\begin{document}

\title{Evangelija pagal Joną}

\chapter{1}


\par 1 Pradžioje buvo Žodis, tas Žodis buvo pas Dievą, ir Žodis buvo Dievas. 
\par 2 Jis pradžioje buvo pas Dievą. 
\par 3 Visa per Jį atsirado, ir be Jo neatsirado nieko, kas yra atsiradę. 
\par 4 Jame buvo gyvybė, ir gyvybė buvo žmonių šviesa. 
\par 5 Šviesa šviečia tamsoje, ir tamsa jos neužgožė. 
\par 6 Buvo Dievo siųstas žmogus, vardu Jonas. 
\par 7 Jis atėjo kaip liudytojas, kad paliudytų apie šviesą ir kad visi per jį įtikėtų. 
\par 8 Jis nebuvo šviesa, bet atėjo liudyti apie šviesą. 
\par 9 Buvo tikroji šviesa, kuri apšviečia kiekvieną žmogų, ateinantį į pasaulį. 
\par 10 Jis buvo pasaulyje, ir pasaulis per Jį atsirado, bet pasaulis Jo nepažino. 
\par 11 Jis atėjo pas savuosius, ir savieji Jo nepriėmė. 
\par 12 Visiems, kurie Jį priėmė, Jis davė galią tapti Dievo vaikais­ tiems, kurie tiki Jo vardą, 
\par 13 kurie ne iš kraujo, ne iš kūno norų ir ne iš vyro norų, bet iš Dievo gimę. 
\par 14 Tas Žodis tapo kūnu ir gyveno tarp mūsų; mes regėjome Jo šlovę­šlovę Tėvo viengimio, pilno malonės ir tiesos. 
\par 15 Jonas apie Jį liudija ir šaukia: “Čia Tas, apie kurį kalbėjau: Tas, kuris eina paskui mane, pirmesnis už mane yra, nes Jis buvo anksčiau už mane”. 
\par 16 Ir iš Jo pilnatvės mes visi gavome malonę po malonės. 
\par 17 Nes Įstatymas buvo duotas per Mozę, o malonė ir tiesa atėjo per Jėzų Kristų. 
\par 18 Dievo niekas niekada nėra matęs, tiktai viengimis Sūnus, Tėvo prieglobstyje esantis, mums Jį apreiškė. 
\par 19 Toks buvo Jono liudijimas, kai žydai iš Jeruzalės atsiuntė pas jį kunigų ir levitų paklausti: “Kas tu esi?” 
\par 20 Jis išpažino ir neišsigynė. Jis išpažino: “Aš nesu Kristus!” 
\par 21 Jie klausė: “Tai kas gi? Gal Elijas?” Jis atsakė: “Ne!”­“Tai gal tu pranašas?” Jis atsakė: “Ne!” 
\par 22 Tada jie tęsė: “Tai kas gi tu, kad mes galėtume duoti atsakymą tiems, kurie mus siuntė? Ką sakai apie save?” 
\par 23 Jis tarė: “Aš­‘dykumoje šaukiančiojo balsas: Ištiesinkite Viešpačiui kelią!’, kaip yra pasakęs pranašas Izaijas”. 
\par 24 Atsiųstieji buvo iš fariziejų. 
\par 25 Jie dar jį paklausė: “Tai kodėl krikštiji, jei nesi nei Kristus, nei Elijas, nei pranašas?” 
\par 26 Jonas jiems atsakė: “Aš krikštiju vandeniu,­bet tarp jūsų stovi Tas, kurio jūs nepažįstate. 
\par 27 Jis yra Tas, kuris po manęs ateina, kuris pirmesnis už mane yra. Jam aš nevertas atrišti sandalų dirželio”. 
\par 28 Tai atsitiko Betanijoje, anapus Jordano, kur Jonas krikštijo. 
\par 29 Kitą dieną Jonas, matydamas pas jį ateinantį Jėzų, prabilo: “Štai Dievo Avinėlis, kuris naikina pasaulio nuodėmę! 
\par 30 Čia Tas, apie kurį pasakiau: po manęs ateina vyras, kuris pirmesnis už mane yra, nes anksčiau už mane buvo. 
\par 31 Aš Jo nepažinojau, bet tam, kad Jis būtų apreikštas Izraeliui, atėjau krikštyti vandeniu”. 
\par 32 Jonas paliudijo, sakydamas: “Aš mačiau Dvasią, lyg balandį nusileidžiančią iš dangaus, ir Ji pasiliko ant Jo. 
\par 33 Aš Jo nepažinojau, bet Tas, kuris mane siuntė krikštyti vandeniu, man pasakė: ‘Ant ko pamatysi nusileidžiančią ir pasiliekančią Dvasią, bus Tas, kuris krikštys Šventąja Dvasia’. 
\par 34 Ir aš mačiau, ir liudiju, kad šitas yra Dievo Sūnus”. 
\par 35 Kitą dieną vėl stovėjo Jonas ir du jo mokiniai. 
\par 36 Išvydęs einantį Jėzų, jis tarė: “Štai Dievo Avinėlis!” 
\par 37 Išgirdę tuos žodžius, abu mokiniai nusekė paskui Jėzų. 
\par 38 Jėzus, atsigręžęs ir pamatęs juos sekančius, paklausė: “Ko ieškote?” Jie atsakė: “Rabi (tai reiškia: “Mokytojau”), kur gyveni?” 
\par 39 Jis jiems tarė: “Ateikite ir pamatysite”. Jie nuėjo, pamatė, kur Jis gyvena, ir tą dieną praleido pas Jį. Tai buvo apie dešimtą valandą. 
\par 40 Vienas iš tų dviejų, kurie girdėjo Jono žodžius ir nusekė paskui Jėzų, buvo Simono Petro brolis Andriejus. 
\par 41 Jis pirmiausia susiieškojo savo brolį Simoną ir jam pranešė: “Radome Mesiją!” (išvertus tai reiškia: “Kristų”). 
\par 42 Ir nusivedė jį pas Jėzų. Jėzus pažvelgė į jį ir tarė: “Tu esi Simonas, Jonos sūnus, o vadinsies Kefas” (išvertus tai reiškia: “Akmuo”). 
\par 43 Kitą dieną Jėzus panoro vykti į Galilėją. Jis sutiko Pilypą ir tarė jam: “Sek paskui mane!” 
\par 44 Pilypas buvo iš Betsaidos­Andriejaus ir Petro miesto. 
\par 45 Pilypas sutiko Natanaelį ir sako jam: “Radome Tą, apie kurį rašė Mozė Įstatyme ir pranašai­ Jėzų iš Nazareto, Juozapo sūnų”. 
\par 46 Natanaelis jam tarė: “Ar iš Nazareto gali būti kas gero?” Pilypas atsakė: “Ateik ir pažiūrėk!” 
\par 47 Pamatęs ateinantį Natanaelį, Jėzus pasakė apie jį: “Štai tikras izraelitas, kuriame nėra klastos!” 
\par 48 O Natanaelis Jam sako: “Iš kur mane pažįsti?” Jėzus atsakė: “Prieš pakviečiant tave Pilypui, kai buvai po figmedžiu, Aš mačiau tave”. 
\par 49 Natanaelis sušuko: “Rabi, Tu Dievo Sūnus, Tu Izraelio karalius!” 
\par 50 Jėzus atsakė: “Tu tiki, kadangi pasakiau tave matęs po figmedžiu? Pamatysi dar didesnių dalykų”. 
\par 51 Ir pridūrė: “Iš tiesų, iš tiesų sakau jums: nuo šiol jūs matysite atvirą dangų ir Dievo angelus, kylančius ir nusileidžiančius ant Žmogaus Sūnaus”.


\chapter{2}


\par 1 Trečią dieną Galilėjos Kanoje buvo vestuvės. Jose dalyvavo Jėzaus motina. 
\par 2 Į vestuves buvo pakviestas ir Jėzus, ir Jo mokiniai. 
\par 3 Pritrūkus vyno, Jėzaus motina Jam sako: “Jie nebeturi vyno”. 
\par 4 Jėzus jai atsakė: “O kas man ir tau, moterie? Dar neatėjo mano valanda”. 
\par 5 Jo motina tarė tarnams: “Darykite, ką tik Jis jums lieps”. 
\par 6 Ten buvo šeši akmeniniai indai žydų apsiplovimams, kiekvienas dviejų trijų saikų talpos. 
\par 7 Jėzus jiems liepė: “Pripilkite indus vandens”. Jie pripylė sklidinus. 
\par 8 Tada Jis sakė: “Dabar semkite ir neškite stalo prižiūrėtojui”. Tie nunešė. 
\par 9 Paragavęs paversto vynu vandens ir nežinodamas, iš kur tai (bet tarnai, kurie sėmė vandenį, žinojo), prižiūrėtojas pasišaukė jaunikį 
\par 10 ir tarė jam: “Kiekvienas žmogus pirmiau stato geresnio vyno, o kai svečiai įgeria, tuomet prastesnio. O tu laikei gerąjį vyną iki šiol”. 
\par 11 Tokią stebuklų pradžią Jėzus padarė Galilėjos Kanoje. Taip Jis parodė savo šlovę, ir mokiniai įtikėjo Jį. 
\par 12 Paskui Jis su savo motina, broliais ir mokiniais nukeliavo į Kafarnaumą. Ten jie pasiliko kelias dienas. 
\par 13 Artėjant žydų Paschai, Jėzus nukeliavo į Jeruzalę. 
\par 14 Šventykloje Jis rado prekiaujančių jaučiais, avimis, karveliais ir prisėdusių pinigų keitėjų. 
\par 15 Susukęs iš virvučių rimbą, Jis išvijo juos visus iš šventyklos, išvarė avis ir jaučius, išbarstė keitėjų pinigus ir išvartė stalus. 
\par 16 Karvelių pardavėjams Jis sakė: “Pasiimkite visa tai iš čia ir iš mano Tėvo namų nedarykite prekybos namų!” 
\par 17 Ir Jo mokiniai prisiminė, kad yra parašyta: “Uolumas dėl Tavo namų sugrauš mane”. 
\par 18 Tada žydai kreipėsi į Jį, sakydami: “Kokį ženklą mums parodysi, jog turi teisę taip daryti?” 
\par 19 Jėzus atsakė: “Sugriaukite šitą šventyklą, ir per tris dienas Aš ją atstatysiu!” 
\par 20 Tada žydai sakė: “Keturiasdešimt šešerius metus šventyklą statė, o Tu atstatysi ją per tris dienas?!” 
\par 21 Bet Jis kalbėjo apie savo kūno šventyklą. 
\par 22 Todėl, Jam prisikėlus iš numirusių, Jo mokiniai prisiminė Jį apie tai kalbėjus ir jie įtikėjo Raštu ir Jėzaus pasakytu žodžiu. 
\par 23 Per Paschos šventę, Jam būnant Jeruzalėje, daugelis įtikėjo Jo vardą, matydami Jo daromus ženklus. 
\par 24 Bet Jėzus, gerai visus pažindamas, jais nepasitikėjo. 
\par 25 Jam nereikėdavo, kad kas paliudytų apie žmogų, nes Jis pats žinojo, kas yra žmoguje.


\chapter{3}


\par 1 Buvo vienas fariziejus, vardu Nikodemas, žydų vyresnysis. 
\par 2 Jis atėjo naktį pas Jėzų ir kreipėsi į Jį: “Rabi, mes žinome, kad esi mokytojas, atėjęs nuo Dievo, nes niekas negalėtų daryti tokių ženklų, kokius Tu darai, jeigu Dievas nebūtų su juo”. 
\par 3 Jėzus jam atsakė: “Iš tiesų, iš tiesų sakau tau: jei kas negims iš naujo, negalės regėti Dievo karalystės”. 
\par 4 Nikodemas paklausė: “Bet kaip gali gimti žmogus, būdamas senas? Argi jis gali antrą kartą įeiti į savo motinos įsčias ir gimti?” 
\par 5 Jėzus atsakė: “Iš tiesų, iš tiesų sakau tau: jei kas negims iš vandens ir Dvasios, negalės įeiti į Dievo karalystę. 
\par 6 Kas gimė iš kūno, yra kūnas, o kas gimė iš Dvasios, yra dvasia. 
\par 7 Nesistebėk, jog pasakiau tau: jums būtina gimti iš naujo. 
\par 8 Vėjas pučia, kur nori; jo ošimą girdi, bet nežinai, iš kur ateina ir kurlink nueina. Taip yra su kiekvienu, kuris gimė iš Dvasios”. 
\par 9 Nikodemas atsiliepė: “Kaip tai gali būti?” 
\par 10 Jėzus jam atsakė: “Tu esi Izraelio mokytojas ir šito nesupranti? 
\par 11 Iš tiesų, iš tiesų sakau tau: mes kalbame, ką žinome, ir liudijame, ką matėme, o jūs nepriimate mūsų liudijimo. 
\par 12 Jei netikite man kalbant apie žemiškuosius dalykus, tai kaipgi tikėsite, jei kalbėsiu jums apie dangiškuosius? 
\par 13 Niekas nėra pakilęs į dangų, kaip tik Tas, kuris nužengė iš dangaus,­Žmogaus Sūnus, esantis danguje. 
\par 14 Kaip Mozė dykumoje iškėlė gyvatę, taip turi būti iškeltas Žmogaus Sūnus, 
\par 15 kad kiekvienas, kuris Jį tiki, nepražūtų, bet turėtų amžinąjį gyvenimą”. 
\par 16 Nes Dievas taip pamilo pasaulį, jog atidavė savo viengimį Sūnų, kad kiekvienas, kuris Jį tiki, nepražūtų, bet turėtų amžinąjį gyvenimą. 
\par 17 Dievas gi nesiuntė savo Sūnaus į pasaulį, kad Jis pasaulį pasmerktų, bet kad pasaulis per Jį būtų išgelbėtas. 
\par 18 Kas Jį tiki, tas neteisiamas, o kas netiki, jau yra pasmerktas už tai, kad netiki viengimio Dievo Sūnaus vardo. 
\par 19 O teismo nuosprendis yra toks: šviesa atėjo į pasaulį, bet žmonės labiau pamilo tamsą nei šviesą, nes jų darbai buvo pikti. 
\par 20 Kiekvienas, kuris daro bloga, neapkenčia šviesos ir neina į šviesą, kad jo darbai nebūtų atskleisti. 
\par 21 O kas vykdo tiesą, tas eina į šviesą, kad pasirodytų, jog jo darbai atlikti Dieve. 
\par 22 Paskui Jėzus su savo mokiniais atėjo į Judėjos kraštą ir, ten su jais būdamas, krikštijo. 
\par 23 Taip pat ir Jonas krikštijo Enone, netoli Salimo, nes ten buvo daug vandens ir žmonės ten ateidavo ir buvo krikštijami. 
\par 24 Tada Jonas dar nebuvo įmestas į kalėjimą. 
\par 25 Tarp Jono mokinių ir vieno žydo kilo ginčas dėl apsivalymo. 
\par 26 Tad jie atėjo pas Joną ir pranešė: “Rabi, Tas kuris buvo su tavimi anapus Jordano, apie kurį tu paliudijai,­štai Jis krikštija, ir visi eina pas Jį”. 
\par 27 Jonas atsakė: “Žmogus nieko negali pasiimti, jeigu jam neduota iš dangaus. 
\par 28 Jūs patys galite man paliudyti, jog sakiau: aš ne Kristus! Aš siųstas būti tik Jo pirmtaku. 
\par 29 Kas turi sužadėtinę, tas sužadėtinis, o sužadėtinio bičiulis, kuris šalia stovi ir jį girdi, džiaugte džiaugiasi jaunikio balsu. Todėl šiam mano džiaugsmui jau nieko netrūksta. 
\par 30 Jis turi augti, o aš­mažėti”. 
\par 31 Kas iš aukštybių ateina, Tas už visus viršesnis, o kas iš žemės,­ žemiškas yra ir žemiškai kalba. Kas iš dangaus ateina, Tas už visus viršesnis. 
\par 32 Jis liudija, ką matė ir girdėjo, tik niekas Jo liudijimo nepriima. 
\par 33 O kas Jo liudijimą priima, tas pripažįsta, jog Dievas teisus, 
\par 34 juk ką Dievas yra siuntęs, Tas kalba Dievo žodžius, nes Dievas teikia Dvasią be saiko. 
\par 35 Tėvas myli Sūnų ir visa yra atidavęs į Jo rankas. 
\par 36 Kas tiki Sūnų, turi amžinąjį gyvenimą, o kas netiki Sūnumi­gyvenimo nematys: ant jo pasilieka Dievo rūstybė.


\chapter{4}


\par 1 Viešpats, sužinojęs, kad fariziejai išgirdo, jog Jo mokinių skaičius labiau auga negu Jono ir Jis gausiau krikštija 
\par 2 (nors pats Jėzus nekrikštydavo, tik Jo mokiniai), 
\par 3 paliko Judėją ir vėl išėjo į Galilėją. 
\par 4 Jam reikėjo eiti per Samariją. 
\par 5 Taigi Jis užsuko į Samarijos miestą, vadinamą Sicharu, netoli nuo lauko, kurį Jokūbas buvo davęs savo sūnui Juozapui. 
\par 6 Tenai buvo Jokūbo šulinys. Nuvargęs iš kelionės, Jėzus atsisėdo prie šulinio. Buvo apie šeštą valandą. 
\par 7 Viena samarietė moteris atėjo semtis vandens. Jėzus ją paprašė: “Duok man gerti”. 
\par 8 (Tuo tarpu Jo mokiniai buvo nuėję į miestą nusipirkti maisto.) 
\par 9 Samarietė atsakė: “Kaip Tu, būdamas žydas, prašai mane, samarietę, gerti?” (Mat žydai nebendrauja su samariečiais.) 
\par 10 Jėzus jai tarė: “Jei tu pažintum Dievo dovaną ir kas yra Tas, kuris tave prašo: ‘Duok man gerti’, pati būtum Jį prašiusi, ir Jis tau būtų davęs gyvojo vandens!” 
\par 11 Moteris atsiliepė: “Viešpatie, bet Tu neturi kuo pasemti, o šulinys gilus. Iš kur Tu imsi gyvojo vandens? 
\par 12 Argi Tu didesnis už mūsų tėvą Jokūbą, kuris tą šulinį mums paliko ir pats iš jo gėrė, ir jo vaikai, ir gyvuliai?” 
\par 13 Jėzus atsakė: “Kiekvienas, kas geria šitą vandenį, ir vėl trokš. 
\par 14 O kas gers vandenį, kurį Aš jam duosiu, tas nebetrokš per amžius, ir vanduo, kurį jam duosiu, taps jame versme vandens, trykštančio į amžinąjį gyvenimą”. 
\par 15 Tada moteris Jam tarė: “Viešpatie, duok man to vandens, kad aš nebetrokščiau ir nebevaikščiočiau jo semtis čionai”. 
\par 16 Jėzus atsiliepė: “Eik, pakviesk savo vyrą ir sugrįžk čia”. 
\par 17 Moteris atsakė: “Aš neturiu vyro”. Jėzus jai tarė: “Gerai pasakei: ‘Neturiu vyro’, 
\par 18 nes esi turėjusi penkis vyrus, ir tas, kurį dabar turi, nėra tavo vyras. Čia tu tiesą pasakei”. 
\par 19 Moteris atsiliepė: “Aš matau, Viešpatie, jog Tu esi pranašas. 
\par 20 Mūsų tėvai garbino ant šito kalno, o jūs sakote, kad Jeruzalė esanti vieta, kur reikia garbinti”. 
\par 21 Jėzus atsakė: “Moterie, tikėk manimi, jog ateina valanda, kada garbinsite Tėvą ne ant šio kalno ir ne Jeruzalėje. 
\par 22 Jūs garbinate, ko nepažįstate, o mes garbiname, ką pažįstame, nes išgelbėjimas­iš žydų. 
\par 23 Bet ateina valanda,­jau dabar ji yra,­kai tikrieji garbintojai garbins Tėvą dvasioje ir tiesoje, nes Tėvas tokių Jo garbintojų ieško. 
\par 24 Dievas yra Dvasia, ir Jį garbinantys turi garbinti dvasioje ir tiesoje”. 
\par 25 Moteris Jam sako: “Žinau, jog ateina Mesijas (tai yra Kristus). Atėjęs Jis mums viską paskelbs”. 
\par 26 Jėzus jai tarė: “Tai Aš, kuris su tavimi kalbu!” 
\par 27 Tuo metu sugrįžo Jo mokiniai ir nustebo, kad Jis kalbėjo su moterimi. Vis dėlto nė vienas nepaklausė: “Ko iš jos nori?” arba: “Apie ką su ja kalbi?” 
\par 28 O moteris, palikusi ąsotį, nubėgo į miestą ir apskelbė žmonėms: 
\par 29 “Eikite pažiūrėti žmogaus, kuris pasakė man viską, ką esu padariusi. Ar tik Jis nebus Kristus?” 
\par 30 Ir žmonės iš miesto ėjo pas Jį. 
\par 31 Tuo tarpu mokiniai ragino Jį, sakydami: “Rabi, valgyk!” 
\par 32 O Jis jiems tarė: “Aš turiu valgyti maisto, kurio jūs nežinote”. 
\par 33 Tada mokiniai pradėjo vienas kitą klausinėti: “Nejaugi kas atnešė Jam valgyti?” 
\par 34 Bet Jėzus tarė: “Mano maistas­vykdyti valią To, kuris mane siuntė, ir baigti Jo darbą. 
\par 35 Argi jūs nesakote: ‘Dar keturi mėnesiai, ir ateis pjūtis’? Štai sakau jums: pakelkite akis ir pažiūrėkite į laukus­jie jau boluoja ir prinokę pjūčiai. 
\par 36 Jau pjovėjas uždarbį gauna ir renka vaisių amžinajam gyvenimui, kad kartu džiaugtųsi ir sėjėjas, ir pjovėjas. 
\par 37 Čia teisingai priežodis sako: ‘Vienas pasėja, kitas nupjauna’. 
\par 38 Aš pasiunčiau jus nuimti derliaus, į kurį jūs neįdėjote darbo. Kiti pasidarbavo, o jūs įstojote į jų darbą”. 
\par 39 Daug samariečių iš ano miesto įtikėjo Jėzų dėl moters žodžių: “Jis man pasakė viską, ką esu padariusi”. 
\par 40 Atėję samariečiai prašė Jį pasilikti pas juos, ir Jis ten pasiliko dvi dienas. 
\par 41 Ir dar daug žmonių įtikėjo dėl Jo žodžių. 
\par 42 O moteriai jie pasakė: “Dabar tikime ne dėl tavo šnekos. Mes patys išgirdome ir žinome, kad Jis iš tiesų yra Kristus, pasaulio Gelbėtojas”. 
\par 43 Po dviejų dienų Jis išvyko iš ten į Galilėją. 
\par 44 Pats Jėzus buvo paliudijęs: “Pranašas negerbiamas savo tėviškėje”. 
\par 45 Kai Jis pasiekė Galilėją, galilėjiečiai priėmė Jį, nes buvo matę visa, ką Jis padarė per šventę Jeruzalėje; mat ir jie buvo nukeliavę į tą šventę. 
\par 46 Taigi Jėzus vėl atėjo į Galilėjos Kaną, kur buvo pavertęs vandenį vynu. Kafarnaume buvo vienas karaliaus valdininkas, kurio sūnus sirgo. 
\par 47 Išgirdęs, jog Jėzus iš Judėjos sugrįžo į Galilėją, jis atėjo pas Jį ir maldavo ateiti ir išgydyti jo sūnų, kuris buvo prie mirties. 
\par 48 Tuomet Jėzus jam atsakė: “Kol nepamatysite ženklų ir stebuklų, niekaip netikėsite”. 
\par 49 Karaliaus valdininkas prašė: “Viešpatie, ateik, kol mano vaikas dar nenumirė”. 
\par 50 Jėzus jam tarė: “Eik, tavo sūnus gyvas!” Žmogus patikėjo Jėzaus jam pasakytu žodžiu ir išėjo. 
\par 51 Pareinantį pasitiko Jį tarnai ir pranešė: “Tavo vaikas gyvas”. 
\par 52 Jis paklausė, kurią valandą jam pasidarė geriau. Jie atsakė: “Vakar septintą valandą dingo jam karštis”. 
\par 53 Taip tėvas patyrė, kad tai buvo ta valanda, kada Jėzus pasakė jam: “Tavo sūnus gyvas”. Ir įtikėjo jis pats bei visi jo namai. 
\par 54 Tai buvo antras stebuklas, kurį Jėzus padarė, sugrįžęs iš Judėjos į Galilėją.


\chapter{5}


\par 1 Kiek vėliau buvo žydų šventė, ir Jėzus nukeliavo į Jeruzalę. 
\par 2 Jeruzalėje, prie Avių vartų, yra maudyklė, žydiškai vadinama Betezda, turinti penkias stogines. 
\par 3 Jose gulėdavo daugybė ligonių­ aklų, luošų, paralyžiuotų, kurie laukė vandens sujudant. 
\par 4 Mat kartkartėmis maudyklėn nusileisdavo angelas ir sujudindavo vandenį. Kas, vandeniui pajudėjus, pirmas įlipdavo į tvenkinį, pagydavo, kad ir kokia liga sirgdavo. 
\par 5 Ten buvo vienas žmogus, išsirgęs trisdešimt aštuonerius metus. 
\par 6 Pamatęs jį gulintį ir sužinojęs jį labai seniai sergant, Jėzus paklausė: “Ar nori pasveikti?” 
\par 7 Ligonis Jam atsakė: “Viešpatie, aš neturiu žmogaus, kuris, vandeniui sujudėjus, mane įkeltų į tvenkinį. O kol pats einu, kitas įlipa greičiau už mane”. 
\par 8 Jėzus jam tarė: “Kelkis, imk savo gultą ir vaikščiok!” 
\par 9 Žmogus bematant išgijo, pasiėmė gultą ir pradėjo vaikščioti. O toji diena buvo sabatas. 
\par 10 Todėl žydai sakė išgydytajam: “Šiandien sabatas, tau negalima nešti gulto”. 
\par 11 Jis jiems atsakė: “Tas, kuris mane išgydė, man liepė: ‘Imk savo gultą ir vaikščiok!’ ” 
\par 12 Jie klausinėjo: “O kas Tas žmogus, kuris tau liepė: ‘Imk savo gultą ir vaikščiok’?” 
\par 13 Išgydytasis nežinojo, kas Jis, kadangi Jėzus pasitraukė, miniai susirinkus toje vietoje. 
\par 14 Vėliau Jėzus jį sutiko šventykloje ir tarė: “Štai tu esi pasveikęs. Daugiau nebenusidėk, kad neatsitiktų tau kas blogesnio!” 
\par 15 Žmogus nuėjo ir pranešė žydams, kad jį išgydė Jėzus. 
\par 16 Žydai dėl to persekiojo Jėzų ir norėjo Jį nužudyti, kad Jis taip darė per sabatą. 
\par 17 O Jėzus jiems atsakė: “Mano Tėvas darbuojasi lig šiolei, ir Aš darbuojuosi”. 
\par 18 Dėl to žydai dar labiau ieškojo progos Jį nužudyti, nes Jis ne tik laužė sabatą, bet ir vadino Dievą savo Tėvu, lygindamas save su Dievu. 
\par 19 Jėzus jiems atsakė: “Iš tiesų, iš tiesų sakau jums: Sūnus nieko negali daryti iš savęs, o vien tai, ką mato darant Tėvą; nes ką Jisai daro, lygiai daro ir Sūnus. 
\par 20 Nes Tėvas myli Sūnų ir rodo Jam visa, ką pats daro. Ir Jam parodys dalykų, dar didesnių už šituos, kad jūs stebėsitės. 
\par 21 Kaip Tėvas prikelia numirusius ir juos atgaivina, taip ir Sūnus grąžina gyvybę, kam nori. 
\par 22 Ir Tėvas nieko neteisia, bet visą teismą pavedė Sūnui, 
\par 23 kad visi gerbtų Sūnų, kaip gerbia Tėvą. Kas negerbia Sūnaus, tas negerbia Jį siuntusio Tėvo. 
\par 24 Iš tiesų, iš tiesų sakau jums: kas mano žodžių klauso ir mane atsiuntusį tiki, tas turi amžinąjį gyvenimą ir nepateks į teismą, nes iš mirties yra perėjęs į gyvenimą. 
\par 25 Iš tiesų, iš tiesų sakau jums: ateina valanda,­ir dabar jau yra,­ kada mirusieji išgirs Dievo Sūnaus balsą, ir kurie išgirs, tie atgis. 
\par 26 Nes kaip Tėvas turi gyvybę pats savyje, taip davė ir Sūnui turėti gyvybę pačiam savyje, 
\par 27 ir taip pat suteikė Jam valdžią teisti, nes Jis­Žmogaus Sūnus. 
\par 28 Nesistebėkite tuo, nes ateina valanda, kai visi, esantys kapuose, išgirs Jo balsą. 
\par 29 Ir tie, kurie darė gera, išeis gyvenimo prisikėlimui, o kurie darė bloga­teismo prisikėlimui. 
\par 30 Iš savęs Aš nieko negaliu daryti. Aš teisiu, kaip girdžiu, ir mano teismas teisingas, nes Aš ieškau ne savo valios, bet valios Tėvo, kuris mane siuntė”. 
\par 31 “Jei Aš pats apie save liudiju, mano liudijimas nėra tikras. 
\par 32 Bet apie mane liudija kitas, ir Aš žinau, kad Jo liudijimas, kuriuo Jis liudija apie mane, yra tikras. 
\par 33 Jūs buvote nusiuntę pas Joną, ir jis paliudijo tiesą. 
\par 34 Aš neieškau žmogaus liudijimo, bet šitai kalbu tam, kad būtumėte išgelbėti. 
\par 35 Jonas buvo degantis ir šviečiantis žiburys, tačiau jūs panorėjote tik valandėlę jo šviesa pasidžiaugti. 
\par 36 O Aš turiu aukštesnį liudijimą negu Jono: tie darbai, kuriuos man skyrė nuveikti Tėvas,­patys darbai, kuriuos Aš darau,­liudija apie mane, kad mane siuntė Tėvas. 
\par 37 Ir mane pasiuntęs Tėvas pats paliudijo apie mane. Bet jūs niekad nesate girdėję Jo balso, nei regėję Jo pavidalo, 
\par 38 ir neturite Jo žodžio, jumyse pasiliekančio, nes netikite Tuo, kurį Jis siuntė. 
\par 39 Jūs tyrinėjate Raštus, nes manote juose turį amžinąjį gyvenimą. O Raštai liudija apie mane, 
\par 40 bet jūs nenorite ateiti pas mane, kad turėtumėte gyvenimą”. 
\par 41 “Šlovės iš žmonių Aš nepriimu. 
\par 42 Aš žinau, kad neturite savyje Dievo meilės. 
\par 43 Aš atėjau savo Tėvo vardu, o jūs manęs nepriimate. Jei kitas ateitų savo vardu, tą jūs priimtumėte. 
\par 44 Kaipgi jūs galite tikėti, jeigu vienas iš kito priimate šlovę, o tos šlovės, kuri iš vieno Dievo ateina, neieškote? 
\par 45 Nemanykite, kad Aš jus kaltinsiu Tėvui! Jūsų kaltintojas yra Mozė, į kurį esate savo viltis sudėję. 
\par 46 Jeigu jūs tikėtumėte Moze, tai tikėtumėte ir manimi, nes jis rašė apie mane. 
\par 47 Bet jeigu jūs netikite jo raštais, kaipgi patikėsite mano žodžiais?!”


\chapter{6}


\par 1 Paskui Jėzus nuvyko į kitą pusę Galilėjos, arba Tiberiados, ežero. 
\par 2 Jį lydėjo gausi minia, nes žmonės matė stebuklus, kuriuos Jis darė ligoniams. 
\par 3 Jėzus užkopė į kalną ir ten atsisėdo su savo mokiniais. 
\par 4 Artėjo žydų šventė Pascha. 
\par 5 Pakėlęs akis ir pamatęs, kokia didelė minia pas Jį atėjusi, Jėzus paklausė Pilypą: “Kur pirksime duonos jiems pavalgydinti?” 
\par 6 Jis klausė, mėgindamas jį, nes pats žinojo, ką darysiąs. 
\par 7 Pilypas Jam atsakė: “Už du šimtus denarų duonos neužteks, kad kiekvienas gautų bent gabalėlį”. 
\par 8 Vienas iš mokinių, Simono Petro brolis Andriejus, Jam pasakė: 
\par 9 “Čia yra vienas berniukas, kuris turi penkis miežinės duonos kepalus ir dvi žuveles. Bet ką tai reiškia tokiai daugybei!” 
\par 10 Jėzus tarė: “Susodinkite žmones!” Toje vietoje buvo daug žolės. Taigi jie susėdo, iš viso kokie penki tūkstančiai vyrų. 
\par 11 Tada Jėzus paėmė duoną ir padėkojęs išdalino mokiniams, o mokiniai ten sėdintiems; taip pat ir žuvis, kiek kas norėjo. 
\par 12 Kai žmonės pasisotino, Jis pasakė savo mokiniams: “Surinkite trupinius, kad niekas nepražūtų”. 
\par 13 Taigi jie surinko ir iš penkių miežinės duonos kepalų pripylė dvylika pintinių trupinių, kurie buvo atlikę nuo valgiusiųjų. 
\par 14 Pamatę Jėzaus padarytą stebuklą, žmonės sakė: “Jis tikrai yra Tas pranašas, kuris turi ateiti į pasaulį”. 
\par 15 O Jėzus, supratęs, kad jie ruošiasi pasigriebti Jį ir paskelbti karaliumi, vėl pasitraukė pats vienas į kalną. 
\par 16 Atėjus vakarui, Jo mokiniai nusileido prie ežero, 
\par 17 sulipo į valtį ir išplaukė į kitą pusę ežero, į Kafarnaumą. Jau sutemo, o Jėzus vis dar nebuvo grįžęs pas juos. 
\par 18 Ežeras bangavo, nes pūtė smarkus vėjas. 
\par 19 Nusiyrę apie dvidešimt penkias­trisdešimt stadijų, jie pamatė Jėzų, einantį ežeru ir besiartinantį prie valties. Jie išsigando. 
\par 20 O Jis jiems tarė: “Tai Aš, nebijokite!” 
\par 21 Jie norėjo Jį paimti į valtį, bet valtis netrukus priartėjo prie kranto, į kurį jie yrėsi. 
\par 22 Kitą dieną minia, buvusi anoje pusėje, pamatė, kad ten nebuvo kitos valties, tik ta, į kurią įlipo Jo mokiniai. O Jėzus su jais neįlipo, ir šie išplaukė vieni. 
\par 23 Iš Tiberiados atplaukė kitų valčių netoli tos vietos, kur žmonės buvo valgę duonos, Viešpačiui padėkojus. 
\par 24 Pamatę, kad čia nėra nei Jėzaus, nei Jo mokinių, žmonės lipo į valtis ir plaukė į Kafarnaumą, ieškodami Jėzaus. 
\par 25 Suradę Jį kitoje ežero pusėje, jie klausinėjo: “Rabi, kada čia atvykai?” 
\par 26 Jėzus, atsakydamas jiems, tarė: “Iš tiesų, iš tiesų sakau jums: jūs ieškote manęs ne todėl, kad matėte ženklų, bet kad prisivalgėte duonos lig soties. 
\par 27 Darbuokitės ne dėl žūvančio maisto, bet dėl išliekančio amžinajam gyvenimui! Jį duos jums Žmogaus Sūnus, nes Tėvas­Dievas Jį savo antspaudu yra pažymėjęs”. 
\par 28 Jie paklausė: “Ką mums daryti, kad vykdytume Dievo darbus?” 
\par 29 Jėzus atsakė: “Tai yra Dievo darbas: tikėkite Tą, kurį Jis siuntė”. 
\par 30 Tada jie klausė: “Kokį padarysi ženklą, kad pamatytume ir Tave įtikėtume? Ką nuveiksi? 
\par 31 Mūsų tėvai dykumoje valgė maną, kaip parašyta: ‘Jis davė jiems valgyti duonos iš dangaus’ ”. 
\par 32 Tuomet Jėzus jiems tarė: “Iš tiesų, iš tiesų sakau jums: ne Mozė davė jums duonos iš dangaus, bet mano Tėvas duoda jums iš dangaus tikrosios duonos. 
\par 33 Nes Dievo duona yra Tas, kuris nužengia iš dangaus ir duoda pasauliui gyvybę”. 
\par 34 Tada jie tarė Jam: “Viešpatie, duok visuomet mums tos duonos!” 
\par 35 Jėzus atsakė: “Aš esu gyvenimo duona! Kas ateina pas mane, niekuomet nebealks, ir kas tiki mane, niekuomet nebetrokš. 
\par 36 Bet Aš jums sakiau: jūs mane matėte, ir netikite. 
\par 37 Visi, kuriuos man duoda Tėvas, ateis pas mane, ir ateinančio pas mane Aš neišvarysiu lauk, 
\par 38 nes Aš nužengiau iš dangaus vykdyti ne savo valios, bet valios To, kuris mane siuntė. 
\par 39 O mane siuntusio Tėvo valia,­ kad nepražudyčiau nė vieno iš tų, kuriuos Jis man davė, bet kad prikelčiau juos paskutiniąją dieną. 
\par 40 Tokia mano Siuntėjo valia, kad kiekvienas, kuris regi Sūnų ir tiki Jį, turėtų amžinąjį gyvenimą; ir Aš jį prikelsiu paskutiniąją dieną”. 
\par 41 Tada žydai ėmė murmėti prieš Jį dėl to, kad Jis pasakė: “Aš duona, nužengusi iš dangaus”. 
\par 42 Jie sakė: “Argi Jis ne Jėzus, Juozapo sūnus?! Argi nepažįstame Jo tėvo ir motinos? Tad kodėl Jis sako: ‘Aš nužengiau iš dangaus’?” 
\par 43 Jėzus jiems atsakė: “Nemurmėkite tarpusavyje! 
\par 44 Niekas negali ateiti pas mane, jei mane siuntęs Tėvas jo nepatraukia; ir tą Aš prikelsiu paskutiniąją dieną. 
\par 45 Pranašų parašyta: ‘Ir visi bus mokomi Dievo’. Todėl, kas išgirdo iš Tėvo ir pasimokė, ateina pas mane. 
\par 46 Bet tai nereiškia, jog kas nors būtų Tėvą regėjęs; tiktai Tas, kuris iš Dievo yra, Jis matė Tėvą. 
\par 47 Iš tiesų, iš tiesų sakau jums: kas tiki mane, tas turi amžinąjį gyvenimą. 
\par 48 Aš esu gyvenimo duona. 
\par 49 Jūsų tėvai dykumoje valgė maną ir mirė. 
\par 50 O ši duona yra nužengusi iš dangaus, kad, kas ją valgys, nemirtų. 
\par 51 Aš esu gyvoji duona, nužengusi iš dangaus. Kas valgo šitos duonos­gyvens per amžius. Duona, kurią Aš duosiu, yra mano kūnas, kurį Aš atiduosiu už pasaulio gyvybę”. 
\par 52 Tada žydai ėmė tarp savęs ginčytis ir klausinėti: “Kaip Jis gali duoti mums valgyti savo kūną?!” 
\par 53 O Jėzus jiems kalbėjo: “Iš tiesų, iš tiesų sakau jums: jei nevalgysite Žmogaus Sūnaus kūno ir negersite Jo kraujo, neturėsite savyje gyvybės! 
\par 54 Kas valgo mano kūną ir geria mano kraują, tas turi amžinąjį gyvenimą, ir Aš jį prikelsiu paskutiniąją dieną. 
\par 55 Nes mano kūnas tikrai yra valgis, ir mano kraujas tikrai yra gėrimas. 
\par 56 Kas valgo mano kūną ir geria mano kraują, tas pasilieka manyje, ir Aš jame. 
\par 57 Kaip mane siuntė gyvasis Tėvas ir Aš gyvenu per Tėvą, taip ir tas, kuris mane valgo, gyvens per mane. 
\par 58 Štai duona, nužengusi iš dangaus,­ne taip, kaip jūsų tėvai valgė maną ir mirė. Kas valgo šią duoną­gyvens per amžius”. 
\par 59 Visa tai Jis paskelbė, mokydamas sinagogoje, Kafarnaume. 
\par 60 Tai išgirdę, daugelis Jo mokinių sakė: “Kieti šie žodžiai, kas gali jų klausytis!” 
\par 61 Jėzus, žinodamas, kad mokiniai dėl to murma, paklausė: “Jus tai piktina? 
\par 62 O kas būtų, jei pamatytumėte Žmogaus Sūnų, pakylantį ten, kur Jis buvo pirmiau?! 
\par 63 Dvasia teikia gyvybę, o kūnas nieko neduoda. Žodžiai, kuriuos jums kalbu, yra dvasia ir gyvenimas. 
\par 64 Bet kai kurie iš jūsų netiki”. Jėzus iš pat pradžių žinojo, kas netiki ir kas Jį išduos. 
\par 65 Ir Jis sakė: “Štai kodėl Aš jums sakiau: niekas negali ateiti pas mane, jeigu jam nėra duota mano Tėvo”. 
\par 66 Nuo to laiko daug Jo mokinių pasitraukė ir daugiau su Juo nebevaikščiojo. 
\par 67 Tada Jėzus paklausė dvylika: “Gal ir jūs norite pasitraukti?” 
\par 68 Simonas Petras atsakė: “Viešpatie, pas ką mes eisime?! Tu turi amžinojo gyvenimo žodžius. 
\par 69 Mes įtikėjome ir pažinome, kad Tu esi Kristus, gyvojo Dievo Sūnus”. 
\par 70 Jėzus jiems atsakė: “Argi Aš neišsirinkau jūsų, dvylikos? Tačiau tarp jūsų vienas yra velnias”. 
\par 71 Jis kalbėjo apie Judą, Simono Iskarijoto sūnų. Šis, vienas iš dvylikos, turėjo Jį išduoti.


\chapter{7}


\par 1 Tada Jėzus vaikščiojo po Galilėją. Jis nenorėjo eiti Judėjon, nes žydai ieškojo progos Jį nužudyti. 
\par 2 Artėjo žydų Palapinių šventė. 
\par 3 Jo broliai Jam kalbėjo: “Keliauk iš čia į Judėją, kad ir Tavo mokiniai pamatytų, kokius darbus darai. 
\par 4 Juk, norėdamas iškilti viešumon, niekas neveikia slapčiomis. Jei darai tokius darbus, parodyk save pasauliui”. 
\par 5 (Mat netgi Jo broliai Juo netikėjo.) 
\par 6 Jėzus jiems atsakė: “Mano laikas dar neatėjo, o jums laikas visada tinkamas. 
\par 7 Pasaulis negali jūsų nekęsti, o manęs jis nekenčia, nes Aš liudiju, kad jo darbai pikti. 
\par 8 Jūs eikite į iškilmes. Aš į šitą šventę neisiu, nes mano laikas dar neatėjo”. 
\par 9 Tai jiems pasakęs, Jis pasiliko Galilėjoje. 
\par 10 Bet kai Jo broliai iškeliavo į šventę, tada ir Jis išėjo, bet ne viešai, o tarsi slapčiomis. 
\par 11 Tuo tarpu iškilmėse žydai Jo ieškojo, klausinėdami: “O kur Tas?” 
\par 12 Apie Jį taip pat ėjo kalbos miniose. Vieni sakė: “Jis geras!” Kiti neigė: “Visai ne. Jis tik klaidina žmones”. 
\par 13 Tačiau nė vienas apie Jį viešai nekalbėjo, nes bijojo žydų. 
\par 14 Šventei įpusėjus, Jėzus atėjo į šventyklą ir ėmė mokyti. 
\par 15 Žydai stebėjosi ir sakė: “Iš kur Jis žino raštą, visai nesimokęs?” 
\par 16 Jėzus jiems atsakė: “Mano mokslas ne iš manęs, bet iš To, kuris mane siuntė. 
\par 17 Kas nori vykdyti Jo valią, supras, ar tas mokymas iš Dievo, ar Aš kalbu iš savęs. 
\par 18 Kas iš savęs kalba, ieško sau šlovės. O kuris ieško Jo siuntėjo šlovės, Tas teisus, ir nėra Jame neteisybės”. 
\par 19 “Argi Mozė nedavė jums Įstatymo? Tačiau niekas iš jūsų Įstatymo nesilaiko. Kodėl gi norite mane nužudyti?” 
\par 20 Žmonės atsiliepė: “Tu turi demoną! Kas gi nori tave nužudyti?” 
\par 21 Jėzus jiems atsakė: “Aš padariau tik vieną darbą, ir jūs visi nustebote. 
\par 22 Mozė jums davė apipjaustymą,­ nors jis kilęs ne iš Mozės, bet iš protėvių,­ir jūs apipjaustote žmogų per sabatą. 
\par 23 Jei žmogus apipjaustomas sabato dieną, kad nebūtų sulaužytas Mozės Įstatymas, tai kodėl pykstate ant manęs, kad Aš visą žmogų pagydžiau per sabatą? 
\par 24 Tad neteiskite pagal išorę, bet teiskite teisingai”. 
\par 25 Kai kurie Jeruzalės gyventojai klausinėjo: “Ar tik ne šitą nori nužudyti? 
\par 26 Štai Jis viešai kalba, ir niekas Jam nieko nesako. Gal vyresnybė įsitikino, jog Jis iš tiesų Kristus? 
\par 27 Tačiau mes žinome, iš kur šis yra. O kai ateis Kristus, niekas nežinos, iš kur Jis”. 
\par 28 Tuomet Jėzus, mokydamas šventykloje, šaukė: “Jūs pažįstate mane ir žinote, iš kur Aš. Ne pats nuo savęs Aš atėjau, bet teisingas yra Tas, kuris mane siuntė, o jūs Jo nepažįstate. 
\par 29 Aš Jį pažįstu, nes Aš esu iš Jo, ir Jis mane pasiuntė”. 
\par 30 Tada jie norėjo Jėzų suimti, bet nė vienas nepakėlė prieš Jį rankos, nes dar nebuvo atėjusi Jo valanda. 
\par 31 Bet daugelis iš minios įtikėjo Jį ir kalbėjo: “Argi atėjęs Kristus padarytų daugiau ženklų, kaip kad šis yra padaręs?” 
\par 32 Fariziejai išgirdo žmones šitaip kalbant apie Jį, ir aukštieji kunigai bei fariziejai pasiuntė sargybinius Jį suimti. 
\par 33 Tuomet Jėzus jiems pasakė: “Dar trumpą laiką būsiu su jumis. Paskui iškeliausiu pas Tą, kuris mane siuntė. 
\par 34 Jūs manęs ieškosite ir nerasite; ir ten, kur Aš būsiu, jūs negalėsite nueiti”. 
\par 35 Tuomet žydai ėmė kalbėtis tarp savęs: “Kurgi Jis žada keliauti, kad mes negalėsime Jo rasti? Gal Jis rengiasi išvykti pas išsisklaidžiusius tarp graikų ir mokyti graikus? 
\par 36 Ką reiškia tie Jo žodžiai: ‘Jūs manęs ieškosite ir nerasite; ir ten, kur Aš būsiu, jūs negalėsite nueiti’?” 
\par 37 Paskutinę, didžiąją šventės dieną Jėzus stovėjo ir šaukė: “Jei kas trokšta, teateina pas mane ir tegeria! 
\par 38 Kas mane tiki, kaip Raštas sako, iš jo vidaus plūs gyvojo vandens upės”. 
\par 39 Jis kalbėjo apie Dvasią, kurią turės gauti Jį įtikėjusieji. Mat Šventoji Dvasia dar nebuvo nužengusi, nes Jėzus dar nebuvo pašlovintas. 
\par 40 Išgirdę tuos žodžius, daugelis iš minios sakė: “Jis iš tiesų pranašas!” 
\par 41 Kiti tvirtino: “Jis Kristus!” Dar kiti prieštaravo: “Nejaugi Kristus ateitų iš Galilėjos? 
\par 42 Argi Raštas nesako, jog Kristus ateis iš Dovydo palikuonių, iš Betliejaus miestelio, iš kur kilo Dovydas?” 
\par 43 Taigi minioje dėl Jo kilo nesutarimas. 
\par 44 Kai kurie norėjo Jį suimti, bet nė vienas nepakėlė prieš Jį rankos. 
\par 45 Sargybiniai sugrįžo pas aukštuosius kunigus bei fariziejus, o tie klausė: “Kodėl Jo neatvedėte?” 
\par 46 Sargybiniai atsakė: “Niekados žmogus nėra taip kalbėjęs, kaip šis!” 
\par 47 Fariziejai atsakė: “Gal ir jūs jau suvedžioti? 
\par 48 Ar tiki Jį bent vienas iš vyresnybės ar fariziejų? 
\par 49 O ta minia, nežinanti Įstatymo,­prakeikta”. 
\par 50 Tada prabilo vienas iš jų, Nikodemas, kuris buvo aplankęs Jėzų nakčia: 
\par 51 “Argi mūsų Įstatymas teisia žmogų, jeigu jis pirmiau neišklausytas ir nežinoma, ką jis padaręs?” 
\par 52 Jie jam tarė: “Gal ir tu iš Galilėjos? Patyrinėk, ir pamatysi, kad joks pranašas nėra kilęs iš Galilėjos”. 
\par 53 Ir taip išsiskirstė kiekvienas po namus.


\chapter{8}


\par 1 Jėzus nuėjo į Alyvų kalną. 
\par 2 Anksti rytą Jis vėl atėjo į šventyklą. Visi žmonės rinkosi prie Jo, ir Jis atsisėdęs juos mokė. 
\par 3 Rašto žinovai ir fariziejai atvedė pas Jį moterį, sugautą svetimaujant. Pastatė ją viduryje 
\par 4 ir kreipėsi į Jį: “Mokytojau, ši moteris buvo pagauta svetimaujant. 
\par 5 Mozė Įstatyme mums įsakė tokias užmėtyti akmenimis. O ką Tu pasakysi?” 
\par 6 Jie tai sakė, mėgindami Jį, kad turėtų kuo apkaltinti. Bet Jėzus, jų nepaisydamas, pasilenkęs rašė pirštu ant žemės. 
\par 7 Jiems nesiliaujant klausinėti, Jis atsitiesė ir tarė: “Kas iš jūsų be nuodėmės, tegul pirmas sviedžia į ją akmenį”. 
\par 8 Ir vėl pasilenkęs rašė ant žemės. 
\par 9 Tai išgirdę, sąžinės apkaltinti jie vienas po kito ėmė trauktis šalin, pradedant nuo vyresniųjų iki paskutiniojo. 
\par 10 Atsitiesęs ir nebematydamas nė vieno, tik moterį, Jėzus paklausė: “Moterie, kur tie tavo kaltintojai? Niekas tavęs nepasmerkė?” 
\par 11 Ji atsiliepė: “Niekas, Viešpatie”. Jėzus jai tarė: “Nė Aš tavęs nesmerkiu. Eik ir daugiau nebenusidėk”. 
\par 12 Jėzus vėl prabilo: “Aš­pasaulio šviesa. Kas seka manimi, nebevaikščios tamsoje, bet turės gyvenimo šviesą”. 
\par 13 Tada fariziejai Jam pasakė: “Tu pats apie save liudiji,­tavo liudijimas netikras”. 
\par 14 Jėzus jiems atsakė: “Nors Aš ir liudiju pats apie save, mano liudijimas yra tikras, nes Aš žinau, iš kur atėjau ir kur einu. O jūs nežinote, nei iš kur Aš atėjau, nei kur einu. 
\par 15 Jūs teisiate pagal kūną, Aš neteisiu nė vieno. 
\par 16 O jeigu Aš ir teisiu­mano teismas teisingas, nes Aš ne vienas, bet esu Aš ir mane siuntęs Tėvas. 
\par 17 Ir jūsų Įstatyme parašyta, kad dviejų asmenų liudijimas tikras. 
\par 18 Aš liudiju pats apie save, ir apie mane liudija mane siuntęs Tėvas”. 
\par 19 Jie paklausė: “Kur yra Tavo Tėvas?” Jėzus atsakė: “Jūs nepažįstate nei manęs, nei mano Tėvo. Jei pažintumėte mane, pažintumėte ir mano Tėvą”. 
\par 20 Šiuos žodžius Jėzus pasakė, mokydamas šventyklos iždinėje. Ir niekas Jo nesuėmė, nes dar nebuvo atėjusi Jo valanda. 
\par 21 Jėzus vėl jiems kalbėjo: “Aš išeinu, ir jūs ieškosite manęs ir mirsite savo nuodėmėse. Kur Aš einu, jūs negalite nueiti”. 
\par 22 Tada žydai ėmė kalbėti: “Nejaugi Jis nusižudys, kad sako: ‘Kur Aš einu, jūs negalite nueiti’?” 
\par 23 Jis atsakė: “Jūs esate iš pažemių, o Aš esu iš aukštybės. Jūs­iš šio pasaulio, o Aš­ne iš šio pasaulio. 
\par 24 Dėl to Aš jums sakiau, kad mirsite savo nuodėmėse. Jeigu netikėsite, kad Aš Esu,­mirsite savo nuodėmėse”. 
\par 25 Tada jie klausė: “Kas Tu esi?” Jėzus atsakė: “Tas, ką nuo pradžios jums sakiau. 
\par 26 Daug turiu ką apie jus kalbėti ir teisti, bet teisingas yra mano Siuntėjas, ir Aš skelbiu pasauliui, ką iš Jo girdėjau”. 
\par 27 Jie nesuprato, kad Jis kalbėjo jiems apie Tėvą. 
\par 28 O Jėzus tęsė: “Kai Žmogaus Sūnų būsite aukštyn iškėlę, tuomet suprasite, kad Aš Esu ir kad nieko nedarau iš savęs, bet skelbiu tai, ko mane Tėvas išmokė. 
\par 29 Mano Siuntėjas yra su manimi; Tėvas nepaliko manęs vieno, nes visuomet darau, kas Jam patinka”. 
\par 30 Jam tai kalbant, daugelis įtikėjo Jį. 
\par 31 Jėzus kalbėjo įtikėjusiems Jį žydams: “Jei jūs pasiliekate mano žodyje, iš tiesų esate mano mokiniai; 
\par 32 ir jūs pažinsite tiesą, ir tiesa padarys jus laisvus”. 
\par 33 Jie Jam atsakė: “Mes esame Abraomo palikuonys ir niekada niekam nevergavome. Kaipgi Tu sakai: ‘Tapsite laisvi’?” 
\par 34 Jėzus jiems tarė: “Iš tiesų, iš tiesų sakau jums: kiekvienas, kas daro nuodėmę, yra nuodėmės vergas. 
\par 35 Bet vergas nepasilieka namuose amžinai, o Sūnus pasilieka amžinai. 
\par 36 Jei tad Sūnus jus išlaisvins, iš tiesų būsite laisvi. 
\par 37 Aš žinau, kad jūs Abraomo palikuonys. Bet jūs norite mane nužudyti, nes mano žodžiui nėra jumyse vietos. 
\par 38 Aš kalbu, ką esu matęs pas savo Tėvą. O jūs darote, ką matėte pas savo tėvą”. 
\par 39 Jie atsiliepė: “Mūsų tėvas Abraomas!” Jėzus jiems tarė: “Jei jūs būtumėte Abraomo vaikai, darytumėte Abraomo darbus. 
\par 40 Bet dabar jūs norite nužudyti mane­žmogų, kuris kalbėjo jums tiesą, girdėtą iš Dievo. Šitaip Abraomas nedarė! 
\par 41 Jūs darote savo tėvo darbus”. Jie atsakė: “Mes nesame pavainikiai ir turime vieną Tėvą­Dievą”. 
\par 42 Jėzus kalbėjo jiems toliau: “Jei Dievas būtų jūsų Tėvas, jūs mylėtumėte mane, nes Aš iš Dievo išėjau ir čia atėjau. Aš ne pats nuo savęs atėjau, bet Jis mane siuntė. 
\par 43 Kodėl gi nesuprantate, ką jums sakau? Todėl, kad negalite mano žodžių klausyti. 
\par 44 Jūsų tėvas­velnias, ir jūs norite vykdyti savo tėvo troškimus. Jis nuo pat pradžios buvo žmogžudys ir nesilaikė tiesos, nes jame nėra tiesos. Kalbėdamas melą, jis kalba, kas jam sava, nes jis melagis ir melo tėvas. 
\par 45 O kadangi Aš kalbu tiesą, jūs netikite manimi. 
\par 46 Kas iš jūsų apkaltins mane nuodėme? Jeigu tad tiesą sakau, kodėl netikite manimi? 
\par 47 Kas iš Dievo yra, tas Dievo žodžių klauso. Jūs neklausote todėl, kad nesate iš Dievo”. 
\par 48 Tuomet žydai atsiliepė: “Argi mes ne teisingai sakome, kad Tu samarietis ir Tavyje demonas?” 
\par 49 Jėzus jiems atsakė: “Nėra manyje demono. Aš gerbiu savo Tėvą, o jūs negerbiate manęs. 
\par 50 Aš neieškau sau šlovės: yra, kas ieško ir teisia. 
\par 51 Iš tiesų, iš tiesų sakau jums: kas laikysis mano žodžio, nematys mirties per amžius”. 
\par 52 Žydai Jam pasakė: “Dabar mes žinome, kad Tu demono apsėstas! Juk mirė Abraomas ir pranašai, o Tu sakai: ‘Kas laikysis mano žodžio, tas neragaus mirties per amžius’. 
\par 53 Argi Tu didesnis už mūsų tėvą Abraomą, kuris mirė? Pranašai irgi mirė. Kuo Tu dediesi?” 
\par 54 Jėzus atsakė: “Jei Aš save šlovinčiau, mano šlovė būtų niekai. Mane šlovina mano Tėvas, apie kurį jūs sakote: ‘Jis yra mūsų Dievas’. 
\par 55 Ir jūs Jo nepažįstate, o Aš Jį pažįstu. Jei sakyčiau, kad Jo nepažįstu, būčiau toks kaip jūs­melagis. Bet Aš Jį pažįstu ir laikausi Jo žodžio. 
\par 56 Jūsų tėvas Abraomas džiūgavo, kad matysiąs manąją dieną; jis ją išvydo ir džiaugėsi”. 
\par 57 Tada žydai Jam sakė: “Dar neturi nė penkiasdešimt metų ir esi matęs Abraomą?!” 
\par 58 Jėzus jiems tarė: “Iš tiesų, iš tiesų sakau jums: pirmiau, negu buvo Abraomas, Aš Esu!” 
\par 59 Tada jie griebėsi akmenų, norėdami mesti į Jį, bet Jėzus pasislėpė ir, praeidamas pro juos, išėjo iš šventyklos ir nuėjo tolyn.


\chapter{9}


\par 1 Eidamas pro šalį, Jėzus pamatė žmogų, aklą gimusį. 
\par 2 Jo mokiniai paklausė: “Rabi, kas nusidėjo,­jis pats ar jo tėvai,­ kad gimė aklas?” 
\par 3 Jėzus atsakė: “Nei jis nusidėjo, nei jo tėvai, bet dėl to, kad jame apsireikštų Dievo darbai. 
\par 4 Man reikia dirbti darbus To, kuris mane siuntė, kol yra diena. Ateina naktis, kada niekas negalės dirbti. 
\par 5 Kol esu pasaulyje, esu pasaulio šviesa!” 
\par 6 Tai taręs, Jis spjovė žemėn, padarė purvo iš seilių, patepė juo neregio akis 
\par 7 ir tarė jam: “Eik ir nusiprausk Siloamo tvenkinyje”. (Išvertus “Siloamas” reiškia: “Pasiųstasis”.) Tas nuėjo, nusiplovė ir sugrįžo regintis. 
\par 8 Kaimynai ir tie, kurie anksčiau matydavo jį aklą, klausė: “Ar čia ne tas, kuris sėdėdavo elgetaudamas?” 
\par 9 Vieni sakė: “Tai jis”. Kiti: “Ne, tik panašus į jį”. O jis atsakė: “Tai aš”. 
\par 10 Tada jie klausė jį: “Kaip atsivėrė tau akys?” 
\par 11 Jis atsakė: “Žmogus, vadinamas Jėzumi, padarė purvo, patepė juo mano akis ir pasakė: ‘Eik į Siloamo tvenkinį ir nusiprausk’. Aš nuėjau, nusiprausiau ir praregėjau”. 
\par 12 Tada jie paklausė: “Kur Jis?” Šis atsakė: “Nežinau”. 
\par 13 Jie nusivedė buvusį neregį pas fariziejus. 
\par 14 O toji diena, kai Jėzus padarė purvo ir atvėrė akis, buvo sabatas. 
\par 15 Fariziejai jį iš naujo paklausė, kaip jis praregėjęs. Tas jiems paaiškino: “Jis uždėjo man ant akių purvo, aš nusiprausiau, ir dabar regiu”. 
\par 16 Kai kurie fariziejai kalbėjo: “Tas žmogus ne iš Dievo, nes nesilaiko sabato”. O kiti sakė: “Kaip galėtų nuodėmingas žmogus daryti tokius ženklus?!” Ir jie tarp savęs nesutarė. 
\par 17 Tuomet jie vėl paklausė buvusį neregį: “O ką tu pasakysi apie žmogų, atvėrusį tau akis?” Šis atsakė: “Jis pranašas”. 
\par 18 Bet žydai netikėjo, kad jis buvo aklas ir praregėjo, kol pašaukė praregėjusiojo tėvus 
\par 19 ir paklausė jų: “Ar šitas jūsų sūnus, kurį sakote gimus aklą? Tai kaip jis dabar regi?” 
\par 20 Jo tėvai jiems atsakė: “Mes žinome, kad jis mūsų sūnus ir kad jis gimė aklas. 
\par 21 O kaip jis praregėjo, mes nežinome, nei kas jam atvėrė akis, nežinome. Klauskite jį patį, jis suaugęs ir pats tegul kalba už save”. 
\par 22 Jo tėvai taip kalbėjo, bijodami žydų. Nes žydai jau buvo nutarę: jei kas tik išpažintų Jėzų esant Kristų, turėtų būti pašalintas iš sinagogos. 
\par 23 Todėl jo tėvai pasakė: “Jis suaugęs, klauskite jį patį”. 
\par 24 Tada jie antrą kartą pasišaukė buvusį neregį ir pasakė jam: “Šlovink Dievą! Mes žinome, kad Tas žmogus nusidėjėlis”. 
\par 25 Jis atsiliepė: “Ar Jis nusidėjėlis, aš nežinau. Viena žinau: buvau aklas, o dabar regiu”. 
\par 26 Jie vėl klausė: “Ką Jis tau darė? Kaip Jis tau atvėrė akis?” 
\par 27 Šis atsakė: “Aš jau sakiau jums, tik jūs neklausėte. Ar dar kartą norite išgirsti? Gal ir jūs norite tapti Jo mokiniais?” 
\par 28 Tada jie išplūdo jį ir pasakė: “Tu esi Jo mokinys, o mes­Mozės mokiniai. 
\par 29 Mes žinome, kad Mozei Dievas kalbėjo, o iš kur šitas, nežinome”. 
\par 30 Žmogus jiems atsakė: “Tai tikrai nuostabu, kad nežinote, iš kur Jis. O juk Jis man atvėrė akis! 
\par 31 Žinome, kad Dievas neišklauso nusidėjėlių. Bet jei kas yra Dievo garbintojas ir vykdo Jo valią­tą Jis išklauso. 
\par 32 Nuo amžių negirdėta, kad kas būtų atvėręs aklo gimusio akis! 
\par 33 Jei šitas nebūtų iš Dievo, Jis nieko negalėtų padaryti”. 
\par 34 Jie atsakė jam: “Tu visas gimęs nuodėmėse ir dar mus mokai?!” Ir išvarė jį lauk. 
\par 35 Jėzus, išgirdęs, kad jie išvarė jį lauk, surado jį ir paklausė: “Ar tiki Dievo Sūnų?” 
\par 36 Šis atsakė: “O kas Jis, Viešpatie, kad Jį tikėčiau?” 
\par 37 Jėzus jam tarė: “Tu jau esi Jį matęs, ir dabar Jis su tavimi kalba”. 
\par 38 Tada jis sušuko: “Tikiu, Viešpatie!”, ir pagarbino Jį. 
\par 39 O Jėzus pasakė: “Aš atėjau į šį pasaulį daryti teismo,­kad neregiai praregėtų, o regintieji apaktų”. 
\par 40 Prie Jo esantys fariziejai, tai išgirdę, paklausė: “Tai gal ir mes akli?” 
\par 41 Jėzus jiems atsakė: “Jei būtumėte akli, neturėtumėte nuodėmės, bet dabar sakote: ‘Mes regime!’­Todėl jūsų nuodėmė pasilieka”.


\chapter{10}


\par 1 “Iš tiesų, iš tiesų sakau jums: kas neįeina pro vartus į avių gardą, bet įlipa pro kur kitur, tas vagis ir plėšikas. 
\par 2 O kas pro vartus įeina, tas avių ganytojas. 
\par 3 Jam sargas atidaro, ir avys klauso jo balso. Jis šaukia savąsias avis vardais ir jas išveda. 
\par 4 Išsivaręs savąsias avis, jis eina priešakyje, o avys paskui jį seka, nes pažįsta jo balsą. 
\par 5 O paskui svetimą jos neseks, bet nuo jo bėgs, nes nepažįsta svetimųjų balso”. 
\par 6 Jėzus pasakė jiems tą palyginimą, bet jie nesuprato, ką tai reiškia. 
\par 7 Tuomet Jėzus kalbėjo jiems toliau: “Iš tiesų, iš tiesų sakau jums: Aš­avių vartai. 
\par 8 Visi, kurie pirma manęs atėjo, buvo vagys ir plėšikai, todėl avys jų neklausė. 
\par 9 Aš esu vartai. Jei kas įeis per mane, bus išgelbėtas. Jis įeis ir išeis ir ganyklą sau ras. 
\par 10 Vagis ateina tik vogti, žudyti ir naikinti. Aš atėjau, kad jie turėtų gyvenimą,­kad apsčiai jo turėtų. 
\par 11 Aš esu gerasis ganytojas. Geras ganytojas už avis guldo savo gyvybę. 
\par 12 O samdinys, ne ganytojas, kuriam avys ne savos, pamatęs sėlinantį vilką, palieka avis ir pabėga, o vilkas griebia jas ir išsklaido. 
\par 13 Samdinys pabėga, nes jis samdinys, jam avys nerūpi. 
\par 14 Aš esu gerasis ganytojas: Aš pažįstu savąsias, ir manosios pažįsta mane. 
\par 15 Kaip mane pažįsta Tėvas, taip ir Aš pažįstu Tėvą ir už avis guldau savo gyvybę. 
\par 16 Ir kitų avių turiu, kurios ne iš šios avidės; ir jas man reikia atvesti; jos girdės mano balsą, ir bus viena kaimenė, vienas ganytojas. 
\par 17 Todėl Tėvas myli mane, kad Aš guldau savo gyvybę, jog ir vėl ją pasiimčiau. 
\par 18 Niekas neatima jos iš manęs, bet Aš pats ją atiduodu. Aš turiu galią ją atiduoti ir turiu galią vėl ją pasiimti; tokį įsakymą gavau iš savo Tėvo”. 
\par 19 Tarp žydų vėl kilo nesutarimas dėl šitų žodžių. 
\par 20 Daugelis iš jų sakė: “Jis turi demoną ir šėlsta. Kodėl Jo klausote?” 
\par 21 Kiti tvirtino: “Tai ne demono apsėstojo kalbos. Argi gali demonas atverti neregiui akis?!” 
\par 22 Jeruzalėje buvo Šventyklos pašventinimo šventė. Buvo žiema. 
\par 23 Jėzus vaikščiojo šventykloje, po Saliamono stoginę. 
\par 24 Ten Jį apstojo žydai ir ėmė klausinėti: “Kaip ilgai laikysi mus abejonėse? Jeigu esi Kristus, pasakyk mums atvirai!” 
\par 25 Jėzus jiems atsakė: “Aš jums sakiau, bet jūs netikite. Darbai, kuriuos darau savo Tėvo vardu, liudija apie mane. 
\par 26 Bet jūs netikite, nes jūs­ne iš manųjų avių, kaip jums ir sakiau. 
\par 27 Manosios avys girdi mano balsą; Aš jas pažįstu, ir jos seka paskui mane. 
\par 28 Aš duodu joms amžinąjį gyvenimą; jos nepražus per amžius, ir niekas jų neišplėš iš mano rankos. 
\par 29 Mano Tėvas, kuris man jas davė, yra aukščiau už viską, ir niekas negali jų išplėšti iš mano Tėvo rankos. 
\par 30 Aš ir Tėvas esame viena”. 
\par 31 Tada žydai vėl stvėrėsi akmenų, norėdami Jį užmėtyti. 
\par 32 O Jėzus paklausė jų: “Daug gerų darbų esu jums parodęs iš savo Tėvo. Už kurį gi darbą užmėtysite mane akmenimis?” 
\par 33 Žydai Jam atsakė: “Ne už gerą darbą užmėtysime Tave akmenimis, bet už piktžodžiavimą ir kad Tu, būdamas žmogus, dediesi Dievu”. 
\par 34 Jėzus jiems atsakė: “Argi jūsų Įstatyme nėra parašyta: ‘Aš tariau: jūs esat dievai’? 
\par 35 Jeigu Jis vadina dievais tuos, kuriems skirtas Dievo žodis (o Raštas negali būti panaikintas), 
\par 36 kaip tad jūs galite sakyti Tam, kurį Tėvas pašventino ir pasiuntė į pasaulį: ‘Tu piktžodžiauji’, kai Jis pasakė: ‘Aš esu Dievo Sūnus!’? 
\par 37 Jei Aš nedarau savo Tėvo darbų,­ netikėkite manimi! 
\par 38 O jeigu darau ir manimi netikite,­tikėkite darbais, kad pažintumėte ir patikėtumėte, jog Tėvas manyje ir Aš Jame”. 
\par 39 Tuomet jie vėl norėjo Jį suimti, bet Jis ištrūko jiems iš rankų. 
\par 40 Jėzus vėl pasitraukė anapus Jordano, kur pirma Jonas krikštijo, ir apsistojo ten. 
\par 41 Daugelis atėjo pas Jį ir kalbėjo: “Jonas nepadarė nė vieno ženklo, bet ką jis sakė apie šį žmogų, yra teisybė”. 
\par 42 Ir daugelis tenai Jį įtikėjo.


\chapter{11}


\par 1 Buvo vienas ligonis, Lozorius iš Betanijos kaimo, kur gyveno Marija ir jos sesuo Morta. 
\par 2 Marija buvo ta pati moteris, kuri patepė Viešpatį kvepalais ir nušluostė savo plaukais Jo kojas. Jos brolis Lozorius sirgo. 
\par 3 Todėl seserys nusiuntė Jam žinią: “Viešpatie! Tas, kurį Tu myli, serga!” 
\par 4 Tai išgirdęs, Jėzus tarė: “Šita liga ne mirčiai, bet Dievo šlovei, kad per ją būtų pašlovintas Dievo Sūnus”. 
\par 5 Jėzus mylėjo Mortą, jos seserį ir Lozorių. 
\par 6 Taigi išgirdęs, kad tasai serga, Jis dar dvi dienas užtruko ten, kur buvo. 
\par 7 Po to pasakė mokiniams: “Eikime vėl į Judėją!” 
\par 8 Mokiniai Jam atsakė: “Rabi, ką tik žydai kėsinosi užmėtyti Tave akmenimis, o Tu vėl ten eini?” 
\par 9 Jėzus tarė: “Argi ne dvylika valandų turi diena? Kas vaikščioja dieną, tas nesuklumpa, nes mato šio pasaulio šviesą. 
\par 10 O kas vaikščioja naktį, suklumpa, nes jame nėra šviesos”. 
\par 11 Tai pasakęs, pridūrė: “Mūsų bičiulis Lozorius užmigo, bet Aš einu jo pažadinti”. 
\par 12 Jo mokiniai atsiliepė: “Viešpatie, jeigu jis miega, pasveiks”. 
\par 13 Tačiau Jėzus kalbėjo apie jo mirtį, o jie manė, kad Jis kalba apie poilsio miegą. 
\par 14 Pagaliau Jėzus atvirai jiems pasakė: “Lozorius mirė. 
\par 15 Ir Aš džiaugiuosi, kad ten nebuvau,­dėl jūsų, kad tikėtumėte. Tad eikime pas jį”. 
\par 16 Tuomet Tomas, vadinamas Dvyniu, tarė kitiems mokiniams: “Eikime ir mes numirti su Juo!” 
\par 17 Atėjęs Jėzus rado Lozorių jau keturias dienas išgulėjusį kape. 
\par 18 O Betanija buvo arti Jeruzalės, maždaug penkiolika stadijų atstu. 
\par 19 Daug žydų buvo atėję pas Mortą ir Mariją paguosti jų dėl brolio. 
\par 20 Morta, išgirdusi ateinant Jėzų, išėjo Jo pasitikti. Marija sėdėjo namie. 
\par 21 Morta tarė Jėzui: “Viešpatie, jei būtum čia buvęs, mano brolis nebūtų miręs. 
\par 22 Bet ir dabar žinau: ko tik prašysi Dievo, Dievas Tau duos”. 
\par 23 Jėzus jai pasakė: “Tavo brolis prisikels!” 
\par 24 Morta atsiliepė: “Aš žinau, jog jis prisikels prisikėlime, paskutinę dieną”. 
\par 25 Jėzus jai tarė: “Aš esu prisikėlimas ir gyvenimas. Kas tiki mane, nors ir mirtų, bus gyvas. 
\par 26 Ir kiekvienas, kuris gyvena ir tiki mane, nemirs per amžius. Ar tai tiki?” 
\par 27 Ji atsakė: “Taip, Viešpatie! Aš tikiu, jog Tu esi Kristus, Dievo Sūnus, kuris turi ateiti į šį pasaulį”. 
\par 28 Tai pasakiusi, ji nuėjo ir slapčiomis pašaukė savo seserį Mariją, jai pranešdama: “Mokytojas atėjo ir šaukia tave”. 
\par 29 Išgirdusi ši greitai pakilo ir nuėjo pas Jį. 
\par 30 Jėzus dar nebuvo įėjęs į kaimą, bet tebebuvo toje vietoje, kur Jį pasitiko Morta. 
\par 31 Kai žydai, buvę su ja namuose ir ją guodę, pamatė ją skubiai keliantis ir išeinant, nusekė paskui, sakydami: “Ji eina prie kapo”. 
\par 32 Marija, atėjusi ten, kur buvo Jėzus, ir Jį pamačiusi, puolė Jam po kojų, sakydama: “Viešpatie, jei būtum čia buvęs, mano brolis nebūtų miręs”. 
\par 33 Jėzus, pamatęs ją ir kartu atėjusius žydus verkiančius, sudejavo dvasioje ir susijaudinęs 
\par 34 paklausė: “Kur jį paguldėte?” Jie atsakė: “Viešpatie, eik ir pažiūrėk”. 
\par 35 Jėzus pravirko. 
\par 36 Tada žydai ėmė kalbėti: “Štai kaip Jis jį mylėjo!” 
\par 37 O kiti sakė: “Argi Tas, kuris atvėrė neregiui akis, negalėjo padaryti, kad šitas nemirtų?” 
\par 38 Ir vėl dejuodamas Jėzus atėjo prie kapo. Tai buvo ola, užrista akmeniu. 
\par 39 Jėzus tarė: “Nuriskite akmenį!” Mirusiojo sesuo Morta įspėjo: “Viešpatie, jis jau dvokia, nes jau keturios dienos, kai jis miręs”. 
\par 40 Jėzus jai tarė: “Argi nesakiau tau, kad jei tikėsi, pamatysi Dievo šlovę?!” 
\par 41 Jie nurito akmenį nuo ten, kur gulėjo numirėlis. Jėzus pakėlė akis aukštyn ir prabilo: “Tėve, dėkoju Tau, kad mane išgirdai. 
\par 42 Aš žinojau, kad visada mane girdi. Tačiau tai sakau dėl čia esančiųjų, kad jie tikėtų, jog Tu esi mane siuntęs”. 
\par 43 Tai pasakęs, Jis galingu balsu sušuko: “Lozoriau, išeik!” 
\par 44 Ir numirėlis išėjo. Jo rankos ir kojos buvo apvyniotos laidojimo aprišalais, veidas aprištas drobule. Jėzus jiems liepė: “Atvyniokite jį ir leiskite jam eiti”. 
\par 45 Daugelis žydų, kurie buvo atėję pas Mariją ir matė, ką Jėzus padarė, įtikėjo Jį. 
\par 46 Bet kai kurie nuėjo pas fariziejus ir pranešė jiems, ką Jėzus padaręs. 
\par 47 Tada aukštieji kunigai ir fariziejai sušaukė sinedrioną ir svarstė: “Ką darysime? Šitas žmogus daro daug ženklų. 
\par 48 Jei taip Jį paliksime, visi įtikės Jį; ateis romėnai ir užims mūsų vietą bei tautą”. 
\par 49 Vienas iš jų­Kajafas, tais metais vyriausiasis kunigas­jiems tarė: “Jūs nieko neišmanote 
\par 50 ir nepagalvojate, jog mums geriau, kad vienas žmogus mirtų už tautą, o ne visa tauta žūtų”. 
\par 51 Jis tai pasakė ne iš savęs, bet, būdamas tų metų vyriausiasis kunigas, pranašavo, jog Jėzui reikės mirti už tautą, 
\par 52 ir ne tik už tautą, bet tam, kad suburtų į viena išsklaidytuosius Dievo vaikus. 
\par 53 Nuo tos dienos jie buvo nusprendę Jį nužudyti. 
\par 54 Todėl Jėzus nebevaikščiojo viešai tarp žydų, bet pasitraukė iš ten į vietovę netoli dykumos, į miestelį, vadinamą Efraimu, ir ten apsistojo kartu su savo mokiniais. 
\par 55 Artinosi žydų Pascha. Daugelis iš viso krašto prieš Paschą atėjo į Jeruzalę apsivalyti. 
\par 56 Jie ieškojo Jėzaus ir, stoviniuodami šventykloje, kalbėjosi: “Kaip jūs manote? Nejaugi Jis nebeateis į šventę?” 
\par 57 Mat aukštieji kunigai ir fariziejai išleido įsakymą, kad žinantys praneštų, kur Jis esąs, kad galėtų Jį suimti.


\chapter{12}


\par 1 Šešioms dienoms belikus iki Paschos, Jėzus atėjo į Betaniją, kur gyveno jo prikeltasis iš numirusių Lozorius. 
\par 2 Ten iškėlė Jam vaišes. Morta patarnavo, o Lozorius kartu su kitais sėdėjo prie stalo su Jėzumi. 
\par 3 Paėmusi svarą labai brangaus tepalo iš gryno nardo, Marija patepė Jėzui kojas ir nušluostė jas savo plaukais. Namuose pasklido tepalo kvapas. 
\par 4 Vienas iš Jo mokinių, Simono sūnus Judas Iskarijotas, kuris turėjo Jį išduoti, pasakė: 
\par 5 “Kodėl to tepalo neparduoda už tris šimtus denarų ir neatiduoda vargšams?” 
\par 6 Jis taip sakė ne todėl, kad jam būtų rūpėję vargšai, bet kad pats buvo vagis ir, turėdamas kasą, grobstė įplaukas. 
\par 7 O Jėzus tarė: “Palik ją ramybėje. Ji tai laikė mano laidotuvių dienai. 
\par 8 Vargšų jūs visada turėsite su savimi, o mane ne visuomet turėsite”. 
\par 9 Daug žydų sužinojo Jėzų ten esant ir atėjo ne tik dėl Jėzaus, bet taip pat pamatyti Lozoriaus, kurį Jis prikėlė iš numirusių. 
\par 10 O aukštieji kunigai nusprendė nužudyti ir Lozorių, 
\par 11 nes daugybė žydų per jį pasitraukė nuo jų ir įtikėjo Jėzų. 
\par 12 Kitą dieną gausi minia, susirinkusi į šventę, sužinojo, kad Jėzus ateinąs į Jeruzalę. 
\par 13 Žmonės pasiėmė palmių šakų ir išėjo Jo pasitikti, šaukdami: “Osana! Palaimintas Tas, kuris ateina Viešpaties vardu­Izraelio karalius!” 
\par 14 Jėzus, radęs asiliuką, užsėdo ant jo, kaip parašyta: 
\par 15 “Nebijok, Siono dukra: štai atvyksta tavo karalius, sėdėdamas ant asilaičio!” 
\par 16 Iš pradžių mokiniai šito nesuprato, bet, kai Jėzus buvo pašlovintas, atsiminė, kad tai buvo apie Jį parašyta ir jie buvo Jam tai padarę. 
\par 17 Taigi dabar liudijo minia, buvusi su Juo, kai Jis pašaukė Lozorių iš kapo ir prikėlė iš numirusių. 
\par 18 Žmonės todėl ir išėjo Jo pasitikti, kad buvo girdėję Jį padarius tą ženklą. 
\par 19 O fariziejai kalbėjo vieni kitiems: “Žiūrėkite, jūs nieko negalite padaryti. Štai visas pasaulis nuėjo paskui Jį!” 
\par 20 Tarp atėjusių per šventes pagarbinti buvo ir graikų. 
\par 21 Jie priėjo prie Pilypo, kilusio iš Galilėjos miesto Betsaidos, ir paprašė: “Gerbiamasis, mes norėtume pamatyti Jėzų”. 
\par 22 Pilypas nuėjo ir pasakė Andriejui. Paskui Andriejus ir Pilypas pranešė Jėzui. 
\par 23 O Jėzus jiems tarė: “Atėjo valanda, kad būtų pašlovintas Žmogaus Sūnus. 
\par 24 Iš tiesų, iš tiesų sakau jums: jei kviečio grūdas nekris į žemę ir nenumirs, jis pasiliks vienas, o jei numirs, jis duos gausių vaisių. 
\par 25 Kas myli savo gyvybę, ją praras, o kas nekenčia savo gyvybės šiame pasaulyje, išsaugos ją amžinajam gyvenimui. 
\par 26 Kas man tarnauja, tegul seka paskui mane; ir kur Aš esu, ten bus ir mano tarnas. Kas man tarnauja, tą pagerbs mano Tėvas. 
\par 27 Dabar mano siela sukrėsta. Ir ką Aš pasakysiu: ‘Tėve, gelbėk mane nuo šios valandos!’? Bet juk tam Aš ir atėjau į šią valandą. 
\par 28 Tėve, pašlovink savo vardą!” Tada iš dangaus pasigirdo balsas: “Aš jį pašlovinau ir dar pašlovinsiu!” 
\par 29 Aplink stovinti minia, tai išgirdusi, sakė griaustinį sugriaudus. Kai kurie tvirtino: “Angelas Jam kalbėjo”. 
\par 30 O Jėzus atsakė: “Ne dėl manęs, o dėl jūsų pasigirdo šitas balsas. 
\par 31 Dabar teisiamas šitas pasaulis. Dabar šio pasaulio kunigaikštis bus išmestas laukan. 
\par 32 O Aš, kai būsiu pakeltas nuo žemės, visus patrauksiu prie savęs”. 
\par 33 Jis tai pasakė, nurodydamas, kokia mirtimi Jam reikės mirti. 
\par 34 O žmonės Jam sakė: “Mes girdėjome iš Įstatymo, kad Kristus pasilieka per amžius. Kodėl Tu sakai, kad Žmogaus Sūnus turės būti iškeltas aukštyn? Kas gi tas Žmogaus Sūnus?” 
\par 35 Jėzus jiems tarė: “Dar trumpą laiką šviesa bus su jumis. Vaikščiokite, kol turite šviesą, kad neužkluptų jūsų tamsa. Kas vaikščioja tamsoje, tas nežino, kur eina. 
\par 36 Kol turite šviesą, tikėkite ją, kad taptumėte šviesos vaikais”. Tai pasakęs, Jėzus pasišalino ir pasislėpė nuo jų. 
\par 37 Nors Jis jų akivaizdoje padarė tiek daug ženklų, jie Juo netikėjo,­ 
\par 38 kad išsipildytų pranašo Izaijo žodžiai: “Viešpatie, kas patikėjo mūsų skelbimu ir kam buvo apreikšta Viešpaties rankos galybė?” 
\par 39 Jie neįstengė tikėti, nes vėl, anot Izaijo: 
\par 40 “Jis apakino jų akis ir sukietino jų širdį, kad nematytų akimis ir nesuvoktų širdimi, ir neatsiverstų, ir Aš jų nepagydyčiau”. 
\par 41 Izaijas tai pasakė, regėdamas Jo šlovę ir kalbėdamas apie Jį. 
\par 42 Vis dėlto daugelis įtikėjo Jėzų net iš vyresnybės, tačiau dėl fariziejų Jo neišpažino, kad nebūtų pašalinti iš sinagogos, 
\par 43 nes žmonių šlovę jie brangino labiau už Dievo šlovę. 
\par 44 O Jėzus šaukė: “Kas mane tiki, tiki ne mane, bet Tą, kuris mane siuntė. 
\par 45 Ir kas mane mato, mato Tą, kuris mane siuntė. 
\par 46 Aš atėjau į pasaulį kaip šviesa, kad visi, kurie mane tiki, neliktų tamsoje. 
\par 47 Jei kas klausosi mano žodžių ir netiki, Aš jo neteisiu, nes atėjau ne teisti pasaulio, bet gelbėti pasaulį. 
\par 48 Kas mane atstumia ir mano žodžių nepriima, tas turi savo teisėją: žodis, kurį kalbėjau, teis jį paskutiniąją dieną. 
\par 49 Nes Aš kalbėjau ne iš savęs,­Tėvas, kuris mane siuntė, įsakė man, ką turiu sakyti ir ką skelbti. 
\par 50 Aš žinau, kad Jo įsakymas yra amžinasis gyvenimas. Tad ką Aš kalbu, kalbu taip, kaip Tėvas yra man sakęs”.


\chapter{13}


\par 1 Prieš Paschos šventę Jėzus, žinodamas, kad atėjo Jo valanda iš šio pasaulio keliauti pas Tėvą, ir mylėdamas savuosius pasaulyje, parodė jiems savo meilę iki galo. 
\par 2 Vakarienės metu, kai velnias jau buvo įkvėpęs Simono sūnaus Judo Iskarijoto širdin sumanymą išduoti Jį, 
\par 3 žinodamas, kad Tėvas visa atidavęs į Jo rankas ir kad Jis išėjo iš Dievo ir eina pas Dievą, 
\par 4 Jėzus pakilo nuo stalo, nusivilko viršutinius drabužius ir persijuosė rankšluosčiu. 
\par 5 Po to įpylė vandens į praustuvą ir ėmė plauti mokiniams kojas bei šluostyti rankšluosčiu, kuriuo buvo persijuosęs. 
\par 6 Taip Jis priėjo prie Simono Petro. Šis Jam tarė: “Viešpatie, nejaugi Tu plausi man kojas?” 
\par 7 Jėzus jam atsakė: “Tu dabar nesupranti, ką Aš darau, bet vėliau suprasi”. 
\par 8 Petras atsiliepė: “Tu neplausi man kojų per amžius!” Jėzus jam atsakė: “Jei nenuplausiu tavęs, neturėsi dalies su manimi”. 
\par 9 Tada Simonas Petras sušuko: “Viešpatie, ne tik mano kojas, bet ir rankas, ir galvą!” 
\par 10 Jėzus jam atsakė: “Kas nuplautas, tam reikia tik kojas nusiplauti, nes jis visas švarus. Ir jūs esate švarūs, bet ne visi”. 
\par 11 Nes Jis žinojo, kas Jį išduos, ir todėl sakė: “Jūs ne visi švarūs”. 
\par 12 Nuplovęs jų kojas, Jis užsivilko drabužius ir atsisėdęs paklausė: “Ar supratote, ką jums padariau? 
\par 13 Jūs vadinate mane ‘Mokytoju’ ir ‘Viešpačiu’, ir gerai sakote, nes Aš Tas esu. 
\par 14 Jei tad Aš­Viešpats ir Mokytojas­nuploviau jums kojas, tai ir jūs turite vieni kitiems kojas plauti. 
\par 15 Aš jums daviau pavyzdį, kad Jūs darytumėte, kaip Aš jums dariau. 
\par 16 Iš tiesų, iš tiesų sakau jums: tarnas ne didesnis už savo šeimininką ir pasiuntinys ne didesnis už savo siuntėją. 
\par 17 Jeigu tai suprantate, palaiminti esate, taip elgdamiesi. 
\par 18 Ne apie jus visus tai sakau. Aš žinau, ką išsirinkau, bet turi išsipildyti Raštas: ‘Tas, kuris valgo su manimi duoną, pakėlė virš manęs savo kulną’. 
\par 19 Sakau jums dabar, prieš įvykstant, kad įvykus tikėtumėte, jog Aš Esu. 
\par 20 Iš tiesų, iš tiesų sakau jums: kas priima mano pasiuntinį, tas priima mane, o kas mane priima, priima Tą, kuris mane siuntė”. 
\par 21 Tai pasakęs, Jėzus, sukrėstas dvasioje, tarė: “Iš tiesų, iš tiesų sakau jums: vienas iš jūsų išduos mane!” 
\par 22 Tada mokiniai ėmė žvalgytis vienas į kitą, nesusivokdami, apie kurį Jis taip pasakė. 
\par 23 Vienas iš Jo mokinių, kurį Jėzus mylėjo, buvo prisiglaudęs prie Jėzaus krūtinės. 
\par 24 Simonas Petras pamojo jam, kad šis paklaustų, apie kurį Jis kalba. 
\par 25 Tasai, pasilenkęs prie Jėzaus krūtinės, paklausė: “Kas jis, Viešpatie?” 
\par 26 Jėzus atsiliepė: “Tai tas, kuriam padažęs paduosiu kąsnį duonos”. Ir, pamirkęs duoną, Jis padavė Judui Iskarijotui, Simono sūnui. 
\par 27 Ir po šio kąsnio įėjo į jį šėtonas. Tada Jėzus jam pasakė: “Ką darai, daryk greičiau!” 
\par 28 Bet nė vienas iš esančių prie stalo nesuprato, kodėl Jis jam taip pasakė. 
\par 29 Kadangi Judo žinioje buvo kasa, kai kurie manė, jog Jėzus jam liepė: “Nupirk, ko mums reikia šventei”, arba kad jis duotų ką nors vargšams. 
\par 30 Taigi, paėmęs duonos kąsnį, anas tuojau išėjo. Buvo naktis. 
\par 31 Jam išėjus, Jėzus prabilo: “Dabar Žmogaus Sūnus pašlovintas, ir Dievas pašlovintas Jame. 
\par 32 O jeigu Dievas pašlovintas Jame, tai Dievas pašlovins Jį savyje,­bematant Jį pašlovins”. 
\par 33 “Vaikeliai, Aš jau nebeilgai būsiu su jumis. Jūs ieškosite manęs, bet sakau jums tą patį, ką žydams pasakiau: kur Aš einu, jūs negalite nueiti. 
\par 34 Aš jums duodu naują įsakymą, kad jūs vienas kitą mylėtumėte: kaip Aš jus pamilau, kad ir jūs mylėtumėte vienas kitą. 
\par 35 Iš to visi pažins, kad esate mano mokiniai, jei mylėsite vieni kitus”. 
\par 36 Simonas Petras Jį paklausė: “Viešpatie, kur Tu eini?” Jėzus atsakė: “Kur Aš einu, tu dabar negali paskui mane sekti, bet vėliau nuseksi mane”. 
\par 37 Petras vėl klausė: “Viešpatie, kodėl gi negaliu dabar paskui Tave sekti? Aš savo gyvybę už Tave guldysiu!” 
\par 38 Jėzus atsakė: “Tu guldysi už mane gyvybę? Iš tiesų, iš tiesų sakau tau: dar gaidžiui nepragydus, tu tris kartus manęs išsiginsi!”


\chapter{14}


\par 1 “Tenebūgštauja jūsų širdys! Tikite Dievą­tikėkite ir mane! 
\par 2 Mano Tėvo namuose daug buveinių. Jeigu taip nebūtų, būčiau jums pasakęs. Einu jums vietos paruošti. 
\par 3 Kai nuėjęs paruošiu, vėl sugrįšiu ir jus pas save pasiimsiu, kad jūs būtumėte ten, kur ir Aš. 
\par 4 Kur Aš einu, jūs žinote, ir kelią jūs žinote”. 
\par 5 Tomas Jam sako: “Viešpatie, mes nežinome, kur Tu eini, tai kaip žinosime kelią?” 
\par 6 Jėzus jam sako: “Aš esu kelias, tiesa ir gyvenimas. Niekas nenueina pas Tėvą kitaip, kaip tik per mane. 
\par 7 Jeigu pažinote mane, tai pažinsite ir mano Tėvą. Jau nuo šiol Jį pažįstate ir esate Jį matę”. 
\par 8 Pilypas Jam sako: “Viešpatie, parodyk mums Tėvą, ir bus mums gana”. 
\par 9 Jėzus jam taria: “Jau tiek laiko esu su jumis, ir tu, Pilypai, vis dar manęs nepažįsti? Kas matė mane, matė Tėvą! Tad kaip gali sakyti: ‘Parodyk mums Tėvą’? 
\par 10 Nejaugi tu netiki, kad Aš esu Tėve ir Tėvas manyje? Žodžius, kuriuos jums kalbu, ne iš savęs kalbu; Tėvas, esantis manyje­Jis daro darbus. 
\par 11 Tikėkite manimi, kad Aš esu Tėve ir Tėvas manyje, arba tikėkite mane dėl pačių darbų! 
\par 12 Iš tiesų, iš tiesų sakau jums: kas mane tiki, darys darbus, kuriuos Aš darau, ir dar už juos didesnių darys, nes Aš einu pas savo Tėvą. 
\par 13 Ir ko tik prašysite mano vardu, Aš padarysiu, kad Tėvas būtų pašlovintas Sūnuje. 
\par 14 Ko tik prašysite mano vardu, Aš padarysiu. 
\par 15 Jei mylite mane, laikykitės mano įsakymų. 
\par 16 Ir Aš paprašysiu Tėvą, ir Jis duos jums kitą Guodėją, kad Jis liktų su jumis per amžius. 
\par 17 Tiesos Dvasią, kurios pasaulis neįstengia priimti, nes Jos nemato ir nepažįsta. O jūs Ją pažįstate, nes Ji yra su jumis ir bus jumyse. 
\par 18 Nepaliksiu jūsų našlaičiais­ ateisiu pas jus. 
\par 19 Dar valandėlė, ir pasaulis manęs nebematys. O jūs mane matysite, nes Aš gyvenu ir jūs gyvensite. 
\par 20 Tą dieną jūs suprasite, kad Aš esu savo Tėve, ir jūs manyje, ir Aš jumyse. 
\par 21 Kas žino mano įsakymus ir jų laikosi, tas myli mane. O kas mane myli, tą mylės mano Tėvas, ir Aš jį mylėsiu ir pats jam apsireikšiu”. 
\par 22 Judas­ne Iskarijotas­paklausė: “Viešpatie, kas atsitiko, jog ketini apsireikšti mums, o ne pasauliui?” 
\par 23 Jėzus jam atsakė: “Jei kas mane myli, laikysis mano žodžio ir mano Tėvas jį mylės; mes pas jį ateisime ir apsigyvensime. 
\par 24 Kas manęs nemyli, mano žodžių nesilaiko. O žodis, kurį girdite, ne mano, bet Tėvo, kuris mane siuntė. 
\par 25 Aš jums tai pasakiau, būdamas su jumis, 
\par 26 o Guodėjas­Šventoji Dvasia, kurią mano vardu Tėvas atsiųs,­ mokys jus visko ir viską primins, ką jums sakiau. 
\par 27 Aš palieku jums ramybę, duodu jums savo ramybę. Ne kaip pasaulis duoda, Aš jums duodu. Tenebūgštauja jūsų širdys ir teneišsigąsta. 
\par 28 Jūs girdėjote, kaip Aš pasakiau: ‘Aš išeinu ir vėl sugrįšiu pas jus!’ Jei mylėtumėte mane, džiaugtumėtės, kad pasakiau: ‘Einu pas Tėvą’, nes mano Tėvas už mane didesnis. 
\par 29 Ir dabar, prieš įvykstant, jums tai pasakiau, kad tikėtumėte, kada tai įvyks. 
\par 30 Jau nebedaug su jumis kalbėsiu, nes ateina šio pasaulio kunigaikštis, ir manyje jis neturi nieko. 
\par 31 Bet pasaulis turi pažinti, jog Aš myliu Tėvą ir darau taip, kaip Jis man įsakė.­Kelkitės, eikime iš čia!”


\chapter{15}


\par 1 “Aš esu tikrasis vynmedis, o mano Tėvas­vynininkas. 
\par 2 Kiekvieną šakelę manyje, neduodančią vaisiaus, Jis išpjauna, o kiekvieną šakelę, nešančią vaisių, apvalo, kad ji duotų daugiau vaisiaus. 
\par 3 Jūs jau esate švarūs dėl žodžio, kurį jums kalbėjau. 
\par 4 Pasilikite manyje, ir Aš jumyse. Kaip šakelė negali duoti vaisiaus pati iš savęs, nepasilikdama vynmedyje,­taip ir jūs, jei nepasiliksite manyje. 
\par 5 Aš esu vynmedis, o jūs šakelės. Kas pasilieka manyje ir Aš jame, tas duoda daug vaisių; nes be manęs jūs negalite nieko nuveikti. 
\par 6 Kas nepasiliks manyje, bus išmestas laukan ir sudžius kaip šakelė. Paskui surinks šakeles, įmes į ugnį, ir jos sudegs. 
\par 7 Jei pasiliksite manyje ir mano žodžiai pasiliks jumyse,­jūs prašysite, ko tik norėsite, ir bus jums duota. 
\par 8 Tuo bus pašlovintas mano Tėvas, kad duosite gausių vaisių ir būsite mano mokiniai. 
\par 9 Kaip mane Tėvas pamilo, taip ir Aš jus pamilau. Pasilikite mano meilėje! 
\par 10 Jei laikysitės mano įsakymų, pasiliksite mano meilėje, kaip Aš kad vykdau savo Tėvo įsakymus ir pasilieku Jo meilėje. 
\par 11 Aš jums tai kalbėjau, kad jumyse liktų manasis džiaugsmas ir kad jūsų džiaugsmui nieko netrūktų. 
\par 12 Tai yra mano įsakymas, kad mylėtumėte vienas kitą, kaip Aš jus pamilau. 
\par 13 Niekas neturi didesnės meilės kaip tas, kuris savo gyvybę už draugus atiduoda. 
\par 14 Jūs esate mano draugai, jei darote, ką jums įsakau. 
\par 15 Jau nebevadinu jūsų tarnais, nes tarnas nežino, ką veikia jo šeimininkas. Jus Aš draugais vadinu, nes jums viską paskelbiau, ką iš savo Tėvo girdėjau. 
\par 16 Ne jūs mane išsirinkote, bet Aš jus išsirinkau ir paskyriau, kad eitumėte, duotumėte vaisių ir jūsų vaisiai išliktų,­kad ko tik prašytumėte Tėvą mano vardu, Jis jums duotų. 
\par 17 Aš jums tai įsakau: mylėkite vienas kitą!” 
\par 18 “Jei pasaulis jūsų nekenčia, žinokite, kad manęs jis nekentė pirmiau negu jūsų. 
\par 19 Jei jūs būtumėte pasaulio, jis mylėtų jus kaip savuosius. Kadangi jūs­ne pasaulio, bet Aš jus iš pasaulio išskyriau, todėl jis jūsų nekenčia. 
\par 20 Atsiminkite žodžius, kuriuos jums sakiau: ‘Tarnas ne didesnis už savo šeimininką!’ Jei persekiojo mane, tai ir jus persekios; jeigu laikėsi mano žodžio, laikysis ir jūsų. 
\par 21 Bet visa tai jums darys dėl mano vardo, nes jie nepažįsta To, kuris mane siuntė. 
\par 22 Jei nebūčiau atėjęs ir jiems kalbėjęs, jie neturėtų nuodėmės. O dabar jie neturi kuo pateisinti savo nuodėmės. 
\par 23 Kas manęs nekenčia, nekenčia ir mano Tėvo. 
\par 24 Jeigu nebūčiau tarp jų padaręs darbų, kurių niekas kitas nedarė, jie neturėtų nuodėmės. O dabar jie matė, ir nekenčia ir manęs, ir mano Tėvo. 
\par 25 Bet kad išsipildytų užrašytas Įstatyme žodis: ‘Jie manęs nekentė be priežasties’. 
\par 26 Kai ateis Guodėjas, kurį jums atsiųsiu nuo Tėvo,­Tiesos Dvasia, kuri eina iš Tėvo,­Jis liudys apie mane. 
\par 27 Ir jūs liudysite, nes nuo pradžios su manimi buvote”.


\chapter{16}


\par 1 “Aš jums tai pasakiau, kad nepasipiktintumėte. 
\par 2 Jie šalins jus iš sinagogų, ir ateina valanda, kada tie, kurie jus žudys, tarsis tarnaują Dievui. 
\par 3 Jie jums tai darys, nes nei Tėvo, nei manęs nepažįsta. 
\par 4 Aš jums visa tai kalbėjau, kad tai valandai atėjus, atsimintumėte, jog buvau jus įspėjęs. Aš jums to nesakiau iš pradžių, nes buvau su jumis. 
\par 5 O dabar einu pas Tą, kuris mane siuntė, ir niekas iš jūsų neklausia: ‘Kur Tu eini?’ 
\par 6 Kadangi jums tai pasakiau, liūdesys jūsų širdis užliejo. 
\par 7 Bet sakau jums tiesą: jums geriau, kad Aš išeinu, nes jei neišeičiau, pas jus neateitų Guodėjas. O nuėjęs Jį jums atsiųsiu. 
\par 8 Atėjęs Jis įtikins pasaulį dėl nuodėmės, dėl teisumo ir dėl teismo. 
\par 9 Dėl nuodėmės,­kad netiki manimi. 
\par 10 Dėl teisumo,­kad Aš pas savo Tėvą einu, ir jūs manęs daugiau nebematysite. 
\par 11 Dėl teismo,­kad šio pasaulio kunigaikštis yra nuteistas. 
\par 12 Dar daug ką turiu jums pasakyti, bet dabar negalite pakelti. 
\par 13 Bet kai ateis Ji, Tiesos Dvasia, jus Ji įves į visą tiesą. Ji nekalbės iš savęs, bet kalbės, ką išgirs, ir praneš, kas dar turi įvykti. 
\par 14 Ji pašlovins mane, nes ims iš to, kas mano, ir jums tai paskelbs. 
\par 15 Visa, ką turi Tėvas, yra mano, todėl Aš pasakiau, kad Ji ims iš to, kas mano, ir jums tai paskelbs. 
\par 16 Prabėgs valandėlė­ir manęs nematysite, ir dar valandėlė­ir vėl mane pamatysite, nes Aš pas Tėvą einu”. 
\par 17 Tada kai kurie mokiniai ėmė vienas kitą klausinėti: “Ką reiškia Jo pasakyti žodžiai: ‘Prabėgs valandėlė­ir manęs nematysite, ir dar valandėlė­ir vėl mane pamatysite’ ir: ‘Aš pas Tėvą einu’?” 
\par 18 Tad jie klausinėjo: “Ką reiškia ‘valandėlė’? Mums neaišku, ką Jis kalba”. 
\par 19 Supratęs, kad jie norėjo Jį klausti, Jėzus tarė: “Klausinėjate vieni kitus dėl mano žodžių: ‘Prabėgs valandėlė­ir manęs nematysite, ir dar valandėlė­ir vėl mane pamatysite’? 
\par 20 Iš tiesų, iš tiesų sakau jums: jūs verksite ir vaitosite, o pasaulis džiaugsis. Jūs liūdėsite, bet jūsų liūdesys pavirs džiaugsmu. 
\par 21 Gimdydama moteris būna sielvarto prislėgta, nes atėjo jos valanda, bet, kūdikiui gimus, ji skausmą užmiršta iš džiaugsmo, kad gimė į pasaulį žmogus. 
\par 22 Taip ir jūs dabar nuliūdę, bet Aš vėl jus pamatysiu; ir jūsų širdys džiūgaus, ir jūsų džiaugsmo niekas iš jūsų nebeatims. 
\par 23 Tą dieną jūs manęs nieko neklausinėsite. Iš tiesų, iš tiesų sakau jums: ko tik prašysite Tėvą mano vardu, Jis jums duos. 
\par 24 Iki šiol jūs nieko neprašėte mano vardu. Prašykite ir gausite, kad jūsų džiaugsmui nieko netrūktų. 
\par 25 Aš jums kalbėjau palyginimais, bet ateina valanda, kada nebekalbėsiu jums palyginimais, bet atvirai apie Tėvą jums skelbsiu. 
\par 26 Tą dieną jūs prašysite mano vardu, ir Aš nesakau, kad Aš prašysiu Tėvą už jus,­ 
\par 27 juk pats Tėvas jus myli, nes jūs mane pamilote ir įtikėjote, jog Aš esu iš Dievo išėjęs. 
\par 28 Išėjau iš Tėvo ir atėjau į pasaulį. Vėl palieku pasaulį ir einu pas Tėvą”. 
\par 29 Mokiniai tarė: “Štai dabar Tu aiškiai kalbi ir nebesakai jokių palyginimų. 
\par 30 Mes dabar suprantame, kad Tu viską žinai ir nereikia, kad kas Tave klausinėtų. Todėl tikime, kad Tu esi išėjęs iš Dievo”. 
\par 31 Jėzus jiems atsakė: “Dabar tikite? 
\par 32 Štai ateina valanda,­ir jau atėjo,­kai jūs išsisklaidysite kas sau ir paliksite mane vieną. Tačiau Aš ne vienas, nes su manimi yra Tėvas. 
\par 33 Aš jums tai kalbėjau, kad manyje turėtumėte ramybę. Pasaulyje jūs turėsite priespaudą, bet būkite drąsūs: Aš nugalėjau pasaulį!”


\chapter{17}


\par 1 Tai pasakęs, Jėzus pakėlė akis į dangų ir prabilo: “Tėve, atėjo valanda! Pašlovink savo Sūnų, kad ir Tavo Sūnus pašlovintų Tave; 
\par 2 nes Tu davei Jam valdžią kiekvienam kūnui, kad visiems, kuriuos esi Jam davęs, Jis teiktų amžinąjį gyvenimą. 
\par 3 Tai yra amžinasis gyvenimas: kad jie pažintų Tave, vienintelį tikrąjį Dievą ir Tavo siųstąjį Jėzų Kristų. 
\par 4 Aš pašlovinau Tave žemėje. Atlikau darbą, kurį buvai man davęs nuveikti. 
\par 5 Dabar Tu, Tėve, pašlovink mane pas save ta šlove, kurią pas Tave turėjau dar prieš pasaulio buvimą. 
\par 6 Aš apreiškiau Tavo vardą žmonėms, kuriuos man davei iš pasaulio. Jie buvo Tavo, ir Tu juos davei man, ir jie laikėsi Tavojo žodžio. 
\par 7 Dabar jie suprato, kad visa, ką man davei, yra iš Tavęs. 
\par 8 Nes Tavo man duotus žodžius Aš perdaviau jiems, ir jie priėmė juos ir tikrai pažino, kad iš Tavęs išėjau, ir jie įtikėjo, kad mane siuntei. 
\par 9 Aš meldžiu už juos. Ne už pasaulį meldžiu, bet už tuos, kuriuos man davei, nes jie yra Tavo! 
\par 10 Ir visa, kas mano, yra Tavo, o kas Tavo, yra mano, ir Aš pašlovintas juose. 
\par 11 Aš jau nebe pasaulyje, bet jie dar pasaulyje. Aš einu pas Tave. Šventasis Tėve, išlaikyk juos savo vardu­tuos, kuriuos man davei, kad jie būtų viena kaip ir mes. 
\par 12 Kol buvau su jais pasaulyje, Aš išlaikiau juos Tavo vardu; Aš išsaugojau tuos, kuriuos man davei, ir nė vienas iš jų nepražuvo, išskyrus pražūties sūnų, kad išsipildytų Raštas. 
\par 13 Bet dabar Aš einu pas Tave ir tai kalbu pasaulyje, kad jie turėtų savyje tobulą mano džiaugsmą. 
\par 14 Aš jiems daviau Tavo žodį, ir pasaulis jų nekentė, nes jie ne iš pasaulio, kaip ir Aš ne iš pasaulio. 
\par 15 Aš neprašau, kad juos paimtum iš pasaulio, bet kad apsaugotum juos nuo pikto. 
\par 16 Jie nėra iš pasaulio, kaip ir Aš ne iš pasaulio. 
\par 17 Pašventink juos savo tiesa! Tavo žodis yra tiesa. 
\par 18 Kaip Tu mane siuntei į pasaulį, taip ir Aš juos pasiunčiau į pasaulį. 
\par 19 Dėl jų Aš pašventinu save, kad ir jie būtų pašventinti tiesa. 
\par 20 Ne tik už juos meldžiu, bet ir už tuos, kurie per jų žodį mane įtikės,­ 
\par 21 kad jie visi būtų viena. Kaip Tu, Tėve, manyje ir Aš Tavyje, kad ir jie būtų viena mumyse, kad pasaulis įtikėtų, jog Tu mane siuntei. 
\par 22 Ir tą šlovę, kurią man suteikei, daviau jiems, kad jie būtų viena, kaip mes esame viena: 
\par 23 Aš juose ir Tu manyje, kad jie pasiektų tobulą vienybę, ir pasaulis pažintų, jog Tu mane siuntei ir pamilai juos taip, kaip myli mane. 
\par 24 Tėve, Aš noriu, kad tie, kuriuos davei, taip pat būtų su manimi ten, kur Aš esu, kad jie matytų mano šlovę, kurią man suteikei, nes pamilai mane prieš pasaulio sukūrimą. 
\par 25 Teisingasis Tėve, pasaulis Tavęs nepažino, o Aš Tave pažinau. Ir šitie pažino, kad Tu mane siuntei. 
\par 26 Aš paskelbiau jiems Tavo vardą ir dar skelbsiu, kad meilė, kuria mane pamilai, būtų juose ir Aš juose”.


\chapter{18}


\par 1 Baigęs kalbėti, Jėzus su savo mokiniais nuėjo anapus Kedrono upelio, kur buvo sodas. Jis ir mokiniai įėjo į sodą. 
\par 2 Jo išdavėjas Judas taip pat žinojo tą vietą, nes Jėzus ir Jo mokiniai dažnai ten rinkdavosi. 
\par 3 Taigi Judas, gavęs kareivių būrį ir aukštųjų kunigų bei fariziejų tarnų, atėjo ten su žibintais, deglais ir ginklais. 
\par 4 Tuomet Jėzus, žinodamas visa, kas Jo laukia, išėjo į priekį ir paklausė: “Ko ieškote?” 
\par 5 Jie atsakė: “Jėzaus iš Nazareto!” Jėzus tarė jiems: “Tai Aš”. Jo išdavėjas Judas irgi stovėjo tarp jų. 
\par 6 Kai Jėzus ištarė: “Tai Aš”, jie atšoko atgal ir parpuolė ant žemės. 
\par 7 O Jis vėl juos klausė: “Ko ieškote?” Jie atsakė: “Jėzaus iš Nazareto”. 
\par 8 Jėzus atsiliepė: “Jau jums sakiau, kad tai Aš. Jei tad manęs ieškote, leiskite šitiem pasišalinti”,­ 
\par 9 kad išsipildytų Jo pasakyti žodžiai: “Iš tų, kuriuos man davei, nepražudžiau nė vieno”. 
\par 10 Tuomet Simonas Petras, kuris turėjo kalaviją, išsitraukė jį, smogė vyriausiojo kunigo tarnui ir nukirto jam dešinę ausį. Tarnas buvo vardu Malkus. 
\par 11 Jėzus pasakė Petrui: “Kišk kalaviją į makštį! Nejaugi Aš negersiu tos taurės, kurią Tėvas man davė?” 
\par 12 Būrys, jo vadas ir žydų tarnai suėmė Jėzų, surišo 
\par 13 ir nuvedė pirmiausia pas Aną. Mat jis buvo tų metų vyriausiojo kunigo Kajafo uošvis. 
\par 14 Tai tas pats Kajafas, kuris buvo žydams pataręs: “Geriau, kad vienas žmogus mirtų už tautą”. 
\par 15 Paskui Jėzų nusekė Simonas Petras ir kitas mokinys. Tas mokinys buvo pažįstamas su vyriausiuoju kunigu ir įėjo su Jėzumi į vyriausiojo kunigo kiemą. 
\par 16 O Petras liko stovėti lauke prie vartų. Tada anas mokinys, kuris buvo pažįstamas su vyriausiuoju kunigu, išėjo laukan, pasikalbėjo su durininke ir įsivedė Petrą vidun. 
\par 17 Tarnaitė durininkė tarė Petrui: “Ar tik ir tu nebūsi vienas iš to žmogaus mokinių?” Šis atsakė: “Ne!” 
\par 18 Ten stoviniavo samdiniai ir tarnai, dėl šalčio susikūrę ugnį, ir šildėsi. Prie jų atsistojo Petras ir taip pat šildėsi. 
\par 19 Vyriausiasis kunigas paklausė Jėzų apie Jo mokinius bei Jo mokymą. 
\par 20 Jėzus jam atsakė: “Aš viešai kalbėjau pasauliui. Aš visada mokiau sinagogoje ir šventykloje, kur visuomet susirenka žydai, ir nieko nekalbėjau slapčia. 
\par 21 Kodėl tad mane klausi? Klausk tų, kurie girdėjo, ką jiems kalbėjau. Jie žino, ką Aš sakiau”. 
\par 22 Jam tai pasakius, vienas iš ten buvusių tarnų smogė Jėzui per veidą, tardamas: “Šitaip atsakai vyriausiajam kunigui?!” 
\par 23 Jėzus jam tarė: “Jei kalbėjau negerai, pasakyk, kas negerai, o jei gerai,­kodėl mane muši?” 
\par 24 Tada Anas pasiuntė Jį surištą pas vyriausiąjį kunigą Kajafą. 
\par 25 Simonas Petras tebestovėjo ir šildėsi. Aplinkiniai paklausė jį: “Ar tik nebūsi vienas iš Jo mokinių?” Tas išsigynė: “Ne!” 
\par 26 Vienas iš vyriausiojo kunigo tarnų, giminaitis to, kuriam Petras nukirto ausį, pasakė: “Argi aš nemačiau tavęs sode kartu su Juo?” 
\par 27 Petras ir vėl išsigynė, ir tuojau pragydo gaidys. 
\par 28 Iš Kajafo rūmų jie nuvedė Jėzų į pretorijų. Buvo ankstyvas rytas. Jie patys nėjo į pretorijų, kad nesusiteptų ir galėtų valgyti Paschą. 
\par 29 Todėl Pilotas išėjo laukan pas juos ir paklausė: “Kuo kaltinate šitą žmogų?” 
\par 30 Jie atsakė: “Jeigu Jis nebūtų piktadarys, nebūtume tau Jo atvedę”. 
\par 31 Pilotas jiems tarė: “Pasiimkite Jį ir teiskite pagal savo Įstatymą”. Žydai jam atsakė: “Mums neleista nieko bausti mirtimi”,­ 
\par 32 kad išsipildytų Jėzaus žodžiai, kuriais Jis nurodė, kokia mirtimi Jam reikės mirti. 
\par 33 Tada Pilotas vėl įėjo į pretorijų, pasišaukė Jėzų ir paklausė: “Ar Tu esi žydų karalius?” 
\par 34 Jėzus jam atsakė: “Ar nuo savęs šito klausi, ar kiti apie mane tau pasakė?” 
\par 35 Pilotas tarė: “Bene aš žydas?! Tavoji tauta ir aukštieji kunigai man Tave atvedė. Ką padarei?” 
\par 36 Jėzus atsakė: “Mano karalystė ne iš šio pasaulio. Jei mano karalystė būtų iš šio pasaulio, mano tarnai kovotų, kad nebūčiau atiduotas žydams. Bet mano karalystė ne iš čia”. 
\par 37 Tada Pilotas Jį paklausė: “Vadinasi, Tu esi karalius?” Jėzus atsakė: “Taip yra, kaip sakai: Aš esu karalius. Aš tam gimiau ir atėjau į šį pasaulį, kad liudyčiau tiesą. Kiekvienas, kas laikosi tiesos, klauso mano balso”. 
\par 38 Pilotas Jo paklausė: “O kas yra tiesa?!” Po šių žodžių jis vėl išėjo pas žydus ir tarė jiems: “Aš nerandu Jame jokios kaltės. 
\par 39 Yra jūsų paprotys, kad per Paschą aš paleisčiau vieną suimtąjį. Tad ar norite, kad paleisčiau jums žydų karalių?” 
\par 40 Tada jie vėl visi ėmė šaukti: “Ne šitą, bet Barabą!” O Barabas buvo plėšikas.


\chapter{19}


\par 1 Tuomet Pilotas ėmė ir nuplakdino Jėzų. 
\par 2 Kareiviai, nupynę vainiką iš erškėčių, uždėjo Jam ant galvos, apsiautė Jį purpurine skraiste 
\par 3 ir sakė: “Sveikas, žydų karaliau!” Ir daužė Jam per veidą. 
\par 4 O Pilotas dar kartą išėjo laukan ir kalbėjo žydams: “Štai išvedu Jį jums, kad žinotumėte, jog nerandu Jame jokios kaltės”. 
\par 5 Jėzus išėjo laukan su erškėčių vainiku ir purpurine skraiste. Pilotas tarė: “Štai žmogus!” 
\par 6 Jį pamatę, aukštieji kunigai ir tarnai pradėjo šaukti: “Nukryžiuok Jį, nukryžiuok!” Pilotas jiems sako: “Jūs imkite Jį ir nukryžiuokite! Aš nerandu Jame jokios kaltės”. 
\par 7 Žydai jam atsakė: “Mes turime Įstatymą, ir pagal mūsų Įstatymą Jis turi mirti, nes laikė save Dievo Sūnumi”. 
\par 8 Išgirdęs tuos žodžius, Pilotas dar labiau nusigando. 
\par 9 Jis vėl nuėjo į pretorijų ir klausė Jėzų: “Iš kur Tu?” Bet Jėzus jam neatsakė. 
\par 10 Tada Pilotas Jam tarė: “Tu nekalbi su manimi? Ar nežinai, kad turiu valdžią Tave nukryžiuoti ir turiu valdžią Tave paleisti?” 
\par 11 Jėzus atsakė: “Tu neturėtum prieš mane jokios valdžios, jeigu tau nebūtų jos duota iš aukštybių. Todėl tam, kuris mane tau įdavė, didesnė nuodėmė”. 
\par 12 Nuo tol Pilotas stengėsi Jį paleisti, bet žydai šaukė: “Jei šitą paleidi, nebesi ciesoriaus draugas. Kiekvienas, kas skelbiasi karaliumi, kalba prieš ciesorių”. 
\par 13 Tai išgirdęs, Pilotas išvedė Jėzų laukan ir atsisėdo į teisėjo krasę, kuri stovėjo vietoje, vadinamoje “Akmeninis grindinys”, hebrajiškai Gabata. 
\par 14 Buvo diena prieš Paschą, apie šeštą valandą. Jis tarė žydams: “Štai jūsų karalius!” 
\par 15 Bet tie šaukė: “Šalin, šalin! Nukryžiuok Jį!” Pilotas paklausė: “Nejaugi turiu nukryžiuoti jūsų karalių?” Aukštieji kunigai atsakė: “Mes neturime karaliaus, tiktai ciesorių”. 
\par 16 Tada Pilotas atidavė jiems Jį nukryžiuoti. Jie pasiėmė Jėzų ir išsivedė. 
\par 17 Nešdamas savo kryžių, Jis ėjo į vadinamąją Kaukolės vietą, hebrajiškai Golgotą. 
\par 18 Tenai jie Jį nukryžiavo; kartu su Juo ir kitus du, vienoje ir antroje pusėje, o Jėzų viduryje. 
\par 19 Pilotas parašė užrašą ir prikalė ant kryžiaus. Buvo parašyta: “Jėzus Nazarietis, žydų karalius”. 
\par 20 Šį užrašą skaitė daugybė žydų, nes vieta, kur Jėzų nukryžiavo, buvo arti miesto, o parašyta buvo hebrajiškai, graikiškai ir lotyniškai. 
\par 21 Žydų aukštieji kunigai sakė Pilotui: “Nerašyk: ‘Žydų karalius’, bet: ‘Šitas sakė: Aš esu žydų karalius’ ”. 
\par 22 Pilotas atsakė: “Ką parašiau, parašiau!” 
\par 23 Kareiviai, nukryžiavę Jėzų, pasiėmė Jo drabužius ir pasidalino juos į keturias dalis­kiekvienam kareiviui po dalį; pasiėmė ir tuniką. Ji buvo be siūlės, nuo viršaus iki apačios ištisai megzta. 
\par 24 Todėl jie tarėsi: “Neplėšykime jos, bet meskime burtą, kuriam ji atiteks”,­kad išsipildytų Raštas: “Jie pasidalijo mano drabužius tarp savęs ir dėl mano apdaro metė burtą”. Šitaip kareiviai ir padarė. 
\par 25 Prie Jėzaus kryžiaus stovėjo Jo motina, Jo motinos sesuo, Marija—Kleopo žmona, ir Marija Magdalietė. 
\par 26 Pamatęs stovinčius savo motiną ir mokinį, kurį mylėjo, Jėzus tarė motinai: “Moterie, štai tavo sūnus!” 
\par 27 Paskui tarė mokiniui: “Štai tavo motina!” Ir nuo tos valandos mokinys pasiėmė ją pas save. 
\par 28 Tada, žinodamas, jog viskas įvykdyta,­kad išsipildytų Raštas, Jėzus tarė: “Trokštu!” 
\par 29 Tenai stovėjo indas, pilnas rūgštaus vyno. Jie pakėlė ant yzopo šakelės kempinę, pamirkytą vyne, ir prinešė prie Jo lūpų. 
\par 30 Paragavęs to vyno, Jėzus tarė: “Atlikta!” Ir, nuleidęs galvą, Jis atidavė dvasią. 
\par 31 Kad kūnai neliktų ant kryžiaus per sabatą,­nes tas sabatas buvo didi diena,­žydai Prisirengimo dieną prašė Pilotą, kad nukryžiuotiesiems būtų sulaužyti blauzdikauliai ir kūnai nuimti. 
\par 32 Tad atėjo kareiviai ir sulaužė blauzdas vienam ir antram, kurie buvo su Juo nukryžiuoti. 
\par 33 Priėję prie Jėzaus ir pamatę, kad Jis jau miręs, jie nebelaužė Jam blauzdų, 
\par 34 tik vienas kareivis ietimi perdūrė Jam šoną, ir tuojau ištekėjo kraujo ir vandens. 
\par 35 Tai matęs paliudijo, ir jo liudijimas teisingas; jis žino sakąs tiesą, kad jūs tikėtumėte. 
\par 36 Taip įvyko, kad išsipildytų Raštas: “Nė vienas Jo kaulas nebus sulaužytas”. 
\par 37 Ir vėl kitoje vietoje Raštas sako: “Jie žiūrės į Tą, kurį perdūrė”. 
\par 38 Po to Juozapas iš Arimatėjos, kuris buvo Jėzaus mokinys, tik slaptas dėl žydų baimės, paprašė Pilotą leisti nuimti Jėzaus kūną. Pilotas leido. Jis atėjo ir nuėmė Jėzaus kūną. 
\par 39 Taip pat atvyko ir Nikodemas, kuris anksčiau buvo atėjęs pas Jėzų nakčia. Jis atsivežė apie šimtą svarų miros ir alavijo mišinio. 
\par 40 Taigi jie paėmė Jėzaus kūną ir suvyniojo į drobules su kvepalais, kaip reikalavo žydų laidojimo paprotys. 
\par 41 Toje vietoje, kur Jį nukryžiavo, buvo sodas ir sode naujas kapas, kuriame dar niekas nebuvo laidotas. 
\par 42 Ten jie ir paguldė Jėzų, nes buvo žydų Prisirengimo diena, o kapas arti.


\chapter{20}


\par 1 Pirmąją savaitės dieną, labai anksti, dar neišaušus, Marija Magdalietė atėjo prie kapo ir pamatė, kad akmuo nuo kapo nuristas. 
\par 2 Ji nubėgo pas Simoną Petrą ir kitą mokinį, kurį Jėzus mylėjo, ir pranešė jiems: “Paėmė Viešpatį iš kapo, ir nežinome, kur Jį padėjo”. 
\par 3 Petras ir tas kitas mokinys nuskubėjo prie kapo. 
\par 4 Bėgo abu kartu, bet tasai kitas mokinys pralenkė Petrą ir pirmas pasiekė kapą. 
\par 5 Pasilenkęs jis pamatė numestas drobules, tačiau į vidų nėjo. 
\par 6 Netrukus iš paskos atbėgo ir Simonas Petras. Jis įėjo į rūsį ir pamatė numestas drobules 
\par 7 ir skarą, buvusią ant Jėzaus galvos, ne su drobulėmis paliktą, bet suvyniotą ir atskirai padėtą. 
\par 8 Tada įėjo ir kitas mokinys, kuris pirmas buvo atbėgęs prie kapo. Jis pamatė ir įtikėjo. 
\par 9 Mat jie dar nebuvo supratę Rašto, kad Jis turėsiąs prisikelti iš numirusių. 
\par 10 Paskui mokiniai vėl sugrįžo namo. 
\par 11 O Marija stovėjo lauke prie kapo ir verkė. Verkdama ji pasilenkė, pažvelgė į kapo vidų 
\par 12 ir pamatė du angelus baltais drabužiais sėdinčius­vieną galvūgalyje, kitą kojų vietoje­ten, kur būta Jėzaus kūno. 
\par 13 Jie paklausė ją: “Moterie, ko verki?” Ji atsakė: “Paėmė mano Viešpatį ir nežinau, kur Jį padėjo”. 
\par 14 Tai tarusi, ji atsisuko ir pamatė stovintį Jėzų, bet nepažino, kad tai Jėzus. 
\par 15 Jėzus jai tarė: “Moterie, ko verki? Ko ieškai?” Ji, manydama, jog tai sodininkas, atsakė: “Gerbiamasis, jei tamsta Jį išnešei, pasakyk man, kur Jį padėjai. Aš Jį pasiimsiu”. 
\par 16 Jėzus jai sako: “Marija!” Ji atsigręžė ir sušuko: “Rabuni!” (Tai reiškia: “Mokytojau”). 
\par 17 Jėzus jai tarė: “Neliesk manęs! Aš dar neįžengiau pas savo Tėvą. Eik pas mano brolius ir pasakyk jiems: ‘Aš žengiu pas savo Tėvą ir jūsų Tėvą, pas savo Dievą ir jūsų Dievą’ ”. 
\par 18 Marija Magdalietė nuėjo ir pranešė mokiniams, kad mačiusi Viešpatį ir ką Jis jai sakęs. 
\par 19 Tos pirmosios savaitės dienos vakare, durims, kur buvo susirinkę mokiniai, dėl žydų baimės esant užrakintoms, atėjo Jėzus, atsistojo viduryje ir tarė: “Ramybė jums!” 
\par 20 Tai pasakęs, Jis parodė jiems rankas ir šoną. Mokiniai nudžiugo, išvydę Viešpatį. 
\par 21 Jėzus vėl tarė: “Ramybė jums! Kaip mane siuntė Tėvas, taip ir Aš jus siunčiu”. 
\par 22 Tai pasakęs, Jis kvėpė į juos ir tarė: “Priimkite Šventąją Dvasią. 
\par 23 Kam atleisite nuodėmes, tiems jos bus atleistos, o kam sulaikysite,­sulaikytos”. 
\par 24 Vieno iš dvylikos­Tomo, vadinamo Dvyniu,­nebuvo su jais, kai Jėzus atėjo. 
\par 25 Tad kiti mokiniai jam kalbėjo: “Mes matėme Viešpatį!” O jis atsakė: “Jeigu aš nepamatysiu Jo rankose vinių dūrio ir neįleisiu piršto į vinių vietą, ir jeigu ranka nepaliesiu Jo šono­netikėsiu”. 
\par 26 Po aštuonių dienų Jo mokiniai vėl buvo kambaryje, ir Tomas su jais. Jėzus atėjo, durims esant užrakintoms, atsistojo viduryje ir prabilo: “Ramybė jums!” 
\par 27 Paskui kreipėsi į Tomą: “Įleisk čia pirštą ir pažiūrėk į mano rankas. Pakelk ranką ir paliesk mano šoną; nebūk netikintis­būk tikintis”. 
\par 28 Tomas atsakė Jam: “Mano Viešpats ir mano Dievas!” 
\par 29 Jėzus jam tarė: “Tomai, tu įtikėjai, nes mane pamatei. Palaiminti, kurie tiki nematę!” 
\par 30 Savo mokinių akivaizdoje Jėzus padarė ir daug kitų ženklų, kurie nesurašyti šitoje knygoje. 
\par 31 O šitie yra surašyti, kad tikėtumėte, jog Jėzus yra Kristus, Dievo Sūnus, ir kad tikėdami turėtumėte gyvenimą per Jo vardą.


\chapter{21}


\par 1 Paskui Jėzus vėl pasirodė mokiniams prie Tiberiados ežero. Pasirodė taip. 
\par 2 Buvo drauge Simonas Petras, Tomas, vadinamas Dvyniu, Natanaelis iš Galilėjos Kanos, Zebediejaus sūnūs ir dar du kiti Jėzaus mokiniai. 
\par 3 Simonas Petras jiems sako: “Einu žvejoti”. Jie pasisiūlė: “Ir mes einame su tavimi”. Nuėję jie tuojau sulipo į valtį, tačiau tą naktį nieko nesugavo. 
\par 4 Rytui auštant, ant kranto pasirodė bestovįs Jėzus. Bet mokiniai nepažino, kad tai buvo Jėzus. 
\par 5 O Jėzus jiems tarė: “Vaikeliai, ar neturite ko valgyti?” Jie atsakė: “Ne”. 
\par 6 Tada Jis pasakė: “Užmeskite tinklą į dešinę nuo valties, ir pagausite”. Jie užmetė ir nebeįstengė jo patraukti dėl žuvų gausybės. 
\par 7 Tuomet tasai mokinys, kurį Jėzus mylėjo, sako Petrui: “Tai Viešpats!” Išgirdęs, jog tai Viešpats, Simonas Petras persijuosė drabužį,­mat buvo neapsirengęs,­ir šoko į ežerą. 
\par 8 Kiti mokiniai atsiyrė valtimi, nes buvo netoli kranto­maždaug už dviejų šimtų mastų­ir atitempė tinklą su žuvimis. 
\par 9 Išlipę į krantą, jie pamatė žėrinčias žarijas, ant jų padėtą žuvį ir duonos. 
\par 10 Jėzus jiems tarė: “Atneškite ką tik pagautų žuvų”. 
\par 11 Simonas Petras nuėjo ir išvilko į krantą tinklą, pilną didelių žuvų, iš viso šimtą penkiasdešimt tris. Nors jų buvo tokia gausybė, tačiau tinklas nesuplyšo. 
\par 12 Jėzus tarė: “Eikite šen pusryčių!” Ir nė vienas iš mokinių neišdrįso paklausti: “Kas Tu esi?”, nes jie žinojo, jog tai Viešpats. 
\par 13 Jėzus priėjo, paėmė duonos ir davė jiems, taip pat ir žuvies. 
\par 14 Tai jau trečią kartą pasirodė savo mokiniams Jėzus, prisikėlęs iš numirusių. 
\par 15 Papusryčiavus Jėzus paklausė Simoną Petrą: “Simonai, Jonos sūnau, ar myli mane labiau už šituos?” Jis atsakė: “Taip, Viešpatie, Tu žinai, kad Tave myliu”. Jėzus jam tarė: “Ganyk mano avinėlius”. 
\par 16 Ir antrą kartą Jėzus paklausė: “Simonai, Jonos sūnau, ar myli mane?” Tas atsiliepė: “Taip, Viešpatie, Tu žinai, kad Tave myliu”. Jėzus jam pasakė: “Ganyk mano avis”. 
\par 17 Jėzus paklausė trečią kartą: “Simonai, Jonos sūnau, ar myli mane?” Petras nuliūdo, kad Jis trečią kartą klausia: “Ar myli mane?”, ir atsakė: “Viešpatie, Tu viską žinai. Tu žinai, kad Tave myliu”. Jėzus jam tarė: “Ganyk mano avis. 
\par 18 Iš tiesų, iš tiesų sakau tau: kai buvai jaunas, pats susijuosdavai ir vaikščiojai, kur norėjai. O pasenęs tu ištiesi rankas, ir kitas tave perjuos ir ves, kur tu nenori”. 
\par 19 Jis tai pasakė, nurodydamas, kokia mirtimi Petras pašlovinsiąs Dievą. Tai pasakęs, dar pridūrė: “Sek paskui mane!” 
\par 20 Petras atsisukęs pamatė iš paskos einantį mokinį, kurį Jėzus mylėjo, kuris vakarienės metu buvo prisiglaudęs prie Jėzaus krūtinės ir klausė: “Viešpatie, kas Tave išduos?” 
\par 21 Pamatęs jį, Petras tarė Jėzui: “Viešpatie, o kas bus šitam?” 
\par 22 Jėzus atsakė: “Jei Aš noriu, kad jis pasiliktų, kol ateisiu, kas gi tau? Tu sek paskui mane!” 
\par 23 Todėl pasklido šis žodis tarp brolių, jog tas mokinys nemirsiąs. Bet Jėzus nesakė, kad Jis nemirs, tik: “Jei noriu, kad jis pasiliktų, kol ateisiu, kas gi tau?” 
\par 24 Tai ir yra tas mokinys, kuris liudija apie tuos dalykus ir juos aprašė, ir mes žinome, kad jo liudijimas tikras. 
\par 25 Yra dar daug kitų dalykų, kuriuos Jėzus padarė. Jeigu juos visus atskirai aprašytume, manau, kad visas pasaulis nesutalpintų knygų, kurios būtų parašytos. Amen.





\end{document}