\begin{document}

\title{Zecharijo knyga}

\chapter{1}


\par 1 Antraisiais Darijaus metais, aštuntą mėnesį, Viešpats kalbėjo Zacharijui, Idojo sūnaus Berechijos sūnui, sakydamas: 
\par 2 “Viešpats buvo labai užsirūstinęs ant jūsų tėvų. 
\par 3 Todėl sakyk jiems: ‘Taip sako kareivijų Viešpats: ‘Gręžkitės į mane, tai Aš gręšiuosi į jus. 
\par 4 Nebūkite kaip jūsų tėvai, kuriems prieš tai buvę pranašai skelbė: ‘Taip sako kareivijų Viešpats: ‘Nusisukite nuo savo piktų kelių ir darbų!’ Bet jie neklausė ir nekreipė dėmesio,­sako Viešpats.­ 
\par 5 Jūsų tėvų nebėra ir pranašai negyvena per amžius, 
\par 6 bet mano žodžiai ir nuostatai, kuriuos buvau paskelbęs savo tarnams pranašams, pasivijo jūsų tėvus. Jie sugrįžo ir sakė: ‘Kaip kareivijų Viešpats sumanė mums padaryti pagal mūsų kelius ir darbus, taip Jis ir pasielgė’ ”. 
\par 7 Antraisiais Darijaus metais, vienuolikto mėnesio dvidešimt ketvirtą dieną, Viešpats kalbėjo pranašui Zacharijui, Idojo sūnaus Berechijos sūnui. 
\par 8 Naktį aš regėjau vyrą. Jis sėdėjo ant sarto žirgo, kuris stovėjo tarp mirtų dauboje; už jo buvo sartų, margų ir baltų žirgų. 
\par 9 Aš paklausiau: “Kas jie, mano viešpatie?” Angelas, kuris kalbėjo su manimi, atsakė: “Aš tau paaiškinsiu”. 
\par 10 Vyras, kuris stovėjo tarp mirtų, tarė: “Viešpats siuntė juos apvaikščioti žemę”. 
\par 11 Jie atsakė Viešpaties angelui, stovinčiam tarp mirtų: “Mes apvaikščiojome žemę­visur ramybė ir taika”. 
\par 12 Viešpaties angelas klausė: “Kareivijų Viešpatie, kiek dar laiko nepasigailėsi Jeruzalės ir Judo miestų, prieš kuriuos rūstinaisi septyniasdešimt metų?” 
\par 13 Viešpats gerais ir paguodžiančiais žodžiais atsakė su manimi kalbančiam angelui. 
\par 14 Angelas, kuris kalbėjo su manimi, tarė: “Skelbk: ‘Taip sako kareivijų Viešpats: ‘Mano pavydas dėl Jeruzalės ir Siono yra didelis. 
\par 15 Ir Aš labai supykęs ant tautų, kurios jaučiasi saugios. Kai Aš buvau truputį supykęs, jos padidino jūsų vargus’. 
\par 16 Todėl taip sako Viešpats: ‘Aš pasigailėsiu Jeruzalės, mano šventykla bus statoma joje. 
\par 17 Mano miestai bus perpildyti gėrybių. Viešpats paguos Sioną, vėl išsirinks Jeruzalę’ ”. 
\par 18 Pakėlęs akis, pamačiau: štai buvo keturi ragai! 
\par 19 Aš klausiau angelą, kuris su manimi kalbėjo: “Ką reiškia šitie ragai?” Jis atsakė: “Šitie ragai išsklaidė Izraelį, Judą ir Jeruzalę”. 
\par 20 Po to Viešpats parodė man keturis kalvius. 
\par 21 Aš klausiau: “Ką šitie darys?” Jis atsakė: “Ragai išsklaidė Judą taip, kad niekas nebegalėjo pakelti galvos. O šitie atėjo numušti ragus toms tautoms, kurios pakėlė ragus prieš Judo žemę, kad ją išsklaidytų”.


\chapter{2}


\par 1 Pakėlęs akis, pamačiau vyrą, kuris laikė rankoje matavimo virvę. 
\par 2 Aš paklausiau: “Kur tu eini?” Jis man atsakė: “Matuoti Jeruzalės, kad sužinočiau jos plotį ir ilgį”. 
\par 3 Angelas, kalbėjęs su manimi, išėjo priekin, o kitas angelas atėjo priešais jį. 
\par 4 Jis tarė anam: “Bėk, pranešk šiam jaunuoliui, kad Jeruzalėje bus gyvenama ir už miesto sienų dėl daugybės žmonių ir gyvulių”. 
\par 5 “Aš,­sako Viešpats,­būsiu jai ugnies siena ir būsiu šlovė joje. 
\par 6 Bėkite iš šiaurės krašto,­sako Viešpats,­nes Aš jus išsklaidžiau į visas keturias puses,­sako Viešpats.­ 
\par 7 Gelbėkis, Sione, gyvenantis su Babilono dukterimi!” 
\par 8 Taip sako kareivijų Viešpats, kurio šlovė siuntė mane pas tautas, kurios plėšė jus: “Kas paliečia jus, paliečia mano akies vyzdį! 
\par 9 Aš pakelsiu ranką prieš juos, ir jie taps savo tarnų grobiu. Tada jūs žinosite, kad kareivijų Viešpats siuntė mane. 
\par 10 Džiūgauk ir būk linksma, Siono dukra, nes Aš ateinu ir gyvensiu tavyje,­sako Viešpats.­ 
\par 11 Tuomet daug tautų prisijungs prie Viešpaties ir taps mano tauta. Aš gyvensiu tavyje ir tu žinosi, kad kareivijų Viešpats siuntė mane pas tave. 
\par 12 Viešpats vėl valdys Judą, kaip savo paveldą šventame krašte, ir Jeruzalė vėl taps išrinktuoju miestu. 
\par 13 Tenutyla kiekvienas kūnas Viešpaties akivaizdoje, nes Jis pakilo iš savo šventos buveinės”.



\chapter{3}


\par 1 Jis man parodė vyriausiąjį kunigą Jozuę, stovintį prie Viešpaties angelo. Šėtonas stovėjo jo dešinėje, kad jam pasipriešintų. 
\par 2 Viešpats tarė šėtonui: “Viešpats, kuris išsirinko Jeruzalę, tesudraudžia tave, šėtone! Argi šitas ne nuodėgulis, ištrauktas iš ugnies?” 
\par 3 Jozuė stovėjo prie angelo, apsivilkęs nešvariais drabužiais. 
\par 4 Angelas sakė stovintiems prie jo: “Nuvilkite nuo jo nešvarius drabužius!” O jam jis tarė: “Štai aš pašalinau tavo kaltę, dabar aprengiu tave puošniais rūbais!” 
\par 5 Jis liepė: “Uždėkite jam ant galvos švarią mitrą!” Jie uždėjo jam mitrą ir apvilko rūbais. Viešpaties angelas stovėjo šalia. 
\par 6 Viešpaties angelas kalbėjo Jozuei: 
\par 7 “Taip sako kareivijų Viešpats: ‘Jei vaikščiosi mano keliais ir atliksi tarnystę, tada teisi mano namus ir prižiūrėsi mano kiemus. Aš suteiksiu tau vietą tarp tų, kurie čia stovi. 
\par 8 Klausykis, Jozue, vyriausiasis kunige, tu ir tavo draugai, žymūs žmonės, kurie sėdi su tavimi. Aš atvesiu savo tarną­Atžalą. 
\par 9 Štai akmenį padėjau Jozuės akivaizdoje. Tame akmenyje yra septynios akys. Aš įrėšiu jame įrašą,­sako kareivijų Viešpats,­ir pašalinsiu šito krašto kaltę per vieną dieną. 
\par 10 Tą dieną,­sako kareivijų Viešpats,­jūs kviesite vienas kitą po vynmedžiu ir figmedžiu’ ”.



\chapter{4}


\par 1 Angelas, kuris kalbėjo su manimi, vėl sugrįžo ir žadino mane, kaip žmogus žadinamas iš miego. 
\par 2 Jis klausė: “Ką matai?” Aš atsakiau: “Matau auksinę žvakidę, aliejaus indą jos viršuje ir septynias lempas su septyniais vamzdeliais. 
\par 3 Prie jos yra du alyvmedžiai: vienas aliejaus indo dešinėje, o kitas kairėje”. 
\par 4 Aš klausiau angelo, kuris kalbėjo su manimi: “Kas tai yra, mano viešpatie?” 
\par 5 Angelas atsakė: “Ar nežinai, kas tai yra?” Aš atsakiau: “Ne, mano viešpatie”. 
\par 6 Jis tarė: “Tai yra Viešpaties žodis Zorobabeliui: ‘Ne galybe ir ne jėga, bet mano dvasia,­sako kareivijų Viešpats’ ”. 
\par 7 Kas esi, didis kalne? Zorobabelio akivaizdoje tu pavirsi lyguma! Jis padės paskutinį akmenį, skambant džiaugsmo šūksniams: “Malonė, malonė jam”. 
\par 8 Viešpats kalbėjo man: 
\par 9 “Zorobabelio rankos padėjo pamatus šiems namams, ir jo rankos užbaigs juos. Tada žinosi, kad kareivijų Viešpats siuntė mane pas jus. 
\par 10 Kurie paniekino mažą pradžią, džiaugsis, matydami paskutinį akmenį Zorobabelio rankoje. Septynios lempos yra septynios Viešpaties akys, kurios viską stebi žemėje”. 
\par 11 Aš klausiau angelą: “Kas yra šitie du alyvmedžiai žvakidės dešinėje ir kairėje?” 
\par 12 Aš antrą kartą klausiau jį. “Kas yra tos dvi alyvmedžių šakelės prie auksinių vamzdelių, kuriais teka aliejus?” 
\par 13 Jis klausė manęs: “Ar nežinai, kas tai yra?” Aš atsakiau: “Ne, mano viešpatie”. 
\par 14 Jis tarė: “Šie du yra pateptieji, kurie stovi prie visos žemės Viešpaties”.



\chapter{5}


\par 1 Aš pakėliau akis ir regėjau skrendantį raštų ritinį. 
\par 2 Angelas klausė: “Ką matai?” Aš atsakiau: “Matau skrendantį ritinį dvidešimties uolekčių ilgio ir dešimties pločio”. 
\par 3 Angelas aiškino: “Tai prakeikimas, einantis per visą žemę. Kiekvienas vagis ir melagingai prisiekiantis bus teisiamas pagal tai, kaip ten parašyta. 
\par 4 ‘Aš jį pasiunčiau,­sako kareivijų Viešpats,­į vagies namus ir į melagingai prisiekiančio mano vardu namus, kad juos sunaikintų iš pamatų’ ”. 
\par 5 Po to angelas, kuris kalbėjo su manimi, išėjo priekin ir man tarė: “Pakelk savo akis ir pažiūrėk, ką matai”. 
\par 6 Aš klausiau: “Kas tai yra?” Angelas atsakė: “Tai indas, kuris vaizduoja viso krašto nuodėmę”. 
\par 7 Švininis dangtis pakilo, ir štai inde sėdėjo moteris. 
\par 8 Angelas tarė: “Tai nedorybė”. Jis ją įstūmė atgal į indą ir uždengė. 
\par 9 Pakėlęs akis, pamačiau dvi moteris. Vėjas buvo jų sparnuose, kurie buvo panašūs į gandro sparnus. Jos pakėlė indą aukštyn. 
\par 10 Aš klausiau angelo: “Kur jos neša indą?” 
\par 11 Jis atsakė: “Šinaro krašte jam bus pastatyti namai. Ten jis bus padėtas ant savo pamato”.



\chapter{6}


\par 1 Pakėlęs akis, pamačiau iš dviejų kalnų tarpo išvažiuojančius keturis vežimus. Tie kalnai buvo variniai. 
\par 2 Pirmojo kovos vežimo žirgai buvo sarti, antrojo vežimo­juodi, 
\par 3 trečiojo­balti, o ketvirtojo­ kerši ir bėri. 
\par 4 Aš klausiau angelo: “Kas tai yra, mano viešpatie?” 
\par 5 Angelas man atsakė: “Tai keturios dangaus dvasios. Jos buvo visos žemės Viešpaties akivaizdoje ir dabar išvažiuoja. 
\par 6 Juodi žirgai eis į šiaurės šalį, ir balti eis paskui juos. Kerši žirgai eis į pietų šalį”. 
\par 7 Bėri žirgai nerimavo, norėjo išvažinėti visą žemę. Angelas jiems sakė: “Eikite ir važinėkite po žemę!” Jie padarė tai. 
\par 8 Dabar jis tarė man: “Kurie išvažiavo į šiaurės kraštą, nuramino mano dvasią toje šalyje”. 
\par 9 Ir Viešpats tarė man: 
\par 10 “Paimk iš tremtinių, sugrįžusių iš Babilono­Heldajo, Tobijos bei Jedajos­aukso bei sidabro ir eik su juo tą pačią dieną į Sofonijos sūnaus Jozijo namus. 
\par 11 Padirbdink iš to sidabro ir aukso karūną ir, uždėjęs ją ant Jehocadako sūnaus Jozuės, vyriausiojo kunigo, galvos, 
\par 12 sakyk: ‘Taip sako kareivijų Viešpats: ‘Štai vyras, kurio vardas Atžala. Jis išaugs iš savo vietos ir pastatys Viešpaties šventyklą. 
\par 13 Jis garbingai sėdės soste ir valdys kraštą. Bus ir kunigas savo soste, ir jie abu taikiai sutars’. 
\par 14 Karūna bus padėta Viešpaties šventykloje Helemo, Tobijos, Jedajos ir Sofonijo sūnaus Heno atminimui. 
\par 15 Iš tolimų kraštų atėję žmonės padės statyti Viešpaties šventyklą. Tada žinosite, kad kareivijų Viešpats siuntė mane pas jus. Tai įvyks, jei uoliai klausysite Viešpaties, savo Dievo”.



\chapter{7}


\par 1 Ketvirtaisiais karaliaus Darijaus metais, devinto mėnesio ketvirtą dieną, Viešpats kalbėjo Zacharijui, 
\par 2 kai Betelio gyventojai siuntė Sarecerą ir Regem Melechą su jų žmonėmis melstis Viešpačiui 
\par 3 ir kareivijų Viešpaties kunigų bei pranašų paklausti: “Ar man pasninkauti ir verkti penktąjį mėnesį, kaip tai dariau daugelį metų?” 
\par 4 Kareivijų Viešpats atsakė man: 
\par 5 “Sakyk viso krašto žmonėms ir kunigams: ‘Kai pasninkavote ir gedėjote penktąjį ir septintąjį mėnesį per šituos septyniasdešimt metų, ar jūs man pasninkavote? 
\par 6 O kai valgote ir geriate, ar ne sau valgote ir geriate?’ 
\par 7 Argi ne tokius pat žodžius Viešpats skelbė per ankstesniuosius pranašus, kai Jeruzalė buvo dar nesunaikinta ir jos aplinkiniai miestai, pietų kraštas ir žemuma tebebuvo apgyventa?” 
\par 8 Viešpats kalbėjo Zacharijui: 
\par 9 “Taip kalbėjo kareivijų Viešpats: ‘Teiskite teisingai ir būkite pasigailintys bei užjaučiantys vienas kitą! 
\par 10 Neskriauskite našlių ir našlaičių, ateivių ir beturčių. Nemąstykite pikta savo širdyse prieš savo brolį’. 
\par 11 Bet jie nepaisė to, atkakliai priešinosi ir užsikimšo ausis, kad negirdėtų. 
\par 12 Jie sukietino savo širdį kaip akmenį, kad negirdėtų įstatymo ir žodžių, kuriuos kareivijų Viešpats siuntė savo dvasia per ankstesniuosius pranašus. Todėl išsiliejo kareivijų Viešpaties rūstybė. 
\par 13 Jis šaukė, bet jie neatsiliepė. ‘Todėl ir Aš neatsiliepiau jiems šaukiant,­sako kareivijų Viešpats.­ 
\par 14 Aš išblaškiau juos po visas tautas, kurių jie nepažino. Kraštas, kurį jie paliko, virto dykuma­niekas nebekeliaudavo per jį. Jie pavertė tą puikų kraštą dykuma’ ”.



\chapter{8}


\par 1 Kareivijų Viešpats kalbėjo: 
\par 2 “Taip sako kareivijų Viešpats: ‘Didelis mano pavydas dėl Siono, labai užsirūstinęs pavydžiu dėl jo’. 
\par 3 Taip sako Viešpats: ‘Aš sugrįžau į Sioną ir gyvensiu Jeruzalėje. Jeruzalė bus vadinama tiesos miestu, o kareivijų Viešpaties kalnas­šventu kalnu’. 
\par 4 Taip sako kareivijų Viešpats: ‘Seneliai ir senelės sėdės Jeruzalės gatvėse su lazda rankoje dėl senatvės. 
\par 5 Miesto aikštės bus pilnos žaidžiančių berniukų ir mergaičių’. 
\par 6 Taip sako kareivijų Viešpats: ‘Jei šiomis dienomis tautos likučiui tai atrodo neįmanomas dalykas, ar ir man tai neįmanoma?­sako kareivijų Viešpats’. 
\par 7 Taip sako kareivijų Viešpats: ‘Aš išgelbėsiu savo tautą ir parvesiu ją iš rytų ir vakarų. 
\par 8 Jie gyvens Jeruzalėje ir bus mano tauta, o Aš būsiu jų Dievas tiesoje ir teisume’. 
\par 9 Taip sako kareivijų Viešpats: ‘Tesustiprėja jūsų rankos, kurie girdite šiomis dienomis žodžius pranašų, buvusių tą dieną, kai dėjo pamatus kareivijų Viešpaties šventyklai. 
\par 10 Iki tol žmogus negaudavo atlyginimo nei už savo darbą, nei už gyvulį. Priešai užpuldinėjo įeinančius ir išeinančius miesto gyventojus, nes Aš sukėliau žmones vieną prieš kitą. 
\par 11 Bet dabar Aš nebesielgsiu su tautos likučiu kaip iki šiol,­sako kareivijų Viešpats.­ 
\par 12 Pasėliai klestės, vynmedis neš vaisių, laukai duos derlių ir dangus­rasą. Visa tai Aš duosiu šitos tautos likučiui. 
\par 13 Judo ir Izraelio namai, jūs buvote prakeikimu tautose. Dabar jus išgelbėsiu ir jūs būsite palaiminimu. Nebijokite! Stiprios tebūna jūsų rankos!’ 
\par 14 Taip sako kareivijų Viešpats: ‘Kaip Aš nusprendžiau bausti jus, kai jūsų tėvai užrūstino mane, ir negailėjau jūsų, 
\par 15 taip dabar Aš nusprendžiau daryti gera Jeruzalei ir Judo namams. Tad nebijokite! 
\par 16 Štai ką turite daryti: kalbėkite tiesą vienas kitam ir teisingai teiskite. 
\par 17 Neplanuokite pikto vienas prieš kitą savo širdyje! Venkite klastingos priesaikos, nes viso to nekenčiu!­sako Viešpats’ ”. 
\par 18 Viešpats kalbėjo man, sakydamas: 
\par 19 “Taip sako kareivijų Viešpats: ‘Ketvirto, penkto, septinto ir dešimto mėnesio pasninkų dienos bus Judo gyventojų džiaugsmas, linksmybė ir iškilmingos šventės, tik mylėkite tiesą ir taiką!’ 
\par 20 Taip sako kareivijų Viešpats: ‘Ateis tautos ir daugelio miestų gyventojai. 
\par 21 Vieno miesto gyventojai kalbės kito miesto gyventojams: ‘Eikime drauge melstis Viešpačiui ir ieškoti kareivijų Viešpaties’. 
\par 22 Daug tautų ir giminių ateis Jeruzalėn ieškoti kareivijų Viešpaties ir melstis Viešpačiui’. 
\par 23 Taip sako kareivijų Viešpats: ‘Tuomet dešimt vyrų iš įvairių tautų atėję įsikibs žydui už skverno ir sakys: ‘Mes eisime su jumis, nes girdėjome, kad su jumis yra Dievas’ ”.



\chapter{9}


\par 1 Viešpaties žodis Hadracho kraštui ir Damasko miestui. Viešpačiui priklauso Sirijos miestai ir visos Izraelio giminės. 
\par 2 Taip pat Hamatas, kuris ribojasi su Hadrachu, Tyras ir Sidonas, nors jie ir labai išmintingi. 
\par 3 Tyras pasistatė tvirtovę ir surinko sidabro kaip dulkių ir aukso kaip gatvių purvo. 
\par 4 Štai Viešpats padarys jį beturtį, jo galia žus jūroje, o jis pats­ugnies liepsnose. 
\par 5 Aškelonas, tai matydamas, nusigąs; taip pat Gaza ir Ekronas didžiai sielvartaus, nes niekais nuėjo jų viltis. Gazos karalius žus, Aškelonas bus nebegyvenamas. 
\par 6 Įvairių tautybių žmonės gyvens Ašdode. Aš sunaikinsiu filistinų išdidumą. 
\par 7 Aš atimsiu kraują iš jo burnos ir bjaurystę nuo jo dantų. Išlikę priklausys mūsų Dievui ir bus kaip Judo gyventojai, Ekronas­kaip jebusiečiai. 
\par 8 Aš stovėsiu kaip sargybinis savo tautos namuose, kad niekas jų nepultų ir nenaikintų. Prispaudėjas nebeateis, nes dabar Aš pats saugosiu juos. 
\par 9 Linksmai džiūgauk, Siono dukra! Šauk, Jeruzalės dukra! Tavo karalius ateina pas tave. Jis teisus ir atneša išgelbėjimą, nuolankus ir joja ant asilo, ant asilės jauniklio. 
\par 10 Aš sunaikinsiu Efraimo kovos vežimus ir Jeruzalės žirgus, jos kovos lanką sulaužysiu. Jis skelbs taiką pagonims ir viešpataus nuo jūros iki jūros ir nuo upės iki žemės pakraščių. 
\par 11 Dėl padarytos su tavimi kraujo sandoros Aš išleisiu tavo kalinius iš duobės, kurioje nėra vandens. 
\par 12 Kaliniai, pasitikėkite ir sugrįžkite į tvirtovę! Dabar skelbiu, kad Aš jums atlyginsiu dvigubai. 
\par 13 Aš naudosiu Judą kaip lanką, Efraimą kaip strėlę. Sione, Aš pakelsiu tavo sūnus prieš Graikiją ir padarysiu tave karžygio kardu. 
\par 14 Viešpats bus virš jų ir šaudys strėlėmis kaip žaibais. Viešpats Dievas trimituodamas ateis su viesulais iš pietų. 
\par 15 Kareivijų Viešpats juos gins, jie ris ir naikins mėtyklių akmenimis. Jie gers ir bus triukšmingi kaip nuo vyno, jie bus pilni kaip indai, kaip aukuro ragai. 
\par 16 Tą dieną Viešpats, jų Dievas, juos išgelbės ir ganys kaip savo kaimenę. Jie bus kaip brangakmeniai karūnoje ir spindės virš Jo krašto. 
\par 17 Koks didis Jo gerumas ir grožis. Javai pradžiugins jaunuolius ir jaunas vynas mergaites.



\chapter{10}


\par 1 Melskite Viešpatį lietaus, atėjus vėlyvo lietaus metui! Viešpats surinks debesis ir duos gausų lietų, palaistys visus laukų augalus. 
\par 2 Stabai kalba tuštybes, būrėjai skelbia melą, tuščių sapnų aiškinimą. Jų paguoda yra bevertė. Todėl jie klaidžiojo kaip avys ir skurdo, nes neturėjo ganytojo. 
\par 3 “Prieš ganytojus užsidegė mano rūstybė, Aš nubaudžiau ožius. Kareivijų Viešpats aplankė savo kaimenę, Judo namus, ir padarė juos tarsi gerą kovos žirgą. 
\par 4 Iš jų kils kertinis akmuo, palapinės stulpas, kovos lankas, galingi vadai. 
\par 5 Jie bus karžygių tauta: sutryps kovoje priešą kaip gatvių purvą, nes Viešpats bus su jais ir išdidūs raiteliai bus sugėdinti. 
\par 6 Aš sustiprinsiu Judo namus, išgelbėsiu Juozapą. Pasigailėsiu jų ir parvesiu juos iš tremties. Jie bus tokie, lyg niekada nebūčiau jų atmetęs. Aš esu Viešpats, jų Dievas, ir išklausysiu juos. 
\par 7 Efraimai bus lyg karžygiai, jie džiūgaus kaip nuo vyno. Jų vaikai, tai matydami, džiaugsis ir dėkos Viešpačiui. 
\par 8 Aš juos surinksiu, nes išpirkau juos. Jie bus gausūs, kaip buvo anksčiau. 
\par 9 Aš juos išsklaidysiu tarp tautų. Tolimose šalyse jie neužmirš manęs, gyvens su savo vaikais ir sugrįš. 
\par 10 Aš parvesiu juos iš Egipto krašto, surinksiu iš Asirijos. Į Gileado ir Libano kraštą juos atvesiu, nes jiems savame krašte bus ankšta. 
\par 11 Jie pereis per vargų jūrą. Aš sulaikysiu jūros bangas, išdžiovinsiu upės gelmes. Asirijos išdidumas bus pažemintas ir Egipto skeptras pašalintas. 
\par 12 Aš sustiprinsiu juos Viešpatyje ir jie vaikščios Jo vardu”,­sako Viešpats.



\chapter{11}


\par 1 Libane, atidaryk savo vartus, ir ugnis tesuėda tavo kedrus! 
\par 2 Verk, kiparise, nes kedras krito, didingi medžiai sunaikinti! Verkite Bašano ąžuolai, nes iškirsta neįžengiamoji giria! 
\par 3 Klausyk­piemenų šauksmas! Sunaikintas jų išdidumas. Klausyk! Tai jaunų liūtų riaumojimas, nes sunaikinta Jordano didybė. 
\par 4 Viešpats, mano Dievas, man kalbėjo: “Ganyk pjovimui skirtas avis. 
\par 5 Pirkliai nesigailėdami pjauna jas. Pirkliai džiaugiasi, nes pralobo. Jų ganytojai taip pat nesigaili jų. 
\par 6 Aš irgi nebesigailėsiu krašto gyventojų. Aš pats atiduosiu kiekvieną į jo valdovo rankas; jie naikins šalį, o Aš jų negelbėsiu”. 
\par 7 Taip aš pradėjau ganyti pjovimui skirtas avis. Aš paėmiau dvi lazdeles: vieną pavadinau Malone, o kitą­Sąjunga. 
\par 8 Per vieną mėnesį aš atleidau tris piemenis. Mano siela bjaurėjosi jais ir jie bjaurėjosi manimi. 
\par 9 Tada tariau: “Nebeganysiu jūsų. Kuri miršta, temiršta, kuri pražūva, tepražūva, o kurios išlieka, tegul ėda viena kitą!” 
\par 10 Aš ėmiau lazdą Malonę ir ją sulaužiau, kad panaikinčiau savo sandorą su visomis tautomis. 
\par 11 Tą pačią dieną ji buvo panaikinta. Avių pirkliai, kurie stebėjo mane, suprato, kad tai buvo Viešpaties žodis. 
\par 12 Aš jiems tariau: “Jei jums atrodo teisinga, užmokėkite man, o jei ne­nemokėkite!” Jie pasvėrė mano užmokestį­trisdešimt sidabrinių. 
\par 13 Viešpats tarė man: “Mesk juos į šventyklos iždą, tą aukštą kainą, kuria jie mane įvertino!” Aš įmečiau tuos trisdešimt sidabrinių į šventyklos iždą. 
\par 14 Po to sulaužiau antrąją lazdą Sąjungą ir tuo nutraukiau brolystę tarp Judo ir Izraelio. 
\par 15 Viešpats tarė man: “Imk dar kartą kvailo ganytojo reikmenis. 
\par 16 Aš paskiriu krašte ganytoją, kuris paklydusia nesirūpins, žūstančios neieškos, sužeistos negydys ir sveikos neganys, tik ės riebiųjų mėsą ir jų nagus nuplėš. 
\par 17 Vargas blogam ganytojui, kuris palieka avis! Kardas jo rankai ir dešinei akiai. Jo ranka nudžius ir dešinioji akis apaks!”



\chapter{12}


\par 1 Viešpaties žodis apie Izraelį. Taip sako Viešpats, kuris sukūrė dangų, žemę ir sutvėrė žmoguje jo dvasią: 
\par 2 “Aš padarysiu Jeruzalę svaiginančia taure visoms aplinkinėms tautoms, ir taip pat Judui per Jeruzalės apgulimą. 
\par 3 Tą dieną Jeruzalę padarysiu sunkiu akmeniu. Visos tautos, norinčios jį pakelti, susižeis. Visos žemės tautos susirinks prieš ją. 
\par 4 Aš pabaidysiu visus žirgus, jų raiteliai bus apimti baimės. Bet Judą stebėsiu atviromis akimis, kai tuo metu visų tautų žirgai bus apakinti. 
\par 5 Tada Judo kunigaikščiai sakys: ‘Jeruzalės gyventojai semiasi stiprybės iš kareivijų Viešpaties­jų Dievo!’ 
\par 6 Judo kunigaikščius padarysiu kaip ugnį malkose ir kaip degantį deglą javų pėduose; jie praris visas aplinkines tautas dešinėje ir kairėje. Jeruzalė vėl bus apgyvendinta savo vietoje,­sako Viešpats”. 
\par 7 Viešpats pirma išgelbės Judo palapines, kad Dovydo namai ir Jeruzalės gyventojai per daug nesiaukštintų prieš Judą. 
\par 8 Tuomet Viešpats apgins Jeruzalės gyventojus, silpniausias iš jų taps kaip Dovydas, o Dovydo namai­kaip Dievas, lyg Viešpaties angelas priešais juos. 
\par 9 Aš sunaikinsiu tautas, sukilusias prieš Jeruzalę, 
\par 10 o Dovydo namams ir Jeruzalės gyventojams išliesiu malonės ir maldos dvasią. Jie žvelgs į mane, kurį jie perdūrė, apraudos, kaip aprauda vienintelį sūnų, ir liūdės, kaip liūdi, netekę pirmagimio. 
\par 11 Tą dieną Jeruzalėje bus toks didelis gedulas, koks jis buvo prie Hadad Rimono, Megido lygumoje. 
\par 12 Visas kraštas liūdės, kiekviena šeima atskirai: Dovydo namų šeima ir jų moterys, Natano namų šeima ir jų moterys, 
\par 13 Levio namų šeima ir jų moterys, Šimio namų šeima ir jų moterys 
\par 14 ir visų likusių namų šeimos ir jų moterys.



\chapter{13}


\par 1 Tą dieną atsivers šaltinis Dovydo namams ir Jeruzalės gyventojams nuplauti nuodėmę ir nešvarą. 
\par 2 Viešpats sako: “Aš išnaikinsiu krašte stabų vardus, ir jų daugiau neminės. Pranašus ir netyras dvasias pašalinsiu iš šalies. 
\par 3 Jei kas dar pasirodys kaip pranašas, jo tėvas ir motina jam tars: ‘Tu mirsi, nes kalbėjai melą Viešpaties vardu’. Jo paties gimdytojai jį perdurs, jam pranašaujant. 
\par 4 Pranašai gėdysis savo regėjimų ir nebevilkės šiurkštaus apsiausto pranašaudami ir apgaudinėdami. 
\par 5 Kiekvienas sakys: ‘Aš ne pranašas, aš žemdirbys nuo pat savo jaunystės!’ 
\par 6 Ir jei kas klaus: ‘Ką reiškia šitie randai tavo nugaroje?’, jis atsakys: ‘Mane sumušė mano artimieji’. 
\par 7 Karde, pakilk prieš mano ganytoją, mano artimiausią!­sako kareivijų Viešpats.­Aš ištiksiu ganytoją, ir avys išsisklaidys. Aš atgręšiu ranką prieš mažuosius. 
\par 8 Visame krašte,­sako Viešpats,­ du trečdaliai bus sunaikinti ir žus, tik trečdalis išliks. 
\par 9 Ir tą trečdalį vesiu per ugnį; valysiu ir gryninsiu juos kaip sidabrą ir auksą. Jie šauksis mano vardo, ir Aš juos išklausysiu. Aš sakysiu: ‘Tai mano tauta’, o jie atsakys: ‘Viešpats yra mūsų Dievas’ ”.



\chapter{14}


\par 1 Viešpaties diena ateina, ir tavo grobis bus išdalintas tavo viduryje. 
\par 2 Visos tautos bus sušauktos kovai prieš Jeruzalę. Miestas kris, namai bus apiplėšti, moterys išprievartautos. Pusė miesto gyventojų bus išvesti nelaisvėn, o kita pusė liks mieste. 
\par 3 Tada Viešpats kariaus prieš anas tautas, kaip kariavo mūšio dieną. 
\par 4 Tą dieną Jis stovės Alyvų kalne, kuris yra į rytus nuo Jeruzalės. Alyvų kalnas perskils į dvi dalis­iš rytų į vakarus, ir pasidarys labai didelis slėnis. Viena kalno pusė nuslinks į šiaurę, o kita­į pietus. 
\par 5 Jūs bėgsite į kalnų slėnį, kuris sieks iki Acalo. Jūs bėgsite, kaip bėgote nuo žemės drebėjimo Judo karaliaus Uzijo dienomis. Tada ateis Viešpats, mano Dievas, ir visi šventieji su Juo. 
\par 6 Tuomet visi šviesuliai užges. 
\par 7 Bus viena diena, kurią žino Viešpats­ne diena ir ne naktis, ir vakare bus šviesu. 
\par 8 Tą dieną gyvieji vandenys tekės iš Jeruzalės dviem kryptim: viena į Rytų, kita į Vakarų jūrą; taip bus vasarą ir žiemą. 
\par 9 Viešpats karaliaus visai žemei. Bus vienas Viešpats ir Jo vardas bus vienas. 
\par 10 Visa šalis nuo Gebos iki Rimono, esančio į pietus nuo Jeruzalės, taps lyguma. Miestas, nuo Benjamino vartų iki Kertinių vartų ir nuo Hananelio bokšto iki karaliaus vyno spaustuvų, bus apgyventas. 
\par 11 Prakeikimo nebebus ir Jeruzalėje visi saugiai gyvens. 
\par 12 Maras, kurį Viešpats siųs visoms tautoms, kariaujančioms prieš Jeruzalę, bus baisus. Jų kūnai pradės pūti dar jiems stovint ant savo kojų, akys­akiduobėse, o jų liežuviai­burnose. 
\par 13 Tuomet žmonėse bus toks didelis sąmyšis nuo Viešpaties, kad pakels ranką artimas prieš artimą. 
\par 14 Net Judas kariaus prieš Jeruzalę. Visų aplinkinių tautų turtai bus surinkti: auksas, sidabras ir drabužiai. 
\par 15 Toks pat maras kankins žirgus, mulus, kupranugarius, asilus ir visus kitus gyvulius, kurie bus kariaujančiųjų stovyklose. 
\par 16 Tie, kurie liks iš kariavusių prieš Jeruzalę, kas metai eis pagarbinti Karalių, kareivijų Viešpatį ir švęsti Palapinių šventę. 
\par 17 Jei kuri žemės giminė neateis į Jeruzalę pagarbinti Karaliaus, kareivijų Viešpaties, tai pas juos nebus lietaus. 
\par 18 Jei neateis Egipto giminė, juos ištiks tas pats maras, kurį Viešpats siųs pagonims, neateinantiems švęsti Palapinių šventės. 
\par 19 Tai bus bausmė Egiptui ir visoms tautoms, kurios neateis į Palapinių šventę. 
\par 20 Tuomet ant žirgų varpelių bus užrašyta: “Pašvęstas Viešpačiui”. Puodai Viešpaties namuose bus kaip aukojimo taurės prie aukuro. 
\par 21 Kiekvienas puodas Jeruzalėje ir Jude bus pašvęstas kareivijų Viešpačiui. Visi aukotojai galės ateiti, paimti bet kurį iš jų ir jame virti. Tomis dienomis nebebus kanaaniečio kareivijų Viešpaties namuose.




\end{document}