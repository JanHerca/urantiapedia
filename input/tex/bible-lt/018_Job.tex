\begin{document}

\title{Jobo knyga}

\chapter{1}

\par 1 Uco krašte gyveno vyras, vardu Jobas. Jis buvo tobulas ir teisus vyras, bijojo Dievo ir vengė pikto. 
\par 2 Jis turėjo septynis sūnus ir tris dukteris, 
\par 3 septynis tūkstančius avių, tris tūkstančius kupranugarių, penkis šimtus jungų jaučių, penkis šimtus asilių ir labai didelę šeimyną. Tas vyras buvo žymiausias Rytuose. 
\par 4 Jo sūnūs keldavo vaišes kiekvienas savo namuose savo dieną, pasikvietę tris seseris kartu su jais valgyti ir gerti. 
\par 5 Vaišių dienoms pasibaigus, Jobas juos šventindavo. Atsikėlęs anksti rytą, jis aukodavo deginamąsias aukas pagal jų skaičių, galvodamas: “Gal mano sūnūs nusidėjo ir keikė Dievą savo širdyse”. Taip Jobas visuomet darydavo. 
\par 6 Vieną dieną Dievo sūnūs susirinko pas Viešpatį; atėjo ir šėtonas. 
\par 7 Viešpats klausė šėtoną: “Iš kur ateini?” Šėtonas atsakė Viešpačiui: “Aš vaikštinėjau po visą žemę”. 
\par 8 Viešpats vėl klausė šėtoną: “Ar atkreipei dėmesį į mano tarną Jobą? Juk žemėje nėra nė vieno jam lygaus. Jis yra tobulas ir teisus, bijo Dievo ir vengia pikto”. 
\par 9 Šėtonas atsakė Viešpačiui: “Ne veltui Jobas bijo Dievo. 
\par 10 Juk Tu saugoji jį, jo namus ir viską, ką jis turi! Jo darbus Tu laimini ir jo turtas didėja. 
\par 11 Bet ištiesk savo ranką ir paliesk tai, ką jis turi, ir jis keiks Tave į akis”. 
\par 12 Viešpats tarė šėtonui: “Visa, ką jis turi, atiduodu tavo valdžion, bet prieš jį neištiesk rankos”. Šėtonas pasišalino iš Viešpaties akivaizdos. 
\par 13 Tą dieną Jobo sūnūs ir dukterys valgė ir gėrė vyną vyriausiojo brolio namuose. 
\par 14 Pasiuntinys, atėjęs pas Jobą, pranešė: “Jaučiai arė ir asilai ganėsi šalia jų. 
\par 15 Šebiečiai užpuolė ir juos pagrobė, o tarnus užmušė kardu; tik aš vienas ištrūkau, kad tau tai praneščiau”. 
\par 16 Jam tebekalbant, atėjo kitas ir pranešė: “Dievo ugnis krito iš dangaus ir sudegino avis ir tarnus, ir juos prarijo; tik aš vienas ištrūkau, kad tau praneščiau”. 
\par 17 Jam tebekalbant, atėjo kitas ir pranešė: “Chaldėjai, pasiskirstę trimis grupėmis, puolė kupranugarius ir juos pagrobė, o tarnus nužudė kardu; tik aš vienas ištrūkau, kad tau praneščiau”. 
\par 18 Jam tebekalbant, atėjo kitas ir pranešė: “Tavo sūnūs ir dukterys valgė ir gėrė vyną vyriausiojo brolio namuose. 
\par 19 Pakilęs smarkus vėjas iš dykumos sugriovė namą ir užmušė visus jaunuolius; tik aš vienas ištrūkau, kad tau praneščiau”. 
\par 20 Jobas atsikėlė, perplėšė savo apsiaustą, nusiskuto plaukus ir, puolęs ant žemės, pagarbino, 
\par 21 tardamas: “Nuogas gimiau, nuogas ir mirsiu. Viešpats davė, Viešpats ir atėmė; tebūna palaimintas Viešpaties vardas”. 
\par 22 Visu tuo Jobas nenusidėjo ir nekalbėjo kvailai prieš Dievą.


\chapter{2}


\par 1 Vieną dieną Dievo sūnūs vėl susirinko Viešpaties akivaizdoje; šėtonas atėjo su jais ir stojo prieš Dievą. 
\par 2 Viešpats paklausė jo: “Iš kur ateini?” Šėtonas atsakė Viešpačiui: “Aš vaikštinėjau po visą žemę”. 
\par 3 Viešpats klausė šėtoną: “Ar atkreipei dėmesį į mano tarną Jobą? Juk žemėje nėra jam lygaus. Jis yra tobulas ir teisus, bijo Dievo ir vengia pikto. Jis ir toliau lieka man ištikimas, nors tu mane sukurstei prieš jį, kad be priežasties jam kenkčiau”. 
\par 4 Šėtonas atsakė Viešpačiui: “Oda už odą, bet žmogus viską atiduos, kad liktų gyvas. 
\par 5 Bet ištiesk savo ranką ir paliesk jo kaulus ir kūną, ir jis keiks Tave į akis”. 
\par 6 Viešpats tarė šėtonui: “Jis tavo rankose. Tik nepaliesk jo gyvybės”. 
\par 7 Šėtonas pasišalino iš Viešpaties akivaizdos ir ištiko Jobą skaudžiomis votimis nuo kojų padų iki viršugalvio. 
\par 8 Jis sėdėjo pelenuose ir, paėmęs šukę, gramdė pūlius. 
\par 9 Jo žmona jam tarė: “Ar tu vis dar laikaisi savo ištikimybės? Prakeik Dievą ir mirk”. 
\par 10 Jobas jai atsakė: “Tu kalbi kaip viena iš kvailų moterų. Argi priimtume gera iš Dievo rankos, o pikta nepriimtume?” Visu tuo Jobas nenusidėjo savo lūpomis. 
\par 11 Trys Jobo draugai: temanas Elifazas, šuachas Bildadas ir naamatietis Cofaras, išgirdę apie Jobo nelaimę, susitarė ir atėjo gedėti su juo ir jį paguosti. 
\par 12 Pamatę jį iš tolo, nepažino jo. Jie perplėšė savo apsiaustus, balsiai verkė ir mėtė dulkes virš savo galvų link dangaus. 
\par 13 Septynias dienas ir septynias naktis jie sėdėjo su juo ant žemės; nė vienas nepratarė žodžio, nes jie matė, kad jo kančia labai didelė.



\chapter{3}


\par 1 Pagaliau Jobas atvėrė burną ir prakeikė savo dieną. 
\par 2 Jobas prabilo ir tarė: 
\par 3 “Tegul pražūna diena, kurią gimiau, ir naktis, kurią buvau pradėtas. 
\par 4 Tegul ta diena tampa tamsybe. Dieve, neprisimink jos ir neduok jai šviesos. 
\par 5 Te tamsa ir mirties šešėlis apgaubia ją, te debesis aptemdo ją ir juoduma tepadaro ją baisią. 
\par 6 Ta naktis tegul būna tamsi; tegul ji bus išbraukta iš metų ir mėnesių dienų skaičiaus. 
\par 7 Ta naktis tegul būna apleista ir tenesigirdi joje džiaugsmingo balso. 
\par 8 Tegul prakeikia tą dieną tie, kurie gali pažadinti leviataną. 
\par 9 Tegul aptemsta aušros žvaigždės ir nepasirodo laukiama šviesa, akys teneišvysta aušros spindulių. 
\par 10 Nes ji neužvėrė mano motinos įsčių ir nepaslėpė vargo nuo manęs. 
\par 11 Kodėl nemiriau gimdamas ir kodėl neatidaviau dvasios, išeidamas iš pilvo? 
\par 12 Kodėl mane laikė ant kelių ir maitino krūtimi? 
\par 13 Tada gulėčiau ramus ir tylus ir miegočiau, ir ilsėčiausi 
\par 14 kartu su žemės karaliais ir patarėjais, kurie atstatė sau apleistas vietas, 
\par 15 arba su kunigaikščiais, kurie turėjo aukso ir pripildė savo namus sidabro, 
\par 16 arba kaip paslėptas nelaiku gimęs kūdikis, neregėjęs šviesos. 
\par 17 Ten piktadariai nebesiaučia ir pavargusieji ilsisi. 
\par 18 Ten belaisviai ilsisi kartu ir nebegirdi prižiūrėtojo balso. 
\par 19 Didelis ir mažas yra ten, vergas ten yra laisvas nuo savo valdovo. 
\par 20 Kodėl šviesa duodama tam, kuris kenčia, ir gyvybė apkartusiai sielai? 
\par 21 Laukiantieji mirties jos nesulaukia; jie jos ieško labiau negu paslėptų turtų. 
\par 22 Jie džiaugiasi ir yra labai patenkinti, kai suranda sau kapą. 
\par 23 Kodėl duota šviesa žmogui, kurio kelias paslėptas ir kurį Dievas spaudžia iš visų pusių? 
\par 24 Mano dūsavimai kyla prieš valgant, o aimanos liejasi kaip tekantis vanduo. 
\par 25 Tai, ko labai bijojau, užgriuvo mane, ir tai, dėl ko nuogąstavau, ištiko mane. 
\par 26 Aš nebuvau saugus ir neturėjau poilsio, aš nenurimdavau, tačiau bėda atėjo”.



\chapter{4}


\par 1 Temanas Elifazas atsakydamas tarė: 
\par 2 “Jei kalbėsime tau, gal tau ir nepatiks, tačiau kas gali susilaikyti nekalbėjęs? 
\par 3 Tu daugelį pamokei ir sustiprinai jų pailsusias rankas. 
\par 4 Klumpantį tavo žodžiai palaikė, linkstančius jo kelius tu sutvirtinai. 
\par 5 Dabar tai užgriuvo tave, ir tu nusilpai; tai palietė tave, ir tu sunerimęs. 
\par 6 Ar tai tavo baimė, pasitikėjimas, viltis ir tiesumas tavo kelių? 
\par 7 Pagalvok, kas, būdamas nekaltas, pražuvo? Ar teisusis buvo sunaikintas? 
\par 8 Kiek esu matęs, kas aparė blogį ir pasėjo piktadarystes, tai ir nupjovė. 
\par 9 Nuo Dievo pūstelėjimo jie žuvo; Jo rūstybės kvapas juos sunaikino. 
\par 10 Liūto riaumojimas ir piktos liūtės balsas nutildomi, jaunų liūtų dantys išdaužomi. 
\par 11 Senas liūtas žūva, neradęs grobio, ir liūtės jaunikliai išsisklaido. 
\par 12 Paslaptis mane aplankė ir mano ausis ją nugirdo. 
\par 13 Mąstant apie nakties regėjimus, kai gilus miegas buvo apėmęs žmones, 
\par 14 mane apėmė išgąstis ir drebėjimas, ir visi mano kaulai tirtėjo. 
\par 15 Dvasia praėjo pro mano veidą, ir mano plaukai pasišiaušė. 
\par 16 Ji stovėjo, tačiau jos neatpažinau. Pavidalas buvo prieš mano akis; buvo tylu, ir aš išgirdau balsą: 
\par 17 ‘Ar mirtingas žmogus gali būti teisesnis už Dievą? Ar jis gali būti tyresnis už savo Kūrėją? 
\par 18 Savo tarnais Jis nepasitiki ir mato angelų klaidas. 
\par 19 Juo labiau tie, kurie gyvena molio namuose, kurių pamatai­dulkės. Jie sunyks kandžių suėsti. 
\par 20 Jie naikinami nuo ryto iki vakaro ir pražūna niekieno nepastebimi. 
\par 21 Argi jų didybė nepranyksta? Jie miršta tiesos nesuvokdami’ ”.



\chapter{5}


\par 1 “Šauk, jei kas nors tau atsakys. Į kurį iš šventųjų kreipsies? 
\par 2 Kvailį sunaikina pyktis, ir prastuolis žūva dėl pavydo. 
\par 3 Aš mačiau kvailį, kuris suleido šaknis, tačiau tuoj pat prakeikiau jo buveinę. 
\par 4 Jo vaikai nėra saugūs; jie yra mušami vartuose, ir niekas jų neišgelbsti. 
\par 5 Alkanas suvalgo jų derlių, erškėčiai nekliudo jam pasiimti. Plėšikas praryja jo nuosavybę. 
\par 6 Vargas neiškyla iš dulkių ir bėda neišauga iš žemės. 
\par 7 Tačiau žmogus gimęs vargti, kaip paukštis skrajoti. 
\par 8 Aš ieškočiau Dievo ir patikėčiau savo bylą Jam, 
\par 9 kuris daro didelių, neištiriamų ir nuostabių dalykų be skaičiaus. 
\par 10 Jis duoda žemei lietaus ir siunčia vandens laukams. 
\par 11 Jis pakelia pažemintus ir liūdinčius nuramina. 
\par 12 Gudriųjų sumanymus Jis paverčia niekais, todėl jų darbai nesėkmingi. 
\par 13 Jis sugauna gudriuosius jų pačių klastose, ir sukčių sumanymai nueina niekais. 
\par 14 Dienos metu jie susiduria su tamsa ir vidudienį vaikšto apgraibomis kaip naktį. 
\par 15 Jis išgelbsti vargšą nuo kardo, nuo jų kalbų ir stipriųjų rankos. 
\par 16 Vargšas turi viltį, o neteisybei užčiaupiama burna. 
\par 17 Laimingas žmogus, kurį Dievas pamoko, todėl nepaniekink Visagalio drausmės. 
\par 18 Jis sužeidžia, bet ir aptvarsto, Jis sumuša, tačiau ir pagydo. 
\par 19 Jis išgelbės tave iš šešių nelaimių, o septintoje pikta nepalies tavęs. 
\par 20 Bado metu Jis išpirks tave iš mirties, o kare­nuo kardo jėgos. 
\par 21 Tavęs nepalies liežuvių plakimai ir nebaugins gresiantis sunaikinimas. 
\par 22 Sunaikinimo ir bado metu tu juoksiesi, laukinių žvėrių nebijosi. 
\par 23 Lauko akmenys bus tavo sąjungininkai, o laukiniai žvėrys bus taikoje su tavimi. 
\par 24 Tu patirsi, kad tavo palapinė bus saugi, tu lankysiesi savo buveinėje ir nenusidėsi. 
\par 25 Tu patirsi, kad tavo sėkla bus gausi, o tavo palikuonys kaip žolė lankoje. 
\par 26 Tu nueisi į kapą senatvėje, būsi kaip javų pėdai, suvežami savo laiku. 
\par 27 Mes tai ištyrėme ir taip yra. Klausyk ir žinok tai savo labui”.



\chapter{6}

\par 1 Jobas atsakydamas tarė: 
\par 2 “O kad pasvertų mano vargą ir ant svarstyklių uždėtų mano kentėjimus! 
\par 3 Visa tai svertų daugiau už jūros smėlį. Todėl aš nuryju savo žodžius. 
\par 4 Visagalio strėlės įsmeigtos į mane, jų nuodus turi gerti mano dvasia. Dievo baisenybės išsirikiavę prieš mane. 
\par 5 Ar žvengia laukinis asilas, turėdamas žolės? Ar baubia jautis prie savo pašaro? 
\par 6 Ar galima valgyti beskonį dalyką be druskos? Ar kiaušinio baltymas turi skonį? 
\par 7 Tai, kuo bjaurėdavosi mano siela, yra mano suspaudimo maistas. 
\par 8 O kad įvyktų, ko prašau, ir Dievas suteiktų man, ko ilgiuosi. 
\par 9 Kad patiktų Dievui sunaikinti mane, rankos pakėlimu pribaigti mane. 
\par 10 Tai būtų man paguoda ir aš džiaugčiausi kentėdamas. Tenesigaili Jis manęs, nes aš neišsigyniau Šventojo žodžių. 
\par 11 Iš kur man jėgos, kad turėčiau viltį? Koks galas, kad aš toliau gyvenčiau? 
\par 12 Ar mano jėga yra akmens jėga? Ar mano kūnas iš vario? 
\par 13 Manyje nėra pagalbos ir išmintis pasitraukė nuo manęs. 
\par 14 Kenčiantis turėtų susilaukti gailestingumo iš savo draugo, tačiau jis atsisako Visagalio baimės. 
\par 15 Mano broliai yra klastingi kaip upelis, kaip vandens srovės, tekančios pro šalį. 
\par 16 Jie yra lyg tamsus ledas, padengtas sniegu. 
\par 17 Saulei kaitinant, jie pradingsta, karščiui užėjus­išnyksta. 
\par 18 Jų kelias pasuka į šalį, jie teka į tuštumą ir pranyksta. 
\par 19 Temos ir Šebos karavanai tyrinėjo juos ir pasitikėjo jais. 
\par 20 Tačiau jų viltis apvylė juos, jie atėjo ir buvo sugėdinti. 
\par 21 Jūs esate niekas, nes pamatę mano pažeminimą, išsigandote. 
\par 22 Argi ar prašiau: ‘Duokite man dovanų iš savo turto?’ 
\par 23 Arba: ‘Išgelbėkite mane iš priešo rankų. Išpirkite mane iš prispaudėjų’. 
\par 24 Pamokykite mane, ir aš nutilsiu; duokite man suprasti mano klaidas. 
\par 25 Kokie stiprūs yra tiesos žodžiai, o jūsų kalbos nieko neįrodo. 
\par 26 Jūs savo žodžiais man tik prikaišiojate; jie nuliūdusiam praeina lyg vėjas. 
\par 27 Jūs puolate našlaitį ir savo draugui kasate duobę. 
\par 28 Dabar pažvelkite į mane ir matysite, ar aš meluoju. 
\par 29 Atsakykite, kad nebūtų netiesos. Atsakykite, ar aš ne teisus? 
\par 30 Spręskite, ar aš netiesą kalbu? Ar aš neatskiriu tiesos nuo melo?”



\chapter{7}


\par 1 “Ar nėra žmogui skirto laiko žemėje? Ar jo dienos nėra kaip samdinio dienos? 
\par 2 Kaip vergas trokšta pavėsio ir samdinys laukia algos, 
\par 3 taip aš gavau tuštybės mėnesius ir vargo naktys skirtos man. 
\par 4 Kai atsigulu, galvoju: ‘Kada pasibaigs naktis ir aš atsikelsiu?’ Taip aš vargstu ir kenčiu iki aušros. 
\par 5 Mano kūnas aplipęs kirmėlėmis ir purvais, mano oda sutrūkinėjusi ir susitraukusi. 
\par 6 Mano dienos greitesnės už audėjo šaudyklę ir baigiasi neviltimi. 
\par 7 Atsimink, kad mano gyvenimas tėra vėjas; mano akys neberegės gero. 
\par 8 Akys to, kuris mane matė, nebematys manęs; Tu žiūrėsi, bet manęs nebebus. 
\par 9 Kaip debesis nueina ir dingsta, taip nuėjęs į kapą nebesugrįžta. 
\par 10 Jis nebegrįš į savo namus, jo vieta nebepažins jo. 
\par 11 Aš neužversiu savo burnos, kalbėsiu dvasios skausme, skųsiuos savo sielos kartume. 
\par 12 Ar aš esu jūra, ar banginis, kad statai man sargybą? 
\par 13 Kai sakau: ‘Mano lova paguos mane, mano guolis palengvins mano skundą’, 
\par 14 Tu baugini mane sapnais ir gąsdini regėjimais. 
\par 15 Todėl mano siela pasirinktų būti pasmaugta, ir mirtis man geriau už gyvenimą. 
\par 16 Aš bjauriuosi juo ir nebenoriu gyventi. Palik mane, mano dienos­tuštybė. 
\par 17 Kas yra žmogus, kad jį laikai pagarboje ir kreipi į jį savo dėmesį? 
\par 18 Aplankai jį kas rytą, kas akimirką jį mėgini. 
\par 19 Kada paliksi mane ir leisi ramiai nuryti seilę? 
\par 20 Jei nusidėjau, ką Tau padarysiu, žmonių sarge? Kodėl mane pasirinkai taikiniu, kad būčiau sau našta? 
\par 21 Kodėl neatleidi mano kaltės ir nepanaikini mano nusikaltimo? Aš gulėsiu dulkėse; Tu ieškosi manęs rytą, tačiau manęs nebebus”.



\chapter{8}


\par 1 Šuachas Bildadas atsakydamas tarė: 
\par 2 “Ar dar ilgai tu šitaip kalbėsi? Tavo žodžiai yra kaip stiprus vėjas. 
\par 3 Argi Dievas neteisingai teisia, ar Visagalis iškreipia teisingumą? 
\par 4 Jei tavo sūnūs Jam nusidėjo, Jis juos atidavė jų nusikaltimams. 
\par 5 Jei tu ieškosi Dievo ir maldausi Visagalį, 
\par 6 būsi tyras ir doras, tai Jis pakils dėl tavęs ir duos klestėjimą tavo teisumo buveinei. 
\par 7 Nors tavo pradžia buvo maža, tačiau galiausiai tai labai išaugs. 
\par 8 Patyrinėk ankstesnius laikus ir sužinok, ką patyrė jų tėvai. 
\par 9 Mes gyvename tik nuo vakar dienos ir nieko nežinome, nes mūsų dienos žemėje lyg šešėlis. 
\par 10 Jie tikrai pamokys tave, duos nuoširdžių patarimų. 
\par 11 Ar auga papirusas, kur nėra drėgmės, ir nendrės be vandens? 
\par 12 Dar žydėdamas ir nenuskintas, jis sudžiūsta pirma visų žolių. 
\par 13 Tokie keliai yra visų, kurie pamiršta Dievą; ir veidmainių viltis pražus. 
\par 14 Jų viltis sunyks ir jų pasitikėjimas­ tik voratinklis. 
\par 15 Atsirems į savo namus, bet jie sugrius, įsitvers jų, bet jie neatlaikys. 
\par 16 Jis žaliuoja saulėje, jo atžalos plečiasi sode. 
\par 17 Akmenų krūvą apraizgo jo šaknys, jos laikosi akmenuotoje žemėje. 
\par 18 Jei Jis išraus jį iš tos vietos, ši išsigins jo: ‘Aš tavęs niekada nemačiau’. 
\par 19 Toks yra jo kelių džiaugsmas, o iš žemės auga kiti. 
\par 20 Dievas neatmes tobulojo ir nepadės piktadariams. 
\par 21 Jis pripildys tavo burną juoko ir tavo lūpas džiaugsmo. 
\par 22 Tie, kurie tavęs nekenčia, bus aprengti gėda, ir nedorėlio palapinės sunyks”.



\chapter{9}

\par 1 Jobas atsakydamas tarė: 
\par 2 “Tikrai žinau, kad taip yra. Bet kaip žmogus gali būti teisus prieš Dievą? 
\par 3 Jei jis ginčytųsi su Juo, negalėtų Jam atsakyti nė į vieną iš tūkstančio. 
\par 4 Jis išmintingas širdyje ir galingas jėga. Kas, užsikietinęs prieš Jį, turėjo sėkmę? 
\par 5 Jis perkelia kalnus nepastebimai, rūstaudamas juos sunaikina. 
\par 6 Jis sudrebina žemę ir supurto jos stulpus. 
\par 7 Jis įsako saulei, ir ji nepateka, ir žvaigždes užantspauduoja. 
\par 8 Jis vienas ištiesė dangus ir žingsniuoja jūros bangomis. 
\par 9 Jis padarė Grįžulo ratus, Orioną ir Sietyną bei Pietų skliauto žvaigždynus. 
\par 10 Jis padaro didelių, mums nesuvokiamų dalykų ir nesuskaičiuojamų stebuklų. 
\par 11 Štai Jis praeina pro mane, bet aš Jo nepastebiu; Jis eina tolyn, o aš Jo nematau. 
\par 12 Jei Jis atims, kas sutrukdys Jam ir paklaus: ‘Ką Tu darai?’ 
\par 13 Jei Dievas nesulaikys savo rūstybės, išdidūs padėjėjai nusilenks prieš Jį. 
\par 14 Kaip tad aš galėčiau Jam pasiteisinti ir parinkti tinkamus žodžius? 
\par 15 Jei aš ir būčiau teisus, negalėčiau atsakyti Jam, bet turėčiau maldauti savo Teisėją. 
\par 16 Jei aš šaukčiausi ir Jis atsakytų man, nepatikėčiau, kad Jis manęs klauso. 
\par 17 Jis viesulu palaužia mane ir daugina mano žaizdas be priežasties; 
\par 18 Jis neleidžia man atsikvėpti, bet pripildo mane kartybių. 
\par 19 Jei kalbėčiau apie jėgą, Jis stiprus! O jei apie teismą, kas paskirs man laiką bylinėtis? 
\par 20 Jei teisinčiau save, mano paties žodžiai pasmerktų mane; jei būčiau nekaltas, jie mane kaltintų. 
\par 21 Aš esu nekaltas. Bet nebenoriu pažinti savo sielos ir niekinu savo gyvybę. 
\par 22 Nėra jokio skirtumo. Todėl sakau: ‘Jis sunaikina kaltą ir nekaltą’. 
\par 23 Jei netikėta nelaimė pražudo, Jis juokiasi iš nekaltųjų išmėginimų. 
\par 24 Žemė atiduota nedorėliams, Jis uždengia teisėjų veidus. Kas gi visa tai daro, ar ne Jis? 
\par 25 Mano dienos greitesnės už pasiuntinį; jos nubėgo, nematę nieko gero. 
\par 26 Jos pralėkė kaip greiti laivai, kaip erelis, puoląs grobį. 
\par 27 Jei sakyčiau: ‘Aš pamiršiu savo skundą, paliksiu savo sunkumą ir paguosiu save’, 
\par 28 tai bijausi visų savo kančių, žinodamas, kad Tu manęs nelaikysi nekaltu. 
\par 29 Jei aš esu nedorėlis, tai kam veltui stengtis? 
\par 30 Jei nusiprausčiau sniego vandeniu ir kaip niekada švariai nusiplaučiau rankas, 
\par 31 Tu vis tiek įstumtum mane į purvą, ir mano rūbai baisėtųsi manimi. 
\par 32 Jis nėra žmogus kaip aš, kad Jam galėčiau atsakyti ir abu galėtumėme stoti į teismą. 
\par 33 Tarp mūsų nėra tarpininko, kuris galėtų uždėti ant mūsų rankas. 
\par 34 O kad Jis patrauktų nuo manęs savo lazdą ir manęs nebegąsdintų. 
\par 35 Tada kalbėčiau nebijodamas Jo, bet dabar taip nėra”.



\chapter{10}


\par 1 “Mano siela pavargo nuo gyvenimo, tad skųsiuos atvirai, kalbėsiu iš sielos kartėlio. 
\par 2 Sakysiu Dievui: ‘Nepasmauk manęs, parodyk man, kodėl su manimi kovoji. 
\par 3 Ar Tau gera prispausti ir paniekinti savo rankų darbą, ir šviesti nedorėlių pasitarime? 
\par 4 Ar Tavo akys kūniškos, ar matai taip, kaip žmogus? 
\par 5 Ar Tavo dienos kaip žmogaus dienos ir Tavo metai kaip žmonių laikas, 
\par 6 kad ieškai mano kaltės ir teiraujiesi mano nuodėmių, 
\par 7 nors žinai, kad nesu nedorėlis? Nėra nė vieno, kuris mane išgelbėtų iš Tavo rankų. 
\par 8 Tavo rankos padarė mane, o dabar nori mane sunaikinti. 
\par 9 Atsimink, kad mane iš molio padarei ir vėl į dulkes paversi. 
\par 10 Kaip pieną mane išliejai ir kaip sūrį suspaudei. 
\par 11 Tu apvilkai mane kūnu ir oda, kaulais ir gyslomis sutvirtinai mane. 
\par 12 Gyvybę ir palankumą man suteikei, Tavo aplankymas saugojo mano dvasią. 
\par 13 Visa tai paslėpei savo širdyje; žinau, kad tai yra su Tavimi. 
\par 14 Jeigu nusidedu, Tu pastebi tai ir mano kalčių neatleidi. 
\par 15 Jei aš nedorėlis, vargas man! O jei aš ir teisus, negaliu pakelti galvos, nes esu pilnas gėdos. Pažiūrėk į mano vargą, 
\par 16 nes jis didėja. Kaip liūtas mane medžioji ir pasirodai baisingas prieš mane. 
\par 17 Tu pastatai naujus liudytojus prieš mane ir daugini savo pasipiktinimą; permainos ir karai kyla prieš mane. 
\par 18 Kodėl leidai man gimti? O kad būčiau miręs ir niekas nebūtų manęs matęs. 
\par 19 Aš būčiau tarsi nebuvęs ir iš įsčių būčiau nuneštas į kapą. 
\par 20 Mano gyvenimo dienų mažai; palik mane, kad nors kiek atsikvėpčiau, 
\par 21 prieš išeidamas ten, iš kur negrįžta, į tamsos šalį ir mirties šešėlį. 
\par 22 Į gūdžios tamsos šalį, kur mirties šešėlis, kur nėra skirtumo tarp šviesos ir tamsos’ ”.



\chapter{11}


\par 1 Naamatietis Cofaras atsakydamas tarė: 
\par 2 “Ar žodžių gausybė neturi būti atsakyta? Ar, daug kalbėdamas, būsi išteisintas? 
\par 3 Ar tavo melai galėtų nutildyti vyrus? Kai tu tyčiojiesi, ar niekas tavęs nesugėdins? 
\par 4 Tu sakei: ‘Mano mokslas yra tikras, aš esu teisus Tavo akyse’. 
\par 5 O kad Dievas, pravėręs lūpas, prabiltų prieš tave. 
\par 6 Jis apreikštų tau išminties paslaptis, ir tu pamatytum, kad du kartus daugiau turėtum kentėti. Žinok, kad Dievas iš tavęs reikalauja mažiau, negu verta tavo neteisybė. 
\par 7 Ar gali tyrinėdamas suprasti Dievą? Ar gali tobulai suprasti Visagalį? 
\par 8 Jis aukščiau už dangų. Ką tu gali padaryti? Jis giliau už pragarą. Ką tu gali žinoti? 
\par 9 Jis ilgesnis už žemę ir platesnis už jūrą. 
\par 10 Jei Jis praeidamas suima ir patraukia tieson, kas Jam užgins? 
\par 11 Jis pažįsta žmonių tuštybę, mato jų nedorybes. Ar Jis nekreips į tai dėmesio? 
\par 12 Tuščias žmogus dedasi išmintingas, nors gimsta kaip laukinio asilo jauniklis. 
\par 13 Jei tu paruoši savo širdį ir ištiesi savo rankas į Jį, 
\par 14 jei pašalinsi savo kaltes ir neleisi nedorybei gyventi tavyje, 
\par 15 tada pakelsi savo veidą be dėmės, būsi tvirtas ir nieko nebijosi. 
\par 16 Tada pamirši buvusį vargą, prisiminsi jį kaip nutekėjusį vandenį. 
\par 17 Tavo gyvenimas bus šviesus kaip vidudienis, tu nušvisi kaip rytas. 
\par 18 Tu būsi saugus, nes yra viltis, tu apsiklosi ir ilsėsies ramybėje. 
\par 19 Tu atsigulsi ir niekas tavęs neišgąsdins, daugelis ieškos tavo pagalbos. 
\par 20 Tačiau nedorėlių akys užges, jie nepaspruks, jų viltis kaip paskutinis atodūsis”.



\chapter{12}


\par 1 Jobas atsakydamas tarė: 
\par 2 “Iš tikrųjų jūs esate žmonės, su kuriais išmintis mirs! 
\par 3 Aš turiu supratimą kaip ir jūs ir nesu už jus menkesnis. O kas viso to nežino? 
\par 4 Mano artimas išjuokia mane. Aš šaukiausi Dievo, ir Jis mane išklausė, o jūs juokiatės iš teisaus ir doro žmogaus. 
\par 5 Tas, kuris saugus, paniekina žiburį, paruoštą tam, kurio koja paslysta. 
\par 6 Plėšikų palapinės pilnos ir Dievą rūstinantys saugūs, bet Dievas atidavė viską į jų rankas. 
\par 7 Paklausk žvėrių, jie tave pamokys, ir padangių paukščių­jie tau pasakys. 
\par 8 Kalbėk žemei, ji tau patars, jūros žuvys paaiškins tau. 
\par 9 Kas iš viso to nepažins, kad Viešpaties ranka tai padarė? 
\par 10 Jo rankoje yra kiekvieno gyvio siela ir kiekvieno žmogaus kvapas. 
\par 11 Ausis skiria žodžius, burna jaučia maisto skonį. 
\par 12 Su senoliais išmintis, ir ilgaamžiai turi supratimą. 
\par 13 Su Juo išmintis ir galia, Jis išmano ir pamoko. 
\par 14 Jei Jis sugriauna, niekas nebegali atstatyti; jei Jis uždaro žmogų, niekas nebeatidaro. 
\par 15 Kai Jis sulaiko vandenis, jie išdžiūsta; kai siunčia juos, sunaikina žemę. 
\par 16 Jis yra galingas ir išmintingas, Jo žinioje apgavikas ir apgautasis. 
\par 17 Krašto patarėjams jis atima išmintį, o teisėjus padaro kvailus. 
\par 18 Jis atpalaiduoja karalių pančius ir juos sujuosia. 
\par 19 Jis atima kunigaikščiams garbę ir nuverčia galinguosius. 
\par 20 Jis nutildo patikimus kalbėtojus, o iš vyresniųjų atima supratimą. 
\par 21 Jis lieja paniekinimą ant kunigaikščių ir susilpnina galiūnų jėgas. 
\par 22 Jis atidengia, kas paslėpta tamsoje, ir iškelia švieson mirties šešėlį. 
\par 23 Jis iškelia tautas ir jas sunaikina; išsklaido ir vėl surenka. 
\par 24 Jis aptemdo protą tautų vadams, ir jie klaidžioja dykumoje be jokio kelio, 
\par 25 jie grabalioja tamsoje ir blaškosi kaip girti”.



\chapter{13}


\par 1 “Mano akys viską matė, ausys girdėjo ir suprato. 
\par 2 Kiek jūs žinote, tiek ir aš žinau, aš nesu menkesnis už jus. 
\par 3 Norėčiau kalbėti su Visagaliu ir ginčytis su Dievu. 
\par 4 Jūs esate melo kalviai, niekam tikę gydytojai. 
\par 5 Jei jūs tylėtumėte, tuo parodytumėte savo išmintį. 
\par 6 Pasiklausykite mano svarstymų, įsidėmėkite mano kalbą. 
\par 7 Ar kalbėsite nedorai už Dievą ir sakysite melą už Jį? 
\par 8 Ar, būdami šališki, norite Dievą ginti? 
\par 9 Ar bus gerai, kai Jis jus ištirs? Ar pasijuoksite iš Dievo kaip iš žmogaus? 
\par 10 Jis jus tikrai sudraus, jei būsite užslėpę šališkumą. 
\par 11 Ar Jo didybė jūsų negąsdins? Ar Jo baimė neapims jūsų? 
\par 12 Jūsų kalbos tarsi pelenai, o jūsų kūnai kaip molis. 
\par 13 Nutilkite ir leiskite man kalbėti, kas man bebūtų. 
\par 14 Kam aš draskau dantimis savo kūną ir nešioju savo sielą savo rankose? 
\par 15 Jei Jis mane ir nužudys, aš Juo pasitikėsiu ir išlaikysiu savo kelius Jo akivaizdoje. 
\par 16 Jis bus mano išgelbėjimas, nes veidmainis nepasirodys Dievo akivaizdoje. 
\par 17 Atidžiai klausykite mano kalbos ir savo ausimis išgirskite mano pasiteisinimą. 
\par 18 Štai aš iškėliau savo bylą, nes žinau, kad būsiu išteisintas. 
\par 19 Kas galėtų mane kaltinti? Jei aš nutilčiau, atiduočiau savo dvasią. 
\par 20 Nedaryk man dviejų dalykų, tada nesislėpsiu nuo Tavęs: 
\par 21 atitrauk nuo manęs savo ranką ir negąsdink manęs. 
\par 22 Tada Tu šauksi, ir aš atsiliepsiu; arba leisk man kalbėti ir atsakyk man. 
\par 23 Kiek nusikaltimų ir nuodėmių padariau? Parodyk man mano kaltes ir nuodėmes. 
\par 24 Kodėl slepi savo veidą ir mane laikai priešu? 
\par 25 Kodėl rodai savo jėgą prieš vėjo blaškomą lapą ir persekioji sausą šiaudą? 
\par 26 Tu rašai prieš mane karčius dalykus ir baudi už jaunystės nuodėmes; 
\par 27 Tu įtveri mano kojas į šiekštą ir seki visus mano žingsnius ir takus. 
\par 28 Esu sunaikintas kaip puvėsis, kaip drabužis, suėstas kandžių”.



\chapter{14}


\par 1 “Žmogus, gimęs iš moters, gyvena trumpai, bet daug vargsta. 
\par 2 Jis kaip gėlė auga ir nuvysta. Jis dingsta kaip šešėlis ir nepasilieka. 
\par 3 Ar Tu atversi savo akis į tokį ir nusivesi mane į teismą su savimi? 
\par 4 Kas gali iš netyro padaryti tyrą? Niekas! 
\par 5 Jo dienos yra tiksliai nustatytos ir mėnesiai suskaičiuoti. Tu nustatai jam ribą, ir jis jos neperžengs. 
\par 6 Atsitrauk nuo jo, kad jis pailsėtų, kol kaip samdinys sulauks savo dienos. 
\par 7 Medžiui yra viltis, kad ir nukirstas atžels ir iš kelmo išaugs atžalos. 
\par 8 Nors žemėje jo šaknys pasensta ir jo kelmas apmiršta dulkėse, 
\par 9 bet, gavęs vandens, jis atželia, krauna pumpurus ir išleidžia šakeles kaip jaunas augalas. 
\par 10 O žmogus miršta, ir nebėra jo; atiduoda žmogus savo dvasią, kur jis yra? 
\par 11 Kaip vanduo išgaruoja iš jūrų, upės nusenka ir išdžiūsta, 
\par 12 taip žmogus atsigula ir nebeatsikelia. Kol dangūs pasibaigs, jis neatsibus; niekas jo nepažadins iš miego. 
\par 13 O kad paslėptum mane kape ir laikytum paslėpęs, kol praeis Tavo rūstybė; nustatytam laikui praėjus, vėl mane atsimintum. 
\par 14 Ar miręs žmogus prisikels? Per visas man skirtas dienas aš lauksiu permainos. 
\par 15 Tu šauksi, ir aš atsiliepsiu; Tu ilgėsiesi savo rankų kūrinio. 
\par 16 Tu skaičiuoji mano žingsnius, bet neįskaityk mano nuodėmės. 
\par 17 Mano nusikaltimas paslėptas maišelyje ir mano kaltė užrišta. 
\par 18 Kaip kalnas krisdamas subyra ir uola pajuda iš savo vietos, 
\par 19 kaip vanduo nuneša akmenis ir liūtys nuplauna dirvožemį, taip Tu sunaikini žmogaus viltį. 
\par 20 Tu nugali jį nuolat, ir jo nebelieka, Tu pakeiti jo veidą ir pavarai jį. 
\par 21 Jei jo sūnūs gerbiami, jis nežino; jei jie niekinami, jis nepastebi. 
\par 22 Jis jaučia savo kūno skausmus, ir jo siela kenčia”.



\chapter{15}


\par 1 Temanas Elifazas atsakydamas tarė: 
\par 2 “Ar išmintingas žmogus kalba tuščius žodžius ir pripildo savo vidurius rytų vėjo? 
\par 3 Ar jis kalba netinkamus žodžius ir sako tai, kas neatneša nieko gero? 
\par 4 Tu atmetei baimę ir nesivaržai, kalbėdamas prieš Dievą. 
\par 5 Tavo lūpos kalba apie tavo kaltę ir tu pasirinkai klastingą liežuvį. 
\par 6 Tavo paties burna pasmerkia tave, o ne aš, tavo paties lūpos liudija prieš tave. 
\par 7 Bene tu esi pirmas gimęs žmogus, sutvertas pirma kalnų? 
\par 8 Ar tu girdėjai Dievo paslaptis ir turi visą išmintį? 
\par 9 Ką žinai, ko mes nežinome? Ką supranti, ko mes nesuprantame? 
\par 10 Tarp mūsų yra ilgaamžių ir žilagalvių, senesnių už tavo tėvą. 
\par 11 Ar tau neužtenka Dievo paguodos? Ar turi kokią paslaptį? 
\par 12 Kodėl tavo širdis tave nunešė į šalį ir prieš ką merki savo akį? 
\par 13 Kodėl nukreipi savo dvasią prieš Dievą ir leidi tokiems žodžiams išeiti iš tavo lūpų? 
\par 14 Kas yra žmogus, kad būtų tyras; tas, kuris gimęs iš moters, kad būtų teisus? 
\par 15 Jis nepasitiki savo šventaisiais ir dangūs nėra tyri Jo akivaizdoje, 
\par 16 tuo labiau bjaurus ir purvinas žmogus, kuris geria nedorybę kaip vandenį. 
\par 17 Klausyk manęs, aš tave pamokysiu, ką patyriau, pasakysiu, 
\par 18 ką išminčiai skelbė, sužinoję iš savo tėvų, ir nenuslėpė. 
\par 19 Jiems vieniems buvo atiduota žemė ir joks svetimšalis nevaikščiojo tarp jų. 
\par 20 Nedorėlis kenčia per visas savo dienas, metų skaičius paslėptas nuo prispaudėjo. 
\par 21 Baisūs garsai jo ausyse, taikos metu jį užklumpa naikintojas. 
\par 22 Jis nesitiki ištrūkti iš tamsos, jis paskirtas kardui. 
\par 23 Jis klaidžioja, ieškodamas duonos. Bet kur ji? Jis žino, kad jo laukia tamsi diena. 
\par 24 Jį gąsdina vargas ir pavojus, jie nugalės jį kaip karalius, pasiruošęs kovai. 
\par 25 Nes jis grasina Dievui, kėsinasi į Visagalį, 
\par 26 su skydu rankoje puola Jį. 
\par 27 Jo veidas padengtas riebalais, jis pats aptekęs taukais. 
\par 28 Jis gyvena sugriautuose miestuose, namuose, kuriuose niekas negyvena, kurie skirti nugriauti. 
\par 29 Jis nepraturtės, jo nuosavybė neišliks; jo turtai nepasklis po žemę. 
\par 30 Jis neištrūks iš tamsos, jo šakas sudegins liepsna, nuo Jo burnos kvapo jis pranyks. 
\par 31 Tenepasitiki apsigaudamas tuštybe, nes tuštybe bus jam atlyginta. 
\par 32 Jis pražus ne laiku ir jo šakos nežaliuos. 
\par 33 Jo neprinokusios vynuogės nukris, nubyrės kaip alyvmedžio žiedai. 
\par 34 Veidmainių susirinkimas bus nevaisingas, ir ugnis sudegins kyšių ėmėjų palapines. 
\par 35 Jie pastoja piktu sumanymu ir pagimdo blogį; jų pilvas paruošia apgaulę”.



\chapter{16}


\par 1 Jobas atsakydamas tarė: 
\par 2 “Tokių kalbų aš jau daug girdėjau. Netikę guodėjai jūs visi. 
\par 3 Kada pasibaigs tuščios kalbos? Kas verčia tave man atsakyti? 
\par 4 Ir aš galėčiau taip kalbėti, jei jūs būtumėte mano vietoje. Aš galėčiau užversti jus žodžiais ir kraipyti galvą prieš jus. 
\par 5 Tačiau aš stiprinčiau jus savo burna ir savo lūpų paguoda lengvinčiau jūsų kančią. 
\par 6 Jei kalbu, mano skausmas nesumažėja; jei tyliu, man nelengviau. 
\par 7 Bet dabar Jis vargina mane; Tu sunaikinai visą mano šeimą. 
\par 8 Tu pripildei mane raukšlių, kurios liudija prieš mane, mano liesumas pakyla manyje liudyti man į veidą. 
\par 9 Jis savo rūstybe parbloškė mane ir griežia dantimis prieš mane. Mano priešo akys įsmeigtos į mane. 
\par 10 Jie atvėrė prieš mane savo burnas, plūsdami smogia man į veidą, jie susirinko prieš mane. 
\par 11 Dievas atidavė mane bedieviams, perdavė į nedorėlių rankas. 
\par 12 Aš gyvenau ramiai, bet Jis supurtė mane; nutvėręs už sprando, sutraiškė mane ir pastatė sau taikiniu. 
\par 13 Jo šauliai apsupo mane ir be pasigailėjimo perveria mano inkstus, išlieja mano tulžį. 
\par 14 Jis daro man žaizdą po žaizdos, puola mane kaip milžinas. 
\par 15 Aš savo kūną apdengiau ašutine; savo ragą paslėpiau dulkėse. 
\par 16 Mano veidas ištino nuo ašarų, mano akys­mirties šešėlis, 
\par 17 nors mano rankose nėra neteisybės; mano malda yra tyra. 
\par 18 Žeme, neuždenk mano kraujo, ir mano šauksmas tegul nenutyla. 
\par 19 Štai dabar mano liudytojas yra danguje, Tas, kuris pažįsta mane, yra aukštybėse. 
\par 20 Mano draugai tyčiojasi iš manęs, tačiau Dievas mato mano ašaras. 
\par 21 O kad kas galėtų apginti žmogų prieš Dievą, kaip žmogus apgina savo artimą. 
\par 22 Po kelerių metų aš nueisiu tuo keliu, kuriuo nebegrįžtama”.



\chapter{17}


\par 1 “Mano kvėpavimas nusilpo, dienos trumpėja, kapai paruošti man. 
\par 2 Mane apspito išjuokėjai, mano akys pavargo bežiūrėdamos į juos. 
\par 3 Tu pats laiduok už mane, nes kas kitas paduos man ranką? 
\par 4 Tu paslėpei supratimą nuo jų širdžių, todėl jų neišaukštinsi. 
\par 5 Kas pataikauja savo draugams, to vaikai nesidžiaugs laimikiu. 
\par 6 Jis padarė mane priežodžiu žmonėms, visi spjaudo man į veidą. 
\par 7 Mano akys aptemo nuo sielvarto, mano kūnas kaip šešėlis. 
\par 8 Teisieji pasibaisės tuo, o nekaltieji pakils prieš veidmainius. 
\par 9 Teisusis laikysis savo kelio, o tas, kurio rankos švarios, stiprės ir stiprės. 
\par 10 Ateikite jūs visi dar kartą, nes tarp jūsų nerandu nė vieno išmintingo. 
\par 11 Mano dienos praėjo; sumanymai ir mano širdies siekiai sudužo. 
\par 12 Jie naktį padarė diena, tačiau trūksta šviesos tamsoje. 
\par 13 Ko gi aš dar laukiu? Mano namai yra kapas; aš savo guolį pasiklojau tamsoje. 
\par 14 Sugedimą aš vadinu tėvu, o kirmėles­motina ir seserimi. 
\par 15 Kur yra mano viltis? Kas pamatys, kuo viliuosi? 
\par 16 Ji nueis su manimi į gelmes ir ilsėsis su manimi dulkėse”.



\chapter{18}


\par 1 Šuachas Bildadas atsakydamas tarė: 
\par 2 “Ar ilgai jūs dar taip kalbėsite? Nutilkite ir leiskite mums kalbėti. 
\par 3 Kodėl mes laikomi gyvuliais ir neišmanėliais? 
\par 4 Tu plėšai save pykčiu. Ar dėl tavęs žemė taps tuščia ir uolos pajudės iš savo vietos? 
\par 5 Nedorėlio šviesa užges ir jo liepsnos kibirkštis nebešvies. 
\par 6 Jo palapinėje bus tamsu ir jo žiburys užges su juo. 
\par 7 Jo buvę tvirti žingsniai sutrumpės, jo paties sumanymas jį parblokš. 
\par 8 Jo kojos įkliuvusios į tinklą, jis eina uždengta duobe. 
\par 9 Jo kulnis pateks į spąstus, ir jis paklius plėšikui. 
\par 10 Kilpa padėta jam ant žemės, spąstai ant tako. 
\par 11 Pavojai gąsdins jį iš visų pusių ir nuvarys jį nuo kojų. 
\par 12 Jo jėgos išseks nuo bado, žlugimas pasiruošęs prie jo šono. 
\par 13 Jo odą suės ligos, mirties pirmagimis prarys jo jėgą. 
\par 14 Jo pasitikėjimas bus išrautas iš jo palapinės ir nuves jį pas siaubų karalių. 
\par 15 Tai gyvens jo palapinėje, nes ji nebepriklauso jam, jo buveinė bus apibarstyta siera. 
\par 16 Jo šaknys nudžius žemėje, o šakos bus nukirstos viršuje. 
\par 17 Jo atsiminimas dings krašte ir jo vardas nebus minimas gatvėje. 
\par 18 Jis bus išvytas iš šviesos į tamsą ir išvarytas iš pasaulio. 
\par 19 Palikuonių jis nepaliks tautoje ir niekas nepasiliks jo buveinėje. 
\par 20 Po jo atėję stebėsis jo diena ir prieš jį buvę bus išgąsdinti. 
\par 21 Tikrai tokia yra nedorėlio buveinė ir vieta to, kuris nepažįsta Dievo”.



\chapter{19}


\par 1 Jobas atsakydamas tarė: 
\par 2 “Kaip ilgai varginsite mano sielą ir kankinsite mane tuščiais žodžiais? 
\par 3 Jau dešimt kartų mane plūdote ir nesigėdite mane kankinti. 
\par 4 O jei iš tikrųjų nusikaltau, tai mano klaida yra su manimi. 
\par 5 Ar iš tikrųjų norite didžiuotis prieš mane ir įrodyti mano nusikaltimą? 
\par 6 Žinokite, kad Dievas mane pargriovė ir savo tinklu mane pagavo. 
\par 7 Aš šaukiu apie priespaudą, bet niekas neatsiliepia, šaukiu garsiai, bet nėra teisybės. 
\par 8 Jis užtvėrė man kelią, kad negaliu praeiti; mano takus Jis apsupo tamsa. 
\par 9 Jis nuplėšė mano garbę ir nuėmė karūną man nuo galvos. 
\par 10 Iš visų pusių Jo naikinamas krintu, mano viltis lyg nukirstas medis. 
\par 11 Jo rūstybė užsidegė prieš mane, Jis laiko mane savo priešu. 
\par 12 Jo būriai traukia kartu prieš mane ir apgula mano palapinę. 
\par 13 Jis patraukė mano brolius nuo manęs ir pažįstami šalinasi manęs. 
\par 14 Mano artimieji paliko mane, mano draugai mane užmiršo. 
\par 15 Tie, kurie gyvena mano namuose, net mano tarnaitės, laiko mane svetimu, nepažįstamas tapau jų akyse. 
\par 16 Aš šaukiu savo tarną, bet jis neatsiliepia, aš turiu jį maldauti savo burna. 
\par 17 Mano kvapas bjaurus mano žmonai ir mano kūno vaikai atmetė mane. 
\par 18 Maži vaikai niekina mane. Kai noriu atsikelti, jie šaiposi iš manęs. 
\par 19 Manimi bjaurisi mano artimiausi draugai; tie, kuriuos mylėjau, tapo mano priešais. 
\par 20 Iš manęs liko tik oda ir kaulai, prie dantų liko tik lūpos. 
\par 21 Draugai, pasigailėkite, pasigailėkite manęs, nes Dievo ranka palietė mane. 
\par 22 Kodėl mane persekiojate kaip Dievas ir nepasisotinate mano kūnu? 
\par 23 O kad mano žodžiai būtų įrašyti į knygą, 
\par 24 įrėžti geležiniu rašikliu bei švinu amžiams į uolą. 
\par 25 Nes aš žinau, kad mano Atpirkėjas gyvas ir kad Jis atsistos galiausiai ant žemės. 
\par 26 Ir kai mano oda sunyks, aš savo kūne regėsiu Dievą. 
\par 27 Aš pats Jį matysiu, savo akimis žiūrėsiu į Jį, o ne svetimomis. Mano širdis krūtinėje ilgisi Jo. 
\par 28 Jūs turėtumėte sakyti: ‘Kodėl mes jį persekiojame?’, tarsi priežastis būtų manyje. 
\par 29 Bijokite kardo, nes pyktis baudžiamas kardu, kad žinotumėte, jog yra teismas”.



\chapter{20}


\par 1 Naamatietis Cofaras atsakydamas tarė: 
\par 2 “Mano mintys verčia mane atsiliepti, ir dėl to aš skubu kalbėti. 
\par 3 Aš girdėjau priekaištus man ir mano supratimo dvasia verčia mane atsakyti. 
\par 4 Ar nežinai, kad nuo seno, kai žmogus buvo įkurdintas žemėje, 
\par 5 nedorėlių džiaugsmas trumpas ir veidmainių džiaugsmas tik akimirka? 
\par 6 Nors jo puikybė pasiektų dangus ir jo galva liestų debesis, 
\par 7 jis pražus kaip jo paties išmatos. Kas jį matė, klaus: ‘Kur jis?’ 
\par 8 Kaip sapnas jis dings, pranyks kaip nakties regėjimas. 
\par 9 Akis, kuri jį matė, nebematys jo daugiau ir jo vieta jo neberegės. 
\par 10 Jo vaikai ieškos beturčių palankumo, jo rankos sugrąžins jo turtus. 
\par 11 Jo kaulai pilni jaunystės nuodėmių, kurios gulės dulkėse kartu su juo. 
\par 12 Nors nedorybė saldi jo burnoje, nors jis slepia ją po savo liežuviu, 
\par 13 saugo ją ir nepaleidžia, paslėpęs savo burnoje, 
\par 14 tačiau maistas jo viduriuose virs gyvačių tulžimi. 
\par 15 Prarytus turtus jis išvems; Dievas iš jo pilvo ištrauks juos. 
\par 16 Gyvačių nuodus jis čiulps, jį nužudys angies liežuvis. 
\par 17 Jis nematys upių ir upelių, tekančių medumi ir pienu. 
\par 18 Ką jis uždirbo, turės atiduoti ir neprarys to. Jis atlygins savo turtais ir nepasidžiaugs jais. 
\par 19 Nes jis nuskriaudė ir apleido vargšą, pasisavino namus, kurių nestatė. 
\par 20 Jo godumui nebuvo ribų, bet jis nieko neišgelbės. 
\par 21 Neliks jo valgio ir niekas nežiūrės į jo gėrybes. 
\par 22 Kai jis bus apsirūpinęs, jam bus ankšta, nelaimės rankos apims jį. 
\par 23 Kai jis norės prikimšti savo pilvą, Dievas pasiųs savo rūstybę ant jo, išlies ją, jam bevalgant. 
\par 24 Jis bėgs nuo geležinio ginklo, bet jį pervers varinis lankas. 
\par 25 Strėlė bus ištraukta iš jo kūno, žibantis antgalis bus išliejęs jo tulžį; siaubai apniks jį. 
\par 26 Tamsa paslėpta jo viduje; prarys jį ugnis, kurios niekas neužkūrė. Bus varginami tie, kas liks jo palapinėje. 
\par 27 Dangus atskleis jo kaltę ir žemė sukils prieš jį. 
\par 28 Visas jo turtas bus sunaikintas ir nuneštas Jo rūstybės dieną. 
\par 29 Tokia dalis yra nuo Dievo nedorėliui ir toks palikimas yra Dievo jam skirtas”.



\chapter{21}


\par 1 Jobas atsakydamas tarė: 
\par 2 “Klausykite atidžiai mano žodžių, ir tokia bus jūsų paguoda. 
\par 3 Kantriai išklausykite mano kalbą; kai baigsiu, galite toliau tyčiotis. 
\par 4 Argi mano skundas žmogui? Jei taip būtų, mano dvasia nesijaudintų. 
\par 5 Pažvelkite į mane ir nusigąskite, uždenkite ranka savo burnas. 
\par 6 Tai prisiminęs, pats nusigąstu, drebėjimas apima mano kūną. 
\par 7 Kodėl nedorėliai gyvena iki senatvės ir yra kupini jėgos? 
\par 8 Jų palikuonys įsikuria jų aplinkoje, jų vaikaičiai gyvena su jais. 
\par 9 Jų namai saugūs, jie nieko nebijo, ir Dievo lazda jų neplaka. 
\par 10 Jų galvijai veisiasi, karvės apsiveršiuoja ir neišsimeta. 
\par 11 Jų vaikai šokinėja kaip ėriukai ir žaidžia. 
\par 12 Jie paima būgnelius ir arfas ir džiaugiasi, skambant fleitai. 
\par 13 Jie gyvena pasiturinčiai ir per akimirksnį nueina į kapus, 
\par 14 nors jie sako Dievui: ‘Atsitrauk nuo mūsų. Mes nenorime pažinti Tavo kelių. 
\par 15 Kas yra Visagalis, kad Jam turėtume tarnauti? Kokią naudą turėsime, jei melsimės Jam?’ 
\par 16 Bet jų gerovė nėra jų pačių rankose; todėl nedorėlių patarimas yra toli nuo manęs. 
\par 17 Juk dažnai užgęsta nedorėlių žibintas ir juos prislegia nelaimės! Dievas užsirūstinęs siunčia jiems vargų. 
\par 18 Jie yra kaip šiaudai prieš vėją, kaip audros nunešami pelai. 
\par 19 Jūs sakote, kad Dievas kaupia nedorybes jo vaikams! Tegul Jis atlygina jam pačiam, kad jis tai žinotų. 
\par 20 Jo akys tepamato savo pražūtį ir jis tegeria Visagalio rūstybę. 
\par 21 Kam jam rūpintis savo namais po savęs, kai jo mėnesių skaičius bus nutrauktas? 
\par 22 Argi galima pamokyti išminties Dievą, kuris teisia valdovus? 
\par 23 Vienas miršta kupinas jėgų, laisvas nuo rūpesčių ir ramus, 
\par 24 jo viduriai pilni taukų ir jo kaulai prisigėrę smegenų. 
\par 25 Kitas miršta su apkartusia siela, nieko gero neragavęs. 
\par 26 Abu paguldomi į dulkes, ir kirmėlės juos apdengia. 
\par 27 Aš žinau jūsų mintis ir jūsų nedoras užmačias prieš mane. 
\par 28 Jūs sakote: ‘Kur kunigaikščių namai ir kur nedorėlių buveinės?’ 
\par 29 Pasiklausykite keliautojų ir pasimokykite iš jų pasakojimų, 
\par 30 kad nedorėlis palaikomas žlugimo dienai ir bus atvestas į rūstybės dieną. 
\par 31 Kas pasakys jam į akis apie jo kelius? Kas atlygins jam už tai, ką jis padarė? 
\par 32 Jis bus nuvežtas į kapines ir pasiliks kape. 
\par 33 Slėnio grumstai bus jam mieli; kiekvienas žmogus nueis paskui jį ir prieš jį buvo nesuskaitoma daugybė. 
\par 34 Kaip tad jūs mane tuščiai guodžiate, nes jūsų atsakymai yra tik apgaulė”.



\chapter{22}


\par 1 Temanas Elifazas atsakydamas tarė: 
\par 2 “Ar gali žmogus būti naudingas Dievui? Išminčiaus nauda yra jam pačiam. 
\par 3 Ar Visagaliui malonu, jei esi teisus? Ar Jis turi naudos, jei tavo kelias nepeiktinas? 
\par 4 Ar dėl to, kad Jo bijai, Jis bara tave ir patraukia tave į teismą? 
\par 5 Ar ne dėl tavo didelės nedorybės ir nesuskaičiuojamų kalčių? 
\par 6 Ėmei užstatą iš savo brolių už nieką, nuplėšei nuogiesiems drabužius, 
\par 7 nepagirdei ištroškusio, alkano nepamaitinai. 
\par 8 Savo jėga įsigijai nuosavybę, būdamas galingas, pasilaikei ją. 
\par 9 Našles išvarei be nieko ir našlaičius palikai tuščiomis rankomis. 
\par 10 Todėl tu patekai į spąstus ir netikėta baimė vargina tave, 
\par 11 tamsoje tu negali matyti; vandens gausybė apdengė tave. 
\par 12 Argi Dievo nėra dangaus aukštybėje? Žiūrėk, kaip aukštai yra žvaigždės. 
\par 13 Tu sakai: ‘Ką Dievas žino? Ar Jis gali teisti pro tamsius debesis? 
\par 14 Tamsūs debesys dengia Jį, Jis nieko nemato vaikščiodamas dangaus skliautu’. 
\par 15 Ar vis dar laikaisi seno kelio, kuriuo piktadariai vaikščiojo? 
\par 16 Todėl jie nelaiku žuvo, jų pamatą nunešė tvanas. 
\par 17 Jie sakė Dievui: ‘Atsitrauk nuo mūsų’. Ką Visagalis padarys jiems? 
\par 18 Jis pripildė jų namus gėrybių, tačiau nedorėlių patarimas yra toli nuo manęs. 
\par 19 Teisusis mato tai ir džiaugiasi, nekaltieji tyčiojasi iš jų: 
\par 20 ‘Mūsų turtai nepražuvo, o tai, ką jie turėjo, suėdė ugnis’. 
\par 21 Susitaikyk su Juo ir nusiramink; tuomet ateis tau gerovė. 
\par 22 Priimk Jo pamokymus ir įsidėk Jo žodžius į širdį. 
\par 23 Jei grįši prie Visagalio, būsi sutvirtintas ir pašalinsi nedorybę iš savo palapinių, 
\par 24 tada krausi auksą kaip dulkes, Ofyro auksą kaip upelių akmenis. 
\par 25 Visagalis bus tau apsauga ir tu turėsi gausybę sidabro. 
\par 26 Tada gėrėsies Visagaliu ir pakelsi savo akis į Dievą. 
\par 27 Kai tu melsi Jį, Jis tave išklausys, ir tu ištesėsi savo įžadus. 
\par 28 Tu nuspręsi, ir bus tau, šviesa švies tau kelyje. 
\par 29 Kai žmogus pažemintas, tu sakysi: ‘Yra išaukštinimas’, ir Jis išgelbės nuolankų žmogų. 
\par 30 Jis išgelbės ir kaltą; jis bus išgelbėtas dėl tavo rankų švarumo”.



\chapter{23}


\par 1 Jobas atsakydamas tarė: 
\par 2 “Mano skundas dar ir šiandien kartus; mano kentėjimai didesni už mano vaitojimą. 
\par 3 O kad žinočiau, kur Jį rasti, ir galėčiau ateiti prie Jo sosto! 
\par 4 Jam pateikčiau savo bylą ir savo burną pripildyčiau įrodymų. 
\par 5 Tada išgirsčiau, ką Jis man atsakytų, ir suprasčiau, ką man kalbėtų. 
\par 6 Ar Jis priešintųsi man savo galinga jėga? Ne! Jis pažvelgtų į mane. 
\par 7 Teisusis galėtų aiškintis su Juo, taip aš būčiau išlaisvintas amžiams nuo savo teisėjo. 
\par 8 Jei einu pirmyn, ten Jo nėra, o jei atgal, Jo nerandu. 
\par 9 Jei Jis yra kairėje, aš Jo nematau, o jei pasislėpęs dešinėje, Jo nepastebiu. 
\par 10 Bet Jis žino mano kelią; jei Jis mane ištirtų, būčiau kaip auksas. 
\par 11 Ėjau Jo pėdomis, iš Jo kelio neiškrypau. 
\par 12 Nuo Jo įsakymų nepasitraukiau, Jo burnos žodžius vertinau labiau negu būtiną maistą. 
\par 13 Jis vienintelis, kas gali Jį pakeisti? Ko Jo siela geidžia, tą Jis padaro. 
\par 14 Jis įvykdys, kas man skirta; daug panašių dalykų Jis turi. 
\par 15 Todėl man baugu Jo akivaizdoje; apie tai galvodamas, bijau Jo. 
\par 16 Dievas susilpnina mano širdį; Visagalis gąsdina mane. 
\par 17 Aš nepražuvau prieš tamsą, Jis nepaslėpė tamsybės nuo mano veido”.



\chapter{24}


\par 1 “Visagalis žino laikus, tai kodėl tie, kurie Jį pažįsta, nesulaukia Jo dienos? 
\par 2 Vieni perkelia ribas, smurtu pagrobia bandas ir jas gano. 
\par 3 Našlaičių asilą jie nusivaro, našlės jautį paima kaip užstatą. 
\par 4 Vargšus jie nustumia nuo kelio, krašto beturčiai slepiasi nuo jų. 
\par 5 Jie eina į savo darbą kaip laukiniai asilai dykumoje, anksti keliasi ieškoti grobio, nes tyruose jie randa maisto sau ir savo vaikams. 
\par 6 Jie renka varpas ne savo lauke ir nedorėlio vynuogyne pasiskina vynuogių. 
\par 7 Jie miega neapsikloję ir neturi kuo apsidengti šaltyje. 
\par 8 Jie šlapi nuo kalnų lietaus ir glaudžiasi prie uolos, ieškodami slėptuvės. 
\par 9 Anie atplėšia našlaitį nuo krūtinės ir ima užstatą iš vargšo, 
\par 10 kuris vaikščioja nuogas, neturėdamas drabužių; atima varpas iš alkstančių, 
\par 11 kurie gamina aliejų šeimininkui, mina vyno spaustuvus, tačiau kenčia troškulį. 
\par 12 Žmonės dejuoja mieste ir sužeistųjų sielos šaukiasi pagalbos, tačiau Dievas neapkaltina jų. 
\par 13 Jie sukyla prieš šviesą, nepažįsta jos kelių, neina jos takais. 
\par 14 Anksti rytą keliasi žudikas ir žudo beturtį ir vargšą; naktį jis slankioja kaip vagis. 
\par 15 Svetimautojas laukia sutemų, tikėdamas, kad niekas jo nematys, jis užsidengia savo veidą. 
\par 16 Tamsoje jie plėšia namus, kuriuos numatė dieną; šviesos jie nežino. 
\par 17 Nes rytas jiems kaip mirties šešėlis, jei kas pažins juos, tai juos gąsdina kaip mirtis. 
\par 18 Jis greitas kaip vanduo; jo dalis prakeikta žemėje. Jis neina keliu į vynuogyną. 
\par 19 Kaip sausra ir kaitra praryja sniego vandenis, taip pragaras praryja nusidėjėlį. 
\par 20 Įsčios pamirš jį, kirmėlės skaniai maitinsis juo, jo niekas nebeprisimins. Piktadarys kaip nukirstas medis, 
\par 21 nes išnaudojo nevaisingąją ir našlei nepadėjo. 
\par 22 Dievas pašalina galiūnus savo jėga; Jis pakyla, ir joks žmogus nėra tikras dėl savo gyvybės. 
\par 23 Jis teikia jiems saugumą ir poilsį, bet Jo akys stebi jų kelius. 
\par 24 Jie išaukštinami trumpam laikui, bet pranyksta ir tampa nieku. Jie pašalinami iš kelio ir sudžiūsta kaip javų varpos. 
\par 25 Argi taip nėra? Kas įrodys, kad aš netiesą kalbu ir esu melagis?”



\chapter{25}


\par 1 Šuachas Bildadas atsakydamas tarė: 
\par 2 “Valdžia ir pagarba priklauso Jam, nes Jis palaiko tvarką aukštybėse. 
\par 3 Ar Jo pulkai suskaičiuojami? Kam Jo šviesa neužteka? 
\par 4 Kaip tad žmogus gali būti nuteisintas prieš Dievą? Kaip gali būti švarus, gimęs iš moters? 
\par 5 Jo akyse net mėnulis ir žvaigždės nėra skaistūs, 
\par 6 tuo labiau žmogus, kuris yra tik kirmėlė, žmogaus sūnus, kuris yra tik kirminas”.



\chapter{26}


\par 1 Jobas atsakydamas tarė: 
\par 2 “Kaip tu padėjai bejėgiui ir parėmei nusilpusio ranką! 
\par 3 Koks geras ir išmintingas buvo tavo patarimas! 
\par 4 Kam tu kalbėjai šiuos žodžius? Kokia dvasia atėjo iš tavęs? 
\par 5 Prieš Jį dreba mirusieji, vandenys ir jų gyventojai. 
\par 6 Mirusiųjų pasaulis yra atviras Jam ir pražūtis neuždengta. 
\par 7 Jis ištiesia šiaurę ant tuštumos ir žemę pakabina ant nieko. 
\par 8 Jis surenka vandenis į tamsius debesis, tačiau debesys neplyšta. 
\par 9 Jis uždengia savo sosto veidą ir ištiesia savo debesį ant jo. 
\par 10 Vandens paviršiuje Jis nubrėžė ribą ir atskyrė šviesą nuo tamsos. 
\par 11 Dangaus kolonos svyruoja ir dreba, kai Jis grūmoja. 
\par 12 Savo galia Jis sujaudina jūrą, savo išmintimi nutildo jos išdidumą. 
\par 13 Savo dvasia Jis papuošė dangus, Jo ranka padarė gyvatę. 
\par 14 Čia tik Jo kelių pašaliai; mes girdime tik Jo šnibždesį. O Jo galybės griaustinį kas supras?”



\chapter{27}


\par 1 Jobas tęsė savo palyginimą: 
\par 2 “Gyvas Dievas, kuris nedaro man teisybės, ir Visagalis, kuris apkartino mano sielą. 
\par 3 Kol aš kvėpuoju ir Dievo kvapas yra mano šnervėse, 
\par 4 mano lūpos nekalbės netiesos ir mano liežuvis neapgaudinės. 
\par 5 Taip nebus, kad aš pateisinčiau jus. Savo nekaltumo neatsisakysiu iki mirties. 
\par 6 Teisumo tvirtai laikausi ir nepaleisiu; mano širdis man nepriekaištaus, kol gyvensiu. 
\par 7 Te mano priešas bus kaip nedorėlis, tas, kuris puola mane, kaip neteisusis. 
\par 8 Ar yra veidmainiui kokia viltis, nors jis ir daug turi, kai Dievas atima jo sielą? 
\par 9 Ar Dievas išklausys jo šauksmą, kai nelaimės užgrius jį? 
\par 10 Ar jis linksminsis Visagalyje ir nuolat šauksis Dievo? 
\par 11 Aš jus pamokysiu apie Dievo ranką, Visagalio kelių aš neslėpsiu. 
\par 12 Jūs patys tai regėjote; kodėl jūs tad taip tuščiai kalbate? 
\par 13 Štai nedorėlio dalis nuo Dievo ir prispaudėjų paveldėjimas iš Visagalio: 
\par 14 jei jo vaikų padaugėja, jie skirti kardui; jo palikuonys nepasisotina duona. 
\par 15 Kurie išliks po jo, tuos mirtis nuves į kapą ir našlės jų neapraudos. 
\par 16 Nors jis turi sidabro kaip smėlio ir drabužių kaip molio, 
\par 17 bet jo rūbus teisusis vilkės, o sidabrą padalins nekaltasis. 
\par 18 Jis stato namus kaip kandis, kaip sargas būdelę pasidaro. 
\par 19 Jis atsigula turtingas, o pabudęs ir atvėręs akis nieko nebeturi. 
\par 20 Išgąstis užklumpa jį kaip vanduo ir audra naktį nuneša jį. 
\par 21 Rytų vėjas jį pakelia ir viesulas išplėšia jį iš jo vietos, 
\par 22 svaido be pasigailėjimo, nors jis labai stengiasi pabėgti nuo jo. 
\par 23 Žmonės plos savo rankomis dėl jo ir švilpdami išlydės jį”.



\chapter{28}


\par 1 “Yra sidabro gyslų ir vietų, kur gryninamas auksas. 
\par 2 Geležis kasama iš žemės. Iš akmens išlydomas varis. 
\par 3 Žmogus nustato ribą tamsai ir ieško visose įdubose rūdos­tamsos ir mirties šešėlio akmens. 
\par 4 Jis įrengia kasyklas vietose, kur nėra žengusi koja, leidžiasi į gelmes toli nuo žmonių. 
\par 5 Žemės paviršiuje užauga duona, o gelmėje žemė išrausiama kaip po gaisro. 
\par 6 Safyrą randa jos uolienose, aukso dulkės yra joje. 
\par 7 Kelio tenai nežino plėšrus paukštis, nematė jo nė vanago akis. 
\par 8 Liūto jaunikliai jų nemindo, ir liūtas tais takais nevaikštinėja. 
\par 9 Jis ištiesia ranką į kietas uolas ir kalnų pamatus kasinėja. 
\par 10 Jis iškerta upes uolose, jo akis pamato kiekvieną brangų daiktą. 
\par 11 Jis užtvenkia upes ir tai, kas jose paslėpta, iškelia į šviesą. 
\par 12 Bet kur randama išmintis, kur yra supratimo šaltinis? 
\par 13 Joks žmogus nežino jos kainos, jos nėra gyvųjų krašte. 
\par 14 Gelmė sako: ‘Ji ne pas mane’, jūra sako: ‘Jos nėra manyje’. 
\par 15 Jos negali pirkti už auksą nė įsigyti už sidabrą. 
\par 16 Negalima jos palyginti su Ofyro auksu nė su oniksu ar safyru. 
\par 17 Neprilygsta jai auksas ir krištolas ir negalima jos gauti už brangius aukso dirbinius. 
\par 18 Koralai ir perlai neverti minėti, nes išmintis brangesnė už rubinus. 
\par 19 Jai neprilygsta Etiopijos topazas, už gryną auksą jos nenupirksi. 
\par 20 Iš kur tad ateina išmintis ir kur supratimo šaltinis? 
\par 21 Ji yra paslėpta gyvųjų akims, net padangių paukščiai jos nesuranda. 
\par 22 Mirtis ir prapultis sako: ‘Mes girdėjome apie jos garsą’. 
\par 23 Dievas žino jos kelią ir vietą, kur ji yra. 
\par 24 Jis stebi visus žemės kraštus ir mato po visu dangumi. 
\par 25 Jis pasveria vėją ir išmatuoja vandenis. 
\par 26 Jis nustatė lietui laiką ir nurodė žaibui kelią. 
\par 27 Tada Jis matė ją ir paskelbė ją, paruošė ją ir ją išmėgino. 
\par 28 Žmogui Jis pasakė: ‘Viešpaties baimė yra išmintis ir šalintis nuo pikta yra supratimas’ ”.



\chapter{29}


\par 1 Jobas tęsė savo palyginimą: 
\par 2 “O kad aš būčiau kaip anksčiau, kaip tomis dienomis, kai Dievas mane saugojo. 
\par 3 Kai Jo žiburys švietė virš mano galvos ir prie Jo šviesos vaikščiojau tamsumoje, 
\par 4 kai mano jaunystės dienomis Dievo paslaptis buvo virš mano palapinės. 
\par 5 Kai Visagalis dar buvo su manimi ir mano vaikai buvo šalia manęs, 
\par 6 kai ploviau kojas piene ir uolos liejo man aliejaus upes. 
\par 7 Kai išeidavau prie miesto vartų, kai aikštėje paruošdavau sau vietą, 
\par 8 jaunuoliai, mane pamatę, slėpdavosi, o seniai atsikėlę stovėdavo, 
\par 9 kunigaikščiai liaudavosi kalbėję ir užsidengdavo ranka savo burnas. 
\par 10 Net kilmingieji nutildavo, ir jų liežuvis prilipdavo prie gomurio. 
\par 11 Kas mane matė ir girdėjo, kalbėjo gera apie mane ir man pritarė, 
\par 12 nes aš išgelbėjau vargšą, prašantį pagalbos, ir našlaitį, kuris neturėjo kas jam padėtų. 
\par 13 To, kuris būtų pražuvęs, palaiminimas pasiekė mane, ir aš suteikdavau džiaugsmo našlės širdžiai. 
\par 14 Teisumas man buvo rūbas, o teisingumas­apsiaustas ir vainikas galvai. 
\par 15 Aš buvau akys aklam ir kojos raišam. 
\par 16 Aš buvau tėvas beturčiams ir ištirdavau bylą, kurios nežinodavau. 
\par 17 Aš sulaužydavau nedorėlio žandikaulius ir iš jo dantų išplėšdavau grobį. 
\par 18 Tuomet sakiau: ‘Mirsiu savo lizde, o mano dienų bus kaip smėlio. 
\par 19 Mano šaknys įleistos prie vandens, ir rasa vilgo mano šakas. 
\par 20 Mano garbė nesensta ir lankas mano rankoje tvirtėja’. 
\par 21 Žmonės klausė manęs ir laukdavo tylėdami mano patarimo. 
\par 22 Po mano žodžių jie nebekalbėdavo, mano kalba krisdavo ant jų. 
\par 23 Jie laukdavo manęs kaip lietaus, plačiai išsižiodavo kaip per vėlyvąjį lietų. 
\par 24 Jei šypsodavausi jiems, jie netikėdavo, mano veido šviesos jie netemdydavo. 
\par 25 Aš parinkdavau jiems kelius ir sėdėjau garbingiausioje vietoje kaip karalius tarp kariuomenės, kaip verkiančiųjų guodėjas”.



\chapter{30}


\par 1 “Dabar juokiasi iš manęs jaunesni už mane, kurių tėvų nebūčiau laikęs prie savo avių bandos šunų. 
\par 2 Kurių rankų stiprumas neturėjo vertės man, jie nesulaukė senatvės. 
\par 3 Dėl neturto ir bado visai nusilpę, jie bėgdavo į dykumą, tuščią ir apleistą. 
\par 4 Jie raudavo dilgėles iš pakrūmių ir kadagių šaknys buvo jų maistas. 
\par 5 Jie būdavo varomi iš bendruomenės su triukšmu kaip vagys. 
\par 6 Jie gyveno kalnų pašlaitėse, žemės olose ir ant uolų, 
\par 7 rinkdavosi tarp erškėčių ir šūkaudavo krūmuose. 
\par 8 Kvailių ir netikėlių vaikai, kuriuos iš krašto išveja. 
\par 9 O dabar tapau priežodis jų dainose, 
\par 10 jie bjaurisi manimi, traukiasi nuo manęs ir nesidrovi spjauti man į veidą. 
\par 11 Kadangi Jis atleido savo templę ir ištiko mane, jie taip pat nebesivaržo mano akivaizdoje. 
\par 12 Man iš dešinės pakyla gauja, stumia mane nuo kelio ir siekia mane sunaikinti. 
\par 13 Jie išardo mano taką, apsunkina mano nelaimę, jie neturi pagalbininko. 
\par 14 Lyg pro plačią spragą įsiveržę, jie neša man pražūtį. 
\par 15 Mane apėmė baimė; jie persekiojo mano sielą kaip vėjas, ir mano laimė praeina kaip debesis. 
\par 16 Dabar mano siela suvargusi ir mano dienos gausios kančių. 
\par 17 Naktį man kaulus gelia ir skausmai nesiliauja. 
\par 18 Daug jėgų reikia man, kad pasikeisčiau drabužį, jis varžo mane kaip rūbo apykaklė. 
\par 19 Jis įmetė mane į purvą, tapau kaip dulkės ir pelenai. 
\par 20 Aš šaukiuosi Tavęs, bet Tu man neatsakai; stoviu, bet Tu nekreipi dėmesio į mane. 
\par 21 Tu tapai man žiaurus, savo stipria ranka mane prislėgei. 
\par 22 Tu pakeli mane vėju ir blaškai, Tu išplėši mano nuosavybę. 
\par 23 Aš žinau, kad nuvesi mane į mirtį, į namus, skirtus visiems gyviesiems. 
\par 24 Tačiau Jis neištiesia rankos į kapą, nors jie šaukia pražūdami. 
\par 25 Ar aš neverkiau dėl kenčiančio, nesisielojau dėl vargšo? 
\par 26 Aš ieškojau gero­gavau pikta; laukiau šviesos­atėjo tamsa. 
\par 27 Mano viduriai virė ir neturėjo poilsio, pasitiko mane vargo dienos. 
\par 28 Aš vaikštinėju gedėdamas, nematydamas saulės; stoviu susirinkime ir šaukiu. 
\par 29 Aš tapau broliu šakalams ir draugu stručiams. 
\par 30 Mano oda pajuodusi, mano kaulai dega nuo karščio. 
\par 31 Mano arfa virto rauda, o mano fleita­verkiančiojo balsu”.



\chapter{31}


\par 1 “Pasižadėjau niekada geisdamas nežiūrėti į mergaitę. 
\par 2 Kokia yra man Dievo skirta dalia? Ką paveldėsiu iš aukštybių nuo Visagalio? 
\par 3 Ar žlugimas neskirtas nedorėliui ir pasmerkimas piktadariui? 
\par 4 Jis juk mato visus mano kelius ir skaičiuoja mano žingsnius. 
\par 5 Ar aš kada vaikščiojau tuštybėje ir ar mano koja skubėjo į apgaulę? 
\par 6 Tegul pasveria mane teisingomis svarstyklėmis ir Dievas žinos mano nekaltumą. 
\par 7 Jei mano žingsnis nukrypo nuo kelio, jei mano širdis sekė paskui akis, jei mano rankos nešvarios, 
\par 8 tada, ką pasėsiu, tegul naudoja kiti ir mano palikuonys tebūna išrauti su šaknimis. 
\par 9 Jei mano širdis geidė moters ir jei tykojau prie savo artimo durų, 
\par 10 tai mano žmona tegul mala kitam ir kitas tegul pasilenkia virš jos. 
\par 11 Tai begėdiškas nusikaltimas, kuris turi būti baudžiamas teismo; 
\par 12 tai ugnis, naikinanti viską, kuri išrautų visas mano gėrybes. 
\par 13 Jei aš nepaisiau savo tarno ar tarnaitės teisių, kai jie ginčijosi su manimi, 
\par 14 tai ką darysiu, kai Dievas pakils? Ką Jam atsakysiu, kai Jis aplankys mane? 
\par 15 Ar ne Tas, kuris padarė mane įsčiose, padarė ir juos? Ar ne vienas sutvėrė mus įsčiose? 
\par 16 Ar aš neišpildžiau beturčio noro ir našlės prašymo? 
\par 17 Ar aš vienas valgiau ir nedaviau našlaičiui? 
\par 18 Nuo savo jaunystės aš auginau jį ir nuo savo gimimo padėdavau našlėms. 
\par 19 Ar aš elgetai nedaviau drabužio ir beturčiui kuo apsikloti? 
\par 20 Ar jis nelinkėjo man laimės, susišildęs mano avių vilnomis? 
\par 21 Jei pakėliau ranką prieš našlaitį, matydamas vartuose man pritariančius, 
\par 22 tai tegul mano petys išnyra ir ranka tebūna sulaužyta. 
\par 23 Dievo bausmės aš bijojau ir Jo didybės akivaizdoje negalėčiau pakelti. 
\par 24 Ar pasitikėjau auksu ir sakiau grynam auksui: ‘Tu mano viltis’? 
\par 25 Ar džiaugiausi dideliais turtais, mano rankų sukauptais? 
\par 26 Ar man žvelgiant į šviečiančią saulę ir į keliaujantį mėnulį, 
\par 27 mano širdis buvo slapta suvedžiota, ar aš bučiavau savo ranką? 
\par 28 Tai būtų nusikaltimas, už kurį reikėtų bausti teisme, nes būčiau išsigynęs Dievo, kuris yra aukštybėse. 
\par 29 Ar aš džiaugiausi manęs nekenčiančio nelaime ir nesėkme? 
\par 30 Aš neleidau savo lūpoms nusidėti, nelinkėjau prakeikimo jo sielai. 
\par 31 Mano palapinės vyrai sakė: ‘Ar yra tokių, kurie būtų nepasisotinę jo maistu?’ 
\par 32 Gatvėje nenakvojo joks ateivis; keleiviui aš atidarydavau duris. 
\par 33 Aš nedangsčiau savo nuodėmių kaip Adomas ir neslėpiau savo kalčių; 
\par 34 nebijojau minios, artimųjų panieka nebaugino manęs, nesėdėjau savo namuose ir netylėjau. 
\par 35 O kad nors kas išklausytų mane! Štai mano parašas. Visagalis teatsako man, mano priešas teparašo knygą. 
\par 36 Tikrai ją ant pečių užsidėčiau arba kaip karūną ant savo galvos. 
\par 37 Aš skelbčiau Jam apie kiekvieną savo žingsnį, kaip kunigaikštis prie Jo ateičiau. 
\par 38 Jei mano žemė šaukia prieš mane ir jos vagos skundžiasi, 
\par 39 jei valgiau jos derlių neapmokėjęs ir jos darbininkams apsunkinau gyvenimą, 
\par 40 tai kviečių vietoje tegul auga erškėčiai, o miežių vietoje­piktžolės”. Taip Jobas baigė savo kalbą.



\chapter{32}


\par 1 Tie trys vyrai liovėsi atsakinėti Jobui, nes jis laikė save teisiu. 
\par 2 Barachelio sūnus Elihuvas, buzitas iš Ramo giminės, supyko ant Jobo, nes jis teisino save, o ne Dievą. 
\par 3 Jis supyko ir ant jo trijų draugų, nes jie nesurado atsakymo, tačiau kaltino Jobą. 
\par 4 Jobui kalbant, Elihus laukė, nes jie buvo vyresni už jį. 
\par 5 Kai Elihus pamatė, kad tie trys vyrai nesurado atsakymo, užsidegė jo pyktis. 
\par 6 Barakelio sūnus Elihus, buzitas, atsakydamas tarė: “Aš dar jaunas, jūs senesni amžiumi, todėl bijojau ir nedrįsau jums pareikšti savo nuomonės. 
\par 7 Aš galvojau: ‘Amžius tegul kalba, metų skaičius tepamoko išminties’. 
\par 8 Tačiau dvasia yra žmoguje ir Visagalio įkvėpimas duoda jam supratimą. 
\par 9 Seniai ne visados išmintingi ir ne amžius leidžia suvokti, kas teisinga. 
\par 10 Todėl pasiklausykite manęs. Aš irgi pareikšiu savo nuomonę. 
\par 11 Aš laukiau jūsų žodžių, klausiau jūsų svarstymų, kai ieškojote, ką atsakyti. 
\par 12 Aš atidžiai jus stebėjau, tačiau nė vienas iš jūsų neįtikino Jobo ir neatsakė į jo žodžius. 
\par 13 Nesakykite, kad atsakėte išmintingai: ‘Dievas jį įveiks, ne žmogus’. 
\par 14 Jis nesikreipė savo žodžiais į mane, ir aš jam neatsakysiu jūsų žodžiais. 
\par 15 Jie nustebę stovi, netekę žado, nebežino, ką sakyti. 
\par 16 Kai aš laukiau, o jie stovėjo tylėdami ir nieko nesakė, 
\par 17 aš nusprendžiau atsakyti ir pareikšti savo nuomonę. 
\par 18 Aš turiu žodžių pakankamai, o dvasia mane ragina. 
\par 19 Mano pilvas kaip vynas, nerandąs išėjimo, plėšantis naujas odines. 
\par 20 Aš turiu kalbėti, kad man būtų lengviau; praversiu savo lūpas ir atsakysiu. 
\par 21 Nebūsiu šališkas ir niekam nepataikausiu. 
\par 22 Jei pataikaučiau, mano Kūrėjas greitai pašalintų mane”.



\chapter{33}

\par 1 “Jobai, klausyk mano žodžių ir juos įsidėmėk. 
\par 2 Atvėriau burną, ir mano liežuvis prabilo. 
\par 3 Mano žodžiai eis iš neklastingos širdies; mano lūpos kalbės tyrą pažinimą. 
\par 4 Dievo Dvasia mane sukūrė, Visagalis įkvėpė man gyvybę. 
\par 5 Atsakyk man, jei gali, parink tinkamus žodžius ir ginkis. 
\par 6 Štai pagal tavo norą aš esu vietoje Dievo; padarytas iš molio, kaip ir tu. 
\par 7 Aš tavęs negaliu išgąsdinti ir mano ranka neprislėgs tavęs. 
\par 8 Ką tu sakei, aš girdėjau, klausiausi tavo žodžių: 
\par 9 ‘Aš esu tyras, be nuodėmės, esu nenusikaltęs ir nėra manyje neteisybės. 
\par 10 Jis kaltina mane, laiko mane savo priešu. 
\par 11 Jis įtvėrė mano kojas į šiekštą, seka visus mano žingsnius’. 
\par 12 Štai čia tu klysti, nes Dievas yra didesnis už žmogų. 
\par 13 Kodėl tu ginčijiesi su Juo? Jis neatsiskaito už jokius savo darbus. 
\par 14 Dievas kalba vienu ar kitu būdu, bet žmogus to nesupranta. 
\par 15 Sapne, nakties regėjime, kai žmonės giliai įmigę ar snaudžia ant lovos, 
\par 16 Jis atidaro žmonių ausis savo įspėjimams, 
\par 17 norėdamas atitraukti žmogų nuo jo poelgių ir puikybės. 
\par 18 Jis saugo jo sielą nuo pražūties ir gyvybę nuo mirties. 
\par 19 Žmogus baudžiamas skausmais savo lovoje, visi jo kaulai apimti stipraus skausmo. 
\par 20 Mėgstamiausio maisto jis nebegali valgyti, 
\par 21 jo kūnas sunyksta, kad negali jo atpažinti, lieka tik vieni kaulai. 
\par 22 Jo siela artėja prie kapo, gyvybė­prie mirties. 
\par 23 Jei pas jį ateitų pasiuntinys kaip tarpininkas, vienas iš tūkstančio, ir parodytų žmogui Jo teisingumą, 
\par 24 Jis būtų maloningas jam ir sakytų: ‘Išlaisvink jį, kad nenueitų į duobę; Aš suradau išpirką’. 
\par 25 Jo kūnas atsinaujins ir jis grįš į jaunystės dienas. 
\par 26 Jis melsis Dievui, ir Tas bus maloningas jam. Su džiaugsmu jis regės Jo veidą, nes Jis sugrąžins žmogui savo teisumą. 
\par 27 Jis žiūrės į žmones ir sakys: ‘Buvau nusidėjęs ir nukrypęs nuo tiesos, bet man už tai neatlygino’. 
\par 28 Jis išgelbės jo sielą iš duobės ir jis gyvendamas matys šviesą. 
\par 29 Dievas visa tai kartoja žmogui du ar tris kartus, 
\par 30 norėdamas išgelbėti jo sielą, kad jis matytų šviesą ir gyventų. 
\par 31 Jobai, tylėk, klausyk ir įsidėmėk, ką aš sakysiu. 
\par 32 Jei turi ką pasakyti, kalbėk, nes aš trokštu tave pateisinti. 
\par 33 Jei ne, paklausyk manęs, ir aš pamokysiu tave išminties”.



\chapter{34}


\par 1 Elihuvas tęsė: 
\par 2 “Išminčiai, paklausykite mano žodžių ir supraskite juos, turintieji išmanymą. 
\par 3 Ausis skiria žodžius, kaip burna jaučia maisto skonį. 
\par 4 Kartu patyrinėkime, kas tiesa, ir nustatykime, kas gera. 
\par 5 Juk Jobas sakė: ‘Aš esu teisus, bet Dievas nedaro man teisybės. 
\par 6 Nors esu teisus, mane laiko melagiu; mano žaizda nepagydoma, nors esu nekaltas’. 
\par 7 Ar yra kitas toks žmogus kaip Jobas, kuris geria paniekinimus kaip vandenį, 
\par 8 kuris draugauja su piktadariais ir bendrauja su nedorėliais? 
\par 9 Jis sakė: ‘Žmogui jokios naudos, jei jis stengiasi patikti Dievui’. 
\par 10 Vyrai, kurie išmanote, paklausykite manęs. Negali būti, kad Dievas darytų neteisybę ir Visagalis nusikalstų. 
\par 11 Jis atlygina žmogui pagal jo darbus ir užmoka pagal jo kelius. 
\par 12 Tikrai Dievas nedaro neteisybės ir Visagalis neiškraipo teisės. 
\par 13 Kas Jam patikėjo žemę ir kas pavedė Jam visatą? 
\par 14 Jei Jis savo dvasią ir kvapą atimtų iš žmogaus, 
\par 15 tai žmogaus kūnas pražūtų ir virstų dulkėmis. 
\par 16 Jei ką nors supranti, tai paklausyk, ką sakau. 
\par 17 Ar gali būti valdovu tas, kuris nepakenčia teisingumo? Ar galėtum pasmerkti Tą, kuris yra visų teisiausias? 
\par 18 Kas sako karaliui, kad jis nedorėlis, arba kunigaikščiui, kad jis bedievis? 
\par 19 O Jis neatsižvelgia į kunigaikštį ir neteikia turtuoliams pirmenybės prieš vargšus, nes jie visi yra Jo kūriniai? 
\par 20 Staiga jie mirs, tautos bus išgąsdintos naktį ir pranyks. Galiūnus Jis pašalins, žmogui nepridėjus rankos. 
\par 21 Jis stebi žmogaus kelius ir mato visus jo žingsnius. 
\par 22 Jam nėra sutemų nei tamsos, kurioje piktadariai galėtų pasislėpti. 
\par 23 Todėl Jis nereikalauja iš žmogaus, kad tas eitų į teismą su Dievu. 
\par 24 Jis sutrupins galinguosius ir paskirs kitus į jų vietą. 
\par 25 Jis žino jų darbus, todėl parbloškia juos naktį ir sunaikina. 
\par 26 Jis baudžia juos kaip piktadarius visų akivaizdoje, 
\par 27 nes jie pasitraukė nuo Jo ir nepaisė Jo kelių. 
\par 28 Vargšų šauksmas pasiekė Jį ir Jis išklausė nuskriaustuosius. 
\par 29 Kai Jis duoda ramybę, kas gali varginti? Kas Jį suras, jei Jis pasislėps nuo tautos ar nuo atskiro žmogaus? 
\par 30 Jis apsaugo žmones, kad jiems nekaraliautų veidmainis. 
\par 31 Derėtų sakyti Dievui: ‘Aš nusipelniau Tavo bausmės, ateityje nebenusikalsiu. 
\par 32 Pamokyk mane, ko nežinau; jei nusikaltau, daugiau to nedarysiu’. 
\par 33 Ar Jis turėtų atlyginti pagal tavo supratimą dėl to, kad tu prieštarauji? Tu pasirenki, o ne aš. Todėl kalbėk, ką žinai. 
\par 34 Supratingi žmonės sako man, išminčiai, kurie klauso manęs: 
\par 35 ‘Jobas kalba nesuprasdamas ir jo žodžiai neapgalvoti’. 
\par 36 Jobo žodžius reikia iki galo ištirti, nes jis kalba kaip piktadarys. 
\par 37 Jis prideda maištą prie savo nuodėmės, ploja rankomis tarp mūsų ir kalba žodžių gausybę prieš Dievą”.



\chapter{35}


\par 1 Elihuvas tęsė: 
\par 2 “Ar manai, kad tu teisingai kalbi sakydamas: ‘Aš esu teisesnis už Dievą’? 
\par 3 Nes tu sakai: ‘Kokia nauda man iš to, jei aš nenusidedu?’ 
\par 4 Aš atsakysiu tau ir tavo draugams. 
\par 5 Pažvelk į dangaus debesis, kurie yra aukštai. 
\par 6 Jei nusikaltai, ar Jam pakenkei? Jei savo nuodėmes daugini, ar Jam ką padarai? 
\par 7 Jei teisus esi, kokia nauda Jam? Ką Jis gaus iš tavęs? 
\par 8 Tavo nedorybės kenkia tokiems kaip tu, ir tavo teisumas naudingas žmogaus sūnui. 
\par 9 Didelių vargų prispausti, žmonės šaukiasi pagalbos prieš smurtininkus. 
\par 10 Bet niekas neklausia: ‘Kur yra Dievas, mano Kūrėjas, kuris duoda giesmes naktį, 
\par 11 kuris sutvėrė mus išmintingesnius už gyvulius ir padangių paukščius?’ 
\par 12 Ten jie šaukia, bet niekas neatsako dėl piktadarių išdidumo. 
\par 13 Dievas nepaiso tuščių kalbų ir Visagalis nekreipia į jas dėmesio. 
\par 14 Nors tu sakai, kad Jo nematai, bet teisingumas yra prieš Jį, todėl pasitikėk Juo. 
\par 15 Kadangi Jis neaplankė savo rūstybėje ir nekreipė dėmesio į kvailybę, 
\par 16 todėl Jobas tuščiai atveria savo burną, išdidžiais žodžiais neišmintingai kalba”.



\chapter{36}


\par 1 Elihuvas kalbėjo toliau: 
\par 2 “Turėk kantrybės dar valandėlę, kad aš kalbėčiau už Dievą. 
\par 3 Aš remsiuosi praeitimi ir įrodysiu, kad mano Kūrėjas yra teisus. 
\par 4 Mano žodžiai teisingi, turintis tobulą pažinimą yra priešais tave. 
\par 5 Dievas yra galingas ir stiprus, tačiau neniekina nieko. 
\par 6 Jis nepalieka nedorėlio gyvo, bet apgina nuskriaustojo teises. 
\par 7 Jis neatitraukia nuo teisiųjų savo akių, bet su karaliais pasodina soste, įtvirtina ir išaukština juos. 
\par 8 Jei jie sukaustomi grandinėmis ir priespaudos metu surišami virvėmis, 
\par 9 tai Jis parodo jiems jų darbus ir nusikaltimus. 
\par 10 Dievas atveria jų ausis pamokymui ir įsako atsisakyti nusikaltimų. 
\par 11 Jei jie paklauso ir tarnauja Jam, praleis savo dienas klestėdami ir metus besidžiaugdami. 
\par 12 Bet jei jie nepaklūsta, pražus nuo kardo, mirs, neįgavę išminties. 
\par 13 Veidmainiai kaupia širdyje rūstybę, jie nesišaukia Jo, net būdami surišti. 
\par 14 Tokie miršta jaunystėje, jų gyvenimas tarp netyrųjų. 
\par 15 Jis išgelbsti vargšą iš jo vargo ir atveria jam ausis priespaudos metu. 
\par 16 Jis ir tave išlaisvintų ir padengtų tau stalą gėrybėmis. 
\par 17 Tu susilaukei nedorėlio bausmės, teismas ir teisingumas pasiekė tave. 
\par 18 Būk atsargus, kad Jis nepašalintų tavęs savo rūstybėje, nes tada ir didelė išpirka neišgelbės tavęs. 
\par 19 Ar Jis atsižvelgs į tavo turtus? Ne! Nei į auksą, nei į tavo galybę. 
\par 20 Nelauk nakties, kai tautos bus pašalintos ir sunaikintos. 
\par 21 Saugokis, nepalink į neteisybę, kurią tu pasirinktum vietoje kentėjimų. 
\par 22 Dievas išaukštintas savo galybe, kas gali pamokyti kaip Jis? 
\par 23 Kas Jam nurodė Jo kelius ir kas Jam pasakytų: ‘Tu klysti’? 
\par 24 Atsimink, kad galėtum išaukštinti Jo darbus, kuriuos žmonės matė. 
\par 25 Visi žmonės gali juos matyti ir pastebėti iš tolo. 
\par 26 Dievas yra didis, ne mums Jį suprasti; Jo metų tu negali suskaičiuoti. 
\par 27 Jis padaro iš vandens lašus, iš rūko lietų, 
\par 28 kurį debesys lieja gausiai ant žmonių. 
\par 29 Kas supras debesų išsidėstymą ir Jo palapinės garsus? 
\par 30 Jis paskleidžia savo šviesą ir apdengia jūros gelmes. 
\par 31 Jis teisia žmones ir teikia jiems maisto apsčiai. 
\par 32 Jis pažaboja žaibus ir įsako jiems smogti į tikslą. 
\par 33 Griaustinis praneša apie tai, ir gyvuliai jaučia, kas vyksta”.



\chapter{37}


\par 1 “Dėl to mano širdis dreba ir pasitraukė iš savo vietos. 
\par 2 Klausykite Jo balso, griaudėjimo, kuris sklinda iš Jo burnos. 
\par 3 Jis siunčia jį po visą padangę, Jo žaibai iki žemės pakraščių. 
\par 4 Po to aidi balsas. Jis sugriaudžia savo didybės balsu ir nieko nepasilieka, kai Jo balsas pasigirsta. 
\par 5 Dievas didingai griaudėja savo balsu, Jis daro mums nesuvokiamų dalykų. 
\par 6 Sniegui Jis įsako snigti, silpnas ir stiprus lietus priklauso nuo Jo. 
\par 7 Kad žmonės pažintų Jo darbą, Jis užantspauduoja žmonių rankas. 
\par 8 Tuomet ir žvėrys slepiasi savo lindynėse. 
\par 9 Iš pietų ateina audra, iš šiaurės­šaltis. 
\par 10 Dievo kvapu padaromas ledas, ir platūs vandenys sustingsta. 
\par 11 Jis pripildo debesis drėgmės, iš jų sklinda žaibai. 
\par 12 Jie plaukia, kur Jis nukreipia, ir vykdo, ką Jis įsako, visuose žemės kraštuose. 
\par 13 Jis tai daro norėdamas sudrausti, palaiminti arba pasigailėti. 
\par 14 Jobai, stebėk ir apsvarstyk Dievo nuostabius darbus. 
\par 15 Ar žinai, kaip Dievas juos suvaldo ir parodo savo debesies šviesą? 
\par 16 Ar žinai, kaip debesys laikosi, šitie nuostabūs darbai To, kuris turi tobulą pažinimą? 
\par 17 Ar žinai, kodėl drabužiai įkaista, kai Jis ramina žemę pietų vėju? 
\par 18 Ar tu su Juo ištiesei dangaus skliautą tvirtą kaip veidrodį, iš vario nulietą? 
\par 19 Pamokyk mus, ką turime Jam sakyti, nes mes nesusigaudome tamsoje. 
\par 20 Ar bus Jam pranešta, ką kalbu? Jei žmogus kalbėtų, jis būtų prarytas. 
\par 21 Kai debesys uždengia saulę, šviesos nematyti, bet, vėjui papūtus, dangus nuskaidrėja. 
\par 22 Iš šiaurės ateina giedra, o Dievas yra bauginančiai didingas. 
\par 23 Visagalis mums nepasiekiamas; Jis galingas jėga, tiesa ir teisingumu, Jis neišnaudoja. 
\par 24 Todėl žmonės Jo bijo. Jis nepaiso tų, kurie dedasi išmintingi”.



\chapter{38}


\par 1 Viešpats atsiliepė Jobui iš audros ir tarė: 
\par 2 “Kas aptemdo patarimą neprotingais žodžiais? 
\par 3 Susijuosk dabar kaip vyras; Aš klausiu tave, o tu atsakyk man. 
\par 4 Kur buvai, kai Aš dėjau žemės pamatus? Atsakyk, jei supranti. 
\par 5 Ar žinai, kas nustatė jos dydį, kas ją išmatavo? 
\par 6 Ant ko pritvirtintas jos pamatas arba kas padėjo jos kertinį akmenį, 
\par 7 kai kartu giedojo ryto žvaigždės ir šaukė iš džiaugsmo visi Dievo sūnūs? 
\par 8 Kas uždarė jūros duris, kai ji veržėsi, tarsi išeidama iš įsčių? 
\par 9 Aš aprengiau ją debesimis lyg drabužiu ir apsupau tamsa lyg vystyklais, 
\par 10 kai jai paskyriau ribas, įdėjau skląstį bei duris 
\par 11 ir pasakiau: ‘Iki čia ateisi, ne toliau; čia sustos tavo puikiosios bangos’. 
\par 12 Ar kada nors savo gyvenime įsakei rytui ir nurodei aušrai jos vietą, 
\par 13 kad ji, pasiekus žemės pakraščius, nukratytų nedorėlius nuo jos? 
\par 14 Žemė keičiasi kaip molis po antspaudu, susiklosto kaip drabužis. 
\par 15 Nedorėliams saulė nebešviečia, jų pakelta ranka sulaužoma. 
\par 16 Ar kada pasiekei jūros šaltinius ir vaikštinėjai tyrinėdamas gelmes? 
\par 17 Ar mirties vartai tau buvo atverti, ar matei mirties šešėlio duris? 
\par 18 Ar išmatavai žemės platybes? Atsakyk, jei visa tai žinai. 
\par 19 Kur yra kelias į šviesą, kur gyvena tamsa? 
\par 20 Ar gali pasiekti jų ribas ir surasti taką į jų namus? 
\par 21 Aišku, tu žinai, nes tada jau buvai gimęs ir gyveni nuo amžių! 
\par 22 Ar buvai nuėjęs į sniego sandėlius ir sandėlius krušos, 
\par 23 kurią laikau sielvarto metui, karo ir kovos dienai? 
\par 24 Kur yra kelias, kuriuo ateina šviesa ir iš kur pakyla žemėje rytys? 
\par 25 Kas nustatė lietui ir žaibui kryptį, 
\par 26 kad lytų negyvenamose vietose, dykumose, kur nėra žmonių, 
\par 27 drėkintų tuščią ir apleistą žemę, kad želtų žolė? 
\par 28 Ar lietus turi tėvą? Kas pagimdė rasos lašus? 
\par 29 Kur gimė ledas? Kas pagimdė šerkšną po dangumi? 
\par 30 Kodėl vanduo sukietėja į akmenį ir gelmių paviršius užšąla? 
\par 31 Ar gali surišti Sietyno raiščius ir atrišti Oriono? 
\par 32 Ar gali liepti užtekėti Zodiakui ir Grįžulo ratams jų laiku? 
\par 33 Ar pažįsti dangaus nuostatus ir gali juos pritaikyti žemei? 
\par 34 Ar gali įsakyti debesiui, kad jo srovės išsilietų ant tavęs? 
\par 35 Ar gali pasiųsti žaibus, kad jie išeitų, sakydami: ‘Štai mes čia’? 
\par 36 Kas įdėjo išmintį į širdį ir davė supratimą protui? 
\par 37 Kas gali suskaičiuoti debesis? Kas gali uždaryti dangaus indus, 
\par 38 kai dulkės tampa purvu ir grumstai sulimpa? 
\par 39 Ar gali sumedžioti liūtei grobį ir pasotinti jos jauniklius, 
\par 40 gulinčius olose ir lindynėse? 
\par 41 Kas paruošia varnui peną, kai jo jaunikliai šaukiasi Dievo, klaidžiodami be maisto?”



\chapter{39}

\par 1 “Ar žinai kalnų ožių atsivedimo laiką? Ar stebėjai stirnų gimimą? 
\par 2 Ar gali suskaičiuoti jų nėštumo mėnesius ir ar žinai laiką, kada jos atsives? 
\par 3 Jos susiriečia, dejuoja ir atsiveda vaikų. 
\par 4 Jų jaunikliai, sustiprėję ir užaugę atvirame lauke, atsiskiria ir nebesugrįžta. 
\par 5 Kas leido laukiniam asilui laisvai bėgioti ir kas atrišo jo pančius? 
\par 6 Aš paskyriau jam namais tyrus, nederlingoje žemėje jį apgyvendinau. 
\par 7 Jis juokiasi iš miesto spūsties, vežiko šauksmų negirdi. 
\par 8 Aukštai kalnuose jis randa sau ganyklą, ieško žaliuojančių plotų. 
\par 9 Ar stumbras tau tarnaus, ar jis stovės naktį prie tavo ėdžių? 
\par 10 Ar gali jį pakinkyti ir ar jis ars slėnį paskui tave? 
\par 11 Ar pasitikėsi juo ir jo didele jėga? Ar paliksi jam savo darbą? 
\par 12 Ar tiki, kad jis suveš tavo pasėlius į klojimą? 
\par 13 Ar tu davei povui gražius sparnus ir plunksnas bei sparnus stručiui? 
\par 14 Jis pakasa žemėje savo kiaušinius ir smėlyje leidžia jiems šilti. 
\par 15 Jis nesupranta, kad koja gali juos sutraiškyti ir laukinis žvėris sumindyti. 
\par 16 Jis šiurkščiai elgiasi su savo vaikais, tarsi jie būtų svetimi; jis nebijo, kad darbuojasi veltui, 
\par 17 nes Dievas neapdovanojo jo išmintimi ir nedavė jam supratimo. 
\par 18 Jei jis pasikelia bėgti, pasijuokia iš žirgo ir raitelio. 
\par 19 Ar tu suteikei žirgui stiprybės? Ar papuošei jo sprandą karčiais? 
\par 20 Ar gali jį išgąsdinti kaip žiogą? Jo šnervių prunkštimas baisus. 
\par 21 Jis kasa žemę ir džiaugiasi savo jėga, bėga prieš ginkluotų žmonių būrį. 
\par 22 Jis nepažįsta baimės ir nenusigąsta, jis nesitraukia nuo kardo. 
\par 23 Jei žvanga strėlinės, žiba ietys ir skydai, 
\par 24 jis trypia ir kasa žemę, nerimsta gaudžiant trimitui. 
\par 25 Trimitams pasigirdus, jis žvengia: Y-ha-ha! Jis iš tolo nujaučia kovą, girdi vado įsakymus ir kovos šauksmą. 
\par 26 Ar tavo išmintimi pakyla sakalas, išskleidžia savo sparnus ir skrenda link pietų? 
\par 27 Ar tavo įsakymu sklando erelis ir krauna savo lizdą aukštumose? 
\par 28 Jis gyvena ant aukščiausios uolos neprieinamoje vietoje. 
\par 29 Iš ten jis dairosi grobio, jo akys pamato jį iš tolo. 
\par 30 Jo jaunikliai geria kraują; kur yra žuvusių, ten ir jis”.



\chapter{40}


\par 1 Viešpats, atsakydamas Jobui, tarė: 
\par 2 “Ar tas, kuris ginčijasi su Visagaliu, pamokys Jį? Teatsako tas, kuris priekaištauja Dievui”. 
\par 3 Jobas atsakė Viešpačiui: 
\par 4 “Aš per menkas Tau atsakyti. Aš uždengsiu ranka savo burną. 
\par 5 Kartą kalbėjau, bet daugiau nebekalbėsiu ir nebeatsakysiu”. 
\par 6 Tuomet Viešpats kalbėjo Jobui iš audros: 
\par 7 “Susijuosk dabar kaip vyras, Aš klausiu tavęs, o tu atsakyk man. 
\par 8 Ar tu panaikinsi mano sprendimą? Ar mane smerksi, o save teisinsi? 
\par 9 Ar tavo ranka tokia kaip Dievo? Ar tavo balsas toks stiprus kaip Jo balsas? 
\par 10 Pasipuošk garbe ir kilnumu, apsisiausk šlove ir spindesiu. 
\par 11 Išliek savo rūstybę, pažvelk į kiekvieną išdidų ir pažemink jį. 
\par 12 Pažemink visus išdidžiuosius ir sutrypk nedorėlius ten, kur jie yra. 
\par 13 Paslėpk juos visus dulkėse ir nuvesk į mirties tamsą. 
\par 14 Tada Aš pripažinsiu, kad tavo dešinė gali tave išgelbėti. 
\par 15 Štai begemotas, kurį padariau kartu su tavimi; jis ėda žolę kaip jautis. 
\par 16 Jo jėga strėnose ir pilvo raumenyse. 
\par 17 Jis iškelia savo uodegą kaip kedrą; jo šlaunų raumenys tvirtai susipynę. 
\par 18 Jo kaulai kaip variniai vamzdžiai, o skeletas kaip geležiniai virbai. 
\par 19 Jis yra Dievo kelių pradžia. Tik Tas, kuris jį sukūrė, gali jį įveikti. 
\par 20 Kalnuose, kur gyvena laukiniai žvėrys, auga jam maistas. 
\par 21 Jis guli po ūksmingais medžiais, pasislėpęs tarp nendrių pelkėse. 
\par 22 Jį dengia ūksmingų medžių šešėliai, paupio gluosniai jį supa. 
\par 23 Jis geria iš upės neskubėdamas, jam atrodo, kad Jordanas sutilps į jo nasrus. 
\par 24 Kas galėtų jį pagauti kabliais arba žabangais?”



\chapter{41}


\par 1 “Ar gali pagauti leviataną kabliu ir užnerti virvę jam ant liežuvio? 
\par 2 Ar gali perverti kabliu jo šnerves ir akstinu perdurti jo žiaunas? 
\par 3 Ar jis maldaus tave, ar kalbės švelniais žodžiais? 
\par 4 Ar jis sudarys sutartį su tavimi ir tarnaus tau amžinai? 
\par 5 Ar gali žaisti su juo kaip su paukščiu ir jį pririšti savo mergaitėms? 
\par 6 Ar dėl jo tarsis žvejai ir pasidalins jį pirkliai? 
\par 7 Ar gali prismaigstyti strėlių į jo odą ir žeberklu persmeigti jo galvą? 
\par 8 Jei paliesi jį ranka, atsiminsi ir daugiau to nebedarysi. 
\par 9 Tuščia viltis jį nugalėti, nes vien į jį pažvelgus, baimė ima. 
\par 10 Niekas nedrįstų jo erzinti. Kas tad galėtų atsilaikyti prieš mane? 
\par 11 Kas davė man pirmas, kad jam atlyginčiau? Viskas po dangumi priklauso man. 
\par 12 Neslėpsiu jo didybės, galios ir tobulos sandaros. 
\par 13 Kas atidengs jo drabužio kraštą? Kas sieks tarp jo dantų? 
\par 14 Kas jį pražiodys? Jo dantys baisūs. 
\par 15 Jis didžiuojasi savo žvynais, kurie sutvirtinti lyg antspaudu. 
\par 16 Jie taip arti vienas kito, kad nė oras nepatenka į jų tarpą. 
\par 17 Jie vienas su kitu neatskiriamai sujungti. 
\par 18 Nuo jo čiaudėjimo blykčioja žaibai, jo akys spindi lyg aušra. 
\par 19 Iš jo nasrų eina ugnis, skraido ugnies kibirkštys. 
\par 20 Iš jo šnervių kyla garai kaip iš verdančio katilo. 
\par 21 Jo kvapas uždega anglis, liepsna veržiasi iš jo nasrų. 
\par 22 Jo jėga sprande; jo išvaizda baugina. 
\par 23 Jo kūno dalys tvirtai sujungtos, jos nepajudinamos. 
\par 24 Širdis jo kieta kaip akmuo, tvirta kaip apatinė girnų pusė. 
\par 25 Prieš jį dreba galiūnai, išgąsdinti pasitraukia. 
\par 26 Nei kardu, nei strėle ar ietimi jo nesužeisi. 
\par 27 Geležis jam kaip šiaudai, varis kaip supuvęs medis. 
\par 28 Strėlės jo negąsdina, mėtyklės akmenys jam tik pelai. 
\par 29 Lazdos jam kaip ražienos, jis juokiasi iš švilpiančių iečių. 
\par 30 Po juo aštrūs akmenys, ant aštrių šukių jis guli kaip ant dumblo. 
\par 31 Jis užvirina gelmę kaip puodą, jūrą padaro kaip tepalų puodą. 
\par 32 Jam nuplaukus, lieka šviesus takas, gelmė atrodo pražilusi. 
\par 33 Žemėje nėra jam lygaus; jis nepažįsta baimės. 
\par 34 Jis žiūri į viską iš aukšto; jis karalius visų išdidumo vaikų”.



\chapter{42}


\par 1 Jobas atsakė Viešpačiui, tardamas: 
\par 2 “Žinau, kad Tu esi Visagalis; niekas nesutrukdys Tau padaryti, ką sumanei. 
\par 3 Kas paniekina patarimą, neturėdamas supratimo? Aš kalbėjau tai, ko nesuprantu, kas man per daug nuostabu ir nežinoma. 
\par 4 Paklausyk, ir aš kalbėsiu, aš klausiu, o Tu man atsakyk. 
\par 5 Anksčiau savo ausimis girdėjau apie Tave, o dabar mano akys mato Tave. 
\par 6 Todėl aš baisiuosi savimi ir atgailauju dulkėse ir pelenuose”. 
\par 7 Po pasikalbėjimo su Jobu Viešpats tarė temaniui Elifazui: “Mano rūstybė užsidegė prieš tave ir abu tavo draugus, nes jūs nekalbėjote teisingai apie mane kaip mano tarnas Jobas. 
\par 8 Dabar imkite septynis jaučius bei septynis avinus, eikite pas mano tarną Jobą ir juos aukokite kaip deginamąją auką už save, o mano tarnas Jobas melsis už jus. Aš išklausysiu jo maldas ir nebausiu jūsų, kaip esate nusipelnę, nes jūs nekalbėjote apie mane teisingai kaip mano tarnas Jobas”. 
\par 9 Tuomet temanas Elifazas, šuachas Bildadas ir naamatietis Cofaras padarė, ką Viešpats buvo jiems įsakęs, o Viešpats išklausė Jobą. 
\par 10 Kai jis meldėsi už savo draugus, Viešpats išvadavo Jobą ir davė jam visko dvigubai, negu jis anksčiau turėjo. 
\par 11 Pas jį atėjo jo broliai, seserys ir visi jo pirmiau buvę pažįstami; jie valgė duoną su juo jo namuose, užjautė jį ir guodė nelaimėje, kurią Viešpats jam buvo siuntęs; kiekvienas dovanojo jam pinigų ir auksinį auskarą. 
\par 12 Viešpats laimino Jobo senatvę labiau negu jaunystę. Jis turėjo keturiolika tūkstančių avių, šešis tūkstančius kupranugarių, tūkstantį jungų jaučių ir tūkstantį asilių. 
\par 13 Be to, jis turėjo septynis sūnus ir tris dukteris. 
\par 14 Pirmosios vardas buvo Jamima, antrosios­Kecija ir trečiosios­ Keren Hapucha. 
\par 15 Visoje šalyje nebuvo gražesnių moterų už Jobo dukteris. Jų tėvas davė joms paveldėjimą tarp jų brolių. 
\par 16 Jobas po to dar gyveno šimtą keturiasdešimt metų; jis matė savo vaikus ir vaikaičius iki ketvirtos kartos. 
\par 17 Jobas mirė senatvėje, pasisotinęs gyvenimo dienomis.



\end{document}