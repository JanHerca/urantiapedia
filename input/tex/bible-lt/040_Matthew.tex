\begin{document}

\title{Evangelija pagal Matą}

\chapter{1}


\par 1 Jėzaus Kristaus, Dovydo Sūnaus, Abraomo Sūnaus, kilmės knyga. 
\par 2 Abraomui gimė Izaokas, Izaokui gimė Jokūbas, Jokūbui gimė Judas ir jo broliai. 
\par 3 Judui gimė Faras ir Zara iš Tamaros, Farui gimė Esromas, Esromui gimė Aramas. 
\par 4 Aramui gimė Aminadabas, Aminadabui gimė Naasonas, Naasonui gimė Salmonas. 
\par 5 Salmonui gimė Boozas iš Rahabos, Boozui gimė Jobedas iš Rūtos, Jobedui gimė Jesė. 
\par 6 Jesei gimė karalius Dovydas. Dovydui gimė Saliamonas iš Ūrijos žmonos. 
\par 7 Saliamonui gimė Roboamas, Roboamui gimė Abija, Abijai gimė Asa. 
\par 8 Asai gimė Juozapatas, Juozapatui gimė Joramas, Joramui gimė Ozijas. 
\par 9 Ozijui gimė Joatamas, Joatamui gimė Achazas, Achazui gimė Ezekijas. 
\par 10 Ezekijui gimė Manasas, Manasui gimė Amonas, Amonui gimė Jozijas. 
\par 11 Jozijui gimė Jechonijas ir jo broliai ištrėmimo į Babiloniją laikais. 
\par 12 Po ištrėmimo į Babiloniją Jechonijui gimė Salatielis, Salatieliui gimė Zorobabelis. 
\par 13 Zorobabeliui gimė Abijudas, Abijudui gimė Eliakimas, Eliakimui gimė Azoras. 
\par 14 Azorui gimė Sadokas, Sadokui gimė Achimas, Achimui gimė Elijudas. 
\par 15 Elijudui gimė Eleazaras, Eleazarui gimė Matanas, Matanui gimė Jokūbas. 
\par 16 Jokūbui gimė Juozapas­vyras Marijos, iš kurios gimė Jėzus, vadinamas Kristumi. 
\par 17 Taigi nuo Abraomo iki Dovydo iš viso keturiolika kartų, nuo Dovydo iki ištrėmimo į Babiloniją keturiolika kartų ir nuo ištrėmimo į Babiloniją iki Kristaus keturiolika kartų. 
\par 18 Jėzaus Kristaus gimimas buvo toksai. Jo motina Marija buvo susižadėjusi su Juozapu; dar nepradėjus jiems kartu gyventi, ji tapo nėščia iš Šventosios Dvasios. 
\par 19 Jos vyras Juozapas, būdamas teisus ir nenorėdamas daryti jai nešlovės, sumanė tylomis ją atleisti. 
\par 20 Kai jis nusprendė taip padaryti, sapne pasirodė jam Viešpaties angelas ir tarė: “Juozapai, Dovydo sūnau, nebijok parsivesti į namus savo žmonos Marijos, nes jos vaisius yra iš Šventosios Dvasios. 
\par 21 Ji pagimdys Sūnų, kuriam tu duosi vardą Jėzus, nes Jis išgelbės savo tautą iš jos nuodėmių”. 
\par 22 Visa tai įvyko, kad išsipildytų, kas buvo Viešpaties pasakyta per pranašą: 
\par 23 “Štai mergelė pradės įsčiose ir pagimdys Sūnų, ir Jį pavadins Emanueliu”, tai reiškia: “Dievas su mumis”. 
\par 24 Atsikėlęs Juozapas padarė taip, kaip Viešpaties angelas jam įsakė, ir parsivedė žmoną pas save. 
\par 25 Jam negyvenus su ja kaip vyrui, ji pagimdė Sūnų, kurį jis pavadino Jėzumi.


\chapter{2}


\par 1 Jėzui gimus Judėjos Betliejuje karaliaus Erodo dienomis, štai atkeliavo į Jeruzalę išminčiai iš Rytų ir klausinėjo: 
\par 2 “Kur yra gimęs žydų karalius? Mes matėme Jo žvaigždę Rytuose ir atvykome pagarbinti Jį”. 
\par 3 Tai išgirdęs, karalius Erodas sunerimo, o su juo ir visa Jeruzalė. 
\par 4 Jis sukvietė visus tautos aukštuosius kunigus bei Rašto žinovus ir teiravosi, kur turėjęs gimti Kristus. 
\par 5 Tie jam atsakė: “Judėjos Betliejuje, nes taip pranašo parašyta: 
\par 6 ‘Ir tu, Judo žemės Betliejau, anaiptol nesi mažiausias tarp Judo valdovų, nes iš tavęs išeis Valdovas, kuris ganys mano tautą­ Izraelį’ ”. 
\par 7 Tada Erodas, slapta pasikvietęs išminčius, sužinojo iš jų apie žvaigždės pasirodymo metą 
\par 8 ir, siųsdamas į Betliejų, tarė: “Keliaukite ir viską kruopščiai sužinokite apie kūdikį. Radę praneškite man, kad ir aš nuvykęs Jį pagarbinčiau”. 
\par 9 Išklausę karaliaus, išminčiai leidosi kelionėn. Ir štai žvaigždė, kurią jie matė Rytuose, traukė pirma, kol sustojo ties ta vieta, kur buvo kūdikis. 
\par 10 Išvydę žvaigždę, jie labai džiaugėsi. 
\par 11 Įžengę į namus, rado kūdikį su motina Marija ir, parpuolę ant žemės, Jį pagarbino. Jie atidarė savo brangenybių dėžes ir davė Jam dovanų: aukso, smilkalų ir miros. 
\par 12 Sapne įspėti nebegrįžti pas Erodą, kitu keliu pasuko į savo kraštą. 
\par 13 Jiems iškeliavus, štai pasirodė Juozapui sapne Viešpaties angelas ir tarė: “Kelkis, imk kūdikį, Jo motiną ir bėk į Egiptą. Pasilik ten, kol tau pasakysiu, nes Erodas ieškos kūdikio, norėdamas Jį nužudyti”. 
\par 14 Atsikėlęs Juozapas paėmė kūdikį ir Jo motiną ir pasitraukė į Egiptą. 
\par 15 Ten jis prabuvo iki Erodo mirties, kad išsipildytų, kas Viešpaties buvo pasakyta per pranašą: “Iš Egipto pašaukiau savo Sūnų”. 
\par 16 Erodas, pamatęs, kad jį išminčiai apgavo, baisiai įniršo ir pasiuntė išžudyti Betliejuje ir jo apylinkėse visus berniukus, dvejų metų ir jaunesnius, pagal laiką, kurį buvo patyręs iš išminčių. 
\par 17 Tada išsipildė, kas buvo pasakyta per pranašą Jeremiją: 
\par 18 “Ramoje pasigirdo šauksmas, raudos, aimanos ir garsios dejonės: tai Rachelė rauda savo vaikų; ir niekas jos nepaguos, nes jų nebėra”. 
\par 19 Erodui mirus, štai Viešpaties angelas pasirodė per sapną Juozapui Egipte 
\par 20 ir tarė: “Kelkis, imk kūdikį su Jo motina ir keliauk į Izraelio žemę, nes jau mirė tie, kurie ieškojo kūdikio gyvybės”. 
\par 21 Tuomet Juozapas atsikėlė, paėmė kūdikį ir Jo motiną ir sugrįžo į Izraelio žemę. 
\par 22 Bet, išgirdęs, jog Archelajas valdo Judėją po savo tėvo Erodo, pabūgo ten vykti. Įspėtas sapne, nukeliavo į Galilėjos sritį 
\par 23 ir apsigyveno Nazareto mieste, kad išsipildytų, kas buvo pranašų pasakyta: “Jį vadins Nazariečiu”.



\chapter{3}


\par 1 Anomis dienomis pasirodė Jonas Krikštytojas ir pamokslavo Judėjos dykumoje, 
\par 2 skelbdamas: “Atgailaukite, nes prisiartino dangaus karalystė”. 
\par 3 O jis buvo tas, apie kurį pranašas Izaijas yra pasakęs: “Dykumoje šaukiančiojo balsas: ‘Paruoškite Viešpačiui kelią! Ištiesinkite Jam takus!’ ” 
\par 4 Pats Jonas vilkėjo kupranugario vilnų apdaru, o strėnas buvo susijuosęs odiniu diržu. Jis maitinosi skėriais ir lauko medumi. 
\par 5 Tuomet Jeruzalė, visa Judėja ir visa Pajordanė ėjo pas jį. 
\par 6 Žmonės buvo jo krikštijami Jordano upėje ir išpažindavo savo nuodėmes. 
\par 7 Pamatęs daug fariziejų ir sadukiejų, einančių pas jį krikštytis, jis jiems sakė: “Angių išperos, kas perspėjo jus bėgti nuo ateinančios rūstybės? 
\par 8 Duokite vaisių, vertų atgailos! 
\par 9 Ir nebandykite ramintis: ‘Mūsų tėvas­Abraomas’. Aš jums sakau, kad Dievas gali pažadinti Abraomui vaikų iš šitų akmenų. 
\par 10 Štai kirvis jau prie medžių šaknų, ir kiekvienas medis, kuris neduoda gerų vaisių, yra nukertamas ir įmetamas į ugnį. 
\par 11 Aš jus krikštiju vandeniu atgailai, bet Tas, kuris ateina po manęs,­galingesnis už mane, aš nevertas net Jo sandalų nuauti. Jis krikštys jus Šventąja Dvasia ir ugnimi. 
\par 12 Jo rankoje vėtyklė, ir Jis kruopščiai išvalys savo kluoną. Kviečius surinks į klėtį, o pelus sudegins neužgesinama ugnimi”. 
\par 13 Tuomet Jėzus iš Galilėjos atėjo prie Jordano pas Joną krikštytis. 
\par 14 Jonas Jį atkalbinėjo: “Aš turėčiau būti Tavo pakrikštytas, o Tu ateini pas mane!” 
\par 15 Bet Jėzus jam atsakė: “Šį kartą paklausyk! Taip mudviem dera įvykdyti visą teisumą”. Tada Jonas sutiko. 
\par 16 Pakrikštytas Jėzus tuoj išbrido iš vandens. Ir štai Jam atsivėrė dangus, ir Jis pamatė Dievo Dvasią, sklendžiančią žemyn it balandį ir nusileidžiančią ant Jo. 
\par 17 Ir štai balsas iš dangaus prabilo: “Šitas yra mano mylimas Sūnus, kuriuo Aš gėriuosi”.



\chapter{4}


\par 1 Tuomet Jėzus buvo Dvasios nuvestas į dykumą, kad būtų velnio gundomas. 
\par 2 Išpasninkavęs keturiasdešimt dienų ir keturiasdešimt naktų, Jis buvo alkanas. 
\par 3 Prie Jo prisiartino gundytojas ir tarė: “Jei Tu Dievo Sūnus, liepk, kad šie akmenys pavirstų duona”. 
\par 4 Bet Jėzus atsakė: “Parašyta: ‘Žmogus gyvens ne viena duona, bet kiekvienu žodžiu, išeinančiu iš Dievo lūpų’ ”. 
\par 5 Tada velnias paėmė Jį į šventąjį miestą, pastatė ant šventyklos šelmens 
\par 6 ir tarė Jam: “Jei Tu Dievo Sūnus, pulk žemyn, nes parašyta: ‘Jis lieps savo angelams globoti Tave, ir jie nešios Tave ant rankų, kad neužsigautum kojos į akmenį’ ”. 
\par 7 Jėzus jam atsakė: “Taip pat parašyta: ‘Negundyk Viešpaties, savo Dievo’ ”. 
\par 8 Velnias vėl paėmė Jį į labai aukštą kalną ir, rodydamas viso pasaulio karalystes bei jų šlovę, 
\par 9 tarė Jam: “Visa tai aš Tau atiduosiu, jei parpuolęs pagarbinsi mane”. 
\par 10 Tada Jėzus jam atsakė: “Eik šalin nuo manęs, šėtone! Nes parašyta: ‘Viešpatį, savo Dievą, tegarbink ir Jam vienam tetarnauk!’ ” 
\par 11 Tuomet velnias nuo Jo atsitraukė, ir štai angelai prisiartino ir Jam tarnavo. 
\par 12 Išgirdęs, kad Jonas suimtas, Jėzus pasitraukė į Galilėją. 
\par 13 Jis paliko Nazaretą ir apsigyveno Kafarnaume, prie ežero, kur susieina Zabulono ir Neftalio sritys, 
\par 14 kad išsipildytų, kas buvo pasakyta per pranašą Izaiją: 
\par 15 “Zabulono ir Neftalio žeme! Paežerės juosta, žeme už Jordano­pagonių Galilėja! 
\par 16 Tamsybėje sėdinti tauta išvydo skaisčią šviesą, gyvenantiems mirties šalyje ir šešėlyje užtekėjo šviesa”. 
\par 17 Nuo to laiko Jėzus pradėjo pamokslauti, skelbdamas: “Atgailaukite, nes prisiartino dangaus karalystė!” 
\par 18 Vaikščiodamas palei Galilėjos ežerą, Jėzus pamatė du brolius­ Simoną, vadinamą Petru, ir jo brolį Andriejų­metančius tinklą į ežerą; mat jie buvo žvejai. 
\par 19 Jis tarė jiems: “Sekite paskui mane, ir Aš padarysiu jus žmonių žvejais”. 
\par 20 Tuodu tuojau paliko tinklus ir nusekė paskui Jį. 
\par 21 Paėjęs toliau, Jis pamatė kitus du brolius­Zebediejaus sūnų Jokūbą ir jo brolį Joną. Jiedu su savo tėvu Zebediejumi valtyje taisė tinklus. Jėzus juos pašaukė, 
\par 22 ir tie, tučtuojau palikę valtį ir tėvą, nusekė paskui Jį. 
\par 23 Jėzus vaikščiojo po visą Galilėją, mokydamas sinagogose, pamokslaudamas karalystės Evangeliją ir gydydamas visas žmonių ligas bei negalias. 
\par 24 Garsas apie Jį pasklido visoje Sirijoje. Žmonės nešė pas Jį visus sergančius, įvairiausių ligų bei kentėjimų suimtus, demonų apsėstus, nakvišas bei paralyžiuotus,­ir Jis išgydydavo juos. 
\par 25 Paskui Jį sekė didelės minios žmonių iš Galilėjos, Dekapolio, Jeruzalės, Judėjos ir Užjordanės.



\chapter{5}


\par 1 Matydamas minias, Jėzus užkopė į kalną ir atsisėdo. Prie Jo priėjo mokiniai. 
\par 2 Atvėręs lūpas, Jis ėmė mokyti: 
\par 3 “Palaiminti vargšai dvasia, nes jų yra dangaus karalystė. 
\par 4 Palaiminti, kurie liūdi, nes jie bus paguosti. 
\par 5 Palaiminti romieji, nes jie paveldės žemę. 
\par 6 Palaiminti, kurie alksta ir trokšta teisumo, nes jie bus pasotinti. 
\par 7 Palaiminti gailestingieji, nes jie susilauks gailestingumo. 
\par 8 Palaiminti tyraširdžiai, nes jie regės Dievą. 
\par 9 Palaiminti taikdariai, nes jie bus vadinami Dievo vaikais. 
\par 10 Palaiminti, kurie persekiojami dėl teisumo, nes jų yra dangaus karalystė. 
\par 11 Palaiminti jūs, kai dėl manęs jus šmeižia ir persekioja bei meluodami visaip piktžodžiauja. 
\par 12 Būkite linksmi ir džiūgaukite, nes didelis jūsų atlygis danguje. Juk lygiai taip persekiojo ir iki jūsų buvusius pranašus”. 
\par 13 “Jūs esate žemės druska. Jei druska netenka sūrumo, kuo gi ją reikėtų pasūdyti? Ji niekam netinka, ir belieka ją išberti žmonėms sumindžioti. 
\par 14 Jūs esate pasaulio šviesa. Neįmanoma nuslėpti miesto, pastatyto ant kalno. 
\par 15 Ir niekas, uždegęs žiburį, nevožia jo indu, bet stato į žibintuvą, kad šviestų visiems, kas yra namuose. 
\par 16 Taip tešviečia ir jūsų šviesa žmonių akivaizdoje, kad jie matytų jūsų gerus darbus ir šlovintų jūsų Tėvą, kuris danguje”. 
\par 17 “Nemanykite, jog Aš atėjau panaikinti Įstatymo ar Pranašų. Ne panaikinti jų atėjau, bet įvykdyti. 
\par 18 Iš tiesų sakau jums: kol dangus ir žemė nepraeis, nė viena raidelė ir nė vienas brūkšnelis neišnyks iš Įstatymo, kol viskas išsipildys. 
\par 19 Todėl, kas sulaužytų bent vieną iš mažiausių įsakymų ir taip mokytų žmones, tas bus vadinamas mažiausiu dangaus karalystėje. O kas juos vykdys ir jų mokys, bus vadinamas didžiu dangaus karalystėje. 
\par 20 Taigi sakau jums: jeigu jūsų teisumas nepranoks Rašto žinovų ir fariziejų teisumo,­neįeisite į dangaus karalystę”. 
\par 21 “Jūs girdėjote, kad protėviams buvo pasakyta: ‘Nežudyk’; o kas nužudo, turės atsakyti teisme. 
\par 22 O Aš jums sakau: kas be reikalo pyksta ant savo brolio, turės atsakyti teisme. Kas sako savo broliui: ‘Pusgalvi’, turės stoti prieš sinedrioną. O kas sako: ‘Beproti’, tas smerktinas į pragaro ugnį. 
\par 23 Todėl jei neši dovaną prie aukuro ir ten prisimeni, jog tavo brolis turi šį tą prieš tave, 
\par 24 palik savo dovaną ten prie aukuro, eik pirmiau susitaikinti su savo broliu, ir tik tada sugrįžęs aukok savo dovaną. 
\par 25 Greitai susitark su savo kaltintoju, dar kelyje į teismą, kad kaltintojas neįduotų tavęs teisėjui, o teisėjas­teismo vykdytojui ir kad nepakliūtum į kalėjimą. 
\par 26 Iš tiesų sakau tau: neišeisi iš ten, kol neatsiteisi iki paskutinio skatiko”. 
\par 27 “Jūs girdėjote, jog protėviams buvo pasakyta: ‘Nesvetimauk!’ 
\par 28 O Aš jums sakau: kiekvienas, kuris geidulingai žiūri į moterį, jau svetimauja savo širdyje. 
\par 29 Jeigu tavo dešinioji akis skatina tave nusidėti, išlupk ją ir mesk šalin. Geriau tau netekti vieno nario, negu kad visas kūnas būtų įmestas į pragarą. 
\par 30 Ir jeigu tavo dešinioji ranka skatina tave nusidėti, nukirsk ją ir mesk šalin. Geriau tau netekti vieno nario, negu kad visas kūnas būtų įmestas į pragarą”. 
\par 31 “Taip pat buvo pasakyta: ‘Kas atleidžia savo žmoną, teišduoda jai skyrybų raštą’. 
\par 32 O Aš jums sakau: kiekvienas, kuris atleidžia savo žmoną,­jei ne ištvirkavimo atveju,­skatina ją svetimauti; ir jeigu kas atleistąją veda­svetimauja”. 
\par 33 “Taip pat girdėjote, jog protėviams buvo pasakyta: ‘Neprisiek melagingai, bet ištesėk Viešpačiui savo priesaikas’. 
\par 34 O Aš jums sakau: iš viso neprisiekinėkite nei dangumi, nes jis­ Dievo sostas, 
\par 35 nei žeme, nes ji­Jo pakojis, nei Jeruzale, nes ji­didžiojo Karaliaus miestas. 
\par 36 Neprisiek nei savo galva, nes negali nė vieno plauko padaryti balto ar juodo. 
\par 37 Verčiau jūs sakykite: ‘Taip’, jei taip, ‘Ne’, jei ne, o kas viršaus, tai iš pikto”. 
\par 38 “Jūs girdėjote, jog buvo pasakyta: ‘Akis už akį’ ir ‘dantis už dantį’. 
\par 39 O Aš jums sakau: nesipriešinkite piktam, bet, jei kas tave muštų per dešinį skruostą, atsuk jam ir kitą. 
\par 40 Jei kas nori su tavimi bylinėtis ir paimti tavo tuniką, atiduok jam ir apsiaustą. 
\par 41 Jei kas verstų tave nueiti mylią, nueik su juo dvi. 
\par 42 Prašančiam duok ir nuo norinčio iš tavęs pasiskolinti nenusigręžk. 
\par 43 Jūs girdėjote, jog buvo pasakyta: ‘Mylėk savo artimą’ ir nekęsk savo priešo. 
\par 44 O Aš jums sakau: mylėkite savo priešus, laiminkite jus keikiančius, darykite gera tiems, kurie nekenčia jūsų, ir melskitės už savo skriaudėjus ir persekiotojus, 
\par 45 kad būtumėte vaikai savo Tėvo, kuris danguje; Jis juk leidžia savo saulei tekėti blogiesiems ir geriesiems, siunčia lietų ant teisiųjų ir neteisiųjų. 
\par 46 Jei mylite tuos, kurie jus myli, kokį gi atlygį turite? Argi taip nesielgia ir muitininkai? 
\par 47 Ir jeigu sveikinate tik savo brolius, kuo gi viršijate kitus? Argi to nedaro ir muitininkai? 
\par 48 Taigi būkite tobuli, kaip ir jūsų Tėvas, kuris danguje, yra tobulas”.



\chapter{6}


\par 1 “Žiūrėkite, jog nedarytumėte savo gailestingumo darbų žmonių akyse, kad būtumėte jų matomi, kitaip negausite atlygio iš savo Tėvo, kuris danguje. 
\par 2 Todėl, duodamas išmaldą, netrimituok sinagogose ir gatvėse, kaip daro veidmainiai, kad būtų žmonių giriami. Iš tiesų sakau jums: jie jau atsiėmė savo atlygį. 
\par 3 Kai aukoji, tenežino tavo kairė, ką daro dešinė, 
\par 4 kad tavo gailestingumo auka būtų slaptoje, o tavo Tėvas, regintis slaptoje, tau atlygins viešai”. 
\par 5 “Kai meldžiatės, nebūkite kaip veidmainiai, kurie mėgsta melstis, stovėdami sinagogose ir gatvių kampuose, kad būtų žmonių matomi. Iš tiesų sakau jums: jie jau atsiėmė savo atlygį. 
\par 6 Kai meldiesi, eik į savo kambarėlį ir, užsirakinęs duris, melskis savo Tėvui, kuris yra slaptoje, o tavo Tėvas, regintis slaptoje, tau atlygins viešai. 
\par 7 Melsdamiesi nedaugiažodžiaukite kaip pagonys: jie mano būsią išklausyti dėl žodžių gausumo. 
\par 8 Nebūkite panašūs į juos, nes jūsų Tėvas žino, ko jums reikia, dar prieš jums prašant Jo. 
\par 9 Todėl melskitės taip: ‘Tėve mūsų, kuris esi danguje, teesie šventas Tavo vardas, 
\par 10 teateinie Tavo karalystė, tebūnie Tavo valia kaip danguje, taip ir žemėje. 
\par 11 Kasdienės mūsų duonos duok mums šiandien 
\par 12 ir atleisk mums mūsų kaltes, kaip ir mes atleidžiame savo kaltininkams. 
\par 13 Ir nevesk mūsų į pagundymą, bet gelbėk mus nuo pikto; nes Tavo yra karalystė, jėga ir šlovė per amžius. Amen’. 
\par 14 Jeigu jūs atleisite žmonėms jų nusižengimus, tai ir jūsų dangiškasis Tėvas atleis jums, 
\par 15 o jeigu jūs neatleisite žmonėms jų nusižengimų, tai ir jūsų Tėvas neatleis jūsų nusižengimų”. 
\par 16 “Kai pasninkaujate, nebūkite paniurę kaip veidmainiai: jie perkreipia veidus, kad žmonės matytų juos pasninkaujant. Iš tiesų sakau jums: jie jau atsiėmė savo atlygį. 
\par 17 O tu, kai pasninkauji, pasitepk galvą ir nusiprausk veidą, 
\par 18 kad ne žmonėms rodytumeis pasninkaująs, bet savo Tėvui, kuris yra slaptoje. Ir tavo Tėvas, regintis slaptoje, tau atlygins viešai”. 
\par 19 “Nekraukite sau turtų žemėje, kur kandys ir rūdys ėda, kur vagys įsilaužia ir vagia. 
\par 20 Bet kraukite sau turtus danguje, kur nei kandys, nei rūdys neėda, kur vagys neįsilaužia ir nevagia, 
\par 21 nes kur tavo turtas, ten ir tavo širdis”. 
\par 22 “Kūno žiburys yra akis. Todėl, jei tavo akis sveika, visas tavo kūnas bus šviesus. 
\par 23 O jei tavo akis pikta, visas tavo kūnas bus tamsus. Taigi, jei tavyje esanti šviesa yra tamsa, tai kokia baisi toji tamsa!” 
\par 24 “Niekas negali tarnauti dviems šeimininkams: arba jis vieno nekęs, o kitą mylės, arba vienam bus atsidavęs, o kitą nieku vers. Negalite tarnauti Dievui ir Mamonai”. 
\par 25 “Todėl sakau jums: nesirūpinkite savo gyvybe, ką valgysite ar ką gersite, nei savo kūnu, kuo vilkėsite. Argi gyvybė ne daugiau už maistą ir kūnas už drabužį? 
\par 26 Pažvelkite į padangių paukščius: nei jie sėja, nei pjauna, nei į kluonus krauna, o jūsų dangiškasis Tėvas juos maitina. Argi jūs ne daug vertesni už juos? 
\par 27 O kas iš jūsų gali savo rūpesčiu bent per sprindį pridėti sau ūgio? 
\par 28 Ir kam gi rūpinatės drabužiu? Žiūrėkite, kaip auga lauko lelijos. Jos nesidarbuoja ir neverpia, 
\par 29 bet sakau jums: nė Saliamonas visoje savo šlovėje nebuvo taip pasipuošęs, kaip kiekviena iš jų. 
\par 30 Jeigu Dievas taip aprengia laukų žolę, kuri šiandien žaliuoja, o rytoj metama į krosnį, tai argi Jis dar labiau nepasirūpins jumis, mažatikiai? 
\par 31 Todėl nesirūpinkite ir neklausinėkite: ‘Ką valgysime?’, arba: ‘Ką gersime?’, arba: ‘Kuo vilkėsime?’ 
\par 32 Visų tų dalykų ieško pagonys. Jūsų dangiškasis Tėvas juk žino, kad viso to jums reikia. 
\par 33 Pirmiausia ieškokite Dievo karalystės ir Jo teisumo, o visa tai bus jums pridėta. 
\par 34 Taigi nesirūpinkite rytdiena, nes rytojus pats pasirūpins savimi. Kiekvienai dienai užtenka savo vargo”.



\chapter{7}


\par 1 “Neteiskite, kad nebūtumėte teisiami. 
\par 2 Kokiu teismu teisiate, tokiu ir patys būsite teisiami, ir kokiu saiku seikite, tokiu ir jums bus atseikėta. 
\par 3 Kodėl matai krislą savo brolio akyje, o nepastebi rąsto savojoje? 
\par 4 Arba kaip gali sakyti broliui: ‘Leisk, išimsiu krislą iš tavo akies’, kai tavo akyje rąstas?! 
\par 5 Veidmainy, pirmiau išritink rąstą iš savo akies, o paskui pažiūrėsi, kaip išimti krislelį iš savo brolio akies”. 
\par 6 “Neduokite to, kas šventa, šunims ir nebarstykite savo perlų kiaulėms, kad kartais kojomis jų nesutryptų ir atsigręžusios jūsų pačių nesudraskytų”. 
\par 7 “Prašykite, ir jums bus duota, ieškokite, ir rasite, belskite, ir bus jums atidaryta. 
\par 8 Kiekvienas, kas prašo, gauna, kas ieško, randa, ir beldžiančiam atidaroma. 
\par 9 Argi atsiras iš jūsų žmogus, kuris savo vaikui, prašančiam duonos, duotų akmenį?! 
\par 10 Arba jeigu jis prašytų žuvies, nejaugi duotų jam gyvatę? 
\par 11 Jei tad jūs, būdami blogi, mokate savo vaikams duoti gerų dalykų, tai juo labiau jūsų Tėvas, kuris yra danguje, duos gera tiems, kurie Jį prašo. 
\par 12 Tad visa, ko norite, kad jums darytų žmonės, ir jūs patys jiems darykite; nes tai Įstatymas ir Pranašai”. 
\par 13 “Įeikite pro ankštus vartus, nes erdvūs vartai ir platus kelias veda į pražūtį, ir daug yra juo įeinančių. 
\par 14 O ankšti vartai ir siauras kelias veda į gyvenimą, ir tik nedaugelis jį randa”. 
\par 15 “Saugokitės netikrų pranašų, kurie ateina pas jus avių kailyje, o viduje yra plėšrūs vilkai. 
\par 16 Jūs pažinsite juos iš vaisių. Argi kas gali priskinti vynuogių nuo erškėčių ar figų nuo usnių?! 
\par 17 Juk kiekvienas geras medis duoda gerus vaisius, o blogas medis­blogus. 
\par 18 Geras medis negali duoti blogų vaisių, o blogas­gerų. 
\par 19 Kiekvienas medis, kuris neduoda gerų vaisių, nukertamas ir įmetamas į ugnį. 
\par 20 Taigi jūs pažinsite juos iš vaisių”. 
\par 21 “Ne kiekvienas, kuris man sako: ‘Viešpatie, Viešpatie!’, įeis į dangaus karalystę, bet tas, kuris vykdo valią mano Tėvo, kuris yra danguje. 
\par 22 Daugelis man sakys aną dieną: ‘Viešpatie, Viešpatie, argi mes nepranašavome Tavo vardu, argi neišvarinėjome demonų Tavo vardu, argi nedarėme daugybės stebuklų Tavo vardu?!’ 
\par 23 Tada Aš jiems pareikšiu: ‘Aš niekuomet jūsų nepažinojau. Šalin nuo manęs, jūs piktadariai!’ 
\par 24 Taigi kiekvienas, kuris klauso šitų mano žodžių ir juos vykdo, panašus į išmintingą žmogų, pasistačiusį savo namą ant uolos. 
\par 25 Prapliupo liūtys, ištvino upės, pakilo vėjai ir daužėsi į tą namą. Tačiau jis nesugriuvo, nes buvo pastatytas ant uolos. 
\par 26 Ir kiekvienas, kuris klauso šitų mano žodžių ir jų nevykdo, panašus į kvailą žmogų, pasistačiusį savo namą ant smėlio. 
\par 27 Prapliupo liūtys, ištvino upės, pakilo vėjai ir daužėsi į tą namą, ir jis sugriuvo, o jo griuvimas buvo smarkus”. 
\par 28 Kai Jėzus baigė šiuos žodžius, minios stebėjosi Jo mokymu, 
\par 29 nes Jis mokė juos kaip turintis valdžią, o ne kaip Rašto žinovai.



\chapter{8}


\par 1 Kai Jėzus leidosi nuo kalno, Jį sekė didelės minios. 
\par 2 Ir štai priėjo raupsuotasis ir, pagarbinęs Jį, sakė: “Viešpatie, jei nori, Tu gali mane padaryti švarų”. 
\par 3 Jėzus ištiesė ranką, palietė jį ir tarė: “Noriu, būk švarus!” Ir tuojau raupsai išnyko. 
\par 4 Jėzus pasakė jam: “Žiūrėk, niekam nepasakok, bet eik pasirodyti kunigui ir paaukok Mozės įsakytą atnašą jiems paliudyti”. 
\par 5 Jėzui sugrįžus į Kafarnaumą, prie Jo priėjo šimtininkas, maldaudamas: 
\par 6 “Viešpatie, mano tarnas guli namie paralyžiuotas ir baisiai kankinasi”. 
\par 7 Jėzus jam tarė: “Einu ir išgydysiu jį”. 
\par 8 Šimtininkas atsakė: “Viešpatie, nesu vertas, kad užeitum po mano stogu, bet tik tark žodį, ir mano tarnas pasveiks. 
\par 9 Juk ir aš, būdamas valdinys, turiu sau pavaldžių kareivių. Sakau vienam: ‘Eik!’, ir jis eina; sakau kitam: ‘Ateik!’, ir jis ateina; sakau tarnui: ‘Daryk tai!’, ir jis daro”. 
\par 10 Tai girdėdamas, Jėzus stebėjosi ir kalbėjo einantiems iš paskos: “Iš tiesų sakau jums: net Izraelyje neradau tokio tikėjimo! 
\par 11 Todėl sakau jums: daugelis ateis iš rytų ir vakarų ir susės dangaus karalystėje su Abraomu, Izaoku ir Jokūbu. 
\par 12 O karalystės vaikai bus išmesti laukan į tamsybes. Ten bus verksmas ir dantų griežimas”. 
\par 13 Šimtininkui Jėzus tarė: “Eik, ir tebūnie tau, kaip tikėjai!” Ir tą pačią valandą tarnas pagijo. 
\par 14 Atėjęs į Petro namus, Jėzus pamatė jo uošvę gulinčią ir karščiuojančią. 
\par 15 Jis palietė jos ranką, ir karštis praėjo. Toji atsikėlė ir patarnavo jiems. 
\par 16 Vakarui atėjus, žmonės sugabeno pas Jėzų daug demonų apsėstųjų. Jis išvarė dvasias žodžiu ir išgydė visus ligonius, 
\par 17 kad išsipildytų, kas buvo pasakyta per pranašą Izaiją: “Jis pasiėmė mūsų negalias ir nešė mūsų ligas”. 
\par 18 Matydamas aplinkui didžiulę minią, Jėzus įsakė irtis į kitą krantą. 
\par 19 Tuomet priėjo vienas Rašto žinovas ir tarė Jam: “Mokytojau, aš seksiu paskui Tave, kur tik Tu eisi!” 
\par 20 Jėzus jam atsakė: “Lapės turi urvus, padangių paukščiai­lizdus, o Žmogaus Sūnus neturi kur galvos priglausti”. 
\par 21 Kitas Jo mokinys prašė: “Viešpatie, leisk man pirmiau pareiti tėvo palaidoti”. 
\par 22 Bet Jėzus atsakė: “Sek paskui mane ir palik mirusiems laidoti savo numirėlius”. 
\par 23 Jėzus įlipo į valtį, ir mokiniai paskui Jį. 
\par 24 Ir štai ežere pakilo smarki audra, ir bangos liejo valtį. O Jis miegojo. 
\par 25 Mokiniai pripuolę ėmė Jį žadinti, šaukdami: “Viešpatie, gelbėk mus, žūvame!” 
\par 26 Jis jiems tarė: “Kodėl jūs tokie bailūs, mažatikiai?” Paskui Jis atsikėlė, sudraudė vėjus bei ežerą, ir pasidarė visiškai ramu. 
\par 27 Žmonės stebėjosi ir kalbėjo: “Kas Jis per vienas, kad net vėjai ir ežeras Jo klauso?” 
\par 28 Kai Jėzus priplaukė kitą krantą gergeziečių krašte, Jam priešais atbėgo du demonų apsėstieji, išlindę iš kapinių rūsių. Juodu buvo tokie pavojingi, kad niekas negalėjo praeiti anuo keliu. 
\par 29 Ir štai jiedu ėmė šaukti: “Ko Tau iš mūsų reikia, Jėzau, Dievo Sūnau?! Atėjai pirma laiko mūsų kankinti?” 
\par 30 Toli nuo jų ganėsi didelė banda kiaulių. 
\par 31 Demonai prašė: “Jeigu mus išvarysi, tai leisk sueiti kiaulių bandon”. 
\par 32 Ir Jis jiems tarė: “Eikite!” Tuomet demonai išėjo ir apniko kiaules. Ir štai visa banda metėsi nuo skardžio į ežerą ir prigėrė vandenyje. 
\par 33 Piemenys pabėgo ir, pasiekę miestą, viską išpasakojo, taip pat ir apie apsėstuosius. 
\par 34 Tada visas miestas išėjo pasitikti Jėzaus ir, Jį pamatę, maldavo pasišalinti iš jų krašto.



\chapter{9}


\par 1 Įlipęs į valtį, Jėzus persikėlė per ežerą ir sugrįžo į savo miestą. 
\par 2 Ir štai Jam atnešė paralyžiuotą žmogų, paguldytą ant gulto. Pamatęs jų tikėjimą, Jėzus tarė paralyžiuotajam: “Būk drąsus, sūnau, tavo nuodėmės atleistos!” 
\par 3 Kai kurie Rašto žinovai ėmė murmėti: “Jis piktžodžiauja!” 
\par 4 Žinodamas jų mintis, Jėzus tarė: “Kam piktai mąstote savo širdyje? 
\par 5 Kas gi lengviau­ar pasakyti: ‘Tavo nuodėmės atleistos!’, ar liepti: ‘Kelkis ir vaikščiok!’? 
\par 6 Ir todėl, kad žinotumėte Žmogaus Sūnų turint galią žemėje atleisti nuodėmes,­čia Jis kreipėsi į paralyžiuotąjį:­Kelkis, pasiimk savo gultą ir eik namo!” 
\par 7 Šis atsikėlė ir nuėjo į savo namus. 
\par 8 Visa tai pamačiusios, minios stebėjosi ir šlovino Dievą, suteikusį tokią galią žmonėms. 
\par 9 Iškeliaudamas iš ten, Jėzus pamatė muitinėje sėdintį žmogų, vardu Matą, ir tarė jam: “Sek paskui mane!” Šis atsikėlė ir nusekė paskui Jį. 
\par 10 Kai Jėzus sėdėjo namuose prie stalo, ten susirinko daug muitininkų bei nusidėjėlių, kurie susėdo šalia Jo ir Jo mokinių. 
\par 11 Fariziejai, tai išvydę, sakė Jo mokiniams: “Kodėl jūsų Mokytojas valgo su muitininkais ir nusidėjėliais?” 
\par 12 Tai išgirdęs, Jėzus atsiliepė: “Ne sveikiesiems reikia gydytojo, o ligoniams. 
\par 13 Eikite ir pasimokykite, ką reiškia žodžiai: ‘Aš noriu gailestingumo, o ne aukos’. Aš atėjau šaukti ne teisiųjų, bet nusidėjėlių atgailai”. 
\par 14 Tada priėjo Jono mokiniai ir paklausė: “Kodėl mes ir fariziejai daug pasninkaujame, o Tavo mokiniai nepasninkauja?” 
\par 15 Jėzus jiems atsakė: “Argi gali vestuvininkai gedėti, kol su jais yra jaunikis? Bet ateis dienos, kai jaunikis iš jų bus atimtas, ir tada jie pasninkaus. 
\par 16 Niekas sudėvėto drabužio nelopo naujo audinio lopu, nes toks lopinys atplėšia drabužio gabalą, ir pasidaro dar didesnė skylė. 
\par 17 Taip pat niekas nepila jauno vyno į senus vynmaišius, nes antraip vynmaišiai plyštų, vynas išsilietų ir vynmaišiai niekais nueitų. Bet jaunas vynas pilamas į naujus vynmaišius, ir abeji išsilaiko”. 
\par 18 Jam taip bekalbant, prisiartino vienas vyresnysis, pagarbino Jį ir tarė: “Ką tik mirė mano dukrelė. Bet ateik, uždėk ant jos ranką, ir ji atgis”. 
\par 19 Jėzus atsikėlė ir nuėjo paskui jį kartu su savo mokiniais. 
\par 20 Ir štai moteris, dvylika metų serganti kraujoplūdžiu, prisiartino iš paskos ir palietė Jo apsiausto apvadą. 
\par 21 Mat ji pati sau kalbėjo: “Jei tik palytėsiu Jo drabužį­išgysiu”. 
\par 22 Jėzus, atsigręžęs ir ją pamatęs, tarė: “Pasitikėk, dukra, tavo tikėjimas išgydė tave”. Ir tą pačią akimirką moteris pagijo. 
\par 23 Atėjęs į vyresniojo namus ir pamatęs vamzdininkus bei šurmuliuojančią minią, 
\par 24 Jėzus paliepė: “Pasitraukite, nes mergaitė ne mirus, o miega”. Jie tik šaipėsi iš Jo. 
\par 25 Kai minia buvo išvaryta, Jis įėjo vidun, paėmė mergaitę už rankos, ir ji atsikėlė. 
\par 26 Garsas apie tai pasklido po visą aną kraštą. 
\par 27 Jėzui išeinant, du neregiai sekė paskui Jį ir šaukė: “Pasigailėk mūsų, Dovydo Sūnau!” 
\par 28 Kai Jis pasiekė namus, neregiai užėjo pas Jį. Jėzus paklausė: “Ar tikite, kad Aš galiu tai padaryti?” Šie atsakė: “Taip, Viešpatie!” 
\par 29 Tada Jis palietė jų akis ir tarė: “Tebūna jums pagal jūsų tikėjimą”. 
\par 30 Ir jų akys atsivėrė. Jėzus griežtai jiems įsakė: “Žiūrėkite, kad niekas nesužinotų!” 
\par 31 Tačiau tie išėję išgarsino Jį po visą tą kraštą. 
\par 32 Jiems išėjus, štai atvedė pas Jį demono apsėstą nebylį. 
\par 33 Išvarius demoną, nebylys prakalbo. Minios stebėjosi ir sakė: “Dar niekad Izraelyje nebuvo tokių dalykų”. 
\par 34 O fariziejai kalbėjo: “Jis išvaro demonus jų valdovo jėga”. 
\par 35 Jėzus ėjo per visus miestus ir kaimus, mokydamas jų sinagogose, skelbdamas karalystės Evangeliją ir gydydamas visas žmonių ligas bei negalias. 
\par 36 Matydamas minias, Jis gailėjosi žmonių, nes jie buvo suvargę ir išsklaidyti lyg avys be piemens. 
\par 37 Tuomet Jis tarė savo mokiniams: “Pjūtis didelė, o darbininkų maža. 
\par 38 Todėl melskite pjūties Viešpatį, kad siųstų darbininkų į savo pjūtį”.



\chapter{10}


\par 1 Pasišaukęs dvylika savo mokinių, Jis suteikė jiems valdžią netyrosioms dvasioms, kad išvarinėtų jas ir gydytų visas ligas bei negalias. 
\par 2 Dvylikos apaštalų vardai: pirmasis Simonas, vadinamas Petru, ir jo brolis Andriejus; Zebediejaus sūnus Jokūbas ir jo brolis Jonas; 
\par 3 Pilypas ir Baltramiejus; Tomas ir muitininkas Matas; Alfiejaus sūnus Jokūbas ir Lebėjus, vadinamas Tadu; 
\par 4 Simonas Kananietis ir Judas Iskarijotas, kuris išdavė Jėzų. 
\par 5 Šiuos dvylika Jėzus išsiuntė, įsakydamas jiems: “Nepasukite keliu pas pagonis ir neužeikite į samariečių miestą, 
\par 6 bet verčiau eikite pas pražuvusias Izraelio namų avis. 
\par 7 Keliaudami skelbkite: ‘Prisiartino dangaus karalystė!’ 
\par 8 Gydykite ligonius, apvalykite raupsuotuosius, prikelkite mirusius, išvarinėkite demonus. Dovanai gavote, dovanai ir duokite! 
\par 9 Neimkite nei aukso, nei sidabro, nei variokų į savo juostas; 
\par 10 nei kelionmaišio, nei dviejų tunikų, nei sandalų, nei lazdos, nes darbininkas vertas savo valgio. 
\par 11 Atėję į kokį nors miestą ar kaimą, susiieškokite vertą žmogų ir apsistokite pas jį, kol išvyksite. 
\par 12 Įeidami į namus, pasveikinkite juos. 
\par 13 Ir jeigu namai bus verti, teateinie jiems jūsų ramybė. O jeigu nebus verti­jūsų ramybė tesugrįžta jums. 
\par 14 Jei kas jūsų nepriimtų ir neklausytų jūsų žodžių, tai, išėję iš tokių namų ar tokio miesto, nusikratykite ir dulkes nuo kojų. 
\par 15 Iš tiesų sakau jums: Sodomos ir Gomoros žemei bus lengviau teismo dieną negu tokiam miestui”. 
\par 16 “Štai Aš siunčiu jus kaip avis tarp vilkų. Todėl būkite gudrūs kaip žalčiai ir neklastingi kaip balandžiai. 
\par 17 Saugokitės žmonių, nes jie įskųs jus teismams ir plaks savo sinagogose. 
\par 18 Jūs būsite dėl manęs vedžiojami pas valdytojus ir karalius liudyti jiems ir pagonims. 
\par 19 Kai jie jus įskųs, nesirūpinkite, kaip arba ką kalbėsite, nes tą valandą jums bus duota, ką jūs turite sakyti. 
\par 20 Nes ne jūs kalbėsite, o jūsų Tėvo Dvasia kalbės jumyse. 
\par 21 Brolis išduos mirti brolį ir tėvas­sūnų, o vaikai sukils prieš gimdytojus ir žudys juos. 
\par 22 Jūs būsite visų nekenčiami dėl mano vardo. Bet kas ištvers iki galo, bus išgelbėtas. 
\par 23 Kai jus persekios viename mieste, bėkite į kitą. Iš tiesų sakau jums: dar nebūsite išvaikščioję Izraelio miestų, kai ateis Žmogaus Sūnus. 
\par 24 Mokinys nėra aukštesnis už savo mokytoją nei tarnas už šeimininką. 
\par 25 Pakanka, jei mokinys prilygsta mokytojui ir tarnas šeimininkui. Jei namų šeimininką jie praminė Belzebulu, tai ką bekalbėti apie namiškius!” 
\par 26 “Taigi nebijokite jų. Nes nieko nėra uždengta, kas nebus atidengta, ir nieko paslėpta, kas nepasidarys žinoma. 
\par 27 Ką jums kalbu tamsoje, sakykite šviesoje, ir ką šnibždu į ausį, skelbkite nuo stogų. 
\par 28 Nebijokite tų, kurie žudo kūną, bet negali užmušti sielos. Verčiau bijokite to, kuris gali pražudyti ir sielą, ir kūną pragare. 
\par 29 Argi ne du žvirbliai parduodami už skatiką? Ir nė vienas iš jų nekrinta žemėn be jūsų Tėvo valios. 
\par 30 O jūsų net visi galvos plaukai suskaičiuoti. 
\par 31 Tad nebijokite! Jūs vertesni už daugybę žvirblių. 
\par 32 Kas išpažins mane žmonių akivaizdoje, ir Aš jį išpažinsiu savo dangiškojo Tėvo akivaizdoje. 
\par 33 O kas išsižadės manęs žmonių akivaizdoje, ir Aš jo išsižadėsiu savo dangiškojo Tėvo akivaizdoje”. 
\par 34 “Nemanykite, jog Aš atėjau atnešti žemėn ramybės. Atėjau atnešti ne ramybės, o kalavijo. 
\par 35 Atėjau sukiršinti ‘sūnaus prieš tėvą, dukters prieš motiną ir marčios prieš anytą. 
\par 36 Žmogaus namiškiai taps jam priešais’. 
\par 37 Kas myli tėvą ar motiną labiau negu mane­nevertas manęs. Kas myli sūnų ar dukterį labiau negu mane­nevertas manęs. 
\par 38 Kas neima savo kryžiaus ir neseka paskui mane, tas nevertas manęs. 
\par 39 Kas išsaugo savo gyvybę, praras ją, o kas praranda savo gyvybę dėl manęs­atras ją. 
\par 40 Kas jus priima, tas mane priima. O kas priima mane, priima Tą, kuris mane siuntė. 
\par 41 Kas priima pranašą dėl to, kad jis pranašas, gaus pranašo atlygį. Kas priima teisųjį dėl to, kad jis teisusis, gaus teisiojo atlygį. 
\par 42 Ir kas paduos bent taurę šalto vandens vienam iš šitų mažųjų dėl to, kad jis yra mokinys,­iš tiesų sakau jums,­tas nepraras savo atlygio”.



\chapter{11}


\par 1 Baigęs nurodymus dvylikai savo mokinių, Jėzus iškeliavo toliau mokyti ir pamokslauti kituose miestuose. 
\par 2 Jonas, išgirdęs kalėjime apie Kristaus darbus, nusiuntė du savo mokinius 
\par 3 Jo paklausti: “Ar Tu esi Tas, kuris turi ateiti, ar mums laukti kito?” 
\par 4 Jėzus jiems atsakė: “Eikite ir pasakykite Jonui, ką girdite ir matote: 
\par 5 aklieji praregi, luošieji vaikščioja, raupsuotieji apvalomi, kurtieji girdi, mirusieji prikeliami, vargšams skelbiama Evangelija. 
\par 6 Ir palaimintas, kas nepasipiktina manimi”. 
\par 7 Jiems nueinant, Jėzus ėmė kalbėti minioms apie Joną: “Ko išėjote į dykumą pažiūrėti? Ar vėjo linguojamos nendrės? 
\par 8 Ko gi išėjote pamatyti? Ar švelniais drabužiais vilkinčio žmogaus? Švelniais drabužiais vilkintys gyvena karaliaus rūmuose. 
\par 9 Tai ko gi išėjote pamatyti? Ar pranašo? Taip, sakau jums, ir kur kas daugiau negu pranašo! 
\par 10 Jis yra tas, apie kurį parašyta: ‘Štai Aš siunčiu pirma Tavęs savo pasiuntinį, kuris nuties prieš Tave kelią’. 
\par 11 Iš tiesų sakau jums: tarp gimusių iš moterų nepakilo didesnis už Joną Krikštytoją, bet ir mažiausias dangaus karalystėje didesnis už jį. 
\par 12 Nuo Jono Krikštytojo dienų iki dabar dangaus karalystė grobiama, ir stiprieji ją jėga ima. 
\par 13 Visi pranašai ir Įstatymas pranašavo iki Jono 
\par 14 ir, jeigu norite priimti, tai jis ir yra Elijas, kuris turi ateiti. 
\par 15 Kas turi ausis klausyti­teklauso!” 
\par 16 “Su kuo galėčiau palyginti šią kartą? Ji panaši į vaikus, kurie sėdi prekyvietėje ir šaukia savo draugams: 
\par 17 ‘Mes jums grojome, o jūs nešokote. Mes giedojome raudas, o jūs neraudojote’. 
\par 18 Atėjo Jonas, nevalgus ir negeriantis, tai jie sako: ‘Jis demono apsėstas’. 
\par 19 Atėjo Žmogaus Sūnus, valgantis ir geriantis, tai jie sako: ‘Štai rijūnas ir vyno gėrėjas, muitininkų ir nusidėjėlių bičiulis’. Bet išmintį pateisina jos vaikai”. 
\par 20 Tada Jis pradėjo priekaištauti miestams, kuriuose buvo padaryta daugumas Jo stebuklų, kad jie neatgailavo: 
\par 21 “Vargas tau, Chorazine! Vargas tau, Betsaida! Jeigu Tyre ir Sidone būtų įvykę tokių stebuklų, kokie padaryti pas jus, jie seniai būtų atgailavę su ašutine bei pelenuose. 
\par 22 Todėl sakau jums: Tyrui ir Sidonui bus lengviau teismo dieną negu jums! 
\par 23 Ir tu, Kafarnaume, išaukštintas iki dangaus, nugarmėsi iki pragaro! Jeigu Sodomoje būtų įvykę tokių stebuklų, kokių įvyko tavyje, ji būtų išlikusi iki šios dienos. 
\par 24 Todėl sakau jums: Sodomos žemei bus lengviau teismo dieną negu tau”. 
\par 25 Anuo metu Jėzus kalbėjo: “Aš šlovinu Tave, Tėve, dangaus ir žemės Viešpatie, kad paslėpei tai nuo išmintingųjų ir gudriųjų, o apreiškei mažutėliams. 
\par 26 Taip, Tėve, nes Tau taip patiko. 
\par 27 Viskas man yra mano Tėvo atiduota; ir niekas nepažįsta Sūnaus, tik Tėvas, nei Tėvo niekas nepažįsta, tik Sūnus ir kam Sūnus nori apreikšti. 
\par 28 Ateikite pas mane visi, kurie vargstate ir esate prislėgti, ir Aš jus atgaivinsiu. 
\par 29 Imkite ant savęs mano jungą ir mokykitės iš manęs, nes Aš romus ir nuolankios širdies, ir jūs rasite savo sieloms atgaivą. 
\par 30 Nes mano jungas švelnus ir mano našta lengva”.



\chapter{12}


\par 1 Anuo metu sabato dieną Jėzus ėjo per javų lauką. Jo mokiniai buvo išalkę, tad skynė varpas ir valgė. 
\par 2 Tai pamatę, fariziejai Jam sakė: “Žiūrėk, Tavo mokiniai daro, kas per sabatą draudžiama”. 
\par 3 Jis jiems atsakė: “Ar neskaitėte, ką darė Dovydas ir jo palydovai, būdami išalkę? 
\par 4 Kaip jie įėjo į Dievo namus ir valgė padėtinės duonos, nors nevalia buvo jos valgyti nei jam, nei jo palydovams, o vien tik kunigams. 
\par 5 Arba, ar neskaitėte Įstatyme, jog per sabatą kunigai šventykloje pažeidžia sabatą ir nenusikalsta? 
\par 6 Bet sakau jums: čia daugiau negu šventykla! 
\par 7 Jei būtumėte supratę, ką reiškia ‘Aš noriu gailestingumo, o ne aukos’, nebūtumėte pasmerkę nekaltų. 
\par 8 Žmogaus Sūnus yra ir sabato Viešpats”. 
\par 9 Iš ten išėjęs, Jis atėjo į jų sinagogą. 
\par 10 Ir štai ten buvo žmogus padžiūvusia ranka. Jie paklausė Jėzų (kad galėtų apkaltinti): “Ar leistina sabato dieną gydyti?” 
\par 11 Jis jiems atsakė: “Kas iš jūsų, turėdamas vieną avį, jeigu ji per sabatą įkris į duobę, nestvers ir neištrauks? 
\par 12 O kaip daug brangesnis už avį žmogus! Todėl leistina daryti gera sabato dieną”. 
\par 13 Tada Jis tarė žmogui: “Ištiesk ranką!” Tas ištiesė, ir ji tapo sveika kaip ir antroji. 
\par 14 Tuomet išėję fariziejai tarėsi, kaip Jėzų pražudyti. 
\par 15 Tai sužinojęs, Jėzus pasitraukė iš ten. Didelės minios sekė paskui Jį, ir Jis visus išgydė, 
\par 16 bet įspėjo, kad Jo negarsintų. 
\par 17 Kad išsipildytų, kas buvo pasakyta per pranašą Izaiją: 
\par 18 “Štai mano išrinktasis tarnas, mano mylimasis, kuriuo gėrisi mano siela. Aš duosiu Jam savo Dvasią, ir Jis skelbs pagonims teisingumą. 
\par 19 Jis nesiginčys, nešauks, ir niekas gatvėje negirdės Jo balso. 
\par 20 Jis nenulauš palūžusios nendrės ir neužgesins rusenančio dagčio, kol nenuves teisingumo į pergalę; 
\par 21 o Jo vardas teiks viltį pagonims!” 
\par 22 Tuomet atvedė pas Jį demono apsėstąjį, kuris buvo aklas ir nebylys. Jėzus išgydė jį, ir šis prakalbo ir praregėjo. 
\par 23 Ištisos minios netvėrė iš nuostabos ir klausinėjo: “Ar nebus šitas Dovydo Sūnus?!” 
\par 24 Tai išgirdę, fariziejai sakė: “Jis išvaro demonus ne kitaip, kaip tik demonų valdovo Belzebulo jėga”. 
\par 25 Žinodamas jų mintis, Jėzus tarė: “Kiekviena suskilusi karalystė bus nusiaubta, ir joks suskilęs miestas ir namas neišsilaikys. 
\par 26 Jeigu tad šėtonas išvarinėtų šėtoną, irgi būtų savyje susiskaldęs. Kaipgi tada galėtų išsilaikyti jo karalystė? 
\par 27 Ir jeigu Aš išvarau demonus Belzebulo jėga, tai kieno jėga išvaro jūsų sūnūs? Todėl jie bus jūsų teisėjai. 
\par 28 Bet jeigu Aš išvarau demonus Dievo Dvasia, tai tikrai pas jus atėjo Dievo karalystė. 
\par 29 Argi gali kas nors įeiti į galiūno namus ir pasigrobti jo turtą, pirmiau nesurišęs galiūno? Tik tada jis apiplėš jo namus. 
\par 30 Kas ne su manimi, tas prieš mane, ir kas su manimi nerenka, tas barsto”. 
\par 31 “Sakau jums: kiekviena nuodėmė ir piktžodžiavimas bus žmonėms atleisti, bet piktžodžiavimas Dvasiai nebus jiems atleistas. 
\par 32 Jei kas tartų žodį prieš Žmogaus Sūnų, tam bus atleista, o kas kalbėtų prieš Šventąją Dvasią, tam nebus atleista nei šiame, nei būsimajame amžiuje. 
\par 33 Arba sakykite medį esant gerą ir jo vaisių gerą, arba sakykite medį esant blogą ir jo vaisių blogą, nes medis pažįstamas iš vaisių. 
\par 34 Angių išperos, kaip jūs galite kalbėti gera, būdami blogi?! Juk lūpos kalba tai, ko pertekusi širdis. 
\par 35 Geras žmogus iš gero širdies lobyno iškelia gera, o blogas iš blogo lobyno iškelia bloga. 
\par 36 Todėl sakau jums: teismo dieną žmonės turės duoti apyskaitą už kiekvieną pasakytą tuščią žodį. 
\par 37 Pagal savo žodžius būsi išteisintas ir pagal savo žodžius būsi pasmerktas”. 
\par 38 Tada kai kurie Rašto žinovai ir fariziejai sakė: “Mokytojau, norime, kad parodytum ženklą”. 
\par 39 Jis jiems atsakė: “Pikta ir svetimaujanti karta ieško ženklo, bet nebus jai duota kito ženklo, kaip tik pranašo Jonos ženklas. 
\par 40 Kaip Jona išbuvo tris dienas ir tris naktis banginio pilve, taip ir Žmogaus Sūnus išbus tris dienas ir tris naktis žemės širdyje. 
\par 41 Ninevės žmonės teismo dieną prisikels drauge su šia karta ir ją pasmerks, nes jie atgailavo, išgirdę Jonos pamokslą, o štai čia daugiau negu Jona. 
\par 42 Pietų šalies karalienė teismo dieną prisikels drauge su šia karta ir ją pasmerks, nes ji atkeliavo nuo žemės pakraščių pasiklausyti Saliamono išminties, o štai čia daugiau negu Saliamonas”. 
\par 43 “Netyroji dvasia, išėjusi iš žmogaus, klaidžioja bevandenėse vietose, ieškodama poilsio, ir neranda. 
\par 44 Tada ji sako: ‘Grįšiu į savo namus, iš kur išėjau’. Sugrįžusi randa juos tuščius, iššluotus ir išpuoštus. 
\par 45 Tada eina, pasiima kitas septynias dvasias, dar piktesnes už save, ir įėjusios jos ten apsigyvena. Ir paskui tam žmogui darosi blogiau negu pirma. Taip atsitiks ir šiai piktai kartai”. 
\par 46 Jam tebekalbant minioms, štai Jo motina ir broliai stovėjo lauke ir norėjo su Juo pasikalbėti. 
\par 47 Tada kažkas pranešė Jam: “Štai Tavo motina ir broliai stovi lauke ir nori su Tavim pasikalbėti”. 
\par 48 Jis atsakė pranešusiam: “Kas yra mano motina ir kas yra mano broliai?” 
\par 49 Ir, ištiesęs ranką į savo mokinius, tarė: “Štai mano motina ir mano broliai! 
\par 50 Kiekvienas, kas vykdo mano dangiškojo Tėvo valią, yra mano brolis, ir sesuo, ir motina”.



\chapter{13}


\par 1 Tą dieną, išėjęs iš namų, Jėzus atsisėdo ant ežero kranto. 
\par 2 Prie Jo susirinko didžiulė minia; todėl Jis įlipo į valtį ir atsisėdo, o žmonės stovėjo pakrantėje. 
\par 3 Jis daug jiems kalbėjo palyginimais: “Štai sėjėjas išėjo sėti. 
\par 4 Jam besėjant, vieni grūdai nukrito palei kelią, ir atskridę paukščiai juos sulesė. 
\par 5 Kiti nukrito uolėtoj vietoj, kur buvo nedaug žemės. Jie greit sudygo, nes neturėjo gilesnio žemės sluoksnio. 
\par 6 Saulei patekėjus, daigai išdegė ir, neturėdami šaknų, sudžiūvo. 
\par 7 Kiti nukrito tarp erškėčių. Erškėčiai išaugo ir nusmelkė juos. 
\par 8 Dar kiti nukrito į gerą žemę ir davė derlių: vieni šimteriopą, kiti šešiasdešimteriopą, dar kiti trisdešimteriopą. 
\par 9 Kas turi ausis klausyti­teklauso!” 
\par 10 Priėję mokiniai paklausė Jo: “Kodėl jiems kalbi palyginimais?” 
\par 11 Jėzus atsakė: “Jums duota pažinti dangaus karalystės paslaptis, o jiems neduota. 
\par 12 Mat, kas turi, tam bus duota, ir jis turės su pertekliumi, o iš neturinčio bus atimta ir tai, ką jis turi. 
\par 13 Aš jiems kalbu palyginimais todėl, kad jie žiūrėdami nemato, klausydami negirdi ir nesupranta. 
\par 14 Jiems pildosi Izaijo pranašystės žodžiai: ‘Girdėti girdėsite, bet nesuprasite, žiūrėti žiūrėsite, bet nematysite. 
\par 15 Šitų žmonių širdys aptuko. Jie prastai girdėjo ausimis ir užmerkė akis, kad nepamatytų akimis, neišgirstų ausimis, nesuprastų širdimi ir neatsiverstų, ir Aš jų nepagydyčiau’. 
\par 16 Bet palaimintos jūsų akys, nes mato, ir jūsų ausys, nes girdi. 
\par 17 Iš tiesų sakau jums: daugelis pranašų ir teisiųjų troško išvysti, ką jūs matote, bet neišvydo, ir girdėti, ką jūs girdite, bet neišgirdo”. 
\par 18 “Tad pasiklausykite palyginimo apie sėjėją. 
\par 19 Pas kiekvieną, kuris girdi karalystės žodį ir nesupranta, ateina piktasis ir išplėšia, kas buvo pasėta jo širdyje. Tai yra pasėlis prie kelio. 
\par 20 Pasėlis uolėtoje vietoje­tai tas, kuris, girdėdamas žodį, tuojau su džiaugsmu jį priima. 
\par 21 Tačiau jis be šaknų­nepastovus žmogus. Kilus kokiam sunkumui ar persekiojimui dėl žodžio, jis tuoj pat pasipiktina. 
\par 22 Pasėlis tarp erškėčių­tai tas, kuris klauso žodžio, bet šio pasaulio rūpesčiai ir turtų apgaulė nustelbia žodį, ir jis lieka nevaisingas. 
\par 23 O pasėlis geroje žemėje­tas, kuris girdi ir supranta žodį; tas ir neša vaisių: kas duoda šimteriopą, kas šešiasdešimteriopą, o kas trisdešimteriopą”. 
\par 24 Jis pateikė jiems kitą palyginimą: “Su dangaus karalyste yra kaip su žmogumi, kuris pasėjo dirvoje gerą sėklą. 
\par 25 Žmonėms bemiegant, atėjo jo priešas, pasėjo kviečiuose raugių ir nuėjo. 
\par 26 Kai želmuo paūgėjo ir subrandino vaisių, pasirodė ir raugės. 
\par 27 Šeimininko tarnai atėję klausė: ‘Šeimininke, argi ne gerą sėklą pasėjai savo lauke? Iš kurgi atsirado raugių?’ 
\par 28 Jis atsakė: ‘Tai padarė priešas’. Tarnai pasiūlė: ‘Jei nori, eisime ir jas išravėsime’. 
\par 29 Jis atsakė: ‘Ne, kad kartais, ravėdami rauges, neišrautumėte kartu su jomis ir kviečių. 
\par 30 Palikite abejus augti iki pjūties. Pjūties metu pasakysiu pjovėjams: ‘Pirmiau išrinkite rauges ir suriškite į pėdelius sudeginti, o kviečius sukraukite į mano kluoną’ ”. 
\par 31 Jis pateikė jiems dar vieną palyginimą: “Dangaus karalystė yra kaip garstyčios grūdelis, kurį žmogus ėmė ir pasėjo savo dirvoje. 
\par 32 Nors jis mažiausias iš visų sėklų, bet užaugęs būna didesnis už visus augalus ir tampa medeliu; net padangių paukščiai atskridę susisuka lizdus jo šakose”. 
\par 33 Jis pasakė ir dar kitą palyginimą: “Dangaus karalystė yra kaip raugas, kurį moteris įmaišė trijuose saikuose miltų, ir nuo jo viskas įrūgo”. 
\par 34 Visa tai Jėzus kalbėjo minioms palyginimais, ir be palyginimų Jis jiems nekalbėjo, 
\par 35 kad išsipildytų, kas buvo per pranašą pasakyta: “Aš atversiu savo burną palyginimais, skelbsiu nuo pasaulio sukūrimo paslėptus dalykus”. 
\par 36 Paleidęs minias, Jėzus parėjo namo. Prie Jo priėjo mokiniai ir prašė: “Išaiškink mums palyginimą apie rauges dirvoje”. 
\par 37 Jis jiems atsakė: “Sėjantysis gerą sėklą yra Žmogaus Sūnus. 
\par 38 Dirva­tai pasaulis. Gera sėkla­karalystės vaikai, o raugės­ piktojo vaikai. 
\par 39 Jas pasėjęs priešas­velnias. Pjūtis­pasaulio pabaiga, o pjovėjai­ angelai. 
\par 40 Taigi, kaip surenkamos ir sudeginamos ugnyje raugės, taip bus ir pasaulio pabaigoje. 
\par 41 Žmogaus Sūnus išsiųs savo angelus, tie išrankios iš Jo karalystės visus papiktinimus bei piktadarius 
\par 42 ir įmes juos į ugnies krosnį. Ten bus verksmas ir dantų griežimas. 
\par 43 Tada teisieji spindės kaip saulė savo Tėvo karalystėje. Kas turi ausis klausyti­teklauso!” 
\par 44 “Dangaus karalystė yra kaip dirvoje paslėptas lobis. Atradęs jį, žmogus tai nuslepia; iš to džiaugsmo eina, parduoda visa, ką turi, ir perka tą dirvą. 
\par 45 Vėl su dangaus karalyste yra kaip su pirkliu, ieškančiu gerų perlų. 
\par 46 Atradęs vieną brangų perlą, jis eina, parduoda visa, ką turi, ir nusiperka jį”. 
\par 47 “Ir vėl su dangaus karalyste yra kaip su jūron metamu tinklu, užgriebiančiu įvairiausių žuvų. 
\par 48 Kai jis pilnas, jį išvelka į krantą, susėda ir surenka gerąsias į indus, o blogąsias išmeta. 
\par 49 Taip bus ir pasaulio pabaigoje: išeis angelai, išrankios bloguosius iš gerųjų 
\par 50 ir įmes juos į ugnies krosnį. Ten bus verksmas ir dantų griežimas”. 
\par 51 Jėzus paklausė jų: “Ar supratote visa tai?” Jie atsakė: “Taip, Viešpatie”. 
\par 52 Tada Jis jiems tarė: “Todėl kiekvienas Rašto žinovas, tapęs dangaus karalystės mokiniu, panašus į šeimininką, kuris iškelia iš savo lobyno naujų ir senų daiktų”. 
\par 53 Baigęs sakyti tuos palyginimus, Jėzus iškeliavo iš ten. 
\par 54 Jis parėjo į savo tėviškę ir mokė žmones jų sinagogoje taip, kad jie stebėjosi ir klausinėjo: “Iš kur šitam tokia išmintis ir stebuklingi darbai? 
\par 55 Argi Jis ne dailidės sūnus?! Argi Jo motina nesivadina Marija, o Jokūbas, Jozė, Simonas ir Judas argi ne Jo broliai? 
\par 56 Ir Jo seserys­argi jos ne visos yra pas mus? Iš kur Jam visa tai?” 
\par 57 Ir jie ėmė piktintis Juo. O Jėzus jiems atsakė: “Pranašas nebūna be pagarbos, nebent savo tėviškėje ir savo namuose”. 
\par 58 Ir Jis ten nedarė daug stebuklų dėl jų netikėjimo.



\chapter{14}


\par 1 Anuo metu garsas apie Jėzų pasiekė tetrarchą Erodą, 
\par 2 ir jis savo tarnams pasakė: “Tai Jonas Krikštytojas! Jis prisikėlė iš numirusių, ir todėl jame veikia stebuklingos jėgos”. 
\par 3 Mat Erodas buvo suėmęs Joną, sukaustęs jį ir įmetęs į kalėjimą dėl savo brolio Pilypo žmonos Erodiados. 
\par 4 Nes Jonas jam sakė: “Tau nevalia jos turėti”. 
\par 5 Erodas norėjo nužudyti Joną, bet bijojo žmonių, nes jie laikė jį pranašu. 
\par 6 Erodo gimimo dieną Erodiados duktė šoko svečiams ir patiko Erodui. 
\par 7 Todėl jis su priesaika pažadėjo jai duoti, ko tik ji paprašys. 
\par 8 O ši, savo motinos primokyta, tarė: “Duok man čia dubenyje Jono Krikštytojo galvą”. 
\par 9 Karalius nuliūdo, bet dėl priesaikos ir svečių įsakė duoti. 
\par 10 Jis pasiuntė nukirsti kalėjime Jonui galvą. 
\par 11 Jo galva buvo atnešta dubenyje ir įteikta mergaitei, kuri ją nunešė motinai. 
\par 12 Jono mokiniai atėję pasiėmė kūną, palaidojo ir nuėję pranešė Jėzui. 
\par 13 Tai išgirdęs, Jėzus valtimi nuplaukė į dykvietę, į vienumą. Minios sužinojo ir iš miestų pėsčiomis nusekė paskui. 
\par 14 Išlipęs Jėzus pamatė daugybę žmonių. Jėzui pagailo jų, ir Jis išgydė jų ligonius. 
\par 15 Atėjus vakarui, priėjo mokiniai ir tarė: “Čia dykvietė, ir jau vėlus metas. Paleisk žmones, kad, nuėję į kaimus, nusipirktų maisto”. 
\par 16 Bet Jėzus jiems atsakė: “Nėra reikalo jiems iš čia eiti. Jūs duokite jiems valgyti”. 
\par 17 Jie atsiliepė: “Mes čia turime tik penkis kepalus duonos ir dvi žuvis”. 
\par 18 Jis tarė: “Atneškite juos man”. 
\par 19 Ir, liepęs miniai susėsti ant žolės, Jis paėmė penkis duonos kepalus ir dvi žuvis, pažvelgė į dangų, palaimino, laužė ir davė kepalus mokiniams, o tie dalijo žmonėms. 
\par 20 Visi valgė ir pasisotino. Ir surinko dvylika pilnų pintinių likusių trupinių. 
\par 21 O valgytojų buvo apie penkis tūkstančius vyrų, neskaičiuojant moterų ir vaikų. 
\par 22 Tuojau pat Jėzus privertė savo mokinius sėsti į valtį ir pirma Jo plaukti į kitą ežero pusę, kol Jis paleisiąs minią. 
\par 23 Paleidęs minią, Jis užkopė nuošaliai į kalną melstis. Atėjus vakarui, Jis buvo ten vienas. 
\par 24 Tuo tarpu valtis jau buvo ežero viduryje, blaškoma bangų, nes pūtė priešingas vėjas. 
\par 25 Ketvirtos nakties sargybos metu Jėzus atėjo pas juos, žengdamas ežero paviršiumi. 
\par 26 Pamatę Jį einantį ežero paviršiumi, mokiniai nusigando ir iš baimės ėmė šaukti: “Tai šmėkla!” 
\par 27 Jėzus tuojau juos prakalbino: “Drąsos! Tai Aš. Nebijokite!” 
\par 28 Petras atsiliepė: “Viešpatie, jei čia Tu, liepk man ateiti pas Tave vandeniu”. 
\par 29 Jis atsakė: “Ateik!” Petras, išlipęs iš valties, ėjo vandeniu, norėdamas ateiti pas Jėzų. 
\par 30 Bet, pamatęs vėjo smarkumą, jis išsigando ir, pradėjęs skęsti, sušuko: “Viešpatie, gelbėk mane!” 
\par 31 Tuojau ištiesęs ranką, Jėzus sugriebė jį ir tarė: “Mažatiki, ko suabejojai?” 
\par 32 Jiems įlipus į valtį, vėjas nurimo. 
\par 33 Tie, kurie buvo valtyje, prisiartinę pagarbino Jį, sakydami: “Tikrai Tu esi Dievo Sūnus!” 
\par 34 Perplaukę jie išlipo į krantą Genezarete. 
\par 35 Pažinę Jį, tos vietos gyventojai pasiuntė į visas to krašto apylinkes ir sugabeno pas Jį visus sergančius. 
\par 36 Jie maldavo Jį leisti palytėti nors Jo apsiausto apvadą. Ir kurie tik palietė­tapo visiškai sveiki.



\chapter{15}


\par 1 Tuomet prie Jėzaus priėjo Rašto žinovų ir fariziejų iš Jeruzalės ir klausė: 
\par 2 “Kodėl Tavo mokiniai laužo prosenių tradiciją? Jie, prieš valgydami duoną, nesiplauna rankų”. 
\par 3 Jis, atsakydamas jiems, tarė: “O kodėl ir jūs laužote Dievo įsakymą savo tradicija?! 
\par 4 Juk Dievas įsakė: ‘Gerbk savo tėvą ir motiną!’, ir: ‘Kas keiktų tėvą ar motiną, mirtimi temiršta!’ 
\par 5 O jūs sakote: ‘Kiekvienas, kas pasakys tėvui ar motinai:­Viskas, kas tau būtų naudinga iš manęs, tebūnie dovana Dievui­ 
\par 6 tas gali negerbti tėvo ir motinos’. Taip jūs savo tradicija Dievo įsakymą padarote negaliojantį. 
\par 7 Veidmainiai! Gerai apie jus pranašavo Izaijas: 
\par 8 ‘Ši tauta artinasi prie manęs savo lūpomis ir gerbia mane savo liežuviu, bet jos širdis toli nuo manęs. 
\par 9 Veltui jie mane garbina, žmogiškus priesakus paversdami mokymu’ ”. 
\par 10 Sušaukęs minią, Jis kalbėjo: “Klausykite ir supraskite! 
\par 11 Ne kas patenka į burną, suteršia žmogų, bet kas išeina iš burnos, tai suteršia žmogų”. 
\par 12 Tada priėję Jo mokiniai pranešė: “Ar žinai, kad fariziejai pasipiktino, išgirdę tuos žodžius?” 
\par 13 Jis atsakė: “Kiekvienas augalas, kurio nesodino mano dangiškasis Tėvas, bus išrautas. 
\par 14 Palikite juos! Jie akli aklųjų vadovai. O jeigu aklas aklą ves, abu į duobę įkris”. 
\par 15 Tuomet Petras paprašė Jo: “Išaiškink mums tą palyginimą”. 
\par 16 Jėzus atsakė: “Ar ir jūs dar nesuprantate?! 
\par 17 Argi nesuprantate, kad visa, kas patenka į burną, eina į pilvą ir išmetama laukan? 
\par 18 O kas išeina iš burnos, eina iš širdies, ir tai suteršia žmogų. 
\par 19 Iš širdies išeina pikti sumanymai, žmogžudystės, svetimavimai, paleistuvystės, vagystės, melagingi liudijimai, piktžodžiavimai. 
\par 20 Šitie dalykai suteršia žmogų, o valgymas neplautomis rankomis žmogaus nesuteršia”. 
\par 21 Iš ten išėjęs, Jėzus pasitraukė į Tyro ir Sidono sritį. 
\par 22 Ir štai iš ano krašto atėjo moteris kanaanietė ir šaukė Jam: “Pasigailėk manęs, Viešpatie, Dovydo Sūnau! Mano dukterį baisiai kankina demonas!” 
\par 23 Bet Jėzus neatsakė nė žodžio. Tada priėjo mokiniai ir ėmė Jį maldauti: “Paleisk ją, nes ji šaukia mums iš paskos!” 
\par 24 Bet Jis atsakė: “Aš esu siųstas tik pas pražuvusias Izraelio namų avis”. 
\par 25 Tada ji priėjusi Jį pagarbino ir tarė: “Viešpatie, padėk man!” 
\par 26 Jis atsakė: “Nedera imti vaikų duoną ir mesti šunyčiams”. 
\par 27 O ji atsiliepė: “Taip, Viešpatie, bet ir šunyčiai ėda trupinius, nukritusius nuo jų šeimininko stalo”. 
\par 28 Tada Jėzus jai tarė: “O moterie, didis tavo tikėjimas! Tebūnie tau, kaip tu nori”. Ir tą pačią valandą jos duktė pasveiko. 
\par 29 Iš ten išėjęs, Jėzus atvyko prie Galilėjos ežero. Jis užkopė ant kalno ir atsisėdo. 
\par 30 Prie Jo susirinko didžiulės minios, kurios atsigabeno su savimi luošų, aklų, nebylių, raišų ir daugelį kitokių. Žmonės suguldė juos prie Jėzaus kojų, o Jis pagydė juos. 
\par 31 Minia stebėjosi, matydama nebylius kalbančius, luošius išgijusius, raišius vaikščiojančius ir akluosius reginčius. Ir jie šlovino Izraelio Dievą. 
\par 32 Pasišaukęs savo mokinius, Jėzus tarė: “Gaila man minios, nes jau tris dienas jie pasilieka su manimi ir neturi ko valgyti. Aš nepaleisiu jų alkanų, kad nenusilptų kelyje”. 
\par 33 Mokiniai Jam atsakė: “Iš kur mums imti dykumoje tiek duonos, kad galėtume pasotinti tokią didelę minią?” 
\par 34 Jėzus paklausė jų: “Kiek turite duonos?” Jie atsakė: “Septynis kepalus ir kelias žuveles”. 
\par 35 Jėzus liepė žmonėms susėsti ant žemės. 
\par 36 Tada paėmė septynis duonos kepalus ir žuvis, padėkojo, sulaužė ir davė savo mokiniams, o mokiniai miniai. 
\par 37 Visi valgė ir pasisotino. Ir surinko septynias pilnas pintines likusių trupinių. 
\par 38 O valgytojų buvo keturi tūkstančiai vyrų, neskaičiuojant moterų ir vaikų. 
\par 39 Paleidęs minią, Jis sėdo į valtį ir nuplaukė į Magadano sritį.



\chapter{16}


\par 1 Atėjo fariziejų ir sadukiejų ir mėgindami prašė Jį parodyti jiems ženklą iš dangaus. 
\par 2 Jis jiems atsakė: “Atėjus vakarui, jūs sakote: ‘Bus giedra, nes dangus raudonas’, 
\par 3 ir rytmetį: ‘Šiandien bus lietaus, nes rausta apsiniaukęs dangus’. Veidmainiai! Jūs mokate atpažinti dangaus veidą, o laiko ženklų­ ne. 
\par 4 Pikta ir svetimaujanti karta ieško ženklo, tačiau jai nebus duota kito ženklo, kaip tik pranašo Jonos ženklas”. Ir, palikęs juos, nuėjo šalin. 
\par 5 Keldamiesi į kitą ežero pusę, mokiniai buvo užmiršę pasiimti duonos. 
\par 6 Jėzus jiems tarė: “Būkite atidūs ir saugokitės fariziejų bei sadukiejų raugo”. 
\par 7 O jie tarpusavy svarstė: “Tai todėl, kad nepasiėmėme duonos”. 
\par 8 Tai supratęs, Jėzus tarė: “Mažatikiai! Kodėl svarstote, kad nepasiėmėte duonos? 
\par 9 Argi dar nesuprantate? Ar neprisimenate penkių kepalų penkiems tūkstančiams ir kiek pintinių surinkote trupinių? 
\par 10 Arba septynių kepalų keturiems tūkstančiams ir kiek pintinių surinkote trupinių?! 
\par 11 Tad kaip nesuprantate, jog kalbėjau jums ne apie duoną. Fariziejų ir sadukiejų raugo saugokitės!” 
\par 12 Tada jie suprato, kad Jis liepė saugotis ne duonos raugo, bet fariziejų ir sadukiejų mokslo. 
\par 13 Atėjęs į Pilypo Cezarėjos apylinkes, Jėzus paklausė savo mokinius: “Kuo žmonės mane, Žmogaus Sūnų, laiko?” 
\par 14 Jie atsakė: “Vieni Jonu Krikštytoju, kiti Eliju, kiti Jeremiju ar dar kuriuo iš pranašų”. 
\par 15 Jis vėl paklausė: “O kuo jūs mane laikote?” 
\par 16 Tada Simonas Petras atsakė: “Tu esi Kristus, gyvojo Dievo Sūnus”. 
\par 17 Jėzus jam atsakė: “Palaimintas tu, Simonai, Jonos sūnau, nes ne kūnas ir kraujas tau tai apreiškė, bet mano Tėvas, kuris yra danguje. 
\par 18 Ir Aš tau sakau: tu esi Petras, ir ant šios uolos Aš pastatysiu savo bažnyčią, ir pragaro vartai jos nenugalės. 
\par 19 Tau duosiu dangaus karalystės raktus; ką tu suriši žemėje, bus surišta ir danguje, ir ką tu atriši žemėje, bus atrišta ir danguje”. 
\par 20 Tada Jis griežtai įsakė savo mokiniams niekam neskelbti, kad Jis yra Jėzus­Kristus. 
\par 21 Nuo tada Jėzus pradėjo aiškinti savo mokiniams, kad Jis turįs eiti į Jeruzalę ir daug iškentėti nuo vyresniųjų, aukštųjų kunigų ir Rašto žinovų, būti nužudytas ir trečią dieną prisikelti. 
\par 22 Tada Petras, pasivadinęs Jį į šalį, ėmė drausti: “Jokiu būdu, Viešpatie, Tau neturi taip atsitikti!” 
\par 23 Bet Jis atsisukęs pasakė Petrui: “Eik šalin, šėtone! Tu man papiktinimas, nes mąstai ne apie tai, kas Dievo, o kas žmonių”. 
\par 24 Tuomet Jėzus savo mokiniams pasakė: “Jei kas nori eiti paskui mane, teišsižada pats savęs, teima savo kryžių ir teseka manimi. 
\par 25 Nes, kas nori išgelbėti savo gyvybę, tas ją praras; o kas praras savo gyvybę dėl manęs, tas ją atras. 
\par 26 Kokia gi žmogui nauda, jeigu jis laimėtų visą pasaulį, o pakenktų savo sielai? Arba kuo žmogus galėtų išsipirkti savo sielą? 
\par 27 Nes Žmogaus Sūnus ateis savo Tėvo šlovėje su savo angelais, ir tuomet Jis atlygins kiekvienam pagal jo darbus. 
\par 28 Iš tiesų sakau jums: kai kurie iš čia stovinčių neragaus mirties, kol pamatys Žmogaus Sūnų, ateinantį savo karalystėje”.



\chapter{17}


\par 1 Po šešių dienų Jėzus pasiėmė Petrą, Jokūbą ir jo brolį Joną ir užsivedė juos nuošaliai ant aukšto kalno. 
\par 2 Ten Jis atsimainė jų akivaizdoje. Jo veidas švietė kaip saulė, o Jo drabužiai tapo balti kaip šviesa. 
\par 3 Ir štai jiems pasirodė Mozė ir Elijas, kurie kalbėjosi su Juo. 
\par 4 Tada Petras kreipėsi į Jėzų: “Viešpatie, gera mums čia būti! Jei nori, mes pastatysime čia tris palapines: vieną Tau, kitą Mozei, trečią Elijui”. 
\par 5 Dar jam tebekalbant, štai šviesus debesis apgaubė juos, ir štai balsas iš debesies prabilo: “Šitas yra mano mylimas Sūnus, kuriuo Aš gėriuosi. Jo klausykite!” 
\par 6 Tai išgirdę, mokiniai puolė veidais į žemę ir labai išsigando. 
\par 7 Tuomet Jėzus priėjo, palietė juos ir tarė: “Kelkitės, nebijokite!” 
\par 8 Pakėlę akis, jie nieko daugiau nebematė, tik vieną Jėzų. 
\par 9 Besileidžiant nuo kalno, Jėzus jiems įsakė: “Niekam nepasakokite apie regėjimą, kol Žmogaus Sūnus prisikels iš numirusių”. 
\par 10 Tada Jo mokiniai Jį paklausė: “Kodėl Rašto žinovai sako, jog pirmiau turįs ateiti Elijas?” 
\par 11 Jėzus atsakė: “Iš tiesų Elijas turi ateiti pirma ir viską atstatyti. 
\par 12 Bet Aš jums sakau, kad Elijas jau atėjo, ir jie jo nepažino, bet padarė su juo, ką norėjo. Taip nuo jų turės kentėti ir Žmogaus Sūnus”. 
\par 13 Tuomet mokiniai suprato, kad Jis kalbėjo jiems apie Joną Krikštytoją. 
\par 14 Jiems atėjus prie minios, priėjo vienas vyras ir puolė prieš Jį ant kelių, sakydamas: 
\par 15 “Viešpatie, pasigailėk mano sūnaus! Jis per miegus vaikščioja ir labai kankinasi: dažnai įpuola į ugnį ir į vandenį. 
\par 16 Aš atvedžiau jį pas Tavo mokinius, bet jie nepajėgė išgydyti”. 
\par 17 Tada Jėzus atsakė: “O netikinti ir iškrypusi karta! Kaip ilgai man reikės su jumis būti? Kaip ilgai jus kęsti? Atveskite jį pas mane”. 
\par 18 Jėzus sudraudė demoną, šis išėjo iš berniuko, ir tą pačią akimirką jis pasveiko. 
\par 19 Tuomet mokiniai priėjo prie Jėzaus vieni ir klausė: “Dėl ko mes negalėjome jo išvaryti?” 
\par 20 Jėzus jiems atsakė: “Dėl jūsų netikėjimo. Iš tiesų sakau jums: jei turėtumėte tikėjimą kaip garstyčios grūdelį, jūs tartumėte šitam kalnui: ‘Persikelk iš čia į tenai’, ir jis persikeltų. Ir nieko jums nebūtų neįmanomo. 
\par 21 O šita veislė kitaip neišvaroma, kaip tik malda ir pasninku”. 
\par 22 Būdamas su mokiniais Galilėjoje, Jėzus jiems sakė: “Žmogaus Sūnus bus atiduotas į žmonių rankas, 
\par 23 ir jie nužudys Jį, o trečią dieną Jis prisikels”. Tada jie labai nuliūdo. 
\par 24 Atėjus jiems į Kafarnaumą, prie Petro priėjo didrachmų rinkėjai ir paklausė: “Ar jūsų Mokytojas nemoka didrachmos?” 
\par 25 Jis atsakė: “Taip!” Kai parėjo į namus, Jėzus pirmas jį prakalbino: “Kaip manai, Simonai? Iš ko žemės karaliai ima muitą ar mokestį: iš savo vaikų ar iš svetimųjų?” 
\par 26 Petras Jam atsakė: “Iš svetimųjų”. Jėzus jam tarė: “Taigi vaikai laisvi. 
\par 27 Tačiau, kad jų nepapiktintume, nueik prie ežero, užmesk meškerę, paimk pirmą užkibusią žuvį; ją pražiodęs, rasi staterą. Paimk ją ir atiduok jiems už mane ir už save”.



\chapter{18}


\par 1 Tuo metu prie Jėzaus priėjo mokiniai ir paklausė: “Kas yra didžiausias dangaus karalystėje?” 
\par 2 Pasišaukęs vaikutį, Jėzus pastatė tarp jų 
\par 3 ir tarė: “Iš tiesų sakau jums: jeigu neatsiversite ir nepasidarysite kaip maži vaikai, niekaip neįeisite į dangaus karalystę. 
\par 4 Taigi kiekvienas, kas nusižemins kaip šis vaikelis, bus didžiausias dangaus karalystėje”. 
\par 5 “Kas priima tokį vaikelį mano vardu, tas mane priima. 
\par 6 O kas pastūmėtų į nuodėmę vieną iš šitų mažutėlių, kurie tiki manim, tam būtų geriau, kad girnų akmuo būtų užkabintas jam ant kaklo ir jis būtų paskandintas jūros gelmėje. 
\par 7 Vargas pasauliui dėl papiktinimų! Papiktinimai neišvengiami, bet vargas tam žmogui, per kurį papiktinimas ateina. 
\par 8 Jei tavo ranka ar koja traukia tave nusidėti, nukirsk ją ir mesk šalin. Geriau tau sužalotam ar luošam įeiti į gyvenimą, negu su abiem rankom ir kojom būti įmestam į amžiną ugnį. 
\par 9 Ir jeigu tavo akis traukia tave nusidėti, išlupk ją ir mesk šalin. Tau geriau vienakiui įeiti į gyvenimą, negu su abiem akim būti įmestam į pragaro ugnį. 
\par 10 Žiūrėkite, kad nepaniekintumėte nė vieno iš šitų mažutėlių, nes, sakau jums, jų angelai danguje visuomet mato mano dangiškojo Tėvo veidą”. 
\par 11 “Žmogaus Sūnus atėjo gelbėti, kas buvo pražuvę. 
\par 12 Kaip jums atrodo: jeigu kas turėtų šimtą avių ir viena nuklystų, argi jis nepaliktų devyniasdešimt devynių ir neitų į kalnus ieškoti nuklydusios? 
\par 13 Ir jei surastų­iš tiesų sakau jums­jis džiaugtųsi dėl jos labiau negu dėl devyniasdešimt devynių, kurios nebuvo nuklydusios. 
\par 14 Taip ir jūsų Tėvas, kuris danguje, nenori, kad pražūtų bent vienas iš šitų mažutėlių”. 
\par 15 “Jei tavo brolis tau nusidėtų, eik ir pasakyk jam apie jo kaltę prie keturių akių. Jeigu jis paklausys tavęs, tu laimėjai savo brolį. 
\par 16 O jei nepaklausytų, pasiimk su savimi dar vieną ar du, kad ‘dviejų ar trijų liudytojų parodymais būtų patvirtintas kiekvienas žodis’. 
\par 17 Jeigu jis jų nepaklausytų, pranešk bažnyčiai. O jei neklausys nė bažnyčios, tebūna jis tau kaip pagonis ir muitininkas. 
\par 18 Iš tiesų sakau jums: ką tik jūs surišite žemėje, bus surišta ir danguje, ir ką tik atrišite žemėje, bus atrišta ir danguje”. 
\par 19 “Ir dar sakau jums: jeigu du iš jūsų susitars žemėje prašyti bet kokio dalyko, jiems mano dangiškasis Tėvas jį suteiks. 
\par 20 Kur du ar trys susirinkę mano vardu, ten ir Aš esu tarp jų”. 
\par 21 Tuomet Petras priėjo ir paklausė: “Viešpatie, kiek kartų turiu atleisti savo broliui, kai jis man nusideda? Ar iki septynių kartų?” 
\par 22 Jėzus jam atsakė: “Aš nesakau tau iki septynių kartų, bet iki septyniasdešimt septynių”. 
\par 23 “Todėl su dangaus karalyste yra panašiai kaip su karaliumi, kuris sumanė atsiskaityti su savo tarnais. 
\par 24 Jam pradėjus apyskaitą, atvedė pas jį vieną, kuris buvo jam skolingas dešimt tūkstančių talentų. 
\par 25 Kadangi šis neturėjo iš ko grąžinti skolą, valdovas įsakė parduoti jį, jo žmoną ir vaikus bei visą jo nuosavybę, kad būtų sumokėta skola. 
\par 26 Tada parpuolęs tarnas jį pagarbino ir tarė: ‘Turėk man kantrybės! Aš viską tau sumokėsiu’. 
\par 27 Pasigailėjęs to tarno, valdovas paleido jį ir dovanojo skolą. 
\par 28 Vos išėjęs, tas tarnas sutiko vieną savo tarnybos draugą, kuris buvo jam skolingas šimtą denarų, ir nutvėręs smaugė jį, sakydamas: ‘Atiduok skolą!’ 
\par 29 Puolęs ant kelių, draugas maldavo: ‘Turėk man kantrybės! Aš viską sumokėsiu’. 
\par 30 Bet tas nesutiko, ėmė ir įmetė jį į kalėjimą, iki atiduos skolą. 
\par 31 Matydami, kas nutiko, kiti tarnai labai nuliūdo. Jie nuėjo ir papasakojo valdovui, kas įvyko. 
\par 32 Tada, pasišaukęs jį, valdovas tarė: ‘Nedorasis tarne, visą tavo skolą aš tau dovanojau, nes labai manęs prašei. 
\par 33 Argi neturėjai ir tu pasigailėti savo draugo, kaip aš pasigailėjau tavęs?!’ 
\par 34 Užsirūstinęs valdovas atidavė jį kankintojams, iki jis sumokės visą skolą. 
\par 35 Taip ir mano dangiškasis Tėvas pasielgs su jumis, jeigu kiekvienas iš širdies neatleisite savo broliui jo nusižengimų”.



\chapter{19}


\par 1 Baigęs tai kalbėti, Jėzus pasitraukė iš Galilėjos ir atėjo į Judėjos sritį, anapus Jordano. 
\par 2 Paskui Jį sekė didelės minios žmonių, ir Jis ten juos išgydė. 
\par 3 Fariziejai taip pat atėjo pas Jį ir, mėgindami Jį, klausė: “Ar galima vyrui dėl kokios nors priežasties atleisti savo žmoną?” 
\par 4 Jis atsakė: “Argi neskaitėte, jog Kūrėjas iš pradžių ‘sukūrė juos, vyrą ir moterį’, 
\par 5 ir pasakė: ‘Todėl žmogus paliks tėvą ir motiną ir susijungs su savo žmona, ir du taps vienu kūnu’. 
\par 6 Taigi jie jau nebėra du, o vienas kūnas. Todėl ką Dievas sujungė, žmogus teneperskiria”. 
\par 7 Tada jie paklausė Jo: “O kodėl Mozė įsakė duoti skyrybų raštą, atleidžiant žmoną?” 
\par 8 Jis atsakė: “Mozė leido jums atleisti savo žmonas dėl jūsų širdies kietumo, bet pradžioje taip nebuvo. 
\par 9 Ir Aš jums sakau: kas atleidžia savo žmoną,­jei ne dėl ištvirkavimo,­ir veda kitą, svetimauja. Ir kas atleistąją veda, svetimauja”. 
\par 10 Jo mokiniai pasakė Jam: “Jei tokie vyro ir žmonos reikalai, tai geriau nevesti”. 
\par 11 Jis jiems atsakė: “Ne visi gali išmanyti tuos žodžius, o tik tie, kuriems duota. 
\par 12 Nes yra eunuchų, kurie gimė tokie iš motinos įsčių. Yra eunuchų, kuriuos tokius padarė žmonės. Ir yra eunuchų, kurie patys save tokius padarė dėl dangaus karalystės. Kas pajėgia išmanyti, teišmano”. 
\par 13 Tuomet atvedė pas Jį vaikučių, kad Jis uždėtų rankas ant jų ir pasimelstų, o mokiniai draudė jiems. 
\par 14 Bet Jėzus tarė: “Leiskite mažutėlius ir nedrauskite jiems ateiti pas mane, nes tokių yra dangaus karalystė”. 
\par 15 Ir, uždėjęs ant jų rankas, Jis iš ten išėjo. 
\par 16 Ir štai vienas, prie Jo priėjęs, klausė: “Gerasis Mokytojau, ką gero turiu daryti, kad turėčiau amžinąjį gyvenimą?” 
\par 17 Jis jam atsakė: “Kodėl vadini mane geru? Nė vieno nėra gero, tik vienas Dievas. O jei nori įeiti į gyvenimą, laikykis įsakymų”. 
\par 18 Tas paklausė Jo: “Kokių?” Jėzus atsakė: “Nežudyk, nesvetimauk, nevok, melagingai neliudyk; 
\par 19 gerbk savo tėvą ir motiną; mylėk savo artimą kaip save patį”. 
\par 20 Jaunuolis Jam tarė: “Viso to laikausi nuo savo jaunystės. Ko dar man trūksta?” 
\par 21 Jėzus atsakė: “Jei nori būti tobulas, eik, parduok, ką turi, išdalink vargšams, ir turėsi turtą danguje. Tada ateik ir sek paskui mane”. 
\par 22 Išgirdęs tuos žodžius, jaunuolis nuliūdęs pasitraukė, nes turėjo daug turto. 
\par 23 Tada Jėzus tarė savo mokiniams: “Iš tiesų sakau jums: turtingas sunkiai įeis į dangaus karalystę. 
\par 24 Ir dar kartą jums sakau: lengviau kupranugariui išlįsti pro adatos ausį, negu turtingam įeiti į Dievo karalystę”. 
\par 25 Tai išgirdę, Jo mokiniai labai nustebo ir klausė: “Kas tada gali būti išgelbėtas?” 
\par 26 Jėzus pažvelgė į juos ir tarė: “Žmonėms tai neįmanoma, bet Dievui viskas įmanoma”. 
\par 27 Tada Petras Jį paklausė: “Štai mes viską palikome ir sekame paskui Tave. Kas mums bus už tai?” 
\par 28 Jėzus jiems atsakė: “Iš tiesų sakau jums: atgimime, kai Žmogaus Sūnus sėdės savo šlovės soste, jūs, mano sekėjai, irgi sėdėsite dvylikoje sostų, teisdami dvylika Izraelio giminių. 
\par 29 Ir kiekvienas, kas paliko namus ar brolius, ar seseris, ar tėvą, ar motiną, ar žmoną, ar vaikus, ar laukus dėl mano vardo, gaus šimteriopai ir paveldės amžinąjį gyvenimą. 
\par 30 Tačiau daug pirmųjų bus paskutiniai, ir paskutiniai­pirmi”.



\chapter{20}


\par 1 “Su dangaus karalyste yra panašiai, kaip su šeimininku, kuris anksti rytą išėjo samdytis darbininkų savo vynuogynui. 
\par 2 Susiderėjęs su darbininkais po denarą dienai, jis nusiuntė juos į savo vynuogyną. 
\par 3 Išėjęs apie trečią valandą, jis pamatė kitus, stovinčius aikštėje be darbo. 
\par 4 Jis tarė jiems: ‘Eikite ir jūs į mano vynuogyną, ir, kas bus teisinga, aš jums užmokėsiu!’ Jie nuėjo. 
\par 5 Ir vėl išėjęs apie šeštą ir devintą valandą, jis taip pat padarė. 
\par 6 Išėjęs apie vienuoliktą, jis rado dar kitus stovinčius be darbo ir sako jiems: ‘Ko čia stovite visą dieną be darbo?’ 
\par 7 Jie atsakė: ‘Kad niekas mūsų nepasamdė’. Jis tarė jiems: ‘Eikite ir jūs į vynuogyną, ir, kas bus teisinga, jūs gausite’. 
\par 8 Atėjus vakarui, vynuogyno šeimininkas liepė ūkvedžiui: ‘Pašauk darbininkus ir išmokėk jiems atlyginimą, pradėdamas nuo paskutiniųjų ir baigdamas pirmaisiais!’ 
\par 9 Atėję pasamdytieji apie vienuoliktą valandą kiekvienas gavo po denarą. 
\par 10 Prisiartinę pirmieji manė daugiau gausią, bet irgi gavo po denarą. 
\par 11 Paėmę jie murmėjo prieš šeimininką, 
\par 12 sakydami: ‘Šitie paskutinieji tedirbo vieną valandą, o tu sulyginai juos su mumis, nešusiais dienos ir kaitros naštą’. 
\par 13 Bet jis vienam iš jų atsakė: ‘Bičiuli, aš tavęs neskriaudžiu! Argi ne už denarą susiderėjai su manimi? 
\par 14 Imk, kas tavo, ir eik sau. Aš noriu ir šitam paskutiniam duoti tiek, kiek tau. 
\par 15 Argi aš neturiu teisės daryti ką noriu su tuo, kas mano? Ar todėl tavo akis pikta, kad aš geras?’ 
\par 16 Taip paskutinieji bus pirmi, o pirmieji­paskutiniai; nes daug yra pašauktų, bet maža išrinktų”. 
\par 17 Išvykdamas į Jeruzalę, Jėzus pasiėmė skyrium dvylika mokinių ir kelyje kalbėjo jiems: 
\par 18 “Štai einame į Jeruzalę, ir Žmogaus Sūnus bus išduotas aukštiesiems kunigams bei Rašto žinovams. Jie nuteis Jį mirti, 
\par 19 atiduos pagonims tyčiotis, nuplakti ir nukryžiuoti, ir trečią dieną Jis prisikels”. 
\par 20 Tada prie Jėzaus priėjo Zebediejaus sūnų motina kartu su savo sūnumis ir, pagarbinusi Jį, ėmė kažko prašyti. 
\par 21 Jis paklausė jos: “Ko nori?” Toji atsakė: “Leisk, kad šitie abu mano sūnūs Tavo karalystėje sėdėtų vienas Tavo dešinėje, o kitas kairėje”. 
\par 22 Jėzus atsakė: “Nežinote, ko prašote. Ar galite gerti taurę, kurią Aš gersiu, ir būti krikštijami krikštu, kuriuo Aš krikštijamas?” Jie atsakė: “Galime”. 
\par 23 Tuomet Jis tarė: “Mano taurę, tiesa, gersite, ir krikštu, kuriuo Aš krikštijamas, būsite pakrikštyti, bet vietą mano dešinėje ar kairėje ne Aš duodu; tai bus tiems, kuriems mano Tėvo paruošta”. 
\par 24 Tai išgirdę, kiti dešimt mokinių supyko ant dviejų brolių. 
\par 25 O Jėzus, pasikvietęs juos pas save, tarė: “Jūs žinote, kad pagonių valdovai jiems viešpatauja ir didieji juos valdo. 
\par 26 Bet tarp jūsų taip neturi būti. Kas iš jūsų nori būti didžiausias, tebūnie jūsų tarnas, 
\par 27 ir kas nori būti pirmas tarp jūsų, tebūnie jūsų vergas. 
\par 28 Ir Žmogaus Sūnus atėjo, ne kad Jam tarnautų, bet pats tarnauti ir savo gyvybės atiduoti kaip išpirkos už daugelį”. 
\par 29 Jiems išeinant iš Jericho, paskui Jį sekė didelė minia. 
\par 30 Ir štai pakelėje sėdėjo du neregiai. Išgirdę praeinantį Jėzų, jie ėmė šaukti: “Viešpatie, Dovydo Sūnau, pasigailėk mūsų!” 
\par 31 Minia draudė juos, kad tylėtų, bet anie dar garsiau šaukė: “Viešpatie, Dovydo Sūnau, pasigailėk mūsų!” 
\par 32 Jėzus sustojo, pašaukė juos ir paklausė: “Ko norite, kad jums padaryčiau?” 
\par 33 Neregiai Jam atsakė: “Viešpatie, kad atsivertų mūsų akys”. 
\par 34 Pasigailėjęs Jėzus palietė jų akis; jie tučtuojau praregėjo ir nusekė paskui Jį.



\chapter{21}


\par 1 Kai jie prisiartino prie Jeruzalės ir atėjo į Betfagę prie Alyvų kalno, Jėzus pasiuntė du mokinius, 
\par 2 liepdamas: “Eikite į priešais esantį kaimą ir tuojau rasite pririštą asilę su asilaičiu. Atriškite ir atveskite juos man. 
\par 3 O jeigu kas imtų dėl to teirautis, atsakykite: ‘Jų reikia Viešpačiui’, ir iš karto juos paleis”. 
\par 4 Tai įvyko, kad išsipildytų, kas per pranašą buvo pasakyta: 
\par 5 “Sakykite Siono dukrai: štai atkeliauja tavo karalius, romus, ir joja ant asilės, lydimas asilaičio, darbinio gyvulio jauniklio”. 
\par 6 Mokiniai nuėjo ir padarė, kaip Jėzus jiems įsakė. 
\par 7 Jie atvedė asilę su asilaičiu, apdengė juos savo apsiaustais, ir Jis užsėdo ant viršaus. 
\par 8 Didžiulė minia tiesė drabužius ant kelio. Kiti kirto ir klojo ant kelio medžių šakas. 
\par 9 Iš priekio ir iš paskos einančios minios šaukė: “Osana Dovydo Sūnui! Palaimintas, kuris ateina Viešpaties vardu! Osana aukštybėse!” 
\par 10 Jam įėjus į Jeruzalę, sujudo visas miestas ir klausinėjo: “Kas yra šitas?” 
\par 11 O minios kalbėjo: “Tai pranašas Jėzus iš Galilėjos Nazareto”. 
\par 12 Įėjęs į Dievo šventyklą, Jėzus išvarė visus parduodančius ir perkančius šventykloje, išvartė pinigų keitėjų stalus bei karvelių pardavėjų suolus 
\par 13 ir tarė jiems: “Parašyta: ‘Mano namai vadinsis maldos namai’, o jūs pavertėte juos ‘plėšikų lindyne!’ ” 
\par 14 Šventykloje prie Jo susirinko aklų ir luošų, ir Jis išgydė juos. 
\par 15 O aukštieji kunigai ir Rašto žinovai, pamatę stebuklus, kuriuos Jis padarė, ir vaikus, šaukiančius šventykloje: “Osana Dovydo Sūnui!”, įpyko 
\par 16 ir sakė Jam: “Ar girdi, ką jie sako?” Jėzus atsiliepė: “Girdžiu. Argi niekada neskaitėte: ‘Iš vaikų ir žindomų kūdikių lūpų Tu paruošei sau tobulą gyrių’?” 
\par 17 Ir, palikęs juos, Jis išėjo iš miesto į Betaniją ir ten apsinakvojo. 
\par 18 Rytą, grįždamas į miestą, Jis išalko. 
\par 19 Pamatęs pakelėje figmedį, priėjo prie jo, bet nieko nerado, vien tik lapus. Ir tarė jam: “Tegul per amžius ant tavęs neaugs vaisiai!” Ir figmedis kaipmat nudžiūvo. 
\par 20 Tai pamatę, mokiniai nustebo ir sakė: “Kaip tas figmedis taip greit nudžiūvo?!” 
\par 21 Jėzus atsakė: “Iš tiesų sakau jums: jeigu turėsite tikėjimą ir neabejosite, jūs ne tik padarysite taip su figmedžiu, bet ir jei pasakysite šitam kalnui: ‘Pasikelk ir meskis į jūrą!’, taip įvyks. 
\par 22 Ir visa, ko tik prašysite maldoje tikėdami,­gausite”. 
\par 23 Kai Jėzus atėjo į šventyklą ir pradėjo mokyti, priėjo prie Jo aukštųjų kunigų ir tautos vyresniųjų, kurie klausė: “Kokią teisę turi taip daryti? Ir kas Tau davė šitą valdžią?” 
\par 24 Jėzus atsakė: “Aš irgi paklausiu jus vieno dalyko. Jei man atsakysite, ir Aš pasakysiu, kokia valdžia tai darau. 
\par 25 Iš kur buvo Jono krikštas? Iš dangaus ar iš žmonių?” Jie samprotavo tarpusavy: “Jei pasakysime­iš dangaus, Jis mums sakys: ‘Tai kodėl juo netikėjote?’ 
\par 26 O pasakyti­iš žmonių, baisu prieš minią, nes visi laiko Joną pranašu”. 
\par 27 Todėl jie atsakė Jėzui: “Mes nežinome”. Tada Jis tarė: “Tai ir Aš jums nesakysiu, kokia valdžia tai darau”. 
\par 28 “Kaip jūs manote? Vienas žmogus turėjo du sūnus. Kartą, priėjęs prie pirmojo, tarė: ‘Sūnau, eik ir padirbėk šiandien mano vynuogyne’. 
\par 29 Šis atsakė: ‘Nenoriu’, bet vėliau apsigalvojo ir nuėjo. 
\par 30 Paskui tėvas kreipėsi į antrąjį sūnų tais pačiais žodžais. Šis jam atsakė: ‘Einu, viešpatie’, bet nenuėjo. 
\par 31 Kuris iš jų įvykdė tėvo valią?” Jie atsakė: “Pirmasis”. Tada Jėzus jiems tarė: “Iš tiesų sakau jums: muitininkai ir paleistuvės pirma jūsų eina į Dievo karalystę. 
\par 32 Nes Jonas atėjo pas jus teisumo keliu, bet jūs netikėjote juo. O muitininkai ir paleistuvės juo tikėjo. Bet jūs, tai matydami, nė vėliau neatgailavote ir netikėjote juo”. 
\par 33 “Pasiklausykite kito palyginimo. Buvo vienas šeimininkas, kuris pasodino vynuogyną, aptvėrė jį tvora, įrengė spaustuvą, pastatė bokštą, išnuomojo vynininkams ir iškeliavo į tolimą šalį. 
\par 34 Atėjus vaisių metui, jis nusiuntė savo tarnus pas vynininkus atsiimti savo vaisių. 
\par 35 Bet vynininkai, nutvėrę jo tarnus, vieną primušė, kitą nužudė, o trečią užmėtė akmenimis. 
\par 36 Jis vėl nusiuntė tarnų, daugiau negu pirma. Bet vynininkai ir su šitais pasielgė kaip su anais. 
\par 37 Galiausiai jis išsiuntė pas juos savo sūnų, sakydamas: ‘Jie gerbs mano sūnų’. 
\par 38 Tačiau vynininkai, išvydę sūnų, ėmė kalbėtis: ‘Tai paveldėtojas! Eime, užmuškime jį ir pagrobkime palikimą’. 
\par 39 Nutvėrę jie išmetė jį iš vynuogyno ir užmušė. 
\par 40 Tad ką gi atvykęs vynuogyno šeimininkas padarys su tais vynininkais?” 
\par 41 Jie atsakė Jam: “Jis žiauriai nužudys piktadarius ir išnuomos vynuogyną kitiems vynininkams, kurie, atėjus metui, atiduos vaisių”. 
\par 42 Jėzus jiems tarė: “Ar niekada neskaitėte Raštuose: ‘Akmuo, kurį statytojai atmetė, tapo kertiniu akmeniu. Tai Viešpaties padaryta ir nuostabu mūsų akyse’. 
\par 43 Todėl sakau jums: Dievo karalystė bus iš jūsų atimta ir duota tautai, kuri neš jos vaisių. 
\par 44 Kas kris ant to akmens, tas suduš, o ant ko tas akmuo užgrius, tą sutriuškins”. 
\par 45 Išgirdę Jo palyginimus, aukštieji kunigai ir fariziejai suprato, kad Jis kalbėjo apie juos. 
\par 46 Jie stengėsi Jį suimti, tačiau bijojo minios, nes ji laikė Jį pranašu.



\chapter{22}


\par 1 Jėzus vėl kalbėjo palyginimais: 
\par 2 “Su dangaus karalyste yra panašiai kaip su karaliumi, kuris kėlė savo sūnui vestuves. 
\par 3 Jis išsiuntė tarnus šaukti pakviestųjų į vestuvių pokylį, bet tie nenorėjo ateiti. 
\par 4 Tada jis vėl siuntė kitus tarnus, liepdamas: ‘Sakykite pakviestiesiems: Štai surengiau pokylį, mano jaučiai ir nupenėti veršiai papjauti, ir viskas paruošta. Ateikite į vestuves!’ 
\par 5 Tačiau kviečiamieji jo nepaisė ir nuėjo kas sau: vienas į ūkį, kitas prekiauti, 
\par 6 o kiti tarnus nutvėrę išniekino ir užmušė. 
\par 7 Tai išgirdęs, karalius užsirūstino ir, išsiuntęs kariuomenes, sunaikino tuos žmogžudžius ir padegė jų miestą. 
\par 8 Tuomet jis tarė savo tarnams: ‘Vestuvės surengtos, bet pakviestieji nebuvo verti. 
\par 9 Todėl eikite į kryžkeles ir, ką tik rasite, kvieskite į vestuves’. 
\par 10 Tarnai išėjo į kelius ir surinko visus, ką tik sutiko, blogus ir gerus. Ir vestuvės buvo pilnos svečių. 
\par 11 Karalius atėjo pasižiūrėti svečių ir pamatė žmogų, neapsirengusį vestuviniu drabužiu. 
\par 12 Jis tarė jam: ‘Bičiuli, kaip čia įėjai, neturėdamas vestuvių drabužio?’ Tasai tylėjo. 
\par 13 Tada karalius paliepė tarnams: ‘Suriškite jam rankas ir kojas ir išmeskite jį laukan į tamsybes. Ten bus verksmas ir dantų griežimas’. 
\par 14 Nes daug pašauktų, bet maža išrinktų”. 
\par 15 Tuomet fariziejai pasitraukė ir tarėsi, kaip Jį sugauti kalboje. 
\par 16 Jie nusiuntė pas Jį savo mokinių kartu su erodininkais, kurie klausė: “Mokytojau, mes žinome, kad esi tiesus, mokai Dievo kelio, kaip reikalauja tiesa, ir niekam nepataikauji, nes neatsižvelgi į asmenis. 
\par 17 Tad pasakyk mums, kaip manai: reikia mokėti ciesoriui mokesčius ar ne?” 
\par 18 Suprasdamas jų klastą, Jėzus tarė: “Kam spendžiate man pinkles, veidmainiai? 
\par 19 Parodykite man mokesčių pinigą!” Jie padavė Jam denarą. 
\par 20 Jis paklausė: “Kieno čia atvaizdas ir įrašas?” 
\par 21 Jie atsakė: “Ciesoriaus”. Tuomet Jėzus jiems tarė: “Atiduokite tad, kas ciesoriaus, ciesoriui, o kas Dievo­Dievui”. 
\par 22 Tai girdėdami, jie stebėjosi ir, palikę Jį, nuėjo. 
\par 23 Tą pačią dieną atėjo pas Jį sadukiejų, kurie nepripažįsta mirusiųjų prisikėlimo, ir klausė: 
\par 24 “Mokytojau, Mozė yra pasakęs: ‘Jei kas mirtų bevaikis, tegul jo brolis veda jo žmoną ir pažadina savo broliui palikuonių’. 
\par 25 Štai pas mus buvo septyni broliai. Pirmasis vedęs mirė ir, neturėdamas vaikų, paliko žmoną savo broliui. 
\par 26 Taip atsitiko antrajam ir trečiajam iki septintojo. 
\par 27 Po jų visų numirė ir ta moteris. 
\par 28 Tad kurio iš septynių ji bus žmona prisikėlime? Juk visi yra ją turėję”. 
\par 29 Jėzus jiems atsakė: “Jūs klystate, nepažindami nei Raštų, nei Dievo jėgos. 
\par 30 Prisikėlime nei ves, nei tekės, bet bus kaip Dievo angelai danguje. 
\par 31 O apie mirusiųjų prisikėlimą ar neskaitėte, kas jums Dievo pasakyta: 
\par 32 ‘Aš esu Abraomo Dievas, Izaoko Dievas ir Jokūbo Dievas’. Dievas nėra mirusiųjų Dievas, bet gyvųjų!” 
\par 33 Tai girdėdama, minia stebėjosi Jo mokymu. 
\par 34 Fariziejai, išgirdę, kad Jėzus nutildė sadukiejus, susirinko kartu, 
\par 35 ir vienas iš jų, Įstatymo mokytojas, mėgindamas Jį, paklausė: 
\par 36 “Mokytojau, koks įsakymas yra didžiausias Įstatyme?” 
\par 37 Jėzus jam atsakė: “ ‘Mylėk Viešpatį, savo Dievą, visa savo širdimi, visa savo siela ir visu savo protu’. 
\par 38 Tai pirmasis ir didžiausias įsakymas. 
\par 39 Antrasis­panašus į jį: ‘Mylėk savo artimą kaip save patį’. 
\par 40 Šitais dviem įsakymais remiasi visas Įstatymas ir Pranašai”. 
\par 41 Kol fariziejai tebebuvo susirinkę, Jėzus juos paklausė: 
\par 42 “Ką jūs manote apie Kristų? Kieno Jis Sūnus?” Jie atsakė: “Dovydo”. 
\par 43 Jis tarė jiems: “O kodėl gi Dovydas, Dvasios įkvėptas, vadina Jį Viešpačiu, sakydamas: 
\par 44 ‘Viešpats tarė mano Viešpačiui: sėskis mano dešinėje, kol patiesiu Tavo priešus tarsi pakojį po Tavo kojų’. 
\par 45 Jei tad Dovydas vadina Jį Viešpačiu, kaipgi tada Jis gali būti jo Sūnus?” 
\par 46 Ir nė vienas negalėjo Jam atsakyti nė žodžio, ir niekas nedrįso nuo tos dienos Jį klausinėti.



\chapter{23}


\par 1 Tuomet Jėzus ėmė kalbėti minioms ir savo mokiniams: 
\par 2 “Į Mozės krasę atsisėdo Rašto žinovai ir fariziejai. 
\par 3 Todėl visko, ko jie liepia jums laikytis, laikykitės ir vykdykite, tačiau nesielkite, kaip jie elgiasi, nes jie kalba, bet nedaro. 
\par 4 Jie riša sunkias, nepanešamas naštas ir krauna žmonėms ant pečių, o patys nenori jų nė pirštu pajudinti. 
\par 5 Jie viską daro, kad būtų žmonių matomi. Jie pasiplatina maldos diržus ir pasididina apsiaustų kutus. 
\par 6 Jie mėgsta garbės vietas pokyliuose bei pirmuosius krėslus sinagogose, 
\par 7 mėgsta sveikinimus aikštėse ir, kad žmonės juos vadintų ‘Rabi, Rabi’. 
\par 8 Bet jūs nesivadinkite ‘Rabi’, nes vienas yra jūsų Mokytojas­Kristus, o jūs visi esate broliai. 
\par 9 Ir nė vieno žemėje nevadinkite tėvu, nes vienas jūsų Tėvas, kuris yra danguje. 
\par 10 Taip pat nesivadinkite mokytojais, nes vienas jūsų Mokytojas­ Kristus. 
\par 11 Kas iš jūsų didžiausias, tebūna jums tarnas. 
\par 12 Ir kas save aukština, bus pažemintas, o kas save žemina, bus išaukštintas. 
\par 13 Bet vargas jums, veidmainiai Rašto žinovai ir fariziejai! Nes jūs užrakinate žmonėms dangaus karalystę ir nei patys neinate, nei į ją einantiems neleidžiate įeiti. 
\par 14 Vargas jums, veidmainiai Rašto žinovai ir fariziejai! Nes jūs suryjate našlių namus ir dedatės kalbą ilgas maldas, todėl gausite dar didesnį pasmerkimą. 
\par 15 Vargas jums, veidmainiai Rašto žinovai ir fariziejai! Nes jūs keliaujate per jūrą ir sausumą, kad laimėtumėte vieną naujatikį, ir kai jis tokiu tampa, jūs padarote iš jo pragaro vaiką, dvigubai blogesnį už jus pačius. 
\par 16 Vargas jums, aklieji vadai, kurie sakote: ‘Jei kas prisiektų šventykla, tai nieko, o jei kas prisiektų šventyklos auksu, tai jis įsipareigoja’. 
\par 17 Kvaili jūs ir akli! Kas gi didesnis­auksas ar šventykla, kuri pašventina auksą? 
\par 18 Arba vėl sakote: ‘Jei kas prisiektų aukuru, tai nieko, o jei kas prisiektų atnaša ant aukuro, tai jis įsipareigoja’. 
\par 19 Kvaili ir akli! Kas gi didesnis­ atnaša ar aukuras, kuris pašventina atnašą? 
\par 20 Todėl, kas prisiekia aukuru, prisiekia juo ir viskuo, kas ant jo padėta, 
\par 21 o kas prisiekia šventykla, prisiekia ja ir Tuo, kuris joje gyvena. 
\par 22 Ir kas prisiekia dangumi, prisiekia Dievo sostu ir Tuo, kuris jame sėdi. 
\par 23 Vargas jums, veidmainiai Rašto žinovai ir fariziejai! Nes jūs duodate dešimtinę nuo mėtų, krapų ir kmynų, o paliekate, kas Įstatyme svarbiau,­teisingumą, gailestingumą ir tikėjimą. Tai turite daryti ir ano nepalikti! 
\par 24 Akli vadai, jūs iškošiate uodą, o praryjate kupranugarį. 
\par 25 Vargas jums, veidmainiai Rašto žinovai ir fariziejai! Nes jūs valote taurės bei dubens išorę, o viduje esate pilni gobšumo ir nesusilaikymo. 
\par 26 Aklas fariziejau! Pirmiau išvalyk taurės ir dubens vidų, kad būtų švari ir išorė! 
\par 27 Vargas jums, veidmainiai Rašto žinovai ir fariziejai! Nes jūs panašūs į pabaltintus kapus, kurie iš paviršiaus atrodo gražiai, o viduje pilni numirėlių kaulų ir visokių nešvarumų. 
\par 28 Taip ir jūs iš paviršiaus atrodote žmonėms teisūs, o viduje esate pilni veidmainystės ir nedorumo. 
\par 29 Vargas jums, veidmainiai Rašto žinovai ir fariziejai! Nes jūs statote pranašams antkapius, puošiate teisiųjų kapus 
\par 30 ir sakote: ‘Jei būtume gyvenę savo protėvių dienomis, nebūtume kartu su jais susitepę pranašų krauju’. 
\par 31 Taigi jūs patys prieš save liudijate, jog esate pranašų žudytojų vaikai. 
\par 32 Pripildykite tad savo tėvų saiką! 
\par 33 Gyvatės! Angių išperos! Kaip jūs ištrūksite nuo pasmerkimo į pragarą?! 
\par 34 Todėl štai siunčiu pas jus pranašų, išminčių ir Rašto žinovų. Vienus iš jų užmušite ir nukryžiuosite, kitus plaksite savo sinagogose ir persekiosite nuo miesto iki miesto, 
\par 35 kad ant jūsų kristų visas teisus kraujas, pralietas žemėje, pradedant teisiojo Abelio krauju ir baigiant krauju Barachijo sūnaus Zacharijo, kurį nužudėte tarp šventyklos ir aukuro. 
\par 36 Iš tiesų sakau jums: visa tai ištiks šitą kartą”. 
\par 37 “Jeruzale, Jeruzale! Tu žudai pranašus ir užmėtai akmenimis tuos, kurie pas tave siųsti. Kiek kartų norėjau surinkti tavo vaikus, kaip višta surenka savo viščiukus po sparnais, o tu nenorėjai! 
\par 38 Štai jūsų namai jums paliekami tušti. 
\par 39 Ir sakau jums: nuo dabar jūs manęs nebematysite, kol tarsite: ‘Palaimintas Tas, kuris ateina Viešpaties vardu!’ ”



\chapter{24}


\par 1 Išėjęs iš šventyklos, Jėzus ėjo tolyn. Priėjo Jo mokiniai, rodydami Jam šventyklos pastatus. 
\par 2 O Jis jiems tarė: “Ar matote visa tai? Iš tiesų sakau jums: čia neliks akmens ant akmens, viskas bus išgriauta!” 
\par 3 Kai Jis sėdėjo Alyvų kalne, priėjo vieni mokiniai ir klausė: “Pasakyk mums, kada tai įvyks? Ir koks Tavo atėjimo ir pasaulio pabaigos ženklas?” 
\par 4 Jėzus jiems atsakė: “Žiūrėkite, kad niekas jūsų nesuklaidintų. 
\par 5 Daug kas ateis mano vardu ir sakys: ‘Aš esu Kristus!’, ir daugelį suklaidins. 
\par 6 Girdėsite apie karus ir karų gandus. Žiūrėkite, kad neišsigąstumėte, nes visa tai turi įvykti. Bet tai dar ne galas. 
\par 7 Tauta sukils prieš tautą ir karalystė prieš karalystę. Įvairiose vietose bus badmečių, marų ir žemės drebėjimų. 
\par 8 Tačiau visa tai­gimdymo skausmų pradžia. 
\par 9 Tada jus atiduos kankinti ir žudyti. Jūs būsite visų tautų nekenčiami dėl mano vardo. 
\par 10 Daugelis pasipiktins, vieni kitus išdavinės ir vieni kitų nekęs. 
\par 11 Atsiras daug netikrų pranašų, kurie daugelį suvedžios. 
\par 12 Kadangi įsigalės neteisumas, daugelio meilė atšals. 
\par 13 Bet kas ištvers iki galo, tas bus išgelbėtas. 
\par 14 Ir bus paskelbta ši karalystės Evangelija visame pasaulyje paliudyti visoms tautoms. Ir tada ateis galas”. 
\par 15 “Todėl, kai pamatysite per pranašą Danielių paskelbtą naikinimo bjaurastį, stovinčią šventoje vietoje (kas skaito, teišmano), 
\par 16 tada, kas bus Judėjoje, tebėga į kalnus, 
\par 17 kas ant stogo, tenelipa žemėn pasiimti ko nors iš savo namų, 
\par 18 o kas laukuose, tenegrįžta pasiimti apsiausto. 
\par 19 Vargas nėščioms ir žindančioms tomis dienomis! 
\par 20 Bet melskitės, kad jums netektų bėgti žiemą ar per sabatą. 
\par 21 Tuomet bus didelis suspaudimas, kokio nėra buvę nuo pasaulio pradžios iki dabar ir kokio daugiau nebebus. 
\par 22 Ir jeigu tos dienos nebūtų sutrumpintos, neišsigelbėtų nė vienas kūnas. Bet dėl išrinktųjų tos dienos bus sutrumpintos. 
\par 23 Jei tada kas nors jums sakys: ‘Štai čia Kristus’, arba: ‘Jis tenai!’,­netikėkite, 
\par 24 nes atsiras netikrų kristų ir netikrų pranašų, ir jie darys didelių ženklų bei stebuklų, kad suklaidintų, jei įmanoma, net išrinktuosius. 
\par 25 Štai Aš jums iš anksto tai pasakiau!” 
\par 26 “Todėl, jeigu jums sakytų: ‘Štai Jis dykumoje!’, neikite, ‘Štai Jis kambariuose!’, netikėkite. 
\par 27 Kaip žaibas tvyksteli iš rytų ir nušvinta iki vakarų, toks bus ir Žmogaus Sūnaus atėjimas. 
\par 28 Kur tik bus lavonų, ten sulėks ir maitvanagiai. 
\par 29 Tuoj pat po tų suspaudimo dienų saulė užtems, mėnulis nebeduos šviesos, žvaigždės kris iš dangaus, ir dangaus jėgos bus sudrebintos. 
\par 30 Tada danguje pasirodys Žmogaus Sūnaus ženklas, ir visos žemės giminės raudos ir pamatys Žmogaus Sūnų, ateinantį dangaus debesyse su galybe ir didžia šlove. 
\par 31 Jis pasiųs savo angelus su skardžiais trimitų garsais, ir jie surinks Jo išrinktuosius iš keturių žemės pusių, nuo vieno dangaus pakraščio iki kito. 
\par 32 Pasimokykite iš palyginimo su figmedžiu: kai jo šaka suminkštėja ir sprogsta lapai, jūs žinote, jog artėja vasara. 
\par 33 Taip pat, visa šita išvydę, žinokite, jog tai arti, prie durų. 
\par 34 Iš tiesų sakau jums: ši karta nepraeis, iki visa tai įvyks. 
\par 35 Dangus ir žemė praeis, bet mano žodžiai nepraeis. 
\par 36 Tačiau tos dienos ir valandos niekas nežino, nė dangaus angelai, o vien tik mano Tėvas”. 
\par 37 “Kaip buvo Nojaus dienomis, taip bus ir tada, kai ateis Žmogaus Sūnus. 
\par 38 Kaip dienomis prieš tvaną žmonės valgė, gėrė, tuokėsi ir tuokė iki tos dienos, kurią Nojus įžengė į laivą, 
\par 39 nieko nenumanydami, kol užėjo tvanas ir visus nusinešė; taip bus ir tada, kai ateis Žmogaus Sūnus. 
\par 40 Tada du bus kartu lauke, ir vienas bus paimtas, o kitas paliktas. 
\par 41 Dvi mals girnomis, ir viena bus paimta, o kita palikta. 
\par 42 Todėl budėkite, nes nežinote, kurią valandą ateis jūsų Viešpats. 
\par 43 Supraskite ir tai: jeigu šeimininkas žinotų, kurią nakties valandą ateis vagis, jis budėtų ir neleistų jam įsilaužti į namus. 
\par 44 Todėl ir jūs būkite pasiruošę, nes Žmogaus Sūnus ateis tą valandą, kurią nemanote”. 
\par 45 “Kas yra tas ištikimas bei protingas tarnas, kurį šeimininkas paskyrė tarnauti savo šeimynai, kad ją maitintų deramu laiku? 
\par 46 Palaimintas tarnas, kurį sugrįžęs šeimininkas ras taip darantį. 
\par 47 Iš tiesų sakau jums: jį paskirs valdyti visų savo turtų. 
\par 48 Bet jei blogas tarnas tartų savo širdyje: ‘Mano šeimininkas neskuba grįžti’, 
\par 49 ir pradėtų mušti savo tarnybos draugus, valgyti ir gerti su girtuokliais,­ 
\par 50 to tarno šeimininkas sugrįš tą dieną, kai jis nelaukia, ir tą valandą, kurią jis nemano. 
\par 51 Jis perkirs jį pusiau ir paskirs jam dalį su veidmainiais. Ten bus verksmas ir dantų griežimas”.



\chapter{25}


\par 1 “Tada su dangaus karalyste bus panašiai kaip su dešimtimi mergaičių, kurios, pasiėmusios savo žibintus, išėjo pasitikti jaunikio. 
\par 2 Penkios iš jų buvo protingos ir penkios kvailos. 
\par 3 Kvailosios pasiėmė žibintus, bet nepasiėmė aliejaus. 
\par 4 Protingosios kartu su žibintais pasiėmė induose ir aliejaus. 
\par 5 Jaunikiui vėluojant, visos pradėjo snausti ir užmigo. 
\par 6 Vidurnaktį pasigirdo šauksmas: ‘Štai jaunikis ateina! Išeikite jo pasitikti!’ 
\par 7 Tada visos mergaitės atsikėlė ir taisėsi žibintus. 
\par 8 Kvailosios prašė protingųjų: ‘Duokite mums savo aliejaus, nes mūsų žibintai gęsta!’ 
\par 9 Protingosios atsakė: ‘Kad kartais nepristigtų ir mums, ir jums, verčiau eikite pas pardavėjus ir nusipirkite’. 
\par 10 Joms beeinant pirkti, atėjo jaunikis. Kurios buvo pasiruošusios, įėjo kartu su juo į vestuves, ir durys buvo uždarytos. 
\par 11 Vėliau atėjo ir anos mergaitės ir ėmė prašyti: ‘Viešpatie, viešpatie, atidaryk mums!’ 
\par 12 O jis atsakė: ‘Iš tiesų sakau jums: aš jūsų nepažįstu!’ 
\par 13 Taigi budėkite, nes nežinote nei dienos, nei valandos, kurią Žmogaus Sūnus ateis”. 
\par 14 “Bus taip, kaip atsitiko žmogui, kuris, iškeliaudamas į tolimą šalį, pasišaukė savo tarnus ir patikėjo jiems savo turtą. 
\par 15 Vienam jis davė penkis talentus, kitam du, trečiam vieną­kiekvienam pagal jo gabumus­ir tuojau iškeliavo. 
\par 16 Tas, kuris gavo penkis talentus, nuėjęs ėmė su jais verstis ir pelnė kitus penkis. 
\par 17 Taip pat tas, kuris gavo du talentus, pelnė kitus du. 
\par 18 O kuris gavo vieną, nuėjo, iškasė duobę ir paslėpė šeimininko pinigus. 
\par 19 Praėjus nemaža laiko, tų tarnų šeimininkas grįžo ir pradėjo daryti su jais apyskaitą. 
\par 20 Atėjo tas, kuris buvo gavęs penkis talentus; jis atnešė kitus penkis ir tarė: ‘Šeimininke, davei man penkis talentus, štai aš pelniau kitus penkis’. 
\par 21 Jo šeimininkas atsakė: ‘Gerai, šaunusis ir ištikimasis tarne! Kadangi buvai ištikimas mažuose dalykuose, pavesiu tau didelius. Eikš į savo šeimininko džiaugsmą!’ 
\par 22 Taip pat tas, kuris buvo gavęs du talentus, atėjęs pasakė: ‘Šeimininke, davei man du talentus, štai aš pelniau kitus du’. 
\par 23 Jo šeimininkas tarė: ‘Gerai, šaunusis ir ištikimasis tarne! Kadangi buvai ištikimas mažuose dalykuose, pavesiu tau didelius. Eikš į savo šeimininko džiaugsmą!’ 
\par 24 Priėjęs tas, kuris buvo gavęs vieną talentą, sakė: ‘Šeimininke, aš žinojau, kad tu­žmogus kietas: pjauni, kur nesėjai, ir renki, kur nebarstei. 
\par 25 Pabijojęs nuėjau ir paslėpiau tavo talentą žemėje. Še, turėk, kas tavo’. 
\par 26 Jo šeimininkas jam atsakė: ‘Blogasis tarne, tinginy! Tu žinojai, kad aš pjaunu, kur nesėjau, ir renku, kur nebarsčiau. 
\par 27 Taigi privalėjai duoti mano pinigus pinigų keitėjams, o sugrįžęs būčiau atsiėmęs, kas mano, su palūkanomis. 
\par 28 Todėl atimkite iš jo talentą ir atiduokite tam, kuris turi dešimt talentų. 
\par 29 Nes kiekvienam, kas turi, bus duota, ir jis turės su perteklium, o iš neturinčio bus atimta ir tai, ką jis turi. 
\par 30 Šitą niekam tikusį tarną išmeskite laukan į tamsybes. Ten bus verksmas ir dantų griežimas’ ”. 
\par 31 “Kai ateis Žmogaus Sūnus savo šlovėje ir kartu su Juo visi šventi angelai, tada Jis atsisės savo šlovės soste. 
\par 32 Jo akivaizdoje bus surinkti visų tautų žmonės, ir Jis atskirs juos vienus nuo kitų, kaip piemuo atskiria avis nuo ožių. 
\par 33 Avis Jis pastatys dešinėje, o ožius­kairėje. 
\par 34 Tuomet Karalius tars stovintiems dešinėje: ‘Ateikite, mano Tėvo palaimintieji, paveldėkite nuo pasaulio sukūrimo jums paruoštą karalystę! 
\par 35 Nes Aš buvau išalkęs, ir jūs mane pavalgydinote, buvau ištroškęs, ir mane pagirdėte, buvau keleivis, ir mane priėmėte, 
\par 36 buvau nuogas, ir mane aprengėte, buvau ligonis, ir mane aplankėte, buvau kalinys, ir atėjote pas mane’. 
\par 37 Tada teisieji klaus: ‘Viešpatie, kada gi matėme Tave alkaną ir pavalgydinome, ištroškusį ir pagirdėme? 
\par 38 Kada gi matėme Tave keliaujantį ir priėmėme ar nuogą ir aprengėme? 
\par 39 Kada gi matėme Tave sergantį ar kalinį ir aplankėme?’ 
\par 40 Ir atsakys jiems Karalius: ‘Iš tiesų sakau jums, kiek kartų tai padarėte vienam iš šitų mažiausiųjų mano brolių, man padarėte’. 
\par 41 Tada Jis prabils ir į stovinčius kairėje: ‘Eikite šalin nuo manęs, prakeiktieji, į amžinąją ugnį, kuri paruošta velniui ir jo angelams! 
\par 42 Nes Aš buvau išalkęs, ir jūs manęs nepavalgydinote, buvau ištroškęs, ir manęs nepagirdėte, 
\par 43 buvau keleivis, ir manęs nepriėmėte, nuogas, ir manęs neaprengėte, ligonis ir kalinys, ir manęs neaplankėte’. 
\par 44 Tada jie atsakys: ‘Viešpatie, kada gi matėme Tave alkaną ar ištroškusį, ar keleivį, ar nuogą, ar ligonį, ar kalinį ir Tau nepatarnavome?’ 
\par 45 Tada Jis atsakys jiems: ‘Iš tiesų sakau jums: kiek kartų taip nepadarėte vienam šitų mažiausiųjų, man nepadarėte’. 
\par 46 Ir šitie eis į amžinąjį kentėjimą, o teisieji į amžinąjį gyvenimą”.



\chapter{26}


\par 1 Baigęs visa tai kalbėti, Jėzus tarė savo mokiniams: 
\par 2 “Jūs žinote, kad po dviejų dienų bus Pascha, ir Žmogaus Sūnus bus atiduotas nukryžiuoti”. 
\par 3 Tuomet aukštieji kunigai, Rašto žinovai bei tautos vyresnieji susirinko į vyriausiojo kunigo Kajafo rūmus 
\par 4 ir nusprendė suimti Jėzų klasta ir Jį nužudyti. 
\par 5 Bet jie sakė: “Tik ne per šventes, kad žmonėse nekiltų sąmyšio”. 
\par 6 Kai Jėzus buvo Betanijoje, Simono Raupsuotojo namuose, 
\par 7 atėjo moteris su alebastriniu labai brangaus kvapiojo aliejaus indu ir Jam sėdinčiam prie stalo išpylė ant galvos. 
\par 8 Tai pamatę, mokiniai pasipiktino ir kalbėjo: “Kam toks eikvojimas? 
\par 9 Juk buvo galima aliejų brangiai parduoti ir išdalyti pinigus vargšams”. 
\par 10 Tai sužinojęs, Jėzus tarė: “Kam skaudinate moterį? Ji man padarė gerą darbą! 
\par 11 Vargšų jūs visuomet turite su savimi, o mane ne visuomet turėsite. 
\par 12 Išpildama aliejų ant mano kūno, ji tai padarė mano laidotuvėms. 
\par 13 Iš tiesų sakau jums: visame pasaulyje, kur tik bus skelbiama ši Evangelija, jos atminimui bus pasakojama ir tai, ką ji padarė”. 
\par 14 Tada vienas iš dvylikos, vardu Judas Iskarijotas, nuėjo pas aukštuosius kunigus 
\par 15 ir tarė: “Ką man duosite, jeigu Jį jums išduosiu?” Tie suderėjo su juo trisdešimt sidabrinių. 
\par 16 Ir nuo to laiko jis ieškojo progos išduoti Jį. 
\par 17 Pirmąją Neraugintos duonos dieną mokiniai priėjo prie Jėzaus ir paklausė: “Kur nori, kad Tau paruoštume valgyti Paschą?” 
\par 18 Jis atsakė: “Eikite į miestą pas tokį žmogų ir sakykite jam: ‘Mokytojas sako: Mano metas arti. Pas tave švęsiu Paschą su savo mokiniais’ ”. 
\par 19 Mokiniai padarė, kaip Jėzus nurodė, ir paruošė Paschą. 
\par 20 Atėjus vakarui, Jėzus su dvylika sėdosi prie stalo. 
\par 21 Jiems bevalgant, Jėzus tarė: “Iš tiesų sakau jums: vienas iš jūsų mane išduos”. 
\par 22 Jie labai nuliūdo ir kiekvienas klausė Jo: “Nejaugi aš, Viešpatie?” 
\par 23 Jis atsakė: “Mane išduos dažantis kartu su manim duoną dubenyje. 
\par 24 Žmogaus Sūnus, tiesa, eina, kaip apie Jį parašyta, bet vargas tam žmogui, per kurį Žmogaus Sūnus išduodamas. Geriau būtų buvę tam žmogui negimti”. 
\par 25 Jo išdavėjas Judas paklausė: “Nejaugi aš, Rabi?!” Jis atsakė: “Tu pasakei”. 
\par 26 Kai jie valgė, Jėzus paėmė duoną, palaimino, laužė ją ir davė mokiniams, sakydamas: “Imkite ir valgykite: tai yra mano kūnas”. 
\par 27 Po to paėmė taurę, padėkojo ir davė jiems, tardamas: “Gerkite iš jos visi, 
\par 28 nes tai yra mano kraujas, Naujosios Sandoros kraujas, kuris už daugelį išliejamas nuodėmėms atleisti. 
\par 29 Ir sakau jums: nuo šiol nebegersiu šito vynmedžio vaisiaus iki tos dienos, kada su jumis gersiu jį naują savo Tėvo karalystėje”. 
\par 30 Pagiedoję himną, jie išėjo į Alyvų kalną. 
\par 31 Tada Jėzus jiems kalbėjo: “Šią naktį jūs visi manimi pasipiktinsite, nes parašyta: ‘Ištiksiu piemenį, ir kaimenės avys išsisklaidys’. 
\par 32 Bet kai prisikelsiu, Aš pirma jūsų nueisiu į Galilėją”. 
\par 33 Petras atsiliepė: “Jei net visi Tavimi pasipiktins, aš niekuomet nepasipiktinsiu!” 
\par 34 Jėzus jam tarė: “Iš tiesų sakau tau: šią naktį, dar gaidžiui nepragydus, tu tris kartus manęs išsiginsi”. 
\par 35 Bet Petras tvirtino: “Jei man reikėtų net mirti kartu su Tavimi, aš vis tiek Tavęs neišsiginsiu!” Taip kalbėjo ir visi mokiniai. 
\par 36 Tuomet Jėzus su jais atėjo į vietą, vadinamą Getsemane, ir tarė mokiniams: “Pasėdėkite čia, kol Aš ten nuėjęs melsiuos”. 
\par 37 Pasiėmęs Petrą ir abu Zebediejaus sūnus, pradėjo liūdėti ir sielvartauti. 
\par 38 Tada tarė jiems: “Mano siela mirtinai nuliūdusi. Likite čia ir budėkite su manimi”. 
\par 39 Paėjęs kiek toliau, Jėzus parpuolė veidu į žemę ir meldėsi: “Mano Tėve, jeigu įmanoma, teaplenkia mane ši taurė. Tačiau ne kaip Aš noriu, bet kaip Tu!” 
\par 40 Jis sugrįžo pas mokinius ir, radęs juos miegančius, tarė Petrui: “Nepajėgėte nė vienos valandos pabudėti su manimi? 
\par 41 Budėkite ir melskitės, kad nepatektumėte į pagundymą. Dvasia ryžtinga, bet kūnas silpnas”. 
\par 42 Ir vėl nuėjęs antrą kartą, meldėsi: “Mano Tėve, jei ši taurė negali praeiti mano negerta, tebūnie Tavo valia!” 
\par 43 Sugrįžęs vėl rado juos miegančius, nes jų akys buvo apsunkusios. 
\par 44 Tada, palikęs juos, vėl nuėjo ir trečią kartą meldėsi tais pačiais žodžiais. 
\par 45 Paskui grįžo pas mokinius ir tarė: “Jūs vis dar miegate ir ilsitės... Štai atėjo valanda, ir Žmogaus Sūnus išduodamas į nusidėjėlių rankas. 
\par 46 Kelkitės, eime! Štai mano išdavėjas čia pat”. 
\par 47 Dar Jam tebekalbant, štai pasirodė Judas, vienas iš dvylikos, o su juo didelė minia, ginkluota kalavijais ir vėzdais, pasiųsta aukštųjų kunigų ir tautos vyresniųjų. 
\par 48 Jo išdavėjas nurodė jiems ženklą, sakydamas: “Kurį pabučiuosiu, tai Tas! Suimkite Jį!” 
\par 49 Ir tuojau priėjęs prie Jėzaus, tarė: “Sveikas, Rabi!”, ir pabučiavo Jį. 
\par 50 O Jėzus jam tarė: “Bičiuli, ko atėjai?” Tada jie priėjo, čiupo Jėzų rankomis ir suėmė. 
\par 51 Ir štai vienas iš buvusių su Jėzumi ištiesė ranką, išsitraukė kalaviją, puolė vyriausiojo kunigo tarną ir nukirto jam ausį. 
\par 52 Tuomet Jėzus jam tarė: “Kišk kalaviją atgal, kur buvo, nes visi, kurie griebiasi kalavijo, nuo kalavijo ir žus. 
\par 53 O gal manai, jog negaliu paprašyti savo Tėvą ir Jis bematant neatsiųstų man daugiau kaip dvylika legionų angelų? 
\par 54 Bet kaip tada išsipildytų Raštai, kad šitaip turi įvykti?!” 
\par 55 Tą valandą Jėzus tarė miniai: “Kaip prieš plėšiką išėjote su kalavijais ir vėzdais suimti mane. Aš kasdien sėdėjau šventykloje su jumis ir mokiau, ir manęs nesuėmėte. 
\par 56 Bet viskas šitaip įvyko, kad išsipildytų pranašų Raštai”. Tada visi mokiniai paliko Jį ir pabėgo. 
\par 57 Tie, kurie suėmė Jėzų, nuvedė Jį pas vyriausiąjį kunigą Kajafą, kur buvo susirinkę Rašto žinovai ir vyresnieji. 
\par 58 O Petras sekė Jį iš tolo iki vyriausiojo kunigo rūmų kiemo ir įėjęs atsisėdo su tarnais pasižiūrėti, kaip viskas baigsis. 
\par 59 Tuo tarpu aukštieji kunigai, vyresnieji ir visas sinedrionas ieškojo melagingo parodymo prieš Jėzų, kad galėtų Jį nuteisti mirti, 
\par 60 tačiau nerado, nors buvo išėję į priekį daug melagingų liudytojų. Pagaliau išėjo priekin du 
\par 61 ir tarė: “Šitas sakė: ‘Aš galiu sugriauti Dievo šventyklą ir per tris dienas ją atstatyti’ ”. 
\par 62 Tada atsistojo vyriausiasis kunigas ir paklausė Jėzų: “Tu nieko neatsakai į šituos kaltinimus?!” 
\par 63 Bet Jėzus tylėjo. Tuomet vyriausiasis kunigas Jam tarė: “Prisaikdinu Tave gyvuoju Dievu, kad mums pasakytum, ar Tu esi Kristus, Dievo Sūnus?!” 
\par 64 Jėzus atsakė: “Taip yra, kaip sakai. Bet Aš jums sakau: nuo šiol jūs matysite Žmogaus Sūnų, sėdintį Galybės dešinėje ir ateinantį dangaus debesyse!” 
\par 65 Tada vyriausiasis kunigas persiplėšė drabužius ir sušuko: “Jis piktžodžiauja! Kam dar mums liudytojai?! Štai girdėjote Jo piktžodžiavimą. 
\par 66 Kaip jums atrodo?” Jie atsakė: “Vertas mirties!” 
\par 67 Tada jie ėmė spjaudyti Jam į veidą ir daužyti Jį kumščiais. Kiti mušė Jį per veidą, 
\par 68 sakydami: “Pranašauk mums, Kristau! Kas Tau smogė?” 
\par 69 Tuo metu Petras sėdėjo kieme. Viena tarnaitė priėjo prie jo ir tarė: “Ir tu buvai su Jėzumi Galilėjiečiu”. 
\par 70 Bet jis išsigynė visų akivaizdoje: “Aš nežinau, ką tu sakai”. 
\par 71 Einantį vartų link, jį pastebėjo kita tarnaitė ir kalbėjo aplinkiniams: “Šitas buvo su Jėzumi Nazariečiu”. 
\par 72 Jis ir vėl išsigynė, prisiekdamas: “Aš nepažįstu to žmogaus!” 
\par 73 Po kiek laiko priėjo ten stovėjusieji ir sakė Petrui: “Tikrai tu vienas iš jų, nes tavo tarmė tave išduoda”. 
\par 74 Tada jis ėmė keiktis ir prisiekinėti: “Aš nepažįstu to žmogaus!” Ir tuojau pragydo gaidys. 
\par 75 Petras prisiminė Jėzaus žodžius: “Dar gaidžiui nepragydus, tu tris kartus manęs išsiginsi”. Jis išėjo laukan ir karčiai verkė.



\chapter{27}


\par 1 Rytui išaušus, visi aukštieji kunigai ir tautos vyresnieji nusprendė, kad Jėzus turi būti nužudytas. 
\par 2 Surišę nuvedė ir atidavė Jį valdytojui Poncijui Pilotui. 
\par 3 Pamatęs, jog Jėzus pasmerktas, išdavikas Judas gailėjosi ir nunešė atgal aukštiesiems kunigams ir vyresniesiems trisdešimt sidabrinių, 
\par 4 sakydamas: “Nusidėjau, išduodamas nekaltą kraują”. Tie atsakė: “Kas mums darbo? Tu žinokis!” 
\par 5 Nusviedęs šventykloje pinigus, jis išbėgo ir pasikorė. 
\par 6 Aukštieji kunigai paėmė sidabrinius ir kalbėjo: “Jų negalima dėti į šventyklos iždą, nes tai užmokestis už kraują”. 
\par 7 Jie pasitarė ir už juos nupirko puodžiaus dirvą ateiviams laidoti. 
\par 8 Todėl ta dirva iki šios dienos vadinasi Kraujo dirva. 
\par 9 Tuomet išsipildė, kas buvo parašyta per pranašą Jeremiją: “Ir paėmė trisdešimt sidabrinių­įkainotojo kainą, už kurią buvo Jį suderėję iš Izraelio vaikų,­ 
\par 10 ir atidavė juos už puodžiaus dirvą; taip man Viešpats buvo paskyręs”. 
\par 11 O Jėzus stovėjo valdytojo akivaizdoje. Valdytojas Jį klausė: “Ar Tu esi žydų karalius?” Jėzus atsakė: “Taip yra, kaip sakai”. 
\par 12 Aukštųjų kunigų ir vyresniųjų kaltinamas, Jis nieko neatsakė. 
\par 13 Tada Pilotas klausė: “Ar negirdi, kiek daug jie prieš Tave liudija?” 
\par 14 Bet Jis neatsakė jam nė žodžio ir tuo labai nustebino valdytoją. 
\par 15 Per šventę valdytojas buvo pratęs paleisti miniai vieną kalinį, kurio ji norėdavo. 
\par 16 Tuo metu jie turėjo pagarsėjusį kalinį, vardu Barabą. 
\par 17 Todėl, žmonėms susirinkus, Pilotas klausė: “Kurį norite, kad jums paleisčiau: Barabą ar Jėzų, vadinamą Kristumi?” 
\par 18 Nes jis žinojo, kad Jėzų jie buvo išdavę iš pavydo. 
\par 19 Sėdinčiam teismo krėsle Pilotui žmona atsiuntė įspėjimą: “Nieko nedaryk šitam teisiajam, nes šiąnakt sapne labai dėl Jo kentėjau”. 
\par 20 Bet aukštieji kunigai ir vyresnieji įtikino minią, kad prašytų paleisti Barabą, o Jėzų pražudytų. 
\par 21 Tada valdytojas jiems tarė: “Kurį iš šių dviejų norite, kad jums paleisčiau?” Jie šaukė: “Barabą!” 
\par 22 Pilotas paklausė: “Ką gi man daryti su Jėzumi, kuris vadinamas Kristumi?” Jie visi rėkė: “Nukryžiuok Jį!” 
\par 23 Jis klausė: “Ką bloga Jis padarė?” Bet jie dar garsiau šaukė: “Nukryžiuok Jį!” 
\par 24 Pilotas, pamatęs, kad nieko nelaimi, o sąmyšis tik didėja, paėmė vandens, nusiplovė rankas minios akivaizdoje ir tarė: “Aš nekaltas dėl šio teisiojo kraujo. Jūs žinokitės!” 
\par 25 Visi žmonės šaukė: “Jo kraujas tekrinta ant mūsų ir ant mūsų vaikų!” 
\par 26 Tada jis paleido jiems Barabą, o Jėzų nuplakdinęs atidavė nukryžiuoti. 
\par 27 Valdytojo kareiviai nusivedė Jėzų į pretorijų ir surinko aplink jį visą kuopą. 
\par 28 Jie išrengė Jį ir apsiautė raudonu apsiaustu. 
\par 29 Nupynę erškėčių vainiką, uždėjo Jam ant galvos, o į Jo dešinę įspraudė nendrę. Po to tyčiodamiesi klūpčiojo prieš Jį ir sakė: “Sveikas, žydų karaliau!” 
\par 30 Spjaudydami Jį, stvėrė iš Jo nendrę ir čaižė per galvą. 
\par 31 Pasityčioję iš Jo, jie nusiautė apsiaustą, apvilko Jo paties drabužiais ir išsivedė nukryžiuoti. 
\par 32 Išeidami jie sutiko žmogų iš Kirėnės, vardu Simoną. Tą privertė nešti Jėzaus kryžių. 
\par 33 Atėję į vietą, vadinamą Golgota (tai yra: “Kaukolės vieta”), 
\par 34 davė Jam gerti rūgštaus vyno, sumaišyto su tulžimi, bet Jėzus paragavęs negėrė. 
\par 35 Nukryžiavę Jį, pasidalijo Jo drabužius, mesdami burtą, kad išsipildytų, kas buvo per pranašą pasakyta: “Jie dalinosi mano drabužius ir dėl mano apdaro metė burtą”. 
\par 36 Ir ten pat susėdę, saugojo Jį. 
\par 37 Viršum Jo galvos jie prisegė užrašytą Jo kaltinimą: “Šitas yra Jėzus, žydų karalius”. 
\par 38 Kartu su Juo buvo nukryžiuoti du plėšikai, vienas iš dešinės, kitas iš kairės. 
\par 39 Einantys pro šalį užgauliojo Jėzų, kraipydami galvas 
\par 40 ir sakydami: “Še Tau, kuris sugriauni šventyklą ir per tris dienas atstatai; išgelbėk save! Jei esi Dievo Sūnus, nuženk nuo kryžiaus!” 
\par 41 Taip pat tyčiojosi aukštieji kunigai su Rašto žinovais ir vyresniaisiais, kalbėdami: 
\par 42 “Kitus išgelbėdavo, o pats negali išsigelbėti. Jeigu Jis Izraelio karalius, tenužengia dabar nuo kryžiaus, ir mes Juo tikėsime. 
\par 43 Jis pasitikėjo Dievu, tad teišvaduoja Tas dabar Jį, jeigu Jam Jo reikia, nes Jis sakė: ‘Aš Dievo Sūnus’ ”. 
\par 44 Taip pat Jį užgauliojo ir kartu nukryžiuoti plėšikai. 
\par 45 Nuo šeštos iki devintos valandos visą kraštą gaubė tamsa. 
\par 46 O apie devintą valandą Jėzus sušuko garsiu balsu: “Elí, Elí, lemá sabachtáni?”, tai reiškia: “Mano Dieve, mano Dieve, kodėl mane palikai?!” 
\par 47 Kai kurie iš ten stovėjusiųjų, tai išgirdę, sakė: “Jis šaukiasi Elijo”. 
\par 48 Ir tuoj vienas iš jų pribėgęs paėmė kempinę, primirkė ją rūgštaus vyno, užmovė ant nendrės ir padavė Jam gerti. 
\par 49 Kiti kalbėjo: “Liaukis! Pažiūrėsim, ar ateis Elijas Jo išgelbėti”. 
\par 50 Tada Jėzus, dar kartą sušukęs garsiu balsu, atidavė dvasią. 
\par 51 Ir štai šventyklos uždanga perplyšo pusiau nuo viršaus iki apačios, ir žemė sudrebėjo, ir uolos ėmė skeldėti. 
\par 52 Atsidarė kapai, ir daug užmigusių šventųjų kūnų prisikėlė. 
\par 53 Išėję iš kapų po Jo prisikėlimo, jie atėjo į šventąjį miestą ir daug kam pasirodė. 
\par 54 Šimtininkas ir kiti su juo saugojantys Jėzų, pamatę žemės drebėjimą ir visa, kas dėjosi, labai išsigando ir sakė: “Tikrai šitas buvo Dievo Sūnus!” 
\par 55 Tenai buvo daug moterų, kurios žiūrėjo iš tolo. Jos sekė paskui Jėzų nuo Galilėjos, Jam tarnaudamos. 
\par 56 Tarp jų buvo Marija Magdalietė, Jokūbo ir Jozės motina Marija ir Zebediejaus sūnų motina. 
\par 57 Vakarui atėjus, atvyko vienas turtingas žmogus iš Arimatėjos, vardu Juozapas, kuris irgi buvo tapęs Jėzaus mokiniu. 
\par 58 Jis nuėjo pas Pilotą ir paprašė Jėzaus kūno. Pilotas įsakė kūną atiduoti. 
\par 59 Juozapas paėmė kūną, įvyniojo į švarią drobulę 
\par 60 ir paguldė savo naujame kape, kurį buvo išsikaldinęs uoloje. Užritęs didelį akmenį ant kapo angos, nuėjo. 
\par 61 Ten buvo Marija Magdalietė ir kita Marija, kurios sėdėjo priešais kapą. 
\par 62 Kitą dieną, po Prisirengimo dienos, susirinko pas Pilotą aukštieji kunigai bei fariziejai 
\par 63 ir kalbėjo: “Valdove, mes prisimename, jog tas suvedžiotojas, dar gyvas būdamas, sakė: ‘Po trijų dienų prisikelsiu’. 
\par 64 Įsakyk tad saugoti kapą iki trečios dienos, kad kartais nakčia atėję Jo mokiniai nepavogtų Jo ir nepaskelbtų žmonėms: ‘Jis prisikėlė iš numirusių’. Pastaroji apgavystė būtų blogesnė už pirmąją”. 
\par 65 Pilotas jiems atsakė: “Štai jums sargyba­eikite ir saugokite, kaip išmanote”. 
\par 66 Jie nuėjo, pastatė prie kapo sargybą ir paženklino antspaudu akmenį.



\chapter{28}


\par 1 Sabatui pasibaigus, auštant pirmajai savaitės dienai, Marija Magdalietė ir kita Marija atėjo pažiūrėti kapo. 
\par 2 Ir štai kilo smarkus žemės drebėjimas, nes Viešpaties angelas, nužengęs iš dangaus, atėjo, nurito akmenį nuo angos, ir atsisėdo ant jo. 
\par 3 Jo išvaizda buvo tarsi žaibo, o drabužiai balti kaip sniegas. 
\par 4 Išsigandę jo, sargybiniai ėmė drebėti ir sustingo lyg negyvi. 
\par 5 O angelas tarė moterims: “Nebijokite! Aš žinau, kad ieškote Jėzaus, kuris buvo nukryžiuotas. 
\par 6 Jo čia nėra! Jis prisikėlė, kaip buvo sakęs. Įeikite, apžiūrėkite vietą, kur Viešpats gulėjo. 
\par 7 Ir skubiai duokite žinią Jo mokiniams: ‘Jis prisikėlė iš numirusių ir eina pirma jūsų į Galilėją; tenai Jį pamatysite’. Štai aš jums tai pasakiau”. 
\par 8 Jos skubiai paliko kapą, apimtos išgąsčio bei didelio džiaugsmo, ir bėgo pranešti mokiniams. 
\par 9 Joms beeinant pranešti Jo mokiniams, štai Jėzus sutiko jas ir tarė: “Sveikos!” Jos priėjo, apkabino Jo kojas ir pagarbino Jį. 
\par 10 Jėzus joms tarė: “Nebijokite! Eikite ir pasakykite mano broliams, kad keliautų į Galilėją; ten jie mane pamatys”. 
\par 11 Joms beeinant, keli sargybiniai atbėgo į miestą ir pranešė aukštiesiems kunigams, kas įvyko. 
\par 12 Tie susitiko su vyresniaisiais, pasitarę davė kareiviams daug pinigų 
\par 13 ir primokė: “Sakykite, kad, jums bemiegant, Jo mokiniai atėję naktį Jį pavogė. 
\par 14 O jeigu apie tai išgirstų valdytojas, mes jį įtikinsime ir apsaugosime jus nuo nemalonumų”. 
\par 15 Šie, paėmę pinigus, taip ir padarė, kaip buvo pamokyti. Šis žodis yra pasklidęs tarp žydų iki šios dienos. 
\par 16 Vienuolika mokinių nuvyko į Galilėją, ant kalno, kurį jiems buvo nurodęs Jėzus. 
\par 17 Jį pamatę, jie pagarbino Jį, tačiau kai kurie abejojo. 
\par 18 Tuomet priėjęs Jėzus jiems pasakė: “Man duota visa valdžia danguje ir žemėje. 
\par 19 Todėl eikite ir padarykite mano mokiniais visų tautų žmones, krikštydami juos Tėvo ir Sūnaus, ir Šventosios Dvasios vardu, 
\par 20 mokydami juos laikytis visko, ką tik esu jums įsakęs. Ir štai Aš esu su jumis per visas dienas iki pasaulio pabaigos. Amen”.



\end{document}