\begin{document}

\title{Michėjo knyga}

\chapter{1}


\par 1 Viešpaties žodis, atėjęs Michėjui iš Morešeto Judo karalių Joatamo, Ahazo ir Ezekijo dienomis apie Samariją ir Jeruzalę. 
\par 2 Klausykite, visos tautos, įsidėmėk, žeme ir visa, kas joje yra. Viešpats Dievas tebūna liudytojas prieš jus iš savo šventyklos. 
\par 3 Štai Viešpats leidžiasi iš savo vietos ir žengs žemės aukštumomis. 
\par 4 Kalnai po Juo tirps ir slėniai lydysis lyg vaškas nuo ugnies, lyg vandenys, tekantys nuo skardžio. 
\par 5 Visa tai dėl Jokūbo nusikaltimo ir Izraelio nuodėmių. Kas yra Jokūbo nusikaltimas? Ar ne Samarija? Kur yra Judo namų aukštumos? Ar ne Jeruzalėje? 
\par 6 “Aš paversiu Samariją griuvėsių krūva, padarysiu ją vynuogynų lauku, nuritinsiu jos akmenis į slėnį ir atidengsiu jos pamatus. 
\par 7 Visi jos drožiniai bus sudaužyti ir sudeginti, visi stabai sunaikinti. Jie buvo įgyti iš paleistuvių užmokesčio ir jie vėl taps paleistuvių užmokesčiu”. 
\par 8 Aš raudosiu ir dejuosiu, vaikščiosiu basas ir nuogas, kauksiu kaip šakalai, dejuosiu lyg pelėdos, 
\par 9 nes jos žaizda nepagydoma, ji pasiekė Judą, palietė mano tautos vartus­Jeruzalę. 
\par 10 Neskelbkite Gate, neverkite garsiai! Bet Leafroje voliokitės dulkėse! 
\par 11 Išeikite visi, Šafyro gyventojai, gėdingai apnuoginti! Caanano gyventojai, pasilikite mieste, neišeikite! Bet Ecelyje raudos visi, netekę savo vietos. 
\par 12 Ko gero gali laukti Maroto gyventojai? Juk nelaimė atėjo nuo Viešpaties ir pasiekė Jeruzalės vartus. 
\par 13 Lachišo miesto gyventojai, kinkykite žirgus ir bėkite! Jūs buvote Siono dukteriai nuodėmės pradžia, nes jumyse atrasti Izraelio nusikaltimai. 
\par 14 Todėl jūs duosite dovanas Morešet Gatui. Achzibo namai taps apgaule Izraelio karaliams. 
\par 15 Marešos gyventojai, jums atvesiu paveldėtoją, jis nueis iki Adulamo­Izraelio šlovės. 
\par 16 Nusikirpk ir nusiskusk plaukus, liūdėdamas savo mylimų vaikų, tapk plikas kaip erelis, nes jie išvesti iš tavęs į nelaisvę.


\chapter{2}


\par 1 Vargas tiems, kurie planuoja neteisybes ir piktybę gulėdami lovose, kad jas įvykdytų rytui išaušus, nes jie turi tam galios. 
\par 2 Jie užsigeidžia laukų­ir pagrobia juos, namų­ir atima juos. Jie apiplėšia žmogų ir jo namus, savininką ir jo nuosavybę. 
\par 3 Todėl Viešpats taip sako: “Ruošiu šitai giminei nelaimę ir jos nenusimesite nuo savo sprandų, ir nevaikščiosite, iškėlę galvas, nes laikas bus piktas”. 
\par 4 Tada sakys apie jus patarlę ir užtrauks graudulingą raudą: “Mes esame visiškai sunaikinti, mūsų tautos nuosavybė atiteko svetimiems. Jis atima žemę, išdalina mūsų laukus kitiems”. 
\par 5 Todėl tu neturėsi nė vieno, kuris, mesdamas burtą, matuotų tau dalį Viešpaties bendruomenėje. 
\par 6 Jie šaukia pranašams: “Nepranašaukite mums taip! To niekad nebus!” 
\par 7 Jūs, vadinami Jokūbo namais! Argi Viešpaties dvasia apribota? Argi tokie Jo darbai? Argi mano žodžiai nėra geri tiems, kurie elgiasi teisingai? 
\par 8 Tie, kurie buvo mano tauta, pakilo kaip priešai! Jūs nutraukiate apsiaustą nuo ramiai einančių keliu, visai negalvojančių apie karą. 
\par 9 Mano tautos moteris išvarėte iš jų gražių namų, iš jų vaikų amžiams atėmėte mano šlovę. 
\par 10 Išeikite, nes ši šalis nėra poilsio vieta. Ji buvo sutepta, todėl ji bus žiauriai sunaikinta. 
\par 11 Jei kas meluotų ir pataikautų, kalbėdamas apie vyną ir stiprius gėrimus, tas būtų priimtinas pranašas šitiems žmonėms! 
\par 12 Aš surinksiu tave, Jokūbai, visą Izraelio likutį, surinksiu kaip avis garde, kaip bandą ganykloje, ir girdėsis žmonių minios klegėjimas. 
\par 13 Pralaužėjas eis priekyje jų, jie prasilauš ir išeis pro vartus; jų karalius eis pirma jų, o Viešpats visų priekyje.



\chapter{3}


\par 1 Aš tariau: “Klausykite jūs, Jokūbo vadai, Izraelio namų kunigaikščiai! Ar ne jūsų reikalas žinoti, kas teisinga? 
\par 2 Tačiau jūs nekenčiate gero, o mylite pikta; nulupate odą nuo mano žmonių ir mėsą nuo jų kaulų. 
\par 3 Jūs ėdate mano tautos kūną, nulupate jiems odą, sulaužote jų kaulus, sukapojate į gabalus kaip puodui, kaip mėsą katilui”. 
\par 4 Tada jie šauksis Viešpaties, bet Jis neatsakys jiems. Jis paslėps savo veidą nuo jų, nes jie elgėsi piktai. 
\par 5 Taip sako Viešpats apie pranašus, kurie suvedžioja mano tautą: “Jei jie turi ką kramtyti dantimis, jie šaukia: ‘Taika!’, o tiems, kurie jiems nieko neduoda, jie skelbia karą. 
\par 6 Todėl jums užeis naktis, kad neturėtumėte regėjimų, ir tamsa, kad negalėtumėte burti. Saulė nusileis pranašams, jų dienos aptems. 
\par 7 Tada regėtojai bus sugėdinti ir būrėjai užsičiaups, nes nebus atsakymo iš Dievo”. 
\par 8 Tačiau aš esu pilnas jėgos iš Viešpaties dvasios, teisybės bei drąsos skelbti Jokūbui jo nusikaltimą ir Izraeliui jo nuodėmę. 
\par 9 Klausykite, Jokūbo namų vadai, Izraelio namų kunigaikščiai, kurie paminate teisingumą ir iškraipote tiesą, 
\par 10 kurie statote Sioną krauju ir Jeruzalę neteisybe! 
\par 11 Jos vadai teisia už kyšius, kunigai moko už atlyginimą, pranašai pranašauja už pinigus, tačiau jie remiasi Viešpačiu, sakydami: “Argi Viešpaties nėra tarp mūsų? Nelaimė mūsų neužklups!” 
\par 12 Dėl jūsų kaltės Sionas bus ariamas laukas, Jeruzalė pavirs griuvėsiais, o šventyklos kalnas apaugs mišku.



\chapter{4}


\par 1 Paskutinėmis dienomis Viešpaties namų kalnas tvirtai stovės kaip aukščiausias kalnas. Jis iškils aukščiau nei kalvos, tautos plauks į jį. 
\par 2 Daug tautų ateis ir sakys: “Ateikite, eikime visi į Viešpaties kalną ir į Jokūbo Dievo namus, kad Jis mus mokytų savo kelių ir mes vaikščiotume Jo takais!” Iš Siono išeis įstatymas, o Viešpaties žodis­iš Jeruzalės. 
\par 3 Jis teis daugelį tautų, sudraus toli esančias galingas tautas. Tada jos perkals savo kardus į noragus ir ietis į pjautuvus. Tauta nebekels kardo prieš tautą ir jie nebesimokys kariauti. 
\par 4 Kiekvienas saugiai sėdės po savo vynmedžiu ir po savo figmedžiu, niekieno negąsdinamas, nes kareivijų Viešpats taip kalbėjo. 
\par 5 Nes visos tautos vaikšto kiekviena savo dievo vardu, tačiau mes vaikščiosime Viešpaties, savo Dievo, vardu visados ir per amžius. 
\par 6 “Tuomet,­sako Viešpats,­Aš surinksiu iš tremties raišuosius, išblaškytuosius ir tuos, kuriuos varginau. 
\par 7 Aš padarysiu raišuosius likučiu, toli išsklaidytuosius­galinga tauta. Viešpats bus jų karalius Siono kalne per amžius. 
\par 8 Tu, kaimenės bokšte, Siono dukters tvirtove, tau bus sugrąžinta ankstesnė valdžia, Jeruzalės dukrai sugrįš karalystė!” 
\par 9 Kodėl dabar taip garsiai šauki? Ar neturi karaliaus? Ar žuvo tavo patarėjas? Skausmai suėmė tave lyg gimdyvę. 
\par 10 Raitykis ir dejuok iš skausmo kaip gimdyvė, Siono dukra! Tu turėsi palikti miestą ir gyventi atvirame lauke, tu nueisi iki Babilono. Ten tu būsi išgelbėta, nes Viešpats išpirks tave iš priešo rankos. 
\par 11 Susirinkusios prieš tave tautos sako: “Ji tebūna išniekinta, tepasigėri Sionu mūsų akys!” 
\par 12 Bet jos nežino Viešpaties minčių ir nesupranta Jo sprendimų. Jis jas surinks kaip pėdus į klojimą. 
\par 13 Siono dukra, pakilk ir kulk. Aš padarysiu tavo ragą geležinį, o kanopas varines. Tu sudaužysi daug tautų, pašvęsi Viešpačiui, viso pasaulio valdovui, jų grobį ir jų turtus.



\chapter{5}


\par 1 Dabar surink kariuomenę, kad kariautų su priešu, kuris apgulė mus. Priešas trenks į veidą Izraelio teisėjui! 
\par 2 O tu, Efrata­Betliejau, nors esi mažas tarp Judo miestų, bet iš tavęs kils Tas, kuris bus valdovu Izraelyje. Jo kilmė siekia pradžios laikus, amžinybės dienas. 
\par 3 Jis juos atiduos į vergiją, kol gimdančioji pagimdys. Tada Jo brolių likutis sugrįš į Izraelį. 
\par 4 Jis stovės ir ganys juos Viešpaties galia ir Viešpaties, savo Dievo, vardo didybe. Jie saugiai gyvens, nes Jis bus didis iki žemės pakraščių. 
\par 5 Jis bus taika. Kai asirai įžengs į mūsų kraštą, mes pastatysime prieš juos septynis ganytojus ir aštuonis kunigaikščius. 
\par 6 Jie nuniokos Asirijos ir Nimrodo kraštą kardu. Taip Jis išlaisvins mus, asirams įsiveržus į mūsų kraštą. 
\par 7 Jokūbo likutis tarp daugelio tautų bus kaip rasa, kuri ateina iš Viešpaties, kaip lietus augalams, kuris nieko nelauks iš žmonių ir nedės vilčių į žmonių sūnus. 
\par 8 Jokūbo likutis tarp tautų bus kaip liūtas tarp laukinių žvėrių, kaip jauniklis liūtas avių bandose, kuris įsibrovęs mindo, drasko, ir nėra galinčio išgelbėti. 
\par 9 Tu pakelsi savo ranką prieš prispaudėjus, ir visi tavo priešai žus! 
\par 10 “Tą dieną,­sako Viešpats,­Aš išnaikinsiu tavo žirgus, sudaužysiu kovos vežimus; 
\par 11 sunaikinsiu miestus krašte ir sugriausiu tvirtoves; 
\par 12 išnaikinsiu burtininkus tavo krašte, ir nebebus tavyje žynių. 
\par 13 Aš išnaikinsiu jūsų drožinius ir statulas, kad nebegarbintumėte savo rankų darbų; 
\par 14 išrausiu giraites tavo krašte ir nusiaubsiu miestus. 
\par 15 Užsirūstinęs ir įtūžęs atkeršysiu tautoms, kurios neklausė manęs”.



\chapter{6}


\par 1 Klausykite, ką sako Viešpats: “Susirinkite, bylinėkitės kalnų akivaizdoje, kalvos tegirdi jūsų balsą!” 
\par 2 Girdėkite, kalnai, Viešpaties bylą, jūs, tvirtieji žemės pamatai, Viešpaties bylą su Jo tauta­Izraeliu: 
\par 3 “Mano tauta, ką tau padariau ir kuo apsunkinau? Paliudyk prieš mane. 
\par 4 Aš juk išvedžiau tave iš Egipto šalies, iš jos vergijos išpirkau tave; pasiunčiau Mozę, Aaroną ir Mirjamą išlaisvinti tave. 
\par 5 Mano tauta, atsimink, ką buvo sumanęs Balakas, Moabo karalius, ir ką jam atsakė Balaamas, Beoro sūnus. Pagalvok, kas vyko nuo Šitimų iki Gilgalo, kad pažintum Viešpaties teisumą”. 
\par 6 Su kuo man ateiti pas Viešpatį, nusilenkti aukštybių Dievui? Ar ateiti su metinių veršių deginamosiomis aukomis? 
\par 7 Ar Viešpats priims tūkstančius avinų ir daugybę aliejaus? O gal man atiduoti savo pirmagimį už nusikaltimus, savo kūno vaisių už savo nuodėmę? 
\par 8 Tau, žmogau, Jis pasakė, kas gera ir ko Viešpats reikalauja iš tavęs: teisingai elgtis, būti gailestingam ir vaikščioti nuolankiai su Dievu. 
\par 9 Viešpaties balsas šaukia miestui (išmintinga yra bijotis Tavo vardo): “Klausykite skeptro ir To, kuris jį paskyrė! 
\par 10 Ar dar tebėra nedorėlio namuose nedorybės turtai ir pasibjaurėtinai sumažintas saikas? 
\par 11 Ar laikyti nekaltais tuos, kurie naudoja neteisingas svarstykles ir apgaulingus svarsčius? 
\par 12 Turtuoliai pilni smurto, gyventojai kalba melą ir apgaulę. 
\par 13 Taigi ir Aš tave, Izraeli, bausiu dėl tavo nuodėmių! 
\par 14 Tu valgysi, bet nepasisotinsi­ liksi alkanas. Ką sutaupysi, to neturėsi, o jei ką išsaugosi­karai sunaikins. 
\par 15 Tu sėsi, bet nepjausi; spausi alyvas, bet nesitepsi aliejumi; spausi vynuoges, bet negersi vyno. 
\par 16 Jūs vykdote Omrio ir Ahabo nuostatus ir elgiatės pagal jų patarimus. Todėl sunaikinsiu tave, o tavo gyventojai taps pajuoka. Jūs kentėsite mano tautos paniekinimą”.



\chapter{7}


\par 1 Vargas man, kaip nuskynus vasaros vaisius, kaip nurinkus vynuogių likučius: nėra jokios kekės pavalgyti nė nunokusių figų, kurias taip mėgstu! 
\par 2 Dingo geri žmonės krašte, nebėra dorų žmonių. Jie visi tykoja kraujo, spendžia pinkles broliui. 
\par 3 Jų rankos yra stropios piktam. Kunigaikštis ir teisėjas reikalauja dovanų, o turtuolis reiškia savo nedorus norus. Taip jie iškraipo teisingumą. 
\par 4 Jų geriausias yra kaip usnis, o doriausias kaip erškėčių tvora. Sąmyšio diena artėja, pranašų žodžiai pildosi. 
\par 5 Netikėkite artimu, nepasitikėkite draugu! Saugokis net tos, kuri guli prie tavo šono. 
\par 6 Sūnus niekina tėvą, duktė sukyla prieš motiną, marti prieš anytą; žmogaus priešai yra jo paties namiškiai. 
\par 7 O aš ieškosiu Viešpaties, lauksiu savo išgelbėjimo Dievo! Mano Dievas išklausys mane. 
\par 8 Nesidžiauk, mano prieše! Nors suklupau­atsikelsiu, nors sėdžiu tamsoje­Viešpats yra mano šviesa. 
\par 9 Viešpaties rūstybę kęsiu, nes Jam nusidėjau. Jis išspręs mano bylą ir bus man teisingas­Jis išves mane į šviesą ir aš matysiu Jo teisumą. 
\par 10 Mano priešai tai matys ir bus sugėdinti. Jie kalbėjo: “Kur yra Viešpats, tavo Dievas?” Aš matysiu juos sutryptus gatvių purve. 
\par 11 Ateina dienos tavo sienoms atstatyti, dienos tavo riboms praplėsti. 
\par 12 Tą dieną ateis pas tave iš visur: nuo Asūro iki Egipto, nuo Tyro iki Eufrato, nuo jūros iki jūros ir nuo kalno iki kalno. 
\par 13 Tačiau kraštas virs dykyne dėl jo gyventojų, dėl jų darbų vaisiaus. 
\par 14 Ganyk savo tautą, savo paveldo avis, kurios vienišos gyvena Karmelio miške. Tegul jos ganosi Bašano ir Gileado vešliuose laukuose kaip senomis dienomis. 
\par 15 Parodysiu jums nuostabių dalykų, kaip jums išeinant iš Egipto krašto. 
\par 16 Tautos, tai matydamos, susigės dėl savo jėgos: nebenorės nei girdėti, nei kalbėti. 
\par 17 Jos ris dulkes kaip gyvatės, kaip žemės kirmėlės išlįs iš savo skylių. Jos bijos Viešpaties, mūsų Dievo, ir drebės dėl Tavęs. 
\par 18 Kur yra toks Dievas kaip Tu, kuris atleidžia kaltę savo išrinktosios tautos likučiui? Jis nerūstauja per amžius, nes Jam patinka gailestingumas. 
\par 19 Jis pasigailės mūsų, sunaikins nusikaltimus ir paskandins jūros gelmėse visas mūsų nuodėmes. 
\par 20 Tu parodysi ištikimybę Jokūbui ir gailestingumą Abraomui, kaip su priesaika pažadėjai mūsų tėvams senomis dienomis.



\end{document}