\begin{document}

\title{Ozėjo knyga}

\chapter{1}


\par 1 Viešpaties žodis buvo suteiktas Beerio sūnui Ozėjui Judo karalių Uzijo, Jotamo, Ahazo ir Ezekijo dienomis ir Jehoašo sūnaus Jeroboamo, Izraelio karaliaus, dienomis. 
\par 2 Viešpaties žodžių Ozėjui pradžia. Viešpats tarė: “Eik, imk sau žmoną paleistuvę ir susilauk iš jos paleistuvių vaikų, nes kraštas padarė didelę paleistuvystę, atitoldamas nuo Viešpaties”. 
\par 3 Jis vedė Gomerą, Diblaimo dukterį. Ji pastojo ir pagimdė sūnų. 
\par 4 Viešpats tarė pranašui: “Tebūna jo vardas Jezreelis, nes netrukus nubausiu Jehuvo namus dėl Jezreelyje pralieto kraujo ir padarysiu galą Izraelio karalystei. 
\par 5 Tą dieną sulaužysiu Izraelio lanką Jezreelio slėnyje”. 
\par 6 Gomera vėl pastojo ir pagimdė dukterį. Viešpats tarė jam: “Jos vardas tebūna Lo Ruhama, nes Aš daugiau nebepasigailėsiu Izraelio namų ir jiems nebeatleisiu. 
\par 7 Bet Aš pasigailėsiu Judo namų ir išgelbėsiu juos Viešpatyje, jų Dieve, ne lankais, kardais, karu, žirgais ar raiteliais”. 
\par 8 Gomera, nustojusi maitinti Lo Ruhamą, pastojo ir pagimdė sūnų. 
\par 9 Viešpats tarė: “Vadink jį Lo Amiu, nes jūs ne mano tauta ir Aš nebūsiu jūsų Dievas. 
\par 10 Bet Izraelio palikuonių bus kaip jūros smilčių, kurių negalima išmatuoti ar suskaičiuoti. Ir toje vietoje, kur jiems buvo pasakyta: ‘Jūs ne mano tauta’, juos vadins gyvojo Dievo vaikais. 
\par 11 Judo ir Izraelio vaikai susirinks, paskirs sau vieną vadą ir išeis iš krašto. Didi bus Jezrahelio diena”.


\chapter{2}


\par 1 “Vadinkite savo brolius ‘Mano tauta’ ir savo seseris ‘Ta, kurios pagailėjo’. 
\par 2 Darykite teismą savo motinai, teiskite ją, nes ji ne mano žmona ir Aš ne jos vyras! Teatsisako ji paleistuvystės ir svetimavimo, 
\par 3 kad nenurengčiau jos nuogai ir nepastatyčiau tokios, kokia ji buvo tą dieną, kai gimė, kad nepadaryčiau jos panašios į dykumą, į išdžiūvusią žemę, ir nenumarinčiau jos troškuliu. 
\par 4 Nepasigailėsiu ir jos vaikų, nes jie paleistuvystės vaikai. 
\par 5 Jų motina elgėsi begėdiškai. Ji sakė: ‘Aš seksiu savo meilužius, duodančius man duonos, vandens, vilnų, linų, aliejaus ir gėrimų’. 
\par 6 Aš užtversiu jos kelią erškėčiais, pastatysiu sieną, kad ji neberastų savo takų. 
\par 7 Ji bėgs paskui savo meilužius, bet jų nepavys, ji ieškos jų, bet nesuras. Tada ji sakys: ‘Grįšiu pas savo pirmąjį vyrą, nes pas Jį man buvo geriau negu dabar’. 
\par 8 Ji nesuprato, kad Aš jai daviau javų, vyno, aliejaus ir parūpinau daug sidabro bei aukso, kurį ji panaudojo Baalui! 
\par 9 Todėl Aš sugrįšiu ir atimsiu iš jos javus jų metu, vyną jo metu ir vilnas bei linus, kuriais ji dengė savo kūną. 
\par 10 Dabar Aš atidengsiu jos gėdą meilužių akyse, ir niekas jos neišgelbės iš mano rankos. 
\par 11 Padarysiu galą jos džiaugsmui, puotoms, sabatams, jauno mėnulio ir metinėms šventėms. 
\par 12 Aš sunaikinsiu jos vynmedžius ir figmedžius, apie kuriuos ji kalbėjo: ‘Tai mano meilužių duotas užmokestis’. Aš tuos sodus paversiu mišku, juose ganysis laukiniai žvėrys. 
\par 13 Bausiu ją dėl švenčių Baalams, nes ji smilkydavo jiems, dabindavosi auskarais ir papuošalais, sekdavo paskui meilužius, o mane pamiršdavo”,­sako Viešpats. 
\par 14 “Aš ją viliosiu, išvesiu į dykumą ir paguosiu ją. 
\par 15 Aš ten jai duosiu vynuogynus ir Achoro slėnį­vilties vartus. Ji giedos ten kaip jaunystėje, kaip išėjusi iš Egipto krašto. 
\par 16 Tą dieną,­sako Viešpats,­tu vadinsi mane savo vyru, ir nebevadinsi savo šeimininku. 
\par 17 Aš pašalinsiu Baalų vardus iš jos burnos, ir ji nebeminės jų. 
\par 18 Tuomet padarysiu sandorą su lauko žvėrimis, padangių paukščiais ir žemės ropliais; karą, lanką ir kardą pašalinsiu iš krašto ir leisiu jiems saugiai gyventi. 
\par 19 Aš susižadėsiu su tavimi amžiams, susižadėsiu teisume ir teisingume, malonėje ir gailestingume. 
\par 20 Susižadėsiu su tavimi ištikimybėje, ir tu pažinsi Viešpatį. 
\par 21 Tą dieną Aš išgirsiu,­sako Viešpats.­Aš išgirsiu dangus ir jie išgirs žemę. 
\par 22 Žemė išgirs javus, vyną ir aliejų, ir šie išgirs Jezrahelį. 
\par 23 Aš ją pasėsiu sau tame krašte, pasigailėsiu tos, kurios nebuvo pasigailėta ir sakysiu ne savo tautai: ‘Tu mano tauta’, o jie atsakys: ‘Tu mano Dievas’ ”.



\chapter{3}


\par 1 Viešpats sakė Ozėjui: “Eik ir mylėk turinčią meilužį moterį, svetimautoją, kaip Viešpats myli izraelitus, nors jie ieško kitų dievų ir mėgsta džiovintas vynuoges”. 
\par 2 Aš ją nusipirkau už penkiolika sidabrinių ir pusantro homero miežių 
\par 3 ir sakiau jai: “Pasiliksi pas mane daug dienų, nepaleistuvausi ir neturėsi vyro, ir aš pasiliksiu prie tavęs”. 
\par 4 Izraelitai ilgai bus be karaliaus ir be kunigaikščio, be aukos ir be aukuro, be efodo ir be terafimo. 
\par 5 Galiausiai jie sugrįš ir ieškos Viešpaties ir savo karaliaus Dovydo. Jie pagarbiai artinsis prie Viešpaties ir Jo gerumo.



\chapter{4}


\par 1 Izraelitai, klausykite Viešpaties žodžio. Viešpats kaltina krašto gyventojus, nes nėra šalyje tiesos, gailestingumo nė Dievo pažinimo. 
\par 2 “Šmeižtas, melagystė, žmogžudystė, vagystė ir svetimavimas įsigalėjo krašte, vienas kraujo praliejimas po kito. 
\par 3 Todėl žemė liūdės, visa, kas gyvena joje, nusilps, net laukiniai žvėrys, padangių paukščiai ir jūros žuvys pražus. 
\par 4 Niekas tegul nesiginčija ir tenekaltina kitų, nes tavo tauta yra kaip besiginčijantys su kunigu. 
\par 5 Todėl tu krisi dienos metu, ir pranašai kris su tavimi kaip naktį, ir Aš sunaikinsiu jūsų motiną. 
\par 6 Mano tauta žūsta dėl pažinimo stokos! Kadangi tu atmetei pažinimą, tai ir Aš tave atmesiu, kad nebebūtum mano kunigu. Kadangi pamiršai savo Dievo įstatymą, tai ir Aš pamiršiu tavo vaikus. 
\par 7 Kuo labiau jų daugėjo, tuo daugiau jie nuodėmiavo. Aš jų šlovę pakeisiu į gėdą. 
\par 8 Jie minta mano tautos nuodėme ir trokšta jos nusikaltimo. 
\par 9 Tautai ir kunigams bus tas pats­ Aš bausiu juos už jų kelius, užmokėsiu jiems už jų darbus. 
\par 10 Jie valgys, bet nepasisotins, paleistuvaus, bet jų nepadaugės, nes jie užmiršo Viešpatį. 
\par 11 Paleistuvystės, senas vynas ir jaunas vynas užvaldė jų širdis. 
\par 12 Mano tauta klausia medžio gabalo ir laukia atsakymo iš lazdos. Paleistuvystės dvasia juos suklaidino, ir paleistuvaudami jie paliko Dievą. 
\par 13 Kalnų viršūnėse ir ant kalvų, malonioje ąžuolų, topolių ir terebintų pavėsyje, jie aukoja aukas ir degina smilkalus. Todėl jūsų dukterys paleistuvauja ir sužadėtinės svetimauja. 
\par 14 Aš nebausiu jūsų dukterų už paleistuvavimą nė sužadėtinių už neištikimybę, nes jūs patys paleistuvaujate ir aukojate su kekšėmis. Tauta, neturinti Dievo pažinimo, pražus. 
\par 15 Tu, Izraeli, jau paleistuvauji, tenenusikalsta bent Judas! Neikite į Gilgalą, nekeliaukite į Bet Aveną ir neprisiekite, sakydami: ‘Kaip Viešpats gyvas!’ 
\par 16 Jei Izraelis užsispyręs kaip užsispyrusi karvė, argi Viešpats ganys juos kaip avinėlį plačioje ganykloje? 
\par 17 Efraimas prisirišo prie stabų­ palik jį! 
\par 18 Jie girtuokliauja ir paleistuvauja, jų valdytojai gėdą pamėgo labiau negu garbę. 
\par 19 Vėjas apsupo juos savo sparnais, ir jie gėdysis savo aukų”.



\chapter{5}


\par 1 “Kunigai, išgirskite, Izraelio namai, įsidėmėkite, ir jūs, karaliaus namai, klausykite, nes jūs būsite nuteisti, kadangi tapote spąstais Micpoje ir ištiestu tinklu Tabore. 
\par 2 Maištininkai įgudę žudyti, tačiau Aš juos visus nubausiu. 
\par 3 Aš pažįstu Efraimą, ir Izraelis nepaslėptas nuo manęs. Tu, Efraimai, paleistuvauji ir tu, Izraeli, esi susitepęs. 
\par 4 Jų darbai neleidžia jiems sugrįžti pas savo Dievą. Paleistuvystės dvasia yra juose ir Viešpaties jie nepažino. 
\par 5 Izraelio išdidumas liudija prieš jį patį. Todėl Efraimas ir Izraelis kris dėl savo kaltės, Judas kris su jais. 
\par 6 Su avių ir galvijų bandomis jie eis ieškoti Viešpaties, bet neras­ Jis pasitraukė nuo jų. 
\par 7 Jie buvo neištikimi Viešpačiui, gimdė svetimus vaikus. Vienas mėnuo praris juos su jų dalimis. 
\par 8 Pūskite ragą Gibėjoje, trimituokite Ramoje. Šaukite Bet Avene, gąsdinkite Benjaminą! 
\par 9 Efraimas virs dykuma bausmės dieną. Tą paskelbsiu visoms Izraelio giminėms. 
\par 10 Judo kunigaikščiai tapo kaip perkeliantieji ribą, todėl Aš išliesiu ant jų savo rūstybę kaip vandenį. 
\par 11 Efraimas prispaustas ir palaužtas teismo, nes sekė tuštybę. 
\par 12 Aš būsiu lyg kandis Efraimui, lyg kirminas Judo namams. 
\par 13 Efraimas matė savo ligą, ir Judas­savo žaizdą. Tada kreipėsi Efraimas į Asiriją, siuntė pas didį karalių. Bet jis negali jūsų pagydyti nė žaizdos pašalinti. 
\par 14 Aš esu lyg liūtas Efraimui, lyg jaunas liūtas Judui. Aš, Aš sudraskysiu ir nueisiu, nusinešiu, ir niekas neišgelbės. 
\par 15 Aš pasitrauksiu, kol jie pripažins savo kaltę ir ieškos mano veido. Savo varge jie anksti ieškos manęs”.



\chapter{6}


\par 1 Eikime, sugrįžkime pas Viešpatį: Jis mus sudraskė­Jis ir pagydys; Jis sumušė­Jis ir aptvarstys. 
\par 2 Jis atgaivins mus po dviejų dienų, trečią dieną pakels, kad gyventume Jo akivaizdoje. 
\par 3 Stenkimės pažinti Viešpatį. Kaip aušra Jis pasirodys ir ateis pas mus kaip lietus, kaip vėlyvas ir ankstyvas lietus į žemę. 
\par 4 “Ką darysiu tau, Efraimai? Ką darysiu tau, Judai? Jūsų gerumas kaip rytmečio migla, kaip rasa, kuri greitai išnyksta. 
\par 5 Aš tašiau juos per pranašus, žudžiau savo burnos žodžiais. Tavo teismai yra kaip nušvintanti šviesa. 
\par 6 Aš noriu gailestingumo, o ne aukos, ir Dievo pažinimo labiau, negu deginamųjų aukų. 
\par 7 Jie kaip žmonės sulaužė sandorą, buvo man neištikimi. 
\par 8 Gileadas yra piktadarių miestas, suteptas krauju. 
\par 9 Kaip tykojanti plėšikų gauja­tokie yra kunigai, kurie žudo einančius į Sichemą ir elgiasi bjauriai. 
\par 10 Izraelio namuose mačiau baisių dalykų: ten paleistuvauja Efraimas, susitepęs Izraelis. 
\par 11 Tau, Judai, taip pat paruošta pjūtis, kai parvesiu savo tautos ištremtuosius”.



\chapter{7}


\par 1 “Kai norėjau pagydyti Izraelį, paaiškėjo Efraimo kaltė ir Samarijos nedorybės. Jie apgaudinėja, vagys įsilaužia į namus, plėšikų gaujos plėšia gatvėse. 
\par 2 Jie nepagalvoja, kad Aš prisimenu jų nedorybes! Dabar jų darbai apsupo juos, jie visi yra mano akivaizdoje. 
\par 3 Jų nedorybėmis džiaugiasi karalius, jų apgaulėmis­kunigaikščiai. 
\par 4 Jie visi yra svetimautojai kaip pakūrenta krosnis, kurios nebereikia kurstyti nuo tešlos įmaišymo iki iškilimo. 
\par 5 Karaliaus dieną kunigaikščiai susirgo nuo vyno, jis ištiesė ranką akiplėšoms. 
\par 6 Jų širdys paruoštos kaip krosnis­kepėjas miega naktį, o rytą ugnis įsiliepsnoja. 
\par 7 Jie visi, įkaitę kaip krosnis, prarijo savo teisėjus. Visi jų karaliai krito, bet nė vienas tarp jų nesišaukia manęs. 
\par 8 Efraimas maišosi su tautomis! Jis yra lyg neapverstas paplotis. 
\par 9 Svetimšaliai suėdė jo jėgą, bet jis to nepastebėjo; plaukai jam pražilo, bet jis to nežinojo. 
\par 10 Izraelio išdidumas liudija prieš jį patį. Bet jie vis dėlto nesugrįžta pas Viešpatį, savo Dievą, ir neieško Jo. 
\par 11 Efraimas elgiasi kaip kvailas balandis: tai šaukiasi Egipto, tai bėga į Asiriją. 
\par 12 Jiems einant, Aš ištiesiu tinklą­kaip padangių paukščius juos pagausiu; bausiu juos, kaip esu jiems sakęs. 
\par 13 Vargas jiems, nes jie pasitraukė nuo manęs! Sunaikinimas jiems, nes jie sukilo prieš mane! Aš išpirkau juos, bet jie kalbėjo melą. 
\par 14 Jie nesišaukė manęs nuoširdžiai, kai dejavo savo guoliuose. Dėl javų ir vyno jie susirinko, bet prieš mane maištavo. 
\par 15 Nors Aš juos mokiau ir stiprinau, tačiau jie piktu man atlygindavo. 
\par 16 Jie kreipėsi, bet ne į Aukščiausiąjį, jie kaip netikras ginklas. Jų kunigaikščiai žus nuo kardo dėl jų akiplėšiškumo, Egipto žemė tyčiosis iš jų”.



\chapter{8}


\par 1 “Pūskite trimitą! Erelis leidžiasi ant Viešpaties namų, nes jie sulaužė mano sandorą ir pažeidė įstatymą. 
\par 2 Izraelis šauksis manęs: ‘Mano Dieve! Mes pažįstame Tave’. 
\par 3 Izraelis atmetė tai, kas gera, priešas persekios jį! 
\par 4 Jie išsirinko karalius be manęs ir paskyrė kunigaikščius be mano žinios. Iš sidabro ir aukso jie pasidirbdino stabų sau patiems sunaikinti. 
\par 5 Samarija, tavo veršis atmetė tave. Mano rūstybė dega prieš tave, ar ilgai dar Izraelis neapsivalys? 
\par 6 Amatininkas padirbo jį, todėl jis ne dievas. Samarijos veršis bus paverstas dulkėmis. 
\par 7 Kas sėja vėją, pjaus audrą. Tuščios varpos neneš grūdų, o jei ir neštų, svetimieji juos surytų. 
\par 8 Izraelis prarytas. Dabar jis bus tarp pagonių kaip nenaudingas indas. 
\par 9 Jis išėjo į Asiriją kaip vienišas laukinis asilas. Efraimas pasisamdė meilužių. 
\par 10 Nors jie samdo svetimšalius, tačiau Aš juos greitai išsklaidysiu, jie nebeturės nei karaliaus, nei kunigaikščių. 
\par 11 Efraimas pasistatė daugybę aukurų, jie padėjo jam nusikalsti. 
\par 12 Aš surašiau jam didžius dalykus savo įstatyme, bet jis palaikė juos svetimais. 
\par 13 Jie aukoja aukas ir valgo mėsą, bet Viešpačiui jos nepatinka. Viešpats atsimins jų kaltes ir baus juos už nuodėmes­jie grįš į Egiptą. 
\par 14 Izraelis užmiršo savo Kūrėją ir statosi šventyklas, o Judas­sutvirtintus miestus. Bet Aš siųsiu ugnį į jų miestus, sudeginsiu jų rūmus”.



\chapter{9}


\par 1 Izraeli, nesidžiauk, nedžiūgauk kaip tautos! Tu paleistuvaudamas palikai savo Dievą, pamėgai užmokestį kiekviename klojime. 
\par 2 Klojimas ir vynuogių spaustuvas jų nemaitins, ir šviežias vynas jiems nepadės. 
\par 3 Jie neliks Viešpaties krašte: Efraimas grįš į Egiptą ir Asirijoje valgys nešvarų maistą. 
\par 4 Jie neaukos Viešpačiui geriamųjų aukų, ir kitos jų aukos Jam nepatiks. Jų aukos bus kaip gedinčiųjų duona; visi, kurie jų valgys, susiteps, nes jų duona tinka tik pasisotinti, bet į Viešpaties namus ji nepateks. 
\par 5 Ką jūs darysite iškilmių dienomis ir Viešpaties šventės dieną? 
\par 6 Kas išliks po sunaikinimo, išeis į Egiptą, Nofe jie bus palaidoti. Jų sidabras apaugs piktžolėmis, ir jų palapinėse augs erškėčiai. 
\par 7 Priartėjo aplankymo dienos ir atsiskaitymo metas. Izraelis tai žino! Pranašas­kvailys, dvasinis žmogus­beprotis! Tai dėl tavo kalčių daugybės, dėl didelės neapykantos. 
\par 8 Efraimo sargas yra su mano Dievu. Bet pranašas­žabangai visuose jo keliuose, neapykanta Dievo namuose. 
\par 9 Žmonės sugedo kaip Gibėjos dienomis. Jis atsimins jų kaltę, baus juos už jų nuodėmes. 
\par 10 Man Izraelis buvo kaip vynuogės dykumoje, jūsų tėvai kaip ankstyvi figmedžio vaisiai. O jie nuėjo pas Baal Peorą, pasišventė gėdai ir tapo pasibjaurėtini kaip ir tie, kuriuos jie pamilo. 
\par 11 Efraimo šlovė nuskrido kaip paukštis­nebėra nei nėštumo, nei gimdymo, nei pradėjimo. 
\par 12 Jeigu jie užaugintų vaikus, Aš iš jų atimsiu juos. Vargas jiems, kai nuo jų pasitrauksiu! 
\par 13 Efraimas mano akyse buvo kaip Tyras, pasodintas geroje vietoje. Tačiau jis atiduos savo vaikus žudikui. 
\par 14 Viešpatie, duok jiems, ką esi numatęs duoti! Duok jiems nevaisingas įsčias ir tuščias krūtis. 
\par 15 Visa jų nedorybė pasirodė Gilgaloje, ten Aš pradėjau jų nebekęsti. Dėl jų piktų darbų pašalinsiu juos iš savo namų, daugiau jų nebemylėsiu, visi jų kunigaikščiai­maištininkai. 
\par 16 Efraimas nubaustas­jų šaknis padžiūvusi, jie nebeneš vaisiaus. Jei ir pagimdytų, nužudysiu jų mylimą įsčių vaisių. 
\par 17 Dievas atmes juos, nes jie neklausė Jo. Jie taps klajūnais tarp tautų.



\chapter{10}


\par 1 Izraelis yra vešlus vynmedis, nešantis vaisių sau. Kuo daugiau jis nešė vaisiaus, tuo daugiau statė aukurų. Kuo labiau turtėjo kraštas, tuo gražesnius darė atvaizdus. 
\par 2 Jų širdis pasidalinusi! Dabar jie atkentės už savo kaltes. Viešpats sulaužys jų aukurus ir sunaikins atvaizdus. 
\par 3 Dabar jie sako: “Mes neturime karaliaus, nes nebijome Viešpaties. O ką gali padaryti mums karalius?” 
\par 4 Jie tuščiai kalba ir klastingai prisiekia, darydami sutartis. Todėl ateis jiems teismas lyg nuodingos piktžolės, kurios užauga lauko vagose. 
\par 5 Dėl Bet Aveno veršio drebės Samarijos gyventojai. Dėl jo tauta liūdės ir jo kunigai, kurie anksčiau juo džiaugėsi, dejuos, nes šlovės nebeliko. 
\par 6 Jis bus nugabentas Asirijon kaip duoklė didžiajam karaliui. Tada bus gėda Efraimui ir Izraelis gėdinsis dėl savo sumanymo. 
\par 7 Samarijos karalius pradings kaip puta nuo vandens paviršiaus. 
\par 8 Aveno aukštumos­Izraelio nuodėmė­bus sunaikintos. Erškėčiai ir usnys augs ant jų aukurų. Jie sakys kalnams ir kalvoms: “Griūkite ant mūsų ir apdenkite mus!” 
\par 9 Izraeli, tu nuodėmiauji nuo Gibėjos dienų. Ten jie pasiliko, mūšis Gibėjoje prieš nedorybės vaikus neužklupo jų. 
\par 10 Aš juos nubausiu. Tautos kariaus prieš juos, jie bus nubausti už dvigubą savo nusikaltimą. 
\par 11 Efraimas­telyčia, išmokyta kulti javus. Aš uždėjau jungą ant jos gražaus kaklo. Aš atsisėsiu ant Efraimo, Judas ars, o Jokūbas akės. 
\par 12 Sėkite sau teisumą, pjaukite gailestingumą. Plėškite dirvoną, nes laikas ieškoti Viešpaties, kad Jis ateitų ir išlietų ant jūsų teisumą. 
\par 13 Jūs arėte nedorybę, pjovėte neteisybę ir valgėte melo vaisius. Kadangi pasitikėjote savo keliais, karžygių daugybe, 
\par 14 kils karas prieš tavo tautą ir visos tvirtovės bus sunaikintos, kaip Šalmanas sunaikino Bet Arbelį kovos metu: motinos buvo sutraiškytos drauge su vaikais. 
\par 15 Taip atsitiks tau, Beteli, dėl tavo nedorybių. Rytą pražus Izraelio karalius.



\chapter{11}


\par 1 “Kai Izraelis buvo jaunas, mylėjau jį ir iš Egipto pašaukiau savo sūnų. 
\par 2 Kuo labiau juos šaukiau, tuo toliau jie traukėsi nuo manęs. Jie aukodavo Baalams ir smilkydavo drožiniams. 
\par 3 Aš mokiau Efraimą vaikščioti, paėmęs už rankos, bet jie nesuprato, kad Aš juos išgelbėjau. 
\par 4 Žmonių virvėmis ir meilės raiščiais traukiau juos. Aš buvau kaip tas, kuris nuima jungą nuo kaklo ir maitina juos. 
\par 5 Jis negrįš į Egiptą, bet Asirijos karalius valdys jį, nes jis nenorėjo sugrįžti. 
\par 6 Kardas siaus jo miestuose, sunaikins užkaiščius ir praris juos dėl jų sprendimų. 
\par 7 Mano tauta linkusi nuklysti nuo manęs. Nors juos kvietė pas Aukščiausiąjį, nė vienas Jo neaukština. 
\par 8 Kaipgi paliksiu tave, Efraimai, kaip atstumsiu tave, Izraeli? Ar galiu tau padaryti kaip Admai, ar galiu su tavimi pasielgti kaip su Ceboimais? Mano širdis suminkštėjo, jaučiu tau gailestį. 
\par 9 Aš nesielgsiu pagal savo rūstybės užsidegimą ir nesunaikinsiu Efraimo. Aš esu Dievas, ne žmogus. Šventasis tarp jūsų, Aš neateisiu naikinti. 
\par 10 Jie seks Viešpatį, o Jis riaumos kaip liūtas. Kai Jis riaumos, sūnūs grįš drebėdami iš vakarų. 
\par 11 Jie atskubės kaip paukščiai iš Egipto, kaip balandžiai iš Asūro. Aš sugrąžinsiu juos į jų namus,­ sako Viešpats,­ 
\par 12 Efraimas apsupo mane melu, o Izraelio namai­apgaule. Bet Judas dar tebėra su Dievu ir lieka ištikimas Šventajam”.



\chapter{12}


\par 1 Efraimas gano vėją ir vejasi rytų vėją, kasdien daugina melus ir sunaikinimą. Jie tariasi su Asirija ir gabena aliejų į Egiptą. 
\par 2 Viešpats nepatenkintas Judu. Jis nubaus Jokūbą už jo kelius, pagal jo darbus atmokės jam. 
\par 3 Dar negimęs, jis laikė už kulnies savo brolį, o subrendęs grūmėsi su Dievu. 
\par 4 Jis kovojo su angelu ir nugalėjo. Jis verkė ir maldavo jį. Betelyje Jis jį surado ir ten Jis kalbėjo su mumis. 
\par 5 Tai Viešpats, kareivijų Dievas. Viešpats­Jo vardas. 
\par 6 Gręžkis į savo Dievą, laikykis gailestingumo bei teisybės ir nuolat lauk Dievo! 
\par 7 Kanaanietis laiko rankoje neteisingas svarstykles, jis mėgsta skriausti. 
\par 8 Efraimas sakė: “Aš pralobau, praturtėjau, daug ką įsigijau ir niekuo nenusikaltau”. 
\par 9 “Aš esu Viešpats, tavo Dievas nuo dienų Egipto krašte; Aš dar kartą apgyvendinsiu tave palapinėse kaip iškilmingos šventės metu. 
\par 10 Aš kalbėjau per pranašus, per regėjimus ir palyginimais”. 
\par 11 Jie Gileade nusikalto ir tapo tuštybe; Gilgaloje aukojo jaučius, todėl jų aukurai pavirs akmenų krūvomis. 
\par 12 Jokūbas pabėgo į Siriją, Izraelis tarnavo už žmoną ir ganė bandą, kad ją gautų. 
\par 13 Per pranašą Viešpats išvedė Izraelį iš Egipto ir per pranašą saugojo jį. 
\par 14 Efraimas labai supykdė Viešpatį, todėl Jis paliks jo kraują ant jo ir jo panieką Viešpats sugrąžins jam.



\chapter{13}


\par 1 Kai Efraimas kalbėjo drebėdamas, jis iškilo Izraelyje, o kai nusikalto Baalu­mirė. 
\par 2 Dabar jie nusideda dar labiau: lieja sidabrinius atvaizdus, stabus pagal savo sugebėjimus. Tai yra amatininkų darbas. Jie ragina: “Žmonės, kurie aukoja, tebučiuoja veršius!” 
\par 3 Todėl jie bus kaip rytmečio migla, kaip rasa, kuri anksti pranyksta, kaip pelai, nupučiami nuo klojimo, arba dūmai iš kamino. 
\par 4 “Aš esu Viešpats, tavo Dievas nuo dienų Egipto krašte. Tu nepažinsi kito dievo, tik mane, nes be manęs nėra gelbėtojo. 
\par 5 Aš pažinau tave dykumoje, išdžiūvusioje žemėje. 
\par 6 Kai jie prasigyveno ir pasisotino, jų širdis išpuiko, jie pamiršo mane. 
\par 7 Aš būsiu jiems kaip liūtas, kaip šalia kelio tykojantis leopardas. 
\par 8 Aš juos užpulsiu kaip lokė, netekusi jauniklių, ir draskysiu jų krūtines. Ten surysiu juos kaip liūtas, sudraskysiu kaip laukiniai žvėrys. 
\par 9 Izraeli, tu sunaikinai save, nes tik manyje tavo pagalba. 
\par 10 Kur yra tavo karalius, kuris tave išgelbėtų? Kur tavo teisėjai, apie kuriuos sakei: ‘Duok man karalių ir kunigaikščių’? 
\par 11 Aš tau daviau karalių supykęs ir atėmiau jį užsirūstinęs. 
\par 12 Efraimo kaltė surišta, jo nuodėmė paslėpta. 
\par 13 Jis yra lyg neišmintingas sūnus­atėjus laikui gimti, jis neturėtų laukti. 
\par 14 Aš išpirksiu juos iš mirusiųjų buveinės galios, išgelbėsiu nuo mirties. Mirtie, Aš būsiu tavo galas, mirusiųjų buveine, Aš būsiu tavo sunaikinimas. Gailestis bus paslėptas nuo mano akių. 
\par 15 Nors jis klestės tarp savo brolių, pakils Viešpaties vėjas iš rytų, iš dykumos, išdžiovins versmes ir šaltinius, išplėš turtus ir visus brangius indus. 
\par 16 Samarija kentės už tai, kad maištavo prieš savo Dievą. Jie žus nuo kardo, jos kūdikius sutraiškys, nėščias moteris perskros”.



\chapter{14}


\par 1 “Izraeli, sugrįžk pas Viešpatį, savo Dievą, nes tu suklupai dėl savo kaltės. 
\par 2 Pripažinkite kaltę ir gręžkitės į Viešpatį, sakydami: ‘Pašalink mūsų kaltes ir priimk mus maloningai, tai aukosime Tau savo lūpų aukas. 
\par 3 Asirija neišgelbės mūsų, nebejosime ant žirgų ir nebesakysime savo rankų darbams: ‘Tu mūsų dievas’. Tik Tu, Viešpatie, pasigaili našlaičių’. 
\par 4 Aš išgydysiu jų paklydimą, gera valia juos mylėsiu, nes mano rūstybė nusigręžė nuo jų. 
\par 5 Aš būsiu kaip rasa Izraeliui, jis žydės kaip lelija ir išleis šaknis kaip Libano kedras. 
\par 6 Jis išsikeros, išleis atžalas, bus gražus kaip alyvmedis ir kvepės kaip Libanas. 
\par 7 Gyvenusieji jo pavėsyje sugrįš, atsigaus kaip javai, žydės lyg vynmedis; jų kvapas bus kaip Libano vyno. 
\par 8 Efraimai, ką Aš turiu bendro su stabais? Aš išklausiau ir pastebėjau tave! Aš esu kaip žaliuojantis kiparisas, iš manęs sulauksi savo vaisiaus. 
\par 9 Išmintingas tai supras, sumanus tai žinos! Viešpaties keliai teisingi, ir teisieji vaikščioja jais, o nusidėjėliai suklumpa”.


\end{document}