\begin{document}

\title{Exodus}


\chapter{1}


\par 1 Šitie yra vardai Izraelio sūnų, kurie atėjo į Egiptą kartu su Jokūbu, kiekvienas su savo šeimomis: 
\par 2 Rubenas, Simeonas, Levis, Judas, 
\par 3 Isacharas, Zabulonas, Benjaminas, 
\par 4 Danas, Neftalis, Gadas ir Ašeras. 
\par 5 Iš Jokūbo kilusių buvo septyniasdešimt sielų. Juozapas jau buvo Egipte. 
\par 6 Ir numirė Juozapas, visi jo broliai ir visa ta karta. 
\par 7 Izraelitai buvo vaisingi, labai pagausėjo, išsiplėtė ir tapo galingi; jie pripildė visą kraštą. 
\par 8 Egiptą pradėjo valdyti naujas karalius, kuris nieko nežinojo apie Juozapą. 
\par 9 Jis kalbėjo savo tautai: “Žiūrėkite, izraelitų tauta yra gausesnė ir galingesnė už mus. 
\par 10 Pasielkime išmintingai su jais, kad jie nebesiplėstų. Jei kiltų karas, kad jie neprisijungtų prie mūsų priešų, nekariautų prieš mus ir nepasitrauktų iš šalies”. 
\par 11 Egiptiečiai paskyrė jiems prižiūrėtojus, kad juos prislėgtų sunkiais darbais. Jie pastatė faraonui sandėlių miestus Pitomą ir Ramzį. 
\par 12 Bet kuo labiau egiptiečiai spaudė juos, tuo labiau jie gausėjo ir plito taip, kad egiptiečiai pradėjo bijoti izraelitų. 
\par 13 Egiptiečiai vertė izraelitus tarnauti be gailesčio, 
\par 14 apkartino jų gyvenimą sunkia vergyste prie molio, plytų bei visokių ūkio darbų ir juos negailestingai spaudė. 
\par 15 Egipto karalius įsakė hebrajų pribuvėjoms, kurių viena vadinosi Šifra ir kita­Pūva: 
\par 16 “Pašauktos prie hebrajų moterų gimdymo, žiūrėkite, kas gims: jei sūnus, nužudykite jį, bet jei duktė­tegyvena!” 
\par 17 Tačiau pribuvėjos bijojo Dievo ir nevykdė Egipto karaliaus įsakymo, bet palikdavo berniukus gyvus. 
\par 18 Egipto karalius, tai sužinojęs, pasišaukė pribuvėjas ir joms tarė: “Kodėl jūs nevykdote įsakymo ir paliekate berniukus gyvus?” 
\par 19 Pribuvėjos atsakė faraonui: “Hebrajų moterys ne tokios kaip egiptietės, jos stiprios ir pagimdo prieš pribuvėjai ateinant”. 
\par 20 Todėl Dievas darė gera pribuvėjoms. Tauta gausėjo ir pasidarė labai galinga. 
\par 21 Kadangi pribuvėjos bijojo Dievo, Jis suteikė joms palikuonių. 
\par 22 Faraonas įsakė visai tautai: “Kiekvieną berniuką, gimusį hebrajams, meskite į upę, o mergaites palikite gyvas!”


\chapter{2}

\par 1 Vienas Levio giminės vyras pasirinko ir vedė tos pačios giminės mergaitę. 
\par 2 Moteris pastojo ir pagimdė sūnų. Kai ji pamatė, koks jis buvo gražus, slėpė jį tris mėnesius. 
\par 3 Negalėdama ilgiau jo slėpti, ji paėmė iš papiruso nendrių nupintą krepšį, ištepė jį derva ir sakais, įdėjo į jį kūdikį ir nunešusi padėjo nendryne prie upės kranto. 
\par 4 Jo sesuo atsistojo kiek toliau, norėdama pamatyti, kas atsitiks. 
\par 5 Faraono duktė atėjo prie upės praustis. Jos tarnaitės vaikščiojo upės pakraščiu. Ji pamatė nendryne pintinę ir pasiuntė tarnaitę ją atnešti. 
\par 6 Atidarius ją, rado verkiantį berniuką. Jai pagailo jo ir ji tarė: “Šitas yra vienas iš hebrajų kūdikių!” 
\par 7 Jo sesuo tarė faraono dukteriai: “Ar norėtum, kad surasčiau hebrajų moterį, kuri maitintų kūdikį?” 
\par 8 Faraono duktė jai atsakė: “Eik”. Mergaitė nuėjus pakvietė kūdikio motiną. 
\par 9 Tada Faraono duktė tarė jai: “Imk šitą kūdikį ir maitink jį! Aš mokėsiu tau už tai”. Moteris paėmė kūdikį ir maitino jį. 
\par 10 Berniukui paaugus, ji nunešė jį faraono dukteriai. Ji įsūnijo berniuką ir pavadino Moze, sakydama: “Aš jį ištraukiau iš vandens!” 
\par 11 Kartą Mozė, jau užaugęs, išėjo pas savo brolius ir pamatė jų naštas. Ir jis pamatė egiptietį bemušantį hebrają, vieną iš jo brolių. 
\par 12 Apsidairęs ir nieko nematydamas, jis užmušė tą egiptietį ir užkasė jį smėlyje. 
\par 13 Išėjęs kitą dieną, pamatė besivaidijančius du hebrajus ir klausė kaltininko: “Kodėl muši savo artimą?” 
\par 14 Tas atsakė: “Kas tave paskyrė mūsų kunigaikščiu ir teisėju? Bene nori ir mane užmušti, kaip užmušei egiptietį?” Mozė nusigando ir sakė: “Iš tiesų tai tapo žinoma!” 
\par 15 Faraonas, išgirdęs apie tą įvykį, norėjo Mozę nužudyti. Bet Mozė pabėgo nuo faraono ir apsigyveno Midjano šalyje. Vieną dieną, sėdint jam prie šulinio, 
\par 16 midjaniečių kunigo septynios dukterys atėjusios sėmė vandenį ir pylė į lovius, norėdamos pagirdyti savo tėvo avis. 
\par 17 Atėję piemenys jas nuvarė. Tada Mozė atsistojo, apgynė mergaites ir pagirdė jų avis. 
\par 18 Joms sugrįžus pas savo tėvą Reuelį, jis paklausė: “Kodėl šiandien taip anksti parėjote?” 
\par 19 Jos atsakė: “Vienas egiptietis mus apgynė nuo piemenų. Be to, jis sėmė vandenį ir pagirdė avis. 
\par 20 Tėvas tarė savo dukterims: “Kur jis yra? Kodėl palikote tą vyrą? Pakvieskite jį valgyti su mumis”. 
\par 21 Mozė sutiko gyventi pas Reuelį ir vedė jo dukterį Ciporą. 
\par 22 Ji pagimdė sūnų, kurį Mozė pavadino Geršomu, sakydamas: “Tapau ateivis svetimoje šalyje”. 
\par 23 Praėjus daug laiko, mirė Egipto karalius. Izraelio vaikai vaitojo dėl vergystės ir šaukė, ir jų šauksmas pasiekė Dievą. 
\par 24 Dievas išgirdo jų vaitojimą ir atsiminė savo sandorą su Abraomu, Izaoku ir Jokūbu. 
\par 25 Ir Dievas pažvelgė į Izraelio vaikus, ir ėmė jais rūpintis.



\chapter{3}

\par 1 Mozė ganė savo uošvio, midjaniečių kunigo Jetro, avis. Kartą jis buvo nusivaręs avis už dykumos, prie Dievo kalno Horebo. 
\par 2 Jam pasirodė Viešpaties angelas ugnies liepsnoje, kylančioje iš krūmo vidurio. Jis matė krūmą degantį, tačiau nesudegantį. 
\par 3 Mozė pasakė: “Eisiu ir pažiūrėsiu į šį didingą reginį. Kodėl nesudega tas krūmas?” 
\par 4 Viešpats pamatė jį artėjant ir pašaukė iš krūmo: “Moze! Moze!” Tas atsiliepė: “Aš čia!” 
\par 5 Jis sakė: “Nesiartink prie šios vietos! Nusiauk nuo kojų apavą, nes vieta, ant kurios stovi, yra šventa žemė! 
\par 6 Aš esu tavo tėvo Dievas, Abraomo, Izaoko ir Jokūbo Dievas”. Mozė užsidengė veidą, nes bijojo pažvelgti į Dievą. 
\par 7 Viešpats tarė: “Aš pamačiau savo tautos vargą Egipte ir išgirdau jos šauksmą dėl jų prižiūrėtojų. Aš žinau jos sielvartą. 
\par 8 Aš nužengiau jos išvaduoti iš egiptiečių rankos ir išvesti jos iš tos šalies į gerą ir plačią šalį, plūstančią pienu ir medumi: į kanaaniečių, hetitų, amoritų, perizų, hivų ir jebusiečių žemę. 
\par 9 Izraelio vaikų šauksmas pasiekė mane. Aš mačiau jų priespaudą, kaip egiptiečiai juos engė. 
\par 10 Taigi dabar eik. Aš siunčiu tave pas faraoną, kad išvestum iš Egipto mano tautą, Izraelio vaikus!” 
\par 11 Mozė atsakė Dievui: “Kas aš, kad eičiau pas faraoną ir išvesčiau iš Egipto izraelitus?” 
\par 12 Dievas atsakė: “Aš būsiu su tavimi! Štai ženklas, kad Aš tave siunčiu: kai išvesi tautą iš Egipto, jūs tarnausite Dievui ant šito kalno”. 
\par 13 Mozė klausė Dievo: “Kai aš ateisiu pas izraelitus ir jiems sakysiu: ‘Jūsų tėvų Dievas mane siuntė pas jus’, jie manęs klaus: ‘Koks yra Jo vardas?’ Ką turiu jiems atsakyti?” 
\par 14 Dievas tarė Mozei: “AŠ ESU, KURIS ESU. Sakyk izraelitams: ‘AŠ ESU mane siuntė pas jus’ ”. 
\par 15 Dievas dar kalbėjo Mozei: “Šitaip sakyk izraelitams: ‘Viešpats, jūsų tėvų Dievas, Abraomo Dievas, Izaoko Dievas ir Jokūbo Dievas, mane siuntė pas jus’. Toks mano vardas per amžius ir taip mane vadinsite per kartų kartas. 
\par 16 Eik, surink Izraelio vyresniuosius ir jiems sakyk: ‘Viešpats, jūsų tėvų Dievas, Abraomo Dievas, Izaoko Dievas ir Jokūbo Dievas, man pasirodė ir kalbėjo: ‘Aš aplankiau jus ir mačiau, kas jums buvo padaryta Egipte. 
\par 17 Aš išvesiu jus iš Egipto vargo į kanaaniečių, hetitų, amoritų, perizų, hivų ir jebusiečių žemę, plūstančią pienu ir medumi’. 
\par 18 Jie klausys tavo balso. Tu su Izraelio vyresniaisiais nueisite pas Egipto karalių ir jam sakysite: ‘Viešpats, hebrajų Dievas, mus pašaukė, ir dabar prašome: leisk mums tris dienas keliauti į dykumą, aukoti Viešpačiui, savo Dievui’. 
\par 19 Aš žinau, kad Egipto karalius neišleis jūsų, jei nebus priverstas galingos rankos. 
\par 20 Aš ištiesiu savo ranką ir ištiksiu Egiptą visais savo stebuklais, kuriuos darysiu jų tarpe. Po to jis išleis jus. 
\par 21 Ir suteiksiu šiai tautai malonę egiptiečių akyse, kad jūs neišeisite tuščiomis rankomis. 
\par 22 Kiekviena moteris paprašys savo kaimynės ir savo įnamės sidabrinių bei auksinių indų ir drabužių; juos uždėsite ant savo sūnų ir dukterų ir taip išsinešite egiptiečių turtus”.



\chapter{4}


\par 1 Mozė sakė: “O jeigu jie netikės manimi ir neklausys mano balso, sakydami: ‘Tau nepasirodė Viešpats’ ”. 
\par 2 Tuomet Viešpats paklausė jį: “Ką laikai rankoje?” Jis atsakė: “Lazdą”. 
\par 3 “Mesk žemėn!”­tarė Dievas. Jis numetė ją ir ji pavirto gyvate; Mozė ėmė bėgti nuo jos. 
\par 4 Viešpats pasakė Mozei: “Ištiesk savo ranką ir nutverk ją už uodegos!” Jis ištiesė savo ranką, nutvėrė ją ir ji pavirto lazda. 
\par 5 “Tai daryk, kad jie tikėtų, jog tau pasirodė Viešpats, jų tėvų Dievas, Abraomo, Izaoko ir Jokūbo Dievas”. 
\par 6 Viešpats kalbėjo toliau ir liepė jam įkišti ranką į savo užantį. Jis įkišo, ir kai ištraukė ją, ranka buvo balta nuo raupsų kaip sniegas. 
\par 7 “Vėl įkišk savo ranką į užantį!”­tarė Dievas. Jis padarė, kaip buvo liepta. Kai jis ištraukė ją, ji buvo sveika. 
\par 8 “Jeigu jie netikės tavimi ir nekreips dėmesio į pirmojo ženklo balsą, tai jie patikės kito ženklo balsu. 
\par 9 O jeigu jie netikės šiais dviem ženklais ir neklausys tavo balso, tai pasemk upės vandens ir išliek ant žemės! Tada vanduo, kurį pasemsi iš upės ir išliesi ant žemės, pavirs krauju”. 
\par 10 Mozė tarė Viešpačiui: “Viešpatie, aš nesu iškalbingas: nebuvau toks anksčiau nei dabar, kai Tu prabilai į savo tarną. Man sunku kalbėti ir mano liežuvis pinasi”. 
\par 11 Viešpats jam atsakė: “Kas sutvėrė žmogaus burną? Kas padaro jį nebylų, kurčią, matantį ar aklą? Argi ne Aš, Viešpats? 
\par 12 Taigi dabar eik, o Aš būsiu su tavo lūpomis ir tave pamokysiu, ką kalbėti”. 
\par 13 Jis atsakė: “Prašau, Viešpatie, siųsk ką nors kitą!” 
\par 14 Tada Viešpats supykęs tarė: “Žinau, kad tavo brolis Aaronas, levitas, yra iškalbus ir jis pasitiks tave. Tave pamatęs, jis džiaugsis savo širdyje. 
\par 15 Tu kalbėsi jam ir įdėsi žodžius į jo lūpas. Aš būsiu su tavo ir jo lūpomis ir jus pamokysiu, ką turite daryti. 
\par 16 Jis kalbės už tave tautai. Jis bus tavo lūpomis, o tu būsi jam vietoje Dievo. 
\par 17 Pasiimk šitą lazdą, su kuria darysi ženklus”. 
\par 18 Mozė sugrįžo pas savo uošvį Jetrą ir tarė jam: “Prašau, leisk man sugrįžti į Egiptą pas savo brolius ir pasižiūrėti, ar jie tebėra gyvi”. Jetras atsakė: “Eik ramybėje!” 
\par 19 Viešpats tarė Mozei Midjane: “Sugrįžk į Egiptą! Nes visi, kurie ieškojo tavo gyvybės, yra mirę”. 
\par 20 Mozė pasiėmė žmoną ir sūnus, užsodino juos ant asilo ir grįžo į Egiptą; ir Dievo lazdą jis pasiėmė į savo ranką. 
\par 21 Viešpats pasakė Mozei: “Kai sugrįši į Egiptą, žiūrėk, kad padarytum faraono akivaizdoje visus stebuklus, kuriuos įdėjau į tavo ranką. Bet Aš užkietinsiu jo širdį, ir jis neišleis tautos. 
\par 22 Sakyk faraonui: ‘Taip kalba Viešpats: ‘Izraelis yra mano pirmagimis sūnus. 
\par 23 Aš tau sakau: išleisk mano sūnų, kad jis man tarnautų; jei neišleisi jo, nužudysiu tavo pirmagimį sūnų’ ”. 
\par 24 Pakeliui, nakvynės namuose, sutiko jį Viešpats ir norėjo nužudyti. 
\par 25 Tada Cipora, paėmusi aštrų akmenį, apipjaustė sūnų, numetė odelę prie Mozės kojų ir tarė: “Tu esi mano kruvinas vyras!” 
\par 26 Ir Viešpats leido jam eiti. Tada ji pasakė: “Kruvinas vyras dėl apipjaustymo”. 
\par 27 Viešpats tarė Aaronui: “Eik į dykumą pasitikti Mozės”. Jis ėjo ir, sutikęs jį prie Dievo kalno, pabučiavo. 
\par 28 Mozė papasakojo Aaronui visus Viešpaties, kuris jį siuntė, žodžius ir apie visus ženklus, kuriuos Jis jam įsakė daryti. 
\par 29 Mozė ir Aaronas nuėję surinko visus izraelitų vyresniuosius. 
\par 30 Aaronas kalbėjo visus žodžius, kuriuos Viešpats buvo kalbėjęs Mozei, ir padarė ženklus tautos akyse. 
\par 31 Tauta patikėjo. Išgirdę, kad Viešpats aplankė izraelitus ir pamatė jų priespaudą, jie žemai nusilenkė ir pagarbino Jį.



\chapter{5}

\par 1 Mozė ir Aaronas atėję tarė faraonui: “Taip sako Viešpats Izraelio Dievas: ‘Išleisk mano tautą, kad jie aukotų man dykumoje’ ”. 
\par 2 Faraonas atsakė: “Kas yra jūsų Viešpats, kad klausyčiau Jo balso ir išleisčiau Izraelį? Aš nepažįstu Viešpaties ir neišleisiu Izraelio”. 
\par 3 Jie sakė: “Hebrajų Dievas mus pašaukė. Prašome, leisk mums keliauti tris dienas į dykumą ir ten aukoti Viešpačiui, mūsų Dievui, kad Jis nebaustų mūsų maru ar kardu”. 
\par 4 Egipto karalius jiems atsakė: “Moze ir Aaronai, kodėl atitraukiate žmones nuo jų darbų? Eikite savo darbų dirbti! 
\par 5 Matote, kiek daug žmonių yra krašte, o jūs atitraukiate juos nuo darbų”. 
\par 6 Tą pačią dieną faraonas įsakė darbų prievaizdams ir vadovams: 
\par 7 “Nebeduokite, kaip iki šiol, žmonėms šiaudų plytoms gaminti: jie patys tegul eina ir prisirenka šiaudų. 
\par 8 Tačiau reikalaukite iš jų tokio pat plytų skaičiaus, kokį jie lig šiol pagamindavo, jo nesumažinkite, nes jie dykinėja ir todėl šaukia: ‘Išleisk mus aukoti savo Dievui!’ 
\par 9 Duokite jiems daugiau darbo; tegul dirba ir nepaiso tuščių kalbų”. 
\par 10 Darbų prievaizdai ir vadovai išėję kalbėjo: “Taip sako faraonas: ‘Aš nebeduosiu jums šiaudų. 
\par 11 Eikite ir rinkite, kur rasite! Tačiau turėsite padaryti tiek, kiek ir anksčiau’ ”. 
\par 12 Žmonės išsisklaidė po visą Egipto šalį rinkti ražienų vietoje šiaudų. 
\par 13 Prievaizdai spaudė juos, sakydami: “Atlikite dienai skirtą darbą, kaip anksčiau, kai gaudavote šiaudų”. 
\par 14 Ir izraelitus, kuriuos faraono prievaizdai buvo paskyrę darbų prižiūrėtojais, mušė, sakydami: “Kodėl vakar ir šiandien nepagaminote nustatyto plytų kiekio kaip anksčiau?” 
\par 15 Izraelitų prižiūrėtojai atėjo verkdami ir skundėsi faraonui: “Kodėl taip elgiesi su savo tarnais? 
\par 16 Šiaudų nebeduoda, o mums liepia gaminti plytas. Be to, tavo tarnai dar mušami, nors kalti dėl to yra tavo žmonės”. 
\par 17 Jis atsakė: “Jūs vien tik dykinėjate. Todėl sakote: ‘Išleisk mus aukoti Viešpačiui!’ 
\par 18 Taigi dabar eikite, dirbkite! Šiaudų jums neduos, tačiau nustatytą plytų kiekį privalote padaryti”. 
\par 19 Izraelitų prižiūrėtojai matė esą patekę į sunkią padėtį, nes jiems buvo pasakyta: “Nemažinkite dienai skirto plytų kiekio”. 
\par 20 Išėję iš faraono, jie susitiko Mozę ir Aaroną, kurie jų laukė, 
\par 21 ir tarė jiems: “Viešpats tepažvelgia ir tegul teisia! Jūs sukėlėte faraono bei jo tarnų neapykantą mums, įduodami kardą į jų ranką, kad mus išžudytų”. 
\par 22 Mozė kreipėsi į Viešpatį: “Viešpatie, kodėl taip piktai pasielgei su šitais žmonėmis? Kodėl siuntei mane? 
\par 23 Nuo to laiko, kai nuėjau pas faraoną kalbėti Tavo vardu, jis dar blogiau elgiasi su mano tauta. Tu neišvadavai savo tautos”.



\chapter{6}


\par 1 Viešpats atsakė Mozei:“Tu matysi, ką Aš padarysiu faraonui. Stiprios rankos priverstas, jis išleis juos; stiprios rankos priverstas, jis išvarys juos iš savo šalies”. 
\par 2 Dievas kalbėjo Mozei: “Aš esu Viešpats. 
\par 3 Aš pasirodžiau Abraomui, Izaokui ir Jokūbui kaip Dievas Visagalis, tačiau savo vardu Viešpats jiems neapsireiškiau. 
\par 4 Aš įtvirtinau su jais sandorą, norėdamas jiems duoti Kanaano šalį, kurioje jie buvo ateiviai. 
\par 5 Ir Aš išgirdau Izraelio vaikų, kuriuos yra pavergę egiptiečiai, vaitojimą, ir atsiminiau savo sandorą. 
\par 6 Todėl sakyk izraelitams, kad Aš esu Viešpats; Aš jus išvesiu iš egiptiečių priespaudos, išlaisvinsiu iš vergystės ir išgelbėsiu jus savo ištiesta ranka ir dideliais teismais. 
\par 7 Aš jus priimsiu, kad būtumėte mano tauta, ir Aš būsiu jūsų Dievas, kad žinotumėte, jog esu Viešpats, jūsų Dievas, kuris jus išvedė iš egiptiečių vergijos. 
\par 8 Aš nuvesiu jus į šalį, kurią pažadėjau duoti Abraomui, Izaokui ir Jokūbui; ją duosiu jums paveldėti, nes Aš esu Viešpats!” 
\par 9 Mozė taip kalbėjo izraelitams, bet jie neklausė jo dėl dvasios suspaudimo ir žiaurios vergystės. 
\par 10 Viešpats kalbėjo Mozei: 
\par 11 “Eik, sakyk Egipto faraonui, kad jis išleistų izraelitus iš savo šalies!” 
\par 12 Mozė atsakė Viešpačiui: “Izraelitai manęs neklauso, tai kaipgi manęs klausys faraonas? Mano lūpos neapipjaustytos”. 
\par 13 Ir Viešpats kalbėjo Mozei ir Aaronui, ir davė jiems įsakymą Izraelio vaikams ir faraonui, Egipto karaliui, kad išvestų Izraelio vaikus iš Egipto šalies. 
\par 14 Šitie yra vyresnieji savo tėvų namuose. Izraelio pirmagimio Rubeno sūnūs: Henochas, Paluvas, Hecronas ir Karmis. Tai yra Rubeno giminės šeimos. 
\par 15 Simeono sūnūs: Jemuelis, Jaminas, Ohadas, Jachinas, Coharas ir kanaanietės sūnus Saulius. Tai yra Simeono giminės šeimos. 
\par 16 Šitie yra Levio sūnų vardai: Geršonas, Kehatas ir Meraris. Levis gyveno šimtą trisdešimt septynerius metus. 
\par 17 Geršono sūnūs: Libnis ir Šimis. 
\par 18 Kehato sūnūs: Amramas, Iccharas, Hebronas ir Uzielis. Kehatas gyveno šimtą trisdešimt trejus metus. 
\par 19 Merario sūnūs: Machlis ir Mušis. Tai yra Levio giminė pagal savo šeimas. 
\par 20 Amramas vedė savo tetą Jochebedą. Ji pagimdė Aaroną ir Mozę. Amramas gyveno šimtą trisdešimt septynerius metus. 
\par 21 Iccharo sūnūs: Korachas, Nefegas ir Zichris. 
\par 22 Uzielio sūnūs: Mišaelis, Elicafanas ir Sitris. 
\par 23 Aaronas vedė Elišebą, Aminadabo dukterį, Naasono seserį. Ji pagimdė Nadabą, Abihuvą, Eleazarą ir Itamarą. 
\par 24 Koracho sūnūs: Asiras, Elkana ir Abiasafas. Tai yra korachų šeimos. 
\par 25 Aarono sūnus Eleazaras vedė vieną iš Putielio dukterų. Ji pagimdė Finehasą. Šitie yra Levio giminės šeimų vyresnieji. 
\par 26 Aaronas ir Mozė yra tie, kuriems Viešpats įsakė išvesti izraelitus iš Egipto šalies. 
\par 27 Jie yra tie, kurie kalbėjo faraonui, Egipto karaliui, kad išvestų izraelitus iš Egipto. 
\par 28 Viešpats kalbėjo Mozei Egipto šalyje, 
\par 29 sakydamas: “Aš esu Viešpats! Pasakyk faraonui, Egipto karaliui, visa, ką tau kalbu”. 
\par 30 Bet Mozė sakė Viešpaties akivaizdoje: “Aš esu neapipjaustytomis lūpomis, kaip manęs klausys faraonas?”



\chapter{7}


\par 1 Viešpats tarė Mozei: “Aš tave padariau dievu faraonui; tavo brolis Aaronas bus tavo pranašas. 
\par 2 Tu sakysi visa, ką tau įsakau, o tavo brolis Aaronas kalbės faraonui, kad jis išleistų izraelitus iš savo šalies. 
\par 3 Bet Aš užkietinsiu faraono širdį ir padarysiu daug ženklų bei stebuklų Egipto šalyje. 
\par 4 Tačiau faraonas jūsų neklausys, kad galėčiau uždėti savo ranką ant Egipto ir dideliais teismais išvesti savo pulkus­savo tautą, Izraelio vaikus­iš Egipto žemės. 
\par 5 Tada egiptiečiai žinos, kad Aš esu Viešpats, kai ištiesiu savo ranką virš Egipto ir išvesiu izraelitus iš jų žemės”. 
\par 6 Mozė ir Aaronas padarė, kaip Viešpats jiems įsakė. 
\par 7 Mozė buvo aštuoniasdešimties metų, o Aaronas aštuoniasdešimt trejų metų amžiaus, kai jie kalbėjo faraonui. 
\par 8 Viešpats tarė Mozei ir Aaronui: 
\par 9 “Kai faraonas jums sakys: ‘Padarykite stebuklą patvirtinimui’, tai sakyk Aaronui: ‘Imk lazdą ir mesk ją prieš faraoną!’ Tada ji pavirs gyvate”. 
\par 10 Mozė ir Aaronas atėjo pas faraoną ir padarė taip, kaip Viešpats įsakė: Aaronas metė savo lazdą prieš faraoną ir jo tarnus, ir ji pavirto gyvate. 
\par 11 Faraonas pasišaukė išminčių ir burtininkų. Ir tie Egipto žyniai savo kerais padarė tą patį: 
\par 12 kiekvienas jų numetė savo lazdą, ir jos pavirto gyvatėmis. Tačiau Aarono lazda prarijo jų lazdas. 
\par 13 Faraono širdis liko užkietėjusi, ir jis jų neklausė, kaip Viešpats ir buvo kalbėjęs. 
\par 14 Viešpats tarė Mozei: “Faraono širdis tebėra užkietėjusi, jis nesutinka išleisti tautos. 
\par 15 Rytoj anksti rytą nueik prie upės, kai faraonas eis prie vandens, ir lauk jo ten. Pasiimk tą lazdą, kuri buvo pavirtusi gyvate. 
\par 16 Jam atėjus, sakyk: ‘Viešpats, hebrajų Dievas, mane siuntė pas tave, sakydamas: ‘Išleisk mano tautą, kad ji man tarnautų dykumoje’. Tačiau tu ligi šiol nenorėjai klausyti. 
\par 17 Todėl taip sako Viešpats: ‘Iš to pažinsi, kad Aš esu Viešpats. Štai suduosiu mano rankoje esančia lazda į upės vandenį, ir jis pavirs krauju. 
\par 18 Upėje plaukiojančios žuvys išgaiš, ir upė pradės taip dvokti, kad egiptiečiai nebegalės gerti jos vandens’ ”. 
\par 19 Viešpats toliau kalbėjo Mozei: “Sakyk Aaronui: ‘Imk lazdą ir ištiesk savo ranką virš egiptiečių vandenų, upių, perkasų, balų, vandens tvenkinių. Vanduo pavirs krauju visoje Egipto šalyje, net mediniuose ir akmeniniuose induose!’ ” 
\par 20 Mozė ir Aaronas taip padarė, kaip Viešpats buvo įsakęs. Jis pakėlė lazdą ir sudavė į upės vandenį, faraonui ir jo tarnams matant. Visas vanduo upėje pavirto krauju. 
\par 21 Upėje plaukiojančios žuvys išgaišo. Vanduo ėmė taip dvokti, kad egiptiečiai nebegalėjo gerti vandens iš upės. Kraujas buvo visoje Egipto žemėje. 
\par 22 Tą patį padarė ir egiptiečių žyniai savo kerais. Faraono širdis liko užkietėjusi, ir jis neklausė jų, kaip Viešpats ir buvo sakęs. 
\par 23 Faraonas nusigręžė ir nuėjo į savo namą. Jis viso to neėmė į širdį. 
\par 24 Egiptiečiai kasė upės pakraščiuose šulinius, ieškodami geriamojo vandens, nes jie nebegalėjo upės vandens gerti. 
\par 25 Praėjo septynios dienos, kai Viešpats buvo ištikęs upę.



\chapter{8}

\par 1 Tada Viešpats tarė Mozei: “Eik pas faraoną ir jam sakyk: ‘Taip sako Viešpats: ‘Išleisk mano žmones, kad jie man tarnautų. 
\par 2 Jei nesutiksi jų išleisti, užleisiu visą tavo kraštą varlėmis. 
\par 3 Upė knibždės varlėmis. Jos iš upės atrėplios į tavo namus, į tavo miegamąjį ir į tavo lovą, taip pat į tavo tarnų ir tarnaičių namus, į tavo krosnis ir į duonkubilius. 
\par 4 Per tave, tavo žmones ir visus tavo tarnus rėplios varlės’. 
\par 5 Sakyk Aaronui: ‘Ištiesk savo ranką su lazda ant upių, perkasų, balų ir padaryk, kad varlės užplūstų Egipto šalį!’ ” 
\par 6 Aaronas ištiesė ranką ant Egipto vandenų, ir varlės išrėpliojo ir užpildė Egipto kraštą. 
\par 7 Tą patį padarė žyniai savo kerais ir iššaukė varles Egipto šalyje. 
\par 8 Faraonas, pasišaukęs Mozę ir Aaroną, tarė: “Melskite Viešpatį, kad Jis pašalintų varles nuo manęs ir mano žmonių; tada išleisiu izraelitus aukoti Viešpačiui”. 
\par 9 Mozė atsakė faraonui: “Paskirk laiką, kada melsti už tave, tavo tarnus ir tautą, kad varlės būtų pašalintos nuo tavęs, iš tavo namų ir kad jos tik upėje tepasiliktų”. 
\par 10 Jis atsakė: “Rytoj”. Mozė tarė: “Tebūna, kaip sakai, kad žinotum, jog nėra lygaus Viešpačiui, mūsų Dievui! 
\par 11 Varlės pasišalins nuo tavęs, iš tavo namų, nuo tavo tarnų ir tarnaičių, jos tik upėje tepasiliks!” 
\par 12 Mozė ir Aaronas, grįžę iš faraono, meldė Viešpatį pašalinti varles, kurias Jis buvo užleidęs faraonui. 
\par 13 Viešpats padarė, kaip Mozė prašė: varlės išgaišo namuose, kiemuose ir laukuose. 
\par 14 Jie suvertė jas į krūvas. Visa šalis dvokė. 
\par 15 Faraonas, matydamas, kad atėjo ramybė, užkietino savo širdį ir neklausė, kaip Viešpats ir buvo sakęs. 
\par 16 Tada Viešpats tarė Mozei: “Liepk Aaronui ištiesti lazdą ir suduoti į žemės dulkes, kad jos pavirstų mašalais visoje Egipto šalyje!” 
\par 17 Aaronas ištiesė ranką su lazda ir sudavė į žemę. Tuoj mašalai apniko žmones ir gyvulius. Visos žemės dulkės virto mašalais Egipto krašte. 
\par 18 Žyniai bandė tą patį padaryti savo kerais, bet negalėjo. Mašalai apniko žmones ir gyvulius. 
\par 19 Tada žyniai tarė faraonui: “Tai Dievo pirštas!” Bet faraono širdis buvo užkietinta ir jis neklausė jų, kaip Viešpats ir buvo sakęs. 
\par 20 Viešpats tarė Mozei: “Atsikelk anksti rytą, prieik prie faraono, kai jis eis prie vandens, ir pasakyk jam: ‘Taip sako Viešpats: ‘Išleisk mano žmones, kad jie man tarnautų! 
\par 21 Jei neišleisi mano tautos, užleisiu musių ant tavęs, tavo tarnų, tavo žmonių ir ant tavo namų taip, kad egiptiečių namai ir visa žemė bus pilna musių. 
\par 22 Bet tą dieną Aš atskirsiu Gošeno kraštą, kuriame gyvena mano tauta. Ten nebus musių, kad žinotum, jog Aš esu visos žemės Viešpats. 
\par 23 Taip Aš atskirsiu savo ir tavo tautą. Rytoj įvyks šitas ženklas’ ”. 
\par 24 Viešpats taip ir padarė. Dideli musių spiečiai apniko faraono ir jo tarnų namus ir visą Egipto šalį. Musės vargino kraštą. 
\par 25 Faraonas, pasišaukęs Mozę ir Aaroną, jiems tarė: “Eikite, aukokite savo Dievui šioje šalyje!” 
\par 26 Mozė atsakė: “Netinka taip daryti. Egiptiečiai bjaurisi tuo, ką aukosime Viešpačiui, savo Dievui. Jei aukosime, kas egiptiečiams bjauru, argi jie mūsų neužmuš akmenimis? 
\par 27 Tris dienas keliausime į dykumą aukoti Viešpačiui, savo Dievui, kaip Jis mums įsakė”. 
\par 28 Faraonas atsakė: “Aš išleisiu jus į dykumą aukoti Viešpačiui, savo Dievui. Tik nenueikite labai toli! Melskitės už mane!” 
\par 29 Mozė atsakė: “Iš tavęs išėjęs, melsiu Viešpatį, kad muses pašalintų nuo tavęs, nuo tavo tarnų ir tarnaičių. Tik kad vėl faraonas neapgautų, neišleisdamas tautos Viešpačiui aukoti!” 
\par 30 Mozė išėjo iš faraono ir meldė Viešpatį. 
\par 31 Viešpats padarė, kaip Mozė prašė. Jis pašalino muses nuo faraono, nuo jo tarnų ir žmonių taip, kad nė vienos nebeliko. 
\par 32 Bet ir šįkart faraonas užkietino savo širdį ir neišleido tautos.



\chapter{9}


\par 1 Viešpats sakė Mozei: “Eik pas faraoną ir jam sakyk: ‘Taip sako Viešpats, hebrajų Dievas: ‘Išleisk mano žmones, kad jie man tarnautų! 
\par 2 Jei jų neišleisi ir nepaliausi jų laikęs, 
\par 3 Viešpaties ranka bus ant tavo lauke besiganančių gyvulių: arklių, asilų, kupranugarių, avių ir galvijų­bus labai sunkus maras. 
\par 4 Viešpats atskirs izraelitų ir egiptiečių gyvulius; niekas nežus, kas priklauso izraelitams’ ”. 
\par 5 Ir Viešpats paskyrė laiką, sakydamas: “Rytoj Viešpats įvykdys šitą dalyką krašte!” 
\par 6 Kitą rytą Viešpats įvykdė tai: visi egiptiečių gyvuliai nugaišo, bet iš izraelitų gyvulių nepražuvo nė vienas. 
\par 7 Faraonas pasiuntė pasižiūrėti. Pasirodė, kad izraelitų gyvulių nė vienas nebuvo nugaišęs. Tačiau faraono širdis liko kieta ir jis neišleido tautos. 
\par 8 Tada Viešpats tarė Mozei ir Aaronui: “Imkite pilnas saujas pelenų iš krosnies ir Mozė teberia juos į orą faraono akivaizdoje. 
\par 9 Jie taps dulkėmis visoje Egipto šalyje, ir ant žmonių bei gyvulių iškils votys su pūslėmis”. 
\par 10 Jie pasiėmė pelenų iš krosnies ir, atsistoję prieš faraoną, išbėrė juos į orą. Ir atsirado votys ant žmonių ir gyvulių. 
\par 11 Žyniai negalėjo pasirodyti Mozei, nes votys buvo ant jų ir visų egiptiečių. 
\par 12 Tačiau Viešpats užkietino faraono širdį, ir jis neklausė jų, kaip Viešpats ir buvo sakęs Mozei. 
\par 13 Tada Viešpats tarė Mozei: “Atsikelk anksti rytą, nueik pas faraoną ir sakyk: ‘Taip sako Viešpats, hebrajų Dievas: ‘Išleisk mano tautą man tarnauti. 
\par 14 Nes šį kartą Aš siųsiu įvairias negalias ir vargus tau, tavo tarnams ir tavo žmonėms, kad žinotum, jog nėra man lygaus visoje žemėje. 
\par 15 Aš galėjau ištiesti savo ranką ir ištikti tave ir tavo tautą maru, kad būtumėte visi pranykę nuo žemės paviršiaus. 
\par 16 Bet Aš tam išaukštinau tave, kad parodyčiau savo galią ir mano vardas būtų skelbiamas visoje žemėje. 
\par 17 Tu vis dar didžiuojiesi prieš mano tautą ir neišleidi jos. 
\par 18 Rytoj apie šitą laiką kris labai smarki kruša, kokios nėra buvę Egipte nuo jo įsikūrimo dienos. 
\par 19 Taigi dabar siųsk žmones surinkti iš lauko gyvulius ir visa, kas tau priklauso. Visi žmonės ir gyvuliai, kurie bus lauke ir nebus parvesti namo, krušai krintant, pražus’ ”. 
\par 20 Faraono tarnai, kurie bijojo Viešpaties, sugabeno į namus savo tarnus ir gyvulius. 
\par 21 O kas nekreipė dėmesio į Viešpaties žodį, paliko savo tarnus ir gyvulius lauke. 
\par 22 Viešpats tarė Mozei: “Ištiesk savo ranką, kad kristų kruša visoje Egipto šalyje: ant žmonių, gyvulių ir visos laukų augmenijos”. 
\par 23 Mozė ištiesė lazdą, ir Viešpats pasiuntė perkūniją, krušą ir žaibus. Viešpats siuntė krušą į visą Egipto žemę. 
\par 24 Kruša susimaišė su žaibais ir buvo tokia smarki, kokios Egipto šalis nebuvo mačiusi. 
\par 25 Kruša išmušė Egipto šalyje visa, kas buvo lauke: žmones, gyvulius, augalus ir medžius. 
\par 26 Tik Gošeno krašte, kur gyveno izraelitai, nebuvo krušos. 
\par 27 Faraonas, pasišaukęs Mozę ir Aaroną, jiems kalbėjo: “Aš nusidėjau! Viešpats yra teisus, o aš ir mano tauta esame nusikaltę. 
\par 28 Melskite Viešpatį, kad liautųsi stipri perkūnija ir kruša! Aš jus išleisiu ir daugiau nebesulaikysiu”. 
\par 29 Mozė jam atsakė: “Kai tik išeisiu iš miesto, pakelsiu rankas į Viešpatį. Tada perkūnija ir kruša liausis, kad žinotum, jog Viešpačiui priklauso visa žemė. 
\par 30 Bet aš žinau, kad nei tu, nei tavo tarnai dar nesibijote Viešpaties Dievo”. 
\par 31 Linus ir miežius kruša išmušė, nes miežiai buvo išplaukę ir linai jau žydėjo. 
\par 32 Bet kviečių ir rugių neišmušė, nes jie vėliau pribręsta. 
\par 33 Mozė išėjo nuo faraono iš miesto ir iškėlė rankas į Viešpatį: perkūnija ir kruša liovėsi, lietus nustojo lijęs. 
\par 34 Bet faraonas, matydamas, kad liovėsi lietus, kruša ir perkūnija, vėl nusidėjo ir užkietino savo širdį kartu su savo tarnais. 
\par 35 Faraono širdis pasiliko užkietėjusi, ir jis neišleido Izraelio vaikų, kaip Viešpats ir buvo sakęs Mozei.



\chapter{10}

\par 1 Viešpats tarė Mozei: “Eik pas faraoną, nes Aš užkietinau jo ir jo tarnų širdis, kad padaryčiau šituos ženklus jų tarpe 
\par 2 ir kad galėtum papasakoti savo vaikams ir vaikų vaikams, ką Aš padariau Egipte ir kokius ženklus parodžiau jų tarpe, kad žinotumėte, jog Aš esu Viešpats”. 
\par 3 Mozė ir Aaronas atėjo pas faraoną ir jam kalbėjo: “Taip sako Viešpats, hebrajų Dievas: ‘Ar ilgai dar tu nenusižeminsi prieš mane? Išleisk mano žmones, kad jie man tarnautų! 
\par 4 Jei neišleisi mano tautos, rytoj užleisiu skėrius ant tavo krašto. 
\par 5 Jie taip apdengs šalį, kad nesimatys žemės; jie nuės išlikusį nuo krušos derlių, nugrauš kiekvieną žaliuojantį medį; 
\par 6 jų bus pilni tavo namai, tavo tarnų namai ir visų egiptiečių namai, kaip to dar nėra matę tavo tėvai ir seneliai per visą savo amžių’ ”. Po to Mozė apsisuko ir išėjo iš faraono namų. 
\par 7 Tada faraono tarnai kalbėjo savo valdovui: “Ar ilgai mes kentėsime? Išleisk tuos žmones, kad jie tarnautų Viešpačiui, savo Dievui! Ar dar nematai, kad Egiptas žūva?” 
\par 8 Mozė ir Aaronas buvo pašaukti pas faraoną. Jis tarė jiems: “Eikite, tarnaukite Viešpačiui, savo Dievui! Bet kas yra tie, kurie eis?” 
\par 9 Mozė atsakė: “Eisime visi: jaunimas ir seneliai, sūnūs ir dukterys, avys ir galvijai. Mes švęsime šventę Viešpačiui”. 
\par 10 Jis atsakė jiems: “Tebūna Viešpats su jumis! Kaip galiu išleisti jus ir jūsų vaikus? Žiūrėkite, nes jūs sumanėte pikta! 
\par 11 Taip nebus. Eikite vieni vyrai ir aukokite Viešpačiui, nes to juk jūs ir prašėte!” Ir juos išvarė iš faraono akivaizdos. 
\par 12 Tada Viešpats tarė Mozei: “Ištiesk savo ranką ant Egipto šalies, kad skėriai apniktų ir nugraužtų visus šalies augalus ir visa, kas dar liko nuo krušos!” 
\par 13 Mozė ištiesė lazdą. Viešpats leido rytų vėjui pūsti visą dieną ir naktį. Rytmečiui išaušus, rytų vėjas atnešė skėrius. 
\par 14 Skėriai nusileido visoje Egipto šalyje. Jų buvo tiek, kiek niekad nėra buvę ir jų tiek nebus ateityje. 
\par 15 Jie taip užplūdo visą kraštą, kad apdengė visą žemę. Jie nugraužė visus augalus ir medžių vaisius, išlikusius nuo krušos, kad nieko žaliuojančio nebeliko visoje Egipto šalyje. 
\par 16 Tada faraonas skubiai pasišaukė Mozę ir Aaroną ir tarė: “Nusidėjau Viešpačiui, jūsų Dievui, ir jums! 
\par 17 Prašau, atleiskite man dar kartą mano nusikaltimą ir melskite Viešpatį, savo Dievą, kad Jis pašalintų nuo manęs šitą mirtį”. 
\par 18 Mozė išėjo iš faraono ir meldė Viešpatį. 
\par 19 Viešpats sukėlė labai smarkų vakarų vėją, kuris skėrius nupūtė į Raudonąją jūrą; nė vieno skėrio nebeliko visoje Egipto šalyje. 
\par 20 Tačiau Viešpats užkietino faraono širdį, ir jis neišleido izraelitų. 
\par 21 Viešpats tarė Mozei: “Ištiesk savo ranką, kad tamsa apgaubtų visą Egipto šalį, tamsa, kurią galima būtų pajusti”. 
\par 22 Mozė ištiesė savo ranką, ir tirščiausia tamsa tris dienas buvo visame Egipto krašte. 
\par 23 Tris dienas žmonės negalėjo matyti vienas kito ir pajudėti iš vietos. Tačiau izraelitų namuose buvo šviesu. 
\par 24 Tada faraonas, pasišaukęs Mozę, tarė: “Eikite, tarnaukite Viešpačiui! Tik jūsų avys ir galvijai tepasilieka! O jūsų vaikai teeina su jumis!” 
\par 25 Mozė atsakė: “Privalai mums duoti atnašas ir deginamąsias aukas, kad mes galėtume aukoti Viešpačiui, mūsų Dievui. 
\par 26 Mūsų visi gyvuliai eis su mumis; nepaliksime nė kanopos. Nes iš jų privalome imti auką Viešpačiui, savo Dievui. Mes net nežinome, ko mums reikės Viešpaties aukai, kol ten nenuėjome”. 
\par 27 Tačiau Viešpats užkietino faraono širdį, ir jis nesutiko jų išleisti. 
\par 28 Ir faraonas tarė: “Šalin nuo manęs! Saugokis! Nebepasirodyk daugiau mano akyse, nes tą dieną, kurią pasirodysi, mirsi!” 
\par 29 Mozė atsakė: “Tebūna, kaip pasakei! Daugiau nebepasirodysiu”.



\chapter{11}


\par 1 Viešpats tarė Mozei: “Dar vieną bausmę užleisiu faraonui ir Egiptui. Po to jis išleis jus, netgi varyte išvarys! 
\par 2 Kalbėk žmonėms, kad kiekvienas vyras iš savo kaimyno ir kiekviena moteris iš savo kaimynės paprašytų sidabrinių ir auksinių daiktų”. 
\par 3 Viešpats palankiai nuteikė izraelitams egiptiečius. Be to, Mozė buvo labai žymus vyras Egipto šalyje, faraono tarnų ir tautos akyse. 
\par 4 Tada Mozė kalbėjo: “Taip sako Viešpats: ‘Apie vidurnaktį Aš pereisiu Egiptą. 
\par 5 Tada mirs visi pirmagimiai, pradedant faraono, sėdinčio soste, pirmagimiu, ir baigiant pirmagimiu vergės, esančios prie girnų, ir visi gyvulių pirmagimiai. 
\par 6 Visoje Egipto šalyje kils verksmas, kokio nėra buvę ir kokio daugiau nebus. 
\par 7 Bet prieš nė vieną izraelitą, ar jo gyvulį nė šuo nesulos, kad žinotumėte, jog Viešpats daro skirtumą tarp egiptiečių ir izraelitų’. 
\par 8 Tada tavo tarnai ateis ir, nusilenkę prieš mane, sakys: ‘Išeik tu ir visi tavo žinioje esantys žmonės!’ Ir tada aš išeisiu”. Mozė išėjo iš faraono degdamas pykčiu. 
\par 9 Tada Viešpats tarė Mozei: “Faraonas jūsų neklausys, kad Aš galėčiau padaryti daugiau stebuklų Egipte”. 
\par 10 Mozė ir Aaronas padarė visus šituos stebuklus faraonui matant. Tačiau Viešpats taip užkietino faraono širdį, kad jis neišleido izraelitų iš savo šalies.



\chapter{12}

\par 1 Viešpats kalbėjo Mozei ir Aaronui Egipto šalyje: 
\par 2 “Šitas mėnuo tebūna jums pirmasis metų mėnuo. 
\par 3 Pasakykite visiems izraelitams, kad šito mėnesio dešimtąją dieną jie paimtų po avinėlį savo tėvų namams, po vieną avinėlį kiekvienai šeimai. 
\par 4 O jei šeima yra per maža avinėlį suvalgyti, tepaima kartu su savo artimiausiu kaimynu, kad susidarytų tiek asmenų, kiek gali suvalgyti avinėlį. 
\par 5 Avinėlis privalo būti be trūkumų, metinis patinėlis; paimsite jį iš avių ar ožkų. 
\par 6 Laikykite jį iki šio mėnesio keturioliktos dienos; kiekviena Izraelio tautos šeima turi jį papjauti tos dienos vakare. 
\par 7 Jo krauju patepkite abi durų staktas ir skersinį tų namų, kuriuose valgysite avinėlį. 
\par 8 Tą naktį valgykite mėsą, keptą ant ugnies, su nerauginta duona ir karčiomis žolėmis. 
\par 9 Jūs neturite jos valgyti žalios ar išvirtos vandenyje, tik keptą ugnyje, taip pat galvą, kojas ir vidurius. 
\par 10 Nieko iš jo nepalikite iki ryto. O kas paliks ligi ryto, tą sudeginkite. 
\par 11 Valgykite jį paskubomis, susijuosę strėnas, apsiavę, laikydami lazdą rankoje; tai Viešpaties Pascha. 
\par 12 Tą naktį pereisiu visą Egipto šalį ir išžudysiu visus pirmagimius, žmones ir gyvulius; ir visiems Egipto dievams įvykdysiu teismą. Aš­Viešpats! 
\par 13 O kraujas ant namų bus ženklas, kur jūs gyvenate. Pamatęs kraują, aplenksiu jus, kad Egipto šalies bausmė nepaliestų jūsų. 
\par 14 Ta diena tebūna jums atmintina diena; jūs privalote ją švęsti kaip šventę Viešpačiui per kartų kartas. 
\par 15 Septynias dienas valgysite neraugintą duoną. Jau pirmąją dieną pašalinkite raugą iš savo namų, nes kiekvienas, kuris valgys raugintą maistą nuo pirmosios iki septintosios dienos, bus išnaikintas iš Izraelio. 
\par 16 Pirmąją dieną susirinkite šventei, taip pat ir septintąją. Jokio darbo nevalia dirbti tomis dienomis, išskyrus tai, ko reikia kiekvieno žmogaus maistui. 
\par 17 Jūs švęsite Neraugintos duonos šventę per kartų kartas, nes tą dieną Aš išvedžiau jūsų pulkus iš Egipto šalies. 
\par 18 Pirmojo mėnesio keturioliktos dienos vakare pradėsite valgyti neraugintą duoną ir valgysite iki dvidešimt pirmosios dienos vakaro. 
\par 19 Septynias dienas nebus raugo jūsų namuose, nes kiekvienas, kuris valgys ką nors rauginto, bus išnaikintas iš Izraelio, ar jis būtų ateivis, ar vietinis gyventojas. 
\par 20 Nieko rauginto jūs nevalgysite. Savo namuose valgysite neraugintą duoną”. 
\par 21 Tada Mozė sušaukė visus Izraelio vyresniuosius ir jiems tarė: “Eikite, imkite avinėlį savo šeimoms ir pjaukite jį Paschai. 
\par 22 Po to imkite yzopo ryšulėlį ir, pamirkę dubenyje su krauju, patepkite juo abi durų staktas ir skersinį; nė vienas jūsų teneišeina iki ryto iš savo namų! 
\par 23 Nes Viešpats pereis, žudydamas egiptiečius; pamatęs kraują ant abiejų durų staktų ir skersinio, aplenks ir neleis žudytojui įeiti į jūsų namus. 
\par 24 Tai bus įstatas jums ir jūsų vaikams per amžius. 
\par 25 Kai atvyksite į šalį, kurią Viešpats jums duos, kaip Jis pažadėjo, laikykitės šitų nuostatų. 
\par 26 Kai jūsų vaikai klaus: ‘Ką reiškia šitos apeigos?’, 
\par 27 atsakykite: ‘Tai yra Paschos auka Viešpačiui, kuris aplenkė izraelitų namus Egipte, kai Jis išžudė egiptiečius ir išlaisvino mūsų namus’ ”. Tada žmonės nusilenkė ir pagarbino Viešpatį. 
\par 28 Izraelitai ėjo ir padarė, kaip Viešpats įsakė Mozei ir Aaronui. 
\par 29 Vidurnaktį Viešpats išžudė visus pirmagimius Egipto šalyje, pradedant faraono, kuris sėdėjo soste, pirmagimiu, baigiant pirmagimiu kalinio, sėdinčio kalėjime, ir visus gyvulių pirmagimius. 
\par 30 Naktį atsikėlė faraonas, jo tarnai ir visi egiptiečiai. Ir kilo didelis verksmas Egipte, nes nebuvo namų, kuriuose nebūtų mirusio. 
\par 31 Tą naktį faraonas, pasišaukęs Mozę ir Aaroną, tarė: “Išeikite iš mano krašto, jūs ir izraelitai! Eikite, tarnaukite Viešpačiui, kaip sakėte. 
\par 32 Pasiimkite avis ir galvijus, kaip sakėte! Eikite ir palaiminkite mane taip pat!” 
\par 33 Egiptiečiai ragino tautą skubiai išeiti iš jų šalies, sakydami: “Mes visi išmirsime!” 
\par 34 Žmonės ėmė ant pečių dar neįrūgusią tešlą duonkubiliuose, įvyniotą į apsiaustus. 
\par 35 Ir izraelitai padarė, kaip Mozė buvo sakęs; jie paprašė iš egiptiečių sidabrinių bei auksinių daiktų ir drabužių. 
\par 36 Viešpats palankiai nuteikė izraelitams egiptiečius, kurie patenkino jų prašymą. Taip jie apiplėšė egiptiečius. 
\par 37 Apie šeši šimtai tūkstančių vyrų iš Izraelio sūnų, neskaičiuojant vaikų, išėjo iš Ramzio į Sukotą. 
\par 38 Taip pat daugybė kitų žmonių ėjo su jais ir didelės kaimenės avių bei galvijų. 
\par 39 Jie kepė iš tešlos, kurią išsinešė iš Egipto, neraugintus papločius. Ji dar nebuvo įrūgusi, nes jie buvo skubiai išvaryti iš Egipto ir nespėjo pasigaminti maisto kelionei. 
\par 40 Izraelitai pragyveno Egipte keturis šimtus trisdešimt metų. 
\par 41 Praėjus keturiems šimtams trisdešimčiai metų, vieną dieną visi Viešpaties pulkai iškeliavo iš Egipto. 
\par 42 Ta naktis, kurią juos išvedė iš Egipto krašto, yra Viešpaties šventė. Izraelitai privalo tą naktį švęsti per kartų kartas. 
\par 43 Viešpats kalbėjo Mozei ir Aaronui: “Šitas yra Paschos įstatymas: jokiam svetimšaliui nevalia Paschos valgyti. 
\par 44 Tačiau pirktas vergas turi teisę ją valgyti tada, kai jį apipjaustai. 
\par 45 Svetimšaliui ir samdiniui negalima jos valgyti. 
\par 46 Ji turi būti valgoma namuose. Nevalia mėsos išnešti iš namų ir neleidžiama sulaužyti jokio kaulo. 
\par 47 Visa Izraelio tauta privalo ją švęsti. 
\par 48 Jei ateivis apsistotų pas tave ir norėtų švęsti Paschą, tai visi jo vyrai privalo būti apipjaustyti. Tuo atveju leidžiama jam švęsti Paschą, ir jis bus tarsi vietinis šalies gyventojas. Bet niekam neapipjaustytam neleistina jos valgyti. 
\par 49 Tas pats įstatymas galioja ir pas jus gimusiam, ir ateiviui, apsigyvenusiam tarp jūsų”. 
\par 50 Izraelitai darė, kaip Viešpats įsakė Mozei ir Aaronui. 
\par 51 Tą dieną Viešpats išvedė izraelitų pulkus iš Egipto šalies.



\chapter{13}


\par 1 Viešpats kalbėjo Mozei: 
\par 2 “Pašvęsk man visus pirmagimius: ar jie būtų žmonių, ar gyvulių. Jie­mano”. 
\par 3 Mozė kalbėjo tautai: “Atsiminkite šitą dieną, kurią išėjote iš Egipto, iš vergijos namų. Savo rankos galybe Viešpats jus išvedė, todėl nevalgykite nieko rauginto. 
\par 4 Šiandien, Abibo mėnesį, jūs išėjote. 
\par 5 Kai Viešpats tave įves į kanaaniečių, hetitų, amoritų, hivų ir jebusiečių šalį, plūstančią pienu ir medumi, kurią tau duoti Jis prisiekė tavo tėvams, tu privalai švęsti tą šventę šį mėnesį. 
\par 6 Septynias dienas valgysi neraugintą duoną, septintą dieną bus šventė Viešpačiui. 
\par 7 Neraugintą duoną privalai valgyti septynias dienas, ir nei pas tave, nei tavo gyvenamose vietose nebus raugintos duonos. 
\par 8 Tą dieną tu sakysi savo sūnui: ‘Tai daroma dėl to, ką Viešpats padarė išvesdamas mane iš Egipto’. 
\par 9 Tai bus kaip ženklas tau ant rankos ir prisiminimas ant kaktos, kad Viešpaties įstatymas būtų tavo lūpose. Juk galinga ranka Viešpats tave išvedė iš Egipto. 
\par 10 Vykdyk šitą nurodymą kasmet skirtu laiku. 
\par 11 Kai Viešpats tave įves į kanaaniečių šalį, kaip Jis prisiekė tau ir tavo tėvams, ir ją tau atiduos, 
\par 12 tai kiekvieną savo pirmagimį atskirsi Viešpačiui ir kiekvieną gyvulių pirmagimį. Vyriškos lyties pirmagimiai priklauso Viešpačiui. 
\par 13 Kiekvieną pirmagimį asilaitį privalai išpirkti ėriuku, bet jei neišpirksi, užmušk jį. Kiekvieną savo žmonių pirmagimį sūnų privalai išpirkti. 
\par 14 Kai ateityje tavo sūnus paklaus: ‘Ką tai reiškia?’, jam atsakyk: ‘Galinga ranka Viešpats mus išvedė iš Egipto, iš vergijos namų. 
\par 15 Kai faraonas užsikietino ir nenorėjo mūsų išleisti, Viešpats nužudė visus pirmagimius Egipto šalyje, tiek žmonių, tiek gyvulių pirmagimius. Todėl aukoju Viešpačiui kiekvieną pirmagimį vyriškos lyties, o kiekvieną pirmagimį savo vaikų išperku’. 
\par 16 Tai bus kaip ženklas ant tavo rankos ir prisiminimas ant tavo kaktos, nes savo galinga ranka Viešpats mus išvedė iš Egipto”. 
\par 17 Kai faraonas išleido žmones, Dievas jų nevedė filistinų šalies keliu, nors tai būtų buvę arčiausia. Nes Dievas sakė: “Kad išvydę karą žmonės nepradėtų gailėtis ir nesugrįžtų į Egiptą”. 
\par 18 Dievas vedė tautą aplinkiniu dykumų keliu, Raudonosios jūros link. Izraelitai išėjo iš Egipto apsiginklavę. 
\par 19 Mozė paėmė su savimi Juozapo kaulus, nes jis buvo prisaikdinęs izraelitus, sakydamas: “Dievas tikrai jus aplankys, ir jūs išnešite iš čia mano kaulus”. 
\par 20 Jie keliavo iš Sukoto ir pasistatė stovyklą Etame, dykumos pakraštyje. 
\par 21 Viešpats ėjo pirma jų dieną debesies stulpe, rodydamas kelią, o naktį ugnies stulpe, šviesdamas jiems, kad jie galėtų keliauti dieną ir naktį. 
\par 22 Jis nepatraukė nuo jų debesies stulpo dieną nė ugnies stulpo naktį.



\chapter{14}

\par 1 Viešpats kalbėjo Mozei: 
\par 2 “Sakyk izraelitams, kad jie apsistotų ties Pi Hahirotu, tarp Migdolo ir jūros. Priešais Baal Cefoną pasistatykite stovyklą prie jūros. 
\par 3 Tada faraonas sakys apie jus: ‘Jie pasiklydo krašte, dykumos juos sulaiko!’ 
\par 4 Aš užkietinsiu faraono širdį, ir jis jus vysis, kad būčiau pašlovintas per faraoną ir visą jo kariuomenę ir egiptiečiai žinotų, jog Aš esu Viešpats”. Jie taip ir padarė. 
\par 5 Egipto karaliui pranešė, kad tauta pabėgo. Tada faraono ir jo tarnų širdys atsisuko prieš šitą tautą ir jie kalbėjo: “Ką padarėme, išleisdami Izraelį, kad mums nebetarnautų?” 
\par 6 Jis paruošė savo vežimą ir pasiėmė savo žmones su savimi. 
\par 7 Jis pasiėmė šešis šimtus rinktinių kovos vežimų ir paskyrė viršininkus jiems. 
\par 8 Viešpats užkietino faraono, Egipto karaliaus, širdį, ir jis persekiojo Izraelio vaikus. Bet Izraelio vaikus išvedė galinga ranka. 
\par 9 Faraono žirgai, vežimai, raiteliai ir visa jo kariuomenė pasivijo juos, apsistojusius prie jūros, ties Pi Hahirotu, priešais Baal Cefoną. 
\par 10 Kai faraonas priartėjo, Izraelio vaikai pakėlė akis ir pamatė atžygiuojančius iš paskos egiptiečius. Jie labai išsigando ir šaukėsi Viešpaties. 
\par 11 Jie sakė Mozei: “Nejaugi Egipte nebuvo kapų, kad mus išvedei mirti dykumoje? Ką mums padarei, išvesdamas iš Egipto? 
\par 12 Argi mes tau nesakėme, būdami Egipte: ‘Palik mus tarnauti egiptiečiams’. Mums juk būtų buvę geriau tarnauti egiptiečiams, negu mirti dykumoje”. 
\par 13 Mozė atsakė: “Nebijokite, stovėkite ramiai ir stebėkite Viešpaties išgelbėjimą, kurį Jis šiandien įvykdys. Egiptiečių, kuriuos šiandien matote, niekados daugiau nebematysite. 
\par 14 Viešpats kariaus už jus, o jūs būkite ramūs!” 
\par 15 Viešpats tarė Mozei: “Ko šauki? Sakyk izraelitams, kad eitų pirmyn. 
\par 16 O tu pakelk lazdą, ištiesk ranką link jūros ir perskirk ją. Izraelitai sausuma pereis per jūrą. 
\par 17 Aš užkietinsiu egiptiečių širdis, ir jie vysis jus, ir Aš būsiu pašlovintas per faraoną ir visą jo kariuomenę, kovos vežimus ir raitelius. 
\par 18 Egiptiečiai žinos, kad Aš esu Viešpats, kai būsiu pašlovintas per faraoną, jo vežimus ir raitelius”. 
\par 19 Tada Dievo angelas, kuris ėjo Izraelio pulkų priekyje, pasitraukė už jų. Debesies stulpas iš jų priekio atsistojo užpakalyje jų. 
\par 20 Jis buvo tarp egiptiečių stovyklos ir Izraelio stovyklos; tamsus debesis dengė egiptiečius, o izraelitams buvo šviesu. Taip jie per naktį nepriartėjo vieni prie kitų. 
\par 21 Mozė ištiesė ranką link jūros. Viešpats smarkiu rytų vėju, kuris pūtė per naktį, išdžiovino jūrą, ir vandenys persiskyrė. 
\par 22 Izraelitai ėjo sausu jūros dugnu; vanduo jiems buvo siena iš dešinės ir kairės. 
\par 23 Egiptiečiai, juos vydamiesi, sekė jūros dugnu su kovos vežimais ir raiteliais. 
\par 24 Rytui auštant, Viešpats iš ugnies ir debesies stulpo pažvelgė į egiptiečių kariuomenę ir sukėlė sąmyšį jų tarpe. 
\par 25 Jis numovė vežimų ratus, ir tie sunkiai begalėjo judėti. Egiptiečiai sakė: “Bėkime nuo izraelitų, nes Viešpats kovoja už juos prieš egiptiečius”. 
\par 26 Tada Viešpats tarė Mozei: “Ištiesk ranką link jūros, kad vanduo užlietų egiptiečius, jų vežimus ir raitelius!” 
\par 27 Mozė ištiesė ranką, ir, rytui brėkštant, jūra sugrįžo į pirmykštę vietą; bėgančius egiptiečius užliejo vanduo. Taip Viešpats sunaikino egiptiečius jūroje. 
\par 28 Vanduo sugrįžo ir užliejo visą faraono kariuomenę, karo vežimus ir raitelius. Nė vienas jų neišliko. 
\par 29 O izraelitai perėjo sausu jūros dugnu, vanduo jiems buvo siena dešinėje ir kairėje. 
\par 30 Taip Viešpats išgelbėjo izraelitus iš egiptiečių; jie matė negyvus egiptiečius ant jūros kranto. 
\par 31 Izraelis matė galingą darbą, kurį Viešpats padarė egiptiečiams. Tauta bijojo Viešpaties. Jie tikėjo Viešpačiu ir Jo tarnu Moze.
Online Parallel Study Bible



\chapter{15}


\par 1 Tada Mozė ir izraelitai giedojo Viešpačiui šią giesmę: “Giedosiu Viešpačiui, nes Jis šlovingai nugalėjo! Jis žirgą ir raitelį įmetė į jūrą. 
\par 2 Mano stiprybė ir giesmė yra Viešpats. Jis tapo mano išgelbėjimu. Jis yra mano Dievas, aš Jį šlovinsiu. Jis mano tėvų Dievas, aš Jį aukštinsiu. 
\par 3 Viešpats yra karžygys, Jahvė yra Jo vardas. 
\par 4 Faraono kovos vežimus ir jo kariuomenę Jis įmetė į jūrą, jo rinktinius karo vadus paskandino Raudonojoje jūroje. 
\par 5 Gelmės apdengė juos, jie nugrimzdo į dugną kaip akmuo. 
\par 6 Tavo dešinė, Viešpatie, pasirodė šlovinga savo jėga! Tavo dešinė, Viešpatie, sutriuškino priešą. 
\par 7 Savo šlovės didybe Tu parbloškei tuos, kurie kėlėsi prieš Tave. Tu siuntei savo rūstybę, kuri suėdė juos kaip ražienas. 
\par 8 Tau papūtus, sujudo vandenys, sustojo kaip siena ir gelmės sustingo jūros širdyje. 
\par 9 Priešas tarė: ‘Vysiuos, sugausiu, pasidalinsiu grobį, pasitenkins mano aistra! Ištrauksiu kardą­juos sunaikins mano ranka!’ 
\par 10 Tu papūtei savo vėju, ir jūra uždengė juos, kaip švinas jie nuskendo galinguose vandenyse! 
\par 11 Viešpatie, kas yra Tau lygus tarp dievų? Kas yra toks šlovingas šventumu, didingas gyriumi ir savo stebuklais? 
\par 12 Tu ištiesei savo dešinę­juos prarijo žemė. 
\par 13 Tu, būdamas gailestingas, vedei tautą, kurią atpirkai. Savo galia vedei ją į savo šventąją buveinę. 
\par 14 Išgirs tautos ir sudrebės, baimė apims filistinus. 
\par 15 Nusigąs ir Edomo kunigaikščiai. Moabo galinguosius apims drebulys. Neteks jėgų visi Kanaano gyventojai. 
\par 16 Juos apims siaubas ir išgąstis. Dėl Tavo rankos galybės jie sustings kaip akmuo, kol pereis Tavo tauta, Viešpatie, kurią įsigijai. 
\par 17 Tu įvesi juos ir pasodinsi savo paveldėjimo kalne, į vietą, kurią Tu, Viešpatie, pasirinkai, kad gyventum, į šventyklą, kurią padarė Tavo rankos. 
\par 18 Viešpats karaliaus per amžius!” 
\par 19 Faraono žirgams, kovos vežimams ir raiteliams įėjus į jūrą, Viešpats užliejo juos vandenimis, o izraelitai perėjo sausu jūros dugnu. 
\par 20 Tada pranašė Mirjama, Aarono sesuo, paėmė būgną, ir visos moterys, eidamos paskui ją su būgnais, šoko. 
\par 21 Mirjama pritardama giedojo: “Giedokite Viešpačiui, nes Jis šlovingai nugalėjo. Žirgą ir raitelį Jis įmetė į jūrą”. 
\par 22 Mozė vedė Izraelį nuo Raudonosios jūros, ir jie įėjo į Šūro dykumą. Tris dienas jie keliavo dykuma ir nerado vandens. 
\par 23 Atėję į Marą, jie negalėjo gerti Maros vandens, nes jis buvo kartus. Todėl ta vieta vadinama Mara. 
\par 24 Tauta pradėjo murmėti prieš Mozę sakydami: “Ką gersime?” 
\par 25 Jis šaukėsi Viešpaties. Viešpats parodė jam medį, kurį įmetus į vandenį, vanduo tapo saldus. Čia Jis davė jiems įstatymą ir nuostatus, ir čia Jis išbandė juos. 
\par 26 Jis sakė: “Jei atidžiai klausysi Viešpaties, savo Dievo, ir darysi, kas teisu Jo akyse, kreipsi dėmesį į Jo įsakymus ir laikysiesi visų Jo nuostatų, tai ant tavęs neužleisiu nė vienos tų nelaimių, kurias užleidau ant egiptiečių, nes Aš esu Viešpats, tavo gydytojas”. 
\par 27 Jie atėjo į Elimą, kur buvo dvylika vandens šaltinių ir septyniasdešimt palmių; ten pasistatė stovyklą.



\chapter{16}


\par 1 Antrojo mėnesio penkioliktą dieną po to, kai jie paliko Egiptą, visi izraelitai išėjo iš Elimo ir atėjo į Sino dykumą, esančią tarp Elimo ir Sinajaus. 
\par 2 Visa Izraelio vaikų tauta murmėjo prieš Mozę ir Aaroną dykumoje. 
\par 3 Jie sakė: “Geriau mes būtume mirę nuo Viešpaties rankos Egipto šalyje, kai sėdėjome prie mėsos puodų ir valgėme duonos sočiai! Jūs mus atvedėte į šią dykumą, kad numarintumėte visus badu”. 
\par 4 Tada Viešpats tarė Mozei: “Aš jums duosiu duonos iš dangaus. Žmonės teišeina ir tesusirenka dienos davinį, kad išbandyčiau juos, ar jie laikysis mano įstatymo, ar ne. 
\par 5 O šeštą dieną teprisirenka dvigubai tiek, kiek kasdien prisirinkdavo”. 
\par 6 Tada Mozė ir Aaronas tarė visiems izraelitams: “Šį vakarą žinosite, kad Viešpats jus išvedė iš Egipto. 
\par 7 Ir rytą išvysite Viešpaties šlovę; Jis išgirdo jūsų murmėjimą prieš Jį. O kas esame mudu, kad murmate prieš mus? 
\par 8 Viešpats duos jums vakare mėsos, o rytą­duonos sočiai. Viešpats išgirdo, kad murmėjote prieš Jį. O kas mudu esame? Ne prieš mus jūs murmate, bet prieš Viešpatį”. 
\par 9 Tada Mozė tarė Aaronui: “Sakyk visiems izraelitams: ‘Priartėkite prie Viešpaties, nes Jis išgirdo jūsų murmėjimą’ ”. 
\par 10 Aaronui tebekalbant izraelitams, jie pažvelgė į dykumas ir pamatė Viešpaties šlovę debesyje. 
\par 11 Viešpats kalbėjo Mozei: 
\par 12 “Aš girdėjau Izraelio vaikų murmėjimą. Sakyk jiems: ‘Vakare jūs gausite mėsos, o rytą pasisotinsite duona. Ir jūs žinosite, kad Aš esu Viešpats, jūsų Dievas’ ”. 
\par 13 Vakare atskrido putpelės ir apdengė stovyklą, o rytą aplink stovyklą buvo rasa. 
\par 14 Rasai pranykus, dykumoje pasirodė kažkas, tarsi šerkšnas ant žemės. 
\par 15 Tai pamatę, izraelitai klausė vienas kito: “Kas čia?” Nė vienas nežinojo, kas tai yra. Tada Mozė jiems tarė: “Tai duona, kurią Viešpats jums davė maistui. 
\par 16 Štai ką Viešpats įsakė: ‘Kiekvienas teprisirenka tiek, kiek jis suvalgo; teparsineša po omerą kiekvienam žmogui, atsižvelgdamas į asmenų skaičių savo palapinėje’ ”. 
\par 17 Izraelitai darė, kaip buvo įsakyta, ir prisirinko vieni daugiau, kiti mažiau. 
\par 18 Ir kai jie seikėjo omeru, kas buvo prisirinkęs daug, neturėjo pertekliaus, o kas mažai, tam nestigo. Kiekvienas turėjo tiek, kiek galėjo suvalgyti. 
\par 19 Mozė jiems sakė: “Nė vienas tenepalieka nieko rytojui”. 
\par 20 Bet jie nepaklausė Mozės. Kai kurie paliko dalį maisto kitai dienai, tačiau atsirado kirmėlių ir jis pradėjo dvokti. Mozė užsirūstino ant jų. 
\par 21 Kiekvienas kas rytą rinkdavosi, kiek jis galėjo suvalgyti. O saulei kaitinant, tie grūdeliai laukuose sutirpdavo. 
\par 22 Šeštą dieną jie prisirinko dvigubai tiek: po du omerus kiekvienam. Izraelio vyresnieji apie tai pranešė Mozei. 
\par 23 Jis sakė jiems: “Taip Viešpats liepė: ‘Rytoj yra sabatas, šventa poilsio diena, skirta Viešpačiui. Ką norite kepti, iškepkite, ir ką norite virti, išvirkite, o kas lieka, atidėkite į šalį ir palaikykite rytojui’ ”. 
\par 24 Jie pasidėjo rytojui, kaip Mozė buvo įsakęs. Ir tai nesugedo ir neatsirado kirmėlių. 
\par 25 Po to Mozė tarė: “Valgykite tai! Nes šiandien yra Viešpaties sabatas, nieko laukuose nerasite. 
\par 26 Šešias dienas rinksite, o septintoji diena yra sabatas; joje nieko nerasite”. 
\par 27 Vis dėlto septintąją dieną kai kurie išėjo rinkti, bet nieko nerado. 
\par 28 Viešpats tarė Mozei: “Ar ilgai nesilaikysite mano įstatymų ir įsakymų? 
\par 29 Viešpats jums įsakė švęsti sabatą. Todėl šeštąją dieną Jis duoda jums duonos dviem dienom. Kiekvienas pasilikite savo vietoje. Nė vienas neišeikite iš savo namų septintąją dieną”. 
\par 30 Taip tauta ilsėjosi septintąją dieną. 
\par 31 Dievo duotą maistą izraelitai praminė mana. Ji buvo tarsi balti kalendros grūdeliai, o jos skonis kaip papločio su medumi. 
\par 32 Mozė paskelbė Viešpaties įsakymą: “Prisipilkite omerą manos, kad būsimos kartos žinotų, kokia duona jus maitinau dykumoje, kai išvedžiau iš Egipto”. 
\par 33 Mozė sakė Aaronui: “Paimk indą, supilk į jį vieną omerą manos ir jį padėk Viešpaties akivaizdoje, kad išliktų būsimoms kartoms”. 
\par 34 Kaip Viešpats įsakė Mozei, taip Aaronas padėjo indą palapinės švenčiausioje vietoje. 
\par 35 Izraelitai valgė maną keturiasdešimt metų, kol atėjo į gyvenamas žemes. Jie valgė maną, kol atėjo prie Kanaano šalies ribų. 
\par 36 Omeras yra dešimtoji efos dalis.



\chapter{17}


\par 1 Visi izraelitai keliavo toliau iš Sino dykumos sustodami, kai Viešpats įsakydavo. Jie pasistatė stovyklą Refidime. Čia jie nerado geriamo vandens 
\par 2 ir priekaištavo Mozei, sakydami: “Duok mums vandens atsigerti!” Mozė jiems atsakė: “Ko burnojate prieš mane? Kodėl gundote Viešpatį?” 
\par 3 Žmonės, ištroškę vandens, murmėjo prieš Mozę: “Kodėl mus išvedei iš Egipto, kad mus, mūsų vaikus ir gyvulius numarintum troškuliu?” 
\par 4 Tada Mozė šaukėsi Viešpaties: “Ką man daryti su šita tauta? Nedaug trūksta, kad jie užmuštų mane akmenimis”. 
\par 5 Viešpats atsakė Mozei: “Išeik prieš tautą kartu su Izraelio vyresniaisiais; laikyk rankoje lazdą, kuria sudavei į upę, ir eik. 
\par 6 Aš stovėsiu prieš tave ant uolos Horebe. Tu suduosi lazda į uolą, iš jos ištekės vanduo ir žmonės galės atsigerti”. Mozė taip ir padarė Izraelio vyresniųjų akyse. 
\par 7 Jis praminė tą vietą Masa ir Meriba dėl to, kad izraelitai priekaištavo ir gundė Viešpatį, sakydami: “Ar yra Viešpats tarp mūsų?” 
\par 8 Tada atėjo amalekiečiai ir kariavo su Izraeliu Refidime. 
\par 9 Mozė įsakė Jozuei: “Pasirink vyrų ir eik kariauti prieš amalekiečius! Rytoj aš atsistosiu ant kalvos viršūnės ir laikysiu Dievo lazdą rankoje”. 
\par 10 Jozuė padarė, kaip jam Mozė buvo įsakęs, ir kovojo prieš amalekiečius. Mozė, Aaronas ir Hūras užlipo ant kalvos viršūnės. 
\par 11 Kol Mozė laikydavo pakėlęs savo rankas, laimėdavo izraelitai; kai tik jas nuleisdavo, laimėdavo amalekiečiai. 
\par 12 Mozės rankos pavargo; jie tad paėmė akmenį, ir jis atsisėdo ant jo. Aaronas ir Hūras laikė jo rankas, vienas iš vienos, kitas iš kitos pusės; taip jo rankos buvo pakeltos iki saulės nusileidimo. 
\par 13 Tuo būdu Jozuė nugalėjo Amaleką ir jo tautą savo kardu. 
\par 14 Viešpats liepė Mozei: “Įrašyk tai į knygą atminimui ir perskaityk Jozuei girdint, kad Aš išnaikinsiu atminimą apie amalekiečius iš po dangaus”. 
\par 15 Mozė pastatė aukurą ir jį pavadino “Viešpats yra mano vėliava”. 
\par 16 Nes jis sakė: “Viešpats prisiekė ir Jis kariaus prieš amalekiečius per kartų kartas”.



\chapter{18}


\par 1 Jetras, Midjano kunigas, Mozės uošvis, išgirdo visa, ką Dievas padarė Mozei ir savo tautai Izraeliui, ir kad Viešpats išvedė Izraelį iš Egipto. 
\par 2 Tai sužinojęs, Jetras, Mozės uošvis, paėmė Ciporą, Mozės žmoną, kurią Mozė buvo parsiuntęs atgal, 
\par 3 ir jos abu sūnus, kurių vieno vardas buvo Geršomas, nes jis sakė: “Buvau ateivis svetimoje šalyje”, 
\par 4 o kito Eliezeras, nes sakė: “Mano tėvo Dievas buvo mano pagalba ir mane išgelbėjo nuo faraono kardo”. 
\par 5 Ir Jetras, Mozės uošvis, atėjo su jo sūnumis ir žmona pas Mozę į dykumą, kur jis stovyklavo prie Dievo kalno. 
\par 6 Jetras pranešė Mozei: “Štai aš, tavo uošvis Jetras, su tavo žmona ir abiem sūnumis einame pas tave!” 
\par 7 Tada Mozė, išėjęs pasitikti uošvio, nusilenkė jam ir jį pabučiavo. Paklausę vienas kito, kaip sekasi, įėjo į palapinę. 
\par 8 Mozė papasakojo savo uošviui visa, ką Viešpats padarė faraonui ir egiptiečiams dėl Izraelio, ir visą vargą kelyje, ir kaip Viešpats juos išgelbėjo. 
\par 9 Jetras džiaugėsi visu tuo, ką Viešpats padarė Izraeliui išgelbėdamas jį iš egiptiečių, 
\par 10 ir tarė: “Tebūna palaimintas Viešpats, kuris jus išgelbėjo iš egiptiečių ir faraono, kuris išgelbėjo tautą iš Egipto vergovės. 
\par 11 Dabar žinau, kad Viešpats yra aukščiau visų kitų dievų, nes pranoko juos tuo, kuo jie didžiavosi”. 
\par 12 Po to Jetras, Mozės uošvis, aukojo Dievui deginamąją auką ir atnašas. Tada Aaronas ir visi Izraelio vyresnieji drauge su Mozės uošviu valgė Dievo akivaizdoje. 
\par 13 Kitą dieną Mozė atsisėdo žmonių teisti, ir žmonės stovėjo prie jo nuo ryto iki vakaro. 
\par 14 Mozės uošvis, pamatęs visa, ką jis darė žmonėms, klausė: “Ką tu čia darai? Kodėl tu sėdi vienas, o visi žmonės stovi nuo ryto iki vakaro?” 
\par 15 Mozė atsakė: “Žmonės ateina pas mane pasiklausti Dievo patarimų. 
\par 16 Kai jie nesutaria, ateina pas mane, kad būčiau jų teisėju ir paskelbčiau Dievo nuostatus ir įstatymus”. 
\par 17 Mozės uošvis atsakė: “Negerai darai! 
\par 18 Tu pats ir šitie žmonės, kurie yra su tavimi, visiškai nuvargsite, nes tau tai per sunku; tu negali vienas atlikti to darbo. 
\par 19 Dabar paklausyk mano balso. Aš tau patarsiu, ir Dievas bus su tavimi! Būk tarpininkas tarp tautos ir Dievo ir pranešk jų reikalus Dievui. 
\par 20 Mokyk juos įstatymų bei nuostatų, parodyk jiems kelią, kuriuo jie turi eiti, ir darbus, kuriuos turi daryti. 
\par 21 Be to, išsirink iš tautos sumanius, Dievo bijančius, patikimus, negodžius vyrus, paskirk juos tūkstantininkais, šimtininkais, penkiasdešimtininkais ir dešimtininkais, 
\par 22 ir tegu jie teisia žmones. Kiekvieną didelę bylą jie perduos tau, o mažas bylas spręs patys. Tau bus lengviau, ir jie pasidalins naštą su tavimi. 
\par 23 Jeigu taip darysi ir Dievas įsakys tau, pats ištversi ir visi žmonės grįš į savo namus ramybėje”. 
\par 24 Mozė paklausė savo uošvio patarimo. 
\par 25 Jis išsirinko sumanius vyrus iš viso Izraelio ir paskyrė juos vyresniaisiais: tūkstantininkais, šimtininkais, penkiasdešimtininkais ir dešimtininkais. 
\par 26 Ir jie teisė žmones visą laiką. Sunkias bylas jie perduodavo Mozei, bet mažas sprendė patys. 
\par 27 Po to Mozės uošvis iškeliavo į savo šalį.



\chapter{19}

\par 1 Trečią mėnesį po to, kai išėjo iš Egipto, Izraelio vaikai tą pačią dieną atvyko į Sinajaus dykumą. 
\par 2 Nes jie buvo išėję iš Refidimo ir, atėję į Sinajaus dykumą, apsistojo. Jie pasistatė stovyklą ties kalnu, 
\par 3 ir Mozė užkopė ant kalno pas Dievą. Viešpats pašaukė jį nuo kalno, sakydamas: “Taip sakysi Jokūbo namams ir Izraelio vaikams: 
\par 4 ‘Jūs matėte, ką padariau egiptiečiams, kaip nešiau jus lyg ant erelio sparnų ir atgabenau jus prie savęs. 
\par 5 Taigi dabar, jei paklusite mano balsui ir laikysitės mano sandoros, būsite ypatinga mano nuosavybė tarp visų tautų. Man priklauso visa žemė. 
\par 6 Jūs būsite man kunigų karalystė ir šventa tauta’. Šiuos žodžius kalbėk izraelitams”. 
\par 7 Sugrįžęs Mozė sušaukė tautos vyresniuosius ir jiems pranešė viską, ką Viešpats įsakė. 
\par 8 Tada visa tauta atsakė: “Visa, ką Viešpats kalbėjo, darysime”. Mozė perdavė tautos atsakymą Viešpačiui. 
\par 9 Viešpats sakė Mozei: “Aš ateisiu pas tave tirštame debesyje, kad, man kalbant su tavimi, tauta girdėtų ir tavimi tikėtų per amžius”. Mozė pranešė tautos žodžius Viešpačiui, 
\par 10 o Jis tarė Mozei: “Eik pas žmones ir pašventink juos šiandien ir rytoj. Teišsiskalbia jie drabužius 
\par 11 ir tepasirengia trečiajai dienai. Trečią dieną Aš nužengsiu ant Sinajaus kalno visai tautai matant. 
\par 12 Nužymėk ribas aplink kalną ir įsakyk žmonėms, kad nedrįstų lipti į kalną ar paliesti jo kraštą, nes kiekvienas, kuris palies kalną, numirs. 
\par 13 Kas palies, tas bus užmuštas akmenimis arba pervertas strėle; nei žmogus, nei gyvulys neišliks gyvas. Išgirdę ilgą trimito garsą, jie tepriartėja prie kalno”. 
\par 14 Mozė, grįžęs nuo kalno, pašventino žmones. Ir jie išsiplovė drabužius. 
\par 15 Jis jiems tarė: “Būkite pasiruošę trečiajai dienai; nė vienas tenesiartina prie savo žmonos!” 
\par 16 Trečiąją dieną, rytui išaušus griaudėjo perkūnija ir žaibavo, tamsūs debesys apdengė kalną ir pasigirdo labai stiprus trimito garsas. Visi žmonės stovykloje pradėjo drebėti. 
\par 17 Mozė išvedė tautą iš stovyklos susitikti su Dievu. Jie sustojo kalno papėdėje. 
\par 18 Visas Sinajaus kalnas buvo apgaubtas dūmų, nes Viešpats nužengė ugnyje ant jo. Dūmai kilo tarsi iš krosnies, visas kalnas smarkiai drebėjo. 
\par 19 Trimito garsas vis stiprėjo. Mozė kalbėjo, o Dievas jam atsakinėjo balsu. 
\par 20 Viešpats nužengė ant Sinajaus kalno viršūnės ir, pasišaukęs Mozę ant kalno, 
\par 21 jam tarė: “Nulipk žemyn, įspėk žmones, kad nesiveržtų, norėdami išvysti Viešpatį, ir daugelis jų nežūtų. 
\par 22 Taip pat ir kunigai, kurie artinasi prie Viešpaties, tegul pasišventina, kad Viešpats jų neištiktų”. 
\par 23 Mozė sakė Viešpačiui: “Žmonės negali užlipti į Sinajaus kalną. Tu mus įspėjai nužymėti ribą aplink kalną ir pašventinti jį”. 
\par 24 Viešpats jam atsakė: “Nusileisk ir grįžk su Aaronu. Tačiau kunigai ir tauta tegul nesiveržia prie Viešpaties, kad Jis jų nesunaikintų”. 
\par 25 Mozė grįžo pas tautą ir kalbėjo jiems.



\chapter{20}

\par 1 Dievas kalbėjo visus šiuos žodžius: 
\par 2 “Aš esu Viešpats, tavo Dievas, kuris tave išvedžiau iš Egipto žemės, iš vergijos namų. 
\par 3 Neturėk kitų dievų šalia manęs. 
\par 4 Nedaryk sau jokio drožinio nei jokio atvaizdo to, kas yra aukštai danguje, žemai žemėje ar po žeme vandenyje. 
\par 5 Nesilenk prieš juos ir netarnauk jiems! Nes Aš, Viešpats, tavo Dievas, esu pavydus Dievas, baudžiąs vaikus už tėvų kaltes iki trečios ir ketvirtos kartos tų, kurie manęs nekenčia, 
\par 6 bet rodąs gailestingumą iki tūkstantosios kartos tiems, kurie mane myli ir laikosi mano įsakymų. 
\par 7 Netark Viešpaties, savo Dievo, vardo be reikalo, nes Viešpats nepaliks be kaltės to, kuris be reikalo mini Jo vardą. 
\par 8 Atsimink sabato dieną, kad ją švęstum. 
\par 9 Šešias dienas dirbk ir atlik visus savo darbus, 
\par 10 o septintoji diena yra sabatas Viešpačiui, tavo Dievui. Joje nevalia dirbti jokio darbo nei tau, nei tavo sūnui ar dukteriai, nei tavo tarnui ar tarnaitei, nei tavo gyvuliui, nei ateiviui, kuris yra tavo namuose, 
\par 11 nes per šešias dienas Viešpats sukūrė dangų, žemę, jūrą ir visa, kas juose yra, o septintąją dieną ilsėjosi. Todėl Viešpats palaimino sabatą ir pašventino jį. 
\par 12 Gerbk savo tėvą ir motiną, kad ilgai gyventum žemėje, kurią Viešpats Dievas tau duoda. 
\par 13 Nežudyk. 
\par 14 Nesvetimauk. 
\par 15 Nevok. 
\par 16 Neliudyk neteisingai prieš savo artimą. 
\par 17 Negeisk savo artimo namų, negeisk savo artimo žmonos, nei jo tarno, nei tarnaitės, nei jaučio, nei asilo—nieko, kas yra tavo artimo”. 
\par 18 Visi žmonės girdėjo griaustinį ir trimito garsą, matė žaibus ir rūkstantį kalną. Matydami tai žmonės atsitraukė ir stovėjo atokiai. 
\par 19 Jie sakė Mozei: “Tu kalbėk su mumis! Mes klausysime. Tačiau tenekalba su mumis Dievas, kad nemirtume!” 
\par 20 Mozė atsakė žmonėms: “Nebijokite! Dievas atėjo tam, kad jus išmėgintų, ir kad Jo baimė būtų prieš jus ir jūs nenusidėtumėte”. 
\par 21 Žmonės stovėjo toli, o Mozė priartėjo prie tamsaus debesies, kur buvo Dievas. 
\par 22 Viešpats liepė Mozei pasakyti izraelitams: “Jūs patys girdėjote, kad Aš iš dangaus kalbėjau su jumis. 
\par 23 Nedarykite šalia manęs sau sidabrinių ar auksinių dievų. 
\par 24 Padaryk man iš žemės aukurą ir aukok ant jo deginamąsias ir padėkos aukas, savo avis ir galvijus. Kiekvienoje vietoje, kurioje bus minimas mano vardas, Aš ateisiu ir laiminsiu tave. 
\par 25 O jei man statysi akmeninį aukurą, nestatyk jo iš tašytų akmenų, nes, naudodamas įrankį, suterši jį. 
\par 26 Taip pat nelipk laiptais prie mano aukuro, kad nebūtų atidengtas tavo nuogumas”.



\chapter{21}


\par 1 “Tai nuostatai, kuriuos jiems pateiksi. 
\par 2 Jei pirksi vergą hebrają, jis tau tarnaus šešerius metus, o septintaisiais paleisi jį be išpirkimo. 
\par 3 Jei jis atėjo vienas, vienas teišeina. Jei atėjo vedęs, jo žmona teišeina su juo. 
\par 4 Jei jo šeimininkas davė jam žmoną ir ji pagimdė sūnų ir dukterų, žmona ir jos vaikai lieka šeimininkui, o jis vienas teišeina. 
\par 5 Bet jei vergas aiškiai pasakys: ‘Aš myliu savo šeimininką, žmoną bei vaikus ir atsisakau laisvės’, 
\par 6 tai jo šeimininkas atves jį pas teisėjus, prives jį prie durų arba prie durų staktos ir perdurs yla jo ausį; ir jis liks jam tarnauti visą gyvenimą. 
\par 7 Jei kas parduoda savo dukterį vergijon, ji neišeis, kaip išeina vergai. 
\par 8 Jei ji nepatinka šeimininkui, kuris su ja susižadėjo, jis turi leisti ją išpirkti. Jis neturi teisės parduoti jos svetimšaliams, nes apgavo ją. 
\par 9 O jei jis sužadėjo ją su savo sūnumi, privalo elgtis su ja kaip su dukterimi. 
\par 10 Jei jis paims jam kitą, jis neturi teisės sumažinti jai maisto, rūbų ir santuokinių teisių. 
\par 11 Jei jis neatlieka jai šitų trijų dalykų, ji išeis be išpirkimo mokesčio. 
\par 12 Kas sumuša žmogų taip, kad jis miršta, tas baudžiamas mirtimi. 
\par 13 O jei žmogus negalvojo žudyti, bet Dievas atidavė jį į jo rankas, tai Aš paskirsiu vietą, kur jis galėtų pabėgti. 
\par 14 Jei kas savo artimą tyčiomis nužudo panaudodamas klastą, tą paimk ir nuo mano aukuro, kad jis mirtų. 
\par 15 Kas suduoda savo tėvui arba motinai, tas baudžiamas mirtimi. 
\par 16 Kas pavagia žmogų ir parduoda jį, ar jis surandamas pas jį, baudžiamas mirtimi. 
\par 17 Kas keikia savo tėvą ar motiną, tas baudžiamas mirtimi. 
\par 18 Jei vyrams susivaidijus, vienas taip sumuša kitą akmeniu ar kumščiu, kad tas nemiršta, bet atsigula į lovą, 
\par 19 ir jeigu jis atsikelia ir pasiremdamas lazda gali vaikščioti, sumušėjas nebaudžiamas, tik privalo atlyginti už sugaištą laiką ir sumokėti visas gydymo išlaidas. 
\par 20 Jei kas taip sumuša vergą ar vergę lazda, kad tas tuojau miršta,­bus nubaustas. 
\par 21 O jei jis išgyvena vieną ar dvi dienas, jis nebaudžiamas, nes vergas yra jo nuosavybė. 
\par 22 Jei vyrai vaidijasi ir užgauna nėščią moterį ir ji persileidžia, bet jos pačios nesužaloja, tada užgavėjas baudžiamas pinigine bauda, kokią jam paskiria tos moters vyras, teisėjams tarpininkaujant. 
\par 23 Bet jei sužaloja­gyvybė už gyvybę, 
\par 24 akis už akį, dantis už dantį, ranka už ranką, koja už koją, 
\par 25 nudeginimas už nudeginimą, žaizda už žaizdą, randas už randą. 
\par 26 Jei kas išmuša savo vergui ar vergei akį, jis privalo už tai paleisti jį laisvėn. 
\par 27 Jei jis išmuša savo vergui ar vergei dantį, jis privalo už tai paleisti jį laisvėn. 
\par 28 Jei jautis taip subado vyrą ar moterį, kad tas miršta, jautį užmuškite akmenimis ir nevalgykite jo mėsos. Tačiau jaučio savininkas yra nekaltas. 
\par 29 Bet jeigu jautis jau anksčiau badydavo ir jo savininkas buvo įspėtas, tačiau jo neuždarė, ir jei jis mirtinai subadė vyrą ar moterį, jautį užmuškite akmenimis, o jo savininką taip pat bauskite mirtimi. 
\par 30 O jei jam bus leista išsipirkti, jis mokės išpirką už savo gyvybę tiek, kiek jam bus paskirta. 
\par 31 Jei jautis subado sūnų ar dukterį, laikykitės tos pačios taisyklės. 
\par 32 Jei jautis subado vergę ar vergą, tai savininkas sumokės vergo šeimininkui trisdešimt šekelių sidabro, o jautį užmuškite akmenimis. 
\par 33 Jei kas atidengia duobę arba jei kas iškasa duobę, bet jos neuždengia, ir į ją įkrinta jautis ar asilas, 
\par 34 duobės savininkas atlygins nuostolį, sumokėdamas pinigus gyvulio savininkui, o nugaišęs gyvulys liks jam. 
\par 35 Jei kieno jautis taip sužaloja kito jautį, kad tas nugaišta, tai jie parduos gyvąjį jautį ir pasidalins už jį gautus pinigus. Taip pat jie pasidalins ir nugaišusį gyvulį. 
\par 36 O jei buvo žinoma, kad jautis jau anksčiau badydavo ir savininkas jo neuždarė, tai jis atiduos jautį už jautį, o nugaišęs priklausys jam”.



\chapter{22}


\par 1 “Jei kas pavagia jautį ar avį ir jį papjauna ar parduoda, jis sugrąžins penkis jaučius už jautį ir keturias avis už avį. 
\par 2 Jei kas užklumpa vagį besilaužiantį ir suduoda jam taip, kad tas numiršta, jis nebaudžiamas už pralietą kraują. 
\par 3 Jei jis tai padarytų dienos metu, jis kaltinamas už pralietą kraują. Vagis privalo viską atlyginti. Jei jis nieko neturi, jį parduosite už vagystę. 
\par 4 Jei pas jį randamas dar gyvas pavogtas jautis ar avis, jis atlygins dvigubai. 
\par 5 Jei kas nugano lauką ar vynuogyną ir leidžia savo gyvulius ganytis svetimame lauke, tas privalo atlyginti geriausiu, kas yra jo paties lauke ar vynuogyne. 
\par 6 Jei ugnis išsiplečia ir apima erškėčius, ir sudega sustatyti pėdai ar tebeaugantys javai lauke, tai tas, kuris užkūrė ugnį, atlygina visą nuostolį. 
\par 7 Jei kas paveda savo artimui saugoti pinigus ar kitokius daiktus ir jie pavagiami iš jo namų, tai, suradus vagį, jis privalo dvigubai atlyginti. 
\par 8 Jei vagies nesuranda, namų savininką atvesite pas teisėjus, kad ištirtų, ar jis nepridėjo rankos prie savo artimo nuosavybės. 
\par 9 Kai kyla ginčas dėl nuosavybės: jaučio, asilo, avies, apsiausto ar dėl bet kokio kito pamesto daikto, apie kurį kitas tvirtina, kad tai jo,­abu privalo ateiti pas teisėjus. Kuris kaltas, privalo dvigubai atlyginti savo artimui. 
\par 10 Jei kas paveda savo artimui saugoti asilą, jautį, avį ar bet kokį gyvulį ir tas, niekam nematant, pastimpa, susižeidžia ar nuvaromas, 
\par 11 tas, kuris saugojo, turi prisiekti prieš Viešpatį, kad nepridėjo rankos prie artimo nuosavybės. Tada savininkas privalo sutikti su tuo ir jis neturės atlyginti nuostolio. 
\par 12 O jei bus pavogta iš jo, jis atlygins savininkui. 
\par 13 Jei gyvulį sudrasko žvėrys, jis privalo liekanas atgabenti įrodymui ir jam nereikės atlyginti už tai, kas sudraskyta. 
\par 14 Jei kas pasiskolina iš savo artimo ir tai sugadinama ar pastimpa, kai savininko nėra šalia, tai jis privalo atlyginti nuostolį. 
\par 15 Jei savininkas buvo šalia, atlyginti nereikia. Jei buvo išnuomota, reikia sumokėti tik už nuomą. 
\par 16 Jei kas suvedžioja mergaitę dar nesužadėtą, jis privalo ją vesti ir duoti jai kraitį. 
\par 17 Jei jos tėvas nesutinka jos išleisti už jo, tai jis sumokės tiek pinigų, kiek mokama kraičiui mergaitei. 
\par 18 Būrėjams neleisi gyviems išlikti. 
\par 19 Kas santykiauja su gyvuliu, turi būti baudžiamas mirtimi. 
\par 20 Kas aukoja kitiems dievams, o ne Viešpačiui, turi būti sunaikintas. 
\par 21 Ateivio neskriausi ir nespausi, nes jūs patys buvote ateiviai Egipto šalyje. 
\par 22 Neskriauskite našlės ar našlaičio. 
\par 23 Jei juos skriausi ir jie šauksis mano pagalbos, Aš tikrai išklausysiu jų šauksmą. 
\par 24 Tada savo rūstybėje išžudysiu jus kardu: jūsų žmonos liks našlėmis ir vaikai našlaičiais. 
\par 25 Jei paskolinsi pinigų mano tautos beturčiui, gyvenančiam šalia tavęs, nepasidaryk lupikas ir neapkrauk jo palūkanomis. 
\par 26 Jei paimsi kaip užstatą savo artimo apsiaustą, privalai jam grąžinti jį iki saulės laidos. 
\par 27 Nes tai yra jo vienintelis apsiaustas kūnui pridengti. Kuo kitu jis apsidengs miegodamas? Ir kai jis šauksis manęs, išklausysiu, nes esu gailestingas. 
\par 28 Nekalbėk pikta prieš tautos teisėjus ir nekeik tautos vadovo. 
\par 29 Savo javų ir vaisių pirmienų nedelsk pristatyti. Pirmagimį savo sūnų atiduosi man. 
\par 30 Taip pat pasielgsi su savo jaučiais ir avimis. Septynias dienas jis pasiliks prie savo motinos, aštuntąją dieną atiduosi jį man. 
\par 31 Jūs būsite šventi žmonės man, ir žvėries sudraskyto gyvulio nevalgysite; šuniui jį numeskite”.



\chapter{23}

\par 1 “Nepatikėk melagingu skundu ir neprisidėk prie piktadario neteisingai liudyti. 
\par 2 Nesek paskui minią daryti pikta ir neprisidėk teisme prie daugumos iškreipti teisingumo. 
\par 3 Nepataikauk beturčiui jo byloje. 
\par 4 Jei sutinki beklaidžiojantį savo priešo jautį ar asilą, privalai jį nuvesti jam. 
\par 5 Jei matai tavęs nekenčiančio žmogaus asilą, parkritusį po našta, nepraeik, bet padėk jį pakelti. 
\par 6 Neiškraipyk teisingumo beturčio byloje. 
\par 7 Šalinkis neteisybės, nekalto bei teisaus nežudyk, nes Aš neišteisinsiu piktadario. 
\par 8 Kyšių nepriimk, nes kyšiai apakina net išmintinguosius ir iškraipo teisiųjų žodžius. 
\par 9 Neskriausk ateivio. Jūs žinote, kaip jaučiasi ateivis, nes buvote ateiviai Egipto šalyje. 
\par 10 Šešerius metus sėsi savo žemėje ir surinksi jos vaisius, 
\par 11 o septintaisiais metais leisi jai pailsėti, paliksi ją neapsėtą, kad tavo tautos beturčiai iš jos maitintųsi, ir tai, ką jie paliks, ėstų laukiniai žvėrys. Taip pat pasielgsi su savo vynuogynu ir alyvmedžiais. 
\par 12 Šešias dienas dirbk savo darbą, o septintąją dieną ilsėkis, kad pailsėtų tavo jautis bei asilas ir atsikvėptų vergės sūnus bei ateivis. 
\par 13 Laikykitės viso, ką jums įsakiau. Neminėkite kitų dievų vardo, tenesigirdi to iš jūsų lūpų. 
\par 14 Tris kartus per metus švęsi man šventes. 
\par 15 Švęsk Neraugintos duonos šventę. Septynias dienas valgyk neraugintą duoną, kaip įsakiau nustatytu laiku, Abibo mėnesį, nes jį išėjai iš Egipto. Nepasirodyk mano akyse tuščiomis rankomis. 
\par 16 Švęsk Pirmųjų vaisių derliaus šventę ir metų pabaigoje, kai nuimsi laukų derlių, Derliaus nuėmimo šventę. 
\par 17 Tris sykius per metus kiekvienas vyras tepasirodo Viešpaties akivaizdoje. 
\par 18 Neaukok mano aukos kraujo drauge su rauginta duona ir nelaikyk iki ryto mano aukos riebalų. 
\par 19 Pirmuosius savo dirvos vaisius atgabenk į Viešpaties, tavo Dievo, namus. Nevirk ožiuko jo motinos piene. 
\par 20 Aš siunčiu angelą pirma tavęs, kad saugotų tave kelyje ir nuvestų į vietą, kurią tau paruošiau. 
\par 21 Saugokis jo ir klausyk jo balso, nesipriešink jam, nes jis neatleis jūsų nusižengimų, kadangi jame yra mano vardas. 
\par 22 Bet jei tu iš tiesų paklusi jo balsui ir vykdysi visa, ką kalbu, tai Aš būsiu priešas tavo priešams ir prispaudėjas tavo prispaudėjams. 
\par 23 Mano angelas eis pirma tavęs ir tave nuves pas amoritus, hetitus, perizus, kanaaniečius, hivus ir jebusiečius, ir Aš juos sunaikinsiu. 
\par 24 Nesilenk prieš jų dievus ir netarnauk jiems. Nesielk, kaip jie elgiasi, bet visiškai sunaikink juos ir sudaužyk jų atvaizdus. 
\par 25 Jūs tarnausite Viešpačiui, savo Dievui, ir Jis laimins tavo duoną bei vandenį; ir Aš pašalinsiu jų ligas. 
\par 26 Nebus nelaiku gimdančios ir nevaisingos tavo šalyje, ir išpildysiu tavo dienų skaičių. 
\par 27 Sukelsiu baimę tarp tavo priešų ir sąmyšį tautoje, su kuria kariausi, ir priversiu visus tavo priešus bėgti. 
\par 28 Aš siųsiu pirma tavęs širšes, kad jos nuvytų nuo tavęs hivus, kanaaniečius ir hetitus. 
\par 29 Aš nenuvarysiu jų nuo tavo veido per vienerius metus, kad žemė netaptų dykuma ir nepadaugėtų laukinių žvėrių. 
\par 30 Pamažu juos išvarysiu iš krašto, kol tu išsiplėsi ir užvaldysi žemę. 
\par 31 Aš nustatysiu tavo krašto ribas nuo Raudonosios jūros iki filistinų jūros ir nuo dykumos iki upės. Atiduosiu tos šalies gyventojus į tavo rankas, ir tu juos išvarysi iš krašto. 
\par 32 Nedaryk sandoros nei su jais, nei su jų dievais. 
\par 33 Jie neturi gyventi tavo šalyje, kad tavęs nesugundytų nusidėti prieš mane. Jei tarnausi jų dievams, tai taps tau spąstais”.



\chapter{24}


\par 1 Viešpats sakė Mozei: “Tu, Aaronas, Nadabas ir Abi-huvas bei septyniasdešimt Izraelio vyresniųjų užlipkite ant kalno ir pagarbinkite mane iš tolo. 
\par 2 Mozė vienas priartės prie Viešpaties, o jie tegul nesiartina, ir tauta tegu neina su juo”. 
\par 3 Mozė atėjo ir pranešė tautai visus Viešpaties žodžius ir nuostatus. Tada visa tauta atsakė vienu balsu: “Visus žodžius, kuriuos Viešpats kalbėjo, vykdysime”. 
\par 4 Mozė surašė visus Viešpaties žodžius. Anksti rytą atsikėlęs, jis pastatė aukurą kalno papėdėje ir dvylika stulpų pagal dvylika Izraelio giminių. 
\par 5 Ir jis pasiuntė jaunus vyrus iš Izraelio vaikų, kurie aukojo deginamąsias aukas ir jaučius kaip padėkos auką Viešpačiui. 
\par 6 Mozė ėmė pusę aukojamųjų gyvulių kraujo ir supylė į dubenis, o kitą pusę iššlakstė ant aukuro. 
\par 7 Po to jis, paėmęs sandoros knygą, tautai girdint, perskaitė. Jie sakė: “Visa, ką Viešpats įsakė, vykdysime ir būsime klusnūs”. 
\par 8 Tada Mozė ėmė kraujo ir, šlakstydamas jį ant tautos, tarė: “Štai kraujas sandoros, kurią Viešpats padarė su jumis šiais žodžiais”. 
\par 9 Tada užkopė Mozė, Aaronas, Nadabas, Abihuvas ir septyniasdešimt Izraelio vyresniųjų. 
\par 10 Ir jie matė Izraelio Dievą: po Jo kojomis buvo tarsi grindinys iš safyro akmens, tarsi skaidrus dangus. 
\par 11 Ir Jis nekėlė savo rankos prieš Izraelio vaikų vyresniuosius. Jie matė Dievą, valgė ir gėrė. 
\par 12 Viešpats tarė Mozei: “Užkopk pas mane ant kalno ir būk čia. Aš duosiu tau akmenines plokštes, įstatymą ir įsakymus, kuriuos surašiau, kad galėtum juos pamokyti”. 
\par 13 Tai pakilo Mozė ir jo tarnas Jozuė, ir Mozė užlipo į Dievo kalną. 
\par 14 Mozė tarė vyresniesiems: “Palaukite čia mūsų, kol sugrįšime. Aaronas ir Hūras yra su jumis; kas turi kokią bylą, kreipkitės į juos”. 
\par 15 Mozė užkopė į kalną, ir debesis apdengė kalną. 
\par 16 Viešpaties šlovė nusileido ant Sinajaus kalno ir pasiliko šešias dienas. Septintą dieną Jis pašaukė Mozę iš debesies. 
\par 17 Viešpaties šlovė atrodė izraelitams lyg deganti ugnis kalno viršūnėje. 
\par 18 Mozė įėjo į debesį ir užlipo į kalną. Jis pasiliko kalne keturiasdešimt parų.



\chapter{25}


\par 1 Viešpats kalbėjo Mozei: 
\par 2 “Sakyk Izraelio vaikams, kad jie atneštų man auką. Priimkite auką iš kiekvieno, kuris duoda laisva valia. 
\par 3 Štai kokias aukas imsite iš jų: auksą, sidabrą, varį, 
\par 4 mėlynus, violetinius ir raudonus siūlus, ploną drobę, ožkų vilną, 
\par 5 raudonai dažytus avinų kailius, opšrų kailius, akacijos medį, 
\par 6 aliejų lempoms, kvepalus patepimo aliejui ir kvepiantiems smilkalams, 
\par 7 onikso akmenėlius ir brangius akmenis efodui bei krūtinės skydeliui. 
\par 8 Padarykite man šventyklą, kad galėčiau gyventi tarp jūsų. 
\par 9 Tau parodysiu palapinės ir visų daiktų, kurie turės būti joje, pavyzdį; viską privalote taip padaryti. 
\par 10 Padarykite skrynią iš akacijos medžio, pustrečios uolekties ilgio, pusantros pločio ir pusantros aukščio. 
\par 11 Aptraukite ją iš vidaus ir iš išorės grynu auksu ir jos viršuje padarykite auksinį apvadą. 
\par 12 Nuliekite keturias auksines grandis ir jas pritvirtinkite prie keturių kampų­dvi grandis iš vienos pusės ir dvi iš kitos. 
\par 13 Be to, padarykite kartis iš akacijos medžio ir jas aptraukite auksu. 
\par 14 Įkiškite kartis į grandis skrynios šonuose, kad galėtumėte nešioti skrynią. 
\par 15 Kartys telieka skrynios grandyse­ neištraukite jų. 
\par 16 Į skrynią įdėsi liudijimą, kurį tau duosiu. 
\par 17 Padarysi dangtį iš gryno aukso pustrečios uolekties ilgio ir pusantros pločio. 
\par 18 Iš gryno aukso padarysi du cherubus, nukalsi juos abiejuose dangčio galuose: 
\par 19 vieną cherubą viename gale, o kitą kitame. Ant dangčio padarykite cherubus abiejuose jo galuose. 
\par 20 Cherubų sparnai bus išskėsti, jie gaubs dangtį savo sparnais. Jų veidai bus nukreipti vienas į kitą ir į dangtį. 
\par 21 Į skrynią įdėsi liudijimą, kurį tau duosiu, ir užvoši dangčiu. 
\par 22 Ten Aš susitiksiu su tavimi ir Aš kalbėsiu su tavimi nuo dangčio viršaus tarp abiejų cherubų, kurie yra ant Liudijimo skrynios, ir duosiu tau įsakymus Izraelio vaikams. 
\par 23 Padarysi stalą iš akacijos medžio: dviejų uolekčių ilgio, uolekties pločio ir pusantros uolekties aukščio. 
\par 24 Jį aptrauksi grynu auksu, pakraščiu aplinkui pritaisysi auksinį apvadą. 
\par 25 Aplinkui padarysi briauną plaštakos platumo ir ant jos auksinį apvadą. 
\par 26 Padarysi keturias auksines grandis ir jas pritvirtinsi prie keturių kampų, prie kiekvienos kojos. 
\par 27 Grandys kartims įkišti bus prie pat briaunos, kad stalą galima būtų nešti. 
\par 28 Padarysi kartis iš akacijos medžio ir jas aptrauksi auksu, kad jomis galima būtų nešti stalą. 
\par 29 Iš gryno aukso padarysi dubenis, taures, smilkytuvus ir puodelius, kuriais bus liejami skysčiai. 
\par 30 Padėtinę duoną nuolat laikysi padėjęs ant stalo mano akivaizdoje. 
\par 31 Padarysi žvakidę iš gryno aukso; žvakidė turi būti nukalta iš vieno gabalo: jos šakos, kotas, taurelės, buoželės ir žiedai. 
\par 32 Šešios šakos eis iš jos šonų: trys šakos iš vienos ir trys iš kitos pusės. 
\par 33 Trys riešuto pavidalo taurelės, buoželė ir žiedas bus ant vienos šakos; trys riešuto pavidalo taurelės, buoželė ir žiedas ant kitos šakos, taip visoms šakoms, išeinančioms iš žvakidės. 
\par 34 O prie pačios žvakidės koto bus keturios riešuto pavidalo taurelės, buoželės ir žiedai. 
\par 35 Viena buoželė po dviem iš jos išeinančiom šakom, kita po dviem iš jos išeinančiom šakom ir trečia po likusiom dviem šakom, išeinančiom iš žvakidės. 
\par 36 Jų buoželės ir šakos turi būti iš vieno gabalo; visa žvakidė turi būti nukalta iš gryno aukso. 
\par 37 Padarysi septynis žibintus ir juos pastatysi ant žvakidės. 
\par 38 Gnybtuvus ir indą nuognaibom padarysi iš gryno aukso. 
\par 39 Visa tai pagaminsi iš vieno talento gryno aukso. 
\par 40 Žiūrėk, kad viską padarytum pagal pavyzdį, kurį tau parodžiau kalne”.



\chapter{26}


\par 1 “Padarysi palapinei dešimt uždangalų iš suktų plonų siūlų drobės su mėlynų, violetinių bei raudonų siūlų su išsiuvinėtais cherubais. 
\par 2 Vieno uždangalo ilgis bus dvidešimt aštuonios uolektys, plotis­keturios. Visi uždangalai vienodo dydžio. 
\par 3 Penkis uždangalus susegsi vieną su kitu, kitus penkis taip pat susegsi vieną su kitu. 
\par 4 Padarysi mėlynos spalvos kilpas prie abiejų uždangalų šonų. 
\par 5 Penkiasdešimt kilpų įtaisysi viename uždangale ir penkiasdešimt kitame taip, kad iš abiejų šonų kilpos būtų viena prieš kitą ir jas būtų galima sukabinti vieną su kita. 
\par 6 Padarysi taip pat penkiasdešimt auksinių kabių, kuriomis abiejų uždangalų šonus sukabinsi, kad būtų viena palapinė. 
\par 7 Padarysi iš ožkų vilnos vienuolika uždangalų palapinei apdengti iš viršaus. 
\par 8 Uždangalo ilgis bus trisdešimt uolekčių, plotis keturios uolektys. Visi uždangalai vienodo dydžio. 
\par 9 Penkis iš jų sujungsi atskirai ir kitus šešis taip pat. Šeštąjį uždangalą sudėsi dvilinką ant palapinės priekio. 
\par 10 Padarysi penkiasdešimt kilpų prie vieno uždangalo šono ir penkiasdešimt kilpų prie kito uždangalo šono, kad galėtum juos sukabinti. 
\par 11 Padarysi penkiasdešimt varinių kabių, kuriomis bus sukabinamos kilpos, kad pasidarytų vienas uždangalas. 
\par 12 Atliekančia nuo stogo uždengimo dalimi, tai yra vieno uždangalo likusia puse, pridengsi palapinės galą. 
\par 13 Tai, kas liks nuo palapinės uždengimo, po uolektį uždangalo iš abiejų pusių, tedengia palapinės šonus. 
\par 14 Palapinei pridengti padirbsi dar vieną uždangalą iš raudonai dažytų avinų kailių ir virš tos­iš mėlynai dažytų opšrų kailių. 
\par 15 Padirbdinsi palapinei lentų iš akacijos medžio, kurias reikės pastatyti stačias. 
\par 16 Kiekvienos iš jų ilgis bus dešimt uolekčių, o plotis­pusantros uolekties. 
\par 17 Lentos turi turėti šonuose po du išsikišimus, kuriais viena su kita bus sukabinamos; taip padarysi visas lentas. 
\par 18 Dvidešimt lentų padarysi palapinės pietiniam šonui. 
\par 19 Nuliesi keturiasdešimt sidabrinių pakojų dvidešimčiai lentų, kad po kiekvienos lentos kampu būtų pakištas pakojis. 
\par 20 Taip pat kitam palapinės šonui, kuris atgręžtas į šiaurę, padarysi dvidešimt lentų 
\par 21 ir keturiasdešimt sidabrinių pakojų padėti po du kiekvienos lentos apačioje. 
\par 22 O palapinės šonui, atgręžtam į vakarus, padirbdinsi šešias lentas 
\par 23 ir dvi lentas padarysi palapinės kampams iš abiejų pusių. 
\par 24 Jos turi būti sujungtos apačioje ir viršuje, kad sudarytų vieną sunėrimą. Taip padarysi abiejuose kampuose. 
\par 25 Iš viso bus aštuonios lentos, o jų sidabrinių pakojų­šešiolika, po du pakojus kiekvienai lentai. 
\par 26 Iš akacijos medžio padirbsi užkaiščius. Penkis vienos palapinės pusės lentoms 
\par 27 ir penkis kitos palapinės pusės lentoms, taip pat penkis palapinės galui vakarų pusėje. 
\par 28 Padarysi vidinį užkaištį, kad eitų per lentas nuo vieno galo iki kito. 
\par 29 Pačias lentas aptrauksi auksu; nuliesi auksines grandis užkaiščiams ir užkaiščius aptrauksi auksu. 
\par 30 Pastatysi palapinę pagal pavyzdį, kuris tau buvo parodytas kalne. 
\par 31 Padarysi uždangą iš mėlynų, raudonų ir violetinių siūlų ir plonos suktų siūlų drobės ir ant jos išsiuvinėsi cherubus. 
\par 32 Ją pakabinsi ant keturių akacijos medžio stulpų, aptrauktų auksu, jiems padirbdinsi auksinius kablius bei sidabrinius pakojus. 
\par 33 Uždangą prikabinsi kabėmis. Už uždangos pastatysi Liudijimo skrynią. Ta uždanga skirs šventąją dalį nuo Švenčiausiosios. 
\par 34 Uždėsi dangtį ant Liudijimo skrynios Švenčiausiojoje. 
\par 35 Stalą pastatysi šiapus uždangos, o žvakidę­priešais stalą pietiniame palapinės šone. 
\par 36 Padirbdinsi palapinės įėjimui užuolaidą iš mėlynų, violetinių ir raudonų siūlų ir plonos suktų siūlų drobės, visą išsiuvinėtą. 
\par 37 Aptrauksi auksu penkis akacijos medžio stulpus, ant kurių bus kabinama užuolaida; stulpų kabliai bus auksiniai, o pakojai­variniai”.



\chapter{27}


\par 1 “Padirbdinsi iš akacijos medžio keturkampį aukurą, penkių uolekčių ilgio, tiek pat pločio ir trijų uolekčių aukščio. 
\par 2 Keturiuose kampuose padarysi ragus ir visa tai aptrauksi variu. 
\par 3 Padarysi aukurui puodus pelenams supilti, semtuvėlius, dubenis, šakutes ir indus anglims; visus šiuos indus pagaminsi iš vario. 
\par 4 Padirbdinsi iš vario nupintas groteles, prie kurių keturiuose kampuose bus keturios varinės grandys. 
\par 5 Jas pritvirtinsi aplinkui aukurą, kad grotelės siektų nuo apačios iki pusės. 
\par 6 Padirbdinsi iš akacijos medžio aukurui kartis, kurias aptrauksi variu; 
\par 7 jas įkiši į grandis, kurios bus abiejuose šonuose aukurui nešti. 
\par 8 Aukurą padarysi tuščiavidurį, kaip tau buvo parodyta kalne. 
\par 9 Ir padarysi palapinei kiemą. Pietų pusėje bus užkabos iš plonos suktų siūlų drobės, kurių ilgis bus šimtas uolekčių. 
\par 10 Padarysi dvidešimt stulpų ir tiek pat varinių pakojų. Stulpai turės sidabrinius kablius ir skersinius. 
\par 11 Taip pat ir šiaurės pusėje bus užkabos šimto uolekčių ilgio, dvidešimt stulpų ir tiek pat varinių pakojų. Stulpai turės sidabrinius kablius ir skersinius. 
\par 12 Kiemo vakarinės pusės užkabos bus penkiasdešimties uolekčių ilgio, dešimt stulpų ir tiek pat pakojų. 
\par 13 Kiemo plotis rytų pusėje bus penkiasdešimt uolekčių. 
\par 14 Viename įėjimo šone bus užkaba penkiolikos uolekčių ilgio su trimis stulpais ir tiek pat pakojų. 
\par 15 Taip pat kitame šone bus užkaba penkiolikos uolekčių, trys stulpai ir tiek pat pakojų. 
\par 16 Kiemo įėjimą dengs dvidešimties uolekčių išsiuvinėta užkaba iš mėlynų, raudonų, violetinių ir plonų suktų siūlų. Jame bus keturi stulpai su tiek pat pakojų. 
\par 17 Visi kiemo stulpai bus aptraukti variu, turės sidabrinius kablius ir varinius pakojus. 
\par 18 Kiemo ilgis bus šimtas uolekčių, plotis penkiasdešimt uolekčių ir aukštis penkios uolektys. Jo užkabas padarysi iš plonos suktų siūlų drobės, o pakojus­iš vario. 
\par 19 Visus palapinės indus įvairiems reikalams ir visus kuolelius palapinei bei kiemui padarysi iš vario. 
\par 20 Įsakyk Izraelio vaikams, kad atneštų tyriausio alyvmedžių aliejaus, išspausto piestoje, kad lempa visada degtų 
\par 21 Susitikimo palapinėje šiapus uždangos, kuri pakabinta prieš Švenčiausiąją. Aaronas ir jo sūnūs ją prižiūrės, kad ji degtų per naktį ligi ryto Viešpaties akivaizdoje. Tai yra amžina Izraelio vaikų pareiga kartų kartoms”.



\chapter{28}


\par 1 “Tegul Aaronas ir jo sūnūs Nadabas, Abihuvas, Eleazaras ir Itamaras eina kunigų tarnystę. 
\par 2 Savo broliui Aaronui padarysi šventus drabužius, kad jis atrodytų iškilmingai ir gražiai. 
\par 3 Kalbėk visiems išmintingiesiems, kuriuos Aš pripildžiau išminties dvasios, kad jie padarytų Aaronui drabužius ir jis būtų įšventintas tarnauti kunigu mano akivaizdoje. 
\par 4 Jie turi padaryti šiuos drabužius: krūtinės skydelį, efodą, tuniką, siaurą drobinę jupą, mitrą ir juostą. Jie padarys šventus drabužius tavo broliui Aaronui ir jo sūnums, kurie eis kunigų tarnystę. 
\par 5 Rūbams imk auksą, mėlynų, violetinių, raudonų ir plonų lininių siūlų. 
\par 6 Efodą padarysi iš aukso, mėlynų, violetinių, raudonų ir plonų suktų lininių siūlų, meniškai juos išausdamas. 
\par 7 Ant jo bus dvi petnešos, sujungtos savo kraštais, ir taip jis bus sujungtas. 
\par 8 Juostą efodui padarysi taip pat kaip efodą: iš aukso, mėlynų, violetinių, raudonų ir suktų lininių siūlų. 
\par 9 Imsi du onikso akmenėlius ir juose įrėši Izraelio sūnų vardus: 
\par 10 šešis vardus viename akmenėlyje ir šešis antrame jų gimimo eile. 
\par 11 Kaip auksakaliai ir brangių akmenų raižytojai daro, taip įrėši juose Izraelio sūnų vardus ir įstatysi akmenėlius į auksinius įdėklus. 
\par 12 Pritvirtinsi juos ant abiejų efodo pečių kaip atminimo ženklą Izraelio sūnums. Ir Aaronas nešios jų vardus Viešpaties akivaizdoje ant abiejų pečių atminimui. 
\par 13 Padarysi iš aukso taip pat dvi sagtis 
\par 14 ir iš tyriausio aukso dvi grandinėles, sukabinėtas nareliais, kurias įversi į sagtis. 
\par 15 Padirbdinsi teismo krūtinės skydelį taip pat meniškai kaip efodą: iš aukso, mėlynų, violetinių, raudonų ir plonų suktų lininių siūlų padarysi jį. 
\par 16 Jis bus keturkampis ir dvilinkas; plotis ir ilgis vieno sprindžio. 
\par 17 Jį papuoši keturiomis brangakmenių eilėmis. Pirmoje eilėje bus sardis, topazas ir smaragdas; 
\par 18 antroje­rubinas, safyras ir jaspis; 
\par 19 trečioje­hiacintas, agatas ir ametistas; 
\par 20 ketvirtoje­chrizolitas, oniksas ir berilis. Jie bus aukse įrėminti. 
\par 21 Juose bus įrėžti dvylikos Izraelio sūnų vardai. Kiekviename brangakmenyje vardas vienos iš dvylikos giminių. 
\par 22 Krūtinės skydeliui padarysi iš gryno aukso grandinėles, 
\par 23 taip pat du auksinius žiedus, kuriuos pritaisysi prie dviejų krūtinės skydelio kampų. 
\par 24 Auksines grandinėles įversi į žiedus, esančius kampuose; 
\par 25 prie tų grandinėlių galų pritvirtinsi sagtis ir prisegsi jas prie efodo petnešų ties krūtinės skydeliu. 
\par 26 Padirbdinsi du auksinius žiedus, kuriuos pritaisysi prie apatinių krūtinės skydelio kampų iš apačios toje pusėje, kuri siekia efodą. 
\par 27 Du kitus auksinius žiedus pritaisysi prie abiejų efodo šonų žemai, kur apatinis sujungimas, kad krūtinės skydelis galėtų būti sukabintas su efodu. 
\par 28 Efodo žiedai bus surišti mėlyna juosta su krūtinės skydelio žiedais, kad krūtinės skydelis tvirtai prigultų ir negalėtų būti atskirtas nuo efodo. 
\par 29 Aaronas nešios Izraelio sūnų vardus teismo krūtinės skydelyje kaip amžiną atminimo ženklą Viešpaties akivaizdoje, eidamas į šventyklą. 
\par 30 Į teismo krūtinės skydelį įdėsi Urimą ir Tumimą, jie bus ant Aarono širdies, kai jis pasirodys Viešpaties akivaizdoje. Jis visuomet nešios ant savo širdies Viešpaties teismą Izraelio sūnums. 
\par 31 Padarysi efodui mėlyną tuniką, 
\par 32 kurios viršuje per vidurį bus skylė galvai įkišti, apvesta auksiniu apvadu, kad nesuplyštų. 
\par 33 Tunikos apačioje padarysi aplinkui iš mėlynų, violetinių ir raudonų siūlų granato vaisius, o tarp jų­auksinius varpelius, 
\par 34 taip, kad pakaitomis būtų auksinis varpelis ir granato vaisius aplink visą tuniką. 
\par 35 Jį dėvės Aaronas, eidamas tarnystę, kad girdėtųsi skambėjimas, jam einant į šventyklą Viešpaties akivaizdon ir išeinant, kad jis nemirtų. 
\par 36 Padarysi iš gryno aukso plokštelę, kurioje įrėši auksakalio darbu: ‘Pašvęstas Viešpačiui’. 
\par 37 Ją pririši mėlynu raiščiu ant mitros, kad būtų mitros priekyje. 
\par 38 Ji bus ant Aarono kaktos, kad Aaronas galėtų nešti visus trūkumus šventų dalykų, kuriuos Izraelio sūnūs aukoja kaip šventas dovanas. Ji nuolat bus ant jo kaktos, kad jie galėtų būti priimtini Viešpačiui. 
\par 39 Padarysi siaurą jupą iš baltos plonos drobės, mitrą iš tokios pat drobės ir juostą, margai išsiuvinėtą. 
\par 40 Aarono sūnums padarysi drobines jupas, juostas ir kunigiškus gobtuvus, kad atrodytų iškilmingai ir gražiai. 
\par 41 Šitais apdarais apvilksi savo brolį Aaroną ir jo sūnus. Patepsi juos, pašventinsi ir įšventinsi, kad jie galėtų būti mano kunigais. 
\par 42 Padarysi jiems trumpas drobines kelnes, kad pridengtų jų kūno nuogumą. Jos bus nuo strėnų iki šlaunų. 
\par 43 Jas dėvės Aaronas ir jo sūnūs, eidami į Susitikimo palapinę arba prie aukuro tarnauti šventykloje, kad nenusikalstų ir nemirtų. Tas nuostatas bus amžinas Aaronui ir jo palikuonims”.



\chapter{29}


\par 1 “Tai darysi, kad įšventintum juos būti mano kunigais. Imk iš bandos veršį ir du avinus be trūkumų, 
\par 2 neraugintos duonos ir nerauginto ragaišio, suvilgyto aliejumi, taip pat neraugintų papločių, apipiltų aliejumi. Visa tai pagaminsi iš kvietinių miltų. 
\par 3 Sudėjęs juos į pintinę, atneši pintinėje kartu su veršiu ir dviem avinais. 
\par 4 Aaroną ir jo sūnus pastatysi prie Susitikimo palapinės įėjimo ir apiplausi juos vandeniu. 
\par 5 Paimsi drabužius ir apvilksi Aaroną drobine jupa, tunika, efodu, uždėsi krūtinės skydelį ir sujuosi juosta. 
\par 6 Jam ant galvos uždėsi kunigišką mitrą, o ant jos šventąją plokštelę, 
\par 7 tada paimsi patepimo aliejų ir išpilsi ant jo galvos, jį patepdamas. 
\par 8 Pakvietęs Aarono sūnus, apvilksi juos drobinėmis jupomis, 
\par 9 apjuosi juostomis, uždėsi kunigiškus gobtuvus, ir jie tarnaus man kunigais per amžius. Ir tu pašventinsi Aaroną ir jo sūnus. 
\par 10 Atvesi veršį ties Susitikimo palapine, Aaronas ir jo sūnūs uždės jam ant galvos rankas. 
\par 11 Jį papjausi Viešpaties akivaizdoje prie Susitikimo palapinės įėjimo. 
\par 12 Ėmęs veršio kraujo, patepsi savo pirštu aukuro ragus, likusį kraują išliesi prie jo papėdės. 
\par 13 Imsi visus vidurių taukus, kepenų tinklelį, abu inkstus, jų taukus ir sudeginsi juos ant aukuro. 
\par 14 Veršio mėsą, odą ir mėšlus sudeginsi lauke už stovyklos, nes tai yra auka už nuodėmę. 
\par 15 Imsi taip pat aviną, ant kurio galvos Aaronas ir jo sūnūs uždės rankas. 
\par 16 Jį papjovęs, jo kraują šlakstysi aplink aukurą. 
\par 17 Patį aviną sukaposi į gabalus, apiplausi jo vidurius ir kojas ir uždėsi kartu su mėsos gabalais ir galva ant aukuro. 
\par 18 Sudeginsi ant aukuro visą aviną; tai yra deginamoji auka, malonus kvapas Viešpačiui. 
\par 19 Imsi ir kitą aviną, ant kurio galvos Aaronas ir jo sūnūs uždės rankas. 
\par 20 Jį papjovęs, jo krauju patepsi Aarono ir jo sūnų dešinę ausį, dešinės rankos nykštį ir didįjį dešinės kojos pirštą, likusį kraują šlakstysi aplink aukurą. 
\par 21 Krauju nuo aukuro ir patepimo aliejumi apšlakstysi Aaroną ir jo rūbus, sūnus ir jų rūbus; ir bus pašventintas jis, jo drabužiai, jo sūnūs ir sūnų drabužiai. 
\par 22 Tada imsi avino taukus, uodegą, taukus, dengiančius vidurius, kepenų tinklelį, abu inkstus su jų taukais ir dešinį petį, nes tai yra įšventinimo avinas, 
\par 23 vieną duonos kepalą, aliejumi apšlakstytą ragaišį, paplotį iš neraugintos duonos pintinės, padėtos Viešpaties akivaizdoje, 
\par 24 ir viską padėjęs ant Aarono ir jo sūnų rankų pašventinsi, siūbuodamas Viešpaties akivaizdoje. 
\par 25 Visa tai paimsi iš jų rankų ir sudeginsi ant aukuro kaip malonaus kvapo deginamąją auką Viešpačiui. 
\par 26 Iš avino, kuriuo bus įšventintas Aaronas, paimsi krūtinę ir ją pasiūbuosi Viešpaties akivaizdoje; tai bus tavo dalis. 
\par 27 Pašventinsi aukos krūtinę ir petį, kuriuos atskyrei iš avino Aaronui ir jo sūnums. 
\par 28 Tai bus Aarono ir jo sūnų amžinoji dalis iš izraelitų, nes tai yra padėkos aukos, kurias Izraelio sūnūs atneša Viešpačiui. 
\par 29 Šventą apdarą, kuriuo naudosis Aaronas, paveldės jo sūnūs, kad jame būtų įšventinti ir patepti. 
\par 30 Septynias dienas juo vilkės tas jo sūnus, kuris jo vieton bus paskirtas kunigu, kuris eis į Susitikimo palapinę tarnauti šventykloje. 
\par 31 Įšventinimo avino mėsą išvirsi šventoje vietoje. 
\par 32 Aaronas ir jo sūnūs valgys ją ir pintinėje esančią duoną Susitikimo palapinės kieme. 
\par 33 Ir jie valgys tai, kad būtų atliktas sutaikinimas, kad jie būtų pašvęsti ir įšventinti. Niekas kitas negali to valgyti, nes tai šventa. 
\par 34 O jei liktų pašvęstos mėsos ar duonos ligi ryto, liekanas sudeginsi; jų valgyti negalima, nes jos šventos. 
\par 35 Visa, ką tau įsakiau, padarysi Aaronui ir jo sūnums. Septynias dienas šventinsi juos 
\par 36 ir už nuodėmę kasdien aukosi veršį. Paaukojęs sutaikinimo auką, apvalysi aukurą, patepsi jį ir vėl pašventinsi. 
\par 37 Septynias dienas aukosi sutaikinimo auką ir šventinsi aukurą. Jis bus labai šventas aukuras. Kiekvienas, prie jo prisilietęs, taps šventas. 
\par 38 Štai ką aukosi ant aukuro: kasdien aukosi du metinius avinėlius­ 
\par 39 vieną avinėlį rytą, antrą vakare. 
\par 40 Kartu su vienu avinėliu paimsi dešimtą dalį efos miltų, sumaišysi juos su ketvirtadaliu hino aliejaus ir paimsi ketvirtadalį hino vyno geriamajai aukai. 
\par 41 Kitą gi avinėlį aukosi vakare tuo pačiu būdu, kaip ir rytą, kartu su geriamąja auka, kad būtų malonus aukos kvapas Viešpačiui. 
\par 42 Tai deginamoji auka Viešpačiui, aukojama amžinai per kartų kartas prie Susitikimo palapinės įėjimo Viešpaties akivaizdoje. Čia Aš susitiksiu su tavimi ir kalbėsiu tau. 
\par 43 Ir čia Aš susitiksiu su izraelitais ir pašventinsiu palapinę savo šlove. 
\par 44 Pašventinsiu Susitikimo palapinę, aukurą ir Aaroną su jo sūnumis, kad jie būtų man kunigais. 
\par 45 Aš gyvensiu tarp izraelitų ir būsiu jų Dievas. 
\par 46 Jie žinos, kad Aš­Viešpats, jų Dievas, kuris juos išvedžiau iš Egipto žemės, kad gyvenčiau tarp jų. Aš Viešpats, jų Dievas”.



\chapter{30}


\par 1 “Padirbdinsi iš akacijos medžio aukurą smilkalams deginti, 
\par 2 uolekties ilgio, uolekties pločio, keturkampį ir dviejų uolekčių aukščio. Ant jo bus ragai. 
\par 3 Aptrauksi jo viršų, šonus ir ragus grynu auksu. Padarysi jam auksinį apvadą aplinkui 
\par 4 ir auksines grandis po apvadu dviejuose kampuose kartims įkišti, kad jį būtų galima nešti. 
\par 5 Kartis padarysi iš akacijos medžio ir aptrauksi jas auksu. 
\par 6 Pastatysi aukurą prie uždangos, pakabintos prieš Liudijimo skrynią, ties dangčiu, kuris dengia liudijimą. 
\par 7 Kiekvieną rytą Aaronas degins ant jo maloniai kvepiančius smilkalus. Paruošęs lempas, jis degins smilkalus ant šio aukuro. 
\par 8 Ir vakare, uždegdamas lempas, jis degins smilkalus ant aukuro. Tai nuolatinis smilkymas Viešpaties akivaizdoje per visas jūsų kartas. 
\par 9 Neaukosite ant jo svetimų smilkalų nei deginamųjų aukų, nei valgio aukų ir neliesite geriamųjų aukų. 
\par 10 Vieną kartą per metus Aaronas ant aukuro ragų atliks sutaikinimą, aukos už nuodėmes krauju. Tai bus daroma per visas jūsų kartas, nes tai labai šventa Viešpačiui”. 
\par 11 Ir Viešpats kalbėjo Mozei: 
\par 12 “Kada skaičiuosi izraelitus, paskirk kiekvienam išpirką Viešpačiui, kad jų nepaliestų nelaimės. 
\par 13 Kiekvienas turės duoti po pusę šekelio, pagal šventyklos šekelį. Šekelis turi dvidešimt gerų. Pusė šekelio bus jų auka Viešpačiui. 
\par 14 Visi, sulaukę dvidešimties ar daugiau metų amžiaus, duos auką Viešpačiui. 
\par 15 Turtingas nemokės daugiau ir beturtis nemokės mažiau­pusę šekelio jie duos kaip auką Viešpačiui, kad jų sielos būtų sutaikintos. 
\par 16 Izraelitų sutaikinimo pinigus atiduosi Susitikimo palapinės reikalams”. 
\par 17 Ir Viešpats kalbėjo Mozei: 
\par 18 “Padirbdinsi varinę praustuvę su stovu ir ją pastatysi tarp Susitikimo palapinės ir aukuro. 
\par 19 Aaronas ir jo sūnūs mazgos joje savo rankas ir kojas. 
\par 20 Eidami į Susitikimo palapinę, jie nusimazgos vandeniu, kad nenumirtų, taip pat eidami prie aukuro aukoti deginamosios aukos Viešpačiui. 
\par 21 Jie mazgos savo rankas ir kojas, kad nemirtų. Tai yra amžinas nuostatas Aaronui ir jo palikuonims per visas kartas”. 
\par 22 Ir dar Viešpats kalbėjo Mozei: 
\par 23 “Imk geriausių kvepalų: penkis šimtus šekelių miros ir pusę tiek cinamono, tai yra du šimtus penkiasdešimt šekelių, ir kvepiančių nendrių du šimtus penkiasdešimt šekelių, 
\par 24 kasijos penkis šimtus šekelių pagal šventyklos šekelį ir vieną hiną alyvmedžių aliejaus. 
\par 25 Padarysi šventojo patepimo aliejų, sumaišydamas visa tai vaistininkų būdu; tai bus šventas patepimo aliejus. 
\par 26 Juo patepsi Susitikimo palapinę, Liudijimo skrynią, 
\par 27 stalą su jo priedais, žvakidę su jos priedais, smilkymo aukurą, 
\par 28 deginamųjų aukų aukurą ir praustuvę. 
\par 29 Tu pašventinsi visa, ir tai bus labai šventa; kas prie jų prisilies, taps šventas. 
\par 30 Patepsi Aaroną bei jo sūnus ir juos įšventinsi būti mano kunigais. 
\par 31 Izraelitams sakysi, kad toks patepimo aliejus bus šventas visoms jūsų kartoms. 
\par 32 Žmogaus kūnas juo nebus tepamas; jokio kito tos sudėties tepalo nedarysite, nes jis yra šventas ir bus jums šventas. 
\par 33 Kas tokį pat padarytų ar juo pateptų pašalietį, bus išnaikintas iš savo tautos”. 
\par 34 Ir Viešpats tarė Mozei: “Imk kvepalų: stakto, miros, balzamo, galbano ir skaidrių sakų visų vienodą svorį. 
\par 35 Padarysi iš jų smilkalus, sumaišydamas vaistininkų būdu, grynus ir šventus. 
\par 36 Sutrinsi tai smulkiai ir padėsi ties Susitikimo palapine, kurioje su tavimi susitiksiu. Tie smilkalai bus jums labai šventi. 
\par 37 Tokio mišinio nedarysite savo reikalams, nes jis bus šventas Viešpačiui. 
\par 38 Kiekvienas, kuris sau pasidarytų tokį pat mišinį ir juo naudotųsi, bus išnaikintas iš savo tautos”.



\chapter{31}


\par 1 Ir Viešpats kalbėjo Mozei: 
\par 2 “Aš pašaukiau vardu Hūro sūnaus Ūrio sūnų Becalelį iš Judo giminės 
\par 3 ir jį pripildžiau Dievo dvasios, išminties, sumanumo, pažinimo ir amato išmanymo, 
\par 4 kad sugebėtų viską padaryti iš aukso, sidabro, vario, 
\par 5 brangių akmenų ir medžio. 
\par 6 Daviau jam padėjėją Ahisamako sūnų Oholiabą iš Dano giminės; kiekvienam daviau išmintį padaryti viską, apie ką tau kalbėjau. 
\par 7 Jie padarys Susitikimo palapinę, Liudijimo skrynią, jos dangtį ir visus palapinės reikmenis: 
\par 8 stalą su jo priedais, gryno aukso žvakidę su jos priedais, aukurus smilkymui 
\par 9 ir deginamosioms aukoms bei visus jų reikmenis, praustuvę su jos stovu, 
\par 10 tarnavimo apdarus ir šventus drabužius kunigui Aaronui ir jo sūnums, kad būtų man kunigais, 
\par 11 patepimo aliejų, kvepiančius smilkalus šventyklai­viską, ką tau įsakiau, jie padarys”. 
\par 12 Viešpats toliau kalbėjo Mozei: 
\par 13 “Kalbėk izraelitams: ‘Privalote laikytis sabato, nes tai yra ženklas tarp manęs ir jūsų kartų kartoms, kad žinotumėte, jog Aš esu Viešpats, kuris jus pašventinu. 
\par 14 Laikykitės sabato, nes jis šventas; kas jį suterš, turi mirti. Kas dirbs tą dieną, bus išnaikintas iš savo tautos. 
\par 15 Šešias dienas dirbsite, o septintoji diena yra sabatas­poilsis, pašvęstas Viešpačiui. Kiekvienas, kuris dirbs sabato dieną, turi mirti. 
\par 16 Izraelitai privalo laikytis sabato per visas savo kartas kaip amžinos sandoros 
\par 17 tarp manęs ir Izraelio vaikų. Nes Viešpats per šešias dienas sukūrė dangų ir žemę, o septintąją dieną ilsėjosi ir atsigaivino’ ”. 
\par 18 Baigęs kalbėti, Viešpats davė Mozei ant Sinajaus kalno dvi akmenines liudijimo plokštes, parašytas Dievo pirštu.



\chapter{32}


\par 1 Tauta, nesulaukdama Mozės grįžtant, susirinko apie Aaroną ir sakė jam: “Padaryk mums dievą, kuris eitų pirma mūsų, nes mes nežinome, kas atsitiko tam žmogui Mozei, kuris mus išvedė iš Egipto žemės”. 
\par 2 Aaronas jiems atsakė: “Imkite iš savo žmonų, sūnų ir dukterų auksinius auskarus ir atneškite man”. 
\par 3 Visi žmonės išsiėmė auksinius auskarus iš savo ausų ir atnešė juos Aaronui. 
\par 4 Tas juos išlydė ir nuliedino veršį. Žmonės tarė: “Izraeli, štai tavo dievas, kuris tave išvedė iš Egipto žemės”. 
\par 5 Aaronas, matydamas tai, pastatė aukurą ir paskelbė: “Rytoj Viešpaties šventė!” 
\par 6 Atsikėlę anksti rytą, jie aukojo deginamąsias ir padėkos aukas, sėdo valgyti ir gerti, ir kėlėsi žaisti. 
\par 7 Ir Viešpats pasakė Mozei: “Eik, nusileisk žemyn, nes tavo tauta, kurią išvedei iš Egipto žemės, nusidėjo. 
\par 8 Jie greitai nuklydo nuo kelio, kurį jiems nurodžiau. Nusiliedinę veršį, jį garbina, jam aukoja aukas ir sako: ‘Šitas yra tavo dievas, Izraeli, kuris tave išvedė iš Egipto žemės’. 
\par 9 Matau, kad tai kietasprandė tauta. 
\par 10 Palik mane, kad mano rūstybė užsidegtų prieš juos ir juos sunaikinčiau, o iš tavęs padarysiu didelę tautą”. 
\par 11 Mozė maldavo Viešpatį, savo Dievą, sakydamas: “Kodėl, Viešpatie, Tavo rūstybė užsidega prieš Tavo tautą, kurią išvedei iš Egipto žemės didžia jėga ir galinga ranka? 
\par 12 Kodėl egiptiečiai turėtų sakyti: ‘Išvedė juos į pražūtį, kad nužudytų kalnuose ir išnaikintų nuo žemės paviršiaus’. Liaukis rūstavęs ir nesielk piktai su šia tauta. 
\par 13 Atsimink savo tarnus: Abraomą, Izaoką ir Izraelį, kuriems prisiekei: ‘Padauginsiu jūsų palikuonis kaip dangaus žvaigždes ir visą šitą žemę, apie kurią kalbėjau, duosiu jūsų palikuonims, kad jie paveldėtų ją amžiams’ ”. 
\par 14 Ir Viešpats nepasielgė piktai su savo tauta, kaip buvo sumanęs. 
\par 15 Mozė sugrįžo nuo kalno, nešdamas rankoje dvi liudijimo plokštes, abiejose pusėse prirašytas. 
\par 16 Plokštės ir jose įrėžtas raštas buvo Dievo darbas. 
\par 17 Jozuė, išgirdęs šūkaujančios tautos triukšmą, tarė Mozei: “Karo šauksmas girdisi stovykloje”. 
\par 18 O Mozė atsakė: “Tai ne nugalėtojų šauksmas ir ne pralaimėjusiųjų balsai, bet dainuojančių balsus aš girdžiu”. 
\par 19 Prisiartinęs prie stovyklos, jis išvydo veršį ir šokius. Mozė užsidegė pykčiu ir išmetė iš rankų abi plokštes, ir jas sudaužė kalno papėdėje. 
\par 20 Paėmęs veršį, kurį jie buvo pasidarę, sudegino jį ir sutrynė į dulkes; jas subėrė į vandenį ir davė gerti Izraelio vaikams. 
\par 21 Mozė klausė Aaroną: “Ką tau padarė šita tauta, kad užtraukei jiems tokią didelę nuodėmę?” 
\par 22 Aaronas atsakė: “Tenesirūstina mano valdovas. Tu žinai šitą tautą, kad jie greit nusikalsta. 
\par 23 Nes jie man sakė: ‘Padirbk mums dievą, kuris eitų pirma mūsų, nes mes nežinome, kas atsitiko tam žmogui Mozei, kuris mus išvedė iš Egipto žemės’. 
\par 24 Aš jiems sakiau: ‘Kas iš jūsų turi aukso, suneškite!’ Jie atidavė man auksą, aš įmečiau jį į ugnį ir išėjo šitas veršis”. 
\par 25 Mozė matė, kad tauta yra apnuoginta, nes Aaronas buvo ją apnuoginęs ir pastatęs priešų pajuokai. 
\par 26 Tada Mozė, stovėdamas stovyklos vartuose, sakė: “Kas esate Viešpaties, susirinkite prie manęs!” Prie jo susirinko visi Levio sūnūs. 
\par 27 Jis tarė: “Taip sako Viešpats: ‘Kiekvienas prisijuoskite kardą prie juosmens. Eikite per stovyklą išilgai nuo vartų ligi vartų ir nužudykite savo brolį, draugą ir artimą’ ”. 
\par 28 Levitai padarė, kaip Mozė įsakė. Tą dieną krito maždaug trys tūkstančiai žmonių. 
\par 29 Nes Mozė sakė: “Pasišvęskite šiandien Viešpačiui, nesigailėkite savo sūnaus nė brolio, kad jums tektų palaiminimas”. 
\par 30 Kitą dieną Mozė kalbėjo tautai: “Labai nusidėjote: eisiu pas Viešpatį, gal kaip nors sutaikinsiu jus dėl jūsų nusikaltimo”. 
\par 31 Sugrįžęs pas Viešpatį, tarė: “Šita tauta, pasidarydama auksinį dievą, labai nusidėjo. 
\par 32 Bet aš maldauju, atleisk jiems tą kaltę: jei ne, išbrauk mane iš knygos, kurion įrašei”. 
\par 33 Viešpats jam atsakė: “Kas man nusideda, tą išbrauksiu iš savo knygos. 
\par 34 Todėl eik ir vesk tautą, kur tau įsakiau. Mano angelas eis pirma tavęs. O kai laikas ateis, Aš juos nubausiu už jų nuodėmę”. 
\par 35 Viešpats baudė tautą dėl veršio, kurį Aaronas buvo padaręs.



\chapter{33}

\par 1 Viešpats kalbėjo Mozei: “Eik ir keliauk iš šitos vietos su tauta, kurią išvedei iš Egipto, į šalį, apie kurią prisiekiau Abraomui, Izaokui ir Jokūbui, sakydamas: ‘Aš ją duosiu tavo palikuonims’. 
\par 2 Aš siųsiu pirma tavęs angelą ir išstumsiu kanaaniečius, amoritus, hetitus, perizus, hivus ir jebusiečius. 
\par 3 Eik į žemę, plūstančią pienu ir medumi; tačiau Aš pats neisiu su jumis, kadangi esate kietasprandė tauta, kad kartais nesunaikinčiau jūsų kelyje”. 
\par 4 Tauta, išgirdusi tokią blogą žinią, nuliūdo, nė vienas nesipuošė papuošalais. 
\par 5 Nes Viešpats sakė Mozei: “Kalbėk Izraelio tautai: ‘Jūs esate kietasprandė tauta: jei įeičiau pas jus nors trumpam, sunaikinčiau jus. Todėl nusiimkite savo papuošalus, kad žinočiau, ką su jumis daryti’ ”. 
\par 6 Izraelitai nusiėmė nuo savęs papuošalus prie Horebo kalno. 
\par 7 Mozė ištiesė palapinę toli už stovyklos ir pavadino ją Susitikimo palapine. Visi žmonės, kurie ieškojo Viešpaties, eidavo iš stovyklos prie Susitikimo palapinės. 
\par 8 Kai Mozė eidavo į palapinę, visi žmonės pakildavo ir stovėdavo savo palapinių angose, sekdami jį akimis, kol jis įeidavo palapinėn. 
\par 9 Mozei įėjus į Susitikimo palapinę, debesies stulpas nusileisdavo ir stovėdavo prie palapinės įėjimo ir Viešpats kalbėdavosi su Moze. 
\par 10 Visa tauta matė debesies stulpą prie palapinės įėjimo, ir visi žmonės pakildavo ir pagarbindavo savo palapinių angose. 
\par 11 Viešpats kalbėdavo su Moze veidas į veidą, kaip žmogus kalbasi su savo draugu. Jam grįžtant į stovyklą, jo tarnas Jozuė, Nūno sūnus, jaunas vyras, nepasitraukdavo nuo palapinės. 
\par 12 Mozė tarė Viešpačiui: “Liepei išvesti šitą tautą ir nepasakei, ką siųsi su manimi, tačiau sakei: ‘Aš tave pažįstu ir žinau tavo vardą, tu radai malonę mano akyse’. 
\par 13 Jei tad radau malonę Tavo akyse, apreikšk man savo kelius, kad Tave pažinčiau ir galėčiau atrasti malonę Tavo akyse, nes tie žmonės yra Tavo tauta”. 
\par 14 Viešpats atsakė: “Mano artumas eis su tavimi, ir Aš įvesiu tave į poilsį”. 
\par 15 Mozė atsakė: “Jei Tavo artumas neis, nevesk mūsų niekur iš šitos vietos. 
\par 16 Nes kaip man sužinoti, kad aš ir Tavo tauta atradome malonę Tavo akyse? Ar ne iš to, kad Tu eisi su mumis? Taip aš ir Tavo tauta būsime išskirti iš visų žemės tautų”. 
\par 17 Viešpats atsakė Mozei: “Ir šį prašymą patenkinsiu, nes radai malonę mano akyse ir Aš žinau tavo vardą”. 
\par 18 Mozė prašė: “Parodyk man savo šlovę”. 
\par 19 Viešpats atsakė: “Aš leisiu visai savo šlovei praeiti pro tave ir paskelbsiu tau Viešpaties vardą, ir būsiu maloningas tam, kam būsiu maloningas, ir pasigailėsiu to, ko pasigailėsiu. 
\par 20 Mano veido negalėsi matyti, nes žmogus, mane pamatęs, negali išlikti gyvas. 
\par 21 Štai vieta šalia manęs! Atsistok ant šios uolos. 
\par 22 Kai mano šlovė eis pro šalį, tave pastatysiu uolos plyšyje ir pridengsiu savo ranka, kol praeisiu. 
\par 23 Po to atitrauksiu ranką ir matysi mane iš užpakalio, mano gi veido nematysi”.



\chapter{34}

\par 1 Viešpats tarė Mozei: “Išsikirsk dvi akmenines plokštes. Aš jose įrašysiu žodžius, kurie buvo sudaužytose plokštėse. 
\par 2 Rytą būk pasiruošęs, užlipk į Sinajaus kalną ir atsistok mano akivaizdoje. 
\par 3 Nė vienas su tavimi tegul neateina ir nė vienas tenepasirodo visame kalne. Taip pat galvijai ir avys tenesigano prie kalno”. 
\par 4 Jis iškirto iš akmens dvi plokštes, kokios buvo pirmosios; atsikėlęs anksti rytą, užlipo į Sinajaus kalną, kaip Viešpats buvo įsakęs, nešdamas rankose plokštes. 
\par 5 Viešpats nužengė debesyje ir atsistojo šalia jo, ir paskelbė Viešpaties vardą. 
\par 6 Viešpats praėjo pro jį ir paskelbė: “Viešpats, Viešpats Dievas, gailestingas ir maloningas, kantrus ir kupinas gerumo bei tiesos, 
\par 7 parodantis gailestingumą tūkstančiams, atleidžiantis nusikaltimus, neteisybes ir nuodėmes, tačiau nepaliekantis kalto nenubausto, bet baudžiantis už tėvų nusikaltimus vaikus ir vaikaičius iki trečios ir ketvirtos kartos”. 
\par 8 Mozė skubiai nusilenkė iki žemės ir pagarbino Viešpatį. 
\par 9 Ir jis sakė: “Jei radau malonę Tavo akyse, Viešpatie, maldauju Tave, eik kartu su mumis, nes tauta yra kietasprandė; atleisk mūsų neteisybes ir nuodėmes, padaryk mus savo nuosavybe!” 
\par 10 Viešpats atsakė: “Štai darau sandorą ir visos tavo tautos akyse darysiu stebuklus, kokių niekas nedarė žemėje ir jokiose tautose. Šita tauta matys Viešpaties darbą, nes Aš su tavimi darysiu baisių dalykų. 
\par 11 Įsidėmėk, ką šiandien tau įsakau. Aš išvarysiu tavo akivaizdoje amoritus, kanaaniečius, hetitus, perizus, hivus ir jebusiečius. 
\par 12 Saugokis ir nedaryk sandoros su tų kraštų gyventojais, kad jie netaptų spąstais tarp jūsų: 
\par 13 sugriauk jų aukurus, sutrupink atvaizdus ir iškirsk giraites. 
\par 14 Negarbink svetimų dievų, nes Viešpats yra pavydus Dievas. 
\par 15 Nedaryk sandoros su anų kraštų žmonėmis, kad kas nors iš jų garbinęs stabus nepasikviestų tavęs jų aukų valgyti. 
\par 16 Neimk savo sūnums žmonų iš jų dukterų, kad jos, garbindamos stabus, neįtrauktų ir tavo sūnų garbinti jų dievų. 
\par 17 Nepasigamink nulietų dievų. 
\par 18 Švęsk Neraugintos duonos šventę. Septynias dienas valgyk neraugintą duoną Abibo mėnesį, kaip įsakiau, nes Abibo mėnesį išėjai iš Egipto. 
\par 19 Visų gyvulių pirmagimiai yra mano­patinėliai jautukai ir ėriukai. 
\par 20 Asilo pirmagimį išpirksi avinu, o jei jo neišpirksi, nusuk jam sprandą. Pirmagimius savo sūnus išpirk ir nepasirodyk mano akivaizdoje tuščiomis rankomis. 
\par 21 Šešias dienas dirbk, septintą dieną ilsėkis, net sėjos ir pjūties metu. 
\par 22 Švęsk Savaičių šventę, tai yra pirmųjų kviečių derliaus šventę, taip pat Derliaus nuėmimo šventę, metams baigiantis. 
\par 23 Tris kartus per metus visi tavo vyrai privalo pasirodyti Viešpaties, Izraelio Dievo, akivaizdoje. 
\par 24 Aš išvarysiu tautas prieš tave ir išplėsiu tavo krašto ribas; niekas nepuls tavo žemių, tau išėjus pasirodyti tris kartus per metus Viešpaties, tavo Dievo, akivaizdoje. 
\par 25 Neaukok mano aukos kraujo kartu su raugu; nieko nepalik iki ryto iš Paschos aukos. 
\par 26 Savo lauko pirmuosius vaisius atgabenk į Viešpaties Dievo namus. Nevirk ožiuko jo motinos piene”. 
\par 27 Viešpats tarė Mozei: “Užrašyk žodžius, kuriais padariau sandorą su tavimi ir Izraeliu”. 
\par 28 Jis buvo ant kalno su Viešpačiu keturiasdešimt dienų ir keturiasdešimt naktų, nevalgė duonos ir negėrė vandens. Ir Jis įrašė plokštėse sandoros žodžius, dešimt įsakymų. 
\par 29 Nužengdamas nuo Sinajaus kalno, Mozė nešėsi dvi liudijimo plokštes ir nežinojo, kad jo veidas po pašnekesio su Viešpačiu spindėjo. 
\par 30 Aaronas ir izraelitai, matydami spindintį Mozės veidą, bijojo prie jo priartėti. 
\par 31 Tik jam pašaukus, Aaronas ir visi vyresnieji susirinko pas jį. Mozė kalbėjo su jais. 
\par 32 Po to susirinko ir visi izraelitai. Jis jiems perdavė viską, ką Viešpats jam kalbėjo Sinajaus kalne. 
\par 33 Baigęs kalbėti, jis užsidėjo ant veido gaubtuvą. 
\par 34 Įėjęs pas Viešpatį ir su Juo kalbėdamas, Mozė gaubtuvą nusiimdavo. Išėjęs perduodavo izraelitams viską, kas jam buvo įsakyta. 
\par 35 Izraelio vaikai matė Mozės veidą, kad jis spindėjo, ir Mozė užsidėdavo gaubtuvą, iki eidavo kalbėti su Juo.



\chapter{35}


\par 1 Mozė, sušaukęs visus izraelitus, paskelbė, ką Viešpats įsakė: 
\par 2 “Šešias dienas dirbsite, o septintoji diena yra šventa, sabato poilsis Viešpačiui. Kas tą dieną dirbs, bus baudžiamas mirtimi. 
\par 3 Nekurkite ugnies sabato dieną savo būstuose”. 
\par 4 Ir Mozė kalbėjo visam Izraelio vaikų susirinkimui: “Štai ką įsakė Viešpats: 
\par 5 ‘Kiekvienas atneškite Viešpačiui auką, laisva valia atneškite ją Viešpačiui: auksą, sidabrą, varį, 
\par 6 mėlynus, raudonus ir violetinius siūlus, ploną drobę, ožkų vilnas, 
\par 7 raudonai dažytus avinų kailius, opšrų kailius, akacijos medį, 
\par 8 aliejų lempoms, kvepalus patepimo aliejui ir kvepiantiems smilkalams, 
\par 9 onikso akmenėlius ir brangius akmenis efodui bei krūtinės skydeliui. 
\par 10 Kas tarp jūsų sumanus, teateina ir tedirba, ką Viešpats įsakė: 
\par 11 palapinę ir jos uždangalus, grandis, lentas ir kartis, stulpus ir pakojus; 
\par 12 skrynią, jos kartis bei dangtį ir uždangą; 
\par 13 stalą su kartimis bei reikmenimis ir padėtine duona; 
\par 14 žvakidę lempoms, jos reikmenis ir aliejų deginimui; 
\par 15 smilkymo aukurą ir kartis, patepimo aliejų ir kvapius smilkalus; užuolaidą palapinės įėjimui, 
\par 16 deginamųjų aukų aukurą, jo varines groteles su kartimis ir reikmenimis, praustuvę ir jos stovą; 
\par 17 kiemo užkabas su stulpais ir pakojais, užkabą kiemo įėjimui; 
\par 18 palapinės ir kiemo kuolelius su virvelėmis; 
\par 19 apdarus, naudojamus tarnaujant šventykloje, vyriausiojo kunigo Aarono ir jo sūnų drabužius kunigų tarnystei atlikti’ ”. 
\par 20 Tada visi izraelitai išsiskirstė. 
\par 21 Ir atėjo kiekvienas, kurio širdis buvo sujaudinta ir kiekvienas, kurio dvasioje buvo noras, ir atnešė Viešpačiui auką Susitikimo palapinei statyti, ko reikia jos tarnavimui ir šventiems rūbams. 
\par 22 Vyrai ir moterys, kurių širdyse buvo noras, davė sagtis, auskarus, žiedus, apyrankes­įvairius auksinius daiktus. Visi aukojusieji atnešė auksą kaip auką Viešpačiui. 
\par 23 Kas tik turėjo mėlynų, raudonų ir violetinių siūlų, plonos drobės, ožkų vilnų, raudonai dažytų avinų kailių, opšrų kailių, 
\par 24 sidabro ir vario ir akacijos medžio, tinkamo įvairiems daiktams, atnešė ir paaukojo Viešpačiui. 
\par 25 Sumanios moterys davė, ką buvo suverpusios: mėlynų, raudonų, violetinių ir plonų lininių siūlų. 
\par 26 Moterys, kurių širdys buvo sujaudintos, sumaniai suverpė ožkų vilnas. 
\par 27 Kunigaikščiai atnešė onikso akmenėlių ir brangiųjų akmenų efodui ir krūtinės skydeliui, 
\par 28 kvepalų, aliejaus lempoms, patepimo aliejui ir kvapniems smilkalams. 
\par 29 Vyrai ir moterys, kurių širdyse buvo noras, aukojo, kad būtų atlikti darbai, kuriuos Viešpats jiems įsakė per Mozę. Visi izraelitai aukojo Viešpačiui laisva valia. 
\par 30 Tada Mozė vėl kalbėjo: “Viešpats pašaukė vardu Hūro sūnaus Ūrio sūnų Becalelį iš Judo giminės, 
\par 31 pripildė jį Dievo dvasios, išminties, sumanumo, pažinimo ir amato išmanymo 
\par 32 padaryti įvairių meniškų dalykų iš aukso, sidabro ir vario, 
\par 33 taip pat iškalti akmenį ir drožti medį­visus menininko darbus padaryti. 
\par 34 Ir Jis įdėjo sugebėjimą mokyti į jo ir Ahisamako sūnaus Oholiabo iš Dano giminės širdis. 
\par 35 Juos apdovanojo išmintimi atlikti raižytojo, įgudusio audėjo ir siuvinėtojo iš mėlynų, raudonų, violetinių ir plonų lininių siūlų darbus, kad jie meniškai padarytų kiekvieną darbą”.



\chapter{36}


\par 1 Tai ėmėsi darbo Becalelis, Oholiabas ir kiti sumanūs vyrai, kuriems Viešpats davė išminties ir supratimo, kaip padaryti įvairius reikmenis šventyklai pagal visus Viešpaties nurodymus. 
\par 2 Mozė pasišaukė Becalelį, Oholiabą ir visus sumanius vyrus, kurie savo noru sutiko dirbti prie šventyklos darbų. 
\par 3 Mozė atidavė jiems visas aukas, kurias Izraelio vaikai atnešė šventyklos darbams. Tauta kas rytą vis dar nešė laisvos valios aukas. 
\par 4 Visi sumanūs vyrai, kurie darė šventyklos darbus, atėjo nuo savo darbų ir kalbėjo Mozei: 
\par 5 “Tauta aukoja daugiau negu reikia, kad įvykdytume šį Viešpaties įsakytą darbą”. 
\par 6 Tada Mozė liepė paskelbti, kad nei vyras, nei moteris nieko daugiau nebeaukotų šventyklos statybai. Tada žmonės nustojo aukoti, 
\par 7 nes visko, kas buvo sunešta, užteko visiems darbams padaryti ir dar liko. 
\par 8 Sumanūs vyrai, kurie darė palapinę, padarė iš plonos suktų siūlų drobės ir mėlynų, raudonų ir violetinių siūlų dešimt uždangalų su išsiuvinėtais cherubų vaizdais. 
\par 9 Kiekvienas jų buvo dvidešimt aštuonių uolekčių ilgio ir keturių uolekčių pločio­visi uždangalai buvo vienodo dydžio. 
\par 10 Jie sujungė penkis uždangalus vieną su kitu ir likusius penkis sujungė vieną su kitu. 
\par 11 Be to, padarė mėlynas kilpas vienam ir antram uždangalui, 
\par 12 po penkiasdešimt kilpų, kad kilpos būtų viena prieš kitą ir uždangalai galėtų būti sukabinti. 
\par 13 Nuliejo taip pat penkiasdešimt auksinių kabių, kuriomis sukabino abu uždangalus taip, kad pasidarė viena palapinė. 
\par 14 Padarė ir vienuolika uždangalų iš ožkų plaukų palapinės viršui apdengti. 
\par 15 Kiekvienas uždangalas buvo trisdešimties uolekčių ilgio ir keturių uolekčių pločio­visi vienuolika uždangalų buvo vienodo dydžio. 
\par 16 Penkis jų sujungė į vieną, kitus šešis taip pat sujungė. 
\par 17 Padarė penkiasdešimt kilpų vieno sujungto uždangalo šone ir penkiasdešimt kito šone, kad abu būtų galima sukabinti. 
\par 18 Taip pat padarė ir penkiasdešimt varinių kabių, kuriomis būtų sukabinti abu sujungti uždangalai, kad susidarytų vienas uždangalas. 
\par 19 Dar padarė palapinės uždangalą iš raudonai dažytų avinų kailių ir uždangalą iš opšrų kailių. 
\par 20 Iš akacijos medžio padarė lentas palapinei, kad jas būtų galima pastatyti. 
\par 21 Kiekviena lenta buvo dešimties uolekčių ilgio ir pusantros uolekties pločio. 
\par 22 Jos turėjo po du išsikišimus, kad vieną lentą su kita būtų galima sukabinti. Taip buvo padarytos visos palapinės lentos. 
\par 23 Ir padarė lentas palapinei: dvidešimt lentų šonui, atgręžtam į pietus, 
\par 24 su keturiasdešimt sidabrinių pakojų­kiekviena lenta turėjo po du pakojus savo apačioje. 
\par 25 Taip pat į šiaurę atgręžtam šonui buvo padaryta dvidešimt lentų 
\par 26 su keturiasdešimt sidabrinių pakojų, po du pakojus kiekvienai lentai. 
\par 27 Palapinės šonui, atgręžtam į vakarus, padarė šešias lentas 
\par 28 ir dvi lentas palapinės kampams iš abiejų pusių. 
\par 29 Jos buvo sujungtos apačioje ir viršuje ir sudarė vieną sunėrimą. Taip jis padarė abiejuose kampuose. 
\par 30 Ir buvo iš viso aštuonios lentos ir šešiolika sidabrinių pakojų, po du pakojus kiekvienai lentai. 
\par 31 Ir padarė užkaiščius iš akacijos medžio. Penkis užkaiščius vienos palapinės pusės lentoms, 
\par 32 penkis užkaiščius kitos palapinės pusės lentoms ir penkis užkaiščius palapinės galui vakarų pusėje. 
\par 33 Ir padarė vidinį užkaištį, kuris eitų per lentas nuo vieno galo iki kito. 
\par 34 Aptraukė lentas auksu, nuliejo auksines grandis užkaiščiams ir užkaiščius aptraukė auksu. 
\par 35 Padarė uždangą iš mėlynų, raudonų, violetinių siūlų ir plonos suktų siūlų drobės ir ant jos išsiuvinėjo cherubus. 
\par 36 Jai pakabinti padarė keturis stulpus iš akacijos medžio, aptrauktus auksu, su auksiniais kabliais ir sidabriniais pakojais. 
\par 37 Palapinės įėjimui padarė užuolaidą iš mėlynų, raudonų, violetinių siūlų ir plonos suktų siūlų drobės, visą išsiuvinėtą, 
\par 38 ir penkis stulpus su jų kabliais. Jų pagrindus ir skersinius aptraukė auksu, bet jų penki pakojai buvo variniai.



\chapter{37}


\par 1 Becalelis padarė iš akacijos medžio skrynią pustrečios uolekties ilgio, pusantros pločio ir pusantros aukščio. 
\par 2 Ją aptraukė grynu auksu iš vidaus ir iš išorės, auksinis apvadas buvo aplinkui. 
\par 3 Nuliejo keturias auksines grandis ir pritvirtino keturiuose kampuose: po dvi grandis iš kiekvienos pusės. 
\par 4 Iš akacijos medžio padarė kartis, kurias aptraukė auksu, 
\par 5 ir įkišo į grandis, esančias skrynios šonuose, jai nešioti. 
\par 6 Padarė dangtį iš gryno aukso pustrečios uolekties ilgio ir pusantros uolekties pločio. 
\par 7 Iš aukso padarė du cherubus, nukaltus iš vieno gabalo, abiejuose dangčio galuose. 
\par 8 Vieną cherubą viename gale, antrą­kitame. Ant dangčio padarė cherubus abiejuose galuose 
\par 9 išskėstais sparnais, gaubiančiais dangtį. Jų veidai buvo atgręžti vienas į kitą ir žiūrėjo į dangtį. 
\par 10 Padarė dar ir stalą iš akacijos medžio: dviejų uolekčių ilgio, uolekties pločio ir pusantros uolekties aukščio. 
\par 11 Jį aptraukė grynu auksu ir padarė auksinį apvadą aplinkui 
\par 12 ir auksinę briauną plaštakos platumo, o ant jos­kitą auksinį apvadą. 
\par 13 Nuliejo keturias auksines grandis, kurias pritaisė keturiuose kampuose prie kiekvienos stalo kojos, 
\par 14 ir į jas įkišo kartis stalui nešioti. 
\par 15 Kartis padarė iš akacijos medžio ir aptraukė auksu. 
\par 16 Padarė ir įvairius stalo reikmenis iš gryno aukso: dubenis, taures, smilkytuvus ir puodelius skysčiams aukoti. 
\par 17 Žvakidė buvo nukalta iš gryno aukso. Iš jos liemens ėjo šakos, taurelės, buoželės ir žiedai; 
\par 18 šešios šakos iš abiejų šonų: trys iš vienos pusės ir trys iš kitos; 
\par 19 po tris riešuto pavidalo taureles, po buoželę ir žiedą buvo ant kiekvienos šakos, kurios ėjo iš žibinto liemens. 
\par 20 Ant pačios žvakidės buvo keturios riešuto pavidalo taurelės su buoželėm ir žiedais. 
\par 21 Viena buoželė po dviem iš jos išeinančiom šakom, kita po dviem iš jos išeinančiom šakom ir trečia po dviem iš žvakidės išeinančiom šakom, visoms šešioms šakoms. 
\par 22 Buoželės ir šakos buvo nukaltos iš vieno gabalo gryno aukso. 
\par 23 Iš gryno aukso buvo padarytos ir septynios lempos, gnybtuvai ir indai nuognaiboms. 
\par 24 Žvakidė ir visi jos priedai buvo padaryti iš talento gryno aukso. 
\par 25 Iš akacijos medžio padarė smilkymo aukurą. Jis buvo keturkampis, uolekties ilgio, uolekties pločio ir dviejų uolekčių aukščio. Keturiuose jo kampuose buvo po ragą. 
\par 26 Jį aptraukė grynu auksu: viršų, šonus ir ragus, ir padarė jam auksinį apvadą aplinkui 
\par 27 ir dvi auksines grandis po apvadu dviejuose kampuose kartims įkišti, kad jį būtų galima nešti. 
\par 28 Kartis taip pat padarė iš akacijos medžio ir aptraukė auksu. 
\par 29 Patepimo aliejus ir smilkalai buvo pagaminti vaistininkų būdu iš tyriausių kvepalų.



\chapter{38}


\par 1 Padarė deginamųjų aukų aukurą iš akacijos medžio: penkių uolekčių ilgio, tiek pat pločio, keturkampį ir trijų uolekčių aukščio, 
\par 2 su ragais kampuose, ir aptraukė jį variu. 
\par 3 Jo reikalams padarė iš vario: puodus, semtuvėlius, dubenis, šakutes ir indus anglims. 
\par 4 Aukurui padarė iš vario išpintas groteles aplinkui jį nuo apačios iki pusės. 
\par 5 Nuliejo keturias varines grandis grotelių kampuose kartims įkišti. 
\par 6 Padarė kartis iš akacijos medžio ir aptraukė jas variu. 
\par 7 Kartis įkišo į grandis aukuro šonuose, kad būtų galima nešti. Aukuras buvo sukaltas iš lentų, vidurys buvo tuščias. 
\par 8 Padarė ir praustuvę su stovu iš vario, kurį paėmė iš budėjusių ties palapinės įėjimu moterų veidrodžių. 
\par 9 Padarė ir kiemą. Jo pietų pusėje buvo šimto uolekčių ilgio užkabos, padarytos iš plonos suktų siūlų drobės, 
\par 10 ir dvidešimt stulpų su jų variniais pakojais; taip pat kablius ir skersinius iš sidabro. 
\par 11 Šiaurės pusėje buvo šimto uolekčių ilgio užkabos, dvidešimt stulpų, dvidešimt varinių pakojų ir sidabriniai kabliai bei skersiniai. 
\par 12 Vakarų pusėje buvo penkiasdešimties uolekčių ilgio užkabos, dešimt stulpų su variniais pakojais; jų kabliai ir skersiniai buvo padaryti iš sidabro. 
\par 13 Rytinė pusė buvo taip pat penkiasdešimties uolekčių pločio; 
\par 14 viename krašte buvo penkiolikos uolekčių ilgio užkaba su trimis stulpais ir jų pakojais 
\par 15 ir kitame krašte taip pat penkiolikos uolekčių užkaba, trys stulpai ir tiek pat pakojų. 
\par 16 Visos užkabos aplinkui kiemą buvo iš plonos suktų siūlų drobės. 
\par 17 Stulpų pakojai buvo variniai, o jų kabliai ir skersiniai­iš sidabro; stulpus aptraukė sidabru ir sujungė juos sidabriniais skersiniais. 
\par 18 Įėjimui į kiemą padarė iš mėlynų, raudonų, violetinių ir plonų suktų siūlų išsiuvinėtą užkabą, kuri buvo dvidešimties uolekčių ilgio, penkių uolekčių pločio pagal visų kiemo užkabų aukštį. 
\par 19 Užkabai padarė keturis stulpus su variniais pakojais ir sidabriniais kabliais bei skersiniais. 
\par 20 Palapinės ir kiemo kuoleliai buvo variniai. 
\par 21 Tai sąrašas to, kas buvo sunaudota Susitikimo palapinei, kaip suskaičiavo Mozei įsakius kunigo Aarono sūnus Itamaras, padedamas levitų. 
\par 22 Hūro sūnaus Ūrio sūnus Becalelis iš Judo giminės padarė viską, ką Viešpats įsakė Mozei. 
\par 23 Su juo buvo Ahisamako sūnus Oholiabas iš Dano giminės. Jis buvo sumanus raižytojas, audėjas bei siuvinėtojas iš mėlynų, raudonų, violetinių ir plonų lininių siūlų. 
\par 24 Šventyklai iš viso buvo sunaudota dvidešimt devyni talentai ir septyni šimtai trisdešimt šekelių aukso pagal šventyklos šekelį. 
\par 25 Sidabro, kurį davė visi tautoje suskaičiuoti vyrai, buvo šimtas talentų ir tūkstantis septyni šimtai septyniasdešimt penki šekeliai pagal šventyklos šekelį. 
\par 26 Nuo kiekvieno, kuris buvo dvidešimties metų amžiaus arba vyresnis, buvo surinkta po pusę šekelio pagal šventyklos šekelį. Iš viso jų buvo šeši šimtai trys tūkstančiai penki šimtai penkiasdešimt. 
\par 27 Iš šimto talentų sidabro nuliejo šimtą pakojų šventyklai ir uždangai, kiekvienam pakojui sunaudojo po vieną talentą. 
\par 28 Iš tūkstančio septynių šimtų septyniasdešimt penkių šekelių padarė kablius stulpams, aptraukė stulpus sidabru ir sujungė juos skersiniais. 
\par 29 Dar buvo paaukota septyniasdešimt talentų ir du tūkstančiai keturi šimtai šekelių vario. 
\par 30 Iš jo padarė pakojus Susitikimo palapinės įėjimui, varinį aukurą su grotelėmis ir visus aukuro reikmenis, 
\par 31 kiemo ir įėjimo pakojus ir palapinės bei kiemo kuolelius.



\chapter{39}

\par 1 Iš mėlynų, raudonų ir violetinių siūlų padarė apdarus, kuriais turėjo vilkėti Aaronas, tarnaudamas šventykloje, kaip Viešpats įsakė Mozei. 
\par 2 Efodą padarė iš aukso, mėlynų, raudonų, violetinių ir plonų suktų lininių siūlų. 
\par 3 Supjaustę išplotą auksinę skardą, padarė iš jos siūlus, kad jie galėtų būti įausti į mėlynų, raudonų, violetinių ir plonų lininių siūlų audinį. 
\par 4 Padarė jam petnešas, kad jį sujungtų; iš abiejų pusių sujungė jį. 
\par 5 Padarė ir juostą efodui tų pačių spalvų ir taip pat padarytą, kaip Viešpats įsakė Mozei. 
\par 6 Du onikso akmenėlius su įrėžtais ant jų Izraelio sūnų vardais įstatė į auksinius įdėklus. 
\par 7 Juos pritvirtino ant abiejų efodo pečių Izraelio sūnų atminimui, kaip Viešpats įsakė Mozei. 
\par 8 Krūtinės skydelį padarė taip pat meniškai, kaip ir efodą, iš auksinių, mėlynų, violetinių, raudonų ir plonų suktų lininių siūlų. 
\par 9 Jis buvo keturkampis, dvilinkas, sprindžio ilgumo ir sprindžio platumo. 
\par 10 Ant jo pritvirtino keturias brangakmenių eiles. Pirmoje eilėje buvo sardis, topazas ir smaragdas; 
\par 11 antroje­rubinas, safyras, jaspis; 
\par 12 trečioje­hiacintas, agatas ir ametistas; 
\par 13 ketvirtoje­chrizolitas, oniksas ir berilis. Eilėmis sustatyti brangakmeniai buvo įtvirtinti aukse. 
\par 14 Pačiuose akmenyse buvo įrėžti dvylikos Izraelio giminių vardai, kiekviename po vieną vardą. 
\par 15 Krūtinės skydelio kampams padarė iš gryno aukso grandinėles, 
\par 16 dvi sagtis ir du auksinius žiedus. Žiedus pritvirtino dviejuose krūtinės skydelio kampuose 
\par 17 ir įvėrė dvi auksines grandinėles į žiedus krūtinės skydelio kampuose. 
\par 18 Ir prie dviejų auksinių grandinėlių galų pritvirtino sagtis ir prisegė jas prie efodo petnešų ties krūtinės skydeliu. 
\par 19 Ir padarė du auksinius žiedus, ir pritvirtino prie dviejų krūtinės skydelio kampų toje pusėje, kuri buvo prie efodo. 
\par 20 Ir padarė du kitus auksinius žiedus, ir pritvirtino prie efodo šonų žemai, kur apatinis sujungimas virš juostos. 
\par 21 Ir pririšo krūtinės skydelį už jo žiedų prie efodo žiedų mėlynu raiščiu, kad jis būtų virš efodo juostos ir neatsiskirtų nuo efodo, kaip Viešpats įsakė Mozei. 
\par 22 Efodui padarė mėlyną tuniką. 
\par 23 Jos viršuje per vidurį buvo skylė galvai, apsiūta aplinkui, kad nesuplyštų; 
\par 24 apačioje prie kojų buvo granato vaisiai iš mėlynų, raudonų, violetinių ir suktų lininių siūlų 
\par 25 ir varpeliai iš gryno aukso, kurie buvo prikabinti tarp granato vaisių aplink visą tuniką. 
\par 26 Auksiniai varpeliai ir granato vaisiai ėjo pakaitomis aplink visą tunikos kraštą, kaip Viešpats įsakė Mozei. 
\par 27 Aaronui ir jo sūnums taip pat padarė drobines jupas, 
\par 28 mitrą iš drobės, drobinius gobtuvus ir kelnes iš drobės. 
\par 29 O juostą išaudė raštais iš mėlynų, raudonų, violetinių ir plonų suktų lininių siūlų, kaip Viešpats įsakė Mozei. 
\par 30 Iš gryno aukso padarė plokštelę ir joje įrėžė, kaip yra išraižomi antspaudai: “Pašvęstas Viešpačiui”. 
\par 31 Ją pririšo ant mitros mėlynu raiščiu, kaip Viešpats įsakė Mozei. 
\par 32 Susitikimo palapinės darbai buvo baigti; izraelitai padarė visa, ką Viešpats buvo įsakęs Mozei. 
\par 33 Jie atnešė Mozei palapinę, uždangalus, visus daiktus bei reikmenis: kabes, lentas, kartis, stulpus ir pakojus; 
\par 34 uždangalą iš raudonai dažytų avinų kailių, uždangalą iš opšrų kailių, uždangą, 
\par 35 Liudijimo skrynią, jos kartis, dangtį, 
\par 36 stalą su jo reikmenimis ir padėtinę duoną, 
\par 37 žvakidę, lempas, sustatytas į savo vietas, jos priedus ir aliejų deginti, 
\par 38 auksinį aukurą, patepimo aliejų ir kvepiančius smilkalus, užuolaidą palapinės įėjimui, 
\par 39 varinį aukurą su grotelėmis, kartimis ir visais jo reikmenimis, praustuvę ir jos stovą, 
\par 40 kiemo užkabas ir stulpus su jų pakojais, užkabas kiemo įėjimui, jo virveles ir kuolelius,­visa, ko reikia tarnavimui Susitikimo palapinėje. 
\par 41 Apdarus tarnavimui šventykloje, šventus drabužius kunigui Aaronui ir drabužius jo sūnums, kad atliktų kunigų tarnystę. 
\par 42 Visa padarė izraelitai, kaip Viešpats įsakė Mozei. 
\par 43 Mozė apžiūrėjo darbą ir pamatė, kad viskas padaryta pagal Viešpaties nurodymus, ir Mozė palaimino juos.



\chapter{40}


\par 1 Ir Viešpats kalbėjo Mozei: 
\par 2 “Pirmo mėnesio pirmą dieną ištiesk Susitikimo palapinę, 
\par 3 pastatyk joje skrynią, pakabink prieš ją uždangą, 
\par 4 įnešk stalą ir padėk ant jo tai, kas turi ant jo būti. Įnešk žvakidę ir sustatyk ant jos lempas. 
\par 5 Auksinį smilkymo aukurą padėk priešais Liudijimo skrynią. Palapinės įėjime pakabink užuolaidą. 
\par 6 Pastatyk deginamųjų aukų aukurą priešais įėjimą į Susitikimo palapinę, 
\par 7 o tarp aukuro ir palapinės­ praustuvę, į kurią pripilk vandens. 
\par 8 Atskirk kiemą užkabomis ir pakabink užkabą įėjime. 
\par 9 Paėmęs patepimo aliejaus, patepk palapinę ir visus jos daiktus, ir tai bus šventa; 
\par 10 taip pat deginamųjų aukų aukurą, visus jo reikmenis 
\par 11 ir praustuvę su jos stovu. 
\par 12 Aaroną ir jo sūnus privesk prie Susitikimo palapinės įėjimo, apiplauk vandeniu 
\par 13 ir apvilk Aaroną šventais rūbais man tarnauti: patepk ir pašventink jį, kad jis galėtų būti man kunigu. 
\par 14 Taip pat ir jo sūnus privesk ir apvilk jupomis, 
\par 15 ir juos patepk, kaip jų tėvą patepei, kad galėtų būti man kunigais. Tas patepimas jiems bus amžinai kunigystei per kartų kartas”. 
\par 16 Mozė padarė visa, ką Viešpats įsakė. 
\par 17 Taigi antrųjų metų pirmojo mėnesio pirmą dieną palapinė buvo pastatyta. 
\par 18 Mozė pastatė ją: sustatė lentas, padėjo pakojus, įkišo užkaiščius, pastatė stulpus, 
\par 19 užtiesė palapinės uždangalą ir ant jo uždėjo kitus uždangalus, kaip Viešpats įsakė Mozei. 
\par 20 Į skrynią įdėjo liudijimą, įkišo kartis ir uždėjo dangtį. 
\par 21 Kai įnešė skrynią palapinėn, prieš ją pakabino uždangą, kaip Viešpats įsakė Mozei. 
\par 22 Pastatė ir stalą Susitikimo palapinės šiauriniame šone šiapus uždangos 
\par 23 ir ant jo sudėjo padėtinės duonos kepalus, kaip Viešpats įsakė Mozei. 
\par 24 Susitikimo palapinėje priešais stalą pastatė žvakidę 
\par 25 ir paruošė lempas, kaip Viešpats įsakė Mozei. 
\par 26 Prieš uždangą pastatė ir auksinį aukurą, 
\par 27 ant kurio degino kvapnius smilkalus, kaip Viešpats įsakė Mozei. 
\par 28 Susitikimo palapinės įėjime pakabino užuolaidą, 
\par 29 pastatė deginamųjų aukų aukurą priešais Susitikimo palapinę ir ant to aukuro aukojo deginamąją ir duonos auką, kaip Viešpats įsakė Mozei. 
\par 30 Tarp Susitikimo palapinės ir aukuro pastatė praustuvę ir į ją pripylė vandens. 
\par 31 Mozė, Aaronas ir jo sūnūs plaudavosi joje rankas ir kojas, 
\par 32 prieš eidami į Susitikimo palapinę ir prieš artindamiesi prie aukuro, kaip Viešpats įsakė Mozei. 
\par 33 Aplink palapinę ir aukurą atitvėrė kiemą ir jo įėjime pakabino užkabą. Taip Mozė užbaigė visus darbus. 
\par 34 Kai visa buvo baigta, debesis apgaubė Susitikimo palapinę ir Viešpaties šlovė pripildė ją. 
\par 35 Mozė negalėjo įeiti Susitikimo palapinėn, nes debesis buvo ant jos ir Viešpaties šlovė buvo ją pripildžiusi. 
\par 36 Debesiui pakilus nuo palapinės, izraelitai keliaudavo toliau; 
\par 37 jei debesis būdavo virš palapinės, jie pasilikdavo toje pačioje vietoje, kol debesis pakildavo. 
\par 38 Viešpaties debesis buvo virš palapinės dienos metu, o naktį virš jos buvo liepsna visų Izraelio vaikų akivaizdoje per visą kelionės laiką.




\end{document}