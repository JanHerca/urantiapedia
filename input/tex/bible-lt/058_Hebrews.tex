\begin{document}

\title{Laiškas žydams}


\chapter{1}


\par 1 Daugel kartų ir įvairiais būdais praeityje Dievas yra kalbėjęs tėvams per pranašus, 
\par 2 o šiomis paskutinėmis dienomis prakalbo mums per Sūnų, kurį paskyrė visa ko paveldėtoju ir per kurį sutvėrė pasaulius. 
\par 3 Jis, Dievo šlovės spindesys ir Jo esybės tikslus atvaizdas, viską laikantis savo jėgos žodžiu, pats nuplovęs mūsų nuodėmes, atsisėdo Didybės dešinėje aukštybėse, 
\par 4 tapdamas tiek pranašesnis už angelus, kiek prakilnesnį už juos paveldėjo vardą. 
\par 5 Kuriam gi angelų kada nors Jis yra pasakęs: “Tu esi mano Sūnus, šiandien pagimdžiau Tave”?! Ir vėl: “Aš Jam būsiu Tėvas, o Jis bus man Sūnus”. 
\par 6 Ir vėl, įvesdamas Pirmagimį į pasaulį, Jis sako: “Tepagarbina Jį visi Dievo angelai”. 
\par 7 O apie angelus sako: “Jis daro savo angelus vėjais ir savo tarnus­ugnies liepsnomis”. 
\par 8 O Sūnui: “Tavo sostas, Dieve, amžių amžiams, ir teisingumo skeptras yra tavo karalystės skeptras. 
\par 9 Tu pamilai teisumą ir nekentei nedorybės, todėl patepė Tave Dievas, Tavo Dievas, džiaugsmo aliejumi gausiau negu Tavo bičiulius”. 
\par 10 Ir: “Iš pradžių Tu, Viešpatie, padėjai pamatus žemei, ir dangūs­ Tavo rankų darbas. 
\par 11 Jie pražus, o Tu pasiliksi, jie visi sudils lyg drabužis, 
\par 12 ir kaip apsiaustą Tu juos suvyniosi, ir jie bus pakeisti. Bet Tu esi tas pats, ir Tavo metai nesibaigs”. 
\par 13 O kuriam iš angelų Jis yra kada sakęs: “Sėskis mano dešinėje, kol Aš patiesiu Tavo priešus, kaip pakojį po Tavo kojomis”? 
\par 14 Argi jie visi nėra tarnaujančios dvasios, išsiųstos tarnauti tiems, kurie paveldės išgelbėjimą?


\chapter{2}


\par 1 Todėl turime būti labai dėmesingi tam, ką girdėjome, kad nepraplauktume pro šalį. 
\par 2 Nes jei per angelus paskelbtas žodis buvo tvirtas ir kiekvienas nusižengimas bei neklusnumas susilaukdavo teisėto atlygio, 
\par 3 tai kaipgi pabėgsime mes, nepaisydami tokio didžio išgelbėjimo? Jis, prasidėjęs Viešpaties skelbimu, buvo mums patvirtintas tų, kurie Jį girdėjo, 
\par 4 Dievui liudijant ženklais ir stebuklais, visokiais galingais darbais ir Šventosios Dvasios dovanomis, paskirstytomis Jo valia. 
\par 5 Ne angelams Jis pajungė ateities pasaulį, apie kurį kalbame. 
\par 6 Bet kažkas kažkur paliudijo, sakydamas: “Kas yra žmogus, kad jį atsimeni, ar žmogaus sūnus, kad juo rūpiniesi? 
\par 7 Padarei jį trumpam laikui žemesnį už angelus, šlove ir garbe jį apvainikavai ir pastatei jį virš savo rankų darbų, 
\par 8 visa paklojai po jo kojomis”. Jeigu jau visa paklojo, tai nepaliko nieko, kas nebūtų jam paklota. Vis dėlto dabar mes dar nematome, jog visa jam paklota. 
\par 9 Bet matome Tą, trumpam laikui padarytą žemesnį už angelus­Jėzų, už mirties kentėjimus apvainikuotą šlove ir garbe. Reikėjo, kad Dievo malone už kiekvieną Jis paragautų mirties. 
\par 10 Nes Tam, dėl kurio ir iš kurio yra viskas, priderėjo, vedant daugybę vaikų į garbę, kentėjimais ištobulinti jų išgelbėjimo Vadovą. 
\par 11 Juk šventintojas ir šventinamieji­visi kyla iš vieno. Todėl Jis nesigėdija juos vadinti broliais, 
\par 12 sakydamas: “Aš paskelbsiu Tavo vardą savo broliams, vidury susirinkimo Tave šlovinsiu giesme”. 
\par 13 Ir vėl: “Aš Juo pasitikėsiu”. Ir vėl: “Štai Aš ir mano vaikai, kuriuos man davė Dievas”. 
\par 14 Kadangi vaikų kraujas ir kūnas bendri, tai ir Jis lygiomis juos prisiėmė, kad mirtimi sunaikintų tą, kuris turėjo mirties jėgą, tai yra velnią, 
\par 15 ir išvaduotų tuos, kurie, bijodami mirties, visam gyvenimui buvo patekę į vergiją. 
\par 16 Iš tiesų Jam rūpėjo ne angelai, o Abraomo palikuonys. 
\par 17 Todėl Jis turėjo visu kuo tapti panašus į brolius, kad būtų gailestingas ir ištikimas Dievui vyriausiasis Kunigas ir permaldautų už žmonių nuodėmes. 
\par 18 Pats iškentęs gundymus, Jis gali padėti tiems, kurie yra gundomi.


\chapter{3}


\par 1 Todėl, šventieji broliai, dangiškojo pašaukimo dalininkai, įsižiūrėkite mūsų išpažinimo Apaštalą ir vyriausiąjį Kunigą Jėzų Kristų, 
\par 2 kuris buvo ištikimas Jį paskyrusiam, kaip ir Mozė visuose Jo namuose. 
\par 3 Juk Jis yra pripažintas vertu didesnės šlovės už Mozę, kaip didesnės pagarbos vertas statytojas už namą. 
\par 4 Mat kiekvienas namas kažkieno pastatytas, o viską pastatęs yra Dievas. 
\par 5 Ir Mozė buvo ištikimas visuose Jo namuose kaip tarnas, kad paliudytų tai, kas ateity turėjo būti pasakyta. 
\par 6 O Kristus kaip Sūnus viešpatauja Jo namuose, ir tie namai esame mes, jei pasitikėjimą ir tvirtą vilties pasigyrimą išsaugosime iki galo. 
\par 7 Todėl, kaip Šventoji Dvasia sako: “Jei šiandien išgirsite Jo balsą, 
\par 8 neužkietinkite savo širdžių, kaip per maištą, gundymo dieną dykumoje, 
\par 9 kur jūsų tėvai gundė mane, mėgino ir matė mano darbus per keturiasdešimt metų. 
\par 10 Todėl Aš užpykau ant tos kartos ir pasakiau: ‘Visada jie klysta savo širdyje ir nepažino mano kelių’. 
\par 11 Taigi Aš prisiekiau savo rūstybėje: ‘Jie neįeis į mano poilsį!’ ” 
\par 12 Žiūrėkite, broliai, kad kuris iš jūsų nebūtų piktos, netikinčios širdies, atitolusios nuo gyvojo Dievo. 
\par 13 Verčiau raginkite vieni kitus kasdien,­kol dar sakoma “šiandien”,­kad kurio iš jūsų neužkietintų nuodėmės klasta. 
\par 14 Juk mes esame tapę Kristaus dalininkais,­jei tik išlaikysime tvirtą pradinį pasitikėjimą iki galo, 
\par 15 kol yra sakoma: “Šiandien, jei išgirsite Jo balsą, neužkietinkite savo širdžių kaip per maištą”. 
\par 16 Kas gi buvo tie, kurie išgirdę maištavo? Ar ne visi Mozės išvestieji iš Egipto? 
\par 17 Ant ko Jis buvo užpykęs per keturiasdešimt metų? Ar ne ant nusidėjusių, kurių lavonai krito dykumoje? 
\par 18 O kam prisiekė, jog neįeis į Jo poilsį, jei ne tiems, kurie nepakluso? 
\par 19 Taigi matome, kad jie negalėjo įeiti dėl netikėjimo.


\chapter{4}


\par 1 Todėl, kol tebegalioja pažadas įeiti į Jo poilsį, bijokime, kad kuris nors iš jūsų nepasirodytų pavėlavęs. 
\par 2 Mums, kaip ir jiems, buvo paskelbta Evangelija. Bet išgirstas žodis neišėjo jiems į naudą, nes nebuvo sujungtas su girdėjusiųjų tikėjimu. 
\par 3 O mes, įtikėjusieji, einame į tą poilsį, kaip Jo pasakyta: “Aš prisiekiau savo rūstybėje: ‘Jie neįeis į mano poilsį’ ”, nors darbai buvo užbaigti nuo pasaulio sutvėrimo. 
\par 4 Jis vienoje vietoje pasakė apie septintąją dieną: “Septintąją dieną Dievas ilsėjosi po visų savo darbų”. 
\par 5 Ir vėl anoje vietoje: “Jie neįeis į mano poilsį”. 
\par 6 Kadangi kai kuriems belieka įeiti, o tie, kuriems pirma buvo paskelbta Evangelija, neįėjo dėl neklusnumo, 
\par 7 Jis vėl nustato tam tikrą dieną, po tiek daug laiko, kartodamas Dovydo lūpomis,­“šiandien”,­kaip ir buvo pasakyta: “Šiandien, jei išgirsite Jo balsą, neužkietinkite savo širdžių”. 
\par 8 Jeigu Jozuė būtų juos įvedęs į poilsį, Dievas nebūtų po to kalbėjęs apie kitą dieną. 
\par 9 Taigi sabato poilsis tebepasilieka Dievo tautai, 
\par 10 nes, kas įeina į Jo poilsį, taip pat ilsisi po savo darbų, kaip Dievas ilsėjosi po savųjų. 
\par 11 Tad stenkimės įeiti į tą poilsį, kad niekas nebenupultų, sekdamas ano neklusnumo pavyzdžiu. 
\par 12 Dievo žodis yra gyvas ir veiksmingas, aštresnis už bet kokį dviašmenį kalaviją. Jis prasiskverbia iki pat sielos ir dvasios atšakos, iki sąnarių ir kaulų smegenų, ir teisia širdies mintis bei sumanymus. 
\par 13 Ir joks kūrinys nėra paslėptas nuo Jo žvilgsnio, bet visa yra nuoga ir atidengta akims To, kuriam turėsime duoti apyskaitą. 
\par 14 Taigi, turėdami didį vyriausiąjį Kunigą, praėjusį pro dangus Dievo Sūnų Jėzų, tvirtai laikykimės mūsų išpažinimo. 
\par 15 Juk mes turime ne tokį vyriausiąjį Kunigą, kuris negalėtų atjausti mūsų silpnybių, bet, kaip ir mes, visaip gundytą, tačiau nenusidėjusį. 
\par 16 Todėl drąsiai artinkimės prie malonės sosto, kad gautume gailestingumą ir rastume malonę pagalbai reikiamu metu.


\chapter{5}


\par 1 Kiekvienas vyriausiasis kunigas imamas iš žmonių ir skiriamas atstovauti žmonėms pas Dievą, kad aukotų dovanas ir aukas už nuodėmes. 
\par 2 Jis sugeba užjausti nežinančius ir klystančius, nes ir pats yra apgaubtas silpnumo 
\par 3 ir dėl to turi aukoti aukas­tiek už tautos, tiek ir už savo nuodėmes. 
\par 4 Ir niekas pats nepasiima tos garbės, vien tik tas, kuris Dievo pašauktas kaip Aaronas. 
\par 5 Taip pat ir Kristus ne pats sau suteikė šlovę tapti vyriausiuoju Kunigu, bet Tas, kuris Jam pasakė: “Tu esi mano Sūnus, šiandien Aš Tave pagimdžiau”. 
\par 6 Ir kitoje vietoje sako: “Tu esi kunigas per amžius Melchizedeko būdu”. 
\par 7 Jis savo kūno dienomis siuntė maldas bei prašymus su garsiu šauksmu bei ašaromis Tam, kuris galėjo išgelbėti Jį nuo mirties, ir buvo išklausytas dėl savo dievobaimingumo. 
\par 8 Nors būdamas Sūnus, Jis per savo kentėjimus išmoko paklusnumo 
\par 9 ir ištobulintas tapo amžinojo išgelbėjimo priežastimi visiems, kurie Jam paklūsta, 
\par 10 Dievo pavadintas vyriausiuoju Kunigu Melchizedeko būdu. 
\par 11 Apie tai mums reikėtų daug kalbėti, bet sunku jums išaiškinti, nes pasidarėte nerangūs klausyti. 
\par 12 Ir nors, žiūrint laiko, jūs jau turėtumėte būti mokytojai, iš tiesų reikia, kad jus vėl kas nors pamokytų Dievo žodžio pagrindų. Jūs tapote tokie, kuriems reikia pieno, o ne stipraus valgio. 
\par 13 Juk kiekvienas, maitinamas pienu, dar nepatyręs teisumo žodyje, nes tebėra kūdikis. 
\par 14 Tik subrendusiems dera stiprus maistas­tiems, kurie pratybomis išlavino savo pojūčius, kad atskirtų gera nuo blogo.


\chapter{6}


\par 1 Todėl, palikę pradinį Kristaus mokymą, veržkimės prie tobulumo, užuot vėl dėję pamatus iš atsivertimo nuo negyvų darbų, iš tikėjimo Dievu, 
\par 2 iš mokymo apie krikštus, rankų uždėjimo, mirusiųjų prikėlimo ir amžinybės teismo. 
\par 3 Jei Dievas leis, ir tai padarysime. 
\par 4 Kurie kartą jau buvo apšviesti, paragavo dangiškos dovanos, tapo Šventosios Dvasios dalininkais, 
\par 5 paragavo gerojo Dievo žodžio bei ateinančio amžiaus jėgų 
\par 6 ir atpuolė, tų nebeįmanoma vėl grąžinti naujai atgailai, nes jie kryžiuoja sau Dievo Sūnų ir išstato Jį viešai paniekai. 
\par 7 Jeigu žemė sugeria dažną lietų ir iš jos želia želmenys, naudingi tiems, kurie ją dirba, tai ji susilaukia Dievo palaiminimo, 
\par 8 bet jeigu ji teželdo erškėčius bei usnis, tai ji niekam tikusi, greitai bus prakeikta ir galiausiai sudeginta. 
\par 9 Nors taip kalbame, bet jums, mylimieji, tikimės geresnių, vedančių į išgelbėjimą dalykų. 
\par 10 Dievas nėra neteisingas, kad pamirštų jūsų darbą ir meilės triūsą, kurį parodėte Jo vardui, kai tarnavote ir tebetarnaujate šventiesiems. 
\par 11 Todėl trokštame, kad kiekvienas iš jūsų rodytų ankstesnį uolumą iki galo, kol pasieks visišką vilties užtikrintumą,­ 
\par 12 kad neaptingtumėte, bet būtumėte sekėjai tų, kurie tikėjimu ir kantrybe paveldi pažadus. 
\par 13 Kai Dievas davė Abraomui pažadą, neturėdamas kuo aukštesniu prisiekti, prisiekė savimi, 
\par 14 sakydamas: “Iš tiesų, Aš laiminte palaiminsiu tave, dauginte padauginsiu tave”. 
\par 15 Ir Abraomas, kantriai laukdamas, gavo, kas buvo pažadėta. 
\par 16 Žmonės prisiekia aukštesniais dalykais ir kiekvieno ginčo pabaigoje patvirtinimui imasi priesaikos. 
\par 17 Todėl Dievas, norėdamas stipriau pabrėžti pažado paveldėtojams savo valios nekintamumą, pridėjo priesaiką. 
\par 18 Tad du nekintami dalykai, kuriuose neįmanoma, kad Dievas meluotų, tvirtai guodžia mus, atradusius prieglobstį mums skirtoje viltyje. 
\par 19 Ji mums yra tarsi saugus ir tvirtas sielos inkaras, prasiskverbiantis pro uždangą vidun, 
\par 20 kur už mus kaip pirmtakas įžengė Jėzus, tapęs amžiams vyriausiuoju Kunigu Melchizedeko būdu.


\chapter{7}


\par 1 Tasai Melchizedekas, Salemo karalius, aukščiausiojo Dievo kunigas, sutiko Abraomą, kai jis grįžo, sumušęs karalius, ir palaimino jį. 
\par 2 Abraomas atidavė jam nuo visko dešimtinę. Išvertus jo vardą, jis pirmiausia teisumo karalius, paskui Salemo, arba ramybės, karalius,­ 
\par 3 be tėvo, be motinos, be kilmės sąrašo, neturintis nei dienų pradžios, nei gyvenimo pabaigos, panašus į Dievo Sūnų. Jis lieka kunigas visam laikui. 
\par 4 Pažvelkite, koks didis tas, kuriam ir patriarchas Abraomas davė dešimtinę nuo karo grobio! 
\par 5 Beje, ir Levio sūnūs, kurie gauna kunigystę, pagal įstatymą turi įsakymą imti dešimtines iš žmonių, tai yra imti iš savo brolių, nors ir jie kilę iš Abraomo strėnų. 
\par 6 O tas, kuris nėra kilęs iš jų, ėmė dešimtinę iš Abraomo ir palaimino tą, kuris turėjo pažadus. 
\par 7 Be jokių abejonių, visada žemesnį laimina aukštesnis. 
\par 8 Čia dešimtines ima mirtingi žmonės, o tenai­tas, apie kurį paliudyta, jog jis gyvena. 
\par 9 Ir, taip sakant, per Abraomą ir Levis mokėjo dešimtines, pats būdamas dešimtinių ėmėjas. 
\par 10 Žinoma, jis dar tebebuvo strėnose tėvo, kai šį pasitiko Melchizedekas. 
\par 11 Jeigu tobulumas būtų buvęs pasiekiamas levitų kunigystės dėka,­o tauta jos pagrindu buvo gavusi įstatymą,­tai kam dar būtų reikėję iškilti kitam kunigui Melchizedeko būdu ir nesivadinti kunigu Aarono būdu? 
\par 12 Juk, besikeičiant kunigystei, keičiasi ir įstatymas. 
\par 13 O tas, apie kurį tai sakoma, priklausė kitai giminei, iš kurios niekas netarnavo aukurui. 
\par 14 Juk aišku, kad mūsų Viešpats kilo iš Judo giminės, apie kurią Mozė nieko nėra kalbėjęs dėl kunigystės. 
\par 15 Tai dar labiau paaiškėja, kai iškyla kitas kunigas, panašus į Melchizedeką, 
\par 16 tapęs kunigu ne kūniško įstatymo įsakymu, bet nesibaigiančio gyvenimo jėga. 
\par 17 Juk Jis liudija: “Tu esi kunigas per amžius Melchizedeko būdu”. 
\par 18 Šitaip atšaukiamas ankstyvesnis įsakymas dėl savo silpnumo ir bergždumo,­ 
\par 19 įstatymas juk nieko nepadarė tobulo,­ir įvedama tvirtesnė viltis, kuria priartėjame prie Dievo. 
\par 20 Juo labiau, kad tai neįvyko be priesaikos, 
\par 21 mat anie tapdavo kunigais be priesaikos, o šis su priesaika To, kuris Jam pasakė: “Viešpats prisiekė, ir Jis nesigailės: ‘Tu esi kunigas per amžius Melchizedeko būdu’ ”. 
\par 22 Taigi Jėzus yra tapęs geresnės Sandoros garantu. 
\par 23 Anų kunigų buvo daug, nes mirtis jiems sukliudydavo ilgiau pasilikti. 
\par 24 O kadangi šis išlieka per amžius, Jis turi neatšaukiamą kunigystę. 
\par 25 Todėl Jis ir gali visada išgelbėti tuos, kurie per Jį eina prie Dievo, nes Jis amžinai gyvas, kad juos užtartų. 
\par 26 Mums ir priderėjo turėti tokį vyriausiąjį Kunigą: šventą, nekaltą, tyrą, atskirtą nuo nusidėjėlių ir išaukštintą virš dangų. 
\par 27 Jam nereikia, kaip tiems vyriausiesiems kunigams, kasdien atnašauti aukas pirma už savo nuodėmes, paskui už tautos, nes Jis tai atliko vieną kartą visiems laikams, paaukodamas pats save. 
\par 28 Įstatymas skiria vyriausiaisiais kunigais žmones su silpnybėmis, o priesaikos žodis, duotas po įstatymo, paskyrė amžiams ištobulintą Sūnų.


\chapter{8}


\par 1 Bet iš to, ką sakome, svarbiausia yra štai kas: mes turime tokį vyriausiąjį Kunigą, kuris danguje atsisėdo Didybės sosto dešinėje 
\par 2 ir tarnauja šventykloje bei tikroje palapinėje, kurią pastatė ne žmogus, o Viešpats. 
\par 3 Kiekvienas vyriausiasis kunigas skiriamas aukoti dovanas ir aukas, tad ir šis privalo turėti, ką aukoti. 
\par 4 Jeigu Jis būtų žemėje, Jis nebūtų kunigas, nes čia yra kunigai, kurie pagal įstatymą aukoja dovanas. 
\par 5 Jie tarnauja dangiškųjų dalykų paveikslui ir šešėliui, panašiai kaip buvo pamokytas Mozė, kai rengėsi statyti palapinę: “Žiūrėk,­sako Jis,­kad visa padarytum pagal tą pavyzdį, kuris tau buvo parodytas ant kalno”. 
\par 6 Bet dabar Jis gavo juo prakilnesnį tarnavimą, juo tobulesnės, geresniais pažadais besiremiančios Sandoros tarpininkas yra. 
\par 7 Jeigu ta pirmoji Sandora būtų buvusi be trūkumų, nebūtų reikėję ieškoti vietos antrajai. 
\par 8 Jis sakė, peikdamas juos: “Štai ateina dienos,­sako Viešpats,­ kai Aš su Izraelio namais ir Judo namais sudarysiu naują Sandorą. 
\par 9 Ne tokią Sandorą, kokią buvau sudaręs su jų tėvais tą dieną, kai paėmiau juos už rankos, norėdamas išvesti iš Egipto žemės. Jie neišsaugojo mano Sandoros, ir Aš juos apleidau,­sako Viešpats.­ 
\par 10 Štai kokia bus Sandora, kurią sudarysiu su Izraelio namais, praslinkus anoms dienoms,­sako Viešpats:­Aš duosiu savo įstatymus jų protams ir juos įrašysiu jų širdyse, ir būsiu jų Dievas, ir jie bus mano tauta. 
\par 11 Ir nė vienas nebemokys savo artimo nei savo brolio, sakydamas: ‘Pažink Viešpatį!’, nes visi mane pažins nuo mažiausio iki didžiausio. 
\par 12 Aš būsiu gailestingas dėl jų neteisumo ir jų nuodėmių bei nedorybių daugiau nebeprisiminsiu”. 
\par 13 Sakydamas “naują Sandorą”, Jis nurodė, kad pirmoji yra pasenusi. O kas pasensta ir nukaršta, to greit nebeliks.


\chapter{9}


\par 1 Tiesa, ir pirmoji Sandora turėjo tarnavimo Dievui potvarkius ir žemišką šventyklą. 
\par 2 Buvo įrengta palapinė: priešakinė dalis, kur buvo žvakidė, stalas ir padėtinės duonos kepalai, vadinosi Šventoji. 
\par 3 Už antrosios uždangos buvo palapinės dalis, vadinama Šventų švenčiausioji. 
\par 4 Ten stovėjo auksinis smilkytuvas ir iš visų šonų auksu apmušta Sandoros skrynia, kurioje buvo auksinis ąsotis su mana, išsprogusi Aarono lazda ir Sandoros plokštės. 
\par 5 Viršum jos buvo šlovės cherubai, kurie gaubė sutaikinimo dangtį. Apie tai dabar nėra reikalo smulkiau kalbėti. 
\par 6 Esant tokiai sąrangai, į priekinę palapinės dalį visada eina kunigai atlikti apeigų, 
\par 7 o į antrąją­kartą per metus vienas tik vyriausiasis kunigas, ir tai ne be kraujo, kurį aukoja už save ir už tautos nuodėmes, padarytas dėl nežinojimo. 
\par 8 Šitaip Šventoji Dvasia nurodo, kad kelias į Šventų švenčiausiąją dar nėra atviras, kol tebestovi pirmoji palapinė, 
\par 9 kuri yra dabartinio laikotarpio atvaizdas. Joje aukojamos dovanos ir aukos, kurios negali padaryti aukotojo sąžinėje tobulo, 
\par 10 bet apima tik valgius, gėrimus, įvairius apiplovimus ir kūniškus potvarkius, galiojančius iki atnaujinimo. 
\par 11 Bet Kristus, atėjęs kaip būsimųjų gėrybių vyriausiasis Kunigas, pro aukštesnę ir tobulesnę palapinę, ne rankų darbo, tai yra ne šitos kūrinijos, 
\par 12 taip pat ne ožių ar veršių krauju, o savuoju krauju vieną kartą visiems laikams įžengė į Švenčiausiąją ir įvykdė amžinąjį atpirkimą. 
\par 13 Jeigu jaučių bei ožių kraujas ir telyčios pelenai per apšlakstymą pašventina susitepusius, kad kūnas būtų švarus, 
\par 14 tai nepalyginti labiau kraujas Kristaus, kuris per amžinąją Dvasią paaukojo save kaip auką be dėmės Dievui, nuvalys jūsų sąžinę nuo mirties darbų, kad tarnautumėte gyvajam Dievui. 
\par 15 Ir todėl Jis yra naujosios Sandoros tarpininkas, kad, įvykus mirčiai pirmojoje Sandoroje padarytiems nusikaltimams atpirkti, pašauktieji gautų amžinojo palikimo pažadą. 
\par 16 Kur tik yra testamentas, ten būtina įrodyti testatoriaus mirtį. 
\par 17 Testamentas įgyja galią, mirus jo sudarytojui, o jam tebesant gyvam, testamentas negalioja. 
\par 18 Štai kodėl ir pirmoji Sandora nebuvo patvirtinta be kraujo. 
\par 19 Paskelbęs visai tautai visus įstatymo nuostatus, Mozė, paėmęs veršių bei ožių kraujo su vandeniu ir purpurinės vilnos su yzopu, apšlakstė pačią knygą ir visą tautą, 
\par 20 sakydamas: “Tai yra kraujas Sandoros, kurią įsakė jums Dievas!” 
\par 21 Jis taip pat apšlakstė krauju palapinę ir visus tarnavimo indus. 
\par 22 Taip pat beveik viskas pagal įstatymą apvaloma krauju, ir be kraujo praliejimo nėra atleidimo. 
\par 23 Todėl dangiškųjų dalykų atvaizdai turėjo būti šitaip apvalomi, o patys dangaus dalykai­geresnėmis aukomis negu šitos. 
\par 24 Mat Kristus įžengė ne į rankų darbo šventyklą­tikrosios atvaizdą, bet į patį dangų, kad dabar pasirodytų už mus Dievo akivaizdoje. 
\par 25 Ir ne tam, kad pakartotinai aukotų save, kaip daro vyriausiasis kunigas, kuris kasmet įeina į Švenčiausiąją su svetimu krauju,­ 
\par 26 tuomet Kristui būtų reikėję daugelį kartų kentėti nuo pasaulio sutvėrimo. Bet dabar Jis vieną kartą pasirodė amžių pabaigoje, kad savo auka sunaikintų nuodėmę. 
\par 27 Ir kaip žmonėms skirta vieną kartą mirti, o po to­teismas, 
\par 28 taip ir Kristus, vieną kartą paaukotas, kad pasiimtų daugelio nuodėmes, antrą kartą pasirodys be nuodėmės Jo laukiančiųjų išgelbėjimui.


\chapter{10}


\par 1 Kadangi Įstatymas turi tiktai būsimųjų gėrybių šešėlį, o ne patį dalykų vaizdą, jis niekada negali tomis pačiomis aukomis, kurios kasmet vis aukojamos ir aukojamos, padaryti tobulus tuos, kurie artinasi. 
\par 2 Argi tos aukos nesiliautų, jeigu aukotojai, vienąkart apvalyti, daugiau nebejaustų sąžinėje nuodėmių? 
\par 3 Priešingai: jos metai iš metų vis primena nuodėmes. 
\par 4 Juk neįmanoma, kad jaučių ir ožių kraujas panaikintų nuodėmes. 
\par 5 Todėl, ateidamas į pasaulį, Jis sako: “Aukų ir atnašų Tu nenorėjai, bet paruošei man kūną. 
\par 6 Tau nepatiko deginamosios atnašos ir aukos už nuodėmes. 
\par 7 Tuomet tariau: ‘Štai ateinu, kaip knygos rietime apie mane parašyta, vykdyti Tavo, o Dieve, valios!’ ” 
\par 8 Anksčiau pasakęs: “Aukų ir atnašų, deginamųjų atnašų ir atnašų už nuodėmes Tu nenorėjai ir nemėgai”,­jos aukojamos pagal Įstatymą,­ 
\par 9 paskui paskelbė: “Štai ateinu vykdyti Tavo, o Dieve, valios”. Jis panaikina viena, kad įtvirtintų kita. 
\par 10 Tos valios dėka esame Jėzaus Kristaus kūno auka vieną kartą pašventinti visiems laikams. 
\par 11 Kiekvienas kunigas diena iš dienos tarnauja ir daug kartų aukoja tas pačias aukas, kurios niekada negali panaikinti nuodėmių. 
\par 12 O šis, paaukojęs vienintelę auką už nuodėmes, amžiams atsisėdo Dievo dešinėje, 
\par 13 nuo tol laukdamas, kol Jo priešai bus patiesti tarsi pakojis po Jo kojų. 
\par 14 Vienintele atnaša Jis amžiams padarė tobulus šventinamuosius. 
\par 15 Tai mums liudija ir Šventoji Dvasia. Ji yra pasakiusi: 
\par 16 “Štai Sandora, kurią su jais sudarysiu, praslinkus anoms dienoms,­sako Viešpats:­Aš įdėsiu savo įstatymus į jų širdis ir juos įrašysiu jų mintyse, 
\par 17 ir jų nuodėmių bei jų nedorybių daugiau nebeprisiminsiu”. 
\par 18 O kur jos atleistos, ten nebereikia atnašos už nuodėmę. 
\par 19 Taigi, broliai, galėdami drąsiai įeiti į Švenčiausiąją dėl Jėzaus kraujo 
\par 20 nauju ir gyvu keliu, kurį Jis atvėrė mums per uždangą, tai yra savąjį kūną, 
\par 21 ir turėdami didį Kunigą Dievo namams, 
\par 22 artinkimės su tyra širdimi ir giliu, užtikrintu tikėjimu, apšlakstymu apvalę širdis nuo nešvarios sąžinės ir nuplovę kūną švariu vandeniu! 
\par 23 Išlaikykime nepajudinamą vilties išpažinimą, nes ištikimas Tas, kuris pažadėjo. 
\par 24 Žiūrėkime vieni kitų, skatindami mylėti ir daryti gerus darbus. 
\par 25 Neapleiskime savųjų susirinkimo, kaip kai kurie yra pratę, bet raginkime vieni kitus juo labiau, juo aiškiau regime besiartinančią dieną. 
\par 26 Jeigu, pasiekę tiesos pažinimą, sąmoningai nusidedame, tada nebelieka aukos už nuodėmes, 
\par 27 bet kažkoks baisus laukimas teismo ir liepsnojančio pykčio, kuris praris priešininkus. 
\par 28 Jei kas atstumia Mozės Įstatymą, tas be jokio pasigailėjimo turi mirti, dviem ar trims liudytojams paliudijus. 
\par 29 Tik pagalvokite: kaip dar sunkesnės bausmės nusipelnys tas, kuris sutrypė kojomis Dievo Sūnų, nešventu palaikė Sandoros kraują, kuriuo buvo pašventintas, ir įžeidė malonės Dvasią! 
\par 30 Juk pažįstame Tą, kuris pasakė: “Mano kerštas, Aš atsilyginsiu,­ sako Viešpats”. Ir vėl: “Viešpats teis savo tautą”. 
\par 31 Baisu pakliūti į gyvojo Dievo rankas! 
\par 32 Prisiminkite ankstesnes dienas, kada jūs apšviesti ištvėrėte didelę kentėjimų kovą, 
\par 33 tiek patys išstatyti viešam reginiui su paniekinimais ir smurtu, tiek būdami dalininkai tų, su kuriais buvo taip elgiamasi. 
\par 34 Jūs užjautėte mane, kalinį, ir linksmai sutikote savo turto išplėšimą, žinodami, jog turite danguje geresnį ir išliekantį turtą. 
\par 35 Tad nepameskite savo pasitikėjimo, už kurį skirtas didelis atlygis! 
\par 36 Taip, reikia jums ištvermės, kad, įvykdę Dievo valią, gautumėte, kas pažadėta. 
\par 37 Nes “dar trumpa, trumpa valandėlė, ir ateis Tas, kuris turi ateiti, ir neužtruks. 
\par 38 Bet teisusis gyvens tikėjimu, ir, jeigu jis atsitrauktų, mano siela juo nebesigėrės”. 
\par 39 Tačiau mes nesame tie, kurie atsitraukia savo pražūčiai, bet tie, kurie tiki, kad išgelbėtume sielą.


\chapter{11}


\par 1 Tikėjimas užtikrina tai, ko viliamės, ir parodo tai, ko nematome. 
\par 2 Per jį protėviai gavo gerą liudijimą. 
\par 3 Tikėjimu suvokiame, kad pasauliai buvo sutverti Dievo žodžiu, būtent iš neregimybės atsirado regima. 
\par 4 Tikėjimu Abelis aukojo geresnę auką negu Kainas ir dėl tikėjimo gavo liudijimą, kad yra teisus, Dievui paliudijus apie jo dovanas. Dėl tikėjimo jis ir miręs tebekalba. 
\par 5 Tikėjimu Henochas buvo perkeltas, kad nematytų mirties, ir “jo neberado, nes Dievas jį perkėlė”. Mat prieš perkeliamas, jis gavo liudijimą, kad patikęs Dievui. 
\par 6 O be tikėjimo neįmanoma patikti Dievui. Kas artinasi prie Dievo, tam būtina tikėti, kad Jis yra ir kad uoliai Jo ieškantiems atsilygina. 
\par 7 Tikėjimu Nojus, Dievo perspėtas apie tuo metu dar nematomus dalykus, būdamas dievobaimingas, pastatė arką savo šeimai išgelbėti; tikėjimu jis pasmerkė pasaulį ir paveldėjo tikėjimo teisumą. 
\par 8 Tikėjimu Abraomas pakluso, kai buvo pašauktas keliauti į šalį, kurią turėjo paveldėti, ir išvyko, nežinodamas kur einąs. 
\par 9 Tikėjimu jis apsigyveno pažado žemėje, tarytum svetimoje, gyvendamas palapinėse su Izaoku ir Jokūbu, to paties pažado bendrapaveldėtojais. 
\par 10 Mat jis laukė miesto su pamatais, kurio statytojas ir kūrėjas yra Dievas. 
\par 11 Tikėjimu ir pati Sara­nevaisinga ir nebe to amžiaus­gavo galios pastoti ir pagimdė vaiką, nes ji laikė ištikimu Tą, kuris pažadėjo. 
\par 12 Todėl iš vieno vyro, ir dar apmirusio, gimė palikuonys, gausūs tartum dangaus žvaigždės ir nesuskaitomi kaip jūros pakrantės smiltys. 
\par 13 Jie visi mirė tikėdami, dar negavę pažadėtųjų dalykų, bet iš tolo juos regėdami, buvo įsitikinę jais ir priėmė juos, išpažindami, jog jie žemėje svečiai ir keleiviai. 
\par 14 Kurie taip kalba, parodo, kad ieško tėvynės. 
\par 15 Jeigu jie būtų minėję aną, iš kurios iškeliavo, jie būtų turėję laiko sugrįžti atgal. 
\par 16 Bet dabar jie siekė geresnės tėvynės, tai yra dangiškosios. Todėl Dievas nesigėdija vadintis jų Dievu: juk Jis paruošė jiems Miestą! 
\par 17 Tikėjimu Abraomas aukojo Izaoką, kai buvo mėginamas. Jis, kuris buvo gavęs pažadą, aukojo savo viengimį sūnų, 
\par 18 apie kurį buvo pasakyta: “Iš Izaoko bus pašaukti tavo palikuonys”. 
\par 19 Jis suprato, kad Dievas gali prikelti net iš mirties, ir atgavo sūnų tarytum iš numirusių. 
\par 20 Tikėjimu Izaokas palaimino ateičiai Jokūbą ir Ezavą. 
\par 21 Tikėjimu Jokūbas mirties valandą palaimino kiekvieną Juozapo sūnų ir pagarbino, atsirėmęs į savo lazdos drūtgalį. 
\par 22 Tikėjimu merdintis Juozapas priminė apie Izraelio vaikų iškeliavimą ir davė nurodymų dėl savo palaikų. 
\par 23 Tikėjimu Mozė tris mėnesius buvo tėvų paslėptas, nes jie matė, koks kūdikis dailus, ir neišsigando karaliaus įsakymo. 
\par 24 Tikėjimu Mozė užaugęs atsisakė vadintis faraono dukters sūnumi. 
\par 25 Jis verčiau pasirinko su Dievo tauta kęsti sunkumus negu laikinai džiaugtis nuodėmės malonumais. 
\par 26 Jis Kristaus paniekinimą laikė didesniu turtu negu Egipto brangenybes, nes jis žvelgė į atlygį. 
\par 27 Tikėjimu jis paliko Egiptą, neišsigandęs karaliaus rūstybės, nes liko nepajudinamas, tarsi regėtų Neregimąjį. 
\par 28 Tikėjimu jis įsteigė Paschą ir apšlakstymą krauju, kad naikintojas nepaliestų jų pirmagimių. 
\par 29 Tikėjimu jie perėjo per Raudonąją jūrą tartum per sausumą, o tai daryti mėginantys egiptiečiai prigėrė. 
\par 30 Tikėjimu buvo sugriauti Jericho mūrai po septynių dienų žygiavimo aplinkui. 
\par 31 Tikėjimu paleistuvė Rahaba nepražuvo kartu su neklusniaisiais; mat ji taikingai buvo priėmusi žvalgus. 
\par 32 Ką dar pasakyti? Man neužtektų laiko, jeigu imčiau pasakoti apie Gedeoną, Baraką, Samsoną, Jeftę, Dovydą, Samuelį ir pranašus, 
\par 33 kurie tikėjimu nugalėjo karalystes, vykdė teisumą, įgijo pažadus, užčiaupė liūtams nasrus, 
\par 34 užgesino ugnies karštį, paspruko nuo kalavijo ašmenų, sustiprėjo iš silpnumo, tapo galiūnais kovoje, privertė bėgti svetimųjų pulkus. 
\par 35 Moterys atgavo prikeltus savo mirusiuosius, kiti buvo kankinami ir atsisakė išlaisvinimo, kad gautų prakilnesnį prisikėlimą. 
\par 36 Dar kiti iškentė patyčias ir plakimus, taip pat pančius ir kalėjimą. 
\par 37 Jie buvo akmenimis užmušami, pjaustomi pusiau, gundomi, kardu žudomi, klajojo prisidengę avių ir ožkų kailiais, vargo, kentė priespaudą ir kankinimus. 
\par 38 Jie, kurių pasaulis nebuvo vertas, klajojo dykumose ir kalnuose, slapstėsi olose ir žemės plyšiuose. 
\par 39 Ir jie visi, per tikėjimą gavę gerą liudijimą, negavo to, kas buvo pažadėta, 
\par 40 nes Dievas geresnius dalykus buvo numatęs mums, kad jie ne be mūsų pasiektų tobulumą.


\chapter{12}


\par 1 Todėl ir mes, tokio didelio debesies liudytojų apsupti, nusimeskime visus apsunkinimus bei lengvai apraizgančią nuodėmę ir ištvermingai bėkime mums paskirtose lenktynėse, 
\par 2 žiūrėdami į mūsų tikėjimo pradininką ir atbaigėją Jėzų. Jis vietoj sau priderančio džiaugsmo, nepaisydamas gėdos, iškentėjo kryžių ir atsisėdo Dievo sosto dešinėje. 
\par 3 Apsvarstykite, kaip Jis iškentė nuo nusidėjėlių tokį priešiškumą, kad nepailstumėte ir nesuglebtumėte savo sielomis! 
\par 4 Jums dar neteko priešintis iki kraujų, kovojant su nuodėme. 
\par 5 Jūs pamiršote paraginimą, kuris sako jums kaip sūnums: “Mano sūnau, nepaniekink Viešpaties auklybos ir nenusimink Jo baramas: 
\par 6 nes kurį Viešpats myli, tą griežtai auklėja, ir plaka kiekvieną sūnų, kurį priima”. 
\par 7 Jeigu jūs pakenčiat drausmę, Dievas elgiasi su jumis kaip su sūnumis. O kurio gi sūnaus tėvas griežtai neauklėja? 
\par 8 Bet jeigu jūs be drausmės, kuri visiems privaloma, vadinasi, jūs ne sūnūs, o pavainikiai. 
\par 9 Jau mūsų kūno tėvai mus bausdavo, ir mes juos gerbėme. Tad argi nebūsime dar klusnesni dvasių Tėvui, kad gyventume? 
\par 10 Juk anie savo nuožiūra mus drausmino neilgą laiką, o šis tai daro mūsų labui, kad taptume Jo šventumo dalininkais. 
\par 11 Beje, kiekviena auklyba tam kartui atrodo ne linksma, o karti, bet vėliau ji atneša taikingų teisumo vaisių auklėtiniams. 
\par 12 Todėl pakelkite nuleistas rankas, sustiprinkite linkstančius kelius 
\par 13 ir ištiesinkite takus po savo kojomis, kad, kas luoša, neišnirtų, bet verčiau sugytų. 
\par 14 Siekite santaikos su visais, siekite šventumo, be kurio niekas neregės Viešpaties. 
\par 15 Žiūrėkite, kad kas neprarastų Dievo malonės, kad neišleistų daigų kokia karti šaknis ir nepadarytų vargo, suteršdama daugelį; 
\par 16 kad neatsirastų ištvirkėlių ir bedievių kaip Ezavas, už valgio kąsnį pardavęs pirmagimio teises. 
\par 17 Jūs žinote, kad jis ir paskui, norėdamas paveldėti palaiminimą, buvo atmestas, nes nerado progos atgailai, nors su ašaromis jos ieškojo. 
\par 18 Jūs prisiartinote ne prie apčiuopiamo ir liepsnojančio ugnimi kalno ar prie tamsos, ar ūkanų, ar viesulo, 
\par 19 ar trimito skardenimo, ar žodžių skambesio, kurį išgirdę žmonės meldė, kad daugiau nebūtų ištarta nė žodžio. 
\par 20 Mat jie negalėjo pakelti įsakymo: “Net jeigu gyvulys paliestų kalną, jis privalo būti užmuštas akmenimis arba nušautas strėle”. 
\par 21 Anas reginys buvo toks baisus, jog Mozė pasakė: “Labai išsigandau ir visas drebu!” 
\par 22 Bet jūs prisiartinote prie Siono kalno bei gyvojo Dievo miesto, dangiškosios Jeruzalės, prie nesuskaitomų tūkstančių angelų 
\par 23 ir šventiško susirinkimo, prie danguje įrašytųjų pirmagimių bažnyčios, prie visų Teisėjo Dievo, prie ištobulintų teisiųjų dvasių 
\par 24 ir prie Naujosios Sandoros Tarpininko Jėzaus bei prie apšlakstymo kraujo, kuris kalba apie geresnius dalykus negu Abelio kraujas. 
\par 25 Žiūrėkite, kad neatstumtumėte kalbančiojo, nes jeigu anie neištrūko, kai atmetė Tą, kuris žemėje davė įspėjimų, tai juo labiau neištrūksime mes, nusigręžę nuo To, kuris kalba iš dangaus. 
\par 26 Jo balsas anuomet drebino žemę, o dabar Jis pažadėjo, sakydamas: “Aš dar kartą sudrebinsiu ne tik žemę, bet ir dangų!” 
\par 27 Žodžiai “dar kartą” rodo, kad iš sutvertųjų dalykų bus pašalinti sudrebinamieji, kad pasiliktų tai, kas nesudrebinama. 
\par 28 Todėl, gaudami nesudrebinamą karalystę, tvirtai laikykimės malonės, kuria galime deramai tarnauti Dievui su pagarba ir baime, 
\par 29 nes mūsų Dievas yra ryjanti ugnis.


\chapter{13}


\par 1 Teišsilaiko broliška meilė. 
\par 2 Nepamirškite svetingumo, nes per jį kai kurie, patys to nežinodami, priėmė viešnagėn angelus. 
\par 3 Prisiminkite kalinius, tarsi kartu būdami įkalinti, prisiminkite tuos, su kuriais piktai elgiamasi, nes patys tebesate kūne. 
\par 4 Tebūna visų gerbiama santuoka ir nesuteptas santuokos patalas. O ištvirkėlius ir svetimautojus teis Dievas. 
\par 5 Gyvenkite be godumo pinigams, būkite patenkinti tuo, ką turite, nes Jis pats yra pasakęs: “Niekad Aš tavęs nepaliksiu ir nepamiršiu”. 
\par 6 Todėl galime su pasitikėjimu tarti: “Viešpats mano padėjėjas­aš nebijosiu! Ką gali padaryti man žmogus?” 
\par 7 Atsiminkite savo vadovus, kurie jums skelbė Dievo žodį. Įsižiūrėkite į jų gyvenimo vaisius, sekite jų tikėjimu. 
\par 8 Jėzus Kristus yra tas pats vakar, šiandien ir per amžius. 
\par 9 Nesiduokite suklaidinami įvairių ir svetimų mokslų, nes gera, kai širdis sustiprinama malone, o ne valgiais, kurie nedavė naudos tiems, kurie jų laikėsi. 
\par 10 Mes turime aukurą, nuo kurio valgyti neturi teisės tie, kurie tarnauja palapinei. 
\par 11 Juk kūnai gyvulių, kurių kraujas vyriausiojo kunigo įnešamas į šventyklą už nuodėmes, sudeginami už stovyklos. 
\par 12 Todėl ir Jėzus, norėdamas savo krauju pašventinti tautą, kentėjo už vartų. 
\par 13 Taigi išeikime pas Jį už stovyklos, nešdami Jo paniekinimą. 
\par 14 Čia mes neturime išliekančio miesto, bet ieškome būsimojo. 
\par 15 Todėl per Jį visada aukokime Dievui šlovinimo auką, tai yra Jo vardą garbinančių lūpų vaisių. 
\par 16 Nepamirškite daryti gera ir dalintis su kitais, nes tokios aukos patinka Dievui. 
\par 17 Klausykite savo vadovų ir būkite jiems atsidavę, nes jie budi jūsų sielų labui, būdami atsakingi už jas; jie tai tedaro su džiaugsmu, o ne dūsaudami, nes tai nebūtų jums naudinga. 
\par 18 Melskite už mus, nes esame įsitikinę turį gerą sąžinę ir norį visame kame dorai elgtis. 
\par 19 Itin prašau melsti, kad būčiau greičiau jums sugrąžintas. 
\par 20 Ramybės Dievas, amžinosios Sandoros krauju išvedęs iš numirusių didįjį avių Ganytoją­mūsų Viešpatį Jėzų, 
\par 21 teištobulina jus kiekvienam geram darbui, kad vykdytumėte Jo valią, Jam veikiant jumyse, kas Jo akims patinka per Jėzų Kristų. Jam šlovė per amžių amžius! Amen. 
\par 22 Aš maldauju jus, broliai, kantriai priimkite šį paraginimo žodį: juk parašiau jums trumpai. 
\par 23 Žinokite, kad mūsų brolis Timotiejus yra išleistas laisvėn. Jei jis greitai atvyks, ir aš su juo drauge pamatysiu jus. 
\par 24 Sveikinkite visus savo vadovus ir visus šventuosius. Jus sveikina broliai iš Italijos. 
\par 25 Malonė teesie su jumis visais! Amen.



\end{document}