\begin{document}

\title{Ezechiele}


\chapter{1}

\par 1 Or avvenne l'anno trentesimo, il quinto giorno del quarto mese, che, essendo presso al fiume Kebar, fra quelli ch'erano stati menati in cattività, i cieli s'aprirono, e io ebbi delle visioni divine.
\par 2 Il quinto giorno del mese (era il quinto anno della cattività del re Joiakin),
\par 3 la parola dell'Eterno fu espressamente rivolta al sacerdote Ezechiele, figliuolo di Buzi, nel paese dei Caldei, presso al fiume Kebar; e la mano dell'Eterno fu quivi sopra lui.
\par 4 Io guardai, ed ecco venire dal settentrione un vento di tempesta, una grossa nuvola con un globo di fuoco che spandeva tutto all'intorno d'essa uno splendore; e nel centro di quel fuoco si vedeva come del rame sfavillante in mezzo al fuoco.
\par 5 Nel centro del fuoco appariva la forma di quattro esseri viventi; e questo era l'aspetto loro: avevano sembianza umana.
\par 6 Ognuno d'essi aveva quattro facce, e ognuno quattro ali.
\par 7 I loro piedi eran diritti, e la pianta de' loro piedi era come la pianta del piede d'un vitello; e sfavillavano come il rame terso.
\par 8 Avevano delle mani d'uomo sotto le ali ai loro quattro lati; e tutti e quattro avevano le loro facce e le loro ali.
\par 9 Le loro ali si univano l'una all'altra; camminando, non si voltavano; ognuno camminava dritto dinanzi a sé.
\par 10 Quanto all'aspetto delle loro facce, essi avevan tutti una faccia d'uomo, tutti e quattro una faccia di leone a destra, tutti e quattro una faccia di bue a sinistra, e tutti e quattro una faccia d'aquila.
\par 11 Le loro facce e le loro ali erano separate nella parte superiore; ognuno aveva due ali che s'univano a quelle dell'altro, e due che coprivan loro il corpo.
\par 12 Camminavano ognuno dritto davanti a sé, andavano dove lo spirito li faceva andare, e, camminando, non si voltavano.
\par 13 Quanto all'aspetto degli esseri viventi, esso era come di carboni ardenti, come di fiaccole; quel fuoco circolava in mezzo agli esseri viventi, era un fuoco sfavillante, e dal fuoco uscivan de' lampi.
\par 14 E gli esseri viventi correvano in tutti i sensi, simili al fulmine.
\par 15 Or com'io stavo guardando gli esseri viventi, ecco una ruota in terra, presso a ciascun d'essi, verso le loro quattro facce.
\par 16 L'aspetto delle ruote e la loro forma eran come l'aspetto del crisolito; tutte e quattro si somigliavano; il loro aspetto e la loro forma eran quelli d'una ruota che fosse attraversata da un'altra ruota.
\par 17 Quando si movevano, andavano tutte e quattro dal proprio lato, e, andando, non si voltavano.
\par 18 Quanto ai loro cerchi, essi erano alti e formidabili; e i cerchi di tutte e quattro eran pieni d'occhi d'ogn'intorno.
\par 19 Quando gli esseri viventi camminavano, le ruote si movevano allato a loro; e quando gli esseri viventi s'alzavan su da terra, s'alzavano anche le ruote.
\par 20 Dovunque lo spirito voleva andare, andavano anch'essi; e le ruote s'alzavano allato a quelli, perché lo spirito degli esseri viventi era nelle ruote.
\par 21 Quando quelli camminavano, anche le ruote si movevano; quando quelli si fermavano, anche queste si fermavano; e quando quelli s'alzavano su dalla terra, anche queste s'alzavano allato ad essi perché lo spirito degli esseri viventi era nelle ruote.
\par 22 Sopra le teste degli esseri viventi c'era come una distesa di cielo, di colore simile a cristallo d'ammirabile splendore, e s'espandeva su in alto, sopra alle loro teste.
\par 23 E sotto la distesa si drizzavano le loro ali, l'una verso l'altra; e ne avevano ciascuno due che coprivano loro il corpo.
\par 24 E quand'essi camminavano, io sentivo il rumore delle loro ali, come il rumore delle grandi acque, come la voce dell'Onnipotente: un rumore di gran tumulto, come il rumore d'un accampamento; quando si fermavano, abbassavano le loro ali;
\par 25 e s'udiva un rumore che veniva dall'alto della distesa ch'era sopra le loro teste.
\par 26 E al disopra della distesa che stava sopra le loro teste, c'era come una pietra di zaffiro, che pareva un trono; e su questa specie di trono appariva come la figura d'un uomo, che vi stava assiso sopra, su in alto.
\par 27 Vidi pure come del rame terso, come del fuoco, che lo circondava d'ogn'intorno dalla sembianza dei suoi fianchi in su; e dalla sembianza dei suoi fianchi in giù vidi come del fuoco, come uno splendore tutto attorno a lui.
\par 28 Qual è l'aspetto dell'arco ch'è nella nuvola in un giorno di pioggia, tal era l'aspetto di quello splendore che lo circondava. Era una apparizione dell'immagine della gloria dell'Eterno. A questa vista caddi sulla mia faccia, e udii la voce d'uno che parlava.

\chapter{2}

\par 1 E mi disse: 'Figliuol d'uomo, rizzati in piedi, e io ti parlerò'.
\par 2 E com'egli mi parlava, lo spirito entrò in me, e mi fece rizzare in piedi; e io udii colui che mi parlava.
\par 3 Egli mi disse: 'Figliuol d'uomo, io ti mando ai figliuoli d'Israele, a nazioni ribelli, che si son ribellate a me; essi e i loro padri si son rivoltati contro di me fino a questo giorno.
\par 4 A questi figliuoli dalla faccia dura e dal cuore ostinato io ti mando, e tu dirai loro: Così parla il Signore, l'Eterno.
\par 5 E sia che t'ascoltino o non t'ascoltino - giacché è una casa ribelle - essi sapranno che v'è un profeta in mezzo a loro.
\par 6 E tu, figliuol d'uomo, non aver paura di loro, né delle loro parole, giacché tu stai colle ortiche e colle spine, e abiti fra gli scorpioni; non aver paura delle loro parole, non ti sgomentare davanti a loro, poiché sono una casa ribelle.
\par 7 Ma tu riferirai loro le mie parole, sia che t'ascoltino o non t'ascoltino, poiché sono ribelli.
\par 8 E tu, figliuol d'uomo, ascolta ciò che ti dico; non esser ribelle com'è ribelle questa casa; apri la bocca, e mangia ciò che ti do'.
\par 9 Io guardai, ed ecco una mano stava stesa verso di me, la quale teneva il rotolo d'un libro;
\par 10 ed egli lo spiegò davanti a me; era scritto di dentro e di fuori, e conteneva delle lamentazioni, de' gemiti e de' guai.

\chapter{3}

\par 1 Ed egli mi disse: 'Figliuol d'uomo, mangia ciò che tu trovi; mangia questo rotolo, e va' e parla alla casa d'Israele'.
\par 2 Io aprii la bocca, ed egli mi fece mangiare quel rotolo.
\par 3 E mi disse: 'Figliuol d'uomo, nutriti il ventre e riempiti le viscere di questo rotolo che ti dò'. E io lo mangiai, e mi fu dolce in bocca, come del miele.
\par 4 Ed egli mi disse: 'Figliuol d'uomo, va', recati alla casa d'Israele, e riferisci loro le mie parole;
\par 5 poiché tu sei mandato, non a un popolo dal parlare oscuro e dalla lingua non intelligibile, ma alla casa d'Israele:
\par 6 non a molti popoli dal parlare oscuro e dalla lingua non intelligibile, di cui tu non intenda le parole. Certo, s'io ti mandassi a loro, essi ti darebbero ascolto;
\par 7 ma la casa d'Israele non ti vorrà ascoltare, perché non vogliono ascoltar me; giacché tutta la casa d'Israele ha la fronte dura e il cuore ostinato.
\par 8 Ecco, io t'induro la faccia, perché tu l'opponga alla faccia loro; induro la tua fronte, perché tu l'opponga alla fronte loro;
\par 9 io rendo la tua fronte come un diamante più dura della selce; non li temere, non ti sgomentare davanti a loro, perché sono una casa ribelle'.
\par 10 Poi mi disse: 'Figliuol d'uomo, ricevi nel cuor tuo tutte le parole che io ti dirò, e ascoltale con le tue orecchie.
\par 11 E va' dai figliuoli del tuo popolo che sono in cattività, parla loro, e di' loro: - Così parla il Signore, l'Eterno; sia che t'ascoltino o non ti ascoltino'.
\par 12 E lo spirito mi levò in alto, e io udii dietro a me il suono d'un gran fragore che diceva: 'Benedetta sia la gloria dell'Eterno dalla sua dimora!'
\par 13 e udii pure il rumore delle ali degli esseri viventi che battevano l'una contro l'altra, il rumore delle ruote allato ad esse e il suono d'un gran fragore.
\par 14 E lo spirito mi levò in alto, e mi portò via; e io andai, pieno d'amarezza nello sdegno del mio spirito; e la mano dell'Eterno era forte su di me.
\par 15 E giunsi da quelli ch'erano in cattività a Tel-abib presso al fiume Kebar, e mi fermai dov'essi dimoravano; e dimorai quivi sette giorni, mesto e silenzioso, in mezzo a loro.
\par 16 E in capo a sette giorni, la parola dell'Eterno mi fu rivolta in questi termini:
\par 17 'Figliuol d'uomo, io t'ho stabilito come sentinella per la casa d'Israele; e quando tu udrai dalla mia bocca una parola, tu li avvertirai da parte mia.
\par 18 Quando io dirò all'empio: - Certo morrai, - se tu non l'avverti, e non parli per avvertire quell'empio di abbandonar la sua via malvagia, e salvargli così la vita, quell'empio morrà per la sua iniquità; ma io domanderò conto del suo sangue alla tua mano.
\par 19 Ma, se tu avverti l'empio, ed egli non si ritrae dalla sua empietà e dalla sua via malvagia, egli morrà per la sua iniquità, ma tu avrai salvata l'anima tua.
\par 20 E quando un giusto si ritrae dalla sua giustizia e commette l'iniquità, se io gli pongo davanti una qualche occasione di caduta, egli morrà, perché tu non l'avrai avvertito; morrà per il suo peccato, e le cose giuste che avrà fatte non saranno più ricordate; ma io domanderò conto del suo sangue alla tua mano.
\par 21 Però se tu avverti quel giusto perché non pecchi, e non pecca, egli certamente vivrà, perch'è stato avvertito, e tu avrai salvata l'anima tua'.
\par 22 E la mano dell'Eterno fu quivi sopra me, ed egli mi disse: 'Lèvati, va' nella pianura, e quivi io parlerò teco'.
\par 23 Io dunque mi levai, uscii nella pianura, ed ecco che quivi stava la gloria dell'Eterno, gloria simile a quella che avevo veduta presso il fiume Kebar; e caddi sulla mia faccia.
\par 24 Ma lo spirito entrò in me; mi fece rizzare in piedi, e l'Eterno mi parlò e mi disse: 'Va', chiuditi in casa tua!
\par 25 E a te, figliuol d'uomo, ecco, ti si metteranno addosso delle corde, con esse ti si legherà, e tu non andrai in mezzo a loro.
\par 26 E io farò che la lingua ti s'attacchi al palato, perché tu rimanga muto e tu non possa esser per essi un censore; perché sono una casa ribelle.
\par 27 Ma quando io ti parlerò, t'aprirò la bocca, e tu dirai loro: - Così parla il Signore, l'Eterno; chi ascolta, ascolti; chi non vuole ascoltare non ascolti; poiché sono una casa ribelle.

\chapter{4}

\par 1 E tu, figliuol d'uomo, prenditi un mattone, mettitelo davanti e disegnavi sopra una città, Gerusalemme;
\par 2 cingila d'assedio, costruisci contro di lei una torre, fa' contro di lei dei bastioni, circondala di vari accampamenti, e disponi contro di lei, d'ogn'intorno, degli arieti.
\par 3 Prenditi poi una piastra di ferro, e collocala come un muro di ferro fra te e la città; vòlta la tua faccia contro di lei; sia ella assediata, e tu cingila d'assedio. Questo sarà un segno per la casa d'Israele.
\par 4 Poi sdràiati sul tuo lato sinistro, e metti su questo lato l'iniquità della casa d'Israele; e per il numero di giorni che starai sdraiato su quel lato, tu porterai la loro iniquità.
\par 5 E io ti conterò gli anni della loro iniquità in un numero pari a quello di que' giorni: trecentonovanta giorni. Tu porterai così l'iniquità della casa d'Israele.
\par 6 E quando avrai compiuti que' giorni, ti sdraierai di nuovo sul tuo lato destro, e porterai l'iniquità della casa di Giuda per quaranta giorni: t'impongo un giorno per ogni anno.
\par 7 Tu volgerai la tua faccia e il tuo braccio nudo verso l'assedio di Gerusalemme, e profeterai contro di lei.
\par 8 Ed ecco, io ti metterò addosso delle corde, e tu non potrai voltarti da un lato sull'altro, finché tu non abbia compiuti i giorni del tuo assedio.
\par 9 Prenditi anche del frumento, dell'orzo, delle fave, delle lenticchie, del miglio, del farro, mettili in un vaso, fattene del pane durante tutto il tempo che starai sdraiato sul tuo lato; ne mangerai per trecentonovanta giorni.
\par 10 Il cibo che mangerai sarà del peso di venti sicli per giorno; lo mangerai di tempo in tempo.
\par 11 Berrai pure dell'acqua a misura: la sesta parte d'un hin; la berrai di tempo in tempo.
\par 12 Mangerai delle focacce d'orzo, che cuocerai in loro presenza con escrementi d'uomo'.
\par 13 E l'Eterno disse: 'Così i figliuoli d'Israele mangeranno il loro pane contaminato, fra le nazioni dove io li caccerò'.
\par 14 Allora io dissi: 'Ahimè, Signore, Eterno, ecco, l'anima mia non è stata contaminata; dalla mia fanciullezza a ora, non ho mai mangiato carne di bestia morta da sé o sbranata, e non m'è mai entrata in bocca alcuna carne infetta'.
\par 15 Ed egli mi disse: 'Guarda io ti do dello sterco bovino, invece d'escrementi d'uomo; sopra quello cuocerai il tuo pane!'
\par 16 Poi mi disse: 'Figliuol d'uomo, io farò mancar del tutto il sostegno del pane a Gerusalemme; essi mangeranno il pane a peso e con angoscia e berranno l'acqua a misura e con costernazione,
\par 17 perché mancheranno di pane e d'acqua; e saranno costernati tutti quanti, e si struggeranno a motivo della loro iniquità.

\chapter{5}

\par 1 E tu, figliuol d'uomo, prenditi un ferro tagliente, prenditi un rasoio da barbiere, e fattelo passare sul capo e sulla barba; poi prenditi una bilancia da pesare, e dividi i peli che avrai tagliati.
\par 2 Bruciane una terza parte nel fuoco in mezzo alla città, quando i giorni dell'assedio saranno compiuti; poi prendine un'altra terza parte, e percuotila con la spada attorno alla città; e disperdi al vento l'ultima terza parte, dietro alla quale io sguainerò la spada.
\par 3 E di questa prendi una piccola quantità, e légala nei lembi della tua veste;
\par 4 e di questa prendi ancora una parte, gettala nel fuoco, e bruciala nel fuoco; di là uscirà un fuoco contro tutta la casa d'Israele.
\par 5 Così parla il Signore, l'Eterno: Ecco Gerusalemme! Io l'avevo posta in mezzo alle nazioni e agli altri paesi che la circondavano;
\par 6 ed ella, per darsi all'empietà, s'è ribellata alle mie leggi, più delle nazioni, e alle mie prescrizioni più de' paesi che la circondano; poiché ha sprezzato le mie leggi; e non ha camminato seguendo le mie prescrizioni.
\par 7 Perciò, così parla il Signore, l'Eterno: Poiché voi siete stati più insubordinati delle nazioni che vi circondano, in quanto non avete camminato seguendo le mie prescrizioni e non avete messo ad effetto le mie leggi e non avete neppur agito seguendo le leggi delle nazioni che vi circondano,
\par 8 così parla il Signore, l'Eterno: Eccomi, vengo io da te! ed eseguirò in mezzo a te i miei giudizi, nel cospetto delle nazioni;
\par 9 e farò a te quello che non ho mai fatto e che non farò mai più così, a motivo di tutte le tue abominazioni.
\par 10 Perciò, in mezzo a te, dei padri mangeranno i loro figliuoli, e dei figliuoli mangeranno i loro padri; ed io eseguirò su di te dei giudizi, e disperderò a tutti i venti quel che rimarrà di te.
\par 11 Perciò, com'è vero ch'io vivo, dice il Signore, l'Eterno, perché tu hai contaminato il mio santuario con tutte le tue infamie e con tutte le tue abominazioni, anch'io ti raderò, l'occhio mio non risparmierà nessuno e anch'io non avrò pietà.
\par 12 Una terza parte di te morrà di peste, e sarà consumata dalla fame in mezzo a te; una terza parte cadrà per la spada attorno a te, e ne disperderò a tutti i venti l'altra terza parte, e sguainerò contro ad essa la spada.
\par 13 Così si sfogherà la mia ira, e io soddisfarò su loro il mio furore, e sarò pago; ed essi conosceranno che io, l'Eterno, ho parlato nella mia gelosia, quando avrò sfogato su loro il mio furore.
\par 14 E farò di te, sotto gli occhi di tutti i passanti, una desolazione, il vituperio delle nazioni che ti circondano.
\par 15 E il tuo obbrobrio e la tua ignominia saranno un ammaestramento e un oggetto di stupore per le nazioni che ti circondano, quand'io avrò eseguito su di te i miei giudizi con ira, con furore, con indignati castighi - son io l'Eterno, che parlo -
\par 16 quando avrò scoccato contro di loro i letali dardi della fame, apportatori di distruzione e che io tirerò per distruggervi, quando avrò aggravata su di voi la fame e vi avrò fatto venir meno il sostegno del pane,
\par 17 quando avrò mandato contro di voi la fame e le male bestie che ti priveranno de' figliuoli, quando la peste e il sangue saran passati per mezzo a te, e quando io avrò fatto venire su di te la spada. Io, l'Eterno, son quegli che parla!'

\chapter{6}

\par 1 La parola dell'Eterno mi fu rivolta in questi termini: 'Figliuol d'uomo,
\par 2 volgi la tua faccia verso i monti d'Israele, profetizza contro di loro, e di':
\par 3 O monti d'Israele, ascoltate la parola del Signore, dell'Eterno! Così parla il Signore, l'Eterno, ai monti ed ai colli, ai burroni ed alle valli: Eccomi, io fo venire su di voi la spada, e distruggerò i vostri alti luoghi.
\par 4 I vostri altari saranno desolati, le vostre colonne solari saranno infrante, e io farò cadere i vostri uccisi davanti ai vostri idoli.
\par 5 E metterò i cadaveri de' figliuoli d'Israele davanti ai loro idoli, e spargerò le vostre ossa attorno ai vostri altari.
\par 6 Dovunque abitate, le città saranno deserte e gli alti luoghi desolati, affinché i vostri altari siano deserti e desolati, i vostri idoli siano infranti e scompaiano, le vostre colonne solari siano abbattute, e tutte le vostre opere siano spazzate via.
\par 7 I morti cadranno in mezzo a voi, e voi conoscerete che io sono l'Eterno.
\par 8 Nondimeno, io vi lascerò un residuo; poiché avrete alcuni scampati dalla spada fra le nazioni, quando sarete dispersi in vari paesi.
\par 9 E i vostri scampati si ricorderanno di me fra le genti dove saranno stati menati in cattività, poiché io spezzerò il loro cuore adultero che s'è stornato da me, e farò piangere i loro occhi che han commesso adulterio coi loro idoli; e avranno disgusto di loro stessi, per i mali che hanno commessi con tutte le loro abominazioni.
\par 10 E conosceranno che io sono l'Eterno, e che non invano li ho minacciati di far loro questo male.
\par 11 Così parla il Signore, l'Eterno: Batti le mani, batti del piede, e di': Ahimè! a motivo di tutte le scellerate abominazioni della casa d'Israele, che cadrà per la spada, per la fame, per la peste.
\par 12 Chi sarà lontano morirà di peste; chi sarà vicino cadrà per la spada; e chi sarà rimasto e sarà assediato, perirà di fame; e io sfogherò così il mio furore su di loro.
\par 13 E voi conoscerete che io sono l'Eterno, quando i loro morti saranno in mezzo ai loro idoli, attorno ai loro altari, sopra ogni alto colle, su tutte le vette dei monti, sotto ogni albero verdeggiante, sotto ogni querce dal folto fogliame, là dove essi offrivano profumi d'odor soave a tutti i loro idoli.
\par 14 E io stenderò su di loro la mia mano, e renderò il paese più solitario e desolato del deserto di Dibla, dovunque essi abitano; e conosceranno che io sono l'Eterno'.

\chapter{7}

\par 1 E la parola dell'Eterno mi fu rivolta in questi termini:
\par 2 'E tu, figliuol d'uomo, così parla il Signore, l'Eterno, riguardo al paese d'Israele: La fine! la fine viene sulle quattro estremità del paese!
\par 3 Ora ti sovrasta la fine, e io manderò contro di te la mia ira, ti giudicherò secondo la tua condotta, e ti farò ricadere addosso tutte le tue abominazioni.
\par 4 E l'occhio mio non ti risparmierà, io sarò senza pietà, ti farò ricadere addosso la tua condotta e le tue abominazioni saranno in mezzo a te; e voi conoscerete che io sono l'Eterno.
\par 5 Così parla il Signore, l'Eterno: Una calamità! ecco viene una calamità!
\par 6 La fine viene! viene la fine! Ella si desta per te! ecco ella viene!
\par 7 Vien la tua volta, o abitante del paese! Il tempo viene, il giorno s'avvicina: giorno di tumulto, e non di grida di gioia su per i monti.
\par 8 Ora, in breve, io spanderò su di te il mio furore sfogherò su di te la mia ira, ti giudicherò secondo la tua condotta, e ti farò ricadere addosso tutte le tue abominazioni.
\par 9 E l'occhio mio non ti risparmierà, io non avrò pietà, ti farò ricadere addosso la tua condotta, le tue abominazioni saranno in mezzo a te, e voi conoscerete che io, l'Eterno, son quegli che colpisce.
\par 10 Ecco il giorno! ecco ei viene! giunge la tua volta! La verga è fiorita! l'orgoglio è sbocciato!
\par 11 La violenza s'eleva e divien la verga dell'empietà; nulla più riman d'essi, della loro folla tumultuosa, del loro fracasso, nulla della loro magnificenza!
\par 12 Giunge il tempo, il giorno s'avvicina! Chi compra non si rallegri, chi vende non si dolga, perché un'ira ardente sovrasta a tutta la loro moltitudine.
\par 13 Poiché chi vende non tornerà in possesso di ciò che avrà venduto, anche se fosse tuttora in vita; poiché la visione contro tutta la loro moltitudine non sarà revocata, e nessuno potrà col suo peccato mantenere la propria vita.
\par 14 Suona la tromba, tutto è pronto, ma nessuno va alla battaglia; poiché l'ardore della mia ira sovrasta a tutta la loro moltitudine.
\par 15 Di fuori, la spada; di dentro, la peste e la fame! Chi è nei campi morrà per la spada: chi è in città sarà divorato dalla fame e dalla peste.
\par 16 E quelli di loro che riusciranno a scampare staranno su per i monti come le colombe delle valli, tutti quanti gemendo, ognuno per la propria iniquità.
\par 17 Tutte le mani diverranno fiacche, tutte le ginocchia si scioglieranno in acqua.
\par 18 E si cingeranno di sacchi, e lo spavento sarà la loro coperta; la vergogna sarà su tutti i volti, e avran tutti il capo rasato.
\par 19 Getteranno il loro argento per le strade, e il loro oro sarà per essi una immondezza; il loro argento e il loro oro non li potranno salvare nel giorno del furore dell'Eterno; non potranno saziare la loro fame, né empir loro le viscere, perché furon quelli la pietra d'intoppo per cui caddero nella loro iniquità.
\par 20 La bellezza dei loro ornamenti era per loro fonte d'orgoglio; e ne han fatto delle immagini delle loro abominazioni, delle loro divinità esecrande; perciò io farò che siano per essi una cosa immonda
\par 21 e abbandonerò tutto come preda in man degli stranieri e come bottino in man degli empi della terra, che lo profaneranno.
\par 22 E stornerò la mia faccia da loro; e i nemici profaneranno il mio intimo santuario; de' furibondi entreranno in Gerusalemme, e la profaneranno.
\par 23 Prepara le catene! poiché questo paese è pieno di delitti di sangue, e questa città è piena di violenza.
\par 24 E io farò venire le più malvage delle nazioni, che s'impossesseranno delle loro case: farò venir meno la superbia de' potenti, e i loro santuari saran profanati.
\par 25 Vien la ruina! Essi cercheranno la pace, ma non ve ne sarà alcuna.
\par 26 Verrà calamità su calamità, allarme sopra allarme; essi chiederanno delle visioni al profeta e la legge mancherà ai sacerdoti, il consiglio agli anziani.
\par 27 Il re farà cordoglio, il principe si rivestirà di desolazione, e le mani del popolo del paese tremeranno di spavento. Io li tratterò secondo la loro condotta, e li giudicherò secondo che meritano: e conosceranno che io sono l'Eterno'.

\chapter{8}

\par 1 E il sesto anno, il quinto giorno del sesto mese, avvenne che, come io stavo seduto in casa mia e gli anziani di Giuda eran seduti in mia presenza, la mano del Signore, dell'Eterno, cadde quivi su me.
\par 2 Io guardai, ed ecco una figura d'uomo, che aveva l'aspetto del fuoco; dai fianchi in giù pareva di fuoco; e dai fianchi in su aveva un aspetto risplendente, come di terso rame.
\par 3 Egli stese una forma di mano, e mi prese per una ciocca de' miei capelli; e lo spirito mi sollevò fra terra e cielo, e mi trasportò in visioni divine a Gerusalemme, all'ingresso della porta interna che guarda verso il settentrione, dov'era posto l'idolo della gelosia, che eccita a gelosia.
\par 4 Ed ecco che quivi era la gloria dell'Iddio d'Israele, come nella visione che avevo avuta nella valle.
\par 5 Ed egli mi disse: 'Figliuol d'uomo, alza ora gli occhi verso il settentrione'. Ed io alzai gli occhi verso il settentrione, ed ecco che al settentrione della porta dell'altare, all'ingresso, stava quell'idolo della gelosia.
\par 6 Ed egli mi disse: 'Figliuol d'uomo, vedi tu quello che costoro fanno? le grandi abominazioni che la casa d'Israele commette qui, perché io m'allontani dal mio santuario? Ma tu vedrai ancora altre più grandi abominazioni'.
\par 7 Ed egli mi condusse all'ingresso del cortile. Io guardai, ed ecco un buco nel muro.
\par 8 Allora egli mi disse: 'Figliuol d'uomo, adesso fora il muro'. E quand'io ebbi forato il muro, ecco una porta.
\par 9 Ed egli mi disse: 'Entra, e guarda le scellerate abominazioni che costoro commettono qui'.
\par 10 Io entrai, e guardai: ed ecco ogni sorta di figure di rettili e di bestie abominevoli, e tutti gl'idoli della casa d'Israele dipinti sul muro attorno attorno;
\par 11 e settanta fra gli anziani della casa d'Israele, in mezzo ai quali era Jaazania, figliuolo di Shafan, stavano in piè davanti a quelli, avendo ciascuno un turibolo in mano, dal quale saliva il profumo d'una nuvola d'incenso.
\par 12 Ed egli mi disse: 'Figliuol d'uomo, hai tu visto quello che gli anziani della casa d'Israele fanno nelle tenebre, ciascuno nelle camere riservate alle sue immagini? poiché dicono: - L'Eterno non ci vede, l'Eterno ha abbandonato il paese'.
\par 13 Poi mi disse: 'Tu vedrai ancora altre più grandi abominazioni che costoro commettono'.
\par 14 E mi menò all'ingresso della porta della casa dell'Eterno, che è verso il settentrione; ed ecco quivi sedevano delle donne che piangevano Tammuz.
\par 15 Ed egli mi disse: 'Hai tu visto figliuol d'uomo? Tu vedrai ancora delle abominazioni più grandi di queste'.
\par 16 E mi menò nel cortile della casa dell'Eterno; ed ecco, all'ingresso del tempio dell'Eterno, fra il portico e l'altare, circa venticinque uomini che voltavano le spalle alla casa dell'Eterno, e la faccia verso l'oriente; e si prostravano verso l'oriente, davanti al sole.
\par 17 Ed egli mi disse: 'Hai visto, figliuol d'uomo? È egli poca cosa per la casa di Giuda di commettere le abominazioni che commette qui, perché abbia anche a riempire il paese di violenza, e a tornar sempre a provocarmi ad ira? Ed ecco che s'accostano il ramo al naso.
\par 18 E anch'io agirò con furore; l'occhio mio non li risparmierà, e io non avrò pietà; e per quanto gridino ad alta voce ai miei orecchi, io non darò loro ascolto'.

\chapter{9}

\par 1 Poi gridò ad alta voce ai miei orecchi, dicendo: 'Fate accostare quelli che debbon punire la città, e ciascuno abbia in mano la sua arma di distruzione'.
\par 2 Ed ecco venire dal lato della porta superiore che guarda verso settentrione sei uomini, ognun dei quali aveva in mano la sua arma di distruzione; e in mezzo a loro stava un uomo vestito di lino, che aveva un corno da scrivano alla cintura; e vennero a mettersi di fianco all'altare di rame.
\par 3 E la gloria dell'Iddio d'Israele s'alzò di sul cherubino sul quale stava, e andò verso la soglia della casa; e l'Eterno chiamò l'uomo vestito di lino, che aveva il corno da scrivano alla cintura, e gli disse:
\par 4 'Passa in mezzo alla città, in mezzo a Gerusalemme, e fa' un segno sulla fronte degli uomini che sospirano e gemono per tutte le abominazioni che si commettono in mezzo di lei'.
\par 5 E agli altri disse, in modo ch'io intesi: 'Passate per la città dietro a lui, e colpite; il vostro occhio non risparmi alcuno, e siate senza pietà;
\par 6 uccidete, sterminate vecchi, giovani, vergini, bambini e donne, ma non vi avvicinate ad alcuno che porti il segno; e cominciate dal mio santuario'. Ed essi cominciarono da quegli anziani che stavano davanti alla casa.
\par 7 Poi egli disse loro: 'Contaminate la casa ed empite di morti i cortili! Uscite!' E quelli uscirono, e andarono colpendo per la città.
\par 8 E com'essi colpivano ed io ero rimasto solo, caddi sulla mia faccia, e gridai: 'Ahimè, Signore, Eterno, distruggerai tu tutto ciò che rimane d'Israele, riversando il tuo furore su Gerusalemme?'
\par 9 Ed egli mi rispose: 'L'iniquità della casa d'Israele e di Giuda è oltremodo grande; il paese è pieno di sangue, e la città è piena di prevaricazioni; poiché dicono: - L'Eterno ha abbandonato il paese, l'Eterno non vede nulla. -
\par 10 Perciò, anche l'occhio mio non risparmierà nessuno, io non avrò pietà, e farò ricadere sul loro capo la loro condotta'.
\par 11 Ed ecco, l'uomo vestito di lino, che aveva il corno dello scrivano alla cintura, venne a fare il suo rapporto, dicendo: 'Ho fatto come tu m'hai comandato'.

\chapter{10}

\par 1 Io guardai, ed ecco, sulla distesa sopra il capo dei cherubini, v'era come una pietra di zaffiro; si vedeva come una specie di trono che stava sopra loro.
\par 2 E l'Eterno parlò all'uomo vestito di lino, e disse: 'Va' fra le ruote sotto i cherubini, empiti le mani di carboni ardenti tolti di fra i cherubini, e spargili sulla città'. Ed egli v'andò in mia presenza.
\par 3 Or i cherubini stavano al lato destro della casa, quando l'uomo entrò là; e la nuvola riempì il cortile interno.
\par 4 E la gloria dell'Eterno s'alzò di sui cherubini, movendo verso la soglia della casa; e la casa fu ripiena della nuvola; e il cortile fu ripieno dello splendore della gloria dell'Eterno.
\par 5 E il rumore delle ali dei cherubini s'udì fino al cortile esterno, simile alla voce dell'Iddio onnipotente quand'egli parla.
\par 6 E quando l'Eterno ebbe dato all'uomo vestito di lino l'ordine di prender del fuoco di fra le ruote che son tra i cherubini, quegli venne a fermarsi presso una delle ruote.
\par 7 E uno dei cherubini stese la mano fra gli altri cherubini verso il fuoco ch'era fra i cherubini, ne prese e lo mise nelle mani dell'uomo vestito di lino, che lo ricevette, ed uscì.
\par 8 Or ai cherubini si vedeva una forma di mano d'uomo sotto alle ali.
\par 9 E io guardai, ed ecco quattro ruote presso ai cherubini, una ruota presso ogni cherubino; e le ruote avevano l'aspetto di una pietra di crisolito.
\par 10 E, a vederle, tutte e quattro avevano una medesima forma, come se una ruota passasse attraverso all'altra.
\par 11 Quando si movevano, si movevano dai loro quattro lati; e, movendosi non si voltavano, ma seguivano la direzione del luogo verso il quale guardava il capo, e, andando, non si voltavano.
\par 12 E tutto il corpo dei cherubini, i loro dossi, le loro mani, le loro ali, come pure le ruote di tutti e quattro, eran pieni d'occhi tutto attorno.
\par 13 E udii che le ruote eran chiamate 'Il Turbine'.
\par 14 E ogni cherubino aveva quattro facce: la prima faccia era una faccia di cherubino; la seconda faccia, una faccia d'uomo; la terza, una faccia di leone; la quarta, una faccia d'aquila.
\par 15 E i cherubini s'alzarono. Erano gli stessi esseri viventi, che avevo veduti presso il fiume Kebar.
\par 16 E quando i cherubini si movevano, anche le ruote si movevano allato a loro; e quando i cherubini spiegavano le ali per alzarsi da terra, anche le ruote non deviavano da presso a loro.
\par 17 Quando quelli si fermavano, anche queste si fermavano; quando quelli s'innalzavano anche queste s'innalzavano con loro, perché lo spirito degli esseri viventi era in esse.
\par 18 E la gloria dell'Eterno si partì di sulla soglia della casa, e si fermò sui cherubini.
\par 19 E i cherubini spiegarono le loro ali e s'innalzarono su dalla terra; e io li vidi partire, con le ruote allato a loro. Si fermarono all'ingresso della porta orientale della casa dell'Eterno; e la gloria dell'Iddio d'Israele stava sopra di loro, su in alto.
\par 20 Erano gli stessi esseri viventi, che avevo veduti sotto l'Iddio d'Israele presso il fiume Kebar; e riconobbi che erano cherubini.
\par 21 Ognun d'essi avea quattro facce, ognuno quattro ali; e sotto le loro ali appariva la forma di mani d'uomo.
\par 22 E quanto all'aspetto delle loro facce, eran le facce che avevo vedute presso il fiume Kebar; erano gli stessi aspetti, i medesimi cherubini. Ognuno andava dritto davanti a sé.

\chapter{11}

\par 1 Poi lo spirito mi levò in alto, e mi menò alla porta orientale della casa dell'Eterno che guarda verso levante; ed ecco, all'ingresso della porta, venticinque uomini; e in mezzo ad essi vidi Jaazania, figliuolo di Azzur, e Pelatia, figliuolo di Benaia, capi del popolo.
\par 2 E l'Eterno mi disse: 'Figliuol d'uomo, questi sono gli uomini che meditano l'iniquità, e danno cattivi consigli in questa città.
\par 3 Essi dicono: - Il tempo non è così vicino! Edifichiamo pur case! Questa città è la pentola e noi siamo la carne. -
\par 4 Perciò profetizza contro di loro, profetizza, figliuol d'uomo!'
\par 5 E lo spirito dell'Eterno cadde su di me, e mi disse: 'Di': Così parla l'Eterno: Voi parlate a quel modo, o casa d'Israele, e io conosco le cose che vi passan per la mente.
\par 6 Voi avete moltiplicato i vostri omicidi in questa città, e ne avete riempite d'uccisi le strade.
\par 7 Perciò così parla il Signore, l'Eterno: I vostri morti che avete stesi in mezzo a questa città sono la carne, e la città è la pentola; ma voi ne sarete tratti fuori.
\par 8 Voi avete paura della spada, e io farò venire su di voi la spada, dice il Signore, l'Eterno.
\par 9 Io vi trarrò fuori dalla città, e vi darò in man di stranieri; ed eseguirò su di voi i miei giudizi.
\par 10 Voi cadrete per la spada, io vi giudicherò sulle frontiere d'Israele, e voi conoscerete che io sono l'Eterno.
\par 11 Questa città non sarà per voi una pentola, e voi non sarete in mezzo a lei la carne; io vi giudicherò sulle frontiere d'Israele;
\par 12 e voi conoscerete che io sono l'Eterno, del quale non avete seguito le prescrizioni né messe in pratica le leggi, ma avete agito secondo le leggi delle nazioni che vi circondano'.
\par 13 Or avvenne che, come io profetavo a Pelatia, figliuolo di Benaia, morì; e io mi gettai con la faccia a terra, e gridai ad alta voce: 'Ahimè, Signore, Eterno, farai tu una completa distruzione di quel che rimane d'Israele?'
\par 14 E la parola dell'Eterno mi fu rivolta in questi termini:
\par 15 'Figliuol d'uomo, i tuoi fratelli, gli uomini del tuo parentado e tutta quanta la casa d'Israele son quelli ai quali gli abitanti di Gerusalemme hanno detto: - Statevene lontani dall'Eterno! a noi è dato il possesso del paese. -
\par 16 Perciò di': Così parla il Signore, l'Eterno: Benché io li abbia allontanati fra le nazioni e li abbia dispersi per i paesi, io sarò per loro, per qualche tempo, un santuario nei paesi dove sono andati.
\par 17 Perciò di': Così parla il Signore, l'Eterno: Io vi raccoglierò di fra i popoli, vi radunerò dai paesi dove siete stati dispersi, e vi darò il suolo d'Israele.
\par 18 E quelli vi verranno, e ne torranno via tutte le cose esecrande e tutte le abominazioni.
\par 19 E io darò loro un medesimo cuore, metterò dentro di loro un nuovo spirito, torrò via dalla loro carne il cuore di pietra, e darò loro un cuor di carne,
\par 20 perché camminino secondo le mie prescrizioni, e osservino le mie leggi e le mettano in pratica; ed essi saranno il mio popolo, e io sarò il loro Dio.
\par 21 Ma quanto a quelli il cui cuore segue l'affetto che hanno alle loro cose esecrande e alle loro abominazioni, io farò ricadere sul loro capo la loro condotta, dice il Signore, l'Eterno'.
\par 22 Poi i cherubini spiegarono le loro ali, e le ruote si mossero allato a loro; e la gloria dell'Iddio d'Israele stava su loro, in alto.
\par 23 E la gloria dell'Eterno s'innalzò di sul mezzo della città, e si fermò sul monte ch'è ad oriente della città.
\par 24 E lo spirito mi trasse in alto, e mi menò in Caldea presso quelli ch'erano in cattività, in visione, mediante lo spirito di Dio; e la visione che avevo avuta scomparve d'innanzi a me;
\par 25 e io riferii a quelli ch'erano in cattività tutte le parole che l'Eterno m'aveva dette in visione.

\chapter{12}

\par 1 La parola dell'Eterno mi fu ancora rivolta in questi termini:
\par 2 'Figliuol d'uomo, tu abiti in mezzo a una casa ribelle che ha occhi per vedere e non vede, orecchi per udire e non ode perché è una casa ribelle.
\par 3 Perciò, figliuol d'uomo, prepàrati un bagaglio da esiliato, e parti di giorno in loro presenza, come se tu andassi in esilio; parti, in loro presenza, dal luogo dove tu sei, per un altro luogo; forse vi porranno mente; perché sono una casa ribelle.
\par 4 Metti dunque fuori, di giorno, in loro presenza, il tuo bagaglio, simile a quello di chi va in esilio; poi la sera, esci tu stesso, in loro presenza, come fanno quelli che sen vanno esuli.
\par 5 Fa', in loro presenza, un foro nel muro, e porta fuori per esso il tuo bagaglio.
\par 6 Portalo sulle spalle, in loro presenza; portalo fuori quando farà buio; copriti la faccia per non veder la terra; perché io faccio di te un segno per la casa d'Israele'.
\par 7 E io feci così come m'era stato comandato; trassi fuori di giorno il mio bagaglio, bagaglio di esiliato, e sulla sera feci con le mie mani un foro nel muro; e quando fu buio portai fuori il bagaglio, e me lo misi sulle spalle in loro presenza.
\par 8 E la mattina la parola dell'Eterno mi fu rivolta in questi termini:
\par 9 'Figliuol d'uomo, la casa d'Israele, questa casa ribelle, non t'ha ella detto: - Che fai? -
\par 10 Di' loro: Così parla il Signore, l'Eterno: Quest'oracolo concerne il principe ch'è in Gerusalemme, e tutta la casa d'Israele di cui essi fan parte.
\par 11 Di': Io sono per voi un segno: come ho fatto io, così sarà fatto a loro: essi andranno in esilio, in cattività.
\par 12 Il principe ch'è in mezzo a loro porterà il suo bagaglio sulle spalle quando farà buio, e partirà; si farà un foro nel muro, per farlo uscire di lì; egli si coprirà la faccia per non veder coi suoi occhi la terra;
\par 13 e io stenderò su di lui la mia rete, ed egli sarà preso nel mio laccio; lo menerò a Babilonia, nella terra dei Caldei, ma egli non la vedrà, e quivi morrà.
\par 14 E io disperderò a tutti i venti quelli che lo circondano per aiutarlo, e tutti i suoi eserciti, e sguainerò la spada dietro a loro.
\par 15 Ed essi conosceranno che io sono l'Eterno quando li avrò sparsi fra le nazioni e dispersi nei paesi stranieri.
\par 16 Ma lascerò di loro alcuni pochi uomini scampati dalla spada, dalla fame e dalla peste, affinché narrino tutte le loro abominazioni fra le nazioni dove saran giunti; e conosceranno che io sono l'Eterno'.
\par 17 La parola dell'Eterno mi fu ancora rivolta in questi termini:
\par 18 'Figliuol d'uomo, mangia il tuo pane con tremore, e bevi la tua acqua con trepidazione ed ansietà;
\par 19 e di' al popolo del paese: Così parla il Signore, l'Eterno, riguardo agli abitanti di Gerusalemme nella terra d'Israele: Mangeranno il loro pane con ansietà e berranno la loro acqua con desolazione, poiché il loro paese sarà desolato, spogliato di tutto ciò che contiene, a motivo della violenza di tutti quelli che l'abitano.
\par 20 Le città abitate saranno ridotte in rovine, e il paese sarà desolato; e voi conoscerete che io sono l'Eterno'.
\par 21 E la parola dell'Eterno mi fu rivolta in questi termini:
\par 22 'Figliuol d'uomo: Che proverbio è questo che voi ripetete nel paese d'Israele quando dite - I giorni si prolungano e ogni visione è venuta meno?
\par 23 - Perciò di' loro: Così parla il Signore, l'Eterno: Io farò cessare questo proverbio, e non lo si ripeterà più in Israele; di' loro, invece: I giorni s'avvicinano, s'avvicina l'avveramento d'ogni visione;
\par 24 poiché nessuna visione sarà più vana, né vi sarà più divinazione ingannevole in mezzo alla casa d'Israele.
\par 25 Poiché io sono l'Eterno; qualunque sia la parola che avrò detta, ella sarà messa ad effetto; non sarà più differita; poiché nei vostri giorni, o casa ribelle, io pronunzierò una parola, e la metterò ad effetto, dice il Signore, l'Eterno'.
\par 26 La parola dell'Eterno mi fu ancora rivolta in questi termini:
\par 27 'Figliuol d'uomo, ecco, quelli della casa d'Israele dicono: - La visione che costui contempla concerne lunghi giorni avvenire, ed egli profetizza per dei tempi lontani. -
\par 28 Perciò di' loro: Così parla il Signore, l'Eterno: Nessuna delle mie parole sarà più differita; la parola che avrò pronunziata sarà messa ad effetto, dice il Signore, l'Eterno'.

\chapter{13}

\par 1 La parola dell'Eterno mi fu rivolta in questi termini:
\par 2 'Figliuol d'uomo, profetizza contro i profeti d'Israele che profetano, e di' a quelli che profetano di loro senno: Ascoltate la parola dell'Eterno.
\par 3 Così parla il Signore, l'Eterno: Guai ai profeti stolti, che seguono il loro proprio spirito, e parlano di cose che non hanno vedute!
\par 4 O Israele, i tuoi profeti sono stati come volpi tra le ruine!
\par 5 Voi non siete saliti alle brecce e non avete costruito riparo attorno alla casa d'Israele, per poter resistere alla battaglia nel giorno dell'Eterno.
\par 6 Hanno delle visioni vane, delle divinazioni menzognere, costoro che dicono: - L'Eterno ha detto! - mentre l'Eterno non li ha mandati; e sperano che la loro parola s'adempirà!
\par 7 Non avete voi delle visioni vane e non pronunziate voi divinazioni menzognere, quando dite: - L'Eterno ha detto - e io non ho parlato?
\par 8 Perciò, così parla il Signore, l'Eterno: Poiché profferite cose vane e avete visioni menzognere, eccomi contro di voi, dice il Signore, l'Eterno.
\par 9 La mia mano sarà contro i profeti dalle visioni vane e dalle divinazioni menzognere; essi non saranno più nel consiglio del mio popolo, non saranno più iscritti nel registro della casa d'Israele, e non entreranno nel paese d'Israele; e voi conoscerete che io sono il Signore, l'Eterno.
\par 10 Giacché, sì, giacché sviano il mio popolo, dicendo: Pace! quando non v'è alcuna pace, e giacché quando il popolo edifica un muro, ecco che costoro lo intònacano di malta che non regge,
\par 11 di' a quelli che lo intònacano di malta che non regge, ch'esso cadrà; verrà una pioggia scrosciante, e voi, o pietre di grandine, cadrete; e si scatenerà un vento tempestoso;
\par 12 ed ecco, quando il muro cadrà, non vi si dirà egli: E dov'è la malta con cui l'avevate intonacato?
\par 13 Perciò così parla il Signore, l'Eterno: Io, nel mio furore, farò scatenare un vento tempestoso, e, nella mia ira, farò cadere una pioggia scrosciante, e, nella mia indignazione, delle pietre di grandine sterminatrice.
\par 14 E demolirò il muro che voi avete intonacato con malta che non regge, lo rovescerò a terra, e i suoi fondamenti saranno messi allo scoperto; ed esso cadrà, e voi sarete distrutti in mezzo alle sue ruine, e conoscerete che io sono l'Eterno.
\par 15 Così sfogherò il mio furore su quel muro, e su quelli che l'hanno intonacato di malta che non regge; e vi dirò: Il muro non è più, e quelli che lo intonacavano non sono più:
\par 16 cioè i profeti d'Israele, che profetano riguardo a Gerusalemme e hanno per lei delle visioni di pace, benché non vi sia pace alcuna, dice il Signore, l'Eterno.
\par 17 E tu, figliuol d'uomo, volgi la faccia verso le figliuole del tuo popolo che profetano di loro senno, e profetizza contro di loro,
\par 18 e di': Così parla il Signore, l'Eterno: Guai alle donne che cuciono de' cuscini per tutti i gomiti, e fanno de' guanciali per le teste d'ogni altezza, per prendere le anime al laccio! Vorreste voi prendere al laccio le anime del mio popolo e salvare le vostre proprie anime?
\par 19 Voi mi profanate fra il mio popolo per delle manate d'orzo e per de' pezzi di pane, facendo morire anime che non devono morire, e facendo vivere anime che non devono vivere, mentendo al mio popolo, che dà ascolto alle menzogne.
\par 20 Perciò, così parla il Signore, l'Eterno: Eccomi ai vostri cuscini, coi quali voi prendete le anime al laccio, come uccelli! io ve li strapperò dalle braccia, e lascerò andare le anime: le anime, che voi prendete al laccio come gli uccelli.
\par 21 Strapperò pure i vostri guanciali, e libererò il mio popolo dalle vostre mani; ed egli non sarà più nelle vostre mani per cadere nei lacci, e voi saprete che io sono l'Eterno.
\par 22 Poiché avete contristato il cuore del giusto con delle menzogne, quand'io non lo contristavo, e avete fortificate le mani dell'empio perché non si convertisse dalla sua via malvagia per ottenere la vita,
\par 23 voi non avrete più visioni vane e non praticherete più la divinazione; e io libererò il mio popolo dalle vostre mani e voi conoscerete che io sono l'Eterno'.

\chapter{14}

\par 1 Or vennero a me alcuni degli anziani d'Israele, e si sedettero davanti a me.
\par 2 E la parola dell'Eterno mi fu rivolta in questi termini:
\par 3 'Figliuol d'uomo, questi uomini hanno innalzato i loro idoli nel loro cuore, e si son messi davanti l'intoppo che li fa cadere nella loro iniquità; come potrei io esser consultato da costoro?
\par 4 Perciò parla e di' loro: Così dice il Signore, l'Eterno: Chiunque della casa d'Israele innalza i suoi idoli nel suo cuore e pone davanti a sé l'intoppo che lo fa cadere nella sua iniquità, e poi viene al profeta, io, l'Eterno, gli risponderò come si merita per la moltitudine de' suoi idoli,
\par 5 affin di prendere per il loro cuore quelli della casa d'Israele che si sono alienati da me tutti quanti per i loro idoli.
\par 6 Perciò di' alla casa d'Israele: Così parla il Signore, l'Eterno: Tornate, ritraetevi dai vostri idoli, stornate le vostre facce da tutte le vostre abominazioni.
\par 7 Poiché, a chiunque della casa d'Israele o degli stranieri che soggiornano in Israele si separa da me, innalza i suoi idoli nel suo cuore e pone davanti a sé l'intoppo che lo fa cadere nella sua iniquità e poi viene al profeta per consultarmi per suo mezzo, risponderò io, l'Eterno, da me stesso.
\par 8 Io volgerò la mia faccia contro a quell'uomo, ne farò un segno e un proverbio, e lo sterminerò di mezzo al mio popolo; e voi conoscerete che io sono l'Eterno.
\par 9 E se il profeta si lascia sedurre e dice qualche parola, io, l'Eterno, sono quegli che avrò sedotto quel profeta; e stenderò la mia mano contro di lui, e lo distruggerò di mezzo al mio popolo d'Israele.
\par 10 E ambedue porteranno la pena della loro iniquità: la pena del profeta sarà pari alla pena di colui che lo consulta,
\par 11 affinché quelli della casa d'Israele non vadano più errando lungi da me, e non si contaminino più con tutte le loro trasgressioni, e siano invece mio popolo, e io sia il loro Dio, dice il Signore, l'Eterno'.
\par 12 La parola dell'Eterno mi fu ancora rivolta in questi termini:
\par 13 'Figliuol d'uomo, se un paese peccasse contro di me commettendo qualche prevaricazione, e io stendessi la mia mano contro di lui, e gli spezzassi il sostegno del pane, e gli mandassi contro la fame, e ne sterminassi uomini e bestie,
\par 14 e in mezzo ad esso si trovassero questi tre uomini: Noè, Daniele e Giobbe, questi non salverebbero che le loro persone, per la loro giustizia, dice il Signore, l'Eterno.
\par 15 Se io facessi passare per quel paese delle male bestie che lo spopolassero, sì ch'esso rimanesse un deserto dove nessuno passasse più a motivo di quelle bestie,
\par 16 se in mezzo ad esso si trovassero quei tre uomini, com'è vero ch'io vivo, dice il Signore, l'Eterno, essi non salverebbero né figliuoli né figliuole; essi soltanto sarebbero salvati, ma il paese rimarrebbe desolato.
\par 17 O se io facessi venire la spada contro quel paese, e dicessi: - Passi la spada per il paese! - in guisa che ne sterminasse uomini e bestie,
\par 18 se in mezzo ad esso si trovassero quei tre uomini, com'è vero ch'io vivo, dice il Signore, l'Eterno, essi non salverebbero né figliuoli né figliuole, ma essi soltanto sarebbero salvati.
\par 19 O se contro quel paese mandassi la peste, e riversassi su d'esso il mio furore fino al sangue, per sterminare uomini e bestie,
\par 20 se in mezzo ad esso si trovassero Noè, Daniele e Giobbe, com'è vero ch'io vivo, dice il Signore, l'Eterno, essi non salverebbero né figliuoli né figliuole; non salverebbero che le loro persone, per la loro giustizia.
\par 21 Poiché così parla il Signore, l'Eterno: Non altrimenti avverrà quando manderò contro Gerusalemme i miei quattro tremendi giudizi: la spada, la fame, le male bestie e la peste, per sterminare uomini e bestie.
\par 22 Ma ecco, ne scamperà un residuo, de' figliuoli e delle figliuole, che saran menati fuori, che giungeranno a voi, e di cui vedrete la condotta e le azioni; e allora vi consolerete del male che io faccio venire su Gerusalemme, di tutto quello che faccio venire su di lei.
\par 23 Essi vi consoleranno quando vedrete la loro condotta e le loro azioni, e riconoscerete che, non senza ragione, io faccio quello che faccio contro di lei, dice il Signore, l'Eterno'.

\chapter{15}

\par 1 E la parola dell'Eterno mi fu rivolta in questi termini:
\par 2 'Figliuol d'uomo, il legno della vite che cos'è egli più di qualunque altro legno? che cos'è il tralcio ch'è fra gli alberi della foresta?
\par 3 Se ne può egli prendere il legno per farne un qualunque lavoro? Si può egli trarne un cavicchio da appendervi un qualche oggetto?
\par 4 Ecco, esso è gettato nel fuoco, perché si consumi; il fuoco ne consuma i due capi, e il mezzo si carbonizza; è egli atto a farne qualcosa?
\par 5 Ecco, mentr'era intatto, non se ne poteva fare alcun lavoro; quanto meno se ne potrà fare qualche lavoro, quando il fuoco l'abbia consumato o carbonizzato!
\par 6 Perciò, così parla il Signore, l'Eterno: Com'è fra gli alberi della foresta il legno della vite che io destino al fuoco perché lo consumi, così farò degli abitanti di Gerusalemme.
\par 7 Io volgerò la mia faccia contro di loro; dal fuoco sono usciti, e il fuoco li consumerà; e riconoscerete che io sono l'Eterno, quando avrò vòlto la mia faccia contro di loro.
\par 8 E renderò il paese desolato, perché hanno agito in modo infedele, dice il Signore, l'Eterno'.

\chapter{16}

\par 1 La parola dell'Eterno mi fu ancora rivolta in questi termini:
\par 2 'Figliuol d'uomo, fa' conoscere a Gerusalemme le sue abominazioni,
\par 3 e di': Così parla il Signore, l'Eterno, a Gerusalemme: Per la tua origine e per la tua nascita sei del paese del Cananeo; tuo padre era un Amoreo, tua madre una Hittea.
\par 4 Quanto alla tua nascita, il giorno che nascesti l'ombelico non ti fu tagliato, non fosti lavata con acqua per nettarti, non fosti sfregata con sale, né fosti fasciata.
\par 5 Nessuno ebbe sguardi di pietà per te, per farti una sola di queste cose, avendo compassione di te; ma fosti gettata nell'aperta campagna, il giorno che nascesti, pel disprezzo che si aveva di te.
\par 6 E io ti passai accanto, vidi che ti dibattevi nel sangue, e ti dissi: - Vivi, tu che sei nel sangue! - E ti ripetei: - Vivi, tu che sei nel sangue!
\par 7 Io ti farò moltiplicare per miriadi, come il germe dei campi. - E tu ti sviluppasti, crescesti, giungesti al colmo della bellezza, il tuo seno si formò, la tua capigliatura crebbe abbondante, ma tu eri nuda e scoperta.
\par 8 Io ti passai accanto, ti guardai, ed ecco il tuo tempo era giunto: il tempo degli amori; io stesi su di te il lembo della mia veste, e copersi la tua nudità; ti feci un giuramento, fermai un patto con te, dice il Signore, l'Eterno, e tu fosti mia.
\par 9 Ti lavai con acqua, ti ripulii del sangue che avevi addosso, e ti unsi con olio.
\par 10 Ti misi delle vesti ricamate, de' calzari di pelle di tasso, ti cinsi il capo di lino fino, ti ricopersi di seta.
\par 11 Ti fornii d'ornamenti, ti misi de' braccialetti ai polsi, e una collana al collo.
\par 12 Ti misi un anello al naso, dei pendenti agli orecchi, e una magnifica corona in capo.
\par 13 Così fosti adorna d'oro e d'argento, e fosti vestita di lino fino, di seta e di ricami; e tu mangiasti fior di farina, miele e olio; diventasti sommamente bella, e giungesti fino a regnare.
\par 14 E la tua fama si sparse fra le nazioni, per la tua bellezza; poich'essa era perfetta, avendoti io coperta della mia magnificenza, dice il Signore, l'Eterno.
\par 15 Ma tu confidasti nella tua bellezza, e ti prostituisti in grazia della tua fama, e prodigasti le tue prostituzioni a ogni passante, a chi voleva.
\par 16 Tu prendesti delle tue vesti, ti facesti degli alti luoghi parati di vari colori, e quivi ti prostituisti: cose tali, che non ne avvennero mai, e non ne avverranno più.
\par 17 Prendesti pure i tuoi bei gioielli fatti del mio oro e del mio argento, che io t'avevo dati, te ne facesti delle immagini d'uomo, e ad esse ti prostituisti;
\par 18 e prendesti le tue vesti ricamate e ne ricopristi quelle immagini, dinanzi alle quali tu ponesti il mio olio e il mio profumo.
\par 19 Parimente il mio pane che t'avevo dato, il fior di farina, l'olio e il miele con cui ti nutrivo, tu li ponesti davanti a loro, come un profumo di soave odore. Questo si fece! dice il Signore, l'Eterno.
\par 20 Prendesti inoltre i tuoi figliuoli e le tue figliuole che mi avevi partoriti, e li offristi loro in sacrificio, perché li divorassero. Non bastavan esse le tue prostituzioni,
\par 21 perché tu avessi anche a scannare i miei figliuoli, e a darli loro facendoli passare per il fuoco?
\par 22 E in mezzo a tutte le tue abominazioni e alle tue prostituzioni, non ti sei ricordata de' giorni della tua giovinezza, quand'eri nuda, scoperta, e ti dibattevi nel sangue.
\par 23 Ora dopo tutta la tua malvagità - guai! guai a te! dice il Signore, l'Eterno, -
\par 24 ti sei costruita un bordello, e ti sei fatto un alto luogo in ogni piazza pubblica:
\par 25 hai costruito un alto luogo a ogni capo di strada, hai reso abominevole la tua bellezza, ti sei offerta ad ogni passante, ed hai moltiplicato le tue prostituzioni.
\par 26 Ti sei pure prostituita agli Egiziani, tuoi vicini dalle membra vigorose, e hai moltiplicato le tue prostituzioni per provocarmi ad ira.
\par 27 Perciò, ecco, io ho steso la mia mano contro di te, ho diminuito la provvisione che t'avevo fissata, e t'ho abbandonata in balìa delle figliuole dei Filistei, che t'odiano e hanno vergogna della tua condotta scellerata.
\par 28 Non sazia ancora, ti sei pure prostituita agli Assiri; ti sei prostituita a loro; e neppure allora sei stata sazia;
\par 29 e hai moltiplicato le tue prostituzioni col paese di Canaan fino in Caldea, e neppure con questo sei stata sazia.
\par 30 Com'è vile il tuo cuore, dice il Signore, l'Eterno, a ridurti a fare tutte queste cose, da sfacciata prostituta!
\par 31 Quando ti costruivi il bordello a ogni capo di strada e ti facevi gli alti luoghi in ogni piazza pubblica, tu non eri come una prostituta, giacché sprezzavi il salario,
\par 32 ma come una donna adultera, che riceve gli stranieri invece del suo marito.
\par 33 A tutte le prostitute si danno dei regali: ma tu hai fatto de' regali a tutti i tuoi amanti, e li hai sedotti con de' doni, perché venissero da te, da tutte le parti, per le tue prostituzioni.
\par 34 Con te, nelle tue prostituzioni, è avvenuto il contrario delle altre donne; giacché non eri tu la sollecitata; in quanto tu pagavi, invece d'esser pagata, facevi il contrario delle altre.
\par 35 Perciò, o prostituta, ascolta la parola dell'Eterno.
\par 36 Così parla il Signore, l'Eterno: Poiché il tuo danaro è stato dissipato e la tua nudità è stata scoperta nelle tue prostituzioni coi tuoi amanti, e a motivo di tutti i tuoi idoli abominevoli, e a cagione del sangue dei tuoi figliuoli che hai dato loro,
\par 37 ecco, io radunerò tutti i tuoi amanti ai quali ti sei resa gradita, e tutti quelli che hai amati insieme a quelli che hai odiati; li radunerò da tutte le parti contro di te, e scoprirò davanti a loro la tua nudità, ed essi vedranno tutta la tua nudità.
\par 38 Io ti giudicherò alla stregua delle donne che commettono adulterio e spandono il sangue, e farò che il tuo sangue sia sparso dal furore e dalla gelosia.
\par 39 E ti darò nelle loro mani, ed essi abbatteranno il tuo bordello, distruggeranno i tuoi alti luoghi, ti spoglieranno delle tue vesti, ti prenderanno i bei gioielli, e ti lasceranno nuda e scoperta;
\par 40 e faranno salire contro di te una folla, e ti lapideranno e ti trafiggeranno con le loro spade;
\par 41 daranno alle fiamme le tue case, faranno giustizia di te nel cospetto di molte donne, e io ti farò cessare dal far la prostituta, e tu non pagherai più nessuno.
\par 42 Così io sfogherò il mio furore su di te, e la mia gelosia si stornerà da te; m'acqueterò, e non sarò più adirato.
\par 43 Poiché tu non ti sei ricordata dei giorni della tua giovinezza e m'hai provocato ad ira con tutte queste cose, ecco, anch'io ti farò ricadere sul capo la tua condotta, dice il Signore, l'Eterno, e tu non aggiungerai altri delitti a tutte le tue abominazioni.
\par 44 Ecco, tutti quelli che usano proverbi faranno di te un proverbio, e diranno: - Quale la madre, tale la figlia.
\par 45 Tu sei figliuola di tua madre, ch'ebbe a sdegno il suo marito e i suoi figliuoli, e sei sorella delle tue sorelle, ch'ebbero a sdegno i loro mariti e i loro figliuoli. Vostra madre era una Hittea, e vostro padre un Amoreo.
\par 46 La tua sorella maggiore, che ti sta a sinistra, è Samaria, con le sue figliuole; e la tua sorella minore, che ti sta a destra, è Sodoma, con le sue figliuole.
\par 47 E tu, non soltanto hai camminato nelle loro vie e commesso le stesse loro abominazioni; era troppo poco; ma in tutte le tue vie ti sei corrotta più di loro.
\par 48 Com'è vero ch'io vivo, dice il Signore, l'Eterno, Sodoma, la tua sorella, e le sue figliuole, non hanno fatto quel che avete fatto tu e le figliuole tue.
\par 49 Ecco questa fu l'iniquità di Sodoma, tua sorella: lei e le sue figliuole vivevano nell'orgoglio, nell'abbondanza del pane, e nell'ozio indolente; ma non sostenevano la mano dell'afflitto e del povero.
\par 50 Erano altezzose, e commettevano abominazioni nel mio cospetto; perciò le feci sparire, quando vidi ciò.
\par 51 E Samaria non ha commesso la metà de' tuoi peccati; tu hai moltiplicato le tue abominazioni più che l'una e l'altra, e hai giustificato le tue sorelle, con tutte le abominazioni che hai commesse.
\par 52 Anche tu porta il vituperio che hai inflitto alle tue sorelle! Coi tuoi peccati tu ti sei resa più abominevole di loro, ed esse son più giuste di te; tu pure dunque, vergognati e porta il tuo vituperio, poiché tu hai giustificato le tue sorelle!
\par 53 Io farò tornare dalla cattività quelli che là si trovano di Sodoma e delle sue figliuole, quelli di Samaria e delle sue figliuole e quelli de' tuoi che sono in mezzo ad essi,
\par 54 affinché tu porti il tuo vituperio, che tu senta l'onta di tutto quello che hai fatto, e sii così loro di conforto.
\par 55 La tua sorella Sodoma e le sue figliuole torneranno nella loro condizione di prima, Samaria e le sue figliuole torneranno nella loro condizione di prima, e tu e le tue figliuole tornerete nella vostra condizione di prima.
\par 56 Sodoma, la tua sorella, non era neppur mentovata dalla tua bocca, ne' giorni della tua superbia,
\par 57 prima che la tua malvagità fosse messa a nudo, come avvenne quando fosti oltraggiata dalle figliuole della Siria e da tutti i paesi circonvicini, dalle figliuole dei Filistei, che t'insultavano da tutte le parti.
\par 58 Tu porti alla tua volta il peso della tua scelleratezza e delle tue abominazioni, dice l'Eterno.
\par 59 Poiché, così parla il Signore, l'Eterno: Io farò a te come hai fatto tu, che hai sprezzato il giuramento, infrangendo il patto.
\par 60 Nondimeno io mi ricorderò del patto che fermai teco nei giorni della tua giovinezza, e stabilirò per te un patto eterno.
\par 61 E tu ti ricorderai della tua condotta, e ne avrai vergogna, quando riceverai le tue sorelle, quelle che sono più grandi e quelle che sono più piccole di te, e io te le darò per figliuole, ma non in virtù del tuo patto.
\par 62 E io fermerò il mio patto con te, e tu conoscerai che io sono l'Eterno,
\par 63 affinché tu ricordi, e tu arrossisca, e tu non possa più aprir bocca dalla vergogna, quand'io t'avrò perdonato tutto quello che hai fatto, dice il Signore, l'Eterno'.

\chapter{17}

\par 1 E la parola dell'Eterno mi fu rivolta in questi termini:
\par 2 'Figliuol d'uomo, proponi un enigma e narra una parabola alla casa d'Israele, e di':
\par 3 Così parla il Signore, l'Eterno: Una grande aquila, dalle ampie ali, dalle lunghe penne, coperta di piume di svariati colori, venne al Libano, e tolse la cima a un cedro;
\par 4 ne spiccò il più alto dei ramoscelli, lo portò in un paese di commercio, e lo mise in una città di mercanti.
\par 5 Poi prese un germoglio del paese, e lo mise in un campo da sementa; lo collocò presso acque abbondanti, e lo piantò a guisa di magliolo.
\par 6 Esso crebbe, e diventò una vite estesa, di pianta bassa, in modo da avere i suoi tralci vòlti verso l'aquila, e le sue radici sotto di lei. Così diventò una vite che fece de' pampini e mise de' rami.
\par 7 Ma c'era un'altra grande aquila, dalle ampie ali, e dalle piume abbondanti; ed ecco che questa vite volse le sue radici verso di lei; e, dal suolo dov'era piantata, stese verso l'aquila i suoi tralci perch'essa l'annaffiasse.
\par 8 Or essa era piantata in buon terreno, presso acque abbondanti, in modo da poter mettere de' rami, portar frutto e diventare una vite magnifica.
\par 9 Di': Così parla il Signore, l'Eterno: può essa prosperare? La prima aquila non svellerà essa le sue radici e non taglierà essa via i suoi frutti sì che si secchi, e si secchino tutte le giovani foglie che metteva? Né ci sarà bisogno di molta forza né di molta gente per svellerla dalle radici.
\par 10 Ecco, essa è piantata. Prospererà? Non si seccherà essa del tutto dacché l'avrà toccata il vento d'oriente? Seccherà sul suolo dove ha germogliato'.
\par 11 Poi la parola dell'Eterno mi fu rivolta in questi termini:
\par 12 'Di' dunque a questa casa ribelle: Non sapete voi che cosa voglian dire queste cose? Di' loro: Ecco, il re di Babilonia è venuto a Gerusalemme, ne ha preso il re ed i capi, e li ha menati con sé a Babilonia.
\par 13 Poi ha preso uno del sangue reale, ha fermato un patto con lui, e gli ha fatto prestar giuramento; e ha preso pure gli uomini potenti del paese,
\par 14 perché il regno fosse tenuto basso senza potersi innalzare, e quegli osservasse il patto fermato con lui, per poter sussistere.
\par 15 Ma il nuovo re s'è ribellato contro di lui, e ha mandato i suoi ambasciatori in Egitto perché gli fossero dati cavalli e gran gente. Colui che fa tali cose potrà prosperare? Scamperà? Ha rotto il patto e scamperebbe?
\par 16 Com'è vero ch'io vivo, dice il Signore, l'Eterno, nella residenza stessa di quel re che l'avea fatto re, e verso il quale non ha tenuto il giuramento fatto né osservato il patto concluso, vicino a lui, in mezzo a Babilonia, egli morrà:
\par 17 Faraone non andrà col suo potente esercito e con gran gente a soccorrerlo in guerra, quando si eleveranno dei bastioni e si costruiranno delle torri per sterminare gran numero d'uomini.
\par 18 Egli ha violato il giuramento infrangendo il patto, eppure, avea dato la mano! Ha fatto tutte queste cose, e non scamperà.
\par 19 Perciò così parla il Signore, l'Eterno: Com'è vero ch'io vivo, il mio giuramento ch'egli ha violato, il mio patto ch'egli ha infranto, io glieli farò ricadere sul capo.
\par 20 Io stenderò su di lui la mia rete, ed egli rimarrà preso nel mio laccio; lo menerò a Babilonia, e quivi entrerò in giudizio con lui, per la perfidia di cui s'è reso colpevole verso di me.
\par 21 E tutti i fuggiaschi delle sue schiere cadranno per la spada; e quelli che rimarranno saranno dispersi a tutti i venti; e voi conoscerete che io, l'Eterno, son quegli che ho parlato.
\par 22 Così dice il Signore, l'Eterno: Ma io prenderò l'alta vetta del cedro, e la porrò in terra; dai più elevati dei suoi giovani rami spiccherò un tenero ramoscello, e lo pianterò sopra un monte alto, eminente.
\par 23 Lo pianterò sull'alto monte d'Israele; ed esso metterà rami, porterà frutto, e diventerà un cedro magnifico. Gli uccelli d'ogni specie faranno sotto di lui la loro dimora; faran la loro dimora all'ombra dei suoi rami.
\par 24 E tutti gli alberi della campagna sapranno che io, l'Eterno, son quegli che ho abbassato l'albero ch'era su in alto, che ho innalzato l'albero ch'era giù in basso, che ho fatto seccare l'albero verde, e che ho fatto germogliare l'albero secco. Io, l'Eterno, l'ho detto, e lo farò'.

\chapter{18}

\par 1 E la parola dell'Eterno mi fu rivolta in questi termini:
\par 2 'Perché dite nel paese d'Israele questo proverbio: - I padri han mangiato l'agresto e ai figliuoli s'allegano i denti? -
\par 3 Com'è vero ch'io vivo, dice il Signore, l'Eterno, non avrete più occasione di dire questo proverbio in Israele.
\par 4 Ecco, tutte le anime sono mie; è mia tanto l'anima del padre quanto quella del figliuolo; l'anima che pecca sarà quella che morrà.
\par 5 Se uno è giusto e pratica l'equità e la giustizia,
\par 6 se non mangia sui monti e non alza gli occhi verso gl'idoli della casa d'Israele, se non contamina la moglie del suo prossimo, se non s'accosta a donna mentre è impura,
\par 7 se non opprime alcuno, se rende al debitore il suo pegno, se non commette rapine, se dà il suo pane a chi ha fame e copre di vesti l'ignudo,
\par 8 se non presta a interesse e non dà ad usura, se ritrae la sua mano dall'iniquità e giudica secondo verità fra uomo e uomo,
\par 9 se segue le mie leggi e osserva le mie prescrizioni operando con fedeltà, quel tale è giusto; certamente egli vivrà, dice il Signore, l'Eterno.
\par 10 Ma se ha generato un figliuolo ch'è un violento, che spande il sangue e fa al suo fratello qualcuna di queste cose
\par 11 (cose che il padre non commette affatto), e mangia sui monti, e contamina la moglie del suo prossimo,
\par 12 opprime l'afflitto e il povero, commette rapine, non rende il pegno, alza gli occhi verso gl'idoli, fa delle abominazioni,
\par 13 presta a interesse e dà ad usura, questo figlio vivrà egli? No, non vivrà! Egli ha commesso tutte queste abominazioni, e sarà certamente messo a morte; il suo sangue ricadrà su lui.
\par 14 Ma ecco che questi ha generato un figliuolo, il quale, avendo veduto tutti i peccati che suo padre ha commesso, vi pon mente, e non fa cotali cose:
\par 15 non mangia sui monti, non alza gli occhi verso gl'idoli della casa d'Israele, non contamina la moglie del suo prossimo,
\par 16 non opprime alcuno, non prende pegni, non commette rapine, ma dà il suo pane a chi ha fame, copre di vesti l'ignudo,
\par 17 non fa pesar la mano sul povero, non prende interesse né usura, osserva le mie prescrizioni e segue le mie leggi, questo figliuolo non morrà per l'iniquità del padre; egli certamente vivrà.
\par 18 Suo padre, siccome è stato un oppressore, ha commesso rapine a danno del fratello e ha fatto ciò che non è bene in mezzo al suo popolo, ecco che muore per la sua iniquità.
\par 19 Che se diceste: - Perché il figliuolo non porta l'iniquità del padre? - Egli è perché quel figliuolo pratica l'equità e la giustizia, osserva tutte le mie leggi e le mette ad effetto. Certamente egli vivrà.
\par 20 L'anima che pecca è quella che morrà, il figliuolo non porterà l'iniquità del padre, e il padre non porterà l'iniquità del figliuolo; la giustizia del giusto sarà sul giusto, l'empietà dell'empio sarà sull'empio.
\par 21 E se l'empio si ritrae da tutti i peccati che commetteva, se osserva tutte le mie leggi e pratica l'equità e la giustizia, egli certamente vivrà, non morrà.
\par 22 Nessuna delle trasgressioni che ha commesse sarà più ricordata contro di lui; per la giustizia che pratica, egli vivrà.
\par 23 Provo io forse piacere se l'empio muore? dice il Signore, l'Eterno. Non ne provo piuttosto quand'egli si converte dalle sue vie e vive?
\par 24 E se il giusto si ritrae dalla sua giustizia e commette l'iniquità e imita tutte le abominazioni che l'empio fa, vivrà egli? Nessuno de' suoi atti di giustizia sarà ricordato; per la prevaricazione di cui s'è reso colpevole e per il peccato che ha commesso, per tutto questo, morrà.
\par 25 Ma voi dite: 'La via del Signore non è retta...' Ascoltate dunque, o casa d'Israele! - È proprio la mia via quella che non è retta? Non son piuttosto le vie vostre quelle che non son rette?
\par 26 Se il giusto si ritrae dalla sua giustizia e commette l'iniquità, e per questo muore, muore per l'iniquità che ha commessa.
\par 27 E se l'empio si ritrae dall'empietà che commetteva e pratica l'equità e la giustizia, farà vivere l'anima sua.
\par 28 Se ha cura di ritrarsi da tutte le trasgressioni che commetteva, certamente vivrà; non morrà.
\par 29 Ma la casa d'Israele dice: - La via del Signore non è retta. - Son proprio le mie vie quelle che non son rette, o casa d'Israele? Non son piuttosto le vie vostre quelle che non son rette?
\par 30 Perciò, io vi giudicherò ciascuno secondo le vie sue, o casa d'Israele! dice il Signore, l'Eterno. Tornate, convertitevi da tutte le vostre trasgressioni, e non avrete più occasione di caduta nell'iniquità!
\par 31 Gettate lungi da voi tutte le vostre trasgressioni per le quali avete peccato, e fatevi un cuor nuovo e uno spirito nuovo; e perché morreste, o casa d'Israele?
\par 32 Poiché io non ho alcun piacere nella morte di colui che muore, dice il Signore, l'Eterno. Convertitevi dunque, e vivrete!

\chapter{19}

\par 1 E tu pronunzia una lamentazione sui principi d'Israele, e di':
\par 2 Che cos'era tua madre? Una leonessa. Fra i leoni stava accovacciata; in mezzo ai leoncelli, allevava i suoi piccini.
\par 3 Allevò uno de' suoi piccini, il quale divenne un leoncello, imparò a sbranar la preda, e divorò gli uomini.
\par 4 Ma le nazioni ne sentiron parlare, ed ei fu preso nella lor fossa; lo menaron, con de' raffi alle mascelle, nel paese d'Egitto.
\par 5 E quando ella vide che aspettava invano e la sua speranza era delusa, prese un altro de' suoi piccini, e ne fece un leoncello.
\par 6 Questo andava e veniva fra i leoni, e divenne un leoncello; imparò a sbranar la preda, e divorò gli uomini.
\par 7 Devastò i loro palazzi, desolò le loro città; il paese, con tutto quello che conteneva, fu atterrito al rumor dei suoi ruggiti.
\par 8 Ma da tutte le province all'intorno le nazioni gli diedero addosso, gli tesero contro le loro reti, e fu preso nella loro fossa.
\par 9 Lo misero in una gabbia con dei raffi alle mascelle e lo menarono al re di Babilonia; lo menarono in una fortezza, perché la sua voce non fosse più udita sui monti d'Israele.
\par 10 Tua madre era, come te, simile a una vigna, piantata presso alle acque; era feconda, ricca di tralci, per l'abbondanza dell'acque.
\par 11 Aveva de' rami forti, da servire di scettri a sovrani; s'ergeva nella sua sublimità, tra il folto dei tralci; era appariscente per la sua elevatezza, per la moltitudine de' suoi sarmenti.
\par 12 Ma è stata divelta con furore, e gettata a terra; il vento orientale ne ha seccato il frutto; i rami forti ne sono stati rotti e seccati, il fuoco li ha divorati.
\par 13 Ed ora è piantata nel deserto in un suolo arido ed assetato;
\par 14 un fuoco è uscito dal suo ramo fronzuto, e ne ha divorato il frutto, sì che non v'è in essa più ramo forte né scettro per governare'. Questa la lamentazione, ch'è diventata una lamentazione.

\chapter{20}

\par 1 Or avvenne, il settimo anno, il decimo giorno del quinto mese, che alcuni degli anziani d'Israele vennero a consultare l'Eterno, e si misero a sedere davanti a me.
\par 2 E la parola dell'Eterno mi fu rivolta in questi termini:
\par 3 'Figliuol d'uomo, parla agli anziani d'Israele, e di' loro: Così parla il Signore, l'Eterno: Siete venuti per consultarmi? Com'è vero ch'io vivo, io non mi lascerò consultare da voi! dice il Signore, l'Eterno.
\par 4 Giudicali tu, figliuol d'uomo! giudicali tu! Fa' loro conoscere le abominazioni dei loro padri; e di' loro:
\par 5 Così parla il Signore, l'Eterno: Il giorno ch'io scelsi Israele e alzai la mano per fare un giuramento alla progenie della casa di Giacobbe, e mi feci loro conoscere nel paese d'Egitto, e alzai la mano per loro, dicendo: Io son l'Eterno, il vostro Dio,
\par 6 quel giorno alzai la mano, giurando che li trarrei fuori del paese d'Egitto per introdurli in un paese che io avevo cercato per loro, paese ove scorre il latte e il miele, il più splendido di tutti i paesi.
\par 7 E dissi loro: Gettate via, ognun di voi, le abominazioni che attirano i vostri sguardi, e non vi contaminate con gl'idoli d'Egitto; io sono l'Eterno, il vostro Dio!
\par 8 Ma essi si ribellarono contro di me, e non mi vollero dare ascolto; nessun d'essi gettò via le abominazioni che attiravano il suo sguardo, e non abbandonò gl'idoli d'Egitto; allora parlai di voler riversare su loro il mio furore e sfogare su loro la mia ira in mezzo al paese d'Egitto.
\par 9 Nondimeno, io agii per amor del mio nome, perché non fosse profanato agli occhi delle nazioni in mezzo alle quali essi si trovavano, in presenza delle quali io m'ero fatto loro conoscere, allo scopo di trarli fuori dal paese d'Egitto.
\par 10 E li trassi fuori dal paese d'Egitto, e li condussi nel deserto.
\par 11 Diedi loro le mie leggi e feci loro conoscere le mie prescrizioni, per le quali l'uomo che le metterà in pratica vivrà.
\par 12 E diedi pur loro i miei sabati perché servissero di segno fra me e loro, perché conoscessero che io sono l'Eterno che li santifico.
\par 13 Ma la casa d'Israele si ribellò contro di me nel deserto; non camminarono secondo le mie leggi e rigettarono le mie prescrizioni, per le quali l'uomo che le metterà in pratica vivrà, e profanarono gravemente i miei sabati; perciò io parlai di riversare su loro il mio furore sul deserto, per consumarli.
\par 14 Nondimeno io agii per amor del mio nome, perché non fosse profanato agli occhi delle nazioni, in presenza delle quali io li avevo tratti fuori dall'Egitto.
\par 15 E alzai perfino la mano nel deserto, giurando loro che non li farei entrare nel paese che avevo loro dato, paese ove scorre il latte e il miele, il più splendido di tutti i paesi,
\par 16 perché aveano rigettato le mie prescrizioni, non avean camminato secondo le mie leggi e aveano profanato i miei sabati, poiché il loro cuore andava dietro ai loro idoli.
\par 17 Ma l'occhio mio li risparmiò dalla distruzione, e io non li sterminai del tutto nel deserto.
\par 18 E dissi ai loro figliuoli nel deserto: Non camminate secondo i precetti de' vostri padri, non osservate le loro prescrizioni, e non vi contaminate mediante i loro idoli!
\par 19 Io sono l'Eterno, il vostro Dio; camminate secondo le mie leggi, osservate le mie prescrizioni, e mettetele in pratica;
\par 20 santificate i miei sabati, e siano essi un segno fra me e voi, dal quale si conosca che io sono l'Eterno, il vostro Dio.
\par 21 Ma i figliuoli si ribellarono contro di me; non camminarono secondo le mie leggi, e non osservarono le mie prescrizioni per metterle in pratica: le leggi per le quali l'uomo che le mette in pratica vivrà; profanarono i miei sabati, ond'io parlai di riversare su loro il mio furore e di sfogare su loro la mia ira nel deserto.
\par 22 Nondimeno io ritirai la mia mano, ed agii per amor del mio nome, perché non fosse profanato agli occhi delle nazioni, in presenza delle quali li avevo tratti fuori dall'Egitto.
\par 23 Ma alzai pure la mano nel deserto, giurando loro che li disperderei fra le nazioni e li spargerei per tutti i paesi,
\par 24 perché non mettevano in pratica le mie prescrizioni, rigettavano le mie leggi, profanavano i miei sabati, e i loro occhi andavan dietro agl'idoli dei loro padri.
\par 25 E detti loro perfino delle leggi non buone e delle prescrizioni per le quali non potevano vivere;
\par 26 e li contaminai coi loro propri doni, quando facevan passare per il fuoco ogni primogenito, per ridurli alla desolazione affinché conoscessero che io sono l'Eterno.
\par 27 Perciò, figliuol d'uomo, parla alla casa d'Israele e di' loro: Così parla il Signore, l'Eterno: I vostri padri m'hanno ancora oltraggiato in questo, conducendosi perfidamente verso di me:
\par 28 quando li ebbi introdotti nel paese che avevo giurato di dar loro, portarono i loro sguardi sopra ogni alto colle, e sopra ogni albero fronzuto, e quivi offrirono i loro sacrifizi, quivi presentarono le loro offerte provocanti, quivi misero i loro profumi d'odor soave, e quivi sparsero le loro libazioni.
\par 29 Ed io dissi loro: Che cos'è l'alto luogo dove andate? E nondimeno, s'è continuato a chiamarlo 'alto luogo' fino al dì d'oggi.
\par 30 Perciò, di' alla casa d'Israele: Così parla il Signore, l'Eterno: Quando vi contaminate seguendo le vie de' vostri padri e vi prostituite ai loro idoli esecrandi
\par 31 e quando, offrendo i vostri doni e facendo passare per il fuoco i vostri figliuoli, vi contaminate fino al dì d'oggi con tutti i vostri idoli, mi lascerei io consultare da voi, o casa d'Israele? Com'è vero ch'io vivo, dice il Signore, l'Eterno, io non mi lascerò consultare da voi!
\par 32 E non avverrà affatto quello che vi passa per la mente quando dite: Noi saremo come le nazioni, come le famiglie degli altri paesi, e renderemo un culto al legno ed alla pietra!
\par 33 Com'è vero ch'io vivo, dice il Signore, l'Eterno, con mano forte, con braccio disteso, con scatenamento di furore, io regnerò su voi!
\par 34 E vi trarrò fuori di tra i popoli, e vi raccoglierò dai paesi dove sarete stati dispersi, con mano forte, con braccio disteso e con scatenamento di furore,
\par 35 e vi condurrò nel deserto dei popoli, e quivi verrò in giudizio con voi a faccia a faccia;
\par 36 come venni in giudizio coi vostri padri nel deserto del paese d'Egitto, così verrò in giudizio con voi, dice il Signore, l'Eterno;
\par 37 e vi farò passare sotto la verga, e vi rimetterò nei vincoli del patto;
\par 38 e separerò da voi i ribelli e quelli che mi sono infedeli; io li trarrò fuori dal paese dove sono stranieri, ma non entreranno nel paese d'Israele, e voi conoscerete che io sono l'Eterno.
\par 39 Voi dunque, casa d'Israele, così parla il Signore, l'Eterno: Andate, servite ognuno ai vostri idoli, giacché non mi volete ascoltare! Ma il mio santo nome non lo profanerete più coi vostri doni e coi vostri idoli!
\par 40 Poiché sul mio monte santo, e sull'alto monte d'Israele, dice il Signore, l'Eterno, là tutti quelli della casa d'Israele, tutti quanti saranno nel paese, mi serviranno; là io mi compiacerò di loro, là io chiederò le vostre offerte e le primizie dei vostri doni in tutto quello che mi consacrerete.
\par 41 Io mi compiacerò di voi come di un profumo d'odor soave, quando vi avrò tratto fuori di tra i popoli, e vi avrò radunati dai paesi dove sarete stati dispersi; e io sarò santificato in voi nel cospetto delle nazioni;
\par 42 e voi conoscerete che io sono l'Eterno, quando vi avrò condotti nella terra d'Israele, paese che giurai di dare ai vostri padri.
\par 43 E là vi ricorderete della vostra condotta e di tutte le azioni con le quali vi siete contaminati, e sarete disgustati di voi stessi, per tutte le malvagità che avete commesse;
\par 44 e conoscerete che io sono l'Eterno, quando avrò agito con voi per amor del mio nome, e non secondo la vostra condotta malvagia, né secondo le vostre azioni corrotte, o casa d'Israele! dice il Signore, l'Eterno'.

\chapter{21}

\par 1 E la parola dell'Eterno mi fu rivolta in questi termini:
\par 2 'Figliuol d'uomo, volta la faccia dal lato di mezzogiorno, rivolgi la parola al mezzogiorno, e profetizza contro la foresta della campagna meridionale,
\par 3 e di' alla foresta del mezzodì: Ascolta la parola dell'Eterno! Così parla il Signore, l'Eterno: Ecco, io accendo in te un fuoco che divorerà in te ogni albero verde e ogni albero secco; la fiamma dell'incendio non si estinguerà, e tutto ciò ch'è sulla faccia del suolo ne sarà divampato, dal mezzogiorno al settentrione;
\par 4 e ogni carne vedrà che io, l'Eterno, son quegli che ho acceso il fuoco, che non s'estinguerà'.
\par 5 E io dissi: 'Ahimè, Signore, Eterno! Costoro dicon di me: Egli non fa che parlare in parabole'.
\par 6 E la parola dell'Eterno mi fu rivolta in questi termini:
\par 7 'Figliuol d'uomo, volta la faccia verso Gerusalemme, e rivolgi la parola ai luoghi santi, e profetizza contro il paese d'Israele;
\par 8 e di' al paese d'Israele: Così parla l'Eterno: Eccomi a te! Io trarrò la mia spada dal suo fodero, e sterminerò in mezzo a te giusti e malvagi.
\par 9 Appunto perché voglio sterminare in mezzo a te giusti e malvagi, la mia spada uscirà dal suo fodero per colpire ogni carne dal mezzogiorno al settentrione;
\par 10 e ogni carne conoscerà che io, l'Eterno, ho tratto la mia spada dal suo fodero; e non vi sarà più rimessa.
\par 11 E tu, figliuol d'uomo, gemi! Coi lombi rotti e con dolore amaro, gemi dinanzi agli occhi loro.
\par 12 E quando ti chiederanno: Perché gemi? rispondi: Per la notizia che sta per giungere; ogni cuore si struggerà, tutte le mani diverran fiacche, tutti gli spiriti verranno meno, tutte le ginocchia si scioglieranno in acqua. Ecco, la cosa giunge, ed avverrà! dice il Signore, l'Eterno'.
\par 13 E la parola dell'Eterno mi fu rivolta in questi termini:
\par 14 'Figliuol d'uomo, profetizza, e di': Così parla il Signore. Di': La spada! la spada! è aguzzata ed anche forbita:
\par 15 aguzzata, per fare un macello; forbita, perché folgoreggi. Ci rallegrerem noi dunque? ripetendo: 'Lo scettro del mio figliuolo disprezza ogni legno'.
\par 16 Il Signore l'ha data a forbire, perché la s'impugni; la spada è aguzza, essa è forbita, per metterla in mano di chi uccide.
\par 17 Grida e urla, figliuol d'uomo, poich'essa è per il mio popolo, è per tutti i principi d'Israele; essi son dati in balìa della spada col mio popolo; perciò percuotiti la coscia!
\par 18 Poiché la prova è stata fatta; e che dunque, se perfino lo scettro sprezzante non sarà più? dice il Signore, l'Eterno.
\par 19 E tu, figliuol d'uomo, profetizza, e batti le mani; la spada raddoppi, triplichi i suoi colpi, la spada che fa strage, la spada che uccide anche chi è grande, la spada che li attornia.
\par 20 Io ho rivolto la punta della spada contro tutte le loro porte, perché il loro cuore si strugga e cresca il numero dei caduti; sì, essa è fatta per folgoreggiare, è aguzzata per il macello.
\par 21 Spada! raccogliti! volgiti a destra, attenta! volgiti a sinistra, dovunque è diretto il tuo filo!
\par 22 E anch'io batterò le mani, e sfogherò il mio furore! Io, l'Eterno, son quegli che ho parlato'.
\par 23 E la parola dell'Eterno mi fu rivolta in questi termini:
\par 24 'E tu, figliuol d'uomo, fatti due vie, per le quali passi la spada del re di Babilonia; partano ambedue dal medesimo paese; e traccia un indicatore, tracciato al capo della strada d'una città.
\par 25 Fa' una strada per la quale la spada vada a Rabba, città de' figliuoli d'Ammon, e un'altra perché vada in Giuda, a Gerusalemme, città fortificata.
\par 26 Poiché il re di Babilonia sta sul bivio, in capo alle due strade, per tirare presagi: scuote le frecce, consulta gl'idoli, esamina il fegato.
\par 27 La sorte, ch'è nella destra, designa Gerusalemme per collocarvi degli arieti, per aprir la bocca a ordinare il massacro, per alzar la voce in gridi di guerra, per collocare gli arieti contro le porte, per elevare bastioni, per costruire delle torri.
\par 28 Ma essi non vedono in questo che una divinazione bugiarda; essi, a cui sono stati fatti tanti giuramenti! Ma ora egli si ricorderà della loro iniquità, perché siano presi.
\par 29 Perciò così parla il Signore, l'Eterno: Poiché avete fatto ricordare la vostra iniquità mediante le vostre manifeste trasgressioni, sì che i vostri peccati si manifestano in tutte le vostre azioni, poiché ne rievocate il ricordo, sarete presi dalla sua mano.
\par 30 E tu, o empio, dannato alla spada, o principe d'Israele, il cui giorno è giunto al tempo del colmo dell'iniquità;
\par 31 così parla il Signore, l'Eterno: La tiara sarà tolta, il diadema sarà levato; tutto sarà mutato; ciò ch'è in basso sarà innalzato; ciò ch'è in alto sarà abbassato.
\par 32 Ruina! ruina! ruina! Questo farò di lei; anch'essa non sarà più, finché non venga colui a cui appartiene il giudizio, e al quale lo rimetterò.
\par 33 E tu, figliuol d'uomo, profetizza, e di': Così parla il Signore, l'Eterno, riguardo ai figliuoli d'Ammon ed al loro obbrobrio; e di': La spada, la spada è sguainata; è forbita per massacrare, per divorare, per folgoreggiare.
\par 34 Mentre s'hanno per te delle visioni vane, mentre s'hanno per te divinazioni bugiarde, essa ti farà cadere fra i cadaveri degli empi, il cui giorno è giunto al tempo del colmo dell'iniquità.
\par 35 Riponi la spada nel suo fodero! Io ti giudicherò nel luogo stesso dove fosti creata, nel paese della tua origine;
\par 36 e riverserò su di te la mia indignazione, soffierò contro di te nel fuoco della mia ira, e ti darò in mano d'uomini brutali, artefici di distruzione.
\par 37 Tu sarai pascolo al fuoco, il tuo sangue sarà in mezzo al paese; tu non sarai più ricordata, perché io, l'Eterno, son quegli che ho parlato'.

\chapter{22}

\par 1 E la parola dell'Eterno mi fu rivolta in questi termini:
\par 2 'Ora, figliuol d'uomo, non giudicherai tu, non giudicherai tu questa città di sangue? Falle dunque conoscere tutte le sue abominazioni! e di':
\par 3 Così parla il Signore, l'Eterno: O città, che spandi il sangue in mezzo a te perché il tuo tempo giunga, e che ti fai degl'idoli per contaminarti!
\par 4 Per il sangue che hai sparso ti sei resa colpevole, e per gl'idoli che hai fatto ti sei contaminata; tu hai fatto avvicinare i tuoi giorni, e sei giunta al termine de' tuoi anni; perciò io ti espongo al vituperio delle nazioni e allo scherno di tutti i paesi.
\par 5 Quelli che ti son vicini e quelli che son lontani da te si faran beffe di te, o tu contaminata di fama, e piena di disordine!
\par 6 Ecco, i principi d'Israele, ognuno secondo il suo potere, sono occupati in te a spandere il sangue;
\par 7 in te si sprezza padre e madre; in mezzo a te si opprime lo straniero; in te si calpesta l'orfano e la vedova.
\par 8 Tu disprezzi le mie cose sante, tu profani i miei sabati.
\par 9 In te c'è della gente che calunnia per spandere il sangue, in te si mangia sui monti, in mezzo a te si commettono scelleratezze.
\par 10 In te si scoprono le vergogne del padre, in te si violenta la donna durante la sua impurità;
\par 11 in te l'uno commette abominazione con la moglie del suo prossimo, l'altro contamina d'incesto la sua nuora, l'altro violenta la sua sorella, figliuola di suo padre.
\par 12 In te si ricevono regali per spandere del sangue; tu prendi interesse, dài ad usura, trai guadagno dal prossimo con la violenza, e dimentichi me, dice il Signore, l'Eterno.
\par 13 Ma ecco, io batto le mani, a motivo del disonesto guadagno che fai, e del sangue da te sparso, ch'è in mezzo di te.
\par 14 Il tuo cuore reggerà egli, o le tue mani saranno esse forti il giorno che io agirò contro di te? Io, l'Eterno, son quegli che ho parlato, e lo farò.
\par 15 Io ti disperderò fra le nazioni, ti spargerò per i paesi, e torrò via da te tutta la tua immondezza;
\par 16 e tu sarai profanata da te stessa agli occhi delle nazioni, e conoscerai che io sono l'Eterno'.
\par 17 E la parola dell'Eterno mi fu rivolta in questi termini:
\par 18 'Figliuol d'uomo, quelli della casa d'Israele mi son diventati tante scorie: tutti quanti non son che rame, stagno, ferro, piombo, in mezzo al fornello; son tutti scorie d'argento.
\par 19 Perciò, così parla il Signore, l'Eterno: Poiché siete tutti diventati tante scorie, ecco, io vi raduno in mezzo a Gerusalemme.
\par 20 Come si raduna l'argento, il rame, il ferro, il piombo e lo stagno in mezzo al fornello e si soffia nel fuoco per fonderli, così, nella mia ira e nel mio furore io vi radunerò, vi metterò là, e vi fonderò.
\par 21 Vi radunerò, soffierò contro di voi nel fuoco del mio furore e voi sarete fusi in mezzo a Gerusalemme.
\par 22 Come l'argento è fuso in mezzo al fornello, così voi sarete fusi in mezzo alla città; e voi saprete che io, l'Eterno, son quegli che riverso su di voi il mio furore'.
\par 23 E la parola dell'Eterno mi fu rivolta in questi termini:
\par 24 'Figliuol d'uomo, di' a Gerusalemme: Tu sei una terra che non è stata purificata, che non è stata bagnata da pioggia in un giorno d'indignazione.
\par 25 V'è una cospirazione de' suoi profeti in mezzo a lei; come un leone ruggente che sbrana una preda, costoro divorano le anime, piglian tesori e cose preziose, moltiplican le vedove in mezzo a lei.
\par 26 I suoi sacerdoti violano la mia legge e profanano le mie cose sante; non distinguono fra santo e profano, non fan conoscere la differenza che passa fra ciò ch'è impuro e ciò ch'è puro, chiudon gli occhi sui miei sabati, e io son profanato in mezzo a loro.
\par 27 I suoi capi, in mezzo a lei, son come lupi che sbranano la loro preda: spandono il sangue, perdono le anime per saziare la loro cupidigia.
\par 28 E i loro profeti intonacan loro tutto questo con malta che non regge: hanno delle visioni vane, pronostican loro la menzogna, e dicono: - Così parla il Signore, l'Eterno - mentre l'Eterno non ha parlato affatto.
\par 29 Il popolo del paese si dà alla violenza, commette rapine, calpesta l'afflitto e il povero, opprime lo straniero, contro ogni equità.
\par 30 Ed io ho cercato fra loro qualcuno che riparasse la cinta e stesse sulla breccia davanti a me in favore del paese, perché io non lo distruggessi; ma non l'ho trovato.
\par 31 Perciò, io riverserò su loro la mia indignazione; io li consumerò col fuoco della mia ira, e farò ricadere sul loro capo la loro condotta, dice il Signore, l'Eterno'.

\chapter{23}

\par 1 E la parola dell'Eterno mi fu rivolta in questi termini:
\par 2 'Figliuol d'uomo, c'erano due donne, figliuole d'una medesima madre,
\par 3 le quali si prostituirono in Egitto; si prostituirono nella loro giovinezza; là furon premute le loro mammelle, e là fu compromesso il loro vergine seno.
\par 4 I loro nomi sono: quello della maggiore, Ohola; quello della sorella, Oholiba. Esse divennero mie, e mi partorirono figliuoli e figliuole; e questi sono i loro veri nomi: Ohola è Samaria, Oholiba è Gerusalemme.
\par 5 E, mentre era mia, Ohola si prostituì, e s'appassionò per i suoi amanti,
\par 6 gli Assiri, ch'eran suoi vicini, vestiti di porpora, governatori e magistrati, tutti bei giovani, cavalieri montati sui loro cavalli.
\par 7 Ella si prostituì con loro, ch'eran tutti il fior fiore de' figliuoli d'Assiria, e si contaminò con tutti quelli per i quali s'appassionava, con tutti i loro idoli.
\par 8 Ed ella non abbandonò le prostituzioni che commetteva con gli Egiziani, quando questi giacevano con lei nella sua giovinezza, quando comprimevano il suo vergine seno e sfogavano su lei la loro lussuria.
\par 9 Perciò io l'abbandonai in balìa de' suoi amanti, in balìa de' figliuoli d'Assiria, per i quali s'era appassionata.
\par 10 Essi scoprirono la sua nudità, presero i suoi figliuoli e le sue figliuole e la uccisero con la spada. Ed ella diventò famosa fra le donne, e su lei furono eseguiti dei giudizi.
\par 11 E la sua sorella vide questo, e nondimeno si corruppe più di lei ne' suoi amori, e le sue prostituzioni sorpassarono le prostituzioni della sua sorella.
\par 12 S'appassionò per i figliuoli d'Assiria, che eran suoi vicini, governatori e magistrati, vestiti pomposamente, cavalieri montati sui loro cavalli, tutti giovani e belli.
\par 13 E io vidi ch'ella si contaminava; ambedue seguivano la medesima via;
\par 14 ma questa superò l'altra nelle sue prostituzioni; vide degli uomini disegnati sui muri, delle immagini di Caldei dipinte in rosso,
\par 15 con delle cinture ai fianchi, con degli ampi turbanti in capo, dall'aspetto di capitani, tutti quanti, ritratti de' figliuoli di Babilonia, della Caldea, loro terra natìa;
\par 16 e, come li vide, s'appassionò per loro e mandò ad essi de' messaggeri, in Caldea.
\par 17 E i figliuoli di Babilonia vennero a lei, al letto degli amori, e la contaminarono con le loro fornicazioni; ed ella si contaminò con essi; poi, l'anima sua s'alienò da loro.
\par 18 Ella mise a nudo le sue prostituzioni, mise a nudo la sua vergogna, e l'anima mia s'alienò da lei, come l'anima mia s'era alienata dalla sua sorella.
\par 19 Nondimeno, ella moltiplicò le sue prostituzioni, ricordandosi dei giorni della sua giovinezza quando s'era prostituita nel paese d'Egitto;
\par 20 e s'appassionò per quei fornicatori dalle membra d'asino, dall'ardor di stalloni.
\par 21 Così tu tornasti alle turpitudini della tua giovinezza, quando gli Egiziani ti premevan le mammelle a motivo del tuo vergine seno.
\par 22 Perciò, Oholiba, così parla il Signore, l'Eterno: Ecco, io susciterò contro di te i tuoi amanti, dai quali l'anima tua s'è alienata, e li farò venire contro di te da tutte le parti:
\par 23 i figliuoli di Babilonia e tutti i Caldei, principi, ricchi e grandi, e tutti i figliuoli d'Assiria con loro, giovani e belli, tutti governatori e magistrati, capitani e consiglieri, tutti montati sui loro cavalli.
\par 24 Essi vengono contro di te con armi, carri e ruote, e con una folla di popoli; con targhe, scudi, ed elmi si schierano contro di te d'ogni intorno; io rimetto in mano loro il giudizio, ed essi ti giudicheranno secondo le loro leggi.
\par 25 Io darò corso alla mia gelosia contro di te, ed essi ti tratteranno con furore: ti taglieranno il naso e gli orecchi, e ciò che rimarrà di te cadrà per la spada; prenderanno i tuoi figliuoli e le tue figliuole, e ciò che rimarrà di te sarà divorato dal fuoco.
\par 26 E ti spoglieranno delle tue vesti, e porteran via gli oggetti di cui t'adorni.
\par 27 E io farò cessare la tua lussuria e la tua prostituzione cominciata nel paese d'Egitto, e tu non alzerai più gli occhi verso di loro, e non ti ricorderai più dell'Egitto.
\par 28 Poiché così parla il Signore, l'Eterno: Ecco, io ti do in mano di quelli che tu hai in odio, in mano di quelli, dai quali l'anima tua s'è alienata.
\par 29 Essi ti tratteranno con odio, porteran via tutto il frutto del tuo lavoro, e ti lasceranno nuda e scoperta; e così saran messe allo scoperto la vergogna della tua impudicizia, la tua lussuria e le tue prostituzioni.
\par 30 Queste cose ti saran fatte, perché ti sei prostituita correndo dietro alle nazioni, perché ti sei contaminata coi loro idoli.
\par 31 Tu hai camminato per la via della tua sorella, e io ti metto in mano la sua coppa.
\par 32 Così parla il Signore, l'Eterno: Tu berrai la coppa della tua sorella: coppa profonda ed ampia; sarai esposta alle risa ed alle beffe; la coppa è di gran capacità.
\par 33 Tu sarai riempita d'ebbrezza e di dolore: è la coppa della desolazione e della devastazione, è la coppa della tua sorella Samaria.
\par 34 E tu la berrai, la vuoterai, ne morderai i pezzi, e te ne squarcerai il seno; poiché son io quegli che ho parlato, dice il Signore, l'Eterno.
\par 35 Perciò così parla il Signore, l'Eterno: Poiché tu m'hai dimenticato e m'hai buttato dietro alle spalle, porta dunque anche tu, la pena della tua scelleratezza e delle tue prostituzioni'.
\par 36 E l'Eterno mi disse: 'Figliuol d'uomo, non giudicherai tu Ohola e Oholiba? Dichiara loro adunque le loro abominazioni!
\par 37 Poiché han commesso adulterio, han del sangue sulle loro mani; han commesso adulterio coi loro idoli, e gli stessi figliuoli che m'avean partorito, li han fatti passare per il fuoco perché servissero loro di pasto.
\par 38 E anche questo m'hanno fatto: in quel medesimo giorno han contaminato il mio santuario e han profanato i miei sabati.
\par 39 Dopo aver immolato i loro figliuoli ai loro idoli, in quello stesso giorno sono venute nel mio santuario per profanarlo; ecco, quello che hanno fatto in mezzo alla mia casa.
\par 40 E, oltre a questo, hanno mandato a cercare uomini che vengon da lontano; ad essi hanno inviato de' messaggeri, ed ecco che son venuti. Per loro ti sei lavata, ti sei imbellettata gli occhi, ti sei parata d'ornamenti;
\par 41 ti sei assisa sopra un letto sontuoso, davanti al quale era disposta una tavola; e su quella hai messo il mio profumo e il mio olio.
\par 42 E là s'udiva il rumore d'una folla sollazzante, e oltre alla gente presa tra la folla degli uomini, sono stati introdotti degli ubriachi venuti dal deserto, che han messo de' braccialetti ai polsi delle due sorelle, e de' magnifici diademi sul loro capo.
\par 43 E io ho detto di quella invecchiata negli adulterî: Anche ora commettono prostituzioni con lei!... proprio con lei!
\par 44 E si viene ad essa, come si va da una prostituta! Così si viene da Ohola e da Oholiba, da queste donne scellerate.
\par 45 Ma degli uomini giusti le giudicheranno, come si giudican le adultere, come si giudican le donne che spandon il sangue; perché sono adultere, e hanno del sangue sulle mani.
\par 46 Perciò così parla il Signore, l'Eterno: Sarà fatta salire contro di loro una moltitudine, ed esse saranno date in balìa del terrore e del saccheggio.
\par 47 E quella moltitudine le lapiderà, e le farà a pezzi con la spada; ucciderà i loro figliuoli e le loro figliuole, e darà alle fiamme le loro case.
\par 48 E io farò cessare la scelleratezza nel paese, e tutte le donne saranno ammaestrate a non commetter più turpitudini come le vostre.
\par 49 E la vostra scelleratezza vi sarà fatta ricadere addosso, e voi porterete la pena della vostra idolatria, e conoscerete che io sono il Signore, l'Eterno'.

\chapter{24}

\par 1 E la parola dell'Eterno mi fu rivolta il nono anno, il decimo mese, il decimo giorno del mese, in questi termini:
\par 2 'Figliuol d'uomo, scriviti la data di questo giorno, di quest'oggi! Oggi stesso, il re di Babilonia investe Gerusalemme.
\par 3 E proponi una parabola a questa casa ribelle, e di' loro: Così parla il Signore, l'Eterno: Metti, metti la pentola al fuoco, e versaci dentro dell'acqua;
\par 4 raccoglici dentro i pezzi di carne, tutti i buoni pezzi, coscia e spalla; riempila d'ossa scelte.
\par 5 Prendi il meglio del gregge, ammonta sotto la pentola la legna per far bollire le ossa; falla bollire a gran bollore, affinché anche le ossa che ci son dentro, cuociano.
\par 6 Perciò, così parla il Signore, l'Eterno: Guai alla città sanguinaria, pentola piena di verderame, il cui verderame non si stacca! Vuotala de' pezzi, uno a uno, senza tirare a sorte!
\par 7 Poiché il sangue che ha versato è in mezzo a lei; essa lo ha posto sulla roccia nuda; non l'ha sparso in terra, per coprirlo di polvere.
\par 8 Per eccitare il furore, per farne vendetta, ho fatto mettere quel sangue sulla roccia nuda, perché non fosse coperto.
\par 9 Perciò, così parla il Signore, l'Eterno: Guai alla città sanguinaria! Anch'io voglio fare un gran fuoco!
\par 10 Ammonta le legna, fa' levar la fiamma, fa' cuocer bene la carne, fa' struggere il grasso, e fa' che le ossa si consumino!
\par 11 Poi metti la pentola vuota sui carboni perché si riscaldi e il suo rame diventi rovente, affinché la sua impurità si strugga in mezzo ad essa, e il suo verderame sia consumato.
\par 12 Ogni sforzo è inutile; il suo abbondante verderame non si stacca; il suo verderame non se n'andrà che mediante il fuoco.
\par 13 V'è della scelleratezza nella tua impurità; poiché io t'ho voluto purificare e tu non sei diventata pura; non sarai più purificata della tua impurità, finché io non abbia sfogato su di te il mio furore.
\par 14 Io, l'Eterno, son quegli che ho parlato; la cosa avverrà, io la compirò; non indietreggerò, non avrò pietà, non mi pentirò; tu sarai giudicata secondo la tua condotta, secondo le tue azioni, dice il Signore, l'Eterno'.
\par 15 E la parola dell'Eterno mi fu rivolta in questi termini:
\par 16 'Figliuol d'uomo, ecco, con un colpo improvviso io ti tolgo la delizia dei tuoi occhi; e tu non far cordoglio, non piangere, non spander lacrime.
\par 17 Sospira in silenzio; non portar lutto per i morti, cingiti il capo col turbante, mettiti i calzari ai piedi, non ti coprire la barba, e non mangiare il pane che la gente ti manda'.
\par 18 La mattina parlai al popolo, e la sera mi morì la moglie; e la mattina dopo feci come mi era stato comandato.
\par 19 E il popolo mi disse: 'Non ci spiegherai tu che cosa significhi quello che fai?'
\par 20 E io risposi loro: 'La parola dell'Eterno m'è stata rivolta, in questi termini:
\par 21 Di' alla casa d'Israele: Così parla il Signore, l'Eterno: Ecco, io profanerò il mio santuario, l'orgoglio della vostra forza, la delizia degli occhi vostri, il desìo dell'anima vostra; e i vostri figliuoli e le vostre figliuole che avete lasciati a Gerusalemme, cadranno per la spada.
\par 22 E voi farete come ho fatto io: non vi coprirete la barba e non mangerete il pane che la gente vi manda;
\par 23 avrete i vostri turbanti in capo, i vostri calzari ai piedi; non farete cordoglio e non piangerete, ma vi consumerete di languore per le vostre iniquità, e gemerete l'uno con l'altro.
\par 24 Ed Ezechiele sarà per voi un simbolo; tutto quello che fa lui, lo farete voi; e, quando queste cose accadranno, voi conoscerete che io sono il Signore, l'Eterno.
\par 25 E tu, figliuol d'uomo, il giorno ch'io torrò loro ciò che fa la loro forza, la gioia della loro gloria, il desìo de' loro occhi, la brama dell'anima loro, i loro figliuoli e le loro figliuole,
\par 26 in quel giorno un fuggiasco verrà da te a recartene la notizia.
\par 27 In quel giorno la tua bocca s'aprirà, all'arrivo del fuggiasco; e tu parlerai, non sarai più muto, e sarai per loro un simbolo; ed essi conosceranno che io sono l'Eterno'.

\chapter{25}

\par 1 E la parola dell'Eterno mi fu rivolta in questi termini:
\par 2 'Figliuol d'uomo, volgi la tua faccia verso i figliuoli d'Ammon, e profetizza contro di loro; e di' ai figliuoli d'Ammon:
\par 3 Ascoltate la parola del Signore, dell'Eterno: Così parla il Signore, l'Eterno: Poiché tu hai detto: Ah! Ah! quando il mio santuario è stato profanato, quando il suolo d'Israele è stato desolato, e quando la casa di Giuda è andata in cattività,
\par 4 ecco, io ti do in possesso de' figliuoli dell'Oriente, ed essi porranno in te i loro accampamenti, e stabiliranno in mezzo a te le loro dimore; e saranno essi che mangeranno i tuoi frutti, essi che berranno il tuo latte.
\par 5 Io farò di Rabba un pascolo per i cammelli, e del paese de' figliuoli d'Ammon, un ovile per le pecore; e voi conoscerete che io sono l'Eterno.
\par 6 Poiché così parla il Signore, l'Eterno: Poiché tu hai applaudito, e battuto de' piedi, e ti sei rallegrato con tutto lo sprezzo che nutrivi nell'anima per la terra d'Israele,
\par 7 ecco, io stendo la mia mano contro di te, ti do in pascolo alle nazioni, ti stermino di fra i popoli, ti fo sparire dal novero dei paesi, ti distruggo, e tu conoscerai che io sono l'Eterno.
\par 8 Così parla il Signore, l'Eterno: Poiché Moab e Seir dicono: Ecco, la casa di Giuda è come tutte le altre nazioni,
\par 9 ecco, io aprirò il fianco di Moab dal lato delle città, dal lato delle città che stanno alle sue frontiere e sono lo splendore del paese, Beth-Ieschimoth, Baal-meon e Kiriathaim;
\par 10 aprirò il fianco di Moab ai figliuoli dell'Oriente, nello stesso modo che aprirò loro il fianco de' figliuoli d'Ammon; e darò questi paesi in loro possesso, affinché i figliuoli d'Ammon non sian più mentovati fra le nazioni;
\par 11 ed eserciterò i miei giudizi su Moab, ed essi conosceranno che io sono l'Eterno.
\par 12 Così parla il Signore, l'Eterno: Poiché quelli d'Edom si sono crudelmente vendicati della casa di Giuda e si sono resi gravemente colpevoli vendicandosi d'essa,
\par 13 così parla il Signore, l'Eterno: Io stenderò la mia mano contro Edom, ne sterminerò uomini e bestie, ne farò un deserto fino da Theman, e fino a Dedan cadranno per la spada.
\par 14 E rimetterò la mia vendetta sopra Edom nelle mani del mio popolo d'Israele; esso tratterà Edom secondo la mia ira e secondo il mio furore; ed essi conosceranno la mia vendetta, dice il Signore, l'Eterno.
\par 15 Così parla il Signore, l'Eterno: Poiché i Filistei si sono abbandonati alla vendetta e si sono crudelmente vendicati, collo sprezzo che nutrivano nell'anima, dandosi alla distruzione per odio antico,
\par 16 così parla il Signore, l'Eterno: Ecco, io stenderò la mia mano contro i Filistei, sterminerò i Kerethei, e distruggerò il rimanente della costa del mare;
\par 17 ed eserciterò su loro grandi vendette, e li riprenderò con furore; ed essi conosceranno che io sono l'Eterno, quando avrò fatto loro sentire la mia vendetta'.

\chapter{26}

\par 1 E avvenne, l'anno undecimo, il primo giorno del mese, che la parola dell'Eterno mi fu rivolta in questi termini:
\par 2 'Figliuol d'uomo, poiché Tiro ha detto di Gerusalemme: - Ah! Ah! è infranta colei ch'era la porta dei popoli! La gente si volge verso me! Io mi riempirò di lei ch'è deserta! -
\par 3 perciò così parla il Signore, l'Eterno: Eccomi contro di te, o Tiro! Io farò salire contro di te molti popoli, come il mare fa salire le proprie onde.
\par 4 Ed essi distruggeranno le mura di Tiro, e abbatteranno le sue torri: io spazzerò via di su lei la sua polvere, e farò di lei una roccia nuda.
\par 5 Ella sarà, in mezzo al mare, un luogo da stender le reti, poiché son io quegli che ho parlato, dice il Signore, l'Eterno; ella sarà abbandonata al saccheggio delle nazioni;
\par 6 e le sue figliuole che sono nei campi saranno uccise dalla spada, e quei di Tiro sapranno che io sono l'Eterno.
\par 7 Poiché così dice il Signore, l'Eterno: Ecco, io fo venire dal settentrione contro Tiro Nebucadnetsar, re di Babilonia, il re dei re, con de' cavalli, con de' carri e con de' cavalieri, e una gran folla di gente.
\par 8 Egli ucciderà con la spada le tue figliuole che sono nei campi, farà contro di te delle torri, innalzerà contro di te de' bastioni, leverà contro di te le targhe;
\par 9 dirigerà contro le tue mura i suoi arieti, e coi suoi picconi abbatterà le tue torri.
\par 10 La moltitudine de' suoi cavalli sarà tale che la polvere sollevata da loro ti coprirà; lo strepito de' suoi cavalieri, delle sue ruote e de' suoi carri, farà tremare le tue mura, quand'egli entrerà per le tue porte, come s'entra in una città dove s'è aperta una breccia.
\par 11 Con gli zoccoli de' suoi cavalli egli calpesterà tutte le tue strade; ucciderà il tuo popolo con la spada, e le colonne in cui riponi la tua forza cadranno a terra.
\par 12 Essi faranno lor bottino delle tue ricchezze, saccheggeranno le tue mercanzie, abbatteranno le tue mura, distruggeranno le tue case deliziose, e getteranno in mezzo alle acque le tue pietre, il tuo legname, la tua polvere.
\par 13 Io farò cessare il rumore de' tuoi canti, e il suono delle tue arpe non s'udrà più.
\par 14 E ti ridurrò ad essere una roccia nuda; tu sarai un luogo da stendervi le reti; tu non sarai più riedificata, perché io, l'Eterno, son quegli che ho parlato, dice il Signore, l'Eterno.
\par 15 Così parla il Signore, l'Eterno, a Tiro: Sì, al rumore della tua caduta, al gemito dei feriti a morte, al massacro che si farà in mezzo a te, tremeranno le isole.
\par 16 Tutti i principi del mare scenderanno dai loro troni, si torranno i loro manti, deporranno le loro vesti ricamate; s'avvolgeranno nello spavento, si sederanno per terra, tremeranno ad ogni istante saranno costernati per via di te.
\par 17 E prenderanno a fare su di te un lamento, e ti diranno: Come mai sei distrutta, tu che eri abitata da gente di mare, la città famosa, ch'eri così potente in mare, tu che al pari dei tuoi abitanti incutevi terrore a tutti gli abitanti della terra!
\par 18 Ora le isole tremeranno il giorno della tua caduta, le isole del mare saranno spaventate per la tua fine.
\par 19 Poiché così parla il Signore, l'Eterno: Quando farò di te una città desolata come le città che non han più abitanti, quando farò salire su di te l'abisso e le grandi acque ti copriranno,
\par 20 allora ti trarrò giù, con quelli che scendon nella fossa, fra il popolo d'un tempo, ti farò dimorare nelle profondità della terra, nelle solitudini eterne, con quelli che scendon nella fossa, perché tu non sia più abitata; mentre rimetterò lo splendore sulla terra de' viventi.
\par 21 Io ti ridurrò uno spavento, e non sarai più; ti si cercherà ma non ti si troverà mai più, dice il Signore, l'Eterno'.

\chapter{27}

\par 1 La parola dell'Eterno mi fu rivolta in questi termini:
\par 2 'E tu, figliuol d'uomo, pronunzia una lamentazione su Tiro,
\par 3 e di' a Tiro che sta agli approdi del mare, che porta le mercanzie de' popoli a molte isole: Così parla il Signore, l'Eterno: O Tiro, tu dici: Io sono di una perfetta bellezza.
\par 4 Il tuo dominio è nel cuore dei mari; i tuoi edificatori t'hanno fatto di una bellezza perfetta;
\par 5 hanno costruito di cipresso di Senir tutte le tue pareti; hanno preso dei cedri del Libano per fare l'alberatura delle tue navi;
\par 6 han fatto i tuoi remi di quercia di Bashan, han fatto i ponti del tuo naviglio d'avorio incastonato in larice, portato dalle isole di Kittim.
\par 7 Il lino fino d'Egitto lavorato a ricami, t'ha servito per le tue vele e per le tue bandiere; la porpora e lo scarlatto delle isole d'Elisha formano i tuoi padiglioni.
\par 8 Gli abitanti di Sidon e d'Arvad sono i tuoi rematori; i tuoi savi, o Tiro, sono in mezzo a te; son dessi i tuoi piloti.
\par 9 Tu hai in mezzo a te gli anziani di Ghebel e i suoi savi, a calafatare le tue falle; in te son tutte le navi del mare coi loro marinai, per far lo scambio delle tue mercanzie.
\par 10 Dei Persiani, dei Lidî, dei Libî servono nel tuo esercito; son uomini di guerra, che sospendono in mezzo a te lo scudo e l'elmo; sono la tua magnificenza.
\par 11 I figliuoli d'Arvad e il tuo esercito guarniscono d'ogn'intorno le tue mura, e degli uomini prodi stanno nelle tue torri; essi sospendono le loro targhe tutt'intorno alle tue mura; essi rendon perfetta la tua bellezza.
\par 12 Tarsis traffica teco con la sua abbondanza d'ogni sorta di ricchezze; fornisce i tuoi mercati d'argento, di ferro, di stagno e di piombo.
\par 13 Javan, Tubal e Mescec anch'essi traffican teco; danno anime umane e utensili di rame in scambio delle tue mercanzie.
\par 14 Quelli della casa di Togarma pagano le tue mercanzie con cavalli da tiro, con cavalli da corsa, e con muli.
\par 15 I figliuoli di Dedan trafficano teco; il commercio di molte isole passa per le tue mani; ti pagano con denti d'avorio e con ebano.
\par 16 La Siria commercia con te, per la moltitudine de' tuoi prodotti; fornisce i tuoi scambi di carbonchi, di porpora, di stoffe ricamate, di bisso, di corallo, di rubini.
\par 17 Giuda e il paese d'Israele anch'essi trafficano teco, ti danno in pagamento grano di Minnith, pasticcerie, miele, olio e balsamo.
\par 18 Damasco commercia teco, scambiando i tuoi numerosi prodotti con abbondanza d'ogni sorta di beni, con vino di Helbon e con lana candida.
\par 19 Vedan e Javan d'Uzzal provvedono i tuoi mercati; ferro lavorato, cassia, canna aromatica, sono fra i prodotti di scambio.
\par 20 Dedan traffica teco in coperte da cavalcatura.
\par 21 L'Arabia e tutti i principi di Kedar fanno commercio teco, trafficando in agnelli, in montoni, in capri.
\par 22 I mercanti di Sceba e di Raama anch'essi trafficano teco; provvedono i tuoi mercati di tutti i migliori aromi, d'ogni sorta di pietre preziose, e d'oro.
\par 23 Haran, Canné e Eden, i mercanti di Sceba, d'Assiria, di Kilmad, trafficano teco;
\par 24 trafficano teco in oggetti di lusso, in mantelli di porpora, in ricami, in casse di stoffe preziose legate con corde, e fatte di cedro.
\par 25 Le navi di Tarsis son la tua flotta per il tuo commercio. Così ti sei riempita, e ti sei grandemente arricchita nel cuore dei mari.
\par 26 I tuoi rematori t'han menata nelle grandi acque; il vento d'oriente s'infrange nel cuore de' mari.
\par 27 Le tue ricchezze, i tuoi mercati, la tua mercanzia, i tuoi marinai, i tuoi piloti, i tuoi calafati, i tuoi negozianti, tutta la tua gente di guerra ch'è in te, e tutta la moltitudine ch'è in mezzo a te, cadranno nel cuore de' mari, il giorno della tua rovina.
\par 28 Alle grida de' tuoi piloti, i lidi tremeranno;
\par 29 e tutti quelli che maneggiano il remo, i marinai e tutti i piloti del mare scenderanno dalle loro navi, e si terranno sulla terra ferma.
\par 30 E faranno sentire la lor voce su di te; grideranno amaramente, si getteranno della polvere sul capo, si rotoleranno nella cenere.
\par 31 A causa di te si raderanno il capo, si cingeranno di sacchi; per te piangeranno con amarezza d'animo, con cordoglio amaro;
\par 32 e, nella loro angoscia, pronunzieranno su di te una lamentazione, e si lamenteranno così riguardo a te: Chi fu mai come Tiro, come questa città, ora muta in mezzo al mare?
\par 33 Quando i tuoi prodotti uscivano dai mari, tu saziavi gran numero di popoli; con l'abbondanza delle ricchezze e del tuo traffico, arricchivi i re della terra.
\par 34 Quando sei stata infranta dai mari, nelle profondità delle acque, la tua mercanzia e tutta la moltitudine ch'era in mezzo di te, sono cadute.
\par 35 Tutti gli abitanti delle isole sono sbigottiti a causa di te; i loro re son presi da orribile paura, il loro aspetto è sconvolto.
\par 36 I mercanti fra i popoli fischiano su di te; sei diventata uno spavento, e non esisterai mai più!'

\chapter{28}

\par 1 E la parola dell'Eterno mi fu rivolta in questi termini:
\par 2 'Figliuol d'uomo, di' al principe di Tiro: Così parla il Signore, l'Eterno: Il tuo cuore s'è fatto altero, e tu dici: Io sono un dio! Io sto assiso sopra un trono di Dio nel cuore de' mari! mentre sei un uomo non un Dio, quantunque tu ti faccia un cuore simile al cuore d'un Dio.
\par 3 Ecco, tu sei più savio di Daniele, nessun mistero è oscuro per te;
\par 4 con la tua saviezza e con la tua intelligenza ti sei procurato ricchezza, hai ammassato oro e argento nei tuoi tesori;
\par 5 con la tua gran saviezza e col tuo commercio hai accresciuto le tue ricchezze, e a motivo delle tue ricchezze il tuo cuore s'è fatto altero.
\par 6 Perciò così parla il Signore, l'Eterno: Poiché tu ti sei fatto un cuore come un cuore di Dio,
\par 7 ecco, io fo venire contro di te degli stranieri, i più violenti di fra le nazioni; ed essi sguaineranno le loro spade contro lo splendore della tua saviezza, e contamineranno la tua bellezza;
\par 8 ti trarranno giù nella fossa, e tu morrai della morte di quelli che sono trafitti nel cuore de' mari.
\par 9 Continuerai tu a dire: 'Io sono un Dio', in presenza di colui che ti ucciderà? Sarai un uomo e non un Dio nelle mani di chi ti trafiggerà!
\par 10 Tu morrai della morte degl'incirconcisi, per man di stranieri; poiché io ho parlato, dice il Signore, l'Eterno'.
\par 11 E la parola dell'Eterno mi fu rivolta in questi termini:
\par 12 'Figliuol d'uomo, pronunzia una lamentazione sul re di Tiro, e digli: Così parla il Signore, l'Eterno: Tu mettevi il suggello alla perfezione, eri pieno di saviezza, di una bellezza perfetta;
\par 13 eri in Eden il giardino di Dio; eri coperto d'ogni sorta di pietre preziose: rubini, topazi, diamanti, crisoliti, onici, diaspri, zaffiri, carbonchi, smeraldi, oro; tamburi e flauti erano al tuo servizio, preparati il giorno che fosti creato.
\par 14 Eri un cherubino dalle ali distese, un protettore. Io t'avevo stabilito, tu stavi sul monte santo di Dio, camminavi in mezzo a pietre di fuoco.
\par 15 Tu fosti perfetto nelle tue vie dal giorno che fosti creato, perché non si trovò in te la perversità.
\par 16 Per l'abbondanza del tuo commercio, tutto in te s'è riempito di violenza, e tu hai peccato; perciò io ti caccio come un profano dal monte di Dio, e ti farò sparire, o cherubino protettore, di mezzo alle pietre di fuoco.
\par 17 Il tuo cuore s'è fatto altero per la tua bellezza; tu hai corrotto la tua saviezza a motivo del tuo splendore; io ti getto a terra, ti do in ispettacolo ai re.
\par 18 Con la moltitudine delle tue iniquità, colla disonestà del tuo commercio, tu hai profanato i tuoi santuari; ed io faccio uscire di mezzo a te un fuoco che ti divori, e ti riduco in cenere sulla terra, in presenza di tutti quelli che ti guardano.
\par 19 Tutti quelli che ti conoscevano fra i popoli restano stupefatti al vederti; tu sei diventato oggetto di terrore e non esisterai mai più'.
\par 20 La parola dell'Eterno mi fu rivolta in questi termini:
\par 21 'Figliuol d'uomo, volgi la faccia verso Sidon, profetizza contro di lei,
\par 22 e di': Così parla il Signore, l'Eterno: Eccomi contro di te, o Sidon! e io mi glorificherò in mezzo di te: e si conoscerà che io sono l'Eterno, quando avrò eseguiti i miei giudizi contro di lei, e mi sarò santificato in lei.
\par 23 Io manderò contro di lei la peste, e ci sarà del sangue nelle sue strade; e in mezzo ad essa cadranno gli uccisi dalla spada, che piomberà su lei da tutte le parti; e si conoscerà che io sono l'Eterno.
\par 24 E non ci sarà più per la casa d'Israele né spina maligna né rovo irritante fra tutti i suoi vicini che la disprezzano; e si conoscerà che io sono il Signore, l'Eterno.
\par 25 Così parla il Signore, l'Eterno: Quando avrò raccolto la casa d'Israele di mezzo ai popoli fra i quali essa è dispersa, io mi santificherò in loro nel cospetto delle nazioni, ed essi abiteranno il loro paese che io ho dato al mio servo Giacobbe;
\par 26 vi abiteranno al sicuro; edificheranno case e pianteranno vigne; abiteranno al sicuro, quand'io avrò eseguito i miei giudizi su tutti quelli che li circondano e li disprezzano; e conosceranno che io sono l'Eterno, il loro Dio'.

\chapter{29}

\par 1 L'anno decimo, il decimo mese, il dodicesimo giorno del mese, la parola dell'Eterno mi fu rivolta in questi termini:
\par 2 'Figliuol d'uomo volgi la tua faccia contro Faraone, re d'Egitto, e profetizza contro di lui e contro l'Egitto tutto quanto;
\par 3 parla e di': Così parla il Signore, l'Eterno: Eccomi contro di te, Faraone, re d'Egitto, gran coccodrillo, che giaci in mezzo ai tuoi fiumi, e dici: - Il mio fiume è mio, e son io che me lo son fatto!
\par 4 Io metterò dei ganci nelle tue mascelle, e farò sì che i pesci de' tuoi fiumi s'attaccheranno alle tue scaglie, e ti trarrò fuori di mezzo ai tuoi fiumi, con tutti i pesci de' tuoi fiumi attaccati alle tue scaglie.
\par 5 E ti getterò nel deserto, te e tutti i pesci de' tuoi fiumi, e tu cadrai sulla faccia de' campi; non sarai né adunato né raccolto, e io ti darò in pasto alle bestie della terra e agli uccelli del cielo.
\par 6 E tutti gli abitanti dell'Egitto conosceranno che io sono l'Eterno, perché essi sono stati per la casa d'Israele un sostegno di canna.
\par 7 Quando t'hanno preso in mano tu ti sei rotto e hai forato loro tutta la spalla, e quando si sono appoggiati su di te tu ti sei spezzato e li hai fatti stare tutti ritti sui loro fianchi.
\par 8 Perciò, così parla il Signore, l'Eterno: Ecco, io farò venire sopra di te la spada e sterminerò in mezzo a te uomini e bestie:
\par 9 il paese d'Egitto sarà ridotto in una desolazione, in un deserto, e si conoscerà che io sono l'Eterno, perché Faraone ha detto: - Il fiume è mio, e son io che l'ho fatto! -
\par 10 Perciò, eccomi contro di te e contro il tuo fiume; e ridurrò il paese d'Egitto in un deserto, in una desolazione, da Migdol a Syene, fino alle frontiere dell'Etiopia.
\par 11 Non vi passerà piè d'uomo, non vi passerà piè di bestia, né sarà più abitato per quarant'anni;
\par 12 e ridurrò il paese d'Egitto in una desolazione in mezzo a contrade desolate, e le sue città saranno una desolazione, per quarant'anni, in mezzo a città devastate; e disperderò gli Egiziani fra le nazioni, e li spargerò per tutti i paesi.
\par 13 Poiché, così parla il Signore, l'Eterno: Alla fine dei quarant'anni io raccoglierò gli Egiziani di fra i popoli dove saranno stati dispersi,
\par 14 e farò tornare gli Egiziani dalla loro cattività e li ricondurrò nel paese di Patros, nel loro paese natio, e quivi saranno un umile regno.
\par 15 L'Egitto sarà il più umile dei regni, e non si eleverà più sopra le nazioni; e io ridurrò il loro numero, perché non dominino più sulle nazioni;
\par 16 e la casa d'Israele non riporrà più la sua fiducia in quelli che le ricorderanno l'iniquità da lei commessa quando si volgeva verso di loro; e si conoscerà che io sono il Signore, l'Eterno'.
\par 17 E il ventisettesimo anno, il primo mese, il primo giorno del mese, la parola dell'Eterno mi fu rivolta in questi termini:
\par 18 'Figliuol d'uomo, Nebucadnetsar, re di Babilonia, ha fatto fare al suo esercito un duro servizio contro Tiro; ogni testa n'è divenuta calva, ogni spalla scorticata; e né egli né il suo esercito hanno ricavato da Tiro alcun salario del servizio ch'egli ha fatto contro di essa.
\par 19 Perciò così parla il Signore, l'Eterno: Ecco, io do a Nebucadnetsar, re di Babilonia, il paese d'Egitto; ed egli ne porterà via le ricchezze, lo spoglierà d'ogni sua spoglia, vi prederà ciò che v'è da predare, e questo sarà il salario del suo esercito.
\par 20 Come retribuzione del servizio ch'egli ha fatto contro Tiro, io gli do il paese d'Egitto, poiché han lavorato per me, dice il Signore, l'Eterno.
\par 21 In quel giorno io farò rispuntare la potenza della casa d'Israele, e darò a te di parlar liberamente in mezzo a loro, ed essi conosceranno che io sono l'Eterno'.

\chapter{30}

\par 1 E la parola dell'Eterno mi fu rivolta in questi termini:
\par 2 'Figliuol d'uomo, profetizza e di': Così parla il Signore, l'Eterno: Urlate: Ahi, che giorno!
\par 3 Poiché il giorno è vicino, è vicino il giorno dell'Eterno: giorno di nuvole, il tempo delle nazioni.
\par 4 La spada verrà sull'Egitto, e vi sarà terrore in Etiopia quando in Egitto cadranno i feriti a morte, quando si porteran via le sue ricchezze, e le sue fondamenta saranno rovesciate.
\par 5 L'Etiopia, la Libia, la Lidia, Put, Lud, gli stranieri d'ogni sorta, Cub e i figli del paese dell'alleanza, cadranno con loro per la spada.
\par 6 Così parla l'Eterno: Quelli che sostengono l'Egitto cadranno, e l'orgoglio della sua forza sarà abbattuto: da Migdol a Syene essi cadranno per la spada, dice il Signore, l'Eterno,
\par 7 e saranno desolati in mezzo a terre desolate, le loro città saranno devastate in mezzo a città devastate;
\par 8 e conosceranno che io sono l'Eterno, quando metterò il fuoco all'Egitto, e tutti i suoi ausiliari saranno fiaccati.
\par 9 In quel giorno, partiranno de' messi dalla mia presenza su delle navi per spaventare l'Etiopia nella sua sicurtà; e regnerà fra loro il terrore come nel giorno dell'Egitto; poiché, ecco, la cosa sta per avvenire.
\par 10 Così parla il Signore, l'Eterno: Io farò sparire la moltitudine dell'Egitto, per man di Nebucadnetsar, re di Babilonia.
\par 11 Egli e il suo popolo con lui, i più violenti fra le nazioni, saranno condotti a distruggere il paese; sguaineranno le spade contro l'Egitto, e riempiranno il paese d'uccisi.
\par 12 E io muterò i fiumi in luoghi aridi, darò il paese in balìa di gente malvagia, e per man di stranieri desolerò il paese e tutto ciò che contiene. Io, l'Eterno, sono quegli che ho parlato.
\par 13 Così parla il Signore, l'Eterno: Io sterminerò da Nof gl'idoli, e ne farò sparire i falsi dèi; non ci sarà più principe che venga dal paese d'Egitto, e metterò lo spavento nel paese d'Egitto.
\par 14 Desolerò Patros, darò alle fiamme Tsoan, eserciterò i miei giudizi su No,
\par 15 riverserò il mio furore sopra Sin, la fortezza dell'Egitto, e sterminerò la moltitudine di No.
\par 16 Applicherò il fuoco all'Egitto; Sin si torcerà dal dolore, No sarà squarciata, Nof sarà presa da nemici in pieno giorno.
\par 17 I giovani di Aven e di Pibeseth cadranno per la spada, e queste città andranno in cattività.
\par 18 E a Tahpanes il giorno s'oscurerà, quand'io spezzerò quivi i gioghi imposti dall'Egitto; e l'orgoglio della sua forza avrà fine. Quanto a lei, una nuvola la coprirà, e le sue figliuole andranno in cattività.
\par 19 Così eserciterò i miei giudizi sull'Egitto, e si conoscerà che io sono l'Eterno'.
\par 20 L'anno undicesimo, il primo mese, il settimo giorno del mese, la parola dell'Eterno mi fu rivolta in questi termini:
\par 21 'Figliuol d'uomo, io ho spezzato il braccio di Faraone, re d'Egitto; ed ecco, il suo braccio non è stato fasciato applicandovi rimedi e mettendovi delle bende per fasciarlo e fortificarlo, in guisa da poter maneggiare una spada.
\par 22 Perciò, così parla il Signore, l'Eterno: Eccomi contro Faraone, re d'Egitto, per spezzargli le braccia, tanto quello ch'è ancora forte, quanto quello ch'è già spezzato; e gli farò cader di mano la spada.
\par 23 E disperderò gli Egiziani fra le nazioni, e li spargerò per tutti i paesi;
\par 24 e fortificherò le braccia del re di Babilonia, e gli metterò in mano la mia spada; e spezzerò le braccia di Faraone, ed egli gemerà davanti a lui, come geme un uomo ferito a morte.
\par 25 Fortificherò le braccia del re di Babilonia, e le braccia di Faraone cadranno; e si conoscerà che io sono l'Eterno, quando metterò la mia spada in man del re di Babilonia, ed egli la volgerà contro il paese d'Egitto.
\par 26 E io disperderò gli Egiziani fra le nazioni, e li spargerò per tutti i paesi; e si conoscerà che io sono l'Eterno'.

\chapter{31}

\par 1 L'anno undecimo, il terzo mese, il primo giorno del mese, la parola dell'Eterno mi fu rivolta in questi termini:
\par 2 'Figliuol d'uomo, di' a Faraone re d'Egitto, e alla sua moltitudine: A chi somigli tu nella tua grandezza?
\par 3 Ecco, l'Assiro era un cedro del Libano, dai bei rami, dall'ombra folta, dal tronco slanciato, dalla vetta sporgente fra il folto de' rami.
\par 4 Le acque lo nutrivano, l'abisso lo facea crescere, andando, coi suoi fiumi, intorno al luogo dov'era piantato, mentre mandava i suoi canali a tutti gli alberi dei campi.
\par 5 Perciò la sua altezza era superiore a quella di tutti gli alberi della campagna, i suoi rami s'eran moltiplicati, i suoi ramoscelli s'erano allungati per l'abbondanza delle acque che lo faceano sviluppare.
\par 6 Tutti gli uccelli del cielo s'annidavano fra i suoi rami, tutte le bestie de' campi figliavano sotto i suoi ramoscelli, e tutte le grandi nazioni dimoravano alla sua ombra.
\par 7 Era bello per la sua grandezza, per la lunghezza dei suoi rami, perché la sua radice era presso acque abbondanti.
\par 8 I cedri non lo sorpassavano nel giardino di Dio; i cipressi non uguagliavano i suoi ramoscelli, e i platani non eran neppure come i suoi rami; nessun albero nel giardino di Dio lo pareggiava in bellezza.
\par 9 Io l'avevo reso bello per l'abbondanza de' suoi rami, e tutti gli alberi d'Eden, che sono nel giardino di Dio, gli portavano invidia.
\par 10 Perciò, così parla il Signore, l'Eterno: Perché era salito a tanta altezza e sporgeva la sua vetta tra il folto de' rami e perché il suo cuore s'era insuperbito della sua altezza,
\par 11 io lo diedi in mano del più forte fra le nazioni perché lo trattasse a suo piacimento; per la sua empietà io lo scacciai.
\par 12 Degli stranieri, i più violenti fra le nazioni, l'hanno tagliato e l'han lasciato in abbandono; sui monti e in tutte le valli son caduti i suoi rami, i suoi ramoscelli sono stati spezzati in tutti i burroni del paese, e tutti i popoli della terra si son ritirati dalla sua ombra, e l'hanno abbandonato.
\par 13 Sul suo tronco caduto si posano tutti gli uccelli del cielo, e sopra i suoi rami stanno tutte le bestie de' campi.
\par 14 Così è avvenuto affinché gli alberi tutti piantati presso alle acque non sian fieri della propria altezza, non sporgan più la vetta fra il folto de' rami, e tutti gli alberi potenti che si dissetano alle acque non persistano nella loro fierezza; poiché tutti quanti son dati alla morte, alle profondità della terra, assieme ai figliuoli degli uomini, a quelli che scendono nella fossa.
\par 15 Così parla il Signore, l'Eterno: Il giorno ch'ei discese nel soggiorno de' morti, io feci fare cordoglio; a motivo di lui velai l'abisso, ne arrestai i fiumi, e le grandi acque furon fermate; a motivo di lui abbrunai il Libano, e tutti gli alberi de' campi vennero meno a motivo di lui.
\par 16 Al rumore della sua caduta feci tremare le nazioni, quando lo feci scendere nel soggiorno de' morti con quelli che scendon nella fossa; e nelle profondità della terra si consolarono tutti gli alberi di Eden, i più scelti e i più belli del Libano, tutti quelli che si dissetavano alle acque.
\par 17 Anch'essi discesero con lui nel soggiorno de' morti, verso quelli che la spada ha uccisi: verso quelli che erano il suo braccio, e stavano alla sua ombra in mezzo alle nazioni.
\par 18 A chi dunque somigli tu per gloria e per grandezza fra gli alberi d'Eden? Così tu sarai precipitato con gli alberi d'Eden nelle profondità della terra; tu giacerai in mezzo agl'incirconcisi, fra quelli che la spada ha uccisi. Tal sarà di Faraone con tutta la sua moltitudine, dice il Signore, l'Eterno'.

\chapter{32}

\par 1 L'anno dodicesimo, il dodicesimo mese, il primo giorno del mese, la parola dell'Eterno mi fu rivolta in questi termini:
\par 2 'Figliuol d'uomo, pronunzia una lamentazione su Faraone, re d'Egitto, e digli: Tu eri simile ad un leoncello fra le nazioni; eri come un coccodrillo nei mari; ti slanciavi ne' tuoi fiumi, e coi tuoi piedi agitavi le acque e ne intorbidavi i canali.
\par 3 Così parla il Signore, l'Eterno: Io stenderò su di te la mia rete mediante gran moltitudine di popoli, i quali ti trarranno fuori con la mia rete;
\par 4 e t'abbandonerò sulla terra e ti getterò sulla faccia dei campi, e farò che su di te verranno a posarsi tutti gli uccelli del cielo, e sazierò di te le bestie di tutta la terra;
\par 5 metterò la tua carne su per i monti, e riempirò le valli de' tuoi avanzi;
\par 6 annaffierò del tuo sangue, fin sui monti, il paese dove nuoti; e i canali saran ripieni di te.
\par 7 Quando t'estinguerò, velerò i cieli e ne oscurerò le stelle; coprirò il sole di nuvole, e la luna non darà la sua luce.
\par 8 Per via di te, oscurerò tutti i luminari che splendono in cielo, e stenderò le tenebre sul tuo paese, dice il Signore, l'Eterno.
\par 9 Affliggerò il cuore di molti popoli, quando farò giungere la notizia della tua ruina fra le nazioni, in paesi che tu non conosci;
\par 10 e farò sì che di te resteranno attoniti molti popoli, e i loro re saran presi da spavento per causa tua, quand'io brandirò la mia spada dinanzi a loro; e ognun d'essi tremerà ad ogni istante per la sua vita, nel giorno della tua caduta.
\par 11 Poiché così parla il Signore, l'Eterno: La spada del re di Babilonia ti piomberà addosso.
\par 12 Io farò cadere la moltitudine del tuo popolo per la spada d'uomini potenti, tutti quanti i più violenti fra le nazioni, ed essi distruggeranno il fasto dell'Egitto, e tutta la sua moltitudine sarà annientata.
\par 13 E farò perire tutto il suo bestiame di sulle rive delle grandi acque; nessun piede d'uomo le intorbiderà più, non le intorbiderà più unghia di bestia.
\par 14 Allora lascerò posare le loro acque, e farò scorrere i loro fiumi come olio, dice il Signore, l'Eterno,
\par 15 quando avrò ridotto il paese d'Egitto in una desolazione, in un paese spogliato di ciò che conteneva, e quando ne avrò colpito tutti gli abitanti; e si conoscerà che io sono l'Eterno.
\par 16 Ecco la lamentazione che sarà pronunziata; la pronunzieranno le figliuole delle nazioni; pronunzieranno questa lamentazione sull'Egitto e su tutta la sua moltitudine, dice il Signore, l'Eterno'.
\par 17 Il dodicesimo anno, il quindicesimo giorno del mese, la parola dell'Eterno mi fu rivolta in questi termini:
\par 18 'Figliuol d'uomo, intuona un lamento sulla moltitudine dell'Egitto, e falle scendere, lei e le figliuole delle nazioni illustri, nelle profondità della terra, con quelli che scendon nella fossa.
\par 19 Chi mai sorpassi tu in bellezza? Scendi, e sarai posto a giacere con gl'incirconcisi!
\par 20 Essi cadranno in mezzo agli uccisi per la spada. La spada v'è data; trascinate l'Egitto con tutte le sue moltitudini!
\par 21 I più forti fra i prodi e quelli che gli davan soccorso gli rivolgeranno la parola, di mezzo al soggiorno de' morti. Sono scesi, gl'incirconcisi; giacciono uccisi dalla spada.
\par 22 Là è l'Assiro con tutta la sua moltitudine; attorno a lui stanno i suoi sepolcri; tutti son uccisi, caduti per la spada.
\par 23 I suoi sepolcri son posti nelle profondità della fossa, e la sua moltitudine sta attorno al suo sepolcro; tutti sono uccisi, caduti per la spada, essi che spandevano il terrore sulla terra de' viventi.
\par 24 Là è Elam con tutta la sua moltitudine, attorno al suo sepolcro; tutti sono uccisi, caduti per la spada, incirconcisi scesi nelle profondità della terra: essi, che spandevano il terrore sulla terra de' viventi; e han portato il loro obbrobrio con quelli che scendon nella fossa.
\par 25 Han fatto un letto, per lui e per la sua moltitudine, in mezzo a quelli che sono stati uccisi; attorno a lui stanno i suoi sepolcri; tutti costoro sono incirconcisi, sono morti per la spada, perché spandevano il terrore sulla terra de' viventi; e hanno portato il loro obbrobrio con quelli che scendono nella fossa; sono stati messi fra gli uccisi.
\par 26 Là è Mescec, Tubal e tutta la loro moltitudine; attorno a loro stanno i lor sepolcri; tutti costoro sono incirconcisi, uccisi dalla spada, perché spandevano il terrore sulla terra de' viventi.
\par 27 Non giacciono coi prodi che sono caduti fra gl'incirconcisi, che sono scesi nel soggiorno de' morti con le loro armi da guerra, e sotto il capo de' quali sono state poste le loro spade; ma le loro iniquità stanno sulle loro ossa, perché erano il terrore de' prodi sulla terra de' viventi.
\par 28 Tu pure sarai fiaccato in mezzo agl'incirconcisi e giacerai con gli uccisi dalla spada.
\par 29 Là è Edom coi suoi re e con tutti i suoi principi, i quali, nonostante tutto il loro valore, sono stati messi cogli uccisi di spada. Anch'essi giacciono con gl'incirconcisi e con quelli che scendono nella fossa.
\par 30 Là son tutti principi del settentrione e tutti i Sidonî, che son discesi in mezzo agli uccisi, coperti d'onta, nonostante il terrore che incuteva la loro bravura. Giacciono incirconcisi con gli uccisi di spada, e portano il loro obbrobrio con quelli che scendono nella fossa.
\par 31 Faraone li vedrà, e si consolerà d'aver perduto tutta la sua moltitudine; Faraone e tutto il suo esercito saranno uccisi per la spada, dice il Signore, l'Eterno,
\par 32 poiché io spanderò il mio terrore nella terra de' viventi; e Faraone con tutta la sua moltitudine sarà posto a giacere in mezzo agl'incirconcisi, con quelli che sono stati uccisi dalla spada, dice il Signore, l'Eterno'.

\chapter{33}

\par 1 E la parola dell'Eterno mi fu rivolta in questi termini:
\par 2 'Figliuol d'uomo, parla ai figliuoli del tuo popolo, e di' loro: Quando io farò venire la spada contro un paese, e il popolo di quel paese prenderà nel proprio seno un uomo e se lo stabilirà come sentinella,
\par 3 ed egli, vedendo venire la spada contro il paese, sonerà il corno e avvertirà il popolo,
\par 4 se qualcuno, pur udendo il suono del corno, non se ne cura, e la spada viene e lo porta via, il sangue di quel tale sarà sopra il suo capo;
\par 5 egli ha udito il suono del corno, e non se n'è curato; il suo sangue sarà sopra lui; se se ne fosse curato, avrebbe scampato la sua vita.
\par 6 Ma se la sentinella vede venir la spada e non suona il corno, e il popolo non è stato avvertito, e la spada viene e porta via qualcuno di loro, questi sarà portato via per la propria iniquità, ma io domanderò conto del suo sangue alla sentinella.
\par 7 Ora, o figliuol d'uomo, io ho stabilito te come sentinella per la casa d'Israele; quando dunque udrai qualche parola dalla mia bocca, avvertili da parte mia.
\par 8 Quando avrò detto all'empio: - Empio, per certo tu morrai! - e tu non avrai parlato per avvertir l'empio che si ritragga dalla sua via, quell'empio morrà per la sua iniquità, ma io domanderò conto del suo sangue alla tua mano.
\par 9 Ma, se tu avverti l'empio che si ritragga dalla sua via, e quegli non se ne ritrae, esso morrà per la sua iniquità, ma tu avrai scampato l'anima tua.
\par 10 E tu, figliuol d'uomo, di' alla casa d'Israele: Voi dite così: - Le nostre trasgressioni e i nostri peccati sono su noi, e a motivo d'essi noi languiamo: come potremmo noi vivere? -
\par 11 Di' loro: Com'è vero ch'io vivo, dice il Signore, l'Eterno, io non mi compiaccio della morte dell'empio, ma che l'empio si converta dalla sua via e viva; convertitevi, convertitevi dalle vostre vie malvagie! E perché morreste voi, o casa d'Israele?
\par 12 E tu, figliuol d'uomo, di' ai figliuoli del tuo popolo: La giustizia del giusto non lo salverà nel giorno della sua trasgressione; e l'empio non cadrà per la sua empietà nel giorno in cui si sarà ritratto dalla sua empietà; nello stesso modo che il giusto non potrà vivere per la sua giustizia nel giorno in cui peccherà.
\par 13 Quand'io avrò detto al giusto che per certo egli vivrà, s'egli confida nella propria giustizia e commette l'iniquità, tutti i suoi atti giusti non saranno più ricordati, e morrà per l'iniquità che avrà commessa.
\par 14 E quando avrò detto all'empio: - Per certo tu morrai, - s'egli si ritrae dal suo peccato e pratica ciò ch'è conforme al diritto e alla giustizia,
\par 15 se rende il pegno, se restituisce ciò che ha rapito, se cammina secondo i precetti che danno la vita, senza commettere l'iniquità, per certo egli vivrà, non morrà;
\par 16 tutti i peccati che ha commessi non saranno più ricordati contro di lui; egli ha praticato ciò ch'è conforme al diritto ed alla giustizia; per certo vivrà.
\par 17 Ma i figliuoli del tuo popolo dicono: - La via del Signore non è ben regolata; - ma è la via loro quella che non è ben regolata.
\par 18 Quando il giusto si ritrae dalla sua giustizia e commette l'iniquità, egli muore a motivo di questo;
\par 19 e quando l'empio si ritrae dalla sua empietà e si conduce secondo il diritto e la giustizia, a motivo di questo, vive.
\par 20 Voi dite: - La via del Signore non è ben regolata! - Io vi giudicherò ciascuno secondo le vostre vie, o casa d'Israele!'
\par 21 Il dodicesimo anno della nostra cattività, il decimo mese, il quinto giorno del mese, un fuggiasco da Gerusalemme venne a me, e mi disse: - La città è presa! -
\par 22 La sera avanti la venuta del fuggiasco, la mano dell'Eterno era stata sopra di me, ed egli m'aveva aperto la bocca, prima che quegli venisse a me la mattina; la bocca mi fu aperta, ed io non fui più muto.
\par 23 E la parola dell'Eterno mi fu rivolta in questi termini:
\par 24 'Figliuol d'uomo, gli abitanti di quelle rovine, nel paese d'Israele, dicono: - Abrahamo era solo, eppure ebbe il possesso del paese; e noi siamo molti, il possesso del paese è dato a noi.
\par 25 Perciò, di' loro: Così parla il Signore, l'Eterno: Voi mangiate la carne col sangue, alzate gli occhi verso i vostri idoli, spargete il sangue, e possedereste il paese?
\par 26 Voi v'appoggiate sulla vostra spada, commettete abominazioni, ciascun di voi contamina la moglie del prossimo, e possedereste il paese?
\par 27 Di' loro così: Così parla il Signore, l'Eterno: Com'è vero ch'io vivo, quelli che stanno fra quelle ruine cadranno per la spada; quelli che son per i campi li darò in pasto alle bestie; e quelli che son nelle fortezze e nelle caserme morranno di peste!
\par 28 E io ridurrò il paese in una desolazione, in un deserto; l'orgoglio della sua forza verrà meno, e i monti d'Israele saranno così desolati, che nessuno vi passerà più.
\par 29 Ed essi conosceranno che io sono l'Eterno, quando avrò ridotto il paese in una desolazione, in un deserto, per tutte le abominazioni che hanno commesse.
\par 30 E quant'è a te, figliuol d'uomo, i figliuoli del tuo popolo discorrono di te presso le mura e sulle porte delle case; e parlano l'uno con l'altro e ognuno col suo fratello, e dicono: - Venite dunque ad ascoltare qual è la parola che procede dall'Eterno! -
\par 31 E vengon da te come fa la folla, e il mio popolo si siede davanti a te, e ascolta le tue parole, ma non le mette in pratica; perché, con la bocca fa mostra di molto amore, ma il suo cuore va dietro alla sua cupidigia.
\par 32 Ed ecco, tu sei per loro come una canzone d'amore d'uno che abbia una bella voce, e sappia suonar bene; essi ascoltano le tue parole, ma non le mettono in pratica;
\par 33 ma quando la cosa avverrà - ed ecco che sta per avvenire - essi conosceranno che in mezzo a loro c'è stato un profeta'.

\chapter{34}

\par 1 E la parola dell'Eterno mi fu rivolta in questi termini:
\par 2 'Figliuol d'uomo, profetizza contro i pastori d'Israele; profetizza, e di' a quei pastori: Così parla il Signore, l'Eterno: Guai ai pastori d'Israele, che non han fatto se non pascer se stessi! Non è forse il gregge quello che i pastori debbon pascere?
\par 3 Voi mangiate il latte, vi vestite della lana, ammazzate ciò ch'è ingrassato, ma non pascete il gregge.
\par 4 Voi non avete fortificato le pecore deboli, non avete guarito la malata, non avete fasciato quella ch'era ferita, non avete ricondotto la smarrita, non avete cercato la perduta, ma avete dominato su loro con violenza e con asprezza.
\par 5 Ed esse, per mancanza di pastore, si sono disperse, son diventate pasto a tutte le fiere dei campi, e si sono disperse.
\par 6 Le mie pecore vanno errando per tutti i monti e per ogni alto colle; le mie pecore si disperdono su tutta la faccia del paese, e non v'è alcuno che ne domandi, alcuno che le cerchi!
\par 7 Perciò, o pastori, ascoltate la parola dell'Eterno!
\par 8 Com'è vero ch'io vivo, dice il Signore, l'Eterno, poiché le mie pecore sono abbandonate alla rapina; poiché le mie pecore, essendo senza pastore, servon di pasto a tutte le fiere de' campi, e i miei pastori non cercano le mie pecore; poiché i pastori pascon se stessi e non pascono le mie pecore,
\par 9 perciò, ascoltate, o pastori, la parola dell'Eterno!
\par 10 Così parla il Signore, l'Eterno: Eccomi contro i pastori; io ridomanderò le mie pecore alle loro mani; li farò cessare dal pascer le pecore; i pastori non pasceranno più se stessi; io strapperò le mie pecore dalla loro bocca, ed esse non serviran più loro di pasto.
\par 11 Poiché, così dice il Signore, l'Eterno: Eccomi! io stesso domanderò delle mie pecore, e ne andrò in cerca.
\par 12 Come un pastore va in cerca del suo gregge il giorno che si trova in mezzo alle sue pecore disperse, così io andrò in cerca delle mie pecore, e le ritrarrò da tutti i luoghi dove sono state disperse in un giorno di nuvole e di tenebre;
\par 13 e le trarrò di fra i popoli e le radunerò dai diversi paesi, e le ricondurrò sul loro suolo, e le pascerò sui monti d'Israele, lungo i ruscelli e in tutti i luoghi abitati del paese.
\par 14 Io le pascerò in buoni pascoli, e i loro ovili saranno sugli alti monti d'Israele; esse riposeranno quivi in buoni ovili, e pascoleranno in grassi pascoli sui monti d'Israele.
\par 15 Io stesso pascerò le mie pecore, e io stesso le farò riposare, dice il Signore, l'Eterno.
\par 16 Io cercherò la perduta, ricondurrò la smarrita, fascerò la ferita, fortificherò la malata, ma distruggerò la grassa e la forte: io le pascerò con giustizia.
\par 17 E quant'è a voi, o pecore mie, così dice il Signore, l'Eterno: Ecco, io giudicherò fra pecora e pecora, fra montoni e capri.
\par 18 Vi par egli troppo poco il pascolare in questo buon pascolo, che abbiate a pestare co' piedi ciò che rimane del vostro pascolo? Il bere le acque più chiare, che abbiate a intorbidare co' piedi quel che ne resta?
\par 19 E le mie pecore hanno per pascolo quello che i vostri piedi han calpestato; e devono bere, ciò che i vostri piedi hanno intorbidato!
\par 20 Perciò, così dice loro il Signore, l'Eterno: Eccomi, io stesso giudicherò fra la pecora grassa e la pecora magra.
\par 21 Siccome voi avete spinto col fianco e con la spalla e avete cozzato con le corna tutte le pecore deboli finché non le avete disperse e cacciate fuori,
\par 22 io salverò le mie pecore, ed esse non saranno più abbandonate alla rapina; e giudicherò fra pecora e pecora.
\par 23 E susciterò sopra d'esse un solo pastore, che le pascolerà: il mio servo Davide; egli le pascolerà, egli sarà il loro pastore.
\par 24 E io, l'Eterno, sarò il loro Dio, e il mio servo Davide sarà principe in mezzo a loro. Io, l'Eterno, son quegli che ho parlato.
\par 25 E fermerò con esse un patto di pace; farò sparire le male bestie dal paese, e le mie pecore dimoreranno al sicuro nel deserto e dormiranno nelle foreste.
\par 26 E farò ch'esse e i luoghi attorno al mio colle saranno una benedizione; farò scender la pioggia a suo tempo, e saran piogge di benedizione.
\par 27 L'albero dei campi darà il suo frutto, e la terra darà i suoi prodotti. Esse staranno al sicuro sul loro suolo, e conosceranno che io sono l'Eterno, quando spezzerò le sbarre del loro giogo e le libererò dalla mano di quelli che le tenevano schiave.
\par 28 E non saranno più preda alle nazioni; le fiere dei campi non le divoreranno più, ma se ne staranno al sicuro, senza che nessuno più le spaventi.
\par 29 E farò sorgere per loro una vegetazione, che le farà salire in fama; e non saranno più consumate dalla fame nel paese, e non porteranno più l'obbrobrio delle nazioni.
\par 30 E conosceranno che io, l'Eterno, l'Iddio loro, sono con esse, e che esse, la casa d'Israele, sono il mio popolo, dice il Signore, l'Eterno.
\par 31 E voi, pecore mie, pecore del mio pascolo, siete uomini, e io sono il vostro Dio, dice l'Eterno'.

\chapter{35}

\par 1 E la parola dell'Eterno mi fu rivolta in questi termini:
\par 2 'Figliuol d'uomo, volgi la tua faccia verso il monte di Seir, e profetizza contro di esso,
\par 3 e digli: Così parla il Signore, l'Eterno: Eccomi a te, o monte di Seir! Io stenderò la mia mano contro di te, e ti renderò una solitudine, un deserto.
\par 4 Io ridurrò le tue città in rovine, tu diventerai una solitudine, e conoscerai che io sono l'Eterno.
\par 5 Poiché tu hai avuto una inimicizia eterna e hai abbandonato i figliuoli d'Israele in balìa della spada nel giorno della loro calamità, nel giorno che l'iniquità era giunta al colmo,
\par 6 com'è vero ch'io vivo, dice il Signore, l'Eterno, io ti metterò a sangue, e il sangue t'inseguirà; giacché non hai avuto in odio il sangue, il sangue t'inseguirà.
\par 7 E ridurrò il monte di Seir in una solitudine, in un deserto, e ne sterminerò chi va e chi viene.
\par 8 Io riempirò i suoi monti dei suoi uccisi; sopra i tuoi colli, nelle tue valli, in tutti i tuoi burroni cadranno gli uccisi dalla spada.
\par 9 Io ti ridurrò in una desolazione perpetua, e le tue città non saranno più abitate; e voi conoscerete che io sono l'Eterno.
\par 10 Siccome tu hai detto: - Quelle due nazioni e que' due paesi saranno miei, e noi ne prenderemo possesso - (e l'Eterno era quivi!),
\par 11 com'è vero ch'io vivo, dice il Signore, l'Eterno, io agirò con l'ira e con la gelosia, che tu hai mostrate nel tuo odio contro di loro; e mi farò conoscere in mezzo a loro, quando ti giudicherò.
\par 12 Tu conoscerai che io, l'Eterno, ho udito tutti gli oltraggi che hai proferiti contro i monti d'Israele, dicendo: - Essi son desolati; son dati a noi, perché ne facciam nostra preda. -
\par 13 Voi, con la vostra bocca, vi siete inorgogliti contro di me, e avete moltiplicato contro di me i vostri discorsi. Io l'ho udito!
\par 14 Così parla il Signore, l'Eterno: Quando tutta la terra si rallegrerà, io ti ridurrò in una desolazione.
\par 15 Siccome tu ti sei rallegrato perché l'eredità della casa d'Israele era devastata, io farò lo stesso di te: tu diventerai una desolazione, o monte di Seir: tu, e Edom tutto quanto; e si conoscerà che io sono l'Eterno.

\chapter{36}

\par 1 E tu, figliuol d'uomo, profetizza ai monti d'Israele, e di': O monti d'Israele, ascoltate la parola dell'Eterno!
\par 2 Così parla il Signore, l'Eterno: Poiché il nemico ha detto di voi: - Ah! ah! Queste alture eterne sono diventate nostro possesso! - tu profetizza, e di':
\par 3 Così parla il Signore, l'Eterno: Sì, poiché da tutte le parti han voluto distruggervi e inghiottirvi perché diventaste possesso del resto delle nazioni, e perché siete stati oggetto de' discorsi delle male lingue e delle maldicenze della gente,
\par 4 perciò, o monti d'Israele, ascoltate la parola del Signore, dell'Eterno! Così parla il Signore, l'Eterno, ai monti e ai colli, ai burroni ed alle valli, alle ruine desolate e alle città abbandonate, che sono state date in balìa del saccheggio e delle beffe delle altre nazioni d'ogn'intorno;
\par 5 così parla il Signore, l'Eterno: Sì, nel fuoco della mia gelosia, io parlo contro il resto delle altre nazioni e contro Edom tutto quanto, che hanno fatto del mio paese il loro possesso con tutta la gioia del loro cuore e con tutto lo sprezzo dell'anima loro, per ridurlo in bottino.
\par 6 Perciò, profetizza sopra la terra d'Israele, e di' ai monti e ai colli, ai burroni ed alle valli: Così parla il Signore, l'Eterno: Ecco, io parlo nella mia gelosia e nel mio furore, perché voi portate l'obbrobrio delle nazioni.
\par 7 Perciò, così parla il Signore, l'Eterno: Io l'ho giurato! Le nazioni che vi circondano porteranno anch'esse il loro obbrobrio;
\par 8 ma voi, o monti d'Israele, metterete i vostri rami e porterete i vostri frutti al mio popolo d'Israele, perch'egli sta per arrivare.
\par 9 Poiché, ecco, io vengo a voi, mi volgerò verso voi, e voi sarete coltivati e seminati;
\par 10 io moltiplicherò su voi gli uomini, tutta quanta la casa d'Israele; le città saranno abitate e le ruine saranno ricostruite;
\par 11 moltiplicherò su voi uomini e bestie; essi moltiplicheranno e saranno fecondi, e farò sì che sarete abitati com'eravate prima, e vi farò del bene più che nei vostri primi tempi; e voi conoscerete che io sono l'Eterno.
\par 12 Io farò camminar su voi degli uomini, il mio popolo d'Israele. Essi ti possederanno, o paese; tu sarai la loro eredità, e non li priverai più de' loro figliuoli.
\par 13 Così parla il Signore, l'Eterno: Poiché vi si dice: - Tu, o paese, hai divorato gli uomini, hai privato la tua nazione dei suoi figliuoli, -
\par 14 tu non divorerai più gli uomini, e non priverai più la tua nazione de' suoi figliuoli, dice il Signore, l'Eterno.
\par 15 Io non ti farò più udire gli oltraggi delle nazioni, e tu non porterai più l'obbrobrio dei popoli, e non farai più cader la tua gente, dice il Signore, l'Eterno'.
\par 16 E la parola dell'Eterno mi fu rivolta in questi termini:
\par 17 'Figliuol d'uomo, quando quelli della casa d'Israele abitavano il loro paese, lo contaminavano con la loro condotta e con le loro azioni; la loro condotta era nel mio cospetto come la immondezza della donna quand'è impura.
\par 18 Ond'io riversai su loro il mio furore a motivo del sangue che aveano sparso sul paese, e perché l'aveano contaminato coi loro idoli;
\par 19 e li dispersi fra le nazioni, ed essi furono sparsi per tutti i paesi; io li giudicai secondo la loro condotta e secondo le loro azioni.
\par 20 E, giunti fra le nazioni dove sono andati, hanno profanato il nome mio santo, giacché si diceva di loro: - Costoro sono il popolo dell'Eterno, e sono usciti dal suo paese. -
\par 21 Ed io ho avuto pietà del nome mio santo, che la casa d'Israele profanava fra le nazioni dov'è andata.
\par 22 Perciò, di' alla casa d'Israele: Così parla il Signore, l'Eterno: Io agisco così, non per cagion di voi, o casa d'Israele, ma per amore del nome mio santo, che voi avete profanato fra le nazioni dove siete andati.
\par 23 E io santificherò il mio gran nome che è stato profanato fra le nazioni, in mezzo alle quali voi l'avete profanato; e le nazioni conosceranno che io sono l'Eterno, dice il Signore, l'Eterno, quand'io mi santificherò in voi, sotto gli occhi loro.
\par 24 Io vi trarrò di fra le nazioni, vi radunerò da tutti i paesi, e vi ricondurrò nel vostro paese;
\par 25 v'aspergerò d'acqua pura, e sarete puri; io vi purificherò di tutte le vostre impurità e di tutti i vostri idoli.
\par 26 E vi darò un cuor nuovo, e metterò dentro di voi uno spirito nuovo; torrò dalla vostra carne il cuore di pietra, e vi darò un cuore di carne.
\par 27 Metterò dentro di voi il mio spirito, e farò sì che camminerete secondo le mie leggi, e osserverete e metterete in pratica le mie prescrizioni.
\par 28 E voi abiterete nel paese ch'io detti ai vostri padri, e voi sarete mio popolo, e io sarò vostro Dio.
\par 29 Io vi libererò da tutte le vostre impurità; chiamerò il frumento, lo farò abbondare, e non manderò più contro di voi la fame;
\par 30 e farò moltiplicare il frutto degli alberi e il prodotto de' campi, affinché non siate più esposti all'obbrobrio della fame tra le nazioni.
\par 31 Allora vi ricorderete delle vostre vie malvage e delle vostre azioni, che non eran buone, e prenderete disgusto di voi stessi a motivo delle vostre iniquità e delle vostre abominazioni.
\par 32 Non è per amor di voi, che agisco così, dice il Signore, l'Eterno: siavi pur noto! Vergognatevi, e siate confusi a motivo delle vostre vie, o casa d'Israele!
\par 33 Così parla il Signore, l'Eterno: Il giorno che io vi purificherò di tutte le vostre iniquità, farò sì che le città saranno abitate, e le ruine saranno ricostruite;
\par 34 la terra desolata sarà coltivata, invece d'essere una desolazione agli occhi di tutti i passanti;
\par 35 e si dirà: Questa terra ch'era desolata, è divenuta come il giardino d'Eden; e queste città ch'erano deserte, desolate, ruinate, sono fortificate e abitate.
\par 36 E le nazioni che saran rimaste attorno a voi conosceranno che io, l'Eterno, son quegli che ho ricostruito i luoghi ruinati, e ripiantato il luogo deserto. Io, l'Eterno, son quegli che parlo, e che mando la cosa ad effetto.
\par 37 Così parla il Signore, l'Eterno: Anche in questo mi lascerò supplicare dalla casa d'Israele, e glielo concederò: io moltiplicherò loro gli uomini come un gregge.
\par 38 Come greggi di pecore consacrate, come i greggi di Gerusalemme nelle sue feste solenni, così le città deserte saranno riempite di greggi d'uomini; e si conoscerà che io sono l'Eterno'.

\chapter{37}

\par 1 La mano dell'Eterno fu sopra di me, e l'Eterno mi trasportò in ispirito, e mi depose in mezzo a una valle ch'era piena d'ossa.
\par 2 E mi fece passare presso d'esse, tutt'attorno; ed ecco erano numerosissime sulla superficie della valle ed erano anche molto secche.
\par 3 E mi disse: 'Figliuol d'uomo, queste ossa potrebbero esse rivivere?' E io risposi: 'O Signore, o Eterno, tu il sai'.
\par 4 Ed egli mi disse: 'Profetizza su queste ossa, e di' loro: Ossa secche, ascoltate la parola dell'Eterno!
\par 5 Così dice il Signore, l'Eterno, a queste ossa: Ecco, io faccio entrare in voi lo spirito, e voi rivivrete;
\par 6 e metterò su voi de' muscoli, farò nascere su voi della carne, vi coprirò di pelle, metterò in voi lo spirito, e rivivrete; e conoscerete che io sono l'Eterno'.
\par 7 E io profetizzai come mi era stato comandato; e come io profetizzavo, si fece un rumore; ed ecco un movimento, e le ossa si accostarono le une alle altre.
\par 8 Io guardai, ed ecco venir su d'esse de' muscoli, crescervi della carne, e la pelle ricoprirle; ma non c'era in esse spirito alcuno.
\par 9 Allora egli mi disse: 'Profetizza allo spirito, profetizza, figliuol d'uomo, e di' allo spirito: Così parla il Signore, l'Eterno: Vieni dai quattro venti, o spirito, soffia su questi uccisi, e fa' che rivivano!'
\par 10 E io profetizzai, com'egli m'aveva comandato; e lo spirito entrò in essi, e tornarono alla vita, e si rizzarono in piedi: erano un esercito grande, grandissimo.
\par 11 Ed egli mi disse: 'Figliuol d'uomo, queste ossa sono tutta la casa d'Israele. Ecco, essi dicono: - Le nostre ossa sono secche, la nostra speranza è perita noi siam perduti! -
\par 12 Perciò, profetizza e di' loro: Così parla il Signore, l'Eterno: Ecco, io aprirò i vostri sepolcri, vi trarrò fuori dalle vostre tombe, o popolo mio, e vi ricondurrò nel paese d'Israele.
\par 13 E voi conoscerete che io sono l'Eterno, quando aprirò i vostri sepolcri e vi trarrò fuori dalle vostre tombe, o popolo mio!
\par 14 E metterò in voi il mio spirito, e voi tornerete alla vita; vi porrò sul vostro suolo, e conoscerete che io, l'Eterno, ho parlato e ho messo la cosa ad effetto, dice l'Eterno'.
\par 15 E la parola dell'Eterno mi fu rivolta in questi termini:
\par 16 'E tu, figliuol d'uomo, prenditi un pezzo di legno, e scrivici sopra: - Per Giuda e per i figliuoli d'Israele, che gli sono associati. - Poi prenditi un altro pezzo di legno, e scrivici sopra: - Per Giuseppe, bastone d'Efraim e di tutta la casa d'Israele, che gli è associata. -
\par 17 Poi accostali l'uno all'altro per farne un solo pezzo di legno, in modo che siano uniti nella tua mano.
\par 18 E quando i figliuoli del tuo popolo ti parleranno e ti diranno: - Non ci spiegherai tu che cosa vuoi dire con queste cose? -
\par 19 tu rispondi loro: Così parla il Signore, l'Eterno: Ecco, io prenderò il pezzo di legno di Giuseppe ch'è in mano d'Efraim e le tribù d'Israele che sono a lui associate, e li unirò a questo, ch'è il pezzo di legno di Giuda, e ne farò un solo legno, in modo che saranno una sola cosa nella mia mano.
\par 20 E i legni sui quali tu avrai scritto, li terrai in mano tua, sotto i loro occhi.
\par 21 E di' loro: Così parla il Signore, l'Eterno: Ecco, io prenderò i figliuoli d'Israele di fra le nazioni dove sono andati, li radunerò da tutte le parti, e li ricondurrò nel loro paese;
\par 22 e farò di loro una stessa nazione, nel paese, sui monti d'Israele; un solo re sarà re di tutti loro; e non saranno più due nazioni, e non saranno più divisi in due regni.
\par 23 E non si contamineranno più coi loro idoli, con le loro abominazioni né colle loro numerose trasgressioni; io li trarrò fuori da tutti i luoghi dove hanno abitato e dove hanno peccato, e li purificherò; essi saranno mio popolo, e io sarò loro Dio.
\par 24 Il mio servo Davide sarà re sopra loro, ed essi avranno tutti un medesimo pastore; cammineranno secondo le mie prescrizioni, osserveranno le mie leggi, e le metteranno in pratica;
\par 25 e abiteranno nel paese che io detti al mio servo Giacobbe, e dove abitarono i vostri padri; vi abiteranno essi, i loro figliuoli e i figliuoli dei loro figliuoli in perpetuo; e il mio servo Davide sarà loro principe in perpetuo.
\par 26 E io fermerò con loro un patto di pace: sarà un patto perpetuo con loro; li stabilirò fermamente, li moltiplicherò, e metterò il mio santuario in mezzo a loro per sempre;
\par 27 la mia dimora sarà presso di loro, e io sarò loro Dio, ed essi saranno mio popolo.
\par 28 E le nazioni conosceranno che io sono l'Eterno che santifico Israele, quando il mio santuario sarà per sempre in mezzo ad essi.

\chapter{38}

\par 1 E la parola dell'Eterno mi fu rivolta in questi termini:
\par 2 'Figliuol d'uomo, volgi la tua faccia verso Gog del paese di Magog, principe sovrano di Mescec e di Tubal, e profetizza contro di lui, e di':
\par 3 Così parla il Signore, l'Eterno: Eccomi da te, o Gog, principe sovrano di Mescec e di Tubal!
\par 4 Io ti menerò via, ti metterò degli uncini nelle mascelle e ti trarrò fuori, te e tutto il tuo esercito, cavalli e cavalieri, tutti quanti vestiti pomposamente, gran moltitudine con targhe e scudi, tutti maneggianti la spada;
\par 5 e con loro Persiani, Etiopi e gente di Put, tutti con scudi ed elmi.
\par 6 Gomer e tutte le sue schiere, la casa di Togarma dell'estremità del settentrione e tutte le sue schiere, de' popoli numerosi saranno con te.
\par 7 Mettiti in ordine, prepàrati, tu con tutte le tue moltitudini che s'adunano attorno a te, e sii tu per essi colui al quale si ubbidisce.
\par 8 Dopo molti giorni tu riceverai l'ordine; negli ultimi anni verrai contro il paese sottratto alla spada, contro la nazione raccolta di fra molti popoli sui monti d'Israele, che sono stati per tanto tempo deserti; ma, tratta fuori di fra i popoli, essa abiterà tutta quanta al sicuro.
\par 9 Tu salirai, verrai come un uragano, sarai come una nuvola che sta per coprire il paese, tu con tutte le tue schiere e coi popoli numerosi che son teco.
\par 10 Così parla il Signore, l'Eterno: In quel giorno, de' pensieri ti sorgeranno in cuore, e concepirai un malvagio disegno.
\par 11 Dirai: - Io salirò contro questo paese di villaggi aperti; piomberò su questa gente che vive tranquilla ed abita al sicuro, che dimora tutta in luoghi senza mura, e non ha né sbarre né porte.
\par 12 Verrai per far bottino e predare, per stendere la tua mano contro queste ruine ora ripopolate, contro questo popolo raccolto di fra le nazioni, che s'è procurato bestiame e facoltà, e dimora sulle alture del paese.
\par 13 Sceba, Dedan, i mercanti di Tarsis e tutti i suoi leoncelli ti diranno: - Vieni tu per far bottino? Hai tu adunato la tua moltitudine per predare, per portar via l'argento e l'oro, per pigliare bestiame e beni, per fare un gran bottino? -
\par 14 Perciò, figliuol d'uomo, profetizza, e di' a Gog: Così parla il Signore, l'Eterno: In quel giorno, quando il mio popolo d'Israele dimorerà al sicuro, tu lo saprai;
\par 15 e verrai dal luogo dove stai, dall'estremità del settentrione, tu con de' popoli numerosi teco, tutti quanti a cavallo, una grande moltitudine, un potente esercito;
\par 16 e salirai contro il mio popolo d'Israele, come una nuvola che sta per coprire il paese. Questo avverrà alla fine de' giorni: io ti condurrò contro il mio paese affinché le nazioni mi conoscano, quand'io mi santificherò in te sotto gli occhi di loro, o Gog!
\par 17 Così parla il Signore, l'Eterno: non sei tu quello del quale io parlai ai tempi antichi mediante i miei servi, i profeti d'Israele, i quali profetarono allora per degli anni che io ti farei venire contro di loro?
\par 18 In quel giorno, nel giorno che Gog verrà contro la terra d'Israele, dice il Signore, l'Eterno, il mio furore mi monterà nelle narici;
\par 19 e nella mia gelosia, e nel fuoco della mia ira, io lo dico, certo, in quel giorno, vi sarà un gran commovimento nel paese d'Israele:
\par 20 i pesci del mare, gli uccelli del cielo, le bestie de' campi, tutti i rettili che strisciano sul suolo e tutti gli uomini che sono sulla faccia della terra, tremeranno alla mia presenza; i monti saranno rovesciati, le balze crolleranno, e tutte le mura cadranno al suolo.
\par 21 Io chiamerò contro di lui la spada su tutti i miei monti, dice il Signore, l'Eterno; la spada d'ognuno si volgerà contro il suo fratello.
\par 22 E verrò in giudizio contro di lui, con la peste e col sangue; e farò piovere torrenti di pioggia e grandine, e fuoco e zolfo su lui, sulle sue schiere e sui popoli numerosi che saranno con lui.
\par 23 Così mi magnificherò e mi santificherò e mi farò conoscere agli occhi di molte nazioni, ed esse sapranno che io sono l'Eterno.

\chapter{39}

\par 1 E tu, figliuol d'uomo, profetizza contro Gog, e di': Così parla il Signore, l'Eterno: Eccomi da te, o Gog, principe sovrano di Mescec e di Tubal!
\par 2 Io ti menerò via, ti spingerò innanzi, ti farò salire dalle estremità del settentrione e ti condurrò sui monti d'Israele;
\par 3 butterò giù l'arco dalla tua mano sinistra, e ti farò cadere le frecce dalla destra.
\par 4 Tu cadrai sui monti d'Israele, tu con tutte le tue schiere e coi popoli che saranno teco; ti darò in pasto agli uccelli rapaci, agli uccelli d'ogni specie, e alle bestie de' campi.
\par 5 Tu cadrai sulla faccia de' campi, poiché io ho parlato, dice il Signore, l'Eterno.
\par 6 E manderò il fuoco su Magog e su quelli che abitano sicuri nelle isole; e conosceranno che io sono l'Eterno.
\par 7 E farò conoscere il mio nome santo in mezzo al mio popolo d'Israele, e non lascerò più profanare il mio nome santo; e le nazioni conosceranno che io sono l'Eterno, il Santo in Israele.
\par 8 Ecco, la cosa sta per avvenire, si effettuerà, dice il Signore, l'Eterno; questo è il giorno di cui io ho parlato.
\par 9 E gli abitanti delle città d'Israele usciranno e faranno de' fuochi, bruciando armi, scudi, targhe, archi, frecce, picche e lance; e ne faranno del fuoco per sette anni;
\par 10 non porteranno legna dai campi, e non ne taglieranno nelle foreste; giacché faran del fuoco con quelle armi; e spoglieranno quelli che li spogliavano, e prederanno quelli che li predavano, dice il Signore, l'Eterno.
\par 11 In quel giorno, io darò a Gog un luogo che gli servirà di sepoltura in Israele, la Valle de' viandanti, a oriente del mare; e quel sepolcro chiuderà la via ai viandanti; quivi sarà sepolto Gog con tutta la sua moltitudine; e quel luogo sarà chiamato la Valle d'Hamon-Gog.
\par 12 La casa d'Israele li sotterrerà per purificare il paese; e ciò durerà sette mesi.
\par 13 Tutto il popolo del paese li sotterrerà; e per questo ei salirà in fama il giorno in cui mi glorificherò, dice il Signore, l'Eterno.
\par 14 E metteranno da parte degli uomini i quali percorreranno del continuo il paese a sotterrare, con l'aiuto de' viandanti, i corpi che saran rimasti sul suolo del paese, per purificarlo; alla fine dei sette mesi faranno questa ricerca.
\par 15 E quando i viandanti passeranno per il paese, chiunque di loro vedrà delle ossa umane, rizzerà lì vicino un segnale, finché i seppellitori non le abbiano sotterrate nella Valle di Hamon-Gog.
\par 16 E Hamonah sarà pure il nome d'una città. E così purificheranno il paese.
\par 17 E tu, figliuol d'uomo, così parla il Signore, l'Eterno: Di' agli uccelli d'ogni specie e a tutte le bestie dei campi: Riunitevi, e venite! Raccoglietevi da tutte le parti attorno al banchetto del sacrificio che sto per immolare per voi, del gran sacrificio sui monti d'Israele! Voi mangerete carne e berrete sangue.
\par 18 Mangerete carne di prodi e berrete sangue di principi della terra: montoni, agnelli, capri, giovenchi, tutti quanti ingrassati in Basan.
\par 19 Mangerete del grasso a sazietà, e berrete del sangue fino a inebriarvi, al banchetto del sacrificio che io immolerò per voi;
\par 20 e alla mia mensa sarete saziati di carne di cavalli e di bestie da tiro, di prodi e di guerrieri d'ogni sorta, dice il Signore, l'Eterno.
\par 21 E io manifesterò la mia gloria fra le nazioni, e tutte le nazioni vedranno il giudizio che io eseguirò, e la mia mano che metterò su loro.
\par 22 E da quel giorno in poi la casa d'Israele conoscerà che io sono l'Eterno, il suo Dio;
\par 23 e le nazioni conosceranno che la casa d'Israele è stata menata in cattività a motivo della sua iniquità, perché m'era stata infedele; ond'io ho nascosto a loro la mia faccia, e li ho dati in mano de' loro nemici; e tutti quanti sono caduti per la spada.
\par 24 Io li ho trattati secondo la loro impurità e secondo le loro trasgressioni, e ho nascosto loro la mia faccia.
\par 25 Perciò, così parla il Signore, l'Eterno: Ora io farò tornare Giacobbe dalla cattività, e avrò pietà di tutta la casa d'Israele, e sarò geloso del mio santo nome.
\par 26 Ed essi avran finito di portare il loro obbrobrio e la pena di tutte le infedeltà che hanno commesse contro di me, quando dimoreranno al sicuro nel loro paese, e non vi sarà più alcuno che li spaventi;
\par 27 quando li ricondurrò di fra i popoli e li raccoglierò dai paesi de' loro nemici, e mi santificherò in loro in presenza di molte nazioni;
\par 28 ed essi conosceranno che io sono l'Eterno, il loro Dio, quando, dopo averli fatti andare in cattività fra le nazioni, li avrò raccolti nel loro paese; e non lascerò là più alcuno d'essi;
\par 29 e non nasconderò più loro la mia faccia, perché avrò sparso il mio spirito sulla casa d'Israele, dice il Signore, l'Eterno'.

\chapter{40}

\par 1 L'anno venticinquesimo della nostra cattività, al principio dell'anno, il decimo giorno del mese, quattordici anni dopo la presa della città, in quello stesso giorno, la mano dell'Eterno fu sopra me, ed egli mi trasportò nel paese d'Israele.
\par 2 In una visione divina mi trasportò là, e mi posò sopra un monte altissimo, sul quale stava, dal lato di mezzogiorno, come la costruzione di una città.
\par 3 Egli mi menò là, ed ecco che v'era un uomo, il cui aspetto era come aspetto di rame; aveva in mano una corda di lino e una canna da misurare, e stava in piè sulla porta.
\par 4 E quell'uomo mi disse: 'Figliuol d'uomo, apri gli occhi e guarda, porgi l'orecchio e ascolta, e poni mente a tutte le cose che io ti mostrerò; poiché tu sei stato menato qua perché io te le mostri. Riferisci alla casa d'Israele tutto quello che vedrai'.
\par 5 Ed ecco un muro esterno circondava la casa d'ogn'intorno. L'uomo aveva in mano una canna da misurare, lunga sei cubiti, ogni cubito d'un cubito e un palmo. Egli misurò la larghezza del muro, ed era una canna; l'altezza, ed era una canna.
\par 6 Poi venne alla porta che guardava verso oriente, ne salì la gradinata, e misurò la soglia della porta, ch'era della larghezza d'una canna: questa prima soglia aveva la larghezza d'una canna.
\par 7 Ogni camera di guardia aveva una canna di lunghezza, e una canna di larghezza. Fra le camere era uno spazio di cinque cubiti. La soglia della porta verso il vestibolo della porta, dal lato della casa, era d'una canna.
\par 8 Misurò il vestibolo della porta dal lato della casa, ed era una canna.
\par 9 Misurò il vestibolo della porta, ed era otto cubiti; i suoi pilastri, ed erano due cubiti. Il vestibolo della porta era dal lato della casa.
\par 10 Le camere di guardia della porta orientale erano tre da un lato e tre dall'altro; tutte e tre avevano la stessa misura; e i pilastri, da ogni lato, avevano pure la stessa misura.
\par 11 Misurò la larghezza dell'apertura della porta, ed era dieci cubiti; e la lunghezza della porta, ed era tredici cubiti.
\par 12 E davanti alle camere c'era una chiusura d'un cubito da un lato, e una chiusura d'un cubito dall'altro; e ogni camera aveva sei cubiti da un lato, e sei dall'altro.
\par 13 E misurò la porta dal tetto d'una delle camere al tetto dell'altra; e c'era una larghezza di venticinque cubiti, da porta a porta.
\par 14 Contò sessanta cubiti per i pilastri, e dopo i pilastri veniva il cortile tutt'attorno alle porte.
\par 15 Lo spazio fra la porta d'ingresso e il vestibolo della porta interna era di cinquanta cubiti.
\par 16 E c'erano delle finestre, con delle grate, alle camere e ai loro pilastri, verso l'interno della porta, tutt'all'intorno; lo stesso agli archi; così c'erano delle finestre tutt'all'intorno, verso l'interno; e sopra i pilastri c'erano delle palme.
\par 17 Poi mi menò nel cortile esterno, ed ecco c'erano delle camere, e un lastrico tutt'all'intorno del cortile: trenta camere davano su quel lastrico.
\par 18 Il lastrico era allato alle porte, e corrispondeva alla lunghezza delle porte; era il lastrico inferiore.
\par 19 Poi misurò la larghezza, dal davanti della porta inferiore fino alla cinta del cortile interno: cento cubiti a oriente e a settentrione.
\par 20 Misurò la lunghezza e la larghezza della porta settentrionale del cortile esterno;
\par 21 le sue camere di guardia erano tre di qua e tre di là; i suoi pilastri e i suoi archi avevano la stessa misura della prima porta: cinquanta cubiti di lunghezza e venticinque di larghezza.
\par 22 Le sue finestre, i suoi archi, le sue palme avevano la stessa misura della porta orientale; vi si saliva per sette gradini, davanti ai quali stavano i suoi archi.
\par 23 Al cortile interno c'era una porta difaccia alla porta settentrionale e difaccia alla porta orientale; ed egli misurò da porta a porta: cento cubiti.
\par 24 Poi mi menò verso mezzogiorno, ed ecco una porta che guardava a mezzogiorno; egli ne misurò i pilastri e gli archi, che avevano le stesse dimensioni.
\par 25 Questa porta e i suoi archi avevano delle finestre tutt'all'intorno, come le altre finestre: cinquanta cubiti di lunghezza e venticinque cubiti di larghezza.
\par 26 Vi si saliva per sette gradini, davanti ai quali stavano gli archi; ed essa aveva le sue palme, una di qua e una di là sopra i suoi pilastri.
\par 27 E il cortile interno aveva una porta dal lato di mezzogiorno; ed egli misurò da porta a porta, in direzione di mezzogiorno, cento cubiti.
\par 28 Poi mi menò nel cortile interno per la porta di mezzogiorno, e misurò la porta di mezzogiorno, che aveva quelle stesse dimensioni.
\par 29 Le sue camere di guardia, i suoi pilastri e i suoi archi avevano le stesse dimensioni. Questa porta e i suoi archi avevano delle finestre tutt'all'intorno; aveva cinquanta cubiti di lunghezza e venticinque di larghezza.
\par 30 E c'erano tutt'all'intorno degli archi di venticinque cubiti di lunghezza e di cinque cubiti di larghezza.
\par 31 Gli archi della porta erano dal lato del cortile esterno, c'erano delle palme sui suoi pilastri, e vi si saliva per otto gradini.
\par 32 Poi mi menò nel cortile interno per la porta orientale, e misurò la porta, che aveva le stesse dimensioni.
\par 33 Le sue camere, i suoi pilastri e i suoi archi avevano quelle stesse dimensioni. Questa porta e i suoi archi avevano tutt'all'intorno delle finestre; aveva cinquanta cubiti di lunghezza e venticinque cubiti di larghezza.
\par 34 Gli archi della porta erano dal lato del cortile esterno, c'erano delle palme sui suoi pilastri di qua e di là e vi si saliva per otto gradini.
\par 35 E mi menò alla porta settentrionale; la misurò, e aveva le stesse dimensioni;
\par 36 così delle sue camere, de' suoi pilastri e de' suoi archi; e c'erano delle finestre tutt'all'intorno, e aveva cinquanta cubiti di lunghezza e venticinque cubiti di larghezza.
\par 37 I pilastri della porta erano dal lato del cortile esterno, c'erano delle palme sui suoi pilastri di qua e di là, e vi si saliva per otto gradini.
\par 38 E c'era una camera con l'ingresso vicino ai pilastri delle porte; quivi si lavavano gli olocausti.
\par 39 E nel vestibolo delle porte c'erano due tavole di qua e due tavole di là per scannarvi su gli olocausti, i sacrifizi per il peccato e per la colpa.
\par 40 E a uno de' lati esterni, a settentrione di chi saliva all'ingresso della porta, c'erano due tavole; e dall'altro lato, verso il vestibolo della porta, c'erano due tavole.
\par 41 Così c'erano quattro tavole di qua e quattro tavole di là, ai lati della porta: in tutto otto tavole, per scannarvi su i sacrifizi.
\par 42 C'erano ancora, per gli olocausti, quattro tavole di pietra tagliata, lunghe un cubito e mezzo, larghe un cubito e mezzo e alte un cubito, per porvi su gli strumenti coi quali si scannavano gli olocausti e gli altri sacrifizi.
\par 43 E degli uncini d'un palmo eran fissati nella casa tutt'all'intorno; e sulle tavole doveva esser messa la carne delle offerte.
\par 44 E fuori dalla porta interna c'erano due camere, nel cortile interno: una era allato alla porta settentrionale, e guardava a mezzogiorno; l'altra era allato alla porta meridionale, e guardava a settentrione.
\par 45 Ed egli mi disse: 'Questa camera che guarda verso mezzogiorno è per i sacerdoti che sono incaricati del servizio della casa;
\par 46 e la camera che guarda verso settentrione è per i sacerdoti incaricati del servizio dell'altare; i figliuoli di Tsadok son quelli che, tra i figliuoli di Levi, s'accostano all'Eterno per fare il suo servizio'.
\par 47 Ed egli misurò il cortile, che era quadrato, e aveva cento cubiti di lunghezza, e cento cubiti di larghezza; e l'altare stava davanti alla casa.
\par 48 Poi mi menò nel vestibolo della casa, e misurò i pilastri del vestibolo: cinque cubiti di qua e cinque di là; la larghezza della porta era di tre cubiti di qua e di tre di là.
\par 49 La lunghezza del vestibolo era di venti cubiti; e la larghezza, di undici cubiti; vi si saliva per de' gradini; e presso ai pilastri c'erano delle colonne, una di qua e una di là.

\chapter{41}

\par 1 Poi mi condusse nel tempio, e misurò i pilastri: sei cubiti di larghezza da un lato e sei cubiti di larghezza dall'altro, larghezza della tenda.
\par 2 La larghezza dell'ingresso era di dieci cubiti; le pareti laterali dell'ingresso avevano cinque cubiti da un lato e cinque cubiti dall'altro. Egli misurò la lunghezza del tempio: quaranta cubiti, e venti cubiti di larghezza.
\par 3 Poi entrò dentro, e misurò i pilastri dell'ingresso: due cubiti; e l'ingresso: sei cubiti; e la larghezza dell'ingresso: sette cubiti.
\par 4 E misurò una lunghezza di venti cubiti e una larghezza di venti cubiti in fondo al tempio; e mi disse: 'Questo è il luogo santissimo'.
\par 5 Poi misurò il muro della casa: sei cubiti; e la larghezza delle camere laterali tutt'attorno alla casa: quattro cubiti.
\par 6 Le camere laterali erano una accanto all'altra, in numero di trenta, e c'eran tre piani; stavano in un muro, costruito per queste camere tutt'attorno alla casa, perché fossero appoggiate senz'appoggiarsi al muro della casa.
\par 7 E le camere occupavano maggiore spazio man mano che si saliva di piano in piano, poiché la casa aveva una scala circolare a ogni piano tutt'attorno alla casa; perciò questa parte della casa si allargava a ogni piano, e si saliva dal piano inferiore al superiore per quello di mezzo.
\par 8 E io vidi pure che la casa tutta intorno stava sopra un piano elevato; così le camere laterali avevano un fondamento: una buona canna, e sei cubiti fino all'angolo.
\par 9 La larghezza del muro esterno delle camere laterali era di cinque cubiti;
\par 10 e lo spazio libero intorno alle camere laterali della casa e fino alle stanze attorno alla casa aveva una larghezza di venti cubiti tutt'attorno.
\par 11 Le porte delle camere laterali davano sullo spazio libero: una porta a settentrione, una porta a mezzogiorno; e la larghezza dello spazio libero era di cinque cubiti tutt'all'intorno.
\par 12 L'edifizio ch'era davanti allo spazio vuoto dal lato d'occidente aveva settanta cubiti di larghezza, il muro dell'edifizio aveva cinque cubiti di spessore tutt'attorno, e una lunghezza di novanta cubiti.
\par 13 Poi misurò la casa, che aveva cento cubiti di lunghezza. Lo spazio vuoto, l'edifizio e i suoi muri avevano una lunghezza di cento cubiti.
\par 14 La larghezza della facciata della casa e dello spazio vuoto dal lato d'oriente era di cento cubiti.
\par 15 Egli misurò la lunghezza dell'edifizio davanti allo spazio vuoto, sul di dietro, e le sue gallerie da ogni lato: cento cubiti. L'interno del tempio, i vestiboli che davano sul cortile,
\par 16 gli stipiti, le finestre a grata, le gallerie tutt'attorno ai tre piani erano ricoperti, all'altezza degli stipiti, di legno tutt'attorno. Dall'impiantito fino alle finestre (le finestre erano sbarrate),
\par 17 fino al disopra della porta, l'interno della casa, l'esterno, e tutte le pareti tutt'attorno, all'interno e all'esterno, tutto era fatto secondo precise misure.
\par 18 E v'erano degli ornamenti di cherubini e di palme, una palma fra cherubino e cherubino,
\par 19 e ogni cherubino aveva due facce: una faccia d'uomo, vòlta verso la palma da un lato, e una faccia di leone vòlta verso l'altra palma, dall'altro lato. E ve n'era per tutta la casa, tutt'attorno.
\par 20 Dall'impiantito fino al disopra della porta c'erano dei cherubini e delle palme; così pure sul muro del tempio.
\par 21 Gli stipiti del tempio erano quadrati, e la facciata del santuario aveva lo stesso aspetto.
\par 22 L'altare era di legno, alto tre cubiti, lungo due cubiti; aveva degli angoli; e le sue pareti, per tutta la lunghezza, erano di legno. L'uomo mi disse: 'Questa è la tavola che sta davanti all'Eterno'.
\par 23 Il tempio e il santuario avevano due porte;
\par 24 e ogni porta aveva due battenti; due battenti che si piegavano in due pezzi: due pezzi per ogni battente.
\par 25 E su d'esse, sulle porte del tempio, erano scolpiti dei cherubini e delle palme, come quelli sulle pareti. E sulla facciata del vestibolo, all'esterno, c'era una tettoia di legno.
\par 26 E c'erano delle finestre a grata e delle palme, da ogni lato, alle pareti laterali del vestibolo, alle camere laterali della casa e alle tettoie.

\chapter{42}

\par 1 Poi egli mi menò fuori verso il cortile esterno dal lato di settentrione, e mi condusse nelle camere che si trovavano davanti allo spazio vuoto, e di fronte all'edifizio verso settentrione.
\par 2 Sulla facciata, dov'era la porta settentrionale, la lunghezza era di cento cubiti, e la larghezza era di cinquanta cubiti:
\par 3 dirimpetto ai venti cubiti del cortile interno, e dirimpetto al lastrico del cortile esterno, dove si trovavano tre gallerie a tre piani.
\par 4 Davanti alle camere c'era un corridoio largo dieci cubiti; e per andare nell'interno c'era un passaggio d'un cubito; e le loro porte guardavano a settentrione.
\par 5 Le camere superiori erano più strette di quelle inferiori e di quelle del centro dell'edifizio, perché le loro gallerie toglievano dello spazio.
\par 6 Poiché esse erano a tre piani, e non avevano colonne come le colonne dei cortili; perciò a partire dal suolo, le camere superiori erano più strette di quelle in basso, e di quelle del centro.
\par 7 Il muro esterno, parallelo alle camere dal lato del cortile esterno, difaccia alle camere, aveva cinquanta cubiti di lunghezza;
\par 8 poiché la lunghezza delle camere, dal lato del cortile esterno, era di cinquanta cubiti, mentre dal lato della facciata del tempio era di cento cubiti.
\par 9 In basso a queste camere c'era un ingresso dal lato d'oriente per chi v'entrava dal cortile esterno.
\par 10 Nella larghezza del muro del cortile, in direzione d'oriente, difaccia allo spazio vuoto e difaccia all'edifizio, c'erano delle camere;
\par 11 e, davanti a queste, c'era un corridoio come quello delle camere di settentrione; la loro lunghezza e la loro larghezza erano come la lunghezza e la larghezza di quelle, e così tutte le loro uscite, le loro disposizioni e le loro porte.
\par 12 Così erano anche le porte delle camere di mezzogiorno; c'era parimente una porta in capo al corridoio: al corridoio che si trovava proprio davanti al muro, dal lato d'oriente di chi v'entrava.
\par 13 Ed egli mi disse: 'Le camere di settentrione e le camere di mezzogiorno che stanno difaccia allo spazio vuoto, sono le camere sante, dove i sacerdoti che s'accostano all'Eterno mangeranno le cose santissime; quivi deporranno le cose santissime, le oblazioni e le vittime per il peccato e per la colpa; poiché quel luogo è santo.
\par 14 Quando i sacerdoti saranno entrati, non usciranno dal luogo santo per andare nel cortile esterno, senz'aver prima deposti quivi i paramenti coi quali fanno il servizio, perché questi paramenti sono santi; indosseranno altre vesti, poi potranno accostarsi alla parte che è riservata al popolo'.
\par 15 Quando ebbe finito di misurare così l'interno della casa, egli mi menò fuori per la porta ch'era al lato d'oriente e misurò il recinto tutt'attorno.
\par 16 Misurò il lato orientale con la canna da misurare: cinquecento cubiti della canna da misurare, tutto attorno.
\par 17 Misurò il lato settentrionale: cinquecento cubiti della canna da misurare, tutt'attorno.
\par 18 Misurò il lato meridionale con la canna da misurare: cinquecento cubiti.
\par 19 Si volse al lato occidentale, e misurò: cinquecento cubiti della canna da misurare.
\par 20 Misurò dai quattro lati il muro che formava il recinto: tutt'attorno la lunghezza era di cinquecento, e la larghezza di cinquecento; il muro faceva la separazione fra il sacro e il profano.

\chapter{43}

\par 1 Poi mi condusse alla porta, alla porta che guardava a oriente.
\par 2 Ed ecco, la gloria dell'Iddio d'Israele veniva dal lato d'oriente. La sua voce era come il rumore di grandi acque, e la terra risplendeva della sua gloria.
\par 3 La visione ch'io n'ebbi era simile a quella ch'io ebbi quando venni per distruggere la città; e queste visioni erano simili a quella che avevo avuta presso il fiume Kebar; e io caddi sulla mia faccia.
\par 4 E la gloria dell'Eterno entrò nella casa per la via della porta che guardava a oriente.
\par 5 Lo spirito mi levò in alto, e mi menò nel cortile interno; ed ecco, la gloria dell'Eterno riempiva la casa.
\par 6 Ed io udii qualcuno che mi parlava dalla casa, e un uomo era in piedi presso di me.
\par 7 Egli mi disse: 'Figliuol d'uomo, questo è il luogo del mio trono, e il luogo dove poserò la pianta dei miei piedi; io vi abiterò in perpetuo in mezzo ai figliuoli d'Israele; e la casa d'Israele e i suoi re non contamineranno più il mio santo nome con le loro prostituzioni e con le carogne dei loro re sui loro alti luoghi,
\par 8 come facevano quando mettevano la loro soglia presso la mia soglia, i loro stipiti presso i miei stipiti, talché non c'era che una parete fra me e loro. Essi contaminavano così il mio santo nome con le abominazioni che commettevano; ond'io li consumai, nella mia ira.
\par 9 Ora allontaneranno da me le loro prostituzioni e le carogne dei loro re, e io abiterò in mezzo a loro in perpetuo.
\par 10 E tu, figliuol d'uomo, mostra questa casa alla casa d'Israele, e si vergognino delle loro iniquità.
\par 11 Ne misurino il piano, e se si vergognano di tutto quello che hanno fatto, fa' loro conoscere la forma di questa casa, la sua disposizione, le sue uscite e i suoi ingressi, tutti i suoi disegni e tutti i suoi regolamenti, tutti i suoi riti e tutte le sue leggi; mettili per iscritto sotto ai loro occhi affinché osservino tutti i suoi riti e tutti i suoi regolamenti, e li mettano in pratica.
\par 12 Tal è la legge della casa. Sulla sommità del monte, tutto lo spazio che deve occupare tutt'attorno sarà santissimo. Ecco, tal è la legge della casa.
\par 13 E queste sono le misure dell'altare, in cubiti, de' quali ogni cubito è un cubito e un palmo. La base ha un cubito d'altezza e un cubito di larghezza; l'orlo che termina tutto il suo contorno, una spanna di larghezza; tale, il sostegno dell'altare.
\par 14 Dalla base, sul suolo, fino al gradino inferiore, due cubiti, e un cubito di larghezza; dal piccolo gradino fino al gran gradino, quattro cubiti, e un cubito di larghezza.
\par 15 La parte superiore dell'altare ha quattro cubiti d'altezza: e dal fornello dell'altare s'elevano quattro corni;
\par 16 il fornello dell'altare ha dodici cubiti di lunghezza e dodici cubiti di larghezza, e forma un quadrato perfetto.
\par 17 Il gradino ha dai quattro lati quattordici cubiti di lunghezza e quattordici cubiti di larghezza; e l'orlo che termina il suo contorno ha un mezzo cubito; la base ha tutt'attorno un cubito, e i suoi scalini son vòlti verso oriente'.
\par 18 Ed egli mi disse: 'Figliuol d'uomo, così parla il Signore, l'Eterno: Ecco i regolamenti dell'altare per il giorno che sarà costruito per offrirvi su l'olocausto e per farvi l'aspersione del sangue.
\par 19 Ai sacerdoti levitici che sono della stirpe di Tsadok, i quali s'accostano a me per servirmi, dice il Signore, l'Eterno, darai un giovenco per un sacrifizio per il peccato.
\par 20 E prenderai del suo sangue, e ne metterai sopra i quattro corni dell'altare e ai quattro angoli dei gradini e sull'orlo tutt'attorno, e purificherai così l'altare e farai l'espiazione per esso.
\par 21 E prenderai il giovenco del sacrifizio per il peccato, e lo si brucerà in un luogo designato della casa, fuori del santuario.
\par 22 E il secondo giorno offrirai come sacrifizio per il peccato un capro senza difetto, e con esso si purificherà l'altare come lo si è purificato col giovenco.
\par 23 Quando avrai finito di fare quella purificazione, offrirai un giovenco senza difetto, e un capro del gregge, senza difetto.
\par 24 Li presenterai davanti all'Eterno; e i sacerdoti vi getteranno su del sale, e li offriranno in olocausto all'Eterno.
\par 25 Per sette giorni offrirai ogni giorno un capro, come sacrifizio per il peccato; e s'offrirà pure un giovenco e un montone del gregge, senza difetto.
\par 26 Per sette giorni si farà l'espiazione per l'altare, lo si purificherà, e lo si consacrerà.
\par 27 E quando que' giorni saranno compiuti, l'ottavo giorno e in sèguito, i sacerdoti offriranno sull'altare i vostri olocausti e i vostri sacrifizi d'azione di grazie; e io vi gradirò, dice il Signore, l'Eterno'.

\chapter{44}

\par 1 Poi egli mi ricondusse verso la porta esterna del santuario, che guarda a oriente. Essa era chiusa.
\par 2 E l'Eterno mi disse: 'Questa porta sarà chiusa; essa non s'aprirà, e nessuno entrerà per essa, poiché per essa è entrato l'Eterno, l'Iddio d'Israele; perciò rimarrà chiusa.
\par 3 Quanto al principe, siccome è principe, egli potrà sedervi per mangiare il pane davanti all'Eterno; egli entrerà per la via del vestibolo della porta, e uscirà per la medesima via'.
\par 4 Poi mi menò davanti alla casa per la via della porta settentrionale. Io guardai, ed ecco, la gloria dell'Eterno riempiva la casa dell'Eterno; e io caddi sulla mia faccia.
\par 5 E l'Eterno mi disse: 'Figliuol d'uomo, sta' bene attento, apri gli occhi per guardare e gli orecchi per udire tutto quello che ti dirò circa tutti i regolamenti della casa dell'Eterno e tutte le sue leggi; e considera attentamente l'ingresso della casa, e tutti gli egressi del santuario.
\par 6 E di' a questi ribelli, alla casa d'Israele: Così parla il Signore, l'Eterno: O casa d'Israele, bastano tutte le vostre abominazioni!
\par 7 Avete fatto entrare degli stranieri, incirconcisi di cuore e incirconcisi di carne, perché stessero nel mio santuario a profanare la mia casa, quando offrivate il mio pane, il grasso e il sangue, violando così il mio patto con tutte le vostre abominazioni.
\par 8 Voi non avete serbato l'incarico che avevate delle mie cose sante; ma ne avete fatti custodi quegli stranieri, nel mio santuario, a vostro pro.
\par 9 Così parla il Signore, l'Eterno: Nessun straniero incirconciso di cuore, e incirconciso di carne, entrerà nel mio santuario: nessuno degli stranieri che saranno in mezzo ai figliuoli d'Israele.
\par 10 Inoltre, i Leviti che si sono allontanati da me quando Israele si sviava, e si sono sviati da me per seguire i loro idoli, porteranno la pena della loro iniquità;
\par 11 e saranno nel mio santuario come de' servi, con l'incarico di guardare le porte della casa; e faranno il servizio della casa; scanneranno per il popolo le vittime degli olocausti e degli altri sacrifizi, e si terranno davanti a lui per essere al tuo servizio.
\par 12 Siccome han servito il popolo davanti agl'idoli suoi e sono stati per la casa d'Israele un'occasione di caduta nell'iniquità, io alzo la mia mano contro di loro, dice il Signore, l'Eterno, giurando ch'essi porteranno la pena della loro iniquità.
\par 13 E non s'accosteranno più a me per esercitare il sacerdozio, e non s'accosteranno ad alcuna delle mie cose sante, alle cose che sono santissime; ma porteranno il loro obbrobrio, e la pena delle abominazioni che hanno commesse;
\par 14 ne farò de' guardiani della casa, incaricati di tutto il servigio d'essa e di tutto ciò che vi si deve fare.
\par 15 Ma i sacerdoti Leviti, figliuoli di Tsadok, i quali hanno serbato l'incarico che avevano del mio santuario quando i figliuoli d'Israele si sviavano da me, saranno quelli che si accosteranno a me per fare il mio servizio, e che si terranno davanti a me per offrirmi il grasso e il sangue, dice il Signore, l'Eterno.
\par 16 Essi entreranno nel mio santuario, essi s'accosteranno alla mia tavola per servirmi, e compiranno tutto il mio servizio.
\par 17 E quando entreranno per le porte del cortile interno, indosseranno vesti di lino; non avranno addosso lana di sorta, quando faranno il servizio alle porte del cortile interno e nella casa.
\par 18 Avranno in capo delle tiare di lino, e delle brache di lino ai fianchi; non si cingeranno con ciò che fa sudare.
\par 19 Ma quando usciranno per andare nel cortile esterno, nel cortile esterno verso il popolo, si toglieranno i paramenti coi quali avranno fatto il servizio, e li deporranno nelle camere del santuario; e indosseranno altre vesti, per non santificare il popolo con i loro paramenti.
\par 20 Non si raderanno il capo, e non si lasceranno crescere i capelli; ma porteranno i capelli corti.
\par 21 Nessun sacerdote berrà vino, quand'entrerà nel cortile interno.
\par 22 Non prenderanno per moglie né una vedova, né una donna ripudiata, ma prenderanno delle vergini della progenie della casa d'Israele; potranno però prendere delle vedove, che sian vedove di sacerdoti.
\par 23 Insegneranno al mio popolo a distinguere fra il sacro e il profano, e gli faranno conoscere la differenza tra ciò ch'è impuro e ciò ch'è puro.
\par 24 In casi di processo, spetterà a loro il giudicare; e giudicheranno secondo le mie prescrizioni, e osserveranno le mie leggi e i miei statuti in tutte le mie feste, e santificheranno i miei sabati.
\par 25 Il sacerdote non entrerà dov'è un morto, per non rendersi impuro; non si potrà rendere impuro che per un padre, per una madre, per un figliuolo, per una figliuola, per un fratello o per una sorella non maritata.
\par 26 Dopo la sua purificazione, gli conteranno sette giorni;
\par 27 e il giorno che entrerà nel santuario, nel cortile interno, per fare il servizio nel santuario, offrirà il suo sacrifizio per il peccato, dice il Signore, l'Eterno.
\par 28 E avranno una eredità: Io sarò la loro eredità; e voi non darete loro alcun possesso in Israele: Io sono il loro possesso.
\par 29 Essi si nutriranno delle oblazioni, dei sacrifizi per il peccato e dei sacrifizi per la colpa: e ogni cosa votata allo sterminio in Israele sarà loro.
\par 30 E le primizie dei primi prodotti d'ogni sorta, tutte le offerte di qualsivoglia cosa che offrirete per elevazione, saranno dei sacerdoti; darete parimente al sacerdote le primizie della vostra pasta, affinché la benedizione riposi sulla vostra casa.
\par 31 I sacerdoti non mangeranno carne di nessun uccello né d'alcun animale morto da sé o sbranato.

\chapter{45}

\par 1 Quando spartirete a sorte il paese per esser vostra eredità, preleverete come offerta all'Eterno una parte consacrata del paese, della lunghezza di venticinquemila cubiti e della larghezza di diecimila; sarà sacra in tutta la sua estensione.
\par 2 Di questa parte prenderete per il santuario un quadrato di cinquecento per cinquecento cubiti, e cinquanta cubiti per uno spazio libero, tutt'attorno.
\par 3 Su questa estensione di venticinquemila cubiti di lunghezza per diecimila di larghezza misurerai un'area per il santuario, per il luogo santissimo.
\par 4 È la parte consacrata del paese, la quale apparterrà ai sacerdoti, che fanno il servizio del santuario che s'accostano all'Eterno per servirlo; sarà un luogo per le loro case, un santuario per il santuario.
\par 5 Venticinquemila cubiti di lunghezza e diecimila di larghezza saranno per i Leviti che faranno il servizio della casa; sarà il loro possesso, con venti camere.
\par 6 Come possesso della città destinerete cinquemila cubiti di larghezza e venticinquemila di lunghezza, parallelamente alla parte sacra prelevata; esso sarà per tutta la casa d'Israele.
\par 7 Per il principe riserberete uno spazio ai due lati della parte sacra e del possesso della città, difaccia alla parte sacra offerta, e difaccia al possesso della città, dal lato d'occidente verso occidente, e dal lato d'oriente verso oriente, per una lunghezza parallela a una delle divisioni del paese, dal confine occidentale al confine orientale.
\par 8 Questo sarà territorio suo, suo possesso in Israele; e i miei principi non opprimeranno più il mio popolo, ma lasceranno il paese alla casa d'Israele secondo le sue tribù.
\par 9 Così parla il Signore, l'Eterno: Basta, o principi d'Israele! Lasciate da parte la violenza e le rapine, praticate il diritto e la giustizia, liberate il mio popolo dalle vostre estorsioni! dice il Signore, l'Eterno.
\par 10 Abbiate bilance giuste, efa giusto, bat giusto.
\par 11 L'efa e il bat avranno la stessa capacità: il bat conterrà la decima parte d'un omer e l'efa la decima parte d'un omer; la loro capacità sarà regolata dall'omer.
\par 12 Il siclo sarà di venti ghere; venti sicli, venticinque sicli, quindici sicli, formeranno la vostra mina.
\par 13 Questa è l'offerta che preleverete: la sesta parte d'un efa da un omer di frumento, e la sesta parte d'un efa da un omer d'orzo.
\par 14 Questa è la norma per l'olio: un decimo di bat d'olio per un cor, che è dieci bati, cioè un omer; poiché dieci bati fanno un omer.
\par 15 Una pecora su d'un gregge di dugento capi nei grassi pascoli d'Israele sarà offerta per le oblazioni, gli olocausti e i sacrifizi di azioni di grazie, per fare la propiziazione per essi, dice il Signore, l'Eterno.
\par 16 Tutto il popolo del paese dovrà prelevare quest'offerta per il principe d'Israele.
\par 17 E al principe toccherà di fornire gli olocausti, le oblazioni e le libazioni per le feste, per i noviluni, per i sabati, per tutte le solennità della casa d'Israele; egli provvederà i sacrifizi per il peccato, l'oblazione, l'olocausto e i sacrifizi d'azioni di grazie, per fare la propiziazione per la casa d'Israele.
\par 18 Così parla il Signore, l'Eterno: Il primo mese, il primo giorno del mese, prenderai un giovenco senza difetto, e purificherai il santuario.
\par 19 Il sacerdote prenderà del sangue della vittima per il peccato, e ne metterà sugli stipiti della porta della casa, sui quattro angoli de' gradini dell'altare, e sugli stipiti della porta del cortile interno.
\par 20 Farai lo stesso il settimo giorno del mese per chi avrà peccato per errore, e per il semplice; e così purificherete la casa.
\par 21 Il quattordicesimo giorno del primo mese avrete la Pasqua. La festa durerà sette giorni; si mangeranno pani senza lievito.
\par 22 In quel giorno, il principe offrirà per sé e per tutto il popolo del paese un giovenco, come sacrifizio per il peccato.
\par 23 Durante i sette giorni della festa, offrirà in olocausto all'Eterno, sette giovenchi e sette montoni senza difetto, ognuno de' sette giorni, e un capro per giorno come sacrifizio per il peccato.
\par 24 E v'aggiungerà l'offerta d'un efa per ogni giovenco e d'un efa per ogni montone, con un hin d'olio per efa.
\par 25 Il settimo mese, il quindicesimo giorno del mese, alla festa, egli offrirà per sette giorni gli stessi sacrifizi per il peccato, gli stessi olocausti, le stesse oblazioni e la stessa quantità d'olio.

\chapter{46}

\par 1 Così parla il Signore, l'Eterno: La porta del cortile interno, che guarda verso levante, resterà chiusa durante i sei giorni di lavoro; ma sarà aperta il giorno di sabato; sarà pure aperta il giorno del novilunio.
\par 2 Il principe entrerà per la via del vestibolo della porta esteriore, e si fermerà presso allo stipite della porta; e i sacerdoti offriranno il suo olocausto e i suoi sacrifizi di azioni di grazie. Egli si prostrerà sulla soglia della porta, poi uscirà; ma la porta non sarà chiusa fino alla sera.
\par 3 Parimente il popolo del paese si prostrerà davanti all'Eterno all'ingresso di quella porta, nei giorni di sabato e nei noviluni.
\par 4 E l'olocausto che il principe offrirà all'Eterno il giorno del sabato sarà di sei agnelli senza difetto, e d'un montone senza difetto;
\par 5 e la sua oblazione sarà d'un efa per il montone, e l'oblazione per gli agnelli sarà quello che vorrà dare, e d'un hin d'olio per efa.
\par 6 Il giorno del novilunio offrirà un giovenco senza difetto, sei agnelli e un montone, che saranno senza difetti;
\par 7 e darà come oblazione un efa per il giovenco, un efa per il montone, per gli agnelli nella misura de' suoi mezzi, e un hin d'olio per efa.
\par 8 Quando il principe entrerà, passerà per la via del vestibolo della porta, e uscirà per la stessa via.
\par 9 Ma quando il popolo del paese verrà davanti all'Eterno nelle solennità, chi sarà entrato per la via della porta settentrionale per prostrarsi, uscirà per la via della porta meridionale; e chi sarà entrato per la via della porta meridionale uscirà per la via della porta settentrionale; nessuno se ne tornerà per la via della porta per la quale sarà entrato, ma si uscirà per la porta opposta.
\par 10 E il principe, quando quelli entreranno, entrerà in mezzo a loro; e quando quelli usciranno, egli uscirà insieme ad essi.
\par 11 Nelle feste e nelle solennità, l'oblazione sarà d'un efa per giovenco, d'un efa per montone, per gli agnelli quello che vorrà dare, e un hin d'olio per efa.
\par 12 E quando il principe farà all'Eterno un'offerta volontaria, olocausto o sacrifizio di azioni di grazie, come offerta volontaria all'Eterno, gli si aprirà la porta che guarda a levante, ed egli offrirà il suo olocausto e il suo sacrifizio di azioni di grazie come fa nel giorno del sabato; poi uscirà; e, quando sarà uscito, si chiuderà la porta.
\par 13 Tu offrirai ogni giorno, come olocausto all'Eterno, un agnello d'un anno, senza difetto; l'offrirai ogni mattina.
\par 14 E v'aggiungerai ogni mattina, come oblazione, la sesta parte d'un efa e la terza parte d'un hin d'olio per intridere il fior di farina: è un'oblazione all'Eterno, da offrirsi del continuo per prescrizione perpetua.
\par 15 Si offriranno l'agnello, l'oblazione e l'olio ogni mattina, come olocausto continuo.
\par 16 Così parla il Signore, l'Eterno: Se il principe fa a qualcuno de' suoi figliuoli un dono preso dal proprio possesso, questo dono apparterrà ai suoi figliuoli; sarà loro proprietà ereditaria.
\par 17 Ma s'egli fa a uno de' suoi servi un dono preso dal proprio possesso, questo dono apparterrà al servo fino all'anno della liberazione; poi, tornerà al principe; la sua eredità apparterrà soltanto ai suoi figliuoli.
\par 18 E il principe non prenderà nulla dell'eredità del popolo, spogliandolo delle sue possessioni; quello che darà come eredità ai suoi figliuoli, lo prenderà da ciò che possiede, affinché nessuno del mio popolo sia cacciato dalla sua possessione'.
\par 19 Poi egli mi menò, per l'ingresso ch'era allato alla porta, nelle camere sante destinate ai sacerdoti, le quali guardavano a settentrione; ed ecco che là in fondo, verso occidente, c'era un luogo.
\par 20 Ed egli mi disse: 'Questo è il luogo dove i sacerdoti faranno cuocere la carne dei sacrifizi per la colpa e per il peccato, e faranno cuocere l'oblazione, per non farle portare fuori nel cortile esterno, in guisa che il popolo sia santificato'.
\par 21 Poi mi menò fuori nel cortile esterno, e mi fece passare presso i quattro angoli del cortile; ed ecco, in ciascun angolo del cortile c'era un cortile.
\par 22 Nei quattro angoli del cortile c'erano de' cortili chiusi, di quaranta cubiti di lunghezza e di trenta di larghezza; questi quattro cortili negli angoli avevano le stesse dimensioni.
\par 23 E intorno a tutti e quattro c'era un recinto, e de' fornelli per cuocere erano praticati in basso al recinto, tutt'attorno.
\par 24 Ed egli mi disse: 'Queste son le cucine dove quelli che fanno il servizio della casa faranno cuocere i sacrifizi del popolo'.

\chapter{47}

\par 1 Ed egli mi rimenò all'ingresso della casa; ed ecco delle acque uscivano di sotto la soglia della casa, dal lato d'oriente; perché la facciata della casa guardava a oriente; e le acque uscite di là scendevano dal lato meridionale della casa, a mezzogiorno dell'altare.
\par 2 Poi mi menò fuori per la via della porta settentrionale, e mi fece fare il giro, di fuori, fino alla porta esterna, che guarda a oriente; ed ecco, le acque scendevano dal lato destro.
\par 3 Quando l'uomo fu uscito verso oriente, aveva in mano una cordicella, e misurò mille cubiti; mi fece attraversare le acque, ed esse m'arrivavano alle calcagna.
\par 4 Misurò altri mille cubiti, e mi fece attraversare le acque, ed esse m'arrivavano alle ginocchia. Misurò altri mille cubiti, e mi fece attraversare le acque, ed esse m'arrivavano sino ai fianchi.
\par 5 E ne misurò altri mille: era un torrente che io non potevo attraversare, perché le acque erano ingrossate; erano acque che bisognava attraversare a nuoto: un torrente, che non si poteva guadare.
\par 6 Ed egli mi disse: 'Hai visto, figliuol d'uomo?' E mi ricondusse sulla riva del torrente.
\par 7 Tornato che vi fu, ecco che sulla riva del torrente c'erano moltissimi alberi, da un lato e dall'altro.
\par 8 Ed egli mi disse: 'Queste acque si dirigono verso la regione orientale, scenderanno nella pianura ed entreranno nel mare; e quando saranno entrate nel mare, le acque del mare saran rese sane.
\par 9 E avverrà che ogni essere vivente che si muove, dovunque giungerà il torrente ingrossato, vivrà, e ci sarà grande abbondanza di pesce; poiché queste acque entreranno là, quelle del mare saranno risanate, e tutto vivrà dovunque arriverà il torrente.
\par 10 E dei pescatori staranno sulle rive del mare; da En-ghedi fino ad En-eglaim si stenderanno le reti; vi sarà del pesce di diverse specie come il pesce del mar Grande, e in grande abbondanza.
\par 11 Ma le sue paludi e le sue lagune non saranno rese sane; saranno abbandonate al sale.
\par 12 E presso il torrente, sulle sue rive, da un lato e dall'altro, crescerà ogni specie d'alberi fruttiferi, le cui foglie non appassiranno e il cui frutto non verrà mai meno; ogni mese faranno dei frutti nuovi, perché quelle acque escono dal santuario; e quel loro frutto servirà di cibo, e quelle loro foglie, di medicamento'.
\par 13 Così parla il Signore, l'Eterno: 'Questa è la frontiera del paese che voi spartirete come eredità fra le dodici tribù d'Israele. Giuseppe ne avrà due parti.
\par 14 Voi avrete ciascuno, tanto l'uno quanto l'altro, una parte di questo paese, che io giurai di dare ai vostri padri. Questo paese vi toccherà quindi in eredità.
\par 15 E queste saranno le frontiere del paese. Dalla parte di settentrione: partendo dal mar Grande, in direzione di Hethlon, venendo verso Tsedad;
\par 16 Hamath, Berotha, Sibraim, che è tra la frontiera di Damasco, e la frontiera di Hamath; Hatser-hattikon, che è sulla frontiera dell'Hauran.
\par 17 Così la frontiera sarà dal mare fino a Hatsar-Enon, frontiera di Damasco, avendo a settentrione il paese settentrionale e la frontiera di Hamath. Tale, la parte di settentrione.
\par 18 Dalla parte d'oriente: partendo di fra l'Hauran e Damasco, poi di fra Galaad e il paese d'Israele, verso il Giordano, misurerete dalla frontiera di settentrione, fino al mare orientale. Tale, la parte d'oriente.
\par 19 La parte meridionale si dirigerà verso mezzogiorno, da Tamar fino alle acque di Meriboth di Kades, fino al torrente che va nel mar Grande. Tale, la parte meridionale, verso mezzogiorno.
\par 20 La parte occidentale sarà il mar Grande, da quest'ultima frontiera, fino difaccia all'entrata di Hamath. Tale, la parte occidentale.
\par 21 Dividerete così questo paese fra voi, secondo le tribù d'Israele.
\par 22 Ne spartirete a sorte de' lotti d'eredità fra voi e gli stranieri che soggiorneranno fra voi, i quali avranno generato de' figliuoli fra voi. Questi saranno per voi come dei nativi di tra i figliuoli d'Israele; trarranno a sorte con voi la loro parte d'eredità in mezzo alle tribù d'Israele.
\par 23 E nella tribù nella quale lo straniero soggiorna, quivi gli darete la sua parte, dice il Signore, l'Eterno.

\chapter{48}

\par 1 E questi sono i nomi delle tribù. Partendo dall'estremità settentrionale, lungo la via di Hethlon per andare ad Hamath, fino ad Hatsar-Enon, frontiera di Damasco a settentrione verso Hamath, avranno questo: dal confine orientale al confine occidentale, Dan, una parte.
\par 2 Sulla frontiera di Dan, dal confine orientale al confine occidentale: Ascer, una parte.
\par 3 Sulla frontiera di Ascer, dal confine orientale al confine occidentale: Neftali, una parte.
\par 4 Sulla frontiera di Neftali, dal confine orientale al confine occidentale: Manasse, una parte.
\par 5 Sulla frontiera di Manasse, dal confine orientale al confine occidentale: Efraim, una parte.
\par 6 Sulla frontiera di Efraim, dal confine orientale al confine occidentale: Ruben, una parte.
\par 7 Sulla frontiera di Ruben, dal confine orientale al confine occidentale: Giuda, una parte.
\par 8 Sulla frontiera di Giuda, dal confine orientale al confine occidentale, sarà la parte che preleverete di venticinquemila cubiti di larghezza e lunga come una delle altre parti dal confine orientale al confine occidentale; e quivi in mezzo sarà il santuario.
\par 9 La parte che preleverete per l'Eterno avrà venticinquemila cubiti di lunghezza e diecimila di larghezza.
\par 10 E questa parte santa prelevata apparterrà ai sacerdoti: venticinquemila cubiti di lunghezza al settentrione, diecimila di larghezza all'occidente, diecimila di larghezza all'oriente, e venticinquemila di lunghezza al mezzogiorno; e il santuario dell'Eterno sarà quivi in mezzo.
\par 11 Essa apparterrà ai sacerdoti consacrati di tra i figliuoli di Tsadok che hanno fatto il mio servizio, e non si sono sviati quando i figliuoli d'Israele si sviavano, come si sviavano i Leviti.
\par 12 Essa apparterrà loro come parte prelevata dalla parte del paese che sarà stata prelevata: una cosa santissima verso la frontiera di Leviti.
\par 13 I Leviti avranno, parallelamente alla frontiera de' sacerdoti, una lunghezza di venticinquemila cubiti e una larghezza di diecimila: tutta la lunghezza sarà di venticinquemila e la larghezza di diecimila.
\par 14 Essi non potranno venderne nulla; questa primizia del paese non potrà essere né scambiata né alienata, perché è cosa consacrata all'Eterno.
\par 15 I cinquemila cubiti che rimarranno di larghezza sui venticinquemila, formeranno un'area non consacrata destinata alla città, per le abitazioni e per il contado; la città sarà in mezzo,
\par 16 ed eccone le dimensioni: dal lato settentrionale, quattromilacinquecento cubiti; dal lato meridionale, quattromilacinquecento; dal lato orientale, quattromilacinquecento; dal lato occidentale, quattromilacinquecento.
\par 17 La città avrà un contado di duecentocinquanta cubiti a settentrione, di duecentocinquanta a mezzogiorno; di duecentocinquanta a oriente; e di duecentocinquanta a occidente.
\par 18 Il resto della lunghezza, parallelamente alla parte santa, cioè diecimila cubiti a oriente e diecimila a occidente, parallelamente alla parte santa, servirà, coi suoi prodotti, al mantenimento dei lavoratori della città.
\par 19 I lavoratori della città, di tutte le tribù d'Israele, ne lavoreranno il suolo.
\par 20 Tutta la parte prelevata sarà di venticinquemila cubiti di lunghezza per venticinquemila di larghezza; ne preleverete così una parte uguale al quarto della parte santa, come possesso della città.
\par 21 Il rimanente sarà del principe; da un lato e dall'altro della parte santa prelevata e del possesso della città, difaccia ai venticinquemila cubiti della parte santa sino alla frontiera d'oriente e a occidente difaccia ai venticinquemila cubiti verso la frontiera d'occidente, parallelamente alle parti; questo sarà del principe; e la parte santa e il santuario della casa saranno in mezzo.
\par 22 Così, toltone il possesso dei Leviti e il possesso della città situati in mezzo a quello del principe, ciò che si troverà tra la frontiera di Giuda e la frontiera di Beniamino, apparterrà al principe.
\par 23 Poi verrà il resto delle tribù. Dal confine orientale al confine occidentale: Beniamino, una parte.
\par 24 Sulla frontiera di Beniamino, dal confine orientale al confine occidentale: Simeone, una parte.
\par 25 Sulla frontiera di Simeone, dal confine orientale al confine occidentale: Issacar, una parte.
\par 26 Sulla frontiera d'Issacar, dal confine orientale al confine occidentale: Zabulon, una parte.
\par 27 Sulla frontiera di Zabulon, dal confine orientale al confine occidentale: Gad, una parte.
\par 28 Sulla frontiera di Gad, dal lato meridionale, verso mezzogiorno, la frontiera sarà da Tamar fino alle acque di Meriba di Kades, fino al torrente che va nel mar Grande.
\par 29 Tale è il paese che vi spartirete a sorte, come eredità delle tribù d'Israele, e tali ne sono le parti, dice il Signore, l'Eterno.
\par 30 E queste sono le uscite della città. Dal lato settentrionale, quattromilacinquecento cubiti misurati;
\par 31 le porte della città porteranno i nomi delle tribù d'Israele, e ci saranno tre porte a settentrione: la Porta di Ruben, l'una; la Porta di Giuda, l'altra; la Porta di Levi, l'altra.
\par 32 Dal lato orientale, quattromilacinquecento cubiti, e tre porte: la Porta di Giuseppe, l'una; la Porta di Beniamino, l'altra; la Porta di Dan, l'altra.
\par 33 Dal lato meridionale, quattromilacinquecento cubiti, e tre porte: la Porta di Simeone, l'una; la Porta d'Issacar, l'altra; la Porta di Zabulon, l'altra.
\par 34 Dal lato occidentale, quattromilacinquecento cubiti, e tre porte: la Porta di Gad, l'una; la Porta d'Ascer, l'altra; la Porta di Neftali, l'altra.
\par 35 La circonferenza sarà di diciottomila cubiti. E, da quel giorno, il nome della città sarà: L'Eterno è quivi'.


\end{document}