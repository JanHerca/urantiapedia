\begin{document}

\title{Song of Solomon}


\chapter{1}

\par 1 Il Cantico de' cantici di Salomone.
\par 2 Mi baci egli de' baci della sua bocca!... poiché le tue carezze son migliori del vino.
\par 3 I tuoi profumi hanno un odore soave; il tuo nome è un profumo che si spande; perciò t'aman le fanciulle!
\par 4 Attirami a te! Noi ti correremo dietro! Il re m'ha condotta ne' suoi appartamenti; noi gioiremo, ci rallegreremo a motivo di te; noi celebreremo le tue carezze più del vino! A ragione sei amato!
\par 5 Io son nera ma son bella, o figliuole di Gerusalemme, come le tende di Chedar, come i padiglioni di Salomone.
\par 6 Non guardate se son nera; è il sole che m'ha bruciata; i figliuoli di mia madre si sono adirati contro di me; m'hanno fatta guardiana delle vigne, ma io, la mia vigna, non l'ho guardata.
\par 7 O tu che il mio cuore ama, dimmi dove meni a pascere il tuo gregge, e dove lo fai riposare sul mezzogiorno. Poiché, perché sarei io come una donna sperduta, presso i greggi de' tuoi compagni?
\par 8 Se non lo sai, o la più bella delle donne, esci e segui le tracce delle pecore, e fa' pascere i tuoi capretti presso alle tende de' pastori.
\par 9 Amica mia io t'assomiglio alla mia cavalla che s'attacca ai carri di Faraone.
\par 10 Le tue guance son belle in mezzo alle collane, e il tuo collo è bello tra i filari di perle.
\par 11 Noi ti faremo delle collane d'oro con de' punti d'argento.
\par 12 Mentre il re è nel suo convito, il mio nardo esala il suo profumo.
\par 13 Il mio amico m'è un sacchetto di mirra, che passa la notte sul mio seno.
\par 14 Il mio amico m'è un grappolo di cipro delle vigne d'Enghedi.
\par 15 Come sei bella, amica mia, come sei bella! I tuoi occhi son come quelli dei colombi.
\par 16 Come sei bello, amico mio, come sei amabile! Anche il nostro letto è verdeggiante.
\par 17 Le travi delle nostre case sono cedri, i nostri soffitti sono di cipresso.

\chapter{2}

\par 1 Io sono la rosa di Saron, il giglio delle valli.
\par 2 Quale un giglio tra le spine, tale è l'amica mia tra le fanciulle.
\par 3 Qual è un melo fra gli alberi del bosco, tal è l'amico mio fra i giovani. Io desidero sedermi alla sua ombra, e il suo frutto è dolce al mio palato.
\par 4 Egli m'ha condotta nella casa del convito, e l'insegna che spiega su di me è Amore.
\par 5 Fortificatemi con delle schiacciate d'uva, sostentatemi con de' pomi, perch'io son malata d'amore.
\par 6 La sua sinistra sia sotto al mio capo, e la sua destra m'abbracci!
\par 7 O figliuole di Gerusalemme, io vi scongiuro per le gazzelle, per le cerve dei campi, non svegliate, non svegliate l'amor mio, finch'essa non lo desideri!
\par 8 Ecco la voce del mio amico! Eccolo che viene, saltando per i monti, balzando per i colli.
\par 9 L'amico mio è simile a una gazzella o ad un cerbiatto. Eccolo, egli sta dietro al nostro muro, e guarda per la finestra, lancia occhiate attraverso alle persiane.
\par 10 Il mio amico parla e mi dice: 'Lèvati, amica mia, mia bella, e vientene,
\par 11 poiché, ecco, l'inverno è passato, il tempo delle piogge è finito, se n'è andato;
\par 12 i fiori appaion sulla terra, il tempo del cantare è giunto, e la voce della tortora si fa udire nelle nostre contrade.
\par 13 Il fico ha messo i suoi ficucci, e le viti fiorite esalano il loro profumo. Lèvati, amica mia, mia bella, e vientene'.
\par 14 O mia colomba, che stai nelle fessure delle rocce, nel nascondiglio delle balze, mostrami il tuo viso, fammi udire la tua voce; poiché la tua voce è soave, e il tuo viso è bello.
\par 15 Pigliateci le volpi, le volpicine che guastano le vigne, poiché le nostre vigne sono in fiore!
\par 16 Il mio amico è mio, ed io son sua: di lui, che pastura il gregge fra i gigli.
\par 17 Prima che spiri l'aura del giorno e che le ombre fuggano, torna, amico mio, come la gazzella od il cerbiatto sui monti che si separano!

\chapter{3}

\par 1 Sul mio letto, durante la notte, ho cercato colui che l'anima mia ama; l'ho cercato, ma non l'ho trovato.
\par 2 Ora mi leverò, e andrò attorno per la città, per le strade e per le piazze; cercherò colui che l'anima mia ama; l'ho cercato, ma non l'ho trovato.
\par 3 Le guardie che vanno attorno per la città m'hanno incontrata; e ho chiesto loro: 'Avete visto colui che l'anima mia ama?'
\par 4 Di poco le avevo passate, quando trovai colui che l'anima mia ama; io l'ho preso, e non lo lascerò, finché non l'abbia menato in casa di mia madre, e nella camera di colei che m'ha concepita.
\par 5 Io vi scongiuro, o figliuole di Gerusalemme, per le gazzelle, per le cerve de' campi, non svegliate, non svegliate l'amor mio, finch'essa non lo desideri!
\par 6 Chi è colei che sale dal deserto, simile a colonne di fumo, profumata di mirra e d'incenso e d'ogni aroma de' mercanti?
\par 7 Ecco la lettiga di Salomone, intorno alla quale stanno sessanta prodi, fra i più prodi d'Israele.
\par 8 Tutti maneggiano la spada, sono esperti nelle armi; ciascuno ha la sua spada al fianco, per gli spaventi notturni.
\par 9 Il re Salomone s'è fatto una lettiga di legno del Libano.
\par 10 Ne ha fatto le colonne d'argento, la spalliera d'oro, il sedile di porpora; in mezzo è un ricamo, lavoro d'amore delle figliuole di Gerusalemme.
\par 11 Uscite, figliuole di Sion, mirate il re Salomone con la corona di cui l'ha incoronato sua madre, il giorno de' suoi sponsali, il giorno dell'allegrezza del suo cuore.

\chapter{4}

\par 1 Come sei bella, amica mia, come sei bella! I tuoi occhi, dietro al tuo velo, somiglian quelli delle colombe; i tuoi capelli son come un gregge di capre, sospese ai fianchi del monte di Galaad.
\par 2 I tuoi denti son come un branco di pecore tosate, che tornano dal lavatoio; tutte hanno de' gemelli, non ve n'è alcuna che sia sterile.
\par 3 Le tue labbra somigliano un filo di scarlatto, e la tua bocca è graziosa; le tue gote, dietro al tuo velo, son come un pezzo di melagrana.
\par 4 Il tuo collo è come la torre di Davide, edificata per essere un'armeria; mille scudi vi sono appesi, tutte le targhe dei prodi.
\par 5 Le tue due mammelle son due gemelli di gazzella, che pasturano, fra i gigli.
\par 6 Prima che spiri l'aura del giorno e che le ombre fuggano, io me ne andrò al monte della mirra e al colle dell'incenso.
\par 7 Tu sei tutta bella, amica mia, e non v'è difetto alcuno in te.
\par 8 Vieni meco dal Libano, o mia sposa, vieni meco dal Libano! Guarda dalla sommità dell'Amana, dalla sommità del Senir e dell'Hermon, dalle spelonche de' leoni, dai monti dei leopardi.
\par 9 Tu m'hai rapito il cuore, o mia sorella, o sposa mia! Tu m'hai rapito il cuore con un solo de' tuoi sguardi, con un solo de' monili del tuo collo.
\par 10 Quanto son dolci le tue carezze, o mia sorella, o sposa mia! Come le tue carezze son migliori del vino, come l'odore de' tuoi profumi è più soave di tutti gli aromi!
\par 11 O sposa mia, le tue labbra stillano miele, miele e latte son sotto la tua lingua, e l'odore delle tue vesti è come l'odore del Libano.
\par 12 O mia sorella, o sposa mia, tu sei un giardino serrato, una sorgente chiusa, una fonte sigillata.
\par 13 I tuoi germogli sono un giardino di melagrani e d'alberi di frutti deliziosi, di piante di cipro e di nardo;
\par 14 di nardo e di croco, di canna odorosa e di cinnamomo, e d'ogni albero da incenso; di mirra e d'aloe, e d'ogni più squisito aroma.
\par 15 Tu sei una fontana di giardino, una sorgente d'acqua viva, un ruscello che scende giù dal Libano.
\par 16 Lèvati, Aquilone, e vieni, o Austro! Soffiate sul mio giardino, sì che se ne spandano gli aromi! Venga l'amico mio nel suo giardino, e ne mangi i frutti deliziosi!

\chapter{5}

\par 1 Son venuto nel mio giardino, o mia sorella, o sposa mia; ho còlto la mia mirra e i miei aromi; ho mangiato il mio favo di miele; ho bevuto il mio vino ed il mio latte. Amici, mangiate, bevete, inebriatevi d'amore!
\par 2 Io dormivo, ma il mio cuore vegliava. Sento la voce del mio amico, che picchia e dice: 'Aprimi, sorella mia, amica mia, colomba mia, o mia perfetta! Poiché il mio capo è coperto di rugiada e le mie chiome son piene di gocce della notte'.
\par 3 Io mi son tolta la gonna; come me la rimetterei? Mi son lavata i piedi; come l'insudicerei?
\par 4 L'amico mio ha passato la mano per il buco della porta, e le mie viscere si son commosse per lui.
\par 5 Mi son levata per aprire al mio amico, e le mie mani hanno stillato mirra, le mie dita mirra liquida, sulla maniglia della serratura.
\par 6 Ho aperto all'amico mio, ma l'amico mio s'era ritirato, era partito. Ero fuori di me mentr'egli parlava; l'ho cercato, ma non l'ho trovato; l'ho chiamato, ma non m'ha risposto.
\par 7 Le guardie che vanno attorno per la città m'hanno incontrata, m'hanno battuta, m'hanno ferita; le guardie delle mura m'hanno strappato il velo.
\par 8 Io vi scongiuro, o figliuole di Gerusalemme, se trovate il mio amico, che gli direte?... Che son malata d'amore.
\par 9 Che è dunque, l'amico tuo, più d'un altro amico, o la più bella fra le donne? Che è dunque, l'amico tuo, più d'un altro amico, che così ci scongiuri?
\par 10 L'amico mio è bianco e vermiglio, e si distingue fra diecimila.
\par 11 Il suo capo è oro finissimo, le sue chiome sono crespe, nere come il corvo.
\par 12 I suoi occhi paion colombe in riva a de' ruscelli, lavati nel latte, incassati ne' castoni d'un anello.
\par 13 Le sue gote son come un'aia d'aromi, come aiuole di fiori odorosi; le sue labbra son gigli, e stillano mirra liquida.
\par 14 Le sue mani sono anelli d'oro, incastonati di berilli; il suo corpo è d'avorio terso, coperto di zaffiri.
\par 15 Le sue gambe son colonne di marmo, fondate su basi d'oro puro. Il suo aspetto è come il Libano, superbo come i cedri;
\par 16 il suo palato è tutto dolcezza, tutta la sua persona è un incanto. Tal è l'amor mio, tal è l'amico mio, o figliuole di Gerusalemme.

\chapter{6}

\par 1 Dov'è andato il tuo amico, o la più bella fra le donne? Da che parte s'è vòlto l'amico tuo? Noi lo cercheremo teco.
\par 2 Il mio amico è disceso nel suo giardino, nell'aie degli aromi a pasturare i greggi ne' giardini, e coglier gigli.
\par 3 Io sono dell'amico mio; e l'amico mio, che pastura il gregge fra i gigli, è mio.
\par 4 Amica mia, tu sei bella come Tirtsa, vaga come Gerusalemme, tremenda come un esercito a bandiere spiegate.
\par 5 Storna da me gli occhi tuoi, che mi turbano. I tuoi capelli son come una mandra di capre, sospese ai fianchi di Galaad.
\par 6 I tuoi denti son come un branco di pecore, che tornano dal lavatoio; tutte hanno de' gemelli, non ve n'è alcuna che sia sterile;
\par 7 le tue gote, dietro al tuo velo, son come un pezzo di melagrana.
\par 8 Ci son sessanta regine, ottanta concubine, e fanciulle senza numero;
\par 9 ma la mia colomba, la perfetta mia, è unica; è l'unica di sua madre, la prescelta di colei che l'ha partorita. Le fanciulle la vedono, e la proclaman beata; la vedon pure le regine e le concubine, e la lodano.
\par 10 Chi è colei che appare come l'alba, bella come la luna, pura come il sole, tremenda come un esercito a bandiere spiegate?
\par 11 Io son discesa nel giardino dei noci a vedere le piante verdi della valle, a veder se le viti mettevan le loro gemme, se i melagrani erano in fiore.
\par 12 Io non so come, il mio desiderio m'ha resa simile ai carri d'Amminadab.

\chapter{7}

\par 1 Torna, torna, o Sulamita, torna, torna che ti miriamo. Perché mirate la Sulamita come una danza a due schiere?
\par 2 Come son belli i tuoi piedi ne' loro calzari, o figliuola di principe! I contorni delle tue anche son come monili, opera di mano d'artefice.
\par 3 Il tuo seno è una tazza rotonda, dove non manca mai vino profumato. Il tuo corpo è un mucchio di grano, circondato di gigli.
\par 4 Le tue due mammelle paion due gemelli di gazzella.
\par 5 Il tuo collo è come una torre d'avorio; i tuoi occhi son come le piscine d'Heshbon presso la porta di Bath-Rabbim. Il tuo naso è come la torre del Libano, che guarda verso Damasco.
\par 6 Il tuo capo s'eleva come il Carmelo, e la chioma del tuo capo sembra di porpora; un re è incatenato dalle tue trecce!
\par 7 Quanto sei bella, quanto sei piacevole, o amor mio, in mezzo alle delizie!
\par 8 La tua statura è simile alla palma, e le tue mammelle a de' grappoli d'uva.
\par 9 Io ho detto: 'Io salirò sulla palma, e m'appiglierò ai suoi rami'. Siano le tue mammelle come grappoli di vite, il profumo del tuo fiato, come quello dei pomi,
\par 10 e la tua bocca come un vino generoso, che cola dolcemente per il mio amico, e scivola fra le labbra di quelli che dormono.
\par 11 Io sono del mio amico, e verso me va il suo desiderio.
\par 12 Vieni, amico mio, usciamo ai campi, passiam la notte ne' villaggi!
\par 13 Fin dal mattino andremo nelle vigne; vedremo se la vite ha sbocciato, se il suo fiore s'apre, se i melagrani fioriscono. Quivi ti darò le mie carezze.
\par 14 Le mandragole mandano profumo, e sulle nostre porte stanno frutti deliziosi d'ogni sorta, nuovi e vecchi, che ho serbati per te, amico mio.

\chapter{8}

\par 1 Oh perché non sei tu come un mio fratello, allattato dalle mammelle di mia madre! Trovandoti fuori, ti bacerei, e nessuno mi sprezzerebbe.
\par 2 Ti condurrei, t'introdurrei in casa di mia madre, tu mi ammaestreresti, e io ti darei a bere del vino aromatico, del succo del mio melagrano.
\par 3 La sua sinistra sia sotto il mio capo, e la sua destra m'abbracci!
\par 4 O figliuole di Gerusalemme, io vi scongiuro, non svegliate, non svegliate l'amor mio, finch'essa non lo desideri!
\par 5 Chi è colei che sale dal deserto appoggiata all'amico suo? Io t'ho svegliata sotto il melo, dove tua madre t'ha partorito, dove quella che t'ha partorito, s'è sgravata di te.
\par 6 Mettimi come un sigillo sul tuo cuore, come un sigillo sul tuo braccio; perché l'amore è forte come la morte, la gelosia è dura come il soggiorno de' morti. I suoi ardori sono ardori di fuoco, fiamma dell'Eterno.
\par 7 Le grandi acque non potrebbero spegnere l'amore, e dei fiumi non potrebbero sommergerlo. Se uno desse tutti i beni di casa sua in cambio dell'amore, sarebbe del tutto disprezzato.
\par 8 Noi abbiamo una piccola sorella, che non ha ancora mammelle; che farem noi della nostra sorella, quando si tratterà di lei?
\par 9 S'ella è un muro, costruiremo su lei una torretta d'argento; se ella è un uscio, la chiuderemo con una tavola di cedro.
\par 10 Io sono un muro, e le mie mammelle sono come torri; io sono stata ai suoi occhi come colei che ha trovato pace.
\par 11 Salomone aveva una vigna a Baal-Hamon; egli affidò la vigna a de' guardiani, ognun de' quali portava, come frutto, mille sicli d'argento.
\par 12 La mia vigna, ch'è mia, la guardo da me; tu, Salomone, tienti pure i tuoi mille sicli, e se n'abbian duecento quei che guardano il frutto della tua!
\par 13 O tu che dimori ne' giardini, dei compagni stanno intenti alla tua voce! Fammela udire!
\par 14 Fuggi, amico mio, come una gazzella od un cerbiatto, sui monti degli aromi!


\end{document}