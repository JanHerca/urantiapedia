\begin{document}

\title{1 Timothy}


\chapter{1}

\par 1 Paolo, apostolo di Cristo Gesù per comandamento di Dio nostro Salvatore e di Cristo Gesù nostra speranza,
\par 2 a Timoteo mio vero figliuolo in fede, grazia, misericordia, pace, da Dio Padre e da Cristo Gesù nostro Signore.
\par 3 Ti ripeto l'esortazione che ti feci quando andavo in Macedonia, di rimanere ad Efeso per ordinare a certuni che non insegnino dottrina diversa
\par 4 né si occupino di favole e di genealogie senza fine, le quali producono questioni, anziché promuovere la dispensazione di Dio, che è in fede.
\par 5 Ma il fine di quest'incarico è l'amore procedente da un cuor puro, da una buona coscienza e da fede non finta;
\par 6 dalle quali cose certuni avendo deviato, si sono rivolti a un vano parlare,
\par 7 volendo esser dottori della legge, quantunque non intendano quello che dicono, né quello che danno per certo.
\par 8 Or noi sappiamo che la legge è buona, se uno l'usa legittimamente,
\par 9 riconoscendo che la legge è fatta non per il giusto, ma per gl'iniqui e i ribelli, per gli empî e i peccatori, per gli scellerati e gl'irreligiosi, per i percuotitori di padre e madre,
\par 10 per gli omicidi, per i fornicatori, per i sodomiti, per i ladri d'uomini, per i bugiardi, per gli spergiuri e per ogni altra cosa contraria alla sana dottrina,
\par 11 secondo l'evangelo della gloria del beato Iddio, che m'è stato affidato.
\par 12 Io rendo grazie a colui che mi ha reso forte, a Cristo Gesù, nostro Signore, dell'avermi egli reputato degno della sua fiducia, ponendo al ministerio me,
\par 13 che prima ero un bestemmiatore, un persecutore e un oltraggiatore; ma misericordia mi è stata fatta, perché lo feci ignorantemente nella mia incredulità;
\par 14 e la grazia del Signor nostro è sovrabbondata con la fede e con l'amore che è in Cristo Gesù.
\par 15 Certa è questa parola e degna d'essere pienamente accettata: che Cristo Gesù è venuto nel mondo per salvare i peccatori, dei quali io sono il primo.
\par 16 Ma per questo mi è stata fatta misericordia, affinché Gesù Cristo dimostrasse in me per il primo tutta la sua longanimità, e io servissi d'esempio a quelli che per l'avvenire crederebbero in lui per aver la vita eterna.
\par 17 Or al re dei secoli, immortale, invisibile, solo Dio, siano onore e gloria ne' secoli de' secoli. Amen.
\par 18 Io t'affido quest'incarico, o figliuol mio Timoteo, in armonia con le profezie che sono state innanzi fatte a tuo riguardo, affinché tu guerreggi in virtù d'esse la buona guerra,
\par 19 avendo fede e buona coscienza; della quale alcuni avendo fatto getto, hanno naufragato quanto alla fede.
\par 20 Fra questi sono Imeneo ed Alessandro, i quali ho dati in man di Satana affinché imparino a non bestemmiare.

\chapter{2}

\par 1 Io esorto dunque, prima d'ogni altra cosa, che si facciano supplicazioni, preghiere, intercessioni, ringraziamenti per tutti gli uomini,
\par 2 per i re e per tutti quelli che sono in autorità, affinché possiamo menare una vita tranquilla e quieta, in ogni pietà e onestà.
\par 3 Questo è buono e accettevole nel cospetto di Dio, nostro Salvatore,
\par 4 il quale vuole che tutti gli uomini siano salvati e vengano alla conoscenza della verità.
\par 5 Poiché v'è un solo Dio ed anche un solo mediatore fra Dio e gli uomini, Cristo Gesù uomo,
\par 6 il quale diede se stesso qual prezzo di riscatto per tutti; fatto che doveva essere attestato a suo tempo,
\par 7 e per attestare il quale io fui costituito banditore ed apostolo (io dico il vero, non mentisco), dottore dei Gentili in fede e in verità.
\par 8 Io voglio dunque che gli uomini faccian orazione in ogni luogo, alzando mani pure, senz'ira e senza dispute.
\par 9 Similmente che le donne si adornino d'abito convenevole, con verecondia e modestia: non di trecce e d'oro o di perle o di vesti sontuose,
\par 10 ma d'opere buone, come s'addice a donne che fanno professione di pietà.
\par 11 La donna impari in silenzio con ogni sottomissione.
\par 12 Poiché non permetto alla donna d'insegnare, né d'usare autorità sul marito, ma stia in silenzio.
\par 13 Perché Adamo fu formato il primo, e poi Eva;
\par 14 e Adamo non fu sedotto; ma la donna, essendo stata sedotta, cadde in trasgressione
\par 15 nondimeno sarà salvata partorendo figliuoli, se persevererà nella fede, nell'amore e nella santificazione con modestia.

\chapter{3}

\par 1 Certa è questa parola: Se uno aspira all'ufficio di vescovo, desidera un'opera buona.
\par 2 Bisogna dunque che il vescovo sia irreprensibile, marito di una sola moglie, sobrio, assennato, costumato, ospitale, atto ad insegnare,
\par 3 non dedito al vino né violento, ma sia mite, non litigioso, non amante del danaro
\par 4 che governi bene la propria famiglia e tenga i figliuoli in sottomissione e in tutta riverenza
\par 5 (che se uno non sa governare la propria famiglia, come potrà aver cura della chiesa di Dio?),
\par 6 che non sia novizio, affinché, divenuto gonfio d'orgoglio, non cada nella condanna del diavolo.
\par 7 Bisogna inoltre che abbia una buona testimonianza da quelli di fuori, affinché non cada in vituperio e nel laccio del diavolo.
\par 8 Parimente i diaconi debbono esser dignitosi, non doppi in parole, non proclivi a troppo vino, non avidi di illeciti guadagni;
\par 9 uomini che ritengano il mistero della fede in pura coscienza.
\par 10 E anche questi siano prima provati; poi assumano l'ufficio di diaconi se sono irreprensibili.
\par 11 Parimente siano le donne dignitose, non maldicenti, sobrie, fedeli in ogni cosa.
\par 12 I diaconi siano mariti di una sola moglie, e governino bene i loro figliuoli e le loro famiglie.
\par 13 Perché quelli che hanno ben fatto l'ufficio di diaconi, si acquistano un buon grado e una gran franchezza nella fede che è in Cristo Gesù.
\par 14 Io ti scrivo queste cose sperando di venir tosto da te;
\par 15 e, se mai tardo, affinché tu sappia come bisogna comportarsi nella casa di Dio, che è la Chiesa dell'Iddio vivente, colonna e base della verità.
\par 16 E, senza contraddizione, grande è il mistero della pietà: Colui che è stato manifestato in carne, è stato giustificato nello spirito, è apparso agli angeli, è stato predicato fra i Gentili, è stato creduto nel mondo, è stato elevato in gloria.

\chapter{4}

\par 1 Ma lo Spirito dice espressamente che nei tempi a venire alcuni apostateranno dalla fede, dando retta a spiriti seduttori, e a dottrine di demonî
\par 2 per via della ipocrisia di uomini che proferiranno menzogna, segnati di un marchio nella loro propria coscienza;
\par 3 i quali vieteranno il matrimonio e ordineranno l'astensione da cibi che Dio ha creati affinché quelli che credono e hanno ben conosciuta la verità, ne usino con rendimento di grazie.
\par 4 Poiché tutto quel che Dio ha creato è buono; e nulla è da riprovare, se usato con rendimento di grazie;
\par 5 perché è santificato dalla parola di Dio e dalla preghiera.
\par 6 Rappresentando queste cose ai fratelli, tu sarai un buon ministro di Cristo Gesù, nutrito delle parole della fede e della buona dottrina che hai seguita da presso.
\par 7 Ma schiva le favole profane e da vecchie; esèrcitati invece alla pietà;
\par 8 perché l'esercizio corporale è utile a poca cosa, mentre la pietà è utile ad ogni cosa, avendo la promessa della vita presente e di quella a venire.
\par 9 Certa è questa parola, e degna d'esser pienamente accettata.
\par 10 Poiché per questo noi fatichiamo e lottiamo: perché abbiamo posto la nostra speranza nell'Iddio vivente, che è il Salvatore di tutti gli uomini, principalmente dei credenti.
\par 11 Ordina queste cose e insegnale. Nessuno sprezzi la tua giovinezza;
\par 12 ma sii d'esempio ai credenti, nel parlare, nella condotta, nell'amore, nella fede, nella castità.
\par 13 Attendi finché io torni, alla lettura, all'esortazione, all'insegnamento.
\par 14 Non trascurare il dono che è in te, il quale ti fu dato per profezia quando ti furono imposte le mani dal collegio degli anziani.
\par 15 Cura queste cose e datti ad esse interamente, affinché il tuo progresso sia manifesto a tutti.
\par 16 Bada a te stesso e all'insegnamento; persevera in queste cose, perché, facendo così, salverai te stesso e quelli che ti ascoltano.

\chapter{5}

\par 1 Non riprendere aspramente l'uomo anziano, ma esortalo come un padre;
\par 2 i giovani, come fratelli; le donne anziane, come madri; le giovani, come sorelle, con ogni castità.
\par 3 Onora le vedove che son veramente vedove.
\par 4 Ma se una vedova ha dei figliuoli o de' nipoti, imparino essi prima a mostrarsi pii verso la propria famiglia e a rendere il contraccambio ai loro genitori, perché questo è accettevole nel cospetto di Dio.
\par 5 Or la vedova che è veramente tale e sola al mondo, ha posto la sua speranza in Dio, e persevera in supplicazioni e preghiere notte e giorno;
\par 6 ma quella che si dà ai piaceri, benché viva, è morta.
\par 7 Anche queste cose ordina, onde siano irreprensibili.
\par 8 Che se uno non provvede ai suoi, e principalmente a quelli di casa sua, ha rinnegato la fede, ed è peggiore dell'incredulo.
\par 9 Sia la vedova iscritta nel catalogo quando non abbia meno di sessant'anni: quando sia stata moglie d'un marito solo,
\par 10 quando sia conosciuta per le sue buone opere: per avere allevato figliuoli, esercitato l'ospitalità, lavato i piedi ai santi, soccorso gli afflitti, concorso ad ogni opera buona.
\par 11 Ma rifiuta le vedove più giovani, perché, dopo aver lussureggiato contro Cristo, vogliono maritarsi,
\par 12 e sono colpevoli perché hanno rotta la prima fede;
\par 13 ed oltre a ciò imparano anche ad essere oziose, andando attorno per le case; e non soltanto ad esser oziose, ma anche cianciatrici e curiose, parlando di cose delle quali non si deve parlare.
\par 14 Io voglio dunque che le vedove giovani si maritino, abbiano figliuoli, governino la casa, non diano agli avversari alcuna occasione di maldicenza,
\par 15 poiché già alcune si sono sviate per andar dietro a Satana.
\par 16 Se qualche credente ha delle vedove, le soccorra, e la chiesa non ne sia gravata, onde possa soccorrer quelle che son veramente vedove.
\par 17 Gli anziani che tengon bene la presidenza, siano reputati degni di doppio onore, specialmente quelli che faticano nella predicazione e nell'insegnamento;
\par 18 poiché la Scrittura dice: Non metter la museruola al bue che trebbia; e l'operaio è degno della sua mercede.
\par 19 Non ricevere accusa contro un anziano, se non sulla deposizione di due o tre testimoni.
\par 20 Quelli che peccano, riprendili in presenza di tutti, onde anche gli altri abbian timore.
\par 21 Io ti scongiuro, dinanzi a Dio, dinanzi a Cristo Gesù e agli angeli eletti, che tu osservi queste cose senza prevenzione, non facendo nulla con parzialità.
\par 22 Non imporre con precipitazione le mani ad alcuno, e non partecipare ai peccati altrui; conservati puro.
\par 23 Non continuare a bere acqua soltanto, ma prendi un poco di vino a motivo del tuo stomaco e delle tue frequenti infermità.
\par 24 I peccati d'alcuni uomini sono manifesti e vanno innanzi a loro al giudizio; ad altri uomini, invece, essi tengono dietro.
\par 25 Similmente, anche le opere buone sono manifeste; e quelle che non lo sono, non possono rimanere occulte.

\chapter{6}

\par 1 Tutti coloro che sono sotto il giogo della servitù, reputino i loro padroni come degni d'ogni onore, affinché il nome di Dio e la dottrina non vengano biasimati.
\par 2 E quelli che hanno padroni credenti non li disprezzino perché son fratelli, ma tanto più li servano, perché quelli che ricevono il beneficio del loro servizio sono fedeli e diletti. Queste cose insegna e ad esse esorta.
\par 3 Se qualcuno insegna una dottrina diversa e non s'attiene alle sane parole del Signor nostro Gesù Cristo e alla dottrina che è secondo pietà,
\par 4 esso è gonfio e non sa nulla; ma langue intorno a questioni e dispute di parole, dalle quali nascono invidia, contenzione, maldicenza, cattivi sospetti,
\par 5 acerbe discussioni d'uomini corrotti di mente e privati della verità, i quali stimano la pietà esser fonte di guadagno.
\par 6 Or la pietà con animo contento del proprio stato, è un gran guadagno;
\par 7 poiché non abbiam portato nulla nel mondo, perché non ne possiamo neanche portar via nulla;
\par 8 ma avendo di che nutrirci e di che coprirci, saremo di questo contenti.
\par 9 Ma quelli che vogliono arricchire cadono in tentazione, in laccio, e in molte insensate e funeste concupiscenze, che affondano gli uomini nella distruzione e nella perdizione.
\par 10 Poiché l'amor del danaro è radice d'ogni sorta di mali; e alcuni che vi si sono dati, si sono sviati dalla fede e si son trafitti di molti dolori.
\par 11 Ma tu, o uomo di Dio, fuggi queste cose, e procaccia giustizia, pietà, fede, amore, costanza, dolcezza.
\par 12 Combatti il buon combattimento della fede, afferra la vita eterna alla quale sei stato chiamato e in vista della quale facesti quella bella confessione in presenza di molti testimoni.
\par 13 Nel cospetto di Dio che vivifica tutte le cose, e di Cristo Gesù che rese testimonianza dinanzi a Ponzio Pilato con quella bella confessione,
\par 14 io t'ingiungo d'osservare il comandamento divino da uomo immacolato, irreprensibile, fino all'apparizione del nostro Signor Gesù Cristo,
\par 15 la quale sarà a suo tempo manifestata dal beato e unico Sovrano, il Re dei re e Signor dei signori,
\par 16 il quale solo possiede l'immortalità ed abita una luce inaccessibile; il quale nessun uomo ha veduto né può vedere; al quale siano onore e potenza eterna. Amen.
\par 17 A quelli che son ricchi in questo mondo ordina che non siano d'animo altero, che non ripongano la loro speranza nell'incertezza delle ricchezze, ma in Dio, il quale ci somministra copiosamente ogni cosa perché ne godiamo;
\par 18 che facciano del bene, che siano ricchi in buone opere, pronti a dare, a far parte dei loro averi,
\par 19 in modo da farsi un tesoro ben fondato per l'avvenire, a fin di conseguire la vera vita.
\par 20 O Timoteo, custodisci il deposito, schivando le profane vacuità di parole e le opposizioni di quella che falsamente si chiama scienza,
\par 21 della quale alcuni facendo professione, si sono sviati dalla fede. La grazia sia con voi.


\end{document}