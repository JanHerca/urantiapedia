\begin{document}

\title{Giovanni}


\chapter{1}

\par 1 Nel principio era la Parola, e la Parola era con Dio, e la Parola era Dio.
\par 2 Essa era nel principio con Dio.
\par 3 Ogni cosa è stata fatta per mezzo di lei; e senza di lei neppure una delle cose fatte è stata fatta.
\par 4 In lei era la vita; e la vita era la luce degli uomini;
\par 5 e la luce splende nelle tenebre, e le tenebre non l'hanno ricevuta.
\par 6 Vi fu un uomo mandato da Dio, il cui nome era Giovanni.
\par 7 Egli venne come testimone per render testimonianza alla luce, affinché tutti credessero per mezzo di lui.
\par 8 Egli stesso non era la luce, ma venne per render testimonianza alla luce.
\par 9 La vera luce che illumina ogni uomo, era per venire nel mondo.
\par 10 Egli era nel mondo, e il mondo fu fatto per mezzo di lui, ma il mondo non l'ha conosciuto.
\par 11 È venuto in casa sua, e i suoi non l'hanno ricevuto;
\par 12 ma a tutti quelli che l'hanno ricevuto egli ha dato il diritto di diventar figliuoli di Dio; a quelli, cioè, che credono nel suo nome;
\par 13 i quali non son nati da sangue, né da volontà di carne, né da volontà d'uomo, ma son nati da Dio.
\par 14 E la Parola è stata fatta carne ed ha abitato per un tempo fra noi, piena di grazia e di verità; e noi abbiam contemplata la sua gloria, gloria come quella dell'Unigenito venuto da presso al Padre.
\par 15 Giovanni gli ha resa testimonianza ed ha esclamato, dicendo: Era di questo che io dicevo: Colui che vien dietro a me mi ha preceduto, perché era prima di me.
\par 16 Infatti, è della sua pienezza che noi tutti abbiamo ricevuto, e grazia sopra grazia.
\par 17 Poiché la legge è stata data per mezzo di Mosè; la grazia e la verità son venute per mezzo di Gesù Cristo.
\par 18 Nessuno ha mai veduto Iddio; l'unigenito Figliuolo, che è nel seno del Padre, è quel che l'ha fatto conoscere.
\par 19 E questa è la testimonianza di Giovanni, quando i Giudei mandarono da Gerusalemme de' sacerdoti e dei leviti per domandargli: Tu chi sei?
\par 20 Ed egli lo confessò e non lo negò; lo confessò dicendo: Io non sono il Cristo.
\par 21 Ed essi gli domandarono: Che dunque? Sei Elia? Ed egli rispose: Non lo sono. Sei tu il profeta? Ed egli rispose: No.
\par 22 Essi dunque gli dissero: Chi sei? affinché diamo una risposta a coloro che ci hanno mandato. Che dici tu di te stesso?
\par 23 Egli disse: Io son la voce d'uno che grida nel deserto: Addirizzate la via del Signore, come ha detto il profeta Isaia.
\par 24 Or quelli ch'erano stati mandati a lui erano de' Farisei:
\par 25 e gli domandarono: Perché dunque battezzi se tu non sei il Cristo, né Elia, né il profeta?
\par 26 Giovanni rispose loro, dicendo: Io battezzo con acqua; nel mezzo di voi è presente uno che voi non conoscete,
\par 27 colui che viene dietro a me, al quale io non son degno di sciogliere il legaccio de' calzari.
\par 28 Queste cose avvennero in Betania al di là del Giordano, dove Giovanni stava battezzando.
\par 29 Il giorno seguente, Giovanni vide Gesù che veniva a lui, e disse: Ecco l'Agnello di Dio, che toglie il peccato del mondo!
\par 30 Questi è colui del quale dicevo: Dietro a me viene un uomo che mi ha preceduto, perché egli era prima di me.
\par 31 E io non lo conoscevo; ma appunto perché egli sia manifestato ad Israele, son io venuto a battezzar con acqua.
\par 32 E Giovanni rese la sua testimonianza, dicendo: Ho veduto lo Spirito scendere dal cielo a guisa di colomba, e fermarsi su di lui.
\par 33 E io non lo conoscevo; ma Colui che mi ha mandato a battezzare con acqua, mi ha detto: Colui sul quale vedrai lo Spirito scendere e fermarsi, è quel che battezza con lo Spirito Santo.
\par 34 E io ho veduto e ho attestato che questi è il Figliuol di Dio.
\par 35 Il giorno seguente, Giovanni era di nuovo là con due de' suoi discepoli;
\par 36 e avendo fissato lo sguardo su Gesù che stava passando, disse: Ecco l'Agnello di Dio!
\par 37 E i suoi due discepoli, avendolo udito parlare, seguirono Gesù.
\par 38 E Gesù, voltatosi, e osservando che lo seguivano, domandò loro: Che cercate? Ed essi gli dissero: Rabbì (che, interpretato, vuol dire: Maestro), ove dimori?
\par 39 Egli rispose loro: Venite e vedrete. Essi dunque andarono, e videro ove dimorava, e stettero con lui quel giorno. Era circa la decima ora.
\par 40 Andrea, il fratello di Simon Pietro, era uno dei due che aveano udito Giovanni ed avean seguito Gesù.
\par 41 Egli pel primo trovò il proprio fratello Simone e gli disse: Abbiam trovato il Messia (che, interpretato, vuol dire: Cristo); e lo menò da Gesù.
\par 42 E Gesù, fissato in lui lo sguardo, disse: Tu sei Simone, il figliuol di Giovanni; tu sarai chiamato Cefa (che significa Pietro).
\par 43 Il giorno seguente, Gesù volle partire per la Galilea; trovò Filippo, e gli disse: Seguimi.
\par 44 Or Filippo era di Betsaida, della città d'Andrea e di Pietro.
\par 45 Filippo trovò Natanaele, e gli disse: Abbiam trovato colui del quale hanno scritto Mosè nella legge, ed i profeti: Gesù figliuolo di Giuseppe, da Nazaret.
\par 46 E Natanaele gli disse: Può forse venir qualcosa di buono da Nazaret? Filippo gli rispose: Vieni a vedere.
\par 47 Gesù vide Natanaele che gli veniva incontro, e disse di lui: Ecco un vero israelita in cui non c'è frode.
\par 48 Natanaele gli chiese: Da che mi conosci? Gesù gli rispose: Prima che Filippo ti chiamasse, quand'eri sotto il fico, io t'ho veduto.
\par 49 Natanaele gli rispose: Maestro, tu sei il Figliuol di Dio, tu sei il Re d'Israele.
\par 50 Gesù rispose e gli disse: Perché t'ho detto che t'avevo visto sotto il fico, tu credi? Tu vedrai cose maggiori di queste.
\par 51 Poi gli disse: In verità, in verità vi dico che vedrete il cielo aperto e gli angeli di Dio salire e scendere sopra il Figliuol dell'uomo.

\chapter{2}

\par 1 Tre giorni dopo, si fecero delle nozze in Cana di Galilea, e c'era la madre di Gesù.
\par 2 E Gesù pure fu invitato co' suoi discepoli alle nozze.
\par 3 E venuto a mancare il vino, la madre di Gesù gli disse: Non han più vino.
\par 4 E Gesù le disse: Che v'è fra me e te, o donna? L'ora mia non è ancora venuta.
\par 5 Sua madre disse ai servitori: Fate tutto quel che vi dirà.
\par 6 Or c'erano quivi sei pile di pietra, destinate alla purificazione de' Giudei, le quali contenevano ciascuna due o tre misure.
\par 7 Gesù disse loro: Empite d'acqua le pile. Ed essi le empirono fino all'orlo.
\par 8 Poi disse loro: Ora attingete, e portatene al maestro di tavola. Ed essi gliene portarono.
\par 9 E quando il maestro di tavola ebbe assaggiata l'acqua ch'era diventata vino (or egli non sapea donde venisse, ma ben lo sapeano i servitori che aveano attinto l'acqua), chiamò lo sposo e gli disse:
\par 10 Ognuno serve prima il vin buono; e quando si è bevuto largamente, il men buono; tu, invece, hai serbato il vin buono fino ad ora.
\par 11 Gesù fece questo primo de' suoi miracoli in Cana di Galilea, e manifestò la sua gloria; e i suoi discepoli credettero in lui.
\par 12 Dopo questo, scese a Capernaum, egli con sua madre, co' suoi fratelli e i suoi discepoli; e stettero quivi non molti giorni.
\par 13 Or la Pasqua de' Giudei era vicina, e Gesù salì a Gerusalemme.
\par 14 E trovò nel tempio quelli che vendevano buoi e pecore e colombi, e i cambiamonete seduti.
\par 15 E fatta una sferza di cordicelle, scacciò tutti fuori del tempio, pecore e buoi; e sparpagliò il danaro dei cambiamonete, e rovesciò le tavole;
\par 16 e a quelli che vendeano i colombi, disse: Portate via di qui queste cose; non fate della casa del Padre mio una casa di mercato.
\par 17 E i suoi discepoli si ricordarono che sta scritto: Lo zelo della tua casa mi consuma.
\par 18 I Giudei allora presero a dirgli: Qual segno ci mostri tu che fai queste cose?
\par 19 Gesù rispose loro: Disfate questo tempio, e in tre giorni lo farò risorgere.
\par 20 Allora i Giudei dissero: Quarantasei anni è durata la fabbrica di questo tempio e tu lo faresti risorgere in tre giorni?
\par 21 Ma egli parlava del tempio del suo corpo.
\par 22 Quando dunque fu risorto da' morti, i suoi discepoli si ricordarono ch'egli avea detto questo; e credettero alla Scrittura e alla parola che Gesù avea detta.
\par 23 Mentr'egli era in Gerusalemme alla festa di Pasqua, molti credettero nel suo nome, vedendo i miracoli ch'egli faceva.
\par 24 Ma Gesù non si fidava di loro, perché conosceva tutti,
\par 25 e perché non avea bisogno della testimonianza d'alcuno sull'uomo, poiché egli stesso conosceva quello che era nell'uomo.

\chapter{3}

\par 1 Or v'era tra i Farisei un uomo, chiamato Nicodemo, un de' capi de' Giudei.
\par 2 Egli venne di notte a Gesù, e gli disse: Maestro, noi sappiamo che tu sei un dottore venuto da Dio; perché nessuno può fare questi miracoli che tu fai, se Dio non è con lui.
\par 3 Gesù gli rispose dicendo: In verità, in verità io ti dico che se uno non è nato di nuovo, non può vedere il regno di Dio.
\par 4 Nicodemo gli disse: Come può un uomo nascere quand'è vecchio? Può egli entrare una seconda volta nel seno di sua madre e nascere?
\par 5 Gesù rispose: In verità, in verità io ti dico che se uno non è nato d'acqua e di Spirito, non può entrare nel regno di Dio.
\par 6 Quel che è nato dalla carne, è carne; e quel che è nato dallo Spirito, è spirito.
\par 7 Non ti maravigliare se t'ho detto: Bisogna che nasciate di nuovo.
\par 8 Il vento soffia dove vuole, e tu ne odi il rumore, ma non sai né d'onde viene né dove va; così è di chiunque è nato dallo Spirito.
\par 9 Nicodemo replicò e gli disse: Come possono avvenir queste cose?
\par 10 Gesù gli rispose: Tu se' il dottor d'Israele e non sai queste cose?
\par 11 In verità, in verità io ti dico che noi parliamo di quel che sappiamo, e testimoniamo di quel che abbiamo veduto; ma voi non ricevete la nostra testimonianza.
\par 12 Se vi ho parlato delle cose terrene e non credete, come crederete se vi parlerò delle cose celesti?
\par 13 E nessuno è salito in cielo, se non colui che è disceso dal cielo: il Figliuol dell'uomo che è nel cielo.
\par 14 E come Mosè innalzò il serpente nel deserto, così bisogna che il Figliuol dell'uomo sia innalzato,
\par 15 affinché chiunque crede in lui abbia vita eterna.
\par 16 Poiché Iddio ha tanto amato il mondo, che ha dato il suo unigenito Figliuolo, affinché chiunque crede in lui non perisca, ma abbia vita eterna.
\par 17 Infatti Iddio non ha mandato il suo Figliuolo nel mondo per giudicare il mondo, ma perché il mondo sia salvato per mezzo di lui.
\par 18 Chi crede in lui non è giudicato; chi non crede è già giudicato, perché non ha creduto nel nome dell'unigenito Figliuol di Dio.
\par 19 E il giudizio è questo: che la luce è venuta nel mondo, e gli uomini hanno amato le tenebre più che la luce, perché le loro opere erano malvage.
\par 20 Poiché chiunque fa cose malvage odia la luce e non viene alla luce, perché le sue opere non siano riprovate;
\par 21 ma chi mette in pratica la verità viene alla luce, affinché le opere sue siano manifestate, perché son fatte in Dio.
\par 22 Dopo queste cose, Gesù venne co' suoi discepoli nelle campagne della Giudea; quivi si trattenne con loro, e battezzava.
\par 23 Or anche Giovanni stava battezzando a Enon, presso Salim, perché c'era là molt'acqua; e la gente veniva a farsi battezzare.
\par 24 Poiché Giovanni non era ancora stato messo in prigione.
\par 25 Nacque dunque una discussione fra i discepoli di Giovanni e un Giudeo intorno alla purificazione.
\par 26 E vennero a Giovanni e gli dissero: Maestro, colui che era con te di là dal Giordano, e al quale tu rendesti testimonianza, eccolo che battezza, e tutti vanno a lui.
\par 27 Giovanni rispose dicendo: L'uomo non può ricever cosa alcuna, se non gli è data dal cielo.
\par 28 Voi stessi mi siete testimoni che ho detto: Io non sono il Cristo; ma son mandato davanti a lui.
\par 29 Colui che ha la sposa è lo sposo; ma l'amico dello sposo, che è presente e l'ascolta, si rallegra grandemente alla voce dello sposo; questa allegrezza che è la mia è perciò completa.
\par 30 Bisogna che egli cresca, e che io diminuisca.
\par 31 Colui che vien dall'alto è sopra tutti; colui che vien dalla terra è della terra e parla com'essendo della terra: colui che vien dal cielo è sopra tutti.
\par 32 Egli rende testimonianza di quel che ha veduto e udito, ma nessuno riceve la sua testimonianza.
\par 33 Chi ha ricevuto la sua testimonianza ha confermato che Dio è verace.
\par 34 Poiché colui che Dio ha mandato, proferisce le parole di Dio; perché Dio non gli dà lo Spirito con misura.
\par 35 Il Padre ama il Figliuolo, e gli ha dato ogni cosa in mano.
\par 36 Chi crede nel Figliuolo ha vita eterna; ma chi rifiuta di credere al Figliuolo non vedrà la vita, ma l'ira di Dio resta sopra lui.

\chapter{4}

\par 1 Quando dunque il Signore ebbe saputo che i Farisei aveano udito ch'egli faceva e battezzava più discepoli di Giovanni
\par 2 (quantunque non fosse Gesù che battezzava, ma i suoi discepoli),
\par 3 lasciò la Giudea e se n'andò di nuovo in Galilea.
\par 4 Or doveva passare per la Samaria.
\par 5 Giunse dunque a una città della Samaria, chiamata Sichar, vicina al podere che Giacobbe dette a Giuseppe, suo figliuolo;
\par 6 e quivi era la fonte di Giacobbe. Gesù dunque, stanco del cammino, stava così a sedere presso la fonte. Era circa l'ora sesta.
\par 7 Una donna samaritana venne ad attingere l'acqua. Gesù le disse: Dammi da bere.
\par 8 (Giacché i suoi discepoli erano andati in città a comprar da mangiare).
\par 9 Onde la donna samaritana gli disse: Come mai tu che sei giudeo chiedi da bere a me che sono una donna samaritana? Infatti i Giudei non hanno relazioni co' Samaritani.
\par 10 Gesù rispose e le disse: Se tu conoscessi il dono di Dio e chi è che ti dice: Dammi da bere, tu stessa gliene avresti chiesto, ed egli t'avrebbe dato dell'acqua viva.
\par 11 La donna gli disse: Signore, tu non hai nulla per attingere, e il pozzo è profondo; donde hai dunque cotest'acqua viva?
\par 12 Sei tu più grande di Giacobbe nostro padre che ci dette questo pozzo e ne bevve egli stesso co' suoi figliuoli e il suo bestiame?
\par 13 Gesù rispose e le disse: Chiunque beve di quest'acqua avrà sete di nuovo;
\par 14 ma chi beve dell'acqua che io gli darò, non avrà mai più sete; anzi, l'acqua che io gli darò, diventerà in lui una fonte d'acqua che scaturisce in vita eterna.
\par 15 La donna gli disse: Signore, dammi di cotest'acqua, affinché io non abbia più sete, e non venga più sin qua ad attingere.
\par 16 Gesù le disse: Va' a chiamar tuo marito e vieni qua.
\par 17 La donna gli rispose: Non ho marito. E Gesù: Hai detto bene: Non ho marito;
\par 18 perché hai avuto cinque mariti; e quello che hai ora, non è tuo marito; in questo hai detto il vero.
\par 19 La donna gli disse: Signore, io vedo che tu sei un profeta.
\par 20 I nostri padri hanno adorato su questo monte, e voi dite che a Gerusalemme è il luogo dove bisogna adorare.
\par 21 Gesù le disse: Donna, credimi; l'ora viene che né su questo monte né a Gerusalemme adorerete il Padre.
\par 22 Voi adorate quel che non conoscete; noi adoriamo quel che conosciamo, perché la salvazione vien da' Giudei.
\par 23 Ma l'ora viene, anzi è già venuta, che i veri adoratori adoreranno il Padre in ispirito e verità; poiché tali sono gli adoratori che il Padre richiede.
\par 24 Iddio è spirito; e quelli che l'adorano, bisogna che l'adorino in ispirito e verità.
\par 25 La donna gli disse: Io so che il Messia (ch'è chiamato Cristo) ha da venire; quando sarà venuto, ci annunzierà ogni cosa.
\par 26 Gesù le disse: Io che ti parlo, son desso.
\par 27 In quel mentre giunsero i suoi discepoli, e si maravigliarono ch'egli parlasse con una donna; ma pur nessuno gli chiese: Che cerchi? o: Perché discorri con lei?
\par 28 La donna lasciò dunque la sua secchia, se ne andò in città e disse alla gente:
\par 29 Venite a vedere un uomo che m'ha detto tutto quello che ho fatto; non sarebb'egli il Cristo?
\par 30 La gente uscì dalla città e veniva a lui.
\par 31 Intanto i discepoli lo pregavano, dicendo: Maestro, mangia.
\par 32 Ma egli disse loro: Io ho un cibo da mangiare che voi non sapete.
\par 33 Perciò i discepoli si dicevano l'uno all'altro: Forse qualcuno gli ha portato da mangiare?
\par 34 Gesù disse loro: Il mio cibo è di far la volontà di Colui che mi ha mandato, e di compiere l'opera sua.
\par 35 Non dite voi che ci sono ancora quattro mesi e poi vien la mietitura? Ecco, io vi dico: Levate gli occhi e mirate le campagne come già son bianche da mietere.
\par 36 Il mietitore riceve premio e raccoglie frutto per la vita eterna, affinché il seminatore e il mietitore si rallegrino assieme.
\par 37 Poiché in questo è vero il detto: L'uno semina e l'altro miete.
\par 38 Io v'ho mandati a mieter quello intorno a cui non avete faticato; altri hanno faticato, e voi siete entrati nella lor fatica.
\par 39 Or molti de' Samaritani di quella città credettero in lui a motivo della testimonianza resa da quella donna: Egli m'ha detto tutte le cose che ho fatte.
\par 40 Quando dunque i Samaritani furon venuti a lui, lo pregarono di trattenersi da loro; ed egli si trattenne quivi due giorni.
\par 41 E più assai credettero a motivo della sua parola;
\par 42 e dicevano alla donna: Non è più a motivo di quel che tu ci hai detto, che crediamo; perché abbiamo udito da noi, e sappiamo che questi è veramente il Salvator del mondo.
\par 43 Passati que' due giorni, egli partì di là per andare in Galilea;
\par 44 poiché Gesù stesso aveva attestato che un profeta non è onorato nella sua propria patria.
\par 45 Quando dunque, fu venuto in Galilea, fu accolto dai Galilei, perché avean vedute tutte le cose ch'egli avea fatte in Gerusalemme alla festa; poiché anch'essi erano andati alla festa.
\par 46 Gesù dunque venne di nuovo a Cana di Galilea, dove avea cambiato l'acqua in vino. E v'era un certo ufficial reale, il cui figliuolo era infermo a Capernaum.
\par 47 Come egli ebbe udito che Gesù era venuto dalla Giudea in Galilea, andò a lui e lo pregò che scendesse e guarisse il suo figliuolo, perché stava per morire.
\par 48 Perciò Gesù gli disse: Se non vedete segni e miracoli, voi non crederete.
\par 49 L'ufficial reale gli disse: Signore, scendi prima che il mio bambino muoia.
\par 50 Gesù gli disse: Va', il tuo figliuolo vive. Quell'uomo credette alla parola che Gesù gli avea detta, e se ne andò.
\par 51 E come già stava scendendo, i suoi servitori gli vennero incontro e gli dissero: Il tuo figliuolo vive.
\par 52 Allora egli domandò loro a che ora avesse cominciato a star meglio; ed essi gli risposero: Ieri, all'ora settima, la febbre lo lasciò.
\par 53 Così il padre conobbe che ciò era avvenuto nell'ora che Gesù gli avea detto: Il tuo figliuolo vive; e credette lui con tutta la sua casa.
\par 54 Questo secondo miracolo fece di nuovo Gesù, tornando dalla Giudea in Galilea.

\chapter{5}

\par 1 Dopo queste cose ci fu una festa de' Giudei, e Gesù salì a Gerusalemme.
\par 2 Or a Gerusalemme, presso la porta delle Pecore, v'è una vasca, chiamata in ebraico Betesda, che ha cinque portici.
\par 3 Sotto questi portici giaceva un gran numero d'infermi, di ciechi, di zoppi, di paralitici.
\par 4 sql
\par 5 E quivi era un uomo, che da trentott'anni era infermo.
\par 6 Gesù, vedutolo che giaceva e sapendo che già da gran tempo stava così, gli disse: Vuoi esser risanato?
\par 7 L'infermo gli rispose: Signore, io non ho alcuno che, quando l'acqua è mossa, mi metta nella vasca, e mentre ci vengo io, un altro vi scende prima di me.
\par 8 Gesù gli disse: Lèvati, prendi il tuo lettuccio, e cammina.
\par 9 E in quell'istante quell'uomo fu risanato; e preso il suo lettuccio, si mise a camminare.
\par 10 Or quel giorno era un sabato; perciò i Giudei dissero all'uomo guarito: È sabato, e non ti è lecito portare il tuo lettuccio.
\par 11 Ma egli rispose loro: È colui che m'ha guarito, che m'ha detto: Prendi il tuo lettuccio e cammina.
\par 12 Essi gli domandarono: Chi è quell'uomo che t'ha detto: Prendi il tuo lettuccio e cammina?
\par 13 Ma colui ch'era stato guarito non sapeva chi fosse; perché Gesù era scomparso, essendovi in quel luogo molta gente.
\par 14 Di poi Gesù lo trovò nel tempio, e gli disse: Ecco, tu sei guarito; non peccar più, che non t'accada di peggio.
\par 15 Quell'uomo se ne andò, e disse ai Giudei che Gesù era quel che l'avea risanato.
\par 16 E per questo i Giudei perseguitavano Gesù e cercavan d'ucciderlo; perché facea quelle cose di sabato.
\par 17 Gesù rispose loro: Il Padre mio opera fino ad ora, ed anche io opero.
\par 18 Perciò dunque i Giudei più che mai cercavan d'ucciderlo; perché non soltanto violava il sabato, ma chiamava Dio suo Padre, facendosi uguale a Dio.
\par 19 Gesù quindi rispose e disse loro: In verità, in verità io vi dico che il Figliuolo non può da se stesso far cosa alcuna, se non la vede fare dal Padre; perché le cose che il Padre fa, anche il Figlio le fa similmente.
\par 20 Poiché il Padre ama il Figliuolo, e gli mostra tutto quello che Egli fa; e gli mostrerà delle opere maggiori di queste, affinché ne restiate maravigliati.
\par 21 Difatti, come il Padre risuscita i morti e li vivifica, così anche il Figliuolo vivifica chi vuole.
\par 22 Oltre a ciò, il Padre non giudica alcuno, ma ha dato tutto il giudicio al Figliuolo,
\par 23 affinché tutti onorino il Figliuolo come onorano il Padre. Chi non onora il Figliuolo non onora il Padre che l'ha mandato.
\par 24 In verità, in verità io vi dico: Chi ascolta la mia parola e crede a Colui che mi ha mandato, ha vita eterna; e non viene in giudizio, ma è passato dalla morte alla vita.
\par 25 In verità, in verità io vi dico: L'ora viene, anzi è già venuta, che i morti udranno la voce del Figliuol di Dio; e quelli che l'avranno udita, vivranno.
\par 26 Perché come il Padre ha vita in se stesso, così ha dato anche al Figliuolo d'aver vita in se stesso;
\par 27 e gli ha dato autorità di giudicare, perché è il Figliuol dell'uomo.
\par 28 Non vi maravigliate di questo; perché l'ora viene in cui tutti quelli che son nei sepolcri, udranno la sua voce e ne verranno fuori:
\par 29 quelli che hanno operato bene, in risurrezione di vita; e quelli che hanno operato male, in risurrezion di giudicio.
\par 30 Io non posso far nulla da me stesso; come odo, giudico; e il mio giudicio è giusto, perché cerco non la mia propria volontà, ma la volontà di Colui che mi ha mandato.
\par 31 Se io rendo testimonianza di me stesso, la mia testimonianza non è verace.
\par 32 V'è un altro che rende testimonianza di me; e io so che la testimonianza ch'egli rende di me, è verace.
\par 33 Voi avete mandato da Giovanni, ed egli ha reso testimonianza alla verità.
\par 34 Io però la testimonianza non la prendo dall'uomo, ma dico questo affinché voi siate salvati.
\par 35 Egli era la lampada ardente e splendente e voi avete voluto per breve ora godere alla sua luce.
\par 36 Ma io ho una testimonianza maggiore di quella di Giovanni; perché le opere che il Padre mi ha dato a compiere, quelle opere stesse che io fo, testimoniano di me che il Padre mi ha mandato.
\par 37 E il Padre che mi ha mandato, ha Egli stesso reso testimonianza di me. La sua voce, voi non l'avete mai udita; il suo sembiante, non l'avete mai veduto;
\par 38 e la sua parola non l'avete dimorante in voi, perché non credete in colui ch'Egli ha mandato.
\par 39 Voi investigate le Scritture, perché pensate aver per mezzo d'esse vita eterna, ed esse son quelle che rendon testimonianza di me;
\par 40 eppure non volete venire a me per aver la vita!
\par 41 Io non prendo gloria dagli uomini;
\par 42 ma vi conosco, che non avete l'amor di Dio in voi.
\par 43 Io son venuto nel nome del Padre mio, e voi non mi ricevete; se un altro verrà nel suo proprio nome, voi lo riceverete.
\par 44 Come potete credere, voi che prendete gloria gli uni dagli altri e non cercate la gloria che vien da Dio solo?
\par 45 Non crediate che io sia colui che vi accuserà davanti al Padre; v'è chi v'accusa, ed è Mosè, nel quale avete riposta la vostra speranza.
\par 46 Perché se credeste a Mosè, credereste anche a me; poiché egli ha scritto di me.
\par 47 Ma se non credete agli scritti di lui, come crederete alle mie parole?

\chapter{6}

\par 1 Dopo queste cose, Gesù se ne andò all'altra riva del mar di Galilea, ch'è il mar di Tiberiade.
\par 2 E una gran moltitudine lo seguiva, perché vedeva i miracoli ch'egli faceva sugl'infermi.
\par 3 Ma Gesù salì sul monte e quivi si pose a sedere co' suoi discepoli.
\par 4 Or la Pasqua, la festa dei Giudei, era vicina.
\par 5 Gesù dunque, alzati gli occhi e vedendo che una gran folla veniva a lui, disse a Filippo: Dove comprerem noi del pane perché questa gente abbia da mangiare?
\par 6 Diceva così per provarlo; perché sapeva bene quel che stava per fare.
\par 7 Filippo gli rispose: Dugento denari di pane non bastano perché ciascun di loro n'abbia un pezzetto.
\par 8 Uno de' suoi discepoli, Andrea, fratello di Simon Pietro, gli disse:
\par 9 V'è qui un ragazzo che ha cinque pani d'orzo e due pesci; ma che cosa sono per tanta gente?
\par 10 Gesù disse: Fateli sedere. Or v'era molt'erba in quel luogo. La gente dunque si sedette, ed eran circa cinquemila uomini.
\par 11 Gesù quindi prese i pani; e dopo aver rese grazie, li distribuì alla gente seduta; lo stesso fece de' pesci, quanto volevano.
\par 12 E quando furon saziati, disse ai suoi discepoli: Raccogliete i pezzi avanzati, ché nulla se ne perda.
\par 13 Essi quindi li raccolsero, ed empiron dodici ceste di pezzi che di que' cinque pani d'orzo erano avanzati a quelli che avean mangiato.
\par 14 La gente dunque, avendo veduto il miracolo che Gesù avea fatto, disse: Questi è certo il profeta che ha da venire al mondo.
\par 15 Gesù quindi, sapendo che stavan per venire a rapirlo per farlo re, si ritirò di nuovo sul monte, tutto solo.
\par 16 E quando fu sera, i suoi discepoli scesero al mare;
\par 17 e montati in una barca, si dirigevano all'altra riva, verso Capernaum. Già era buio, e Gesù non era ancora venuto a loro.
\par 18 E il mare era agitato, perché tirava un gran vento.
\par 19 Or com'ebbero vogato circa venticinque o trenta stadi, videro Gesù che camminava sul mare e s'accostava alla barca; ed ebbero paura.
\par 20 Ma egli disse loro: Son io, non temete.
\par 21 Essi dunque lo vollero prendere nella barca, e subito la barca toccò terra là dove eran diretti.
\par 22 La folla che era rimasta all'altra riva del mare avea notato che non v'era quivi altro che una barca sola, e che Gesù non v'era entrato co' suoi discepoli, ma che i discepoli eran partiti soli.
\par 23 Or altre barche eran giunte da Tiberiade, presso al luogo dove avean mangiato il pane dopo che il Signore avea reso grazie.
\par 24 La folla, dunque, quando l'indomani ebbe veduto che Gesù non era quivi, né che v'erano i suoi discepoli, montò in quelle barche, e venne a Capernaum in cerca di Gesù.
\par 25 E trovatolo di là dal mare, gli dissero: Maestro, quando se' giunto qua?
\par 26 Gesù rispose loro e disse: In verità, in verità vi dico che voi mi cercate, non perché avete veduto dei miracoli, ma perché avete mangiato de' pani e siete stati saziati.
\par 27 Adopratevi non per il cibo che perisce, ma per il cibo che dura in vita eterna, il quale il Figliuol dell'uomo vi darà; poiché su lui il Padre, cioè Dio, ha apposto il proprio suggello.
\par 28 Essi dunque gli dissero: Che dobbiam fare per operare le opere di Dio?
\par 29 Gesù rispose e disse loro: Questa è l'opera di Dio: che crediate in colui che Egli ha mandato.
\par 30 Allora essi gli dissero: Qual segno fai tu dunque perché lo vediamo e ti crediamo? Che operi?
\par 31 I nostri padri mangiaron la manna nel deserto, com'è scritto: Egli diè loro da mangiare del pane venuto dal cielo.
\par 32 E Gesù disse loro: In verità vi dico che non Mosè vi ha dato il pane che vien dal cielo, ma il Padre mio vi dà il vero pane che viene dal cielo.
\par 33 Poiché il pan di Dio è quello che scende dal cielo, e dà vita al mondo. Essi quindi gli dissero:
\par 34 Signore, dacci sempre di codesto pane.
\par 35 Gesù disse loro: Io sono il pan della vita; chi viene a me non avrà fame, e chi crede in me non avrà mai sete.
\par 36 Ma io ve l'ho detto: Voi m'avete veduto, eppur non credete!
\par 37 Tutto quel che il Padre mi dà, verrà a me; e colui che viene a me, io non lo caccerò fuori;
\par 38 perché son disceso dal cielo per fare non la mia volontà, ma la volontà di Colui che mi ha mandato.
\par 39 E questa è la volontà di Colui che mi ha mandato: ch'io non perda nulla di tutto quel ch'Egli m'ha dato, ma che lo risusciti nell'ultimo giorno.
\par 40 Poiché questa è la volontà del Padre mio: che chiunque contempla il Figliuolo e crede in lui, abbia vita eterna; e io lo risusciterò nell'ultimo giorno.
\par 41 I Giudei perciò mormoravano di lui perché avea detto: Io sono il pane che è disceso dal cielo.
\par 42 E dicevano: Non è costui Gesù, il figliuol di Giuseppe, del quale conosciamo il padre e la madre? Come mai dice egli ora: Io son disceso dal cielo?
\par 43 Gesù rispose e disse loro: Non mormorate fra voi.
\par 44 Niuno può venire a me se non che il Padre, il quale mi ha mandato, lo attiri; e io lo risusciterò nell'ultimo giorno.
\par 45 È scritto nei profeti: E saranno tutti ammaestrati da Dio. Ogni uomo che ha udito il Padre ed ha imparato da lui, viene a me.
\par 46 Non che alcuno abbia veduto il Padre, se non colui che è da Dio; egli ha veduto il Padre.
\par 47 In verità, in verità io vi dico: Chi crede ha vita eterna.
\par 48 Io sono il pan della vita.
\par 49 I vostri padri mangiarono la manna nel deserto e morirono.
\par 50 Questo è il pane che discende dal cielo, affinché chi ne mangia non muoia.
\par 51 Io sono il pane vivente, che è disceso dal cielo; se uno mangia di questo pane vivrà in eterno; e il pane che darò è la mia carne, che darò per la vita del mondo.
\par 52 I Giudei dunque disputavano fra di loro, dicendo: Come mai può costui darci a mangiare la sua carne?
\par 53 Perciò Gesù disse loro: In verità, in verità io vi dico che se non mangiate la carne del Figliuol dell'uomo e non bevete il suo sangue, non avete la vita in voi.
\par 54 Chi mangia la mia carne e beve il mio sangue ha vita eterna; e io lo risusciterò nell'ultimo giorno.
\par 55 Perché la mia carne è vero cibo e il mio sangue è vera bevanda.
\par 56 Chi mangia la mia carne e beve il mio sangue dimora in me, ed io in lui.
\par 57 Come il vivente Padre mi ha mandato e io vivo a cagion del Padre, così chi mi mangia vivrà anch'egli a cagion di me.
\par 58 Questo è il pane che è disceso dal cielo; non qual era quello che i padri mangiarono e morirono; chi mangia di questo pane vivrà in eterno.
\par 59 Queste cose disse Gesù, insegnando nella sinagoga di Capernaum.
\par 60 Onde molti dei suoi discepoli, udite che l'ebbero, dissero: Questo parlare è duro; chi lo può ascoltare?
\par 61 Ma Gesù, conoscendo in se stesso che i suoi discepoli mormoravan di ciò, disse loro: Questo vi scandalizza?
\par 62 E che sarebbe se vedeste il Figliuol dell'uomo ascendere dov'era prima?
\par 63 È lo spirito quel che vivifica; la carne non giova nulla; le parole che vi ho dette sono spirito e vita.
\par 64 Ma fra voi ve ne sono alcuni che non credono. Poiché Gesù sapeva fin da principio chi eran quelli che non credevano, e chi era colui che lo tradirebbe.
\par 65 E diceva: Per questo v'ho detto che niuno può venire a me, se non gli è dato dal Padre.
\par 66 D'allora molti de' suoi discepoli si ritrassero indietro e non andavan più con lui.
\par 67 Perciò Gesù disse ai dodici: Non ve ne volete andare anche voi?
\par 68 Simon Pietro gli rispose: Signore, a chi ce ne andremmo noi? Tu hai parole di vita eterna;
\par 69 e noi abbiam creduto ed abbiam conosciuto che tu sei il Santo di Dio.
\par 70 Gesù rispose loro: Non ho io scelto voi dodici? Eppure, uno di voi è un diavolo.
\par 71 Or egli parlava di Giuda, figliuol di Simone Iscariota, perché era lui, uno di quei dodici, che lo dovea tradire.

\chapter{7}

\par 1 Dopo queste cose, Gesù andava attorno per la Galilea; non voleva andare attorno per la Giudea perché i Giudei cercavan d'ucciderlo.
\par 2 Or la festa de' Giudei, detta delle Capanne, era vicina.
\par 3 Perciò i suoi fratelli gli dissero: Partiti di qua e vattene in Giudea, affinché i tuoi discepoli veggano anch'essi le opere che tu fai.
\par 4 Poiché niuno fa cosa alcuna in segreto, quando cerca d'esser riconosciuto pubblicamente. Se tu fai codeste cose, palesati al mondo.
\par 5 Poiché neppure i suoi fratelli credevano in lui.
\par 6 Gesù quindi disse loro: Il mio tempo non è ancora venuto; il vostro tempo, invece, è sempre pronto.
\par 7 Il mondo non può odiar voi; ma odia me, perché io testimonio di lui che le sue opere sono malvage.
\par 8 Salite voi alla festa; io non salgo ancora a questa festa, perché il mio tempo non è ancora compiuto.
\par 9 E dette loro queste cose, rimase in Galilea.
\par 10 Quando poi i suoi fratelli furono saliti alla festa, allora vi salì anche lui; non palesemente, ma come di nascosto.
\par 11 I Giudei dunque lo cercavano durante la festa, e dicevano: Dov'è egli?
\par 12 E v'era fra le turbe gran mormorio intorno a lui. Gli uni dicevano: È un uomo dabbene! Altri dicevano: No, anzi, travia la moltitudine!
\par 13 Nessuno però parlava di lui apertamente, per paura de' Giudei.
\par 14 Or quando s'era già a metà della festa, Gesù salì al tempio e si mise a insegnare.
\par 15 Onde i Giudei si maravigliavano e dicevano: Come mai s'intende costui di lettere, senz'aver fatto studi?
\par 16 E Gesù rispose loro e disse: La mia dottrina non è mia, ma di Colui che mi ha mandato.
\par 17 Se uno vuol fare la volontà di lui, conoscerà se questa dottrina è da Dio o se io parlo di mio.
\par 18 Chi parla di suo cerca la propria gloria; ma chi cerca la gloria di colui che l'ha mandato, egli è verace e non v'è ingiustizia in lui.
\par 19 Mosè non v'ha egli data la legge? Eppure nessun di voi mette ad effetto la legge! Perché cercate d'uccidermi?
\par 20 La moltitudine rispose: Tu hai un demonio! Chi cerca di ucciderti?
\par 21 Gesù rispose e disse loro: Un'opera sola ho fatto, e tutti ve ne maravigliate.
\par 22 Mosè v'ha dato la circoncisione (non che venga da Mosè, ma viene dai padri); e voi circoncidete l'uomo in giorno di sabato.
\par 23 Se un uomo riceve la circoncisione di sabato affinché la legge di Mosè non sia violata, vi adirate voi contro a me perché in giorno di sabato ho guarito un uomo tutto intero?
\par 24 Non giudicate secondo l'apparenza, ma giudicate con giusto giudizio.
\par 25 Dicevano dunque alcuni di Gerusalemme: Non è questi colui che cercano di uccidere?
\par 26 Eppure, ecco, egli parla liberamente, e non gli dicon nulla. Avrebbero mai i capi riconosciuto per davvero ch'egli è il Cristo?
\par 27 Eppure, costui sappiamo donde sia; ma quando il Cristo verrà, nessuno saprà donde egli sia.
\par 28 Gesù dunque, insegnando nel tempio, esclamò: Voi e mi conoscete e sapete di dove sono; però io non son venuto da me, ma Colui che mi ha mandato è verità, e voi non lo conoscete.
\par 29 Io lo conosco, perché vengo da lui, ed è Lui che mi ha mandato.
\par 30 Cercavan perciò di pigliarlo, ma nessuno gli mise le mani addosso, perché l'ora sua non era ancora venuta.
\par 31 Ma molti della folla credettero in lui, e dicevano: Quando il Cristo sarà venuto, farà egli più miracoli che questi non abbia fatto?
\par 32 I Farisei udirono la moltitudine mormorare queste cose di lui; e i capi sacerdoti e i Farisei mandarono delle guardie a pigliarlo.
\par 33 Perciò Gesù disse loro: Io sono ancora con voi per poco tempo; poi me ne vo a Colui che mi ha mandato.
\par 34 Voi mi cercherete e non mi troverete; e dove io sarò, voi non potete venire.
\par 35 Perciò i Giudei dissero fra loro: Dove dunque andrà egli che noi non lo troveremo? Andrà forse a quelli che son dispersi fra i Greci, ad ammaestrare i Greci?
\par 36 Che significa questo suo dire: Voi mi cercherete e non mi troverete; e: Dove io sarò voi non potete venire?
\par 37 Or nell'ultimo giorno, il gran giorno della festa, Gesù stando in piè, esclamò: Se alcuno ha sete, venga a me e beva.
\par 38 Chi crede in me, come ha detto la Scrittura, fiumi d'acqua viva sgorgheranno dal suo seno.
\par 39 Or disse questo dello Spirito, che doveano ricevere quelli che crederebbero in lui; poiché lo Spirito non era ancora stato dato, perché Gesù non era ancora glorificato.
\par 40 Una parte dunque della moltitudine, udite quelle parole, diceva: Questi è davvero il profeta.
\par 41 Altri dicevano: Questi è il Cristo. Altri, invece, dicevano: Ma è forse dalla Galilea che viene il Cristo?
\par 42 La Scrittura non ha ella detto che il Cristo viene dalla progenie di Davide e da Betleem, il villaggio dove stava Davide?
\par 43 Vi fu dunque dissenso fra la moltitudine, a motivo di lui;
\par 44 e alcuni di loro lo voleano pigliare, ma nessuno gli mise le mani addosso.
\par 45 Le guardie dunque tornarono dai capi sacerdoti e dai Farisei, i quali dissero loro: Perché non l'avete condotto?
\par 46 Le guardie risposero: Nessun uomo parlò mai come quest'uomo!
\par 47 Onde i Farisei replicaron loro: Siete stati sedotti anche voi?
\par 48 Ha qualcuno de' capi o de' Farisei creduto in lui?
\par 49 Ma questa plebe, che non conosce la legge, è maledetta!
\par 50 Nicodemo (un di loro, quello che prima era venuto a lui) disse loro:
\par 51 La nostra legge giudica ella un uomo prima che sia stato udito e che si sappia quel che ha fatto?
\par 52 Essi gli risposero: Sei anche tu di Galilea? Investiga, e vedrai che dalla Galilea non sorge profeta.
\par 53 E ognuno se ne andò a casa sua;

\chapter{8}

\par 1 ma Gesù andò al monte degli Ulivi.
\par 2 E sul far del giorno, tornò nel tempio, e tutto il popolo venne a lui; ed egli, postosi a sedere, li ammaestrava.
\par 3 Allora gli scribi e i Farisei gli menarono una donna còlta in adulterio; e fattala stare in mezzo,
\par 4 gli dissero: Maestro, questa donna è stata còlta in flagrante adulterio.
\par 5 Or Mosè, nella legge, ci ha comandato di lapidare queste tali; e tu che ne dici?
\par 6 Or dicean questo per metterlo alla prova, per poterlo accusare. Ma Gesù, chinatosi, si mise a scrivere col dito in terra.
\par 7 E siccome continuavano a interrogarlo, egli, rizzatosi, disse loro: Chi di voi è senza peccato, scagli il primo la pietra contro di lei.
\par 8 E chinatosi di nuovo, scriveva in terra.
\par 9 Ed essi, udito ciò, e ripresi dalla loro coscienza, si misero ad uscire ad uno ad uno, cominciando dai più vecchi fino agli ultimi; e Gesù fu lasciato solo con la donna che stava là in mezzo.
\par 10 E Gesù, rizzatosi e non vedendo altri che la donna, le disse: Donna, dove sono que' tuoi accusatori? Nessuno t'ha condannata?
\par 11 Ed ella rispose: Nessuno, Signore. E Gesù le disse: Neppure io ti condanno; va' e non peccar più.
\par 12 Or Gesù parlò loro di nuovo, dicendo: Io son la luce del mondo; chi mi seguita non camminerà nelle tenebre, ma avrà la luce della vita.
\par 13 Allora i Farisei gli dissero: Tu testimoni di te stesso; la tua testimonianza non è verace.
\par 14 Gesù rispose e disse loro: Quand'anche io testimoni di me stesso, la mia testimonianza è verace, perché so donde son venuto e dove vado; ma voi non sapete donde io vengo né dove vado.
\par 15 Voi giudicate secondo la carne; io non giudico alcuno.
\par 16 E anche se giudico, il mio giudizio è verace, perché non son solo, ma son io col Padre che mi ha mandato.
\par 17 D'altronde nella vostra legge è scritto che la testimonianza di due uomini è verace.
\par 18 Or son io a testimoniar di me stesso, e il Padre che mi ha mandato testimonia pur di me.
\par 19 Onde essi gli dissero: Dov'è tuo padre? Gesù rispose: Voi non conoscete né me né il Padre mio: se conosceste me, conoscereste anche il Padre mio.
\par 20 Queste parole disse Gesù nel tesoro, insegnando nel tempio; e nessuno lo prese, perché l'ora sua non era ancora venuta.
\par 21 Egli dunque disse loro di nuovo: Io me ne vado, e voi mi cercherete, e morrete nel vostro peccato; dove vado io, voi non potete venire.
\par 22 Perciò i Giudei dicevano: S'ucciderà egli forse, poiché dice: Dove vado io voi non potete venire?
\par 23 Ed egli diceva loro: Voi siete di quaggiù; io sono di lassù; voi siete di questo mondo; io non sono di questo mondo.
\par 24 Perciò v'ho detto che morrete ne' vostri peccati; perché se non credete che sono io (il Cristo), morrete nei vostri peccati.
\par 25 Allora gli domandarono: Chi sei tu? Gesù rispose loro: Sono per l'appunto quel che vo dicendovi.
\par 26 Ho molte cose da dire e da giudicare sul conto vostro; ma Colui che mi ha mandato è verace, e le cose che ho udito da lui, le dico al mondo.
\par 27 Essi non capirono ch'egli parlava loro del Padre.
\par 28 Gesù dunque disse loro: Quando avrete innalzato il Figliuol dell'uomo, allora conoscerete che son io (il Cristo) e che non fo nulla da me, ma dico queste cose secondo che il Padre m'ha insegnato.
\par 29 E Colui che mi ha mandato è meco; Egli non mi ha lasciato solo, perché fo del continuo le cose che gli piacciono.
\par 30 Mentr'egli parlava così, molti credettero in lui.
\par 31 Gesù allora prese a dire a que' Giudei che aveano creduto in lui: Se perseverate nella mia parola, siete veramente miei discepoli;
\par 32 e conoscerete la verità, e la verità vi farà liberi.
\par 33 Essi gli risposero: Noi siamo progenie d'Abramo, e non siamo mai stati schiavi di alcuno; come puoi tu dire: Voi diverrete liberi?
\par 34 Gesù rispose loro: In verità, in verità vi dico che chi commette il peccato è schiavo del peccato.
\par 35 Or lo schiavo non dimora per sempre nella casa: il figliuolo vi dimora per sempre.
\par 36 Se dunque il Figliuolo vi farà liberi, sarete veramente liberi.
\par 37 Io so che siete progenie d'Abramo; ma cercate d'uccidermi, perché la mia parola non penetra in voi.
\par 38 Io dico quel che ho veduto presso il Padre mio; e voi pure fate le cose che avete udite dal padre vostro.
\par 39 Essi risposero e gli dissero: Il padre nostro è Abramo. Gesù disse loro: Se foste figliuoli d'Abramo, fareste le opere d'Abramo;
\par 40 ma ora cercate d'uccider me, uomo che v'ho detta la verità che ho udita da Dio; così non fece Abramo.
\par 41 Voi fate le opere del padre vostro. Essi gli dissero: Noi non siam nati di fornicazione; abbiamo un solo Padre: Iddio.
\par 42 Gesù disse loro: Se Dio fosse vostro Padre, amereste me, perché io son proceduto e vengo da Dio, perché io non son venuto da me, ma è Lui che mi ha mandato.
\par 43 Perché non comprendete il mio parlare? Perché non potete dare ascolto alla mia parola.
\par 44 Voi siete progenie del diavolo, ch'è vostro padre, e volete fare i desideri del padre vostro. Egli è stato omicida fin dal principio e non si è attenuto alla verità, perché non c'è verità in lui. Quando parla il falso, parla del suo, perché è bugiardo e padre della menzogna.
\par 45 E a me, perché dico la verità, voi non credete.
\par 46 Chi di voi mi convince di peccato? Se vi dico la verità, perché non mi credete?
\par 47 Chi è da Dio ascolta le parole di Dio. Per questo voi non le ascoltate; perché non siete da Dio.
\par 48 I Giudei risposero e gli dissero: Non diciam noi bene che sei un Samaritano e che hai un demonio?
\par 49 Gesù rispose: Io non ho un demonio, ma onoro il Padre mio e voi mi disonorate.
\par 50 Ma io non cerco la mia gloria; v'è Uno che la cerca e che giudica.
\par 51 In verità, in verità vi dico che se uno osserva la mia parola, non vedrà mai la morte.
\par 52 I Giudei gli dissero: Ora vediam bene che tu hai un demonio. Abramo e i profeti son morti, e tu dici: Se uno osserva la mia parola, non gusterà mai la morte.
\par 53 Sei tu forse maggiore del padre nostro Abramo, il quale è morto? Anche i profeti son morti; chi pretendi d'essere?
\par 54 Gesù rispose: S'io glorifico me stesso, la mia gloria è un nulla; chi mi glorifica è il Padre mio, che voi dite esser vostro Dio,
\par 55 e non l'avete conosciuto; ma io lo conosco, e se dicessi di non conoscerlo, sarei un bugiardo come voi; ma io lo conosco e osservo la sua parola.
\par 56 Abramo, vostro padre, ha giubilato nella speranza di vedere il mio giorno; e l'ha veduto, e se n'è rallegrato.
\par 57 I Giudei gli dissero: Tu non hai ancora cinquant'anni e hai veduto Abramo?
\par 58 Gesù disse loro: In verità, in verità vi dico: Prima che Abramo fosse nato, io sono.
\par 59 Allora essi presero delle pietre per tirargliele; ma Gesù si nascose ed uscì dal tempio.

\chapter{9}

\par 1 E passando vide un uomo ch'era cieco fin dalla nascita.
\par 2 E i suoi discepoli lo interrogarono, dicendo: Maestro, chi ha peccato, lui o i suoi genitori, perché sia nato cieco?
\par 3 Gesù rispose: Né lui peccò, né i suoi genitori; ma è così, affinché le opere di Dio siano manifestate in lui.
\par 4 Bisogna che io compia le opere di Colui che mi ha mandato, mentre è giorno; la notte viene in cui nessuno può operare.
\par 5 Mentre sono nel mondo, io son la luce del mondo.
\par 6 Detto questo, sputò in terra, fece del fango con la saliva e ne spalmò gli occhi del cieco,
\par 7 e gli disse: Va', lavati nella vasca di Siloe (che significa: mandato). Egli dunque andò e si lavò, e tornò che ci vedeva.
\par 8 Perciò i vicini e quelli che per l'innanzi l'avean veduto, perché era mendicante, dicevano: Non è egli quello che stava seduto a chieder l'elemosina?
\par 9 Gli uni dicevano: È lui. Altri dicevano: No, ma gli somiglia. Egli diceva: Son io.
\par 10 Allora essi gli domandarono: Com'è che ti sono stati aperti gli occhi?
\par 11 Egli rispose: Quell'uomo che si chiama Gesù fece del fango, me ne spalmò gli occhi e mi disse: Vattene a Siloe e lavati. Io quindi sono andato, e mi son lavato e ho ricuperato la vista.
\par 12 Ed essi gli dissero: Dov'è costui? Egli rispose: Non so.
\par 13 Menarono a' Farisei colui ch'era stato cieco.
\par 14 Or era in giorno di sabato che Gesù avea fatto il fango e gli avea aperto gli occhi.
\par 15 I Farisei dunque gli domandaron di nuovo anch'essi com'egli avesse ricuperata la vista. Ed egli disse loro: Egli mi ha messo del fango sugli occhi, mi son lavato, e ci veggo.
\par 16 Perciò alcuni dei Farisei dicevano: Quest'uomo non è da Dio perché non osserva il sabato. Ma altri dicevano: Come può un uomo peccatore far tali miracoli? E v'era disaccordo fra loro.
\par 17 Essi dunque dissero di nuovo al cieco: E tu, che dici di lui, dell'averti aperto gli occhi? Egli rispose: È un profeta.
\par 18 I Giudei dunque non credettero di lui che fosse stato cieco e avesse ricuperata la vista, finché non ebbero chiamato i genitori di colui che avea ricuperata la vista,
\par 19 e li ebbero interrogati così: È questo il vostro figliuolo che dite esser nato cieco? Com'è dunque che ora ci vede?
\par 20 I suoi genitori risposero: Sappiamo che questo è nostro figliuolo, e che è nato cieco;
\par 21 ma come ora ci veda, non sappiamo; né sappiamo chi gli abbia aperti gli occhi; domandatelo a lui; egli è d'età; parlerà lui di sé.
\par 22 Questo dissero i suoi genitori perché avean paura de' Giudei; poiché i Giudei avean già stabilito che se uno riconoscesse Gesù come Cristo, fosse espulso dalla sinagoga.
\par 23 Per questo dissero i suoi genitori: Egli è d'età, domandatelo a lui.
\par 24 Essi dunque chiamarono per la seconda volta l'uomo ch'era stato cieco, e gli dissero: Da' gloria a Dio! Noi sappiamo che quell'uomo è un peccatore.
\par 25 Egli rispose: S'egli sia un peccatore, non so, una cosa so, che ero cieco e ora ci vedo.
\par 26 Essi allora gli dissero: Che ti fece egli? Come t'aprì gli occhi?
\par 27 Egli rispose loro: Ve l'ho già detto e voi non avete ascoltato: perché volete udirlo di nuovo? Volete forse anche voi diventar suoi discepoli?
\par 28 Essi l'ingiuriarono e dissero: Sei tu discepolo di costui; ma noi siam discepoli di Mosè.
\par 29 Noi sappiamo che a Mosè Dio ha parlato; ma quant'è a costui, non sappiamo di dove sia.
\par 30 Quell'uomo rispose e disse loro: Questo poi è strano: che voi non sappiate di dove sia; eppure, m'ha aperto gli occhi!
\par 31 Si sa che Dio non esaudisce i peccatori; ma se uno è pio verso Dio e fa la sua volontà, quello egli esaudisce.
\par 32 Da che mondo è mondo non s'è mai udito che uno abbia aperto gli occhi ad un cieco nato.
\par 33 Se quest'uomo non fosse da Dio, non potrebbe far nulla.
\par 34 Essi risposero e gli dissero: Tu sei tutto quanto nato nel peccato e insegni a noi? E lo cacciaron fuori.
\par 35 Gesù udì che l'avean cacciato fuori; e trovatolo gli disse: Credi tu nel Figliuol di Dio?
\par 36 Colui rispose: E chi è egli, Signore, perché io creda in lui?
\par 37 Gesù gli disse: Tu l'hai già veduto; e quei che parla teco, è lui.
\par 38 Ed egli disse: Signore, io credo. E gli si prostrò dinanzi.
\par 39 E Gesù disse: Io son venuto in questo mondo per fare un giudizio, affinché quelli che non vedono vedano, e quelli che vedono diventino ciechi.
\par 40 E quelli de' Farisei che eran con lui udirono queste cose e gli dissero: Siamo ciechi anche noi?
\par 41 Gesù rispose loro: Se foste ciechi, non avreste alcun peccato; ma siccome dite: Noi vediamo, il vostro peccato rimane.

\chapter{10}

\par 1 In verità, in verità io vi dico che chi non entra per la porta nell'ovile delle pecore, ma vi sale da un'altra parte, esso è un ladro e un brigante.
\par 2 Ma colui che entra per la porta è pastore delle pecore.
\par 3 A lui apre il portinaio, e le pecore ascoltano la sua voce, ed egli chiama le proprie pecore per nome e le mena fuori.
\par 4 Quando ha messo fuori tutte le sue pecore, va innanzi a loro, e le pecore lo seguono, perché conoscono la sua voce.
\par 5 Ma un estraneo non lo seguiranno; anzi, fuggiranno via da lui perché non conoscono la voce degli estranei.
\par 6 Questa similitudine disse loro Gesù; ma essi non capirono di che cosa parlasse loro.
\par 7 Onde Gesù di nuovo disse loro: In verità, in verità vi dico: Io son la porta delle pecore.
\par 8 Tutti quelli che son venuti prima di me, sono stati ladri e briganti; ma le pecore non li hanno ascoltati.
\par 9 Io son la porta; se uno entra per me, sarà salvato, ed entrerà ed uscirà, e troverà pastura.
\par 10 Il ladro non viene se non per rubare e ammazzare e distruggere; io son venuto perché abbian la vita e l'abbiano ad esuberanza.
\par 11 Io sono il buon pastore; il buon pastore mette la sua vita per le pecore.
\par 12 Il mercenario, che non è pastore, a cui non appartengono le pecore, vede venire il lupo, abbandona le pecore e si dà alla fuga, e il lupo le rapisce e disperde.
\par 13 Il mercenario si dà alla fuga perché è mercenario e non si cura delle pecore.
\par 14 Io sono il buon pastore, e conosco le mie, e le mie mi conoscono,
\par 15 come il Padre mi conosce ed io conosco il Padre; e metto la mia vita per le pecore.
\par 16 Ho anche delle altre pecore, che non son di quest'ovile; anche quelle io devo raccogliere, ed esse ascolteranno la mia voce, e vi sarà un solo gregge, un solo pastore.
\par 17 Per questo mi ama il Padre; perché io depongo la mia vita, per ripigliarla poi.
\par 18 Nessuno me la toglie, ma la depongo da me. Io ho potestà di deporla e ho potestà di ripigliarla. Quest'ordine ho ricevuto dal Padre mio.
\par 19 Nacque di nuovo un dissenso fra i Giudei a motivo di queste parole.
\par 20 E molti di loro dicevano: Egli ha un demonio ed è fuor di sé; perché l'ascoltate?
\par 21 Altri dicevano: Queste non son parole di un indemoniato. Può un demonio aprir gli occhi a' ciechi?
\par 22 In quel tempo ebbe luogo in Gerusalemme la festa della Dedicazione. Era d'inverno,
\par 23 e Gesù passeggiava nel tempio, sotto il portico di Salomone.
\par 24 I Giudei dunque gli si fecero attorno e gli dissero: Fino a quando terrai sospeso l'animo nostro? Se tu sei il Cristo, diccelo apertamente.
\par 25 Gesù rispose loro: Ve l'ho detto, e non lo credete; le opere che fo nel nome del Padre mio, son quelle che testimoniano di me;
\par 26 ma voi non credete, perché non siete delle mie pecore.
\par 27 Le mie pecore ascoltano la mia voce, e io le conosco, ed esse mi seguono;
\par 28 e io do loro la vita eterna, e non periranno mai, e nessuno le rapirà dalla mia mano.
\par 29 Il Padre mio che me le ha date è più grande di tutti; e nessuno può rapirle di mano al Padre.
\par 30 Io ed il Padre siamo uno.
\par 31 I Giudei presero di nuovo delle pietre per lapidarlo.
\par 32 Gesù disse loro: Molte buone opere v'ho mostrate da parte del Padre mio; per quale di queste opere mi lapidate voi?
\par 33 I Giudei gli risposero: Non ti lapidiamo per una buona opera, ma per bestemmia; e perché tu, che sei uomo, ti fai Dio.
\par 34 Gesù rispose loro: Non è egli scritto nella vostra legge: Io ho detto: Voi siete dèi?
\par 35 Se chiama dèi coloro a' quali la parola di Dio è stata diretta (e la Scrittura non può essere annullata),
\par 36 come mai dite voi a colui che il Padre ha santificato e mandato nel mondo, che bestemmia, perché ho detto: Son Figliuolo di Dio?
\par 37 Se non faccio le opere del Padre mio, non mi credete;
\par 38 ma se le faccio, anche se non credete a me, credete alle opere, affinché sappiate e riconosciate che il Padre è in me e che io sono nel Padre.
\par 39 Essi cercavan di nuovo di pigliarlo; ma egli sfuggì loro dalle mani.
\par 40 E Gesù se ne andò di nuovo al di là del Giordano, nel luogo dove Giovanni da principio stava battezzando; e quivi dimorò.
\par 41 E molti vennero a lui, e dicevano: Giovanni, è vero, non fece alcun miracolo; ma tutto quello che Giovanni disse di quest'uomo, era vero.
\par 42 E quivi molti credettero in lui.

\chapter{11}

\par 1 Or v'era un ammalato, un certo Lazzaro di Betania, del villaggio di Maria e di Marta sua sorella.
\par 2 Maria era quella che unse il Signore d'olio odorifero e gli asciugò i piedi co' suoi capelli; e Lazzaro, suo fratello, era malato.
\par 3 Le sorelle dunque mandarono a dire a Gesù: Signore, ecco, colui che tu ami è malato.
\par 4 Gesù, udito ciò, disse: Questa malattia non è a morte, ma è per la gloria di Dio, affinché per mezzo d'essa il Figliuol di Dio sia glorificato.
\par 5 Or Gesù amava Marta e sua sorella e Lazzaro.
\par 6 Come dunque ebbe udito ch'egli era malato, si trattenne ancora due giorni nel luogo dov'era;
\par 7 poi dopo, disse a' discepoli: Torniamo in Giudea!
\par 8 I discepoli gli dissero: Maestro, i Giudei cercavano or ora di lapidarti, e tu vuoi tornar là?
\par 9 Gesù rispose: Non vi son dodici ore nel giorno? Se uno cammina di giorno, non inciampa, perché vede la luce di questo mondo;
\par 10 ma se uno cammina di notte, inciampa, perché la luce non è in lui.
\par 11 Così parlò; e poi disse loro: Il nostro amico Lazzaro s'è addormentato; ma io vado a svegliarlo.
\par 12 Perciò i discepoli gli dissero: Signore, s'egli dorme, sarà salvo.
\par 13 Or Gesù avea parlato della morte di lui; ma essi pensarono che avesse parlato del dormir del sonno.
\par 14 Allora Gesù disse loro apertamente: Lazzaro è morto;
\par 15 e per voi mi rallegro di non essere stato là, affinché crediate; ma ora, andiamo a lui!
\par 16 Allora Toma, detto Didimo, disse ai suoi condiscepoli: Andiamo anche noi, per morire con lui!
\par 17 Gesù dunque, arrivato, trovò che Lazzaro era già da quattro giorni nel sepolcro.
\par 18 Or Betania non distava da Gerusalemme che circa quindici stadî;
\par 19 e molti Giudei eran venuti da Marta e Maria per consolarle del loro fratello.
\par 20 Come dunque Marta ebbe udito che Gesù veniva, gli andò incontro; ma Maria stava seduta in casa.
\par 21 Marta dunque disse a Gesù: Signore, se tu fossi stato qui, mio fratello non sarebbe morto;
\par 22 e anche adesso so che tutto quel che chiederai a Dio, Dio te lo darà.
\par 23 Gesù le disse: Tuo fratello risusciterà.
\par 24 Marta gli disse: Lo so che risusciterà, nella risurrezione, nell'ultimo giorno.
\par 25 Gesù le disse: Io son la risurrezione e la vita; chi crede in me, anche se muoia, vivrà;
\par 26 e chiunque vive e crede in me, non morrà mai. Credi tu questo?
\par 27 Ella gli disse: Sì, o Signore, io credo che tu sei il Cristo, il Figliuol di Dio che dovea venire nel mondo.
\par 28 E detto questo, se ne andò, e chiamò di nascosto Maria, sua sorella, dicendole: il Maestro è qui, e ti chiama.
\par 29 Ed ella, udito questo, si alzò in fretta e venne a lui.
\par 30 Or Gesù non era ancora entrato nel villaggio, ma era sempre nel luogo dove Marta l'aveva incontrato.
\par 31 Quando dunque i Giudei ch'erano in casa con lei e la consolavano, videro che Maria s'era alzata in fretta ed era uscita, la seguirono, supponendo che si recasse al sepolcro a piangere.
\par 32 Appena Maria fu giunta dov'era Gesù e l'ebbe veduto, gli si gettò a' piedi dicendogli: Signore, se tu fossi stato qui, mio fratello non sarebbe morto.
\par 33 E quando Gesù la vide piangere, e vide i Giudei ch'eran venuti con lei piangere anch'essi, fremé nello spirito, si conturbò, e disse:
\par 34 Dove l'avete posto? Essi gli dissero: Signore, vieni a vedere!
\par 35 Gesù pianse.
\par 36 Onde i Giudei dicevano: Guarda come l'amava!
\par 37 Ma alcuni di loro dicevano: Non poteva, lui che ha aperto gli occhi al cieco, fare anche che questi non morisse?
\par 38 Gesù dunque, fremendo di nuovo in se stesso, venne al sepolcro. Era una grotta, e una pietra era posta all'apertura.
\par 39 Gesù disse: Togliete via la pietra! Marta, la sorella del morto, gli disse: Signore, egli puzza già, perché siamo al quarto giorno.
\par 40 Gesù le disse: Non t'ho io detto che se credi, tu vedrai la gloria di Dio?
\par 41 Tolsero dunque la pietra. E Gesù, alzati gli occhi in alto, disse: Padre, ti ringrazio che m'hai esaudito.
\par 42 Io ben sapevo che tu mi esaudisci sempre; ma ho detto questo a motivo della folla che mi circonda, affinché credano che tu m'hai mandato.
\par 43 E detto questo, gridò con gran voce: Lazzaro vieni fuori!
\par 44 E il morto uscì, avendo i piedi e le mani legati da fasce, e il viso coperto d'uno sciugatoio. Gesù disse loro: Scioglietelo, e lasciatelo andare.
\par 45 Perciò molti dei Giudei che eran venuti da Maria e avean veduto le cose fatte da Gesù, credettero in lui.
\par 46 Ma alcuni di loro andarono dai Farisei e raccontaron loro quel che Gesù avea fatto.
\par 47 I capi sacerdoti quindi e i Farisei radunarono il Sinedrio e dicevano: Che facciamo? perché quest'uomo fa molti miracoli.
\par 48 Se lo lasciamo fare, tutti crederanno in lui; e i Romani verranno e ci distruggeranno e città e nazione.
\par 49 E un di loro, Caiàfa, che era sommo sacerdote di quell'anno, disse loro: Voi non capite nulla;
\par 50 e non riflettete come vi torni conto che un uomo solo muoia per il popolo, e non perisca tutta la nazione.
\par 51 Or egli non disse questo di suo; ma siccome era sommo sacerdote di quell'anno, profetò che Gesù dovea morire per la nazione;
\par 52 e non soltanto per la nazione, ma anche per raccogliere in uno i figliuoli di Dio dispersi.
\par 53 Da quel giorno dunque deliberarono di farlo morire.
\par 54 Gesù quindi non andava più apertamente fra i Giudei, ma si ritirò di là nella contrada vicina al deserto, in una città detta Efraim; e quivi si trattenne co' suoi discepoli.
\par 55 Or la Pasqua de' Giudei era vicina; e molti di quella contrada salirono a Gerusalemme prima della Pasqua per purificarsi.
\par 56 Cercavan dunque Gesù; e stando nel tempio dicevano tra loro: Che ve ne pare? Che non abbia a venire alla festa?
\par 57 Or i capi sacerdoti e i Farisei avean dato ordine che se alcuno sapesse dove egli era, ne facesse denunzia perché potessero pigliarlo.

\chapter{12}

\par 1 Gesù dunque, sei giorni avanti la Pasqua, venne a Betania dov'era Lazzaro ch'egli avea risuscitato dai morti.
\par 2 E quivi gli fecero una cena; Marta serviva, e Lazzaro era uno di quelli ch'erano a tavola con lui.
\par 3 Allora Maria, presa una libbra d'olio odorifero di nardo schietto, di gran prezzo, unse i piedi di Gesù e glieli asciugò co' suoi capelli; e la casa fu ripiena del profumo dell'olio.
\par 4 Ma Giuda Iscariot, uno dei suoi discepoli, che stava per tradirlo, disse:
\par 5 Perché non s'è venduto quest'olio per trecento denari e non si son dati ai poveri?
\par 6 Diceva così, non perché si curasse de' poveri, ma perché era ladro, e tenendo la borsa, ne portava via quel che vi si metteva dentro.
\par 7 Gesù dunque disse: Lasciala stare; ella lo ha serbato per il giorno della mia sepoltura.
\par 8 Poiché i poveri li avete sempre con voi; ma me non avete sempre.
\par 9 La gran folla dei Giudei seppe dunque ch'egli era quivi; e vennero non solo a motivo di Gesù, ma anche per vedere Lazzaro che egli avea risuscitato dai morti.
\par 10 Ma i capi sacerdoti deliberarono di far morire anche Lazzaro,
\par 11 perché, per cagion sua, molti de' Giudei andavano e credevano in Gesù.
\par 12 Il giorno seguente, la gran folla che era venuta alla festa, udito che Gesù veniva a Gerusalemme,
\par 13 prese de' rami di palme, e uscì ad incontrarlo, e si mise a gridare: Osanna! Benedetto colui che viene nel nome del Signore, il Re d'Israele!
\par 14 E Gesù, trovato un asinello, vi montò su, secondo ch'è scritto:
\par 15 Non temere, o figliuola di Sion! Ecco, il tuo Re viene, montato sopra un puledro d'asina!
\par 16 Or i suoi discepoli non intesero da prima queste cose; ma quando Gesù fu glorificato, allora si ricordarono che queste cose erano state scritte di lui, e che essi gliele aveano fatte.
\par 17 La folla dunque che era con lui quando avea chiamato Lazzaro fuor dal sepolcro e l'avea risuscitato dai morti, ne rendea testimonianza.
\par 18 E per questo la folla gli andò incontro, perché aveano udito ch'egli avea fatto quel miracolo.
\par 19 Onde i Farisei dicevano fra loro: Vedete che non guadagnate nulla? Ecco, il mondo gli corre dietro!
\par 20 Or fra quelli che salivano alla festa per adorare, v'erano certi Greci.
\par 21 Questi dunque, accostatisi a Filippo, che era di Betsaida di Galilea, gli fecero questa richiesta: Signore, vorremmo veder Gesù.
\par 22 Filippo lo venne a dire ad Andrea; e Andrea e Filippo vennero a dirlo a Gesù.
\par 23 E Gesù rispose loro dicendo: L'ora è venuta, che il Figliuol dell'uomo ha da esser glorificato.
\par 24 In verità, in verità io vi dico che se il granello di frumento caduto in terra non muore, riman solo; ma se muore, produce molto frutto.
\par 25 Chi ama la sua vita, la perde; e chi odia la sua vita in questo mondo, la conserverà in vita eterna.
\par 26 Se uno mi serve, mi segua; e là dove son io, quivi sarà anche il mio servitore; se uno mi serve, il Padre l'onorerà.
\par 27 Ora è turbata l'anima mia; e che dirò? Padre, salvami da quest'ora! Ma è per questo che son venuto incontro a quest'ora.
\par 28 Padre, glorifica il tuo nome! Allora venne una voce dal cielo: E l'ho glorificato, e lo glorificherò di nuovo!
\par 29 Onde la moltitudine ch'era quivi presente e aveva udito, diceva ch'era stato un tuono. Altri dicevano: Un angelo gli ha parlato.
\par 30 Gesù rispose e disse: Questa voce non s'è fatta per me, ma per voi.
\par 31 Ora avviene il giudizio di questo mondo; ora sarà cacciato fuori il principe di questo mondo;
\par 32 e io, quando sarò innalzato dalla terra, trarrò tutti a me.
\par 33 Così diceva per significare di qual morte dovea morire.
\par 34 La moltitudine quindi gli rispose: Noi abbiamo udito dalla legge che il Cristo dimora in eterno: come dunque dici tu che bisogna che il Figliuol dell'uomo sia innalzato? Chi è questo Figliuol dell'uomo?
\par 35 Gesù dunque disse loro: Ancora per poco la luce è fra voi. Camminate mentre avete la luce, affinché non vi colgano le tenebre; chi cammina nelle tenebre non sa dove vada.
\par 36 Mentre avete la luce, credete nella luce, affinché diventiate figliuoli di luce. Queste cose disse Gesù, poi se ne andò e si nascose da loro.
\par 37 E sebbene avesse fatto tanti miracoli in loro presenza, pure non credevano in lui;
\par 38 affinché s'adempisse la parola detta dal profeta Isaia: Signore, chi ha creduto a quel che ci è stato predicato? E a chi è stato rivelato il braccio del Signore?
\par 39 Perciò non potevan credere, per la ragione detta ancora da Isaia:
\par 40 Egli ha accecato gli occhi loro e ha indurato i loro cuori, affinché non veggano con gli occhi, e non intendano col cuore, e non si convertano, e io non li sani.
\par 41 Queste cose disse Isaia, perché vide la gloria di lui e di lui parlò.
\par 42 Pur nondimeno molti, anche fra i capi, credettero in lui; ma a cagione dei Farisei non lo confessavano, per non essere espulsi dalla sinagoga;
\par 43 perché amarono la gloria degli uomini più della gloria di Dio.
\par 44 Ma Gesù ad alta voce avea detto: Chi crede in me, crede non in me, ma in Colui che mi ha mandato;
\par 45 e chi vede me, vede Colui che mi ha mandato.
\par 46 Io son venuto come luce nel mondo, affinché chiunque crede in me, non rimanga nelle tenebre.
\par 47 E se uno ode le mie parole e non le osserva, io non lo giudico; perché io non son venuto a giudicare il mondo, ma a salvare il mondo.
\par 48 Chi mi respinge e non accetta le mie parole, ha chi lo giudica: la parola che ho annunziata è quella che lo giudicherà nell'ultimo giorno.
\par 49 Perché io non ho parlato di mio; ma il Padre che m'ha mandato, m'ha comandato lui quel che debbo dire e di che debbo ragionare;
\par 50 ed io so che il suo comandamento è vita eterna. Le cose dunque che dico, così le dico, come il Padre me le ha dette.

\chapter{13}

\par 1 Or avanti la festa di Pasqua, Gesù, sapendo che era venuta per lui l'ora di passare da questo mondo al Padre, avendo amato i suoi che erano nel mondo, li amò sino alla fine.
\par 2 E durante la cena, quando il diavolo avea già messo in cuore a Giuda Iscariot, figliuol di Simone, di tradirlo,
\par 3 Gesù, sapendo che il Padre gli avea dato tutto nelle mani e che era venuto da Dio e a Dio se ne tornava,
\par 4 si levò da tavola, depose le sue vesti, e preso un asciugatoio, se ne cinse.
\par 5 Poi mise dell'acqua nel bacino, e cominciò a lavare i piedi a' discepoli, e ad asciugarli con l'asciugatoio del quale era cinto.
\par 6 Venne dunque a Simon Pietro, il quale gli disse: Tu, Signore, lavare i piedi a me?
\par 7 Gesù gli rispose: Tu non sai ora quello che io fo, ma lo capirai dopo.
\par 8 Pietro gli disse: Tu non mi laverai mai i piedi! Gesù gli rispose: Se non ti lavo, non hai meco parte alcuna.
\par 9 E Simon Pietro: Signore, non soltanto i piedi, ma anche le mani e il capo!
\par 10 Gesù gli disse: Chi è lavato tutto non ha bisogno che d'aver lavato i piedi; è netto tutto quanto; e voi siete netti, ma non tutti.
\par 11 Perché sapeva chi era colui che lo tradirebbe; per questo disse: Non tutti siete netti.
\par 12 Come dunque ebbe loro lavato i piedi ed ebbe ripreso le sue vesti, si mise di nuovo a tavola, e disse loro: Capite quel che v'ho fatto?
\par 13 Voi mi chiamate Maestro e Signore; e dite bene, perché lo sono.
\par 14 Se dunque io, che sono il Signore e il Maestro, v'ho lavato i piedi, anche voi dovete lavare i piedi gli uni agli altri.
\par 15 Poiché io v'ho dato un esempio, affinché anche voi facciate come v'ho fatto io.
\par 16 In verità, in verità vi dico che il servitore non è maggiore del suo signore, né il messo è maggiore di colui che l'ha mandato.
\par 17 Se sapete queste cose, siete beati se le fate.
\par 18 Io non parlo di voi tutti; io so quelli che ho scelti; ma, perché sia adempita la Scrittura, colui che mangia il mio pane, ha levato contro di me il suo calcagno.
\par 19 Fin da ora ve lo dico, prima che accada; affinché, quando sia accaduto, voi crediate che sono io (il Cristo).
\par 20 In verità, in verità vi dico: Chi riceve colui che io avrò mandato, riceve me; e chi riceve me, riceve Colui che mi ha mandato.
\par 21 Dette queste cose, Gesù fu turbato nello spirito, e così apertamente si espresse: In verità, in verità vi dico che uno di voi mi tradirà.
\par 22 I discepoli si guardavano l'un l'altro, stando in dubbio di chi parlasse.
\par 23 Or, a tavola, inclinato sul seno di Gesù, stava uno de' discepoli, quello che Gesù amava.
\par 24 Simon Pietro quindi gli fe' cenno e gli disse: Di', chi è quello del quale parla?
\par 25 Ed egli, chinatosi così sul petto di Gesù, gli domandò: Signore, chi è? Gesù rispose:
\par 26 È quello al quale darò il boccone dopo averlo intinto. E intinto un boccone, lo prese e lo diede a Giuda figlio di Simone Iscariota.
\par 27 E allora, dopo il boccone, Satana entrò in lui. Per cui Gesù gli disse: Quel che fai, fallo presto.
\par 28 Ma nessuno de' commensali intese perché gli avesse detto così.
\par 29 Difatti alcuni pensavano, siccome Giuda tenea la borsa, che Gesù gli avesse detto: Compra quel che ci abbisogna per la festa; ovvero che desse qualcosa ai poveri.
\par 30 Egli dunque, preso il boccone, uscì subito; ed era notte.
\par 31 Quand'egli fu uscito, Gesù disse: Ora il Figliuol dell'uomo è glorificato, e Dio è glorificato in lui.
\par 32 Se Dio è glorificato in lui, Dio lo glorificherà anche in se stesso, e presto lo glorificherà.
\par 33 Figliuoletti, è per poco che sono ancora con voi. Voi mi cercherete; e, come ho detto ai Giudei: 'Dove vo io, voi non potete venire', così lo dico ora a voi.
\par 34 Io vi do un nuovo comandamento: che vi amiate gli uni gli altri. Com'io v'ho amati, anche voi amatevi gli uni gli altri.
\par 35 Da questo conosceranno tutti che siete miei discepoli, se avete amore gli uni per gli altri.
\par 36 Simon Pietro gli domandò: Signore, dove vai? Gesù rispose: Dove io vado, non puoi per ora seguirmi; ma mi seguirai più tardi.
\par 37 Pietro gli disse: Signore, perché non posso seguirti ora? Metterò la mia vita per te!
\par 38 Gesù gli rispose: Metterai la tua vita per me? In verità, in verità ti dico che il gallo non canterà che già tu non m'abbia rinnegato tre volte.

\chapter{14}

\par 1 Il vostro cuore non sia turbato; abbiate fede in Dio, e abbiate fede anche in me!
\par 2 Nella casa del Padre mio ci son molte dimore; se no, ve l'avrei detto; io vo a prepararvi un luogo;
\par 3 e quando sarò andato e v'avrò preparato un luogo, tornerò, e v'accoglierò presso di me, affinché dove son io, siate anche voi;
\par 4 e del dove io vo sapete anche la via.
\par 5 Toma gli disse: Signore, non sappiamo dove vai; come possiamo saper la via?
\par 6 Gesù gli disse: Io son la via, la verità e la vita; nessuno viene al Padre se non per mezzo di me.
\par 7 Se m'aveste conosciuto, avreste conosciuto anche mio Padre; e fin da ora lo conoscete, e l'avete veduto.
\par 8 Filippo gli disse: Signore, mostraci il Padre, e ci basta.
\par 9 Gesù gli disse: Da tanto tempo sono con voi e tu non m'hai conosciuto, Filippo? Chi ha veduto me, ha veduto il Padre; come mai dici tu: Mostraci il Padre?
\par 10 Non credi tu ch'io sono nel Padre e che il Padre è in me? Le parole che io vi dico, non le dico di mio; ma il Padre che dimora in me, fa le opere sue.
\par 11 Credetemi che io sono nel Padre e che il Padre è in me; se no, credete a cagion di quelle opere stesse.
\par 12 In verità, in verità vi dico che chi crede in me farà anch'egli le opere che fo io; e ne farà di maggiori, perché io me ne vo al Padre;
\par 13 e quel che chiederete nel mio nome, lo farò; affinché il Padre sia glorificato nel Figliuolo.
\par 14 Se chiederete qualche cosa nel mio nome, io la farò.
\par 15 Se voi mi amate, osserverete i miei comandamenti.
\par 16 E io pregherò il Padre, ed Egli vi darà un altro Consolatore, perché stia con voi in perpetuo,
\par 17 lo Spirito della verità, che il mondo non può ricevere, perché non lo vede e non lo conosce. Voi lo conoscete, perché dimora con voi, e sarà in voi.
\par 18 Non vi lascerò orfani; tornerò a voi.
\par 19 Ancora un po', e il mondo non mi vedrà più; ma voi mi vedrete, perché io vivo e voi vivrete.
\par 20 In quel giorno conoscerete che io sono nel Padre mio, e voi in me ed io in voi.
\par 21 Chi ha i miei comandamenti e li osserva, quello mi ama; e chi mi ama sarà amato dal Padre mio, e io l'amerò e mi manifesterò a lui.
\par 22 Giuda (non l'Iscariota) gli domandò: Signore, come mai ti manifesterai a noi e non al mondo?
\par 23 Gesù rispose e gli disse: Se uno mi ama, osserverà la mia parola; e il Padre mio l'amerà, e noi verremo a lui e faremo dimora presso di lui.
\par 24 Chi non mi ama non osserva le mie parole; e la parola che voi udite non è mia, ma è del Padre che mi ha mandato.
\par 25 Queste cose v'ho detto, stando ancora con voi;
\par 26 ma il Consolatore, lo Spirito Santo che il Padre manderà nel mio nome, egli v'insegnerà ogni cosa e vi rammenterà tutto quello che v'ho detto.
\par 27 Io vi lascio pace; vi do la mia pace. Io non vi do come il mondo dà. Il vostro cuore non sia turbato e non si sgomenti.
\par 28 Avete udito che v'ho detto: 'Io me ne vo, e torno a voi'; se voi m'amaste, vi rallegrereste ch'io vo al Padre, perché il Padre è maggiore di me.
\par 29 E ora ve l'ho detto prima che avvenga, affinché, quando sarà avvenuto, crediate.
\par 30 Io non parlerò più molto con voi, perché viene il principe di questo mondo. Ed esso non ha nulla in me;
\par 31 ma così avviene affinché il mondo conosca che amo il Padre, e opero come il Padre m'ha ordinato. Levatevi, andiamo via di qui.

\chapter{15}

\par 1 Io sono la vera vite, e il Padre mio è il vignaiuolo. Ogni tralcio che in me non dà frutto,
\par 2 Egli lo toglie via; e ogni tralcio che dà frutto, lo rimonda affinché ne dia di più.
\par 3 Voi siete già mondi a motivo della parola che v'ho annunziata.
\par 4 Dimorate in me, e io dimorerò in voi. Come il tralcio non può da sé dar frutto se non rimane nella vite, così neppur voi, se non dimorate in me.
\par 5 Io son la vite, voi siete i tralci. Colui che dimora in me e nel quale io dimoro, porta molto frutto; perché senza di me non potete far nulla.
\par 6 Se uno non dimora in me, è gettato via come il tralcio, e si secca; cotesti tralci si raccolgono, si gettano nel fuoco e si bruciano.
\par 7 Se dimorate in me e le mie parole dimorano in voi, domandate quel che volete e vi sarà fatto.
\par 8 In questo è glorificato il Padre mio: che portiate molto frutto, e così sarete miei discepoli.
\par 9 Come il Padre mi ha amato, così anch'io ho amato voi; dimorate nel mio amore.
\par 10 Se osservate i miei comandamenti, dimorerete nel mio amore; com'io ho osservato i comandamenti del Padre mio, e dimoro nel suo amore.
\par 11 Queste cose vi ho detto, affinché la mia allegrezza dimori in voi, e la vostra allegrezza sia resa completa.
\par 12 Questo è il mio comandamento: che vi amiate gli uni gli altri, come io ho amato voi.
\par 13 Nessuno ha amore più grande che quello di dar la sua vita per i suoi amici.
\par 14 Voi siete miei amici, se fate le cose che io vi comando.
\par 15 Io non vi chiamo più servi; perché il servo non sa quel che fa il suo signore; ma voi vi ho chiamati amici, perché vi ho fatto conoscere tutte le cose che ho udite dal Padre mio.
\par 16 Non siete voi che avete scelto me, ma son io che ho scelto voi, e v'ho costituiti perché andiate, e portiate frutto, e il vostro frutto sia permanente; affinché tutto quel che chiederete al Padre nel mio nome, Egli ve lo dia.
\par 17 Questo vi comando: che vi amiate gli uni gli altri.
\par 18 Se il mondo vi odia, sapete bene che prima di voi ha odiato me.
\par 19 Se foste del mondo, il mondo amerebbe quel ch'è suo; ma perché non siete del mondo, ma io v'ho scelti di mezzo al mondo, perciò vi odia il mondo.
\par 20 Ricordatevi della parola che v'ho detta: Il servitore non è da più del suo signore. Se hanno perseguitato me, perseguiteranno anche voi; se hanno osservato la mia parola, osserveranno anche la vostra.
\par 21 Ma tutto questo ve lo faranno a cagion del mio nome, perché non conoscono Colui che m'ha mandato.
\par 22 S'io non fossi venuto e non avessi loro parlato, non avrebbero colpa; ma ora non hanno scusa del loro peccato.
\par 23 Chi odia me, odia anche il Padre mio.
\par 24 Se non avessi fatto tra loro le opere che nessun altro ha fatte mai, non avrebbero colpa; ma ora le hanno vedute, ed hanno odiato e me e il Padre mio.
\par 25 Ma quest'è avvenuto affinché sia adempita la parola scritta nella loro legge: Mi hanno odiato senza cagione.
\par 26 Ma quando sarà venuto il Consolatore che io vi manderò da parte del Padre, lo Spirito della verità che procede dal Padre, egli testimonierà di me;
\par 27 e anche voi mi renderete testimonianza, perché siete stati meco fin da principio.

\chapter{16}

\par 1 Io vi ho dette queste cose, affinché non siate scandalizzati.
\par 2 Vi espelleranno dalle sinagoghe; anzi, l'ora viene che chiunque v'ucciderà, crederà di offrir servigio a Dio.
\par 3 E questo faranno, perché non hanno conosciuto né il Padre né me.
\par 4 Ma io v'ho dette queste cose, affinché quando sia giunta l'ora in cui avverranno, vi ricordiate che ve l'ho dette. Non ve le dissi da principio, perché ero con voi.
\par 5 Ma ora me ne vo a Colui che mi ha mandato; e niun di voi mi domanda: Dove vai?
\par 6 Invece, perché v'ho detto queste cose, la tristezza v'ha riempito il cuore.
\par 7 Pure, io vi dico la verità, egli v'è utile ch'io me ne vada; perché, se non me ne vo, non verrà a voi il Consolatore; ma se me ne vo, io ve lo manderò.
\par 8 E quando sarà venuto, convincerà il mondo quanto al peccato, alla giustizia, e al giudizio.
\par 9 Quanto al peccato, perché non credono in me;
\par 10 quanto alla giustizia, perché me ne vo al Padre e non mi vedrete più;
\par 11 quanto al giudizio, perché il principe di questo mondo è stato giudicato.
\par 12 Molte cose ho ancora da dirvi; ma non sono per ora alla vostra portata;
\par 13 ma quando sia venuto lui, lo Spirito della verità, egli vi guiderà in tutta la verità, perché non parlerà di suo, ma dirà tutto quello che avrà udito, e vi annunzierà le cose a venire.
\par 14 Egli mi glorificherà perché prenderà del mio e ve l'annunzierà.
\par 15 Tutte le cose che ha il Padre, son mie: per questo ho detto che prenderà del mio e ve l'annunzierà.
\par 16 Fra poco non mi vedrete più; e fra un altro poco mi vedrete, perché me ne vo al Padre.
\par 17 Allora alcuni dei suoi discepoli dissero tra loro: Che cos'è questo che ci dice: 'Fra poco non mi vedrete più'; e 'Fra un altro poco mi vedrete'; e: 'Perché me ne vo al Padre?'
\par 18 Dicevano dunque: Che cos'è questo 'fra poco' che egli dice? Noi non sappiamo quello ch'egli voglia dire.
\par 19 Gesù conobbe che lo volevano interrogare, e disse loro: Vi domandate voi l'un l'altro che significhi quel mio dire 'Fra poco non mi vedrete più', e 'fra un altro poco mi vedrete?'
\par 20 In verità, in verità vi dico che voi piangerete e farete cordoglio, e il mondo si rallegrerà. Voi sarete contristati, ma la vostra tristezza sarà mutata in letizia.
\par 21 La donna, quando partorisce, è in dolore, perché è venuta la sua ora; ma quando ha dato alla luce il bambino, non si ricorda più dell'angoscia, per l'allegrezza che sia nata al mondo una creatura umana.
\par 22 E così anche voi siete ora nel dolore; ma io vi vedrò di nuovo, e il vostro cuore si rallegrerà e nessuno vi torrà la vostra allegrezza.
\par 23 E in quel giorno non rivolgerete a me alcuna domanda. In verità, in verità vi dico che quel che chiederete al Padre, Egli ve lo darà nel nome mio.
\par 24 Fino ad ora non avete chiesto nulla nel nome mio; chiedete e riceverete, affinché la vostra allegrezza sia completa.
\par 25 Queste cose v'ho dette in similitudini; l'ora viene che non vi parlerò più in similitudini, ma apertamente vi farò conoscere il Padre.
\par 26 In quel giorno chiederete nel mio nome; e non vi dico che io pregherò il Padre per voi;
\par 27 poiché il Padre stesso vi ama, perché mi avete amato e avete creduto che son proceduto da Dio.
\par 28 Son proceduto dal Padre e son venuto nel mondo; ora lascio il mondo, e torno al Padre.
\par 29 I suoi discepoli gli dissero: Ecco, adesso tu parli apertamente e non usi similitudine.
\par 30 Ora sappiamo che sai ogni cosa, e non hai bisogno che alcuno t'interroghi; perciò crediamo che sei proceduto da Dio.
\par 31 Gesù rispose loro: Adesso credete?
\par 32 Ecco, l'ora viene, anzi è venuta, che sarete dispersi, ciascun dal canto suo, e mi lascerete solo; ma io non son solo, perché il Padre è meco.
\par 33 V'ho dette queste cose, affinché abbiate pace in me. Nel mondo avrete tribolazione; ma fatevi animo, io ho vinto il mondo.

\chapter{17}

\par 1 Queste cose disse Gesù; poi levati gli occhi al cielo, disse: Padre, l'ora è venuta; glorifica il tuo Figliuolo, affinché il Figliuolo glorifichi te,
\par 2 poiché gli hai data potestà sopra ogni carne, onde egli dia vita eterna a tutti quelli che tu gli hai dato.
\par 3 E questa è la vita eterna: che conoscano te, il solo vero Dio, e colui che tu hai mandato, Gesù Cristo.
\par 4 Io ti ho glorificato sulla terra, avendo compiuto l'opera che tu m'hai data a fare.
\par 5 Ed ora, o Padre, glorificami tu presso te stesso della gloria che avevo presso di te avanti che il mondo fosse.
\par 6 Io ho manifestato il tuo nome agli uomini che tu m'hai dati dal mondo; erano tuoi, e tu me li hai dati; ed essi hanno osservato la tua parola.
\par 7 Ora hanno conosciuto che tutte le cose che tu m'hai date, vengon da te;
\par 8 poiché le parole che tu mi hai date, le ho date a loro; ed essi le hanno ricevute, e hanno veramente conosciuto ch'io son proceduto da te, e hanno creduto che tu m'hai mandato.
\par 9 Io prego per loro; non prego per il mondo, ma per quelli che tu m'hai dato, perché son tuoi;
\par 10 e tutte le cose mie son tue, e le cose tue son mie; ed io son glorificato in loro.
\par 11 E io non sono più nel mondo, ma essi sono nel mondo, e io vengo a te. Padre santo, conservali nel tuo nome, essi che tu m'hai dati, affinché siano uno, come noi.
\par 12 Mentre io ero con loro, io li conservavo nel tuo nome; quelli che tu mi hai dati, li ho anche custoditi, e niuno di loro è perito, tranne il figliuol di perdizione, affinché la Scrittura fosse adempiuta.
\par 13 Ma ora io vengo a te; e dico queste cose nel mondo, affinché abbiano compita in se stessi la mia allegrezza.
\par 14 Io ho dato loro la tua parola; e il mondo li ha odiati, perché non sono del mondo, come io non sono del mondo.
\par 15 Io non ti prego che tu li tolga dal mondo, ma che tu li preservi dal maligno.
\par 16 Essi non sono del mondo, come io non sono del mondo.
\par 17 Santificali nella verità: la tua parola è verità.
\par 18 Come tu hai mandato me nel mondo, anch'io ho mandato loro nel mondo.
\par 19 E per loro io santifico me stesso, affinché anch'essi siano santificati in verità.
\par 20 Io non prego soltanto per questi, ma anche per quelli che credono in me per mezzo della loro parola:
\par 21 che siano tutti uno; che come tu, o Padre, sei in me, ed io sono in te, anch'essi siano in noi: affinché il mondo creda che tu mi hai mandato.
\par 22 E io ho dato loro la gloria che tu hai dato a me, affinché siano uno come noi siamo uno;
\par 23 io in loro, e tu in me; acciocché siano perfetti nell'unità, e affinché il mondo conosca che tu m'hai mandato, e che li ami come hai amato me.
\par 24 Padre, io voglio che dove son io, siano meco anche quelli che tu m'hai dati, affinché veggano la mia gloria che tu m'hai data; poiché tu m'hai amato avanti la fondazion del mondo.
\par 25 Padre giusto, il mondo non t'ha conosciuto, ma io t'ho conosciuto; e questi hanno conosciuto che tu mi hai mandato;
\par 26 ed io ho fatto loro conoscere il tuo nome, e lo farò conoscere, affinché l'amore del quale tu m'hai amato sia in loro, e io in loro.

\chapter{18}

\par 1 Dette queste cose, Gesù uscì coi suoi discepoli di là dal torrente Chedron, dov'era un orto, nel quale egli entrò co' suoi discepoli.
\par 2 Or Giuda, che lo tradiva, conosceva anch'egli quel luogo, perché Gesù s'era molte volte ritrovato là coi suoi discepoli.
\par 3 Giuda dunque, presa la coorte e delle guardie mandate dai capi sacerdoti e dai Farisei, venne là con lanterne e torce ed armi.
\par 4 Onde Gesù, ben sapendo tutto quel che stava per accadergli, uscì e chiese loro: Chi cercate?
\par 5 Gli risposero: Gesù il Nazareno! Gesù disse loro: Son io. E Giuda, che lo tradiva, era anch'egli là con loro.
\par 6 Come dunque ebbe detto loro: 'Son io', indietreggiarono e caddero in terra.
\par 7 Egli dunque domandò loro di nuovo: Chi cercate? Ed essi dissero: Gesù il Nazareno.
\par 8 Gesù rispose: V'ho detto che son io; se dunque cercate me, lasciate andar questi.
\par 9 E ciò affinché s'adempisse la parola ch'egli avea detta: Di quelli che tu m'hai dato, non ne ho perduto alcuno.
\par 10 Allora Simon Pietro, che avea una spada, la trasse, e percosse il servo del sommo sacerdote, e gli recise l'orecchio destro. Quel servo avea nome Malco.
\par 11 Per il che Gesù disse a Pietro: Rimetti la spada nel fodero; non berrò io il calice che il Padre mi ha dato?
\par 12 La coorte dunque e il tribuno e le guardie de' Giudei, presero Gesù e lo legarono,
\par 13 e lo menaron prima da Anna, perché era suocero di Caiàfa, il quale era sommo sacerdote di quell'anno.
\par 14 Or Caiàfa era quello che avea consigliato a' Giudei esser cosa utile che un uomo solo morisse per il popolo.
\par 15 Or Simon Pietro e un altro discepolo seguivano Gesù; e quel discepolo era noto al sommo sacerdote, ed entrò con Gesù nella corte del sommo sacerdote;
\par 16 ma Pietro stava di fuori, alla porta. Allora quell'altro discepolo che era noto al sommo sacerdote, uscì, parlò con la portinaia e fece entrar Pietro.
\par 17 La serva portinaia dunque disse a Pietro: Non sei anche tu de' discepoli di quest'uomo? Egli disse: Non lo sono.
\par 18 Or i servi e le guardie avevano acceso un fuoco, perché faceva freddo, e stavan lì a scaldarsi; e anche Pietro stava con loro e si scaldava.
\par 19 Il sommo sacerdote dunque interrogò Gesù intorno ai suoi discepoli e alla sua dottrina.
\par 20 Gesù gli rispose: Io ho parlato apertamente al mondo; ho sempre insegnato nelle sinagoghe e nel tempio, dove tutti i Giudei si radunano; e non ho detto nulla in segreto. Perché m'interroghi?
\par 21 Domanda a quelli che m'hanno udito, quel che ho detto loro; ecco, essi sanno le cose che ho detto.
\par 22 E com'ebbe detto questo, una delle guardie che gli stava vicino, dette uno schiaffo a Gesù, dicendo: Così rispondi tu al sommo sacerdote?
\par 23 Gesù gli disse: Se ho parlato male, dimostra il male che ho detto; ma se ho parlato bene, perché mi percuoti?
\par 24 Quindi Anna lo mandò legato a Caiàfa, sommo sacerdote.
\par 25 Or Simon Pietro stava quivi a scaldarsi; e gli dissero: Non sei anche tu dei suoi discepoli? Egli lo negò e disse: Non lo sono.
\par 26 Uno de' servi del sommo sacerdote, parente di quello a cui Pietro avea tagliato l'orecchio, disse: Non t'ho io visto nell'orto con lui?
\par 27 E Pietro da capo lo negò, e subito il gallo cantò.
\par 28 Poi, da Caiàfa, menarono Gesù nel pretorio. Era mattina, ed essi non entrarono nel pretorio per non contaminarsi e così poter mangiare la pasqua.
\par 29 Pilato dunque uscì fuori verso di loro, e domandò: Quale accusa portate contro quest'uomo?
\par 30 Essi risposero e gli dissero: Se costui non fosse un malfattore, non te lo avremmo dato nelle mani.
\par 31 Pilato quindi disse loro: Pigliatelo voi, e giudicatelo secondo la vostra legge. I Giudei gli dissero: A noi non è lecito far morire alcuno.
\par 32 E ciò affinché si adempisse la parola che Gesù aveva detta, significando di qual morte dovea morire.
\par 33 Pilato dunque rientrò nel pretorio; chiamò Gesù e gli disse: Sei tu il Re dei Giudei?
\par 34 Gesù gli rispose: Dici tu questo di tuo, oppure altri te l'hanno detto di me?
\par 35 Pilato gli rispose: Son io forse giudeo? La tua nazione e i capi sacerdoti t'hanno messo nelle mie mani; che hai fatto?
\par 36 Gesù rispose: il mio regno non è di questo mondo; se il mio regno fosse di questo mondo, i miei servitori combatterebbero perch'io non fossi dato in man de' Giudei; ma ora il mio regno non è di qui.
\par 37 Allora Pilato gli disse: Ma dunque, sei tu re? Gesù rispose: Tu lo dici; io sono re; io son nato per questo, e per questo son venuto nel mondo, per testimoniare della verità. Chiunque è per la verità ascolta la mia voce.
\par 38 Pilato gli disse: Che cos'è verità? E detto questo, uscì di nuovo verso i Giudei, e disse loro: Io non trovo alcuna colpa in lui.
\par 39 Ma voi avete l'usanza ch'io vi liberi uno per la Pasqua; volete dunque che vi liberi il Re de' Giudei?
\par 40 Allora gridaron di nuovo: Non costui, ma Barabba! Or Barabba era un ladrone.

\chapter{19}

\par 1 Allora dunque Pilato prese Gesù e lo fece flagellare.
\par 2 E i soldati, intrecciata una corona di spine, gliela posero sul capo, e gli misero addosso un manto di porpora; e s'accostavano a lui e dicevano:
\par 3 Salve, Re de' Giudei! e gli davan degli schiaffi.
\par 4 Pilato uscì di nuovo, e disse loro: Ecco, ve lo meno fuori, affinché sappiate che non trovo in lui alcuna colpa.
\par 5 Gesù dunque uscì, portando la corona di spine e il manto di porpora. E Pilato disse loro: Ecco l'uomo!
\par 6 Come dunque i capi sacerdoti e le guardie l'ebbero veduto, gridarono: Crocifiggilo, crocifiggilo! Pilato disse loro: Prendetelo voi e crocifiggetelo; perché io non trovo in lui alcuna colpa.
\par 7 I Giudei gli risposero: Noi abbiamo una legge, e secondo questa legge egli deve morire, perché egli s'è fatto Figliuol di Dio.
\par 8 Quando Pilato ebbe udita questa parola, temette maggiormente;
\par 9 e rientrato nel pretorio, disse a Gesù: Donde sei tu? Ma Gesù non gli diede alcuna risposta.
\par 10 Allora Pilato gli disse: Non mi parli? Non sai che ho potestà di liberarti e potestà di crocifiggerti?
\par 11 Gesù gli rispose: Tu non avresti potestà alcuna contro di me, se ciò non ti fosse stato dato da alto; perciò chi m'ha dato nelle tue mani, ha maggior colpa.
\par 12 Da quel momento Pilato cercava di liberarlo; ma i Giudei gridavano, dicendo: Se liberi costui, non sei amico di Cesare. Chiunque si fa re, si oppone a Cesare.
\par 13 Pilato dunque, udite queste parole, menò fuori Gesù, e si assise al tribunale nel luogo detto Lastrico, e in ebraico Gabbatà.
\par 14 Era la preparazione della Pasqua, ed era circa l'ora sesta. Ed egli disse ai Giudei: Ecco il vostro Re!
\par 15 Allora essi gridarono: Tòglilo, tòglilo di mezzo, crocifiggilo! Pilato disse loro: Crocifiggerò io il vostro Re? I capi sacerdoti risposero: Noi non abbiamo altro re che Cesare.
\par 16 Allora lo consegnò loro perché fosse crocifisso.
\par 17 Presero dunque Gesù; ed egli, portando la sua croce, venne al luogo del Teschio, che in ebraico si chiama Golgota,
\par 18 dove lo crocifissero, assieme a due altri, uno di qua, l'altro di là, e Gesù nel mezzo.
\par 19 E Pilato fece pure un'iscrizione, e la pose sulla croce. E v'era scritto: GESÙ IL NAZARENO, IL RE DE' GIUDEI.
\par 20 Molti dunque dei Giudei lessero questa iscrizione, perché il luogo dove Gesù fu crocifisso era vicino alla città; e l'iscrizione era in ebraico, in latino e in greco.
\par 21 Perciò i capi sacerdoti dei Giudei dicevano a Pilato: Non scrivere: Il Re dei Giudei; ma che egli ha detto: Io sono il Re de' Giudei.
\par 22 Pilato rispose: Quel che ho scritto, ho scritto.
\par 23 I soldati dunque, quando ebbero crocifisso Gesù, presero le sue vesti, e ne fecero quattro parti, una parte per ciascun soldato e la tunica. Or la tunica era senza cuciture, tessuta per intero dall'alto in basso.
\par 24 Dissero dunque tra loro: Non la stracciamo, ma tiriamo a sorte a chi tocchi; affinché si adempisse la Scrittura che dice: Hanno spartito fra loro le mie vesti, e han tirato la sorte sulla mia tunica. Questo dunque fecero i soldati.
\par 25 Or presso la croce di Gesù stavano sua madre e la sorella di sua madre, Maria moglie di Cleopa, e Maria Maddalena.
\par 26 Gesù dunque, vedendo sua madre e presso a lei il discepolo ch'egli amava, disse a sua madre: Donna, ecco il tuo figlio!
\par 27 Poi disse al discepolo: Ecco tua madre! E da quel momento, il discepolo la prese in casa sua.
\par 28 Dopo questo, Gesù, sapendo che ogni cosa era già compiuta, affinché la Scrittura fosse adempiuta, disse: Ho sete.
\par 29 V'era quivi un vaso pieno d'aceto; i soldati dunque, posta in cima a un ramo d'issopo una spugna piena d'aceto, gliel'accostarono alla bocca.
\par 30 E quando Gesù ebbe preso l'aceto, disse: È compiuto! E chinato il capo, rese lo spirito.
\par 31 Allora i Giudei, perché i corpi non rimanessero sulla croce durante il sabato (poiché era la Preparazione, e quel giorno del sabato era un gran giorno), chiesero a Pilato che fossero loro fiaccate le gambe, e fossero tolti via.
\par 32 I soldati dunque vennero e fiaccarono le gambe al primo, e poi anche all'altro che era crocifisso con lui;
\par 33 ma venuti a Gesù, come lo videro già morto, non gli fiaccarono le gambe,
\par 34 ma uno de' soldati gli forò il costato con una lancia, e subito ne uscì sangue ed acqua.
\par 35 E colui che l'ha veduto, ne ha reso testimonianza, e la sua testimonianza è verace, ed egli sa che dice il vero, affinché anche voi crediate.
\par 36 Poiché questo è avvenuto affinché si adempisse la Scrittura: Niun osso d'esso sarà fiaccato.
\par 37 E anche un'altra Scrittura dice: Volgeranno lo sguardo a colui che hanno trafitto.
\par 38 Dopo queste cose, Giuseppe d'Arimatea, che era discepolo di Gesù, ma occulto per timore de' Giudei, chiese a Pilato di poter togliere il corpo di Gesù; e Pilato glielo permise. Egli dunque venne e tolse il corpo di Gesù.
\par 39 E Nicodemo, che da prima era venuto a Gesù di notte, venne anche egli, portando una mistura di mirra e d'aloe di circa cento libbre.
\par 40 Essi dunque presero il corpo di Gesù e lo avvolsero in pannilini con gli aromi, com'è usanza di seppellire presso i Giudei.
\par 41 Or nel luogo dov'egli fu crocifisso c'era un orto; e in quell'orto un sepolcro nuovo, dove nessuno era ancora stato posto.
\par 42 Quivi dunque posero Gesù, a motivo della Preparazione dei Giudei, perché il sepolcro era vicino.

\chapter{20}

\par 1 Or il primo giorno della settimana, la mattina per tempo, mentr'era ancora buio, Maria Maddalena venne al sepolcro, e vide la pietra tolta dal sepolcro.
\par 2 Allora corse e venne da Simon Pietro e dall'altro discepolo che Gesù amava, e disse loro: Han tolto il Signore dal sepolcro, e non sappiamo dove l'abbiano posto.
\par 3 Pietro dunque e l'altro discepolo uscirono e si avviarono al sepolcro.
\par 4 Correvano ambedue assieme; ma l'altro discepolo corse innanzi più presto di Pietro, e giunse primo al sepolcro;
\par 5 e chinatosi, vide i pannilini giacenti, ma non entrò.
\par 6 Giunse intanto anche Simon Pietro che lo seguiva, ed entrò nel sepolcro, e vide i pannilini giacenti,
\par 7 e il sudario ch'era stato sul capo di Gesù, non giacente coi pannilini, ma rivoltato in un luogo a parte.
\par 8 Allora entrò anche l'altro discepolo che era giunto primo al sepolcro, e vide, e credette.
\par 9 Perché non aveano ancora capito la Scrittura, secondo la quale egli dovea risuscitare dai morti.
\par 10 I discepoli dunque se ne tornarono a casa.
\par 11 Ma Maria se ne stava di fuori presso al sepolcro a piangere. E mentre piangeva, si chinò per guardar dentro al sepolcro,
\par 12 ed ecco, vide due angeli, vestiti di bianco, seduti uno a capo e l'altro ai piedi, là dov'era giaciuto il corpo di Gesù.
\par 13 Ed essi le dissero: Donna, perché piangi? Ella disse loro: Perché han tolto il mio Signore, e non so dove l'abbiano posto.
\par 14 Detto questo, si voltò indietro, e vide Gesù in piedi; ma non sapeva che era Gesù.
\par 15 Gesù le disse: Donna, perché piangi? Chi cerchi? Ella, pensando che fosse l'ortolano, gli disse: Signore, se tu l'hai portato via, dimmi dove l'hai posto, e io lo prenderò.
\par 16 Gesù le disse: Maria! Ella, rivoltasi, gli disse in ebraico: Rabbunì! che vuol dire: Maestro!
\par 17 Gesù le disse: Non mi toccare, perché non sono ancora salito al Padre; ma va' dai miei fratelli, e di' loro: Io salgo al Padre mio e Padre vostro, all'Iddio mio e Iddio vostro.
\par 18 Maria Maddalena andò ad annunziare ai discepoli che avea veduto il Signore, e ch'egli le avea dette queste cose.
\par 19 Or la sera di quello stesso giorno, ch'era il primo della settimana, ed essendo, per timor de' Giudei, serrate le porte del luogo dove si trovavano i discepoli, Gesù venne e si presentò quivi in mezzo, e disse loro:
\par 20 Pace a voi! E detto questo, mostrò loro le mani ed il costato. I discepoli dunque, com'ebbero veduto il Signore, si rallegrarono.
\par 21 Allora Gesù disse loro di nuovo: Pace a voi! Come il Padre mi ha mandato, anch'io mando voi.
\par 22 E detto questo, soffiò su loro e disse: Ricevete lo Spirito Santo.
\par 23 A chi rimetterete i peccati, saranno rimessi; a chi li riterrete, saranno ritenuti.
\par 24 Or Toma, detto Didimo, uno de' dodici, non era con loro quando venne Gesù.
\par 25 Gli altri discepoli dunque gli dissero: Abbiam veduto il Signore! Ma egli disse loro: Se io non vedo nelle sue mani il segno de' chiodi, e se non metto il mio dito nel segno de' chiodi, e se non metto la mia mano nel suo costato, io non crederò.
\par 26 E otto giorni dopo, i suoi discepoli erano di nuovo in casa, e Toma era con loro. Venne Gesù, a porte chiuse, e si presentò in mezzo a loro, e disse: Pace a voi!
\par 27 Poi disse a Toma: Porgi qua il dito, e vedi le mie mani; e porgi la mano e mettila nel mio costato; e non essere incredulo, ma credente.
\par 28 Toma gli rispose e disse: Signor mio e Dio mio!
\par 29 Gesù gli disse: Perché m'hai veduto, tu hai creduto; beati quelli che non han veduto, e hanno creduto!
\par 30 Or Gesù fece in presenza dei discepoli molti altri miracoli, che non sono scritti in questo libro;
\par 31 ma queste cose sono scritte, affinché crediate che Gesù è il Cristo, il Figliuol di Dio, e affinché, credendo, abbiate vita nel suo nome.

\chapter{21}

\par 1 Dopo queste cose, Gesù si fece veder di nuovo ai discepoli presso il mar di Tiberiade; e si fece vedere in questa maniera.
\par 2 Simon Pietro, Toma detto Didimo, Natanaele di Cana di Galilea, i figliuoli di Zebedeo e due altri de' suoi discepoli erano insieme.
\par 3 Simon Pietro disse loro: Io vado a pescare. Essi gli dissero: Anche noi veniamo con te. Uscirono, e montarono nella barca; e quella notte non presero nulla.
\par 4 Or essendo già mattina, Gesù si presentò sulla riva; i discepoli però non sapevano che fosse Gesù.
\par 5 Allora Gesù disse loro: Figliuoli, avete voi del pesce? Essi gli risposero: No.
\par 6 Ed egli disse loro: Gettate la rete dal lato destro della barca, e ne troverete. Essi dunque la gettarono, e non potevano più tirarla su per il gran numero dei pesci.
\par 7 Allora il discepolo che Gesù amava disse a Pietro: È il Signore! E Simon Pietro, udito ch'era il Signore, si cinse il camiciotto, perché era nudo, e si gettò nel mare.
\par 8 Ma gli altri discepoli vennero con la barca, perché non erano molto distanti da terra (circa duecento cubiti), traendo la rete coi pesci.
\par 9 Come dunque furono smontati a terra, videro quivi della brace, e del pesce messovi su, e del pane.
\par 10 Gesù disse loro: Portate qua de' pesci che avete presi ora.
\par 11 Simon Pietro quindi montò nella barca, e tirò a terra la rete piena di centocinquantatre grossi pesci; e benché ce ne fossero tanti, la rete non si strappò.
\par 12 Gesù disse loro: Venite a far colazione. E niuno dei discepoli ardiva domandargli: Chi sei? sapendo che era il Signore.
\par 13 Gesù venne, e prese il pane e lo diede loro; e il pesce similmente.
\par 14 Quest'era già la terza volta che Gesù si faceva vedere ai suoi discepoli, dopo essere risuscitato da' morti.
\par 15 Or quand'ebbero fatto colazione, Gesù disse a Simon Pietro: Simon di Giovanni, m'ami tu più di questi? Ei gli rispose: Sì, Signore tu sai che io t'amo. Gesù gli disse: Pasci i miei agnelli.
\par 16 Gli disse di nuovo una seconda volta: Simon di Giovanni, m'ami tu? Ei gli rispose: Sì, Signore; tu sai che io t'amo. Gesù gli disse: Pastura le mie pecorelle.
\par 17 Gli disse per la terza volta: Simon di Giovanni, mi ami tu? Pietro fu attristato ch'ei gli avesse detto per la terza volta: Mi ami tu? E gli rispose: Signore, tu sai ogni cosa; tu conosci che io t'amo. Gesù gli disse: Pasci le mie pecore.
\par 18 In verità, in verità ti dico che quand'eri più giovane, ti cingevi da te e andavi dove volevi; ma quando sarai vecchio, stenderai le tue mani, e un altro ti cingerà e ti condurrà dove non vorresti.
\par 19 Or disse questo per significare con qual morte egli glorificherebbe Iddio. E dopo aver così parlato, gli disse: Seguimi.
\par 20 Pietro, voltatosi, vide venirgli dietro il discepolo che Gesù amava; quello stesso, che durante la cena stava inclinato sul seno di Gesù e avea detto: Signore, chi è che ti tradisce?
\par 21 Pietro dunque, vedutolo, disse a Gesù: Signore, e di lui che sarà?
\par 22 Gesù gli rispose: Se voglio che rimanga finch'io venga, che t'importa? Tu, seguimi.
\par 23 Ond'è che si sparse tra i fratelli la voce che quel discepolo non morrebbe; Gesù però non gli avea detto che non morrebbe, ma: Se voglio che rimanga finch'io venga, che t'importa?
\par 24 Questo è il discepolo che rende testimonianza di queste cose, e che ha scritto queste cose; e noi sappiamo che la sua testimonianza è verace.
\par 25 Or vi sono ancora molte altre cose che Gesù ha fatte, le quali se si scrivessero ad una ad una, credo che il mondo stesso non potrebbe contenere i libri che se ne scriverebbero.


\end{document}