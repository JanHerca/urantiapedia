\begin{document}

\title{Genesi}


\chapter{1}

\par 1 Nel principio Iddio creò i cieli e la terra.
\par 2 E la terra era informe e vuota, e le tenebre coprivano la faccia dell'abisso, e lo spirito di Dio aleggiava sulla superficie delle acque. E Dio disse:
\par 3 'Sia la luce!' E la luce fu.
\par 4 E Dio vide che la luce era buona; e Dio separò la luce dalle tenebre.
\par 5 E Dio chiamò la luce 'giorno', e le tenebre 'notte'. Così fu sera, poi fu mattina: e fu il primo giorno.
\par 6 Poi Dio disse: 'Ci sia una distesa tra le acque, che separi le acque dalle acque'.
\par 7 E Dio fece la distesa e separò le acque ch'erano sotto la distesa, dalle acque ch'erano sopra la distesa. E così fu.
\par 8 E Dio chiamò la distesa 'cielo'. Così fu sera, poi fu mattina: e fu il secondo giorno.
\par 9 Poi Dio disse: 'Le acque che son sotto il cielo siano raccolte in un unico luogo, e apparisca l'asciutto'. E così fu.
\par 10 E Dio chiamò l'asciutto 'terra', e chiamò la raccolta delle acque 'mari'. E Dio vide che questo era buono.
\par 11 Poi Dio disse: 'Produca la terra della verdura, dell'erbe che faccian seme e degli alberi fruttiferi che, secondo la loro specie, portino del frutto avente in sé la propria semenza, sulla terra'. E così fu.
\par 12 E la terra produsse della verdura, dell'erbe che facevan seme secondo la loro specie, e degli alberi che portavano del frutto avente in sé la propria semenza, secondo la loro specie. E Dio vide che questo era buono.
\par 13 Così fu sera, poi fu mattina: e fu il terzo giorno.
\par 14 Poi Dio disse: 'Sianvi de' luminari nella distesa dei cieli per separare il giorno dalla notte; e siano dei segni e per le stagioni e per i giorni e per gli anni;
\par 15 e servano da luminari nella distesa dei cieli per dar luce alla terra'. E così fu.
\par 16 E Dio fece i due grandi luminari: il luminare maggiore, per presiedere al giorno, e il luminare minore per presiedere alla notte; e fece pure le stelle.
\par 17 E Dio li mise nella distesa dei cieli per dar luce alla terra,
\par 18 per presiedere al giorno e alla notte e separare la luce dalle tenebre. E Dio vide che questo era buono.
\par 19 Così fu sera, poi fu mattina: e fu il quarto giorno.
\par 20 Poi Dio disse: 'Producano le acque in abbondanza animali viventi, e volino degli uccelli sopra la terra per l'ampia distesa del cielo'.
\par 21 E Dio creò i grandi animali acquatici e tutti gli esseri viventi che si muovono, i quali le acque produssero in abbondanza secondo la loro specie, ed ogni volatile secondo la sua specie. E Dio vide che questo era buono.
\par 22 E Dio li benedisse, dicendo: 'Crescete, moltiplicate, ed empite le acque dei mari, e moltiplichino gli uccelli sulla terra'.
\par 23 Così fu sera, poi fu mattina: e fu il quinto giorno.
\par 24 Poi Dio disse: 'Produca la terra animali viventi secondo la loro specie: bestiame, rettili e animali selvatici della terra, secondo la loro specie'. E così fu.
\par 25 E Dio fece gli animali selvatici della terra, secondo le loro specie, il bestiame secondo le sue specie, e tutti i rettili della terra, secondo le loro specie. E Dio vide che questo era buono.
\par 26 Poi Dio disse: 'Facciamo l'uomo a nostra immagine e a nostra somiglianza, ed abbia dominio sui pesci del mare e sugli uccelli del cielo e sul bestiame e su tutta la terra e su tutti i rettili che strisciano sulla terra'.
\par 27 E Dio creò l'uomo a sua immagine; lo creò a immagine di Dio; li creò maschio e femmina.
\par 28 E Dio li benedisse; e Dio disse loro: 'Crescete e moltiplicate e riempite la terra, e rendetevela soggetta, e dominate sui pesci del mare e sugli uccelli del cielo e sopra ogni animale che si muove sulla terra'.
\par 29 E Dio disse: 'Ecco, io vi do ogni erba che fa seme sulla superficie di tutta la terra, ed ogni albero fruttifero che fa seme; questo vi servirà di nutrimento.
\par 30 E ad ogni animale della terra e ad ogni uccello dei cieli e a tutto ciò che si muove sulla terra ed ha in sé un soffio di vita, io do ogni erba verde per nutrimento'. E così fu.
\par 31 E Dio vide tutto quello che aveva fatto, ed ecco, era molto buono. Così fu sera, poi fu mattina: e fu il sesto giorno.

\chapter{2}

\par 1 Così furono compiti i cieli e la terra e tutto l'esercito loro.
\par 2 Il settimo giorno, Iddio compì l'opera che aveva fatta, e si riposò il settimo giorno da tutta l'opera che aveva fatta.
\par 3 E Dio benedisse il settimo giorno e lo santificò, perché in esso si riposò da tutta l'opera che aveva creata e fatta.
\par 4 Queste sono le origini dei cieli e della terra quando furono creati, nel giorno che l'Eterno Iddio fece la terra e i cieli.
\par 5 Non c'era ancora sulla terra alcun arbusto della campagna, e nessuna erba della campagna era ancora spuntata, perché l'Eterno Iddio non avea fatto piovere sulla terra, e non c'era alcun uomo per coltivare il suolo:
\par 6 ma un vapore saliva dalla terra e adacquava tutta la superficie del suolo.
\par 7 E l'Eterno Iddio formò l'uomo dalla polvere della terra, gli soffiò nelle narici un alito vitale, e l'uomo divenne un'anima vivente.
\par 8 E l'Eterno Iddio piantò un giardino in Eden, in oriente, e quivi pose l'uomo che aveva formato.
\par 9 E l'Eterno Iddio fece spuntare dal suolo ogni sorta d'alberi piacevoli a vedersi e il cui frutto era buono da mangiare, e l'albero della vita in mezzo al giardino, e l'albero della conoscenza del bene e del male.
\par 10 E un fiume usciva d'Eden per adacquare il giardino, e di là si spartiva in quattro bracci.
\par 11 Il nome del primo è Pishon, ed è quello che circonda tutto il paese di Havila dov'è l'oro;
\par 12 e l'oro di quel paese è buono; quivi si trovan pure il bdellio e l'ònice.
\par 13 Il nome del secondo fiume è Ghihon, ed è quello che circonda tutto il paese di Cush.
\par 14 Il nome del terzo fiume è Hiddekel, ed è quello che scorre a oriente dell'Assiria. E il quarto fiume è l'Eufrate.
\par 15 L'Eterno Iddio prese dunque l'uomo e lo pose nel giardino d'Eden perché lo lavorasse e lo custodisse.
\par 16 E l'Eterno Iddio diede all'uomo questo comandamento: 'Mangia pure liberamente del frutto d'ogni albero del giardino;
\par 17 ma del frutto dell'albero della conoscenza del bene e del male non ne mangiare; perché, nel giorno che tu ne mangerai, per certo morrai'.
\par 18 Poi l'Eterno Iddio disse: 'Non è bene che l'uomo sia solo; io gli farò un aiuto che gli sia convenevole'.
\par 19 E l'Eterno Iddio avendo formato dalla terra tutti gli animali dei campi e tutti gli uccelli dei cieli, li menò all'uomo per vedere come li chiamerebbe, e perché ogni essere vivente portasse il nome che l'uomo gli darebbe.
\par 20 E l'uomo dette de' nomi a tutto il bestiame, agli uccelli dei cieli e ad ogni animale dei campi; ma per l'uomo non si trovò aiuto che gli fosse convenevole.
\par 21 Allora l'Eterno Iddio fece cadere un profondo sonno sull'uomo, che s'addormentò; e prese una delle costole di lui, e richiuse la carne al posto d'essa.
\par 22 E l'Eterno Iddio, con la costola che avea tolta all'uomo, formò una donna e la menò all'uomo.
\par 23 E l'uomo disse: 'Questa, finalmente, è ossa delle mie ossa e carne della mia carne. Ella sarà chiamata donna perché è stata tratta dall'uomo'.
\par 24 Perciò l'uomo lascerà suo padre e sua madre e si unirà alla sua moglie, e saranno una stessa carne.
\par 25 E l'uomo e la sua moglie erano ambedue ignudi e non ne aveano vergogna.

\chapter{3}

\par 1 Or il serpente era il più astuto di tutti gli animali dei campi che l'Eterno Iddio aveva fatti; ed esso disse alla donna: 'Come! Iddio v'ha detto: Non mangiate del frutto di tutti gli alberi del giardino?'
\par 2 E la donna rispose al serpente: 'Del frutto degli alberi del giardino ne possiamo mangiare;
\par 3 ma del frutto dell'albero ch'è in mezzo al giardino Iddio ha detto: Non ne mangiate e non lo toccate, che non abbiate a morire'.
\par 4 E il serpente disse alla donna: 'No, non morrete affatto;
\par 5 ma Iddio sa che nel giorno che ne mangerete, gli occhi vostri s'apriranno, e sarete come Dio, avendo la conoscenza del bene e del male'.
\par 6 E la donna vide che il frutto dell'albero era buono a mangiarsi, ch'era bello a vedere, e che l'albero era desiderabile per diventare intelligente; prese del frutto, ne mangiò, e ne dette anche al suo marito ch'era con lei, ed egli ne mangiò.
\par 7 Allora si apersero gli occhi ad ambedue e s'accorsero ch'erano ignudi; e cucirono delle foglie di fico, e se ne fecero delle cinture.
\par 8 E udirono la voce dell'Eterno Iddio, il quale camminava nel giardino sul far della sera; e l'uomo e sua moglie si nascosero dalla presenza dell'Eterno Iddio, fra gli alberi del giardino.
\par 9 E l'Eterno Iddio chiamò l'uomo e gli disse: 'Dove sei?' E quegli rispose:
\par 10 'Ho udito la tua voce nel giardino, e ho avuto paura, perch'ero ignudo, e mi sono nascosto'.
\par 11 E Dio disse: 'Chi t'ha mostrato ch'eri ignudo? Hai tu mangiato del frutto dell'albero del quale io t'avevo comandato di non mangiare?'
\par 12 L'uomo rispose: 'La donna che tu m'hai messa accanto, è lei che m'ha dato del frutto dell'albero, e io n'ho mangiato'.
\par 13 E l'Eterno Iddio disse alla donna: 'Perché hai fatto questo?' E la donna rispose: 'Il serpente mi ha sedotta, ed io ne ho mangiato'.
\par 14 Allora l'Eterno Iddio disse al serpente: 'Perché hai fatto questo, sii maledetto fra tutto il bestiame e fra tutti gli animali dei campi! Tu camminerai sul tuo ventre, e mangerai polvere tutti i giorni della tua vita.
\par 15 E io porrò inimicizia fra te e la donna, e fra la tua progenie e la progenie di lei; questa progenie ti schiaccerà il capo, e tu le ferirai il calcagno'.
\par 16 Alla donna disse: 'Io moltiplicherò grandemente le tue pene e i dolori della tua gravidanza; con dolore partorirai figliuoli; i tuoi desiderî si volgeranno verso il tuo marito, ed egli dominerà su te'.
\par 17 E ad Adamo disse: 'Perché hai dato ascolto alla voce della tua moglie e hai mangiato del frutto dell'albero circa il quale io t'avevo dato quest'ordine: Non ne mangiare, il suolo sarà maledetto per causa tua; ne mangerai il frutto con affanno, tutti i giorni della tua vita.
\par 18 Esso ti produrrà spine e triboli, e tu mangerai l'erba dei campi;
\par 19 mangerai il pane col sudore del tuo volto finché tu ritorni nella terra donde fosti tratto; perché sei polvere, e in polvere ritornerai'.
\par 20 E l'uomo pose nome Eva alla sua moglie, perch'è stata la madre di tutti i viventi.
\par 21 E l'Eterno Iddio fece ad Adamo e alla sua moglie delle tuniche di pelle, e li vestì.
\par 22 Poi l'Eterno Iddio disse: 'Ecco, l'uomo è diventato come uno di noi, quanto a conoscenza del bene e del male. Guardiamo ch'egli non stenda la mano e prenda anche del frutto dell'albero della vita, e ne mangi, e viva in perpetuo'.
\par 23 Perciò l'Eterno Iddio mandò via l'uomo dal giardino d'Eden, perché lavorasse la terra donde era stato tratto.
\par 24 Così egli scacciò l'uomo; e pose ad oriente del giardino d'Eden i cherubini, che vibravano da ogni parte una spada fiammeggiante, per custodire la via dell'albero della vita.

\chapter{4}

\par 1 Or Adamo conobbe Eva sua moglie, la quale concepì e partorì Caino, e disse: 'Ho acquistato un uomo, con l'aiuto dell'Eterno'.
\par 2 Poi partorì ancora Abele, fratello di lui. E Abele fu pastore di pecore; e Caino, lavoratore della terra.
\par 3 E avvenne, di lì a qualche tempo, che Caino fece un'offerta di frutti della terra all'Eterno;
\par 4 e Abele offerse anch'egli dei primogeniti del suo gregge e del loro grasso. E l'Eterno guardò con favore Abele e la sua offerta,
\par 5 ma non guardò con favore Caino e l'offerta sua. E Caino ne fu molto irritato, e il suo viso ne fu abbattuto.
\par 6 E l'Eterno disse a Caino: 'Perché sei tu irritato? e perché hai il volto abbattuto?
\par 7 Se fai bene non rialzerai tu il volto? ma, se fai male, il peccato sta spiandoti alla porta, e i suoi desideri son vòlti a te; ma tu lo devi dominare!'
\par 8 E Caino disse ad Abele suo fratello: 'Usciamo fuori ai campi!' E avvenne che, quando furono nei campi, Caino si levò contro Abele suo fratello, e l'uccise.
\par 9 E l'Eterno disse a Caino: 'Dov'è Abele tuo fratello?' Ed egli rispose: 'Non lo so; sono io forse il guardiano di mio fratello?'
\par 10 E l'Eterno disse: 'Che hai tu fatto? la voce del sangue di tuo fratello grida a me dalla terra.
\par 11 E ora tu sarai maledetto, condannato ad errar lungi dalla terra che ha aperto la sua bocca per ricevere il sangue del tuo fratello dalla tua mano.
\par 12 Quando coltiverai il suolo, esso non ti darà più i suoi prodotti, e tu sarai vagabondo e fuggiasco sulla terra'.
\par 13 E Caino disse all'Eterno: 'Il mio castigo è troppo grande perch'io lo possa sopportare.
\par 14 Ecco, tu mi scacci oggi dalla faccia di questo suolo, ed io sarò nascosto dal tuo cospetto, e sarò vagabondo e fuggiasco per la terra; e avverrà che chiunque mi troverà mi ucciderà'.
\par 15 E l'Eterno gli disse: 'Perciò, chiunque ucciderà Caino, sarà punito sette volte più di lui'. E l'Eterno mise un segno su Caino, affinché nessuno, trovandolo, l'uccidesse.
\par 16 E Caino si partì dal cospetto dell'Eterno e dimorò nel paese di Nod, ad oriente di Eden.
\par 17 E Caino conobbe la sua moglie, la quale concepì e partorì Enoc. Poi, si mise a edificare una città, a cui diede il nome di Enoc, dal nome del suo figliuolo.
\par 18 E ad Enoc nacque Irad; Irad generò Mehujael; Mehujael generò Methushael, e Methushael generò Lamec.
\par 19 E Lamec prese due mogli: il nome dell'una era Ada, e il nome dell'altra, Zilla.
\par 20 E Ada partorì Iabal, che fu il padre di quelli che abitano sotto le tende presso i greggi.
\par 21 E il nome del suo fratello era Jubal, che fu il padre di tutti quelli che suonano la cetra ed il flauto.
\par 22 E Zilla partorì anch'essa Tubal-cain, l'artefice d'ogni sorta di strumenti di rame e di ferro; e la sorella di Tubal-cain fu Naama.
\par 23 E Lamec disse alle sue mogli: 'Ada e Zilla, ascoltate la mia voce; mogli di Lamec, porgete orecchio al mio dire! Sì, io ho ucciso un uomo perché m'ha ferito, e un giovine perché m'ha contuso.
\par 24 Se Caino sarà vendicato sette volte, Lamec lo sarà settantasette volte'.
\par 25 E Adamo conobbe ancora la sua moglie, ed essa partorì un figliuolo, a cui pose nome Seth, 'perché' ella disse, 'Iddio m'ha dato un altro figliuolo al posto d'Abele, che Caino ha ucciso'.
\par 26 E anche a Seth nacque un figliuolo, a cui pose nome Enosh. Allora si cominciò a invocare il nome dell'Eterno.

\chapter{5}

\par 1 Questo è il libro della posterità d'Adamo. Nel giorno che Dio creò l'uomo, lo fece a somiglianza di Dio;
\par 2 li creò maschio e femmina, li benedisse e dette loro il nome di 'uomo', nel giorno che furon creati.
\par 3 Adamo visse centotrent'anni, generò un figliuolo, a sua somiglianza, conforme alla sua immagine, e gli pose nome Seth;
\par 4 e il tempo che Adamo visse, dopo ch'ebbe generato Seth, fu ottocent'anni, e generò figliuoli e figliuole;
\par 5 e tutto il tempo che Adamo visse fu novecentotrent'anni; poi morì.
\par 6 E Seth visse centocinque anni, e generò Enosh.
\par 7 E Seth, dopo ch'ebbe generato Enosh, visse ottocentosette anni, e generò figliuoli e figliuole;
\par 8 e tutto il tempo che Seth visse fu novecentododici anni; poi morì.
\par 9 Ed Enosh visse novant'anni, e generò Kenan.
\par 10 Ed Enosh, dopo ch'ebbe generato Kenan, visse ottocentoquindici anni, e generò figliuoli e figliuole;
\par 11 e tutto il tempo che Enosh visse fu novecentocinque anni; poi morì.
\par 12 E Kenan visse settant'anni, e generò Mahalaleel.
\par 13 E Kenan, dopo ch'ebbe generato Mahalaleel, visse ottocentoquarant'anni, e generò figliuoli e figliuole;
\par 14 e tutto il tempo che Kenan visse fu novecentodieci anni; poi morì.
\par 15 E Mahalaleel visse sessantacinque anni, e generò Jared.
\par 16 E Mahalaleel, dopo ch'ebbe generato Jared, visse ottocentotrent'anni, e generò figliuoli e figliuole;
\par 17 e tutto il tempo che Mahalaleel visse fu ottocentonovantacinque anni; poi morì.
\par 18 E Jared visse centosessantadue anni, e generò Enoc.
\par 19 E Jared, dopo ch'ebbe generato Enoc, visse ottocent'anni, e generò figliuoli e figliuole;
\par 20 e tutto il tempo che Jared visse fu novecentosessantadue anni; poi morì.
\par 21 Ed Enoc visse sessantacinque anni, e generò Methushelah.
\par 22 Ed Enoc, dopo ch'ebbe generato Methushelah, camminò con Dio trecent'anni, e generò figliuoli e figliuole;
\par 23 e tutto il tempo che Enoc visse fu trecentosessantacinque anni.
\par 24 Ed Enoc camminò con Dio; poi disparve, perché Iddio lo prese.
\par 25 E Methushelah visse centottantasette anni e generò Lamec.
\par 26 E Methushelah, dopo ch'ebbe generato Lamec, visse settecentottantadue anni, e generò figliuoli e figliuole;
\par 27 e tutto il tempo che Methushelah visse fu novecentosessantanove anni; poi morì.
\par 28 E Lamec visse centottantadue anni, e generò un figliuolo;
\par 29 e gli pose nome Noè, dicendo: 'Questo ci consolerà della nostra opera e della fatica delle nostre mani cagionata dal suolo che l'Eterno ha maledetto'.
\par 30 E Lamec, dopo ch'ebbe generato Noè, visse cinquecentonovantacinque anni, e generò figliuoli e figliuole;
\par 31 e tutto il tempo che Lamec visse fu settecentosettantasette anni; poi morì.
\par 32 E Noè, all'età di cinquecent'anni, generò Sem, Cam e Jafet.

\chapter{6}

\par 1 Or quando gli uomini cominciarono a moltiplicarsi sulla faccia della terra e furon loro nate delle figliuole,
\par 2 avvenne che i figliuoli di Dio videro che le figliuole degli uomini erano belle, e presero per mogli quelle che si scelsero fra tutte.
\par 3 E l'Eterno disse: 'Lo spirito mio non contenderà per sempre con l'uomo; poiché, nel suo traviamento, egli non è che carne; i suoi giorni saranno quindi centovent'anni'.
\par 4 In quel tempo c'erano sulla terra i giganti, e ci furono anche di poi, quando i figliuoli di Dio si accostarono alle figliuole degli uomini, e queste fecero loro de' figliuoli. Essi sono gli uomini potenti che, fin dai tempi antichi, sono stati famosi.
\par 5 E l'Eterno vide che la malvagità degli uomini era grande sulla terra, e che tutti i disegni dei pensieri del loro cuore non erano altro che male in ogni tempo.
\par 6 E l'Eterno si pentì d'aver fatto l'uomo sulla terra, e se ne addolorò in cuor suo.
\par 7 E l'Eterno disse: 'Io sterminerò di sulla faccia della terra l'uomo che ho creato: dall'uomo al bestiame, ai rettili, agli uccelli dei cieli; perché mi pento d'averli fatti'.
\par 8 Ma Noè trovò grazia agli occhi dell'Eterno.
\par 9 Questa è la posterità di Noè. Noè fu uomo giusto, integro, ai suoi tempi; Noè camminò con Dio.
\par 10 E Noè generò tre figliuoli: Sem, Cam e Jafet.
\par 11 Or la terra era corrotta davanti a Dio; la terra era ripiena di violenza.
\par 12 E Dio guardò la terra; ed ecco, era corrotta, poiché ogni carne aveva corrotto la sua via sulla terra.
\par 13 E Dio disse a Noè: 'Nei miei decreti, la fine d'ogni carne è giunta; poiché la terra, per opera degli uomini, è piena di violenza; ecco, io li distruggerò, insieme con la terra.
\par 14 Fatti un'arca di legno di gofer; falla a stanze, e spalmala di pece, di dentro e di fuori.
\par 15 Ed ecco come la dovrai fare: la lunghezza dell'arca sarà di trecento cubiti; la larghezza, di cinquanta cubiti, e l'altezza, di trenta cubiti.
\par 16 Farai all'arca una finestra, in alto, e le darai la dimensione d'un cubito; metterai la porta da un lato, e farai l'arca a tre piani: uno da basso, un secondo e un terzo piano.
\par 17 Ed ecco, io sto per far venire il diluvio delle acque sulla terra, per distruggere di sotto i cieli ogni carne in cui è alito di vita; tutto quello ch'è sopra la terra, morrà.
\par 18 Ma io stabilirò il mio patto con te; e tu entrerai nell'arca: tu e i tuoi figliuoli, la tua moglie e le mogli de' tuoi figliuoli, con te.
\par 19 E di tutto ciò che vive, d'ogni carne, fanne entrare nell'arca due d'ogni specie, per conservarli in vita con te; e siano maschio e femmina.
\par 20 Degli uccelli secondo le loro specie, del bestiame secondo le sue specie, e di tutti i rettili della terra secondo le loro specie, due d'ogni specie verranno a te, perché tu li conservi in vita.
\par 21 E tu prenditi d'ogni cibo che si mangia, e fattene provvista, perché serva di nutrimento a te e a loro'.
\par 22 E Noè fece così; fece tutto quello che Dio gli avea comandato.

\chapter{7}

\par 1 E l'Eterno disse a Noè: 'Entra nell'arca tu con tutta la tua famiglia, poiché t'ho veduto giusto nel mio cospetto, in questa generazione.
\par 2 D'ogni specie di animali puri prendine sette paia, maschio e femmina; e degli animali impuri un paio, maschio e femmina;
\par 3 e parimente degli uccelli dei cieli prendine sette paia, maschio e femmina, per conservarne in vita la razza sulla faccia di tutta la terra;
\par 4 poiché di qui a sette giorni farò piovere sulla terra per quaranta giorni e quaranta notti, e sterminerò di sulla faccia della terra tutti gli esseri viventi che ho fatto'.
\par 5 E Noè fece tutto quello che l'Eterno gli avea comandato.
\par 6 Noè era in età di seicent'anni, quando il diluvio delle acque inondò la terra.
\par 7 E Noè, coi suoi figliuoli, con la sua moglie e con le mogli de' suoi figliuoli, entrò nell'arca per scampare dalle acque del diluvio.
\par 8 Degli animali puri e degli animali impuri, degli uccelli e di tutto quello che striscia sulla terra,
\par 9 vennero delle coppie, maschio e femmina, a Noè nell'arca, come Dio avea comandato a Noè.
\par 10 E, al termine dei sette giorni, avvenne che le acque del diluvio furono sulla terra.
\par 11 L'anno seicentesimo della vita di Noè, il secondo mese, il diciassettesimo giorno del mese, in quel giorno, tutte le fonti del grande abisso scoppiarono e le cateratte del cielo s'aprirono.
\par 12 E piovve sulla terra per quaranta giorni e quaranta notti.
\par 13 In quello stesso giorno, Noè, Sem, Cam e Jafet, figliuoli di Noè, la moglie di Noè e le tre mogli dei suoi figliuoli con loro, entrarono nell'arca:
\par 14 essi, e tutti gli animali secondo le loro specie, e tutto il bestiame secondo le sue specie, e tutti i rettili che strisciano sulla terra, secondo le loro specie, e tutti gli uccelli secondo le loro specie, tutti gli uccelletti, tutto quel che porta ali.
\par 15 D'ogni carne in cui è alito di vita venne una coppia a Noè nell'arca:
\par 16 venivano maschio e femmina d'ogni carne, come Dio avea comandato a Noè; poi l'Eterno lo chiuse dentro l'arca.
\par 17 E il diluvio venne sopra la terra per quaranta giorni; e le acque crebbero e sollevarono l'arca, che fu levata in alto d'in su la terra.
\par 18 E le acque ingrossarono e crebbero grandemente sopra la terra, e l'arca galleggiava sulla superficie delle acque.
\par 19 E le acque ingrossarono oltremodo sopra la terra; e tutte le alte montagne che erano sotto tutti i cieli, furono coperte.
\par 20 Le acque salirono quindici cubiti al disopra delle vette dei monti; e le montagne furon coperte.
\par 21 E perì ogni carne che si moveva sulla terra: uccelli, bestiame, animali selvatici, rettili d'ogni sorta striscianti sulla terra, e tutti gli uomini.
\par 22 Tutto quello ch'era sulla terra asciutta ed aveva alito di vita nelle sue narici, morì.
\par 23 E tutti gli esseri che erano sulla faccia della terra furono sterminati: dall'uomo fino al bestiame, ai rettili e agli uccelli del cielo; furono sterminati di sulla terra; non scampò che Noè con quelli ch'erano con lui nell'arca.
\par 24 E le acque rimasero alte sopra la terra per centocinquanta giorni.

\chapter{8}

\par 1 Or Iddio si ricordò di Noè, di tutti gli animali e di tutto il bestiame ch'era con lui nell'arca; e Dio fece passare un vento sulla terra, e le acque si calmarono;
\par 2 le fonti dell'abisso e le cateratte del cielo furono chiuse; e cessò la pioggia dal cielo;
\par 3 le acque andarono del continuo ritirandosi di sulla terra, e alla fine di centocinquanta giorni cominciarono a scemare.
\par 4 E nel settimo mese, il decimosettimo giorno del mese, l'arca si fermò sulle montagne di Ararat.
\par 5 E le acque andarono scemando fino al decimo mese. Nel decimo mese, il primo giorno del mese, apparvero le vette dei monti.
\par 6 E in capo a quaranta giorni, Noè aprì la finestra che avea fatta nell'arca,
\par 7 e mandò fuori il corvo, il quale uscì, andando e tornando, finché le acque furono asciugate sulla terra.
\par 8 Poi mandò fuori la colomba, per vedere se le acque fossero diminuite sulla superficie della terra.
\par 9 Ma la colomba non trovò dove posar la pianta del suo piede, e tornò a lui nell'arca, perché c'eran delle acque sulla superficie di tutta la terra; ed egli stese la mano, la prese, e la portò con sé dentro l'arca.
\par 10 E aspettò altri sette giorni, poi mandò di nuovo la colomba fuori dell'arca.
\par 11 E la colomba tornò a lui, verso sera; ed ecco, essa aveva nel becco una foglia fresca d'ulivo; onde Noè capì che le acque erano scemate sopra la terra.
\par 12 E aspettò altri sette giorni, poi mandò fuori la colomba; ma essa non tornò più a lui.
\par 13 L'anno seicentesimoprimo di Noè, il primo mese, il primo giorno del mese, le acque erano asciugate sulla terra; e Noè scoperchiò l'arca, guardò, ed ecco che la superficie del suolo era asciutta.
\par 14 E il secondo mese, il ventisettesimo giorno del mese, la terra era asciutta.
\par 15 E Dio parlò a Noè, dicendo:
\par 16 'Esci dall'arca tu e la tua moglie, i tuoi figliuoli e le mogli dei tuoi figliuoli con te.
\par 17 Fa' uscire con te tutti gli animali che sono teco, d'ogni carne: uccelli, bestiame, e tutti i rettili che strisciano sulla terra, perché abbondino sulla terra, e figlino e moltiplichino sulla terra'.
\par 18 E Noè uscì con i suoi figliuoli, con la sua moglie, e con le mogli dei suoi figliuoli.
\par 19 Tutti gli animali, tutti i rettili, tutti gli uccelli, tutto quel che si muove sulla terra, secondo le loro famiglie, uscirono dall'arca.
\par 20 E Noè edificò un altare all'Eterno; prese d'ogni specie d'animali puri e d'ogni specie d'uccelli puri, e offrì olocausti sull'altare.
\par 21 E l'Eterno sentì un odor soave; e l'Eterno disse in cuor suo: 'Io non maledirò più la terra a cagione dell'uomo, poiché i disegni del cuor dell'uomo sono malvagi fin dalla sua fanciullezza; e non colpirò più ogni cosa vivente, come ho fatto.
\par 22 Finché la terra durerà, sementa e raccolta, freddo e caldo, estate e inverno, giorno e notte, non cesseranno mai'.

\chapter{9}

\par 1 E Dio benedisse Noè e i suoi figliuoli, e disse loro: 'Crescete, moltiplicate, e riempite la terra.
\par 2 E avranno timore e spavento di voi tutti gli animali della terra e tutti gli uccelli del cielo. Essi son dati in poter vostro con tutto ciò che striscia sulla terra e con tutti i pesci del mare.
\par 3 Tutto ciò che si muove ed ha vita vi servirà di cibo; io vi do tutto questo, come l'erba verde;
\par 4 ma non mangerete carne con la vita sua, cioè col suo sangue.
\par 5 E, certo, io chiederò conto del vostro sangue, del sangue delle vostre vite; ne chiederò conto ad ogni animale; e chiederò conto della vita dell'uomo alla mano dell'uomo, alla mano d'ogni suo fratello.
\par 6 Il sangue di chiunque spargerà il sangue dell'uomo sarà sparso dall'uomo, perché Dio ha fatto l'uomo a immagine sua.
\par 7 Voi dunque crescete e moltiplicate; spandetevi sulla terra, e moltiplicate in essa'.
\par 8 Poi Dio parlò a Noè e ai suoi figliuoli con lui, dicendo:
\par 9 'Quanto a me, ecco, stabilisco il mio patto con voi e con la vostra progenie dopo voi,
\par 10 e con tutti gli esseri viventi che sono con voi: uccelli, bestiame, e tutti gli animali della terra con voi; da tutti quelli che sono usciti dall'arca, a tutti quanti gli animali della terra.
\par 11 Io stabilisco il mio patto con voi, e nessuna carne sarà più sterminata dalle acque del diluvio, e non ci sarà più diluvio per distruggere la terra'.
\par 12 E Dio disse: 'Ecco il segno del patto che io fo tra me e voi e tutti gli esseri viventi che sono con voi, per tutte le generazioni a venire.
\par 13 Io pongo il mio arco nella nuvola, e servirà di segno del patto fra me e la terra.
\par 14 E avverrà che quando avrò raccolto delle nuvole al disopra della terra, l'arco apparirà nelle nuvole,
\par 15 e io mi ricorderò del mio patto fra me e voi e ogni essere vivente d'ogni carne, e le acque non diventeranno più un diluvio per distruggere ogni carne.
\par 16 L'arco dunque sarà nelle nuvole, e io lo guarderò per ricordarmi del patto perpetuo fra Dio e ogni essere vivente, di qualunque carne che è sulla terra'.
\par 17 E Dio disse a Noè: 'Questo è il segno del patto che io ho stabilito fra me e ogni carne che è sulla terra'.
\par 18 E i figliuoli di Noè che uscirono dall'arca furono Sem, Cam e Jafet; e Cam è il padre di Canaan.
\par 19 Questi sono i tre figliuoli di Noè; e da loro fu popolata tutta la terra.
\par 20 Or Noè, ch'era agricoltore, cominciò a piantar la vigna;
\par 21 e bevve del vino e s'inebriò e si scoperse in mezzo alla sua tenda.
\par 22 E Cam, padre di Canaan, vide la nudità del padre suo, e andò a dirlo fuori, ai suoi fratelli.
\par 23 Ma Sem e Jafet presero il suo mantello, se lo misero assieme sulle spalle, e, camminando all'indietro, coprirono la nudità del loro padre; e siccome aveano la faccia vòlta alla parte opposta, non videro la nudità del loro padre.
\par 24 E quando Noè si svegliò dalla sua ebbrezza, seppe quello che gli avea fatto il suo figliuolo minore, e disse:
\par 25 'Maledetto sia Canaan! Sia servo dei servi de' suoi fratelli!' E disse ancora:
\par 26 'Benedetto sia l'Eterno, l'Iddio di Sem, e sia Canaan suo servo!
\par 27 Iddio estenda Jafet, ed abiti egli nelle tende di Sem, e sia Canaan suo servo!'
\par 28 E Noè visse, dopo il diluvio, trecentocinquant'anni.
\par 29 E tutto il tempo che Noè visse fu novecentocinquant'anni; poi morì.

\chapter{10}

\par 1 Questa è la posterità dei figliuoli di Noè: Sem, Cam e Jafet; e a loro nacquero de' figliuoli, dopo il diluvio.
\par 2 I figliuoli di Jafet furono Gomer, Magog, Madai, Javan, Tubal, Mescec e Tiras.
\par 3 I figliuoli di Gomer: Ashkenaz, Rifat e Togarma.
\par 4 I figliuoli di Javan: Elisha, Tarsis, Kittim e Dodanim.
\par 5 Da essi vennero i popoli sparsi nelle isole delle nazioni, nei loro diversi paesi, ciascuno secondo la propria lingua, secondo le loro famiglie, nelle loro nazioni.
\par 6 I figliuoli di Cam furono Cush, Mitsraim, Put e Canaan.
\par 7 I figliuoli di Cush: Seba, Havila, Sabta, Raama e Sabteca; e i figliuoli di Raama: Sceba e Dedan.
\par 8 E Cush generò Nimrod, che cominciò a esser potente sulla terra.
\par 9 Egli fu un potente cacciatore nel cospetto dell'Eterno; perciò si dice: 'Come Nimrod, potente cacciatore nel cospetto dell'Eterno'.
\par 10 E il principio del suo regno fu Babel, Erec, Accad e Calne nel paese di Scinear.
\par 11 Da quel paese andò in Assiria ed edificò Ninive, Rehoboth-Ir e Calah;
\par 12 e, fra Ninive e Calah, Resen, la gran città.
\par 13 Mitsraim generò i Ludim, gli Anamim, i Lehabim, i Naftuhim,
\par 14 i Pathrusim, i Casluhim (donde uscirono i Filistei) e i Caftorim.
\par 15 Canaan generò Sidon, suo primogenito, e Heth,
\par 16 e i Gebusei, gli Amorei, i Ghirgasei,
\par 17 gli Hivvei, gli Archei, i Sinei,
\par 18 gli Arvadei, i Tsemarei e gli Hamattei. Poi le famiglie dei Cananei si sparsero.
\par 19 E i confini dei Cananei andarono da Sidon, in direzione di Gherar, fino a Gaza; e in direzione di Sodoma, Gomorra, Adma e Tseboim, fino a Lesha.
\par 20 Questi sono i figliuoli di Cam, secondo le loro famiglie, secondo le loro lingue, nei loro paesi, nelle loro nazioni.
\par 21 Anche a Sem, padre di tutti i figliuoli di Eber e fratello maggiore di Jafet, nacquero de' figliuoli.
\par 22 I figliuoli di Sem furono Elam, Assur, Arpacshad, Lud e Aram.
\par 23 I figliuoli di Aram: Uz, Hul, Gheter e Mash.
\par 24 E Arpacshad generò Scelah, e Scelah generò Eber.
\par 25 E ad Eber nacquero due figliuoli; il nome dell'uno fu Peleg, perché ai suoi giorni la terra fu spartita; e il nome del suo fratello fu Jokthan.
\par 26 E Jokthan generò Almodad, Scelef, Hatsarmaveth,
\par 27 Jerah, Hadoram, Uzal,
\par 28 Diklah, Obal, Abimael, Sceba,
\par 29 Ofir, Havila e Jobab. Tutti questi furono figliuoli di Jokthan.
\par 30 E la loro dimora fu la montagna orientale, da Mesha, fin verso Sefar.
\par 31 Questi sono i figliuoli di Sem, secondo le loro famiglie, secondo le loro lingue, nei loro paesi, secondo le loro nazioni.
\par 32 Queste sono le famiglie dei figliuoli di Noè, secondo le loro generazioni, nelle loro nazioni; e da essi uscirono le nazioni che si sparsero per la terra dopo il diluvio.

\chapter{11}

\par 1 Or tutta la terra parlava la stessa lingua e usava le stesse parole.
\par 2 E avvenne che, essendo partiti verso l'Oriente, gli uomini trovarono una pianura nel paese di Scinear, e quivi si stanziarono.
\par 3 E dissero l'uno all'altro: 'Orsù, facciamo dei mattoni e cociamoli col fuoco!' E si valsero di mattoni invece di pietre, e di bitume invece di calcina.
\par 4 E dissero: 'Orsù, edifichiamoci una città e una torre di cui la cima giunga fino al cielo, e acquistiamoci fama, onde non siamo dispersi sulla faccia di tutta la terra'.
\par 5 E l'Eterno discese per vedere la città e la torre che i figliuoli degli uomini edificavano.
\par 6 E l'Eterno disse: 'Ecco, essi sono un solo popolo e hanno tutti il medesimo linguaggio; e questo è il principio del loro lavoro; ora nulla li impedirà di condurre a termine ciò che disegnano di fare.
\par 7 Orsù, scendiamo e confondiamo quivi il loro linguaggio, sicché l'uno non capisca il parlare dell'altro!'
\par 8 Così l'Eterno li disperse di là sulla faccia di tutta la terra, ed essi cessarono di edificare la città.
\par 9 Perciò a questa fu dato il nome di Babel perché l'Eterno confuse quivi il linguaggio di tutta la terra, e di là l'Eterno li disperse sulla faccia di tutta la terra.
\par 10 Questa è la posterità di Sem. Sem, all'età di cent'anni, generò Arpacshad, due anni dopo il diluvio.
\par 11 E Sem, dopo ch'ebbe generato Arpacshad, visse cinquecent'anni e generò figliuoli e figliuole.
\par 12 Arpacshad visse trentacinque anni e generò Scelah; e Arpacshad, dopo aver generato Scelah,
\par 13 visse quattrocento anni e generò figliuoli e figliuole.
\par 14 Scelah visse trent'anni e generò Eber;
\par 15 e Scelah, dopo aver generato Eber, visse quattrocentotre anni e generò figliuoli e figliuole.
\par 16 Eber visse trentaquattro anni e generò Peleg;
\par 17 ed Eber, dopo aver generato Peleg, visse quattrocentotrent'anni e generò figliuoli e figliuole.
\par 18 Peleg visse trent'anni e generò Reu;
\par 19 e Peleg, dopo aver generato Reu, visse duecentonove anni e generò figliuoli e figliuole.
\par 20 Reu visse trentadue anni e generò Serug;
\par 21 e Reu, dopo aver generato Serug, visse duecentosette anni e generò figliuoli e figliuole.
\par 22 Serug visse trent'anni e generò Nahor;
\par 23 e Serug, dopo aver generato Nahor, visse duecento anni e generò figliuoli e figliuole.
\par 24 Nahor visse ventinove anni e generò Terah;
\par 25 e Nahor, dopo aver generato Terah, visse centodiciannove anni e generò figliuoli e figliuole.
\par 26 Terah visse settant'anni e generò Abramo, Nahor e Haran.
\par 27 E questa è la posterità di Terah. Terah generò Abramo, Nahor e Haran; e Haran generò Lot.
\par 28 Haran morì in presenza di Terah suo padre, nel suo paese nativo, in Ur de' Caldei.
\par 29 E Abramo e Nahor si presero delle mogli: il nome della moglie d'Abramo era Sarai; e il nome della moglie di Nahor, Milca, ch'era figliuola di Haran, padre di Milca e padre di Isca.
\par 30 E Sarai era sterile; non aveva figliuoli.
\par 31 E Terah prese Abramo, suo figliuolo, e Lot, figliuolo di Haran, cioè figliuolo del suo figliuolo e Sarai sua nuora, moglie d'Abramo suo figliuolo, e uscirono insieme da Ur de' Caldei per andare nel paese di Canaan; e, giunti a Charan, dimorarono quivi.
\par 32 E il tempo che Terah visse fu duecentocinque anni; poi Terah morì in Charan.

\chapter{12}

\par 1 Or l'Eterno disse ad Abramo: 'Vattene dal tuo paese e dal tuo parentado e dalla casa di tuo padre, nel paese che io ti mostrerò;
\par 2 e io farò di te una grande nazione e ti benedirò e renderò grande il tuo nome e tu sarai fonte di benedizione:
\par 3 e benedirò quelli che ti benediranno e maledirò chi ti maledirà e in te saranno benedette tutte le famiglie della terra'.
\par 4 E Abramo se ne andò, come l'Eterno gli avea detto, e Lot andò con lui. Abramo aveva settantacinque anni quando partì da Charan.
\par 5 E Abramo prese Sarai sua moglie e Lot, figliuolo del suo fratello, e tutti i beni che possedevano e le persone che aveano acquistate in Charan, e partirono per andarsene nel paese di Canaan; e giunsero nel paese di Canaan.
\par 6 E Abramo traversò il paese fino al luogo di Sichem, fino alla quercia di Moreh. Or in quel tempo i Cananei erano nel paese.
\par 7 E l'Eterno apparve ad Abramo e disse: 'Io darò questo paese alla tua progenie'. Ed egli edificò quivi un altare all'Eterno che gli era apparso.
\par 8 E di là si trasportò verso la montagna a oriente di Bethel, e piantò le sue tende, avendo Bethel a occidente e Ai ad oriente; e quivi edificò un altare all'Eterno e invocò il nome dell'Eterno.
\par 9 Poi Abramo si partì, proseguendo da un accampamento all'altro, verso mezzogiorno.
\par 10 Or venne nel paese una carestia; e Abramo scese in Egitto per soggiornarvi, perché la fame era grave nel paese.
\par 11 E come stava per entrare in Egitto, disse a Sarai sua moglie: 'Ecco, io so che tu sei una donna di bell'aspetto;
\par 12 e avverrà che quando gli Egiziani t'avranno veduta, diranno: Ella è sua moglie; e uccideranno me, ma a te lasceranno la vita.
\par 13 Deh, di' che sei mia sorella, perché io sia trattato bene a motivo di te, e la vita mia sia conservata per amor tuo'.
\par 14 E avvenne che quando Abramo fu giunto in Egitto, gli Egiziani osservarono che la donna era molto bella.
\par 15 E i principi di Faraone la videro e la lodarono dinanzi a Faraone; e la donna fu menata in casa di Faraone.
\par 16 Ed egli fece del bene ad Abramo per amor di lei; ed Abramo ebbe pecore e buoi e asini e servi e serve e asine e cammelli.
\par 17 Ma l'Eterno colpì Faraone e la sua casa con grandi piaghe, a motivo di Sarai, moglie di Abramo.
\par 18 Allora Faraone chiamò Abramo e disse: 'Che m'hai tu fatto? perché non m'hai detto ch'era tua moglie? perché hai detto:
\par 19 È mia sorella? ond'io me la son presa per moglie. Or dunque eccoti la tua moglie; prenditela e vattene!'
\par 20 E Faraone diede alla sua gente ordini relativi ad Abramo, ed essi fecero partire lui, sua moglie, e tutto quello ch'ei possedeva.

\chapter{13}

\par 1 Abramo dunque risalì dall'Egitto con sua moglie, con tutto quel che possedeva e con Lot, andando verso il mezzogiorno di Canaan.
\par 2 Abramo era molto ricco di bestiame, d'argento e d'oro.
\par 3 E continuò il suo viaggio dal mezzogiorno fino a Bethel, al luogo ove da principio era stata la sua tenda, fra Bethel ed Ai,
\par 4 al luogo dov'era l'altare ch'egli avea fatto da prima; e quivi Abramo invocò il nome dell'Eterno.
\par 5 Or Lot, che viaggiava con Abramo, aveva anch'egli pecore, buoi e tende.
\par 6 E il paese non era sufficiente perch'essi potessero abitarvi assieme; poiché le loro facoltà erano grandi ed essi non potevano stare assieme.
\par 7 E nacque una contesa fra i pastori del bestiame d'Abramo e i pastori del bestiame di Lot. I Cananei e i Ferezei abitavano a quel tempo nel paese.
\par 8 E Abramo disse a Lot: 'Deh, non ci sia contesa fra me e te, né fra i miei pastori e i tuoi pastori, poiché siam fratelli!
\par 9 Tutto il paese non sta esso davanti a te? Deh, sepàrati da me! Se tu vai a sinistra, io andrò a destra; e se tu vai a destra, io andrò a sinistra'.
\par 10 E Lot alzò gli occhi e vide l'intera pianura del Giordano. Prima che l'Eterno avesse distrutto Sodoma e Gomorra, essa era tutta quanta irrigata fino a Tsoar, come il giardino dell'Eterno, come il paese d'Egitto.
\par 11 E Lot si scelse tutta la pianura del Giordano, e partì andando verso oriente. Così si separarono l'uno dall'altro.
\par 12 Abramo dimorò nel paese di Canaan, e Lot abitò nelle città della pianura e andò piantando le sue tende fino a Sodoma.
\par 13 Ora la gente di Sodoma era scellerata e oltre modo peccatrice contro l'Eterno.
\par 14 E l'Eterno disse ad Abramo, dopo che Lot si fu separato da lui: 'Alza ora gli occhi tuoi e mira, dal luogo dove sei, a settentrione, a mezzogiorno, a oriente, a occidente.
\par 15 Tutto il paese che vedi, lo darò a te e alla tua progenie, in perpetuo.
\par 16 E farò sì che la tua progenie sarà come la polvere della terra; in guisa che, se alcuno può contare la polvere della terra, anche la tua progenie si potrà contare.
\par 17 Lèvati, percorri il paese quant'è lungo e quant'è largo, poiché io te lo darò'.
\par 18 Allora Abramo levò le sue tende, e venne ad abitare alle querce di Mamre, che sono a Hebron; e quivi edificò un altare all'Eterno.

\chapter{14}

\par 1 Or avvenne, al tempo di Amrafel re di Scinear, d'Arioc re di Ellasar, di Kedorlaomer re di Elam, e di Tideal re dei Goim,
\par 2 ch'essi mossero guerra a Bera re di Sodoma, a Birsha re di Gomorra, a Scinear re di Adma, a Scemeber re di Tseboim e al re di Bela, che è Tsoar.
\par 3 Tutti questi ultimi si radunarono nella valle di Siddim, ch'è il Mar salato.
\par 4 Per dodici anni erano stati soggetti a Kedorlaomer, e al tredicesimo anno si erano ribellati.
\par 5 E nell'anno quattordicesimo, Kedorlaomer e i re ch'erano con lui vennero e sbaragliarono i Refei ad Ashteroth-Karnaim, gli Zuzei a Ham, gli Emei nella pianura di Kiriathaim
\par 6 e gli Horei nella loro montagna di Seir fino a El-Paran, che è presso al deserto.
\par 7 Poi tornarono indietro e vennero a En-Mishpat, che è Kades, e sbaragliarono gli Amalekiti su tutto il loro territorio, e così pure gli Amorei, che abitavano ad Hatsatson-Tamar.
\par 8 Allora il re di Sodoma, il re di Gomorra, il re di Adma, il re di Tseboim e il re di Bela, che è Tsoar, uscirono e si schierarono in battaglia contro quelli, nella valle di Siddim:
\par 9 contro Kedorlaomer re di Elam, Tideal re dei Goim, Amrafel re di Scinear e Arioc re di Ellasar: quattro re contro cinque.
\par 10 Or la valle di Siddim era piena di pozzi di bitume; e i re di Sodoma e di Gomorra si dettero alla fuga e vi caddero dentro; quelli che scamparono fuggirono al monte.
\par 11 E i vincitori presero tutte le ricchezze di Sodoma e di Gomorra, e tutti i loro viveri, e se ne andarono.
\par 12 Presero anche Lot, figliuolo del fratello di Abramo, con la sua roba; e se ne andarono. Lot abitava in Sodoma.
\par 13 E uno degli scampati venne a dirlo ad Abramo, l'Ebreo, che abitava alle querce di Mamre l'Amoreo, fratello di Eshcol e fratello di Aner, i quali aveano fatto alleanza con Abramo.
\par 14 E Abramo, com'ebbe udito che il suo fratello era stato fatto prigioniero, armò trecentodiciotto de' suoi più fidati servitori, nati in casa sua, ed inseguì i re fino a Dan.
\par 15 E, divisa la sua schiera per assalirli di notte, egli coi suoi servi li sconfisse e l'inseguì fino a Hobah, che è a sinistra di Damasco.
\par 16 E ricuperò tutta la roba, e rimenò pure Lot suo fratello, la sua roba, e anche le donne e il popolo.
\par 17 E com'egli se ne tornava dalla sconfitta di Kedorlaomer e dei re ch'eran con lui, il re di Sodoma gli andò incontro nella valle di Shaveh, che è la Valle del re.
\par 18 E Melchisedec, re di Salem, fece portar del pane e del vino. Egli era sacerdote dell'Iddio altissimo.
\par 19 Ed egli benedisse Abramo, dicendo: 'Benedetto sia Abramo dall'Iddio altissimo, padrone de' cieli e della terra!
\par 20 E benedetto sia l'Iddio altissimo, che t'ha dato in mano i tuoi nemici!' E Abramo gli diede la decima d'ogni cosa.
\par 21 E il re di Sodoma disse ad Abramo: 'Dammi le persone, e prendi per te la roba'.
\par 22 Ma Abramo rispose al re di Sodoma: 'Ho alzato la mia mano all'Eterno, l'Iddio altissimo, padrone dei cieli e della terra,
\par 23 giurando che non prenderei neppure un filo, né un laccio di sandalo, di tutto ciò che t'appartiene; perché tu non abbia a dire: Io ho arricchito Abramo.
\par 24 Nulla per me! tranne quello che hanno mangiato i giovani, e la parte che spetta agli uomini che son venuti meco: Aner, Eshcol e Mamre; essi prendano la loro parte'.

\chapter{15}

\par 1 Dopo queste cose, la parola dell'Eterno fu rivolta in visione ad Abramo, dicendo: 'Non temere, o Abramo, io sono il tuo scudo, e la tua ricompensa sarà grandissima'.
\par 2 E Abramo disse: 'Signore, Eterno, che mi darai tu? poiché io me ne vo senza figliuoli, e chi possederà la mia casa è Eliezer di Damasco'.
\par 3 E Abramo soggiunse: 'Tu non m'hai dato progenie; ed ecco, uno schiavo nato in casa mia sarà mio erede'.
\par 4 Allora la parola dell'Eterno gli fu rivolta dicendo: 'Questi non sarà tuo erede; ma colui che uscirà dalle tue viscere sarà erede tuo'.
\par 5 E lo menò fuori, e gli disse: 'Mira il cielo, e conta le stelle, se le puoi contare'. E gli disse: 'Così sarà la tua progenie'.
\par 6 Ed egli credette all'Eterno, che gli contò questo come giustizia.
\par 7 E l'Eterno gli disse: 'Io sono l'Eterno che t'ho fatto uscire da Ur de' Caldei per darti questo paese, perché tu lo possegga'.
\par 8 E Abramo chiese: 'Signore, Eterno, da che posso io conoscere che lo possederò?'
\par 9 E l'Eterno gli rispose: 'Pigliami una giovenca di tre anni, una capra di tre anni, un montone di tre anni, una tortora e un piccione'.
\par 10 Ed egli prese tutti questi animali, li divise per mezzo, e pose ciascuna metà dirimpetto all'altra; ma non divise gli uccelli.
\par 11 Or degli uccelli rapaci calarono sulle bestie morte, ma Abramo li scacciò.
\par 12 E, sul tramontare del sole, un profondo sonno cadde sopra Abramo; ed ecco, uno spavento, una oscurità profonda, cadde su lui.
\par 13 E l'Eterno disse ad Abramo: 'Sappi per certo che i tuoi discendenti dimoreranno come stranieri in un paese che non sarà loro, e vi saranno schiavi, e saranno oppressi per quattrocento anni;
\par 14 ma io giudicherò la gente di cui saranno stati servi; e, dopo questo, se ne partiranno con grandi ricchezze.
\par 15 E tu te n'andrai in pace ai tuoi padri, e sarai sepolto dopo una prospera vecchiezza.
\par 16 E alla quarta generazione essi torneranno qua; perché l'iniquità degli Amorei non è giunta finora al colmo'.
\par 17 Or come il sole si fu coricato e venne la notte scura, ecco una fornace fumante ed una fiamma di fuoco passare in mezzo agli animali divisi.
\par 18 In quel giorno l'Eterno fece patto con Abramo, dicendo: 'Io do alla tua progenie questo paese, dal fiume d'Egitto al gran fiume, il fiume Eufrate;
\par 19 i Kenei, i Kenizei, i Kadmonei,
\par 20 gli Hittei, i Ferezei, i Refei,
\par 21 gli Amorei, i Cananei, i Ghirgasei e i Gebusei'.

\chapter{16}

\par 1 Or Sarai, moglie d'Abramo, non gli aveva dato figliuoli. Essa aveva una serva egiziana per nome Agar.
\par 2 E Sarai disse ad Abramo: 'Ecco, l'Eterno m'ha fatta sterile; deh, va' dalla mia serva; forse avrò progenie da lei'. E Abramo dette ascolto alla voce di Sarai.
\par 3 Sarai dunque, moglie d'Abramo, dopo che Abramo ebbe dimorato dieci anni nel paese di Canaan, prese la sua serva Agar, l'Egiziana, e la diede per moglie ad Abramo, suo marito.
\par 4 Ed egli andò da Agar, che rimase incinta; e quando s'accorse ch'era incinta, guardò la sua padrona con disprezzo.
\par 5 E Sarai disse ad Abramo: 'L'ingiuria fatta a me, ricade su te. Io t'ho dato la mia serva in seno; e da che ella s'è accorta ch'era incinta, mi guarda con disprezzo. L'Eterno sia giudice fra me e te'.
\par 6 E Abramo rispose a Sarai: 'Ecco, la tua serva è in tuo potere; fa' con lei come ti piacerà'. Sarai la trattò duramente, ed ella se ne fuggì da lei.
\par 7 E l'angelo dell'Eterno la trovò presso una sorgente d'acqua, nel deserto, presso la sorgente ch'è sulla via di Shur,
\par 8 e le disse: 'Agar, serva di Sarai, donde vieni? e dove vai?' Ed ella rispose: 'Me ne fuggo dal cospetto di Sarai mia padrona'.
\par 9 E l'angelo dell'Eterno le disse: 'Torna alla tua padrona, e umiliati sotto la sua mano'.
\par 10 L'angelo dell'Eterno soggiunse: 'Io moltiplicherò grandemente la tua progenie, e non la si potrà contare, tanto sarà numerosa'.
\par 11 E l'angelo dell'Eterno le disse ancora: 'Ecco, tu sei incinta e partorirai un figliuolo, al quale porrai nome Ismaele, perché l'Eterno t'ha ascoltata nella tua afflizione;
\par 12 esso sarà tra gli uomini come un asino selvatico; la sua mano sarà contro tutti, e la mano di tutti contro di lui; e abiterà in faccia a tutti i suoi fratelli'.
\par 13 Allora Agar chiamò il nome dell'Eterno che le avea parlato, Atta-El-Roï, perché disse: 'Ho io, proprio qui, veduto andarsene colui che m'ha vista?'
\par 14 Perciò quel pozzo fu chiamato 'il pozzo di Lachai-Roï'. Ecco, esso è fra Kades e Bered.
\par 15 E Agar partorì un figliuolo ad Abramo; e Abramo, al figliuolo che Agar gli avea partorito, pose nome Ismaele.
\par 16 Abramo aveva ottantasei anni quando Agar gli partorì Ismaele.

\chapter{17}

\par 1 Quando Abramo fu d'età di novantanove anni, l'Eterno gli apparve e gli disse: 'Io sono l'Iddio onnipotente; cammina alla mia presenza e sii integro;
\par 2 e io fermerò il mio patto fra me e te, e ti moltiplicherò grandissimamente'.
\par 3 Allora Abramo si prostrò con la faccia in terra, e Dio gli parlò, dicendo:
\par 4 'Quanto a me, ecco il patto che fo con te; tu diverrai padre di una moltitudine di nazioni;
\par 5 e non sarai più chiamato Abramo, ma il tuo nome sarà Abrahamo, poiché io ti costituisco padre di una moltitudine di nazioni.
\par 6 E ti farò moltiplicare grandissimamente, e ti farò divenir nazioni, e da te usciranno dei re.
\par 7 E fermerò il mio patto fra me e te e i tuoi discendenti dopo di te, di generazione in generazione; sarà un patto perpetuo, per il quale io sarò l'Iddio tuo e della tua progenie dopo di te.
\par 8 E a te e alla tua progenie dopo di te darò il paese dove abiti come straniero: tutto il paese di Canaan, in possesso perpetuo; e sarò loro Dio'.
\par 9 Poi Dio disse ad Abrahamo: 'Quanto a te, tu osserverai il mio patto: tu e la tua progenie dopo di te, di generazione in generazione.
\par 10 Questo è il mio patto che voi osserverete, patto fra me e voi e la tua progenie dopo di te: ogni maschio fra voi sia circonciso.
\par 11 E sarete circoncisi; e questo sarà un segno del patto fra me e voi.
\par 12 All'età d'otto giorni, ogni maschio sarà circonciso fra voi, di generazione in generazione: tanto quello nato in casa, quanto quello comprato con danaro da qualsivoglia straniero e che non sia della tua progenie.
\par 13 Quello nato in casa tua e quello comprato con danaro dovrà essere circonciso; e il mio patto nella vostra carne sarà un patto perpetuo.
\par 14 E il maschio incirconciso, che non sarà stato circonciso nella sua carne, sarà reciso di fra il suo popolo: egli avrà violato il mio patto'.
\par 15 E Dio disse ad Abrahamo: 'Quanto a Sarai tua moglie, non la chiamar più Sarai; il suo nome sarà, invece Sara.
\par 16 E io la benedirò, ed anche ti darò di lei un figliuolo; io la benedirò, ed essa diverrà nazioni; re di popoli usciranno da lei'.
\par 17 Allora Abrahamo si prostrò con la faccia in terra e rise; e disse in cuor suo: 'Nascerà egli un figliuolo a un uomo di cent'anni? e Sara, che ha novant'anni, partorirà ella?'
\par 18 E Abrahamo disse a Dio: 'Di grazia, viva Ismaele nel tuo cospetto!'
\par 19 E Dio rispose: 'No, ma Sara tua moglie ti partorirà un figliuolo, e tu gli porrai nome Isacco; e io fermerò il mio patto con lui, un patto perpetuo per la sua progenie dopo di lui.
\par 20 Quanto a Ismaele, io t'ho esaudito. Ecco, io l'ho benedetto, e farò che moltiplichi e s'accresca grandissimamente. Egli genererà dodici principi, e io farò di lui una grande nazione.
\par 21 Ma fermerò il mio patto con Isacco che Sara ti partorirà in questo tempo, l'anno venturo'.
\par 22 E quand'ebbe finito di parlare con lui, Iddio lasciò Abrahamo, levandosi in alto.
\par 23 E Abrahamo prese Ismaele suo figliuolo e tutti quelli che gli erano nati in casa e tutti quelli che avea comprato col suo danaro, tutti i maschi fra la gente della casa d'Abrahamo, e li circoncise, in quello stesso giorno, come Dio gli avea detto di fare.
\par 24 Or Abrahamo aveva novantanove anni quando fu circonciso.
\par 25 E Ismaele suo figliuolo aveva tredici anni quando fu circonciso.
\par 26 In quel medesimo giorno fu circonciso Abrahamo, e Ismaele suo figliuolo.
\par 27 E tutti gli uomini della sua casa, tanto quelli nati in casa quanto quelli comprati con danaro dagli stranieri, furono circoncisi con lui.

\chapter{18}

\par 1 L'Eterno apparve ad Abrahamo alle querce di Mamre, mentre questi sedeva all'ingresso della sua tenda durante il caldo del giorno.
\par 2 Abrahamo alzò gli occhi, ed ecco che scòrse tre uomini, i quali stavano dinanzi a lui; e come li ebbe veduti, corse loro incontro dall'ingresso della tenda, si prostrò fino a terra e disse:
\par 3 'Deh, Signor mio, se ho trovato grazia davanti a te, non passare senza fermarti dal tuo servo!
\par 4 Deh, lasciate che si porti un po' d'acqua; e lavatevi i piedi; e riposatevi sotto quest'albero.
\par 5 Io andrò a prendere un pezzo di pane, e vi fortificherete il cuore; poi, continuerete il vostro cammino; poiché per questo siete passati presso al vostro servo'. E quelli dissero: 'Fa' come hai detto'.
\par 6 Allora Abrahamo andò in fretta nella tenda da Sara, e le disse: 'Prendi subito tre misure di fior di farina, impastala, e fa' delle schiacciate'.
\par 7 Poi Abrahamo corse all'armento, ne tolse un vitello tenero e buono, e lo diede a un servo, il quale s'affrettò a prepararlo.
\par 8 E prese del burro, del latte e il vitello ch'era stato preparato, e li pose davanti a loro; ed egli se ne stette in piè presso di loro sotto l'albero. E quelli mangiarono.
\par 9 Poi essi gli dissero: 'Dov'è Sara tua moglie?' Ed egli rispose: 'È là nella tenda'.
\par 10 E l'altro: 'Tornerò certamente da te fra un anno; ed ecco, Sara tua moglie avrà un figliuolo'. E Sara ascoltava all'ingresso della tenda, ch'era dietro a lui.
\par 11 Or Abrahamo e Sara eran vecchi, bene avanti negli anni, e Sara non aveva più i corsi ordinari delle donne.
\par 12 E Sara rise dentro di sé, dicendo: 'Vecchia come sono, avrei io tali piaceri? e anche il mio signore è vecchio!'
\par 13 E l'Eterno disse ad Abrahamo: 'Perché mai ha riso Sara, dicendo: Partorirei io per davvero, vecchia come sono?
\par 14 V'ha egli cosa che sia troppo difficile per l'Eterno? Al tempo fissato, fra un anno, tornerò, e Sara avrà un figliuolo'.
\par 15 Allora Sara negò, dicendo: 'Non ho riso'; perch'ebbe paura. Ma egli disse: 'Invece, hai riso!'
\par 16 Poi quegli uomini s'alzarono e volsero gli sguardi verso Sodoma; e Abrahamo andava con loro per accomiatarli.
\par 17 E l'Eterno disse: 'Celerò io ad Abrahamo quello che sto per fare,
\par 18 giacché Abrahamo deve diventare una nazione grande e potente e in lui saran benedette tutte le nazioni della terra?
\par 19 Poiché io l'ho prescelto affinché ordini ai suoi figliuoli, e dopo di sé alla sua casa, che s'attengano alla via dell'Eterno per praticare la giustizia e l'equità, onde l'Eterno ponga ad effetto a pro d'Abrahamo quello che gli ha promesso'.
\par 20 E l'Eterno disse: 'Siccome il grido che sale da Sodoma e Gomorra è grande e siccome il loro peccato è molto grave,
\par 21 io scenderò e vedrò se hanno interamente agito secondo il grido che n'è pervenuto a me; e, se così non è, lo saprò'.
\par 22 E quegli uomini, partitisi di là, s'avviarono verso Sodoma; ma Abrahamo rimase ancora davanti all'Eterno.
\par 23 E Abrahamo s'accostò e disse: 'Farai tu perire il giusto insieme con l'empio?
\par 24 Forse ci son cinquanta giusti nella città; farai tu perire anche quelli? o non perdonerai tu a quel luogo per amore de' cinquanta giusti che vi sono?
\par 25 Lungi da te il fare tal cosa! il far morire il giusto con l'empio, in guisa che il giusto sia trattato come l'empio! lungi da te! Il giudice di tutta la terra non farà egli giustizia?'
\par 26 E l'Eterno disse: 'Se trovo nella città di Sodoma cinquanta giusti, perdonerò a tutto il luogo per amor d'essi'.
\par 27 E Abrahamo riprese e disse: 'Ecco, prendo l'ardire di parlare al Signore, benché io non sia che polvere e cenere;
\par 28 forse, a que' cinquanta giusti ne mancheranno cinque; distruggerai tu tutta la città per cinque di meno?' E l'Eterno: 'Se ve ne trovo quarantacinque, non la distruggerò'.
\par 29 Abrahamo continuò a parlargli e disse: 'Forse, vi se ne troveranno quaranta'. E l'Eterno: 'Non lo farò, per amor dei quaranta'.
\par 30 E Abrahamo disse: 'Deh, non si adiri il Signore, ed io parlerò. Forse, vi se ne troveranno trenta'. E l'Eterno: 'Non lo farò, se ve ne trovo trenta'.
\par 31 E Abrahamo disse: 'Ecco, prendo l'ardire di parlare al Signore; forse, vi se ne troveranno venti'. E l'Eterno: 'Non la distruggerò per amore dei venti'.
\par 32 E Abrahamo disse: 'Deh, non si adiri il Signore, e io parlerò ancora questa volta soltanto. Forse, vi se ne troveranno dieci'. E l'Eterno: 'Non la distruggerò per amore dei dieci'.
\par 33 E come l'Eterno ebbe finito di parlare ad Abrahamo, se ne andò. E Abrahamo tornò alla sua dimora.

\chapter{19}

\par 1 Or i due angeli giunsero a Sodoma verso sera; e Lot stava sedendo alla porta di Sodoma; e, come li vide, s'alzò per andar loro incontro e si prostrò con la faccia a terra, e disse:
\par 2 'Signori miei, vi prego, venite in casa del vostro servo, albergatevi questa notte, e lavatevi i piedi; poi domattina vi leverete per tempo e continuerete il vostro cammino'. Ed essi risposero: 'No; passeremo la notte sulla piazza'.
\par 3 Ma egli fe' loro tanta premura, che vennero da lui ed entrarono in casa sua. Ed egli fece loro un convito, cosse dei pani senza lievito, ed essi mangiarono.
\par 4 Ma prima che si fossero coricati, gli uomini della città, i Sodomiti, circondarono la casa: giovani e vecchi, la popolazione intera venuta da ogni lato; e chiamarono Lot, e gli dissero:
\par 5 'Dove sono quegli uomini che son venuti da te stanotte? Menaceli fuori, affinché noi li conosciamo!'
\par 6 Lot uscì verso di loro sull'ingresso di casa, si chiuse dietro la porta, e disse:
\par 7 'Deh, fratelli miei, non fate questo male!
\par 8 Ecco, ho due figliuole che non hanno conosciuto uomo; deh, lasciate ch'io ve le meni fuori, e voi fate di loro quel che vi piacerà; soltanto non fate nulla a questi uomini, poiché son venuti all'ombra del mio tetto'.
\par 9 Ma essi gli dissero: 'Fatti in là!' E ancora: 'Quest'individuo è venuto qua come straniero, e la vuol far da giudice! Ora faremo a te peggio che a quelli!' E, premendo Lot con violenza, s'avvicinarono per sfondare la porta.
\par 10 Ma quegli uomini stesero la mano, trassero Lot in casa con loro, e chiusero la porta.
\par 11 E colpirono di cecità la gente ch'era alla porta della casa, dal più piccolo al più grande, talché si stancarono a cercar la porta.
\par 12 E quegli uomini dissero a Lot: 'Chi hai tu ancora qui? fa' uscire da questo luogo generi, figliuoli, figliuole e chiunque de' tuoi è in questa città;
\par 13 poiché noi distruggeremo questo luogo, perché il grido contro i suoi abitanti è grande nel cospetto dell'Eterno, e l'Eterno ci ha mandati a distruggerlo'.
\par 14 Allora Lot uscì, parlò ai suoi generi che avevano preso le sue figliuole, e disse: 'Levatevi, uscite da questo luogo, perché l'Eterno sta per distruggere la città'. Ma ai generi parve che volesse scherzare.
\par 15 E come l'alba cominciò ad apparire, gli angeli sollecitarono Lot, dicendo: 'Lèvati, prendi tua moglie e le tue due figliuole che si trovan qui, affinché tu non perisca nel castigo di questa città'.
\par 16 Ma egli s'indugiava; e quegli uomini presero per la mano lui, sua moglie e le sue due figliuole, perché l'Eterno lo volea risparmiare; e lo menaron via, e lo misero fuori della città.
\par 17 E avvenne che quando li ebbero fatti uscire, uno di quegli uomini disse: 'Sàlvati la vita! non guardare indietro, e non ti fermare in alcun luogo della pianura; sàlvati al monte, che tu non abbia a perire!'
\par 18 E Lot rispose loro: 'No, mio signore!
\par 19 ecco, il tuo servo ha trovato grazia agli occhi tuoi, e tu hai mostrato la grandezza della tua bontà verso di me conservandomi in vita; ma io non posso salvarmi al monte prima che il disastro mi sopraggiunga, ed io perisca.
\par 20 Ecco, questa città è vicina da potermici rifugiare, ed è piccola. Deh, lascia ch'io scampi quivi - non è essa piccola? - e vivrà l'anima mia!'
\par 21 E quegli a lui: 'Ecco, anche questa grazia io ti concedo: di non distruggere la città, della quale hai parlato.
\par 22 Affréttati, scampa colà, poiché io non posso far nulla finché tu vi sia giunto'. Perciò quella città fu chiamata Tsoar.
\par 23 Il sole si levava sulla terra quando Lot arrivò a Tsoar.
\par 24 Allora l'Eterno fece piovere dai cieli su Sodoma e Gomorra zolfo e fuoco, da parte dell'Eterno;
\par 25 ed egli distrusse quelle città e tutta la pianura e tutti gli abitanti delle città e quanto cresceva sul suolo.
\par 26 Ma la moglie di Lot si volse a guardare indietro, e diventò una statua di sale.
\par 27 E Abrahamo si levò la mattina a buon'ora, e andò al luogo dove s'era prima fermato davanti all'Eterno;
\par 28 guardò verso Sodoma e Gomorra e verso tutta la regione della pianura, ed ecco vide un fumo che si levava dalla terra, come il fumo d'una fornace.
\par 29 Così avvenne che, quando Iddio distrusse le città della pianura, egli si ricordò d'Abrahamo, e fece partir Lot di mezzo al disastro, allorché sovvertì le città dove Lot avea dimorato.
\par 30 Lot salì da Tsoar e dimorò sul monte insieme con le sue due figliuole, perché temeva di stare in Tsoar; e dimorò in una spelonca, egli con le sue due figliuole.
\par 31 E la maggiore disse alla minore: 'Nostro padre è vecchio, e non c'è più nessuno sulla terra per venire da noi, come si costuma in tutta la terra.
\par 32 Vieni, diamo a bere del vino a nostro padre, e giaciamoci con lui, affinché possiamo conservare la razza di nostro padre'.
\par 33 E quella stessa notte dettero a bere del vino al loro padre; e la maggiore entrò e si giacque con suo padre; ed egli non s'accorse né quando essa si coricò né quando si levò.
\par 34 E avvenne che il dì seguente, la maggiore disse alla minore: 'Ecco, la notte passata io mi giacqui con mio padre; diamogli a bere del vino anche questa notte; e tu entra, e giaciti con lui, affinché possiamo conservare la razza di nostro padre'.
\par 35 E anche quella notte dettero a bere del vino al padre loro, e la minore andò a giacersi con lui; ed egli non s'accorse né quando essa si coricò né quando si levò.
\par 36 Così le due figliuole di Lot rimasero incinte del loro padre.
\par 37 E la maggiore partorì un figliuolo, al quale pose nome Moab. Questi è il padre dei Moabiti, che sussistono fino al dì d'oggi.
\par 38 E la minore partorì anch'essa un figliuolo, al quale pose nome Ben-Ammi. Questi è il padre degli Ammoniti, che sussistono fino al dì d'oggi.

\chapter{20}

\par 1 Abrahamo si partì di là andando verso il paese del mezzodì, dimorò fra Kades e Shur, e abitò come forestiero in Gherar.
\par 2 E Abrahamo diceva di Sara sua moglie: 'Ell'è mia sorella'. E Abimelec, re di Gherar, mandò a pigliar Sara.
\par 3 Ma Dio venne, di notte, in un sogno, ad Abimelec, e gli disse: 'Ecco, tu sei morto, a motivo della donna che ti sei presa; perch'ella ha marito'.
\par 4 Or Abimelec non s'era accostato a lei; e rispose: 'Signore, faresti tu perire una nazione, anche se giusta?
\par 5 Non m'ha egli detto: È mia sorella? e anche lei stessa ha detto: Egli è mio fratello. Io ho fatto questo nella integrità del mio cuore e con mani innocenti'.
\par 6 E Dio gli disse nel sogno: 'Anch'io so che tu hai fatto questo nella integrità del tuo cuore; e t'ho quindi preservato dal peccare contro di me; perciò non ti ho permesso di toccarla.
\par 7 Or dunque, restituisci la moglie a quest'uomo, perché è profeta; ed egli pregherà per te, e tu vivrai. Ma, se non la restituisci, sappi che, per certo, morrai: tu e tutti i tuoi'.
\par 8 E Abimelec si levò la mattina per tempo, chiamò tutti i suoi servi, e raccontò in loro presenza tutte queste cose. E quegli uomini furon presi da gran paura.
\par 9 Poi Abimelec chiamò Abrahamo e gli disse: 'Che ci hai tu fatto? E in che t'ho io offeso, che tu abbia fatto venir su me e sul mio regno un sì gran peccato? Tu m'hai fatto cose che non si debbono fare'.
\par 10 E di nuovo Abimelec disse ad Abrahamo: 'A che miravi, facendo questo?'
\par 11 E Abrahamo rispose: 'L'ho fatto, perché dicevo fra me: Certo, in questo luogo non c'è timor di Dio; e m'uccideranno a causa di mia moglie.
\par 12 Inoltre, ella è proprio mia sorella, figliuola di mio padre, ma non figliuola di mia madre; ed è diventata mia moglie.
\par 13 Or quando Iddio mi fece errare lungi dalla casa di mio padre, io le dissi: Questo è il favore che tu mi farai; dovunque giungeremo, dirai di me: È mio fratello'.
\par 14 E Abimelec prese delle pecore, dei buoi, de' servi e delle serve, e li diede ad Abrahamo, e gli restituì Sara sua moglie. E Abimelec disse:
\par 15 'Ecco, il mio paese ti sta dinanzi; dimora dovunque ti piacerà'. E a Sara disse:
\par 16 'Ecco, io ho dato a tuo fratello mille pezzi d'argento; questo ti sarà un velo sugli occhi di fronte a tutti quelli che sono teco, e sarai giustificata dinanzi a tutti'.
\par 17 E Abrahamo pregò Dio, e Dio guarì Abimelec, la moglie e le serve di lui, ed esse poteron partorire.
\par 18 Poiché l'Eterno avea del tutto resa sterile l'intera casa di Abimelec, a motivo di Sara moglie di Abrahamo.

\chapter{21}

\par 1 L'Eterno visitò Sara come avea detto; e l'Eterno fece a Sara come aveva annunziato.
\par 2 E Sara concepì e partorì un figliuolo ad Abrahamo, quand'egli era vecchio, al tempo che Dio gli aveva fissato.
\par 3 E Abrahamo pose nome Isacco al figliuolo che gli era nato, che Sara gli aveva partorito.
\par 4 E Abrahamo circoncise il suo figliuolo Isacco all'età di otto giorni, come Dio gli avea comandato.
\par 5 Or Abrahamo aveva cento anni, quando gli nacque il suo figliuolo Isacco.
\par 6 E Sara disse: 'Iddio m'ha dato di che ridere; chiunque l'udrà riderà con me'.
\par 7 E aggiunse: 'Chi avrebbe mai detto ad Abrahamo che Sara allatterebbe figliuoli? poiché io gli ho partorito un figliuolo nella sua vecchiaia'.
\par 8 Il bambino dunque crebbe e fu divezzato; e nel giorno che Isacco fu divezzato, Abrahamo fece un gran convito.
\par 9 E Sara vide che il figliuolo partorito ad Abrahamo da Agar, l'Egiziana, rideva;
\par 10 allora ella disse ad Abrahamo: 'Caccia via questa serva e il suo figliuolo; perché il figliuolo di questa serva non ha da essere erede col mio figliuolo, con Isacco'.
\par 11 E la cosa dispiacque fortemente ad Abrahamo, a motivo del suo figliuolo.
\par 12 Ma Dio disse ad Abrahamo: 'Questo non ti dispiaccia, a motivo del fanciullo e della tua serva; acconsenti a tutto quello che Sara ti dirà; poiché da Isacco uscirà la progenie che porterà il tuo nome.
\par 13 Ma anche del figliuolo di questa serva io farò una nazione, perché è tua progenie'.
\par 14 Abrahamo dunque si levò la mattina di buon'ora, prese del pane e un otre d'acqua, e lo diede ad Agar, mettendoglielo sulle spalle; le diede anche il fanciullo, e la mandò via. Ed essa partì e andò errando per il deserto di Beer-Sceba.
\par 15 E quando l'acqua dell'otre venne meno, essa lasciò cadere il fanciullo sotto un arboscello.
\par 16 E se ne andò, e si pose a sedere dirimpetto, a distanza d'un tiro d'arco; perché diceva: 'Ch'io non vegga morire il fanciullo!' E sedendo così dirimpetto, alzò la voce e pianse.
\par 17 E Dio udì la voce del ragazzo; e l'angelo di Dio chiamò Agar dal cielo, e le disse: 'Che hai, Agar? non temere, poiché Iddio ha udito la voce del fanciullo là dov'è.
\par 18 Lèvati, prendi il ragazzo e tienlo per la mano; perché io farò di lui una grande nazione'.
\par 19 E Dio le aperse gli occhi, ed ella vide un pozzo d'acqua: e andò, empì d'acqua l'otre, e diè da bere al ragazzo.
\par 20 E Dio fu con lui; ed egli crebbe, abitò nel deserto, e fu tirator d'arco;
\par 21 dimorò nel deserto di Paran, e sua madre gli prese per moglie una donna del paese d'Egitto.
\par 22 Or avvenne in quel tempo che Abimelec, accompagnato da Picol, capo del suo esercito, parlò ad Abrahamo, dicendo: 'Iddio è teco in tutto quello che fai;
\par 23 or dunque giurami qui, nel nome di Dio, che tu non ingannerai né me, né i miei figliuoli, né i miei nipoti; ma che userai verso di me e verso il paese dove hai dimorato come forestiero, la stessa benevolenza che io ho usata verso di te'.
\par 24 E Abrahamo rispose: 'Lo giuro'.
\par 25 E Abrahamo fece delle rimostranze ad Abimelec per cagione di un pozzo di acqua, di cui i servi di Abimelec s'erano impadroniti per forza.
\par 26 E Abimelec disse: 'Io non so chi abbia fatto questo; tu stesso non me l'hai fatto sapere, e io non ne ho sentito parlare che oggi'.
\par 27 E Abrahamo prese pecore e buoi e li diede ad Abimelec; e i due fecero alleanza.
\par 28 Poi Abrahamo mise da parte sette agnelle del gregge.
\par 29 E Abimelec disse ad Abrahamo: 'Che voglion dire queste sette agnelle che tu hai messe da parte?'
\par 30 Abrahamo rispose: 'Tu accetterai dalla mia mano queste sette agnelle, affinché questo mi serva di testimonianza che io ho scavato questo pozzo'.
\par 31 Perciò egli chiamò quel luogo Beer-Sceba, perché ambedue vi avean fatto giuramento.
\par 32 Così fecero alleanza a Beer-Sceba. Poi Abimelec, con Picol, capo del suo esercito, si levò, e se ne tornarono nel paese dei Filistei.
\par 33 E Abrahamo piantò un tamarindo a Beer-Sceba, e invocò quivi il nome dell'Eterno, l'Iddio della eternità.
\par 34 E Abrahamo dimorò come forestiero molto tempo nel paese de' Filistei.

\chapter{22}

\par 1 Dopo queste cose, avvenne che Iddio provò Abrahamo, e gli disse: 'Abrahamo!' Ed egli rispose: 'Eccomi'.
\par 2 E Dio disse: 'Prendi ora il tuo figliuolo, il tuo unico, colui che ami, Isacco, e vattene nel paese di Moriah, e offrilo quivi in olocausto sopra uno dei monti che ti dirò'.
\par 3 E Abrahamo levatosi la mattina di buon'ora, mise il basto al suo asino, prese con sé due de' suoi servitori e Isacco suo figliuolo, spaccò delle legna per l'olocausto, poi partì per andare al luogo che Dio gli avea detto.
\par 4 Il terzo giorno, Abrahamo alzò gli occhi e vide da lontano il luogo.
\par 5 E Abrahamo disse ai suoi servitori: 'Rimanete qui con l'asino; io ed il ragazzo andremo fin colà e adoreremo; poi torneremo a voi'.
\par 6 E Abrahamo prese le legna per l'olocausto e le pose addosso a Isacco suo figliuolo; poi prese in mano sua il fuoco e il coltello, e tutti e due s'incamminarono assieme.
\par 7 E Isacco parlò ad Abrahamo suo padre e disse: 'Padre mio!' Abrahamo rispose: 'Eccomi qui, figlio mio'. E Isacco: 'Ecco il fuoco e le legna; ma dov'è l'agnello per l'olocausto?'
\par 8 Abrahamo rispose: 'Figliuol mio, Iddio se lo provvederà l'agnello per l'olocausto'. E camminarono ambedue assieme.
\par 9 E giunsero al luogo che Dio gli avea detto, e Abrahamo edificò quivi l'altare, e vi accomodò le legna; legò Isacco suo figliuolo, e lo mise sull'altare, sopra le legna.
\par 10 E Abrahamo stese la mano e prese il coltello per scannare il suo figliuolo.
\par 11 Ma l'angelo dell'Eterno gli gridò dal cielo e disse: 'Abrahamo, Abrahamo'.
\par 12 E quegli rispose: 'Eccomi'. E l'angelo: 'Non metter la mano addosso al ragazzo, e non gli fare alcun male; poiché ora so che tu temi Iddio, giacché non m'hai rifiutato il tuo figliuolo, l'unico tuo'.
\par 13 E Abrahamo alzò gli occhi, guardò, ed ecco dietro a sé un montone, preso per le corna in un cespuglio. E Abrahamo andò, prese il montone, e l'offerse in olocausto invece del suo figliuolo.
\par 14 E Abrahamo pose nome a quel luogo Jehovah-jireh. Per questo si dice oggi: 'Al monte dell'Eterno, sarà provveduto'.
\par 15 L'angelo dell'Eterno chiamò dal cielo Abrahamo una seconda volta, e disse:
\par 16 'Io giuro per me stesso, dice l'Eterno, che, siccome tu hai fatto questo e non m'hai rifiutato il tuo figliuolo, l'unico tuo,
\par 17 io certo ti benedirò e moltiplicherò la tua progenie come le stelle del cielo e come la rena ch'è sul lido del mare; e la tua progenie possederà la porta de' suoi nemici.
\par 18 E tutte le nazioni della terra saranno benedette nella tua progenie, perché tu hai ubbidito alla mia voce'.
\par 19 Poi Abrahamo se ne tornò ai suoi servitori; e si levarono e se n'andarono insieme a Beer-Sceba. E Abrahamo dimorò a Beer-Sceba.
\par 20 Dopo queste cose avvenne che fu riferito ad Abrahamo questo: 'Ecco, Milca ha partorito anch'ella de' figliuoli a Nahor, tuo fratello:
\par 21 Uz, suo primogenito, Buz suo fratello, Kemuel padre d'Aram,
\par 22 Kesed, Hazo, Pildash, Jidlaf e Bethuel'.
\par 23 E Bethuel generò Rebecca. Questi otto Milca partorì a Nahor, fratello d'Abrahamo.
\par 24 E la concubina di lui, che si chiamava Reumah, partorì anch'essa Thebah, Gaam, Tahash e Maaca.

\chapter{23}

\par 1 Or la vita di Sara fu di centoventisette anni. Tanti furon gli anni della vita di Sara.
\par 2 E Sara morì a Kiriat-Arba, che è Hebron, nel paese di Canaan; e Abrahamo venne a far duolo di Sara e a piangerla.
\par 3 Poi Abrahamo si levò di presso al suo morto, e parlò ai figliuoli di Heth, dicendo:
\par 4 'Io sono straniero e avventizio fra voi; datemi la proprietà di un sepolcro fra voi, affinché io seppellisca il mio morto e me lo tolga d'innanzi'.
\par 5 E i figliuoli di Heth risposero ad Abrahamo dicendogli:
\par 6 'Ascoltaci, signore; tu sei fra noi un principe di Dio; seppellisci il tuo morto nel migliore dei nostri sepolcri; nessun di noi ti rifiuterà il suo sepolcro perché tu vi seppellisca il tuo morto'.
\par 7 E Abrahamo si levò, s'inchinò dinanzi al popolo del paese, dinanzi ai figliuoli di Heth, e parlò loro dicendo:
\par 8 'Se piace a voi ch'io tolga il mio morto d'innanzi a me e lo seppellisca, ascoltatemi, e intercedete per me presso Efron figliuolo di Zohar
\par 9 perché mi ceda la sua spelonca di Macpela che è all'estremità del suo campo, e me la dia per l'intero suo prezzo, come sepolcro che m'appartenga fra voi'.
\par 10 Or Efron sedeva in mezzo ai figliuoli di Heth; ed Efron, lo Hitteo, rispose ad Abrahamo in presenza dei figliuoli di Heth, di tutti quelli che entravano per la porta della sua città, dicendo:
\par 11 'No, mio signore, ascoltami! Io ti dono il campo, e ti dono la spelonca che v'è; te ne fo dono, in presenza de' figliuoli del mio popolo; seppellisci il tuo morto'.
\par 12 E Abrahamo s'inchinò dinanzi al popolo del paese,
\par 13 e parlò ad Efron in presenza del popolo del paese, dicendo: 'Deh, ascoltami! Io ti darò il prezzo del campo; accettalo da me, e io seppellirò quivi il mio morto'.
\par 14 Ed Efron rispose ad Abrahamo, dicendogli:
\par 15 'Signor mio, ascoltami! Un pezzo di terreno di quattrocento sicli d'argento, che cos'è fra me e te? Seppellisci dunque il tuo morto'.
\par 16 E Abrahamo fece a modo di Efron; e Abrahamo pesò a Efron il prezzo che egli avea detto in presenza de' figliuoli di Heth, quattrocento sicli d'argento, di buona moneta mercantile.
\par 17 Così il campo di Efron ch'era a Macpela dirimpetto a Mamre, il campo con la caverna che v'era, e tutti gli alberi che erano nel campo e in tutti i confini all'intorno,
\par 18 furono assicurati come proprietà d'Abrahamo, in presenza dei figliuoli di Heth e di tutti quelli ch'entravano per la porta della città di Efron.
\par 19 Dopo questo, Abrahamo seppellì Sara sua moglie nella spelonca del campo di Macpela dirimpetto a Mamre, che è Hebron, nel paese di Canaan.
\par 20 E il campo e la spelonca che v'è, furono assicurati ad Abrahamo, dai figliuoli di Heth, come sepolcro di sua proprietà.

\chapter{24}

\par 1 Or Abrahamo era vecchio e d'età avanzata; e l'Eterno avea benedetto Abrahamo in ogni cosa.
\par 2 E Abrahamo disse al più antico servo di casa sua, che aveva il governo di tutti i suoi beni: 'Deh, metti la tua mano sotto la mia coscia;
\par 3 e io ti farò giurare per l'Eterno, l'Iddio dei cieli e l'Iddio della terra, che tu non prenderai per moglie al mio figliuolo alcuna delle figliuole de' Cananei, fra i quali dimoro;
\par 4 ma andrai al mio paese e al mio parentado, e vi prenderai una moglie per il mio figliuolo, per Isacco'.
\par 5 Il servo gli rispose: 'Forse quella donna non vorrà seguirmi in questo paese; dovrò io allora ricondurre il tuo figliuolo nel paese donde tu sei uscito?'
\par 6 E Abrahamo gli disse: 'Guardati dal ricondurre colà il mio figliuolo!
\par 7 L'Eterno, l'Iddio dei cieli, che mi trasse dalla casa di mio padre e dal mio paese natale e mi parlò e mi giurò dicendo: - Io darò alla tua progenie questo paese, - egli stesso manderà il suo angelo davanti a te, e tu prenderai di là una moglie per il mio figliuolo.
\par 8 E se la donna non vorrà seguirti, allora sarai sciolto da questo giuramento che ti faccio fare; soltanto, non ricondurre colà il mio figliuolo'.
\par 9 E il servo pose la mano sotto la coscia d'Abrahamo suo signore, e gli giurò di fare com'egli chiedeva.
\par 10 Poi il servo prese dieci cammelli fra i cammelli del suo signore, e si partì, avendo a sua disposizione tutti i beni del suo signore; e messosi in viaggio, andò in Mesopotamia, alla città di Nahor.
\par 11 E, fatti riposare sulle ginocchia i cammelli fuori della città presso a un pozzo d'acqua, verso sera, all'ora in cui le donne escono ad attinger acqua, disse:
\par 12 'O Eterno, Dio del mio signore Abrahamo, deh, fammi fare quest'oggi un felice incontro, e usa benignità verso Abrahamo mio signore!
\par 13 Ecco, io sto qui presso a questa sorgente; e le figlie degli abitanti della città usciranno ad attinger acqua.
\par 14 Fa' che la fanciulla alla quale dirò: - Deh, abbassa la tua brocca perch'io beva - e che mi risponderà - Bevi, e darò da bere anche ai tuoi cammelli, - sia quella che tu hai destinata al tuo servo Isacco. E da questo comprenderò che tu hai usato benignità verso il mio signore'.
\par 15 Non aveva ancora finito di parlare, quand'ecco uscire con la sua brocca sulla spalla, Rebecca, figliuola di Bethuel figlio di Milca, moglie di Nahor fratello d'Abrahamo.
\par 16 La fanciulla era molto bella d'aspetto, vergine, e uomo alcuno non l'avea conosciuta. Ella scese alla sorgente, empì la brocca, e risalì.
\par 17 E il servo le corse incontro, e le disse: 'Deh, dammi a bere un po' d'acqua della tua brocca'.
\par 18 Ed ella rispose: 'Bevi, signor mio'; e s'affrettò a calarsi la brocca sulla mano, e gli diè da bere.
\par 19 E quand'ebbe finito di dargli da bere disse: 'Io ne attingerò anche per i tuoi cammelli, finché abbian bevuto a sufficienza'.
\par 20 E presto vuotò la sua brocca nell'abbeveratoio, corse di nuovo al pozzo ad attingere acqua, e ne attinse per tutti i cammelli di lui.
\par 21 E quell'uomo la contemplava in silenzio per sapere se l'Eterno avesse o no fatto prosperare il suo viaggio.
\par 22 E quando i cammelli ebbero finito di bere, l'uomo prese un anello d'oro del peso di mezzo siclo, e due braccialetti del peso di dieci sicli d'oro, per i polsi di lei, e disse:
\par 23 'Di chi sei figliuola? deh, dimmelo. V'è posto in casa di tuo padre per albergarci?'
\par 24 Ed ella rispose: 'Son figliuola di Bethuel figliuolo di Milca, ch'ella partorì a Nahor'.
\par 25 E aggiunse: 'C'è da noi strame e foraggio assai, e anche posto da albergare'.
\par 26 E l'uomo s'inchinò, adorò l'Eterno, e disse:
\par 27 'Benedetto l'Eterno, l'Iddio d'Abrahamo mio signore, che non ha cessato d'esser benigno e fedele verso il mio signore! Quanto a me, l'Eterno mi ha messo sulla via della casa dei fratelli del mio signore'.
\par 28 E la fanciulla corse a raccontare queste cose a casa di sua madre.
\par 29 Or Rebecca aveva un fratello chiamato Labano. E Labano corse fuori da quell'uomo alla sorgente.
\par 30 Com'ebbe veduto l'anello e i braccialetti ai polsi di sua sorella ed ebbe udite le parole di Rebecca sua sorella che diceva: 'Quell'uomo m'ha parlato così', venne a quell'uomo, ed ecco ch'egli se ne stava presso ai cammelli, vicino alla sorgente.
\par 31 E disse: 'Entra, benedetto dall'Eterno! perché stai fuori? Io ho preparato la casa e un luogo per i cammelli'.
\par 32 L'uomo entrò in casa, e Labano scaricò i cammelli, diede strame e foraggio ai cammelli, e portò acqua per lavare i piedi a lui e a quelli ch'eran con lui.
\par 33 Poi gli fu posto davanti da mangiare; ma egli disse: 'Non mangerò finché non abbia fatto la mia ambasciata'. E l'altro disse: 'Parla'.
\par 34 E quegli: 'Io sono servo d'Abrahamo.
\par 35 L'Eterno ha benedetto abbondantemente il mio signore, ch'è divenuto grande; gli ha dato pecore e buoi, argento e oro, servi e serve, cammelli e asini.
\par 36 Or Sara, moglie del mio signore, ha partorito nella sua vecchiaia un figliuolo al mio padrone, che gli ha dato tutto quel che possiede.
\par 37 E il mio signore m'ha fatto giurare, dicendo: - Non prenderai come moglie per il mio figliuolo alcuna delle figlie de' Cananei, nel paese de' quali dimoro;
\par 38 ma andrai alla casa di mio padre e al mio parentado e vi prenderai una moglie per il mio figliuolo.
\par 39 - E io dissi al mio padrone: - Forse quella donna non mi vorrà seguire.
\par 40 - Ed egli rispose: - L'Eterno, nel cospetto del quale ho camminato, manderà il suo angelo teco e farà prosperare il tuo viaggio, e tu prenderai al mio figliuolo una moglie del mio parentado e della casa di mio padre.
\par 41 Sarai sciolto dal giuramento che ti fo fare, quando sarai andato dal mio parentado; e, se non vorranno dartela, allora sarai sciolto dal giuramento che mi fai.
\par 42 - Oggi sono arrivato alla sorgente, e ho detto: - O Eterno, Dio del mio signore Abrahamo, se pur ti piace far prosperare il viaggio che ho intrapreso,
\par 43 ecco, io mi fermo presso questa sorgente; fa' che la fanciulla che uscirà ad attinger acqua, alla quale dirò: - Deh, dammi da bere un po' d'acqua della tua brocca, -
\par 44 e che mi dirà: - Bevi pure, e ne attingerò anche per i tuoi cammelli, - sia la moglie che l'Eterno ha destinata al figliuolo del mio signore.
\par 45 E avanti che avessi finito di parlare in cuor mio, ecco uscir fuori Rebecca con la sua brocca sulla spalla, scendere alla sorgente e attinger l'acqua. Allora io le ho detto:
\par 46 - Deh, dammi da bere! - Ed ella s'è affrettata a calare la brocca dalla spalla, e m'ha risposto: - Bevi! e darò da bere anche ai tuoi cammelli. - Così ho bevuto io ed ella ha abbeverato anche i cammelli.
\par 47 Poi l'ho interrogata, e le ho detto: - Di chi sei figliuola? - Ed ella ha risposto: - Son figliuola di Bethuel figlio di Nahor, che Milca gli partorì. - Allora io le ho messo l'anello al naso e i braccialetti ai polsi.
\par 48 E mi sono inchinato, ho adorato l'Eterno e ho benedetto l'Eterno, l'Iddio d'Abrahamo mio signore, che m'ha condotto per la retta via a prendere per il figliuolo di lui la figliuola del fratello del mio signore.
\par 49 E ora, se volete usare benignità e fedeltà verso il mio signore, ditemelo; e se no, ditemelo lo stesso, e io mi volgerò a destra o a sinistra'.
\par 50 Allora Labano e Bethuel risposero e dissero: 'La cosa procede dall'Eterno; noi non possiam dirti né male né bene.
\par 51 Ecco, Rebecca ti sta dinanzi, prendila, va', e sia ella moglie del figliuolo del tuo signore, come l'Eterno ha detto'.
\par 52 E quando il servo d'Abrahamo ebbe udito le loro parole, si prostrò a terra dinanzi all'Eterno.
\par 53 Il servo trasse poi fuori oggetti d'argento e oggetti d'oro, e vesti, e li dette a Rebecca; e donò anche delle cose preziose al fratello e alla madre di lei.
\par 54 Poi mangiarono e bevvero, egli e gli uomini ch'eran con lui, e passaron quivi la notte. La mattina, quando si furono levati, il servo disse: 'Lasciatemi tornare al mio signore'.
\par 55 E il fratello e la madre di Rebecca dissero: 'Rimanga la fanciulla ancora alcuni giorni con noi, almeno una diecina; poi se ne andrà'. Ma egli rispose loro:
\par 56 'Non mi trattenete, giacché l'Eterno ha fatto prosperare il mio viaggio; lasciatemi partire, affinché io me ne torni al mio signore'.
\par 57 Allora dissero: 'Chiamiamo la fanciulla e sentiamo lei stessa'.
\par 58 Chiamarono Rebecca, e le dissero: 'Vuoi tu andare con quest'uomo?' Ed ella rispose:
\par 59 'Sì, andrò'. Così lasciarono andare Rebecca loro sorella e la sua balia col servo d'Abrahamo e la sua gente.
\par 60 E benedissero Rebecca e le dissero: 'Sorella nostra, possa tu esser madre di migliaia di miriadi, e possa la tua progenie possedere la porta de' suoi nemici!'
\par 61 E Rebecca si levò con le sue serve e montarono sui cammelli e seguirono quell'uomo. E il servo prese Rebecca e se ne andò.
\par 62 Or Isacco era tornato dal pozzo di Lachai-Roï, ed abitava nel paese del mezzodì.
\par 63 Isacco era uscito, sul far della sera, per meditare nella campagna; e, alzati gli occhi, guardò, ed ecco venir de' cammelli.
\par 64 E Rebecca, alzati anch'ella gli occhi, vide Isacco, saltò giù dal cammello, e disse al servo:
\par 65 'Chi è quell'uomo che viene nel campo incontro a noi?' Il servo rispose: 'È il mio signore'. Ed ella, preso il suo velo, se ne coprì.
\par 66 E il servo raccontò a Isacco tutto quello che avea fatto.
\par 67 E Isacco menò Rebecca nella tenda di Sara sua madre, se la prese, ed ella divenne sua moglie, ed egli l'amò. Così Isacco fu consolato dopo la morte di sua madre.

\chapter{25}

\par 1 Poi Abrahamo prese un'altra moglie, per nome Ketura.
\par 2 E questa gli partorì Zimran, Jokshan, Medan, Madian, Jishbak e Shuach.
\par 3 Jokshan generò Sceba e Dedan. I figliuoli di Dedan furono gli Asshurim, i Letushim ed i Leummim.
\par 4 E i figliuoli di Madian furono Efa, Efer, Hanoch, Abida ed Eldaa. Tutti questi furono i figliuoli di Ketura.
\par 5 E Abrahamo dette tutto quello che possedeva a Isacco;
\par 6 ma ai figliuoli delle sue concubine fece dei doni e, mentre era ancora in vita, li mandò lungi dal suo figliuolo Isacco, verso levante, nel paese d'oriente.
\par 7 Or tutto il tempo della vita d'Abrahamo fu di centosettantacinque anni.
\par 8 Poi Abrahamo spirò in prospera vecchiezza, attempato e sazio di giorni, e fu riunito al suo popolo.
\par 9 E Isacco e Ismaele, suoi figliuoli, lo seppellirono nella spelonca di Macpela nel campo di Efron figliuolo di Tsoar lo Hitteo, ch'è dirimpetto a Mamre:
\par 10 campo, che Abrahamo avea comprato dai figliuoli di Heth. Quivi furon sepolti Abrahamo e Sara sua moglie.
\par 11 E dopo la morte d'Abrahamo, Iddio benedisse Isacco figliuolo di lui; e Isacco dimorò presso il pozzo di Lachai-Roï.
\par 12 Or questi sono i discendenti d'Ismaele, figliuolo d'Abrahamo, che Agar, l'Egiziana, serva di Sara, avea partorito ad Abrahamo.
\par 13 Questi sono i nomi de' figliuoli d'Ismaele, secondo le loro generazioni: Nebaioth, il primogenito d'Ismaele; poi Kedar, Adbeel, Mibsam,
\par 14 Mishma, Duma, Massa, Hadar, Tema, Jethur,
\par 15 Nafish e Kedma.
\par 16 Questi sono i figliuoli d'Ismaele, e questi i loro nomi, secondo i loro villaggi e i loro accampamenti. Furono i dodici capi dei loro popoli.
\par 17 E gli anni della vita d'Ismaele furono centotrentasette; poi spirò, morì, e fu riunito al suo popolo.
\par 18 E i suoi figliuoli abitarono da Havila fino a Shur, ch'è dirimpetto all'Egitto, andando verso l'Assiria. Egli si stabilì di faccia a tutti i suoi fratelli.
\par 19 E questi sono i discendenti d'Isacco, figliuolo d'Abrahamo.
\par 20 Abrahamo generò Isacco; e Isacco era in età di quarant'anni quando prese per moglie Rebecca, figliuola di Bethuel, l'Arameo di Paddan-Aram, e sorella di Labano, l'Arameo.
\par 21 Isacco pregò istantemente l'Eterno per sua moglie, perch'ella era sterile. L'Eterno l'esaudì, e Rebecca, sua moglie, concepì.
\par 22 E i bambini si urtavano nel suo seno; ed ella disse: 'Se così è, perché vivo?' E andò a consultare l'Eterno.
\par 23 E l'Eterno le disse: 'Due nazioni sono nel tuo seno, e due popoli separati usciranno dalle tue viscere. Uno dei due popoli sarà più forte dell'altro, e il maggiore servirà il minore'.
\par 24 E quando venne per lei il tempo di partorire, ecco ch'ella aveva due gemelli nel seno.
\par 25 E il primo che uscì fuori era rosso, e tutto quanto come un mantello di pelo; e gli fu posto nome Esaù.
\par 26 Dopo uscì il suo fratello, che con la mano teneva il calcagno di Esaù; e gli fu posto nome Giacobbe. Or Isacco era in età di sessant'anni quando Rebecca li partorì.
\par 27 I due fanciulli crebbero, ed Esaù divenne un esperto cacciatore, un uomo di campagna, e Giacobbe un uomo tranquillo, che se ne stava nelle tende.
\par 28 Or Isacco amava Esaù, perché la cacciagione era di suo gusto; e Rebecca amava Giacobbe.
\par 29 Or come Giacobbe s'era fatto cuocere una minestra, Esaù giunse dai campi, tutto stanco.
\par 30 Ed Esaù disse a Giacobbe: 'Deh, dammi da mangiare un po' di cotesta minestra rossa; perché sono stanco'. Per questo fu chiamato Edom.
\par 31 E Giacobbe gli rispose: 'Vendimi prima di tutto la tua primogenitura'.
\par 32 Ed Esaù disse: 'Ecco io sto per morire; che mi giova la primogenitura?'
\par 33 E Giacobbe disse: 'Prima, giuramelo'. Ed Esaù glielo giurò, e vendé la sua primogenitura a Giacobbe.
\par 34 E Giacobbe diede a Esaù del pane e della minestra di lenticchie. Ed egli mangiò e bevve; poi si levò, e se ne andò. Così Esaù sprezzò la primogenitura.

\chapter{26}

\par 1 Or ci fu la carestia nel paese, oltre la prima carestia che c'era stata al tempo d'Abrahamo. E Isacco andò da Abimelec, re dei Filistei, a Gherar.
\par 2 E l'Eterno gli apparve e gli disse: 'Non scendere in Egitto; dimora nel paese che io ti dirò.
\par 3 Soggiorna in questo paese, e io sarò teco e ti benedirò, poiché io darò a te e alla tua progenie tutti questi paesi, e manterrò il giuramento che feci ad Abrahamo tuo padre,
\par 4 e moltiplicherò la tua progenie come le stelle del cielo, darò alla tua progenie tutti questi paesi, e tutte le nazioni della terra saranno benedette nella tua progenie,
\par 5 perché Abrahamo ubbidì alla mia voce e osservò quello che gli aveva ordinato, i miei comandamenti, i miei statuti e le mie leggi'.
\par 6 E Isacco dimorò in Gherar.
\par 7 E quando la gente del luogo gli faceva delle domande intorno alla sua moglie, egli rispondeva: 'È mia sorella'; perché avea paura di dire: 'È mia moglie'. 'Non vorrei', egli pensava, 'che la gente del luogo avesse ad uccidermi, a motivo di Rebecca'. Poiché ella era di bell'aspetto.
\par 8 Ora, prolungandosi quivi il suo soggiorno, avvenne che Abimelec, re dei Filistei, mentre guardava dalla finestra, vide Isacco che scherzava con Rebecca sua moglie.
\par 9 E Abimelec chiamò Isacco, e gli disse: 'Certo, costei è tua moglie; come mai dunque, hai detto: È mia sorella?' E Isacco rispose: 'Perché dicevo: Non vorrei esser messo a morte a motivo di lei'.
\par 10 E Abimelec: 'Che cos'è questo che ci hai fatto? Poco è mancato che qualcuno del popolo si giacesse con tua moglie, e tu ci avresti tirato addosso una gran colpa'.
\par 11 E Abimelec diede quest'ordine a tutto il popolo: 'Chiunque toccherà quest'uomo o sua moglie sia messo a morte'.
\par 12 Isacco seminò in quel paese, e in quell'anno raccolse il centuplo; e l'Eterno lo benedisse.
\par 13 Quest'uomo divenne grande, andò crescendo sempre più, finché diventò grande oltremisura.
\par 14 Fu padrone di greggi di pecore, di mandre di buoi e di numerosa servitù. I Filistei lo invidiavano;
\par 15 e perciò turarono ed empiron di terra tutti i pozzi che i servi di suo padre aveano scavati, al tempo d'Abrahamo suo padre.
\par 16 E Abimelec disse ad Isacco: 'Vattene da noi, poiché tu sei molto più potente di noi'.
\par 17 Isacco allora si partì di là, s'accampò nella valle di Gherar, e quivi dimorò.
\par 18 E Isacco scavò di nuovo i pozzi d'acqua ch'erano stati scavati al tempo d'Abrahamo suo padre, e che i Filistei avean turati dopo la morte d'Abrahamo; e pose loro gli stessi nomi che avea loro posto suo padre.
\par 19 E i servi d'Isacco scavarono nella valle, e vi trovarono un pozzo d'acqua viva.
\par 20 Ma i pastori di Gherar altercarono coi pastori d'Isacco, dicendo: 'L'acqua è nostra'. Ed egli chiamò il pozzo Esek, perché quelli aveano conteso con lui.
\par 21 Poi i servi scavarono un altro pozzo, e per questo ancora quelli altercarono. E Isacco lo chiamò Sitna.
\par 22 Allora egli si partì di là, e scavò un altro pozzo per il quale quelli non altercarono. Ed egli lo chiamò Rehoboth 'perché', disse, 'ora l'Eterno ci ha messi al largo, e noi prospereremo nel paese'.
\par 23 Poi di là Isacco salì a Beer-Sceba.
\par 24 E l'Eterno gli apparve quella stessa notte, e gli disse: 'Io sono l'Iddio d'Abrahamo tuo padre; non temere, poiché io sono teco e ti benedirò e moltiplicherò la tua progenie per amor d'Abrahamo mio servo'.
\par 25 Ed egli edificò quivi un altare, invocò il nome dell'Eterno, e vi piantò la sua tenda. E i servi d'Isacco scavaron quivi un pozzo.
\par 26 Abimelec andò a lui da Gherar con Ahuzath, suo amico, e con Picol, capo del suo esercito.
\par 27 E Isacco disse loro: 'Perché venite da me, giacché mi odiate e m'avete mandato via dal vostro paese?'
\par 28 E quelli risposero: 'Noi abbiamo chiaramente veduto che l'Eterno è teco; e abbiamo detto: Si faccia ora un giuramento fra noi, fra noi e te, e facciam lega teco.
\par 29 Giura che non ci farai alcun male, così come noi non t'abbiamo toccato, e non t'abbiamo fatto altro che del bene, e t'abbiamo lasciato andare in pace. Tu sei ora benedetto dall'Eterno'.
\par 30 E Isacco fece loro un convito, ed essi mangiarono e bevvero.
\par 31 La mattina dipoi si levarono di buon'ora e si fecero scambievole giuramento. Poi Isacco li accomiatò, e quelli si partirono da lui in pace.
\par 32 Or avvenne che, in quello stesso giorno, i servi d'Isacco gli vennero a dar notizia del pozzo che aveano scavato, dicendogli: 'Abbiam trovato dell'acqua'.
\par 33 Ed egli lo chiamò Sciba. Per questo la città porta il nome di Beer-Sceba, fino al dì d'oggi.
\par 34 Or Esaù, in età di quarant'anni, prese per moglie Judith, figliuola di Beeri, lo Hitteo, e Basmath, figliuola di Elon, lo Hitteo. Esse furon cagione d'amarezza d'animo a Isacco ed a Rebecca.

\chapter{27}

\par 1 Or avvenne, quando Isacco era divenuto vecchio e i suoi occhi indeboliti non ci vedevano più, ch'egli chiamò Esaù, suo figliuolo maggiore, e gli disse: 'Figliuol mio!'
\par 2 E quello rispose: 'Eccomi!' E Isacco: 'Ecco, io sono vecchio, e non so il giorno della mia morte.
\par 3 Deh, prendi ora le tue armi, il tuo turcasso e il tuo arco, vattene fuori ai campi, prendimi un po' di caccia,
\par 4 e preparami una pietanza saporita di quelle che mi piacciono; portamela perch'io la mangi e l'anima mia ti benedica prima ch'io muoia'.
\par 5 Ora Rebecca stava ad ascoltare, mentre Isacco parlava ad Esaù suo figliuolo. Ed Esaù se n'andò ai campi per fare qualche caccia e portarla a suo padre.
\par 6 E Rebecca parlò a Giacobbe suo figliuolo, e gli disse: 'Ecco, io ho udito tuo padre che parlava ad Esaù tuo fratello, e gli diceva:
\par 7 Portami un po' di caccia e fammi una pietanza saporita perch'io la mangi e ti benedica nel cospetto dell'Eterno, prima ch'io muoia.
\par 8 Or dunque, figliuol mio, ubbidisci alla mia voce e fa' quello ch'io ti comando.
\par 9 Va' ora al gregge e prendimi due buoni capretti; e io ne farò una pietanza saporita per tuo padre, di quelle che gli piacciono.
\par 10 E tu la porterai a tuo padre, perché la mangi, e così ti benedica prima di morire'.
\par 11 E Giacobbe disse a Rebecca sua madre: 'Ecco, Esaù mio fratello è peloso, e io no.
\par 12 Può darsi che mio padre mi tasti; sarò allora da lui reputato un ingannatore, e mi trarrò addosso una maledizione, invece di una benedizione'.
\par 13 E sua madre gli rispose: 'Questa maledizione ricada su me, figliuol mio! Ubbidisci pure alla mia voce, e va' a prendermi i capretti'.
\par 14 Egli dunque andò a prenderli, e li menò a sua madre; e sua madre ne preparò una pietanza saporita, di quelle che piacevano al padre di lui.
\par 15 Poi Rebecca prese i più bei vestiti di Esaù suo figliuolo maggiore, i quali aveva in casa presso di sé, e li fece indossare a Giacobbe suo figliuolo minore;
\par 16 e con le pelli de' capretti gli coprì le mani e il collo, ch'era senza peli.
\par 17 Poi mise in mano a Giacobbe suo figliuolo la pietanza saporita e il pane che avea preparato.
\par 18 Ed egli venne a suo padre e gli disse: 'Padre mio!' E Isacco rispose: 'Eccomi; chi sei tu, figliuol mio?'
\par 19 E Giacobbe disse a suo padre: 'Sono Esaù, il tuo primogenito. Ho fatto come tu m'hai detto. Deh, lèvati, mettiti a sedere e mangia della mia caccia, affinché l'anima tua mi benedica'.
\par 20 E Isacco disse al suo figliuolo: 'Come hai fatto a trovarne così presto, figliuol mio?' E quello rispose: 'Perché l'Eterno, il tuo Dio, l'ha fatta venire sulla mia via'.
\par 21 E Isacco disse a Giacobbe: 'Fatti vicino, figliuol mio, ch'io ti tasti, per sapere se sei proprio il mio figliuolo Esaù, o no'.
\par 22 Giacobbe dunque s'avvicinò a Isacco suo padre e, come questi l'ebbe tastato, disse: 'La voce è la voce di Giacobbe; ma le mani son le mani d'Esaù'.
\par 23 E non lo riconobbe, perché le mani di lui eran pelose come le mani di Esaù suo fratello: e lo benedisse. E disse:
\par 24 'Sei tu proprio il mio figliuolo Esaù?' Egli rispose: 'Sì'.
\par 25 E Isacco gli disse: 'Servimi, ch'io mangi della caccia del mio figliuolo e l'anima mia ti benedica'. E Giacobbe lo servì, e Isacco mangiò. Giacobbe gli portò anche del vino, ed egli bevve.
\par 26 Poi Isacco suo padre gli disse: 'Deh, fatti vicino e baciami, figliuol mio'.
\par 27 Ed egli s'avvicinò e lo baciò. E Isacco sentì l'odore de' vestiti di lui, e lo benedisse dicendo: 'Ecco, l'odor del mio figliuolo è come l'odor d'un campo, che l'Eterno ha benedetto.
\par 28 Iddio ti dia della rugiada de' cieli e della grassezza della terra e abbondanza di frumento e di vino.
\par 29 Ti servano i popoli, e le nazioni s'inchinino davanti a te. Sii padrone de' tuoi fratelli, e i figli di tua madre s'inchinino davanti a te. Maledetto sia chiunque ti maledice, benedetto sia chiunque ti benedice!'
\par 30 E avvenne che, come Isacco ebbe finito di benedire Giacobbe e Giacobbe se n'era appena andato dalla presenza d'Isacco suo padre, Esaù suo fratello giunse dalla sua caccia.
\par 31 Anch'egli preparò una pietanza saporita, la portò a suo padre, e gli disse: 'Lèvisi mio padre, e mangi della caccia del suo figliuolo, affinché l'anima tua mi benedica'.
\par 32 E Isacco suo padre gli disse: 'Chi sei tu?' Ed egli rispose: 'Sono Esaù, il tuo figliuolo primogenito'.
\par 33 Isacco fu preso da un tremito fortissimo, e disse: 'E allora, chi è che ha preso della caccia e me l'ha portata? Io ho mangiato di tutto prima che tu venissi, e l'ho benedetto; e benedetto ei sarà'.
\par 34 Quando Esaù ebbe udite le parole di suo padre, dette in un grido forte ed amarissimo. Poi disse a suo padre: 'Benedici anche me, padre mio!'
\par 35 E Isacco rispose: 'Il tuo fratello è venuto con inganno e ha preso la tua benedizione'.
\par 36 Ed Esaù: 'Non è forse a ragione ch'egli è stato chiamato Giacobbe? M'ha già soppiantato due volte: mi tolse la mia primogenitura, ed ecco che ora m'ha tolta la mia benedizione'. Poi aggiunse: 'Non hai tu riserbato qualche benedizione per me?'
\par 37 E Isacco rispose e disse a Esaù: 'Ecco, io l'ho costituito tuo padrone, e gli ho dato tutti i suoi fratelli per servi, e l'ho provvisto di frumento e di vino; che potrei dunque fare per te, figliuol mio?'
\par 38 Ed Esaù disse a suo padre: 'Non hai tu che questa benedizione, padre mio? Benedici anche me, o padre mio!' Ed Esaù alzò la voce e pianse.
\par 39 E Isacco suo padre rispose e gli disse: 'Ecco, la tua dimora sarà priva della grassezza della terra e della rugiada che scende dai cieli.
\par 40 Tu vivrai della tua spada, e sarai servo del tuo fratello; ma avverrà che, menando una vita errante, tu spezzerai il suo giogo di sul tuo collo'.
\par 41 Ed Esaù prese a odiare Giacobbe a motivo della benedizione datagli da suo padre; e disse in cuor suo: 'I giorni del lutto di mio padre si avvicinano; allora ucciderò il mio fratello Giacobbe'.
\par 42 Furon riferite a Rebecca le parole di Esaù, suo figliuolo maggiore; ed ella mandò a chiamare Giacobbe, suo figliuolo minore, e gli disse: 'Ecco, Esaù, tuo fratello, si consola riguardo a te, proponendosi d'ucciderti.
\par 43 Or dunque, figliuol mio, ubbidisci alla mia voce: lèvati, e fuggi a Charan da Labano mio fratello;
\par 44 e trattienti quivi qualche tempo, finché il furore del tuo fratello sia passato,
\par 45 finché l'ira del tuo fratello si sia stornata da te ed egli abbia dimenticato quello che tu gli hai fatto; e allora io manderò a farti ricondurre di là. Perché sarei io privata di voi due in uno stesso giorno?'
\par 46 E Rebecca disse ad Isacco: 'Io sono disgustata della vita a motivo di queste figliuole di Heth. Se Giacobbe prende in moglie, tra le figliuole di Heth, tra le figliuole del paese, una donna come quelle, che mi giova la vita?'.

\chapter{28}

\par 1 Allora Isacco chiamò Giacobbe, lo benedisse e gli diede quest'ordine: 'Non prender moglie tra le figliuole di Canaan.
\par 2 Lèvati, vattene in Paddan-Aram, alla casa di Bethuel, padre di tua madre, e prenditi moglie di là, tra le figliuole di Labano, fratello di tua madre.
\par 3 E l'Iddio onnipotente ti benedica, ti renda fecondo e ti moltiplichi, in guisa che tu diventi un'assemblea di popoli,
\par 4 e ti dia la benedizione d'Abrahamo: a te, e alla tua progenie con te; affinché tu possegga il paese dove sei andato peregrinando, e che Dio donò ad Abrahamo'.
\par 5 E Isacco fece partire Giacobbe, il quale se n'andò in Paddan-Aram da Labano, figliuolo di Bethuel, l'Arameo, fratello di Rebecca, madre di Giacobbe e di Esaù.
\par 6 Or Esaù vide che Isacco avea benedetto Giacobbe e l'avea mandato in Paddan-Aram perché vi prendesse moglie; e che, benedicendolo, gli avea dato quest'ordine: 'Non prender moglie tra le figliuole di Canaan',
\par 7 e che Giacobbe aveva ubbidito a suo padre e a sua madre, e se n'era andato in Paddan-Aram.
\par 8 Ed Esaù s'accorse che le figliuole di Canaan dispiacevano ad Isacco suo padre;
\par 9 e andò da Ismaele, e prese per moglie, oltre quelle che aveva già, Mahalath, figliuola d'Ismaele, figliuolo d'Abrahamo, sorella di Nebaioth.
\par 10 Or Giacobbe partì da Beer-Sceba e se n'andò verso Charan.
\par 11 Capitò in un certo luogo, e vi passò la notte, perché il sole era già tramontato. Prese una delle pietre del luogo, la pose come suo capezzale e si coricò quivi.
\par 12 E sognò; ed ecco una scala appoggiata sulla terra, la cui cima toccava il cielo; ed ecco gli angeli di Dio, che salivano e scendevano per la scala.
\par 13 E l'Eterno stava al disopra d'essa, e gli disse: 'Io sono l'Eterno, l'Iddio d'Abrahamo tuo padre e l'Iddio d'Isacco; la terra sulla quale tu stai coricato, io la darò a te e alla tua progenie;
\par 14 e la tua progenie sarà come la polvere della terra, e tu ti estenderai ad occidente e ad oriente, a settentrione e a mezzodì; e tutte le famiglie della terra saranno benedette in te e nella tua progenie.
\par 15 Ed ecco, io son teco, e ti guarderò dovunque tu andrai, e ti ricondurrò in questo paese; poiché io non ti abbandonerò prima d'aver fatto quello che t'ho detto'.
\par 16 E come Giacobbe si fu svegliato dal suo sonno, disse: 'Certo, l'Eterno è in questo luogo ed io non lo sapevo!'
\par 17 Ed ebbe paura, e disse: 'Com'è tremendo questo luogo! Questa non è altro che la casa di Dio, e questa è la porta del cielo!'
\par 18 E Giacobbe si levò la mattina di buon'ora, prese la pietra che avea posta come suo capezzale, la eresse in monumento, e versò dell'olio sulla sommità d'essa.
\par 19 E pose nome a quel luogo Bethel; ma, prima, il nome della città era Luz.
\par 20 E Giacobbe fece un voto, dicendo: 'Se Dio è meco, se mi guarda durante questo viaggio che fo, se mi dà pane da mangiare e vesti da coprirmi,
\par 21 e se ritorno sano e salvo alla casa del padre mio, l'Eterno sarà il mio Dio;
\par 22 e questa pietra che ho eretta in monumento, sarà la casa di Dio; e di tutto quello che tu darai a me, io, certamente, darò a te la decima'.

\chapter{29}

\par 1 Poi Giacobbe si mise in cammino e andò nel paese degli Orientali.
\par 2 E guardò, e vide un pozzo in un campo; ed ecco tre greggi di pecore, giacenti lì presso; poiché a quel pozzo si abbeveravano i greggi; e la pietra sulla bocca del pozzo era grande.
\par 3 Quivi s'adunavano tutti i greggi; i pastori rotolavan la pietra di sulla bocca del pozzo, abbeveravano le pecore, poi rimettevano al posto la pietra sulla bocca del pozzo.
\par 4 E Giacobbe disse ai pastori: 'Fratelli miei, di dove siete?' E quelli risposero: 'Siamo di Charan'.
\par 5 Ed egli disse loro: 'Conoscete voi Labano, figliuolo di Nahor?' Ed essi: 'Lo conosciamo'.
\par 6 Ed egli disse loro: 'Sta egli bene?' E quelli: 'Sta bene; ed ecco Rachele, sua figliuola, che viene con le pecore'.
\par 7 Ed egli disse: 'Ecco, è ancora pieno giorno, e non è tempo di radunare il bestiame; abbeverate le pecore e menatele al pascolo'.
\par 8 E quelli risposero: 'Non possiamo, finché tutti i greggi siano radunati; allora si rotola la pietra di sulla bocca del pozzo, e abbeveriamo le pecore'.
\par 9 Mentr'egli parlava ancora con loro, giunse Rachele con le pecore di suo padre; poich'ella era pastora.
\par 10 E quando Giacobbe vide Rachele figliuola di Labano, fratello di sua madre, e le pecore di Labano fratello di sua madre, s'avvicinò, rotolò la pietra di sulla bocca del pozzo, e abbeverò il gregge di Labano fratello di sua madre.
\par 11 E Giacobbe baciò Rachele, alzò la voce, e pianse.
\par 12 E Giacobbe fe' sapere a Rachele ch'egli era parente del padre di lei, e ch'era figliuolo di Rebecca. Ed ella corse a dirlo a suo padre.
\par 13 E appena Labano ebbe udito le notizie di Giacobbe figliuolo della sua sorella, gli corse incontro, l'abbracciò, lo baciò, e lo menò a casa sua. Giacobbe raccontò a Labano tutte queste cose;
\par 14 e Labano gli disse: 'Tu sei proprio mie ossa e mia carne!' Ed egli dimorò con lui durante un mese.
\par 15 Poi Labano disse a Giacobbe: 'Perché sei mio parente dovrai tu servirmi per nulla? Dimmi quale dev'essere il tuo salario'.
\par 16 Or Labano aveva due figliuole: la maggiore si chiamava Lea, e la minore Rachele.
\par 17 Lea aveva gli occhi delicati, ma Rachele era avvenente e di bell'aspetto.
\par 18 E Giacobbe amava Rachele, e disse a Labano: 'Io ti servirò sette anni, per Rachele tua figliuola minore'.
\par 19 E Labano rispose: 'È meglio ch'io la dia a te che ad un altr'uomo; sta' con me'.
\par 20 E Giacobbe servì sette anni per Rachele; e gli parvero pochi giorni, per l'amore che le portava.
\par 21 E Giacobbe disse a Labano: 'Dammi la mia moglie, poiché il mio tempo è compiuto, ed io andrò da lei'.
\par 22 Allora Labano radunò tutta la gente del luogo, e fece un convito.
\par 23 Ma, la sera, prese Lea, sua figliuola, e la menò da Giacobbe, il quale entrò da lei.
\par 24 E Labano dette la sua serva Zilpa per serva a Lea, sua figliuola.
\par 25 L'indomani mattina, ecco che era Lea. E Giacobbe disse a Labano: 'Che m'hai fatto? Non è egli per Rachele ch'io t'ho servito? Perché dunque m'hai ingannato?'
\par 26 E Labano rispose: 'Non è usanza da noi di dare la minore prima della maggiore. Finisci la settimana di questa;
\par 27 e ti daremo anche l'altra, per il servizio che presterai da me altri sette anni'.
\par 28 Giacobbe fece così, e finì la settimana di quello sposalizio; poi Labano gli dette in moglie Rachele sua figliuola.
\par 29 E Labano dette la sua serva Bilha per serva a Rachele, sua figliuola.
\par 30 E Giacobbe entrò pure da Rachele, ed anche amò Rachele più di Lea, e servì da Labano altri sette anni.
\par 31 L'Eterno, vedendo che Lea era odiata, la rese feconda; ma Rachele era sterile.
\par 32 E Lea concepì e partorì un figliuolo, al quale pose nome Ruben; perché disse: 'L'Eterno ha veduto la mia afflizione; e ora il mio marito mi amerà'.
\par 33 Poi concepì di nuovo e partorì un figliuolo, e disse: 'L'Eterno ha udito ch'io ero odiata, e però m'ha dato anche questo figliuolo'. E lo chiamò Simeone.
\par 34 E concepì di nuovo e partorì un figliuolo, e disse: 'Questa volta, il mio marito sarà ben unito a me, poiché gli ho partorito tre figliuoli'. Per questo fu chiamato Levi.
\par 35 E concepì di nuovo e partorì un figliuolo, e disse: 'Questa volta celebrerò l'Eterno'. Perciò gli pose nome Giuda. E cessò d'aver figliuoli.

\chapter{30}

\par 1 Rachele, vedendo che non dava figliuoli a Giacobbe, portò invidia alla sua sorella, e disse a Giacobbe: 'Dammi de' figliuoli; altrimenti muoio'.
\par 2 E Giacobbe s'accese d'ira contro Rachele, e disse: 'Tengo io il luogo di Dio che t'ha negato d'esser feconda?'
\par 3 Ed ella rispose: 'Ecco la mia serva Bilha; entra da lei; essa partorirà sulle mie ginocchia, e, per mezzo di lei, avrò anch'io dei figliuoli'.
\par 4 Ed ella gli diede la sua serva Bilha per moglie, e Giacobbe entrò da lei.
\par 5 E Bilha concepì e partorì un figliuolo a Giacobbe.
\par 6 E Rachele disse: 'Iddio m'ha reso giustizia, ha anche ascoltato la mia voce, e m'ha dato un figliuolo'. Perciò gli pose nome Dan.
\par 7 E Bilha, serva di Rachele, concepì ancora e partorì a Giacobbe un secondo figliuolo.
\par 8 E Rachele disse: 'Io ho sostenuto con mia sorella lotte di Dio, e ho vinto'. Perciò gli pose nome Neftali.
\par 9 Lea, vedendo che avea cessato d'aver figliuoli, prese la sua serva Zilpa e la diede a Giacobbe per moglie.
\par 10 E Zilpa, serva di Lea, partorì un figliuolo a Giacobbe. E Lea disse:
\par 11 'Che fortuna!' E gli pose nome Gad.
\par 12 Poi Zilpa, serva di Lea, partorì a Giacobbe un secondo figliuolo. E Lea disse:
\par 13 'Me felice! ché le fanciulle mi chiameranno beata'. Perciò gli pose nome Ascer.
\par 14 Or Ruben uscì, al tempo della mietitura del grano, e trovò delle mandragole per i campi, e le portò a Lea sua madre. Allora Rachele disse a Lea: 'Deh, dammi delle mandragole del tuo figliuolo!'
\par 15 Ed ella le rispose: 'Ti par egli poco l'avermi tolto il marito, che mi vuoi togliere anche le mandragole del mio figliuolo?' E Rachele disse: 'Ebbene, si giaccia egli teco questa notte, in compenso delle mandragole del tuo figliuolo'.
\par 16 E come Giacobbe, in sulla sera, se ne tornava dai campi, Lea uscì a incontrarlo, e gli disse: 'Devi entrare da me; poiché io t'ho accaparrato con le mandragole del mio figliuolo'. Ed egli si giacque con lei quella notte.
\par 17 E Dio esaudì Lea, la quale concepì e partorì a Giacobbe un quinto figliuolo.
\par 18 Ed ella disse: 'Iddio m'ha dato la mia mercede, perché diedi la mia serva a mio marito'. E gli pose nome Issacar.
\par 19 E Lea concepì ancora, e partorì a Giacobbe un sesto figliuolo.
\par 20 E Lea disse: 'Iddio m'ha dotata di buona dote; questa volta il mio marito abiterà con me, poiché gli ho partorito sei figliuoli'. E gli pose nome Zabulon.
\par 21 Poi partorì una figliuola, e le pose nome Dina.
\par 22 Iddio si ricordò anche di Rachele; Iddio l'esaudì, e la rese feconda;
\par 23 ed ella concepì e partorì un figliuolo, e disse: 'Iddio ha tolto il mio obbrobrio'.
\par 24 E gli pose nome Giuseppe, dicendo: 'L'Eterno m'aggiunga un altro figliuolo'.
\par 25 Or dopo che Rachele ebbe partorito Giuseppe, Giacobbe disse a Labano: 'Dammi licenza, ch'io me ne vada a casa mia, nel mio paese.
\par 26 Dammi le mie mogli, per le quali t'ho servito, e i miei figliuoli; e lasciami andare; poiché tu ben conosci il servizio che t'ho prestato'.
\par 27 E Labano gli disse: 'Se ho trovato grazia dinanzi a te, rimanti; giacché credo indovinare che l'Eterno mi ha benedetto per amor tuo'.
\par 28 Poi disse: 'Fissami il tuo salario, e te lo darò'.
\par 29 Giacobbe gli rispose: 'Tu sai in qual modo io t'ho servito, e quel che sia diventato il tuo bestiame nelle mie mani.
\par 30 Poiché quel che avevi prima ch'io venissi, era poco; ma ora s'è accresciuto oltremodo, e l'Eterno t'ha benedetto dovunque io ho messo il piede. Ora, quando lavorerò io anche per la casa mia?'
\par 31 Labano gli disse: 'Che ti darò io?' E Giacobbe rispose: 'Non mi dar nulla; se acconsenti a quel che sto per dirti, io pascerò di nuovo i tuoi greggi e n'avrò cura.
\par 32 Passerò quest'oggi fra mezzo a tutti i tuoi greggi, mettendo da parte, di fra le pecore, ogni agnello macchiato e vaiolato, e ogni agnello nero; e di fra le capre, le vaiolate e le macchiate. E quello sarà il mio salario.
\par 33 Così, da ora innanzi, il mio diritto risponderà per me nel tuo cospetto, quando verrai ad accertare il mio salario: tutto ciò che non sarà macchiato o vaiolato fra le capre, e nero fra gli agnelli, sarà rubato, se si troverà presso di me'.
\par 34 E Labano disse: 'Ebbene, sia come tu dici!'
\par 35 E quello stesso giorno mise da parte i becchi striati e vaiolati e tutte le capre macchiate e vaiolate, tutto quello che avea del bianco e tutto quel ch'era nero fra gli agnelli, e li affidò ai suoi figliuoli.
\par 36 E Labano frappose la distanza di tre giornate di cammino fra sé e Giacobbe; e Giacobbe pascolava il rimanente de' greggi di Labano.
\par 37 E Giacobbe prese delle verghe verdi di pioppo, di mandorlo e di platano; vi fece delle scortecciature bianche, mettendo allo scoperto il bianco delle verghe.
\par 38 Poi collocò le verghe che avea scortecciate, in vista delle pecore, ne' rigagnoli, negli abbeveratoi dove le pecore venivano a bere; ed entravano in caldo quando venivano a bere.
\par 39 Le pecore dunque entravano in caldo avendo davanti quelle verghe, e figliavano agnelli striati, macchiati e vaiolati.
\par 40 Poi Giacobbe metteva da parte questi agnelli, e faceva volger gli occhi delle pecore verso tutto quello ch'era striato e tutto quel ch'era nero nel gregge di Labano. Egli si formò così dei greggi a parte, che non unì ai greggi di Labano.
\par 41 Or avveniva che, tutte le volte che le pecore vigorose del gregge entravano in caldo, Giacobbe metteva le verghe ne' rigagnoli, in vista delle pecore, perché le pecore entrassero in caldo vicino alle verghe;
\par 42 ma quando le pecore erano deboli, non ve le metteva; così gli agnelli deboli erano di Labano, e i vigorosi di Giacobbe.
\par 43 E quest'uomo diventò ricco oltremodo, ed ebbe greggi numerosi, serve, servi, cammelli e asini.

\chapter{31}

\par 1 Or Giacobbe udì le parole dei figliuoli di Labano, che dicevano: 'Giacobbe ha tolto tutto quello che era di nostro padre; e con quello ch'era di nostro padre, s'è fatto tutta questa ricchezza'.
\par 2 Giacobbe osservò pure il volto di Labano; ed ecco, non era più, verso di lui, quello di prima.
\par 3 E l'Eterno disse a Giacobbe: 'Torna al paese de' tuoi padri e al tuo parentado; e io sarò teco'.
\par 4 E Giacobbe mandò a chiamare Rachele e Lea perché venissero ai campi, presso il suo gregge, e disse loro:
\par 5 'Io vedo che il volto di vostro padre non è più, verso di me, quello di prima; ma l'Iddio di mio padre è stato meco.
\par 6 E voi sapete che io ho servito il padre vostro con tutto il mio potere,
\par 7 mentre vostro padre m'ha ingannato, e ha mutato il mio salario dieci volte; ma Dio non gli ha permesso di farmi del male.
\par 8 Quand'egli diceva: I macchiati saranno il tuo salario, tutto il gregge figliava agnelli macchiati; e quando diceva: Gli striati saranno il tuo salario, tutto il gregge figliava agnelli striati.
\par 9 Così Iddio ha tolto il bestiame a vostro padre, e me l'ha dato.
\par 10 E una volta avvenne, al tempo che le pecore entravano in caldo, ch'io alzai gli occhi, e vidi, in sogno, che i maschi che montavano le femmine, erano striati, macchiati o chiazzati.
\par 11 E l'Angelo di Dio mi disse nel sogno: Giacobbe! E io risposi: Eccomi!
\par 12 Ed egli: Alza ora gli occhi e guarda; tutti i maschi che montano le femmine, sono striati, macchiati o chiazzati; perché ho veduto tutto quel che Labano ti fa.
\par 13 Io son l'Iddio di Bethel, dove tu ungesti un monumento e mi facesti un voto. Ora lèvati, partiti da questo paese, e torna al tuo paese natìo'.
\par 14 Rachele e Lea risposero e gli dissero: 'Abbiam noi forse ancora qualche parte o eredità in casa di nostro padre?
\par 15 Non ci ha egli trattate da straniere, quando ci ha vendute e ha per di più mangiato il nostro danaro?
\par 16 Tutte le ricchezze che Dio ha tolte a nostro padre, sono nostre e dei nostri figliuoli; or dunque, fa' tutto quello che Dio t'ha detto'.
\par 17 Allora Giacobbe si levò, mise i suoi figliuoli e le sue mogli sui cammelli,
\par 18 e menò via tutto il suo bestiame, tutte le sostanze che aveva acquistate, il bestiame che gli apparteneva e che aveva acquistato in Paddan-Aram, per andarsene da Isacco suo padre, nel paese di Canaan.
\par 19 Or mentre Labano se n'era andato a tosare le sue pecore, Rachele rubò gl'idoli di suo padre.
\par 20 Giacobbe si partì furtivamente da Labano, l'Arameo, senza dirgli che voleva fuggire.
\par 21 Così se ne fuggì, con tutto quello che aveva; e si levò, passò il fiume, e si diresse verso il monte di Galaad.
\par 22 Il terzo giorno, fu annunziato a Labano che Giacobbe se n'era fuggito.
\par 23 Allora egli prese seco i suoi fratelli, lo inseguì per sette giornate di cammino, e lo raggiunse al monte di Galaad.
\par 24 Ma Dio venne a Labano l'Arameo, in un sogno della notte, e gli disse: 'Guardati dal parlare a Giacobbe, né in bene né in male'.
\par 25 Labano dunque raggiunse Giacobbe. Or Giacobbe avea piantata la sua tenda sul monte; e anche Labano e i suoi fratelli avean piantato le loro, sul monte di Galaad.
\par 26 Allora Labano disse a Giacobbe: 'Che hai fatto, partendoti da me furtivamente, e menando via le mie figliuole come prigioniere di guerra?
\par 27 Perché te ne sei fuggito di nascosto, e sei partito da me furtivamente, e non m'hai avvertito? Io t'avrei accomiatato con gioia e con canti, a suon di timpano e di cetra.
\par 28 E non m'hai neppur permesso di baciare i miei figliuoli e le mie figliuole! Tu hai agito stoltamente.
\par 29 Ora è in poter mio farvi del male; ma l'Iddio del padre vostro mi parlò la notte scorsa, dicendo: Guardati dal parlare a Giacobbe, né in bene né in male.
\par 30 Ora dunque te ne sei certo andato, perché anelavi alla casa di tuo padre; ma perché hai rubato i miei dèi?'
\par 31 E Giacobbe rispose a Labano: 'Egli è che avevo paura, perché dicevo fra me che tu m'avresti potuto togliere per forza le tue figliuole.
\par 32 Ma chiunque sia colui presso il quale avrai trovato i tuoi dèi, egli deve morire! In presenza dei nostri fratelli, riscontra ciò ch'è tuo fra le cose mie, e prenditelo!' Or Giacobbe ignorava che Rachele avesse rubato gl'idoli.
\par 33 Labano dunque entrò nella tenda di Giacobbe, nella tenda di Lea e nella tenda delle due serve, ma non trovò nulla. E uscito dalla tenda di Lea, entrò nella tenda di Rachele.
\par 34 Or Rachele avea preso gl'idoli, li avea messi nel basto del cammello, e vi s'era posta sopra a sedere. Labano frugò tutta la tenda, e non trovò nulla.
\par 35 Ed ella disse a suo padre: 'Non s'abbia il mio signore a male s'io non posso alzarmi davanti a te, perché ho le solite ricorrenze delle donne'. Ed egli cercò ma non trovò gl'idoli.
\par 36 Allora Giacobbe si adirò e contese con Labano e riprese a dirgli: 'Qual è il mio delitto, qual è il mio peccato, perché tu m'abbia inseguito con tanto ardore?
\par 37 Tu hai frugato tutta la mia roba; che hai trovato di tutta la roba di casa tua? Mettilo qui davanti ai miei e tuoi fratelli, e giudichino loro fra noi due!
\par 38 Ecco vent'anni che sono stato con te; le tue pecore e le tue capre non hanno abortito, e io non ho mangiato i montoni del tuo gregge.
\par 39 Io non t'ho mai portato quel che le fiere aveano squarciato; n'ho subito il danno io; tu mi ridomandavi conto di quello ch'era stato rubato di giorno o rubato di notte.
\par 40 Di giorno, mi consumava il caldo; di notte, il gelo; e il sonno fuggiva dagli occhi miei.
\par 41 Ecco vent'anni che sono in casa tua; t'ho servito quattordici anni per le tue due figliuole, e sei anni per le tue pecore, e tu hai mutato il mio salario dieci volte.
\par 42 Se l'Iddio di mio padre, l'Iddio d'Abrahamo e il Terrore d'Isacco non fosse stato meco, certo, tu m'avresti ora rimandato a vuoto. Iddio ha veduto la mia afflizione e la fatica delle mie mani, e la notte scorsa ha pronunziato la sua sentenza'.
\par 43 E Labano rispose a Giacobbe, dicendo: 'Queste figliuole son mie figliuole, questi figliuoli son miei figliuoli, queste pecore son pecore mie, e tutto quel che vedi è mio. E che posso io fare oggi a queste mie figliuole o ai loro figliuoli ch'esse hanno partorito?
\par 44 Or dunque vieni, facciamo un patto fra me e te, e serva esso di testimonianza fra me e te'.
\par 45 Giacobbe prese una pietra, e la eresse in monumento.
\par 46 E Giacobbe disse ai suoi fratelli: 'Raccogliete delle pietre'. Ed essi presero delle pietre, ne fecero un mucchio, e presso il mucchio mangiarono.
\par 47 E Labano chiamò quel mucchio Jegar-Sahadutha, e Giacobbe lo chiamò Galed.
\par 48 E Labano disse: 'Questo mucchio è oggi testimonio fra me e te'. Perciò fu chiamato Galed,
\par 49 e anche Mitspa, perché Labano disse: 'L'Eterno tenga l'occhio su me e su te quando non ci potremo vedere l'un l'altro.
\par 50 Se tu affliggi le mie figliuole e se prendi altre mogli oltre le mie figliuole, non un uomo sarà con noi; ma, bada, Iddio sarà testimonio fra me e te'.
\par 51 Labano disse ancora a Giacobbe: 'Ecco questo mucchio di pietre, ed ecco il monumento che io ho eretto fra me e te.
\par 52 Sia questo mucchio un testimonio e sia questo monumento un testimonio che io non passerò oltre questo mucchio per andare a te, e che tu non passerai oltre questo mucchio e questo monumento, per far del male.
\par 53 L'Iddio d'Abrahamo e l'Iddio di Nahor, l'Iddio del padre loro, sia giudice fra noi!' E Giacobbe giurò per il Terrore d'Isacco suo padre.
\par 54 Poi Giacobbe offrì un sacrifizio sul monte, e invitò i suoi fratelli a mangiar del pane. Essi dunque mangiarono del pane, e passarono la notte sul monte.
\par 55 La mattina, Labano si levò di buon'ora, baciò i suoi figliuoli e le sue figliuole, e li benedisse. Poi Labano se ne andò, e tornò a casa sua.

\chapter{32}

\par 1 Giacobbe continuò il suo cammino, e gli si fecero incontro degli angeli di Dio.
\par 2 E come Giacobbe li vide, disse: 'Questo è il campo di Dio'; e pose nome a quel luogo Mahanaim.
\par 3 Giacobbe mandò davanti a sé dei messi a Esaù suo fratello, nel paese di Seir, nella campagna di Edom.
\par 4 E dette loro quest'ordine: 'Direte così ad Esaù, mio signore: Così dice il tuo servo Giacobbe: Io ho soggiornato presso Labano e vi sono rimasto fino ad ora;
\par 5 ho buoi, asini, pecore, servi e serve; e lo mando a dire al mio signore, per trovar grazia agli occhi tuoi'.
\par 6 E i messi tornarono a Giacobbe, dicendo: 'Siamo andati dal tuo fratello Esaù, ed eccolo che ti viene incontro con quattrocento uomini'.
\par 7 Allora Giacobbe fu preso da gran paura ed angosciato; divise in due schiere la gente ch'era con lui, i greggi, gli armenti, i cammelli, e disse:
\par 8 'Se Esaù viene contro una delle schiere e la batte, la schiera che rimane potrà salvarsi'.
\par 9 Poi Giacobbe disse: 'O Dio d'Abrahamo mio padre, Dio di mio padre Isacco! O Eterno, che mi dicesti: Torna al tuo paese e al tuo parentado e ti farò del bene,
\par 10 io son troppo piccolo per esser degno di tutte le benignità che hai usate e di tutta la fedeltà che hai dimostrata al tuo servo; poiché io passai questo Giordano col mio bastone, e ora son divenuto due schiere.
\par 11 Liberami, ti prego, dalle mani di mio fratello, dalle mani di Esaù; perché io ho paura di lui e temo che venga e mi dia addosso, non risparmiando né madre né bambini.
\par 12 E tu dicesti: Certo, io ti farò del bene, e farò diventare la tua progenie come la rena del mare, la quale non si può contare da tanta che ce n'è'.
\par 13 Ed egli passò quivi quella notte; e di quello che avea sotto mano prese di che fare un dono al suo fratello Esaù:
\par 14 duecento capre e venti capri, duecento pecore e venti montoni,
\par 15 trenta cammelle allattanti coi loro parti, quaranta vacche e dieci tori, venti asine e dieci puledri.
\par 16 E li consegnò ai suoi servi, gregge per gregge separatamente, e disse ai suoi servi: 'Passate dinanzi a me, e fate che vi sia qualche intervallo fra gregge e gregge'.
\par 17 E dette quest'ordine al primo: 'Quando il mio fratello Esaù t'incontrerà e ti chiederà: Di chi sei? dove vai? a chi appartiene questo gregge che va dinanzi a te?
\par 18 tu risponderai: Al tuo servo Giacobbe, è un dono inviato al mio signore Esaù; ed ecco, egli stesso vien dietro a noi'.
\par 19 E dette lo stesso ordine al secondo, al terzo, e a tutti quelli che seguivano i greggi, dicendo: 'In questo modo parlerete a Esaù, quando lo troverete,
\par 20 e direte: Ecco il tuo servo Giacobbe, che viene egli stesso dietro a noi'. Perché diceva: 'Io lo placherò col dono che mi precede, e, dopo, vedrò la sua faccia; forse, mi farà buona accoglienza'.
\par 21 Così il dono andò innanzi a lui, ed egli passò la notte nell'accampamento.
\par 22 E si levò, quella notte, prese le sue due mogli, le sue due serve, i suoi undici figliuoli, e passò il guado di Iabbok.
\par 23 Li prese, fece loro passare il torrente, e lo fece passare a tutto quello che possedeva.
\par 24 Giacobbe rimase solo, e un uomo lottò con lui fino all'apparir dell'alba.
\par 25 E quando quest'uomo vide che non lo poteva vincere, gli toccò la commessura dell'anca; e la commessura dell'anca di Giacobbe fu slogata, mentre quello lottava con lui.
\par 26 E l'uomo disse: 'Lasciami andare, ché spunta l'alba'. E Giacobbe: 'Non ti lascerò andare prima che tu m'abbia benedetto!'
\par 27 E l'altro gli disse: 'Qual è il tuo nome?' Ed egli rispose: 'Giacobbe'.
\par 28 E quello disse: 'Il tuo nome non sarà più Giacobbe, ma Israele, poiché tu hai lottato con Dio e con gli uomini, ed hai vinto'.
\par 29 E Giacobbe gli chiese: 'Deh, palesami il tuo nome'. E quello rispose: 'Perché mi chiedi il mio nome?'
\par 30 E lo benedisse quivi. E Giacobbe chiamò quel luogo Peniel, 'perché', disse, 'ho veduto Iddio a faccia a faccia, e la mia vita è stata risparmiata'.
\par 31 Il sole si levava com'egli ebbe passato Peniel; e Giacobbe zoppicava dell'anca.
\par 32 Per questo, fino al dì d'oggi, gl'Israeliti non mangiano il nervo della coscia che passa per la commessura dell'anca, perché quell'uomo avea toccato la commessura dell'anca di Giacobbe, al punto del nervo della coscia.

\chapter{33}

\par 1 Giacobbe alzò gli occhi, guardò, ed ecco Esaù che veniva, avendo seco quattrocento uomini. Allora divise i figliuoli fra Lea, Rachele e le due serve.
\par 2 E mise davanti le serve e i loro figliuoli, poi Lea e i suoi figliuoli, e da ultimo Rachele e Giuseppe.
\par 3 Ed egli stesso passò dinanzi a loro, s'inchinò fino a terra sette volte, finché si fu avvicinato al suo fratello.
\par 4 Ed Esaù gli corse incontro, l'abbracciò, gli si gettò al collo, e lo baciò: e piansero.
\par 5 Poi Esaù, alzando gli occhi, vide le donne e i suoi fanciulli, e disse: 'Chi sono questi qui che hai teco?' Giacobbe rispose: 'Sono i figliuoli che Dio s'è compiaciuto di dare al tuo servo'.
\par 6 Allora le serve s'accostarono, esse e i loro figliuoli, e s'inchinarono.
\par 7 S'accostarono anche Lea e i suoi figliuoli, e s'inchinarono. Poi s'accostarono Giuseppe e Rachele, e s'inchinarono.
\par 8 Ed Esaù disse: 'Che ne vuoi fare di tutta quella schiera che ho incontrata?' Giacobbe rispose: 'È per trovar grazia agli occhi del mio signore'.
\par 9 Ed Esaù: 'Io ne ho assai della roba, fratel mio; tienti per te ciò ch'è tuo'.
\par 10 Ma Giacobbe disse: 'No, ti prego; se ho trovato grazia agli occhi tuoi, accetta il dono dalla mia mano, giacché io ho veduto la tua faccia, come uno vede la faccia di Dio, e tu m'hai fatto gradevole accoglienza.
\par 11 Deh, accetta il mio dono che t'è stato recato; poiché Iddio m'ha usato grande bontà, e io ho di tutto'. E insisté tanto, che Esaù l'accettò.
\par 12 Poi Esaù disse: 'Partiamo, incamminiamoci, e io andrò innanzi a te'.
\par 13 E Giacobbe rispose: 'Il mio signore sa che i fanciulli son di tenera età, e che ho con me delle pecore e delle vacche che allattano; se si forzassero per un giorno solo a camminare, le bestie morrebbero tutte.
\par 14 Deh, passi il mio signore innanzi al suo servo; e io me ne verrò pian piano, al passo del bestiame che mi precederà, e al passo de' fanciulli, finché arrivi presso al mio signore, a Seir'.
\par 15 Ed Esaù disse: 'Permetti almeno ch'io lasci con te un po' della gente che ho meco'. Ma Giacobbe rispose: 'E perché questo? Basta ch'io trovi grazia agli occhi del mio signore'.
\par 16 Così Esaù, in quel giorno stesso, rifece il cammino verso Seir.
\par 17 Giacobbe partì alla volta di Succoth e edificò una casa per sé, e fece delle capanne per il suo bestiame; e per questo quel luogo fu chiamato Succoth.
\par 18 Poi Giacobbe, tornando da Paddan-Aram, arrivò sano e salvo alla città di Sichem, nel paese di Canaan, e piantò le tende dirimpetto alla città.
\par 19 E comprò dai figliuoli di Hemor, padre di Sichem, per cento pezzi di danaro, la parte del campo dove avea piantato le sue tende.
\par 20 Ed eresse quivi un altare, e lo chiamò El-Elohè-Israel.

\chapter{34}

\par 1 Or Dina, la figliuola che Lea aveva partorito a Giacobbe, uscì per vedere le figliuole del paese.
\par 2 E Sichem, figliuolo di Hemor lo Hivveo, principe del paese, vedutala, la rapì, si giacque con lei, e la violentò.
\par 3 E l'anima sua s'appassionò per Dina, figliuola di Giacobbe; egli amò la fanciulla, e parlò al cuore di lei.
\par 4 Poi disse a Hemor suo padre: 'Dammi questa fanciulla per moglie'.
\par 5 Or Giacobbe udì ch'egli avea disonorato la sua figliuola Dina; e come i suoi figliuoli erano ai campi col suo bestiame, Giacobbe si tacque finché non furon tornati.
\par 6 E Hemor, padre di Sichem, si recò da Giacobbe per parlargli.
\par 7 E i figliuoli di Giacobbe, com'ebbero udito il fatto, tornarono dai campi; e questi uomini furono addolorati e fortemente adirati perché costui aveva commessa un'infamia in Israele, giacendosi con la figliuola di Giacobbe: cosa che non era da farsi.
\par 8 Ed Hemor parlò loro, dicendo: 'L'anima del mio figliuolo Sichem s'è unita strettamente alla vostra figliuola; deh, dategliela per moglie;
\par 9 e imparentatevi con noi; dateci le vostre figliuole, e prendetevi le figliuole nostre.
\par 10 Voi abiterete con noi, e il paese sarà a vostra disposizione; dimoratevi, trafficatevi, e acquistatevi delle proprietà'.
\par 11 Allora Sichem disse al padre e ai fratelli di Dina: 'Fate ch'io trovi grazia agli occhi vostri, e vi darò quel che mi direte.
\par 12 Imponetemi pure una gran dote e di gran doni; e io ve li darò come mi direte; ma datemi la fanciulla per moglie'.
\par 13 I figliuoli di Giacobbe risposero a Sichem e ad Hemor suo padre, e parlarono loro con astuzia, perché Sichem avea disonorato Dina loro sorella;
\par 14 e dissero loro: 'Questa cosa non la possiamo fare; non possiam dare la nostra sorella a uno che non è circonciso; giacché questo, per noi, sarebbe un obbrobrio.
\par 15 Soltanto a questa condizione acconsentiremo alla vostra richiesta: se vorrete essere come siam noi, circoncidendo ogni maschio tra voi.
\par 16 Allora vi daremo le nostre figliuole, e noi ci prenderemo le figliuole vostre; abiteremo con voi, e diventeremo un popolo solo.
\par 17 Ma se non ci volete ascoltare e non vi volete far circoncidere, noi prenderemo la nostra fanciulla e ce ne andremo'.
\par 18 Le loro parole piacquero ad Hemor e Sichem figliuolo di Hemor.
\par 19 E il giovane non indugiò a fare la cosa, perché portava affezione alla figliuola di Giacobbe, ed era l'uomo più onorato in tutta la casa di suo padre.
\par 20 Hemor e Sichem, suo figliuolo, vennero alla porta della loro città, e parlarono alla gente della loro città, dicendo:
\par 21 'Questa è gente pacifica, qui tra noi; rimanga dunque pure nel paese, e vi traffichi; poiché, ecco, il paese è abbastanza ampio per loro. Noi prenderemo le loro figliuole per mogli, e daremo loro le nostre.
\par 22 Ma soltanto a questa condizione questa gente acconsentirà ad abitare con noi per formare un popolo solo: che ogni maschio fra noi sia circonciso, come son circoncisi loro.
\par 23 Il loro bestiame, le loro sostanze, tutti i loro animali non saran nostri? Acconsentiamo alla loro domanda ed essi abiteranno con noi'.
\par 24 E tutti quelli che uscivano dalla porta della città diedero ascolto ad Hemor e a Sichem suo figliuolo; e ogni maschio fu circonciso: ognuno di quelli che uscivano dalla porta della città.
\par 25 Or avvenne che il terzo giorno, mentre quelli eran sofferenti, due de' figliuoli di Giacobbe, Simeone e Levi, fratelli di Dina, presero ciascuno la propria spada, assalirono la città che si tenea sicura, e uccisero tutti i maschi.
\par 26 Passarono anche a fil di spada Hemor e Sichem suo figliuolo, presero Dina dalla casa di Sichem, e uscirono.
\par 27 I figliuoli di Giacobbe si gettarono sugli uccisi e saccheggiarono la città, perché la loro sorella era stata disonorata;
\par 28 presero i loro greggi, i loro armenti, i loro asini, quello che era in città, e quello che era per i campi,
\par 29 e portaron via come bottino tutte le loro ricchezze, tutti i loro piccoli bambini, le loro mogli, e tutto quello che si trovava nelle case.
\par 30 Allora Giacobbe disse a Simeone ed a Levi: 'Voi mi date grande affanno, mettendomi in cattivo odore presso gli abitanti del paese, presso i Cananei ed i Ferezei. Ed io non ho che poca gente; essi si raduneranno contro di me e mi daranno addosso, e sarò distrutto: io con la mia casa'.
\par 31 Ed essi risposero: 'Dovrà la nostra sorella esser trattata come una meretrice?'

\chapter{35}

\par 1 Iddio disse a Giacobbe: 'Lèvati, vattene a Bethel, dimora quivi, e fa' un altare all'Iddio che ti apparve, quando fuggivi dinanzi al tuo fratello Esaù'.
\par 2 Allora Giacobbe disse alla sua famiglia e a tutti quelli ch'erano con lui: 'Togliete gli dèi stranieri che sono fra voi, purificatevi, e cambiatevi i vestiti;
\par 3 e leviamoci, andiamo a Bethel, ed io farò quivi un altare all'Iddio che mi esaudì nel giorno della mia angoscia, e ch'è stato con me nel viaggio che ho fatto'.
\par 4 Ed essi dettero a Giacobbe tutti gli dèi stranieri ch'erano nelle loro mani e gli anelli che avevano agli orecchi; e Giacobbe li nascose sotto la quercia ch'è presso a Sichem.
\par 5 Poi si partirono; e un terrore mandato da Dio invase le città ch'erano intorno a loro; talché non inseguirono i figliuoli di Giacobbe.
\par 6 Così Giacobbe giunse a Luz, cioè Bethel, ch'è nel paese di Canaan: egli con tutta la gente che avea seco;
\par 7 ed edificò quivi un altare, e chiamò quel luogo El-Bethel, perché quivi Iddio gli era apparso, quando egli fuggiva dinanzi a suo fratello.
\par 8 Allora morì Debora, balia di Rebecca, e fu sepolta al di sotto di Bethel, sotto la quercia, che fu chiamata Allon-Bacuth.
\par 9 Iddio apparve ancora a Giacobbe, quando questi veniva da Paddan-Aram; e lo benedisse.
\par 10 E Dio gli disse: 'Il tuo nome è Giacobbe; tu non sarai più chiamato Giacobbe, ma il tuo nome sarà Israele'. E gli mise nome Israele.
\par 11 E Dio gli disse: 'Io sono l'Iddio onnipotente; sii fecondo e moltiplica; una nazione, anzi una moltitudine di nazioni discenderà da te, e dei re usciranno dai tuoi lombi;
\par 12 e darò a te e alla tua progenie dopo di te il paese che detti ad Abrahamo e ad Isacco'.
\par 13 E Dio risalì di presso a lui, dal luogo dove gli avea parlato.
\par 14 E Giacobbe eresse un monumento di pietra nel luogo dove Iddio gli avea parlato; vi fece sopra una libazione e vi sparse su dell'olio.
\par 15 E Giacobbe chiamò Bethel il luogo dove Dio gli avea parlato.
\par 16 Poi partirono da Bethel; e c'era ancora qualche distanza per arrivare ad Efrata, quando Rachele partorì. Essa ebbe un duro parto;
\par 17 e mentre penava a partorire, la levatrice le disse: 'Non temere, perché eccoti un altro figliuolo'.
\par 18 E com'ella stava per rendere l'anima (perché morì), pose nome al bimbo Ben-Oni; ma il padre lo chiamò Beniamino.
\par 19 E Rachele morì, e fu sepolta sulla via di Efrata; cioè di Bethlehem.
\par 20 E Giacobbe eresse un monumento sulla tomba di lei. Questo è il monumento della tomba di Rachele, il quale esiste tuttora.
\par 21 Poi Israele si partì, e piantò la sua tenda al di là di Migdal-Eder.
\par 22 E avvenne che, mentre Israele abitava in quel paese, Ruben andò e si giacque con Bilha, concubina di suo padre. E Israele lo seppe.
\par 23 Or i figliuoli di Giacobbe erano dodici. I figliuoli di Lea: Ruben, primogenito di Giacobbe, Simeone, Levi, Giuda, Issacar, Zabulon.
\par 24 I figliuoli di Rachele: Giuseppe e Beniamino.
\par 25 I figliuoli di Bilha, serva di Rachele: Dan e Neftali.
\par 26 I figliuoli di Zilpa, serva di Lea: Gad e Ascer. Questi sono i figliuoli di Giacobbe che gli nacquero in Paddan-Aram.
\par 27 E Giacobbe venne da Isacco suo padre a Mamre, a Kiriath-Arba, cioè Hebron, dove Abrahamo e Isacco aveano soggiornato.
\par 28 E i giorni d'Isacco furono centottant'anni.
\par 29 E Isacco spirò, morì, e fu raccolto presso il suo popolo, vecchio e sazio di giorni; ed Esaù e Giacobbe, suoi figliuoli, lo seppellirono.

\chapter{36}

\par 1 Questa è la posterità di Esaù, cioè Edom.
\par 2 Esaù prese le sue mogli tra le figliuole de' Cananei: Ada, figliuola di Elon, lo Hitteo; Oholibama, figliuola di Ana,
\par 3 figliuola di Tsibeon, lo Hivveo; e Basmath, figliuola d'Ismaele, sorella di Nebaioth.
\par 4 Ada partorì ad Esaù Elifaz;
\par 5 Basmath partorì Reuel; e Oholibama partorì Ieush, Ialam e Korah. Questi sono i figliuoli di Esaù, che gli nacquero nel paese di Canaan.
\par 6 Esaù prese le sue mogli, i suoi figliuoli, le sue figliuole, tutte le persone della sua casa, i suoi greggi, tutto il suo bestiame e tutti i beni che aveva messi assieme nel paese di Canaan, e se ne andò in un altro paese, lontano da Giacobbe suo fratello;
\par 7 giacché i loro beni erano troppo grandi perch'essi potessero dimorare assieme; e il paese nel quale soggiornavano, non era loro sufficiente a motivo del loro bestiame.
\par 8 Ed Esaù abitò sulla montagna di Seir. Esaù è Edom.
\par 9 Questa è la posterità di Esaù, padre degli Edomiti, sulla montagna di Seir.
\par 10 Questi sono i nomi dei figliuoli di Esaù: Elifaz, figliuolo di Ada, moglie di Esaù; Reuel figliuolo di Basmath, moglie di Esaù.
\par 11 I figliuoli di Elifaz furono: Teman, Omar, Tsefo, Gatam e Kenaz.
\par 12 Timna era la concubina di Elifaz, figliuolo di Esaù; essa partorì ad Elifaz Amalek. Questi furono i figliuoli di Ada, moglie di Esaù.
\par 13 E questi furono i figliuoli di Reuel: Nahath e Zerach, Shammah e Mizza. Questi furono i figliuoli di Basmath, moglie di Esaù.
\par 14 E questi furono i figliuoli di Oholibama, figliuola di Ana, figliuola di Tsibeon, moglie di Esaù; essa partorì a Esaù: Ieush, Ialam e Korah.
\par 15 Questi sono i capi de' figliuoli di Esaù: Figliuoli di Elifaz, primogenito di Esaù: il capo Teman, il capo Omar, il capo Tsefo, il capo Kenaz,
\par 16 il capo Korah, il capo Gatam, il capo Amalek; questi sono i capi discesi da Elifaz, nel paese di Edom. E sono i figliuoli di Ada.
\par 17 E questi sono i figliuoli di Reuel, figliuolo di Esaù: il capo Nahath, il capo Zerach, il capo Shammah, il capo Mizza; questi sono i capi discesi da Reuel, nel paese di Edom. E sono i figliuoli di Basmath, moglie di Esaù.
\par 18 E questi sono i figliuoli di Oholibama, moglie di Esaù: il capo Ieush, il capo Ialam, il capo Korah; questi sono i capi discesi da Oholibama, figliuola di Ana, moglie di Esaù.
\par 19 Questi sono i figliuoli di Esaù, che è Edom, e questi sono i loro capi.
\par 20 Questi sono i figliuoli di Seir lo Horeo, che abitavano il paese: Lothan, Shobal, Tsibeon,
\par 21 Ana, Dishon, Etser e Dishan. Questi sono i capi degli Horei, figliuoli di Seir, nel paese di Edom.
\par 22 I figliuoli di Lothan furono: Hori e Hemam; e la sorella di Lothan fu Timna.
\par 23 E questi sono i figliuoli di Shobal: Alvan, Manahath, Ebal, Scefo e Onam.
\par 24 E questi sono i figliuoli di Tsibeon: Aiah e Ana. Questo è quell'Ana che trovò le acque calde nel deserto, mentre pasceva gli asini di Tsibeon suo padre.
\par 25 E questi sono i figliuoli di Ana: Dishon e Oholibama, figliuola di Ana.
\par 26 E questi sono i figliuoli di Dishon: Hemdan, Eshban, Iithran e Keran.
\par 27 Questi sono i figliuoli di Etser: Bilhan, Zaavan e Akan.
\par 28 Questi sono i figliuoli di Dishan: Uts e Aran.
\par 29 Questi sono i capi degli Horei: il capo Lothan, il capo Shobal, il capo Tsibeon, il capo Ana,
\par 30 il capo Dishon, il capo Etser, il capo Dishan. Questi sono i capi degli Horei, i capi ch'essi ebbero nel paese di Seir.
\par 31 Questi sono i re che regnarono nel paese di Edom, prima che alcun re regnasse sui figliuoli d'Israele:
\par 32 Bela, figliuolo di Beor, regnò in Edom, e il nome della sua città fu Dinhaba.
\par 33 Bela morì, e Iobab, figliuolo di Zerach, di Botsra, regnò in luogo suo.
\par 34 Iobab morì, e Husham, nel paese de' Temaniti, regnò in luogo suo.
\par 35 Husham morì, e Hadad, figliuolo di Bedad, che sconfisse i Madianiti ne' campi di Moab, regnò in luogo suo; e il nome della sua città fu Avith.
\par 36 Hadad morì, e Samla, di Masreka, regnò in luogo suo.
\par 37 Samla morì, e Saul di Rehoboth sul Fiume, regnò in luogo suo.
\par 38 Saul morì, e Baal-Hanan, figliuolo di Acbor, regnò in luogo suo.
\par 39 Baal-Hanan, figliuolo di Acbor, morì, e Hadar regnò in luogo suo. Il nome della sua città fu Pau, e il nome della sua moglie, Mehetabeel, figliuola di Matred, figliuola di Mezahab.
\par 40 E questi sono i nomi dei capi di Esaù, secondo le loro famiglie, secondo i loro territori, coi loro nomi: il capo Timna, il capo Alva, il capo Jeteth,
\par 41 il capo Oholibama, il capo Ela,
\par 42 il capo Pinon, il capo Kenaz, il capo Teman, il capo Mibtsar, il capo Magdiel, il capo Iram.
\par 43 Questi sono i capi di Edom secondo le loro dimore, nel paese che possedevano. Questo è Esaù, il padre degli Edomiti.

\chapter{37}

\par 1 Or Giacobbe dimorò nel paese dove suo padre avea soggiornato, nel paese di Canaan.
\par 2 E questa è la posterità di Giacobbe. Giuseppe, all'età di diciassette anni, pasceva il gregge coi suoi fratelli; e, giovinetto com'era, stava coi figliuoli di Bilha e coi figliuoli di Zilpa, mogli di suo padre. E Giuseppe riferì al loro padre la mala fama che circolava sul loro conto.
\par 3 Or Israele amava Giuseppe più di tutti gli altri suoi figliuoli, perché era il figlio della sua vecchiaia; e gli fece una veste lunga con le maniche.
\par 4 E i suoi fratelli, vedendo che il loro padre l'amava più di tutti gli altri fratelli, l'odiavano, e non gli potevan parlare amichevolmente.
\par 5 Or Giuseppe ebbe un sogno, e lo raccontò ai suoi fratelli; e questi l'odiaron più che mai.
\par 6 Egli disse loro: 'Udite, vi prego, il sogno che ho fatto.
\par 7 Noi stavamo legando dei covoni in mezzo ai campi, quand'ecco che il mio covone si levò su e si tenne ritto; ed ecco i covoni vostri farsi d'intorno al mio covone, e inchinarglisi dinanzi'.
\par 8 Allora i suoi fratelli gli dissero: 'Dovrai tu dunque regnare su di noi? o dominarci?' E l'odiarono più che mai a motivo dei suoi sogni e delle sue parole.
\par 9 Egli ebbe ancora un altro sogno, e lo raccontò ai suoi fratelli, dicendo: 'Ho avuto un altro sogno! Ed ecco che il sole, la luna e undici stelle mi s'inchinavano dinanzi'.
\par 10 Ei lo raccontò a suo padre e ai suoi fratelli; e suo padre lo sgridò, e gli disse: 'Che significa questo sogno che hai avuto? Dovremo dunque io e tua madre e i tuoi fratelli venir proprio a inchinarci davanti a te fino a terra?'
\par 11 E i suoi fratelli gli portavano invidia, ma suo padre serbava dentro di sé queste parole.
\par 12 Or i fratelli di Giuseppe erano andati a pascere il gregge del padre a Sichem.
\par 13 E Israele disse a Giuseppe: 'I tuoi fratelli non sono forse alla pastura a Sichem? Vieni, che ti manderò da loro'. Ed egli rispose: 'Eccomi'.
\par 14 Israele gli disse: 'Va' a vedere se i tuoi fratelli stanno bene, e se tutto va bene col gregge; e torna a dirmelo'. Così lo mandò dalla valle di Hebron, e Giuseppe arrivò a Sichem.
\par 15 E un uomo lo trovò che andava errando per i campi e quest'uomo lo interrogò, dicendo: 'Che cerchi?'
\par 16 Egli rispose: 'Cerco i miei fratelli; deh, dimmi dove siano a pascere il gregge'.
\par 17 E quell'uomo gli disse: 'Son partiti di qui, perché li ho uditi che dicevano: Andiamocene a Dotan'. Giuseppe andò quindi in traccia de' suoi fratelli, e li trovò a Dotan.
\par 18 Essi lo scorsero da lontano; e prima ch'egli fosse loro vicino, macchinarono d'ucciderlo.
\par 19 E dissero l'uno all'altro: 'Ecco cotesto sognatore che viene!
\par 20 Ora dunque venite, uccidiamolo, e gettiamolo in una di queste cisterne; diremo poi che una mala bestia l'ha divorato, e vedremo che ne sarà de' suoi sogni'.
\par 21 Ruben udì questo, e lo liberò dalle loro mani. Disse: 'Non gli togliamo la vita'.
\par 22 Poi Ruben aggiunse: 'Non spargete sangue; gettatelo in quella cisterna ch'è nel deserto, ma non lo colpisca la vostra mano'. Diceva così, per liberarlo dalle loro mani e restituirlo a suo padre.
\par 23 Quando Giuseppe fu giunto presso i suoi fratelli, lo spogliarono della sua veste, della veste lunga con le maniche che aveva addosso;
\par 24 lo presero e lo gettarono nella cisterna. Or la cisterna era vuota; non c'era punt'acqua.
\par 25 Poi si misero a sedere per prender cibo; e avendo alzati gli occhi, ecco che videro una carovana d'Ismaeliti, che veniva da Galaad, coi suoi cammelli carichi di aromi, di balsamo e di mirra, che portava in Egitto.
\par 26 E Giuda disse ai suoi fratelli: 'Che guadagneremo a uccidere il nostro fratello e a nascondere il suo sangue?
\par 27 Venite, vendiamolo agl'Ismaeliti, e non lo colpisca la nostra mano, poiché è nostro fratello, nostra carne'. E i suoi fratelli gli diedero ascolto.
\par 28 E come que' mercanti Madianiti passavano, essi trassero e fecero salire Giuseppe su dalla cisterna, e lo vendettero per venti sicli d'argento a quegl'Ismaeliti. E questi menarono Giuseppe in Egitto.
\par 29 Or Ruben tornò alla cisterna; ed ecco, Giuseppe non era più nella cisterna. Allora egli si stracciò le vesti,
\par 30 tornò dai suoi fratelli, e disse: 'Il fanciullo non c'è più; e io, dove andrò io?'
\par 31 Essi presero la veste di Giuseppe, scannarono un becco, e intrisero del sangue la veste.
\par 32 Poi mandarono uno a portare al padre loro la veste lunga con le maniche, e gli fecero dire: 'Abbiam trovato questa veste; vedi tu se sia quella del tuo figliuolo, o no'.
\par 33 Ed egli la riconobbe e disse: 'È la veste del mio figliuolo; una mala bestia l'ha divorato; per certo, Giuseppe è stato sbranato'.
\par 34 E Giacobbe si stracciò le vesti, si mise un cilicio sui fianchi, e fece cordoglio del suo figliuolo per molti giorni.
\par 35 E tutti i suoi figliuoli e tutte le sue figliuole vennero a consolarlo; ma egli rifiutò d'esser consolato, e disse: 'Io scenderò, facendo cordoglio, dal mio figliuolo, nel soggiorno de' morti'. E suo padre lo pianse.
\par 36 E que' Madianiti lo vendettero in Egitto a Potifar, ufficiale di Faraone, capitano delle guardie.

\chapter{38}

\par 1 Or avvenne che, in quel tempo, Giuda discese di presso ai suoi fratelli, e andò a stare da un uomo di Adullam, che avea nome Hira.
\par 2 E Giuda vide quivi la figliuola di un Cananeo, chiamato Shua; e se la prese, e convisse con lei.
\par 3 Ed ella concepì e partorì un figliuolo, al quale egli pose nome Er.
\par 4 Poi ella concepì di nuovo, e partorì un figliuolo, al quale pose nome Onan.
\par 5 E partorì ancora un figliuolo, al quale pose nome Scela. Or Giuda era a Kezib, quand'ella lo partorì.
\par 6 E Giuda prese per Er, suo primogenito, una moglie che avea nome Tamar.
\par 7 Ma Er, primogenito di Giuda, era perverso agli occhi dell'Eterno e l'Eterno lo fece morire.
\par 8 Allora Giuda disse a Onan: 'Va' dalla moglie del tuo fratello, prenditela come cognato, e suscita una progenie al tuo fratello'.
\par 9 E Onan, sapendo che quella progenie non sarebbe sua, quando s'accostava alla moglie del suo fratello, faceva in modo d'impedire il concepimento, per non dar progenie al fratello.
\par 10 Ciò ch'egli faceva dispiacque all'Eterno, il quale fece morire anche lui.
\par 11 Allora Giuda disse a Tamar sua nuora: 'Rimani vedova in casa di tuo padre, finché Scela, mio figliuolo, sia cresciuto'. Perché diceva: 'Badiamo che anch'egli non muoia per i suoi fratelli'. E Tamar se ne andò, e dimorò in casa di suo padre.
\par 12 Passaron molti giorni, e morì la figliuola di Shua, moglie di Giuda; e dopo che Giuda si fu consolato, salì da quelli che tosavan le sue pecore a Timna; egli col suo amico Hira, l'Adullamita.
\par 13 Di questo fu informata Tamar, e le fu detto: 'Ecco, il tuo suocero sale a Timna a tosare le sue pecore'.
\par 14 Allora ella si tolse le vesti da vedova, si coprì d'un velo, se ne avvolse tutta, e si pose a sedere alla porta di Enaim, ch'è sulla via di Timna; poiché vedeva che Scela era cresciuto, e nondimeno, lei non gli era stata data per moglie.
\par 15 Come Giuda la vide, la prese per una meretrice, perch'essa aveva il viso coperto.
\par 16 E accostatosi a lei sulla via, le disse: 'Lasciami venire da te!' Poiché non sapeva ch'ella fosse sua nuora. Ed ella rispose: 'Che mi darai per venire da me?'
\par 17 Ed egli le disse: 'Ti manderò un capretto del mio gregge'. Ed ella: 'Mi darai tu un pegno finché tu me l'abbia mandato?'
\par 18 Ed egli: 'Che pegno ti darò?' E l'altra rispose: 'Il tuo sigillo, il tuo cordone e il bastone che hai in mano'. Egli glieli dette, andò da lei, ed ella rimase incinta di lui.
\par 19 Poi ella si levò, e se ne andò; si tolse il velo, e si rimise le vesti da vedova.
\par 20 E Giuda mandò il capretto per mezzo del suo amico, l'Adullamita, affin di ritirare il pegno di man di quella donna; ma egli non la trovò.
\par 21 Interrogò la gente del luogo, dicendo: 'Dov'è quella meretrice che stava a Enaim, sulla via?' E quelli risposero: 'Qui non c'è stata alcuna meretrice'.
\par 22 Ed egli se ne tornò a Giuda, e gli disse: 'Non l'ho trovata; e, per di più, la gente del luogo m'ha detto: Qui non c'è stata alcuna meretrice'.
\par 23 E Giuda disse: 'Si tenga pure il pegno, che non abbiamo a incorrere nel disprezzo. Ecco, io ho mandato questo capretto, e tu non l'hai trovata'.
\par 24 Or circa tre mesi dopo, vennero a dire a Giuda: 'Tamar, tua nuora, si è prostituita; e, per di più, eccola incinta in seguito alla sua prostituzione'. E Giuda disse: 'Menatela fuori, e sia arsa!'
\par 25 Come la menavano fuori, ella mandò a dire al suocero: 'Sono incinta dell'uomo al quale appartengono queste cose'. E disse: 'Riconosci, ti prego, di chi siano questo sigillo, questi cordoni e questo bastone'.
\par 26 Giuda li riconobbe, e disse: 'Ella è più giusta di me, giacché io non l'ho data a Scela, mio figliuolo'. Ed egli non ebbe più relazioni con lei.
\par 27 Or quando venne il tempo in cui doveva partorire, ecco ch'essa aveva in seno due gemelli.
\par 28 E mentre partoriva, l'un d'essi mise fuori una mano; e la levatrice la prese, e vi legò un filo di scarlatto, dicendo: 'Questo qui esce il primo'.
\par 29 Ma egli ritirò la mano, ed ecco uscir fuori il suo fratello. Allora la levatrice disse: 'Perché ti sei fatta questa breccia?' Per questo motivo gli fu messo nome Perets.
\par 30 Poi uscì il suo fratello, che aveva alla mano il filo di scarlatto; e fu chiamato Zerach.

\chapter{39}

\par 1 Giuseppe fu menato in Egitto; e Potifar, ufficiale di Faraone, capitano delle guardie, un Egiziano, lo comprò da quegl'Ismaeliti, che l'avevano menato quivi.
\par 2 E l'Eterno fu con Giuseppe, il quale prosperava e stava in casa del suo signore, l'Egiziano.
\par 3 E il suo signore vide che l'Eterno era con lui, e che l'Eterno gli faceva prosperare nelle mani tutto quello che intraprendeva.
\par 4 Giuseppe entrò nelle grazie di lui, e attendeva al servizio personale di Potifar, il quale lo fece maggiordomo della sua casa, e gli mise nelle mani tutto quello che possedeva.
\par 5 E da che l'ebbe fatto maggiordomo della sua casa e gli ebbe affidato tutto quello che possedeva, l'Eterno benedisse la casa dell'Egiziano, per amor di Giuseppe; e la benedizione dell'Eterno riposò su tutto quello ch'egli possedeva, in casa e in campagna.
\par 6 Potifar lasciò tutto quello che aveva nelle mani di Giuseppe; e non s'occupava più di cosa alcuna, tranne del suo proprio cibo. - Or Giuseppe era di presenza avvenente e di bell'aspetto.
\par 7 Dopo queste cose avvenne che la moglie del signore di Giuseppe gli mise gli occhi addosso, e gli disse: 'Giaciti meco'.
\par 8 Ma egli rifiutò e disse alla moglie del suo signore: 'Ecco, il mio signore non s'informa da me di nulla ch'è nella casa, e ha messo nelle mie mani tutto quello che ha;
\par 9 egli stesso non è più grande di me in questa casa; e nulla mi ha divietato, tranne che te, perché sei sua moglie. Come dunque potrei io fare questo gran male e peccare contro Dio?'
\par 10 E bench'ella gliene parlasse ogni giorno, Giuseppe non acconsentì, né a giacersi né a stare con lei.
\par 11 Or avvenne che un giorno egli entrò in casa per fare il suo lavoro; e non c'era quivi alcuno della gente di casa;
\par 12 ed essa lo afferrò per la veste, e gli disse: 'Giaciti meco'. Ma egli le lasciò in mano la veste, e fuggì fuori.
\par 13 E quand'ella vide ch'egli le aveva lasciata la veste in mano e ch'era fuggito fuori,
\par 14 chiamò la gente della sua casa, e le parlò così: 'Vedete, ei ci ha menato in casa un Ebreo per pigliarsi giuoco di noi; esso è venuto da me per giacersi meco, ma io ho gridato a gran voce.
\par 15 E com'egli ha udito ch'io alzavo la voce e gridavo, m'ha lasciato qui la sua veste, ed è fuggito fuori'.
\par 16 E si tenne accanto la veste di lui, finché il suo signore non fu tornato a casa.
\par 17 Allora ella gli parlò in questa maniera: 'Quel servo ebreo che tu ci hai menato, venne da me per pigliarsi giuoco di me.
\par 18 Ma com'io ho alzato la voce e ho gridato, egli m'ha lasciato qui la sua veste e se n'è fuggito fuori'.
\par 19 Quando il signore di Giuseppe ebbe intese le parole di sua moglie che gli diceva: 'Il tuo servo m'ha fatto questo!' l'ira sua s'infiammò.
\par 20 E il signore di Giuseppe lo prese e lo mise nella prigione, nel luogo ove si tenevano chiusi i carcerati del re. Egli fu dunque là in quella prigione.
\par 21 Ma l'Eterno fu con Giuseppe, e spiegò a pro di lui la sua benignità, cattivandogli le grazie del governatore della prigione.
\par 22 E il governatore della prigione affidò alla sorveglianza di Giuseppe tutti i detenuti ch'erano nella carcere; e nulla si faceva quivi senza di lui.
\par 23 Il governatore della prigione non rivedeva niente di quello ch'era affidato a lui, perché l'Eterno era con lui, e l'Eterno faceva prosperare tutto quello ch'egli intraprendeva.

\chapter{40}

\par 1 Or, dopo queste cose, avvenne che il coppiere e il panattiere del re d'Egitto offesero il loro signore, il re d'Egitto.
\par 2 E Faraone s'indignò contro i suoi due ufficiali, contro il capo de' coppieri e il capo de' panattieri,
\par 3 e li fece mettere in carcere, nella casa del capo delle guardie; nella prigione stessa dove Giuseppe stava rinchiuso.
\par 4 E il capitano delle guardie li affidò alla sorveglianza di Giuseppe, il quale li serviva. Ed essi rimasero in prigione per un certo tempo.
\par 5 E durante una medesima notte, il coppiere e il panattiere del re d'Egitto, ch'erano rinchiusi nella prigione, ebbero ambedue un sogno, un sogno per uno, e ciascun sogno aveva il suo significato particolare.
\par 6 Giuseppe, venuto la mattina da loro, li guardò, ed ecco, erano conturbati.
\par 7 E interrogò gli ufficiali di Faraone ch'eran con lui in prigione nella casa del suo signore, e disse: 'Perché avete oggi il viso così mesto?'
\par 8 E quelli gli risposero: 'Abbiamo fatto un sogno e non v'è alcuno che ce lo interpreti'. E Giuseppe disse loro: 'Le interpretazioni non appartengono a Dio? Raccontatemi i sogni, vi prego'.
\par 9 E il capo de' coppieri raccontò il suo sogno a Giuseppe, e gli disse: 'Nel mio sogno, ecco, mi stava davanti una vite;
\par 10 e in quella vite c'eran tre tralci; e mi pareva ch'essa germogliasse, poi fiorisse, e desse in fine dei grappoli d'uva matura.
\par 11 E io avevo in mano la coppa di Faraone; presi l'uva, la spremei nella coppa di Faraone, e diedi la coppa in mano a Faraone'.
\par 12 Giuseppe gli disse: 'Questa è l'interpretazione del sogno: i tre tralci sono tre giorni;
\par 13 ancora tre giorni, e Faraone ti farà rialzare il capo, ti ristabilirà nel tuo ufficio, e tu darai in mano a Faraone la sua coppa, nel modo che facevi prima, quand'eri suo coppiere.
\par 14 Ma ricordati di me, quando sarai felice, e siimi benigno, ti prego; parla di me a Faraone, e fammi uscire da questa casa;
\par 15 perché io fui portato via furtivamente dal paese degli Ebrei, e anche qui non ho fatto nulla da esser messo in questa fossa'.
\par 16 Il capo de' panattieri, vedendo che la interpretazione di Giuseppe era favorevole, gli disse: 'Anch'io, nel mio sogno, ecco, avevo tre canestri di pan bianco, sul capo;
\par 17 e nel canestro più alto c'era per Faraone ogni sorta di vivande cotte al forno; e gli uccelli le mangiavano dentro al canestro sul mio capo'.
\par 18 Giuseppe rispose e disse: 'Questa è l'interpretazione del sogno: i tre canestri sono tre giorni;
\par 19 ancora tre giorni, e Faraone ti porterà via la testa di sulle spalle, ti farà impiccare a un albero, e gli uccelli ti mangeranno le carni addosso'.
\par 20 E avvenne, il terzo giorno, ch'era il natalizio di Faraone, che questi dette un convito a tutti i suoi servitori, e fece alzare il capo al gran coppiere, e alzare il capo al gran panattiere in mezzo ai suoi servitori:
\par 21 ristabilì il gran coppiere nel suo ufficio di coppiere, perché mettesse la coppa in man di Faraone,
\par 22 ma fece appiccare il gran panattiere, secondo la interpretazione che Giuseppe avea loro data.
\par 23 Il gran coppiere però non si ricordò di Giuseppe, ma lo dimenticò.

\chapter{41}

\par 1 Or avvenne, in capo a due anni interi, che Faraone ebbe un sogno. Ed ecco che stava presso il fiume;
\par 2 e su dal fiume ecco salire sette vacche, di bell'apparenza e grasse, e mettersi a pascere nella giuncaia.
\par 3 E, dopo quelle, ecco salire dal fiume altre sette vacche di brutt'apparenza e scarne, e fermarsi presso alle prime, sulla riva del fiume.
\par 4 E le vacche di brutt'apparenza e scarne, divorarono le sette vacche di bell'apparenza e grasse. E Faraone si svegliò.
\par 5 Poi si riaddormentò, e sognò di nuovo; ed ecco sette spighe, grasse e belle, venir su da un unico stelo.
\par 6 Poi ecco sette spighe, sottili e arse dal vento orientale, germogliare dopo quelle altre.
\par 7 E le spighe sottili inghiottirono le sette spighe grasse e piene. E Faraone si svegliò: ed ecco, era un sogno.
\par 8 La mattina, lo spirito di Faraone fu conturbato; ed egli mandò a chiamare tutti i magi e tutti i savi d'Egitto, e raccontò loro i suoi sogni; ma non ci fu alcuno che li potesse interpretare a Faraone.
\par 9 Allora il capo de' coppieri parlò a Faraone, dicendo: 'Ricordo oggi i miei falli.
\par 10 Faraone s'era sdegnato contro i suoi servitori, e m'avea fatto mettere in prigione in casa del capo delle guardie: me, e il capo de' panattieri.
\par 11 L'uno e l'altro facemmo un sogno, nella medesima notte: facemmo ciascuno un sogno, avente il suo proprio significato.
\par 12 Or c'era quivi con noi un giovane ebreo, servo del capo delle guardie; a lui raccontammo i nostri sogni, ed egli ce li interpretò, dando a ciascuno l'interpretazione del suo sogno.
\par 13 E le cose avvennero secondo l'interpretazione ch'egli ci aveva data: Faraone ristabilì me nel mio ufficio, e l'altro lo fece appiccare'.
\par 14 Allora Faraone mandò a chiamare Giuseppe, il quale fu tosto tratto fuor dalla prigione sotterranea. Egli si rase, si cambiò il vestito, e venne da Faraone.
\par 15 E Faraone disse a Giuseppe: 'Ho fatto un sogno, e non c'è chi lo possa interpretare; e ho udito dir di te che, quando t'hanno raccontato un sogno, tu lo puoi interpretare'.
\par 16 Giuseppe rispose a Faraone, dicendo: 'Non son io; ma sarà Dio che darà a Faraone una risposta favorevole'.
\par 17 E Faraone disse a Giuseppe: 'Nel mio sogno, io stavo sulla riva del fiume;
\par 18 quand'ecco salir dal fiume sette vacche grasse e di bell'apparenza, e mettersi a pascere nella giuncaia.
\par 19 E, dopo quelle, ecco salire altre sette vacche magre, di bruttissima apparenza e scarne: tali, che non ne vidi mai di così brutte in tutto il paese d'Egitto.
\par 20 E le vacche magre e brutte divorarono le prime sette vacche grasse;
\par 21 e quelle entrarono loro in corpo, e non si riconobbe che vi fossero entrate; erano di brutt'apparenza come prima. E mi svegliai.
\par 22 Poi vidi ancora nel mio sogno sette spighe venir su da un unico stelo, piene e belle;
\par 23 ed ecco altre sette spighe vuote, sottili e arse dal vento orientale, germogliare dopo quelle altre.
\par 24 E le spighe sottili inghiottirono le sette spighe belle. Io ho raccontato questo ai magi; ma non c'è stato alcuno che abbia saputo spiegarmelo'.
\par 25 Allora Giuseppe disse a Faraone: 'Ciò che Faraone ha sognato è una stessa cosa. Iddio ha significato a Faraone quello che sta per fare.
\par 26 Le sette vacche belle sono sette anni, e le sette spighe belle sono sette anni; è uno stesso sogno.
\par 27 E le sette vacche magre e brutte che salivano dopo quelle altre, sono sette anni; come pure le sette spighe vuote e arse dal vento orientale saranno sette anni di carestia.
\par 28 Questo è quel che ho detto a Faraone: Iddio ha mostrato a Faraone quello che sta per fare.
\par 29 Ecco, stanno per venire sette anni di grande abbondanza in tutto il paese d'Egitto;
\par 30 e dopo, verranno sette anni di carestia; e tutta quell'abbondanza sarà dimenticata nel paese d'Egitto, e la carestia consumerà il paese.
\par 31 E uno non si accorgerà più di quell'abbondanza nel paese, a motivo della carestia che seguirà; perché questa sarà molto aspra.
\par 32 E l'essersi il sogno replicato due volte a Faraone vuol dire che la cosa è decretata da Dio, e che Dio l'eseguirà tosto.
\par 33 Or dunque si provveda Faraone d'un uomo intelligente e savio e lo stabilisca sul paese d'Egitto.
\par 34 Faraone faccia così: Costituisca dei commissari sul paese per prelevare il quinto delle raccolte del paese d'Egitto, durante i sette anni dell'abbondanza.
\par 35 E radunino essi tutti i viveri di queste sette buone annate che stan per venire, e ammassino il grano a disposizione di Faraone per l'approvvigionamento delle città, e lo conservino.
\par 36 Questi viveri saranno una riserva per il paese, in vista dei sette anni di carestia che verranno nel paese d'Egitto; e così il paese non perirà per la carestia'.
\par 37 Piacque la cosa a Faraone e a tutti i suoi servitori.
\par 38 E Faraone disse ai suoi servitori: 'Potremmo noi trovare un uomo pari a questo, in cui sia lo spirito di Dio?'
\par 39 E Faraone disse a Giuseppe: 'Giacché Iddio t'ha fatto conoscere tutto questo, non v'è alcuno che sia intelligente e savio al pari di te.
\par 40 Tu sarai sopra la mia casa, e tutto il mio popolo obbedirà ai tuoi ordini; per il trono soltanto, io sarò più grande di te'.
\par 41 E Faraone disse a Giuseppe: 'Vedi, io ti stabilisco su tutto il paese d'Egitto'.
\par 42 E Faraone si tolse l'anello di mano e lo mise alla mano di Giuseppe; lo fece vestire di abiti di lino fino, e gli mise al collo una collana d'oro.
\par 43 Lo fece montare sul suo secondo carro, e davanti a lui si gridava: 'In ginocchio!' Così Faraone lo costituì su tutto il paese d'Egitto.
\par 44 E Faraone disse a Giuseppe: 'Io son Faraone! e senza te; nessuno alzerà la mano o il piede in tutto il paese d'Egitto'.
\par 45 E Faraone chiamò Giuseppe Tsafnath-Paneach e gli dette per moglie Asenath figliuola di Potifera, sacerdote di On. E Giuseppe partì per visitare il paese d'Egitto.
\par 46 Or Giuseppe avea trent'anni quando si presentò dinanzi a Faraone re d'Egitto. E Giuseppe uscì dal cospetto di Faraone, e percorse tutto il paese d'Egitto.
\par 47 Durante i sette anni d'abbondanza, la terra produsse a piene mani;
\par 48 e Giuseppe adunò tutti i viveri di quei sette anni che vennero nel paese d'Egitto, e ripose i viveri nelle città; ripose in ogni città i viveri del territorio circonvicino.
\par 49 Così Giuseppe ammassò grano come la rena del mare; in così gran quantità, che si smise di contarlo, perch'era innumerevole.
\par 50 Or avanti che venisse il primo anno della carestia, nacquero a Giuseppe due figliuoli, che Asenath figliuola di Potifera sacerdote di On gli partorì.
\par 51 E Giuseppe chiamò il primogenito Manasse, perché, disse, 'Iddio m'ha fatto dimenticare ogni mio affanno e tutta la casa di mio padre'.
\par 52 E al secondo pose nome Efraim, perché, disse, 'Iddio m'ha reso fecondo nel paese della mia afflizione'.
\par 53 I sette anni d'abbondanza ch'erano stati nel paese d'Egitto, finirono;
\par 54 e cominciarono a venire i sette anni della carestia, come Giuseppe avea detto. E ci fu carestia in tutti i paesi; ma in tutto il paese d'Egitto c'era del pane.
\par 55 Poi la carestia si estese a tutto il paese d'Egitto, e il popolo gridò a Faraone per aver del pane. E Faraone disse a tutti gli Egiziani: 'Andate da Giuseppe, e fate quello che vi dirà'.
\par 56 La carestia era sparsa su tutta la superficie del paese, e Giuseppe aperse tutti i depositi e vendé grano agli Egiziani. E la carestia s'aggravò nel paese d'Egitto.
\par 57 E da tutti i paesi si veniva in Egitto da Giuseppe per comprar del grano, perché la carestia era grave per tutta la terra.

\chapter{42}

\par 1 Or Giacobbe, vedendo che c'era del grano in Egitto, disse ai suoi figliuoli: 'Perché vi state a guardare l'un l'altro?'
\par 2 Poi disse: 'Ecco, ho sentito dire che c'è del grano in Egitto; scendete colà per comprarcene, onde possiam vivere e non abbiamo a morire'.
\par 3 E dieci de' fratelli di Giuseppe scesero in Egitto per comprarvi del grano.
\par 4 Ma Giacobbe non mandò Beniamino, fratello di Giuseppe, coi suoi fratelli, perché diceva: 'Che non gli abbia a succedere qualche disgrazia!'
\par 5 E i figliuoli d'Israele giunsero per comprare del grano in mezzo agli altri, che pur venivano; poiché nel paese di Canaan c'era la carestia.
\par 6 Or Giuseppe era colui che comandava nel paese; era lui che vendeva il grano a tutta la gente del paese; e i fratelli di Giuseppe vennero, e si prostrarono dinanzi a lui con la faccia a terra.
\par 7 E Giuseppe vide i suoi fratelli e li riconobbe, ma fece lo straniero davanti a loro, e parlò loro aspramente, e disse loro: 'Donde venite?' Ed essi risposero: 'Dal paese di Canaan per comprar de' viveri'.
\par 8 E Giuseppe riconobbe i suoi fratelli, ma essi non riconobbero lui.
\par 9 E Giuseppe si ricordò de' sogni che aveva avuti intorno a loro, e disse: 'Voi siete delle spie! Siete venuti per vedere i luoghi sforniti del paese!'
\par 10 Ed essi a lui: 'No, signor mio; i tuoi servitori son venuti a comprar de' viveri.
\par 11 Siamo tutti figliuoli d'uno stesso uomo; siamo gente sincera; i tuoi servitori non son delle spie'.
\par 12 Ed egli disse loro: 'No, siete venuti per vedere i luoghi sforniti del paese!'
\par 13 E quelli risposero: 'Noi, tuoi servitori, siamo dodici fratelli, figliuoli d'uno stesso uomo, del paese di Canaan. Ed ecco, il più giovane è oggi con nostro padre e uno non è più'.
\par 14 E Giuseppe disse loro: 'La cosa è come v'ho detto; siete delle spie!
\par 15 Ecco come sarete messi alla prova: Per la vita di Faraone, non uscirete di qui prima che il vostro fratello più giovane sia venuto qua.
\par 16 Mandate uno di voi a prendere il vostro fratello; e voi resterete qui in carcere, perché le vostre parole siano messe alla prova, e si vegga se c'è del vero in voi; se no, per la vita di Faraone, siete delle spie!'
\par 17 E li mise assieme in prigione per tre giorni.
\par 18 Il terzo giorno, Giuseppe disse loro: 'Fate questo, e vivrete; io temo Iddio!
\par 19 Se siete gente sincera, uno di voi fratelli resti qui incatenato nella vostra prigione; e voi, andate, portate del grano per la necessità delle vostre famiglie;
\par 20 e menatemi il vostro fratello più giovine; così le vostre parole saranno verificate, e voi non morrete'. Ed essi fecero così.
\par 21 E si dicevano l'uno all'altro: 'Sì, noi fummo colpevoli verso il nostro fratello, giacché vedemmo l'angoscia dell'anima sua quando egli ci supplicava, e noi non gli demmo ascolto! Ecco perché ci viene addosso quest'angoscia'.
\par 22 E Ruben rispose loro, dicendo: 'Non ve lo dicevo io: Non commettete questo peccato contro il fanciullo? Ma voi non mi voleste dare ascolto. Perciò, ecco, che il suo sangue ci è ridomandato'.
\par 23 Or quelli non sapevano che Giuseppe li capiva, perché fra lui e loro c'era un interprete.
\par 24 Ed egli s'allontanò da essi, e pianse. Poi tornò, parlò loro, e prese di fra loro Simeone, che fece incatenare sotto i loro occhi.
\par 25 Poi Giuseppe ordinò che s'empissero di grano i loro sacchi, che si rimettesse il danaro di ciascuno nel suo sacco, e che si dessero loro delle provvisioni per il viaggio. E così fu fatto.
\par 26 Ed essi caricarono il loro grano sui loro asini, e se ne andarono.
\par 27 Or l'un d'essi aprì il suo sacco per dare del foraggio al suo asino, nel luogo ove pernottavano, e vide il suo danaro ch'era alla bocca del sacco;
\par 28 e disse ai suoi fratelli: 'Il mio danaro m'è stato restituito, ed eccolo qui nel mio sacco'. Allora il cuore venne loro meno, e, tremando, dicevano l'uno all'altro: 'Che è mai questo che Dio ci ha fatto?'
\par 29 E vennero a Giacobbe, loro padre, nel paese di Canaan, e gli raccontarono tutto quello ch'era loro accaduto, dicendo:
\par 30 'L'uomo ch'è il signor del paese, ci ha parlato aspramente e ci ha trattato da spie del paese.
\par 31 E noi gli abbiamo detto: Siamo gente sincera; non siamo delle spie;
\par 32 siamo dodici fratelli, figliuoli di nostro padre; uno non è più, e il più giovine è oggi con nostro padre nel paese di Canaan.
\par 33 E quell'uomo, signore del paese, ci ha detto: Da questo conoscerò se siete gente sincera; lasciate presso di me uno dei vostri fratelli, prendete quel che vi necessita per le vostre famiglie, partite, e menatemi il vostro fratello più giovine.
\par 34 Allora conoscerò che non siete delle spie ma gente sincera; io vi renderò il vostro fratello e voi potrete trafficare nel paese'.
\par 35 Or com'essi vuotavano i loro sacchi, ecco che l'involto del danaro di ciascuno era nel suo sacco; essi e il padre loro videro gl'involti del loro danaro, e furon presi da paura.
\par 36 E Giacobbe, loro padre, disse: 'Voi m'avete privato dei miei figliuoli! Giuseppe non è più, Simeone non è più, e mi volete togliere anche Beniamino! Tutto questo cade addosso a me!'
\par 37 E Ruben disse a suo padre: 'Se non te lo rimeno, fa' morire i miei due figliuoli! Affidalo a me, io te lo ricondurrò'.
\par 38 Ma Giacobbe rispose: 'Il mio figliuolo non scenderà con voi; poiché il suo fratello è morto, e questo solo è rimasto: se gli succedesse qualche disgrazia durante il vostro viaggio, fareste scendere con cordoglio la mia canizie nel soggiorno de' morti'.

\chapter{43}

\par 1 Or la carestia era grave nel paese;
\par 2 e quand'ebbero finito di mangiare il grano che aveano portato dall'Egitto, il padre disse loro: 'Tornate a comprarci un po' di viveri'.
\par 3 E Giuda gli rispose, dicendo: 'Quell'uomo ce lo dichiarò positivamente: Non vedrete la mia faccia, se il vostro fratello non sarà con voi.
\par 4 Se tu mandi il nostro fratello con noi, noi scenderemo e ti compreremo dei viveri;
\par 5 ma, se non lo mandi, non scenderemo; perché quell'uomo ci ha detto: Non vedrete la mia faccia, se il vostro fratello non sarà con voi'.
\par 6 E Israele disse: 'Perché m'avete fatto questo torto di dire a quell'uomo che avevate ancora un fratello?'
\par 7 Quelli risposero: 'Quell'uomo c'interrogò partitamente intorno a noi e al nostro parentado, dicendo: Vostro padre vive egli ancora? Avete qualche altro fratello? E noi gli rispondemmo a tenore delle sue domande. Potevam noi mai sapere che ci avrebbe detto: Fate venire il vostro fratello?'
\par 8 E Giuda disse a Israele suo padre: 'Lascia venire il fanciullo con me, e ci leveremo e andremo; e noi vivremo e non morremo: né noi, né tu, né i nostri piccini.
\par 9 Io mi rendo garante di lui; ridomandane conto alla mia mano; se non te lo riconduco e non te lo rimetto davanti, io sarò per sempre colpevole verso di te.
\par 10 Se non ci fossimo indugiati, a quest'ora saremmo già tornati due volte'.
\par 11 Allora Israele, loro padre, disse loro: 'Se così è, fate questo: Prendete ne' vostri sacchi delle cose più squisite di questo paese, e portate a quell'uomo un dono: un po' di balsamo, un po' di miele, degli aromi e della mirra, de' pistacchi e delle mandorle;
\par 12 e pigliate con voi il doppio del danaro, e riportate il danaro che fu rimesso alla bocca de' vostri sacchi; forse fu un errore;
\par 13 prendete anche il vostro fratello, e levatevi, tornate da quell'uomo;
\par 14 e l'Iddio onnipotente vi faccia trovar grazia dinanzi a quell'uomo, sì ch'egli vi rilasci l'altro vostro fratello e Beniamino. E se debbo esser privato dei miei figliuoli, ch'io lo sia!'
\par 15 Quelli presero dunque il dono, presero seco il doppio del danaro, e Beniamino; e, levatisi, scesero in Egitto, e si presentarono dinanzi a Giuseppe.
\par 16 E come Giuseppe vide Beniamino con loro, disse al suo maestro di casa: 'Conduci questi uomini in casa; macella, e prepara tutto; perché questi uomini mangeranno con me a mezzogiorno'.
\par 17 E l'uomo fece come Giuseppe aveva ordinato, e li menò in casa di Giuseppe.
\par 18 E quelli ebbero paura, perché eran menati in casa di Giuseppe, e dissero: 'Siamo menati qui a motivo di quel danaro che ci fu rimesso nei sacchi la prima volta; ei vuol darci addosso, precipitarsi su noi e prenderci come schiavi, coi nostri asini'.
\par 19 E accostatisi al maestro di casa di Giuseppe, gli parlarono sulla porta della casa, e dissero:
\par 20 'Scusa, signor mio! noi scendemmo già una prima volta a comprar dei viveri;
\par 21 e avvenne che, quando fummo giunti al luogo dove pernottammo, aprimmo i sacchi, ed ecco il danaro di ciascun di noi era alla bocca del suo sacco: il nostro danaro del peso esatto; e noi l'abbiam riportato con noi.
\par 22 E abbiam portato con noi dell'altro danaro per comprar de' viveri; noi non sappiamo chi avesse messo il nostro danaro nei nostri sacchi'.
\par 23 Ed egli disse: 'Datevi pace, non temete; l'Iddio vostro e l'Iddio del vostro padre ha messo un tesoro nei vostri sacchi. Io ebbi il vostro danaro'. E, fatto uscire Simeone, lo condusse loro.
\par 24 Quell'uomo li fece entrare in casa di Giuseppe; dette loro dell'acqua, ed essi si lavarono i piedi; ed egli dette del foraggio ai loro asini.
\par 25 Ed essi prepararono il regalo, aspettando che Giuseppe venisse a mezzogiorno; perché aveano inteso che rimarrebbero quivi a mangiare.
\par 26 E quando Giuseppe venne a casa, quelli gli porsero il dono che aveano portato seco nella casa, e s'inchinarono fino a terra davanti a lui.
\par 27 Egli domandò loro come stessero, e disse: 'Vostro padre, il vecchio di cui mi parlaste, sta egli bene? Vive egli ancora?'
\par 28 E quelli risposero: 'Il padre nostro, tuo servo, sta bene; vive ancora'. E s'inchinarono, e gli fecero riverenza.
\par 29 Poi Giuseppe alzò gli occhi, vide Beniamino suo fratello, figliuolo della madre sua, e disse: 'È questo il vostro fratello più giovine di cui mi parlaste?' Poi disse a lui: 'Iddio ti sia propizio, figliuol mio!'
\par 30 E Giuseppe s'affrettò ad uscire, perché le sue viscere s'eran commosse per il suo fratello; e cercava un luogo dove piangere; entrò nella sua camera, e quivi pianse.
\par 31 Poi si lavò la faccia, ed uscì; si fece forza, e disse: 'Portate il pranzo'.
\par 32 Fu dunque portato il cibo per lui a parte, e per loro a parte, e per gli Egiziani che mangiavan con loro, a parte; perché gli Egiziani non possono mangiare con gli Ebrei; per gli Egiziani è cosa abominevole.
\par 33 Ed essi si misero a sedere dinanzi a lui: il primogenito, secondo il suo diritto di primogenitura, e il più giovine secondo la sua età; e si guardavano l'un l'altro con maraviglia.
\par 34 E Giuseppe fe' loro portare delle vivande che aveva dinanzi; ma la porzione di Beniamino era cinque volte maggiore di quella d'ogni altro di loro. E bevvero, e stettero allegri con lui.

\chapter{44}

\par 1 Giuseppe dette quest'ordine al suo maestro di casa: 'Riempi i sacchi di questi uomini di tanti viveri quanti ne posson portare, e metti il danaro di ciascun d'essi alla bocca del suo sacco.
\par 2 E metti la mia coppa, la coppa d'argento, alla bocca del sacco del più giovine, assieme al danaro del suo grano'. Ed egli fece come Giuseppe avea detto.
\par 3 La mattina, non appena fu giorno, quegli uomini furon fatti partire coi loro asini.
\par 4 E quando furono usciti dalla città e non erano ancora lontani, Giuseppe disse al suo maestro di casa: 'Lèvati, va' dietro a quegli uomini; e quando li avrai raggiunti, di' loro: Perché avete reso mal per bene?
\par 5 Non è quella la coppa nella quale il mio signore beve, e della quale si serve per indovinare? Avete fatto male a far questo!'
\par 6 Egli li raggiunse, e disse loro quelle parole.
\par 7 Ed essi gli risposero: 'Perché il mio signore ci rivolge parole come queste? Iddio preservi i tuoi servitori dal fare una tal cosa!
\par 8 Ecco, noi t'abbiam riportato dal paese di Canaan il danaro che avevam trovato alla bocca de' nostri sacchi; come dunque avremmo rubato dell'argento e dell'oro dalla casa del tuo signore?
\par 9 Quello de' tuoi servitori presso il quale si troverà la coppa, sia messo a morte; e noi pure saremo schiavi del tuo signore!'
\par 10 Ed egli disse: 'Ebbene, sia fatto come dite: colui presso il quale essa sarà trovata, sarà mio schiavo; e voi sarete innocenti'.
\par 11 In tutta fretta, ognuno d'essi mise giù il suo sacco a terra, e ciascuno aprì il suo.
\par 12 Il maestro di casa li frugò, cominciando da quello del maggiore, per finire con quello del più giovane; e la coppa fu trovata nel sacco di Beniamino.
\par 13 Allora quelli si stracciarono le vesti, ognuno ricaricò il suo asino, e tornarono alla città.
\par 14 Giuda e i suoi fratelli arrivarono alla casa di Giuseppe, il quale era ancora quivi; e si gettarono in terra dinanzi a lui.
\par 15 E Giuseppe disse loro: 'Che azione è questa che avete fatta? Non lo sapete che un uomo come me ha potere d'indovinare?' Giuda rispose:
\par 16 'Che diremo al mio signore? quali parole useremo? o come ci giustificheremo? Dio ha ritrovato l'iniquità de' tuoi servitori. Ecco, siamo schiavi del mio signore: tanto noi, quanto colui in mano del quale è stata trovata la coppa'.
\par 17 Ma Giuseppe disse: 'Mi guardi Iddio dal far questo! L'uomo in man del quale è stata trovata la coppa, sarà mio schiavo; quanto a voi, risalite in pace dal padre vostro'.
\par 18 Allora Giuda s'accostò a Giuseppe, e disse: 'Di grazia, signor mio, permetti al tuo servitore di far udire una parola al mio signore, e non s'accenda l'ira tua contro il tuo servitore! poiché tu sei come Faraone.
\par 19 Il mio signore interrogò i suoi servitori, dicendo: Avete voi padre o fratello?
\par 20 e noi rispondemmo al mio signore: Abbiamo un padre ch'è vecchio, con un giovane figliuolo, natogli nella vecchiaia; il fratello di questo è morto, talché egli è rimasto solo de' figli di sua madre; e suo padre l'ama.
\par 21 Allora tu dicesti ai tuoi servitori: Menatemelo, perch'io lo vegga co' miei occhi.
\par 22 E noi dicemmo al mio signore: Il fanciullo non può lasciare suo padre; perché, se lo lasciasse, suo padre morrebbe.
\par 23 E tu dicesti ai tuoi servitori: Se il vostro fratello più giovine non scende con voi, voi non vedrete più la mia faccia.
\par 24 E come fummo risaliti a mio padre, tuo servitore, gli riferimmo le parole del mio signore.
\par 25 Poi nostro padre disse: Tornate a comprarci un po' di viveri.
\par 26 E noi rispondemmo: Non possiamo scender laggiù; se il nostro fratello più giovine verrà con noi, scenderemo; perché non possiamo veder la faccia di quell'uomo, se il nostro fratello più giovine non è con noi.
\par 27 E mio padre, tuo servitore, ci rispose: Voi sapete che mia moglie mi partorì due figliuoli;
\par 28 l'un d'essi si partì da me, e io dissi: Certo egli è stato sbranato; e non l'ho più visto da allora;
\par 29 e se mi togliete anche questo, e se gli avviene qualche disgrazia, voi farete scendere con dolore la mia canizie nel soggiorno de' morti.
\par 30 Or dunque, quando giungerò da mio padre, tuo servitore, se il fanciullo, all'anima del quale la sua è legata, non è con noi,
\par 31 avverrà che, come avrà veduto che il fanciullo non c'è, egli morrà; e i tuoi servitori avranno fatto scendere con cordoglio la canizie del tuo servitore nostro padre nel soggiorno de' morti.
\par 32 Ora, siccome il tuo servitore s'è reso garante del fanciullo presso mio padre, e gli ha detto: Se non te lo riconduco sarò per sempre colpevole verso mio padre,
\par 33 deh, permetti ora che il tuo servitore rimanga schiavo del mio signore, invece del fanciullo, e che il fanciullo se ne torni coi suoi fratelli.
\par 34 Perché, come farei a risalire da mio padre senz'aver meco il fanciullo? Ah, ch'io non vegga il dolore che ne verrebbe a mio padre!'

\chapter{45}

\par 1 Allora Giuseppe non poté più contenersi dinanzi a tutti gli astanti, e gridò: 'Fate uscir tutti dalla mia presenza!' E nessuno rimase con Giuseppe quand'egli si diè a conoscere ai suoi fratelli.
\par 2 E alzò la voce piangendo; gli Egiziani l'udirono, e l'udì la casa di Faraone.
\par 3 E Giuseppe disse ai suoi fratelli: 'Io son Giuseppe; mio padre vive egli tuttora?' Ma i suoi fratelli non gli potevano rispondere, perché erano sbigottiti alla sua presenza.
\par 4 E Giuseppe disse ai suoi fratelli: 'Deh, avvicinatevi a me!' Quelli s'avvicinarono, ed egli disse: 'Io son Giuseppe, vostro fratello, che voi vendeste perché fosse menato in Egitto.
\par 5 Ma ora non vi contristate, né vi dolga d'avermi venduto perch'io fossi menato qua; poiché Iddio m'ha mandato innanzi a voi per conservarvi in vita.
\par 6 Infatti, sono due anni che la carestia è nel paese; e ce ne saranno altri cinque, durante i quali non ci sarà né aratura né mèsse.
\par 7 Ma Dio mi ha mandato dinanzi a voi, perché sia conservato di voi un resto sulla terra, e per salvarvi la vita con una grande liberazione.
\par 8 Non siete dunque voi che m'avete mandato qua, ma è Dio; egli m'ha stabilito come padre di Faraone, signore di tutta la sua casa, e governatore di tutto il paese d'Egitto.
\par 9 Affrettatevi a risalire da mio padre, e ditegli: Così dice il tuo figliuolo Giuseppe: Iddio mi ha stabilito signore di tutto l'Egitto; scendi da me; non tardare;
\par 10 tu dimorerai nel paese di Goscen, e sarai vicino a me; tu e i tuoi figliuoli, i figliuoli de' tuoi figliuoli, i tuoi greggi, i tuoi armenti, e tutto quello che possiedi.
\par 11 E quivi io ti sostenterò (perché ci saranno ancora cinque anni di carestia), onde tu non sia ridotto alla miseria: tu, la tua famiglia, e tutto quello che possiedi.
\par 12 Ed ecco, voi vedete coi vostri occhi, e il mio fratello Beniamino vede con gli occhi suoi, ch'è proprio la bocca mia quella che vi parla.
\par 13 Raccontate dunque a mio padre tutta la mia gloria in Egitto, e tutto quello che avete veduto; e fate che mio padre scenda presto qua'.
\par 14 E gettatosi al collo di Beniamino, suo fratello, pianse; e Beniamino pianse sul collo di lui.
\par 15 Baciò pure tutti i suoi fratelli, piangendo. E, dopo questo, i suoi fratelli si misero a parlare con lui.
\par 16 Il rumore della cosa si sparse nella casa di Faraone, e si disse: 'Sono arrivati i fratelli di Giuseppe'. Il che piacque a Faraone ed ai suoi servitori.
\par 17 E Faraone disse a Giuseppe: 'Di' ai tuoi fratelli: Fate questo: caricate le vostre bestie, e andate, tornate al paese di Canaan;
\par 18 prendete vostro padre e le vostre famiglie, e venite da me; io vi darò del meglio del paese d'Egitto, e voi mangerete il grasso del paese.
\par 19 Tu hai l'ordine di dir loro: Fate questo: Prendete nel paese di Egitto dei carri per i vostri piccini e per le vostre mogli; conducete vostro padre, e venite.
\par 20 E non vi rincresca di lasciar le vostre masserizie; perché il meglio di tutto il paese d'Egitto sarà vostro'.
\par 21 I figliuoli d'Israele fecero così, e Giuseppe diede loro dei carri, secondo l'ordine di Faraone, e diede loro delle provvisioni per il viaggio.
\par 22 A tutti dette un abito di ricambio per ciascuno; ma a Beniamino dette trecento sicli d'argento e cinque mute di vestiti;
\par 23 e a suo padre mandò questo: dieci asini carichi delle migliori cose d'Egitto, dieci asine cariche di grano, di pane e di viveri, per suo padre, durante il viaggio.
\par 24 Così licenziò i suoi fratelli, e questi partirono; ed egli disse loro: 'Non ci siano, per via, delle dispute fra voi'.
\par 25 Ed essi risalirono dall'Egitto, e vennero nel paese di Canaan da Giacobbe loro padre.
\par 26 E gli riferirono ogni cosa, dicendo: 'Giuseppe vive tuttora, ed è il governatore di tutto il paese d'Egitto'. Ma il suo cuore rimase freddo, perch'egli non credeva loro.
\par 27 Essi gli ripeterono tutte le parole che Giuseppe avea dette loro; ed egli vide i carri che Giuseppe avea mandato per condurlo via; allora lo spirito di Giacobbe loro padre si ravvivò, e Israele disse:
\par 28 'Basta; il mio figliuolo Giuseppe vive tuttora; io andrò, e lo vedrò prima di morire'.

\chapter{46}

\par 1 Israele dunque si partì con tutto quello che aveva; e, giunto a Beer-Sceba, offrì sacrifizi all'Iddio d'Isacco suo padre.
\par 2 E Dio parlò a Israele in visioni notturne, e disse: 'Giacobbe, Giacobbe!' Ed egli rispose: 'Eccomi'.
\par 3 E Dio disse: 'Io sono Iddio, l'Iddio di tuo padre; non temere di scendere in Egitto, perché là ti farò diventare una grande nazione.
\par 4 Io scenderò con te in Egitto, e te ne farò anche sicuramente risalire; e Giuseppe ti chiuderà gli occhi'.
\par 5 Allora Giacobbe partì da Beer-Sceba; e i figliuoli d'Israele fecero salire Giacobbe loro padre, i loro piccini e le loro mogli sui carri che Faraone avea mandato per trasportarli.
\par 6 Ed essi presero il loro bestiame e i beni che aveano acquistato nel paese di Canaan, e vennero in Egitto: Giacobbe, e tutta la sua famiglia con lui.
\par 7 Egli condusse seco in Egitto i suoi figliuoli, le sue figliuole, le figliuole de' suoi figliuoli, e tutta la sua famiglia.
\par 8 Questi sono i nomi de' figliuoli d'Israele che vennero in Egitto: Giacobbe e i suoi figliuoli. Il primogenito di Giacobbe: Ruben.
\par 9 I figliuoli di Ruben: Henoc, Pallu, Hetsron e Carmi.
\par 10 I figliuoli di Simeone: Iemuel, Iamin, Ohad, Iakin, Tsohar e Saul, figliuolo di una Cananea.
\par 11 I figliuoli di Levi: Gherson, Kehath e Merari.
\par 12 I figliuoli di Giuda: Er, Onan, Scela, Perets e Zerach; ma Er e Onan morirono nel paese di Canaan; e i figliuoli di Perets furono: Hetsron e Hamul.
\par 13 I figliuoli d'Issacar: Tola, Puva, Iob e Scimron.
\par 14 I figliuoli di Zabulon: Sered, Elon, e Iahleel.
\par 15 Cotesti sono i figliuoli che Lea partorì a Giacobbe a Paddan-Aram, oltre Dina, figliuola di lui. I suoi figliuoli e le sue figliuole erano in tutto trentatré persone.
\par 16 I figliuoli di Gad: Tsifion, Haggi, Shuni, Etsbon, Eri, Arodi e Areli.
\par 17 I figliuoli di Ascer: Imna, Tishva, Tishvi, Beria e Serah loro sorella. E i figliuoli di Beria: Heber e Malkiel.
\par 18 Cotesti furono i figliuoli di Zilpa che Labano avea data a Lea sua figliuola; ed essa li partorì a Giacobbe: in tutto sedici persone.
\par 19 I figliuoli di Rachele, moglie di Giacobbe: Giuseppe e Beniamino.
\par 20 E a Giuseppe, nel paese d'Egitto, nacquero Manasse ed Efraim, i quali Asenath, figliuola di Potifera, sacerdote di On, gli partorì.
\par 21 I figliuoli di Beniamino: Bela, Beker, Ashbel, Ghera, Naaman, Ehi, Rosh, Muppim, Huppim e Ard.
\par 22 Cotesti sono i figliuoli di Rachele che nacquero a Giacobbe: in tutto, quattordici persone.
\par 23 I figliuoli di Dan: Huscim.
\par 24 I figliuoli di Neftali: Iahtseel, Guni, Ietser e Shillem.
\par 25 Cotesti sono i figliuoli di Bilha che Labano avea dato a Rachele sua figliuola, ed essa li partorì a Giacobbe: in tutto, sette persone.
\par 26 Le persone che vennero con Giacobbe in Egitto, discendenti da lui, senza contare le mogli de' figliuoli di Giacobbe, erano in tutto sessantasei.
\par 27 E i figliuoli di Giuseppe, natigli in Egitto, erano due. Il totale delle persone della famiglia di Giacobbe che vennero in Egitto, era di settanta.
\par 28 Or Giacobbe mandò avanti a sé Giuda a Giuseppe, perché questi lo introducesse nel paese di Goscen. E giunsero nel paese di Goscen.
\par 29 Giuseppe fece attaccare il suo carro, e salì in Goscen a incontrare Israele, suo padre; e gli si presentò, gli si gettò al collo, e pianse lungamente sul collo di lui.
\par 30 E Israele disse a Giuseppe: 'Ora, ch'io muoia pure, giacché ho veduto la tua faccia, e tu vivi ancora!'
\par 31 E Giuseppe disse ai suoi fratelli e alla famiglia di suo padre: 'Io salirò a informare Faraone, e gli dirò: I miei fratelli e la famiglia di mio padre che erano nel paese di Canaan, sono venuti da me.
\par 32 Questi uomini sono pastori, poiché son sempre stati allevatori di bestiame; e hanno menato seco i loro greggi, i loro armenti, e tutto quello che posseggono.
\par 33 E quando Faraone vi farà chiamare e vi dirà: Qual è la vostra occupazione? risponderete:
\par 34 I tuoi servitori sono stati allevatori di bestiame dalla loro infanzia fino a quest'ora: così noi come i nostri padri. Direte così, perché possiate abitare nel paese di Goscen. Poiché gli Egiziani hanno in abominio tutti i pastori'.

\chapter{47}

\par 1 Giuseppe andò quindi a informare Faraone, e gli disse: 'Mio padre e i miei fratelli coi loro greggi, coi loro armenti e con tutto quello che hanno, son venuti dal paese di Canaan; ed ecco, sono nel paese di Goscen'.
\par 2 E prese cinque uomini di tra i suoi fratelli e li presentò a Faraone.
\par 3 E Faraone disse ai fratelli di Giuseppe: 'Qual è la vostra occupazione?' Ed essi risposero a Faraone: 'I tuoi servitori sono pastori, come furono i nostri padri'.
\par 4 Poi dissero a Faraone: 'Siam venuti per dimorare in questo paese, perché nel paese di Canaan non c'è pastura per i greggi dei tuoi servitori; poiché la carestia v'è grave; deh, permetti ora che i tuoi servi dimorino nel paese di Goscen'.
\par 5 E Faraone parlò a Giuseppe dicendo: 'Tuo padre e i tuoi fratelli son venuti da te;
\par 6 il paese d'Egitto ti sta dinanzi, fa' abitare tuo padre e i tuoi fratelli nella parte migliore del paese; dimorino pure nel paese di Goscen; e se conosci fra loro degli uomini capaci, falli sovrintendenti del mio bestiame'.
\par 7 Poi Giuseppe menò Giacobbe suo padre da Faraone, e glielo presentò. E Giacobbe benedisse Faraone.
\par 8 E Faraone disse a Giacobbe: 'Quanti sono i giorni del tempo della tua vita?'
\par 9 Giacobbe rispose a Faraone: 'I giorni del tempo de' miei pellegrinaggi sono centotrent'anni; i giorni del tempo della mia vita sono stati pochi e cattivi, e non hanno raggiunto il numero dei giorni della vita dei miei padri, ai dì dei loro pellegrinaggi'.
\par 10 Giacobbe benedisse ancora Faraone, e si ritirò dalla presenza di lui.
\par 11 E Giuseppe stabilì suo padre e i suoi fratelli, e dette loro un possesso nel paese d'Egitto, nella parte migliore del paese, nella contrada di Ramses, come Faraone aveva ordinato.
\par 12 E Giuseppe sostentò suo padre, i suoi fratelli e tutta la famiglia di suo padre, provvedendoli di pane, secondo il numero de' figliuoli.
\par 13 Or in tutto il paese non c'era pane, perché la carestia era gravissima; il paese d'Egitto e il paese di Canaan languivano a motivo della carestia.
\par 14 Giuseppe ammassò tutto il danaro che si trovava nel paese d'Egitto e nel paese di Canaan, come prezzo del grano che si comprava; e Giuseppe portò questo danaro nella casa di Faraone.
\par 15 E quando il danaro fu esaurito nel paese d'Egitto e nel paese di Canaan, tutti gli Egiziani vennero a Giuseppe e dissero: 'Dacci del pane! Perché dovremmo morire in tua presenza? giacché il danaro è finito'.
\par 16 E Giuseppe disse: 'Date il vostro bestiame; e io vi darò del pane in cambio del vostro bestiame, se non avete più danaro'.
\par 17 E quelli menarono a Giuseppe il loro bestiame; e Giuseppe diede loro del pane in cambio dei loro cavalli, dei loro greggi di pecore, delle loro mandre di buoi e dei loro asini. Così fornì loro del pane per quell'anno, in cambio di tutto il loro bestiame.
\par 18 Passato quell'anno, tornarono a lui l'anno seguente, e gli dissero: 'Noi non celeremo al mio signore che, il danaro essendo esaurito e le mandre del nostro bestiame essendo passate al mio signore, nulla più resta che il mio signore possa prendere, tranne i nostri corpi e le nostre terre.
\par 19 E perché dovremmo perire sotto gli occhi tuoi: noi e le nostre terre? Compra noi e le terre nostre in cambio di pane; e noi con le nostre terre saremo schiavi di Faraone; e dacci da seminare affinché possiam vivere e non moriamo, e il suolo non diventi un deserto'.
\par 20 Così Giuseppe comprò per Faraone tutte le terre d'Egitto; giacché gli Egiziani venderono ognuno il suo campo, perché la carestia li colpiva gravemente. Così il paese diventò proprietà di Faraone.
\par 21 Quanto al popolo, lo fece passare nelle città, da un capo all'altro dell'Egitto;
\par 22 solo le terre dei sacerdoti non acquistò; perché i sacerdoti ricevevano una provvisione assegnata loro da Faraone, e vivevano della provvisione che Faraone dava loro; per questo essi non venderono le loro terre.
\par 23 E Giuseppe disse al popolo: 'Ecco, oggi ho acquistato voi e le vostre terre per Faraone; eccovi del seme; seminate la terra;
\par 24 e al tempo della raccolta, ne darete il quinto a Faraone, e quattro parti saran vostre, per la sementa dei campi e per il nutrimento vostro, di quelli che sono in casa vostra, e per il nutrimento de' vostri bambini'.
\par 25 E quelli dissero: 'Tu ci hai salvato la vita! ci sia dato di trovar grazia agli occhi del mio signore, e saremo schiavi di Faraone'.
\par 26 Giuseppe ne fece una legge, che dura fino al dì d'oggi, secondo la quale un quinto del reddito delle terre d'Egitto era per Faraone; non ci furono che le terre dei sacerdoti che non furon di Faraone.
\par 27 Così gl'Israeliti abitarono nel paese d'Egitto, nel paese di Goscen; vi ebbero de' possessi, vi s'accrebbero, e moltiplicarono oltremodo.
\par 28 E Giacobbe visse nel paese d'Egitto diciassette anni; e i giorni di Giacobbe, gli anni della sua vita, furono centoquarantasette.
\par 29 E quando Israele s'avvicinò al giorno della sua morte, chiamò il suo figliuolo Giuseppe, e gli disse: 'Deh, se ho trovato grazia agli occhi tuoi, mettimi la mano sotto la coscia, e usami benignità e fedeltà; deh, non mi seppellire in Egitto!
\par 30 ma, quando giacerò coi miei padri, portami fuori d'Egitto, e seppelliscimi nel loro sepolcro!'
\par 31 Ed egli rispose: 'Farò come tu dici'. E Giacobbe disse: 'Giuramelo'. E Giuseppe glielo giurò. E Israele, vòlto al capo del letto, adorò.

\chapter{48}

\par 1 Dopo queste cose, avvenne che fu detto a Giuseppe: 'Ecco, tuo padre è ammalato'. Ed egli prese seco i suoi due figliuoli, Manasse ed Efraim.
\par 2 Giacobbe ne fu informato, e gli fu detto: 'Ecco, il tuo figliuolo Giuseppe viene da te'. E Israele raccolse le sue forze, e si mise a sedere sul letto.
\par 3 E Giacobbe disse a Giuseppe: 'L'Iddio onnipotente mi apparve a Luz nel paese di Canaan, mi benedisse,
\par 4 e mi disse: Ecco, io ti farò fruttare, ti moltiplicherò, ti farò diventare una moltitudine di popoli, e darò questo paese alla tua progenie dopo di te, come un possesso perpetuo.
\par 5 E ora, i tuoi due figliuoli che ti son nati nel paese d'Egitto prima ch'io venissi da te in Egitto, sono miei. Efraim e Manasse saranno miei, come Ruben e Simeone.
\par 6 Ma i figliuoli che hai generati dopo di loro, saranno tuoi; essi saranno chiamati col nome dei loro fratelli, quanto alla loro eredità.
\par 7 Quanto a me, allorché tornavo da Paddan, Rachele morì presso di me, nel paese di Canaan, durante il viaggio, a qualche distanza da Efrata; e la seppellii quivi, sulla via di Efrata, che è Bethlehem.
\par 8 Israele guardò i figliuoli di Giuseppe, e disse: 'Questi, chi sono?'
\par 9 E Giuseppe rispose a suo padre: 'Sono miei figliuoli, che Dio mi ha dati qui'. Ed egli disse: 'Deh, fa' che si appressino a me, e io li benedirò'.
\par 10 Or gli occhi d'Israele erano annebbiati a motivo dell'età, sì che non ci vedeva più. E Giuseppe li fece avvicinare a lui, ed egli li baciò e li abbracciò.
\par 11 E Israele disse a Giuseppe: 'Io non pensavo di riveder più la tua faccia; ed ecco che Iddio m'ha dato di vedere anche la tua progenie'.
\par 12 Giuseppe li ritirò di tra le ginocchia di suo padre, e si prostrò con la faccia a terra.
\par 13 Poi Giuseppe li prese ambedue: Efraim alla sua destra, alla sinistra d'Israele; e Manasse alla sua sinistra, alla destra d'Israele; e li fece avvicinare a lui.
\par 14 E Israele stese la sua man destra, e la posò sul capo di Efraim ch'era il più giovane; e posò la sua mano sinistra sul capo di Manasse, incrociando le mani; poiché Manasse era il primogenito.
\par 15 E benedisse Giuseppe, e disse: 'L'Iddio, nel cui cospetto camminarono i miei padri Abrahamo e Isacco, l'Iddio ch'è stato il mio pastore dacché esisto fino a questo giorno,
\par 16 l'angelo che mi ha liberato da ogni male, benedica questi fanciulli! Siano chiamati col mio nome e col nome de' miei padri Abrahamo ed Isacco, e moltiplichino copiosamente sulla terra!'
\par 17 Or quando Giuseppe vide che suo padre posava la man destra sul capo di Efraim, n'ebbe dispiacere, e prese la mano di suo padre per levarla di sul capo di Efraim e metterla sul capo di Manasse.
\par 18 E Giuseppe disse a suo padre: 'Non così, padre mio; perché questo è il primogenito; metti la tua mano destra sul suo capo'.
\par 19 Ma suo padre ricusò e disse: 'Lo so, figliuol mio, lo so; anch'egli diventerà un popolo, e anch'egli sarà grande; nondimeno, il suo fratello più giovane sarà più grande di lui, e la sua progenie diventerà una moltitudine di nazioni'.
\par 20 E in quel giorno li benedisse, dicendo: 'Per te Israele benedirà, dicendo: Iddio ti faccia simile ad Efraim ed a Manasse!' E mise Efraim prima di Manasse.
\par 21 Poi Israele disse a Giuseppe: 'Ecco, io mi muoio; ma Dio sarà con voi, e vi ricondurrà nel paese dei vostri padri.
\par 22 E io ti do una parte di più che ai tuoi fratelli: quella che conquistai dalle mani degli Amorei, con la mia spada e col mio arco'.

\chapter{49}

\par 1 Poi Giacobbe chiamò i suoi figliuoli, e disse: "Adunatevi, e vi annunzierò ciò che vi avverrà nei giorni a venire.
\par 2 Adunatevi e ascoltate, o figliuoli di Giacobbe! Date ascolto a Israele, vostro padre!
\par 3 Ruben, tu sei il mio primogenito, la mia forza, la primizia del mio vigore, eminente in dignità ed eminente in forza.
\par 4 Impetuoso come l'acqua, tu non avrai la preeminenza, perché sei salito sul letto di tuo padre. Allora tu l'hai profanato. Egli è salito sul mio letto.
\par 5 Simeone e Levi sono fratelli: le loro spade sono strumenti di violenza.
\par 6 Non entri l'anima mia nel loro consiglio segreto, non si unisca la mia gloria alla loro raunanza! Poiché nella loro ira hanno ucciso degli uomini, e nel loro mal animo han tagliato i garetti ai tori.
\par 7 Maledetta l'ira loro, perch'è stata violenta, e il loro furore perch'è stato crudele! Io li dividerò in Giacobbe, e li disperderò in Israele.
\par 8 Giuda, te loderanno i tuoi fratelli; la tua mano sarà sulla cervice de' tuoi nemici; i figliuoli di tuo padre si prostreranno dinanzi a te.
\par 9 Giuda è un giovine leone; tu risali dalla preda, figliuol mio; egli si china, s'accovaccia come un leone, come una leonessa: chi lo farà levare?
\par 10 Lo scettro non sarà rimosso da Giuda, né il bastone del comando di fra i suoi piedi, finché venga Colui che darà il riposo, e al quale ubbidiranno i popoli.
\par 11 Egli lega il suo asinello alla vite, e il puledro della sua asina, alla vite migliore; lava la sua veste col vino, e il suo manto col sangue dell'uva.
\par 12 Egli ha gli occhi rossi dal vino, e i denti bianchi dal latte.
\par 13 Zabulon abiterà sulla costa dei mari; sarà sulla costa ove convengon le navi e il suo fianco s'appoggerà a Sidon.
\par 14 Issacar è un asino robusto, sdraiato fra i tramezzi del chiuso.
\par 15 Egli ha visto che il riposo è buono, e che il paese è ameno; ha curvato la spalla per portare il peso, ed è divenuto un servo forzato al lavoro.
\par 16 Dan giudicherà il suo popolo, come una delle tribù d'Israele.
\par 17 Dan sarà una serpe sulla strada, una cerasta sul sentiero, che morde i talloni del cavallo, sì che il cavaliere cade all'indietro.
\par 18 Io ho aspettato la tua salvezza, o Eterno!
\par 19 Gad, l'assaliranno delle bande armate, ma egli a sua volta le assalirà, e le inseguirà.
\par 20 Da Ascer verrà il pane saporito, ed ei fornirà delizie reali.
\par 21 Neftali è una cerva messa in libertà; egli dice delle belle parole.
\par 22 Giuseppe è un ramo d'albero fruttifero; un ramo d'albero fruttifero vicino a una sorgente; i suoi rami si stendono sopra il muro.
\par 23 Gli arcieri l'hanno provocato, gli han lanciato dei dardi, l'hanno perseguitato;
\par 24 ma l'arco suo è rimasto saldo; le sue braccia e le sue mani sono state rinforzate dalle mani del Potente di Giacobbe, da colui ch'è il pastore e la roccia d'Israele,
\par 25 dall'Iddio di tuo padre che t'aiuterà, e dall'Altissimo che ti benedirà con benedizioni del cielo di sopra, con benedizioni dell'abisso che giace di sotto, con benedizioni delle mammelle e del seno materno.
\par 26 Le benedizioni di tuo padre sorpassano le benedizioni dei miei progenitori, fino a raggiungere la cima delle colline eterne. Esse saranno sul capo di Giuseppe, sulla fronte del principe dei suoi fratelli.
\par 27 Beniamino è un lupo rapace; la mattina divora la preda, e la sera spartisce le spoglie".
\par 28 Tutti costoro sono gli antenati delle dodici tribù d'Israele; e questo è quello che il loro padre disse loro, quando li benedisse. Li benedisse, dando a ciascuno la sua benedizione particolare.
\par 29 Poi dette loro i suoi ordini, e disse: 'Io sto per essere riunito al mio popolo; seppellitemi coi miei padri nella spelonca ch'è nel campo di Efron lo Hitteo,
\par 30 nella spelonca ch'è nel campo di Macpela, dirimpetto a Mamre, nel paese di Canaan, la quale Abrahamo comprò, col campo, da Efron lo Hitteo, come sepolcro di sua proprietà.
\par 31 Quivi furon sepolti Abrahamo e Sara sua moglie; quivi furon sepolti Isacco e Rebecca sua moglie, e quivi io seppellii Lea.
\par 32 Il campo e la spelonca che vi si trova, furon comprati dai figliuoli di Heth'.
\par 33 Quando Giacobbe ebbe finito di dare questi ordini ai suoi figliuoli, ritirò i piedi entro il letto, e spirò, e fu riunito al suo popolo.

\chapter{50}

\par 1 Allora Giuseppe si gettò sulla faccia di suo padre, pianse su lui, e lo baciò.
\par 2 Poi Giuseppe ordinò ai medici ch'erano al suo servizio, d'imbalsamare suo padre; e i medici imbalsamarono Israele.
\par 3 Ci vollero quaranta giorni; perché tanto è il tempo che s'impiega ad imbalsamare; e gli Egiziani lo piansero settanta giorni.
\par 4 E quando i giorni del lutto fatto per lui furon passati, Giuseppe parlò alla casa di Faraone, dicendo: 'Se ora ho trovato grazia agli occhi vostri, fate giungere agli orecchi di Faraone queste parole:
\par 5 Mio padre m'ha fatto giurare e m'ha detto: Ecco, io mi muoio; seppelliscimi nel mio sepolcro, che mi sono scavato nel paese di Canaan. Ora dunque, permetti ch'io salga e seppellisca mio padre; poi tornerò'.
\par 6 E Faraone rispose: 'Sali, e seppellisci tuo padre come t'ha fatto giurare'.
\par 7 Allora Giuseppe salì a seppellire suo padre; e con lui salirono tutti i servitori di Faraone, gli Anziani della sua casa e tutti gli Anziani del paese d'Egitto,
\par 8 e tutta la casa di Giuseppe e i suoi fratelli e la casa di suo padre. Non lasciarono nel paese di Goscen che i loro bambini, i loro greggi e i loro armenti.
\par 9 Con lui salirono pure carri e cavalieri; talché il corteggio era numerosissimo.
\par 10 E come furon giunti all'aia di Atad, ch'è oltre il Giordano, vi fecero grandi e profondi lamenti; e Giuseppe fece a suo padre un lutto di sette giorni.
\par 11 Or quando gli abitanti del paese, i Cananei, videro il lutto dell'aia di Atad, dissero: 'Questo è un grave lutto per gli Egiziani!' Perciò fu messo nome Abel-Mitsraim a quell'aia, ch'è oltre il Giordano.
\par 12 I figliuoli di Giacobbe fecero per lui quello ch'egli aveva ordinato loro:
\par 13 lo trasportarono nel paese di Canaan, e lo seppellirono nella spelonca del campo di Macpela, che Abrahamo avea comprato, col campo, da Efron lo Hitteo, come sepolcro di sua proprietà, dirimpetto a Mamre.
\par 14 Giuseppe, dopo ch'ebbe sepolto suo padre, se ne tornò in Egitto coi suoi fratelli e con tutti quelli ch'eran saliti con lui a seppellire suo padre.
\par 15 I fratelli di Giuseppe, quando videro che il loro padre era morto, dissero: 'Chi sa che Giuseppe non ci porti odio, e non ci renda tutto il male che gli abbiam fatto!'
\par 16 E mandarono a dire a Giuseppe: 'Tuo padre, prima di morire, dette quest'ordine:
\par 17 Dite così a Giuseppe: Deh, perdona ora ai tuoi fratelli il loro misfatto e il loro peccato; perché t'hanno fatto del male. Deh, perdona dunque ora il misfatto de' servi dell'Iddio di tuo padre!' E Giuseppe, quando gli fu parlato così, pianse.
\par 18 E i suoi fratelli vennero anch'essi, si prostrarono ai suoi piedi, e dissero: 'Ecco, siamo tuoi servi'.
\par 19 E Giuseppe disse loro: 'Non temete; poiché son io forse al posto di Dio?
\par 20 Voi avevate pensato del male contro a me; ma Dio ha pensato di convertirlo in bene, per compiere quello che oggi avviene: per conservare in vita un popolo numeroso.
\par 21 Ora dunque non temete; io sostenterò voi e i vostri figliuoli'. E li confortò, e parlò al loro cuore.
\par 22 Giuseppe dimorò in Egitto: egli, con la casa di suo padre; e visse centodieci anni.
\par 23 Giuseppe vide i figliuoli di Efraim, fino alla terza generazione; anche i figliuoli di Makir, figliuolo di Manasse, nacquero sulle sue ginocchia.
\par 24 E Giuseppe disse ai suoi fratelli: 'Io sto per morire; ma Dio per certo vi visiterà, e vi farà salire, da questo paese, nel paese che promise con giuramento ad Abrahamo, a Isacco e a Giacobbe'.
\par 25 E Giuseppe fece giurare i figliuoli d'Israele, dicendo: 'Iddio per certo vi visiterà; allora, trasportate di qui le mie ossa'.
\par 26 Poi Giuseppe morì, in età di centodieci anni; e fu imbalsamato, e posto in una bara in Egitto.


\end{document}