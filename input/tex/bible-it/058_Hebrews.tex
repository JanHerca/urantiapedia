\begin{document}

\title{Ebrei}


\chapter{1}

\par 1 Iddio, dopo aver in molte volte e in molte maniere parlato anticamente ai padri per mezzo de' profeti,
\par 2 in questi ultimi giorni ha parlato a noi mediante il suo Figliuolo, ch'Egli ha costituito erede di tutte le cose, mediante il quale pure ha creato i mondi;
\par 3 il quale, essendo lo splendore della sua gloria e l'impronta della sua essenza e sostenendo tutte le cose con la parola della sua potenza, quand'ebbe fatta la purificazione dei peccati, si pose a sedere alla destra della Maestà ne' luoghi altissimi,
\par 4 diventato così di tanto superiore agli angeli, di quanto il nome che ha eredato è più eccellente del loro.
\par 5 Infatti, a qual degli angeli diss'Egli mai: Tu sei il mio Figliuolo, oggi ti ho generato? e di nuovo: Io gli sarò Padre ed egli mi sarà Figliuolo?
\par 6 E quando di nuovo introduce il Primogenito nel mondo, dice: Tutti gli angeli di Dio l'adorino!
\par 7 E mentre degli angeli dice: Dei suoi angeli Ei fa dei venti, e dei suoi ministri fiamme di fuoco,
\par 8 dice del Figliuolo: Il tuo trono, o Dio, è ne' secoli dei secoli, e lo scettro di rettitudine è lo scettro del tuo regno.
\par 9 Tu hai amata la giustizia e hai odiata l'iniquità; perciò Dio, l'Iddio tuo, ha unto te d'olio di letizia, a preferenza dei tuoi compagni.
\par 10 E ancora: Tu, Signore, nel principio, fondasti la terra, e i cieli son opera delle tue mani.
\par 11 Essi periranno, ma tu dimori; invecchieranno tutti come un vestito,
\par 12 e li avvolgerai come un mantello, e saranno mutati; ma tu rimani lo stesso, e i tuoi anni non verranno meno.
\par 13 Ed a qual degli angeli diss'Egli mai: Siedi alla mia destra finché abbia fatto dei tuoi nemici lo sgabello dei tuoi piedi?
\par 14 Non sono eglino tutti spiriti ministratori, mandati a servire a pro di quelli che hanno da eredare la salvezza?

\chapter{2}

\par 1 Perciò bisogna che ci atteniamo vie più alle cose udite, che talora non siam portati via lungi da esse.
\par 2 Perché, se la parola pronunziata per mezzo d'angeli si dimostrò ferma, e ogni trasgressione e disubbidienza ricevette una giusta retribuzione,
\par 3 come scamperemo noi se trascuriamo una così grande salvezza? La quale, dopo essere stata prima annunziata dal Signore, ci è stata confermata da quelli che l'aveano udita,
\par 4 mentre Dio stesso aggiungeva la sua testimonianza alla loro, con de' segni e de' prodigi, con opere potenti svariate, e con doni dello Spirito Santo distribuiti secondo la sua volontà.
\par 5 Difatti, non è ad angeli ch'Egli ha sottoposto il mondo a venire del quale parliamo;
\par 6 anzi, qualcuno ha in un certo luogo attestato dicendo: Che cos'è l'uomo che tu ti ricordi di lui o il figliuol dell'uomo che tu ti curi di lui?
\par 7 Tu l'hai fatto di poco inferiore agli angeli; l'hai coronato di gloria e d'onore;
\par 8 tu gli hai posto ogni cosa sotto i piedi. Col sottoporgli tutte le cose, Egli non ha lasciato nulla che non gli sia sottoposto. Ma al presente non vediamo ancora che tutte le cose gli siano sottoposte;
\par 9 ben vediamo però colui che è stato fatto di poco inferiore agli angeli, cioè Gesù, coronato di gloria e d'onore a motivo della morte che ha patita, onde, per la grazia di Dio, gustasse la morte per tutti.
\par 10 Infatti, per condurre molti figliuoli alla gloria, ben s'addiceva a Colui per cagion del quale son tutte le cose e per mezzo del quale son tutte le cose, di rendere perfetto, per via di sofferenze, il duce della loro salvezza.
\par 11 Poiché e colui che santifica e quelli che son santificati, provengon tutti da uno; per la qual ragione egli non si vergogna di chiamarli fratelli,
\par 12 dicendo: Annunzierò il tuo nome ai miei fratelli; in mezzo alla raunanza canterò la tua lode.
\par 13 E di nuovo: Io metterò la mia fiducia in Lui. E di nuovo: Ecco me e i figliuoli che Dio mi ha dati.
\par 14 Poiché dunque i figliuoli partecipano del sangue e della carne, anch'egli vi ha similmente partecipato, affinché, mediante la morte, distruggesse colui che avea l'impero della morte, cioè il diavolo,
\par 15 e liberasse tutti quelli che per il timor della morte erano per tutta la vita soggetti a schiavitù.
\par 16 Poiché, certo, egli non viene in aiuto ad angeli, ma viene in aiuto alla progenie d'Abramo.
\par 17 Laonde egli doveva esser fatto in ogni cosa simile ai suoi fratelli, affinché diventasse un misericordioso e fedel sommo sacerdote nelle cose appartenenti a Dio, per compiere l'espiazione dei peccati del popolo.
\par 18 Poiché, in quanto egli stesso ha sofferto essendo tentato, può soccorrere quelli che son tentati.

\chapter{3}

\par 1 Perciò, fratelli santi, che siete partecipi d'una celeste vocazione, considerate Gesù, l'Apostolo e il Sommo Sacerdote della nostra professione di fede,
\par 2 il quale è fedele a Colui che l'ha costituito, come anche lo fu Mosè in tutta la casa di Dio.
\par 3 Poiché egli è stato reputato degno di tanta maggior gloria che Mosè, di quanto è maggiore l'onore di Colui che fabbrica la casa, in confronto di quello della casa stessa.
\par 4 Poiché ogni casa è fabbricata da qualcuno; ma chi ha fabbricato tutte le cose è Dio.
\par 5 E Mosè fu bensì fedele in tutta la casa di Dio come servitore per testimoniar delle cose che dovevano esser dette;
\par 6 ma Cristo lo è come Figlio, sopra la sua casa; e la sua casa siamo noi se riteniam ferma sino alla fine la nostra franchezza e il vanto della nostra speranza.
\par 7 Perciò, come dice lo Spirito Santo, Oggi, se udite la sua voce,
\par 8 non indurate i vostri cuori, come nel dì della provocazione, come nel dì della tentazione nel deserto
\par 9 dove i vostri padri mi tentarono mettendomi alla prova, e videro le mie opere per quarant'anni!
\par 10 Perciò mi disgustai di quella generazione, e dissi: Sempre erra il cuor loro; ed essi non han conosciuto le mie vie,
\par 11 talché giurai nell'ira mia: Non entreranno nel mio riposo!
\par 12 Guardate, fratelli, che talora non si trovi in alcuno di voi un malvagio cuore incredulo, che vi porti a ritrarvi dall'Iddio vivente;
\par 13 ma esortatevi gli uni gli altri tutti i giorni, finché si può dire: 'Oggi', onde nessuno di voi sia indurato per inganno del peccato;
\par 14 poiché siam diventati partecipi di Cristo, a condizione che riteniam ferma sino alla fine la fiducia che avevamo da principio,
\par 15 mentre ci vien detto: Oggi, se udite la sua voce, non indurate i vostri cuori, come nel dì della provocazione.
\par 16 Infatti, chi furon quelli che dopo averlo udito lo provocarono? Non furon forse tutti quelli ch'erano usciti dall'Egitto, condotti da Mosè?
\par 17 E chi furon quelli di cui si disgustò durante quarant'anni? Non furon essi quelli che peccarono, i cui cadaveri caddero nel deserto?
\par 18 E a chi giurò Egli che non entrerebbero nel suo riposo, se non a quelli che furon disubbidienti?
\par 19 E noi vediamo che non vi poterono entrare a motivo dell'incredulità.

\chapter{4}

\par 1 Temiamo dunque che talora, rimanendo una promessa d'entrare nel suo riposo, alcuno di voi non appaia esser rimasto indietro.
\par 2 Poiché a noi come a loro è stata annunziata una buona novella; ma la parola udita non giovò loro nulla non essendo stata assimilata per fede da quelli che l'avevano udita.
\par 3 Poiché noi che abbiam creduto entriamo in quel riposo, siccome Egli ha detto: Talché giurai nella mia ira: Non entreranno nel mio riposo! e così disse, benché le sue opere fossero terminate fin dalla fondazione del mondo.
\par 4 Perché in qualche luogo, a proposito del settimo giorno, è detto così: E Dio si riposò il settimo giorno da tutte le sue opere;
\par 5 e in questo passo di nuovo: Non entreranno nel mio riposo!
\par 6 Poiché dunque è riserbato ad alcuni d'entrarvi e quelli ai quali la buona novella fu prima annunziata non v'entrarono a motivo della loro disubbidienza,
\par 7 Egli determina di nuovo un giorno "Oggi" dicendo nei Salmi, dopo lungo tempo, come s'è detto dianzi: Oggi, se udite la sua voce, non indurate i vostri cuori!
\par 8 Infatti, se Giosuè avesse dato loro il riposo, Iddio non avrebbe di poi parlato d'un altro giorno.
\par 9 Resta dunque un riposo di sabato per il popolo di Dio;
\par 10 poiché chi entra nel riposo di Lui si riposa anch'egli dalle opere proprie, come Dio si riposò dalle sue.
\par 11 Studiamoci dunque d'entrare in quel riposo, onde nessuno cada seguendo lo stesso esempio di disubbidienza.
\par 12 Perché la parola di Dio è vivente ed efficace, e più affilata di qualunque spada a due tagli, e penetra fino alla divisione dell'anima e dello spirito, delle giunture e delle midolle; e giudica i sentimenti ed i pensieri del cuore.
\par 13 E non v'è creatura alcuna che sia occulta davanti a lui; ma tutte le cose sono nude e scoperte dinanzi agli occhi di Colui al quale abbiam da render ragione.
\par 14 Avendo noi dunque un gran Sommo Sacerdote che è passato attraverso i cieli, Gesù, il Figliuol di Dio, riteniamo fermamente la professione della nostra fede.
\par 15 Perché non abbiamo un Sommo Sacerdote che non possa simpatizzare con noi nelle nostre infermità; ma ne abbiamo uno che in ogni cosa è stato tentato come noi, però senza peccare.
\par 16 Accostiamoci dunque con piena fiducia al trono della grazia, affinché otteniamo misericordia e troviamo grazia per esser soccorsi al momento opportuno.

\chapter{5}

\par 1 Poiché ogni sommo sacerdote, preso di fra gli uomini, è costituito a pro degli uomini, nelle cose concernenti Dio, affinché offra doni e sacrificî per i peccati;
\par 2 e può aver convenevole compassione verso gl'ignoranti e gli erranti, perché anch'egli è circondato da infermità;
\par 3 ed è a cagion di questa ch'egli è obbligato ad offrir dei sacrificî per i peccati, tanto per se stesso quanto per il popolo.
\par 4 E nessuno si prende da sé quell'onore; ma lo prende quando sia chiamato da Dio, come nel caso d'Aronne.
\par 5 Così anche Cristo non si prese da sé la gloria d'esser fatto Sommo Sacerdote; ma l'ebbe da Colui che gli disse: Tu sei il mio Figliuolo; oggi t'ho generato;
\par 6 come anche in altro luogo Egli dice: Tu sei sacerdote in eterno secondo l'ordine di Melchisedec.
\par 7 Il quale, ne' giorni della sua carne, avendo con gran grida e con lagrime offerto preghiere e supplicazioni a Colui che lo potea salvar dalla morte, ed avendo ottenuto d'esser liberato dal timore,
\par 8 benché fosse figliuolo, imparò l'ubbidienza dalle cose che soffrì;
\par 9 ed essendo stato reso perfetto, divenne per tutti quelli che gli ubbidiscono,
\par 10 autore d'una salvezza eterna, essendo da Dio proclamato Sommo Sacerdote secondo l'ordine di Melchisedec.
\par 11 Del quale abbiamo a dir cose assai, e malagevoli a spiegare, perché siete diventati duri d'orecchi.
\par 12 Poiché, mentre per ragion di tempo dovreste esser maestri, avete di nuovo bisogno che vi s'insegnino i primi elementi degli oracoli di Dio; e siete giunti a tale che avete bisogno di latte e non di cibo sodo.
\par 13 Perché chiunque usa il latte non ha esperienza della parola della giustizia, poiché è bambino;
\par 14 ma il cibo sodo è per uomini fatti; per quelli, cioè, che per via dell'uso hanno i sensi esercitati a discernere il bene e il male.

\chapter{6}

\par 1 Perciò, lasciando l'insegnamento elementare intorno a Cristo, tendiamo a quello perfetto, e non stiamo a porre di nuovo il fondamento del ravvedimento dalle opere morte e della fede in Dio,
\par 2 della dottrina dei battesimi e della imposizione delle mani, della risurrezione de' morti e del giudizio eterno.
\par 3 E così faremo, se pur Dio lo permette.
\par 4 Perché quelli che sono stati una volta illuminati e hanno gustato il dono celeste e sono stati fatti partecipi dello Spirito Santo
\par 5 e hanno gustato la buona parola di Dio e le potenze del mondo a venire,
\par 6 se cadono, è impossibile rinnovarli da capo a ravvedimento, poiché crocifiggono di nuovo per conto loro il Figliuol di Dio, e lo espongono ad infamia.
\par 7 Infatti, la terra che beve la pioggia che viene spesse volte su lei, e produce erbe utili a quelli per i quali è coltivata, riceve benedizione da Dio;
\par 8 ma se porta spine e triboli, è riprovata e vicina ad esser maledetta; e la sua fine è d'esser arsa.
\par 9 Peraltro, diletti, quantunque parliamo così, siamo persuasi, riguardo a voi, di cose migliori e attinenti alla salvezza;
\par 10 poiché Dio non è ingiusto da dimenticare l'opera vostra e l'amore che avete mostrato verso il suo nome coi servizî che avete reso e che rendete tuttora ai santi.
\par 11 Ma desideriamo che ciascun di voi dimostri fino alla fine il medesimo zelo per giungere alla pienezza della speranza,
\par 12 onde non diventiate indolenti ma siate imitatori di quelli che per fede e pazienza eredano le promesse.
\par 13 Poiché, quando Iddio fece la promessa ad Abramo, siccome non potea giurare per alcuno maggiore di lui, giurò per se stesso,
\par 14 dicendo: Certo, ti benedirò e ti moltiplicherò grandemente.
\par 15 E così, avendo aspettato con pazienza, Abramo ottenne la promessa.
\par 16 Perché gli uomini giurano per qualcuno maggiore di loro; e per essi il giuramento è la conferma che pone fine ad ogni contestazione.
\par 17 Così, volendo Iddio mostrare vie meglio agli eredi della promessa la immutabilità del suo consiglio, intervenne con un giuramento,
\par 18 affinché, mediante due cose immutabili, nelle quali è impossibile che Dio abbia mentito, troviamo una potente consolazione noi, che abbiam cercato il nostro rifugio nell'afferrar saldamente la speranza che ci era posta dinanzi,
\par 19 la quale noi teniamo qual àncora dell'anima, sicura e ferma e penetrante di là dalla cortina,
\par 20 dove Gesù è entrato per noi qual precursore, essendo divenuto Sommo Sacerdote in eterno, secondo l'ordine di Melchisedec.

\chapter{7}

\par 1 Poiché questo Melchisedec, re di Salem, sacerdote dell'Iddio altissimo, che andò incontro ad Abramo quand'egli tornava dalla sconfitta dei re e lo benedisse,
\par 2 a cui Abramo diede anche la decima d'ogni cosa, il quale in prima, secondo la interpretazione del suo nome, è Re di giustizia, e poi anche Re di Salem, vale a dire Re di pace,
\par 3 senza padre, senza madre, senza genealogia, senza principio di giorni né fin di vita, ma rassomigliato al Figliuol di Dio, questo Melchisedec rimane sacerdote in perpetuo.
\par 4 Or considerate quanto grande fosse colui al quale Abramo, il patriarca, dette la decima del meglio della preda.
\par 5 Or quelli d'infra i figliuoli di Levi che ricevono il sacerdozio, hanno bensì ordine, secondo la legge, di prender le decime dal popolo, cioè dai loro fratelli, benché questi siano usciti dai lombi d'Abramo;
\par 6 quello, invece, che non è della loro stirpe, prese la decima da Abramo e benedisse colui che avea le promesse!
\par 7 Ora, senza contraddizione, l'inferiore è benedetto dal superiore;
\par 8 e poi, qui, quelli che prendon le decime son degli uomini mortali; ma là le prende uno di cui si attesta che vive.
\par 9 E, per così dire, nella persona d'Abramo, Levi stesso, che prende le decime, fu sottoposto alla decima;
\par 10 perch'egli era ancora ne' lombi di suo padre, quando Melchisedec incontrò Abramo.
\par 11 Ora, se la perfezione fosse stata possibile per mezzo del sacerdozio levitico (perché su quello è basata la legge data al popolo), che bisogno c'era ancora che sorgesse un altro sacerdote secondo l'ordine di Melchisedec e non scelto secondo l'ordine d'Aronne?
\par 12 Poiché, mutato il sacerdozio, avviene per necessità anche un mutamento di legge.
\par 13 Difatti, colui a proposito del quale queste parole son dette, ha appartenuto a un'altra tribù, della quale nessuno s'è accostato all'altare;
\par 14 perché è ben noto che il nostro Signore è sorto dalla tribù di Giuda, circa la quale Mosè non disse nulla che concernesse il sacerdozio.
\par 15 E la cosa è ancora vie più evidente se sorge, a somiglianza di Melchisedec,
\par 16 un altro sacerdote che è stato fatto tale non a tenore di una legge dalle prescrizioni carnali, ma in virtù della potenza di una vita indissolubile;
\par 17 poiché gli è resa questa testimonianza: Tu sei sacerdote in eterno secondo l'ordine di Melchisedec.
\par 18 Giacché qui v'è bensì l'abrogazione del comandamento precedente a motivo della sua debolezza e inutilità
\par 19 (poiché la legge non ha condotto nulla a compimento), ma v'è altresì l'introduzione d'una migliore speranza, mediante la quale ci accostiamo a Dio.
\par 20 E in quanto ciò non è avvenuto senza giuramento (poiché quelli sono stati fatti sacerdoti senza giuramento,
\par 21 ma egli lo è con giuramento, per opera di Colui che ha detto: Il Signore l'ha giurato e non si pentirà: tu sei sacerdote in eterno),
\par 22 è di tanto più eccellente del primo il patto del quale Gesù è divenuto garante.
\par 23 Inoltre, quelli sono stati fatti sacerdoti in gran numero, perché per la morte erano impediti di durare;
\par 24 ma questi, perché dimora in eterno, ha un sacerdozio che non si trasmette;
\par 25 ond'è che può anche salvar appieno quelli che per mezzo di lui si accostano a Dio, vivendo egli sempre per intercedere per loro.
\par 26 E infatti a noi conveniva un sacerdote come quello, santo, innocente, immacolato, separato dai peccatori ed elevato al disopra de' cieli;
\par 27 il quale non ha ogni giorno bisogno, come gli altri sommi sacerdoti, d'offrir de' sacrificî prima per i proprî peccati e poi per quelli del popolo; perché questo egli ha fatto una volta per sempre, quando ha offerto se stesso.
\par 28 La legge infatti costituisce sommi sacerdoti uomini soggetti a infermità; ma la parola del giuramento fatto dopo la legge costituisce il Figliuolo, che è stato reso perfetto per sempre.

\chapter{8}

\par 1 Ora, il punto capitale delle cose che stiamo dicendo, è questo: che abbiamo un tal Sommo Sacerdote, che si è posto a sedere alla destra del trono della Maestà nei cieli,
\par 2 ministro del santuario e del vero tabernacolo, che il Signore, e non un uomo, ha eretto.
\par 3 Poiché ogni sommo sacerdote è costituito per offrir doni e sacrificî; ond'è necessario che anche questo Sommo Sacerdote abbia qualcosa da offrire.
\par 4 Or, se fosse sulla terra, egli non sarebbe neppur sacerdote, perché ci son quelli che offrono i doni secondo la legge,
\par 5 i quali ministrano in quel che è figura e ombra delle cose celesti, secondo che fu detto da Dio a Mosè quando questi stava per costruire il tabernacolo: Guarda, Egli disse, di fare ogni cosa secondo il modello che ti è stato mostrato sul monte.
\par 6 Ma ora egli ha ottenuto un ministerio di tanto più eccellente, ch'egli è mediatore d'un patto anch'esso migliore, fondato su migliori promesse.
\par 7 Poiché se quel primo patto fosse stato senza difetto, non si sarebbe cercato luogo per un secondo.
\par 8 Difatti, Iddio, biasimando il popolo, dice: Ecco i giorni vengono, dice il Signore, che io concluderò con la casa d'Israele e con la casa di Giuda, un patto nuovo;
\par 9 non un patto come quello che feci coi loro padri nel giorno che li presi per la mano per trarli fuori dal paese d'Egitto; perché essi non han perseverato nel mio patto, ed io alla mia volta non mi son curato di loro, dice il Signore.
\par 10 E questo è il patto che farò con la casa d'Israele dopo quei giorni, dice il Signore: Io porrò le mie leggi nelle loro menti, e le scriverò sui loro cuori; e sarò il loro Dio, ed essi saranno il mio popolo.
\par 11 E non istruiranno più ciascuno il proprio concittadino e ciascuno il proprio fratello, dicendo: Conosci il Signore! Perché tutti mi conosceranno, dal minore al maggiore di loro,
\par 12 poiché avrò misericordia delle loro iniquità, e non mi ricorderò più dei loro peccati.
\par 13 Dicendo: Un nuovo patto, Egli ha dichiarato antico il primo. Ora, quel che diventa antico e invecchia è vicino a sparire.

\chapter{9}

\par 1 Or anche il primo patto avea delle norme per il culto e un santuario terreno.
\par 2 Infatti fu preparato un primo tabernacolo, nel quale si trovavano il candeliere, la tavola, e la presentazione de' pani; e questo si chiamava il Luogo santo.
\par 3 E dietro la seconda cortina v'era il tabernacolo detto il Luogo santissimo,
\par 4 contenente un turibolo d'oro, e l'arca del patto, tutta ricoperta d'oro, nella quale si trovavano un vaso d'oro contenente la manna, la verga d'Aronne che avea fiorito, e le tavole del patto.
\par 5 E sopra l'arca, i cherubini della gloria, che adombravano il propiziatorio. Delle quali cose non possiamo ora parlare partitamente.
\par 6 Or essendo le cose così disposte, i sacerdoti entrano bensì continuamente nel primo tabernacolo per compiervi gli atti del culto;
\par 7 ma nel secondo, entra una volta solamente all'anno il solo sommo sacerdote, e non senza sangue, il quale egli offre per se stesso e per gli errori del popolo.
\par 8 Lo Spirito Santo volea con questo significare che la via al santuario non era ancora manifestata finché sussisteva ancora il primo tabernacolo.
\par 9 Esso è una figura per il tempo attuale, conformemente alla quale s'offron doni e sacrificî che non possono, quanto alla coscienza, render perfetto colui che offre il culto,
\par 10 poiché si tratta solo di cibi, di bevande e di varie abluzioni, insomma, di regole carnali imposte fino al tempo della riforma.
\par 11 Ma venuto Cristo, Sommo Sacerdote dei futuri beni, egli, attraverso il tabernacolo più grande e più perfetto, non fatto con mano, vale a dire non di questa creazione,
\par 12 e non mediante il sangue di becchi e di vitelli, ma mediante il proprio sangue, è entrato una volta per sempre nel santuario, avendo acquistata una redenzione eterna.
\par 13 Perché, se il sangue di becchi e di tori e la cenere d'una giovenca sparsa su quelli che son contaminati santificano in modo da dar la purità della carne,
\par 14 quanto più il sangue di Cristo che mediante lo Spirito eterno ha offerto se stesso puro d'ogni colpa a Dio, purificherà la vostra coscienza dalle opere morte per servire all'Iddio vivente?
\par 15 Ed è per questa ragione che egli è mediatore d'un nuovo patto, affinché, avvenuta la sua morte per la redenzione delle trasgressioni commesse sotto il primo patto, i chiamati ricevano l'eterna eredità promessa.
\par 16 Infatti, dove c'è un testamento, bisogna che sia accertata la morte del testatore.
\par 17 Perché un testamento è valido quand'è avvenuta la morte; poiché non ha valore finché vive il testatore.
\par 18 Ond'è che anche il primo patto non è stato inaugurato senza sangue.
\par 19 Difatti, quando tutti i comandamenti furono secondo la legge proclamati da Mosè a tutto il popolo, egli prese il sangue de' vitelli e de' becchi con acqua, lana scarlatta ed issopo, e ne asperse il libro stesso e tutto il popolo,
\par 20 dicendo: Questo è il sangue del patto che Dio ha ordinato sia fatto con voi.
\par 21 E parimente asperse di sangue il tabernacolo e tutti gli arredi del culto.
\par 22 E secondo la legge, quasi ogni cosa è purificata con sangue; e senza spargimento di sangue non c'è remissione.
\par 23 Era dunque necessario che le cose raffiguranti quelle nei cieli fossero purificate con questi mezzi, ma le cose celesti stesse doveano esserlo con sacrificî più eccellenti di questi.
\par 24 Poiché Cristo non è entrato in un santuario fatto con mano, figura del vero; ma nel cielo stesso, per comparire ora, al cospetto di Dio, per noi;
\par 25 e non per offrir se stesso più volte, come il sommo sacerdote, che entra ogni anno nel santuario con sangue non suo;
\par 26 ché, in questo caso, avrebbe dovuto soffrir più volte dalla fondazione del mondo; ma ora, una volta sola, alla fine de' secoli, è stato manifestato, per annullare il peccato col suo sacrificio.
\par 27 E come è stabilito che gli uomini muoiano una volta sola, dopo di che viene il giudizio,
\par 28 così anche Cristo, dopo essere stato offerto una volta sola, per portare i peccati di molti, apparirà una seconda volta, senza peccato, a quelli che l'aspettano per la loro salvezza.

\chapter{10}

\par 1 Poiché la legge, avendo un'ombra dei futuri beni, non la realtà stessa delle cose, non può mai con quegli stessi sacrificî, che sono offerti continuamente, anno dopo anno, render perfetti quelli che s'accostano a Dio.
\par 2 Altrimenti non si sarebb'egli cessato d'offrirli, non avendo più gli adoratori, una volta purificati, alcuna coscienza di peccati?
\par 3 Invece in quei sacrificî è rinnovato ogni anno il ricordo dei peccati;
\par 4 perché è impossibile che il sangue di tori e di becchi tolga i peccati.
\par 5 Perciò, entrando nel mondo, egli dice: Tu non hai voluto né sacrificio né offerta, ma mi hai preparato un corpo;
\par 6 non hai gradito né olocausti né sacrificî per il peccato.
\par 7 Allora ho detto: Ecco, io vengo (nel rotolo del libro è scritto di me) per fare, o Dio, la tua volontà.
\par 8 Dopo aver detto prima: Tu non hai voluto e non hai gradito né sacrificî, né offerte, né olocausti, né sacrificî per il peccato (i quali sono offerti secondo la legge), egli dice poi:
\par 9 Ecco, io vengo per fare la tua volontà. Egli toglie via il primo per stabilire il secondo.
\par 10 In virtù di questa "volontà" noi siamo stati santificati, mediante l'offerta del corpo di Gesù Cristo fatta una volta per sempre.
\par 11 E mentre ogni sacerdote è in piè ogni giorno ministrando e offrendo spesse volte gli stessi sacrificî che non possono mai togliere i peccati,
\par 12 questi, dopo aver offerto un unico sacrificio per i peccati, e per sempre, si è posto a sedere alla destra di Dio,
\par 13 aspettando solo più che i suoi nemici sian ridotti ad essere lo sgabello dei suoi piedi.
\par 14 Perché con un'unica offerta egli ha per sempre resi perfetti quelli che son santificati.
\par 15 E anche lo Spirito Santo ce ne rende testimonianza. Infatti, dopo aver detto:
\par 16 Questo è il patto che farò con loro dopo que' giorni, dice il Signore: Io metterò le mie leggi ne' loro cuori, e le scriverò nelle loro menti, egli aggiunge:
\par 17 E non mi ricorderò più de' loro peccati e delle loro iniquità.
\par 18 Ora, dov'è remissione di queste cose, non c'è più luogo a offerta per il peccato.
\par 19 Avendo dunque, fratelli, libertà d'entrare nel santuario in virtù del sangue di Gesù,
\par 20 per quella via recente e vivente che egli ha inaugurata per noi attraverso la cortina, vale a dire la sua carne,
\par 21 e avendo noi un gran Sacerdote sopra la casa di Dio,
\par 22 accostiamoci di vero cuore, con piena certezza di fede, avendo i cuori aspersi di quell'aspersione che li purifica dalla mala coscienza, e il corpo lavato d'acqua pura.
\par 23 Riteniam fermamente la confessione della nostra speranza, senza vacillare; perché fedele è Colui che ha fatte le promesse.
\par 24 E facciamo attenzione gli uni agli altri per incitarci a carità e a buone opere,
\par 25 non abbandonando la nostra comune adunanza come alcuni son usi di fare, ma esortandoci a vicenda; e tanto più, che vedete avvicinarsi il gran giorno.
\par 26 Perché, se pecchiamo volontariamente dopo aver ricevuto la conoscenza della verità, non resta più alcun sacrificio per i peccati;
\par 27 rimangono una terribile attesa del giudizio e l'ardor d'un fuoco che divorerà gli avversarî.
\par 28 Uno che abbia violato la legge di Mosè, muore senza misericordia sulla parola di due o tre testimoni.
\par 29 Di qual peggior castigo stimate voi che sarà giudicato degno colui che avrà calpestato il Figliuol di Dio e avrà tenuto per profano il sangue del patto col quale è stato santificato, e avrà oltraggiato lo Spirito della grazia?
\par 30 Poiché noi sappiamo chi è Colui che ha detto: A me appartiene la vendetta! Io darò la retribuzione! E ancora: Il Signore giudicherà il suo popolo.
\par 31 È cosa spaventevole cadere nelle mani dell'Iddio vivente.
\par 32 Ma ricordatevi dei giorni di prima, quando, dopo essere stati illuminati, voi sosteneste una così gran lotta di patimenti:
\par 33 sia coll'essere esposti a vituperio e ad afflizioni, sia coll'esser partecipi della sorte di quelli che eran così trattati.
\par 34 Infatti, voi simpatizzaste coi carcerati, e accettaste con allegrezza la ruberia de' vostri beni, sapendo d'aver per voi una sostanza migliore e permanente.
\par 35 Non gettate dunque via la vostra franchezza la quale ha una grande ricompensa!
\par 36 Poiché voi avete bisogno di costanza, affinché, avendo fatta la volontà di Dio, otteniate quel che v'è promesso. Perché:
\par 37 Ancora un brevissimo tempo, e colui che ha da venire verrà e non tarderà;
\par 38 ma il mio giusto vivrà per fede; e se si trae indietro, l'anima mia non lo gradisce.
\par 39 Ma noi non siamo di quelli che si traggono indietro a loro perdizione, ma di quelli che hanno fede per salvar l'anima.

\chapter{11}

\par 1 Or la fede è certezza di cose che si sperano, dimostrazione di cose che non si vedono.
\par 2 Infatti, per essa fu resa buona testimonianza agli antichi.
\par 3 Per fede intendiamo che i mondi sono stati formati dalla parola di Dio; cosicché le cose che si vedono non sono state tratte da cose apparenti.
\par 4 Per fede Abele offerse a Dio un sacrificio più eccellente di quello di Caino; per mezzo d'essa gli fu resa testimonianza ch'egli era giusto, quando Dio attestò di gradire le sue offerte; e per mezzo d'essa, benché morto, egli parla ancora.
\par 5 Per fede Enoc fu trasportato perché non vedesse la morte; e non fu più trovato, perché Dio l'avea trasportato; poiché avanti che fosse trasportato fu di lui testimoniato ch'egli era piaciuto a Dio.
\par 6 Or senza fede è impossibile piacergli; poiché chi s'accosta a Dio deve credere ch'Egli è, e che è il rimuneratore di quelli che lo cercano.
\par 7 Per fede Noè, divinamente avvertito di cose che non si vedevano ancora, mosso da pio timore, preparò un'arca per la salvezza della propria famiglia; e per essa fede condannò il mondo e fu fatto erede della giustizia che si ha mediante la fede.
\par 8 Per fede Abramo, essendo chiamato, ubbidì, per andarsene in un luogo ch'egli avea da ricevere in eredità; e partì senza sapere dove andava.
\par 9 Per fede soggiornò nella terra promessa, come in terra straniera, abitando in tende con Isacco e Giacobbe, eredi con lui della stessa promessa,
\par 10 perché aspettava la città che ha i veri fondamenti e il cui architetto e costruttore è Dio.
\par 11 Per fede Sara anch'ella benché fuori d'età, ricevette forza di concepire, perché reputò fedele Colui che avea fatto la promessa.
\par 12 E perciò, da uno solo, e già svigorito, è nata una discendenza numerosa come le stelle del cielo, come la rena lungo la riva del mare che non si può contare.
\par 13 In fede moriron tutti costoro, senz'aver ricevuto le cose promesse, ma avendole vedute e salutate da lontano, e avendo confessato che erano forestieri e pellegrini sulla terra.
\par 14 Poiché quelli che dicon tali cose dimostrano che cercano una patria.
\par 15 E se pur si ricordavano di quella ond'erano usciti, certo avean tempo di ritornarvi.
\par 16 Ma ora ne desiderano una migliore, cioè una celeste; perciò Iddio non si vergogna d'esser chiamato il loro Dio, poiché ha preparato loro una città.
\par 17 Per fede Abramo, quando fu provato, offerse Isacco; ed egli, che avea ricevuto le promesse, offerse il suo unigenito: egli, a cui era stato detto:
\par 18 È in Isacco che ti sarà chiamata una progenie,
\par 19 ritenendo che Dio è potente anche da far risuscitare dai morti; ond'è che lo riebbe per una specie di risurrezione.
\par 20 Per fede Isacco diede a Giacobbe e ad Esaù una benedizione concernente cose future.
\par 21 Per fede Giacobbe, morente, benedisse ciascuno dei figliuoli di Giuseppe, e adorò appoggiato in cima al suo bastone.
\par 22 Per fede Giuseppe, quando stava per morire, fece menzione dell'èsodo de' figliuoli d'Israele, e diede ordini intorno alle sue ossa.
\par 23 Per fede Mosè, quando nacque, fu tenuto nascosto per tre mesi dai suoi genitori, perché vedevano che il bambino era bello; e non temettero il comandamento del re.
\par 24 Per fede Mosè, divenuto grande, rifiutò d'esser chiamato figliuolo della figliuola di Faraone,
\par 25 scegliendo piuttosto d'esser maltrattato col popolo di Dio, che di godere per breve tempo i piaceri del peccato;
\par 26 stimando egli il vituperio di Cristo ricchezza maggiore de' tesori d'Egitto, perché riguardava alla rimunerazione.
\par 27 Per fede abbandonò l'Egitto, non temendo l'ira del re, perché stette costante, come vedendo Colui che è invisibile.
\par 28 Per fede celebrò la Pasqua e fece lo spruzzamento del sangue affinché lo sterminatore dei primogeniti non toccasse quelli degli Israeliti.
\par 29 Per fede passarono il Mar Rosso come per l'asciutto; il che tentando di fare gli Egizî, furono inabissati.
\par 30 Per fede caddero le mura di Gerico, dopo essere state circuite per sette giorni.
\par 31 Per fede Raab, la meretrice, non perì coi disubbidienti, avendo accolto le spie in pace.
\par 32 E che dirò di più? poiché il tempo mi verrebbe meno se narrassi di Gedeone, di Barac, di Sansone, di Jefte, di Davide, di Samuele e dei profeti,
\par 33 i quali per fede vinsero regni, operarono giustizia, ottennero adempimento di promesse, turaron le gole di leoni,
\par 34 spensero la violenza del fuoco, scamparono al taglio della spada, guarirono da infermità, divennero forti in guerra, misero in fuga eserciti stranieri.
\par 35 Le donne ricuperarono per risurrezione i loro morti; e altri furon martirizzati non avendo accettata la loro liberazione affin di ottenere una risurrezione migliore;
\par 36 altri patirono scherni e flagelli, e anche catene e prigione.
\par 37 Furon lapidati, furon segati, furono uccisi di spada; andarono attorno coperti di pelli di pecora e di capra; bisognosi, afflitti,
\par 38 maltrattati (di loro il mondo non era degno), vaganti per deserti e monti e spelonche e per le grotte della terra.
\par 39 E tutti costoro, pur avendo avuta buona testimonianza per la loro fede, non ottennero quello ch'era stato promesso,
\par 40 perché Iddio aveva in vista per noi qualcosa di meglio, ond'essi non giungessero alla perfezione senza di noi.

\chapter{12}

\par 1 Anche noi, dunque, poiché siam circondati da sì gran nuvolo di testimoni, deposto ogni peso e il peccato che così facilmente ci avvolge, corriamo con perseveranza l'arringo che ci sta dinanzi, riguardando a Gesù,
\par 2 duce e perfetto esempio di fede, il quale per la gioia che gli era posta dinanzi sopportò la croce sprezzando il vituperio, e s'è posto a sedere alla destra del trono di Dio.
\par 3 Poiché, considerate colui che sostenne una tale opposizione dei peccatori contro a sé, onde non abbiate a stancarvi, perdendovi d'animo.
\par 4 Voi non avete ancora resistito fino al sangue, lottando contro il peccato;
\par 5 e avete dimenticata l'esortazione a voi rivolta come a figliuoli: Figliuol mio, non far poca stima della disciplina del Signore, e non ti perder d'animo quando sei da lui ripreso;
\par 6 perché il Signore corregge colui ch'Egli ama, e flagella ogni figliuolo ch'Egli gradisce.
\par 7 È a scopo di disciplina che avete a sopportar queste cose. Iddio vi tratta come figliuoli; poiché qual è il figliuolo che il padre non corregga?
\par 8 Che se siete senza quella disciplina della quale tutti hanno avuto la loro parte, siete dunque bastardi, e non figliuoli.
\par 9 Inoltre, abbiamo avuto per correttori i padri della nostra carne, eppur li abbiamo riveriti; non ci sottoporremo noi molto più al Padre degli spiriti per aver vita?
\par 10 Quelli, infatti, per pochi giorni, come parea loro, ci correggevano; ma Egli lo fa per l'util nostro, affinché siamo partecipi della sua santità.
\par 11 Or ogni disciplina sembra, è vero, per il presente non esser causa d'allegrezza, ma di tristizia; però rende poi un pacifico frutto di giustizia a quelli che sono stati per essa esercitati.
\par 12 Perciò, rinfrancate le mani cadenti e le ginocchia vacillanti;
\par 13 e fate de' sentieri diritti per i vostri passi, affinché quel che è zoppo non esca fuor di strada, ma sia piuttosto guarito.
\par 14 Procacciate pace con tutti e la santificazione senza la quale nessuno vedrà il Signore;
\par 15 badando bene che nessuno resti privo della grazia di Dio; che nessuna radice velenosa venga fuori a darvi molestia sì che molti di voi restino infetti;
\par 16 che nessuno sia fornicatore, o profano, come Esaù che per una sola pietanza vendette la sua primogenitura.
\par 17 Poiché voi sapete che anche quando più tardi volle eredare la benedizione fu respinto, perché non trovò luogo a pentimento, sebbene la richiedesse con lagrime.
\par 18 Poiché voi non siete venuti al monte che si toccava con la mano, avvolto nel fuoco, né alla caligine, né alla tenebria, né alla tempesta,
\par 19 né al suono della tromba, né alla voce che parlava in modo che quelli che la udirono richiesero che niuna parola fosse loro più rivolta
\par 20 perché non poteano sopportar l'ordine: Se anche una bestia tocchi il monte sia lapidata;
\par 21 e tanto spaventevole era lo spettacolo, che Mosè disse: Io son tutto spaventato e tremante;
\par 22 ma voi siete venuti al monte di Sion, e alla città dell'Iddio vivente, che è la Gerusalemme celeste, e alla festante assemblea delle miriadi degli angeli,
\par 23 e alla Chiesa de' primogeniti che sono scritti nei cieli, e a Dio, il Giudice di tutti, e agli spiriti de' giusti resi perfetti,
\par 24 e a Gesù, il mediatore del nuovo patto, e al sangue dell'aspersione che parla meglio di quello d'Abele.
\par 25 Guardate di non rifiutare Colui che parla; perché, se quelli non scamparono quando rifiutarono Colui che rivelava loro in terra la sua volontà, molto meno scamperemo noi se voltiam le spalle a Colui che parla dal cielo;
\par 26 la cui voce scosse allora la terra, ma che adesso ha fatto questa promessa: Ancora una volta farò tremare non solo la terra, ma anche il cielo.
\par 27 Or questo 'ancora una volta' indica la remozione delle cose scosse, come di cose fatte, onde sussistan ferme quelle che non sono scosse.
\par 28 Perciò, ricevendo un regno che non può essere scosso, siamo riconoscenti, e offriamo così a Dio un culto accettevole, con riverenza e timore!
\par 29 Perché il nostro Dio è anche un fuoco consumante.

\chapter{13}

\par 1 L'amor fraterno continui fra voi. Non dimenticate l'ospitalità;
\par 2 perché, praticandola, alcuni, senza saperlo, hanno albergato degli angeli.
\par 3 Ricordatevi de' carcerati, come se foste in carcere con loro; di quelli che sono maltrattati, ricordando che anche voi siete nel corpo.
\par 4 Sia il matrimonio tenuto in onore da tutti, e sia il talamo incontaminato; poiché Iddio giudicherà i fornicatori e gli adulteri.
\par 5 Non siate amanti del danaro, siate contenti delle cose che avete; poiché Egli stesso ha detto: Io non ti lascerò, e non ti abbandonerò.
\par 6 Talché possiam dire con piena fiducia: Il Signore è il mio aiuto; non temerò. Che mi potrà far l'uomo?
\par 7 Ricordatevi dei vostri conduttori, i quali v'hanno annunziato la parola di Dio; e considerando com'hanno finito la loro carriera, imitate la loro fede.
\par 8 Gesù Cristo è lo stesso ieri, oggi, e in eterno.
\par 9 Non siate trasportati qua e là da diverse e strane dottrine; poiché è bene che il cuore sia reso saldo dalla grazia, e non da pratiche relative a vivande, dalle quali non ritrassero alcun giovamento quelli che le osservarono.
\par 10 Noi abbiamo un altare del quale non hanno diritto di mangiare quelli che servono il tabernacolo.
\par 11 Poiché i corpi degli animali il cui sangue è portato dal sommo sacerdote nel santuario come un'offerta per il peccato, sono arsi fuori del campo.
\par 12 Perciò anche Gesù, per santificare il popolo col proprio sangue, soffrì fuor della porta.
\par 13 Usciamo quindi fuori del campo e andiamo a lui, portando il suo vituperio.
\par 14 Poiché non abbiam qui una città stabile, ma cerchiamo quella futura.
\par 15 Per mezzo di lui, dunque, offriam del continuo a Dio un sacrificio di lode: cioè, il frutto di labbra confessanti il suo nome!
\par 16 E non dimenticate di esercitar la beneficenza e di far parte agli altri de' vostri beni; perché è di tali sacrificî che Dio si compiace.
\par 17 Ubbidite ai vostri conduttori e sottomettetevi a loro, perché essi vegliano per le vostre anime, come chi ha da renderne conto; affinché facciano questo con allegrezza e non sospirando; perché ciò non vi sarebbe d'alcun utile.
\par 18 Pregate per noi, perché siam persuasi d'aver una buona coscienza, desiderando di condurci onestamente in ogni cosa.
\par 19 E vie più v'esorto a farlo, onde io vi sia più presto restituito.
\par 20 Or l'Iddio della pace che in virtù del sangue del patto eterno ha tratto dai morti il gran Pastore delle pecore, Gesù nostro Signore,
\par 21 vi renda compiuti in ogni bene, onde facciate la sua volontà, operando in voi quel che è gradito nel suo cospetto, per mezzo di Gesù Cristo; a Lui sia la gloria ne' secoli dei secoli. Amen.
\par 22 Or, fratelli, comportate, vi prego, la mia parola d'esortazione; perché v'ho scritto brevemente.
\par 23 Sappiate che il nostro fratello Timoteo è stato messo in libertà; con lui, se vien presto, io vi vedrò.
\par 24 Salutate tutti i vostri conduttori e tutti i santi. Quei d'Italia vi salutano.
\par 25 La grazia sia con tutti voi. Amen.


\end{document}