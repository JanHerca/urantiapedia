\begin{document}

\title{Esther}


\chapter{1}

\par 1 Al tempo d'Assuero, di quell'Assuero che regnava dall'India sino all'Etiopia sopra centoventisette province,
\par 2 in quel tempo, dico, il re Assuero, che sedeva sul trono del suo regno a Susa, la residenza reale,
\par 3 l'anno terzo del suo regno, fece un convito a tutti i suoi principi e ai suoi servi; l'esercito di Persia e di Media, i nobili e i governatori delle province furono riuniti in sua presenza,
\par 4 ed egli mostrò le ricchezze e la gloria del suo regno e il fasto magnifico della sua grandezza per molti giorni, per centottanta giorni.
\par 5 Scorsi che furon questi giorni, il re fece un altro convito di sette giorni, nel cortile del giardino del palazzo reale, per tutto il popolo che si trovava a Susa, la residenza reale, dal più grande al più piccolo.
\par 6 Arazzi di cotone finissimo, bianchi e violacei, stavan sospesi con cordoni di bisso e di scarlatto a degli anelli d'argento e a delle colonne di marmo. V'eran dei divani d'oro e d'argento sopra un pavimento di porfido, di marmo bianco, di madreperla e di pietre nere.
\par 7 Si porgeva da bere in vasi d'oro di forme svariate, e il vino reale era abbondante, grazie alla liberalità del re.
\par 8 E l'ordine era dato di non forzare alcuno a bere, poiché il re avea prescritto a tutti i grandi della sua casa che lasciassero fare a ciascuno secondo la propria volontà.
\par 9 La regina Vashti fece anch'ella un convito alle donne nella casa reale del re Assuero.
\par 10 Il settimo giorno, il re, che aveva il cuore reso allegro dal vino, ordinò a Mehuman, a Biztha, a Harbona, a Bigtha, ad Abagtha, a Zethar ed a Carcas, i sette eunuchi che servivano in presenza del re Assuero,
\par 11 che conducessero davanti a lui la regina Vashti con la corona reale, per far vedere ai popoli ed ai grandi la sua bellezza; poich'essa era bella d'aspetto.
\par 12 Ma la regina Vashti rifiutò di venire secondo l'ordine che il re le avea dato per mezzo degli eunuchi; e il re ne fu irritatissimo, e l'ira divampò dentro di lui.
\par 13 Allora il re interrogò i savi che aveano la conoscenza de' tempi. - Poiché gli affari del re si trattavano così in presenza di tutti quelli che conoscevano la legge e il diritto;
\par 14 e i più vicini a lui erano Carscena, Scethar, Admatha, Tarscish, Meres, Marsena e Memucan, sette principi di Persia e di Media che vedevano la faccia del re e occupavano i primi posti nel regno. -
\par 15 'Secondo la legge', disse, 'che si dev'egli fare alla regina Vashti per non aver ella eseguito l'ordine datole dal re Assuero per mezzo degli eunuchi?'
\par 16 Memucan rispose in presenza del re e dei principi: 'La regina Vashti ha mancato non solo verso il re, ma anche verso tutti i principi e tutti i popoli che sono in tutte le province del re Assuero.
\par 17 Poiché quello che la regina ha fatto si saprà da tutte le donne, e le indurrà a disprezzare i loro propri mariti; giacché esse diranno: - Il re Assuero aveva ordinato che si conducesse in sua presenza la regina Vashti, ed ella non v'è andata. -
\par 18 Da ora innanzi le principesse di Persia e di Media che avranno udito il fatto della regina ne parleranno a tutti i principi del re, e ne nascerà un gran disprezzo e molto sdegno.
\par 19 Se così piaccia al re, venga da lui emanato un editto reale, e sia iscritto fra le leggi di Persia e di Media talché sia irrevocabile, per il quale Vashti non possa più comparire in presenza del re Assuero, e il re conferisca la dignità reale ad una sua compagna migliore di lei.
\par 20 E quando l'editto che il re avrà emanato sarà conosciuto nell'intero suo regno ch'è vasto, tutte le donne renderanno onore ai loro mariti, dal più grande al più piccolo'.
\par 21 La cosa piacque al re ed ai principi, e il re fece come avea detto Memucan;
\par 22 e mandò lettere a tutte le province del regno, a ogni provincia secondo il suo modo di scrivere e ad ogni popolo secondo la sua lingua; per esse lettere ogni uomo doveva esser padrone in casa propria, e parlare la lingua del suo popolo.

\chapter{2}

\par 1 Dopo queste cose, quando l'ira del re fu calmata, egli si ricordò di Vashti, di ciò ch'ella avea fatto, e di quanto era stato deciso a suo riguardo.
\par 2 E quelli che stavano al servizio del re dissero: 'Si cerchino per il re delle fanciulle vergini e belle d'aspetto;
\par 3 stabilisca il re in tutte le province del suo regno de' commissari, i quali radunino tutte le fanciulle vergini e belle alla residenza reale di Susa, nella casa delle donne, sotto la sorveglianza di Hegai, eunuco del re, guardiano delle donne, che darà loro i cosmetici di cui abbisognano;
\par 4 e la fanciulla che piacerà al re diventi regina invece di Vashti'. La cosa piacque al re, e così si fece.
\par 5 Or nella residenza reale di Susa v'era un Giudeo per nome Mardocheo, figliuolo di Jair, figliuolo di Scimei, figliuolo di Kis, un Beniaminita,
\par 6 ch'era stato menato via da Gerusalemme fra gli schiavi trasportati in cattività con Jeconia, re di Giuda, da Nebucadnetsar, re di Babilonia.
\par 7 Egli aveva allevata la figliuola di suo zio, Hadassa, che è Ester, perch'essa non avea né padre né madre; la fanciulla era formosa e di bell'aspetto; e alla morte del padre e della madre, Mardocheo l'aveva adottata per figliuola.
\par 8 E come l'ordine del re e il suo editto furon divulgati, e un gran numero di fanciulle furon radunate nella residenza reale di Susa sotto la sorveglianza di Hegai, Ester fu menata anch'essa nella casa del re, sotto la sorveglianza di Hegai, guardiano delle donne.
\par 9 La fanciulla piacque a Hegai, ed entrò nelle buone grazie di lui; ei s'affrettò a fornirle i cosmetici di cui ell'avea bisogno e i suoi alimenti, le diede sette donzelle scelte nella casa del re, e assegnò a lei e alle sue donzelle l'appartamento migliore della casa delle donne.
\par 10 Ester non avea detto nulla né del suo popolo né del suo parentado, perché Mardocheo le avea proibito di parlarne.
\par 11 E Mardocheo tutti i giorni passeggiava davanti al cortile della casa delle donne per sapere se Ester stava bene e che cosa si farebbe di lei.
\par 12 Or quando veniva la volta per una fanciulla d'andare dal re Assuero alla fine dei dodici mesi prescritti alle donne per i loro preparativi - perché tanto durava il tempo dei loro preparativi: sei mesi per profumarsi con olio di mirra e sei mesi con aromi e altri cosmetici usati dalle donne, - la fanciulla andava dal re,
\par 13 e le si permetteva di portar seco, dalla casa delle donne alla casa del re, tutto quello che chiedeva.
\par 14 V'andava la sera, e la mattina dipoi passava nella seconda casa delle donne, sotto la sorveglianza di Shaashgaz, eunuco del re, guardiano delle concubine. Ella non tornava più dal re, a meno che il re la desiderasse ed ella fosse chiamata nominatamente.
\par 15 Quando venne la volta per Ester - la figliuola d'Abihail, zio di Mardocheo che l'aveva adottata per figliuola - d'andare dal re, ella non domandò altro fuori di quello che le fu indicato da Hegai, eunuco del re, guardiano delle donne. Ed Ester si guadagnava il favore di tutti quelli che la vedevano.
\par 16 Ester fu dunque condotta dal re Assuero, nella casa reale, il decimo mese, ch'è il mese di Tebeth, il settimo anno del regno di lui.
\par 17 E il re amò Ester più di tutte le altre donne, ed ella trovò grazia e favore agli occhi di lui più di tutte le altre fanciulle. Ei le pose in testa la corona reale e la fece regina in luogo di Vashti.
\par 18 E il re fece un gran convito a tutti i suoi principi ed ai suoi servi, che fu il convito d'Ester; concedette sgravi alle province, e fece doni con munificenza di re.
\par 19 Or la seconda volta che si radunavano delle fanciulle, Mardocheo stava seduto alla porta del re.
\par 20 Ester, secondo l'ordine che Mardocheo le avea dato, non avea detto nulla né del suo parentado né del suo popolo; perché ella faceva quello che Mardocheo le diceva, come quand'era sotto la tutela di lui.
\par 21 In que' giorni, come Mardocheo stava seduto alla porta del re, Bigthan e Teresh, due eunuchi del re di fra le guardie della soglia, irritatisi contro il re Assuero, cercarono d'attentargli alla vita.
\par 22 Mardocheo, avuto sentore della cosa, ne informò la regina Ester, ed Ester ne parlò al re in nome di Mardocheo.
\par 23 Investigato e verificato il fatto, i due eunuchi furono appiccati a un legno; e la cosa fu registrata nel libro delle Cronache, in presenza del re.

\chapter{3}

\par 1 Dopo queste cose, il re Assuero promosse Haman, figliuolo di Hammedatha, l'Agaghita, alla più alta dignità, e pose il suo seggio al disopra di quelli di tutti i principi ch'eran con lui.
\par 2 E tutti i servi del re che stavano alla porta del re s'inchinavano e si prostravano davanti a Haman, perché così aveva ordinato il re a suo riguardo. Ma Mardocheo non s'inchinava né si prostrava.
\par 3 E i servi del re che stavano alla porta del re dissero a Mardocheo: 'Perché trasgredisci l'ordine del re?'
\par 4 Or com'essi glielo ripetevano tutti i giorni, ed egli non dava loro ascolto, quelli riferirono la cosa a Haman, per vedere se Mardocheo persisterebbe nel suo dire; perch'egli avea lor detto ch'era Giudeo.
\par 5 Haman vide che Mardocheo non s'inchinava né si prostrava davanti a lui, e ne fu ripieno d'ira;
\par 6 ma sdegnò di metter le mani addosso a Mardocheo soltanto, giacché gli avean detto a qual popolo Mardocheo apparteneva; e cercò di distruggere il popolo di Mardocheo, tutti i Giudei che si trovavano in tutto il regno d'Assuero.
\par 7 Il primo mese ch'è il mese di Nisan, il dodicesimo anno del re Assuero, si tirò il Pur, vale a dire si tirò a sorte, in presenza di Haman, un giorno dopo l'altro e un mese dopo l'altro, finché sortì designato il dodicesimo mese, ch'è il mese di Adar.
\par 8 E Haman disse al re Assuero: 'V'è un popolo appartato e disperso fra i popoli di tutte le province del tuo regno, le cui leggi sono diverse da quelle d'ogni altro popolo, e che non osserva le leggi del re; non conviene quindi che il re lo tolleri.
\par 9 Se così piace al re, si scriva ch'esso sia distrutto; e io pagherò diecimila talenti d'argento in mano di quelli che fanno gli affari del re, perché sian portati nel tesoro reale'.
\par 10 Allora il re si tolse l'anello di mano, e lo diede a Haman l'Agaghita figliuolo di Hammedatha, e nemico de' Giudei.
\par 11 E il re disse a Haman: 'Il danaro t'è dato, e il popolo pure; fagli quel che ti pare'.
\par 12 Il tredicesimo giorno del primo mese furon chiamati i segretari del re, e fu scritto, seguendo in tutto gli ordini di Haman, ai satrapi del re, ai governatori d'ogni provincia e ai capi d'ogni popolo, a ogni provincia secondo il suo modo di scrivere, e ad ogni popolo nella sua lingua. Lo scritto fu redatto in nome del re Assuero e sigillato col sigillo reale.
\par 13 E furon mandate delle lettere, a mezzo di corrieri, in tutte le province del re perché si distruggessero, si uccidessero, si sterminassero tutti i Giudei, giovani e vecchi, bambini e donne, in un medesimo giorno, il tredici del dodicesimo mese, ch'è il mese d'Adar, e si abbandonassero al saccheggio i loro beni.
\par 14 Queste lettere contenevano una copia dell'editto che doveva esser pubblicato in ogni provincia, e invitavano tutti i popoli a tenersi pronti per quel giorno.
\par 15 I corrieri partirono in tutta fretta per ordine del re, e il decreto fu promulgato nella residenza reale di Susa; e mentre il re e Haman se ne stavano a sedere bevendo, la città di Susa era costernata.

\chapter{4}

\par 1 Or quando Mardocheo seppe tutto quello ch'era stato fatto, si stracciò le vesti, si coprì d'un sacco, si cosparse di cenere, e uscì fuori in mezzo alla città, mandando alte ed amare grida;
\par 2 e venne fin davanti alla porta del re, poiché a nessuno che fosse coperto di sacco era permesso di passare per la porta del re.
\par 3 In ogni provincia, dovunque giungevano l'ordine del re e il suo decreto, ci fu gran desolazione fra i Giudei: digiunavano, piangevano, si lamentavano, e a molti serviron di letto il sacco e la cenere.
\par 4 Le donzelle d'Ester e i suoi eunuchi vennero a riferirle la cosa; e la regina ne fu fortemente angosciata; e mandò delle vesti a Mardocheo, perché se le mettesse e si levasse di dosso il sacco; ma egli non le accettò.
\par 5 Allora Ester chiamò Hathac, uno degli eunuchi che il re avea messo al servizio di lei, e gli ordinò d'andare da Mardocheo per domandargli che cosa questo significasse, e perché agisse così.
\par 6 Hathac dunque si recò da Mardocheo sulla piazza della città, di faccia alla porta del re.
\par 7 E Mardocheo gli narrò tutto quello che gli era avvenuto, e gl'indicò la somma di danaro che Haman avea promesso di versare al tesoro reale per far distruggere i Giudei;
\par 8 e gli diede pure una copia del testo del decreto ch'era stato promulgato a Susa per il loro sterminio, affinché lo mostrasse a Ester, la informasse di tutto, e le ordinasse di presentarsi al re per domandargli grazia e per intercedere a pro del suo popolo.
\par 9 E Hathac tornò da Ester, e le riferì le parole di Mardocheo.
\par 10 Allora Ester ordinò a Hathac d'andare a dire a Mardocheo:
\par 11 'Tutti i servi del re e il popolo delle sue province sanno che se qualcuno, uomo o donna che sia, entra dal re nel cortile interno, senza essere stato chiamato, per una legge ch'è la stessa per tutti, ei dev'esser messo a morte, a meno che il re non stenda verso di lui il suo scettro d'oro; nel qual caso, colui ha salva la vita. E io son già trenta giorni che non sono stata chiamata per andare dal re'.
\par 12 Le parole di Ester furon riferite a Mardocheo;
\par 13 e Mardocheo fece dare questa risposta a Ester: 'Non ti mettere in mente che tu sola scamperai fra tutti i Giudei perché sei nella casa del re.
\par 14 Poiché se oggi tu ti taci, soccorso e liberazione sorgeranno per i Giudei da qualche altra parte; ma tu e la casa di tuo padre perirete; e chi sa se non sei pervenuta ad esser regina appunto per un tempo come questo?'
\par 15 Allora Ester ordinò che si rispondesse a Mardocheo:
\par 16 'Va', raduna tutti i Giudei che si trovano a Susa, e digiunate per me; state senza mangiare e senza bere per tre giorni, notte e giorno. Anch'io con le mie donzelle digiunerò nello stesso modo; e dopo entrerò dal re, quantunque ciò sia contro la legge; e, s'io debbo perire ch'io perisca!'
\par 17 Mardocheo se ne andò, e fece tutto quello che Ester gli aveva ordinato.

\chapter{5}

\par 1 Il terzo giorno, Ester si mise la veste reale, e si presentò nel cortile interno della casa del re, di faccia all'appartamento del re. Il re era assiso sul trono reale nella casa reale, di faccia alla porta della casa.
\par 2 E come il re ebbe veduta la regina Ester in piedi nel cortile, ella si guadagnò la sua grazia; e il re stese verso Ester lo scettro d'oro che teneva in mano; ed Ester s'appressò, e toccò la punta dello scettro.
\par 3 Allora il re le disse: 'Che hai, regina Ester? che domandi? Quand'anche tu chiedessi la metà del regno, ti sarà data'.
\par 4 Ester rispose: 'Se così piace al re, venga oggi il re con Haman al convito che gli ho preparato'.
\par 5 E il re disse: 'Fate venir subito Haman, per fare ciò che Ester ha detto'. Così il re e Haman vennero al convito che Ester avea preparato.
\par 6 E il re disse ad Ester, mentre si beveva il vino: 'Qual è la tua richiesta? Ti sarà concessa. Che desideri? Fosse anche la metà del regno, l'avrai'.
\par 7 Ester rispose: 'Ecco la mia richiesta, e quel che desidero:
\par 8 se ho trovato grazia agli occhi del re, e se piace al re di concedermi quello che chiedo e di soddisfare il mio desiderio, venga il re con Haman al convito ch'io preparerò loro, e domani farò come il re ha detto'.
\par 9 E Haman uscì, quel giorno, tutto allegro e col cuor contento; ma quando vide, alla porta del re, Mardocheo che non s'alzava né si moveva per lui, fu pieno d'ira contro Mardocheo.
\par 10 Nondimeno Haman si contenne, se ne andò a casa, e mandò a chiamare i suoi amici e Zeresh, sua moglie.
\par 11 E Haman parlò loro della magnificenza delle sue ricchezze, del gran numero de' suoi figliuoli, di tutto quello che il re aveva fatto per aggrandirlo, e del come l'aveva innalzato al disopra dei capi e dei servi del re.
\par 12 E aggiunse: 'Anche la regina Ester non ha fatto venire col re altri che me al convito che ha dato; e anche per domani sono invitato da lei col re.
\par 13 Ma tutto questo non mi soddisfa finché vedrò quel Giudeo di Mardocheo sedere alla porta del re'.
\par 14 Allora Zeresh sua moglie, e tutti i suoi amici gli dissero: 'Si prepari una forca alta cinquanta cubiti; e domattina di' al re che vi s'appicchi Mardocheo; poi vattene allegro al convito col re'. E la cosa piacque a Haman, che fece preparare la forca.

\chapter{6}

\par 1 Quella notte il re, non potendo prender sonno, ordinò che gli si portasse il libro delle Memorie, le Cronache; e ne fu fatta la lettura in presenza del re.
\par 2 Vi si trovò scritto che Mardocheo avea denunziato Bigthan e Teresh, i due eunuchi del re di fra i guardiani della soglia, i quali avean cercato d'attentare alla vita del re Assuero.
\par 3 Allora il re chiese: 'Qual onore e qual distinzione s'è dato a Mardocheo per questo?' Quelli che servivano il re risposero: 'Non s'è fatto nulla per lui'.
\par 4 E il re disse: 'Chi è nel cortile?' - Or Haman era venuto nel cortile esterno della casa del re, per dire al re di fare appiccare Mardocheo alla forca ch'egli avea preparata per lui. -
\par 5 I servi del re gli risposero: - 'Ecco, c'è Haman nel cortile'. E il re: 'Fatelo entrare'.
\par 6 Haman entrò, e il re gli disse: 'Che bisogna fare a un uomo che il re voglia onorare?' Haman disse in cuor suo: 'Chi altri vorrebbe il re onorare, se non me?'
\par 7 E Haman rispose al re: 'All'uomo che il re voglia onorare?
\par 8 Si prenda la veste reale che il re suol portare, e il cavallo che il re suol montare, e sulla cui testa è posta una corona reale;
\par 9 si consegni la veste e il cavallo a uno dei principi più nobili del re; si rivesta di quella veste l'uomo che il re vuole onorare, lo si faccia percorrere a cavallo le vie della città, e si gridi davanti a lui: - Così si fa all'uomo che il re vuole onorare!' -
\par 10 Allora il re disse a Haman: 'Fa' presto, e prendi la veste e il cavallo, come hai detto, e fa' a quel modo a Mardocheo, a quel Giudeo che siede alla porta del re; e non tralasciar nulla di quello che hai detto'.
\par 11 E Haman prese la veste e il cavallo, rivestì della veste Mardocheo, lo fece percorrere a cavallo le vie della città, e gridava davanti a lui: 'Così si fa all'uomo che il re vuole onorare!'
\par 12 Poi Mardocheo tornò alla porta del re, ma Haman s'affrettò d'andare a casa sua, tutto addolorato, e col capo coperto.
\par 13 E Haman raccontò a Zeresh sua moglie e a tutti i suoi amici tutto quello che gli era accaduto. E i suoi savi e Zeresh sua moglie gli dissero: 'Se Mardocheo davanti al quale tu hai cominciato a cadere è della stirpe de' Giudei, tu non potrai nulla contro di lui e cadrai completamente davanti ad esso'.
\par 14 Mentr'essi parlavano ancora con lui, giunsero gli eunuchi del re, i quali s'affrettarono a condurre Haman al convito che Ester aveva preparato.

\chapter{7}

\par 1 Il re e Haman andarono dunque al convito con la regina Ester.
\par 2 E il re anche questo secondo giorno disse a Ester, mentre si beveva il vino: 'Qual è la tua richiesta, o regina Ester? Ti sarà concessa. Che desideri? Fosse anche la metà del regno, l'avrai'.
\par 3 Allora la regina Ester rispose dicendo: 'Se ho trovato grazia agli occhi tuoi, o re, e se così piace al re, la mia richiesta è che mi sia donata la vita; e il mio desiderio, che mi sia donato il mio popolo.
\par 4 Perché io e il mio popolo siamo stati venduti per esser distrutti, uccisi, sterminati. Ora se fossimo stati venduti per diventare schiavi e schiave, mi sarei taciuta; ma il nostro avversario non potrebbe riparare al danno fatto al re con la nostra morte'.
\par 5 Il re Assuero prese a dire alla regina Ester: 'Chi è, e dov'è colui che ha tanta presunzione da far questo?'
\par 6 Ester rispose: 'L'avversario, il nemico, è quel malvagio di Haman'. Allora Haman fu preso da terrore in presenza del re e della regina.
\par 7 E il re tutto adirato si alzò, e dal luogo del convito andò nel giardino del palazzo; ma Haman rimase per chiedere la grazia della vita alla regina Ester, perché vedeva bene che nell'animo del re la sua rovina era decisa.
\par 8 Poi il re tornò dal giardino del palazzo nel luogo del convito; intanto Haman s'era gettato sul divano sul quale si trovava Ester; e il re esclamò: 'Vuol egli anche far violenza alla regina, davanti a me, in casa mia?' Non appena questa parola fu uscita dalla bocca del re, copersero a Haman la faccia;
\par 9 e Harbona, uno degli eunuchi, disse in presenza del re: 'Ecco, è perfino rizzata, in casa d'Haman, la forca alta cinquanta cubiti che Haman ha fatto preparare per Mardocheo, il qual avea parlato per il bene del re'. E il re disse: 'Appiccatevi lui!'
\par 10 Così Haman fu appiccato alla forca ch'egli avea preparata per Mardocheo. E l'ira del re si calmò.

\chapter{8}

\par 1 In quello stesso giorno il re Assuero donò alla regina Ester la casa di Haman, il nemico dei Giudei. E Mardocheo si presentò al re, al quale Ester avea dichiarato la parentela che l'univa a lui.
\par 2 E il re si cavò l'anello che avea fatto togliere a Haman, e lo diede a Mardocheo. Ed Ester diede a Mardocheo il governo della casa di Haman.
\par 3 Poi Ester parlò di nuovo in presenza del re, gli si gittò ai piedi, e lo supplicò con le lacrime agli occhi d'impedire gli effetti della malvagità di Haman, l'Agaghita, e delle trame ch'egli aveva ordite contro i Giudei.
\par 4 Allora il re stese lo scettro d'oro verso Ester; ed Ester s'alzò, rimase in piedi davanti al re,
\par 5 e disse: 'Se così piace al re, se io ho trovato grazia agli occhi suoi, se la cosa gli par giusta, e se io gli sono gradita, si scriva per revocare le lettere scritte da Haman, figliuolo di Hammedatha, l'Agaghita, col perfido disegno di far perire i Giudei che sono in tutte le province del re.
\par 6 Perché come potrei io reggere a vedere la calamità che colpirebbe il mio popolo? Come potrei reggere a vedere la distruzione della mia stirpe?'
\par 7 Allora il re Assuero disse alla regina Ester e a Mardocheo, il Giudeo: 'Ecco, io ho dato a Ester la casa di Haman, e questi è stato appeso alla forca, perché avea voluto metter la mano addosso ai Giudei.
\par 8 Scrivete dunque, a pro dei Giudei, come vi parrà meglio, nel nome del re, e suggellate coll'anello reale; perché ciò ch'è scritto in nome del re e sigillato con l'anello reale, è irrevocabile?'
\par 9 Senza perdere tempo, il ventitreesimo giorno del terzo mese, ch'è il mese di Sivan, furon chiamati i segretari del re, e fu scritto, seguendo in tutto l'ordine di Mardocheo, ai Giudei, ai satrapi, ai governatori e ai capi delle centoventisette province, dall'India all'Etiopia, a ogni provincia secondo il suo modo di scrivere, a ogni popolo nella sua lingua, e ai Giudei secondo il loro modo di scrivere e nella loro lingua.
\par 10 Fu dunque scritto in nome del re Assuero, si sigillaron le lettere con l'anello reale, e le si mandarono per mezzo di corrieri che cavalcavano veloci corsieri usati per il servizio del re, nati da stalloni reali.
\par 11 In esse il re permetteva ai Giudei, in qualunque città si trovassero, di radunarsi e di difendere la loro vita, di distruggere, uccidere, sterminare, non esclusi i bambini e le donne, tutta la gente armata, di qualunque popolo e di qualunque provincia si fosse, che li assalisse, e di abbandonare al saccheggio i suoi beni;
\par 12 e ciò, in un medesimo giorno, in tutte le province del re Assuero: il tredici del dodicesimo mese, ch'è il mese di Adar.
\par 13 Queste lettere contenevano una copia dell'editto che doveva esser bandito in ogni provincia e pubblicato fra tutti i popoli, perché i Giudei si tenessero pronti per quel giorno a vendicarsi dei loro nemici.
\par 14 Così i corrieri che montavano veloci corsieri usati per il servizio del re partirono tosto, in tutta fretta, per ordine del re; e il decreto fu promulgato nella residenza reale di Susa.
\par 15 Mardocheo uscì dalla presenza del re con una veste reale di porpora e di lino bianco, con una grande corona d'oro, e un manto di bisso e di scarlatto; la città di Susa mandava gridi di gioia, ed era in festa.
\par 16 I Giudei poi erano raggianti di gioia, d'allegrezza, di gloria.
\par 17 E in ogni provincia, in ogni città, dovunque giungevano l'ordine del re e il suo decreto, vi furon, tra i Giudei, gioia, allegrezza, conviti, e giorni lieti. E molti appartenenti ai popoli del paese si fecero Giudei, perché lo spavento dei Giudei s'era impossessato di loro.

\chapter{9}

\par 1 Il dodicesimo mese, ch'è il mese d'Adar, il tredicesimo giorno del mese, quando l'ordine del re e il suo decreto doveano esser mandati ad effetto, il giorno che i nemici de' Giudei speravano d'averli in loro potere, avvenne invece tutto il contrario; poiché furono i Giudei ch'ebbero in loro potere i loro nemici.
\par 2 I Giudei si radunarono nelle loro città, in tutte le province del re Assuero, per metter la mano su quelli che cercavano far ad essi del male; e nessuno poté resister loro, perché lo spavento de' Giudei s'era impossessato di tutti i popoli.
\par 3 E tutti i capi delle province, i satrapi, i governatori e quelli che facevano gli affari del re dettero man forte ai Giudei, perché lo spavento di Mardocheo s'era impossessato di loro.
\par 4 Poiché Mardocheo era grande nella casa del re, e la sua fama si spandeva per tutte le province, perché quest'uomo, Mardocheo, diventava sempre più grande.
\par 5 I Giudei dunque colpirono tutti i loro nemici, mettendoli a fil di spada, uccidendoli e sterminandoli; fecero dei loro nemici quello che vollero.
\par 6 Alla residenza reale di Susa i Giudei uccisero e sterminarono cinquecento uomini,
\par 7 e misero a morte Parshandatha, Dalfon, Aspatha, Poratha,
\par 8 Adalia, Aridatha,
\par 9 Parmashta, Arisai, Aridai, e Vaizatha, i dieci figliuoli di Haman,
\par 10 figliuolo di Hammedatha, il nemico de' Giudei, ma non si diedero al saccheggio.
\par 11 Quel giorno stesso il numero di quelli ch'erano stati uccisi alla residenza reale di Susa fu recato a conoscenza del re.
\par 12 E il re disse alla regina Ester: 'Alla residenza reale di Susa i Giudei hanno ucciso, hanno sterminato cinquecento uomini e dieci figliuoli di Haman; che avranno essi mai fatto nelle altre province del re? Or che chiedi tu ancora? Ti sarà dato. Che altro desideri? L'avrai'.
\par 13 Allora Ester disse: 'Se così piace al re, sia permesso ai Giudei che sono a Susa di fare anche domani quello ch'era stato decretato per oggi; e siano appesi alla forca i dieci figliuoli di Haman'.
\par 14 E il re ordinò che così fosse fatto. Il decreto fu promulgato a Susa, e i dieci figliuoli di Haman furono appiccati.
\par 15 E i Giudei ch'erano a Susa si radunarono ancora il quattordicesimo giorno del mese d'Adar e uccisero a Susa trecento uomini; ma non si diedero al saccheggio.
\par 16 Gli altri Giudei ch'erano nelle province del re si radunarono anch'essi, difesero la loro vita, ed ebbero requie dagli attacchi de' loro nemici; uccisero settantacinquemila di quelli che li aveano in odio, ma non si diedero al saccheggio.
\par 17 Questo avvenne il tredicesimo giorno del mese d'Adar; il quattordicesimo giorno si riposarono, e ne fecero un giorno di convito e di gioia.
\par 18 Ma i Giudei ch'erano a Susa si radunarono il tredicesimo e il quattordicesimo giorno di quel mese; il quindicesimo giorno si riposarono, e ne fecero un giorno di conviti e di gioia.
\par 19 Perciò i Giudei della campagna che abitano in città non murate fanno del quattordicesimo giorno del mese di Adar un giorno di gioia, di conviti e di festa, nel quale gli uni mandano dei regali agli altri.
\par 20 Mardocheo scrisse queste cose, e mandò delle lettere a tutti i Giudei ch'erano in tutte le province del re Assuero, vicini e lontani,
\par 21 ordinando loro che ogni anno celebrassero il quattordicesimo e il quindicesimo giorno del mese d'Adar,
\par 22 come i giorni ne' quali i Giudei ebbero requie dagli attacchi de' loro nemici, e il mese in cui il loro dolore era stato mutato in gioia, il loro lutto in festa, e facessero di questi giorni de' giorni di conviti e di gioia, nei quali gli uni manderebbero de' regali agli altri, e si farebbero dei doni ai bisognosi.
\par 23 I Giudei s'impegnarono a continuare quello che avean già cominciato a fare, e che Mardocheo avea loro scritto;
\par 24 poiché Haman, figliuolo di Hammedatha, l'Agaghita, il nemico di tutti i Giudei, aveva ordito una trama contro i Giudei per distruggerli, e avea gettato il Pur, vale a dire la sorte, per sgominarli e farli perire;
\par 25 ma quando Ester si fu presentata al cospetto del re, questi ordinò per iscritto che la scellerata macchinazione che Haman aveva ordita contro i Giudei fosse fatta ricadere sul capo di lui, e ch'egli e i suoi figliuoli fossero appesi alla forca.
\par 26 Perciò que' giorni furon detti Purim, dal termine Pur. Conforme quindi a tutto il contenuto di quella lettera, in seguito a tutto quello che avean visto a questo proposito e ch'era loro avvenuto,
\par 27 i Giudei stabilirono e presero per sé, per la loro progenie e per tutti quelli che si aggiungerebbero a loro, l'impegno inviolabile di celebrare ogni anno que' due giorni secondo il tenore di quello scritto e al tempo fissato.
\par 28 Que' giorni dovevano esser commemorati e celebrati di generazione in generazione, in ogni famiglia, in ogni provincia, in ogni città; e que' giorni di Purim non dovevano cessar mai d'esser celebrati fra i Giudei, e il loro ricordo non dovea mai cancellarsi fra i loro discendenti.
\par 29 La regina Ester, figliuola d'Abihail, e il Giudeo Mardocheo riscrissero con ogni autorità, per dar peso a questa loro seconda lettera relativa ai Purim.
\par 30 E si mandaron delle lettere a tutti i Giudei nelle centoventisette province del regno di Assuero: lettere contenenti parole di pace e di fedeltà,
\par 31 per fissar bene que' giorni di Purim nelle loro date precise, come li aveano ordinati il Giudeo Mardocheo e la regina Ester, e com'essi stessi li aveano stabiliti per sé e per i loro discendenti, in occasione del loro digiuno e del loro grido.
\par 32 Così l'ordine d'Ester fissò l'istituzione dei Purim, e ciò fu scritto in un libro.

\chapter{10}

\par 1 Il re Assuero impose un tributo al paese e alle isole del mare.
\par 2 Or quanto a tutti i fatti concernenti la potenza e il valore di Mardocheo e quanto alla completa descrizione della sua grandezza e del come il re lo ingrandì, sono cose scritte nel libro delle Cronache dei re di Media e di Persia.
\par 3 Poiché il Giudeo Mardocheo era il secondo dopo il re Assuero: grande fra i Giudei, e amato dalla moltitudine dei suoi fratelli; cercò il bene del suo popolo, e parlò per la pace di tutta la sua stirpe.


\end{document}