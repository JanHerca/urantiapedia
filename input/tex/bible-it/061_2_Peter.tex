\begin{document}

\title{2 Peter}


\chapter{1}

\par 1 Simon Pietro, servitore e apostolo di Gesù Cristo, a quelli che hanno ottenuto una fede preziosa quanto la nostra nella giustizia del nostro Dio e Salvatore Gesù Cristo:
\par 2 grazia e pace vi siano moltiplicate nella conoscenza di Dio e di Gesù nostro Signore.
\par 3 Poiché la sua potenza divina ci ha donate tutte le cose che appartengono alla vita e alla pietà mediante la conoscenza di Colui che ci ha chiamati mercé la propria gloria e virtù,
\par 4 per le quali Egli ci ha largito le sue preziose e grandissime promesse onde per loro mezzo voi foste fatti partecipi della natura divina dopo esser fuggiti dalla corruzione che è nel mondo per via della concupiscenza,
\par 5 voi, per questa stessa ragione, mettendo in ciò dal canto vostro ogni premura, aggiungete alla fede vostra la virtù; alla virtù la conoscenza;
\par 6 alla conoscenza la continenza; alla continenza la pazienza; alla pazienza la pietà; alla pietà l'amor fraterno;
\par 7 e all'amor fraterno la carità.
\par 8 Perché se queste cose si trovano e abbondano in voi, non vi renderanno né oziosi né sterili nella conoscenza del Signor nostro Gesù Cristo.
\par 9 Poiché colui nel quale queste cose non si trovano, è cieco, ha la vista corta avendo dimenticato il purgamento dei suoi vecchi peccati.
\par 10 Perciò, fratelli, vie più studiatevi di render sicura la vostra vocazione ed elezione; perché, facendo queste cose, non inciamperete giammai,
\par 11 poiché così vi sarà largamente provveduta l'entrata nel regno eterno del nostro Signore e Salvatore Gesù Cristo.
\par 12 Perciò avrò cura di ricordarvi del continuo queste cose, benché le conosciate, e siate stabiliti nella verità che vi è stata recata.
\par 13 E stimo cosa giusta finché io sono in questa tenda, di risvegliarvi ricordandovele,
\par 14 perché so che presto dovrò lasciare questa mia tenda, come il Signore nostro Gesù Cristo me lo ha dichiarato.
\par 15 Ma mi studierò di far sì che dopo la mia dipartenza abbiate sempre modo di ricordarvi di queste cose.
\par 16 Poiché non è coll'andar dietro a favole artificiosamente composte che vi abbiamo fatto conoscere la potenza e la venuta del nostro Signor Gesù Cristo, ma perché siamo stati testimoni oculari della sua maestà.
\par 17 Poiché egli ricevette da Dio Padre onore e gloria quando giunse a lui quella voce dalla magnifica gloria: Questo è il mio diletto Figliuolo, nel quale mi son compiaciuto.
\par 18 E noi stessi udimmo quella voce che veniva dal cielo, quand'eravamo con lui sul monte santo.
\par 19 Abbiamo pure la parola profetica, più ferma, alla quale fate bene di prestare attenzione, come a una lampada splendente in luogo oscuro, finché spunti il giorno e la stella mattutina sorga ne' vostri cuori;
\par 20 sapendo prima di tutto questo: che nessuna profezia della Scrittura procede da vedute particolari;
\par 21 poiché non è dalla volontà dell'uomo che venne mai alcuna profezia, ma degli uomini hanno parlato da parte di Dio, perché sospinti dallo Spirito Santo.

\chapter{2}

\par 1 Ma sorsero anche falsi profeti fra il popolo, come ci saranno anche fra voi falsi dottori che introdurranno di soppiatto eresie di perdizione, e, rinnegando il Signore che li ha riscattati, si trarranno addosso subita rovina.
\par 2 E molti seguiranno le loro lascivie; e a cagion loro la via della verità sarà diffamata.
\par 3 Nella loro cupidigia vi sfrutteranno con parole finte; il loro giudicio già da tempo è all'opera, e la loro ruina non sonnecchia.
\par 4 Perché se Dio non risparmiò gli angeli che aveano peccato, ma li inabissò, confinandoli in antri tenebrosi per esservi custoditi pel giudizio;
\par 5 e se non risparmiò il mondo antico ma salvò Noè predicator di giustizia, con sette altri, quando fece venire il diluvio sul mondo degli empi;
\par 6 e se, riducendo in cenere le città di Sodoma e Gomorra, le condannò alla distruzione perché servissero d'esempio a quelli che in avvenire vivrebbero empiamente;
\par 7 e se salvò il giusto Lot che era contristato dalla lasciva condotta degli scellerati
\par 8 (perché quel giusto, che abitava fra loro, per quanto vedeva e udiva si tormentava ogni giorno l'anima giusta a motivo delle loro inique opere),
\par 9 il Signore sa trarre i pii dalla tentazione e riserbare gli ingiusti ad esser puniti nel giorno del giudizio;
\par 10 e massimamente quelli che van dietro alla carne nelle immonde concupiscenze, e sprezzano l'autorità. Audaci, arroganti, non hanno orrore di dir male delle dignità;
\par 11 mentre gli angeli, benché maggiori di loro per forza e potenza, non portano contro ad esse, dinanzi al Signore, alcun giudizio maldicente.
\par 12 Ma costoro, come bruti senza ragione, nati alla vita animale per esser presi e distrutti, dicendo male di quel che ignorano, periranno per la loro propria corruzione, ricevendo il salario della loro iniquità.
\par 13 Essi trovano il loro piacere nel gozzovigliare in pieno giorno; son macchie e vergogne, godendo dei loro inganni mentre partecipano ai vostri conviti;
\par 14 hanno occhi pieni d'adulterio e che non possono smetter di peccare; adescano le anime instabili; hanno il cuore esercitato alla cupidigia; son figliuoli di maledizione.
\par 15 Lasciata la diritta strada, si sono smarriti, seguendo la via di Balaam, figliuolo di Beor, che amò il salario d'iniquità,
\par 16 ma fu ripreso per la sua prevaricazione: un'asina muta, parlando con voce umana, represse la follia del profeta.
\par 17 Costoro son fonti senz'acqua, e nuvole sospinte dal turbine; a loro è riserbata la caligine delle tenebre.
\par 18 Perché, con discorsi pomposi e vacui, adescano con le concupiscenze carnali e le lascivie quelli che si erano già un poco allontanati da coloro che vivono nell'errore,
\par 19 promettendo loro la libertà, mentre essi stessi sono schiavi della corruzione; giacché uno diventa schiavo di ciò che l'ha vinto.
\par 20 Poiché, se dopo esser fuggiti dalle contaminazioni del mondo mediante la conoscenza del Signore e Salvatore Gesù Cristo, si lascian di nuovo avviluppare in quelle e vincere, la loro condizione ultima diventa peggiore della prima.
\par 21 Perché meglio sarebbe stato per loro non aver conosciuta la via della giustizia, che, dopo averla conosciuta, voltar le spalle al santo comandamento ch'era loro stato dato.
\par 22 È avvenuto di loro quel che dice con verità il proverbio: Il cane è tornato al suo vomito, e: La troia lavata è tornata a voltolarsi nel fango.

\chapter{3}

\par 1 Diletti, questa è già la seconda epistola che vi scrivo; e in ambedue io tengo desta la vostra mente sincera facendo appello alla vostra memoria,
\par 2 onde vi ricordiate delle parole dette già dai santi profeti, e del comandamento del Signore e Salvatore, trasmessovi dai vostri apostoli;
\par 3 sapendo questo, prima di tutto: che negli ultimi giorni verranno degli schernitori coi loro scherni i quali si condurranno secondo le loro concupiscenze
\par 4 e diranno: Dov'è la promessa della sua venuta? perché dal giorno in cui i padri si sono addormentati, tutte le cose continuano nel medesimo stato come dal principio della creazione.
\par 5 Poiché costoro dimenticano questo volontariamente: che ab antico, per effetto della parola di Dio, esistettero de' cieli e una terra tratta dall'acqua e sussistente in mezzo all'acqua;
\par 6 per i quali mezzi il mondo d'allora, sommerso dall'acqua, perì;
\par 7 mentre i cieli d'adesso e la terra, per la medesima Parola son custoditi, essendo riservati al fuoco per il giorno del giudizio e della distruzione degli uomini empî.
\par 8 Ma voi, diletti, non dimenticate quest'unica cosa, che per il Signore, un giorno è come mille anni, e mille anni son come un giorno.
\par 9 Il Signore non ritarda l'adempimento della sua promessa, come alcuni reputano che faccia; ma egli è paziente verso voi, non volendo che alcuni periscano, ma che tutti giungano a ravvedersi.
\par 10 Ma il giorno del Signore verrà come un ladro; in esso i cieli passeranno stridendo, e gli elementi infiammati si dissolveranno, e la terra e le opere che sono in essa saranno arse.
\par 11 Poiché dunque tutte queste cose hanno da dissolversi, quali non dovete voi essere, per santità di condotta e per pietà,
\par 12 aspettando e affrettando la venuta del giorno di Dio, a cagion del quale i cieli infocati si dissolveranno e gli elementi infiammati si struggeranno?
\par 13 Ma, secondo la sua promessa, noi aspettiamo nuovi cieli e nuova terra, ne' quali abiti la giustizia.
\par 14 Perciò, diletti, aspettando queste cose, studiatevi d'esser trovati, agli occhi suoi, immacolati e irreprensibili nella pace;
\par 15 e ritenete che la pazienza del Signor nostro è per la vostra salvezza, come anche il nostro caro fratello Paolo ve l'ha scritto, secondo la sapienza che gli è stata data;
\par 16 e questo egli fa in tutte le sue epistole, parlando in esse di questi argomenti; nelle quali epistole sono alcune cose difficili a capire, che gli uomini ignoranti e instabili torcono, come anche le altre Scritture, a loro propria perdizione.
\par 17 Voi dunque, diletti, sapendo queste cose innanzi, state in guardia, che talora, trascinati anche voi dall'errore degli scellerati, non iscadiate dalla vostra fermezza;
\par 18 ma crescete nella grazia e nella conoscenza del nostro Signore e Salvatore Gesù Cristo. A lui sia la gloria, ora e in sempiterno. Amen.


\end{document}