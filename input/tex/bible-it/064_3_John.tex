\begin{document}

\title{3 John}


\chapter{1}

\par 1 L'anziano al diletto Gaio, che io amo nella verità.
\par 2 Diletto, io faccio voti che tu prosperi in ogni cosa e stii sano, come prospera l'anima tua.
\par 3 Perché mi sono grandemente rallegrato quando son venuti dei fratelli che hanno reso testimonianza della tua verità, del modo nel quale tu cammini in verità.
\par 4 Io non ho maggiore allegrezza di questa, d'udire che i miei figliuoli camminano nella verità.
\par 5 Diletto, tu operi fedelmente in quel che fai a pro dei fratelli che sono, per di più, forestieri.
\par 6 Essi hanno reso testimonianza del tuo amore, dinanzi alla chiesa; e farai bene a provvedere al loro viaggio in modo degno di Dio;
\par 7 perché sono partiti per amor del nome di Cristo, senza prendere alcun che dai pagani.
\par 8 Noi dunque dobbiamo accogliere tali uomini, per essere cooperatori con la verità.
\par 9 Ho scritto qualcosa alla chiesa; ma Diotrefe che cerca d'avere il primato fra loro, non ci riceve.
\par 10 Perciò, se vengo, io ricorderò le opere che fa, cianciando contro di noi con male parole; e non contento di questo, non solo non riceve egli stesso i fratelli, ma a quelli che vorrebbero riceverli impedisce di farlo, e li caccia fuor della chiesa.
\par 11 Diletto non imitare il male, ma il bene. Chi fa il bene è da Dio; chi fa il male non ha veduto Iddio.
\par 12 A Demetrio è resa testimonianza da tutti e dalla verità stessa; e anche noi ne testimoniamo; e tu sai che la nostra testimonianza è vera.
\par 13 Avevo molte cose da scriverti, ma non voglio scrivertele con inchiostro e penna.
\par 14 Ma spero vederti tosto, e ci parleremo a voce.
\par 15 La pace sia teco. Gli amici ti salutano. Saluta gli amici ad uno ad uno.


\end{document}