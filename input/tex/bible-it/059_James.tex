\begin{document}

\title{James}


\chapter{1}

\par 1 Giacomo, servitore di Dio e del Signor Gesù Cristo, alle dodici tribù che sono nella dispersione, salute.
\par 2 Fratelli miei, considerate come argomento di completa allegrezza le prove svariate in cui venite a trovarvi,
\par 3 sapendo che la prova della vostra fede produce costanza.
\par 4 E la costanza compia appieno l'opera sua in voi, onde siate perfetti e completi, di nulla mancanti.
\par 5 Che se alcuno di voi manca di sapienza, la chiegga a Dio che dona a tutti liberalmente senza rinfacciare, e gli sarà donata.
\par 6 Ma chiegga con fede, senza star punto in dubbio; perché chi dubita è simile a un'onda di mare, agitata dal vento e spinta qua e là.
\par 7 Non pensi già quel tale di ricever nulla dal Signore,
\par 8 essendo uomo d'animo doppio, instabile in tutte le sue vie.
\par 9 Or il fratello d'umil condizione si glorî della sua elevazione;
\par 10 e il ricco, della sua umiliazione, perché passerà come fior d'erba.
\par 11 Il sole si leva col suo calore ardente e fa seccare l'erba, e il fiore d'essa cade, e la bellezza della sua apparenza perisce; così anche il ricco appassirà nelle sue imprese.
\par 12 Beato l'uomo che sostiene la prova; perché, essendosi reso approvato, riceverà la corona della vita, che il Signore ha promessa a quelli che l'amano.
\par 13 Nessuno, quand'è tentato, dica: Io son tentato da Dio; perché Dio non può esser tentato dal male, né Egli stesso tenta alcuno;
\par 14 ma ognuno è tentato dalla propria concupiscenza che lo attrae e lo adesca.
\par 15 Poi la concupiscenza avendo concepito partorisce il peccato; e il peccato, quand'è compiuto, produce la morte.
\par 16 Non errate, fratelli miei diletti;
\par 17 ogni donazione buona e ogni dono perfetto vengon dall'alto, discendendo dal Padre degli astri luminosi presso il quale non c'è variazione né ombra prodotta da rivolgimento.
\par 18 Egli ci ha di sua volontà generati mediante la parola di verità, affinché siamo in certo modo le primizie delle sue creature.
\par 19 Questo lo sapete, fratelli miei diletti; ma sia ogni uomo pronto ad ascoltare, tardo al parlare, lento all'ira;
\par 20 perché l'ira dell'uomo non mette in opra la giustizia di Dio.
\par 21 Perciò, deposta ogni lordura e resto di malizia, ricevete con mansuetudine la Parola che è stata piantata in voi, e che può salvare le anime vostre.
\par 22 Ma siate facitori della Parola e non soltanto uditori, illudendo voi stessi.
\par 23 Perché, se uno è uditore della Parola e non facitore, è simile a un uomo che mira la sua natural faccia in uno specchio;
\par 24 e quando s'è mirato se ne va, e subito dimentica qual era.
\par 25 Ma chi riguarda bene addentro nella legge perfetta, che è la legge della libertà, e persevera, questi, non essendo un uditore dimentichevole ma facitore dell'opera, sarà beato nel suo operare.
\par 26 Se uno pensa d'esser religioso, e non tiene a freno la sua lingua ma seduce il cuor suo, la religione di quel tale è vana.
\par 27 La religione pura e immacolata dinanzi a Dio e Padre è questa: visitar gli orfani e le vedove nelle loro afflizioni, e conservarsi puri dal mondo.

\chapter{2}

\par 1 Fratelli miei, la vostra fede nel nostro Signor Gesù Cristo, il Signor della gloria, sia scevra da riguardi personali.
\par 2 Perché, se nella vostra raunanza entra un uomo con l'anello d'oro, vestito splendidamente, e v'entra pure un povero vestito malamente,
\par 3 e voi avete riguardo a quello che veste splendidamente e gli dite: Tu, siedi qui in un posto onorevole; e al povero dite: Tu, stattene là in piè, o siedi appiè del mio sgabello,
\par 4 non fate voi una differenza nella vostra mente, e non diventate giudici dai pensieri malvagi?
\par 5 Ascoltate, fratelli miei diletti: Iddio non ha egli scelto quei che sono poveri secondo il mondo perché siano ricchi in fede ed eredi del Regno che ha promesso a coloro che l'amano?
\par 6 Ma voi avete disprezzato il povero! Non son forse i ricchi quelli che vi opprimono e che vi traggono ai tribunali?
\par 7 Non sono essi quelli che bestemmiano il buon nome che è stato invocato su di voi?
\par 8 Certo, se adempite la legge reale, secondo che dice la Scrittura: Ama il tuo prossimo come te stesso, fate bene;
\par 9 ma se avete dei riguardi personali, voi commettete un peccato essendo dalla legge convinti quali trasgressori.
\par 10 Poiché chiunque avrà osservato tutta la legge, e avrà fallito in un sol punto, si rende colpevole su tutti i punti.
\par 11 Poiché Colui che ha detto: Non commettere adulterio, ha detto anche: Non uccidere. Ora, se tu non commetti adulterio ma uccidi, sei diventato trasgressore della legge.
\par 12 Parlate e operate come dovendo esser giudicati da una legge di libertà.
\par 13 Perché il giudicio è senza misericordia per colui che non ha usato misericordia: la misericordia trionfa del giudicio.
\par 14 Che giova, fratelli miei, se uno dice d'aver fede ma non ha opere? Può la fede salvarlo?
\par 15 Se un fratello o una sorella son nudi e mancanti del cibo quotidiano,
\par 16 e un di voi dice loro: Andatevene in pace, scaldatevi e satollatevi; ma non date loro le cose necessarie al corpo, che giova?
\par 17 Così è della fede; se non ha opere, è per se stessa morta.
\par 18 Anzi uno piuttosto dirà: Tu hai la fede, ed io ho le opere; mostrami la tua fede senza le tue opere, e io con le mie opere ti mostrerò la mia fede.
\par 19 Tu credi che v'è un sol Dio, e fai bene; anche i demonî lo credono e tremano.
\par 20 Ma vuoi tu, o uomo vano, conoscere che la fede senza le opere non ha valore?
\par 21 Abramo, nostro padre, non fu egli giustificato per le opere quando offrì il suo figliuolo Isacco sull'altare?
\par 22 Tu vedi che la fede operava insieme con le opere di lui, e che per le opere la sua fede fu resa compiuta;
\par 23 e così fu adempiuta la Scrittura che dice: E Abramo credette a Dio, e ciò gli fu messo in conto di giustizia; e fu chiamato amico di Dio.
\par 24 Voi vedete che l'uomo è giustificato per opere, e non per fede soltanto.
\par 25 Parimente, Raab, la meretrice, non fu anch'ella giustificata per le opere quando accolse i messi e li mandò via per un altro cammino?
\par 26 Infatti, come il corpo senza lo spirito è morto, così anche la fede senza le opere è morta.

\chapter{3}

\par 1 Fratelli miei, non siate molti a far da maestri, sapendo che ne riceveremo un più severo giudicio.
\par 2 Poiché tutti falliamo in molte cose. Se uno non falla nel parlare, esso è un uomo perfetto, capace di tenere a freno anche tutto il corpo.
\par 3 Se mettiamo il freno in bocca ai cavalli perché ci ubbidiscano, noi guidiamo anche tutto quanto il loro corpo.
\par 4 Ecco, anche le navi, benché siano così grandi e sian sospinte da fieri venti, son dirette da un piccolissimo timone, dovunque vuole l'impulso di chi le governa.
\par 5 Così anche la lingua è un piccol membro, e si vanta di gran cose. Vedete un piccol fuoco, che gran foresta incendia!
\par 6 Anche la lingua è un fuoco, è il mondo dell'iniquità. Posta com'è fra le nostre membra, contamina tutto il corpo e infiamma la ruota della vita, ed è infiammata dalla geenna.
\par 7 Ogni sorta di fiere e d'uccelli, di rettili e di animali marini si doma, ed è stata domata dalla razza umana;
\par 8 ma la lingua, nessun uomo la può domare; è un male senza posa, è piena di mortifero veleno.
\par 9 Con essa benediciamo il Signore e Padre; e con essa malediciamo gli uomini che son fatti a somiglianza di Dio.
\par 10 Dalla medesima bocca procede benedizione e maledizione.
\par 11 Fratelli miei, non dev'essere così. La fonte getta essa dalla medesima apertura il dolce e l'amaro?
\par 12 Può, fratelli miei, un fico fare ulive, o una vite fichi? Neppure può una fonte salata dare acqua dolce.
\par 13 Chi è savio e intelligente fra voi? Mostri con la buona condotta le sue opere in mansuetudine di sapienza.
\par 14 Ma se avete nel cuor vostro dell'invidia amara e uno spirito di contenzione, non vi gloriate e non mentite contro la verità.
\par 15 Questa non è la sapienza che scende dall'alto, anzi ella è terrena, carnale, diabolica.
\par 16 Poiché dove sono invidia e contenzione, quivi è disordine ed ogni mala azione.
\par 17 Ma la sapienza che è da alto, prima è pura; poi pacifica, mite, arrendevole, piena di misericordia e di buoni frutti, senza parzialità, senza ipocrisia.
\par 18 Or il frutto della giustizia si semina nella pace per quelli che s'adoprano alla pace.

\chapter{4}

\par 1 Donde vengon le guerre e le contese fra voi? Non è egli da questo: cioè dalle vostre voluttà che guerreggiano nelle vostre membra?
\par 2 Voi bramate e non avete; voi uccidete ed invidiate e non potete ottenere; voi contendete e guerreggiate; non avete, perché non domandate;
\par 3 domandate e non ricevete, perché domandate male per spendere ne' vostri piaceri.
\par 4 O gente adultera, non sapete voi che l'amicizia del mondo è inimicizia contro Dio? Chi dunque vuol essere amico del mondo si rende nemico di Dio.
\par 5 Ovvero pensate voi che la Scrittura dichiari invano che lo Spirito ch'Egli ha fatto abitare in noi ci brama fino alla gelosia?
\par 6 Ma Egli dà maggior grazia; perciò la Scrittura dice:
\par 7 Iddio resiste ai superbi e dà grazia agli umili. Sottomettetevi dunque a Dio; ma resistete al diavolo, ed egli fuggirà da voi.
\par 8 Appressatevi a Dio, ed Egli si appresserà a voi. Nettate le vostre mani, o peccatori; e purificate i vostri cuori, o doppi d'animo!
\par 9 Siate afflitti e fate cordoglio e piangete! Sia il vostro riso convertito in lutto, e la vostra allegrezza in mestizia!
\par 10 Umiliatevi nel cospetto del Signore, ed Egli vi innalzerà.
\par 11 Non parlate gli uni contro gli altri, fratelli. Chi parla contro un fratello, o giudica il suo fratello, parla contro la legge e giudica la legge. Ora, se tu giudichi la legge, non sei un osservatore della legge, ma un giudice.
\par 12 Uno soltanto è il legislatore e il giudice, Colui che può salvare e perdere; ma tu chi sei, che giudichi il tuo prossimo?
\par 13 Ed ora a voi che dite: Oggi o domani andremo nella tal città e vi staremo un anno, e trafficheremo, e guadagneremo;
\par 14 mentre non sapete quel che avverrà domani! Che cos'è la vita vostra? Poiché siete un vapore che appare per un po' di tempo e poi svanisce.
\par 15 Invece di dire: Se piace al Signore, saremo in vita e faremo questo o quest'altro.
\par 16 Ma ora vi vantate con le vostre millanterie. Ogni cotal vanto è cattivo.
\par 17 Colui dunque che sa fare il bene, e non lo fa, commette peccato.

\chapter{5}

\par 1 A voi ora, o ricchi; piangete e urlate per calamità che stanno per venirvi addosso!
\par 2 Le vostre ricchezze son marcite, e le vostre vesti son rôse dalle tignuole.
\par 3 Il vostro oro e il vostro argento sono arrugginiti, e la loro ruggine sarà una testimonianza contro a voi, e divorerà le vostre carni a guisa di fuoco. Avete accumulato tesori negli ultimi giorni.
\par 4 Ecco, il salario dei lavoratori che han mietuto i vostri campi, e del quale li avete frodati, grida; e le grida di quelli che han mietuto sono giunte alle orecchie del Signor degli eserciti.
\par 5 Voi siete vissuti sulla terra nelle delizie e vi siete dati ai piaceri; avete pasciuto i vostri cuori in giorno di strage.
\par 6 Avete condannato, avete ucciso il giusto; egli non vi resiste.
\par 7 Siate dunque pazienti, fratelli, fino alla venuta del Signore. Ecco, l'agricoltore aspetta il prezioso frutto della terra pazientando, finché esso abbia ricevuto la pioggia della prima e dell'ultima stagione.
\par 8 Siate anche voi pazienti; rinfrancate i vostri cuori, perché la venuta del Signore è vicina.
\par 9 Fratelli, non mormorate gli uni contro gli altri, onde non siate giudicati; ecco, il Giudice è alla porta.
\par 10 Prendete, fratelli, per esempio di sofferenza e di pazienza i profeti che han parlato nel nome del Signore.
\par 11 Ecco, noi chiamiam beati quelli che hanno sofferto con costanza. Avete udito parlare della costanza di Giobbe, e avete veduto la fine riserbatagli dal Signore, perché il Signore è pieno di compassione e misericordioso.
\par 12 Ma, innanzi tutto, fratelli miei, non giurate né per il cielo, né per la terra, né con altro giuramento; ma sia il vostro sì, sì, e il vostro no, no, affinché non cadiate sotto giudicio.
\par 13 C'è fra voi qualcuno che soffre? Preghi. C'è qualcuno d'animo lieto? Salmeggi.
\par 14 C'è qualcuno fra voi infermo? Chiami gli anziani della chiesa, e preghino essi su lui, ungendolo d'olio nel nome del Signore;
\par 15 e la preghiera della fede salverà il malato, e il Signore lo ristabilirà; e s'egli ha commesso dei peccati, gli saranno rimessi.
\par 16 Confessate dunque i falli gli uni agli altri, e pregate gli uni per gli altri onde siate guariti; molto può la supplicazione del giusto, fatta con efficacia.
\par 17 Elia era un uomo sottoposto alle stesse passioni che noi, e pregò ardentemente che non piovesse, e non piovve sulla terra per tre anni e sei mesi.
\par 18 Pregò di nuovo, e il cielo diede la pioggia, e la terra produsse il suo frutto.
\par 19 Fratelli miei, se qualcuno fra voi si svia dalla verità e uno lo converte,
\par 20 sappia colui che chi converte un peccatore dall'error della sua via salverà l'anima di lui dalla morte e coprirà moltitudine di peccati.


\end{document}