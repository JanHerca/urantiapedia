\begin{document}

\title{II Corinzi}


\chapter{1}

\par 1 Paolo, apostolo di Cristo Gesù per la volontà di Dio, e il fratello Timoteo, alla chiesa di Dio che è in Corinto, con tutti i santi che sono in tutta l'Acaia,
\par 2 grazia a voi e pace da Dio nostro Padre e dal Signor Gesù Cristo.
\par 3 Benedetto sia Iddio, il Padre del nostro Signore Gesù Cristo, il Padre delle misericordie e l'Iddio d'ogni consolazione,
\par 4 il quale ci consola in ogni nostra afflizione, affinché, mediante la consolazione onde noi stessi siam da Dio consolati, possiam consolare quelli che si trovano in qualunque afflizione.
\par 5 Perché, come abbondano in noi le sofferenze di Cristo, così, per mezzo di Cristo, abbonda anche la nostra consolazione.
\par 6 Talché se siamo afflitti, è per la vostra consolazione e salvezza; e se siamo consolati, è per la vostra consolazione, la quale opera efficacemente nel farvi capaci di sopportare le stesse sofferenze che anche noi patiamo.
\par 7 E la nostra speranza di voi è ferma, sapendo che come siete partecipi delle sofferenze siete anche partecipi della consolazione.
\par 8 Poiché, fratelli, non vogliamo che ignoriate, circa l'afflizione che ci colse in Asia, che siamo stati oltremodo aggravati, al di là delle nostre forze, tanto che stavamo in gran dubbio anche della vita.
\par 9 Anzi, avevamo già noi stessi pronunciata la nostra sentenza di morte, affinché non ci confidassimo in noi medesimi, ma in Dio che risuscita i morti,
\par 10 il quale ci ha liberati e ci libererà da un così gran pericolo di morte, e nel quale abbiamo la speranza che ci libererà ancora;
\par 11 aiutandoci anche voi con le vostre supplicazioni, affinché del favore ottenutoci per mezzo di tante persone, grazie siano rese per noi da molti.
\par 12 Questo, infatti, è il nostro vanto: la testimonianza della nostra coscienza, che ci siam condotti nel mondo, e più che mai verso voi, con santità e sincerità di Dio, non con sapienza carnale, ma con la grazia di Dio.
\par 13 Poiché noi non vi scriviamo altro se non quel che leggete o anche riconoscete;
\par 14 e spero che sino alla fine riconoscerete, come in parte avete già riconosciuto, che noi siamo il vostro vanto, come anche voi sarete il nostro nel giorno del nostro Signore, Gesù.
\par 15 E in questa fiducia, per procurarvi un duplice beneficio, io volevo venire prima da voi,
\par 16 e, passando da voi, volevo andare in Macedonia; e poi dalla Macedonia venir di nuovo a voi, e da voi esser fatto proseguire per la Giudea.
\par 17 Prendendo dunque questa decisione ho io agito con leggerezza? Ovvero, le cose che delibero, le delibero io secondo la carne, talché un momento io dica 'Sì, sì' e l'altro 'No, no?'
\par 18 Or com'è vero che Dio è fedele, la parola che vi abbiam rivolta non è 'sì' e 'no'.
\par 19 Perché il Figliuol di Dio, Cristo Gesù, che è stato da noi predicato fra voi, cioè da me, da Silvano e da Timoteo, non è stato 'sì' e 'no'; ma è 'sì' in lui.
\par 20 Poiché quante sono le promesse di Dio, tutte hanno in lui il loro 'sì'; perciò pure per mezzo di lui si pronunzia l'Amen alla gloria di Dio, in grazia del nostro ministerio.
\par 21 Or Colui che con voi ci rende fermi in Cristo e che ci ha unti, è Dio,
\par 22 il quale ci ha pur segnati col proprio sigillo, e ci ha data la caparra dello Spirito nei nostri cuori.
\par 23 Or io chiamo Iddio a testimone sull'anima mia ch'egli è per risparmiarvi ch'io non son più venuto a Corinto.
\par 24 Non già che signoreggiamo sulla vostra fede, ma siamo aiutatori della vostra allegrezza; poiché nella fede voi state saldi.

\chapter{2}

\par 1 Io avevo dunque meco stesso determinato di non venire a voi per rattristarvi una seconda volta.
\par 2 Perché, se io vi contristo, chi sarà dunque colui che mi rallegrerà, se non colui che sarà stato da me contristato?
\par 3 E vi ho scritto a quel modo onde, al mio arrivo, io non abbia tristezza da coloro dai quali dovrei aver allegrezza; avendo di voi tutti fiducia che la mia allegrezza è l'allegrezza di tutti voi.
\par 4 Poiché in grande afflizione ed in angoscia di cuore vi scrissi con molte lagrime, non già perché foste contristati, ma perché conosceste l'amore che nutro abbondantissimo per voi.
\par 5 Or se qualcuno ha cagionato tristezza, egli non ha contristato me, ma, in parte, per non esagerare, voi tutti.
\par 6 Basta a quel tale la riprensione inflittagli dalla maggioranza;
\par 7 onde ora, al contrario, dovreste piuttosto perdonarlo e confortarlo, che talora non abbia a rimaner sommerso da soverchia tristezza.
\par 8 Perciò vi prego di confermargli l'amor vostro;
\par 9 poiché anche per questo vi ho scritto: per conoscere alla prova se siete ubbidienti in ogni cosa.
\par 10 Or a chi voi perdonate qualcosa, perdono anch'io; poiché anch'io quel che ho perdonato, se ho perdonato qualcosa, l'ho fatto per amor vostro, nel cospetto di Cristo,
\par 11 affinché non siamo soverchiati da Satana, giacché non ignoriamo le sue macchinazioni.
\par 12 Or essendo venuto a Troas per l'Evangelo di Cristo ed essendomi aperta una porta nel Signore,
\par 13 non ebbi requie nel mio spirito perché non vi trovai Tito, mio fratello; così, accomiatatomi da loro, partii per la Macedonia.
\par 14 Ma grazie siano rese a Dio che sempre ci conduce in trionfo in Cristo, e che per mezzo nostro spande da per tutto il profumo della sua conoscenza.
\par 15 Poiché noi siamo dinanzi a Dio il buon odore di Cristo fra quelli che son sulla via della salvezza e fra quelli che son sulla via della perdizione:
\par 16 a questi, un odore di morte, a morte; a quelli, un odore di vita, a vita. E chi è sufficiente a queste cose?
\par 17 Poiché noi non siamo come quei molti che adulterano la parola di Dio; ma parliamo mossi da sincerità, da parte di Dio, in presenza di Dio, in Cristo.

\chapter{3}

\par 1 Cominciamo noi di nuovo a raccomandar noi stessi? O abbiam noi bisogno, come alcuni, di lettere di raccomandazione presso di voi o da voi?
\par 2 Siete voi la nostra lettera, scritta nei nostri cuori, conosciuta e letta da tutti gli uomini;
\par 3 essendo manifesto che voi siete una lettera di Cristo, scritta mediante il nostro ministerio, scritta non con inchiostro, ma con lo Spirito dell'Iddio vivente; non su tavole di pietra, ma su tavole che son cuori di carne.
\par 4 E una tal confidanza noi l'abbiamo per mezzo di Cristo presso Dio.
\par 5 Non già che siam di per noi stessi capaci di pensare alcun che, come venendo da noi;
\par 6 ma la nostra capacità viene da Dio, che ci ha anche resi capaci d'esser ministri di un nuovo patto, non di lettera, ma di spirito; perché la lettera uccide, ma lo spirito vivifica.
\par 7 Ora se il ministerio della morte scolpito in lettere su pietre fu circondato di gloria, talché i figliuoli d'Israele non poteano fissar lo sguardo nel volto di Mosè a motivo della gloria, che pur svaniva, del volto di lui,
\par 8 non sarà il ministerio dello Spirito circondato di molto maggior gloria?
\par 9 Se, infatti, il ministerio della condanna fu con gloria, molto più abbonda in gloria il ministerio della giustizia.
\par 10 Anzi, quel che nel primo fu reso glorioso, non fu reso veramente glorioso, quando lo si confronti colla gloria di tanto superiore del secondo;
\par 11 perché, se ciò che aveva da sparire fu circondato di gloria, molto più ha da esser glorioso ciò che ha da durare.
\par 12 Avendo dunque una tale speranza, noi usiamo grande franchezza,
\par 13 e non facciamo come Mosè, che si metteva un velo sulla faccia, perché i figliuoli d'Israele non fissassero lo sguardo nella fine di ciò che doveva sparire.
\par 14 Ma le loro menti furon rese ottuse; infatti, sino al dì d'oggi, quando fanno la lettura dell'antico patto, lo stesso velo rimane, senz'essere rimosso, perché è in Cristo ch'esso è abolito.
\par 15 Ma fino ad oggi, quando si legge Mosè, un velo rimane steso sul cuor loro;
\par 16 quando però si saranno convertiti al Signore, il velo sarà rimosso.
\par 17 Ora, il Signore è lo Spirito; e dov'è lo Spirito del Signore, quivi è libertà.
\par 18 E noi tutti contemplando a viso scoperto, come in uno specchio, la gloria del Signore, siamo trasformati nell'istessa immagine di lui, di gloria in gloria, secondo che opera il Signore, che è Spirito.

\chapter{4}

\par 1 Perciò, avendo questo ministerio in virtù della misericordia che ci è stata fatta, noi non veniam meno nell'animo,
\par 2 ma abbiam rinunziato alle cose nascoste e vergognose, non procedendo con astuzia né falsificando la parola di Dio, ma mediante la manifestazione della verità raccomandando noi stessi alla coscienza di ogni uomo nel cospetto di Dio.
\par 3 E se il nostro vangelo è ancora velato, è velato per quelli che son sulla via della perdizione,
\par 4 per gl'increduli, dei quali l'iddio di questo secolo ha accecato le menti, affinché la luce dell'evangelo della gloria di Cristo, che è l'immagine di Dio, non risplenda loro.
\par 5 Poiché noi non predichiamo noi stessi, ma Cristo Gesù qual Signore, e quanto a noi ci dichiariamo vostri servitori per amor di Gesù;
\par 6 perché l'Iddio che disse: Splenda la luce fra le tenebre, è quel che risplendé ne' nostri cuori affinché noi facessimo brillare la luce della conoscenza della gloria di Dio che rifulge nel volto di Gesù Cristo.
\par 7 Ma noi abbiamo questo tesoro in vasi di terra, affinché l'eccellenza di questa potenza sia di Dio e non da noi.
\par 8 Noi siamo tribolati, in ogni maniera, ma non ridotti all'estremo; perplessi, ma non disperati;
\par 9 perseguitati, ma non abbandonati; atterrati, ma non uccisi;
\par 10 portiam sempre nel nostro corpo la morte di Gesù, perché anche la vita di Gesù si manifesti nel nostro corpo;
\par 11 poiché noi che viviamo, siam sempre esposti alla morte per amor di Gesù, onde anche la vita di Gesù sia manifestata nella nostra carne mortale.
\par 12 Talché la morte opera in noi, ma la vita in voi.
\par 13 Ma siccome abbiam lo stesso spirito di fede, ch'è in quella parola della Scrittura: Ho creduto, perciò ho parlato, anche noi crediamo, e perciò anche parliamo,
\par 14 sapendo che Colui che risuscitò il Signor Gesù, risusciterà anche noi con Gesù, e ci farà comparir con voi alla sua presenza.
\par 15 Poiché tutte queste cose avvengono per voi, affinché la grazia essendo abbondata, faccia sì che sovrabbondi per bocca di un gran numero il ringraziamento alla gloria di Dio.
\par 16 Perciò noi non veniamo meno nell'animo; ma quantunque il nostro uomo esterno si disfaccia, pure il nostro uomo interno si rinnova di giorno in giorno.
\par 17 Perché la nostra momentanea, leggera afflizione ci produce un sempre più grande, smisurato peso eterno di gloria,
\par 18 mentre abbiamo lo sguardo intento non alle cose che si vedono, ma a quelle che non si vedono; poiché le cose che si vedono son solo per un tempo, ma quelle che non si vedono sono eterne.

\chapter{5}

\par 1 Noi sappiamo infatti che se questa tenda ch'è la nostra dimora terrena viene disfatta, noi abbiamo da Dio un edificio, una casa non fatta da mano d'uomo, eterna, nei cieli.
\par 2 Poiché in questa tenda noi gemiamo, bramando di esser sopravvestiti della nostra abitazione che è celeste,
\par 3 se pur sarem trovati vestiti e non ignudi.
\par 4 Poiché noi che siamo in questa tenda, gemiamo, aggravati; e perciò desideriamo non già d'esser spogliati, ma d'essere sopravvestiti, onde ciò che è mortale sia assorbito dalla vita.
\par 5 Or Colui che ci ha formati per questo stesso è Dio, il quale ci ha dato la caparra dello Spirito.
\par 6 Noi siamo dunque sempre pieni di fiducia, e sappiamo che mentre abitiamo nel corpo, siamo assenti dal Signore
\par 7 (poiché camminiamo per fede e non per visione);
\par 8 ma siamo pieni di fiducia e abbiamo molto più caro di partire dal corpo e d'abitare col Signore.
\par 9 Ed è perciò che ci studiamo d'essergli grati, sia che abitiamo nel corpo, sia che ne partiamo.
\par 10 Poiché dobbiamo tutti comparire davanti al tribunale di Cristo, affinché ciascuno riceva la retribuzione delle cose fatte quand'era nel corpo, secondo quel che avrà operato, o bene, o male.
\par 11 Sapendo dunque il timor che si deve avere del Signore, noi persuadiamo gli uomini; e Dio ci conosce a fondo, e spero che nelle vostre coscienze anche voi ci conoscete.
\par 12 Noi non ci raccomandiamo di nuovo a voi, ma vi diamo l'occasione di gloriarvi di noi, affinché abbiate di che rispondere a quelli che si gloriano di ciò che è apparenza e non di ciò che è nel cuore.
\par 13 Perché, se siamo fuor di senno, lo siamo a gloria di Dio e se siamo di buon senno lo siamo per voi;
\par 14 poiché l'amore di Cristo ci costringe; perché siamo giunti a questa conclusione: che uno solo morì per tutti, quindi tutti morirono;
\par 15 e ch'egli morì per tutti, affinché quelli che vivono non vivano più per loro stessi, ma per colui che è morto e risuscitato per loro.
\par 16 Talché, da ora in poi, noi non conosciamo più alcuno secondo la carne; e se anche abbiam conosciuto Cristo secondo la carne, ora però non lo conosciamo più così.
\par 17 Se dunque uno è in Cristo, egli è una nuova creatura; le cose vecchie son passate: ecco, son diventate nuove.
\par 18 E tutto questo vien da Dio che ci ha riconciliati con sé per mezzo di Cristo e ha dato a noi il ministerio della riconciliazione;
\par 19 in quanto che Iddio riconciliava con sé il mondo in Cristo non imputando agli uomini i loro falli, ed ha posta in noi la parola della riconciliazione.
\par 20 Noi dunque facciamo da ambasciatori per Cristo, come se Dio esortasse per mezzo nostro; vi supplichiamo nel nome di Cristo: Siate riconciliati con Dio.
\par 21 Colui che non ha conosciuto peccato, Egli l'ha fatto esser peccato per noi, affinché noi diventassimo giustizia di Dio in lui.

\chapter{6}

\par 1 Come collaboratori di Dio, noi v'esortiamo pure a far sì che non abbiate ricevuta la grazia di Dio invano;
\par 2 poiché egli dice: T'ho esaudito nel tempo accettevole, e t'ho soccorso nel giorno della salvezza. Eccolo ora il tempo accettevole; eccolo ora il giorno della salvezza!
\par 3 Noi non diamo motivo di scandalo in cosa alcuna, onde il ministerio non sia vituperato;
\par 4 ma in ogni cosa ci raccomandiamo come ministri di Dio per una grande costanza, per afflizioni, necessità, angustie,
\par 5 battiture, prigionie, sommosse, fatiche, veglie, digiuni,
\par 6 per purità, conoscenza, longanimità, benignità, per lo Spirito Santo, per carità non finta;
\par 7 per la parola di verità, per la potenza di Dio; per le armi di giustizia a destra e a sinistra,
\par 8 in mezzo alla gloria e all'ignominia, in mezzo alla buona ed alla cattiva riputazione; tenuti per seduttori, eppur veraci;
\par 9 sconosciuti, eppur ben conosciuti moribondi, eppur eccoci viventi castigati, eppur non messi a morte;
\par 10 contristati, eppur sempre allegri; poveri, eppure arricchenti molti; non avendo nulla, eppur possedenti ogni cosa!
\par 11 La nostra bocca vi ha parlato apertamente, o Corinzî; il nostro cuore s'è allargato.
\par 12 Voi non siete allo stretto in noi, ma è il vostro cuore che si è ristretto.
\par 13 Ora, per renderci il contraccambio (parlo come a figliuoli), allargate il cuore anche voi!
\par 14 Non vi mettete con gl'infedeli sotto un giogo che non è per voi; perché qual comunanza v'è egli fra la giustizia e l'iniquità? O qual comunione fra la luce e le tenebre?
\par 15 E quale armonia fra Cristo e Beliar? O che v'è di comune tra il fedele e l'infedele?
\par 16 E quale accordo fra il tempio di Dio e gl'idoli? Poiché noi siamo il tempio dell'Iddio vivente, come disse Iddio: Io abiterò in mezzo a loro e camminerò fra loro; e sarò loro Dio, ed essi saranno mio popolo.
\par 17 Perciò Uscite di mezzo a loro e separatevene, dice il Signore, e non toccate nulla d'immondo; ed io v'accoglierò,
\par 18 e vi sarò per Padre e voi mi sarete per figliuoli e per figliuole, dice il Signore onnipotente.

\chapter{7}

\par 1 Poiché dunque abbiam queste promesse, diletti, purifichiamoci d'ogni contaminazione di carne e di spirito, compiendo la nostra santificazione nel timor di Dio.
\par 2 Fateci posto nei vostri cuori! Noi non abbiam fatto torto ad alcuno, non abbiam nociuto ad alcuno, non abbiamo sfruttato alcuno.
\par 3 Non lo dico per condannarvi, perché ho già detto prima che voi siete nei nostri cuori per la morte e per la vita.
\par 4 Grande è la franchezza che uso con voi; molto ho da gloriarmi di voi; son ripieno di consolazione, io trabocco d'allegrezza in tutta la nostra afflizione.
\par 5 Poiché, anche dopo che fummo giunti in Macedonia, la nostra carne non ha avuto requie alcuna, ma siamo stati afflitti in ogni maniera; combattimenti di fuori, di dentro timori.
\par 6 Ma Iddio che consola gli abbattuti, ci consolò con la venuta di Tito;
\par 7 e non soltanto con la venuta di lui, ma anche con la consolazione da lui provata a vostro riguardo. Egli ci ha raccontato la vostra bramosia di noi, il vostro pianto, il vostro zelo per me; ond'io mi sono più che mai rallegrato.
\par 8 Poiché, quand'anche io v'abbia contristati con la mia epistola, non me ne rincresce; e se pur ne ho provato rincrescimento (poiché vedo che quella epistola, quantunque per un breve tempo, vi ha contristati),
\par 9 ora mi rallegro, non perché siete stati contristati, ma perché siete stati contristati a ravvedimento; poiché siete stati contristati secondo Iddio, onde non aveste a ricever alcun danno da noi.
\par 10 Poiché, la tristezza secondo Dio produce un ravvedimento che mena alla salvezza, e del quale non c'è mai da pentirsi; ma la tristezza del mondo produce la morte.
\par 11 Infatti, questo essere stati contristati secondo Iddio, vedete quanta premura ha prodotto in voi! Anzi, quanta giustificazione, quanto sdegno, quanto timore, quanta bramosia, quanto zelo, qual punizione! In ogni maniera avete dimostrato d'esser puri in quest'affare.
\par 12 Sebbene dunque io v'abbia scritto non è a motivo di chi ha fatto l'ingiuria né a motivo di chi l'ha patita, ma perché la premura che avete per noi fosse manifestata presso di voi nel cospetto di Dio.
\par 13 Perciò siamo stati consolati; e oltre a questa nostra consolazione ci siamo più che mai rallegrati per l'allegrezza di Tito, perché il suo spirito è stato ricreato da voi tutti.
\par 14 Che se mi sono in qualcosa gloriato di voi con lui, non sono stato confuso; ma come v'abbiam detto in ogni cosa la verità, così anche il nostro vanto di voi con Tito è risultato verità.
\par 15 Ed egli vi ama più che mai svisceratamente, quando si ricorda dell'ubbidienza di voi tutti, e come l'avete ricevuto con timore e tremore.
\par 16 Io mi rallegro che in ogni cosa posso aver fiducia in voi.

\chapter{8}

\par 1 Or, fratelli, vogliamo farvi sapere la grazia da Dio concessa alle chiese di Macedonia.
\par 2 In mezzo alle molte afflizioni con le quali esse sono provate, l'abbondanza della loro allegrezza e la loro profonda povertà hanno abbondato nelle ricchezze della loro liberalità.
\par 3 Poiché, io ne rendo testimonianza, secondo il poter loro, anzi al di là del poter loro, hanno dato volonterosi,
\par 4 chiedendoci con molte istanze la grazia di contribuire a questa sovvenzione destinata ai santi.
\par 5 E l'hanno fatto non solo come avevamo sperato; ma prima si sono dati loro stessi al Signore, e poi a noi, per la volontà di Dio.
\par 6 Talché abbiamo esortato Tito che, come l'ha già cominciata, così porti a compimento fra voi anche quest'opera di carità.
\par 7 Ma siccome voi abbondate in ogni cosa, in fede, in parola, in conoscenza, in ogni zelo e nell'amore che avete per noi, vedete d'abbondare anche in quest'opera di carità.
\par 8 Non lo dico per darvi un ordine, ma per mettere alla prova, con l'esempio dell'altrui premura, anche la schiettezza del vostro amore.
\par 9 Perché voi conoscete la carità del Signor nostro Gesù Cristo il quale, essendo ricco, s'è fatto povero per amor vostro, onde, mediante la sua povertà, voi poteste diventar ricchi.
\par 10 E qui vi do un consiglio; il che conviene a voi i quali fin dall'anno passato avete per i primi cominciato non solo a fare ma anche a volere:
\par 11 Portate ora a compimento anche il fare; onde, come ci fu la prontezza del volere, così ci sia anche il compiere secondo i vostri mezzi.
\par 12 Poiché, se c'è la prontezza dell'animo, essa è gradita in ragione di quello che uno ha, e non di quello che non ha.
\par 13 Poiché questo non si fa per recar sollievo ad altri ed aggravio a voi, ma per principio di uguaglianza;
\par 14 nelle attuali circostanze, la vostra abbondanza serve a supplire al loro bisogno, onde la loro abbondanza supplisca altresì al bisogno vostro, affinché ci sia uguaglianza, secondo che è scritto:
\par 15 Chi avea raccolto molto non n'ebbe di soverchio, e chi avea raccolto poco, non n'ebbe mancanza.
\par 16 Or ringraziato sia Iddio che ha messo in cuore a Tito lo stesso zelo per voi;
\par 17 poiché non solo egli ha accettata la nostra esortazione, ma mosso da zelo anche maggiore si è spontaneamente posto in cammino per venire da voi.
\par 18 E assieme a lui abbiam mandato questo fratello, la cui lode nella predicazione dell'Evangelo è sparsa per tutte le chiese;
\par 19 non solo, ma egli è stato anche eletto dalle chiese a viaggiare con noi per quest'opera di carità, da noi amministrata per la gloria del Signore stesso e per dimostrare la prontezza dell'animo nostro.
\par 20 Evitiamo così che qualcuno abbia a biasimarci circa quest'abbondante colletta che è da noi amministrata;
\par 21 perché ci preoccupiamo d'agire onestamente non solo nel cospetto del Signore, ma anche nel cospetto degli uomini.
\par 22 E con loro abbiamo mandato quel nostro fratello del quale spesse volte e in molte cose abbiamo sperimentato lo zelo, e che ora è più zelante che mai per la gran fiducia che ha in voi.
\par 23 Quanto a Tito, egli è mio compagno e collaboratore in mezzo a voi; quanto ai nostri fratelli, essi sono gli inviati delle chiese, e gloria di Cristo.
\par 24 Date loro dunque, nel cospetto delle chiese, la prova del vostro amore e mostrate loro che abbiamo ragione di gloriarci di voi.

\chapter{9}

\par 1 Quanto alla sovvenzione destinata ai santi, è superfluo ch'io ve ne scriva,
\par 2 perché conosco la prontezza dell'animo vostro, per la quale mi glorio di voi presso i Macedoni, dicendo che l'Acaia è pronta fin dall'anno passato; e il vostro zelo ne ha stimolati moltissimi.
\par 3 Ma ho mandato i fratelli onde il nostro gloriarci di voi non riesca vano per questo rispetto; affinché, come dissi, siate pronti;
\par 4 che talora, se venissero meco dei Macedoni e vi trovassero non preparati, noi (per non dir voi) non avessimo ad essere svergognati per questa nostra fiducia.
\par 5 Perciò ho reputato necessario esortare i fratelli a venire a voi prima di me e preparare la vostra già promessa liberalità, ond'essa sia pronta come atto di liberalità e non d'avarizia.
\par 6 Or questo io dico: chi semina scarsamente mieterà altresì scarsamente; e chi semina liberalmente mieterà altresì liberalmente.
\par 7 Dia ciascuno secondo che ha deliberato in cuor suo; non di mala voglia, né per forza perché Iddio ama un donatore allegro.
\par 8 E Dio è potente da far abbondare su di voi ogni grazia, affinché, avendo sempre in ogni cosa tutto quel che vi è necessario, abbondiate in ogni opera buona;
\par 9 siccome è scritto: Egli ha sparso, egli ha dato ai poveri, la sua giustizia dimora in eterno.
\par 10 Or Colui che fornisce al seminatore la semenza, e il pane da mangiare, fornirà e moltiplicherà la semenza vostra e accrescerà i frutti della vostra giustizia.
\par 11 Sarete così arricchiti in ogni cosa onde potere esercitare una larga liberalità, la quale produrrà per nostro mezzo rendimento di grazie a Dio.
\par 12 Poiché la prestazione di questo servigio sacro non solo supplisce ai bisogni dei santi ma più ancora produce abbondanza di ringraziamenti a Dio;
\par 13 in quanto che la prova pratica fornita da questa sovvenzione li porta a glorificare Iddio per l'ubbidienza con cui professate il Vangelo di Cristo, e per la liberalità con cui partecipate ai bisogni loro e di tutti.
\par 14 E con le loro preghiere a pro vostro essi mostrano d'esser mossi da vivo affetto per voi a motivo della sovrabbondante grazia di Dio che è sopra voi.
\par 15 Ringraziato sia Dio del suo dono ineffabile!

\chapter{10}

\par 1 Io poi, Paolo, vi esorto per la mansuetudine e la mitezza di Cristo, io che quando sono presente fra voi son umile, ma quando sono assente sono ardito verso voi,
\par 2 vi prego di non obbligarmi, quando sarò presente, a procedere arditamente con quella sicurezza onde fo conto d'essere audace contro taluni che ci stimano come se camminassimo secondo la carne.
\par 3 Perché sebbene camminiamo nella carne, non combattiamo secondo la carne;
\par 4 infatti le armi della nostra guerra non sono carnali, ma potenti nel cospetto di Dio a distruggere le fortezze;
\par 5 poiché distruggiamo i ragionamenti ed ogni altezza che si eleva contro alla conoscenza di Dio, e facciam prigione ogni pensiero traendolo all'ubbidienza di Cristo;
\par 6 e siam pronti a punire ogni disubbidienza, quando la vostra ubbidienza sarà completa.
\par 7 Voi guardate all'apparenza delle cose. Se uno confida dentro di sé d'esser di Cristo, consideri anche questo dentro di sé: che com'egli è di Cristo, così siamo anche noi.
\par 8 Poiché, quand'anche io mi gloriassi un po' di più dell'autorità che il Signore ci ha data per la edificazione vostra e non per la vostra rovina, non ne sarei svergognato.
\par 9 Dico questo perché non paia ch'io cerchi di spaventarvi con le mie lettere.
\par 10 Difatti, dice taluno, ben sono le sue lettere gravi e forti; ma la sua presenza personale è debole, e la sua parola è cosa da nulla.
\par 11 Quel tale tenga questo per certo: che quali siamo a parole, per via di lettere, quando siamo assenti, tali saremo anche a fatti quando saremo presenti.
\par 12 Poiché noi non osiamo annoverarci o paragonarci con certuni che si raccomandano da sé; i quali però, misurandosi alla propria stregua e paragonando sé con se stessi, sono senza giudizio.
\par 13 Noi, invece, non ci glorieremo oltre misura, ma entro la misura del campo di attività di cui Dio ci ha segnato i limiti, dandoci di giungere anche fino a voi.
\par 14 Poiché non ci estendiamo oltre il dovuto, quasi che non fossimo giunti fino a voi; perché fino a voi siamo realmente giunti col Vangelo di Cristo.
\par 15 E non ci gloriamo oltre misura di fatiche altrui, ma nutriamo speranza che, crescendo la fede vostra, noi, senza uscire dai nostri limiti, saremo fra voi ampiamente ingranditi
\par 16 in guisa da poter evangelizzare anche i paesi che sono al di là del vostro, e da non gloriarci, entrando nel campo altrui, di cose bell'e preparate.
\par 17 Ma chi si gloria, si glorî nel Signore.
\par 18 Poiché non colui che raccomanda se stesso è approvato, ma colui che il Signore raccomanda.

\chapter{11}

\par 1 Oh quanto desidererei che voi sopportaste da parte mia un po' di follia! Ma pure, sopportatemi!
\par 2 Poiché io son geloso di voi d'una gelosia di Dio, perché v'ho fidanzati ad un unico sposo, per presentarvi come una casta vergine a Cristo.
\par 3 Ma temo che come il serpente sedusse Eva con la sua astuzia, così le vostre menti siano corrotte e sviate dalla semplicità e dalla purità rispetto a Cristo.
\par 4 Infatti, se uno viene a predicarvi un altro Gesù, diverso da quello che abbiamo predicato noi, o se si tratta di ricevere uno Spirito diverso da quello che avete ricevuto, o un Vangelo diverso da quello che avete accettato, voi ben lo sopportate!
\par 5 Ora io stimo di non essere stato in nulla da meno di cotesti sommi apostoli.
\par 6 Che se pur sono rozzo nel parlare, tale non sono nella conoscenza; e l'abbiamo dimostrato fra voi, per ogni rispetto e in ogni cosa.
\par 7 Ho io commesso peccato quando, abbassando me stesso perché voi foste innalzati, v'ho annunziato l'evangelo di Dio gratuitamente?
\par 8 Ho spogliato altre chiese, prendendo da loro uno stipendio, per poter servir voi;
\par 9 e quando, durante il mio soggiorno fra voi, mi trovai nel bisogno, non fui d'aggravio a nessuno, perché i fratelli, venuti dalla Macedonia, supplirono al mio bisogno; e in ogni cosa mi sono astenuto e m'asterrò ancora dall'esservi d'aggravio.
\par 10 Com'è vero che la verità di Cristo è in me, questo vanto non mi sarà tolto nelle contrade dell'Acaia.
\par 11 Perché? Forse perché non v'amo? Lo sa Iddio.
\par 12 Ma quel che fo lo farò ancora per togliere ogni occasione a coloro che desiderano un'occasione; affinché in quello di cui si vantano siano trovati uguali a noi.
\par 13 Poiché cotesti tali sono dei falsi apostoli, degli operai fraudolenti, che si travestono da apostoli di Cristo.
\par 14 E non c'è da maravigliarsene, perché anche Satana si traveste da angelo di luce.
\par 15 Non è dunque gran che se anche i suoi ministri si travestono da ministri di giustizia; la fine loro sarà secondo le loro opere.
\par 16 Lo dico di nuovo: Nessuno mi prenda per pazzo; o se no, anche come pazzo accettatemi, onde anch'io possa gloriarmi un poco.
\par 17 Quello che dico, quando mi vanto con tanta fiducia, non lo dico secondo il Signore, ma come in pazzia.
\par 18 Dacché molti si gloriano secondo la carne, anch'io mi glorierò.
\par 19 Difatti, voi, che siete assennati, li sopportate volentieri i pazzi.
\par 20 Che se uno vi riduce in schiavitù, se uno vi divora, se uno vi prende il vostro, se uno s'innalza sopra voi, se uno vi percuote in faccia, voi lo sopportate.
\par 21 Lo dico a nostra vergogna, come se noi fossimo stati deboli; eppure, in qualunque cosa uno possa essere baldanzoso (parlo da pazzo), sono baldanzoso anch'io.
\par 22 Son dessi Ebrei? Lo sono anch'io. Son dessi Israeliti? Lo sono anch'io. Son dessi progenie d'Abramo? Lo sono anch'io.
\par 23 Son dessi ministri di Cristo? (Parlo come uno fuor di sé), io lo sono più di loro; più di loro per le fatiche, più di loro per le carcerazioni, assai più di loro per le battiture sofferte. Sono spesso stato in pericolo di morte.
\par 24 Dai Giudei cinque volte ho ricevuto quaranta colpi meno uno:
\par 25 tre volte sono stato battuto con le verghe; una volta sono stato lapidato; tre volte ho fatto naufragio; ho passato un giorno e una notte sull'abisso.
\par 26 Spesse volte in viaggio, in pericoli sui fiumi, in pericoli di ladroni, in pericoli per parte de' miei connazionali, in pericoli per parte dei Gentili, in pericoli in città, in pericoli nei deserti, in pericoli sul mare, in pericoli tra falsi fratelli;
\par 27 in fatiche ed in pene; spesse volte in veglie, nella fame e nella sete, spesse volte nei digiuni, nel freddo e nella nudità.
\par 28 E per non parlar d'altro, c'è quel che m'assale tutti i giorni, l'ansietà per tutte le chiese.
\par 29 Chi è debole ch'io non sia debole? Chi è scandalizzato, che io non arda?
\par 30 Se bisogna gloriarsi, io mi glorierò delle cose che concernono la mia debolezza.
\par 31 L'Iddio e Padre del nostro Signor Gesù che è benedetto in eterno, sa ch'io non mento.
\par 32 A Damasco, il governatore del re Areta avea posto delle guardie alla città dei Damasceni per pigliarmi;
\par 33 e da una finestra fui calato, in una cesta, lungo il muro, e scampai dalle sue mani.

\chapter{12}

\par 1 Bisogna gloriarmi: non è cosa giovevole, ma pure, verrò alle visioni e alle rivelazioni del Signore.
\par 2 Io conosco un uomo in Cristo, che quattordici anni fa (se fu col corpo non so, né so se fu senza il corpo; Iddio lo sa), fu rapito fino al terzo cielo.
\par 3 E so che quel tale (se fu col corpo o senza il corpo non so;
\par 4 Iddio lo sa) fu rapito in paradiso, e udì parole ineffabili che non è lecito all'uomo di proferire.
\par 5 Di quel tale io mi glorierò; ma di me stesso non mi glorierò se non nelle mie debolezze.
\par 6 Che se pur volessi gloriarmi, non sarei un pazzo, perché direi la verità; ma me ne astengo, perché nessuno mi stimi al di là di quel che mi vede essere, ovvero ode da me.
\par 7 E perché io non avessi ad insuperbire a motivo della eccellenza delle rivelazioni, m'è stata messa una scheggia nella carne, un angelo di Satana, per schiaffeggiarmi ond'io non insuperbisca.
\par 8 Tre volte ho pregato il Signore perché l'allontanasse da me;
\par 9 ed egli mi ha detto: La mia grazia ti basta, perché la mia potenza si dimostra perfetta nella debolezza. Perciò molto volentieri mi glorierò piuttosto delle mie debolezze, onde la potenza di Cristo riposi su me.
\par 10 Per questo io mi compiaccio in debolezze, in ingiurie, in necessità, in persecuzioni, in angustie per amor di Cristo; perché, quando son debole, allora sono forte.
\par 11 Son diventato pazzo; siete voi che mi ci avete costretto; poiché io avrei dovuto esser da voi raccomandato; perché in nulla sono stato da meno di cotesti sommi apostoli, benché io non sia nulla.
\par 12 Certo, i segni dell'apostolo sono stati manifestati in atto fra voi nella perseveranza a tutta prova, nei miracoli, nei prodigî ed opere potenti.
\par 13 In che siete voi stati da meno delle altre chiese se non nel fatto che io stesso non vi sono stato d'aggravio? Perdonatemi questo torto.
\par 14 Ecco, questa è la terza volta che son pronto a recarmi da voi; e non vi sarò d'aggravio, poiché io non cerco i vostri beni, ma voi; perché non sono i figliuoli che debbono far tesoro per i genitori, ma i genitori per i figliuoli.
\par 15 E io molto volentieri spenderò e sarò speso per le anime vostre. Se io v'amo tanto, devo esser da voi amato meno?
\par 16 Ma sia pure così, ch'io non vi sia stato d'aggravio; ma, forse, da uomo astuto, v'ho presi con inganno.
\par 17 Mi son io approfittato di voi per mezzo di qualcuno di quelli ch'io v'ho mandato?
\par 18 Ho pregato Tito di venire da voi, e ho mandato quell'altro fratello con lui. Tito si è forse approfittato di voi? Non abbiam noi camminato col medesimo spirito e seguito le medesime orme?
\par 19 Da tempo voi v'immaginate che noi ci difendiamo dinanzi a voi. Egli è nel cospetto di Dio, in Cristo, che noi parliamo; e tutto questo, diletti, per la vostra edificazione.
\par 20 Poiché io temo, quando verrò, di trovarvi non quali vorrei, e d'essere io stesso da voi trovato quale non mi vorreste; temo che vi siano tra voi contese, gelosie, ire, rivalità, maldicenze, insinuazioni, superbie, tumulti;
\par 21 e che al mio arrivo l'Iddio mio abbia di nuovo ad umiliarmi dinanzi a voi, ed io abbia a pianger molti di quelli che hanno per lo innanzi peccato, e non si sono ravveduti della impurità, della fornicazione e della dissolutezza a cui si erano dati.

\chapter{13}

\par 1 Questa è la terza volta ch'io vengo da voi. Ogni parola sarà confermata dalla bocca di due o di tre testimoni.
\par 2 Ho avvertito quand'ero presente fra voi la seconda volta, e avverto, ora che sono assente, tanto quelli che hanno peccato per l'innanzi, quanto tutti gli altri, che, se tornerò da voi, non userò indulgenza;
\par 3 giacché cercate la prova che Cristo parla in me: Cristo che verso voi non è debole, ma è potente in voi.
\par 4 Poiché egli fu crocifisso per la sua debolezza; ma vive per la potenza di Dio; e anche noi siam deboli in lui, ma vivremo con lui per la potenza di Dio, nel nostro procedere verso di voi.
\par 5 Esaminate voi stessi per vedere se siete nella fede; provate voi stessi. Non riconoscete voi medesimi che Gesù Cristo è in voi? A meno che proprio siate riprovati.
\par 6 Ma io spero che riconoscerete che noi non siamo riprovati.
\par 7 Or noi preghiamo Iddio che non facciate alcun male; non già per apparir noi approvati, ma perché voi facciate quello che è bene, anche se noi abbiam da passare per riprovati.
\par 8 Perché noi non possiamo nulla contro la verità; quel che possiamo è per la verità.
\par 9 Poiché noi ci rallegriamo quando siamo deboli e voi siete forti; e i nostri voti sono per il vostro perfezionamento.
\par 10 Perciò vi scrivo queste cose mentre sono assente, affinché, quando sarò presente, io non abbia a procedere rigorosamente secondo l'autorità che il Signore mi ha data per edificare, e non per distruggere.
\par 11 Del resto, fratelli, rallegratevi, procacciate la perfezione, siate consolati, abbiate un medesimo sentimento, vivete in pace; e l'Iddio dell'amore e della pace sarà con voi.
\par 12 Salutatevi gli uni gli altri con un santo bacio.
\par 13 La grazia del Signor Gesù Cristo e l'amore di Dio e la comunione dello Spirito Santo siano con tutti voi.


\end{document}