\begin{document}

\title{Mark}


\chapter{1}

\par 1 Principio dell'evangelo di Gesù Cristo, Figliuolo di Dio.
\par 2 Secondo ch'egli è scritto nel profeta Isaia: Ecco, io mando davanti a te il mio messaggero a prepararti la via...
\par 3 V'è una voce di uno che grida nel deserto: Preparate la via del Signore, addirizzate i suoi sentieri,
\par 4 apparve Giovanni il Battista nel deserto predicando un battesimo di ravvedimento per la remissione dei peccati.
\par 5 E tutto il paese della Giudea e tutti quei di Gerusalemme accorrevano a lui; ed erano da lui battezzati nel fiume Giordano, confessando i loro peccati.
\par 6 Or Giovanni era vestito di pel di cammello, con una cintura di cuoio intorno ai fianchi, e si nutriva di locuste e di miele selvatico.
\par 7 E predicava, dicendo: Dopo di me vien colui che è più forte di me; al quale io non son degno di chinarmi a sciogliere il legaccio dei calzari.
\par 8 Io vi ho battezzati con acqua, ma lui vi battezzerà con lo Spirito Santo.
\par 9 Ed avvenne in que' giorni che Gesù venne da Nazaret di Galilea e fu battezzato da Giovanni nel Giordano.
\par 10 E ad un tratto, com'egli saliva fuori dell'acqua, vide fendersi i cieli, e lo Spirito scendere su di lui in somiglianza di colomba.
\par 11 E una voce venne dai cieli: Tu sei il mio diletto Figliuolo; in te mi sono compiaciuto.
\par 12 E subito dopo lo Spirito lo sospinse nel deserto;
\par 13 e nel deserto rimase per quaranta giorni, tentato da Satana; e stava tra le fiere e gli angeli lo servivano.
\par 14 Dopo che Giovanni fu messo in prigione, Gesù si recò in Galilea, predicando l'evangelo di Dio e dicendo:
\par 15 Il tempo è compiuto e il regno di Dio è vicino; ravvedetevi e credete all'evangelo.
\par 16 Or passando lungo il mar della Galilea, egli vide Simone e Andrea, il fratello di Simone, che gettavano la rete in mare, perché erano pescatori. E Gesù disse loro:
\par 17 Seguitemi, ed io farò di voi dei pescatori d'uomini.
\par 18 Ed essi, lasciate subito le reti, lo seguirono.
\par 19 Poi, spintosi un po' più oltre, vide Giacomo di Zebedeo e Giovanni suo fratello, che anch'essi in barca rassettavano le reti;
\par 20 e subito li chiamò; ed essi, lasciato Zebedeo loro padre nella barca con gli operai, se n'andarono dietro a lui.
\par 21 E vennero in Capernaum; e subito, il sabato, Gesù, entrato nella sinagoga, insegnava.
\par 22 E la gente stupiva della sua dottrina, perch'egli li ammaestrava come avente autorità e non come gli scribi.
\par 23 In quel mentre, si trovava nella loro sinagoga un uomo posseduto da uno spirito immondo, il quale prese a gridare:
\par 24 Che v'è fra noi e te, o Gesù Nazareno? Se' tu venuto per perderci? Io so chi tu sei: il Santo di Dio!
\par 25 E Gesù lo sgridò, dicendo: Ammutolisci ed esci da costui!
\par 26 E lo spirito immondo, straziatolo e gridando forte, uscì da lui.
\par 27 E tutti sbigottirono talché si domandavano fra loro: Che cos'è mai questo? È una dottrina nuova! Egli comanda con autorità perfino agli spiriti immondi, ed essi gli ubbidiscono!
\par 28 E la sua fama si divulgò subito per ogni dove, in tutta la circostante contrada della Galilea.
\par 29 Ed appena usciti dalla sinagoga, vennero con Giacomo e Giovanni in casa di Simone e d'Andrea.
\par 30 Or la suocera di Simone era a letto con la febbre; ed essi subito gliene parlarono;
\par 31 ed egli, accostatosi, la prese per la mano e la fece levare; e la febbre la lasciò ed ella si mise a servirli.
\par 32 Poi, fattosi sera, quando il sole fu tramontato, gli menarono tutti i malati e gl'indemoniati.
\par 33 E tutta la città era raunata all'uscio.
\par 34 Ed egli ne guarì molti che soffrivan di diverse malattie, e cacciò molti demonî; e non permetteva ai demonî di parlare, poiché sapevano chi egli era.
\par 35 Poi, la mattina, essendo ancora molto buio, Gesù, levatosi, uscì e se ne andò in un luogo deserto; e quivi pregava.
\par 36 Simone e quelli ch'eran con lui gli tennero dietro;
\par 37 e trovatolo, gli dissero: Tutti ti cercano.
\par 38 Ed egli disse loro: Andiamo altrove, per i villaggi vicini, ond'io predichi anche là; poiché è per questo che io sono uscito.
\par 39 E andò per tutta la Galilea, predicando nelle loro sinagoghe e cacciando i demonî.
\par 40 E un lebbroso venne a lui e buttandosi in ginocchio lo pregò dicendo: Se tu vuoi, tu puoi mondarmi!
\par 41 E Gesù, mosso a pietà, stese la mano, lo toccò e gli disse: Lo voglio; sii mondato!
\par 42 E subito la lebbra sparì da lui, e fu mondato.
\par 43 E Gesù, avendogli fatte severe ammonizioni, lo mandò subito via e gli disse:
\par 44 Guàrdati dal farne parola ad alcuno; ma va', mostrati al sacerdote ed offri per la tua purificazione quel che Mosè ha prescritto; e questo serva loro di testimonianza.
\par 45 Ma colui, appena partito, si dette a proclamare e a divulgare il fatto; di modo che Gesù non poteva più entrar palesemente in città; ma se ne stava fuori in luoghi deserti, e da ogni parte la gente accorreva a lui.

\chapter{2}

\par 1 E dopo alcuni giorni, egli entrò di nuovo in Capernaum, e si seppe che era in casa;
\par 2 e si raunò tanta gente che neppure lo spazio dinanzi alla porta la potea contenere. Ed egli annunziava loro la Parola.
\par 3 E vennero a lui alcuni che menavano un paralitico portato da quattro.
\par 4 E non potendolo far giungere fino a lui a motivo della calca, scoprirono il tetto dalla parte dov'era Gesù; e fattavi un'apertura, calarono il lettuccio sul quale il paralitico giaceva.
\par 5 E Gesù, veduta la loro fede, disse al paralitico: Figliuolo, i tuoi peccati ti sono rimessi.
\par 6 Or alcuni degli scribi eran quivi seduti e così ragionavano in cuor loro:
\par 7 Perché parla costui in questa maniera? Egli bestemmia! Chi può rimettere i peccati, se non un solo, cioè Dio?
\par 8 E Gesù, avendo subito conosciuto nel suo spirito che ragionavano così dentro di sé, disse loro: Perché fate voi cotesti ragionamenti ne' vostri cuori?
\par 9 Che è più agevole, dire al paralitico: I tuoi peccati ti sono rimessi, oppur dirgli: Lèvati, togli il tuo lettuccio e cammina?
\par 10 Ora, affinché sappiate che il Figliuol dell'uomo ha potestà in terra di rimettere i peccati:
\par 11 Io tel dico (disse al paralitico), lèvati, togli il tuo lettuccio, e vattene a casa tua.
\par 12 E colui s'alzò, e subito, preso il suo lettuccio, se ne andò via in presenza di tutti; talché tutti stupivano e glorificavano Iddio dicendo: Una cosa così non la vedemmo mai.
\par 13 E Gesù uscì di nuovo verso il mare; e tutta la moltitudine andava a lui, ed egli li ammaestrava.
\par 14 E passando, vide Levi d'Alfeo seduto al banco della gabella, e gli disse: Seguimi. Ed egli, alzatosi, lo seguì.
\par 15 Ed avvenne che, mentre Gesù era a tavola in casa di lui, molti pubblicani e peccatori erano anch'essi a tavola con lui e coi suoi discepoli; poiché ve n'erano molti e lo seguivano.
\par 16 E gli scribi d'infra i Farisei vedutolo mangiar coi pubblicani e coi peccatori, dicevano ai suoi discepoli: Come mai mangia e beve coi pubblicani e i peccatori?
\par 17 E Gesù, udito ciò, disse loro: Non sono i sani che hanno bisogno del medico, ma i malati. Io non son venuto a chiamar de' giusti, ma dei peccatori.
\par 18 Or i discepoli di Giovanni e i Farisei solevano digiunare. E vennero a Gesù e gli dissero: Perché i discepoli di Giovanni e i discepoli dei Farisei digiunano, e i discepoli tuoi non digiunano?
\par 19 E Gesù disse loro: Possono gli amici dello sposo digiunare, mentre lo sposo è con loro? Finché hanno con sé lo sposo, non possono digiunare.
\par 20 Ma verranno i giorni che lo sposo sarà loro tolto; ed allora, in quei giorni, digiuneranno.
\par 21 Niuno cuce un pezzo di stoffa nuova sopra un vestito vecchio; altrimenti la toppa nuova porta via del vecchio, e lo strappo si fa peggiore.
\par 22 E niuno mette del vin nuovo in otri vecchi; altrimenti il vino fa scoppiare gli otri, ed il vino si perde insieme con gli otri; ma il vin nuovo va messo in otri nuovi.
\par 23 Or avvenne che in un giorno di sabato egli passava per i seminati, e i suoi discepoli, cammin facendo, si misero a svellere delle spighe.
\par 24 E i Farisei gli dissero: Vedi! perché fanno di sabato quel che non è lecito?
\par 25 Ed egli disse loro: Non avete voi mai letto quel che fece Davide, quando fu nel bisogno ed ebbe fame, egli e coloro ch'eran con lui?
\par 26 Com'egli, sotto il sommo sacerdote Abiatar, entrò nella casa di Dio e mangiò i pani di presentazione, che a nessuno è lecito mangiare se non ai sacerdoti, e ne diede anche a coloro che eran con lui?
\par 27 Poi disse loro: Il sabato è stato fatto per l'uomo e non l'uomo per il sabato;
\par 28 perciò il Figliuol dell'uomo è Signore anche del sabato.

\chapter{3}

\par 1 Poi entrò di nuovo in una sinagoga; e quivi era un uomo che avea la mano secca.
\par 2 E l'osservavano per vedere se lo guarirebbe in giorno di sabato, per poterlo accusare.
\par 3 Ed egli disse all'uomo che avea la mano secca: Lèvati là nel mezzo!
\par 4 Poi disse loro: È egli lecito, in giorno di sabato, di far del bene o di far del male? di salvare una persona o di ucciderla? Ma quelli tacevano.
\par 5 Allora Gesù, guardatili tutt'intorno con indignazione, contristato per l'induramento del cuor loro, disse all'uomo: Stendi la mano! Egli la stese, e la sua mano tornò sana.
\par 6 E i Farisei, usciti, tennero subito consiglio con gli Erodiani contro di lui, con lo scopo di farlo morire.
\par 7 Poi Gesù co' suoi discepoli si ritirò verso il mare; e dalla Galilea gran moltitudine lo seguitò;
\par 8 e dalla Giudea e da Gerusalemme e dalla Idumea e da oltre il Giordano e dai dintorni di Tiro e di Sidone una gran folla, udendo quante cose egli facea, venne a lui.
\par 9 Ed egli disse ai suoi discepoli che gli tenessero sempre pronta una barchetta a motivo della calca, che talora non l'affollasse.
\par 10 Perché egli ne aveva guariti molti; cosicché tutti quelli che aveano qualche flagello gli si precipitavano addosso per toccarlo.
\par 11 E gli spiriti immondi, quando lo vedevano, si gittavano davanti a lui e gridavano: Tu sei il Figliuol di Dio!
\par 12 Ed egli li sgridava forte, affinché non facessero conoscere chi egli era.
\par 13 Poi Gesù salì sul monte e chiamò a sé quei ch'egli stesso volle, ed essi andarono a lui.
\par 14 E ne costituì dodici per tenerli con sé
\par 15 e per mandarli a predicare con la potestà di cacciare i demonî.
\par 16 Costituì dunque i dodici, cioè: Simone, al quale mise nome Pietro;
\par 17 e Giacomo di Zebedeo e Giovanni fratello di Giacomo, ai quali pose nome Boanerges, che vuol dire figliuoli del tuono;
\par 18 e Andrea e Filippo e Bartolommeo e Matteo e Toma e Giacomo di Alfeo e Taddeo e Simone il Cananeo
\par 19 e Giuda Iscariot, quello che poi lo tradì.
\par 20 Poi entrò in una casa, e la moltitudine si adunò di nuovo, talché egli ed i suoi non potevan neppur prender cibo.
\par 21 Or i suoi parenti, udito ciò, vennero per impadronirsi di lui, perché dicevano:
\par 22 È fuori di sé. E gli scribi, ch'eran discesi da Gerusalemme, dicevano: Egli ha Beelzebub, ed è per l'aiuto del principe dei demonî ch'ei caccia i demonî.
\par 23 Ma egli, chiamatili a sé, diceva loro in parabole: Come può Satana cacciar Satana?
\par 24 E se un regno è diviso in parti contrarie, quel regno non può durare.
\par 25 E se una casa è divisa in parti contrarie, quella casa non potrà reggere.
\par 26 E se Satana insorge contro se stesso ed è diviso, non può reggere, ma deve finire.
\par 27 Ed anzi niuno può entrar nella casa dell'uomo forte e rapirgli le sue masserizie, se prima non abbia legato l'uomo forte; allora soltanto gli prederà la casa.
\par 28 In verità io vi dico: Ai figliuoli degli uomini saranno rimessi tutti i peccati e qualunque bestemmia avranno proferita;
\par 29 ma chiunque avrà bestemmiato contro lo Spirito Santo, non ha remissione in eterno, ma è reo d'un peccato eterno.
\par 30 Or egli parlava così perché dicevano: Ha uno spirito immondo.
\par 31 E giunsero sua madre ed i suoi fratelli; e fermatisi fuori, lo mandarono a chiamare.
\par 32 Una moltitudine gli stava seduta attorno, quando gli fu detto: Ecco tua madre, i tuoi fratelli e le tue sorelle là fuori che ti cercano.
\par 33 Ed egli rispose loro: Chi è mia madre? e chi sono i miei fratelli?
\par 34 E guardati in giro coloro che gli sedevano d'intorno, disse: Ecco mia madre e i miei fratelli!
\par 35 Chiunque avrà fatta la volontà di Dio, mi è fratello, sorella e madre.

\chapter{4}

\par 1 Gesù prese di nuovo ad insegnare presso il mare: e una gran moltitudine si radunò intorno a lui; talché egli, montato in una barca, vi sedette stando in mare, mentre tutta la moltitudine era a terra sulla riva.
\par 2 Ed egli insegnava loro molte cose in parabole e diceva loro nel suo insegnamento:
\par 3 Udite: Ecco, il seminatore uscì a seminare.
\par 4 Ed avvenne che mentre seminava, una parte del seme cadde lungo la strada; e gli uccelli vennero e lo mangiarono.
\par 5 Ed un'altra cadde in un suolo roccioso ove non avea molta terra; e subito spuntò, perché non avea terreno profondo;
\par 6 ma quando il sole si levò, fu riarsa; e perché non avea radice, si seccò.
\par 7 Ed un'altra cadde fra le spine; e le spine crebbero e l'affogarono e non fece frutto.
\par 8 Ed altre parti caddero nella buona terra; e portaron frutto che venne su e crebbe, e giunsero a dare qual trenta, qual sessanta e qual cento.
\par 9 Poi disse: Chi ha orecchi da udire oda.
\par 10 Quand'egli fu in disparte, quelli che gli stavano intorno coi dodici, lo interrogarono sulle parabole.
\par 11 Ed egli disse loro: A voi è dato di conoscere il mistero del regno di Dio; ma a quelli che son di fuori, tutto è presentato per via di parabole, affinché:
\par 12 vedendo, vedano sì, ma non discernano; udendo, odano sì, ma non intendano; che talora non si convertano, e i peccati non siano loro rimessi.
\par 13 Poi disse loro: Non intendete voi questa parabola? E come intenderete voi tutte le parabole?
\par 14 Il seminatore semina la Parola.
\par 15 Quelli che sono lungo la strada, sono coloro nei quali è seminata la Parola; e quando l'hanno udita, subito viene Satana e porta via la Parola seminata in loro.
\par 16 E parimente quelli che ricevono la semenza in luoghi rocciosi sono coloro che, quando hanno udito la Parola, la ricevono subito con allegrezza;
\par 17 e non hanno in sé radice ma son di corta durata; e poi, quando venga tribolazione o persecuzione a cagion della Parola, son subito scandalizzati.
\par 18 Ed altri sono quelli che ricevono la semenza fra le spine; cioè coloro che hanno udita la Parola;
\par 19 poi le cure mondane e l'inganno delle ricchezze e le cupidigie delle altre cose, penetrati in loro, affogano la Parola, e così riesce infruttuosa.
\par 20 Quelli poi che hanno ricevuto il seme in buona terra, sono coloro che odono la Parola e l'accolgono e fruttano qual trenta, qual sessanta e qual cento.
\par 21 Poi diceva ancora: Si reca forse la lampada per metterla sotto il moggio o sotto il letto? Non è ella recata per esser messa sul candeliere?
\par 22 Poiché non v'è nulla che sia nascosto se non in vista d'esser manifestato; e nulla è stato tenuto segreto, se non per esser messo in luce.
\par 23 Se uno ha orecchi da udire oda.
\par 24 Diceva loro ancora: Ponete mente a ciò che voi udite. Con la misura con la quale misurate, sarà misurato a voi; e a voi sarà data anche la giunta;
\par 25 poiché a chi ha sarà dato, e a chi non ha, anche quello che ha gli sarà tolto.
\par 26 Diceva ancora: Il regno di Dio è come un uomo che getti il seme in terra,
\par 27 e dorma e si levi, la notte e il giorno; il seme intanto germoglia e cresce nel modo ch'egli stesso ignora.
\par 28 La terra da se stessa dà il suo frutto: prima l'erba; poi la spiga; poi, nella spiga, il grano ben formato.
\par 29 E quando il frutto è maturo, subito e' vi mette la falce perché la mietitura è venuta.
\par 30 Diceva ancora: A che assomiglieremo il regno di Dio, o con qual parabola lo rappresenteremo?
\par 31 Esso è simile ad un granello di senapa, il quale, quando lo si semina in terra, è il più piccolo di tutti i semi che son sulla terra;
\par 32 ma quando è seminato, cresce e diventa maggiore di tutti i legumi; e fa de' rami tanto grandi, che all'ombra sua possono ripararsi gli uccelli del cielo.
\par 33 E con molte cosiffatte parabole esponeva loro la Parola, secondo che potevano intendere;
\par 34 e non parlava loro senza parabola; ma in privato spiegava ogni cosa ai suoi discepoli.
\par 35 In quel medesimo giorno, fattosi sera, Gesù disse loro: Passiamo all'altra riva.
\par 36 E i discepoli, licenziata la moltitudine, lo presero, così com'era, nella barca. E vi erano delle altre barche con lui.
\par 37 Ed ecco levarsi un gran turbine di vento che cacciava le onde nella barca, talché ella già si riempiva.
\par 38 Or egli stava a poppa, dormendo sul guanciale. I discepoli lo destano e gli dicono: Maestro, non ti curi tu che noi periamo?
\par 39 Ed egli, destatosi, sgridò il vento e disse al mare: Taci, calmati! E il vento cessò, e si fece gran bonaccia.
\par 40 Ed egli disse loro: Perché siete così paurosi? Come mai non avete voi fede?
\par 41 Ed essi furono presi da gran timore e si diceano gli uni agli altri: Chi è dunque costui, che anche il vento ed il mare gli ubbidiscono?

\chapter{5}

\par 1 E giunsero all'altra riva del mare nel paese de' Geraseni.
\par 2 E come Gesù fu smontato dalla barca, subito gli venne incontro dai sepolcri un uomo posseduto da uno spirito immondo,
\par 3 il quale nei sepolcri avea la sua dimora; e neppure con una catena poteva più alcuno tenerlo legato;
\par 4 poiché spesso era stato legato con ceppi e con catene; e le catene erano state da lui rotte, ed i ceppi spezzati, e niuno avea forza da domarlo.
\par 5 E di continuo, notte e giorno, fra i sepolcri e su per i monti, andava urlando e percotendosi con delle pietre.
\par 6 Or quand'ebbe veduto Gesù da lontano, corse e gli si prostrò dinanzi;
\par 7 e dato un gran grido, disse: Che v'è fra me e te, o Gesù, Figliuolo dell'Iddio altissimo? Io ti scongiuro, in nome di Dio, di non tormentarmi;
\par 8 perché Gesù gli diceva: Spirito immondo, esci da quest'uomo!
\par 9 E Gesù gli domandò: Qual è il tuo nome? Ed egli rispose: Il mio nome è Legione perché siamo molti.
\par 10 E lo pregava con insistenza che non li mandasse via dal paese.
\par 11 Or quivi pel monte stava a pascolare un gran branco di porci.
\par 12 E gli spiriti lo pregarono dicendo: Mandaci ne' porci, perché entriamo in essi.
\par 13 Ed egli lo permise loro. E gli spiriti immondi, usciti, entrarono ne' porci, ed il branco si avventò giù a precipizio nel mare.
\par 14 Eran circa duemila ed affogarono nel mare. E quelli che li pasturavano fuggirono e portaron la notizia in città e per la campagna; e la gente andò a vedere ciò che era avvenuto.
\par 15 E vennero a Gesù e videro l'indemoniato seduto, vestito ed in buon senno, lui che aveva avuto la legione; e s'impaurirono.
\par 16 E quelli che aveano visto, raccontarono loro ciò che era avvenuto all'indemoniato e il fatto de' porci.
\par 17 Ed essi presero a pregar Gesù che se ne andasse dai loro confini.
\par 18 E come egli montava nella barca, l'uomo ch'era stato indemoniato lo pregava di poter stare con lui.
\par 19 E Gesù non glielo permise, ma gli disse: Va' a casa tua dai tuoi, e racconta loro le grandi cose che il Signore ti ha fatto, e come egli ha avuto pietà di te.
\par 20 E quello se ne andò e cominciò a pubblicare per la Decapoli le grandi cose che Gesù avea fatto per lui. E tutti si maravigliavano.
\par 21 Ed essendo Gesù passato di nuovo in barca all'altra riva, una gran moltitudine si radunò attorno a lui; ed egli stava presso il mare.
\par 22 Ed ecco venire uno dei capi della sinagoga, chiamato Iairo, il quale, vedutolo, gli si getta ai piedi
\par 23 e lo prega instantemente, dicendo: La mia figliuolina è agli estremi. Vieni a metter sopra lei le mani, affinché sia salva e viva.
\par 24 E Gesù andò con lui, e gran moltitudine lo seguiva e l'affollava.
\par 25 Or una donna che avea un flusso di sangue da dodici anni,
\par 26 e molto avea sofferto da molti medici, ed avea speso tutto il suo senz'alcun giovamento, anzi era piuttosto peggiorata,
\par 27 avendo udito parlar di Gesù, venne per di dietro fra la calca e gli toccò la veste, perché diceva:
\par 28 Se riesco a toccare non foss'altro che le sue vesti, sarò salva.
\par 29 E in quell'istante il suo flusso ristagnò; ed ella sentì nel corpo d'esser guarita di quel flagello.
\par 30 E subito Gesù, conscio della virtù ch'era emanata da lui, voltosi indietro in quella calca, disse: Chi mi ha toccato le vesti?
\par 31 E i suoi discepoli gli dicevano: Tu vedi come la folla ti si serra addosso e dici: Chi mi ha toccato?
\par 32 Ed egli guardava attorno per vedere colei che avea ciò fatto.
\par 33 Ma la donna, paurosa e tremante, ben sapendo quel che era avvenuto in lei, venne e gli si gettò ai piedi, e gli disse tutta la verità.
\par 34 Ma Gesù le disse: Figliuola, la tua fede t'ha salvata; vattene in pace e sii guarita del tuo flagello.
\par 35 Mentr'egli parlava ancora, ecco arrivar gente da casa del capo della sinagoga, che gli dice: La tua figliuola è morta; perché incomodare più oltre il Maestro?
\par 36 Ma Gesù, inteso quel che si diceva, disse al capo della sinagoga: Non temere; solo abbi fede!
\par 37 E non permise ad alcuno di accompagnarlo, salvo che a Pietro, a Giacomo e a Giovanni, fratello di Giacomo.
\par 38 E giungono a casa del capo della sinagoga; ed egli vede del tumulto e gente che piange ed urla forte.
\par 39 Ed entrato, dice loro: Perché fate tanto strepito e piangete? La fanciulla non è morta, ma dorme.
\par 40 E si ridevano di lui. Ma egli, messili tutti fuori, prende seco il padre e la madre della fanciulla e quelli che eran con lui, ed entra là dove era la fanciulla.
\par 41 E presala per la mano le dice: Talithà cumì! che interpretato vuol dire: Giovinetta, io tel dico, lèvati!
\par 42 E tosto la giovinetta s'alzò e camminava, perché avea dodici anni. E furon subito presi da grande stupore;
\par 43 ed egli comandò loro molto strettamente che non lo risapesse alcuno: e disse che le fosse dato da mangiare.

\chapter{6}

\par 1 Poi si partì di là e venne nel suo paese e i suoi discepoli lo seguitarono.
\par 2 E venuto il sabato, si mise ad insegnar nella sinagoga; e la maggior parte, udendolo, stupivano dicendo: Donde ha costui queste cose? e che sapienza è questa che gli è data? e che cosa sono cotali opere potenti fatte per mano sua?
\par 3 Non è costui il falegname, il figliuol di Maria, e il fratello di Giacomo e di Giosè, di Giuda e di Simone? E le sue sorelle non stanno qui da noi? E si scandalizzavano di lui.
\par 4 Ma Gesù diceva loro: Niun profeta è sprezzato se non nella sua patria e tra i suoi parenti e in casa sua.
\par 5 E non poté far quivi alcun'opera potente, salvo che, imposte le mani ad alcuni pochi infermi, li guarì.
\par 6 E si maravigliava della loro incredulità. E andava attorno per i villaggi circostanti, insegnando.
\par 7 Poi chiamò a sé i dodici e cominciò a mandarli a due a due; e dette loro potestà sugli spiriti immondi.
\par 8 E comandò loro di non prender nulla per viaggio, se non un bastone soltanto; non pane, non sacca, non danaro nella cintura:
\par 9 ma di calzarsi di sandali e di non portar tunica di ricambio.
\par 10 E diceva loro: Dovunque sarete entrati in una casa, trattenetevi quivi, finché non ve ne andiate di là;
\par 11 e se in qualche luogo non vi ricevono né v'ascoltano, andandovene di là, scotetevi la polvere di sotto ai piedi; e ciò serva loro di testimonianza.
\par 12 E partiti, predicavano che la gente si ravvedesse;
\par 13 cacciavano molti demonî, ungevano d'olio molti infermi e li guarivano.
\par 14 Ora il re Erode udì parlar di Gesù (ché la sua rinomanza s'era sparsa), e diceva: Giovanni Battista è risuscitato dai morti; ed è per questo che agiscono in lui le potenze miracolose.
\par 15 Altri invece dicevano: È Elia! Ed altri: È un profeta come quelli di una volta.
\par 16 Ma Erode, udito ciò, diceva: Quel Giovanni ch'io ho fatto decapitare, è lui che è risuscitato!
\par 17 Poiché esso Erode avea fatto arrestare Giovanni e l'avea fatto incatenare in prigione a motivo di Erodiada, moglie di Filippo suo fratello, ch'egli, Erode, avea sposata.
\par 18 Giovanni infatti gli diceva: È non t'è lecito di tener la moglie di tuo fratello!
\par 19 Ed Erodiada gli serbava rancore e bramava di farlo morire, ma non poteva;
\par 20 perché Erode avea soggezione di Giovanni, sapendolo uomo giusto e santo, e lo proteggeva; dopo averlo udito era molto perplesso, e l'ascoltava volentieri.
\par 21 Ma venuto un giorno opportuno che Erode, nel suo natalizio, fece un convito ai grandi della sua corte, ai capitani ed ai primi della Galilea,
\par 22 la figliuola della stessa Erodiada, essendo entrata, ballò e piacque ad Erode ed ai commensali. E il re disse alla fanciulla: Chiedimi quello che vuoi e te lo darò.
\par 23 E le giurò: Ti darò quel che mi chiederai; fin la metà del mio regno.
\par 24 Costei, uscita, domandò a sua madre: Che chiederò? E quella le disse: La testa di Giovanni Battista.
\par 25 E rientrata subito frettolosamente dal re, gli fece così la domanda: Voglio che sul momento tu mi dia in un piatto la testa di Giovanni Battista.
\par 26 Il re ne fu grandemente attristato; ma a motivo de' giuramenti fatti e dei commensali, non volle dirle di no;
\par 27 e mandò subito una guardia con l'ordine di portargli la testa di lui.
\par 28 E quegli andò, lo decapitò nella prigione, e ne portò la testa in un piatto, e la dette alla fanciulla, e la fanciulla la dette a sua madre.
\par 29 I discepoli di Giovanni, udita la cosa, andarono a prendere il suo corpo e lo deposero in un sepolcro.
\par 30 Or gli apostoli, essendosi raccolti presso Gesù, gli riferirono tutto quello che avean fatto e insegnato.
\par 31 Ed egli disse loro: Venitevene ora in disparte, in luogo solitario, e riposatevi un po'. Difatti, era tanta la gente che andava e veniva, che essi non aveano neppur tempo di mangiare.
\par 32 Partirono dunque nella barca per andare in un luogo solitario in disparte.
\par 33 E molti li videro partire e li riconobbero; e da tutte le città accorsero là a piedi e vi giunsero prima di loro.
\par 34 E come Gesù fu sbarcato, vide una gran moltitudine e n'ebbe compassione, perché erano come pecore che non hanno pastore; e si mise ad insegnar loro molte cose.
\par 35 Ed essendo già tardi, i discepoli gli s'accostarono e gli dissero: Questo luogo è deserto ed è già tardi;
\par 36 licenziali, affinché vadano per le campagne e per i villaggi d'intorno a comprarsi qualcosa da mangiare.
\par 37 Ma egli rispose loro: Date lor voi da mangiare. Ed essi a lui: Andremo noi a comprare per dugento danari di pane e daremo loro da mangiare?
\par 38 Ed egli domandò loro: Quanti pani avete? andate a vedere. Ed essi, accertatisi, risposero: Cinque, e due pesci.
\par 39 Allora egli comandò loro di farli accomodar tutti a brigate sull'erba verde;
\par 40 e si assisero per gruppi di cento e di cinquanta.
\par 41 Poi Gesù prese i cinque pani e i due pesci, e levati gli occhi al cielo, benedisse e spezzò i pani, e li dava ai discepoli, affinché li mettessero dinanzi alla gente; e i due pesci spartì pure fra tutti.
\par 42 E tutti mangiarono e furon sazî;
\par 43 e si portaron via dodici ceste piene di pezzi di pane, ed anche i resti dei pesci.
\par 44 E quelli che avean mangiato i pani erano cinquemila uomini.
\par 45 Subito dopo Gesù obbligò i suoi discepoli a montar nella barca e a precederlo sull'altra riva, verso Betsaida, mentre egli licenzierebbe la moltitudine.
\par 46 E preso commiato, se ne andò sul monte a pregare.
\par 47 E fattosi sera, la barca era in mezzo al mare ed egli era solo a terra.
\par 48 E vedendoli che si affannavano a remare perché il vento era loro contrario, verso la quarta vigilia della notte, andò alla loro volta, camminando sul mare; e voleva oltrepassarli;
\par 49 ma essi, vedutolo camminar sul mare, pensarono che fosse un fantasma e si dettero a gridare;
\par 50 perché tutti lo videro e ne furono sconvolti. Ma egli subito parlò loro e disse: State di buon cuore, son io; non temete!
\par 51 E montò nella barca con loro, e il vento s'acquetò; ed essi più che mai sbigottirono in loro stessi,
\par 52 perché non avean capito il fatto de' pani, anzi il cuor loro era indurito.
\par 53 Passati all'altra riva, vennero a Gennesaret e vi presero terra.
\par 54 E come furon sbarcati, subito la gente, riconosciutolo,
\par 55 corse per tutto il paese e cominciarono a portare qua e là i malati sui loro lettucci, dovunque sentivano dire ch'egli si trovasse.
\par 56 E da per tutto dov'egli entrava, ne' villaggi, nelle città, e nelle campagne, posavano gl'infermi per le piazze e lo pregavano che li lasciasse toccare non foss'altro che il lembo del suo vestito. E tutti quelli che lo toccavano, erano guariti.

\chapter{7}

\par 1 Allora si radunarono presso di lui i Farisei ed alcuni degli scribi venuti da Gerusalemme.
\par 2 E videro che alcuni de' suoi discepoli prendevano cibo con mani impure, cioè non lavate.
\par 3 Poiché i Farisei e tutti i Giudei non mangiano se non si sono con gran cura lavate le mani, attenendosi alla tradizione degli antichi;
\par 4 e quando tornano dalla piazza non mangiano se non si sono purificati con delle aspersioni. E vi sono molte altre cose che ritengono per tradizione: lavature di calici, d'orciuoli e di vasi di rame.
\par 5 E i Farisei e gli scribi gli domandarono: Perché i tuoi discepoli non seguono essi la tradizione degli antichi, ma prendon cibo con mani impure?
\par 6 Ma Gesù disse loro: Ben profetò Isaia di voi ipocriti, com'è scritto: Questo popolo mi onora con le labbra, ma il cuor loro è lontano da me.
\par 7 Ma invano mi rendono il loro culto insegnando dottrine che son precetti d'uomini.
\par 8 Voi, lasciato il comandamento di Dio, state attaccati alla tradizione degli uomini.
\par 9 E diceva loro ancora: Come ben sapete annullare il comandamento di Dio per osservare la tradizione vostra!
\par 10 Mosè infatti ha detto: Onora tuo padre e tua madre; e: Chi maledice padre o madre sia punito di morte;
\par 11 voi, invece, se uno dice a suo padre od a sua madre: Quello con cui potrei assisterti è Corban, (vale a dire, offerta a Dio),
\par 12 non gli permettete più di far cosa alcuna a pro di suo padre o di sua madre;
\par 13 annullando così la parola di Dio con la tradizione che voi vi siete tramandata. E di cose consimili ne fate tante!
\par 14 Poi, chiamata a sé di nuovo la moltitudine, diceva loro: Ascoltatemi tutti ed intendete:
\par 15 Non v'è nulla fuori dell'uomo che entrando in lui possa contaminarlo; ma son le cose che escon dall'uomo quelle che contaminano l'uomo.
\par 16 tex
\par 17 E quando, lasciata la moltitudine, fu entrato in casa, i suoi discepoli lo interrogarono intorno alla parabola.
\par 18 Ed egli disse loro: Siete anche voi così privi d'intendimento? Non capite voi che tutto ciò che dal di fuori entra nell'uomo non lo può contaminare,
\par 19 perché gli entra non nel cuore ma nel ventre e se ne va nella latrina? Così dicendo, dichiarava puri tutti quanti i cibi.
\par 20 Diceva inoltre: È quel che esce dall'uomo che contamina l'uomo;
\par 21 poiché è dal di dentro, dal cuore degli uomini, che escono cattivi pensieri, fornicazioni, furti, omicidî,
\par 22 adulterî, cupidigie, malvagità, frode, lascivia, sguardo maligno, calunnia, superbia, stoltezza.
\par 23 Tutte queste cose malvage escono dal di dentro e contaminano l'uomo.
\par 24 Poi, partitosi di là, se ne andò verso i confini di Tiro. Ed entrato in una casa, non voleva che alcuno lo sapesse; ma non poté restar nascosto,
\par 25 ché anzi, subito, una donna la cui figliuolina aveva uno spirito immondo, avendo udito parlar di lui, venne e gli si gettò ai piedi.
\par 26 Quella donna era pagana, di nazione sirofenicia; e lo pregava di cacciare il demonio dalla sua figliuola.
\par 27 Ma Gesù le disse: Lascia che prima siano saziati i figliuoli; ché non è bene prendere il pan dei figliuoli per buttarlo a' cagnolini.
\par 28 Ma ella gli rispose: Dici bene, Signore; e i cagnolini, sotto la tavola, mangiano de' minuzzoli dei figliuoli.
\par 29 E Gesù le disse: Per cotesta parola, va'; il demonio è uscito dalla tua figliuola.
\par 30 E la donna, tornata a casa sua, trovò la figliuolina coricata sul letto e il demonio uscito di lei.
\par 31 Partitosi di nuovo dai confini di Tiro, Gesù, passando per Sidone, tornò verso il mare di Galilea traversando il territorio della Decapoli.
\par 32 E gli menarono un sordo che parlava a stento; e lo pregarono che gl'imponesse la mano.
\par 33 Ed egli, trattolo in disparte fuor dalla folla, gli mise le dita negli orecchi e con la saliva gli toccò la lingua;
\par 34 poi, levati gli occhi al cielo, sospirò e gli disse: Effathà! che vuol dire: Apriti!
\par 35 E gli si aprirono gli orecchi; e subito gli si sciolse lo scilinguagnolo e parlava bene.
\par 36 E Gesù ordinò loro di non parlarne ad alcuno; ma più lo divietava loro e più lo divulgavano;
\par 37 e stupivano oltremodo, dicendo: Egli ha fatto ogni cosa bene; i sordi li fa udire, e i mutoli li fa parlare.

\chapter{8}

\par 1 In que' giorni, essendo di nuovo la folla grandissima, e non avendo ella da mangiare, Gesù, chiamati a sé i discepoli, disse loro:
\par 2 Io ho pietà di questa moltitudine; poiché già da tre giorni sta con me e non ha da mangiare.
\par 3 E se li rimando a casa digiuni, verranno meno per via; e ve n'hanno alcuni che son venuti da lontano.
\par 4 E i suoi discepoli gli risposero: Come si potrebbe mai saziarli di pane qui, in un deserto?
\par 5 Ed egli domandò loro: Quanti pani avete? Essi dissero: Sette.
\par 6 Ed egli ordinò alla folla di accomodarsi per terra; e presi i sette pani, dopo aver rese grazie, li spezzò e diede ai discepoli perché li ponessero dinanzi alla folla; ed essi li posero.
\par 7 Avevano anche alcuni pochi pescetti; ed egli, fatta la benedizione, comandò di porre anche quelli dinanzi a loro.
\par 8 E mangiarono e furon saziati; e de' pezzi avanzati si levarono sette panieri.
\par 9 Or erano circa quattromila persone. Poi Gesù li licenziò;
\par 10 e subito, montato nella barca co' suoi discepoli, andò dalle parti di Dalmanuta.
\par 11 E i Farisei si recaron colà e si misero a disputar con lui, chiedendogli, per metterlo alla prova, un segno dal cielo.
\par 12 Ma egli, dopo aver sospirato nel suo spirito, disse: Perché questa generazione chiede ella un segno? In verità io vi dico: Non sarà dato alcun segno a questa generazione.
\par 13 E lasciatili, montò di nuovo nella barca e passò all'altra riva.
\par 14 Or i discepoli aveano dimenticato di prendere dei pani, e non aveano seco nella barca che un pane solo.
\par 15 Ed egli dava loro de' precetti dicendo: Badate, guardatevi dal lievito de' Farisei e dal lievito d'Erode!
\par 16 Ed essi si dicevano gli uni agli altri: Egli è perché non abbiam pane.
\par 17 E Gesù, accortosene, disse loro: Perché ragionate voi del non aver pane? Non riflettete e non capite voi ancora? Avete il cuore indurito?
\par 18 Avendo occhi non vedete? e avendo orecchie non udite? e non avete memoria alcuna?
\par 19 Quand'io spezzai i cinque pani per i cinquemila, quante ceste piene di pezzi levaste? Essi dissero: Dodici.
\par 20 E quando spezzai i sette pani per i quattromila, quanti panieri pieni di pezzi levaste?
\par 21 Ed essi risposero: Sette. E diceva loro: Non capite ancora?
\par 22 E vennero in Betsaida; e gli fu menato un cieco, e lo pregarono che lo toccasse.
\par 23 Ed egli, preso il cieco per la mano, lo condusse fuor del villaggio; e sputatogli negli occhi e impostegli le mani, gli domandò:
\par 24 Vedi tu qualche cosa? Ed egli, levati gli occhi, disse: Scorgo gli uomini, perché li vedo camminare, e mi paion alberi.
\par 25 Poi Gesù gli mise di nuovo le mani sugli occhi; ed egli riguardò e fu guarito e vedeva ogni cosa chiaramente.
\par 26 E Gesù lo rimandò a casa sua e gli disse: Non entrar neppure nel villaggio.
\par 27 Poi Gesù, co' suoi discepoli, se ne andò verso le borgate di Cesarea di Filippo; e cammin facendo domandò ai suoi discepoli: Chi dice la gente ch'io sia?
\par 28 Ed essi risposero: Gli uni, Giovanni Battista: altri, Elia; ed altri, uno de' profeti.
\par 29 Ed egli domandò loro: E voi, chi dite ch'io sia? E Pietro gli rispose: Tu sei il Cristo.
\par 30 Ed egli vietò loro severamente di dir ciò di lui ad alcuno.
\par 31 Poi cominciò ad insegnar loro ch'era necessario che il Figliuol dell'uomo soffrisse molte cose, e fosse reietto dagli anziani e dai capi sacerdoti e dagli scribi, e fosse ucciso, e in capo a tre giorni risuscitasse.
\par 32 E diceva queste cose apertamente. E Pietro, trattolo da parte, prese a rimproverarlo.
\par 33 Ma egli, rivoltosi e guardati i suoi discepoli, rimproverò Pietro dicendo: Vattene via da me, Satana! Tu non hai il senso delle cose di Dio, ma delle cose degli uomini.
\par 34 E chiamata a sé la folla coi suoi discepoli, disse loro: Se uno vuol venir dietro a me, rinunzi a se stesso e prenda la sua croce e mi segua.
\par 35 Perché chi vorrà salvare la sua vita, la perderà; ma chi perderà la sua vita per amor di me e del Vangelo, la salverà.
\par 36 E che giova egli all'uomo se guadagna tutto il mondo e perde l'anima sua?
\par 37 E infatti, che darebbe l'uomo in cambio dell'anima sua?
\par 38 Perché se uno si sarà vergognato di me e delle mie parole in questa generazione adultera e peccatrice, anche il Figliuol dell'uomo si vergognerà di lui quando sarà venuto nella gloria del Padre suo

\chapter{9}

\par 1 coi santi angeli. E diceva loro: In verità io vi dico che alcuni di coloro che son qui presenti non gusteranno la morte, finché non abbian visto il regno di Dio venuto con potenza.
\par 2 Sei giorni dopo, Gesù prese seco Pietro e Giacomo e Giovanni e li condusse soli, in disparte, sopra un alto monte.
\par 3 E fu trasfigurato in presenza loro; e i suoi vestiti divennero sfolgoranti, candidissimi, di un tal candore che niun lavator di panni sulla terra può dare.
\par 4 Ed apparve loro Elia con Mosè, i quali stavano conversando con Gesù.
\par 5 E Pietro rivoltosi a Gesù: Maestro, disse, egli è bene che stiamo qui; facciamo tre tende; una per te, una per Mosè ed una per Elia.
\par 6 Poiché non sapeva che cosa dire, perché erano stati presi da spavento.
\par 7 E venne una nuvola che li coperse della sua ombra; e dalla nuvola una voce: Questo è il mio diletto figliuolo; ascoltatelo.
\par 8 E ad un tratto guardatisi attorno, non videro più alcuno con loro, se non Gesù solo.
\par 9 Or come scendevano dal monte, egli ordinò loro di non raccontare ad alcuno le cose che aveano vedute, se non quando il Figliuol dell'uomo sarebbe risuscitato dai morti.
\par 10 Ed essi tennero in sé la cosa, domandandosi fra loro che cosa fosse quel risuscitare dai morti.
\par 11 Poi gli chiesero: Perché dicono gli scribi che prima deve venire Elia?
\par 12 Ed egli disse loro: Elia deve venir prima e ristabilire ogni cosa; e come mai è egli scritto del Figliuol dell'uomo che egli ha da patir molte cose e da essere sprezzato?
\par 13 Ma io vi dico che Elia è già venuto, ed anche gli hanno fatto quello che hanno voluto, com'è scritto di lui.
\par 14 E venuti ai discepoli, videro intorno a loro una gran folla, e degli scribi che discutevan con loro.
\par 15 E subito tutta la folla, veduto Gesù, sbigottì e accorse a salutarlo.
\par 16 Ed egli domandò loro: Di che discutete voi con loro?
\par 17 E uno della folla gli rispose: Maestro, io t'ho menato il mio figliuolo che ha uno spirito mutolo;
\par 18 e dovunque esso lo prende, lo atterra; ed egli schiuma, stride dei denti e rimane stecchito. Ho detto a' tuoi discepoli che lo cacciassero, ma non hanno potuto.
\par 19 E Gesù, rispondendo, disse loro: O generazione incredula! Fino a quando sarò io con voi? Fino a quando vi sopporterò? Menatemelo.
\par 20 E glielo menarono; e come vide Gesù, subito lo spirito lo torse in convulsione; e caduto in terra, si rotolava schiumando. E Gesù domandò al padre:
\par 21 Da quanto tempo gli avviene questo? Ed egli disse:
\par 22 Dalla sua infanzia e spesse volte l'ha gettato anche nel fuoco e nell'acqua per farlo perire; ma tu, se ci puoi qualcosa, abbi pietà di noi ed aiutaci.
\par 23 E Gesù: Dici: Se puoi?! Ogni cosa è possibile a chi crede.
\par 24 E subito il padre del fanciullo esclamò: Io credo; sovvieni alla mia incredulità.
\par 25 E Gesù, vedendo che la folla accorreva, sgridò lo spirito immondo, dicendogli: Spirito muto e sordo, io tel comando, esci da lui e non entrar più in lui.
\par 26 E lo spirito, gridando e straziandolo forte, uscì; e il fanciullo rimase come morto; talché quasi tutti dicevano: È morto.
\par 27 Ma Gesù lo sollevò, ed egli si rizzò in piè.
\par 28 E quando Gesù fu entrato in casa, i suoi discepoli gli domandarono in privato: Perché non abbiam potuto cacciarlo noi?
\par 29 Ed egli disse loro: Cotesta specie di spiriti non si può far uscire in altro modo che con la preghiera.
\par 30 Poi, essendosi partiti di là, traversarono la Galilea; e Gesù non voleva che alcuno lo sapesse.
\par 31 Poich'egli ammaestrava i suoi discepoli, e diceva loro: Il Figliuol dell'uomo sta per esser dato nelle mani degli uomini ed essi l'uccideranno; e tre giorni dopo essere stato ucciso, risusciterà.
\par 32 Ma essi non intendevano il suo dire e temevano d'interrogarlo.
\par 33 E vennero a Capernaum; e quand'egli fu in casa, domandò loro: Di che discorrevate per via?
\par 34 Ed essi tacevano, perché per via aveano questionato fra loro chi fosse il maggiore.
\par 35 Ed egli postosi a sedere, chiamò i dodici e disse loro: Se alcuno vuol essere il primo, dovrà essere l'ultimo di tutti e il servitor di tutti.
\par 36 E preso un piccolo fanciullo, lo pose in mezzo a loro; e recatoselo in braccio, disse loro:
\par 37 Chiunque riceve uno di tali piccoli fanciulli nel nome mio, riceve me; e chiunque riceve me, non riceve me, ma colui che mi ha mandato.
\par 38 Giovanni gli disse: Maestro, noi abbiam veduto uno che cacciava i demonî nel nome tuo, il quale non ci seguita; e glielo abbiamo vietato perché non ci seguitava.
\par 39 Ma Gesù disse: Non glielo vietate, poiché non v'è alcuno che faccia qualche opera potente nel mio nome, e che subito dopo possa dir male di me.
\par 40 Poiché chi non è contro a noi, è per noi.
\par 41 Perché chiunque vi avrà dato a bere un bicchier d'acqua in nome mio perché siete di Cristo, in verità vi dico che non perderà punto il suo premio.
\par 42 E chiunque avrà scandalizzato uno di questi piccoli che credono, meglio sarebbe per lui che gli fosse messa al collo una macina da mulino, e fosse gettato in mare.
\par 43 E se la tua mano ti fa intoppare, mozzala; meglio è per te entrar monco nella vita, che aver due mani e andartene nella geenna, nel fuoco inestinguibile.
\par 44 Ver
\par 45 E se il tuo piede ti fa intoppare, mozzalo; meglio è per te entrar zoppo nella vita, che aver due piedi ed esser gittato nella geenna.
\par 46 Cha
\par 47 E se l'occhio tuo ti fa intoppare, cavalo; meglio è per te entrar con un occhio solo nel regno di Dio, che aver due occhi ed esser gittato nella geenna,
\par 48 dove il verme loro non muore ed il fuoco non si spegne.
\par 49 Poiché ognuno sarà salato con fuoco.
\par 50 Il sale è buono; ma se il sale diventa insipido, con che gli darete sapore?
\par 51 Abbiate del sale in voi stessi e state in pace gli uni con gli altri.

\chapter{10}

\par 1 Poi, levatosi di là, se ne andò sui confini della Giudea, ed oltre il Giordano; e di nuovo si raunarono presso a lui delle turbe; ed egli di nuovo, come soleva, le ammaestrava.
\par 2 E de' Farisei, accostatisi, gli domandarono, tentandolo: È egli lecito ad un marito di mandar via la moglie?
\par 3 Ed egli rispose loro: Mosè che v'ha egli comandato?
\par 4 Ed essi dissero: Mosè permise di scrivere un atto di divorzio e di mandarla via.
\par 5 E Gesù disse loro: È per la durezza del vostro cuore ch'egli scrisse per voi quel precetto;
\par 6 ma al principio della creazione Iddio li fece maschio e femmina.
\par 7 Perciò l'uomo lascerà suo padre e sua madre, e i due saranno una sola carne.
\par 8 Talché non sono più due, ma una stessa carne.
\par 9 Quello dunque che Iddio ha congiunto l'uomo non separi.
\par 10 E in casa i discepoli lo interrogarono di nuovo sullo stesso soggetto.
\par 11 Ed egli disse loro: Chiunque manda via sua moglie e ne sposa un'altra, commette adulterio verso di lei;
\par 12 e se la moglie, ripudiato il marito, ne sposa un altro, commette adulterio.
\par 13 Or gli presentavano dei bambini perché li toccasse; ma i discepoli sgridavan coloro che glieli presentavano.
\par 14 E Gesù, veduto ciò, s'indignò e disse loro: Lasciate i piccoli fanciulli venire a me; non glielo vietate, perché di tali è il regno di Dio.
\par 15 In verità io vi dico che chiunque non avrà ricevuto il regno di Dio come un piccolo fanciullo, non entrerà punto in esso.
\par 16 E presili in braccio ed imposte loro le mani, li benediceva.
\par 17 Or com'egli usciva per mettersi in cammino, un tale accorse e inginocchiatosi davanti a lui, gli domandò: Maestro buono, che farò io per ereditare la vita eterna?
\par 18 E Gesù gli disse: Perché mi chiami buono? Nessuno è buono, tranne uno solo, cioè Iddio.
\par 19 Tu sai i comandamenti: Non uccidere; non commettere adulterio; non rubare; non dir falsa testimonianza; non far torto ad alcuno; onora tuo padre e tua madre.
\par 20 Ed egli rispose: Maestro, tutte queste cose io le ho osservate fin dalla mia giovinezza.
\par 21 E Gesù, riguardatolo in viso, l'amò e gli disse: Una cosa ti manca; va', vendi tutto ciò che hai, e dàllo ai poveri, e tu avrai un tesoro nel cielo; poi vieni e seguimi.
\par 22 Ma egli, attristato da quella parola, se ne andò dolente, perché avea di gran beni.
\par 23 E Gesù, guardatosi attorno, disse ai suoi discepoli: Quanto malagevolmente coloro che hanno delle ricchezze entreranno nel regno di Dio!
\par 24 E i discepoli sbigottirono a queste sue parole. E Gesù da capo replicò loro: Figliuoli, quant'è malagevole a coloro che si confidano nelle ricchezze entrare nel regno di Dio!
\par 25 È più facile a un cammello passare per la cruna d'un ago, che ad un ricco entrare nel regno di Dio.
\par 26 Ed essi vie più stupivano, dicendo fra loro: Chi dunque può esser salvato?
\par 27 E Gesù, riguardatili, disse: Agli uomini è impossibile, ma non a Dio; perché tutto è possibile a Dio.
\par 28 E Pietro prese a dirgli: Ecco, noi abbiamo lasciato ogni cosa e t'abbiam seguitato.
\par 29 E Gesù rispose: Io vi dico in verità che non v'è alcuno che abbia lasciato casa, o fratelli, o sorelle, o madre, o padre, o figliuoli, o campi, per amor di me e per amor dell'evangelo,
\par 30 il quale ora, in questo tempo, non ne riceva cento volte tanto: case, fratelli, sorelle, madri, figliuoli, campi, insieme a persecuzioni; e nel secolo avvenire, la vita eterna.
\par 31 Ma molti primi saranno ultimi e molti ultimi, primi.
\par 32 Or erano per cammino salendo a Gerusalemme, e Gesù andava innanzi a loro; ed essi erano sbigottiti; e quelli che lo seguivano eran presi da timore. Ed egli, tratti di nuovo da parte i dodici, prese a dir loro le cose che gli avverrebbero:
\par 33 Ecco, noi saliamo a Gerusalemme, e il Figliuol dell'uomo sarà dato nelle mani de' capi sacerdoti e degli scribi; ed essi lo condanneranno a morte e lo metteranno nelle mani dei Gentili;
\par 34 e lo scherniranno e gli sputeranno addosso e lo flagelleranno e l'uccideranno; e dopo tre giorni egli risusciterà.
\par 35 E Giacomo e Giovanni, figliuoli di Zebedeo, si accostarono a lui, dicendogli: Maestro, desideriamo che tu ci faccia quello che ti chiederemo.
\par 36 Ed egli disse loro: Che volete ch'io vi faccia?
\par 37 Essi gli dissero: Concedici di sedere uno alla tua destra e l'altro alla tua sinistra nella tua gloria. Ma Gesù disse loro:
\par 38 Voi non sapete quel che chiedete. Potete voi bere il calice ch'io bevo, o esser battezzati del battesimo del quale io son battezzato? Essi gli dissero: Sì, lo possiamo.
\par 39 E Gesù disse loro: Voi certo berrete il calice ch'io bevo e sarete battezzati del battesimo del quale io sono battezzato;
\par 40 ma quant'è al sedermi a destra o a sinistra, non sta a me il darlo, ma è per quelli cui è stato preparato.
\par 41 E i dieci, udito ciò, presero a indignarsi di Giacomo e di Giovanni.
\par 42 Ma Gesù, chiamatili a sé, disse loro: Voi sapete che quelli che son reputati principi delle nazioni, le signoreggiano; e che i loro grandi usano potestà sopra di esse.
\par 43 Ma non è così tra voi; anzi chiunque vorrà esser grande fra voi, sarà vostro servitore;
\par 44 e chiunque fra voi vorrà esser primo, sarà servo di tutti.
\par 45 Poiché anche il Figliuol dell'uomo non è venuto per esser servito, ma per servire, e per dar la vita sua come prezzo di riscatto per molti.
\par 46 Poi vennero in Gerico. E come egli usciva di Gerico coi suoi discepoli e con gran moltitudine, il figliuol di Timeo, Bartimeo, cieco mendicante, sedeva presso la strada.
\par 47 E udito che chi passava era Gesù il Nazareno, prese a gridare e a dire: Gesù, Figliuol di Davide, abbi pietà di me!
\par 48 E molti lo sgridavano perché tacesse; ma quello gridava più forte: Figliuol di Davide, abbi pietà di me!
\par 49 E Gesù, fermatosi, disse: Chiamatelo! E chiamarono il cieco, dicendogli: Sta' di buon cuore! Alzati! Egli ti chiama.
\par 50 E il cieco, gettato via il mantello, balzò in piedi e venne a Gesù.
\par 51 E Gesù, rivoltosi a lui, gli disse: Che vuoi tu ch'io ti faccia? E il cieco gli rispose: Rabbuni, ch'io ricuperi la vista.
\par 52 E Gesù gli disse: Va', la tua fede ti ha salvato. E in quell'istante egli ricuperò la vista e seguiva Gesù per la via.

\chapter{11}

\par 1 E quando furon giunti vicino a Gerusalemme, a Betfage e Betania, presso al monte degli Ulivi, Gesù mandò due dei suoi discepoli, e disse loro:
\par 2 Andate nella borgata che è dirimpetto a voi; e subito, appena entrati, troverete legato un puledro d'asino, sopra il quale non è montato ancora alcuno; scioglietelo e menatemelo.
\par 3 E se qualcuno vi dice: Perché fate questo? rispondete: Il Signore ne ha bisogno, e lo rimanderà subito qua.
\par 4 Ed essi andarono e trovarono un puledro legato ad una porta, fuori, sulla strada, e lo sciolsero.
\par 5 Ed alcuni di coloro ch'eran lì presenti, dissero loro: Che fate, che sciogliete il puledro?
\par 6 Ed essi risposero come Gesù aveva detto. E quelli li lasciaron fare.
\par 7 Ed essi menarono il puledro a Gesù, e gettarono su quello i loro mantelli, ed egli vi montò sopra.
\par 8 E molti stendevano i loro mantelli sulla via; ed altri, delle fronde che avean tagliate nei campi.
\par 9 E coloro che andavano avanti e coloro che venivano dietro, gridavano: Osanna! Benedetto colui che viene nel nome del Signore!
\par 10 Benedetto il regno che viene, il regno di Davide nostro padre! Osanna ne' luoghi altissimi!
\par 11 E Gesù entrò in Gerusalemme, nel tempio; e avendo riguardata ogni cosa attorno attorno, essendo già l'ora tarda, uscì per andare a Betania coi dodici.
\par 12 E il giorno seguente, quando furon usciti da Betania, egli ebbe fame.
\par 13 E veduto di lontano un fico che avea delle foglie, andò a vedere se per caso vi trovasse qualche cosa; ma venuto al fico non vi trovò nient'altro che foglie; perché non era la stagion dei fichi.
\par 14 E Gesù prese a dire al fico: Niuno mangi mai più in perpetuo frutto da te! E i suoi discepoli udirono.
\par 15 E vennero a Gerusalemme; e Gesù, entrato nel tempio, prese a cacciarne coloro che vendevano e che compravano nel tempio; e rovesciò le tavole de' cambiamonete e le sedie de' venditori di colombi;
\par 16 e non permetteva che alcuno portasse oggetti attraverso il tempio.
\par 17 Ed insegnava, dicendo loro: Non è egli scritto: La mia casa sarà chiamata casa d'orazione per tutte le genti? ma voi ne avete fatta una spelonca di ladroni.
\par 18 Ed i capi sacerdoti e gli scribi udirono queste cose e cercavano il modo di farlo morire, perché lo temevano; poiché tutta la moltitudine era rapita in ammirazione della sua dottrina.
\par 19 E quando fu sera, uscirono dalla città.
\par 20 E la mattina, passando, videro il fico seccato fin dalle radici;
\par 21 e Pietro, ricordatosi, gli disse: Maestro, vedi, il fico che tu maledicesti, è seccato.
\par 22 E Gesù, rispondendo, disse loro: Abbiate fede in Dio!
\par 23 In verità io vi dico che chi dirà a questo monte: Togliti di là e gettati nel mare, se non dubita in cuor suo, ma crede che quel che dice avverrà, gli sarà fatto.
\par 24 Perciò vi dico: Tutte le cose che voi domanderete pregando, crediate che le avete ricevute, e voi le otterrete.
\par 25 E quando vi mettete a pregare, se avete qualcosa contro a qualcuno, perdonate; affinché il Padre vostro che è nei cieli, vi perdoni i vostri falli.
\par 26 tex
\par 27 Poi vennero di nuovo in Gerusalemme; e mentr'egli passeggiava per il tempio, i capi sacerdoti e gli scribi e gli anziani s'accostarono a lui e gli dissero:
\par 28 Con quale autorità fai tu queste cose? O chi ti ha data codesta autorità di far queste cose?
\par 29 E Gesù disse loro: Io vi domanderò una cosa; rispondetemi e vi dirò con quale autorità io faccio queste cose.
\par 30 Il battesimo di Giovanni era esso dal cielo o dagli uomini? Rispondetemi.
\par 31 Ed essi ragionavan fra loro dicendo: Se diciamo: Dal cielo, egli dirà: Perché dunque non gli credeste?
\par 32 Diremo invece: Dagli uomini?... Essi temevano il popolo, perché tutti stimavano che Giovanni fosse veramente profeta.
\par 33 E risposero a Gesù: Non lo sappiamo. E Gesù disse loro: E neppur io vi dico con quale autorità fo queste cose.

\chapter{12}

\par 1 E prese a dir loro in parabole: Un uomo piantò una vigna e le fece attorno una siepe e vi scavò un luogo da spremer l'uva e vi edificò una torre; l'allogò a de' lavoratori, e se ne andò in viaggio.
\par 2 E a suo tempo mandò a que' lavoratori un servitore per ricevere da loro de' frutti della vigna.
\par 3 Ma essi, presolo, lo batterono e lo rimandarono a vuoto.
\par 4 Ed egli di nuovo mandò loro un altro servitore; e anche lui ferirono nel capo e vituperarono.
\par 5 Ed egli ne mandò un altro, e anche quello uccisero; e poi molti altri, de' quali alcuni batterono ed alcuni uccisero.
\par 6 Aveva ancora un unico figliuolo diletto; e quello mandò loro per ultimo, dicendo: Avranno rispetto al mio figliuolo.
\par 7 Ma que' lavoratori dissero fra loro: Costui è l'erede; venite, uccidiamolo, e l'eredità sarà nostra.
\par 8 E presolo, l'uccisero, e lo gettarono fuor della vigna.
\par 9 Che farà dunque il padrone della vigna? Egli verrà e distruggerà que' lavoratori, e darà la vigna ad altri.
\par 10 Non avete voi neppur letta questa scrittura: La pietra che gli edificatori hanno riprovata, è quella che è divenuta pietra angolare;
\par 11 ciò è stato fatto dal Signore, ed è cosa maravigliosa agli occhi nostri?
\par 12 Ed essi cercavano di pigliarlo, ma temettero la moltitudine; perché si avvidero bene ch'egli avea detto quella parabola per loro. E lasciatolo, se ne andarono.
\par 13 E gli mandarono alcuni dei Farisei e degli Erodiani per coglierlo in parole.
\par 14 Ed essi, venuti, gli dissero: Maestro, noi sappiamo che tu sei verace, e che non ti curi d'alcuno, perché non guardi all'apparenza delle persone, ma insegni la via di Dio secondo verità. È egli lecito pagare il tributo a Cesare o no? Dobbiamo darlo o non darlo?
\par 15 Ma egli, conosciuta la loro ipocrisia, disse loro: Perché mi tentate? Portatemi un denaro, ch'io lo vegga.
\par 16 Ed essi glielo portarono. Ed egli disse loro: Di chi è questa effigie e questa iscrizione? Essi gli dissero:
\par 17 Di Cesare. Allora Gesù disse loro: Rendete a Cesare quel ch'è di Cesare, e a Dio quel ch'è di Dio. Ed essi si maravigliarono di lui.
\par 18 Poi vennero a lui de' Sadducei, i quali dicono che non v'è risurrezione, e gli domandarono:
\par 19 Maestro, Mosè ci lasciò scritto che se il fratello di uno muore e lascia moglie senza figliuoli, il fratello ne prenda la moglie e susciti progenie a suo fratello.
\par 20 Or v'erano sette fratelli. Il primo prese moglie; e morendo, non lasciò progenie.
\par 21 E il secondo la prese e morì senza lasciar progenie.
\par 22 Così il terzo. E i sette non lasciaron progenie. Infine, dopo tutti, morì anche la donna.
\par 23 Nella risurrezione, quando saranno risuscitati, di chi di loro sarà ella moglie? Poiché tutti e sette l'hanno avuta per moglie.
\par 24 Gesù disse loro: Non errate voi per questo, che non conoscete le Scritture né la potenza di Dio?
\par 25 Poiché quando gli uomini risuscitano da' morti, né prendono né danno moglie, ma son come angeli ne' cieli.
\par 26 Quanto poi ai morti ed alla loro risurrezione, non avete voi letto nel libro di Mosè, nel passo del "pruno", come Dio gli parlò dicendo: Io sono l'Iddio d'Abramo e l'Iddio d'Isacco e l'Iddio di Giacobbe?
\par 27 Egli non è un Dio di morti, ma di viventi. Voi errate grandemente.
\par 28 Or uno degli scribi che li aveva uditi discutere, visto ch'egli avea loro ben risposto, si accostò e gli domandò: Qual è il comandamento primo fra tutti?
\par 29 E Gesù rispose: Il primo è: Ascolta, Israele: Il Signore Iddio nostro è l'unico Signore:
\par 30 ama dunque il Signore Iddio tuo con tutto il tuo cuore e con tutta l'anima tua e con tutta la mente tua e con tutta la forza tua.
\par 31 Il secondo è questo: Ama il tuo prossimo come te stesso. Non v'è alcun altro comandamento maggiore di questi.
\par 32 E lo scriba gli disse: Maestro, ben hai detto secondo verità che v'è un Dio solo e che fuor di lui non ve n'è alcun altro;
\par 33 e che amarlo con tutto il cuore, con tutto l'intelletto e con tutta la forza e amare il prossimo come se stesso, è assai più che tutti gli olocausti e i sacrifici.
\par 34 E Gesù, vedendo ch'egli avea risposto avvedutamente, gli disse: Tu non sei lontano dal regno di Dio. E niuno ardiva più interrogarlo.
\par 35 E Gesù, insegnando nel tempio, prese a dire: Come dicono gli scribi che il Cristo è figliuolo di Davide?
\par 36 Davide stesso ha detto, per lo Spirito Santo: Il Signore ha detto al mio Signore: Siedi alla mia destra, finché io abbia posto i tuoi nemici per sgabello dei tuoi piedi.
\par 37 Davide stesso lo chiama Signore; e onde viene ch'egli è suo figliuolo? E la massa del popolo l'ascoltava con piacere.
\par 38 E diceva nel suo insegnamento: Guardatevi dagli scribi, i quali amano passeggiare in lunghe vesti, ed esser salutati nelle piazze,
\par 39 ed avere i primi seggi nelle sinagoghe e i primi posti ne' conviti;
\par 40 essi che divorano le case delle vedove, e fanno per apparenza lunghe orazioni. Costoro riceveranno una maggior condanna.
\par 41 E postosi a sedere dirimpetto alla cassa delle offerte, stava guardando come la gente gettava danaro nella cassa; e molti ricchi ne gettavano assai.
\par 42 E venuta una povera vedova, vi gettò due spiccioli che fanno un quarto di soldo.
\par 43 E Gesù, chiamati a sé i suoi discepoli, disse loro: In verità io vi dico che questa povera vedova ha gettato nella cassa delle offerte più di tutti gli altri;
\par 44 poiché tutti han gettato del loro superfluo; ma costei, del suo necessario, vi ha gettato tutto ciò che possedeva, tutto quanto avea per vivere.

\chapter{13}

\par 1 E com'egli usciva dal tempio uno de' suoi discepoli gli disse: Maestro, guarda che pietre e che edifizî!
\par 2 E Gesù gli disse: Vedi tu questi grandi edifizî? Non sarà lasciata pietra sopra pietra che non sia diroccata.
\par 3 Poi sedendo egli sul monte degli Ulivi dirimpetto al tempio, Pietro e Giacomo e Giovanni e Andrea gli domandarono in disparte:
\par 4 Dicci, quando avverranno queste cose, e qual sarà il segno del tempo in cui tutte queste cose staranno per compiersi?
\par 5 E Gesù prese a dir loro: Guardate che nessuno vi seduca!
\par 6 Molti verranno sotto il mio nome, dicendo: Son io; e ne sedurranno molti.
\par 7 Or quando udrete guerre e rumori di guerre, non vi turbate; è necessario che ciò avvenga, ma non sarà ancora la fine.
\par 8 Poiché si leverà nazione contro nazione e regno contro regno: vi saranno terremoti in vari luoghi; vi saranno carestie. Questo non sarà che un principio di dolori.
\par 9 Or badate a voi stessi! Vi daranno in man dei tribunali e sarete battuti nelle sinagoghe e sarete fatti comparire davanti a governatori e re, per cagion mia, affinché ciò serva loro di testimonianza.
\par 10 E prima convien che fra tutte le genti sia predicato l'evangelo.
\par 11 E quando vi meneranno per mettervi nelle loro mani, non state innanzi in sollecitudine di ciò che avrete a dire, ma dite quel che vi sarà dato in quell'ora; perché non siete voi che parlate, ma lo Spirito Santo.
\par 12 E il fratello darà il fratello alla morte, e il padre il figliuolo; e i figliuoli si leveranno contro i genitori e li faranno morire.
\par 13 E sarete odiati da tutti a cagion del mio nome; ma chi avrà sostenuto sino alla fine, sarà salvato.
\par 14 Quando poi avrete veduta l'abominazione della desolazione posta là dove non si conviene (chi legge pongavi mente), allora quelli che saranno nella Giudea, fuggano ai monti;
\par 15 e chi sarà sulla terrazza non scenda e non entri in casa sua per toglierne cosa alcuna;
\par 16 e chi sarà nel campo non torni indietro a prender la sua veste.
\par 17 Or guai alle donne che saranno incinte ed a quelle che allatteranno in que' giorni!
\par 18 E pregate che ciò non avvenga d'inverno!
\par 19 Poiché quelli saranno giorni di tale tribolazione, che non v'è stata l'uguale dal principio del mondo che Dio ha creato, fino ad ora, né mai più vi sarà.
\par 20 E se il Signore non avesse abbreviato que' giorni, nessuno scamperebbe; ma a cagion dei suoi propri eletti, egli ha abbreviato que' giorni.
\par 21 E allora, se alcuno vi dice: "Il Cristo eccolo qui, eccolo là", non lo credete;
\par 22 perché sorgeranno falsi cristi e falsi profeti, e faranno segni e prodigi per sedurre, se fosse possibile, anche gli eletti.
\par 23 Ma voi, state attenti; io v'ho predetta ogni cosa.
\par 24 Ma in que' giorni, dopo quella tribolazione, il sole si oscurerà e la luna non darà il suo splendore;
\par 25 e le stelle cadranno dal cielo e le potenze che son nei cieli saranno scrollate.
\par 26 E allora si vedrà il Figliuol dell'uomo venir sulle nuvole con gran potenza e gloria.
\par 27 Ed egli allora manderà gli angeli e raccoglierà i suoi eletti dai quattro venti, dall'estremo della terra all'estremo del cielo.
\par 28 Or imparate dal fico questa similitudine: Quando già i suoi rami si fanno teneri e metton le foglie, voi sapete che l'estate è vicina.
\par 29 Così anche voi, quando vedrete avvenir queste cose, sappiate che egli è vicino, alle porte.
\par 30 In verità io vi dico che questa generazione non passerà prima che tutte queste cose siano avvenute.
\par 31 Il cielo e la terra passeranno, ma le mie parole non passeranno.
\par 32 Ma quant'è a quel giorno ed a quell'ora, nessuno li sa, neppur gli angeli nel cielo, né il Figliuolo, ma solo il padre.
\par 33 State in guardia, vegliate, poiché non sapete quando sarà quel tempo.
\par 34 Egli è come se un uomo, andando in viaggio, lasciasse la sua casa e ne desse la potestà ai suoi servitori, a ciascuno il compito suo, e al portinaio comandasse di vegliare.
\par 35 Vegliate dunque perché non sapete quando viene il padron di casa: se a sera, o a mezzanotte, o al cantar del gallo o la mattina;
\par 36 che talora, venendo egli all'improvviso, non vi trovi addormentati.
\par 37 Ora, quel che dico a voi, lo dico a tutti: Vegliate.

\chapter{14}

\par 1 Ora, due giorni dopo, era la pasqua e gli azzimi; e i capi sacerdoti e gli scribi cercavano il modo di pigliar Gesù con inganno ed ucciderlo;
\par 2 perché dicevano: Non lo facciamo durante la festa, che talora non vi sia qualche tumulto del popolo.
\par 3 Ed essendo egli in Betania, nella casa di Simone il lebbroso, mentre era a tavola, venne una donna che aveva un alabastro d'olio odorifero di nardo schietto, di gran prezzo; e rotto l'alabastro, glielo versò sul capo.
\par 4 E alcuni, sdegnatisi, dicevano fra loro: Perché s'è fatta questa perdita dell'olio?
\par 5 Questo olio si sarebbe potuto vendere più di trecento danari e darli ai poveri. E fremevano contro a lei.
\par 6 Ma Gesù disse: Lasciatela stare! Perché le date noia? Ella ha fatto un'azione buona inverso me.
\par 7 Poiché i poveri li avete sempre con voi; e quando vogliate, potete far loro del bene; ma me non mi avete sempre.
\par 8 Ella ha fatto ciò che per lei si poteva; ha anticipato d'ungere il mio corpo per la sepoltura.
\par 9 E in verità io vi dico che per tutto il mondo, dovunque sarà predicato l'evangelo, anche quello che costei ha fatto sarà raccontato, in memoria di lei.
\par 10 E Giuda Iscariot, uno dei dodici, andò dai capi sacerdoti per darglielo nelle mani.
\par 11 Ed essi, uditolo, si rallegrarono e promisero di dargli del danaro. Ed egli cercava il modo opportuno di tradirlo.
\par 12 E il primo giorno degli azzimi, quando si sacrificava la pasqua, i suoi discepoli gli dissero: Dove vuoi che andiamo ad apparecchiarti da mangiar la pasqua?
\par 13 Ed egli mandò due dei suoi discepoli, e disse loro: Andate nella città, e vi verrà incontro un uomo che porterà una brocca d'acqua; seguitelo;
\par 14 e dove sarà entrato, dite al padron di casa: Il Maestro dice: Dov'è la mia stanza da mangiarvi la pasqua coi miei discepoli?
\par 15 Ed egli vi mostrerà di sopra una gran sala ammobiliata e pronta; quivi apparecchiate per noi.
\par 16 E i discepoli andarono e giunsero nella città e trovarono come egli avea lor detto, e apparecchiarono la pasqua.
\par 17 E quando fu sera Gesù venne co' dodici.
\par 18 E mentre erano a tavola e mangiavano, Gesù disse: In verità io vi dico che uno di voi, il quale mangia meco, mi tradirà.
\par 19 Essi cominciarono ad attristarsi e a dirgli ad uno ad uno: Sono io desso?
\par 20 Ed egli disse loro: È uno dei dodici, che intinge meco nel piatto.
\par 21 Certo il Figliuol dell'uomo se ne va, com'è scritto di lui; ma guai a quell'uomo per cui il Figliuol dell'uomo è tradito! Ben sarebbe per quell'uomo di non esser nato!
\par 22 E mentre mangiavano, Gesù prese del pane; e fatta la benedizione, lo ruppe e lo diede loro e disse: Prendete, questo è il mio corpo.
\par 23 Poi, preso un calice e rese grazie, lo diede loro, e tutti ne bevvero.
\par 24 E disse loro: Questo è il mio sangue, il sangue del patto, il quale è sparso per molti.
\par 25 In verità io vi dico che non berrò più del frutto della vigna fino a quel giorno che lo berrò nuovo nel regno di Dio.
\par 26 E dopo ch'ebbero cantato l'inno, uscirono per andare al monte degli Ulivi.
\par 27 E Gesù disse loro: Voi tutti sarete scandalizzati; perché è scritto: Io percoterò il pastore e le pecore saranno disperse.
\par 28 Ma dopo che sarò risuscitato, vi precederò in Galilea.
\par 29 Ma Pietro gli disse: Quand'anche tutti fossero scandalizzati, io però non lo sarò.
\par 30 E Gesù gli disse: In verità io ti dico che tu, oggi, in questa stessa notte, avanti che il gallo abbia cantato due volte, mi rinnegherai tre volte.
\par 31 Ma egli vie più fermamente diceva: Quantunque mi convenisse morir teco non però ti rinnegherò. E lo stesso dicevano pure tutti gli altri.
\par 32 Poi vennero in un podere detto Getsemani; ed egli disse ai suoi discepoli: Sedete qui finché io abbia pregato.
\par 33 E prese seco Pietro e Giacomo e Giovanni e cominciò ad essere spaventato ed angosciato.
\par 34 E disse loro: L'anima mia è oppressa da tristezza mortale; rimanete qui e vegliate.
\par 35 E andato un poco innanzi, si gettò a terra; e pregava che, se fosse possibile, quell'ora passasse oltre da lui.
\par 36 E diceva: Abba, Padre! ogni cosa ti è possibile; allontana da me questo calice! Ma pure, non quello che io voglio, ma quello che tu vuoi.
\par 37 E venne, e li trovò che dormivano, e disse a Pietro: Simone, dormi tu? non sei stato capace di vegliare un'ora sola?
\par 38 Vegliate e pregate, affinché non cadiate in tentazione; ben è lo spirito pronto, ma la carne è debole.
\par 39 E di nuovo andò e pregò, dicendo le medesime parole.
\par 40 E tornato di nuovo, li trovò che dormivano perché gli occhi loro erano aggravati; e non sapevano che rispondergli.
\par 41 E venne la terza volta, e disse loro: Dormite pure oramai, e riposatevi! Basta! L'ora è venuta: ecco, il Figliuol dell'uomo è dato nelle mani dei peccatori.
\par 42 Levatevi, andiamo; ecco, colui che mi tradisce, è vicino.
\par 43 E in quell'istante, mentr'egli parlava ancora, arrivò Giuda, l'uno dei dodici, e con lui una gran turba con ispade e bastoni, da parte de' capi sacerdoti, degli scribi e degli anziani.
\par 44 Or colui che lo tradiva, avea dato loro un segnale, dicendo: Colui che bacerò è desso; pigliatelo e menatelo via sicuramente.
\par 45 E come fu giunto, subito si accostò a lui e disse: Maestro! e lo baciò.
\par 46 Allora quelli gli misero le mani addosso e lo presero;
\par 47 ma uno di coloro ch'erano quivi presenti, tratta la spada, percosse il servitore del sommo sacerdote, e gli spiccò l'orecchio.
\par 48 E Gesù, rivolto a loro, disse: Voi siete usciti con ispade e bastoni come contro ad un ladrone per pigliarmi.
\par 49 Ogni giorno ero fra voi insegnando nel tempio, e voi non mi avete preso; ma ciò è avvenuto, affinché le Scritture fossero adempiute.
\par 50 E tutti, lasciatolo, se ne fuggirono.
\par 51 Ed un certo giovane lo seguiva, avvolto in un panno lino sul nudo; e lo presero;
\par 52 ma egli, lasciando andare il panno lino, se ne fuggì ignudo.
\par 53 E menarono Gesù al sommo sacerdote; e s'adunarono tutti i capi sacerdoti e gli anziani e gli scribi.
\par 54 E Pietro lo avea seguito da lungi, fin dentro la corte del sommo sacerdote, ove stava a sedere con le guardie e si scaldava al fuoco.
\par 55 Or i capi sacerdoti e tutto il Sinedrio cercavano qualche testimonianza contro a Gesù per farlo morire; e non ne trovavano alcuna.
\par 56 Poiché molti deponevano il falso contro a lui; ma le testimonianze non erano concordi.
\par 57 Ed alcuni, levatisi, testimoniarono falsamente contro a lui, dicendo:
\par 58 Noi l'abbiamo udito che diceva: Io disfarò questo tempio fatto di man d'uomo, e in tre giorni ne riedificherò un altro, che non sarà fatto di mano d'uomo.
\par 59 Ma neppur così la loro testimonianza era concorde.
\par 60 Allora il sommo sacerdote, levatosi in piè quivi in mezzo, domandò a Gesù: Non rispondi tu nulla? Che testimoniano costoro contro a te?
\par 61 Ma egli tacque e non rispose nulla. Daccapo il sommo sacerdote lo interrogò e gli disse: Sei tu il Cristo, il Figliuol del Benedetto?
\par 62 E Gesù disse: Sì, lo sono: e vedrete il Figliuol dell'uomo seduto alla destra della Potenza e venire sulle nuvole del cielo.
\par 63 Ed il sommo sacerdote, stracciatesi le vesti, disse: Che abbiam noi più bisogno di testimoni?
\par 64 Voi avete udita la bestemmia. Che ve ne pare? E tutti lo condannarono come reo di morte.
\par 65 Ed alcuni presero a sputargli addosso ed a velargli la faccia e a dargli dei pugni e a dirgli: Indovina, profeta! E le guardie presero a schiaffeggiarlo.
\par 66 Ed essendo Pietro giù nella corte, venne una delle serve del sommo sacerdote;
\par 67 e veduto Pietro che si scaldava, lo riguardò in viso e disse: Anche tu eri con Gesù Nazareno.
\par 68 Ma egli lo negò, dicendo: Io non so, né capisco quel che tu ti dica. Ed uscì fuori nell'antiporto, e il gallo cantò.
\par 69 E la serva, vedutolo, cominciò di nuovo a dire a quelli ch'eran quivi presenti: Costui è di quelli. Ma egli daccapo lo negò.
\par 70 E di nuovo, di lì a poco, quelli ch'erano quivi, dicevano a Pietro: Per certo tu sei di quelli, perché poi sei galileo.
\par 71 Ma egli prese ad imprecare ed a giurare: Non conosco quell'uomo che voi dite.
\par 72 E subito, per la seconda volta, il gallo cantò. E Pietro si ricordò della parola che Gesù gli aveva detta: Avanti che il gallo abbia cantato due volte, tu mi rinnegherai tre volte. Ed a questo pensiero si mise a piangere.

\chapter{15}

\par 1 E subito la mattina, i capi sacerdoti, con gli anziani e gli scribi e tutto il Sinedrio, tenuto consiglio, legarono Gesù e lo menarono via e lo misero in man di Pilato.
\par 2 E Pilato gli domandò: Sei tu il re dei Giudei? Ed egli, rispondendo, gli disse: Sì, lo sono.
\par 3 E i capi sacerdoti l'accusavano di molte cose;
\par 4 e Pilato daccapo lo interrogò dicendo: Non rispondi nulla? Vedi di quante cose ti accusano!
\par 5 Ma Gesù non rispose più nulla; talché Pilato se ne maravigliava.
\par 6 Or ogni festa di pasqua ei liberava loro un carcerato, qualunque chiedessero.
\par 7 C'era allora in prigione un tale chiamato Barabba, insieme a de' sediziosi, i quali, nella sedizione, avean commesso omicidio.
\par 8 E la moltitudine, venuta su, cominciò a domandare ch'e' facesse come sempre avea lor fatto.
\par 9 E Pilato rispose loro: Volete ch'io vi liberi il Re de' Giudei?
\par 10 Poiché capiva bene che i capi sacerdoti glielo aveano consegnato per invidia.
\par 11 Ma i capi sacerdoti incitarono la moltitudine a chiedere che piuttosto liberasse loro Barabba.
\par 12 E Pilato, daccapo replicando, diceva loro: Che volete dunque ch'io faccia di colui che voi chiamate il Re de' Giudei?
\par 13 Ed essi di nuovo gridarono: Crocifiggilo!
\par 14 E Pilato diceva loro: Ma pure, che male ha egli fatto? Ma essi gridarono più forte che mai: Crocifiggilo!
\par 15 E Pilato, volendo soddisfare la moltitudine, liberò loro Barabba; e consegnò Gesù, dopo averlo flagellato, per esser crocifisso.
\par 16 Allora i soldati lo menarono dentro la corte che è il pretorio, e radunarono tutta la coorte.
\par 17 E lo vestirono di porpora; e intrecciata una corona di spine, gliela misero intorno al capo,
\par 18 e cominciarono a salutarlo: Salve, Re de' Giudei!
\par 19 E gli percotevano il capo con una canna, e gli sputavano addosso, e postisi inginocchioni, si prostravano dinanzi a lui.
\par 20 E dopo che l'ebbero schernito, lo spogliarono della porpora e lo rivestirono dei suoi propri vestimenti. E lo menaron fuori per crocifiggerlo.
\par 21 E costrinsero a portar la croce di lui un certo Simon cireneo, il padre di Alessandro e di Rufo, il quale passava di là, tornando dai campi.
\par 22 E menarono Gesù al luogo detto Golgota; il che, interpretato, vuol dire luogo del teschio.
\par 23 E gli offersero da bere del vino mescolato con mirra; ma non ne prese.
\par 24 Poi lo crocifissero e si spartirono i suoi vestimenti, tirandoli a sorte per sapere quel che ne toccherebbe a ciascuno.
\par 25 Era l'ora terza quando lo crocifissero.
\par 26 E l'iscrizione indicante il motivo della condanna, diceva: IL RE DE' GIUDEI.
\par 27 E con lui crocifissero due ladroni, uno alla sua destra e l'altro alla sua sinistra.
\par 28 tex
\par 29 E quelli che passavano lì presso lo ingiuriavano, scotendo il capo e dicendo: Eh, tu che disfai il tempio e lo riedifichi in tre giorni,
\par 30 salva te stesso e scendi giù di croce!
\par 31 Parimente anche i capi sacerdoti con gli scribi, beffandosi, dicevano l'uno all'altro: Ha salvato altri e non può salvar se stesso!
\par 32 Il Cristo, il Re d'Israele, scenda ora giù di croce, affinché vediamo e crediamo! Anche quelli che erano stati crocifissi con lui, lo insultavano.
\par 33 E venuta l'ora sesta, si fecero tenebre per tutto il paese, fino all'ora nona.
\par 34 Ed all'ora nona, Gesù gridò con gran voce: Eloì, Eloì, lamà sabactanì? il che, interpretato, vuol dire: Dio mio, Dio mio, perché mi hai abbandonato?
\par 35 E alcuni degli astanti, udito ciò, dicevano: Ecco, chiama Elia!
\par 36 E uno di loro corse, e inzuppata d'aceto una spugna, e postala in cima ad una canna, gli diè da bere dicendo: Aspettate, vediamo se Elia viene a trarlo giù.
\par 37 E Gesù, gettato un gran grido, rendé lo spirito.
\par 38 E la cortina del tempio si squarciò in due, da cima a fondo.
\par 39 E il centurione ch'era quivi presente dirimpetto a Gesù, avendolo veduto spirare a quel modo, disse: Veramente, quest'uomo era Figliuol di Dio!
\par 40 Or v'erano anche delle donne, che guardavan da lontano; fra le quali era Maria Maddalena e Maria madre di Giacomo il piccolo e di Iose, e Salome;
\par 41 le quali, quand'egli era in Galilea, lo seguivano e lo servivano; e molte altre, che eran salite con lui a Gerusalemme.
\par 42 Ed essendo già sera (poiché era Preparazione, cioè la vigilia del sabato),
\par 43 venne Giuseppe d'Arimatea, consigliere onorato, il quale aspettava anch'egli il Regno di Dio; e, preso ardire, si presentò a Pilato e domandò il corpo di Gesù.
\par 44 Pilato si maravigliò ch'egli fosse già morto; e chiamato a sé il centurione, gli domandò se era morto da molto tempo;
\par 45 e saputolo dal centurione, donò il corpo a Giuseppe.
\par 46 E questi, comprato un panno lino e tratto Gesù giù di croce, l'involse nel panno e lo pose in una tomba scavata nella roccia, e rotolò una pietra contro l'apertura del sepolcro.
\par 47 E Maria Maddalena e Maria madre di Iose stavano guardando dove veniva deposto.

\chapter{16}

\par 1 E passato il sabato, Maria Maddalena e Maria madre di Giacomo e Salome comprarono degli aromi per andare a imbalsamar Gesù.
\par 2 E la mattina del primo giorno della settimana, molto per tempo, vennero al sepolcro sul levar del sole.
\par 3 E dicevano tra loro: Chi ci rotolerà la pietra dall'apertura del sepolcro?
\par 4 E alzati gli occhi, videro che la pietra era stata rotolata; ed era pur molto grande.
\par 5 Ed essendo entrate nel sepolcro, videro un giovinetto, seduto a destra, vestito d'una veste bianca, e furono spaventate.
\par 6 Ma egli disse loro: Non vi spaventate! Voi cercate Gesù il Nazareno che è stato crocifisso; egli è risuscitato; non è qui; ecco il luogo dove l'aveano posto.
\par 7 Ma andate a dire ai suoi discepoli ed a Pietro, ch'egli vi precede in Galilea; quivi lo vedrete, come v'ha detto.
\par 8 Ed esse, uscite, fuggiron via dal sepolcro, perché eran prese da tremito e da stupore, e non dissero nulla ad alcuno, perché aveano paura.
\par 9 Or Gesù, essendo risuscitato la mattina del primo giorno della settimana, apparve prima a Maria Maddalena, dalla quale avea cacciato sette demonî.
\par 10 Costei andò ad annunziarlo a coloro ch'erano stati con lui, i quali facean cordoglio e piangevano.
\par 11 Ed essi, udito ch'egli viveva ed era stato veduto da lei, non lo credettero.
\par 12 Or dopo questo, apparve in altra forma a due di loro ch'erano in cammino per andare ai campi;
\par 13 e questi andarono ad annunziarlo agli altri; ma neppure a quelli credettero.
\par 14 Di poi, apparve agli undici, mentre erano a tavola; e li rimproverò della loro incredulità e durezza di cuore, perché non avean creduto a quelli che l'avean veduto risuscitato.
\par 15 E disse loro: Andate per tutto il mondo e predicate l'evangelo ad ogni creatura.
\par 16 Chi avrà creduto e sarà stato battezzato sarà salvato; ma chi non avrà creduto sarà condannato.
\par 17 Or questi sono i segni che accompagneranno coloro che avranno creduto: nel nome mio cacceranno i demonî; parleranno in lingue nuove;
\par 18 prenderanno in mano dei serpenti; e se pur bevessero alcunché di mortifero, non ne avranno alcun male; imporranno le mani agl'infermi ed essi guariranno.
\par 19 Il Signor Gesù dunque, dopo aver loro parlato, fu assunto nel cielo, e sedette alla destra di Dio.
\par 20 E quelli se ne andarono a predicare da per tutto, operando il Signore con essi e confermando la Parola coi segni che l'accompagnavano.


\end{document}