\begin{document}

\title{1 Peter}


\chapter{1}

\par 1 Pietro, apostolo di Gesù Cristo, agli eletti che vivono come forestieri nella dispersione del Ponto, della Galazia, della Cappadocia, dell'Asia e della Bitinia,
\par 2 eletti secondo la prescienza di Dio Padre, mediante la santificazione dello Spirito, ad ubbidire e ad esser cosparsi del sangue di Gesù Cristo: grazia e pace vi siano moltiplicate.
\par 3 Benedetto sia l'Iddio e Padre del Signor nostro Gesù Cristo, il quale nella sua gran misericordia ci ha fatti rinascere, mediante la risurrezione di Gesù Cristo dai morti,
\par 4 ad una speranza viva in vista di una eredità incorruttibile, immacolata ed immarcescibile, conservata ne' cieli per voi,
\par 5 che dalla potenza di Dio, mediante la fede, siete custoditi per la salvazione che sta per esser rivelata negli ultimi tempi.
\par 6 Nel che voi esultate, sebbene ora, per un po' di tempo, se così bisogna, siate afflitti da svariate prove,
\par 7 affinché la prova della vostra fede, molto più preziosa dell'oro che perisce, eppure è provato col fuoco, risulti a vostra lode, gloria ed onore alla rivelazione di Gesù Cristo:
\par 8 il quale, benché non l'abbiate veduto, voi amate; nel quale credendo, benché ora non lo vediate, voi gioite d'un'allegrezza ineffabile e gloriosa,
\par 9 ottenendo il fine della fede: la salvezza delle anime.
\par 10 Questa salvezza è stata l'oggetto delle ricerche e delle investigazioni dei profeti che profetizzarono della grazia a voi destinata.
\par 11 Essi indagavano qual fosse il tempo e quali le circostanze a cui lo Spirito di Cristo che era in loro accennava, quando anticipatamente testimoniava delle sofferenze di Cristo, e delle glorie che dovevano seguire.
\par 12 E fu loro rivelato che non per se stessi ma per voi ministravano quelle cose che ora vi sono state annunziate da coloro che vi hanno evangelizzato per mezzo dello Spirito Santo mandato dal cielo; nelle quali cose gli angeli desiderano riguardare bene addentro.
\par 13 Perciò, avendo cinti i fianchi della vostra mente, e stando sobrî, abbiate piena speranza nella grazia che vi sarà recata nella rivelazione di Gesù Cristo;
\par 14 e, come figliuoli d'ubbidienza, non vi conformate alle concupiscenze del tempo passato quand'eravate nell'ignoranza;
\par 15 ma come Colui che vi ha chiamati è santo, anche voi siate santi in tutta la vostra condotta;
\par 16 poiché sta scritto: Siate santi, perché io son santo.
\par 17 E se invocate come Padre Colui che senza riguardi personali giudica secondo l'opera di ciascuno, conducetevi con timore durante il tempo del vostro pellegrinaggio;
\par 18 sapendo che non con cose corruttibili, con argento o con oro, siete stati riscattati dal vano modo di vivere tramandatovi dai padri,
\par 19 ma col prezioso sangue di Cristo, come d'agnello senza difetto né macchia,
\par 20 ben preordinato prima della fondazione del mondo, ma manifestato negli ultimi tempi per voi,
\par 21 i quali per mezzo di lui credete in Dio che l'ha risuscitato dai morti e gli ha dato gloria, onde la vostra fede e la vostra speranza fossero in Dio.
\par 22 Avendo purificate le anime vostre coll'ubbidienza alla verità per arrivare a un amor fraterno non finto, amatevi l'un l'altro di cuore, intensamente,
\par 23 poiché siete stati rigenerati non da seme corruttibile, ma incorruttibile, mediante la parola di Dio vivente e permanente.
\par 24 Poiché Ogni carne è com'erba, e ogni sua gloria come il fior dell'erba. L'erba si secca, e il fiore cade;
\par 25 ma la parola del Signore permane in eterno. E questa è la Parola della Buona Novella che vi è stata annunziata.

\chapter{2}

\par 1 Gettando dunque lungi da voi ogni malizia, e ogni frode, e le ipocrisie, e le invidie, ed ogni sorta di maldicenze, come bambini pur ora nati,
\par 2 appetite il puro latte spirituale, onde per esso cresciate per la salvezza,
\par 3 se pure avete gustato che il Signore è buono.
\par 4 Accostandovi a lui, pietra vivente, riprovata bensì dagli uomini ma innanzi a Dio eletta e preziosa, anche voi,
\par 5 come pietre viventi, siete edificati qual casa spirituale, per esser un sacerdozio santo per offrire sacrificî spirituali, accettevoli a Dio per mezzo di Gesù Cristo.
\par 6 Poiché si legge nella Scrittura: Ecco, io pongo in Sion una pietra angolare, eletta, preziosa; e chiunque crede in lui non sarà confuso.
\par 7 Per voi dunque che credete ell'è preziosa; ma per gl'increduli la pietra che gli edificatori hanno riprovata è quella ch'è divenuta la pietra angolare, e una pietra d'inciampo, e un sasso d'intoppo:
\par 8 essi, infatti, essendo disubbidienti, intoppano nella Parola; ed a questo sono stati anche destinati.
\par 9 Ma voi siete una generazione eletta, un real sacerdozio, una gente santa, un popolo che Dio s'è acquistato, affinché proclamiate le virtù di Colui che vi ha chiamati dalle tenebre alla sua maravigliosa luce;
\par 10 voi, che già non eravate un popolo, ma ora siete il popolo di Dio; voi che non avevate ottenuto misericordia, ma ora avete ottenuto misericordia.
\par 11 Diletti, io v'esorto come stranieri e pellegrini ad astenervi dalle carnali concupiscenze, che guerreggiano contro l'anima,
\par 12 avendo una buona condotta fra i Gentili; affinché laddove sparlano di voi come di malfattori, essi, per le vostre buone opere che avranno osservate, glorifichino Iddio nel giorno ch'Egli li visiterà.
\par 13 Siate soggetti, per amor del Signore, ad ogni autorità creata dagli uomini: al re, come al sovrano;
\par 14 ai governatori, come mandati da lui per punire i malfattori e per dar lode a quelli che fanno il bene.
\par 15 Poiché questa è la volontà di Dio: che, facendo il bene, turiate la bocca alla ignoranza degli uomini stolti;
\par 16 come liberi, ma non usando già della libertà qual manto che copra la malizia, ma come servi di Dio.
\par 17 Onorate tutti. Amate la fratellanza. Temete Iddio. Rendete onore al re.
\par 18 Domestici, siate con ogni timore soggetti ai vostri padroni; non solo ai buoni e moderati, ma anche a quelli che son difficili.
\par 19 Poiché questo è accettevole: se alcuno, per motivo di coscienza davanti a Dio, sopporta afflizioni, patendo ingiustamente.
\par 20 Infatti, che vanto c'è se, peccando ed essendo malmenati, voi sopportate pazientemente? Ma se facendo il bene, eppur patendo, voi sopportate pazientemente, questa è cosa grata a Dio.
\par 21 Perché a questo siete stati chiamati: poiché anche Cristo ha patito per voi, lasciandovi un esempio, onde seguiate le sue orme;
\par 22 egli, che non commise peccato, e nella cui bocca non fu trovata alcuna frode;
\par 23 che, oltraggiato, non rendeva gli oltraggi; che, soffrendo, non minacciava, ma si rimetteva nelle mani di Colui che giudica giustamente;
\par 24 egli, che ha portato egli stesso i nostri peccati nel suo corpo, sul legno, affinché, morti al peccato, vivessimo per la giustizia, e mediante le cui lividure siete stati sanati.
\par 25 Poiché eravate erranti come pecore; ma ora siete tornati al Pastore e Vescovo delle anime vostre.

\chapter{3}

\par 1 Parimente voi, mogli, siate soggette ai vostri mariti, affinché, se anche ve ne sono che non ubbidiscono alla Parola, siano guadagnati senza parola dalla condotta delle loro mogli,
\par 2 quand'avranno considerato la vostra condotta casta e rispettosa.
\par 3 Il vostro ornamento non sia l'esteriore che consiste nell'intrecciatura dei capelli, nel mettersi attorno dei gioielli d'oro, nell'indossar vesti sontuose
\par 4 ma l'essere occulto del cuore fregiato dell'ornamento incorruttibile dello spirito benigno e pacifico, che agli occhi di Dio è di gran prezzo.
\par 5 E così infatti si adornavano una volta le sante donne speranti in Dio, stando soggette ai loro mariti,
\par 6 come Sara che ubbidiva ad Abramo, chiamandolo signore; della quale voi siete ora figliuole, se fate il bene e non vi lasciate turbare da spavento alcuno.
\par 7 Parimente, voi, mariti, convivete con esse colla discrezione dovuta al vaso più debole ch'è il femminile. Portate loro onore, poiché sono anch'esse eredi con voi della grazia della vita, onde le vostre preghiere non siano impedite.
\par 8 Infine, siate tutti concordi, compassionevoli, pieni d'amor fraterno, pietosi, umili;
\par 9 non rendendo male per male, od oltraggio per oltraggio, ma, al contrario, benedicendo; poiché a questo siete stati chiamati onde ereditiate la benedizione.
\par 10 Perché: Chi vuol amar la vita e veder buoni giorni, rattenga la sua lingua dal male e le sue labbra dal parlar con frode;
\par 11 si ritragga dal male e faccia il bene; cerchi la pace e la procacci;
\par 12 perché gli occhi del Signore sono sui giusti e i suoi orecchi sono attenti alle loro supplicazioni; ma la faccia del Signore è contro quelli che fanno il male.
\par 13 E chi è colui che vi farà del male, se siete zelanti del bene?
\par 14 Ma anche se aveste a soffrire per cagion di giustizia, beati voi! E non vi sgomenti la paura che incutono e non vi conturbate;
\par 15 anzi abbiate nei vostri cuori un santo timore di Cristo il Signore, pronti sempre a rispondere a vostra difesa a chiunque vi domanda ragione della speranza che è in voi, ma con dolcezza e rispetto; avendo una buona coscienza;
\par 16 onde laddove sparlano di voi, siano svergognati quelli che calunniano la vostra buona condotta in Cristo.
\par 17 Perché è meglio, se pur tale è la volontà di Dio, che soffriate facendo il bene, anziché facendo il male.
\par 18 Poiché anche Cristo ha sofferto una volta per i peccati, egli giusto per gl'ingiusti, per condurci a Dio; essendo stato messo a morte, quanto alla carne, ma vivificato quanto allo spirito;
\par 19 e in esso andò anche a predicare agli spiriti ritenuti in carcere,
\par 20 i quali un tempo furon ribelli, quando la pazienza di Dio aspettava, ai giorni di Noè, mentre si preparava l'arca; nella quale poche anime, cioè otto, furon salvate tra mezzo all'acqua.
\par 21 Alla qual figura corrisponde il battesimo (non il nettamento delle sozzure della carne ma la richiesta di una buona coscienza fatta a Dio), il quale ora salva anche voi, mediante la risurrezione di Gesù Cristo,
\par 22 che, essendo andato in cielo, è alla destra di Dio, dove angeli, principati e potenze gli son sottoposti.

\chapter{4}

\par 1 Poiché dunque Cristo ha sofferto nella carne, anche voi armatevi di questo stesso pensiero, che, cioè, colui che ha sofferto nella carne ha cessato dal peccato,
\par 2 per consacrare il tempo che resta da passare nella carne, non più alle concupiscenze degli uomini, ma alla volontà di Dio.
\par 3 Poiché basta l'aver dato il vostro passato a fare la volontà de' Gentili col vivere nelle lascivie, nelle concupiscenze, nelle ubriachezze, nelle gozzoviglie, negli sbevazzamenti, e nelle nefande idolatrie.
\par 4 Per la qual cosa trovano strano che voi non corriate con loro agli stessi eccessi di dissolutezza, e dicon male di voi.
\par 5 Essi renderanno ragione a colui ch'è pronto a giudicare i vivi ed i morti.
\par 6 Poiché per questo è stato annunziato l'Evangelo anche ai morti; onde fossero bensì giudicati secondo gli uomini quanto alla carne, ma vivessero secondo Dio quanto allo spirito.
\par 7 Or la fine d'ogni cosa è vicina; siate dunque temperati e vigilanti alle orazioni.
\par 8 Soprattutto, abbiate amore intenso gli uni per gli altri, perché l'amore copre moltitudine di peccati.
\par 9 Siate ospitali gli uni verso gli altri senza mormorare.
\par 10 Come buoni amministratori della svariata grazia di Dio, ciascuno, secondo il dono che ha ricevuto, lo faccia valere al servizio degli altri.
\par 11 Se uno parla, lo faccia come annunziando oracoli di Dio; se uno esercita un ministerio, lo faccia come con la forza che Dio fornisce, onde in ogni cosa sia glorificato Iddio per mezzo di Gesù Cristo, al quale appartengono la gloria e l'imperio nei secoli de' secoli. Amen.
\par 12 Diletti, non vi stupite della fornace accesa in mezzo a voi per provarvi, quasiché vi avvenisse qualcosa di strano.
\par 13 Anzi in quanto partecipate alle sofferenze di Cristo, rallegratevene, affinché anche alla rivelazione della sua gloria possiate rallegrarvi giubilando.
\par 14 Se siete vituperati per il nome di Cristo, beati voi! perché lo Spirito di gloria, lo Spirito di Dio, riposa su voi.
\par 15 Nessun di voi patisca come omicida, o ladro, o malfattore, o come ingerentesi nei fatti altrui;
\par 16 ma se uno patisce come Cristiano, non se ne vergogni, ma glorifichi Iddio portando questo nome.
\par 17 Poiché è giunto il tempo in cui il giudicio ha da cominciare dalla casa di Dio; e se comincia prima da noi, qual sarà la fine di quelli che non ubbidiscono al Vangelo di Dio?
\par 18 E se il giusto è appena salvato, dove comparirà l'empio e il peccatore?
\par 19 Perciò anche quelli che soffrono secondo la volontà di Dio, raccomandino le anime loro al fedel Creatore, facendo il bene.

\chapter{5}

\par 1 Io esorto dunque gli anziani che sono fra voi, io che sono anziano con loro e testimone delle sofferenze di Cristo e che sarò pure partecipe della gloria che ha da essere manifestata:
\par 2 Pascete il gregge di Dio che è fra voi, non forzatamente, ma volonterosamente secondo Dio; non per un vil guadagno, ma di buon animo;
\par 3 e non come signoreggiando quelli che vi son toccati in sorte, ma essendo gli esempi del gregge.
\par 4 E quando sarà apparito il sommo Pastore, otterrete la corona della gloria che non appassisce.
\par 5 Parimente, voi più giovani, siate soggetti agli anziani. E tutti rivestitevi d'umiltà gli uni verso gli altri, perché Dio resiste ai superbi ma dà grazia agli umili.
\par 6 Umiliatevi dunque sotto la potente mano di Dio, affinché Egli v'innalzi a suo tempo,
\par 7 gettando su lui ogni vostra sollecitudine, perch'Egli ha cura di voi.
\par 8 Siate sobrî, vegliate; il vostro avversario, il diavolo, va attorno a guisa di leon ruggente cercando chi possa divorare.
\par 9 Resistetegli stando fermi nella fede, sapendo che le medesime sofferenze si compiono nella vostra fratellanza sparsa per il mondo.
\par 10 Or l'Iddio d'ogni grazia, il quale vi ha chiamati alla sua eterna gloria in Cristo, dopo che avrete sofferto per breve tempo, vi perfezionerà Egli stesso, vi renderà saldi, vi fortificherà.
\par 11 A lui sia l'imperio, nei secoli dei secoli. Amen.
\par 12 Per mezzo di Silvano, nostro fedel fratello, com'io lo stimo, v'ho scritto brevemente esortandovi, e attestando che questa è la vera grazia di Dio; in essa state saldi.
\par 13 La chiesa che è in Babilonia eletta come voi, vi saluta; e così fa Marco, il mio figliuolo.
\par 14 Salutatevi gli uni gli altri con un bacio d'amore. Pace a voi tutti che siete in Cristo.


\end{document}