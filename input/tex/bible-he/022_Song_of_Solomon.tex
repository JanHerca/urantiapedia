\begin{document}

\title{שיר השירים}


\chapter{1}

\par 1 שׁיר השׁירים אשׁר לשׁלמה׃
\par 2 ישׁקני מנשׁיקות פיהו כי־טובים דדיך מיין׃
\par 3 לריח שׁמניך טובים שׁמן תורק שׁמך על־כן עלמות אהבוך׃
\par 4 משׁכני אחריך נרוצה הביאני המלך חדריו נגילה ונשׂמחה בך נזכירה דדיך מיין מישׁרים אהבוך׃
\par 5 שׁחורה אני ונאוה בנות ירושׁלם כאהלי קדר כיריעות שׁלמה׃
\par 6 אל־תראוני שׁאני שׁחרחרת שׁשׁזפתני השׁמשׁ בני אמי נחרו־בי שׂמני נטרה את־הכרמים כרמי שׁלי לא נטרתי׃
\par 7 הגידה לי שׁאהבה נפשׁי איכה תרעה איכה תרביץ בצהרים שׁלמה אהיה כעטיה על עדרי חבריך׃
\par 8 אם־לא תדעי לך היפה בנשׁים צאי־לך בעקבי הצאן ורעי את־גדיתיך על משׁכנות הרעים׃
\par 9 לססתי ברכבי פרעה דמיתיך רעיתי׃
\par 10 נאוו לחייך בתרים צוארך בחרוזים׃
\par 11 תורי זהב נעשׂה־לך עם נקדות הכסף׃
\par 12 עד־שׁהמלך במסבו נרדי נתן ריחו׃
\par 13 צרור המר דודי לי בין שׁדי ילין׃
\par 14 אשׁכל הכפר דודי לי בכרמי עין גדי׃
\par 15 הנך יפה רעיתי הנך יפה עיניך יונים׃
\par 16 הנך יפה דודי אף נעים אף־ערשׂנו רעננה׃
\par 17 קרות בתינו ארזים רחיטנו ברותים׃

\chapter{2}

\par 1 אני חבצלת השׁרון שׁושׁנת העמקים׃
\par 2 כשׁושׁנה בין החוחים כן רעיתי בין הבנות׃
\par 3 כתפוח בעצי היער כן דודי בין הבנים בצלו חמדתי וישׁבתי ופריו מתוק לחכי׃
\par 4 הביאני אל־בית היין ודגלו עלי אהבה׃
\par 5 סמכוני באשׁישׁות רפדוני בתפוחים כי־חולת אהבה אני׃
\par 6 שׂמאלו תחת לראשׁי וימינו תחבקני׃
\par 7 השׁבעתי אתכם בנות ירושׁלם בצבאות או באילות השׂדה אם־תעירו ואם־תעוררו את־האהבה עד שׁתחפץ׃
\par 8 קול דודי הנה־זה בא מדלג על־ההרים מקפץ על־הגבעות׃
\par 9 דומה דודי לצבי או לעפר האילים הנה־זה עומד אחר כתלנו משׁגיח מן־החלנות מציץ מן־החרכים׃
\par 10 ענה דודי ואמר לי קומי לך רעיתי יפתי ולכי־לך׃
\par 11 כי־הנה הסתו עבר הגשׁם חלף הלך לו׃
\par 12 הנצנים נראו בארץ עת הזמיר הגיע וקול התור נשׁמע בארצנו׃
\par 13 התאנה חנטה פגיה והגפנים סמדר נתנו ריח קומי לכי רעיתי יפתי ולכי־לך׃
\par 14 יונתי בחגוי הסלע בסתר המדרגה הראיני את־מראיך השׁמיעיני את־קולך כי־קולך ערב ומראיך נאוה׃
\par 15 אחזו־לנו שׁועלים שׁועלים קטנים מחבלים כרמים וכרמינו סמדר׃
\par 16 דודי לי ואני לו הרעה בשׁושׁנים׃
\par 17 עד שׁיפוח היום ונסו הצללים סב דמה־לך דודי לצבי או לעפר האילים על־הרי בתר׃

\chapter{3}

\par 1 על־משׁכבי בלילות בקשׁתי את שׁאהבה נפשׁי בקשׁתיו ולא מצאתיו׃
\par 2 אקומה נא ואסובבה בעיר בשׁוקים וברחבות אבקשׁה את שׁאהבה נפשׁי בקשׁתיו ולא מצאתיו׃
\par 3 מצאוני השׁמרים הסבבים בעיר את שׁאהבה נפשׁי ראיתם׃
\par 4 כמעט שׁעברתי מהם עד שׁמצאתי את שׁאהבה נפשׁי אחזתיו ולא ארפנו עד־שׁהביאתיו אל־בית אמי ואל־חדר הורתי׃
\par 5 השׁבעתי אתכם בנות ירושׁלם בצבאות או באילות השׂדה אם־תעירו ואם־תעוררו את־האהבה עד שׁתחפץ׃
\par 6 מי זאת עלה מן־המדבר כתימרות עשׁן מקטרת מור ולבונה מכל אבקת רוכל׃
\par 7 הנה מטתו שׁלשׁלמה שׁשׁים גברים סביב לה מגברי ישׂראל׃
\par 8 כלם אחזי חרב מלמדי מלחמה אישׁ חרבו על־ירכו מפחד בלילות׃
\par 9 אפריון עשׂה לו המלך שׁלמה מעצי הלבנון׃
\par 10 עמודיו עשׂה כסף רפידתו זהב מרכבו ארגמן תוכו רצוף אהבה מבנות ירושׁלם׃
\par 11 צאינה וראינה בנות ציון במלך שׁלמה בעטרה שׁעטרה־לו אמו ביום חתנתו וביום שׂמחת לבו׃

\chapter{4}

\par 1 הנך יפה רעיתי הנך יפה עיניך יונים מבעד לצמתך שׂערך כעדר העזים שׁגלשׁו מהר גלעד׃
\par 2 שׁניך כעדר הקצובות שׁעלו מן־הרחצה שׁכלם מתאימות ושׁכלה אין בהם׃
\par 3 כחוט השׁני שׂפתתיך ומדבריך נאוה כפלח הרמון רקתך מבעד לצמתך׃
\par 4 כמגדל דויד צוארך בנוי לתלפיות אלף המגן תלוי עליו כל שׁלטי הגבורים׃
\par 5 שׁני שׁדיך כשׁני עפרים תאומי צביה הרועים בשׁושׁנים׃
\par 6 עד שׁיפוח היום ונסו הצללים אלך לי אל־הר המור ואל־גבעת הלבונה׃
\par 7 כלך יפה רעיתי ומום אין בך׃
\par 8 אתי מלבנון כלה אתי מלבנון תבואי תשׁורי מראשׁ אמנה מראשׁ שׂניר וחרמון ממענות אריות מהררי נמרים׃
\par 9 לבבתני אחתי כלה לבבתיני באחד מעיניך באחד ענק מצורניך׃
\par 10 מה־יפו דדיך אחתי כלה מה־טבו דדיך מיין וריח שׁמניך מכל־בשׂמים׃
\par 11 נפת תטפנה שׂפתותיך כלה דבשׁ וחלב תחת לשׁונך וריח שׂלמתיך כריח לבנון׃
\par 12 גן נעול אחתי כלה גל נעול מעין חתום׃
\par 13 שׁלחיך פרדס רמונים עם פרי מגדים כפרים עם־נרדים׃
\par 14 נרד וכרכם קנה וקנמון עם כל־עצי לבונה מר ואהלות עם כל־ראשׁי בשׂמים׃
\par 15 מעין גנים באר מים חיים ונזלים מן־לבנון׃
\par 16 עורי צפון ובואי תימן הפיחי גני יזלו בשׂמיו יבא דודי לגנו ויאכל פרי מגדיו׃

\chapter{5}

\par 1 באתי לגני אחתי כלה אריתי מורי עם־בשׂמי אכלתי יערי עם־דבשׁי שׁתיתי ייני עם־חלבי אכלו רעים שׁתו ושׁכרו דודים׃
\par 2 אני ישׁנה ולבי ער קול דודי דופק פתחי־לי אחתי רעיתי יונתי תמתי שׁראשׁי נמלא־טל קוצותי רסיסי לילה׃
\par 3 פשׁטתי את־כתנתי איככה אלבשׁנה רחצתי את־רגלי איככה אטנפם׃
\par 4 דודי שׁלח ידו מן־החר ומעי המו עליו׃
\par 5 קמתי אני לפתח לדודי וידי נטפו־מור ואצבעתי מור עבר על כפות המנעול׃
\par 6 פתחתי אני לדודי ודודי חמק עבר נפשׁי יצאה בדברו בקשׁתיהו ולא מצאתיהו קראתיו ולא ענני׃
\par 7 מצאני השׁמרים הסבבים בעיר הכוני פצעוני נשׂאו את־רדידי מעלי שׁמרי החמות׃
\par 8 השׁבעתי אתכם בנות ירושׁלם אם־תמצאו את־דודי מה־תגידו לו שׁחולת אהבה אני׃
\par 9 מה־דודך מדוד היפה בנשׁים מה־דודך מדוד שׁככה השׁבעתנו׃
\par 10 דודי צח ואדום דגול מרבבה׃
\par 11 ראשׁו כתם פז קוצותיו תלתלים שׁחרות כעורב׃
\par 12 עיניו כיונים על־אפיקי מים רחצות בחלב ישׁבות על־מלאת׃
\par 13 לחיו כערוגת הבשׂם מגדלות מרקחים שׂפתותיו שׁושׁנים נטפות מור עבר׃
\par 14 ידיו גלילי זהב ממלאים בתרשׁישׁ מעיו עשׁת שׁן מעלפת ספירים׃
\par 15 שׁוקיו עמודי שׁשׁ מיסדים על־אדני־פז מראהו כלבנון בחור כארזים׃
\par 16 חכו ממתקים וכלו מחמדים זה דודי וזה רעי בנות ירושׁלם׃

\chapter{6}

\par 1 אנה הלך דודך היפה בנשׁים אנה פנה דודך ונבקשׁנו עמך׃
\par 2 דודי ירד לגנו לערוגות הבשׂם לרעות בגנים וללקט שׁושׁנים׃
\par 3 אני לדודי ודודי לי הרעה בשׁושׁנים׃
\par 4 יפה את רעיתי כתרצה נאוה כירושׁלם אימה כנדגלות׃
\par 5 הסבי עיניך מנגדי שׁהם הרהיבני שׂערך כעדר העזים שׁגלשׁו מן־הגלעד׃
\par 6 שׁניך כעדר הרחלים שׁעלו מן־הרחצה שׁכלם מתאימות ושׁכלה אין בהם׃
\par 7 כפלח הרמון רקתך מבעד לצמתך׃
\par 8 שׁשׁים המה מלכות ושׁמנים פילגשׁים ועלמות אין מספר׃
\par 9 אחת היא יונתי תמתי אחת היא לאמה ברה היא ליולדתה ראוה בנות ויאשׁרוה מלכות ופילגשׁים ויהללוה׃
\par 10 מי־זאת הנשׁקפה כמו־שׁחר יפה כלבנה ברה כחמה אימה כנדגלות׃
\par 11 אל־גנת אגוז ירדתי לראות באבי הנחל לראות הפרחה הגפן הנצו הרמנים׃
\par 12 לא ידעתי נפשׁי שׂמתני מרכבות עמי־נדיב׃

\chapter{7}

\par 1 שׁובי שׁובי השׁולמית שׁובי שׁובי ונחזה־בך מה־תחזו בשׁולמית כמחלת המחנים׃
\par 2 מה־יפו פעמיך בנעלים בת־נדיב חמוקי ירכיך כמו חלאים מעשׂה ידי אמן׃
\par 3 שׁררך אגן הסהר אל־יחסר המזג בטנך ערמת חטים סוגה בשׁושׁנים׃
\par 4 שׁני שׁדיך כשׁני עפרים תאמי צביה׃
\par 5 צוארך כמגדל השׁן עיניך ברכות בחשׁבון על־שׁער בת־רבים אפך כמגדל הלבנון צופה פני דמשׂק׃
\par 6 ראשׁך עליך ככרמל ודלת ראשׁך כארגמן מלך אסור ברהטים׃
\par 7 מה־יפית ומה־נעמת אהבה בתענוגים׃
\par 8 זאת קומתך דמתה לתמר ושׁדיך לאשׁכלות׃
\par 9 אמרתי אעלה בתמר אחזה בסנסניו ויהיו־נא שׁדיך כאשׁכלות הגפן וריח אפך כתפוחים׃
\par 10 וחכך כיין הטוב הולך לדודי למישׁרים דובב שׂפתי ישׁנים׃
\par 11 אני לדודי ועלי תשׁוקתו׃
\par 12 לכה דודי נצא השׂדה נלינה בכפרים׃
\par 13 נשׁכימה לכרמים נראה אם פרחה הגפן פתח הסמדר הנצו הרמונים שׁם אתן את־דדי לך׃
\par 14 הדודאים נתנו־ריח ועל־פתחינו כל־מגדים חדשׁים גם־ישׁנים דודי צפנתי לך׃

\chapter{8}

\par 1 מי יתנך כאח לי יונק שׁדי אמי אמצאך בחוץ אשׁקך גם לא־יבוזו לי׃
\par 2 אנהגך אביאך אל־בית אמי תלמדני אשׁקך מיין הרקח מעסיס רמני׃
\par 3 שׂמאלו תחת ראשׁי וימינו תחבקני׃
\par 4 השׁבעתי אתכם בנות ירושׁלם מה־תעירו ומה־תעררו את־האהבה עד שׁתחפץ׃
\par 5 מי זאת עלה מן־המדבר מתרפקת על־דודה תחת התפוח עוררתיך שׁמה חבלתך אמך שׁמה חבלה ילדתך׃
\par 6 שׂימני כחותם על־לבך כחותם על־זרועך כי־עזה כמות אהבה קשׁה כשׁאול קנאה רשׁפיה רשׁפי אשׁ שׁלהבתיה׃
\par 7 מים רבים לא יוכלו לכבות את־האהבה ונהרות לא ישׁטפוה אם־יתן אישׁ את־כל־הון ביתו באהבה בוז יבוזו לו׃
\par 8 אחות לנו קטנה ושׁדים אין לה מה־נעשׂה לאחתנו ביום שׁידבר־בה׃
\par 9 אם־חומה היא נבנה עליה טירת כסף ואם־דלת היא נצור עליה לוח ארז׃
\par 10 אני חומה ושׁדי כמגדלות אז הייתי בעיניו כמוצאת שׁלום׃
\par 11 כרם היה לשׁלמה בבעל המון נתן את־הכרם לנטרים אישׁ יבא בפריו אלף כסף׃
\par 12 כרמי שׁלי לפני האלף לך שׁלמה ומאתים לנטרים את־פריו׃
\par 13 היושׁבת בגנים חברים מקשׁיבים לקולך השׁמיעיני׃
\par 14 ברח דודי ודמה־לך לצבי או לעפר האילים על הרי בשׂמים׃


\end{document}