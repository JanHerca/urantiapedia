\begin{document}

\title{קהלת}


\chapter{1}

\par 1 דברי קהלת בן־דוד מלך בירושׁלם׃
\par 2 הבל הבלים אמר קהלת הבל הבלים הכל הבל׃
\par 3 מה־יתרון לאדם בכל־עמלו שׁיעמל תחת השׁמשׁ׃
\par 4 דור הלך ודור בא והארץ לעולם עמדת׃
\par 5 וזרח השׁמשׁ ובא השׁמשׁ ואל־מקומו שׁואף זורח הוא שׁם׃
\par 6 הולך אל־דרום וסובב אל־צפון סובב סבב הולך הרוח ועל־סביבתיו שׁב הרוח׃
\par 7 כל־הנחלים הלכים אל־הים והים איננו מלא אל־מקום שׁהנחלים הלכים שׁם הם שׁבים ללכת׃
\par 8 כל־הדברים יגעים לא־יוכל אישׁ לדבר לא־תשׂבע עין לראות ולא־תמלא אזן משׁמע׃
\par 9 מה־שׁהיה הוא שׁיהיה ומה־שׁנעשׂה הוא שׁיעשׂה ואין כל־חדשׁ תחת השׁמשׁ׃
\par 10 ישׁ דבר שׁיאמר ראה־זה חדשׁ הוא כבר היה לעלמים אשׁר היה מלפננו׃
\par 11 אין זכרון לראשׁנים וגם לאחרנים שׁיהיו לא־יהיה להם זכרון עם שׁיהיו לאחרנה׃
\par 12 אני קהלת הייתי מלך על־ישׂראל בירושׁלם׃
\par 13 ונתתי את־לבי לדרושׁ ולתור בחכמה על כל־אשׁר נעשׂה תחת השׁמים הוא ענין רע נתן אלהים לבני האדם לענות בו׃
\par 14 ראיתי את־כל־המעשׂים שׁנעשׂו תחת השׁמשׁ והנה הכל הבל ורעות רוח׃
\par 15 מעות לא־יוכל לתקן וחסרון לא־יוכל להמנות׃
\par 16 דברתי אני עם־לבי לאמר אני הנה הגדלתי והוספתי חכמה על כל־אשׁר־היה לפני על־ירושׁלם ולבי ראה הרבה חכמה ודעת׃
\par 17 ואתנה לבי לדעת חכמה ודעת הוללות ושׂכלות ידעתי שׁגם־זה הוא רעיון רוח׃
\par 18 כי ברב חכמה רב־כעס ויוסיף דעת יוסיף מכאוב׃

\chapter{2}

\par 1 אמרתי אני בלבי לכה־נא אנסכה בשׂמחה וראה בטוב והנה גם־הוא הבל׃
\par 2 לשׂחוק אמרתי מהולל ולשׂמחה מה־זה עשׂה׃
\par 3 תרתי בלבי למשׁוך ביין את־בשׂרי ולבי נהג בחכמה ולאחז בסכלות עד אשׁר־אראה אי־זה טוב לבני האדם אשׁר יעשׂו תחת השׁמים מספר ימי חייהם׃
\par 4 הגדלתי מעשׂי בניתי לי בתים נטעתי לי כרמים׃
\par 5 עשׂיתי לי גנות ופרדסים ונטעתי בהם עץ כל־פרי׃
\par 6 עשׂיתי לי ברכות מים להשׁקות מהם יער צומח עצים׃
\par 7 קניתי עבדים ושׁפחות ובני־בית היה לי גם מקנה בקר וצאן הרבה היה לי מכל שׁהיו לפני בירושׁלם׃
\par 8 כנסתי לי גם־כסף וזהב וסגלת מלכים והמדינות עשׂיתי לי שׁרים ושׁרות ותענוגת בני האדם שׁדה ושׁדות׃
\par 9 וגדלתי והוספתי מכל שׁהיה לפני בירושׁלם אף חכמתי עמדה לי׃
\par 10 וכל אשׁר שׁאלו עיני לא אצלתי מהם לא־מנעתי את־לבי מכל־שׂמחה כי־לבי שׂמח מכל־עמלי וזה־היה חלקי מכל־עמלי׃
\par 11 ופניתי אני בכל־מעשׂי שׁעשׂו ידי ובעמל שׁעמלתי לעשׂות והנה הכל הבל ורעות רוח ואין יתרון תחת השׁמשׁ׃
\par 12 ופניתי אני לראות חכמה והוללות וסכלות כי מה האדם שׁיבוא אחרי המלך את אשׁר־כבר עשׂוהו׃
\par 13 וראיתי אני שׁישׁ יתרון לחכמה מן־הסכלות כיתרון האור מן־החשׁך׃
\par 14 החכם עיניו בראשׁו והכסיל בחשׁך הולך וידעתי גם־אני שׁמקרה אחד יקרה את־כלם׃
\par 15 ואמרתי אני בלבי כמקרה הכסיל גם־אני יקרני ולמה חכמתי אני אז יותר ודברתי בלבי שׁגם־זה הבל׃
\par 16 כי אין זכרון לחכם עם־הכסיל לעולם בשׁכבר הימים הבאים הכל נשׁכח ואיך ימות החכם עם־הכסיל׃
\par 17 ושׂנאתי את־החיים כי רע עלי המעשׂה שׁנעשׂה תחת השׁמשׁ כי־הכל הבל ורעות רוח׃
\par 18 ושׂנאתי אני את־כל־עמלי שׁאני עמל תחת השׁמשׁ שׁאניחנו לאדם שׁיהיה אחרי׃
\par 19 ומי יודע החכם יהיה או סכל וישׁלט בכל־עמלי שׁעמלתי ושׁחכמתי תחת השׁמשׁ גם־זה הבל׃
\par 20 וסבותי אני ליאשׁ את־לבי על כל־העמל שׁעמלתי תחת השׁמשׁ׃
\par 21 כי־ישׁ אדם שׁעמלו בחכמה ובדעת ובכשׁרון ולאדם שׁלא עמל־בו יתננו חלקו גם־זה הבל ורעה רבה׃
\par 22 כי מה־הוה לאדם בכל־עמלו וברעיון לבו שׁהוא עמל תחת השׁמשׁ׃
\par 23 כי כל־ימיו מכאבים וכעס ענינו גם־בלילה לא־שׁכב לבו גם־זה הבל הוא׃
\par 24 אין־טוב באדם שׁיאכל ושׁתה והראה את־נפשׁו טוב בעמלו גם־זה ראיתי אני כי מיד האלהים היא׃
\par 25 כי מי יאכל ומי יחושׁ חוץ ממני׃
\par 26 כי לאדם שׁטוב לפניו נתן חכמה ודעת ושׂמחה ולחוטא נתן ענין לאסוף ולכנוס לתת לטוב לפני האלהים גם־זה הבל ורעות רוח׃

\chapter{3}

\par 1 לכל זמן ועת לכל־חפץ תחת השׁמים׃
\par 2 עת ללדת ועת למות עת לטעת ועת לעקור נטוע׃
\par 3 עת להרוג ועת לרפוא עת לפרוץ ועת לבנות׃
\par 4 עת לבכות ועת לשׂחוק עת ספוד ועת רקוד׃
\par 5 עת להשׁליך אבנים ועת כנוס אבנים עת לחבוק ועת לרחק מחבק׃
\par 6 עת לבקשׁ ועת לאבד עת לשׁמור ועת להשׁליך׃
\par 7 עת לקרוע ועת לתפור עת לחשׁות ועת לדבר׃
\par 8 עת לאהב ועת לשׂנא עת מלחמה ועת שׁלום׃
\par 9 מה־יתרון העושׂה באשׁר הוא עמל׃
\par 10 ראיתי את־הענין אשׁר נתן אלהים לבני האדם לענות בו׃
\par 11 את־הכל עשׂה יפה בעתו גם את־העלם נתן בלבם מבלי אשׁר לא־ימצא האדם את־המעשׂה אשׁר־עשׂה האלהים מראשׁ ועד־סוף׃
\par 12 ידעתי כי אין טוב בם כי אם־לשׂמוח ולעשׂות טוב בחייו׃
\par 13 וגם כל־האדם שׁיאכל ושׁתה וראה טוב בכל־עמלו מתת אלהים היא׃
\par 14 ידעתי כי כל־אשׁר יעשׂה האלהים הוא יהיה לעולם עליו אין להוסיף וממנו אין לגרע והאלהים עשׂה שׁיראו מלפניו׃
\par 15 מה־שׁהיה כבר הוא ואשׁר להיות כבר היה והאלהים יבקשׁ את־נרדף׃
\par 16 ועוד ראיתי תחת השׁמשׁ מקום המשׁפט שׁמה הרשׁע ומקום הצדק שׁמה הרשׁע׃
\par 17 אמרתי אני בלבי את־הצדיק ואת־הרשׁע ישׁפט האלהים כי־עת לכל־חפץ ועל כל־המעשׂה שׁם׃
\par 18 אמרתי אני בלבי על־דברת בני האדם לברם האלהים ולראות שׁהם־בהמה המה להם׃
\par 19 כי מקרה בני־האדם ומקרה הבהמה ומקרה אחד להם כמות זה כן מות זה ורוח אחד לכל ומותר האדם מן־הבהמה אין כי הכל הבל׃
\par 20 הכל הולך אל־מקום אחד הכל היה מן־העפר והכל שׁב אל־העפר׃
\par 21 מי יודע רוח בני האדם העלה היא למעלה ורוח הבהמה הירדת היא למטה לארץ׃
\par 22 וראיתי כי אין טוב מאשׁר ישׂמח האדם במעשׂיו כי־הוא חלקו כי מי יביאנו לראות במה שׁיהיה אחריו׃

\chapter{4}

\par 1 ושׁבתי אני ואראה את־כל־העשׁקים אשׁר נעשׂים תחת השׁמשׁ והנה דמעת העשׁקים ואין להם מנחם ומיד עשׁקיהם כח ואין להם מנחם׃
\par 2 ושׁבח אני את־המתים שׁכבר מתו מן־החיים אשׁר המה חיים עדנה׃
\par 3 וטוב משׁניהם את אשׁר־עדן לא היה אשׁר לא־ראה את־המעשׂה הרע אשׁר נעשׂה תחת השׁמשׁ׃
\par 4 וראיתי אני את־כל־עמל ואת כל־כשׁרון המעשׂה כי היא קנאת־אישׁ מרעהו גם־זה הבל ורעות רוח׃
\par 5 הכסיל חבק את־ידיו ואכל את־בשׂרו׃
\par 6 טוב מלא כף נחת ממלא חפנים עמל ורעות רוח׃
\par 7 ושׁבתי אני ואראה הבל תחת השׁמשׁ׃
\par 8 ישׁ אחד ואין שׁני גם בן ואח אין־לו ואין קץ לכל־עמלו גם־עיניו לא־תשׂבע עשׁר ולמי אני עמל ומחסר את־נפשׁי מטובה גם־זה הבל וענין רע הוא׃
\par 9 טובים השׁנים מן־האחד אשׁר ישׁ־להם שׂכר טוב בעמלם׃
\par 10 כי אם־יפלו האחד יקים את־חברו ואילו האחד שׁיפול ואין שׁני להקימו׃
\par 11 גם אם־ישׁכבו שׁנים וחם להם ולאחד איך יחם׃
\par 12 ואם־יתקפו האחד השׁנים יעמדו נגדו והחוט המשׁלשׁ לא במהרה ינתק׃
\par 13 טוב ילד מסכן וחכם ממלך זקן וכסיל אשׁר לא־ידע להזהר עוד׃
\par 14 כי־מבית הסורים יצא למלך כי גם במלכותו נולד רשׁ׃
\par 15 ראיתי את־כל־החיים המהלכים תחת השׁמשׁ עם הילד השׁני אשׁר יעמד תחתיו׃
\par 16 אין־קץ לכל־העם לכל אשׁר־היה לפניהם גם האחרונים לא ישׂמחו־בו כי־גם־זה הבל ורעיון רוח׃
\par 17 שׁמר רגליך כאשׁר תלך אל־בית האלהים וקרוב לשׁמע מתת הכסילים זבח כי־אינם יודעים לעשׂות רע׃

\chapter{5}

\par 1 אל־תבהל על־פיך ולבך אל־ימהר להוציא דבר לפני האלהים כי האלהים בשׁמים ואתה על־הארץ על־כן יהיו דבריך מעטים׃
\par 2 כי בא החלום ברב ענין וקול כסיל ברב דברים׃
\par 3 כאשׁר תדר נדר לאלהים אל־תאחר לשׁלמו כי אין חפץ בכסילים את אשׁר־תדר שׁלם׃
\par 4 טוב אשׁר לא־תדר משׁתדור ולא תשׁלם׃
\par 5 אל־תתן את־פיך לחטיא את־בשׂרך ואל־תאמר לפני המלאך כי שׁגגה היא למה יקצף האלהים על־קולך וחבל את־מעשׂה ידיך׃
\par 6 כי ברב חלמות והבלים ודברים הרבה כי את־האלהים ירא׃
\par 7 אם־עשׁק רשׁ וגזל משׁפט וצדק תראה במדינה אל־תתמה על־החפץ כי גבה מעל גבה שׁמר וגבהים עליהם׃
\par 8 ויתרון ארץ בכל היא מלך לשׂדה נעבד׃
\par 9 אהב כסף לא־ישׂבע כסף ומי־אהב בהמון לא תבואה גם־זה הבל׃
\par 10 ברבות הטובה רבו אוכליה ומה־כשׁרון לבעליה כי אם־ראית עיניו׃
\par 11 מתוקה שׁנת העבד אם־מעט ואם־הרבה יאכל והשׂבע לעשׁיר איננו מניח לו לישׁון׃
\par 12 ישׁ רעה חולה ראיתי תחת השׁמשׁ עשׁר שׁמור לבעליו לרעתו׃
\par 13 ואבד העשׁר ההוא בענין רע והוליד בן ואין בידו מאומה׃
\par 14 כאשׁר יצא מבטן אמו ערום ישׁוב ללכת כשׁבא ומאומה לא־ישׂא בעמלו שׁילך בידו׃
\par 15 וגם־זה רעה חולה כל־עמת שׁבא כן ילך ומה־יתרון לו שׁיעמל לרוח׃
\par 16 גם כל־ימיו בחשׁך יאכל וכעס הרבה וחליו וקצף׃
\par 17 הנה אשׁר־ראיתי אני טוב אשׁר־יפה לאכול־ולשׁתות ולראות טובה בכל־עמלו שׁיעמל תחת־השׁמשׁ מספר ימי־חיו אשׁר־נתן־לו האלהים כי־הוא חלקו׃
\par 18 גם כל־האדם אשׁר נתן־לו האלהים עשׁר ונכסים והשׁליטו לאכל ממנו ולשׂאת את־חלקו ולשׂמח בעמלו זה מתת אלהים היא׃
\par 19 כי לא הרבה יזכר את־ימי חייו כי האלהים מענה בשׂמחת לבו׃

\chapter{6}

\par 1 ישׁ רעה אשׁר ראיתי תחת השׁמשׁ ורבה היא על־האדם׃
\par 2 אישׁ אשׁר יתן־לו האלהים עשׁר ונכסים וכבוד ואיננו חסר לנפשׁו מכל אשׁר־יתאוה ולא־ישׁליטנו האלהים לאכל ממנו כי אישׁ נכרי יאכלנו זה הבל וחלי רע הוא׃
\par 3 אם־יוליד אישׁ מאה ושׁנים רבות יחיה ורב שׁיהיו ימי־שׁניו ונפשׁו לא־תשׂבע מן־הטובה וגם־קבורה לא־היתה לו אמרתי טוב ממנו הנפל׃
\par 4 כי־בהבל בא ובחשׁך ילך ובחשׁך שׁמו יכסה׃
\par 5 גם־שׁמשׁ לא־ראה ולא ידע נחת לזה מזה׃
\par 6 ואלו חיה אלף שׁנים פעמים וטובה לא ראה הלא אל־מקום אחד הכל הולך׃
\par 7 כל־עמל האדם לפיהו וגם־הנפשׁ לא תמלא׃
\par 8 כי מה־יותר לחכם מן־הכסיל מה־לעני יודע להלך נגד החיים׃
\par 9 טוב מראה עינים מהלך־נפשׁ גם־זה הבל ורעות רוח׃
\par 10 מה־שׁהיה כבר נקרא שׁמו ונודע אשׁר־הוא אדם ולא־יוכל לדין עם שׁהתקיף ממנו׃
\par 11 כי ישׁ־דברים הרבה מרבים הבל מה־יתר לאדם׃
\par 12 כי מי־יודע מה־טוב לאדם בחיים מספר ימי־חיי הבלו ויעשׂם כצל אשׁר מי־יגיד לאדם מה־יהיה אחריו תחת השׁמשׁ׃

\chapter{7}

\par 1 טוב שׁם משׁמן טוב ויום המות מיום הולדו׃
\par 2 טוב ללכת אל־בית־אבל מלכת אל־בית משׁתה באשׁר הוא סוף כל־האדם והחי יתן אל־לבו׃
\par 3 טוב כעס משׂחק כי־ברע פנים ייטב לב׃
\par 4 לב חכמים בבית אבל ולב כסילים בבית שׂמחה׃
\par 5 טוב לשׁמע גערת חכם מאישׁ שׁמע שׁיר כסילים׃
\par 6 כי כקול הסירים תחת הסיר כן שׂחק הכסיל וגם־זה הבל׃
\par 7 כי העשׁק יהולל חכם ויאבד את־לב מתנה׃
\par 8 טוב אחרית דבר מראשׁיתו טוב ארך־רוח מגבה־רוח׃
\par 9 אל־תבהל ברוחך לכעוס כי כעס בחיק כסילים ינוח׃
\par 10 אל־תאמר מה היה שׁהימים הראשׁנים היו טובים מאלה כי לא מחכמה שׁאלת על־זה׃
\par 11 טובה חכמה עם־נחלה ויתר לראי השׁמשׁ׃
\par 12 כי בצל החכמה בצל הכסף ויתרון דעת החכמה תחיה בעליה׃
\par 13 ראה את־מעשׂה האלהים כי מי יוכל לתקן את אשׁר עותו׃
\par 14 ביום טובה היה בטוב וביום רעה ראה גם את־זה לעמת־זה עשׂה האלהים על־דברת שׁלא ימצא האדם אחריו מאומה׃
\par 15 את־הכל ראיתי בימי הבלי ישׁ צדיק אבד בצדקו וישׁ רשׁע מאריך ברעתו׃
\par 16 אל־תהי צדיק הרבה ואל־תתחכם יותר למה תשׁומם׃
\par 17 אל־תרשׁע הרבה ואל־תהי סכל למה תמות בלא עתך׃
\par 18 טוב אשׁר תאחז בזה וגם־מזה אל־תנח את־ידך כי־ירא אלהים יצא את־כלם׃
\par 19 החכמה תעז לחכם מעשׂרה שׁליטים אשׁר היו בעיר׃
\par 20 כי אדם אין צדיק בארץ אשׁר יעשׂה־טוב ולא יחטא׃
\par 21 גם לכל־הדברים אשׁר ידברו אל־תתן לבך אשׁר לא־תשׁמע את־עבדך מקללך׃
\par 22 כי גם־פעמים רבות ידע לבך אשׁר גם־את קללת אחרים׃
\par 23 כל־זה נסיתי בחכמה אמרתי אחכמה והיא רחוקה ממני׃
\par 24 רחוק מה־שׁהיה ועמק עמק מי ימצאנו׃
\par 25 סבותי אני ולבי לדעת ולתור ובקשׁ חכמה וחשׁבון ולדעת רשׁע כסל והסכלות הוללות׃
\par 26 ומוצא אני מר ממות את־האשׁה אשׁר־היא מצודים וחרמים לבה אסורים ידיה טוב לפני האלהים ימלט ממנה וחוטא ילכד בה׃
\par 27 ראה זה מצאתי אמרה קהלת אחת לאחת למצא חשׁבון׃
\par 28 אשׁר עוד־בקשׁה נפשׁי ולא מצאתי אדם אחד מאלף מצאתי ואשׁה בכל־אלה לא מצאתי׃
\par 29 לבד ראה־זה מצאתי אשׁר עשׂה האלהים את־האדם ישׁר והמה בקשׁו חשׁבנות רבים׃

\chapter{8}

\par 1 מי כהחכם ומי יודע פשׁר דבר חכמת אדם תאיר פניו ועז פניו ישׁנא׃
\par 2 אני פי־מלך שׁמור ועל דברת שׁבועת אלהים׃
\par 3 אל־תבהל מפניו תלך אל־תעמד בדבר רע כי כל־אשׁר יחפץ יעשׂה׃
\par 4 באשׁר דבר־מלך שׁלטון ומי יאמר־לו מה־תעשׂה׃
\par 5 שׁומר מצוה לא ידע דבר רע ועת ומשׁפט ידע לב חכם׃
\par 6 כי לכל־חפץ ישׁ עת ומשׁפט כי־רעת האדם רבה עליו׃
\par 7 כי־איננו ידע מה־שׁיהיה כי כאשׁר יהיה מי יגיד לו׃
\par 8 אין אדם שׁליט ברוח לכלוא את־הרוח ואין שׁלטון ביום המות ואין משׁלחת במלחמה ולא־ימלט רשׁע את־בעליו׃
\par 9 את־כל־זה ראיתי ונתון את־לבי לכל־מעשׂה אשׁר נעשׂה תחת השׁמשׁ עת אשׁר שׁלט האדם באדם לרע לו׃
\par 10 ובכן ראיתי רשׁעים קברים ובאו וממקום קדושׁ יהלכו וישׁתכחו בעיר אשׁר כן־עשׂו גם־זה הבל׃
\par 11 אשׁר אין־נעשׂה פתגם מעשׂה הרעה מהרה על־כן מלא לב בני־האדם בהם לעשׂות רע׃
\par 12 אשׁר חטא עשׂה רע מאת ומאריך לו כי גם־יודע אני אשׁר יהיה־טוב ליראי האלהים אשׁר ייראו מלפניו׃
\par 13 וטוב לא־יהיה לרשׁע ולא־יאריך ימים כצל אשׁר איננו ירא מלפני אלהים׃
\par 14 ישׁ־הבל אשׁר נעשׂה על־הארץ אשׁר ישׁ צדיקים אשׁר מגיע אלהם כמעשׂה הרשׁעים וישׁ רשׁעים שׁמגיע אלהם כמעשׂה הצדיקים אמרתי שׁגם־זה הבל׃
\par 15 ושׁבחתי אני את־השׂמחה אשׁר אין־טוב לאדם תחת השׁמשׁ כי אם־לאכול ולשׁתות ולשׂמוח והוא ילונו בעמלו ימי חייו אשׁר־נתן־לו האלהים תחת השׁמשׁ׃
\par 16 כאשׁר נתתי את־לבי לדעת חכמה ולראות את־הענין אשׁר נעשׂה על־הארץ כי גם ביום ובלילה שׁנה בעיניו איננו ראה׃
\par 17 וראיתי את־כל־מעשׂה האלהים כי לא יוכל האדם למצוא את־המעשׂה אשׁר נעשׂה תחת־השׁמשׁ בשׁל אשׁר יעמל האדם לבקשׁ ולא ימצא וגם אם־יאמר החכם לדעת לא יוכל למצא׃

\chapter{9}

\par 1 כי את־כל־זה נתתי אל־לבי ולבור את־כל־זה אשׁר הצדיקים והחכמים ועבדיהם ביד האלהים גם־אהבה גם־שׂנאה אין יודע האדם הכל לפניהם׃
\par 2 הכל כאשׁר לכל מקרה אחד לצדיק ולרשׁע לטוב ולטהור ולטמא ולזבח ולאשׁר איננו זבח כטוב כחטא הנשׁבע כאשׁר שׁבועה ירא׃
\par 3 זה רע בכל אשׁר־נעשׂה תחת השׁמשׁ כי־מקרה אחד לכל וגם לב בני־האדם מלא־רע והוללות בלבבם בחייהם ואחריו אל־המתים׃
\par 4 כי־מי אשׁר יבחר אל כל־החיים ישׁ בטחון כי־לכלב חי הוא טוב מן־האריה המת׃
\par 5 כי החיים יודעים שׁימתו והמתים אינם יודעים מאומה ואין־עוד להם שׂכר כי נשׁכח זכרם׃
\par 6 גם אהבתם גם־שׂנאתם גם־קנאתם כבר אבדה וחלק אין־להם עוד לעולם בכל אשׁר־נעשׂה תחת השׁמשׁ׃
\par 7 לך אכל בשׂמחה לחמך ושׁתה בלב־טוב יינך כי כבר רצה האלהים את־מעשׂיך׃
\par 8 בכל־עת יהיו בגדיך לבנים ושׁמן על־ראשׁך אל־יחסר׃
\par 9 ראה חיים עם־אשׁה אשׁר־אהבת כל־ימי חיי הבלך אשׁר נתן־לך תחת השׁמשׁ כל ימי הבלך כי הוא חלקך בחיים ובעמלך אשׁר־אתה עמל תחת השׁמשׁ׃
\par 10 כל אשׁר תמצא ידך לעשׂות בכחך עשׂה כי אין מעשׂה וחשׁבון ודעת וחכמה בשׁאול אשׁר אתה הלך שׁמה׃
\par 11 שׁבתי וראה תחת־השׁמשׁ כי לא לקלים המרוץ ולא לגבורים המלחמה וגם לא לחכמים לחם וגם לא לנבנים עשׁר וגם לא לידעים חן כי־עת ופגע יקרה את־כלם׃
\par 12 כי גם לא־ידע האדם את־עתו כדגים שׁנאחזים במצודה רעה וכצפרים האחזות בפח כהם יוקשׁים בני האדם לעת רעה כשׁתפול עליהם פתאם׃
\par 13 גם־זה ראיתי חכמה תחת השׁמשׁ וגדולה היא אלי׃
\par 14 עיר קטנה ואנשׁים בה מעט ובא־אליה מלך גדול וסבב אתה ובנה עליה מצודים גדלים׃
\par 15 ומצא בה אישׁ מסכן חכם ומלט־הוא את־העיר בחכמתו ואדם לא זכר את־האישׁ המסכן ההוא׃
\par 16 ואמרתי אני טובה חכמה מגבורה וחכמת המסכן בזויה ודבריו אינם נשׁמעים׃
\par 17 דברי חכמים בנחת נשׁמעים מזעקת מושׁל בכסילים׃
\par 18 טובה חכמה מכלי קרב וחוטא אחד יאבד טובה הרבה׃

\chapter{10}

\par 1 זבובי מות יבאישׁ יביע שׁמן רוקח יקר מחכמה מכבוד סכלות מעט׃
\par 2 לב חכם לימינו ולב כסיל לשׂמאלו׃
\par 3 וגם־בדרך כשׁהסכל הלך לבו חסר ואמר לכל סכל הוא׃
\par 4 אם־רוח המושׁל תעלה עליך מקומך אל־תנח כי מרפא יניח חטאים גדולים׃
\par 5 ישׁ רעה ראיתי תחת השׁמשׁ כשׁגגה שׁיצא מלפני השׁליט׃
\par 6 נתן הסכל במרומים רבים ועשׁירים בשׁפל ישׁבו׃
\par 7 ראיתי עבדים על־סוסים ושׂרים הלכים כעבדים על־הארץ׃
\par 8 חפר גומץ בו יפול ופרץ גדר ישׁכנו נחשׁ׃
\par 9 מסיע אבנים יעצב בהם בוקע עצים יסכן בם׃
\par 10 אם־קהה הברזל והוא לא־פנים קלקל וחילים יגבר ויתרון הכשׁיר חכמה׃
\par 11 אם־ישׁך הנחשׁ בלוא־לחשׁ ואין יתרון לבעל הלשׁון׃
\par 12 דברי פי־חכם חן ושׂפתות כסיל תבלענו׃
\par 13 תחלת דברי־פיהו סכלות ואחרית פיהו הוללות רעה׃
\par 14 והסכל ירבה דברים לא־ידע האדם מה־שׁיהיה ואשׁר יהיה מאחריו מי יגיד לו׃
\par 15 עמל הכסילים תיגענו אשׁר לא־ידע ללכת אל־עיר׃
\par 16 אי־לך ארץ שׁמלכך נער ושׂריך בבקר יאכלו׃
\par 17 אשׁריך ארץ שׁמלכך בן־חורים ושׂריך בעת יאכלו בגבורה ולא בשׁתי׃
\par 18 בעצלתים ימך המקרה ובשׁפלות ידים ידלף הבית׃
\par 19 לשׂחוק עשׂים לחם ויין ישׂמח חיים והכסף יענה את־הכל׃
\par 20 גם במדעך מלך אל־תקלל ובחדרי משׁכבך אל־תקלל עשׁיר כי עוף השׁמים יוליך את־הקול ובעל הכנפים יגיד דבר׃

\chapter{11}

\par 1 שׁלח לחמך על־פני המים כי־ברב הימים תמצאנו׃
\par 2 תן־חלק לשׁבעה וגם לשׁמונה כי לא תדע מה־יהיה רעה על־הארץ׃
\par 3 אם־ימלאו העבים גשׁם על־הארץ יריקו ואם־יפול עץ בדרום ואם בצפון מקום שׁיפול העץ שׁם יהוא׃
\par 4 שׁמר רוח לא יזרע וראה בעבים לא יקצור׃
\par 5 כאשׁר אינך יודע מה־דרך הרוח כעצמים בבטן המלאה ככה לא תדע את־מעשׂה האלהים אשׁר יעשׂה את־הכל׃
\par 6 בבקר זרע את־זרעך ולערב אל־תנח ידך כי אינך יודע אי זה יכשׁר הזה או־זה ואם־שׁניהם כאחד טובים׃
\par 7 ומתוק האור וטוב לעינים לראות את־השׁמשׁ׃
\par 8 כי אם־שׁנים הרבה יחיה האדם בכלם ישׂמח ויזכר את־ימי החשׁך כי־הרבה יהיו כל־שׁבא הבל׃
\par 9 שׂמח בחור בילדותיך ויטיבך לבך בימי בחורותך והלך בדרכי לבך ובמראי עיניך ודע כי על־כל־אלה יביאך האלהים במשׁפט׃
\par 10 והסר כעס מלבך והעבר רעה מבשׂרך כי־הילדות והשׁחרות הבל׃

\chapter{12}

\par 1 וזכר את־בוראיך בימי בחורתיך עד אשׁר לא־יבאו ימי הרעה והגיעו שׁנים אשׁר תאמר אין־לי בהם חפץ׃
\par 2 עד אשׁר לא־תחשׁך השׁמשׁ והאור והירח והכוכבים ושׁבו העבים אחר הגשׁם׃
\par 3 ביום שׁיזעו שׁמרי הבית והתעותו אנשׁי החיל ובטלו הטחנות כי מעטו וחשׁכו הראות בארבות׃
\par 4 וסגרו דלתים בשׁוק בשׁפל קול הטחנה ויקום לקול הצפור וישׁחו כל־בנות השׁיר׃
\par 5 גם מגבה יראו וחתחתים בדרך וינאץ השׁקד ויסתבל החגב ותפר האביונה כי־הלך האדם אל־בית עולמו וסבבו בשׁוק הספדים׃
\par 6 עד אשׁר לא־ירחק חבל הכסף ותרץ גלת הזהב ותשׁבר כד על־המבוע ונרץ הגלגל אל־הבור׃
\par 7 וישׁב העפר על־הארץ כשׁהיה והרוח תשׁוב אל־האלהים אשׁר נתנה׃
\par 8 הבל הבלים אמר הקוהלת הכל הבל׃
\par 9 ויתר שׁהיה קהלת חכם עוד למד־דעת את־העם ואזן וחקר תקן משׁלים הרבה׃
\par 10 בקשׁ קהלת למצא דברי־חפץ וכתוב ישׁר דברי אמת׃
\par 11 דברי חכמים כדרבנות וכמשׂמרות נטועים בעלי אספות נתנו מרעה אחד׃
\par 12 ויתר מהמה בני הזהר עשׂות ספרים הרבה אין קץ ולהג הרבה יגעת בשׂר׃
\par 13 סוף דבר הכל נשׁמע את־האלהים ירא ואת־מצותיו שׁמור כי־זה כל־האדם׃
\par 14 כי את־כל־מעשׂה האלהים יבא במשׁפט על כל־נעלם אם־טוב ואם־רע׃


\end{document}