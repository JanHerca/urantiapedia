\begin{document}

\title{Esimene Makkabite raamat}

\chapter{1}

\section*{Aleksander Suur jaotab oma riigi}

\par 1 Pärast seda kui Makedoonia Aleksander, Filippose poeg, kes tuli kittide maalt, oli võitnud pärslaste ja meedlaste kuninga Dareiose, sündis, et ta hakkas valitsema tema asemel, olles juba enne Kreeka kuningas.
\par 2 Ta pidas palju sõdu ja vallutas kindlusi ning tappis maa kuningaid.
\par 3 Ta tungis maailma äärteni ja võttis sõjasaaki paljudelt rahvastelt. Aga kui maailm oli temale alistunud, siis läks ta ülbeks ja tema süda suurustas.
\par 4 Ta kogus väga võimsa sõjaväe ning valitses maid, rahvaid ja vürste, ja nemad pidid talle makse maksma.
\par 5 Aga seejärel heitis ta voodisse ja kui ta tundis, et sureb,
\par 6 siis kutsus ta oma auväärseimad teenrid, kes noorusest alates olid koos temaga kasvanud, ja jaotas neile oma kuningriigi veel oma eluajal.
\par 7 Aleksander suri, olles valitsenud kaksteist aastat.
\par 8 Ja tema teenrid hakkasid valitsema, igaüks oma piirkonnas.
\par 9 Pärast tema surma panid nad kõik krooni pähe, palju aastaid pärast neid ka nende pojad nõndasamuti, ja nad tegid maailmas palju kurja. 

\section*{Usust taganevad juudid ühinevad paganatega}

\par 10 Ja neist võrsus patune võsu, Antiohhos Epifanes, kuningas Antiohhose poeg, kes oli olnud Roomas pantvangiks. Tema sai kuningaks Kreeka kuningriigi saja kolmekümne seitsmendal aastal.
\par 11 Neil päevil tõusis Iisraelis jumalavallatuid mehi, kes hukutasid paljusid, üteldes: „Mingem ja ühinegem ümberkaudsete paganatega, sest sellest ajast peale, kui endid neist eraldasime, on meid tabanud palju õnnetusi!”
\par 12 See kõne oli nende meelest hea.
\par 13 Mõned rahva hulgast olid nõus ja läksid kuninga juurde ning tema andis neile loa talitada paganate kommete järgi.
\par 14 Siis nad ehitasid Jeruusalemma maadluskooli paganate eeskujul.
\par 15 Ja nad taastasid oma eesnahad ning taganesid pühast lepingust, ühinesid paganatega ja müüsid endid kurja tegema. 

\section*{Antiohhose sõjakäik}

\par 16 Kui kuningriik oli Antiohhose meelest kindlustatud, siis ta tahtis saada ka Egiptuse kuningaks, et nõnda valitseda kahte kuningriiki.
\par 17 Ta tungis Egiptusesse võimsa väehulgaga, sõjavankrite ja elevantidega, ratsaväe ja suure laevastikuga
\par 18 ning hakkas sõdima Egiptuse kuninga Ptolemaiose vastu. Ptolemaios tundis aga tema ees hirmu ja põgenes ning paljud langesid mahalööduna.
\par 19 Egiptuse kindlustatud linnad vallutati ja ta võttis Egiptusest sõjasaaki.
\par 20 Kui Antiohhos oli Egiptust löönud, pöördus ta tagasi aastal sada nelikümmend kolm ja läks üles Iisraeli ja Jeruusalemma vastu võimsa sõjaväega.
\par 21 Oma suures ülbuses tungis ta pühakotta ja võttis ära kuldaltari, lambijala ja kõik pühakoja riistad:
\par 22 vaateleibade laua, ohvripeekrid, ohvriliuad, kullast suitsutusrohupannid, eesriide ja selle vanikud, ja kuldilustused, mis olid templi esiküljel. Ta riisus kõik ära.
\par 23 Ta võttis ära hõbeda ja kulla ning hinnalised riistad. Ta võttis ära ka need peidetud varandused, mis ta leidis.
\par 24 Ja olles ära võtnud kõik, läks ta oma maale. Ta pani toime ka veretöö ning kõneles suurustavaid sõnu.
\par 25 Siis haaras Iisraeli suur lein kõigis tema asulais:
\par 26 „Vürstid ja vanemad oigasid, neitsid ja noored mehed nõrkesid ning naiste kaunidus kadus.
\par 27 Peig alustas nutulaulu ja mõrsjakambris istuja oli kurb.
\par 28 Maa värises oma elanike pärast ja häbi kattis kogu Jaakobi soo.” 

\section*{Uus sõjakäik}

\par 29 Kaks aastat pärast seda läkitas kuningas maksunõudjate ülema Juuda linnadesse. Ja see tuli Jeruusalemma võimsa sõjaväega.
\par 30 Ta rääkis neile kavalalt rahusõnu ja nad uskusid teda. Aga äkitselt ta ründas linna ja lõi seda võimsa rünnakuga ning tappis palju Iisraeli rahvast.
\par 31 Ta riisus linna, pani selle põlema ning kiskus maha selle hooned ja ümbritsevad müürid.
\par 32 Naised ja lapsed viidi vangi ning karjad võeti ära.
\par 33 Aga Taaveti linna nad kindlustasid suure ja tugeva müüriga, tugevate tornidega, ja see sai neile kindluseks.
\par 34 Sinna paigutasid nad patuse rahva, nurjatud mehed, ja need kindlustasid endid seal.
\par 35 Nad varusid sinna sõjariistu ning moona ja panid sinna Jeruusalemmast kogutud saagi. Ja sellest sai suur lõks.
\par 36 „Sellest sai varitsuspaik pühamu vastu ja alaline kuri ähvardus Iisraelile.
\par 37 Pühamu ümber valasid nad süütut verd ja rüvetasid ka pühamut.
\par 38 Nende pärast siis Jeruusalemma elanikud põgenesid ja linn sai muulaste eluasemeks. See muutus oma järglastele võõraks ja tema oma lapsed jätsid ta maha.
\par 39 Pühamu rüüstati otsekui kõrbeks, pühad muutusid leinaks, hingamispäevad teotuseks, au mitte millekski.
\par 40 Nõnda suur kui oli olnud ta auhiilgus, nõnda suureks sai ta häbi, tema uhkusest sai lein.” 

\section*{Juudiusu tagakiusamine}

\par 41 Kuningas kirjutas kogu oma kuningriigile, et kõik peavad saama üheks rahvaks
\par 42 ja igaüks peab loobuma oma kombeist. Ja kõik rahvad alistusid kuninga käsule.
\par 43 Ka paljudele Iisraeli lastele meeldis tema jumalateenistus ja nad ohverdasid ebajumalaile ning teotasid hingamispäeva.
\par 44 Kuningas läkitas käskjalgade kaudu käsukirjad Jeruusalemma ja Juuda linnadesse, et need elaksid kommete järgi, mis maale olid võõrad,
\par 45 ja loobuksid põletusohvreist, tapaohvreist ja joogiohvreist pühamus, teotaksid hingamispäevi ja pühi,
\par 46 rüvetaksid pühamu ja selle pühitsetud mehed,
\par 47 ehitaksid ohvrikünkaid, pühamuid ja ebajumalate kodasid, ohverdaksid sigu ja muid rüvedaid loomi,
\par 48 jätaksid oma pojad ümber lõikamata ja rüvetaksid oma hinge kõigega, mis ei ole puhas, vaid on roojane,
\par 49 nõnda et nad unustaksid Seaduse ja tühistaksid kõik selle korraldused.
\par 50 Ja kes kuninga käsu järgi ei tee, peab surema.
\par 51 Nii nagu olid kõik need sõnad, nõnda ta kirjutas tervele oma kuningriigile ja pani ülevaatajad kogu rahvale. Ta andis Juuda linnadele käsu ohverdada igas linnas.
\par 52 Rahva hulgast kogunesid siis paljud nende ülevaatajate juurde, kõik, kes hülgasid Seaduse. Need tegid maal kurja
\par 53 ning sundisid Iisraeli rahva pelgupaikadesse, kuhu iganes nad said endid peita.
\par 54 Ja kislevikuu viieteistkümnendal päeval aastal sada nelikümmend viis püstitasid nad ohvrialtari peale „hävituse koletise” ja rajasid ohvrikünkaid ümberkaudseisse Juuda linnadesse.
\par 55 Nad suitsutasid kodade uste ees ja tänavail.
\par 56 Seaduse raamatud, mis nad leidsid, rebisid nad katki ja põletasid tules.
\par 57 Kui kelleltki leiti lepinguraamat või kui keegi Seadusest kinni pidas, siis ta mõisteti surma kuninga otsuse kohaselt.
\par 58 Kuust kuusse talitasid nad vägivaldselt Iisraelis kõigiga, keda nad linnades tabasid.
\par 59 Ja iga kuu kahekümne viiendal päeval ohverdasid nad sellel altaril, mis oli põletusohvrialtari peal.
\par 60 Naised, kes olid lasknud oma lapsed ümber lõigata, surmasid nad vastavalt käsule
\par 61 ja nad poosid kaelapidi nende lapsukesed ja kodakondsed ning need, kes olid olnud ümberlõikajaiks.
\par 62 Aga paljud Iisraelist kinnitasid endid ja otsustasid, et nad ei söö rüvedat.
\par 63 Nad tahtsid pigem surra kui ennast toitudega rüvetada ja rikkuda püha lepingut. Ja nad suridki.
\par 64 Jah, suur karistus lasus rängalt Iisraeli peal.

\chapter{2}

\section*{Mattatias ja tema pojad}

\par 1 Neil päevil tõusis Mattatias, Siimeoni poja Johannese poeg, preester Joojaribi poegade hulgast Jeruusalemmast. Ta elas aga Moodeinis.
\par 2 Ja temal oli viis poega: Johannes, keda hüüti Gaddiks,
\par 3 Siimon, keda hüüti Tassiks,
\par 4 Juudas, keda hüüti Makkabiks,
\par 5 Eleasar, keda hüüti Avaraniks, Joonatan, keda hüüti Apfuseks.
\par 6 Kui tema nägi neid pühaduseteotusi, mis Juudas ja Jeruusalemmas sündisid,
\par 7 siis ta ütles: „Häda mulle! Miks olen sündinud nägema oma rahva hukkumist ja püha linna hävingut, istudes siin jõude, kui linn on antud vaenlastele ja pühakoda võõraste kätte?
\par 8 Tema templist on saanud otsekui põlatud mees,
\par 9 selle toredad riistad on saagiks viidud. Tema lapsukesed on tapetud tänavail, noored mehed vaenlase mõõgaga.
\par 10 Kas on rahvast, kes ei ole vallutanud tema kuningriiki ega ole võtnud temalt saaki?
\par 11 Kõik tema ehted on röövitud, vabast on saanud ori.
\par 12 Jah, vaata, mis meile oli püha, mis meile oli ilus ja uhke, see on hävitatud ja paganad on seda teotanud.
\par 13 Milleks me veel elame?” 
\par 14 Ja Mattatias ja tema pojad käristasid oma riided lõhki, panid endile kotiriide ümber ja leinasid väga. Võõras ohvritalitus Moodeinis
\par 15 Kuninga saadikud, kes pidid sundima usust taganemisele, tulid Moodeini linna, et seal hakataks ohverdama.
\par 16 Ja paljud Iisraelist ühinesid nendega. Aga ka Mattatias ja tema pojad olid kokku tulnud.
\par 17 Ja kuninga saadikud hakkasid rääkima ning ütlesid Mattatiasele: „Sina oled juht, austatud ja suur selles linnas ja sul on toetus poegade ja vendade poolt.
\par 18 Astu nüüd esimesena ette ja tee kuninga käsu järgi, nõnda nagu on teinud kõik rahvad ja Juuda mehed ning need, kes Jeruusalemma on alles jäänud. Siis saate sina ja sinu pojad kuninga sõpradeks ja sind ning sinu poegi austatakse hõbeda ja kullaga ning paljude kingitustega.”
\par 19 Mattatias aga vastas ja ütles valju häälega: „Isegi kui kõik rahvad kuninga valitsuse all oleval alal teda kuulda võtavad, nõnda et igaüks loobub oma vanemate jumalateenistusest ja alistub tema käskudele,
\par 20 siis mina ja minu pojad ja minu vennad tahame käia oma vanemate lepingu kohaselt.
\par 21 Hoitagu meid loobumast Seadusest ja korraldustest!
\par 22 Meie ei võta seda kuninga sõna kuulda, et me oma jumalateenistusest kalduksime kõrvale, paremale või vasakule.” 

\section*{Usust taganeja tapetakse, tapja põgeneb}

\par 23 Kui ta need sõnad oli ütelnud, siis astus ette üks Juuda mees kõigi nähes ja hakkas ohverdama Moodeini altaril kuninga käsu kohaselt.
\par 24 Kui Mattatias seda nägi, siis ta vihastas ja tema sisemus kees. Õigusega tõusis temas viha. Ta jooksis ning tappis mehe altari juures.
\par 25 Ka kuningamehe, kes sundis ohverdama, tappis ta selsamal viivul. Ja ta lõhkus altari.
\par 26 Ta vihastas Seaduse kaitseks, tehes nagu Piinehas oli teinud Simriga, Salu pojaga.
\par 27 Ja Mattatias hüüdis linnas valju häälega, üteldes: „Igaüks, kes tahab kindlaks jääda Seadusele ja lepingule, tulgu minu järel!”
\par 28 Tema ja ta pojad põgenesid siis mägedesse ning jätsid linna maha kõik, mis neil oli. 

\section*{Hingamispäeva pühitsemine pannakse proovile}

\par 29 Siis ka paljud, kes nõudsid õigust ja õiglust, läksid kõrbe, et elada seal,
\par 30 nemad ise, nende lapsed ja naised ning nende karjad, sest vaen nende vastu süvenes üha.
\par 31 Kuningameestele ja neile väehulkadele, kes olid Jeruusalemmas, Taaveti linnas, anti aga teada, et mehed, kes olid põlanud kuninga käsku, olid läinud kõrbe koobastesse.
\par 32 Paljud jooksid siis neile järele ja leidnud nad, lõid leeri üles nende vastu ja valmistusid nendega sõdima hingamispäeval.
\par 33 Nad ütlesid neile: „Küllalt nüüd! Tulge välja, tehke kuninga sõna järgi, siis te jääte elama!”
\par 34 Need aga vastasid: „Meie ei tule välja ega tee kuninga sõna järgi, et peaksime teotama hingamispäeva.”
\par 35 Ja nad hakkasid nendega sõdima.
\par 36 Need aga ei hakanud vastu, ei visanud mitte kivigi nende poole ega sulgenud koopaid,
\par 37 vaid ütlesid: „Meie kõik tahame surra oma vagaduses. Taevas ja maa on meie tunnistajad, et teie meid ülekohtuselt tapate.”
\par 38 Teised alustasid aga hingamispäeval sõda nende vastu, ja nõnda nad surid, nemad ja nende naised ja lapsed ning nende karjad, ligi tuhat inimhinge. 

\section*{Mattatias pooldajatega alustab võitlust}

\par 39 Kui Mattatias ja tema sõbrad sellest teada said, siis nad leinasid nende pärast väga.
\par 40 Ja nad ütlesid üksteisele: „Kui meie kõik teeme nõnda, nagu meie vennad tegid, ega võitle paganate vastu oma hinge ja õiguste eest, siis nad hävitavad meid varsti maa pealt.”
\par 41 Selsamal päeval nad otsustasid, üteldes: „Meie võitleme igaühe vastu, kes tuleb meiega sõdima hingamispäeval, et me kõik ei sureks, nõnda nagu surid meie vennad koobastes.”
\par 42 Siis kogunes nende juurde jõuk hassiide, vapraid võitlejaid Iisraelist, igaüks neist ustav Seadusele.
\par 43 Ja kõik, kes vaenu eest põgenesid, ühinesid nendega ning said neile toeks.
\par 44 Nõnda moodustasid nad sõjaväe ning oma vihas ja raevus lõid nad maha patused ja ülekohtused mehed. Ülejäänud põgenesid aga paganate juurde, et endid päästa.
\par 45 Mattatias ja tema sõbrad aga käisid ringi ning kiskusid maha altarid
\par 46 ja lõikasid vägisi ümber veel ümber lõikamata poeglapsed, keda nad leidsid Iisraeli alalt.
\par 47 Nad jälitasid jultunuid ja see tegu läks korda nende käes.
\par 48 Nõnda nad kaitsesid Seadust paganate ja kuningate vägivalla vastu ega andnud patustele võimust. 

\section*{Mattatiase viimased sõnad}

\par 49 Kui päevad kätte jõudsid, et Mattatias pidi surema, siis ta ütles oma poegadele: „Nüüd on võimust võtnud jultumus ja karistus, häving ja pöörane viha.
\par 50 Võidelge nüüd, lapsed, Seaduse pärast ja andke oma hinged meie vanemate lepingu eest!
\par 51 Ja meenutage vanemate tegusid, mida nende põlvkond tegi, siis saate teiegi suure au ja igavese nime!
\par 52 Eks Aabraham ju leitud katsumuses ustav olevat ja eks see arvatud temale õiguseks.
\par 53 Joosep pidas oma kitsikuse ajal kinni käsust ja sai Egiptuse isandaks.
\par 54 Meie esiisa Piinehas, vihastades tõsise vihaga, sai igavese preestriameti lepingu.
\par 55 Joosua täitis Jumala sõna ja sai Iisraeli kohtumõistjaks.
\par 56 Kaaleb andis koguduses tunnistuse ja sai pärisosaks maa.
\par 57 Taavet päris oma halastuse tõttu kuningliku aujärje igaveseks ajaks.
\par 58 Eelija ägestus tõsiselt Seaduse pärast ja võeti taevasse.
\par 59 Hananja, Asarja ja Miisael uskusid ja päästeti tulest.
\par 60 Taaniel oma vagaduses kisti lõvide suust.
\par 61 Nõnda pidage meeles põlvest põlve, et ükski, kes Issanda peale loodab, ei jää jõuetuks!
\par 62 Ärge siis kartke patuse mehe sõnu, sest tema au saab sõnnikuks ja ussipuruks!
\par 63 Täna on ta ülendatud, aga homme teda enam ei ole, sest ta pöördub tagasi põrmu ja tema kavatsus läheb tühja.
\par 64 Lapsed, olge kui mehed ja jääge kindlaks Seaduses, sest selle eest austatakse teid!
\par 65 Vaata, Siimon, teie vend, on siin. Ma tean, et tema on tark mees. Kuulake alati teda! Tema olgu teile isaks!
\par 66 Ja Juudas Makkabi, jõult tugev noorusest alates, tema olgu teie sõjaväe pealik ja juhtigu suguharude sõda!
\par 67 Ja koguge endi juurde kõik, kes teevad Seaduse järgi, ja makske kätte oma rahva eest!
\par 68 Tasuge paganaile ja pidage Seaduse käske!”
\par 69 Ja ta õnnistas neid. Siis ta võeti ära oma vanemate juurde.
\par 70 Ta suri aastal sada nelikümmend kuus ja ta maeti oma isade hauda Moodeinis ning kogu Iisrael leinas teda suure leinakaebusega.

\chapter{3}

\section*{Ülistuslaul Juudas Makkabist}

\par 1 Siis tõusis Juudas, tema poeg, keda hüüti Makkabiks, tema asemele.
\par 2 Kõik tema vennad ja kõik, kes olid läbi käinud tema isaga, toetasid teda, ja nad sõdisid rõõmuga Iisraeli sõda. 
\par 3 „Tema kasvatas oma rahva kuulsust. Otse hiiglasena riietus ta soomusrüüsse, pani oma relvad vööle, juhtis võitlusi, kaitses oma mõõgaga sõjaleeri.
\par 4 Oma tegudes oli ta lõviga sarnane, oli otsekui noor lõvi, kes möirgab saagi pärast.
\par 5 Ta jälitas jumalavallatuid, otsis nad üles ja põletas ära need, kes tema rahvast eksitasid.
\par 6 Kartusest tema ees värisesid jumalavallatud, kohkusid kõik, kes tegid seadusevastaseid tegusid, jah, vabastus sündis tema käe läbi.
\par 7 Ta vihastas paljusid kuningaid, aga rõõmustas Jaakobit oma tegudega ja mälestus temast on õnnistuseks igavesti.
\par 8 Ta käis läbi Juuda linnad, hävitas sealt jumalavallatud ja pööras viha ära Iisraeli pealt.
\par 9 Tema kuulsus levis maailma äärteni ja ta kogus kokku laiali pillatud.” 

\section*{Juuda esimesed võidud}

\par 10 Aga Apolloonios kogus kokku paganad ja suure sõjaväe Samaariast Iisraeli vastu sõdima.
\par 11 Kui Juudas sai sellest teada, siis ta läks välja temale vastu, võitis ja tappis tema. Ja paljud langesid haavatuna ning ülejäänud põgenesid.
\par 12 Nad võtsid ära nende sõjavarustuse. Juudas võttis aga Appollooniose mõõga ja võitles alati sellega.
\par 13 Kui Süüria väepealik Seeron kuulis, et Juudas oli kogunud enese ümber salga ustavaid, kes olid valmis sõtta minema,
\par 14 siis ta ütles: „Ma tahan enesele nime teha ja kuningriigis kuulsaks saada, ma tahan võidelda Juuda ja tema meeste vastu, kes ei täida kuninga käsku.”
\par 15 Ta läks taas üles ja koos temaga tugev jumalakartmatute sõjavägi, kes pidi teda aitama kätte maksta Iisraeli lastele.
\par 16 Kui ta jõudis Beet-Hooroni mägiteeni, siis läks Juudas väikese salgaga talle vastu.
\par 17 Aga nähes sõjaväge talle vastu tulevat, ütlesid mehed Juudale: „Kuidas meie, keda on nii vähe, suudame sõdida nii võimsa hulga vastu? Ja me oleme väsinud, olles täna söömata.”
\par 18 Aga Juudas ütles: „Hõlpsasti võib juhtuda, et paljud antakse väheste kätte, ja taeva ees ei ole vahet, kas päästa palju või pisku läbi.
\par 19 Sest võit ei olene sõjaväe suurusest, vaid jõud tuleb taevast.
\par 20 Nemad tulevad meie vastu suure ülbuse ja ülekohtuga, hävitama meid ja meie naisi ja lapsi, et meid paljaks riisuda.
\par 21 Meie aga võitleme oma hingede ja oma õiguste eest.
\par 22 Tema ise põrmustab nad meie nähes, teie aga ärge kartke neid!”
\par 23 Kui ta kõne oli lõpetanud, siis tungis ta äkitselt neile kallale ja Seeron ning tema sõjavägi hävitati tema nähes.
\par 24 Siis nad ajasid neid taga Beet-Hooroni mägiteel kuni tasandikuni. Ja neist langes ligi kaheksasada meest ning ülejäänud põgenesid vilistite maale.
\par 25 Nüüd hakati kartma Juudast ja tema vendi ja hirm valdas ümberkaudseid paganaid.
\par 26 Tema kuulsus jõudis kuningani ja kõik rahvad jutustasid Juuda võitlustest. 

\section*{Antiohhos valmistub sõjakäiguks}

\par 27 Kui kuningas Antiohhos neid sõnumeid kuulis, siis ta vihastas väga ja läkitas käskjalad ning laskis koguda kõik oma kuningriigi jõud, väga tugeva sõjaväe.
\par 28 Ta avas oma varaaida ja andis sõjameestele aastapalga, käskides neid valmis olla igaks olukorraks.
\par 29 Aga kui ta nägi, et raha varaaidast lõppes ja maa sissetulekud olid väikesed segaduse ja kahju tõttu, mis ta oli maale toonud ammust ajast kehtinud seaduste tühistamisega,
\par 30 siis ta kartis, et temal midagi ei ole, nagu juba varem oli mõnikord sündinud, kulutuseks ja kingitusteks, mida ta varem oli andnud helde käega ja rikkalikumalt kui endised kuningad.
\par 31 Ja tema hing oli suures kimbatuses. Ta otsustas siis minna Pärsiasse ja võtta maksu sealsetest piirkondadest ning koguda palju raha. 

\section*{Lüüsias hakkab valitsema}

\par 32 Ning ta jättis Lüüsiase, auväärse ja kuninglikust soost mehe riigi asju ajama Eufrati jõest kuni Egiptuse piirideni
\par 33 ja kasvatama tema poega Antiohhost, kuni ta ise tagasi tuleb.
\par 34 Ta andis temale poole sõjaväge ja elevandid ning meelevalla kõige üle, mis temal endal kavas oli, nõndasamuti Juudamaa ja Jeruusalemma elanike üle,
\par 35 et ta saadaks nende vastu sõjaväe purustama ja hävitama Iisraeli väge ning seda, mis Jeruusalemmast oli järele jäänud, ja kaotama sellest paigast mälestuse neist,
\par 36 ja et ta pidi asustama võõramaalasi kõigisse nende paigusse ning liisu läbi jaotama nende maa.
\par 37 Kuningas ise võttis järelejäänud poole sõjaväge ja läks aastal sada nelikümmend seitse Antiookiast, oma kuninglikust linnast, et minna üle Eufrati ja läbida kõrgendikku. 

\section*{Nikanor ja Gorgias tungivad Juudamaale}

\par 38 Lüüsias valis Ptolemaiose, Dorümenese poja, ja Nikanori ja Gorgiase, vaprad mehed kuninga sõprade hulgast,
\par 39 ja saatis koos nendega nelikümmend tuhat jalameest ja seitse tuhat ratsanikku Juudamaale rüüstama seda kuninga käsu kohaselt.
\par 40 Nad läksid siis kogu oma sõjaväega ja tulid ning lõid leeri üles lagendikule Emmause lähedal.
\par 41 Kui piirkonna kaupmehed neist kuulda said, siis nad võtsid väga palju hõbedat ja kulda ning jalaraudu, ja tulid leeri, et osta Iisraeli lapsi orjadeks. Ja nendega ühinesid Süüria ja vilistite maa sõjaväed.
\par 42 Kui Juudas ja tema vennad nägid, et kurjus üha kasvas ja et sõjaväed olid leeris nende aladel, ja kui nad said teada, mida kuningas oli käskinud teha rahva hävitamiseks ja hukkamiseks,
\par 43 siis nad ütlesid üksteisele: „Takistagem meie rahva hävingut ja võidelgem oma rahva ja pühamu eest!”
\par 44 Ja rahvahulk kogunes, et olla valmis võitluseks ja et palvetada ning anuda armu ja halastust.
\par 45 Aga Jeruusalemm oli tühi nagu kõrb, tema lastest ei olnud sisseminejat ega väljatulijat, pühamu oli ära tallatud ja kindluses olid võõrad, see oli eluasemeks paganaile. Rõõm oli võetud Jaakobist, vaikisid vile ja kannel.
\par 46 Siis nad kogunesid ja tulid Mispasse, mis on Jeruusalemma vastas, sest Mispa oli varem olnud Iisraeli palvepaik.
\par 47 Ja nad paastusid sel päeval ning panid endale kotiriided selga ja tuhka pähe, käristades oma riided lõhki.
\par 48 Nad rullisid lahti Seaduse raamatu, sest eks paganadki küsi nõu oma ebajumalate kujudelt?
\par 49 Ja nad tõid välja preestrirüüd, esmaannid ja kümnised. Nad käskisid ette astuda neid nasiire+ , kelle aeg oli täis saanud.
\par 50 Siis hüüdsid nad suure häälega taeva poole, üteldes: „Mida me nendega teeme ja kuhu me need viime?
\par 51 Sest sinu pühamu on ära tallatud ja teotatud ja sinu preestrid on leinas ja alanduses.
\par 52 Vaata, paganad on kogunenud meid hävitama. Sina tead, mida nad meie vastu kavatsevad.
\par 53 Kuidas suudame neile vastu panna, kui sina meid ei aita?”
\par 54 Ja nad puhusid pasunaid ning hüüdsid suure häälega.
\par 55 Seejärel seadis Juudas rahvale juhid, tuhande-, saja-, viiekümne- ja kümnepealikud.
\par 56 Seaduse kohaselt ta käskis neid, kes olid kodasid ehitanud, naisi võtnud või viinamägesid istutanud, ja argu, et igaüks neist läheks koju.
\par 57 Siis läks sõjavägi teele ja lõi leeri üles Emmausest lõuna poole.
\par 58 Ja Juudas ütles: „Pange vööd vööle ja olge vaprad! Olge homme varakult valmis võitlema nende paganatega, kes on kogunenud meie vastu, et hävitada meid ja meie pühamut!
\par 59 Sest parem on meil surra sõjas kui näha meie rahva ja pühamu hukkumist.
\par 60 Aga sündigu nõnda, nagu on taeva tahtmine!”

\chapter{4}

\section*{Juudas võidab Gorgiase}

\par 1 Gorgias võttis aga viis tuhat jalameest ja tuhat valitud ratsanikku ning sõjavägi läks öösel teele,
\par 2 et juutide leerile kallale tungida ja neid ootamatult lüüa. Mehed kindlusest olid temale teejuhtideks.
\par 3 Aga kui Juudas sellest kuulda sai, läks ka tema koos vapratega teele, et lüüa Emmauses olevat kuninga sõjaväge,
\par 4 kuni väeosad olid alles hajali väljaspool leeri.
\par 5 Kui nüüd Gorgias öösel Juuda leeri tuli, ei leidnud ta seal kedagi. Ta otsis neid siis mäestikust, sest ta mõtles: „Nad on meie eest põgenenud!”
\par 6 Aga päeva koites ilmus Juudas tasandikule kolme tuhande mehega, kuigi neil ei olnud soovitud hulgal kaitsevarustust ega mõõku.
\par 7 Kui nad nägid paganate hästi kindlustatud ja relvastatud sõjaleeri ning selle ümber võitlusvalmis ratsaväge,
\par 8 siis ütles Juudas meestele, kes koos temaga olid: „Ärge kartke nende suurt hulka ja ärge kohkuge nende rünnaku ees!
\par 9 Meenutage, kuidas meie vanemad päästeti Punases meres, kui vaarao sõjaväega neid taga ajas!
\par 10 Hüüdkem nüüd appi taevast, et ta halastaks meie peale ja meenutaks meie vanemate lepingut ja et ta selle sõjaleeri täna purustaks meie silma ees,
\par 11 et kõik paganad teaksid, et on olemas see, kes lunastab ja päästab Iisraeli!”
\par 12 Kui nüüd muulased oma silmad üles tõstsid ja nägid neid vastu tulevat,
\par 13 siis tulid nad leerist välja sõdima. Juuda mehed aga puhusid pasunaid.
\par 14 Seejärel põrkasid sõjaväed kokku, paganad võideti täielikult ja need põgenesid tasandikule.
\par 15 Need, kes jäid viimasteks, langesid kõik mõõga läbi. Neid aeti taga kuni Geserini ja Idumea lagendikkudeni ja Asdodi ning Jamniani. Ja neist langes ligi kolm tuhat meest.
\par 16 Kui Juudas koos sõjaväega jälitamast tagasi tuli,
\par 17 siis ütles ta rahvale: „Ärge himustage saaki, sest võitlus on meil alles ees!
\par 18 Gorgias oma sõjaväega on ju meie lähedal mäestikus. Kuid astuge nüüd meie vaenlaste vastu ja võidelge nendega, siis võite pärast vabalt saaki võtta!”
\par 19 Vaevalt oli Juudas kõneluse lõpetanud, kui nähti ühte väeosa mäestikust välja tulevat.
\par 20 See nägi nüüd, et teised olid löödud põgenema ja leer oli pandud põlema, sest nähtav suits osutas, mis oli sündinud.
\par 21 Seda taibates nad kohkusid väga. Kui nad nägid ka, et Juuda leer oli lagendikul võitlusvalmis,
\par 22 siis nad kõik põgenesid muulaste maale.
\par 23 Juudas pöördus aga tagasi leeri saagi juurde. Nad said palju kulda ja hõbedat, siniseid ja meripurpurseid kangaid ja suuri rikkusi.
\par 24 Ja koju minnes lasksid nad taeva poole kõlada kiitus- ja ülistuslaule, „et tema on hea, et tema heldus kestab igavesti”.
\par 25 Jah, suure võidu sai Iisrael sel päeval! 

\section*{Lüüsiase esimene sõjakäik}

\par 26 Aga võõramaalased, kes olid pääsenud, läksid ja teatasid Lüüsiasele kõigest, mis oli sündinud.
\par 27 Seda kuuldes oli ta vapustatud ja meeleheitel, et Iisraeliga ei olnud sündinud tema tahtmist mööda ja et temale antud kuninga käsk oli jäänud täitmata.
\par 28 Ja järgmisel aastal kogus ta kuuskümmend tuhat valitud jalameest ja viis tuhat ratsanikku, et taas sõdida nende vastu.
\par 29 Siis nad tungisid Idumeasse ja lõid leeri üles Beet-Suuri, aga Juudas läks neile vastu kümne tuhande mehega.
\par 30 Kui ta nägi seda vägevat sõjaleeri, siis ta palvetas ja ütles: „Ole kiidetud, Iisraeli päästja, kes lõpetasid vägilase rünnaku oma sulase Taaveti käe läbi ja andsid muulaste leeri Sauli poja Joonatani ja tema sõjariistade kandja kätte!
\par 31 Jäta nüüd see leer oma Iisraeli rahva kätte ja jäägu nad häbisse oma sõjaväe ja ratsanikega!
\par 32 Sisenda neisse argust ja halva nende suur enesekindlus, et nad väriseksid oma tulevase kaotuse pärast!
\par 33 Langeta nad sinu armastajate mõõga läbi, et kõik, kes sinu nime tunnevad, ülistaksid sind kiituslauludega!”
\par 34 Siis väed ründasid teineteist ja Lüüsiase leerist langes ligi viis tuhat meest lähivõitluses.
\par 35 Kui Lüüsias nägi oma sõjajõudude kaotust ja et Juuda mehed olid julged, valmis ausaks eluks või surmaks, siis ta läks Antiookiasse ja värbas sealt uut väge, et tugevamana jälle Juudamaale tulla. 

\section*{Templi puhastamine ja pühitsemine}

\par 36 Juudas ja tema vennad aga ütlesid: „Vaata, meie vaenlased on hävitatud! Mingem nüüd pühamut puhastama ja pühitsema!”
\par 37 Siis tuli kogu leer kokku ja nad läksid Siioni mäele.
\par 38 Kui nad nägid, et pühamu oli rüüstatud, ohvrialtar teotatud, väravad põletatud ja et eesõuedes kasvasid põõsad otsekui metsas või mäel ja et külgmised kambrid olid hävitatud,
\par 39 siis nad käristasid oma riided lõhki, tegid suurt kaebekisa, raputasid tuhka pähe
\par 40 ja heitsid silmili maha. Ja nad puhusid märguandeks pasunaid ning hüüdsid taeva poole.
\par 41 Juudas käskis siis mehi minna võitlema kindluses olijate vastu, senikaua kui tema pühamut puhastab.
\par 42 Ja ta valis nüüd preestreiks niisugused, kes olid laitmatud ja armastasid Seadust.
\par 43 Need puhastasid pühamu ja viisid rüvetatud kivid rüvedasse paika.
\par 44 Siis nad pidasid nõu, mida tuleks teha rüvetatud põletusohvrialtariga.
\par 45 Neile tuli hea mõte: altar lõhkuda, et see ei jääks neile teotuseks, sest paganad olid selle ju rüvetanud. Ja nad lõhkusid altari.
\par 46 Aga kivid panid nad templimäele sobivasse paika, seniks kui tuleb prohvet, kes annab nende kohta seletuse.
\par 47 Siis nad võtsid tahumata kive, nagu Seadus ette näeb, ja ehitasid uue altari, endisega sarnase.
\par 48 Nad parandasid ka pühamut ja koja sisemust ning pühitsesid õuesid.
\par 49 Nad valmistasid uued pühad riistad ning viisid templisse lambijala, suitsutusaltari ja laua.
\par 50 Siis nad suitsutasid altaril ja süütasid lambid, mis olid lambijalal, ja need valgustasid templit.
\par 51 Ja nad panid laua peale leibu, riputasid ette eesriided ning lõpetasid kõik tööd, mis nad olid ette võtnud.
\par 52 Aga üheksanda kuu, see on kislevikuu kahekümne viiendal päeval, aastal sada nelikümmend kaheksa, tõusid nad varahommikul
\par 53 ja ohverdasid seadusekohaseid ohvreid uuel põletusohvrialtaril, mille nad olid teinud.
\par 54 Just selsamal aastaajal, selsamal päeval, mil paganad olid selle rüvetanud, pühitseti see nüüd uuesti laulude, tsitrite, kannelde ja simblite saatel.
\par 55 Ja kogu rahvas heitis silmili maha; nad palvetasid ja saatsid tänu selle poole taevas, kes nende töö oli lasknud korda minna.
\par 56 Nad pühitsesid altarit kaheksa päeva ja tõid rõõmuga põletusohvreid ning ohverdasid tänu- ja kiitusohvreid.
\par 57 Ja nad kaunistasid templi esikülje kuldvanikute ja kilpidega, tegid uued väravad ja kambrid ning panid neile uksed ette.
\par 58 Siis oli rahval väga suur rõõm, et paganate häbistus oli kõrvaldatud.
\par 59 Juudas ja tema vennad ning kogu Iisraeli kogudus seadsid nüüd, et altari pühitsuspäevi pidi hõiskamise ja rõõmuga peetama igal aastal omal ajal kaheksa päeva, alates kislevikuu kahekümne viiendast päevast.
\par 60 Selsamal ajal ehitasid nad ümber Siioni mäe kõrged müürid ja tugevad tornid, et paganad ei saaks jälle tulla seda tallama, nõnda nagu nad enne olid teinud.
\par 61 Ja selle kaitseks paigutasid nad sinna sõjaväge. Nad kindlustasid ka Beet-Suuri, et see oleks kaitstud, et rahval oleks kindlus Idumea vastu.

\chapter{5}

\section*{Sõjakäik idumealaste ja ammonlaste vastu}

\par 1 Aga kui ümberkaudu elavad paganad kuulsid, et altar oli ehitatud ja pühamu oli pühitsetud nagu ennegi, siis nad vihastasid väga.
\par 2 Nad otsustasid hävitada Jaakobi soo, kes elas nende keskel, ja nad hakkasid rahva hulgas tapma ja hävitama.
\par 3 Juudas sõdis siis Eesavi järeltulijate vastu Idumeas Akrabattene paikkonnas, sest need piirasid Iisraeli. Ta tekitas neile suure kaotuse, alistas nad ja võttis neilt saaki.
\par 4 Ta meenutas ka Bajani poegade kuritegu; need olid rahvale püüniseks ja lõksuks, varitsedes teede peal.
\par 5 Ta sulges nad nende tornidesse, piiras neid ja hävitas nad sootuks, põletades nende tornid koos kõigiga, kes sees olid.
\par 6 Siis ta läks edasi ammonlaste juurde ja kohtas seal tugevat sõjaväge ning palju rahvast Timoteose juhtimisel.
\par 7 Ta pidas nende vastu palju lahinguid, lõi nad puruks ja võitis nad.
\par 8 Ja olles vallutanud Jaaseri ning selle tütarlinnad, pöördus ta tagasi Juudamaale. 

\section*{Ettevalmistus uuteks sõjakäikudeks}

\par 9 Siis kogunesid Gileadi paganad Iisraeli laste vastu, kes elasid nende alal, et neid hävitada. Need aga põgenesid Datema kindlusesse.
\par 10 Nad saatsid kirju Juudale ja tema vendadele, teatades: „Meie ümberkaudsed paganad on kogunenud meie vastu, et meid hävitada.
\par 11 Nad valmistuvad tulema, et vallutada kindlus, kuhu oleme põgenenud, ja Timoteos on nende sõjaväe juht.
\par 12 Tule siis nüüd ja päästa meid nende käest, sest meist on juba paljud langenud!
\par 13 Kõik meie vennad, kes elasid Toobimaal, on tapetud, nende naised ja lapsed ning vara on ära viidud. Seal on surmatud ligi tuhat meest.”
\par 14 Kui kirja parajasti loeti, vaata, siis tulid teised käskjalad Galileast, riided lõhki käristatud, tuues samasuguseid sõnumeid,
\par 15 üteldes, et nende vastu on kogunetud Ptolemaisist, Tüürosest ja Siidonist, ka kogu paganlikust Galileast, et neid hävitada.
\par 16 Kui Juudas ja rahvas neid sõnumeid kuulsid, siis kogunes suur rahvakogu nõu pidama, mida tuleks teha oma vendade heaks, kes olid kitsikuses ja sõjaohus.
\par 17 Ja Juudas ütles oma vennale Siimonile: „Vali enesele mehi, mine ja päästa oma Galileas olevad vennad! Mina ja mu vend Joonatan läheme aga Gileadi.”
\par 18 Ta jättis Joosepi, Sakariase poja, ja rahvajuhi Asarja koos ülejäänud sõjaväega Juudamaale selle kaitseks.
\par 19 Ja ta käskis neid, üteldes: „Juhtige seda rahvast, aga ärge hakake sõdima paganate vastu, enne kui oleme tagasi tulnud!”
\par 20 Siimonile jaotati kolm tuhat meest Galileasse minekuks, Juudale aga kaheksa tuhat meest Gileadi jaoks. 

\section*{Võitlused Galileas ja Gileadis}

\par 21 Siimon läks siis Galileasse ja pidas palju lahinguid paganate vastu. Ta lõi paganad pihuks ja põrmuks
\par 22 ning ajas neid taga kuni Ptolemaisi väravani. Paganaid langes ligi kolm tuhat meest ja ta võttis neilt saaki.
\par 23 Siis ta võttis enesega kaasa Galileas ja Arbattas olevad juudid, ka nende naised ja lapsed ning kõik, mis neil oli, ja viis nad Juudamaale suure rõõmuga.
\par 24 Aga Juudas Makkabi ja Joonatan, tema vend, läksid üle Jordani ja käisid kolm päevateekonda kõrbes.
\par 25 Seal kohtasid nad nabatlasi, kes neid sõbralikult vastu võtsid ja neile jutustasid kõigest, mis Gileadis oli nende vendadega juhtunud,
\par 26 ja et paljud neist on suletud Bosrasse, Beserisse, Alemasse, Hasfosse, Makedisse ja Karnaimi - need kõik on kindlad ja suured linnad;
\par 27 et teisteski Gileadi linnades on sissepiiratuid. Homseks on otsustatud kindluste ründamine ja vallutamine ning kõigi nende hävitamine samal päeval.
\par 28 Juudas koos oma sõjaväega pöördus siis otsekohe kõrbe, Beserisse viivale teele. Ta vallutas linna, tappis kõik mehed mõõgateraga, võttis saagiks kogu nende varanduse ja põletas linna tulega.
\par 29 Aga öösel läks ta sealt ära ja tuli kindluse juurde.
\par 30 Kui siis hommik koitis, tõstsid nad silmad üles, ja vaata, seal oli loendamatult palju rahvast kandmas redeleid ja seadmeid kindluse vallutamiseks, olles juba sõdimas seesolijate vastu.
\par 31 Kui Juudas nägi, et võitlus oli alanud ja lärm tõusis linnas taevani pasuna puhumise ja suure kisa saatel,
\par 32 siis ta ütles oma sõjameestele: „Täna sõdige oma vendade eest!”
\par 33 Ja ta ründas selja tagant kolme väeosaga. Nad puhusid pasunaid ja hüüdsid palvesõnu.
\par 34 Aga kui Timoteose sõjavägi märkas, et see oli Makkabi, siis nad põgenesid tema eest. Ja tema tekitas neile suure kaotuse ning sel päeval langes neist kaheksa tuhat meest.
\par 35 Seejärel pöördus ta Alema vastu, ründas seda ja vallutas selle, tappis kõik mehed, võttis saagiks nende varanduse ja põletas linna.
\par 36 Sealt läks ta edasi ja vallutas Hasfo, Makedi, Bosra ning teised Gileadi linnad.
\par 37 Pärast neid sündmusi kogus Timoteos uue sõjaväe ja lõi leeri üles Rafooni ette, teisele poole mägioja.
\par 38 Juudas läkitas aga mehi leeri uurima. Need teatasid temale, üteldes: „Kõik meie ümberkaudsed paganad on kogunenud tema juurde, väga suur sõjavägi.
\par 39 Araablasigi on ta enesele abiks palganud ja need on leeri üles löönud teisele poole mägioja, olles valmis tulema võitlusse sinu vastu.” Siis läks Juudas neile vastu.
\par 40 Kui Juudas ja tema sõjavägi lähenesid voolavale mägiojale, siis ütles Timoteos oma sõjaväe pealikuile: „Kui tema tuleb enne üle meie juurde, siis me ei suuda talle vastu seista, sest ta on meist vägevam.
\par 41 Aga kui ta kardab ja jääb leeri teisele poole jõge, siis läheme meie üle tema juurde ja võidame tema.”
\par 42 Kui Juudas voolavale mägiojale oli lähenenud, siis ta paigutas kirjatundjad oja kaldale ja käskis neid, üteldes: „Ärge laske ühtki inimest leeri jääda, vaid kõik mingu sõdima!”
\par 43 Siis läks ta ise esimesena üle ja kogu rahvas tema järel. Ta lõi laiali kõik paganad, kes viskasid maha oma sõjariistad ja põgenesid Karnaimi pühapaika.
\par 44 Seejärel vallutasid nad linna ja põletasid tulega pühamu koos kõigi seesolijatega. Nõnda alistati Karnaim ja nad ei suutnud enam Juudale vastu panna.
\par 45 Juudas kogus nüüd kõik Gileadis olevad Iisraeli lapsed, väikesed ja suured, nende naised ja lapsed ning varanduse, väga suure inimhulga, Juudamaale minekuks.
\par 46 Ja nad tulid kuni Efronini. See on suur linn kitsa tee suudmes, väga hästi kindlustatud. Sellest ei saanud mööduda ei paremalt ega vasakult, vaid tuli minna otse läbi.
\par 47 Aga linna elanikud sulgesid neile tee ja kuhjasid väravate taha kive.
\par 48 Juudas läkitas siis neile rahusõnumeid, üteldes: „Meie tahame sinu maast läbi minna, et jõuda oma maale. Ükski ei tee teile kurja, me läheme ainult jalgsi läbi.” Nemad aga ei tahtnud temale avada.
\par 49 Juudas käskis nüüd leeris kuulutada, et igaüks peab jääma paika, kus ta parajasti on.
\par 50 Ja sõjamehed jäid paigale. Siis nad sõdisid linna vastu kogu selle päeva ja kogu öö ja linn anti tema kätte.
\par 51 Ja ta hukkas mõõgateraga kõik mehed, hävitas linna, võttis selle varanduse ja läks linnast läbi üle tapetute.
\par 52 Siis nad läksid üle Jordani suurele lagendikule, mis on vastu Beet-Seani.
\par 53 Ja Juudas kogus mahajäänuid ning ergutas rahvast kogu teekonnal, kuni ta jõudis Juudamaale.
\par 54 Siis nad läksid üles Siioni mäele hõiskamise ja rõõmuga ning ohverdasid põletusohvreid, et teekonnal - kuni õnneliku tagasitulekuni - ei olnud ükski neist langenud. 

\section*{Kaotus Jamnia juures}

\par 55 Aga neil päevil kui Juudas Joonataniga oli Gileadimaal ja tema vend Siimon Galileas Ptolemaisi ees,
\par 56 said Joosep, Sakariase poeg, ja Asarja, sõjaväe pealik, kuulda kangelastegudest ja võitlustest, mis teistel olid olnud.
\par 57 Siis nad ütlesid: „Teeme ka meie endile nime ja läheme sõdima paganate vastu, kes meil ümberkaudu on!”
\par 58 Nad andsid käsu sõjameestele, kes neile allusid, ning läksid Jamnia poole.
\par 59 Aga Gorgias ja tema mehed tulid linnast välja sõdima nende vastu.
\par 60 Joosep ja Asarja löödi põgenema ja neid aeti taga kuni Juudamaa piirideni. Sel päeval langes Iisraeli rahvast ligi kaks tuhat meest.
\par 61 Nõnda sai rahvas suure kaotuse, sest nad ei olnud kuulda võtnud Juudast ja tema vendi, vaid arvasid ise teha võivat kangelastegusid.
\par 62 Nemad aga ei olnud nende meeste soost, kelle käte varal Iisrael päästeti. 

\section*{Edu Idumeas ja vilistite maal}

\par 63 Aga seda meest Juudast ja tema vendi austati väga kogu Iisraelis ning kõigi paganate keskel, kus iganes tema nime kuuldi.
\par 64 Ja tema juurde koguneti õnne soovima.
\par 65 Pärast seda läks Juudas koos vendadega sõdima Eesavi järglaste vastu lõunapoolsel maal. Ta lõi Hebronit ja selle tütarlinnu, kiskus maha selle kindluse ja põletas ära tornid, mis olid ümberringi.
\par 66 Seejärel asus ta teele, et minna vilistite maale. Ja ta läks Marisast läbi.
\par 67 Sel ajal langes lahingus preestreid, kes tahtsid teha kangelastegusid, minnes aga võitlusse mõtlematult.
\par 68 Juudas pöördus nüüd Asdodisse vilistite maal, kiskus maha nende altarid, põletas nende jumalakujud, riisus linnade varandused ja läks tagasi Juudamaale.

\chapter{6}

\section*{Antiohhos Epifanese valitsemine lõpeb}

\par 1 Kuningas Antiohhos sai minnes läbi kõrgmaade kuulda, et Pärsias on linn Elümais, kuulus oma rikkuse, hõbeda ja kulla poolest,
\par 2 et sealne tempel on eriti rikas ja et seal on kullast soomusrüüd, rinnakilbid ja sõjariistad, mis sinna oli jätnud Makedoonia kuningas Aleksander, Filippose poeg, kes oli kreeklaste esimene kuningas.
\par 3 Siis ta läks ja püüdis linna vallutada, et seda riisuda, aga ei suutnud, sest asi sai linnaelanikele teatavaks.
\par 4 Nad tõusid võitluseks tema vastu ning ta pidi põgenema ja ta läks sealt ära suures meelekibeduses, et pöörduda tagasi Paabelisse.
\par 5 Siis tuli keegi Pärsiasse temale teatama, et need sõjaväed, mis olid läinud Juudamaale, olid löödud põgenema,
\par 6 et Lüüsias oli läinud sinna tugeva sõjaväe eesotsas, oli aga löödud põgenema ja juudid olid saanud tugevamaks sõjariistade, vägede ja rohke saagi poolest, mida nad võtsid löödud vägedelt.
\par 7 Nad olid maha kiskunud ka jäleduse, mille ta oli püstitanud altari peale Jeruusalemmas, ja olid pühamu ümbritsenud kõrgete müüridega nagu ennegi, nõndasamuti ka oma linna Beet-Suuri.
\par 8 Ja sündis, kui kuningas neid sõnumeid kuulis, et ta kohkus ja oli väga vapustatud. Ta heitis voodisse ja jäi haigeks meelekibedusest, et ei olnud sündinud nõnda, nagu tema oli tahtnud.
\par 9 Ta oli voodis kaua aega, sest üha uuesti valdas teda suur meelekibedus ja ta arvas, et ta sureb.
\par 10 Siis ta kutsus kõik oma sõbrad ja ütles neile: „Uni põgeneb mu silmist ja süda on murest murtud.
\par 11 Ma mõtlesin oma südames, missugusesse kitsikusse ja suurde segadusse ma olen jõudnud, kus ma nüüd olen. Ma olin ju hea ja armastatud oma valitsemisajal.
\par 12 Aga nüüd ma meenutan seda kurja, mis ma Jeruusalemmas tegin. Tõesti, ma võtsin kõik seal olevad hõbe- ja kuldriistad ning saatsin sõjaväe hävitama süütuid Juuda elanikke.
\par 13 Ma mõistan, et sellepärast on need õnnetused mind tabanud, ja vaata, ma suren võõral maal suures meelekibeduses.” 

\section*{Antiohhos Eupator astub aujärjele}

\par 14 Siis ta kutsus Filippose, ühe oma sõpradest, ja seadis tema kogu oma kuningriigi üle.
\par 15 Ja ta andis temale oma krooni, mantli ja pitserisõrmuse, et ta juhataks tema poega Antiohhost ja kasvataks temast valitseja.
\par 16 Ja kuningas Antiohhos suri seal aastal sada nelikümmend üheksa.
\par 17 Kui Lüüsias teada sai, et kuningas oli surnud, siis ta seadis tema asemele valitsema tema poja Antiohhose, keda ta oli kasvatanud tema nooruse päevil, ja pani temale nimeks Eupator. 

\section*{Juudas Makkabi piirab Jeruusalemma}

\par 18 Aga need, kes kindluses olid, sulgesid pühamu Iisraelile igalt poolt, püüdsid üha kurja teha ja paganaid toetada.
\par 19 Siis Juudas kavatses need hävitada ja kutsus kogu rahva neid piirama.
\par 20 Nad tulid kokku ning piirasid neid aastal sada viiskümmend.
\par 21 Aga mõningad pääsesid sissepiiramisest ja nendega ühines jumalakartmatuid Iisraelist.
\par 22 Nad läksid kuninga juurde ja ütlesid: „Kui kaua see kestab, et sa ei tee õigust ega maksa kätte meie vendade eest?
\par 23 Me olime valmis teenima sinu isa, täitma tema käske ja jälgima tema korraldusi.
\par 24 Sellepärast meie rahva pojad piirasid kindlust, ja kes meist iganes tabati, see tapeti, ja meie omand rööviti.
\par 25 Nad sirutasid käe mitte ainult meie, vaid ka kõigi oma naabrite vastu.
\par 26 Vaata, nüüd on nad leeri üles löönud Jeruusalemma kindluse alla, et seda vallutada. Ja nad on kindlustanud pühamu ning Beet-Suuri.
\par 27 Kui sa nüüd kiiresti neist ette ei jõua, siis nad teevad veel rohkemgi, ja sina ei suuda neid enam peatada.” 

\section*{Antiohhos Eupatori ja Lüüsiase sõjakäik}

\par 28 Seda kuuldes kuningas vihastas. Ta kogus kokku kõik oma sõbrad, sõjaväepealikud ja need, kes juhtisid ratsaväge.
\par 29 Tema juurde tuli palgasõdureid ka teistest riikidest ja meresaartelt.
\par 30 Arvuliselt oli tema sõjaväge sada tuhat jalameest ja kakskümmend tuhat ratsanikku ning kolmkümmend kaks sõjaks õpetatud elevanti.
\par 31 Nad läksid läbi Idumea ja lõid leeri üles Beet-Suuri alla. Nad sõdisid kaua aega ja valmistasid piiramisseadmeid. Sissepiiratud aga tungisid välja, põletasid need tules ja võitlesid kui mehed.
\par 32 Ja Juudas läks teele Jeruusalemma kindluse alt ning lõi leeri üles Beet-Sakarja alla, kuninga leeri vastu.
\par 33 Aga kuningas liikus varahommikul oma võitlushimulise sõjaväega Beet-Sakarja poole. Väeosad seadsid endid siis võitlusvalmis ja puhuti pasunaid.
\par 34 Elevantidele pandi ette viinamarja- ja mooruspuumarja mahla, et neid ergutada võitluseks.
\par 35 Siis jaotati loomad väeosade vahel, ja iga elevandi kohta tuli tuhat meest, varustatud soomusrüüga ja peas vaskkiivrid. Ka oli iga looma juurde määratud viissada valitud ratsanikku.
\par 36 Need olid juba ennegi seal, kus loom oli, ja kuhu see läks, sinna läksid nemadki kaasa ega lahkunud kunagi tema juurest.
\par 37 Iga elevandi seljas oli tugev sõjariistadega varustatud puutorn, mis loomale oli selga seotud eriliste sidemetega, ja igas tornis oli neli sõjameest, kes sealt võitlesid, ja üks indialane, elevandijuht.
\par 38 Ülejäänud ratsavägi paigutati kummalegi poole, sõjaväe mõlemale tiivale, vaenlast häirima ja oma väeosi kaitsma.
\par 39 Kui nüüd päike paistis kuld- ja vaskkilpide peale, siis hiilgasid neist mäed ja särasid otsekui tulelondid.
\par 40 Osa kuninga sõjaväest hargnes kõrgetele mägedele, osa tasandikule. Nad astusid välja kindlalt ning korraldatult.
\par 41 Ja kõik, kes kuulsid nende sõjasalkade kära, inimhulkade sammumist ja sõjariistade tärinat, värisesid, sest see sõjavägi oli väga suur ja võimas.
\par 42 Aga Juudas ja tema sõjavägi lähenesid võitluseks, ja siis langes kuninga sõjaväest kuussada meest.
\par 43 Kui Eleasar Avaran nägi üht elevanti, kes oli soomustatud kuningliku soomustusega ja oli kõigist teistest loomadest suurem, ja paistis, et kuningas oli selle seljas,
\par 44 siis ta ohverdas enese, et päästa oma rahvas ja teha enesele igavene nimi:
\par 45 ta tormas julgesti sellele kallale keset võitlusrinnet, tappes paremalt ja vasakult, nõnda et taanduti kummaltki poolt.
\par 46 Siis ta puges elevandi alla ja alt torgates tappis looma, kes langes tema enese peale, nõnda et ta seal suri.
\par 47 Aga nähes kuninga vägevust ja sõjaväe pealetungi, tõmbusid nad selle ees tagasi. 

\section*{Beet-Suuri vallutamine ja Siioni mäe piiramine}

\par 48 Kuninga sõjamehed läksid nüüd Jeruusalemmas olijate vastu. Ja kuningas lõi leeri üles Juudamaa ning Siioni mäe vastu.
\par 49 Aga Beet-Suuris olijatega tegi ta rahu. Need olid sunnitud linnast lahkuma, sest neil ei olnud seal tagavarasid piiramise ajaks. Maal oli just hingamisaasta.
\par 50 Nõnda vallutas kuningas Beet-Suuri ning paigutas sinna väesalga seda valvama.
\par 51 Siis piiras ta kaua aega pühamut ja asetas selle ette heite- ja piiramisseadmeid, tule- ja kiviheitjaid, noolte pildujaid ja lingutajaid.
\par 52 Aga sissepiiratudki tegid kaitseseadmeid vaenlaste omade vastu ja võitlesid kaua.
\par 53 Kuid neil ei olnud aitades toitu, sest oli seitsmes aasta, ja need, kes olid pääsenud Juudamaa paganate käest, olid söönud ülejäägi talletatust.
\par 54 Seetõttu jäi pühamusse vähe mehi, sest nälg sundis neid ja nad valgusid igaüks oma kodupaika. 

\section*{Juutidele antakse usuvabadus}

\par 55 Kui Lüüsias kuulis, et Filippos, kelle kuningas Antiohhos oma eluajal oli seadnud kasvatama oma poega Antiohhost kuningaks,
\par 56 oli tagasi tulnud Pärsiast ja Meediast ühes kuningaga koos sinna läinud vägedega, ja et ta püüdis võtta valitsust enese kätte,
\par 57 siis tahtis ta kohe ära minna ja ütles kuningale, sõjaväepealikuile ja meestele: „Me jääme päev-päevalt nõrgemaks, meie toit on napp ja paik, mida piirame, on tugevasti kindlustatud. Ja meil on kohustus kuningriigi vastu.
\par 58 Ulatagem nüüd käsi neile meestele ja tehkem rahu nendega ning kogu nende rahvaga!
\par 59 Lubagem neid elada nende oma seaduste järgi nagu ennegi! Sest nende seaduste pärast, mis meie oleme tühjaks teinud, on nad vihastanud ja kõike seda teinud.”
\par 60 See kõne meeldis kuningale ja pealikuile ja ta saatis nende juurde rahutegijad. Ja nad nõustusid.
\par 61 Kuningas ja pealikud andsid siis neile vande. Seejärel tulid nad kindlusest välja
\par 62 ja kuningas läks Siioni mäele. Aga kui ta nägi selle paiga tugevat kindlustust, murdis ta antud vande ja käskis müürid ümberringi maha kiskuda.
\par 63 Siis ta läks rutates minema ja pöördus tagasi Antiookiasse. Ja kui ta leidis Filippose linna valitsemas, siis ta sõdis tema vastu ja vallutas linna väevõimuga.

\chapter{7}

\section*{Demeetrios saab Süüria kuningaks}

\par 1 Aastal sada viiskümmend üks tuli Demeetrios, Seleukose poeg, Roomast ja läks väheste meestega ühte rannikulinna ning sai seal kuningaks.
\par 2 Ja kui ta läks oma isade kuninglikku kotta, siis sõjavägi võttis kinni Antiohhose ja Lüüsiase, et viia need tema kätte.
\par 3 Aga teost teada saades ütles ta: „Ärge näidake mulle nende nägusid!”
\par 4 Siis sõjavägi tappis nad ja Demeetrios istus oma kuningriigi aujärjele.
\par 5 Nüüd tulid tema juurde Iisraelist kõik nurjatud ja jumalakartmatud mehed, ja neid juhtis Alkimos, kes tahtis saada ülempreestriks.
\par 6 Nad süüdistasid kuninga ees rahvast, üteldes: „Juudas ja tema vennad on tapnud kõik sinu sõbrad ja on meid ära ajanud meie maalt.
\par 7 Läkita siis nüüd üks mees, keda sa usaldad, ja tema mingu vaatama kõike seda hävitust, mis Juudas on teinud meile ja kuninga maale, ja ta nuhelgu neid ning kõiki nende aitajaid!”
\par 8 Kuningas valis siis Bakhidese, kes oli üks kuninga sõpradest ja valitses teisel pool jõge. Ta oli riigis tähtsaim mees ja kuningale ustav.
\par 9 Ja kuningas läkitas tema ning jumalakartmatu Alkimose, kellele ta andis ülempreestriameti, käskides teda Iisraeli lastele kätte maksta.
\par 10 Nad läksid teele ja tulid Juudamaale suure sõjaväega. Bakhides läkitas nüüd käskjalad petlike rahusõnadega Juuda ja tema vendade juurde.
\par 11 Nemad aga ei hoolinud nende sõnadest, sest nad nägid, et oli tuldud suure sõjaväega.
\par 12 Siis kogunes kirjatundjate kogu Alkimose ja Bakhidese juurde selgust saama.
\par 13 Hassiidid olid Iisraeli lastest esimesed, kes taotlesid neilt rahu,
\par 14 sest nad mõtlesid: „Sõjaväega on kaasa tulnud üks preester Aaroni soost: tema meile kurja ei tee.”
\par 15 See rääkiski nendega rahusõnu ning vandus neile, üteldes: „Meie ei soovi halba teile ega teie sõpradele.”
\par 16 Ja nad uskusid teda. Aga tema võttis neist kinni kuuskümmend meest ja tappis need selsamal päeval kirjasõna kohaselt:
\par 17 „Sinu pühade ihud ja nende veri valati ümber Jeruusalemma ja neil ei olnud matjat.”
\par 18 Siis haaras kogu rahvast hirm ja kartus nende ees ja nad ütlesid: „Ei ole neil tõde ega õigust, sest nad on rikkunud kokkulepet ja vannutud vannet.”
\par 19 Ja Bakhides läks ära Jeruusalemmast, lõi leeri üles Beet-Saiti ning laskis kinni võtta palju tema juurde põgenenud mehi, nõndasamuti mõningaid rahva hulgast, tappis need ja viskas suurde kaevu.
\par 20 Siis ta andis maa Alkimose hooleks ja jättis sõjaväge temale toetuseks. Bakhides ise läks aga kuninga juurde.
\par 21 Alkimos võitles nüüd ülempreestriameti pärast.
\par 22 Tema juurde kogunesid kõik, kes rahvast segadusse viisid. Need võtsid võimu Juudamaal ja tõid Iisraelile suure kaotuse.
\par 23 Kui Juudas nägi kõike seda kurja, mida Alkimos ja need, kes olid koos temaga, olid Iisraeli lastele teinud veel rohkem kui paganad,
\par 24 siis ta käis läbi kõik Juudamaa paikkonnad ümberkaudu ning maksis kätte põgenenud meestele, takistades nõnda nende liikumist maal.
\par 25 Aga kui Alkimos nägi, et Juudas ja need, kes olid koos temaga, olid saanud tugevaks, ja kui ta sai aru, et ta ei suuda neile vastu panna, siis ta läks tagasi kuninga juurde ja süüdistas neid rängalt. 

\section*{Juuda võit Nikanori üle}

\par 26 Kuningas läkitas siis Nikanori, ühe oma kuulsamaid väepealikuid, kes oli Iisraeli vaenlane ja vihkaja, ja käskis tal rahvas hävitada.
\par 27 Nikanor tuli suure sõjaväega Jeruusalemma ning läkitas Juudale ja tema vendadele petlikke rahusõnu, öeldes:
\par 28 „Ärgu olgu võitlust minu ja teie vahel! Mina tulen väheste meestega, et rahus näha teie palet.”
\par 29 Ta tuli Juuda juurde ning nad tervitasid teineteist rahumeeles. Vaenlased olid aga valmis Juudast vägisi ära viima.
\par 30 Kuid Juudale sai teatavaks see asjaolu, et Nikanor oli tagamõttega tema juurde tulnud, ja ta hakkas teda kartma ega tahtnud enam tema nägu näha.
\par 31 Kui Nikanor sai aru, et tema kavatsus oli ilmsiks tulnud, siis ta läks Hafar-Salamasse lahingusse Juuda vastu.
\par 32 Nikanori väest langes siis ligi viissada meest, ülejäänud aga põgenesid Taaveti linna. 

\section*{Ähvardus templi vastu}

\par 33 Ja pärast seda läks Nikanor üles Siioni mäele. Siis tulid mõned pühamu preestreist ja rahvavanemaist teda sõbralikult tervitama ja temale näitama seda põletusohvrit, mida kuninga auks ohverdati.
\par 34 Aga tema pilkas neid ja naeris nad välja, rüvetas neid ja kõneles kõrgilt.
\par 35 Ta vandus tulises vihas, üteldes: „Kui Juudast ja tema sõjaväge nüüd kohe minu kätte ei anta, siis mina, kui ma tervena tagasi tulen, põletan selle koja!” Ja ta läks ära väga vihasena.
\par 36 Siis läksid preestrid ning seisid altari ja templi ette, nutsid ja ütlesid:
\par 37 „Sina, Issand, oled valinud selle koja, et seda nimetataks sinu nimega, et see oleks sinu rahvale palve- ja anumiskojaks.
\par 38 Tasu kätte sellele inimesele ja tema sõjaväele, et nad langeksid mõõga läbi! Meenuta nende teotamisi ja ära anna neile püsi!” 

\section*{Nikanori kaotus Adaasas}

\par 39 Nikanor läks Jeruusalemmast ära ning lõi leeri üles Beet-Hooronis. Seal liitus temaga süürlaste väehulk.
\par 40 Aga Juudas lõi kolme tuhande mehega leeri üles Adaasas. Ja Juudas palvetas ning ütles:
\par 41 „Ükskord kui Assüüria kuninga mehed teotasid, läks sinu ingel ja lõi neist maha sada kaheksakümmend viis tuhat.
\par 42 Purusta nõnda ka täna see leer meie ees, et nad mõistaksid, et ta on pahatahtlikult rääkinud sinu pühamu vastu! Nuhtle teda tema kurjuse kohaselt!”
\par 43 Ja sõjaväed põrkasid kokku tapluseks adarikuu kolmeteistkümnendal päeval. Nikanori sõjavägi löödi puruks ja tema ise langes tapluses esimesena.
\par 44 Aga kui tema sõjavägi nägi, et Nikanor on langenud, siis nad viskasid sõjariistad maha ja põgenesid.
\par 45 Juuda mehed jälitasid neid ühe päevateekonna Adaasast kuni Geseri teelahkmeni ning puhusid nende taga märguandeks pasunaid.
\par 46 Siis tuldi kõigist ümberkaudseist Juudamaa küladest ja tungiti Nikanori sõjaväele kallale külgedelt, nii et nad pöördusid ründajate vastu. Ja nad kõik langesid mõõga läbi ning neist ei jäänud järele ühtainsatki.
\par 47 Siis Juuda mehed võtsid saagi ja röövitud varanduse, raiusid maha Nikanori pea ja tema parema käe, mille ta oli suureliselt tõstnud, võtsid need kaasa ja riputasid üles Jeruusalemma lähedal.
\par 48 Ning rahvas rõõmustas väga ja nad pidasid seda päeva suureks rõõmupäevaks.
\par 49 Ja nad seadsid, et seda päeva pidi peetama igal aastal adarikuu kolmeteistkümnendal päeval.
\par 50 Ja Juudamaal oli üürikeseks ajaks rahu.

\chapter{8}

\section*{Roomlaste ülistamine}

\par 1 Juudas kuulis ka roomlaste kuulsusest, et nad on sõjaliselt tugevad ja sõbralikud kõigile, kes nendega on liitunud, et nad peavad sõprust kõigiga, kes nende juurde tulevad, ja et nende võimsus on suur.
\par 2 Temale jutustati nende sõdadest ja kangelastegudest, mis nad olid teinud galaatlaste vastu, kuidas nad olid need võitnud ja maksu maksma sundinud,
\par 3 ja mis nad olid teinud Hispaania aladel, sealseid hõbeda- ja kullakaevandusi vallutades,
\par 4 ja kuidas nad oma tarkuse ja püsivusega olid vallutanud kogu maa, kuigi see maa oli neist väga kaugel, ja kuidas nad olid võitnud kuningaid, kes maailma äärtest olid tulnud nende vastu, olid need alistanud, lüües neid suurtes lahingutes, ja kuidas ülejäänud pidid neile igal aastal maksu maksma,
\par 5 ja kuidas nad olid sõjas hävitanud ja alistanud Filippose ja Perseuse, kittide kuninga, ja need, kes nende vastu olid tõusnud,
\par 6 ja kuidas nad olid hävitanud Antiohhos Suure, Aasia kuninga, kes nende vastu oli tulnud sõdima saja kahekümne elevandiga, ratsaväe ja vankritega ning väga suure sõjaväega.
\par 7 Tema enese olid nad kinni võtnud elavana ja olid pannud tema ja need, kes pärast teda valitsesid, suurt maksu maksma, pantvange andma ja maid loovutama,
\par 8 nimelt India, Meedia ja Lüüdia ühes teiste osadega tema ilusamatest aladest, mis nad temalt ära võttes olid andnud kuningas Eumenesele.
\par 9 Kreeka rahvas - nõnda jutustati - oli otsustanud astuda nende vastu, et nad hävitada,
\par 10 aga kui see kavatsus neile teatavaks sai, siis nad läkitasid neile vastu ühe väepealiku nendega sõdima. Neist langesid siis paljud mahalööduna, nende naised ja lapsed viidi vangidena ära, nad riisuti paljaks, nende maa vallutati, kindlused hävitati, ja nad on orjastatud tänapäevani.
\par 11 Ka muud kuningriigid ja saared, mis olid tõusnud nende vastu, nad hävitasid ja alistasid, aga oma sõpradega ja nendega, kes neid usaldasid, nad pidasid sõprust.
\par 12 Ligemal ja kaugemal olevad kuningriigid nad aga vallutasid, ja kõik, kes nende nime kuulsid, kartsid neid.
\par 13 Need, keda nad tahtsid aidata ja lasta valitseda, need valitsevad; aga need, keda nad tahtsid kõrvaldada, nad kõrvaldasid, ja nõnda said nad väga suure võimu.
\par 14 Ent kõigest sellest hoolimata ei olnud ükski neist pannud enesele krooni pähe ega purpurit selga, et sellega uhkeldada,
\par 15 vaid nad olid endale asutanud nõukogu, kus iga päev pidas nõu kolmsada kakskümmend meest, alati mures olles rahva pärast, et teda paremini valitseda.
\par 16 Nad usaldasid igal aastal ühele mehele endi juhtimise ja kogu maa valitsemise; kõik kuulasid selle ühe sõna ja nende keskel ei olnud kadedust ega riidu. Juutide liit roomlastega
\par 17 Juudas valis siis Eupolemose, Johannese poja, Akkose pojapoja, ja Jaasoni, Eleasari poja, ja läkitas nad Rooma sõlmima sõprust ja liitu,
\par 18 et roomlased võtaksid neilt ikke, sest nad ju pidid nägema, et kreeklaste valitsus pidas Iisraeli orjuses.
\par 19 Nad läksid siis Rooma - tee sinna oli väga pikk - ning tulid nõukogusse ja võtsid sõna ning ütlesid:
\par 20 „Juudas, nimetatud ka Makkabiks, ja tema vennad ja juudi rahvas on meid läkitanud teie juurde sõlmima teiega liitu ja rahu, ja et meid pandaks kirja kui teie liitlased ja sõbrad.”
\par 21 See kõne meeldis neile.
\par 22 Ja see on ärakiri sellest kirjast, mille nad kirjutasid vasktahvlitele ja saatsid Jeruusalemma, et see oleks seal nende juures rahu ja liidu mälestuseks.
\par 23 „Roomlaste ja juudi rahva käsi käigu igavesti hästi merel ja maal, mõõk ja vaenlane jäägu neist kaugele!
\par 24 Aga kui sõda peaks puhkema, kõigepealt Rooma vastu või mõne tema liitlase vastu kogu tema valitsusalal,
\par 25 siis peab juudi rahvas kõigest südamest kaasa võitlema, nõnda nagu olukord seda nõuab.
\par 26 Nad ei tohi anda ega hankida vaenlastele vilja, sõjariistu, raha ja laevu, nagu roomlased on otsustanud. See kohustus tuleb täita midagi vastu saamata.
\par 27 Aga nõndasamuti, kui sõda puhkeb kõigepealt juudi rahva vastu, siis tulevad roomlased meeleldi neile appi, nõnda nagu olukord neilt nõuab.
\par 28 Nende vaenlastele ei anta vilja, sõjariistu, raha ja laevu, nagu roomlased on otsustanud. See kohustus tuleb täita tingimusteta!”
\par 29 Neil alustel on roomlased sõlminud lepingu juudi rahvaga.
\par 30 „Aga kui pärast seda üks või teine pool tahab midagi lisada või ära jätta, siis ta võib seda teha oma tahte järgi. Ja mida nad lisavad või ära jätavad, see kehtib.”
\par 31 Ja selle kuriteo pärast, mida kuningas Demeetrios teile on teinud, oleme temale kirjutanud nõnda: „Mispärast oled oma ikke raskeks teinud meie sõpradele ja liitlastele, juutidele?
\par 32 Kui nad nüüd sinu peale kaebavad, siis meie muretseme neile õigust ja sõdime sinu vastu merel ja maal.”

\chapter{9}

\section*{Juudas Makkabi kangelassurm}

\par 1 Kui Demeetrios kuulis, et Nikanor ja tema sõjavägi olid lahingus langenud, siis ta läkitas Bakhidese ja Alkimose teist korda Juudamaale, ja koos nendega oma sõjaväe parema tiiva.
\par 2 Nad rühkisid Galgala teed mööda ja lõid leeri üles Arbeelasse, vastu Maisalotti, vallutasid selle ja tapsid palju inimesi.
\par 3 Ja aastal sada viiskümmend kaks, esimesel kuul, lõid nad leeri üles Jeruusalemma alla.
\par 4 Siis nad läksid teele ja liikusid Bereasse kahekümne tuhande jalamehega ja kahe tuhande ratsanikuga.
\par 5 Juudas oli aga leeri üles löönud Elasas, ja koos temaga oli kolm tuhat valitud meest.
\par 6 Aga kui nad nägid, et vaenlase sõjavägi oli arvult nõnda suur, siis nad kartsid väga. Ja leerist põgenejaid oli palju, seetõttu ei jäänud sinna mitte rohkem kui kaheksasada meest.
\par 7 Kui nüüd Juudas nägi, et tema leer oli laiali läinud ja et võitlus teda ähvardas, siis oli ta süda ahastuses, et tal ei olnud enam aega neid kokku koguda.
\par 8 Ta oli meeleheitel ja ütles neile, kes olid paigale jäänud: „Tõuskem ja mingem üles oma vaenlaste vastu, vahest suudame nad võita!”
\par 9 Aga nad keelasid teda, üteldes: „Seda me ei suuda, vaid päästkem nüüd oma hinged ja pöördugem tagasi koos meie vendadega ning võidelgem siis! Praegu on meid liiga vähe.”
\par 10 Kuid Juudas vastas: „Ärgu sündigu, et ma nõnda teeksin ja põgeneksin nende eest! Kui meie tund tuleb, siis surgem kui mehed oma vendade eest ja ärgem jätkem häbiplekki oma au peale!”
\par 11 Aga vaenlaste vägi lahkus leerist ja asetus nende vastu. Ratsavägi jagunes kaheks, lingumehed ja ammukütid käisid sõjaväe ees, nõndasamuti kõik need, kes suutsid võidelda esirivis. Bakhides ise oli aga paremal tiival.
\par 12 Võitlusrivi lähenes siis pasunaid puhudes kummaltki poolt. Ja need, kes Juuda juures olid, puhusid ka pasunaid.
\par 13 Maa müdises nüüd sõjaleeride mürast ja vastastikune taplus kestis hommikust õhtuni.
\par 14 Kui Juudas nägi, et Bakhides ja tema leeri südamik olid paremal tiival, siis tulid tema juurde kõik vaprad mehed
\par 15 ja nad purustasid parempoolse tiiva ning ajasid neid taga kuni Asdodi mäeni.
\par 16 Aga kui vasakul tiival olijad nägid, et parem tiib oli purustatud, siis nad pöördusid Juuda ja tema meeste kannule ning ajasid neid taga.
\par 17 Tekkis äge taplus ja palju mahalööduid langes kummaltki poolt.
\par 18 Langes ka Juudas ja ülejäänud põgenesid. 

\section*{Juuda matmine}

\par 19 Aga Joonatan ja Siimon võtsid oma venna Juuda ning matsid tema ta isade hauda Moodeinis.
\par 20 Nad nutsid teda taga ja kogu Iisrael itkes väga leinates. Nad leinasid kaua ja ütlesid:
\par 21 „Kuidas küll on langenud kangelane, Iisraeli päästja!”
\par 22 Aga muud Juuda lood, tema sõjad ja kangelasteod, mis ta tegi, ja tema tõeline suurus ei ole kirja pandudki, sest seda oleks väga palju. 

\section*{Joonatan vastupanu juhiks}

\par 23 Aga pärast Juuda surma sündis, et usust taganejad tõstsid pead kõigis Iisraeli paikkondades ja kõik, kes olid vääriti teinud, tulid esile.
\par 24 Neil päevil oli väga suur näljahäda, nõnda et maa rahvaski jooksis üle vaenlase poolele.
\par 25 Ja Bakhides valis jumalakartmatuid mehi ning pani need maale valitsejaiks.
\par 26 Need siis otsisid ja jälitasid Juuda sõpru ning viisid nad Bakhidese juurde. Tema nuhtles ja pilkas neid.
\par 27 Siis oli Iisraelis nii suur viletsus, missugust ei ole olnud sellest ajast, mil nende keskel viimati nähti prohvetit.
\par 28 Nüüd kogunesid kõik Juuda sõbrad ja ütlesid Joonatanile:
\par 29 „Pärast seda kui sinu vend Juudas suri, ei ole olnud temaga sarnast meest, kes suudaks minna vaenlaste ja Bakhidese vastu ja nende vastu, kes meie rahvast vihkavad.
\par 30 Sellepärast valime nüüd täna sinu, et oleksid tema asemel meile pealikuks ja juhiks, kes võitleb meie võitlusi.”
\par 31 Ja Joonatan võttis sel ajal juhtimise vastu ning astus oma venna Juuda asemele. 

\section*{Verine sündmus Meedabas}

\par 32 Kui Bakhides sellest teada sai, siis ta püüdis teda tappa.
\par 33 Aga kui Joonatan ja tema vend Siimon ja kõik, kes olid koos temaga, sellest teada said, siis nad põgenesid Tekoa kõrbe ja lõid leeri üles Asfari veehoidla äärde.
\par 34 Bakhides sai sellest teada ühel hingamispäeval ja ta läks koos kogu oma sõjaväega üle Jordani.
\par 35 Joonatan oli läkitanud oma venna Johannese, vooripealiku, paluma sõpru nabatlasi, et nad võiksid nabatlaste juurde paigutada oma varustuse, mida oli palju.
\par 36 Aga ambrilased tulid Meedabast ja võtsid Johannese kinni koos kõigega, mis tal oli, ja läksid saagiga ära.
\par 37 Pärast seda sündmust aga teatati Joonatanile ja tema vennale Siimonile, et ambrilased peavad suurt pulmapidu ja on suure saatjaskonnaga toomas Nadabatist pruuti, kes oli ühe kaananlaste tähtsa suurmehe tütar.
\par 38 Meenutades siis oma venna Johannese verd, läksid nad ja peitsid endid mäekurus.
\par 39 Kui nad silmad üles tõstsid ja vaatasid, siis ennäe, missugune lärm ja kui palju kraami! Peigmees koos sõprade ja vendadega tuli pruudivoorile vastu trummide ja pillimänguga ning paljude sõjariistadega.
\par 40 Siis nad ründasid neid oma varitsuspaigast ja tapsid neid, paljud langesid haavatuna ja ülejäänud põgenesid mäestikku. Ja nad võtsid saagiks kogu nende varustuse.
\par 41 Pulm muutus seega leinaks ja pillimänguviis nutulauluks.
\par 42 Nõnda nad maksid kätte oma venna vere eest ja pöördusid tagasi Jordani madalikule. 

\section*{Üleminek Jordanist}

\par 43 Kui Bakhides sellest kuulis, siis ta tuli hingamispäeval Jordani kaldaile suure sõjaväega.
\par 44 Siis ütles Joonatan neile, kes tema juures olid: „Võtkem nüüd kätte ja võidelgem oma elu eest, ei ole ju täna nagu eile ja üleeile!
\par 45 Sest vaata, võitlus on meil ees ja taga, Jordani vesi on meil kummalgi pool, ka raba ja tihnik, ja ei ole paika, kuhu põgeneda.
\par 46 Hüüdke nüüd taeva poole, et teid päästetaks meie vaenlaste käest!”
\par 47 Ja taplus tuli. Joonatan sirutas käe, et tappa Bakhides, see aga sööstis tema juurest eemale.
\par 48 Siis hüppas Joonatan koos nendega, kes tema juures olid, Jordanisse ja nad ujusid teisele kaldale. Vaenlased aga ei läinud üle Jordani neile järele.
\par 49 Bakhidese poolt langes sel päeval ligi tuhat meest. 

\section*{Bakhides kindlustab linnu}

\par 50 Bakhides pöördus tagasi Jeruusalemma ning ehitas Juudamaal kindlustatud linnad: Jeeriko kindluse, Emmause, Beet-Hooroni, Peeteli, Tamnata-Faratoni ja Tefoni - kõrgete müüride, väravate ja riividega.
\par 51 Ja ta paigutas neisse väesalgad Iisraeli ahistamiseks.
\par 52 Ta kindlustas ka Beet-Suuri linna, Geseri ja kindluse, ja paigutas neisse sõjaväe ning moonavarud.
\par 53 Ja ta võttis pantvangideks maa ülikute pojad ning pani need vangi Jeruusalemma kindlusse. 

\section*{Alkimose surm}

\par 54 Aastal sada viiskümmend kolm, teisel kuul, käskis Alkimos maha kiskuda pühamu sisemise eesõue müürid ja hävitada see, mille prohvetid olid teinud. Aga kui Alkimos oli hävitamist alustanud,
\par 55 just siis tabas teda rabandus ja katkestas tema töö. Tema suu jäi kinni ja ta oli halvatud, nõnda et ta ei saanud sõnagi rääkida ega käskida oma kojas.
\par 56 Ja Alkimos suri sel ajal suures piinas.
\par 57 Kui Bakhides nägi, et Alkimos oli surnud, siis ta pöördus tagasi kuninga juurde. Ja Juudamaal oli rahu kaks aastat. 

\section*{Beet-Basi piiramine}

\par 58 Aga kõik taganejad pidasid nõu, üteldes: „Vaata, Joonatan ja need, kes tema juures on, elavad muretult rahus. Toogem nüüd Bakhides, et ta võtaks nad kõik vangi ühel ööl!”
\par 59 Nad läksid siis temaga nõu pidama.
\par 60 Ta asus siis teele ja läks suure sõjaväega ning saatis salaja kirju kõigile liitlastele Juudamaal, et nad võtaksid kinni Joonatani ja need, kes on koos temaga. Nemad aga ei saanud seda teha, sest see nõu oli teatavaks saanud.
\par 61 Ja Joonatani mehed võtsid kinni ligi viiskümmend meest maa elanikest, kes olid kuritööks ässitajad olnud, ja tapsid need.
\par 62 Joonatan ja Siimon ning need, kes koos nendega olid, läksid siis kõrbes olevasse Beet-Basi, ehitasid üles, mis seal oli purustatud, ja kindlustasid selle.
\par 63 Kui Bakhides sellest teada sai, siis ta kogus kokku kõik oma väehulgad ja kutsus ka Juudamaal olijad.
\par 64 Ja ta tuli ning lõi leeri üles Beet-Basi alla, sõdis selle vastu kaua aega ja ehitas piiramisseadmed.
\par 65 Joonatan jättis siis oma venna Siimoni linna ja läks ise väheste meestega maale.
\par 66 Ta lõi seal Odomeerat ja tema vendi, nõndasamuti Fasirooni poegi nende telkides. Kui ta rünnakut alustas ja oma väeosaga linnale lähenes,
\par 67 siis tuli ka Siimon koos oma meestega linnast välja ja põletas piiramisseadmed.
\par 68 Nad sõdisid Bakhidese vastu ning võitsid ta, valmistades talle suurt meelehärmi, et ta nõu ja kallaletung oli tühja läinud.
\par 69 Tema viha süttis nende nurjatute meeste vastu, kes olid talle nõu andnud maale tulla, ja paljud neist surmati. Siis ta otsustas minna oma maale.
\par 70 Kui Joonatan sellest teada sai, siis ta läkitas tema juurde saadikud, et temaga rahu teha ja vange tagasi saada.
\par 71 Bakhides nõustus, tegi tema ettepanekute järgi ja vandus talle, et ta oma eluajal enam ei tee talle kurja.
\par 72 Ta andis temale tagasi vangid, keda ta varem oli Juudamaal vangistanud. Siis ta pöördus ja läks tagasi oma maale ega tulnud enam neisse paigusse.
\par 73 Mõõk puhkas siis Iisraelis. Joonatan asus elama Mikmassi. Ja Joonatan hakkas nüüd rahvale kohut mõistma, kaotades Iisraelist jumalakartmatud.

\chapter{10}

\section*{Joonatan sõbrustab kuningas Aleksandriga}

\par 1 Aastal sada kuuskümmend tuli Aleksander, Antiohhos Epifanese poeg, ja vallutas Ptolemaisi. Ta võeti vastu ja ta valitses seal kuningana.
\par 2 Kui kuningas Demeetrios seda kuulis, siis ta kogus väga suure sõjaväe ja läks sõtta tema vastu.
\par 3 Ja Demeetrios saatis Joonatanile kirju leplike sõnadega, et teda austada.
\par 4 Sest ta mõtles: „Ruttame temaga rahu tegema, enne kui tema seda teeb Aleksandriga meie vastu!
\par 5 Sest kindlasti meenutab ta kõike seda kurja, mida oleme teinud temale, tema vendadele ja tema rahvale.”
\par 6 Ta andis temale loa koguda sõjaväge, muretseda sõjariistu ja olla tema liitlane. Ja ta käskis anda temale need pantvangid, kes kindluses olid.
\par 7 Joonatan läks siis Jeruusalemma ja luges kirja kogu rahva ning kindluses olijate kuuldes.
\par 8 Need aga hakkasid väga kartma, kui nad kuulsid, et kuningas oli temale loa andnud sõjaväe kogumiseks.
\par 9 Kindluses olijad loovutasid siis pantvangid Joonatanile ja tema andis nad nende vanemaile.
\par 10 Joonatan asus nüüd Jeruusalemma elama ning hakkas linna ehitama ja taastama.
\par 11 Ta käskis töömehi ehitada müürid neljakandilistest kividest ning muuta Siioni mägi ümberringi kindluseks. Ja nad tegid nõnda.
\par 12 Siis need muulased, kes olid neis kindlustes, mis Bakhides oli ehitanud, põgenesid,
\par 13 igaüks jättis oma asupaiga ja läks oma maale.
\par 14 Ainult Beet-Suuri jäid mõned, kes olid hüljanud Seaduse ja korraldused, sest see oli neile pelgulinnaks.
\par 15 Kui kuningas Aleksander kuulis lubadustest, mis Demeetrios Joonatanile oli andnud, ja kui temale jutustati neist võitlustest ja kangelastegudest, mida Joonatan ja tema vennad olid teinud, nõndasamuti raskustest, mis neil olid olnud,
\par 16 siis ta ütles: „Teist niisugust meest me ei leia. Tehkem tema nüüd meie sõbraks ja liitlaseks!”
\par 17 Ja ta kirjutas kirju ning saatis need temale. Sõnastus oli niisugune:
\par 18 „Kuningas Aleksander tervitab vend Joonatani!
\par 19 Meie oleme sinust kuulnud, et oled väga vapper mees ja väärt olema meie sõber.
\par 20 Ja nüüd me seame su täna sinu rahvale ülempreestriks ja sind hüütakse kuninga sõbraks,” - ühtlasi saatis ta temale purpurmantli ja kuldkrooni - „ole meiega ühes nõus ja pea meiega sõprust!”
\par 21 Joonatan pani püha kuue selga saja kuuekümnenda aasta seitsmendal kuul, lehtmajadepühal. Ja ta kogus sõjaväge ning hankis palju sõjariistu. 

\section*{Demeetriose kiri Joonatanile}

\par 22 Kui Demeetrios neist asjust kuulis, siis ta jäi murelikuks ja ütles:
\par 23 „Miks lasksime sündida, et Aleksander meist ette jõudis, sõlmides sõpruse juutidega enesele toeks?
\par 24 Minagi tahan neile kirjutada nende ergutamiseks ülendamistest ja kingitustest, et nad oleksid mulle abiks.”
\par 25 Ja ta saatis neile kirja niisuguses sõnastuses: „Kuningas Demeetrios tervitab juudi rahvast!
\par 26 Et te olete pidanud meiega tehtud lepinguid ja olete jäänud meiega sõprusesse ega ole ühinenud meie vaenlastega, sellest oleme rõõmuga kuulnud.
\par 27 Jääge nüüd edaspidigi selle juurde, et olete meile truud, siis me tasume teile selle hea eest, mida te meile teete!
\par 28 Me lubame teile rohkeid soodustusi ja teeme kingitusi.
\par 29 Nüüd ma kuulutan teid priiks ning vabastan kõik juudid lõivudest, soola- ja kroonimaksust.
\par 30 Ka kolmandikust seemneviljast ja poolest puuviljast, mis mul on õigus saada, ma loobun tänasest päevast alates. Enam ei võeta seda Juudamaalt ja kolmest selle juurde kuuluvast piirkonnast, Samaariast ja Galileast, tänasest päevast alates igavesti.
\par 31 Jeruusalemm koos piirkonnaga olgu püha ja maksuvaba, mis puutub kümnisesse ja tollirahasse!
\par 32 Ma loobun õigusest Jeruusalemma kindlusele ja loovutan selle ülempreestrile, et ta paneks sinna mehed, keda ta valib seda kaitsma.
\par 33 Kõik juudid, kes Juudamaalt on vangi viidud kuhugi minu kuningriiki, lasen ma vabaks ilma lunata. Keegi ei tohi neid maksustada, ka mitte nende loomade eest!
\par 34 Kõik pühad, hingamispäevad, noorkuupühad ja muud kindlaksmääratud päevad, nõndasamuti kolm päeva pärast püha - kõik need päevad olgu tolli- ja maksuvabad kõigile juutidele, kes minu kuningriigis on!
\par 35 Mitte kellelgi ärgu siis olgu meelevalda neilt midagi nõuda ja neid koormata, ükskõik missugune põhjus ka on!
\par 36 Ja kuninga sõjaväe tarvis kirjutatagu üles juutide hulgast kolmkümmend tuhat meest ja neile antagu moona, nagu seda antakse kõigile kuninga väeosadele!
\par 37 Osa neist paigutatagu kuninga suurtesse kindlustesse ja osa neist seatagu kuningriigi usaldusameteisse! Nende ülemad ja pealikud võetagu nende endi hulgast! Nad elagu oma seaduste järgi, nõnda nagu kuningas on määranud ka Juudamaal!
\par 38 Need kolm maakonda, mis Samaaria alast on liidetud Juudamaaga, ühendatagu Juudamaaga nõnda, et neid arvestatakse alluvaks ühele, ja nad ei tarvitse kuulata muud võimu kui ülempreestrit!
\par 39 Ptolemaisi linna koos selle juurde kuuluvaga annan ma kingituseks Jeruusalemma pühamule, pühamu vajalike kulutuste katteks.
\par 40 Mina ise annan igal aastal viisteist tuhat seeklit hõbedat kuninga tuludest, võetud selleks kohastest paikadest.
\par 41 Ja kõik muu, mis on maksmata jäänud eelmistel aastatel, tuleb nüüdsest peale anda templiteenistuse tarvis.
\par 42 Aga peale selle veel viis tuhat seeklit hõbedat, mis võeti pühamu varadest, iga-aastastest tuludest, seegi jäetakse, sest see kuulub teenistust toimetavaile preestreile.
\par 43 Igaüks, kes iganes põgeneb Jeruusalemma templisse või kogu selle piirkonda ja kellel on võlg kuningale või mõni muu süü, olgu karistusest vaba, nõndasamuti kõik tema omand minu kuningriigis!
\par 44 Ja pühamu ehitus- ning uuendustöö kulud makstakse kuninga tuludest.
\par 45 Ka Jeruusalemma müüride ehitamiseks ja selle ümberringi kindlustamiseks vajalikud kulud makstakse kuninga tuludest, nõnda ka teiste linnade müüride ehitamine Juudamaal.” 

\section*{Joonatan lükkab tagasi Demeetriose pakkumise}

\par 46 Aga kui Joonatan ja rahvas kuulsid neid sõnu, siis nad ei uskunud neid ega võtnud vastu, sest neil oli meeles see suur kuritegu, mida kuningas Iisraelis oli teinud, ja kui väga ta neid oli rõhunud.
\par 47 Nad pidasid paremaks hoida Aleksandri poole, sest tema oli esimesena rahu pakkunud, ja nad olid kogu aja olnud tema liitlased.
\par 48 Kuningas Aleksander kogus suure sõjaväe ning lõi leeri üles Demeetriose vastu.
\par 49 Ja mõlemad kuningad läksid kokku tapluseks. Demeetriose leer põgenes ja Aleksander ajas teda taga ning sai võidu nende üle.
\par 50 Ta jätkas ägedat võitlust kuni päikeseloojakuni, ja Demeetrios langes sel päeval. 

\section*{Aleksandri abielu Kleopatraga}

\par 51 Aleksander läkitas siis saadikud Egiptuse kuninga Ptolemaiose juurde viima niisuguseid sõnumeid:
\par 52 „Mina olen nüüd tagasi tulnud oma kuningriiki, istun oma isade aujärjel ja olen võtnud valitsuse enda kätte. Ma olen võitnud Demeetriose ja taas võtnud oma valdusesse meie maa.
\par 53 Ma läksin temaga kokku lahingus ja me hävitasime tema enese ja tema leeri ning istusime tema kuninglikule aujärjele.
\par 54 Tehkem nüüd teineteisega sõprust: sina anna oma tütar mulle naiseks, siis saan ma sinu väimeheks ning annan sinule ja temale kingitusi, mis on sinu väärilised!”
\par 55 Kuningas Ptolemaios vastas, üteldes: „Õnnelik on see päev, mil sa oled tagasi tulnud oma isade maale ja oled istunud nende kuningriigi aujärjele.
\par 56 Ja nüüd ma teen, nagu sa oled kirjutanud. Aga tule minuga kohtuma Ptolemaisi linna, et näeksime teineteist! Siis ma võtan sinu väimeheks, nagu sa oled soovinud.”
\par 57 Ptolemaios tuli siis Egiptusest, tema ise ja tema tütar Kleopatra, ja saabus Ptolemaisi linna aastal sada kuuskümmend kaks.
\par 58 Ja kuningas Aleksander kohtas teda seal. Ja tema andis oma tütre Kleopatra temale ning tegi temale Ptolemaisis pulmad suure toredusega, nagu kuningail on kombeks. 

\section*{Aleksander austab Joonatani}

\par 59 Ja kuningas Aleksander kirjutas Joonatanile, et ta tuleks temaga kohtuma.
\par 60 Ta läks siis Ptolemaisi suure toredusega ja kohtas mõlemat kuningat. Ta andis neile ja nende sõpradele hõbedat ja kulda ja palju kingitusi ning leidis nende silmis armu.
\par 61 Tema vastu kogunes nüüd nurjatuid mehi Iisraelist, Seadusest taganenud mehi, teda süüdistama. Kuningas aga ei võtnud neid kuulda.
\par 62 Kuningas käskis võtta Joonatanilt tema oma riided ning panna talle selga purpurmantli. Ja nõnda tehti.
\par 63 Kuningas pani ta enese kõrvale istuma ning ütles oma vürstidele: „Minge koos temaga linna keskele ja kuulutage, et ükski ärgu süüdistagu teda mingisugusel põhjusel ja ükski ärgu tülitagu teda mingi asjaga!”
\par 64 Ja sündis, kui süüdistajad nägid tema austamist, nõnda nagu kuulutati, ja et ta oli riietatud purpurisse, et nad kõik siis põgenesid.
\par 65 Kuningas austas teda veelgi rohkem, kirjutas ta oma parimate sõprade nimekirja ning pani väejuhiks ja asehalduriks.
\par 66 Joonatan läks tagasi Jeruusalemma rahu ja rõõmuga. 

\section*{Joonatan ja Demeetrios II}

\par 67 Aastal sada kuuskümmend viis tuli Demeetrios, Demeetriose poeg, Kreetalt oma isade maale.
\par 68 Sellest kuuldes jäi kuningas Aleksander väga murelikuks ja läks tagasi Antiookiasse.
\par 69 Demeetrios pani jälle ametisse Apollooniose, Koile-Süüria asehalduri, kes kogus suure sõjaväe, lõi leeri üles Jamniasse ja läkitas ülempreester Joonatanile sõnumi:
\par 70 „Sina oled ainus, kes meile vastu hakkab. Mina olen nüüd sinu pärast saanud naeru- ja pilkealuseks. Miks sa tahad meiega jõudu katsuda mäestikus?
\par 71 Kui sa nüüd loodad oma sõjaväe peale, siis tule meie juurde alla lagendikule! Seal võime teineteisega jõudu katsuda, sest minuga on linnade sõjajõud.
\par 72 Küsi ja õpi tundma, kes olen mina ja kes on need teised, kes meid aitavad, siis vastatakse sulle: ei ole teie jalgel peatust meie ees, sest kaks korda on sinu isad põgenema aetud nende enda maal!
\par 73 Ega nüüd sinagi suuda vastu seista ratsaväele ja nii suurele sõjaväele lagendikul, kus ei ole kaljut ega kivi ega paika, kuhu põgeneda!”
\par 74 Aga kui Joonatan kuulis Apollooniose sõnu, siis oli ta hingelt vapustatud ja, valides kümme tuhat meest, läks Jeruusalemmast välja. Tema vend Siimon tuli temale vastu, et teda aidata.
\par 75 Ta lõi leeri üles Joppe alla, aga linnas olijad ei lasknud teda sisse, sest Apollooniosel oli Joppes linnavägi. Siis nad ründasid seda.
\par 76 Linnas olijad avasid siis kartuse pärast väravad ja Joonatan vallutas Joppe.
\par 77 Kui Apolloonios sellest kuulda sai, siis ta seadis võitlusvalmis kolm tuhat ratsameest ja palju jalaväge ning läks Asdodi poole, nagu tahaks ta sellest läbi minna, läks aga ühtlasi ka lagendikule, sest temal oli palju ratsaväge, kelle peale ta lootis.
\par 78 Joonatan aga jälgis teda kuni Asdodini ja seal läksid leerid kokku tapluseks.
\par 79 Aga Apolloonios oli salaja jätnud tuhat ratsanikku nende selja taha.
\par 80 Kuid Joonatan sai teada, et tema selja taga olid varitsejad. Need piirasid nüüd sisse tema leeri ja ambusid nooli rahva peale hommikust õhtuni.
\par 81 Ent rahvas jäi paigale, nõnda nagu Joonatan oli käskinud, ja vaenlase hobused väsisid.
\par 82 Nüüd tõi Siimon oma väeosa välja ja ründas jalaväe rinnet, sest ratsavägi oli juba jõuetu. Ja vaenlane purustati ning ta põgenes.
\par 83 Aga ratsavägi hajus lagendikule ja nad põgenesid Asdodisse ning läksid Beet-Daagonisse, oma ebajumala templisse, et endid päästa.
\par 84 Joonatan põletas Asdodi ja selle ümberkaudsed linnad ning võttis neist sõjasaaki. Ka Daagoni templi ja need, kes sinna olid põgenenud, põletas ta tulega.
\par 85 Mõõga läbi langenuid ning põletatuid oli ligi kaheksa tuhat meest.
\par 86 Siis läks Joonatan sealt teele ja lõi leeri üles Askeloni alla. Ja linnas olijad tulid välja temale vastu suurte austusavaldustega.
\par 87 Siis Joonatan ja need, kes olid koos temaga, pöördusid rikkaliku saagiga tagasi Jeruusalemma.
\par 88 Kui kuningas Aleksander nendest sündmustest kuulis, siis ta austas Joonatani veelgi rohkem.
\par 89 Ta saatis temale kuldpandla, nagu on viisiks anda kuninga sugulastele. Ja ta andis temale pärisosaks Ekroni ning kõik selle alad.

\chapter{11}

\section*{Ptolemaios VI toetab Demeetrios IIst ja sureb koos Aleksandriga}

\par 1 Egiptuse kuningas kogus palju sõjaväge, mida oli otsekui liiva mererannas, ja palju laevu. Siis ta püüdis kavalusega vallutada Aleksandri kuningriiki ja liita oma kuningriigiga.
\par 2 Ta läks Süüriasse rahusõnadega, ja need, kes linnades olid, avasid talle väravad ning läksid talle vastu, sest kuningas Aleksander oli käskinud teda vastu võtta, kuna ta oli tema äi.
\par 3 Aga kui Ptolemaios linnadesse sisse tuli, siis ta jättis igasse linna sõjaväge linnaväeks.
\par 4 Kui ta lähenes aga Asdodile, siis näidati temale põletatud Daagoni templit ja selle ümber olevaid hävitatud linnu, pillutatud laipu ja põletatute jäänuseid, kes sõjas olid põletatud, sest need olid kuhjatud tema tee äärde.
\par 5 Ja kuningale jutustati, mida Joonatan oli teinud, et teda teotada. Aga kuningas vaikis.
\par 6 Joonatan tuli Joppesse kuningale vastu austusavaldustega, nad tervitasid teineteist ning jäid sinna ööseks.
\par 7 Siis Joonatan saatis kuningat kuni jõeni, mida hüütakse Eleuteroseks, ja pöördus tagasi Jeruusalemma.
\par 8 Aga kuningas Ptolemaios vallutas rannikulinnad kuni Seleukeiani ja haudus kurje kavatsusi Aleksandri vastu.
\par 9 Ja ta läkitas saadikud kuningas Demeetriose juurde sõnumiga: „Tule, teeme teineteisega liidu ja ma annan sulle oma tütre, kes nüüd on Aleksandril, ja sina saad kuningaks oma isa kuningriigis.
\par 10 Ma kahetsen, et andsin oma tütre temale, sest ta püüab mind tappa.”
\par 11 Ta teotas teda, sest ta himustas tema kuningriiki.
\par 12 Ta võttis temalt ära oma tütre ja andis Demeetriosele ning võõrdus Aleksandrist ja nende vahel tuli ilmsiks vaen.
\par 13 Ptolemaios läks siis Antiookiasse ja pani enesele pähe Aasia krooni, nõnda et temal oli peas kaks krooni: Egiptuse ja Aasia.
\par 14 Kuningas Aleksander oli aga sel ajal Kiliikias, sest selle piirkonna elanikud olid temast taganenud.
\par 15 Kui Aleksander sellest nüüd kuulda sai, läks ta sõtta Ptolemaiose vastu. Aga seegi läks teele ja tuli temale vastu tugeva sõjaväega ning sundis ta põgenema.
\par 16 Aleksander põgenes siis Araabiasse, et seal enesele kaitset otsida. Kuningas Ptolemaios jäi võitjaks.
\par 17 Aga araablane Sabdiel raius maha Aleksandri pea ning läkitas selle Ptolemaiosele.
\par 18 Kolmandal päeval pärast seda aga kuningas Ptolemaios suri. Tema linnaväed, kes olid kindlustes, surmati nende linnade elanike poolt.
\par 19 Ja Demeetrios sai kuningaks aastal sada kuuskümmend seitse. 

\section*{Demeetrios II ja Joonatani esimesed kokkupuuted}

\par 20 Neil päevil kogus Joonatan kokku juudid võitluseks Jeruusalemma kindluse vastu. Ja nad valmistasid selle jaoks palju piiramisseadmeid.
\par 21 Siis mõned, kes oma rahvast vihkasid, Seaduse-vastased mehed, läksid kuninga juurde ja kuulutasid temale, et Joonatan piirab kindlust.
\par 22 Seda kuuldes kuningas vihastas. Kui ta seda oli kuulnud, läks ta otsekohe teele ning tuli Ptolemaisi. Ja ta kirjutas Joonatanile, et ta lõpetaks piiramise ja tuleks kiiremas korras Ptolemaisi temaga kohtuma.
\par 23 Aga kui Joonatan seda kuulis, siis ta käskis piiramist jätkata ja valis Iisraeli vanemate ja preestrite hulgast mõned saadikuiks ning läks ka ise hädaohtlikule teekonnale.
\par 24 Ta võttis kaasa hõbedat ja kulda, riideid ja palju muid kingitusi ning läks Ptolemaisi kuninga juurde ja ta leidis armu tema silmis.
\par 25 Siis süüdistasid teda mõned jumalakartmatud tema oma rahva hulgast.
\par 26 Aga kuningas kohtles teda, nagu tema eelkäijad olid teda kohelnud, ja ülendas teda kõigi oma sõprade ees.
\par 27 Ta kinnitas temale ülempreestriameti ja kõik muud aunimetused, mis tal enne olid olnud, ja pidas teda oma parimaks sõbraks.
\par 28 Joonatan palus nüüd kuningat, et ta jätaks maksuvabaks Juudamaa ja need kolm piirkonda Samaarias, ning pakkus temale kolmsada talenti. 

\section*{Uus ürik juutide kasuks}

\par 29 See oli kuningale meelepärane ja ta kirjutas selle kõige kohta Joonatanile kirja, mille sisu oli niisugune:
\par 30 „Kuningas Demeetrios tervitab oma venda Joonatani ja juudi rahvast!
\par 31 Ärakirja kirjast, mille oleme teie pärast kirjutanud oma sugulasele Lastenesele, oleme kirjutanud ka teile teadmiseks:
\par 32 Kuningas Demeetrios tervitab isa Lastenest!
\par 33 Juudi rahvale, meie sõpradele ja ustavatele liitlastele, oleme otsustanud head teha nende poolt meile osutatud heatahtlikkuse pärast.
\par 34 Meie kinnitame neile Juuda alad ja need kolm piirkonda: Afairema, Lüdda ja Raamataimi, mis Samaariast on liidetud Juudamaaga, ja kõik, mis neile kuulub. Kõigile, kes Jeruusalemmas ohverdavad, me jätame kuningale kuuluva osa, mida kuningas varem igal aastal maa saadustest ja puuviljast neilt võttis.
\par 35 Ja muust, mis praegu meile kuulub, kümnisest ja meile tulevatest tollimaksudest, soolatiikidest ja meile tulevatest kroonimaksudest - kõigest sellest me loobume nende heaks.
\par 36 Mitte midagi sellest lepingust ärgu tühistatagu alates tänasest kuni igavese ajani!
\par 37 Kandke siis hoolt, et sellest tehtaks ärakiri ja antaks Joonatanile! See pandagu pühale mäele, nähtavasse paika!” 

\section*{Joonatan Demeetrios II liitlasena}

\par 38 Kui kuningas Demeetrios nüüd nägi, et tema maal on rahu ja keegi ei pane temale vastu, siis ta saatis ära kogu oma sõjaväe, igaühe tema kodupaika, välja arvatud võõrad palgasõdurid, keda ta oli värvanud paganate saartelt. Sellepärast vihkasid teda kõik need väed, kes olid tema isade ajast.
\par 39 Trüfon aga, kes oli endine Aleksandri poolehoidja, kui ta nägi, et kõik väeosad nurisesid Demeetriose vastu, läks araablase Imalku juurde, kes oli Aleksandri poja Antiohhose kasvataja.
\par 40 Ta palus teda tungivalt, et ta annaks poja temale, et see võiks saada kuningaks oma isa asemel. Ta jutustas temale kõigest, mida Demeetrios oli teinud, ja vihast, mis tema vägedel tema vastu on. Ja ta jäi sinna kauaks ajaks.
\par 41 Joonatan saatis aga sõna kuningas Demeetriosele, et ta võtaks ära Jeruusalemma kindluses ja muudes kindlustes olevad väesalgad, sest need olid Iisraelile vaenulikud.
\par 42 Demeetrios läkitas siis Joonatanile sõna: „Ma teen sinule ja sinu rahvale mitte ainult seda, vaid ma tahan sind ja sinu rahvast ka väga austada, kui selleks on sobiv silmapilk.
\par 43 Nüüd siis sa teed hästi, kui läkitad mulle mehi, kes aitaksid mul sõdida, sest kogu mu sõjavägi on minust taganenud.”
\par 44 Joonatan läkitaski temale Antiookiasse kolm tuhat vaprat meest ja kui need kuninga juurde tulid, tundis kuningas rõõmu nende saabumisest.
\par 45 Linna elanikud aga kogunesid linna keskele, ligi sada kakskümmend tuhat meest, ja tahtsid kuninga tappa.
\par 46 Siis kuningas põgenes paleesse, linna elanikud aga vallutasid linna sissepääsud ja alustasid võitlust.
\par 47 Nüüd kutsus kuningas juudid appi ja need kõik kogunesid kohe tema juurde ning sedamaid hajus kogu rahvas linnas ja nad tapsid neist sel päeval ligi sada tuhat.
\par 48 Nad süütasid linna põlema ning võtsid sel päeval palju saaki ja päästsid kuninga.
\par 49 Aga kui linna elanikud nägid, et juudid olid saanud linna oma võimusesse, nagu need olid tahtnud, siis nad kaotasid julguse ja kisendasid kuninga poole palvega, üteldes:
\par 50 „Anna meile käsi, et juudid lõpetaksid sõdimise meie ja linna vastu!”
\par 51 Siis nad heitsid sõjariistad ära ja tegid rahu. Juudid tõusid aga ausse kuninga ja kõigi tema kuningriigis olijate ees ja nad läksid tagasi Jeruusalemma suure saagiga.
\par 52 Ja kuningas Demeetrios istus jälle oma kuningriigi aujärjel ning tema maal oli rahu.
\par 53 Aga ta salgas ära kõik, mis ta oli lubanud, ja võõrdus Joonatanist ega tasunud temale tehtud heategude eest, vaid rõhus teda väga. 

\section*{Joonatan Demeetrios II vastu}

\par 54 Pärast seda tuli aga Trüfon tagasi ja koos temaga oli Antiohhos, kes oli alles noor poiss. Antiohhos sai kuningaks ja temale pandi kroon pähe.
\par 55 Tema juurde kogunes terve see sõjavägi, kes Demeetriose poolt oli laiali aetud, ja sõdis nüüd Demeetriose vastu. See põgenes ja võideti.
\par 56 Ja Trüfon võttis enesele elevandid ning vallutas Antiookia.
\par 57 Noor Antiohhos kirjutas siis Joonatanile nõnda: „Mina kinnitan sulle ülempreestriameti ja panen su valitsema nelja piirkonda ja sa kuulud kuninga sõprade hulka.”
\par 58 Ta saatis temale kuldriistu ja lauatarbeid ning lubas tal juua kuldkarikaist, riietuda purpurisse ja kanda kuldpannalt.
\par 59 Tema venna Siimoni pani ta väepealikuks „Tüürose trepist” kuni Egiptuse piirini.
\par 60 Joonatan läks nüüd välja, ületas jõe ja läks läbi linnade. Tema juurde kogunesid kõik Süüria sõjajõud, et teda võitluses aidata. Siis ta tuli Askeloni ja linna elanikud tulid temale vastu austusavaldustega.
\par 61 Sealt läks ta Gazasse, aga Gaza elanikud sulgesid väravad. Siis ta piiras linna, põletas tulega eeslinnad ja rüüstas need.
\par 62 Nüüd palusid Gaza elanikud Joonatani ja tema andis neile rahukäe, võttis nende ülikute pojad pantvangideks ja saatis need Jeruusalemma. Siis ta käis maa läbi kuni Damaskuseni.
\par 63 Kui Joonatan kuulis, et Demeetriose pealikud olid tulnud suure sõjaväega Galilea Kaadesisse, et teda tema tegevuses takistada,
\par 64 siis ta läks nende vastu. Aga oma venna Siimoni jättis ta kodumaale. 

\section*{Siimon vallutab Beet-Suuri}

\par 65 Siimon lõi leeri üles Beet-Suuri alla ja sõdis kaua aega selle vastu, seda sisse piirates.
\par 66 Siis nad palusid temalt rahukätt ja tema andis selle neile, ajas aga nad sealt ära, vallutas linna ja paigutas sinna kaitseväe. 

\section*{Taplus Haasoris}

\par 67 Joonatan ja tema sõjavägi lõid leeri üles Genneesareti järve kaldale ning läksid varahommikul Haasori lagendikule.
\par 68 Vaata, siis tuli lagendikul tema vastu võõramaalaste sõjavägi. Ja need, olles mäestikku paigutanud ka varitsejad, ründasid teda otsekohe.
\par 69 Siis tulid varitsejad oma paikadest välja ja sekkusid taplusesse.
\par 70 Kõik Joonatani mehed põgenesid nüüd, ja neist jäid paigale ainult Mattatias, Absalomi poeg, ja Juudas, Halfi poeg, kes olid sõjaväe ülempealikud.
\par 71 Joonatan käristas siis oma riided lõhki, raputas enesele mulda pähe ja palvetas.
\par 72 Seejärel pöördus ta tagasi võitlema nende vastu, võitis nad, ja nad põgenesid.
\par 73 Kui need tema mehed, kes olid põgenenud, seda nägid, siis nad pöördusid tagasi tema juurde ja ajasid koos temaga vaenlasi taga kuni Kaadesini, kuni nende leerini, ja jäid ise sinna leeri.
\par 74 Võõramaalasi langes sel päeval ligi kolm tuhat meest. Ja Joonatan läks tagasi Jeruusalemma.

\chapter{12}

\section*{Joonatani sidemed Rooma ja Spartaga}

\par 1 Kui Joonatan nägi, et aeg on temale soodus, siis ta valis mehed ja läkitas need Rooma, kinnitama ja uuendama nende vahel olevat sõprust.
\par 2 Ka Spartasse ja teistesse paikadesse saatis ta samasisulisi kirju.
\par 3 Kui saadikud olid saabunud Rooma, siis nad läksid senatisse ja ütlesid: „Ülempreester Joonatan ja Juuda rahvas on meid läkitanud uuendama teiega sõprust ja liitu, mis varem on olnud.”
\par 4 Ja roomlased andsid neile kaasa kirjad, et nad rahuga saaksid tagasi Juudamaale.
\par 5 See on ärakiri kirjast, mille Joonatan kirjutas spartalastele:
\par 6 „Joonatan, ülempreester, ja rahva vanematekogu, preestrid ja muu juudi rahvas, tervitavad oma vendi spartalasi.
\par 7 Juba varem saatis teie kuningas Areios ülempreester Oniasele kirja, et teie olete meie vennad. Selle ärakiri on siin juures.
\par 8 Onias kohtles läkitatud meest austusega ning võttis vastu kirjad, milles tehti selgeks liit ja sõprus.
\par 9 Kuigi me nüüd neid kirju ei vaja, sest meie julgustus on nendes pühades raamatutes, mis meil on,
\par 10 oleme siiski ette võtnud saadikute läbi uuendada teiega vennalikkust ja sõprust, et meie teist ei võõrduks, sest teie läkitusest meile on möödunud palju aega.
\par 11 Seepärast me nüüd meenutame teid lakkamatult igal ajal: pühade ajal ja muil kohaseil päevil, ohvreid ohverdades ja palvetades, nõnda nagu on kohus ja õige vendi meenutada.
\par 12 Meie tunneme rõõmu teie auväärsusest.
\par 13 Meid aga on piiranud palju muresid ja palju võitlusi, ja ümberkaudsed kuningad on sõdinud meie vastu.
\par 14 Ometi ei ole meie tahtnud nende sõdadega tüli teha teile ja oma teistele liitlastele ja sõpradele.
\par 15 Sest meil on abi taevast, mis meid aitab, ja seepärast oleme päästetud vaenlaste käest, meie vaenlased on aga alandatud.
\par 16 Aga nüüd oleme valinud Numeeniose, Antiohhose poja, ja Antipatrose, Jaasoni poja, ja läkitanud need roomlaste juurde uuendama nendega endist sõprust ning liitu.
\par 17 Me oleme nüüd neile ülesandeks teinud minna ka teie juurde ja teid tervitada ning anda teile meie kirjad meie vendluse uuendamiseks.
\par 18 Ja te teete nüüd hästi, kui nende peale meile vastate.”
\par 19 See on ärakiri kirjast, mis Oniasele saadeti:
\par 20 „Areios, spartalaste kuningas, tervitab ülempreester Oniast!
\par 21 Ühest kirjast on leitud, et spartalased ja juudid on vennad ning Aabrahami soost.
\par 22 Kui me nüüd seda teame, siis teeksite hästi meile kirjutades, kuidas teie käsi käib.
\par 23 Ja meie kirjutame teile vastuseks, et teie karjad ja varandus on meie ja meie omand on teie. Seepärast käsime nüüd, et see teile teatavaks tehtaks.” 

\section*{Joonatani võitlused}

\par 24 Kui Joonatan kuulis, et Demeetriose väepealikud veel suurema sõjaväega kui enne olid tagasi tulnud tema vastu sõdima,
\par 25 siis ta läks Jeruusalemmast teele ja läks neile vastu Hamati piirkonda, sest ta ei tahtnud anda neile võimalust tungida tema maale.
\par 26 Ja ta läkitas nende leeri salakuulajad. Need tulid tagasi ning teatasid temale, et valmistutakse kallaletungiks öösel.
\par 27 Aga kui päike loojus, siis käskis Joonatan oma mehi valvata ja jääda sõjariistade juurde ning olla valmis võitluseks kogu öö. Ta paigutas vahid ümber leeri.
\par 28 Kui vastased kuulsid, et Joonatan ja tema mehed on võitluseks valmis, siis nad kartsid ja kaotasid julguse, süütasid leeris tuled ja läksid ära.
\par 29 Aga Joonatan ja tema mehed ei märganud seda enne kui hommikul, nähes põlevaid tulesid.
\par 30 Joonatan ajas siis neid taga, ei saanud aga kätte, sest nad olid juba läinud üle Eleuterose jõe.
\par 31 Joonatan pöördus nüüd nende araablaste vastu, keda hüütakse sabadlasteks, võitis nad ja võttis neilt saaki.
\par 32 Pärast seda läks ta edasi ja tuli Damaskusesse ning käis läbi kogu maa. 

\section*{Siimoni sõjakäik}

\par 33 Ka Siimon oli välja läinud ja liikunud kuni Askelonini ja selle läheduses olevate kindlusteni, pöördus siis Joppe vastu ja vallutas selle.
\par 34 Sest ta oli kuulnud, et kindlus tahetakse anda Demeetriose poolehoidjatele. Ta pani sinna väesalga, et seda valvata. 

\section*{Jeruusalemma kindlustustöö}

\par 35 Kui Joonatan oli tagasi tulnud, siis ta kutsus kokku rahva vanemad ja pidas nendega nõu kindluse ehitamiseks Juudamaale,
\par 36 Jeruusalemma müüride kõrgendamiseks, suure, kõrge müüri püstitamiseks kindluse ja linna vahele, et kindlust linnast eraldada, nõnda et see jääks omaette ja et kindluses olijad ei saaks osta ega müüa.
\par 37 Ja nad kogunesid linna üles ehitama. Kuna osa idapoolset müüri oli varisenud jõkke, siis nad taastasid selle, Hafenataks nimetatu.
\par 38 Siimon ehitas aga üles Adida Sefeelas, kindlustas selle ning varustas väravate ja riividega. 

\section*{Joonatan langeb vaenlaste kätte}

\par 39 Aga Trüfon tahtis saada Aasia kuningaks, panna enesele kroon pähe ja sirutada käsi kuningas Antiohhose vastu.
\par 40 Siiski ta kartis, et Joonatan seda temale iialgi ei luba, küll aga sõdib tema vastu. Ta otsis viisi, kuidas teda vangistada ja tappa. Ta läks teele ja tuli Beet-Seani.
\par 41 Joonatan läks välja temale vastu neljakümne tuhande valitud, võitlusvalmis mehega ja tuli ka Beet-Seani.
\par 42 Kui Trüfon nägi, et ta oli tulnud suure sõjaväega, siis ta kartis pista oma kätt tema külge.
\par 43 Ta võttis tema vastu austusega, kiitis teda kõigile oma sõpradele, andis temale kingitusi ja käskis oma sõpru ning vägesid kuulata Joonatani sõna nagu tema enese sõna.
\par 44 Ta ütles Joonatanile: „Mispärast sa oled väsitanud kogu seda rahvast, sest meid ei ähvarda ju sõda?
\par 45 Seepärast saada need nüüd koju, aga vali enesele pisut mehi, kes sind saadaksid, ja tule koos minuga Ptolemaisi, ja ma annan sinule selle ja ka teised kindlused ning teised väeosad, nõndasamuti kõik ametimehed. Mina ise pöördun tagasi ja lähen ära, sest sellepärast ma siin olengi.”
\par 46 Joonatan uskus teda ja tegi nõnda, nagu Trüfon oli ütelnud: ta saatis väehulgad ära ja need läksid Juudamaale.
\par 47 Aga kolm tuhat meest jättis ta enese juurde: neist kaks tuhat jättis ta Galileasse, kuid tuhat läks koos temaga.
\par 48 Aga kui Joonatan oli tulnud Ptolemaisi, siis ptolemaislased sulgesid väravad, võtsid ta vangi ja tapsid mõõgaga kõik, kes koos temaga olid sisse tulnud.
\par 49 Trüfon läkitas nüüd jalaväge ja ratsanikke Galileasse ja suurele lagendikule tapma kõiki Joonatani mehi.
\par 50 Aga kui need teada said, et Joonatan koos oma meestega on vangistatud ja hukatud, siis nad julgustasid üksteist ja liikusid tihedais ridades võitlusvalmina.
\par 51 Kui tagaajajad nägid, et teised olid valmis võitlema oma hinge eest, siis nad pöördusid tagasi.
\par 52 Nõnda pääsesid nad kõik rahuga Juudamaale. Nad leinasid Joonatani ning tema mehi, tundes ühtlasi tõsist kartust. Ja kogu Iisrael leinas väga.
\par 53 Kõik ümberkaudsed rahvad aga püüdsid neid hävitada. Sest nad ütlesid: „Neil ei ole pealikut ega aitajat. Sõdime nüüd nende vastu ja kaotame mälestuse neist inimeste keskel.”

\chapter{13}

\section*{Siimon saab ülempreestriks}

\par 1 Kui Siimon kuulis, et Trüfon oli kogunud suure sõjaväe Juudamaale tulekuks ja selle hävitamiseks,
\par 2 ja kui ta nägi, et rahvas oli hirmul ja kartis, siis ta läks üles Jeruusalemma ja kogus rahva kokku.
\par 3 Ta julgustas neid ning ütles neile: „Teie teate ise, mida mina ja minu vennad ja minu isakoda oleme teinud Seaduse ja pühamu heaks, ja missugust võitlust ja hädasid me oleme näinud.
\par 4 Sellepärast on kõik minu vennad hukkunud Iisraeli eest ja mina üksi olen järele jäänud.
\par 5 Nüüd ärgu sündigu minuga seda, et säästaksin oma hinge ükskõik missugusel viletsusajal! Sest mina ei ole parem kui mu vennad.
\par 6 Tõesti, ma tahan kätte maksta oma rahva ja pühamu eest, meie naiste ja laste eest, sest kõik paganad on vihavaenus ühinenud meie hävitamiseks.”
\par 7 Ja rahva vaim elustus, niipea kui nad neid sõnu kuulsid.
\par 8 Nad vastasid suure häälega ning ütlesid: „Sina oled meie juht Juuda ja sinu venna Joonatani asemel!
\par 9 Juhi sina meie võitlust ja me teeme kõike, mida sa käsid!”
\par 10 Siis ta kogus kõik võitlusvõimelised mehed, laskis kiiresti valmis ehitada Jeruusalemma müürid ning kindlustas seda ümberringi.
\par 11 Ja ta läkitas Joppesse Joonatani, Absalomi poja, ja koos temaga küllalt suure väeosa, kes ajas seal olijad välja ja jäi ise sinna. 

\section*{Siimon tõrjub Trüfoni Juudamaalt}

\par 12 Nüüd lahkus Trüfon Ptolemaisist suure sõjaväega, et minna Juudamaale, ja Joonatan oli vangina temaga kaasas.
\par 13 Siimon aga lõi leeri üles Adidasse, lagendiku äärde.
\par 14 Kui Trüfon teada sai, et Siimon oli astunud oma venna Joonatani asemele ja tahab hakata temaga sõdima, siis ta läkitas tema juurde saadikud ütlema nõnda:
\par 15 „Selle raha pärast, mida sinu vend Joonatan võlgneb kuninglikule varakambrile ametite eest, mis tal on olnud, peame teda kinni.
\par 16 Saada nüüd sada talenti hõbedat ja kaks tema poega pandiks, et ta vabaks saades meist ei taganeks, siis me vabastame tema!”
\par 17 Siimon sai aru küll, et nad temale valet rääkisid, ometi andis ta käsu tuua raha ja pojad, et mitte saada rahva suure viha aluseks
\par 18 ja et ei öeldaks: „Sellepärast, et ta ei ole saatnud temale raha ja poegi, on Joonatan hukkunud.”
\par 19 Ta saatis siis pojad ja sada talenti, aga Trüfon oligi valetanud ega vabastanud Joonatani.
\par 20 Seejärel tuli Trüfon, et maale kallale tungida ja seda rüüstata. Ta läks kõverat teed Adoora kaudu, aga Siimon ja tema sõjavägi liikusid kõrvuti temaga igasse paika, kuhu ta iganes läks.
\par 21 Kindluses olijad läkitasid nüüd saadikud Trüfoni juurde, et ta kiiresti tuleks nende juurde läbi kõrbe ja saadaks neile moona.
\par 22 Siis seadis Trüfon minekuvalmis kogu oma ratsaväe. Aga selsamal ööl sadas väga palju lund ja lume pärast ei saanud ta tulla, vaid asus teele ja läks Gileadi.
\par 23 Aga jõudes Baskama lähedale, surmas ta Joonatani, kes maeti sinna.
\par 24 Trüfon pöördus siis tagasi ja läks oma maale. 

\section*{Siimon matab Joonatani luud}

\par 25 Siimon aga laskis tuua oma venna Joonatani luud ja mattis need Moodeini, tema vanemate linna.
\par 26 Ja kogu Iisrael leinas teda väga ning kaua aega.
\par 27 Siimon ehitas oma isa ja vendade hauale kõrge, kaugelt nähtava ehitise, tahutud kividega esi- ja tagaküljel.
\par 28 Selle peale püstitas ta seitse püramiidi üksteisega vastakuti: isale, emale ja neljale vennale.
\par 29 Ja püramiidide peale tegi ta kunstipärased kaunistused, paigutas ümberringi suured sambad, sammaste peale täieliku sõjavarustuse igaveseks mälestuseks, ja sõjariistade kõrvale kivisse raiutud laevad, nähtavad kõigile meresõitjaile.
\par 30 See haud, mille ta Moodeinis tegi, seisab tänapäevani. 

\section*{Demeetrios tunnustab Siimonit}

\par 31 Aga Trüfon käitus petlikult noore kuninga Antiohhosega ja tappis ta.
\par 32 Ta ise sai siis tema asemel kuningaks, pani Aasia krooni enesele pähe ja tegi maale suurt kahju.
\par 33 Siimon taastas aga Juuda kindlused, ümbritses need kõrgete tornide ja tugevate müüridega koos väravate ja riividega, ning varus kindlustesse toidumoona.
\par 34 Siimon valis siis mehed ja läkitas need kuningas Demeetriose juurde, et nõutada maale maksukergendust, sest kõik Trüfoni teod olid riisumised.
\par 35 Ja kuningas Demeetrios läkitas temale nõusoleku, vastas temale ning kirjutas talle niisuguse kirja:
\par 36 „Kuningas Demeetrios tervitab ülempreestrit ja kuninga sõpra Siimonit, ka vanemaid ja juudi rahvast!
\par 37 Kuldkrooni ja palmioksa, mis te olete saatnud, oleme kätte saanud, ja me oleme valmis tegema teiega täielikku rahu ning kirjutama ametimeestele, et nad annaksid teile maksukergenduse.
\par 38 Mis me teie kohta oleme kindlaks määranud, see jääb kindlaks, ja kindlused, mis te olete ehitanud, kuulugu teile!
\par 39 Me andestame teile ka tänapäevani juhtunud eksimused ja väärsammud, ja kroonimaksu, milleks olete kohustatud; ka muud maksu, mida Jeruusalemmas on makstud, ei tule enam maksta.
\par 40 Kui mõned teie seast sobivad meie ihukaitsesse, siis võetagu nad sinna! Ja meie vahel olgu rahu!”
\par 41 Saja seitsmekümnendal aastal võeti paganate ike Iisraeli pealt.
\par 42 Ja rahvas hakkas kirjutama ürikuis ja lepinguis: „Siimoni, juutide suure ülempreestri, väepealiku ja valitseja esimesel aastal.” 

\section*{Siimon vallutab Geseri}

\par 43 Neil päevil lõi Siimon leeri üles Geseri ette ja piiras seda sõjaväega. Ta valmistas ründemasina, tõi selle linna alla ning purustas ühe torni ja vallutas selle.
\par 44 Ründemasinas olijad sööstsid nüüd linna ja linnas sündis suur segadus.
\par 45 Linna elanikud koos naiste ja lastega läksid üles müürile, käristasid oma riided lõhki ja kisendasid suure häälega, anudes, et Siimon annaks neile rahukäe.
\par 46 Nad ütlesid: „Ära talita meiega meie süütegusid mööda, vaid talita oma halastust mööda!”
\par 47 Siis leppis Siimon nendega kokku ega sõdinud enam nende vastu. Aga ta ajas nad linnast välja ja puhastas kojad, kus ebajumalakujud sees olid, ning läks siis sisse lauldes ja kiites.
\par 48 Ja ta kõrvaldas sealt kõik rüveduse ning pani sinna mehi, kes Seadusest kinni pidasid. Ta kindlustas linna veelgi ja ehitas sinna enesele eluaseme. 

\section*{Siimon vallutab Jeruusalemma kindluse}

\par 49 Aga Jeruusalemma kindluses olijaid takistati maale minemast ja maalt tulemast, ostmast ja müümast, ja neil oli väga suur puudus ning paljud neist surid nälga.
\par 50 Nad hüüdsid siis Siimoni poole, et nad saaksid rahukäe. Ja tema andis selle neile. Aga ta ajas ka nemad sealt välja ja puhastas kindluse saastast.
\par 51 Ta läks sinna teise kuu kahekümne kolmandal päeval aastal sada seitsekümmend üks tänulauludega ja palmioksi kandes, kannelde, simblite ja naablite ning kiitus- ja ülistuslaulude kõlades, et suur vaenlane oli Iisraelist hävitatud.
\par 52 Ja ta seadis, et seda päeva tuli igal aastal rõõmuga pühitseda. Siis ta kindlustas kindluse kõrval olevat templimäge veel rohkem ja asus ise omastega sinna.
\par 53 Kui nüüd Siimon nägi, et tema poeg Johannes oli võimekas mees, siis ta pani tema kogu sõjaväe juhiks. Ja Johannes asus Geserisse.

\chapter{14}

\section*{Ülistuslaul Siimonile}

\par 1 Aastal sada seitsekümmend kaks kogus kuningas Demeetrios oma sõjaväe ja läks Meediasse, et hankida enesele abi sõjaks Trüfoni vastu.
\par 2 Kui Arsakes, Pärsia ja Meedia kuningas, kuulis, et Demeetrios oli tunginud tema valdustesse, siis ta läkitas ühe oma pealikuist teda elusana kinni võtma.
\par 3 See läks teele, võitis Demeetriose sõjaväe, võttis tema enese kinni ja viis Arsakese ette, kes pani ta vangi.
\par 4 Juudamaal oli siis rahu kõigil Siimoni päevil. Tema tahtis oma rahvale head ja tema valitsus ning kuulsus oli rahvale meelepärane kõigil neil päevil.
\par 5 Lisaks kõigele oma kuulsusele vallutas ta Joppe sadamalinnaks ja kindlustas pääsu meresaartele.
\par 6 „Ta laiendas piirid oma rahvale ja valitses maad.
\par 7 Ta tõi koju palju vangistatuid ja sai Geseri, Beet-Suuri ning kindluse valitsejaks. Ta koristas sealt saasta ja temale ei olnud vastupanijat.
\par 8 Nad harisid oma maad rahus ja maa andis saaki ning väljade puud kandsid vilja.
\par 9 Vanemad mehed istusid tänavail, kõik kõnelesid headest asjadest, ja noored mehed kandsid auga sõjarüüd.
\par 10 Ta muretses linnadele toitu ja andis neile varustuse. Jah, tema auväärset nime nimetati maailma ääreni.
\par 11 Ta tõi maale rahu ja Iisrael oli väga rõõmus.
\par 12 Igaüks istus oma viinapuu ja viigipuu all, keegi ei peletanud neid.
\par 13 Olid kadunud need, kes maal sõdisid nende vastu, ja neil päevil võideti kuningad lõplikult.
\par 14 Tema toetas kõiki oma rahva rõhutuid, ja nõudis Seaduse täitmist ning hävitas kõik taganejad ja kurjategijad.
\par 15 Ta kaunistas templi toredaks ja suurendas pühamu riistade rohkust.” 

\section*{Uuendatakse liidud Sparta ja Roomaga}

\par 16 Kui Roomas ja Spartas kuuldi, et Joonatan oli surnud, siis nad leinasid väga.
\par 17 Aga kui nad kuulsid, et tema vend Siimon oli saanud tema asemel ülempreestriks ning valitses nüüd maad ja selle linnasid,
\par 18 siis nad kirjutasid temale vasktahvlitel, et uuendada temaga sõprust ja sõjaliitu, mille nad olid teinud tema vendadega, Juuda ja Joonataniga.
\par 19 See läkitus loeti ette kogudusele Jeruusalemmas.
\par 20 Ja see on ärakiri sellest kirjast, mille spartalased saatsid: „Spartalaste peamehed ja linn tervitavad ülempreester Siimonit, vanemaid, preestreid ja muud juudi rahvast, oma vendi!
\par 21 Saadikud, kes meie rahva juurde on läkitatud, on meile jutustanud teie kuulsusest ja aust, ja meie oleme rõõmu tundnud nende tulekust.
\par 22 Ja sellest, mida nad rääkisid, oleme märkinud üles nõnda: Numeenios, Antiohhose poeg, ja Antipatros, Jaasoni poeg, juutide saadikud, on tulnud meie juurde uuendama sõprust meiega.
\par 23 Ja rahval oli hea meel neid austusega vastu võtta ning panna ärakiri nende kõnest ametlikesse kirjadesse, meelespidamiseks Sparta rahvale. Ärakirja sellest me kirjutame ülempreester Siimonile.”
\par 24 Pärast seda läkitas Siimon Numeeniose Rooma, kaasas suur kuldkilp, tuhat miini raske, et kinnitada liitu nendega. 

\section*{Siimoni austamine}

\par 25 Kui rahvas nendest sündmustest kuulis, siis nad ütlesid: „Kuidas peaksime tänama Siimonit ja tema poegi?
\par 26 Sest tema, tema vennad ja isakoda on kindlaks jäänud ja on võidelnud Iisraeli vaenlaste vastu, ajades need ära, ja on kindlustanud maale vabaduse.” Nad kirjutasid sellest vasktahvlitele ja panid need sammaste külge Siioni mäel.
\par 27 See on selle kirja ärakiri: „Elulikuu kaheksateistkümnendal päeval saja seitsmekümne teisel aastal, mis on suure ülempreestri Siimoni, Jumala rahva vürsti kolmas aasta,
\par 28 on meile preestrite ja rahva, rahva ülemate ja maa vanemate suures koguduses teatavaks tehtud see:
\par 29 „Kuna maal on sageli olnud sõdu, siis on Siimon, Mattatiase poeg, Joojaribi poegade järglaste hulgast, ja tema vennad oma elu hädaohtu seadnud ning oma rahva vaenlastele vastu pannud, et nende pühamu ja Seadus jääksid püsima, ja nad on toonud oma rahvale suure kuulsuse.
\par 30 Kui Joonatan oli ühendanud oma rahva ning oli saanud neile ülempreestriks, ja kui ta oli ära võetud oma rahva juurde,
\par 31 siis tahtsid nende vaenlased tungida nende maale ja oma käe sirutada nende pühamu vastu.
\par 32 Aga Siimon tõusis ja sõdis oma rahva eest. Ta kulutas palju oma varandusest, varustas sõjariistadega oma rahva sõjamehi ja andis neile tasu.
\par 33 Ta kindlustas Juudamaa linnasid ning Beet-Suuri Juudamaa piiril, kus enne olid vaenlaste sõjariistad olnud, ja paigutas juudid sinna vahimeesteks.
\par 34 Ta kindlustas ka Joppe, mis on mere ääres, ja Geseri Asdodi piirkonnas, kus enne olid vaenlased elanud, ja saatis sinna juute elama. Ja ta andis neile kõik, mis nende korrashoiuks oli tarvilik.
\par 35 Kui rahvas nägi Siimoni ustavust ja kuulsust, mida ta oli tahtnud oma rahvale tuua, siis nad tegid tema enestele juhiks ja ülempreestriks, sellepärast et ta seda kõike oli teinud, ja et ta oli osutanud õiglust ja truudust oma rahva vastu, ja et ta oli püüdnud oma rahvast igati ülendada.
\par 36 Tema päevil õnnestus tema juhatusel ajada maalt paganad, nõndasamuti need, kes olid Taaveti linnas Jeruusalemmas ja olid sinna teinud enestele kindluse, kust nad välja tulid ja pühamu ümbruse rüvetasid, teotades suuresti selle pühadust.
\par 37 Tema asustas sinna juute ning kindlustas seda, julgestuseks maale ja linnale, ja tegi Jeruusalemma müürid kõrgemaks.
\par 38 Selle põhjal kinnitas kuningas Demeetrios talle ülempreestriameti
\par 39 ning võttis ta oma sõprade hulka ja austas teda suurte austusavaldustega.
\par 40 Sest ta oli kuulda saanud, et roomlased olid kuulutanud juudid oma sõpradeks, liitlasteks ja vendadeks ning olid Siimoni saadikud austusega vastu võtnud.
\par 41 Ja nüüd on juudid ja preestrid otsustanud, et Siimon on igavesti neile juhiks ja ülempreestriks, kuni tõuseb usaldatav prohvet,
\par 42 ja et ta oleks neile väepealikuks, hoolitseks ka pühamu eest, seades ülevaatajaid selle tööde, nõndasamuti ka maa, sõjariistade ja kindluste üle.
\par 43 Tema hooleks on pühamu ja kõik peavad tema sõna kuulama. Tema nimel kirjutatagu kõik maa ametlikud kirjad! Tema võib kanda purpurit ja kulda.
\par 44 Mitte keegi rahva ja preestrite hulgast ei tohi midagi sellest tühjaks teha, tema korraldustele vastu rääkida, ilma tema loata maal rahvast kokku kutsuda, purpurmantlit kanda ja ennast ehtida kuldpandlaga!
\par 45 Kui aga keegi selle vastu talitab või sellest midagi tühjaks teeb, siis karistatagu teda!”
\par 46 Ja rahvas otsustas, et Siimon peab tegema nende sõnade kohaselt.
\par 47 Siimon kiitis otsuse heaks ning nõustus olema ülempreester, väepealik, juutide ja preestrite vürst ning juhtima kõiki.
\par 48 Nad otsustasid, et see kiri märgitakse vasktahvlitele, mis seatakse üles nähtavatesse paikadesse pühamu ümber,
\par 49 ja et ärakiri neist paigutatakse varakambrisse Siimoni ja tema poegade jaoks.

\chapter{15}

\section*{Antiohhos tunnustab Siimoni õigusi}

\par 1 Antiohhos, kuningas Demeetriose poeg, saatis meresaartelt kirju Siimonile, juutide ülempreestrile ja vürstile, ja kogu rahvale.
\par 2 Nende sisu oli niisugune: „Kuningas Antiohhos tervitab Siimonit, ülempreestrit ja vürsti, ja juudi rahvast!
\par 3 Kuna nurjatud on vallutanud meie isade kuningriigi, mina tahan aga riiki tagasi saada, siis olen ma kogunud suure sõjaväe ja varustanud sõjalaevu,
\par 4 ja ma tahan tungida maale, et karistada neid, kes on hävitanud meie maa ning on rüüstanud paljusid minu riigi linnasid,
\par 5 ja sellepärast ma kinnitan sinule kõik need soodustused, mis enne mind olnud kuningad sinule on andnud, ja igasugu andamid, mis sinult on ära jäetud.
\par 6 Ja ma luban sind müntida eriraha sinu oma maa tarvis.
\par 7 Jeruusalemm ja pühamu olgu maksuvabad! Ja kõik sõjariistad, mis sa oled hankinud, ja sinu poolt ehitatud kindlused, mis on sinu valduses, jäävad sinule.
\par 8 Ja seda, mida sa võlgned kuningale ja mida sa tulevikus võiksid võlgneda, ei nõuta sinult kunagi tagasi.
\par 9 Kui me oma riigi oleme kätte saanud, siis me austame sind ja sinu rahvast ning templit suurte austusavaldustega, nõnda et teie kuulsus saab teatavaks kogu maailmas.” 

\section*{Antiohhos kimbutab Trüfonit}

\par 10 Aastal sada seitsekümmend neli läks Antiohhos oma isade maale, ja kõik väesalgad tulid üle tema poole, nõnda et vähesed jäid Trüfoni juurde.
\par 11 Ja Antiohhos ajas Trüfonit taga, kes põgenedes tuli Doorasse, mis on mere ääres.
\par 12 Sest ta sai aru, et teda olid õnnetused tabanud ja sõjavägi oli tema maha jätnud.
\par 13 Antiohhos lõi leeri üles Doora vastu, ja temaga oli sada kakskümmend tuhat jalameest ja kaheksa tuhat ratsanikku.
\par 14 Ta piiras linna ümber ning laevad võtsid võitlusest osa mere poolt. Nõnda ta ründas linna maa ja mere poolt ega lasknud kedagi välja tulla või sisse minna. 

\section*{Uuendatud liiduleping roomlastega}

\par 15 Siis tulid Numeenios ja need, kes koos temaga olid, Roomast, kaasas kirjad kuningatele ja maadele. Neis oli kirjutatud nõnda:
\par 16 „Luukius, roomlaste konsul, tervitab kuningas Ptolemaiost!
\par 17 Juutide saadikud tulid meie juurde kui meie sõbrad ja liitlased uuendama vana sõprust ja liitu, läkitatuna ülempreester Siimoni ja juudi rahva poolt.
\par 18 Ja nad tõid ühe kuldkilbi, tuhat miini raske.
\par 19 Sellepärast on meile nüüd meeltmööda kirjutada kuningatele ja maadele, et nad ei püüaks neile kurja teha, ei sõdiks nende vastu, nende linnade ja maade vastu, ega teeks liitu nende vaenlastega.
\par 20 Me oleme otsustanud kilbi neilt vastu võtta.
\par 21 Kui nüüd mõned nurjatud on oma maalt põgenenud teie juurde, siis andke need välja ülempreester Siimonile, et ta saaks neid karistada nende seaduse kohaselt!”
\par 22 Sedasama kirjutas ta kuningas Demeetriosele, Attalosele, Ariaratesele ja Arsakesele,
\par 23 ning kõigile maadele: Sampsame, Sparta, Deelos, Mündos, Siküon, Kaaria, Samos, Pamfüülia, Lüükia, Halikarnassos, Roodos, Faseelis, Koos, Siide, Arados, Gortüna, Knidos, Küpros ja Küreene.
\par 24 Aga ärakirja sellest kirjast saatsid nad ülempreester Siimonile. 

\section*{Antiohhos asub Siimoniga vaenujalale}

\par 25 Kuningas Antiohhos lõi aga teisel päeval leeri üles Doora vastu, ründas seda lakkamatult ja valmistas piiramisseadmeid. Ta sulges Trüfoni sinna, nõnda et keegi ei pääsenud ei sisse ega välja.
\par 26 Siimon läkitas temale kaks tuhat valitud meest aitama teda võitluses, hõbedat, kulda ja küllaldaselt varustust.
\par 27 Aga kuningas ei tahtnud seda vastu võtta, vaid tühistas kõik, mis ta varem oli talle lubanud, ja ei tahtnud teda enam tunda.
\par 28 Ta läkitas Siimoni juurde ühe oma sõpradest, Atenoobiose, temaga läbi rääkima, käskides ütelda: „Teie peate oma valduses Joppet, Geserit ja Jeruusalemma kindlust, minu kuningriigi linnu.
\par 29 Te olete nende alad rüüstanud, tehes maale suurt kahju, ja olete vallutanud palju paiku minu kuningriigis.
\par 30 Andke nüüd tagasi äravõetud linnad ja makske teie poolt anastatud, väljaspool Juuda ala olevate paikade makse!
\par 31 Aga kui mitte, siis andke selle asemel viissada talenti hõbedat, ja tehtud kahju ning linnade maksu eest veel viissada talenti! Kui mitte, siis tuleme teie vastu sõdima.”
\par 32 Ja kuninga sõber Atenoobios tuli Jeruusalemma. Kui ta nägi Siimoni toredust, lauda kuld- ja hõbenõudega ja väga suurt teenijaskonda, siis oli ta hämmastunud. Ta andis temale edasi kuninga sõnad.
\par 33 Aga Siimon vastas temale, öeldes: „Meie ei ole võtnud võõrast maad ega ole vallutanud teistele kuuluvat, vaid ainult meie isade pärandiosa, mille meie vaenlased olid mõneks ajaks õigusevastaselt anastanud.
\par 34 Aga et meil on nüüd õige aeg käes, siis me hoiame kinni oma isade pärandist.
\par 35 Mis puutub Joppesse ja Geserisse, mida sa tagasi nõuad, siis on need meie rahvale ja maale palju kahju teinud. Nende eest me siiski anname sada talenti.”
\par 36 Aga Atenoobios ei vastanud temale sõnagi, vaid läks suure vihaga tagasi kuninga juurde ja kuulutas temale need sõnad ning jutustas Siimoni toredusest ja kõigest, mida ta oli näinud. Siis sai kuningas väga vihaseks.
\par 37 Trüfon läks aga laeva ja põgenes Ortoosiasse. 

\section*{Kendebaios ründab Juudamaad}

\par 38 Kuningas pani Kendebaiose rannikuala asevalitsejaks ning andis temale jala- ja ratsaväge.
\par 39 Ja käskis tal leer üles lüüa Juudamaa vastu, käskis tal ka Kedroon üles ehitada, kindlustada kitsast teed ja sõdida rahvaga. Kuningas ise aga ajas Trüfonit taga.
\par 40 Ja Kendebaios tuli nüüd Jamniasse ning hakkas rahvast ässitama ja Juudamaale tungima, inimesi vangi võtma ja tapma.
\par 41 Ta ehitas üles Kedrooni ning paigutas sinna ratsa- ja jalaväge, mis pidi minema retkile Juudamaa teedel, nõnda nagu kuningas oli temale käsu andnud.

\chapter{16}

\section*{Siimoni poegade võit Kendebaiose üle}

\par 1 Aga Johannes läks Geserist üles ja teatas oma isale Siimonile, mida Kendebaios oli korda saatnud.
\par 2 Siimon kutsus siis oma kaks vanemat poega, Juuda ja Johannese, ja ütles neile: „Mina ja minu vennad ja minu isakoda oleme võidelnud Iisraeli võitlusi noorpõlvest kuni tänapäevani, ja meie käte läbi on läinud korda, et Iisraeli on sageli päästetud.
\par 3 Aga nüüd olen mina vanaks jäänud ja teie olete taeva armust oma parimas elueas: astuge minu ja minu venna asemele ning minge võitlema meie rahva eest! Taeva abi olgu teiega!”
\par 4 Siis ta valis maalt kakskümmend tuhat sõjameest koos ratsaväega, ja nad läksid Kendebaiose vastu ning ööbisid Moodeinis.
\par 5 Varahommikul asusid nad aga teele ja läksid lagendikule, ja vaata, suur sõjavägi tuli neile vastu, jala- ja ratsavägi. Nende vahele jäi mäestikujõgi.
\par 6 Johannes ja tema rahvas lõid leeri üles nendega kohakuti. Aga kui ta nägi, et rahvas kartis jõest läbi minna, siis läks ta ise esimesena läbi. Kui mehed seda nägid, siis nad läksid temale järele.
\par 7 Siis ta jaotas rahva nõnda, et ratsavägi jäi jalaväe keskele, sest vastase ratsavägi oli väga suur.
\par 8 Nad puhusid pasunaid, Kendebaios ja tema leer aeti põgenema ning paljud neist langesid läbitorgatuna, ülejäänud aga põgenesid kindlusesse.
\par 9 Siis sai haavata Juudas, Johannese vend. Johannes aga ajas neid taga, kuni nad jõudsid Kendebaiose poolt üles ehitatud Kedroonini.
\par 10 Nad põgenesid ka vahitornideni Asdodi väljadel ja Johannes põletas selle linna tulega. Vaenlastest langes ligi kaks tuhat meest. Johannes läks siis rahuga tagasi Juudamaale. 

\section*{Siimoni surm}

\par 11 Ptolemaios, Abubu poeg, oli aga seatud Jeeriko tasandiku pealikuks. Temal oli palju hõbedat ja kulda,
\par 12 sest ta oli ülempreestri väimees.
\par 13 See tegi tema südame ülbeks ja ta tahtis saada maa oma valdusesse. Ta pidas salanõu Siimoni ja tema poegade vastu, et neid kõrvaldada.
\par 14 Siimonil oli kombeks käia läbi maa linnad, hoolitsedes nende vajaduste eest. Nii tuli ta koos oma poegade Mattatiase ja Juudaga alla Jeerikosse aastal sada seitsekümmend seitse, üheteistkümnendal kuul, see on sabatikuus.
\par 15 Aga Abubu poeg võttis nad vastu kavalal kombel väikeses Dooki-nimelises kindluses, mille ta oli ehitanud, ja tegi neile suure võõruspeo, pannes aga sinna ka mehi varitsema.
\par 16 Kui siis Siimon ja tema pojad olid joobnud, tõusid Ptolemaios ja tema mehed, võtsid ära nende sõjariistad ja tungisid peosaali Siimoni juurde ning tapsid tema ja ta kaks poega, nõndasamuti mõned tema teenrid.
\par 17 Nõnda ta näitas üles suurt truudusetust ja tasus head kurjaga.
\par 18 Ptolemaios kirjutas siis sellest ja saatis kirja kuningale, et ta saadaks temale appi sõjaväge ning annaks temale nende maa ja linnad.
\par 19 Ta läkitas mehi Geserisse Johannest oma teelt kõrvaldama. Aga ülempealikuile saatis ta kirju, et need tuleksid üle tema juurde, et ta saaks anda neile hõbedat, kulda ja kingitusi.
\par 20 Ta läkitas ka mehi võtma oma valdusesse Jeruusalemm ja templimägi.
\par 21 Aga keegi mees ruttas ette ja kuulutas Johannesele Geseris, et tema isa ja vennad on tapetud: „Ta on läkitanud kellegi tapma ka sind!”
\par 22 Seda kuuldes kohkus Johannes väga ja laskis kinni võtta need mehed, kes olid tulnud teda tapma, ja hukkas need, sest ta teadis nüüd, et teda taheti tappa.
\par 23 Ja mida veel oleks ütelda Johannesest ja tema sõdadest ja tema kangelastegudest, mida ta mehiselt tegi, ja müüride ehitamisest, mida ta ette võttis, ja muust tegemisest,
\par 24 vaata, see on kirja pandud tema ülempreestriameti ajaraamatus, alates ülempreestriks saamisest oma isa asemel.



\end{document}