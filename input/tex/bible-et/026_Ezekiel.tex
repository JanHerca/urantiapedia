\begin{document}

\title{Hesekiel}

\chapter{1}

\par 1 „Kolmekümnendal aastal, neljanda kuu viiendal päeval, kui ma olin vangide seas Kebari jõe ääres, sündis, et taevad avanesid ja ma nägin Jumala nägemusi.”
\par 2 Kuu viiendal päeval, kuningas Joojakini vangiviimise viiendal aastal
\par 3 tuli Issanda sõna preester Hesekielile, Buusi pojale, kaldealaste maal Kebari jõe ääres; ja Issanda käsi tuli seal tema peale.
\par 4 „Ja ma vaatasin, ja ennäe, põhja poolt tuli marutuul suure pilve ja lõõmava tulega. Ja pilve ümber oli kuma, ja keskelt, tule keskelt, paistis otsekui hiilgav metall.
\par 5 Ja keset seda paistis midagi, mis koosnes neljast olevusest. Nende välimus oli niisugune: nad olid inimese sarnased,
\par 6 aga igaühel oli neli nägu ja igaühel oli neli tiiba.
\par 7 Nende jalad olid sirged ja nende labajalad olid nagu vasika sõrad. Ja nad hiilgasid nagu läikiv vask.
\par 8 Neil olid inimese käed tiibade all, neljas küljes; ja neil neljal olid oma näod ja oma tiivad.
\par 9 Nende tiivad puudutasid üksteist. Liikudes nad ei pöördunud, igaüks läks otse edasi.
\par 10 Ja nende näod olid niisugused: esiküljel inimese nägu, neliku paremal küljel lõvi nägu, neliku vasakul küljel härja nägu ja neliku tagaküljel kotka nägu.
\par 11 Niisugused olid nende näod. Ja nende tiivad olid laiali laotatud ülespoole; igaühel oli kaks tiiba vastastikuseks puudutamiseks ja kaks tiiba katsid nende ihu.
\par 12 Nad liikusid igaüks otse edasi: kuhu vaim iganes tahtis minna, sinna nad läksid, liikudes nad ei pöördunud.
\par 13 Ja olevuste vahel oli näha otsekui tuliseid süsi, mis põlesid tõrvikute sarnaselt, liikudes olevuste vahel sinna-tänna; ja tulel oli kuma ning tulest tuli välja välk.
\par 14 Ja olevused jooksid sinna-tänna, otsekui lööks välku.
\par 15 Siis ma vaatasin olevusi, ja ennäe, olevuste kõrval, igas neljas esiküljes, oli maas üks ratas.
\par 16 Rataste välimus ja nende tegumood: need helkisid nagu krüsoliit ja kõigil neljal oli ühesugune kuju. Need olid välimuselt ja tegumoelt, nagu oleks ratas olnud ratta sees.
\par 17 Liikudes nad läksid neljas eri suunas, liikudes nad ei pöördunud.
\par 18 Ja nende rattapöiad olid kõrged ning kohutavad; ja sel nelikul olid rattapöiad ümberringi täis silmi.
\par 19 Ja kui olevused liikusid, siis liikusid rattad nende kõrval; ja kui olevused tõusid maast üles, siis tõusid üles ka rattad.
\par 20 Kuhu vaim iganes läks, sinna läksid needki, sinna, kuhu vaim tahtis minna. Rattad tõusid üles samuti kui nemad, sest rattais oli olevuste vaim.
\par 21 Kui olevused liikusid, siis liikusid nemadki, ja kui need seisid, siis seisid nemadki; kui olevused tõusid maast üles, siis tõusid selsamal kombel ka rattad, sest rattais oli olevuste vaim.
\par 22 Ja olevuste peade kohal oli midagi taevalaotuse taolist, kohutav nagu jää, mis oli välja laotatud ülal nende peade kohal.
\par 23 Ja laotuse all olid nende tiivad üksteise poole välja sirutatud; igaühel oli kaks tiiba oma keha katmiseks.
\par 24 Kui nad liikusid, siis ma kuulsin nende tiibade kahinat otsekui suurte vete kohinat, otsekui Kõigevägevama häält, müristamise kaja - otsekui sõjaleeri kära. Kui nad seisid, siis nad lasksid oma tiivad longu.
\par 25 Ja hääl kuuldus taevalaotuse pealt, mis oli nende peade kohal; kui nad seisid, siis nad lasksid oma tiivad longu.
\par 26 Ja nende peade kohal oleva taevalaotuse peal oli midagi, mis paistis safiirikividena, midagi aujärje sarnast. Ja selle aujärje sarnase peal, ülal selle peal, oli keegi, kes oli välimuselt inimese sarnane.
\par 27 Ja ma nägin otsekui hiilgavat metalli, pealtnäha nagu tuld, millel oli kuma ümber, ülalpool sellest, mis paistis olevat ta niuded; allpool sellest, mis paistis olevat ta niuded, nägin ma pealtnäha nagu tuld ja sellel oli kuma ümber.
\par 28 Otsekui vikerkaare paiste, mis vihmapäeval on pilvis, oli kuma paiste ümberringi. See oli Issanda auhiilguse ilmutuse paiste! Ja kui ma seda nägin, siis ma langesin silmili maha ja kuulsin häält, mis kõneles.

\chapter{2}

\par 1 Ja ta ütles mulle: „Inimesepoeg, tõuse püsti, ma tahan sinuga rääkida!”
\par 2 Kui ta minuga rääkis, siis tuli mu sisse Vaim ja pani mind jalgadele seisma; ja ma kuulsin teda, kes minuga rääkis.
\par 3 Ta ütles mulle: „Inimesepoeg, mina läkitan sind Iisraeli laste juurde, vastupanija rahva juurde, kes on mulle vastu pannud; nemad ja nende vanemad on mu vastu üles astunud otse tänase päevani.
\par 4 Neil lastel on jultunud näod ja paadunud südamed; ma läkitan sind nende juurde ja sa pead neile ütlema: Nõnda ütleb Issand Jumal;
\par 5 ja kuulaku nad või ärgu kuulaku - nad on ju vastupanija sugu -, nad peavad siiski teadma, et nende keskel on olnud prohvet.
\par 6 Aga sina, inimesepoeg, ära karda neid ja ära karda nende sõnu, kuigi nad on tõrksad ja hülgavad sinu ja sa pead elama skorpionide seas! Ära karda nende sõnu ja ära kohku nende ees - nad on ju vastupanija sugu -,
\par 7 vaid sina räägi neile minu sõnu, kuulaku nad või ärgu kuulaku - nad on ju vastupanijad!
\par 8 Aga sina, inimesepoeg, kuula, mis mina sulle räägin! Ära ole vastupanija nagu see vastupanija sugu! Ava oma suu ja söö, mis mina sulle annan!”
\par 9 Siis ma vaatasin, ja ennäe, mu poole oli sirutatud käsi, ja vaata, selles oli rullraamat.
\par 10 Ta tegi selle minu ees lahti ja see oli mõlemalt poolt täis kirjutatud; sinna oli kirjutatud nutulaule, halinaid ja hädahüüdeid.

\chapter{3}

\par 1 Ja ta ütles mulle: „Inimesepoeg, söö, mis sulle antakse! Söö ära see rullraamat ja mine räägi Iisraeli soole!”
\par 2 Siis ma avasin oma suu ja ta andis mulle süüa selle rullraamatu.
\par 3 Ja ta ütles mulle: „Inimesepoeg, täida oma kõht ja toida oma sisemus selle rullraamatuga, mille ma sulle annan!” Siis ma sõin ja see oli mu suus magus nagu mesi.
\par 4 Ja ta ütles mulle: „Inimesepoeg, tule! Mine Iisraeli soo juurde ja räägi neile minu sõnu!
\par 5 Sest sind ei läkitata mitte umbkeelse või kange keelega rahva juurde, vaid Iisraeli soo juurde,
\par 6 mitte paljude rahvaste juurde, kes on umbkeelsed ja kange keelega, kelle sõnu sa ei mõista. Tõesti, kui ma läkitaksin sind nende juurde, nad kuulaksid sind.
\par 7 Aga Iisraeli sugu ei taha sind kuulata, sest nad ei taha mindki kuulata. Sest kogu Iisraeli sugu on kõva laubaga ja paadunud südamega.
\par 8 Vaata, ma teen su palge karmiks, nagu on nende palged, ja su lauba kõvaks, nagu on nende laubad.
\par 9 Teemandi sarnaseks, ränikivist kõvemaks teen ma su lauba. Ära karda neid ja ära kohku nende ees - nad on ju vastupanija sugu!”
\par 10 Ja ta ütles mulle: „Inimesepoeg, kõik mu sõnad, mis ma sulle räägin, võta südamesse ja kuule neid oma kõrvaga!
\par 11 Ja mine vangide juurde, oma rahva laste juurde, ja räägi nendega ning ütle neile: Nõnda ütleb Issand Jumal; kuulaku nad või ärgu kuulaku!”
\par 12 Siis tõstis Vaim mu üles ja ma kuulsin enda taga suurt müra; „Õnnistatud olgu Issanda auhiilgus oma paigas!”,
\par 13 ja olevuste tiibade kahinat, mis üksteist puudutasid, ja nendega üheaegselt rataste raginat ja suurt mürinat.
\par 14 Ja Vaim tõstis mu üles ja võttis minu, ja ma läksin sügava sisemise erutusega; Issanda käsi oli tugevasti mu peal.
\par 15 Ja ma tulin Tel-Abibi vangide juurde, kes elasid Kebari jõe ääres; ja seal, kus nad elasid, istusin ma vaikides nende keskel seitse päeva.
\par 16 Aga seitsme päeva pärast tuli mulle Issanda sõna; ta ütles:
\par 17 „Inimesepoeg, ma olen sind pannud vahimeheks Iisraeli soole. Ja kui sa kuuled sõna mu suust, siis sa pead neid minu poolt manitsema.
\par 18 Kui ma ütlen õelale: „Sa pead surema!”, kuid sina ei manitse teda ega räägi, et manitseda õelat pöörduma oma õeluse teelt, et hoida teda elus, siis see õel küll sureb oma süü pärast, aga tema verd ma nõuan sinu käest.
\par 19 Aga kui sa õelat manitsed ja tema ei pöördu oma õelusest või oma õeluse teelt, siis ta sureb oma süü pärast, sina aga oled päästnud oma hinge.
\par 20 Ja kui õige pöördub oma õigusest ja teeb ülekohut, siis ma panen ta ette komistuskivi ja ta peab surema; kui sa teda ei ole manitsenud, siis ta peab surema oma patu pärast ja ta õigeid tegusid, mis ta oli teinud, ei peeta meeles; aga tema verd ma nõuan sinu käest.
\par 21 Ent kui sa õiget manitsed, et õige ei teeks pattu, ja tema ei teegi pattu, siis ta jääb tõesti elama, sest ta on lasknud ennast manitseda ja sina oled päästnud oma hinge.”
\par 22 Issanda käsi tuli seal minu peale ja ta ütles mulle: „Tõuse, mine orgu, seal räägin ma sinuga!”
\par 23 Ja ma tõusin ning läksin orgu, ja vaata, seal seisis Issanda auhiilgus, samasugune auhiilgus, mida ma olin näinud Kebari jõe ääres; ja ma langesin silmili.
\par 24 Aga minu sisse tuli Vaim ja tõstis mu jalgadele seisma; ta rääkis minuga ja ütles mulle: „Mine, sule ennast oma kotta!
\par 25 Ja sina, inimesepoeg! Vaata, su peale pannakse köied ja sind seotakse nendega, et sa ei saaks minna välja nende sekka.
\par 26 Ja ma kinnitan su keele su suulae külge, nõnda et sa jääd tummaks ega saa olla neile noomijaks; sest nad on vastupanija sugu.
\par 27 Aga kui mina räägin sinuga, siis ma avan sinu suu, et sa räägiksid nendega: Nõnda ütleb Issand Jumal: Kes tahab kuulata, see kuulaku, kes mitte, see jätku kuulamata, sest nad on vastupanija sugu!

\chapter{4}

\par 1 Ja sina, inimesepoeg, võta enesele telliskivi, pane see enese ette ja uurenda sellesse linn - Jeruusalemm!
\par 2 Pane see piiramisseisukorda, ehita selle vastu piiramisseadmed, kuhja piiramisvall, aseta sõjaleerid ja paiguta müüripurustajad ümberringi!
\par 3 Võta siis enesele raudpann ja pane see raudmüüriks enese ja linna vahele ning pööra oma pale kindlalt selle poole, nõnda et see oleks piiratav ja sina selle piiraja; see olgu Iisraeli soole märgiks!
\par 4 Ja sina heida magama oma vasaku külje peale ja võta enda peale Iisraeli soo süü: niipalju päevi kui sa magad selle külje peal, kannad sa nende süüd!
\par 5 Ja mina määran sulle selleks päevade arvuks nende süüaastad: kolmsada üheksakümmend päeva pead sa kandma Iisraeli soo süüd!
\par 6 Ja kui sa need oled lõpetanud, siis teisel korral pead sa magama parema külje peal ja kandma Juuda soo süüd nelikümmend päeva; ma olen sulle määranud iga aasta kohta ühe päeva.
\par 7 Ja pööra oma pale ja paljastatud käsivars ümberpiiratud Jeruusalemma poole ja ennusta sellele!
\par 8 Ja vaata, ma panen su peale köied, et sa ei saaks pöörduda ühelt küljelt teisele, kuni sa oma piiramispäevad oled lõpetanud.
\par 9 Võta nüüd enesele nisu, otri, ube, läätsi, hirssi ja okasnisu, pane need ühte astjasse ja tee neist enesele leiba, niipalju päevi kui sa magad oma külje peal: kolmsada üheksakümmend päeva pead sa seda sööma!
\par 10 Ja su roog, mida sa päevas sööd, vaagigu kakskümmend seeklit; seda pead sa sööma aeg-ajalt.
\par 11 Ja vett pead sa jooma mõõdu järgi: pool toopi; seda pead sa jooma aeg-ajalt.
\par 12 Sa pead seda sööma nagu odrakooki ja küpsetama inimeste silme ees nende rooja peal.”
\par 13 Ja Issand ütles: „Nõnda peavad Iisraeli lapsed sööma oma roojast leiba paganate keskel, kuhu ma nad pillutan.”
\par 14 Aga mina ütlesin: „Oh Issand Jumal! Vaata, mu hing ei ole iialgi saanud roojaseks; oma noorusest tänini ei ole ma söönud raibet ega mahamurtut, ja mu suhu ei ole saanud roiskunud liha.”
\par 15 Siis ta ütles mulle: „Vaata, ma luban sulle inimrooja asemel veisesõnnikut; valmista siis oma leiba selle peal!”
\par 16 Ja ta ütles mulle: „Inimesepoeg, vaata, ma murran katki leivatoe Jeruusalemmas; siis nad saavad leiba süüa kaalu järgi ja murega ning vett juua mõõdu järgi ja hirmuga,
\par 17 sellepärast et neil on puudus leivast ja veest - et nad hakkaksid üheskoos lõdisema ja kõduneksid oma süütegude sees.

\chapter{5}

\par 1 Ja sina, inimesepoeg, võta enesele terav mõõk; tarvita seda kui habemenuga ja lase see käia üle oma pea ning habeme; siis võta enesele vaekausid ja jaota karvad:
\par 2 kolmandik põleta tules keset linna, siis kui piiramispäevad on lõppenud; kolmandik võta ja raiu mõõgaga ümber linna; kolmandik puista tuulde, ja mina tõmban mõõga välja nende taga;
\par 3 võta neid vähesel määral ja seo oma kuuehõlma;
\par 4 võta neist veel osa ja viska tulle ning põleta tules; siit läheb tuli kogu Iisraeli soo kallale.
\par 5 Nõnda ütleb Issand Jumal: See on Jeruusalemm; ma olen pannud selle paganate keskele ja selle ümber on maad.
\par 6 Aga see on vastu pannud mu seadustele õelamalt kui paganad ja mu määrustele rohkem kui maad, mis on selle ümber; sest nad on põlanud mu seadusi ega ole käinud mu määruste järgi.
\par 7 Seepärast ütleb Issand Jumal nõnda: Et te olete rohkem vastu pannud kui teie ümberkaudsed paganad ega ole käinud mu määruste järgi, ei ole teinud mu seaduste järgi ega ole teinud ka oma ümberkaudsete paganate seaduste järgi,
\par 8 seepärast ütleb Issand Jumal nõnda: Vaata, ka mina olen sinu vastu ja mõistan kohut su keskel paganate silme all
\par 9 ja ma talitan sinuga kõigi su jäleduste pärast, nagu ma enne ei ole talitanud ja milletaoliselt ma enam ei talitagi.
\par 10 Sellepärast peavad sinu keskel isad sööma oma lapsi ja lapsed sööma oma isasid; ja ma mõistan kohut sinu üle ning puistan kogu su jäägi kõigi tuulte poole.
\par 11 Sellepärast, nii tõesti kui ma elan, ütleb Issand Jumal, tõesti, sellepärast et sa oled rüvetanud mu pühamu kõigi oma põlastusväärsustega ja kõigi oma jäledustega, siis lõikan minagi: mu silm ei kurvasta ja ma ei anna armu.
\par 12 Kolmandik sinust sureb katku ja lõpeb su keskel nälga, kolmandik langeb su ümber mõõga läbi ja kolmandiku puistan ma kõigi tuulte poole ning tõmban mõõga välja nende taga.
\par 13 Mu viha mõõt saab täis ja ma vaigistan oma raevu nende kallal ning maksan kätte. Ja nad saavad tunda, et mina, Issand, olen rääkinud oma pühas vihas, kui ma olen tühjendanud oma raevu nende peale.
\par 14 Ja ma teen sind varemeks ning teotuseks su ümberkaudsete paganate seas, iga möödamineja silma ees.
\par 15 Ja sa saad teotuseks ja sõimuks, hoiatuseks ja ehmatuseks su ümberkaudseile paganaile, kui ma mõistan kohut su üle vihas ja raevus ning vihaste manitsustega - mina, Issand, olen rääkinud -,
\par 16 kui ma läkitan nende kallale nälja kurjad nooled, mis ma läkitan teid hävitama ja mis saavad hukatuseks; ja ma suurendan veelgi teie nälga ning murran katki teie leivatoe.
\par 17 Ja ma läkitan teie kallale nälja ja kurjad metsloomad, kes jätavad sind lasteta; katk ja veri käivad sinust üle ja ma toon mõõga su kallale. Mina, Issand, olen rääkinud.”

\chapter{6}

\par 1 Ja mulle tuli Issanda sõna; ta ütles:
\par 2 „Inimesepoeg, pööra oma pale Iisraeli mägede poole ja kuuluta neile prohvetlikult
\par 3 ning ütle: Iisraeli mäed, kuulge Issanda Jumala sõna! Nõnda ütleb Issand Jumal mägedele ja küngastele, jäärakutele ja orgudele: Vaata, ma toon teie kallale mõõga ja hävitan teie ohvrikünkad.
\par 4 Teie altarid rüüstatakse ja teie suitsutusaltarid purustatakse; ja ma lasen teie hulgast mahalööduil langeda teie ebajumalate ette.
\par 5 Ma panen Iisraeli laste laibad nende ebajumalate ette ja puistan teie luud teie altarite ümber.
\par 6 Kõigis teie asupaigus muutuvad linnad varemeiks ja ohvrikünkad laastatuks; nõnda jäävad teie altarid varemeiks ja rüüstatuks, teie ebajumalad purustatuks ja hävitatuks, teie suitsutusaltarid lammutatuks ja teie tööd ärapühituks.
\par 7 Mahalöödud langevad teie keskel, ja siis te tunnete, et mina olen Issand!
\par 8 Aga ma jätan järele jäägi: teil on mõõga eest pääsenuid paganate seas, kui teid maid mööda pillutatakse.
\par 9 Ja need teie hulgast, kes pääsevad, mõtlevad minu peale paganate seas, kuhu nad on vangi viidud, siis kui ma olen purustanud nende truuduseta südamed, mis minu juurest on lahkunud, ja nende silmad, mis truuduseta on järginud nende ebajumalaid; ja nad tülgastuvad oma kuritegude pärast, mis nad on teinud, kõigi oma jäleduste pärast.
\par 10 Ja nad saavad tunda, et mina olen Issand; ma ei ole ilmaaegu rääkinud, et ma valmistan neile selle õnnetuse.
\par 11 Nõnda ütleb Issand Jumal: Löö käsi kokku ja trambi jalgadega ning ütle: „Oh häda!” kõigi Iisraeli soo vastikute jäleduste pärast. Nad langevad mõõga, nälja ja katku läbi.
\par 12 Kes on kaugel, sureb katku, ja kes on ligidal, langeb mõõga läbi, aga järelejäänu ja hoitu sureb nälga; nõnda ma valan nende peale oma tulise viha.
\par 13 Ja te saate tunda, et mina olen Issand, kui teie hulgast mahalöödud on teie ebajumalate keskel ümber teie altarite igal kõrgel künkal, kõigil mäetippudel ning iga halja puu ja iga oksliku tamme all paigus, kus nad on teinud uimastuslõhna kõigile oma ebajumalaile.
\par 14 Ja ma sirutan oma käe välja nende vastu ning teen maa lagedaks ja laastatuks kõrbest kuni Riblani kõigis nende asupaigus; ja nad saavad tunda, et mina olen Issand.”

\chapter{7}

\par 1 Ja mulle tuli Issanda sõna; ta ütles:
\par 2 „Ja sina, inimesepoeg, nõnda ütleb Issand Jumal Iisraeli maale: Lõpp! Üle nelja maapiiri tuleb lõpp!
\par 3 Nüüd tuleb sulle lõpp, sest ma läkitan oma viha su peale ning mõistan kohut su üle su eluviiside kohaselt ja panen su peale kõik su jäledused!
\par 4 Mu silm ei kurvasta sinu pärast ja ma ei anna armu, vaid panen su eluviisid su peale ja su jäledused tulevad su keskele; ja te saate tunda, et mina olen Issand.
\par 5 Nõnda ütleb Issand Jumal: Õnnetus! Üksnes õnnetus! Vaata, see tuleb!
\par 6 Lõpp tuleb, tuleb lõpp! See saabub sulle! Vaata, see tuleb!
\par 7 Järg tuleb sinu kätte, maa elanik! Aeg tuleb, päev on ligidal: jahmatuseks, aga mitte rõõmuhõikeiks mägedel.
\par 8 Nüüd varsti ma valan su peale oma tulise viha ja lasen sul tunda oma viha; ma mõistan kohut su üle su eluviiside kohaselt ja panen su peale kõik su jäledused.
\par 9 Mu silm ei kurvasta ja ma ei anna armu; su eluviiside kohaselt tasun ma sulle ja su jäledused tulevad su keskele; ja te saate tunda, et mina, Issand, olen see, kes teid lööb.
\par 10 Vaata, päev! Vaata, see tuleb! Järg on tulnud, väärus õitseb, ülbus lokkab!
\par 11 Vägivald tõuseb õeluse vitsaks; neist ei jää midagi üle, ei nende rikkusest ega nende kärast ega nende ilust.
\par 12 Aeg tuleb, päev ligineb! Ostja ärgu rõõmustagu ja müüja ärgu kurvastagu, sest tuline viha tuleb kogu selle rahvahulga peale!
\par 13 Jah, müüja ei pääse tagasi müüdu juurde, kui nad ongi veel elus elavate keskel; sest nägemus kogu selle rahvahulga kohta ei pöördu ära ja oma patusse jäädes ei säilita ükski oma elu.
\par 14 Puhutakse küll sarve ja seatakse kõik valmis, aga ükski ei lähe sõtta, sest mu tuline viha on kogu selle rahvahulga vastu.
\par 15 Väljas on mõõk ning sees on katk ja nälg; kes on väljal, see sureb mõõga läbi, ja kes on linnas, selle sööb nälg ja katk.
\par 16 Ja kui neist mõned pääsevad, siis on need mägedel otsekui kudrutavad kuristike tuvid, igaüks nutab oma süü pärast.
\par 17 Kõik käed lõtvuvad ja kõik põlved muutuvad nõrgaks.
\par 18 Nad rõivastuvad kotiriidesse ja neid katab värin; kõigi nägudel on häbi ja kõigi pead on paljaks pöetud.
\par 19 Oma hõbeda nad viskavad tänavaile ja nende kuld on roojane; nende hõbe ja kuld ei päästa neid Issanda vihapäeval, nad ei saa sellega toita oma hinge ega täita oma kõhtu; sest see on saanud neile komistuskiviks süüsse.
\par 20 Nad on tarvitanud oma kallisasju uhkuseks ja on neist valmistanud oma jäledad, põlastusväärsed kujud; seepärast ma teen selle kõik neile roojaseks.
\par 21 Ma annan selle riisumiseks võõrastele ja saagiks maa õelaile, ja need teotavad seda.
\par 22 Ma pööran neist ära oma palge ja mu kallisvara teotatakse; selle kallale tulevad röövlid ja need teotavad seda.
\par 23 Valmista ahelad, sest maa on täis veresüüd ja linn on täis vägivalda!
\par 24 Ja ma toon kõige halvemad paganaist ja need pärivad nende kojad; ma lõpetan nende vägevate kõrkuse ja nende pühamud teotatakse.
\par 25 Ahastus tuleb ja nad otsivad rahu, aga seda ei ole.
\par 26 Õnnetus tuleb õnnetuse peale, kuulujutt kuulujutu järele; siis nad nõuavad prohvetilt nägemust, aga Seadus on kadunud preestreilt ja nõu vanemailt.
\par 27 Kuningas leinab, vürst rüütab enese ehmatusega ja maa rahva käed värisevad; ma kohtlen neid nende eluviiside kohaselt ja mõistan kohut nende üle nende oma seaduste järgi ning nad saavad tunda, et mina olen Issand.”

\chapter{8}

\par 1 Ja kuuendal aastal, kuuenda kuu viiendal päeval sündis, et ma istusin oma kojas ja Juuda vanemad istusid mu ees; siis langes seal Issanda Jumala käsi mu peale.
\par 2 Ja ma vaatasin, ja ennäe, see oli mingi tule sarnane ilmutis: allpool sellest, mis paistis olevat ta niuded, oli tuli, ja ta niudeist ülalpool paistis nagu sära, otsekui mingi metalli hiilgus.
\par 3 Ja ta sirutas midagi käe sarnast ning võttis kinni mu juuksetukast; ja Vaim tõstis mu maa ja taeva vahele ning viis mu Jumala nägemustes Jeruusalemma, sisemise värava suhu, mis on põhja pool, sinna, kus oli selle kuju asukoht, mis põhjustas Issanda püha viha.
\par 4 Ja vaata, seal oli Iisraeli Jumala auhiilgus selle nägemuse sarnaselt, mida ma orus olin näinud.
\par 5 Ja ta ütles mulle: „Inimesepoeg, tõsta nüüd oma silmad üles põhja poole!” Siis ma tõstsin oma silmad põhja poole, ja vaata, põhja pool, altari väravas, sissekäigu juures oli see püha viha põhjustav kuju.
\par 6 Ja ta ütles mulle: „Inimesepoeg, kas sa näed, mis nad teevad, neid suuri häbitegusid, mis Iisraeli sugu siin teeb, et ma peaksin minema kaugele oma pühamust? Aga sa saad näha veel suuremaid jäledusi.”
\par 7 Siis ta viis mind õueukse juurde; ja ma vaatasin, ja ennäe, seinas oli auk.
\par 8 Ja ta ütles mulle: „Inimesepoeg, murra nüüd läbi seina!” Ja ma murdsin seinast läbi, ja vaata, seal oli uks.
\par 9 Ja ta ütles mulle: „Mine sisse ja vaata neid nurjatuid häbitegusid, mida nad siin teevad!”
\par 10 Siis ma läksin sisse ja vaatasin, ja ennäe, kõiksugu pilte jäledaist roomajaist ja loomadest ning kõiksugu Iisraeli soo ebajumalaid oli joonistatud seina peale ümberringi.
\par 11 Ja nende piltide ees seisis seitsekümmend meest Iisraeli soo vanemaist, ja Jaasanja, Saafani poeg, seisis nende keskel; igaühel oli käes oma suitsutuspann ja lõhnav suitsupilv tõusis üles.
\par 12 Siis ta ütles mulle: „Kas sa näed, inimesepoeg, mida Iisraeli soo vanemad pimedas teevad, igaüks oma kuju kambris? Sest nad ütlevad: Issand ei näe meid, Issand on maa maha jätnud.”
\par 13 Ja ta ütles mulle: „Sa saad näha veelgi suuremaid häbitegusid, mida nad teevad.”
\par 14 Siis ta viis mind Issanda koja värava suhu, mis on põhja pool, ja vaata, seal istusid naised, nuttes taga Tammust.
\par 15 Ja ta ütles mulle: „Kas sa näed, inimesepoeg? Sa saad näha veelgi suuremaid häbitegusid kui need.”
\par 16 Siis ta viis mind Issanda koja sisemisse õue, ja vaata, Issanda templi ukse juures, eeskoja ja altari vahel, oli umbes kakskümmend viis meest; neil olid seljad Issanda templi poole ja näod ida poole, ja nad kummardasid päikest ida pool.
\par 17 Ja ta ütles mulle: „Kas sa näed, inimesepoeg? Kas on Juuda sool veel vähe neid häbitegusid, mida nad siin on teinud, et nad täidavad maa vägivallaga ja ärritavad mind veel rohkem? Ja vaata, nad pistavad viinapuuvääte endale ninasse!
\par 18 Seepärast talitan minagi tulises vihas: mu silm ei kurvasta ja ma ei anna armu; ja kuigi nad hüüavad mu kõrvu suure häälega, ei kuule ma neid.”

\chapter{9}

\par 1 Siis ta hüüdis mu kuuldes valju häälega, öeldes: „Tulge siia, linna nuhtlejad, ja igaühel olgu käes hävitusriist!”
\par 2 Ja vaata, kuus meest tuli Ülavärava poolt, mis on põhja poole, ja igaühel oli käes oma purustusriist; aga nende seas oli mees, linased riided seljas ja kirjutustarbed puusal; ja nad tulid ning asusid vaskaltari kõrvale.
\par 3 Ja Iisraeli Jumala auhiilgus tõusis keerubi pealt, mille peal ta oli, koja lävele ja hüüdis linaste riietega meest, kellel olid kirjutustarbed puusal.
\par 4 Ja Issand ütles temale: „Mine läbi linna, läbi Jeruusalemma, ja tee märk nende laubale, kes ohkavad ja ägavad jäleduste pärast, mida selles linnas tehakse!”
\par 5 Ja neile teistele ütles ta minu kuuldes: „Minge tema järel läbi linna ja lööge! Teie silm ärgu kurvastagu ja ärge andke armu!
\par 6 Tapke sootuks vanad, noored mehed ja neitsid, lapsed ja naised, aga ärge puudutage ühtegi, kellel on märk küljes! Ja alustage minu pühamust!” Ja nad alustasid meestest, vanemaist, kes olid koja ees.
\par 7 Ja ta ütles neile: „Rüvetage koda ja täitke õued mahalöödutega! Minge!” Ja nad läksid välja ning alustasid linnas tapatööd.
\par 8 Aga kui nad olid maha löömas ja mina olin üksi jäänud, siis ma langesin silmili maha ja kisendasin ning ütlesin: „Oh Issand Jumal! Kas sa tahad hävitada kogu Iisraeli jäägi, et sa valad oma tulise viha Jeruusalemma peale?”
\par 9 Siis ta ütles mulle: „Iisraeli ja Juuda soo süü on väga suur; maa on täis veresüüd ja linn on täis õiguseväänamist, sest nad ütlevad: Issand on maa maha jätnud, Issand ei näe!
\par 10 Sellepärast minugi silm ei kurvasta ja ma ei anna armu. Ma panen nende eluviisid neile pea peale.”
\par 11 Ja vaata, linaste riietega mees, kellel olid kirjutustarbed puusal, tõi sõna, öeldes: „Ma tegin, nagu sa mind käskisid.”

\chapter{10}

\par 1 Siis ma vaatasin, ja ennäe, taevalaotuses, mis oli keerubite pea kohal, olid nagu safiirikivid; midagi, mis paistis aujärjena, nähti nende kohal.
\par 2 Ja ta rääkis mehega, kellel olid linased riided seljas, ja ütles: „Mine rataste vahele, mis on iga keerubi all, ja täida pihud tuliste sütega keerubite vahelt ning pillu need linna peale!” Ja mees läks sisse minu nähes.
\par 3 Keerubid seisid kojas paremal pool, kui mees läks, ja pilv täitis sisemise õue.
\par 4 Siis tõusis Issanda auhiilgus keerubi kohalt koja lävele, koda täitus pilvest ja õu täitus Issanda auhiilguse särast.
\par 5 Ja keerubite tiibade kahinat kuuldus kuni välimise õueni nagu Kõigeväelise Jumala häält, kui ta kõneleb.
\par 6 Ja kui ta käskis meest, kellel olid linased riided seljas, öeldes: „Võta tuld rataste vahelt, keerubite vahelt!”, siis läks see ja asetus ratta kõrvale.
\par 7 Ja üks keerub pistis oma käe keerubite vahelt tule juurde, mis oli keerubite vahel, ja võttis seda ning andis linasesse riietatu pihkudesse; ja see võttis ning läks välja.
\par 8 Keerubitel paistis tiibade all olevat otsekui inimese käsi.
\par 9 Ja ma vaatasin, ja ennäe, keerubite kõrval oli neli ratast, iga keerubi kõrval oli ratas, ja rattad olid välimuselt nagu krüsoliidikivid.
\par 10 Ja neil neljal oli ühesugune välimus, otsekui oleks ratas olnud ratta sees.
\par 11 Kui nad liikusid, siis nad said minna iga nelja külje suunas, käigul pöördumata; sest sinna, kuhu oli pööratud esikülg, läksid nad selle järel; käigul nad ei pöördunud.
\par 12 Ja kogu nende keha, seljad, käed, tiivad ja rattad olid ümberringi täis silmi; sel nelikul olid rattad
\par 13 ja rattaid nimetati, nagu ma kuulsin, „ratastikuks”.
\par 14 Ja igaühel oli neli nägu: üks oli keerubi nägu, teine inimese nägu, kolmas lõvi nägu ja neljas kotka nägu.
\par 15 Ja keerubid tõusid üles, need olid samad olevused, keda ma olin näinud Kebari jõe ääres.
\par 16 Ja kui keerubid liikusid, siis liikusid rattad nende kõrval; ja kui keerubid tõstsid tiibu, et maa pealt üles tõusta, siis ei pöördunud ka rattad ära nende kõrvalt.
\par 17 Kui ühed seisid, siis seisid ka teised, ja kui ühed tõusid üles, siis tõusid teised koos nendega, sest nende sees oli olevuse vaim.
\par 18 Siis Issanda auhiilgus läks ära koja lävelt ja seisis keerubite kohal.
\par 19 Ja minnes tõstsid keerubid oma tiibu ning tõusid maast üles mu silme all, ja nende rattad samuti nagu nad isegi; nad jäid seisma Issanda koja idapoolse värava suus ja Iisraeli Jumala auhiilgus oli ülal nende kohal.
\par 20 Need olid needsamad olevused, keda ma olin näinud Iisraeli Jumala all Kebari jõe ääres; ja ma mõistsin, et need olid keerubid.
\par 21 Igaühel oli neli nägu ja igaühel oli neli tiiba, ja neil olid otsekui inimese käed tiibade all.
\par 22 Ja nende nägude kuju: need olid näod, mida ma olin näinud Kebari jõe ääres, nende välimus ja need ise; igaüks läks edasi omaette.

\chapter{11}

\par 1 Siis Vaim tõstis minu üles ja viis mu Issanda koja Idaväravasse, mis on ida poole; ja vaata, värava suus oli kakskümmend viis meest ja ma nägin nende keskel Jaasanjat, Assuri poega, ja Pelatjat, Benaja poega, rahva ülemaid.
\par 2 Ja ta ütles mulle: „Inimesepoeg, need on mehed, kes kavatsevad nurjatust ja peavad kurja nõu selle linna vastu;
\par 3 nad ütlevad: „Aeg ei ole käes, et ehitada kodasid. See linn on pott ja meie oleme liha.”
\par 4 Seepärast kuuluta neile prohvetlikult, kuuluta prohvetlikult, inimesepoeg!”
\par 5 Siis langes mu peale Issanda Vaim ja ütles mulle: „Räägi: Nõnda ütleb Issand: Nii te ütlete, Iisraeli sugu, ja mis teil mõttes on, seda ma tean!
\par 6 Palju on teie poolt mahalööduid selles linnas ja te olete mahalöödutega täitnud selle tänavad.
\par 7 Seepärast ütleb Issand Jumal nõnda: Teie mahalöödud, keda te olete pannud selle keskele, on liha, ja see linn on pott, aga ma viin teid sellest välja.
\par 8 Mõõka te kardate, aga mina toon mõõga teie kallale, ütleb Issand Jumal.
\par 9 Ma viin teid välja selle keskelt ja annan teid võõraste kätte; otsused teie kohta viin ma täide.
\par 10 Te langete mõõga läbi, Iisraeli piiril mõistan ma kohut teie üle ja te saate tunda, et mina olen Issand.
\par 11 See linn ei ole teile potiks ja teie ei ole lihaks selle sees: Iisraeli piiril mõistan ma kohut teie üle.
\par 12 Ja te saate tunda, et mina olen Issand, sest te ei ole käinud mu määruste järgi ega ole teinud mu seaduste järgi, vaid olete teinud nende paganate seaduste järgi, kes asuvad teil ümberkaudu.”
\par 13 Aga kui ma prohvetlikult kuulutasin, suri Pelatja, Benaja poeg; siis ma langesin silmili ja kisendasin suure häälega ning ütlesin: „Oh Issand Jumal! Kas sa teed lõpu Iisraeli jäägile?”
\par 14 Ja mulle tuli Issanda sõna; ta ütles:
\par 15 „Inimesepoeg, su vennad, su vennad, sugulased ja kogu Iisraeli sugu üheskoos on need, kelle kohta Jeruusalemma elanikud ütlevad: „Nad on Issandast kaugel, maa on antud omandiks meile!”
\par 16 Seepärast ütle: Nõnda ütleb Issand Jumal: Kuigi ma olen nad viinud kaugele paganate sekka ja kuigi ma olen nad pillutanud mööda maid, olen ma siiski pisut olnud neile pühamuks maades, kuhu nad on sattunud.
\par 17 Seepärast ütle: Nõnda ütleb Issand Jumal: Mina kogun teid rahvaste seast ja korjan teid maadest, kuhu teid on pillutatud, ja ma annan teile Iisraeli maa.
\par 18 Siis nad tulevad sinna ja kõrvaldavad sealt kõik selle põlastusväärsused ja kõik selle jäledused.
\par 19 Mina annan neile ühesuguse südame ja annan nende sisse uue vaimu: ma kõrvaldan nende ihust kivise südame ja annan neile lihase südame,
\par 20 et nad käiksid mu määruste järgi ning peaksid mu seadusi ja täidaksid neid; siis on nad mulle rahvaks ja mina olen neile Jumalaks.
\par 21 Aga kelle süda käib nende põlastusväärsuste ja nende jäleduste meele järgi, nende eluviisid panen ma nende oma pea peale, ütleb Issand Jumal.”
\par 22 Siis tõstsid keerubid oma tiivad ja üheaegselt nendega tõusid rattad ning ülal nende kohal oli Iisraeli Jumala auhiilgus.
\par 23 Ja Issanda auhiilgus tõusis üles linna keskelt ning jäi seisma mäele, mis on ida pool linna.
\par 24 Aga Vaim tõstis mind üles ja viis mind Kaldeasse vangide juurde nägemuses, Jumala Vaimus; siis kadus mul nägemus, mida ma olin näinud.
\par 25 Ja ma jutustasin vangidele kõigist Issanda sõnadest, mis ta mulle oli ilmutanud.

\chapter{12}

\par 1 Ja mulle tuli Issanda sõna; ta ütles:
\par 2 „Inimesepoeg, sa elad vastupanija soo hulgas, kellel on silmad nägemiseks, aga nad ei näe, kõrvad kuulmiseks, aga nad ei kuule, sest see on vastupanija sugu.
\par 3 Aga sina, inimesepoeg, valmista enesele teekonnavarustus ja mine teele päevaajal nende nähes; mine nende nähes oma asupaigast teise paika; vahest nad märkavad, kuigi nad on vastupanija sugu!
\par 4 Vii välja päeva ajal nende nähes oma asjad, nagu teekonnavarustus; ise aga mine õhtul välja nende nähes, nagu vangiminejad lähevad!
\par 5 Murra nende nähes müürist läbi ja vii asjad sealtkaudu välja!
\par 6 Tõsta need nende nähes õlale, vii pimedas välja; kata nägu, et sa ei näeks maad, sest ma olen pannud sind endeks Iisraeli soole!”
\par 7 Ja ma tegin, nagu mind kästi: ma viisin päeval välja asjad, nagu teekonnavarustuse; õhtul aga murdsin ma oma käega müürist läbi, pimeduses viisin need välja, tõstsin nende nähes õlale.
\par 8 Aga hommikul tuli mulle Issanda sõna; ta ütles:
\par 9 „Inimesepoeg, kas pole Iisraeli sugu, see vastupanija sugu, sinult küsinud: „Mis sa teed?”
\par 10 Vasta neile: Nõnda ütleb Issand Jumal: See on ennustus Jeruusalemma vürsti ja kogu Iisraeli soo kohta, kes teie keskel on.
\par 11 Ütle: Mina olen teile endeks. Nõnda nagu mina tegin, nõnda tehakse nendega - nad lähevad vangiteekonnale.
\par 12 Ja vürst, kes on nende keskel, peab pimedas kandami õlale tõstma ja välja minema - müürist murtakse läbi, et asju sealtkaudu välja viia -; ta peab katma oma näo, et ta ei näeks oma silmaga maad.
\par 13 Mina laotan oma võrgu tema peale ja ta püütakse mu püünisesse; ma viin ta Paabelisse, kaldealaste maale, aga ta ei saa seda näha ja seal ta sureb.
\par 14 Ja kõik, kes ümberkaudu on temale abiks, ja kõik ta väesalgad hajutan ma kõigi tuulte poole ja tõmban mõõga nende taga.
\par 15 Ja nad saavad tunda, et mina olen Issand, kui ma neid olen pillutanud paganate sekka ja puistanud mööda maid.
\par 16 Aga ma jätan neist pisut inimesi mõõga, nälja ja katku käest alles, et nad jutustaksid paganate seas, kuhu nad tulevad, kõigist oma häbitegudest; ja nad saavad tunda, et mina olen Issand.”
\par 17 Ja mulle tuli Issanda sõna; ta ütles:
\par 18 „Inimesepoeg, söö oma leiba vabisedes ja joo oma vett värisedes ja murega
\par 19 ning ütle maa rahvale: Nõnda ütleb Issand Jumal Jeruusalemma elanike kohta Iisraeli maal: Nad peavad sööma oma leiba murega ja jooma oma vett suure hirmuga, sellepärast et nende maa laastatakse kõigest, mis seal on, kõigi seal elavate vägivalla pärast.
\par 20 Asustatud linnad muutuvad varemeiks ja maa jääb lagedaks; ja te saate tunda, et mina olen Issand.”
\par 21 Ja mulle tuli Issanda sõna; ta ütles:
\par 22 „Inimesepoeg, mis kõneviis see teil on Iisraeli maal, et te ütlete: „Aeg venib pikale ja kõik nägemused lähevad tühja”?
\par 23 Seepärast ütle neile: Nõnda ütleb Issand Jumal: Ma lõpetan selle kõneviisi ja nõnda ei kõnelda enam Iisraelis. Ja ütle neile: Ligidal on aeg ja kõigi nägemuste täitumine.
\par 24 Sest edaspidi ei ole enam ühtegi valenägemust ega kahtlast ennustust Iisraeli soo keskel.
\par 25 Sest mina, Issand, räägin: See sõna, mis ma ütlen, läheb täide ega viibi enam; sest teie päevil, sa vastupanija sugu, ütlen ma sõna ja viin selle täide, ütleb Issand Jumal.”
\par 26 Ja mulle tuli Issanda sõna; ta ütles:
\par 27 „Inimesepoeg, vaata, Iisraeli sugu ütleb: „Nägemus, mida see näeb, täitub ehk hulga aja pärast, ja ta kuulutab prohvetlikult kaugetest aegadest.”
\par 28 Seepärast ütle neile: Nõnda ütleb Issand Jumal: Ei viibi enam ükski mu sõna. sõna, mis ma ütlen, läheb täide, ütleb Issand Jumal.”

\chapter{13}

\par 1 Ja mulle tuli Issanda sõna; ta ütles:
\par 2 „Inimesepoeg, kuuluta prohvetlikult Iisraeli kuulutavaile prohveteile ja ütle neile, kes enda arvates on prohvetid: Kuulge Issanda sõna!
\par 3 Nõnda ütleb Issand Jumal: Häda jõledaile prohveteile, kes käivad omaenese vaimu järgi ega ole midagi näinud!
\par 4 Nagu rebased varemete vahel on su prohvetid, Iisrael.
\par 5 Te ei ole läinud pragude ette ega ole teinud müüri Iisraeli soo ümber, et see püsiks võitluses Issanda päeval.
\par 6 Nad on näinud tühja ja nende ennustused, kes ütlevad: „See on Issanda sõna!”, on valed, sest Issand ei ole neid läkitanud; ometi nad ootavad, et sõna täide läheks!
\par 7 Eks te ole näinud tühja nägemust ja ennustanud valet, kui olete öelnud: „See on Issanda sõna!”, kuigi mina ei ole rääkinud?
\par 8 Seepärast ütleb Issand Jumal nõnda: Et te räägite tühja ja näete valet, vaata, siis ma tulen teile kallale, ütleb Issand Jumal.
\par 9 Minu käsi on prohvetite vastu, kes näevad tühja ja ennustavad valet; nad ei tohi olla minu rahva osaduses, neid ei tohi kirjutada Iisraeli soo kirja ja nad ei tohi tulla Iisraeli maale. Ja te saate tunda, et mina olen Issand Jumal.
\par 10 Sellepärast, jah sellepärast, et nad eksitavad mu rahvast, öeldes: „On rahu”, kuigi rahu ei ole. Ja kui rahvas ehitab seina, vaata, siis nad võõpavad seda lubjaga.
\par 11 Ütle lubjaga võõpajaile, et see variseb! Tuleb uputav sadu, teie, raheterad, langete, ja sina, marutuul, puhked.
\par 12 Jah, vaata, sein variseb! Eks siis küsita teilt: „Kus on võõp, millega te võõpasite?”
\par 13 Seepärast ütleb Issand Jumal nõnda: Ma lasen oma tulises vihas puhkeda marutuule, mu vihas tuleb uputav sadu ja mu raevus tulevad raheterad, et teha lõpp.
\par 14 Ma lõhun ära seina, mille te olete lubjaga võõbanud, ja paiskan selle maha, nõnda et alusmüür paljastub; kui see langeb, saate te sealjuures otsa. Ja te saate tunda, et mina olen Issand.
\par 15 Ma vaigistan oma viha seina kallal ja nende kallal, kes on seda lubjaga võõbanud, ja ütlen teile: Ei ole enam seina ega selle võõpajaid,
\par 16 Iisraeli prohveteid, kes kuulutavad prohvetlikult Jeruusalemmale ja näevad sellele nägemust rahust, kuigi rahu ei ole, ütleb Issand Jumal.
\par 17 Aga sina, inimesepoeg, pööra pale oma rahva tütarde poole, kes enda arvates kuulutavad prohvetlikult, ja kuuluta neile
\par 18 ning ütle: Nõnda ütleb Issand Jumal: Häda neile, kes õmblevad sidemeid kõigile randmeile ja valmistavad peakatteid nii suurtele kui väikestele, et püüda hingi! Kas tahate püüda mu rahva hingi ja iseenese jaoks jätta hinged elama?
\par 19 Te teotate mind mu rahva ees peotäie otrade ja leivapalukeste eest, surmates hinged, kes ei pea surema, ja jättes elama hinged, kes ei pea jääma ellu, valetades mu rahvale, kes kuulab meeleldi valet.
\par 20 Seepärast ütleb Issand Jumal nõnda: Vaata, ma tulen teie sidemete kallale, millega te seal püüate hingi nagu linde, ja rebin need teie käsivartelt ning päästan hinged lahti, hinged, keda te olete püüdnud nagu linde.


\end{document}