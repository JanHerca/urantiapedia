\begin{document}

\title{Markuse evangeelium}

\chapter{1}

\section*{Ristija Johannese jutlus}

\par 1 Jeesuse Kristuse, Jumala Poja evangeeliumi algus.
\par 2 Nõnda nagu on kirjutatud prohvet Jesaja raamatus: „Vaata, ma läkitan su palge eele oma ingli, kes valmistab su teed;
\par 3 hüüdja hääl on kõrbes: valmistage Issanda teed, õgvendage tema teerajad!”,
\par 4 nõnda oli Ristija Johannes kõrbes kuulutamas meeleparanduse ristimist pattude andekssaamiseks.
\par 5 Ja tema juurde rändasid kogu Juudamaa ja kõik Jeruusalemma elanikud ja said temalt ristimise Jordani jões, tunnistades oma patud.
\par 6 Ja Johannes oli riietatud kaameli karvusse ja tal oli nahkvöö vööl, ja ta sõi rohutirtse ja metsmett.
\par 7 Ja ta kuulutas ning ütles: „Pärast mind tuleb see, kes on vägevam minust, kellele ma ei kõlba kummardades lahti päästma tema jalatsite paelu.
\par 8 Mina olen teid veega ristinud; aga tema ristib teid Püha Vaimuga!”

\section*{Jeesuse ristimine}

\par 9 Neil päevil sündis, et Jeesus tuli Galilea Naatsaretist ja Johannes ristis teda Jordanis.
\par 10 Ja sedamaid, kui ta veest välja tuli, nägi ta taeva avanevat ja Vaimu nagu tuvi laskuvat tema peale.
\par 11 Ja hääl kostis taevast: „Sina oled mu armas Poeg, kellest mul on hea meel!”

\section*{Jeesust kiusatakse kõrbes}

\par 12 Ja kohe ajas Püha Vaim tema kõrbe.
\par 13 Ja ta oli kõrbes nelikümmend päeva saatana kiusata; ja ta oli metselajate seas, ja inglid teenisid teda.
\section*{Jeesus hakkab kuulutama evangeeliumi}

\par 14 Aga pärast seda, kui Johannes oli pandud vangi, tuli Jeesus Galileasse ja kuulutas Jumala evangeeliumi
\par 15 ja ütles: „Aeg on täis saanud ja Jumala riik on lähedal; parandage meelt ja uskuge evangeeliumisse!”

\section*{Jeesus kutsub oma esimesed jüngrid}

\par 16 Ja kui ta Galilea mere ääres kõndis, nägi ta Siimonat ja Andreast võrku merre heitvat; sest nad olid kalurid.
\par 17 Ja Jeesus ütles neile: „Tulge minu järele ja ma teen teist inimesepüüdjad!”
\par 18 Ja sedamaid jätsid nad võrgud maha ja järgisid teda.
\par 19 Ja pisut eemale minnes nägi ta Jakoobust, Sebedeuse poega, ja Johannest, ta venda, neidki paadis võrke parandamas.
\par 20 Ja kohe kutsus ta nad. Ja nad jätsid oma isa Sebedeuse paati ühes palgalistega ning läksid ära tema järele.

\section*{Jeesus ajab inimesest välja rüveda vaimu}

\par 21 Ja nad läksid Kapernauma, ja ta läks hingamispäeval kohe kogudusekotta ja õpetas.
\par 22 Ja nad hämmastusid tema õpetusest; sest ta õpetas neid kui see, kellel on meelevald, ja mitte nõnda nagu kirjatundjad.
\par 23 Ja nende kogudusekojas oli parajasti inimene, kelles oli rüve vaim, ja see kisendas
\par 24 ning ütles: „Mis meil on tegemist sinuga, Jeesus Naatsaretlane? Oled sa tulnud meid hävitama? Ma tunnen sind, kes sa oled: Jumala Püha!”
\par 25 Siis Jeesus sõitles teda ning ütles: „Ole vait ja mine temast välja!”
\par 26 Ja rüve vaim raputas teda ja läks suure häälega kisendades temast välja.
\par 27 Ja nad kõik kohkusid väga, nii et nad üksteiselt küsisid: „Mis see on? Uus võimas õpetus! Ka rüvedaile vaimudele ta annab käsku ja nad kuulevad tema sõna!”
\par 28 Ja kuuldus temast levis varsti igale poole Galilea ümbrusesse.

\section*{Jeesus teeb terveks Siimona ämma ja teisi}

\par 29 Ja kogudusekojast väljudes tulid nad kohe Siimona ja Andrease majasse ühes Jakoobuse ja Johannesega.
\par 30 Aga Siimona ämm lamas maas palavikus, ja varsti öeldi temale seda.
\par 31 Ja ta läks tema juurde ja aitas ta üles, kinni hakates ta käest. Ja palavik lahkus temast, ja ta ümmardas neid.
\par 32 Aga õhtu tulles, kui päike looja läks, toodi tema juurde kõik haiged ja seestunud;
\par 33 ja kogu linn oli tulnud kokku ukse ette.
\par 34 Ja ta tegi terveks paljud, kes põdesid mõnesugust haigust, ja ajas välja mitu kurja vaimu ega lasknud kurje vaime rääkida, sest et nad teda tundsid.

\section*{Jeesus palvetab tühjas paigas ja lahkub Kapernaumast}

\par 35 Ja vara hommikul enne valget tõusis ta üles ja väljus ning läks ära tühja paika ja palvetas seal.
\par 36 Ja Siimon ning tema kaaslased tõttasid ta järele
\par 37 ja leidsid tema ning ütlesid talle: „Kõik otsivad sind!”
\par 38 Tema ütles neile: „Lähme teisale, lähemaisse aleveisse, et ma sealgi kuulutaksin, sest selleks ma olen välja tulnud!”
\par 39 Ja ta läks ning kuulutas nende kogudusekodades kogu Galileas ja ajas kurjad vaimud välja.

\section*{Jeesus teeb terveks pidalitõbise}

\par 40 Ja tema juurde tuleb pidalitõbine, palub teda ja heidab ta ette põlvili ning ütleb temale: „Kui sa tahad, võid sa mind teha puhtaks!”
\par 41 Ja Jeesusel hakkas temast hale meel; ta sirutas oma käe, puudutas teda ja ütles talle: „Ma tahan, saa puhtaks!”
\par 42 Ja kohe lahkus pidalitõbi temast ja ta sai puhtaks.
\par 43 Ja ta hoiatas teda kõvasti ja saatis ta kohe ära
\par 44 ning ütles talle: „Katsu, et sa kellelegi midagi ei ütle, vaid mine näita ennast preestrile ja ohverda oma puhastamise eest, mis Mooses on käskinud, neile tunnistuseks!”
\par 45 Aga kui see oli välja läinud, hakkas ta palju kuulutama ja levitama teateid sellest asjast, nõnda et Jeesus enam ei võinud avalikult linna minna, vaid viibis väljas tühjes paigus. Ja tema juurde tuldi kõikjalt.


\chapter{2}

\section*{Halvatu tervekstegemine}

\par 1 Mõne päeva pärast läks ta jälle Kapernauma, ja saadi kuulda, et ta on kodus.
\par 2 Siis kogunes sinna palju rahvast, nii et neil väljas ukse eeski ei olnud enam maad. Ja ta rääkis neile Jumala sõna.
\par 3 Ja ta juurde tuldi ning toodi halvatut, keda kandsid neli meest.
\par 4 Ja kui nad rahva pärast ei pääsenud teda tooma tema ligi, võtsid nad sealt kohalt, kus ta oli, katuse lahti, tegid augu ja lasksid alla sängi, millel halvatu lamas.
\par 5 Kui Jeesus nende usku nägi, ütles ta halvatule: „Poeg, su patud antakse andeks!”
\par 6 Aga seal istusid mõned kirjatundjad, ja need mõtlesid oma südames:
\par 7 „Mida see räägib? Ta pilkab Jumalat. Kes muu võib patte andeks anda kui ainuüksi Jumal?”
\par 8 Ja kohe tundis Jeesus oma vaimus, et nad iseeneses nõnda mõtlevad, ja ütles neile: „Miks te mõtlete seda oma südames?
\par 9 Kumb on kergem, kas öelda halvatule: sulle antakse patud andeks! või: tõuse üles ja võta oma säng ning kõnni?
\par 10 Aga et te teaksite, et Inimese Pojal on meelevald patte andeks anda, siis - ütleb ta halvatule -
\par 11 ma ütlen sulle: tõuse üles, võta oma säng ja mine koju!”
\par 12 Ja sedamaid tõusis ta üles ja võttis oma sängi ja läks välja kõigi nähes, nõnda et kõik ehmusid ja andsid au Jumalale ning ütlesid: „Niisugust asja me pole ilmaski näinud!”

\section*{Leevi kutsumine}

\par 13 Ja ta läks jälle välja mere äärde, ja kõik rahvas tuli tema juurde, ja ta õpetas neid.
\par 14 Ja mööda minnes nägi ta Leevi, Alfeuse poja, tollihoone juures istuvat; ja ta ütleb temale: „Järgi mind!” Ja tema tõusis ja järgis teda.
\par 15 Ja sündis, kui ta lauas istus tema majas, et istus ka palju tölnereid ja patuseid lauas ühes Jeesuse ja ta jüngritega; sest palju oli neid, kes teda järgisid.
\par 16 Ja kui variseride kirjatundjad nägid, et ta sööb tölnerite ja patustega, ütlesid nad tema jüngritele: „Miks ta sööb ja joob tölnerite ja patustega?”
\par 17 Ja kui Jeesus seda kuulis, ütles ta neile: „Ei terved vaja arsti, vaid haiged. Ma ei ole tulnud kutsuma õigeid, vaid patuseid!”

\section*{Paastumise küsimus}

\par 18 Ja Johannese ja variseride jüngrid pidasid paastu. Siis tuldi ja öeldi temale: „Miks Johannese ja variseride jüngrid paastuvad, aga sinu jüngrid ei paastu?”
\par 19 Jeesus ütles neile: „Ega peiupoisid või paastuda sel ajal, kui peigmees on nende juures? Niikaua kui peigmees on nende juures, ei või nad paastuda.
\par 20 Aga päev tuleb, mil peigmees neilt ära võetakse, ja siis, sel päeval, nad paastuvad.
\par 21 Ükski ei pane vanutamata riidetükki paigaks vanale kuuele, muidu augutäidis rebeneb selle küljest lahti, uus vana küljest, ja auk läheb pahemaks.
\par 22 Ka ei pane ükski värsket viina vanadesse nahklähkritesse; muidu värske viin ratkub lähkrid ja viin ning lähkrid lähevad raisku; vaid värske viin valatakse uutesse lähkritesse!”

\section*{Hingamispäeva pühitsemisest}

\par 23 Ja juhtus, et ta ühel hingamispäeval kõndis läbi viljapõldude ja ta jüngrid hakkasid teed käies viljapäid katkuma.
\par 24 Ja variserid ütlesid temale: „Vaata, miks nad hingamispäeval teevad, mida ei tohi?”
\par 25 Tema ütles neile: „Kas te ei ole lugenud, mis tegi Taavet, kui tal puudus oli ning nälg temal ja ta kaaslastel,
\par 26 kuidas ta läks Jumala kotta ülempreester Ebjatari ajal ja sõi ära vaateleivad, mida ühelgi ei olnud luba süüa kui vaid preestritel, ja andis ka neile, kes olid temaga?”
\par 27 Ja ta ütles neile: „Hingamispäev on tehtud inimese pärast, aga mitte inimene hingamispäeva pärast!
\par 28 Nii on Inimese Poeg ka hingamispäeva isand!”


\chapter{3}

\section*{Kuivanud käega mehe tervekstegemine}

\par 1 Ja ta läks jälle kogudusekotta. Ja seal oli mees, kellel oli kuivanud käsi.
\par 2 Ja nad varitsesid teda, kas ta hingamispäeval peaks terveks tegema, et kaevata tema peale.
\par 3 Siis ta ütles mehele, kellel oli kuivanud käsi: „Tõuse ja astu esile!”
\par 4 Ja ta küsis neilt: „Kas tohib hingamispäeval teha head või kurja, et elu päästa või hukata?” Aga nad jäid vait.
\par 5 Ja ta vaatas ümber nende peale vihaga ja oli ühtlasi kurb nende südamekanguse pärast ja ütles mehele: „Siruta käsi!” Ja ta sirutas; ja ta käsi sai jälle terveks.
\par 6 Ja variserid läksid välja ja pidasid kohe nõu Heroodese seltsiga Jeesuse vastu, et teda hukata.

\section*{Muud tervekstegemised}

\par 7 Aga Jeesus läks ära ühes oma jüngritega mere poole, ja suur rahvahulk järgis teda Galileast ja Judeast,
\par 8 ja Jeruusalemmast ja Idumeast ja teiselt poolt Jordanit ja Tüürose ja Siidoni ümbrusest, suur rahvakogu, kes kuuldes, mis suuri asju ta tegi, tuli tema juurde.
\par 9 Ja ta ütles oma jüngritele, et nad hoiaksid lootsiku temale valmis rahva pärast, et see ei tungiks ta peale,
\par 10 sest ta tegi terveks paljusid, mistõttu temale peale rõhusid kõik, kellel oli häda, et nad saaksid teda puudutada.
\par 11 Ja rüvedad vaimud, kui nad teda nägid, heitsid maha tema ette, kisendasid ning ütlesid: „Sina oled Jumala Poeg!”
\par 12 Ja ta hoiatas neid kangesti, et nad ei teeks teda avalikuks.

\section*{Jeesus määrab kaksteist apostlit}

\par 13 Ja ta läks üles mäele ja kutsus enese juurde, keda ta tahtis, ja nad tulid ta juurde.
\par 14 Ja ta määras kaksteist, et need oleksid tema juures ja ta võiks neid läkitada kuulutama
\par 15 ja et neil oleks meelevald välja ajada kurje vaime.
\par 16 Ta määras need kaksteist: Siimona, ja pani temale nimeks Peetrus;
\par 17 ja Jakoobuse, Sebedeuse poja, ja Johannese, Jakoobuse venna, ja pani neile nimeks Boanerges, see on Kõuepojad;
\par 18 ja Andrease ja Filippuse ja Bartolomeuse ja Matteuse ja Tooma ja Jakoobuse, Alfeuse poja, ja Taddeuse ja Siimona Kaanast,
\par 19 ja Juudas Iskarioti, kes ta ära andis.

\section*{Jeesus ei vaja saatana abi}

\par 20 Ja ta tuli koju. Ja rahvas tuli jälle kokku, nõnda et nad ei saanud leibagi võtta.
\par 21 Ja kui tema juures olijad seda kuulsid, läksid nad välja teda kinni võtma, sest nad ütlesid: „Ta on arust ära!”
\par 22 Ja kirjatundjad, kes olid tulnud Jeruusalemmast, ütlesid: „Temas on Peeltsebul, ja kurjade vaimude ülema abil ajab ta välja kurje vaime!”
\par 23 Tema kutsus need enese juurde ja ütles neile võrdumitega: „Kuidas võib saatan välja ajada saatanat?
\par 24 Ja kui kuningriik on isekeskis riius, siis ei või see kuningriik püsida.
\par 25 Ja kui koda on isekeskis riius, siis ei või see koda püsida.
\par 26 Ja kui saatan paneb iseenesele vastu, siis ei või ta püsida, vaid tal on ots käes!
\par 27 Ükski ei või minna vägeva majasse ja riisuda ta riistu, kui ta enne seda vägevat ei seo. Ja siis ta võib rüüstata tema maja.
\par 28 Tõesti ma ütlen teile, et kõik patud antakse inimlastele andeks, ja Jumala pilkamised, niipalju kui nad iganes on Jumalat pilganud;
\par 29 kuid kes iganes on pilganud Püha Vaimu, sellel ei ole andekssaamist iialgi, vaid ta on süüdlane igaveses patus!”
\par 30 Sest nad ütlesid: „Temas on rüve vaim!”

\section*{Tõeline sugulus}

\par 31 Siis tulid tema ema ja vennad ja seisid õues ja läkitasid tema juurde teda kutsuma.
\par 32 Ja rahvas istus ta ümber ja temale öeldi: „Vaata, su ema ja su vennad õues otsivad sind!”
\par 33 Ta vastas neile ning ütles: „Kes on mu ema või mu vennad?”
\par 34 Ja ta vaatas enese ümber nende peale, kes istusid sõõris ta ümber, ning ütles: „Vaata, siin on mu ema ja mu vennad!
\par 35 Sest kes iganes teeb Jumala tahtmist, see on mu vend ja õde ja ema!”


\chapter{4}

\section*{Tähendamissõna külvajast}

\par 1 Ja ta hakkas jälle õpetama mere ääres. Ja üsna palju rahvast kogunes ta juurde, nõnda et ta pidi minema paati ja istuma merel. Ja kõik rahvas oli maal mererannas.
\par 2 Ja ta õpetas neid palju tähendamissõnadega, ja oma õpetuses ta ütles neile:
\par 3 „Kuulge! Vaata, külvaja läks välja külvama.
\par 4 Ja sündis, kui ta külvas, et muist kukkus tee äärde, ja linnud tulid ja sõid selle.
\par 5 Ja muist kukkus kaljusele maale, kus tal ei olnud palju mulda; ja see tärkas varsti, sest tal ei olnud sügavat maad.
\par 6 Aga kui päike tõusis, kõrbes see, ja et tal ei olnud juurt, kuivas ta ära.
\par 7 Ja muist kukkus ohakate sekka, ja ohakad tõusid ning lämmatasid selle, ja ta ei andnud vilja.
\par 8 Ja muist kukkus heale maale, ja kui see tärkas ja kasvas, andis see vilja, ja mõni kandis kolmkümmend seemet ja mõni kuuskümmend ja mõni sada.”
\par 9 Ja ta ütles: „Kel kõrvad on kuulda, see kuulgu!”

\section*{Tähendamissõnade otstarve}

\par 10 Aga kui ta üksi oli, küsisid temalt need, kes ühes nende kaheteistkümnega olid tema ümber, nende tähendamissõnade mõtet.
\par 11 Ja ta ütles neile: „Teile on antud mõista Jumala riigi saladust, aga neile, kes on väljas, sünnib see kõik tähendamissõnade läbi,
\par 12 et nad nähes näeksid ega taipaks, ja kuuldes kuuleksid ega mõistaks, et nad ei pöörduks ja neile ei antaks andeks!”

\section*{Jeesus seletab tähendamissõna külvajast}

\par 13 Ja ta ütles neile: „Te ei taipa seda tähendamissõna, kuidas te siis tahate mõista kõiki teisi tähendamissõnu?
\par 14 Külvaja külvab sõna.
\par 15 Need on teeäärsed, kuhu sõna külvatakse, ja kui nad on kuulnud, tuleb kohe saatan ja võtab sõna ära, mis nendesse oli külvatud.
\par 16 Ja nõndasamuti, mis külvati kaljusele maale, on need, kes, kui nad sõna kuulevad, seda kohe rõõmuga vastu võtavad;
\par 17 aga neil ei ole juurt enestes, vaid nad kestavad ainult üürikese aja: pärast, kui tõuseb viletsus või tagakiusamine sõna pärast, nad taganevad varsti.
\par 18 Ja teised on ohakate sekka külvatud. Need on, kes sõna kuulevad,
\par 19 aga selle maailma mured ja rikkuse petlikkus ja muude asjade himud saavad võimule ja lämmatavad sõna ära, ja see jääb viljatuks.
\par 20 Ja kes on heale maale külvatud, on need, kes sõna kuulevad ja vastu võtavad ja kannavad vilja, mõni kolmekümnekordselt ja mõni kuuekümnekordselt ja mõni sajakordselt.”

\section*{Tähendamissõnad küünlast ja seemnekasvust}

\par 21 Ja ta ütles neile: „Kas küünal tuuakse selleks, et panna teda vaka alla või voodi alla? Kas mitte selleks, et teda panna küünlajalale?
\par 22 Sest midagi ei ole varjul muu jaoks, kui et ta saaks avalikuks, ja midagi ei ole peidetud muu jaoks, kui et ta tuleks ilmsiks.
\par 23 Kui kellelgi kõrvad on kuulda, see kuulgu!”
\par 24 Ja ütles neile: „Pange tähele, mida te kuulete. Mis mõõduga te mõõdate, sellega mõõdetakse teile ja teile lisatakse veel juurde.
\par 25 Sest kellel on, sellele antakse; ja kellel ei ole, sellelt võetakse see, mis tal on!”
\par 26 Ja ta ütles: „Nõnda on Jumala riik, otsekui inimene viskab seemet maa peale,
\par 27 ja heidab magama ja tõuseb üles öösel ja päeval, ja seeme tärkab ja sirgub, nõnda et ta isegi ei tea kuidas.
\par 28 Sest maa kannab vilja iseenesest, esmalt orast, pärast päid, siis täit nisu pea sees.
\par 29 Aga kui vili on valminud, saadab ta sedamaid sirbi sinna, sest lõikus on käes.”
\par 30 Ja ta ütles: „Millega võiksime võrrelda Jumala riiki? Või missuguse tähendamissõnaga me seda tähendame?
\par 31 Sinepiivakesega, mis on, kui see maha külvatakse, väiksem kui kõik muud seemned maa peal,
\par 32 aga kui see on külvatud, tõuseb ta ning saab suuremaks kõigist taimedest ja kasvatab suured oksad, nõnda et taeva linnud võivad pesitada ta varju all.”
\par 33 Ja mitme niisuguse tähendamissõnaga rääkis ta neile sõnu sedamööda, kuidas nad suutsid kuulda.
\par 34 Aga ilma tähendamissõnata ei rääkinud ta neile midagi; ja isepäinis ta seletas jüngritele kõik ära.

\section*{Jeesus vaigistab tormi}

\par 35 Ja samal päeval, kui õhtu käes oli, ütles ta neile: „Lähme üle teisele poole.”
\par 36 Ja kui nad rahva olid lasknud ära minna, võtavad nad tema paati, nõnda nagu ta oli; ja ka teisi paate oli temaga.
\par 37 Siis tõusis kange tuulispea ja lõi laineid paati, nõnda et paat juba täitus.
\par 38 Ja tema ise oli paadi päras magamas, toetudes peaalusele; ja nad äratasid ta üles ja ütlesid talle: „Õpetaja, kas sa ei hooli sellest, et me hukkume?”
\par 39 Tema tõusis üles ja sõitles tuult ning ütles merele: „Ole vait ja vaga!” Ja tuul rauges, ja meri jäi täiesti vaikseks.
\par 40 Ja ta ütles neile: „Miks te olete nii arad? Kuidas teil ei ole usku?”
\par 41 Ja nad lõid väga kartma ning ütlesid üksteisele: „Kes see siis õieti on, et ka tuul ja meri kuulevad tema sõna?”


\chapter{5}

\section*{Jeesus teeb terveks seestunu}

\par 1 Ja nad tulid teisele poole merd gerasalaste maale.
\par 2 Ja kui ta paadist väljus, kohtas teda varsti haudadest tulnud mees, kellel oli rüve vaim.
\par 3 Mehe eluase oli surnuhaudades ja keegi ei suutnud teda ahelatega kinni panna,
\par 4 sest ta oli mitu korda olnud kinni jalgraudus ja ahelais, aga oli ahelad puruks kiskunud ja jalgrauad katki murdnud, nii et ükski ei suutnud teda taltsutada.
\par 5 Ööd ja päevad läbi oli ta aina mägedel ja surnuhaudades ning karjus ja peksis ennast kividega.
\par 6 Aga kui ta Jeesust kaugelt nägi, jooksis ta ja kummardas teda
\par 7 ja ütles suure häälega kisendades: „Mis sinul on minuga tegemist, Jeesus, kõige kõrgema Jumala Poeg? Ma vannutan sind Jumala juures, et sa mind ei piinaks!”
\par 8 Sest ta oli öelnud temale: „Mine välja inimesest, sa rüve vaim!”
\par 9 Ja ta küsis temalt: „Mis su nimi on?” Ta ütles talle: „Mu nimi on Leegion, sest meid on palju!”
\par 10 Ja ta anus teda väga, et ta neid välja ei saadaks sealt maalt.
\par 11 Ent seal oli mäe all suur seakari söömas.
\par 12 Ja rüvedad vaimud anusid teda ning ütlesid: „Läheta meid sigade sisse, et me läheksime neisse!”
\par 13 Ja tema andis neile loa. Siis rüvedad vaimud väljusid ja läksid sigade sisse. Ja seakari kukutas enese ülepeakaela kaldalt merre, arvult ligi kaks tuhat, ja uppus merre.
\par 14 Aga nende karjatajad põgenesid ja teatasid sellest linnas ja maal. Ja inimesi tuli välja vaatama, mis on sündinud.
\par 15 Ja nad tulid Jeesuse juurde ja nägid seestunut, kelles oli olnud Leegion, istuvat, riietatud ja selge aruga olevat. Ja nad kartsid.
\par 16 Ja need, kes seda olid näinud, jutustasid neile, mis seestunuga oli sündinud, ja sigadest.
\par 17 Ja inimesed hakkasid teda paluma, et ta nende maa-alalt ära läheks.
\par 18 Kui ta paati astus, palus talt see, kes oli olnud seestunud, et ta tohiks jääda tema juurde.
\par 19 Aga tema ei lubanud seda talle mitte, vaid ütles talle: „Mine koju omaste juurde ja teata neile, mis suuri asju Issand sulle on teinud ja kuidas ta sinu peale on halastanud.”
\par 20 Ja ta läks ära ja hakkas Dekapolis kuulutama, mis suuri asju Jeesus temale oli teinud. Ja kõik panid seda imeks.

\section*{Jairuse tütar ja veritõbine naine}

\par 21 Ja kui Jeesus paadiga jälle siiapoole kaldale oli jõudnud, kogunes palju rahvast tema juurde, ja ta oli mererannas.
\par 22 Siis tuleb üks kogudusekoja ülemaid, Jairus nimi, ja langeb teda nähes tema jalge ette maha
\par 23 ja palub teda väga ning ütleb: „Mu tütreke on hinge vaakumas; ma palun, et sa tuleksid ja paneksid oma käed tema peale, et ta saaks terveks ning jääks ellu!”
\par 24 Ja ta läks temaga. Ja palju rahvast käis tema järel ja rõhus ta peale.
\par 25 Ja seal oli naine, kes oli kaksteist aastat olnud veritõves
\par 26 ja oli saanud palju kannatada mitme arsti all ning oli kulutanud kogu oma vara, ilma et ta oleks mingit abi saanud, kuna haigus oli läinud veel pahemaks.
\par 27 See oli kuulnud Jeesusest ja tuli rahva seas tema selja taha ja puudutas ta kuube.
\par 28 Sest ta mõtles: kui ma aga puudutaksin tema riideid, saaksin ma terveks.
\par 29 Ja kohe kuivas tema vereläte, ja ta tundis oma ihu vaevast terveks saanud olevat.
\par 30 Ka Jeesus tundis kohe iseeneses, et vägi temast oli väljunud, pöördus ümber rahva seas ja küsis: „Kes puudutas mu riideid?”
\par 31 Tema jüngrid ütlesid temale: „Sa näed, et rahvas rõhub su peale, ja küsid: kes puudutas mind?”
\par 32 Ent ta vaatas ümber, et näha, kes seda oli teinud.
\par 33 Siis lõi naine kartma ning värises, sest et ta teadis, mis temale oli sündinud, ja tuli ning heitis maha tema ette ja ütles temale kogu tõe.
\par 34 Aga ta ütles naisele: „Tütar, sinu usk on sind aidanud; mine rahuga ja ole terve oma vaevast!”
\par 35 Kui ta alles rääkis, tulid mõned kogudusekoja ülema perest ja ütlesid: „Su tütar on surnud, mis sa õpetajat veel tülitad?”
\par 36 Aga Jeesus jättis tähele panemata, mida räägiti, ja ütles kogudusekoja ülemale: „Ära karda, usu vaid!”
\par 37 Ja ta ei lasknud kedagi tulla ühes enesega kui vaid Peetruse, Jakoobuse ja Johannese, Jakoobuse venna.
\par 38 Ja nad tulevad kogudusekoja ülema majasse, ja ta näeb käratsemist ja nutjaid ja ulujaid;
\par 39 ja sisse minnes ütleb ta neile: „Mis te käratsete ja nutate? Laps ei ole surnud, vaid magab!”
\par 40 Ja nad naersid teda. Aga kui ta kõik on välja ajanud, võtab ta enesega lapse isa ja ema ja enese juures olijad ja läheb sisse sinna, kus laps maas oli.
\par 41 Ja lapse käest kinni hakates ütleb ta temale: „Talitaa kuum!” see on tõlgitult: neitsike, ma ütlen sulle, tõuse üles!
\par 42 Ja sedamaid tõusis neitsike üles ja kõndis; ta oli juba kaksteist aastat vana. Ja nad ehmusid üliväga.
\par 43 Ja ta keelas neid kangesti, et keegi seda ei saaks teada, ja käskis tütarlapsele süüa anda.


\chapter{6}

\section*{Jeesust põlatakse kodukohas}

\par 1 Ja ta läks sealt ära ja tuli oma kodukohta, ja ta jüngrid järgisid teda.
\par 2 Ja kui hingamispäev tuli, hakkas ta kogudusekojas õpetama. Ja paljud, kes teda kuulsid, hämmastusid väga ja ütlesid: „Kust sellele see kõik on tulnud ja mis tarkus see on, mis temale on antud? Ja kuidas niisugused vägevad teod sünnivad ta käte läbi?
\par 3 Eks tema ole see puusepp, Maarja poeg, Jakoobuse ja Joosese ja Juuda ja Siimona vend? Ja eks ka tema õed ole siin meie juures?” Ja nad pahandusid temast.
\par 4 Aga Jeesus ütles neile: „Prohvet ei ole mujal autu kui oma kodukohas ja oma sugulaste seas ning omas majas!”
\par 5 Ja ta ei saanud seal teha ühtki muud vägevat tegu kui ainult panna oma käed väheste haigete peale ja nad terveks teha.
\par 6 Ja ta pani imeks nende uskmatust. Ja tema käis läbi ümberkaudsed külad ning õpetas.

\section*{Apostlite läkitamine}

\par 7 Ja ta kutsus need kaksteist enese juurde ja hakkas neid läkitama kahekaupa ja andis neile meelevalla rüvedate vaimude üle,
\par 8 ja käskis neid, et nad midagi muud ei võtaks teele kaasa kui ainult kepi, ei pauna ega leiba ega raha vöö vahele,
\par 9 vaid sidugu paeltega kingad jalga ja ärgu pangu kaht kuube selga.
\par 10 Ja ta ütles neile: „Kuhu kotta te iganes sisse lähete, sinna jääge, seni kui te sealt väljute.
\par 11 Ja kus paigas iganes teid vastu ei võeta ega kuulata, sealt väljuge ja puistake tolm oma jalgadelt neile tunnistuseks!”
\par 12 Ja nad läksid välja ja kuulutasid, et tuleb meelt parandada,
\par 13 ja ajasid palju kurje vaime välja ja võidsid palju tõbiseid õliga ja tegid nad terveks.

\section*{Ristija Johannese surm}

\par 14 Ja kuningas Heroodes sai kuulda Jeesusest, sest ta nimi oli juba tuttav, ja ütles: „Ristija Johannes on surnuist üles tõusnud ja sellepärast on imelised väed tema sees tegevad!”
\par 15 Aga teised ütlesid: „Tema on Eelija!” Teised aga: „Tema on prohvet nagu üks prohveteid!”
\par 16 Aga kui Heroodes seda kuulis, ütles ta: „Johannes, kelle pea ma otsast raiusin, on üles tõusnud!”
\par 17 Sest Heroodes ise oli läkitanud ja Johannese kinni võtnud ja tema sidunud ning vangitorni pannud Heroodiase, oma venna Filippuse naise pärast. Sest ta oli tema enesele naiseks võtnud.
\par 18 Oli ju Johannes öelnud Heroodesele: „Sul ei ole luba oma venna naist pidada!”
\par 19 Aga Heroodias kandis ta peale viha ja tahtis teda tappa, ja ei saanud mitte.
\par 20 Sest Heroodes kartis Johannest; ta teadis teda õige ja püha mehe olevat ning kaitses teda, ja kui ta teda kuulis, jäi ta mõneski asjas kahevahele ja kuulas teda siiski hea meelega.
\par 21 Kui nüüd paras aeg oli tulnud, mil Heroodes oma sünnipäeval tegi söömaaja oma suurtele isandatele ja pealikuile ning Galilea ülemaile,
\par 22 siis tuli sama Heroodiase tütar sisse ja tantsis. See oli meelt mööda Heroodesele ja neile, kes lauas istusid. Siis ütles kuningas neitsile: „Palu mult, mida sa iganes tahad, ja ma annan sulle!”
\par 23 Ja ta vandus temale: „Ma annan sulle, mis sa iganes minult palud, olgu kas või pool minu kuningriiki!”
\par 24 Aga neitsi läks välja ja küsis oma emalt: „Mis ma pean paluma?” Ent ema ütles: „Ristija Johannese pea!”
\par 25 Ja neitsi läks kohe rutuga kuninga juurde, palus ning ütles: „Ma tahan, et sa mulle sedamaid annad vaagnal Ristija Johannese pea!”
\par 26 Siis sai kuningas väga kurvaks, kuid vande ja lauasistujate pärast ta ei tahtnud temale seda keelata.
\par 27 Ja kuningas läkitas kohe valvuri ja käskis tuua tema pea.
\par 28 See läks ja raius vangitornis ta pea otsast ära ja tõi ta pea vaagnal ning andis selle neitsile, ja neitsi andis selle oma emale.
\par 29 Kui tema jüngrid seda kuulsid, tulid nad ja võtsid ta keha ja panid selle hauda.

\section*{Apostlite tagasitulek}

\par 30 Ja apostlid tulid kokku Jeesuse juurde ja kuulutasid temale kõik, mida nad olid teinud ja mida õpetanud.
\par 31 Siis ta ütles neile: „Tulge teie kõrvale üksikusse paika ja puhake pisut!” Sest tulijaid ja minejaid oli palju ja nad ei saanud mahti süüagi.
\par 32 Ja nad sõudsid paadiga kõrvale üksikusse paika.
\par 33 Ja paljud nägid neid ära sõudvat ning tundsid tema ära ja jooksid jala sinna kokku kõigist linnadest ja jõudsid neist ette.
\par 34 Ja kui ta paadist välja astus ja nägi palju rahvast, hakkas tal neist hale meel, et nad olid otsekui lambad, kellel pole karjast. Ja ta hakkas neid pikalt õpetama.

\section*{Viie tuhande mehe söötmine}

\par 35 Aga kui aeg oli läinud hiliseks, tulid ta jüngrid ta juurde ja ütlesid: „Paik on tühi ja aeg jääb juba hiliseks;
\par 36 lase nad minema, et nad läheksid ümberkaudu asulaisse ja küladesse ja ostaksid endile leiba, sest neil ei ole midagi süüa!”
\par 37 Aga tema kostis ja ütles neile: „Andke teie neile süüa!” Nemad ütlesid temale: „Kas me peaksime minema kahesaja teenari eest leiba ostma ja neile süüa andma?”
\par 38 Aga tema küsis neilt: „Mitu leiba teil on? Minge vaadake.” Kui nad olid teada saanud, ütlesid nad: „Viis leiba ja kaks kala.”
\par 39 Ja ta käskis neid asetada kõik istuma, salkkond salkkonna kõrvale haljale murule.
\par 40 Ja nad istusid maha ridamisi, sajakaupa ja viiekümnekaupa.
\par 41 Ja ta võttis need viis leiba ja kaks kala, vaatas üles taeva poole ja õnnistas ja murdis leivad ja andis need oma jüngrite kätte rahva ette panemiseks. Ja ka need kaks kala jagas ta kõikidele.
\par 42 Ja kõik sõid ja nende kõhud said täis.
\par 43 Ja korjati kokku kaksteist korvitäit palukesist ja kalust.
\par 44 Ja neid, kes leivust olid söönud, oli viis tuhat meest.

\section*{Jeesus kõnnib vee peal}

\par 45 Ja sedamaid sundis ta oma jüngreid astuma paati ja sõitma tema eele teisele rannale Betsaidasse, kuni tema laseb rahva minema.
\par 46 Ja kui ta lahkumistervituse oli öelnud, läks ta mäele palvetama.
\par 47 Kui siis õhtu jõudis, oli paat keset merd ja tema üksi maal.
\par 48 Ja ta nägi neil sõudes püsti häda käes olevat, sest tuul oli neile vastu. Ja neljandal öövahi-korral tuli ta mere peal kõndides nende juurde ja tahtis neist mööda minna.
\par 49 Aga kui nad teda nägid mere peal kõndivat, arvasid nad, et see on tont, ja hakkasid karjuma.
\par 50 Sest nad kõik nägid teda ja ehmusid. Aga tema rääkis kohe nendega ja ütles neile: „Olge julged, mina olen, ärge kartke!”
\par 51 Ja ta läks nende juurde paati, ja tuul rauges. Ja nad kohkusid üpris väga iseenestes,
\par 52 sest nad ei olnud veel aru saanud leibade loost, vaid nende süda oli läinud kõvaks.

\section*{Tervekstegemised Gennesareti mere ääres}

\par 53 Ja kui nad üle mere olid maale sõudnud, jõudsid nad Gennesaretti ja astusid rannale.
\par 54 Ja kui nad paadist olid väljunud, tunti ta sedamaid ära
\par 55 ja joosti kogu sealne ümberkaudne maa läbi ning hakati haigeid vooditega kandma igale poole, kus teda kuuldi olevat.
\par 56 Ja kuhu ta iganes läks küladesse või linnadesse või asulaisse, asetati põdejaid turgudele ja paluti teda, et nad saaksid vaid puudutada tema kuue palistust. Ja kõik, kes teda puudutasid, said terveks.


\chapter{7}

\section*{Jeesus vastab etteheiteile pärimuste rikkumise kohta}

\par 1 Ja variserid ja mõned kirjatundjaist, kes olid tulnud Jeruusalemmast, kogunesid tema juurde,
\par 2 ja nähes mõningaid ta jüngritest ebapühade, see on pesemata kätega leiba võtvat -
\par 3 sest variserid ja ükski juut ei söö enne, kui nad on hoolsasti käsi pesnud, nii pidades esivanemate pärimust;
\par 4 ka turult tulles ei söö nad muidu kui et nad endid enne pesevad; ja palju muid asju on, mida nad on võtnud pidada, nagu karikate ja kiviriistade ja vasknõude pesemised -
\par 5 küsisid temalt variserid ja kirjatundjad: „Miks sinu jüngrid ei ela vanemate pärimuse järgi, vaid söövad leiba ebapühade kätega?”
\par 6 Aga ta ütles neile: „Jesaja on hästi teist silmakirjatsejaist ennustanud, nõnda nagu on kirjutatud: see rahvas austab mind huultega, kuid nende süda on minust kaugel;
\par 7 ilmaaegu nad teenivad mind, õpetades õpetusi, mis on inimeste käskimised.
\par 8 Sest hüljates Jumala käsu, te peate inimeste pärimust!”
\par 9 Ja ta ütles neile: „Ilus küll! Te heidate Jumala käsu kõrvale, et pidada oma pärimust!
\par 10 Sest Mooses ütles: sa pead oma isa ja ema austama, ja kes isa või ema neab, peab surma surema!
\par 11 Teie aga ütlete: kui inimene ütleb oma isale või emale: korbaan - see tähendab ohvriand - on see, mis sa iganes minu käest võiksid saada!,
\par 12 siis te ei lase teda enam ühtki head teha oma isale ega emale
\par 13 ja teete nõnda Jumala sõna tühjaks oma pärimusega, mille te olete pärinud, ja teete palju muud seesugust.”
\par 14 Ja ta kutsus jälle kõik rahva enese juurde ja ütles neile: „Kuulge mind kõik ja mõistke?
\par 15 Miski, mis läheb väljastpoolt inimest tema sisse, ei või teda rüvetada; aga mis läheb inimesest välja, see rüvetab inimest.
\par 16 [Kui kellelgi kõrvad on kuulda, see kuulgu!]
\par 17 Ja kui ta rahva juurest oli läinud ühte majasse, küsisid ta jüngrid temalt selle tähendamissõna mõtet.
\par 18 Ja ta ütles neile: „Kas teiegi olete mõistmatud? Eks te saa aru, et kõik, mis väljastpoolt läheb inimese sisse, ei või teda rüvetada?
\par 19 Sest see ei lähe mitte tema südamesse, vaid kõhtu, ja väljub eri paika.” Nii ta kuulutas kõik toidused puhtaks.
\par 20 Ja ta ütles: „Mis inimesest välja läheb, see rüvetab inimest.
\par 21 Sest seestpoolt, inimeste südamest, lähtuvad kurjad mõtted, hoorus, vargused, tapmised,
\par 22 abielurikkumised, ahnus, kurjused, kavalus, kiimalus, kade silm, Jumala pilkamine, kõrkus, rumalus.
\par 23 Kõik need pahad asjad lähtuvad seestpoolt ja rüvetavad inimest.”

\section*{Jeesus teeb terveks Sürofoiniikia naise tütre}

\par 24 Ja ta tõusis ning läks sealt ära Tüürose ja Siidoni alale. Ja ta läks ühte majasse ega tahtnud, et keegi seda teada saaks. Kuid ta ei võinud jääda varjule.
\par 25 Sest üks naine, kelle tütrekesel oli rüve vaim, oli temast kohe kuulda saanud ja tuli ning heitis ta jalge ette maha.
\par 26 See naine oli kreeklane, sündimise poolest Sürofoiniikia inimene. Ja see palus teda, et ta ajaks kurja vaimu välja ta tütrest.
\par 27 Siis ütles Jeesus temale: „Lase esmalt laste kõhud täis saada; sest ei ole mitte hea võtta laste leib ja heita koerakeste ette!”
\par 28 Ent naine kostis ja ütles temale: „Jah, Issand, aga koerakesed laua all söövad laste raasukesi!”
\par 29 Siis ta ütles naisele: „Selle sõna pärast mine; kuri vaim on sinu tütrest lahkunud!”
\par 30 Ja naine läks koju ja leidis tütre voodis magavat ning kurja vaimu lahkunud olevat.

\section*{Kurdi ja keeletu tervekstegemine}

\par 31 Ja kui ta jälle välja läks Tüürose aladelt, tuli ta Siidoni kaudu Galilea mere äärde Dekapoli alade keskkohale.
\par 32 Ja ta juurde toodi kurt, kellel oli kidakeel, ja paluti teda, et ta paneks oma käe tema peale.
\par 33 Siis ta võttis tema rahva seast kõrvale ja pistis oma sõrmed tema kõrvu ja sülitas ning puudutas ta keelt,
\par 34 vaatas üles taevasse, ohkas ja ütles talle: „Effataa!”, see on: mine lahti!
\par 35 Ja tema kõrvad läksid lahti ja ta keeleköidik pääses kohe valla, ja ta rääkis selgesti.
\par 36 Ja ta keelas neid seda ühelegi rääkimast. Aga mida enam ta neid keelas, seda enam nad kuulutasid seda.
\par 37 Ja inimesed hämmastusid üliväga ja ütlesid: „Kõik on ta hästi teinud; kurdid teeb ta kuulma ja keeletud rääkima!”


\chapter{8}

\section*{Nelja tuhande mehe söötmine}

\par 1 Neil päevil, kui jälle väga palju rahvast koos oli ja neil ei olnud midagi süüa, kutsus Jeesus jüngrid enese juurde ja ütles neile:
\par 2 „Mul on väga hale meel rahvast, sest nad on juba kolm päeva olnud minu juures, ja neil ei ole midagi süüa.
\par 3 Ja kui ma nad lasen koju minna söömata, nõrkevad nad teel!” Sest mõned neist olid kaugelt tulnud.
\par 4 Ja ta jüngrid vastasid temale: „Kust võib keegi siin kõrbes neid leivaga täita?”
\par 5 Ja ta küsis neilt: „Mitu leiba teil on?” Nad ütlesid: „Seitse.”
\par 6 Ja ta käskis rahva istuda maa peale. Siis ta võttis need seitse leiba, tänas, murdis ja andis oma jüngrite kätte, et nad paneksid ette. Ja nad panid rahvale ette.
\par 7 Ja neil oli pisut kalu; ja ta õnnistas ja käskis ka need ette panna.
\par 8 Siis nad sõid ja nende kõhud said täis, ja ülejäänud palukesi korjati kokku seitse korvitäit.
\par 9 Aga sööjaid oli olnud arvata neli tuhat. Ja ta laskis nad minna.
\par 10 Ja varsti ta astus ühes oma jüngritega paati ja jõudis Dalmanuuta paikadesse.

\section*{Jeesus ei hooli variseride tunnustähe nõudmisest}

\par 11 Ja variserid tulid välja ja hakkasid temaga vaidlema ning nõudsid temalt tunnustähte taevast, et teda kiusata.
\par 12 Siis ta ohkas oma vaimus ja ütles: „Miks see sugu otsib tähte? Tõesti, ma ütlen teile, sellele soole ei anta tähte!”
\par 13 Ja ta jättis nad maha ja astus jälle paati ja läks ära teisele poole.

\section*{Jeesus noomib jüngreid taipamatuse pärast}

\par 14 Ja nad olid unustanud leiba kaasa võtta ja neil ei olnud rohkem kui üksainus leib paadis kaasas.
\par 15 Ja ta käskis neid ning ütles: „Vaadake ette, hoiduge variseride haputaignast ja Heroodese haputaignast!”
\par 16 Ja nad arutlesid isekeskis seda, et neil ei ole leiba.
\par 17 Tema märkas seda ning ütles neile: „Mis te arutate, et teil ei ole leiba? Kas te veel ei mõista ega saa aru? Kas teil on ikka alles kõva süda?
\par 18 Silmad teil on, aga te ei näe; kõrvad teil on, aga te ei kuule ega mäleta.
\par 19 Kui ma viis leiba murdsin viiele tuhandele, mitu korvitäit palukesi te siis korjasite?„ Nad ütlesid: ”Kaksteist!”
\par 20 „Kui ma seitse murdsin neljale tuhandele, mitu korvitäit palukesi te siis korjasite?” Nad ütlesid: „Seitse.”
\par 21 Siis ta ütles neile: „Kuidas te siis ei mõista?”

\section*{Jeesus teeb Betsaidas pimeda nägijaks}

\par 22 Ja nad tulevad Betsaidasse. Ja ta juurde tuuakse pime, ja nad paluvad teda, et ta teda puudutaks.
\par 23 Siis ta haaras pimeda käest kinni ja viis ta alevist välja, sülitas temale silmi ning pani oma käed ta peale ja küsis talt: „Kas sa näed midagi?”
\par 24 Tema vaatas üles ning ütles: „Ma näen inimesi, ma näen neid nagu puid kõndimas!”
\par 25 Siis ta pani taas oma käed ta silmadele; ja nüüd ta vaatas teravasti ja oli jälle terve; ja ta nägi kõike selgesti.
\par 26 Ja ta saatis tema koju ning ütles: „Ära minegi külasse!”

\section*{Peetrus tunnistab Jeesuse Kristuseks}

\par 27 Ja Jeesus ja ta jüngrid läksid Filippuse Kaisarea küladesse. Ja teel küsis ta oma jüngritelt ning ütles neile: „Kelle mind inimesed ütlevad olevat?”
\par 28 Aga nemad vastasid temale ning ütlesid: „Ristija Johannese, ja teised Eelija, ja veel teised ühe prohveteist!”
\par 29 Siis ta küsis neilt: „Aga teie, kelle teie ütlete mind olevat?” Peetrus vastas ning ütles temale: „Sina oled Kristus!”
\par 30 Ja ta hoiatas neid kõvasti, et nad temast ühelegi ei räägiks.

\section*{Jeesus kuulutab ette oma surma ja ülestõusmist}

\par 31 Ja ta hakkas neid õpetama: „Inimese Poeg peab palju kannatama ja hüljatama vanemate ja ülempreestrite poolt ning tapetama ja kolme päeva pärast üles tõusma.”
\par 32 Ja ta rääkis seda üsna avalikult. Ja Peetrus võttis ta isepäinis ja hakkas teda noomima.
\par 33 Aga tema pöördus, vaatas oma jüngritele ja sõitles Peetrust ning ütles: „Tagane minust, saatan, sest sa ei mõtle seda, mis on Jumala, vaid mis on inimeste kohane!”

\section*{Nõue iseenese salgamiseks}

\par 34 Ja ta kutsus rahva enese juurde ühes oma jüngritega ning ütles neile: „Kui keegi tahab minu järele tulla, siis ta salaku end ära ja võtku oma rist enese peale ja järgigu mind.
\par 35 Sest kes iganes oma hinge tahab päästa, see kaotab selle; aga kes iganes oma hinge kaotab minu ja armuõpetuse pärast, see päästab selle.
\par 36 Sest mis kasu on inimesel sellest, kui ta kogu maailma kasuks saaks, aga oma hingele kahju teeks?
\par 37 Või mis võib inimene anda oma hinge vahetushinnaks?
\par 38 Sest kes häbeneb minu ja mu sõnade pärast selles abielurikkujas ja patuses tõus, selle pärast häbeneb ka Inimese Poeg, kui ta tuleb oma Isa auhiilguses ühes pühade inglitega!”


\chapter{9}

\section*{Nõue iseenese salgamiseks}

\par 1 Ja tema ütles neile: „Tõesti, ma ütlen teile, neist, kes siin seisavad, on mõned, kes ei maitse surma, kuni nad näevad Jumala riigi tulnud olevat väega!”

\section*{Jeesuse muutumine}

\par 2 Ja kuue päeva pärast võttis Jeesus enesega Peetruse, Jakoobuse ja Johannese ning viis nad üles kõrgele mäele isepäinis, ja tema muudeti nende ees.
\par 3 Tema riided läksid hiilgavaks, helevalgeks otsekui lumi, nagu neid ükski vanutaja maa peal ei suuda teha nii heledaks.
\par 4 Ja Eelija ühes Moosesega ilmus neile, ja nad kõnelesid Jeesusega.
\par 5 Ja Peetrus kostis ning ütles Jeesusele: „Rabi, siin on hea olla! Teeme nüüd kolm telki: sinule ühe ja Moosesele ühe ja Eelijale ühe!”
\par 6 Aga ta ei teadnud, mida öelda, sest nad olid täis kartust.
\par 7 Siis tuli pilv ja heitis varju nende üle; ja hääl kostis pilvest: „See on mu armas Poeg, teda kuulake!”
\par 8 Ja kui nad äkitselt ümber vaatasid, ei näinud nad enam kedagi muud kui üksnes Jeesust eneste juures.
\par 9 Ja kui nad mäelt alla läksid, keelas ta neid, et nad ühelegi ei jutustaks, mis nad olid näinud, enne kui Inimese Poeg on surnuist üles tõusnud.
\par 10 Ja nad pidasid meeles selle sõna ja küsisid isekeskis: „Mis tähendab surnuist üles tõusta?”
\par 11 Siis nad küsisid temalt ning ütlesid: „Miks kirjatundjad ütlevad, et Eelija peab enne tulema?”
\par 12 Aga tema ütles neile: „Eelija tuleb küll enne korda seadma kõike. Ent kuidas on kirjutatud Inimese Pojast, et ta peab palju kannatama ja teda peetakse halvaks?
\par 13 Aga ma ütlen teile, et ka Eelija on tulnud ja et nad tegid temale, mida nad tahtsid, nõnda nagu temast on kirjutatud!”

\section*{Jeesus teeb terveks vaimuhaige poisi}

\par 14 Ja kui nad tulid jüngrite juurde, nägi ta palju rahvast nende ümber ja kirjatundjaid vaidlevat nendega.
\par 15 Ja kohe, kui rahvahulk teda nägi, kohkusid kõik ja jooksid temale vastu ning teretasid teda.
\par 16 Ja ta küsis nendelt: „Miks te vaidlete nendega?”
\par 17 Siis vastas keegi rahva seast: „Õpetaja, ma tõin su juurde oma poja, kellel on keeletu vaim;
\par 18 ja kus ta iganes tuleb tema kallale, kisub ta teda, ja tema ajab vahtu ja kiristab hambaid ja kuivetub. Ja ma ütlesin su jüngritele, et nad ajaksid ta välja, kuid nad ei suutnud!”
\par 19 Aga tema vastas neile ja ütles: „Oh sina uskmatu tõug! Kui kaua ma pean olema teie juures? Kui kaua ma pean teiega kannatama? Tooge ta minu juurde!”
\par 20 Ja nad tõid ta tema juurde. Ja kohe, kui ta Jeesust nägi, raputas vaim teda ja ta kukkus maa peale ja aeles ning ajas vahtu.
\par 21 Ja ta küsis tema isalt: „Kui kaua aega see on temal olnud?” Tema vastas: „Lapsest saadik!
\par 22 See on teda mitu puhku küll tulle, küll vette visanud, et teda hukata; aga kui sa kuidagi võid, siis olgu sul meist hale meel ning aita meid!”
\par 23 Aga Jeesus ütles temale: „Sa ütled: kui sa võid! Kõik on võimalik sellele, kes usub!”
\par 24 Ja poisi isa hüüdis sedamaid ning ütles: „Ma usun, aita mu uskmatust!”
\par 25 Aga nähes, et rahvast ikka rohkem kokku jooksis, sõitles Jeesus rüvedat vaimu ning ütles temale: „Sina, keeletu ja kurt vaim, ma käsin sind: mine temast välja ja ära tule enam tema sisse!”
\par 26 Siis see kisendas ja raputas teda väga ja väljus. Ja poiss jäi otsekui surnuks, nõnda et mitmed ütlesid: „Ta on surnud!”
\par 27 Kuid Jeesus hakkas ta käest kinni ja tõstis ta üles; ja ta tõusis püsti.
\par 28 Ja kui ta koju tuli, küsisid ta jüngrid temalt isepäinis: „Miks meie ei võinud teda välja ajada?”
\par 29 Tema ütles neile: „See sugu ei või teisiti välja minna kui vaid palvega!”

\section*{Jeesus kuulutab ette teist korda oma surma ja ülestõusmist}

\par 30 Ja nad läksid sealt ära ja rändasid läbi Galilea. Ja ta ei tahtnud, et keegi saaks teada.
\par 31 Sest ta õpetas oma jüngreid ja ütles neile: „Inimese Poeg antakse inimeste kätte ja nad tapavad tema, ja kui ta on tapetud, tõuseb ta kolmandal päeval üles!”
\par 32 Aga nemad ei saanud aru sellest sõnast ja kartsid temalt küsida.

\section*{Õpetus alandlikkuseks}

\par 33 Ja nad tulid Kapernauma. Ja koju jõudes ta küsis neilt: „Mille üle te teel vaidlesite?”
\par 34 Aga nad jäid vait; sest nad olid teel isekeskis vaielnud selle üle, kes on suurem.
\par 35 Ja tema istus maha ja kutsus need kaksteist ning ütles neile: „Kui keegi tahab olla esimene, siis olgu ta kõikidest viimne ja kõikide teenija!”
\par 36 Ja ta võttis lapse ja pani selle nende keskele, ja kaisutas teda ja ütles neile:
\par 37 „Kes iganes ühe niisuguseist lapsist vastu võtab minu nimel, see võtab mind vastu, ja kes iganes mind vastu võtab, see ei võta mind vastu, vaid teda, kes mind on läkitanud!”

\section*{Sallivuse käsk}

\par 38 Johannes ütles temale: „Õpetaja, me nägime sinu nimel kurje vaime välja ajavat meest, kes meid ei järgi, ja me keelasime teda, sest ta ei järgi meid!”
\par 39 Aga Jeesus ütles: „Ärge keelake teda! Sest kedagi ei ole, kes teeb vägeva teo minu nimel ja suudab sedamaid rääkida minust kurja.
\par 40 Sest kes ei ole meie vastu, see on meie poolt!
\par 41 Sest kes teid joodab karikatäie veega minu nimel, sellepärast et te olete Kristuse omad, tõesti, ma ütlen teile, see ei jää ilma oma palgast!

\section*{Hoiatus pahanduste eest}

\par 42 Ja kes pahandab üht neist pisukesist, kes minusse usuvad, sellele oleks parem, et veskikivi temale kaela pandaks ja ta merre visataks!
\par 43 Ja kui su käsi sind pahandab, raiu ta maha; parem on sul vigasena minna ellu kui et sul on kaks kätt ja pead minema põrgusse, kustumatusse tulle,
\par 44 [kus nende uss ei sure ja tuli ei kustu!]
\par 45 Ja kui su jalg sind pahandab, raiu ta maha; parem on sul jalutuna minna ellu kui et sul on kaks jalga ja sind heidetakse põrgusse,
\par 46 [kus nende uss ei sure ja tuli ei kustu!]
\par 47 Ja kui su silm sind pahandab, kisu ta välja; parem on sul ühe silmaga minna Jumala riiki kui et sul on kaks silma ja sind heidetakse põrgusse,
\par 48 kus nende uss ei sure ja tuli ei kustu!

\section*{Soola vajadusest}

\par 49 Sest igaüht peab tulega soolatama!
\par 50 Sool on hea; aga kui sool tuimaks läheb, millega te teete ta maitsekaks? Olgu teil enestes soola ja pidage rahu isekeskis!”


\chapter{10}

\section*{Abielu lahutamisest}

\par 1 Ja tema läks sealt teele ja tuli Judea aladele ja maale sealpool Jordanit. Ja rahvast voolas jälle kokku tema juurde, ja ta õpetas neid jälle nagu tavaliselt.
\par 2 Ja varisere tuli tema juurde, ja teda kiusates küsisid nad temalt: „Kas on mehel luba oma naist enesest lahutada?”
\par 3 Tema vastas ja ütles neile: „Mida Mooses teid käskis?”
\par 4 Nemad ütlesid: „Mooses andis loa kirjutada lahutuskiri ja lahutada!”
\par 5 Aga Jeesus kostis ja ütles: „Teie südame kanguse pärast ta kirjutas teile selle käsu,
\par 6 ent loomise algusest Jumal lõi nad meheks ja naiseks.
\par 7 Sellepärast loobub inimene oma isast ja emast,
\par 8 ja need kaks, mees ja naine, saavad üheks lihaks. Nii ei ole nad enam kaks, vaid üks liha.
\par 9 Mis nüüd Jumal on ühte pannud, seda inimene ärgu lahutagu!”
\par 10 Ja kodus küsisid tema jüngrid jälle temalt sama asja pärast.
\par 11 Ja ta ütles neile: „Kes iganes oma naise enesest lahutab ja võtab teise, rikub abielu tema vastu.
\par 12 Ja kui naine enese lahutab oma mehest ja läheb teisele, siis ta rikub abielu!”

\section*{Jeesus õnnistab lapsi}

\par 13 Ja nad tõid tema juurde lapsukesi, et ta neid puudutaks. Aga jüngrid sõitlesid toojaid.
\par 14 Kui Jeesus seda nägi, läks ta tusaseks ja ütles neile: „Laske lapsukesed minu juurde tulla, ärge keelake neid, sest niisuguste päralt on Jumala riik!
\par 15 Tõesti, ma ütlen teile, kes Jumala riiki vastu ei võta nagu lapsuke, see ei pääse sinna sisse!”
\par 16 Ja ta kaisutas neid ja pani oma käed nende peale ning õnnistas neid.

\section*{Rikas noormees ja igavene elu}

\par 17 Ja kui ta ära läks teele, jooksis keegi ta juurde, langes põlvili ta ette ning küsis temalt: „Hea õpetaja, mis ma pean tegema, et ma igavese elu päriksin?”
\par 18 Aga Jeesus ütles talle: „Miks sa mind nimetad heaks? Keegi muu ei ole hea kui ainult Jumal.
\par 19 Käsud sa tead: sa ei tohi tappa; sa ei tohi abielu rikkuda; sa ei tohi varastada; sa ei tohi valet tunnistada; sa ei tohi kedagi petta; sa pead oma isa ja ema austama!”
\par 20 Aga ta kostis ja ütles temale: „Õpetaja seda kõike ma olen pidanud oma noorpõlvest alates!”
\par 21 Ja Jeesus vaatas tema peale ja tundis armastust ta vastu ning ütles talle: „Üht asja on sulle vaja: mine ja müü ära kõik, mis sul on, ja anna vaestele, ja sul on siis varandus taevas; ja tule ning järgi mind!”
\par 22 Tema aga jäi murelikuks selle sõna pärast ja läks ära kurva meelega, sest tal oli palju vara.

\section*{Rikkal on raske pääseda Jumala riiki}

\par 23 Siis Jeesus vaatas ümber ja ütles oma jüngritele: „Kui raske on neil, kellel on palju vara, sisse minna Jumala riiki!”
\par 24 Aga jüngrid kohkusid tema sõnade pärast. Ent Jeesus kostis jälle ja ütles neile: „Lapsed, kui raske on sisse minna Jumala riiki!
\par 25 Hõlpsam on kaamelil minna läbi nõelasilma kui rikkal pääseda Jumala riiki!”
\par 26 Aga nemad hämmastusid veel enam ja ütlesid isekeskis: „Kes siis võib õndsaks saada?”
\par 27 Jeesus vaatas nendele ja ütles: „Inimestel on see võimatu, aga mitte Jumalal; sest Jumalal on kõik võimalik!”
\par 28 Peetrus hakkas temale ütlema: „Vaata, meie oleme jätnud maha kõik ja oleme järginud sind!”
\par 29 Jeesus ütles: „Tõesti, ma ütlen teile, ei ole kedagi, kes maja või vennad või õed või ema või isa või naise või põllud on maha jätnud minu ja armuõpetuse pärast
\par 30 ega saaks sajakordselt nüüd sellel ajal maju ja vendi ja õdesid ja emasid ja lapsi ja põlde keset tagakiusamisi ja tulevases maailma-ajastus igavest elu!
\par 31 Aga paljud esimesed saavad viimasteks ja viimased esimesiks!”

\section*{Jeesus kuulutab ette kolmandat korda oma surma ja ülestõusmist}

\par 32 Aga nad olid teel minemas Jeruusalemma; ja Jeesus käis nende ees, ja nad kohkusid, aga järelkäijad kartsid. Ja ta võttis jälle need kaksteist enese juurde ja hakkas neile rääkima, mis temale peab sündima:
\par 33 „Vaata, me läheme üles Jeruusalemma; ja Inimese Poeg antakse ülempreestrite ja kirjatundjate kätte, ja nad mõistavad ta surma ja annavad ta paganate kätte.
\par 34 Ja need pilkavad teda ja sülitavad ta peale ja peksavad teda rooskadega ja tapavad tema, ja kolme päeva pärast tõuseb ta üles!”

\section*{Sebedeuse poegade igatsus au järele}

\par 35 Siis tulid tema juurde Jakoobus ja Johannes, mõlemad Sebedeuse pojad, ja ütlesid: „Õpetaja, me tahame, et sa meile teeksid, mis me sinult palume!”
\par 36 Aga tema ütles neile: „Mida te tahate, et ma teile teeksin?”
\par 37 Nemad ütlesid talle: „Anna meile, et me üks sinu paremal ja teine sinu vasakul pool saaksime istuda sinu auhiilguses!”
\par 38 Aga Jeesus ütles neile: „Te ei tea, mida te palute. Kas te võite juua seda karikat, mida mina joon, või endid lasta ristida selle ristimisega, millega mind ristitakse?”
\par 39 Aga nad ütlesid talle: „Võime küll!” Jeesus ütles neile: „Seda karikat, mida mina joon, peate te jooma, ja selle ristimisega, millega mind ristitakse, peab teid ristitama,
\par 40 kuid istuda mu paremal ja vasakul pool ei ole minu anda, vaid see antakse neile, kellele see on valmistatud!”

\section*{Inimese tõeline suurus}

\par 41 Kui need kümme seda kuulsid, said nad pahaseks Jakoobuse ja Johannese peale.
\par 42 Ja Jeesus kutsus nad enese juurde ja ütles neile: „Te teate, et need, keda arvatakse rahvaste ülemaiks, valitsevad nende üle ja rahvaste suured tarvitavad võimust nende vastu.
\par 43 Aga nõnda ärgu olgu teie seas, vaid kes teie seast tahab saada suureks, olgu teie teenija,
\par 44 ja kes teie seast tahab olla kõige ülem, olgu kõikide ori!
\par 45 Sest Inimese Poeg ei ole tulnud ennast laskma teenida, vaid ise teenima ja andma oma hinge lunaks paljude eest!”

\section*{Pime Bartimeus tehakse nägijaks}

\par 46 Ja nad tulid Jeerikosse. Ja kui ta Jeerikost välja läks ja tema jüngrid ja suur hulk rahvast, siis istus Timeuse poeg, pime Bartimeus, tee ääres ja kerjas.
\par 47 Ja kui ta kuulis, et Jeesus Naatsaretlane on seal, hakkas ta kisendama ja ütlema: „Jeesus, Taaveti poeg, halasta minu peale!”
\par 48 Ja paljud sõitlesid teda, et ta jääks vait; kuid ta kisendas veel enam: „Taaveti poeg, halasta minu peale!”
\par 49 Ja Jeesus jäi seisma ja ütles: „Kutsuge ta siia!” Ja nad kutsusid pimedat ja ütlesid temale: „Ole julge, tõuse üles, tema kutsub sind!”
\par 50 Ent tema viskas kuue seljast, tõusis püsti ja tuli Jeesuse juurde.
\par 51 Ja Jeesus hakkas temaga rääkima ning ütles: „Mis sa tahad, et ma sulle teeksin?” Ent pime ütles talle: „Rabunii, et ma jälle näeksin!”
\par 52 Jeesus ütles talle: „Mine, su usk on sind aidanud!” Ja sedamaid nägi ta jälle ja järgis teda teekonnal.


\chapter{11}

\section*{Jeesus tuleb Jeruusalemma kui Kuningas}

\par 1 Ja kui nad Jeruusalemma ligi said, Betfage ja Betaania poole Õlimäe juurde, läkitab tema kaks oma jüngreist
\par 2 ja ütleb neile: „Minge alevisse, mis teie ees on, ja kohe, kui te sinna saate, leiate kinniseotud sälu, kelle seljas ükski inimene pole istunud; päästke see lahti ja tooge siia!
\par 3 Ja kui keegi teile ütleb: miks te seda teete? siis öelge: Issandal on teda tarvis! Siis ta lähetab selle sedamaid siia.”
\par 4 Ja nad läksid ja leidsid sälu ukse äärde kinniseotuna väljas teeharul ja päästsid ta lahti.
\par 5 Ja mõned neist, kes seal seisid, ütlesid neile: „Mis te teete, et te sälu lahti päästate?”
\par 6 Aga nemad ütlesid neile nõnda, kuidas Jeesus oli käskinud. Siis lasti nad tulema.
\par 7 Ja nad tõid sälu Jeesuse juurde ja panid oma riided selle peale, ja tema istus ta selga.
\par 8 Ja paljud laotasid oma riided tee peale, aga teised oksi, mida nad raiusid väljadelt.
\par 9 Ja need, kes käisid ees ja taga, hüüdsid: „Hoosianna! Õnnistatud olgu, kes tuleb Issanda nimel!
\par 10 Õnnistatud olgu meie isa Taaveti kuningriik! Hoosianna kõrges!”
\par 11 Ja ta tuli Jeruusalemma pühakotta. Ja kui ta kõike oli vaadelnud ja hiline aeg juba käes oli, läks ta välja Betaaniasse nende kaheteistkümnega.

\section*{Jeesus neab viljatu viigipuu}

\par 12 Ja kui nad teisel päeval Betaaniast väljusid, oli tal nälg.
\par 13 Ja nähes kaugelt viigipuud lehis, läks ta vaatama, kas ta sellelt midagi leiab. Aga sinna juurde jõudes ei leidnud ta midagi muud kui lehti; sest ei olnud veel viigimarja-aeg.
\par 14 Ja Jeesus hakkas kõnelema ning ütles: „Ärgu iialgi enam sinust ükski vilja söögu!” Ja tema jüngrid kuulsid seda.

\section*{Jeesus puhastab templi}

\par 15 Ja nad tulid Jeruusalemma. Ja kui Jeesus oli pühakotta sisse läinud, hakkas ta välja ajama neid, kes müüsid ja ostsid pühakojas; ja rahavahetajate lauad ja tuvimüüjate istmed ta lükkas kummuli
\par 16 ega sallinud, et keegi kandis astjaid pühakojast läbi.
\par 17 Ja ta õpetas ja ütles neile: „Eks ole kirjutatud: minu koda peab hüütama palvekojaks kõigile rahvaile? Kuid teie olete selle teinud röövliauguks!”
\par 18 Ja kirjatundjad ja ülempreestrid kuulsid seda ja otsisid nõu, kuidas teda hukata. Kuid nad kartsid teda, sest kõik rahvas oli hämmastuses tema õpetuse pärast.
\par 19 Ja kui õhtu tuli, läks ta linnast välja.

\section*{Usu jõud}

\par 20 Ja vara hommikul nägid nad mööda minnes viigipuu juureni ära kuivanud olevat.
\par 21 Siis Peetrusele tuli meelde Issanda sõna ning ta ütles temale: „Rabi, vaata, viigipuu, mille sa needsid, on ära kuivanud!”
\par 22 Ja Jeesus vastas ning ütles neile: „Olgu teil usku Jumalasse!
\par 23 Tõesti, ma ütlen teile, kui keegi ütleb sellele mäele: tõuse paigast ja lange merre! ja ei mõtle kaksipidi oma südames, vaid usub, et see, mis ta ütleb, sünnib, siis see saab temale!
\par 24 Sellepärast ma ütlen teile: kõik, mida te palute ja anute, uskuge, et te seda saate, siis see saab teile!
\par 25 Ja kui te seisate ja palvetate, siis andke andeks, kui teil midagi on kellegi vastu, et ka teie Isa, kes on taevas, annaks teile andeks teie üleastumised.
\par 26 [Aga kui te andeks ei anna, siis ei anna ka teie Isa, kes on taevas, teie üleastumisi andeks!]

\section*{Küsimus Jeesuse meelevallast}

\par 27 Ja nad tulid jälle Jeruusalemma. Ja kui ta pühakojas kõndis, tulid ta juurde ülempreestrid ja kirjatundjad ja vanemad
\par 28 ja ütlesid talle: „Missuguse meelevallaga sa seda teed? Ja kes on sulle andnud selle meelevalla?”
\par 29 Aga Jeesus ütles neile: „Ma tahan ka teilt küsida üht asja; vastake mulle ja mina ütlen teile, missuguse meelevallaga ma seda teen.
\par 30 Kas Johannese ristimine oli taevast või inimestest? Vastake mulle!”
\par 31 Nad arutasid isekeskis ja ütlesid: „Kui me ütleme taevast, siis ta ütleb: mispärast te siis teda ei uskunud?
\par 32 Või ütleme: inimestest?” Siis oli neil hirm rahva ees, sest kõik pidasid Johannest tõesti prohvetiks.
\par 33 Ja nad vastasid Jeesusele ning ütlesid: „Me ei tea!” Siis ütles Jeesus neile: „Ega minagi teile ütle, missuguse meelevallaga ma neid asju teen!”


\chapter{12}

\section*{Tähendamissõna viinamäest}

\par 1 Ja ta hakkas neile rääkima tähendamissõnadega: „Inimene istutas viinamäe ta tegi aia ümber ja kaevas sinna surutõrre ja ehitas torni ja andis selle rendile aednike kätte, ja läks ise võõrale maale.
\par 2 Ja ta läkitas omal ajal aednike juurde sulase aednikelt vastu võtma viinamäe vilja.
\par 3 Aga nad võtsid ja peksid teda ja läkitasid ta minema tühje käsi.
\par 4 Ja ta läkitas jälle teise sulase nende juurde. Seda nad lõid pähe ja teotasid.
\par 5 Siis ta läkitas teise. Ja selle nad tapsid. Ja veel muid; ühtesid nad peksid, teised nad tapsid.
\par 6 Tal oli veel üks, ta armas poeg. Selle ta läkitas viimseks nende juurde, öeldes: ehk nad häbenevad mu poega?
\par 7 Aga aednikud ütlesid üksteisele: see ongi pärija. Tulge, tapkem ta ära, siis jääb pärand meile.
\par 8 Ja nad võtsid ja tapsid tema ja viskasid ta viinamäest välja.
\par 9 Mis teeb nüüd viinamäe isand? Ta tuleb ja hukkab aednikud ja annab viinamäe teiste kätte.
\par 10 Kas te pole lugenud seda kirja: kivi, mille hooneehitajad kõrvale heitsid, on saanud nurgakiviks?
\par 11 See tuli Issandalt ja on imeasi meie silmis!”
\par 12 Ja nad püüdsid teda kinni võtta, kuid kartsid rahvast; sest nad mõistsid, et ta selle tähendamissõna oli öelnud nende kohta. Ja nad jätsid ta rahule ja läksid ära.

\section*{Maksu maksmisest keisrile}

\par 13 Ja nad läkitavad tema juurde mõningaid varisere ja Heroodese käsilasi, et need tabaksid teda sõnast.
\par 14 Ja need tulid ning ütlesid talle: „Õpetaja, me teame, et sa oled tõearmastaja ega hooli kellestki, sest sa ei vaata inimese isikule, vaid õpetad Jumala teed tões. Kas on vaja anda keisrile maksu või mitte? Kas anname või ei anna?”
\par 15 Tema aga, nähes nende silmakirjalikkust, ütles neile: „Miks te mind kiusate? Tooge mulle näha teenariraha!”
\par 16 Nemad tõid. Siis ta ütleb neile: „Kelle kuju ja pealkiri see on?” Nemad ütlesid temale: „Keisri!”
\par 17 Jeesus aga ütles neile: „Andke keisrile, mis kuulub keisrile, ja Jumalale, mis kuulub Jumalale!” Ja nemad panid teda imeks.

\section*{Surnute ülestõusmise küsimus}

\par 18 Ja tema juurde tuleb sadusere, kes ütlevad, et ei ole ülestõusmist, ja nad küsivad temalt, öeldes:
\par 19 „Õpetaja, Mooses on meile kirjutanud, et kui kellegi vend sureb ja jätab naise järele, aga ei jäta lapsi, siis võtku ta vend see naine ja soetagu oma vennale järglane.
\par 20 Oli seitse venda; ja esimene võttis naise, ja kui ta suri, ei jätnud ta järglast järele.
\par 21 Ja teine võttis naise ja suri, ja temagi ei jätnud järglast, ja samuti kolmas.
\par 22 Ja seitsmest ei jätnud keegi järglast järele. Kõige viimaks suri ka naine.
\par 23 Kui nad nüüd ülestõusmises üles tõusevad, nende seast kelle naine ta siis on? Sest ta on olnud naiseks seitsmele!”
\par 24 Jeesus ütles neile: „Kas te mitte ei eksi, sellepärast et te ei tunne pühi kirju ega Jumala väge?
\par 25 Sest kui surnuist üles tõustakse, ei võeta naisi ega minda mehele, vaid ollakse nagu inglid, kes on taevas.
\par 26 Aga surnute kohta, et need üles äratatakse, eks te ole lugenud Moosese raamatust kibuvitsapõõsa loos, kuidas Jumal temaga kõneles ning ütles: mina olen Aabrahami Jumal ja Iisaki Jumal ja Jaakobi Jumal!
\par 27 Ta ei ole mitte surnute Jumal, vaid elavate Jumal! Te eksite küll väga!”

\section*{Missugune käsk on suur?}

\par 28 Siis tuli ta juurde üks kirjatundjaist, kes oli kuulnud neid vaidlevat ja märkas, et ta neile oli hästi kostnud, ja küsis temalt: „Missugune käsk on kõige esimene?”
\par 29 Jeesus vastas talle: „Esimene on see: kuule Iisrael, Issand sinu Jumal, on ainus Issand!
\par 30 Ja sina armasta Issandat, oma Jumalat, kõigest oma südamest ja kõigest oma hingest ja kõigest oma meelest ja kõigest oma väest.
\par 31 Teine on see: armasta oma ligimest nagu iseennast. Neist suuremat muud käsku ei ole!”
\par 32 Ja kirjatundja ütles talle: „Õige küll, õpetaja, sina oled tõtt mööda öelnud, sest üksainus on olemas ja ei ole muud peale tema!
\par 33 Ja teda armastada kõigest südamest ja kõigest meelest ja kõigest väest ja ligimest armastada nagu iseennast, see on rohkem kui kõik põletusohvrid ja muud ohvrid!”
\par 34 Ja kui Jeesus nägi, et ta vastas mõistlikult, ütles ta temale: „Sina ei ole kaugel Jumala riigist!” Ja ükski ei julgenud temalt enam küsida.

\section*{Kelle poeg on Kristus?}

\par 35 Ja Jeesus kõneles edasi ning ütles pühakojas õpetades: „Kuidas kirjatundjad ütlevad, et Kristus on Taaveti poeg?
\par 36 Ütleb ju Taavet ise Püha Vaimu läbi: Issand ütles mu Issandale: istu mu paremale käele, kuni ma su vaenlased panen su jalgade alla!
\par 37 Taavet ise nimetab teda Issandaks. Kuidas ta siis on tema poeg?” Ja palju rahvast kuulas teda hea meelega.

\section*{Hoiatus kirjatundjate eest}

\par 38 Ja ta ütles oma õpetuses: „Hoiduge kirjatundjaist, kes tahavad käia pikis rüüdes ja armastavad teretust turgudel
\par 39 ja esimesi istmeid kogudusekodades ja ülemat paika lauas õhtusöömaaegadel, nende eest,
\par 40 kes söövad leskede hooned ja loevad silmakirjaks pikki palveid! Need saavad seda raskema hukatuse!”

\section*{Lesknaise ohvriand}

\par 41 Ja kui Jeesus ohvrirahakirstu kohal istus, vaatas ta, kuidas rahvas ohvriraha kirstu pani. Ja paljud rikkad panid sinna palju.
\par 42 Siis tuli üks vaene lesknaine ja pani sisse kaks leptonit, see on üks veering.
\par 43 Ja ta kutsus oma jüngrid enese juurde ning ütles neile: „Tõesti, ma ütlen teile, see vaene lesk pani rohkem ohvrikirstu kui kõik, kes panid!
\par 44 Sest need kõik panid oma küllusest, ent tema pani oma vaesusest kõik, mis tal oli, kogu oma elatise!”


\chapter{13}

\section*{Jeesus kuulutab ette templi hävitamist}

\par 1 Ja kui tema pühakojast väljus, ütles üks tema jüngreist temale: „Õpetaja, näe, millised kivid ja millised hooned need on!”
\par 2 Siis Jeesus vastas ja ütles talle: „Kas sa vaatad neid suuri hooneid? Ei jäeta siia mitte kivi kivi peale, mida maha ei kistaks!”

\section*{Tulevased õnnetused ja hädad}

\par 3 Ja kui ta istus Õlimäel pühakoja vastas, küsisid temalt Peetrus ja Jakoobus ja Johannes ja Andreas isepäinis:
\par 4 „Ütle meile, millal see sünnib ja mis on tunnuseks, kui see kõik hakkab täide minema?”
\par 5 Aga Jeesus hakkas neile rääkima: „Katsuge, et keegi teid ei eksitaks!
\par 6 Paljud tulevad minu nimel ning ütlevad: „Mina olen see!” ja eksitavad paljusid.
\par 7 Aga kui te kuulete sõdadest ja sõjasõnumeid, siis ärge ehmuge; see peab sündima, aga ots ei ole veel käes.
\par 8 Sest rahvas tõuseb rahva vastu ja riik riigi vastu ja maavärisemisi on mõnes paigas ja on näljahädad. Need on valude algus.
\par 9 Aga pidage silmas iseendid: teid antakse ära suurkohtute kätte ja kogudusekodades teid pekstakse, ja teid seatakse maavalitsejate ja kuningate ette minu pärast neile tunnistuseks.
\par 10 Ja kõigile rahvaile peab enne kuulutatama evangeeliumi.
\par 11 Aga kui nad teid viivad ja annavad kohtu kätte, ärge muretsege enneaegu, mida teil tuleb rääkida, vaid mis teile antakse sel tunnil, seda rääkige. Sest teie pole rääkijad, vaid Püha Vaim.
\par 12 Ja vend annab surma venna ja isa lapse, ja lapsed hakkavad vastu vanemaile ja saadavad nad surma.
\par 13 Ja teid vihatakse kõigilt poolt minu nime pärast. Aga kes otsani jääb püsima, see pääseb.
\par 14 Kui te siis näete hävituse koletist seisvat seal, kus see ei peaks seisma - kes seda loeb, see pangu tähele - siis põgenegu need, kes on Juudamaal, mägedele;
\par 15 kes on katusel, ärgu tulgu maha ja ärgu mingu sisse midagi võtma oma majast,
\par 16 ja kes on väljal, ärgu mingu tagasi võtma oma kuube.
\par 17 Aga häda neile, kes on käima peal, ja neile, kes imetavad neil päevil!
\par 18 Aga paluge, et teie põgenemine ei juhtuks talvel.
\par 19 Sest neil päevil on niisugune viletsus, millist ei ole olnud loodu algusest, mille Jumal on loonud, kuni siiamaani, ega tulegi.
\par 20 Ja kui Issand ei lühendaks neid päevi, ei pääseks ükski liha; aga äravalitute pärast, keda tema on ära valinud, on ta need päevad lühendanud.
\par 21 Ja kui siis keegi teile ütleb: „Vaata siin on Kristus, vaata seal!” ärge uskuge.
\par 22 Sest valekristusi ja valeprohveteid tõuseb, ja need teevad tunnustähti ja imesid, et eksitada, kui võimalik, ka äravalituid.
\par 23 Aga teie vaadake ette! Mina olen teile kõik ette öelnud.

\section*{Ajastu lõpp}

\par 24 Aga neil päevil pärast seda viletsuseaega pimeneb päike ja kuu ei anna oma valgust;
\par 25 ja taeva tähed peavad taevast maha langema ja vägesid, mis on taevas, kõigutatakse.
\par 26 Ja siis nähakse Inimese Poega tulevat pilvedes suure väe ja auhiilgusega.
\par 27 Ja siis ta läkitab inglid ja kogub kokku oma äravalitud neljast tuulest ilmamaa otsast taeva otsani.

\section*{Vajadus valvsuseks}

\par 28 Aga viigipuust õppige võrdumit: kui ta oksad on juba pungas ja ajavad lehti, siis te tunnete, et suvi on lähedal.
\par 29 Nõnda ka teie, kui te näete seda sündivat, teadke, et see on lähedal ukse ees.
\par 30 Tõesti, ma ütlen teile, selle põlve rahvas ei lõpe mitte ära, enne kui see kõik sünnib!
\par 31 Taevas ja maa hävivad, kuid minu sõnad ei hävi mitte!
\par 32 Aga sellest päevast või tunnist ei tea ükski, ei inglidki taevas ega Poeg, muud kui Isa üksi.
\par 33 Vaadake ette, olge valvel, sest te ei tea, millal see aeg on käes!
\par 34 Otsekui inimene, kes läks võõrale maale, jättis maha oma koja ja andis oma sulastele meelevalla, igaühele ta töö, ja käskis uksehoidjat valvata,
\par 35 nõnda valvake; sest te ei tea, millal kojaisand tuleb, kas õhtul või kesköö ajal, või kukelaulu ajal või vara hommikul,
\par 36 et ta ei tuleks äkitselt ja ei leiaks teid magavat.
\par 37 Aga mida ma ütlen teile, ütlen kõigile: valvake!”


\chapter{14}

\section*{Vandenõu Jeesuse tapmiseks}

\par 1 Aga kahe päeva pärast olid paasa- ja hapnemata leibade pühad, ja ülempreestrid ja kirjatundjad otsisid, kuidas kavalusega teda kinni võtta ja ära tappa.
\par 2 Sest nad ütlesid: „Mitte pühade ajal, et ei tõuseks rahva mässu.”

\section*{Jeesuse võidmine Betaanias}

\par 3 Ja kui ta istus lauas Betaanias, olles pidalitõbise Siimona kojas, tuli üks naine, kellel oli alabasterriist selge ning kalli nardisalviga. Ja ta murdis alabasterriista katki ja valas salvi tema pea peale.
\par 4 Aga seal olid mõned, kes pahameelega ütlesid üksteisele: „Mistarvis on see salvi raiskamine?
\par 5 Sest selle oleks võinud ära müüa enam kui kolmesaja teenari eest ja raha vaestele anda.” Ja nad sõitlesid naist.
\par 6 Aga Jeesus ütles: „Jätke ta rahule! Miks te teete talle südamevalu? Ta on teinud mulle heateo.
\par 7 Sest vaeseid on ikka teie juures ja, kui te tahate, võite neile head teha; aga mind ei ole teil mitte alati.
\par 8 Ta on teinud, mis ta võis. Ta on ennakult mu ihu võidnud matmiseks.
\par 9 Tõesti, ma ütlen teile, kus iganes evangeeliumi kuulutatakse kõiges maailmas, peab ka räägitama tema mälestuseks, mis tema on teinud.”

\section*{Juudas reedab Jeesuse}

\par 10 Ja Juudas Iskariot, üks neist kaheteistkümnest, läks ülempreestrite juurde, et teda neile ära anda.
\par 11 Kui nad seda kuulsid, said nad rõõmsaks ja lubasid anda talle raha. Ja ta hakkas otsima, kuidas ta saaks teda ära anda parajal ajal.

\section*{Jeesuse viimne paasatalle söömine}

\par 12 Ja hapnemata leibade esimesel päeval, kui paasatalle veristati, ütlesid tema jüngrid talle: „Kuhu sa tahad, et me läheme ja teeme ettevalmistuse sulle paasatalle söömiseks?”
\par 13 Ja tema läkitas kaks oma jüngreist ning ütles neile: „Minge linna, ja teile tuleb vastu inimene, kes kannab veekruusi; järgige teda.
\par 14 Ja seal, kuhu ta sisse läheb, öelge majaisandale: „Õpetaja küsib: kus on võõrastetuba, milles ma võin oma jüngritega süüa paasatalle?
\par 15 Ja tema näitab teile suure ülemise toa, mis on valmis korraldatud. Seal valmistage meile.”
\par 16 Ja tema jüngrid läksid ära ja tulid linna ning leidsid nõnda, nagu ta neile oli öelnud. Ja nad valmistasid paasatalle.
\par 17 Aga kui õhtu jõudis, tuleb tema nende kaheteistkümnega.
\par 18 Ja kui nad lauas istusid ja sõid, ütles Jeesus: „Tõesti ma ütlen teile, üks teie seast annab mu ära, üks, kes minuga ühes sööb!”
\par 19 Siis nad hakkasid kurvaks minema ja üksteise järele temale ütlema: „Ega ometi mina?”
\par 20 Tema ütles neile: „Üks neist kaheteistkümnest, kes ühes minuga pistab käe vaagnasse, annab mind ära.
\par 21 Inimese Poeg läheb küll ära, nõnda nagu temast on kirjutatud, kuid häda sellele inimesele, kelle läbi Inimese Poeg ära antakse. Hea oleks niisugusele inimesele, kui ta ei oleks sündinud!”
\par 22 Ja kui nad sõid, võttis Jeesus leiva, õnnistas, murdis ja andis neile ning ütles: „Võtke! See on minu ihu.”
\par 23 Ja ta võttis karika, tänas ja andis neile; ja nad kõik jõid selle seest.
\par 24 Ja ta ütles neile: „See on minu veri, lepingu veri, mis ära valatakse paljude eest.
\par 25 Tõesti ma ütlen teile: mina ei joo enam viinapuu viljast kuni selle päevani, mil ma joon uut Jumala riigis!”
\par 26 Ja kui nad kiituslaulu olid laulnud, läksid nad välja Õlimäele.

\section*{Jeesus kuulutab ette, et jüngrid jätavad ta maha}

\par 27 Ja Jeesus ütleb neile: „Te kõik taganete minust, sest kirjutatud on: ma löön karjast ja lambad pillutatakse!
\par 28 Aga pärast oma ülestõusmist ma lähen teie eele Galileasse.”
\par 29 Siis Peetrus ütleb temale: „Kui ka kõik sinust taganevad, siis mina mitte!”
\par 30 Jeesus ütles talle: „Tõesti ma ütlen sulle, täna, sellel ööl, enne kui kukk on kaks korda laulnud, salgad sina mind kolm korda!”
\par 31 Aga tema ütles veel eriti: „Kui ma sinuga peaksin ka surema, ei salga ma sind mitte!” Ja samuti ütlesid ka kõik.

\section*{Jeesus Ketsemanis}

\par 32 Ja nad tulid paika, mille nimi on Ketsemani, ja tema ütles oma jüngritele: „Istuge siin niikaua kui ma palvetan!”
\par 33 Ja ta võttis enesega ühes Peetruse, Jakoobuse ja Johannese ja hakkas vabisema ning suurt õudsust tundma,
\par 34 ja ta ütles neile: „Minu hing on väga kurb surmani, jääge siia ja valvake!”
\par 35 Ja ta läks pisut eemale ja heitis maha maa peale ning palus, et see tund, kui on võimalik, läheks temast mööda,
\par 36 ja ta ütles: „Aba Isa! Sinul on kõik võimalik; võta see karikas minult ära! Siiski mitte, mida mina tahan, vaid mida sina tahad!”
\par 37 Ja ta tuleb ja leiab nad magamast, ja ütleb Peetrusele: „Siimon, kas sa magad? Kas sa ei suuda ühtki tundi valvata?
\par 38 Valvake ja paluge, et te ei satuks kiusatusse! Vaim on küll valmis, kuid liha on nõder!”
\par 39 Ja ta läks jälle ära ja palvetas sama palvet.
\par 40 Ja ta tuli tagasi ja leidis nad taas magamast, sest nende silmad olid rasked unest, ja nad ei teadnud, mida temale vastata.
\par 41 Ja ta tuli kolmandat puhku ja ütles neile: „Te ikka veel magate ja puhkate; aeg on möödas, tund on tulnud! Vaata, Inimese Poeg antakse patuste kätte!
\par 42 Tõuske üles, lähme! Vaata, see, kes mu ära annab, on lähedal!”

\section*{Jeesuse vangistamine}

\par 43 Ja sedamaid, kui tema alles rääkis, saabus Juudas, üks neist kaheteistkümnest, ja ühes temaga jõuk mõõkade ja nuiadega ülempreestrite ja kirjatundjate ja vanemate poolt.
\par 44 Aga see, kes tema ära andis, oli neile andnud tähe ning öelnud: „Keda ma suudlen, see ta on; tema võtke kinni ja viige ta kindla valve all minema!”
\par 45 Ja kui ta sinna tuli, astus ta kohe tema juurde ja ütles: „Rabi!” ja andis temale suud.
\par 46 Aga nemad pistsid oma käed tema külge ja võtsid ta kinni.
\par 47 Siis keegi juuresseisjaist tõmbas mõõga ja lõi ülempreestri sulast ning raius ta kõrva ära.
\par 48 Ja Jeesus kostis ja ütles neile: „Otsekui röövli vastu te olete väljunud mõõkade ja nuiadega mind kinni võtma!
\par 49 Ma olen olnud iga päev teie juures pühakojas õpetamas, ja te ei ole mind mitte kinni võtnud; kuid see sünnib, et kirjad täide läheksid!”
\par 50 Siis kõik jätsid ta maha ja põgenesid.

\section*{Põgenev noormees}

\par 51 Ja üks noor mees järgis teda; sel oli linane riie seljas paljal ihul; ja nad võtsid ta kinni.
\par 52 Aga tema jättis linase riide maha ja põgenes ära alasti.

\section*{Jeesus ülempreestri ees}

\par 53 Ja nad viisid Jeesuse ülempreestri juurde, ja sinna tulid kokku kõik ülempreestrid ja vanemad ja kirjatundjad.
\par 54 Ja Peetrus järgis teda kaugelt kuni ülempreestri õue. Ja ta istus ühes sulastega ja soojendas ennast tule ääres.
\par 55 Aga ülempreestrid ja kogu Suurkohus otsisid Jeesuse vastu tunnistust, et saata teda surma, kuid ei leidnud mitte.
\par 56 Sest paljud tunnistasid valet tema peale, aga tunnistused ei läinud ühte.
\par 57 Seal tõusid mõned ning tunnistasid valet tema peale, öeldes:
\par 58 „Me oleme kuulnud teda ütlevat: mina kisun selle kätega tehtud templi maha ja ehitan kolme päevaga teise, mis ei ole kätega tehtud!”
\par 59 Ja nõndagi ei läinud nende tunnistus ühte.
\par 60 Siis ülempreester astus keskele ja küsis Jeesuselt ning ütles: „Eks sa kosta midagi selle vastu, mida need tunnistavad sinu kohta?”
\par 61 Aga tema jäi vait ega vastanud midagi. Taas küsis talt ülempreester ja ütles temale: „Kas sina oled Kristus, Kiidetava Poeg?”
\par 62 Ja Jeesus ütles: „Mina olen see, ja te peate nägema Inimese Poja istuvat Jumala väe paremal poolel ja tulevat taeva pilvedega.”
\par 63 Siis rebestas ülempreester oma kuue ja ütles: „Mis meil veel on vaja tunnistajaid?
\par 64 Te olete kuulnud jumalapilget! Mis te arvate?” Aga nad kõik otsustasid, et ta on surma väärt.
\par 65 Ja mõned hakkasid ta peale sülitama ja ta nägu kinni katma ja teda rusikatega lööma vastu kõrvu ja temale ütlema: „Kuuluta kui prohvet!” Ja sulased lõid teda keppidega.

\section*{Peetrus salgab Jeesuse}

\par 66 Ja kui Peetrus oli all õues, tuli üks ülempreestri tüdrukuist,
\par 67 ja nähes Peetrust end soojendamas, vaatas ta temale otsa ja ütles: „Ka sina olid ühes selle Naatsareti Jeesusega!”
\par 68 Aga tema salgas, öeldes: „Ei ma tea ega saa aru, mida sa räägid!” Ja ta läks välja eesõue. Ja kukk laulis.
\par 69 Ja nähes teda seal, hakkas tüdruk jälle ütlema neile, kes seal seisid: „See on üks nende seast!”
\par 70 Aga ta salgas taas. Ja vähe aja pärast ütlesid juuresseisjad jälle Peetrusele: „Tõesti sa oled nende seast; sest sa oled galilealane!”
\par 71 Aga tema hakkas needma ja vanduma: „Ei ma tunne seda inimest, kellest te räägite!”
\par 72 Ja kohe laulis kukk teist korda. Siis Peetrusele tuli meelde sõna, mis Jeesus temale oli öelnud: „Enne kui kukk on kaks korda laulnud, salgad sa mind kolm korda!” Ja ta ruttas välja ja nuttis.


\chapter{15}

\section*{Jeesus maavalitseja Pilaatuse ees}

\par 1 Ja kohe vara hommikul pidasid ülempreestrid ühes vanemate ja kirjatundjatega ja kõik Suurkohus nõu ja sidusid Jeesuse kinni ning viisid tema ära ja andsid ta Pilaatuse kätte.
\par 2 Ja Pilaatus küsis temalt: „Oled sa juutide kuningas?” Tema vastas ning ütles temale: „Jah, olen!”
\par 3 Ja ülempreestrid kaebasid palju tema peale.
\par 4 Aga Pilaatus küsis talt jälle ning ütles: „Kas sa midagi ei vasta? Vaata, kui raskesti nad sind süüdistavad!”
\par 5 Kuid Jeesus ei vastanud enam midagi, nõnda et Pilaatus imestas.
\par 6 Aga igaks pühaks laskis ta neile vabaks ühe vangi, keda nad talt palusid.
\par 7 Ja seal oli mees nimega Barabas, vangi pandud ühes mässulistega, kes olid mässus toime pannud mõrva.
\par 8 Ja rahvas läks üles ja hakkas paluma teha nii, nagu ta ikka neile oli teinud.
\par 9 Aga Pilaatus vastas neile ning ütles: „Kas tahate, et ma teile lasen vabaks juutide kuninga?”
\par 10 Sest ta teadis, et ülempreestrid tema olid andnud ta kätte kadedusest.
\par 11 Aga ülempreestrid kihutasid rahvast, et ta neile laseks vabaks pigemini Barabase.
\par 12 Siis vastas Pilaatus jälle ning ütles neile: „Mis ma siis teen temaga, keda te nimetate juutide kuningaks?”
\par 13 Aga nemad kisendasid taas: „Löö ta risti!”
\par 14 Pilaatus ütles neile: „Mis kurja ta siis on teinud?” Aga nad kisendasid veel enam: „Löö ta risti!”
\par 15 Ent Pilaatus, tahtes teha rahva meele järgi, andis neile vabaks Barabase, aga Jeesust ta laskis rooskadega peksta, et teda risti lüüa.

\section*{Sõjamehed mõnitavad Jeesust}

\par 16 Siis viisid sõjamehed ta siseõue, see on kohtukotta, ja kutsusid kokku terve sõdurite salga.
\par 17 Ja nad riietasid ta purpurkuuega ja punusid kibuvitsakrooni ning panid temale pähe
\par 18 ja hakkasid teda teretama: „Tere, juutide kuningas!”
\par 19 Ja nad lõid teda pähe pillirooga ja sülitasid tema peale, heitsid põlvili maha ja kummardasid teda.
\par 20 Ja kui nad teda olid naernud, võtsid nad purpurkuue tema seljast ära ja panid tema omad riided temale selga ja viisid ta välja risti löödavaks.

\section*{Jeesuse ristilöömine}

\par 21 Ja nad sundisid üht möödaminevat meest, Siimonat Küreenest, kes tuli väljalt, Aleksandrose ja Ruufuse isa, kandma tema risti.
\par 22 Ja nad viisid ta Kolgata paika, see on meie keeli: Pealae ase.
\par 23 Ja nad pakkusid temale juua mürriga segatud viina. Kuid tema ei võtnud.
\par 24 Ja nad löövad ta risti ja jagavad ta riided, heites liisku selle kohta, mis keegi saab.
\par 25 Ent see oli kolmas tund, kui nad ta risti lõid.
\par 26 Ja pealkirjaks oli kirjutatud tema süü: „Juutide kuningas!”
\par 27 Ka lõid nad ühes temaga risti kaks röövlit, ühe paremale ja teise vasakule poolele.
\par 28 [Siis läks täide Kiri, mis ütleb: „Ja teda arvati üleastujate hulka!”]
\par 29 Ja möödaminejad pilkasid teda, vangutasid pead ja ütlesid: „Oh sind, kes sa maha kisud templi ja ehitad üles kolme päevaga,
\par 30 aita iseennast ja astu ristilt maha!”
\par 31 Samuti naersid teda ka ülempreestrid isekeskis ühes kirjatundjatega ja ütlesid: „Muid ta on aidanud, iseennast ta ei saa aidata!
\par 32 Kristus, Iisraeli kuningas, astugu nüüd ristilt maha, et me näeksime ja usuksime!” Ja need, kes olid ühes temaga risti löödud, teotasid teda.

\section*{Jeesuse surm}

\par 33 Ja kuuendal tunnil tekkis pimedus üle kogu maa ja kestis üheksanda tunnini.
\par 34 Ja üheksandal tunnil kisendas Jeesus suure häälega ning ütles: „Eloii, Eloii, lamaa sabahtaani!” see on meie keeli: mu Jumal, mu Jumal, miks sa mind maha jätsid?
\par 35 Kui mõned juuresseisjaist seda kuulsid, ütlesid nad: „Vaata, ta kutsub Eelijat!”
\par 36 Siis jooksis keegi ja täitis käsna äädikaga, pistis selle pilliroo otsa ja jootis teda ning ütles: „Oodake, saame näha, kas Eelija tuleb teda maha võtma!”
\par 37 Aga Jeesus kisendas suure häälega ja heitis hinge.
\par 38 Ja templi eesriie kärises lõhki kahest servast, ülemisest alamani!
\par 39 Aga kui pealik, kes tema ees seisis, nägi, et ta nõnda kisendades hinge heitis, ütles ta: „Tõesti, see inimene oli Jumala Poeg!”
\par 40 Seal oli ka naisi kaugelt vaatamas, nende seas ka Maarja Magdaleena ja Maarja, noorema Jakoobuse ja Joosese ema, ja Saloome,
\par 41 kes olid teda järginud ja teda teeninud, kui ta Galileamaal oli, ja palju muid naisi, kes ühes temaga olid läinud üles Jeruusalemma.

\section*{Jeesuse matmine}

\par 42 Aga kui päev juba oli jõudmas õhtule, kuna käes oli pühade valmistuspäev, see on hingamispäeva eelpäev,
\par 43 siis tuli Joosep Arimaatiast, lugupeetud kohtunõunik, kes ka ise ootas Jumala riiki, ja julges minna Pilaatuse juurde ja paluda enesele Jeesuse ihu.
\par 44 Siis Pilaatus imestas, et ta juba oli surnud, ja kutsus enese juurde pealiku ning küsis temalt, kas ta ammu suri.
\par 45 Ja kui ta pealikult oli saanud teate, andis ta surnukeha Joosepile.
\par 46 Tema ostis lõuendit, võttis ta keha ning mähkis ta lõuendisse ja asetas ta hauda, mis oli raiutud kaljusse, ja veeretas kivi uksele.
\par 47 Aga Maarja Magdaleena ja Maarja, Joosese ema, vaatasid, kuhu ta asetati.


\chapter{16}

\section*{Jeesuse ülestõusmine}

\par 1 Ja kui hingamispäev oli möödunud, ostsid Maarja Magdaleena ja Maarja, Jakoobuse ema, ja Saloome lõhnaaineid, et minna teda võidma.
\par 2 Ja nädala esimesel päeval tulid nad väga vara päeva tõustes hauale
\par 3 ja ütlesid isekeskis: „Kes veeretab meile ära kivi haua ukselt?”
\par 4 Ja silmi tõstes nad nägid, et kivi oli ära veeretatud; aga see oli väga suur.
\par 5 Ja nad läksid hauakoopasse ja nägid paremal poolel noore mehe istuvat, pikk valge rüü seljas. Ja nad kohkusid väga.
\par 6 Aga tema ütles neile: „Ärge kohkuge! Te otsite Jeesust Naatsaretlast, kes oli risti löödud; ta on üles tõusnud, teda ei ole siin. Ennäe aset, kuhu nad ta panid!
\par 7 Ent minge öelge tema jüngritele, ka Peetrusele, et ta läheb teie eele Galileasse; seal te saate teda näha, nõnda nagu ta teile ütles.”
\par 8 Nad tulid välja, jooksid haua juurest ära, sest neid oli vallanud väristus ja hämmastus; ja nad ei öelnud kellelegi midagi sest nad kartsid.

\section*{Jeesuse uued ilmumised}

\par 9 [Aga kui tema nädala esimesel päeval vara oli üles tõusnud, ilmus ta esiti Maarja Magdaleenale, kellest tema oli välja ajanud seitse kurja vaimu.
\par 10 Tema läks ja teatas neile, kes Jeesusega olid olnud, ja leinasid ning nutsid.
\par 11 Ja kui need kuulsid, et ta elab ja Maarja Magdaleena olevat teda näinud, ei uskunud nad.
\par 12 Selle järel ta ilmus teisel näol kahele nende seast, kui need olid teel maale minemas.
\par 13 Ja ka need läksid ja kuulutasid seda teistele, kuid neidki nad ei uskunud.
\par 14 Pärast ta ilmus neile üheteistkümnele, kui nad lauas istusid, ja sõitles nende uskmatust ning kõva südant, sest nad ei olnud uskunud neid, kes olid näinud teda üles tõusnud olevat.
\par 15 Ja ta ütles neile: „Minge kõike maailma ja kuulutage evangeeliumi kõigele loodule.
\par 16 Kes usub ja keda ristitakse, see saab õndsaks; aga kes ei usu, see mõistetakse hukka.
\par 17 Aga tunnustähed järgivad neid, kes usuvad. Minu nimel nad ajavad kurje vaime välja, räägivad uusi keeli,
\par 18 võtavad üles madusid, ja kui nad mõnd surmajooki joovad, ei tee see neile mitte kahju; haigete peale nad panevad käed, ja need saavad terveks.”

\section*{Jeesuse taevaminek}

\par 19 Ja Issand Jeesus, kui ta oli nendega rääkinud, võeti üles taevasse ja ta istus Isa paremale poolele.
\par 20 Aga nemad läksid välja ja jutlustasid igal pool. Ja Issand töötas ühes nendega ning kinnitas sõna tunnustähtedega, mis pärast seda sündisid.]





\end{document}