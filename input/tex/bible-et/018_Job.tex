\begin{document}

\title{Iiobi raamat}

\chapter{1}

\par 1 Uusimaal oli mees, Iiob nimi. See mees oli vaga ja õiglane, ta kartis Jumalat ja hoidus kurjast.
\par 2 Temale sündis seitse poega ja kolm tütart.
\par 3 Tema karjas oli seitse tuhat lammast ja kitse, kolm tuhat kaamelit, viissada paari härgi, viissada emaeeslit, ja tal oli väga palju peret; see mees oli rikkaim kõigist hommikumaalastest.
\par 4 Tema pojad käisid ja pidasid pidusid üksteise kodades, igaüks omal päeval; nad läkitasid käskjalad oma kolme õe järele ja kutsusid nad enestega sööma ja jooma.
\par 5 Aga kui pidupäevad olid ringi teinud, siis Iiob läkitas käskjalad poegade järele ja pühitses neid; ta tõusis hommikul vara ja ohverdas põletusohvreid, igaühe eest ühe, sest Iiob mõtles: „Võib-olla on mu pojad pattu teinud ja südames Jumalat neednud.” Nõnda tegi Iiob alati.
\par 6 Ent ühel päeval, kui Jumala lapsed tulid ja seisid Issanda ees, tuli ka saatan nende sekka.
\par 7 Ja Issand küsis saatanalt: „Kust sa tuled?„ Ja saatan vastas Issandale ning ütles: ”Maad mööda hulkumast ja rändamast.”
\par 8 Siis Issand ütles saatanale: „Kas oled pannud tähele mu sulast Iiobit? Sest tema sarnast maa peal ei ole: vaga ja õiglane mees, kardab Jumalat ja hoidub kurjast.”
\par 9 Saatan vastas Issandale ning ütles: „Kas Iiob asjata Jumalat kardab?
\par 10 Eks sa ole teinud igast küljest aia ümber temale ja ta kojale ning kõigele, mis tal on? Sa oled õnnistanud tema kätetööd ja ta kari on siginenud maal.
\par 11 Aga pista ometi käsi tema külge ja puuduta kõike, mis tal on! Kas ta siis õnnistab su palet?”
\par 12 Siis Issand ütles saatanale: „Vaata, kõik, mis tal on, olgu sinu käes! Ära ainult pista kätt tema enese külge!” Ja saatan läks Issanda juurest ära.
\par 13 Ja ühel päeval, kui ta pojad ja tütred olid söömas ja veini joomas oma vanima venna kojas,
\par 14 tuli käskjalg Iiobi juurde ja ütles: „Härjad olid kündmas ja emaeeslid nende kõrval söömas,
\par 15 aga seebalased tulid kallale ja võtsid need ära ja lõid poisid mõõgateraga maha. Ainult mina üksi pääsesin seda sulle teatama.”
\par 16 Kui ta alles rääkis, tuli teine ja ütles: „Jumala tuli langes taevast ning süütas lambad, kitsed ja poisid ning põletas nad ära. Ainult mina üksi pääsesin seda sulle teatama.”
\par 17 Kui ta alles rääkis, tuli teine ja ütles: „Kaldealased tegid kolm salka ja tungisid kaamelite kallale ning võtsid need ära ja lõid poisid mõõgateraga maha. Ainult mina üksi pääsesin seda sulle teatama.”
\par 18 Kui ta alles rääkis, tuli veel üks käskjalg ja ütles: „Su pojad ja tütred olid söömas ja veini joomas oma vanima venna kojas.
\par 19 Vaata, siis tuli kõrbe poolt suur tuul, tõukas koja nelja nurka ja see langes noorte meeste peale, nõnda et nad surid. Ainult mina üksi pääsesin seda sulle teatama.”
\par 20 Siis Iiob tõusis ja käristas oma kuue lõhki, ajas pea paljaks, heitis maha, kummardas
\par 21 ja ütles: „Alasti olen ma emaihust tulnud ja Alasti pöördun ma tagasi. Issand on andnud ja Issand on võtnud; Issanda nimi olgu kiidetud!”
\par 22 Kõige selle juures ei teinud Iiob pattu ega rääkinud Jumala kohta halba.

\chapter{2}

\par 1 Ja ühel päeval, kui Jumala lapsed tulid ja seisid Issanda ees, tuli ka saatan nende sekka ja seisis Issanda ees.
\par 2 Ja Issand küsis saatanalt: „Kust sa tuled?„ Ja saatan vastas Issandale: ”Maad mööda hulkumast ja rändamast.”
\par 3 Siis Issand ütles saatanale: „Kas oled pannud tähele mu sulast Iiobit? Sest tema sarnast maa peal ei ole: ta on vaga ja õiglane mees, kardab Jumalat ja hoidub kurjast. Ikka veel peab ta kinni oma vagadusest, kuigi sa kihutasid mind tema vastu teda ilma põhjuseta hävitama.”
\par 4 Aga saatan vastas Issandale ning ütles: „Nahk naha vastu, ja mees annab kõik, mis tal on, oma hinge eest.
\par 5 Aga siruta ometi oma käsi ning puuduta tema luud ja liha! Kas ta siis õnnistab su palet?”
\par 6 Ja Issand ütles saatanale: „Vaata, ta on su käes! Säästa ainult tema hing!”
\par 7 Siis saatan läks Issanda juurest ära ja lõi Iiobit kurjade paisetega jalatallast pealaeni.
\par 8 Iiob aga võttis enesele potikillu, et sellega ennast kaapida; ja ta istus tuha sees.
\par 9 Siis ta naine ütles temale: „Kas sa veelgi oma vagadusest kinni pead? Nea Jumalat ja sure!”
\par 10 Aga ta vastas temale: „Sinagi räägid, nagu rumalad naised räägivad. Kas me peaksime Jumalalt vastu võtma ainult head, aga mitte kurja?” Kõige selle juures ei teinud Iiob oma huultega pattu.
\par 11 Kui Iiobi kolm sõpra kuulsid kogu sellest õnnetusest, mis teda oli tabanud, siis tulid nad igaüks oma kodupaigast: teemanlane Eliifas, suhiit Bildad ja naamalane Soofar; ja nad tulid kokku, et minna temale kaastunnet avaldama ja teda trööstima.
\par 12 Aga kui nad eemalt oma silmad üles tõstsid, siis nad ei tundnud teda. Ja nad tõstsid häält ning nutsid, ja igamees käristas oma kuue lõhki ning riputas enesele vastu taevast tuhka pea peale.
\par 13 Siis nad istusid maas koos temaga seitse päeva ja seitse ööd, ja ükski ei rääkinud temaga sõnagi, sest nad nägid, et ta valu oli väga suur.

\chapter{3}

\par 1 Seejärel avas Iiob suu ja needis oma sündimispäeva.
\par 2 Ja Iiob hakkas rääkima ning ütles:
\par 3 „Kadugu see päev, mil ma sündisin, ja see öö, mil öeldi: „Poeglaps on eostunud!”
\par 4 Muutugu pimeduseks see päev; ärgu hooligu temast Jumal ülal ja ärgu paistku temale valgust!
\par 5 Nõudku teda pimedus ja surmavari, pilved lasugu ta peal, kohutagu teda päeva pimendused!
\par 6 See öö - võtku teda pilkane pimedus! Ärgu ta seltsigu aasta päevadega, kuude hulka ta ärgu tulgu!
\par 7 Vaata, see öö jäägu viljatuks, ärgu olgu tal hõiskamist!
\par 8 Vandugu teda päevaneedjad, kes on valmis Leviatanit äratama!
\par 9 Pimenegu ta puhtetähed, oodaku ta valgust, mis ei tule, ärgu saagu ta näha koidukiiri,
\par 10 sellepärast et ta ei sulgenud mu emaihu ust ega varjanud vaeva mu silma eest.
\par 11 Miks ma ei surnud emakotta, üsast välja tulles ei heitnud hinge?
\par 12 Miks võtsid põlved mind vastu ja miks olid rinnad, et sain imeda?
\par 13 Tõesti, ma oleksin nüüd maganud ja mul oleks olnud rahu; oleksin siis uinunud, mul oleks puhkus
\par 14 koos kuningate ja maanõunikega, kes ehitasid endile hauamärgid,
\par 15 või koos vürstidega, kellel oli kulda, kes täitsid oma kojad hõbedaga.
\par 16 Või miks ma ei olnud nagu varjatud nurisünnitis, nagu lapsukesed, kes päevavalgust ei saa näha?
\par 17 Seal jätavad õelad ässituse ja seal saavad väsinud puhata,
\par 18 seal on kõik vangid muretud: nad ei kuule enam sundija häält.
\par 19 Seal on pisike ja suur ühesugused ja ori on vaba oma isandast.
\par 20 Miks antakse valgust vaevatule ja elu neile, kelle hing on kibestunud,
\par 21 kes ootavad surma, mis ei tule, ent kes otsivad seda enam kui varandust,
\par 22 kes hõiskavad juubeldades ja on rõõmsad, kui nad leiavad haua?
\par 23 Miks antakse valgust mehele, kelle tee on varjul, kellele Jumal igast küljest on pannud takistusi?
\par 24 Sest ohkamine on mulle leivaks ja mu kaebed voolavad nagu vesi.
\par 25 Sest see, mille ees ma tundsin hirmu, tuli mulle kätte, ja mida ma kartsin, see tabas mind.
\par 26 Ei ole mul rahu, ei vaikust ega hingamist, küll aga on tulnud rahutus.”

\chapter{4}

\par 1 Siis rääkis teemanlane Eliifas ja ütles:
\par 2 „Ega püüe sinuga rääkida tüüta sind? Aga kes võiks sõnu peatada?
\par 3 Vaata, sina õpetasid paljusid ja kinnitasid nõrku käsi.
\par 4 Su sõnad tõstsid üles komistaja ja sa tegid nõtkuvad põlved tugevaks.
\par 5 Aga nüüd on see juhtunud sulle ja sa nõrked, see puudutab sind ja sa jahmud.
\par 6 Kas mitte jumalakartus ei ole su lootus ja su laitmatud eluviisid su ootus?
\par 7 Mõtle ometi: kes on süütult hukkunud ja kus on õiged hävitatud?
\par 8 Niipalju kui mina olen näinud: kes künnavad ülekohut ja külvavad õnnetust, need lõikavadki seda.
\par 9 Nad hukkuvad Jumala hingeõhust ja hävivad tema vihapuhangust.
\par 10 Lõvi möirgamine, Lõvi hääl, ja noorte lõvide hambad - need murtakse.
\par 11 Lõvi hukkub saagi puudusel ja emalõvi kutsikad aetakse laiali.
\par 12 Mulle tuli vargsi sõna ja mu kõrv kuulis sellest sosinat
\par 13 öiste nägemuste rahutuis mõtteis, kui sügav uni on langenud inimeste peale.
\par 14 Hirm haaras mind ja värin ja pani kõik mu luud-liikmed vappuma.
\par 15 Üks vaim liugles üle mu näo; mu ihukarvad tõusid püsti.
\par 16 Ta seisatas, aga ta välimust ma ei tundnud - üks kuju mu silma ees. Vaikus. Siis ma kuulsin häält:
\par 17 „Ons inimene õige Jumala ees või mees puhas oma Looja ees?”
\par 18 Vaata, oma sulaseidki ta ei usu ja oma ingleid ta peab eksijaiks,
\par 19 veel vähem siis neid, kes elavad savihooneis, mille alusmüürid on põrmus. Need lüüakse rutemini pihuks kui koi.
\par 20 Hommikust õhtuni lüüakse neid puruks, märkamata hukkuvad nad igaveseks.
\par 21 Eks nende telginöörid kista üles? Nad surevad, ilma et taipaksidki.

\chapter{5}

\par 1 Hüüa ometi! Ons keegi, kes sulle vastab? Kelle poole pühadest sa pöördud?
\par 2 Tõesti, meelehärm tapab meeletu ja õhin surmab narri.
\par 3 Ma nägin meeletut juurduvat ja ma sajatasin äkitselt tema eluaset.
\par 4 Tema lapsed jäid kaitsetuks, neid rõhuti väravas ja päästjat ei olnud.
\par 5 Tema lõikuse sõi näljane: kibuvitste vaheltki võttis ta selle, ja janused kahmasid ta varanduse.
\par 6 Sest õnnetus ei tõuse põrmust ega võrsu kannatused mullast,
\par 7 vaid inimene ise sünnib vaevaks ja sädemed lendavad kõrgele.
\par 8 Ometi pöörduksin ma Jumala poole ja viiksin oma asja Jumala ette,
\par 9 kes teeb suuri ja uurimatuid asju, arvutuid imetegusid,
\par 10 kes annab maa peale vihma ja saadab põldudele vee,
\par 11 et tõsta alandlikke kõrgele ja anda leinajaile suurt õnne,
\par 12 kes teeb tühjaks kavalate kavatsused, et nende kätetöö ei läheks korda,
\par 13 kes tabab tarku nende tarkuses, et salalike nõu läheks luhta,
\par 14 et nad päevaajal kohtaksid pimedust ja lõunaajal kobaksid otsekui öösel.
\par 15 Nõnda ta päästab mõõga eest, nende suu eest, vaese vägeva käest,
\par 16 et viletsal oleks lootust ja ülekohus suleks oma suu.
\par 17 Vaata, õnnis on inimene, keda Jumal noomib. Ära siis põlga Kõigevägevama karistust!
\par 18 Sest tema valmistab küll valu, aga tema ka seob haavad, tema lööb, aga tema käed ka parandavad.
\par 19 Kuuest hädast päästab ta sind ja seitsmendas ei puutu sinusse kuri.
\par 20 Näljas lunastab ta sind surmast ja sõjas mõõga käest.
\par 21 Keelepeksu eest oled sa peidetud ja sul pole karta hävingut, kui see tuleb.
\par 22 Sa võid naerda hävingut ja nälga ja sul pole vaja karta maa loomi.
\par 23 Sest sul on leping kividega väljal ja metsloomadel on sinuga rahu.
\par 24 Siis sa tunned, et su telgil on rahu, ja kui sa vaatad oma karjamaad, siis ei puudu seal midagi.
\par 25 Ja sa märkad, et su sugu on suur, su järeltulijaid on otsekui rohtu maa peal.
\par 26 Küpses vanuses sa lähed hauda, otsekui naber pannakse rehepõrandale omal ajal.
\par 27 Vaata, seda me oleme uurinud, nõnda see on. Sina aga kuula ja võta see teatavaks!”

\chapter{6}

\par 1 Siis vastas Iiob ja ütles:
\par 2 „Kui ometi mu meelehärm saaks vaetud ja mu õnnetus oleks pandud vaekaussidele.
\par 3 Tõesti, see oleks nüüd raskem kui mereliiv. Seepärast on mu sõnadki tormakad.
\par 4 Sest minus on Kõigevägevama nooled, mu vaim joob nende mürki. Jumala hirmutamised on rivistunud mu vastu.
\par 5 Kas metseesel kisendab noore rohu peal või ammub härg oma sulbi juures?
\par 6 Kas magedat süüakse ilma soolata või on siis maitset kassinaeri limal?
\par 7 Mu hing tõrgub neid puudutamast, need on mulle nagu rüvetatud roog.
\par 8 Oh, et ometi mu palve täide läheks ja Jumal annaks, mida soovin!
\par 9 Otsustaks ometi Jumal mind murda, sirutaks oma käe ja lõikaks katki mu elulõnga,
\par 10 siis oleks mul veel troostigi: ma hüppaksin rõõmu pärast isegi armutus valus, sest ma ei ole salanud Püha sõnu.
\par 11 Mis on mu jõud, et jaksaksin oodata, ja missugune peaks olema mu eesmärk, et suudaksin kindlaks jääda?
\par 12 Ons mu tugevus nagu kivide tugevus või ons mu ihu vaskne?
\par 13 Tõesti, mul enesel ei ole abi ja pääsemine on mu juurest peletatud.
\par 14 Kes põlgab sõbra sõprust, see loobub Kõigevägevama kartusest.
\par 15 Mu vennad on petlikud nagu jõgi, otsekui kuivaks valguvad jõesängid,
\par 16 mis jääst on muutunud tumedaks, kuhu lumi on pugenud peitu;
\par 17 veevaeseks jäädes need vaikivad, kuumuses kaovad oma asemelt.
\par 18 Karavanid põikavad teelt kõrvale, lähevad kõrbe ja hukkuvad.
\par 19 Teema karavanid heidavad pilke, Seeba teekäijad loodavad nende peale:
\par 20 oma lootuses nad jäävad häbisse, sinna jõudes nad pettuvad.
\par 21 Seesuguseiks olete nüüd saanud: te näete kohutavat asja ja kardate.
\par 22 Kas ma olen öelnud: „Andke mulle kingitusi!” või: ”Makske oma varandusest minu eest,
\par 23 päästke mind vaenlase võimusest ja lunastage mind vägivallategijate käest!?
\par 24 Õpetage mind, siis ma vaikin! Tehke mulle selgeks, milles olen eksinud!
\par 25 Otsekohesed sõnad ei olekski kibedad. Aga mis tähendus on teie noomimisel?
\par 26 On teil kavatsus mu sõnu laita? Kas meeltheitja kõne peaks olema nagu tuul?
\par 27 Te heidate liisku isegi vaeslapse pärast ja müüte maha oma sõbra.
\par 28 Aga nüüd vaadake mu peale! Ma tõesti ei valeta teile näkku!
\par 29 Jätke ometi järele, et ei sünniks ülekohut! Jah, jätke järele, sest veel on mul selles asjas õigus!
\par 30 Ons mu keelel ülekohut? Või ei taipa mu suulagi, mis on õnnetus?

\chapter{7}

\par 1 Eks ole inimesel maa peal võitlemist ja tema päevad nagu palgalise päevad?
\par 2 Otsekui sulane, kes igatseb varju, ja nagu palgaline, kes ootab oma tasu,
\par 3 nõnda sain ma enesele pärisosaks piinakuud ja mulle määrati vaevaööd.
\par 4 Kui ma magama heidan, siis ma mõtlen: „Millal võin tõusta?„ ja kui olen üles tõusnud, siis: ”Millal tuleb õhtu?” Ma olen koiduni täis rahutust.
\par 5 Mu ihu on kaetud ussikeste ja mullakamaraga, mu nahk on kärnas ja kurtunud.
\par 6 Mu päevad on kärmemad kui süstik ja lõpevad lootuseta.
\par 7 Pea meeles, et mu elu on otsekui tuuleiil, ei saa mu silm enam õnne näha.
\par 8 Enam ei näe mind silm, mis mind praegu vaatab, su silmad otsivad mind, aga mind ei ole enam.
\par 9 Pilv kaob ja läheb ära: nõnda ei tõuse ka see, kes läheb alla surmavalda.
\par 10 Ta ei tule enam tagasi oma kotta ja tema ase ei tunne teda enam.
\par 11 Seepärast ei taha minagi keelata oma suud: ma räägin oma vaimu ahastuses, kaeblen oma hinge kibeduses.
\par 12 Kas ma olen meri või meremadu, et sa paned mu üle valve?
\par 13 Kui ma mõtlen: „Mu voodi peab mind trööstima, mu magamisase mu kaebust kandma”,
\par 14 siis sa kohutad mind unenägudes ja teed mulle hirmu nägemustega,
\par 15 nõnda et mu hing peab paremaks lämbumist, pigem surm kui need mu kondid!
\par 16 Küllalt! Ma ei taha elada igavesti! Jäta mind! Mu päevad ongi ju ainult õhk.
\par 17 Mis on inimene, et sa pead teda suureks ja et sa paned teda tähele,
\par 18 vaatad ta järele igal hommikul ja katsud teda läbi igal hetkel?
\par 19 Kas sa ei pööragi oma pilku ära mu pealt ega jäta mind süljeneelamise ajakski?
\par 20 Kui ma ka olen pattu teinud, mida ma siis sellega teen sinule, sa inimese valvur? Miks panid minu enesele märklauaks ja miks olen saanud sulle koormaks?
\par 21 Miks sa ei anna andeks mu üleastumist ega võta ära mu süüd? Sest nüüd ma lähen mulda magama ja kui sa mind otsid, siis ei ole mind enam.”

\chapter{8}

\par 1 Siis rääkis suhiit Bildad ja ütles:
\par 2 „Kui kaua sa tahad seda kuulutada, kui kaua on sõnad su suust nagu tugev tuul?
\par 3 Kas Jumal peaks väänama õigust või Kõigevägevam väänama õiglust?
\par 4 Kui su lapsed pattu tegid tema vastu, siis ta andis nad nende üleastumise võimusesse.
\par 5 Kui sa otsid Jumalat ja anud Kõigevägevamat,
\par 6 kui sa oled aus ja otsekohene, siis ta ärkab nüüd sinu pärast ja taastab su eluaseme, nagu see peab olema.
\par 7 Ja olgugi su algus väike, sa saad siiski lõpuks väga suureks.
\par 8 Aga küsi ometi eelmiselt põlvelt ja pane tähele, mida nende isad on uurinud.
\par 9 Sest meie oleme eilsed ega tea midagi, meie päevad maa peal on ju nagu vari.
\par 10 Eks nad õpeta sind ja räägi sulle ja too oma südamest sõnu esile?
\par 11 Kas kõrkjas kasvab seal, kus pole muda, või sirgub pilliroog ilma veeta?
\par 12 Alles veel kasvujõus, lõikamata, kuivab see kiiremini kui kõik muu rohi.
\par 13 Niisugune lõpp on kõigil, kes unustavad Jumala, ja jumalatu inimese lootus kaob.
\par 14 Ta usk on härmalõnga sarnane ja ta usaldus on otsekui ämblikuvõrk.
\par 15 Ta toetub oma kojale, aga see ei pea vastu, ta haarab sellest kinni, aga see ei jää püsti.
\par 16 Ta haljendab päikese paistel ja tema võsud levivad üle ta aia.
\par 17 Ta juured on põimunud kivikangrusse, ta näeb kivide koda.
\par 18 Aga kui Jumal hävitab tema ta asemelt, siis ta salgab teda: „Ma ei ole sind näinudki!”
\par 19 Vaata, niipalju oligi temal teekonnarõõmu. Ja mullast võrsub juba teine.
\par 20 Vaata, Jumal ei hülga vaga ega võta kinni pahategijate käest.
\par 21 Veel täidab ta sinu suu naeruga ja su huuled hõiskamisega.
\par 22 Kes sind vihkavad, peavad ennast katma häbiga ja õelate telki ei ole enam.”

\chapter{9}

\par 1 Siis rääkis Iiob ja ütles:
\par 2 „Ma tean tõesti, et see nõnda on. Kuidas võib inimene õige olla Jumala ees?
\par 3 Kui keegi tahaks temaga vaielda, ei suudaks ta temale vastata mitte ainsalgi korral tuhandest.
\par 4 Ta on südamelt tark ja jõult tugev, kes võiks teda trotsida ja ise pääseda?
\par 5 Tema liigutab mägesid, ilma et need märkaksid, kui ta oma vihas neid kummutab;
\par 6 tema põrutab maa oma asemelt, nõnda et selle sambad vabisevad;
\par 7 tema käsib päikest, et see ei tõuseks, ja paneb tähed pitseriga kinni;
\par 8 tema üksinda laotab taevaid ja kõnnib mere lainete peal;
\par 9 tema teeb Vankri-, Varda- ja Sõelatähed ja lõunapoolsed tähtkujud;
\par 10 tema teeb suuri ja mõistmatuid asju ning otsatuid imetegusid.
\par 11 Vaata, ta läheb minust mööda, aga mina ei näe, ta käib üha, aga mina ei märka teda.
\par 12 Vaata, ta napsab ära, kes võiks teda takistada? Kes ütleks temale: „Mis sa teed?”
\par 13 Jumal ei hoia tagasi oma viha, temale peavad alistuma Rahabi aitajad.
\par 14 Kuidas võiksin siis mina temale vastata, oma sõnu tema jaoks valida?
\par 15 Kuigi olen õige, ma ei saa vastata, vaid pean anuma oma kohtumõistjat.
\par 16 Kuigi ma hüüaksin ja tema vastaks mulle, ei usu ma siiski, et ta mu häält kuulda võtab,
\par 17 tema, kes haarab mu järele tormis ja lisab mulle ilma põhjuseta haavu,
\par 18 kes ei lase mind hinge tõmmata, vaid täidab mind kibedusega.
\par 19 Kui on küsimus jõust, vaata, ta on tugevam. Või kui on kohtuasi, kes mind ette kutsub?
\par 20 Kuigi olen õige, mõistab mind hukka mu oma suu; kuigi olen süütu, peab tema mind süüdlaseks.
\par 21 Ma olen süütu! Ma ei hooli oma hingest, ma põlgan oma elu!
\par 22 Ükskõik! Seepärast ma ütlen: „Tema hävitab niihästi õige kui õela.”
\par 23 Kui uputus äkitselt surmab, siis ta pilkab süütute meeleheidet.
\par 24 Kui maa on antud õela kätte, ta katab selle kohtumõistja palge - kui mitte tema, kes siis muu?
\par 25 Mu päevad on jooksjast nobedamad, kaovad õnne nägemata.
\par 26 Need mööduvad otsekui pilliroost vened, nagu kotkas, kes sööstab oma saagi kallale.
\par 27 Kui ma mõtlen: „Ma unustan oma kaebuse, jätan oma kurva näo ja olen rõõmus”,
\par 28 siis on mul hirm kõigi oma kannatuste ees, ma tean, et sa ei pea mind süütuks.
\par 29 Olgu ma siis juba süüdi! Miks peaksin ennast veel ilmaasjata vaevama?
\par 30 Isegi kui ma peseksin ennast lumega ja puhastaksin oma käsi leelisega,
\par 31 pistaksid sina mind poriauku ja siis jälestaksid mind mu enda riidedki.
\par 32 Sest Jumal ei ole inimene nagu mina, et ma temale saaksin vastata, et me üheskoos saaksime kohut käia.
\par 33 Ei ole meie vahel vahemeest, kes oma käe saaks panna meie mõlema peale.
\par 34 Võtku ta oma vits ära mu pealt, et hirm tema ees mind ei heidutaks!
\par 35 Siis ma saaksin rääkida ilma teda kartmata. Sest ma ei ole omast meelest mitte niisugune.

\chapter{10}

\par 1 Mu hing on elust tüdinud, ma annan voli oma kaebusele, ma räägin oma hingekibeduses,
\par 2 ma ütlen Jumalale: Ära mõista mind hukka! Anna mulle teada, miks sa mind kimbutad!
\par 3 Kas sinu meelest on hea, et sa rõhud mind, et sa põlgad oma kätetööd, aga valgustad õelate nõupidamist?
\par 4 Ons sul lihalikud silmad või näed sa, nagu inimene näeb?
\par 5 Ons su päevad nagu inimese päevad või on su aastad nagu mehe aastad,
\par 6 et sa uurid mu süüd ja nõuad taga mu pattu,
\par 7 kuigi sa tead, et mina ei ole süüdlane ja et ei ole kedagi, kes sinu käest võiks päästa?
\par 8 Sinu käed kujundasid ja valmistasid mind üheskoos ümberringi - ja sa neelad mu ära!
\par 9 Pea meeles, et sa mind oled teinud otsekui savist. Ja nüüd tahad mind jälle viia tagasi põrmu.
\par 10 Eks sa ole mind valanud nagu piima ja lasknud kalgendada juustu sarnaselt?
\par 11 Sa katsid mind naha ja lihaga ning põimisid ühte kontide ja kõõlustega.
\par 12 Sa andsid mulle elu ja osaduse, ja su hoolitsus hoidis mu vaimu.
\par 13 Aga selle sa talletasid oma südamesse, ma tean, et sul oli meeles see:
\par 14 kui mina pattu teen, siis valvad sina mind ega jäta mind karistamata.
\par 15 Häda mulle, kui oleksin süüdi! Aga isegi õigena ei julge ma pead tõsta, olles täis häbi ja nähes oma viletsust.
\par 16 Ja kui ma tõusekski, sa ajaksid mind taga nagu lõvi ja teeksid jälle mu kallal imetegusid.
\par 17 Sa tood mu vastu uusi tunnistajaid ja kasvatad oma viha mu vastu, üha pannes vaeva mulle peale.
\par 18 Miks tõid mind välja emaihust? Oleksin ometi hinge heitnud, et ükski silm ei oleks mind näinud!
\par 19 Siis oleks mind viidud emaüsast hauda, otsekui mind ei oleks olnudki.
\par 20 Eks lakka mu päevade pisku? Jäta mind, et võiksin olla pisutki rõõmsam,
\par 21 enne kui ma tagasitulekuta lähen pimeduse ja surmavarju maale,
\par 22 maale, kus pimedus on pilkane, kus on surmavari ja kaos ja kus valguski on pimedus!”

\chapter{11}

\par 1 Siis rääkis naamalane Soofar ja ütles:
\par 2 „Kas peaks see sõnaküllus jääma vastuseta või lobisejal olema õigus?
\par 3 Kas su vada peaks panema mehed vaikima või tohid sa mõnitada, ilma et ükski sind häbistaks?
\par 4 Sest sa ütled: „Mu õpetus on selge ja ma olen tema silmis puhas.”
\par 5 Kui ometi Jumal räägiks ja avaks oma huuled su vastu
\par 6 ning ilmutaks sulle tarkuse saladusi, mis on mõistusele otsekui imed. Siis sa mõistaksid, et Jumal su süüst mõndagi unustab.
\par 7 Kas sa suudad leida Jumala sügavuse? Või tahad sa jõuda Kõigevägevama täiuseni?
\par 8 Need on kõrgemad kui taevad - mida sina suudad teha? Sügavamad kui surmavald - mida sina sellest tead?
\par 9 Nende mõõt on pikem kui maa ja laiem kui meri.
\par 10 Kui tema mööda läheb ja vangistab ning kohtu kokku kutsub, kes teda siis võiks keelata?
\par 11 Sest tema tunneb valelikke inimesi, näeb nurjatust ega pane mikski,
\par 12 et ka tühipäine mees saab oidu, kui inimene sünnib nagu metseesli varss.
\par 13 Kui sinagi oma südant valmistaksid ja sirutaksid oma käed tema poole -
\par 14 kui su kätes on süütegu, siis saada see kaugele ja ära lase ülekohut jääda oma telkidesse -,
\par 15 siis sa võiksid küll häbimärgita oma palge üles tõsta, võiksid olla kindel ega tarvitseks karta.
\par 16 Siis sa võiksid unustada vaeva, mõelda sellele kui äravoolanud veele.
\par 17 Su eluiga saaks kaunimaks kui keskpäev, pimedus oleks nagu hommik.
\par 18 Siis sa võiksid olla kindel, et lootust on. Sa tunneksid ennast turvalisena, ja saaksid rahulikult magada.
\par 19 Sa võiksid maha heita, ilma et keegi hirmutaks. Ja paljud tuleksid sind meelitama.
\par 20 Aga õelate silmad kustuksid, neil kaoks pelgupaik ja nende lootuseks oleks hingeheitmine.”

\chapter{12}

\par 1 Siis rääkis Iiob ja ütles:
\par 2 „Tõsi, teie olete inimesed ja koos teiega sureb tarkus.
\par 3 Ka minul on mõistus nagu teil, ma ei jää teist maha. Ja kes ei mõistaks seda kõike?
\par 4 Naerualuseks ligimesele on saanud, kes hüüab Jumala poole, et see teda kuuleks. Naerualuseks on õige ja vaga.
\par 5 Ülbe arvates ei ole vaja õnnetusest hoolida, see tabab neid, kelle jalg libastub.
\par 6 Rahus on rüüstajate telgid ja julgeolek on Jumala ärritajal, sellel, kes Jumalat oma käes kannab.
\par 7 Aga küsi ometi loomadelt, et need õpetaksid sind, ja taeva lindudelt, et need annaksid sulle teada,
\par 8 või kõnele maale, et ta teeks sind targaks, ja et mere kalad jutustaksid sulle!
\par 9 Kes neist kõigist ei tea, et seda on teinud Issanda käsi,
\par 10 kelle käes on kõigi elavate hing ja iga lihase inimese vaim.
\par 11 Eks kõrv katsu sõnad läbi ja eks suulagi maitse rooga?
\par 12 Elatanuil on tarkus ja pikaealistel mõistus.
\par 13 Jumalal on tarkus ja vägi, temal on nõu ja mõistus.
\par 14 Mida tema maha kisub, seda ei ehitata üles; keda tema vangistab, seda ei vabastata mitte.
\par 15 Vaata, kui tema peatab veed, tekib kuivus, ja kui ta need valla päästab, hävitavad need maa.
\par 16 Temal on tugevus ja tulemus, tema päralt on eksija ja eksitaja.
\par 17 Tema viib nõuandjad riisutavaks ja teeb kohtumõistjad narriks.
\par 18 Tema tühistab kuningate karistused ja köidab neile endile köie vööle.
\par 19 Tema saadab preestrid paljajalu teele ja paiskab ümber vägevate istmed.
\par 20 Tema riisub usaldusväärsetelt huuled ja võtab vanadelt aru.
\par 21 Tema valab õilsate peale põlgust ja lõdvendab tugevate vööd.
\par 22 Tema ilmutab saladused pimedusest ja toob surmavarju valguse kätte.
\par 23 Tema teeb rahvaid suureks ja hukkab neid, tema laotab rahvaid laiali ja hävitab neid.
\par 24 Maa rahva peameestelt võtab ta mõistuse ja paneb nad eksima tühjal teeta maal.
\par 25 Nad kobavad pimeduses valguseta ja ekslevad nagu joobnud.

\chapter{13}

\par 1 Vaata, kõike seda on mu silm näinud, kõrv kuulnud ja tähele pannud.
\par 2 Mida teie teate, seda tean minagi, ma ei jää teist maha.
\par 3 Aga ma tahaksin Kõigevägevamaga rääkida ja meeleldi ennast kaitsta Jumala ees.
\par 4 Sest teie olete valega võõpajad, ebaarstid olete teie kõik!
\par 5 Oleks küll parem, kui te vaikiksite, siis võiks seda pidada teie tarkuseks.
\par 6 Kuulge ometi mu manitsust ja pange tähele mu huulte väiteid!
\par 7 Kas tahate Jumala kasuks valet rääkida ja tema heaks pettust sõnastada?
\par 8 Kas tahate olla tema poolt, Jumala kasuks asja ajada?
\par 9 Ons see hea, kui ta teid läbi katsub? Või tahate teda petta nagu inimest?
\par 10 Ta karistab teid karmilt, kui te oma poolehoidu osutate salalikult.
\par 11 Kas tema auväärsus ei teegi teile hirmu ja kartus tema ees ei valda teid?
\par 12 Teie väited on tuhka kirjutatud, teie kilbid on savikilbid.
\par 13 Olge vait mu ees ja laske mind rääkida, tulgu siis mu peale mis tahes!
\par 14 Ma võtan oma ihu hammaste vahele ja panen oma hinge pihku.
\par 15 Vaata, ta tapab mu! Ma ei looda enam! Ma tahan tema ees ainult õigustada oma teed.
\par 16 Seegi on mulle abiks, et jumalatu ei pääse tema palge ette.
\par 17 Kuulge hästi mu sõnu ja võtke kõrvu mu seletus!
\par 18 Vaata ometi, ma olen asunud õigust jalule seadma: ma tean, et mul on õigus.
\par 19 Kes tahaks mulle vastu vaielda? Siis ma vaikiksin nüüd ja heidaksin hinge.
\par 20 Ainult kaks asja tee mulle, siis pole mul vaja ennast sinu eest peita:
\par 21 võta käsi mu pealt ära ja hirm sinu ees ärgu heidutagu mind!
\par 22 Hüüa siis ja mina vastan, või räägin mina ja sina vasta mulle!
\par 23 Kui palju on mul süütegusid ja patte? Tee mulle teatavaks mu üleastumine ja patt!
\par 24 Miks peidad oma palge ja pead mind oma vaenlaseks?
\par 25 Kas tahad hirmutada lendlevat lehte ja taga ajada kuivanud kõrt,
\par 26 et sa mu vastu kirja paned kibedaid asju ja lased mind tasuda mu noorpõlve süütegusid,
\par 27 paned mu jalad pakku, valvad kõiki mu radu ja märgid enesele mu jalajälgi?
\par 28 Jah, inimene on kõdunenud nahklähkri sarnane, otsekui koi järatud riie.

\chapter{14}

\par 1 Inimesel, naisest sündinul, on lühikesed elupäevad ja palju tüli.
\par 2 Ta tõuseb nagu lilleke ja ta lõigatakse ära, ta põgeneb nagu vari ega jää püsima.
\par 3 Ometi pead sa seesugust silmas ja viid mind enesega kohtu ette.
\par 4 Kes võib roojasest teha puhta? Mitte keegi!
\par 5 Kuna tema elupäevad on määratud ja tema kuude arv on sinu käes - sa oled pannud piiri, millest ta ei saa üle minna -,
\par 6 siis pööra oma pilk tema pealt ära ja jäta ta rahule, seni kui ta päevilisena oma päevast rõõmu tunneb!
\par 7 Sest puulgi on lootus: kui ta maha raiutakse, siis ta võrsub taas ja tal pole võsudest puudu.
\par 8 Kuigi ta juur maa sees kõduneb ja känd mullas sureb,
\par 9 hakkab ta veehõngust haljendama ja võsusid ajama otsekui istik.
\par 10 Aga kui mees sureb ja kaob, kui inimene hinge heidab - kus on ta siis?
\par 11 Vesi voolab järvest ja jõgi taheneb ning kuivab,
\par 12 nõnda heidab inimene magama ega tõuse enam. Enne kui taevaid pole enam, nad ei ärka, neid ei äratata unest.
\par 13 Oh, et sa varjaksid mind surmavallas, peidaksid, kuni su viha möödub; et sa määraksid mulle aja ja siis peaksid mind meeles.
\par 14 Kui mees sureb, kas ta ärkab jälle ellu? Ma ootaksin kogu oma sundaja, kuni mu vabastus tuleb.
\par 15 Sa hüüaksid ja ma vastaksin sulle, sa igatseksid oma kätetööd.
\par 16 Nüüd loed sa aga mu samme, ei lähe mööda mu patust.
\par 17 Mu üleastumine on pitseriga suletud kukrusse ja sa katad kinni mu süü.
\par 18 Aga variseb ju ka mägi ja kalju nihkub paigast,
\par 19 vesi kulutab kive, vihmavaling uhub maamulla - nõnda hävitad sina inimese lootuse.
\par 20 Sina alistad tema igaveseks ja ta peab minema, muudad ta näo ja saadad ta ära.
\par 21 Kas ta lapsi austatakse - tema ei saa seda teada, või kas neid põlatakse - tema seda ei märka.
\par 22 Tema tunneb valu ainult omaenese ihus ja leina omaenese hinges.”

\chapter{15}

\par 1 Siis rääkis teemanlane Eliifas ja ütles:
\par 2 „Kas tark tohib vastata tuulepäiselt ja täita oma rinda idatuulega,
\par 3 seletada kõlbmatute kõnedega, sõnadega, millest pole kasu?
\par 4 Sa teed tühjaks isegi jumalakartuse ja rikud hardust Jumala ees.
\par 5 Sest su süü paneb sulle sõnad suhu ja sa valid kavalate keele.
\par 6 Su oma suu süüdistab sind, aga mitte mina, su oma huuled kostavad su vastu.
\par 7 Kas oled sina esimese inimesena sündinud? Ons sind enne mäekünkaid sünnitatud?
\par 8 Kas oled sina Jumala nõupidamist kuulnud ja nõnda enesele tarkuse toonud?
\par 9 Mis see on, mida sina tead, aga meie ei tea, mida sina mõistad, aga meie mitte?
\par 10 Meiegi hulgas on hallpäid ja elatanuid, ealt vanemad kui su isa.
\par 11 Ons sinu jaoks väike Jumala troost, või sõna, mis kohtleb sind leebelt?
\par 12 Kuhu su süda sind kisub ja kuhu su silmad sihivad,
\par 13 et sa pöörad oma vaimu Jumala vastu ja paiskad sõnu suust välja?
\par 14 Kuidas võiks inimene olla puhas, naisest sündinul olla õigus?
\par 15 Vaata, tema ei usu oma ingleidki ja tema silmis ei ole taevadki selged,
\par 16 veel vähem siis põlastusväärset ja laostunut, meest, kes väärtegusid joob nagu vett.
\par 17 Mina kuulutan sulle, kuule mind, ja ma jutustan, mida olen näinud,
\par 18 mida targad on teada andnud, mida ei olnud salanud nende vanemad,
\par 19 kellele üksi oli antud maa ja kelle seas veel võõras ei olnud käinud:
\par 20 süüdlane vaevleb kogu eluaja ja jõhkrale on talletatud pisut aastaid.
\par 21 Hirmuhääled on tal kõrvus, rahuajalgi tuleb hävitaja temale kallale.
\par 22 Ei ta usu, et ta pimedusest välja pääseb: ta on mõõgale määratud.
\par 23 Ta peab hulkuma leiva pärast: kus seda on? Ta teab, et pimedusepäev on temale valmis.
\par 24 Ahastus ja häda hirmutavad teda, vallutavad tema nagu tapluseks valmis kuningas.
\par 25 Sest ta on sirutanud oma käe Jumala vastu ja on suurustanud Kõigevägevama ees,
\par 26 joostes kangekaelselt tema vastu oma paksukühmuliste kilpidega.
\par 27 Sest ta on katnud oma näo rasvaga, on kasvatanud puusadele lihavust
\par 28 ja on elanud hävitatud linnades, kodades, kus ei olnud luba elada, mis olid määratud varemeiks.
\par 29 Ta ei saa rikkaks, ta varandus ei kesta kaua ja tema omand ei kaldu maha.
\par 30 Ta ei pääse pimedusest, kuumus kuivatab ta võsu ja ta taandub tema suu hinguse ees.
\par 31 Ärgu ta lootku tühjale - ta eksib! Sest temale saab tasuks tühjus.
\par 32 See läheb täide enneaegselt ja tema võsud ei haljenda enam.
\par 33 Ta ajab otsekui viinapuu maha oma küpsemata kobarad ja pillab õisi nagu õlipuu.
\par 34 Sest jumalatute jõuk jääb viljatuks ja tuli põletab meeleheavõtjate telgid.
\par 35 Nad on lapseootel vaevaga ja sünnitavad nurjatust, nende ihu saab toime pettusega.”

\chapter{16}

\par 1 Aga Iiob rääkis ja ütles:
\par 2 „Sellesarnast olen ma palju kuulnud, te kõik olete tüütavad trööstijad.
\par 3 Kas sel tuule rääkimisel ei olegi lõppu, või mis kihutab sind rääkima?
\par 4 Minagi võiksin rääkida nagu teie, kui oleksite minu asemel; minagi võiksin teie vastu sõnadega hiilata ja teie pärast pead vangutada.
\par 5 Ma võiksin teid oma suuga kinnitada ja mitte keelata oma huulte kaastundeavaldust.
\par 6 Kui ma räägin, ei vähene mu valu, ja kas see siiski minust lahkub, kui ma lõpetan?
\par 7 Aga nüüd on Jumal mind väsitanud. Ta on hävitanud kõik mu omaksed.
\par 8 Ta on haaranud mind - sellest tuli tunnistaja, kes tõusis mu vastu -, mu kõhnus süüdistab mind näkku.
\par 9 Ta viha rebis ja kiusas mind, ta kiristas mu pärast hambaid. Minu vaenlane teritab mu vastu oma silmi.
\par 10 Nad ajavad suu ammuli mu vastu, löövad mind pilgates põsele; nad kogunevad hulganisti mu vastu.
\par 11 Jumal andis mind nurjatuile ja tõukas mind õelate kätte.
\par 12 Ma elasin rahus, aga ta vapustas mind, ta haaras mind kuklast ja lõi mind puruks; ta pani mind enesele märklauaks,
\par 13 ta nooled ümbritsevad mind. Ta lõhestab armutult mu neerud, ta valab mu sapi maa peale.
\par 14 Ta murrab mind murd murru peale, ta ründab mind otsekui sõdur.
\par 15 Ma õmblesin enesele ihu jaoks kotiriide ja torkasin oma sarve põrmu.
\par 16 Mu nägu punetab nutust ja mu laugudel on sünge vari,
\par 17 kuigi mu käes ei ole ülekohut ja kuigi mu palve on siiras.
\par 18 Oh maa, ära kata mu verd, ja mu kisendamisel ärgu olgu puhkepaika!
\par 19 Vaata, nüüdki on mul tunnistaja taevas ja eestkõneleja kõrgustes.
\par 20 Mu sõbrad on pilkajad - Jumala poole on mu pisarais silm,
\par 21 et ta kohut mõistaks mehe ja Jumala vahel, otsekui inimese ja ta ligimese vahel.
\par 22 Sest veel pisut aastaid ja ma lähen teele, millelt ma ei pöördu tagasi.

\chapter{17}

\par 1 Mu vaim on murtud, mu päevad on kustutatud, mind ootavad hauad.
\par 2 Tõesti, pilked on mu osa, ja nende tõrksusest on mu silm väsinud.
\par 3 Pane ometi minu heaks pant tallele enese juurde! Kes muidu mu kasuks kätt lööks?
\par 4 Et sa oled nende südamed arusaamisest võõrutanud, siis sa ei ülenda neid.
\par 5 Kes kutsub sõbrad jagamisele, selle lastel tuhmuvad silmad.
\par 6 Mind on antud inimestele sõnakõlksuks ja ma olen pealesülitamiseks nende ees.
\par 7 Mu silm on kurbusest tuhm ja kõik mu liikmed on otsekui varjud.
\par 8 Õiglased ehmuvad sellest ja süütu ärritub jumalavallatute pärast.
\par 9 Aga õige püsib oma teel, ja kellel on puhtad käed, kasvab tugevuses.
\par 10 Te kõik aga võite tulla taas, tarka ma teie hulgast ei leia.
\par 11 Mu päevad on möödunud, katki kistud on mu kavatsused, mu südame soovid.
\par 12 Need tegid öö päevaks: valgus oli lähemal kui pimedus.
\par 13 Kui ma veel võin loota, siis on surmavald mu koda, ma laotan oma aseme pimedusse.
\par 14 Ma hüüan hauale: „Sa oled mu isa!„ ja ussikestele: ”Mu ema ja õde!”
\par 15 Kus on siis mu lootus, ja kes saaks mu lootust näha?
\par 16 Minuga astud sa alla surmavalda, kui üheskoos põrmu vajume.”

\chapter{18}

\par 1 Siis rääkis suhiit Bildad ja ütles:
\par 2 „Millal sa teed sõnadele lõpu? Mõtle järele, ja rääkigem siis.
\par 3 Miks peetakse meid loomadeks, oleme rumalad teie silmis?
\par 4 Sina, kes vihas oma hinge lõhki käristad - kas sinu pärast jäetakse maha maa või nihutatakse kalju oma asemelt?
\par 5 Jah, õela valgus kustub ja tema tuleleek ei paista.
\par 6 Ta telgis pimeneb valgus ja ta kohal kustub tema lamp.
\par 7 Ta jõudsad sammud jäävad lühikeseks ja ta oma nõu paiskab ta maha.
\par 8 Sest ta oma jalad viivad ta võrku ja ta käib püüniste peal.
\par 9 Püüdepael haarab teda kannast, lõks hoiab teda kinni.
\par 10 Tema jaoks on peidetud maa peale köis, teerajale silmus.
\par 11 Kõikjal kohutab teda suur hirm ja kihutab tema kannul.
\par 12 Õnnetus tunneb nälga tema järele, hukatus on valmis tema kukutamiseks.
\par 13 Tõbi sööb ta naha, surma esmasündinu sööb ta liikmed.
\par 14 Tema telgist kistakse ta lootus ja teda aetakse suure hirmu kuninga juurde.
\par 15 Tema telki asub elama see, mis pole tema oma, ta eluaseme peale puistatakse väävlit.
\par 16 Temal kuivavad juured alt ja närtsivad oksad pealt.
\par 17 Mälestus temast kaob maalt ja ta nime ei nimetata tänaval.
\par 18 Ta tõugatakse valgusest pimedusse ja aetakse maailmast ära.
\par 19 Temale ei jää järglast ega sugu oma rahva seas, ja mitte ühtegi pääsenut sealt, kus ta viibis.
\par 20 Inimesed läänes ehmuvad tema hukatuspäevast ja inimesi idas haarab hirm.
\par 21 Tõesti, nõnda sünnib ülekohtutegija hoonega ja nõnda selle paigaga, kes Jumalat ei tunne.”

\chapter{19}

\par 1 Siis rääkis Iiob ja ütles:
\par 2 „Kui kaua te piinate mu hinge ja jahvatate mind sõnadega?
\par 3 Te mõnitate mind juba kümnendat korda häbenematult mulle peale käies.
\par 4 Ja kui ma ka tõesti oleksin eksinud, siis jääks mu eksimus ainult mulle.
\par 5 Kui te tõesti mu ees tahate suurustada ja mulle mu alandust ette heita,
\par 6 siis teadke, et Jumal on mind maha paisanud ja piiranud mind oma võrguga.
\par 7 Vaata, ma kisendan: „Vägivald!”, aga ei saa vastust; hüüan appi, aga õigust ei ole.
\par 8 Ta tegi mu teele tõkke ja ma ei pääse üle, ta pani mu radade peale pimeduse.
\par 9 Ta riisus minult au ja võttis mul krooni peast.
\par 10 Ta kiskus mind igapidi maha, et kaoksin, ja juuris mu lootuse välja nagu puu.
\par 11 Ta süütas oma viha põlema mu vastu ja pidas mind oma vaenlaseks.
\par 12 Tema väesalgad tulid üheskoos, rajasid tee mu juurde ja lõid leeri üles mu telgi ümber.
\par 13 Mu vennad hoidis ta minust eemale ja mu tuttavad võõrdusid minust hoopis.
\par 14 Mu lähedased jätsid mind maha ja mu sõbrad unustasid mind ära.
\par 15 Mu kodakondsed ja teenijad peavad mind võõraks - ma olen nende silmis otsekui muulane.
\par 16 Ma hüüan oma sulast, aga ta ei vasta, ma pean teda anuma, nagu mu suu võtab.
\par 17 Mu naisele ei meeldi mu hingeõhk ja oma lihastele vendadele olen ma vastik.
\par 18 Poisidki põlgavad mind; kui ma tõusen, siis nad räägivad mulle vastu.
\par 19 Mind jälestavad kõik mu lähemad sõbrad, ja need, keda ma armastasin, on pöördunud mu vastu.
\par 20 Mu luud on jäänud kinni naha ja liha külge, mu kondid tungivad välja nagu hambad.
\par 21 Halastage mu peale, halastage, mu sõbrad, sest mind on tabanud Jumala käsi!
\par 22 Miks ajate teiegi mind taga nagu Jumal? Kas te ei küllastu mu lihast?
\par 23 Oh, et mu sõnad ometi kirja pandaks, et need raamatusse kirjutataks,
\par 24 raudsule ja tinaga uurendataks kaljusse igaveseks ajaks!
\par 25 Sest ma tean, et mu Lunastaja elab, ja tema jääb viimsena põrmu peale seisma.
\par 26 Ja kuigi mu nahka on nõnda nülitud, saan ma ilma ihutagi näha Jumalat,
\par 27 teda, keda ma ise näen, keda näevad mu oma silmad, aga mitte mõne võõra. Mul kõdunevad neerud sisikonnas.
\par 28 Kui te mõtlete: „Me ajame teda taga, asja juur leidub temas”,
\par 29 siis kartke mõõka, sest viha toob mõõka väärt süüteod, et te teaksite: kohus on olemas!”

\chapter{20}

\par 1 Siis rääkis naamalane Soofar ja ütles:
\par 2 „Selle peale mu rahutud mõtted tulevad tagasi ja sellepärast tormitseb mu sees.
\par 3 Ma pean kuulma häbistavat noomimist ja saan vastuseks tühja tuult.
\par 4 Kas sa ei tea juba muistsest ajast, sellest ajast, kui inimene maa peale pandi,
\par 5 et õelate hõiskamine on üürike ja jumalatu rõõm ainult hetkeline?
\par 6 Kuigi ta kõrgus tõuseks taevani ja ta pea puudutaks pilvi,
\par 7 ta kaob, nagu ta roegi, igaveseks; need, kes teda nägid, küsivad: „Kus ta on?”
\par 8 Ta lendab ära otsekui unenägu ja teda ei leita enam, ta haihtub nagu öine nägemus.
\par 9 Silm, mis teda nägi, ei näe teda enam, ja ta ase ei pane enam teda tähele.
\par 10 Tema lapsed peavad vaeseid hüvitama ja tema enese käed ta varanduse tagasi andma.
\par 11 Ta kondid on küll täis noorusjõudu, aga ta heidab koos sellega põrmu magama.
\par 12 Kuigi kurjus on ta suus nõnda magus, et ta peidab selle oma keele alla,
\par 13 kuigi ta säästab seda ega loobu sellest, vaid hoiab seda keset suulage,
\par 14 muutub ometi tema roog ta kõhus madude mürgiks ta sisikonnas.
\par 15 Ta peab oksendama neelatud varandust - Jumal ajab selle ta kõhust välja.
\par 16 Ta imes madude mürki, ussi keel tapab ta.
\par 17 Ei saa ta näha ojasid, mee ja piima voolude jõgesid.
\par 18 Ta peab oma töövilja ära andma ega tohi ise seda neelata, ja oma kaubakasust ei tunne ta rõõmu.
\par 19 Sest ta murdis, jättis maha vaesed, röövis endale koja, mida ta polnud ehitanud.
\par 20 Sest ta ei tundnud küllastust kõhus - aga oma kalliste asjadega ei päästa ta ennast.
\par 21 Ükski ei pääsenud tema neelamisest, seepärast ta õnn ei kesta.
\par 22 Tema ülikülluseski tuleb temale kitsas kätte, teda tabab õnnetuse kogu jõud.
\par 23 Et tema kõhtu täita, läkitab Jumal temasse oma tulise viha ja laseb seda sadada tema peale ta toiduga.
\par 24 Kui ta põgeneb raudrelva eest, siis laseb vaskamb temast läbi,
\par 25 viskoda tuleb välja seljast ja mõõgatera sapist, ja temale tulevad hirmuvärinad peale.
\par 26 Suur pimedus on temale varaks pandud, teda neelab õhutamata tuli; mis tema telki on alles jäänud, hävitatakse.
\par 27 Taevad ilmutavad tema süüd ja maa tõuseb tema vastu.
\par 28 Tema koja uhub vihmavaling, ta vihapäeva uputusvesi.
\par 29 See on õela inimese osa Jumalalt, pärisosa, mis Jumal temale määrab.”

\chapter{21}

\par 1 Siis rääkis Iiob ja ütles:
\par 2 „Kuulge ometi mu sõnu ja see olgu mulle troostiks!
\par 3 Olge minuga kannatlikud, siis ma räägin, ja kui olen rääkinud, võite irvitada!
\par 4 Kas ma kaeban inimese peale? Ja miks ei peakski mu vaim muutuma kannatamatuks?
\par 5 Vaadake minu poole, siis te ehmute ja panete käe suu peale.
\par 6 Sest kui ma sellele mõtlen, siis ma jahmun ja värin haarab mu ihu.
\par 7 Miks jäävad õelad elama, saavad vanaks, võtavad isegi jõudu juurde?
\par 8 Nende sugu seisab kindlana nendega nende ees ja nende järglased on nende silma all.
\par 9 Nende kojad on säästetud hirmust ja Jumala vitsa pole nende peal.
\par 10 Nende sõnn kargab, ja mitte asjata, nende lehmad poegivad loodet heitmata.
\par 11 Nad lasevad oma lapsukesi joosta nagu lambaid ja nende noorukid tantsivad.
\par 12 Nad laulavad trummi ja kandle saatel ning tunnevad rõõmu vilepilli häälest.
\par 13 Nad veedavad oma päevi õnnes ja lähevad rahus alla surmavalda.
\par 14 Nad ütlevad Jumalale: „Tagane meist, sest sinu teede tundmiseks pole meil lusti!
\par 15 Kes on Kõigevägevam, et peaksime teda teenima? Ja mis kasu meil on, kui me ta poole palvetame?”
\par 16 Vaata, eks ole nende õnn nende endi käes, õelate nõu minust kaugel?
\par 17 Kui sageli siis kustub õelate lamp ja tabab neid õnnetus? Kui sageli ta jagab oma vihas hukatust,
\par 18 et nad oleksid nagu õled tuules, otsekui aganad, mida tuulekeeris hajutab?
\par 19 Jumal talletavat õela süü tema laste jaoks. Ta tasugu temale enesele, nõnda et ta tunneb!
\par 20 Nähku ta oma silmad tema langust ja ta ise joogu Kõigevägevama viha!
\par 21 Tõesti, ei ole siis rõõmu ta kojal pärast teda, kui ta kuude arv on napiks mõõdetud.
\par 22 Aga kas võiks Jumalale tarkust õpetada, temale, kes taevalistelegi kohut mõistab?
\par 23 Üks sureb oma täies elujõus, kõigiti rahulikult ja muretult,
\par 24 reied lihavad ja kondid täis üdi.
\par 25 Teine sureb kibestunud hingega, õnne maitsta saamata.
\par 26 Nad magavad üheskoos põrmus ja ussikesed katavad neid.
\par 27 Vaata, ma tean teie mõtteid ja riukaid, mis te minu vastu sepitsete.
\par 28 Sest te küsite: „Kus on siis nüüd see võimumehe koda? Ja kus on telk, milles õelad elasid?”
\par 29 Kas te ei ole küsinud teekäijailt ega ole tähele pannud nende märguandeid,
\par 30 et kurjale antakse armu õnnetusepäeval ja ta päästetakse vihapäeval?
\par 31 Kes kuulutaks temale näkku ta käitumist ja kes tasuks temale, mis ta on teinud?
\par 32 Ja kui ta hauda viiakse, siis hoolitsetakse isegi ta kääpa eest.
\par 33 Oru kivipangadki on temale magusad. Tema järele lähevad kõik inimesed, ja enne teda läinuid on arvutult.
\par 34 Kuidas te siis mulle toote nõnda tühist troosti? Ja teie vastused - neist jääb järele ainult vale.”

\chapter{22}

\par 1 Siis rääkis teemanlane Eliifas ja ütles:
\par 2 „Kas Jumalal on inimesest kasu? Ei, aga mõistlik mees toob kasu iseenesele.
\par 3 Kas Kõigevägevamal on head sellest, et sa oled õiglane, või kasu, kui su eluviisid on laitmatud?
\par 4 Kas ta noomib sind su jumalakartuse pärast, ja läheb sellepärast sinuga kohtusse?
\par 5 Kas mitte su kurjus pole suur ja su süüteod otsatud?
\par 6 Sest sa võtsid oma vendadelt panti ilmaaegu ja riisusid alastiolijailt riided.
\par 7 Sa ei andnud väsinule juua ja keelasid näljasele leiba.
\par 8 Aga rusikamees - tema päralt oli maa ja ainult armualune võis seal elada.
\par 9 Sa saatsid lesknaised ära tühje käsi ja vaeslaste käsivarred murti.
\par 10 Seepärast on nüüd su ümber silmused ja äkiline hirm kohutab sind.
\par 11 Või on pimedus, sa ei näe enam, ja veetõus katab sind.
\par 12 Eks ole Jumal kõrgel taevas? Ja vaata ülemaid tähti - kui kõrgel need on!
\par 13 Seepärast sa mõtled: „Mis teab Jumal? Kas ta võib kohut mõista läbi suure pimeduse?
\par 14 Pilved on tal varjuks ees, ei ta siis näe, ta ju kõnnib taevavõlvil.”
\par 15 Kas tahad hoida vana teed, mida on käinud pahad inimesed,
\par 16 need, keda enneaegselt haarati, kelle aluse uhtus vool,
\par 17 kes Jumalale ütlesid: „Tagane meist!„ ja veel: ”Mida võiks Kõigevägevam meile teha?”
\par 18 Ja ometi oli tema täitnud nende kojad õnnistusega ja õelate nõu oli minust kaugel.
\par 19 Õiglased nägid seda ja rõõmustasid, ja süütu irvitas nende üle:
\par 20 „Tõesti, meie vastased on hävitatud, ja mis neist järele jäi, selle sõi tuli.”
\par 21 Saa siis Jumalaga sõbraks ja ole rahul, nõnda sa saavutad õnne!
\par 22 Võta ometi tema suust õpetust ja pane tema sõnad oma südamesse!
\par 23 Kui sa pöördud tagasi Kõigevägevama juurde, sind ehitatakse üles; kui sa saadad ülekohtu oma telgist kaugele,
\par 24 viskad kulla põrmu peale ja Oofiri kulla ojakivide sekka,
\par 25 siis on Kõigevägevam sulle kullaks ja puhtaimaks hõbedaks,
\par 26 sest siis sa võid rõõmu tunda Kõigevägevamast ja tõsta oma palge Jumala poole.
\par 27 Kui sa ta poole palvetad, siis ta kuuleb sind, ja sina saad tasuda oma tõotused.
\par 28 Kui sa midagi ette võtad, siis läheb see sul korda ja sinu teede peale paistab valgus.
\par 29 Kui kedagi alandatakse, ütleb ta: „Üles!” ja ta aitab neid, kes silmad maha löövad.
\par 30 Ta päästab selle, kes ei ole süütu, ja ta päästetakse su käte puhtuse pärast.”

\chapter{23}

\par 1 Siis rääkis Iiob ja ütles:
\par 2 „Mu kaebus on tänagi mässuline. Minu käsi lasub raskesti mu ohkamise kohal.
\par 3 Oh, kui ma teaksin, kuidas ma teda leian, kuidas ma saaksin minna tema aujärje ette!
\par 4 Siis ma paneksin oma kohtuasja tema ette ja täidaksin oma suu vastulausetega.
\par 5 Ma tahaksin teada sõnu, millega ta mulle vastab, ja mõista, mis tal mulle on öelda.
\par 6 Kas ta oma kõikvõimsuses peaks minuga vaidlema? Ei, hea, kui ta mind tähelegi paneb.
\par 7 Siis võiks õiglane temaga arutada ja ma pääseksin oma kohtumõistjast igaveseks.
\par 8 Vaata, kui ma lähen itta, siis ei ole teda seal; või läände, siis ma teda ei märka.
\par 9 Kui ta on põhjas tegutsemas, siis ma teda ei silma; kui ta pöörab lõunasse, siis ma teda ei näe.
\par 10 Kuid tema tunneb teed, mida ma käin. Katsub ta mind läbi - ma tulen sellest välja nagu kuld.
\par 11 Mu jalg on püsinud tema jälgedes, ma olen pidanud tema teed ega ole kõrvale läinud.
\par 12 Tema huulte käskudest ei ole ma taganenud, ma olen oma põues talletanud sõnad tema suust.
\par 13 Aga tema on ainus ja kes saaks teda keelata? Mida tema hing ihaldab, seda ta ka teeb.
\par 14 Sest ta viib täide, mis mulle on määratud, ja seesugust on tal palju.
\par 15 Seepärast ma tunnen hirmu tema palge ees, ja kui ma järele mõtlen, siis ma kardan teda.
\par 16 Jumal on teinud araks mu südame ja Kõigevägevam on see, kes mind hirmutab.
\par 17 Kas ma pole hävinud pimeduse pärast, ja pilkane pimedus katab mu palet?

\chapter{24}

\par 1 Miks ei ole Kõigevägevam talletanud karistusaegu, ja miks need, kes teda tunnevad, ei saa näha tema päevi?
\par 2 Piirimärke nihutatakse, karja riisutakse ja karjatatakse kui oma.
\par 3 Vaeslaste eesel viiakse ära, lesknaise härg võetakse pandiks.
\par 4 Vaesed tõugatakse teelt kõrvale, kõik maa viletsad peavad peitu pugema.
\par 5 Vaata, otsekui metseeslid kõrbes lähevad nad oma tööle toidust otsima: lagendik on leib tema lastele.
\par 6 Väljal peavad nad lõikama seda, mis pole nende oma, ja koristama õela viinamäge.
\par 7 Nad peavad ööbima alasti, riieteta, ja külmas olema katteta.
\par 8 Mägede äikesevihmast saavad nad märjaks, ja olles varjupaigata, liibuvad nad vastu kaljut.
\par 9 Tissi otsast riisutakse vaeslaps ja vaese vastu võetakse panti.
\par 10 Nad käivad alasti, riieteta, ja kannavad näljastena viljavihke.
\par 11 Müüride vahel nad peavad suruma õli, sõtkuma surutõrt, aga kannatama janu.
\par 12 Rahva linnades ägatakse ja haavatute hing hüüab appi; ent Jumal ei pane tähele jõledust.
\par 13 On neid, kes on valguse vastased, kes ei taha tunda selle teid ega jääda selle radadele.
\par 14 Koiduajal tõuseb mõrvar üles, et tappa viletsat ja vaest, ja öösel on ta nagu varas.
\par 15 Abielurikkuja silm ootab hämarust, mõeldes: „Ükski silm ei näe mind!” ja paneb katte näole.
\par 16 Pimeduses murtakse kodadesse: päeval nad redutavad kodus, nad ei taha valgusest teada.
\par 17 Sest sügav pimedus on neile kõigile hommikuks ja pimeduse hirmud on nende sõbrad.
\par 18 Aga päeva koites on neil kiire. Neetud olgu maal nende põld, ükski surutõrre sõtkuja ärgu pöördugu nende viinamäele!
\par 19 Põud ja kuumus võtavad lumeveed, surmavald võtku patustajad!
\par 20 Üsk unustab ta, uss mõnuleb temast, teda ei mäletata enam: nõnda murtakse ülekohus nagu puu.
\par 21 Ta vaevab lastetut, kes pole sünnitanud, ja ei tee head lesknaisele.
\par 22 Ent Jumal oma jõuga pikendab vägivaldsete iga: nad tõusevad üles, isegi kui neil ei ole enam lootustki eluks.
\par 23 Ta annab neile julgeoleku ja nad saavad sellest tuge; aga ta silmad on nende teede peal.
\par 24 Nad on tõusnud üürikeseks - ja siis ei ole neid enam. Neid alandatakse, neid kistakse nagu kõiki muidki ja nad lõigatakse ära otsekui viljapead.
\par 25 Eks ole ju nõnda? Kes tahaks teha mind valelikuks ja mu sõnad tühjaks?”

\chapter{25}

\par 1 Siis rääkis suhiit Bildad ja ütles:
\par 2 „Valitsus ja hirm on temal, kes oma kõrgustes rahu loob.
\par 3 Kas on määra tema väehulkadel? Ja kellele tema valgus ei tõuse?
\par 4 Kuidas võiks siis inimene olla õige Jumala ees? Ja kuidas võiks naisest sündinu olla puhas?
\par 5 Vaata, isegi kuu ei hiilga ja tähedki pole tema silmis selged,
\par 6 veel vähem siis inimene, see ussike, inimlaps, see vaglake.”

\chapter{26}

\par 1 Siis rääkis Iiob ja ütles:
\par 2 „Kuidas sa küll oled aidanud rammetut, toetanud jõuetut käsivart!
\par 3 Kuidas sa küll oled rumalale nõu andnud ja suurt tarkust õpetanud!
\par 4 Kelle abiga sa oled pidanud kõnesid ja kelle vaim on sinust välja tulnud?
\par 5 Surnute vaimud all värisevad, hirmu täis on veed ja nende elanikud.
\par 6 Paljas on surmavald Jumala ees ja kadupaigal pole katet.
\par 7 Ta laotab põhjakaare tühjuse üle ja riputab maa eimillegi kohale.
\par 8 Ta seob veed oma pilvedesse ja pilv ei rebenegi nende raskuse all.
\par 9 Ta peidab oma aujärje, laotab selle üle oma pilve.
\par 10 Ta joonistab vetepinnale sõõri kuni valguse ja pimeduse piirini.
\par 11 Taeva sambad kõiguvad ja kohkuvad tema sõitlusest.
\par 12 Oma rammuga liigutab ta merd ja oma taibukuses peksab ta Rahabit.
\par 13 Tema hingusest selgib taevas, tema käsi torkab läbi põgeneva mao.
\par 14 Vaata, need on ainult tema tee ääred. Ja see on siiski ainult sosin, mida me temast kuuleme. Aga tema vägevuse äikest - kes seda suudaks mõista?”

\chapter{27}

\par 1 Ja Iiob jätkas oma kõnet ning ütles:
\par 2 „Nii tõesti kui elab Jumal, kes võttis ära mu õiguse, ja Kõigevägevam, kes kibestas mu hinge;
\par 3 niikaua kui minus on veel hingeõhku ja mu ninas Jumala hõngu,
\par 4 ei räägi mu huuled valet ega kõnele mu keel pettust.
\par 5 Jäägu see minust kaugele, et annaksin teile õiguse! Kuni ma pole hinge heitnud, ei loobu ma oma vagadusest.
\par 6 Ma hoian kinni oma õigusest ega jäta seda, mu süda ei laida mind ühegi mu elupäeva pärast.
\par 7 Käigu mu vaenlasel käsi nagu õelal ja mu vastasel otsekui kurjategijal!
\par 8 Sest mis lootus on jumalatul, kui Jumal ta ära lõikab, kui ta tema hinge nõuab?
\par 9 Kas Jumal peaks kuulma ta kisendamist, kui talle kitsas kätte tuleb?
\par 10 Kas ta tunneb rõõmu Kõigevägevamast? Kas ta hüüab Jumalat appi igal ajal?
\par 11 Ma tahan teile õpetada Jumala kätt; mida Kõigevägevam mõtleb, seda ma ei salga.
\par 12 Vaata, te kõik olete ise seda näinud, mispärast te siis lobisete tühja?
\par 13 See on õela inimese osa Jumala juures, ja pärisosa, mille vägivalla teostajad saavad Kõigevägevamalt:
\par 14 kui tal on palju lapsi, siis on need mõõga jaoks, ja tema järglastel ei ole küllalt leiba.
\par 15 Kes temale järele jäävad, need matab katk, ja ta lesknaised ei saa neid taga nutta.
\par 16 Kui ta hõbedat kokku kuhjab nagu põrmu ja riideid varub nagu savi,
\par 17 siis varugu; aga õige paneb need selga ja süütu pärib hõbeda.
\par 18 Ta ehitab oma koja riidekoi eeskujul, hüti sarnaselt, mille vahimees teeb.
\par 19 Rikas on ta magama heites, kuid mitte kauem; kui ta silmad avab, siis ei ole tal enam midagi.
\par 20 Suur hirm saab ta kätte nagu voolas vesi ja öösel viib tuulekeeris ta kaasa.
\par 21 Idatuul tõstab ta üles ja ta läheb - see pühib ta ära tema asupaigast.
\par 22 Tema peale visatakse ilma armuta kive ja ta peab põgenema nende käe eest.
\par 23 Tema pärast plaksutatakse käsi, ja paigast, kus ta asus, vilistatakse temale järele.

\chapter{28}

\par 1 Tõesti, hõbedal on leiukoht ja kullal pesemispaik,
\par 2 raud võetakse mullast ja kivist valatakse vask.
\par 3 Piir pannakse pimedusele ning viimseni otsitakse üles pimeduse ja sünguse kivi.
\par 4 Kaevandusi murtakse eluasemeist eemal, paigus, mille on unustanud jalad; nad kõlguvad kõikudes, inimestest kaugel.
\par 5 Maast kasvab leib, aga maa sügavused pööratakse tulega pahupidi.
\par 6 Selle kivimid on safiiride asukohaks ja seal on kullatolmu.
\par 7 Teed sinna ei tea kotkaski ega märka kulli silm.
\par 8 Seda ei talla uhked metsloomad, seal ei kõnni noor lõvigi.
\par 9 Ränikivi külge pistetakse käsi, mäed pööratakse juurteni ümber.
\par 10 Kaljudesse lõhutakse käike ja silm saab näha kõiksugu aardeid.
\par 11 Vooluste nõrgumiskohad tõkestatakse, ja mis peidetud on, tuuakse valguse kätte.
\par 12 Aga tarkus - kust seda leitakse? ja kus on arukuse asupaik?
\par 13 Ei tunne inimene selle hinda ja seda ei leidu elavate maal.
\par 14 Sügavus ütleb: „Minu sees seda pole.„ Ja meri ütleb: ”Ei ole minu juures.”
\par 15 Puhta kulla eestki ei saa seda osta ega hõbedaga selle hinda vaagida.
\par 16 See pole makstav Oofiri kullaga, ei kalli oonüksi ega safiiriga.
\par 17 Sellega pole võrreldav ei kuld ega klaas, ka pole see vahetatav kuldriista vastu.
\par 18 Koralle ja mägikristalli ei maksa mainidagi - omandatud tarkus on enam väärt kui pärlid.
\par 19 Sellega pole võrreldav Etioopia krüsoliit, puhta kullagagi ei saa selle eest maksta.
\par 20 Kust tuleb siis tarkus ja kus on arukuse asupaik?
\par 21 See on varjul kõigi elavate silmade eest, peidetud taeva lindudegi eest.
\par 22 Kadupaik ja surm ütlevad: „Meie kõrvus on sellest ainult kuuldus.”
\par 23 Jumal, tema tunneb teed sinna ja tema teab selle asupaika.
\par 24 Sest tema ulatub vaatama maa äärteni, tema näeb kõike, mis taeva all on.
\par 25 Kui ta andis tuulele kaalu ja määras mõõduga veed,
\par 26 kui ta andis vihmale seaduse ja kõuepilvele tee,
\par 27 siis ta nägi seda ja seletas, valmistas ta ja uuris läbi.
\par 28 Ja ta ütles inimesele: „Vaata, Issanda kartus - see on tarkus, ja hoidumine kurjast on arukus!”

\chapter{29}

\par 1 Ja Iiob jätkas oma kõnet ning ütles:
\par 2 „Kes annaks mulle tagasi endised kuud, need päevad, mil Jumal mind hoidis,
\par 3 kui oma lampi mu pea kohal laskis paista tema, kelle valgusega ma käisin pimeduses,
\par 4 et ma võiksin olla nagu oma nooruspäevil, mil Jumala osadus oli mu telgi peal,
\par 5 kui Kõigevägevam oli alles mu juures ja mu lapsed viibisid mu ümber,
\par 6 kui mu sammud ujusid piimas ja kalju laskis mulle voolata õliojasid?
\par 7 Kui ma siis läksin linna värava juurde, kui ma seadsin oma istme turu peale,
\par 8 siis mind nähes pugesid noored mehed peitu ja elatanud tõusid üles ning jäid seisma,
\par 9 pealikud lakkasid kõnelemast ja panid käe suu peale,
\par 10 vürstide hääl vaikis ja nende keel kleepus suulakke.
\par 11 Tõesti, kelle kõrv mind kuulis, see kiitis mind õnnelikuks, ja kelle silm mind nägi, see tunnistas minu kasuks,
\par 12 sest ma päästsin viletsa, kes appi hüüdis, ja vaeslapse, kellel ei olnud aitajat.
\par 13 Mulle sai osaks hukkuja õnnistus ja ma panin hõiskama lese südame.
\par 14 Ma riietusin õiglusesse, ja mu õigus ehtis mind nagu kuub ja kübar.
\par 15 Ma olin pimedale silmadeks ja jalutule jalgadeks.
\par 16 Ma olin vaestele isaks ja ma uurisin isegi tundmatu tüliasja.
\par 17 Ma purustasin ülekohtutegija lõualuud ja tõmbasin saagi ta hammaste vahelt.
\par 18 Seepärast ma mõtlesin: „Küllap ma heidan hinge oma pesas ja mu päevade hulk on nagu liiv.
\par 19 Mu juur jääb avatuks veele ja mu okste peal on öösiti kaste.
\par 20 Mu au on alati uus ja amb mu käes on ikka laskevalmis.”
\par 21 Nad kuulasid mind ja ootasid, ning vaikisid, kui ma nõu andsin.
\par 22 Pärast mu kõnet nad ei rääkinud enam, sest mu sõnad otse voolasid nende peale.
\par 23 Nad ootasid mind nagu vihma ja ajasid suud ammuli otsekui hilisvihma pärast.
\par 24 Ma naeratasin neile, kui neil puudus usk, ja nad ei tumestanud mu lahket nägu.
\par 25 Mina valisin neile tee ja istusin ise esikohal, elasin nagu kuningas väehulga keskel, otsekui leinajate trööstija.

\chapter{30}

\par 1 Aga nüüd naeravad mind need, kes elupäevilt on minust nooremad, kelle isasid ma ei arvanud väärt panna oma karjakoerte sekka.
\par 2 Mis kasu oleks mul isegi nende käte rammust, kui neil elujõudki on kadunud,
\par 3 kui nad puudusest ja näljast kurnatuina närivad puhtaks isegi põuase maa, kus eile oli laastamine ja hävitus?
\par 4 Nad nopivad põõsaste juurest soolaheina ja leetpõõsa juur on neile leivaks.
\par 5 Nad on teiste hulgast ära aetud, nende peale karjutakse nagu varga peale.
\par 6 Nad peavad elama orunõlvades, muld- ja kaljukoobastes.
\par 7 Nad karjuvad põõsaste vahel, nad kogunevad kureherneste alla -
\par 8 jõledad inimesed, nimetu rahva lapsed, kes on maalt välja aetud.
\par 9 Ja nüüd olen mina saanud neile pilkelauluks, pean olema neile sõnakõlksuks.
\par 10 Nad jälestavad mind, hoiduvad minust ega jäta mulle näkku sülitamata.
\par 11 Sest Jumal vallandas mu ammunööri ja alandas mind, ja nemad heitsid ohjad mu ees ära.
\par 12 Paremalt poolt tõuseb see rämps: nad löövad mul jalad alt ja valmistavad minu jaoks oma hukatusteid.
\par 13 Nad rikuvad mu jalgraja, aitavad kaasa mu hukkumiseks, ükski ei hoia neid tagasi.
\par 14 Nad tulevad nagu läbi laia prao, veeretavad endid sisse tormi ajal.
\par 15 Mu kallale on asunud suur hirm: mu väärikus on otsekui tuulest viidud ja mu õnn on nagu möödunud pilv.
\par 16 Ja nüüd on hing mu sees valatud tühjaks, viletsuspäevad on mind kätte saanud.
\par 17 Öösiti pistab mul kontides ja mu valud ei pea vahet.
\par 18 Suure jõuga haarab ta mind rõivaist; see pigistab mind otsekui kuuekaelus.
\par 19 Jumal on heitnud mind savi sisse, ma olen saanud põrmu ja tuha sarnaseks.
\par 20 Ma kisendan su poole, aga sa ei vasta mulle, ma seisan siin, aga sa ainult silmitsed mind.
\par 21 Sa oled muutunud julmaks mu vastu, sa kiusad mind oma vägeva käega.
\par 22 Sa tõstad mind tuule kätte, lased mind ajada, tormihood lohistavad mind.
\par 23 Tõesti, ma tean, et sa saadad mind surma - kõigi elavate kogunemiskotta.
\par 24 Eks rusu all olija siruta käed, eks hukkuja karju appi?
\par 25 Kas mina ei nutnud selle pärast, kellel oli raske aeg? Kas mul ei olnud kaastunnet vaesele?
\par 26 Aga kui ma ootasin head, tuli õnnetus, ja kui ma igatsesin valgust, tuli pimedus.
\par 27 Mu sisemus keeb ega rahune, mind on tabanud viletsuspäevad.
\par 28 Ma käin mustana, aga mitte päikesest, tõusen koguduses üles ja karjun appi.
\par 29 Ma olen ðaakalitele vennaks ja jaanalindudele seltsiliseks.
\par 30 Mu ihunahk on muutunud mustaks ja mu kondid hõõguvad kuumusest.
\par 31 Mu kandlemängust kujunes lein ja mu vilepilliloost nutjate hääl.

\chapter{31}

\par 1 Ma andsin ju oma silmadele seaduse. Kuidas ma võiksin siis vaadata neitsi poole?
\par 2 Sest mis oleks muidu mu osa Jumalalt ülal ja mu pärisosa Kõigevägevamalt kõrgustes?
\par 3 Eks hukatus tule ülekohtutegijale ja õnnetus sellele, kes nurjatust teeb?
\par 4 Eks tema näe mu teed ja loe kõiki mu samme?
\par 5 Kui ma olen valega kaasas käinud või on mu jalg tõtanud pettuse poole,
\par 6 vaetagu mind õigetel vaekaussidel - siis saab Jumal teada mu ausust!
\par 7 Kui mu sammud on kaldunud teelt kõrvale ja mu süda on järgnenud silmadele ja kui mulle on midagi pihku jäänud,
\par 8 siis söögu keegi teine, mida ma külvan, ja mis mulle võrsub, juuritagu välja!
\par 9 Kui mu süda on lasknud ennast ahvatleda mõne naise poole ja ma olen varitsenud oma ligimese ukse taga,
\par 10 siis jahvatagu mu naine võõrale ja heitku teised tema peale!
\par 11 Sest see oleks olnud häbitegu ja kohut väärt süü,
\par 12 tuli, mis põletab kadupaigani ja hävitab juurteni kõik mu saagi.
\par 13 Kui ma oleksin põlanud oma sulase ja teenija õigust, kui neil oli minuga tüli,
\par 14 mis ma oleksin teinud siis, kui Jumal oleks tõusnud? Ja kui ta oleks katsuma tulnud, mis ma siis temale oleksin kostnud?
\par 15 Eks minu looja ole loonud emaihus ka teda, eks ole üks ja seesama meid emaüsas valmistanud?
\par 16 Kas ma olen tagasi lükanud vaeste soove või lasknud tuhmuda lesknaise silmi?
\par 17 Kas ma üksinda olen söönud oma palukest, ilma et vaeslaps oleks saanud sellest süüa?
\par 18 Sest ta kasvas mu noorusest peale minuga kui oma isaga, ja emaihust alates olen ma teda juhatanud.
\par 19 Kas ma olen võinud näha kedagi hukkuvat riiete puudusel või vaest ilma katteta?
\par 20 Eks ta niuded õnnistanud mind ja eks ta saanud mu lambaniidust sooja?
\par 21 Kui ma olen tõstnud käe vaeslapse vastu, kuna ma väravas nägin enesele abi,
\par 22 siis langegu mu õlg liigesest välja ja murdugu mu käsivars õlavarreluust saadik!
\par 23 Sest mul oli hirm Jumalalt tuleva karistuse ees, ja tema ülevõimu vastu ei oleks ma suutnud püsida.
\par 24 Kas ma panin oma lootuse kullale või ütlesin ma kalleimale kullale: „Ma usaldan sind!”?
\par 25 Kas ma rõõmustasin, et mu varandus oli suur või et mu käsi saavutas väga palju?
\par 26 Kui ma oleksin vaadanud päikest, kuidas see paistab, või kuud, mis nii suurepäraselt rändab,
\par 27 kui mu süda seejuures oleks lasknud ennast salaja petta, nõnda et oleksin saatnud neile käesuudlusi,
\par 28 siis oleks seegi olnud kohut väärt süü, sest ma oleksin siis ju salanud Jumala, kes on ülal.
\par 29 Kas ma olen rõõmustanud oma vihamehe hukkumise pärast või hõisanud, kui teda tabas õnnetus?
\par 30 Ei ole ma lubanud oma suud pattu teha, et oleksin sajatades nõudnud tema hinge.
\par 31 Eks ole inimesed mu telgis öelnud: „Kes ei oleks temalt liha küllalt saanud?”
\par 32 Võõras ei ole pidanud ööbima tänaval, ma avasin teekäijale oma ukse.
\par 33 Kas ma oleksin nagu Aadam varjanud oma üleastumisi, oma süütegu põue peites,
\par 34 sellepärast et pidin kartma suurt rahvahulka ja et mind hirmutas suguvõsa põlgus, nõnda et oleksin pidanud vaikima, ilma uksestki välja minemata?
\par 35 Oh, et mul oleks keegi, kes mind kuuleks! Siin on mu nimemärk - Kõigevägevam kostku mulle! Oleks mul ometi vastase kirjutatud süüdistuskiri!
\par 36 Tõesti, ma tõstaksin selle õlale, paneksin selle enesele krooniks.
\par 37 Ma teeksin temale teatavaks kõik oma sammud, astuksin tema ette nagu vürst.
\par 38 Kui mu põld mu vastu kisendab ja selle vaod üheskoos nutavad,
\par 39 et ma olen söönud ta saaki ilma tasuta ja olen pannud ta õiged omanikud hingeldama,
\par 40 siis kasvatagu ta nisu asemel kibuvitsu ja odra asemel umbrohtu.” Iiobi sõnad on lõppenud.

\chapter{32}

\par 1 Kui need kolm meest loobusid Iiobile vastamast, sellepärast et ta oma silmis õige oli,
\par 2 siis süttis põlema buuslase Eliihu, Baarakeli poja viha; ta oli Raami suguvõsast. Ja ta viha süttis põlema Iiobi vastu, sellepärast et too pidas ennast Jumalast õigemaks.
\par 3 Ta viha süttis põlema ka tema kolme sõbra vastu, sellepärast et need ei olnud leidnud vastust, millega nad oleksid saanud Iiobi süüdi mõista.
\par 4 Eliihu oli oodanud, kuni nad Iiobiga rääkisid, sest nad olid temast vanemad.
\par 5 Aga kui Eliihu nägi, et nende kolme mehe suus ei olnud vastust, siis ta viha süttis põlema.
\par 6 Ja buuslane Eliihu, Baarakeli poeg, rääkis ning ütles: „Mina olen ealt noorim, aga teie olete elatanud, sellepärast ma olin tagasihoidlik ja kartsin teile kuulutada oma arvamust.
\par 7 Ma mõtlesin: Kõnelgu päevad ja aastate rohkus õpetagu tarkust.
\par 8 Tõesti, inimeste sees olev vaim, Kõigevägevama hingeõhk annab neile arukuse.
\par 9 Ei ole eakad mitte alati targemad ega mõista üksnes vanad, mis on õige.
\par 10 Seepärast ma ütlen: Kuule mind, minagi tahan kuulutada oma arvamust.
\par 11 Vaata, ma ootasin, mis teil on öelda, ma kuulatasin teie tarkust, siis kui te otsisite sõnu.
\par 12 Ma panin siis teid tähele, ja vaata, ükski teist ei kutsunud Iiobit korrale, ükski ei vastanud tema sõnadele.
\par 13 Ärge seepärast mõtelge: „Meie oleme leidnud tarkuse.” Jumal ajab ta minema, mitte inimene.
\par 14 Minu vastu ei ole ta oma sõnu seadnud ja teie sõnadega ei taha ma temale vastata.
\par 15 Nad on ehmunud, nad ei vasta enam, neil on sõnad lõppenud.
\par 16 Kas pean ootama, sellepärast et nad ei räägi, et nad seisavad siin ega vasta enam?
\par 17 Ma tahan anda vastuseks omaltki poolt osa, ma tahan kuulutada ka oma arvamust.
\par 18 Sest ma olen täis sõnu, vaim mu sees sunnib mind.
\par 19 Vaata, mu sees on nagu vein, millel puudub väljapääs: otsekui uus nahkastja on see lõhkemas.
\par 20 Ma pean rääkima, et saada enesele õhku, pean avama huuled ja vastama.
\par 21 Ma ei taha nüüd olla erapoolik ega kedagi meelitada.
\par 22 Sest ma ei või meelitada: kui kergesti võiks mu Looja mind siis ära võtta.

\chapter{33}

\par 1 Aga kuule nüüd ometi, Iiob, mu kõnet, ja pane tähele kõiki mu sõnu!
\par 2 Vaata, ma avan nüüd oma suu, keel mu suulaes kõneleb.
\par 3 Mu sõnad tulevad õiglasest südamest ja mu huuled kuulutavad selgesti seda, mis ma tean.
\par 4 Jumala Vaim on mind loonud ja Kõigevägevama hingeõhk on andnud mulle elu.
\par 5 Kui sa suudad, siis vasta mulle, sea ennast valmis, astu ette!
\par 6 Vaata, Jumala ees olen mina samasugune kui sina: savist olen minagi voolitud.
\par 7 Vaata, hirm minu ees ei saa kohutada sind ja minu surve ei ole sulle raske.
\par 8 Tõesti, sa oled kõnelnud mulle kõrvu ja ma olen kuulnud su sõnade kõla:
\par 9 „Mina olen puhas, mina pole üle astunud, ma olen laitmatu ja mul ei ole süüd.
\par 10 Vaata, ta leiab mu vastu põhjusi, peab mind oma vaenlaseks.
\par 11 Ta pistab mu jalad pakku, valvab kõiki mu radu.”
\par 12 Vaata, selles ei ole sul õigus, vastan ma sulle, sest Jumal on suurem kui inimene!
\par 13 Mispärast sa riidled temaga, kui ta ei vasta inimesele iga sõna peale?
\par 14 Sest Jumal kõneleb ühel ja teisel viisil, aga seda ei märgata.
\par 15 Unenäos, öises nägemuses, kui sügav uni inimest valdab, voodis suikudes -
\par 16 seal ta avab inimeste kõrvad ja paneb pitseri nende manitsustele,
\par 17 et inimest pahateost eemal hoida ja mehe kõrkust kinni katta,
\par 18 et säästa tema hinge hauast ja tema elu oda otsa jooksmast.
\par 19 Ka manitsetakse teda valu läbi voodis ja kestva vaevusega kontides.
\par 20 Siis muutub vastikuks ta elule leib ja hingele maiusroog:
\par 21 ta ihu kõhetub nähtamatuks, ja ta luud, mida näha ei olnud, paljastuvad.
\par 22 Nõnda ligineb ta hing hauale ja ta elu sellele, mis toob surma.
\par 23 Aga kui tema jaoks on ingel, eestkostja, üks tuhandest, kes kuulutab inimesele, mis tema kohus on,
\par 24 siis Jumal halastab tema peale ning ütleb: „Vabasta teda haudaminekust, ma olen lunaraha saanud!”
\par 25 Siis muutub ta ihu noorusvärskeks, ta pöördub tagasi oma nooruspäevadesse.
\par 26 Ta palvetab Jumala poole ja sellel on temast hea meel; ta saab hõisates näha tema palet ja tema annab inimesele tagasi ta õiguse.
\par 27 Ta laulab siis inimeste ees ja ütleb: „Ma tegin pattu ja väänasin õigust, aga selle eest ei tasutud mulle kätte.
\par 28 Tema lunastas mu hinge haudaminekust ja mu elu saab valgust näha.”
\par 29 Vaata, seda kõike teeb Jumal inimesele kaks, kolm korda,
\par 30 et tema hinge hauast tagasi tuua ja valgustada teda elu valgusega.
\par 31 Pane tähele, Iiob, kuule mind, vaiki, et mina saaksin rääkida!
\par 32 Kui sul on sõnu, siis vasta mulle, räägi, sest meeleldi annaksin sulle õiguse!
\par 33 Aga kui mitte, siis kuule sina mind, vaiki, siis ma õpetan sulle tarkust!”

\chapter{34}

\par 1 Ja Eliihu jätkas ning ütles:
\par 2 „Kuulge mu sõnu, te targad, ja pöörake oma kõrv minu poole, te teadjad!
\par 3 Sest kõrv katsub sõnad läbi ja suulagi maitseb rooga.
\par 4 Valigem endile, mis on õige, tunnetagem isekeskis, mis on hea!
\par 5 Sest Iiob on öelnud: „Ma olen õige, aga Jumal on võtnud minult õiguse.
\par 6 Kuigi mul on õigus, peetakse mind valetajaks. Mul on parandamatu haav, kuigi ma pole üle astunud.”
\par 7 Kas on Iiobi sarnast meest, kes pilget joob nagu vett,
\par 8 kes läheb ülekohtutegijate kilda ja käib koos õelate inimestega?
\par 9 Sest tema ütleb: „Ei ole inimesel sellest kasu, et tal on sõprus Jumalaga!”
\par 10 Seepärast kuulge, mõistlikud mehed, mind: kaugel on Jumalast õelus ja Kõigevägevamast ülekohtutegu.
\par 11 Sest inimese tegu mööda tasub ta temale, ja nagu on kellegi tee, nõnda temale antakse.
\par 12 Jah, tõesti, Jumal ei tee ülekohut ja Kõigevägevam ei vääna õigust.
\par 13 Kes on temale usaldanud maa? Ja kes on loonud kogu maailma?
\par 14 Kui ta iseenesesse tõmbuks, oma Vaimu ja hingeõhu tagasi võtaks,
\par 15 siis heidaks kõik liha üheskoos hinge ja inimene saaks jälle põrmuks.
\par 16 Kui sul nüüd on arukust, siis kuule seda, võta kõrvu mu sõnade kõla!
\par 17 Kas siis tõesti peaks valitsema õiguse vihkaja? Või tahad sa süüdi mõista õiget ja võimast,
\par 18 kes ütleb kuningale: „Kõlvatu!„, vürstidele: ”Õel!”,
\par 19 kes ei pea lugu vürstidest ega eelista suursugust viletsale, sest et need kõik on tema kätetöö?
\par 20 Nad surevad äkitselt keset ööd: inimesed vaaruvad ja lähevad, ja ilma käetagi võetakse ära vägev.
\par 21 Sest tema silmad on igaühe teede peal ja ta näeb kõiki tema samme.
\par 22 Ei ole pimedust ega varjulist paika, kuhu ülekohtutegijad saaksid peitu pugeda.
\par 23 Sest tema ei anna inimesele aega Jumala juurde kohtusse minekuks:
\par 24 ta peksab vägevad üle kuulamata puruks ja paneb teised nende asemele.
\par 25 Sellepärast, et ta nende tegusid tunneb, hävitab ta nad öösel ja nad purustatakse.
\par 26 Ta peksab neid nagu kurjategijaid paigas, kus on nägijaid,
\par 27 sellepärast et nad taganesid tema järelt ega hoolinud ühestki tema teest.
\par 28 Nende pärast tõusis viletsate hädakisa tema ette, ja tema kuulis vaeste appihüüdeid.
\par 29 Kui ta on vait, kes võiks teda süüdistada? Ja kui ta oma palge peidab, kes saaks teda näha? Niihästi rahva kui inimese üle
\par 30 paneb ta kuningaks jumalavallatu inimese rahva võrgutajate seast.
\par 31 Kui keegi ütleb Jumalale: „Ma olen eksinud, ma ei tee enam kurja;
\par 32 mida ma ei mõista, seda õpeta sina mulle, ja kui olen ülekohut teinud, siis ma seda enam ei tee!”,
\par 33 kas ta siis sinu arust peaks kätte maksma, sellepärast et sina ei ole sellega rahul? Kuid sina pead otsustama, mitte mina, seepärast räägi, mida sa tead!
\par 34 Mõistlikud inimesed ütlevad mulle, samuti tark mees, kes mind kuuleb:
\par 35 „Iiob räägib mõistmatult ja tema sõnad pole targad.”
\par 36 Ah, peaks ometi Iiob pandama lõpuni proovile, kuna ta on vastanud nurjatul viisil!
\par 37 Sest ta lisab oma patule üleastumise, peksab meie keskel keelt ja räägib palju Jumala vastu.”

\chapter{35}

\par 1 Ja Eliihu jätkas ning ütles:
\par 2 „Seda sa pead siis õigeks, nimetad oma õiguseks Jumala ees,
\par 3 et sa küsid: „Mis kasu mul on, mis abi on mul sellest, et ma pattu ei tee?”
\par 4 Ma annan sõnadega vastuse sulle ja koos sinuga su sõpradele.
\par 5 Tõsta oma pilk taeva poole ja vaata, pane tähele pilvi, mis on sinust kõrgemal!
\par 6 Kui sina pattu teed, mida sa sellega temale võiksid teha? Ja kui sinu üleastumisi on palju, mida sa nendega tema vastu võiksid korda saata?
\par 7 Kui sa oled õige, mida sa temale võiksid anda? Või mida olekski tal võtta sinu käest?
\par 8 Su ülekohus mõjutab vaid sinusugust meest ja su õiglus inimlast.
\par 9 Nad kisendavad küll paljude rõhujate pärast, hüüavad appi vägevate käsivarre vastu,
\par 10 aga ükski ei küsi: „Kus on Jumal, mu Looja, kes öösiti põhjustab kiituslaule,
\par 11 kes õpetab meid rohkem kui loomi maa peal ja teeb meid taeva lindudest targemaks?”
\par 12 Seal nad siis kisendavad, aga tema ei vasta kurjade kõrkuse pärast.
\par 13 Tõesti asjata, Jumal ei kuule seda ja Kõigevägevam ei vaata sinna.
\par 14 Kuigi sa ütled, et sa teda ei näe, on asi tema ees, seepärast oota teda!
\par 15 Aga nüüd, kui ta viha ei karista ja ta ei tahagi nii väga teada ülemeelikusest,
\par 16 ajab Iiob asjata oma suu pärani ja teeb mõistmatusest palju sõnu.”

\chapter{36}

\par 1 Ja Eliihu jätkas ning ütles:
\par 2 „Olgu sul minuga pisut kannatust, siis ma õpetan sind, sest mul on veel Jumala heaks sõnu!
\par 3 Ma tahan tuua oma tarkuse kaugemalt ja anda õiguse oma Loojale.
\par 4 Sest mu sõnad pole tõesti mitte valed, täiuslik teadja on su ees.
\par 5 Vaata, Jumal on vägev, kuid ta ei põlga kedagi, vägev on tema südame jõud!
\par 6 Õelat ei jäta ta elama, aga viletsaile annab ta õiguse.
\par 7 Ta ei pööra oma silmi ära õigete pealt, vaid paneb need koos kuningatega igaveseks istuma aujärjele, ja nad ülendatakse.
\par 8 Ja kui nad on pandud ahelaisse, vangistatud viletsuse köidikuisse,
\par 9 siis ta heidab neile ette nende tegu ja üleastumisi, et nad on olnud ülemeelikud,
\par 10 ja avab nende kõrvad hoiatuseks ning käsib neid taganeda ülekohtust.
\par 11 Kui nad siis kuulavad ja teenivad teda, siis nad lõpetavad oma päevad õnnes ja aastad õndsuses.
\par 12 Aga kui nad ei kuula, siis nad tormavad oda otsa ja heidavad arutuina hinge.
\par 13 Südamest jumalatud peavad viha, nad ei hüüa appi, kui ta on nad vangistanud.
\par 14 Nende hing sureb noorelt ja nende elu pordumeeste seas.
\par 15 Ta päästab viletsa tema viletsuse kaudu ja avab tema kõrva, kui ta on hädas.
\par 16 Ta meelitaks sindki kitsikuse kurgust piiramata avarusse ja su laud oleks täis rasvast rooga.
\par 17 Aga sa oled täis õelate kohut, kohus ja õigus tabavad sind.
\par 18 Hoia, et viha sind liialt ei mõjustaks ja rohke lunaraha sind ei eksitaks!
\par 19 Kas su õilsus on küllaldane? Ei aita kuld ega kõik su jõupingutused!
\par 20 Ära igatse ööd, kui rahvad tõusevad oma asemeilt!
\par 21 Hoidu pöördumast ülekohtu poole, sest seda sa eelistad viletsusele!
\par 22 Vaata, Jumal toimib võimsalt oma jõus! Kes oleks seesugune õpetaja nagu tema?
\par 23 Kes on temale määranud ta tee, ja kes julgeks öelda: „Sa oled ülekohut teinud”?
\par 24 Pea meeles, et sinagi pead ülistama tema tööd, millest inimesed on laulnud!
\par 25 Kõik inimesed näevad seda - inimene vaatab seda kaugelt.
\par 26 Vaata, Jumal on suur ja meie ei mõista teda, tema aastate arv on uurimatu.
\par 27 Sest ta tõmbab veepiisad üles: need nõrguvad tema udust vihmaks,
\par 28 mida pilved kallavad ja tilgutavad paljude inimeste peale.
\par 29 Tema, kes korraldab ka pilvekihte, oma kojast müristamist,
\par 30 vaata, ta laotab talle oma valgust ja katab sellega mere põhjad.
\par 31 Sest sellega ta toidab rahvaid ja annab külluses rooga.
\par 32 Ta võtab välgu oma kätte ja käsib sellel märki tabada.
\par 33 Tema mürin kuulutab sellest, loomakarigi kuulutab tõusvat tormi.

\chapter{37}

\par 1 Sellepärast väriseb minugi süda ja hüppab paigast.
\par 2 Kuulge, kuulge tema hääle mürinat ja kõminat, mis tuleb ta suust!
\par 3 Selle ta päästab lahti kogu taeva alla ja oma välgu maailma äärteni.
\par 4 Selle taga möirgab tema hääl: ta müristab oma võimsa häälega ega peata välke, kui tema häält kuuldub.
\par 5 Jumal müristab oma häälega imepärasel viisil, ta teeb suuri tegusid, mida meie ei mõista.
\par 6 Sest ta ütleb lumele: „Lange maa peale!” ja vihmasabin ning vihmavaling on tema tugevus.
\par 7 Ta paneb siis igamehe käe kinni otsekui pitseriga, et kõik inimesed tunneksid ta tegu.
\par 8 Siis läheb metsloom oma peidupaika ja jääb oma pessa.
\par 9 Lõunakambrist tuleb tuulispea ja põhjatuulega külm.
\par 10 Jumala hingusest tekib jää ja tardub veepind.
\par 11 Ta koormab ka pilvi niiskusega, pilved pilluvad tema välku.
\par 12 Need rändavad ringi tema juhtimisel, et teha kõike, mida ta neil käsib maailmas ja maa peal,
\par 13 olgu vitsana, kui ta maale tarvis, olgu armuna, kui ta seda osutab.
\par 14 Kuule seda, Iiob, peatu ja pane tähele Jumala imetegusid!
\par 15 Kas sa mõistad, kuidas Jumal annab neile käsu ja laseb oma pilvest välgu sähvatada?
\par 16 Kas sa mõistad pilvede sõudu, ülima tarkuse imetegu?
\par 17 Sina, kellel riided kuumavad, kui ta lõunatuulega suigutab maad,
\par 18 kas sa koos temaga laotad pilvekatet, vastupidavat nagu valatud peeglit?
\par 19 Õpeta meid, mida me peaksime temale ütlema; pimeduse tõttu ei saa me millestki aru.
\par 20 Kas peaks temale jutustatama, et mina tahan rääkida? Kui keegi kõneleb, kas seda siis temale teada antakse?
\par 21 Kui nüüd ei nähta valgust, mis hiilgab pilvedes, siis puhub tuul ja toob selguse.
\par 22 Põhja poolt tuleb kuldne hiilgus - Jumala ümber on hirmuäratav aupaiste.
\par 23 Kõigevägevam - temani me ei jõua, tema on suur jõult ja rikas õiglusest, tema ei riku õigust.
\par 24 Seepärast peavad inimesed teda kartma, tema ei vaata kedagi, kes on enese meelest tark.”

\chapter{38}

\par 1 Siis vastas Issand Iiobile tormituulest ja ütles:
\par 2 „Kes on see, kes mõistmatute sõnadega tahab varjutada minu nõu?
\par 3 Pane nüüd vöö vööle kui mees, mina küsin sinult ja sina seleta mulle!
\par 4 Kus olid sina siis, kui mina rajasin maa? Vasta, kui sul niipalju tarkust on!
\par 5 Kes määras selle mõõdud? Küllap sa tead. Või kes vedas selle üle mõõdunööri?
\par 6 Mille peale on selle alussambad paigale pandud? Või kes asetas selle nurgakivi,
\par 7 kui hommikutähed üheskoos hõiskasid ja kõik Jumala pojad tõstsid rõõmukisa?
\par 8 Ja kes sulges ustega mere, kui see esile murdes emaüsast välja tuli,
\par 9 kui ma panin pilved selle katteks ja pilkase pimeduse mähkmeks,
\par 10 kui ma sellele lõin korra, panin riivid ja uksed
\par 11 ning ütlesin: „Siiani sa võid tulla, mitte edasi, siin vaibugu su uhked lained!”?
\par 12 Kas sa oled kunagi oma elupäevil andnud hommikule käsu, määranud koidule tema paiga,
\par 13 et ta haaraks kinni maa äärtest ja puistaks õelad sealt välja,
\par 14 et see muutuks nagu savi pitsati all ja jääks seisma otsekui riie,
\par 15 et õelailt võetaks nende valgus ja tõstetud käsivars murtaks?
\par 16 Kas sa oled jõudnud mere allikateni ja kõndinud sügavuse põhjas?
\par 17 Ons sulle ilmutatud surma väravaid, või oled sa näinud pilkase pimeduse väravaid?
\par 18 Kas sa taipad, kui avar on maa? Jutusta, kui sa kõike seda tead!
\par 19 Kus on tee valguse asukohta ja kus on pimeduse paik,
\par 20 et sa võiksid teda viia ta piiridesse ja märkaksid teeradu tema koju?
\par 21 Sa tead seda, sest olid ju siis juba sündinud ja su päevade arv on suur.
\par 22 Kas sa oled käinud lumeaitade juures ja näinud raheaitu,
\par 23 mida ma olen hoidnud hädaaja tarvis, sõja ja võitluse päevaks?
\par 24 Kus on tee sinna, kus valgus jaguneb, kust idatuul levib üle maa?
\par 25 Kes on lõhestanud vihmavalingule vao ja kõuepilvele tee,
\par 26 et see võiks sadada asustamata maale, kõrbe, kus ei ole inimest,
\par 27 laastatu ja lageda küllastuseks, et lasta tärgata noort rohtu?
\par 28 Ons vihmal isa, või kes on sünnitanud kastetilgad?
\par 29 Kelle üsast on välja tulnud jää ja kes on sünnitanud taeva halla,
\par 30 kui veed tarduvad otsekui kiviks ja sügavuse pinnad tõmbuvad kokku?
\par 31 Kas sa suudad siduda Sõelatähtede tõrksust või valla päästa Vardatähtede köidikuid?
\par 32 Kas sa võid Vihmatähti välja saata õigel ajal ja juhatada Vankritähtede kogumit?
\par 33 Kas sa tunned taeva seadusi? Või kehtestad sina maa peal tema kirja?
\par 34 Kas sa suudad tõsta oma häält pilvedeni, et veevoolus sind kataks?
\par 35 Ons sul võimalik saata välke, et need läheksid ja ütleksid sulle: „Vaata, siin me oleme”?
\par 36 Kes on teinud iibise targaks või kes on andnud kukele mõistuse?
\par 37 Kellel oleks osavust lugeda pilvi? Või kes võiks kummutada taevalähkreid,
\par 38 kui muld on kokku vajunud känkraks ja kamakad on takerdunud üksteise külge?
\par 39 Kas sina ajad lõvidele saaki ja täidad noorte lõvide isu,
\par 40 kui need kükitavad pesades, lebavad varitsedes tihnikus?
\par 41 Kes valmistab kaarnale roa, kui ta pojad karjuvad Jumala poole, toiduta ümber hulkudes?

\chapter{39}

\par 1 Kas sa tead kaljukitsede poegimisaega, kas valvad hirvede sünnitust?
\par 2 Kas sa loed nende tiinuskuid ja tead aega, millal nad poegivad,
\par 3 kui nad kõveraks tõmbudes vasikaid heites oma ihuviljast vabanevad?
\par 4 Nende vasikad kosuvad, kasvavad avaral aasal, lähevad ära ega tule tagasi nende juurde.
\par 5 Kes on sebra lahti lasknud ja kes on valla päästnud metseesli köidikud,
\par 6 kellele ma olen seadnud koduks lagendiku ja elukohaks soolakõrbe?
\par 7 Ta naerab linna kära, sundija kisa ta ei kuule.
\par 8 Ta uitab mägedel, oma karjamaal, ja otsib kõike, mis aga haljas on.
\par 9 Kas metshärg tahab sind teenida? Kas ta jääb ööseks su sõime juurde?
\par 10 Kas sa saad metshärja siduda köiega vaole? Kas ta äestab su järel orumaad?
\par 11 Kas sa võid loota tema peale, kuigi ta rammult on suur, ja jätta oma töö tema hooleks?
\par 12 Kas sa usud, et ta toob tagasi ja kogub su seemne rehealusesse?
\par 13 Jaanalinnu tiivad lehvivad rõõmsasti, aga ons need tiivad ja suled tegusad?
\par 14 Sest ta jätab ju oma munad maa peale ja laseb neid liivas soojeneda,
\par 15 unustades, et jalg võib need purustada ja metsloom tallata.
\par 16 Ta on oma poegade vastu vali, nagu ei olekski need ta omad; see pole tema mure, et ta vaev võiks olla asjatu.
\par 17 Sest Jumal on teda lasknud unustada tarkuse ega ole jaganud temale mõistust.
\par 18 Aga kui ta siis üles kargab, ta naerab hobust ja ratsanikku.
\par 19 Kas sina annad hobusele jõu, ehid ta kaela lakaga?
\par 20 Kas sina paned ta hüppama nagu rohutirtsu? Tema võimas nooskamine on kohutav.
\par 21 Ta kaabib orus ja tunneb rõõmu, ta läheb jõuliselt relvadele vastu.
\par 22 Ta naerab hirmu, ta ei kohku ega tagane mõõga eest.
\par 23 Ta kohal kõliseb nooletupp, piigitera ja viskoda.
\par 24 Ta kihutab tuhinal ja hoogsasti ega püsi paigal, kui sarv hüüab.
\par 25 Siis kui puhutakse sarve, hirnub tema: iihahaa! Juba kaugelt haistab ta võitlust, pealikute kisa ja hõikeid.
\par 26 Kas sinu mõistuse abil lendab kull kõrgele, laotab oma tiibu lõuna poole?
\par 27 Kas sinu käsul kerkib kotkas kõrgustesse ja teeb oma pesa kõrgele?
\par 28 Ta elab ja ööbib kalju peal, kaljuserval ja ligipääsmatus paigas.
\par 29 Sealt ta luurab saaki ja ta silmad näevad kaugele.
\par 30 Ta pojad rüübivad verd, ja kus on mahalööduid, seal on temagi.”

\chapter{40}

\par 1 Ja Issand vastas Iiobile ning ütles:
\par 2 „Kas tahab nuriseja vaielda Kõigevägevamaga? Jumala süüdistaja andku vastus!”
\par 3 Siis Iiob vastas Issandale ja ütles:
\par 4 „Vaata, ma olen selleks liiga tühine. Mida peaksin sulle vastama? Ma panen käe suu peale.
\par 5 Ma olen kord rääkinud ja ma ei vasta enam, koguni kaks korda, ja enam ma seda ei tee.”
\par 6 Aga Issand vastas Iiobile tormituulest ja ütles:
\par 7 „Pane nüüd vöö vööle nagu mees, mina küsin sinult ja sina seleta mulle!
\par 8 Kas sina tahad minu õigust tühjaks teha, süüdistades mind, et õigustada iseennast?
\par 9 Ons sul siis käsivars nagu Jumalal, või kas sina oma häälega suudad müristada nagu tema?
\par 10 Ehi ennast siis uhkuse ja väärikusega, pane enesele ülle hiilgust ja ilu!
\par 11 Päästa valla oma vihavood, vaata kõigi kõrkide peale ja alanda neid!
\par 12 Vaata kõigi kõrkide peale, rudju neid, ja rõhu õelad maha otse kohapeal!
\par 13 Mata nad kõik üheskoos põrmu, sule nende silmnäod salapaika,
\par 14 siis minagi kiidan sind, et su parem käsi on sind aidanud!
\par 15 Vaata ometi jõehobu, kelle ma olen loonud nagu sinugi: ta sööb rohtu nagu veis.
\par 16 Ent vaata tema ristluude rammu ja kõhulihaste jõudu.
\par 17 Ta saba on nagu seedritüvi, ta reite kõõlused otsekui põimitud.
\par 18 Ta luud on nagu vaskputked, ta kondid otsekui raudtalad.
\par 19 Ta on Jumala töödest parim: kes ta on teinud, toob ta lähedale mõõga.
\par 20 Jah, toidust toodavad temale mäed, seal, kus kõik metsloomad mängivad.
\par 21 Ta magab lootospõõsaste all, redutab pilliroos ja mudas.
\par 22 Lootospõõsad, olles temale varjuks, katavad teda, jõeremmelgad on ta ümber.
\par 23 Vaata, kui ka jõgi peale surub, ta ei tõtta minema, ta jääb kindlaks, isegi kui Jordan tormaks temale suhu.
\par 24 Kas saaks tema nähes teda kinni võtta, temale püüniseid ninna torgata?

\chapter{41}

\par 1 Vaata, niisugune lootuski on petlik: kas teda nähes juba ei olda pikali?
\par 2 Ükski ei ole nõnda julge, et teda ärritada. Ja kes suudaks siis püsida minu ees?
\par 3 Kes on esmalt andnud mulle, et peaksin tasuma? Kõik, mis on taeva all, on minu oma.
\par 4 Ma ei taha vaikida tema liikmeist, ta vägitegude asjast ning ta oleku ilust.
\par 5 Kes paljastaks tema ta pealmisest kattest? Kes tungiks tema kahekordse soomuse vahele?
\par 6 Kes avaks tema silmnäo väravad? Ta hambad ajavad hirmu peale.
\par 7 Ta selg on kilpide rida, tihedalt pitseriga suletud:
\par 8 üks on teisele nõnda ligi, et tuul ei pääse läbi nende vahelt.
\par 9 Need on liibunud üksteise külge, seisavad koos ja neid ei saa lahutada.
\par 10 Tema aevastustest välgub valgust ja ta silmad on nagu koidupuna kiired.
\par 11 Tema suust käivad otsekui põlevad tõrvikud, kargavad välja tulesädemed.
\par 12 Tema sõõrmeist tuleb välja suits nagu potist, mille all põlevad kõrkjad.
\par 13 Tema hingeõhk süütab söed ja tema suust tuleb välja leek.
\par 14 Tema turjal asub tugevus ja tema ees kargab kartus.
\par 15 Tema lihav vats on otsekui valatud ta külge, liikumatult paigal.
\par 16 Tema süda on kõva nagu kivi, otse alumise veskikivi kõvadune.
\par 17 Kui ta tõuseb, siis kohkuvad vägevadki, tema murdmise tõttu on nad nagu arust ära.
\par 18 Kui teda tabab mõõk, siis see ei pea vastu, samuti mitte piik, viskoda ega nool.
\par 19 Ta peab rauda õleks ja vaske pehkinud puuks.
\par 20 Nool ei aja teda põgenema, lingukivid muutuvad tema vastu kõrteks.
\par 21 Vemblad on temale nagu kõrred ja ta naerab oda vihinat.
\par 22 Tal on kõhu all otsekui teravad kivikillud, ta hüpleb nagu pahmavanker mööda muda.
\par 23 Ta paneb sügavuse keema nagu paja, teeb mere salvipoti sarnaseks.
\par 24 Enese järel ta jätab läikiva raja, sügavus näib olevat otsekui raugajuuksed.
\par 25 Ei ole maa peal tema sarnast, ta on kartmatuks loodud.
\par 26 Kõik, mis kõrge on, kardab teda, tema on kõigi uhkete loomade kuningas.”

\chapter{42}

\par 1 Siis Iiob vastas Issandale ja ütles:
\par 2 „Mina tean, et sina suudad kõike ja et sinul ei ole ükski asi võimatu.
\par 3 Kes on see, kes mõistmatult püüab varjata sinu nõu? Seepärast ma olen jutustanud, millest ma pole aru saanud, asjadest, mis mulle on imelikud ja mida ma ei mõista.
\par 4 Kuule siis ometi ja ma räägin, mina küsin sinult ja sina seleta mulle!
\par 5 Ma olin ainult kõrvaga kuulnud kuuldusi sinust, aga nüüd on mu silm sind näinud.
\par 6 Seepärast ma võtan kõik tagasi ning kahetsen põrmus ja tuhas.”
\par 7 Ja pärast seda kui Issand oli rääkinud need sõnad Iiobile, ütles Issand teemanlasele Eliifasele: „Mu viha on süttinud põlema sinu vastu ja su mõlema sõbra vastu, sest teie ei ole rääkinud minu kohta tõtt nagu mu sulane Iiob!
\par 8 Võtke siis nüüd seitse härjavärssi ja seitse jäära ning minge mu sulase Iiobi juurde ja ohverdage eneste eest põletusohver, ja mu sulane Iiob palvetagu teie eest; sest ma kuulen ainult tema palvet, et mitte talitada teiega häbistavalt, kuigi te ei ole rääkinud minu kohta tõtt nagu mu sulane Iiob!”
\par 9 Siis läksid teemanlane Eliifas, suhiit Bildad ja naamalane Soofar ja tegid, nagu Issand neile oli öelnud; ja Issand kuulis Iiobi palvet.
\par 10 Ja Issand pööras Iiobi saatuse, kui too palvetas oma sõprade eest; ja Issand andis Iiobile kahekordselt kõike, mis tal oli olnud.
\par 11 Ja tema juurde tulid kõik ta vennad ja kõik ta õed ja kõik, kes teda enne olid tundnud, ja võtsid leiba koos temaga ta kojas; ja nad avaldasid temale kaastunnet ja trööstisid teda selle õnnetuse pärast, mis Issand temale oli lasknud tulla; ja igaüks andis temale ühe kaaluühiku raha ja ühe kuldrõnga.
\par 12 Ja Issand õnnistas Iiobi viimast põlve enam kui ta esimest; ja tal oli neliteist tuhat lammast ja kitse, kuus tuhat kaamelit, tuhat paari härgi ja tuhat emaeeslit.
\par 13 Ja tal oli seitse poega ja kolm tütart.
\par 14 Ta pani ühele tütrele nimeks Jemiima, teisele Ketsia ja kolmandale Keren-Hapuuk.
\par 15 Ja kogu maal ei leidunud nii ilusaid naisi kui Iiobi tütred; ja nende isa andis neile pärisosa nende vendade keskel.
\par 16 Pärast seda elas Iiob veel sada nelikümmend aastat ja nägi oma lapsi ja oma laste lapsi neli põlve.
\par 17 Ja Iiob suri, olles vana ja elatanud.



\end{document}