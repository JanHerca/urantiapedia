\begin{document}

\title{Baaruk}

\chapter{1}

\section*{Kiri Paabeli juutidele}

\par 1 Need on selle kirja sõnad, mille Baaruk, Neerija poeg, kes oli Mahseja poeg, kes oli Sedekija poeg, kes oli Hasadja poeg, kes oli Hilkija poeg, kirjutas Paabelis
\par 2 viiendal aastal, seitsmendal kuupäeval, sel ajal kui kaldealased olid vallutanud Jeruusalemma ja selle tulega põletanud.
\par 3 Baaruk luges selle kirja sõnu Juuda kuninga Jekonja, Joojakimi poja kuuldes ja kogu rahva kuuldes, kes oli tulnud kirja kuulama,
\par 4 võimukandjate ja kuningapoegade kuuldes ja vanemate kuuldes ja kõigi inimeste, niihästi pisikeste kui suurte kuuldes, kõigi kuuldes, kes elasid Paabelis, Suudi jõe ääres.
\par 5 Siis nad nutsid, paastusid ja palvetasid Issanda ees
\par 6 ning kogusid raha - igaüks andis jõu kohaselt -
\par 7 ja saatsid selle Jeruusalemma ülempreester Joojakimile, Hilkija pojale, Sallumi pojapojale, ja preestritele ning kogu rahvale, kes koos temaga oli Jeruusalemmas.
\par 8 Siivanikuu kümnendal päeval võttis Baaruk templist äratoodud jumalakoja riistad ja saatis need tagasi Juudamaale, hõberiistad, mis Juuda kuningas Sidkija, Joosija poeg, oli lasknud teha
\par 9 pärast seda, kui Paabeli kuningas Nebukadnetsar oli Jeruusalemmast ära viinud Jekonja ja vürstid, vangid ja võimukandjad ning maa rahva ja oli need toonud Paabelisse.
\par 10 Nad käskisid ütelda: „Vaata, me saadame teile raha! Ostke nüüd raha eest põletus- ja patuohvreid ning suitsutusrohtu, valmistage roaohver ja ohverdage Issanda, meie Jumala altaril!
\par 11 Ja palvetage Paabeli kuninga Nebukadnetsari ja tema poja Belsassari elu pärast, et nende päevad maa peal oleksid otsekui taeva päevad!
\par 12 Siis Issand annab meile jõudu ja valgustab meie silmi, siis me võime elada Paabeli kuninga Nebukadnetsari varju all ja tema poja Belsassari varju all, võime teenida neid kaua aega ja leida armu nende ees.
\par 13 Ja tehke palvet meie eest Issanda, meie Jumala ees, sest me oleme pattu teinud Issanda, meie Jumala vastu, ja Issanda viha ning raev ei ole tänapäevani ära pöördunud meie pealt!
\par 14 Ja lugege ette see kiri, mille meie teile saadame, kuulutades seda Issanda kojas pühade ajal ja pühapäevil.

\section*{Patutunnistus}

\par 15 Ja ütelge nõnda: „Issandal, meie Jumalal, on õigus, meil on aga häbi silmis, nõnda nagu see tänapäeval on Juuda meestel ja Jeruusalemma elanikel,
\par 16 meie kuningail ja meie vürstidel, meie preestreil ja meie prohveteil ja meie vanemail,
\par 17 sellepärast et me oleme pattu teinud Issanda vastu.
\par 18 Meie ei ole teda uskunud ega ole võtnud kuulda Issanda, meie Jumala häält, et käiksime Issanda käskude järgi, mis ta meile on andnud.
\par 19 Alates päevast, mil Issand tõi ära meie vanemad Egiptusemaalt, kuni tänapäevani oleme olnud sõnakuulmatud Issanda, meie Jumala vastu ja oleme olnud hooletud kuulama tema häält.
\par 20 Ja nõnda on tabanud meid see õnnetus ning needus, mida Issand oma sulase Moosese läbi kuulutas sel päeval, mil ta tõi ära meie vanemad Egiptusemaalt, et anda meile maa, mis piima ja mett voolab, nõnda nagu see tänapäeval ongi.
\par 21 Meie ei ole võtnud kuulda Issanda, oma Jumala häält kõigis sõnus neilt prohveteilt, keda tema meie juurde läkitas,
\par 22 vaid oleme ära läinud, igaüks oma kurja südame ajel, et teenida võõraid jumalaid ja teha seda, mis on paha Issanda, meie Jumala silmis.

\chapter{2}

\section*{Patutunnistus}

\par 1 Siis Issand pidas oma sõna, mis oli öeldud meie ja meie  kohtumõistjate kohta, kes Iisraelile kohut mõistsid, ja meie kuningate ja meie vürstide kohta ja Iisraeli ja Juuda meeste kohta.
\par 2 Kogu taeva all ei ole sündinud seesugust, mida ta laskis sündida Jeruusalemmas, just nõnda nagu Moosese Seaduses on kirjutatud,
\par 3 et pidime sööma üks oma poja ja teine oma tütre liha.
\par 4 Ta andis nad kõigi kuningriikide alla, mis meil ümberkaudu olid, teotamiseks ja põlualuseiks kõigile ümberkaudseile rahvaile, kuhu Issand nad oli pillutanud.
\par 5 Ja nad olid ainult alamad, aga mitte ülemad, sellepärast et me tegime pattu Issanda, oma Jumala vastu, sest me ei võtnud kuulda tema häält.
\par 6 Issandal, meie Jumalal, on õigus, meil ja meie vanemail on häbi silmis, nõnda nagu see tänapäeval on.
\par 7 Mida Issand meie kohta on rääkinud, kogu see õnnetus on meie peale tulnud.
\par 8 Me ei ole ka Issanda ees palvetanud, et igaüks pöörduks oma kurja südame soovidest.
\par 9 Sellepärast oli Issand valvas ja saatis selle õnnetuse meie peale, sest Issand on õiglane kõigis oma tegudes, mis ta on otsustanud meiega teha.
\par 10 Meie ei võtnud aga kuulda tema häält, et pidime käima Issanda seaduste järgi, mis tema meile on andnud.

\section*{Palve}

\par 11 Issand, Iisraeli Jumal, sina oled see, kes tõi oma rahva Egiptusemaalt välja vägeva käega, tunnustähtede ja imetegudega, suure rammu ja ülestõstetud käsivarrega, ja andsid enesele nime, nagu see tänapäevalgi on.
\par 12 Oh Issand, meie Jumal, me oleme pattu teinud, oleme olnud jumalakartmatud ja ülekohtused kõigi sinu õigete käskude vastu.
\par 13 Pöördugu sinu viha meie pealt, sest meid on pisut järele jäänud nende rahvaste keskel, kuhu sa meid oled pillutanud!
\par 14 Kuule, Issand, meie palvet ja anumist ja päästa meid iseenese pärast ning lase meid armu leida nende silmis, kes meid on pagendanud,
\par 15 et kogu maailm saaks teada, et sina oled Issand, meie Jumal, et Iisraelile ja tema soole on pandud sinu nimi!
\par 16 Issand, vaata alla oma pühast eluasemest ja pane meid tähele! Pööra, Issand, oma kõrv ja kuule!
\par 17 Ava, Issand, oma silmad ja vaata! Sest mitte surnud surmavallas, kellel hing on rinnast võetud, ei anna Issandale au ja tunnustust,
\par 18 vaid see hing, kes on väga kurb, käib küürus ja on jõuetu, ja äranutetud silmad ning nälgiv hing - need annavad sulle, Issand, au ja tunnustust.
\par 19 Sest mitte meie vanemate ja meie kuningate teenete pärast ei heida me oma anumist sinu ette, Issand, meie Jumal.
\par 20 Sina oled oma viha ja raevu valanud välja ju meie vastu, nõnda nagu sa oled rääkinud oma sulaste, prohvetite läbi, öeldes:
\par 21 „Nõnda ütleb Issand: Painutage oma selga ja teenige Paabeli kuningat, siis te jääte elama maale, mille mina teie vanemaile olen andnud!
\par 22 Aga kui te ei kuula Issanda häält, et teenida Paabeli kuningat,
\par 23 siis ma lõpetan Juuda linnadest ja Jeruusalemmast lusti- ja rõõmuhääle, peigmehe hääle ja pruudi hääle, ja kogu maa jääb tühjaks, ükski ei ela seal.”
\par 24 Et me ei ole võtnud kuulda sinu häält, et teenida Paabeli kuningat, sellepärast oled sina pidanud oma sõna, mille sa oled ütelnud oma sulaste, prohvetite läbi, et meie kuningate ja meie vanemate luud võetakse nende asemeist.
\par 25 Ja vaata, need ongi visatud päeva palavuse ja öö külma kätte. Nad surid suurtes piinades, nälja, mõõga ja katku läbi.
\par 26 Ja sina oled teinud selle koja, mida nimetatakse sinu nimega, selliseks, nagu see nüüd on, Iisraeli soo ja Juuda soo kuritöö pärast.
\par 27 Ometi oled sina, Issand, meie Jumal, talitanud meiega kogu oma headuse ja kogu oma suure halastuse kohaselt,
\par 28 nõnda nagu sa rääkisid oma sulase Moosese läbi päeval, mil sa teda käskisid üles kirjutada oma Seaduse Iisraeli lastele, öeldes:
\par 29 „Kui te ei võta kuulda minu häält, tõesti, siis see suur ja rohke rahvameri jääb piskuks rahvaste seas, kuhu mina nad pillutan.
\par 30 Sest ma tean, et nad ei võta mind kuulda, see on ju kangekaelne rahvas. Aga nad muudavad meelt oma vangipõlvemaal
\par 31 ja mõistavad, et mina olen Issand, nende Jumal. Mina annan neile südame ja kõrvad, mis kuulevad.
\par 32 Siis nad kiidavad mind oma vangipõlvemaal ja meenutavad minu nime.
\par 33 Ja nad pöörduvad oma kangekaelsusest ning kuritegudest, sest nad meenutavad oma vanemate teed, kes tegid pattu Issanda ees.
\par 34 Siis ma viin nad tagasi maale, mille ma vandega tõotasin anda nende vanemaile, Aabrahamile, Iisakile ja Jaakobile; nad valitsevad seda jälle ning ma teen nad rohkearvuliseks ja nad ei saa enam piskuks.
\par 35 Ma teen nendega igavese lepingu, et mina olen nende Jumal ja nemad on minu rahvas, ja ma ei aja enam oma Iisraeli rahvast ära maalt, mille ma neile olen andnud.”

\chapter{3}

\section*{Palve}

\par 1 Issand, kõigeväeline Iisraeli Jumal! Ahistatud hing ja murelik vaim hüüab sinu poole!
\par 2 Kuule, Issand, ja halasta, sest meie oleme sinu vastu pattu teinud!
\par 3 Sina ju istud igavesti aujärjel, meie aga peame igavesti hukkuma!
\par 4 Issand, kõigeväeline Iisraeli Jumal! Kuule Iisraeli palvet, niihästi nende, kes on surnud, kui ka nende laste, kes sinu vastu on pattu teinud, kes ei ole võtnud kuulda sinu, oma Jumala häält, mispärast nüüd see õnnetus meie peal ongi!
\par 5 Ära meenuta meie vanemate väärtegusid, vaid meenuta praegu oma käsivart ja oma nime!
\par 6 Sest sina oled Issand, meie Jumal, ja me tahame kiita sind, Issand!
\par 7 Sest sa oled pannud meie südamesse kartuse sinu ees, selleks et hüüaksime appi sinu nime ja kiidaksime sind oma vangipõlvemaal, kui oleme oma südamest kõrvaldanud kõik oma vanemate väärteod, kes tegid pattu sinu vastu.
\par 8 Vaata, me oleme tänapäevalgi oma vangipõlvemaal, kuhu sa meid pillutasid teotamiseks ja needmiseks ning süüaluseks kõigi nende väärtegude pärast, mida tegid meie vanemad, taganedes Issandast, meie Jumalast.”

\section*{Laul elutarkusest}

\par 9 „Kuule, Iisrael, elu käske, pange tähele, et õpiksite mõistma!
\par 10 Mispärast, Iisrael, mispärast sa oled vaenlaste maal? Vanaks oled saanud võõral maal!
\par 11 Sa oled rüve koos surnutega, sind loetakse surmavallas olijate hulka.
\par 12 Sa oled hüljanud tarkuse allika.
\par 13 Kui oleksid käinud Jumala teel, siis oleksid rahus elanud igavesti.
\par 14 Õpi nüüd, kus on mõistmine, kus on jõud, kus on tarkus, et sa ühtlasi teaksid, kus on pikk iga ja elu, kus on silmavalgus ja rahu!
\par 15 Kes on leidnud tarkuse asupaiga ja kes on pääsenud selle varamusse?
\par 16 Kus on rahvaste juhid ja need, kes valitsevad maa loomi,
\par 17 need, kes mängisid taeva lindudega, kes kuhjasid hõbedat ja kulda, mille peale inimesed loodavad? Jah, ei olnud piiri nende rikkusel!
\par 18 Ja need, kes hõbedat sepistasid niisuguse hoolega, et nende tööd on seletamatud?
\par 19 Need on kadunud ja läinud alla surmavalda, teised on astunud nende asemele.
\par 20 Nooremad nägid valgust ja asustasid maa, aga tarkuse teed nad ei tundnud
\par 21 ega mõistnud selle radu; nende pojadki ei püsinud seal, vaid jäid kaugele selle teest.
\par 22 Sellest ei kuuldud Kaananis ega nähtud seda Teemanis.
\par 23 Haagari lapsed, kes otsisid arukust maa peal, Merrani ja Teemani kaupmehed, muinasjuttude jutustajad ja arukuse otsijad - tarkuse teed nad ei tundnud ega märganud selle radu.
\par 24 Oh Iisrael, kui suur on Jumala koda ja kui avar tema omandi asupaik!
\par 25 See on suur ja otsatu, kõrge ja mõõtmatu.
\par 26 Seal sündisid hiiglased, kuulsad muistsest ajast, pikakasvulised sõjakangelased.
\par 27 Neid aga Jumal ei valinud ega andnud neile tarkuse teed.
\par 28 Nad hukkusid, sest neil ei olnud tarkust, hukkusid rumaluse pärast.
\par 29 Kes on läinud taevasse ja võtnud tarkuse ning toonud selle alla pilvedest?
\par 30 Kes on üle mere läinud ja selle leidnud, tuues seda puhta kulla eest?
\par 31 Ei ole kedagi, kes tunneks tarkuse teed, mitte kedagi, kes südamesse võtaks selle teeraja.
\par 32 Aga tema, kes kõike teab, tunneb seda, mõistusega on ta selle leidnud, tema, kes maa on teinud igaveseks ajaks, täitnud selle neljajalgsete loomadega,
\par 33 tema, kes läkitab valguse ja see läheb, kes kutsub seda ja see kuulab teda värisedes.
\par 34 Tähed säravad oma valvepostidel ja rõõmustavad.
\par 35 Ta kutsub neid, ja nad ütlevad: „Siin me oleme!” Nad säravad rõõmsasti temale, kes nad on teinud.
\par 36 Niisugune on meie Jumal, temaga sarnaseks ei saa pidada mitte ühtki teist.
\par 37 Tema leidis iga tarkuse tee ja andis selle oma sulasele Jaakobile ning oma armastatud Iisraelile.
\par 38 Seejärel nähti seda maa peal ja läbi käimas inimestega.

\chapter{4}

\section*{Laul elutarkusest}

\par 1 Tarkus on Jumala käskude raamat, seadus, mis jääb igavesti. Kõik, kes seda peavad, jäävad elama, aga need, kes selle hülgavad, surevad.
\par 2 Pöördu, Jaakob, ja haara sellest kinni, käi selle valguse paistuses!
\par 3 Ära anna oma au teisele, ega seda, mis sulle on kasulik, võõrale rahvale!
\par 4 Õndsad oleme, oh Iisrael, sest meie teame, mis on Jumalale meelepärane.”

\section*{Jeruusalemma kurb saatus ja troost}

\par 5 „Ole julge, minu rahvas, „Iisraeli” mälestuse hoidja!
\par 6 Teid on küll müüdud paganaile, aga mitte hävitamiseks. Et olete Jumalat vihastanud, siis on teid antud vaenlaste kätte.
\par 7 Sest teie äratasite pahameele temal, kes teid on teinud, ohverdades kurjadele vaimudele, aga mitte Jumalale.
\par 8 Teie unustasite oma toitja, igavese Jumala, ja kurvastasite oma kasvatajat, Jeruusalemma.
\par 9 Sest tema nägi Jumala viha tulevat teie peale ja ütles: „Kuulge, Siioni naabrid, Jumal on mulle saatnud suure leina!
\par 10 Sest ma nägin oma poegade ja tütarde vangistamist, mis Igavese tahtel neid tabas.
\par 11 Mina kasvatasin neid küll rõõmuga, aga nutu ja leinaga pidin nad ära saatma.
\par 12 Ükski ärgu rõõmustagu minu pärast, et olen lesknaine ja kõigist maha jäetud! Ma olen üksi jäänud oma laste pattude pärast, sest nad taganesid Jumala Seadusest
\par 13 ega hoolinud tema määrustest, ei käinud Jumala käskude teed ega astunud tema õiguse õpetusteel.
\par 14 Tulge, Siioni naabrid, ja meenutage minu poegade ja tütarde vangistamist, mis Igavese tahtel neid tabas!
\par 15 Sest ta tõi nende kallale kauge rahva, jultunud ja umbkeelse rahva, kes ei häbenenud vana ega säästnud last,
\par 16 vaid viis lesknaiselt tema armsamad ja röövis tütred sellelt, kes oli jäänud üksikuks.
\par 17 Aga mina - kuidas saan mina teid aidata?
\par 18 Tema, kes õnnetuse laskis tulla, tema vabastab teid vaenlaste käest.
\par 19 Minge, lapsed, minge! Sest mina pean üksinda maha jääma.
\par 20 Mina panin ära õnneaja rüü ja riietusin palvetaja kotiriidesse. Ma tahan hüüda Igavese poole, niikaua kui ma elan.
\par 21 Olge julged, lapsed, hüüdke appi Jumalat, siis ta vabastab teid vägivallast, vaenlaste käest!
\par 22 Sest ma loodan Igaveselt teie päästmist ja minul on rõõm Pühalt halastuse pärast, mis teile peagi osaks saab teie igaveselt Päästjalt.
\par 23 Mina küll saatsin teid ära leina ja nutuga, aga Jumal annab teid minule tagasi igaveseks ajaks hõiskamise ja rõõmuga.
\par 24 Nõnda nagu Siioni naabrid nüüd on näinud teie vangistamist, nõnda nad saavad peagi näha teie päästmist Jumala poolt, mis teile osaks saab suure kirkuse ja hiilgusega Igaveselt.
\par 25 Lapsed, kandke kannatlikult Jumala viha, mis teie peale on tulnud! Vaenlane on sind küll jälitanud, aga sina saad peagi näha tema hukkumist ja sina paned jala neile kaela peale.
\par 26 Minu lemmikud on pidanud käima konarlikke teid, neid aeti just nagu vaenlaste poolt röövitud karja.
\par 27 Olge tublid, lapsed, ja hüüdke Jumala poole, sest tema, kes seda laskis sündida, peab teid meeles!
\par 28 Sest nõnda nagu teil oli mõte taganeda Jumalast, nõnda pöörduge nüüd teda otsima kümnekordse innuga!
\par 29 Sest tema, kes selle õnnetuse teile saatis, toob teile koos päästega igavese rõõmu.”
\par 30 Ole julge, Jeruusalemm, tema, kes sulle nime andis, trööstib sind!
\par 31 Õnnetud on need, kes sulle kurja tegid ja sinu languse pärast rõõmustasid.
\par 32 Õnnetud on linnad, keda sinu lapsed pidid orjama. Õnnetu on, kes võttis vastu sinu pojad.
\par 33 Sest nõnda nagu ta hõiskas sinu languse pärast ja tundis rõõmu sinu õnnetusest, nõnda peab ta leinama omaenese hävingu pärast.
\par 34 Ja „mina võtan sealt rahvahulkade rõõmurõkatuse ning tema uhkus muutub leinaks”.
\par 35 Sest Igaveselt tuleb tema külge tuli paljudeks päevadeks, ja kurjad vaimud elavad seal kaua.
\par 36 Tõsta silmad päikesetõusu poole, Jeruusalemm, ja vaata rõõmu, mis sulle tuleb Jumalalt!
\par 37 Vaata, sinu pojad tulevad, need, keda sa pidid laskma minna. Nad tulevad, kogutuna päikesetõusu ja päikeseloojaku poolt Püha käsul, Jumala aulikkusest rõõmu tundes.

\chapter{5}

\section*{Jeruusalemma kurb saatus ja troost}

\par 1 Heida ära, Jeruusalemm, oma leina ja alanduse rüü ja riietu igaveseks Jumala auhiilguse toredusega!
\par 2 Pane selga Jumala õigusekuub, aseta Igavese auhiilgus krooniks pähe!
\par 3 Sest Jumal näitab sinu sära kõigile, kes taeva all on.
\par 4 Sest sina saad igaveseks Jumala nime: „Õiguse rahu” ja „Jumalakartlikkuse kirkus”.
\par 5 Tõuse, Jeruusalemm, ja astu kõrgendikule, tõsta silmad päikesetõusu poole ja vaata oma lapsi, kes Püha käsul on tulnud kokku päikeseloojaku ja päikesetõusu poolt, tundes rõõmu, et Jumal on neid meeles pidanud!
\par 6 Sest jalgsi läksid nad sinu juurest vaenlaste aetuina, aga Jumal toob nad tagasi sinu juurde austusega kantuina nagu kuninga aujärjel.
\par 7 Sest Jumal on käskinud, et kõik kõrged mäed ja igavesed kingud tehtaks madalaks ja orud täidetaks tasaseks maaks, et Iisrael võiks julgesti minna Jumala auhiilgusesse.
\par 8 Metsad ja kõik healõhnalised puudki annavad Iisraelile varju Jumala käsul.
\par 9 Sest Jumal juhib Iisraeli rõõmuga oma auhiilguse valgusesse, halastuse ja õigusega, mis temalt tuleb.”

\chapter{6}

\section*{Jeremija kiri}

\par 1 Ärakiri kirjast, mille Jeremija saatis neile, keda Paabeli  kuningas oli viimas vangidena Paabelisse, et neile kuulutada, mida  Jumal oli käskinud ütelda: „Nende pattude pärast, mida te olete teinud Jumala vastu, viib  Nebukadnetsar, Paabeli kuningas, teid vangidena Paabelisse.
\par 2 Kui olete jõudnud Paabelisse, siis te jääte sinna paljudeks  aastateks ja pikaks ajaks, seitsmeks inimpõlveks. Aga pärast seda  toon ma teid sealt ära rahus.
\par 3 Paabelis te näete nüüd hõbe-, kuld- ja puujumalaid, keda õlgadel  kantakse, kes paganaile hirmu teevad.
\par 4 Vaadake siis ette, et teiegi ei saaks muulaste sarnaseks ega  hakkaks neid ebajumalaid kartma,
\par 5 nähes, et rahvahulk kummardab neid ees ja taga! Pigem ütelge  südames: „Sind, Issand, peab kummardama!”
\par 6 Sest minu ingel on teiega ja tema otsib teie hingi.
\par 7 Ebajumalate keeled on voolitud käsitööliste poolt. Nad ise  on kullatud ja hõbetatud, nad on pettus, ja nad ei saa rääkida.
\par 8 Otsekui ehteid armastava neitsi jaoks võetakse kulda, et  valmistada kroone oma jumalate pähe.
\par 9 Aga juhtub ka, et preestrid varastavad oma jumalailt kulda ja  hõbedat iseenese tarbeks, andes neilt võetut ka hooradele, kes nende  katuse all on.
\par 10 Neid hõbe-, kuld- ja puujumalaid ehivad nad riietega nagu  inimesi. Need aga ei suuda ennast kaitsta tuhmumise ja koide vastu.
\par 11 Olgugi nad riietatud purpurrüüsse, tuleb nende palgeilt pühkida  templi tolmu, mida nende peal on paksult.
\par 12 Ebajumalal on valitsuskepp nagu inimesel, maa kohtumõistjal,  ometi ei saa ta hukata seda, kes tema vastu pattu teeb.
\par 13 Tal on käes mõõk ja kirves, kuigi ta ei saa ennast päästa  sõjast ega röövlite käest.
\par 14 Sellest tuntakse, et nad ei ole jumalad. Ärge siis kartke neid!
\par 15 Sest otsekui inimese katkiläinud tarbeese on muutunud kõlbmatuks,  nõnda on lugu ka nende templitesse pandud jumalatega.
\par 16 Nende silmad on täis tolmu sisseastujate jalgadest.
\par 17 Ja nõnda kui sellele, kes on eksinud kuninga vastu, suletakse väljapääsud  nagu hukkamisele viidavale, nõnda ka preestrid kindlustavad  nende templeid uste, lukkude ja riividega, et röövlid ei saaks neid  röövida.
\par 18 Nad läidavad neile lampe rohkem kui iseendile, ometi ei näe  nad midagi.
\par 19 Nendega on lugu nagu templi talaga: nende süda, nagu öeldakse,  kõduneb. Maa järaskid närivad neid ja nende riideid, ilma et nad  märkaksid.
\par 20 Nende pale on mustunud templi suitsust.
\par 21 Nende keha peale ja pähe lendavad nahkhiired, pääsukesed ja muud  linnud. Kassidki istuvad sinna.
\par 22 Sellest te mõistate, et need ei ole jumalad. Ärge kartke neid!
\par 23 Kui kullalt, millega need on toredaks karratud, mustust ei  pühita, siis nad ei hiilga. Nad ei märganudki, et neid valati.
\par 24 Need on ostetud mis tahes hinnaga, ometi ei ole neis hinge.
\par 25 Et nad on jalutud, siis kantakse neid õlgadel, nõnda näitavad  nad  inimestele oma häbi, ja needki, kes neid teenivad, peavad häbenema,  sest kui ebajumal maha kukub, siis ei tõuse ta eales ise üles.
\par 26 Ja kui ta on püsti pandud, siis ta iseenesest ei liigu, või kui  ta on kokku vajunud, siis ta ennast sirgu ei aja, vaid tema ette pannakse  ohvriande nagu surnutele.
\par 27 Neile ohverdatu müüvad nende preestrid, tarvitades seda omaenese  kasuks. Nõndasamuti teevad ka nende naised: soolavad ohvriliha ega  jaota seda vaesele ja väetile. Nende ohvreid puudutavad roojased ja  nurganaised.
\par 28 Kui te nüüd sellest teate, et need ei ole jumalad, siis ärge  kartke neid!
\par 29 Sest kuidas saakski neid nimetada jumalaiks? Naised ju katavad  laua hõbe-, kuld- ja puujumalaile.
\par 30 Nende templites askeldavad preestrid lõhkikäristatud  kuubedes, pea ja habe pöetud, katmata peaga.
\par 31 Nad uluvad, kisendades jumalate ees, nagu mõningail on viisiks  surnu mälestussöömaajal.
\par 32 Preestrid võtavad nende riideist osa ning riietavad sel  viisil oma naisi ja lapsi.
\par 33 Kui keegi teeb neile kurja või head, siis ei ole nad võimelised  tasuma. Kuningat nad ei saa valitsema panna ega ära ajada.
\par 34 Nõndasamuti ei saa nad anda rikkust ega raha. Kui keegi annab  neile tõotuse, seda aga ei täida, siis ei saa nad täitmist nõuda.
\par 35 Nad ei saa inimest päästa surmast ega väetit vabastada vägevama  käest.
\par 36 Nad ei saa pimedat teha nägijaks ega hädasolijat päästa.
\par 37 Nad ei heida armu lesele ega tee head orvule.
\par 38 Need kullatud ja hõbetatud  puukujud on sarnased mäest murtud kividega. Häbisse jäävad, kes neid  teenivad.
\par 39 Kuidas saab neid siis pidada või hüüda jumalaiks?
\par 40 Pealegi kaldealased ise põlgavad neid, sest kui nad näevad  tumma, kes ei saa rääkida, siis nad toovad välja Beeli ja nõuavad  temalt, justkui tema midagi mõistaks, et tumm saaks kõnelda.
\par 41 Nad ei suuda uskuda ega neist loobuda, sest neil pole  mõistust.
\par 42 Naised istuvad aga teede ääres, vööd vööl, suitsutades kliisid.
\par 43 Kui siis keegi mööduja on mõne neist kaasa viinud ja temaga  armatsenud, siis see naine pilkab kõrvalistujat, et seda ei ole nõnda  austatud kui teda ja et selle vöö ei ole rebitud.
\par 44 Kõik, mis ebajumalate juures sünnib, on pettus. Kuidas saab siis  neid pidada või hüüda jumalaiks?
\par 45 Need on käsitööliste ja kullasseppade valmistatud, need ei  muutu, vaid jäävad selleks, mis nad meistrite tahtel on.
\par 46 Kui nüüd nende valmistajad ise kaua ei ela, kuidas saaksid siis  nende poolt valmistatud olla jumalad?
\par 47 Nad jätavad oma järeltulijaile vale ning häbi.
\par 48 Sest kui neid tabab sõda või muu õnnetus, siis preestrid peavad  omavahel nõu, kus nad saaksid koos nendega end varjata.
\par 49 Kuidas nüüd võiks olla teadmata, et need ei ole jumalad, kes  iseennast ei suuda päästa sõjast või õnnetusest?
\par 50 Et need on ainult kullatud ja hõbetatud puukujud, siis saab  ükskord teatavaks, et nad on pettus. Kõigile rahvaile ja kuningaile  saab avalikuks, et need ei ole jumalad, vaid on inimeste kätetöö, ja  et neis ei  ole mingit jumalikku jõudu.
\par 51 Kes võiks siis olla teadmatuses, et need ei ole jumalad?
\par 52 Nad ei saa tõsta maale kuningat ega anda inimestele vihma.
\par 53 Olles võimetud, ei suuda nad iseenesele teha õigust ega tõrjuda  ülekohut. Nad on nagu varesed taeva ja maa vahel.
\par 54 Ja iga kord, kui tuli lahti pääseb nende puiste, kullatud ja  hõbetatud jumalate templis, nende preestrid põgenevad ja pääsevad,  nemad ise aga põlevad tuhaks nagu palgidki.
\par 55 Kuninga ja vaenlaste vastu nad ei saa võidelda.
\par 56 Kuidas saab siis oodata ja uskuda, et nad on jumalad?
\par 57 Varaste ja röövlite käest ei saa iseennast päästa need puised,  hõbetatud ja kullatud jumalad. Kes on nendest kangemad, võtavad neilt  kulla ja hõbeda ning riided, mis neil seljas on, ja lähevad nendega  ära. Nemad ise ei suuda ennast aidata.
\par 58 Sellepärast on parem olla kuningas, kes osutab vaprust, või maja  tarbeese, mida omanik saab kasutada, kui niisugused ebajumalad. Ka  maja uks, mis kaitseb seesolevat, on väärtuslikum kui niisugused  ebajumalad, ja kuningakoja puutalagi on väärtuslikum kui niisugused  ebajumalad.
\par 59 Päike, kuu ja tähed, mis paistavad ja on läkitatud kasu tooma,  on sõnakuulelikud.
\par 60 Nõndasamuti ka välk: ilus on näha, kui see sähvatab. Nõnda on  lugu ka tuulega, mis puhub üle kogu maa.
\par 61 Ja kui Jumal käsib pilvi rännata üle kogu asustatud maa, siis  need täidavad käsu. Tuli, mis ülevalt on saadetud hävitama mägesid ja  metsi, täidab käsu.
\par 62 Ebajumalad ei ole aga loomult ja jõult nendega võrdsed.
\par 63 Sellepärast ei tule neid pidada ega hüüda jumalaiks. Nad ei ole  võimelised kohut mõistma ega inimestele head tegema.
\par 64 Kui te nüüd teate, et need ei ole jumalad, siis ärge neid  kartke!
\par 65 Sest nad ei saa kuningaid needa ega õnnistada.
\par 66 Nad ei saa rahvaile näidata taevamärke, nad ei paista nagu päike  ega valgusta nagu kuu.
\par 67 Loomadki on neist paremad, sest nad saavad iseennast aidata  varju alla põgenedes.
\par 68 Meil ei ole seega vähimatki teadmist sellest, et nad oleksid  jumalad. Sellepärast ärge neid kartke!
\par 69 Otsekui kurgipõllu linnupeletised, mis midagi ei kaitse, on ka  nende puised, kullatud ja hõbetatud jumalad.
\par 70 Selsamal viisil võib nende puiseid, kullatud ja hõbetatud  jumalaid võrrelda aia põõsastaraga, kuhu kõik linnud istuvad, või  laibaga, mis on heidetud pimedusse.
\par 71 Ka nende kõdunevaist purpur- ja läikriideist tuntakse, et need  ei ole jumalad. Lõpuks järatakse ka neid endid ning nad saavad  maale teotuseks.
\par 72 Sellepärast on parem õige inimene, kellel ei ole  ebajumalakujusid. Sest tema on teotusest kaugel.”

\end{document}