\begin{document}

\title{Jesaja}

\chapter{1}

\par 1 Jesaja, Aamotsi poja nägemus, mida ta nägi Juuda ja Jeruusalemma kohta Juuda kuningate Ussija, Jootami, Aahase ja Hiskija päevil:
\par 2 Kuule, taevas, ja maa, pane tähele, sest Issand kõneleb: Mina kasvatasin lapsi ja lasksin neil sirguda suureks, aga nemad astusid üles mu vastu.
\par 3 Härg tunneb oma peremeest ja eesel oma isanda sõime, aga Iisrael ei tunne, mu rahvas ei taha mõista.
\par 4 Häda patusele rahvale, süüga koormatud rahvale, kurjategijate soole, kõlvatuile lastele! Nad on jätnud maha Issanda, nad on põlanud Iisraeli Püha, nad on pööranud talle selja.
\par 5 Kuhu tuleks teid veel lüüa teie tõrksuse jätkudes? Pea on üleni haige ja süda täiesti jõuetu.
\par 6 Jalatallast pealaeni ei ole midagi tervet: aina vermed, muhud ja värsked haavad, mida ei ole puhastatud ega seotud ega õliga leevendatud.
\par 7 Teie maa on laastatud, linnad tulega põletatud, võõrad söövad teie põllud tühjaks otse teie silme all. Kõik on laastatud otsekui segipaisatud Soodom.
\par 8 Siioni tütar on jäänud üksi otsekui vahionn viinamäel, otsekui öömaja kurgipõllul, otsekui sissepiiratud linn.
\par 9 Kui vägede Issand ei oleks meile jätnud pääsenuid, olgugi pisut, oleksime olnud nagu Soodom, Gomorra sarnased.
\par 10 Kuulge Issanda sõna, Soodoma pealikud, pane tähele meie Jumala õpetust, Gomorra rahvas!
\par 11 Milleks mulle teie tapaohvrite hulk? ütleb Issand. Ma olen küllastunud teie põletusohvrite jääradest ja nuumveiste rasvast. Härjavärsside, tallede ja sikkude veri ei meeldi mulle.
\par 12 Kui te tulete vaatama mu palet, kes nõuab siis teilt mu õuede tallamist?
\par 13 Ärge tooge enam tühiseid ande, ohvrisuits on mulle vastumeelt. Noorkuu ja hingamispäev, koguduse kokkukutsumine - ma ei salli nurjatust ja pühapidamist ühtaegu.
\par 14 Mu hing vihkab teie noorkuid ja seatud pühi; need on mulle koormaks, ma olen tüdinud neid talumast.
\par 15 Kui te käsi sirutate, peidan ma oma silmad teie eest, kui te ka palju palvetate, ei kuule ma mitte, sest teie käed on täis verd!
\par 16 Peske endid, puhastage endid, saatke oma tegude kurjus mu silme eest, lakake paha tegemast!
\par 17 Õppige tegema head, nõudke õigust, aidake rõhutut, mõistke vaeslapsele õigust, lahendage lesknaiste kohtuasju!
\par 18 Tulge nüüd ja seletagem isekeskis, ütleb Issand. Kuigi teie patud on helepunased, saavad need lumivalgeks; kuigi need on purpurpunased, saavad need villa sarnaseks.
\par 19 Kui te tahate ja kuulate, siis te saate süüa maa parimat vilja.
\par 20 Aga kui te tõrgute ja panete vastu, siis sööb teid mõõk. Jah, Issanda suu on kõnelnud.
\par 21 Kuidas küll sai hooraks ustav linn? Ta oli täis õigust, õiglus asus seal. Aga nüüd - tapjad!
\par 22 Su hõbe on muutunud räbuks, su vein on veega lahjendatud.
\par 23 Su vürstid on tõrksad ja varaste seltsimehed: igaüks armastab meelehead ja himustab kingitusi. Vaeslapsele ei tee nad õigust ja lesknaise kohtuasi ei jõua nende ette.
\par 24 Seepärast ütleb Jumal, vägede Issand, Iisraeli Vägev: Ma tunnen rõõmu oma vastastest ja maksan kätte oma vaenlastele.
\par 25 Ma pööran su vastu oma käe ja puhastan su räbu otsekui leelisega ning eraldan sinust kõik tina.
\par 26 Siis ma annan sulle taas kohtumõistjaid nagu muistegi, ja nõuandjaid, nagu oli alguses. Seejärel hüütakse sind õigluse linnaks, ustavaks linnaks.
\par 27 Siion lunastatakse kohtu ja tema pöördunud õigluse läbi.
\par 28 Aga üleastujate ja patuste langus tuleb ühekorraga, ja need, kes hülgavad Issanda, hukkuvad.
\par 29 Sest te peate häbenema tammede pärast, mida te ihaldasite, ja tundma piinlikkust rohuaedade pärast, mis te valisite.
\par 30 Sest te saate tamme sarnaseks, mille lehed närtsivad, ja olete otsekui rohuaed, millel ei ole vett.
\par 31 Tugev saab takuks ja tema tegu sädemeks: mõlemad põlevad üheskoos, aga kustutajat ei ole.

\chapter{2}

\par 1 Kõne Jesaja, Aamotsi poja nägemusest Juuda ja Jeruusalemma kohta:
\par 2 Aga viimseil päevil sünnib, et Issanda koja mägi seisab kindlana kui mägede tipp ja tõuseb kõrgemale küngastest ning kõik paganad voolavad ta juurde.
\par 3 Ja paljud rahvad lähevad ning ütlevad: „Tulge, mingem üles Issanda mäele, Jaakobi Jumala kotta, et ta meile õpetaks oma teid ja et võiksime käia tema radu: sest Siionist lähtub Seadus ja Jeruusalemmast Issanda sõna!”
\par 4 Ja tema mõistab kohut paganate vahel ning juhatab paljusid rahvaid. Siis nad taovad oma mõõgad sahkadeks ja piigid sirpideks; rahvas ei tõsta mõõka rahva vastu ja nad ei õpi enam sõdimist.
\par 5 Jaakobi sugu, tulge, käigem Issanda valguses!
\par 6 Kuid sa oled hüljanud oma rahva, Jaakobi soo, sest nad on täis idamaa kombeid ja nõiuvad vilistite sarnaselt, nad löövad kätt võõramaa lastega.
\par 7 Nende maa on täis hõbedat ja kulda, nende varandustel ei ole lõppu; nende maa on täis hobuseid ja nende sõjavankreid on otsatult.
\par 8 Nende maa on täis ebajumalaid, nad kummardavad oma kätetööd, seda, mis nende sõrmed on teinud.
\par 9 Aga inimene painutatakse ja mees heidetakse maha. Ära anna neile andeks!
\par 10 Poe kalju sisse ja peida ennast põrmu Issanda kartuse pärast ja tema kõrge aupaistuse eest!
\par 11 Inimeste suurelised silmad alandatakse ja meeste kõrkus painutatakse. Jah, sel päeval on Issand üksinda kõrge.
\par 12 Sest vägede Issandal on päev: kõigi suureliste ja kõrkide jaoks, kõigi jaoks, kes kõrgele on tõusnud, et neid alandada,
\par 13 kõigi Liibanoni kõrgete ja uhkete seedrite jaoks ja kõigi Baasani tammede jaoks,
\par 14 kõigi kõrgete mägede jaoks ja kõigi kõrgemate küngaste jaoks,
\par 15 kõigi kõrgete tornide jaoks ja kõigi tugevate müüride jaoks,
\par 16 kõigi Tarsise laevade jaoks, kõigi toredate laevade jaoks.
\par 17 Siis painutatakse inimeste ülbus ja alandatakse meeste kõrkus. Jah, sel päeval on Issand üksinda kõrge.
\par 18 Ebajumalad aga kaovad täiesti.
\par 19 Siis poetakse kaljulõhedesse ja muldkoobastesse Issanda kartuse pärast ja tema kõrge aupaistuse eest, kui ta tõuseb maad hirmutama.
\par 20 Sel päeval viskavad inimesed oma hõbedased ja kuldsed ebajumalad, mis nad endile kummardamiseks on teinud, muttidele ja nahkhiirtele,
\par 21 et aga saaks pugeda kaljupragudesse ja kivilõhedesse Issanda kartuse pärast ja tema kõrge aupaistuse eest, kui ta tõuseb maad hirmutama.
\par 22 Seepärast lakake lootmast inimestele, kellel on ainult hingeõhk ninas, sest neid ei saa ju panna mikski!

\chapter{3}

\par 1 Sest vaata, Jumal, vägede Issand, võtab Jeruusalemmalt ja Juudalt toe ja toetuse, kõik leiva- ja veetoed,
\par 2 sangari ja sõjamehe, kohtumõistja ja prohveti, ennustaja ja vanema,
\par 3 viiekümnepealiku ja lugupeetud mehe, nõuniku, osava meistri ja lausuda oskaja.
\par 4 Ja ma panen poisid neile vürstideks ning lapsed valitsevad nende üle.
\par 5 Siis hakkab rahvas isekeskis rõhuma, üks ühte ja teine teist, noor tõuseb vana vastu ja vääritu auväärse vastu.
\par 6 Kui vend haarab oma isakojas vennast: „Sul on kuub, ole meile juhiks ja võta see rusuhunnik oma käe alla!”,
\par 7 siis tõstab too selsamal päeval häält, öeldes: „Ei ole minust haavasidujat, mu kojas ei ole leiba ega kuube: ärge pange mind rahva juhiks!”
\par 8 Sest Jeruusalemm komistab ja Juuda langeb, kuna nende keel ja teod on Issanda vastu, et trotsida Aulise silme ees.
\par 9 Nende näoilme tunnistab nende endi vastu, nad kuulutavad oma pattu salgamata otsekui Soodom. Häda nende hingedele! Sest nad teevad iseendile kurja.
\par 10 Öelge õigele, et tema käsi käib hästi, sest nad saavad süüa oma tegude vilja!
\par 11 Häda õelale! Tema käsi käib halvasti, sest ta kätetöö tasutakse talle.
\par 12 Mu rahva sundijaks on laps ja teda valitsevad naised. Mu rahvas! Su juhid on eksitajad, nad muudavad valeks su radade suuna.
\par 13 Issand tõuseb nõutama õigust, ta asub kohut mõistma rahvastele.
\par 14 Issand läheb kohtusse oma rahva vanemate ja vürstidega: Te olete laastanud viinamäed ja teie kodades on vaestelt röövitu.
\par 15 Mida te ometi mõtlete, tallates mu rahvast ja rutjudes vaeste nägu? küsib Jumal, vägede Issand.
\par 16 Ja Issand ütleb: Sellepärast et Siioni tütred on upsakad, käivad kange kaelaga ja pööritavad silmi, käies astuvad tippamisi ja kliristavad jalarõngaid,
\par 17 teeb Issand Siioni tütarde pealae kärnaseks ja Issand paljastab nende häbeme.
\par 18 Sel päeval võtab Issand ära toredad jalarõngad, laubaehted ja kaelakeed,
\par 19 kõrvarõngad, käevõrud ja loorid,
\par 20 peamähised, sammuahelakesed, rinnavööd, lõhnatoosid ja amuletid,
\par 21 sõrmused ja ninarõngad,
\par 22 riided, pealisrüüd, suurrätikud ja kukrud,
\par 23 peeglid, särgid, peakatted ja hõlstid.
\par 24 Ja siis sünnib, et palsamilõhna asemel on kõdunemise lehk, vöö asemel nöör, palmitud juuste asemel paljas pea, peene lõuendi asemel kotiriidest kate, ilu asemel põletusmärk.
\par 25 Su mehed langevad mõõga läbi ja su sangarid sõjas.
\par 26 Siis kurvastavad ja leinavad ta väravad, ta istub maa peal tühjaks tehtuna.

\chapter{4}

\par 1 Sel päeval haaravad seitse naist kinni ühest mehest, öeldes: „Me sööme omaenese leiba ja kanname omaenese riideid, kui meid ainult hüütaks sinu nime järgi. Võta ära meie teotus!”
\par 2 Sel päeval on Issanda võsu iluks ja auks, ning maa vili on uhkuseks ja ehteks Iisraeli pääsenuile.
\par 3 Ja kes Siionis alles jääb ning Jeruusalemmas järele jääb, seda hüütakse pühaks, igaühte, kes Jeruusalemmas on eluks kirja pandud,
\par 4 siis kui Issand on ära pesnud Siioni tütarde saasta ja uhtunud Jeruusalemmast ta veresüü kohtu- ning põletusvaimuga.
\par 5 Ja Issand loob kogu Siioni mäe asukoha ning sealse kogunemispaiga kohale pilve päevaks ning suitsu ja leekiva tulepaistuse ööseks, sest kõige kohale asub auhiilgus kaitsevarjuks.
\par 6 Ja katus on varjuks päeva palavuse eest ja pelgupaigaks ning ulualuseks rajuilma ja vihma puhul.

\chapter{5}

\par 1 Ma tahan nüüd laulda oma armsamast, mu armsama laulu tema viinamäest. Mu armsamal on viinamägi viljakal mäenõlvakul.
\par 2 Ta kaevas ja puhastas kividest ning istutas häid viinapuid; ta ehitas selle keskele torni, ta raius sinna ka surutõrre. Siis ta ootas, et see kasvataks häid kobaraid, ent kängunud olid kobarad, mida see andis.
\par 3 Ja nüüd, Jeruusalemma elanikud ja Juuda mehed, mõistke ometi kohut minu ja mu viinamäe vahel!
\par 4 Kas oleks olnud vaja teha mu viinamäega veel midagi, mida olin jätnud tegemata? Miks ma ootasin, et ta kasvatab häid kobaraid, kuna ta andis kängunud kobaraid?
\par 5 Aga nüüd tahan ma ometi teha teile teatavaks, mida ma teen oma viinamäega: ma kisun maha ta aia, ja ta jääb laastatavaks, ma lammutan ta müüri, ja ta jääb tallatavaks.
\par 6 Nõnda ma hävitan ta. Teda ei kärbita ega rohita, ta kasvatab kibuvitsu ja ohakaid ja ma keelan pilvi temale vihma andmast.
\par 7 Sest Iisraeli sugu on vägede Issanda viinamägi ja Juuda mehed tema lemmikistandik. Ta ootas õigust, aga vaata, tuli õigusetus; õiglust, aga vaata, tuli hädakisa.
\par 8 Häda neile, kes reastavad koja kõrvale koja ja lisavad põllule põllu, kuni enam ei jää üle paikagi, vaid üksnes teie elate maal!
\par 9 Mu kõrvus on: Vägede Issand, tõesti, paljud kojad jäävad tühjaks, suured ja kaunid elanikest lagedaks.
\par 10 Sest kümme adramaad viinamäge annab ainult üheksa kannu ja viiest vakast seemnest saab pool vakka.
\par 11 Häda neile, kes tõusevad hommikul vara, et jahtida vägijooki, kes viidavad aega hilja ööni, kuni vein paneb nad õhetama.
\par 12 Kandled ja naablid, trummid ja viled ja vein on nende joominguil, aga Issanda tegusid nad ei märka ja tema kätetööd nad ei näe.
\par 13 Sellepärast läheb mu rahvas ootamatult vangi, tema auväärsed mehed näevad nälga ja ta rahvahulgad närbuvad janust.
\par 14 Sellepärast avab ennast surmavald ja ajab oma suu otsatult pärani: alla peab astuma niihästi hiilgus kui rüsin, niihästi lärm kui rõõmurõkatus.
\par 15 Inimene painutatakse ja mees heidetakse maha ning suureliste silmad alandatakse.
\par 16 Aga vägede Issand on kohtus kõrge ja Püha Jumal osutub pühaks õigluses.
\par 17 Ja seal käivad talled karjas kui oma karjamaal ja rusude vahel söövad võõrad.
\par 18 Häda neile, kes nööridega tõmbavad enese peale süüd ja otse vankriköitega pattu,
\par 19 kes ütlevad: „Tõtaku tema, tehku kiiresti, et me saaksime näha, ja liginegu ning tulgu Iisraeli Püha nõu, et me saaksime teada!”
\par 20 Häda neile, kes hüüavad kurja heaks ja head kurjaks, kes teevad pimeduse valguseks ja valguse pimeduseks, kes teevad kibeda magusaks ja magusa kibedaks!
\par 21 Häda neile, kes on iseenese silmis targad ja iseenese meelest arukad!
\par 22 Häda neile, kes on vägevad veini jooma ja vahvad mehed vägijooki segama,
\par 23 kes meelehea eest annavad õiguse süüdlastele ja võtavad õiguse õigetelt!
\par 24 Sellepärast - nagu tulekeel sööb kõrsi ja kuluhein vajub kokku leegis, nõnda kõduneb nende juur ja nende õied muutuvad tolmuks, sest nad on hüljanud vägede Issanda Seaduse ja on põlanud Iisraeli Püha sõna.
\par 25 Sellepärast on Issanda viha süttinud põlema oma rahva vastu: ta sirutas oma käe tema kohale ja lõi teda, nõnda et mäed vabisesid ja nende laibad olid otsekui sõnnik keset tänavaid. Selle kõige juures ei ole ta viha ära pöördunud ja ta käsi on alles välja sirutatud.
\par 26 Ja ta tõstab lipu kaugele rahvale, vilistab teda maa äärest, ja vaata, see tuleb kiiresti tõtates.
\par 27 Keegi neist ei ole väsinud ja keegi ei komista, ei tuku ega maga; kellelgi ei vallandu vöö vöölt ega katke jalatsipael.
\par 28 Nende nooled on teritatud ja kõik nende ammud on vinnas; nende hobuste kabjad on nagu tulekivi ja nende rattad nagu tuulispea.
\par 29 Nende möirgamine on otsekui emalõvi möirgamine, nad möirgavad nagu noored lõvid; nad urisevad ja haaravad saagi ning viivad oma teed, ja päästjat ei ole.
\par 30 Ja sel päeval on tema kohal mürin otsekui meremüha, ja kui vaadata maale, ennäe, siis on seal ahastuse pimedus ja pilv pimendab valguse.

\chapter{6}

\par 1 Sel aastal, kui kuningas Ussija suri, nägin ma Issandat istuvat suurel ja kõrgel aujärjel, ja tema kuue palistused ulatusid templi seinast seina.
\par 2 Temast kõrgemal seisid seeravid; igaühel neist oli kuus tiiba: kahega ta kattis oma palet, kahega ta kattis oma jalgu ja kahega ta lendas.
\par 3 Ja need hüüdsid üksteisele ning ütlesid: „Püha, püha, püha on vägede Issand! Kogu maailm on täis tema au!”
\par 4 Ja uksepiidad vabisesid hüüdjast häälest ning koda täitus suitsuga.
\par 5 Ja ma ütlesin: „Häda mulle, sest ma olen kadunud! Sellepärast et ma olen roojane mees huultelt ja elan roojaste huultega rahva keskel; sellepärast et mu silmad on näinud kuningat, vägede Issandat.”
\par 6 Siis lendas mu juurde üks seeravitest ja tal oli käes elav süsi, mille ta oli pihtidega altarilt võtnud.
\par 7 Ja ta puudutas mu suud ning ütles: „Vaata, see puudutas sinu huuli ja su süü on lahkunud ning su patt lepitatud.”
\par 8 Ja ma kuulsin Issanda häält küsivat: „Keda ma läkitan? Kes meilt läheks?„ Ja ma ütlesin: ”Vaata, siin ma olen, läkita mind!”
\par 9 Ja tema ütles: „Mine ja ütle sellele rahvale: Kuuldes kuulete ega mõista; nähes näete ega taipa!
\par 10 Tee tuimaks selle rahva süda ja tee raskeks ta kõrvad ning sule ta silmad, et ta oma silmadega ei näeks ja kõrvadega ei kuuleks, oma südamega ei mõistaks ja ei pöörduks ega paraneks!”
\par 11 Aga mina küsisin: „Kui kauaks, Issand?” Ja tema vastas: ”Kuni linnad on rüüstatud, elanikest ilma, kojad inimtühjad, põllud laastatud kõrbeks
\par 12 ja Issand on saatnud inimesed kaugele ning palju sööti on jäänud keset maad.
\par 13 Ja kui seal veel alles jääb kümnes osa, siis seegi põletatakse ära, nagu terebint või tamm, millest langedes jääb järele känd. Aga ta känd on püha seeme.”

\chapter{7}

\par 1 Ja neil päevil, kui Juudas oli kuningaks Aahas, kes oli Ussija poja Jootami poeg, sündis, et Süüria kuningas Retsin ja Iisraeli kuningas Pekah, Remalja poeg, tulid sõdima Jeruusalemma vastu; aga nad ei suutnud seda vallutada.
\par 2 Kui Taaveti kojale anti teada ja öeldi, et Süüria on liidus Efraimiga, siis vabises ta enese ja ta rahva süda, otsekui vabiseksid metsapuud tuule käes.
\par 3 Ja Issand ütles Jesajale: „Mine ometi Aahasele vastu, sina ja su poeg Sear-Jaasub, Ülatiigi veejuhtme otsa juurde, Vanutajavälja maanteele,
\par 4 ja ütle temale: Hoia ennast ja ole rahulik, ära karda ja su süda ärgu mingu araks nende kahe suitseva tukiotsakese pärast - Süüria Retsini ja Remalja poja vihapalangu pärast!
\par 5 Sellepärast et Süüria, Efraim ja Remalja poeg on pidanud sinu vastu kurja nõu, öeldes:
\par 6 „Läki üles Juuda vastu, tükeldagem see ja vallutagem endile ning tõstkem seal Taabali poeg kuningaks!” -
\par 7 ütleb Issand Jumal nõnda: See ei lähe korda ega sünni!
\par 8 Sest Süüria pea on Damaskus ja Damaskuse pea on Retsin. Ja veel kuuskümmend viis aastat - siis on Efraim kui rahvas purustatud.
\par 9 Efraimi pea on Samaaria ja Samaaria pea on Remalja poeg. Kui te ei usu, siis te ei püsi!”
\par 10 Ja Issand käskis öelda Aahasele veel seda:
\par 11 „Küsi enesele Issandalt, oma Jumalalt, tunnustäht sügavusest või kõrgusest!”
\par 12 Aga Aahas vastas: „Ma ei küsi ega kiusa Issandat!”
\par 13 Siis Jesaja ütles: „Kuulge ometi, Taaveti sugu! Ons teil veel vähe inimeste vaevamisest, et te ka mu Jumalat vaevate?
\par 14 Sellepärast annab Issand ise teile tunnustähe: ennäe, neitsi jääb lapseootele ja toob poja ilmale ning paneb temale nimeks Immaanuel.
\par 15 Ta sööb võid ja mett, kuni ta mõistab põlata halba ja valida head.
\par 16 Sest enne kui poeglaps mõistab põlata halba ja valida head, laastatakse see maa, mille kahe kuninga ees sa tunned hirmu.
\par 17 Issand laseb tulla sinule ja su rahvale ning su isa soole Assuri kuninga läbi päevi, milliseid ei ole olnud Efraimi Juudast lahkumise ajast.
\par 18 Sel päeval vilistab Issand kärbseid Egiptuse jõgede lähteilt ja mesilasi Assuri maalt,
\par 19 ja need kõik tulevad ning laskuvad järskudesse orgudesse ja kaljulõhedesse, kõigisse kibuvitsapõõsastesse ja kõigisse jootmispaikadesse.
\par 20 Sel päeval lõikab Issand noaga, mis on palgatud teiselt poolt jõge, Assuri kuninga kaudu - teie pea- ja häbemekarvad maha ning rebib habemedki ära.
\par 21 Sel päeval saab igaüks pidada ainult ühe noore lehma ja paar lammast,
\par 22 ja nende piimaanni rohkuse tõttu on süüa võid, sest võid ja mett söövad kõik, kes maale alles jäävad.
\par 23 Sel päeval jääb iga paik, kus oli tuhat viinapuud, tuhat hõbeseeklit väärt, kibuvitstele ja ohakaile.
\par 24 Sinna minnakse noolte ja ambudega, sest kogu maa on täis kibuvitsu ja ohakaid.
\par 25 Ja mitte ühelegi mäele, mida nüüd kõplaga rohitakse, ei minda enam kibuvitste ja ohakate kartuse pärast: need jäävad härgade uitepaigaks ja lammaste tallermaaks.”

\chapter{8}

\par 1 Ja Issand ütles mulle: „Võta enesele suur tahvel ja kirjuta selle peale tavalise krihvliga: Maher-Saalal, Haas-Bas!
\par 2 Ja ma võtan endale usaldusväärsed tunnistajad, preester Uurija ja Jeberekja poja Sakarja.”
\par 3 Siis ma läksin prohveti naise juurde, ja tema jäi lapseootele ning tõi poja ilmale. Ja Issand ütles mulle: „Pane temale nimeks Maher-Saalal, Haas-Bas!
\par 4 Sest enne kui poiss oskab hüüda isa ja ema, viiakse Damaskuse varandus ja Samaaria saak Assuri kuninga ette.”
\par 5 Ja Issand rääkis minuga veelgi, öeldes:
\par 6 „Sellepärast et see rahvas põlgab Siiloahi hiljukesi voolavaid vesi ning pelgab Retsinit ja Remalja poega,
\par 7 vaata, sellepärast lasebki Issand tulla nende peale Frati jõe võimsad ja suured veed - Assuri kuninga ja kogu ta hiilguse -, mis tõuseb välja kõigist lammidest ja voolab üle kõigist kallastest,
\par 8 tungib Juudasse, tulvab ja ujutab üle, kuni see ulatub kaelani. Ja tema tiibade sirutus täidab su maa kogu selle laiuses, Immaanuel.”
\par 9 Möllake, rahvad, ja värisege, ja kuulge, kõik kauged maad: Varustuge sõjaks ja värisege! Varustuge sõjaks ja värisege!
\par 10 Pidage nõu, aga see läheb tühja, kõnelge sõna, aga see ei täitu, sest Jumal on meiega!
\par 11 Sest nõnda rääkis minuga Issand, kui ta käsi haaras mind ja hoiatas mind käimast selle rahva teed, öeldes:
\par 12 „Ärge nimetage vandenõuks kõike, mida see rahvas nimetab vandenõuks, ärge kartke, mida tema kardab, ja ärge tundke hirmu!
\par 13 Pidage pühaks vägede Issandat, tema olgu teie kartus ja tema olgu teie hirm!
\par 14 Tema on pühamuks, aga ka komistuskiviks ja pahanduskaljuks mõlemale Iisraeli kojale, lõksuks ja püüdepaelaks Jeruusalemma elanikele.
\par 15 Paljud neist komistavad, kukuvad ja vigastavad endid, püütakse kinni ja võetakse vangi.
\par 16 Seo kinni tunnistus, sulge pitseriga õpetus mu õpilastes!”
\par 17 Mina aga ootan Issandat, kes peidab oma palge Jaakobi soo eest, ja loodan tema peale.
\par 18 Vaata, mina ja lapsed, keda Issand mulle on andnud, oleme tunnustähtedeks ja endemärkideks Iisraelis vägede Issandalt, kes elab Siioni mäel.
\par 19 Ja kui teile öeldakse: Küsitlege surnute vaime ja ennustajaid, kes sosistavad ja pomisevad, siis vastake: Kas rahvas ei peaks küsitlema oma Jumalat? Kas tuleb elavate pärast küsitleda surnuid?
\par 20 Õpetuse ja tunnistuse juurde! Kui nõnda ei kõnelda, siis ei ole koitu.
\par 21 Seal nad siis uitavad, vaevatud ja näljased. Aga nälga tundes nad vihastavad ja neavad oma kuningat ja Jumalat. Ja kui nad tõstavad pilgu üles
\par 22 või vaatavad maha, siis - näe, on kitsikus ja pimedus, ahistuse süngus. Ollakse tõugatud pilkasesse pimedusse.

\chapter{9}

\par 1 Rahvas, kes käib pimeduses, näeb suurt valgust; kes elavad surmavarju maal, neile paistab valgus.
\par 2 Sina teed rohkeks rahva, ja valmistad talle suure rõõmu: nad rõõmustavad sinu ees otsekui lõikusajal ollakse rõõmsad, otsekui saaki jagades hõisatakse.
\par 3 Sest tema koorma ikke ja tema õla kepi, tema sundija vitsa sa murrad nagu Midjani päevil.
\par 4 Sest iga sõjakäras sõtkutud saabas ja veres vettinud vammus põletatakse tuleroaks.
\par 5 Sest meile sünnib laps, meile antakse poeg, kelle õlgadel on valitsus ja kellele pannakse nimeks Imeline Nõuandja, Vägev Jumal, Igavene Isa, Rahuvürst.
\par 6 Suur on valitsus ja otsatu on rahu Taaveti aujärjel ja tema kuningriigi üle, et seda kinnitada ja toetada kohtu ja õigusega, sellest ajast ja igavesti. Vägede Issanda püha viha teeb seda.
\par 7 Issand läkitab sõna Jaakobisse ja see langeb Iisraeli peale.
\par 8 Seda saab tunda kogu rahvas, Efraim ja Samaaria elanikud, kes ütlevad upsakusest ja südame ülbusest:
\par 9 „Telliskivid on varisenud, aga meie ehitame tahutud kividest; metsviigipuud on maha raiutud, aga me paneme seedrid asemele.”
\par 10 Ent Issand teeb tugevaks Retsini, ta vastase, ja kihutab üles ta vaenlased:
\par 11 süürlased idast ja vilistid läänest, ja need söövad Iisraeli täie suuga. Selle kõige juures ei ole ta viha veel ära pöördunud ja ta käsi on alles välja sirutatud.
\par 12 Sest rahvas ei ole pöördunud oma peksja poole ega ole nad otsinud vägede Issandat.
\par 13 Seepärast lõikab Issand Iisraelilt ühel päeval pea ja saba, võrse ja varre.
\par 14 Vana ja austatu on pea, aga valet õpetav prohvet on saba.
\par 15 Selle rahva juhid on eksitajad ja nende poolt juhitavad on viidud segadusse.
\par 16 Seepärast ei tunne Issand rõõmu ta noortest meestest ega halasta vaeslaste ja lesknaiste peale; sest kõik on jumalavallatud ja kurjategijad ja iga suu räägib jõledust. Selle kõige juures ei ole ta viha veel ära pöördunud ja ta käsi on alles välja sirutatud.
\par 17 Sest ülekohus põleb otsekui tuli, mis põletab kibuvitsu ja ohakaid ning süütab metsarägastikud kerkiva suitsuna haihtuma.
\par 18 Vägede Issanda raevust on maa kõrbenud ja rahvas on otsekui tuleroog: üks ei halasta teise peale.
\par 19 Ahmivad paremalt, aga on näljased, söövad vasakult, ometi ei saa kõht täis. Igaüks sööb oma käsivarre liha:
\par 20 Manasse Efraimi ja Efraim Manasset, mõlemad on üheskoos Juuda vastu. Selle kõige juures ei ole ta viha veel ära pöördunud ja ta käsi on alles välja sirutatud.

\chapter{10}

\par 1 Häda neile, kes annavad nurjatuid seadusi, ja kirjutajaile, kes kirjutavad vääri otsuseid,
\par 2 et väänata abitute õigust ja kiskuda kohus mu rahva viletsailt, et lesknaised saaksid neile saagiks ja et nad võiksid vaeslapsi paljaks riisuda!
\par 3 Aga mida te teete katsumispäeval ja tormis, mis tuleb kaugelt? Kelle juurde te põgenete abi saama? Ja kuhu jätate oma varanduse?
\par 4 Ei ole muud kui põlvitada vangide seltsi või langeda tapetute sekka. Selle kõige juures ei ole ta viha veel ära pöördunud ja ta käsi on alles välja sirutatud.
\par 5 Häda Assurile, kes on mu viha vits ja kelle käes on mu sajatuse kepp!
\par 6 Tema ma läkitan jumalavallatu rahva vastu ja teda ma käsin oma vihaaluse rahva pärast võtta noosi ja riisuda saaki ja tallata teda nagu tänavapori.
\par 7 Aga tema ei kavatse nõnda ja nõnda ei mõtle ta süda, vaid temal on südames hävitada ja hukata rahvaid, ja mitte väheseid,
\par 8 sest ta ütleb: „Kas pole mu pealikud kõik kuningad?
\par 9 Eks Kalnol käinud käsi nagu Karkemisel, eks Hamatil nagu Arpadil, eks Samaarial nagu Damaskusel?
\par 10 Kuna mu käsi ulatus ebajumalate kuningriikideni, kus nikerdatud kujusid oli rohkem kui Jeruusalemmal ja Samaarial,
\par 11 eks ma või siis talitada ka Jeruusalemma ja selle kujudega nõnda, nagu ma talitasin Samaaria ja selle ebajumalatega?”
\par 12 Aga kui Issand on lõpetanud kõik oma töö Siioni mäel ja Jeruusalemmas, siis ta nuhtleb Assuri kuningat ta südame ülbuse vilja ja ta suureliste silmade kõrkuse pärast.
\par 13 Sest too ütleb: „Ma tegin seda oma käe rammuga ja oma tarkusest, sest ma olen taibukas. Ma nihutasin rahvaste piire, riisusin, mis kuulus neile. Oma vägevuses tõukasin ma valitsejaid troonilt.
\par 14 Mu käsi leidis rahvaste vara otsekui linnupesa; ja nagu korjatakse mahajäetud mune, nõnda ma korjasin ära kõik maad, ja ei olnud kedagi, kes oleks lehvitanud tiibu või teinud noka lahti ja piiksatanud.”
\par 15 Kas kiitleb kirves selle ees, kes temaga raiub, või suurustab saag saagija ees? Nagu viibutaks vits oma ülestõstjat, või nagu kergitaks kepp seda, kes ei ole puu.
\par 16 Sellepärast läkitab Jumal, vägede Issand, tema priskete sekka kõhetustõve, ja tema toreduse all lahvatab tuli.
\par 17 Iisraeli valgus muutub tuleks ja tema Püha leegiks, mis ühel päeval põletab ja neelab ta ohakad ja kibuvitsad.
\par 18 Ja ta lõpetab tema metsa ja viljapuuaia toreduse koos ihu ja hingega. See on otsekui põdeja kidunemine.
\par 19 Vähe jääb siis üle ta metsapuudest: poisike võib need kirja panna.
\par 20 Ja sel päeval sünnib, et Iisraeli jääk ja Jaakobi soo pääsenu ei toetu enam oma peksjale, vaid toetub ustavalt Issandale, Iisraeli Pühale.
\par 21 Jääk pöördub, Jaakobi Jääk, vägeva Jumala poole.
\par 22 Sest kuigi su rahvast, Iisrael, oleks otsekui mereliiva, pöördub sellest ainult jääk. Hävitus on otsustatud, õiglus ujutab üle.
\par 23 Sest hävituse ja otsuse teeb Jumal, vägede Issand, teoks kogu maal.
\par 24 Seepärast ütleb Jumal, vägede Issand, nõnda: Mu rahvas, kes sa elad Siionis, ära karda Assurit, kui ta peksab vitsaga ja Egiptuse kombel tõstab su kohale oma kepi!
\par 25 Sest veel pisut-pisut, siis on mu sajatusel lõpp ja mu viha pöördub neid hävitama.
\par 26 Siis keerutab vägede Issand ta kohal piitsa otsekui Midjani löömisel Oorebi kaljul, ja oma kepi, mis kord oli mere kohal, tõstab ta üles nagu tol korral Egiptuses.
\par 27 Ja sel päeval sünnib, et tema koorem võetakse su õlgadelt ja tema ike su kaela pealt. Ta tuleb üles Samaariast,
\par 28 tungib Ajjatisse, läbib Migroni, Mikmassi ta jätab oma varustuse.
\par 29 Nad läbivad kuru: „Gebas on meie öömaja!” Raama väriseb, Sauli Gibea põgeneb.
\par 30 Hüüa kilavalt, Gallimi tütar! Kuulata, Laisa! Vasta, Anatot!
\par 31 Madmena põgeneb, Geebimi elanikud poevad peitu.
\par 32 Juba täna peatub ta Noobis, viibutab kätt Siioni tütre mäe, Jeruusalemma künka poole.
\par 33 Vaata, Jumal, vägede Issand, laasib oksad vägeval jõul. Kõrgeks kasvanud raiutakse maha: kõrged peavad saama madalaks!
\par 34 Metsarägastik laastatakse raudkirvega ja Liibanon langeb vägeva läbi.

\chapter{11}

\par 1 Aga Iisai kännust tõuseb võrse ja võsu tema juurtest kannab vilja.
\par 2 Ja tema peal hingab Issanda Vaim, tarkuse ja arukuse Vaim, nõu ja väe Vaim, Issanda tundmise ja kartuse Vaim.
\par 3 Tema õndsus on Issanda kartuses. Ei ta mõista kohut oma silma nägemise ega otsusta oma kõrva kuulmise järgi,
\par 4 vaid ta mõistab viletsaile kohut õiguses ja otsustab hädaliste asju maa peal õigluses; ta lööb oma suu vitsaga maad ja surmab oma huulte puhanguga õela.
\par 5 Õigus on kui vöö tema niudeil ja ustavus kui rihm ta puusadel.
\par 6 Siis elab hunt tallega üheskoos ja panter lesib kitsekese kõrval; vasikas, noor lõvi ja nuumveis on üheskoos ning pisike poiss ajab neid.
\par 7 Lehm ja karu käivad karjamaal, nende pojad lesivad üheskoos, ja lõvi sööb õlgi nagu veis.
\par 8 Imik mängib rästiku uru juures ja võõrutatu sirutab käe mürkmao koopasse.
\par 9 Ei tehta paha ega kahju kogu mu pühal mäel, sest maa on täis Issanda tundmist - otsekui veed katavad merepõhja.
\par 10 Sel päeval sünnib, et Iisai juurt, kes seisab rahvaile lipuks, otsivad paganad ja tema asupaik saab auliseks.
\par 11 Ja sel päeval sirutab Issand veel teist korda käe, et lunastada oma rahva jääki, kes on järele jäänud Assurist ja Egiptusest, Patrosest, Etioopiast, Eelamist, Sinearist, Hamatist ja mere saartelt.
\par 12 Ta tõstab rahvastele lipu, kogub Iisraeli hajutatud ja korjab kokku Juuda pillutatud neljast maailmakaarest.
\par 13 Siis lakkab Efraimi kadestamine ja hävitatakse Juuda rõhujad. Efraim ei kadesta Juudat ja Juuda ei rõhu Efraimi.
\par 14 Nad tormavad siis vilistite maale läände, nad riisuvad üheskoos idamaa poegi. Edom ja Moab on neile käealuseks, ammonlased on nende alamad.
\par 15 Ja Issand lõhestab Egiptuse merelahe ning viipab käega, oma tugeva tuulega, üle Frati jõe ja lööb selle seitsmeks ojaks, et kuivijalu saab läbi minna.
\par 16 Siis on maantee tema rahva jäägile, kes Assurist alles jääb, otsekui Iisraelil oli ta Egiptusemaalt tuleku päeval.

\chapter{12}

\par 1 Sel päeval sa ütled: „Ma tänan sind, Issand! Sa olid küll vihane minule, aga su viha pöördus ja sa trööstisid mind.
\par 2 Vaata, Jumal on mu pääste! Ma olen julge ega karda, sest Issand Jumal on mu tugevus ja kiituslaul, ja tema oli mu päästja.”
\par 3 Te ammutate rõõmuga vett päästeallikaist.
\par 4 Ja sel päeval te ütlete: „Tänage Issandat, kuulutage tema nime, tehke teatavaks rahvaste seas tema teod, tunnistage, et tema nimi on kõrge!
\par 5 Ülistage Issandat, sest tema on teinud suuri asju, saagu see teatavaks kogu maal!
\par 6 Hõiska ja rõkata rõõmust, Siioni elanik, sest Iisraeli Püha on teie keskel suur!”

\chapter{13}

\par 1 Ennustus Paabeli kohta, Jesaja, Aamotsi poja poolt nähtud:
\par 2 Tõstke lipp lagedale mäele, hüüdke neile valjusti, viibake käega, et nad läheksid sisse suurte isandate väravaist!
\par 3 Ma olen andnud käsu oma pühitsetuile, olen juba oma vihale appi kutsunud sangarid, kes võivad suurustada minu suurusest.
\par 4 Kuule! Mägedes on mürin, rahvahulkade kõmin. Kuule! Kuningriikide, kogunenud paganate lärm. Vägede Issand vaatab üle sõjaväge.
\par 5 Nad tulevad kaugelt maalt, taevarannalt, Issand ja tema sajatuse relvad, hävitama kogu maad.
\par 6 Ulguge, sest Issanda päev on ligidal, see tuleb kui hävitus Kõigeväeliselt!
\par 7 Seepärast lõtvuvad kõik käed ja kõik inimsüdamed löövad araks.
\par 8 Nad kohkuvad. Vaevused ja valud haaravad neid, nad vaevlevad otsekui sünnitaja. Üks vaatab teisele jahmunult otsa, nende palged on tulipunased.
\par 9 Vaata, Issanda päev tuleb hirmsana, raevu ja tulise vihaga, et tühjendada maa ja hävitada sealt patused.
\par 10 Sest taevatähed, nende hulgas Vardatähed, ei kiirga enam valgust, päike on tõustes pime ja kuu ei anna oma valget.
\par 11 Mina tasun ilmamaale ta kurjuse ja õelatele nende süü, mina lõpetan ülbete kõrkuse ja alandan võimutsejate upsakuse.
\par 12 Mina teen inimesed kallimaks puhtast kullast, inimlapse kallimaks Oofiri kullast.
\par 13 Seepärast ma panen taevad vankuma ja maa vappub paigast vägede Issanda raevus tema tulise viha päeval.
\par 14 Ja otsekui peletatud gasell või kari, keda keegi ei aja kokku, pöördub igaüks oma rahva juurde ja põgeneb igaüks oma maale.
\par 15 Kõik, keda leitakse, torgatakse läbi, ja iga põgenik langeb mõõga ees.
\par 16 Nende lapsed rebitakse tükkideks nende eneste nähes, nende kojad rüüstatakse, nende naised teotatakse.
\par 17 Vaata, ma kihutan nende vastu meedlased, kes ei hooli hõbedast ega himusta kulda.
\par 18 Nende ammud purustavad poisse, neil ei ole halastust ihuvilja vastu, laste pärast nende silm ei kurvasta.
\par 19 Ja Paabel, kuningriikide ehe, kaldealaste kõrk toredus, saab Soodoma ja Gomorra sarnaseks, mis Jumal segi paiskas.
\par 20 Seda ei asustata enam iialgi ega elata seal põlvest põlve; ei löö sinna araablane telki üles ega puhka seal karjased karja.
\par 21 Vaid seal lebavad kurjad vaimud ja nende kojad on täis öökulle; seal asuvad jaanalinnud ja karglevad sikujalgsed paharetid.
\par 22 Ta tornides uluvad kurjad vaimud ja lusthooneis ðaakalid. Tema aeg on liginemas ja tema päevi ei pikendata.

\chapter{14}

\par 1 Issand halastab Jaakobi peale, valib taas Iisraeli ja asustab nad nende oma maale; võõradki seltsivad nendega ja liituvad Jaakobi sooga.
\par 2 Ja rahvad võtavad nad ning toovad nad nende pärispaika; aga Iisraeli sugu teeb nad Issanda pinnal enesele sulaseiks ja teenijaiks: nad võtavad vangi oma vangistajad ja valitsevad oma rõhujate üle.
\par 3 Ja sel päeval, kui Issand annab sulle rahu vaevast ja ahistusest ning raskest tööst, mida sa pidid tegema,
\par 4 hakkad sa laulma seda pilkelaulu Paabeli kuningast ja ütled: Kuidas on siis nüüd kallaletung lakanud, rõhumisele lõpp tulnud?
\par 5 Issand on katki murdnud õelate kepi, võimutsejate vitsa,
\par 6 mis lõi rahvaid raevus lõppemata löökidega, mis rahvusi valitses vihas lakkamatu jälitusega.
\par 7 Kogu maa tõmbab hinge, puhkab, nad rõkatavad rõõmust.
\par 8 Sinust tunnevad rõõmu ka küpressid ja Liibanoni seedrid: „Sest ajast, kui sa lamad maas, ei tule keegi meid raiuma!”
\par 9 Surmavald all on liikvel sinu pärast, et võtta vastu su tulekut: ta äratab sinu pärast surnute vaimud, kõik maa juhid, laseb nende aujärgedelt tõusta kõik rahvaste kuningad.
\par 10 Need kõik kostavad ja ütlevad sulle: „Sinagi oled jäänud jõuetuks nagu meie, oled saanud meie sarnaseks!”
\par 11 Alla surmavalda on tõugatud su kõrkus, su naablite helinad; su Alla on laotatud ussikesi ja sind katavad vaglad.
\par 12 Kuidas sa ometi oled alla langenud taevast, helkjas hommikutäht, koidiku poeg, tükkidena paisatud maha, rahvaste alistaja?
\par 13 Sina ütlesid oma südames: „Mina tõusen taevasse, kõrgemale kui Jumala tähed tõstan ma oma aujärje ja istun kogunemismäele kaugel põhjamaal.
\par 14 Ma lähen üles pilvede kõrgustikele, ma teen ennast Kõigekõrgema sarnaseks.”
\par 15 Aga sind tõugati alla surmavalda, kõige sügavamasse hauda.
\par 16 Kes sind nägid, silmitsevad sind, vaatavad sind üksisilmi: „Kas see on mees, kes pani maa värisema, kuningriigid vabisema,
\par 17 kes tegi ilmamaa kõrbe sarnaseks ja kiskus maha selle linnad, kes ei lasknud oma vange koju?”
\par 18 Kõik rahvaste kuningad magavad auga, igaüks oma hauakambris,
\par 19 sina aga oled nagu põlatud oks, eemale heidetud oma hauast, kaetud tapetutega, mõõgaga läbipistetutega, kes paisatakse alla kivimurdu purukstallatud laibana.
\par 20 Sina ei ühine nendega hauas, sest sa oled hävitanud oma maa, tapnud oma rahva. Iialgi enam ei nimetata kurjategijate sugu.
\par 21 Valmistage tema laste jaoks tapapink nende vanemate süü pärast, et nad ei tõuseks ega vallutaks maad ega täidaks ilmamaad linnadega.
\par 22 Mina tõusen nende vastu, ütleb vägede Issand, ja ma kaotan Paabeli nime ja jäägi, järglased ja järelkasvu, ütleb Issand.
\par 23 Ma teen ta siilide omandiks ja roolaugasteks, ja ma pühin ta ära hävitusluuaga, ütleb vägede Issand.
\par 24 Vägede Issand on vandunud, öeldes: Tõesti, nagu ma olen mõelnud, nõnda sünnib, ja see, mida ma olen kavatsenud, läheb korda:
\par 25 ma murran Assuri oma maal ja tallan teda oma mägedel; siis on nad tema ikkest lahti ja tema koorem võetakse nende õlgadelt.
\par 26 See on nõu, peetud kogu maa kohta, ja see on käsi, sirutatud kõigi rahvaste kohale.
\par 27 Sest vägede Issand on võtnud nõuks, kes teeb selle tühjaks? Tema käsi on välja sirutatud, kes pöörab selle tagasi?
\par 28 Kuningas Aahase surma-aastal sündis see ennustus:
\par 29 Ära rõõmusta, sina Vilistimaa, et murtud on vits, mis sind lõi! Sest mao juure seest tuleb välja rästik ja tema vili on lendav madu.
\par 30 Siis karjatatakse viletsate esmikuid ja vaesed magavad muretult; aga sinu juure ma suretan näljaga ja su jääk tapetakse.
\par 31 Ulu, värav, kisenda, linn! Vabise, kogu Vilistimaa, sest põhja poolt tuleb suits ja keegi tema väehulkades ei jää teistest maha.
\par 32 Ja mida vastata paganate saadikuile? Seda, et Issand on rajanud Siioni ja seal leiavad varju tema rahva vaevatud.

\chapter{15}

\par 1 Ennustus Moabi kohta: Tõesti, ühel ööl on Moabi Aar hävitatud, vaikima pandud. Tõesti, ühel ööl on Moabi Kiir hävitatud, vaikima pandud.
\par 2 Ka Diibon läheb üles templisse, ohvriküngastele nutma, Nebol ja Meedebal ulub Moab. Kõigil on pead pöetud ja kõik habemed on ära aetud.
\par 3 Tänavail pannakse kotiriie selga, katustel ja turgudel uluvad kõik, upuvad pisaraisse.
\par 4 Hesbon ja Elaale kisendavad appi, Jahaseni kostab nende hääl; seepärast lõdisevad Moabi niuded, väriseb ta hing.
\par 5 Mu süda kisendab Moabi pärast. Ta põgenikud põgenevad Soarini, Eglat-Selisijani. Jah, mööda Luuhiti tõusuteed minnakse nuttes üles, jah, Hooronaimi teel tõstetakse hädakisa hävingu pärast.
\par 6 Sest Nimrimi veed muutuvad kõrbeks; jah, rohi kuivab, taimkate kaob, haljust ei ole enam.
\par 7 Seepärast nad viivad oma kogutud vara ja nende poolt tallelepandu üle Remmelgajõe.
\par 8 Jah, appihüüd kajab Moabimaal; ta hädakisa kostab Eglaimini, ta hädakisa kostab Beer-Eelimini.
\par 9 Jah, Diimoni veed on täis verd. Aga veel muudki ma saadan Diimonile: lõvi Moabi pääsenute ja maale jäänute kallale.

\chapter{16}

\par 1 Lähetage maavalitseja jäärad Selast kõrbe kaudu Siioni tütre mäele!
\par 2 Nagu põgenevad linnud, peletatud pesakond, on Moabi tütred Arnoni koolmeil.
\par 3 Anna nõu, tee otsus! Las olla su vari otsekui öö keskpäeval. Varja väljaaetuid, ära reeda põgenikku!
\par 4 Anna enese juures asu Moabi pillutatuile, ole neile pelgupaigaks hävitaja ees! Kui rõhujal on lõpp, hävitus lakkab, tallajad kaovad maalt,
\par 5 siis rajatakse armastuses aujärg ja sellel istub Taaveti telgis kindlalt kohtumõistja, kes nõuab õigust ja on osav õigluses.
\par 6 Me oleme kuulnud Moabi kõrkusest - ta on väga ülbe; tema upsakusest ja uhkustamisest ning tema tühist hooplemist.
\par 7 Seepärast ulub moab moabi pärast, kõik uluvad. Te leinate Kiir-Hareseti rosinakakkude pärast täiesti lööduina.
\par 8 Sest rüüstatud on Hesbon, kolletanud Sibma viinapuud, rahvaste isandad on peksnud maha nende vääriskobarad. Need ulatusid Jaaserini, laiusid kõrbes, nende väädid lokkasid, tungisid üle mere.
\par 9 Seepärast ma nutan koos Jaaseri nutuga Sibma viinapuude pärast; ma kastan sind oma silmaveega, Hesbon ja Elaale. Sest sinu suve ja lõikuse kohal on kõlanud vaenuhüüd.
\par 10 Rõõm ja ilutsemine on võetud viljapuuaiast ja viinamägedes ei ole hõiskamist ega rõõmuhüüdeid. Veini surutõrtes ei sõtku tallaja, tööhõiked ma lõpetasin.
\par 11 Seetõttu kaebleb mu hing otsekui kannel Moabi pärast, ja mu süda Kiir-Hareseti pärast.
\par 12 Ja kui nüüd Moab ennast näitab, kui ta ennast ohvrikünkal väsitab ja läheb oma pühamusse palvetama, siis ta ei suuda midagi.
\par 13 See on sõna, mis Issand Moabi kohta varem oli öelnud.
\par 14 Aga nüüd kõneleb Issand ning ütleb: „Kolme aasta pärast, palgalise aastate sarnaselt, muutub Moabi hiilgus koos kogu tema suure hulgaga väärtusetuks, ja jääk on pisike, tähtsusetu ja jõuetu.”

\chapter{17}

\par 1 Ennustus Damaskuse kohta: Vaata, Damaskus lakkab olemast linn ja muutub varemeks.
\par 2 Aroeri linnad jäetakse maha, jäävad karjadele, kes seal lebavad, ilma et ükski neid peletaks.
\par 3 Efraimist kaob kindel linn ja Damaskusest kuningriik; ja Süüria jäägiga sünnib nagu Iisraeli laste hiilgusegagi, ütleb vägede Issand.
\par 4 Sel päeval jääb väikseks Jaakobi hiilgus ja lahjub tema ihu lihavus,
\par 5 otsekui lõikaja lõikaks kasvavat vilja ja tema käsivars koristaks viljapäid, otsekui nopitaks viljapäid Refaimi orus.
\par 6 Jääb ainult järelkoristus, otsekui õlipuud raputades: kaks, kolm marja ülal ladvas, neli, viis viljapuu okstel, ütleb Issand, Iisraeli Jumal.
\par 7 Sel päeval vaatab inimene oma Looja peale ja ta silmad näevad Iisraeli Püha.
\par 8 Ta ei vaata altareile, oma kätetööle, ega näe seda, mis ta sõrmed on teinud, ei viljakustulpi ega suitsutusaltareid.
\par 9 Sel päeval on tema tugevad linnad otsekui mahajäetud metsad ja mäetipud, mis Iisraeli laste pärast maha jäeti, ja tühjus võtab maad.
\par 10 Sest sa unustasid oma Pääste-Jumala ega tuletanud meelde oma kindlat kaljut. Seepärast sa siis istutad Adonise aedu ja juurutad võõraid pistikuid.
\par 11 Istutamispäeval sa paned nad kasvama ja viid hommikuks oma külvi õitsema, kuid lõikus kaob tõve ja vaigistamatu valu päeval.
\par 12 Häda! Paljude rahvaste möll. Need mühavad, otsekui mühaks meri. Rahvahõimude märatsus. Need kohavad, otsekui kohaksid võimsad veed.
\par 13 Rahvahõimud kohavad, otsekui kohaksid suured veed. Aga tema sõitleb neid ja nad põgenevad kaugele ning neid aetakse taga, otsekui tuul puhub mägedel aganaid, otsekui torm keerutab tolmu.
\par 14 Õhtul, vaata, on kabuhirm, hommiku tulles ei ole neid enam. See on meie rüüstajate osa ja meie riisujate liisk.

\chapter{18}

\par 1 Häda tiivuliste ritsikate maale, mis on teisel pool Etioopia jõgesid,
\par 2 kes läkitab saadikuid merele, pilliroopaatides üle vee. Minge, kiired käskjalad, pikakasvulise ja siledanahkse rahva juurde, rahva juurde, keda kardetakse ligidal ja kaugel, rahvas, kes on vägev ja vallutaja, kelle maad jõed uhuvad.
\par 3 Kõik ilmamaa elanikud ja maa asustajad, vaadake, millal lipp tõstetakse mägedele, ja kuulge, kui puhutakse sarve.
\par 4 Sest Issand on mulle öelnud nõnda: „Mina olen vagusi ja vaatlen oma asupaigast otsekui kuumavirvendus päikesepaistel, otsekui kastepilv lõikusaja palavuses.”
\par 5 Sest enne lõikust, kui õitsemine on lõppenud ja õied on valminud küpsevaiks marjadeks, lõigatakse sirpidega väädid ja kõrvaldatakse kitkudes lokkavad kasvud.
\par 6 Need jäetakse üheskoos maha mägede röövlindudele ja maa loomadele; suviti on nende peal röövlinnud ja talviti kõik maa loomad.
\par 7 Sel ajal tuuakse ande vägede Issandale pikakasvulise ja siledanahkse rahva poolt, rahva poolt, keda kardetakse ligidal ja kaugel, rahvas, kes on vägev ja vallutaja, kelle maad jõed uhuvad - vägede Issanda nime asupaika Siioni mäel.

\chapter{19}

\par 1 Ennustus Egiptuse kohta: Vaata, Issand sõidab nobedal pilvel ja tuleb Egiptusesse. Siis vabisevad tema ees Egiptuse ebajumalad ja egiptlaste süda lööb araks.
\par 2 Mina kihutan egiptlased egiptlaste vastu ja nad sõdivad isekeskis, sõber sõbra vastu, linn linna vastu, kuningriik kuningriigi vastu.
\par 3 Egiptuse vaim satub segadusse ja ma eksitan tema nõu; siis nad küsitlevad ebajumalaid ja lausujaid, surnute vaime ja ennustajaid.
\par 4 Ma annan egiptlased karmi isanda kätte ja nende üle hakkab valitsema kõvakäeline kuningas, ütleb Jumal, vägede Issand.
\par 5 Veed lõpevad Niilusest, jõgi taheneb ja kuivab.
\par 6 Jõed hakkavad haisema, kanalid kahanevad ja jäävad kuivaks, pilliroog ja kõrkjad närbuvad.
\par 7 Kaislad Niiluse kaldal, Niiluse suudmes, ja kõik Niiluse külvimaad kuivavad, puhutakse laiali ja neid ei ole enam.
\par 8 Kalamehed kurvastavad ja leinavad; kõik, kes heidavad õnge jõkke ja lasevad võrke vette, on rammetud.
\par 9 Häbisse jäävad, kes lõugutavad linu, ja kudujad kahvatavad.
\par 10 Ta kangrud lüüakse maha, kõigi palgatööliste meel on nukker.
\par 11 Tõesti, Soani vürstid on meeletud, vaarao targad nõuandjad annavad rumalat nõu. Kuidas te võite öelda vaaraole: „Mina olen tarkade poeg, muistsete kuningate järeltulija”?
\par 12 Kus su targad siis on? Las nad nüüd jutustavad sulle, kui nad teavad, mida vägede Issand on otsustanud Egiptuse kohta.
\par 13 Soani vürstid on muutunud juhmiks, Noofi vürste on narritud ja Egiptuse on pannud taaruma tema enese suguharude juhid.
\par 14 Issand on seganud tema sekka pöörituse vaimu ja nad panevad Egiptuse taaruma kõigis ta tegudes, otsekui joobnu, kes taarub oma okses.
\par 15 Ei lähe korda Egiptuse töö, tehku seda pea või saba, võrse või vars.
\par 16 Sel päeval on egiptlased otsekui naised ning nad vabisevad ja värisevad vägede Issanda viibutava käe ees, mida ta viibutab nende kohal.
\par 17 Ja Juudamaa saab Egiptusele häbistuseks; iga kord, kui temale seda meelde tuletatakse, väriseb ta vägede Issanda nõu ees, mida too on pidanud tema vastu.
\par 18 Sel päeval on Egiptusemaal viis linna, kes räägivad kaanani keelt ja annavad vande vägede Issandale; ühte hüütakse Päikeselinnaks.
\par 19 Sel päeval on Issanda altar keset Egiptusemaad ja selle piiri ääres on Issanda sammas,
\par 20 märgiks ja tunnistuseks vägede Issandale Egiptusemaal. Kui nad rõhujate pärast kisendavad Issanda poole, siis läkitab tema neile päästja, kes nende eest võitleb ja nad vabastab.
\par 21 Ja Issand ilmutab ennast egiptlastele ja egiptlased õpivad sel päeval Issandat tundma ja teenivad teda tapa- ja roaohvritega, ja nad annavad Issandale tõotusi ning viivad need täide.
\par 22 Ja Issand lööb egiptlasi, lööb ja ravib; nad pöörduvad siis Issanda poole ja tema kuuleb nende palveid ning teeb nad terveks.
\par 23 Sel päeval on maantee Egiptusest Assurisse: assüürlased tulevad Egiptusesse ja egiptlased Assurisse, ja egiptlased peavad jumalateenistust koos assüürlastega.
\par 24 Sel päeval on Iisrael kolmandana Egiptuse ja Assuri kõrval, õnnistus on keset maad,
\par 25 mida vägede Issand õnnistab, öeldes: „Õnnistatud olgu Egiptus, mu rahvas, Assur, mu kätetöö, ja Iisrael, mu pärisosa!”

\chapter{20}

\par 1 Sel aastal kui väejuht, kelle Assuri kuningas Sargon läkitas, tuli Asdodi ja taples Asdodi vastu ning vallutas selle -
\par 2 sel ajal kõneles Issand Jesaja, Aamotsi poja läbi, öeldes: „Mine päästa kotiriie oma niudeilt valla ja võta jalatsid jalast!” Ja ta tegi nõnda: käis alasti ja paljajalu.
\par 3 Siis ütles Issand: „Otsekui mu sulane Jesaja on kolm aastat käinud alasti ja paljajalu, märgiks ja endeks Egiptusele ja Etioopiale,
\par 4 nõnda ajab Assuri kuningas Egiptuse vange ja Etioopia asumiselesaadetuid, noori ja vanu, alasti ja paljajalu, istmik paljas, egiptlastele häbiks.
\par 5 Siis nad ehmuvad ja häbenevad Etioopia pärast, kes oli nende lootus, ja Egiptuse pärast, kes oli nende uhkus.
\par 6 Ja sel päeval ütlevad selle ranna elanikud: „Vaata, nõnda läks meie lootusega, kuhu me põgenesime abi saama, et pääseda Assuri kuninga eest. Kuidas me nüüd pääseme?”

\chapter{21}

\par 1 Ennustus mereäärse kõrbe kohta: Otsekui Lõunamaal tormavad tuulehood, nõnda tuleb see kõrbest, kardetavalt maalt.
\par 2 Mulle on ilmutatud karm nägemus: petis petab ja rüüstaja rüüstab. Mine üles, Eelam! Asu piirama, Meedia! Mina lõpetan kõik ohkamised.
\par 3 Seepärast on mu niuded täis valu: mind on haaranud vaevused otsekui tuhud sünnitajat. Ma olen kuuldust segaduses, ma olen nähtust jahmunud.
\par 4 Mu süda väriseb, lõdisemine kohutab mind, puhteaeg, mida igatsesin, sai mulle ängiks.
\par 5 Lauda kaetakse, istejärjestust seatakse, süüakse, juuakse. Tõuske, vürstid, võidke kilpi!
\par 6 Sest Issand on mulle öelnud nõnda: „Mine, pane vahimees välja! Mida ta näeb, seda teatagu!
\par 7 Kui ta näeb karavani, ratsanikke paarikaupa, eeslite rodu, kaamelite rodu, siis ta pangu hästi tähele, pangu hoolsasti tähele!”
\par 8 Ja ta hüüdiski: „Nägija vahitornis! Issand, ma seisan päevad läbi ja olen valves igal ööl.
\par 9 Jah, vaata, sealt tuleb meeste rodu, ratsanikke paarikaupa.„ Siis ta kostis ning ütles: „Langenud, langenud on Paabel, ja kõik ta ebajumalate kujud on maha puruks löödud.”
\par 10 Mu rehena pekstud ja tallatud rahvas! Mida ma kuulsin vägede Issandalt, Iisraeli Jumalalt, seda ma kuulutan teile!
\par 11 Ennustus Duuma kohta: Mulle hüütakse Seirist: „Vahimees, ons palju veel ööd? Vahimees, ons palju veel ööd?”
\par 12 Vahimees vastab: „Hommik tuleb, aga öö tuleb ka! Kui tahate küsida, küsige, tulge uuesti tagasi!”
\par 13 Ennustus Araabia kohta: Ööbige Araabia tihnikuis, dedanlaste karavan!
\par 14 Minge vastu janustele, viige neile vett! Teemamaa elanikud, astuge oma leivaga põgenike ligi!
\par 15 Sest nad põgenevad mõõkade eest, paljastatud mõõga eest, vinnastatud ammu eest ja sõja raskuste eest.
\par 16 Sest Issand ütles mulle nõnda: „Veel aasta, palgalise aastate sarnaselt, ja kõik Keedari hiilgus on kadunud.
\par 17 Jääk ammumeeste arvust, Keedari kangelaste arvust jääb väheseks. Jah, Issand, Iisraeli Jumal, on rääkinud.”

\chapter{22}

\par 1 Ennustus Nägemuseoru kohta: Mis sul siis nüüd on, et oled tervenisti läinud katustele,
\par 2 sa kärarikas, rahutu linn, ülemeelik linnake? Su tapetuid ei tapeta mõõgaga ega sure nad sõjas.
\par 3 Kõik su juhid põgenevad üheskoos, võetakse vangi ammutagi. Kõik, kes su omadest leitakse, võetakse üheskoos vangi, kuigi põgenevad kaugele.
\par 4 Seepärast ma ütlen: Ärge vaadake mind, ma nutan kibedasti; ärge kippuge mind trööstima mu rahva tütre julma kohtlemise pärast!
\par 5 Sest Jumalal, vägede Issandal, on Nägemuseorus jahmatuse, tallamise ja segaduse päev: müüride murdmine ja hädakisa mägedeni.
\par 6 Eelam võtab nooletupe, sõjavankrite vooris on mehed ja ratsanikud, ja Kiir vabastab kilbi kattest.
\par 7 Su ilusad orud täituvad vankritega ja ratsanikud asuvad otse väravate vastas.
\par 8 Ta võtab ära Juuda kaitse ja sel päeval sa vaatad relvade poole Metsakojas.
\par 9 Te näete, et Taaveti linnas on palju müürimurde, ja te kogute Alatiiki vett.
\par 10 Te loete üle Jeruusalemma kojad ja kisute kodasid maha, et kindlustada müüri.
\par 11 Ja te teete müüride vahele mahuti Vanatiigi vee jaoks. Aga te ei vaata selle poole, kes seda teeb, ega näe, kes seda ammu valmistas.
\par 12 Jah, Jumal, vägede Issand, kutsub sel päeval nutma ja leinama, pead pügama ja kotiriiet ülle võtma.
\par 13 Aga vaata, siis on lust ja rõõm: härgade tapp ja lammaste veristus, süüakse liha ja juuakse veini. „Söögem ja joogem, sest homme me sureme!”
\par 14 Kuid vägede Issand on mulle kõrva ilmutanud: Tõesti, seda süüd ei lepitata teile surmani, ütleb Jumal, vägede Issand.
\par 15 Nõnda ütleb Jumal, vägede Issand: Tule, mine selle valitseja juurde, Sebna juurde, kes on kojaülemaks, ja ütle:
\par 16 „Mis sul siin asja on ja kes sul siin on, et sa siin raiud enesele hauda, raiud oma haua kõrgele, uuristad enesele elamu kaljusse?
\par 17 Vaata, Issand viskab sind kaares kaugele vägeva mehe viskega, ja haarates sinust kõvasti kinni,
\par 18 mässib ta sind koguni keraks ning virutab kui palli igati laiale maale: seal sa sured ja sinna jäävad su toredad vankrid, sina, oma isanda koja häbi.”
\par 19 Mina tõukan sinu su ametist, ja sind aetakse ära su kohalt.
\par 20 Ja sel päeval kutsun ma oma sulase Eljakimi, Hilkija poja,
\par 21 panen temale selga sinu kuue, kinnitan talle vööle sinu vöö ja annan tema kätte sinu valitsuse: tema saab siis isandaks Jeruusalemma elanikele ja Juuda soole.
\par 22 Ja ma panen Taaveti koja võtme temale õlale: tema avab ja ükski ei sule, tema suleb ja ükski ei ava.
\par 23 Mina löön ta nagu varna kindlasse kohta ja ta saab aujärjeks oma isa soole.
\par 24 Tema külge riputatakse kogu ta isa soo koorem: võrsed ja lehed, kõik väiksed asjadki, kausikestest kuni igasuguste kruusideni.
\par 25 Siis, ütleb vägede Issand, annab järele kindlasse kohta löödud varn: see murdub ja langeb, ja koorem, mis selle küljes oli, puruneb. Jah, Issand on rääkinud.

\chapter{23}

\par 1 Ennustus Tüürose kohta: Ulguge, Tarsise laevad, sest teie varjupaik hävitatakse! Kui nad tulevad kittide maalt, saab see neile teatavaks.
\par 2 Vaikima peab rannarahvas, Siidoni kaupmees, kelle saadik sõidab merd.
\par 3 Suurtel vetel oli Siihori külv ja Niiluse lõikus, temast sai rahvaste turg.
\par 4 Häbene, Siidon, sa tugev linn, sest meri kõneleb: „Ma pole olnud lapsevaevas ega ole sünnitanud; ma pole kasvatanud noori mehi ega ole lasknud sirguda neitseid.”
\par 5 Kui Egiptuses sellest kuuldakse, siis nad värisevad nagu kuuldusest Tüürosegi kohta.
\par 6 Tõtake Tarsisesse! Ulguge, rannarahvas!
\par 7 On see teie rõõmus linn, mille algus on muistses ajas, keda jalad kandsid kaugele võõrana elama?
\par 8 Kes on seda otsustanud kroonide jagaja Tüürose kohta, kelle kaupmehed on vürstid, kelle kaubitsejad on kõige auväärsemad maa peal?
\par 9 Vägede Issand on otsustanud teotada kogu toreduse kõrkust, alandada kõiki auväärseid maal.
\par 10 Käi oma maa läbi, Tarsise tütar, otsekui Niilus! Vööd ei ole enam!
\par 11 Tema sirutab käe mere kohale, paneb kuningriigid värisema - Issand annab käsu Kaanani kohta ta kindluste hävitamiseks.
\par 12 Ta ütleb: „Ära ole enam ülemeelik, kurjasti koheldud neitsi, Siidoni tütar! Tõuse, siirdu kittide juurde, aga sealgi ei ole sul rahu!”
\par 13 Vaata kittide maad! See on rahvas, keda ei olnud olemas, kes on siidonlaste rajatud: nemad püstitasid tema vahitornid, linnad ja paleed - tema teeb selle varemeiks.
\par 14 Ulguge, Tarsise laevad, sest teie kindlus on hävitatud!
\par 15 Ja sel ajal unustatakse Tüüros seitsmekümneks aastaks, ühe kuninga elupäevadeks. Seitsmekümne aasta pärast sünnib Tüürosega nagu hoora laulus:
\par 16 „Võta kannel, käi linnas ringi, sa unustatud hoor! Mängi ilusasti, laula usinasti, et sind meelde tuletataks!”
\par 17 Seitsmekümne aasta pärast katsub Issand Tüürost ja see saab jälle hoorapalgale ning teeb hooratööd kõigi maailma kuningriikidega maa peal.
\par 18 Aga tema tulu ja hoorapalk on pühitsetud Issandale; seda ei kuhjata ega talletata, sest see tulu saab neile, kes elavad Issanda ees, külluslikuks toiduseks ja kauneiks kehakatteiks.”

\chapter{24}

\par 1 Vaata, Issand laastab maa ja rüüstab selle, paiskab segi selle pinna ja pillutab elanikud.
\par 2 Ja nagu rahvaga, nõnda sünnib preestriga, nagu sulasega, nõnda ta isandaga, nagu teenijaga, nõnda ta emandaga, nagu ostjaga, nõnda müüjaga, nagu laenuandjaga, nõnda laenajaga, nagu võlausaldajaga, nõnda võlgnikuga.
\par 3 Maa laastatakse täiesti ja riisutakse sootuks, sest Issand on öelnud selle sõna.
\par 4 Maa leinab, laostub, ilmamaa närbub, laostub, Maa rahva ülikud rammestuvad.
\par 5 Maa on rüvetunud oma elanike all, sest need on üle astunud Seadusest, muutnud seadlusi, murdnud igavese lepingu.
\par 6 Seepärast neelab needus maa ja selle elanikud peavad kandma oma süüd; seepärast kahaneb maa elanike hulk ja vähe inimesi jääb järele.
\par 7 Vein leinab, viinapuu kuivab, ägavad kõik, kes olid südamest rõõmsad.
\par 8 Lõpeb trummide lõbu, lakkab hõiskajate lärm, kaob kandle rõõm.
\par 9 Enam ei jooda lauldes veini, vägijook on joojale kibe.
\par 10 Tühi linn purustatakse, kõik kojad suletakse, sissepääsu ei ole.
\par 11 Tänavail karjutakse veini järele, kõik rõõm kaob, maa lust lahkub.
\par 12 Linna jääb häving ja värav taotakse puruks.
\par 13 Sest nõnda peab sündima keset maad, rahvaste keskel, otsekui õlipuu raputus, otsekui järelnoppimine pärast viinamarjalõikuse lõppu.
\par 14 Need seal tõstavad häält, rõkatavad rõõmust, Issanda ülevuse tõttu hõiskavad lääne poolt:
\par 15 „Seepärast austage ka ida pool Issandat, mere saartel Issanda, Iisraeli Jumala nime.”
\par 16 Maa äärest kuuleme kiituslaule: „Austus õigele!” Aga mina ütlen: Kadu mulle, kadu mulle! Häda mulle, petised petavad, jah, petised petavad petise kombel!
\par 17 Pelg, püügiauk ja püüdepael on su ees, maa elanik.
\par 18 Kes siis põgeneb peletushüüu eest, langeb püügiauku; aga kes tuleb püügiaugust välja, selle püüab püüdepael. Sest kõrguse luugid avanevad ja maa alused vappuvad.
\par 19 Maa põrub põrmuks, Maa lõheneb lõhki, Maa kõigub kõvasti.
\par 20 Maa vaarub tugevasti, joobnu sarnaselt, ja vangub otsekui vahionn. Tema üleastumine lasub raskesti ta peal, ta langeb ega tõuse enam.
\par 21 Sel päeval nuhtleb Issand kõrguse väehulki kõrgel ja maa kuningaid maa peal.
\par 22 Need kogutakse kokku, kogutakse vangidena auku, suletakse vangitorni ja hulga aja pärast nuheldakse neid.
\par 23 Siis on kuul häbi ja päike häbeneb, sest vägede Issand on kuningas Siioni mäel ja Jeruusalemmas, ja tema vanemate ees paistab auhiilgus.

\chapter{25}

\par 1 Issand, sina oled mu Jumal, ma tahan sind ülistada, su nime kiita, sest sa oled teinud imet, su otsused muistsest ajast on õiged ja kindlad.
\par 2 Sest sa tegid linna kivivaremeks ja kindlustatud linna rusuhunnikuks. Võõraste kants ei ole enam linn, seda ei ehitata iialgi üles.
\par 3 Seepärast austab sind vägev rahvas, aga julmade paganate linnad kardavad sind.
\par 4 Sest sa oled kindluseks viletsale ja pelgupaigaks vaesele ta kitsikuses, ulualuseks raju eest, varjuks palavuse eest; on ju julmade viha otsekui raju, mis raputab müüri,
\par 5 otsekui lõõsk põuaajal. Sina vaigistad võõraste möllu, otsekui palavus pilve varjus vaikib julmade võidulaul.
\par 6 Ja vägede Issand valmistab sellel mäel kõigile rahvaile võõruspeo rammusate roogadega, võõruspeo laagerdatud veiniga, üdirammusate roogadega, pärmi pealt selitatud veiniga.
\par 7 Ta hävitab sel mäel loori, mis looritab kõiki rahvaid, ja katte, mis katab kõiki paganaid.
\par 8 Ta neelab surma ära igaveseks ajaks. Ja Issand Jumal pühib pisarad kõigilt palgeilt ning kõrvaldab oma rahva teotuse kogu maalt. Jah, Issand on rääkinud!
\par 9 Ja sel päeval öeldakse: „Vaata, see on meie Jumal, keda me ootasime, et ta meid päästaks. See on Issand, keda me ootasime, hõisakem ja tundkem rõõmu tema päästest,
\par 10 sest Issanda käsi viibib sellel mäel.” Aga Moab tallatakse maha ta omas paigas, otsekui tallataks õlekubu virtsalompi.
\par 11 Ja kui ta selle sees sirutab oma käsi otsekui ujuja ujudes, alandatakse ta suurelisus ikkagi, hoolimata ta käte kunstist.
\par 12 Jah, kindel linn, ta laseb langeda su kaitsvad müürid, alandab, paiskab maha põrmu.

\chapter{26}

\par 1 Sel päeval lauldakse Juudamaal seda laulu: Meil on tugev linn, pääste on pandud müürideks ja kaitsevalliks.
\par 2 Avage väravad, et saaks sisse minna õige rahvas, kes püsib usus!
\par 3 Kindlameelsele sa hoiad rahu, rahu, sest ta loodab sinu peale.
\par 4 Lootke alati Issanda peale, sest Issand Jumal on igavene kalju.
\par 5 Sest tema langetab need, kes elavad kõrgel, ligipääsmatu linna; ta alandab seda, alandab maani, paiskab põrmu.
\par 6 Seda tallab jalg, õnnetute jalad, viletsate sammud.
\par 7 Õigete tee on tasane, sina sillutad siledaks õigete raja.
\par 8 Issand! Me ootame sind ka su kohtumõistmiste teel, hing igatseb su nime ja su mälestust.
\par 9 Mu hing igatseb sind öösel, vaimgi mu sees otsib sind, sest kui sinu kohtumõistmised tabavad maad, õpivad ilmamaa elanikud õiglust.
\par 10 Leiab aga õel armu, ei õpi ta õiglust, õigusemaalgi teeb ta ülekohut ega näe Issanda kõrgust.
\par 11 Issand, sinu käsi on tõstetud kõrgele, aga nad ei näe seda. Nähku nad siis su püha viha rahva pärast ja häbenegu! Jah, neelaku neid tuli kui sinu vaenlasi!
\par 12 Issand, sina saadad meile rahu, sest kõik meie teodki oled sina teinud.
\par 13 Issand, meie Jumal! Meid on valitsenud muud isandad peale sinu - me kiidame üksnes sind, sinu nime.
\par 14 Surnud ei ärka ellu, kadunud ei tõuse üles. Sest sina nuhtlesid neid ja hävitasid nad ning kaotasid neist iga mälestuse.
\par 15 Sina oled lisanud rahvast, Issand, oled lisanud rahvast, oled näidanud oma au, oled laiendanud kõik maa piirid.
\par 16 Issand, nad otsisid sind kitsikuses, sosistasid lausumisi, kui sina neid karistasid.
\par 17 Otsekui lapseootel olija, kel sünnitus ligineb, vaevleb ja kisendab oma valudes, nõnda olime meiegi sinu ees, Issand.
\par 18 Me olime lapseootel, vaevlesime, aga otsekui oleksime sünnitanud tuult: päästet me maale ei toonud ja ilmamaa elanikke ei sündinud.
\par 19 Aga sinu surnud ärkavad ellu, minu laibad tõusevad üles. Ärgake ja hõisake, põrmus lamajad! Sest sinu kaste on valguse kaste ja maa paiskab välja kadunud.
\par 20 Tule, mu rahvas, mine oma kambritesse ja sule uksed enese tagant, peitu üürikeseks ajaks, kuni raev möödub.
\par 21 Sest vaata, Issand väljub oma asupaigast nuhtlema maa elanikke nende ülekohtu pärast. Siis paljastab maa oma veresüü ega kata enam neid, kes ta peal on tapetud.

\chapter{27}

\par 1 Sel päeval nuhtleb Issand oma terava, suure ja tugeva mõõgaga Leviatanit, põgenevat madu, ja Leviatanit, keerdunud madu, ja tapab meres oleva lohe.
\par 2 Selsamal päeval öeldakse: „Tore viinamägi, laulge sellest vastastikku!”
\par 3 Mina, Issand, olen selle valvur, ma kastan seda igal hetkel, ma valvan öösel ja päeval, et sellele kahju ei tehtaks.
\par 4 Viha mul enam ei ole. Ent oleks mul tarvis võidelda kibuvitste ja ohakatega, ma astuksin neile vastu, süütaksin kõik põlema.
\par 5 Või olgu siis, et haaratakse kinni minu kaitsest, tehakse minuga rahu - jah, minuga tehakse rahu.
\par 6 Tulevasil päevil juurdub Jaakob, Iisrael õitseb ning haljendab ja maailm on täis vilja.
\par 7 Ons ta teda löönud, nõnda nagu ta lõi tema lööjat, või teda tapnud, nõnda nagu tapeti tema tapjad?
\par 8 Mõõt mõõdu vastu! Ta lahendas temaga asja teda maapakku saates: ta pühkis ta ära oma tugeva rajuiiliga idatuule päeval.
\par 9 Seepärast lepitatakse Jaakobi süü nõnda, ja niisugune olgu tema patukustutuse täielik vili: ta peab tegema kõik altarikivid pihustatud lubjakivide sarnaseks; ei tohi tõusta viljakustulpi ega suitsutussambaid.
\par 10 Jah, kindel linn jääb üksi, tühjaks ja mahajäetud karjamaaks, kõrbe sarnaseks. Seal käib vasikas karjas, lebab seal ja näsib paljaks selle raod.
\par 11 Kui selle oksad on kuivanud, siis murtakse need; naised tulevad, teevad nendega tuld. Sest see ei ole mõistlik rahvas: seepärast ta Looja ei halasta tema peale ja ta kujundaja ei anna temale armu.
\par 12 Ja sel päeval sünnib, et Issand rabab viljapäid Frati jõest Egiptuseojani, ja teid, Iisraeli lapsed, korjatakse ükshaaval kokku.
\par 13 Sel päeval puhutakse suurt sarve ja niihästi need, kes olid kadunud Assurimaale, kui ka need, kes olid pillutatud Egiptusemaale, tulevad ja kummardavad Issandat Jeruusalemmas pühal mäel.

\chapter{28}

\par 1 Häda Efraimi joomarite toredale kroonile ja tema hiilgava ilu närtsinud õiele künkal keset veinist roidunute lopsakat orgu!
\par 2 Vaata, Issandal on keegi vägev ja võimas, otsekui rahesadu, hävitav torm, võimsalt voolava veetulva sarnane, kes paiskab jõuga jalust.
\par 3 Jalge alla tallatakse Efraimi joomarite tore kroon
\par 4 ja tema hiilgava ilu närtsinud õiel, mis on künkal keset lopsakat orgu, käib käsi nagu varakult, enne lõikust küpsenud viigimarjal: kes seda iganes näeb, neelab selle alla, kui saab selle pihku.
\par 5 Sel päeval on vägede Issand ilusaks krooniks ja kauniks pärjaks oma rahva jäägile
\par 6 ja õiguse vaimuks kohtus istujale ning rammuks neile, kes tõrjuvad tapluse tagasi väravasse.
\par 7 Aga needki siin vaaruvad veinist ja tuiguvad vägijoogist: preester ja prohvet vaaruvad vägijoogist, on segased veinist, tuiguvad vägijoogist, eksivad nägemuses, kõiguvad otsuses.
\par 8 Sest kõik lauad on täis okset ja rooja, puhast paika ei olegi.
\par 9 Kellele ta tahab õpetada tarkust ja kellele seletada ilmutust? Kas piimast võõrutatuile, rinnalt võetuile?
\par 10 Sest see on: käsk käsu peale, käsu peale käsk, mõõdunöör mõõdunööri peale, mõõdunööri peale mõõdunöör, pisut siin, pisut seal.
\par 11 Tõesti, kogelevate huultega ja võõras keeles kõneleb ta sellele rahvale,
\par 12 tema, kes neile ütles: „See on hingamispaik, laske väsinud hingata, see on kosutuskoht!” Ometi nad ei tahtnud kuulda.
\par 13 Seepärast on Issanda sõna neile: Käsk käsu peale, käsu peale käsk, mõõdunöör mõõdunööri peale, mõõdunööri peale mõõdunöör, pisut siin, pisut seal, et nad läheksid ja kukuksid selili, vigastaksid endid ning laseksid endid püüda ja vangi võtta.
\par 14 Seepärast kuulge Issanda sõna, te hooplejad mehed, pilkajad selle rahva hulgas, kes asub Jeruusalemmas,
\par 15 sest te ütlete: „Me oleme sõlminud surmaga lepingu ja loonud põrguhauaga liidu - tulgu või uputusetulv, meieni see ei ulatu, sest me valisime oma varjupaigaks vale ja pugesime pelgu pettusesse.”
\par 16 Seepärast ütleb Issand Jumal nõnda: Vaata, see olen mina, kes paneb Siionis aluskivi, valitud kivi, kalli nurgakivi, kindla aluse: kes usub, see ei tunne rahutust.
\par 17 Mina panen õiguse mõõdunööriks ja õigluse loodiks. Rahe hävitab vale varjupaiga ja veed uhuvad ulualuse.
\par 18 Teie leping surmaga lõpeb ja teie liit põrguhauaga ei püsi: kui uputusetulv teist üle käib, muutute selle tallermaaks.
\par 19 Alati, kui see üle käib, haarab see teid, sest see tuleb igal hommikul, päeval ja ööl, ja ilmutuse mõistmine muutub lausa hirmuks.
\par 20 Sest voodi on sirutuseks lühike ja vaip katteks kitsas.
\par 21 Sest nagu Peratsimi mäel tõuseb Issand; ta vihastab nagu Gibeoni orus, et teha oma tööd - hämmastav on tema töö! - ja et toimetada oma tegu - võõristav on tema tegu.
\par 22 Nüüd aga ärge enam hoobelge, et teie kütked jääksid pingutamata, sest ma olen kuulnud Jumalalt, vägede Issandalt, otsusest ja hävitusest, mis tabab kogu maad.
\par 23 Kuulatage ja kuulge mu häält, pange tähele ja kuulge mu kõnet!
\par 24 Kas kündja kogu päeva külvi tarvis künnab, kobestab ja äestab oma põllumaad?
\par 25 Eks ole nõnda: kui ta on pinna tasandanud, siis ta puistab tilli ja riputab köömneid, paneb paika nisu, hirsi ja odra ning äärele okasnisu?
\par 26 Nõnda on temale juhatatud õige viis, tema Jumal õpetab teda.
\par 27 Tilli ju ei peksta pahmareega ja köömnete peal ei veeretata vankriratast, vaid tilli klopitakse kepiga ja köömneid vitsaga.
\par 28 Leivavili jahvatatakse peeneks, aga iial ei peksta seda lõpmata: pannakse liikuma oma vankriratas ja hobused, aga seda ei peksta puruks.
\par 29 Ka see on tulnud vägede Issandalt; tema nõu on imeline ja tema tarkus suur.

\chapter{29}

\par 1 Häda sulle, Ariel, Ariel, linn, kus Taavet leeri üles lõi! Lisandugu aastale aasta, tehku seatud pühad oma ringkäiku,
\par 2 mina rõhun Arieli: tuleb kurbus ja kurvastus ja ta muutub mulle otsekui ohvrialtariks.
\par 3 Ma löön su vastu üles ringikujulise leeri, ümbritsen sind valvega ja püstitan su vastu piiramisseadmed.
\par 4 Siis sa räägid maa madalusest ja tasa kõlab põrmust su kõne; siis tuleb su hääl maa seest otsekui vaimul ja sa sosistad põrmust oma sõnu.
\par 5 Su vaenlaste hulk on otsekui peenike tolm, rõhujate jõuk lendlevate aganate sarnane. Sest silmapilkselt, äkitselt sünnib,
\par 6 et vägede Issandalt tuleb sulle katsumine äikese, maavärisemise ja suure mürinaga, tuulispea ja marutuulega ning hävitava tuleleegiga.
\par 7 Ja nagu unenägu, öine nägemus, on kõigi Arieli vastu sõdivate rahvaste hulk, samuti kõik, kes sõdivad tema ja ta kindluste vastu ning ahistavad teda.
\par 8 See on, nagu näeks näljane und, et vaata, ta sööb. Kui ta ärkab, siis on ta hing täitmata. Või nagu näeks janune und, et vaata, ta joob. Kui ta ärkab, vaata, siis on ta nõrk ja ta hing on närbunud. Samuti sünnib kõigi rahvaste hulkadega, kes sõdivad Siioni mäe vastu.
\par 9 Pidage ja hämmastuge, saage pimedaks ja ärge nähke! Nad on joobnud, aga mitte veinist, nad vaaruvad, aga mitte vägijoogist.
\par 10 Sest Issand on valanud teie peale unisuse vaimu ja on sulgenud teie silmad, prohvetid, ning katnud teie pead, nägijad.
\par 11 Seepärast on kogu ilmutus teile nagu pitseeritud raamatu sõnad; kui see antakse mõnele, kes kirja tunneb, ja öeldakse: „Loe ometi!„, siis ta vastab: ”Ma ei saa, sest see on pitseriga kinni.”
\par 12 Või kui raamat antakse sellele, kes kirja ei tunne, ja öeldakse: „Loe ometi!„, siis vastab see: ”Ma ei tunne kirja.”
\par 13 Ja Issand ütleb: Kuna see rahvas ligineb mulle suuga ja austab mind huultega, aga ta süda on minust kaugel ja nende kartus minu ees on ainult päritud inimlik käsk,
\par 14 siis vaata, ma teen sellele rahvale veel imet - imelikul ja kummalisel viisil: tema tarkade tarkus kaob ja tema arukate arukus peidetakse.
\par 15 Häda neile, kes oma nõu Issanda eest sügavale ära peidavad, kelle teod sünnivad pimedas ja kes ütlevad: „Kes meid näeb? Kes meid tunneb?”
\par 16 Oh teie põikpäisust! Kas peetakse savi võrdseks potissepaga, kas ütleb töö oma tegijale: „Tema ei ole mind teinud!„ või ütleb kuju oma voolijale: ”Tema ei oska midagi!”?
\par 17 Eks ole ju veel ainult pisut aega, kuni Liibanon muutub viljapuuaiaks ja viljapuuaeda hakatakse pidama metsaks?
\par 18 Sel päeval kuulevad kurdid kirja sõnu ja pimedate silmad näevad pilkasest pimedusest.
\par 19 Siis tunnevad alandlikud aina rõõmu Issandas ja kõige vaesemad inimesed hõiskavad Iisraeli Pühas.
\par 20 Sest vägivaldsele tuleb lõpp ja pilkaja saab otsa, ja hävitatakse kõik, kes kavatsevad kurja,
\par 21 kes teevad sõnaga inimese süüdlaseks, kes seavad püünise kohtumõistjale väravas ja tõrjuvad õige põhjuseta kõrvale.
\par 22 Seepärast ütleb Issand, kes lunastas Aabrahami, Jaakobi soole nõnda: Jaakob ei jää nüüd enam häbisse ja ta pale ei kahvata enam.
\par 23 Vaid kui ta näeb oma lapsi, mu kätetööd, enese keskel, siis nad pühitsevad minu nime ja peavad pühaks Jaakobi Püha ning kardavad Iisraeli Jumalat.
\par 24 Ja need, kes olid eksivaimus, tulevad mõistusele ja nurisejad võtavad õpetust.

\chapter{30}

\par 1 Häda kangekaelseile lastele, ütleb Issand, kes peavad nõu, aga ilma minuta, ja kes teevad liidu, aga ilma minu Vaimuta, lisades patule patu,
\par 2 kes lähevad alla Egiptusesse, küsimata nõu minu suust, et põgeneda vaarao kaitse alla ja otsida varju Egiptuses.
\par 3 Seepärast tuleb vaarao kaitse teile häbiks ja Egiptuse varju alla kippumine teotuseks.
\par 4 Sest kuigi ta vürstid on Soanis ja ta käskjalad on jõudnud Haanessi,
\par 5 jäävad kõik häbisse rahva pärast, kellest neil ei ole kasu, kes ei ole abiks ega kasuks, vaid häbiks ja koguni teotuseks.
\par 6 Ennustus Lõunamaa Jõehobu kohta: Läbi häda ja viletsuse maa, kust on pärit ema- ja isalõvid, mürkmaod ja lendmaod, viivad nad eeslite seljas oma rikkused ja kaamelite küürudel varandused rahvale, kellest ei ole kasu.
\par 7 Et Egiptuse abi on tühine ja asjatu, siis ma nimetan seda „Paigalpüsiv tormakus”.
\par 8 Nüüd mine kirjuta see nende nähes tahvlile ja märgi raamatusse, et see võiks jääda tulevasiks päeviks, ikka ja igavesti!
\par 9 Sest nad on tõrges rahvas, valelikud lapsed, lapsed, kes ei taha kuulda Issanda õpetust,
\par 10 kes ütlevad nägijaile: „Ärge nähke!” ja ennustajaile: ”Ärge ennustage meile tõtt, kõnelge meile meelt mööda, ennustage pettepilte,
\par 11 taganege teelt, lahkuge rajalt, jätke meid rahule Iisraeli Pühaga!”
\par 12 Seepärast ütleb Iisraeli Püha nõnda: Kuna te põlgate seda sõna ning loodate vale ja riugaste peale, ja toetute nende peale,
\par 13 siis on see süü teile otsekui väljavajunud müüriosa kõrges praoga müüris, mis variseb äkitselt, silmapilgu jooksul,
\par 14 ja puruneb, nagu puruneb potissepa kruus, mis pekstakse armuta puruks, nõnda et tükkidest ei leidu kildu, millega võtta leest tuld või ammutada lombist vett.
\par 15 Sest nõnda ütleb Issand Jumal, Iisraeli Püha: Pöördudes ja vaikseks jäädes te pääseksite, rahus ja lootuses oleks teie jõud, kuid te pole seda tahtnud,
\par 16 vaid ütlete: Ei, me põgeneme hobuste seljas! - sellepärast peategi põgenema - ja: Nobedal ratsul tahame sõita! - sellepärast ongi teie jälitajad kärmed.
\par 17 Tuhat põgeneb üheainsa ähvardaja eest, teie põgenete viie ähvardaja eest, kuni teist jääb jääk, mis on otsekui lipuvarras mäetipul või lipp künkal.
\par 18 Ja ometi ootab Issand, et teile armu anda, ja jääb kõrgeks, et teie peale halastada, sest Issand on õiguse Jumal, õndsad on kõik, kes teda ootavad.
\par 19 Jah, sina, rahvas Siionis, kes elad Jeruusalemmas, ära enam nuta! Tema on sulle tõesti armuline, kui sa appi hüüad. Seda kuuldes ta vastab sulle.
\par 20 Issand annab teile küll hädaleiba ja viletsusevett, kuid su õpetajad ei jää enam kõrvale, vaid su silmad saavad näha su õpetajaid.
\par 21 Ja su kõrvad kuulevad sõna, mis su tagant ütleb, kui te kaldute paremale või vasakule: „See on tee, käige seda!”
\par 22 Siis sa rüvetad oma hõbedaga kaetud nikerdatud kujud ja oma kullaga karratud valatud kujud: sa viskad need ära otsekui mingi jälkuse. „Välja!” ütled sa neile.
\par 23 Siis annab tema vihma su seemnele, mida sa külvad põllule, ja maa saagist leiba, mis on toitev ja rammus; sel päeval sööb su kari avaral karjamaal.
\par 24 Härjad ja eeslid, kellega haritakse põldu, söövad oblikatega segatud toitu, mida on puistatud visklabida ja hanguga.
\par 25 Ja igal kõrgemal mäel ja igal kerkival künkal on ojasid voolava veega suurel tapapäeval, kui tornid langevad.
\par 26 Siis on kuu valgus nagu päikese valgus, ja päikese valgus on seitsmekordne otsekui seitsme päeva valgus, päeval, kui Issand seob oma rahva vigastusi ja parandab tema löögihaavu.
\par 27 Vaata, Issanda nimi tuleb kaugelt, ta viha põleb ja suitsupilv on ränk; ta huuled on täis sajatust ja ta keel on otsekui hävitav tuli.
\par 28 Tema hingus on tulvava jõe sarnane, mis ulatub kaelani, et lennutada paganaid õnnetusikkesse ja panna rahvaste lõuapäradesse eksiteele juhtivad suulised.
\par 29 Siis te laulate otsekui püha pühitsemise ööl ja rõõmustate südamest otsekui vilepilliga käija, kes läheb Issanda mäele, Iisraeli Kalju juurde.
\par 30 Ja Issand annab kuulda oma võimsat häält ja näha oma käsivarre laskumist maruvihas hävitava tuleleegina koos paduvihma ja äikese ning raheteradega.
\par 31 Jah, Assur ehmub Issanda häälest, kui ta vitsaga peksab.
\par 32 Ja iga määratud kepihoop, mis Issand laseb langeda tema peale, sünnib trummide ja kannelde saatel, kui ta ägedas võitluses tema vastu sõdib.
\par 33 Sest ammusest ajast on valmis põletuspaik - see on ka kuninga jaoks. Sügavaks ja laiaks on see tehtud, tuleriidas on ohtrasti tuld ja puid; Issanda hingus, otsekui väävlijõgi, süütab selle.

\chapter{31}

\par 1 Häda neile, kes lähevad Egiptusesse abi saama ja otsivad tuge hobustest ning loodavad sõjavankrite peale, sellepärast et neid on palju, ja ratsanike peale, sellepärast et nende hulk on väga suur, aga ei vaata Iisraeli Püha poole ega küsi nõu Issandalt!
\par 2 Kuid temagi on tark ja laseb tulla õnnetuse ega võta tagasi oma sõnu, vaid tõuseb kurjategijate soo vastu ja nende abi vastu, kes teevad nurjatust.
\par 3 Sest egiptlased on inimesed, aga mitte Jumal, ja nende hobused on liha, aga mitte vaim. Kui Issand sirutab käe, siis komistab aitaja ja langeb aidatav ja nad kõik hukkuvad üheskoos.
\par 4 Sest Issand on mulle öelnud nõnda: Otsekui lõvi või noor lõvi, kes uriseb oma saagi kallal ja kelle vastu on kokku kutsutud karjaste jõuk, ei kohku nende hüüdeist ega lömita nende kisa pärast, nõnda astub alla vägede Issand sõdima Siioni mäel ja künkal.
\par 5 Nagu linnud laotavad oma tiivad laiali, nõnda kaitseb vägede Issand Jeruusalemma, kaitseb ja päästab, säästab ja vabastab.
\par 6 Pöörduge tagasi tema juurde, kellest te olete taganenud nii kaugele, Iisraeli lapsed!
\par 7 Sest sel päeval viskab igaüks ära oma hõbedased ja kuldsed ebajumalad, mis teie käed on teinud teile patuks.
\par 8 Ja Assur langeb, aga mitte mehe mõõga läbi, mitte inimese mõõk ei hävita teda. Ta peab põgenema mõõga eest ja ta noored mehed saavad orjadeks.
\par 9 Tema kalju hääbub hirmu pärast ja ta vürstid ehmuvad võitluslipu ees, ütleb Issand, kellel on tuli Siionis ja põletusahi Jeruusalemmas.

\chapter{32}

\par 1 Vaata, kuningas hakkab valitsema õigluses ja vürstid asuvad juhtima, nagu on kohus.
\par 2 Siis on igaüks neist otsekui pelgupaik tuule või ulualune vihmahoo eest, otsekui veeojad põuases paigas, otsekui võimsa kalju vari märga igatseval maal.
\par 3 Siis pole nägijate silmad suletud ja kuuljate kõrvad panevad tähele.
\par 4 Läbematute süda õpib tundma tõde ja kogelejate keel ruttab selgesti rääkima.
\par 5 Jõledat ei hüüta enam õilsaks ja petist ei nimetata auliseks.
\par 6 Sest jõle räägib jõledust ja ta süda taotleb nurjatust, et teha jumalavallatust ja rääkida Issanda kohta valet, et näljast hinge lasta nälgida ja janusele keelata jooki.
\par 7 Petise relvad on kurjad, ta kavatseb häbitegusid, et hävitada hädalisi valekõnega, isegi kui vaene räägib õigust.
\par 8 Aga õilis kavatseb õilsaid asju ja jääb püsima õilsate asjade juurde.
\par 9 Muretud naised, tõuske üles, kuulake mu häält; hooletud tütred, kuulake mu kõnet!
\par 10 Veel aasta ja mõned päevad, siis te, hooletud, värisete, sest viinamarjalõikusel on lõpp, korjamist enam ei tule.
\par 11 Vabisege, muretud, värisege, muretud, värisege, hooletud, riietuge lahti, võtke endid alasti ja vöötage niuded!
\par 12 Siis lüüakse endile vastu rindu toredate põldude pärast, viljakate viinapuude pärast,
\par 13 minu rahva põllumaa pärast, mis kasvatab kibuvitsu ja ohakaid, jah, kõigi lusthoonete pärast selles ülemeelikus linnas.
\par 14 Sest palee jääb maha, linna lärm lakkab, templiküngas ja vahitorn muutuvad igavesti kõledaiks väljadeks, metseeslitele rõõmuks, karjadele karjamaaks,
\par 15 kuni meie peale valatakse Vaim ülalt. Siis saab kõrb viljapuuaiaks ja viljapuuaeda hakatakse pidama metsaks.
\par 16 Siis elab õigus kõrbes ja õiglus võtab aset viljapuuaias.
\par 17 Ja õigluse vili on rahu, õigluse tulemuseks püsiv rahulik elu ning julgeolek.
\par 18 Ja mu rahvas elab rahu eluasemel, kindlais elamuis ning häirimatuis hingamispaigus,
\par 19 kui sajab rahet, mets langeb maha ja linn vajub madalusse.
\par 20 Õnnelikud olete teie, külvajad kõigi vete ääres, kes te võite härja ja eesli jalga lasta vabalt joosta.

\chapter{33}

\par 1 Häda sulle, hävitaja, kes ise oled hävitamata, reetur, keda ei ole reedetud! Kui oled hävitusega valmis, hävitatakse sind, kui oled reetmise lõpetanud, reedetakse sind.
\par 2 Issand, ole meile armuline, me ootame sind! Ole meie käsivars igal hommikul, ole meie abi kitsikuse ajal.
\par 3 Su kõuehääle eest põgenevad rahvad, kui sa tõused, hajuvad paganad
\par 4 ja saak kahmatakse meie käest, otsekui ahmaksid röövputukad; rohutirtsude rünnaku sarnaselt tormatakse selle kallale.
\par 5 Issand on kõrge, sest ta elab kõrgel, tema täidab Siioni õiguse ja õiglusega.
\par 6 Tema on su aegade tagatis, abi küllus, tarkus ja tunnetus; Issanda kartus - see on Siioni rikkus.
\par 7 Vaata, Arieli elanikud kisendavad õues, rahu saadikud nutavad kibedasti.
\par 8 Maanteed on tühjad, teekäijat ei ole, leping on tühistatud, tunnistajad on hüljatud, inimestest ei hoolita.
\par 9 Maa leinab, närbub, Liibanon häbeneb, närtsib; Saaron on saanud kõnnumaaks, Baasan ja Karmel varistavad lehti.
\par 10 Nüüd ma tõusen, ütleb Issand, nüüd ma tahan olla kõrge, nüüd ma ülendan ennast.
\par 11 Kuluheinast olete rasedad, kõrsi toote ilmale, teie hingeõhk on tuli, mis teid sööb.
\par 12 Jah, rahvad põletatakse lubjaks, otsekui maharaiutud kibuvitsad süüdatakse nad tulega.
\par 13 Kuulge, kaugelviibijad, mida ma olen teinud, ja juuresolijad, tundke mu väge!
\par 14 Patused Siionis värisevad, vabin valdab jumalatuid: „Kes meist saaks elada hävitavas tules? Kes meist saaks elada igavesel leel?”
\par 15 See, kes elab õiguses ja räägib tõtt, kes põlgab rõhumisest saadavat kasu, kes raputab käsi, et keelduda meeleheast, kes topib kõrvad kinni, et mitte kuulda veretööst, ja kes suleb silmad, et mitte näha kurja,
\par 16 see elab kõrgustikel, ligipääsmatud kaljud on temale varjupaigaks; temale antakse ta leib, temale on vesi kindlustatud.
\par 17 Su silmad saavad näha kuningat ta ilus, saavad näha ihaldusväärset maad.
\par 18 Su süda meenutab hirmsaid asju: kus on nüüd kirjutaja, kus on vaagija, kus on tornide lugeja?
\par 19 Enam sa ei näe jultunud rahvast, rahvast, kelle segast kõnet sa ei taipa, kelle kogelevast keelest sa ei saa aru.
\par 20 Vaata Siionit, meie pidupäevade linna! Su silmad näevad Jeruusalemma, rahulikku paika, liigutamatut telki, mille vaiu ei tõmmata iialgi välja ja mille nööridest ei kista ainsatki katki.
\par 21 Sest seal on Vägev: meiega on Issand nagu laiade jõgede ja vooluste kogumik, sellel ei kulge sõudelaev ja seda ei ületa võimsaimgi alus.
\par 22 Sest Issand on meie kohtumõistja, Issand on meie käsuandja, Issand on meie kuningas, tema päästab meid.
\par 23 Su köied lõtvuvad, ei pea paigal mastipuud ega lase purjel paisuda. Siis jagatakse suurt saaki, jalutudki võtavad noosi
\par 24 ja ükski elanik ei ütle: „Ma olen nõder!” Rahvale, kes seal elab, antakse süü andeks.

\chapter{34}

\par 1 Rahvad, astuge ligi kuulma, ja rahvahõimud, pange tähele! Kuulgu maa ja need, kes seda täidavad, maailm ja kõik, kes seal võrsuvad!
\par 2 Sest Issandal on raev kõigi rahvaste ja viha kõigi nende väehulkade vastu: ta on pannud need vande alla, ta on andnud need tapetavaiks.
\par 3 Kes neist maha lüüakse, visatakse ära, nende laipadest tõuseb lehk, nende verest nõretavad mäed.
\par 4 Kõik taevaväed kaovad, taevad keerduvad kokku otsekui rullraamat: kogu nende vägi variseb, nagu variseb viinapuust leht või viigipuust kuivanud mari.
\par 5 Sest mu mõõk on joobunud taevas, vaata, see langeb alla Edomi peale, minu poolt neetud rahva peale kohtumõistmiseks.
\par 6 Issanda mõõk on verine, tilgub rasvast, tallede ja sikkude verest, jäärade neerurasvast, sest Issandal on ohver Bosras ja palju tapmist Edomimaal.
\par 7 Metshärjadki langevad koos nendega, härjavärsid koos sõnnidega: nende maa joobub verest ja muld muutub rasvast rammusaks.
\par 8 Sest see on Issanda kättemaksupäev, tasumisaasta riiu eest Siioniga.
\par 9 Edomi jõed muutuvad tõrvaks, muld väävliks, ja ta maa on otsekui põlev tõrv.
\par 10 See ei kustu ööl ega päeval, igavesti tõuseb selle suits. See jääb laastatuks põlvest põlve, iialgi ei käi keegi sealtkaudu.
\par 11 Puguhani ja siil pärivad selle, öökull ja kaaren elutsevad seal; selle üle veetakse tühjuse mõõdunöör ja paljasoleku lood.
\par 12 Seal ei ole enam ülikuid, et kedagi kuningaks kuulutada, ja kõik ta vürstid saavad otsa.
\par 13 Tema paleedes kasvavad kibuvitsad, linnustes nõgesed ja ohakad; see saab ðaakalite asukohaks ja jaanalindude pesitsuspaigaks.
\par 14 Kurjad vaimud kohtuvad seal sortsidega ja sikujalgne paharet hüüab seltsilist. Tõesti, seal viibib tont ja leiab enesele puhkepaiga.
\par 15 Seal pesitseb süstmadu, muneb, haub pojad välja, ja kogub oma varju alla. Tõesti, sinna kogunevad raisakotkad, igaüks oma paarilisega.
\par 16 Uurige Issanda raamatust ja lugege: ükski neist ei ole puudu, ükski neist ei ole teist kaotanud, sest tema on oma suu kaudu andnud käsu ja tema Vaim on need kogunud.
\par 17 Tema on neile liisku heitnud, tema käsi on selle neile mõõdunööriga jaotanud: nad pärivad selle igaveseks, nad elavad seal põlvest põlve.

\chapter{35}

\par 1 Kõrb ja liivik rõõmutsevad, nõmmemaa hõiskab ja õitseb nagu liilia.
\par 2 Ta õitseb kaunisti ja ilutseb rõõmu ning hõiskamisega, temale antakse Liibanoni toredus, Karmeli ja Saaroni ilu. Nad saavad näha Issanda toredust, meie Jumala ilu.
\par 3 Kinnitage nõrku käsi ja tehke tugevaks komistavad põlved!
\par 4 Öelge neile, kel rahutu süda: Olge kindlad, ärge kartke! Vaata, teie Jumal! Kättemaks tuleb, teie Jumala karistus; tema ise tuleb ja päästab teid.
\par 5 Siis avanevad pimedate silmad ja kurtide kõrvad lähevad lahti.
\par 6 Siis hüppab jalutu otsekui hirv ja keeletu keel hõiskab, sest veed keevad üles kõrbes ja ojad nõmmemaal.
\par 7 Kuumavirvendusest saab järv ja põuasest pinnast keevad veeallikad üles; seal, kus on ðaakalite eluase, kasvab kaislaid, kõrkjaid ja pilliroogu.
\par 8 Ja seal on maantee ja tee, mida nimetatakse pühaks teeks; ükski rüve ei või sellel käia, vaid see on tema rahva jaoks: kes seda teed käib, ei eksi, rumaladki mitte.
\par 9 Seal ei ole lõvi ega lähe sinna murdjad kiskjad, neid seal ei leidu. Aga lunastatud käivad.
\par 10 Ja Issanda vabaksostetud pöörduvad tagasi ning tulevad Siionisse hõisates. Nende pea kohal on igavene rõõm: rõõm ja ilutsemine valdavad neid, aga kurbus ja ohkamine põgenevad ära.

\chapter{36}

\par 1 Kuningas Hiskija neljateistkümnendal aastal tuli Assuri Kuningas Sanherib kõigi Juuda kindlustatud linnade kallale ja vallutas need.
\par 2 Ja Assuri kuningas läkitas Laakisest Jeruusalemma kuningas Hiskija vastu ülemjoogikallaja suure väehulgaga, kes jäi peatuma Ülatiigi veejuhtme juurde, mis on Vanutajavälja maantee ääres.
\par 3 Siis läksid välja tema juurde Eljakim, Hilkija poeg, kes oli kojaülem, kirjutaja Sebna ja nõunik Joah, Aasafi poeg.
\par 4 Ja ülemjoogikallaja ütles neile: „Öelge ometi Hiskijale: Nõnda ütleb suurkuningas, Assuri kuningas: Mis lootus see on, mida sa hellitad?
\par 5 Sa arvad, et paljad sõnad annavad võitluseks nõu ja jõu? Kelle peale sa loodad, et oled mulle vastu hakanud?
\par 6 Vaata, sa loodad Egiptuse, tolle murtud pillirookepi peale, mis tungib pihku ja puurib selle läbi, kui keegi selle peale toetub. Niisugune on vaarao, Egiptuse kuningas, kõigile, kes tema peale loodavad.
\par 7 Või kui sa mulle ütled: Meie loodame Issanda, oma Jumala peale!, kas pole siis mitte tema see, kelle ohvrikünkad ja altarid Hiskija kõrvaldas, öeldes Juudale ja Jeruusalemmale: Selle altari ees peate te kummardama!?
\par 8 Nüüd aga vea ometi kihla mu isandaga, Assuri kuningaga: mina annan sulle kaks tuhat hobust, kui sa suudad hankida neile ratsanikke!
\par 9 Kuidas sa siis saaksid tagasi tõrjuda asevalitseja, ühe mu isanda vähimaist sulaseist? Aga sa loodad Egiptuse, tema sõjavankrite ja ratsanike peale.
\par 10 Kas ma siis nüüd Issandata olen tulnud selle maa vastu, et seda hävitada? Issand ise ütles mulle: Mine sinna maale ja hävita see!”
\par 11 Siis ütlesid Eljakim, Sebna ja Joah ülemjoogikallajale: „Räägi ometi oma sulastega aramea keelt, sest me mõistame seda, aga ära Räägi meiega juudi keelt müüri peal oleva rahva kuuldes!”
\par 12 Kuid ülemjoogikallaja vastas: „Kas mu isand on mind läkitanud kõnelema neid sõnu ainult su isandale ja sinule? Küllap ka müüri peal istuvaile meestele, kes koos teiega peavad sööma oma rooja ja jooma oma kust!”
\par 13 Ning ülemjoogikallaja astus ette ja hüüdis valju häälega juudi keeles ning ütles: „Kuulge suurkuninga, Assuri kuninga sõnu!
\par 14 Nõnda ütleb kuningas: Ärge laske endid Hiskijast petta, sest tema ei suuda teid päästa!
\par 15 Ärgu pangu Hiskija teid lootma Issanda peale, öeldes: Issand päästab meid kindlasti ja seda linna ei anta Assuri kuninga kätte!
\par 16 Ärge kuulake Hiskijat, sest Assuri kuningas ütleb nõnda: Tehke minuga alistusleping ja tulge välja mu juurde, siis te võite süüa igamees oma viinapuust ja igamees oma viigipuust ja juua igamees oma kaevust vett,
\par 17 kuni ma tulen ja viin teid maale, mis on teie maa sarnane, vilja ja veini maa, leiva ja viinamägede maa.
\par 18 Asjata ässitab teid Hiskija ja ütleb: Issand päästab meid! Kas on muude rahvaste jumalaist mõni päästnud oma maa Assuri kuninga käest?
\par 19 Kus on Hamati ja Arpadi jumalad? Kus on Sefarvaimi jumalad? Kas need päästsid Samaaria minu käest?
\par 20 Kes kõigist nende maade jumalaist on päästnud oma maa minu käest, et Issand peaks päästma minu käest Jeruusalemma?”
\par 21 Aga nad vaikisid ega vastanud temale sõnagi, sest niisugune oli kuninga käsk, kes oli öelnud: „Ärge vastake temale!”
\par 22 Siis tulid kojaülem Eljakim, Hilkija poeg, ja kirjutaja Sebna ja nõunik Joah, Aasafi poeg, Hiskija juurde, riided lõhki käristatud, ja kandsid temale ette ülemjoogikallaja sõnad.

\chapter{37}

\par 1 Kui kuningas Hiskija seda kuulis, siis ta käristas oma riided lõhki, kattis ennast kotiriidega ja läks Issanda kotta.
\par 2 Ja ta läkitas kojaülema Eljakimi, kirjutaja Sebna ja preestrite vanemad, kotiriided seljas, prohvet Jesaja, Aamotsi poja juurde,
\par 3 et need ütleksid temale: „Nõnda ütleb Hiskija: See päev on ahastuse, sõitluse ja teotuse päev. Jah, lapsed on küll jõudnud emakasuudmeni, aga sünnituseks ei ole jõudu.
\par 4 Vahest Issand, su Jumal, kuuleb siiski ülemjoogikallaja sõnu? Tema isand, Assuri kuningas, läkitas ta teotama elavat Jumalat, ja nõuab aru nende sõnade pärast, mida Issand, su Jumal, on pidanud kuulma. Tee siis palvet selle jäägi pärast, kes veel on olemas!”
\par 5 Kui kuningas Hiskija sulased tulid Jesaja juurde,
\par 6 siis ütles Jesaja neile: „Öelge oma isandale nii: Nõnda ütleb Issand: Ära karda sõnade pärast, mida sa oled kuulnud, millega Assuri kuninga poisid mind on teotanud!
\par 7 Vaata, ma panen temasse niisuguse vaimu, et kui ta kuuleb kuulujuttu, siis ta läheb tagasi oma maale ja ma lasen ta langeda mõõga läbi tema oma maal.”
\par 8 Ja ülemjoogikallaja pöördus tagasi ning leidis Assuri kuninga sõdivat Libna vastu; sest ta oli kuulnud, et too oli Laakisest edasi läinud.
\par 9 Aga kui Sanherib kuulis kõneldavat Tirhakast, Etioopia kuningast: „Ta on välja tulnud, et sõdida sinu vastu”, siis seda kuuldes läkitas ta käskjalad Hiskija juurde, öeldes:
\par 10 „Rääkige nõnda Hiskijaga, Juuda kuningaga, ja öelge: Ära lase ennast petta oma Jumalast, kelle peale sa loodad, arvates, et Jeruusalemma ei anta Assuri kuninga kätte!
\par 11 Vaata, sa oled ju kuulnud, kuidas Assuri kuningad on talitanud kõigi maadega, neid sootuks hävitades. Ja sind peaks päästetama?
\par 12 Kas rahvaste jumalad päästsid neid, keda mu isad hävitasid: Goosani, Haarani, Resefi ja Telassaris olevaid edenlasi?
\par 13 Kus on Hamati kuningas ja Arpadi kuningas, Sefarvaimi linna, Heena ja Ivva kuningas?”
\par 14 Kui Hiskija oli võtnud käskjalgade käest kirja ja oli seda lugenud, siis ta läks üles Issanda kotta; ja Hiskija laotas selle Issanda ette.
\par 15 Ja Hiskija palvetas Issanda poole, öeldes:
\par 16 „Vägede Issand, Iisraeli Jumal, kes istud keerubite peal! Sina üksi oled kõigi maa kuningriikide Jumal, sina oled teinud taeva ja maa.
\par 17 Pööra, Issand, oma kõrv ja kuule, ava, Issand, oma silm ja vaata! kuule kõiki Sanheribi sõnu, millega ta on läkitanud teotama elavat Jumalat!
\par 18 See on tõsi, Issand, et Assuri kuningad on rüüstanud kõiki rahvaid ja nende maid
\par 19 ja on heitnud tulle nende jumalad; kuid need ei olnud jumalad, vaid olid inimeste kätetöö, puu ja kivi, ja seepärast nad võisid neid hävitada.
\par 20 Aga nüüd, Issand, meie Jumal, päästa meid tema käest, et kõik maa kuningriigid tunneksid, et sina, Issand, oled ainus!”
\par 21 Siis Jesaja, Aamotsi poeg, läkitas Hiskijale ütlema: „Nõnda ütleb Issand, Iisraeli Jumal: Et sa mind oled palunud Assuri kuninga Sanheribi pärast,
\par 22 siis on sõna, mis Issand tema kohta kõneleb, niisugune: Neitsi, Siioni tütar, põlastab sind, pilkab sind; Jeruusalemma tütar vangutab su taga pead.
\par 23 Keda sa oled laimanud ja teotanud ja kelle vastu sa oled kõrgendanud häält? Kõrgele oled tõstnud oma silmad Iisraeli Püha vastu.
\par 24 Sa laimasid oma sulaste läbi Issandat ja ütlesid: „Ma tõusin oma vankrite hulgaga mägede harjadele, Liibanoni kaugemaisse kurudesse; ma raiusin maha ta kõrged seedrid, ta valitud küpressid, ma tungisin ta suurimasse kõrgusse, ta tihedaimasse metsa.
\par 25 Ma kaevasin kaevusid ja jõin vett, ja ma kuivatasin oma jalataldadega kõik Egiptuse jõed.”
\par 26 Kas sa pole kuulnud, et mina olen seda valmistanud ammusest ajast, kavatsenud muistseist päevist peale? Nüüd olen mina lasknud sündida, et sa võisid purustada kindlustatud linnu varisenud kivihunnikuiks,
\par 27 et nende elanikud olid jõuetud, täis hirmu ja häbi, olid nagu rohi väljal, haljad taimekesed, nagu hein katustel või idatuule ees kõrbev vili.
\par 28 Ma tean su istumist ja astumist, su minekut ja tulekut, ka su raevutsemist minu vastu.
\par 29 Aga et sa raevutsed mu vastu ja su ülbus on ulatunud mu kõrvu, siis ma panen konksu sulle ninna, suulised suhu ja viin sind tagasi sedasama teed, mida mööda sa tulidki.
\par 30 Ja see olgu sulle, Hiskija, märgiks: sel aastal tuleb süüa isekasvanud vilja ja teisel aastal järelkasvu, aga kolmandal aastal te külvate ja lõikate ning istutate viinamägesid ja sööte nende vilja!
\par 31 Ja Juuda soost pääsenute jääk juurdub taas alt ja kannab vilja pealt.
\par 32 Sest Jeruusalemmast tuleb välja jääk ja Siioni mäelt pääsenu. Seda teeb vägede Issanda püha viha.
\par 33 Seepärast ütleb Issand Assuri kuninga kohta nõnda: Sellesse linna ta ei tule ja ta ei ammu siia nooli; ta ei tule selle ette kilbiga ega kuhja selle vastu piiramisvalli.
\par 34 Sedasama teed, mida mööda ta tuli, läheb ta tagasi ja sellesse linna ta ei tule, ütleb Issand.
\par 35 Sest ma kaitsen seda linna, et seda päästa iseenese pärast ja oma sulase Taaveti pärast.”
\par 36 Ja Issanda ingel läks välja ning lõi Assuri leeris maha sada kaheksakümmend viis tuhat meest; ja kui hommikul vara üles tõusti, vaata, siis olid need kõik surnud.
\par 37 Siis Assuri kuningas Sanherib asus teele, läks tagasi koju ja jäi Niinevesse.
\par 38 Aga kord, kui ta kummardas oma jumala Nisroki templis, lõid ta pojad Adrammelek ja Sareser tema mõõgaga maha ning põgenesid ise Araratimaale. Ja tema poeg Eesar-Haddon sai tema asemel kuningaks.

\chapter{38}

\par 1 Neil päevil jäi Hiskija haigeks ja oli suremas. Ja prohvet Jesaja, Aamotsi poeg, tuli ta juurde ning ütles temale: „Nõnda ütleb Issand: Sea oma elumaja asjad korda, sest sa sured ega saa terveks!”
\par 2 Siis Hiskija pööras oma näo seina poole ja palus Issandat
\par 3 ning ütles: „Oh Issand, meenuta ometi, kuidas ma sinu ees olen elanud ustavuses ja siira südamega ja olen teinud, mis sinu silmis hea on!” Ja Hiskija nuttis kibedasti.
\par 4 Aga Jesajale tuli Issanda sõna, kes ütles:
\par 5 „Mine ja ütle Hiskijale: Nõnda ütleb Issand, su isa Taaveti Jumal: Ma olen kuulnud su palvet, ma olen näinud su silmavett. Vaata, ma lisan su elupäevadele viisteist aastat
\par 6 ja ma päästan sinu ja selle linna Assuri kuninga pihust ning kaitsen seda linna.
\par 7 Ja see olgu sulle märgiks Issandalt, et Issand teeb, nagu ta on öelnud:
\par 8 Vaata, ma lasen varjul, mis Aahase päikesekellal on laskunud koos päikesega, kümme pügalat tagasi minna.” Ja päike läkski päikesekellal tagasi need kümme pügalat, mis ta oli laskunud.
\par 9 Juuda kuninga Hiskija tänulaul pärast haigusest tervenemist:
\par 10 „Ma ütlesin iseeneses: Oma elupäevade poolel teel pean ma minema surmavalla väravaisse; minult on röövitud mu ülejäänud aastad.
\par 11 Ma mõtlesin: Ma ei saa näha Issandat, Issandat elavate maal, ei enam vaadata inimest maailma elanike keskel.
\par 12 Mu telklaager kistakse üles ja keeratakse minu käest rulli otsekui karjase telk; ma rullin nagu kangur oma elu kokku, mind lõigatakse lõime küljest. Enne kui päev jõuab õhtule, teed sa mulle lõpu.
\par 13 Ma ootasin hommikuni - otsekui murraks lõvi, nõnda murti kõik mu luud-liikmed. Enne kui päev jõuab õhtule, teed sa mulle lõpu.
\par 14 Nagu pääsuke või rästas, nõnda ma häälitsesin, kudrutasin otsekui tuvi. Mu silmad väsisid kõrgusesse vaadates: Issand, ma olen rõhutud, astu mulle appi!
\par 15 Mis oleks mul rääkida ja temale öelda? Tema ju tegi seda! Ma kõnnin tasakesi kõik oma aastad oma hinge kibeduse pärast.
\par 16 Issand, nõndamoodi jäädakse elama ja selles kõiges on mu vaimu elu: sina teed mind terveks ja hoiad mind elus.
\par 17 Vaata, mu suur kibedus muutus rahuks: sina kiindusid mu hinge hukatuse augus, sest sa heitsid kõik mu patud oma selja taha.
\par 18 Sest surmavald ei kiida sind, surm ei ülista sind, hauda vajunud ei looda su ustavuse peale.
\par 19 Elav, ainult elav kiidab sind, nõnda nagu minagi teen täna: isa kuulutab lastele sinu ustavust.
\par 20 Issand oli valmis mind päästma, seepärast mängigem mu keelpillidel kõik oma elupäevad Issanda kojas!”
\par 21 Jesaja oli öelnud: „Tooge viigimarjakakk ja asetage paisele, et ta saaks terveks!”
\par 22 Ja Hiskija oli küsinud: „Mis on märgiks, et ma võin minna Issanda kotta?”

\chapter{39}

\par 1 Sel ajal läkitas Paabeli kuningas Merodak-Baladan, Baladani poeg, saadikud kingitustega Hiskija juurde, kui ta oli kuulnud tema haigusest ning paranemisest.
\par 2 Ja Hiskija tundis neist rõõmu ja näitas neile oma varaaita, hõbedat ja kulda, kalleid rohte ja parimat õli ja kogu oma sõjariistade kambrit ning kõike, mis leidus ta varanduste hulgas; Hiskija ei jätnud neile midagi näitamata oma kojas ja kogu oma valitsemisalal.
\par 3 Aga prohvet Jesaja tuli kuningas Hiskija juurde ja küsis temalt: „Mida need mehed rääkisid ja kust nad su juurde tulid?„ Ja Hiskija vastas: ”Nad tulid mu juurde kaugelt maalt, Paabelist.”
\par 4 Siis ta küsis: „Mida nad su kojas nägid?„ Ja Hiskija vastas: ”Nad nägid kõike, mis mu kojas on; oma varanduste hulgast ei jätnud ma midagi neile näitamata.”
\par 5 Siis Jesaja ütles Hiskijale: „Kuule vägede Issanda sõna:
\par 6 Vaata, päevad tulevad, mil kõik, mis su kojas on ja mis su vanemad tänapäevani on kogunud, viiakse ära Paabelisse! Mitte midagi ei jää järele, ütleb Issand.
\par 7 Ja sinu poegadest, kes sinust põlvnevad, kes sulle sünnivad, võetakse mõned ja neist saavad õukonnateenrid Paabeli kuninga palees.”
\par 8 Siis Hiskija ütles Jesajale: „Issanda sõna, mida sa oled kõnelnud, on hea.„ Sest ta mõtles: ”Minu päevil on ju ometi rahu ning julgeolek.”

\chapter{40}

\par 1 Trööstige, trööstige minu rahvast, ütleb teie Jumal.
\par 2 Rääkige Jeruusalemma meele järgi ja kuulutage temale, et ta vaev on lõppenud, et ta süü on lepitatud, et ta on saanud Issanda käest kahekordselt kõigi oma pattude eest.
\par 3 Hüüdja hääl: „Valmistage kõrbes Issanda teed, tehke lagendikul maantee tasaseks meie Jumalale!
\par 4 Kõik orud ülendatagu ja kõik mäed ning künkad alandatagu: mis mätlik, saagu tasaseks, ja mis konarlik, siledaks maaks!
\par 5 Siis ilmub Issanda au ja kõik liha näeb seda üheskoos. Jah, Issanda suu on rääkinud.”
\par 6 Hääl ütleb: „Kuuluta!„ Ja teine kostab: „Mida ma pean kuulutama?” ”Kõik liha on nagu rohi ja kõik tema hiilgus nagu õieke väljal.
\par 7 Rohi kuivab ära, õieke närtsib, kui Issanda tuul puhub selle peale. Tõesti, rahvas on nagu Rohi.
\par 8 Rohi kuivab ära, õieke närtsib, aga meie Jumala sõna püsib igavesti.”
\par 9 Astu kõrgele mäele, Siioni sõnumiviija, tõsta valjusti häält, Jeruusalemma sõnumiviija, tõsta, ära karda! Ütle Juuda linnadele: „Vaata, teie Jumal!”
\par 10 Vaata, Issand Jumal tuleb jõuliselt ja tema käsivars valitseb. Vaata, temaga koos on te palk ja tema ees on teie töötasu.
\par 11 Otsekui karjane hoiab ta oma karja, kogub oma käsivarrega tallekesi ja kannab neid süles, talutab imetajaid lambaid.
\par 12 Kes on mõõtnud pihuga vee? Või määranud vaksaga taeva? Või kogunud vakaga maa põrmu? Või vaaginud margapuuga mäed ja kaalukaussidega künkad?
\par 13 Kes on juhatanud Issanda Vaimu või teda õpetanud tema nõuandjana?
\par 14 Kellega on ta nõu pidanud, et see annaks temale arusaamise ja õpetaks õiguse rada, õpetaks temale tarkust ja teeks teatavaks mõistuse tee?
\par 15 Vaata, rahvad on otsekui tilk ämbris ja neid peetakse kübemeks kaalukaussidel. Vaata, ta tõstab saared üles kui tolmukübemed.
\par 16 Ja Liibanonist ei jätku tulepuudeks ega selle loomadest põletusohvriks.
\par 17 Kõik rahvad on tema ees nagu eimiski, ta peab neid vähemaks kui olematus ja tühjus.
\par 18 Kellega te siis võrdlete Jumalat ja missuguse kujutise võiksite temast teha?
\par 19 Jumalakuju? Selle valab valaja. Ja kullassepp kuldab selle üle ning sepistab sellele hõbekette.
\par 20 Kes on vaene selleks andi andma, valib kõdunematu puu, otsib enesele osava meistri, et see valmistaks kõikumatu kuju.
\par 21 Kas te ei tea? Kas te ei ole kuulnud? Kas teile ei ole algusest peale kuulutatud? Kas te ei ole taibanud maailma alustest,
\par 22 et see on tema, kes istub maasõõri kohal, mille elanikud on nagu rohutirtsud; kes laotab taevad laiali kui loori ja venitab neid otsekui telki selles elamiseks;
\par 23 kes ei pane aukandjaid mikski ja teeb tühiseks maa kohtumõistjad?
\par 24 Vaevalt on nad istutatud, vaevalt on nad külvatud, vaevalt on nende tüvi maas juurdunud, kui ta juba puhub nende peale ja nad kuivavad ning tuulekeeris puhub nad ära kui kõrred.
\par 25 Kellega te siis võrdlete mind ja kelle sarnane ma olen? ütleb see, kes on Püha.
\par 26 Tõstke oma silmad ja vaadake kõrgusse: kes on loonud need seal? Tema, kes nende väe viib välja täiearvuliselt, kes nimetab neid kõiki nimepidi. Tema suure väe ja võimsa jõu tõttu ei puudu neist ainsatki.
\par 27 Mispärast ütled sina, Jaakob, ja räägid sina, Iisrael: „Minu tee on varjul Issanda eest ja mu õigus läheb mööda mu Jumalast?”
\par 28 Kas sa ei tea? Kas sa ei ole kuulnud? Igavene Jumal, Issand, on maailma äärte Looja; tema ei väsi ega tüdi, tema mõistus on uurimatu.
\par 29 Tema annab rammetule rammu ja jõuetule jõudu.
\par 30 Poisidki väsivad ja tüdivad, noored mehed komistavad ja kukuvad,
\par 31 aga kes ootavad Issandat, saavad uut rammu, need tõusevad tiibadega üles nagu kotkad: nad jooksevad ega tüdi, nad käivad ega väsi.

\chapter{41}

\par 1 Kuulake mind vaikides, saared, ja rahvad, koguge jõudu! Astuge ligi, rääkige siis, mingem üheskoos kohtu ette!
\par 2 Kes äratas ida poolt selle, kel kordaminek kannul käib? Kes andis tema kätte rahvad ja alistas kuningad, et ta mõõk need põrmustaks ja ta amb teeks nad hajuvaiks kõrteks?
\par 3 Tema ajab neid taga, kulgeb ohutult rada, mida ta jalad ei käi.
\par 4 Kes on seda teinud ja teostanud? See, kes algusest peale on kutsunud rahvapõlvi: Mina, Issand, olen esimene ja olen viimastegi juures seesama.
\par 5 Saared nägid ja kartsid, maailma ääred värisesid, nad astusid ligi ja tulid.
\par 6 Üks aitas teist ja ütles oma vennale: „Ole vahva!”
\par 7 Valaja julgustab kullasseppa, naastu vasardaja alasil tagujat, öeldes joodetu kohta: „See on hea!” ja kinnitades naeltega, et see ei kõiguks.
\par 8 Aga sina, Iisrael, mu sulane, Jaakob, kelle ma olen valinud, mu sõbra Aabrahami sugu,
\par 9 sina, kelle ma võtsin maailma äärtest ja kutsusin selle kaugemaist paigust, öeldes sulle: „Sina oled mu sulane, ma valisin sinu ega põlanud sind!” -
\par 10 ära karda, sest mina olen sinuga; ära vaata ümber, sest mina olen su Jumal: ma teen su tugevaks, ma aitan sind, ma toetan sind oma õiguse parema käega!
\par 11 Vaata, häbenema ja piinlikkust tundma peavad kõik, kes on sinule vihased; olematuks saavad ja hukkuvad mehed, kes riidlevad sinuga.
\par 12 Sa otsid, aga enam sa ei leia neid mehi, kes võitlevad sinu vastu; olematuks saavad ja lõpevad mehed, kes sõdivad sinu vastu.
\par 13 Sest mina olen Issand, su Jumal, kes kinnitab su paremat kätt, kes sulle ütleb: „Ära karda, mina aitan sind!”
\par 14 Ära karda, ussike Jaakob, väetike Iisrael, mina aitan sind, ütleb Issand ja su lunastaja, Iisraeli Püha.
\par 15 Vaata, ma panen sind pahmareeks, uueks ja paljuhambuliseks: sa peksad mägesid nagu reht ja jahvatad neid, teed künkad aganaiks;
\par 16 sa tuulad neid, tuul kannab neid, torm puistab need laiali. Aga sina ise ilutsed Issanda tõttu, kiitled Iisraeli Püha tõttu.
\par 17 Kui viletsad ja vaesed asjata otsivad vett ja nende keel kuivab janust, siis mina, Issand, kuulen neid, mina, Iisraeli Jumal, ei jäta neid maha.
\par 18 Ma panen jõed voolama viljatuil künkail ja allikad orgude põhjas; ma muudan järveks kõrbe ja veelätteiks põuase maa.
\par 19 Ma istutan kõrbesse seedreid, akaatsiaid, mürte ja õlipuid; ma panen lagendikule küpresse koos plataanide ja piiniatega,
\par 20 et nad näeksid ja teaksid, tähele paneksid ja ühtlasi mõistaksid, et seda on teinud Issanda käsi ja selle on loonud Iisraeli Püha.
\par 21 Tooge ette oma riiuasi, ütleb Issand, esitage oma tõendid, ütleb Jaakobi kuningas!
\par 22 Esitage ja kuulutage meile, mis on tulemas! Jutustage endisist asjust, missugused need olid, et saaksime südamesse võtta ja mõista nende tulemusi! Või laske meid kuulda tulevasi asju,
\par 23 kuulutage, mis tuleb pärastpoole, et me teaksime, kas te olete jumalad. Tehke ometi midagi, olgu head või kurja, et saaksime üheskoos imetleda ja näha.
\par 24 Vaata, teie ei olegi midagi ja teie töö on tühjast: kes teid eelistab, teeb jäledust.
\par 25 Mina äratasin põhja poolt ühe, ja ta tuli, päikesetõusu poolt mu nime appihüüdja, kes tallab valitsejaid nagu savi, otsekui sõtkuks potissepp saue.
\par 26 Kes on seda kuulutanud algusest peale, et oleksime seda teadnud, et oleksime saanud öelda: „Temal oli õigus!”? Ükski ei teatanud, ükski ei kuulutanud, ükski ei kuulnud teie sõnu.
\par 27 Vaata, mina olin esimene, kes kuulutas Siionile: „Näe, siin nad on!” ja andis Jeruusalemmale hea sõnumi tooja.
\par 28 Ja kui ma vaatan, siis ei olegi kedagi, ja neist ei ole nõuandjat, et saaksin neilt küsida ja nad annaksid vastuse.
\par 29 Vaata, nad kõik on tühised, nende teod on eimiski, nende valatud kujud on tühipaljas tuul.

\chapter{42}

\par 1 Vaata, see on mu sulane, kellesse ma olen kiindunud, mu valitu, kellest mu hingel on hea meel. Ma olen pannud oma Vaimu tema peale, tema toob rahvaile õiguse.
\par 2 Tema ei kisenda ega karju, tema häält ei ole kuulda tänavail.
\par 3 Rudjutud pilliroogu ei murra ta katki ja hõõguvat tahti ei kustuta ta ära, ta levitab ustavalt õigust.
\par 4 Tema ei nõrke ega murdu, kuni ta maa peal on rajanud õiguse ja saared ootavad tema õpetust.
\par 5 Nõnda ütleb Jumal, Issand, kes on loonud taeva ja on selle võlvinud, kes on laotanud maa ja sellest võrsuva, kes on andnud hingeõhu selle rahvale ja vaimu neile, kes seal peal käivad:
\par 6 Mina, Issand, olen õigusega sind kutsunud ja kinnitan su kätt; ma kaitsen sind ja panen sinu rahvale lepinguks, paganaile valguseks,
\par 7 avama pimedate silmi, vabastama vange vangistusest, pimeduses istujaid vangikojast.
\par 8 Mina olen Issand, see on mu nimi, ja mina ei anna oma au teisele ega oma kiidetavust nikerdatud kujudele.
\par 9 Vaata, eelmised sündmused on aset leidnud ja ma kuulutan uusi; enne kui need toimuma hakkavad, teen ma need teile teatavaks.
\par 10 Laulge Issandale uut laulu tema kiituseks maailma äärest! Ilutsegu meri ja mis seda täidab, saared ja nende elanikud!
\par 11 Tõstku häält kõrb ja selle linnad, külad, kus Keedar elab! Kaljude elanikud hõisaku, hüüdku valjusti mägede tippudelt!
\par 12 Andku nad Issandale au ja kuulutagu saartel tema kiidetavust!
\par 13 Issand läheb välja nagu kangelane, ta õhutab võitlusindu otsekui sõjamees; ta tõstab sõjakisa, hüüab valjusti, astub võidukalt oma vaenlaste vastu.
\par 14 Ma olen kaua vaikinud, olnud tegevuseta, hoidnud ennast tagasi. Aga nüüd ma oigan otsekui sünnitaja, hingeldan ja ahmin õhku ühtaegu.
\par 15 Ma laastan mäed ja künkad ning kuivatan kõik nende rohu; ma muudan jõed koolmeiks ja teen järved tahedaks.
\par 16 Ma juhin pimedaid teel, mida nad ei tunne, ma lasen neid käia tundmatuid radu; ma muudan pimeduse nende ees valguseks ja konara tasaseks. Need on asjad, mis ma teen ega jäta tegemata.
\par 17 Aga taganema ja häbenema peavad need, kes loodavad nikerdatud kujude peale, kes ütlevad valatud kujudele: „Teie olete meie jumalad!”
\par 18 Kurdid, kuulge, ja pimedad, tõstke oma pilk üles, et te näeksite!
\par 19 Kes on pime, kui mitte mu sulane, ja kurt nagu mu käskjalg, kelle ma läkitan? Kes on nõnda pime nagu mu usaldusmees ja nõnda pime nagu Issanda sulane?
\par 20 Sa oled näinud palju, aga ei ole tähele pannud; kõrvad on küll lahti, aga sa ei kuule.
\par 21 Issandale meeldis oma õigluse pärast teha Seadus suureks ja võimsaks.
\par 22 Ja see on paljaksriisutud ja rüüstatud rahvas: nad kõik on pandud aukudesse ja peidetud vangikodadesse; nad on jäänud saagiks ja päästjat ei ole, rüüsteks, ja ükski ei ütle: „Anna tagasi!”
\par 23 Kes teist võtab seda kuulda, paneb tähele ja kuuleb tuleviku tarvis?
\par 24 Kes andis Jaakobi rüüstata ja Iisraeli riisujate kätte? Kas mitte Issand, kelle vastu me pattu tegime, kelle teedel ei tahetud käia ja kelle Seadust ei tahetud kuulda?
\par 25 Seepärast ta valas tema peale oma vihalõõma ja sõjajõu: see põletas teda igalt poolt, kuid ta ei saanud targaks; kõrvetas teda, kuid ta ei võtnud seda südamesse.

\chapter{43}

\par 1 Aga nõnda ütleb nüüd Issand, kes sind, Jaakob, on loonud, ja kes sind, Iisrael, on kujundanud: Ära karda, sest ma olen sind lunastanud, ma olen sind nimepidi kutsunud, sa oled minu päralt!
\par 2 Kui sa lähed läbi vee, siis olen mina sinuga, ja kui sa lähed läbi jõgede, siis ei uputa need sind; kui sa käid tules, siis sa ei põle ja leek ei kõrveta sind.
\par 3 Sest mina olen Issand, su Jumal, Iisraeli Püha, su Päästja; su lunahinnaks ma annan Egiptuse, sinu asemel Etioopia ja Seba.
\par 4 Et sa mu silmis oled kallis ja auline ja et ma sind armastan, siis annan ma sinu asemel inimesi ja su hinge eest rahvaid.
\par 5 Ära karda, sest mina olen sinuga, ma toon su soo päikesetõusu poolt ja ma kogun sind päikeseloojaku poolt.
\par 6 Ma ütlen põhjamaale: Anna! ja lõunamaale: Ära keela! Too mu pojad kaugelt ja mu tütred maailma äärest,
\par 7 kõik, keda nimetatakse minu nimega ja keda ma oma auks olen loonud, kujundanud ning valmis teinud!
\par 8 Too välja pime rahvas, kellel siiski on silmad, ja kurdid, kellel siiski on kõrvad!
\par 9 Kõik paganad on kogunenud ühte ja rahvad on kokku tulnud. Kes neist saaks jutustada niisuguseid asju ja kuulutada meile endisi sündmusi? Toogu nad oma tunnistajad ja õigustagu endid, las need kuulda ning öelda: „See on tõsi!”
\par 10 Teie olete minu tunnistajad, ütleb Issand, ja minu sulane, kelle ma olen valinud, et te teaksite, usuksite minusse ja mõistaksite, et mina olen see: enne mind ei ole olnud ühtegi jumalat ega tule ühtegi pärast mind.
\par 11 Mina, mina olen Issand, ei ole muud päästjat kui mina.
\par 12 Mina olen teatavaks teinud, päästnud ja kuulutanud, aga mitte mõni võõras teie hulgas; teie olete minu tunnistajad, ütleb Issand, et mina olen Jumal.
\par 13 Ka edaspidi olen ma seesama ja ükski ei päästa minu käest. Mina teen, ja kes saaks seda takistada?
\par 14 Nõnda ütleb Issand, teie lunastaja, Iisraeli Püha: teie pärast ma läkitan Paabelisse, ja toon alla kõik põgenikud, ka kaldealased, käratsedes laevadesse.
\par 15 Mina, Issand, olen teie Püha, mina, Iisraeli Looja, olen teie kuningas.
\par 16 Nõnda ütleb Issand, kes tegi tee merre ja jalgraja võimsasse vette,
\par 17 kes tõi välja vankrid ja hobused, väe ja vägevad mehed; need lamavad üheskoos ega tõuse enam, nad on vaibunud, kustunud otsekui taht.
\par 18 Ärge tuletage meelde endisi asju ja ärge pange tähele, mis muiste on sündinud.
\par 19 Vaata, mina teen hoopis uut: see juba tärkab, kas te ei märka? Ma teen kõrbessegi tee, tühjale maale jõed.
\par 20 Mind austavad metsloomad, ðaakalid ja jaanalinnud, sest ma annan kõrbes vett, jõgesid tühjal maal, et joota oma valitud rahvast.
\par 21 Rahvas, kelle ma enesele olen valmistanud, peab jutustama minu kiidetavusest.
\par 22 Aga ei ole sina, Jaakob, mind kutsunud, ei ole sina, Iisrael, minu pärast vaeva näinud.
\par 23 Sina ei ole mulle toonud oma põletusohvrite lambaid ega ole mind austanud oma tapaohvritega; mina ei ole sind pannud orjama roaohvriga ega ole vaevanud viirukiga.
\par 24 Sina ei ole mulle ostnud raha eest kalmuseid ega ole mind küllastanud oma tapaohvrite rasvaga, vaid sa oled mulle tüli teinud oma pattudega, oled mind vaevanud oma süütegudega.
\par 25 Mina, mina olen see, kes kustutab su üleastumised iseenese pärast ega pea meeles su patte.
\par 26 Tuleta mulle meelde, käigem isekeskis kohut, räägi sina, et sa võiksid õigeks saada.
\par 27 Su esiisa tegi pattu ja su eestkõnelejad astusid üles mu vastu.
\par 28 Seepärast ma teotasin pühamu vürste, jätsin Jaakobi hävima ja Iisraeli laimualuseks.

\chapter{44}

\par 1 Nüüd aga kuule, Jaakob, mu sulane, ja Iisrael, mu valitu!
\par 2 Nõnda ütleb Issand, sinu Looja, emaihus valmistaja, sinu aitaja: Ära karda, mu sulane Jaakob, ja Jesurun, mu valitu!
\par 3 Sest ma valan janusele vett ja kuivale voogusid; ma valan sinu soo peale oma Vaimu ja su järglaste peale oma õnnistuse,
\par 4 et nad võrsuksid kui roogude vahel, otsekui remmelgad veeojade ääres.
\par 5 Üks ütleb: „Mina kuulun Issandale„, teine nimetab ennast Jaakobi nimega, kolmas kirjutab oma käe peale: ”Issanda oma” ja võtab enesele aunimeks Iisrael.
\par 6 Nõnda ütleb Issand, Iisraeli kuningas, ja tema lunastaja, vägede Issand: Mina olen esimene ja viimane, ja ei ole muud Jumalat kui mina.
\par 7 Kes on minu sarnane? See hüüdku, jutustagu sellest ja kandku mulle ette! Sest ajast kui ma seadsin muistse rahva ja tulevased sündmused, jutustagu nad neile sellest, mis on tulemas.
\par 8 Ärge värisege ja ärge kartke! Kas ma ei ole sulle juba ammu kuulutanud ja teatavaks teinud? Teie olete mu tunnistajad. Kas on muud Jumalat kui mina? Ei, muud kaljut ei ole, mina küll ei tea!
\par 9 Nikerdatud kujude valmistajad on kõik tühised ja nende lemmikud ei suuda aidata; nende tunnistajad ise ei näe ega tea, et nad peaksid häbenema.
\par 10 Kes küll peakski valmistama jumala ja valama kuju, millest ei ole kasu?
\par 11 Vaata, kõik selle austajad jäävad häbisse ja sepistajad ise on ainult inimesed; las nad kõik tulevad kokku, astuvad ette: nad hakkavad üheskoos värisema ja häbenema!
\par 12 Sepp taob rauast tööriista ja töötleb ääsil, vormib vasaratega ja taob oma käsivarre jõul; temalgi tuleb nälg ja ta jõud kaob; kui ta vett ei joo, siis ta väsib.
\par 13 Puusepp pingutab mõõdunööri, tähistab kriidiga, voolib kaabitsatega, märgib sirkliga ja valmistab mehesarnase kuju, ilusa inimese taolise, et see asuks kojas.
\par 14 Ta raiub enesele seedreid; ta valib raudtamme või tamme ja laseb metsapuude hulgas oma tarbeks tugevaks kasvada; ta istutab loorberipuu, vihm kasvatab selle suureks
\par 15 ja sellest saab inimesele kütet: ta võtab seda enese soojendamiseks, samuti süütab ta selle, et leiba küpsetada; kuid ta valmistab sellest ka jumala ja kummardab seda, ta teeb sellest nikerdatud kuju ja põlvitab selle ette.
\par 16 Osa sellest ta põletab tules; ta sööb süte pealt liha: küpsetab prae ja sööb kõhu täis, soojendab siis ennast ja ütleb: „Küll on hea, mul on soe, ma tunnen tulepaistest mõnu!”
\par 17 Aga selle jäänusest teeb ta jumala, oma nikerdatud kuju, põlvitab selle ette ja kummardab, palvetab selle poole ning ütleb: „Päästa mind, sest sina oled mu jumal!”
\par 18 Nad ei tea ega taipa midagi, sest nende silmad on nägemiseks ja südamed mõistmiseks suletud.
\par 19 Ükski ei võta südamesse, ei ole tal teadmist ega taipu, et ta mõtleks: osa sellest olen põletanud tules, samuti olen selle süte peal küpsetanud leiba, praadinud liha ja söönud, ja nüüd peaksin ma selle jäänusest valmistama jäleduse, põlvitama puutüki ette!
\par 20 Kes tuhka taga ajab, seda eksitab petetud süda, see ei suuda päästa oma hinge ega mõtle: Kas ei ole viimaks pettus mu paremas käes?
\par 21 Tuleta seda meelde, Jaakob, ja sina, Iisrael, sest sa oled mu sulane: mina olen su valmistanud, sa oled mu sulane: Iisrael, ma ei unusta sind!
\par 22 Ma kaotan su üleastumised nagu pilved ja su patud nagu pilvituse. Pöördu minu poole, sest ma lunastan sinu!
\par 23 Hõisake, taevad, et Issand seda teeb, hüüdke, maa sügavused, rõkatage rõõmust, mäed, metsad ja kõik metsapuud, et Issand lunastab Jaakobi ja ilmutab Iisraelis oma au!
\par 24 Nõnda ütleb Issand, su lunastaja, kes sind emaihus on valmistanud: Mina olen Issand, kes kõik teeb, kes üksinda võlvis taeva, laotas maa - kes oli koos minuga? -
\par 25 kes teeb tühjaks valetajate tunnustähed ja muudab narrideks lausujad, kes tõrjub tagasi targad ja pöörab nende teadmised jõleduseks,
\par 26 aga kes kinnitab oma sulase sõna ja viib täide oma käskjalgade nõu, kes ütleb Jeruusalemmale: „Sinus hakatakse elama!„ ja Juuda linnadele: ”Teid taastatakse, ma ehitan üles teie varemed!”,
\par 27 kes ütleb sügavikule: „Kuiva! Ma lasen su jõed kuivada!”,
\par 28 kes ütleb Kooresele: „Mu karjane!„ Tema viib täide kõik, mida ma tahan, ja ütleb Jeruusalemmale: ”Sind ehitatakse üles ja rajatakse tempel!”

\chapter{45}

\par 1 Nõnda ütleb Issand oma võitule Kooresele, kelle paremast käest ma olen kinni haaranud, et alistada tema ees rahvaid ja päästa lahti kuningate vöösid, et avada tema ees uksi, et väravad ei jääks suletuks:
\par 2 Mina käin su ees ja teen kingud tasaseks; ma purustan vaskuksed ja raiun katki raudriivid.
\par 3 Ma annan sulle pimeduse varjust varandusi ja peidupaikadest aardeid, selleks et sa teaksid, et see olen mina, Issand, Iisraeli Jumal, kes sind on kutsunud nimepidi.
\par 4 Oma sulase Jaakobi ja oma valitu Iisraeli pärast ma kutsusin sind nimepidi ja andsin sulle aunime, ehkki sa mind ei tundnud.
\par 5 Mina olen Issand ja kedagi teist ei ole, ei ole ühtegi jumalat peale minu; mina vöötasin sind, ehkki sa mind ei tundnud,
\par 6 et teataks päikesetõusu ja -loojaku pool, et ei ole ühtegi peale minu; mina olen Issand ja kedagi teist ei ole.
\par 7 Mina valmistan valguse ja loon pimeduse, mina annan õnne ja toon õnnetuse, mina, Issand, teen seda kõike.
\par 8 Kastke, taevad, ülalt ja pilved, pange voolama õiglus! Avanegu maa, et idaneks õnnistus ja ühtlasi võrsuks õiglus! Mina, Issand, olen selle loonud.
\par 9 Häda sellele, kes riidleb oma valmistajaga - kild teiste savikildude seas. Kas ütleb savi oma vormijale: „Mis sa teed?„ ja su töö: ”Tal ei ole käsi!”?
\par 10 Häda sellele, kes ütleb isale: „Miks sa sigitad?„ või naisele: ”Miks oled sünnitusvaludes?”!
\par 11 Nõnda ütleb Issand, Iisraeli Püha ja tema Looja: Kas te küsitlete mind mu laste kohta ja annate mulle käsu mu kätetöö jaoks?
\par 12 Mina tegin maa ja lõin inimesed selle peale; minu käed võlvisid taeva ja ma käsutasin kõiki tema vägesid.
\par 13 Ma äratasin tema õiguse läbi ja ma teen kõik ta teed tasaseks; tema ehitab mu linna ja vabastab mu vangid hinnata ja meeleheata, ütleb vägede Issand.
\par 14 Nõnda ütleb Issand: Egiptuse töövaev ja Etioopia kaubitsemine ja seebalased, pikakasvulised mehed, tulevad su juurde ja saavad su omaks; nad käivad su järel; tulevad ahelais ja kummardavad sind, palvetavad su poole: „Ainult sinu juures on Jumal ja teist ei ole, ei ole ühtegi muud Jumalat!”
\par 15 Tõesti, sina oled ennast varjav Jumal, Iisraeli Jumal, Päästja!
\par 16 Häbenevad ja piinlikkust tunnevad kõik üheskoos, häbiga lähevad ebajumalate sepad.
\par 17 Aga Issand päästab Iisraeli igavese päästega: teil ei ole vaja häbeneda ega tunda piinlikkust, iialgi mitte.
\par 18 Sest nõnda ütleb Issand, taevaste Looja, tema, kes on Jumal, maa vormija ning valmistaja; tema, selle rajaja, ei ole loonud seda tühjaks, vaid on valmistanud selle, et seal elataks: Mina olen Issand ja kedagi teist ei ole.
\par 19 Ei ole ma rääkinud salajas, maa pimedas paigas, ega ole ma öelnud Jaakobi soole: Otsige mind ilmaaegu. Mina, Issand, kõnelen õigust, kuulutan, mis on tõsi.
\par 20 Kogunege ja tulge, liginege üheskoos, rahvaste pääsenud! Ei ole arusaamist neil, kes kannavad puukujusid ja palvetavad jumala poole, kes ei suuda päästa.
\par 21 Kuulutage ja tooge esile, pidagu nad isekeskis nõu: kes on seda kuulutanud muistsest peale, teatavaks teinud ammusest ajast? Kas mitte mina, Issand? Ei ole ju muud jumalat peale minu, kes oleks õiglane Jumal ja Päästja, mitte ühtegi peale minu.
\par 22 Pöörduge minu poole ja laske endid päästa, kõik maailma ääred, sest mina olen Jumal ja kedagi teist ei ole!
\par 23 Ma olen iseeneses vandunud, mu suust on välja tulnud tõde, tagasivõetamatu sõna: Minu ees peab nõtkuma iga põlv, kõik keeled peavad andma mulle vande.
\par 24 Üksnes Issandas - nõnda öeldakse minu kohta - on õigus ja jõud, tema juurde tulevad häbenedes kõik, kes on olnud tema peale vihased.
\par 25 Issandas saab õigeks ja auliseks kogu Iisraeli sugu.

\chapter{46}

\par 1 Põlvili langeb Beel, küüru tõmbub Nebo, nende kujud pannakse veo- ja kariloomade selga; need, mida te kandsite, tõstetakse koormaks väsinud loomadele.
\par 2 Nad tõmbuvad küüru, langevad üheskoos põlvili, nad ei suuda päästa koormat, vaid lähevad ise vangi.
\par 3 Kuulge mind, Jaakobi sugu, ja kogu Iisraeli soo jääk, keda mul on tulnud kanda emaihust peale, sülle võtta emaüsast alates.
\par 4 Teie vana eani ma olen seesama ja teie hallide juusteni ma kannan teid; mina olen teinud ja mina tõstan üles, mina kannan ja päästan.
\par 5 Kellega te mind võrdlete ja samastate, või keda te seate mu kõrvale, et oleksime sarnased?
\par 6 Need, kes puistavad kukrust kulda ja vaevad vaega hõbedat, palkavad kullassepa ja see valmistab jumala, mille ette nad põlvitavad ja mida nad kummardavad.
\par 7 Nad tõstavad selle õlale, kannavad ja asetavad paigale, ja see seisab ega liigu oma kohalt. Kui siis keegi seda appi hüüab, ei vasta see ega päästa teda ta ahastusest.
\par 8 Pidage seda meeles ja jääge kindlaks, võtke südamesse, te üleastujad!
\par 9 Tuletage meelde endisi asju muistsest ajast, sest mina olen Jumal ja kedagi teist ei ole, mina olen Jumal ja ükski ei ole minu sarnane;
\par 10 kes algusest alates kuulutab lõppu ja aegsasti ette, mida veel ei ole tehtud; kes ütleb: Minu nõu läheb korda ja ma teen kõik, mis ma tahan;
\par 11 kes kutsub ida poolt röövlinnu, kaugelt maalt mehe, kes teostab mu nõu. Mida ma olen rääkinud, seda ma lasen ka sündida, nagu ma olen kavatsenud, nõnda ma teen.
\par 12 Kuulge mind, te arad, kes te olete kaugel õigusest!
\par 13 Ma toon ligidale oma õiguse, see ei ole kaugel, ja mu pääste ei viibi; ma annan Siionile pääste, Iisraelile oma toreduse.

\chapter{47}

\par 1 Astu alla ja istu põrmu, neitsi, Paabeli tütar! istu maha aujärjeta, kaldealaste tütar! Sest enam ei nimetata sind õrnaks ja hellitatuks.
\par 2 Võta käsikivi ja jahvata jahu; pane loor ära, tõsta seelikuserv üles, paljasta sääred, kahla läbi jõgede!
\par 3 Su häbe paljastatakse, sinugi häbistust nähakse. Mina tasun kätte ega lase manguda kedagi, ütleb meie lunastaja,
\par 4 kelle nimi on vägede Issand, Iisraeli Püha.
\par 5 Istu vaikselt ja mine pimedusse, kaldealaste tütar, sest enam ei nimetata sind kuningriikide käskijannaks!
\par 6 Mina vihastasin oma rahva peale, teotasin oma pärisosa ja andsin nad su kätte; sina ei osutanud neile halastust, raugalegi panid peale oma üliränga ikke
\par 7 ja ütlesid: „Ma olen igavesti käskijanna.” Sa ei võtnud neid asjaolusid südamesse ega mõelnud, kuidas see lõpeb.
\par 8 Aga kuule nüüd seda, sa himur, kes elad muretult, üteldes südames: „Mina, ja ei keegi muu! Mina ei ela lesena ega tunne lastetust!”
\par 9 Ent äkki ühel päeval on need mõlemad sul käes: lastetus ja lesepõlv. Need tabavad sind täiel jõul, hoolimata su paljudest nõidustest, hoolimata su suurest lausumisvõimest.
\par 10 Sa olid oma kurjuses julge ja ütlesid: „Ükski ei näe mind!„ Su tarkus ja su teadmised eksitasid sind ja sa ütlesid südames: ”Mina, ja ei keegi muu!”
\par 11 Seepärast tabab sind õnnetus, mida sa ei oska eemale manada; su peale langeb hukatus, mida sa ei suuda tõrjuda; äkitselt tuleb su üle torm, mida sa ei aimagi.
\par 12 Astu ometi ette oma lausumiste ja oma paljude nõidustega, millega sa ennast oled väsitanud noorpõlvest peale; võib-olla sa suudad abi tuua, võib-olla hirmutada!
\par 13 Sa oled ennast väsitanud oma paljude nõuandjatega; astugu nad ometi ette ja päästku sind, taevatundjad, tähtede seletajad, kes kuulutavad iga noorkuu ajal, mis sulle juhtub.
\par 14 Vaata, need on otsekui kõrred, tuli põletab need; nad ei suuda päästa oma hinge leegi võimusest. See ei ole söelõõm soojenduse tarvis ega tulepaistus, mille ees istuda.
\par 15 Nõnda käib su nõidujate käsi, kelle pärast sa oled vaeva näinud noorpõlvest peale: igaüks vaarub omale poole, ükski ei päästa sind.

\chapter{48}

\par 1 Kuulge seda, Jaakobi sugu, keda nimetatakse Iisraeli nimega, kes olete võrsunud Juuda seemnest, kes vannute Issanda nime juures ja kuulutate Iisraeli Jumalat - ometi mitte tões ja õiguses,
\par 2 ehk küll nad nimetavad endid püha linna järgi ja toetuvad Iisraeli Jumalale, kelle nimi on vägede Issand -,
\par 3 endisi asju olen ma teatanud ammu, need on lähtunud minu suust ja mina olen neid kuulutanud. Äkitselt tegin ma need teoks ja need sündisid.
\par 4 Ma teadsin, et sa oled kangekaelne, et su kuklakõõlus on raudne ja otsaesine vaskne.
\par 5 Seepärast ma teatasin sulle ammu, kuulutasin sulle, enne kui see sündis, et sa ei ütleks: „Seda tegi mu ebajumal, mu nikerdatud ja valatud kuju käskis nõnda.”
\par 6 Sa oled kuulnud, nüüd vaata seda kõike! Ja teie, kas te ei tahaks seda tunnustada? Nüüdsest peale ma kuulutan sulle uusi asju ja saladusi, mida sa ei tea.
\par 7 Need on loodud äsja, ja mitte ammu, ja enne tänast päeva ei ole sa neist kuulnud, et sa ei saaks öelda: „Vaata, ma teadsin seda!”
\par 8 Ei ole sa kuulnud ega teadnud, ka ei ole su kõrv varem selleks lahti olnud; sest ma tean, et sa tõepoolest oled truuduseta ja et juba emaihust alates hüütakse sind üleastujaks.
\par 9 Oma nime pärast olen ma pikameelne ja oma kuulsuse pärast hoian ma ennast tagasi, et sind mitte hävitada.
\par 10 Vaata, ma olen sind sulatanud, ometi mitte kui hõbedat: ma olen sind proovinud viletsuse ahjus.
\par 11 Iseenese pärast, iseenese pärast teen ma seda, sest muidu ju teotataks mind. Oma au ma teistele ei anna!
\par 12 Kuule mind, Jaakob, ja Iisrael, mu kutsutu! See olen mina! Mina olen esimene, mina olen ka viimne.
\par 13 Minu käsi on rajanud maale aluse ja mu parem käsi on laotanud taeva; kui ma neid hüüan, siis nad astuvad üheskoos ette.
\par 14 Kogunege kõik ja kuulge! Kes neist on seda kuulutanud? See, keda Issand armastab, näitab oma tahet Paabelis ja oma käsivart kaldealaste seas.
\par 15 Mina, mina olen rääkinud, mina olen ta kutsunud, olen tema toonud ja tema teekond läheb korda.
\par 16 Tulge mu juurde, kuulge seda! Mina ei ole algusest peale rääkinud salajas, juba kui see sündis, olin ma seal. Ja nüüd on Issand Jumal läkitanud minu ja oma Vaimu.
\par 17 Nõnda ütleb Issand, su lunastaja, Iisraeli Püha: Mina olen Issand, su Jumal, kes sulle õpetab, mis on kasulik, kes sind juhatab teele, mida sa pead käima.
\par 18 Kui sa ometi oleksid tähele pannud mu käske! Siis oleks su rahu olnud kui jõgi ja su õigus oleks olnud otsekui mere lained.
\par 19 Su sugu oleks olnud nagu liiva ja su ihuvilja selle sõmerate sarnaselt: ei oleks hävitatud ega kaotatud tema nime mu palge eest.
\par 20 Minge välja Paabelist, põgenege Kaldeast, rõõmuhüüdega andke teada, kuulutage seda, levitage maailma ääreni, öelge: „Issand on lunastanud oma sulase Jaakobi.
\par 21 Nad ei saanud tunda janu, kui ta viis neid läbi kõrbete: ta laskis neile kaljust vett voolata, ta lõhestas kalju ja vesi vulises.”
\par 22 Õelatel ei ole rahu, ütleb Issand.

\chapter{49}

\par 1 Kuulge mind, saared, ja pange tähele, kauged rahvad! Issand on mind kutsunud emaihust, alates mu emaüsast on ta nimetanud mu nime.
\par 2 Ta tegi mu suu vaheda mõõga sarnaseks, peitis mind oma käe varju alla; ta tegi mind teravaks nooleks, talletas mind oma nooletupes.
\par 3 Ja ta ütles mulle: Sina, Iisrael, oled mu sulane, kelle läbi ma ilmutan oma au.
\par 4 Aga mina ütlesin: Ma olen asjata vaeva näinud, kulutanud oma jõudu kasuta ja tühja. Ometi on mu õigus Issanda käes ja mu töötasu on mu Jumala juures.
\par 5 Ja nüüd ütleb Issand, kes mind emaihust alates on valmistanud enesele sulaseks, et tuua Jaakob tagasi tema juurde ja koguda Iisrael tema juurde - sest ma olen Issanda silmis austatud ja mu Jumal on mu tugevus -,
\par 6 ta ütleb: Sellest on vähe, et sa mu sulasena taastad Jaakobi suguharud ja tood tagasi Iisraeli jäägi: ma panen sind paganaile valguseks, et mu pääste ulatuks ilmamaa ääreni.
\par 7 Nõnda ütleb Issand, Iisraeli lunastaja, tema Püha, täiesti põlatule, rahvaste poolt jälestatule, valitsejate sulasele: Kuningad näevad seda ja tõusevad püsti, ja vürstid kummardavad Issanda pärast, kes on ustav, Iisraeli Püha pärast, kes sind on valinud.
\par 8 Nõnda ütleb Issand: Ma olen sind kuulnud hea meele ajal ja aidanud päästepäeval; ma olen sind hoidnud ja pannud rahvale seaduseks, taastama maad, jagama laastatud pärisosi,
\par 9 ütlema vangistatuile: „Minge välja!„, pimeduses olijaile: ”Tulge valguse kätte!” Nad saavad teede ääres karja hoida ja neil on karjamaa kõigil küngastel.
\par 10 Ei ole neil nälga ega janu, neid ei pista palavus ega päike, sest nende peale halastaja juhib neid ja talutab nad veeallikate juurde.
\par 11 Ma teen kõik oma mäed teeks ja mu maanteed on kõrged.
\par 12 Vaata, nad tulevad kaugelt. Ennäe, ühed põhja ja lääne poolt, teised Siinimimaalt.
\par 13 Hõisake, taevad, ja ilutse, maa, mäed, rõkatage rõõmust, sest Issand trööstib oma rahvast ja halastab oma viletsate peale!
\par 14 Aga Siion ütleb: „Issand on mu maha jätnud, Jumal on mu unustanud.”
\par 15 Kas naine unustab oma lapsukese ega halasta oma ihuvilja peale? Ja kui nad ka unustaksid, ei unusta mina sind mitte.
\par 16 Vaata, ma olen sind märkinud oma peopesadesse, su müürid on alati mu silme ees.
\par 17 Su ehitajad tulevad tõtates; kes sind lammutasid ja laastasid, need lähevad ära su kallalt.
\par 18 Tõsta oma silmad üles ja vaata ringi: nad kõik kogunevad, tulevad su juurde. Nii tõesti kui ma elan, ütleb Issand, ehid sa ennast nende kõigiga nagu ehtega ja seod selle endale vööle nagu mõrsja.
\par 19 Jah, su laastatud ja paljaks tehtud paigad ning su hävitatud maa - tõesti jääd sa nüüd elanikele kitsaks, ja kaugele põgenevad su laastajad.
\par 20 Veel sinu kuuldes saavad öelda su lastetuseaja lapsed: „Paik on mulle kitsas, tee ruumi, et ma saaksin elada!”
\par 21 Siis sa mõtled oma südames: Kes on need mulle sünnitanud? Ma olin lasteta ja viljatu, vangi viidud ja ära aetud. Kes on need kasvatanud? Vaata, ma olin üksi järele jäänud. Kus olid siis need?
\par 22 Nõnda ütleb Issand Jumal: Vaata, ma tõstan oma käe paganate poole ja püstitan oma lipu rahvaste suunas: siis nad toovad su poegi süles ja kannavad su tütreid õlgadel.
\par 23 Ja sul on lastehoidjaiks kuningad, ammedeks nende emandad; nad kummardavad silmili maha su ette ja lakuvad su jalgadelt põrmu. Siis sa tead, et mina olen Issand, ei jää häbisse need, kes mind ootavad.
\par 24 Kas sangari käest saab võtta tema saagi või kiskuda vägeva käest tema vange?
\par 25 Tõesti, nõnda ütleb Issand: Küllap võetakse sangari vangid ja pääseb vägeva saak: mina riidlen sellega, kes riidleb sinuga, ja mina päästan su lapsed.
\par 26 Ma söödan su rõhujaid nende eneste lihaga ja nad joobuvad omaenese verest otsekui värskest veinist. Siis saab teada kõik liha, et mina, Issand, olen su päästja, Jaakobi Vägev on su lunastaja.

\chapter{50}

\par 1 Nõnda ütleb Issand: Kus on teie ema lahutuskiri, millega ma tema olen ära saatnud? Või kellele oma võlausaldajaist ma olen teid müünud? Vaata, te olete müüdud oma süütegude pärast, ja teie üleastumiste pärast on teie ema ära saadetud.
\par 2 Miks ei olnud kedagi, kui ma tulin, ei vastanud keegi, kui ma hüüdsin? Ons mu käsi liiga lühike lunastamiseks või ei ole mul jõudu päästmiseks? Vaata, oma sõitlusega ma kuivatan mere, teen jõed kõrbeks, vee puudusel hakkavad kalad haisema ja nad surevad janusse.
\par 3 Ma riietan taevad pimedusega ja annan neile kotiriide katteks.
\par 4 Issand Jumal on mulle andnud õpetatud keele, et ma oskaksin vastata väsinule, virgutada teda sõnaga. Ta äratab igal hommikul, ta äratab mu kõrva, et ma kuuleksin õpilaste kombel.
\par 5 Issand Jumal avas mu kõrva ja ma ei ole vastu pannud, ma ei ole taganenud.
\par 6 Ma andsin oma selja peksjaile ja põsed neile, kes katkusid karvu, ma ei peitnud oma palet teotuse ja sülje eest.
\par 7 Issand Jumal aitab mind: seepärast ma ei jäänud häbisse, seepärast ma muutsin oma näo otsekui ränikiviks, sest ma teadsin, et ma ei jää häbisse,
\par 8 et mu õigusemõistja on ligidal. Kes tahab minuga riielda? Astugem üheskoos ette! Kes on mu vastane? Tulgu ta mu juurde!
\par 9 Vaata, Issand Jumal aitab mind! Kes võib mind süüdi mõista? Vaata, nad kõik kuluvad nagu rüü, koi sööb nad.
\par 10 Kes teist kardab Issandat, see kuulgu tema sulase häält; kes käib pimeduses ja kel puudub valgusekuma, see lootku Issanda nime peale ja toetugu oma Jumalale.
\par 11 Aga vaata, teie kõik, kes läidate tule, süütate tuliseid nooli, minge oma tule leekidesse ja tuliste noolte keskele, mis te olete süüdanud. Minu käest tuleb see teile, et peate lamama piinades.

\chapter{51}

\par 1 Kuulge mind, õigluse nõudjad, Issanda otsijad! Vaadake kaljut, mille küljest te olete raiutud, ja kaevuauku, kust olete välja kaevatud.
\par 2 Vaadake Aabrahami, oma isa, ja Saarat, kes teid sünnitas. Kui ta alles üksi oli, kutsusin ma tema ja ma õnnistasin teda ning tegin ta paljuks.
\par 3 Sest Issand trööstib Siionit, trööstib kõiki selle varemeid; ta teeb selle kõrbe otsekui Eedeniks ja lagendikud Issanda rohuaia sarnaseks. Seal on lusti ja rõõmu, tänulaulu ja pillihäält.
\par 4 Pane mind tähele, mu rahvas, mu hõimud, kuulake mind! Sest minult lähtub Seadus ja mu õigus rahvaile valguseks.
\par 5 Äkitselt ligineb mu õiglus, ilmub mu pääste ja mu käsivarred mõistavad rahvaile kohut: mind ootavad saared ja loodavad mu käsivarre peale.
\par 6 Tõstke oma silmad taeva poole ja vaadake alla maa peale; sest taevad haihtuvad kui suits ja maa kulub nagu kuub, selle elanikud surevad otsekui sääsed. Aga minu pääste jääb igavesti ja minu õiglus ei lõpe.
\par 7 Kuulge mind, õigusetundjad, rahvas, kelle südames on mu Seadus! Ärge kartke inimeste laimu ja ärge ehmuge nende sõimust!
\par 8 Sest neid sööb riidekoi nagu riiet, neid sööb villakoi nagu villa. Aga minu õiglus jääb igavesti ja mu pääste põlvest põlve.
\par 9 Ärka, ärka, ehi ennast jõuga, Issanda käsivars! Ärka nagu muistseil päevil, endiste põlvede ajal! Eks olnud sina see, kes purustas Rahabi, kes torkas läbi merelohe?
\par 10 Eks olnud sina see, kes kuivatas mere, suure sügavuse veed, kes tegi teeks meresügavikud, lunastatuile läbitavaks?
\par 11 Ja Issanda lunastatud pöörduvad tagasi ning tulevad Siionisse hõisates. Nende pea kohal on igavene rõõm, rõõm ja ilutsemine valdavad neid, aga kurbus ja ohkamine põgenevad ära.
\par 12 Mina, mina olen see, kes teid trööstib! Kes oled sina, et sa kardad surelikke inimesi, inimlapsi, kes on nagu rohi,
\par 13 ja unustad Issanda, kes sind on teinud, kes on laotanud taeva ja rajanud maa, ja värised alati, iga päev, rõhuja viha ees, kui ta valmistub hävitama? Aga kus on nüüd rõhuja viha?
\par 14 Varsti vabastatakse aheldatu: ta ei sure vangiauku, tema leib ei lõpe.
\par 15 Sest mina olen Issand, su Jumal, kes liigutab merd ja paneb lained kohisema, kelle nimi on vägede Issand.
\par 16 Ja ma panin oma sõnad sulle suhu, ma peitsin sind oma käe varju alla, et võlvida taevas ja rajada maa, ja et öelda Siionile: „Sina oled mu rahvas.”
\par 17 Ärka, ärka, tõuse üles, Jeruusalemm, kes oled joonud Issanda käest tema viha karika, oled joonud tilgatuks uimastuse peekri!
\par 18 Ei olnud tal talutajat ühestki oma sünnitatud lapsest, ega haaranud tal käest kinni mitte ükski tema kasvatatud laps.
\par 19 Need kaks paari said sulle osaks - kes tunneb sulle kaasa? - rüüstamine ja hävitus, nälg ja mõõk - kes olen mina, et sind trööstiksin?
\par 20 Igal tänavanurgal lamasid su lapsed oimetult, otsekui metskitsed püügivõrgus, löödud Issanda vihast, su Jumala sõitlusest.
\par 21 Seepärast kuule ometi seda, sa vilets, kes oled joobnud, aga mitte veinist:
\par 22 nõnda ütleb su Issand, Issand, su Jumal, kes riidleb oma rahva eest: Vaata, ma võtan su käest uimastuse karika, oma viha peekri - seda ei ole sul enam vaja juua.
\par 23 Ja ma annan selle kätte su piinajaile, kes ütlesid sulle: „Kummarda, et saaksime sinust üle käia!” Sa andsidki oma selja maaks ja käijatele tänavaks.

\chapter{52}

\par 1 Ärka, ärka, ehi ennast oma jõuga, Siion! Pane selga oma ilusad riided, Jeruusalemm, püha linn! Sest sinu sisse ei tule enam ümberlõikamatu ja rüve.
\par 2 Puhasta ennast tolmust, tõuse üles, istu, Jeruusalemm, su kaela köidikud on vallandunud, vang, Siioni tütar!
\par 3 Sest nõnda ütleb Issand: Hinnata teid müüdi ja rahata teid lunastatakse.
\par 4 Sest nõnda ütleb Issand Jumal: Mu rahvas läks esiti alla Egiptusesse, et seal võõrana elada, ja seejärel rõhus Assur teda põhjuseta.
\par 5 Ja nüüd? Mis oleks mul siin teha, ütleb Issand, kui mu rahvas on hinnata ära viidud? Ta valitsejad räuskavad, ütleb Issand, ja mu nime teotatakse lakkamata iga päev.
\par 6 Seepärast peab mu rahvas tundma mu nime, peab seepärast just sel päeval mõistma, et mina olen see, kes ütleb: „Vaata, siin ma olen!”
\par 7 Kui armsad on mägede peal sõnumitooja sammud. Ta kuulutab rahu, toob häid sõnumeid, kuulutab päästet ja ütleb Siionile: „Sinu Jumal on kuningas!”
\par 8 Kuule! Su valvurid tõstavad häält, nad hõiskavad üheskoos, sest nad näevad silmast silma Issanda kojutulekut Siionisse.
\par 9 Rõõmutsege, hõisake üheskoos, Jeruusalemma varemed, sest Issand trööstib oma rahvast, lunastab Jeruusalemma!
\par 10 Issand paljastab oma püha käsivarre kõigi rahvaste nähes, ja kõik maailma ääred saavad näha meie Jumala päästet.
\par 11 Lahkuge, lahkuge, minge sealt ära, ärge puudutage rüvedat; minge ära selle keskelt, puhastage endid, Issanda riistade kandjad!
\par 12 Aga teil ei ole vaja lahkuda tõtates ega minna põgenedes, sest Issand käib teie ees, Iisraeli Jumal on teile järelväeks.
\par 13 Vaata, mu sulane talitab targasti, teda ülistatakse ning ülendatakse ja ta saab väga kõrgeks.
\par 14 Nagu paljud kohkusid tema pärast - nõnda rikutud, ebainimlik oli ta välimus ja ta kuju ei olnud inimlaste taoline -,
\par 15 nõnda ehmatab ta paljusid rahvaid, kuningad sulevad tema pärast suud. Kuid mida neile ei ole jutustatud, seda saavad nad näha, ja mida nad ei ole kuulnud, seda saavad nad teada.

\chapter{53}

\par 1 Kes usub meie kuulutust ja kellele on ilmutatud Issanda käsivars?
\par 2 Sest ta tõusis meie ees nagu võsuke, otsekui juur põuasest maast. Ei olnud tal kuju ega ilu, et teda vaadata, ega olnud tal välimust, et teda ihaldada.
\par 3 Ta oli põlatud ja inimeste poolt hüljatud, valude mees ja haigustega tuttav, niisugune, kelle pealt silmad ära pööratakse: ta oli põlatud ja me ei hoolinud temast.
\par 4 Ent tõeliselt võttis ta enese peale meie haigused ja kandis meie valusid. meie aga pidasime teda vigaseks, Jumalast nuhelduks ja vaevatuks.
\par 5 Ent teda haavati meie üleastumiste pärast, löödi meie süütegude tõttu. Karistus oli tema peal, et meil oleks rahu, ja tema vermete läbi on meile tervis tulnud.
\par 6 Me kõik eksisime nagu lambad, igaüks meist pöördus oma teed, aga Issand laskis meie kõigi süüteod tulla tema peale.
\par 7 Teda piinati ja ta alistus ega avanud suud nagu tall, keda viiakse tappa, nagu lammas, kes on vait oma niitjate ees, nõnda ei avanud ta oma suud.
\par 8 Surve ja kohtu läbi võeti ta ära, kes tema sugupõlvest mõtles sellele, et ta lõigati ära elavate maalt, ja teda tabas surm mu rahva üleastumise pärast?
\par 9 Temale anti haud õelate juurde, kurjategijate juurde, kui ta suri, kuigi ta ei olnud ülekohut teinud ega olnud pettust ta suus.
\par 10 Aga Issand arvas heaks teda alandada haigustega. Kui ta iseenese on andnud süüohvriks, saab ta näha tulevast sugu, ta elab kaua ja Issanda tahe teostub tema läbi.
\par 11 Pärast oma hingevaeva saab ta näha valgust ja rahuldust tunda; oma tarkusega teeb mu õiglane sulane paljusid õigeks, sest ta kannab nende patusüüd.
\par 12 Sellepärast ma annan temale osa paljude hulgas ja ta jagab vägevatega saaki, sest ta tühjendas oma hinge surmani ja ta arvati üleastujate hulka; tema aga kandis paljude pattu ja seisis üleastujate eest.

\chapter{54}

\par 1 Hõiska, sigimatu, kes pole sünnitanud, rõkata rõõmust ja ilutse, kes pole olnud lapsevaevas! Sest vallalisel saab olema rohkem lapsi kui abielunaisel, ütleb Issand.
\par 2 Tee avaraks oma telgipaik ja su elamute vaipu venitatagu! Ära ole kokkuhoidlik! Pikenda oma telginööre ja kinnita vaiu!
\par 3 Sest sa levid paremale ja vasakule, su järglased vallutavad rahvaid ja asustavad tühje linnu.
\par 4 Ära karda, sest sul ei ole vaja häbeneda, ja ära tunne piinlikkust, sest sa ei jää häbisse, vaid unusta oma noorpõlve häbi ja ära meenuta enam oma lesepõlve teotust!
\par 5 Sest sinu Looja on su mees, vägede Issand on tema nimi; sinu lunastaja on Iisraeli Püha, teda nimetatakse kogu maailma Jumalaks.
\par 6 Sest nagu hüljatud ja sügavasti kurvastatud naist kutsub sind Issand. Ja noorpõlve naine - kas teda saakski põlata? ütleb su Jumal.
\par 7 Ma jätsin su maha üürikeseks hetkeks, aga ma kogun sind suure halastusega.
\par 8 Ülevoolavas vihas peitsin ma oma palge silmapilguks su eest, aga ma halastan su peale igavese heldusega, ütleb Issand, su lunastaja.
\par 9 Sest see on mul nagu Noa päevil, kui ma vandusin, et Noa veed enam ei ujuta maad: nõnda ma vannun, et ma ei ole sinule vihane ega sõitle sind.
\par 10 Mäed liiguvad ja künkad kõiguvad küll, aga minu heldus ei liigu su juurest ja minu rahuseadus ei kõigu, ütleb Issand, su halastaja.
\par 11 Sina vilets, vintsutatu, trööstimatu! Vaata, ma ehitan sind türkiisikividega ja rajan su aluse safiiridest.
\par 12 Ma teen su müürisakmed rubiinidest, su väravad säravatest juveelidest ja kogu su ringmüüri kalliskividest.
\par 13 Ja kõik su lapsed on Issanda õpilased ning su lastel on suur rahu.
\par 14 Sind rajatakse õigluses; sa jääd eemale vägivallast, sest sul ei ole midagi karta, ja hirmust, sest see ei ligine sulle.
\par 15 Kui sulle ka kallale kiputakse, siis mitte minu poolt; kes sulle kallale kipub, see sinu pärast langeb.
\par 16 Vaata, mina olen loonud sepa, kes puhub ääsituld ja valmistab relva vastavaks otstarbeks: mina ise olen seega loonud hävitaja hävitama.
\par 17 Aga ei ole edu ühelgi relval, mis valmistatakse sinu vastu, ja sa mõistad hukka iga keele, mis tõuseb sinuga kohut käima. See on Issanda sulaste pärisosa ja nende õigus minult, ütleb Issand.

\chapter{55}

\par 1 Hoi! Kõik janused, tulge vee juurde! Ka see, kellel ei ole raha, tulgu, ostku ja söögu! Tulge, ostke ilma rahata, ilma hinnata veini ja piima!
\par 2 Miks vaete raha selle eest, mis ei ole leib, ja oma vaevatasu selle eest, mis ei toida? Kuulake mind hästi, siis te saate süüa head ja kosutada ennast rammusate roogadega.
\par 3 Pöörake kõrv ja tulge mu juurde, kuulake mind, siis on teil edu; mina teen teiega igavese lepingu, samasuguse, nagu oli mu osadus Taavetiga!
\par 4 Vaata, ma panin tema rahvaile tunnistajaks, rahvaste juhiks ja käskijaks.
\par 5 Näe, ka sina hakkad kutsuma rahvaid, keda sa ei tunne, ja rahvad, kes sind ei tunne, jooksevad sinu juurde Issanda, su Jumala pärast ja Iisraeli Püha pärast, sellepärast et tema sind austab.
\par 6 Otsige Issandat, kui ta on leitav, hüüdke teda, kui ta on ligidal!
\par 7 Õel jätku oma tee ja nurjatu mees oma mõtted ning pöördugu Issanda poole, siis halastab tema ta peale; ja meie Jumala poole, sest tema annab palju andeks.
\par 8 Aga minu mõtted ei ole teie mõtted, ja teie teed ei ole minu teed, ütleb Issand.
\par 9 Sest otsekui taevad on maast kõrgemal, nõnda on minu teed kõrgemad kui teie teed, ja minu mõtted kõrgemad kui teie mõtted.
\par 10 Sest otsekui vihm ja lumi tulevad taevast alla ega lähe sinna tagasi, vaid kastavad maad ja teevad selle sigivaks ning kandvaks, et see annaks külvajale seemet ja sööjale leiba,
\par 11 nõnda on ka minu sõnaga, mis lähtub mu suust: see ei tule tagasi mu juurde tühjalt, vaid teeb, mis on mu meele järgi, ja saadab korda, milleks ma selle läkitasin.
\par 12 Jah, te lähete rõõmsasti välja ja teid tuuakse rahus. Mäed ja künkad rõkatavad rõõmust teie ees ning kõik väljapuud plaksutavad käsi.
\par 13 Kibuvitste asemel kasvavad küpressid, nõgeste asemel kasvavad mürdid. See sünnib Issanda auks, igaveseks, hävimatuks märgiks.

\chapter{56}

\par 1 Nõnda ütleb Issand: Pange tähele õigust ja olge õiglased, sest mu pääste on ligidal ja mu õiglus ilmumas!
\par 2 Õnnis on inimene, kes nõnda teeb, inimlaps, kes selles püsib, kes peab hingamispäeva ega riku seda, ja kes hoiab oma kätt igast kurjast teost.
\par 3 Ärgu rääkigu ega öelgu Issandaga liitunud võõras: „Issand eraldab mind muidugi oma rahvast!„ Ja kohitsetu ärgu öelgu: ”Vaata, ma olen kuivanud puu!”
\par 4 Sest nõnda ütleb Issand: Kohitsetuile, kes peavad mu hingamispäevi ja valivad, mis on mu meele järgi, ja peavad kinni mu lepingust,
\par 5 neile ma annan oma kojas ja oma müüride vahel mälestusmärgi ja nime, parema kui pojad ja tütred: mina annan neile igavese nime, mida ei saa hävitada.
\par 6 Ja võõrad, kes on liitunud Issandaga, teenivad teda ja armastavad Issanda nime, et saada tema sulaseiks, kõik, kes peavad hingamispäeva ega riku seda ja kes peavad kinni mu lepingust -
\par 7 needki ma viin oma pühale mäele ja ma rõõmustan neid oma palvekojas, ja nende põletus- ja tapaohvrid on mu altari peal meelepärased; sest mu koda nimetatakse palvekojaks kõigile rahvastele.
\par 8 Issanda, Iisraeli hajutatuid koguva Jumala ütlus: Ma kogun tema juurde veelgi, lisaks neile, kes tema juurde on kogutud.
\par 9 Kõik loomad väljal, tulge sööma, tulge, kõik metsloomad!
\par 10 Tema vahimehed on kõik pimedad, nad ei mõista midagi, nad kõik on nagu tummad koerad, kes ei saa haukuda; nad armastavad tukkuda, magavad norinal.
\par 11 Nad on aplad koerad, kes ei saa iialgi küllalt. Nad on karjased, kellel ei ole arusaamist. Kõik nad pöörduvad oma teed, igaüks eranditult oma kasu poole:
\par 12 „Tulge, ma toon veini, rüüpame vägijooki! Olgu homne päev nagu tänane, veel palju suurepärasem!”

\chapter{57}

\par 1 Õige sureb, aga ükski ei võta seda südamesse, ja vagad mehed koristatakse, ilma et keegi märkaks. Kuid õige koristatakse õnnetuse eest,
\par 2 ta läheb rahusse: oma voodites hingavad need, kes on käinud sirget teed.
\par 3 Aga teie tulge siia, te nõiamoori lapsed, abielurikkujate ja hoorade sigitis!
\par 4 Kelle üle te tunnete rõõmu? Kelle vastu te ajate suu ammuli, näitate keelt? Eks te ole üleastumise lapsed, vale sigitis,
\par 5 teie, kes kirest hõõgute tammede varjus, iga halja puu all, teie, kes tapate lapsi orgudes, kaljulõhede vahel.
\par 6 Oru siledate kivide seas on su osa, need, need on su liisk. Neile oled sa valanud joogiohvrit, viinud roaohvrit. Kas ma sellega lasen ennast lepitada?
\par 7 Kõrgele ja väljapaistvale mäele seadsid sa oma voodi, sinna üles sa läksid ka tapaohvreid ohverdama.
\par 8 Ja ukse ning piitade taha sa panid oma tunnusmärgi, sest minust lahkudes sa riietusid lahti ja läksid üles, tegid oma voodi laiemaks, kauplesid enesele neid, kelle voodeid sa armastasid, vahtisid kürva.
\par 9 Sa laskusid alla Mooloki juurde õliga ja priiskasid oma võietega; sa läkitasid oma käskjalad kaugele, ja sügavale alla, kuni surmavallani.
\par 10 Sa väsisid oma paljudest rännakutest, aga sa ei öelnud: „Lootuseta!” Sa leidsid uut elujõudu, sellepärast sa ei nõrkenud.
\par 11 Keda sa pelgasid ja kartsid, et sa valetasid? Sa ei meenutanud mind ega võtnud seda südamesse. Eks ole: mina vaikin ammusest ajast ja seepärast sa ei karda mind.
\par 12 Mina teen teatavaks su õiguse ja su teod, aga need ei aita sind.
\par 13 Kui sa kisendad, siis päästku sind su ebajumalate jõuk. Aga tuul viib need kõik, tuuleõhk võtab nad ära. Kes aga otsib abi minu juurest, see pärib maa, omandab mu püha mäe.
\par 14 On öeldud: „Sillutage, sillutage, tehke tee valmis, koristage takistused mu rahva teelt!”
\par 15 Sest nõnda ütleb kõrge ja üllas, kes igavesti elab ja kelle nimi on püha: Ma elan kõrges ja pühas paigas ja rõhutute ning vaimult alandlike juures, et turgutada alandlike vaimu ja elustada rõhutute südameid.
\par 16 Sest mina ei riidle mitte igavesti ega ole jäädavalt vihane, muidu nõrkeksid mu palge ees nende vaim ja hinged, keda ma ise olen teinud.
\par 17 Tema ahnuse süü pärast ma vihastasin, lõin teda, peitsin enese ja olin vihane, ent tema käis taganejana omaenese südame teed.
\par 18 Ma olen näinud tema teid, aga ma parandan ja juhatan teda; ja ma tasun temale troostiga ning pakun ta leinajaile huultevilja.
\par 19 Rahu, rahu kaugel ja lähedal olijale, ütleb Issand, ja ma parandan teda.
\par 20 Aga õelad on otsekui mässav meri, mis ei saa rahuneda ja mille veed kobrutavad kõntsa ja muda.
\par 21 Õelatel ei ole rahu, ütleb minu Jumal.

\chapter{58}

\par 1 Hüüa täiest kõrist, ära peatu, tõsta häält otsekui pasun! Tee teatavaks mu rahvale nende üleastumine ja Jaakobi soole nende patud!
\par 2 Päevast päeva nad küll otsivad mind ja tahavad teada mu teid, nagu oleks see rahvas, kes on õiglane ega hülga oma Jumala õigust. Nad nõuavad minult õiglasi otsuseid, igatsevad Jumala ligiolekut:
\par 3 „Miks me paastume, kui sa seda ei näe, alandame oma hinge, kui sa seda ei märka?” Vaata, oma paastupäeval te teete, mis teile meeldib, ja pigistate kõiki oma võlgnikke.
\par 4 Vaata, te paastute riiuks ja tüliks ja et lüüa õela rusikaga. Praegu te küll ei paastu selleks, et teha oma häält kuuldavaks ülal.
\par 5 Kas niisugune on see paast, mis mulle meeldib, päev, mil inimene alandab oma hinge, et ta painutab oma pead nagu kõrkjas ja teeb enesele aseme kotiriidest ning tuhast? Kas sa seda nimetad paastuks ja Issandale meelepäraseks päevaks?
\par 6 Eks ole ju mulle meeldiv paast niisugune: päästa valla ülekohtused ahelad, teha lahti ikke rihmad, lasta vabaks rõhutud ja purustada kõik ikked?
\par 7 Eks see ole murda oma leiba näljasele ja viia oma kotta viletsad kodutud, kui sa näed alastiolijat ja riietad teda ega hoidu oma ligimesest?
\par 8 Siis ilmub su valgus otsekui koit ja su paranemine edeneb jõudsasti. Sinu õigus käib su ees, Issanda auhiilgus järgneb sulle.
\par 9 Siis sa hüüad ja Issand vastab, kisendad appi ja tema ütleb: „Vaata, siin ma olen!” Kui sa oma keskelt eemaldad ikke, sõrmega näitamise ja nurjatu kõne,
\par 10 kui sa pakud näljasele sedasama, mida sa ka ise himustad, ja toidad alandatud hinge, siis koidab sulle pimeduses valgus ja su pilkane pimedus on otsekui keskpäev.
\par 11 Ja Issand juhatab sind alati ning toidab su hinge põuasel maal; ta teeb tugevaks su luud-liikmed ja sa oled otsekui kastetud rohuaed, veelätte sarnane, mille vesi ei valmista iial pettumust.
\par 12 Ja su omad ehitavad üles muistsed varemed, sa taastad endiste põlvede alusmüürid; sind nimetatakse „lõhutud müüride parandajaks„, ”teeradade käidavaks tegijaks”.
\par 13 Kui sa hingamispäeval seisatad ega tee mu pühal päeval, mis sulle meeldib, kui sa nimetad hingamispäeva rõõmuks ja Issanda püha päeva austusväärseks ning austad seda ega tee, mis sulle meeldib, ei otsi omakasu ega kõnele tühje sõnu,
\par 14 siis sa tunned rõõmu Issandast: mina viin sind üle maa kõrgendike ja toidan sind su isa Jaakobi pärandiga. Jah, Issanda suu on rääkinud.

\chapter{59}

\par 1 Vaata, Issanda käsi ei ole päästmiseks lühike ega ole ta kõrv kuulmiseks kurt,
\par 2 vaid teie süüteod on teinud vahe teie ja teie Jumala vahele, teie patud varjavad tema palge teie eest, sellepärast ta ei kuule.
\par 3 Sest teie käed on rüvetatud verega ja teie sõrmed süüga, teie huuled väidavad valet, teie keel kõneleb kõverust.
\par 4 Ükski ei süüdista õiglaselt ja keegi ei lähe kohtusse tõe nimel: loodetakse tühja peale ja räägitakse valet, sigitatakse pahandust ja sünnitatakse kurjust.
\par 5 Nad hauvad mürkmao mune ja koovad ämblikuvõrku; kes sööb nende mune, see sureb, ja katkivajutatust poeb välja rästik.
\par 6 Nende lõngad ei kõlba riideks ja nad ei saa endid katta oma tegudega; nende teod on nurjatud teod ja nende kätes on vägivallatöö.
\par 7 Nende jalad jooksevad kurja poole ja nad tõttavad valama vaga verd; nende mõtted on nurjatud mõtted, nende teedel on rüüstamine ja hävitus.
\par 8 Rahu rada nad ei tunne ja õigust ei ole nende jälgedes; nad teevad oma teerajad kõveraiks, ükski, kes käib nende peal, ei tunne rahu.
\par 9 Sellepärast on õigus meist kaugel ja õiglus ei ulatu meieni; me ootame valgust, aga vaata, on pimedus, me ootame valgusekuma, aga käime pilkases pimedas.
\par 10 Me kobame seina pimedate sarnaselt, kobame otsekui silmitud; keskpäevaajal me komistame nagu hämarikus, elujõuliste keskel oleme otsekui surnud.
\par 11 Me kõik mõmiseme nagu karud ja kudrutame tuvide sarnaselt; me ootame õigust, aga seda ei ole, päästet, aga see on meist kaugel.
\par 12 Sest meie üleastumisi sinu ees on palju ja meie patud tunnistavad meie vastu, sest meie üleastumised on meiega kaasas ja me tunneme oma süütegusid:
\par 13 vastuhakk ja Issanda salgamine ning loobumine käimisest meie Jumala järel; kõned rõhumisest ja ärataganemisest, valelike sõnade väljamõtlemine ja südamest kuuldavale toomine.
\par 14 Õiglus on tagasi tõrjutud ja õigus seisab kaugel, sest tõde komistab tänaval ja ausus ei saa sisse tulla.
\par 15 Nõnda on tõde kadunud ja kes loobub kurjast, laseb ennast paljaks riisuda. Issand nägi seda ja see oli tema silmis paha, et õigust ei olnud.
\par 16 Tema nägi, et ei olnud ühtegi meest, ja imestas, et ei olnud ühtegi vaheleastujat. Siis aitas teda ta oma käsivars ja teda toetas tema õigus.
\par 17 Ta pani enesele selga õiguse otsekui soomusrüü, ja päästekiivri pähe; ta riietus kättemaksuriietesse ja kattis ennast püha vihaga otsekui ülekuuega.
\par 18 Missugused on teod, niisugune on tasu: viha oma vastaste vastu, kättemaks oma vaenlastele, ta maksab saartele kätte.
\par 19 Siis kardetakse õhtu pool Issanda nime ja päikesetõusu pool tema auhiilgust, sest ta tuleb otsekui paisutatud jõgi, mida Issanda tuul edasi ajab.
\par 20 Aga Issand ütleb: Siionile, üleastumisest pöördujale Jaakobis, tuleb lunastaja.
\par 21 Ja niisugune on mu leping nendega, ütleb Issand: Minu Vaim, kes on su peal, ja minu sõnad, mis ma olen pannud sulle suhu, ei lahku sinu suust ega sinu järglaste suust ega sinu järglaste järglaste suust, ütleb Issand, nüüdsest ajast alates ja igavesti.

\chapter{60}

\par 1 Tõuse, paista, sest sinu valgus tuleb ja Issanda auhiilgus koidab su kohal.
\par 2 Sest vaata, pimedus katab maad ja pilkane pimedus rahvaid, aga sinu kohal koidab Issand ja sinu kohal nähakse tema auhiilgust.
\par 3 Ja rahvad tulevad su valguse juurde ning kuningad paistuse juurde, mis sinust kumab.
\par 4 Tõsta oma silmad ja vaata ringi: nad kõik kogunevad, tulevad su juurde; su pojad tulevad kaugelt, su tütreid kantakse kätel.
\par 5 Siis sa näed ja särad rõõmust, su süda põksub ja avardub, kui su poole pöördub mere ohtrus ja su juurde tuleb rahvaste rikkus.
\par 6 Sind katab kaamelite hulk, Midjani ja Eefa noored kaamelid; kõik tulevad Sebast, kannavad kulda ja viirukit ning kuulutavad Issanda kiiduväärsust.
\par 7 Su juurde kogunevad kõik Keedari karjad, Nebajoti jäärad teenivad sind: meelepärasena tulevad need mu altari peale ja ma teen oma hiilguse koja veel toredamaks.
\par 8 Kes on need, kes lendavad otsekui pilv, otsekui tuvid oma puuriavadesse?
\par 9 Jah, mu juurde kogunevad meresõitjad, eesotsas Tarsise laevadega, et tuua su lapsi kaugelt; nende hõbe ja kuld on neil kaasas - Issanda, su Jumala nimele ja Iisraeli Pühale, sellepärast et tema sind austab.
\par 10 Ja võõrad ehitavad üles su müürid ja nende kuningad teenivad sind; sest oma vihas ma lõin sind, aga oma armus ma halastan su peale.
\par 11 Su väravad on alati lahti, neid ei suleta päeval ega ööl, et su juurde saaks tuua rahvaste rikkusi ja juhatada nende kuningaid.
\par 12 Sest rahvas või kuningriik, kes sind ei teeni, hukkub, seesugused rahvad rüüstatakse sootuks.
\par 13 Liibanoni toredus tuleb su juurde, küpressid, plataanid ja piiniad üheskoos, et kaunistada mu pühamu paika ja et ma saaksin austada oma jalgade aset.
\par 14 Ja kummargil tulevad su juurde su rõhujate lapsed, ja kõik, kes sind põlastasid, peavad kummardama su jalataldadeni; ja nad nimetavad sind „Issanda linnaks„, ”Iisraeli Püha Siioniks”.
\par 15 Selle asemel et sa olid mahajäetud ning vihatud, nii et ükski ei käinud sinu kaudu, teen ma sind igaveseks uhkuseks, rõõmuks põlvest põlve.
\par 16 Sina saad imeda rahvaste piima, imeda kuningategi rinda, ja sa tunned, et mina, Issand, olen su päästja, et Jaakobi Vägev on su lunastaja.
\par 17 Vase asemel ma toon kulda ja raua asemel ma toon hõbedat, puidu asemel vaske ja kivide asemel rauda. Ja ma panen su ülemuseks rahu ning su sundijaks õiguse.
\par 18 Ei kuuldu enam vägivallast su maal, rüüstamisest ja hävitusest su piirides; sa nimetad oma müüre „Päästeks„ ja väravaid ”Kiituseks”.
\par 19 Päike ei ole sulle enam valguseks päeval ega paista sulle kuupaiste, vaid Issand on sulle igaveseks valguseks ja su Jumal on su hiilgus.
\par 20 Su päike ei lähe enam looja ja su kuu ei kahane, sest Issand on sulle igaveseks valguseks ja su leinapäevad lõpevad.
\par 21 Ja igaüks su rahvast on õiglane: nad pärivad igaveseks maa kui minu istutatud võsu, mu kätetöö, millega ma ennast austan.
\par 22 Kõige pisemast saab tuhatkond ja kõige väetimast vägev rahvas. Mina, Issand, tõttan sellega määratud ajal.

\chapter{61}

\par 1 Issanda Jumala Vaim on minu peal, sest Issand on mind võidnud; ta on mind läkitanud viima rõõmusõnumit alandlikele, parandama neid, kel murtud süda, kuulutama vabastust vangidele ja avama pimedate silmi,
\par 2 kuulutama Issanda meelepärast aastat ja meie Jumala kättemaksu päeva, trööstima kõiki leinajaid,
\par 3 andma Siioni leinajaile laubaehte tuha asemel, rõõmuõli leinarüü asemel, ülistusrüü kustuva vaimu asemel, et neid nimetataks „Õigluse tammedeks„, ”Issanda istanduseks”, millega ta ennast ehib.
\par 4 Nad ehitavad üles muistsed varemed, taastavad esivanemate rüüstatud paigad ja uuendavad hävitatud linnad, mis põlvede jooksul on olnud laastatud.
\par 5 Ja võõrad seisavad ning hoiavad teie karja, muulased on teie põllu- ja viinamäeharijad.
\par 6 Aga teid nimetatakse Issanda preestriteks, teist räägitakse kui meie Jumala sulastest; te toitute rahvaste rikkustest ja võite kiidelda nende varandusega.
\par 7 Häbi ja teotuse asemel te saate kahekordse osa ja ilutsete selles: te pärite kahekordselt oma maal, teil on igavene rõõm,
\par 8 sest mina, Issand, armastan õiglust, vihkan röövimist ja nurjatust. Ma annan neile ustavalt töötasu ja teen nendega igavese lepingu.
\par 9 Nende sugu saab tuntuks paganate juures ja nende järglaskond rahvaste keskel; kõik, kes neid näevad, tunnevad ära, et nad on Issanda õnnistatud sugu.
\par 10 Mina rõõmutsen väga Issandas, mu hing ilutseb mu Jumalas, sest ta on mind riietanud päästeriietega, katnud õigusekuuega, otsekui oleks peigmees enesele pähe pannud piduliku peakatte või pruut ennast ehetega ehtinud.
\par 11 Sest nagu maa toob esile oma kasvud ja aed laseb võrsuda oma külvi, nõnda laseb ka Issand Jumal võrsuda õigust ja kiitust kõigi rahvaste ees.

\chapter{62}

\par 1 Siioni pärast ma ei vaiki ega jää rahule Jeruusalemma pärast, enne kui tema õigus hakkab paistma ja tema pääste põleb otsekui tõrvik.
\par 2 Siis saavad rahvad näha su õigust ja kõik kuningad su au. Ja sulle antakse uus nimi, mille määrab Issanda suu.
\par 3 Siis oled sa toredaks krooniks Issanda käes ja kuninglikuks peaehteks oma Jumala pihus.
\par 4 Sinust ei kõnelda enam kui „Hüljatust„ ega kõnelda su maast enam kui „Laastatust„, vaid sind hüütakse ”Minu armsam” ja su maad ”Abikaasa”, sest Issand armastab sind ja su maa saab mehele.
\par 5 Sest nagu noor mees naib neitsi, nõnda naivad sind su pojad; ja nagu peigmees tunneb rõõmu pruudist, nõnda tunneb su Jumal rõõmu sinust.
\par 6 Sinu müüride peale, Jeruusalemm, olen ma seadnud vahid; kogu päeva ja kogu öö ei tohi nad hetkekski vaikida. Issanda meenutajad, ärgu olgu teil puhkust
\par 7 ja ärge andke temale puhkust, enne kui ta on rajanud Jeruusalemma ja seadnud selle maa peale kiituseks!
\par 8 Issand on vandunud oma parema käe ja oma tugeva käsivarre juures: tõesti, enam ma ei anna su vilja su vaenlastele roaks ega saa võõramaalased juua su veini, mille pärast sa oled vaeva näinud,
\par 9 vaid kes koguvad, need söövad ja kiidavad Issandat, ja kes korjavad, need joovad mu pühamu õuedes.
\par 10 Minge, minge väravaist välja, valmistage rahvale teed, sillutage, sillutage maanteed, puhastage kividest, tõstke rahvaile lipp!
\par 11 Vaata, Issand on kuulutanud maa ääreni: Öelge Siioni tütrele: Vaata, su pääste tuleb! Näe, temaga koos on ta palk ja tema ees on ta töötasu.
\par 12 Ja neid hüütakse „Pühaks rahvaks„, „Issanda lunastatuiks„. Ja sind hüütakse ”Otsituks”, ”Linnaks, mis pole maha jäetud”.

\chapter{63}

\par 1 Kes see on, kes tuleb Edomist, erepunaste riietega Bosrast, see silmapaistva kuuega, kes sammub oma jõukülluses? „See olen mina, õiguse kuulutaja, võimas päästma!”
\par 2 Miks on su kuub punane ja riided nagu surutõrresõtkujal?
\par 3 „Mina sõtkusin surutõrt üksinda, rahvaste hulgast ei olnud ükski minuga; mina sõtkusin neid vihas ja tallasin raevus: nende verejoad pritsisid mu riiete peale ja ma määrisin kogu oma kuue.
\par 4 Sest mul oli südames kättemaksupäev ja oli tulnud mu tasumisaasta.
\par 5 Ma vaatasin, aga aitajat ei olnud, imestasin, aga ükski ei toetanud; siis aitas mind mu oma käsivars ja mulle oli toeks mu tuline viha.
\par 6 Oma vihas ma tallasin maha rahvad, purustasin nad raevus ja lasksin nende verejoad voolata maa peale.”
\par 7 Ma meenutan Issanda heldust, Issanda kiiduväärsust, kõige selle pärast, mida Issand meile on osutanud, ja suurt headust Iisraeli soo vastu, mida ta neile on osutanud oma halastuse ja suure helduse pärast.
\par 8 Sest ta ütles: Nemad on tõesti minu rahvas, lapsed, kes ei tee pettust. Ja ta tuli neile päästjaks.
\par 9 Kõigis nende ahistustes tundis ta ahistust ja tema palge ingel päästis nad. Armastuse ja kaastunde pärast ta lunastas nad, tõstis nad üles ja kandis neid kõigil muistseil päevil.
\par 10 Aga nad tõstsid mässu ja kurvastasid tema Püha Vaimu; seepärast ta muutus nende vaenlaseks, võitles ise nende vastu.
\par 11 Siis nad meenutasid muistseid päevi, Moosest ja tema rahvast: Kus on see, kes tõi veest välja oma lammaste karjase? Kus on see, kes pani tema sisse oma Püha Vaimu,
\par 12 kes oma aulist käsivart laskis käia Moosese paremal pool, kes lõhestas nende ees veed, et teha enesele igavest nime,
\par 13 kes talutas neid sügavustest läbi otsekui hobust kõrbes, et nad ei komistaks?
\par 14 Otsekui karja, kes läheb alla orgu, viis Issanda Vaim neid puhkepaika. Nõnda juhtisid sa oma rahvast, et teha enesele aulist nime.
\par 15 Vaata taevast alla ja näe oma pühast ja aulisest eluasemest. Kus on su püha viha ja su võimsad teod, su seesmine liigutus ja su halastus? Ära hoia neid tagasi!
\par 16 Sest sina oled meie isa! Aabraham ju meist ei tea ja Iisrael meid ei tunne. Sina, Issand, oled meie isa, muistsest ajast on su nimi meie Lunastaja.
\par 17 Miks lased meid, Issand, eksida sinu teedelt, lased meie südame kõvaks jääda, nõnda et me sind ei karda? Pöördu oma sulaste, oma pärisosa suguharude pärast!
\par 18 Ainult üürikeseks ajaks oli su pühitsetud rahvas pärijaks, siis tallasid meie vaenlased su pühamu.
\par 19 Meie oleme nagu need, keda sa iialgi ei ole valitsenud, keda ei ole nimetatud sinu nimega.

\chapter{64}

\par 1 Oh, et sa ometi käristaksid taevad lõhki ja tuleksid alla, et mäed kõiguksid su ees - otsekui tuli põletaks risu või tuli paneks vee keema -, et su nimi saaks tuntuks su vaenlastele, et rahvad väriseksid su ees,
\par 2 kui sa teed kardetavaid tegusid üle meie ootuse, tuled alla, mäed kõiguvad su ees - sellest ei ole kuuldud muistsest ajast.
\par 3 Ükski kõrv ei ole kuulnud, ükski silm ei ole näinud muud Jumalat peale sinu, kes tema ootajale seesugust võiks teha.
\par 4 Oh, et sa tuleksid vastu sellele, kes rõõmsasti teeb õigust, neile, kes mõtlevad sinu teedele! Vaata, sina vihastasid, et me tegime patutegusid; neis me oleme olnud kaua ja kas me pääseme?
\par 5 Me kõik oleme saanud rüvedaks ja kõik meie õigused on määrdunud riide sarnased; me kõik oleme närtsinud nagu lehed ja meie süü kannab meid ära otsekui tuul.
\par 6 Ei ole ühtegi, kes hüüaks appi sinu nime, kes ennast õhutaks sinust kinni haarama; sest sina oled peitnud oma palge meie eest ja oled lasknud meid hääbuda meie süütegude tõttu.
\par 7 Nüüd aga, Issand, oled sina meie isa. meie oleme savi ja sina vormid meid, me kõik oleme sinu kätetöö.
\par 8 Issand! Ära vihasta üleliia ja ära meenuta ülekohut lõpmata: näe, vaata ometi - me kõik oleme ju sinu rahvas!
\par 9 Su pühad linnad on saanud kõrbeks, Siion on saanud kõrbeks, Jeruusalemm kõnnumaaks.
\par 10 Meie püha ja ilus koda, kus meie vanemad sind ülistasid, on saanud tuleroaks, ja varemeis on kõik, mis oli meile kallis.
\par 11 Kas sa selle juures, Issand, tahad ennast veel tagasi hoida, vaikida ja meid üliväga alandada?

\chapter{65}

\par 1 Ma olen olnud kättesaadav neile, kes mind ei ole nõudnud; ma olen olnud leitav neile, kes mind ei ole otsinud; ma olen öelnud rahvale, kes mu nime ei ole appi hüüdnud: „Vaata, siin ma olen! Vaata, siin ma olen!”
\par 2 Kogu päeva ma sirutan käsi kangekaelse rahva poole, kes iseenese mõtetele järgnedes käib teed, mis ei ole hea,
\par 3 rahva poole, kes mind ärritab, alati mind trotsib rohuaedades ohverdades ja telliskivide peal suitsutades,
\par 4 kes istub haudades ja ööbib kaljulõhedes, kes sööb sealiha ja kellel on astjais roisklihaleem,
\par 5 kes ütleb: „Jää sinna, kus oled, ära ligine mulle, sest ma olen sulle püha!” Need on suits mu sõõrmeis, tuli, mis põleb kogu päeva.
\par 6 Vaata, see on mu ees kirja pandud: Mina ei rahune, enne kui olen tasunud, jah, kui olen tasunud neile sülle
\par 7 teie ja teie vanemate süüteod üheskoos, ütleb Issand, sellepärast et nad on suitsutanud mägedel ja on küngastel mind teotanud. Jah, ma mõõdan neile rüppe nende varasemad teod.
\par 8 Nõnda ütleb Issand: Otsekui mahlaka viinamarjakobara kohta öeldakse: „Ära seda riku, sest selles on õnnistus!”, nõnda teen mina oma sulaste pärast, et mitte kõiki hävitada.
\par 9 Mina toon Jaakobist järglase ja Juudast oma mägede pärija; minu valitud pärivad maa ja minu sulased elavad seal.
\par 10 Saaron saab lammaste ja kitsede karjamaaks ja Aakori org veiste lebamispaigaks mu rahvale, kes mind otsib.
\par 11 Aga teie, kes hülgate Issanda, unustate mu püha mäe, kes katate õnnejumalale laua ja kallate saatusejumalale tembitud veini -
\par 12 teid ma määran mõõga jaoks ja teil kõigil tuleb põlvitada tapaks, sest kui ma hüüdsin, siis te ei vastanud, kui ma rääkisin, siis te ei kuulnud, vaid tegite kurja mu silmis ja valisite, mis oli mulle vastumeelt.
\par 13 Seepärast ütleb Issand Jumal nõnda: Vaata, minu sulased söövad, aga teie nälgite; vaata, minu sulased joovad, aga teil on janu; vaata, minu sulased rõõmutsevad, aga teie häbenete.
\par 14 Vaata, minu sulased hõiskavad südamerõõmust, aga teie kisendate südamevalust ja ulute meeleheitest.
\par 15 Ja teie jätate oma nime mu valituile sajatuseks: „Nõnda surmaku sindki Issand Jumal!” Aga oma sulaseid ta nimetab teise nimega.
\par 16 Kes maa peal ennast õnnistab, õnnistab ennast tõe Jumala nimel, ja kes maa peal vannub, vannub tõe Jumala juures, sest endised hädad on unustatud ja peidetud mu silme eest.
\par 17 Sest vaata, ma loon uue taeva ja uue maa. Enam ei mõelda endiste asjade peale ja need ei tule meeldegi,
\par 18 vaid rõõmutsetakse ja ollakse igavesti rõõmsad mu loodu pärast. Sest vaata, ma loon Jeruusalemma rõõmuks ja ta rahva rõõmustuseks.
\par 19 Mina rõõmutsen Jeruusalemma pärast ja tunnen rõõmu oma rahvast; seal ei ole enam kuulda nutu- ega hädakisahäält.
\par 20 Seal ei ole enam imikut, kes elab ainult mõne päeva, ega rauka, kellel ei täitu ta päevade määr, nooreks peetakse seda, kes sureb saja-aastaselt, ja neetuks seda, kes ei saa sadat aastat täis.
\par 21 Nad ehitavad kodasid ja elavad neis, istutavad viinamägesid ja söövad nende vilja.
\par 22 Nad ei ehita teistele elamiseks, ei istuta teistele söömiseks, sest mu rahva eluiga on otsekui puu eluiga ja mu valitud kasutavad ise oma kätetööd.
\par 23 Nad ei näe asjata vaeva ega sünnita lapsi hirmu jaoks, sest nad on Issanda õnnistatud sugu ja koos nendega on õnnistus nende võrseil.
\par 24 Enne kui nad hüüavad, vastan mina; kui nad alles räägivad, olen mina kuulnud.
\par 25 Hunt ja tall käivad koos karjas, lõvi sööb õlgi nagu veis ja mao toiduks on põrm: ei tehta paha ega kahju kogu mu pühal mäel, ütleb Issand.

\chapter{66}

\par 1 Nõnda ütleb Issand: Taevas on minu aujärg ja maa on minu jalajärg. Kus saaks siis olla koda, mida te tahate mulle ehitada, ja kus saaks siis olla mu hingamispaik?
\par 2 On ju minu käsi selle kõik teinud ja nõnda on see kõik sündinud, ütleb Issand. Aga mina vaatan ka niisuguse peale, kes on vilets, kellel on purukspekstud vaim ja kes väriseb mu sõna ees.
\par 3 Ons inimese mahalööja härja tapjaga võrdne? Koera kaela käänaja lamba ohverdajaga? Sea vere tooja roaohvri toojaga? Ons ebajumala austaja viiruki suitsutajaga võrdne? Nõnda nagu need on valinud oma teed ja nende hing armastab nende jäledusi,
\par 4 nõnda valin minagi nende jaoks piinu ja saadan neile sellepärast hirmuvärinaid; sest kui ma hüüdsin, siis ei vastanud ükski, kui ma rääkisin, siis nad ei kuulnud, vaid tegid kurja mu silmis ja valisid, mis on mulle vastumeelt.
\par 5 Kuulge Issanda sõna, kes te värisete tema sõna ees: Teie vennad, kes teid vihkavad, kes teid ära tõukavad minu nime pärast, ütlevad: „Issand olgu auline, et saaksime näha teie rõõmu!” Aga nad peavad jääma häbisse!
\par 6 Kuule! Lärm linnast! Hääl templist! Issanda Hääl, kes tasub kätte oma vaenlastele.
\par 7 Juba enne lapsevaeva sünnitab Siion, enne kui temale tulevad valud, toob ta poeglapse ilmale.
\par 8 Kes on kuulnud midagi niisugust? Kes on näinud selliseid asju? Kas maa sünnitatakse üheainsa päevaga või tuleb rahvas ilmale ühekorraga? Kuid niipea kui Siion tunneb valusid, toob ta ka kohe oma lapsed ilmale.
\par 9 Kas mina avan emakoja, ilma et laseksin sünnitada? Või peaksin mina, sünnitama saatja, sulgema emakoja? ütleb su Jumal.
\par 10 Rõõmustage koos Jeruusalemmaga ja hõisake tema pärast kõik, kes teda armastate! Olge väga rõõmsad koos temaga kõik, kes te tema pärast leinasite,
\par 11 et võite imeda ja küllastuda tema troostirindadest, et võite juua ja ennast kosutada tema ohtrast emarinnast!
\par 12 Sest nõnda ütleb Issand: Vaata, ma juhin tema juurde rahu otsekui jõe, ja rahvaste rikkused otsekui tulvava oja; teie lapsi kantakse kätel ja hellitatakse põlvedel.
\par 13 Otsekui trööstiks teid ema, nõnda trööstin ma teid - ja teid trööstitakse Jeruusalemmas.
\par 14 Teie näete seda ja teie süda rõõmustab, teie luud-liikmed kasvavad nagu värske rohi. On tuntav, et Issanda käsi on tema sulastega, aga ta needus tema vaenlastega.
\par 15 Sest vaata, Issand tuleb tules ja tema sõjavankrid on otsekui tuulekeeris, et kätte tasuda vihalõõsas ja sõitluse tuleleekidega.
\par 16 Sest Issand mõistab kõige liha üle tule ja mõõgaga kohut ja Issanda poolt mahalööduid on siis palju.
\par 17 Kes endid pühitsevad ja puhastavad rohuaedade jaoks, järgnedes mõnele endi keskelt, ning sealiha, jäleduste ja hiirte sööjad saavad otsa üheskoos, ütleb Issand.
\par 18 Mina tunnen nende tegusid ja mõtteid, aga ma tulen koguma kõiki rahvaid ning keeli; ja need tulevad ning näevad minu auhiilgust.
\par 19 Ja ma teen nende keskel ühe tunnustähe. Ma läkitan nende hulgast pääsenuid rahvaste juurde Tarsisesse, Puudi ja Luudi ammuküttide juurde, Tubalisse ja Jaavanisse, kaugetele saartele, kes ei ole kuulnud minust räägitavat ega ole neil aimu minu auhiilgusest; ja need kuulutavad rahvaste keskel minu auhiilgust.
\par 20 Ja nad toovad kõik teie vennad kõigi rahvaste seast ohvriannina Issandale; nad toovad nad hobuste, vankrite ja tõldadega, muulade ja kärmete kaamelitega mu pühale mäele Jeruusalemma, ütleb Issand, otsekui Iisraeli lapsed toovad puhtais astjais roaohvri Issanda kotta.
\par 21 Ja nendegi hulgast ma võtan leviitpreestreid, ütleb Issand.
\par 22 Sest otsekui uus taevas ja uus maa, mis ma teen, püsivad minu palge ees, ütleb Issand, nõnda püsib ka teie sugu ja teie nimi.
\par 23 Ja noorkuust noorkuusse ning hingamispäevast hingamispäeva tuleb kogu inimsugu mu ette kummardama, ütleb Issand.
\par 24 Kui nad välja lähevad, siis nad näevad nende meeste laipu, kes astusid üles mu vastu; sest nende uss ei sure ja nende tuli ei kustu, ja nad on jälkuseks kogu inimsoole.”



\end{document}