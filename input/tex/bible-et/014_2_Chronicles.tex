\begin{document}

\title{Teine Ajaraamat}

\chapter{1}

\par 1 Ja Saalomon, Taaveti poeg, osutus tugevaks oma kuningriigis, ja Issand, tema Jumal, oli temaga ning tegi ta väga suureks.
\par 2 Ja Saalomon andis käsu kogu Iisraelile, tuhande- ja sajapealikuile, kohtumõistjaile ja kõigile juhtidele kogu Iisraelis, perekondade peameestele,
\par 3 ja Saalomon ning terve kogudus koos temaga läksid ohvrikünkale, mis oli Gibeonis, sest seal oli Jumala kogudusetelk, mille Issanda sulane Mooses kõrbes oli teinud.
\par 4 Ometi oli Taavet toonud Jumala laeka Kirjat-Jearimist paika, mille Taavet selleks oli valmistanud; sest ta oli Jeruusalemmas selle jaoks telgi üles löönud.
\par 5 Aga vaskaltar, mille Betsaleel, Huuri poja Uuri poeg, oli teinud, oli seal Issanda elamu ees; seal otsisid Saalomon ja kogudus Issandat.
\par 6 Ja Saalomon ohverdas seal Issanda ees kogudusetelgi vaskaltaril; ta ohverdas selle peal tuhat põletusohvrit.
\par 7 Selsamal ööl ilmutas Jumal ennast Saalomonile ja ütles talle: „Palu, mida ma sulle peaksin andma!”
\par 8 Ja Saalomon vastas Jumalale: „Sa oled Taavetile, mu isale, suurt heldust osutanud, ja oled minu tema asemele kuningaks tõstnud.
\par 9 Nüüd, Issand Jumal, saagu tõeks su sõna mu isale Taavetile, sest sa oled mu kuningaks tõstnud rahvale, keda on nõnda palju nagu põrmu maa peal!
\par 10 Anna nüüd mulle tarkust ja mõistust minna ja tulla selle rahva eesotsas, sest kes suudaks muidu kohut mõista sellele sinu suurele rahvale?”
\par 11 Ja Jumal ütles Saalomonile: „Et see sul südame peal on ja et sa ei ole palunud rikkust, varandust ega au, ka mitte oma vaenlaste hinge, ja et sa ei ole palunud isegi mitte pikka iga, vaid oled enesele palunud tarkust ja mõistust, et saaksid kohut mõista rahvale, kellele ma sind olen kuningaks tõstnud,
\par 12 siis olgu sulle antud tarkust ja mõistust! Ja ma annan sulle ka rikkust, varandust ja au, nagu seda ei ole olnud kuningail enne sind ega ole ühelgi pärast sind.”
\par 13 Ja Saalomon tuli ohvrikünkalt Gibeonist, kogudusetelgi eest, Jeruusalemma ja valitses Iisraeli üle.
\par 14 Ja Saalomon kogus sõjavankreid ja ratsanikke, ja tal oli tuhat nelisada sõjavankrit ja kaksteist tuhat ratsanikku; need paigutas ta vankrilinnadesse ja kuninga juurde Jeruusalemma.
\par 15 Ja kuningas hoolitses, et Jeruusalemmas oli hõbedat ja kulda nagu kive, ja seedripuid nõnda palju nagu metsviigipuid Madalmaal.
\par 16 Saalomoni hobused olid toodud Egiptusest ja Kiliikiast; kuninga ülesostjad tõid neid Kiliikiast kindla hinna eest.
\par 17 Egiptusest toodi vanker kuuesaja hõbeseekli eest ja hobune saja viiekümne eest; ja nõnda toodi neid nende vahendusel kõigile hettide ja süürlaste kuningaile.

\chapter{2}

\par 1 Ja Saalomon määras seitsekümmend tuhat meest kandjaiks, kaheksakümmend tuhat meest kiviraiujaiks mäestikus ja kolm tuhat kuussada, kes neid juhatasid.
\par 2 Ja Saalomon läkitas ütlema Huuramile, Tüürose kuningale: „Tee minule sedasama, mida sa tegid mu isale Taavetile, kellele sa saatsid seedripuid, et ta sai enesele ehitada koja, kus ta elas.
\par 3 Vaata, mina tahan ehitada Issanda, oma Jumala nimele koja ja selle pühitseda temale, et tema ees suitsutada hästilõhnavaid suitsutusrohte, tuua igapäevaseid ohvrileibu ja põletusohvreid hommikul ja õhtul, hingamispäevil, noorkuupäevil ja Issanda, meie Jumala pühadel; see on Iisraelile igaveseks kohustuseks.
\par 4 Ja koda, mille ma ehitan, peab olema suur, sest meie Jumal on suurem kõigist jumalaist!
\par 5 Aga kellel jätkuks jõudu temale koda ehitada? Sest taevad ja taevaste taevad ei mahuta teda! Ja kes olen siis mina, et ma tahan temale koja ehitada, kuigi ainult suitsutuseks tema ees?
\par 6 Aga nüüd läkita mulle mees, kes oskab teha töid kullast, hõbedast, vasest ja rauast, purpurpunasest, karmiinpunasest ja sinisest lõngast, ja kes tunneb nikerdustööd, et ta töötaks Juudas ja Jeruusalemmas koos minu oskustöölistega, kes on hangitud mu isa Taaveti poolt.
\par 7 Ja saada mulle seedripuid, küpressi- ja algumipuid Liibanonist, sest ma tean, et sinu sulased oskavad raiuda Liibanoni puid. Ja vaata, minu sulased olgu koos sinu sulastega!
\par 8 Mul on tarvis hankida palju puitu, sest koda, mille ma ehitan, peab olema suur ja imetlusväärne!
\par 9 Ja vaata, raiujaile, kes puid lõikavad, annan ma kakskümmend tuhat koori nisu, su sulastele toiduks, ja kakskümmend tuhat koori otri, kakskümmend tuhat batti veini ja kakskümmend tuhat batti õli!”
\par 10 Ja Huuram, Tüürose kuningas, vastas kirjaga, mille ta Saalomonile läkitas: „Sellepärast, et Issand oma rahvast armastab, on ta sinu pannud neile kuningaks.”
\par 11 Ja edasi ütles Huuram: „Kiidetud olgu Issand, Iisraeli Jumal, kes on teinud taeva ja maa, kes on andnud kuningas Taavetile targa poja, kel on tarkust ja taipu, et ehitada Issandale koda ja iseenesele kuninglik koda!
\par 12 Ja nüüd ma läkitan ühe targa ja aruka mehe, Huuram-Abi,
\par 13 kes on Daani tütreist pärit naise poeg ja kelle isa on Tüürose mees. Tema oskab teha töid kullast, hõbedast, vasest, rauast, kivist ja puust, ja purpurpunasest, sinisest, valgest linasest ja karmiinpunasest lõngast. Ta oskab igasuguseid nikerdusi ja lahendab igasuguseid ülesandeid, mis temale antakse, koos sinu tarkadega ja minu isanda, sinu isa Taaveti tarkadega.
\par 14 Saada siis nüüd oma sulastele nisu, otri, õli ja veini, nagu mu isand on lubanud.
\par 15 Meie raiume siis Liibanonist puid, nõnda palju kui sul on vaja, ja me toome need sulle parvedena merd mööda Jaafosse. Sina vii need siis üles Jeruusalemma!”
\par 16 Ja Saalomon luges ära kõik Iisraelimaal olevad võõrad mehed, pärast seda lugemist, mida tema isa Taavet oli toimetanud, ja neid leiti olevat sada viiskümmend kolm tuhat kuussada.
\par 17 Ja ta tegi neist seitsekümmend tuhat kandjaiks, kaheksakümmend tuhat kiviraiujaiks mäestikus ja kolm tuhat kuussada juhatajaiks, kes panid rahva tööle.

\chapter{3}

\par 1 Ja Saalomon hakkas Issanda koda ehitama Jeruusalemma Morija mäele, kus Taavetil, tema isal, oli olnud ilmutus, paika, mille Taavet oli kindlaks määranud jebuuslase Ornani rehealuse kohale.
\par 2 Ta hakkas ehitama teise kuu teisel päeval, oma valitsemise neljandal aastal.
\par 3 Ja need on mõõdud, mis Saalomon Jumala koja ehituseks määras: pikkus, vana küünramõõdu järgi, kuuskümmend küünart ja laius kakskümmend küünart.
\par 4 Eeskoda selle ees oli, vastavalt koja laiusele, kakskümmend küünart pikk ja selle kõrgus oli sada kakskümmend küünart; ja ta kattis selle seestpoolt puhta kullaga.
\par 5 Ta vooderdas suure koja küpressipuuga ja kattis selle parima kullaga; ta tegi selle peale palmid ja pärjad.
\par 6 Ta kattis koja ilustuseks kalliskividega; ja kuld oli Parvaimi kuld.
\par 7 Ta kattis koja - palgid, läved, seinad ja uksed - kullaga ja nikerdas seinte peale keerubid.
\par 8 Ja ta tegi kõige pühama paiga; selle pikkus oli, vastavalt koja laiusele, kakskümmend küünart ja laius kakskümmend küünart; ta kattis selle parima kullaga, mida oli kuussada talenti.
\par 9 Ja kuldnaelad vaagisid viiskümmend seeklit; ka ülakambrid kattis ta kullaga.
\par 10 Ja kõige pühamasse paika valmistas ta valatud tööna kaks keerubit, need kaeti kullaga.
\par 11 Ja keerubite tiibade pikkus oli kokku kakskümmend küünart: üks tiib, viieküünrane, puudutas koja seina, ja teine tiib, viieküünrane, puudutas teise keerubi tiiba.
\par 12 Ja teise keerubi viieküünrane tiib puudutas koja seina ja teine viieküünrane tiib puutus kokku teise keerubi tiivaga.
\par 13 Nende keerubite tiivad olid laialilaotatult kakskümmend küünart; need seisid oma jalgade peal ja nende näod olid koja poole.
\par 14 Ja ta valmistas eesriide sinisest, purpurpunasest ja karmiinpunasest lõngast ja valgest linasest ning tegi selle peale keerubid.
\par 15 Ja ta valmistas koja ette kaks sammast, kolmkümmend viis küünart kõrged; ja nupp, mis oli nende otsas, oli viieküünrane.
\par 16 Ja ta valmistas ketid nagu tagaruumis ning pani need sammaste otsa; ta valmistas sada granaatõuna ja pani need kettide külge.
\par 17 Ja ta püstitas templi ette sambad: ühe paremale poole ja teise vasakule poole; ta pani parempoolsele nimeks Jaakin ja vasakpoolsele nimeks Boas.

\chapter{4}

\par 1 Ja ta valmistas vaskaltari, kakskümmend küünart pika, kakskümmend küünart laia ja kümme küünart kõrge.
\par 2 Ja ta valmistas valatud vaskmere, kümme küünart äärest ääreni, täiesti ümmarguse, viis küünart kõrge; kolmekümneküünrane mõõdunöör ulatus selle ümber.
\par 3 Ja selle all olid härjakujud ümberringi; need ümbritsesid seda, kümme igal küünral, ringi ümber vaskmere; härjad olid kahes reas, selle valuga koos valatud.
\par 4 See seisis kaheteistkümnel härjal: kolm vaatasid põhja poole, kolm vaatasid lääne poole, kolm vaatasid lõuna poole ja kolm vaatasid ida poole; vaskmeri oli ülal nende peal ja neil kõigil olid tagumikud sissepoole.
\par 5 See oli kämblapaksune ja selle äär oli tehtud karika ääre sarnaselt liilia õiena; see mahutas kolm tuhat batti.
\par 6 Ja ta valmistas kümme pesunõu ning pani viis paremale ja viis vasakule poole pesemise jaoks; neis pidi loputatama seda, mis kuulus põletusohvri juurde, kuna vaskmeri oli preestritele pesemise jaoks.
\par 7 Ja ta valmistas kümme kuldlambijalga, nagu need pidid olema, ja pani need templisse, viis paremale ja viis vasakule poole.
\par 8 Ja ta valmistas kümme lauda ning asetas need templisse, viis paremale ja viis vasakule poole; ja ta valmistas sada kuldpiserdusnõu.
\par 9 Ja ta tegi preestrite õue ja suure õue ning õue uksed; ta kattis uksed vasega.
\par 10 Ja ta asetas vaskmere paremasse tiiba, kagusse.
\par 11 Ja Huuram valmistas tuhanõud, -labidad ja piserdusnõud. Ja Huuram sai valmis töö, mis ta pidi tegema kuningas Saalomonile Jumala koja jaoks:
\par 12 kaks sammast ja kaks kausikujulist nuppu, mis olid sammaste otsas; ja kaks võrestikku mõlema kausikujulise nupu katteks, mis olid sammaste otsas;
\par 13 ja nelisada granaatõuna mõlema võrestiku jaoks, kaks rida granaatõunu kummalegi võrestikule, katteks mõlemale kausikujulisele nupule, mis olid sammaste otsas.
\par 14 Ja ta valmistas alused ning valmistas pesunõud aluste peale;
\par 15 ühe vaskmere ja kaksteist härga selle all.
\par 16 Ja tuhanõud, -labidad ja hargid ja kõik nende juurde kuuluvad riistad valmistas Huuram-Abi kuningas Saalomonile Issanda koja tarvis hiilgavast vasest.
\par 17 Kuningas laskis need valada Jordani uhtmaal savimaa sees Sukkoti ja Seredata vahel.
\par 18 Ja Saalomon valmistas kõiki neid riistu väga palju, sest vase kaalu ei arvestatudki.
\par 19 Saalomon valmistas ka kõik need riistad, mis pidid olema Jumala kojas: kuldaltari ja lauad, mille peal olid ohvrileivad;
\par 20 lambijalad ja nende lambid, vastavas korras süütamiseks kõige pühama paiga ees, puhtast kullast;
\par 21 kuldsed õied, lambid ja tahikäärid puhtast kullast;
\par 22 noad, piserdusnõud, peekrid ja sütepannid puhtast kullast; ja kullast ka koja uste esiküljed kõige pühama paiga sisemistel ja templilöövi ustel.

\chapter{5}

\par 1 Kui kõik tööd, mis Saalomon tegi Issanda koja jaoks, valmis said, viis Saalomon kotta kõik, mis ta isa Taavet oli pühitsenud: ta pani hõbeda ja kulla ja kõik riistad Jumala koja varanduste hulka.
\par 2 Siis kogus Saalomon Jeruusalemma Iisraeli vanemad ja kõik suguharude peamehed, Iisraeli laste perekondade eestseisjad, Issanda seaduselaegast Taaveti linnast, see on Siionist, üles tooma.
\par 3 Ja kõik Iisraeli mehed kogunesid kuninga juurde pühiks, mis on seitsmendas kuus.
\par 4 Ja kui kõik Iisraeli vanemad olid tulnud, siis tõstsid leviidid laeka
\par 5 ja tõid üles laeka, kogudusetelgi ja kõik pühad riistad, mis telgis olid; leviitpreestrid tõid need üles.
\par 6 Ja kuningas Saalomon ning terve Iisraeli kogudus, kes tema juurde oli kogunenud, olid laeka ees; nad ohverdasid nii palju lambaid, kitsi ja veiseid, et neid ei saadud lugeda ega kokku arvata.
\par 7 Ja preestrid tõid Issanda seaduselaeka selle paika koja tagaruumi, kõige pühamasse paika keerubite tiibade alla.
\par 8 Ja keerubid laotasid tiibu laeka paiga üle; keerubid katsid laegast ja selle kandekange pealtpoolt.
\par 9 Ja kangid olid nii pikad, et laekast eemaleulatuvaid kangide otsi oli näha kõige pühama paiga eest; ent neid ei nähtud väljastpoolt; ja need on seal tänapäevani.
\par 10 Laekas ei olnud muud kui need kaks lauda, mis Mooses oli Hoorebil sinna pannud, kui Issand tegi Iisraeli lastega lepingu nende tulles Egiptusest.
\par 11 Kui preestrid pühamust väljusid - sest kõik sealolevad preestrid olid endid pühitsenud rühmadele vaatamata,
\par 12 ja kõik lauljad leviidid, Aasaf, Heeman, Jedutuun ja nende pojad ning vennad, seisid valgeis linaseis riideis simblite, naablite ja kanneldega ida pool altarit, ja koos nendega sada kakskümmend preestrit, kes puhusid pasunaid -
\par 13 pasunad ja lauljad pidid ühekorraga ja kooskõlas kuuldavale tooma kiitust ja tänu Issandale -, ja kui nad häält tõstsid pasunate, simblite ja muude mänguriistadega ning Issanda kiituslauluga: „Sest tema on hea, sest tema heldus kestab igavesti!”, siis täitis pilv koja, Issanda koja,
\par 14 ja preestrid ei võinud teenima jääda pilve pärast: sest Issanda auhiilgus oli täitnud Jumala koja.

\chapter{6}

\par 1 Siis kõneles Saalomon: „Issand on öelnud, et ta tahab elada pimeduses.
\par 2 Mina olen sulle ehitanud valitsuskoja, su igavese eluaseme paiga.”
\par 3 Ja kuningas pööras oma palge ja õnnistas kogu Iisraeli kogudust, ja kogu Iisraeli kogudus seisis.
\par 4 Ja ta ütles: „Kiidetud olgu Issand, Iisraeli Jumal, kes oma kätega on tõeks teinud, mida ta oma suuga on kõnelnud mu isale Taavetile, öeldes:
\par 5 Sellest päevast alates, kui ma tõin oma rahva ära Egiptusemaalt, ei ole ma üheltki Iisraeli suguharult valinud ühtegi linna, kuhu ehitada koda, kus oleks mu nimi, ega ole ma ka valinud ühtegi meest, kes oleks olnud vürstiks mu Iisraeli rahvale.
\par 6 Aga ma olen valinud Jeruusalemma, et seal oleks mu nimi, ja ma olen valinud Taaveti, et ta valitseks mu Iisraeli rahva üle.
\par 7 Mu isal Taavetil oli küll südame peal ehitada koda Issanda, Iisraeli Jumala nimele.
\par 8 Aga Issand ütles mu isale Taavetile: Et sul südame peal on ehitada mu nimele koda, siis oled sa hästi teinud, kui see sul südame peal on.
\par 9 Ometi ei ehita sina koda, vaid su poeg, kes su niudeist välja tuleb, ehitab minu nimele koja.
\par 10 Ja Issand on pidanud oma sõna, mis ta kõneles, ja mina olen tõusnud oma isa Taaveti asemele ja istun Iisraeli aujärjel, nõnda nagu Issand on kõnelnud, ja ma olen ehitanud koja Issanda, Iisraeli Jumala nimele.
\par 11 Ja ma olen sinna pannud laeka, mille sees on Issanda seadus, mille ta Iisraeli lastele andis.”
\par 12 Siis ta astus Issanda altari ette kogu Iisraeli koguduse juuresolekul ja sirutas oma käed välja.
\par 13 Saalomon oli valmistanud vasklava, viis küünart pika, viis küünart laia ja kolm küünart kõrge, ja oli selle pannud keset õue. Ja ta astus selle peale, heitis põlvili kogu Iisraeli koguduse nähes, sirutas oma käed taeva poole
\par 14 ja ütles: „Issand, Iisraeli Jumal! Sinu sarnast jumalat ei ole taevas ega maa peal, sina pead lepingut ja osadust oma sulastega, kes käivad su ees kõigest südamest,
\par 15 sina oled pidanud oma sulasele, mu isale Taavetile, mis sa temale olid lubanud. Jah, mis sa oma suuga oled kõnelnud, selle oled sa oma käega tõeks teinud, nõnda nagu see täna on sündinud.
\par 16 Ja nüüd, Issand, Iisraeli Jumal, pea oma sulasele, mu isale Taavetile, mis sa temale tõotasid, öeldes: Ei puudu sul minu palge ees mees, kes istub Iisraeli aujärjel, kui ainult su pojad hoiavad oma teed, käies mu Seaduse järgi, nõnda nagu sina oled käinud mu ees.
\par 17 Ja nüüd, Issand, Iisraeli Jumal, saagu tõeks su sõna, mis sa oled kõnelnud oma sulasele Taavetile!
\par 18 Aga kas Jumal tõesti peaks elama koos inimestega maa peal? Vaata, taevad ja taevaste taevad ei mahuta sind, veel vähem siis see koda, mille ma olen ehitanud.
\par 19 Aga pöördu oma sulase palve ja ta anumise poole, Issand, mu Jumal, et sa kuuleksid kaebehüüdu ja palvet, mida su sulane palvetab sinu ees,
\par 20 et su silmad oleksid lahti päevad ja ööd selle koja kohal, selle paiga kohal, mille kohta sa oled öelnud, et sa paned sinna oma nime ja kuuled palvet, mida su sulane palvetab selle paiga poole!
\par 21 Kuule siis oma sulase ja oma Iisraeli rahva anumist, kuidas nad palvetavad selle paiga poole! Jah, Kuule paigast, kus sa elad - taevast! Ja kui sa kuuled, siis anna andeks!
\par 22 Kui keegi oma ligimese vastu pattu teeb ja talle pannakse peale vanne teda vannutades, ja vandeasi tuleb sinu altari ette siia kotta,
\par 23 siis kuule sina taevast ja tee ning mõista õigust oma sulastele: mõista süüdlane süüdi, pane tema teod ta pea peale, õige aga mõista õigeks, anna temale ta õigust mööda!
\par 24 Ja kui su Iisraeli rahvas lüüakse maha vaenlase ees, sellepärast et nad sinu vastu on pattu teinud, ent nad pöörduvad ja kiidavad sinu nime, palvetavad ja anuvad su ees siin kojas,
\par 25 siis kuule sina taevast ja anna andeks oma Iisraeli rahva patt ja too nad tagasi maale, mille sa oled andnud neile ja nende vanemaile!
\par 26 Kui taevas on suletud ja vihma ei ole, sellepärast et nad on sinu vastu pattu teinud, aga nad palvetavad selle paiga poole ja kiidavad sinu nime ja pöörduvad oma patust, sellepärast et sa neid oled alandanud,
\par 27 siis kuule sina taevast ja anna andeks oma sulaste ja oma Iisraeli rahva patt, sest sina õpetad neile head teed, mida nad peavad käima, ja anna vihma oma maale, mille sa oled andnud pärisosaks oma rahvale!
\par 28 Kui maale tuleb nälg, katk, kui tuleb viljakõrvetus või -rooste, kui tulevad rohutirtsud ja mardikad, kui vaenlased teda rõhuvad ta maa väravais, kui tabab mingi nuhtlus või mingi haigus,
\par 29 siis iga palvet ja iga anumist, mis iganes tuleb mõne inimese või kogu su Iisraeli rahva poolt, kui nad igaüks tunnevad oma häda ja valu ja sirutavad oma käed selle koja poole,
\par 30 kuule sina siis taevast, oma asupaigast, ja anna andeks ning anna igaühele tema tegusid mööda, nõnda nagu sa tunned tema südant, sest sina üksi tunned inimlaste südant,
\par 31 et nad kardaksid sind ja käiksid su teedel kõigil neil päevil, mil nad elavad maal, mille sa oled andnud meie vanemaile!
\par 32 Aga ka võõramaalast, kes ei ole sinu Iisraeli rahva hulgast, tuleb aga kaugelt maalt sinu suure nime, vägeva käe ja väljasirutatud käsivarre pärast - kui ta tuleb ja palvetab selle koja poole -,
\par 33 kuule sina taevast, oma asupaigast, ja tee kõike, mille pärast võõras sinu poole hüüab, et kõik maa rahvad õpiksid tundma sinu nime ja kardaksid sind, nõnda nagu su Iisraeli rahvas, ja et nad teaksid, et sinu nimi on pandud kojale, mille ma olen ehitanud!
\par 34 Kui su rahvas läheb sõtta oma vaenlaste vastu, teekonnale, kuhu sa nad läkitad, ja nad palvetavad sinu poole selle linna suunas, mille sina oled valinud, ja koja suunas, mille mina olen ehitanud su nimele,
\par 35 siis kuule taevast nende palvet ja anumist ja tee neile õigust!
\par 36 Kui nad sinu vastu pattu teevad - sest pole inimest, kes pattu ei tee - ja sina vihastud nende peale ning annad nad vaenlase kätte, nõnda et nende vangistajad viivad neid vangi kaugemale või lähemale maale,
\par 37 aga kui nad siis seda südamesse võtavad maal, kuhu nad on vangi viidud, ja pöörduvad ning anuvad sind oma vangistusemaal, öeldes: Me oleme pattu teinud, oleme eksinud ja saanud süüdlasteks!
\par 38 ja pöörduvad sinu poole kõigest oma südamest ja kõigest oma hingest oma vangistusemaal, kuhu nad on vangi viidud, ja palvetavad oma maa suunas, mille sa oled andnud nende vanemaile, ja linna suunas, mille sina oled valinud, ja koja suunas, mille mina olen ehitanud sinu nimele,
\par 39 siis kuule taevast, oma asupaigast, nende palvet ja anumist, tee neile õigust ja anna andeks oma rahvale, kes sinu vastu on pattu teinud!
\par 40 Nüüd, mu Jumal, olgu siis su silmad lahti ja su kõrvad pangu tähele palvet selles paigas!
\par 41 Ja nüüd, Issand Jumal, tõuse, tule oma hingamispaika, sina ja su võimsuselaegas! Su preestrid, Issand Jumal, olgu õnnistusega ehitud, ja su vagad tundku rõõmu sellest, mis on hea!
\par 42 Issand Jumal, ära hülga oma võitut! Mõtle oma sulase Taaveti osadusele!”

\chapter{7}

\par 1 Kui Saalomon oma palve oli lõpetanud, siis tuli taevast tuli ja põletas põletusohvri ja tapaohvrid, ja Issanda auhiilgus täitis koja.
\par 2 Ja preestrid ei võinud minna Issanda kotta, sest Issanda auhiilgus oli täitnud Issanda koja.
\par 3 Ja kui kõik Iisraeli lapsed nägid, kuidas tuli alla tuli ja Issanda auhiilgus oli koja kohal, siis heitsid nad silmili maha kivipõrandale, kummardasid ja tänasid Issandat, et tema on hea, et tema heldus kestab igavesti.
\par 4 Siis ohverdasid kuningas ja kogu rahvas Issanda ees tapaohvreid.
\par 5 Kuningas Saalomon ohverdas tapaohvriks kakskümmend kaks tuhat veist ning sada kakskümmend tuhat lammast ja kitse; nõnda pühitsesid nad, Kuningas ja kogu rahvas, Jumala koja.
\par 6 Ja preestrid seisid oma ameteis, nõndasamuti leviidid Issanda mänguriistadega, mis kuningas Taavet oli teinud, et Issandat tänada tema igavesti kestva helduse eest, iga kord kui tema, Taavet, tahtis nende abiga kiitust avaldada; nendega vastamisi puhusid preestrid pasunaid ja kogu Iisrael seisis püsti.
\par 7 Ja Saalomon pühitses Issanda koja ees oleva õue keskpaiga, sest ta ohverdas seal põletusohvreid ja tänuohvrite rasvu, kuna vaskaltar, mille Saalomon oli teinud, ei suutnud mahutada põletusohvreid, roaohvreid ja rasvu.
\par 8 Saalomon pidas sel ajal püha seitse päeva ja koos temaga kogu Iisrael; see oli väga suur kogudus Hamati teelahkmest kuni Egiptuseojani.
\par 9 Kaheksandal päeval pidasid nad lõpetuspüha, sest nad pidasid altari pühitsemist seitse päeva ja püha seitse päeva.
\par 10 Aga seitsmenda kuu kahekümne kolmandal päeval saatis ta rahva nende telkidesse; rahvas oli rõõmus ja heas meeleolus selle hea pärast, mida Issand oli teinud Taavetile, Saalomonile ja oma Iisraeli rahvale.
\par 11 Kui Saalomon oli valmis saanud Issanda koja ja kuningakoja, ja kui kõik, mida Saalomon oli tahtnud teha Issanda kojas ja oma kojas, oli korda läinud,
\par 12 siis ilmutas Issand ennast öösel Saalomonile ja ütles temale: „Ma olen kuulnud su palvet ja olen valinud selle paiga enesele ohvrikojaks.
\par 13 Vaata, kui ma sulen taeva, nõnda et vihma ei saja, ja vaata, kui ma käsin rohutirtse maa paljaks süüa, või kui ma läkitan oma rahva kallale katku,
\par 14 ja kui siis minu rahvas, kellele on pandud minu nimi, alandab ennast ja nad palvetavad ja otsivad minu palet ning pöörduvad oma kurjadelt teedelt, siis ma kuulen taevast ja annan andeks nende patu ning säästan nende maa.
\par 15 Nüüd on mu silmad lahti ja mu kõrvad panevad tähele palvetamist selles paigas.
\par 16 Nüüd olen ma valinud ja pühitsenud selle koja, et minu nimi oleks seal igavesti. Mu silmad ja mu süda on seal iga päev.
\par 17 Ja kui sina käid mu palge ees, nõnda nagu käis su isa Taavet, ja teed kõike, mida ma sind käsin, ja pead mu määrusi ja seadlusi,
\par 18 siis ma kinnitan su kuningriigi aujärje, nõnda nagu ma tegin lepingu su isa Taavetiga, öeldes: Sul ei puudu mees, kes valitseb Iisraeli üle!
\par 19 Aga kui te taganete ja hülgate mu seadlused ja mu käsud, mis ma teile olen andnud, ja lähete ning teenite teisi jumalaid ja kummardate neid,
\par 20 siis ajan ma seesugused välja oma maalt, mille ma neile olen andnud, ja koja, mille ma olen pühitsenud oma nimele, heidan ma ära oma palge eest ning teen selle kõnekäänuks ja pilkesõnaks kõigi rahvaste keskel.
\par 21 Ja kui kõrge see koda ka oli, ometi kohkub igaüks, kes sellest mööda läheb, ja küsib: Mispärast talitas Issand nõnda selle maa ja selle kojaga?
\par 22 Ja siis vastatakse: Sellepärast, et nad jätsid maha Issanda, oma vanemate Jumala, kes tõi nad ära Egiptusemaalt, ja haarasid teiste jumalate järele ning kummardasid ja teenisid neid. Sellepärast on ta lasknud nende peale tulla kogu selle õnnetuse.”

\chapter{8}

\par 1 Möödus kakskümmend aastat, mille jooksul Saalomon ehitas Issanda koja ja oma koja.
\par 2 Linnad, mis Huuram oli Saalomonile andnud, ehitas Saalomon üles ja pani neisse Iisraeli lapsed elama.
\par 3 Siis läks Saalomon Hamat-Soobasse ja vallutas selle.
\par 4 Ja ta kindlustas kõrbes Tadmori ja kõik varustuslinnad, mis ta ehitas Hamatis.
\par 5 Ja ta ehitas ülemise Beet-Hooroni ja alumise Beet-Hooroni kindlustatud linnadeks müüride, väravate ja riividega,
\par 6 samuti Baalati ja kõik varustuslinnad, mis Saalomonil olid, ja kõik sõjavankrite linnad, ja ratsanike linnad, ja kõik, mis Saalomon soovis ehitada Jeruusalemmas ja Liibanonis ning kõikjal oma valitsetaval maal.
\par 7 Kogu selle rahva, kes oli alles jäänud hettidest, emorlastest, perislastest, hiivlastest ja jebuuslastest, need, kes ei olnud Iisraelist,
\par 8 nende järglased, kes pärast neid olid maale alles jäänud, keda Iisraeli lapsed ei olnud hävitanud, need pani Saalomon teoorjusesse kuni tänapäevani.
\par 9 Aga Iisraeli lastest ei teinud Saalomon kedagi oma tööorjaks, vaid nemad olid sõjamehed, tema sangarite pealikud, tema sõjavankrite ja ratsanike pealikud.
\par 10 Ja neid kuningas Saalomoni linnavägede pealikuid, kes valitsesid rahva üle, oli kakssada viiskümmend.
\par 11 Ja Saalomon tõi vaarao tütre Taaveti linnast kotta, mille ta temale oli ehitanud, sest ta ütles: „Ükski naine ei tohi mul elada Iisraeli kuninga Taaveti kojas, sest need paigad, kus Issanda laegas on olnud, on pühad!”
\par 12 Siis ohverdas Saalomon Issandale põletusohvreid Issanda altaril, mille ta oli ehitanud eeskoja ette,
\par 13 ohverdades vastavaks päevaks määratud ohvri, nagu Mooses oli käskinud, hingamispäevil, noorkuupäevil ja seatud pühadel kolm korda aastas: hapnemata leibade pühal, nädalatepühal ja lehtmajadepühal.
\par 14 Ja ta seadis oma isa Taaveti korra järgi preestrite rühmad nende teenistusse ja leviidid nende kohustustesse kiituslaule laulma ja preestrite juures igapäevaseid ülesandeid toimetama, ja väravahoidjad rühmade kaupa iga värava juurde, sest nõnda oli jumalamehe Taaveti käsk.
\par 15 Ja kuninga käsust preestrite, leviitide ning varakambrite kohta ei taganetud üheski asjas.
\par 16 Nõnda tehti kõik Saalomoni tööd Issanda koja rajamise päevani ja kuni selle valmissaamiseni; siis oli Issanda koda täiuslik.
\par 17 Siis läks Saalomon Esjon-Geberisse ja Eelotisse, mereranda Edomimaale.
\par 18 Ja Huuram läkitas temale oma sulastega laevu ning sulaseid, kes merd tundsid; ja need läksid koos Saalomoni sulastega Oofiri ja võtsid sealt nelisada viiskümmend talenti kulda ning tõid kuningas Saalomonile.

\chapter{9}

\par 1 Kui Seeba kuninganna kuulis Saalomoni kuulsusest, siis tuli ta Jeruusalemma Saalomoni mõistuküsimustega kimbutama; ta tuli väga suure saatjaskonnaga ja kaamelitega, kes kandsid palsameid, väga palju kulda ja kalliskive. Ja kui ta jõudis Saalomoni juurde, siis kõneles ta temaga kõigest, mis tal südame peal oli.
\par 2 Aga Saalomon vastas kõigile tema küsimustele; ükski tema küsimus ei olnud Saalomonile liiga raske; ta suutis vastata kõigile.
\par 3 Kui Seeba kuninganna nägi Saalomoni tarkust, ja koda, mille ta oli ehitanud,
\par 4 rooga ta laual ja kuidas ta sulased istusid, tema teenrite teenimist ja nende riietust, tema joogikallajaid ja nende riietust ja tema ülesminemist, kuidas ta läks üles Issanda kotta, siis jäi ta otse hingetuks.
\par 5 Ja ta ütles kuningale: „See kuuldus oli tõsi, mis ma oma maal kuulsin sinust ja su tarkusest.
\par 6 Aga ma ei uskunud neid jutte mitte enne, kui ma tulin ja nägin oma silmaga; ja vaata, mulle ei olnud räägitud pooltki sinu suurest tarkusest. Sa ületasid kuulduse, mida ma olin kuulnud.
\par 7 Õnnelikud on su mehed ja õnnelikud on need su sulased, kes seisavad alati su ees ja kuulevad su tarkust!
\par 8 Kiidetud olgu Issand, su Jumal, kellele sa nõnda meeldisid, et ta pani su oma aujärjele kuningaks Issanda, su Jumala ees! Sellepärast, et su Jumal armastab Iisraeli ja tahab temale anda igavest püsi, on ta sind pannud neile kuningaks, et sa teeksid, mis on kohus ja õige.”
\par 9 Ja ta andis kuningale sada kakskümmend talenti kulda ja väga palju palsameid ja kalliskive; iialgi enam ei ole olnud seesuguseid palsameid kui need, mis Seeba kuninganna andis kuningas Saalomonile.
\par 10 Aga ka Huurami sulased ja Saalomoni sulased, kes tõid kulda Oofirist, tõid algumipuid ja kalliskive.
\par 11 Ja kuningas tegi algumipuust trepid Issanda kojale ja kuningakojale, samuti kandleid ja naableid lauljaile; seesuguseid ei olnud Juudamaal enne nähtud.
\par 12 Ja kuningas Saalomon andis Seeba kuningannale kõike, mida ta soovis ja palus, aga mitte seda, mis ta kuningale oli toonud; siis kuninganna pöördus ümber ja läks tagasi oma maale, tema ja ta sulased.
\par 13 Ja kulla kaal, mis tuli Saalomoni kätte ühe aasta jooksul, oli kuussada kuuskümmend kuus talenti kulda,
\par 14 lisaks see, mida tõid suurkaupmehed ja kaubitsejad; ka kõik Araabia kuningad ja maa asevalitsejad tõid Saalomonile kulda ja hõbedat.
\par 15 Ja kuningas Saalomon valmistas kakssada kilpi taotud kullast, tarvitades kuussada seeklit taotud kulda üheks kilbiks;
\par 16 ja kolmsada väiksemat kilpi taotud kullast, tarvitades kolmsada seeklit kulda üheks kilbiks; ja kuningas pani need Liibanonimetsakotta.
\par 17 Ja kuningas valmistas suure elevandiluust aujärje ning kattis selle puhta kullaga.
\par 18 Aujärjel oli kuus astet ja kullast jalajäri, mis oli kinnitatud aujärje külge; kummalgi pool istepaika olid käetoed ja kaks lõvi seisis käetugede kõrval.
\par 19 Ja seal seisis kaksteist lõvi kuuel astmel kummalgi pool; seesugust pole valmistatud mitte üheski muus kuningriigis.
\par 20 Ja kuningas Saalomoni kõik joogiriistad olid kullast, samuti olid Liibanonimetsakoja riistad puhtast kullast; hõbedat ei peetud Saalomoni ajal mikski,
\par 21 sest kuningal oli laevu, mis käisid Tarsises Huurami sulastega; igal kolmandal aastal tulid Tarsise laevad, tuues kulda ja hõbedat, elevandiluud, pärdikuid ja paabulinde.
\par 22 Ja kuningas Saalomon sai rikkuse ja tarkuse poolest suurimaks kõigist kuningaist maa peal.
\par 23 Ja kõik maailma kuningad püüdsid näha saada Saalomoni palet, et kuulda tema tarkust, mille Jumal oli pannud tema südamesse.
\par 24 Ja nad tõid igaüks oma anni: hõbe- ja kuldriistu, riideid, relvi, palsameid, hobuseid ja muulasid; nõnda aastast aastasse.
\par 25 Ja Saalomonil oli neli tuhat latrit hobuste ja vankrite tarvis, ja kaksteist tuhat ratsanikku; ta paigutas need vankrilinnadesse ja kuninga juurde Jeruusalemma.
\par 26 Ja ta valitses kõigi kuningate üle Frati jõest kuni vilistite maa ja Egiptuse piirini.
\par 27 Ja kuningas hoolitses, et Jeruusalemmas oli hõbedat nagu kive, ja seedripuid nõnda palju nagu metsviigipuid Madalmaal.
\par 28 Ja hobuseid toodi Saalomonile Egiptusest ja kõigist muudest maadest.
\par 29 Ja muud Saalomoni lood, varasemad ja hilisemad, eks need ole kirja pandud prohvet Naatani „Kõnedes„, siilolase Ahija „Kuulutuses” ja nägija Jeddo ”Nägemustes Jerobeami, Nebati poja kohta”?
\par 30 Ja Saalomon valitses Jeruusalemmas kogu Iisraeli üle nelikümmend aastat.
\par 31 Siis Saalomon läks magama oma vanemate juurde ja ta maeti oma isa Taaveti linna. Ja tema poeg Rehabeam sai tema asemel kuningaks.

\chapter{10}

\par 1 Ja Rehabeam läks Sekemisse, sest kogu Iisrael oli tulnud Sekemisse teda kuningaks tõstma.
\par 2 Kui Jerobeam, Nebati poeg, seda kuulis - tema oli Egiptuses, kuhu ta oli põgenenud kuningas Saalomoni eest -, siis Jerobeam tuli Egiptusest tagasi.
\par 3 Ja nad läkitasid käskjalad talle järele ning kutsusid ta; ja Jerobeam ja kogu Iisrael tulid ning rääkisid Rehabeamiga, öeldes:
\par 4 „Sinu isa tegi meie ikke raskeks. Aga kergenda nüüd sina oma isa rasket teenistust ja tema ränka iket, mille ta meile on peale pannud, siis me teenime sind!”
\par 5 Aga tema vastas neile: „Tulge kolme päeva pärast jälle tagasi mu juurde!” Ja rahvas läks ära.
\par 6 Ja kuningas Rehabeam pidas nõu vanematega, kes olid seisnud tema isa Saalomoni teenistuses, kui too veel elas, öeldes: „Mis nõu te annate, et sellele rahvale midagi vastata?”
\par 7 Nad vastasid temale, öeldes: „Kui sa oled selle rahva vastu hea, meelitad neid ja räägid neile häid sõnu, siis jäävad nad sulle sulaseiks kogu eluajaks.”
\par 8 Aga ta hülgas vanemate nõu, mida need temale andsid, ja pidas nõu noorematega, kes olid kasvanud koos temaga ja seisid tema teenistuses.
\par 9 Ta ütles neile: „Mis nõu teie annate, et saaksime midagi vastata sellele rahvale, kes minuga kõneles, öeldes: Kergenda iket, mille su isa on meile peale pannud!?”
\par 10 Ja nooremad, kes olid kasvanud koos temaga, vastasid temale, öeldes: „Ütle nõnda rahvale, kes sinuga on kõnelnud ja sulle on öelnud: Su isa tegi meie ikke raskeks, aga kergenda sina seda meile - ütle neile nõnda: Minu väike sõrm on jämedam kui mu isa puusad.
\par 11 Ja kui nüüd mu isa on teile peale pannud ränga ikke, siis mina annan teie ikkele veel lisa! Kui mu isa karistas teid piitsadega, siis mina karistan teid okaspiitsadega!”
\par 12 Ja Jerobeam ja kogu rahvas tulid kolmandal päeval Rehabeami juurde, nagu kuningas oli käskinud, öeldes: „Tulge kolmandal päeval tagasi minu juurde!”
\par 13 Ja kuningas vastas neile karmilt. kuningas Rehabeam hülgas vanemate nõu
\par 14 ja kõneles neile nooremate nõu järgi, öeldes: „Mu isa on teie ikke teinud rängaks, aga mina annan sellele lisa! Mu isa karistas teid piitsadega, aga mina karistan okaspiitsadega!”
\par 15 Ja kuningas ei võtnud kuulda rahvast, sest see pööre oli Jumalalt, et Issand saaks tõeks teha oma sõna, mis ta siilolase Ahija läbi oli kõnelnud Jerobeamile, Nebati pojale.
\par 16 Kui kogu Iisrael nägi, et kuningas ei võtnud neid kuulda, siis vastas rahvas kuningale ja ütles: „Mis osa on meil Taavetis? Ei ole meil pärisosa Iisai pojas! Igamees oma telkidesse, Iisrael! Karjata nüüd omaenese sugu, Taavet!” Ja kogu Iisrael läks oma telkidesse.
\par 17 Siis Rehabeam valitses nende Iisraeli laste üle, kes elasid Juuda linnades.
\par 18 Ja kui kuningas Rehabeam läkitas Hadorami, kes oli sunnitöö ülevaataja, siis Iisraeli lapsed viskasid teda kividega ja ta suri. kuningas Rehabeam ise aga suutis astuda vankrisse ja põgeneda Jeruusalemma.
\par 19 Nõnda on Iisrael Taaveti soost taganenud kuni tänapäevani.

\chapter{11}

\par 1 Ja kui Rehabeam tuli Jeruusalemma, siis ta kogus kokku Juuda ja Benjamini soo, sada kaheksakümmend tuhat valitud sõjakõlvulist meest, et sõdida Iisraeli vastu ja taastada kuningriik Rehabeamile.
\par 2 Aga jumalamees Semajale tuli Issanda sõna, kes ütles:
\par 3 „Räägi Rehabeamiga, Saalomoni pojaga, Juuda kuningaga, ja kogu Iisraeliga Juudas ja Benjaminis ning ütle:
\par 4 Nõnda ütleb Issand: Te ei tohi minna ega sõdida oma vendade vastu! Igamees mingu tagasi koju, sest see lugu on lastud sündida minu poolt!” Ja nad võtsid kuulda Issanda sõna ja läksid tagasi ega läinud Jerobeami vastu.
\par 5 Ja Rehabeam elas Jeruusalemmas ning ehitas Juudas kindlustatud linnu.
\par 6 Siis ta ehitas Petlemma, Eetami, Tekoa,
\par 7 Beet-Suuri, Sooko, Adullami,
\par 8 Gati, Maaresa, Siifi,
\par 9 Adoraimi, Laakise, Aseka,
\par 10 Sora, Ajjaloni ja Hebroni kindlustatud linnadeks Juudas ja Benjaminis.
\par 11 Ta tegi linnused tugevaks, paigutas neisse vürstid ning toidu-, õli- ja veinivarud;
\par 12 ja igasse linna kilpe ja piike; ta tegi need väga tugevaks. Juuda ja Benjamin jäid temale.
\par 13 Ja preestrid ja leviidid, kes olid kogu Iisraelis, tulid tema juurde kõigist oma paigust.
\par 14 Leviidid jätsid maha oma karjamaad ja omandi ning läksid Juudasse ja Jeruusalemma, kuna Jerobeam ja ta pojad olid nad Issanda preestriametist ära ajanud.
\par 15 Tema oli ju enesele preestrid seadnud ohvriküngaste, kurjade vaimude ja vasikate jaoks, mis ta oli teinud.
\par 16 Ja nende järel tulid Jeruusalemma kõigist Iisraeli suguharudest need, kes olid südamega andunud otsima Issandat, Iisraeli Jumalat, ohverdama Issandale, oma vanemate Jumalale.
\par 17 Nõnda nad kinnitasid Juuda kuningriiki ja toetasid Rehabeami, Saalomoni poega, kolm aastat; sest nad käisid Taaveti ja Saalomoni teed kolm aastat.
\par 18 Ja Rehabeam võttis enesele naiseks Mahalati, Taaveti poja Jerimoti ja Iisai poja Eliabi tütre Abihaili tütre.
\par 19 Ja see sünnitas temale pojad Jeuse, Semarja ja Saahami.
\par 20 Ja selle järel ta võttis Absalomi tütre Maaka, kes sünnitas temale Abija, Attai, Siisa ja Selomiti.
\par 21 Ja Rehabeam armastas Maakat, Absalomi tütart, rohkem kui kõiki oma naisi ja liignaisi; aga ta oli võtnud kaheksateist naist ja kuuskümmend liignaist; temale sündis kakskümmend kaheksa poega ja kuuskümmend tütart.
\par 22 Ja Rehabeam määras Abija, Maaka poja, peameheks, vürstiks ta vendade hulgas, sest ta tahtis teda kuningaks tõsta.
\par 23 Ta tegi targasti ning jaotas kõik oma pojad kõigisse Juuda ja Benjamini maadesse, kõigisse kindlustatud linnadesse, andis neile rikkalikult toidust ja võttis neile palju naisi.

\chapter{12}

\par 1 Aga kui Rehabeami valitsus oli kindlustatud ja ta ise vägevaks saanud, siis ta hülgas Issanda Seaduse, ja koos temaga kogu Iisrael.
\par 2 Siis sündis, et Rehabeami viiendal aastal tuli Egiptuse kuningas Siisak üles Jeruusalemma kallale - sest need seal olid olnud truuduseta Issanda vastu -
\par 3 koos tuhande kahesaja sõjavankriga ja kuuekümne tuhande ratsanikuga; ja lugematu oli rahvas, kes koos temaga Egiptusest tuli: liibüalased, sukkid ja etiooplased.
\par 4 Ja ta vallutas Juudas olevad kindlustatud linnad ning tuli kuni Jeruusalemmani.
\par 5 Siis tuli prohvet Semaja Rehabeami ja Juuda vürstide juurde, kes olid Siisaki eest kogunenud Jeruusalemma, ja ütles neile: „Nõnda ütleb Issand: Te olete hüljanud minu, seepärast olen minagi teid jätnud Siisaki kätte.”
\par 6 Siis Iisraeli vürstid ja kuningas alandasid endid ning ütlesid: „Issand on õiglane!”
\par 7 Ja kui Issand nägi, et nad endid alandasid, siis tuli Issanda sõna Semajale; ta ütles: „Nad on endid alandanud. Mina ei hävita neid, vaid päästan nad varsti ega vala oma viha Jeruusalemma peale Siisaki käe läbi.
\par 8 Aga nad saavad tema alamaiks, et nad õpiksid tundma, mis tähendab teenida mind ja mis tähendab teenida maiseid kuningriike.”
\par 9 Ja Siisak, Egiptuse kuningas, tuli Jeruusalemma kallale ning võttis ära Issanda koja varandused ja kuningakoja varandused - ta võttis kõik ära, ta võttis ära ka kuldkilbid, mis Saalomon oli teinud.
\par 10 Siis kuningas Rehabeam tegi nende asemele vaskkilbid ja andis need ihukaitsepealikute hooleks, kes valvasid kuningakoja ust.
\par 11 Ja iga kord, kui kuningas läks Issanda kotta, tulid ihukaitsjad ja kandsid neid ja viisid need siis jälle tagasi ihukaitse ruumi.
\par 12 Sellepärast, et ta ennast oli alandanud, pöördus Issanda viha temalt, et teda mitte sootuks hävitada; oli ju veel Juudaski mõndagi head.
\par 13 Ja kuningas Rehabeam kindlustas ennast Jeruusalemmas ning valitses edasi; sest Rehabeam oli kuningaks saades olnud nelikümmend üks aastat vana ja ta valitses seitseteist aastat Jeruusalemmas, linnas, mille Issand kõigist Iisraeli suguharudest oli valinud, et panna sinna oma nimi; ta ema oli ammonlanna Naama.
\par 14 Aga ta tegi kurja, sest ta ei valmistanud oma südant Issandat otsima.
\par 15 Ja Rehabeami lood, varasemad ja hilisemad, eks need ole kirja pandud prohvet Semaja ja nägija Iddo lugudes, nõndasamuti see, mis puutub suguvõsakirjadesse? Rehabeami ja Jerobeami vahel olid kogu aja sõjad.
\par 16 Siis Rehabeam läks magama oma vanemate juurde ja ta maeti Taaveti linna. Ja tema poeg Abija sai tema asemel kuningaks.

\chapter{13}

\par 1 Kuningas Jerobeami kaheksateistkümnendal aastal sai Abija Juuda kuningaks.
\par 2 Ta valitses Jeruusalemmas kolm aastat; ta ema nimi oli Miikaja, Uurieli tütar Gibeast. Aga Abija ja Jerobeami vahel oli sõda.
\par 3 Ja Abija alustas sõda sangarite väega, neljasaja tuhande valitud mehega, aga Jerobeam rivistas tapluseks tema vastu kaheksasada tuhat valitud meest, sõjakangelast.
\par 4 Ja Abija tõusis Semaraimi mäele, mis on Efraimi mäestikus, ja ütles: „Kuulge mind, Jerobeam ja kogu Iisrael!
\par 5 Kas te ei tea, et Issand, Iisraeli Jumal, on andnud Taavetile ja tema poegadele valitsuse Iisraeli üle igaveseks ajaks nagu soolaosaduse?
\par 6 Aga Jerobeam, Nebati poeg, Taaveti poja Saalomoni sulane, on tõusnud ja oma isandale vastu hakanud.
\par 7 Tema juurde kogunes tühiseid, kõlvatuid mehi, ja need olid üle Rehabeamist, Saalomoni pojast, kuna Rehabeam oli noor ja arglik ega suutnud ennast nende vastu maksma panna.
\par 8 Ja nüüd mõtlete teie endid maksma panna Issanda kuningriigi vastu, mis on Taaveti poegade käes, sellepärast et teid on suur hulk ja teie poolt on kuldvasikad, mis Jerobeam on teinud teile jumalaiks!
\par 9 Kas te pole mitte ära ajanud Issanda preestreid, Aaroni poegi, ja leviite, ja endile ise preestreid teinud nagu teiste maade rahvad? Igaüks, kes tuli härjavärsi ja seitsme jääraga oma kätt täitma, sai preestriks neile, kes pole jumalad.
\par 10 Aga meie Jumal on Issand ja meie ei ole teda maha jätnud! Aaroni pojad teenivad preestritena Issandat ja leviidid on ametis.
\par 11 Nad ohverdavad Issandale igal hommikul ja õhtul põletusohvreid ja healõhnalisi suitsutusrohte; nad seavad leivad puhta laua peale ja süütavad igal õhtul kuldlambijala lambid põlema. Sest me täidame Issanda, meie Jumala antud kohustusi, teie olete aga tema maha jätnud!
\par 12 Ja vaata, koos meiega, kõige ees on Jumal ja tema preestrid ja märgupasunad, mis teie vastu hüüavad. Iisraeli lapsed, ärge võidelge Issanda, oma vanemate Jumala vastu, sest see teil ei õnnestu!”
\par 13 Aga Jerobeam oli lasknud varitsejad ringi minna, et need läheksid neile tagant kallale; nad ise olid juudalaste ees, kuna varitsejad olid nende taga.
\par 14 Kui juudalased pöördusid, vaata, siis oli neil võitlus eest ja tagant. Nad kisendasid Issanda poole ja preestrid puhusid pasunaid.
\par 15 Ja Juuda mehed tõstsid sõjakisa. Ja sündis, et kui Juuda mehed sõjakisa tõstsid, siis lõi Jumal Jerobeami ja kogu Iisraeli maha Abija ja Juuda ees.
\par 16 Ja Iisraeli lapsed põgenesid Juuda eest, aga Jumal andis nad tema kätte.
\par 17 Abija ja ta rahvas lõid neid suuresti ja Iisraelist langes mahalööduina viissada tuhat valitud meest.
\par 18 Nõnda alandati seekord Iisraeli lapsed; juudalased aga panid endid maksma, sest nad toetusid Issandale, oma vanemate Jumalale.
\par 19 Ja Abija ajas Jerobeami taga ning vallutas temalt linnad: Peeteli ja selle tütarlinnad, Jesana ja selle tütarlinnad, Efroni ja selle tütarlinnad.
\par 20 Jerobeam aga ei saanud Abija päevil enam jõudu koguda, sest Issand lõi teda, nõnda et ta suri.
\par 21 Aga Abija sai vägevaks. Ta võttis enesele neliteist naist ja temale sündis kakskümmend kaks poega ja kuusteist tütart.
\par 22 Ja muud Abija lood, tema teod ja sõnad, on kirja pandud prohvet Iddo „Seletustes”.

\chapter{14}

\par 1 Aasa tegi, mis hea ja õige oli Issanda, tema Jumala silmis.
\par 2 Tema kõrvaldas võõrad altarid ja ohvrikünkad, murdis sambad katki ja raius viljakustulbad tükkideks.
\par 3 Tema käskis Juudat otsida Issandat, nende vanemate Jumalat, ja täita Seadust ning käske.
\par 4 Tema kõrvaldas kõigist Juuda linnadest ohvrikünkad ja suitsutusaltarid; tema ajal oli kuningriigil rahu.
\par 5 Tema ehitas Juudasse kindlustatud linnu, kuna maal oli rahu ja neil aastail ei olnud tema vastu sõda, sest Issand lubas teda hinge tõmmata.
\par 6 Ta ütles Juudale: „Ehitagem neid linnu ja ümbritsegem need müüriga, tornide, väravate ja riividega, niikaua kui maa veel meie päralt on! Sest me oleme otsinud Issandat, oma Jumalat, oleme otsinud ja tema on meile ümberkaudu rahu andnud.” Siis nad ehitasid ja see läks neil korda.
\par 7 Ja Aasal oli sõjavägi: Juudast kolmsada tuhat meest, kes kandsid kilpi ja piiki, ja Benjaminist kakssada kaheksakümmend tuhat kilbikandjat ja ammukütti; need kõik olid vahvad võitlejad.
\par 8 Aga nende vastu tuli etiooplane Serah väega, mille suurus oli tuhat korda tuhat, ja kolmesaja sõjavankriga; ja ta jõudis kuni Maaresani.
\par 9 Ja Aasa läks temale vastu; siis nad rivistusid tapluseks Sefata orus Maaresas.
\par 10 Ja Aasa hüüdis Issanda, oma Jumala poole ning ütles: „Issand! Ei ole peale sinu ühtegi, kes suudaks jõuetut aidata tugeva vastu! Aita meid, Issand, meie Jumal, sest me toetume sinule ja oleme tulnud sinu nimel selle hulga vastu! Issand! Sina oled meie Jumal, ära luba, et inimene oleks sinust üle!”
\par 11 Siis Issand lõi etiooplasi Aasa ja Juuda ees, ja etiooplased põgenesid.
\par 12 Ja Aasa ning rahvas, kes oli koos temaga, ajasid neid taga kuni Gerarini. Ja etiooplasi langes nõnda palju, et nad enam ei toibunud, sest nad löödi puruks Issanda ja tema leeri ees; ja nad võtsid väga palju saaki.
\par 13 Ja nad lõid kõiki linnu Gerari ümbruses, sest neid oli vallanud hirm Issanda ees. Ja nad riisusid kõiki linnu, sest neis oli palju saaki.
\par 14 Ja nad lõid ka karjatelke, viisid ära palju lambaid, kitsi ja kaameleid ning tulid tagasi Jeruusalemma.

\chapter{15}

\par 1 Ja Asarja, Oodedi poja peale tuli Jumala Vaim.
\par 2 Ta läks välja Aasale vastu ja ütles temale: „Kuulge mind, Aasa ja kogu Juuda ja Benjamin! Issand on teiega, kui teie olete temaga, ja kui te otsite teda, siis te leiate tema; aga kui te tema maha jätate, siis jätab tema teid maha!
\par 3 Kaua aega ei olnud Iisraelil tõsist, Jumalat õpetavat preestrit ega Seadust.
\par 4 Aga kui temal kitsas käes oli, siis ta pöördus Issanda, Iisraeli Jumala poole. Ja nad otsisid teda ning leidsid ta.
\par 5 Neil aegadel ei olnud rahu minejal ega tulijal, sest suur segadus valitses kõigi maade elanike hulgas.
\par 6 Rahvas purustas rahvast ja linn linna, sest Jumal tegi nad rahutuks kõiksugu hädade läbi.
\par 7 Aga teie olge vahvad ja ärge laske oma käsi lõdvaks, sest teie tööl on tasu!”
\par 8 Kui Aasa kuulis neid sõnu ja prohvet Oodedi kuulutust, siis ta võttis julguse ja kõrvaldas jäledused kogu Juuda ja Benjamini maalt, samuti linnadest, mis ta Efraimi mäestikus oli vallutanud, ja uuendas Issanda altari, mis oli Issanda eeskoja ees.
\par 9 Ja ta kogus kokku Juuda ja Benjamini, ja need, kes olid võõrastena nende juures Efraimist, Manassest ja Siimeonist, sest Iisraelist olid paljud tulnud üle tema poole, kui nad nägid, et Issand, tema Jumal, oli temaga.
\par 10 Nad kõik kogunesid Jeruusalemma kolmandas kuus, Aasa valitsemise viieteistkümnendal aastal,
\par 11 ja nad ohverdasid sel päeval Issandale saagist, mis nad olid toonud, seitsesada veist ning seitse tuhat lammast ja kitse.
\par 12 Ja nad võtsid endile kohustuse, et nad otsivad Issandat, oma vanemate Jumalat, kõigest oma südamest ja kõigest oma hingest,
\par 13 ja et igaüks, kes ei otsi Issandat, Iisraeli Jumalat, tuleb surmata, olgu väike või suur, mees või naine.
\par 14 Ja nad andsid Issandale vande valju hääle ja hõiskamisega, pasunate ja sarvedega.
\par 15 Ja kogu Juuda rõõmustas vande pärast, sest nad olid vandunud kõigest südamest; ja nad otsisid teda tõesti hea meelega ning leidsid ta. Ja Issand andis neile ümberkaudu rahu.
\par 16 Ja kuningas Aasa kõrvaldas isegi oma vanaema Maaka kui valitsejanna, sellepärast et too oli teinud Aðera häbikuju; Aasa hävitas tema häbikuju, purustas selle ja põletas Kidroni jõe ääres.
\par 17 Aga ohvrikünkad ei kadunud Iisraelist; ometi oli Aasa süda siiras kogu ta eluaja.
\par 18 Ta viis Jumala kotta oma isa pühitsetud annid ja mis ta ise pühitses: hõbeda, kulla ja riistad.
\par 19 Sõda ei olnud kuni Aasa kolmekümne viienda valitsemisaastani.

\chapter{16}

\par 1 Aasa kolmekümne kuuendal valitsemisaastal tuli Iisraeli kuningas Baesa Juuda vastu ja kindlustas Raama, et Juuda kuningal Aasal ei oleks välja- ega sissepääsu.
\par 2 Siis Aasa tõi välja hõbeda ja kulla Issanda koja ja kuningakoja varanduste hulgast ja läkitas Ben-Hadadile, Süüria kuningale, kes elas Damaskuses, öeldes:
\par 3 „Minu ja sinu vahel olgu leping, nagu oli minu isa ja sinu isa vahel! Vaata, ma läkitan sulle hõbedat ja kulda. Mine, tühista oma leping Iisraeli kuninga Baesaga, et ta läheks ära minu kallalt!”
\par 4 Ja Ben-Hadad kuulas Aasat ning läkitas oma sõjaväepealikud Iisraeli linnade vastu ja need vallutasid Ijjoni, Daani ja Aabel-Maimi ja kõik Naftali linnade tagavaralaod.
\par 5 Kui Baesa seda kuulis, siis ta loobus Raamat kindlustamast ja katkestas oma töö.
\par 6 Aga kuningas Aasa tõi kaasa kogu Juuda, ja nad viisid Raamast ära kivid ja palgid, millega Baesa oli ehitanud; ta kindlustas nendega Geba ja Mispa.
\par 7 Sel ajal tuli nägija Hanani Juuda kuninga Aasa juurde ja ütles temale: „Et sa toetusid Süüria kuningale ega toetunud Issandale, oma Jumalale, seepärast on Süüria kuninga sõjavägi sinu käest pääsenud.
\par 8 Eks olnud etiooplased ja liibüalased suur sõjavägi väga paljude vankrite ja ratsanikega? Aga kui sa toetusid Issandale, andis ta need su kätte.
\par 9 Sest Issanda silmad jälgivad kogu maad, et võimsasti aidata neid, kes siira südamega hoiavad tema poole. Seda sa tegid rumalasti, sest nüüdsest peale on sul sõjad.”
\par 10 Aga Aasa sai nägijale pahaseks ja pani ta vangihoonesse, sest selle kõne pärast oli tal viha tema vastu. Aasa kohtles sel ajal mõningaid rahva hulgast julmalt.
\par 11 Ja vaata, Aasa varasemad ja hilisemad lood, vaata, need on kirja pandud Juuda ja Iisraeli Kuningate raamatus.
\par 12 Oma kolmekümne üheksandal valitsemisaastal hakkas Aasa jalgu põdema ja tema haigus oli väga raske. Aga oma haiguseski ei otsinud ta Issandat, vaid arste.
\par 13 Siis Aasa läks magama oma vanemate juurde ja suri oma neljakümne esimesel valitsemisaastal.
\par 14 Ta maeti oma hauda, mille ta enesele oli lasknud raiuda Taaveti linnas; ta sängitati asemele, mis oli täidetud palsamitega ja mitmesuguste segatud salvidega, ja tema auks põletati väga suur koolnusuitsutus.

\chapter{17}

\par 1 Ja tema poeg Joosafat sai tema asemel kuningaks; tema kindlustas ennast Iisraeli vastu.
\par 2 Ta pani sõjaväe kõigisse Juuda kindlustatud linnadesse ja paigutas valveüksused Juuda maale ja Efraimi linnadesse, mis tema isa Aasa oli vallutanud.
\par 3 Ja Issand oli Joosafatiga, sest ta käis oma isa Taaveti varasematel teedel ega otsinud baale,
\par 4 vaid ta otsis oma isa Jumalat ja käis tema käskude järgi ega teinud nagu Iisrael.
\par 5 Issand kinnitas kuningriigi tema kätte ja kogu Juuda tõi Joosafatile ande; ja tal oli palju rikkust ning au.
\par 6 Issanda teedel olles oli ta süda julge ja peale selle kõrvaldas ta Juudast veel ohvrikünkad ja viljakustulbad.
\par 7 Oma valitsemise kolmandal aastal läkitas ta oma vürstid Ben-Haili, Obadja, Sakarja, Netaneeli ja Miika Juuda linnadesse õpetama,
\par 8 ja koos nendega leviidid Semaja, Netanja, Sebadja, Asaeli, Semiramoti, Joonatani, Adonija, Toobija ja Toob-Adonija - leviidid; ja koos nendega preestrid Elisama ja Joorami.
\par 9 Need õpetasid Juudas ja neil oli kaasas Issanda Seaduse raamat; nad käisid läbi kõik Juuda linnad ja õpetasid rahvast.
\par 10 Hirm Issanda ees valdas kõiki Juuda ümberkaudsete maade kuningriike ja need ei sõdinud Joosafati vastu.
\par 11 Ja vilistite poolt toodi Joosafatile ande ja rahamaksu; isegi araablased tõid temale lambaid ja kitsi: seitse tuhat seitsesada jäära ja seitse tuhat seitsesada sikku.
\par 12 Ja Joosafat sai üha vägevamaks ning ta ehitas Juudasse kindlustatud paiku ja tagavaraladude linnu.
\par 13 Temal olid suured tagavarad Juuda linnades ja sõjamehed, vahvad võitlejad, Jeruusalemmas.
\par 14 Ja niisugune oli nende teenistuslik jaotus vastavalt nende perekondadele: Juuda tuhandepealikud: pealik Adna ja koos temaga kolmsada tuhat vahvat võitlejat;
\par 15 tema kõrval pealik Joohanan ja koos temaga kakssada kaheksakümmend tuhat;
\par 16 tema kõrval Amasja, Sikri poeg, kes oli enese vabatahtlikult Issandale pühendanud, ja koos temaga kakssada tuhat vahvat võitlejat.
\par 17 Benjaminist: Eljada, vahva võitleja, ja koos temaga kakssada tuhat, kes olid relvastatud ammu ja kilbiga;
\par 18 tema kõrval Joosabad ja koos temaga sada kaheksakümmend tuhat, kes olid sõjaks varustatud.
\par 19 Need teenisid kuningat; ja peale nende olid need, keda kuningas oli pannud kindlustatud linnadesse kogu Juudas.

\chapter{18}

\par 1 Joosafatil oli palju rikkust ning au ja ta sai Ahabiga languks.
\par 2 Mõne aasta pärast läks ta Ahabi juurde alla Samaariasse; Ahab tappis palju lambaid, kitsi ja veiseid temale ja rahvale, kes oli koos temaga, ja õhutas teda minema üles Gileadi Raamotisse.
\par 3 Ja Iisraeli kuningas Ahab küsis Juuda kuningalt Joosafatilt: „Kas tuled koos minuga Gileadi Raamotisse?„ Ja Joosafat vastas temale: ”Nagu sina, nõnda mina, nagu sinu rahvas, nõnda minu rahvas! Ma tulen koos sinuga sõtta.”
\par 4 Ja Joosafat ütles Iisraeli kuningale: „Küsi ometi enne Issanda sõna!”
\par 5 Siis Iisraeli kuningas kogus kokku prohvetid, nelisada meest, ja küsis neilt: „Kas me võime minna sõdima Gileadi Raamoti vastu või ma pean loobuma?„ Ja nad vastasid: ”Mine, ja Jumal annab selle kuninga kätte!”
\par 6 Aga Joosafat küsis: „Kas ei ole siin veel mõnda Issanda prohvetit, et saaksime temalt küsida?”
\par 7 Ja Iisraeli kuningas vastas Joosafatile: „On küll veel üks mees, kellelt saaks küsida Issanda nõu. Aga mina vihkan teda, sest ta ei kuuluta mulle head, vaid alati halba. See on Miika, Jimla poeg.„ Aga Joosafat ütles: ”kuningas ärgu rääkigu nõnda!”
\par 8 Siis Iisraeli kuningas kutsus ühe hoovkondlase ja ütles: „Kähku siia Miika, Jimla poeg!”
\par 9 Iisraeli kuningas ja Juuda kuningas Joosafat istusid kumbki oma aujärjel, mantlid seljas; nad istusid rehepeksu väljakul Samaaria värava suus ja kõik prohvetid kuulutasid nende ees.
\par 10 Sidkija, Kenaana poeg, oli teinud enesele raudsarved ja ütles: „Nõnda ütleb Issand: Nendega pead sa puskima süürlasi, kuni nad on hävitatud!”
\par 11 Ja kõik prohvetid kuulutasid sedasama ning ütlesid: „Mine Gileadi Raamotisse ja see õnnestub sul, sest Issand annab selle kuninga kätte!”
\par 12 Ja käskjalg, kes oli läinud Miikat kutsuma, rääkis temaga ning ütles: „Vaata, prohvetite sõnad on nagu ühest suust kuningale head; olgu siis sinugi sõna nagu ühel nende hulgast, ja räägi head!”
\par 13 Aga Miika vastas: „Nii tõesti kui Issand elab, mina räägin, mis mu Jumal ütleb!”
\par 14 Kui ta tuli kuninga juurde, siis kuningas ütles temale: „Miika! Kas me võime minna sõdima Gileadi Raamoti vastu või ma pean loobuma?„ Ja ta vastas: ”Minge, ja teil õnnestub see, sest nad antakse teie kätte!”
\par 15 Aga kuningas ütles temale: „Mitu korda ma pean sind vannutama, et sa ei räägiks mulle muud kui ainult tõtt Issanda nimel?”
\par 16 Siis ta ütles: „Ma nägin kogu Iisraeli hajali olevat mägedel nagu lambad, kellel ei ole karjast. Ja Issand ütles: Neil pole isandaid. Igaüks pöördugu rahuga koju!”
\par 17 Siis ütles Iisraeli kuningas Joosafatile: „Eks ma öelnud sulle, et ta ei kuuluta mulle head, küll aga halba!”
\par 18 Aga Miika ütles: „Seepärast kuulge Issanda sõna! Ma nägin Issandat istuvat oma aujärjel ja kogu taeva sõjaväe seisvat temast paremal ja vasakul pool.
\par 19 Ja Issand ütles: Kes tahaks ahvatleda Ahabit, Iisraeli kuningat, et ta läheks ja langeks Gileadi Raamotis? Siis rääkis üks nii ja teine rääkis naa.
\par 20 Aga üks vaim tuli ja seisis Issanda ees ning ütles: Mina ahvatlen teda! Ja Issand küsis temalt: Kuidas?
\par 21 Tema vastas: Ma lähen ja olen vale vaim kõigi ta prohvetite suus. Siis ütles Issand: Sina oled ahvatleja ja küllap sa suudad. Mine ja tee nõnda!
\par 22 Ja nüüd, vaata, Issand on pannud vale vaimu nende sinu prohvetite suhu ja Issand on kuulutanud sulle halba.”
\par 23 Siis astus ligi Sidkija, Kenaana poeg, ja lõi Miikat põse pihta ning küsis: „Missugust teed on Issanda Vaim minust läinud sinusse rääkima?”
\par 24 Ja Miika vastas: „Vaata, sa näed seda päeval, kui sa pead minema kambrist kambrisse, et ennast peita!”
\par 25 Siis ütles Iisraeli kuningas: „Võtke Miika ja viige ta tagasi linnapealik Aamoni ja kuningapoeg Joase juurde
\par 26 ning öelge: Nõnda ütleb kuningas: Pange ta vangikotta ja toitke teda hädapäraselt leiva ja veega, kuni ma rahuga tagasi tulen!”
\par 27 Aga Miika ütles: „Kui sa tõesti rahuga tagasi tuled, siis ei ole Issand minu läbi rääkinud!„ Ja ta ütles veel: ”Kuulge, kõik rahvad!”
\par 28 Ja Iisraeli kuningas ja Joosafat, Juuda kuningas, läksid üles Gileadi Raamotisse.
\par 29 Ja Iisraeli kuningas ütles Joosafatile: „Mina panen sõtta minnes teised riided selga, aga sina kanna oma riideid!” Ja Iisraeli kuningas pani teised riided selga ja nad läksid sõtta.
\par 30 Aga Süüria kuningas oli andnud käsu oma sõjavankrite pealikuile ja oli öelnud: „Ärge tapelge mitte ühegi muu, vähema või suurema vastu kui üksnes Iisraeli kuninga vastu!”
\par 31 Kui siis sõjavankrite pealikud nägid Joosafatti, ütlesid nad: „See on Iisraeli kuningas!” Ja nad pöördusid tema vastu sõdima. Siis Joosafat hakkas kisendama ja Issand aitas teda: Jumal meelitas nad tema kallalt ära.
\par 32 Kui sõjavankrite pealikud nägid, et see ei olnudki Iisraeli kuningas, siis nad pöördusid tagasi tema kannult.
\par 33 Aga keegi mees, kes oli huupi oma ammu vinna tõmmanud, tabas Iisraeli kuningat rihmade ja soomustuse vahele. Ja kuningas ütles sõjavankri juhile: „Pööra ümber ja vii mind võitlusest välja, sest ma olen haavatud!”
\par 34 Kuid taplus üha ägenes sel päeval ja Iisraeli kuningas pidi jääma vankrisse süürlaste vastu kuni õhtuni; aga päikeseloojaku ajal ta suri.

\chapter{19}

\par 1 Aga Juuda kuningas Joosafat tuli rahuga tagasi koju Jeruusalemma.
\par 2 Siis läks nägija Jehu, Hanani poeg, temale vastu ja ütles kuningas Joosafatile: „Kas sa tohid aidata õelat ja armastada neid, kes Issandat vihkavad? Seepärast on su peal Issanda viha!
\par 3 Ometi leidub su juures ka head, sest sa oled maalt kõrvaldanud Aðera kujud ja oled oma südant valmistanud Jumalat otsima.”
\par 4 Ja Joosafat elas jälle Jeruusalemmas. Ta läks jälle rahva sekka Beer-Sebast kuni Efraimi mäestikuni ja tõi nad tagasi Issanda, nende vanemate Jumala juurde.
\par 5 Ta seadis maale kohtumõistjaid kõigisse Juuda kindlustatud linnadesse, linn-linnalt,
\par 6 ja ütles kohtumõistjaile: „Vaadake, mida te teete, sest teie ei mõista kohut inimese nimel, vaid Issanda nimel, ja tema on teiega, kui te õigust mõistate!
\par 7 Nüüd siis valitsegu teid Issanda kartus! Pange tähele, mida te teete, sest Issanda, meie Jumala juures ei ole ülekohut, erapoolikust ega meelehea võtmist!”
\par 8 Ka Jeruusalemmas seadis Joosafat leviite, preestreid ja Iisraeli perekondade peamehi Issanda kohtu ja riiuasjade tarvis, kui nad Jeruusalemma tagasi tulid.
\par 9 Ja ta käskis neid, öeldes: „Te peate Issanda kartuses, ustavuses ja südame siiruses tegema nõnda:
\par 10 igas riiuasjas, mis teie ette tuuakse teie vendade poolt, kes elavad oma linnades, olgu see vere ja vere vahel, või mis puutub Seadusesse, käsusse, määrustesse ja seadlustesse, siis peate teie neid manitsema, et nad ei saaks Issanda ees süüdlasteks ja et viha ei tabaks teid ja teie vendi. Tehke nõnda, et te ei jääks süüdlasteks!
\par 11 Ja vaata, ülempreester Amarja on teie ülemus kõigis Issanda asjus, ja Sebadja, Ismaeli poeg, Juuda soo vürst, kõigis kuninga asjus; ja ametnikena olgu leviidid teie teenistuses! Olge vahvad ja tegutsege, ja Issand olgu sellega, kes on tubli!”

\chapter{20}

\par 1 Aga pärast seda tulid moabid ja ammonlased, ja nendega koos oli meunlasi, Joosafati vastu sõdima.
\par 2 Siis tuldi ja teatati Joosafatile: „Sinu vastu tuleb suur jõuk teiselt poolt merd Aramist. Vaata, nad on Haseson-Taamaris, see on Een-Gedis!”
\par 3 Siis Joosafat hakkas kartma ja pööras oma palge Issandat otsima; ta kuulutas välja paastu kogu Juudas.
\par 4 Ja Juuda kogunes Issandalt abi otsima, ka kõigist Juuda linnadest tuldi Issandat otsima.
\par 5 Ja Joosafat seisis Juuda ja Jeruusalemma koguduses Issanda kojas uue õue ees
\par 6 ning ütles: „Issand, meie vanemate Jumal, kas pole mitte sina Jumal taevas ja kõigi paganate kuningriikide valitseja? Sinu käes on ju jõud ja vägi, ja ükski ei suuda sulle vastu panna.
\par 7 Eks ole sina, meie Jumal, ära ajanud selle maa elanikud oma Iisraeli rahva eest ja andnud maa igaveseks ajaks oma sõbra Aabrahami soole?
\par 8 Nemad on seal elanud ja on seal ehitanud sinule, su nimele pühamu, öeldes:
\par 9 Kui meid tabab õnnetus, mõõk, kohus, katk või nälg, siis me astume selle koja ette ja sinu ette, sest selles kojas on sinu nimi. Ja me hüüame sinu poole oma kitsikuses, et sa meid kuuleksid ja aitaksid.
\par 10 Ja nüüd, vaata, Ammoni, Moabi ja Seiri mäestiku pojad, kelle maast sa ei lubanud Iisraeli läbi minna, kui nad Egiptusemaalt tulid, vaid pidid neist mööda minema ega saanud neid hävitada,
\par 11 vaata, need maksavad nüüd meile kätte, tulles meid välja ajama sinu omandist, mille sa meile oled andnud pärida.
\par 12 Meie Jumal, kas sa ei tahaks nende üle kohut mõista? Sest meil pole jõudu selle suure jõugu ees, kes tuleb meile kallale, ja me ise ei tea, mida peaksime tegema, vaid meie silmad vaatavad sinu poole!”
\par 13 Ja kogu Juuda seisis Issanda ees, isegi nende väetid, nende naised ja lapsed.
\par 14 Ja Issanda Vaim tuli Jahasieli, Sakarja poja peale, kes oli Aasafi järeltulijaist pärit oleva leviit Mattanja poja Jeieli poja Benaja poja Sakarja poeg,
\par 15 ja ta ütles: „Pange tähele, kogu Juuda, Jeruusalemma elanikud ja kuningas Joosafat: Nõnda ütleb teile Issand: Ärge kartke ja ärge tundke hirmu selle suure jõugu ees, sest see pole teie, vaid Jumala sõda!
\par 16 Minge homme alla nende vastu! Vaata, nad tulevad üles mööda Siisi tõusuteed ja te kohtate neid oru lõpus ida pool Jerueli kõrbe.
\par 17 Seejuures ei ole teil vaja sõdida. Astuge ette, seiske ja vaadake, kuidas Issand teid päästab, Juuda ja Jeruusalemm! Ärge kartke ja ärge tundke hirmu! Minge homme nende vastu ja Issand on teiega!”
\par 18 Siis Joosafat kummardas silmili maha ja kõik Juuda ja Jeruusalemma elanikud heitsid Issanda ette ning kummardasid Issandat.
\par 19 Ja need leviidid, kes olid kehatlastest ja korahlastest, tõusid üles, et Issandat, Iisraeli Jumalat, ülivalju häälega kiita.
\par 20 Ja järgmisel hommikul tõusid nad vara ning läksid Tekoa kõrbe. Nende välja minnes astus Joosafat ette ja ütles: „Kuulge mind, Juuda ja Jeruusalemma elanikud! Uskuge Issandasse, oma Jumalasse, siis te jääte püsima! Uskuge tema prohveteid, siis õnnestub teil kõik!”
\par 21 Ja ta pidas rahvaga nõu, seadis Issandale lauljad ja kiitjad, kes pühas ehtes pidid minema relvastatute ees ja ütlema: „Tänage Issandat, sest tema heldus kestab igavesti!”
\par 22 Sel ajal, kui nad hakkasid hõiskama ja kiitma, pani Issand varitsejad Ammoni, Moabi ja Seiri mäestiku poegadele, kes olid tulnud Juuda vastu; ja nad löödi maha.
\par 23 Sest ammonlased ja moabid tõusid Seiri mäestiku elanike vastu, et need hävitada ja kaotada; ja kui nad Seiri elanikele olid lõpu teinud, siis aitasid nad üksteist hävitada.
\par 24 Kui siis Juuda jõudis vaatluspaika kõrbes ja nad vaatasid jõugu poole, ennäe, siis lamasid laibad maas ja ükski polnud pääsenud.
\par 25 Ja Joosafat ja tema rahvas tulid neilt saaki riisuma ja leidsid nende juurest palju varandust, riideid ja kalleid riistu, ja nad võtsid endile rohkem, kui nad suutsid kanda; nad olid kolm päeva saaki riisumas, sest seda oli palju.
\par 26 Ja neljandal päeval kogunesid nad Kiituseorgu, sest seal kiitsid nad Issandat; seepärast nimetatakse seda paika Kiituseoruks tänapäevani.
\par 27 Siis pöördusid kõik Juuda ja Jeruusalemma mehed, ja Joosafat nende eesotsas, rõõmuga tagasi Jeruusalemma, sest Issand oli neid lasknud nende vaenlastest rõõmu tunda.
\par 28 Ja nad tulid naablite, kannelde ja pasunatega Jeruusalemma Issanda kotta.
\par 29 Aga hirm Jumala ees valdas kõigi maade kuningriike, kui nad kuulsid, et Issand oli sõdinud Iisraeli vaenlaste vastu.
\par 30 Siis oli Joosafati kuningriigil vaikne aeg; tema Jumal andis talle ümberkaudu rahu.
\par 31 Ja Joosafat valitses Juuda üle; ta oli kuningaks saades kolmkümmend viis aastat vana ja ta valitses Jeruusalemmas kakskümmend viis aastat; ta ema nimi oli Asuuba, Silhi tütar.
\par 32 Ta käis oma isa Aasa teed ega lahkunud sellelt, tehes, mis õige oli Issanda silmis.
\par 33 Aga ohvrikünkad ei kadunud ja rahvas ei olnud veel valmistanud oma südant oma vanemate Jumalale.
\par 34 Ja muud Joosafati lood, varasemad ja hilisemad, vaata, need on kirja pandud Jehu, Hanani poja lugudes, mis on võetud Iisraeli Kuningate raamatusse.
\par 35 Ja pärast seda astus Juuda kuningas Joosafat ühendusse Iisraeli kuninga Ahasjaga, kes tegi õelat tööd.
\par 36 Ta ühines temaga laevu ehitama, mis pidid minema Tarsisesse; nad ehitasid laevu Esjon-Geberis.
\par 37 Aga Elieser, Dodavahu poeg Maaresast, kuulutas prohvetlikult Joosafati vastu, öeldes: „Sellepärast, et sa oled liitunud Ahasjaga, purustab Issand su töö!” Ja laevad hukkusid ega saanudki minna Tarsisesse.

\chapter{21}

\par 1 Ja Joosafat läks magama oma vanemate juurde ning ta maeti oma vanemate juurde Taaveti linna. Ja tema poeg Jooram sai tema asemel kuningaks.
\par 2 Temal olid vennad, Joosafati pojad: Asarja, Jehiel, Sakarja, Asarjahu, Miikael ja Sefatja; need kõik olid Iisraeli kuninga Joosafati pojad.
\par 3 Nende isa oli neile andnud suuri kingitusi, hõbedat, kulda ja kalleid asju ühes kindlustatud linnadega Juudas; aga kuningriigi andis ta Jooramile, sest tema oli esmasündinu.
\par 4 Kui Jooram oli tõusnud oma isa aujärjele ja oli ennast kindlustanud, siis ta tappis mõõgaga kõik oma vennad, samuti ka mõned Jeruusalemma vürstidest.
\par 5 Jooram oli kuningaks saades kolmkümmend kaks aastat vana ja ta valitses Jeruusalemmas kaheksa aastat.
\par 6 Aga ta käis Iisraeli kuningate teed, nõnda nagu Ahabi sugu oli teinud, sest tal oli Ahabi tütar naiseks; ja ta tegi kurja Issanda silmis.
\par 7 Kuid Issand ei tahtnud Taaveti sugu hävitada lepingu pärast, mille ta Taavetiga oli teinud, ja sellepärast et ta temale ja ta poegadele oli tõotanud anda lambi kõigiks aegadeks.
\par 8 Tema ajal taganesid edomlased Juuda käe alt ja tõstsid endile kuninga.
\par 9 Siis läks Jooram oma pealikutega sinna ja tal olid kaasas kõik sõjavankrid; ta tõusis öösel ja lõi edomlasi, kes olid piiranud teda ja sõjavankrite pealikuid.
\par 10 Siiski on Edom tänapäevani taganenud Juuda käe alt; sel ajal taganes ka Libna tema käe alt, kuna ta oli hüljanud Issanda, oma vanemate Jumala.
\par 11 Temagi tegi ohvrikünkaid Juuda mäestikku, saatis Jeruusalemma elanikud kõlvatust tegema ja eksitas Juudat.
\par 12 Aga temale tuli prohvet Eelijalt kiri, milles öeldi: „Nõnda ütleb Issand, su isa Taaveti Jumal: Sellepärast et sa ei ole käinud oma isa Joosafati teedel ega Juuda kuninga Aasa teedel,
\par 13 vaid oled käinud Iisraeli kuningate teed ja oled saatnud Juuda ja Jeruusalemma elanikud kõlvatust tegema, nõnda nagu Ahabi sugu saatis kõlvatust tegema, ja ka sellepärast, et sa oled tapnud oma isa perest oma vennad, kes olid paremad kui sina,
\par 14 vaata, sellepärast nuhtleb Issand suure nuhtlusega su rahvast, su poegi, su naisi ja kogu su omandit,
\par 15 ja sind ennast raskete haigustega, su sisikonna haigusega, kuni su sisikond on haiguse tõttu aja jooksul välja tulnud!”
\par 16 Ja Issand äratas Joorami vastu vilistite ja araablaste vaimu, kes asusid etiooplaste naabruses,
\par 17 ja need tulid üles Juudasse, tungisid sisse ja viisid ära kogu varanduse, mis leidus kuningakojas, ka tema pojad ja naised, nõnda et temale ei jäänud muud poega kui Jooahas, tema kõige noorem poeg.
\par 18 Ja pärast kõike seda nuhtles Issand teda ravimatu sisikonnahaigusega:
\par 19 aja jooksul, kahe aasta möödudes, tuli ta sisikond haiguse tõttu välja ja ta suri pahaloomulisse haigusesse. Aga tema rahvas ei põletanud tema auks koolnusuitsutust, nagu oli põletatud tema vanemate auks.
\par 20 Ta oli kuningaks saades kolmkümmend kaks aastat vana ja ta valitses Jeruusalemmas kaheksa aastat; ta läks ära, ilma et temast oleks hoolitud, ta maeti Taaveti linna, aga mitte kuningate hauda.

\chapter{22}

\par 1 Siis Jeruusalemma elanikud tõstsid tema noorima poja Ahasja tema asemele kuningaks, sest kõik vanemad pojad oli tapnud see röövjõuk, kes koos araablastega oli tunginud leeri. Ja nõnda sai Ahasja, Juuda kuninga Joorami poeg, kuningaks.
\par 2 Ahasja oli kuningaks saades kakskümmend kaks aastat vana ja ta valitses Jeruusalemmas ühe aasta; ta ema nimi oli Atalja, Omri tütar.
\par 3 Temagi käis Ahabi soo teedel, sest ta ema oli temale nõuandjaks süütegude kordasaatmisel.
\par 4 Ta tegi kurja Issanda silmis, nõnda nagu Ahabi sugu, sest nemad olid tema nõuandjad pärast ta isa surma, hukatuseks temale.
\par 5 Tema käis ka nende nõuande järgi ja läks koos Jooramiga, Iisraeli kuninga Ahabi pojaga, sõtta Süüria kuninga Hasaeli vastu Gileadi Raamotisse; aga süürlased haavasid Jooramit.
\par 6 Siis ta tuli tagasi, et Jisreelis ennast ravida, sest tal olid haavad, mis temale Raamas olid löödud, kui ta sõdis Süüria kuninga Hasaeli vastu. Ja Juuda kuningas Asarja, Joorami poeg, tuli alla Jisreeli vaatama Jooramit, Ahabi poega, kui see oli haige.
\par 7 Aga Jumalalt oli see Ahasjale hukatuseks, et ta tuli Joorami juurde; sest sinna jõudnult läks ta koos Jooramiga Nimsi poja Jehu juurde, keda Issand oli võidnud Ahabi sugu hävitama.
\par 8 Ja kui Jehu mõistis kohut Ahabi soo üle, siis ta kohtas Juuda vürste ja Ahasja vendade poegi, kes Ahasjat teenisid, ja tappis need.
\par 9 Siis ta otsis Ahasjat; too võeti kinni Samaarias, kus ta ennast varjas. Ta viidi Jehu juurde ja surmati. Ja ta maeti maha, sest nad ütlesid: „Tema on selle Joosafati pojapoeg, kes otsis Issandat kõigest oma südamest.” Ja Ahasja sool ei olnud kedagi, kes oleks suutnud valitseda.
\par 10 Kui Ahasja ema Atalja nägi, et ta poeg oli surnud, siis ta võttis kätte ja hävitas kogu Juuda koja kuningliku soo.
\par 11 Aga Joosabat, kuninga tütar, võttis surmale määratud kuningapoegade hulgast salaja Ahasja poja Joase ja pani tema ja ta imetaja ühte magamiskambrisse; nõnda peitis teda Joosabat, kuningas Joorami tütar, preester Joojada naine - sest ta oli Ahasja õde -, Atalja eest, ja nõnda ei saanud see teda surmata.
\par 12 Ja Joas oli nende juures Jumala kojas peidus kuus aastat, kuna Atalja valitses maad.

\chapter{23}

\par 1 Aga seitsmendal aastal võttis Joojada julguse ja tegi lepingu sajapealikutega: Jerohami poja Asarjaga, Joohanani poja Ismaeliga, Oobedi poja Asarjaga, Adaja poja Maasejaga ja Sikri poja Elisafatiga.
\par 2 Nemad käisid Juudas ringi ja kogusid kokku kõigist Juuda linnadest leviidid ja Iisraeli perekondade peamehed; ja nad tulid Jeruusalemma.
\par 3 Ja terve kogudus tegi Jumala kojas lepingu kuninga küsimuses. Ja Joojada ütles neile: „Vaata, kuninga poeg peab saama kuningaks, nõnda nagu Issand on öelnud Taaveti poegade kohta!
\par 4 Teil tuleb teha nõnda: kolmandik teist, preestritest ja leviitidest, kes tuleb hingamispäeval, olgu värava lävehoidjaks;
\par 5 kolmandik olgu kuningakojas ja kolmandik olgu Jesoodi väravas; ja kõik muu rahvas olgu Issanda koja õuedes.
\par 6 Aga Issanda kotta ei tohi minna ükski peale preestrite ja teenistuses olevate leviitide; nemad võivad sisse minna, sest nad on pühitsetud, aga kõik muu rahvas täitku Issanda korraldust.
\par 7 Leviidid asugu ringi ümber kuninga, igaühel oma sõjariistad käes, ja kes kotta tuleb, see surmatagu! Nõnda olge kuninga juures, kui ta sisse tuleb või välja läheb!”
\par 8 Ja leviidid ning kogu Juuda tegid kõik nõnda, nagu preester Joojada käskis; ja igaüks võttis oma mehed, niihästi need, kes hingamispäeval tulid, kui ka need, kes hingamispäeval pidid ära minema, sest preester Joojada ei vabastanud neid rühmi.
\par 9 Ja preester Joojada andis sajapealikuile piigid ning suured ja väikesed kilbid, mis olid kuulunud kuningas Taavetile ja mis olid Jumala kojas.
\par 10 Ja ta paigutas kogu rahva, igaühel oma viskoda käes, koja paremast tiivast koja vasaku tiivani, altari ja koja suunas ümber kuninga.
\par 11 Siis toodi kuningapoeg välja, pandi kroon pähe, anti kinnituskiri kätte ja tõsteti kuningaks, Joojada ja tema pojad võidsid teda ja hüüdsid: „Elagu kuningas!”
\par 12 Kui Atalja kuulis jooksva ja kuningat ülistava rahva häält, siis tuli ta rahva juurde Issanda kotta.
\par 13 Ta vaatas, ja ennäe, kuningas seisis samba juures sissekäigu ligidal. Pealikud ja pasunapuhujad olid kuninga juures, kogu maa rahvas oli rõõmus ja puhus pasunaid, ja lauljad mänguriistadega juhatasid kiituslaulu. Siis käristas Atalja oma riided lõhki ja hüüdis: „Vandenõu! Vandenõu!”
\par 14 Aga preester Joojada saatis välja sajapealikud, sõjaväe ülemad, ja ütles neile: „Viige ta ridade vahelt välja, ja kes läheb temale järele, see surmatagu mõõgaga!„ Sest preester oli öelnud: ”Ärge surmake teda Issanda kojas!”
\par 15 Siis nad panid käed ta külge; ja kui Atalja oli jõudnud kuningakoja Hobuvärava sissekäiguni, siis nad surmasid ta seal.
\par 16 Ja Joojada tegi lepingu iseenese ja kogu rahva ning kuninga vahel, et nad tahavad olla Issanda rahvas.
\par 17 Siis läks kogu rahvas Baali kotta ja nad kiskusid selle maha; nad purustasid selle altarid ja kujud ning tapsid Mattani, Baali preestri, altarite ees.
\par 18 Siis Joojada seadis valvurid Issanda kojale, leviitpreestrite juhatuse alla, keda Taavet Issanda kojale oli jaotanud, et ohverdataks Issandale põletusohvreid, nõnda nagu Moosese Seaduses on kirjutatud, Taaveti korralduse kohaselt rõõmu ja lauluga.
\par 19 Ja ta seadis väravahoidjad Issanda koja väravaile, et ei pääseks sisse see, kes mõnel põhjusel on roojane.
\par 20 Ja ta võttis sajapealikud, ülikud ja rahva valitsejad ja kogu maa rahva ja viis kuninga Issanda kojast alla; ja nad tulid läbi Ülavärava kuningakotta ning panid kuninga istuma kuninglikule aujärjele.
\par 21 Ja kogu maa rahvas oli rõõmus, ja linnas oli rahu, kui Atalja oli mõõgaga surmatud.

\chapter{24}

\par 1 Joas oli kuningaks saades seitse aastat vana ja ta valitses Jeruusalemmas nelikümmend aastat; ta ema nimi oli Sibja, Beer-Sebast.
\par 2 Joas tegi, mis õige oli Issanda silmis preester Joojada kõigil elupäevil.
\par 3 Joojada võttis temale kaks naist, ja temale sündis poegi ja tütreid.
\par 4 Ja pärast seda oli Joasel südame peal Issanda koja parandamine.
\par 5 Ta kutsus kokku preestrid ja leviidid ning ütles neile: „Minge Juuda linnadesse ja koguge kogu Iisraelist raha oma Jumala koja iga-aastaseks kohendamiseks, ja rutake sellega!” Aga leviidid ei rutanud.
\par 6 Siis kutsus kuningas Joojada, ülempreestri, ja ütles temale: „Miks sa ei nõua leviitidelt, et nad tooksid Juudast ja Jeruusalemmast Issanda sulase Moosese seatud maksu Iisraeli kogudusele tunnistustelgi tarvis?
\par 7 Sest nurjatu Atalja ja tema pojad tungisid Jumala kotta ja nad on ka kõik Issanda koja pühitsetud annid kulutanud baalide jaoks.”
\par 8 Siis kuningas andis käsu, ja nad tegid ühe laeka ning asetasid selle Issanda koja väravasse, väljapoole.
\par 9 Ja Juudas ja Jeruusalemmas kuulutati, et Issandale tuleb tuua maksu, mis Jumala sulane Mooses oli kõrbes Iisraelile peale pannud.
\par 10 Ja kõik vürstid ning kogu rahvas olid rõõmsad ja tõid maksu ning heitsid laekasse, kuni see täis sai.
\par 11 Ja iga kord, kui laegas viidi leviitide poolt kuninga ametnike juurde ja kui need nägid, et raha oli palju, tulid kuninga kirjutaja ja ülempreestri volinik, tühjendasid laeka ja kandsid selle siis tagasi ta paika; nõnda tegid nad päev-päevalt ja kogusid palju raha.
\par 12 Ja kuningas ja Joojada andsid seda neile, kelle hooleks oli Issanda kojas tehtav töö, ja nemad palkasid kiviraiujaid ja puuseppi Issanda koda uuendama, samuti ka raud- ja vaskseppi Issanda koda kohendama.
\par 13 Ja töötegijad töötasid ning parandamine edenes nende käes; nad seadsid Jumala koja korda, nagu see pidi olema, ja tegid selle kindlaks.
\par 14 Ja kui nad olid sellega valmis saanud, siis nad tõid ülejäänud raha kuninga ja Joojada ette; ja sellest tehti Issanda koja riistu, teenistus- ja ohvririistu ja peekreid ja kuld- ja hõbeasju. Ja Issanda kojas ohverdati alati põletusohvreid, niikaua kui Joojada elas.
\par 15 Aga Joojada sai vanaks ja elatanuks, ja ta suri; surres oli ta sada kolmkümmend aastat vana.
\par 16 Ja ta maeti Taaveti linna kuningate juurde, sest ta oli Iisraelis head teinud Jumalale ja tema kojale.
\par 17 Aga pärast Joojada surma tulid Juuda vürstid ja kummardasid kuningat; siis kuningas kuulas neid.
\par 18 Ja nad jätsid maha Issanda, oma vanemate Jumala koja ning teenisid viljakustulpi ja jumalakujusid; siis tabas viha Juudat ja Jeruusalemma nende süü pärast.
\par 19 Ta läkitas nende keskele prohveteid pöörama neid Issanda poole; ja need manitsesid neid, aga nad ei võtnud kuulda.
\par 20 Siis täitis Jumala Vaim Sakarja, preester Joojada poja, ja tema astus rahva ette ning ütles neile: „Nõnda ütleb Jumal: Mispärast te astute üle Issanda käskudest? See ei õnnestu teil! Et te olete Issanda maha jätnud, siis jätab tema teid maha.”
\par 21 Aga nad pidasid tema vastu vandenõu ja viskasid tema kuninga käsul Issanda koja õues kividega surnuks.
\par 22 Kuningas Joas ei mõelnud armastusele, mida Sakarja isa Joojada temale oli osutanud, vaid tappis tema poja. Ja too ütles surres: „Issand näeb ja tasub!”
\par 23 Ja aastavahetusel tuligi Süüria sõjavägi tema vastu; nad tulid Juudasse ja Jeruusalemma, hävitasid rahvalt kõik rahvajuhid ja läkitasid kogu oma saagi Damaskuse kuningale.
\par 24 Kuigi Süüria sõjavägi oli tulnud väheste meestega, andis Issand nende kätte väga suure sõjaväe, sest nad olid maha jätnud Issanda, oma vanemate Jumala; nõnda mõisteti kohut Joase üle.
\par 25 Ja kui nad tema kallalt ära läksid - nad jätsid ta maha raskesti haavatuna -, siis pidasid ta sulased tema vastu vandenõu preester Joojada poegade vere pärast; nad tapsid tema voodis ja nõnda ta suri. Ta maeti Taaveti linna, aga teda ei maetud kuningate hauda.
\par 26 Ja need, kes pidasid tema vastu vandenõu, olid Saabad, ammonlanna Simeati poeg, ja Joosabad, moablanna Simriti poeg.
\par 27 Aga tema pojad ja paljud ennustused tema kohta ja Jumala koja ehitamine, vaata, need on kirja pandud Kuningate raamatu „Seletuses”. Ja tema poeg Amasja sai tema asemel kuningaks.

\chapter{25}

\par 1 Amasja oli kuningaks saades kakskümmend viis aastat vana ja ta valitses Jeruusalemmas kakskümmend üheksa aastat; ta ema nimi oli Jooaddan, Jeruusalemmast.
\par 2 Tema tegi, mis õige oli Issanda silmis, aga mitte siirast südamest.
\par 3 Kui tal oli kuningriik kindlalt käes, siis ta tappis ära oma sulased, kes olid maha löönud kuninga, tema isa.
\par 4 Aga nende lapsi ta ei surmanud, sest nõnda on kirjutatud Seaduses, Moosese raamatus, milles Issand on käskinud ja öelnud: „Isad ei pea surema laste pärast ja lapsed ei pea surema isade pärast, vaid igaüks surgu oma patu pärast!”
\par 5 Ja Amasja kogus kokku Juuda mehed ning seadis nad perekondade kaupa tuhande- ja sajapealikute juhatuse alla, kogu Juuda ja Benjamini; ja ta luges nad ära, kahekümneaastased ja üle selle, ja leidis neid olevat kolmsada tuhat valitud sõjakõlvulist meest, piigi- ja kilbikandjat.
\par 6 Ja ta palkas Iisraelist sada tuhat vahvat võitlejat saja hõbetalendi eest.
\par 7 Aga üks jumalamees tuli tema juurde ja ütles: „Kuningas! Iisraeli sõjavägi ei tohi minna koos sinuga, sest Issand ei ole Iisraeliga ega ühegi Efraimi lapsega.
\par 8 Sest kui sa nõnda tugevdatult sõtta lähed, paneb Jumal sind komistama vaenlase ees; Jumalal on ju võimus aidata või panna komistama.”
\par 9 Siis küsis Amasja jumalamehelt: „Aga kuidas jääb selle saja talendiga, mis ma olen andnud Iisraeli väehulgale?„ Ja jumalamees vastas: ”Issand võib sulle anda rohkem, kui see on!”
\par 10 Seepeale eraldas Amasja selle väehulga, kes Efraimist oli tulnud tema juurde, et see läheks koju. Siis süttis nende viha väga põlema Juuda vastu ja nad läksid tulise vihaga koju tagasi.
\par 11 Aga Amasja võttis julguse ja viis oma rahva välja, läks Soolaorgu ja lõi seirlastest maha kümme tuhat.
\par 12 Ja Juuda lapsed võtsid kümme tuhat elusalt vangi, viisid nad ühe kalju tippu ja paiskasid kaljutipust alla, nõnda et nad kõik kukkusid puruks.
\par 13 Aga väehulga mehed, keda Amasja oli tagasi saatnud, et nad ei tuleks sõtta koos temaga, tungisid Juuda linnadesse Samaariast kuni Beet-Hooronini ja lõid neist maha kolm tuhat ning riisusid palju saaki.
\par 14 Ja seejärel, kui Amasja oli edomlasi löömast tagasi tulnud, tõi ta seirlaste jumalad ja pani need enesele jumalaiks, kummardas nende ees ja suitsutas neile.
\par 15 Siis Issanda viha süttis põlema Amasja vastu ja ta läkitas tema juurde prohveti, kes temale ütles: „Mispärast sa otsid teise rahva jumalaid, kes ei suutnud päästa oma rahvast sinu käest?”
\par 16 Ja kui prohvet temale nõnda ütles, vastas Amasja temale: „Kas me oleme pannud sinu kuningale nõuandjaks? Jäta järele! Miks peaks sind maha löödama?„ prohvet peatus ja ütles: ”Ma tean, et Jumal on võtnud nõuks sind hävitada, sellepärast et sa teed nõnda ega võta kuulda minu nõu.”
\par 17 Ja Juuda kuningas Amasja pidas nõu ning läkitas ütlema Iisraeli kuningale Joasele, Jehu poja Jooahase pojale: „Tule, vaadakem teineteisele näkku!”
\par 18 Aga Iisraeli kuningas Joas läkitas Juuda kuningale Amasjale vastuse: „Liibanoni orjavits läkitas ütlema Liibanoni seedrile: Anna oma tütar naiseks mu pojale! Aga Liibanoni metsloomad läksid mööda ja tallasid orjavitsa ära.
\par 19 Sa mõtled, vaata, et sa oled edomlased maha löönud ja seepärast ajab su süda sind püüdma au. Aga jää nüüd koju! Mispärast sa tikud õnnetusse, kus sa langed ise ja koos sinuga Juuda?”
\par 20 Aga Amasja ei võtnud kuulda, sest see tuli Jumalalt, et neid anda Joase kätte, sellepärast et nad olid otsinud Edomi jumalaid.
\par 21 Ja Iisraeli kuningas Joas tuli üles ja nad vaatasid teineteisele näkku, tema ja Juuda kuningas Amasja, Juudale kuuluvas Beet-Semesis.
\par 22 Aga Juuda löödi maha Iisraeli ees ja igamees põgenes oma telki.
\par 23 Ja Iisraeli kuningas Joas võttis kinni Juuda kuninga Amasja, Jooahase poja Joase poja, Beet-Semesis ja viis ta Jeruusalemma. Ta kiskus ka maha nelisada küünart Jeruusalemma müüri Efraimi väravast kuni Nurgaväravani.
\par 24 Ja ta võttis ära kõik kulla ja hõbeda, samuti kõik riistad, mis leidusid Jumala kojas Oobed-Edomi juures, ja kuningakoja varanduse, võttis pantvange ja läks tagasi Samaariasse.
\par 25 Aga Juuda kuningas Amasja, Joase poeg, elas pärast Iisraeli kuninga Joase, Jooahase poja surma viisteist aastat.
\par 26 Ja muud Amasja lood, varasemad ja hilisemad, vaata, eks need ole kirja pandud Juuda ja Iisraeli Kuningate raamatus?
\par 27 Ja alates sellest ajast, kui Amasja taganes Issanda järelt, peeti Jeruusalemmas tema vastu vandenõu. Ta põgenes Laakisesse, aga nad läkitasid mehed Laakisesse temale järele ja surmasid ta seal.
\par 28 Ta tõsteti hobuse selga ja maeti oma vanemate juurde Juuda linna.

\chapter{26}

\par 1 Ja kogu Juuda rahvas võttis Ussija, kes oli kuusteist aastat vana, ja tõstis ta tema isa Amasja asemele kuningaks.
\par 2 Tema ehitas üles Eelati ja tõi selle tagasi Juudale, pärast seda kui kuningas oli läinud magama oma vanemate juurde.
\par 3 Ussija oli kuningaks saades kuusteist aastat vana ja ta valitses Jeruusalemmas viiskümmend kaks aastat; ta ema nimi oli Jekolja, Jeruusalemmast.
\par 4 Tema tegi, mis õige oli Issanda silmis, kõigiti nõnda, nagu tema isa Amasja oli teinud.
\par 5 Ta otsis Jumalat, niikaua kui elas Sakarja, kes oli arukas Jumala nägemises; ja niikaua kui ta otsis Issandat, andis Jumal temale edu.
\par 6 Ta läks välja ja sõdis vilistite vastu ning kiskus maha Gati müüri, Jabne müüri ja Asdodi müüri ning ehitas linnu Asdodi ja vilistite maa-alale.
\par 7 Jumal aitas teda vilistite vastu ja nende araablaste vastu, kes elasid Guur-Baalis, ning meunlaste vastu.
\par 8 Ja ammonlased andsid Ussijale ande ning tema kuulsus levis kuni Egiptuseni, sest ta sai väga võimsaks.
\par 9 Ja Ussija ehitas Jeruusalemma tornid Nurgavärava, Oruvärava ja Kõveriku peale ning kindlustas need.
\par 10 Ta ehitas torne kõrbe ja kaevas palju kaevusid, sest tal oli palju karja madalmaal ja tasandikul, põllumehi ja viinamäetöölisi mäestikus ja viljapuuaedades, kuna ta armastas põllutööd.
\par 11 Ja Ussijal oli võitlusvalmis sõjavägi, kes rühmadena sõtta läks, olles arvuliselt ära loetud kirjutaja Jeieli ja nimestiku hooldaja Maaseja poolt Hananja juhatusel; Hananja oli üks kuninga pealikuid.
\par 12 Perekondade peameeste koguarv vahvate võitlejate hulgas oli kaks tuhat kuussada.
\par 13 Nende juhatada oli sõjavägi, kolmsada seitse tuhat viissada meest, võitlusvalmis ja küllalt tugev, et aidata kuningat vaenlase vastu.
\par 14 Ja Ussija varustas neid, kogu sõjaväge, kilpide, piikide, kiivrite, soomusrüüde, ambude ja lingukividega.
\par 15 Ja ta tegi Jeruusalemma osavalt leiutatud seadeldised tornide ja nurkade peale, noolte ja suurte kivide heitmiseks; ja tema kuulsus levis kaugele, sest ta sai imepäraselt abi, kuni ta sai vägevaks.
\par 16 Aga kui ta oli saanud vägevaks, läks ta süda ülbeks, nõnda et ta talitas kõlvatult ja murdis truudust Issandale, oma Jumalale: ta läks Issanda templisse suitsutama suitsutusaltaril.
\par 17 Siis läks preester Asarja tema järel sisse ja koos temaga kaheksakümmend Issanda preestrit, kõik tublid mehed.
\par 18 Nemad astusid kuningas Ussijale vastu ning ütlesid temale: „Pole sinu asi, Ussija, Issandale suitsutada, vaid see on preestrite, Aaroni poegade asi, kes on pühitsetud suitsutama. Mine pühamust välja, sest sa oled üle astunud ja see ei tule sulle auks Issandalt Jumalalt!”
\par 19 Siis vihastus Ussija, kel oli parajasti käes suitsutuspann suitsutamiseks; ja kui ta preestrite pärast vihastus, lõi Issanda kojas suitsutusaltari juures preestrite nähes ta laubal välja pidalitõbi.
\par 20 Ja kui ülempreester Asarja ja kõik preestrid vaatasid tema poole, vaata, siis oli ta laubal pidalitõbi. Siis nad ajasid ta sealt kiiresti ära, ja ka tema ise tõttas välja, sest Issand oli teda löönud.
\par 21 Ja kuningas Ussija oli pidalitõbine kuni oma surmapäevani ning elas pidalitõbisena omaette kojas, sest teda hoiti Issanda kojast eemal. Ja Jootam, tema poeg, valitses kuningakoja üle ning mõistis kohut maa rahvale.
\par 22 Ja muud Ussija lood, varasemad ja hilisemad, on prohvet Jesaja, Aamotsi poeg, kirja pannud.
\par 23 Siis Ussija läks magama oma vanemate juurde ja ta maeti oma vanemate juurde matmisväljale, mis kuulus kuningatele, sest nad ütlesid: „Ta on pidalitõbine.” Ja tema poeg Jootam sai tema asemel kuningaks.

\chapter{27}

\par 1 Jootam oli kuningaks saades kakskümmend viis aastat vana ja ta valitses Jeruusalemmas kuusteist aastat; ta ema nimi oli Jeruusa, Saadoki tütar.
\par 2 Tema tegi, mis õige oli Issanda silmis, kõigiti nõnda, nagu ta isa Ussija oli teinud, ainult ta ei tunginud Issanda templisse. Aga rahvas talitas veelgi kõlvatult.
\par 3 Ta ehitas Issanda koja ülemise värava ja ta ehitas palju templikünka müüri.
\par 4 Ta ehitas linnu Juuda mäestikku ning ehitas kindlustatud paiku ja torne metsadesse.
\par 5 Ta sõdis ammonlaste kuninga vastu ning võitis nad; ja ammonlased andsid temale sel aastal sada talenti hõbedat, kümme tuhat koori nisu ja kümme tuhat koori otri; seda tõid ammonlased temale ka teisel ja kolmandal aastal.
\par 6 Ja Jootam sai vägevaks, sest ta korrastas oma teid Issanda, oma Jumala ees.
\par 7 Aga muud Jootami lood, kõik ta sõjad ja ettevõtmised, vaata, need on kirja pandud Iisraeli ja Juuda Kuningate raamatus.
\par 8 Ta oli kuningaks saades kakskümmend viis aastat vana ja ta valitses Jeruusalemmas kuusteist aastat.
\par 9 Siis Jootam läks magama oma vanemate juurde ja ta maeti Taaveti linna. Ja tema poeg Aahas sai tema asemel kuningaks.

\chapter{28}

\par 1 Aahas oli kuningaks saades kakskümmend aastat vana ja ta valitses Jeruusalemmas kuusteist aastat. Tema ei teinud, mis õige oli Issanda silmis, nõnda nagu tema isa Taavet,
\par 2 vaid käis Iisraeli kuningate teedel ja tegi ka valatud kujusid baalidele.
\par 3 Ja see oli tema, kes suitsutas Ben-Hinnomi orus ja laskis oma pojad käia läbi tule nende rahvaste jäledate tegude eeskujul, kelle Issand oli ära ajanud Iisraeli laste eest.
\par 4 Ta ohverdas ning suitsutas ohvrikünkail ja kõrgendikel ja iga halja puu all.
\par 5 Ja Issand, tema Jumal, andis ta Süüria kuninga kätte; ja nad lõid teda ning võtsid temalt hulga vange ja viisid Damaskusesse. Ja ta anti ka Iisraeli kuninga kätte, kes valmistas temale suure kaotuse.
\par 6 Pekah, Remalja poeg, tappis ühe päevaga Juudast sada kakskümmend tuhat, kõik vahvad mehed, sellepärast et nad olid hüljanud Issanda, oma vanemate Jumala.
\par 7 Ja Sikri, Efraimi kangelane, tappis Maaseja, kuninga poja, ja Asrikami, kojaülema, ja Elkana, kes oli kuningast järgmine.
\par 8 Ja Iisraeli lapsed viisid oma vendadelt vangi naisi, poegi ja tütreid - kakssada tuhat, riisusid neilt ka palju saaki ja viisid saagi Samaariasse.
\par 9 Aga seal oli Issanda prohvet, Ooded nimi, ja tema läks vastu sõjaväele, kes oli teel Samaariasse, ja ütles neile: „Vaata, Issand, teie vanemate Jumal, on vihastunud Juuda peale ja on andnud nad teie kätte. Ja teie olete neid tapnud niisuguse vihaga, mis ulatub taevani.
\par 10 Ja nüüd te mõtlete alistada Juuda ja Jeruusalemma lapsi endile sulaseiks ja teenijaiks. Kas pole ka teil endil süüd Issanda, teie Jumala ees?
\par 11 Aga kuulge nüüd mind ja viige tagasi vangid, keda te olete võtnud oma vendadelt, muidu tabab teid Issanda tuline viha!”
\par 12 Siis tõusid mõned mehed efraimlaste peameestest: Asarja, Joohanani poeg, Berekja, Mesillemoti poeg, Hiskija, Sallumi poeg, ja Amaasa, Hadlai poeg, sõjast tulijaile vastu
\par 13 ja ütlesid neile: „Ärge tooge vange siia, sest te teete meid süüdlasteks Issanda ees, kui mõtlete lisa tuua meie pattudele ja süütegudele! Tõesti, meie süü on suur ja Iisraeli peal on tuline viha!”
\par 14 Siis sõjamehed loobusid vangidest ja saagist vürstide ja terve koguduse ees.
\par 15 Ja nimeliselt nimetatud mehed tõusid ning võtsid vangid ja riietasid saagist võetuga kõik, kes neist olid alasti: nad andsid neile riided selga ja jalatsid jalga, söötsid ja jootsid neid, võidsid neid ja panid eeslite selga kõik, kes olid nõrgad, ja viisid need Jeerikosse, Palmidelinna, nende vendade juurde. Seejärel tulid nad Samaariasse tagasi.
\par 16 Selsamal ajal läkitas kuningas Aahas sõna Assuri kuningatele, et nad aitaksid teda.
\par 17 Sest ka edomlased tulid jälle ja lõid Juudat ning võtsid vange.
\par 18 Ja vilistid tungisid Juuda madalmaa ja lõunamaa linnadesse ning vallutasid Beet-Semesi, Ajjaloni, Gederoti, Sooko ja selle tütarlinnad, Timna ja selle tütarlinnad, Gimso ja selle tütarlinnad, ja asusid neisse.
\par 19 Sest Issand alandas Juudat Iisraeli kuninga Aahase pärast, sellepärast et ta andis Juudamaal vaba voli korralagedusele ja oli truuduseta Issanda vastu.
\par 20 Assuri kuningas Tiglat-Pileser aga tuli tema vastu ja kimbutas teda ega andnud temale toetust.
\par 21 Sest kuigi Aahas rüüstas Issanda koja ning kuninga ja vürstide kojad ja andis Assuri kuningale, ei olnud tal sellest abi.
\par 22 Aga isegi sel ajal, kui teda rõhuti, murdis ta üha truudust Issandale, seesama kuningas Aahas.
\par 23 Siis ta ohverdas Damaskuse jumalatele, kes teda olid löönud, ja ütles: „Kuna Süüria kuningate jumalad neid aitavad, siis ma ohverdan neile, et nad mindki aitaksid!” Aga need saidki komistuseks temale ja kogu Iisraelile.
\par 24 Ja Aahas kogus kokku Jumala koja riistad ja raius katki Jumala koja riistad; ta sulges Issanda koja uksed ja tegi enesele altarid igasse Jeruusalemma nurka.
\par 25 Ja ta tegi igasse Juuda linna ohvrikünkaid teistele jumalatele suitsutamiseks ning ärritas Issandat, oma vanemate Jumalat.
\par 26 Aga tema muud lood ja kõik tema ettevõtted, varasemad ja hilisemad, vaata, need on kirja pandud Juuda ja Iisraeli Kuningate raamatus.
\par 27 Siis Aahas läks magama oma vanemate juurde ja ta maeti linna, Jeruusalemma, aga teda ei viidud Iisraeli kuningate hauda. Ja tema poeg Hiskija sai tema asemel kuningaks.

\chapter{29}

\par 1 Hiskija oli kuningaks saades kakskümmend viis aastat vana ja ta valitses Jeruusalemmas kakskümmend üheksa aastat; ta ema nimi oli Abija, Sakarja tütar.
\par 2 Tema tegi, mis õige oli Issanda silmis, kõigiti nõnda, nagu ta isa Taavet oli teinud.
\par 3 Oma valitsemise esimese aasta esimeses kuus avas ta Issanda koja uksed ja parandas need.
\par 4 Ja ta laskis tulla preestrid ja leviidid ning kogus need idapoolsele väljakule.
\par 5 Ta ütles neile: „Kuulge mind, leviidid! Pühitsege nüüd iseendid ja pühitsege Issanda, oma vanemate Jumala koda ja viige saast pühamust välja!
\par 6 Sest meie vanemad on üle astunud ja kurja teinud Issanda, meie Jumala silmis ja on tema maha jätnud. Nad pöörasid ära oma palge Issanda elamu poolt ja pöörasid selja.
\par 7 Nad sulgesid ka eeskoja uksed ja kustutasid lambid; nad ei toonud suitsutusohvreid ega ohverdanud pühamus põletusohvreid Iisraeli Jumalale.
\par 8 Seepärast tabas Issanda viha Juudat ja Jeruusalemma ja ta tegi need ehmatuse, hirmu ja vilistamise põhjuseks, nagu te oma silmaga näete.
\par 9 Jah, vaata, meie isad on langenud mõõga läbi ja meie pojad, meie tütred ja meie naised on sellepärast vangi viidud.
\par 10 Nüüd on mul südame peal teha leping Issandaga, Iisraeli Jumalaga, et ta tuline viha meilt pöörduks.
\par 11 Mu pojad, ärge nüüd puhake, sest Issand on teid valinud seisma tema ees, teda teenima ja olema temale teenreiks ning suitsutajaiks!”
\par 12 Siis leviidid Mahat, Amaasai poeg, ja Joel, Asarja poeg, Kehati poegadest; ja Merari poegadest Kiis, Abdi poeg, ja Asarja, Jehalleleeli poeg; ja geersonlastest Joah, Simma poeg, ja Eeden, Joahi poeg;
\par 13 ja Elisafani poegadest Simri ja Jeiel; ja Aasafi poegadest Sakarja ja Mattanja;
\par 14 ja Heemani poegadest Jehiel ja Simei; ja Jedutuuni poegadest Semaja ja Ussiel,
\par 15 võtsid kätte, kogusid oma vennad, pühitsesid endid ja läksid, nagu kuningas Issanda sõna kohaselt oli käskinud, Issanda koda puhastama.
\par 16 Aga preestrid läksid Issanda koja siseruumi puhastama ja tõid kõik saasta, mis nad Issanda templist leidsid, Issanda koja õue; ja leviidid võtsid selle ning viisid välja Kidroni jõkke.
\par 17 Nad hakkasid pühitsema esimese kuu esimesel päeval ja jõudsid kuu kaheksandal päeval Issanda koja eeskotta; nad pühitsesid Issanda koda kaheksa päeva ja lõpetasid esimese kuu kuueteistkümnendal päeval.
\par 18 Siis nad läksid kuningas Hiskija juurde sisse ja ütlesid: „Me oleme puhastanud kogu Issanda koja, põletusohvrialtari ja kõik selle riistad.
\par 19 Nõndasamuti oleme korda seadnud ja pühitsenud kõik riistad, mis kuningas Aahas oma valitsemisajal oli ära visanud, kui ta truudust murdis, ja vaata, need on Issanda altari ees.”
\par 20 Siis kuningas Hiskija tõusis järgmisel hommikul vara ja kogus linna ülemad ning läks üles Issanda kotta.
\par 21 Ja nad tõid seitse härjavärssi, seitse jäära, seitse oinastalle ja seitse sikku patuohvriks kuningakoja, pühamu ja Juuda eest. Ja ta käskis Aaroni poegi, preestreid, et nad ohverdaksid need Issanda altaril.
\par 22 Siis nad tapsid veised ja preestrid võtsid vere ning piserdasid altarile; ja nad tapsid jäärad ning piserdasid vere altarile; siis nad tapsid oinastalled ning piserdasid vere altarile.
\par 23 Seejärel toodi patuohvri sikud kuninga ja koguduse ette, ja nad panid oma käed nende peale.
\par 24 Ja preestrid tapsid need ning ohverdasid nende vere patuohvriks altaril lepituse tegemiseks kogu Iisraeli eest, sest kuningas oli käskinud ohverdada põletus- ja patuohvri kogu Iisraeli eest.
\par 25 Ja ta asetas leviidid Issanda kotta simblite, naablite ja kanneldega Taaveti ja kuninga nägija Gaadi ja prohvet Naatani käsu kohaselt; sest see oli Issanda käsk tema prohvetite läbi.
\par 26 Ja leviidid astusid ette Taaveti mänguriistadega ning preestrid pasunatega.
\par 27 Ja Hiskija käskis ohverdada põletusohvreid altaril; ja kui ohverdamine algas, siis algas Issanda-laul ja pasunate puhumine Iisraeli kuninga Taaveti mänguriistade saatel.
\par 28 Ja terve kogudus kummardas, laul kõlas ja pasunad puhusid - see kõik kestis, kuni põletusohver oli ohverdatud.
\par 29 Ja kui ohverdamine oli lõpetatud, siis põlvitasid kuningas ja kõik, kes tema juures olid, ja kummardasid.
\par 30 Ja kuningas Hiskija ning ülemad käskisid leviite Issandat kiita Taaveti ja nägija Aasafi sõnadega; ja nad kiitsid rõõmsasti, põlvitasid ja kummardasid.
\par 31 Siis Hiskija võttis sõna ja ütles: „Nüüd olete teie oma käe täitnud Issandale. Astuge ette ja tooge tapa- ja tänuohvreid Issanda kojale!” Ja kogudus tõi tapa- ja tänuohvreid, ja igaüks, kes tahtis, põletusohvreid.
\par 32 Koguduse toodud põletusohvrite arv oli seitsekümmend veist, sada jäära, kakssada oinastalle - need kõik olid põletusohvriks Issandale.
\par 33 Ja pühitsetud ande oli kuussada veist ning kolm tuhat lammast ja kitse.
\par 34 Aga preestreid oli vähe ja nad ei suutnud nülgida kõiki põletusohvreid; siis aitasid neid nende vennad leviidid, kuni töö oli tehtud ja kuni preestrid olid endid pühitsenud, sest leviidid olid agaramad olnud endid pühitsema kui preestrid.
\par 35 Ka põletusohvreid oli palju koos tänuohvri rasvadega ja põletusohvri joogiohvritega. Nõnda taastati Issanda koja teenistus.
\par 36 Ja Hiskija ning kogu rahvas olid rõõmsad selle pärast, mis Jumal oli rahvale valmistanud, sest see asi oli sündinud hõlpsasti.

\chapter{30}

\par 1 Ja Hiskija läkitas sõna kogu Iisraelile ja Juudale ja kirjutas kirjad ka Efraimile ja Manassele, et nad tuleksid Issanda kotta Jeruusalemma pidama paasapüha Issanda, Iisraeli Jumala auks.
\par 2 Kuningas oma vürstidega ja kogu Jeruusalemma kogudusega oli otsustanud pidada paasapüha teises kuus.
\par 3 Sest nad ei saanud seda pidada õigel ajal, kuna preestrid ei olnud küllaldasel arvul endid pühitsenud ja rahvas ei olnud kogunenud Jeruusalemma.
\par 4 See asi oli õige kuninga silmis ja terve koguduse silmis.
\par 5 Ja nad otsustasid, et kutse pidi käima läbi kogu Iisraeli Beer-Sebast Daanini, et tuldaks Jeruusalemma paasapüha pidama Issanda, Iisraeli Jumala auks. Seda ei olnud peetud nii rohke osavõtuga, kui oli ette kirjutatud.
\par 6 Ja jooksjad käisid kirjadega kuningalt ja tema vürstidelt läbi kogu Iisraeli ja Juuda ning ütlesid, nagu oli kuninga käsk: „Iisraeli lapsed, pöörduge taas Issanda, Aabrahami, Iisaki ja Iisraeli Jumala juurde, siis ta pöördub selle jäägi juurde, kes on pääsenud Assuri kuningate pihust!
\par 7 Ärge olge oma vanemate ja vendade sarnased, kes murdsid truudust Issandale, oma vanemate Jumalale, ja keda ta andis koleduste kätte, nõnda nagu te ise näete!
\par 8 Ärge tehke nüüd oma kaela kangeks, nagu teie vanemad tegid, andke käsi Issandale ja tulge tema pühamusse, mille ta on pühitsenud igaveseks ajaks, ja teenige Issandat, oma Jumalat, et ta pööraks teie pealt oma tulise viha!
\par 9 Sest kui te pöördute tagasi Issanda juurde, siis leiavad teie vennad ja pojad halastust oma vangistajate ees ja võivad tulla tagasi sellele maale. Sest Issand, teie Jumal, on armuline ja halastaja, ja ta ei pööra teilt oma palet, kui te pöördute tagasi tema juurde.”
\par 10 Ja jooksjad käisid linnast linna Efraimi ja Manasse maal kuni Sebulonini; aga neid naerdi ja pilgati.
\par 11 Ainult mehed Aaserist, Manassest ja Sebulonist alandasid endid ja tulid Jeruusalemma.
\par 12 Ka Juudas oli Jumala käsi, andes neile üksmeelse südame kuninga ja vürstide Issanda sõna kohase käsu täitmiseks.
\par 13 Nõnda kogunes palju rahvast Jeruusalemma, et pidada hapnemata leibade püha teises kuus - väga suur kogudus.
\par 14 Ja nad võtsid kätte ja kõrvaldasid Jeruusalemmas olevad altarid; nad kõrvaldasid ka kõik suitsutusaltarid ja viskasid Kidroni jõkke.
\par 15 Teise kuu neljateistkümnendal päeval tapsid nad paasatalle. Aga preestrid ja leviidid häbenesid, ja nad pühitsesid endid ning tõid Issanda kotta põletusohvreid.
\par 16 Ja nad asusid oma kohtadele jumalamehe Moosese Seaduse korra kohaselt: preestrid piserdasid leviitide käest vastuvõetud verd.
\par 17 Aga koguduses olid paljud, kes ei olnud ennast pühitsenud; seepärast tapsid leviidid paasatallesid kõigile, kes ei olnud puhtad, et neid pühitseda Issandale.
\par 18 Sest suur osa rahvast, paljud Efraimist, Manassest, Issaskarist ja Sebulonist, ei olnud ennast puhastanud, vaid nad sõid paasatalle teisiti, kui oli ette kirjutatud. Aga Hiskija oli nende eest palunud, öeldes: „Hea Issand annab andeks
\par 19 igaühele, kes valmistab oma südant otsima Jumalat, Issandat, oma vanemate Jumalat, kuigi ta ei ole puhas, nagu pühadus nõuab!”
\par 20 Ja Issand kuulis Hiskijat ning säästis rahva.
\par 21 Ja Iisraeli lapsed, kes olid Jeruusalemmas, pidasid hapnemata leibade püha seitse päeva suure rõõmuga. Ja leviidid ja preestrid kiitsid iga päev Issandat võimsate mänguriistadega.
\par 22 Ja Hiskija ütles tunnustust kõigile leviitidele, kes olid näidanud nii head arusaamist Issanda teenistuses. Ja nad sõid pühade ohvreid seitse päeva, ohverdasid tänuohvreid ja ülistasid Issandat, oma vanemate Jumalat.
\par 23 Ja terve kogudus otsustas veel teist seitse päeva püha pidada, ja nõnda nad pidasid rõõmsasti püha ka need seitse päeva.
\par 24 Sest Hiskija, Juuda kuningas, oli kogudusele annetanud tuhat härjavärssi ning seitse tuhat lammast ja kitse; ja vürstid olid kogudusele annetanud tuhat härjavärssi ning kümme tuhat lammast ja kitse; ka olid paljud preestrid endid pühitsenud.
\par 25 Ja terve Juuda kogudus, preestrid ja leviidid, ja kõik Iisraelist tulnud kogudus, nõndasamuti võõrad, kes olid tulnud Iisraelimaalt või elasid Juudas, olid rõõmsad.
\par 26 Ja Jeruusalemmas oli rõõm suur, sest Iisraeli kuninga Saalomoni, Taaveti poja päevist saadik ei olnud midagi niisugust Jeruusalemmas sündinud.
\par 27 Ja leviitpreestrid tõusid ning õnnistasid rahvast. Issand võttis kuulda nende häält ja nende palve jõudis tema pühasse eluasemesse, taevasse.

\chapter{31}

\par 1 Kui see kõik oli lõpetatud, siis läks kogu kohalolev Iisrael välja Juuda linnadesse; ja nad purustasid ebaususambad, raiusid katki viljakustulbad ning kiskusid maha ohvrikünkad ja altarid kogu Juudas, Benjaminis, Efraimis ja Manasses, kuni kõik oli hävitatud. Siis läksid kõik Iisraeli lapsed tagasi, igaüks oma pärisosale, oma linnadesse.
\par 2 Ja Hiskija seadis preestrite ja leviitide rühmad rühmade viisi, igaühe vastavalt teenistusele, mis preestritele ja leviitidele oli ette nähtud, tooma põletus- ja tänuohvreid, teenima, tänama ja kiitma Issanda leeri väravais.
\par 3 Ja kuninga poolt oli toetus tema isiklikust omandist põletusohvrite tarvis: hommikusteks ja õhtusteks põletusohvriteks ja põletusohvriteks hingamispäevil, noorkuupäevil ja pühil, nõnda nagu Issanda Seaduses on kirjutatud.
\par 4 Ja ta käskis rahvast, kes elas Jeruusalemmas, anda preestritele ja leviitidele nende osa, et nad püsiksid Issanda Seaduses.
\par 5 Ja kui see käsk oli laiali läinud, siis tõid Iisraeli lapsed palju uudsesaaki viljast, veinist, õlist ja meest ning igasugu põllusaadustest; nad tõid rohkesti kümnist kõigest.
\par 6 Ja Iisraeli ja Juuda lapsed, kes elasid Juuda linnades, needki tõid kümnist veistest ning lammastest ja kitsedest, samuti kümnist pühadest andidest, mis olid pühitsetud Issandale, nende Jumalale, ja panid selle kuhjadesse.
\par 7 Nad hakkasid panema kuhjadesse kolmandas kuus ja lõpetasid seitsmendas kuus.
\par 8 Kui Hiskija ja vürstid tulid ning nägid kuhje, siis nad kiitsid Issandat ja tema rahvast Iisraeli.
\par 9 Ja kui Hiskija küsitles preestreid ja leviite nende kuhjade kohta,
\par 10 siis vastas temale ülempreester Asarja, Saadoki soost, ja ütles: „Sellest alates, kui Issanda kotta hakati tooma tõstelõivu, oleme söönud ja meie kõhud on täis, ja veelgi on palju järele jäänud; sest Issand on õnnistanud oma rahvast ja nõnda on see suur kogus üle jäänud.”
\par 11 Siis Hiskija käskis kambrid Issanda kojas korda seada; ja kui need olid korda seatud,
\par 12 siis viidi tõstelõiv, kümnis ja pühad annid hoolsasti sisse. Nende ülevaatajaks sai leviit Konanja; ja tema vend Simei oli teiseks.
\par 13 Ja Jehiel, Asasja, Nahat, Asael, Jerimot, Joosabad, Eliel, Jismakja, Mahet ja Benaja olid ametnikud Konanja ja tema venna Simei käe all kuningas Hiskija ja Jumala koja eestseisja Asarja määramise kohaselt.
\par 14 Ja leviit Koore, Jimna poeg, idapoolse värava hoidja, oli Jumalale toodavate vabatahtlike andide hooldaja, kes pidi jaotama Issanda tõstelõivu ja kõige pühamaid ande.
\par 15 Ja tema käe all olid Eeden, Minjamin, Jeesua, Semaja, Amarja ja Sekanja, kes pidid preestrite linnades ustavalt jaotama oma vendadele, niihästi vanemaile kui nooremaile rühmade kaupa,
\par 16 vastavalt nende suguvõsakirja kandmisele, meessoost kolmeaastased ja üle selle, kõigile, kes pidid tulema Issanda kotta igapäevast teenistuskohustust täitma oma rühmade kaupa.
\par 17 Ja preestrid kanti suguvõsakirja nende perekondade kaupa, aga leviidid, kahekümneaastased ja üle selle, vastavalt nende teenistuskohustustele rühmade kaupa.
\par 18 Nad kanti suguvõsakirja kõigi oma laste, naiste, poegade ja tütardega, terve kogudusena, sest oma ametikohustuse tõttu said nad pühitsetuiks.
\par 19 Ja Aaroni poegadel, preestritel, kes elasid oma karjamaalinnade väljadel, olid igas linnas nimeliselt määratud mehed, kes pidid osa kätte andma igale meesterahvale preestrite hulgas ja igale suguvõsakirja kantud leviidile.
\par 20 Nõnda tegi Hiskija kogu Juudas; ta tegi, mis oli hea, õige ja kohus Issanda, tema Jumala ees.
\par 21 Ja iga tööd, mida ta alustas, et otsida oma Jumalat, olgu Jumala koja teenistuses või Seaduse ja käskudega seoses, tegi ta kõigest südamest ja see läks tal korda.

\chapter{32}

\par 1 Pärast neid sündmusi ja seda osutatud ustavust tuli Assuri kuningas Sanherib ja tungis Juudasse, lõi kindlustatud linnade vastu leeri üles ja mõtles need enesele vallutada.
\par 2 Kui Hiskija nägi, et Sanherib tuleb ja et tema eesmärgiks on sõdida Jeruusalemma vastu,
\par 3 siis ta pidas nõu oma vürstide ja võitlejatega veeallikate sulgemiseks, mis olid väljaspool linna; ja temaga oldi nõus.
\par 4 Siis koguti kokku palju rahvast ja nad sulgesid kõik allikad ja jõe, mis maa all voolas, öeldes: „Miks peaksid Assuri kuningad, kui nad tulevad, leidma nii palju vett?”
\par 5 Siis ta kindlustas ennast ja ehitas üles kõik mahakistud müüriosad, ehitas kõrgemaks tornid ja väljapoole veel teise müüri, kindlustas kantsi, Taaveti linna, valmistas palju viskodasid ja kilpe.
\par 6 Ja ta seadis rahvale sõjapealikud ning kogus need enese juurde linna värava väljakule ja kõneles neile julgustavalt, öeldes:
\par 7 „Olge vahvad ja tugevad, ärge kartke ja ärge kohkuge Assuri kuninga ees ja kogu selle jõugu ees, kes on koos temaga! Sest see, kes on meiega, on suurem kui see, kes on temaga.
\par 8 Temaga on lihane käsivars, aga meiega on Issand, meie Jumal, meid aitamas ja meie võitlusi võitlemas.” Ja rahvas toetus Juuda kuninga Hiskija sõnadele.
\par 9 Pärast seda läkitas Assuri kuningas Sanherib, olles ise kogu oma väega Laakise all, oma sulased Jeruusalemma Juuda kuninga Hiskija juurde ja kogu Jeruusalemmas oleva Juuda juurde ütlema:
\par 10 „Nõnda ütleb Sanherib, Assuri kuningas: Mille peale te loodate, et istute sissepiiratud Jeruusalemmas?
\par 11 Kas mitte Hiskija teid ei ässita, saates teid surema nälga ja janusse, kui ta ütleb: Issand, meie Jumal, päästab meid Assuri kuninga pihust?
\par 12 Eks ole Hiskija ise kõrvaldanud tema ohvrikünkad ja altarid ning andnud käsu Juudale ja Jeruusalemmale, öeldes: Te peate kummardama üheainsa altari ees ja suitsutama selle peal?
\par 13 Kas te ei tea, mida mina ja mu isad oleme teinud kõigi teiste maade rahvastega? Kas nende maade rahvaste jumalad on suutnud päästa oma maad minu käest?
\par 14 Kes kõigist nende rahvaste jumalaist, kelle mu isad hävitasid, on suutnud päästa oma rahva minu käest? Kuidas siis teie jumal suudab teid päästa minu käest?
\par 15 Ärge siis nüüd laske endid petta Hiskija poolt ega sel viisil ässitada ja ärge uskuge teda, sest mitte ühegi rahva või kuningriigi jumal pole suutnud päästa oma rahvast minu ja mu isade käest, veel vähem siis teie jumal: ei ta päästa teid minu käest!”
\par 16 Ja tema sulased rääkisid veel rohkemgi Issanda Jumala ja tema sulase Hiskija vastu.
\par 17 Ta kirjutas ka kirju Issanda, Iisraeli Jumala teotamiseks ja rääkis tema vastu, öeldes: „Nõnda nagu muude maade rahvaste jumalad ei päästnud oma rahvast minu käest, nõnda ei päästa ka Hiskija jumal oma rahvast minu käest!”
\par 18 Ja nad hüüdsid valju häälega juudi keeles müüri peal olevale Jeruusalemma rahvale, et neid hirmutada ja kartma panna, et nad siis saaksid linna vallutada.
\par 19 Ja nad rääkisid Jeruusalemma Jumalast nõnda nagu muude maade rahvaste jumalaist, kes olid inimeste kätetöö.
\par 20 Aga kuningas Hiskija ja prohvet Jesaja, Aamotsi poeg, palvetasid selle pärast ja kisendasid taeva poole.
\par 21 Siis Issand läkitas ingli, kes hävitas kõik sõjakangelased, pealikud ja vürstid Assuri kuninga leeris, ja too pidi minema tagasi oma maale, silmad häbi täis. Ja kui ta läks oma jumala kotta, siis ta oma lihased järeltulijad lõid tema seal mõõgaga maha.
\par 22 Nõnda päästis Issand Hiskija ja Jeruusalemma elanikud Assuri kuninga Sanheribi käest ja kõigi käest ja andis neile rahu ümberkaudu.
\par 23 Ja paljud tõid Jeruusalemma ande Issandale ja kalleid asju Juuda kuningale Hiskijale; ja ta tõusis pärast seda kõigi rahvaste silmis.
\par 24 Neil päevil jäi Hiskija haigeks ja oli suremas; siis palvetas ta Issanda poole, kes vastas temale ja andis temale tunnustähe.
\par 25 Aga Hiskija ei tasunud heategu, mis temale tehti, vaid ta süda oli ülbe; seepärast oleks viha pidanud tabama teda, Juudat ja Jeruusalemma.
\par 26 Et aga Hiskija oma südame ülbuses ennast alandas, tema ja Jeruusalemma elanikud, siis ei tabanud Issanda viha neid Hiskija päevil.
\par 27 Ja Hiskijal oli väga palju rikkust ning au; ta hankis enesele hõbeda, kulla, kalliskivide, palsamite, kilpide ja igasugu kalliste asjade tagavarasid,
\par 28 ja varaaitu viljasaagi, veini ja õli tarvis, ja lautu igasugu loomadele ning tarasid karjadele.
\par 29 Ta ehitas enesele linnu ja tal oli suur kari lambaid, kitsi ja veiseid, sest Jumal andis temale väga suure varanduse.
\par 30 Ja tema, Hiskija, sulges ka Giihoni vete ülemjooksu ja juhtis veed alla lääne poole, Taaveti linna; Hiskijal läks korda kõik, mis ta tegi.
\par 31 Ja samuti siis, kui Paabeli vürstide vahemehed olid läkitatud tema juurde küsima tunnustähe asjas, mis maal oli aset leidnud, laskis Jumal teda vabalt toimida, kusjuures ta pani teda proovile, et teada saada kõike, mis tal südames oli.
\par 32 Ja muud Hiskija lood ja tema vagad teod, vaata, need on kirja pandud prohvet Jesaja, Aamotsi poja „Nägemustes”, Juuda ja Iisraeli Kuningate raamatus.
\par 33 Siis Hiskija läks magama oma vanemate juurde ja ta maeti Taaveti poegade haudade astangule; kogu Juuda ja Jeruusalemma elanikud austasid teda, kui ta oli surnud. Ja tema poeg Manasse sai tema asemel kuningaks.

\chapter{33}

\par 1 Manasse oli kuningaks saades kaksteist aastat vana ja ta valitses Jeruusalemmas viiskümmend viis aastat.
\par 2 Tema tegi kurja Issanda silmis nende rahvaste jõleduste eeskujul, kelle Issand oli ära ajanud Iisraeli laste eest.
\par 3 Tema ehitas jälle üles ohvrikünkad, mis ta isa Hiskija oli maha kiskunud, ja ta püstitas altareid baalidele, valmistas Aðera kujusid ja kummardas kõiki taevavägesid ning teenis neid.
\par 4 Tema ehitas altareid Issanda kotta, kuigi Issand oli öelnud: „Jeruusalemmas peab minu nimi olema igavesti!”
\par 5 Tema ehitas altareid kõigile taevavägedele Issanda koja kumbagi õue.
\par 6 Tema laskis oma pojad tulest läbi käia Ben-Hinnomi orus, toimetas lausumist ja kuulutas märkidest, nõidus ning seadis vaimudemanajaid ja ennustajaid; ta tegi palju kurja Issanda silmis ja vihastas teda.
\par 7 Tema paigutas Jumala kotta nikerdatud ebajumalakuju, mille ta oli valmistanud, kuigi Jumal oli öelnud Taavetile ja ta pojale Saalomonile: „Siia kotta ja Jeruusalemma, mille ma olen valinud kõigist Iisraeli suguharudest, panen ma oma nime igaveseks ajaks.
\par 8 Ma ei vii enam Iisraeli jalga ära sellelt maalt, mille ma olen määranud teie vanemaile, kui nad ainult panevad tähele ja teevad kõike, mida ma Moosese läbi neid olen käskinud, mis puutub kogu Seadusesse, määrustesse ja seadlustesse.”
\par 9 Aga Manasse ahvatles Juuda ja Jeruusalemma elanikke tegema rohkem kurja, kui tegid need paganad, kelle Issand oli hävitanud Iisraeli laste eest.
\par 10 Ja Issand rääkis Manassele ja tema rahvale, aga nemad ei pannud tähele.
\par 11 Siis tõi Issand nende kallale Assuri kuninga sõjaväepealikud ja need võtsid Manasse kinni peiduurkas, sidusid ta vaskahelatega ja viisid Paabelisse.
\par 12 Aga kui tal kitsas käes oli, siis ta püüdis leevendada Issanda, oma Jumala palet ja alandas ennast väga oma vanemate Jumala ees.
\par 13 Ja kui ta palvetas tema poole, siis andis Jumal temale järele, kuulis ta anumist ja tõi tema tagasi Jeruusalemma, ta oma kuningriiki. Siis mõistis Manasse, et Issand on Jumal.
\par 14 Ja seejärel ehitas ta Taaveti linna välimise müüri orgu Giihonist lääne pool, Kalavärava suunas ja ümber templikünka, tehes selle väga kõrgeks; ja ta paigutas väepealikud kõigisse Juuda kindlustatud linnadesse.
\par 15 Ta kõrvaldas võõrad jumalad ja kuju Issanda kojast, samuti kõik altarid, mis ta oli ehitanud Issanda koja mäele ja Jeruusalemma, ja viskas need linnast välja.
\par 16 Ja ta seadis korda Issanda altari ning ohverdas selle peal tänu- ja kiituseohvreid, ja andis Juudale käsu teenida Issandat, Iisraeli Jumalat.
\par 17 Ometi ohverdas rahvas veelgi ohvriküngastel, kuigi Issandale, oma Jumalale.
\par 18 Ja muud Manasse lood ja tema palve oma Jumala poole ning nende nägijate sõnad, kes temale rääkisid Issanda, Iisraeli Jumala nimel, vaata, need on Iisraeli Kuningate raamatus.
\par 19 Ja tema palve ning kuidas seda kuulda võeti, ja kõik tema patud ja üleastumised, ja paigad, kuhu ta ehitas ohvrikünkad, püstitas viljakustulbad ja nikerdatud kujud, enne kui ta ennast alandas, vaata, need on kirja pandud Hoosai lugudes.
\par 20 Siis Manasse läks magama oma vanemate juurde ja ta maeti oma kotta. Ja tema poeg Aamon sai tema asemel kuningaks.
\par 21 Aamon oli kuningaks saades kakskümmend kaks aastat vana ja ta valitses Jeruusalemmas kaks aastat.
\par 22 Tema tegi kurja Issanda silmis, nõnda nagu tema isa Manasse oli teinud. Aamon ohverdas kõigile nikerdatud kujudele, mis ta isa Manasse oli teinud, ja teenis neid.
\par 23 Aga ta ei alandanud ennast Issanda ees, nõnda nagu tema isa Manasse oli ennast alandanud, vaid tema, Aamon, suurendas süüd.
\par 24 Siis pidasid ta sulased vandenõu tema vastu ja nad surmasid ta tema oma kojas.
\par 25 Aga maa rahvas lõi maha kõik need, kes olid pidanud vandenõu kuningas Aamoni vastu, ja maa rahvas tõstis tema poja Joosija tema asemele kuningaks.

\chapter{34}

\par 1 Joosija oli kuningaks saades kaheksa aastat vana ja ta valitses Jeruusalemmas kolmkümmend üks aastat.
\par 2 Tema tegi, mis õige oli Issanda silmis, ja käis oma isa Taaveti teedel ega kaldunud paremale või vasakule.
\par 3 Oma valitsemise kaheksandal aastal, kuigi ta oli alles noor, hakkas ta otsima oma isa Taaveti Jumalat; ja kaheteistkümnendal aastal hakkas ta puhastama Juudat ja Jeruusalemma ohvriküngastest, viljakustulpadest ning nikerdatud ja valatud kujudest.
\par 4 Tema juuresolekul kisti maha baalide altarid; ta lõi katki suitsutusaltarid, mis olid ülal nende peal, purustas viljakustulbad ning nikerdatud ja valatud kujud, pihustas ja puistas need nende haudade peale, kes neile olid ohverdanud.
\par 5 Ja ta põletas preestrite luud nende altarite peal; nõnda ta puhastas Juuda ja Jeruusalemma.
\par 6 Ka Manasse, Efraimi ja Siimeoni linnades kuni Naftalini, kõikjal nende varemeil,
\par 7 kiskus ta maha altarid, lõi pihuks ja põrmuks viljakustulbad ja nikerdatud kujud ning raius katki suitsutusaltarid kogu Iisraelimaal. Siis ta läks tagasi Jeruusalemma.
\par 8 Ja oma valitsemise kaheksateistkümnendal aastal, kui ta maa ja koja oli puhastanud, läkitas ta Saafani, Asalja poja, ja linnapealiku Maaseja ja usaldusmehe Joahi, Jooahase poja, kohendama Issanda, oma Jumala koda.
\par 9 Ja need tulid ülempreester Hilkija juurde ja andsid üle raha, mis oli toodud Jumala kotta, mida leviidid, lävehoidjad, olid kogunud Manassest, Efraimist ja kogu ülejäänud Iisraelist ja kogu Juudast ja Benjaminist ja Jeruusalemma elanikelt.
\par 10 Ja nad andsid selle tööjuhatajate kätte, kelle hooleks oli Issanda koda, et need annaksid seda töötegijaile, kes töötasid Issanda kojas koja parandamiseks ja kohendamiseks.
\par 11 Ja need andsid seda meistritele ja ehitustöölistele, et nad ostaksid tahutud kive ja puitu aampalkideks ning nende kodade katmiseks, mis Juuda kuningad olid lasknud laguneda.
\par 12 Ja need mehed tegid ustavalt tööd. Neil olid ülevaatajaiks leviidid Jahat ja Obadja Merari poegadest, Sakarja ja Mesullam Kehati poegadest, et neid juhatada; ja leviidid, kõik, kes oskasid mänguriistu käsitseda,
\par 13 olid ülevaatajaiks kandjaile ja juhatasid kõiki töötegijaid mitmesugustel töödel. Ja leviitidest olid kirjutajad, ametnikud ja väravahoidjad.
\par 14 Ja kui nad võtsid välja raha, mis Issanda kotta oli toodud, siis leidis preester Hilkija Issanda Seaduse raamatu, mis Moosese läbi oli antud.
\par 15 Ja Hilkija võttis sõna ning ütles kirjutaja Saafanile: „Ma leidsin Issanda kojast Seaduse raamatu.” Ja Hilkija andis raamatu Saafanile.
\par 16 Ja Saafan viis raamatu kuningale ja tõi kuningale ka veel sõna, öeldes: „Su sulased on teinud kõik, mis nende kätte oli antud.
\par 17 Raha, mis leidus Issanda kojas, on toodud ja antud ülevaatajaile ning töötegijaile.”
\par 18 Ja kirjutaja Saafan teatas kuningale ning ütles: „Preester Hilkija andis mulle raamatu.” Ja Saafan luges sellest kuningale ette.
\par 19 Aga kui kuningas kuulis Seaduse sõnu, siis ta käristas oma riided lõhki.
\par 20 Ja kuningas andis käsu Hilkijale ja Ahikamile, Saafani pojale, ja Abdonile, Miika pojale, ja kirjutaja Saafanile ja kuninga sulasele Asajale, öeldes:
\par 21 „Minge küsitlege Issandat minu ning Iisraelist ja Juudast ülejäänute nimel leitud raamatu sõnade pärast, sest suur on Issanda viha, mis meie peale on valatud, sellepärast et meie vanemad ei ole tähele pannud Issanda sõna ega ole teinud kõike nõnda, nagu selles raamatus on kirjutatud!”
\par 22 Siis läksid Hilkija ja need, keda kuningas käskis, naisprohvet Hulda juurde, kes oli riietehoidja Sallumi, Tokhati poja, Hasra pojapoja naine ja elas Jeruusalemmas teises linnaosas, ja rääkisid temale, nagu see asi oli.
\par 23 Ja tema ütles neile: „Nõnda ütleb Issand, Iisraeli Jumal: Öelge sellele mehele, kes teid minu juurde läkitas:
\par 24 Nõnda ütleb Issand: Vaata, ma saadan sellele paigale ja selle elanikele õnnetuse, kõik need sajatused, mis on kirja pandud selles raamatus, mida Juuda kuningale ette loeti,
\par 25 sellepärast et nad jätsid minu maha ja suitsutasid teistele jumalatele, et mind vihastada oma igasugu kätetöödega, sellepärast valatakse ka minu tuline viha selle paiga peale ja see ei kustu mitte.
\par 26 Aga Juuda kuningale, kes teid läkitas Issandat küsitlema, öelge nõnda: Nõnda ütleb Issand, Iisraeli Jumal, sõnadest, mis sa oled kuulnud:
\par 27 Et su süda pehmenes ja sa alandasid ennast Jumala ees, kui sa kuulsid tema sõnu selle paiga ja selle elanike kohta, et sa alandasid ennast minu ees, käristasid oma riided lõhki ja nutsid minu ees, siis olen ka mina sind kuulnud, ütleb Issand.
\par 28 Vaata, ma koristan sind su vanemate juurde ja sind koristatakse rahus oma hauda ja sinu silmad ei saa näha kogu seda õnnetust, mille ma saadan sellele paigale ja selle elanikele.” Ja nad tõid kuningale sõna tagasi.
\par 29 Siis kuningas läkitas käsu ja kogus kokku kõik Juuda ja Jeruusalemma vanemad.
\par 30 Ja kuningas läks Issanda kotta, ka kõik Juuda mehed ja Jeruusalemma elanikud, preestrid ja leviidid ja kogu rahvas, niihästi suured kui väikesed, ja ta luges nende kuuldes Issanda kojast leitud Seaduse raamatu kõik sõnad.
\par 31 Ja kuningas astus oma kohale ja tegi Issanda ees lepingu, et nad käivad Issanda järel ja peavad tema käske, manitsusi ja määrusi kõigest südamest ja kõigest hingest, et täita selle Seaduse sõnu, mis sellesse raamatusse oli kirjutatud.
\par 32 Ja ta seadis selle lepingu alla kõik, kes leiti Jeruusalemmast ja Benjaminist. Ja Jeruusalemma elanikud täitsid Jumala, oma vanemate Jumala lepingut.
\par 33 Ja Joosija kõrvaldas kõik jäledused kõigist Iisraeli laste maakondadest ja pani kõik, kes elasid Iisraelis, teenima Issandat, nende Jumalat. Niikaua kui ta elas, ei taganenud nad Issandast, oma vanemate Jumalast.

\chapter{35}

\par 1 Ja Joosija pidas Jeruusalemmas paasapüha Issanda auks; esimese kuu neljateistkümnendal päeval tapeti paasatall.
\par 2 Tema seadis preestrid nende ametisse ja kinnitas nad Issanda koja teenistusse.
\par 3 Ja ta ütles leviitidele, kes õpetasid kogu Iisraeli ja olid pühitsetud Issandale: „Pange püha laegas kotta, mille Saalomon, Taaveti poeg, Iisraeli kuningas, on ehitanud; teil pole enam vaja seda õlgadel kanda. Teenige nüüd Issandat, oma Jumalat, ja tema rahvast Iisraeli!
\par 4 Ja korraldage endid oma perekondade kaupa rühmadena Taaveti, Iisraeli kuninga eeskirja ja tema poja Saalomoni määruse järgi
\par 5 ja astuge pühamusse - leviitide perekonnarühmad olgu vastavalt teie vendade, rahva poegade perekondade jaotusele -
\par 6 ja tapke paasatall, pühitsege endid ja valmistage see oma vendadele, tehes, nagu Issand Moosese läbi on käskinud!”
\par 7 Ja Joosija annetas rahva poegadele lambaid ja kitsi, lamba- ja kitsetallesid - kõik paasaohvreiks kõigile kohalviibijaile - arvult kolmkümmend tuhat, ja kolm tuhat veist; need olid kuninga omandist.
\par 8 Ka tema vürstid annetasid vabatahtlikult rahvale, preestritele ja leviitidele; Hilkija, Sakarja ja Jehiel, Jumala koja eestseisjad, andsid preestritele paasaohvreiks kaks tuhat kuussada talle ja kolmsada veist.
\par 9 Ja Konanja, ja Semaja ja Netaneel, tema vennad, ja Hasabja, Jeiel ja Joosabad, leviitide ülemad, annetasid leviitidele paasaohvreiks viis tuhat talle ja viissada veist.
\par 10 Nõnda valmistuti teenistuseks. Siis asusid preestrid oma kohtadele, nõndasamuti leviidid rühmade kaupa, nagu kuningas oli käskinud,
\par 11 ja nad tapsid paasatalle; preestrid piserdasid verd nende käest ja leviidid nülgisid.
\par 12 Aga nad panid kõrvale põletusohvrid, mis nad pidid andma rahva poegade perekondadele rühmade kaupa ohverdamiseks Issandale, nõnda nagu Moosese raamatus on kirjutatud; ja nõndasamuti oli lugu veistega.
\par 13 Ja nad küpsetasid paasatalle korra kohaselt tulel, pühitsetud annid aga keetsid nad pottides, padades ja pannidel, ja viisid need rutates kõigile rahva poegadele.
\par 14 Seejärel nad valmistasid iseendile ja preestritele, sest preestrid, Aaroni pojad, olid ööni ohverdamas põletusohvreid ja rasvu; seepärast valmistasid leviidid iseendile ja preestritele, Aaroni poegadele.
\par 15 Ja lauljad, Aasafi pojad, olid oma kohal Taaveti, Aasafi, Heemani ja kuninga nägija Jedutuuni käsu kohaselt; ja väravahoidjad olid igas väravas; neil ei olnud vaja oma teenistusest lahkuda, sest nende vennad leviidid valmistasid neile.
\par 16 Nõnda korraldus kogu Issanda teenistus sel päeval - paasapüha pidamine ja põletusohvrite ohverdamine Issanda altaril - kuningas Joosija käsu kohaselt.
\par 17 Ja kohal viibivad Iisraeli lapsed pidasid seekord paasapüha ja hapnemata leibade püha seitse päeva.
\par 18 Niisugust paasapüha ei olnud Iisraelis peetud prohvet Saamueli päevist peale; sest ükski Iisraeli kuningaist ei olnud pidanud niisugust paasapüha, nagu pidasid Joosija, preestrid ja leviidid, kogu kohal viibiv Juuda ja Iisrael ja Jeruusalemma elanikud.
\par 19 Joosija kaheksateistkümnendal valitsemisaastal peeti see paasapüha.
\par 20 Pärast kõike seda, kui Joosija oli koja korda seadnud, tuli Egiptuse kuningas Neko sõdima Karkemise vastu Frati jõe ääres; ja Joosija läks temale vastu.
\par 21 Aga see läkitas tema juurde saadikud ütlema: „Mis on mul sinuga tegemist, Juuda kuningas? Ei ole ma täna tulnud sinu vastu, vaid selle soo vastu, kes minuga sõdib. Jumal on mind käskinud rutata. Jäta rahule Jumal, kes on minuga, et ta sind ei hävitaks!”
\par 22 Aga Joosija ei hoidunud temast eemale, vaid muutis end tundmatuks, et temaga võidelda; ta ei kuulanud Neko sõnu, mis olid Jumala suust, vaid läks sõdima Megiddo orgu.
\par 23 Aga kütid ambusid kuningas Joosijat; ja kuningas ütles oma sulastele: „Viige mind ära, sest ma olen raskesti haavatud!”
\par 24 Siis ta sulased tõstsid ta sõjavankrist teise vankrisse, mis tal oli, ja viisid ta Jeruusalemma. Ta suri ja ta maeti oma vanemate hauda. Ja kogu Juuda ja Jeruusalemm leinasid Joosijat.
\par 25 Ja Jeremija tegi Joosija pärast nutulaulu; kõik lauljad ja lauljannad on oma nutulauludes rääkinud Joosijast kuni tänapäevani. See kujunes Iisraelis tavaks; ja vaata, need on Nutulauludes kirja pandud.
\par 26 Ja muud Joosija lood ja tema vagad teod, mis vastasid Issanda Seaduse eeskirjale,
\par 27 tema varasemad ja hilisemad lood, vaata, need on kirja pandud Iisraeli ja Juuda Kuningate raamatus.

\chapter{36}

\par 1 Ja maa rahvas võttis Jooahase, Joosija poja, ja tõstis tema kuningaks Jeruusalemmas tema isa asemele.
\par 2 Jooahas oli kuningaks saades kakskümmend kolm aastat vana ja ta valitses Jeruusalemmas kolm kuud.
\par 3 Aga Egiptuse kuningas kõrvaldas ta Jeruusalemmas võimult ja pani maale rahatrahvi peale: sada talenti hõbedat ja talent kulda.
\par 4 Ja Egiptuse kuningas tõstis tema venna Eljakimi Juuda ja Jeruusalemma kuningaks ning muutis tema nime Joojakimiks; aga Jooahase, tema venna, võttis Neko ja viis Egiptusesse.
\par 5 Joojakim oli kuningaks saades kakskümmend viis aastat vana ja ta valitses Jeruusalemmas üksteist aastat; tema tegi kurja Issanda, oma Jumala silmis.
\par 6 Nebukadnetsar, Paabeli kuningas, tuli tema vastu ja sidus ta vaskahelatega, et teda Paabelisse viia.
\par 7 Ka osa Issanda koja riistu viis Nebukadnetsar Paabelisse ja pani need Paabelis oma templisse.
\par 8 Ja muud Joojakimi lood ja tema põlastusväärsed teod, mis ta tegi ja mis veel tema vastu leiti, vaata, need on kirja pandud Iisraeli ja Juuda Kuningate raamatus. Ja tema poeg Joojakin sai tema asemel kuningaks.
\par 9 Joojakin oli kuningaks saades kaheksateist aastat vana ja ta valitses Jeruusalemmas kolm kuud ja kümme päeva; tema tegi kurja Issanda silmis.
\par 10 Ja aastavahetusel läkitas kuningas Nebukadnetsar talle järele ja laskis ta koos Issanda koja kalliste riistadega tuua Paabelisse, ja tõstis tema venna Sidkija Juuda ja Jeruusalemma kuningaks.
\par 11 Sidkija oli kuningaks saades kakskümmend üks aastat vana ja ta valitses Jeruusalemmas üksteist aastat.
\par 12 Tema tegi kurja Issanda, oma Jumala silmis; ta ei alandanud ennast prohvet Jeremija ees, kes rääkis Issanda käsul.
\par 13 Ta hakkas vastu ka kuningas Nebukadnetsarile, kes teda oli vannutanud Jumala juures; ja tema tegi oma kaela kangeks ja südame kõvaks ega pöördunud Issanda, Iisraeli Jumala poole.
\par 14 Nõndasamuti ka kõik Juuda vürstid, preestrid ja rahvas lisasid üha üleastumisi paganate igasugu jäleduste eeskujul ja rüvetasid Issanda koja, mille ta Jeruusalemmas oli pühitsenud.
\par 15 Ja Issand, nende vanemate Jumal, läkitas neile korduvalt sõna oma käskjalgade kaudu, sest ta tahtis säästa oma rahvast ja oma eluaset.
\par 16 Aga nad pilkasid Jumala käskjalgu, põlgasid tema sõnu ja tegid nalja tema prohvetitega, kuni Issanda viha tõusis oma rahva vastu, nõnda et enam ei olnud abi.
\par 17 Siis ta tõi nende kallale kaldealaste kuninga ja see tappis mõõgaga nende noored mehed nende pühas paigas ega andnud armu poisile ega tüdrukule, vanale ega raugale - Issand andis kõik tema kätte.
\par 18 Ja kõik Jumala koja riistad, suured ja väikesed, ja Issanda koja varandused ja kuninga ning tema vürstide varandused - need kõik viis ta Paabelisse.
\par 19 Ja nad põletasid Jumala koja, kiskusid maha Jeruusalemma müüri, põletasid tulega kõik selle paleed ja hävitasid kõik selle kallid asjad.
\par 20 Ja kes olid mõõgast üle jäänud, need viis ta Paabelisse vangi ja need said orjadeks temale ja ta poegadele kuni Pärsia kuningriigi valitsuse alguseni -
\par 21 et läheks täide Issanda sõna Jeremija suust -, kuni maa oli hüvitanud oma pidamata jäetud hingamispäevad: kõik laastatud oleku päevad olid hingamiseks, kuni seitsekümmend aastat sai täis.
\par 22 Aga Pärsia kuninga Koorese esimesel aastal - et läheks täide Issanda sõna Jeremija suust - äratas Issand Pärsia kuninga Koorese vaimu, nõnda et ta laskis kogu oma kuningriigis kuulutada ja ka kirjalikult öelda:
\par 23 „Nõnda ütleb Koores, Pärsia kuningas: Issand, taevaste Jumal, on andnud mulle kõik kuningriigid maa peal ja tema on mind käskinud ehitada temale koja Juudas olevas Jeruusalemmas. Kes iganes teie hulgas on tema rahvast, sellega olgu Issand, tema Jumal, ja see võib minna!”



\end{document}