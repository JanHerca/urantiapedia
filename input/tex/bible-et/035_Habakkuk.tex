\begin{document}

\title{Habakuk}

\chapter{1}

\par 1 Prohvet Habakuki nähtud ennustus:
\par 2 „Kui kaua, Jehoova, ma pean appi hüüdma, ilma et sa kuuleksid? Või sulle kisendama vägivalla pärast, ilma et sa aitaksid?
\par 3 Miks sa lased mind näha nurjatust ja vaatad ise seda õnnetust pealt? Hävitus ja vägivald on mu ees - on tulnud riid ja tõusnud tüli!
\par 4 Seepärast on käsuõpetus jõuetu ja õigus ei tule iialgi esile! Sest õel piirab õiget - nõnda tekib väänatud õigus!
\section*{Kaldealased karistajaiks}

\par 5 Vaadake paganate seas ja pange tähele, imestage ja hämmastuge! Sest ma teen teie päevil teo, mida te ei usuks, kui sellest jutustataks!
\par 6 Sest vaata, ma lasen tõusta kaldealased, kibeda ja tormaka rahva, kes käib maa laiuti läbi, et vallutada eluasemeid, mis pole tema omad!
\par 7 Ta on hirmus ja kardetav, temalt eneselt tuleneb ta õigus ja suurus!
\par 8 Tema hobused on pantritest kiiremad, metsikumad kui hundid õhtul! Tema ratsud trambivad - kaugelt tulevad ta ratsanikud, lendavad nagu sööma tõttav kotkas!
\par 9 Kõik nad tulevad teostama vägivalda; nende ees käib hirm ja nad koguvad vange nagu liiva!
\par 10 Ta pilkab kuningaid ja aukandjad on temale naerualuseks! Ta naerab igat kindlust, kuhjab kokku mulda ja vallutab selle!
\par 11 Siis ta tormab tuulena ja tõttab edasi, ta teeb oma rammu enesele jumalaks!
\par 12 Eks ole sina, Jehoova, muistsest ajast mu püha Jumal? Ei me sure! Jehoova! Sina oled teda pannud kohut mõistma, sina, kalju, määrasid teda nuhtlejaks!
\par 13 Sinu silmad on liiga puhtad selleks, et näha kurja, ja sina ei või vaadata õnnetust! Mispärast sa vaatad uskmatute peale, vaikid, kui õel neelab selle, kes on õigem temast,
\par 14 teed inimesed mere kalade sarnaseks, otsekui roomajaiks, kellel pole valitsejat?
\par 15 Need kõik ta tõmbab õngega üles, veab oma noota ja kogub võrku: seepärast ta rõõmutseb ja hõiskab!
\par 16 Seepärast ta ohverdab oma noodale ja suitsutab võrgule, sest nende abil on ta osa rasvane ja ta roog rammus!
\par 17 Kas ta võib alati nõnda tühjendada oma noota, rahvaid armuta tappes?


\chapter{2}

\par 1 Mina seisan oma vahipostil, asun linnusel ja vaatan, et näha, mida ta mulle räägib ja mida ta mulle vastab mu etteheite peale!
\par 2 Ja Jehoova vastas mulle ning ütles: kirjuta üles nägemus ja tähenda selgesti lauakeste peale, et see, kes seda loeb, võiks joosta!
\par 3 Sest nägemus ootab küll oma aega, aga ta ruttab lõpu poole ega peta mitte! Kui ta viibib, siis oota teda, sest ta tuleb kindlasti ega kõhkle mitte!
\par 4 Vaata, kes on ülbe, selle hing ei jää temasse, aga õige elab oma usust!
\par 5 Aga veel enam: viin on petlik - ei jõua sihile hoopleja mees, kes avab nagu haud oma kurgu ja on täitmatu otsekui surm, kes korjab enesele kõik paganad ja kogub enese juurde kõik rahvad!
\par 6 Eks nad kõik alusta tema kohta pilkelaulu ja osatavaid mõistukõnesid ning ütle: häda sellele, kes kuhjab kokku, mis ei ole tema oma - kui kauaks? - ja koormab ennast võlaga!
\par 7 Kas viimaks su võlausaldajad ei tõuse äkitselt ja su vintsutajad ei ärka? Jah, ja sina jääd neile saagiks!
\par 8 Sest sa oled riisunud paljusid rahvaid, nüüd riisuvad kõik ülejäänud rahvad sind, inimeste vere, samuti maale, linnale ja kõigile selle elanikele tehtud vägivalla pärast!
\par 9 Häda sellele, kes ahnitseb kurja kasu oma kojale, et paigutada oma pesa kõrgele, et päästa ennast kurja käest!
\par 10 Sa oled oma kavatsustega häbistanud oma koja, oled hävitanud paljusid rahvaid ja oled teinud pattu oma hinge vastu!
\par 11 Sest kivi müüris kisendab ja krohv seinal vastab temale!
\par 12 Häda sellele, kes ehitab linna verega ja rajab linnust ülekohtuga!
\par 13 Vaata, eks seegi ole vägede Jehoovalt, et rahvad näevad vaeva ainult tule tarvis ja rahvahõimud väsitavad endid ilmaaegu?
\par 14 Sest maa saab täis Jehoova au tundmist - otsekui veed katavad merepõhja!
\par 15 Häda sellele, kes joodab ligimest oma viha karikast ja teeb pealegi joobnuks, et saaks vaadata tema alastiolekut!
\par 16 Au asemel sa küllastud häbist: joo sinagi ja tuigu! Jehoova parema käe karikas jõuab ringiga sinule, ja su au peale häbi!
\par 17 Sest Liibanonile tehtud vägivald katab sind ja lojuste hävitus teeb sulle hirmu, inimeste vere, samuti maale, linnale ja kõigile selle elanikele tehtud vägivalla pärast!
\par 18 Mis kasu on nikerdatud kujust, et ta meister selle nikerdab, valatud kujust ja valeõpetajast, et kuju meister loodab selle peale, valmistades keeletuid ebajumalaid?
\par 19 Häda sellele, kes ütleb puule: „Ärka!” või liikumatule kivile: „Liigu!” Ons see õpetaja? Vaata, see on kulla ja hõbedaga karratud, ometi pole ta sees mingit vaimu!
\par 20 Aga Jehoova on oma pühas templis, tema palge ees vaikigu kogu maa!”


\chapter{3}

\par 1 Prohvet Habakuki palve kaebelaulude kujul:
\par 2 „Jehoova, ma olen kuulnud su sõnumit, ma olen näinud su tegu, Jehoova! Ligemail aastail lase see sündida, ligemail aastail ilmuta! Vihas mõtle halastusele!
\par 3 Jumal tuleb Teemanist, Püha Paarani mäelt! Sela! Tema aulikkus katab taevaid ja maa on täis tema kiitust!
\par 4 Tema all on nagu valguse sära, tema kõrval on kiired, seal on ta võimsuse loor!
\par 5 Tema ees käib katk ja tema kannul tuleb taud!
\par 6 Tema seisab ja mõõdab maad, tema vaatab ja paneb rahvad võpatama! Purunevad igavesed mäed, vajuvad ürgsed künkad, tema igavesed teed!
\par 7 Ma näen Kuusani telke vaeva all, Midjanimaa telgiriided värisevad!
\par 8 Kas oled vihastunud jõgede pärast, Jehoova? Ons sul viha jõgede või raev mere vastu, et sõidad oma hobuste seljas, oma võiduvankreis?
\par 9 Sina paljastad oma ammu täiesti, varustad oma vibunööri nooltega! Sina lõhestad jõgedega maa!
\par 10 Mäed näevad sind, on nagu sünnitusvaevas, pilved kallavad vett, põhjavesi teeb häält, tõstab kõrgele oma käed!
\par 11 Päike, kuu jäävad oma valitsuspaika: nad lähevad sinna su valguse noolte, su välkuvate piikide sära tõttu!
\par 12 Sajatades sa sammud maal, vihas peksad rahvaid nagu reht!
\par 13 Sa lähed aitama oma rahvast, abiks oma võitule! Sa purustad katuse õela kojalt, paljastad aluse kaljuni! Sela!
\par 14 Sa läbistad oma nooltega tema pea, aganate sarnaselt hajutatakse tema ratsanikud, kes oma ülbuses tulevad pillutama vaeseid, neelama viletsaid peidupaikades!
\par 15 Sa tallad tema hobused merre, suurte vete mudasse!
\par 16 Ma kuulen seda ja mu ihu väriseb, hääl paneb lõdisema mu huuled, mädanik tungib mu luudesse ja sammud mu all vanguvad! Ma halisen hädapäeva pärast, mis tõuseb rahvale, kes on tulnud meile kallale!
\par 17 Kuigi viigipuu ei õitse ja viinapuudel pole vilja õlipuu saak äpardub ja põllud ei anna toidust, pudulojused kaovad tarast ja veiseid pole karjaaedades,
\par 18 ometi rõõmutsen mina Jehoovas, hõiskan oma pääste Jumalas!
\par 19 Issand Jehoova on minu jõud! Tema teeb mu jalad emahirve jalgade sarnaseks ja paneb mind käima mu kõrgustikel!” Laulujuhatajale: minu keelpillisaade.




\end{document}