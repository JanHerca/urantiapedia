\begin{document}

\title{Toobit}

\chapter{1}

\section*{Jutustaja päritolu ja elukäik Niineves}

\par 1 Raamat Toobiti loost, kes oli Toobieli, kes oli Hananieli, kes oli Adueli, kes oli Gabaeli poeg, kes oli Asieli soost, kes oli Naftali suguharust,
\par 2 kes Assuri kuninga Salmanassari päevil oli vangi viidud Tisbest, mis on paremal pool Naftali Kedesit Galileas, ülalpool Haasorit.
\par 3 „Mina, Toobit, käisin tõe ja õiguse teed kogu oma eluaja. Ma tegin palju häid tegusid oma vendadele ja rahvale, kes koos minuga olid pidanud minema assüürlaste maale Niineve linna.
\par 4 Kui ma noorena olin alles oma kodus Iisraeli maal, siis oli kogu minu esiisa Naftali suguharu taganenud Jeruusalemma templist selles linnas, mis kõigist Iisraeli suguharudest oli valitud, et kõik suguharud seal ohverdaksid. Ja see tempel, Kõigekõrgema eluase, oli ehitatud ja pühitsetud kõigile sugupõlvedele igaveseks ajaks.
\par 5 Ja kõik suguharud, kes üheskoos olid taganenud, ohverdasid Baali vasikale, nõnda ka minu esiisa Naftali koda.
\par 6 Mina üksi käisin aga sageli Jeruusalemmas pidupäevil, nõnda nagu kogu Iisraelile on igavese seadusega ette kirjutatud, kaasas uudsevili, kümnis saagist ja esimene niiduvill,
\par 7 ja ma andsin need altari tarvis preestritele, Aaroni poegadele. Kümnise kogu saagist andsin ma Jeruusalemmas teenivaile Leevi poegadele. Teise kümnise ma müüsin ja läksin ning kulutasin selle igal aastal Jeruusalemmas.
\par 8 Ja kolmanda kümnise andsin ma neile, kellele see kuulus, nagu minu isaema Deboora oli käskinud, kuna olin jäänud isata orvuks.
\par 9 Kui ma olin meheks saanud, siis võtsin naiseks Hanna oma isa suguvõsast, ja tema sünnitas mulle Tobiase.
\par 10 Kui meid oli Niinevesse asumisele toodud, siis kõik minu vennad ja sugulased sõid paganate leiba.
\par 11 Mina aga hoidusin söömast,
\par 12 sest ma jäin kõigest hingest Jumalale truuks.
\par 13 Kõigekõrgem andis mulle armu ja armsust Salmanassari ees ja ma sain tema kaubamuretsejaks.
\par 14 Ma reisisin Meediasse ja andsin kümme talenti hõbedat hoiule Gabria vennale Gabaelile Meedia Ragaus.
\par 15 Kui Salmanassar suri, siis tema poeg Sanherib sai ta järglasena kuningaks. Tema ajal olid teed ebakindlad, nõnda et ma ei saanud enam Meediasse reisida.
\par 16 Salmanassari päevil andsin ma oma vendadele palju ande.
\par 17 Oma leivad andsin ma näljaseile ja riided alastiolijaile. Ja kui ma nägin kedagi oma rahvast olevat surnud ja heidetud Niineve müüride taha, siis ma matsin selle.
\par 18 Ja kui kuningas Sanherib, tulles põgenikuna Juudast, kellegi tappis, siis ma matsin selle salaja, sest oma vihas tappis ta paljusid. Kui kuningas siis nende kehi otsis, ei leitud neid.
\par 19 Aga üks niinevelane läks ja teatas kuningale, et mina neid matsin. Ma läksin siis peitu. Kui ma teada sain, et mind tahetakse tappa, siis ma kartsin ja põgenesin.
\par 20 Kogu mu vara riisuti ja mulle ei jäetud muud kui ainult mu naine Hanna ja poeg Tobias.
\par 21 Aga ei olnud veel möödunud viiskümmend päeva, kui kaks ta poega tapsid kuninga ja põgenesid Ararati mäestikku. Tema poeg Eesar-Haddon sai tema asemel kuningaks ja too seadis Ahhiahharose, minu venna Hanaeli poja, oma kuningriigi kogu arvepidamise ja kogu halduse üle.
\par 22 Ja Ahhiahharos palus minu eest ning ma tulin jälle Niinevesse. Ahhiahharos oli joogikallaja ja pitsatihoidja, valitseja ja arvepidaja. Eesar-Haddon oli tema pannud enesest järgmiseks. Tema oli aga minu vennapoeg.

\chapter{2}

\section*{Toobit jääb pimedaks}

\par 1 Aga kui ma jälle olin koju tulnud ning oma naise Hanna ja poja  Tobiase olin tagasi saanud, siis oli mul nelipühade pidupäeval, mis  on seitsme nädala püha, hea söömaaeg, ja ma istusin sööma.
\par 2 Nähes nüüd rohkeid roogasid, ütlesin ma oma pojale: „Mine, ja kui  sa leiad mõne meie vendadest, kes midagi vajab, kes Issanda peale mõtleb,  siis too ta siia! Vaata, ma ootan sind!”
\par 3 Ta tuli tagasi ning ütles: „Isa, keegi meie soost on  kägistatud ja heidetud turule.”
\par 4 Siis mina, enne kui ma midagi maitsesin, tõusin kiiresti ja  viisin surnukeha ühte kotta, seniks kui päike loojub.
\par 5 Ma tulin tagasi, pesin ennast ja sõin leinates oma leiba.
\par 6 Mulle meenus Aamose ennustus, kuidas ta oli ütelnud: „Teie pühad muutuvad leinaks ja kõik teie rõõmud nutulauludeks!”Ja ma nutsin.
\par 7 Kui päike oli loojunud, siis ma läksin, kaevasin haua ning  matsin ta maha.
\par 8 Aga naabrid naersid, üteldes: „Ta ei karda enam, et teda selle  teo pärast tapetakse! Ta pidi põgenema, aga vaata, ta matab jälle  surnuid!”
\par 9 Selsamal ööl tulin ma matmast tagasi, ja olles roojaseks saanud,  heitsin ma magama õue müüri äärde ja mu nägu oli katmata.
\par 10 Aga ma ei teadnud, et müüris olid varblased, ja kui mu silmad  lahti olid, siis need varblased sirtsutasid oma sooja väljaheidet mu  silmadesse. Minu silmadesse tekkisid siis valged laigud. Ma läksin  küll arstide juurde, aga nemad ei suutnud mind aidata. Ahhiahharos  toitis siis mind, kuni ta läks Elümaisi. Hanna noomib Toobitit
\par 11 Minu naine Hanna tegi tasu eest naiste käsitöid
\par 12 ja saatis need isandaile: nemad maksid temale palka ja andsid  lisaks ühe kitsetalle.
\par 13 Aga kui ta minu juurde tuli, siis hakkas tall mökitama. Ja ma  küsisin temalt: „Kust see kitsetall on? Ega ometi varastatud? Anna  see tagasi isandaile, sest varastatut ei ole lubatud süüa!”
\par 14 Tema vastas: „See on kingitus, antud mulle palgalisaks.” Aga mina  ei uskunud teda ja käskisin talle isandaile tagasi anda, ja ma  punastasin tema ees. Tema aga vastas ja ütles mulle: „Kus on sinu  almused ja heateod? Vaata, nüüd on sinust kõik teada!”

\chapter{3}

\section*{Toobiti patukahetsus}

\par 1 Siis ma jäin kurvaks, nutsin ja palvetasin ahastuses, üteldes:
\par 2 „Issand, sina oled õiglane ja kõik sinu teod ja teed on halastus  ning tõde. Sina mõistad alati tõelist ja õiget kohut!
\par 3 Mõtle minu peale ja vaata mind! Ära nuhtle mind minu pattude ja  eksimuste pärast, ka mitte selle pärast, millega minu vanemad on pattu  teinud sinu ees!
\par 4 Sest nemad ei võtnud kuulda sinu käske, ja sina andsid meid  rüüstata, vangi viia, surma, ja pilkeiks kõigile rahvaile, kelle  keskele me oleme pillutatud.
\par 5 Ja nüüd on õiglased sinu paljud nuhtlused, mida sa paned minu  peale minu ja minu vanemate pattude pärast, sest meie ei ole pidanud  sinu käske. Meie ei ole ju sinu ees elanud tões.
\par 6 Talita nüüd minuga, nagu sinu meelest hea on! Käsi ära võtta mu  hing, et võiksin lahkuda ja mullaks saada, sest minul on parem surra  kui elada! Olen ju kuulnud valelikke teotusi ja minu kurvastus on  suur. Käsi, et juba nüüd saaksin lahti oma vaevast, igavesse  asupaika! Ära pööra oma palet minult ära!” Saara
\par 7 Selsamal päeval sündis, et Meedia Ekbatanas teotati ka Saarat,  Ragueli tütart, tema isa teenijatüdrukute poolt.
\par 8 Sest tema oli olnud kihlatud seitsme mehega, aga kuri vaim  Asmodeus oli  need tapnud, enne kui nad olid maganud temaga kui naisega. Ja tüdrukud  ütlesid talle: „Oled sa meeletu, et tapad oma mehi? Sul on juba  olnud seitse meest, aga mitte ühegagi pole sul õnne olnud!
\par 9 Miks sa meid peksad? Kui nad on surnud, siis mine koos nendega!  Et me iialgi ei näeks sinu poega või tütart!”
\par 10 Kui ta seda kuulis, siis muutus ta väga kurvaks, nõnda et  tahtis ennast puua. Aga siis ta ütles: „Mina olen oma isa ainuke. Kui  ma seda teen, siis on see temale teotuseks ja ma saadan tema hallid  juuksed murega hauda.”
\par 11 Ja ta palvetas akna ees ning ütles: „Ole kiidetud, Issand, minu  Jumal! Sinu püha ja auline nimi olgu kiidetud igavesti! Kiitku sind  alati kõik sinu teod!
\par 12 Ja nüüd, Issand, olen ma oma silmad ja palge pööranud sinu  poole.
\par 13 Käsi mind maa pealt ära võtta, et ma enam ei kuuleks teotust!
\par 14 Sina tead, Issand, et ma olen puhas igast patust mehega
\par 15 ja et ma ei ole määrinud oma nime ega oma isa nime minu  vangipõlvemaal. Mina olen oma isa ainusündinu ja temal ei ole teist  last, kes oleks tema pärija, ega ole ka lähemat sugulast või selle  poega, et hoiaksin ennast temale naiseks. Olen juba seitse meest  kaotanud, milleks ma veel elan? Kui sa ei taha mind surmata,  siis käsi, et ma enam ei peaks kuulma teotamist!” Jumal kuuleb palveid
\par 16 Ja mõlema palve võeti kuulda suure Raafaeli auhiilguse  juuresolekul;
\par 17 tema läkitati ravima neid mõlemaid: võtma Toobitilt valged laigud  ja andma Ragueli tütar Saara naiseks Toobiti pojale Tobiasele ja  siduma kurja vaimu Asmodeusi, sest Tobiasel oli õigus Saarat pärida.  Selsamal ajal tuli Toobit tagasi ja läks oma kotta, ja Saara, Ragueli  tütar, astus alla ülemisest kambrist.

\chapter{4}

\section*{Toobit õpetab Tobiast}

\par 1 Selsamal päeval meenus Toobitile raha, mille ta oli andnud  Gabaelile Meedia Ragaus.
\par 2 Ja ta mõtles endamisi: „Ma olen soovinud surra. Miks ma ei kutsu  oma poega Tobiast, et temale see teatavaks teha, enne kui ma suren?”
\par 3 Ta kutsus tema ning ütles: „Poeg! Kui ma suren, siis mata mind, ja  kanna hoolt oma ema eest! Austa teda kogu oma eluajal ja tee, mis on  temale meelepärane! Ära teda kurvasta!
\par 4 Mõtle, poeg, et temal on olnud palju hädaohte sind oma üsas  kandes! Kui ta sureb, siis mata ta minu kõrvale samasse  hauda!
\par 5 Iga päev, mu poeg, mõtle Issandale, meie Jumalale, ja ära taha  pattu teha ega tema käskudest üle astuda! Tee õigust kõigil oma  elupäevil ja ära käi ülekohtu teedel!
\par 6 Sest kui sa tõde taotled, siis lähevad su teod korda.
\par 7 Kõigile, kes õigust teevad, anna oma varandusest almuseid, ja su  silm ärgu olgu kade, kui sa almuseid annad! Ära pööra oma palet mitte  üheltki vaeselt, siis ka Jumala pale ei pöördu sinult!
\par 8 Kui sul palju on, siis anna sellest rohkesti almuseid! Kui sul  vähe on, siis ära karda anda almuseid sellest vähesestki!
\par 9 Nõnda sa kogud enesele hea varanduse hädapäevaks.
\par 10 Sest almus päästab surmast ega lase minna pimedusse.
\par 11 Almus on hea ohvriand Kõigekõrgema ees kõigile, kes selle  annavad.
\par 12 Hoia ennast, poeg, igasugusest hooraelust! Ja ennekõike: võta  naine oma isade soost! Ära võta võõrast naist, kes ei ole sinu isa  suguharust, sest meie oleme prohvetite pojad! Noa, Aabraham, Iisak,  Jaakob, meie isad muistsest ajast - meenuta, poeg - kõik võtsid naisi  oma sugulaste hulgast, neid õnnistati nende laste läbi ja nende sugu  pärib maa.
\par 13 Ja nüüd, poeg, armasta oma vendi ja ära pea südames ennast  suuremaks oma vendadest ning oma rahva poegadest ja tütardest, et sa  enesele ei võta naist nende hulgast! Sest kõrkuses on hukatus ja palju  segadust, ja kõlvatus elus on viletsus ning suur vaesus, sest kõlvatu  elu on nälja ema.
\par 14 Ära viivita palgamaksmisega ühelegi inimesele, kes sulle tööd teeb,  vaid anna see temale otsekohe! Kui sa nõnda Jumalat teenid, siis  tasutakse sulle. Poeg, valitse iseennast kõigis oma tegudes ja käitu  korralikult kõigis eluviisides!
\par 15 Mida sa ise vihkad, seda ära tee kellelegi! Ära joo veini,  et jääda joobnuks, ja joomapahe ärgu olgu sulle saatjaks sinu teel!
\par 16 Oma leivast anna näljasele, ja oma riideist alastiolijaile!  Kõigest, mida sul on küllalt, anna almuseid ja su silm ärgu olgu  kade, kui sa almuseid annad!
\par 17 Pane rohkesti leiba õigete hauale, aga ära anna patustele!
\par 18 Otsi nõu igalt mõistlikult ja ära põlga ühtki kasulikku nõuannet!
\par 19 Ja kiida alati Issandat Jumalat ning palu temalt, et su teed  oleksid õiged ja et kõik su teerajad ning kavatsused läheksid korda!  Sest mitte ühelgi rahval ei ole õiget nõu, vaid Issand ise annab kõike  head! Tema alandab, keda tahab, nõnda nagu temale meeldib. Ja nüüd,  poeg, pea meeles minu käsud, ärgu need ununegu sinu südames!
\par 20 Nüüd räägin ma sulle sellest kümnest talendist hõbedast,  mille ma andsin hoiule Gabria vennale Gabaelile Meedia Ragaus.
\par 21 Ära karda, poeg, et oleme vaeseks jäänud! Sinule on küllalt, kui  sa kardad Jumalat ja hoidud igast patust ning teed, mis on temale  meelepärane.”

\chapter{5}

\section*{Tobias läheb teele koos Raafaeliga}

\par 1 Tobias vastas temale, üteldes: „Isa, ma teen kõik, nõnda nagu  sa mind oled käskinud!
\par 2 Aga kuidas ma raha kätte saan, mina ju teda ei tunne?”
\par 3 Siis ta andis temale võlakirja ja ütles: „Otsi enesele inimene,  kes reisiks koos sinuga, ja mina maksan temale palka, kuni ma elan.  Mine ja too raha ära!”
\par 4 Ta läks nüüd otsima üht niisugust inimest ja leidis Raafaeli, kes  oli ingel, mida tema aga ei teadnud.
\par 5 Tobias küsis temalt: „Kas ma võin koos sinuga reisida Meedia  Ragausse, ja kas sina neid paiku tunned?”
\par 6 Ja ingel vastas temale: „Ma reisin koos sinuga; ma tunnen teed ja  olen ööbinud meie venna Gabaeli juures.”
\par 7 Tobias ütles siis temale: „Oota mind, ma räägin seda oma  isale!”
\par 8 Ta vastas: „Mine, aga ära jää kauaks!”
\par 9 Ja ta läks ning ütles isale: „Vaata, ma leidsin ühe, kes minuga  kaasa tuleb.” Isa aga ütles: „Kutsu ta minu juurde, et saaksin teada,  missugusest suguharust ta on ja kas ta on sulle usaldusväärne  teekaaslane!”
\par 10 Siis ta kutsus tema, ja tema astus sisse ning nad teretasid  teineteist.
\par 11 Toobit küsis temalt: „Vend, missugusest suguharust ja  missugusest perekonnast sa oled? Ütle mulle seda!”
\par 12 Ja ta vastas temale: „Kas sa otsid suguharu ja perekonda või  palgalist, kes koos sinu pojaga teele läheks?” Ja Toobit kostis:  „Vend, ma tahan teada saada sinu päritolu ja sinu nime!”
\par 13 Ta vastas: „Mina olen Asarja, suure Ananja poeg sinu vendade  hulgast.”
\par 14 Siis Toobit ütles temale: „Ole terve tulemast, vend! Ära ole  minu peale pahane, et uurisin sinu suguharu ja tahtsin teada saada  sinu perekonnast! Sina juhtudki olema minu vend, heast ja kuulsast  soost. Sest mina õppisin tundma Ananjat ja Joonatani, suure Simei  poegi, kui me koos läksime Jeruusalemma palvetama, viies  esmasündinuid ja kümniseid saagist. Nemad ei olnud langenud meie  vendade eksimusse. Sa oled heast juurest, vend!
\par 15 Aga ütle mulle, kui suure tasu pean sulle andma? Drahm päevas,  ja mis muidu veel sinule ja minu pojale tarvis on?
\par 16 Ma annan su palgale lisa, kui tervisega tagasi tulete!”
\par 17 Nõnda nad leppisidki kokku. Ja isa ütles Tobiasele:  „Valmistu  teekonnaks ja mingu see korda!” Ja tema poeg valmistus teekonnaks.  Isa ütles temale: „Mine koos selle inimesega! Aga Jumal, kes taevas  elab, andku teile head teed ja tema ingel mingu koos teiega!” Siis  mõlemad läksid välja, et teele minna, ja noormehe koer läks nendega  kaasa.
\par 18 Aga tema ema Hanna nuttis ja ütles Toobitile: „Miks sa meie poja  ära saatsid? Eks ta ole ju otsekui meie tugi, kui ta meie nähes  tuleb ja läheb?
\par 19 Raha ärgu tulgu rahale lisaks, vaid see jäägu meie poja  lunarahaks!
\par 20 Sest mis Issand on meile elamiseks andnud, sellest meile  jätkub.”
\par 21 Aga Toobit ütles temale: „Ära muretse, õde! Ta tuleb tervisega  tagasi ja sinu silmad saavad teda näha.
\par 22 Sest hea ingel saadab teda ning tema teekond läheb korda ja ta  tuleb tervisega tagasi.”
\par 23 Siis ta enam ei nutnud.

\chapter{6}

\section*{Teekond Meediasse}

\par 1 Aga nemad jõudsid oma teed käies õhtuks Tigrise jõe äärde ja  ööbisid seal.
\par 2 Kui noormees astus jõkke suplema, siis üks kala kargas veest  välja ja tahtis noormehe ära neelata.
\par 3 Aga ingel ütles talle: „Püüa see kala kinni!” Siis noormees  haaras kala ja viskas selle kaldale.
\par 4 Ja ingel ütles talle: „Lõika kala lõhki, võta süda, maks ja  sapp ja hoia neid hoolsasti!”
\par 5 Ja noormees tegi nõnda, nagu ingel teda käskis. Aga kala nad  küpsetasid ja sõid.
\par 6 Siis läksid mõlemad edasi, kuni nad jõudsid Ekbatana lähedale.
\par 7 Ja noormees ütles inglile: „Asarja, mu vend, milleks need on:  kala maks, süda ja sapp?”
\par 8 Ingel vastas temale: „Süda ja maks on selleks, et kui kedagi  vaevab deemon ehk kuri vaim, siis tuleb nendega suitsutada mehe või  naise ees, ja teda ei vaevata enam.
\par 9 Sapp on aga inimese võidmiseks, kellel on valged laigud silmades.  Siis ta paraneb.”
\par 10 Ja kui nad jõudsid Ekbatanasse,
\par 11 siis ingel ütles noormehele: „Vend, täna ööbime Ragueli juures,  tema on sinu sugulane ja temal on üksainus tütar, Saara nimi.
\par 12 Ma räägin tema pärast, et ta antaks sulle naiseks, sest tema  pärand langeb sinule. Sina oled ju ainus tema suguvõsast. Ja tüdruk on  ka ilus ning tark.
\par 13 Kuule aga nüüd mind, ma räägin tema isaga, ja kui me Ragaust  tagasi läheme, siis teeme pulmad. Sest ma tean, et Moosese seaduse  järgi ei saa Raguel teda anda mitte ühelegi teisele mehele surma ära  teenimata. Sest sinul on õigus pärand saada enne kõiki teisi mehi.”
\par 14 Noormees ütles siis inglile: „Asarja, mu vend! Ma olen kuulnud,  et seda tüdrukut on antud seitsmele mehele, ja need kõik on hukkunud  pruudikambris.
\par 15 Mina olen nüüd oma isa ainus poeg ja ma kardan, et kui ma sisse  lähen, siis ma suren nagu need eelmisedki, sest teda armastab kuri  vaim, kes ei tee kurja teistele kui ainult neile, kes temale  lähenevad.  Nüüd ma kardan, et minagi suren ja saadan oma isa ja ema elu hauda  valuga minu pärast. Neil ei ole teist poega, kes nad mataks.”
\par 16 Aga ingel ütles temale: „Kas sa ei mäleta neid sõnu, mis isa  sulle ütles, et sa võtaksid enesele naise oma suguvõsast? Nüüd  kuule mind, vend: tema peab saama sinu naiseks! Ja üldse ära hooli  kurjast vaimust, sest täna öösel antakse Saara sulle naiseks!
\par 17 Kui sa lähed pruudikambrisse, siis võta suitsutusohvri tuhka ja  pane selle peale pisut kala südant ja maksa ning suitsuta! Kui kuri  vaim seda nuusutab, siis ta põgeneb ega tule iialgi enam tagasi.
\par 18 Kui sa Saara juures oled, siis tõuske mõlemad ja hüüdke appi  armulikku Jumalat! Tema päästab teid ja halastab! Ära karda, sest  Saara on algusest peale sulle määratud ja sina pead ta päästma! Tema  tuleb koos sinuga ja ma usun, et sa saad temalt lapsi.”
\par 19 Kui Tobias seda kuulis, siis ta hakkas Saarat armastama ja ta  hing kiindus temasse üliväga.

\chapter{7}

\section*{Tobias Ragueli kojas}

\par 1 Ja nad jõudsid Ekbatanasse ning tulid Ragueli kotta. Saara tuli  neile vastu ja teretas neid ja nemad omakorda teda. Siis ta viis nad  kotta sisse.
\par 2 Raguel ütles oma naisele Ednale: „Kui sarnane on see noormees  minu nõo Toobitiga!”
\par 3 Ja Raguel küsis neilt: „Kust te olete, vennad?” Ja nad kostsid  temale: „Naftali poegadest, sõjavangidest Niineves.”
\par 4 Siis ta küsis neilt: „Kas tunnete meie venda Toobitit?” Ja nad  kostsid: „Tunneme küll.”
\par 5 Ta küsis neilt veel: „Kas ta on terve?” Nad vastasid: „Ta elab ja  on terve.” Ja Tobias ütles: „Tema on minu isa.”
\par 6 Siis Raguel hüppas üles, suudles teda, nuttis ja õnnistas teda  ning ütles temale: „Sina oled ausa ja hea mehe poeg.” Aga kuuldes, et  Toobit oli kaotanud nägemise, jäi ta kurvaks ja nuttis.
\par 7 Ka Edna, tema naine, ja Saara, tema tütar, nutsid. Ja nad võtsid  neid sõbralikult vastu.
\par 8 Nad tapsid karjast ühe jäära ja panid rikkalikult rooga nende ette.
\par 9 Aga Tobias ütles Raafaelile: „Asarja, mu vend, räägi nüüd  sellest, millest sa tee peal kõnelesid, et kavatsus täide läheks!”
\par 10 Ingel jutustas siis loo Raguelile. Ja Raguel ütles Tobiasele:  „Söö, joo ja ole rõõmus! Sest sina võid võtta minu tütre. Ometi pean  sulle avaldama tõe:
\par 11 ma olen andnud oma tütre seitsmele mehele, ja igaüks neist suri,  kui läks öösel tema juurde. Nüüd aga ole rõõmus!”
\par 12 Aga Tobias ütles: „Mina ei maitse siin midagi, enne kui olete  tema ette toonud ja teie ise seisate minu ees.” Raguel ütles siis:  „Võta tema nüüd seaduse kohaselt! Sina ju oled tema sugulane ja tema  on sinu! Armuline Jumal andku teile kõike head!”
\par 13 Ta kutsus oma tütre Saara, võttis tema käest kinni ja andis  tema Tobiasele naiseks, üteldes: „Vaata, Moosese seaduse järgi võta  tema ja vii ta oma isa juurde!” Ja ta õnnistas neid.
\par 14 Ta kutsus nüüd oma naise Edna, ja võtnud paberilehe, kirjutas ta  selle peale lepingu ning pitseeris selle. Siis nad hakkasid sööma.
\par 15 Pärast seda kutsus Raguel oma naise Edna ja ütles temale: „Õde,  sea korda teine kamber ja vii Saara sinna!”
\par 16 Naine tegi, nagu kästi, ja viis Saara sinna. Saara hakkas  siis nutma. Tema aga kuivatas oma tütre pisaraid ja ütles talle:
\par 17 „Ole julge, mu laps! Taeva ja maa Issand andku sulle rõõmu  kurvastuse asemel! Ole julge, tütar!”

\chapter{8}

\section*{Tobias ei sure pruudikambris}

\par 1 Kui nad söömise olid lõpetanud, siis nad viisid Tobiase Saara  juurde.
\par 2 Aga sisse astudes meenusid temale Raafaeli sõnad ja ta võttis  suitsutamiseks tuhka, pani selle peale kala südame ja maksa ning  suitsutas.
\par 3 Kui kuri vaim seda lõhna haistis, siis ta põgenes  Ülem-Egiptusesse ja ingel sidus ta kinni.
\par 4 Kui uks oli suletud mõlema tagant, siis Tobias tõusis voodist  ja ütles: „Tõuse üles, õde, ja palvetagem, et Issand halastaks meie  peale!”
\par 5 Ja Tobias alustas, üteldes: „Ole kiidetud, meie vanemate Jumal,  ja olgu igavesti kiidetud sinu püha ja auline nimi! Kiitku sind taevad  ja kõik sinu looming!
\par 6 Sina lõid Aadama ja andsid temale Eeva, tema naise, abiks ning  toeks. Neist on inimsugu sündinud. Sina ütlesid: „Inimesel ei ole hea  üksi olla, me teeme temale abi, kes on temaga sarnane.”
\par 7 Ja nüüd, Issand, mina ei võta seda oma õde mitte hooruse pärast,  vaid tõemeelselt. Lase mind armu leida ja koos temaga vanaks saada!”
\par 8 Saara ütles koos temaga: „Aamen.”
\par 9 Ja nad mõlemad magasid sellel ööl.
\par 10 Aga Raguel tõusis üles, läks ja kaevas haua, üteldes: „Küllap ka  tema sureb!”
\par 11 Kui Raguel oli koju tagasi tulnud,
\par 12 siis ta ütles oma naisele Ednale: „Läkita üks tüdrukuist  vaatama, kas ta elab! Kui mitte, siis matame ta ja keegi ei saa  sellest midagi teada.”
\par 13 Ja tüdruk läks, avas ukse ning leidis mõlemad magamas.
\par 14 Välja tulles kuulutas ta neile, et ta elab.
\par 15 Siis Raguel kiitis Jumalat, üteldes: „Ole kiidetud, Jumal,  kõige puhta ja püha kiitusega! Kiitku sind sinu pühad ja kogu sinu  looming! Kõik sinu inglid ja valitud kiitku sind ikka ja igavesti!
\par 16 Ole kiidetud, et oled mind rõõmustanud, et minule ei sündinud  nõnda, nagu ma kartsin, vaid et sa talitasid meiega oma rohket halastust  mööda!
\par 17 Ole kiidetud, et halastasid kahe üksiklapse peale! Osuta neile,  Issand, heldust ja lase neid elada tervises, rõõmu ja armuga!”
\par 18 Siis ta käskis kodakondseil haua kinni ajada.
\par 19 Ja ta korraldas neile pulmad, mis kestsid neliteist  päeva.
\par 20 Ja Raguel vannutas Tobiast, enne kui pulmapäevad olid lõppenud,  et ta ei läheks ära varem, kui need neliteist pulmapäeva on mööda  läinud.
\par 21 Alles siis võis ta võtta poole tema varandusest ja tervisega  minna oma isa juurde! „Ja ülejäänud osa saad siis, kui mina ja minu naine  oleme surnud.”

\chapter{9}

\section*{Raafael läheb raha tooma}

\par 1 Ja Tobias kutsus Raafaeli ja ütles temale:
\par 2 „Asarja, mu vend, võta kaasa sulane ja kaks kaamelit ja mine  Meedia Ragausse Gabaeli juurde, too mulle raha ja too ka tema ise  pulma,
\par 3 sest Raguel on vannutanud, et ma ei läheks ära.
\par 4 Aga mu isa loeb päevi ja kui ma liialt viivitan, siis ta kurvastab  väga.”
\par 5 Ja Raafael läks teele ning peatus Gabaeli juures ja andis temale  võlakirja. Gabael tõi siis välja pitseeritud kukrud ja andis need  temale.
\par 6 Ja varahommikul läksid nad üheskoos teele ning tulid pulma. Ja  Tobias ülistas oma naist.

\chapter{10}

\section*{Kodus oodatakse Tobiast}

\par 1 Aga tema isa Toobit luges iga päeva. Kui teekonna päevad täis  said ja tema ei tulnud,
\par 2 siis ütles Toobit: „Vahest peetakse neid kinni? Või on Gabael  surnud ja keegi ei anna temale raha?”
\par 3 Ta oli väga mures.
\par 4 Aga naine ütles talle: „Poiss on hukkunud, sellepärast ta jääb  nii kauaks.” Ja ta hakkas nutma tema pärast ning ütles:
\par 5 „Ei, mulle teeb muret, mu laps, et lasksin sind minna, mu  silmavalgus!”
\par 6 Toobit aga ütles talle: „Ole vait, ära muretse, ta on terve!”
\par 7 Naine vastas talle: „Ära räägi, ära peta mind, minu laps on  hukkunud!” Ja ta läks iga päev välja tee peale, mida mööda poeg oli  ära läinud. Päeval ta ei söönud, öösiti aga kaebles lakkamatult oma  poja Tobiase pärast, kuni olid möödunud need neliteist  pulmapäeva, milleks Raguel oli Tobiast vannutanud sinna jääma. Lahkumine Ragueli juurest
\par 8 Aga Tobias ütles Raguelile: „Lase mind minna, sest mu isa ja ema  ei loodagi enam mind näha saada!”
\par 9 Tema äi aga ütles talle: „Jää minu juurde, ma läkitan su isale  sõnumi ja teatan talle, kuidas su käsi käib!” Ja Tobias vastas: „Ei,  vaid läkita mind mu isa juurde!”
\par 10 Siis Raguel tõusis üles ja andis temale ta naise Saara ning  poole varandusest: orjad, loomad ja raha.
\par 11 Ta õnnistas neid ning saatis nad teele, üteldes: „Taeva Jumal  andku teile õnne, mu lapsed, enne kui ma suren!”
\par 12 Ja oma tütrele ta ütles: „Austa oma mehe vanemaid, nemad on nüüd  sinu vanemad! Soovin sinust head kuulda!” Ja ta suudles teda.
\par 13 Edna ütles Tobiasele: „Armas vend, taeva Issand viigu sind  koju ja andku mulle näha sinu lapsi minu tütrelt Saaralt, et võiksin  rõõmus olla Issanda ees! Ja vaata, ma usaldan sulle pandiks oma tütre,  ära kurvasta teda!”
\par 14 Pärast seda läks Tobias teele, kiites Jumalat, kes oli lasknud  tema teekonna korda minna. Ja ta õnnistas Ragueli ja tema naist Ednat.

\chapter{11}

\section*{Tobias jõuab koju}

\par 1 Siis ta läks teele ja olles jõudnud Niineve lähedale, ütles Raafael  Tobiasele:
\par 2 „Kas sa ei mäleta, mu vend, missugusesse olukorda sa jätsid oma  isa?
\par 3 Ruttame ette sinu naisest ja seame maja korda!
\par 4 Ja võta kala sapp kaasa!” Nad läksid siis edasi ja koer käis  nende kannul.
\par 5 Hanna aga istus, silmitsedes oma poja tuleku teed.
\par 6 Kui ta märkas teda tulevat, siis ta ütles tema isale: „Vaata,  sinu poeg tuleb, ja mees, kes läks koos temaga!”
\par 7 Ja Raafael ütles: „Mina tean, et su isa teeb silmad lahti!
\par 8 Võia siis sapiga tema silmi, ja kui kipitab, siis ta hõõrub neid,  valged laigud kaovad ja ta näeb sind!”
\par 9 Hanna jooksis tulijaile vastu, kaelustas oma poega ja ütles talle: „Mina  olen sind näinud, mu laps, nüüd ma võin surra!” Ja nad mõlemad  nutsid.

\section*{Toobit näeb jälle}

\par 10 Toobit läks siis ukse juurde ja komistas. Aga poeg jooksis temale  vastu
\par 11 ja haaras oma isa käest kinni ning tilgutas sappi isa  silmadesse, üteldes: „Ole rahulik, isa!”
\par 12 Aga kui silmad kipitasid, siis Toobit hõõrus neid ja tema silma  nurkadest eraldusid valged laigud.
\par 13 Nähes nüüd oma poega, kaelustas ta teda, nuttis ja ütles:
\par 14 „Ole kiidetud, Jumal, olgu igavesti kiidetud sinu nimi ja olgu  kiidetud kõik sinu pühad inglid! Sest sina oled mind karistanud, aga  oled minu peale ka halastanud. Vaata, ma näen oma poega Tobiast!”
\par 15 Tema poeg läks siis rõõmsana sisse ja jutustas isale neist  suurist asjust, mis Meedias olid temale sündinud.
\par 16 Ja Toobit läks välja Niineve väravasse oma miniale vastu,  rõõmustades ja Jumalat kiites. Ja kes nägid teda käimas, imestasid, et  ta jälle näeb. Toobit tunnistas nende ees, et Jumal oli tema peale  halastanud.
\par 17 Ja kui Toobit jõudis oma minia Saara juurde, siis ta õnnistas  teda, üteldes: „Ole terve tulemast, tütar! Kiidetud olgu Jumal, kes  sinu on saatnud meie juurde, nõndasamuti su isa ja ema!”
\par 18 Ja kõigil tema vendadel Niineves oli suur rõõm.
\par 19 Saabusid ka tema vennapoeg Ahhiahharos ja Naadab. Ja Tobiase  pulmi peeti seitse päeva rõõmsasti.

\chapter{12}

\section*{Raafael ilmutab ennast}

\par 1 Ja Toobit kutsus oma poja Tobiase ning ütles temale: „Muretse,  poeg, tasu mehele, kes koos sinuga käis, ja temale tuleb ka lisa  anda!”
\par 2 Tobias vastas temale: „Isa, ma ei kanna kahju, kui annan temale  poole sellest, mida olen kaasa toonud.
\par 3 Sest tema on mind tervisega sinu juurde toonud, on terveks teinud  minu naise, on toonud minule raha ja on ka sinu terveks teinud.”
\par 4 Siis ütles Toobit: „Temal on õigus seda saada!”
\par 5 Ta kutsus ingli ning ütles temale: „Võta pool kõigest, mida  olete kaasa toonud!”
\par 6 Aga ingel kutsus nad kahekesi kõrvale ja ütles neile: „Kiitke  Jumalat ja tänage teda, andke temale suurt au ja tänage teda kõigi  elavate ees selle eest, mis ta teile on teinud! Hea on kiita Jumalat  ja kõrgeks tõsta tema nime, kui te austavalt jutustate Jumala  tegudest! Ärge kõhelge teda tänamast!
\par 7 Ilus on pidada salajas kuninga saladust, Jumala tegude avaldamine  on aga auline. Tehke head, siis ei taba teid kuri!
\par 8 Hea on palve koos paastumise, almuste ja õiglusega. Parem pisut  õiglusega kui palju ülekohtuga! Parem on anda almuseid kui koguda  kulda!
\par 9 Sest almus päästab surmast ja peseb ära iga patu. Kes annavad  almuseid ja on õiglased, need elavad kaua,
\par 10 aga kes pattu teevad, need on omaenese elu vaenlased.
\par 11 Mina ei taha teie eest midagi varjata. Ma olen ju ütelnud, et  ilus on pidada kuninga saladust, Jumala tegude avaldamine on aga  auline.
\par 12 Ja nüüd, siis kui sina ja sinu minia Saara palvetasite,  viisin mina teie palve meeldetuletuse Püha ette. Kui sa surnuid matsid,  siis olin mina nõndasamuti sinu juures.
\par 13 Kui sa ei kõhelnud tõusmast ja jätsid oma söömaaja, et minna  surnut matma, siis heategu ei jäänud mulle saladuseks, vaid ma olin  koos sinuga.
\par 14 Ja nüüd on Jumal mind läkitanud terveks tegema sind ja sinu  miniat Saarat.
\par 15 Mina olen Raafael, üks neist seitsmest pühast inglist, kes  kannavad ette vagade palved ja käivad Püha auhiilguse ees.”
\par 16 Siis mõlemad ehmusid ja langesid silmili maha, sest nad  kartsid.
\par 17 Aga ta ütles neile: „Ärge kartke, rahu olgu teile! Kiitke  aga Jumalat igavesti!
\par 18 Sest mina ei ole tulnud omaenese armu pärast, vaid meie Jumala  tahtel. Seepärast kiitke teda igavesti!
\par 19 Kõik need päevad olen ma olnud teile nähtav, aga ma ei ole söönud  ega joonud, sest teie olete näinud nägemust.
\par 20 Nüüd tänage Jumalat, sest ma lähen üles selle juurde, kes mind  on läkitanud! Ja kirjutage raamatusse kõik, mis on sündinud!”
\par 21 Kui nad üles tõusid, siis nad ei näinud teda enam.
\par 22 Ja nad ülistasid Jumala suuri ja imelisi tegusid, mida Issanda  ingel oli neile ilmutanud.

\chapter{13}

\section*{Toobiti kiituslaul}

\par 1 Ja Toobit kirjutas üles palve, rõõmulaulu, ja ütles:
\par 2 „Kiidetud olgu Jumal, kes elab igavesti, ja tema kuningriik, sest tema karistab ja halastab, viib alla surmavalda ja toob jälle üles, ja ükski ei saa põgeneda tema käest!
\par 3 Ülistage teda, Iisraeli lapsed, paganate ees, sest tema on meid pillutanud nende keskele!
\par 4 Kuulutage seal tema suurust, ülendage teda kõigi elavate ees, sest tema on meie Issand ja Jumal, tema on meie isa ikka ja igavesti!
\par 5 Tema karistab meid meie ülekohtu pärast, aga halastab jälle, ja kogub meid kõigi paganate hulgast, kuhu iganes teid on nende keskele pillutatud.
\par 6 Kui te pöördute tema poole kõigest südamest ja kõigest hingest, ja teete tema ees seda, mis õige on, siis pöördub ka tema teie poole ega peida oma palet teie eest.
\par 7 Pange tähele, mida ta teiega teeb, ja tänage teda täiel häälel, kiitke õigluse Issandat ja ülistage igavest Kuningat!
\par 8 Mina tänan teda oma vangipõlvemaal, kuulutan tema võimsust ja suursugusust patusele rahvale. Pöörduge, patused, ja tehke õigust tema ees! Kes teab, vahest ta võtab teid vastu ja osutab teile halastust!
\par 9 Oma Jumalat ma ülistan, minu hing ülistab taeva Kuningat ja rõõmustab tema suursugususest!
\par 10 Jeruusalemmas tänagu teda kõik ja ütelgu: „Jeruusalemm, pühamu linn, tema küll karistab sind sinu laste tegude pärast, aga ta halastab jälle õigete inimeste laste peale.
\par 11 Täna Issandat kaunilt ja kiida igavest Kuningat, et taas ehitatakse rõõmuga tema eluase sinu sees,
\par 12 et ta sinus rõõmustaks vange ja armastaks õnnetuid igavesti põlvest põlve!
\par 13 Paljud rahvad tulevad kaugelt Issanda Jumala nime juurde, annid käes, annid taeva Kuningale. Põlv põlve järel toob sulle rõõmustust.
\par 14 Neetud olgu kõik, kes sind vihkavad, õnnistatud kõik, kes sind armastavad igavesti!
\par 15 Hõiska ja rõõmusta õigete inimeste laste pärast, sest nemad kogutakse, ja nad ülistavad õigete Issandat! Õndsad on, kes sind armastavad, nad saavad rõõmsaks sinu rahu pärast.
\par 16 Õndsad on, kes olid kurvad kõigi sinu karistuste pärast, sest nad tunnevad sinust rõõmu, kui näevad kogu sinu auhiilgust, ja nad on igavesti rõõmsad. Kiida, mu hing, Jumalat, suurt Kuningat!
\par 17 Sest Jeruusalemm ehitatakse safiirist ja smaragdist, sinu müürid on kalliskividest, tornid ja kaitsevallid on puhtast kullast. Ja Jeruusalemma tänavad sillutatakse berülli, rubiini ja Oofiri kiviga.
\par 18 Kõik tema kurud kuulutavad: „Halleluuja!” ja ülistavad, üteldes: „Kiidetud olgu Jumal, kes sind on ülendanud, ikka ja igavesti!””

\chapter{14}

\section*{Toobiti viimased sõnad}

\par 1 Ja Toobit lõpetas oma kiituslaulu.
\par 2 Ta oli viiekümne kaheksa aastane, kui ta kaotas nägemise, ja  kaheksa aasta pärast nägi ta jälle. Ta andis almuseid ja kartis alati  Issandat Jumalat ning tänas teda.
\par 3 Ta elas väga vanaks. Siis ta kutsus oma poja ja tema pojad ja ta  ütles temale: „Poeg, võta oma pojad! Vaata, ma olen vanaks saanud ja  valmis sellest elust lahkuma.
\par 4 Mine Meediasse, mu poeg, sest ma olen veendunud, et Niineve  hukkub, nagu prohvet Joona on ütelnud. Meedias on aga rahu kauemat  aega. Meie vennad, kes kodumaal on, pillutatakse sellelt healt maalt ja  Jeruusalemm jääb tühjaks. Jumala koda, mis seal on, põletatakse ära ja  jääb tühjaks tükiks ajaks.
\par 5 Siis Jumal halastab jälle nende peale ja laseb nad tagasi minna  nende oma maale. Ja nad ehitavad jälle koja - mitte endisega sarnase - ja  see jääb, kuni maailma ajastud on täis saanud. Pärast seda nad pöörduvad  tagasi vangipõlvest ja ehitavad Jeruusalemma toredasti üles. Ja Jumala  koda seal ehitatakse igaveseks kõigile sugupõlvedele, kuulsusrikkaks  ehitiseks, nagu prohvetid on sellest kõnelnud.
\par 6 Ja kõik paganad pöörduvad tõepoolest kartma Issandat Jumalat ja  matavad maha oma ebajumalad. Kõik paganad kiidavad Issandat.
\par 7 Tema oma rahvas tänab Jumalat ja Issand ülendab oma rahva. Siis  rõõmustavad kõik, kes Issandat Jumalat armastavad tões ja õigluses,  osutades halastust meie vendadele.
\par 8 Ja nüüd, mu poeg, mine ära Niinevest, sest kindlasti sünnib, nagu  prohvet Joona on ütelnud.
\par 9 Sina aga pea Seadust ja käske, ole halastav ja õiglane, et su  käsi hästi käiks! Mata mind ausasti ja oma ema minu kõrvale! Ärge  jääge enam Niinevesse!
\par 10 Mõtle sellele, mu poeg, mida Naadab tegi Ahhiahharosega, oma  toitjaga, kuidas ta viis tema valgusest pimedusse ja kuidas ta temale  tasus! Ahhiahharos küll päästeti, teisele aga maksti kätte ja ta pidi  astuma alla pimedusse. Manasse andis almuseid ja ta päästeti  surmavõrgust, mis temale oli pandud. Naadab langes aga võrku ja  hukkus.
\par 11 Vaadake siis nüüd, lapsed, mida almus suudab ja kuidas õiglus  päästab!” Seda üteldes heitis ta hinge oma voodis. Ta oli juba  sada viiskümmend kaheksa aastat vana. Ja nad matsid ta ausasti.
\par 12 Kui Hanna suri, siis Tobias mattis ta oma isa kõrvale. Siis  aga läks Tobias koos oma naise ja poegadega Ekbatanasse, oma äia  Ragueli juurde.
\par 13 Ja ta sai auga vanaks. Ta mattis ausasti oma naise vanemad ja  päris nende ning oma isa Toobiti varanduse.
\par 14 Ta suri Meedia Ekbatanas, olles sada kakskümmend seitse aastat  vana.
\par 15 Ja enne surma sai ta kuulda Niineve hukkumisest, mille  Nebukadnetsar ja Ahasveros vallutasid. Ta võis enne surma rõõmu tunda  Niineve karistamise pärast. 


\end{document}