\begin{document}

\title{Jeremija Nutulaulud}

\chapter{1}

\par 1 Kuidas küll istub üksinda see kord nii rahvarohke linn! Kes rahvaste seas oli suur, on saanud lesknaise sarnaseks. Vürstitar maakondade hulgas peab tegema orjatööd.
\par 2 Ta nutab öösel kibedasti, pisarad voolavad tal üle näo. Pole tal trööstijat ühestki armastajast: kõik ta sõbrad on teda petnud, on saanud ta vaenlasteks.
\par 3 Juuda on sattunud viletsusse ja ränka orjusesse, ta elab paganate seas ega leia hingamist. Kõik jälitajad saavad ta kätte keset kitsikusi.
\par 4 Siioni teed leinavad, et pole pühiks tulijaid. Kõik ta väravad on tühjad, ta preestrid ohkavad. Tema neitsid on kurvad ja tal enesel on kibe käes.
\par 5 Tema rõhujad on saanud võimu, tema vaenlastel käib käsi hästi. Sest Issand on toonud talle häda tema paljude üleastumiste pärast. Tema lastel on tulnud minna vangidena vaenlase ees.
\par 6 Siioni tütrelt on kõik ta toredus läinud. Tema vürstid on nagu hirved, kes ei leia karjamaad; nad astuvad jõuetult tagaajaja ees.
\par 7 Jeruusalemm mõtleb oma viletsuse ja kodutuse päevil kõigile kallistele asjadele, mis tal muistsest ajast on olnud. Et ta rahvas langes vaenlase kätte ja tal ei olnud aitajat, siis näevad vaenlased seda ja naeravad tema lõppu.
\par 8 Jeruusalemm on raskesti patustanud, seepärast on ta saanud jäleduseks. Kõik, kes teda austasid, põlastavad teda, sest nad on näinud ta alastiolekut. Ka tema ise ohkab ja tõmbub pelgu.
\par 9 Tema roojus on ta hõlmadel. Ta ei mõelnud oma lõpule ja on langenud hämmastaval viisil: tal pole trööstijat. Vaata, Issand, mu viletsust, sest vaenlane suurustab!
\par 10 Vaenlane sirutab käe kõigi ta kalliste asjade järele; sest ta sai näha, et ta pühamusse tulid paganad, need, keda sa olid keelanud, et nad ei tuleks su kogudusse.
\par 11 Kogu ta rahvas ohkab ja otsib leiba: nad annavad oma aarded toidu eest, et pidada hinge sees. Vaata, Issand, ja näe, kui põlatud ma olen!
\par 12 Kas see teile ei lähegi korda, kõik, kes te mööda lähete? Vaadake ja nähke! Ons olemas valu, minu valu sarnast, mis mulle on tehtud, millega Issand mind on nuhelnud oma tulise viha päeval?
\par 13 Ta läkitas kõrgusest tule mu luudesse ja vaevas neid; ta laotas mu jalgadele võrgu, ta tõrjus mind tagasi; ta jättis mind üksikuks igavesti haigena.
\par 14 Mu üleastumiste ike on kokku seotud, tema käed põimisid need ühte: need tulid mu turjale, tema murdis mu jõu; Issand andis mu nende kätte, ma ei suuda üles tõusta.
\par 15 Issand heitis kõrvale mu keskelt kõik vägevad; ta kutsus kokku peo minu vastu, et purustada mu noori mehi. Issand sõtkus surutõrt neitsile, Juuda tütrele.
\par 16 Nende asjade pärast ma nutan, mu silmad, mu silmad voolavad vett. Sest kaugel on minust trööstija, kes turgutaks mu hinge. Mu lapsed on hävitatud, sest vaenlane on vägev.
\par 17 Siion laiutab käsi, tal ei ole trööstijat. Issand on käsutanud vaenlastena Jaakobi vastu selle enese naabrid. Jeruusalemm on muutunud nende seas jäleduseks.
\par 18 Õiglane on tema, Issand, sest mina olen tõrkunud tema käsu vastu. Kuulge ometi, kõik rahvad, ja vaadake mu valu: mu neitsid ja noored mehed on läinud vangi.
\par 19 Ma kutsusin oma armastajaid, nad petsid mind. Mu preestrid ja vanemad surid linnas, kui nad otsisid enesele toitu, et turgutada oma hinge.
\par 20 Vaata, Issand, kuidas mul on kitsas käes: mu sisemus käärib, mu süda pööritab sees, sest ma olen kõvasti tõrkunud. Väljas teeb mõõk mind lastetuks, sees on päris surm.
\par 21 Nad on kuulnud, et ma ägan. Mul pole trööstijat. Kõik mu vaenlased on kuulnud mu õnnetusest, nad rõõmutsevad, et sina seda tegid. Too ometi päev, millest sa kuulutasid, et nende käsi käib samuti kui mul!
\par 22 Tulgu kõik nende kurjus su palge ette, ja talita nendega, nagu sa talitasid minuga kõigi mu üleastumiste pärast! Jah, mu ohkeid on palju ja mu süda on haige.

\chapter{2}

\par 1 Kuidas küll Issand oma vihas kattis pilvedega Siioni tütre! Ta heitis taevast maha Iisraeli ilu ega mõelnud oma jalgade järile oma vihapäeval.
\par 2 Issand hävitas armuta kõik Jaakobi eluasemed; oma vihas lõhkus ta maha Juuda tütre kindlused; ta tegi need maatasa, teotas kuningriigi ja selle vürstid.
\par 3 Oma tulises vihas raius ta maha kõik Iisraeli sarved. Ta tõmbas tagasi oma parema käe vaenlase ees ja süttis Jaakobis otsekui tuleleek, mis põletab ümbruse.
\par 4 Ta vinnastas oma ammu nagu vaenlane, seisis tõstetud parema käega nagu rõhuja ja tappis kõik silmarõõmu Siioni tütre telgis. Ta valas oma raevu välja kui tuld.
\par 5 Issand oli nagu vaenlane, ta hävitas Iisraeli; ta hävitas kõik tema paleed, purustas ta kindlused ja tõi Juuda tütrele hulgana kurvastust ja leina.
\par 6 Ta lammutas oma eluaseme otsekui aia, hävitas oma kogunemispaiga; Issand saatis Siionis unustusse pühad ja hingamispäevad, hülgas oma viha sajatuses kuningad ja preestrid.
\par 7 Issand tõukas ära oma altari, jättis maha oma pühamu, andis vaenlase kätte selle paleede müürid. Issanda kojast kostis kära otsekui pühade ajal.
\par 8 Issand otsustas hävitada Siioni tütre müürid; ta vedas mõõdunööri neist üle ega hoidnud oma kätt tagasi neid hävitamast; ta pani leinama kaitsevalli ja müüri, need varisesid üheskoos.
\par 9 Selle väravad vajusid maasse, ta hävitas ja murdis riivid. Selle kuningas ja vürstid on paganate seas, kus ei ole Seadust, prohvetidki ei saa seal Issandalt nägemust.
\par 10 Vaikides istuvad maas Siioni tütre vanemad: nad on riputanud enesele tuhka pähe, rõivastunud kotiriidesse. Oma pea on painutanud maani Jeruusalemma neitsid.
\par 11 Mu silmad on pisaraist kibedad, mu sisemus käärib, mu maks on valatud maha mu rahva tütre murdumise pärast. Sest linna turgudel on nõrkenud lapsed ja imikud.
\par 12 Nad küsivad emadelt: „Kus on leib ja vein?”, kui nad nõrkevad nagu haavatud linna turgudel, kui nad heidavad hinge oma ema süles.
\par 13 Mida võiksin sulle tunnistada, millega sind võrrelda, Jeruusalemma tütar? Mida võiksin pidada sinu sarnaseks, et sind trööstida, neitsi, Siioni tütar? Sest su purustus on suur nagu meri, kes suudaks sind parandada!
\par 14 Su prohvetid on kuulutanud sulle vääri ja mõttetuid nägemusi; aga nad ei ole paljastanud su süüd, et pöörata su saatust, vaid on sulle ilmutanud petlikke ja eksitavaid ennustusi.
\par 15 Kõik teekäijad löövad sinu pärast käsi kokku, nad vilistavad ja vangutavad pead Jeruusalemma tütre pärast: „Kas see on linn, mille kohta öeldi: ilu täius, kogu maa rõõm?”
\par 16 Kõik su vaenlased ajavad oma suud ammuli su vastu, nad vilistavad ja kiristavad hambaid, nad ütlevad: „Me oleme ta neelanud. See on tõesti päev, mida oleme oodanud, nüüd on see käes, me oleme seda näinud.”
\par 17 Issand tegi, mida ta oli otsustanud, tegi tõeks oma sõna, mida ta oli kuulutanud muistsest ajast: ta lõhkus maha ega halastanud, ta laskis vaenlasel su pärast rõõmu tunda, ta kergitas su rõhujate sarve.
\par 18 Nende süda kisendab Issanda poole. Siioni tütre müür, lase pisarail voolata jõena päeval ja öösel! Ära luba enesele lõtvust, ärgu olgu su silmateral rahu!
\par 19 Tõuse, karju öösel vahikordade alguses! Vala oma süda välja kui vesi Issanda palge ette! Tõsta oma käed tema poole oma laste elu pärast, kes on näljast nõrkemas igal tänavanurgal!
\par 20 Vaata, Issand, ja silmitse, kellele sa nõnda oled teinud: kas naised peavad sööma oma ihuvilja, terveina sündinud lapsi? Kas tohib Issanda pühamus tappa preestrit ja prohvetit?
\par 21 Tänavail lamab maas noor ja vana, mu neitsid ja noored mehed langesid mõõga läbi. Sina surmasid oma vihapäeval, tapsid, ei andnud armu.
\par 22 Sa kutsusid kokku nagu pidupäevaks mu vaenlased igalt poolt; Issanda vihapäeval ei jäänud põgenikku ega pääsenut: neile, keda olin ilmale toonud ja kasvatanud, tegi mu vaenlane lõpu.

\chapter{3}

\par 1 Mina olen mees, kes nägi viletsust tema nuhtluse nuudi all.
\par 2 Ta ajas mind ja pani mind käima pimeduses, mitte valguses.
\par 3 Tõesti, ta pööras oma käe minu vastu ja tõstab seda minu vastu iga päev.
\par 4 Ta kulutas mu liha ja naha, ta murdis mu luud.
\par 5 Ta ehitas mu vastu kindluse ja ümbritses mind kibeduse ning vaevaga.
\par 6 Ta pani mu istuma pimedusse nagu need, kes on ammu surnud.
\par 7 Ta tegi mu ümber müüri ja ma ei pääse välja, ta pani mind raskeisse ahelaisse.
\par 8 Kuigi ma hüüan ja karjun appi, summutab tema mu palved.
\par 9 Ta tegi mu teele tahutud kividest müüri, rikkus mu teerajad.
\par 10 Ta on mulle varitsevaks karuks, peidus olevaks lõviks.
\par 11 Ta paiskas segi mu teed, kiskus mind lõhki, tegi mu lagedaks.
\par 12 Ta tõmbas oma ammu vinna ja pani mind oma nooltele märgiks.
\par 13 Ta laskis oma nooled mu neerudesse.
\par 14 Ma olen kogu oma rahva naeruks, nende igapäevaseks pilkelauluks.
\par 15 Ta toitis mind kibedate taimedega, jootis mind koirohuga.
\par 16 Ta vajutas mu põrmu, laskis mu hambad kuluda sõmeras.
\par 17 Jah, sina tõukasid mu hinge rahupõlvest välja, ma olen unustanud, mis on õnn.
\par 18 Ma ütlen: Kadunud on mu jõud ja mu lootus Issanda peale.
\par 19 Mõtle mu viletsusele ja kodutusele, koirohule ja mürgile!
\par 20 Sina küll mõtled sellele, et mu hing on rõhutud.
\par 21 Seda võtan ma südamesse, sellepärast loodan ma veel:
\par 22 see on Issanda suur heldus, et me pole otsa saanud, sest tema halastused pole lõppenud:
\par 23 need on igal hommikul uued - sinu ustavus on suur!
\par 24 Issand on mu osa, ütleb mu hing, seepärast loodan ma tema peale.
\par 25 Issand on hea neile, kes teda ootavad, hingele, kes teda otsib.
\par 26 Hea on oodata kannatlikult Issanda päästet.
\par 27 Hea on mehele, kui ta kannab iket oma nooruses.
\par 28 Ta istugu üksi ja vakka, kui see on pandud ta peale!
\par 29 Ta pistku oma suu põrmu - võib-olla on veel lootust!
\par 30 Ta andku oma põsk sellele, kes teda lööb, et ta oleks küllalt teotatud!
\par 31 Sest Issand ei tõuka ära igaveseks.
\par 32 Kui ta on kurvastanud, siis ta ka halastab oma suure helduse pärast.
\par 33 Sest ta ei alanda ega kurvasta inimlapsi mitte südamest.
\par 34 Kui jalge alla tallatakse kõik vangid maal,
\par 35 kui väänatakse mehe õigust Kõigekõrgema palge ees,
\par 36 kui inimesele tehakse ülekohut tema riiuasjas - kas Issand seda ei näe?
\par 37 Kes ütleb, et midagi sünnib, ilma et Issand oleks seda käskinud?
\par 38 Eks tule Kõigekõrgema suust niihästi kuri kui hea?
\par 39 Miks inimesed elus nurisevad? Igaüks nurisegu omaenese patu pärast!
\par 40 Uurigem ja proovigem oma teid ja pöördugem tagasi Issanda juurde!
\par 41 Tõstkem oma südamed ja käed Jumala poole taevas!
\par 42 Me oleme üleastujad ja vastuhakkajad, sina ei andnudki meile andeks.
\par 43 Sa oled peitunud vihasse, oled meid jälitanud, armuta surmanud.
\par 44 Sa oled peitunud pilvesse, et palved ei pääseks läbi.
\par 45 Sa oled teinud meid pühkmeiks ja jätisteks rahvaste seas.
\par 46 Kõik meie vaenlased ajavad oma suu ammuli meie vastu.
\par 47 Meile on tulnud hirm ja haud, hävitus ja hukkumine.
\par 48 Veeojad voolavad mu silmist mu rahva tütre hävingu pärast.
\par 49 Mu silmad voolavad lakkamatult, pisaratel ei ole pidamist,
\par 50 kuni Issand vaatab taevast alla ja näeb.
\par 51 Mu silm teeb mu hingele valu kõigi mu linna tütarde pärast.
\par 52 Tõesti, nagu lindu küttisid mind need, kes põhjuseta on mu vaenlased.
\par 53 Nad tahtsid mu elu kustutada kaevus ja pildusid mu peale kive.
\par 54 Vesi tõusis mul üle pea, ma ütlesin: „Nüüd olen kadunud!”
\par 55 Ma hüüdsin su nime, Issand, sügavaimast kaevust.
\par 56 Sina kuulsid mu hüüdu: „Ära peida oma kõrva mu appihüüde eest, et saaksin kergendust!”
\par 57 Sa olid ligi, kui ma sind hüüdsin, sa ütlesid: „Ära karda!”
\par 58 Sina, Issand, seletad mu hinge riiuasja, sina lunastad mu elu.
\par 59 Sina, Issand, näed mu rõhumist, mõista mulle õigust!
\par 60 Sina näed kogu nende kättemaksu, kõiki nende kavatsusi mu vastu.
\par 61 Sina kuuled nende laimamist, Issand, kõiki nende kavatsusi minu vastu.
\par 62 Mu vastaste huuled ja nende pomin on mu vastu iga päev.
\par 63 Vaata, kas nad istuvad või tõusevad - mina olen nende pilkelaul.
\par 64 Maksa neile kätte, Issand, nende kätetööd mööda!
\par 65 Anna neile paadunud süda, tulgu su sajatus nende peale!
\par 66 Aja neid taga raevus ja hävita nad Issanda taeva alt!

\chapter{4}

\par 1 Kuidas küll on tuhmunud kuld, puhas kuld kuidas teiseks saanud! Pühamu kivid on paisatud kõigile tänavanurkadele.
\par 2 Kallid Siioni lapsed, puhtaima kullaga võrdsed, kuidas on nad nüüd saanud saviastjate sarnaseks, potisseppade käsitööks!
\par 3 Ðaakalidki ulatavad nisa, et imetada oma poegi, aga mu rahva tütar on julm, otsekui jaanalind kõrbes.
\par 4 Imiku keel jääb kinni suulakke janu pärast, lapsed paluvad leiba, aga pole, kes neile jagaks.
\par 5 Kes enne sõid maiustusi, närbuvad tänavail; keda hellitati purpuri peal, lebavad sõnnikuhunnikul.
\par 6 Mu rahva tütre süü on suurem kui patt Soodomas, mis paisati segi silmapilkselt, kätega aitamata.
\par 7 Tema vürstid olid puhtamad lumest, valgemad piimast, ihult korallidest verevamad, kujult otsekui safiirid.
\par 8 Nüüd on nad näost mustemad kui nõgi, neid ei tunta tänavail ära; nende nahk on kontidel kortsunud, kuivanud nagu puu.
\par 9 Õnnelikumad olid need, kes mõõgaga maha löödi, kui need, kes surid nälga, kes põllusaagi puudumisel kidusid nagu teibasse aetud.
\par 10 Kaastundlike naiste käed keetsid oma lapsi: need olid neile roaks mu rahva tütre hävingus.
\par 11 Issand valas välja oma viha, tegi teoks oma tulise raevu ja süütas Siionis tule, mis põletas selle alusmüüridki.
\par 12 Ei oleks uskunud maa kuningad ja kõik maailma elanikud, et vihamees ja vaenlane tuleb sisse Jeruusalemma väravaist.
\par 13 See on sündinud tema prohvetite pattude, tema preestrite süü pärast; nende pärast, kes valasid seal õigete verd.
\par 14 Nad vaarusid tänavail nagu pimedad, verega roojastatud, nõnda et nende riideid ei võinud puudutada.
\par 15 „Hoidke eest! Roojane!„ hüüti nende kohta. „Hoidke eest, hoidke eest, ärge puudutage!” Nad vaarusid ka põgenedes, rahvaste seas öeldi: ”Nad ei tohi jääda siia kauemaks!”
\par 16 Issanda pale hajutas nad, ta ei vaata enam nende peale. Preestritest ei peetud lugu, vanadele ei antud armu.
\par 17 Isegi veel siis, väsinud silmadega, me ootasime asjatult endile abi; oma vahitornidest piilusime rahva poole, kes meid ei päästnud.
\par 18 Meie samme luurati, me ei võinud käia oma turgudel; meie lõpp ligines, meie päevad said täis - tõesti, meie lõpp tuli!
\par 19 Meie jälitajad olid kiiremad kui kotkas taeva all; nad ajasid meid taga mägedel, varitsesid meid kõrbes.
\par 20 Issanda võitu, kes oli meile eluõhuks, püüti kinni nende aukudes, tema, kellest me ütlesime: „Tema varjus me elame paganate seas!”
\par 21 Rõõmutse ja ole rõõmus, Edomi tütar, kes elad Uusimaal! Sinulegi tuleb karikas: sa jääd joobnuks ja kisud enese paljaks.
\par 22 Sinu süü on lõppenud, Siioni tütar, enam ta ei vii sind vangi. Aga ta karistab su süüd, Edomi tütar, ta paljastab su patud.

\chapter{5}

\par 1 Mõtle, Issand, sellele, mis on meiega juhtunud, vaata ja näe meie teotust!
\par 2 Meie pärisosa on läinud võõraste, meie kojad muulaste kätte.
\par 3 Me oleme jäänud orbudeks, isatuiks, meie emad on lesed.
\par 4 Vett me joome raha eest, puid me saame ostes.
\par 5 Jälitajad on meil kaela peal, me väsime, meile ei anta asu.
\par 6 Egiptusele ja Assurile me andsime käe, et saada kõhutäit leiba.
\par 7 Meie vanemad tegid pattu: neid ei ole enam. Meie kanname nende süüd.
\par 8 Orjad valitsevad meie üle, ei ole nende käest lahtikiskujat.
\par 9 Elu ohustades toome enestele leiba, sest kõrbes on mõõk.
\par 10 Meie nahk hõõgub nagu ahi näljakõrvetuste pärast.
\par 11 Siionis on naised raisatud ja Juuda linnades neitsid.
\par 12 Vürstid on poodud nende käe läbi, vanade vastu ei ole olnud austust.
\par 13 Noored mehed peavad ajama käsikivi ja poisid komistavad puukoorma all.
\par 14 Vanemad on kadunud väravast, noorukid pillimängude juurest.
\par 15 On lõppenud meie südame rõõm, meie tants on muutunud leinaks.
\par 16 Kroon on langenud meie peast. Häda meile, et oleme pattu teinud!
\par 17 Seepärast on meie süda haige, nende asjade pärast on meie silmad jäänud pimedaks,
\par 18 Siioni mäe pärast, mis on nii laastatud, et seal luusivad rebased.
\par 19 Sina, Issand, valitsed igavesti, sinu aujärg jääb põlvest põlve.
\par 20 Mispärast sa tahad meid unustada alatiseks, meid maha jätta pikaks ajaks?
\par 21 Too meid, Issand, tagasi enese juurde, siis me pöördume! Uuenda meie päevi nagu muiste!
\par 22 Või oled sa meid tõuganud hoopis ära, vihastunud meie peale üliväga?



\end{document}