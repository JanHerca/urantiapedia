\begin{document}

\title{Jeremija}

\chapter{1}

\par 1 Jeremija, Hilkija poja sõnad. Jeremija oli Benjamini maal Anatotis asuvate preestrite hulgast.
\par 2 Ta sai Issanda sõna Juuda kuninga Joosija, Aamoni poja päevil, tema valitsemise kolmeteistkümnendal aastal,
\par 3 ja siis veel Juuda kuninga Joojakimi, Joosija poja päevil kuni Juuda kuninga Sidkija, Joosija poja üheteistkümnenda aasta lõpuni, kuni Jeruusalemma vangiviimiseni viiendas kuus.
\par 4 Mulle tuli Issanda sõna; ta ütles:
\par 5 „Enne kui ma sind emaihus valmistasin, tundsin ma sind, ja enne kui sa emaüsast välja tulid, pühitsesin ma sinu: ma panin su rahvastele prohvetiks.”
\par 6 Aga mina ütlesin: „Oh Issand Jumal! Vaata, ma ei oska rääkida, sest ma olen noor.”
\par 7 Kuid Issand ütles mulle: „Ära ütle: ma olen noor, vaid mine kõikjale, kuhu ma sind läkitan, ja räägi kõike, mida ma sind käsin.
\par 8 Ära karda neid, sest mina olen sinuga, ja päästan sinu, ütleb Issand.”
\par 9 Ja Issand sirutas oma käe ning puudutas mu suud; ja Issand ütles mulle: „Vaata, ma annan oma sõnad sulle suhu.
\par 10 Vaata, ma panen su täna rahvaste ja kuningriikide üle, kitkuma ja rebima, hävitama ja purustama, istutama ja rajama.”
\par 11 Ja mulle tuli Issanda sõna; ta küsis: „Mida sa näed, Jeremija?„ Mina vastasin: ”Ma näen mandlipuu oksa!”
\par 12 Ja Issand ütles mulle: „Sa oled õigesti näinud, sest mina olen valvas oma sõna teoks tegema!”
\par 13 Ja Issanda sõna tuli mulle teist korda; ta küsis: „Mida sa näed?„ Mina vastasin: ”Ma näen ülekeevat pada põhjakaares!”
\par 14 Ja Issand ütles mulle: „Põhjakaarest pääseb lahti õnnetus kõigi maa elanike peale.
\par 15 Sest vaata, ma kutsun kõiki suguvõsasid põhjapoolseist kuningriikidest, ütleb Issand, ja need tulevad ning asetavad igaüks oma aujärje Jeruusalemma väravate ette ja kõigi selle müüride vastu ümberringi, ja kõigi Juuda linnade vastu.
\par 16 Siis ma mõistan nende üle kohut kõigi nende pahategude eest, et nad jätsid minu maha ja suitsutasid teistele jumalatele ning kummardasid oma kätetööd.
\par 17 Aga sina vööta oma niuded, võta kätte ja räägi neile kõik, mida ma sul käsin. Ära kohku nende ees, et mina sind nende ees ei peaks kohutama!
\par 18 Sest vaata, mina panen su täna kindlustatud linnaks ja raudsambaks ning vaskmüüriks kogu maa vastu, Juuda kuningate, selle vürstide, preestrite ja maa rahva vastu.
\par 19 Nad võitlevad sinu vastu, aga nad ei saa võimust su üle, sest mina olen sinuga, ütleb Issand, ja ma päästan sinu.”

\chapter{2}

\par 1 Ja mulle tuli Issanda sõna; ta ütles:
\par 2 „Mine ja kuuluta Jeruusalemma kuuldes ning ütle: Nõnda ütleb Issand: Ma mäletan su noorpõlve kiindumust, su mõrsjapõlve armastust, kui sa käisid mu järel kõrbes, külvamata maal.
\par 3 Iisrael oli pühitsetud Issandale, olles tema uudsevili: kõik, kes teda sõid, said süüdlasteks, nende peale tuli õnnetus, ütleb Issand.
\par 4 Kuulge Issanda sõna, Jaakobi sugu ja kõik Iisraeli soo suguvõsad!
\par 5 Nõnda ütleb Issand: Missuguse ülekohtu leidsid teie vanemad minus, et nad läksid minust eemale ja käisid tühjuse järel ning said tühjuseks?
\par 6 Nad ei küsinud: „Kus on Issand, kes tõi meid Egiptusemaalt, kes juhtis meid kõrbes, lagedal ja auklikul maal, põuasel ja süngel maal, kus ei käi mitte keegi ja kus ei ela ükski inimene?”
\par 7 Mina tõin teid viljakale maale sööma selle vilja ja hüvesid; aga jõudnud sinna, te roojastasite mu maa ja tegite mu pärisosa jäleduseks.
\par 8 Preestrid ei küsinud: „Kus on Issand?” Seaduse seletajad ei tundnud mind, karjased astusid üles mu vastu, prohvetid ennustasid Baali nimel ja käisid asjade järel, millest pole abi.
\par 9 Sellepärast ma veel riidlen teiega, ütleb Issand, ja riidlen ka teie laste lastega.
\par 10 Sest käige läbi kittide saared ja vaadake, läkitage Keedarisse, pange hästi tähele ja vaadake: kas seal on sündinud sellesarnast?
\par 11 Kas on ükski rahvas vahetanud oma jumalaid, kuigi need ei olegi jumalad? Aga minu rahvas on vahetanud oma Aulise selle vastu, millest pole abi.
\par 12 Hämmastu sellest, taevas, vabise väga, ütleb Issand.
\par 13 Sest mu rahvas on teinud kahekordse süüteo: minu, elava vee allika, jätsid nad maha, et raiuda enestele kaevusid, pragulisi kaevusid, mis ei pea vett.
\par 14 Kas Iisrael on sulane? Või on ta pärisori? Miks ta sai riisutavaks?
\par 15 Tema vastu möirgasid noored lõvid, andsid kuulda oma häält; nad tegid ta maa kõledaks, ta linnad on varemeis, elaniketa.
\par 16 Ka Noofi pojad ja tahpaneeslased on su pealae paljaks püganud.
\par 17 Eks sa ise ole seda enesele teinud, hüljates Issanda, oma Jumala, siis kui ta saatis sind teekonnal?
\par 18 Miks on sul nüüd vaja minna Egiptusesse jooma Siihori vett? Ja miks on sul vaja minna Assurisse jooma Frati vett?
\par 19 Sind karistab su oma kurjus ja noomivad su taganemised. Mõista ja näe, et see on halb ja kibe, kui sa jätad maha Issanda, oma Jumala, kui sul pole kartust minu ees, ütleb Issand, vägede Issand.
\par 20 Sest ammusest ajast oled sa murdnud oma ikke, rebinud katki oma köidikud ja öelnud: „Mina ei taha teenida!” Jah, igale kõrgemale künkale ja iga halja puu alla laskusid sa truudust murdma.
\par 21 Mina olen sind istutanud heaks viinapuuks täiesti puhtast seemnest; kuidas sa küll oled muutunud mulle võõraks kärbunud viinapuuks?
\par 22 Kuigi sa ennast peseksid leelisega ja võtaksid palju seepi, jääb su süü minu ees mustuselaiguks, ütleb Issand Jumal.
\par 23 Kuidas sa võid öelda: „Mina ei ole ennast roojastanud, mina ei ole käinud baalide järel!”? Vaata oma teekonda orus, mõista, mis sa oled teinud, sa indlev kaamelimära, kes jooksed oma teedel sinna ja tänna,
\par 24 metsemaeesel kõrbes, kes kangest innast ahmib tuult. Kes saaks keelata tema kiima? Ühelgi, kes teda otsib, ei ole vaja ennast väsitada: ta leiab tema ta innaajal.
\par 25 Hoia oma jalga paljaks jäämast ja oma kurku janu eest! Aga sina ütled: „Asjata! Ei, sest ma armastan võõraid ja käin nende järel.”
\par 26 Otsekui varas jääb häbisse, kui ta tabatakse, nõnda jääb häbisse Iisraeli sugu: nemad, nende kuningad, vürstid, preestrid ja prohvetid,
\par 27 need, kes ütlevad puule: „Sina oled mu isa„, ja kivile: „Sina oled mu sünnitanud”, sest nad on pööranud mu poole kukla, aga mitte näo. Ent oma hädaajal nad ütlevad: ”Tõuse ja päästa meid!”
\par 28 Aga kus on su jumalad, keda sa enesele valmistasid? Tõusku nemad, kui nad su hädaajal suudavad sind päästa! Sest nõnda palju kui sul on linnu, on sul jumalaid, Juuda.
\par 29 Mispärast riidlete minuga? Teie kõik olete astunud üles mu vastu, ütleb Issand.
\par 30 Ilmaaegu olen ma löönud teie lapsi, nad ei ole võtnud õpetust. Teie oma mõõk on õginud teie prohvetid, otsekui murdja lõvi.
\par 31 Oh teie sugupõlve! Nähke Issanda sõna: kas olen mina olnud Iisraelile kõrbeks või pilkase pimeduse maaks? Mispärast ütleb mu rahvas: „Me oleme vabad käima sinna-tänna, me ei tule enam sinu juurde!”?
\par 32 Kas neitsi unustab oma ehte, pruut oma paelad? Aga minu rahvas on unustanud minu loendamatuil päevil.
\par 33 Kui hästi sa oskad otsida armastust! Seepärast sa oledki harjunud kurjaga oma kommetes.
\par 34 Su kuuepalistustelt leitakse ka vaeste süütute verd, kuigi sa ei ole tabanud neid sissemurdmiselt.
\par 35 Aga kõige selle juures ütled sa ometi: „Ma olen süütu, ta viha pöördub tõesti mu pealt.„ Vaata, ma lähen sinuga kohtusse, sellepärast et sa ütled: ”Ma pole pattu teinud!”
\par 36 Miks jooksed sa nii kerglaselt kord siia, kord sinna? Ka Egiptuse pärast pead sa häbenema, nagu sa häbenesid Assuri pärast.
\par 37 Ka siit tuleb sul ära minna käed pea peal, sest Issand põlgab neid, kelle peale sa loodad, ja sul ei õnnestu ennast päästa.

\chapter{3}

\par 1 On öeldud: Vaata, kui mees saadab ära oma naise ja see läheb tema juurest minema ning saab teisele mehele: kas esimene tohib teda jälle tagasi võtta? Kas pole nõnda, et maa saaks sellest roojaseks? Sina oled teinud hooratööd paljude armatsejatega ja tahad pöörduda tagasi minu juurde, ütleb Issand.
\par 2 Tõsta oma silmad üles küngaste poole ja vaata: kus ei ole sind magatatud? Neid oodates istusid sa teede ääres otsekui araablane kõrbes ning roojastasid maad oma hooruse ja pahategudega.
\par 3 Seepärast ei tulnud varast vihmasadu ja ka hiline vihm jäi tulemata: sul on hooranaise laup, aga sa ei häbene.
\par 4 Nüüd hüüad sa mind: „Oh, mu isa! Mu noorpõlvesõber!
\par 5 Kas sa pead igavesti viha ja oled sellepärast alati valvel?„ Vaata, nõnda sa räägid, ise aga teed usinasti kurja.”
\par 6 Ja Issand ütles mulle kuningas Joosija päevil: „Kas sa oled näinud, mida on teinud see taganeja Iisrael? Ta on käinud igal kõrgel mäel ja iga halja puu all ning on teinud seal hooratööd.
\par 7 Ja ma ütlesin talle, kui ta oli teinud seda kõike: „Pöördu tagasi minu juurde!” Aga ta ei pöördunud. Ja seda nägi tema truuduseta õde Juuda.
\par 8 Ja ta nägi, et ma saatsin ära taganeja Iisraeli ja andsin temale lahutuskirja, sellepärast et ta oli abielu rikkunud. Ometi ei kartnud tema truuduseta õde Juuda, vaid läks ka ise tegema hooratööd.
\par 9 Ja oma kergemeelse hooratööga rüvetas ta maa ning rikkus abielu kiviga ja puuga.
\par 10 Sellest hoolimata ei ole tema truuduseta õde Juuda pöördunud minu poole kõigest südamest, vaid on seda teinud teesklemisi, ütleb Issand.”
\par 11 Ja Issand ütles mulle: „Taganenud Iisrael on osutunud süütumaks kui truuduseta Juuda.
\par 12 Mine ja hüüa neid sõnu põhja poole ja ütle: Pöördu, taganeja Iisrael, ütleb Issand, siis ma ei vaata teie peale enam süngel pilgul, sest mina olen armuline, ütleb Issand, mina ei pea viha mitte igavesti.
\par 13 Aga tunne oma süüd, et sa oled üles astunud Issanda, oma Jumala vastu ja oled iga halja puu all ajanud oma põlved laiali võõrastele. Minu häält ei ole te mitte kuulnud, ütleb Issand.
\par 14 Pöörduge ümber, taganenud lapsed, ütleb Issand, sest mina olen võtnud teid oma valdusesse; ja mina võtan teid: igast linnast ühe ja igast suguvõsast kaks ning viin teid Siionisse!
\par 15 Ja ma annan teile karjaseid oma südame järgi, ja need karjatavad teid targasti ja taibukalt.
\par 16 Ja kui neil päevil saab teid sellel maal palju ja te olete viljakad, ütleb Issand, siis ei räägita enam Issanda seaduselaekast ega ole see meeleski, sellele ei mõelda, sellest ei tunta puudust ja seda ei valmistatagi enam.
\par 17 Sel ajal nimetatakse Jeruusalemma „Issanda aujärjeks” ja kõik rahvad kogunevad sinna, Issanda nime juurde Jeruusalemma; ja nad ei käi enam oma kurja südame kanguse järgi.
\par 18 Neil päevil läheb Juuda sugu Iisraeli soo juurde ja nad tulevad üheskoos põhjamaalt maale, mille mina olen andnud pärisosaks teie vanemaile.
\par 19 Ja ma mõtlesin: Küll tahaksin sind panna laste sekka ja anda sulle meeldiva maa, toredaist toredaima pärisosa rahvaste keskel. Ja ma mõtlesin: Sina hüüaksid mind isaks ega taganeks mu järelt.
\par 20 Aga otsekui naine on truuduseta oma eluseltsilise vastu, nõnda olete teie, Iisraeli sugu, olnud truuduseta minu vastu, ütleb Issand.
\par 21 Küngastel kuuldub häält, Iisraeli laste haledat nuttu, sellepärast et nad on käinud vääral teel, on unustanud Issanda, oma Jumala.
\par 22 Pöörduge tagasi, taganenud lapsed, ma teen teid terveks teie taganemisest! „Vaata, me tuleme sinu juurde, sest sina oled Issand, meie Jumal.
\par 23 Tõesti, petlik on see, mis kuuldub küngastelt, lärm mägedel. Tõesti, Issandas, meie Jumalas on Iisraeli pääste.
\par 24 „Häbi” on söönud meie vanemate töövilja, meie noorpõlvest alates: nende lambad ja veised, nende pojad ja tütred.
\par 25 Lamagem siis oma häbis ja katku meid meie teotus, sest me oleme pattu teinud Issanda, oma Jumala vastu, meie ja meie vanemad oma noorpõlvest tänapäevani ega ole võtnud kuulda Issanda, oma Jumala häält.”

\chapter{4}

\par 1 Kui sa, Iisrael, pöördud, siis pöördu minu poole, ütleb Issand; ja kui sa mu palge eest kõrvaldad oma jäledused, siis sa ei jää kodutuks.
\par 2 Ja kui sa vannud: „Nii tõesti kui Issand elab!” - vannud tões, õiguses ja õigluses, siis õnnistavad rahvad endid temaga ja kiitlevad temast.
\par 3 Sest nõnda ütleb Issand Juuda meestele ja Jeruusalemmale: Kündke enestele uudismaad ja ärge külvake kibuvitste sekka!
\par 4 Laske endid ümber lõigata Issandale ja kõrvaldage oma südamete eesnahad, Juuda mehed ja Jeruusalemma elanikud, et mu viha ei süttiks nagu tuli ega põleks teie tegude kurjuse pärast, ilma et keegi kustutaks.
\par 5 Teatage Juudas ja kuulutage Jeruusalemmas ning öelge: Puhuge maal sarve, hüüdke valjusti ja öelge: Tulge kokku ja läki kindlustatud linnadesse!
\par 6 Tõstke lipp Siioni poole, põgenege, ärge peatuge! Sest mina toon põhja poole õnnetuse ja suure hävingu.
\par 7 Lõvi tõuseb oma rägastikust ja rahvaste hävitaja asub teele, tuleb välja oma asupaigast tegema su maad tühjaks, su linnu elaniketa varemeiks.
\par 8 Seepärast rõivastuge kotiriidesse, kurtke ja kaevelge, sest Issanda vihalõõm ei ole meie pealt pöördunud.
\par 9 Ja sel päeval, ütleb Issand, kaob julgus kuningal ja vürstidel, preestrid lõdisevad ja prohvetid on hämmeldunud.
\par 10 Aga mina ütlesin: Oh, Issand Jumal! Sa oled seda rahvast ja Jeruusalemma hoopis petnud, öeldes: Teile tuleb rahu - kuid ometi ulatab mõõk elu ligi!
\par 11 Sel ajal öeldakse sellele rahvale ja Jeruusalemmale: Kuum tuul tuleb lagedailt küngastelt kõrbes mu rahva tütre poole, aga mitte tuulamiseks ega puhastamiseks.
\par 12 Tuul, mis selleks on liiga vali, tuleb minu tarbeks: nüüd kuulutan ka mina neile kohut.
\par 13 Vaata, ta tõuseb pilvede sarnaselt ja ta sõjavankrid on otsekui tuulispea; ta hobused on kotkaist kiiremad. Häda meile, sest me oleme kadunud!
\par 14 Pese oma süda kurjusest, Jeruusalemm, et sind saaks päästa; kui kaua sa lased viibida eneses nurjatuil mõtteil?
\par 15 Sest kisa kostab Daanist ja kuulutab õudust Efraimi mäestikust:
\par 16 „Teatage rahvaile, vaata, kuulutage Jeruusalemmale: Piirajad tulevad kaugelt maalt ja tõstavad sõjakisa Juuda linnade vastu.”
\par 17 Need asuvad ta ümber otsekui väljavahid, sellepärast et ta on hakanud mulle vastu, ütleb Issand.
\par 18 Sinu tee ja sinu teod on seda sulle teinud; jah, sinu oma kurjuse pärast on see nii kibe, et ulatub su südameni.
\par 19 Mu rind, mu rind, ma väänlen valudes. Oh, mu südameseinad! Mu süda tormitseb mu sees. Ma ei saa vaiki olla, sest ma kuulen sarvehäält, sõjakära.
\par 20 Hävingut kuulutatakse hävingu peale, sest kogu maa on rüüstatud; äkitselt rüüstatakse mu telgid, ühe hetkega mu telgiriided.
\par 21 Kui kaua ma pean nägema lippu, kuulma sarvehäält?
\par 22 Mu rahvas on ju meeletu, ta ei tunne mind; nad on rumalad lapsed, neil ei ole arusaamist; nad on küll targad tegema kurja, aga nad ei mõista teha head.
\par 23 Ma vaatasin maad, ja ennäe, see oli tühi ja paljas; ma vaatasin taeva poole, aga seal ei olnud valgust.
\par 24 Ma vaatasin mägesid, ja ennäe, need vabisesid ja kõik künkad kõikusid.
\par 25 Ma vaatasin, ja ennäe, ei olnud ühtegi inimest ja kõik taeva linnud olid põgenenud.
\par 26 Ma vaatasin, ja ennäe, viljakas maa oli kõrb ja kõik linnad lõhutud Issanda vihalõõmast, tema palge ees.
\par 27 Sest Issand ütleb nõnda: Kõik maa peab saama lagedaks. Ometi ma ei tee lõppu.
\par 28 Seepärast leinab maa ja taevas ülal läheb mustaks; sest ma olen rääkinud ja otsustanud, ma ei kahetse seda ega loobu sellest.
\par 29 Ratsanike ja ammuküttide kisa pärast põgeneb iga linn: nad lähevad padrikuisse, tõusevad kaljudele; iga linn jäetakse maha ja keegi ei ela neis.
\par 30 Aga sina, rüüstatu, mida sa teed? Kuigi sa riietud purpurisse, kuigi sa ehid ennast kuldehetega, kuigi sa suurendad oma silmi värviga, ilustad sa ennast asjata. Su armastajad põlgavad sind ja nõuavad su elu.
\par 31 Sest ma kuulen häält, nagu oleks keegi sünnitusvaludes, esmasünnitaja ahastust, Siioni tütre häält. Ta hingeldab ja ringutab käsi: „Häda mulle, ma vaagun hinge tapjate käes!”

\chapter{5}

\par 1 Käige läbi Jeruusalemma tänavad, vaadake ometi ja pange tähele, ja otsige ta turgudelt, kas leiate kedagi, kes teeb õigust, kes nõuab tõde - siis ma annan linnale andeks.
\par 2 Aga kuigi nad ütlevad: „Nii tõesti kui Issand elab”, vannuvad nad siiski valet.
\par 3 Issand! Kas su silmad siis tõe peale ei vaata? Sina lõid neid, aga nad ei tundnud valu; sina hävitasid neid, aga nad ei võtnud õpetust. Nad on teinud oma palged kaljust kõvemaks, nad keelduvad pöördumast.
\par 4 Aga mina mõtlesin: Need on ainult viletsad, need on rumalad, sest nad ei tunne Issanda teed, oma Jumala õigust.
\par 5 Ma lähen ülemate juurde ja räägin nendega, sest nemad tunnevad Issanda teed, oma Jumala õigust. Aga kõik needki olid murdnud ikke, katki rebinud köidikud.
\par 6 Seepärast tapab neid metsa lõvi, hävitab lagendiku hunt, luurab panter nende linnu: kes iganes neist välja tuleb, kistakse lõhki. Sest nende üleastumisi on palju, nende taganemised on suured.
\par 7 Kuidas ma võin sulle andeks anda? Su lapsed on minu maha jätnud ja on andnud vande nende juures, kes ei olegi jumalad. Mina toitsin neid, aga nad rikkusid abielu ja logelevad pordumajas.
\par 8 Nad on lihavad, ringi tõmbavad täkud: igaüks hirnub oma ligimese naise järele.
\par 9 Kas ma ei peaks selliseid karistama? ütleb Issand. Kas ma ei peaks kätte tasuma niisugusele rahvale?
\par 10 Tõuske ta viinamäe astanguile, lõhkuge kaitsemüürid maha, aga ärge tehke neile sootuks lõppu: rebige ta lokkavad kasvud, sest need ei ole Issanda omad.
\par 11 Tõesti, nad on olnud väga truuduseta mu vastu, Iisraeli sugu ja Juuda sugu, ütleb Issand.
\par 12 Nad on salanud Issandat ja on öelnud: „Tema seda küll ei tee! Ei taba meid õnnetus, ei näe me mõõka ega nälga!”
\par 13 Ja prohvetid? Neid peetakse tuuleks ja nende sees ei olevat sõna. Sündigu nende enestega nõnda!
\par 14 Seepärast ütleb Issand, vägede Issand, nõnda: Kuna te olete nõnda rääkinud, vaata, siis ma teen sõnad te suus tuleks ja selle rahva puudeks, et teid põletataks.
\par 15 Vaata, ma toon teie kallale kaugelt ühe rahva, oh Iisraeli sugu, ütleb Issand. See on vastupidav rahvas, see on igivana rahvas, rahvas, kelle keelt te ei oska ega mõista, mida ta räägib.
\par 16 Tema nooletupp on nagu lahtine haud, nad kõik on vaprad võitlejad.
\par 17 Ta sööb su lõikuse ja leiva, ta sööb su pojad ja tütred, ta sööb su lambad ja veised, ta sööb su viinapuud ja viigipuud. Ta hävitab mõõgaga su kindlustatud linnad, mille peale sa loodad.
\par 18 Aga ka neil päevil, ütleb Issand, ei tee ma teile lõppu.
\par 19 Ja kui küsitakse: „Miks on Issand, meie Jumal, seda kõike meile teinud?„, siis vasta neile: ”Nõnda nagu te minu olete maha jätnud ja olete teeninud võõraid jumalaid oma maal, nõnda te peate teenima võõraid maal, mis ei ole teie oma!”
\par 20 Andke seda teada Jaakobi soole, kuulutage Juudale, öeldes:
\par 21 „Kuulge ometi seda, rumal ja südametu rahvas! Silmad teil on, aga te ei näe, kõrvad teil on, aga te ei kuule.
\par 22 Kas te ei karda mind, ütleb Issand, kas te ei vabise minu ees, kes olen pannud liiva merele igaveseks piiriks, millest see üle ei pääse? Kuigi ta lained kohisevad, ei suuda need midagi; kuigi need mühavad, ei pääse nad üle.
\par 23 Aga sellel rahval on tõrges ja vastupanija süda, nad on ära taganenud ja läinud.
\par 24 Nad ei ütle oma südames: „Kartkem ometi Issandat, oma Jumalat, kes annab vihma omal ajal, varajase ja hilise vihma, kes meile hoiab lõikuse nädalaid!”
\par 25 Teie süüteod pöörasid need ära ja teie patud hoidsid hea teist eemal.
\par 26 Sest minu rahva hulgas leidub õelaid: need luuravad kummargil nagu linnupüüdjad, seavad üles püüdepaelu, et püüda inimesi.
\par 27 Otsekui lindudega täidetud puurid on nende kojad täis kavalust. Seetõttu on nad saanud suureks ja rikkaks,
\par 28 on läinud lihavaks ja läikivaks. Nad astuvad üle iga piiri, ka kurjades tegudes: nad ei aja kohtuasju, vaeslapse kohtuasja, et seda lahendada, ja nad ei mõista õigust vaestele.
\par 29 Kas ma ei peaks neid sellepärast karistama? ütleb Issand. Kas mu hing ei peaks kätte tasuma niisugusele rahvale?
\par 30 Midagi kohutavat ja jäledat sünnib maal:
\par 31 prohvetid kuulutavad valet, preestrid õpetavad nendega käsikäes ja mu rahvas armastab seda nõnda. Aga mida te teete, kui sellele tuleb lõpp?

\chapter{6}

\par 1 Põgenege Jeruusalemmast, Benjamini lapsed! Puhuge sarve Tekoas ja pange tähis Beet-Keremisse, sest põhja poolt paistab õnnetus ja suur häving!
\par 2 Kena ja helliku olen ma hävitanud - Siioni tütre.
\par 3 Tema juurde tulevad karjased oma karjadega ja löövad telgid üles tema ümber, nad karjatavad igaüks oma osa peal.
\par 4 „Pühitsegem sõda tema vastu! Tõuskem ja mingem üles keskpäeva ajal!„ ”Häda meile, sest päev veereb ja õhtuvarjud pikenevad!”
\par 5 „Tõuskem ja mingem üles öösel ning hävitagem tema paleed!”
\par 6 Sest vägede Issand ütleb nõnda: Raiuge puid ja kuhjake piiramisvall Jeruusalemma vastu! See on linn, mida tuleb karistada, sest tema sees valitseb vägivald.
\par 7 Otsekui kaev hoiab värske oma vee, nii hoiab Jeruusalemm värske oma kurjuse; tema sees kuulukse ülekohtust ja rüüstamisest, minu palge ees on aina valu ja piin.
\par 8 Lase ennast hoiatada, Jeruusalemm, et mu hing ei võõrduks sinust, et ma ei teeks sind lagedaks, maaks, kus ei elata.
\par 9 Nõnda ütleb vägede Issand: Iisraeli jääki nopitakse otsekui viinapuu järelnoppimist. Siruta oma käsi võrsete kohale otsekui viinamarjakorjaja.
\par 10 Kellele ma peaksin rääkima ja kinnitama, et nad kuuleksid? Vaata, nende kõrvadel on eesnahk, nad ei saa kuulda. Vaata, Issanda sõna on neile teotuseks, see ei kõlba neile.
\par 11 Mina olen täis Issanda viha, ma ei suuda seda peatada. Vala see laste peale tänaval, samuti noorukite jõugu peale; sest kinni võetakse nii mehed kui naised, nii vanad kui raugad.
\par 12 Ja nende kojad antakse teistele koos põldude ja naistega; sest ma sirutan oma käe maa elanike vastu, ütleb Issand.
\par 13 Sest pisemast suuremani ahnitseb igaüks neist omakasu, ja prohvetist preestrini petavad kõik.
\par 14 Ja mu rahva vigastust ravivad nad pinnapealselt, öeldes: „Rahu, rahu!”, kuigi rahu ei ole.
\par 15 Kas nad häbenevad, et nad on teinud jäledust? Ei, nad ei häbene sugugi ega tunne piinlikkust. Seepärast nad langevad langejate hulgas; oma karistusajal nad komistavad, ütleb Issand.
\par 16 Nõnda ütles Issand: Seisatage teedel ja vaadake, küsige muistsete radade kohta, missugune on hea tee, ja käige sellel, siis leiate oma hingele hingamispaiga. Aga nemad ütlesid: „Seda me ei tee!”
\par 17 Ma seadsin vahimehed teie üle: „Kuulake sarvehäält!„ Aga nemad vastasid: ”Me ei kuula!”
\par 18 Seepärast kuulge, rahvad, ja mõista, kogudus, mis nendega juhtub!
\par 19 Kuule, maa! Vaata, ma saadan õnnetuse sellele rahvale, nende mõtete vilja; sest nad ei pannud tähele mu sõnu ja põlgasid mu Seadust.
\par 20 Milleks mulle viiruk, mis tuleb Seebast, ja head kalmused kaugelt maalt? Teie põletusohvrid ei ole mulle meelepärased ja teie tapaohvrid ei kõlba mulle.
\par 21 Seepärast ütleb Issand nõnda: Vaata, ma panen komistuskive sellele rahvale, ja nad komistavad nende otsa, isad ja pojad üheskoos, naaber hukkub koos naabriga.
\par 22 Nõnda ütleb Issand: Vaata, üks rahvas tuleb põhjamaalt, suur rahvas hakkab liikuma maa viimastest äärtest.
\par 23 Nad hoiavad käes ambu ja oda, nad on julmad ega tunne halastust; nende kisa on nagu mere kohin ja nad ratsutavad hobuste seljas; nad on varustatud nagu sõdurid tapluseks sinu vastu, Siioni tütar.
\par 24 Me oleme kuulnud neist sõnumeid, meie käed on lõtvunud; meid haarab ahastus nagu sünnitajat valu.
\par 25 Ärge minge väljale, ärge käige teed, sest seal on vaenlase mõõk - hirm on igal pool!
\par 26 Mu rahva tütar! Rõivastu kotiriidesse ja püherda tuhas, leina nagu ainsat poega, tõsta kibedat kaebust, sest äkitselt tuleb hävitaja meile kallale.
\par 27 Ma olen sinu pannud oma rahva proovijaks, kindlustatud linnaks, et sa õpiksid tundma ja katsuksid järele nende teed.
\par 28 Nad kõik on läinud väga ülekäte, laimu levitajad; nad on vask ja raud, nad kõik on hävitajad.
\par 29 Lõõts ähib, tina läks tules vedelaks, aga kõik sulatamine oli asjata - kurjad ei eraldunud.
\par 30 Neid hüütakse põlatud hõbedaks, sest Issand on nad põlanud.”

\chapter{7}

\par 1 Sõna, mis Jeremijale tuli Issandalt, kes ütles:
\par 2 „Seisa Issanda koja väravas ja kuuluta seal seda sõna ning ütle: Kuulge Issanda sõna, kogu Juuda, kes tulete sisse neist väravaist Issandat kummardama!
\par 3 Nõnda ütleb vägede Issand, Iisraeli Jumal: Parandage oma eluviise ja tegusid, siis ma jätan teid elama siia paika!
\par 4 Ärge lootke valesõnade peale, kui öeldakse: „See on Issanda tempel, Issanda tempel, Issanda tempel!”
\par 5 Aga kui te tõesti parandate oma eluviise ja tegusid, kui te tõesti teete õigust niihästi ühele kui teisele
\par 6 ega tee liiga võõrale, vaeslapsele ja lesknaisele, ega vala süütut verd siin paigas, ega järgne teistele jumalatele teile enestele õnnetuseks,
\par 7 siis ma jätan teid elama siia paika, maale, mille ma andsin teie vanemaile muistsest ajast igavesti.
\par 8 Aga vaata, te loodate valesõnade peale, millest ei ole kasu.
\par 9 Kas tahate varastada, tappa, abielu rikkuda, valet vanduda ja suitsutada Baalile ning käia teiste jumalate järel, keda te ei tunne,
\par 10 ja siis tulla ning seista minu ees selles kojas, millele on pandud minu nimi, ja öelda: „Me oleme päästetud!”, selleks et edasi teha kõiki neid jäledusi?
\par 11 Ons see koda, millele on pandud minu nimi, teie silmis röövlikoobas? Vaata, minagi näen seda seesugusena, ütleb Issand.
\par 12 Siis minge ometi minu asupaika, mis oli Siilos, kuhu ma esiti panin elama oma nime, ja vaadake, kuidas ma sellega talitasin oma Iisraeli rahva kurjuse pärast!
\par 13 Ja nüüd sellepärast, et te olete teinud kõiki neid tegusid, ütleb Issand, ja kuna ma teile aegsasti rääkisin ja rääkisin, aga teie ei kuulanud, ja kuna ma teid hüüdsin, aga teie ei vastanud,
\par 14 siis ma talitan kojaga, millele on pandud minu nimi, mille peale te loodate, ja paigaga, mille ma olen andnud teile ja teie vanemaile, nõnda nagu ma talitasin Siiloga,
\par 15 ja heidan teid ära oma palge eest, nõnda nagu ma heitsin ära kõik teie vennad, kogu Efraimi soo.
\par 16 Aga sina ära palu selle rahva eest ja ära tee nende pärast kisa ega palvet, ja ära käi mulle peale, sest ma ei kuule sind!
\par 17 Eks sa näe, mis nad teevad Juuda linnades ja Jeruusalemma tänavail?
\par 18 Lapsed korjavad puid, isad süütavad tule ja naised sõtkuvad tainast, et valmistada ohvrileibu Taevakuningannale. Ja nad valavad joogiohvreid teistele jumalatele, et mind teotada.
\par 19 Kas nad teotavad mind? küsib Issand. Eks nad teota iseendid oma häbiks?
\par 20 Seepärast ütleb Issand Jumal nõnda: Vaata, mu viha ja mu vihalõõm voolab selle paiga peale, inimeste ja loomade peale, välja puude ja maa vilja peale, ja see põleb kustutamatult.
\par 21 Nõnda ütleb vägede Issand, Iisraeli Jumal: Lisage oma põletusohvrid tapaohvritele ja sööge liha!
\par 22 Sest sel päeval, mil ma tõin nad ära Egiptusemaalt, ei rääkinud ma teie vanematega ega andnud neile käsku põletus- ja tapaohvri kohta,
\par 23 vaid ma andsin neile selle käsu, öeldes: Kuulake mu häält, siis ma olen teie Jumal ja teie olete minu rahvas; ja käige kõigiti seda teed, mida ma teid käsin, et teil oleks hea põli!
\par 24 Aga nemad ei kuulanud ega pööranud kõrva, vaid käisid oma nõu järgi oma kurja südame paadumuses, ja nad läksid tagasi, aga mitte edasi.
\par 25 Alates sellest päevast, kui teie vanemad lahkusid Egiptusemaalt, kuni tänapäevani olen ma läkitanud teie juurde kõik oma sulased prohvetid - läkitanud päevast päeva -,
\par 26 aga teie ei kuulanud mind ega pööranud kõrva, vaid jäite kangekaelseiks; te tegite rohkem paha kui teie vanemad.
\par 27 Ja kui sa neile kõike seda räägid, siis nad ei kuula sind, ja kui sa neid hüüad, siis nad ei vasta sulle.
\par 28 Seepärast ütle neile: See on rahvas, kes ei kuula Issanda, oma Jumala häält ega võta vastu hoiatust. Tõde on kadunud ja hävinud nende suust.
\par 29 Lõika oma juuksed ja viska need ära, ja alusta nutulaulu küngastel, sest Issand on põlanud ja hüljanud oma vihaaluse sugupõlve.
\par 30 Sest Juuda lapsed on kurja teinud minu silmis, ütleb Issand. Nad on asetanud oma jäledused kotta, millele on pandud minu nimi, ja on selle roojastanud.
\par 31 Ja nad on ehitanud Põletuspaiga ohvrikünkad, mis on Ben-Hinnomi orus, et põletada tules oma poegi ja tütreid, mida mina ei ole käskinud ja mis mulle ei ole meeldegi tulnud.
\par 32 Seepärast, vaata, päevad tulevad, ütleb Issand, mil enam ei öelda Põletuspaik ja Ben-Hinnomi org, vaid Tapaorg, ja Põletuspaika maetaksegi, sest muud paika ei ole.
\par 33 Ja selle rahva laibad saavad taeva lindude ja maa loomade roaks, aga peletajat ei ole.
\par 34 Ja ma lõpetan Juuda linnadest ja Jeruusalemma tänavailt lustihääle ja rõõmuhääle, peigmehe hääle ja pruudi hääle, sest maa muutub varemeiks.

\chapter{8}

\par 1 Sel ajal, ütleb Issand, võetakse Juuda kuningate luud ja tema vürstide luud, preestrite luud ja prohvetite luud ja Jeruusalemma elanike luud nende haudadest välja
\par 2 ja laotatakse päikese ja kuu ja kogu taevaväe ette, mida nad armastasid ja mida nad teenisid, mille järel nad käisid ja mille poole nad pöördusid ja mida nad kummardasid; neid ei korjata enam kokku ega maeta, need jäävad maa peale sõnnikuks.
\par 3 Ja surm on elust eelistatum kogu sellele jäägile, kes sellest halvast suguvõsast järele jääb kõigis paigus, kuhu ma pillutan nende jäägi, ütleb vägede Issand.
\par 4 Ja ütle neile: Nõnda ütleb Issand: Eks langenu taha üles tõusta? Eks eksinu taha pöörduda tagasi?
\par 5 Mispärast jääb see rahvas, Jeruusalemm, igavesti eksinuks? Nad hoiavad pettusest kinni, nad tõrguvad pöördumast.
\par 6 Ma olen tähele pannud ja kuulnud: nad ei räägi õigust. Keegi neist ei kahetse oma kurjust, et ta mõtleks: „Mis ma olen teinud!” Igaüks jookseb kiiresti oma teed nagu kihutav hobune lahingus.
\par 7 Isegi toonekurg taeva all teab oma seatud aega, turteltuvi, pääsuke ja rästas peavad kinni oma tulemisajast, aga minu rahvas ei tunne Issanda Seadust.
\par 8 Kuidas te võite öelda: „Me oleme targad ja meil on Issanda Seadus”? Tõesti, vaata, kirjatundjate valesulg on teinud selle valeks.
\par 9 Targad jäävad häbisse, nad ehmuvad ja nad tabatakse. Vaata, nad on põlanud Issanda sõna - mis tarkust võib neil olla?
\par 10 Sellepärast ma annan nende naised teistele, nende põllud vallutajaile. Sest kõik, niihästi väikesed kui suured, ahnitsevad kasu; kõik, niihästi prohvetid kui preestrid, petavad.
\par 11 Ja mu rahva, mu tütre vigastust ravivad nad pinnapealselt, öeldes: „Rahu, rahu!”, kuigi rahu ei ole.
\par 12 Kas nad häbenevad, et nad on teinud jäledust? Ei, nad ei häbene sugugi ega tunne piinlikkust. Seepärast nad langevad langejate hulgas, oma karistusajal nad komistavad, ütleb Issand.
\par 13 Ma nopin nad ära sootuks, ütleb Issand. Ei jää viinamarju viinapuule ega viigimarju viigipuule ja lehed närtsivad. Ja mis ma neile olen andnud, läheb neist mööda.
\par 14 „Miks me siin istume? Kogunegem ja mingem kindlustatud linnadesse ning vaikigem seal, sest Issand, meie Jumal, teeb meid vaikseks ja joodab mürgiveega, sest me oleme pattu teinud Issanda vastu.”
\par 15 Oodatakse rahu, aga head ei ole, paranemisaega, aga vaata, on kohkumus.
\par 16 Daanist kuuldakse tema hobuste nooskamist, tema täkkude hirnumisest väriseb kogu maa. Nad tulevad ja söövad maa koos kõigega, linna ja selle elanikud.
\par 17 Sest vaata, ma läkitan teie sekka madusid, mürkmadusid, kellesse ei mõju lausumine, ja need salvavad teid, ütleb Issand.
\par 18 Mu mure murrab mind, mu süda on haige!
\par 19 Vaata, mu tütre, mu rahva appihüüd kaugelt maalt: „Kas Issand ei ole Siionis või ei ole seal selle kuningas?” Miks nad mind on ärritanud oma nikerdatud kujudega, võõraste ebajumalatega?
\par 20 „Lõikus on lõppenud, suvi on möödas, aga meid ei ole päästetud!”
\par 21 Oma rahva, mu tütre vigastuse pärast olen ma murdunud; ma olen kurb, mind on haaranud hirm.
\par 22 Kas Gileadis ei ole palsamit või ei ole seal ravijat? Miks ei ole mu tütar, mu rahvas, siis terveks saanud?

\chapter{9}

\par 1 Kes annaks mulle kõrbes teekäijate öömaja? Siis ma jätaksin oma rahva ja läheksin ära nende juurest, sest nad kõik on abielurikkujate ja äraandjate jõuk.
\par 2 Nad pingutavad oma keelt nagu valede ambu, ei, mitte tõe pärast pole nad võimsad. Nad lähevad kurjusest kurjusse, aga mind nad ei tunne, ütleb Issand.
\par 3 Igaüks hoidugu oma ligimese eest ja ärgu lootku ühelegi oma vendadest! Sest iga vend kavaldab üle venda ja iga ligimene kannab teise peale keelt.
\par 4 Igaüks petab oma ligimest ja tõtt ei räägita; nad on harjutanud oma keele valetama, patustama, nad on võimetud pöörduma.
\par 5 Surve surve peale, pettus pettuse peale, nad tõrguvad mind tundmast, ütleb Issand.
\par 6 Seepärast ütleb vägede Issand nõnda: Vaata, ma sulatan ja proovin neid, sest mida muud saaksin ma teha oma rahva, mu tütre heaks?
\par 7 Tappev nool on nende keel, mis räägib valet: suuga räägitakse oma ligimesele rahust, südames aga varitsetakse teda.
\par 8 Kas ma selle kõige pärast ei peaks neid karistama? ütleb Issand. Või ei peaks mu hing kätte tasuma rahvale nagu too?
\par 9 Mägede pärast tõstan ma nuttu ja kaebust, ka kõrbe karjamaade pärast nutulaulu; sest need on nõnda hävinud, et ükski seal ei käi ja karja häält pole kuulda. Niihästi taeva linnud kui loomad on põgenenud ja ära läinud.
\par 10 Ma teen Jeruusalemma kivivaremeks, ðaakalite asupaigaks; ja Juuda linnad ma teen lagedaks, et ükski ei saa seal elada.
\par 11 Kes on tark mees ja mõistab seda ning kuulutab, mida Issanda suu on temale rääkinud: miks maa hukkub ja hävib kõrbe sarnaseks, kus ükski ei käi?
\par 12 Ja Issand ütles: Sellepärast et nad hülgasid mu Seaduse, mille ma neile andsin, ega kuulanud mu häält ega käinud selle järgi,
\par 13 vaid käisid oma südame paadumuses ja järgnesid baalidele, nagu nende vanemad olid neid õpetanud,
\par 14 seepärast ütleb vägede Issand, Iisraeli Jumal nõnda: Vaata, ma söödan neid, seda rahvast, koirohuga ja joodan neid mürgiveega.
\par 15 Ja ma pillutan nad rahvaste sekka, keda nemad ega nende vanemad ei ole tundnud, ja ma läkitan neile järele mõõga, kuni ma olen nad hävitanud.
\par 16 Nõnda ütleb vägede Issand: Saage aru ja kutsuge nutunaisi, et nad tuleksid; läkitage sõna tarkadele naistele, et nad tuleksid,
\par 17 ruttaksid ja alustaksid meie kohta kaebelugu, et meie silmist voolaks pisaraid ja meie laugudelt tilguks vett.
\par 18 Sest Siionist kostab kaebehääl: „Kuidas küll oleme hävitatud! Me oleme jäänud suurde häbisse, et pidime lahkuma maalt, et meie kodud lõhuti.”
\par 19 Kuulge siis, naised, Issanda sõna, ja teie kõrv võtku vastu kõne tema suust; õpetage oma tütreile kaebelugu ja üksteisele nutulaulu!
\par 20 Sest surm on tulnud sisse meie akendest, on tulnud meie paleedesse, ta niidab tänavailt lapsi, turgudelt noorukeid.
\par 21 Räägi: Nõnda ütleb Issand: Inimeste laibad langevad väljale nagu sõnnik, nagu loog niitja järel, mida keegi ei korista.
\par 22 Nõnda ütleb Issand: Ärgu kiidelgu tark oma tarkusest, ärgu kiidelgu vägev oma vägevusest, ärgu kiidelgu rikas oma rikkusest,
\par 23 vaid kes kiitleb, kiidelgu sellest, et ta on arukas ja tunneb mind, et mina olen Issand, kes teeb head, õigust ja õiglust maal. Sest seesugused asjad on mu meele järgi, ütleb Issand.
\par 24 Vaata, päevad tulevad, ütleb Issand, mil ma nuhtlen kõiki ümberlõigatuid, kellel siiski on eesnahk:
\par 25 Egiptust, Juudat, Edomit, ammonlasi, Moabit ja kõiki neid pöetudoimulisi, kes elavad kõrbes; sest kõik paganad on ümber lõikamata ja kogu Iisraeli sugu on ümberlõikamata südamega.

\chapter{10}

\par 1 Kuulge sõna, mida Issand teile räägib, Iisraeli sugu!
\par 2 Nõnda ütleb Issand: Ärge õppige paganate teid ja ärge kartke taeva märke, sest paganad kardavad neid.
\par 3 Kuid rahvaste kombed on tühisus: sest puu on raiutud metsast, see on puusepa kätetöö, kirvega tehtud;
\par 4 seda ehitakse hõbeda ja kullaga, kinnitatakse naelte ja haamritega, et see ei kõiguks.
\par 5 Aga nad jäävad linnupeletiste sarnaseks, kurgipõllul on need, ja nad ei räägi; neid peab kandma, sest nad ei kõnni. Ärge kartke neid, sest nad ei tee kurja, aga nad ei saa teha ka head!
\par 6 Ei ole sinu sarnast, Issand! Sina oled suur, ja suur on su nimi vägevuse poolest.
\par 7 Kes ei peaks sind kartma, rahvaste kuningas? Tõesti, seda sa väärid! Sest kõigi rahvaste tarkade hulgas ja kõigis nende kuningriikides ei ole sinu sarnast.
\par 8 Üheskoos on nad rumalad ja narrid. Ebajumalate õpetus: see on puu,
\par 9 õhukeseks taotud hõbe, toodud Tarsisest, ja Uufase kuld, puusepa ja kullassepa kätetöö. Nende riided on sinised ja purpurpunased, kõik meistrite töö.
\par 10 Aga Issand on tõeline Jumal, ta on elav Jumal ja igavene kuningas. Tema vihast väriseb maa ja tema sajatust ei suuda rahvad taluda.
\par 11 Öelge neile nõnda: Jumalad, kes ei ole teinud taevast ja maad, kaovad maa pealt ja taeva alt.
\par 12 Tema on oma rammuga rajanud maa, oma tarkusega loonud maailma ja mõistusega laotanud taeva.
\par 13 Kui tema teeb häält, siis on taevas vete kohin, ja ta kergitab pilved maa äärest; tema teeb vihmale välgud ja toob tuule välja selle aitadest.
\par 14 Inimesed on kõik rumalad, mõistusest ilma. Kõik kullassepad jäävad häbisse jumalakujude pärast; nende valatud kujud on pettus, sest neis pole vaimu.
\par 15 Need on tühised, naeruväärt tööd: oma katsumisajal nad hävivad.
\par 16 Nende sarnane ei ole see, kes on Jaakobi rikkuseks, sest tema on kõige Looja ja Iisrael on ta pärisosaks. Vägede Issand on tema nimi.
\par 17 Korja oma kompsud maast kokku, kitsikuses istuja!
\par 18 Sest nõnda ütleb Issand: Vaata, seekord ma lingutan minema maa elanikud ja rõhun neid nõnda, et nad tunnevad.
\par 19 Häda mulle mu vigastuse pärast! Mu haav on ravimatu! Mina aga mõtlesin: See on ju ainult nõrkus, mida ma suudan taluda.
\par 20 Mu telk on hävitatud ja kõik mu telginöörid katki kistud; mu lapsed on läinud mu juurest ja neid ei ole enam. Ei ole ühtegi, kes jälle püstitaks mu telgi ja seaks üles mu telgiriided.
\par 21 Jah, karjased on läinud arust ära ega otsi enam Issandat: seetõttu tegutsevad nad vääralt ja kõik nende karjad on pillutatud.
\par 22 Hääl kostab. Vaata, see tuleb - suur mürin põhjamaalt, tegema Juuda linnu lagedaks, ðaakalite asupaigaks!
\par 23 Ma tean, Issand, et inimese tee ei olene temast enesest, ei ole ränduri käes juhtida oma sammu.
\par 24 Karista mind, Issand, aga õiglaselt, mitte oma raevus, et sa mind täiesti ei hävitaks.
\par 25 Vala oma vihalõõm paganate peale, kes sind ei tunne, ja suguvõsade peale, kes ei hüüa appi sinu nime. Sest nad on neelanud Jaakobi, jah, on neelanud ta ära ja teinud talle lõpu ning hävitanud ta eluaseme.”

\chapter{11}

\par 1 Sõna, mis Jeremijale tuli Issandalt, kes ütles:
\par 2 „Kuulge selle seaduse sõnu ja rääkige Juuda meestele ja Jeruusalemma rahvale,
\par 3 ütle neile: Nõnda ütleb Issand, Iisraeli Jumal: Neetud olgu mees, kes ei kuule selle seaduse sõnu,
\par 4 mille ma andsin teie vanemaile sel päeval, mil ma tõin nad ära Egiptusemaalt rauasulatusahjust, öeldes: Kuulake mu häält ja tehke kõigiti nõnda, nagu ma teid käsin, siis te olete minu rahvas ja mina olen teie Jumal,
\par 5 et kinnitada vannet, mille ma vandusin teie vanemaile: anda neile maa, mis voolab piima ja mett, nagu see tänapäeval ongi!„ Ja mina vastasin ning ütlesin: „Aamen, Issand!”
\par 6 Ja Issand ütles mulle: „Kuuluta kõiki neid sõnu Juuda linnades ja Jeruusalemma tänavail ning ütle: Kuulge selle seaduse sõnu ja tehke nende järgi!
\par 7 Sest ma hoiatasin tõsiselt teie vanemaid sel päeval, kui ma tõin nad ära Egiptusemaalt, hoiatasin järelejätmatult kuni tänase päevani, öeldes: Kuulake minu häält!
\par 8 Aga nad ei kuulanud ega pööranud kõrva, vaid käisid igaüks oma kurja südame paadumuses; seepärast lasksin ma tulla nende peale kõik selle seaduse sõnad, mida ma olin käskinud täita, aga mida nad ei täitnud.”
\par 9 Ja Issand ütles mulle: „Juuda meeste ja Jeruusalemma rahva hulgas on avastatud vandenõu.
\par 10 Nad on pöördunud tagasi oma esiisade süütegude juurde, kes tõrkusid kuulamast mu sõna, ja nad on käinud teiste jumalate järel neid teenides; Iisraeli sugu ja Juuda sugu on tühistanud mu seaduse, mille ma olin andnud nende vanemaile.
\par 11 Seepärast ütleb Issand nõnda: Vaata, ma toon neile õnnetuse, millest neil pole pääsu; ja kui nad kisendavad minu poole, siis ma ei kuule neid.
\par 12 Kui siis Juuda linnad ja Jeruusalemma elanikud lähevad ja kisendavad jumalate poole, kellele nad on suitsutanud, siis need küll ei suuda neid päästa nende õnnetuse ajal.
\par 13 Sest nõnda palju kui sul on linnu, on sul jumalaid, Juuda! Ja nõnda palju kui Jeruusalemmas on tänavaid, olete te püstitanud altareid häbile, suitsutusaltareid Baalile!
\par 14 Sina ära palu selle rahva eest, ära tee nende pärast kisa ega palvet, sest mina ei kuule, kui nad hüüavad mind oma õnnetuse pärast appi!
\par 15 Mida teeb mu armsam minu kojas? Sepitseb riukaid? Kas tõotused ja pühitsetud liha võiksid ära viia su süü, et sa nende läbi pääseksid?
\par 16 Haljaks õlipuuks, ilusaks, kauniviljaliseks nimetas sind Issand; aga raginal süütab ta tule selle külge ja selle oksad muutuvad kõlbmatuks.
\par 17 Ja vägede Issand, kes sind on istutanud, rääkis sinule tulevasest õnnetusest Iisraeli soo ja Juuda soo häbitegude pärast, mida nad on teinud, et Baalile suitsutades mind ärritada.”
\par 18 Ja Issand ilmutas mulle ning ma sain teada; siis näitasid sa mulle nende tegusid.
\par 19 Mina olin nagu süütu talleke, keda viiakse tappa, ega teadnud, et nad mu vastu mõtteid mõlgutasid: „Hävitagem puu koos ta viljaga ja raiugem ta ära elavate maalt, et ta nimegi enam ei meenutataks!”
\par 20 Aga vägede Issand on õiglane kohtumõistja, kes katsub läbi neerud ja südame. Lase mind näha, et sa neile kätte tasud, sest ma olen oma riiuasja sinule avaldanud!
\par 21 Seepärast ütleb Issand nõnda Anatoti meeste kohta, kes püüavad mu hinge ja ütlevad: „Sina ära ennusta Issanda nimel, et sa ei sureks meie käe läbi!”,
\par 22 seepärast ütleb vägede Issand nõnda: Vaata, ma nuhtlen neid: noored mehed surevad mõõga läbi, nende pojad ja tütred surevad nälga.
\par 23 Neist ei jää järele jääkigi, sest ma toon Anatoti meestele nende karistusaastal õnnetuse.

\chapter{12}

\par 1 „Sina jääd õiglaseks, Issand, kuigi ma riidlen sinuga. Tõesti, ma tahaksin sinuga rääkida õigusemõistmisest. Miks läheb korda õelate tee? Miks kõik petised elavad rahus?
\par 2 Sina istutad neid, nad juurduvad, nad kasvavad, kannavad isegi vilja. Nende suus oled sa ligidal, aga nende neerudest kaugel.
\par 3 Kuid sina, Issand, tunned mind, näed mind ja katsud läbi mu südame, missugune see on sinu ees. Lahuta nad nagu tapalambad ja pühenda veristuspäevaks!
\par 4 Kui kaua peab maa leinama ja kõigi väljade rohi kuivama? Nende kurjuse pärast, kes seal elavad, hukkuvad loomad ja linnud, sest nad ütlevad: „Tema ei näe meie lõppu.”
\par 5 „Kui sa väsid jalameestega koos joostes, kuidas sa võiksid siis võistelda hobustega? Rahulikul maal võid sa olla julge, aga mida sa teeksid Jordani padrikus?
\par 6 Sest isegi su vennad ja su isa pere, needki petavad sind, needki karjuvad täiest kõrist sulle järele. Ära usu neid, isegi kui nad sinuga lahkesti räägivad!”
\par 7 „Ma olen maha jätnud oma koja, hüljanud oma pärisosa, olen andnud, mis oli mu hingele armas, oma vaenlaste kätte.
\par 8 Mu pärisosa oli mulle kui lõvi metsas: ta möirgas mu peale, seepärast ma vihkan teda.
\par 9 Mu pärisosa oli mulle kirjuks linnuks, mida röövlinnud piiravad. Minge, koguge kõik metsloomad, tooge nad sööma!
\par 10 Hulk karjaseid on hävitanud mu viinamäe, on tallanud mu põlluosa, nad on teinud mu armsa põllu lagedaks kõrbeks.
\par 11 Lagedaks on see tehtud, lagedana leinab see mu ees. Kogu maa on laastatud, aga ei ole ühtegi, kes võtaks seda südamesse.
\par 12 Kõigi kõrbe tühermaade peale tulid hävitajad - sest Issanda mõõk neelab maa äärest ääreni - rahu pole kellelgi.
\par 13 Nad külvasid nisu, aga lõikasid kibuvitsu, nad väsitasid endid asjata; häbenege oma saaki Issanda vihalõõma pärast!”
\par 14 Nõnda ütleb Issand kõigi mu kurjade naabrite kohta, kes puudutavad pärisosa, mille ma oma rahvale Iisraelile olen andnud pärida: „Vaata, ma kitkun nad ära nende maalt ja kitkun Juuda soo välja nende keskelt.
\par 15 Aga pärast seda, kui ma olen nad kitkunud, halastan ma jälle nende peale ja toon nad tagasi, igaühe ta pärisosale ja igaühe ta maale.
\par 16 Ja kui nad tõesti õpivad ära mu rahva kombed ja hakkavad vanduma minu nime juures: Nii tõesti kui Issand elab!, nõnda nagu nad õpetasid mu rahvast vanduma Baali juures, siis nad võivad edasi elada mu rahva keskel.
\par 17 Aga kui nad ei kuule, siis ma kitkun selle rahva sootuks ära ja saadan hukatusse, ütleb Issand.”

\chapter{13}

\par 1 Issand ütles mulle nõnda: „Mine ja osta enesele linane vöö ja pane see niuete ümber, aga ära vii seda vette!”
\par 2 Ja ma ostsin Issanda sõna peale vöö ning panin selle enesele niuete ümber.
\par 3 Siis tuli Issanda sõna mulle teist korda; ta ütles:
\par 4 „Võta vöö, mille sa ostsid, mis sul niuete ümber on, ja võta kätte ning mine Frati jõe äärde ja peida see seal kaljulõhesse!”
\par 5 Ja ma läksin ning peitsin selle Frati jõe äärde, nagu Issand mind oli käskinud.
\par 6 Ja kui hea tükk aega oli möödunud, ütles Issand mulle: „Võta kätte, mine Frati jõe äärde ja võta sealt vöö, mille ma sind käskisin sinna peita!”
\par 7 Siis ma läksin Frati jõe äärde ja kaevasin ning võtsin vöö paigast, kuhu ma selle olin peitnud, ja vaata, vöö oli rikutud, see ei kõlvanud kuhugi.
\par 8 Ja mulle tuli Issanda sõna; ta ütles:
\par 9 „Nõnda ütleb Issand: Selsamal kombel ma hävitan Juuda kõrkuse ja Jeruusalemma kõrkuse, mis on suur.
\par 10 See paha rahvas, kes ei taha mu sõna kuulda, kes elab oma südame paadumuses ja käib teiste jumalate järel, et neid teenida ja kummardada, saab selle vöö sarnaseks, mis ei kõlba kuhugi.
\par 11 Sest nagu vöö kinnitub mehe niudeil, nõnda ma kinnitasin enesele kogu Iisraeli soo ja kogu Juuda soo, ütleb Issand, et nad oleksid mulle rahvaks, kuulsuseks, kiituseks ja ehteks; aga nad ei kuulanud.
\par 12 Seepärast ütle neile see sõna: Nõnda ütleb Issand, Iisraeli Jumal: Iga kruus täidetakse veiniga. Ja kui nad sulle ütlevad: Kas me tõesti ise ei tea, et iga kruus täidetakse veiniga?,
\par 13 siis vasta neile: Nõnda ütleb Issand: Vaata, ma täidan joomauimaga kõik selle maa elanikud, kuningad, kes istuvad Taaveti aujärjel, preestrid, prohvetid ja kõik Jeruusalemma elanikud,
\par 14 ja ma purustan nad üksteise vastu, isad ja pojad üheskoos, ütleb Issand. Mina ei tunne kaasa, ei kurvasta ega halasta, nõnda et jätaksin nad hävitamata.
\par 15 Kuulge ja kuulatage, ärge suurustage, sest Issand on rääkinud!
\par 16 Andke au Issandale, oma Jumalale, enne kui saabub pimedus ja enne kui teie jalad hämaruses puutuvad vastu mägesid! Te ootate küll valgust, aga tema teeb selle surmavarjuks, muudab pilkaseks pimeduseks.
\par 17 Aga kui te ei võta seda kuulda, siis mu hing nutab salajas teie kõrkuse pärast: nutab kibedasti ja mu silmist voolavad pisarad, et Issanda kari viiakse vangi.
\par 18 Ütle kuningale ja kuningannale: „Istuge alamale, sest teie tore kroon on langenud teil peast!”
\par 19 Lõunamaa linnad on suletud ja avajat ei ole. Kogu Juuda viiakse vangi, viiakse vangi tervenisti.
\par 20 Tõstke oma silmad üles ja vaadake, kuidas nad tulevad põhja poolt! Kus on kari, mis sulle oli antud, su ilusad kitsed ja lambad?
\par 21 Mis sa siis ütled, kui ta paneb sulle valitsejaiks need, keda sa ise harjutasid enesele armukesteks? Küllap sind valdavad valud otsekui sünnitajat naist.
\par 22 Ja kui sa mõtled oma südames: „Mispärast see juhtub mulle?”, siis sinu suure süü pärast tõstetakse üles su hõlmad, su kannad saavad tunda vägivalda.
\par 23 Kas etiooplane saab muuta oma nahka või panter oma täppe? Sel juhul saaksite head teha ka teie, kes olete harjunud kurja tegema.
\par 24 Ma pillutan teid nagu lendlevaid kõrsi kõrbe tuules.
\par 25 See on sinu liisk, sinu mõõdetud osa minult, ütleb Issand, sest sa oled unustanud minu ja oled lootnud vale peale.
\par 26 Nõnda minagi tõstan üles su hõlmad su näo peale, et nähtaks su häbi,
\par 27 su abielurikkumist ja su hirnumist, su häbiväärset kõlvatust. Välja küngastel olen ma näinud su jäledusi. Häda sulle, Jeruusalemm! Ei sina saa puhtaks! Kui kaua see veel kestab?”

\chapter{14}

\par 1 Issanda sõna, mis Jeremijale tuli ühenduses põuaga:
\par 2 „Juuda leinab, ta väravad jäävad tühjaks, leinates lamatakse maas, Jeruusalemmast tõuseb hädakisa.
\par 3 Nende suurnikud läkitavad oma alamaid vee järele; need lähevad kaevude juurde, aga ei leia vett - nende astjad tulevad tühjalt tagasi. Neil on häbi ja nad tunnevad piinlikkust ning katavad oma pead.
\par 4 Sest põld on kohkunud, et maal ei ole vihma, põllumehed häbenevad ja katavad oma pead.
\par 5 Koguni poeginud emahirv väljal jätab maha oma vasika, sest värsket rohtu ei ole.
\par 6 Ja metseeslid seisavad tühermail ja ahmivad õhku nagu ðaakalid; nende silmad tuhmuvad, sest rohtu ei ole.
\par 7 Kui meie süüteod tunnistavad meie vastu, siis, Issand, toimi meiega oma nime kohaselt! Sest meie taganemisi on palju, me oleme pattu teinud sinu vastu.
\par 8 Sina, Iisraeli lootus, tema päästja hädaajal. Mispärast sa oled nagu võõras maal või nagu rändur, kes lööb telgi üles ainult ööbimiseks?
\par 9 Mispärast sa oled nagu jahmunud mees, nagu keegi, kes on võimetu päästma? Ja ometi oled sa meie keskel, Issand! Meile on pandud sinu nimi, ära jäta meid maha!
\par 10 Nõnda ütleb Issand selle rahva kohta: Nii on siis neile meeldinud hulkuda, nad ei ole säästnud oma jalgu; seepärast ei ole Issandal neist hea meel, ta meenutab nüüd nende süüd ja nuhtleb nende patte.
\par 11 Ja Issand ütles mulle: Ära palu selle rahva eest tema heaks!
\par 12 Kuigi nad paastuvad, ei kuule ma nende hüüdu; ja kuigi nad ohverdavad põletusohvreid ja roaohvreid, ei ole mul neist hea meel, vaid ma teen neile lõpu mõõga, nälja ja katkuga.
\par 13 Siis ma ütlesin: Oh, Issand Jumal! Vaata, prohvetid ütlevad neile: Te ei näe mõõka ega tule teile nälga, sest mina annan teile siin paigas püsiva rahu.
\par 14 Aga Issand ütles mulle: Prohvetid kuulutavad minu nimel valet. Mina ei ole neid läkitanud, ei ole neid käskinud ega nendega rääkinud; nad kuulutavad teile valenägemusi, tühiseid ennustusi ja oma südame pettekujutlusi.
\par 15 Seepärast ütleb Issand prohvetite kohta, kes kuulutavad minu nimel, kuigi ma ei ole neid läkitanud, ja kes ütlevad, et sellele maale ei tule mõõka ja nälga, nõnda: Mõõga ja nälja läbi hukkuvad need prohvetid.
\par 16 Ja rahvas, kellele nad kuulutavad, heidetakse Jeruusalemma tänavaile nälja ja mõõga läbi ning ükski ei mata neid - nemad, nende naised, nende pojad ja tütred. Ja mina valan nende peale nende eneste kurjuse.
\par 17 Ja ütle neile see sõna: Minu silmist voolavad pisarad ööd ja päevad ega lakka, sest neitsi, mu rahva tütar, on vigastatud raske vigastusega, tema haav on ravimatu.
\par 18 Kui ma lähen väljale, siis vaata: mõõgaga tapetud. Ja kui ma tulen linna, siis vaata: näljatõved. Aga niihästi prohvetid kui preestrid uitavad maal ega tea midagi.
\par 19 Kas sa oled Juuda hoopis hüljanud? Või on su hing Siionist tüdinud? Mispärast sa oled meid nõnda löönud, et me ei saa terveks? Oodatakse rahu, aga head ei ole, ja paranemise aega, ent vaata: on kohkumus.
\par 20 Issand, me tunneme oma õelust ja oma vanemate süüd, ja et me oleme pattu teinud sinu vastu.
\par 21 Oma nime pärast ära põlga meid, ära teota oma aujärge! Pea meeles, ära tee tühjaks oma lepingut meiega!
\par 22 Kas paganate ebajumalate seas on vihmategijaid või annab taevas piisku iseenesest? Eks see ole sina, Issand, meie Jumal? Me loodame sinu peale, sest sina oled kõik selle teinud.”

\chapter{15}

\par 1 Ja Issand ütles mulle: Isegi kui Mooses ja Saamuel seisaksid minu ees, ei oleks mu hing siiski mitte selle rahva poolt. Aja nad minu eest, et nad läheksid ära.
\par 2 Ja kui nad sinult küsivad: „Kuhu me peame minema?”, siis vasta neile: Issand ütleb nõnda: Kes surma, see surma, kes mõõga ette, see mõõga ette, kes nälja kätte, see nälja kätte, kes vangi, see vangi.
\par 3 Ja ma panen neile peale nelja liiki nuhtlusi, ütleb Issand: mõõga tapma, koerad lohistama, taeva linnud ja maa loomad sööma ja hävitama.
\par 4 Ja ma panen nad hirmutuseks kõigile kuningriikidele maa peal Juuda kuninga Manasse, Hiskija poja pärast, selle pärast, mis ta Jeruusalemmas tegi.
\par 5 Sest kes annaks sulle armu, Jeruusalemm, või kes trööstiks sind? Või kes küll põikaks kõrvale, et küsida su käekäigu järele?
\par 6 Sina oled mu hüljanud, ütleb Issand, ja oled taganenud; seepärast ma sirutan oma käe su vastu ja hävitan sinu, ma olen tüdinud järele andmast.
\par 7 Mina viskan neid visklabidaga maa väravais; mina teen lastetuks, mina hukkan oma rahva, kui nad ei pöördu oma teedelt.
\par 8 Mul on siis nende lesknaisi rohkem kui liiva mere ääres; mina toon noorukite ema kallale päise päeva ajal rüüstaja, paiskan äkitselt ta peale ärevust ja hirmu.
\par 9 See, kes sünnitas seitse, närbub, vaagub hinge; tema päike loojub, kui alles on päev, ta häbeneb ja jääb häbisse. Ja mis neist üle jääb, selle ma annan mõõga kätte, nende vaenlaste ees, ütleb Issand.
\par 10 „Häda mulle, ema, et sa sünnitasid minu, riiaka ja tülika mehe kogu maale! Ei ole ma võlausaldaja ega võlgnik, ometi sajatavad mind kõik.
\par 11 Issand ütles: Küll ma vabastan su hea jaoks, küll ma panen vaenlase anuma sind õnnetuse ajal ja häda ajal.”
\par 12 „Kas murdub raud, põhjamaine raud ja vask?
\par 13 Sinu varanduse ja tagavarad annan ma ilma hinnata saagiks kõigi su pattude pärast ja kõigis su paigus.
\par 14 Ma panen sind teenima su vaenlasi maal, mida sa ei tunne, sest mu vihatuli on süttinud - see põleb teie kohal.”
\par 15 „Sina tead seda. Issand, pea mind meeles ja kanna hoolt minu eest, tasu kätte minu eest mu tagakiusajaile! Oma suures kannatlikkuses ära võta mind ära, mõtle, et ma sinu pärast kannan teotust!
\par 16 Kui leidus su sõnu, siis ma neelasin neid ja su sõna oli mulle lustiks ja südamerõõmuks, sest mulle on pandud sinu nimi, Issand, vägede Jumal.
\par 17 Ei ole ma istunud naerjate killas ega ole ma ilutsenud; sinu käe pärast olen ma istunud üksinda, sest sa oled mu täitnud oma sajatusega.
\par 18 Mispärast on mu valu nii püsiv ja mu haav ravimatu? See ei taha paraneda. Oh häda! Sa oled mulle nagu kahanev oja, nagu vesi, mis ei jää püsima.”
\par 19 Selle peale ütleb Issand nõnda: „Kui sa pöördud, siis ma lasen sind jälle seista mu palge ees. Ja kui sa lahutad väärtusliku tühisest, siis sa oled otsekui minu suu. Nemad pöörduvad siis sinu poole, aga sina ei pöördu nende poole.
\par 20 Ja mina teen sind kindlaks vaskmüüriks selle rahva vastu; kui nad võitlevad sinu vastu, siis nad ei võida sind, sest mina olen sinuga, et sind aidata ja päästa, ütleb Issand.
\par 21 Ja ma päästan sind kurjade käest ning lunastan vägivallavalitsejate pihust.”

\chapter{16}

\par 1 Ja mulle tuli Issanda sõna; ta ütles:
\par 2 „Ära võta enesele naist ja ärgu olgu sul siin paigas poegi ega tütreid!
\par 3 Sest nõnda ütleb Issand poegade ja tütarde kohta, kes sünnivad siin paigas, ja nende emade kohta, kes neid sünnitavad, ja nende isade kohta, kes neid sigitavad siin maal:
\par 4 Nad surevad surmahaigustesse, neid ei leinata ja neid ei maeta, nad jäävad maale sõnnikuks. Nad hukkuvad mõõga ja nälja läbi ning nende laibad saavad taeva lindude ja maa loomade roaks.
\par 5 Sest Issand ütleb nõnda: Ära mine leinakotta, ära käi kurtmas ja ära tunne neile kaasa, sest mina võtan sellelt rahvalt oma rahu, armu ja halastuse, ütleb Issand.
\par 6 Ja nad surevad, niihästi suured kui väikesed siin maal, neid ei maeta ega peeta neile leinakaebust, ja ükski ei lõika enesele märke ega aja pead paljaks nende pärast.
\par 7 Ei murta neile leinaleiba lohutuseks surnu pärast ega joodeta neid troostikarikast nende isa ja ema pärast.
\par 8 Ära mine peomajja koos nendega istuma, sööma ja jooma!
\par 9 Sest nõnda ütleb vägede Issand, Iisraeli Jumal: Vaata, ma lõpetan siit paigast teie silme ees ja teie päevil lustihääle ja rõõmuhääle, peigmehe hääle ja pruudi hääle.
\par 10 Ja kui sa kuulutad sellele rahvale kõiki neid sõnu ja nemad küsivad sinult: „Mispärast räägib Issand meile kogu sellest suurest õnnetusest ja mis on meie süü ja patt, mida me oleme teinud Issanda, oma Jumala vastu?”,
\par 11 siis vasta neile: Sellepärast et teie vanemad jätsid minu maha, ütleb Issand, ja läksid teiste jumalate järele ja teenisid ning kummardasid neid, jätsid minu maha ega pidanud minu Seadust.
\par 12 Teie aga olete teinud halvemini kui teie vanemad, sest vaata, te käite igaüks oma kurja südame paadumuses mind kuulda võtmata.
\par 13 Sellepärast ma paiskan teid sellelt maalt maale, mis on tundmatu teile ja teie vanemaile, ja te teenite seal teisi jumalaid päevad ja ööd, sest mina ei anna teile armu.
\par 14 Ometi, vaata, päevad tulevad, ütleb Issand, kui enam ei öelda: „Nii tõesti kui elab Issand, kes tõi Iisraeli lapsed ära Egiptusemaalt”,
\par 15 vaid: „Nii tõesti kui elab Issand, kes tõi Iisraeli lapsed ära põhjamaalt ja kõigist maadest, kuhu ta oli nad pillutanud.” Sest ma toon nad tagasi nende maale, mille ma olen andnud nende vanemaile.
\par 16 Vaata, ma läkitan paljude kalameeste järele, ütleb Issand, et nood püüaksid neid; ja pärast seda ma läkitan paljude küttide järele, et nood kütiksid neid kõigilt mägedelt ja kõigilt künkailt ning kaljulõhedest.
\par 17 Sest minu silmad on kõigi nende teede peal, nad ei ole minu eest varjul ja nende süü ei ole peidetud mu silme eest.
\par 18 Aga esmalt tasun ma kahekordselt nende süü ja patu, sellepärast et nad on teotanud minu maad oma põlastusväärsete ebajumalate laipadega ja on täitnud mu pärisosa oma jäledustega.
\par 19 Issand, mu tugevus ja varjupaik, mu pelgupaik hädaajal. Sinu juurde tulevad rahvad maa äärtest ja ütlevad: „Meie vanemail oli pärisosaks ainult vale, tühised ebajumalad, ja nende hulgas ei olnud ühtegi, kellest oleks olnud kasu.”
\par 20 Kas saab inimene valmistada enesele jumalaid? Need ju ei olegi jumalad!
\par 21 Seepärast, vaata, nad saavad tunda, seekord ma annan neile tunda oma kätt ja oma väge, et nad teaksid, et minu nimi on Issand.

\chapter{17}

\par 1 Juuda patt on kirjutatud raudsulega, uurendatud teemanditeravikuga nende südamelauasse ja nende altarisarvede peale.
\par 2 Nõnda nagu nad mäletavad oma lapsi, mäletavad nad oma altareid ja viljakustulpi haljaste puude juures, kõrgetel küngastel.
\par 3 Sinu varanduse, kõik su tagavarad, su ohvrikünkad mägedel ja väljal annan ma riisutavaiks patu pärast kogu su maal.
\par 4 Sa pead oma käe lahti laskma oma pärisosa küljest, mille ma sulle andsin, ja ma panen sind teenima su vaenlasi maal, mida sa ei tunne; sest te olete süüdanud mu vihatule, mis jääb igavesti põlema.
\par 5 Nõnda ütleb Issand: Neetud on mees, kes loodab inimeste peale, kes peab liha oma käsivarreks ja kelle süda lahkub Issandast.
\par 6 Tema on otsekui kadakas nõmmel, mis ei näe head tulemas; ta asub kõrbe kivirägas, soolasel, elamiskõlbmatul maal.
\par 7 Aga õnnistatud on mees, kes loodab Issanda peale, kelle lootuseks on Issand.
\par 8 Tema on otsekui vee äärde istutatud puu, mis ajab oma juuri oja kaldal ega karda, kui palavus tuleb, vaid ta lehed on haljad; ja põua-aastal ta ei muretse ega lakka vilja kandmast.
\par 9 Süda on petlikum kui kõik muu ja äärmiselt rikutud - kes suudab seda mõista?
\par 10 Mina, Issand, uurin südant, katsun läbi neerud, et anda igaühele tema tee kohaselt, tema tegude vilja mööda.
\par 11 Nagu põldpüü, kes haub, mida ta ei ole munenud, on see, kes kogub ülekohtuselt rikkust; pooles eas peab ta selle jätma ja rumalana lõpetab ta oma elu.
\par 12 Aujärg, kõrge algusest peale, on meie pühamu paik.
\par 13 Issand, Iisraeli lootus! Kõik, kes sinu hülgavad, jäävad häbisse. Kes minust taganevad, kirjutatakse põrmu, sest nad on jätnud maha elava vee allika, Issanda.
\par 14 Tee mind terveks, Issand, siis ma saan terveks; aita mind, siis ma saan abi, sest sina oled mu kiitus.
\par 15 Vaata, nad ütlevad mulle: „Kus on Issanda sõna? Mingu see ometi täide!”
\par 16 Ma ei ole põiganud järgnemast sulle, karjasele, ega ole soovinud õnnetusepäeva. Sa tead, et mis mu huultelt on välja tulnud, see on sinu palge ees.
\par 17 Ära ole mulle hirmutuseks, sina, mu varjupaik kurjal ajal!
\par 18 Jäägu häbisse mu jälitajad, aga mind ära lase jääda häbisse; tundku nad hirmu, aga mind ära lase hirmu tunda; too nende peale õnnetusepäev ja purusta nad kahekordselt!
\par 19 Issand ütles mulle nõnda: Mine ja seisa rahva poegade väravas, millest Juuda kuningad käivad sisse ja välja, ja kõigis Jeruusalemma väravais,
\par 20 ja ütle neile: Kuulge Issanda sõna, Juuda kuningad ja kogu Juuda ja kõik Jeruusalemma elanikud, kes tulete sisse neist väravaist!
\par 21 Issand ütleb nõnda: Vaadake, et te ei kanna hingamispäeval koormat ega too seda sisse Jeruusalemma väravaist!
\par 22 Ärge viige hingamispäeval koormat välja oma kodadest ja ärge tehke ühtegi tööd, vaid pühitsege hingamispäeva nõnda, nagu ma olen käskinud teie vanemaid!
\par 23 Nemad aga ei kuulanud ega pööranud oma kõrva, vaid tegid oma kaela kangeks, et mitte kuulda ja õpetust võtta.
\par 24 Aga kui te tõesti kuulate mind, ütleb Issand, ega too hingamispäeval koormat sisse selle linna väravaist, vaid pühitsete hingamispäeva ega tee ühtegi tööd,
\par 25 siis tulevad selle linna väravaist sisse kuningad ja vürstid, kes istuvad Taaveti aujärjele, sõites vankreis ja hobuste seljas, nemad ja nende vürstid, Juuda mehed ja Jeruusalemma elanikud, ja selles linnas elatakse igavesti.
\par 26 Ja Juuda linnadest ja Jeruusalemma ümbrusest, Benjamini maalt, madalmaalt, mäestikust ja Lõunamaalt tulevad põletus-, tapa- ja roaohvrite ning viiruki toojad, tänuohvri toojad Issanda kotta.
\par 27 Aga kui te ei võta mind kuulda, et te peate pühitsema hingamispäeva ja et te ei tohi kanda koormat ega tulla hingamispäeval sisse Jeruusalemma väravaist, siis ma süütan selle väravais tule, mis neelab Jeruusalemma paleed ega kustu.

\chapter{18}

\par 1 Sõna, mis Jeremijale tuli Issandalt, kes ütles:
\par 2 „Tõuse ja mine alla potissepa kotta, ja ma annan seal sulle kuulda oma sõnu!”
\par 3 Siis ma läksin alla potissepa kotta, ja vaata, ta tegi tööd potikedra juures.
\par 4 Ja kui astja, mida ta savist tegi, potissepa käes ebaõnnestus, siis ta tegi sellest teise astja, nagu potissepa silmis õigem näis olevat teha.
\par 5 Siis tuli mulle Issanda sõna; ta ütles:
\par 6 „Kas mina ei või teiega teha nõnda nagu see potissepp, oh Iisraeli sugu! ütleb Issand. Vaata, otsekui savi potissepa käes, nõnda olete teie minu käes, Iisraeli sugu!
\par 7 Kord ma räägin rahva või kuningriigi kohta, et teda kitkutakse, kistakse ja hävitatakse:
\par 8 aga kui see rahvas, kelle kohta ma rääkisin, pöördub oma kurjusest, siis ma kahetsen kurja, mida ma kavatsesin temale teha.
\par 9 Teine kord ma räägin rahva või kuningriigi kohta, et teda ehitatakse ja istutatakse:
\par 10 aga kui ta teeb kurja minu silmis ega kuula mu häält, siis ma kahetsen head, mida ma lubasin talle teha.
\par 11 Ja nüüd räägi ometi Juuda meestele ja Jeruusalemma elanikele ning ütle: Nõnda ütleb Issand: Vaata, mina valmistan teile õnnetuse ja pean nõu teie vastu. Pöördugu seepärast igaüks oma kurjalt teelt ja parandagu oma viise ja tegusid!
\par 12 Aga nad vastavad: Asjata! Sest me käime oma nõu järgi ja talitame igaüks oma kurja südame paadumuses.
\par 13 Seepärast ütleb Issand nõnda: Küsige ometi rahvaste seas, kes on kuulnud sellesarnast? Väga hirmsaid tegusid on teinud Iisraeli neitsi.
\par 14 Kas lahkub Ðaddai kaljult Liibanoni lumi? Või kuivavad veed, võõrad, külmad, voolavad?
\par 15 Kuid mu rahvas on minu unustanud, nad suitsutavad ebajumalaile. Aga need panid nad komistama nende teedel, põlistel teedel, kõndima jalgradadel, sillutamata teel,
\par 16 tehes nende maa õudseks, igavesti pilkealuseks. Igaüks, kes sealt mööda läheb, kohkub ja vangutab pead.
\par 17 Ma pillutan nad vaenlase ees otsekui idatuul, ma näitan neile kukalt, aga mitte palet nende õnnetusepäeval.”
\par 18 Aga nad ütlevad: „Tulge, peame nõu Jeremija vastu! Sest ei lõpe preestril Seadus, targal nõu ega prohvetil sõna. Tulge, lööme teda keelega ja paneme tähele iga tema sõna!”
\par 19 Issand, pane mind tähele ja kuule mu süüdistajate häält!
\par 20 Kas head tasutakse kurjaga? Nad on ju mulle augu kaevanud, et võtta minult elu. Tuleta meelde, kuidas ma seisin sinu ees, rääkides nende kasuks head, et pöörata nende pealt su viha.
\par 21 Seepärast jäta nende lapsed nälga ja tõuka nad mõõga ette; nende naised jäägu lasteta ja leskedeks, nende mehi tapku katk ja nende noored mehed löödagu sõjas mõõgaga maha!
\par 22 Kuuldagu nende kodadest kisendamist, kui sa äkitselt tood nende kallale röövjõugu. Sest nad on kaevanud minu püüdmiseks augu ja on salaja seadnud mu jalgadele paelu.
\par 23 Aga sina, Issand, tead kõiki nende mõrvaplaane minu vastu. Ära lepita nende süüd ja ära kustuta nende pattu oma palge eest! Komistagu nad sinu palge ees - talita nendega nõnda oma viha ajal!

\chapter{19}

\par 1 Nõnda ütleb Issand: „Mine ja osta potissepa tehtud savikruus, võta kaasa rahva vanemaid ja preestrite vanemaid
\par 2 ja mine välja Ben-Hinnomi orgu, mis on Potikillu värava ees, hüüa seal need sõnad, mis ma sulle räägin,
\par 3 ja ütle: Kuulge Issanda sõna, Juuda kuningad ja Jeruusalemma elanikud! Nõnda ütleb vägede Issand, Iisraeli Jumal: Vaata, ma lasen tulla sellele paigale niisuguse õnnetuse, et igaühel, kes sellest kuuleb, hakkavad kõrvad kumisema,
\par 4 sellepärast et nad hülgasid minu ja muutsid tundmatuseni selle paiga ning suitsutasid seal teistele jumalatele, keda ei tundnud nemad ise ega nende vanemad ja Juuda kuningad, ja täitsid selle paiga süütute verega
\par 5 ning ehitasid ohvrikünkaid Baalile, et põletada tules oma lapsi Baalile põletusohvriks, mida mina ei ole käskinud ega millest ma ei ole rääkinud ja mis mulle ei ole meeldegi tulnud.
\par 6 Seepärast, vaata, päevad tulevad, ütleb Issand, kui seda paika enam ei nimetata Põletuspaigaks ega Ben-Hinnomi oruks, vaid Tapaoruks.
\par 7 Ja ma tühistan siin paigas Juuda ja Jeruusalemma nõu ning lasen neid langeda mõõga läbi nende vaenlaste ees, ja nende käe läbi, kes ihkavad võtta neilt elu; ja ma annan nende laibad roaks taeva lindudele ja maa loomadele.
\par 8 Ja ma teen selle linna jubeduseks ja pilkealuseks. Igaüks, kes sellest mööda läheb, kohkub ja vilistab kõigi selle nuhtluste pärast.
\par 9 Ja ma panen nad sööma oma poegade ja tütarde liha, ja nad söövad üksteise liha piiramise ajal ja kitsikuses, millega nende vaenlased ja need, kes püüavad nende elu, neid kimbutavad.
\par 10 Siis purusta kruus nende meeste nähes, kes koos sinuga on tulnud,
\par 11 ja ütle neile: Nõnda ütleb vägede Issand: Nõndasamuti purustan ma selle rahva ja selle linna, otsekui purustatakse potissepa astja, mida ei saa enam parandada; ja Põletuspaika maetaksegi, sest muud matmispaika ei ole.
\par 12 Nõnda talitan ma selle paigaga, ütleb Issand, ja selle elanikega, et teha seda linna Põletuspaiga sarnaseks.
\par 13 Jeruusalemma kojad ja Juuda kuningate kojad saavad Põletuspaiga taoliselt rüvetatuiks, kõik kojad, mille katustel on suitsutatud kõigile taevavägedele ja on valatud joogiohvreid teistele jumalatele.”
\par 14 Kui Jeremija tuli Põletuspaigast, kuhu Issand oli ta läkitanud prohvetlikult kuulutama, siis ta seisis Issanda koja õues ning ütles kogu rahvale:
\par 15 „Nõnda ütleb vägede Issand, Iisraeli Jumal: Vaata, ma toon sellele linnale ja kõigile ta linnadele õnnetuse, mille ma temale olen tõotanud, sest nad on teinud oma kaela kangeks, et mitte kuulda minu sõnu.”

\chapter{20}

\par 1 Kui preester Pashur, Immeri poeg, kes oli Issanda koja kõrgeim ülevaataja, kuulis Jeremijat prohvetlikult rääkivat neid sõnu,
\par 2 siis lõi Pashur prohvet Jeremijat ja pani tema jalapakku, mis oli Issanda koja juures Benjamini Ülaväravas.
\par 3 Kui Pashur järgmisel päeval vabastas Jeremija jalapakust, ütles Jeremija temale: „Issand ei nimeta sind Pashuriks, vaid „Hirm kõikjal”.
\par 4 Sest Issand ütleb nõnda: Vaata, ma panen su hirmuks sulle enesele ja kõigile su sõpradele: need langevad oma vaenlaste mõõga läbi sinu silmade nähes. Ja ma annan kogu Juuda Paabeli kuninga kätte ning tema viib nad Paabelisse vangi ja lööb nad mõõgaga maha.
\par 5 Ja ma annan ära kõik selle linna vara, kõik ta töötulu ja kõik ta kallid asjad; kõik Juuda kuningate varandused annan ma nende vaenlaste kätte: nad riisuvad neid, võtavad kaasa ja viivad Paabelisse.
\par 6 Ja sina, Pashur, ja kõik, kes elavad su kojas, lähete vangi ja sind viiakse Paabelisse: seal sa sured ja sinna sind maetakse, sind ja kõiki su sõpru, kellele sa oled prohvetlikult kuulutanud valet.
\par 7 Sa valmistasid mulle pettumuse, Issand, ja ma olen pettunud; sa oled minust tugevam ja võitsid. Ma olen iga päev olnud naeruks, kõik pilkavad mind.
\par 8 Sest iga kord, kui ma räägin, pean ma kisendama. Ma hüüan: „Vägivald ja rüüstamine!” Sest Issanda sõna on saanud mulle igapäevaseks teotuseks ja pilkeks.
\par 9 Aga kui ma mõtlesin: Ma ei taha enam mõelda tema peale ega rääkida tema nimel, siis oli mu südames nagu põletav tuli, suletud mu luudesse-liikmetesse. Ma nägin vaeva, et seda taluda, aga ma ei suutnud.
\par 10 Sest ma olen kuulnud paljude laimu, ähvardust kõikjal: „Andke üles! Anname ta üles!” Kõik mu sõbrad luuravad mu vääratust: Vahest ta laseb ennast tüssata, et saaksime teda võita ja temale kätte tasuda.
\par 11 Aga Issand on minuga otsekui võimas kangelane; seepärast komistavad mu jälitajad ega suuda midagi. Nad jäävad suurde häbisse, sest neil ei ole tulemust. See on igavene häbi, mida ei unustata.
\par 12 Vägede Issand, kes sa katsud õige läbi, kes sa näed neerusid ja südant, lase mind näha, et sa neile kätte tasud, sest ma olen sinule avaldanud oma riiuasja.
\par 13 Laulge Issandale, kiitke Issandat, sest tema päästab vaese hinge kurjategijate käest!
\par 14 Neetud olgu päev, mil ma sündisin; päev, mil ema mu sünnitas, jäägu õnnistuseta!
\par 15 Neetud olgu mees, kes tõi mu isale sõnumi, öeldes: „Sulle sündis poeglaps!” ja teda väga rõõmustas.
\par 16 Selle mehe käsi käigu nagu linnadel, mis Issand kahetsemata segi paiskas. Kuulgu ta hädahüüdu hommikul ja sõjakisa keskpäeval,
\par 17 et ta mind ei surmanud emaihus, et mu ema oleks olnud mulle hauaks ja ta üsk oleks jäänud igavesti rasedaks.
\par 18 Mispärast ma olen välja tulnud emaihust nägema vaeva ja muret, nõnda et mu päevad lõpevad häbis?”

\chapter{21}

\par 1 Sõna, mis Jeremijale tuli Issandalt, kui kuningas Sidkija läkitas Pashuri, Malkija poja, ja preester Sefanja, Maaseja poja, temale ütlema:
\par 2 „Küsi ometi Issandalt meile nõu, sest Paabeli kuningas Nebukadnetsar sõdib meie vastu! Vahest talitab Issand meiega kõigi oma imetegude eeskujul, nõnda et ta läheb ära meie kallalt?”
\par 3 Ja Jeremija vastas neile: „Öelge Sidkijale nõnda:
\par 4 Nõnda ütleb Issand, Iisraeli Jumal: Vaata, ma pööran ümber sõjariistad, mis teil käes on, millega te väljaspool müüri sõdite Paabeli kuninga ja kaldealaste vastu, kes teid piiravad, ja kogun need selle linna keskele.
\par 5 Ja mina ise sõdin teie vastu väljasirutatud käe ja vägeva käsivarrega vihas ja raevus ja suures meelepahas.
\par 6 Ja ma löön maha selles linnas asujad, niihästi inimesed kui loomad: nad surevad suurde katku.
\par 7 Ja seejärel, ütleb Issand, annan ma Sidkija, Juuda kuninga, ja tema sulased ja rahva, need, kes sellesse linna katkust, mõõgast ja näljast on üle jäänud, Paabeli kuninga Nebukadnetsari kätte, ja nende vaenlaste kätte, nende kätte, kes püüavad nende elu. Ja tema lööb nad maha mõõgateraga, ta ei kurvasta nende pärast, ei anna armu ega halasta.
\par 8 Ja ütle sellele rahvale: Nõnda ütleb Issand: Vaata, ma panen teie ette elu tee ja surma tee!
\par 9 Kes jääb sellesse linna, see sureb mõõga läbi ning nälga ja katku; aga kes läheb välja ja alistub teid piiravatele kaldealastele, see jääb elama ja võib pidada oma hinge enesele saagiks.
\par 10 Sest ma olen pööranud oma palge selle linna vastu õnnetuseks, aga mitte õnneks, ütleb Issand; linn antakse Paabeli kuninga kätte ja ta põletab selle tulega.
\par 11 Ja ütle Juuda kuningakojale: Kuulge Issanda sõna!
\par 12 Taaveti sugu! Nõnda ütleb Issand: Mõistke igal hommikul õiget kohut ja päästke riisutu rõhuja käest, et mu viha ei süttiks nagu tuli ega põleks kustutamata teie kurjade tegude pärast!
\par 13 Vaata, ma olen su vastu, sina, oru elanik, tasandiku kalju, ütleb Issand. Teie ütlete: „Kes astub alla meile vastu ja kes tuleb meie eluasemeisse?”
\par 14 Aga mina nuhtlen teid teie tegude vilja mööda, ütleb Issand; ma süütan teie metsas tule ja see neelab kogu teie ümbruskonna.”

\chapter{22}

\par 1 Nõnda ütleb Issand: „Mine alla Juuda kuningakotta ja ütle seal need sõnad,
\par 2 räägi: Kuule Issanda sõna, Juuda kuningas, kes sa istud Taaveti aujärjel, sina ja su sulased ja su rahvas, kes te tulete sisse neist väravaist!
\par 3 Nõnda ütleb Issand: Tehke õigust ja olge õiglased ning päästke riisutu rõhuja käest; ärge vaevake, ärge tehke liiga võõrale, vaeslapsele ja lesknaisele, ja ärge valage siin paigas süütut verd!
\par 4 Sest kui te tõesti teete selle sõna järgi, siis tulevad selle koja väravaist sisse kuningad, kes istuvad Taaveti aujärjele, vankreis ja hobuste seljas sõites, nemad ise, nende sulased ja rahvas.
\par 5 Aga kui te ei kuula neid sõnu, siis ma olen vandunud iseeneses, ütleb Issand, et see koda lõhutakse maha.
\par 6 Sest Issand ütleb Juuda kuningakoja kohta nõnda: Sina oled olnud mulle nagu Gilead, Liibanoni tipp. Aga ma teen sind tõesti kõrbeks, elaniketa linnaks.
\par 7 Ma pühitsen su kallale hävitajad, igamehe oma tööriistaga: nad raiuvad maha su valitud seedrid ja viskavad tulle.
\par 8 Kui siis sellest linnast möödub palju rahvaid ja nad küsivad üksteiselt: Miks on Issand selle suure linnaga nõnda talitanud?,
\par 9 siis vastatakse: Sellepärast, et nad hülgasid Issanda, oma Jumala lepingu ja kummardasid teisi jumalaid ja teenisid neid.
\par 10 Ärge nutke surnu pärast ja ärge haletsege teda, vaid nutke parem selle pärast, kes läheb ära, sest tema ei tule enam tagasi ega saa näha oma sünnimaad!
\par 11 Sest nõnda ütleb Issand Juuda kuninga Sallumi, Joosija poja kohta, kes valitses oma isa Joosija asemel, kes läks ära siit paigast: Tema ei tule enam siia tagasi,
\par 12 vaid paika, kuhu ta vangi viidi, ta sureb ja ta ei saa enam näha seda maad.
\par 13 Häda sellele, kes ehitab oma koja ebaõiglusega ja oma ülakambrid ülekohtuga, kes paneb oma ligimese palgata orjama ega anna temale ta tasu,
\par 14 kes ütleb: Ma ehitan enesele ruumika koja, avarad ülakambrid, raiudes aknad, vooderdades seedritega ja võõbates punase värviga.
\par 15 Kas sa selleks oled kuningas, et hoobelda seedritega? Eks sinu isagi söönud ja joonud, aga ometi tegi ta õigust ja oli õiglane. Siis oli tal hea põli.
\par 16 Ta mõistis õigust viletsale ja vaesele, ja siis oli tal hea. Eks see tähenda minu tundmist? ütleb Issand.
\par 17 Aga sinu silmad ja su süda ei näe muud kui omakasu, kuidas valada süütut verd ning teostada vägivalda ja rõhumist.
\par 18 Sellepärast ütleb Issand Juuda kuninga Joojakimi, Joosija poja kohta nõnda: Tema pärast ei leinata: „Oh, mu vend!„ või: „Oh, õde!„ Tema pärast ei leinata: ”Oh, isand!” või: ”Oh, kõrgeausus!”
\par 19 Teda maetakse nagu eeslit, lohistatakse ja visatakse kaugele Jeruusalemma väravaist.
\par 20 Mine üles Liibanonile ja hüüa, tõsta häält Baasanis! Ja hõika Abarimist, sest kõik su armukesed on purustatud!
\par 21 Ma rääkisin sulle su heas põlves, aga sa vastasid: „Ma ei taha kuulda!” See on noorusest peale olnud su viis, et sa ei ole kuulanud mu häält.
\par 22 Tuul karjatab kõiki su karjaseid ja su armukesed lähevad vangi. Küll sa siis tunned häbi ja piinlikkust kõige oma kurjuse pärast.
\par 23 Sina, kes elad Liibanonil, pesitsed seedritel, kuidas sa küll siis oigad, kui sulle tulevad vaevused, valud otsekui sünnitajal.
\par 24 Nii tõesti kui ma elan, ütleb Issand, isegi kui Juuda kuningas Konja, Joojakimi poeg, oleks pitserisõrmuseks mu paremas käes - siiski kisuksin ma su sealt ära
\par 25 ja annaksin su nende kätte, kes püüavad su hinge, ja nende kätte, kelle ees sa hirmu tunned: Paabeli kuninga Nebukadnetsari ja kaldealaste kätte.
\par 26 Ja ma paiskan sinu ja su ema, kes sinu on sünnitanud, teisele maale, kus te ei ole sündinud, ja te surete seal.
\par 27 Aga maale, kuhu nende hing igatseb tagasi minna, sinna nad tagasi ei saa.
\par 28 On siis see mees, Konja, mõni tühine asi, mis purustatakse, või astja, mis ei kõlba kellelegi? Mispärast paisati minema tema ja ta sugu, ja heideti maale, mida nad ei tundnud?
\par 29 Maa, Maa, Maa, kuule Issanda sõna!
\par 30 Nõnda ütleb Issand: Pange kirja see mees kui lastetu, kui mees, kellel oma elupäevil ei ole õnne; sest mitte ühelgi tema soost ei õnnestu istuda Taaveti aujärjele ja jälle valitseda Juudas.

\chapter{23}

\par 1 Häda karjaseile, kes hukkavad ja pillutavad mu karjamaa lambaid, ütleb Issand!
\par 2 Seepärast ütleb Issand, Iisraeli Jumal, karjaste kohta, kes karjatavad mu rahvast, nõnda: Te olete mu lambad pillutanud ja ära ajanud ega ole vaadanud nende järele. Vaata, ma nuhtlen teid teie pahategude pärast, ütleb Issand.
\par 3 Ja ma ise kogun oma lammaste jäägi kõigist maadest, kuhu ma olen nad ajanud, ja toon nad tagasi nende karjamaale; nad on siis viljakad ja neid saab palju.
\par 4 Ja ma sean neile karjased ning nood karjatavad neid; siis nad enam ei ehmu ega karda, ja ühtegi neist ei jää vajaka, ütleb Issand.
\par 5 Vaata, päevad tulevad, ütleb Issand, mil ma lasen tõusta Taavetile ühe õige võsu; tema valitseb kui kuningas ja talitab targasti, tema teeb maal õigust ja õiglust.
\par 6 Tema päevil päästetakse Juuda ja Iisrael elab julgesti; ja see on nimi, millega teda hüütakse: „Issand, meie õigus”.
\par 7 Seepärast, vaata, päevad tulevad, ütleb Issand, mil enam ei öelda: „Nii tõesti kui elab Issand, kes tõi Iisraeli lapsed ära Egiptusemaalt”,
\par 8 vaid: „Nii tõesti kui elab Issand, kes tõi ja juhtis Iisraeli soo järglased koju põhjamaalt ja kõigist maadest, kuhu ma olin nad ajanud.” Ja nad hakkavad elama oma maal.
\par 9 Prohvetite kohta: Mu süda on murdunud rinnus, kõik mu luud-liikmed värisevad; ma olen nagu joobnud mees, nagu veinist vallutatu - Issanda ees ja tema pühade sõnade ees.
\par 10 Sest maa on täis abielurikkujaid. Jah, maa leinab needuse pärast, karjamaad kõrbes on kuivanud; nende püüdlused on kurjad ja nende jõuks on ebaõiglus.
\par 11 Niihästi prohvet kui preester on jumalakartmatu; koguni omaenese kojast olen ma leidnud nende kurjust, ütleb Issand.
\par 12 Seepärast on nende tee neile otsekui libastuskohaks pimeduses: neid tõugatakse ja nad langevad seal, sest ma toon neile õnnetuse, nende karistusaasta, ütleb Issand.
\par 13 Ka Samaaria prohvetite juures nägin ma jõledust: nad kuulutasid prohvetlikult Baali nimel ja eksitasid mu rahvast Iisraeli.
\par 14 Aga Jeruusalemma prohvetite juures nägin ma kohutavaid asju: abielurikkumist ja valelikku eluviisi. Nad julgustavad kurjategijaid, et ükski ei pöörduks oma kurjusest. Nad kõik on mulle nagu Soodom ja Jeruusalemma rahvas on nagu Gomorra.
\par 15 Seepärast ütleb vägede Issand prohvetite kohta nõnda: Vaata, ma söödan neid koirohuga ja joodan mürgiveega, sest Jeruusalemma prohveteist on jumalakartmatus levinud kogu maale.
\par 16 Nõnda ütleb vägede Issand: Ärge kuulake nende prohvetite sõnu, kes teile prohvetlikult kuulutavad - nad ainult tüssavad teid tühiste lootustega: nad räägivad oma südame kujutlustest, mitte Issanda suust.
\par 17 Nad ütlevad ühtepuhku mu laimajaile: „Issand on öelnud: Teil on rahu!„ Ja igaühele, kes käib oma südame paadumuses, nad ütlevad: ”Teile ei tule õnnetust!”
\par 18 Aga kes neist on olnud osaduses Issandaga ja on näinud ning kuulnud tema sõna? Kes on tähele pannud ja kuulnud tema sõna?
\par 19 Vaata, Issanda torm, tema raev puhkeb, ja keeristorm keerutab üle õelate pea.
\par 20 Issanda viha ei pöördu enne, kui ta on teoks teinud ja korda saatnud oma tahtmise. Viimseil päevil te mõistate seda hästi.
\par 21 Mina ei ole neid prohveteid läkitanud, vaid nad ise jooksevad; mina ei ole neile rääkinud, vaid nad ise kuulutavad prohveti kombel.
\par 22 Kui nad oleksid olnud osaduses minuga, siis nad kuulutaksid mu rahvale minu sõnu ning pööraksid neid nende kurjadelt teedelt ja kurjadest tegudest.
\par 23 Kas ma ainult ligidal olen Jumal, ütleb Issand, aga kaugemal ei olegi Jumal?
\par 24 Kas saab keegi ennast peita peidupaikadesse, ilma et mina teda näeksin? ütleb Issand. Kas see pole mina, kes täidab taeva ja maa? ütleb Issand.
\par 25 Ma olen kuulnud, mida räägivad need prohvetid, kes minu nimel kuulutavad valet, öeldes: „Ma nägin und, ma nägin und!”
\par 26 Kui kaua see kestab? Ons midagi südames neil prohveteil, kes kuulutavad valet ja kes avaldavad oma südame pettekujutlusi,
\par 27 kes mõtlevad oma unenägudega, mida nad üksteisele jutustavad, panna mu rahva unustama minu nime, nõnda nagu nende vanemad unustasid minu nime Baali pärast?
\par 28 Prohvet, kellel on olnud unenägu, jutustagu oma unenägu, aga kellel on minu sõna, kõnelgu mu sõna kui tõde! Mis on õlgedel tegemist puhta viljaga? ütleb Issand.
\par 29 Eks mu sõna ole nagu tuli, ütleb Issand, või nagu vasar, mis purustab kalju?
\par 30 Sellepärast, vaata, ütleb Issand, olen ma prohvetite vastu, kes varastavad üksteiselt minu sõnu.
\par 31 Vaata, ma olen prohvetite vastu, ütleb Issand, kes pruugivad omaenese keelt ja ütlevad: „Issand ütleb.”
\par 32 Vaata, ma olen nende vastu, kes prohvetlikult kuulutavad vääri unenägusid, ütleb Issand, ja jutustavad neid ning eksitavad mu rahvast oma valede ja kelkimistega. Ometi ei ole mina neid läkitanud ega käskinud ja nad ei too sellele rahvale mingit kasu, ütleb Issand.
\par 33 Ja kui see rahvas või prohvet või preester sinult küsib, öeldes: „Mis on Issanda ennustus?„, siis vasta neile: ”Te olete mulle koormaks ja ma tõukan teid ära, ütleb Issand.”
\par 34 Ja seda prohvetit ja preestrit ja rahvast, kes ütleb: „Issanda ennustus”, seda meest ja tema sugu ma nuhtlen.
\par 35 Igaüks küsigu oma naabrilt ja vennalt nõnda: „Mis Issand kostis?„ või: ”Mis Issand rääkis?”
\par 36 Aga Issanda ennustust ärge enam nimetage, sest igaühele saab koormaks omaenese sõna, sellepärast et te väänate elava Jumala, vägede Issanda, meie Jumala sõnu!
\par 37 Küsi prohvetilt nõnda: „Mis Issand sulle kostis?„ või: ”Mis Issand sulle rääkis?”
\par 38 Aga kui te ütlete: „Issanda ennustus„, siis ütleb Issand nõnda: Kuna te ütlete selle sõna: „Issanda ennustus” - mina aga olen läkitanud teile ütlema: Ärge öelge: ”Issanda ennustus” -,
\par 39 vaata, seepärast ma siis tõstan teid tõesti üles ning tõukan teid ja linna, mille ma andsin teie vanemaile, oma palge eest ära.
\par 40 Ja ma panen teie peale igavese teotuse ja igavese häbi, mida ei unustata.”

\chapter{24}

\par 1 Issand näitas mulle, ja vaata, Issanda templi ette oli pandud kaks korvi viigimarju, pärast seda kui Paabeli kuningas Nebukadnetsar oli Jeruusalemmast vangi viinud Juuda kuninga Jekonja, Joojakimi poja, ja Juuda vürstid, samuti sepad ja ehitustöölised, ja oli viinud need Paabelisse.
\par 2 Ühes korvis olid väga head viigimarjad, varajaste viigimarjade sarnased, ja teises korvis olid väga halvad viigimarjad, nii halvad, et need ei kõlvanud süüa.
\par 3 Ja Issand küsis minult: „Mida sa näed, Jeremija?„ Mina vastasin: ”Viigimarju. Head viigimarjad on väga head, aga halvad on väga halvad, nii halvad, et need ei kõlba süüa.”
\par 4 Siis tuli mulle Issanda sõna; ta ütles:
\par 5 „Nõnda ütleb Issand, Iisraeli Jumal: Otsekui neid häid viigimarju, nõnda hindan ma hästi Juuda vange, keda ma siit paigast olen läkitanud Kaldeasse.
\par 6 Ma heidan silma nende peale heas mõttes ja toon nad tagasi sellele maale. Ma ehitan neid ega kisu maha, ja ma istutan neid ega kitku välja.
\par 7 Ja ma annan neile südame mõistmiseks, et mina olen Issand. Siis on nemad mulle rahvaks ja mina olen neile Jumalaks, sest nad pöörduvad minu poole kõigest südamest.
\par 8 Aga otsekui halbade viigimarjadega, mis on nii halvad, et need ei kõlba süüa, tõesti, nõnda ütleb Issand, nõnda talitan ma Juuda kuninga Sidkijaga ja tema vürstidega ja Jeruusalemma jäägiga, kes on alles jäänud siia maale või kes elavad Egiptusemaal.
\par 9 Ja ma sean nad hirmutuseks ja õnnetuseks kõigile kuningriikidele maa peal, teotuseks ja kõnekäänuks, pilkesõnaks ja needuseks kõigis paigus, kuhu ma nad tõukan.
\par 10 Ja ma läkitan nende sekka mõõga, nälja ja katku, kuni nad on hävitatud maalt, mille ma olen andnud neile ja nende vanemaile.”

\chapter{25}

\par 1 Sõna, mis Jeremijale tuli kogu Juuda rahva kohta Juuda kuninga Joojakimi, Joosija poja neljandal aastal, see on Paabeli kuninga Nebukadnetsari esimesel aastal,
\par 2 mis prohvet Jeremija rääkis kogu Juuda rahva ja kõigi Jeruusalemma elanike kohta, öeldes:
\par 3 „Juuda kuninga Joosija, Aamoni poja kolmeteistkümnendast aastast alates kuni tänase päevani, nende kahekümne kolme aasta jooksul on mulle tulnud Issanda sõna ja ma olen teile rääkinud ja rääkinud, aga teie ei ole kuulanud.
\par 4 Ja Issand on läkitanud teie juurde kõik oma sulased prohvetid, on läkitanud lakkamatult, aga teie ei ole kuulanud ega ole pööranud oma kõrva kuulma,
\par 5 kui ta ütles: Pöördugu ometi igaüks oma kurjalt teelt ja oma kurjadest tegudest, siis te jääte elama maale, mille Issand on andnud teile ja teie vanemaile muistsest ajast igavesti!
\par 6 Ja ärge käige teiste jumalate järel neid teenides ja kummardades, ja ärge pahandage mind oma kätetööga, siis ma ei tee teile kurja!
\par 7 Aga teie ei kuulanud mind, ütleb Issand, vaid pahandasite mind oma kätetööga õnnetuseks teile enestele.
\par 8 Seepärast ütleb vägede Issand nõnda: Kuna te ei võtnud kuulda mu sõnu,
\par 9 vaata, siis ma toon kõik põhjamaa suguvõsad, ütleb Issand, ja läkitan käsu Paabeli kuningale Nebukadnetsarile, oma sulasele, ja toon nemad selle maa ja ta elanike vastu, ja kõigi nende ümberkaudsete rahvaste vastu. Ja ma hävitan nad ning teen nad jubeduseks ja pilkealuseks ning igavesteks varemeteks.
\par 10 Ja ma kaotan nende keskelt lustihääle ja rõõmuhääle, peigmehe hääle ja pruudi hääle, käsikivi mürina ja lambivalguse.
\par 11 Kogu see maa muutub varemeiks, õudseks, ja need rahvad orjavad Paabeli kuningat seitsekümmend aastat.
\par 12 Aga kui seitsekümmend aastat täis saab, siis ma nuhtlen Paabeli kuningat ja seda rahvast, ütleb Issand, nende süüteo pärast, ja kaldealaste maad, tehes selle igavesti lagedaks.
\par 13 Ma lasen täide minna kõik oma sõnad, mis ma olen rääkinud selle maa kohta, kõik, mis on kirjutatud sellesse raamatusse, kõik, mis Jeremija on prohvetlikult kuulutanud kõigi rahvaste kohta.
\par 14 Sest ka nemad peavad orjama paljusid rahvaid ja suuri kuningaid, ja ma tasun neile nende tegude ja kätetöö järgi!
\par 15 Sest Issand, Iisraeli Jumal, ütles mulle nõnda: Võta see vihaveini karikas minu käest ja jooda sellega kõiki rahvaid, kelle juurde ma sind läkitan,
\par 16 et nad jooksid ja tuiguksid ning hulluksid mõõga ees, mille ma läkitan nende sekka!
\par 17 Ja ma võtsin karika Issanda käest ning jootsin kõiki rahvaid, kelle juurde Issand mind läkitas:
\par 18 Jeruusalemma ja Juuda linnu, selle kuningaid ja vürste, et teha nad varemeiks, jubeduseks, pilkealuseks ja needuseks, nagu see tänapäeval on;
\par 19 vaaraod, Egiptuse kuningat, ja tema sulaseid ja vürste ja kogu ta rahvast;
\par 20 ja kõike segarahvast ja kõiki Uusimaa kuningaid ja kõiki vilistite maa kuningaid ja Askeloni, Assat, Ekroni, ja neid, kes on Asdodis alles jäänud;
\par 21 Edomit, Moabit ja ammonlasi;
\par 22 ja kõiki Tüürose kuningaid ja kõiki Siidoni kuningaid ja saarte kuningaid teiselt poolt merd;
\par 23 ja Dedanit, Teemat, Buusi ja kõiki, kellel on pügatud oimud;
\par 24 ja kõiki Araabia kuningaid ja kõiki kõrbes elava segarahva kuningaid;
\par 25 ja kõiki Simri kuningaid, kõiki Eelami kuningaid ja kõiki Meedia kuningaid;
\par 26 ja kõiki põhjamaa kuningaid ligidalt ja kaugelt üksteise järel, ja kõiki maailma kuningriike, mis maa peal on. Ja nende järel peab jooma Seesaki kuningas.
\par 27 Ja ütle neile: Nõnda ütleb vägede Issand, Iisraeli Jumal: Jooge ja joobuge, oksendage ja langege maha ja ärge tõuske enam mõõga ees, mille ma läkitan teie sekka!
\par 28 Aga kui nad tõrguvad karikat joomiseks su käest vastu võtmast, siis ütle neile: Nõnda ütleb vägede Issand: Te peate jooma!
\par 29 Sest vaata, linnast, millele on pandud minu nimi, alustan ma õnnetuse toomist, ja teie peaksite jääma hoopis karistamata! Te ei jää karistamata, sest mina kutsun mõõga kõigi maa elanike kallale, ütleb vägede Issand!
\par 30 Aga sina kuuluta neile prohvetlikult kõik need sõnad ja ütle neile: Issand müristab kõrgusest ja toob oma pühast eluasemest kuuldavale oma hääle. Ta müristab võimsasti oma karjamaa kohal, kõigi maa elanike vastu kostab otsekui viinamarjatallajate hüüd.
\par 31 Mürin jõuab maailma ääreni, sest Issandal on rahvastega riid, ta käib kohut kõigiga: ta annab õelad mõõga kätte, ütleb Issand.
\par 32 Nõnda ütleb vägede Issand: Vaata, õnnetus käib rahvast rahvani ja suur torm tõuseb maa viimastest äärtest.
\par 33 Ja sel päeval on Issanda mahalööduid maa äärest ääreni. Neid ei leinata ja neid ei koristata ega maeta, nad jäävad maa peale sõnnikuks.
\par 34 Ulguge, karjased, ja kisendage, püherdage, karja isandad, sest päevad on täis, et teid tappa ja pillutada, ja te langete nagu hinnaline astja!
\par 35 Siis kaob karjastel varjupaik ja karja isandail pakkupääs.
\par 36 Kuule! Karjaste hädakisa ja karja isandate hala! Sest Issand laastab nende karjamaad
\par 37 ja vaiksed vainud muutuvad elutuks Issanda vihalõõma ees.
\par 38 Otsekui oleks noor lõvi jätnud oma padriku: sest nende maa muutub õudseks hävitaja mõõga ja tema vihalõõma ees.”

\chapter{26}

\par 1 Ja Juuda kuninga Joojakimi, Joosija poja valitsemise alguses tuli see sõna Issandalt, kes ütles:
\par 2 „Nõnda ütleb Issand: Seisa Issanda koja õues ja räägi kõigi Juuda linnade elanike kohta, kes tulevad Issanda kotta kummardama, kõik need sõnad, mis ma sind olen käskinud neile rääkida ilma sõnagi ära jätmata!
\par 3 Vahest nad kuulavad ja pöörduvad igaüks oma kurjalt teelt ja ma võin kahetseda, et olen kavatsenud teha neile kurja nende kurjade tegude pärast.
\par 4 Ja ütle neile: Nõnda ütleb Issand: Kui te ei kuula mind, et käiksite mu Seaduse järgi, mille ma teile olen andnud,
\par 5 et kuulaksite mu sulaste, prohvetite sõnu, keda ma teie juurde olen läkitanud ja läkitanud, aga keda te ei ole kuulanud,
\par 6 siis ma talitan selle kojaga nagu Siiloga ja teen selle linna sajatamissõnaks kõigile maa rahvaile.”
\par 7 Ja preestrid ja prohvetid ja kogu rahvas kuulsid Jeremijat rääkivat neid sõnu Issanda kojas.
\par 8 Ja kui Jeremija oli rääkinud lõpuni kõik, mis Issand oli käskinud rääkida kogu rahvale, siis preestrid ja prohvetid ja kogu rahvas võtsid ta kinni, öeldes: „Sa pead surema!
\par 9 Mispärast sa kuulutad Issanda nimel ja ütled: See koda saab Siilo sarnaseks ja see linn jääb tühjaks, kus ei ole elanikke?” Ja kogu rahvas Issanda kojas kogunes Jeremija vastu.
\par 10 Kui Juuda vürstid kuulsid neid sõnu, siis nad läksid kuningakojast üles Issanda kotta ja istusid Issanda templi uue värava suhu.
\par 11 Ja preestrid ja prohvetid rääkisid vürstidele ja kogu rahvale, öeldes: „See mees on surma väärt, sest ta on prohvetlikult kuulutanud selle linna kohta, nagu te oma kõrvaga olete kuulnud.”
\par 12 Aga Jeremija rääkis kõigile vürstidele ja kogu rahvale, öeldes: „Issand on mind läkitanud prohvetlikult kuulutama selle koja ja selle linna kohta kõiki neid sõnu, mis te olete kuulnud.
\par 13 Ja nüüd parandage oma eluviise ja tegusid ja kuulake Issanda, oma Jumala häält, siis Issand kahetseb õnnetust, mida ta teile on tõotanud.
\par 14 Aga mina, vaata, olen teie käes; tehke minuga, nagu teie silmis on hea ja õige!
\par 15 Ometi teadke kindlasti, et kui te mind surmate, siis tõmbate süütu vere iseeneste ja selle linna ja ta elanike peale, sest Issand on tõesti läkitanud mind teie juurde, et ma räägiksin teie kõrvu kõik need sõnad!”
\par 16 Siis ütlesid vürstid ja kogu rahvas preestritele ja prohvetitele: „See mees ei ole surma väärt, sest ta on meile rääkinud Issanda, meie Jumala nimel.”
\par 17 Ja mehed maa vanemate hulgast tõusid ning rääkisid tervele rahvakogule, öeldes:
\par 18 „Miika, moorastlane, kuulutas prohvetlikult Juuda kuninga Hiskija päevil ja rääkis kogu Juuda rahvale, öeldes: „Nõnda ütleb vägede Issand: Siion küntakse põlluks, Jeruusalemm muutub varemeiks ja templimägi võsastunud künkaks.”
\par 19 Kas Juuda kuningas Hiskija ja kogu Juuda tema sellepärast kohe surmasid? Eks ta kartnud Issandat ja leevendanud Issanda palet, nõnda et Issand kahetses õnnetust, mida ta neile oli tõotanud? Ja meie peaksime valmistama oma hingedele selle suure õnnetuse!”
\par 20 Aga oli ka keegi mees Uurija, Semaja poeg Kirjat-Jearimist, kes Issanda nimel prohvetlikult kuulutas, ja tema kuulutas selle linna ja maa kohta kõigiti sarnaselt Jeremija sõnadega.
\par 21 Aga kui kuningas Joojakim ja kõik ta ihukaitsjad ja kõik vürstid kuulsid tema sõnu, siis püüdis kuningas teda surmata; aga Uurija kuulis sellest, kartis ja põgenes ning jõudis Egiptusesse.
\par 22 Siis kuningas Joojakim läkitas Egiptusesse mehed: Elnatani, Akbori poja, ja mehi koos temaga.
\par 23 Ja need tõid Uurija Egiptusest tagasi ning viisid ta kuningas Joojakimi ette; ja kuningas lõi ta mõõgaga maha ning viskas ta laiba lihtrahva hauda.
\par 24 Aga Jeremijaga oli Ahikami, Saafani poja käsi, nõnda et teda ei antud rahva kätte surmata.

\chapter{27}

\par 1 Juuda kuninga Sidkija, Joosija poja valitsemise alguses tuli Jeremijale Issandalt see sõna; ta ütles:
\par 2 „Issand ütles mulle nõnda: Valmista enesele köidikud ja ikkepuud ning pane need enesele kaela!
\par 3 Läkita siis need Edomi kuningale, Moabi kuningale, ammonlaste kuningale, Tüürose kuningale ja Siidoni kuningale käskjalgade kaudu, kes tulevad Jeruusalemma Juuda kuninga Sidkija juurde,
\par 4 ja käsi neid öelda oma isandaile: Nõnda ütleb vägede Issand, Iisraeli Jumal: Öelge oma isandaile nõnda:
\par 5 Mina olen teinud maa, inimesed ja loomad, kes maa peal on, oma suure rammu ja väljasirutatud käsivarre abil, ja ma annan need sellele, kes minu silmis õige on.
\par 6 Ja nüüd annan ma kõik need maad oma sulase, Paabeli kuninga Nebukadnetsari kätte, ja ma annan temale ka metsloomad teda teenima.
\par 7 Ja kõik rahvad peavad teenima teda ja tema poega ja pojapoega, kuni tuleb temagi aeg ja paljud rahvad ja suured kuningad sunnivad tedagi teenima.
\par 8 Ja seda rahvast ja seda kuningriiki, kes tõrgub teenimast teda, Paabeli kuningat Nebukadnetsarit, ja kes ei anna oma kaela Paabeli kuninga ikke alla, seda rahvast ma nuhtlen mõõga ja nälja ja katkuga, ütleb Issand, kuni ma teen neile tema käe läbi lõpu.
\par 9 Teie ärge kuulake oma prohveteid ja ennustajaid ning unenägusid, samuti mitte märkide seletajaid ja nõidu, kes teile räägivad ja ütlevad: Te ei hakka teenima Paabeli kuningat!
\par 10 Sest nemad kuulutavad teile valet, selleks et teid eemaldada teie oma maalt, et mina peaksin teid ära tõukama, mistõttu te hukkute.
\par 11 Aga selle rahva, kes viib oma kaela Paabeli kuninga ikke alla ja teenib teda, jätan ma tema oma maale, ütleb Issand, ja ta võib seda harida ning elada sel maal.
\par 12 Ja Juuda kuningale Sidkijale rääkisin ma kõik needsamad sõnad, öeldes: Andke oma kaelad Paabeli kuninga ikke alla ja teenige teda ja tema rahvast, siis te jääte elama!
\par 13 Miks peaksite surema, sina ja su rahvas, mõõga, nälja ja katku läbi, nõnda nagu Issand on tõotanud rahvale, kes ei taha teenida Paabeli kuningat?
\par 14 Ja ärge kuulake nende prohvetite sõnu, kes teile räägivad ja ütlevad: „Te ei hakka teenima Paabeli kuningat!”, sest nad kuulutavad teile valet.
\par 15 Sest mina ei ole neid läkitanud, ütleb Issand; ometi kuulutavad nad valet minu nimel, selleks et ma tõukaksin teid ära ja et te hukkuksite, teie ja need prohvetid, kes teile kuulutavad.
\par 16 Ja preestritele ja kogu sellele rahvale ma rääkisin, öeldes: Nõnda ütleb Issand: Ärge kuulake oma prohvetite sõnu, kes teile kuulutavad, öeldes: „Vaata, Issanda koja riistad tuuakse nüüd varsti Paabelist tagasi”, sest nad kuulutavad teile valet!
\par 17 Ärge kuulake neid, teenige Paabeli kuningat, siis te jääte elama! Miks peaks see linn saama varemeiks?
\par 18 Kui nad siiski on prohvetid ja kui neil on Issanda sõna, siis nad palugu ometi vägede Issandat, et riistu, mis on jäänud Issanda kotta ja kuningakotta, ei viidaks Paabelisse!
\par 19 Sest nõnda ütleb vägede Issand sammaste, vaskmere, aluste ja muude riistade kohta, mis on veel jäänud siia linna
\par 20 ja mida Paabeli kuningas Nebukadnetsar ära ei võtnud, kui ta Jeruusalemmast viis Paabelisse vangi Juuda kuninga Jekonja, Joojakimi poja, ja kõik Juuda ja Jeruusalemma suurnikud,
\par 21 sest nõnda ütleb vägede Issand, Iisraeli Jumal, riistade kohta, mis on jäänud Issanda kotta ja Juuda kuningakotta ja Jeruusalemma:
\par 22 Need viiakse Paabelisse ja need jäävad sinna selle päevani, kui ma tunnen muret nende pärast, ütleb Issand, ja toon need ära ning panen tagasi siia paika!”

\chapter{28}

\par 1 Ja selsamal aastal, Juuda kuninga Sidkija valitsemise alguses, neljanda aasta viiendas kuus, rääkis minuga prohvet Hananja, Assuri poeg Gibeonist, Issanda kojas preestrite ja kogu rahva nähes; ta ütles:
\par 2 „Nõnda räägib vägede Issand, Iisraeli Jumal, kes ütleb: Mina murran Paabeli kuninga ikke!
\par 3 Enne kui kaks aastat on möödunud, toon ma tagasi siia paika kõik Issanda koja riistad, mis Paabeli kuningas Nebukadnetsar võttis siit paigast ja viis Paabelisse.
\par 4 Ja Juuda kuninga Jekonja, Joojakimi poja, ja kõik Juuda vangid, kes läksid Paabelisse, toon ma tagasi siia paika, ütleb Issand, sest ma murran Paabeli kuninga ikke.”
\par 5 Aga prohvet Jeremija rääkis prohvet Hananjaga preestrite nähes ja kogu rahva nähes, kes seisid Issanda kojas.
\par 6 Prohvet Jeremija ütles: „Aamen! Issand tehku nõnda! Issand kinnitagu su sõnu, mis sa oled prohvetlikult kuulutanud, et ta toob Paabelist tagasi siia paika Issanda koja riistad ja kõik vangid.
\par 7 Ent kuule ometi seda sõna, mis ma räägin sinu kuuldes ja kogu rahva kuuldes:
\par 8 Prohvetid, kes on olnud enne mind ja enne sind muistsest peale, on prohvetlikult kuulutanud paljudele maadele ja suurtele kuningriikidele sõda, õnnetust ja katku.
\par 9 Aga prohvet, kes kuulutab rahu, tunnistatakse alles siis prohvetiks, kelle Issand tõesti on läkitanud, kui selle prohveti sõna on täide läinud.”
\par 10 Siis prohvet Hananja võttis ikke prohvet Jeremija kaelast ja murdis katki.
\par 11 Ja Hananja rääkis kogu rahva nähes, öeldes: „Nõnda ütleb Issand: Selsamal kombel murran ma Paabeli kuninga Nebukadnetsari ikke kõigi rahvaste kaelast, enne kui kaks aastat on möödunud.” Aga prohvet Jeremija läks oma teed.
\par 12 Ja Issanda sõna tuli Jeremijale pärast seda, kui prohvet Hananja oli katki murdnud ikke prohvet Jeremija kaelast; ta ütles:
\par 13 „Mine ja räägi Hananjaga ning ütle: Nõnda ütleb Issand: Sa oled murdnud puuikked, aga oled teinud nende asemele raudikked!
\par 14 Sest nõnda ütleb vägede Issand, Iisraeli Jumal: Ma olen pannud raudikke kõigi nende rahvaste kaela, et nad teeniksid Paabeli kuningat Nebukadnetsarit, ja nad peavad teda teenima! Ja ma olen andnud temale metsloomigi.”
\par 15 Ja prohvet Jeremija ütles prohvet Hananjale: „Kuule ometi, Hananja! Issand ei ole sind läkitanud, vaid sina oled pannud selle rahva uskuma valet.
\par 16 Seepärast ütleb Issand nõnda: Vaata, ma läkitan sind maa pealt ära. Sel aastal sa sured, sest sa oled kuulutanud taganemist Issandast.”
\par 17 Ja prohvet Hananja suri selsamal aastal seitsmendas kuus.

\chapter{29}

\par 1 Ja need on selle kirja sõnad, mis prohvet Jeremija läkitas Jeruusalemmast vangide vanemaile, kes veel alles olid, ja preestreile ja prohveteile ja kogu rahvale, keda Nebukadnetsar oli Jeruusalemmast viinud vangi Paabelisse,
\par 2 pärast seda kui kuningas Jekonja ja kuninganna, hoovkondlased, Juuda ja Jeruusalemma vürstid ning sepad ja ehitustöölised Jeruusalemmast olid läinud,
\par 3 Elaasa, Saafani poja, ja Gemarja, Hilkija poja, vahendusel, keda Juuda kuningas Sidkija läkitas Paabeli kuninga Nebukadnetsari juurde Paabelisse; selles oli öeldud:
\par 4 „Nõnda ütleb vägede Issand, Iisraeli Jumal, kõigile vangidele, keda ma olen Jeruusalemmast lasknud viia vangi Paabelisse:
\par 5 Ehitage kodasid ja elage neis, istutage rohuaedu ja sööge nende vilja!
\par 6 Võtke naisi ja laske sündida poegi ja tütreid! Võtke oma poegadele naisi ja pange oma tütred mehele, et nad sünnitaksid poegi ja tütreid! Paljunege seal, ärge vähenege!
\par 7 Ja taotlege selle linna heaolu, kuhu ma olen lasknud teid viia, ja paluge selleks Issandat; sest selle hea põli on teie hea põli!
\par 8 Sest nõnda ütleb vägede Issand, Iisraeli Jumal: Ärge laske endid petta oma prohveteist, kes on teie keskel, ja oma ennustajaist, ja ärge võtke tõena oma unistusi, mida te unistate!
\par 9 Sest nad kuulutavad teile minu nimel valet; mina ei ole neid läkitanud, ütleb Issand.
\par 10 Sest Issand ütleb nõnda: Alles kui seitsekümmend Paabeli aastat saab täis, kannan ma hoolt teie eest ja teen oma hea sõna teie kohta tõeks ja toon teid tagasi siia paika.
\par 11 Sest mina tunnen mõtteid, mis ma teie pärast mõlgutan, ütleb Issand: need on rahu, aga mitte õnnetuse mõtted, et anda teile tulevikku ja lootust.
\par 12 Siis te hüüate mind appi ja tulete ning palute mind, ja mina kuulen teid.
\par 13 Ja te otsite mind ja leiate minu, kui te nõuate mind kõigest oma südamest.
\par 14 Ja ma lasen teil mind leida, ütleb Issand, ja ma pööran teie vangipõlve ning kogun teid kõigi rahvaste seast ja kõigist paigust, kuhu ma teid olen tõuganud, ütleb Issand, ja toon teid tagasi paika, kust ma lasksin teid vangi viia.
\par 15 Ja kuna te olete öelnud: Issand on Paabelis lasknud tõusta meile prohveteid -
\par 16 siis ütleb Issand nõnda kuninga kohta, kes istub Taaveti aujärjel, ja kogu rahva kohta, kes elab siin linnas, teie vendade kohta, kes ei ole läinud vangi koos teiega:
\par 17 Nõnda ütleb vägede Issand: Vaata, ma läkitan nende sekka mõõga, nälja ja katku ja teen nad roiskunud viigimarjade sarnaseks, mis on nii halvad, et need ei kõlba süüa.
\par 18 Ja ma ajan neid taga mõõga, nälja ja katkuga ja teen nad hirmutuseks kõigile kuningriikidele maa peal, sajatuseks ja jälestuseks, pilkealuseks ja teotuseks kõigi rahvaste seas, kuhu ma nad olen tõuganud,
\par 19 sellepärast et nad ei kuulanud mu sõnu, ütleb Issand, kui ma läkitasin nende juurde oma sulaseid prohveteid, koguni korduvalt läkitasin; aga teie ei kuulanud, ütleb Issand.
\par 20 Aga kuulge Issanda sõna, teie, kõik vangid, keda ma olen Jeruusalemmast saatnud Paabelisse!
\par 21 Nõnda ütleb vägede Issand, Iisraeli Jumal, Ahabi, Koolaja poja, ja Sidkija, Maaseja poja kohta, kes teile minu nimel kuulutavad valet: Vaata, ma annan nad Paabeli kuninga Nebukadnetsari kätte ja tema lööb nad maha teie silme ees.
\par 22 Ja neist saab sajatussõna kõigile Juuda vangidele, kes on Paabelis; öeldakse: „Issand tehku sinuga nagu Sidkija ja Ahabiga, keda Paabeli kuningas kõrvetas tulega!”,
\par 23 sellepärast et nad tegid Iisraelis jõledust ja rikkusid abielu oma ligimeste naistega ning rääkisid minu nimel valesõnu, mida mina neid ei olnud käskinud; mina tean seda ja olen tunnistaja, ütleb Issand.
\par 24 Ja räägi nehelamlase Semajaga ning ütle:
\par 25 Nõnda räägib vägede Issand, Iisraeli Jumal, ja ütleb: Sellepärast et sa oma nimel läkitasid kirja kogu rahvale, kes on Jeruusalemmas, ja preester Sefanjale, Maaseja pojale, ja kõigile preestritele, ja ütlesid:
\par 26 Issand on sind pannud preestriks preester Joojada asemele, et Issanda kojas oleks järelevaatajaid iga märatseja ja kuulutaja tarvis ja et sa saaksid seesuguse panna jalapakku või kaelaraudu.
\par 27 Aga nüüd, mispärast sa ei ole sõidelnud Jeremijat, anatotlast, kes teile prohvetlikult kuulutab?
\par 28 Sest ta on läkitanud meile Paabelisse sõna: See kestab kaua; ehitage kodasid ja elage neis, istutage rohuaedu ja sööge nende vilja!”
\par 29 Ja preester Sefanja luges seda kirja prohvet Jeremija kuuldes.
\par 30 Siis tuli Jeremijale Issanda sõna; ta ütles:
\par 31 „Läkita ütlema kõigile vangidele: Nõnda ütleb Issand nehelamlase Semaja kohta: Sellepärast et Semaja on teile prohvetlikult kuulutanud, kuigi mina teda ei ole läkitanud, ja on teid pannud uskuma valet,
\par 32 sellepärast ütleb Issand nõnda: Vaata, ma karistan nehelamlast Semajat ja tema sugu; temal ei ole siis enam seda, kes elaks selle rahva keskel, ega saa ta näha head, mida ma teen oma rahvale, ütleb Issand; sest ta on kuulutanud taganemist Issandast.”

\chapter{30}

\par 1 Sõna, mis Jeremijale tuli Issandalt; ta ütles:
\par 2 „Nõnda räägib Issand, Iisraeli Jumal, ja ütleb: Kirjuta kõik sõnad, mis ma sulle olen rääkinud, enesele raamatusse!
\par 3 Sest vaata, päevad tulevad, ütleb Issand, mil ma pööran oma Iisraeli ja Juuda rahva vangipõlve, ütleb Issand, ja ma toon nad tagasi maale, mille ma andsin nende vanemaile, ja nad pärivad selle.”
\par 4 Ja need on sõnad, mis Issand rääkis Iisraeli ja Juuda kohta,
\par 5 sest nõnda ütleb Issand: „Me oleme kuulnud hirmukisa; on kartus, aga mitte rahu.
\par 6 Küsige ometi ja vaadake: kas meesterahvas sünnitab? Mispärast ma näen iga meest käed puusas nagu sünnitajal naisel, ja mispärast on kõik näod muutunud kahvatuks?
\par 7 Häda! Sest see päev on suur, sellesarnast ei ole. See on Jaakobile ahastuse aeg, aga ta päästetakse sellest.
\par 8 Ja sel päeval, ütleb vägede Issand, ma murran ikke su kaelast ja kisun katki su köidikud; ja muulased ei pane neid enam teenima,
\par 9 vaid nad teenivad Issandat, oma Jumalat, ja Taavetit, oma kuningat, kelle ma neile tõstan.
\par 10 Aga sina, mu sulane Jaakob, ära karda, ütleb Issand, ja Iisrael, ära ehmu! Sest vaata, ma päästan sind kaugelt ja sinu soo nende vangipõlvemaalt. Jaakob tuleb tagasi ning elab rahus ja muretult, ilma et keegi teda hirmutaks.
\par 11 Sest mina olen sinuga, ütleb Issand, et sind päästa. Ma teen lõpu kõigile rahvaile, kelle sekka ma sind olen pillutanud. Aga sinule ma ei tee lõppu: ma karistan sind õiglaselt, aga hoopis karistamata ma sind küll ei jäta.
\par 12 Sest nõnda ütleb Issand: Su vigastus on ravimatu, su haav on valus.
\par 13 Ei ole kedagi, kes kaitseks su õigust, mädapaisel ei ole paranemist, sulle ei kasva korpa.
\par 14 Kõik su armukesed on sind unustanud, nad ei küsi su järele. Sest ma olen sind löönud, nagu lüüakse vaenlast halastamatu karistusega, sellepärast et su süü on suur, et su patte on palju.
\par 15 Miks sa kisendad oma haava pärast, et su valu on vaigistamatu? Su suure süü, su paljude pattude pärast olen ma sulle seda teinud.
\par 16 Ometi õgitakse ära kõik, kes sind õgivad, ja kõik, kes sind rõhuvad - viimane kui üks läheb vangi; kes sind rüüstavad, saavad rüüstatavaiks, ja kõik, kes võtavad sind saagiks, annan ma saagiks.
\par 17 Sest ma tahan lasta kasvada sulle korba ja ravida sind sinu haavadest, ütleb Issand, sellepärast et sind on nimetatud hüljatuks: „Siion, kellest ükski ei hooli!”
\par 18 Nõnda ütleb Issand: Vaata, ma pööran Jaakobi telkide vangipõlve ja ma halastan tema eluasemete peale; linn ehitatakse üles oma rusuhunnikule ja palee asub seal, kus see peab asuma.
\par 19 Sealt kostab tänulaul ja rõõmsate hääl. Ma teen nad paljuks ja nad ei vähene, ma teen nad auväärseiks ja neid ei halvustata.
\par 20 Siis on ta pojad nagu muistegi ja ta kogudus seisab kindlana mu ees. Ja ma karistan kõiki ta piinajaid.
\par 21 Temast enesest tuleb ta aukandja ja tema keskelt ilmub ta valitseja; ma luban teda ligineda ja ta tuleb mu juurde, sest kes muidu annaks pandiks oma südame, et minule ligineda? ütleb Issand.
\par 22 Siis te olete mu rahvas ja mina olen teie Jumal.
\par 23 Vaata, Issanda torm, raev, puhkeb, ja keeristorm keerutab üle õelate pea.
\par 24 Issanda viha ei pöördu enne, kui ta on teoks teinud ja korda saatnud oma tahtmise. Viimseil päevil te mõistate seda.

\chapter{31}

\par 1 Sel ajal, ütleb Issand, olen ma kõigi Iisraeli suguvõsade Jumal ja nemad on mu rahvas.
\par 2 Nõnda ütleb Issand: Rahvas, mõõgast pääsenud, on kõrbes leidnud armu: Iisrael läheb oma rahupaika.
\par 3 Issand ilmutas ennast mulle kaugelt: Ma olen sind armastanud igavese armastusega, seepärast jääb mu osadus sinuga.
\par 4 Mina ehitan sind jälle, et sa oleksid üles ehitatud, Iisraeli neitsi. Sa ehid ennast jälle trummidega ja lähed rõõmsatega tantsima.
\par 5 Sa istutad jälle viinapuuaedu Samaaria mägedele; need, kes istutavad, saavad ka kasutada.
\par 6 Sest tuleb päev, kui vahimehed hüüavad Efraimi mäestikus: „Tõuske, mingem üles Siionisse Issanda, meie Jumala juurde!”
\par 7 Sest nõnda ütleb Issand: Hõisake ilutsedes Jaakobi pärast ja tundke rõõmu rahvaste peast; kuulutage, kiitke ja öelge: „Issand, päästa oma rahvas, Iisraeli jääk!”
\par 8 Vaata, ma toon nad põhjamaalt ja kogun nad maa viimastest äärtest; nende hulgas on ka pimedaid ja jalutuid, rasedaid ja sünnitajaid: suure hulgana tulevad nad siia tagasi.
\par 9 Nad tulevad nuttes ja ma toon neid anudes; ma viin nad veeojade äärde tasasel teel, kus nad ei komista; sest ma olen Iisraeli isa ja Efraim on mu esmasündinu.
\par 10 Rahvad, kuulge Issanda sõna ja kuulutage kaugetel saartel ning öelge: „Tema, kes pillutas Iisraeli, kogub ja hoiab teda nagu karjane oma karja!”
\par 11 Sest Issand lunastab Jaakobi ja vabastab tema selle käest, kes on temast vägevam.
\par 12 Ja nad tulevad ning hõiskavad Siioni kõrgendikul, nad säravad rõõmust Issanda headuse pärast, vilja ja veini ja õli pärast, noorte lammaste, kitsede ja veiste pärast; nende hing on nagu kastetud rohuaed ja nad ei närbu enam.
\par 13 Siis neitsid rõõmutsevad tantsides ning noorukid ja raugad üheskoos; ma muudan nende leina rõõmuks, ma trööstin ja rõõmustan neid pärast kurvastust.
\par 14 Ja ma kosutan preestrite südameid rasvaga ning mu rahvas küllastub mu headusest, ütleb Issand.
\par 15 Nõnda ütleb Issand: Raamast kuuldakse häält, halinat, kibedat nuttu: Raahel nutab oma poegi. Ta ei lase ennast trööstida oma poegade pärast, sest neid ei ole enam.
\par 16 Nõnda ütleb Issand: Keela oma häält nutmast ja silmi pisaraid valamast, sest su teol on tasu, ütleb Issand, ja nad tulevad tagasi vaenlase maalt.
\par 17 Sul on lootust tulevikuks, ütleb Issand, su lapsed tulevad tagasi oma maale.
\par 18 Ma olen küll kuulnud, et Efraim haletseb iseennast: „Sina oled mind karistanud ja ma olen saanud karistuse nagu tõrges härjavärss. Too mind tagasi, et saaksin pöörduda, sest sina oled Issand, mu Jumal!
\par 19 Sest pärast taganemist ma kahetsen, ja pärast mõistuseletoomist ma löön enesele vastu puusa; ma häbenen ja tunnen piinlikkust, sest ma kannan oma noorpõlve teotust.”
\par 20 Eks ole Efraim mulle kalliks pojaks ja lemmiklapseks? Sest iga kord, kui ma räägin tema vastu, mõtlen ma ikka temale; seetõttu on mu süda tema pärast rahutu: ma tahan tõesti halastada tema peale, ütleb Issand.
\par 21 Püstita enesele märgikivid, aseta teeviidad, pea meeles maanteed, rada, mida oled käinud! Tule tagasi, Iisraeli neitsi, tule tagasi oma linnadesse!
\par 22 Kui kaua sa kõhkled, taganenud tütar? Sest Issand loob maal midagi uut: naine kaitseb meest.
\par 23 Nõnda ütleb vägede Issand, Iisraeli Jumal: Veel kõneldakse Juudamaal ja selle linnades, kui ma pööran nende vangipõlve, seda sõna: „Issand õnnistagu sind, sa õigluse eluase, sa püha mägi!”
\par 24 Seal elavad üheskoos Juuda ja kõik ta linnad, põllumehed ja rändavad karjased.
\par 25 Sest ma kosutan väsinud hinge ja täidan iga nälginud hinge.”
\par 26 Seepeale ma ärkasin ja vaatasin, ja mul oli olnud magus uni.
\par 27 „Vaata, päevad tulevad, ütleb Issand, mil ma külvan Iisraeli soole ja Juuda soole inimeste seemet ja loomade seemet.
\par 28 Ja nõnda nagu ma valvasin neid, et kitkuda ja maha kiskuda, purustada, hävitada ja tuua õnnetust, nõnda ma valvan neid, et üles ehitada ja istutada, ütleb Issand.
\par 29 Neil päevil ei öelda enam: „Isad sõid tooreid viinamarju, aga laste hambad on hellad”,
\par 30 vaid igaüks sureb oma süü pärast: igal inimesel, kes sööb tooreid viinamarju, lähevad ta oma hambad hellaks.
\par 31 Vaata, päevad tulevad, ütleb Issand, mil ma teen Iisraeli sooga ja Juuda sooga uue lepingu:
\par 32 mitte selle lepingu sarnase, mille ma tegin nende vanematega sel päeval, kui ma võtsin nad kättpidi, et viia nad välja Egiptusemaalt - selle mu lepingu nad murdsid, kuigi ma olin nad võtnud enese omaks, ütleb Issand -,
\par 33 vaid leping, mille ma teen Iisraeli sooga pärast neid päevi, ütleb Issand, on niisugune: ma panen nende sisse oma Seaduse ja kirjutan selle neile südamesse; siis ma olen neile Jumalaks ja nemad on mulle rahvaks.
\par 34 Siis üks ei õpeta enam teist ega vend venda, öeldes: „Tunne Issandat!”, sest nad kõik tunnevad mind, niihästi pisikesed kui suured, ütleb Issand; sest ma annan andeks nende süü ega tuleta enam meelde nende pattu.
\par 35 Nõnda ütleb Issand, kes on pannud päikese valguseks päeval, kuu ja tähtede korrad valguseks öösel, kes liigutab merd, paneb selle lained kohama - vägede Issand on tema nimi:
\par 36 Kui need korrad nihkuksid mu palge eest, ütleb Issand, siis lakkaks ka Iisraeli sugu alatiseks olemast rahvas mu palge ees.
\par 37 Nõnda ütleb Issand: Kui peaks saama mõõta taevast ülal ja uurida maa aluseid all, siis hülgaksin ka mina kogu Iisraeli soo kõige selle pärast, mis nad on teinud, ütleb Issand.
\par 38 Vaata, päevad tulevad, ütleb Issand, mil linn Issandale üles ehitatakse Hananeli tornist Nurgaväravani.
\par 39 Ja mõõdunöör läheb veel edasi otse üle Gaarebi künka ning käändub Goa suunas.
\par 40 Ja kogu surnukehade ja tuha org, ja kõik põllud kuni Kidroni jõeni, Hobuvärava nurgani ida suunas, on pühendatud Issandale. Iialgi enam ei kitkuta seda välja ega kista maha.”

\chapter{32}

\par 1 Sõna, mis Jeremijale tuli Issandalt Juuda kuninga Sidkija kümnendal aastal; see oli Nebukadnetsari kaheksateistkümnes aasta.
\par 2 Paabeli kuninga sõjavägi piiras siis Jeruusalemma ja prohvet Jeremijat peeti kinni vahtkonnaõues, mis oli Juuda kuningakoja juures,
\par 3 kuhu Juuda kuningas Sidkija oli ta kinni pannud, öeldes: „Mispärast sa kuulutad ja ütled: Nõnda ütleb Issand: Vaata, ma annan selle linna Paabeli kuninga kätte ja ta vallutab selle!?
\par 4 Ja Juuda kuningas Sidkija ei pääse kaldealaste käest, vaid antakse kindlasti Paabeli kuninga kätte ja ta peab temaga rääkima suust suhu ja nägema teda silmast silma.
\par 5 Sidkija viiakse tema poolt Paabelisse ja ta jääb sinna, kuni ma võtan tema oma hoolde, ütleb Issand; kui te sõdite kaldealaste vastu, ei ole teil õnne.”
\par 6 Ja Jeremija ütles: „Mulle tuli Issanda sõna; ta ütles:
\par 7 Vaata, Hanameel, su lelle Sallumi poeg, tuleb sinu juurde ja ütleb: Osta enesele minu Anatotis olev põld, sest sul on selle ostuks lunaõigus!”
\par 8 Ja Hanameel, mu lellepoeg, tuli Issanda sõna kohaselt mu juurde vahtkonnaõue ja ütles mulle: „Osta ometi mu põld, mis on Anatotis Benjamini maal, sest sinul on pärimis- ja lunaõigus; osta see enesele!” Siis ma mõistsin, et see oli Issanda sõna.
\par 9 Ja ma ostsin Hanameelilt, oma lellepojalt, Anatotis oleva põllu ning vaagisin temale seitseteist hõbeseeklit hõbedat,
\par 10 kirjutasin üriku ja pitseerisin selle, võtsin siis tunnistajad ja vaagisin hõbeda vaekaussidega.
\par 11 Siis ma võtsin käsu ja seaduste kohaselt pitseeritud ostukirja ja lahtise kirja
\par 12 ning andsin ostukirja Baarukile, Mahseja poja Neerija pojale, oma lellepoja Hanameeli ja tunnistajate nähes, kes olid ostukirjale alla kirjutanud, kõigi juutide nähes, kes istusid vahtkonnaõues.
\par 13 Ja ma käskisin Baarukit nende nähes, öeldes:
\par 14 „Nõnda ütleb vägede Issand, Iisraeli Jumal: Võta need kirjad, see pitseeritud ostukiri ja see lahtine kiri, ja pane saviastjaisse, et need säiliksid kaua aega!
\par 15 Sest nõnda ütleb vägede Issand, Iisraeli Jumal: Kord ostetakse siin maal jälle kodasid, põlde ja viinamägesid.”
\par 16 Ja pärast seda, kui olin andnud ostukirja Baarukile, Neerija pojale, palusin ma Issandat, öeldes:
\par 17 „Oh Issand Jumal! Vaata, sina oled teinud taeva ja maa oma suure rammu ja väljasirutatud käsivarrega: ükski asi ei ole võimatu sinul,
\par 18 kes osutad heldust tuhandeile, aga tasud vanemate süü nende laste sülle pärast neid; sina oled suur, vägev Jumal, kelle nimi on vägede Issand,
\par 19 suur nõu ja vägev teo poolest, sina, kelle silmad on lahti inimlaste kõigi teede kohal, et anda igaühele tema teede ja tegude vilja järgi,
\par 20 sina, kes oled teinud tunnustähti ja imetegusid Egiptusemaal ning teed neid tänapäevani Iisraelis ja muude inimeste seas, ja kes oled teinud enesele nime, nõnda nagu see tänapäeval on.
\par 21 Sina tõid ära oma Iisraeli rahva Egiptusemaalt tunnustähtede ja imetegudega, vägeva käe ja väljasirutatud käsivarrega ning suure hirmu abil
\par 22 ja andsid neile selle maa, mille sa vandega olid tõotanud anda nende vanemaile, maa, mis voolab piima ja mett.
\par 23 Aga kui nad tulid ja võtsid selle omaks, siis nad ei võtnud kuulda su sõna ega käinud su Seaduse järgi, nad ei teinud midagi kõigest sellest, mida sa neid olid käskinud teha; seepärast lasksid sa tulla neile kogu selle õnnetuse.
\par 24 Vaata, piiramisvallid ulatuvad linnani, et seda vallutada, ja linn antakse mõõga, nälja ja katku abil kaldealaste kätte, kes sõdivad selle vastu. Millest sa oled rääkinud, see on sündinud, ja Vaata, sa ise näed seda.
\par 25 Aga sina, Issand Jumal, ütlesid mulle ometi: Osta raha eest enesele põld ja võta tunnistajad, ehk küll linn antakse kaldealaste kätte!”
\par 26 Ja Issanda sõna tuli Jeremijale; ta ütles:
\par 27 „Vaata, mina olen Issand, kõige liha Jumal! Ons mulle mõni asi võimatu?
\par 28 Seepärast ütleb Issand nõnda: Vaata, ma annan selle linna kaldealaste kätte ja Paabeli kuninga Nebukadnetsari kätte ja tema vallutab selle.
\par 29 Ja kaldealased, kes sõdivad selle linna vastu, tulevad ning süütavad selle linna tulega ja põletavad selle ja kojad, mille katustel minu pahandamiseks on suitsutatud Baalile ja on valatud joogiohvreid teistele jumalatele.
\par 30 Sest Iisraeli lapsed ja Juuda lapsed on oma noorusest alates teinud minu silmis ainult kurja; on ju Iisraeli lapsed mind ainult pahandanud oma kätetööga, ütleb Issand.
\par 31 Jah, see linn on mulle olnud viha ja raevu põhjuseks alates päevast, kui see ehitati, kuni tänapäevani, nõnda et ma pean selle kõrvaldama oma palge eest
\par 32 kõigi Iisraeli laste ja Juuda laste kurjuse pärast, mida nad on teinud, et mind pahandada, nemad, nende kuningad, nende vürstid, nende preestrid ja nende prohvetid, ja Juuda mehed ja Jeruusalemma elanikud.
\par 33 Nad on pööranud mu poole kukla, aga mitte näo; ja kuigi ma olen neid õpetanud ja õpetanud, ei ole nad kuulnud, et võtta õpetust.
\par 34 Nad on asetanud oma jäledused kotta, millele on pandud minu nimi, ja on selle roojastanud.
\par 35 Ja nad on ehitanud Baali ohvrikünkad, mis on Ben-Hinnomi orus, et ohverdada Moolokile oma poegi ja tütreid, mida ma neid ei ole käskinud ja mis mulle ei ole tulnud meeldegi, et nad võiksid teha niisugust jõledust, selleks et saata Juudat pattu tegema.
\par 36 Aga nüüd, tõepoolest, ütleb Issand, Iisraeli Jumal, nõnda selle linna kohta, mille kohta te ütlete: „See antakse Paabeli kuninga kätte mõõga, nälja ja katku abil!”
\par 37 Vaata, ma kogun neid kõigist maadest, kuhu ma olen nad tõuganud oma vihas, raevus ja suures meelepahas, ja toon nad tagasi siia paika ning lasen neid elada turvaliselt.
\par 38 Siis on nad mulle rahvaks ja mina olen neile Jumalaks.
\par 39 Ja ma annan neile ühesuguse südame ja ühesuguse tee, nõnda et nad alati kardavad mind, kasuks neile ja nende lastele pärast neid.
\par 40 Ja ma teen nendega igavese lepingu, et ma ei loobu tegemast neile head; ja ma annan neile südamesse minu kartuse, et nad ei lahkuks minust.
\par 41 Ja ma tunnen neist rõõmu, tehes neile head, ma istutan nad ustavalt siia maale kõigest oma südamest ja kõigest oma hingest.
\par 42 Sest nõnda ütleb Issand: Nii nagu ma tõin sellele rahvale kogu selle suure õnnetuse, nõnda ma toon neile kõik selle hea, mis ma neile olen tõotanud.
\par 43 Ja põlde ostetakse veelgi sellel maal, mille kohta te ütlete: „See on lage, inimesteta ja loomadeta, ja see antakse kaldealaste kätte!”
\par 44 Põlde ostetakse raha eest, ostukirju kirjutatakse ja pitseeritakse ja tunnistajaid võetakse Benjamini maal, Jeruusalemma ümbruskonnas, Juuda linnades, mäestiku linnades, madalmaa linnades ja Lõunamaa linnades, sest ma pööran nende vangipõlve, ütleb Issand.”

\chapter{33}

\par 1 Ja Issanda sõna tuli teist korda Jeremijale, kui teda veel kinni peeti vahtkonnaõues; ta ütles:
\par 2 „Nõnda ütleb Issand, kes teeb seda, Issand, kes valmistab ja kinnitab seda - Issand on tema nimi -:
\par 3 Hüüa mind, siis ma vastan sulle ja ilmutan sulle suuri ja salajasi asju, mida sa ei tea!
\par 4 Sest nõnda ütleb Issand, Iisraeli Jumal, selle linna kodade ja Juuda kuningate kodade kohta, mis on maha kistud kaitseks piiramisseadmete ja mõõga vastu,
\par 5 et neid kasutada võitluses kaldealastega ja et neid täita surnutega, keda ma löön maha oma vihas ja raevus, kuna ma olen peitnud oma palge selle linna eest nende kõigi kurjuse pärast:
\par 6 Vaata, ma toon temale paranemist ja tervist, ja ma teen nad terveks ning ilmutan neile rohket rahu ja tõtt.
\par 7 Ma pööran Juuda vangipõlve ja Iisraeli vangipõlve ja ehitan nad üles nõnda nagu muistegi.
\par 8 Ja ma puhastan nad kogu nende süüst, millega nad on pattu teinud minu vastu; ja ma annan neile andeks kõik nende süüteod, millega nad on pattu teinud minu vastu ja millega nad on üles astunud minu vastu.
\par 9 Ja see linn saab mulle rõõmunimeks, kiituseks ning ehteks kõigi maa rahvaste ees, kes kuulevad kõigest heast, mida ma neile teen; siis nad kardavad ja värisevad kõige selle hea ja kõige selle õnne pärast, mida ma linnale annan.
\par 10 Nõnda ütleb Issand: Veel kuuldakse siin paigas, mille kohta te ütlete: „See on laastatud, pole inimest ega looma”, laastatud Juuda linnades ja Jeruusalemma tänavail, kus pole inimest, elanikku ega looma,
\par 11 lustihäält ja rõõmuhäält, peigmehe häält ja pruudi häält, nende häält, kes ütlevad: „Tänage vägede Issandat, sest Issand on hea, sest tema heldus kestab igavesti!”, kes toovad tänuohvrit Issanda kotta. Sest ma pööran maa vangipõlve nõnda nagu muistegi, ütleb Issand.
\par 12 Nõnda ütleb vägede Issand: Veel saab olema siin laastatud paigas, kus pole inimest ega looma, ja kõigil selle linnadel karjamaid karjastele lammaste ja kitsede puhkamiseks.
\par 13 Mäestiku linnades, madalmaa linnades ja Lõunamaa linnades, Benjamini maal, Jeruusalemma ümbruskonnas ja Juuda linnades käib veel lambaid läbi lugeja käte alt, ütleb Issand.
\par 14 Vaata, päevad tulevad, ütleb Issand, mil ma teen tõeks hea sõna, mis ma olen rääkinud Iisraeli soo ja Juuda soo kohta.
\par 15 Neil päevil ja sel ajal lasen ma tärgata Taavetile õiguse võsu, ja tema teeb siis sellel maal õigust ja õiglust.
\par 16 Neil päevil päästetakse Juuda, ja Jeruusalemm võib elada turvaliselt, ja see on nimi, millega teda nimetatakse: Issand, meie õigus.
\par 17 Sest nõnda ütleb Issand: Ei puudu Taavetil mees, kes istub Iisraeli soo aujärjel.
\par 18 Ja leviitpreestreil ei puudu mees mu palge ees, ohverdamas põletusohvrit ja suitsutamas roaohvrit ning toimetamas tapaohvrit iga päev.”
\par 19 Ja Jeremijale tuli Issanda sõna; ta ütles:
\par 20 „Nõnda ütleb Issand: Kui te võite tühjaks teha minu lepingu päevaga ja minu lepingu ööga, nõnda et päev ja öö ei tule neile seatud ajal,
\par 21 siis muutub tühjaks ka minu leping mu sulase Taavetiga, nõnda et tal enam ei ole poega, kes valitseks tema aujärjel, nõndasamuti minu leping leviitidega, preestritega, kes mind teenivad.
\par 22 Nõnda nagu ei saa ära lugeda taeva väge ega mõõta mere liiva, nõnda rohkeks ma teen oma sulase Taaveti soo ja leviidid, kes mind teenivad.”
\par 23 Ja Jeremijale tuli Issanda sõna; ta ütles:
\par 24 „Kas sa ei ole märganud, mis see rahvas räägib ja ütleb: Need mõlemad suguvõsad, kelle Issand valis, on ta hüljanud? Ja nad laimavad mu rahvast, nagu see ei olekski nende meelest enam rahvas.
\par 25 Nõnda ütleb Issand: Kui ei oleks mu lepingut päeva ja ööga, kui ma ei oleks seadnud taeva ja maa korda,
\par 26 siis ma hülgaksin ka Jaakobi ja oma sulase Taaveti soo ega võtaks tema soost valitsejaid Aabrahami, Iisaki ja Jaakobi soole. Aga ma pööran nende vangipõlve ja halastan nende peale.”

\chapter{34}

\par 1 Sõna, mis Jeremijale tuli Issandalt, kui Paabeli kuningas Nebukadnetsar ja kogu ta sõjavägi, kõik maa kuningriigid, mille üle tema käsi valitses, ja kõik rahvad sõdisid Jeruusalemma ja kõigi selle linnade vastu; ta ütles:
\par 2 „Nõnda ütleb Issand, Iisraeli Jumal: Mine ja räägi Juuda kuninga Sidkijaga ja ütle temale: Nõnda ütleb Issand: Vaata, ma annan selle linna Paabeli kuninga kätte ja tema põletab selle tulega maha.
\par 3 Ka sina ei pääse tema käest, sest sind võetakse tõesti kinni ja antakse tema kätte; sina saad Paabeli kuningat näha silmast silma ja tema räägib sinuga suust suhu ning sa lähed Paabelisse.
\par 4 Kummatigi kuule Issanda sõna, Sidkija, Juuda kuningas! Nõnda ütleb Issand sinu kohta: Sa ei sure mõõga läbi!
\par 5 Sa sured rahus! Ja nõnda nagu su isade, endiste kuningate, kes enne sind on olnud, auks põletati koolnusuitsutusi, nõnda põletatakse sinugi auks, ja sõnadega „Oh isand!” leinatakse ka sind! Sest mina ise olen öelnud selle sõna, ütleb Issand.”
\par 6 Ja prohvet Jeremija rääkis Juuda kuningale Sidkijale kõik need sõnad Jeruusalemmas,
\par 7 kui Paabeli kuninga sõjavägi sõdis Jeruusalemma ja kõigi järelejäänud Juuda linnade vastu, Laakise ja Aseka vastu, sest need olid Juuda linnadest, kindlustatud linnadest, järele jäänud.
\par 8 Sõna, mis Jeremijale tuli Issandalt pärast seda, kui kuningas Sidkija oli teinud lepingu kogu Jeruusalemmas oleva rahvaga, et kuulutada neile vabastust:
\par 9 igaüks pidi vabastama oma sulase ja teenija, heebrealase ja heebrealanna, et keegi ei orjaks juuti, oma venda.
\par 10 Ja kõik vürstid ja kogu rahvas, kes olid astunud lepingusse, kuulsid, et igaühel tuleb vabastada oma sulane ja teenija, et neid ei tohi enam orjastada; nad kuulasid sõna ja vabastasid need.
\par 11 Aga hiljem nad taganesid ja võtsid tagasi sulased ja teenijad, keda nad olid vabastanud, ja sundisid neid taas sulaseiks ja teenijaiks.
\par 12 Siis tuli Issandalt Jeremijale Issanda sõna; ta ütles:
\par 13 „Nõnda ütleb Issand, Iisraeli Jumal: Mina tegin teie vanematega lepingu päeval, kui ma tõin nad ära Egiptusemaalt, orjusekojast, öeldes:
\par 14 Igal seitsmendal aastal vabastagu igaüks oma heebrealasest vend, kes enese sulle on müünud; kui ta sind on teeninud kuus aastat, siis lase ta enese juurest vabaks; aga teie vanemad ei kuulanud mind ega pööranud oma kõrva.
\par 15 Teie olete küll äsja pöördunud ja teinud, mis minu silmis õige on, kuulutades igaüks oma ligimesele vabastust, ja olete teinud lepingu mu palge ees kojas, millele on pandud minu nimi.
\par 16 Aga te olete jälle taganenud ja teotanud mu nime: olete võtnud tagasi igaüks oma sulase ja teenija, keda te olite vabastanud nende soovi kohaselt, ja olete neid sundinud olema teile sulaseiks ja teenijaiks.
\par 17 Seepärast ütleb Issand nõnda: Te ei ole kuulanud mind, et oleksite kuulutanud vabastust igaüks oma vennale ja ligimesele; vaata, mina kuulutan teile vabastust, ütleb Issand, mõõga, katku ja nälja läbi ja panen teid hirmutuseks kõigile kuningriikidele maa peal.
\par 18 Ja mehed, kes rikkusid mu lepingut, kes ei pidanud selle lepingu sõnu, mille nad tegid minu ees, ma teen vasikaks, kelle nad raiusid pooleks ja kelle tükkide vahelt nad käisid läbi -
\par 19 Juuda vürstid ja Jeruusalemma vürstid, hoovkondlased ja preestrid ja kogu maa rahva, kes käisid läbi vasika tükkide vahelt -,
\par 20 ja ma annan nad nende vaenlaste kätte ja nende kätte, kes püüavad nende hinge; ja nende surnukehad jäävad taeva lindude ja maa loomade roaks.
\par 21 Ka Juuda kuninga Sidkija ja tema vürstid annan ma nende vaenlaste kätte ja nende kätte, kes püüavad nende hinge, ja Paabeli kuninga sõjaväe kätte, kes teie kallalt on ära läinud.
\par 22 Vaata, ma käsin, ütleb Issand, ja toon nad tagasi selle linna kallale; nad sõdivad selle vastu, vallutavad selle ja põletavad tulega maha; ja ma teen Juuda linnad lagedaks, kus ükski ei ela.”

\chapter{35}

\par 1 Sõna, mis Jeremijale tuli Issandalt Juuda kuninga Joojakimi, Joosija poja päevil; ta ütles:
\par 2 „Mine reekablaste soo juurde, räägi nendega ja too nad Issanda kotta ühte kambrisse ning anna neile veini juua!”
\par 3 Siis ma võtsin Jaasanja, Habassinja poja Jeremija poja, ja tema vennad ja kõik ta pojad ja kogu reekablaste soo
\par 4 ja tõin nad Issanda kotta jumalamehe Haanani, Jigdalja poja poegade kambrisse, mis on vürstide kambri kõrval ülalpool lävehoidja Maaseja, Sallumi poja kambrit.
\par 5 Ja ma panin reekablaste soo poegade ette veiniga täidetud peekrid ja karikad ning ütlesin neile: „Jooge veini!”
\par 6 Aga nad vastasid: „Meie ei joo veini, sest Joonadab, Reekabi poeg, Meie isa, on meid keelanud, öeldes: Ärge iialgi jooge veini, ei teie ega teie lapsed!
\par 7 Ja ärge ehitage kodasid, ärge külvake seemet ja ärge istutage viinamägesid; ärgu teil olgu neid, vaid elage telkides kogu oma eluaja, et võiksite kaua viibida maal, kus te võõraina elate!
\par 8 Ja me oleme kuulanud oma isa Joonadabi, Reekabi poja häält kõiges, mida ta meile on keelanud: me ei joo veini kogu oma eluaja, ei me ise, ei meie naised, ei meie pojad ega meie tütred,
\par 9 ja me ei ehita enestele kodasid elamiseks, meil ei ole viinamäge, põldu ega seemet,
\par 10 vaid me elame telkides ning oleme sõnakuulelikud ja teeme kõigiti nõnda, nagu Joonadab, meie isa, meid on käskinud.
\par 11 Aga kui Paabeli kuningas Nebukadnetsar tuli selle maa vastu, siis me ütlesime: Tulge, läki Jeruusalemma kaldealaste sõjaväe ja süürlaste sõjaväe eest! Ja me asusime Jeruusalemma.”
\par 12 Ja Jeremijale tuli Issanda sõna; ta ütles:
\par 13 „Nõnda ütleb vägede Issand, Iisraeli Jumal: Mine ja ütle Juuda meestele ja Jeruusalemma elanikele: Kas te ei võtaks õpetust, et kuuleksite minu sõnu? ütleb Issand.
\par 14 Joonadabi, Reekabi poja sõnu, millega ta keelas oma poegi veini joomast, peetakse, ja nad ei joo tänapäevani, sest nad kuulavad oma isa keeldu. Kuid mina olen teile rääkinud ja rääkinud, aga te ei ole mind kuulanud.
\par 15 Ja ma olen teie juurde läkitanud kõik oma sulased prohvetid, olen läkitanud korduvalt, öeldes: Pöörduge ometi igaüks oma kurjalt teelt ja tehke heaks oma teod; ja ärge käige teiste jumalate järel neid teenides, siis te jääte maale, mille ma olen andnud teile ja teie vanemaile! Aga te ei ole pööranud oma kõrva ega ole mind kuulanud.
\par 16 Et Joonadabi, Reekabi poja pojad on pidanud oma isa keeldu, mille ta neile andis, aga see rahvas ei ole mind kuulanud,
\par 17 siis ütleb Issand, vägede Jumal, Iisraeli Jumal nõnda: Vaata, ma toon Juudale ja kõigile Jeruusalemma elanikele kogu selle õnnetuse, mille ma neile olen tõotanud, sellepärast et nad ei kuulanud, kui ma neile rääkisin, ega vastanud, kui ma neid hüüdsin.”
\par 18 Aga reekablaste soole ütles Jeremija: „Nõnda ütleb vägede Issand, Iisraeli Jumal: Sellepärast et te olete kuulanud oma isa Joonadabi keeldu ning olete pidanud kõiki tema käske ja olete teinud kõigiti, nagu ta teid on käskinud,
\par 19 sellepärast ütleb vägede Issand, Iisraeli Jumal nõnda: Iialgi ei puudu Joonadabil, Reekabi pojal, mees, kes seisab minu palge ees.”

\chapter{36}

\par 1 Ja Juuda kuninga Joojakimi, Joosija poja neljandal aastal tuli Jeremijale sõna Issandalt, kes ütles:
\par 2 „Võta enesele rullraamat ja kirjuta sinna kõik need sõnad, mis ma sulle olen rääkinud Iisraeli ja Juuda ning kõigi rahvaste kohta, alates päevast, kui ma hakkasin sulle rääkima, Joosija päevist kuni tänase päevani!
\par 3 Võib olla, et kui Juuda sugu kuuleb kõigest sellest õnnetusest, mille ma kavatsen neile saata, siis nad pöörduvad igaüks oma kurjalt teelt ja ma võin neile nende süü ja nende patu andeks anda.”
\par 4 Siis Jeremija kutsus Baaruki, Neerija poja; ja Baaruk kirjutas rullraamatusse Jeremija suust kõik Issanda sõnad, mis ta oli temale rääkinud.
\par 5 Ja Jeremija käskis Baarukit, öeldes: „Mind peetakse kinni, mina ei saa minna Issanda kotta.
\par 6 Aga mine sina ja loe paastupäeval Issanda kojas rahva kuuldes rullraamatust Issanda sõnad, mis sa oled minu suust üles kirjutanud; ja loe need ka kogu Juuda kuuldes kõigile, kes tulevad oma linnadest!
\par 7 Vahest langeb nende alandlik palve Issanda ette ja igaüks pöördub oma kurjalt teelt; sest suur on viha ja raev, millega Issand seda rahvast on ähvardanud.”
\par 8 Ja Baaruk, Neerija poeg, tegi kõik nõnda, nagu prohvet Jeremija teda oli käskinud, ja luges raamatust Issanda sõnad Issanda kojas.
\par 9 Ja Juuda kuninga Joojakimi, Joosija poja viienda aasta üheksandas kuus kuulutati Issanda ees paast kogu Jeruusalemma rahvale ja kogu rahvale, kes Juuda linnadest oli tulnud Jeruusalemma.
\par 10 Siis luges Baaruk raamatust Jeremija sõnad Issanda kojas, kirjutaja Gemarja, Saafani poja kambris ülemises õues, Issanda koja uue värava suus, kogu rahva kuuldes.
\par 11 Kui Miika, Saafani poja Gemarja poeg, kuulis raamatust kõiki Issanda sõnu,
\par 12 siis ta läks alla kuningakotta kirjutaja kambrisse, ja vaata, seal istusid kõik vürstid: kirjutaja Elisama, Delaja, Semaja poeg, Elnatan, Akbori poeg, Gemarja, Saafani poeg, Sidkija, Hananja poeg, ja kõik muud vürstid.
\par 13 Ja Miika andis neile edasi kõik sõnad, mis ta oli kuulnud Baarukit rahva kuuldes raamatust lugevat.
\par 14 Siis läkitasid kõik vürstid Baaruki juurde Jehudi, Netanja poja, Kuusi poja Selemja pojapoja, et ta ütleks: „Võta kaasa rullraamat, millest sa rahva kuuldes lugesid, ja tule siia!” Ja Baaruk, Neerija poeg, võttis rullraamatu kaasa ning tuli nende juurde.
\par 15 Ja nad ütlesid temale: „Istu nüüd ja loe seda meie kuuldes!” Ja Baaruk luges nende kuuldes.
\par 16 Aga kui nad olid kuulnud kõiki sõnu, siis vaatasid nad kohkunult üksteisele otsa ja ütlesid Baarukile: „Me peame kuningale kõik need sõnad teatavaks tegema!”
\par 17 Ja nad küsitlesid Baarukit, öeldes: „Jutusta ometi meile, kuidas sa kõik need sõnad tema suust kirja panid?”
\par 18 Ja Baaruk vastas neile: „Oma suuga ütles ta mulle kõik need sõnad ette ja mina kirjutasin tindiga raamatusse.”
\par 19 Siis ütlesid vürstid Baarukile: „Mine peida ennast, sina ja Jeremija, ja keegi ärgu teadku, kus te olete!”
\par 20 Ja nad läksid kuninga juurde õue, aga rullraamatu jätsid nad kirjutaja Elisama kambrisse; ja nad kandsid kõik sõnad kuninga kõrvu.
\par 21 Siis kuningas läkitas Jehudi rullraamatut tooma ja too võttis selle kirjutaja Elisama kambrist; ja Jehudi luges kuninga kuuldes ja kõigi vürstide kuuldes, kes seisid kuninga juures.
\par 22 Kuningas istus talvekojas - oli üheksas kuu - ja tema ees oli süüdatud sütepann.
\par 23 Ja kui Jehudi oli lugenud kolm või neli veergu, siis lõikas kuningas kirjutaja noaga rullraamatu katki ja viskas sütepannil olevasse tulle, kuni kogu rullraamat oli hävinud sütepanni tules.
\par 24 Aga nad ei tundnud hirmu ega käristanud oma riideid lõhki, ei kuningas ega ükski tema teenritest, kes kuulsid kõiki neid sõnu.
\par 25 Ja kuigi Elnatan, Delaja ja Gemarja kuningat tungivalt palusid rullraamatut mitte põletada, ei võtnud ta neid kuulda.
\par 26 Ja kuningas käskis Jerahmeeli, kuningapoega, ja Serajat, Asrieli poega, ja Selemjat, Abdeeli poega, kirjutaja Baaruki ja prohvet Jeremija kinni võtta; aga Issand peitis nad ära.
\par 27 Ja Jeremijale tuli Issanda sõna, pärast seda kui kuningas oli põletanud rullraamatu ja sõnad, mis Baaruk Jeremija suust oli kirjutanud, ja ta ütles:
\par 28 „Võta enesele teine rullraamat ja kirjuta sinna kõik endised sõnad, mis olid esimeseski rullraamatus, mille Juuda kuningas Joojakim põletas!
\par 29 Ja Juuda kuninga Joojakimi kohta ütle: Nõnda ütleb Issand: Sina oled selle rullraamatu põletanud, öeldes: Mispärast sa oled sinna sedaviisi kirjutanud: Paabeli kuningas tuleb ja hävitab selle maa ning kaotab sealt inimesed ja loomad?
\par 30 Seepärast ütleb Issand Juuda kuninga Joojakimi kohta nõnda: Ei ole temal seda järglast, kes istuks Taaveti aujärjele, ja tema surnukeha visatakse päevapalavuse ja öökülma kätte.
\par 31 Ja ma karistan teda ja tema sugu ning ta sulaseid nende süütegude pärast; ma toon neile ning Jeruusalemma elanikele ja Juuda meestele kogu selle õnnetuse, mille ma neile tõotasin ja mida nad ei võtnud kuulda.”
\par 32 Siis Jeremija võttis teise rullraamatu ja andis selle kirjutaja Baarukile, Neerija pojale, ja tema kirjutas sinna Jeremija suust kõik selle raamatu sõnad, mille Juuda kuningas Joojakim oli tules põletanud; ja neile lisati veel palju samasuguseid sõnu.

\chapter{37}

\par 1 Ja Konja, Joojakimi poja asemel sai kuningaks Sidkija, Joosija poeg, kelle Paabeli kuningas Nebukadnetsar pani Juudamaale kuningaks.
\par 2 Aga ei tema ega ta sulased ega maa rahvas võtnud kuulda Issanda sõnu, mis ta rääkis prohvet Jeremija läbi.
\par 3 Kord läkitas kuningas Sidkija Jehukali, Selemja poja, ja preester Sefanja, Maaseja poja, prohvet Jeremijale ütlema: „Palu ometi meie eest Issandat, meie Jumalat!”
\par 4 Jeremija käis siis alles rahva keskel siia ja sinna, sest veel ei olnud teda vangikotta pandud.
\par 5 Aga Egiptusest oli välja tulnud vaarao sõjavägi; ja kui kaldealased, kes piirasid Jeruusalemma, kuulsid neist sõnumeid, siis nad läksid ära Jeruusalemma alt.
\par 6 Ja prohvet Jeremijale tuli Issanda sõna; ta ütles:
\par 7 „Nõnda ütleb Issand, Iisraeli Jumal: Öelge nõnda Juuda kuningale, kes teid läkitas minu juurde mind küsitlema: Vaata, vaarao sõjavägi, kes tuli teile appi, läheb tagasi oma maale, Egiptusesse.
\par 8 Ja kaldealased tulevad tagasi ning sõdivad selle linna vastu, ja nad vallutavad selle ning põletavad tulega.
\par 9 Nõnda ütleb Issand: Ärge petke iseendid, mõeldes: Kaldealased lähevad kindlasti ära meie kallalt! Sest nad ei lähe.
\par 10 Kuigi te lööksite kogu kaldealaste sõjaväge, kes sõdib teie vastu, ja neist jääksid järele ainult haavatud mehed, tõuseksid nad ometi püsti, igaüks oma telgis, ja põletaksid selle linna tulega maha.”
\par 11 Aga kui kaldealaste sõjavägi oli ära läinud Jeruusalemma alt vaarao sõjaväe pärast,
\par 12 läks Jeremija Jeruusalemmast välja, et minna Benjamini maale osa saama sealsest pärisosa jaotusest rahva keskel.
\par 13 Aga kui ta jõudis Benjamini väravasse, siis oli seal vahtkonnaülem, Jirija nimi, Hananja poja Selemja poeg, ja too võttis prohvet Jeremija kinni, öeldes: „Sa tahad põgeneda kaldealaste juurde!”
\par 14 Aga Jeremija vastas: „See on vale! Mina ei põgene kaldealaste juurde!” Kuid Jirija ei kuulanud teda, vaid võttis Jeremija kinni ja viis vürstide juurde.
\par 15 Ja vürstid vihastusid Jeremija peale ja nad peksid teda ning panid ta vangikotta kirjutaja Joonatani majasse, sest nad olid selle teinud vangikojaks.
\par 16 Nõnda tuli Jeremija kaevuhoonesse, võlvitud ruumi, ja Jeremija jäi sinna kauaks ajaks.
\par 17 Aga kord läkitas kuningas Sidkija tema järele ja laskis ta enese ette tuua; ja kuningas küsis temalt salaja oma kojas ning ütles: „Kas on Issandalt sõna?„ Ja Jeremija vastas: „On!” Ja ta ütles: ”Sind antakse Paabeli kuninga kätte!”
\par 18 Ja Jeremija ütles kuningas Sidkijale: „Mis pattu ma olen teinud sinu ja su sulaste ja selle rahva vastu, et te olete pannud mind vangikotta?
\par 19 Ja kus on teie prohvetid, kes teile kuulutasid, öeldes: Ei tule Paabeli kuningas teie ega selle maa kallale?
\par 20 Ja nüüd kuule ometi, mu isand kuningas! Lase ometi mu alandlik palve langeda sinu ette ja ära saada mind tagasi kirjutaja Joonatani kotta, et ma seal ei sureks!”
\par 21 Siis kuningas Sidkija andis käsu ja Jeremijat valvati vahtkonnaõues, temale anti kakuke leiba päevas pagarite tänavast, kuni linnast lõppes kõik leib; ja Jeremija jäi vahtkonnaõue.

\chapter{38}

\par 1 Aga Sefatja, Mattani poeg, ja Gedalja, Pashuri poeg, ja Juukal, Selemja poeg, ja Pashur, Malkija poeg, kuulsid neid sõnu, mis Jeremija rääkis kogu rahvale, öeldes:
\par 2 „Nõnda ütleb Issand: Kes jääb siia linna, see sureb mõõga läbi, nälja ja katku kätte; aga kes läheb välja kaldealaste juurde, see jääb elama, temale jääb alles ta hing ja ta võib elada.
\par 3 Nõnda ütleb Issand: See linn antakse kindlasti Paabeli kuninga sõjaväe kätte ja ta vallutab selle.”
\par 4 Siis ütlesid vürstid kuningale: „Surmatagu ometi see mees, sest ta teeb lõdvaks nende sõjameeste käed, kes siia linna on alles jäänud, ja kogu rahva käed, rääkides neile seesuguseid sõnu; sest see mees ei otsi sellele rahvale rahu, küll aga õnnetust!”
\par 5 Ja kuningas Sidkija vastas: „Vaata, ta on teie käes, kuningas ei saa ju teha midagi teie vastu!”
\par 6 Siis nad võtsid Jeremija ja viskasid ta kuningapoja Malkija kaevu, mis oli vahtkonnaõues; nad lasksid Jeremija köitega alla, aga kaevus ei olnud vett, vaid oli muda, ja Jeremija vajus mudasse.
\par 7 Kui etiooplane Ebed-Melek, hoovkondlane, kes viibis kuningakojas, kuulis, et nad olid Jeremija pannud kaevu - kuningas istus parajasti Benjamini väravas -,
\par 8 siis Ebed-Melek läks kuningakojast ja rääkis kuningaga, öeldes:
\par 9 „Mu isand kuningas! Need mehed on talitanud kurjasti kõiges, mis nad on teinud prohvet Jeremijale, et nad viskasid ta kaevu. Seal sureb ta nälga, sest linnas ei ole enam leiba.”
\par 10 Siis kuningas andis etiooplasele Ebed-Melekile käsu, öeldes: „Võta siit kolmkümmend meest enesele appi ja tõmba prohvet Jeremija kaevust üles, enne kui ta sureb!”
\par 11 Ja Ebed-Melek võttis mehed enesele appi ning läks kuningakotta, varakambri all olevasse ruumi, võttis sealt räbalaid ja kaltse ja laskis need köitega kaevu Jeremijale.
\par 12 Ja etiooplane Ebed-Melek ütles Jeremijale: „Pane nüüd need räbalad ja kaltsud enesele kaenlaaluseisse köite alla!” Ja Jeremija tegi nõnda.
\par 13 Siis nad tõmbasid Jeremija köitega üles ja võtsid ta kaevust välja; ja Jeremija jäi vahtkonnaõue.
\par 14 Ja kuningas Sidkija läkitas sõna ning laskis tuua prohvet Jeremija enese juurde, Issanda koja kolmanda sissekäigu juurde; ja kuningas ütles Jeremijale: „Ma tahan sinult küsida ühte asja, ära salga mulle midagi!”
\par 15 Aga Jeremija vastas Sidkijale: „Kui ma sulle midagi avaldan, kas sa siis tõesti mind ei tapa? Aga kuigi ma sulle nõu annaksin, sa ei kuulaks mind!”
\par 16 Siis kuningas Sidkija vandus Jeremijale salaja ja ütles: „Nii tõesti kui elab Issand, kes meile hinge on loonud, mina ei tapa sind ega anna sind nende meeste kätte, kes püüavad su hinge.”
\par 17 Siis Jeremija ütles Sidkijale: „Nõnda ütleb Issand, vägede Jumal, Iisraeli Jumal: Kui sa lähed vabatahtlikult välja Paabeli kuninga vürstide juurde, siis su hing jääb elama ja seda linna ei põletata tulega ning sina ja su sugu jääte elama.
\par 18 Aga kui sa ei lähe välja Paabeli kuninga vürstide juurde, siis antakse see linn kaldealaste kätte ja nad põletavad selle tulega ja sina ei pääse nende käest.”
\par 19 Siis kuningas Sidkija ütles Jeremijale: „Mina kardan neid juute, kes on põgenenud kaldealaste juurde; vahest antakse mind nende kätte ja nad teevad minuga halba nalja.”
\par 20 Aga Jeremija ütles: „Ei anta! Kuule ometi Issanda häält selles, mis mina sulle räägin, siis su käsi käib hästi ja su hing jääb elama!
\par 21 Aga kui sa tõrgud välja minemast, siis see on sõna, mille Issand mulle ilmutas:
\par 22 Vaata, kõik naised, kes on jäänud Juuda kuningakotta, viiakse välja Paabeli kuninga vürstide juurde ja nad ütlevad: „Sinu ustavad sõbrad on sind ahvatlenud ja on sinust jagu saanud: su jalad vajusid mudasse, aga nemad tõmbusid tagasi.”
\par 23 Kõik su naised ja lapsed viiakse kaldealaste juurde ja sa ei pääse nende käest, vaid Paabeli kuninga käsi tabab sind ja sinu pärast põletatakse see linn tulega.”
\par 24 Siis Sidkija ütles Jeremijale: „Keegi ärgu saagu teada neid sõnu, muidu sa sured!
\par 25 Ja kui vürstid kuulevad, et ma sinuga olen rääkinud, ja nad tulevad su juurde ning ütlevad sulle: Kõnele ometi meile, mida sa rääkisid kuningale, muidu me surmame sinu, ära meile salga, ja mida kuningas sulle rääkis,
\par 26 siis ütle neile: Ma lasksin langeda kuninga ette oma alandliku palve, et ta ei saadaks mind tagasi Joonatani kotta surema.”
\par 27 Kõik vürstid tulidki Jeremija juurde ja küsisid temalt; aga ta vastas neile just nende sõnadega, nagu kuningas oli käskinud. Siis nad läksid vaikides ära ta juurest, ja asi jäi saladusse.
\par 28 Ja Jeremija jäi vahtkonnaõue kuni päevani, mil Jeruusalemm vallutati.

\chapter{39}

\par 1 Ja kui Jeruusalemm oli vallutatud - Juuda kuninga Sidkija üheksandal aastal kümnendas kuus oli Paabeli kuningas Nebukadnetsar ja kogu tema sõjavägi tulnud Jeruusalemma vastu ja oli asunud seda piirama,
\par 2 ja oli Sidkija üheteistkümnendal aastal neljanda kuu üheksandal päeval tunginud linna sisse -,
\par 3 siis tulid kõik Paabeli kuninga vürstid ja asusid keskmisse väravasse: Neergal Sar-Eser, Samgar-Nebo, Sar-Sekim, ülemkammerhärra, Neergal Sar-Eser, ülemmaag, ja kõik muud Paabeli kuninga vürstid.
\par 4 Aga kui Juuda kuningas Sidkija ja kõik sõjamehed nägid neid, siis nad põgenesid ja läksid öösel linnast välja mööda kuninga rohuaia teed läbi müüridevahelise värava; ta läks välja lagendiku suunas.
\par 5 Ent kaldealaste sõjavägi ajas neid taga ja nad said Sidkija kätte Jeeriko lagendikel; nad võtsid ta kinni ja viisid Riblasse, Hamatimaale, Paabeli kuninga Nebukadnetsari juurde, kes mõistis tema üle kohut.
\par 6 Ja Paabeli kuningas laskis tappa Sidkija pojad Riblas tema silme ees; Paabeli kuningas tappis ka kõik Juuda suurnikud.
\par 7 Ja ta tegi Sidkija silmad pimedaks ning aheldas ta vaskahelaisse, et viia ta Paabelisse.
\par 8 Ja kaldealased põletasid tulega kuningakoja ja rahva kojad ning kiskusid maha Jeruusalemma müürid.
\par 9 Ja rahva säilinud osa, kes oli linna alles jäänud, ja ülejooksikud, kes olid tema poole üle jooksnud, ja muud rahva riismed viis ihukaitsepealik Nebusaradan vangi Paabelisse.
\par 10 Aga osa vaesemast rahvast, kellel midagi ei olnud, jättis ihukaitsepealik Nebusaradan Juudamaale ja andis neile samal ajal viinamägesid ning põlde.
\par 11 Ja Paabeli kuningas Nebukadnetsar oli andnud ihukaitsepealikule Nebusaradanile Jeremija kohta käsu, öeldes:
\par 12 „Võta tema ja kanna hoolt tema eest, ära tee talle midagi kurja, vaid talita temaga nõnda, nagu ta ise sulle ütleb!”
\par 13 Siis ihukaitsepealik Nebusaradan, ülemkammerhärra Nebusasban, ülemmaag Neergal Sar-Eser ja kõik Paabeli kuninga pealikud läkitasid talle järele -
\par 14 nad läkitasid talle järele ja lasksid tuua Jeremija vahtkonnaõuest ning andsid tema Gedalja, Saafani poja Ahikami poja hoolde, et ta viiks tema oma kotta; nõnda jäi ta rahva keskele.
\par 15 Ja Issanda sõna oli tulnud Jeremijale, kui teda vahtkonnaõues kinni peeti; ta oli öelnud:
\par 16 „Mine ja räägi etiooplase Ebed-Melekiga ning ütle: Nõnda ütleb vägede Issand, Iisraeli Jumal: Vaata, ma teen tõeks oma sõnad selle linna kohta, õnnetuseks, aga mitte õnneks, ja sel päeval sünnib see sinu nähes.
\par 17 Aga sinu ma päästan sel päeval, ütleb Issand, ja sind ei anta nende meeste kätte, keda sa kardad.
\par 18 Sest ma päästan su tõesti, nõnda et sa ei lange mõõga läbi, vaid saad enesele osaks oma hinge, sellepärast et sa oled lootnud minu peale, ütleb Issand.”

\chapter{40}

\par 1 Sõna, mis Jeremijale tuli Issandalt pärast seda, kui ihukaitsepealik Nebusaradan oli lasknud tal Raamast ära minna. Kui ihukaitsepealik laskis ta kohale tuua, oli ta käeraudadesse aheldatuna kõigi Jeruusalemma ja Juuda vangide seas, kes viidi Paabelisse.
\par 2 Ja ihukaitsepealik võttis Jeremija ning ütles temale: „Issand, sinu Jumal, on tõotanud selle õnnetuse sellele paigale.
\par 3 Issand on saatnud selle ja on teinud nõnda, nagu ta on rääkinud; sest te olete pattu teinud Issanda vastu ega ole kuulanud tema häält, ja nõnda on see asi teiega sündinud.
\par 4 Ja nüüd, vaata, ma vabastan su täna raudadest, mis sul on käte ümber. Kui sulle meeldib tulla koos minuga Paabelisse, siis tule ja ma kannan su eest hoolt; aga kui sulle ei meeldi tulla koos minuga Paabelisse, siis ära tule. vaata, kogu maa on su ees lahti: kuhu pead heaks ja õigeks minna, sinna mine!”
\par 5 Ja kui ta ei olnud veel tagasi läinud, ütles Nebusaradan: „Mine tagasi Gedalja, Saafani poja Ahikami poja juurde, kelle Paabeli kuningas pani Juuda linnade valitsejaks, ja jää tema juurde rahva sekka; või mine ükskõik kuhu paika sa heaks arvad minna!” Ja ihukaitsepealik andis temale teerooga ja kingituse ning saatis ta ära.
\par 6 Ja Jeremija tuli Gedalja, Ahikami poja juurde Mispasse ning jäi tema juurde rahva sekka, kes oli jäänud maale.
\par 7 Kui kõik sõjaväepealikud, kes olid väljal olnud, kuulsid, nemad ja nende mehed, et Paabeli kuningas oli pannud Gedalja, Ahikami poja, maavalitsejaks ja et ta oli andnud tema hoole alla need mehed ja naised ja lapsed, need maa vaesema rahva hulgast, keda ei viidud vangi Paabelisse,
\par 8 siis tulid nad Gedalja juurde Mispasse: Ismael, Netanja poeg, Joohanan ja Joonatan, Kaareahi pojad, Seraja, Tanhumeti poeg, netofalase Eefai pojad ja Jaasanja, maakatlase poeg, nemad ja nende sõjamehed.
\par 9 Ja Gedalja, Saafani poja Ahikami poeg, vandus neile ja nende sõjameestele, öeldes: „Ärge kartke kaldealasi teenida; jääge maale ja teenige Paabeli kuningat, siis on teil hea põli!
\par 10 Ja mina ise, vaata, jään Mispasse, et olla kaldealaste teenistuses, kes tulevad meie juurde; aga teie koguge veini ja suvist puuvilja ja õli ning pange oma astjaisse ja elage linnades, mis te võtate oma valdusesse!”
\par 11 Ja kõik needki juudid, kes olid Moabis, ammonlaste hulgas, Edomis ja kõigis teistes maades, kuulsid, et Paabeli kuningas oli Juudasse osa alles jätnud, ja et ta oli pannud neile valitsejaks Gedalja, Saafani poja Ahikami poja.
\par 12 Siis tulid kõik juudid tagasi kõigist paigust, kuhu neid oli aetud, ja tulid Juudamaale Gedalja juurde Mispasse; ja nad kogusid väga palju veini ja suvist puuvilja.
\par 13 Aga Joohanan, Kaareahi poeg, ja kõik sõjaväepealikud, kes olid väljal olnud, tulid Gedalja juurde Mispasse
\par 14 ja ütlesid temale: „Kas sa ka tead, et ammonlaste kuningas Baalis on läkitanud Ismaeli, Netanja poja, sind surnuks lööma?” Aga Gedalja, Ahikami poeg, ei uskunud neid.
\par 15 Siis Joohanan, Kaareahi poeg, rääkis salaja Gedaljaga Mispas, öeldes: „Lase ma lähen ja löön kellegi teadmata Ismaeli, Netanja poja maha! Miks ta peaks surnuks lööma sinu, millega kogu Juuda, kes on kogunenud sinu juurde, hajutatakse ja Juuda jääk hukkub?”
\par 16 Aga Gedalja, Ahikami poeg, vastas Joohananile, Kaareahi pojale: „Ära tee seda asja, sest sa räägid valet Ismaeli kohta!”

\chapter{41}

\par 1 Aga seitsmendas kuus tulid Ismael, Elisama poja Netanja poeg, kuninglikust soost ja üks kuninga pealikuid, ja kümme meest koos temaga Gedalja, Ahikami poja juurde Mispasse. Kui nad seal Mispas üheskoos leiba võtsid,
\par 2 siis tõusid Ismael, Netanja poeg, ja need kümme meest, kes olid koos temaga, ja lõid Gedaljat, Saafani poja Ahikami poega mõõgaga ning surmasid tema, kelle Paabeli kuningas oli pannud maavalitsejaks.
\par 3 Ka kõik juudid, kes olid Gedalja juures Mispas, ja kaldealased, sõjamehed, kes juhtusid seal olema, lõi Ismael maha.
\par 4 Ja teisel päeval, kui Gedalja oli tapetud, aga keegi seda veel ei teadnud,
\par 5 tuli Sekemist, Siilost ja Samaariast kaheksakümmend meest, habemed aetud, riided lõhki käristatud ja märgid ihusse lõigatud, käes roaohvrid ja viiruk Issanda kotta viimiseks.
\par 6 Ja Ismael, Netanja poeg, läks Mispast neile vastu lakkamatult nuttes; ja kui ta neid kohtas, ütles ta neile: „Tulge Gedalja, Ahikami poja juurde!”
\par 7 Aga kui nad olid jõudnud linna keskele, siis Ismael, Netanja poeg, ja mehed, kes olid koos temaga, tapsid nad ära ja viskasid kaevu.
\par 8 Aga nende hulgas leidus kümme meest, kes ütlesid Ismaelile: „Ära meid surma, sest meil on peidetud varandusi väljal: nisu ja otri, õli ja mett!” Siis ta jättis need puutumata ega surmanud neid koos nende vendadega.
\par 9 Ja kaev, kuhu Ismael viskas kõik nende meeste surnukehad, keda ta oli maha löönud, nende seas Gedalja kaaslased, oli seesama, mille kuningas Aasa oli teinud Iisraeli kuninga Baesa pärast; selle täitis Ismael, Netanja poeg, mahalöödutega.
\par 10 Ja Ismael võttis vangi kogu rahva jäägi, kes oli Mispas, kuningatütred ja kogu rahva, kes olid jäänud Mispasse, keda ihukaitsepealik Nebusaradan oli jätnud Gedalja, Ahikami poja hoolde; Ismael, Netanja poeg, võttis need vangi ja läks teele, et minna ammonlaste juurde.
\par 11 Aga kui Joohanan, Kaareahi poeg, ja kõik sõjaväepealikud, kes olid koos temaga, kuulsid kõigest sellest kurjast, mida Ismael, Netanja poeg, oli teinud,
\par 12 siis võtsid nad kõik oma mehed ja läksid sõdima Ismaeli, Netanja poja vastu; ja nad leidsid tema suure vee äärest, mis on Gibeonis.
\par 13 Aga kui kõik rahvas, kes oli Ismaeli juures, nägi Joohanani, Kaareahi poega, ja kõiki sõjaväepealikuid, kes olid koos temaga, siis nad rõõmustasid,
\par 14 ja kõik rahvas, kelle Ismael Mispast oli vangi võtnud, pöördus ja läks tagasi ning tuli Joohanani, Kaareahi poja juurde.
\par 15 Aga Ismael, Netanja poeg, pääses Joohanani eest kaheksa mehega ja läks ammonlaste juurde.
\par 16 Ja Joohanan, Kaareahi poeg, ja kõik sõjaväepealikud, kes olid koos temaga, võtsid kogu ülejäänud rahva, kelle ta oli toonud Mispast Ismaeli, Netanja poja käest, pärast seda kui see oli maha löönud Gedalja, Ahikami poja: mehed, sõjakõlvulised mehed, naised ja lapsed ja teenrid, keda ta oli Gibeonist tagasi toonud,
\par 17 ja läksid ning peatusid Kimhami majutuspaigas, mis on Petlemma lähedal, et siis minna ja jõuda Egiptusesse
\par 18 kaldealaste eest, sest nad kartsid neid, kuna Ismael, Netanja poeg, oli maha löönud Gedalja, Ahikami poja, kelle Paabeli kuningas oli pannud maavalitsejaks.

\chapter{42}

\par 1 Siis astusid ligi kõik sõjaväepealikud ja Joohanan, Kaareahi poeg, ja Jesanja, Hoosaja poeg, ja kogu rahvas pisemast suuremani,
\par 2 ja ütlesid prohvet Jeremijale: „Langegu nüüd meie alandlik palve sinu ette ja sina palu meie eest Issandat, oma Jumalat, kogu selle jäägi eest; sest suurest hulgast on meid pisut järele jäänud, nagu sa meid oma silmaga näed!
\par 3 Issand, su Jumal, andku meile teada tee, mida meil tuleks käia, ja seda, mida meil tuleks teha!”
\par 4 Ja prohvet Jeremija vastas neile: „Küllap ma kuulen! Vaata, ma palun Issandat, teie Jumalat, nõnda nagu soovite, ja ma ilmutan teile iga sõna, mis Issand teile kostab, ega salga teie ees sõnagi!”
\par 5 Siis nad ütlesid Jeremijale: „Issand olgu meie vastu tõsine ja ustav tunnistaja, kui me ei tee iga sõna järgi, millega Issand, su Jumal, läkitab sind meie juurde!
\par 6 Olgu see hea või kuri, me kuulame Issanda, oma Jumala häält. Tema juurde me läkitame sinu, et meil võiks olla hea põli, kui me kuulame Issanda, oma Jumala häält.”
\par 7 Kümne päeva pärast tuli Jeremijale Issanda sõna
\par 8 ja ta kutsus Joohanani, Kaareahi poja, ja kõik sõjaväepealikud, kes olid koos temaga, ja kogu rahva pisemast suuremani
\par 9 ning ütles neile: „Nõnda ütleb Issand, Iisraeli Jumal, kelle juurde te mind läkitasite, et ma paneksin teie alandliku palve tema ette:
\par 10 Kui te jääte siia maale, siis ma ehitan teid ega kisu maha, siis ma istutan teid ega kitku välja, sest ma kahetsen kurja, mida ma teile olen teinud.
\par 11 Ärge kartke Paabeli kuningat, keda te nüüd kardate; ärge teda kartke, ütleb Issand, sest mina olen teiega, et teid aidata ja tema käest päästa!
\par 12 Ja mina annan teile armu, et tema halastab teie peale ja laseb teid tulla tagasi teie oma maale.
\par 13 Aga kui te ütlete: Me ei jää siia maale! ega võta kuulda Issanda, oma Jumala häält,
\par 14 vaid ütlete: Ei, me läheme tõesti Egiptusesse, kus meil ei ole vaja näha sõda ega kuulda sarvehäält ja kus meil ei ole leivanälga, ja me jäämegi sinna -
\par 15 siis kuule seepärast nüüd Issanda sõna, Juuda jääk: Nõnda ütleb vägede Issand, Iisraeli Jumal: Kui te tõesti pöörate oma näod Egiptuse poole ja lähete sinna võõraina elama,
\par 16 siis saab mõõk, mida te kardate, teid kätte seal, Egiptusemaal, ja nälg, mille pärast te olete mures, tuleb teile järele sinna, Egiptusesse, ja te surete seal.
\par 17 Ja kõik need mehed, kes pööravad oma näo Egiptuse poole, et seal võõraina elada, surevad mõõga läbi, nälga ja katku, ja ükski neist ei saa põgeneda ega pääse õnnetuse eest, mille ma neile saadan.
\par 18 Sest nõnda ütleb vägede Issand, Iisraeli Jumal: Otsekui mu viha ja raev on valatud Jeruusalemma elanike peale, nõnda valatakse mu raev teie peale, kui te lähete Egiptusesse, ja te saate sajatuseks ja hirmutuseks, needuseks ja teotuseks, ja te ei saa enam näha seda paika.
\par 19 Issand ütleb teile, Juuda jääk: Ärge minge Egiptusesse! Olgu teil hästi teada, et ma täna olen teid hoiatanud!
\par 20 Sest te petsite iseendid, kui te läkitasite mind Issanda, oma Jumala juurde, öeldes: Palu meie eest Issandat, meie Jumalat; ja mida iganes Issand, meie Jumal, ütleb, seda ilmuta meile ja me teeme nõnda!
\par 21 Aga olles teile täna ilmutanud, te siiski ei kuula Issanda, oma Jumala häält kõiges, milleks ta mind teie juurde on läkitanud.
\par 22 Ja nüüd teadke hästi, et mõõga, nälja ja katku läbi te surete seal paigas, kuhu te tahate minna võõraina elama.”

\chapter{43}

\par 1 Ja kui Jeremija oli lõpuni rääkinud kogu rahvale kõik Issanda, nende Jumala sõnad, millega Issand, nende Jumal, oli teda läkitanud nende juurde, kõik need sõnad,
\par 2 siis ütlesid Asarja, Hoosaja poeg, ja Joohanan, Kaareahi poeg, ja kõik ülbed mehed - need ütlesid Jeremijale: „Sa räägid valet! Issand, meie Jumal, ei ole sind läkitanud ütlema: Ärge minge Egiptusesse võõraina elama,
\par 3 vaid Baaruk, Neerija poeg, kihutab sind meie vastu, et meid anda kaldealaste kätte, selleks et nad meid surmaksid või viiksid vangi Paabelisse!”
\par 4 Ja Joohanan, Kaareahi poeg, ja kõik sõjaväepealikud ja kogu rahvas ei kuulanud Issanda häält, et nad jääksid Juudamaale,
\par 5 vaid Joohanan, Kaareahi poeg, ja kõik sõjaväepealikud võtsid kõik Juuda järelejäänud, kes olid tagasi tulnud kõigi rahvaste keskelt, kuhu neid oli tõugatud, et elada Juudamaal:
\par 6 mehed ja naised ja lapsed ja kuningatütred ja kõik hinged, keda ihukaitsepealik Nebusaradan oli jätnud Gedalja, Saafani poja Ahikami poja juurde, ja prohvet Jeremija ja Baaruki, Neerija poja,
\par 7 ja nad läksid Egiptusemaale, sest nad ei kuulanud Issanda häält; ja nad jõudsid Tahpanheesini.
\par 8 Ja Tahpanheesis tuli Jeremijale Issanda sõna; ta ütles:
\par 9 „Võta oma käega suuri kive ja peida need savisse sillutises, mis on vaarao koja sissekäigu juures Tahpanheesis, Juuda meeste nähes
\par 10 ja ütle neile: Nõnda ütleb vägede Issand, Iisraeli Jumal: Vaata, ma läkitan talle järele ja võtan Paabeli kuninga Nebukadnetsari, oma sulase, ja panen tema aujärje nende kivide peale, mis ma olen peitnud, ja ta püstitab nende peale oma toreda telgi.
\par 11 Tema tuleb ja lööb Egiptusemaad: kes surma, see surma, kes vangi, see vangi, ja kes mõõga kätte, see mõõga kätte!
\par 12 Mina süütan tule Egiptuse jumalate templites ja tema põletab need ning viib nad ära; ta puhastab Egiptusemaad, nagu karjane puhastab oma kuube söödikuist; ja ta läheb sealt ära rahuga.
\par 13 Ja ta lõhub Egiptuses Beet-Semesi sambad ning põletab tulega Egiptuse jumalate templid.”

\chapter{44}

\par 1 Sõna, mis Jeremijale tuli kõigi juutide kohta, kes elasid Egiptusemaal, kes elasid Migdolis, Tahpanheesis, Noofis ja Patrosemaal; ta ütles:
\par 2 „Nõnda ütleb vägede Issand, Iisraeli Jumal: Te olete näinud kõike seda õnnetust, mille ma lasksin tulla Jeruusalemmale ja kõigile Juuda linnadele; ja vaata, need on tänapäevani varemeis ja ükski ei ela neis
\par 3 nende pahategude pärast, mis nad tegid minu pahandamiseks, minnes suitsutama ja teenima teisi jumalaid, keda nad ei tundnud, ei nemad, ei teie ega teie vanemad.
\par 4 Ma läkitasin küll teie juurde kõik oma sulased prohvetid, läkitasin lakkamatult, öeldes: Ärge ometi tehke seda jõledust, mida ma vihkan!
\par 5 Aga nemad ei kuulanud ega pööranud kõrva, et nad oleksid pöördunud oma kurjusest ega oleks suitsutanud teistele jumalatele.
\par 6 Seepärast valati välja mu raev ja viha ja see põles Juuda linnades ja Jeruusalemma tänavail ja need muutusid neiks õudseiks varemeiks, nagu need tänapäeval on.
\par 7 Ja nüüd ütleb Issand, vägede Jumal, Iisraeli Jumal, nõnda: Miks valmistate iseendile selle suure õnnetuse, hävitades oma mehed ja naised, lapsed ja imikud Juudast, ilma endile jääkigi järele jätmata,
\par 8 pahandades mind oma kätetööga, suitsutades teistele jumalatele Egiptusemaal, kuhu te olete tulnud võõraina elama, hävitades endid sellega ning saades sajatuseks ja teotuseks kõigi maa rahvaste seas?
\par 9 Kas te olete unustanud oma vanemate pahateod, Juuda kuningate pahateod, omaenese pahateod ja oma naiste pahateod, mis nad tegid Juudamaal ja Jeruusalemma tänavail?
\par 10 Nad ei ole murdunud tänapäevani ja nad ei karda ega käi minu Seaduse ja mu määruste järgi, mis ma olen andnud teile ja teie vanemaile.
\par 11 Seepärast ütleb vägede Issand, Iisraeli Jumal, nõnda: Vaata, ma pööran oma palge teie vastu õnnetuseks ja kogu Juuda hävitamiseks.
\par 12 Ja ma võtan Juuda jäägi, kes pöörasid oma näod Egiptusemaa poole, et seal võõraina elada, ja nad kõik hukkuvad; Egiptusemaal nad langevad mõõga ja nälja läbi; nad hukkuvad pisemast suuremani, nad surevad mõõga ja nälja läbi ning saavad sajatuseks, hirmutuseks, needuseks ja teotuseks.
\par 13 Ja ma karistan neid, kes elavad Egiptusemaal, nõnda nagu ma karistasin Jeruusalemma mõõga, nälja ja katkuga.
\par 14 Ja Juuda jäägist, kes tulid Egiptusemaale võõraina elama, ei jää järele põgenikke ega pääsenuid, et pöörduda tagasi Juudamaale, kuhu nende hing igatseb minna jälle elama; sest nad ei pöördu tagasi, peale mõne päästetu.”
\par 15 Aga kõik mehed, kes teadsid, et nende naised olid suitsutanud teistele jumalatele, ja kõik naised, kes seal seisid suure koguna, ja kõik rahvas, kes elas Egiptusemaal Patroses, kostsid Jeremijale, öeldes:
\par 16 „Sõna, mis sa meile oled rääkinud Issanda nimel, ei võta me sinult kuulda,
\par 17 vaid me teeme teoks iga sõna, mis on tulnud meie eneste suust, suitsutame Taevakuningannale ja valame temale joogiohvreid, nõnda nagu me oleme teinud, meie ja meie vanemad, meie kuningad ja meie vürstid Juuda linnades ja Jeruusalemma tänavail; siis oli meil leiba külluses ja hea põli, ja me ei näinud õnnetust.
\par 18 Aga sellest peale, kui me lakkasime suitsutamast Taevakuningannale ja valamast temale joogiohvreid, on meil olnud puudus kõigest ja me oleme hukkunud mõõga ja nälja läbi.
\par 19 Ja kui me suitsutame Taevakuningannale ja valame temale joogiohvreid, kas me siis vastu oma meeste tahtmist valmistame temale ohvrileibu, tema kuju taolisi, ja valame temale joogiohvreid?”
\par 20 Siis Jeremija kostis kogu rahvale, meestele ja naistele, kogu rahvale, kes temale nõnda oli vastanud, ja ütles:
\par 21 „Eks see olnud suits, mida te suitsutasite Juuda linnades ja Jeruusalemma tänavail, teie ja teie vanemad, teie kuningad ja teie vürstid ja maa rahvas, mida Issand meenutas ja võttis oma südamesse,
\par 22 nõnda et Issand enam ei talunud teie kurje tegusid, jäledusi, mis te tegite, ja nõnda sai teie maa varemeiks, hirmutuseks ja needuseks, elanikest tühjaks, nõnda nagu see tänapäeval on.
\par 23 Sellepärast et te suitsutasite ja tegite pattu Issanda vastu ja et te ei kuulanud Issanda häält ega käinud tema Seaduse ja määruste ning tunnistuste järgi, sellepärast on teid tabanud see õnnetus, nõnda nagu see tänapäeval on.”
\par 24 Ja Jeremija ütles kogu rahvale ja kõigile naistele: „Kuulge Issanda sõna, kõik juudid, kes olete Egiptusemaal!
\par 25 Nõnda räägib vägede Issand, Iisraeli Jumal, ja ütleb: Teie ja teie naised olete rääkinud oma suuga, olete viinud täide oma kätega ja olete öelnud: Me teeme teoks oma tõotused, millega me oleme tõotanud suitsutada Taevakuningannale ja valada temale joogiohvreid. Pidage siis kindlasti oma tõotusi ja tehke oma tõotused teoks!
\par 26 Seepärast kuulge Issanda sõna, kõik juudid, kes te elate Egiptusemaal: Vaata, ma olen vandunud oma suure nime juures, ütleb Issand, et minu nime ei nimetata enam kogu Egiptusemaal mitte ühegi Juuda mehe suus, kes võiks öelda: Nii tõesti kui Issand Jumal elab!
\par 27 Vaata, ma olen nende pärast valvel õnnetuseks, aga mitte õnneks; ja kõik Egiptusemaal olevad Juuda mehed hukkuvad mõõga ja nälja läbi, kuni neile on tulnud lõpp.
\par 28 Aga mõõga eest pääsenud pöörduvad Egiptusemaalt tagasi Juudamaale väikesearvuliselt; siis kogu Juuda jääk, kes on tulnud Egiptusemaale võõraina elama, saab teada, kelle sõna on tõeks läinud - minu või nende.
\par 29 Ja see olgu teile tähiseks, ütleb Issand, kui ma teid nuhtlen siin paigas selleks, et te teaksite, et mu sõnad teie kohta lähevad tõesti täide teile õnnetuseks:
\par 30 Nõnda ütleb Issand: Vaata, ma annan Egiptuse kuninga, vaarao Hofra, tema vaenlaste kätte, nende kätte, kes püüavad tema elu, nõnda nagu ma andsin Juuda kuninga Sidkija Paabeli kuninga Nebukadnetsari kätte, kes oli tema vaenlane ja ähvardas tema elu.”

\chapter{45}

\par 1 Sõna, mis prohvet Jeremija rääkis Baarukile, Neerija pojale, kui see kirjutas raamatusse neid sõnu Jeremija suust Juuda kuninga Joojakimi, Joosija poja neljandal aastal; ta ütles:
\par 2 „Nõnda ütleb Issand, Iisraeli Jumal, sinu kohta, Baaruk:
\par 3 Sa ütled: „Häda mulle! Sest Issand lisab mu valule piina! Ma olen väsinud ohkamisest, ma ei leia hingamist!”
\par 4 Ütle siis temale nõnda: Nõnda ütleb Issand: Vaata, mis ma olen ehitanud, selle ma kisun maha, ja mis ma olen istutanud, selle ma kitkun välja - ja nimelt kogu maa!
\par 5 Ja sina nõuad enesele suuri asju! Ära nõua! Sest vaata, ma toon õnnetuse kõigile, ütleb Issand. Ometi ma jätan sulle osaks su elu kõigis paigus, kuhu sa lähed.”

\chapter{46}

\par 1 Issanda sõna, mis tuli prohvet Jeremijale rahvaste kohta:
\par 2 Egiptuse kohta - Egiptuse kuninga, vaarao Neko sõjaväe kohta, kes oli Frati jõe ääres Karkemises, keda Paabeli kuningas Nebukadnetsar lõi Juuda kuninga Joojakimi, Joosija poja neljandal aastal:
\par 3 „Seadke valmis kilp ja kaitsevari ning tulge sõtta!
\par 4 Rakendage hobused! Ratsanikud, istuge selga! Asuge kohtadele, kiivrid peas! Haljastage piigid, riietuge soomusrüüdesse!
\par 5 Mis ma näen? Neil on hirm, nad taanduvad? Nende sangarid on löödud, nad põgenevad üha ega vaata tagasi. Hirm on igal pool - ütleb Issand.
\par 6 Ei saa pakku kiire ega pääse vägev; põhja pool, Frati jõe ääres, nad komistavad ja langevad.
\par 7 Kes see on, kes tõuseb nagu Niilus, kelle veed voogavad jõgedena?
\par 8 Egiptus tõuseb nagu Niilus ja ta veed voogavad jõgedena. Ta ütleb: „Ma tõusen, katan maa. Ma hävitan linna ja selle elanikud.”
\par 9 Sööstke püsti, hobused, kihutage pööraselt, sõjavankrid! Minge välja, võitlejad: etiooplased ja puudid, kilbikandjad, ja luudid, vilunud ammumehed!
\par 10 See päev on Issandal, vägede Issandal, kättemaksupäev oma vaenlastele tasumiseks. Mõõk sööb, küllastub ja joobub nende verest; sest Issandal, vägede Issandal, on tapaohver põhjamaal Frati jõe ääres.
\par 11 Mine üles Gileadi ja võta palsamit, neitsi, Egiptuse tütar! Asjata tarvitad palju ravimeid - sinul ei ole paranemist.
\par 12 Rahvad kuulevad su häbist ja su hädakisa täidab maa; sest vägev komistab vägeva peale ja üheskoos langevad mõlemad.”
\par 13 Sõna, mis Issand rääkis prohvet Jeremijale Paabeli kuninga Nebukadnetsari tulekust Egiptusemaad lööma:
\par 14 „Andke teada Egiptuses ja kuulutage Migdolis, kuulutage Noofis ja Tahpanheesis, öelge: Astu ette ja ole valmis, sest mõõk õgib juba su ümber!
\par 15 Miks põgenes Apis? Ei jäänud püsima su härg, sest Issand ajas selle ära.
\par 16 Ta pani komistama paljusid, kes ka langesid üksteise peale ja ütlesid: „Tõuskem ja mingem tagasi oma rahva juurde ja oma sünnimaale, kaugele hävitava mõõga eest!”
\par 17 Hüüdke vaaraole, Egiptuse kuningale: Kiidukukk, kes laskis silmapilgu mööda!
\par 18 Nii tõesti kui ma elan, ütleb kuningas, kelle nimi on vägede Issand - tõesti, otsekui Taabor mägede keskel ja Karmel mere ääres - tuleb tema.
\par 19 Sea oma kraam vangiteekonnaks valmis, sina, kes sa seal elad, Egiptuse tütar! Sest Noof muutub õudseks, elaniketa varemeiks.
\par 20 Egiptus on ilus õhvake, aga põhja poolt tuleb parm, tuleb.
\par 21 Palgasõduridki tema keskel on otsekui nuumvasikad; sest needki pöörduvad ja põgenevad üheskoos ega pea vastu, kui neile tuleb õnnetusepäev, nende karistusaeg.
\par 22 Tema hääl on kui vingerdava ussi sisin. Sest nad saabuvad sõjaväega ja tulevad kirvestega tema kallale otsekui puuraiujad.
\par 23 Nad raiuvad maha tema metsa, ütleb Issand, kuigi see on läbitungimatu; sest neid on rohkem kui rohutirtse ja ükski ei suuda neid lugeda.
\par 24 Häbisse jääb Egiptuse tütar, ta antakse põhjamaa rahva kätte.
\par 25 Vägede Issand, Iisraeli Jumal, on öelnud: Vaata, ma karistan Noo Aamonit, ja vaaraod, Egiptust ning selle jumalaid ja kuningaid, vaaraod ja neid, kes loodavad tema peale.
\par 26 Ja ma annan nad nende kätte, kes püüavad nende elu, Paabeli kuninga Nebukadnetsari ja tema sulaste kätte. Aga pärast seda elatakse seal nagu muistseil päevil, ütleb Issand.
\par 27 Aga sina, mu sulane Jaakob, ära karda, ja Iisrael, ära ehmu! Sest vaata, ma päästan sind kaugelt ja sinu soo tema vangipõlvemaalt. Jaakob tuleb tagasi ning elab rahus ja muretult, ilma et keegi teda hirmutaks.
\par 28 Sina, mu sulane Jaakob, ära karda, ütleb Issand, sest mina olen sinuga. Sest ma teen lõpu kõigile rahvaile, kelle sekka ma olen sind pillutanud; aga sinule ma ei tee lõppu: ma karistan sind õiglaselt, aga hoopis karistamata ma sind küll ei jäta.”

\chapter{47}

\par 1 Issanda sõna, mis tuli prohvet Jeremijale vilistite kohta, enne kui vaarao lõi Assat.
\par 2 Nõnda ütleb Issand: „Vaata, veed tõusevad põhja poolt ja neist saab tulvav jõgi; need ujutavad üle maa ja kõik, mis seda täidab, linna ja selle rahva. Siis inimesed kisendavad ja kõik maa elanikud uluvad.
\par 3 Kui kostab tema täkkude kabjaplagin, tema vankrite mürin, tema rataste ragin, siis ei vaata isad oma laste järele - nii jõuetud on nende käed
\par 4 selle päeva pärast, mis tuleb hävitama kõiki vilisteid, kaotama Tüüroselt ja Siidonilt kõiki abistajaid, kes on neile veel jäänud; sest Issand hävitab vilistid, allesjäänud Kaftoori saarelt.
\par 5 Kiilaspäisus tabab Assat, vaikima peab Askelon! Jääk nende orus, kui kaua sa tahad lõigata oma ihu?
\par 6 Oh, Issanda mõõk, kui kaua ei ole sul rahu? Mine tagasi oma tuppe, püsi paigal ja ära liigu!
\par 7 Kuidas sa saad olla rahulik? Sest Issand on andnud temale käsu ning on määranud ta sinna Askeloni ja mereranna vastu.”

\chapter{48}

\par 1 Moabi kohta: Nõnda ütleb vägede Issand, Iisraeli Jumal: „Häda Nebole, sest ta hävitatakse! Häbisse jääb, vallutatakse Kirjataim, häbisse jääb ja purustatakse pelgupaik!
\par 2 Moabi kuulsus kaob, Hesbonis kavatsetakse kurja tema vastu: „Mingem ja hävitagem ta rahvaste hulgast!” Ka sina, Madmen, hukkud - su kannul käib mõõk!
\par 3 Kuule! Hooronaimist kostab kisa: „Hävitus ja suur kokkuvarisemine!”
\par 4 „Moab on purustatud!” kostab ta lapsukeste kisa.
\par 5 Jah, mööda Luuhiti tõusuteed minnakse nuttes üles, jah, Hooronaimi nõlvakult kostab hädakisa hävingu pärast:
\par 6 „Põgenege, päästke oma elu ja olge nagu kadakas kõrbes!”
\par 7 Sellepärast et sa loodad oma tegude ja varanduste peale, vallutatakse ka sind. Ja Kemos läheb vangi üheskoos oma preestrite ja vürstidega.
\par 8 Hävitaja tuleb iga linna kallale ja ükski linn ei pääse; org hukkub ja tasandik rikutakse, nagu Issand on ütelnud.
\par 9 Andke Moabile tiivad, ta lennaku ära! Ta linnad muutuvad õudseks, keegi ei ela neis.
\par 10 Neetud olgu, kes Issanda tööd teeb loiult, ja neetud olgu, kes hoiab oma mõõga verest eemal!
\par 11 Moab on olnud muretu oma noorusest alates ja on vaikselt seisnud oma pärmi peal; teda ei ole kallatud astjast astjasse ega ole ta pidanud minema vangi; seepärast on talle jäänud alles ta maitse ja ta lõhn ei ole muutunud.
\par 12 Seepärast, vaata, päevad tulevad, ütleb Issand, mil ma läkitan tema kallale veinilaskjad, kes teda kallutavad; need tühjendavad tema astjad ja purustavad tema kruusid.
\par 13 Siis Moab tunneb häbi Kemose pärast, nagu Iisraeli sugu tundis häbi Peeteli pärast, kelle peale nad lootsid.
\par 14 Kuidas te võite öelda: „Me oleme vägevad ja vaprad sõjamehed!”?
\par 15 Moab hävitatakse ja ta linnad tõusevad üles kui suits; ta valitud noored mehed laskuvad tapetavaiks, ütleb kuningas, kelle nimi on vägede Issand.
\par 16 Moabi hukatus ligineb ja tema õnnetus tõttab väga.
\par 17 Tundke temale kaasa, kõik ta naabrid ja kõik, kes tunnete tema nime, öelge: „Kuidas küll on murtud tugev kepp, see tore sau!”
\par 18 Astu alla oma auhiilgusest ja istu janus, elanik, Diiboni tütar! Sest Moabi hävitaja tuleb su kallale, hävitama su kindlustatud linnu.
\par 19 Seisa tee ääres ja vaata, Aroeri elanik! Küsi põgenikelt ja pääsenuilt: „Mis on juhtunud?”
\par 20 Moab on jäänud häbisse, sest ta on purustatud. Hüüdke ja karjuge, andke teada Arnoni ääres, et Moab on hävitatud.
\par 21 Ja kohus tuleb tasasele maale, Holonile, Jahsale, Meefatile,
\par 22 Diibonile, Nebole, Beet-Diblataimile,
\par 23 Kirjataimile, Beet-Gaamulile, Beet-Meonile,
\par 24 Kerijotile, Bosrale ja kõigile Moabimaa linnadele kaugel ja ligidal.
\par 25 Moabi sarv raiutakse maha ja tema käsivars murtakse, ütleb Issand.
\par 26 Jootke ta purju, sest ta on suurustanud Issanda vastu, et Moab vehiks kätega oma okse juures ja et temagi saaks naeruks!
\par 27 Eks ole Iisrael olnud sulle naeruks? Või tabati ta varaste hulgast, et sa vangutad pead iga kord, kui sa temast räägid?
\par 28 Jätke linnad ja elage kaljudes, Moabi elanikud, olge nagu tuvi, kes pesitseb koopasuus!
\par 29 Me oleme kuulnud Moabi kõrkusest - ta on väga ülbe, tema uhkusest, hooplemisest ja upsakusest ning tema südame suurelisusest.
\par 30 Mina tunnen, ütleb Issand, tema jultumust ja valejutte - nad on teinud valesti.
\par 31 Seetõttu ulun ma Moabi pärast, kisendan kogu Moabi pärast, kurdan Kiir-Heresi inimeste pärast.
\par 32 Rohkem kui on nuttu Jaaseri pärast, nutan ma sinu pärast, Sibma viinapuu. Su lokkavad kasvud tungisid üle mere, ulatusid Jaaseri mereni. Su suvivilja ja viinamarjalõikuse kallale sööstab hävitaja.
\par 33 Rõõm ja ilutsemine võetakse viljapuuaiast ja Moabimaalt. Ma lõpetan veini surutõrrest, enam ei tallata seda rõõmuhüüetega, rõõmuhüüd ei olegi enam rõõmuhüüd.
\par 34 Hädahüüded on Hesbonist kuni Elaaleni ja Jahaseni, nad annavad kuulda oma häält Soarist kuni Hooronaimini ja Eglat-Selisijani, sest ka Nimrimi veed muutuvad kõrbeks.
\par 35 Ja ma lõpetan Moabist selle, ütleb Issand, kes ohverdab künkal ja suitsutab oma jumalatele.
\par 36 Seetõttu kaebleb mu süda Moabi pärast otsekui vilepill; ja mu süda kaebleb Kiir-Heresi inimeste pärast otsekui vilepill, et nende kogutud vara on hävinud.
\par 37 Tõesti, kõik pead on siis paljad ja kõik habemed aetud, kõigil kätel on sisselõiked ja kõigil niudeil kotiriie.
\par 38 Kõigil Moabi katuseil ja tänavail on üldine leinakaebus, sest ma purustan Moabi otsekui kõlbmatu astja, mis kellelegi ei meeldi, ütleb Issand.
\par 39 Kuidas ta küll on purustatud! Ulguge! Kuidas küll on Moab häbiga pööranud selja! Moab saab naeruks ja hirmutuseks kõigile oma naabritele.
\par 40 Sest nõnda ütleb Issand: Vaata, ta lendab nagu kotkas ja laotab oma tiibu Moabi kohal.
\par 41 Linnad vallutatakse ja kindlused saadakse kätte: sel päeval on Moabi kangelaste süda otsekui lapsevaevas oleva naise süda.
\par 42 Ja Moab hävitatakse rahvaste seast, sest ta on suurustanud Issanda vastu.
\par 43 Pelg, püügiauk ja püüdepael on su ees, Moabi elanik, ütleb Issand.
\par 44 Kes siis põgeneb pelju eest, langeb püügiauku; ja kes tuleb välja püügiaugust, selle püüab püüdepael; sest ma toon temale, Moabile, nende karistusaasta, ütleb Issand.
\par 45 Hesboni varjus seisavad põgenikud jõuetult; sest Hesbonist lähtub tuli ja Siihoni keskelt leek: see põletab Moabi oimud ja lärmajate pealae.
\par 46 Häda sulle, Moab! Kaduma peab Kemose rahvas, sest su pojad võetakse vangi ja su tütred peavad olema vangis!
\par 47 Aga viimseil päevil ma pööran Moabi vangipõlve, ütleb Issand.” Niipalju kohtust Moabi üle.

\chapter{49}

\par 1 Ammonlaste kohta: Nõnda ütleb Issand: „Kas Iisraelil ei ole poegi või ei ole tal pärijat? Miks Milkom on pärinud Gaadi ja ta rahvas on asunud selle linnadesse?
\par 2 Seepärast, vaata, päevad tulevad, ütleb Issand, mil ma toon kuuldavale sõjahüüu ammonlaste Rabba vastu: see muutub hüljatud rusuhunnikuks ja selle tütarlinnad põlevad tules. Siis pärib Iisrael oma pärijad, ütleb Issand.
\par 3 Ulu, Hesbon, sest Ai hävitatakse! Kisendage, Rabba tütred, rõivastage end kotiriidesse! Tehke leinakaebust ja uidake müüridel, sest Milkom läheb vangi koos oma preestrite ja vürstidega!
\par 4 Miks sa kiitled orgudest, et su org voolab üle, sa taganenud tütar, lootes oma varanduste peale: „Kes tuleb mulle kallale?”
\par 5 Vaata, ma toon sulle hirmu kõigilt su naabreilt, ütleb Issand, vägede Issand, ja teid pillutatakse laiali, igaüks omale poole, ja keegi ei kogu põgenikke.
\par 6 Aga pärast seda pööran ma ammonlaste vangipõlve, ütleb Issand.”
\par 7 Edomi kohta: Nõnda ütleb vägede Issand: „Kas ei ole enam tarkust Teemanis? Kas on mõistlikel kadunud nõu? Kas nende tarkus on saanud tühjaks?
\par 8 Põgenege, pöörduge, pugege sügavale peitu, Dedani elanikud! Sest ma toon Eesavile õnnetuse, tema karistusaja.
\par 9 Kui sulle tulevad viinamarjakorjajad, ei jäta nad järelnoppimiseks midagi järele. Kui öösel tulevad vargad, varastavad nad just nõndapalju, kui nad tahavad.
\par 10 Jah, mina ise paljastan Eesavi, avastan tema peidupaigad ja ta ei saa ennast peita. Hävitatakse tema sugu ja ta vennad ja naabrid - ja teda ei ole enam.
\par 11 Jäta oma vaeslapsed minu elatada ja su lesknaised lootku minu peale!
\par 12 Sest nõnda ütleb Issand: Vaata, ka need, kelle kohus ei olekski juua seda karikat, peavad jooma, ja sina peaksid jääma karistuseta? Sina ei jää karistuseta, vaid sa pead tõesti jooma!
\par 13 Sest ma olen vandunud iseenesele, ütleb Issand, et Bosra saab hirmutuseks, teotuseks, kõleduseks ja needuseks, ja kõik ta tütarlinnad muutuvad igavesti varemeiks.
\par 14 Ma olen Issandalt kuulnud sõnumit, et käskjalg on läkitatud rahvaste juurde: „Kogunege ja tulge temale kallale ning hakake sõdima!”
\par 15 Sest vaata, ma teen su pisikeseks rahvaste keskel, põlatuks inimeste seas.
\par 16 Sind on petnud su kardetavus, su südame ülbus, kes elad kaljulõhedes, valitsed kõrgeid künkaid. Kuigi sa pesitseksid kõrgel nagu kotkas, tooksin ma su sealt alla, ütleb Issand.
\par 17 Ja Edom muutub jubeduseks: igaüks, kes läheb sealt mööda, kohkub ja mõnitab teda kõigi tema nuhtluste pärast.
\par 18 Otsekui Soodoma ja Gomorra ja nende naabrite segipaiskamise järel, ütleb Issand, nõnda ei ela seal keegi ega asugi sinna inimlaps.
\par 19 Vaata, otsekui lõvi kargab Jordani padrikust lokkavale karjamaale, nõnda ajan ma tema sealt äkitselt ära, ja selle, kes on valitud, panen ma sinna valitsejaks. Sest kes on minu sarnane? Ja kes võib mind kutsuda kohtusse? Ja kes on karjane, kes võiks seista minu vastu?
\par 20 Seepärast kuulge Issanda nõu, mida ta on pidanud Edomi vastu, ja tema mõtteid, mida ta on mõlgutanud Teemani elanike vastu: tõesti, kõige pisemgi lammas veetakse ära; tõesti, nende karjamaa jääb tühjaks nende eneste pärast.
\par 21 Nende langemise mürinast väriseb maa, hädakisa kaja kostab Kõrkjamereni.
\par 22 Vaata, ta tõuseb üles ja lendab nagu kotkas ja laotab oma tiibu Bosra kohal; ja sel päeval on Edomi kangelaste süda otsekui lapsevaevas oleval naisel.”
\par 23 Damaskuse kohta: „Häbisse jäävad Hamat ja Arpad, sest nad kuulevad halba sõnumit; nad voogavad nagu rahutu meri, mis ei jää vaikseks.
\par 24 Damaskus muutub araks, pöördub põgenema, teda haarab hirm; teda valdab ahastus ja valu nagu sünnitajat.
\par 25 Kuidas küll on maha jäetud see ülistatud linn, mu rõõmulinn!
\par 26 Jah, tema turgudel langevad ta noored mehed ja sel päeval hukkuvad kõik sõjamehed, ütleb vägede Issand.
\par 27 Ja ma süütan Damaskuse müüridesse tule, et see põletaks Ben-Hadadi paleed.”
\par 28 Keedari ja Haasori kuningriikide kohta, mis Paabeli kuningas Nebukadnetsar vallutab: Nõnda ütleb Issand: „Tõuske, minge kallale Keedarile ja hävitage idamaalased!
\par 29 Nende telgid ja karjad võetakse ära, nende telgivaibad ja kõik nende riistad; neilt viiakse nende kaamelid ja neile hüütakse: „Hirm on igal pool!”
\par 30 Põgenege, rännake kaugele, pugege sügavale, Haasori elanikud! ütleb Issand. Sest Paabeli kuningas Nebukadnetsar on pidanud nõu teie vastu ja on mõlgutanud mõtteid.
\par 31 Tõuske, minge kallale muretule rahvale, kes elab julgesti, ütleb Issand; tal ei ole uksi ega riive, ta elab üksinda.
\par 32 Nende kaamelid jäävad riisutavaiks ja nende veiste hulk võitjate saagiks; ma pillutan need pöetudoimulised kõigi tuulte poole ja toon neile igast küljest õnnetuse, ütleb Issand.
\par 33 Ja Haasor jääb ðaakalite pesapaigaks, igavesti laastatuks. Keegi ei ela seal ega asugi sinna inimlaps.”
\par 34 Issanda sõna, mis tuli prohvet Jeremijale Eelami kohta Juuda kuninga Sidkija valitsemise alguses; ta ütles:
\par 35 „Nõnda ütleb vägede Issand: Vaata, ma murran Eelami ammu, nende peamise jõu.
\par 36 Ma toon Eelami vastu neli tuult neljast taevakaarest; ja ma hajutan nad kõigi nende tuulte kätte ega ole rahvast, kelle juurde ei tule Eelami hajutatud.
\par 37 Ma hirmutan eelamlasi nende vaenlaste ees, nende hinge püüdjate ees ja toon neile õnnetuse - oma tulise viha, ütleb Issand; ja ma läkitan mõõga neile järele, kuni olen teinud neile lõpu.
\par 38 Ja ma asetan oma aujärje Eelamisse ning hävitan sealt kuninga ja vürstid, ütleb Issand.
\par 39 Aga viimseil päevil ma pööran Eelami vangipõlve, ütleb Issand.”

\chapter{50}

\par 1 Sõna, mis Issand kõneles Paabeli kohta, kaldealaste maa kohta prohvet Jeremija läbi:
\par 2 „Teatage rahvaste seas ja kuulutage, tõstke lipp üles; kuulutage, ärge salake, öelge: Paabel vallutatakse, Beel jääb häbisse, Merodak lüüakse puruks! Tema pühakujud jäävad häbisse, tema ebajumalad purustatakse.
\par 3 Sest tema kallale tuleb rahvas põhja poolt: see teeb tema maa jubedaks ja seal ei ela ükski - niihästi inimesed kui loomad põgenevad ja kaovad.
\par 4 Neil päevil ja sel ajal, ütleb Issand, tulevad Iisraeli lapsed, nemad ja Juuda lapsed üheskoos, ja käivad üha nuttes ning otsivad Issandat, oma Jumalat.
\par 5 Nad küsivad teed Siionisse, nende silmnäod on siiapoole: „Tulgem ja hoidkem Issanda poole igavese lepinguga, mida ei unustata!”
\par 6 Mu rahvas oli nagu kadunud lambad, nende karjased eksitasid neid, juhtisid kõrvale mägedes; nad rändasid mäelt mäele, nad unustasid oma tarad.
\par 7 Kõik, kes neid leidsid, neelasid neid, ja nende vaenlased ütlesid: „Me ei jää süüdlasteks! Sest nad on pattu teinud Issanda, õiguse eluaseme, ja Issanda, oma vanemate lootuse vastu.”
\par 8 Põgenege Paabelist, minge välja kaldealaste maalt, olge nagu sikud karja ees!
\par 9 Sest vaata, ma äratan ja toon Paabeli kallale hulganisti suuri rahvaid põhjamaalt, kes asuvad rünnakule tema vastu: sealtpoolt ta vallutatakse. Nende nooled on kui vilunud võitlejal - need ei tule tühjalt tagasi.
\par 10 Ja Kaldea rüüstatakse, kõik selle rüüstajad saavad küllaldaselt, ütleb Issand.
\par 11 Ilutsege aga, rõõmutsege aga, mu pärisosa rüüstajad. Jah, kepsutage nagu vasikad aasal ja hirnuge nagu täkud:
\par 12 teie ema jääb suurde häbisse; kes teid on sünnitanud, peab häbenema! Vaata, see on paganate tulevik: kõrb, kuivanud maa ja lagendik.
\par 13 Issanda viha tõttu ei elata seal ja see jääb hoopis tühjaks. Igaüks, kes Paabelist mööda läheb, kohkub ja vilistab kõigi ta nuhtluste pärast.
\par 14 Rivistuge ümber Paabeli, kõik, kes vinnastate ambu! Ambuge teda, ärge säästke nooli, sest ta on pattu teinud Issanda vastu!
\par 15 Tõstke tema ümber võidukisa: „Ta tõstab käe, ta toed langevad, ta müürid kistakse maha.” Sest see on Issanda kättemaks. Makske temale kätte, talitage temaga, nõnda kuidas tema talitas!
\par 16 Kaotage Paabelist külvaja ja lõikusajal sirbihaaraja! Hävitava mõõga eest pöördub igaüks oma rahva juurde ja igaüks põgeneb oma maale.
\par 17 Iisrael oli eksinud lammas, lõvidest aetud. Esiti sõi teda Assuri kuningas, ja nüüd viimaks näris tema luid Paabeli kuningas Nebukadnetsar.
\par 18 Seepärast ütleb vägede Issand, Iisraeli Jumal, nõnda: Vaata, mina karistan Paabeli kuningat ja tema maad, nõnda nagu ma karistan Assuri kuningat.
\par 19 Ja ma toon Iisraeli tagasi tema enese karjamaale ning ta hakkab hoidma karja Karmelil ja Baasanis, ja ta hing küllastub Efraimi mäestikus ja Gileadis.
\par 20 Neil päevil ja sel ajal, ütleb Issand, otsitakse Iisraeli süüd, aga seda ei ole, ja Juuda patte, aga neid ei leita, sest ma annan andeks neile, keda ma alles jätan.
\par 21 Mine „Topelttõrksa„ maa vastu ja ”Karistuse” elanike kallale! Löö maha ja hävita sootuks nende järelt, ütleb Issand, ja tee kõik, nagu ma sind olen käskinud!
\par 22 Maal on sõjahüüd ja suur häving.
\par 23 Kuidas küll on purustatud ja katki murtud kogu maailma vasar! Kuidas küll Paabel on saanud jubeduseks rahvaste seas!
\par 24 Mina panin sulle püüdepaela ja sind, Paabel, saadi kätte, nõnda et sa ise ei teadnudki. Sind tabati ja võeti kinni, sest sa võitlesid Issanda vastu.
\par 25 Issand avas oma relvakambri ja tõi välja oma sajatuse relvad. Sest Issandal, vägede Issandal, on tööd kaldealaste maal.
\par 26 Tulge kõikjalt temale kallale, avage ta aidad, kuhjake teda viljana hunnikusse ja hävitage ta sootuks, ärgu jäägu talle midagi järele.
\par 27 Lööge maha kõik ta härjavärsid, mingu nad tapale! Häda neile! Sest on tulnud nende päev, nende karistusaeg.
\par 28 Kuulge! Põgenikud ja Paabelimaalt pääsenud jutustavad Siionis Issanda, meie Jumala kättemaksust, kättemaksust tema templi eest.
\par 29 Kutsuge Paabeli kallale kütid, kõik ammuvinnastajad! Lööge leer üles tema ümber, ärgu pääsegu ükski! Tasuge temale ta tegude järgi, talitage temaga, nagu tema on talitanud teiega! Sest ta on olnud ülbe Issanda vastu, Iisraeli Püha vastu.
\par 30 Seepärast langevad ta turgudel tema noored mehed ja sel päeval hukkuvad kõik ta sõjamehed, ütleb Issand.
\par 31 Vaata, ma olen su vastu, sina ülbe, ütleb Issand, vägede Issand, sest on tulnud su päev, aeg, mil ma sind karistan.
\par 32 Siis komistab ülbe ja langeb ega ole, kes ta üles tõstaks; ja ma süütan tema linnades tule ning see põletab kõik ümberringi.
\par 33 Nõnda ütleb vägede Issand: Iisraeli lastele ja Juuda lastele on ühtviisi liiga tehtud; kõik nende vangistajad peavad neid kinni, nad ei taha neid vabastada.
\par 34 Aga nende lunastaja on vägev, vägede Issand on tema nimi. Tema seletab hästi nende riiuasja, et tuua maale rahu, aga Paabeli elanikele rahutust.
\par 35 Mõõk kaldealaste kallale, ütleb Issand, ja Paabeli elanike kallale, tema vürstide ja tarkade kallale.
\par 36 Mõõk ennustajate kallale, et need jääksid narrideks. Mõõk tema kangelaste kallale, et need kohkuksid.
\par 37 Mõõk tema hobuste ja sõjavankrite kallale ja kogu segarahva kallale, kes on tema keskel, et see muutuks naisteks. Mõõk tema varanduste kallale, et need riisutaks.
\par 38 Põud tema vete kallale, et need kuivaksid. Sest see on nikerdatud kujude maa, nad on hullud nende peletiste järele.
\par 39 Seepärast hakkavad seal elama kurjad vaimud ja paharetid ja seal elutsevad jaanalinnud. Seda ei asustata enam iialgi, seal ei elata põlvede jooksul.
\par 40 Otsekui Jumal paiskas segi Soodoma ja Gomorra ja nende naabrid, ütleb Issand, nõnda ei ela seal keegi ega asugi sinna inimlaps.
\par 41 Vaata, rahvas tuleb põhja poolt: suur rahvas ja palju kuningaid hakkab liikuma maa viimastest äärtest.
\par 42 Nad hoiavad käes ambu ja oda, nad on julmad ega tunne halastust; nende kisa on nagu mere kohin ja nad ratsutavad hobuste seljas, nad on varustatud nagu mehed tapluseks sinu vastu, Paabeli tütar.
\par 43 Kui Paabeli kuningas kuuleb neist sõnumit, siis lõtvuvad ta käed; teda valdab ahastus nagu sünnitajat valu.
\par 44 Vaata, otsekui lõvi kargab Jordani padrikust lokkavale karjamaale, nõnda äkitselt ajan ma nad sealt ära, ja selle, kes on valitud, panen ma sinna valitsejaks. Sest kes on minu sarnane? Kes võib mind kutsuda kohtusse? Ja kes on karjane, kes võiks seista minu vastu?
\par 45 Seepärast kuulge Issanda nõu, mida ta on pidanud Paabeli vastu, ja tema mõtteid, mida ta on mõlgutanud kaldealaste maa vastu: tõesti, kõige pisemgi lammas veetakse ära; tõesti, nende karjamaa jääb tühjaks nende eneste pärast.
\par 46 Hüüdest: „Paabel on vallutatud!” väriseb maa ja rahvaste seas kuuldakse hädakisa.”

\chapter{51}

\par 1 Nõnda ütleb Issand: „Vaata, ma äratan Paabeli kallale, „mu vastase südame” elanike kallale hävitava tuule.
\par 2 Ja ma läkitan Paabeli kallale võõraid, kes tuulavad teda ja laastavad tema maad, kui nad õnnetusepäeval on igalt poolt tema kallal.
\par 3 Ammukütt vinnastagu oma amb ammuküti vastu ja selle vastu, kes suurustab oma soomusrüüs. Ärge säästke tema noori mehi, hävitage sootuks kogu ta vägi,
\par 4 et mahalöödud langeksid kaldealaste maal ja läbipistetud tema tänavail.
\par 5 Sest Iisrael ja Juuda ei jää oma Jumala, vägede Issanda leskedeks, vaid kaldealaste maa on täis süüd Iisraeli Püha vastu.
\par 6 Põgenege Paabelist ja igaüks päästku oma hing, et teie ei hukkuks tema süü pärast! Sest see on Issanda kättemaksuaeg, kes tasub temale, mis ta on teinud.
\par 7 Paabel oli kuldkarikas Issanda käes, mis tegi joobnuks kogu maailma; rahvad jõid tema veini, seetõttu rahvad hullusid.
\par 8 Äkitselt langeb Paabel ja purustatakse. Ulguge tema pärast! Võtke palsamit tema haavade jaoks, vahest ta saab terveks!
\par 9 Me tahtsime Paabelit ravida, aga ta ei paranenud. Jätkem ta maha ja mingem igaüks oma maale, sest kohus tema üle ulatub taevani ja tõuseb pilvedeni.
\par 10 Issand on toonud esile meie õigused; tulgem ja jutustagem Siionis Issanda, oma Jumala teost.
\par 11 Teritage nooli, haarake kilbid! Issand on äratanud Meedia kuningate vaimu, sest tema kavatsuseks Paabeli kohta on selle hävitamine. Sest see on Issanda kättemaks, kättemaks tema templi eest.
\par 12 Tõstke lipp Paabeli müüride vastu, tugevdage valvet, seadke vahimehi, paigutage varitsejaid! Sest nagu Issand on kavatsenud, nõnda tema teebki teoks, mis ta Paabeli elanike kohta on ütelnud.
\par 13 Sina, kes sa elad suurte vete ääres, varandustest rikas, on tulnud sinu lõpp, küünar, kust sind katki lõigatakse.
\par 14 Vägede Issand on vandunud iseenesele: tõesti, ma täidan sind inimestega otsekui rohutirtsudega, ja nad tõstavad sinu vastu sõjakisa.
\par 15 Tema on oma rammuga teinud maa, tarkusega rajanud maailma ja mõistusega laotanud taeva.
\par 16 Kui tema teeb häält, siis on taevas vee kohin ja ta tõstab pilved maa äärest; tema teeb vihmale välgud ja toob tuule välja selle aitadest.
\par 17 Inimesed on kõik rumalad, mõistusest ilma. Iga kullassepp jääb häbisse jumalakuju pärast, sest tema valatud kujud on pettus ja neis pole vaimu.
\par 18 Need on tühised, naeruväärt töö: oma katsumisajal nad hukkuvad.
\par 19 Nende sarnane ei ole see, kes on Jaakobi osa, sest tema on kõige looja ja ta pärisosa kepp. Vägede Issand on tema nimi.
\par 20 Sina, Paabel, olid mulle vasaraks, relvaks, sinuga purustasin ma rahvaid ja sinuga hävitasin ma kuningriike.
\par 21 Sinuga purustasin ma hobuse ja ratsaniku, sinuga purustasin ma vankri ja sõitja.
\par 22 Sinuga purustasin ma mehe ja naise, sinuga purustasin ma vana ja noore, sinuga purustasin ma noormehe ja neitsi.
\par 23 Sinuga purustasin ma karjase ja karja, sinuga purustasin ma põllumehe ja härjapaari, sinuga purustasin ma maavanemad ja asevalitsejad.
\par 24 Aga nüüd tasun ma Paabelile ja kõigile Kaldea elanikele teie nähes kõik nende kurja, mis nad tegid Siionile, ütleb Issand.
\par 25 Vaata, ma olen su vastu, sa hävitusemägi, ütleb Issand, kes hävitasid kogu maa. Ma sirutan oma käe su vastu ja veeretan sind kaljudelt alla ning teen sind põlenud mäeks.
\par 26 Ja sinust ei võeta nurgakivi ega kivi alusmüüri jaoks, vaid sa jääd igavesti hävitatuks, ütleb Issand.
\par 27 Tõstke maal lipp, puhuge sarve rahvaste seas, pühitsege rahvad tema vastu, kutsuge tema vastu Ararati, Minni ja Askenase kuningriigid, pange tema vastu värbamispealik, tooge üles hobuseid otsekui karuseid rohutirtsuvastseid!
\par 28 Pühitsege ta vastu rahvad, Meedia kuningad, nende maavanemad ja kõik asevalitsejad ning kogu nende valitsusealune maa!
\par 29 Siis väriseb maa ja viskleb valudes, sest täituvad kavatsused, mis Issandal on Paabeli vastu, et teha Paabeli maa lagedaks, kus ükski ei ela.
\par 30 Paabeli kangelased lakkavad sõdimast, nad istuvad kindlustes, nende jõud hääbub, nad muutuvad naisteks; tema hooned põletatakse, tema riivid murtakse.
\par 31 Jooksja jookseb jooksjale ja sõnumiviija sõnumiviijale vastu, kuulutama Paabeli kuningale, et tema linn on igast küljest vallutatud,
\par 32 et koolmed on võetud, pilliroopaadid tules põletatud ja sõjamehed kabuhirmus.
\par 33 Sest nõnda ütleb vägede Issand, Iisraeli Jumal: Paabeli tütar on nagu rehepaik selle kõvakstampimise ajal: veel pisut, siis tuleb temale lõikusaeg.
\par 34 „Nebukadnetsar, Paabeli kuningas, on mind neelanud, kimbutanud, kõrvale pannud nagu tühja astja; ta on mind neelanud nagu lohe, on täitnud oma kõhu minu maiuspaladega, ta on mind ära ajanud.
\par 35 Vägivald, mida olen kannatanud, ja mu kokkuvarisemine tulgu Paabeli peale!„ ütleb Siioni elanik. „Ja mu veri Kaldea elanike peale!” ütleb Jeruusalemm.
\par 36 Seepärast ütleb Issand nõnda: Vaata, mina ajan sinu riiuasja ja tasun, mis sul on tasuda; mina lasen taheneda tema mere ja kuivatan tema allika.
\par 37 Paabel muutub kivihunnikuks, ðaakalite asupaigaks, jubeduseks ja parastamise põhjuseks, kus ükski ei ela.
\par 38 Nad möirgavad küll üheskoos nagu lõvid ja urisevad nagu lõvikutsikad.
\par 39 Kui nad on elevil, siis ma teen neile joomingu ja lasen nad purju jääda, et nad hõiskaksid ja uinuksid igavesse unne ega ärkaks, ütleb Issand.
\par 40 Mina viin nad tappa nagu talled, nagu jäärad koos sikkudega.
\par 41 Kuidas küll vallutatakse Seesak, ja võetakse see, kes on kuulus kogu maal. Kuidas küll Paabel saab jubeduseks rahvaste seas!
\par 42 Meri tõuseb üle Paabeli, ta kaetakse selle kohisevate lainetega.
\par 43 Tema linnad muutuvad õudseks, maa põuaseks ja lagedaks, maaks, kus ükski ei ela ja millest inimlaps läbi ei lähe.
\par 44 Ma karistan Beeli Paabelis ja kisun ta suust, mis ta on neelanud; rahvad ei voola enam tema juurde, Paabeli müürgi variseb maha.
\par 45 Minge ära selle keskelt, minu rahvas, ja päästke igaüks oma elu Issanda tulise viha eest!
\par 46 Ärgu ainult teie süda mingu araks ja ärge kartke kuulujutte, mis maal kostavad, kui ühel aastal tuleb üks kuulujutt ja järgmisel aastal teine, kui maal on vägivald ja valitseja on valitseja vastu!
\par 47 Sest vaata, päevad tulevad, mil ma karistan Paabeli nikerdatud kujusid; ja kogu tema maa jääb häbisse ning kõik langevad surnult maha sel maal.
\par 48 Siis rõõmustavad Paabeli pärast taevas ja maa ja kõik, mis neis on, kui põhja poolt tulevad tema kallale hävitajad, ütleb Issand.
\par 49 Paabelgi langeb Iisraeli mahalöödute pärast, nagu Paabeli pärast langesid mahalöödud kogu maal.
\par 50 Teie, mõõga eest pääsenud, minge, ärge jääge seisma! Mõelge kaugel Issandale ja teile meenugu Jeruusalemm!
\par 51 „Me häbeneme, kui kuuleme laimu; häbi katab meie palet, kui muulased tulevad Issanda koja pühaduste kallale.”
\par 52 Seepärast, vaata, päevad tulevad, ütleb Issand, mil ma karistan tema nikerdatud kujusid ja kogu ta maal oigavad haavatud.
\par 53 Kuigi Paabel peaks tõusma taevani ja kuigi ta kindlustaks oma võimsa kõrgendiku, kummatigi tulevad minult tema kallale hävitajad, ütleb Issand.
\par 54 Kuula! Hädakisa Paabelist ja suurest hävingust kaldealaste maal!
\par 55 Sest Issand hävitab Paabeli ja lõpetab sealt suure lärmi, kuigi ta lained kohisevad nagu suured veed ja kostab ta hüüete kära.
\par 56 Sest tema kallale, Paabeli kallale, tuleb hävitaja ja tema kangelased võetakse vangi, nende ammud murtakse katki; sest kättemaksu Jumal, Issand, maksab kindlasti kätte.
\par 57 Ja ma teen joobnuks tema vürstid ja targad, maavanemad, asevalitsejad ja kangelased, et nad uinuksid igavesse unne ega ärkaks enam, ütleb kuningas, kelle nimi on vägede Issand.
\par 58 Nõnda ütleb vägede Issand: Laiad Paabeli müürid kistakse maani maha, ja tema kõrged väravad põletatakse tules. Rahvad vaevavad endid ilmaaegu ja rahvahõimud väsitavad endid tule tarvis.”
\par 59 Ülesanne, mille prohvet Jeremija andis Serajale, Mahseja poja Neerija pojale, kui see läks koos Juuda kuninga Sidkijaga Paabelisse selle valitsemise neljandal aastal; Seraja oli majutuspealik.
\par 60 Jeremija oli kirjutanud ühte raamatusse kogu selle õnnetuse, mis Paabelile oli tulemas, kõik need sõnad, mis Paabeli kohta on kirjutatud.
\par 61 Ja Jeremija ütles Serajale: „Kui sa jõuad Paabelisse, siis vaata, et sa loed kõik need sõnad
\par 62 ja ütled: Issand, sina oled rääkinud selle paiga kohta, et sa hävitad selle, nõnda et siin ei ela ükski, ei inimene ega loom, vaid et see jääb igavesti laastatuks!
\par 63 Ja kui sa selle raamatu oled lõpuni lugenud, siis seo selle külge kivi ja viska see keset Frati jõge
\par 64 ning ütle: Nõnda vajub Paabel ega tõuse enam õnnetuse pärast, mille ma temale toon: nad peavad väsima!” Siiani on Jeremija sõnad.

\chapter{52}

\par 1 Sidkija oli kuningaks saades kakskümmend üks aastat vana ja ta valitses Jeruusalemmas üksteist aastat; ta ema nimi oli Hamutal, Jeremija tütar Libnast.
\par 2 Ta tegi kurja Issanda silmis, just nagu Joojakim oli teinud.
\par 3 Jah, mis sündis Jeruusalemmas ja Juudas, põhjustas Issanda viha, kuni ta heitis need ära oma palge eest.
\par 4 Sidkija hakkas mässama Paabeli kuninga vastu. Ja tema valitsemise üheksandal aastal, kümnenda kuu kümnendal päeval, tuli Paabeli kuningas Nebukadnetsar, tema ja kogu ta sõjavägi Jeruusalemma vastu ja lõi leeri üles selle alla; ja nad ehitasid selle ümber piiramisseadmed.
\par 5 Ja linna piirati kuni kuningas Sidkija üheteistkümnenda aastani.
\par 6 Neljanda kuu üheksandal päeval võttis nälg linnas võimust ja maa rahval ei olnud leiba.
\par 7 Siis murti linna sisse; aga kõik sõjamehed põgenesid ja läksid öösel linnast välja läbi kuninga rohuaia juures oleva müüridevahelise värava, kuigi kaldealased olid ümber linna; ja nad läksid lagendiku poole.
\par 8 Aga kaldealaste sõjavägi ajas kuningat taga ja nad said Sidkija kätte Jeeriko lagendikel, kui kõik ta sõjavägi tema juurest oli laiali läinud.
\par 9 Ja nad võtsid kuninga kinni ning viisid ta Paabeli kuninga juurde Riblasse Hamatimaale, ja too mõistis kohut tema üle.
\par 10 Ja Paabeli kuningas tappis Sidkija pojad tema silme ees, samuti tappis ta Riblas ka kõik Juuda vürstid.
\par 11 Aga Sidkija silmad tehti pimedaks ja ta aheldati vaskahelaisse; siis Paabeli kuningas viis ta Paabelisse ja pani vangikotta kuni ta surmapäevani.
\par 12 Ja viienda kuu kümnendal päeval, see on kuningas Nebukadnetsari, Paabeli kuninga üheksateistkümnendal aastal, tuli Nebusaradan, ihukaitsepealik, kes oli Paabeli kuninga teenistuses, Jeruusalemma.
\par 13 Ja tema põletas ära Issanda koja ja kuningakoja ning kõik Jeruusalemma kojad; nimelt kõik suurnike kojad põletas ta tulega.
\par 14 Ja kogu kaldealaste sõjavägi, kes oli koos ihukaitsepealikuga, kiskus maha kõik Jeruusalemma müürid ümberringi.
\par 15 Ja osa vaesemast rahvast ning rahva jäägi, kes olid jäänud linna, ja ülejooksikud, kes Paabeli kuninga poole olid üle jooksnud, ja käsitööliste jäägi viis ihukaitsepealik Nebusaradan vangi.
\par 16 Aga Nebusaradan, ihukaitsepealik, jättis maa vaesemast rahvast alles viinamägede harijaid ja teopäevade tegijaid.
\par 17 Vasksambad, mis kuulusid Issanda koja juurde, ja alused ning vaskmere, mis olid Issanda kojas, kaldealased purustasid ja viisid kogu nende vase Paabelisse.
\par 18 Potid, labidad, tahikäärid, piserdusnõud, kausid ja kõik vaskriistad, millega peeti teenistust, võtsid nad ära.
\par 19 Samuti võttis ihukaitsepealik ära liuad, sütepannid, piserdusnõud, potid, lambijalad, kausid ja peekrid, mis olid puhtast kullast ja puhtast hõbedast.
\par 20 Mõlema samba, vaskmere ja kaheteistkümne vaskhärja, mis olid aluste all, mis kuningas Saalomon oli teinud Issanda koja jaoks - kõigi nende asjade vask oli vaagimatu.
\par 21 Mis puutub sammastesse, siis oli ühe samba kõrgus kaheksateist küünart, ja selle ümber ulatas kaheteistküünrane mõõdunöör; selle paksus oli neli sõrme, see oli seest õõnes.
\par 22 Nupp selle peal oli vasest ja nupu kõrgus oli viis küünart; võrestik ja granaatõunad ümber nupu olid kõik vasest; ja samasugune oli teine sammas granaatõuntega.
\par 23 Granaatõunu oli väljaspool üheksakümmend kuus; kõiki granaatõunu võrestikul ümberringi oli sada.
\par 24 Ja ihukaitsepealik võttis Seraja, ülempreestri, ja Sefanja, temast järgmise preestri, ja kolm lävehoidjat,
\par 25 ja võttis linnast ühe hoovkondlase, kes oli olnud sõjameeste käsutaja, ja seitse meest kuninga lähikonnast, kes leiti linnast, ja väepealiku kirjutaja, kes värbas maa rahvast sõjaväkke, ja kuuskümmend meest maa rahva hulgast, kes leiti linnast,
\par 26 - need võttis Nebusaradan, ihukaitsepealik, ja viis Paabeli kuninga juurde Riblasse.
\par 27 Ja Paabeli kuningas lõi need maha ning surmas need Riblas Hamatimaal. Nõnda viidi Juuda vangi omaenese maalt.
\par 28 See on rahva arv, kelle Nebukadnetsar vangi viis: seitsmendal aastal kolm tuhat kakskümmend kolm juuti;
\par 29 kaheksateistkümnendal Nebukadnetsari aastal kaheksasada kolmkümmend kaks hinge Jeruusalemmast;
\par 30 kahekümne kolmandal Nebukadnetsari aastal viis Nebusaradan, ihukaitsepealik, juute vangi seitsesada nelikümmend viis hinge; kõiki hingi oli neli tuhat kuussada.
\par 31 Aga kolmekümne seitsmendal aastal pärast Juuda kuninga Joojakini vangiviimist, kaheteistkümnenda kuu kahekümne viiendal päeval, tõstis Paabeli kuningas Evil-Merodak, sel aastal kui ta sai kuningaks, Juuda kuninga Joojakini pea üles ja tõi tema vangikojast välja.
\par 32 Ja ta kõneles temaga lahkesti ning andis temale istme ülemale nende kuningate istmeist, kes olid tema juures Paabelis.
\par 33 Ja Joojakin vahetas oma vangiriided ning sõi alaliselt tema juures leiba kogu oma eluaja.
\par 34 Ja ta sai Paabeli kuningalt ülalpidamise, kindla ülalpidamise, mida ta vajas iga päev, niikaua kui ta elas.



\end{document}