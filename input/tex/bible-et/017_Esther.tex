\begin{document}

\title{Estri raamat}

\chapter{1}

\par 1 Ja see sündis Ahasverose ajal, selle Ahasverose, kes valitses Indiast Etioopiani - sada kahtkümmend seitset maad -,
\par 2 neil päevil, kui kuningas Ahasveros istus oma kuninglikul aujärjel Suusani palees.
\par 3 Oma valitsemise kolmandal aastal tegi ta võõruspeo kõigile oma vürstidele ja sulastele; Pärsia ja Meedia võimukandjad, ülikud ja maavalitsejad olid tema juures.
\par 4 Ta näitas oma kuningriigi määratut rikkust ja oma võimu toredat hiilgust palju päevi - sada kaheksakümmend päeva.
\par 5 Ja kui need päevad olid möödunud, tegi kuningas seitsmepäevase peo kogu Suusani palees olevale rahvale, niihästi suurtele kui väikestele, kuningliku palee rohuaia väljakul.
\par 6 Peened valged ja sinised kangad olid kinnitatud valgete ja purpurpunaste nööridega hõberõngaste ja marmorsammaste külge; kullast ja hõbedast lamamisasemed olid alabastrist ja marmorist, klaasjaist ja kirjudest kividest ruudulisel põrandal.
\par 7 Juua anti kuldkarikaist, karikaid oli mitmesuguseid, ja kuninglikku veini oli palju, kuninga kohaselt.
\par 8 Aga joomisel kehtis korraldus, et sundust ei ole, sest kuningas oli seda kinnitanud kõigile oma kojaülemaile, et igaüks võib teha, nagu temale meeldib.
\par 9 Kuninganna Vasti tegi peo ka naistele kuningas Ahasverose kuninglikus kojas.
\par 10 Seitsmendal päeval, kui kuninga süda oli veinirõõmus, käskis ta Mehumani, Bistat, Harbonat, Bigtat, Abagtat, Seetarit ja Karkast, seitset eunuhhi, kes kuningas Ahasverost teenisid,
\par 11 tuua kuninganna Vasti kuninga ette koos kuningliku krooniga, et näidata rahvaile ja vürstidele tema ilu, sest ta oli ilusa välimusega.
\par 12 Aga kuninganna Vasti keeldus tulemast kuninga käsu peale, mis eunuhhide kaudu oli antud; siis raevutses kuningas väga ja viha süttis temas põlema.
\par 13 Ja kuningas küsis tarkadelt, kes aegu mõistsid, sest oli viisiks panna kuninga käsk kõigi seaduse- ja õigusetundjate ette,
\par 14 ja temale lähemad olid Karsena, Seetar, Admata, Tarsis, Meres, Marsena ja Memukan, seitse Pärsia ja Meedia vürsti, kes nägid kuninga palet ja istusid kuningriigis esimesel kohal:
\par 15 „Mida tuleks seaduse järgi teha kuninganna Vastiga, sellepärast et ta ei ole täitnud kuninga käsku, mis eunuhhide kaudu oli antud?”
\par 16 Siis vastas Memukan kuninga ja vürstide ees: „Mitte ainult kuninga vastu ei ole kuninganna Vasti eksinud, vaid ka kõigi vürstide ja kõigi rahvaste vastu kuningas Ahasverose kõigis maades.
\par 17 Kui kuninganna tegu saab teatavaks kõigile naistele, siis nad hakkavad põlgusega vaatama oma meestele, öeldes: Kuningas Ahasveros käskis tuua kuninganna Vasti enese ette, aga tema ei tulnud.
\par 18 Juba täna võivad Pärsia ja Meedia vürstinnad, kes sellest kuninganna teost kuulevad, kõigile kuninga vürstidele nõnda vastata, ja sellest tuleb palju põlgust ning viha.
\par 19 Kui kuningas heaks arvab, siis tulgu temalt kuninglik käsk ja see kirjutatagu muutmatuna Pärsia ja Meedia seaduste hulka, et Vasti ei tohi tulla kuningas Ahasverose ette ja et kuningas annab tema kuningliku au teisele, kes on temast parem!
\par 20 Ja kui kuninga otsus, mille ta teeb, saab teatavaks kogu tema kuningriigis, mis on ju suur, siis austavad kõik naised oma mehi, niihästi suured kui väikesed.”
\par 21 See kõne oli kuninga ja vürstide silmis hea, ja kuningas tegi Memukani sõna järgi.
\par 22 Ja ta läkitas kirjad kõigisse kuninga maadesse, igale maale tema oma kirjaviisis ja igale rahvale tema oma keeles, et iga mees olgu valitseja oma kojas ja rääkigu oma rahva keelt!

\chapter{2}

\par 1 Pärast neid sündmusi, kui kuningas Ahasverose viha oli jahtunud, mõtles ta Vastile ja sellele, mis ta oli teinud ning mis tema kohta oli otsustatud.
\par 2 Siis ütlesid kuninga sulased, kes teda teenisid: „Otsitagu kuningale noori neitseid, kes on ilusa välimusega.
\par 3 Kuningas pangu ametnikke kõigisse oma kuningriigi maadesse, et need koguksid kõik ilusa välimusega noored neitsid Suusani palee naistemajasse eunuhhi Heegai, kuninga naistevalvaja käe alla, ja et neile antaks vajalikku iluravi.
\par 4 Ja tütarlaps, kes kuningale meeldib, saagu Vasti asemel kuningannaks.” See kõne oli kuninga meelest hea ja ta tegi nõnda.
\par 5 Suusani palees oli Juuda mees, Mordokai nimi, Jairi poeg, kes oli Simei poeg, kes oli benjaminlase Kiisi poeg.
\par 6 Tema oli Jeruusalemmast vangi viidud koos nende vangidega, kes vangistati koos Juuda kuninga Jekonjaga, kelle Paabeli kuningas Nebukadnetsar oli vangi viinud.
\par 7 Tema oli oma sugulase tütre Hadassa, see on Estri kasvataja, sest temal ei olnud ei isa ega ema; see tütarlaps oli ilusa jume ja kauni välimusega; ja et ta isa ja ema olid surnud, siis oli Mordokai võtnud tema enesele tütreks.
\par 8 Ja kui nüüd kuninga käsk ja seadus said teatavaks ja palju tütarlapsi koguti Suusani paleesse Heegai käe alla, siis võeti ka Ester kuningakotta Heegai käe alla, kes oli naistevalvaja.
\par 9 Tütarlaps meeldis temale ja leidis armu tema ees; ja ta tõttas temale andma iluravivahendeid ja kohast toitu, andis temale seitse valitud teenijat kuningakojast ning määras temale ja ta teenijaile parima paiga naistemajas.
\par 10 Ester ei teatanud oma rahvust ja päritolu, sest Mordokai oli teda keelanud seda teatamast.
\par 11 Ja Mordokai kõndis iga päev naistemaja õue ees, et teada saada, kuidas Estri käsi käib ja mis temaga juhtub.
\par 12 Kui igale tütarlapsele jõudis kätte kord, et tal tuli minna kuningas Ahasverose juurde pärast kaheteistkümne kuu möödumist, nagu naistele oli määratud - sest nõnda kaua kestsid nende iluravipäevad: kuus kuud mürriõliga ja kuus kuud palsamite ning muude naiste iluravivahenditega -,
\par 13 siis võis tütarlaps minna kuninga juurde ja temale anti naistemajast kuningakotta minnes kaasa kõik, mida ta soovis.
\par 14 Ta läks õhtul ja tuli hommikul tagasi teise naistemajasse eunuhhii Saasgase, kuninga liignaistevalvaja käe alla; ta ei läinud enam kuninga juurde, olgu siis, et ta kuningale meeldis ja teda nimeliselt kutsuti.
\par 15 Kui nüüd Estril, Mordokai sugulase Abihaili tütrel, kelle Mordokai oli võtnud enesele tütreks, tuli minna kuninga juurde, siis ei palunud ta midagi muud kui seda, mida eunuhh Heegai, kuninga naistevalvaja, soovitas; aga Ester leidis armu kõigi silmis, kes teda nägid.
\par 16 Ja Ester viidi kuningas Ahasverose juurde tema kuninglikku kotta kümnendas kuus, see on teebetikuus, tema seitsmendal valitsemisaastal.
\par 17 Ja kuningas armastas Estrit enam kui kõiki teisi naisi, ja Ester leidis tema ees armu ja heldust enam kui kõik teised neitsid; ta pani temale kuningliku krooni pähe ja tegi ta Vasti asemel kuningannaks.
\par 18 Ja kuningas tegi suure võõruspeo kõigile oma vürstidele ja sulastele, võõruspeo Estri auks; ta alandas maade makse ja jagas kuningakohaseid kingitusi.
\par 19 Kui teist korda neitseid koguti ja Mordokai istus kuninga väravas -
\par 20 Ester ei olnud teatanud oma päritolu ja rahvust, nagu Mordokai teda oli keelanud, ja Ester tegi Mordokai käsu järgi nagu siis, kui ta oli tema kasvandik -,
\par 21 siis neilsamul päevil, kui Mordokai istus kuninga väravas, vihastasid Bigtan ja Teres, kaks eunuhhi kuninga lävehoidjate hulgast, ja tahtsid pista kätt kuningas Ahasverose külge.
\par 22 Aga Mordokai sai sellest teada ja ta jutustas sellest kuninganna Estrile, ja Ester rääkis seda kuningale Mordokai nimel.
\par 23 Ja kui asja uuriti ning leiti nõnda olevat, siis poodi nad mõlemad puusse; ja see kirjutati kuninga jaoks ajaraamatusse.

\chapter{3}

\par 1 Pärast neid sündmusi ülendas kuningas Ahasveros agaglase Haamani, Hammedata poja, edutas teda ja pani ta istme ülemale kõigist vürstidest, kes ta juures olid.
\par 2 Ja kõik kuninga sulased, kes olid kuninga väravas, pidid heitma põlvili ja kummardama Haamanit, sest nõnda oli kuningas tema suhtes käskinud; aga Mordokai ei põlvitanud ega kummardanud.
\par 3 Siis ütlesid kuninga sulased, kes olid kuninga väravas, Mordokaile: „Miks sa kuninga käsust üle astud?”
\par 4 Ja kui nad seda temale iga päev olid öelnud, aga tema neid kuulda ei võtnud, siis teatasid nad sellest Haamanile, et näha, kas Mordokai põhjus on küllaldane, kuna ta neile oli öelnud enese juudi olevat.
\par 5 Kui Haaman nägi, et Mordokai ei põlvitanud ja teda ei kummardanud, siis täitus Haaman vihaga.
\par 6 Aga sellest oli tema silmis vähe, et pista kätt üksnes Mordokai külge - sest temale oli Mordokai rahvus teatavaks tehtud -, ja nõnda püüdis Haaman hävitada kõiki juute, kes olid kogu Ahasverose kuningriigis.
\par 7 Esimeses kuus, see on niisanikuus, kuningas Ahasverose kaheteistkümnendal aastal, heideti Haamani ees päevast päeva puuri, see on liisku; nõnda kuust kuusse kaheteistkümnendani, see on adarikuuni.
\par 8 Ja Haaman ütles kuningas Ahasverosele: „Üks rahvas on hajutatuna ja eraldatuna rahvaste keskel sinu kuningriigi kõigis maades; nende seadused on teistsugused kui kõigil muudel rahvastel ja nad ei tee kuninga seaduste järgi; seepärast ei ole kuningal kasulik neid sallida.
\par 9 Kui kuningas heaks arvab, siis käskigu ta kirjutada, et nad tuleb hukata; mina vaen siis kümme tuhat talenti hõbedat ametnike kätte, et see viidaks kuninga varaaitadesse.”
\par 10 Siis võttis kuningas oma pitserisõrmuse sõrmest ja andis selle agaglasele Haamanile, Hammedata pojale, juutide vaenlasele.
\par 11 Ja kuningas ütles Haamanile: „Hõbe olgu antud sulle, nõndasamuti rahvas, et sa temaga talitaksid, nagu su meelest hea on!”
\par 12 Ja esimese kuu kolmeteistkümnendal päeval kutsuti kuninga kirjutajad ning lasti kirjutada kõik, mida Haaman käskis, kuninga asehaldureile ja maavalitsejaile, kes olid iga maa üle, ja iga rahva vürstidele, igale maale tema kirjaviisis ja igale rahvale tema keeles; seda kirjutati kuningas Ahasverose nimel ja see kinnitati kuninga pitserisõrmusega.
\par 13 Siis läkitati jooksjatega kirjad kõigisse kuninga maadesse, et tuleb hävitada, tappa ja hukata kõik juudid, noored ja vanad, lapsed ja naised ühel päeval, kaheteistkümnenda kuu, see on adarikuu kolmeteistkümnendal päeval, ja et nende varandus tuleb riisuda.
\par 14 Kirja ärakiri oli antud seadusena igale maale, avaldamiseks kõigile rahvastele, et nad oleksid selleks päevaks valmis.
\par 15 Jooksjad läksid kuninga käsul rutates välja, kui seadus Suusani palees oli antud; kuningas ja Haaman istusid jooma, aga Suusani linn oli ärevuses.

\chapter{4}

\par 1 Kui Mordokai sai teada kõik, mis oli sündinud, siis ta käristas oma riided lõhki, riietus kotiriidesse ja raputas tuhka pea peale, läks linna keskele ning kisendas valjusti ja kibedasti.
\par 2 Ja ta läks kuninga värava ette, sest kotiriides ei olnud lubatud kuninga väravast sisse minna.
\par 3 Igal maal, igas paigas, kuhu kuninga käsk ja tema seadus jõudis, oli juutidel suur lein, paast, nutt ja hala; kotiriie ja tuhk olid paljudele asemeks.
\par 4 Kui Estri teenijad ja eunuhhid tulid ning temale sellest teatasid, siis tundis kuninganna suurt valu ja ta läkitas Mordokaile riideid selga panemiseks, et ta saaks oma kotiriide seljast ära võtta; aga tema ei võtnud neid vastu.
\par 5 Siis kutsus Ester Hataki, ühe kuninga eunuhhidest, kes oli pandud teda teenima, ja andis temale käsu hankida teateid Mordokai kohta, et mis on ja mispärast see nõnda on.
\par 6 Ja Hatak läks Mordokai juurde linna väljakule, mis oli kuninga värava ees,
\par 7 ja Mordokai jutustas talle kõik, mis temaga oli juhtunud, ja andis täpseid andmeid hõbeda kohta, mida Haaman oli lubanud vaagida kuninga varaaitadesse juutide hukkamise eest.
\par 8 Ja ta andis temale ärakirja seadusekirjast, mis Suusanis oli antud nende hävitamiseks, et ta näitaks seda Estrile ja jutustaks talle, ja käsiks teda minna kuninga juurde temalt armu paluma ja oma rahvale abi otsima.
\par 9 Ja Hatak tuli ning andis Estrile edasi Mordokai sõnad.
\par 10 Aga Ester rääkis Hatakiga ja käskis tal Mordokaile öelda:
\par 11 „Kõik kuninga sulased ja kuninga maade rahvad teavad, et igaühe jaoks, olgu see mees või naine, kes kutsumata läheb kuninga juurde siseõue, on ainult üks seadus: ta peab surema; või olgu siis, kui kuningas sirutab oma kuldkepi tema poole, et ta võib jääda elama. Aga mind ei ole kuninga juurde kutsutud juba kolmkümmend päeva.”
\par 12 Kui Estri sõnad Mordokaile edasi anti,
\par 13 siis Mordokai käskis Estrile vastata: „Ära mõtle, et sa kuningakojas paremini pääsed kui kõik muud juudid!
\par 14 Sest kui sa sel ajal tõesti vaikid, tuleb juutidele abi ja pääste mujalt, aga sina ja su isa pere hukkute! Ja kes teab, kas sa mitte ei olegi just selle asja pärast pääsenud kuninglikku seisusesse?”
\par 15 Siis Ester käskis Mordokaile vastata:
\par 16 „Mine kogu kokku kõik Suusanis leiduvad juudid ja paastuge minu pärast; kolm päeva ärge sööge ega jooge ei öösel ega päeval. Minagi paastun oma teenijatega nõnda. Ja siis ma lähen kuninga juurde, kuigi see pole seadusepärane. Aga kui ma hukkun, siis hukkun!”
\par 17 Siis Mordokai läks ja tegi kõik nõnda, nagu Ester teda oli käskinud.

\chapter{5}

\par 1 Ja kolmandal päeval pani Ester selga kuninglikud rõivad ning astus kuningakoja siseõue, mis oli otse kuningakoja vastas; ja kuningas istus oma kuninglikul aujärjel kuninglikus kojas, koja ukse vastas.
\par 2 Kui kuningas nägi kuninganna Estrit õues seisvat, siis leidis too armu tema silmis ja kuningas sirutas Estri poole kuldkepi, mis tal käes oli; siis Ester astus ligi ja puudutas kepiotsa.
\par 3 Ja kuningas ütles temale: „Mis sul tarvis on, kuninganna Ester, ja mida sa palud? Kui see oleks ka pool kuningriiki, see antakse sulle!”
\par 4 Ja Ester vastas: „Kui kuningas heaks arvab, siis tulgu kuningas ja Haaman täna peole, mille ma temale olen korraldanud!”
\par 5 Kuningas ütles: „Saatke kiiresti Haaman, et saaksime teha, nagu Ester on öelnud!” Ja Kuningas ning Haaman tulid peole, mille Ester oli korraldanud.
\par 6 Ja veini juues ütles kuningas Estrile: „Mida sa palud? See antakse sulle! Ja su soov, olgu see või pool kuningriiki, täidetakse!”
\par 7 Aga Ester kostis ning ütles: „Mu palve ja soov on:
\par 8 kui ma kuninga silmis olen armu leidnud ja kui kuningas heaks arvab anda, mida ma palun, ja teha, mida ma soovin, siis tulgu kuningas ja Haaman peole, mille ma neile korraldan; ma teen siis homme, nagu kuningas soovib!”
\par 9 Ja Haaman läks sel päeval sealt ära rõõmsana ja heas tujus; aga kui Haaman nägi Mordokaid kuninga väravas, et too ei tõusnud püsti ega värisenud tema ees, siis Haaman sai täis tulist viha Mordokai vastu.
\par 10 Aga Haaman talitses ennast ja läks oma kotta; siis ta läkitas sõna ja laskis tuua oma sõbrad ja oma naise Serese.
\par 11 Ja Haaman jutustas neile oma suurest rikkusest ja oma paljudest poegadest, ja sellest, kuidas kuningas teda oli ülendanud ja kuidas ta teda oli edutanud kuninga vürstide ja sulaste ees.
\par 12 Ja Haaman ütles: „Isegi kuninganna Ester ei lasknud kedagi muud kui mind tulla koos kuningaga peole, mille ta korraldas. Ja homsekski on ta mind kutsunud koos kuningaga.
\par 13 Aga see kõik ei rahulda mind, niikaua kui ma näen juut Mordokaid istumas kuninga väravas!”
\par 14 Siis ütlesid ta naine Seres ja kõik ta sõbrad temale: „Tehtagu puu, viiskümmend küünart kõrge, ja ütle hommikul kuningale, et Mordokai sinna poodaks. Siis sa võid rõõmsalt koos kuningaga minna peole!” See kõne oli Haamani meelest hea ja ta käskis puu valmis teha.

\chapter{6}

\par 1 Sel ööl ei saanud kuningas und; siis käskis ta tuua meelespeetavate sündmuste kirja, ajaraamatu. Ja kui sellest kuningale ette loeti,
\par 2 siis leiti kirjutatud olevat, kuidas Mordokai oli andnud teateid Bigtani ja Teresi, kuninga kahe eunuhhi kohta lävehoidjate hulgast, et need olid tahtnud pista oma kätt kuningas Ahasverose külge.
\par 3 Ja kuningas küsis: „Missugune austus ja ülendus on Mordokaile selle eest osaks saanud?„ Kuninga sulased, kes teda teenisid, vastasid: ”Ta pole midagi saanud.”
\par 4 Siis küsis kuningas: „Kes on õues?” Haaman oli just tulnud kuningakoja välisõue kuningale ütlema, et Mordokai poodaks puusse, mille ta tema jaoks oli käskinud valmistada.
\par 5 Ja kuninga sulased vastasid temale: „Vaata, Haaman seisab õues.„ Ja kuningas ütles: ”Ta tulgu sisse!”
\par 6 Ja Haaman tuli ning kuningas küsis temalt: „Mida tuleks teha selle mehega, keda kuningas tahab austada?„ Siis mõtles Haaman oma südames: ”Kellele muule tahab kuningas rohkem austust avaldada kui minule?”
\par 7 Ja Haaman vastas kuningale: „Mehele, keda kuningas tahab austada,
\par 8 toodagu kuninglik kuub, millesse kuningas ise on olnud riietatud, ja hobune, kelle seljas kuningas on ratsutanud ja kellele kuninglik peaehe on pähe pandud.
\par 9 Kuub ja hobune antagu kuninga ühe vürsti kätte ülikute hulgast; ja riietatagu mees, keda kuningas tahab austada, ning sõidutatagu teda hobuse seljas linna väljakule ja hüütagu tema ees: Nõnda tehakse mehega, keda kuningas tahab austada!”
\par 10 Siis ütles kuningas Haamanile: „Tõtta, võta kuub ja hobune, nagu sa oled öelnud, ja tee nõnda juut Mordokaiga, kes istub kuninga väravas! Ära jäta midagi tegemata kõigest sellest, mis sa oled öelnud!”
\par 11 Ja Haaman võttis kuue ja hobuse, riietas Mordokai ja sõidutas ta linna väljakule ning hüüdis tema ees: „Nõnda tehakse mehega, keda kuningas tahab austada!”
\par 12 Seejärel pöördus Mordokai tagasi kuninga väravasse, Haaman aga ruttas oma kotta leinates ja kinnikaetud peaga.
\par 13 Kui Haaman jutustas Seresele, oma naisele, ja kõigile oma sõpradele kõik, mis temale oli juhtunud, siis ütlesid talle tema targad ja ta naine Seres: „Kui Mordokai, kelle ees sa oled hakanud langema, on juutide soost, siis sa ei suuda tema vastu midagi, vaid sa langed tõesti tema ees.”
\par 14 Kui nad alles temaga rääkisid, saabusid kuninga eunuhhid ja kiirustasid Haamanit minema sellele peole, mille Ester oli korraldanud.

\chapter{7}

\par 1 Kui kuningas ja Haaman olid tulnud kuninganna Estri juurde peole,
\par 2 siis ütles kuningas veini juues Estrile ka teisel päeval: „Mida sa palud, kuninganna Ester, see antakse sulle! Ja su soov, olgu see või pool kuningriiki, täidetakse!”
\par 3 Ja kuninganna Ester kostis ning ütles: „Kui ma olen sinu silmis armu leidnud, kuningas, ja kui kuningas heaks arvab, siis kingitagu mulle mu palvel mu elu ja vastavalt mu soovile mu rahvas!
\par 4 Sest meie, mina ja mu rahvas, oleme müüdud hävitamiseks, tapmiseks ja hukkamiseks. Kui meid oleks müüdud sulaseiks ja teenijaiks, siis ma oleksin vaikinud, sest seesugune õnnetus ei oleks kuninga tülitamiseks küllaldane.”
\par 5 Siis rääkis kuningas Ahasveros ja küsis kuninganna Estrilt: „Kes see on ja kus see on, kes on julgenud nõnda teha?”
\par 6 Ja Ester vastas: „Vaenlane ja vihamees on see kuri Haaman!” Siis kohkus Haaman kuninga ja kuninganna ees.
\par 7 Ja kuningas tõusis vihasena veini joomast ning läks palee rohuaeda, aga Haaman jäi paluma kuninganna Estrit oma elu pärast, sest ta nägi, et kuninga poolt oli tema kohta halba otsustatud.
\par 8 Kui kuningas tuli palee rohuaiast tagasi kotta, kus veini oli joodud, siis oli Haaman langenud asemele, millel Ester lamas. Ja kuningas ütles: „Kas ta tahab isegi kuningannat mu oma kojas ära naerda?” Vaevalt oli see sõna kuninga suust välja tulnud, kui juba Haamani nägu kaeti.
\par 9 Ja Harbona, üks kuningat teenivatest eunuhhidest, ütles: „Ennäe, seal on ka puu, mille Haaman laskis valmistada Mordokai jaoks, kes kuninga huvides kõneles. See seisab Haamani koja juures, viiskümmend küünart kõrge.„ Ja kuningas ütles: ”Pooge ta sinna!”
\par 10 Ja Haaman poodi puusse, mille ta Mordokai jaoks oli lasknud valmistada; siis lahtus kuninga viha.

\chapter{8}

\par 1 Selsamal päeval andis kuningas Ahasveros kuninganna Estrile juutide vaenlase Haamani koja; ja Mordokai võis tulla kuninga ette, sest Ester oli teatanud, kes ta temale oli.
\par 2 Ja kuningas tõmbas oma pitserisõrmuse, mille ta Haamanilt oli ära võtnud, ja andis selle Mordokaile; ja Ester pani Mordokai Haamani koja üle.
\par 3 Aga Ester rääkis veel kuninga ees ja heitis tema jalge ette maha, nuttis ja anus teda, et ta keelaks agaglase Haamani kurjuse ja tema kava, mille ta juutide vastu oli sepitsenud.
\par 4 Siis sirutas kuningas kuldkepi Estri poole ja Ester tõusis ning seisis kuninga ees
\par 5 ja ütles: „Kui kuningas heaks arvab ja kui ma tema ees olen armu leidnud, kui see asi on kuninga meelest õige ja mina tema silmis vastuvõetav, siis lase kirjutada, et agaglase Haamani, Hammedata poja kirjad kavaga, mis ta kirjutas kõigis kuninga maades olevate juutide hukkamiseks, tagasi võetaks!
\par 6 Sest kuidas võiksin näha õnnetust, mis mu rahvast tabab, ja kuidas võiksin vaadata oma sugulaste hukkumist?”
\par 7 Siis ütles kuningas Ahasveros kuninganna Estrile ja juut Mordokaile: „Vaata, ma olen andnud Haamani koja Estrile ja Haaman ise on puusse poodud, sellepärast et ta tahtis pista oma käe juutide külge.
\par 8 Ja nüüd kirjutage teie ise juutide pärast kuninga nimel, nagu teie meelest hea on, ja kinnitage see kuningliku pitserisõrmusega; sest kirja, mis kuninga nimel on kirjutatud ja kuningliku pitserisõrmusega on kinnitatud, ei saa tagasi võtta!”
\par 9 Siis kutsuti otsekohe kuninga kirjutajad - see oli kolmandas kuus, siivanikuus, selle kahekümne kolmandal päeval - ja kirjutati kõik, mida Mordokai käskis, juutidele ja asehaldureile, maavalitsejaile ja maade vürstidele Indiast Etioopiani, saja kahekümne seitsmel maal, igale maale tema kirjaviisis ja igale rahvale tema keeles; ka juutidele nende kirjaviisis ja keeles.
\par 10 Seda kirjutati kuningas Ahasverose nimel ja kinnitati kuninga pitserisõrmusega; ja kirjad läkitati ratsakäskjalgadega, kes ratsutasid kuninglikel hobustel, võidusõiduratsudel.
\par 11 Kirjas lubas kuningas juutidel igas linnas kokku tulla ja oma elu kaitsta, hävitades, tappes ja hukates iga rahva ja maa kõik relvastatud jõud, kes neile kallale kipuvad, ka lapsed ja naised, ja riisudes nende vara,
\par 12 ühel ja samal päeval kõigis kuningas Ahasverose maades. See päev oli kaheteistkümnenda kuu, see on adarikuu kolmeteistkümnendal päeval.
\par 13 Kirja ärakiri oli antud seadusena igale maale, avaldamiseks kõigile rahvastele, et juudid selleks päevaks oleksid valmis oma vaenlastele kätte maksma.
\par 14 Kuninglikel hobustel ratsutavad käskjalad läksid kuninga käsul kiiresti välja ja ruttasid, niipea kui seadus oli Suusani palees antud.
\par 15 Ja Mordokai tuli kuninga juurest kuninglikus kuues, sinises ja valges, suure kuldkrooniga ning valge ja purpurpunase mantliga; ja Suusani linn hõiskas ja rõõmustas.
\par 16 Juutidel oli õnn ja rõõm, õndsus ja au.
\par 17 Ja igal maal ja igas linnas, igal pool, kuhu kuninga käsk ja tema seadus jõudis, oli juutidel rõõm ja ilutsemine, pidu ja head päevad; ja maa rahvast tunnistasid paljud endid juutideks, sest neid valdas hirm juutide ees.

\chapter{9}

\par 1 Ja kaheteistkümnendas kuus, see on adarikuus, selle kolmeteistkümnendal päeval, kui kuninga käsu ja tema seaduse teostamise aeg oli kätte jõudnud, sel päeval, kui juutide vaenlased lootsid saada võimust nende üle, mis oli aga muudetud, nõnda et juudid pidid saama võimust oma vihkajate üle,
\par 2 tulid juudid kokku oma linnades kõigis kuningas Ahasverose maades, et pista oma käsi nende külge, kes neile kurja olid soovinud; ja ükski ei suutnud neile vastu panna, sest hirm nende ees oli vallanud kõiki rahvaid.
\par 3 Ja kõik maade vürstid, asehaldurid ja maavalitsejad ning kuninga ametnikud aitasid juute, sest neid oli vallanud hirm Mordokai ees.
\par 4 Sest Mordokai oli kuningakojas suur, ja kuuldus temast levis kõigisse maadesse; sest see mees, Mordokai, sai üha vägevamaks.
\par 5 Ja juudid lõid kõiki oma vaenlasi mõõgaga lüües, tappes ja hukates, ja nad talitasid oma vihkajatega, nagu tahtsid.
\par 6 Suusani palees tapsid ja hukkasid juudid viissada meest;
\par 7 ja Parsandata, Dalfon, Aspata,
\par 8 Poorata, Adalja, Aridata,
\par 9 Parmasta, Arisai, Aridai ja Vajesata,
\par 10 Haamani, Hammedata poja, juutide vaenlase kümme poega tapeti, aga tema vara külge nad ei pistnud oma kätt.
\par 11 Selsamal päeval sai Suusani palees tapetute arv kuningale teatavaks.
\par 12 Ja kuningas ütles kuninganna Estrile: „Suusani palees on juudid tapnud ja hukanud viissada meest ja kümme Haamani poega. Mida nad siis küll muudes kuninga maades on teinud? Mida sa nüüd palud? See antakse sulle! Ja mis su soov veel olekski, see täidetakse!”
\par 13 Ja Ester ütles: „Kui kuningas heaks arvab, siis lubatagu ka homme juutidel, kes Suusanis on, talitada nagu täna! Ja kümme Haamani poega poodagu puusse!”
\par 14 Ja kuningas käskis nõnda teha ning Suusanis anti seadus; ja kümme Haamani poega poodi.
\par 15 Ja Suusani juudid tulid kokku ka adarikuu neljateistkümnendal päeval ning tapsid Suusanis kolmsada meest; aga nende vara külge nad ei pistnud oma kätt.
\par 16 Ka muud juudid, kes olid kuninga maades, tulid kokku ja kaitsesid oma elu ja said hingata oma vaenlastest ning tapsid oma vihkajaist seitsekümmend viis tuhat; aga nende vara külge nad ei pistnud oma kätt.
\par 17 See sündis adarikuu kolmeteistkümnendal päeval, aga selle neljateistkümnendal päeval nad puhkasid ning tegid selle pidu- ja rõõmupäevaks.
\par 18 Aga Suusani juudid tulid kokku selle kuu kolmeteistkümnendal ja neljateistkümnendal päeval, ja nad puhkasid selle kuu viieteistkümnendal päeval ning tegid selle pidu- ja rõõmupäevaks.
\par 19 Seepärast tegid juudi maamehed, kes elasid maal külades, adarikuu neljateistkümnenda päeva endile rõõmu- ja pidupäevaks, heaks päevaks, ja läkitasid üksteisele rooga.
\par 20 Ja Mordokai kirjutas need sündmused üles; ja ta läkitas kirjad kõigile juutidele kuningas Ahasverose kõigis maades, niihästi ligidal kui kaugel olijaile,
\par 21 kohustades neid, et nad igal aastal pühitseksid adarikuu neljateistkümnendat ja viieteistkümnendat päeva
\par 22 kui päevi, mil juudid said rahu oma vaenlastest, ja kuud, mil nende mure muutus rõõmuks ja lein pidupäevaks; nad pidid need tegema pidu- ja rõõmupäevadeks ning läkitama üksteisele rooga ja vaestele ande.
\par 23 Ja juudid võtsid tavaks, mida nad juba olid hakanud tegema ja millest Mordokai neile kirjutas.
\par 24 Kuna agaglane Haaman, Hammedata poeg, kõigi juutide vaenlane, oli kavatsenud juute hukata ja oli puuri, see on liisku heitnud nende tagakiusamiseks ja hukkamiseks -
\par 25 ent kui asi jõudis kuninga ette, oli tema kirjalikult käskinud, et Haamani kuri kavatsus, mille ta juutide vastu oli sepitsenud, tuleks ta enese pea peale ja et tema ja ta pojad puusse poodaks -,
\par 26 siis nimetasid nad neid päevi puurimiks, sõna „puur” järgi. Seepärast, vastavalt kõigile selle kirja sõnadele ja vastavalt sellele, mida nad ise olid näinud ja mis neile oli juhtunud,
\par 27 seadsid ja võtsid juudid muutmatuks kohustuseks ja tavaks iseendile ja oma järeltulijaile ja kõigile nendega liitujaile pühitseda igal aastal neid kahte päeva vastavalt eeskirjale ja määratud ajale,
\par 28 ja et neid päevi meenutaks ja pühitseks iga rahvapõlv ja iga suguvõsa igal maal ja igas linnas, et need puurimipäevad ei kaoks juutide hulgast ja et mälestus neist ei lakkaks nende järeltulijail.
\par 29 Ja kuninganna Ester, Abihaili tütar, ja juut Mordokai kirjutasid kogu oma mõjujõuga, et kinnitada seda teist puurimikirja.
\par 30 Ja kirjad läkitati kõigile juutidele Ahasverose kuningriigi saja kahekümne seitsmel maal sõbralike ja usaldusväärsete sõnadega,
\par 31 et neid puurimipäevi peetaks määratud ajal, nagu juut Mordokai ja kuninganna Ester olid need neile seadnud ja nagu nad ise olid endile ja oma järeltulijaile korraldusi teinud paastu ja kaebehüüu asjus.
\par 32 Nõnda kinnitas Estri käsk need puurimi eeskirjad, ja need kirjutati raamatusse.

\chapter{10}

\par 1 Ja kuningas Ahasveros pani maale ja mere saartele peale töökohustuse.
\par 2 Ja kõik tema võimsuse ja vägevuse teod ning täpsed andmed Mordokai suurusest, kuidas kuningas teda ülendas - eks need ole kirja pandud Meedia ja Pärsia kuningate Ajaraamatus?
\par 3 Sest juut Mordokai oli kuningas Ahasverosest järgmine mees, ta oli juutidele suur ja oma paljudele vendadele armas; tema taotles oma rahvale head ja kõneles kogu oma soo hüvanguks.




\end{document}