\begin{document}

\title{Pauluse Teine kiri Timoteusele}

\chapter{1}

\section*{Tervitus ja tänu}

\par 1 Paulus, Jumala tahtel Kristuse Jeesuse apostel tõotuse järgi elust, mis on Kristuses Jeesuses,
\par 2 armsale pojale Timoteosele: armu, halastust, rahu Jumalalt Isalt ja Kristuselt Jeesuselt, meie Issandalt!
\par 3 Ma tänan Jumalat, keda ma oma esivanemaist saadik teenin puhta südametunnistusega, nagu ma ka lakkamata mõtlen sinule oma palvetes ööd ja päevad,
\par 4 igatsedes sind näha ja meelde tuletades sinu pisaraid, et täituda rõõmuga,
\par 5 kui mulle meenub sinu silmakirjatsematu usk, mis enne elas sinu vanaemas Loises ja sinu emas Euniikes ja nüüd, nagu ma olen veendunud, elab ka sinus.

\section*{Manitsus agaruseks ja julgustus}

\par 6 Selles asjas ma tuletan sinule meelde, et sa õhutaksid lõkkele Jumala armuande, mis sulle sai minu käte pealepanemise kaudu.
\par 7 Sest Jumal ei ole meile andnud arguse vaimu, vaid väe ja armastuse ja mõistliku meele vaimu.
\par 8 Ärgu olgu sul siis häbi meie Issanda tunnistusest ega minust, tema vangist, vaid kannata kurja kaasa evangeeliumiga Jumala väge mööda,
\par 9 kes meid on päästnud ja on kutsunud püha kutsega, mitte meie tegusid mööda, vaid oma nõu ja armu järgi, mis meile on antud Kristuses Jeesuses enne igavesi aegu;
\par 10 ent nüüd on saanud avalikuks meie Õnnistegija Kristuse Jeesuse ilmumise läbi, kes on hävitanud surma ja toonud valge ette elu ja kadumatu põlve evangeeliumi kaudu,
\par 11 milleks mina olen seatud kuulutajaks ja apostliks ja õpetajaks.
\par 12 Sel põhjusel ma ka kannatan seda ega ole mul häbi sellest; sest ma tean, kellesse ma usun, ja olen julge selles, et ta on vägev säilitama minu kätte ustud vara tolle päevani.
\par 13 Tervete sõnade eeskujuks võta, mida sa oled kuulnud minult usus ja armastuses, mis on Kristuses Jeesuses.
\par 14 Kaunist sinu hooleks antud vara hoia Püha Vaimu läbi, kes meis elab.
\par 15 Sa tead seda; et kõik, kes on Aasias, on löönud lahku minust, nende seas Fügelos ja Hermogenes.
\par 16 Issand osutagu halastust Onesiforose perele, sest tema on mind sagedasti kosutanud ega ole mitte häbenenud mu ahelaid,
\par 17 vaid kui ta saabus Rooma, otsis ta mind suure hoolega ning leidis mind.
\par 18 Issand andku temale leida halastust Issanda juures tol päeval! Ja kui palju ta Efesoses oli mulle abiks, tead sina paremini.


\chapter{2}

\section*{Hoiatus riiu eest}

\par 1 Saa siis vahvaks, mu poeg, armus, mis on Kristuses Jeesuses!
\par 2 Ja mis sa minult oled kuulnud paljude tunnistajate abil, see anna ustavate inimeste kätte, kes on osavad õpetama ka teisi.
\par 3 Kannata siis ühtlasi kurja kui Jeesuse Kristuse õilis sõdur!
\par 4 Mitte ükski, kes on sõjateenistuses, ei anna ennast peatoiduse hankimise kimpu, muidu ta ei saa olla selle meelt mööda, kes ta on sõtta kutsunud.
\par 5 Ja kui keegi võitleb, ei saa ta võidupärga, kui ta ei võitle seadusepäraselt!
\par 6 Töötaja põllumees peab kõige esiti viljast osa saama!
\par 7 Saa aru, mis ma ütlen! Ent Issand andku sulle arusaamist kõigest.
\par 8 Pea meeles Jeesust Kristust, kes on surnuist üles äratatud, kes on Taaveti soost minu evangeeliumi järgi,
\par 9 mille pärast mina ka kannatan kurja, kandes isegi ahelaid otsekui kurjategija! Kuid Jumala sõna ei ole aheldatud!
\par 10 Sel põhjusel ma talun kõike nende pärast, kes on valitud, et nemadki saavutaksid õndsuse Kristuses Jeesuses igavese auga.
\par 11 Ustav on see sõna: kui me ühes temaga oleme surnud, siis me ka elame ühes temaga;
\par 12 kui me ühes kannatame, siis me ka valitseme ühes temaga; kui me tema salgame, siis tema salgab ka meid;
\par 13 kui me ei usu, tema jääb siiski ustavaks; ta ei või ennast salata.
\par 14 Seda tuleta meelde ja kinnita Jumala ees, et nad ei riidleks sõnade pärast. Sellest ei ole mingit kasu, see segab kuuljate meeli.
\par 15 Püüa hoolega osutuda kõlbavaks Jumalale kui töötegija, kellel pole tarvis häbeneda, kes tõesõna jagab õieti.
\par 16 Kuid kõlvatute tühjade juttude eest hoidu eemale; nende rääkijad lähevad ju ikka kaugemale jumalakartmatuses
\par 17 ja nende kõne sööbib enese ümber otsekui vähktõbi; nende seast on Hümenaios ja Fileetos,
\par 18 kes on eksinud ära tõest ja ütlevad ülestõusmise juba olnud olevat, ja rikuvad mõnede usu.
\par 19 Ent Jumala rajatud alus püsib kindlana ja temal on see pitser: Issand tunneb neid, kes tema omad on; ja: ülekohtust loobugu igaüks, kes Issanda nime nimetab.
\par 20 Aga suures majas ei ole mitte ainult kuld- ja hõbe-, vaid ka puuastjaid ja saviastjaid, ja muist on väärikamaks ja muist halvemaks tarvitamiseks.
\par 21 Kui keegi nüüd iseennast neist puhastab, saab ta astjaks väärikama tarvituse jaoks ja on pühitsetud ning tarvilik oma isandale, kõlvuline igaks heaks teoks.
\par 22 Põgene nooreea himude eest! Taotle õigust, usku, armastust, rahu nendega, kes Issandat appi hüüavad puhtast südamest!
\par 23 Väldi rumalaid ja lapsikuid küsimusi; sest sa tead, et need toovad tüli.
\par 24 Ent Issanda sulane ärgu tülitsegu, vaid olgu lahke kõikide vastu, osav õpetama, valmis kannatama kurja,
\par 25 kes tasase meelega noomib vastupanijaid, et Jumal neile kuidagi annaks meelt parandada ja tunnetada tõde,
\par 26 kaineneda ja vabaneda kuradi paelust, kes on nad kinni võtnud täitma tema tahtmist.


\chapter{3}

\section*{Kurjus ähvardab maad võtta}

\par 1 Aga seda tea, et viimseil päevil tuleb raskeid aegu.
\par 2 Sest inimesed on siis enesearmastajad, rahaahned, hooplejad, ülbed, teotajad, sõnakuulmatud vanemaile, tänamatud, õelad,
\par 3 südametud, leppimatud, laimajad, pillajad, toored, hea põlgajad,
\par 4 petturid, kergemeelsed, sõgedad, rohkem lõbuarmastajad kui jumalaarmastajad,
\par 5 kellel on jumalakartuse nägu, aga kes salgavad tema väge. Ja nendest hoidu eemale.
\par 6 Sest nende seas on ka mõned, kes poetuvad majadesse ja võtavad oma võrku pattudega koormatud ja mõnesuguste himudega aetavaid naisterahvaid,
\par 7 kes alati on õppijad, kuid kunagi ei pääse tõe tunnetusele:
\par 8 Aga otsekui Jannes ja Jambres hakkasid vastu Moosesele, nõnda hakkavad need vastu tõele; nad on inimesed, kes on arust ära ega pea paika usu poolest.
\par 9 Aga nad ei saa ka mitte enam kaugele, sest nende hullustus saab kõigile avalikuks, nagu ka nondega sündis.

\section*{Pääste tagatis Timoteosele}

\par 10 Ent sina oled järginud mind minu õpetuses, eluviisis, nõuandes, usus, pikas meeles, armastuses, kannatlikkuses,
\par 11 tagakiusamistes, kannatamistes, mis said mulle osaks Antiookias, Ikoonionis, Lüstras; milliseid tagakiusamisi ma kannatasin ja millest kõigist mind päästis Issand.
\par 12 Ja kõiki, kes tahavad elada jumalakartlikult Kristuses Jeesuses, kiusatakse taga.
\par 13 Aga kurjad inimesed ja petised lähevad ikka pahemaks, eksitades teisi ja eksides.
\par 14 Ent sina jää sellesse, mida oled õppinud ja milles oled kindel, sest sa tead, kellelt sa selle oled õppinud;
\par 15 ja et sina lapsest saadik tunned pühi kirju, mis võivad sind teha targaks, õndsuseks usu kaudu, mis on Kristuses Jeesuses.
\par 16 Kõik Kiri on Jumala Vaimu poolt sisendatud ja on kasulik õpetuseks, noomimiseks, parandamiseks, juhatamiseks õiguses,
\par 17 et Jumala inimene oleks täielik ja kõigele heale tööle valmistunud.


\chapter{4}

\section*{Pääste tagatis Timoteosele}

\par 1 Ma tunnistan kindlasti Jumala ja Kristuse Jeesuse ees, kes mõistab kohut elavate ja surnute üle, niihästi tema tulemist kui ka tema riiki.
\par 2 Kuuluta sõna, astu esile, olgu parajal või ebasobival ajal, noomi, hoiata, manitse kõige pika meelega ja õpetamisega.
\par 3 Sest tuleb aeg, et nad tervet õpetust ei taha sallida, vaid enestele otsivad õpetajaid iseeneste himude järgi, sedamööda kuidas nende kõrvad sügelevad,
\par 4 ja käänavad kõrvad ära tõest ning pöörduvad tühjade juttude poole.
\par 5 Aga sina ole igapidi kaine, kannata kurja, tee evangeeliumikuulutaja tööd, täida oma ametit õieti.
\par 6 Sest mind juba ohverdatakse ja minu lahkumiseaeg on jõudnud ligi.
\par 7 Ma olen head võitlemist võidelnud, ma olen oma jooksmise lõpetanud, ma olen usu säilitanud!
\par 8 Nüüd on minule tallele pandud õiguse pärg, mille Issand, õige kohtumõistja, mulle annab tol päeval, aga mitte ainult minule, vaid ka kõigile, kes armastavad ta ilmumist.

\section*{Isiklikud sõnumid Timoteosele ja jumalagajätt}

\par 9 Katsu ruttu minu juurde tulla!
\par 10 Sest Deemas jättis mind maha ning hakkas armastama seda maailma ja läks Tessaloonikasse, Kreskes Galaatiasse, Tiitus Dalmaatiasse.
\par 11 Luukas üksi on mu juures. Võta Markus ja too ta enesega, sest teda on mulle väga vaja abiliseametisse.
\par 12 Tühhikose ma läkitasin Efesosse.
\par 13 Kuub, mille ma jätsin Troasse Karpose juurde, too tulles kaasa, ja raamatud, eriti pärgamendid.
\par 14 Vasksepp Aleksandros on mulle teinud palju paha. Issand tasugu talle ta tegusid mööda!
\par 15 Hoidu ka sina tema eest, sest ta on kõvasti vastu pannud meie sõnadele.
\par 16 Kui ma esimest korda kohtus kostsin enese eest, ei olnud ükski mulle toeks, vaid kõik jätsid mind maha. Ärgu arvatagu seda neile süüks.
\par 17 Aga Issand seisis minu eest ja kinnitas mind, et sõna kuulutamine minu läbi teostuks täiel määral ja kõik paganad kuuleksid seda; ja ma pääsesin lõukoera suust.
\par 18 Küll Issand mind ka välja tõmbab kõigist kurjadest tegudest ja aitab oma taevasesse riiki. Temale olgu austus ajastute ajastuteni! Aamen.
\par 19 Tervita Priskat ja Akvilat ja Onesiforose peret.
\par 20 Erastos jäi Korintosesse; Trofimose ma jätsin haigelt Mileetosesse.
\par 21 Rutta, et sa enne talve saabud! Eubulos ja Puudes ja Linos ja Klaudia ja kõik vennad tervitavad sind.
\par 22 Issand Jeesus Kristus olgu sinu vaimuga! Arm olgu teiega!





\end{document}