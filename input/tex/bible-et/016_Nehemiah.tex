\begin{document}

\title{Nehemja raamat}

\chapter{1}

\par 1 Nehemja, Hakalja poja sõnad. „Ja see sündis kahekümnenda aasta kislevikuus, et ma olin Suusani palees.
\par 2 Siis tuli Hanani, üks mu vendadest, tema ise ja mõned mehed Juudamaalt. Ma küsitlesin neid pääsenud juutide kohta, kes vangiviimisel olid alles jäänud, ja Jeruusalemma kohta.
\par 3 Ja nemad ütlesid mulle: „Need, kes seal maal vangiviimisest on alles jäänud, on suures õnnetuses ja häbis, Jeruusalemma müürid on maha kistud ja selle väravad tulega põletatud.”
\par 4 Kui ma neid sõnu kuulsin, siis ma istusin maha, nutsin ja leinasin mitu päeva, ja ma paastusin ja palvetasin taeva Jumala ees.
\par 5 Ja ma ütlesin: „Oh Issand, taeva Jumal, suur ja kartustäratav Jumal, kes hoiab lepingut ja heldust neile, kes teda armastavad ja tema käske peavad!
\par 6 Olgu nüüd su kõrv tähelepanelik ja su silmad lahti kuulma oma sulase palvet, mida ma nüüd sinu ees palvetan päevad ja ööd, Iisraeli laste, su sulaste pärast, ja millega ma tunnistan Iisraeli laste patte, mis me sinu vastu oleme teinud! Ka mina ja mu isa pere oleme pattu teinud.
\par 7 Me oleme väga kõlvatult talitanud sinu vastu ega ole pidanud käske, määrusi ja seadlusi, mis sa Moosesele, oma sulasele, andsid.
\par 8 Tuleta nüüd meelde seda sõna, mille sa andsid Moosesele, oma sulasele, öeldes: Kui te olete truuduseta, siis ma pillutan teid rahvaste sekka,
\par 9 aga kui te pöördute minu juurde ja peate minu käske ning teete nende järgi, siis kui teie pillutatud oleksid kas või taeva servas, ma kogun nad sealtki ja viin paika, mille ma olen valinud eluasemeks oma nimele.
\par 10 Need on ju sinu sulased ja sinu rahvas, kelle sa oled lunastanud oma suure rammu ja vägeva käega.
\par 11 Oh Issand, pangu nüüd su kõrv tähele oma sulase palvet ja oma sulaste palvet, kes tahavad karta su nime! Ja lase siis täna korda minna oma sulase nõu ja anna temale halastust selle mehe ees! Mina olin nimelt kuninga joogikallajate ülem.

\chapter{2}

\par 1 Ja niisanikuus, kuningas Artahsasta kahekümnendal aastal, oli tema ees vein; mina tõin veini ja andsin kuningale. Kuna ma tema palge ees ei olnud iialgi olnud kurb,
\par 2 siis kuningas ütles mulle: „Mispärast sa oled kurva näoga? Ega sa ometi haige ole? See ei ole muud kui südame kurbus.” Siis ma kartsin üliväga
\par 3 ja ütlesin kuningale: „Kuningas elagu igavesti! Miks ma ei peaks olema kurva näoga, kui see linn, kus on mu vanemate hauad, on varemeis ja selle väravad on tulega põletatud?”
\par 4 Siis küsis kuningas minult: „Mida sa nüüd soovid?” Aga mina palusin taeva Jumalat
\par 5 ja ütlesin kuningale: „Kui kuningas heaks arvab ja su sulane on sulle meelepärane, siis läkita mind Juudamaale, mu vanemate haudade linna, et ma saaksin selle üles ehitada!”
\par 6 Siis kuningas, kelle kõrval istus kuninganna, küsis minult: „Kui kaua su teekond kestab ja millal sa tagasi tuled?” Kui ma nimetasin temale vastava aja, siis oli see kuningale meele järgi ja ta läkitas mind.
\par 7 Ja ma ütlesin kuningale: „Kui kuningas heaks arvab, siis antagu mulle kaasa kirjad maavalitsejaile teisel pool Frati jõge, et nad laseksid mu läbi, kuni ma jõuan Juudamaale,
\par 8 nõndasamuti kiri Aasafile, kuninga metsaülemale, et ta annaks mulle puid templipalee väravate ehitamiseks ja linna müüri ning koja jaoks, kuhu ma asun.” Ja kuningas andis mulle need, sest mu peal oli mu Jumala hea käsi.
\par 9 Siis ma tulin maavalitsejate juurde teisele poole Frati jõge ja andsin neile kuninga kirjad; ja kuningas oli koos minuga läkitanud väepealikuid ja ratsanikke.
\par 10 Aga kui Sanballat, hooronlane, ja ametnik Toobija, ammonlane, seda kuulsid, siis pahandas see neid väga, et keegi oli tulnud nõutama head Iisraeli lastele.
\par 11 Ja kui ma olin jõudnud Jeruusalemma ning olin seal olnud kolm päeva,
\par 12 siis ma tõusin öösel üles, mina ja mõned mehed koos minuga, ilma et ma kellelegi oleksin rääkinud, mis Jumal mu südamesse oli pannud Jeruusalemma heaks teha; ka ei olnud mul kaasas muud looma kui see, kellega ma ise ratsutasin.
\par 13 Ja ma läksin öösel Oruväravast välja Loheallika poole ja Sõnnikuvärava juurde, ja vaatlesin Jeruusalemma müüre, mis olid maha kistud, ja selle väravaid, mis olid tulega põletatud.
\par 14 Ja ma läksin edasi Allikavärava ja Kuningatiigi juurde; aga seal ei olnud eeslil, kelle seljas ma istusin, edasipääsuks maad.
\par 15 Siis ma läksin öösel orgu mööda ülespoole ja vaatlesin müüri; seejärel ma pöördusin ümber ja tulin tagasi tulles Oruväravast sisse.
\par 16 Aga ülemad ei teadnud, kuhu ma läksin ja mis ma tegin, sest ma ei olnud senini seda avaldanud ei juutidele ega preestritele, ei suurnikele ega ülemaile, ka mitte muile, kes seda tööd pidid tegema.
\par 17 Aga nüüd ma ütlesin neile: „Te näete, missuguses õnnetuses me oleme, kuidas Jeruusalemm on varemeis ja selle väravad on tulega põletatud. Mingem ja ehitagem üles Jeruusalemma müür, et meid enam ei teotataks!”
\par 18 Ja ma jutustasin neile, kuidas mu peal oli olnud mu Jumala hea käsi, ja ka neist sõnadest, mis kuningas mulle oli öelnud. Siis nad ütlesid: „Võtkem kätte ja ehitagem!” Ja nad kinnitasid oma käsi selleks heaks tööks.
\par 19 Aga kui hooronlane Sanballat ja ammonlane, ametnik Toobija, ja araablane Gesem seda kuulsid, siis nad pilkasid meid, osutasid meile põlgust ja küsisid: „Mis see on, mida te teete? Kas tahate kuningale vastu hakata?”
\par 20 Siis ma vastasin ja ütlesin neile: „Taeva Jumal annab meile kordamineku, ja meie, tema sulased, võtame kätte ja ehitame! Aga teil ei ole osa, õigust ega mälestust Jeruusalemmas!”

\chapter{3}

\par 1 Siis ülempreester Eljasib ja tema vennad preestrid võtsid kätte ja ehitasid Lambavärava; nad pühitsesid selle ja panid sellele väravad ette, ja pühitsesid kuni Sajatornini ja kuni Hananeli tornini.
\par 2 Ja nende kõrval ehitasid Jeeriko mehed; nende kõrval ehitas Sakkur, Imri poeg.
\par 3 Ja Kalavärava ehitasid Senaa pojad; nemad raiusid selle palkidest ja panid sellele ette väravad, lukud ja riivid.
\par 4 Ja nende kõrval parandas Meremot, Hakkosi poja Uurija poeg; nende kõrval parandas Mesullam, Mesesabeli poja Berekja poeg; nende kõrval parandas Saadek, Baana poeg.
\par 5 Ja nende kõrval parandasid tekoalased; aga nende suurnikud ei painutanud oma kaela oma isanda teenistusse.
\par 6 Ja Vanavärava parandasid Joojada, Paaseahi poeg, ja Mesullam, Besodja poeg; nemad raiusid selle palkidest ja panid sellele ette väravad, lukud ja riivid.
\par 7 Ja nende kõrval parandasid Melatja, gibeonlane, ja Jaadon, meeronotlane, Gibeoni ja Mispa mehed Frati-taguse maavalitseja piirkonnast.
\par 8 Nende kõrval parandas Ussiel, Harhaja poeg, kullassepp; tema kõrval parandas Hananja, salvisegaja; nemad taastasid Jeruusalemma müüri kuni laia müürini.
\par 9 Ja nende kõrval parandas Refaja, Huuri poeg, Jeruusalemma piirkonna ühe poole ülem.
\par 10 Ja tema kõrval parandas Jedaja, Harumafi poeg, oma koja kohal; tema kõrval parandas Hattus, Hasabneja poeg.
\par 11 Järgmist lõiku parandasid Malkija, Haarimi poeg, ja Hassub, Pahat-Moabi poeg, kuni Ahjutornini.
\par 12 Ja nende kõrval parandas Sallum, Halloohesi poeg, Jeruusalemma piirkonna teise poole ülem, tema ise ja ta tütred.
\par 13 Oruvärava parandasid Haanun ja Saanoahi elanikud; nemad ehitasid selle ja panid sellele ette väravad, lukud ja riivid; ja veel tuhat küünart müüri kuni Sõnnikuväravani.
\par 14 Ja Sõnnikuvärava parandas Malkija, Reekabi poeg, Beet-Keremi piirkonna ülem; ta ehitas selle ja pani sellele ette väravad, lukud ja riivid.
\par 15 Ja Allikavärava parandas Sallun, Kol-Hose poeg, Mispa piirkonna ülem; tema ehitas selle, tegi sellele katuse ja pani sellele ette väravad, lukud ja riivid; ja veel parandas ta Veejuhtmetiigi müüri kuninga rohuaia juures kuni astmeteni, mis Taaveti linnast alla viivad.
\par 16 Tema järel parandas Nehemja, Asbuki poeg, Beet-Suuri piirkonna ühe poole ülem, kuni Taaveti haudade kohani ja kaevatud tiigi ning sõjameeste hooneni.
\par 17 Tema järel parandasid leviidid: Rehum, Baani poeg; tema kõrval parandas Hasabja, Keila piirkonna ühe poole ülem, oma piirkonna eest.
\par 18 Tema järel parandasid nende vennad: Bavvai, Heenadadi poeg, Keila piirkonna teise poole ülem.
\par 19 Ja tema kõrval parandas Eeser, Jeesua poeg, Mispa ülem, kahekordset osa selle koha vastas, kust nurgal olevasse lattu üles minnakse.
\par 20 Tema järel parandas agarasti Baaruk, Sabbai poeg, järgmist lõiku nurgast alates kuni ülempreester Eljasibi koja ukseni.
\par 21 Tema järel parandas Meremot, Hakkosi poja Uurija poeg, järgmist lõiku Eljasibi koja uksest Eljasibi koja otsani.
\par 22 Ja tema järel parandasid preestrid, ümbruskonna mehed.
\par 23 Nende järel parandasid Benjamin ja Hassub oma koja kohal; nende järel parandas Ananja poja Maaseja poeg Asarja oma koja kõrval.
\par 24 Tema järel parandas Binnui, Heenadadi poeg, järgmist lõiku Asarja kojast kuni nurgani ja kuni käänakuni.
\par 25 Paalal, Uusai poeg, parandas Nurga- ja Ülatorni kohal, mis kuningakojast välja ulatub vahtkonnaõue juures; tema järel Pedaja, Parosi poeg, -
\par 26 ja Ofelil elavad templisulased - idapoolse Veevärava ja väljaulatuva torni kohal.
\par 27 Tema järel parandasid tekoalased järgmist lõiku suure väljaulatuva torni kohalt kuni templikünka müürini.
\par 28 Ülalpool Hobuväravat parandasid preestrid, igaüks oma koja kohal.
\par 29 Nende järel parandas Saadok, Immeri poeg, oma koja kohal; ja tema järel parandas Semaja, Sekanja poeg, Idavärava valvur.
\par 30 Tema järel parandasid Hananja, Selemja poeg, ja Haanun, Salafi kuues poeg, järgmist lõiku; nende järel parandas Mesullam, Berekja poeg, oma kambri kohal.
\par 31 Tema järel parandas Malkija, kullassepp, templisulaste ja kaupmeeste kojani, Ülevaatusevärava kohal ja kuni nurgarõduni.
\par 32 Ja nurgarõdu ning Lambavärava vahel parandasid kullassepad ja kaupmehed.

\chapter{4}

\par 1 Aga kui Sanballat ja Toobija, araablased, ammonlased ja asdodlased kuulsid, et Jeruusalemma müüride parandamine edenes, et mahakistud kohad hakkasid kinni saama, siis nad vihastusid väga.
\par 2 Ja nad kõik pidasid üheskoos vandenõu, et tulla Jeruusalemma vastu sõdima ja temale tüli tegema.
\par 3 Aga me palusime oma Jumalat ja panime nende vastu vahimehi päeval ja öösel, et nende eest kaitset leida.
\par 4 Juudas öeldi: „Kandja ramm raugeb, prügi on palju - meil pole jaksu ehitada müüri.”
\par 5 Meie vaenlased aga mõtlesid: „Enne kui nad märkavad ja näevad, tuleme meie nende sekka ja tapame nad ning lõpetame töö.”
\par 6 Kui juudid, kes elasid nende naabruses, tulid ja ütlesid meile vähemalt kümme korda, et kõigist paigust, kus nad elavad, tullakse meile kallale,
\par 7 siis ma panin mehi madalamatesse paikadesse müüri taha, lahtistele kohtadele, ja seadsin rahva üles suguvõsade kaupa mõõkade, piikide ja ambudega.
\par 8 Siis ma vaatasin ja võtsin kätte ning ütlesin suurnikele ja ülemaile ning muule rahvale: „Ärge kartke neid! Mõelge suurele ja kardetavale Issandale ja võidelge oma vendade, poegade ja tütarde, naiste ja kodade eest!”
\par 9 Ja kui meie vaenlased kuulsid, et see meile oli teatavaks saanud ja Jumal nende nõu oli tühistanud, siis me pöördusime kõik tagasi müüri juurde, igaüks oma tööle.
\par 10 Ja sellest päevast alates tegid tööd ainult pooled minu noortest meestest ja pooled kandsid piike, kilpe, ambusid ja soomusrüüsid; ja vürstid seisid kogu Juuda soo taga,
\par 11 kes müüri ehitas. Ja koormakandjad kandsid, tehes ühe käega tööd ja hoides teisega viskoda.
\par 12 Ja ehitajail oli ehitades igaühel oma mõõk vööle seotud; ja minu kõrval oli sarvepuhuja.
\par 13 Ja ma ütlesin suurnikele, ülemaile ja muule rahvale: „Töö on suur ja laialdane ja me oleme hajali piki müüri, üksteisest kaugel.
\par 14 Aga kust te kuulete sarvehäält, sinna paika kogunege meie juurde! Meie Jumal sõdib meie eest.”
\par 15 Nõnda me siis tegime tööd, aga pooled hoidsid piike, koiduajast kuni tähtede ilmumiseni.
\par 16 Sel ajal ma ütlesin ka rahvale, et igaüks jääks oma sulasega ööseks Jeruusalemma, et nad oleksid meil öösel valves ja päeval töös.
\par 17 Ja meie, ei mina ega mu vennad, ei mu sulased ega vahimehed, kes mulle järgnesid, ei võtnud endil riideid seljast; igaüks võttis end lahti ainult vette minnes.

\chapter{5}

\par 1 Aga rahva ja nende naiste kisa oli suur oma vendade juutide vastu.
\par 2 Oli neid, kes ütlesid: „Meid, meie poegi ja tütreid on palju. Antagu meile vilja, et saaksime süüa ja jääksime ellu!”
\par 3 Ja oli neid, kes ütlesid: „Me peame oma põllud, viinamäed ja kojad nälja pärast pandiks andma, et vilja saada.”
\par 4 Ja oli neid, kes ütlesid: „Me oleme pidanud kuninga maksuks raha laenama oma põldude ja viinamägede vastu.
\par 5 On ju küll meiegi ihu nagu meie vendade ihu, meiegi lapsed nagu nende lapsed, aga vaata, meie peame siiski oma pojad ja tütred orjadeks andma, ja mõned meie tütreist ongi juba orjastatud; kuid me oleme selle vastu võimetud, sest meie põllud ja viinamäed on teiste käes.”
\par 6 Kui ma nende kisa ja neid sõnu kuulsin, siis ma vihastusin väga.
\par 7 Ma pidasin oma südames nõu ja riidlesin siis suurnikega ja ülematega ning ütlesin neile: „Te võtate üksteiselt liigkasu!” Siis ma seadsin neile vastu suure koguduse
\par 8 ja ütlesin neile: „Meie oleme, nõnda palju kui suutsime, lunastanud oma vennad juudid, kes ennast paganaile olid müünud. Teie seevastu müüte oma vendi, et neid siis jälle meile peaks müüdama!” Siis nad vaikisid ega leidnud vastust.
\par 9 Ja ma ütlesin: „See pole hea, mis te teete! Kas te ei peaks mitte elama meie Jumala kartuses, paganate, meie vaenlaste teotuse vältimiseks?
\par 10 Ka mina, mu vennad ja mu sulased oleme neile laenanud raha ja vilja. Loobugem nüüd selle nõudmisest!
\par 11 Andke neile juba täna tagasi nende põllud, viinamäed, õlipuud ja kojad ja sajas osa rahast, viljast, veinist ja õlist, mida te neile olete laenanud!”
\par 12 Ja nad vastasid: „Me anname tagasi ega nõua neilt midagi; me teeme nõnda, nagu sa oled öelnud.” Siis, kutsudes preestrid, ma vannutasin neid, et nad teeksid nõnda.
\par 13 Minagi puistasin oma põue ja ütlesin: „Nõnda puistaku Jumal tema kojast ja tööst igaühte, kes seda sõna ei pea! Jäägu ta seesuguseks puistatuks ja tühjaks!„ Ja terve kogudus ütles: ”Aamen!” ning kiitis Issandat. Ja rahvas tegi selle sõna järgi.
\par 14 Ja edasi - alates sellest päevast, mil mind kästi olla Juudamaa maavalitsejaks, kuningas Artahsasta kahekümnendast valitsemisaastast kolmekümne teise valitsemisaastani, kaksteist aastat, ei ole mina ega mu vennad söönud maavalitseja leiba.
\par 15 Endised maavalitsejad, kes olid enne mind, olid rahvast raskesti koormanud ja neilt võtnud leiba ja veini, peale selle veel nelikümmend hõbeseeklit; ka oli nende sulastel meelevald rahva üle. Aga mina ei teinud nõnda Jumala kartuse pärast.
\par 16 Ka võtsin ma ise osa sellest müüritööst, kuigi me enestele põldu ei olnud ostnud; ja kõik mu sulased olid sinna tööle kogutud.
\par 17 Juudid ja ülemad, sada viiskümmend meest, ja need, kes ümberkaudseist paganaist meie juurde tulid, olid mu lauas.
\par 18 Seda, mis üheks päevaks valmistati, oli üks härg, kuus valitud lammast ja linnud - neidki valmistati mulle - ja iga kümne päeva järel oli rohkesti kõiksugust veini, aga ma siiski ei nõudnud maavalitseja leiba, sest selle rahva peal oli raske orjus.
\par 19 Meenuta minu heaks, mu Jumal, kõike, mis ma sellele rahvale olen teinud!

\chapter{6}

\par 1 Aga kui Sanballat ja Toobija ja araablane Gesem ja meie muud vaenlased said kuulda, et mina olin ehitanud müüri ja et sellesse ei olnud enam jäänud pragugi, kuigi ma selle ajani veel ei olnud väravaile uksi ette pannud,
\par 2 siis läkitasid Sanballat ja Gesem mulle ütlema: „Tule, saame üksteisega kokku mõnes külas Oono orus!” Aga nad mõtlesid teha mulle kurja.
\par 3 Ma läkitasin nende juurde käskjalad, et need ütleksid: „Mul on suur töö teha ja ma ei saa tulla. töö jääks seisma, kui ma selle jätan ja teie juurde tulen.”
\par 4 Nad läkitasid sel viisil mu juurde neli korda, kuid ma andsin neile sellesama vastuse.
\par 5 Siis läkitas Sanballat sel viisil viiendat korda oma sulase mu juurde, ja sel oli käes lahtine kiri,
\par 6 milles oli kirjutatud: „Rahvaste hulgas on saanud teatavaks, ja Gesem ütleb, et sina ja juudid kavatsete mässu. Seepärast sa ehitad müüri ja sa ise tahad saada neile kuningaks, nagu räägitakse.
\par 7 Sa oled seadnud isegi prohveteid, et nad Jeruusalemmas sinust kuulutaksid ja ütleksid: Juudal on kuningas! Nüüd saab kuningas sellest kuulda. Tule siis ja peame üheskoos nõu!”
\par 8 Aga ma läkitasin temale ütlema: „Ei ole sündinud midagi sellesarnast, millest sa räägid, vaid sa ise oled selle välja mõelnud.”
\par 9 Sest nad kõik tahtsid meid hirmutada, mõeldes: „Nende käed muutuvad töö jaoks lõdvaks ja see jääb tegemata.” Aga kinnita nüüd sina, Jumal, minu käsi!
\par 10 Ja kui ma läksin Mehetabeli poja Delaja poja Semaja kotta - tema ise oli takistatud -, siis ta ütles: „Kogunegem Jumala kotta, templi sisemusse, ja sulgegem templi uksed, sest nad tulevad sind tapma! Öösel tullakse sind tapma!”
\par 11 Aga mina vastasin: „Kas mees nagu mina peaks põgenema? Või kes minutaolistest võiks minna templisse ja jääda ellu? Mina ei lähe.”
\par 12 Sest vaata, ma sain aru, et Jumal ei olnud teda läkitanud, kuigi ta kõneles mu vastu prohvetisõna, vaid Toobija ja Sanballat olid teda palganud:
\par 13 ta oli palgatud selleks, et ma hakkaksin kartma ja teeksin nõnda ning patustaksin; see oleks toonud mulle halva kuulsuse, millega nad oleksid saanud mind teotada.
\par 14 Pea meeles, mu Jumal, Toobijat ja Sanballatit nende tegude pärast, samuti ka naisprohvet Nooadjat ja teisi prohveteid, kes tahtsid mind hirmutada!
\par 15 Nõnda sai müür valmis elulikuu kahekümne viiendal päeval viiekümne kahe päevaga.
\par 16 Ja kui kõik meie vaenlased seda kuulsid, siis kõik meie ümberkaudsed paganad kartsid ja langesid väga eneste silmis, sest nad mõistsid, et see töö oli tehtud meie Jumala abiga.
\par 17 Neil päevil saatsid Juuda suurnikud Toobijale palju kirju, ja Toobijalt tuli neile.
\par 18 Sest Juudas olid paljud tema vandeseltslased, kuna ta oli Sakanja, Aarahi poja väimees ja tema poeg Joohanan oli võtnud naiseks Berekja poja Mesullami tütre.
\par 19 Mullegi räägiti tema kohta kuuldusi ja minu sõnad viidi temale; aga Toobija läkitas kirju, et mind hirmutada.

\chapter{7}

\par 1 Kui müür oli üles ehitatud ja ma olin väravad ette pannud, siis seati väravahoidjad, lauljad ja leviidid kohtadele.
\par 2 Ja ma andsin käsutuse Jeruusalemma üle Hananile, oma vennale, ja Hananjale, paleeülemale, kes oli ustav mees ja kartis Jumalat enam kui paljud teised.
\par 3 Ja ma ütlesin neile: „Jeruusalemma väravaid ei tohi lahti teha enne, kui päike palavasti paistab. Ja kui veel seistakse, suletagu ja riivistatagu väravad! Ja Jeruusalemma elanikest pandagu vahte, igaüks oma vahipostile ja igaüks oma koja kohale!”
\par 4 Linn oli kõikepidi lai ja suur, aga rahvast oli selles vähe ja kojad ei olnud üles ehitatud.
\par 5 Siis pani mu Jumal mulle südamele, et ma koguksin ülikud, peamehed ja rahva suguvõsakirja kandmiseks. Ma leidsin nende suguvõsakirja, kes esimestena olid tulnud, ja leidsin selles olevat kirjutatud:
\par 6 Need on maa pojad, kes neist asumisele viidud vangidest teele läksid, keda Nebukadnetsar, Paabeli kuningas, oli asumisele viinud ja kes tagasi pöördusid Jeruusalemma ja Juudamaale igaüks oma linna,
\par 7 need, kes tulid koos Serubbaabeliga, Jeesuaga, Nehemjaga, Asarjaga, Raamjaga, Nahamaniga, Mordokaiga, Bilsaniga, Misperetiga, Bigvaiga, Nehumiga ja Baanaga; Iisraeli rahva meeste arv oli:
\par 8 Parosi poegi kaks tuhat ükssada seitsekümmend kaks;
\par 9 Sefatja poegi kolmsada seitsekümmend kaks;
\par 10 Aarahi poegi kuussada viiskümmend kaks;
\par 11 Pahat-Moabi poegi, Jeesua ja Joabi poegadest, kaks tuhat kaheksasada kaheksateist;
\par 12 Eelami poegi tuhat kakssada viiskümmend neli;
\par 13 Sattu poegi kaheksasada nelikümmend viis;
\par 14 Sakkai poegi seitsesada kuuskümmend;
\par 15 Binnui poegi kuussada nelikümmend kaheksa;
\par 16 Beebai poegi kuussada kakskümmend kaheksa;
\par 17 Asgadi poegi kaks tuhat kolmsada kakskümmend kaks;
\par 18 Adonikami poegi kuussada kuuskümmend seitse;
\par 19 Bigvai poegi kaks tuhat kuuskümmend seitse;
\par 20 Aadini poegi kuussada viiskümmend viis;
\par 21 Aateri poegi, Hiskija harust, üheksakümmend kaheksa;
\par 22 Haasumi poegi kolmsada kakskümmend kaheksa;
\par 23 Beesai poegi kolmsada kakskümmend neli;
\par 24 Haarifi poegi sada kaksteist;
\par 25 Gibeoni mehi üheksakümmend viis;
\par 26 Petlemma ja Netofa mehi sada kaheksakümmend kaheksa;
\par 27 Anatoti mehi sada kakskümmend kaheksa;
\par 28 Beet-Asmaveti mehi nelikümmend kaks;
\par 29 Kirjat-Jearimi, Kefiira ja Beeroti mehi seitsesada nelikümmend kolm;
\par 30 Raama ja Geba mehi kuussada kakskümmend üks;
\par 31 Mikmasi mehi sada kakskümmend kaks;
\par 32 Peeteli ja Ai mehi sada kakskümmend kolm;
\par 33 teise Nebo mehi viiskümmend kaks;
\par 34 teise Eelami poegi tuhat kakssada viiskümmend neli;
\par 35 Haarimi poegi kolmsada kakskümmend;
\par 36 Jeeriko poegi kolmsada nelikümmend viis;
\par 37 Loodi, Haadidi ja Oono mehi seitsesada kakskümmend üks;
\par 38 Senaa poegi kolm tuhat üheksasada kolmkümmend.
\par 39 Preestreid oli: Jedaja poegi, Jeesua soost, üheksasada seitsekümmend kolm;
\par 40 Immeri poegi tuhat viiskümmend kaks;
\par 41 Pashuri poegi tuhat kakssada nelikümmend seitse;
\par 42 Haarimi poegi tuhat seitseteist.
\par 43 Leviite oli: Jeesua poegi Kadmielist, Hoodavja poegadest seitsekümmend neli.
\par 44 Lauljaid oli: Aasafi poegi sada nelikümmend kaheksa.
\par 45 Väravahoidjaid oli: Sallumi poegi, Aateri poegi, Talmoni poegi, Akkubi poegi, Hatita poegi, Soobai poegi - sada kolmkümmend kaheksa.
\par 46 Templisulased olid: Siiha pojad, Hasuufa pojad, Tabbaoti pojad;
\par 47 Keerosi pojad, Siia pojad, Paadoni pojad;
\par 48 Lebana pojad, Hagaba pojad, Salmai pojad;
\par 49 Haanani pojad, Giddeli pojad, Gehari pojad;
\par 50 Reaja pojad, Resini pojad, Nekooda pojad;
\par 51 Gessami pojad, Ussa pojad, Paaseahi pojad;
\par 52 Beesai pojad, meunlaste pojad, nefuslaste pojad;
\par 53 Bakbuki pojad, Hakufa pojad, Harhuuri pojad;
\par 54 Basluti pojad, Mehiida pojad, Harsa pojad;
\par 55 Barkosi pojad, Siisera pojad, Taamahi pojad;
\par 56 Nesiahi pojad, Hatiifi pojad.
\par 57 Saalomoni orjade pojad olid: Sootai pojad, Soofereti pojad, Periida pojad;
\par 58 Jaala pojad, Darkoni pojad, Giddeli pojad;
\par 59 Sefatja pojad, Hattili pojad, Pokeret-Hassebaimi pojad, Aamoni pojad.
\par 60 Templisulaseid ja Saalomoni orjade poegi oli kokku kolmsada üheksakümmend kaks.
\par 61 Ja need olid teeleminejad Tel-Melahist, Tel-Harsast, Kerubist, Addonist ja Immerist, kes ei suutnud selgeks teha, kas nende vanemate kodu ja sugu pärines Iisraelist:
\par 62 Delaja pojad, Toobija pojad, Nekooda pojad - kuussada nelikümmend kaks.
\par 63 Ja preestreist olid: Habaja pojad, Hakkosi pojad, Barsillai pojad; Barsillai oli naise võtnud gileadlase Barsillai tütreist ja teda nimetati selle nimega, -
\par 64 need, kes otsisid oma suguvõsakirja, aga ei leidnud, ja nad vabastati kui kõlbmatud preestriametist.
\par 65 Ja maavalitseja keelas neid söömast kõige pühamat, enne kui uurimi ja tummimi jaoks on taas preester.
\par 66 Terve kogudus kokku oli nelikümmend kaks tuhat kolmsada kuuskümmend hinge;
\par 67 peale selle nende sulased ja teenijad, keda oli seitse tuhat kolmsada kolmkümmend seitse; ja neil oli kakssada nelikümmend viis mees- ja naislauljat;
\par 68 kaameleid oli nelisada kolmkümmend viis, eesleid kuus tuhat seitsesada kakskümmend.”
\par 69 Ja perekondade peameestest annetasid mõned töö tarvis. Maavalitseja andis alusvaraks tuhat kulddrahmi, viiskümmend piserdusnõu ja viissada kolmkümmend preestrikuube.
\par 70 Ja perekondade peameestest annetasid mõned alusvaraks kakskümmend tuhat kulddrahmi ja kaks tuhat kakssada hõbemiini.
\par 71 Ja mis muu rahvas andis, oli kakskümmend tuhat kulddrahmi, kaks tuhat hõbemiini ja kuuskümmend seitse preestrikuube.
\par 72 Ja preestrid, leviidid, väravahoidjad, lauljad ning muu rahvas ja templisulased, kogu Iisrael, asusid oma linnadesse.Ja kui seitsmes kuu kätte jõudis, olid Iisraeli lapsed oma linnades.

\chapter{8}

\par 1 Siis kogunes terve rahvas nagu üks mees Veevärava esisele väljakule, ja nad ütlesid kirjatundjale Esrale, et ta tooks Moosese Seaduse raamatu, mille Issand Iisraelile oli andnud.
\par 2 Ja preester Esra tõi Seaduse koguduse ette, niihästi meeste kui naiste ette, ja kõigi ette, kes mõistsid kuulata, seitsmenda kuu esimesel päeval.
\par 3 Ja ta luges seda Veevärava esisel väljakul koidust keskpäevani meeste ja naiste ning arusaajate ees; ja kogu rahva tähelepanu oli suunatud Seaduse raamatule.
\par 4 Ja Esra, kirjatundja, seisis puust laval, mis selleks oli ehitatud, ja tema kõrval seisid Mattitja, Sema, Anaja, Uurija, Hilkija ja Maaseja paremal pool, ja Pedaja, Miisael, Malkija, Haasum, Hasbaddana, Sakarja ja Mesullam vasakul pool.
\par 5 Ja Esra avas raamatu kogu rahva nähes, sest ta seisis kõrgemal kui kõik muu rahvas; ja kui ta selle avas, tõusis kogu rahvas püsti.
\par 6 Ja Esra kiitis Issandat, suurt Jumalat, ja kogu rahvas vastas käsi tõstes: „Aamen, aamen!” Ja nad põlvitasid ning heitsid Issanda ette silmili maha.
\par 7 Ja Jeesua, Baani, Seerebja, Jaamin, Akkub, Sabbetai, Hoodija, Maaseja, Keliita, Asarja, Joosabad, Haanan, Pelaja, leviidid, õpetasid rahvale Seadust ja rahvas seisis püsti.
\par 8 Ja nad lugesid raamatut, Jumala Seadust, peatükkide kaupa ja andsid seletust, nõnda et loetust aru saadi.
\par 9 Ja Nehemja, kes oli maavalitseja, ja preester Esra, kirjatundja, ja leviidid, kes rahvast õpetasid, ütlesid kogu rahvale: „See päev on pühitsetud Issandale, teie Jumalale. Ärge leinake ja ärge nutke!” Sest kogu rahvas nuttis, kui nad Seaduse sõnu kuulsid.
\par 10 Ja ta ütles neile: „Minge sööge rasvaseid roogi ja jooge magusaid jooke, ja läkitage osa neile, kellel midagi ei ole valmistatud! Sest see päev on pühitsetud meie Issandale. Ja ärge kurvastage, sest rõõm Issandast on teie ramm!”
\par 11 Ja leviidid vaigistasid rahvakogu, öeldes: „Rahunege, sest see päev on püha! Ja ärge kurvastage!”
\par 12 Siis läks kogu rahvas sööma ja jooma, läkitama teistelegi osa ja pidama suurt rõõmupidu, sest nad mõistsid sõnu, mis neile oli teatavaks tehtud.
\par 13 Ja teisel päeval tulid kokku kogu rahva perekondade peamehed, preestrid ja leviidid kirjatundja Esra juurde Seaduse sõnu tundma õppima.
\par 14 Ja nad leidsid Seaduses, mille Issand Moosese läbi oli andnud, kirjutatud olevat, et Iisraeli lapsed pidid pühade ajal seitsmendas kuus elama lehtmajades,
\par 15 ja et kõigis nende linnades ja Jeruusalemmas pidi kuulutatama ja hüütama, öeldes: „Minge mägedele ja tooge õlipuu või metsõlipuu oksi ja mürdi, palmi või muu leherikka puu oksi lehtmajade tegemiseks, nagu on kirjutatud!”
\par 16 Ja rahvas läks ja tõi ning tegi enesele lehtmaju, igaüks oma katusele ja õue ja Jumala koja õuedesse, Veevärava väljakule ja Efraimi värava väljakule.
\par 17 Ja terve kogudus, kes vangist oli tagasi tulnud, tegi lehtmaju ja elas lehtmajades, sest Joosua, Nuuni poja päevist selle päevani ei olnud Iisraeli lapsed nõnda teinud. Ja rõõm oli väga suur.
\par 18 Ja Jumala Seaduse raamatut loeti iga päev, esimesest päevast viimase päevani. Ja nad pidasid püha seitse päeva, ja kaheksandal päeval oli lõpetuspüha seatud viisi järgi.

\chapter{9}

\par 1 Ja sellesama kuu kahekümne neljandal päeval kogunesid Iisraeli lapsed paastuma, kotiriided seljas ja mulda pea peal.
\par 2 Ja Iisraeli sugu lahutas enese kõigist muulastest, astus ette ja tunnistas oma patte ning oma vanemate süütegusid.
\par 3 Siis nad tõusid oma kohal püsti ja lugesid Issanda, oma Jumala Seaduse raamatut kolm tundi; ja kolm tundi tunnistasid nad patte ning kummardasid Issandat, oma Jumalat.
\par 4 Ja Jeesua, Baani, Kadmiel, Sebanja, Bunni, Seerebja, Baani ja Kenani läksid üles leviitide lavale ja hüüdsid suure häälega Issanda, oma Jumala poole.
\par 5 Ja leviidid Jeesua, Kadmiel, Baani, Hasabneja, Seerebja, Hoodija, Sebanja ja Petahja ütlesid: „Tõuske üles ja kiitke Issandat, oma Jumalat, igavesest ajast igavesti, ja õnnistage tema aulist nime, mis on ülem kui kõik kiitus ja ülistus!
\par 6 Sina üksi oled Issand, sina oled teinud taeva, taevaste taevad ja kõik nende väe, maa ja kõik, mis selle peal on, mered ja kõik, mis neis on. Sina annad neile kõigile elu ja taevavägi kummardab sind.
\par 7 Sina, Issand, oled Jumal, kes valis Aabrami ja tõi tema kaldealaste Uurist ning pani temale nimeks Aabraham.
\par 8 Sina leidsid tema südame sinu ees ustava olevat ja sa tegid temaga lepingu, et sa annad tema soole kaananlaste, hettide, emorlaste, perislaste, jebuuslaste ja girgaaslaste maa. Ja sa pidasid oma sõna, sest sina oled õige.
\par 9 Sina nägid meie vanemate viletsust Egiptuses ja kuulsid nende kisendamist Kõrkjamere ääres.
\par 10 Sina tegid tunnustähti ja imetegusid vaarao ja kõigi tema sulaste ja kogu tema maa rahva kallal, sest sa teadsid, et nad nende vastu ülbed olid olnud, ja sa tegid enesele nime, nagu see tänapäevalgi on.
\par 11 Sina lõhestasid nende ees mere ja nad läksid kuiva mööda mere keskelt läbi; aga nende tagaajajad sa viskasid sügavustesse otsekui kivi võimsasse vette.
\par 12 Sina juhtisid neid pilvesambas päeval ja tulesambas öösel, et valgustada neile teed, mida nad pidid käima.
\par 13 Ja sa tulid alla Siinai mäele ja rääkisid nendega taevast ning andsid neile õiged seadlused, tõsised õpetused, head määrused ja käsud.
\par 14 Sina tegid neile teatavaks oma püha hingamispäeva ja sa andsid neile käsud, määrused ja Seaduse oma sulase Moosese läbi.
\par 15 Sina andsid neile taevast leiba, kui neil oli nälg, ja lasksid neile kaljust vett voolata, kui neil oli janu. Ja sa käskisid neid minna pärima maad, mille sa kätt tõstes olid lubanud neile anda.
\par 16 Aga nemad, meie vanemad, läksid ülbeks, olid kangekaelsed ega kuulnud su käske.
\par 17 Nad tõrkusid kuulmast ega tuletanud meelde sinu imetegusid, mis sa neile olid teinud, vaid olid kangekaelsed ja oma tõrksuses võtsid nad pähe oma orjapõlve tagasi minna. Aga sina oled andeksandja Jumal, armuline ja halastaja, pika meelega ja rikas heldusest; sina ei jätnud neid maha.
\par 18 Kuigi nad tegid endile valatud vasika ja ütlesid: „See on su jumal, kes sind Egiptusest välja tõi!” ja kuigi nad suuri teotusi korda saatsid,
\par 19 ei jätnud sina siiski neid kõrbes maha oma suure halastuse pärast. Pilvesammas ei lahkunud neist päeval, juhatamast neid teekonnal, ega tulesammas öösel, valgustamast neile teed, mida nad käisid.
\par 20 Sina andsid neile oma hea Vaimu, et neid targaks teha, oma mannat sa ei keelanud nende suudele ja andsid neile vett, kui neil oli janu.
\par 21 Nelikümmend aastat toitsid sa neid kõrbes, neil ei puudunud midagi, ei kulunud nende riided ega paistetanud jalad.
\par 22 Ja sa andsid neile kuningriigid ja rahvad ning jagasid need jaokaupa: nõnda pärisid nad Siihoni maa, Hesboni kuninga maa, ja Oogi, Baasani kuninga maa.
\par 23 Ja sa tegid nende lapsed paljuks nagu taevatähed ja viisid nad maale, mille kohta sa nende vanemaile olid öelnud, et nad lähevad seda pärima.
\par 24 Ja lapsed tulid ning pärisid maa, ja sina alandasid nende ees maa elanikud, kaananlased, ja andsid nad nende kätte, nende kuningad ja maa rahvad, et nad talitaksid nendega, nagu neile meeldib.
\par 25 Ja nad vallutasid kindlustatud linnad ja rammusa maa ning pärisid kojad, mis olid täis kõike head, suurel hulgal raiutud kaevusid, viinamägesid, õlipuid ja viljapuid. Nad sõid, nende kõht sai täis, nad läksid lihavaks ja elasid hästi sinu suure headuse tõttu.
\par 26 Ometi olid nad tõrksad ja hakkasid sulle vastu, heitsid sinu Seaduse selja taha ja tapsid sinu prohvetid, kes neid manitsesid sinu juurde tagasi pöörduma; ja nad saatsid korda suuri teotusi.
\par 27 Seepärast sa andsid nad nende vaenlaste kätte ja need rõhusid neid. Aga sel ajal, kui neil kitsas käes oli, kisendasid nad sinu poole ja sina kuulsid taevast ning andsid neile oma suure halastuse pärast päästjaid, kes päästsid nad nende vaenlaste käest.
\par 28 Aga kui nad said hingata, siis nad tegid jälle kurja sinu ees. Ja sina jätsid nad nende vaenlaste kätte, et need valitseksid nende üle. Aga kui nad jälle sind hüüdsid, siis sa kuulsid taevast ja päästsid nad oma halastuse pärast palju kordi.
\par 29 Ja sa manitsesid neid tagasi pöörduma sinu Seaduse juurde, aga nemad olid ülbed ega kuulnud su käske, vaid patustasid su seadluste vastu, mida inimene peab täitma, et ellu jääda, ja pöörasid tõrksalt selja ning olid kangekaelsed ega kuulnud mitte.
\par 30 Sa kannatasid nendega palju aastaid ja manitsesid neid oma Vaimu läbi oma prohvetite kaudu, aga nad ei pannud seda tähele; seepärast sa andsid nad teiste maade rahvaste kätte.
\par 31 Aga oma suure halastuse pärast ei teinud sa neile lõppu ega jätnud neid maha, sest sina oled armuline ja halastaja Jumal!
\par 32 Ja nüüd, meie Jumal, suur, vägev ja kartustäratav Jumal, kes peab lepingut ja osutab heldust: ära pea väheseks kõike seda kannatust, mis on tabanud meid, meie kuningaid, meie vürste, meie preestreid, meie prohveteid, meie vanemaid ja kogu su rahvast Assuri kuningate päevist kuni tänapäevani!
\par 33 Sina oled õiglane kõiges, mis meie peale on tulnud, sest sina oled osutanud ustavust, meie aga oleme olnud õelad.
\par 34 Jah, meie kuningad, meie vürstid, meie preestrid ja meie vanemad ei ole teinud su Seaduse järgi, ei ole tähele pannud su käske ja manitsusi, millega sa neid manitsesid.
\par 35 Nad ei teeninud sind oma kuningriigis, hoolimata sinu suurest headusest, mida sa neile osutasid, ja hoolimata avarast ja rammusast maast, mille sa neile olid andnud, ega pöördunud oma kurjadest tegudest.
\par 36 Vaata, me oleme nüüd sulased. Maal, mille sa andsid meie vanemaile, et nad sööksid selle vilja ja häid asju, vaata, seal me oleme sulased.
\par 37 Selle rikkalik saak läheb neile kuningaile, keda sa meie pattude pärast oled pannud meie üle; nemad valitsevad meie ihu ja meie karjade üle, nagu neile meeldib, ja meie oleme suures hädas.”

\chapter{10}

\par 1 Selle kõige tõttu me teeme ja kirjutame kindla lepingu, ja meie vürstid, meie leviidid ja meie preestrid kinnitavad seda pitseriga.
\par 2 Ja pitseriga kinnitajad on: maavalitseja Nehemja, Hakalja poeg, ja Sidkija,
\par 3 Seraja, Asarja, Jeremija,
\par 4 Pashur, Amarja, Malkija,
\par 5 Hattus, Sebanja, Malluk,
\par 6 Haarim, Meremot, Obadja,
\par 7 Taaniel, Ginneton, Baaruk,
\par 8 Mesullam, Abija, Mijamin,
\par 9 Maasja, Bilgai, Semaja - need on preestrid.
\par 10 Ja leviidid on: Jeesua, Asanja poeg, Binnui, Heenadadi poegadest, Kadmiel,
\par 11 ja nende vennad Sebanja, Hoodija, Keliita, Pelaja, Haanan,
\par 12 Miika, Rehob, Hasabja,
\par 13 Sakkur, Seerebja, Sebanja,
\par 14 Hoodija, Baani, Beniinu.
\par 15 Rahva peamehed on: Paros, Pahat-Moab, Eelam, Sattu, Baani,
\par 16 Bunni, Asgad, Beebai,
\par 17 Adonija, Bigvai, Aadin,
\par 18 Aater, Hiskija, Assur,
\par 19 Hoodija, Haasum, Beesai,
\par 20 Haarif, Anatot, Neebai,
\par 21 Magpias, Mesullam, Heesir,
\par 22 Mesesabel, Saadok, Jaddua,
\par 23 Pelatja, Haanan, Anaja,
\par 24 Hoosea, Hananja, Hassub,
\par 25 Halloohes, Pilha, Soobek,
\par 26 Rehum, Hasabna, Maaseja,
\par 27 Ahija, Haanan, Aanan,
\par 28 Malluk, Haarim, Baana.
\par 29 Ja ülejäänud rahvas, preestrid, leviidid, väravahoidjad, lauljad, templisulased ja kõik, kes olid end lahti öelnud teiste maade rahvaist Jumala Seaduse kasuks, nende naised, nende pojad ja tütred, kõik, kes olid võimelised aru saama,
\par 30 need liitusid oma vägevamate vendadega ja võtsid needuse ja vande kinnitusel endile kohustuse käia Jumala Seaduse järgi, mis Jumala sulase Moosese läbi oli antud, ja tähele panna ning täita kõiki Issanda, meie Jumala käske, tema seadlusi ja määrusi:
\par 31 „Me ei anna oma tütreid maa rahvaile ega võta nende tütreid oma poegadele.
\par 32 Ja kui maa rahvad toovad hingamispäeval müügiks kaupa ja kõiksugu vilja, siis me ei võta neilt seda hingamispäeval ega muul pühal päeval. Me loobume seitsmenda aasta saagist ja igasugusest võlanõudest.
\par 33 Me võtame endile kohustuse anda kolmandik seeklit aastas meie Jumala koja teenistuse tarvis:
\par 34 ohvrileibadeks, alaliseks roaohvriks ja alaliseks põletusohvriks, ohvreiks hingamispäevil, noorkuupäevil ja pühadel, pühadeks asjadeks, patuohvriks, et toimetada Iisraelile lepitust, ja igaks tegevuseks meie Jumala kojas.
\par 35 Meie, preestrid, leviidid ja rahvas, oleme heitnud liisku puude muretsemiseks, nende toomiseks meie perekondade kaupa oma Jumala kojale igal aastal kindlaksmääratud ajal, põletamiseks Issanda, meie Jumala altaril, nagu Seaduses on kirjutatud.
\par 36 Me kohustume tooma Jumala kotta igal aastal uudsevilja oma põldudelt ja uudsevilja kõigist puudest,
\par 37 ja esmasündinud oma poegadest ja kariloomadest, nagu Seaduses on kirjutatud; ja kohustume tooma oma veiste ning lammaste ja kitsede esmasündinuid oma Jumala kotta preestritele, kes teenivad meie Jumala kojas.
\par 38 Me toome ka parimat oma taignast ja oma tõstelõivudest, kõiksugu puude vilja, värsket veini ja õli preestritele meie Jumala koja kambrites; ja oma põldude kümnist leviitidele, neile leviitidele, kes koguvad kümnist kõigis meie põllutöölinnades.
\par 39 Ja üks preester, Aaroni poegadest, olgu koos leviitidega, kui leviidid koguvad kümnist, ja leviidid peavad viima kümnise kümnisest meie Jumala kotta varaaida kambritesse.

\chapter{11}

\par 1 Ja rahva vürstid asusid Jeruusalemma, aga muu rahvas heitis liisku, et üks kümnest läheks elama Jeruusalemma, pühasse linna, ja üheksa teistesse linnadesse.
\par 2 Ja rahvas õnnistas kõiki neid mehi, kes vabatahtlikult läksid elama Jeruusalemma.
\par 3 Ja need olid maa peamehed, kes elasid Jeruusalemmas; aga Juuda linnades elas igaüks oma pärandiosas, neile kuuluvais linnades: iisraellased, preestrid ja leviidid, templisulased ja Saalomoni orjade pojad.
\par 4 Jeruusalemmas elas Juuda ja Benjamini poegi. Juuda poegadest: Ataja, Ussija poeg, kes oli Sakarja poeg, kes oli Amarja poeg, kes oli Sefatja poeg, kes oli Mahalaleli poeg Peretsi poegadest,
\par 5 ja Maaseja, Baaruki poeg, kes oli Kol-Hose poeg, kes oli Hasaja poeg, kes oli Joojaribi poeg, kes oli Sakarja poeg, kes oli seelalase poeg.
\par 6 Kõiki Jeruusalemmas elavaid Peretsi poegi oli nelisada kuuskümmend kaheksa vahvat meest.
\par 7 Ja need olid Benjamini pojad: Sallu, Mesullami poeg, kes oli Joedi poeg, kes oli Pedaja poeg, kes oli Koolaja poeg, kes oli Maaseja poeg, kes oli Iitieli poeg, kes oli Jesaja poeg,
\par 8 ja tema järel Gabbai ja Sallai - üheksasada kakskümmend kaheksa.
\par 9 Ja Joel, Sikri poeg, oli nende ülevaataja, ja Juuda, Hassenua poeg, oli teine linnaülem.
\par 10 Preestritest: Jedaja, Joojaribi poeg, Jaakin,
\par 11 Seraja, Hilkija poeg, kes oli Mesullami poeg, kes oli Saadoki poeg, kes oli Merajoti poeg, kes oli Ahituubi poeg, Jumala koja eestseisja,
\par 12 ja nende vennad, kes toimetasid templiteenistust - kaheksasada kakskümmend kaks; ja Adaja, Jerohami poeg, kes oli Pelalja poeg, kes oli Amsi poeg, kes oli Sakarja poeg, kes oli Pashuri poeg, kes oli Malkija poeg,
\par 13 ja tema vennad, perekondade peamehed - kakssada nelikümmend kaks; ja Amassai, Asareli poeg, kes oli Ahsai poeg, kes oli Mesillemoti poeg, kes oli Immeri poeg,
\par 14 ja nende vennad, vahvad mehed - sada kakskümmend kaheksa; nende ülevaataja oli Sabdiel, Gedolimi poeg.
\par 15 Ja leviitidest: Semaja, Hassubi poeg, kes oli Asrikami poeg, kes oli Hasabja poeg, kes oli Bunni poeg;
\par 16 Sabtai ja Joosabad, leviitide peameestest, kes olid Jumala koja välistööde ülevaatajad;
\par 17 Mattanja, Miika poeg, kes oli Sabdi poeg, kes oli palvusel tänulaulu alustava laulujuhataja Aasafi poeg, ja Bakbukja, tema vendade hulgas teisel kohal olija, ja Abda, Sammua poeg, kes oli Gaalali poeg, kes oli Jedutuuni poeg.
\par 18 Kõiki leviite pühas linnas oli kakssada kaheksakümmend neli.
\par 19 Ja väravahoidjad olid: Akkub, Talmon ja nende vennad, kes valvasid väravaid - sada seitsekümmend kaks.
\par 20 Ja ülejäänud Iisrael, preestrid ja leviidid olid kõigis Juuda linnades, igaüks oma pärisosas.
\par 21 Ja templisulased elasid künkal; Siiha ja Gispa olid templisulaste ülemad.
\par 22 Ja leviitide ülevaataja Jeruusalemmas oli Ussi, Baani poeg, kes oli Hasabja poeg, kes oli Mattanja poeg, kes oli Miika poeg Aasafi poegadest, kes laulsid Jumala koja teenistusel.
\par 23 Sest nende kohta oli kuninga käsk ja kindel korraldus, kes nimelt iga päev pidid laulma.
\par 24 Ja Petahja, Mesesabeli poeg, Serahi, Juuda poja poegadest, oli kuninga volinik kõiges, mis rahvasse puutus.
\par 25 Ja nende põldude juures olevais asulais elas Juuda poegi Kirjat-Arbas ja selle tütarlinnades, Diibonis ja selle tütarlinnades, Jekabseelis ja selle külades,
\par 26 Jeesuas, Mooladas, Beet-Peletis,
\par 27 Hasar-Suualis, Beer-Sebas ja selle tütarlinnades,
\par 28 Siklagis, Mekonas ja selle tütarlinnades,
\par 29 Een-Rimmonis, Soras, Jarmutis,
\par 30 Saanoahis, Adullamis ja nende külades, Laakises ja selle väljadel, Asekas ja selle tütarlinnades; nad lõid leeri üles Beer-Sebast Hinnomi oruni.
\par 31 Ja Benjamini pojad asusid alates Gebast, Mikmasis, Ajjas, Peetelis ja selle tütarlinnades,
\par 32 Anatotis, Noobis, Ananjas,
\par 33 Haasoris, Raamas, Gittaimis,
\par 34 Haadidis, Seboimis, Neballatis,
\par 35 Loodis ja Oonos, Sepaorus.
\par 36 Ja Juuda leviitide rühmad olid ka Benjamini maal.

\chapter{12}

\par 1 Ja need olid preestrid ja leviidid, kes tulid tagasi koos Serubbaabeliga, Sealtieli pojaga, ja Jeesuaga: Seraja, Jeremija, Esra,
\par 2 Amarja, Malluk, Hattus,
\par 3 Sekanja, Rehum, Meremot,
\par 4 Iddo, Ginnetoi, Abija,
\par 5 Miijamin, Maadja, Bilga,
\par 6 Semaja, Joojarib, Jedaja,
\par 7 Sallu, Aamok, Hilkija, Jedaja; need olid preestrite ja nende vendade peamehed Jeesua päevil.
\par 8 Ja leviidid olid: Jeesua, Binnui, Kadmiel, Seerebja, Juuda ja Mattanja, kes koos oma vendadega juhatas tänulaulu,
\par 9 kuna Bakbukja ja Unni ning nende vennad teenistusel nendega vastamisi seisid.
\par 10 Ja Jeesuale sündis Joojakim, ja Joojakimile sündis Eljasib, ja Eljasibile sündis Joojada,
\par 11 ja Joojadale sündis Joonatan, ja Joonatanile sündis Jaddua.
\par 12 Ja Joojakimi päevil olid need preestrid perekondade peameesteks: Serajal Meraja, Jeremijal Hananja,
\par 13 Esral Mesullam, Amarjal Joohanan,
\par 14 Mallukil Joonatan, Sebanjal Joosep,
\par 15 Haarimil Adna, Merajotil Helkai,
\par 16 Iddol Sakarja, Ginnetonil Mesullam,
\par 17 Abijal Sikri, Minjaminil, Moadjal Piltai,
\par 18 Bilgal Sammua, Semajal Joonatan,
\par 19 Joojaribil Mattenai, Jedajal Ussi,
\par 20 Sallail Kallai, Aamokil Eeber,
\par 21 Hilkijal Hasabja, Jedajal Netaneel.
\par 22 Leviidid: Eljasibi, Joojada, Joohanani ja Jaddua päevil on perekondade peamehed kirja pandud, nõndasamuti preestrid, kuni pärslase Daarjavese valitsemisajani.
\par 23 Leevi pojad, perekondade peamehed, on ajaraamatusse kirja pandud Joohanani, Eljasibi poja päevini.
\par 24 Ja leviitide peamehed olid: Hasabja, Seerebja ja Jeesua, Kadmieli poeg, ja nende vennad, kes olid nendega vastamisi kiitmas ja tänamas jumalamehe Taaveti käsu kohaselt, rühm vastavalt rühmale.
\par 25 Mattanja, Bakbukja, Obadja, Mesullam, Talmon ja Akkub olid väravahoidjad, kes valvasid väravate varakambreid.
\par 26 Need elasid Joojakimi, Joosadaki poja Jeesua päevil ja maavalitseja Nehemja ja preestri ning kirjatundja Esra päevil.
\par 27 Ja kui Jeruusalemma müüri pühitseti, siis otsiti leviidid üles kõigist nende asupaikadest ja toodi Jeruusalemma, et pidada rõõmsat pühitsuspidu tänulaulude, simblite, naablite ja kanneldega.
\par 28 Ja lauljate pojad kogunesid Jeruusalemma ümberkaudsest piirkonnast ja netofalaste küladest,
\par 29 Beet-Gilgalist ja Geba ja Asmaveti väljadelt, sest lauljad olid endile ehitanud külasid ümber Jeruusalemma.
\par 30 Preestrid ja leviidid puhastasid endid, nõndasamuti puhastasid nad rahva ning väravad ja müüri.
\par 31 Siis ma käskisin Juuda vürste minna müüri peale, ja ma moodustasin kaks suurt laulukoori ja rongkäiku müüril paremale poole Sõnnikuvärava suunas,
\par 32 ja nende järel käisid Hoosaja ning pooled Juuda vürstidest,
\par 33 ja Asarja, Esra, Mesullam,
\par 34 Juuda, Benjamin, Semaja ja Jeremija,
\par 35 ja mõned preestrite pojad pasunatega: Sakarja, Joonatani poeg, kes oli Semaja poeg, kes oli Mattanja poeg, kes oli Miikaja poeg, kes oli Sakkuri poeg, kes oli Aasafi poeg,
\par 36 ja tema vennad Semaja, Asarel, Milalai, Gilalai, Maai, Netaneel, Juuda ja Hananai jumalamees Taaveti mänguriistadega; ja Esra, kirjatundja, käis nende ees.
\par 37 Ja nad läksid Allikaväravast mööda, tõusid otseteed Taaveti linna astmeid pidi üles, mööda müürile viivat treppi ülalpool Taaveti koda, kuni Veeväravani ida pool.
\par 38 Ja teine laulukoor, ja selle järel mina ja pool rahvast, läks müüril vastassuunas, Ahjutornist mööda kuni laia müürini,
\par 39 Efraimi väravast mööda ja üle Vanavärava, Kalavärava, Hananeli torni ja Sajatorni Lambaväravani; nad peatusid Vahtkonnaväravas.
\par 40 Siis asetusid mõlemad laulukoorid Jumala kotta, ka mina ja pooled ülemaist, kes koos minuga olid,
\par 41 ja preestrid Eljakim, Maaseja, Minjamin, Miikaja, Eljoenai, Sakarja ja Hananja pasunatega,
\par 42 ja Maaseja, Semaja, Eleasar, Ussi, Joohanan, Malkija, Eelam ja Eser. Lauljad laulsid ja Jisrahja oli juhatajaks.
\par 43 Ja nad ohverdasid sel päeval suuri ohvreid ning olid rõõmsad, sest Jumal oli neile suure rõõmu valmistanud; ka naised ja lapsed rõõmustasid, ja Jeruusalemma rõõm oli kaugele kuuldav.
\par 44 Sel päeval määrati mehed varakambrite ülemaiks; varakambrid olid tõstelõivude, uudsevilja ja kümniste jaoks, et nad neisse koguksid linnade põldudelt preestritele ja leviitidele kuuluvad seaduspärased osad, sest Juudal oli rõõm teenivaist preestreist ja leviitidest.
\par 45 Nemad toimetasid nende Jumala teenistust ja puhastusteenistust, nõndasamuti lauljad ja väravahoidjad, nagu Taavet ja tema poeg Saalomon olid käskinud.
\par 46 Sest juba muiste, Taaveti ja Aasafi päevil, oli lauljail juhataja, ja oli olemas kiitus- ja tänulaule Jumalale.
\par 47 Ja kogu Iisrael andis Serubbaabeli päevil ja Nehemja päevil lauljaile ja väravahoidjaile neile kuuluva igapäevase osa; pühitsetud annid anti leviitidele ja leviidid andsid pühitsetud annid Aaroni poegadele.

\chapter{13}

\par 1 Sel päeval loeti rahva kuuldes Moosese raamatut ja selles leiti olevat kirjutatud, et ammonlane ja moab ei tohtinud iialgi tulla Jumala kogudusse,
\par 2 sellepärast et nad ei olnud Iisraeli lastele vastu tulnud leiva ja veega, vaid palkasid Bileami neid needma; meie Jumal aga muutis needmise õnnistamiseks.
\par 3 Ja kui nad Seadust olid kuulnud, siis nad lahutasid kõik segarahva Iisraelist.
\par 4 Aga enne seda oli preester Eljasib, Toobija sugulane, kes oli pandud meie Jumala koja kambrite ülemaks,
\par 5 sisustanud temale suure kambri, kuhu varem oli pandud roaohvreid, viirukit, riistu ning vilja-, veini- ja õlikümnist, mis oli määratud leviitidele, lauljatele ja väravahoidjatele, ja preestrite tõstelõive.
\par 6 Ma ei olnud Jeruusalemmas, kui see kõik juhtus, sest Artahsasta, Paabeli kuninga kolmekümne teisel aastal olin ma läinud kuninga juurde. Aga mõne aja pärast palusin ma kuningalt luba
\par 7 ja tulin Jeruusalemma. Ja kui ma märkasin, missuguse pahateo Eljasib Toobija pärast oli teinud, sisustades temale kambri Jumala koja õues,
\par 8 siis ma sain väga pahaseks ja viskasin kõik Toobija koja asjad sellest kambrist välja.
\par 9 Ja ma käskisin kambrid puhtaks teha ning viisin sinna tagasi Jumala koja riistad, roaohvrid ja viiruki.
\par 10 Siis ma sain teada, et leviitidele ei olnud antud nende osa ja seepärast olid leviidid ja lauljad, kes pidid teenistust pidama, põgenenud igaüks oma põllule.
\par 11 Ja ma riidlesin ülematega ning ütlesin: „Mispärast on Jumala koda maha jäetud?” Ja ma kogusin nad kokku ning panin nad nende ametisse.
\par 12 Ja kogu Juuda tõi kümnist viljast, veinist ja õlist varaaitadesse.
\par 13 Ja ma määrasin varaaitade ülemaiks preester Selemja ja kirjatundja Saadoki, ja leviitide hulgast Pedaja, ja neile abiliseks Haanani, Mattanja poja Sakkuri poja; sest neid peeti ustavaiks ja nende kohus oli jagada oma vendadele.
\par 14 Mõtle, mu Jumal, minu peale selle pärast! Ära kustuta mu häid tegusid, mis ma olen teinud oma Jumala koja ja selle teenistuse heaks!
\par 15 Neil päevil nägin ma Juudamaal mõningaid, kes hingamispäeval surutõrsi sõtkusid ja kes viljakoormaid tõid ja eeslite selga panid, nõndasamuti ka veini, viinamarju, viigimarju ja igasuguseid koormaid hingamispäeval Jeruusalemma tõid; ja ma manitsesin neid, et nad sel päeval toiduaineid müüsid.
\par 16 Seal elas ka tüüroslasi, kes tõid kala ja igasugust kaupa ning müüsid hingamispäeval juutidele Jeruusalemmas.
\par 17 Siis ma riidlesin Juuda suurnikega ja ütlesin neile: „Mis kõlvatu tegu see on, mida te teete, et te hingamispäeva teotate?
\par 18 Eks teinud teie vanemad nõndasamuti ja eks lasknud meie Jumal tulla kogu selle õnnetuse meie ja selle linna peale? Ja hingamispäeva teotades te suurendate veelgi viha Iisraeli vastu.”
\par 19 Ja niipea kui Jeruusalemma väravais hingamispäeva eel oli läinud pimedaks, käskisin ma väravad sulgeda ja lubasin need avada alles pärast hingamispäeva; ja ma seadsin mõned oma sulastest väravate juurde, et hingamispäeval ükski koorem sisse ei tuleks.
\par 20 Siis jäid kaupmehed ja igasuguse kauba müüjad kord või kaks väljapoole Jeruusalemma.
\par 21 Ja ma manitsesin neid ning ütlesin neile: „Miks jääte ööseks müüri taha? Kui te seda veel kord teete, siis ma pistan oma käe teie külge!” Sellest ajast ei tulnud nad enam hingamispäeval.
\par 22 Ja ma käskisin leviite, et nad endid puhastaksid ja tuleksid väravaid valvama, et hingamispäeva pühitsetaks. Mõtle minu peale, mu Jumal, sellegi pärast, ja ole mulle armuline oma rohke helduse pärast!
\par 23 Neil päevil nägin ma ka juute, kes olid naiseks võtnud asdodlasi, ammonlasi ja moabe.
\par 24 Ja nende lastest rääkisid pooled asdodi keelt või mõne muu rahva keelt, aga juudi keelt nad ei osanud rääkida.
\par 25 Siis ma riidlesin nendega ja sajatasin neid, peksin neist mõningaid mehi ja katkusin neilt karvu; ja ma vannutasin neid Jumala nimel: „Te ei tohi anda oma tütreid nende poegadele ega võtta nende tütreid oma poegadele või iseendile!
\par 26 Eks ole Saalomon, Iisraeli kuningas, selle läbi pattu teinud? Paljude rahvaste hulgas ei olnud tema sarnast kuningat; ta oli oma Jumalale armas ja Jumal pani tema kuningaks kogu Iisraelile, aga tedagi saatsid võõramaa naised pattu tegema.
\par 27 Kas me siis peame kuulma teistki, et teete samasugust suurt pahategu, võttes võõramaa naisi ja murdes nõnda meie Jumalale truudust?”
\par 28 Ja üks ülempreester Eljasibi poja Joojada poegadest oli hooronlase Sanballati väimees, aga ma ajasin ta enese juurest ära.
\par 29 Mu Jumal, pea neid meeles, et nad on rüvetanud preestriametit ning preestrite ja leviitide lepingut!
\par 30 Nõnda ma puhastasin neid kõigest võõrast ja seadsin teenistuskorrad preestritele ja leviitidele, igaühele tema töö,
\par 31 ning puude ja uudsevilja toomiseks määratud ajad. Meenuta, mu Jumal, seda minu heaks!”



\end{document}