\begin{document}

\title{Pauluse Esimene kiri tessalooniklastele}

\chapter{1}

\section*{Tervitus ja tänu tessalooniklastele nende usu eest}

\par 1 Paulus ja Silvaanus ja Timoteos tessalooniklaste kogudusele Jumalas Isas ja Issandas Jeesuses Kristuses. Armu teile ja rahu!
\par 2 Me täname alati Jumalat teie kõikide eest ning tuletame teid meelde oma palvetes, lakkamatult
\par 3 meenutades teie usu tegu ja armastuse tööd ning kannatlikkust lootuses meie Issandale Jeesusele Kristusele Jumala ja meie Isa ees,
\par 4 teades, et teie, Jumala armastatud vennad, olete valitud.
\par 5 Sest meie evangeelium ei tulnud teie juurde mitte ainult sõnaga, vaid ka väega ning Püha Vaimuga ja väga kindla veendumusega, nagu te ju isegi teate, missugused me olime teie seas teie pärast.
\par 6 Ja teist said meie ja Issanda järgijad, kui te võtsite sõna vastu suures viletsuses Püha Vaimu rõõmuga,
\par 7 nii et te olete saanud eeskujuks kõigile usklikele Makedoonias ja Ahhaias.
\par 8 Sest teie juurest on Issanda sõna kostnud mitte ainult Makedooniasse ja Ahhaiasse, vaid ka igale poole on levinud teie usk Jumalasse, nõnda et meil pole tarvis sellest midagi rääkida.
\par 9 Ise nad ju jutustavad meist, kuidas me oleme tulnud teie juurde ja kuidas te olete pöördunud ebajumalaist Jumala poole teenima elavat ja tõelist Jumalat
\par 10 ning ootama taevast tema Poega, kelle tema on surnuist üles äratanud, Jeesust, kes meid tõmbab välja tulevasest vihast.


\chapter{2}

\section*{Apostel õigustab oma kuulutustööd}

\par 1 Teie ise ju teate, vennad, et meie tulek teie juurde ei ole olnud tühine;
\par 2 vaid, kuigi meie, nagu te teate, olime kannatanud Filipis ja saanud teotada, võtsime siiski oma Jumalas julguse kuulutada teile Jumala evangeeliumi suure võitlusega.
\par 3 Sest meie manitsus ei tule mitte eksimeelest, ei ebapuhtast mõttest ega kavalusest,
\par 4 vaid nõnda nagu Jumal on meid arvanud kõlbavaks usaldada evangeelium meie kätte, nõnda me ka räägime, ei mitte, nagu tahaksime olla inimestele, vaid Jumalale meelepärast, kes katsub läbi meie südamed.
\par 5 Sest me ei ole iialgi tarvitanud meelitavaid sõnu, nagu te teate, ega ka ettekäändeid ahnuseks: Jumal on tunnistaja.
\par 6 Me ei otsi ka austust inimestelt, ei teilt ega muilt,
\par 7 kuigi me Kristuse apostlitena oleksime võinud esineda tähtsatena; vaid me oleme saanud leebeiks teie seas nagu imetaja ema, kes hellitab oma lapsi;
\par 8 nõnda meie, teid ihaldades, olime valmis teile andma mitte ainult Jumala evangeeliumi, vaid ka oma hinged, sest te olite meile armsaks saanud.
\par 9 Te mäletate ju, vennad, meie tööd ja vaeva: ööd ja päevad me tegime tööd selleks, et mitte kedagi teie seast koormata, ja kuulutasime teie seas Jumala evangeeliumi.
\par 10 Teie ja Jumal olete tunnistajad, kui pühasti ja õigesti ja laitmatult me suhtusime teisse usklikesse;
\par 11 nagu te ka teate, kuidas me igaüht teie seast otsekui isa oma lapsi
\par 12 õhutasime ja julgustasime, teid manitsedes elama nõnda, kuidas kohus on Jumala ees, kes teid kutsub oma riiki ja ausse.

\section*{Koguduse tagakiusamisest}

\par 13 Ja seepärast meiegi täname lakkamatult Jumalat, et teie, kuuldes meilt Jumala sõna, ei võtnud seda vastu mitte inimeste sõnana, vaid sellena, mida see tõesti on, Jumala sõnana, mis ka on tegev teie sees, kes usute.
\par 14 Sest teie, vennad, olete saanud nende Kristuses Jeesuses olevate Jumala koguduste järgijaiks, mis on Judeas, sest ka teie olete saanud sedasama kannatada oma suguvendade poolt, nõnda nagu nemadki juutide poolt,
\par 15 kes tapsid Issanda Jeesuse ja prohvetid, ja kiusasid meid taga, ja ei ole Jumalale meelepärast, ja on kõigi inimeste vastased
\par 16 ning keelavad meid rääkimast paganaile nende päästmiseks, et täita kõigiti oma pattude mõõtu. Aga viha ongi juba tulnud nende peale lõpuni.

\section*{Apostli igatsus külastada kogudust}

\par 17 Meie aga, vennad, olles teist lahutatud natukeseks ajaks küll palge, mitte aga südame poolest, oleme seda suurema himuga püüdnud näha saada teie palet.
\par 18 Seepärast oleme tahtnud tulla teie juurde, mina, Paulus, küll üks kord ja teinegi kord, aga saatan on meid takistanud.
\par 19 Sest kes on meie lootus või rõõm või kiitlemise pärg - kas mitte teiegi - meie Issanda Jeesuse Kristuse ees tema tulemises?
\par 20 Sest teie olete meie au ja rõõm.


\chapter{3}

\section*{Timoteose ülesanne}

\par 1 Sellepärast meie, kui me enam ei läbenud kannatada, arvasime heaks jääda üksinda Ateenasse
\par 2 ja läkitasime Timoteose, oma venna ja Jumala kaastöölise Kristuse evangeeliumi kuulutamises, teid kinnitama ja manitsema teie usu pärast,
\par 3 et ükski ennast ei laseks vintsutada neis viletsusis. Sest te isegi teate, et meid on neisse pandud.
\par 4 Sest ka siis, kui olime teie juures, ütlesime teile ette, et meid vaevad ootavad, nõnda nagu ongi sündinud ja on teile teada.
\par 5 Sellepärast ka mina, kui ma enam ei läbenud, läkitasin tema kuulama teie usku, kas ehk vahest kiusaja ei ole teid kiusanud ja kas meie vaev ei ole saanud tühjaks.
\par 6 Aga kui Timoteos nüüd tuli meie juurde teie juurest ja tõi rõõmsaid sõnumeid teie usust ja armastusest ja et te peate meid alati heas mälestuses ning igatsete meid näha saada, nõnda nagu meiegi teid,
\par 7 siis oleme trööstitud teie suhtes, vennad, kõiges oma hädas ja kitsikuses teie usu läbi,
\par 8 sest et nüüd me elame, kui teie püsite kindlasti Issandas.
\par 9 Sest missugust tänu me võimegi anda Jumalale teie pärast kõige selle rõõmu eest, mis meil on teist oma Jumala ees,
\par 10 kuna me ööd ja päevad oleme kõige palavamas palves, et saaksime näha teie palet ja täita seda, mis teie usul veel on puudu.
\par 11 Aga meie Jumal ja Isa ise ja meie Issand Jeesus Kristus juhtigu meie tee teie juurde.
\par 12 Ent teid kasvatagu Issand ja tehku teid rikkaks armastuse poolest üksteise vastu ja kõikide vastu, nagu meiegi oleme teie vastu,
\par 13 et kinnitada teie südamed laitmatuiks pühitsuses Jumala ja meie Isa ees meie Issanda Jeesuse tulekul kõigi ta pühadega.


\chapter{4}

\section*{Manitsused laitmatuks eluks}

\par 1 Peale selle nüüd, vennad, me palume ja manitseme teid Issandas Jeesuses, et teie, nõnda nagu te meilt olete õppinud, kuidas elada ja olla Jumalale meeltmööda, ja nõnda nagu te juba elategi, kasvaksite veel täiuslikumaks.
\par 2 Sest te teate, mis käsud me teile oleme andnud Issanda Jeesuse läbi.
\par 3 Sest see on Jumala tahtmine, teie pühitsus, et te hoiduksite hooruse eest,
\par 4 et igaüks teie seast teaks hoida oma astjat pühaduses ja aus,
\par 5 mitte himude kiimas, nõnda nagu paganad, kes ei tunne Jumalat,
\par 6 et ükski teist ei teeks ülekohut ega petaks oma venda asjaajamises, sest et Issand on kättemaksja kõigi niisuguste asjade eest, nõnda nagu me olemegi teile enne öelnud ning tunnistanud.
\par 7 Sest Jumal ei ole meid mitte kutsunud rüvedusele, vaid pühitsusele.
\par 8 Seepärast, kes seda hülgab, ei hülga inimest, vaid Jumala, kes ka annab oma Püha Vaimu teie sisse.
\par 9 Aga vennalikust armastusest ei ole vaja teile kirjutada, sest teid endid on Jumal õpetanud armastama üksteist;
\par 10 sest te teete seda ka kõigile vendadele kogu Makedoonias. Me ainult manitseme teid, vennad, et te kasvaksite veel täiuslikumaks
\par 11 ja otsiksite au selles, et elate vaiksesti ja ajate oma asju ning teete tööd oma kätega, nõnda nagu me oleme teid käskinud,
\par 12 et te käitumises oleksite ausad nende vastu, kes on väljaspool, ega vajaks ühegi abi.

\section*{Surnute olukorrast}

\par 13 Aga me ei taha, et teil, vennad, oleks teadmata nende järg, kes on läinud magama, et teiegi ei läheks nõnda kurvaks nagu teised, kellel ei ole lootust.
\par 14 Sest kui me usume, et Jeesus on surnud ning üles tõusnud, nõnda ka Jumal toob esile Jeesuse kaudu need, kes ühes temaga on läinud magama.
\par 15 Sest seda me ütleme teile Issanda sõnaga, et meie, kes elame ja üle jääme Issanda tulemiseni, ei jõua mitte ette neist, kes on läinud magama;
\par 16 sest et Issand ise tuleb taevast alla sõjahüüuga, peaingli hääle ning Jumala pasunaga, ja Kristuses surnud tõusevad üles esmalt;
\par 17 selle järele kistakse meid, kes elame ja üle jääme, ühtlasi nendega pilvede peal Issandale vastu üles õhku, ja nõnda saame olla ikka ühes Issandaga.
\par 18 Kinnitage nüüd üksteist nende sõnadega!


\chapter{5}

\section*{Issanda tulemisest ja valvsuse vajadusest}

\par 1 Aga aegadest ja tundidest ei ole teile, vennad, tarvis kirjutada,
\par 2 sest te ise teate selgesti, et Issanda päev tuleb nagu varas öösel.
\par 3 Kui nad ütlevad: „Nüüd on rahu ja julgeolek!”, tabab neid äkiline hukatus nõnda nagu lapsevaev naise, kes on käima peal; ja nad ei pääse mitte pakku.
\par 4 Aga teie, vennad, ei ole mitte pimeduses, nii et see päev teid saaks haarata kui varas.
\par 5 Teie kõik olete ju valguse lapsed ja päeva lapsed. Meie ei ole mitte öö ega pimeduse lapsed.
\par 6 Siis ärgem magagem nagu teised, vaid valvakem ja olgem kained.
\par 7 Sest kes magavad, need magavad öösel, ja kes joobnud on, need on öösel joobnud.
\par 8 Aga meie, kes oleme päeva lapsed, olgem kained, varustatud usu ja armastuse raudrüüga ja päästelootuse kiivriga.
\par 9 Sest Jumal ei ole meid pannud viha alla, vaid pääste omandamisele meie Issanda Jeesuse Kristuse läbi,
\par 10 kes meie eest suri, et meie, kas valvame või magame, ühes temaga elaksime.
\par 11 Sellepärast manitsege üksteist ja kosutage üksteist, nagu te seda teetegi.

\section*{Mitmesugused manitsused ja lõppsõna}

\par 12 Aga me palume teid, vennad, tunnustada neid, kes teevad tööd teie seas ja on teie ülevaatajad Issandas ning juhatavad teid,
\par 13 ja pidada neid üpris väga armsaks nende töö pärast. Pidage rahu isekeskis!
\par 14 Me manitseme teid, vennad: noomige korratuid, julgustage argu, toetage nõtru, olge pika meelega kõikide vastu.
\par 15 Katsuge, et ükski teisele ei tasu kurja kurjaga, vaid püüdke ikka teha head üksteisele ja kõikidele.
\par 16 Olge ikka rõõmsad!
\par 17 Palvetage lakkamata!
\par 18 Olge tänulikud kõige eest; sest see on Jumala tahtmine teie suhtes Kristuses Jeesuses.
\par 19 Ärge kustutage Vaimu!
\par 20 Ärge pange halvaks prohvetlikult kõnelemist!
\par 21 Katsuge kõike läbi; pidage kinni, mis hea on!
\par 22 Hoiduge kõiksuguse kurja eest!
\par 23 Aga rahu Jumal ise pühitsegu teid täielikult ja kogu teie vaim ja hing ja ihu säiligu laitmatuina meie Issanda Jeesuse Kristuse tulekuks.
\par 24 Ustav on see, kes teid kutsub; küll ta teebki seda.
\par 25 Vennad, palvetage meie eest!
\par 26 Tervitage kõiki vendi püha suudlusega.
\par 27 Ma vannutan teid Issanda juures, et te selle kirja ette loeksite kõigile vendadele.
\par 28 Meie Issanda Jeesuse Kristuse arm olgu teiega!




\end{document}