\begin{document}

\title{Johannese evangeelium}

\chapter{1}

\section*{Sõna lihakssaamisest Kristuses}

\par 1 Alguses oli Sõna, ja Sõna oli Jumala juures, ja Sõna oli Jumal.
\par 2 Seesama oli alguses Jumala juures.
\par 3 Kõik on tekkinud tema läbi, ja ilma temata ei ole tekkinud midagi, mis on tekkinud.
\par 4 Temas oli elu, ja elu oli inimeste valgus,
\par 5 ja valgus paistab pimeduses, ja pimedus ei ole seda võtnud omaks.
\par 6 Oli mees, Jumala läkitatud; selle nimi oli Johannes.
\par 7 See tuli tunnistuseks, tunnistama valgusest, et kõik usuksid tema kaudu.
\par 8 Tema ei olnud valgus, vaid ta tuli tunnistama valgusest.
\par 9 See tõeline valgus, mis valgustab igat inimest, oli tulemas maailma.
\par 10 Tema oli maailmas, ja maailm on tekkinud tema läbi, ja maailm ei tundnud teda.
\par 11 Ta tuli sellesse, mis tema oma, ja tema omad ei võtnud teda vastu.
\par 12 Aga kõigile, kes teda vastu võtsid, andis ta meelevalla saada Jumala lasteks, kes usuvad tema nimesse,
\par 13 kes ei ole sündinud verest, ei liha tahtest ega mehe tahtest, vaid Jumalast.
\par 14 Ja Sõna sai lihaks ja elas meie keskel, ja me nägime tema au kui Isast ainusündinud Poja au, täis armu ja tõde.
\par 15 Johannes annab tunnistust temast, ja ta hüüdis nõnda: „Tema oli see, kellest ma ütlesin: kes tuleb pärast mind, see on olnud enne mind, sest ta oli enne kui mina!”
\par 16 Sest tema täiusest me kõik oleme saanud, ja armu armu peale.
\par 17 Sest käsuõpetus on antud Moosese kaudu, arm ja tõde on tulnud Jeesuse Kristuse kaudu.
\par 18 Ükski ei ole Jumalat iialgi näinud; ainusündinud Poeg, kes on Isa süles, on temast kõnelnud.

\section*{Ristija Johannese tunnistus}

\par 19 Ja see on Johannese tunnistus, kui juudid läkitasid Jeruusalemmast preestreid ja leviite temalt küsima: „Kes sa oled?”
\par 20 Tema tunnistas ega salanud mitte, vaid tunnistas: „Mina ei ole Kristus!”
\par 21 Ja nad küsisid temalt: „Kes siis? Oled sa Eelija?” Tema ütles: „Ei ole!” - „Oled sina see prohvet?” Tema vastas: „Ei!”
\par 22 Siis nad ütlesid talle: „Kes sa oled? Et annaksime vastuse neile, kes meid läkitasid. Mis sa ütled enesest?”
\par 23 Tema ütles: „Mina olen hüüdja hääl kõrbes: tehke tasaseks Issanda tee, nõnda nagu prohvet Jesaja on öelnud!”
\par 24 Ja need, kes olid läkitatud, olid variseride seast.
\par 25 Ja nad küsisid temalt ja ütlesid talle: „Miks sa siis ristid, kui sa pole Kristus ega Eelija ega see prohvet?
\par 26 Johannes vastas neile ning ütles: „Mina ristin veega; aga teie keskel seisab see, keda te ei tunne,
\par 27 kes tuleb pärast mind, kelle jalatsi paela lahti päästma mina ei ole vääriline!”
\par 28 See sündis Betaanias, sealpool Jordanit, kus Johannes ristis.
\par 29 Järgmisel päeval näeb Johannes Jeesust tulevat tema juurde ja ta ütleb: „Vaata, see on Jumala Tall, kes võtab ära maailma patu!
\par 30 See on, kellest ma ütlesin: pärast mind tuleb mees, kes on olnud enne mind, sest ta oli enne kui mina!
\par 31 Ja mina ei tundnud teda, kuid selleks, et ta saaks ilmsiks Iisraelile, olen mina tulnud veega ristima!”
\par 32 Ja Johannes tunnistas ning ütles: „Ma nägin Vaimu taevast alla laskuvat nagu tuvi; ja ta jäi tema peale.
\par 33 Ja mina ei tundnud teda mitte; aga kes mind läkitas veega ristima, see ütles mulle: kelle peale sa näed Vaimu alla laskuvat ja tema peale jäävat, see on, kes ristib Püha Vaimuga!
\par 34 Ja mina olen näinud ja tunnistanud, et seesinane on Jumala Poeg!”

\section*{Jeesuse esimesed jüngrid}

\par 35 Järgmisel päeval seisis Johannes jälle seal, ja kaks tema jüngrit.
\par 36 Ja kui ta nägi Jeesust kõndivat, ütles ta: „Ennäe, see on Jumala Tall!”
\par 37 Ja need kaks jüngrit kuulsid tema ütlust ja järgisid Jeesust.
\par 38 Aga Jeesus pöördus ümber, ja nähes neid teda järgivat, ütles ta neile: „Mida te otsite?” Aga nemad ütlesid talle: „Rabi!” - see tähendab meie keeli õpetaja - „Kus sa asud?”
\par 39 Ta ütles neile: „Tulge ja vaadake!” Nad tulidki ja nägid, kus ta asus, ja nad jäid tema juurde selle päeva. See oli arvata kümnes tund.
\par 40 Andreas, Siimon Peetruse vend, oli üks neist kahest, kes oli kuulnud Johannese ütlust ja hakanud Jeesust järgima.
\par 41 Tema leiab esiteks oma venna Siimona ja ütleb temale: „Me oleme leidnud Messia!” - see on meie keeli Kristuse?
\par 42 Ja ta viis tema Jeesuse juurde. Jeesus vaatas temale otsa ja ütles: „Sina oled Siimon, Joona poeg, sind peab hüütama Keefaseks!” - see tähendab Peetruseks.
\par 43 Järgmisel päeval Jeesus tahtis minna Galileasse; ja ta leiab Filippuse ning ütleb temale: „Järgi mind!”
\par 44 Aga Filippus oli pärit Betsaidast, Andrease ja Peetruse linnast.
\par 45 Filippus leiab Naatanaeli ja ütleb temale: „Kellest Mooses on kirjutanud käsuõpetuses, ja prohvetid, selle me oleme leidnud, Jeesuse, Joosepi poja, Naatsaretist!”
\par 46 Naatanael ütles talle: „Kas Naatsaretist võib tulla midagi head?” Filippus ütles temale: „Tule ja vaata!”
\par 47 Jeesus nägi Naatanaeli tema juurde tulevat ja ütles temast: „Ennäe, tõeline iisraellane, kelles ei ole kavalust!”
\par 48 Naatanael ütles talle: „Kust sa mind tunned?” Jeesus vastas ja ütles temale: „Enne kui Filippus sind kutsus, nägin ma sind, kui sa olid viigipuu all!”
\par 49 Naatanael vastas temale: „Rabi, sina oled Jumala Poeg, sina oled Iisraeli kuningas!”
\par 50 Jeesus vastas ja ütles talle: „Sellepärast et ma ütlesin: ma nägin sind viigipuu all! usud sa. Sa saad näha suuremaid asju kui need!”
\par 51 Ja ta ütles talle: „Tõesti, tõesti ma ütlen teile, te näete ka taeva olevat avatud ja Jumala inglid astuvat üles ja alla Inimese Poja peale!”


\chapter{2}

\section*{Kaana pulmad}

\par 1 Ja kolmandal päeval olid pulmad Kaanas Galileamaal, ja Jeesuse ema oli seal.
\par 2 Ja ka Jeesus ja tema jüngrid olid kutsutud pulma.
\par 3 Ja kui tuli puudus viinast, ütles Jeesuse ema temale: „Neil ei ole viina!”
\par 4 Jeesus ütles talle: „Mis sul minuga asja, naine? Minu aeg ei ole veel tulnud!”
\par 5 Tema ema ütles teenijaile: „Mis tema teile iganes ütleb, seda tehke!”
\par 6 Aga seal oli kuus kivist anumat seismas juutide puhastuskombe pärast ja igaühte mahtus kaks või kolm mõõtu.
\par 7 Jeesus ütles neile: „Täitke anumad veega!” Ja nemad täitsid need ääretasa.
\par 8 Ja ta ütles neile: „Ammutage nüüd neist ja viige pidukorraldajale!” Ja nemad viisid.
\par 9 Aga kui pidukorraldaja maitses vett, mis oli viinaks saanud, ega teadnud, kust see oli, kuna teenijad, kes vee olid toonud, teadsid, siis kutsus pidukorraldaja peigmehe
\par 10 ja ütles talle: „Igaüks paneb enne lauale hea viina, ja kui juba küllalt on joodud, lahjema. Sina oled hoidnud hea viina siitsaadik!”
\par 11 Selle esimese tunnustähe tegi Jeesus Kaanas Galileamaal ja ilmutas oma au. Ja tema jüngrid uskusid temasse.
\par 12 Pärast ta läks alla Kapernauma, tema ja ta ema ja ta vennad ja ta jüngrid; ja sinna nad ei jäänud kauaks ajaks.

\section*{Jeesus puhastab templi}

\par 13 Ja juutide paasapühad olid ligidal. Ja Jeesus läks üles Jeruusalemma.
\par 14 Seal ta leidis pühakojas neid, kes müüsid härgi ja lambaid ja tuvisid, ja rahavahetajaid istumas.
\par 15 Ja ta tegi paelust piitsa ja ajas kõik pühakojast välja, niihästi lambad kui härjad, ja rahavahetajate raha ta puistas laiali ning kummutas nende lauad
\par 16 ja ütles tuvimüüjaile: „Viige need siit ära! Ärge tehke minu Isa koda kaubakojaks!”
\par 17 Tema jüngritele meenus, et on kirjutatud: „Püha viha sinu koja pärast sööb mind ära!”
\par 18 Siis kostsid juudid ning ütlesid temale: „Mis tunnustähte sa näitad meile, et sa seda teed?”
\par 19 Jeesus vastas ning ütles neile: „Lammutage see tempel, ja ma püstitan selle kolme päevaga!”
\par 20 Siis ütlesid juudid: „Seda templit on ehitatud nelikümmend ja kuus aastat, ja sina püstitad selle kolme päevaga?”
\par 21 Ent tema rääkis oma ihu templist.
\par 22 Aga kui ta oli surnuist üles tõusnud, meenus tema jüngritele, et ta seda oli öelnud. Ja nad uskusid Kirja ja sõna, mis Jeesus oli öelnud.
\par 23 Aga kui ta oli Jeruusalemmas paasapühal, uskusid paljud tema nimesse, nähes tema imetegusid, mida ta tegi.
\par 24 Kuid Jeesus ise ei usaldanud ennast nende kätte, sest ta tundis neid kõiki
\par 25 ega olnud tal tarvis, et keegi oleks andnud tunnistust kellegi inimese kohta; sest ta ise teadis, mis oli inimeses.


\chapter{3}

\section*{Nikodeemuse kõnelus Jeesusega}

\par 1 Oli variseride seas mees, Nikodeemus nimi, üks juutide ülem.
\par 2 See tuli Jeesuse juurde öösel ja ütles temale: „Rabi, me teame, et sa Jumalalt oled tulnud õpetajaks; sest keegi ei või teha neid imetähti, mida sina teed, kui Jumal ei ole temaga!”
\par 3 Jeesus vastas ning ütles talle: „Tõesti, tõesti ma ütlen sulle, kui keegi ei sünni ülalt, ei või ta Jumala riiki näha!”
\par 4 Nikodeemus ütles temale: „Kuidas võib inimene sündida, kui ta on vana? Ega ta või teist korda minna oma ema ihusse ja sündida?”
\par 5 Jeesus vastas: „Tõesti, tõesti ma ütlen sulle kui keegi ei sünni veest ja Vaimust, ei või ta Jumala riiki pääseda!
\par 6 Mis lihast on sündinud, on liha, ja mis Vaimust on sündinud, on vaim!
\par 7 Ära pane imeks, et ma ütlesin sulle: te peate uuesti sündima!
\par 8 Tuul puhub, kus ta tahab, ja sa kuuled ta häält, aga sa ei tea, kust ta tuleb ja kuhu ta läheb; nõnda on igaüks, kes on sündinud Vaimust!”
\par 9 Nikodeemus vastas ning ütles talle: „Kuidas see võib sündida?”
\par 10 Jeesus vastas ning ütles talle: „Sina oled Iisraeli õpetaja, ja ei tea seda?
\par 11 Tõesti, tõesti ma ütlen sulle, me räägime, mida me teame, ja tunnistame, mis me oleme näinud, ja te ei võta meie tunnistust vastu!
\par 12 Kui ma teile räägin maisist asjust, ja te ei usu, kuidas te usuksite, kui ma teile räägiksin taevasist asjust?
\par 13 Ja ometi ei ole ükski muu läinud taevasse kui aga see, kes taevast on maha tulnud, Inimese Poeg!
\par 14 Ja nõnda nagu Mooses kõrbes mao ülendas, nõnda ülendatakse Inimese Poeg,
\par 15 et igaühel, kes usub temasse, oleks igavene elu!
\par 16 Sest nõnda on Jumal maailma armastanud, et ta oma ainusündinud Poja on andnud, et ükski, kes temasse usub, ei saaks hukka, vaid et temal oleks igavene elu!
\par 17 Sest Jumal ei ole oma Poega läkitanud maailma, et ta maailma üle kohut mõistaks, vaid et maailm tema läbi õndsaks saaks!
\par 18 Kes usub temasse, selle üle ei mõisteta kohut; aga kes ei usu, selle üle on juba kohus mõistetud, sest ta ei ole uskunud Jumala ainusündinud Poja nimesse.
\par 19 Aga see on kohus, et valgus on tulnud maailma ja inimesed armastasid pimedust rohkem kui valgust, sest nende teod olid kurjad.
\par 20 Sest igaüks, kes teeb kurja, vihkab valgust ega tule valguse juurde, et ta tegusid ei laidetaks.
\par 21 Aga kes teeb tõtt, see tuleb valguse juurde, et ta teod saaksid avalikuks, sest need on Jumalas tehtud!”

\section*{Ristija Johannese uus tunnistus Kristusest}

\par 22 Pärast seda tulid Jeesus ja tema jüngrid Juudamaale; ja seal ta viibis nendega ja ristis.
\par 23 Aga ka Johannes oli ristimas Ainonis Salimi lähedal, sest seal oli palju vett ja sinna tuli inimesi ja nad ristiti.
\par 24 Sest Johannes ei olnud veel vangitorni heidetud.
\par 25 Siis hakkasid Johannese jüngrid vaidlema ühe juudiga puhastuse pärast.
\par 26 Ja nad tulid Johannese juurde ja ütlesid temale: „Rabi! See, kes oli sinuga sealpool Jordanit, kellest sina tunnistasid, vaata, see ristib ja kõik lähevad tema juurde!”
\par 27 Johannes vastas ning ütles: „Inimene ei või midagi võtta, kui see temale ei ole antud taevast.
\par 28 Te ise olete minu tunnistajad, et ma ütlesin: mina ei ole Kristus, vaid mind on läkitatud tema eele!
\par 29 Kellel on pruut, see on peigmees; aga peigmehe sõber seisab ja kuulab teda ja on väga rõõmus peigmehe häälest. See mu rõõm on nüüd saanud täielikuks!
\par 30 Tema peab kasvama, aga mina pean kahanema!

\section*{See, kes tuleb taevast}

\par 31 Kes tuleb ülalt, see on üle kõikide! Kes on maast, see on maast ja räägib maast; kes tuleb taevast, on üle kõikide
\par 32 ja tunnistab seda, mida ta on näinud ja kuulnud, ja tema tunnistust ei võta ükski vastu!
\par 33 Kes võtab vastu tema tunnistuse, see kinnitab, et Jumal on tõeline!
\par 34 Sest see, kelle Jumal on läkitanud, räägib Jumala sõnu; sest Jumal ei anna Vaimu mõõdu järgi!
\par 35 Isa armastab Poega ja on kõik andnud tema kätte!
\par 36 Kes usub Pojasse, sellel on igavene elu; aga kes ei kuule Poja sõna, see ei saa elu näha, vaid Jumala viha jääb tema peale!”


\chapter{4}

\section*{Jeesus ja Samaaria naine}

\par 1 Kui nüüd Issand sai teada, et variserid olid kuulnud, et Jeesus rohkem jüngreid tegevat ja ristivat kui Johannes -
\par 2 ehk küll Jeesus ise ei ristinud, vaid tema jüngrid -
\par 3 siis ta lahkus Judeast ja läks Galileasse.
\par 4 Aga tal tuli Samaariast läbi minna.
\par 5 Siis ta tuli Samaaria linna, mida hüütakse Sühhariks, põllu lähedale, mille Jaakob oli andnud oma pojale Joosepile.
\par 6 Seal oli Jaakobi allikas. Väsinuna teekäimisest istus siis Jeesus maha allika äärde. Oli arvata kuues tund.
\par 7 Siis tuleb Samaariast naine vett ammutama. Jeesus ütleb talle: „Anna mulle juua!”
\par 8 Ent tema jüngrid olid läinud linna toidust ostma.
\par 9 Siis ütleb Samaaria naine temale: „Kuidas sina, olles juut, küsid juua minult, kes olen Samaaria naine?” Sest juudid ei lepi kokku samaarlastega.
\par 10 Jeesus vastas ning ütles talle: „Kui sa teaksid Jumala andi ja kes see on, kes sinule ütleb: anna mulle juua! siis sa paluksid teda ja ta annaks sinule elavat vett!”
\par 11 Naine ütleb temale: „Isand, sul ei ole ämbrit ja kaev on sügav; kust sa siis saad selle elava vee?
\par 12 Kas sina oled suurem kui meie isa Jaakob, kes meile andis selle kaevu ja jõi sealt ise, samuti ta pojad ja ta lojused?”
\par 13 Jeesus vastas ning ütles temale: „Igaüks, kes seda vett joob, see januneb jälle;
\par 14 aga kes iganes joob seda vett, mida mina temale annan, see ei janune igavesti mitte, vaid see vesi, mida mina temale annan, saab tema sees veeallikaks, mis voolab igavesse ellu!”
\par 15 Naine ütleb temale: „Isand, anna mulle seda vett, et ma ei januneks ega oleks mul vaja siia tulla vett ammutama!”
\par 16 Jeesus ütleb talle: „Mine kutsu oma mees ja tule siia!”
\par 17 Naine vastas ning ütles: „Minul ei ole meest!” Jeesus ütleb temale: „Sina ütled õigesti: mul ei ole meest!
\par 18 sest viis meest on sul olnud ja kes sul nüüd on, ei ole mitte sinu mees; seda sa oled õigesti öelnud.”
\par 19 Naine ütleb temale: „Isand, ma näen, et sa oled prohvet!
\par 20 Meie esiisad kummardasid Jumalat sellel mäel ja teie ütlete, et Jeruusalemmas olevat paik, kus tuleb kummardada!”
\par 21 Jeesus ütleb talle: „Naine, usu mind, et tund tuleb, mil te ei kummarda Isa ei sellel mäel ega Jeruusalemmas!
\par 22 Teie kummardate, mida te ei tea; meie kummardame, mida me teame, sest õndsus tuleb juutidelt.
\par 23 Aga tund tuleb ja on nüüd, et tõelised kummardajad kummardavad Isa vaimus ja tões; sest Isa otsib neid, kes teda nõnda kummardavad.
\par 24 Jumal on Vaim, ja kes teda kummardavad, need peavad vaimus ja tões teda kummardama!”
\par 25 Naine ütles temale: „Ma tean, et Messias tuleb, keda nimetatakse Kristuseks; kui see tuleb, siis ta kuulutab meile kõik!”
\par 26 Jeesus ütleb talle: „Mina, kes sinuga räägin, olen see!”
\par 27 Ja seepeale tulid tema jüngrid ja imestasid, et ta naisega kõneles. Siiski keegi ei öelnud: „Mis sa küsid või mis sa räägid temaga?”
\par 28 Siis naine jättis oma veekannu sinna ja läks linna ning ütles inimestele:
\par 29 „Tulge vaadake inimest, kes mulle on öelnud kõik, mis ma olen teinud! Kas see ei ole Kristus?”
\par 30 Nad väljusid linnast ja tulid tema juurde.
\par 31 Vaheajal palusid jüngrid teda ning ütlesid: „Rabi, söö!”
\par 32 Aga tema ütles neile: „Minul on süüa rooga, millest teie ei tea!”
\par 33 Siis ütlesid jüngrid isekeskis: „Kas ehk keegi on temale süüa toonud?”
\par 34 Jeesus ütleb neile: „Minu roog on see, et ma teen selle tahtmist, kes mind on läkitanud, ja lõpetan tema töö.
\par 35 Eks te ise ütle, et on veel neli kuud, ja siis tuleb lõikus? Vaata, ma ütlen teile, tõstke oma silmad üles ja vaadake põlde, et need on valged lõikuseks!
\par 36 Ja nüüd saab lõikaja palka ja kogub vilja igaveseks eluks, et niihästi külvaja kui lõikaja ühtlasi saaksid rõõmutseda.
\par 37 Sest siin on sõna tõsi: üks on, kes külvab, ja teine, kes lõikab!
\par 38 Mina olen teid läkitanud lõikama seda, mille kallal teie pole vaeva näinud; teised on vaeva näinud, ja teie olete tulnud nende vaevavilja lõikama!”
\par 39 Aga palju samaarlasi sellest linnast uskus temasse naise kõne tõttu, kes tunnistas: „Tema ütles mulle kõik, mis ma olen teinud!”
\par 40 Kui nüüd samaarlased tulid tema juurde, palusid nad teda jääda nende juurde. Ja ta viibis seal kaks päeva.
\par 41 Ja veel palju rohkem samaarlasi hakkas uskuma tema sõna tõttu.
\par 42 Nad ütlesid naisele: „Me ei usu mitte enam sinu kõne pärast; sest me ise oleme kuulnud ja teame, et tema on tõesti maailma Õnnistegija!”

\section*{Jeesus teeb terveks kuninga ametniku poja}

\par 43 Aga kahe päeva pärast tuli ta sealt ära Galileasse.
\par 44 Sest Jeesus ise tunnistas, et prohvetist ei peeta lugu ta omal isamaal.
\par 45 Kui ta nüüd oli tulnud Galileasse, võtsid galilealased tema vastu, sest nad olid näinud kõike seda, mis ta oli teinud Jeruusalemmas suurel pühal, sest ka nemad olid käinud neil pühil.
\par 46 Siis tuli Jeesus jälle Galilea Kaanasse, kus ta oli teinud vee viinaks. Ja seal oli üks kuninga ametnik, kelle poeg oli haige Kapernaumas.
\par 47 Kui see sai kuulda, et Jeesus on tulnud Judeast Galileasse, läks ta tema juurde ja palus teda, et ta tuleks ja teeks terveks tema poja, sest see oli suremas.
\par 48 Siis Jeesus ütles talle: „Kui te ei näe tunnustähti ega imetegusid, siis te ei usu!”
\par 49 Kuninga ametnik ütles temale: „Issand, tule alla, enne kui mu laps sureb!”
\par 50 Jeesus ütles talle: „Mine, su poeg elab!” Ja inimene uskus sõna, mis Jeesus talle ütles, ja läks.
\par 51 Aga kui ta alles oli minemas, tulid ta sulased talle vastu ja kuulutasid ning ütlesid, et ta poeg elab.
\par 52 Siis ta kuulas neilt tundi, mil ta oli hakanud toibuma. Nad ütlesid siis talle: „Eile seitsmendal tunnil lahkus temast palavik!”
\par 53 Siis isa märkas, et see oli sündinud samal tunnil, mil Jeesus talle ütles: „Su poeg elab!” Ja tema uskus, samuti kõik ta pere.
\par 54 Selle teise tunnustähe tegi Jeesus, kui ta Judeast oli tulnud Galileasse.


\chapter{5}

\section*{Haige Betsata tiigi ääres}

\par 1 Pärast seda oli juutide püha ja Jeesus läks Jeruusalemma.
\par 2 Aga Jeruusalemmas on Lambavärava ligi tiik; seda hüütakse heebrea keeli Betsataks. Sellel on viis võlvitud hoonet.
\par 3 Neis lamas hulk haigeid, pimedaid, jalutuid, kõhetuid, [kes ootasid vee liikumist.
\par 4 Sest aeg-ajalt tuli ingel alla tiiki ja segas vett. Kes siis esimesena pärast veesegamist sisse astus, sai terveks, olgu mis tahes haiguses ta oli.]
\par 5 Aga seal oli inimene, kes oli kolmkümmend kaheksa aastat haige olnud.
\par 6 Kui Jeesus teda nägi seal lamavat ja teada sai, et ta juba kaua aega oli maas olnud, ütles ta talle: „Kas sa tahad terveks saada?”
\par 7 Haige vastas temale: „Isand, mul ei ole seda inimest, kes mind aitaks tiiki, kui vett segatakse; seni, kui ma tulen, astub teine minu eel sisse!”
\par 8 Jeesus ütles temale: „Tõuse üles, võta oma voodi ja kõnni!”
\par 9 Ja inimene sai kohe terveks, võttis oma voodi ja kõndis! Aga see päev oli hingamispäev.
\par 10 Siis ütlesid juudid sellele, kes oli terveks saanud: „Nüüd on hingamispäev, sul ei sobi voodit kanda!”
\par 11 Tema vastas neile: „Kes mind terveks tegi, ütles mulle: võta oma voodi ja kõnni!”
\par 12 Nad küsisid temalt: „Kes on see inimene, kes sulle ütles: võta oma voodi ja kõnni?”
\par 13 Aga tervekssaanu ei teadnud, kes ta on; sest Jeesus oli läinud kõrvale, seal paigas oleva rahva hulka.
\par 14 Pärast seda leiab Jeesus tema pühakojast ja ütleb talle: „Vaata, sa oled terveks saanud, ära tee enam pattu, et sulle ei juhtuks midagi halvemat!”
\par 15 Inimene läks ära ja ütles juutidele, et see oli Jeesus, kes ta terveks tegi.
\par 16 Ja sellepärast juudid kiusasid Jeesust taga, et ta seda oli teinud hingamispäeval.
\par 17 Aga Jeesus kostis neile: „Minu Isa tegutseb tänini ja mina tegutsen!”
\par 18 Sellepärast püüdsid nüüd juudid veel enam teda tappa, et ta mitte ainult ei olnud pannud mikski hingamispäeva, vaid oli ka öelnud Jumala oma Isa olevat, tehes ennast Jumala sarnaseks.

\section*{Jeesus seletab oma sõltuvust Isast}

\par 19 Siis Jeesus vastas ning ütles neile: „Tõesti, tõesti ma ütlen teile, Poeg ei või iseenesest teha midagi kui vaid seda, mida ta näeb Isa tegevat! Sest mida Isa teeb, seda teeb Poeg nõndasamuti.
\par 20 Sest Isa armastab Poega ja näitab talle kõik, mida ta ise teeb, ja tahab temale näidata suuremaid tegusid kui need on, nõnda et te seda panete imeks.
\par 21 Sest otsekui Isa äratab üles surnuid ja teeb elavaks, nõnda teeb ka Poeg elavaks, keda tahab.
\par 22 Sest Isa ei mõista kohut kellegi üle, vaid kõik kohtu on ta andnud Poja kätte,
\par 23 et kõik austaksid Poega, nagu nad austavad Isa. Kes ei austa Poega, see ei austa Isa, kes tema on läkitanud.
\par 24 Tõesti, tõesti ma ütlen teile, kes minu sõna kuuleb ja usub seda, kes mind on läkitanud, sellel on igavene elu ja see ei tule mitte kohtu alla, vaid on surmast läinud ellu!
\par 25 Tõesti, tõesti ma ütlen teile, tund tuleb ja on juba käes, mil surnud peavad kuulma Jumala Poja häält, ja kes seda kuulevad, peavad elama!
\par 26 Sest otsekui Isal on elu iseeneses, nõnda on ta andnud ka Pojale, et elu on temas eneses.
\par 27 Ja ta on temale andnud meelevalla ka kohut pidada, sellepärast et ta on Inimese Poeg.
\par 28 Ärge pange seda imeks; sest tuleb tund, mil kõik, kes on haudades, kuulevad tema häält
\par 29 ning tulevad välja need, kes on teinud head, elu ülestõusmiseks, aga kes on teinud halba, hukkamõistmise ülestõusmiseks.

\section*{Tunnistus Jeesusest}

\par 30 Mina ei või iseenesest ühtki teha. Nagu ma kuulen, nõnda ma mõistan kohut ja minu otsus on õige, sest mina ei nõua oma tahtmist, vaid selle tahtmist, kes mind on läkitanud.
\par 31 Kui mina tunnistan iseenesest, siis ei ole mu tunnistus mitte tõsi.
\par 32 On teine, kes minust tunnistab, ja ma tean, et see tunnistus, mida ta minust tunnistab, on tõsi.
\par 33 Teie läkitasite Johannese juurde ja tema andis tõele tunnistust.
\par 34 Aga mina ei võta tunnistust inimeselt, vaid räägin seda, et teie saaksite õndsaks.
\par 35 Tema oli küünal, mis põles ja paistis; aga teie tahtsite natuke aega ilutseda tema valguses.
\par 36 Aga minul on suurem tunnistus kui Johannese oma; sest teod, mis Isa minule on andnud, et ma need lõpetaksin, needsamad teod, mida ma teen, tunnistavad minust, et Isa mind on läkitanud.
\par 37 Ja Isa, kes mind on läkitanud, see on tunnistanud minust. Teie ei ole iialgi kuulnud tema häält ega ole näinud tema nägu,
\par 38 ega ole teil tema sõna püsivana teie sees; sest te ei usu teda, kelle tema on läkitanud.
\par 39 Te uurite pühi kirju, sest te arvate enestel neis olevat igavese elu, ja need on, mis tunnistavad minust;
\par 40 ja te ei taha tulla minu juurde, et te saaksite elu.
\par 41 Mina ei võta vastu austust inimestelt;
\par 42 aga mina tunnen teid, et teil ei ole Jumala armastust iseenestes.
\par 43 Mina olen tulnud oma Isa nimel, ja te ei võta mind vastu; kui teine tuleb iseenese nimel, tema te võtate vastu.
\par 44 Kuidas te võite uskuda, kui te võtate austust vastu üksteiselt ega otsi seda austust, mis tuleb üksnes Jumalalt?
\par 45 Ärge mõelge, et mina kaebaksin teie peale Isa ees; on olemas, kes teie peale kaebab - Mooses, kelle peale te loodate.
\par 46 Sest kui te usuksite Moosest, usuksite te ka mind; sest tema on kirjutanud minust.
\par 47 Aga kui te tema kirju ei usu, kuidas te siis usuksite minu sõnu?”


\chapter{6}

\section*{Viie tuhande mehe söötmine}

\par 1 Pärast seda läks Jeesus Tibeeria poolt Galilea mere teisele poole.
\par 2 Ja palju rahvast järgis teda, sest nad nägid tema imetähti, mida ta tegi haigete juures.
\par 3 Aga Jeesus läks üles mäele ja istus sinna oma jüngritega.
\par 4 Ja paasa, juutide püha, oli ligi.
\par 5 Kui nüüd Jeesus oma silmad üles tõstis ja nägi palju rahvast enese juurde tulevat, ütles ta Filippusele: „Kust me ostame leiba, et need saaksid süüa?”
\par 6 Ent seda ta ütles teda kiusates, sest ta teadis küll, mida ta tahtis teha.
\par 7 Filippus vastas talle: „Leibu kahesaja teenari eest ei jätku neile, et igaüks natukesegi võtaks!”
\par 8 Üks tema jüngritest, Andreas, Siimon Peetruse vend, ütleb talle:
\par 9 „Siin on üks poisike, kellel on viis odraleiba ja kaks kalakest; aga mis saab sellest nii paljudele?”
\par 10 Jeesus ütles: „Asetage inimesed maha istuma!” Ja seal paigas oli palju rohtu. Siis asetusid maha arvult umbes viis tuhat meest.
\par 11 Jeesus võttis nüüd leivad, tänas ja andis neile, kes maas istusid, samuti ka neist kalukesist, niipalju kui nad tahtsid.
\par 12 Aga kui nende kõhud olid täis saanud, ütles ta oma jüngritele: „Koguge ülejäänud palukesed kokku, et midagi ei läheks raisku!”
\par 13 Siis nad kogusid kokku kaksteist korvitäit palukesi, mis oli üle jäänud viiest odraleivast nendelt, kes olid söönud.
\par 14 Kui nüüd rahvas nägi tunnustähte, mille Jeesus oli teinud, ütlesid nad: „See on tõesti see prohvet, kes maailma pidi tulema!”
\par 15 Siis Jeesus sai aru, et nad tahavad tulla ja teda vägisi võtta ning teha kuningaks. Ja ta läks jälle kõrvale mäele ainuüksi.

\section*{Jeesus kõnnib vee peal}

\par 16 Aga kui õhtu tuli, läksid tema jüngrid mereranda
\par 17 ja astusid paati ja tulid teisele poole merd Kapemauma. Aga pimedus oli juba käes ja Jeesus ei olnud veel tulnud nende juurde.
\par 18 Ja meri hakkas lainetama, sest suur tuul puhus.
\par 19 Kui nad nüüd olid sõudnud umbes kakskümmend viis või kolmkümmend vagu maad, nägid nad Jeesust merel kõndivat ja paadi lähedale jõudvat; ja nad lõid kartma.
\par 20 Aga tema ütles neile: „Mina olen see, ärge kartke!”
\par 21 Nüüd nad tahtsid teda paati võtta, ja varsti sai paat sinna randa, kuhu nad sõitsid.

\section*{Rahvas otsib Jeesust}

\par 22 Järgmisel päeval nägi rahvas, kes viibis mere teisel rannal, et teist lootsikut ei olnud seal, vaid oli ainult üks, ja et Jeesus ei olnud astunud ühes oma jüngritega paati, vaid et ainult tema jüngrid olid ära läinud.
\par 23 Oli aga tulnud veel lootsikuid Tibeeriast selle paiga lähedale, kus nad olid leiba söönud, kui Issand oli Jumalat tänanud.
\par 24 Kui nüüd rahvas nägi, et seal ei olnud Jeesust ega tema jüngreid, läksid nad lootsikusse ja tulid Kapernauma ning otsisid Jeesust.
\par 25 Ja kui nad ta leidsid mere teiselt rannalt, ütlesid nad talle: „Rabi, millal sa siia said?”

\section*{Eluleib}

\par 26 Jeesus vastas neile ning ütles: „Tõesti, tõesti ma ütlen teile, et te ei otsi mind, sellepärast et te nägite tunnustähti, vaid et te sõite leiba ja teie kõhud said täis!
\par 27 Ärge hankige rooga, mis hävib, vaid rooga, mis jääb igaveseks eluks, mida Inimese Poeg tahab teile anda; sest teda on Jumal Isa pitseriga kinnitanud!”
\par 28 Siis ütlesid nad temale: „Mis me peame tegema, et võiksime teha Jumala tegusid?”
\par 29 Jeesus vastas ning ütles neile: „See on Jumala tegu, et te usute temasse, kelle ta on läkitanud!”
\par 30 Siis nad ütlesid temale: „Mis tunnustähe sa siis teed, et me näeksime ja usuksime sind? Mis sa teed?
\par 31 Meie esiisad sõid kõrbes mannat, nagu on kirjutatud: ta andis neile süüa leiba taevast!”
\par 32 Siis Jeesus ütles neile: „Tõesti tõesti ma ütlen teile! Mooses ei andnud teile leiba taevast, aga minu Isa annab teile tõelise leiva taevast!
\par 33 Sest Jumala leib on see, kes tuleb taevast alla ja annab maailmale elu!”
\par 34 Siis nad ütlesid talle: „Issand, anna meile ikka seda leiba!”
\par 35 Jeesus ütles neile: „Mina olen eluleib. Kes tuleb minu juurde, see ei näe nälga, ja kes minusse usub, sellele ei tule iialgi janu!
\par 36 Aga mina olen teile öelnud, et te olete mind näinud ega usu mitte!
\par 37 Kõik, mis Isa minule annab, tuleb minu juurde, ja kes minu juurde tuleb, seda ma ei lükka välja.
\par 38 Sest ma olen taevast alla tulnud mitte oma tahtmist tegema, vaid selle tahtmist, kes mind on läkitanud.
\par 39 Aga see on selle tahtmine, kes mind on läkitanud, et ma ühtki ei kaota sellest kõigest, mis ta minule on andnud, vaid et ma selle üles äratan viimsel päeval.
\par 40 Sest see on mu Isa tahtmine, et igaühel, kes Poega näeb ja temasse usub, on igavene elu, ja ma äratan tema üles viimsel päeval!”
\par 41 Siis nurisesid juudid tema üle, et ta oli öelnud: „Mina olen leib, mis taevast on alla tulnud!”
\par 42 Ja nad ütlesid: „Eks see ole Jeesus, Joosepi poeg, kelle isa ja ema me tunneme? Kuidas ta siis ütleb: ma olen tulnud taevast?”
\par 43 Jeesus vastas ning ütles neile: „Ärge nurisege isekeskis!
\par 44 Ükski ei või tulla minu juurde, kui teda ei tõmba Isa, kes mind on läkitanud; ja mina äratan tema üles viimsel päeval.
\par 45 Prohvetite raamatuis on kirjutatud: nad kõik peavad olema Jumala poolt õpetatud! Igaüks, kes on Isalt kuulnud ja on õppinud, tuleb minu juurde.
\par 46 Mitte, et keegi teine oleks näinud Isa kui see, kes on Jumala juurest; see on näinud Isa.
\par 47 Tõesti, tõesti ma ütlen teile: kes usub, sellel on igavene elu!
\par 48 Mina olen eluleib.
\par 49 Teie esiisad sõid kõrbes mannat ja surid.
\par 50 See on leib, mis taevast alla tuleb, et inimene seda sööks ega sureks.
\par 51 Mina olen elav leib, mis taevast on alla tulnud. Kui keegi sööb seda leiba, siis ta elab igavesti; ja leib, mille mina annan, on minu liha, mille annan maailma elu eest!”
\par 52 Siis riidlesid juudid isekeskis, öeldes: „Kuidas see võib meile anda oma liha süüa?”
\par 53 Siis Jeesus ütles neile: „Tõesti, tõesti ma ütlen teile, et kui te ei söö Inimese Poja liha ega joo tema verd, siis ei ole elu teis enestes!
\par 54 Kes sööb minu liha ja joob minu verd, sel on igavene elu; ja mina äratan tema üles viimsel päeval.
\par 55 Sest mu liha on tõeline roog ja mu veri on tõeline jook.
\par 56 Kes sööb minu liha ja joob minu verd, jääb minusse ja mina temasse.
\par 57 Nagu elav Isa mind on läkitanud ja mina elan Isa läbi, nõnda ka see, kes mind sööb, elab minu läbi.
\par 58 See on leib, mis taevast on alla tulnud; see ei ole niisugune, mida sõid teie vanemad ja surid. Kes seda leiba sööb, elab igavesti!”
\par 59 Seda ta ütles kogudusekojas, õpetades Kapernaumas.

\section*{Kahtlevad jüngrid}

\par 60 Siis paljud tema jüngrid, kui nad seda kuulsid, ütlesid: „See kõne on kõva, kes võib seda kuulda?”
\par 61 Aga et Jeesus iseenesest teadis, et tema jüngrid selle üle nurisesid, ütles ta neile: „Kas see teid pahandab?
\par 62 Aga mis on siis, kui te näete Inimese Poja üles minevat sinna, kus ta enne oli?
\par 63 Vaim on, kes teeb elavaks; lihast ei ole kasu millekski; sõnad, mis mina teile olen rääkinud, on vaim ja on elu.
\par 64 Aga teie seas on mõned, kes ei usu!” Sest Jeesus teadis algusest peale, kes need olid, kes ei uskunud, ja kes see on, kes tema ära annab.
\par 65 Ja ta ütles: „Sellepärast ma olen teile öelnud, et ükski ei või tulla minu juurde, kui temale seda ei ole andnud minu Isa!”
\par 66 Sellest ajast läksid paljud tema jüngrid ära ta järelt ega kõndinud enam temaga.

\section*{Peetruse tunnistus Jeesusest}

\par 67 Siis Jeesus ütles neile kaheteistkümnele: „Kas ka teie tahate ära minna?”
\par 68 Siimon Peetrus vastas temale: „Issand, kelle juurde me läheme? Sinul on igavese elu sõnad,
\par 69 ja me oleme uskunud ja tundnud, et sina oled Jumala püha!”
\par 70 Jeesus kostis neile: „Eks ole mina teid kahtteistkümmend ära valinud, ja üks teie seast on kurat?”
\par 71 Aga ta rääkis Juudast, Iskarioti Siimona pojast; sest see pidi tema ära andma ja oli üks neist kaheteistkümnest.


\chapter{7}

\section*{Jeesus ja tema vennad}

\par 1 Ja pärast seda käis Jeesus mööda Galileamaad, sest ta ei tahtnud käia Juudamaal, kuna juudid püüdsid teda tappa.
\par 2 Aga juutide lehtmajade püha oli ligi.
\par 3 Siis ütlesid ta vennad temale: „Mine siit ära Judeasse, et ka su jüngrid näeksid su tegusid, mida sa teed;
\par 4 sest ükski ei tee midagi salajas, kui ta tahab olla avalikkuses. Kui sa niisuguseid asju teed, siis ilmuta ennast maailmale!”
\par 5 Sest ta vennadki ei uskunud temasse.
\par 6 Siis ütles Jeesus neile: „Minu aeg ei ole veel tulnud, aga teie aeg on alati soodus.
\par 7 Maailm ei või teid vihata, aga mind ta vihkab, sest mina tunnistan temast, et tema teod on kurjad.
\par 8 Minge teie üles pühiks; mina veel ei lähe üles neiks pühiks, sest minu aeg ei ole veel täis saanud.”
\par 9 Kui ta neile seda oli öelnud, jäi ta Galileasse.
\par 10 Ent kui ta vennad olid üles läinud pühiks, läks temagi üles, mitte avalikult, vaid otsekui salaja.
\par 11 Siis otsisid juudid teda pühil ja ütlesid: „Kus ta on?”
\par 12 Ja nurisemine tema pärast oli suur hulkade seas. Mõned ütlesid: „Tema on hea!” Teised ütlesid: „Ei, vaid ta eksitab rahvast!”
\par 13 Kuid ükski ei rääkinud temast julgesti hirmu pärast juutide ees.

\section*{Jumal on läkitanud Jeesuse}

\par 14 Aga kui pühad juba olid poole peale jõudnud, läks Jeesus üles pühakotta ja õpetas.
\par 15 Siis juudid panid seda imeks ning ütlesid: „Kuidas see Kirja tunneb ilma õppimata?”
\par 16 Jeesus aga vastas neile ning ütles: „Minu õpetus ei ole minu oma, vaid selle, kes mind on läkitanud.
\par 17 Kui keegi tahab teha tema tahtmist, see tunneb, kas see õpetus on Jumalast või kas mina räägin iseenesest.
\par 18 Kes iseenesest räägib, see otsib iseenese austust; aga kes otsib selle austust, kes teda on läkitanud, see on tõeline ja temas ei ole ülekohut.
\par 19 Eks Mooses ole teile andnud käsuõpetuse, ja ükski teie seast ei tee käsuõpetuse järgi. Miks te püüate mind tappa?”
\par 20 Rahvas vastas: „Sul on kuri vaim; kes püüab sind tappa?”
\par 21 Jeesus kostis ning ütles neile: „Ühe teo ma tegin ja te kõik imestate.
\par 22 Sellepärast on Mooses teile andnud ümberlõikamise - mitte nii, et see on Moosesest, vaid esiisadest - ja te lõikate inimese ümber ka hingamispäeval.
\par 23 Kui inimene lõigatakse ümber hingamispäeval, et Moosese käsuõpetust ei rikutaks, miks te siis olete pahased minule, et ma olen kogu inimese terveks teinud hingamispäeval?
\par 24 Ärge mõistke kohut silmnäo järgi, vaid mõistke õiget kohut!”
\par 25 Siis ütlesid mõned Jeruusalemma elanikest: „Eks see ole see, keda nad püüavad tappa?
\par 26 Ja ennäe, tema räägib vabalt ja nad ei ütle talle midagi. Kas on ehk meie ülemad tõesti ära tundnud, et tema on Kristus?
\par 27 Ometi me teame, kust ta on. Aga kui Kristus tuleb, siis ei tea ükski, kust ta on!”
\par 28 Siis Jeesus hüüdis pühakojas õpetades ning ütles: „Küll te tunnete mind ja teate, kust ma olen, ja ma ei ole tulnud iseenesest; aga tõeline on see, kes mind on läkitanud, keda te ei tunne.
\par 29 Mina tunned teda, sest ma olen temast, ja tema on mind läkitanud!”
\par 30 Siis nad püüdsid teda kinni võtta, aga ükski ei pistnud kätt tema külge, sest tema tund ei olnud veel tulnud.
\par 31 Aga paljud rahva seast uskusid temasse ja ütlesid: „Kui Kristus tuleb, kas ta peaks tegema veel rohkem tunnustähti kui see on teinud?”
\par 32 Variserid kuulsid rahvast isekeskis seda temast sosistavat. Ja variserid ja ülempreestrid läkitasid sulaseid teda kinni võtma.
\par 33 Siis ütles Jeesus neile: „Ma olen veel üürikese aja teie juures ja siis ma lähen selle juurde, kes mind on läkitanud.
\par 34 Siis te otsite mind, aga ei leia, ja kus ma olen, sinna te ei või tulla!”
\par 35 Siis ütlesid juudid isekeskis: „Kuhu ta tahab minna, et me teda ei peaks leidma? Mõtleb ta minna hajuvil asuvate kreeklaste juurde ja õpetada kreeklasi?
\par 36 Mis sõna see on, mis ta ütles: te otsite mind, aga ei leia, ja kus ma olen, sinna te ei või tulla?”

\section*{Elav vesi}

\par 37 Aga pühade viimsel, suurel päeval seisis Jeesus ja kõneles valju häälega ning ütles: „Kui kellelgi on janu, see tulgu minu juurde ja joogu!
\par 38 Kes usub minusse, nagu Kiri ütleb, selle ihust peavad voolama elava vee jõed!”
\par 39 Aga seda ta ütles Vaimust, kelle pidid saama need, kes temasse usuvad; sest veel ei olnud Vaimu, ei olnud ju Jeesust veel mitte austatud.
\par 40 Siis ütlesid paljud rahva seast, kes seda kõnet kuulsid: „Tema on tõesti see prohvet!”
\par 41 Teised ütlesid: „Tema on Kristus!” Aga muud ütlesid: „Ega siis Kristus Galileast tule?
\par 42 Eks Kiri ütle, et Kristus tuleb Taaveti soost ja Petlemmast, alevist, kust oli Taavet?”
\par 43 Siis tekkis lahkmeel rahva sekka tema pärast.
\par 44 Aga mõned neist tahtsid teda kinni võtta, kuid ükski ei pistnud kätt tema külge.
\par 45 Nii tulid sulased ülempreestrite ja variseride juurde tagasi. Ja need küsisid neilt: „Mispärast te ei ole teda toonud?”
\par 46 Sulased kostsid: „Ükski inimene ei ole iialgi nõnda rääkinud nagu see inimene!”
\par 47 Siis vastasid variserid neile: „Kas teiegi olete eksitatud?
\par 48 Kas ükski ülemaist või variseridest on uskunud temasse?
\par 49 Aga see rahvahulk, kes ei tunne käsuõpetust, on neetud!”
\par 50 Nikodeemus, kes enne oli tema juurde tulnud ja oli üks nende seast, ütleb neile:
\par 51 „Kas meie seadus inimese hukka mõistab, enne kui ta tema üle kuulab ja teada saab, mis ta on teinud?”
\par 52 Nad vastasid ning ütlesid temale: „Oled ka sina Galileast? Uuri ja vaata, et Galileast ei tõuse prohvetit!”
\par 53 [Ja igaüks läks koju.


\chapter{8}

\section*{Abielurikkumiselt tabatud naine}

\par 1 Aga Jeesus läks Õlimäele.
\par 2 Vara hommikul ta tuli jälle pühakotta ja kõik rahvas tuli tema juurde; ja ta istus maha ja õpetas neid.
\par 3 Seal kirjatundjad ja variserid tõid tema juurde naise, kes oli tabatud abielurikkumiselt. Ja nad panid ta keskpaika seisma
\par 4 ning ütlesid temale: „Õpetaja, see naine tabati abielurikkumiselt!
\par 5 Aga Mooses on käsuõpetuses käskinud meid niisugused kividega surnuks visata. Mis siis sina ütled?”
\par 6 Seda nad ütlesid teda kiusates, et neil oleks kaebamist tema peale. - Jeesus kummardas ja kirjutas sõrmega maa peale.
\par 7 Kui nad siis küsides käisid temale peale, ajas ta enese sirgu ning ütles neile: „Kes teie seast on patuta, see olgu esimene tema peale kivi viskama!”
\par 8 Ja ta kummardas jälle ning kirjutas maa peale.
\par 9 Kui nad seda kuulsid ja südametunnistus neid süüdistas, läksid nad välja üksteise järel, vanemad eesotsas. Ja Jeesus jäi sinna üksi ja naine, nagu ta seal seisis.
\par 10 Ent Jeesus ajas enese sirgu ega näinud kedagi muud kui naist; ja ta ütles naisele: „Naine, kus on need sinu süüdistajad? Ega keegi ole sind hukka mõistnud?”
\par 11 Naine vastas: „Mitte keegi, Issand!” Jeesus ütles: „Ega minagi sind hukka mõista; mine ja ära tee enam pattu!”]

\section*{Maailma valgus}

\par 12 Siis rääkis Jeesus jälle neile ning ütles: „Mina olen maailma valgus. Kes mind järgib, see ei käi pimeduses, vaid temal on elu valgus!”
\par 13 Siis variserid ütlesid talle: „Sina tunnistad iseenesest, sinu tunnistus ei ole tõsi!”
\par 14 Jeesus vastas ning ütles neile: „Ehk mina küll iseenesest tunnistan, on mu tunnistus siiski tõsi, sest ma tean, kust ma olen tulnud ja kuhu ma lähen. Aga teie ei tea, kust ma tulen ja kuhu ma lähen.
\par 15 Teie mõistate kohut liha järgi, mina ei mõista kohut kellegi üle!
\par 16 Aga kui mina ka kohut mõistan, siis on minu otsus tõsi, sest mina ei ole üksi, vaid minuga on see, kes mind on läkitanud.
\par 17 Ja teie käsuõpetuseski on kirjutatud, et kahe inimese tunnistus on tõsi.
\par 18 Mina annan enesest ise tunnistust, ja ka Isa, kes mind on läkitanud, tunnistab minust!”
\par 19 Siis nad ütlesid temale: „Kus on sinu Isa!” Jeesus vastas: „Teie ei tunne mind ega mu Isa. Kui te mind tunneksite, siis te tunneksite ka mu Isa!”
\par 20 Need sõnad rääkis Jeesus ohvriraha kirstu juures, kui ta pühakojas õpetas. Ja ükski ei võtnud teda kinni, sest tema tund ei olnud veel tulnud.

\section*{Jeesus ei ole sellest maailmast}

\par 21 Siis ütles Jeesus taas neile: „Mina lähen ära ja te otsite mind, ja te surete oma pattudesse. Kuhu mina lähen, sinna teie ei või tulla!”
\par 22 Siis ütlesid juudid: „Kas ta õige tahab ennast tappa, et ta ütleb: kuhu mina lähen, sinna teie ei või tulla?”
\par 23 Ja ta ütles neile: „Teie olete alt, mina olen ülalt; teie olete sellest maailmast, mina ei ole sellest maailmast.
\par 24 Sellepärast ma ütlesin teile, et te surete oma pattudesse; sest kui te ei usu, et mina see olen, siis te surete oma pattudesse!”
\par 25 Nemad aga ütlesid temale: „Kes sa siis oled?” Jeesus vastas neile: „Kõigepealt see, mida ma teile ütlengi.
\par 26 Mul on teist palju öelda ja teie kohta otsustada; aga see, kes mind on läkitanud, on tõeline, ja mis ma temalt olen kuulnud, seda ma räägin maailmale!”
\par 27 Nad ei saanud aru, et ta rääkis neile Isast.
\par 28 Siis ütles Jeesus: „Kui te Inimese Poja olete ülendanud, siis te mõistate, et mina see olen ja et ma ei tee midagi iseenesest, vaid räägin seda selle järgi, kuidas mu Isa mind on õpetanud.
\par 29 Ja see, kes mind on läkitanud, on minuga; ta ei ole mind üksi jätnud, sest ma teen ikka, mis on tema meelt mööda!”
\par 30 Kui ta seda rääkis, uskusid paljud temasse.

\section*{Vaidlus juutidega}

\par 31 Siis ütles Jeesus juutidele, kes teda uskusid: „Kui te jääte minu sõnasse, siis te olete tõesti minu jüngrid
\par 32 ja tunnetate tõe, ja tõde teeb teid vabaks!”
\par 33 Nemad kostsid temale: „Meie oleme Aabrahami sugu ega ole veel iialgi kedagi orjanud; kuidas sa siis ütled: te saate vabaks?”
\par 34 Jeesus kostis neile: „Tõesti, tõesti ma ütlen teile, et igaüks, kes teeb pattu, on patu ori!
\par 35 Ent ori ei jää majasse igavesti, poeg jääb igavesti.
\par 36 Kui nüüd Poeg teid vabaks teeb, siis te olete õieti vabad!
\par 37 Ma tean, et te olete Aabrahami sugu; aga te püüate mind tappa, sest mu sõna ei mahu teisse.
\par 38 Mina räägin, mida olen näinud oma Isalt, ja teie teete, mida olete kuulnud oma isalt!”
\par 39 Nad vastasid ning ütlesid temale: „Meie isa on Aabraham!” Jeesus ütles neile: „Kui te oleksite Aabrahami lapsed, siis te teeksite ka Aabrahami tegusid.
\par 40 Aga nüüd te püüate tappa mind, inimest, kes olen rääkinud teile tõtt, mida ma olen kuulnud Jumalalt. Seda ei ole Aabraham teinud.
\par 41 Teie teete oma isa tegusid!„ Nad ütlesid temale: ”Meie ei ole sündinud porduelust; meil on üks isaks, Jumal!”
\par 42 Jeesus ütles neile: „Oleks Jumal teie isa, siis te armastaksite mind, sest mina olen lähtunud Jumalast ja tulen temast, sest ma ei ole ju tulnud iseenesest, vaid tema on mind läkitanud.
\par 43 Mispärast te ei mõista minu kõnet? Sellepärast et te ei või kuulda minu sõna.
\par 44 Teie olete oma isast kuradist ja oma isa himude järgi te tahate teha! Tema on inimese tapja olnud algusest peale ega ole jäänud püsima tõesse, sest tõde ei ole tema sees. Kui ta räägib valet, siis ta räägib omast, sest ta on valetaja ja vale isa!
\par 45 Aga et mina tõtt räägin, siis te ei usu mind!
\par 46 Kes teie seast võib mind patus süüdistada? Kui mina tõtt räägin, miks te siis ei usu mind?
\par 47 Kes Jumalast on, see kuuleb Jumala sõnu. Teie ei kuule selle pärast, et teie ei ole Jumalast!”
\par 48 Siis kostsid juudid ning ütlesid talle: „Eks me ütle õigesti, et sina oled samaarlane ja et sul on kuri vaim?”
\par 49 Jeesus vastas: „Mul ei ole kurja vaimu, aga ma austan oma Isa ja teie teotate mind.
\par 50 Mina aga ei otsi oma au: üks on, kes seda otsib ja kohut mõistab!
\par 51 Tõesti, tõesti ma ütlen teile, et kui keegi minu sõna peab, siis ta ei näe surma igavesti!”
\par 52 Juudid ütlesid temale: „Nüüd me oleme ära tundnud, et sul on kuri vaim! Aabraham on surnud ja prohvetid, ja sina ütled: Kui keegi minu sõna peab, siis ta ei maitse surma igavesti!
\par 53 Kas sina oled suurem kui meie isa Aabraham, kes on surnud? Ja prohvetid on surnud. Kelleks sina ennast pead?”
\par 54 Jeesus vastas: „Kui mina iseenesele annan austust, siis ei ole mu austus midagi; minu Isa on, kes mulle annab austuse, keda te ütlete oma Jumala olevat.
\par 55 Ja teie ei ole teda tundnud, aga mina tunnen teda. Ja kui ma ütleksin, et ma teda ei tunne, siis ma oleksin teie sarnane valelik; aga ma tunnen teda ja pean tema sõna.
\par 56 Aabraham, teie isa, hakkas rõõmutsema, et tema saab näha minu päeva. Ja ta nägi seda ja oli rõõmus!”
\par 57 Siis ütlesid juudid temale: „Sa ei ole veel viiskümmend aastat vana ja oled näinud Aabrahami?”
\par 58 Jeesus ütles neile: „Tõesti, tõesti ma ütlen teile, et enne kui Aabraham sündis, olin mina!”
\par 59 Siis nad haarasid kive, et tema peale visata. Aga Jeesus peitis enese ära ja väljus pühakojast.


\chapter{9}

\section*{Jeesus teeb nägijaks pimedana sündinu}

\par 1 Ja mööda minnes ta nägi inimest, kes sündimisest saadik oli olnud pime.
\par 2 Ja tema jüngrid küsisid temalt: „Rabi, kes on pattu teinud, kas tema või ta vanemad, et ta on sündinud pimedana?”
\par 3 Jeesus vastas: „Ei ole tema pattu teinud ega ta vanemad, vaid et Jumala teod saaksid temas avalikuks.
\par 4 Meie peame tegema selle tegusid, kes mind on läkitanud, niikaua kui päeva on; öö tuleb, mil ükski ei või midagi teha!
\par 5 Niikaua kui ma olen maailmas, olen ma maailma valgus!”
\par 6 Kui ta seda oli öelnud, sülitas ta maha ja tegi süljest muda ning võidis selle mudaga tema silmad
\par 7 ja ütles talle: „Mine pese silmi Siiloa tiigis!” - Siiloa tähendab „Läkitatu”. - Siis ta läks sinna ja pesi, ja tuli nägijana tagasi.
\par 8 Aga naabrid ja kes teda varemini olid näinud kerjuse olevat, ütlesid: „Eks see ole sama, kes istus ja kerjas?”
\par 9 Ühed ütlesid: „See on sama!” Aga teised: „Ei ole, vaid ta on tema sarnane!” Tema ise ütles: „Mina olen see!”
\par 10 Siis nad küsisid temalt: „Kuidas siis sinu silmad avati?”
\par 11 Tema vastas: „Inimene, keda kutsutakse Jeesuseks, tegi muda ja võidis sellega mu silmi ja ütles mulle: Mine Siiloa tiigi äärde ja pese silmi! Aga kui ma sinna läksin ja pesin, sain ma nägijaks!”
\par 12 Siis ütlesid nad temale: „Kus ta on?” Tema ütles: „Ei mina tea!”
\par 13 Siis nad viisid tema, kes enne oli pime olnud, variseride juurde.
\par 14 Aga see oli hingamispäev, kui Jeesus tegi muda ja avas ta silmad,
\par 15 Siis küsisid variseridki taas temalt, kuidas ta oli nägijaks saanud. Aga tema ütles neile: „Ta pani muda mu silmadele ja ma pesin näo ning sain nägijaks!”
\par 16 Siis ütlesid mõned variseridest: „See inimene ei ole Jumalast, sest ta ei pea hingamispäeva!” Teised ütlesid: „Kuidas võib patune inimene teha niisuguseid tunnustähti?” Ja nende seas oli lahkmeel.
\par 17 Nüüd nad ütlesid taas pimedale: „Sina, mis sina ütled temast, et ta su silmad avas?” Tema ütles: „Ta on prohvet!”
\par 18 Aga juudid ei uskunud, et ta oli olnud pime ja oli saanud nägijaks, kuni nad kutsusid nägijaks-saanu vanemad
\par 19 ja küsisid neilt ning ütlesid: „Kas see on teie poeg, kelle te ütlete pimedana sündinud olevat? Kuidas ta siis nüüd näeb?”
\par 20 Ta vanemad kostsid neile ning ütlesid: „Me teame, et see on meie poeg ja et ta sündis pimedana.
\par 21 Aga kuidas ta nüüd näeb, seda me ei tea; või kes ta silmad avas, seda me ei tea. Ta on küllalt vana, küsige temalt eneselt, küllap ta ise kostab enese eest!”
\par 22 Seda ütlesid ta vanemad, sest nad kartsid juute; sest juudid olid juba isekeskis võtnud nõuks, et kui keegi peaks teda tunnistama Kristuseks, siis see lükatakse kogudusest välja.
\par 23 Sellepärast ütlesid ta vanemad: „Ta on küllalt vana, küsige temalt eneselt!”
\par 24 Siis nad kutsusid teist korda inimese, kes oli pime olnud, ja ütlesid temale: „Anna Jumalale au! Me teame, et see inimene on patune!”
\par 25 Tema vastas nüüd: „Kas ta on patune, seda ma ei tea; üht ma tean, et olin pime ja nüüd näen!”
\par 26 Nad küsisid nüüd temalt: „Mis ta tegi sulle? Kuidas ta su silmad avas?”
\par 27 Ta vastas neile: „Ma juba ütlesin teile, aga te ei võtnud kuulda. Miks te jälle tahate kuulda? Või tahate teiegi hakata tema jüngreiks?”
\par 28 Siis nad hakkasid teda sõimama ja ütlesid: „Sina oled tema jünger, meie aga oleme Moosese jüngrid!
\par 29 Me teame, et Jumal on rääkinud Moosesega, aga kust see inimene on, seda me ei tea!”
\par 30 Mees kostis ning ütles neile: „See ongi imelik, et te ei tea, kust ta on, ja tema avas mu silmad!
\par 31 Me ju teame, et Jumal ei kuule patuseid, vaid kui keegi on jumalakartlik ja teeb tema tahtmist, siis seda ta kuuleb.
\par 32 Maailma algusest ei ole kuuldud, et keegi on avanud sündinud pimeda silmad.
\par 33 Kui ta ei oleks Jumalast, siis ta ei võiks midagi teha!”
\par 34 Nad kostsid ning ütlesid talle: „Sina oled koguni pattudes sündinud ja sina õpetad meid?” Ja nad tõukasid ta välja.
\par 35 Jeesus sai kuulda, et nad olid ta välja tõuganud; ja kui ta tema leidis, ütles ta: „Kas sa usud Inimese Pojasse?”
\par 36 Tema vastas ja ütles: „Issand, kes see on, et võiksin uskuda temasse?”
\par 37 Jeesus ütles temale: „Sa oled ju teda näinud ja see on tema, kes sinuga räägib!”
\par 38 Aga tema ütles: „Issand, ma usun!” Ja ta kummardas teda.
\par 39 Ja Jeesus ütles: „Mina olen tulnud kohtumõistmiseks sellesse maailma, et need, kes ei näe, saaksid nägijaks, ja kes näevad, saaksid pimedaks!”
\par 40 Ja mõned variseridest, kes olid tema juures, kuulsid seda ja ütlesid: „Kas meiegi oleme pimedad?”
\par 41 Jeesus ütles neile: „Kui te oleksite pimedad, ei oleks teil pattu; aga et te nüüd ütlete: me näeme! siis jääb teie patt teile!


\chapter{10}

\section*{Võrdluspilt lambatarast}

\par 1 Tõesti, tõesti ma ütlen teile: kes ei lähe uksest sisse lambatarasse, vaid astub sisse mujalt, see on varas ja röövel!
\par 2 Aga kes uksest sisse läheb, see on lammaste karjane;
\par 3 sellele avab uksehoidja, ja lambad kuulevad ta häält, ja ta kutsub omi lambaid nimepidi ja viib nad välja.
\par 4 Ja kui ta oma lambad on välja ajanud, käib ta nende ees ja lambad järgivad teda, sest nad tunnevad tema häält.
\par 5 Aga võõrast nad ei järgi, vaid põgenevad tema eest, sest nad ei tunne võõraste häält!”
\par 6 Selle võrdumi Jeesus ütles neile; nemad aga ei saanud aru, mis see on, mida ta neile rääkis.
\par 7 Sellepärast ütles Jeesus taas: „Tõesti, tõesti ma ütlen teile, et mina olen lammaste uks!
\par 8 Kõik, kes on tulnud enne mind, on vargad ja röövlid; lambad aga ei ole neid kuulnud!
\par 9 Mina olen uks. Kui keegi minu kaudu sisse läheb, siis ta saab õndsaks ja käib sisse ja välja ja leiab karjamaad.
\par 10 Varas ei tule muu pärast kui varastama ja tapma ning hukkama. Mina olen tulnud, et neil oleks elu ja kõike ülirohkesti!

\section*{Hea karjane}

\par 11 Mina olen hea karjane. Hea karjane jätab oma elu lammaste eest.
\par 12 Palgaline ja kes ei ole karjane, kelle omad lambad ei ole, näeb hundi tulevat ja jätab lambad maha ja põgeneb - ja hunt kisub neid ja ajab nad laiali.
\par 13 - Ta põgeneb, sest ta on palgaline ega hooli lammastest.
\par 14 Mina olen hea karjane ja tunnen omi ja minu omad tunnevad mind,
\par 15 nõnda nagu Isa tunneb mind ja mina tunnen Isa, ja ma jätan oma elu lammaste eest.
\par 16 Ja mul on veel teisi lambaid, kes ei ole sellest tarast; needki ma pean tooma siia, ja nad kuulevad minu häält, ja siis on üks kari ja üks karjane.
\par 17 Sellepärast Isa armastab mind, et ma jätan oma elu, et seda jälle võtta.
\par 18 Ükski ei võta seda minult, vaid ma jätan selle iseenesest. Mul on meelevald seda jätta ja mul on meelevald seda jälle võtta. Selle käsusõna ma olen saanud oma Isalt!”
\par 19 Siis tekkis taas vaidlus juutide keskel nende sõnade pärast.
\par 20 Ja paljud nende seast ütlesid: „Temas on kuri vaim ja ta jampsib! Miks te teda kuulate?”
\par 21 Teised ütlesid: „Need ei ole seestunu sõnad; ega kuri vaim või avada pimedate silmi?”

\section*{Jeesus pühade puhul Jeruusalemmas}

\par 22 Siis tulid templi uuendamise pühad Jeruusalemmas. Oli talv.
\par 23 Ja Jeesus kõndis pühakojas Saalomoni võlvitud hoones.
\par 24 Siis juudid asusid ta ümber ja ütlesid temale: „Kaua sa pead meie hinge kahevahel? Oled sa Kristus, siis ütle meile seda lausa!”
\par 25 Jeesus kostis neile: „Ma olen teile juba öelnud, aga te ei usu! Teod, mis ma teen oma Isa nimel, need tunnistavad minust.
\par 26 Teie aga ei usu, sest teie ei ole minu lammaste hulgast.
\par 27 Minu lambad kuulevad minu häält ja mina tunnen neid ja nemad järgivad mind.
\par 28 Ja mina annan neile igavese elu ja nemad ei saa iialgi hukka, ja ükski ei kisu neid minu käest.
\par 29 Minu Isa, kes nad mulle on andnud, on suurem kui kõik; ja ükski ei saa neid minu Isa käest ära kiskuda.
\par 30 Mina ja Isa oleme üks!”

\section*{Juutide vaenulikkus}

\par 31 Siis võtsid juudid jälle maast kive, et teda nendega visata.
\par 32 Jeesus kostis neile: „Palju häid tegusid ma olen teile näidanud oma Isast; missuguse teo pärast nende seast te tahate mind kividega visata?”
\par 33 Juudid vastasid temale: „Hea teo pärast me ei viska sind kividega, vaid Jumala pilkamise pärast ja et sina, kes oled inimene, pead ennast Jumalaks!”
\par 34 Jeesus vastas neile: „Eks teie käsuõpetuses ole kirjutatud: mina olen öelnud: teie olete jumalad!?
\par 35 Kui ta nimetab jumalaiks neid, kelle kätte sai Jumala sõna - ja Kiri ei või jääda valeks -
\par 36 kuidas te siis ütlete sellele, keda Isa on pühitsenud ja läkitanud maailma: sina pilkad Jumalat! sellepärast et ma ütlesin: mina olen Jumala Poeg?
\par 37 Kui ma ei tee oma Isa tegusid, siis ärge mind uskuge;
\par 38 aga kui ma teen, siis uskuge neid tegusid, kui te ka mind ei usu, et te tunneksite ja saaksite aru, et Isa on minus ja mina olen Isas!”
\par 39 Siis nad püüdsid taas teda kinni võtta, aga ta pääses nende käest.
\par 40 Ja tema läks jälle teisele poole Jordanit, sinna paika, kus Johannes esmalt oli olnud ristimas; ja ta viibis seal.
\par 41 Ja paljud tulid tema juurde ja ütlesid: „Johannes ei teinud küll ühtki tunnustähte; aga kõik, mis Johannes sellest mehest on öelnud, on tõsi!”
\par 42 Ja seal uskusid paljud temasse.


\chapter{11}

\section*{Laatsaruse ülesäratamine}

\par 1 Aga keegi Laatsarus Betaaniast, Maarja ja tema õe Marta alevist, oli haige.
\par 2 Maarja aga oli see, kes Issandat oli kalli salviga võidnud ja tema jalgu oma juustega kuivatanud. Tema vend Laatsarus oligi haigestunud.
\par 3 Siis läkitasid õed tema juurde ütlema: „Issand, vaata, see, keda sa armastad, on haige!”
\par 4 Kui Jeesus seda kuulis, ütles ta: „See haigus ei ole surmaks, vaid Jumala austuseks, et Jumala Poega selle läbi austataks!”
\par 5 Aga Jeesus armastas Martat ja tema õde ja Laatsarust.
\par 6 Kui ta nüüd kuulis tema haige olevat, siis ta viibis veel kaks päeva seal paigas, kus ta oli.
\par 7 Pärast seda ütles ta jüngritele: „Lähme jälle Juudamaale!”
\par 8 Jüngrid ütlesid temale: „Rabi, äsja püüdsid juudid sind kividega visata ja sa lähed jälle sinna?”
\par 9 Jeesus kostis: „Eks kaksteist tundi ole päevas? Kui keegi kõnnib päeval, siis ta ei komista, sest ta näeb selle maailma valgust.
\par 10 Aga kui keegi kõnnib öösel, siis ta komistab, sest temas ei ole valgust!”
\par 11 Seda ta rääkis. Ja pärast seda ta ütles neile: „Laatsarus, meie sõber, magab, aga ma lähen teda unest äratama!”
\par 12 Siis ütlesid tema jüngrid: „Issand, kui ta magab, siis ta saab terveks!”
\par 13 Aga Jeesus rääkis ta surmast; nemad aga mõtlesid, et ta räägib une magamisest.
\par 14 Siis Jeesus ütles neile lausa: „Laatsarus on surnud,
\par 15 ja ma olen rõõmus teie pärast, et ma ei olnud seal, et te usuksite. Aga läki tema juurde!”
\par 16 Siis ütles Toomas, keda hüütakse Kaksikuks, kaasjüngritele: „Lähme ka meie, et ühes temaga surra!”
\par 17 Kui nüüd Jeesus tuli, leidis ta, et Laatsarus oli juba neli päeva hauas olnud.
\par 18 Aga Betaania oli Jeruusalemma lähedal, umbes viisteist vagu maad sellest.
\par 19 Ja palju juute oli tulnud Marta ja Maarja juurde neid trööstima nende venna pärast.
\par 20 Kui nüüd Marta kuulis, et Jeesus on tulemas, läks ta temale vastu; aga Maarja jäi koju.
\par 21 Siis Marta ütles Jeesusele: „Issand, kui sa oleksid siin olnud, mu vend ei oleks surnud!
\par 22 Aga nüüdki ma tean, et Jumal sulle annab, mis sa iganes Jumalalt palud!”
\par 23 Jeesus ütles talle: „Sinu vend tõuseb üles!”
\par 24 Marta ütles talle: „Ma tean, et ta üles tõuseb ülestõusmises viimsel päeval!”
\par 25 Jeesus ütles temale: „Mina olen ülestõusmine ja elu; kes minusse usub, see elab, ehk ta küll sureb!
\par 26 Ja igaüks, kes elab ja minusse usub, see ei sure igavesti. Kas sa usud seda?”
\par 27 Ta ütleb temale: „Jah, Issand, mul on see usk, et sina oled Kristus Jumala Poeg, kes maailma on tulemas!”
\par 28 Ja kui ta seda oli öelnud, läks ta ja kutsus oma õe Maarja salaja ning ütles: „Õpetaja on siin ja kutsub sind!”
\par 29 Kui õde seda kuulis, tõusis ta sedamaid ja läks tema juurde.
\par 30 Sest Jeesus ei olnud veel jõudnud alevisse, vaid oli veel seal paigas, kus Marta teda oli kohanud.
\par 31 Kui nüüd juudid, kes olid Maarja juures toas teda trööstimas, nägid, et Maarja rutates tõusis ja väljus, järgisid nad teda mõeldes, et ta läheb haua juurde nutma.
\par 32 Kui siis Maarja jõudis sinna, kus Jeesus oli, ja teda nägi, heitis ta maha tema jalgade ette ning ütles talle: „Issand, oleksid sa siin olnud, mu vend ei oleks surnud!”
\par 33 Kui nüüd Jeesus nägi teda nutvat ja juute, kes temaga olid tulnud, ka nutvat, ärritus ta vaimus ja võpatas.
\par 34 Ja ta ütles: „Kuhu te olete ta pannud?” Nad ütlesid temale: „Issand, tule ja vaata!”
\par 35 Jeesus nuttis.
\par 36 Siis ütlesid juudid: „Vaata, kuidas ta teda armastas!”
\par 37 Aga mõned nende seast ütlesid: „Kas tema, kes avas pimeda silmad, ei võinud teha, et ka see ei oleks surnud?”
\par 38 Siis Jeesus ärritus taas iseeneses ning tuli haua juurde. See oli koobas ja selle avausel lasus kivi.
\par 39 Jeesus ütleb: „Tõstke kivi ära!” Siis ütleb surnu õde Marta temale: „Issand, ta lehkab juba, sest on juba neljas päev!”
\par 40 Jeesus ütleb temale: „Eks ma öelnud sulle: kui sa usuksid, saaksid sa näha Jumala au?”
\par 41 Siis nad tõstsid kivi ära. Aga Jeesus tõstis oma silmad üles ja ütles: „Ma tänan sind, Isa, et sa mind oled kuulnud!
\par 42 Mina ju teadsin, et sa mind ikka kuuled; kuid rahva pärast, kes siin ümber seisab, ma räägin, et nad usuksid, et sina mind oled läkitanud!”
\par 43 Ja kui ta seda oli öelnud, hüüdis ta suure häälega: „Laatsarus, tule välja!”
\par 44 Ja surnu tuli välja jalust ja käsist mähkmetega mähitud ja nägu higirätikusse mässitud. Jeesus ütleb neile: „Päästke ta lahti ja laske ta minna!”

\section*{Vandenõu Jeesuse tapmiseks}

\par 45 Siis palju juute, kes olid tulnud Maarja juurde ja olid näinud, mis Jeesus oli teinud, uskusid temasse.
\par 46 Aga mõned neist läksid variseride juurde ja ütlesid neile, mis Jeesus oli teinud.
\par 47 Siis ülempreestrid ja variserid kogusid kokku Suurkohtu ja ütlesid: „Mida me peame tegema? Sest see inimene teeb palju tunnustähti.
\par 48 Kui me ta nõnda jätame, usuvad kõik temasse; siis tulevad roomlased ja võtavad ära meie paiga ja rahva!”
\par 49 Aga üks nende seast, Kaifas, kes oli sel aastal ülempreester, ütles neile: „Te ei tea midagi
\par 50 ega mõtle sellele, et teile on parem, et üks inimene sureb rahva eest kui et kogu rahvas hukkub!”
\par 51 Ent seda ta ei öelnud iseenesest, vaid olles sel aastal ülempreester, kuulutas ta prohveti viisil, et Jeesus pidi surema rahva eest,
\par 52 ja mitte ainult selle rahva eest, vaid et ta ka hajuvil elavad Jumala lapsed koguks ühtekokku.
\par 53 Sellest päevast alates nad pidasid nüüd üheskoos nõu teda ära tappa.
\par 54 Siis ei käinud Jeesus enam avalikult juutide seas, vaid läks sealt ära kõrbelähedasse kohta, Efraimi-nimelisse linna; ja ta viibis seal oma jüngritega.
\par 55 Aga juutide paasapühad olid ligi, ja paljud läksid sealt maalt enne paasapühi üles Jeruusalemma endid puhastama.
\par 56 Siis nad otsisid Jeesust ja rääkisid üksteisega pühakojas seistes: „Mis te arvate, kas ta ei tule pühiks?”
\par 57 Ülempreestrid ja variserid olid aga andnud käsud, et kui keegi peaks teadma, kus ta on, see annaks seda teada, et nad ta kinni võtaksid.


\chapter{12}

\section*{Jeesust võitakse Betaanias}

\par 1 Kuus päeva enne paasapüha tuli Jeesus Betaaniasse, kus elas Laatsarus, kelle Jeesus oli surnuist üles äratanud.
\par 2 Siis nad tegid seal temale söömaaja ja Marta ümmardas. Ja Laatsarus oli üks neist, kes ühes temaga lauas istusid.
\par 3 Siis võttis Maarja naela selget, väga kallist nardisalvi ning võidis Jeesuse jalgu ja kuivatas ta jalgu oma juustega. Ja maja sai täis salvi lõhna.
\par 4 Aga üks tema jüngritest, Juudas Iskariot, kes pärast tema ära andis, ütles:
\par 5 „Mispärast ei ole see salv ära müüdud kolmesaja teenari eest ja raha antud vaestele?”
\par 6 Aga seda ta ei öelnud, et ta vaestest hoolis, vaid et ta oli varas ja ta käes oli kukkur, ja ta kõrvaldas, mis sisse pandi.
\par 7 Siis ütles Jeesus: „Jäta ta rahule, et ta seda saaks hoida minu matusepäevaks.
\par 8 Sest vaeseid on ikka teie juures, aga mind ei ole teil ikka!”
\par 9 Siis sai suur hulk juute teada, et tema on seal, ja nad ei tulnud sinna mitte ainult Jeesuse pärast, vaid et ka näha saada Laatsarust, kelle ta oli surnuist üles äratanud.
\par 10 Aga ülempreestrid pidasid nõu Laatsarusegi ära tappa,
\par 11 sest tema pärast läks sinna palju juute ja uskus Jeesusesse.

\section*{Jeesus tuleb Jeruusalemma kui Kuningas}

\par 12 Järgmisel päeval, kui palju rahvast, kes oli tulnud pühi pidama, kuulis, et Jeesus tuleb Jeruusalemma,
\par 13 võtsid nad palmipuude oksi ja läksid välja temale vastu ning hüüdsid: „Hoosianna, õnnistatud olgu, kes tuleb Issanda nimel, Iisraeli kuningas!”
\par 14 Aga Jeesus leidis noore eesli ja istus tema selga, nõnda nagu on kirjutatud:
\par 15 „Ära karda, Siioni tütar! Vaata, sinu kuningas tuleb ning istub eeslisälu seljas!”
\par 16 Sellest ei saanud jüngrid esialgu aru; aga kui Jeesus oli austatud, siis meenus neile, et see oli kirjutatud tema kohta ja et nad temale seda olid teinud.
\par 17 Nii andis temast tunnistust rahvas, kes oli tema juures, kui ta Laatsaruse välja hüüdis hauast ja tema üles äratas surnuist.
\par 18 Sellepärast ka rahvas läks temale vastu, sest nad kuulsid teda selle tunnustähe teinud olevat.
\par 19 Siis variserid rääkisid isekeskis: „Eks te näe, et te ei saa midagi teha? Vaata, maailm jookseb tema järele!”

\section*{Kreeklased tahavad Jeesust näha}

\par 20 Aga nende seas, kes olid tulnud pühiks palvetama, leidusid mõned kreeklased.
\par 21 Need tulid nüüd Filippuse juurde, kes oli Betsaidast, Galilea linnast, ja palusid teda, öeldes: „Isand, me tahame Jeesust näha!”
\par 22 Filippus tuleb ja ütleb seda Andreasele, ja Andreas ja Filippus tulevad ning ütlevad seda Jeesusele.
\par 23 Aga Jeesus vastab neile ning ütleb: „Tund on tulnud, et Inimese Poega austataks!
\par 24 Tõesti, tõesti ma ütlen teile: kui nisuiva ei kuku mullasse ega sure, jääb ta üksi; aga kui ta sureb, siis ta kannab palju vilja!
\par 25 Kes oma elu armastab, see kaotab selle, ja kes oma elu vihkab siin maailmas, see hoiab selle igaveseks eluks.
\par 26 Kui keegi mind teenib, see järgigu mind, ja kus mina olen, seal peab olema ka minu teenija; ja kes mind teenib, seda tahab Isa austada.
\par 27 Nüüd on mu hing ehmunud! Ja mis ma pean ütlema? Isa, päästa mind sellest tunnist! Aga sellepärast ma olen tulnud sellesse tundi!
\par 28 Isa, austa oma nime!„ Siis tuli hääl taevast: ”Mina olen juba austanud ja austan veel!”
\par 29 Siis rahvahulk, kes seal seisis ja seda kuulis, ütles pikse olevat müristanud. Aga teised ütlesid: „Ingel rääkis temaga!”
\par 30 Jeesus kostis ning ütles: „See hääl ei sündinud minu pärast, vaid teie pärast.
\par 31 Nüüd käib kohus üle selle maailma; nüüd tõugatakse välja selle maailma vürst.
\par 32 Ja mina tahan, kui mind maast üles tõstetakse, kõik tõmmata enese juurde!”
\par 33 Aga seda ta ütles tähendades, mis surma ta pidi surema.
\par 34 Siis vastas rahvas temale: „Me oleme kuulnud käsuõpetusest, et Kristus jääb igavesti. Kuidas siis sina ütled, et Inimese Poeg ülendatakse? Kes on see Inimese Poeg?”
\par 35 Jeesus ütles nüüd neile: „Valgus on veel üürikeseks ajaks teie keskel; käige niikaua kui teil valgust on, et pimedus teid ei tabaks; sest kes käib pimeduses, ei tea, kuhu ta läheb.
\par 36 Uskuge valgusesse, niikaua kui teil valgus on, et te saaksite valguse lasteks!” Seda rääkis Jeesus ja läks ning peitis enese ära nende eest.

\section*{Juutide uskmatus ja Jeesuse salajased jüngrid}

\par 37 Ja ehk ta küll palju tunnustähti oli teinud nende nähes, ei uskunud nad siiski temasse,
\par 38 et läheks täide prohvet Jesaja sõna, mis ta ütles: „Issand, kes usub meie kuulutust ja kellele on ilmutatud Issanda käsivars?”
\par 39 Sellepärast nad ei võinud uskuda, et Jesaja oli veel öelnud:
\par 40 „Tema on teinud nende silmad pimedaks ja nende südamed kõvaks, et nad silmadega ei näeks ja südamega ei mõistaks ega pöörduks, et mina neid parandaksin!”
\par 41 Seda ütles Jesaja, kui ta nägi Jeesuse auhiilgust ja rääkis temast.
\par 42 Siiski uskus ka palju ülemaid temasse; kuid variseride pärast nad ei tunnistanud seda, et neid ei lükataks kogudusest välja.
\par 43 Sest nad armastasid austust inimestelt rohkem kui austust Jumalalt.

\section*{Jeesuse kuulutuse kokkuvõtlik sisu}

\par 44 Aga Jeesus hüüdis ning ütles: „Kes minusse usub, see ei usu mitte minusse, vaid sellesse, kes mind on läkitanud!
\par 45 Ja kes näeb mind, see näeb seda, kes mind on läkitanud.
\par 46 Mina olen tulnud valguseks maailma, et ükski, kes usub minusse ei jääks pimedusse.
\par 47 Ja kui keegi kuuleb minu sõnu ega pea neid, selle üle ei mõista mina kohut; sest mina ei ole tulnud kohut mõistma maailma üle, vaid maailma õndsaks tegema.
\par 48 Kes mind põlgab ega võta vastu minu sõna, sellel on oma kohtumõistja. Sõna, mis ma olen rääkinud, see mõistab tema üle kohut viimsel päeval.
\par 49 Sest ma ei ole rääkinud iseenesest, vaid Isa, kes mind on läkitanud, on mulle andnud käsu, mida ma pean ütlema ja mida ma pean rääkima.
\par 50 Ja ma tean, et tema käsk on igavene elu. Mida ma nüüd räägin, seda ma räägin nõnda, kuidas Isa mulle on öelnud!”


\chapter{13}

\section*{Jeesus peseb jüngrite jalgu}

\par 1 Aga enne paasapühi, kui Jeesus teadis tunni olevat tulnud, et ta siit maailmast läheb Isa juurde, armastas ta neid otsani, nõnda nagu ta oli armastanud omi, kes olid maailmas.
\par 2 Ja õhtusöömaajal olles, kui kurat juba oli Juuda, Siimona poja Iskarioti südamesse pannud, et ta tema ära annaks,
\par 3 ja kui Jeesus teadis, et Isa kõik tema kätte oli andnud ja et tema oli lähtunud Jumalast ja läheb Jumala juurde,
\par 4 tõuseb ta üles õhtusöömaajalt ja võtab oma kuue seljast ja võtab rätiku ja vöötub sellega.
\par 5 Pärast seda ta valab vett vaagnasse ja hakkab pesema jüngrite jalgu ja kuivatama rätikuga, millega ta oli vöötatud.
\par 6 Siis ta tuleb Siimon Peetruse juurde, ja see ütleb temale: „Issand, kas sina mu jalgu pesed?”
\par 7 Jeesus vastas ning ütles temale: „Mida mina teen, seda sina nüüd ei tea, aga pärast seda sa pead teada saama!”
\par 8 Peetrus ütleb temale: „Eladeski ei või sina minu jalgu pesta!” Jeesus vastas temale: „Kui ma sind ei pese, siis ei ole sul osa minuga!”
\par 9 Siimon Peetrus ütleb temale: „Issand, mitte üksnes minu jalgu, vaid ka käed ja pea!”
\par 10 Jeesus ütleb temale: „Kes on pestud, sellel ei ole vaja muud kui jalgu pesta, sest ta on üldse puhas; ja teie olete puhtad, kuid mitte kõik!”
\par 11 Sest ta teadis, kes tema ära annab; sellepärast ta ütles: „Teie ei ole kõik puhtad!”
\par 12 Kui ta nüüd nende jalad oli pesnud ja oma kuue oli võtnud, istus ta jälle maha ja ütles neile: „Kas teate, mis ma teile olen teinud?
\par 13 Te hüüate mind Õpetajaks ja Issandaks ja ütlete õigesti, sest mina olen see.
\par 14 Kui nüüd mina, Issand ja õpetaja, teie jalad olen pesnud, siis peate teiegi üksteise jalgu pesema.
\par 15 Sest ma olen teile andnud eeskuju, et ka teie nõnda teeksite, nagu mina teile olen teinud.
\par 16 Tõesti, tõesti ma ütlen teile: ei ole ori suurem kui tema isand ega käskjalg suurem kui tema läkitaja!
\par 17 Kui te seda teate, õndsad olete, kui te seda teete!
\par 18 Ma ei ütle seda teie kõikide kohta; ma tean, keda ma olen ära valinud. Aga ma ütlen seda, et Kiri täide läheks: kes minu leiba sööb, on tõstnud oma kanna minu vastu!
\par 19 Nüüd ütlen ma teile seda, enne kui see sünnib, et kui see sünnib, te usuksite, et mina see olen.
\par 20 Tõesti, tõesti ma ütlen teile, et kes vastu võtab selle, kelle ma läkitan, see võtab mind vastu; aga kes mind vastu võtab, see võtab vastu tema, kes mind on läkitanud!”

\section*{Jeesus vihjab äraandjale}

\par 21 Kui Jeesus seda oli öelnud, võpatas ta oma vaimus ja tunnistas ning ütles: „Tõesti, tõesti ma ütlen teile, et üks teie seast annab mind ära!”
\par 22 Siis vaatasid jüngrid üksteisele otsa ja olid kahevahel, kellest ta räägib.
\par 23 Üks tema jüngritest, see, keda Jeesus armastas, oli lauas istumas Jeesuse rinna najal.
\par 24 Selle poole nüüd Siimon Peetrus noogutas peaga ning ütles temale: „Ütle, kes see on, kellest ta räägib?”
\par 25 See laskus siis Jeesuse rinnale ja ütles temale: „Issand, kes see on?”
\par 26 Jeesus vastas: „See on see, kellele mina kastutan ja annan selle palukese!” Ja tema kastis palukese ning võttis ja andis selle Juudale, Siimona pojale Iskariotile.
\par 27 Ja palukese järel läks saatan temasse. Siis Jeesus ütles temale: „Mis sa teed, seda tee kähku!”
\par 28 Aga sellest ei saanud aru ükski neist, kes lauas istusid, miks ta seda temale ütles.
\par 29 Sest mõned mõtlesid, et Jeesus, kuna Juuda käes seisis kukkur, temale ütleb: „Osta, mis meile pühiks vaja on!” Või et ta midagi vaestele annaks.
\par 30 Kui ta nüüd palukese oli võtnud, läks ta sedamaid välja. Aga oli öö.

\section*{Uus armastusekäsk}

\par 31 Kui ta nüüd oli väljunud, ütles Jeesus: „Nüüd on Inimese Poeg austatud ja Jumal on austatud temas!
\par 32 Ons Jumal temas austatud, siis austab teda ka Jumal eneses ja austab teda varsti!
\par 33 Lapsukesed, ma olen veel üürikese aja teie juures; te hakkate mind otsima. Ja nagu ma juutidele ütlesin: kuhu mina lähen, sinna teie ei või tulla! nõnda ma ütlen nüüd ka teile.
\par 34 Uue käsusõna ma annan teile, et te üksteist peate armastama, nõnda nagu mina teid olen armastanud; et teiegi üksteist armastaksite!
\par 35 Sellest tunnevad kõik, et teie olete minu jüngrid, kui teil on armastus isekeskis!”

\section*{Jeesus kuulutab ette, et Peetrus teda salgab}

\par 36 Siimon Peetrus ütleb temale: „Issand, kuhu sa lähed?” Jeesus vastas: „Kuhu ma lähen, sinna sa ei või mind nüüd järgida, aga edaspidi sa järgid mind!”
\par 37 Peetrus ütleb temale: „Issand, miks ma nüüd ei või sind järgida? Ma annan oma elu sinu eest!”
\par 38 Jeesus vastab temale: „Annad sina oma elu minu eest? Tõesti, tõesti ma ütlen sulle, et kukk ei laula mitte enne, kui sa mind juba kolm korda oled salanud!


\chapter{14}

\section*{Tee Isa juurde}

\par 1 Teie süda ärgu ehmugu! Uskuge Jumalasse ja uskuge minusse.
\par 2 Minu Isa majas on palju eluasemeid. Kui see nii ei oleks, kas ma oleksin teile öelnud: ma lähen teile aset valmistama?
\par 3 Ja kui ma olen läinud ja teile aseme valmistanud, tulen ma jälle tagasi ja võtan teid enese juurde, et teiegi oleksite, kus mina olen.
\par 4 Ja kuhu mina lähen, seda teed te teate!”
\par 5 Toomas ütleb temale: „Issand, me ei tea, kuhu sa lähed! Kuidas võime teada teed?”
\par 6 Jeesus ütleb temale: „Mina olen tee ja tõde ja elu, ükski ei saa Isa juurde muidu kui minu kaudu!
\par 7 Kui te mind oleksite ära tundnud, siis te tunneksite ka minu Isa, ja sellest ajast te tunnete teda ja olete teda näinud!”
\par 8 Filippus ütleb temale: „Issand, näita meile Isa, siis me jääme rahule!”
\par 9 Jeesus ütleb temale: „Niikaua aega olen ma teie juures ja sa ei ole mind tundnud, Filippus? Kes mind on näinud, see on näinud Isa. Kuidas sa ütled: näita meile Isa!?
\par 10 Kas sa ei usu, et mina olen Isas ja Isa on minus? Neid sõnu, mida ma räägin, ei räägi ma iseenesest, vaid Isa, kes asub minus, teeb oma tegusid.
\par 11 Uskuge mind, et mina olen Isas ja et Isa on minus; aga kui mitte, siis uskuge nende tegude pärast.
\par 12 Tõesti, tõesti ma ütlen teile, et kes usub minusse, see teeb ka neid tegusid, mida mina teen, ja teeb veel suuremaid kui need on, sest mina lähen Isa juurde!
\par 13 Ja mida te iganes palute minu nimel, seda ma teen, et Isa austataks Pojas!
\par 14 Kui te midagi minult palute minu nimel, siis ma teen seda.

\section*{Püha Vaimu tuleku tõotus}

\par 15 Kui te mind armastate, siis pidage minu käsusõnu!
\par 16 Ja mina palun Isa, ja tema annab teile teise Trööstija, et see teie juurde jääks igavesti,
\par 17 tõe Vaimu, keda maailm ei või vastu võtta, sellepärast et ta teda ei näe ega tunne; aga teie tunnete teda, sest tema jääb teie juurde ja tahab olla teie sees.
\par 18 Ma ei jäta teid vaestekslasteks; ma tulen teie juurde.
\par 19 Veel on pisut aega, siis maailm ei näe mind enam, aga teie näete mind; sest mina elan ja teie saate elama!
\par 20 Sel päeval te saate tunda, et mina olen oma Isas ja teie minus ja mina teis.
\par 21 Kellel on minu käsusõnad ja kes neid peab, see on, kes mind armastab; kes aga mind armastab, seda armastab minu Isa ja mina tahan teda armastada ja iseennast temale ilmutada!”
\par 22 Temale ütleb Juudas, mitte Iskariot: „Issand, millest see oleneb, et sa meile tahad ennast ilmutada, aga mitte maailmale?”
\par 23 Jeesus vastas ning ütles temale: „Kui keegi mind armastab, küll ta peab minu sõna, ja minu Isa armastab teda, ja me tuleme tema juurde ja teeme eluaseme tema juurde.
\par 24 Kes mind ei armasta, see ei pea minu sõnu; ja sõna, mida te kuulete, ei ole minu, vaid on Isa sõna, kes mind on läkitanud.
\par 25 Seda olen mina teile rääkinud teie juures viibides.
\par 26 Aga Trööstija, Püha Vaim, kelle minu Isa läkitab minu nimel, see õpetab teile kõik ja tuletab teile meelde kõik, mis mina olen öelnud.
\par 27 Rahu ma jätan teile; oma rahu ma annan teile; mina ei anna teile nõnda nagu maailm annab. Teie süda ärgu ehmugu ja ärgu mingu araks!
\par 28 Te kuulsite mind teile ütlevat: ma lähen ära ja tulen taas teie juurde. Kui te mind armastaksite, siis te rõõmutseksite, et ma lähen Isa juurde; sest minu Isa on suurem minust!
\par 29 Ja nüüd ma olen teile seda öelnud, enne kui see sünnib, et te usuksite, kui see sünnib.
\par 30 Ma ei räägi enam palju teiega, sest selle maailma vürst tuleb ja ta ei saa minust midagi.
\par 31 Aga et maailm saaks aru, et ma armastan Isa ja teen nõnda, kuidas Isa mind on käskinud. Tõuske, mingem siit ära!”


\chapter{15}

\section*{Viinapuu ja oksad}

\par 1 „Mina olen tõeline viinapuu ja minu Isa on viinamäe aednik.
\par 2 Iga oksa minu küljes, mis ei kanna vilja, tema kõrvaldab ja igaüht, mis kannab vilja, tema puhastab, et see kannaks rohkem vilja.
\par 3 Teie olete juba puhtad selle sõna pärast, mis mina teile olen rääkinud!
\par 4 Jääge minusse ja mina jään teisse. Nagu oks ei või vilja kanda iseenesest, kui ta ei jää viinapuu külge, nõnda ka teie, kui te ei jää minusse.
\par 5 Mina olen viinapuu, teie olete oksad; kes jääb minusse ja mina temasse, see kannab palju vilja; sest ilma minuta ei või te midagi teha!
\par 6 Kui keegi ei jää minusse, siis ta heidetakse välja nagu viinapuu oks ning kuivab ära. Ja nad kogutakse kokku ja heidetakse tulle ning põletatakse ära.
\par 7 Kui te jääte minusse ja minu sõnad jäävad teisse, siis paluge, mis te iganes tahate, ja see sünnib teile!
\par 8 Selles on minu Isa austatud, et te kannate palju vilja ja osutute minu jüngriteks.
\par 9 Otsekui minu Isa on armastanud mind, nõnda olen ka mina teid armastanud; jääge minu armastusse.
\par 10 Kui te peate minu käsusõnu, siis te jääte minu armastusse, nõnda nagu mina olen pidanud oma Isa käsusõnu ja jään tema armastusse.
\par 11 Seda ma olen teile rääkinud, et minu rõõm oleks teie sees ja teie rõõm saaks täielikuks.
\par 12 See on minu käsusõna, et te armastaksite üksteist nõnda nagu mina teid olen armastanud!
\par 13 Suuremat armastust ei ole kellelgi kui see, et ta jätab oma elu oma sõprade eest!
\par 14 Teie olete mu sõbrad, kui te teete, mida mina teid käsin.
\par 15 Ma ei ütle teid enam orjad olevat; sest ori ei tea, mida ta isand teeb. Vaid ma olen teid nimetanud sõpradeks, sest ma olen teile teada andnud kõik, mis ma oma Isalt olen kuulnud.
\par 16 Mitte teie ei ole mind valinud, vaid mina olen teid valinud ja olen teid seadnud, et te läheksite ja kannaksite vilja ja et teie vili jääks ja Isa teile annaks, mida te iganes palute minu nimel.

\section*{Maailm ja tõe Vaim}

\par 17 Seda ma käsin teid, et te armastaksite üksteist!
\par 18 Kui maailm teid vihkab, siis teadke, et ta mind on enne teid vihanud.
\par 19 Kui te oleksite maailmast, siis maailm armastaks oma. Aga et te ei ole maailmast, vaid mina olen teid ära valinud maailmast, sellepärast vihkab teid maailm!
\par 20 Pidage meeles seda sõna, mis ma teile olen öelnud: ei ole ori suurem oma isandast! On nemad mind taga kiusanud, siis nad kiusavad teidki taga. On nad pidanud minu sõna, siis nad peavad teiegi sõna.
\par 21 Aga seda kõike nad teevad teile minu nime pärast, sest nad ei tunne teda, kes mind on läkitanud.
\par 22 Kui ma ei oleks tulnud ega oleks neile rääkinud, siis ei oleks neil pattu. Aga nüüd ei ole neil oma pattu millegagi vabandada.
\par 23 Kes vihkab mind, see vihkab mu Isa!
\par 24 Kui ma ei oleks nende keskel teinud neid tegusid, mida ükski muu ei ole teinud, ei oleks neil pattu; aga nüüd on nad näinud ja vihanud niihästi mind kui mu Isa.
\par 25 Aga see on sündinud, et läheks täide sõna, mis on kirjutatud nende käsuõpetuses: nad on mind vihanud ilmaasjata!
\par 26 Aga kui Trööstija tuleb, kelle mina teile läkitan Isalt, tõe Vaim, kes lähtub Isast, see tunnistab minust.
\par 27 Ja teiegi tunnistate, sest te olete algusest peale olnud ühes minuga!


\chapter{16}

\section*{Maailm ja tõe Vaim}

\par 1 Seda ma olen teile rääkinud, et te ei taganeks.
\par 2 Nad heidavad teid kogudusest välja ja tuleb aeg, mil igaüks, kes teid tapab, arvab au andvat Jumalale.
\par 3 Ja seda nad teevad, sellepärast et nad ei tunne Isa ega mind.
\par 4 Aga seda ma olen teile rääkinud, et kui tuleb see aeg, te mõtleksite sellele, et mina teile seda olen öelnud. Ent seda ma ei ole teile öelnud algusest, sest ma olin ühes teiega.
\par 5 Aga nüüd ma lähen selle juurde, kes mind on läkitanud, ja keegi teie seast ei küsi minult: kuhu sa lähed?
\par 6 Ainult et mina seda teile olen öelnud, siis on kurbus täitnud teie südame.
\par 7 Aga mina ütlen teile tõtt: see tuleb teile heaks, et mina lähen ära. Sest kui ma ei läheks, ei tuleks Trööstija teie juurde. Aga kui ma lähen, siis ma läkitan tema teie juurde.
\par 8 Ja kui ta tuleb, siis ta toob maailmale selguse patu kohta ja õiguse kohta ja kohtu kohta:
\par 9 patu kohta, et nad ei usu minusse;
\par 10 õiguse kohta, et ma lähen Isa juurde ja te ei näe mind enam;
\par 11 ja kohtu kohta, et selle maailma vürsti üle on kohus mõistetud.
\par 12 Mul on teile veel palju ütlemist, aga te ei või nüüd seda kanda.
\par 13 Aga kui tema, tõe Vaim, tuleb, siis ta juhatab teid kõigesse tõesse, sest tema ei räägi iseenesest, vaid mida ta kuuleb, seda ta räägib ja tulevasi asju ta kuulutab teile.
\par 14 Tema austab mind, sest minu omast ta võtab ja kuulutab teile.
\par 15 Kõik, mis Isal on, see on minu päralt; sellepärast ma ütlesin, et ta võtab minu omast ja kuulutab teile.

\section*{Jeesuse lahkumiskõne}

\par 16 Üürikese aja pärast, siis te ei näe mind, ja taas üürikese aja pärast, siis te näete mind!”
\par 17 Siis ütlesid mõned tema jüngritest isekeskis: „Mis see on, mis ta ütleb meile: üürikese aja pärast, siis te ei näe mind, ja taas üürikese aja pärast, siis te näete mind! ja: mina lähen Isa juurde?”
\par 18 Nimelt nad küsisid: „Mis see on, mis ta ütleb: üürikese aja pärast? Meie ei tea, mida ta räägib.”
\par 19 Siis Jeesus sai aru, et nad temalt tahtsid küsida, ja ütles neile: „Seda te küsitlete isekeskis, et ma ütlesin: üürikese aja pärast, siis te ei näe mind, ja taas üürikese aja pärast te näete mind!
\par 20 Tõesti, tõesti ma ütlen teile, te nutate ja kaeblete, aga maailm on rõõmus! Teid kurvastatakse, aga teie kurvastus muutub rõõmuks!
\par 21 Kui naine sünnitab, on ta murelik, sest tema hetk on tulnud; aga kui ta lapse on saanud, ei mõtle ta enam ahastusele rõõmu pärast, et inimene on sündinud maailma.
\par 22 Ja teilgi on nüüd meel murelik, aga ma tahan teid jälle näha ja teie süda rõõmustub, ja ükski ei võta teie rõõmu teilt ära.
\par 23 Ja sel päeval te ei küsi minult midagi. Tõesti, tõesti ma ütlen teile, et mida te iganes Isalt palute, seda ta annab teile minu nimel!
\par 24 Tänini ei ole te midagi palunud minu nimel. Paluge, siis te saate, et teie rõõm võiks olla täielik.
\par 25 Seda ma olen teile rääkinud võrdumitega; aga tuleb aeg, mil ma teile enam ei räägi võrdumitega, vaid kuulutan Isast teile avalikult.
\par 26 Sel päeval te palute minu nimel. Ja ma ei ütle teile, et ma palun Isa teie eest.
\par 27 Sest Isa ise armastab teid, sellepärast et teie olete mind armastanud ja uskunud, et mina olen lähtunud Jumalast.
\par 28 Ma olen lähtunud Isast ja tulnud maailma; taas ma jätan maha maailma ja lähen Isa juurde!”
\par 29 Tema jüngrid ütlesid: „Vaata, nüüd sa ütled lausa ega räägi võrdumitega!
\par 30 Nüüd me teame, et sa tead kõik ja et sul ei ole vaja, et keegi sinult küsib. Sellest me usume, et sa oled lähtunud Jumala juurest!”
\par 31 Jeesus vastas neile: „Kas te nüüd usute?
\par 32 Vaata, tund tuleb ja on juba tulnud, et teid hajutatakse igaüks ise kohta ja te jätate mind üksi! Ja mina ei ole ometi mitte üksi, sest Isa on minuga.
\par 33 Seda ma olen teile rääkinud, et teil oleks rahu minus. Maailmas on teil ahastust, aga olge julged, mina olen maailma ära võitnud!”


\chapter{17}

\section*{Jeesuse palve}

\par 1 Seda rääkis Jeesus ja tõstis oma silmad taeva poole ning ütles: „Isa, tund on tulnud! Austa oma Poega, et ka Poeg austaks sind,
\par 2 nagu sa oled temale andnud meelevalla kõige liha üle, et ta annaks igavese elu kõigile, keda sina oled temale andnud!
\par 3 Aga see on igavene elu, et nad tunneksid sind, ainust tõelist Jumalat ja Jeesust Kristust, kelle sina oled läkitanud.
\par 4 Mina olen sind austanud maa peal; ma olen lõpetanud selle töö, mille sa oled andnud mulle teha.
\par 5 Ja nüüd austa mind sina, Isa, enese juures selle auga, mis mul oli sinu juures enne maailma olemasolu.
\par 6 Mina olen sinu nime ilmutanud inimestele, keda sa mulle oled andnud maailmast. Nemad olid sinu omad ja sina andsid nad mulle, ja nad on pidanud sinu sõna.
\par 7 Nüüd nad teavad, et kõik, mis sina mulle oled andnud, on sinult.
\par 8 Sest need sõnad, mis sa andsid mulle, olen ma andnud neile, ja nemad on vastu võtnud ja teavad, et ma tõesti olen lähtunud sinu juurest, ja nemad on uskunud, et sina oled mind läkitanud.
\par 9 Mina palun nende eest; maailma eest ma ei palu, vaid nende eest, keda sina oled mulle andnud, sest nad on sinu omad.
\par 10 Ja kõik, mis on minu oma, on sinu, ja mis on sinu oma, on minu, ja nendes ma olen austatud.
\par 11 Ja ma ei ole edaspidi enam maailmas, aga nemad on maailmas, ja mina tulen sinu juurde. Püha Isa, hoia neid oma nimes, mille sa oled mulle andnud, et nad oleksid üks, nõnda nagu meie!
\par 12 Kui ma olin ühes nendega, hoidsin ma neid sinu nimes, mille sa oled mulle andnud; ja ma hoidsin neid ning ükski neist ei ole kadunud, kui ainult kadumisepoeg, et Kiri täide läheks.
\par 13 Aga nüüd ma tulen sinu juurde ja räägin seda maailmas, et minu rõõm neil oleks täieline neis enestes.
\par 14 Mina olen neile andnud sinu sõna ja maailm vihkab neid; sest nad ei ole maailmast, nõnda nagu mina ei ole maailmast.
\par 15 Mina ei palu, et sa võtaksid nad ära maailmast, vaid et sa neid hoiaksid tigeda eest.
\par 16 Nad ei ole maailmast, nõnda nagu minagi ei ole maailmast.
\par 17 Pühitse neid tões: sinu sõna on tõde!
\par 18 Otsekui sina oled mind läkitanud maailma, nõnda minagi olen nad läkitanud maailma.
\par 19 Ja mina pühitsen iseennast nende eest, et nemadki oleksid pühitsetud tõe sees.
\par 20 Aga ma ei palu mitte üksnes nende eest, vaid ka nende eest, kes nende sõna kaudu usuvad minusse,
\par 21 et nemad kõik oleksid üks, nõnda nagu sina, Isa, minus ja mina sinus, et nemadki meis oleksid ja maailm usuks, et sa mind oled läkitanud.
\par 22 Selle au, mille sa andsid mulle, olen mina andnud neile, et nad oleksid üks, nõnda nagu meie oleme üks:
\par 23 mina nendes ja sina minus, et nad täielikult saaksid üheks ja maailm tunneks, et sina oled mind läkitanud ja oled armastanud neid, nõnda nagu sa mind oled armastanud.
\par 24 Isa, ma tahan, et kus mina olen, ka nemad oleksid minu juures, keda sa mulle oled andnud; et nad näeksid minu auhiilgust, mille sa mulle oled andnud; sest sa oled mind armastanud enne maailma rajamist.
\par 25 Õige Isa, maailm ei ole sind tundnud; aga mina tunnen sind ja need on hakanud tundma, et sina oled mind läkitanud.
\par 26 Ja ma olen neile andnud teada sinu nime ja annan seda teada selleks, et armastus, millega sa mind oled armastanud, oleks nendes ja mina oleksin nendes!”


\chapter{18}

\section*{Jeesuse vangistamine}

\par 1 Kui Jeesus seda oli rääkinud, läks ta oma jüngritega välja üle Kiidroni jõe. Seal oli aed, kuhu ta läks oma jüngritega.
\par 2 Aga Juudas, kes tema ära andis, teadis ka seda paika, sest et Jeesus sagedasti sinna oli kokku tulnud oma jüngritega.
\par 3 Juudas võttis nüüd enesega kaasa väesalga ning ülempreestritelt ja variseridelt sulaseid ja tuli sinna tulelontide, lampide ja relvadega.
\par 4 Siis Jeesus, kuna ta teadis kõik, mis tema peale tuleb, läks välja ja ütles neile: „Keda te otsite?”
\par 5 Nad vastasid temale: „Jeesust Naatsaretlast!” Tema ütles neile: „Mina olen see!” Ka Juudas, kes tema ära andis, seisis nende juures.
\par 6 Kui ta nüüd neile ütles: „Mina olen see!” taganesid nad ja langesid maha.
\par 7 Siis ta küsis taas neilt: „Keda te otsite?” Nemad ütlesid: „Jeesust Naatsaretlast!”
\par 8 Jeesus vastas: „Ma olen teile öelnud, et mina see olen! Kui te nüüd mind otsite, siis laske need ära minna!” -
\par 9 et täide läheks sõna, mis ta oli öelnud: mina ei ole kaotanud kedagi neist, keda sa mulle oled andnud!
\par 10 Siis Siimon Peetrus, kellel oli mõõk, tõmbas selle välja ja lõi ülempreestri sulast ning raius tema parema kõrva ära, sulase nimi aga oli Malkus.
\par 11 Siis Jeesus ütles Peetrusele: „Pista oma mõõk tuppe! Eks mina pea jooma seda karikat, mille Isa mulle on andnud?”
\par 12 Aga väesalk ja ülempealik ja juutide sulased võtsid Jeesuse ja sidusid ta kinni
\par 13 ja viisid esiti Annase juurde; sest see oli Kaifase äi, ja Kaifas oli tol aastal ülempreester.
\par 14 Kaifas aga oli see, kes juutidele nõu andes oli öelnud: „Parem on, et üks inimene sureb rahva eest!”

\section*{Peetrus salgab Jeesuse. Ülempreester küsitleb Jeesust}

\par 15 Aga Siimon Peetrus järgis Jeesust ja üks teine jünger. See jünger oli tuttav ülempreestrile ja läks Jeesusega ühes ülempreestri õue.
\par 16 Ja Peetrus seisis väljas ukse ees. Siis läks teine jünger, kes oli tuttav ülempreestrile, välja ja rääkis uksehoidjaga ning viis Peetruse sisse.
\par 17 Siis ütles uksehoidjatüdruk Peetrusele: „Eks sinagi ole üks selle inimese jüngritest?” Tema ütles: „Mina ei ole!”
\par 18 Aga teenrid ja sulased seisid ja olid teinud sütetule, sest oli külm, ja soojendasid endid. Ja Peetrus seisis nende juures ning soojendas ennast.
\par 19 Ülempreester küsis nüüd Jeesuselt tema jüngrite ja tema õpetuse kohta.
\par 20 Jeesus kostis temale: „Mina olen avalikult rääkinud maailmale; mina olen alati õpetanud koguduse- ja pühakojas, kuhu kõik juudid kokku tulevad, ja salaja ma ei ole midagi rääkinud.
\par 21 Miks sa minult küsid? Küsi nendelt, kes on kuulnud, mida ma neile olen rääkinud. Vaata, need teavad, mis ma olen rääkinud!”
\par 22 Aga kui ta seda oli öelnud, lõi üks sulaseist, kes seisid seal juures, Jeesust kõrva äärde ning ütles: „Kas sa nõnda vastad ülempreestrile?”
\par 23 Jeesus kostis temale: „Kui ma olen rääkinud pahasti, siis tõesta, et see on paha; aga kui ma olen hästi rääkinud, miks sa mind siis lööd?”
\par 24 Siis Annas saatis tema seotult ülempreester Kaifase juurde.
\par 25 Aga Siimon Peetrus seisis ja soojendas ennast. Siis ütlesid nad temale: „Eks sinagi ole tema jüngrite seast?” Tema salgas ja ütles: „Mina ei ole!”
\par 26 Üks ülempreestri teenreist, selle sugulane, kelle kõrva Peetrus oli ära raiunud, ütleb: „Kas mina sind ei näinud ühes temaga aias?”
\par 27 Siis Peetrus salgas jälle, ja sedamaid laulis kukk.

\section*{Jeesus Pilaatuse ees}

\par 28 Siis nad viisid Jeesuse Kaifase juurest kohtukotta; aga see oli puhteajal. Ja nad ise ei läinud kohtukotta, et nad ei rüvetuks, vaid võiksid süüa paasatalle.
\par 29 Siis läks Pilaatus välja nende juurde ja ütles: „Mis teil on kaebamist selle inimese peale?”
\par 30 Nemad vastasid ning ütlesid temale: „Kui see ei oleks kurjategija, me ei oleks andnud teda sinu kätte!”
\par 31 Siis ütles Pilaatus neile: „Võtke teie ta ja mõistke oma käsuõpetuse järgi kohut tema üle!” Juudid ütlesid temale: „Meil ei ole luba kedagi tappa!” -
\par 32 et täide läheks Jeesuse sõna, mis ta oli öelnud, kui ta tähendas, millist surma ta sureb.
\par 33 Siis läks Pilaatus jälle kohtukotta ja kutsus Jeesuse ning ütles temale: „Oled sina juutide kuningas?”
\par 34 Jeesus vastas: „Räägid sa seda iseenesest või on sulle teised seda minust öelnud?”
\par 35 Pilaatus vastas: „Egas ma juut ole! Sinu oma rahvas ja ülempreestrid on sind andnud minu kätte. Mis sa oled teinud?”
\par 36 Jeesus kostis: „Minu riik ei ole sellest maailmast; oleks mu riik sellest maailmast, küll mu sulased oleksid võidelnud, et mind ei oleks antud juutide kätte. Ent nüüd ei ole mu riik mitte siit!”
\par 37 Siis ütles Pilaatus temale: „Siis oled sina ometi kuningas?” Jeesus vastas: „Jah, olen, sest mina olen kuningas! Ma olen selleks sündinud ja selleks tulnud maailma, et ma tõele tunnistust annaksin. Kes tõe seest on, see kuuleb minu häält!”
\par 38 Pilaatus ütleb temale: „Mis on tõde?” Kui ta seda oli öelnud, läks ta uuesti välja juutide juurde ja ütles neile: „Mina ei leia temast ühtki süüd!
\par 39 Aga teil on kombeks, et ma paasapühal lasen ühe teile vabaks. Kas tahate nüüd, et ma teile vabaks lasen juutide kuninga?”
\par 40 Siis nad kisendasid jälle kõik ning ütlesid: „Mitte teda, vaid Barabas!” Aga Barabas oli röövel.


\chapter{19}

\section*{Pilaatus täidab juutide nõude}

\par 1 Siis Pilaatus võttis Jeesuse ja laskis teda rooskadega peksta.
\par 2 Ja sõjamehed punusid kibuvitstest krooni ja panid selle temale pähe ja riietasid teda purpurkuuega
\par 3 ja tulid tema juurde ja ütlesid: „Tere, juutide kuningas!” Ja nad lõid teda kõrva äärde.
\par 4 Siis Pilaatus läks taas välja ja ütles neile: „Vaata, ma toon ta teile välja, et te aru saaksite, et ma ei leia temast ühtki süüd!”
\par 5 Siis tuli Jeesus välja ja kandis kibuvitsakrooni ja purpurkuube. Ja Pilaatus ütles neile: „Ennäe inimest!”
\par 6 Kui nüüd ülempreestrid ja sulased teda nägid, kisendasid nad ning ütlesid: „Löö risti! Löö risti!” Pilaatus ütles neile: „Võtke teie tema ja lööge ta risti, sest mina ei leia temast süüd!”
\par 7 Juudid vastasid temale: „Meil on käsuõpetus, ja käsu järgi ta peab surema, sest ta on enese teinud Jumala Pojaks!”
\par 8 Kui nüüd Pilaatus seda sõna kuulis, kartis ta veel enam.
\par 9 Ja ta läks jälle kohtukotta ja ütleb Jeesusele: „Kust sa oled?” Aga Jeesus ei andnud temale vastust.
\par 10 Siis ütleb Pilaatus temale: „Kas sa ei räägi minuga? Eks sa tea, et mul on meelevald sind vabaks lasta ja meelevald sind risti lüüa?”
\par 11 Jeesus kostis: „Sinul ei oleks mingit meelevalda minu üle, kui see sulle ei oleks antud ülalt; sellepärast on sellel, kes mind sinu kätte andis, suurem patt!”
\par 12 Sellest alates püüdis Pilaatus teda vabaks lasta. Aga juudid kisendasid ning ütlesid: „Kui sa selle vabaks lased, siis sa ei ole keisri sõber! Igaüks, kes iseenese teeb kuningaks, on keisri vastane!”
\par 13 Kui nüüd Pilaatus neid sõnu kuulis, viis ta Jeesuse välja ja istus maha kohtujärjele sinna paika, mida hüütakse Kivipõrandaks, aga heebrea keeli Gabbataks.
\par 14 Aga oli paasapüha valmistuspäev, arvata kuues tund. Ja tema ütles juutidele: „Ennäe teie kuningat!”
\par 15 Aga nad kisendasid: „Vii ära, vii ära, löö ta risti!” Pilaatus ütleb neile: „Kas ma pean teie kuninga risti lööma?” Ülempreestrid vastasid: „Meil ei ole kuningat, vaid on keiser!”
\par 16 Siis ta andis tema nende kätte risti lüüa. Ja nad võtsid Jeesuse.

\section*{Jeesuse ristilöömine}

\par 17 Ja tema kandis ise oma risti ning väljus nõndanimetatud Pealae asemele, mida heebrea keeli hüütakse Kolgataks.
\par 18 Seal nad lõid ta risti ja teised kaks temaga, teise teisele poole, aga Jeesuse keskele.
\par 19 Aga Pilaatus kirjutas ka pealkirja ja pani selle risti külge; ja sellele oli kirjutatud: „Jeesus Naatsaretlane, juutide kuningas!”
\par 20 Seda pealkirja luges nüüd palju juute, sest paik, kus Jeesus risti löödi, oli linna ligi. Ja pealkiri oli kirjutatud heebrea, ladina ja kreeka keeli.
\par 21 Siis ütlesid juutide ülempreestrid Pilaatusele: „Ära kirjuta ”Juutide kuningas„, vaid: ”Ta on öelnud: mina olen juutide kuningas!”
\par 22 Pilaatus kostis: „Mis ma olen kirjutanud, olen ma kirjutanud!”
\par 23 Kui nüüd sõjamehed Jeesuse olid risti löönud, võtsid nad tema riided ja tegid neli osa, igale sõjamehele ühe osa, ja kuue; kuub aga oli õmbluseta, ülemisest äärest alumiseni ühes tükis kootud.
\par 24 Siis nad ütlesid üksteisele: „Ärgem kiskugem seda lõhki, vaid heitkem liisku selle kohta, kellele see saab!” - et läheks täide Kiri, mis ütleb: „Nad on mu riided isekeskis jaganud ja minu kuue kohta liisku heitnud!” Seda sõjamehed tegidki.
\par 25 Aga Jeesuse risti juures seisid tema ema ja ta ema õde Maarja, Kloopase naine, ja Maarja Magdaleena.
\par 26 Kui nüüd Jeesus nägi risti kõrval seisvat oma ema ja jüngrit, keda ta armastas, ütleb ta emale: „Naine, vaata, see on su poeg!”
\par 27 Pärast ta ütleb jüngrile: „Vaata, see on su ema!” Ja sestsamast tunnist võttis jünger ta oma kotta.

\section*{Jeesuse surm}

\par 28 Pärast seda ütleb Jeesus, teades, et kõik juba on lõpetatud, et Kiri täide läheks: „Mul on janu!”
\par 29 Seal seisis astja täis äädikat. Siis nad pistsid äädikaga täidetud käsna iisopi otsa ja panid selle tema suu ette.
\par 30 Kui nüüd Jeesus äädikat oli võtnud, ütles ta: „See on lõpetatud!” Ja ta nõrgutas pead ning heitis hinge.

\section*{Odapiste Jeesuse küljesse. Jeesuse matmine}

\par 31 Aga et oli valmistuspäev ja et kehad ei jääks hingamispäevaks ristile - sest see hingamispäev oli suur - palusid juudid Pilaatust, et ristilöödute sääreluud murtaks ja nad maha võetaks.
\par 32 Siis tulid sõjamehed ja murdsid esimese sääreluud ja teise omad, kes ühes temaga olid risti löödud.
\par 33 Aga kui nad tulid Jeesuse juurde ja nägid ta juba surnud olevat, ei murdnud nad tema sääreluid,
\par 34 vaid üks sõjameestest pistis odaga tema küljesse; ja kohe tuli välja verd ja vett.
\par 35 Ja see, kes seda nägi, on seda tunnistanud, ja tema tunnistus on tõsi, ja tema teab, et ta räägib tõtt, et teiegi usuksite.
\par 36 Sest see on sündinud, et Kiri täide läheks: „Tema luid ärgu murtagu!”
\par 37 Ja taas ütleb teine Kiri: „Nad saavad näha, kellesse nad on pistnud!”
\par 38 Aga pärast seda Joosep Arimaatiast, kes oli Jeesuse jünger, kuid salaja, kartusest juutide eest, palus Pilaatuselt, et ta tohiks maha võtta Jeesuse ihu. Ja Pilaatus andis temale loa. Siis ta tuli ja võttis Jeesuse ihu maha.
\par 39 Nikodeemuski, kes varem oli öösel Jeesuse juurde tulnud, tuli ja tõi segatud mürri ja aaloet ligi sada naela.
\par 40 Siis nad võtsid Jeesuse ihu ja mähkisid ta linastesse riietesse lõhnarohtudega, nõnda nagu on juutide matmisviis.
\par 41 Aga seal paigas, kus ta oli risti löödud, oli aed ja aias uus haud, kuhu veel iialgi ei olnud kedagi pandud.
\par 42 Sinna nad panid siis Jeesuse juutide valmistuspäeva pärast; sest see haud oli lähedal.


\chapter{20}

\section*{Jeesuse ülestõusmine}

\par 1 Aga nädala esimesel päeval tuli Maarja Magdaleena vara, kui alles pime oli, hauale ja näeb, et kivi on haua eest ära võetud.
\par 2 Siis ta jookseb ja tuleb Siimon Peetruse ja teise jüngri juurde, keda Jeesus armastas, ja ütleb neile: „Nad on Issanda hauast ära võtnud ja me ei tea, kuhu nad ta on pannud!”
\par 3 Siis läksid välja Peetrus ja teine jünger ja tulid hauale.
\par 4 Aga nad jooksid mõlemad üheskoos, ja teine jünger jooksis ees usinamini kui Peetrus ja jõudis enne hauale.
\par 5 Ja kummargile sisse vaadates näeb ta surnulinad seal olevat; aga ta ei läinud sisse.
\par 6 Siis tuli Siimon Peetrus, kes teda järgis, ja läks haua sisse. Ja ta näeb surnulinad maas olevat,
\par 7 ja higirätiku, mis oli tema pea peal olnud, et see ei ole surnulinadega maas, vaid on isepäinis kokku mähitud teises kohas.
\par 8 Siis läks sisse ka teine jünger, kes esimesena oli tulnud hauale, ja nägi ja uskus.
\par 9 Sest nad ei saanud veel aru Kirjast, et tema pidi surnuist üles tõusma.
\par 10 Siis need jüngrid läksid jälle tagasi omaste juurde.

\section*{Jeesus ilmub Maarja Magdaleenale}

\par 11 Maarja aga seisis haua juures väljas ja nuttis. Kui ta nõnda nuttis, vaatas ta kummargile hauda
\par 12 ja näeb kaht inglit valgeis riideis istuvat, ühe peatsis ja teise jalutsis, seal kus Jeesuse ihu oli maganud.
\par 13 Ja need ütlevad temale: „Naine, miks sa nutad?” Ta ütleb neile: „Nemad on mu Issanda ära viinud ja ma ei tea, kuhu nad ta on pannud!”
\par 14 Kui ta seda oli öelnud, pöördus ta ümber ja näeb Jeesust seisvat, ja ta ei teadnud, et see on Jeesus.
\par 15 Jeesus ütleb temale: „Naine, miks sa nutad? Keda sa otsid?” Naine mõtleb tema aedniku olevat ja ütleb talle: „Isand, kui sina ta oled ära kandnud, siis ütle mulle, kuhu sa ta oled pannud, ja mina toon ta ära!”
\par 16 Jeesus ütleb temale: „Maarja!” See pöördub ümber ja ütleb talle heebrea keeli: „Rabuuni!” See tähendab: õpetaja.
\par 17 Jeesus ütleb talle: „Ära puuduta mind, sest ma pole veel üles läinud oma Isa juurde! Kuid mine mu vendade juurde ja ütle neile: mina lähen üles oma Isa ja teie Isa juurde ja oma Jumala ja teie Jumala juurde!”
\par 18 Maarja Magdaleena tuleb ja teatab jüngritele, et ta on Issandat näinud ja et ta temale seda on öelnud.

\section*{Jeesus ilmub jüngritele}

\par 19 Kui nüüd õhtu aeg oli samal nädala esimesel päeval ja uksed olid lukus seal, kus jüngrid olid kartuse pärast juutide eest, siis tuli Jeesus ja seisis keset nende seas ja ütles neile: „Rahu olgu teile!”
\par 20 Ja kui ta seda oli öelnud, näitas ta neile oma käsi ja külge. Siis jüngrid said rõõmsaks Issandat nähes.
\par 21 Jeesus ütles nüüd taas neile: „Rahu olgu teile! Nõnda nagu minu Isa on mind läkitanud, nõnda läkitan ka mina teid!”
\par 22 Ja kui ta seda oli öelnud, puhus ta nende peale ja ütles neile: „Võtke vastu Püha Vaim!
\par 23 Kellele te iganes patud andeks annate, neile on need andeks antud; kellele te iganes patud kinnitate, neile on need kinnitatud!”

\section*{Jeesus ilmub Toomale}

\par 24 Aga Toomas, üks neist kaheteistkümnest, keda nimetatakse Kaksikuks, ei olnud nendega, kui Jeesus tuli.
\par 25 Siis ütlesid teised jüngrid temale: „Me nägime Issandat!” Aga tema ütles neile: „Kui ma ei näe tema kätes naelte jälgi ega pane oma sõrme naelte asemeisse ja oma kätt tema külje sisse, siis ma ei usu!”
\par 26 Ja kaheksa päeva pärast olid tema jüngrid jälle toas ja Toomas nendega. Siis tuli Jeesus uste lukus olles ja seisis nende keskel ja ütles: „Rahu olgu teile!”
\par 27 Selle järel ta ütleb Toomale: „Pane oma sõrm siia ja vaata minu käsi, ja siruta oma käsi siia ja pista see minu küljesse, ja ära ole uskmatu, vaid usklik!”
\par 28 Toomas vastas ning ütles temale: „Minu Issand ja minu Jumal!”
\par 29 Jeesus ütleb temale: „Et sa mind oled näinud, siis sa usud. Õndsad on need, kes ei näe ja siiski usuvad!”

\section*{Johannese raamatu eesmärk}

\par 30 Nii tegi Jeesus ka veel muid tunnustähti jüngrite ees, mida ei ole kirjutatud sellesse raamatusse.
\par 31 Aga need on kirjutatud, et te usuksite, et Jeesus on Kristus, Jumala Poeg, ja et teil uskudes oleks elu tema nime sees.


\chapter{21}

\section*{Jeesus ilmub Tibeeria mere rannal}

\par 1 Pärast seda ilmus Jeesus jälle jüngritele Tibeeria mere ääres. Aga ta ilmus nõnda:
\par 2 Siimon Peetrus ja Toomas, keda kutsutakse Kaksikuks, ja Naatanael Kaanast Galileamaalt, ja Sebedeuse pojad ja veel teised kaks tema jüngritest olid seal koos.
\par 3 Siis ütleb Siimon Peetrus neile: „Ma lähen kalale.” Nad ütlevad temale: „Me tuleme ka ühes sinuga.” Nad läksid välja ja astusid varsti paati, aga ei saanud ühtki sel ööl.
\par 4 Aga kui juba oli hommik, seisis Jeesus rannal. Ent jüngrid ei teadnud, et see oli Jeesus.
\par 5 Siis ütleb Jeesus neile: „Lapsed, kas teil on midagi toidupoolist?” Nemad vastasid temale: „Ei ole.”
\par 6 Aga tema ütles neile: „Heitke võrk välja paremale poole paati, siis te leiate!” Nad heitsid selle siis välja ega suutnud seda enam vedada kalade hulga pärast!
\par 7 Siis ütles see jünger, keda Jeesus armastas, Peetrusele: „See on Issand!” Kui nüüd Siimon Peetrus kuulis, et see on Issand, pani ta kuue selga, sest ta oli alasti, ja heitis enese merre.
\par 8 Aga teised jüngrid tulid paadiga - sest nad ei olnud kaldast kaugemal kui kakssada küünart maad - ja vedasid võrku kaladega.
\par 9 Kui nad nüüd astusid maale, näevad nad sütetule maas olevat ja kalu seal peal ja leiba.
\par 10 Jeesus ütleb neile: „Tooge neid kalu, mis te praegu püüdsite!”
\par 11 Siimon Peetrus läks ja vedas võrgu mäele, mis oli täis suuri kalu, arvult sada ja viiskümmend kolm. Ja ehk küll kalu nii palju oli, võrk ei rebenenud mitte.
\par 12 Jeesus ütleb neile: „Tulge einestama!” Aga ükski jüngritest ei julgenud temalt küsida: „Kes sa oled?” Sest nad teadsid, et see oli Issand.
\par 13 Siis Jeesus tuleb ja võtab leiva ja annab neile, ja samuti kalu.
\par 14 See oli juba kolmas kord, et Jeesus pärast oma ülestõusmist surnuist näitas ennast oma jüngritele.

\section*{Jeesuse viimsed sõnad Peetrusele}

\par 15 Aga kui nad olid einestanud, ütleb Jeesus Siimon Peetrusele: „Siimon, Joona poeg, kas sa armastad mind rohkem kui need?” Ta ütleb temale: „Jah, Issand, sina tead, et sa oled mulle armas!” Ta ütleb temale: „Sööda mu tallekesi!”
\par 16 Ta ütleb temale jälle teist korda: „Siimon, Joona poeg, kas sa armastad mind?” Ta ütleb temale: „Jah, Issand, sina tead, et sa oled mulle armas!” Ta ütleb temale: „Hoia mu lambaid kui karjane!”
\par 17 Kolmandat korda ta ütleb temale: „Siimon, Joona poeg, kas ma olen sulle armas?” Peetrus sai kurvaks, et ta temale kolmandat korda ütles: „Kas ma olen sulle armas?” Ja ta ütles temale: „Issand, sina tead kõik, sina tunned, et oled mulle armas!” Jeesus ütleb temale: „Sööda mu lambaid!
\par 18 Tõesti, tõesti ma ütlen sulle: kui sa olid noorem, siis sa vöötasid ennast ise ja läksid, kuhu sa tahtsid. Aga kui sa saad vanaks, siis sa sirutad oma käed ja keegi teine vöötab sind ja viib sind, kuhu sa ei taha!”
\par 19 Aga seda ta ütles tähendades, missuguse surmaga ta pidi Jumalat austama. Ja kui ta seda oli rääkinud, ütleb ta temale: „Järgi mind!”
\par 20 Aga Peetrus pöördus ja näeb jüngrit, keda Jeesus armastas, järel sammuvat, selle, kes õhtusöömaajal oli ka laskunud Jeesuse rinnale ja öelnud: „Issand, kes see on, kes sind ära annab?”
\par 21 Kui nüüd Peetrus teda nägi, ütleb ta Jeesusele: „Issand, aga kuidas jääb temaga?”
\par 22 Jeesus ütleb talle: „Kui ma tahan, et ta jääb, kuni ma tulen, mis see sinusse puutub? Järgi sina mind!”
\par 23 Siis tõusis jutt vendade seas, et see jünger ei surevat. Ent Jeesus ei olnud temale mitte öelnud, et ta ei sure, vaid: „Kui ma tahan, et ta jääb, kuni ma tulen, mis see sinusse puutub?”

\section*{Lõppsõna}

\par 24 See on see jünger, kes neist asjust tunnistab ja on selle kirjutanud; ja me teame, et tema tunnistus on tõsi.
\par 25 On veel palju muudki, mida Jeesus tegi; kui seda kõike üksikult üles kirjutataks, siis ma arvan, et maailm ei mahutaks kirjutatavaid raamatuid!






\end{document}