\begin{document}

\title{Pauluse kiri Tiitusele}

\chapter{1}

\section*{Tervitus}

\par 1 Paulus, Jumala sulane, Jeesuse Kristuse apostel Jumala valitute usu ja tõetunnetuse järgi, mis vastab jumalakartusele,
\par 2 apostel igavese elu lootuses, mille Jumal, kes ei valeta, on tõotanud enne igavesi aegu -
\par 3 omal ajal aga tegi tema oma sõna avalikuks jutlustamises, mis minu kätte on usaldatud Jumala, meie Õnnistegija käsul -
\par 4 Tiitusele, tõelisele pojale meie ühise usu poolest: armu ja rahu Jumalalt Isalt ja Issandalt Kristuselt Jeesuselt, meie Õnnistegijalt!

\section*{Kogudusevanemate määramisest}

\par 5 Selle tarvis ma jätsin sind Kreetasse, et sa korraldaksid, mis veel jäi korraldamata, ja seaksid vanemaid igasse linna, nagu ma sind käskisin,
\par 6 kui kuskil leidub laitmatu isik, ühe naise mees, kellel on usklikud lapsed, nõnda et neile ei saa ette heita kõlvatut elu ega kangekaelsust.
\par 7 Sest koguduse ülevaataja peab olema kui Jumala majapidaja, laitmatu, mitte isemeelne, mitte äkiline, mitte joodik, mitte riiakas, mitte liigkasupüüdja,
\par 8 vaid külalislahke, headuse armastaja, mõistlik, õige, püha, kasin;
\par 9 ta pidagu kinni ustavast sõnast vastavalt õpetusele, et ta oleks võimeline niihästi manitsema terves õpetuses kui ka kummutama vasturääkijate väited.

\section*{Valeõpetajaist}

\par 10 Sest on olemas palju kangekaelseid, tühja jutu mehi ja eksitajaid, kõige enam ümberlõigatute seas,
\par 11 kelle suu tuleb sulgeda, kes pööravad ära terved pered, õpetades sündmatuid asju häbematu kasu pärast.
\par 12 Üks neist, nende oma prohvet, on öelnud: „Kreetalased on luiskajad, õelad loomad ja laisklejad vatsad!”
\par 13 See tunnistus on tõsi. Sellepärast noomi neid valjusti, et nad saaksid terveks usus
\par 14 ega paneks tähele juutide tühje jutte ja tõest eemalepöördunud inimeste käske!
\par 15 Puhtaile on kõik puhas, aga rüvedaile ja uskmatuile pole miski puhas, vaid rüve on niihästi nende meel kui südametunnistus.
\par 16 Nemad väidavad, et nad tunnevad Jumalat, aga tegudega nad salgavad; sest nad on jälgid ja sõnakuulmatud ja igaks heaks tööks kõlbmatud.


\chapter{2}

\section*{Kristlikust eluviisist}

\par 1 Ent sina räägi seda, mis sünnib ühte terve õpetusega.
\par 2 Vanad mehed olgu karsked, ausad, mõistlikud, terved usus, armastuses, kannatlikkuses.
\par 3 Samuti käitugu elatanud naised nõnda, nagu on kohane pühadele; nad ärgu olgu keelekandjad ega suured viinavõtjad, vaid õpetagu head,
\par 4 et nad võiksid mõistlikult juhatada noori naisi armastama oma mehi ja lapsi,
\par 5 olema viisakad, kasinad, majatalituses hoolikad, head, allaheitlikud oma meestele, et Jumala sõna mitte ei pilgataks.
\par 6 Samuti käsi nooremaid mehi käituda viisakalt.
\par 7 Sea ennast igapidi heategude eeskujuks, õpetuses osuta selgust, ausust,
\par 8 tervet vääramata sõna, et vastane tunneks häbi, kui tal ei ole teist rääkida midagi paha.
\par 9 Orje käsi alistuda oma isandaile ja olla kõiges nende meele järgi, mitte vastu rääkida,
\par 10 mitte midagi näpata, vaid osutada täit head ustavust, et kaunistada kõiges Jumala, meie Õnnistegija õpetust.

\section*{Kristliku eluviisi alus ja eesmärk}

\par 11 Sest Jumala õndsakstegev arm on ilmunud kõigile inimestele
\par 12 ja kasvatab meid, et me hülgaksime jumalakartmatu elu ja maailma himud ning elaksime mõistlikult ja õieti ja jumalakartlikult praegusel maailmaajastul,
\par 13 oodates õndsa lootuse täitumist ja suure Jumala ning meie Õnnistegija Jeesuse Kristuse auhiilguse ilmumist,
\par 14 kes iseenese andis meie eest, et meid lunastada kõigest ülekohtust ja puhastada enesele pärisrahvaks, kes agar on tegema häid tegusid.
\par 15 Seda räägi ja manitse ning veena kõige nõudlikkusega. Keegi ärgu põlaku sind!


\chapter{3}

\section*{Veel õigest kristlikust elust}

\par 1 Tuleta neile meelde, et nad alistuksid valitsejaile ja ülemaile, oleksid sõnakuulelikud, valmis igaks heaks teoks,
\par 2 ei laimaks kedagi, ei oleks riiakad, vaid leebed, osutades kõike tasadust kõigi inimeste vastu.
\par 3 Sest ka meie olime kord mõistmatud, sõnakuulmatud, eksijad, mitmesuguste himude ja lõbude orjad ning elasime kurjuses ja kadeduses, olime vihatavad ja vihkasime üksteist.
\par 4 Aga kui Jumala, meie Õnnistegija heldus ja inimesearmastus ilmus,
\par 5 siis ta päästis meid, ei mitte õiguse tegude tõttu, mida me oleksime teinud, vaid oma halastust mööda uuestisünni pesemise ja Püha Vaimu uuendamise kaudu,
\par 6 keda tema on rikkalikult välja valanud meie peale Jeesuse Kristuse, meie Õnnistegija läbi,
\par 7 et meie, tema armu läbi õigeks tehtud, saaksime igavese elu pärijaiks lootuse järgi.
\par 8 See sõna on ustav; ja ma tahan, et sa neid asju teravasti rõhutaksid, et need, kes usuvad Jumalasse, oleksid hoolsad heade tegude tegemises; need on head ja kasulikud inimestele.
\par 9 Aga väldi rumalaid uurimusi ja sugukondade nimekirju ning riidu ja vaidlusi käsu pärast; sest need on kasutud ja tühised.
\par 10 Eksiõpetusega inimesest pöördu ära, kui sa teda ühe korra või kaks oled noominud,
\par 11 teades, et niisugune on pöörane ja teeb pattu ning on iseenese hukka mõistnud.

\section*{Viimsed manitsused ja jumalagajätt}

\par 12 Kui ma Artemase või Tühhikose läkitan sinu juurde, siis tõtta tulema minu juurde Nikoopolisse; sest ma olen otsustanud seal viibida talve.
\par 13 Käsutundja Zeenas ja Apollos varusta hoolsasti teekonnale, et neil midagi ei puuduks.
\par 14 Õppigu meiegi omad harrastama häid tegusid hädavajalikeks tarvidusteks, et nad ei jääks viljatuks.
\par 15 Sind tervitavad kõik, kes on minuga. Tervita neid, kes meid armastavad usus. Arm olgu teie kõikidega!




\end{document}