\begin{document}

\title{Pauluse kiri galaatlastele}

\chapter{1}

\section*{Tervitus}

\par 1 Paulus, apostel, mitte inimeste poolt valitud ega ühegi inimese kaudu, vaid Jeesuse Kristuse ja Jumala Isa läbi, kes tema on surnuist üles äratanud,
\par 2 ja kõik vennad, kes on minu juures, Galaatia kogudustele:
\par 3 Armu teile ja rahu Jumalalt, meie Isalt, ja Issandalt Jeesuselt Kristuselt,
\par 4 kes enese andis meie pattude eest, et ta meid viiks välja praegusest kurjast ajastust Jumala ja meie Isa tahte järgi,
\par 5 kellele olgu austus ajastute ajastuteni! Aamen.

\section*{Apostel on pettunud galaatlastes}

\par 6 Ma panen imeks, et te Kristusest, kes teid on armus kutsunud, nii ruttu ära pöördute teistsuguse evangeeliumi juurde,
\par 7 mis ei olegi teine evangeelium; on vaid mõningaid, kes teid segavad ja tahavad Kristuse evangeeliumi pöörata teiseks.
\par 8 Aga kui me ise, või isegi mõni ingel taevast, peaks teile kuulutama evangeeliumi peale selle, mida meie teile oleme kuulutanud - ta olgu neetud!
\par 9 Nagu me ennegi oleme öelnud, nõnda ütlen mina ka nüüd jälle: kui keegi teile kuulutab evangeeliumi peale selle, mis te olete saanud, siis ta olgu neetud!
\par 10 Kas ma nüüd püüan inimeste või Jumala heakskiitmist? Või püüan ma olla inimestele meelepärane? Sest kui ma veel tahaksin olla inimestele meelepärane, siis ma ei oleks Kristuse sulane!

\section*{Pauluse evangeelium on Kristuselt}

\par 11 Sest ma annan teile teada, vennad, et evangeelium, mida mina olen kuulutanud, ei ole inimestelt.
\par 12 Sest ma pole seda saanud ega õppinud inimestelt, vaid Jeesuse Kristuse ilmutuse kaudu.
\par 13 Sest te olete kuulnud minu endisest elust juudi õpetuses, et ma üliväga taga kiusasin Jumala kogudust ja hävitasin seda,
\par 14 ja et ma juudi õpetuses ette jõudsin mitmest samaealisest oma hõimus, olles palju innukam harrastama oma esiisade pärimusi.
\par 15 Aga kui oli sellele meelepärane, kes mind mu ema ihust valis ja oma armu läbi kutsus,
\par 16 ilmutada minus oma Poega, et ma kuulutaksin teda paganate seas, siis ma ei hakanud kohe nõu pidama liha ja verega
\par 17 ega läinud ka Jeruusalemma nende juurde, kes enne mind olid apostlid, vaid ma läksin Araabiasse ja tulin jälle tagasi Damaskusesse.
\par 18 Hiljemini alles, kolme aasta pärast, ma läksin Jeruusalemma tegema tutvust Keefasega ja jäin tema juurde viieteistkümneks päevaks.
\par 19 Kedagi muud apostlitest ma ei näinud kui vaid Jakoobust, Issanda venda.
\par 20 Ja mida ma teile kirjutan, vaata, ma kinnitan Jumala ees, et ma ei valeta!
\par 21 Pärast ma tulin Süüria ja Kiliikia maakohtadesse.
\par 22 Aga ma olin näo poolest tundmatu Judea kogudustele, kes on Kristuses.
\par 23 Nad olid ainult kuulnud, et see, kes meid enne taga kiusas, nüüd jutlustab seda usku, mida ta enne hävitas.
\par 24 Ja nad andsid Jumalale austust minu pärast.


\chapter{2}

\section*{Pauluse tööd tunnustatakse Jeruusalemmas}

\par 1 Siis, neljateistkümne aasta pärast, ma läksin jälle üles Jeruusalemma ühes Barnabasega ja võtsin ka Tiituse kaasa.
\par 2 Ent ma läksin sinna ilmutuse tagajärjel ja esitasin neile evangeeliumi, mida ma kuulutasin paganate seas, eriti neile tähtsamaile, et ma kuidagi tühja ei jookseks ega oleks jooksnud.
\par 3 Aga Tiitustki, mu kaaslast, ehk ta küll on kreeklane, ei sunnitud end laskma ümber lõigata.
\par 4 Ent sissepugenud valevendade pärast, kes olid hiilinud varitsema meie vabadust, mis meil on Kristuses Jeesuses, et meid orjastada,
\par 5 ei andnud me silmapilkugi järele ega alistunud neile, et evangeeliumi tõde jääks teie juurde.
\par 6 Ja need, kes näisid midagi olevat - olgu missugused nad iganes olid, mulle on see ükskõik, Jumal ei pea ju ühest inimesest rohkem lugu kui teisest - need, kes olid tähtsamad, ei lisanud mulle midagi juurde,
\par 7 vaid vastupidi, kui nad nägid, et minu hooleks on usaldatud evangeeliumi kuulutamine ümberlõikamatuile, nõnda nagu Peetruse hooleks ümberlõigatuile -
\par 8 sest see, kes tegev oli Peetruses apostliametiks ümberlõigatute seas, oli ka minus tegev paganate heaks -
\par 9 ja kui Jakoobus ja Keefas ja Johannes, keda peeti sambaiks, tundsid ära armu, mis mulle oli antud, andsid nad mulle ja Barnabasele kätt osaduse täheks, et meie paganate seas ja nemad ümberlõigatute seas, kuulutaksime armuõpetust,
\par 10 ainult et me ka mõtleksime vaestele, ja just seda ma olen olnud usin tegema.

\section*{Paulus noomib Peetrust}

\par 11 Aga kui Keefas tuli Antiookiasse, astusin ma isiklikult ta vastu, sest teda oli tarvis noomida.
\par 12 Sest enne kui Jakoobuse juurest oli tulnud mõningaid, sõi ta ühes paganatega; aga kui need tulid, tõmbus ta tagasi ja läks kõrvale, kartes ümberlõigatuid.
\par 13 Ja ühes temaga hakkasid ka teised juudid silmakirjatsema, nõnda et ka Barnabas kaasa tõmmati nende silmakirjatsemisse.
\par 14 Aga kui ma nägin, et nad õigesti ei käinud evangeeliumi tõe järgi, ütlesin ma kõikide ees Keefale: „Kui sina, olles juut, elad paganate ja mitte juutide kommete järgi, miks sa siis sunnid paganaid elama juudi kommete järgi?

\section*{Käsuõpetusest ei piisa, õigeks saab üksnes usu läbi}

\par 15 Meie oleme sünnilt juudid ja mitte patused paganate seast,
\par 16 aga teades, et inimene ei saa õigeks käsu tegudest, vaid üksnes Kristuse Jeesuse usu kaudu, oleme ka meie uskunud Kristusesse Jeesusesse, et me saaksime õigeks Kristuse usust ja mitte käsu tegudest, sest käsu tegudest ei saa õigeks ükski liha.
\par 17 Aga kui otsides õigekssaamist Kristuses, meiegi osutume patusteks, kas on siis Kristus patu teenija? Ei sugugi mitte!
\par 18 Sest kui ma jälle ehitan üles selle, mille ma olen maha kiskunud, siis ma teen enese üleastujaks.
\par 19 Sest mina olen käsu läbi käsule surnud, et ma elaksin Jumalale; ma olen ühes Kristusega risti löödud!
\par 20 Ent nüüd ei ela enam mina, vaid Kristus elab minu sees! Ja mida ma nüüd elan lihas, seda ma elan usus Jumala Pojasse, kes mind on armastanud ja on iseenese andnud minu eest.
\par 21 Ma ei tee Jumala armu tühjaks. Sest kui käsu kaudu tuleks õigus, siis oleks Kristus ilmaasjata surnud!



\chapter{3}

\section*{Usk on olulisem kui käsuteod}

\par 1 Oh te mõistmatud galaatlased, kes teid on võlunud? Teid, kelle silmade ette Jeesus Kristus oli joonistatud ristilööduna?
\par 2 Seda üksnes ma tahan teilt teada, kas te saite Vaimu käsu tegudest või usu kuulutamisest?
\par 3 Kas te olete nii mõistmatud? Te algasite Vaimus; kas tahate nüüd lihas lõpetada?
\par 4 Kas te nii suurt olete kannatanud ilmaasjata? Teisiti see olekski ilmaasjata!
\par 5 Tema, kes siis annab teile Vaimu ja teeb vägevaid tegusid teie sees, kas ta teeb seda käsu tegudest või usu kuulutamisest,
\par 6 nõnda nagu „Aabraham uskus Jumalat, ja see arvati temale õiguseks!”?
\par 7 Te ju teate, et need, kes on usust, on Aabrahami lapsed.
\par 8 Aga et Kiri nägi ette ära, et Jumal teeb paganad usust õigeks, kuulutas see Aabrahamile selle hea sõnumi: „Sinus õnnistatakse kõik rahvad!”
\par 9 Sellepärast õnnistatakse neid, kes on usust, ühes uskliku Aabrahamiga.
\par 10 Sest kõik, kes endid rajavad käsutegudele, on needuse all; sest on kirjutatud: „Neetud on igaüks, kes ei püsi kõiges, mis on kirja pandud käsuraamatus, nii et ta seda teeks!”
\par 11 Et nüüd käsu kaudu ükski ei saa õigeks Jumala ees, on ilmne, sest õige peab usust elama.
\par 12 Käsk aga ei ole usust, vaid „kes teeb nende järgi, see elab nende varal!”
\par 13 Kristus on meid lahti ostnud käsu needusest, kui ta sai needuseks meie eest - sest on kirjutatud: „Neetud on igaüks, kes puu küljes ripub” -
\par 14 et Aabrahami õnnistus saaks paganaile osaks Jeesuses Kristuses ja me usu kaudu saaksime Vaimu tõotuse.
\par 15 Vennad, mina räägin inimese viisil! Inimesegi viimset tahet, kui see on nimetatud, ei tee ükski tühjaks ega lisa sellele midagi juurde.
\par 16 Aga nüüd on tõotused antud Aabrahamile ja tema soole. Ei ole mitte öeldud: „Ja sinu sugudele”, otsekui paljude kohta, vaid nagu ühe kohta: „Ja sinu soole,” kes on Kristus.
\par 17 Aga ma ütlen seda: lepingut, mille Jumal on enne kinnitanud, ei tee nelisada kolmkümmend aastat hiljemini tekkinud käsk mitte tühjaks, nii et see hävitaks tõotuse.
\par 18 Sest kui pärand tuleks käsust, ei oleks see enam tõotusest. Ent Jumal on selle kinkinud Aabrahamile tõotuse kaudu.

\section*{Käsu otstarve}

\par 19 Milleks on siis käsk? See on üleastumiste pärast juurde lisatud seniks, kui tuleb sugu, kellele oli antud tõotus; ja käsk seati inglite kaudu vahemehe käe läbi.
\par 20 Kuid vahemeest ei vajata ühe jaoks; ent Jumal on üks.
\par 21 Kas siis käsk on Jumala tõotuste vastu? Ei sugugi. Sest kui oleks antud käsk, mis võiks teha elavaks, siis tuleks õigus tõesti käsust.
\par 22 Aga Kiri on kõik pannud kinni patu alla, et tõotus antaks Jeesuse Kristuse usust neile, kes usuvad.
\par 23 Aga enne kui tuli usk, olime käsu valve all, kinni pandud tulevase usu ilmumise jaoks.
\par 24 Nõnda on käsk saanud meie kasvatajaks Kristuse poole, et me saaksime õigeks usust.

\section*{Usu tagajärg}

\par 25 Aga et usk on tulnud, ei ole me mitte enam kasvataja all.
\par 26 Sest te olete kõik usu kaudu Jumala lapsed Kristuses Jeesuses.
\par 27 Sest nii paljud kui teid on Kristusesse ristitud, olete Kristusega riietatud!
\par 28 Ei ole siin juuti ega kreeklast, ei ole siin orja ega vaba, ei ole siin meest ega naist, sest te kõik olete üks Kristuses Jeesuses.
\par 29 Aga kui te olete Kristuse omad, siis olete ka Aabrahami sugu ja pärijad tõotuse järgi.


\chapter{4}

\section*{Jumala lapse seisus saadakse usust Kristusesse}

\par 1 Aga ma ütlen: niikaua kui pärija on alaealine, ei ole mingit vahet orja ja tema vahel, ehk ta küll on kõige isand;
\par 2 vaid ta on eestkostjate ja valitsejate all kuni isa poolt määratud ajani.
\par 3 Nõnda ka meie: kui olime alaealised, siis olime orjastatud maailma algjõudude alla.
\par 4 Aga kui aeg täis sai, läkitas Jumal oma Poja, kes sündis naisest ja sai käsu alla,
\par 5 lahti ostma käsualuseid, et me saaksime lapse seisuse.
\par 6 Aga et te nüüd olete lapsed, on Jumal läkitanud teie südamesse oma Poja Vaimu, kes hüüab: „Abba, Isa!”
\par 7 Nõnda ei ole sa enam ori, vaid laps; aga kui sa oled laps, siis oled ka pärija Jumala kaudu.

\section*{Apostel on mures, et vastsed kristlased taganevad usust}

\par 8 Aga siis, kui te ei tundnud Jumalat, te orjasite neid, kes olemise poolest ei olegi jumalad.
\par 9 Ent nüüd, kus te olete Jumala ära tundnud, või õigemini, Jumal on teid ära tundnud, kuidas te pöördute jälle nõrkade ja viletsate algjõudude poole, mida te jälle tahate uuesti orjata?
\par 10 Te arvestate päevi ja kuid ja aegu ja aastaid.
\par 11 Ma kardan teie pärast, et ma teie kallal ehk asjata olen vaeva näinud.
\par 12 Saage minusuguseiks, sest minagi olen saanud teiesuguseks, vennad, ma palun teid! Te ei ole mind millegagi solvanud.
\par 13 Te ju teate, et ma liha nõtruses teile kuulutasin evangeeliumi esimesel korral,
\par 14 ja te ei pannud kiusatust, mis teil oli minu kehalisest seisukorrast, mitte halvaks ega võõrastanud mind, vaid võtsite mind vastu kui Jumala ingli, kui Kristuse Jeesuse.
\par 15 Kus on nüüd teie õndsuse kiitlus? Sest ma tunnistan teile, et teie, kui see oleks olnud võimalik, oleksite oma silmad kiskunud välja ja andnud minule.
\par 16 Kas ma nüüd olen saanud teie vaenlaseks, et ma olen teile tõtt öelnud?
\par 17 Nad on õhinal teie ümber, kuid mitte kaunil viisil, vaid nad tahavad teid välja jätta, et te innukad oleksite nende kasuks.
\par 18 Hea on olla innukas hea suhtes alatasa ja mitte ainult siis, kui mina teie juures olen.
\par 19 Mu lapsukesed, kelle pärast ma nüüd jälle olen lapsevaevas, kuni Kristus teie sees saab kuju!
\par 20 Küll ma tahaksin nüüd olla teie juures ja oma häältki muuta, sest ma olen päris nõutu teie pärast!

\section*{Kaks poega}

\par 21 Öelge mulle, kes tahate olla käsu all, kas te ei kuule käsuõpetust?
\par 22 Sest on kirjutatud, et Aabrahamil oli kaks poega, üks ümmardajast ja teine vabast naisest.
\par 23 Aga ümmardaja poeg oli sündinud liha järgi, vaba naise poeg aga tõotuse kaudu.
\par 24 Need on võrdluskujud. Sest need naised on kaks lepingut; üks Siinai mäelt, mis sünnitab orjapõlveks, see on Haagar.
\par 25 Sest Haagar tähendab Siinai mäge Araabias ja vastab praegusele Jeruusalemmale, sest see orjab ühes oma lastega.
\par 26 Aga see Jeruusalemm, mis on ülal, on vaba naine, ja see on meie kõikide ema.
\par 27 Sest on kirjutatud: „Rõõmutse, sigimatu, kes pole sünnitanud; tõsta häält ja hüüa valjusti, kes pole olnud lapsevaevas, sest vallalisel saab olema rohkem lapsi kui sellel, kellel on mees!”
\par 28 Ent teie, vennad, olete nõnda nagu Iisak, tõotuse lapsed!
\par 29 Aga nõnda nagu tol ajal see, kes oli sünnitatud liha järgi, taga kiusas seda, kes oli sündinud Vaimu järgi, nõnda on ka nüüd.
\par 30 Aga mida ütleb Kiri? „Kihuta minema ümmardaja ja tema poeg, sest ümmardaja poeg ei või pärida ühes vaba naise pojaga!”
\par 31 Niisiis, vennad, ei ole me ümmardaja, vaid oleme vaba naise lapsed!


\chapter{5}

\section*{Kaks poega}

\par 1 Vabaduseks on Kristus meid vabastanud. Püsige siis selles ja ärge laske endid jälle panna orjaikkesse.

\section*{Valik Kristuse ja käsuõpetuse vahel}

\par 2 Vaata, mina, Paulus, ütlen teile: kui te lasete endid ümber lõigata, siis ei ole teil Kristusest mingit kasu.
\par 3 Aga ma tunnistan jälle igale inimesele, kes laseb ennast ümber lõigata, et ta kohus on täita kõike käsku.
\par 4 Te olete jäänud ilma Kristusest, kes tahate saada õigeks käsu kaudu; te olete langenud ära armust.
\par 5 Sest me ootame usust õiguse lootust Vaimu kaudu.
\par 6 Sest Kristuses Jeesuses ei kehti ümberlõikamine ega ümberlõikamatus, vaid usk, mis on tegev armastuse kaudu.
\par 7 Te jooksite hästi; kes on teid takistanud olemast tõele sõnakuulelikud?
\par 8 Veenmine selleks ei tulnud mitte temalt, kes teid kutsub.
\par 9 Pisut haputaignat teeb kogu taigna hapuks.
\par 10 Mina loodan teist Issandas, et te ei hakka teisiti mõtlema; aga kes teid teeb segaseks, peab kandma oma nuhtlust, olgu see kes tahes.
\par 11 Aga mind, vennad, kui ma veel jutlustan ümberlõikamist, miks mind veel taga kiusatakse? Siis ju oleks pahandus ristist kõrvaldatud.
\par 12 Oh et need, kes teid tülitavad, endid ära lõikaksid!

\section*{Armastus piiraku vabadust}

\par 13 Sest teie, vennad, olete kutsutud vabaduseks; aga mitte lihale ajet andvaks vabaduseks, vaid teenige üksteist armastuses.
\par 14 Sest kõik käsk on täidetud ühes sõnas, nimelt selles: „Armasta oma ligimest nagu iseennast!”
\par 15 Aga kui te isekeskis purelete ja üksteist sööte, siis katsuge, et te üksteist ära ei neela.

\section*{Elu Vaimus}

\par 16 Aga ma ütlen: käige Vaimus, siis te ei täida mitte liha himusid.
\par 17 Sest liha himustab Vaimu vastu ja Vaim liha vastu; need on üksteise vastu, et te ei teeks seda, mida te tahate.
\par 18 Aga kui te olete Vaimu juhitavad, siis te ei ole käsu all.
\par 19 Aga liha teod on ilmsed, need on: hoorus, rüvedus, kiimalus,
\par 20 ebajumalateenistus, nõidus, vaen, riid, kade meel, vihastumised, jonn, kildkonnad, lahkõpetused,
\par 21 tapmised, joomised, pidutsemised ja muud sellesarnast, millest ma teile ette ütlen, nagu ma ka juba enne olen öelnud, et need, kes teevad seesugust, ei päri Jumala riiki.
\par 22 Aga Vaimu vili on armastus, rõõm, rahu, pikk meel, lahkus, heatahtlikkus, ustavus,
\par 23 tasadus, kasinus; selliste vastu ei ole käsk.
\par 24 Aga kes on Kristuse Jeesuse päralt, need on oma liha risti löönud ühes tema ihalduste ja himudega.
\par 25 Kui me elame Vaimus, siis käigem ka Vaimus.
\par 26 Ärgem olgem ahned tühjale aule, üksteist ärritades, üksteist kadestades.


\chapter{6}

\section*{Manitsused laitmatuks eluks}

\par 1 Vennad, kui inimene ka satuks mingisse eksitusse, siis teie, vaimulikud, parandage niisugust jälle tasase vaimuga; ja valva sa iseennastki, et ka sind ei kiusataks.
\par 2 Kandke üksteise koormat, ja nõnda te täidate Kristuse käsku.
\par 3 Sest kui keegi mõtleb enese midagi olevat, ilma midagi olemata, see petab iseennast.
\par 4 Ent igaüks katsugu läbi oma töö, ja siis on tal üksi eneses kiitlemist, mitte teisele;
\par 5 sest igaüks kandku oma koormat.
\par 6 Aga keda õpetatakse sõnaga, see jagagu kõike head sellele, kes teda õpetab.
\par 7 Ärge eksige, Jumal ei anna ennast pilgata, sest mis inimene külvab, seda ta ka lõikab!
\par 8 Sest kes külvab oma lihale, see lõikab lihast kadu; aga kes külvab Vaimule, see lõikab Vaimust igavest elu.
\par 9 Ja kui me teeme head, siis ärgem tüdigem, sest me saame omal ajal lõigata, kui me ei väsi.
\par 10 Seepärast nüüd, et meil veel on aega, tehkem head kõikidele, aga kõige enam usukaaslastele!

\section*{Lõppsõna}

\par 11 Vaadake, kui suurte tähtedega ma oma käega teile olen kirjutanud!
\par 12 Kõik, kes püüavad meeldida liha poolest, need sunnivad teid ümberlõikamisele ainult sellepärast, et neid taga ei kiusataks Kristuse risti tõttu.
\par 13 Sest needki ise, kes lasevad endid ümber lõigata, ei pea käsku, vaid tahavad, et teid ümber lõigataks ja nad teie lihast võiksid kiidelda.
\par 14 Aga mulle ärgu juhtugu seda, et ma kiitleksin muust kui meie Issanda Jeesuse Kristuse ristist, kelle läbi maailm on minule risti löödud ja mina maailmale.
\par 15 Sest ümberlõikamine ei tähenda midagi ega ümberlõikamatus, vaid uus loodu.
\par 16 Ja kõigile, kes käivad seda juhtnööri mööda, neile olgu rahu ja halastus, ja ka Jumala Iisraelile!
\par 17 Viimaks ärgu tehku mulle keegi vaeva, sest mina kannan Jeesuse arme oma ihul.
\par 18 Meie Issanda Jeesuse Kristuse arm olgu teie vaimuga, vennad! Aamen.




\end{document}