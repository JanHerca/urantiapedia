\begin{document}

\title{Joonan kirja}


\chapter{1}

\par 1 Joonalle, Amittain pojalle, tuli tämä Herran sana:
\par 2 "Nouse, mene Niiniveen, siihen suureen kaupunkiin, ja saarnaa sitä vastaan; sillä heidän pahuutensa on noussut minun kasvojeni eteen".
\par 3 Mutta Joona nousi paetaksensa Tarsiiseen Herran kasvojen edestä ja meni alas Jaafoon ja löysi laivan, joka oli lähtevä Tarsiiseen. Ja hän suoritti laivamaksun ja astui siihen mennäkseen heidän kanssansa Tarsiiseen, pois Herran kasvojen edestä.
\par 4 Mutta Herra heitti suuren tuulen merelle, niin että merellä nousi suuri myrsky ja laiva oli särkymäisillään.
\par 5 Niin merimiehet pelkäsivät ja huusivat avuksi itsekukin jumalaansa. Ja he heittivät mereen tavarat, mitä laivassa oli, keventääkseen sitä. Mutta Joona oli mennyt alas laivan pohjalle ja pannut maata, ja hän nukkui raskaasti.
\par 6 Niin laivuri tuli hänen luoksensa ja sanoi hänelle: "Mitäs nukut? Nouse ja huuda jumalaasi. Ehkäpä se jumala muistaa meitä, niin ettemme huku."
\par 7 Ja he sanoivat toisillensa: "Tulkaa, heittäkäämme arpaa, saadaksemme tietää, kenen tähden tämä onnettomuus on meille tullut". Mutta kun he heittivät arpaa, lankesi arpa Joonalle.
\par 8 Niin he sanoivat hänelle: "Ilmoita meille, kenen tähden tämä onnettomuus on meille tullut. Mikä on toimesi ja mistä tulet? Mikä on sinun maasi ja mistä kansasta olet?"
\par 9 Hän vastasi heille: "Minä olen hebrealainen, ja minä pelkään Herraa, taivaan Jumalaa, joka on tehnyt meren ja kuivan maan".
\par 10 Niin miehet peljästyivät suuresti ja sanoivat hänelle: "Miksi olet tehnyt näin?" Sillä miehet tiesivät, että hän oli pakenemassa Herran kasvojen edestä; hän oli näet ilmaissut sen heille.
\par 11 He sanoivat hänelle: "Mitä on meidän tehtävä sinulle, että meri tulisi meille tyveneksi?" Sillä meri myrskysi myrskyämistänsä.
\par 12 Hän vastasi heille: "Ottakaa minut ja heittäkää minut mereen, niin meri teille tyventyy. Sillä minä tiedän, että tämä suuri myrsky on tullut teille minun tähteni."
\par 13 Miehet soutivat päästäkseen jälleen kuivalle maalle, mutta eivät voineet, sillä meri myrskysi vastaan myrskyämistänsä.
\par 14 Ja he huusivat Herraa ja sanoivat: "Voi, Herra, älä anna meidän hukkua tämän miehen hengen tähden äläkä lue syyksemme viatonta verta, sillä sinä, Herra, teet, niinkuin sinulle otollista on".
\par 15 Sitten he ottivat Joonan ja heittivät hänet mereen, ja meri asettui raivostansa.
\par 16 Ja miehet pelkäsivät suuresti Herraa, uhrasivat Herralle teurasuhrin ja tekivät lupauksia.

\chapter{2}

\par 1 Mutta Herra toimitti suuren kalan nielaisemaan Joonan. Ja Joona oli kalan sisässä kolme päivää ja kolme yötä.
\par 2 Ja Joona rukoili Herraa, Jumalaansa, kalan sisässä
\par 3 ja sanoi: "Minä huusin ahdistuksessani Herraa, ja hän vastasi minulle. Tuonelan kohdussa minä huusin apua, ja sinä kuulit minun ääneni.
\par 4 Sinä syöksit minut syvyyteen, merten sydämeen, ja virta ympäröitsi minut, kaikki sinun kuohusi ja aaltosi vyöryivät minun ylitseni.
\par 5 Minä ajattelin: Olen karkoitettu pois sinun silmiesi edestä. Kuitenkin minä saan vielä katsella sinun pyhää temppeliäsi.
\par 6 Vedet piirittivät minut aina sieluun asti, syvyys ympäröitsi minut, kaisla kietoutui päähäni.
\par 7 Minä vajosin alas vuorten perustuksiin asti, maan salvat sulkeutuivat minun ylitseni iankaikkisesti. Mutta sinä nostit minun henkeni ylös haudasta, Herra, minun Jumalani.
\par 8 Kun sieluni nääntyi minussa, minä muistin Herraa, ja minun rukoukseni tuli sinun tykösi, sinun pyhään temppeliisi.
\par 9 Ne, jotka kunnioittavat vääriä jumalia, hylkäävät armonantajansa.
\par 10 Mutta minä tahdon uhrata sinulle kiitoksen kaikuessa. Mitä olen luvannut, sen minä täytän. Herrassa on pelastus."
\par 11 Sitten Herra käski kalaa, ja se oksensi Joonan kuivalle maalle.

\chapter{3}

\par 1 Joonalle tuli toistamiseen tämä Herran sana:
\par 2 "Nouse ja mene Niiniveen, siihen suureen kaupunkiin, ja saarnaa sille se saarna, minkä minä sinulle puhun".
\par 3 Niin Joona nousi ja meni Niiniveen Herran sanan mukaan. Ja Niinive oli suuri kaupunki Jumalan edessä: kolme päivänmatkaa.
\par 4 Ja Joona käveli kaupungissa aluksi yhden päivänmatkan ja saarnasi sanoen: "Vielä neljäkymmentä päivää, ja Niinive hävitetään".
\par 5 Niin Niiniven miehet uskoivat Jumalaan, kuuluttivat paaston ja pukeutuivat säkkeihin, niin suuret kuin pienet.
\par 6 Ja kun tieto tästä tuli Niiniven kuninkaalle, nousi hän valtaistuimeltaan, riisui yltään vaippansa, verhoutui säkkiin ja istui tuhkaan.
\par 7 Ja hän huudatti Niinivessä: "Kuninkaan ja hänen ylimystensä määräys kuuluu: Älkööt ihmiset älköötkä eläimet - raavaat ja lampaat - maistako mitään, käykö laitumella tai vettä juoko.
\par 8 Verhoutukoot ihmiset ja eläimet säkkeihin, huutakoot väkevästi Jumalaa ja kääntykööt itsekukin pois pahalta tieltänsä sekä väkivallasta, mikä heidän käsiänsä tahraa.
\par 9 Ehkäpä Jumala jälleen katuu ja kääntyy vihansa hehkusta, niin ettemme huku."
\par 10 Kun Jumala näki heidän tekonsa, että he kääntyivät pois pahalta tieltänsä, niin Jumala katui sitä pahaa, minkä hän oli sanonut tekevänsä heille, eikä tehnyt sitä.

\chapter{4}

\par 1 Mutta Joona pahastui tästä kovin, ja hän vihastui.
\par 2 Ja hän rukoili Herraa ja sanoi: "Voi Herra! Enkö minä sitä sanonut, kun olin vielä omassa maassani? Siksihän minä ehätin pakenemaan Tarsiiseen. Sillä minä tiesin, että sinä olet armahtavainen ja laupias Jumala, pitkämielinen ja armosta rikas, ja että sinä kadut pahaa.
\par 3 Ja nyt, Herra, ota minun henkeni, sillä kuolema on minulle parempi kuin elämä."
\par 4 Mutta Herra sanoi: "Onko vihastumisesi oikea?"
\par 5 Niin Joona lähti kaupungista ja asettui kaupungin itäpuolelle. Hän teki itsellensä sinne lehtimajan ja kävi istumaan sen alle varjoon, kunnes näkisi, miten kaupungin oli käyvä.
\par 6 Mutta Herra Jumala toimitti risiinikasvin kasvamaan Joonan pään ylitse, varjostamaan hänen päätänsä ja päästämään häntä hänen mielipahastaan. Ja Joona iloitsi suuresti risiinikasvista.
\par 7 Mutta seuraavana päivänä, aamun sarastaessa, Jumala toimitti madon kalvamaan risiinikasvia, niin että se kuivui.
\par 8 Ja auringon noustua Jumala toimitti tulikuuman itätuulen, ja aurinko paahtoi Joonaa päähän, niin että häntä näännytti. Niin hän toivotti itsellensä kuolemaa ja sanoi: "Parempi on minulle kuolema kuin elämä".
\par 9 Mutta Jumala sanoi Joonalle: "Onko vihastumisesi risiinikasvin tähden oikea?" Tämä vastasi: "Oikea on vihastumiseni kuolemaan asti".
\par 10 Niin Herra sanoi: "Sinä armahdat risiinikasvia, josta et ole vaivaa nähnyt ja jota et ole kasvattanut, joka yhden yön lapsena syntyi ja yhden yön lapsena kuoli.
\par 11 Enkö siis minä armahtaisi Niiniveä, sitä suurta kaupunkia, jossa on enemmän kuin sata kaksikymmentä tuhatta ihmistä, jotka eivät vielä tiedä, kumpi käsi on oikea, kumpi vasen, niin myös paljon eläimiä?"


\end{document}