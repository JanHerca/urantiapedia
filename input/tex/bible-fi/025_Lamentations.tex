\begin{document}

\title{Valitusvirret}


\chapter{1}

\par 1 Kuinka istuukaan yksinänsä tuo väkirikas kaupunki! Kuinka onkaan lesken kaltaiseksi tullut se, joka oli suuri kansakuntien joukossa! Ruhtinatar maakuntien joukossa on tullut työveron alaiseksi.
\par 2 Se katkerasti itkee yössä, ja sillä on kyyneleet poskillansa. Ei ole sillä lohduttajaa kaikkien sen rakastajain seassa. Uskottomia ovat sille olleet kaikki sen ystävät, ovat sille vihamiehiksi tulleet.
\par 3 Juuda on siirtynyt maastansa kurjuutta ja työn kovuutta pakoon. Se istuu pakanakansain seassa eikä lepoa löydä. Kaikki sen vainoojat saavuttivat sen, kun se oli ahdistuksien keskellä.
\par 4 Siionin tiet surevat, sillä juhlilletulijoita ei ole. Kaikki sen portit ovat autioina, sen papit huokailevat, sen neitsyet ovat murheissaan, ja sillä itsellään on katkera mieli.
\par 5 Sen viholliset ovat voitolle päässeet, sen vihamiehet menestyvät. Sillä Herra on saattanut sen murheelliseksi sen rikkomusten paljouden tähden. Sen pienet lapset ovat menneet vankeuteen vihollisen vieminä.
\par 6 Mennyt on tytär Siionilta kaikki hänen kauneutensa. Hänen ruhtinaansa ovat kuin peurat, jotka eivät laidunta löydä; he kulkivat voimattomina vainoojan edessä.
\par 7 Jerusalem muistelee kurjuutensa päivinä ja kodittomuudessaan kaikkea kallisarvoista, mitä sillä oli muinaisista päivistä asti. Kun sen kansa sortui vihollisen käsiin eikä sillä ollut auttajaa, katsoivat viholliset sitä ja nauroivat sen turmiota.
\par 8 Jerusalem on raskaasti syntiä tehnyt, sentähden se on saastaksi tullut. Kaikki, jotka sitä kunnioittivat, halveksivat sitä nyt, kun ovat nähneet sen alastomuuden. Itsekin se huokaa ja kääntyy poispäin.
\par 9 Sen tahrat ovat sen liepeissä. Se ei ajatellut, mikä sille tuleva oli, ja niin se sortui hämmästyttävästi. Ei ole sillä lohduttajaa. Katso, Herra, minun kurjuuttani, sillä vihamies ylvästelee.
\par 10 Vihollinen levitti kätensä kaikkia sen kalleuksia kohti. Niin, sen täytyi nähdä, kuinka pakanat tunkeutuivat sen pyhäkköön, ne, jotka sinä olit kieltänyt tulemasta sinun seurakuntaasi.
\par 11 Kaikki sen kansa huokaa etsiessänsä leipää: he antavat, mitä heillä kallista on, ruuasta, saadakseen nälkänsä tyydytetyksi. Katso, Herra, ja näe, kuinka halveksittu minä olen.
\par 12 Eikö tämä koske teihin, kaikki ohikulkijat? Katsokaa ja nähkää: onko kipua, minun kipuni vertaista, joka on minun kannettavakseni pantu, jolla Herra on murehduttanut minut vihansa hehkun päivänä?
\par 13 Hän lähetti tulen korkeudesta minun luihini ja kuritti niitä. Hän viritti verkon minun jalkaini varalle, hän pani minut peräytymään, hän teki minut autioksi, ainiaan sairaaksi.
\par 14 Hänen kätensä sitoi minun rikosteni ikeen: ne kiedottiin yhteen, tulivat minun niskalleni; hän saattoi horjumaan minun voimani. Herra antoi minut niiden käsiin, joita minä en voi vastustaa.
\par 15 Herra hylkäsi kaikki minun urhoni, joita minun keskuudessani oli; hän kutsui kokoon juhlan minua vastaan murskatakseen minun nuorukaiseni. Herra polki viinikuurnan neitsyelle, tytär Juudalle.
\par 16 Näitä minä itken, minun silmäni, silmäni vuotaa vettä; sillä kaukana on minusta lohduttaja, joka virvoittaisi minun sieluani. Hävitetyt ovat minun lapseni, sillä vihamies on väkevä.
\par 17 Siion levittää käsiänsä: ei ole hänellä lohduttajaa. Herra on nostattanut Jaakobia vastaan sen viholliset joka taholta; Jerusalem on tullut saastaksi heidän keskellänsä.
\par 18 Vanhurskas on hän, Herra; sillä hänen käskyänsä vastaan minä olen niskoitellut. Kuulkaa, te kansat kaikki, ja katsokaa minun kipuani: minun neitsyeni ja nuorukaiseni ovat vankeuteen menneet.
\par 19 Minä kutsuin rakastajiani: ne pettivät minut. Minun pappini ja vanhimpani ovat kuolleet kaupungissa, kun he etsivät ruokaa tyydyttääkseen nälkäänsä.
\par 20 Katso, Herra, kuinka ahdistettu minä olen, minun sisukseni kuohuvat, sydämeni vääntyy rinnassani, sillä minä olen ollut ylen uppiniskainen. Ulkona on miekka minulta riistänyt lapset, sisällä rutto.
\par 21 He ovat kuulleet, kuinka minä huokailen. Ei ole minulla lohduttajaa. Kaikki minun vihamieheni ovat kuulleet minun onnettomuuteni; he iloitsevat, kun sinä olet tämän tehnyt: sinä olet antanut tulla sen päivän, jonka olit ilmoittanut. Mutta käyköön heidän samoin kuin minun.
\par 22 Tulkoon kaikki heidän pahuutensa sinun kasvojesi eteen, ja pane heille kannettavaksi se, minkä olet minulle pannut kaikkien minun rikoksieni tähden. Sillä monet ovat minun huokaukseni, ja minun sydämeni on sairas.

\chapter{2}

\par 1 Kuinka onkaan Herra vihassaan pilvillä peittänyt tytär Siionin! Hän heitti taivaasta maahan Israelin kunnian eikä muistanut jalkainsa astinlautaa vihansa päivänä.
\par 2 Herra on hävittänyt säälimättä kaikki Jaakobin majat, on hajottanut vihastuksissaan tytär Juudan linnoitukset, pannut ne maan tasalle. Hän on häväissyt valtakunnan ja sen ruhtinaat.
\par 3 Hän on vihan hehkussa hakannut poikki Israelin sarven kokonansa. Hän veti oikean kätensä takaisin vihamiehen edestä ja poltti Jaakobia kuin liekitsevä tuli, joka kuluttaa kaiken yltympäri.
\par 4 Hän jännitti jousensa kuin vihamies, seisoi oikea käsi koholla kuin vihollinen ja tappoi kaiken, mihin silmä oli ihastunut. Tytär Siionin majaan hän vuodatti kiivautensa kuin tulen.
\par 5 Herra on ollut niinkuin vihamies, on hävittänyt Israelin: hän hävitti kaikki sen palatsit, turmeli sen linnoitukset ja antoi tytär Juudalle paljon valitusta ja vaikerrusta.
\par 6 Hän särki aitauksensa niinkuin puutarhan aidan, hävitti juhlanviettopaikkansa. Herra on saattanut unhotuksiin Siionissa juhla-ajat ja sapatit ja on kiivaassa suuttumuksessaan pitänyt halpana kuninkaat ja papit.
\par 7 Herra on hyljännyt alttarinsa, syössyt häväistykseen pyhäkkönsä, luovuttanut vihamiehen käteen palatsiensa muurit. He nostivat huudon Herran huoneessa, huudon kuin juhlapäivänä.
\par 8 Herra oli päättänyt turmella tytär Siionin muurit: hän jännitti mittanuoran, ei pidättänyt kättänsä hävittämästä, saattoi murheeseen varustukset ja muurit; ne yhdessä nääntyvät.
\par 9 Sen portit ovat vajonneet maahan, hän poisti ja särki sen salvat. Sen kuningas ja ruhtinaat ovat pakanain seassa; lakia ei ole, eivätkä sen profeetat saa näkyjä Herralta.
\par 10 Maassa istuvat ääneti tytär Siionin vanhimmat. He ovat heittäneet tomua päänsä päälle, vyöttäytyneet säkkeihin; maata kohden ovat painaneet päänsä Jerusalemin neitsyet.
\par 11 Minun silmäni ovat itkusta hiuenneet, minun sisukseni kuohuvat, minun maksani on maahan vuodatettu tyttären, minun kansani, sortumisen tähden. Sillä lapset ja imeväiset nääntyvät kaupungin kaduilla.
\par 12 He sanovat äideillensä: "Missä on leipää ja viiniä?" kun he nääntyvät niinkuin kaatuneet kaupungin kaduilla, heittävät henkensä äitiensä syliin.
\par 13 Minkä sinulle mainitsisin, mihin vertaisin sinua, tytär Jerusalem? Minkä asettaisin rinnallesi lohduttaakseni sinua, neitsyt, tytär Siion? Sillä suuri niinkuin meri on sinun sortumisesi; kuka voi sinut parantaa?
\par 14 Profeettasi ovat sinulle nähneet petollisia, äiteliä näkyjä. Eivät he ole paljastaneet sinun syntiäsi, niin että olisivat kääntäneet sinun kohtalosi, vaan ovat nähneet sinulle petollisia, eksyttäväisiä ennustuksia.
\par 15 Sinulle paukuttavat kämmeniänsä kaikki ohikulkijat, he viheltävät ja nyökyttävät ilkkuen päätänsä tytär Jerusalemille: "Tämäkö on kaupunki, jota sanottiin kauneuden täydellisyydeksi, kaiken maan ihastukseksi?"
\par 16 Suu ammollaan sinua vastaan ovat kaikki sinun vihamiehesi. He viheltävät, kiristelevät hampaitaan ja sanovat: "Me olemme sen hävittäneet. Tämä on juuri se päivä, jota olemme toivoneet; se on meille tullut, olemme sen nähneet."
\par 17 Herra on tehnyt, mitä oli aikonut. Hän on täyttänyt sanansa, sen, mitä oli päättänyt muinaisista päivistä asti: hän on repinyt alas säälimättä, on antanut vihamiesten iloita sinusta ja kohottanut sinun vihollistesi sarven.
\par 18 Heidän sydämensä huutaa Herran puoleen. Sinä, tytär Siionin muuri, anna kyyneltesi virtana vuotaa yötä päivää! Älä suo itsellesi lepoa, älköön silmäteräsi ummistuko.
\par 19 Nouse, kohota valitushuuto yöllä, kun alkavat yövartiot. Anna sydämesi vuotaa kuin vesi Herran kasvojen edessä. Kohota kätesi häntä kohden pienten lastesi elämän puolesta, kun ne nääntyvät nälkään kaikkien katujen kulmissa.
\par 20 Katso, Herra, ja tarkkaa, ketä olet antanut tämän kohdata: pitäisikö vaimojen syödä oma hedelmänsä, vaalimansa pienet lapset; pitäisikö Herran pyhäkössä tapettaman papit ja profeetat?
\par 21 Maassa, kaduilla, makaa nuorta ja vanhaa; minun neitsyeni ja nuorukaiseni ovat kaatuneet miekkaan. Sinä olet surmannut vihasi päivänä, olet teurastanut säälimättä.
\par 22 Sinä kutsuit kuin juhlapäivän viettoon minun peljättäjäni joka taholta; eikä jäänyt Herran vihan päivänä pelastunutta, ei pakoonpäässyttä. Jotka minä olin vaalinut ja isoiksi saanut, ne minun vihamieheni lopetti.

\chapter{3}

\par 1 Minä olen se mies, joka olen kurjuutta nähnyt hänen vihastuksensa vitsan alla.
\par 2 Minut hän on johdattanut ja kuljettanut pimeyteen eikä valkeuteen.
\par 3 Juuri minua vastaan hän kääntää kätensä, yhäti, kaiken päivää.
\par 4 Hän on kalvanut minun lihani ja nahkani, musertanut minun luuni.
\par 5 Hän on rakentanut varustukset minua vastaan ja piirittänyt minut myrkyllä ja vaivalla.
\par 6 Hän on pannut minut asumaan pimeydessä niinkuin ikiaikojen kuolleet.
\par 7 Hän on tehnyt muurin minun ympärilleni, niin etten pääse ulos, on pannut minut raskaisiin vaskikahleisiin.
\par 8 Vaikka minä huudan ja parun, hän vaientaa minun rukoukseni.
\par 9 Hakatuista kivistä hän on tehnyt minun teilleni muurin, on mutkistanut minun polkuni.
\par 10 Vaaniva karhu on hän minulle, piilossa väijyvä leijona.
\par 11 Hän on vienyt harhaan minun tieni ja repinyt minut kappaleiksi, hän on minut autioksi tehnyt.
\par 12 Hän on jännittänyt jousensa ja asettanut minut nuoltensa maalitauluksi.
\par 13 Hän on ampunut munuaisiini nuolet, viinensä lapset.
\par 14 Minä olen joutunut koko kansani nauruksi, heidän jokapäiväiseksi pilkkalauluksensa.
\par 15 Hän on ravinnut minua katkeruudella, juottanut minua koiruoholla.
\par 16 Hän on purettanut minulla hampaat rikki soraan, painanut minut alas tomuun.
\par 17 Sinä olet syössyt minun sieluni ulos, rauhasta pois, minä olen unhottanut onnen.
\par 18 Ja minä sanon: mennyt on minulta kunnia ja Herran odotus.
\par 19 Muista minun kurjuuttani ja kodittomuuttani, koiruohoa ja myrkkyä.
\par 20 Sinä kyllä muistat sen, että minun sieluni on alaspainettu.
\par 21 Tämän minä painan sydämeeni, sentähden minä toivon.
\par 22 Herran armoa on, ettemme ole aivan hävinneet, sillä hänen laupeutensa ei ole loppunut:
\par 23 se on joka aamu uusi, ja suuri on hänen uskollisuutensa.
\par 24 Minun osani on Herra, sanoo minun sieluni; sentähden minä panen toivoni häneen.
\par 25 Hyvä on Herra häntä odottaville, sille sielulle, joka häntä etsii.
\par 26 Hyvä on hiljaisuudessa toivoa Herran apua.
\par 27 Hyvä on miehelle, että hän kantaa iestä nuoruudessaan.
\par 28 Istukoon hän yksin ja hiljaa, kun Herra on sen hänen päällensä pannut.
\par 29 Laskekoon suunsa tomuun - ehkä on vielä toivoa.
\par 30 Ojentakoon hän posken sille, joka häntä lyö, saakoon kyllälti häväistystä.
\par 31 Sillä ei Herra hylkää iankaikkisesti;
\par 32 vaan jos hän on murheelliseksi saattanut, hän osoittaa laupeutta suuressa armossansa.
\par 33 Sillä ei hän sydämensä halusta vaivaa eikä murehduta ihmislapsia.
\par 34 Kun jalkojen alle poljetaan kaikki vangit maassa,
\par 35 kun väännetään miehen oikeutta Korkeimman kasvojen edessä,
\par 36 kun ihmiselle tehdään vääryyttä hänen riita-asiassaan - eikö Herra sitä näkisi?
\par 37 Onko kukaan sanonut, ja se on tapahtunut, jos ei Herra ole käskenyt?
\par 38 Eikö lähde Korkeimman suusta paha ja hyvä?
\par 39 Miksi tuskittelee ihminen eläessään, mies syntiensä palkkaa?
\par 40 Koetelkaamme teitämme, tutkikaamme niitä ja palatkaamme Herran tykö.
\par 41 Kohottakaamme sydämemme ynnä kätemme Jumalan puoleen, joka on taivaassa.
\par 42 Me olemme luopuneet pois ja olleet kapinalliset; sinä et ole antanut anteeksi,
\par 43 olet peittänyt itsesi vihassasi, ajanut meitä takaa, surmannut säälimättä;
\par 44 olet peittänyt itsesi pilvellä, niin ettei rukous pääse lävitse.
\par 45 Tunkioksi ja hylyksi sinä olet meidät tehnyt kansojen seassa.
\par 46 Suut ammollaan meitä vastaan ovat kaikki meidän vihamiehemme.
\par 47 Osaksemme on tullut kauhu ja kuoppa, turmio ja sortuminen.
\par 48 Vesipurot juoksevat minun silmistäni tyttären, minun kansani, sortumisen tähden.
\par 49 Minun silmäni vuotaa lakkaamatta, hellittämättä
\par 50 siihen asti, kunnes katsoo, kunnes näkee Herra taivaasta.
\par 51 Silmäni tuottaa tuskaa minun sielulleni kaikkien minun kaupunkini tyttärien tähden.
\par 52 Kiihkeästi pyydystivät minua kuin lintua ne, jotka syyttä ovat vihamiehiäni.
\par 53 He sulkivat kuoppaan minun elämäni ja heittivät päälleni kiviä.
\par 54 Vedet tulvivat minun pääni ylitse; minä sanoin: olen hukassa.
\par 55 Minä huusin sinun nimeäsi, Herra, kuopan syvyydestä.
\par 56 Sinä kuulit minun huutoni: "Älä peitä korvaasi minun avunhuudoltani, että saisin hengähtää".
\par 57 Sinä olit läsnä silloin, kun minä sinua huusin; sinä sanoit: "Älä pelkää".
\par 58 Sinä, Herra, ajoit minun riita-asiani, lunastit minun henkeni.
\par 59 Olethan nähnyt, Herra, minun kärsimäni sorron: hanki minulle oikeus.
\par 60 Olethan nähnyt kaiken heidän kostonhimonsa, kaikki heidän juonensa minua vastaan.
\par 61 Sinä olet kuullut heidän häväistyksensä, Herra, kaikki heidän juonensa minua vastaan.
\par 62 Minun vastustajaini huulet ja heidän aikeensa ovat minua vastaan kaiken päivää.
\par 63 Istuivatpa he tai nousivat, katso: minä olen heillä pilkkalauluna.
\par 64 Kosta heille, Herra, heidän kättensä teot.
\par 65 Paaduta heidän sydämensä, kohdatkoon heitä sinun kirouksesi.
\par 66 Aja heitä takaa vihassasi ja hävitä heidät Herran taivaan alta.

\chapter{4}

\par 1 Kuinka onkaan kulta tummunut, muuttunut hyvä kulta; kuinka ovat pyhät kivet viskeltyinä kaikkien katujen kulmiin!
\par 2 Siionin pojat, nuo kalliit, punnitut puhtaimman kullan arvoisiksi - kuinka he ovatkaan saviastiain arvossa, savenvalajan kätten tekojen!
\par 3 Aavikkosudetkin taritsevat nisiänsä, imettävät pentujansa; mutta tytär, minun kansani, on tullut tylyksi kuin kamelikurki erämaassa.
\par 4 Imeväisen kieli tarttuu suulakeen janon tähden. Lapsukaiset pyytävät leipää; ei ole, kuka sitä heille taittaisi.
\par 5 Jotka herkkuja söivät, ne nääntyvät kaduilla. Joita punapurppuran päällä kanneltiin, ne tunkioita syleilevät.
\par 6 Tyttären, minun kansani, syntivelka on Sodoman syntiä suurempi; Sodoma hävitettiin yhtäkkiä kätten siellä riehumatta.
\par 7 Siionin ruhtinaat olivat lunta puhtaammat, maitoa valkoisemmat, heidän ruumiinsa oli koralleja rusottavampi, heidän hahmonsa oli kuin safiiri.
\par 8 Nyt on heidän muotonsa nokea mustempi, ei voi heitä tuntea kaduilla. Rypyssä on heillä nahka luitten päällä, se on kuivettunut kuin puu.
\par 9 Parempi oli miekan kaatamien kuin nälän kaatamien, jotka menehtyivät, kuin lävitsepistetyt, pellon viljaa vailla.
\par 10 Armeliaat vaimot keittivät omin käsin lapsiansa: ne tulivat heille ruuaksi tyttären, minun kansani, sortuessa.
\par 11 Herra on pannut täytäntöön kiivautensa, vuodattanut vihansa hehkun; hän on sytyttänyt Siioniin tulen, joka on kuluttanut sen perustukset.
\par 12 Eivät olisi uskoneet maan kuninkaat, ei maanpiirin asukkaista kenkään, että vihollinen ja vainomies hyökkää sisään Jerusalemin porteista.
\par 13 Sen profeettain syntien tähden kävi näin, sen pappien pahain tekojen tähden, niiden, jotka siellä olivat vuodattaneet vanhurskaitten verta.
\par 14 He harhailivat sokeina kaduilla, verellä tahrattuina, niin ettei voinut koskea heidän vaatteisiinsa.
\par 15 "Väistykää! Saastainen!" huudettiin heistä. "Väistykää, väistykää, älkää koskeko!" Paettuaankin he yhä harhailivat; pakanain seassa sanottiin: "Eivät he saa kauemmin asustaa täällä".
\par 16 Herran kasvot ovat hajottaneet heidät, hän ei heihin enää katso. Papeista ei välitetty, vanhimpia ei armahdettu.
\par 17 Vieläkin me, silmät rauenneina, turhaan odotimme apua; tähystyspaikastamme me tähyilimme kansaa, josta ei pelastusta tullut.
\par 18 Meidän askeleitamme vaanittiin, niin ettemme voineet kulkea kaduillamme. Meidän loppumme lähestyi, päivämme täyttyivät - niin, loppumme tuli.
\par 19 Meidän vainoojamme olivat nopeammat kuin kotkat taivaalla. Vuorilla he ajoivat meitä takaa, väijyivät meitä erämaassa.
\par 20 Hän, meidän elämänhenkemme, Herran voideltu, joutui vangiksi heidän kuoppiinsa, hän, josta me olimme sanoneet: hänen varjossaan me saamme elää pakanakansain seassa.
\par 21 Iloitse ja riemuitse, tytär Edom, joka asut Uusin maassa! Mutta malja on tuleva sinunkin kohdallesi: sinä juovut ja paljastat itsesi.
\par 22 Sinun syntivelkasi, tytär Siion, on loppunut: ei Herra ole enää siirtävä sinua pois. Sinun syntivelkasi, tytär Edom, hän on etsiskelevä, on paljastava sinun syntisi.

\chapter{5}

\par 1 Muista, Herra, mitä meille on tapahtunut; katso ja näe meidän häväistyksemme.
\par 2 Meidän perintöosamme on siirtynyt vieraille, meidän talomme muukalaisille.
\par 3 Me olemme tulleet orvoiksi, isättömiksi, meidän äitimme ovat kuin lesket.
\par 4 Oman juomavetemme me ostamme rahalla; omat puumme saamme, jos maksamme hinnan.
\par 5 Vainoojamme ovat meidän niskassamme; kun uuvumme, ei meille lepoa suoda.
\par 6 Egyptille me lyömme kättä ja Assurille saadaksemme leipää ravinnoksi.
\par 7 Meidän isämme ovat syntiä tehneet; heitä ei enää ole. Me kannamme heidän syntivelkaansa.
\par 8 Orjat hallitsevat meitä; ei ole sitä, joka tempaisi meidät heidän käsistänsä.
\par 9 Henkemme kaupalla me noudamme leipämme, väistäen miekkaa erämaassa.
\par 10 Meidän ihomme halkeilee kuin uuni nälän poltteiden takia.
\par 11 Naisia raiskataan Siionissa, neitsyitä Juudan kaupungeissa.
\par 12 Ruhtinaita heidän kätensä hirttävät, vanhinten kasvoja ei pidetä arvossa.
\par 13 Nuorukaiset kantavat myllynkiviä, poikaset kompastelevat puutaakkojen alla.
\par 14 Poissa ovat vanhukset porteista, nuorukaiset kielisoittimiensa äärestä.
\par 15 Poissa on ilo sydämistämme, karkelomme on valitukseksi muuttunut.
\par 16 Pudonnut on päästämme kruunu. Voi meitä, sillä me olemme syntiä tehneet!
\par 17 Tästä syystä on sydämemme tullut sairaaksi, näitten tähden ovat silmämme pimenneet -
\par 18 Siionin vuoren tähden, joka on autiona, jolla ketut juoksentelevat.
\par 19 Sinä, Herra, hallitset iankaikkisesti, sinun valtaistuimesi pysyy suvusta sukuun.
\par 20 Miksi unhotat meidät ainiaaksi, hylkäät meidät ikipäiviksi?
\par 21 Palauta meidät, Herra, tykösi, niin me palajamme; uudista meidän päivämme muinaiselleen.
\par 22 Vai oletko meidät peräti hyljännyt, vihastunut meihin ylenmäärin?


\end{document}