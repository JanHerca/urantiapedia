\begin{document}

\title{Toinen aikakirja}


\chapter{1}

\par 1 Salomo, Daavidin poika, vahvistui kuninkuudessansa, ja Herra, hänen Jumalansa, oli hänen kanssansa ja teki hänet ylen suureksi.
\par 2 Ja Salomo antoi kutsun koko Israelille, tuhannen- ja sadanpäämiehille, tuomareille ja kaikille ruhtinaille koko Israelissa, perhekunta-päämiehille;
\par 3 ja niin Salomo ja koko seurakunta hänen kanssaan menivät uhrikukkulalle, joka oli Gibeonissa, sillä siellä oli Jumalan ilmestysmaja, jonka Herran palvelija Mooses oli tehnyt erämaassa.
\par 4 Mutta Jumalan arkin oli Daavid tuonut Kirjat-Jearimista siihen paikkaan, jonka Daavid oli sille valmistanut; sillä hän oli pystyttänyt sille majan Jerusalemiin.
\par 5 Ja vaskialttari, jonka Besalel, Uurin poika, Huurin pojanpoika, oli rakentanut, oli siellä Herran asumuksen edessä; Salomo ja seurakunta etsivät häntä siellä.
\par 6 Ja Salomo uhrasi siellä Herran edessä ilmestysmajan vaskialttarilla; hän uhrasi sen päällä tuhat polttouhria.
\par 7 Sinä yönä Jumala ilmestyi Salomolle ja sanoi hänelle: "Ano, mitä tahdot minun sinulle antavan".
\par 8 Salomo vastasi Jumalalle: "Sinä olet tehnyt suuren laupeuden minun isälleni Daavidille ja tehnyt minut kuninkaaksi hänen sijaansa.
\par 9 Niin toteutukoon nyt, Herra Jumala, sinun isälleni Daavidille antamasi sana; sillä sinä olet pannut minut hallitsemaan kansaa, jota on paljon niinkuin tomua maassa.
\par 10 Anna siis minulle viisaus ja taito lähteä ja tulla tämän kansan edellä, sillä kuka voi muuten tätä sinun suurta kansaasi tuomita?"
\par 11 Ja Jumala sanoi Salomolle: "Koska sinulla on tämä mieli etkä anonut rikkautta, tavaraa ja kunniaa, et vihamiestesi henkeä, etkä myöskään anonut pitkää ikää, vaan anoit itsellesi viisautta ja taitoa tuomitaksesi minun kansaani, jonka kuninkaaksi minä olen sinut tehnyt,
\par 12 niin annetaan sinulle viisaus ja taito; ja lisäksi minä annan sinulle rikkautta, tavaraa ja kunniaa, niin ettei sitä ole ollut niin paljoa kenelläkään kuninkaalla ennen sinua eikä tule olemaan sinun jälkeesi".
\par 13 Käytyään uhrikukkulalla, joka oli Gibeonissa, Salomo palasi ilmestysmajalta Jerusalemiin ja hallitsi Israelia.
\par 14 Ja Salomo kokosi sotavaunuja ja ratsumiehiä, niin että hänellä oli tuhannet neljätsadat sotavaunut ja kaksitoista tuhatta ratsumiestä. Ne hän sijoitti vaunukaupunkeihin ja kuninkaan luo Jerusalemiin.
\par 15 Ja kuningas toimitti niin, että Jerusalemissa oli hopeata ja kultaa kuin kiviä, ja setripuuta niin paljon kuin metsäviikunapuita Alankomaassa.
\par 16 Ja hevoset, mitä Salomolla oli, tuotiin Egyptistä ja Kuvesta; kuninkaan kauppiaat noutivat niitä Kuvesta maksua vastaan.
\par 17 Egyptistä tuodut vaunut maksoivat kuusisataa hopeasekeliä ja hevonen sata viisikymmentä. Samoin tuotiin niitä heidän välityksellään kaikille heettiläisten ja aramilaisten kuninkaille.

\chapter{2}

\par 1 Ja Salomo käski ryhtyä rakentamaan temppeliä Herran nimelle ja kuninkaallista linnaa itsellensä.
\par 2 Ja Salomo määräsi seitsemänkymmentä tuhatta miestä taakankantajiksi ja kahdeksankymmentä tuhatta miestä kivenhakkaajiksi vuoristoon ja niille kolmetuhatta kuusisataa työnjohtajaa.
\par 3 Sitten Salomo lähetti Huuramille, Tyyron kuninkaalle, sanan: "Tee minulle sama, minkä teit isälleni Daavidille, jolle lähetit setripuita, että hän rakentaisi itsellensä linnan asuaksensa siinä.
\par 4 Katso, minä rakennan temppelin Herran, Jumalani, nimelle, pyhittääkseni sen hänelle, että siinä poltettaisiin hyvänhajuista suitsutusta hänen edessänsä, pidettäisiin aina esillä näkyleipiä ja uhrattaisiin polttouhreja aamuin ja illoin, sapatteina, uudenkuun päivinä ja Herran, meidän Jumalamme, juhlina; tämä on Israelin ikuinen velvollisuus.
\par 5 Ja temppeli, jonka minä rakennan, on oleva suuri; sillä meidän Jumalamme on suurin kaikista jumalista.
\par 6 Kuka kykenisi rakentamaan hänelle huoneen? Taivaisiin ja taivasten taivaisiin hän ei mahdu. Mikä siis olen minä rakentamaan hänelle temppeliä muuta varten kuin uhratakseni hänen edessänsä?
\par 7 Niin lähetä nyt minulle mies, joka on taitava tekemään kulta-, hopea-, vaski- ja rautatöitä, niin myös töitä purppuranpunaisista, helakanpunaisista ja punasinisistä langoista, ja joka osaa veistää veistoksia, työskentelemään yhdessä niiden taitajien kanssa, joita minulla on Juudassa ja Jerusalemissa ja jotka minun isäni Daavid on hankkinut.
\par 8 Ja lähetä minulle setripuita, kypressipuita ja santelipuita Libanonilta, sillä minä tiedän, että sinun palvelijasi osaavat hakata Libanonin puita; ja katso, minun palvelijani olkoot sinun palvelijaisi kanssa.
\par 9 Minulle on hankittava paljon puuta, sillä temppeli, jonka minä rakennan, on oleva suuri ja ihmeellinen.
\par 10 Ja katso, minä annan kirvesmiehille, puitten hakkaajille, kaksikymmentä tuhatta koor-mittaa nisuja, sinun palvelijaisi ravinnoksi, ja kaksikymmentä tuhatta koor-mittaa ohria, kaksikymmentä tuhatta bat-mittaa viiniä ja kaksikymmentä tuhatta bat-mittaa öljyä."
\par 11 Huuram, Tyyron kuningas, vastasi kirjeellä, jonka hän lähetti Salomolle: "Sentähden, että Herra rakastaa kansaansa, on hän asettanut sinut heidän kuninkaakseen".
\par 12 Ja Huuram sanoi vielä: "Kiitetty olkoon Herra, Israelin Jumala, joka on tehnyt taivaan ja maan, siitä että hän on antanut kuningas Daavidille viisaan pojan, jolla on älyä ja ymmärrystä rakentaa temppeli Herralle ja kuninkaallinen linna itsellensä.
\par 13 Ja nyt minä lähetän taitavan, ymmärtäväisen miehen, Huuram-Aabin,
\par 14 joka on daanilaisen vaimon poika ja jonka isä on tyyrolainen. Hän osaa tehdä kulta- ja hopea-, vaski- ja rauta-, kivi- ja puutöitä, niin myös töitä purppuranpunaisista ja punasinisistä langoista, valkoisista pellavalangoista ja helakanpunaisista langoista. Hän osaa veistää veistoksia ja sommitella kaikkinaisia taideteoksia, joita hänen tehtäväkseen annetaan, yhdessä sinun taitomiestesi ja herrani, sinun isäsi Daavidin, taitomiesten kanssa.
\par 15 Lähettäköön siis herrani palvelijoillensa nisut, ohrat, öljyn ja viinin, niinkuin on sanonut.
\par 16 Silloin me hakkaamme puita Libanonilta niin paljon, kuin sinä tarvitset, ja me kuljetamme ne lautoissa meritse sinulle Jaafoon. Toimita sinä ne sitten ylös Jerusalemiin."
\par 17 Ja Salomo luetti kaikki muukalaiset miehet Israelin maassa, senjälkeen kuin hänen isänsä Daavid oli heidät luettanut. Ja heitä huomattiin olevan sata viisikymmentäkolme tuhatta kuusisataa.
\par 18 Näistä hän teki seitsemänkymmentä tuhatta taakankantajiksi, kahdeksankymmentä tuhatta kivenhakkaajiksi vuoristoon ja kolmetuhatta kuusisataa työnjohtajiksi, joiden oli pidettävä väki työssä.

\chapter{3}

\par 1 Sitten Salomo alkoi rakentaa Herran temppeliä Jerusalemiin, Moorian vuorelle, jossa Herra oli ilmestynyt hänen isällensä Daavidille, siihen paikkaan, jonka Daavid oli valmistanut, jebusilaisen Ornanin puimatantereelle.
\par 2 Hän alkoi rakentaa toisen kuun toisena päivänä, neljäntenä hallitusvuotenansa.
\par 3 Tällaisen perustuksen Salomo laski rakennettavalle Jumalan temppelille: pituus oli, vanhan mitan mukaan, kuusikymmentä kyynärää, leveys kaksikymmentä kyynärää.
\par 4 Eteinen, temppelin itäpäässä, oli yhtä pitkä kuin temppeli leveä, kahtakymmentä kyynärää, ja sen korkeus oli kaksikymmentä kyynärää; ja hän päällysti sen sisältä puhtaalla kullalla.
\par 5 Suuren huoneen hän laudoitti sisältä kypressipuulla, jonka hän silasi parhaalla kullalla ja johon hän laittoi palmuja ja vitjoja.
\par 6 Ja hän koristi huoneen kalliilla kivillä. Ja kulta oli Parvaimin kultaa.
\par 7 Ja hän silasi huoneen, palkit, kynnykset, seinät ja ovet, kullalla ja kaiverrutti kerubeja seiniin.
\par 8 Ja hän teki kaikkeinpyhimmän huoneen. Se oli yhtä pitkä kuin temppeli leveä, kahtakymmentä kyynärää, ja myös kahtakymmentä kyynärää leveä. Ja hän silasi sen parhaalla kullalla, joka painoi kuusisataa talenttia.
\par 9 Ja kultanaulojen paino oli viisikymmentä sekeliä. Myöskin yläsalit hän silasi kullalla.
\par 10 Kaikkein pyhimpään huoneeseen hän teki kaksi kerubia, taidollisesti tehtyä, ja päällysti ne kullalla.
\par 11 Kerubien siipien pituus oli yhteensä kaksikymmentä kyynärää. Toisen kerubin toinen siipi oli viittä kyynärää pitkä ja kosketti huoneen toista seinää, ja sen toinen siipi oli viittä kyynärää pitkä ja ulottui toisen kerubin siipeen.
\par 12 Ja toisen kerubin toinen siipi oli viittä kyynärää pitkä ja ulottui huoneen toiseen seinään, ja sen toinen siipi oli viittä kyynärää pitkä ja oli kiinni toisen kerubin siivessä.
\par 13 Näin nämä kerubit levittivät siipensä kahtakymmentä kyynärää laajalti; ja ne seisoivat jaloillaan, kasvot käännettyinä huoneeseen päin.
\par 14 Ja hän teki esiripun punasinisistä, purppuranpunaisista ja helakanpunaisista langoista ja valkoisista pellavalangoista ja laittoi siihen kerubeja.
\par 15 Ja hän teki temppelin eteen kaksi pylvästä, kolmenkymmenen viiden kyynärän korkuista; ja pylväänpää, joka oli niiden päässä, oli viittä kyynärää korkea.
\par 16 Ja hän teki vitjoja kaikkeinpyhimpään, ja hän pani niitä pylväitten päähän; ja hän teki sata granaattiomenaa ja asetti ne vitjoihin.
\par 17 Ja hän pystytti pylväät temppelin eteen, toisen oikealle puolelle ja toisen vasemmalle puolelle: oikeanpuoliselle hän antoi nimen Jaakin ja vasemmanpuoliselle nimen Booas.

\chapter{4}

\par 1 Hän teki alttarin vaskesta, kahtakymmentä kyynärää pitkän, kahtakymmentä kyynärää leveän ja kymmentä kyynärää korkean.
\par 2 Hän teki myös meren, valetun, kymmentä kyynärää leveän reunasta reunaan, ympärinsä pyöreän, ja viittä kyynärää korkean; ja kolmenkymmenen kyynärän pituinen mittanuora ulottui sen ympäri.
\par 3 Ja sen alaosassa oli yltympäri raavaankuvia, jotka kulkivat sen ympäri; ne ympäröivät merta yltympäri, kymmenen kullakin kyynärällä. Raavaita oli kahdessa rivissä, valettuina meren kanssa yhteen.
\par 4 Ja se seisoi kahdentoista raavaan varassa, joista kolme oli käännettynä pohjoiseen, kolme länteen, kolme etelään ja kolme itään päin; meri oli niiden yläpuolella, niiden varassa, ja kaikkien niiden takapuolet olivat sisään päin.
\par 5 Se oli kämmenen paksuinen, ja sen reuna oli maljan reunan kaltainen, puhjenneen liljan muotoinen. Siihen mahtui, se veti kolmetuhatta bat-mittaa.
\par 6 Hän teki myös kymmenen allasta ja asetti viisi oikealle puolelle ja viisi vasemmalle puolelle pesemistä varten; sillä niissä huuhdottiin se, mikä kuului polttouhriin. Mutta meri oli pappien peseytymistä varten.
\par 7 Hän teki myös kymmenen kultaista lampunjalkaa, niinkuin niistä oli säädetty, ja pani ne temppelisaliin, viisi oikealle puolelle ja viisi vasemmalle puolelle.
\par 8 Hän teki myös kymmenen pöytää ja asetti ne temppelisaliin, viisi oikealle puolelle ja viisi vasemmalle puolelle. Hän teki myös sata kultamaljaa.
\par 9 Ja hän teki pappien esipihan ja suuren esikartanon sekä esikartanon ovet; ja ovet hän päällysti vaskella.
\par 10 Ja meren hän asetti oikealle sivulle, kaakkoa kohti.
\par 11 Huuram teki myös kattilat, lapiot ja maljat. Ja niin Huuram sai suoritetuksi työn, mikä hänen oli tehtävä kuningas Salomolle Jumalan temppeliin:
\par 12 kaksi pylvästä ja kaksi palloa, pylväänpäätä, pylväiden päähän, ja kaksi ristikkokoristetta peittämään kahta pallonmuotoista pylväänpäätä, jotka olivat pylväitten päässä,
\par 13 ja neljäsataa granaattiomenaa kahteen ristikkokoristeeseen, kaksi riviä granaattiomenia kumpaankin ristikkokoristeeseen, peittämään kahta pallonmuotoista pylväänpäätä, jotka olivat pylväitten päällä.
\par 14 Ja hän teki telineet ja teki altaat telineitten päälle,
\par 15 Ja yhden meren ja kaksitoista raavasta sen alle.
\par 16 Ja kattilat, lapiot ja haarukat ynnä kaikki niihin kuuluvat kalut Huuram-Aabiv teki kuningas Salomolle Herran temppeliin kiilloitetusta vaskesta.
\par 17 Jordanin lakeudella kuningas ne valatti savimuotteihin, Sukkotin ja Seredan välillä.
\par 18 Ja Salomo teetti kaikkia näitä kaluja ylen paljon, sillä vasken painoa ei määrätty.
\par 19 Salomo teetti myös kaikki kalut, joita tuli olla Herran temppelissä: kulta-alttarin, pöydät, joilla näkyleivät olivat,
\par 20 lampunjalat lamppuineen, jotka oli sytytettävä säädetyllä tavalla kaikkeinpyhimmän eteen, puhtaasta kullasta,
\par 21 kultaisine kukkalehtineen, lamppuineen ja lamppusaksineen - puhtainta kultaa;
\par 22 veitset, maljat, kupit ja hiilipannut puhtaasta kullasta; ja temppelin oviaukkojen sisemmät ovet, jotka veivät kaikkeinpyhimpään, sekä ne temppelin ovet, jotka veivät temppelisaliin, kullasta.

\chapter{5}

\par 1 Kun kaikki työ, minkä Salomo teetti Herran temppeliin, oli valmis, vei Salomo sinne isänsä Daavidin pyhät lahjat. Hopean, kullan ja kaikki kalut hän pani Jumalan temppelin aarrekammioihin.
\par 2 Sitten Salomo kokosi Israelin vanhimmat ja kaikki sukukuntien johtomiehet, israelilaisten perhekunta-päämiehet, Jerusalemiin, tuomaan Herran liitonarkkia Daavidin kaupungista, se on Siionista.
\par 3 Niin kokoontuivat kuninkaan luo kaikki Israelin miehet juhlapäivänä, joka on seitsemännessä kuussa.
\par 4 Ja kun kaikki Israelin vanhimmat olivat tulleet saapuville, nostivat leeviläiset arkin,
\par 5 ja he veivät arkin ja ilmestysmajan sinne, sekä kaiken pyhän kaluston, joka oli majassa; leeviläiset papit veivät ne sinne.
\par 6 Ja kuningas Salomo seisoi arkin edessä ynnä koko Israelin kansa, joka oli kokoontunut hänen luoksensa; ja he uhrasivat lampaita ja raavaita niin paljon, että niitä ei voitu luetella eikä laskea.
\par 7 Ja papit toivat Herran liitonarkin paikoilleen temppelin kuoriin, kaikkeinpyhimpään, kerubien siipien alle.
\par 8 Sillä kerubit levittivät siipensä sen paikan ylitse, missä arkki oli, ja niin kerubit peittivät ylhäältä päin arkin ja sen korennot.
\par 9 Ja korennot olivat niin pitkät, että niiden arkista ulkonevat päät voi nähdä kaikkeinpyhimmän edustalta, mutta ulkoa niitä ei voinut nähdä. Ja ne jäivät sinne tähän päivään asti.
\par 10 Arkissa ei ollut muuta kuin ne kaksi taulua, jotka Mooses oli pannut sinne Hoorebilla, kun Herra oli tehnyt liiton israelilaisten kanssa, heidän lähdettyänsä Egyptistä.
\par 11 Kun papit lähtivät pyhäköstä - sillä kaikki siellä olevat papit olivat pyhittäneet itsensä, osastoihin katsomatta;
\par 12 ja kaikki leeviläiset veisaajat, Aasaf, Heeman ja Jedutun poikinensa ja veljinensä, seisoivat hienoihin pellavavaatteisiin puettuina kymbaaleineen, harppuineen ja kanteleineen itään päin alttarista, ja heidän kanssaan sata kaksikymmentä pappia, jotka puhalsivat torviin;
\par 13 ja puhaltajien ja veisaajien oli yhdyttävä yhtaikaa ja yhteen ääneen ylistämään ja kiittämään Herraa - ja kun torvet, kymbaalit ja muut soittokoneet soivat ja viritettiin Herran ylistys: "Sillä hän on hyvä, sillä hänen armonsa pysyy iankaikkisesti", silloin pilvi täytti huoneen, Herran temppelin,
\par 14 niin että papit eivät voineet astua toimittamaan virkaansa pilven tähden; sillä Herran kirkkaus täytti Jumalan temppelin.

\chapter{6}

\par 1 Silloin Salomo sanoi: "Herra on sanonut tahtovansa asua pimeässä.
\par 2 Mutta minä olen rakentanut huoneen sinulle asunnoksi, asuinsijan, asuaksesi siinä iäti."
\par 3 Sitten kuningas käänsi kasvonsa ja siunasi koko Israelin seurakunnan; ja koko Israelin seurakunta seisoi.
\par 4 Hän sanoi: "Kiitetty olkoon Herra, Israelin Jumala, joka kädellänsä on täyttänyt sen, mitä hän suullansa puhui minun isälleni Daavidille, sanoen:
\par 5 'Siitä päivästä saakka, jona minä vein kansani Egyptistä, en minä ole mistään Israelin sukukunnasta valinnut yhtään kaupunkia, että siihen rakennettaisiin temppeli, missä minun nimeni asuisi, enkä myöskään ole valinnut ketään olemaan kansani Israelin ruhtinaana.
\par 6 Mutta Jerusalemin minä olen valinnut nimeni asuinsijaksi, ja Daavidin minä olen valinnut vallitsemaan kansaani Israelia.'
\par 7 Ja minun isäni Daavid aikoi rakentaa temppelin Herran, Israelin Jumalan, nimelle.
\par 8 Mutta Herra sanoi minun isälleni Daavidille: 'Kun sinä aiot rakentaa temppelin minun nimelleni, niin tosin teet siinä hyvin, että sitä aiot;
\par 9 kuitenkaan et sinä ole sitä temppeliä rakentava, vaan sinun poikasi, joka lähtee sinun kupeistasi, hän on rakentava temppelin minun nimelleni'.
\par 10 Ja Herra on täyttänyt sanansa, minkä hän puhui: minä olen noussut isäni Daavidin sijalle ja istun Israelin valtaistuimella, niinkuin Herra puhui, ja minä olen rakentanut temppelin Herran, Israelin Jumalan, nimelle.
\par 11 Ja minä olen sijoittanut siihen arkin, jossa on se Herran liitto, minkä hän teki israelilaisten kanssa."
\par 12 Sitten hän astui Herran alttarin eteen koko Israelin seurakunnan nähden ja ojensi kätensä.
\par 13 Salomo oli näet teettänyt lavan vaskesta, viittä kyynärää pitkän, viittä kyynärää leveän ja kolmea kyynärää korkean, ja asettanut sen esikartanon keskelle. Sille hän nousi, polvistui koko Israelin seurakunnan nähden, ojensi kätensä taivasta kohti
\par 14 ja sanoi: "Herra, Israelin Jumala, ei ole sinun vertaistasi Jumalaa, ei taivaassa eikä maan päällä, sinun, joka pidät liiton ja säilytät laupeuden palvelijoitasi kohtaan, jotka vaeltavat sinun edessäsi kaikesta sydämestänsä.
\par 15 Sinä olet pitänyt, mitä puhuit palvelijallesi Daavidille, minun isälleni. Minkä sinä suullasi puhuit, sen sinä kädelläsi täytit, niinkuin nyt on tapahtunut.
\par 16 Niin pidä nytkin, Herra, Israelin Jumala, mitä puhuit palvelijallesi Daavidille, minun isälleni, sanoen: 'Aina on mies sinun suvustasi oleva minun edessäni, istumassa Israelin valtaistuimella, jos vain sinun poikasi pitävät vaarin teistänsä, niin että he vaeltavat minun lakini mukaan, niinkuin sinä olet minun edessäni vaeltanut'.
\par 17 Niin toteutukoon nyt, Herra, Israelin Jumala, sinun sanasi, jonka puhuit palvelijallesi Daavidille.
\par 18 Mutta asuuko todella Jumala maan päällä ihmisten seassa? Katso, taivaisiin ja taivasten taivaisiin sinä et mahdu; kuinka sitten tähän temppeliin, jonka minä olen rakentanut!
\par 19 Käänny kuitenkin palvelijasi rukouksen ja anomisen puoleen, Herra, minun Jumalani, niin että kuulet huudon ja rukouksen, jonka palvelijasi sinun edessäsi rukoilee,
\par 20 ja että silmäsi ovat päivät ja yöt avoinna tätä temppeliä kohti, tätä paikkaa kohti, johon sinä olet sanonut asettavasi nimesi, niin että kuulet rukouksen, jonka palvelijasi tähän paikkaan päin kääntyneenä rukoilee.
\par 21 Kuule palvelijasi ja kansasi Israelin rukoukset, jotka he rukoilevat tähän paikkaan päin kääntyneinä; kuule asuinpaikastasi, taivaasta, ja kun kuulet, niin anna anteeksi.
\par 22 Jos joku rikkoo lähimmäistänsä vastaan ja hänet pannaan valalle ja vannotetaan, ja jos hän tulee ja vannoo sinun alttarisi edessä tässä temppelissä,
\par 23 niin kuule taivaasta ja auta palvelijasi oikeuteensa; tee niin, että kostat syylliselle ja annat hänen tekojensa tulla hänen päänsä päälle, mutta julistat syyttömän syyttömäksi ja annat hänelle hänen vanhurskautensa mukaan.
\par 24 Jos vihollinen voittaa sinun kansasi Israelin, sentähden että he ovat tehneet syntiä sinua vastaan, mutta he kääntyvät ja kiittävät sinun nimeäsi, rukoilevat ja anovat armoa sinun kasvojesi edessä tässä temppelissä,
\par 25 niin kuule taivaasta ja anna anteeksi kansasi Israelin synti ja tuo heidät takaisin tähän maahan, jonka olet antanut heille ja heidän isillensä.
\par 26 Jos taivas suljetaan, niin ettei tule sadetta, koska he ovat tehneet syntiä sinua vastaan, mutta he rukoilevat kääntyneinä tähän paikkaan päin ja ylistävät sinun nimeäsi ja kääntyvät synnistänsä, koska sinä nöyryytät heitä,
\par 27 niin kuule taivaasta ja anna anteeksi palvelijaisi ja kansasi Israelin synti - sillä sinä osoitat heille hyvän tien, jota heidän on vaeltaminen - ja suo sade maallesi, jonka olet antanut kansallesi perintöosaksi.
\par 28 Jos maahan tulee nälänhätä, rutto, jos tulee nokitähkä ja viljanruoste, jos tulevat heinäsirkat ja tuhosirkat, jos sen viholliset ahdistavat sitä maassa, jossa sen portit ovat, jos tulee mikä tahansa vitsaus tai vaiva,
\par 29 ja jos silloin joku ihminen, kuka hyvänsä, tai koko sinun kansasi Israel, rukoilee ja anoo armoa, kun he kukin tuntevat vitsauksen ja tuskan, joka on kohdannut heitä, ja ojentavat kätensä tähän temppeliin päin,
\par 30 niin kuule silloin taivaasta, asuinpaikastasi, ja anna anteeksi ja anna jokaiselle aivan hänen tekojensa mukaan, koska sinä tunnet hänen sydämensä - sillä sinä yksin tunnet ihmislasten sydämet -
\par 31 että he pelkäisivät sinua ja vaeltaisivat sinun teitäsi, niin kauan kuin he elävät tässä maassa, jonka sinä olet meidän isillemme antanut.
\par 32 Myös jos joku muukalainen, joka ei ole sinun kansaasi Israelia, tulee kaukaisesta maasta sinun suuren nimesi, väkevän kätesi ja ojennetun käsivartesi tähden - jos hän tulee ja rukoilee kääntyneenä tähän temppeliin päin,
\par 33 niin kuule taivaasta, asuinpaikastasi, häntä ja tee kaikki, mitä muukalainen sinulta rukoilee, että kaikki maan kansat tuntisivat sinun nimesi ja pelkäisivät sinua, samoin kuin sinun kansasi Israel, ja tulisivat tietämään, että sinä olet ottanut nimiisi tämän temppelin, jonka minä olen rakentanut.
\par 34 Jos sinun kansasi lähtee sotaan vihollisiansa vastaan sitä tietä, jota sinä heidät lähetät, ja he rukoilevat sinua kääntyneinä tähän kaupunkiin päin, jonka sinä olet valinnut, ja tähän temppeliin päin, jonka minä olen sinun nimellesi rakentanut,
\par 35 niin kuule taivaasta heidän rukouksensa ja anomisensa ja hanki heille oikeus.
\par 36 Jos he tekevät syntiä sinua vastaan - sillä ei ole ihmistä, joka ei syntiä tee - ja sinä vihastut heihin ja annat heidät vihollisen valtaan, niin että heidän vangitsijansa vievät heidät vangeiksi kaukaiseen tai läheiseen maahan,
\par 37 mutta jos he sitten menevät itseensä siinä maassa, johon heidät on viety vangeiksi, kääntyvät ja anovat sinulta armoa vankeutensa maassa, sanoen: 'Me olemme tehneet syntiä, tehneet väärin ja olleet jumalattomat',
\par 38 ja palajavat sinun tykösi kaikesta sydämestään ja kaikesta sielustaan vankeutensa maassa, johon heidät on vangeiksi viety, ja rukoilevat kääntyneinä tähän maahan päin, jonka sinä olet heidän isillensä antanut, tähän kaupunkiin päin, jonka sinä olet valinnut, ja tähän temppeliin päin, jonka minä olen sinun nimellesi rakentanut,
\par 39 niin kuule taivaasta, asuinpaikastasi, heidän rukouksensa ja anomisensa, hanki heille oikeus ja anna anteeksi kansallesi, mitä he ovat rikkoneet sinua vastaan.
\par 40 Olkoot siis, minun Jumalani, sinun silmäsi avoinna, ja tarkatkoot sinun korvasi rukouksia tässä paikassa.
\par 41 Ja nyt: Nouse, Herra Jumala, tule leposijaasi, sinä ja sinun voimasi arkki. Sinun pappisi, Herra Jumala, olkoot puetut autuuteen, ja sinun hurskaasi riemuitkoot siitä, mikä hyvä on.
\par 42 Herra Jumala, älä hylkää voideltuasi; muista armolupauksiasi, jotka annoit Daavidille, palvelijallesi."

\chapter{7}

\par 1 Kun Salomo oli lakannut rukoilemasta, tuli taivaasta tuli ja kulutti polttouhrin ja teurasuhrit, ja Herran kirkkaus täytti temppelin,
\par 2 niin että papit eivät voineet mennä Herran temppeliin; sillä Herran kirkkaus täytti Herran temppelin.
\par 3 Ja kun kaikki israelilaiset näkivät tulen tulevan alas ja näkivät temppelin päällä Herran kirkkauden, polvistuivat he kivillä lasketulla pihalla kasvoillensa maahan ja rukoilivat ja kiittivät Herraa siitä, että hän on hyvä, että hänen armonsa pysyy iankaikkisesti.
\par 4 Sitten kuningas ja kaikki kansa uhrasivat teurasuhrin Herran edessä.
\par 5 Ja kuningas Salomo uhrasi teurasuhrina kaksikymmentäkaksi tuhatta raavasta ja satakaksikymmentä tuhatta lammasta. Näin he, kuningas ja koko kansa, vihkivät Herran temppelin.
\par 6 Papit seisoivat kukin virkatoimessaan. Ja leeviläiset, käsissään Herran soittimet, jotka kuningas Daavid oli teettänyt, seisoivat kiittämässä Herraa Daavidin ylistysvirsillä siitä, että hänen armonsa pysyy iankaikkisesti. Ja heitä vastapäätä papit soittivat torvia, kaiken Israelin seisoessa.
\par 7 Ja Salomo pyhitti Herran temppelin edessä olevan esipihan keskiosan, sillä hänen oli siellä uhrattava polttouhri ja yhteysuhrin rasvat, koska polttouhri, ruokauhri ja rasvat eivät mahtuneet vaskialttarille, jonka Salomo oli teettänyt.
\par 8 Näin Salomo siihen aikaan vietti juhlaa seitsemän päivää, ja koko Israel hänen kanssansa. Se oli sangen suuri kokous, johon kokoonnuttiin aina sieltä, mistä mennään Hamatiin, ja aina Egyptin purolta asti.
\par 9 Kahdeksantena päivänä heillä oli juhlakokous. Sillä alttarin vihkimistä he viettivät seitsemän päivää ja juhlaa seitsemän päivää.
\par 10 Mutta seitsemännen kuun kahdentenakymmenentenä kolmantena päivänä hän päästi kansan menemään majoillensa. He iloitsivat ja olivat hyvillä mielin siitä hyvästä, minkä Herra oli tehnyt Daavidille ja Salomolle ja kansallensa Israelille.
\par 11 Kun Salomo oli saanut valmiiksi Herran temppelin ja kuninkaan linnan ja menestynyt kaikessa, mitä oli ajatellut tehdä Herran temppelissä ja omassa linnassaan,
\par 12 ilmestyi Herra Salomolle yöllä ja sanoi hänelle: "Minä olen kuullut sinun rukouksesi ja valinnut tämän paikan uhripaikakseni.
\par 13 Jos minä suljen taivaan, niin ettei tule sadetta, jos minä käsken heinäsirkkain syödä maan tahi jos minä lähetän ruton kansaani,
\par 14 mutta minun kansani, joka on otettu minun nimiini, nöyrtyy, ja he rukoilevat ja etsivät minun kasvojani ja palajavat pahoilta teiltänsä, niin minä kuulen taivaasta ja annan anteeksi heidän syntinsä ja teen heidän maansa jälleen terveeksi.
\par 15 Nyt minun silmäni ovat avoinna, ja minun korvani tarkkaavat rukouksia tässä paikassa.
\par 16 Ja nyt minä olen valinnut ja pyhittänyt tämän temppelin, että minun nimeni olisi siinä iäti; ja minun silmäni ja sydämeni tulevat aina olemaan siinä.
\par 17 Ja jos sinä vaellat minun edessäni, niinkuin sinun isäsi Daavid vaelsi, ja teet kaiken, mitä minä olen käskenyt sinun tehdä, ja noudatat minun käskyjäni ja oikeuksiani,
\par 18 niin minä pidän pystyssä sinun kuninkaallisen valtaistuimesi, niinkuin minä lupasin sinun isällesi Daavidille sanoen: 'Aina on mies sinun suvustasi hallitseva Israelia'.
\par 19 Mutta jos te käännytte pois ja hylkäätte minun käskyni ja säädökseni, jotka minä olen teille antanut, ja menette ja palvelette muita jumalia ja kumarratte niitä,
\par 20 niin minä kiskaisen sellaiset irti maastani, jonka minä olen heille antanut; ja tämän temppelin, jonka minä olen nimelleni pyhittänyt, minä heitän pois kasvojeni edestä, ja minä teen sen sananparreksi ja pistopuheeksi kaikille kansoille.
\par 21 Ja tämän temppelin tähden, joka ennen oli korkein, on jokainen ohikulkija tyrmistyvä. Ja kun hän kysyy: 'Miksi on Herra näin tehnyt tälle maalle ja tälle temppelille?',
\par 22 niin vastataan: 'Siksi, että he hylkäsivät Herran, isiensä Jumalan, joka oli vienyt heidät pois Egyptin maasta, ja liittyivät muihin jumaliin, kumarsivat niitä ja palvelivat niitä; sentähden hän on antanut kaiken tämän pahan heitä kohdata'."

\chapter{8}

\par 1 Niiden kahdenkymmenen vuoden kuluttua, joina Salomo oli rakentanut Herran temppelin ja oman linnansa,
\par 2 linnoitti Salomo ne kaupungit, jotka Huuram oli antanut hänelle, ja hän asetti israelilaisia niihin asumaan.
\par 3 Ja Salomo meni Hamat-Soobaan ja valloitti sen.
\par 4 Hän linnoitti myös Tadmorin erämaassa ja kaikki varastokaupungit, jotka hän oli rakentanut Hamatiin.
\par 5 Vielä hän linnoitti Ylä-Beet-Hooronin ja Ala-Beet-Hooronin muureilla, porteilla ja salvoilla varustetuiksi kaupungeiksi;
\par 6 samoin Baalatin ja kaikki varastokaupungit, jotka hänellä oli, ja kaikki vaunukaupungit ja ratsumiesten kaupungit ja kaikki muut paikat, mitkä hän tahtoi linnoittaa Jerusalemissa ja Libanonilla ja kaikessa hallitsemassaan maassa.
\par 7 Kaiken kansan, mitä oli jäänyt jäljelle heettiläisistä, amorilaisista, perissiläisistä, hivviläisistä ja jebusilaisista, kaikki, jotka eivät olleet israelilaisia,
\par 8 ne niiden jälkeläiset, jotka vielä olivat jäljellä maassa ja joita israelilaiset eivät olleet tuhonneet, ne Salomo saattoi työveron alaisiksi, aina tähän päivään asti.
\par 9 Mutta israelilaisista Salomo ei tehnyt ketään työorjakseen, vaan heitä oli sotilaina, hänen vaunusoturiensa päällikköinä ja hänen sotavaunujensa ja ratsumiestensä päällikköinä.
\par 10 Maaherrojen virkamiehiä oli Salomolla kaksisataa viisikymmentä, jotka vallitsivat väkeä.
\par 11 Ja Salomo toi faraon tyttären Daavidin kaupungista linnaan, jonka hän oli tälle rakentanut; sillä hän sanoi: "Älköön nainen asuko Daavidin, Israelin kuninkaan, linnassa; sillä se on pyhäkkö, koska Herran arkki on tullut sinne".
\par 12 Siihen aikaan Salomo uhrasi polttouhreja Herralle Herran alttarilla, jonka hän oli rakentanut eteisen eteen.
\par 13 Hän uhrasi kunakin päivänä sen päivän uhrit Mooseksen käskyn mukaan, sapatteina ja uudenkuun päivinä sekä juhlina kolme kertaa vuodessa: happamattoman leivän juhlana, viikkojuhlana ja lehtimajanjuhlana.
\par 14 Ja hän asetti, niinkuin hänen isänsä Daavid oli säätänyt, pappien osastot toimittamaan virkaansa ja leeviläiset suorittamaan tehtäviänsä, veisaamaan ylistysvirsiä ja palvelemaan pappeja, kunakin päivänä sen päivän tehtävissä, sekä ovenvartijat osastoittain joka ovelle, sillä niin oli Jumalan mies Daavid käskenyt.
\par 15 Eikä missään poikettu kuninkaan käskyistä, mitä tuli pappeihin, leeviläisiin ja aarrekammioihin.
\par 16 Ja niin suoritettiin Salomon kaikki työt, Herran temppelin perustamispäivään asti, sekä siihen saakka, kunnes se valmistui. Niin oli Herran temppeli valmis.
\par 17 Siihen aikaan Salomo meni Esjon-Geberiin ja Elatiin, joka on meren rannalla Edomin maassa.
\par 18 Ja Huuram lähetti palvelijansa tuomaan hänelle laivoja. Palvelijat olivat meritaitoisia ja menivät yhdessä Salomon palvelijain kanssa Oofiriin ja noutivat sieltä kultaa neljäsataa viisikymmentä talenttia ja toivat sen kuningas Salomolle.

\chapter{9}

\par 1 Kun Saban kuningatar kuuli Salomon maineen, tuli hän Jerusalemiin koettelemaan Salomoa arvoituksilla. Hän tuli sangen suuren seurueen kanssa, mukanaan kameleja, jotka kantoivat hajuaineita, kultaa ylen paljon ja kalliita kiviä. Ja kun hän tuli Salomon luo, puhui hän tälle kaikki, mitä hänellä oli mielessänsä.
\par 2 Mutta Salomo selitti hänelle kaikki hänen kysymyksensä; Salomolle ei mikään jäänyt ongelmaksi, jota hän ei olisi hänelle selittänyt.
\par 3 Kun Saban kuningatar näki kaiken Salomon viisauden, linnan, jonka hän oli rakentanut,
\par 4 ruuat hänen pöydällänsä, kuinka hänen palvelijansa asuivat ja hänen palvelusväkensä palveli ja kuinka he olivat puetut, ja näki hänen juomanlaskijansa ja kuinka he olivat puetut, ja hänen yläsalinsa, josta hän nousi Herran temppeliin, meni hän miltei hengettömäksi.
\par 5 Sitten hän sanoi kuninkaalle: "Totta oli se puhe, jonka minä kotimaahani sinusta ja sinun viisaudestasi kuulin.
\par 6 Minä en uskonut, mitä sanottiin, ennenkuin itse tulin ja sain omin silmin nähdä; ja katso, ei puoltakaan sinun suuresta viisaudestasi oltu minulle kerrottu. Sinä olet paljon suurempi, kuin minä olin kuullut huhuttavan.
\par 7 Onnellisia ovat sinun miehesi, onnellisia nämä palvelijasi, jotka aina saavat olla sinun edessäsi ja kuulla sinun viisauttasi.
\par 8 Kiitetty olkoon Herra, sinun Jumalasi, joka sinuun on niin mielistynyt, että on asettanut sinut valtaistuimellensa olemaan kuninkaana Herran, sinun Jumalasi, kunniaksi. Sentähden, että sinun Jumalasi rakastaa Israelia ja tahtoo pitää sen pystyssä ainiaan, on hän antanut sinut heille kuninkaaksi, tekemään sitä, mikä oikeus ja vanhurskaus on."
\par 9 Ja hän antoi kuninkaalle sata kaksikymmentä talenttia kultaa, sangen paljon hajuaineita ja kalliita kiviä. Ei milloinkaan ole ollut moisia hajuaineita kuin ne, jotka Saban kuningatar antoi kuningas Salomolle.
\par 10 Huuramin palvelijat ja Salomon palvelijat, jotka toivat kultaa Oofirista, toivat myöskin santelipuuta ja kalliita kiviä.
\par 11 Ja kuningas teetti santelipuusta portaat Herran temppeliin ja kuninkaan palatsiin ja kanteleita ja harppuja laulajille. Moisia ei oltu ennen nähty Juudan maassa.
\par 12 Kuningas Salomo taas antoi Saban kuningattarelle kaikki, mitä tämä halusi ja pyysi, antoi vielä enemmän, kuin mitä tämä oli tuonut kuninkaalle. Sitten hän lähti paluumatkalle ja meni palvelijoineen omaan maahansa.
\par 13 Kullan paino, mikä yhtenä vuotena tuli Salomolle, oli kuusisataa kuusikymmentä kuusi talenttia kultaa,
\par 14 paitsi mitä kauppamiehet ja kaupustelijat toivat. Sen lisäksi kaikki Arabian kuninkaat ja maan käskynhaltijat toivat Salomolle kultaa ja hopeata.
\par 15 Ja kuningas Salomo teetti kaksisataa suurta kilpeä pakotetusta kullasta ja käytti jokaiseen kilpeen kuusisataa sekeliä pakotettua kultaa;
\par 16 samoin kolmesataa pienempää kilpeä pakotetusta kullasta, ja käytti jokaiseen kilpeen kolmesataa sekeliä kultaa. Ja kuningas asetti ne Libanoninmetsä-taloon.
\par 17 Vielä kuningas teetti suuren norsunluisen valtaistuimen ja päällysti sen puhtaalla kullalla.
\par 18 Valtaistuimessa oli kuusi porrasta ja astinlauta, kullalla kiinnitettyinä valtaistuimeen. Istuimen kummallakin puolella oli käsinoja, ja kaksi leijonaa seisoi käsinojan vieressä.
\par 19 Ja kaksitoista leijonaa seisoi siinä kuudella portaalla, kummallakin puolella. Senkaltaista ei ole tehty missään muussa valtakunnassa.
\par 20 Ja kaikki kuningas Salomon juoma-astiat olivat kultaa, ja kaikki Libanoninmetsä-talon astiat olivat puhdasta kultaa. Hopeata ei Salomon päivinä pidetty minkään arvoisena.
\par 21 Kuninkaalla oli näet laivoja, jotka kulkivat Tarsiiseen Huuramin palvelijain kanssa; kerran kolmessa vuodessa Tarsiin-laivat tulivat ja toivat kultaa ja hopeata, norsunluuta, apinoita ja riikinkukkoja.
\par 22 Ja kuningas Salomo oli kaikkia maan kuninkaita suurempi rikkaudessa ja viisaudessa.
\par 23 Ja kaikki maan kuninkaat pyrkivät näkemään Salomoa kuullaksensa hänen viisauttaan, jonka Jumala oli antanut hänen sydämeensä.
\par 24 Ja he toivat kukin lahjansa: hopea- ja kultakaluja, vaatteita, aseita, hajuaineita, hevosia ja muuleja, joka vuosi vuoden tarpeen.
\par 25 Salomolla oli neljätuhatta hevosvaljakkoa vaunuineen ja kaksitoista tuhatta ratsumiestä; ne hän sijoitti vaunukaupunkeihin ja kuninkaan luo Jerusalemiin.
\par 26 Ja hän vallitsi kaikkia kuninkaita Eufrat-virrasta aina filistealaisten maahan ja Egyptin rajaan asti.
\par 27 Ja kuningas toimitti niin, että Jerusalemissa oli hopeata kuin kiviä, ja setripuuta niin paljon kuin metsäviikunapuita Alankomaassa.
\par 28 Ja hevosia tuotiin Salomolle Egyptistä ja kaikista muista maista.
\par 29 Mitä muuta Salomosta on kerrottavaa, hänen sekä aikaisemmista että myöhemmistä vaiheistaan, se on kirjoitettuna profeetta Naatanin historiassa, siilolaisen Ahian ennustuksessa ja näkijä Jeddon näyssä Jerobeamista, Nebatin pojasta.
\par 30 Salomo hallitsi Jerusalemissa koko Israelia neljäkymmentä vuotta.
\par 31 Sitten Salomo meni lepoon isiensä tykö, ja hänet haudattiin isänsä Daavidin kaupunkiin. Ja hänen poikansa Rehabeam tuli kuninkaaksi hänen sijaansa.

\chapter{10}

\par 1 Rehabeam meni Sikemiin, sillä koko Israel oli tullut Sikemiin tekemään häntä kuninkaaksi.
\par 2 Kun Jerobeam, Nebatin poika, kuuli sen - hän oli Egyptissä, jonne oli paennut kuningas Salomoa - palasi Jerobeam Egyptistä.
\par 3 Ja he lähettivät kutsumaan hänet. Niin Jerobeam ja koko Israel tuli saapuville, ja he puhuivat Rehabeamille sanoen:
\par 4 "Sinun isäsi teki meidän ikeemme raskaaksi; mutta huojenna sinä nyt se kova työ, jota isäsi teetti, ja se raskas ies, jonka hän pani meidän niskaamme, niin me palvelemme sinua".
\par 5 Hän vastasi heille: "Odottakaa kolme päivää ja tulkaa sitten takaisin minun tyköni". Ja kansa meni.
\par 6 Kuningas Rehabeam neuvotteli vanhain kanssa, jotka olivat palvelleet hänen isäänsä Salomoa, kun tämä vielä eli, ja kysyi: "Kuinka te neuvotte vastaamaan tälle kansalle?"
\par 7 He vastasivat hänelle ja sanoivat: "Jos sinä olet hyvä tätä kansaa kohtaan, olet armollinen heille ja puhut heille hyviä sanoja, niin he ovat sinun palvelijoitasi kaiken elinaikasi".
\par 8 Mutta hän hylkäsi tämän neuvon, jonka vanhat hänelle antoivat, ja neuvotteli nuorten miesten kanssa, jotka olivat kasvaneet hänen kanssaan ja jotka palvelivat häntä.
\par 9 Hän kysyi heiltä: "Kuinka te neuvotte meitä vastaamaan tälle kansalle, joka on puhunut minulle sanoen: 'Huojenna se ies, jonka sinun isäsi on pannut meidän niskaamme'?"
\par 10 Niin nuoret miehet, jotka olivat kasvaneet hänen kanssaan, vastasivat hänelle sanoen: "Sano näin tälle kansalle, joka on puhunut sinulle sanoen: 'Sinun isäsi teki meidän ikeemme raskaaksi, mutta huojenna sinä se meiltä' - sano heille näin: 'Minun pikkusormeni on paksumpi kuin minun isäni lantio.
\par 11 Jos siis isäni on sälyttänyt teidän selkäänne raskaan ikeen, niin minä teen teidän ikeenne vielä raskaammaksi; jos isäni on kurittanut teitä raipoilla, niin minä kuritan teitä piikkiruoskilla.'"
\par 12 Niin Jerobeam ja koko kansa tuli Rehabeamin tykö kolmantena päivänä, niinkuin kuningas oli käskenyt, sanoen: "Tulkaa takaisin minun tyköni kolmantena päivänä".
\par 13 Ja kuningas antoi heille kovan vastauksen. Sillä kuningas Rehabeam hylkäsi vanhain neuvon
\par 14 ja puhui heille nuorten miesten neuvon mukaan, sanoen: "Jos minun isäni on tehnyt teidän ikeenne raskaaksi, niin minä teen sen vielä raskaammaksi; jos minun isäni on kurittanut teitä raipoilla, niin minä kuritan teitä piikkiruoskilla".
\par 15 Kuningas ei siis kuullut kansaa; sillä Jumala sen niin salli täyttääkseen sanansa, jonka hän oli puhunut Jerobeamille, Nebatin pojalle, siilolaisen Ahian kautta.
\par 16 Kun koko Israel huomasi, ettei kuningas heitä kuullut, vastasi kansa kuninkaalle näin: "Mitä osaa meillä on Daavidiin? Ei meillä ole perintöosaa Iisain poikaan. Majoillesi, Israel, joka mies! Valvo nyt huonettasi, Daavid!" Ja koko Israel meni majoillensa.
\par 17 Niin Rehabeam tuli ainoastaan niiden israelilaisten kuninkaaksi, jotka asuivat Juudan kaupungeissa.
\par 18 Ja kun kuningas Rehabeam lähetti matkaan verotöiden valvojan Hadoramin, kivittivät israelilaiset hänet kuoliaaksi. Silloin kuningas Rehabeam nousi nopeasti vaunuihinsa ja pakeni Jerusalemiin.
\par 19 Näin Israel luopui Daavidin suvusta, aina tähän päivään asti.

\chapter{11}

\par 1 Kun Rehabeam tuli Jerusalemiin, kokosi hän Juudan heimon ja Benjaminin, sata kahdeksankymmentä tuhatta sotakuntoista valiomiestä, sotimaan Israelia vastaan ja palauttamaan kuninkuutta Rehabeamille.
\par 2 Mutta Jumalan miehelle Semajalle tuli tämä Herran sana:
\par 3 "Sano Rehabeamille, Salomon pojalle, Juudan kuninkaalle, ja koko Israelille Juudassa ja Benjaminissa näin:
\par 4 'Näin sanoo Herra: Älkää menkö sotimaan veljiänne vastaan. Palatkaa kukin kotiinne, sillä minä olen sallinut tämän tapahtua.'" Niin he kuulivat Herran sanoja, kääntyivät takaisin eivätkä menneet Jerobeamia vastaan.
\par 5 Mutta Rehabeam asui Jerusalemissa ja linnoitti lujasti Juudan kaupunkeja.
\par 6 Hän linnoitti Beetlehemin, Eetanin, Tekoan,
\par 7 Beet-Suurin, Sookon, Adullamin,
\par 8 Gatin, Maaresan, Siifin,
\par 9 Adoraimin, Laakiin, Asekan,
\par 10 Soran, Aijalonin ja Hebronin, jotka ovat Juudassa ja Benjaminissa; ne hän linnoitti lujiksi kaupungeiksi.
\par 11 Hän varusti linnoitukset ja sijoitti niihin päämiehiä sekä muonavarastoja, öljyä ja viiniä
\par 12 ja jokaiseen kaupunkiin kilpiä ja keihäitä; hän varusti ne hyvin lujasti. Ja Juuda ja Benjamin olivat hänen.
\par 13 Ja papit ja leeviläiset, mitä koko Israelissa oli, tulivat kaikilta alueiltaan ja asettuivat palvelemaan häntä.
\par 14 Sillä leeviläiset jättivät laidunmaansa ja perintömaansa ja lähtivät Juudaan ja Jerusalemiin, koska Jerobeam ja hänen poikansa olivat hyljänneet heidät, niin etteivät he saaneet pappeina palvella Herraa,
\par 15 Jerobeam kun oli asettanut itselleen pappeja uhrikukkuloita varten ja metsänpeikkoja varten ja vasikoita varten, jotka hän oli teettänyt.
\par 16 Ja heitä seurasivat kaikista Israelin sukukunnista Jerusalemiin ne, jotka sydämestään antautuivat etsimään Herraa, Israelin Jumalaa, uhratakseen Herralle, isiensä Jumalalle.
\par 17 Näin he vahvistivat Juudan valtakuntaa ja tukivat Rehabeamia, Salomon poikaa, kolme vuotta; sillä he vaelsivat Daavidin ja Salomon teitä kolme vuotta.
\par 18 Ja Rehabeam otti vaimokseen Mahlatin, joka oli Daavidin pojan Jerimotin ja Iisain pojan Eliabin tyttären Abihailin tytär.
\par 19 Tämä synnytti hänelle pojat Jeuksen, Semarjan ja Saahamin.
\par 20 Hänen jälkeensä hän nai Maakan, Absalomin tyttären, joka synnytti hänelle Abian, Attain, Siisan ja Selomitin.
\par 21 Ja Rehabeam rakasti Maakaa, Absalomin tytärtä, enemmän kuin kaikkia muita vaimojaan ja sivuvaimojaan. Sillä hän oli ottanut kahdeksantoista vaimoa ja kuusikymmentä sivuvaimoa; ja hänelle syntyi kaksikymmentä kahdeksan poikaa ja kuusikymmentä tytärtä.
\par 22 Ja Rehabeam asetti Abian, Maakan pojan, päämieheksi, ruhtinaaksi hänen veljiensä joukossa, sillä hän aikoi tehdä hänet kuninkaaksi.
\par 23 Ja viisaasti hän jakoi kaikki Juudan ja Benjaminin maakunnat ja kaikki varustetut kaupungit kaikkien poikiensa kesken ja antoi heille runsaasti elintarpeita ja hankki heille paljon vaimoja.

\chapter{12}

\par 1 Kun Rehabeamin kuninkuus oli vahvistunut ja hän oli voimistunut, hylkäsi hän Herran lain, hän ja koko Israel hänen kanssaan.
\par 2 Mutta kuningas Rehabeamin viidentenä hallitusvuotena hyökkäsi Siisak, Egyptin kuningas, Jerusalemin kimppuun, sillä he olivat tulleet uskottomiksi Herraa kohtaan.
\par 3 Hänellä oli mukanaan sotavaunuja tuhat kaksisataa ja ratsumiehiä kuusikymmentä tuhatta, ja lukematon oli väki, joka tuli hänen kanssaan Egyptistä: liibyalaisia, sukkilaisia ja etiopialaisia.
\par 4 Ja hän valloitti Juudan varustetut kaupungit ja tuli aina Jerusalemiin saakka.
\par 5 Niin profeetta Semaja tuli Rehabeamin ja Juudan päämiesten tykö, jotka olivat kokoontuneet Jerusalemiin Siisakia pakoon, ja hän sanoi heille: "Näin sanoo Herra: Te olette hyljänneet minut, sentähden olen minäkin hyljännyt teidät Siisakin käsiin".
\par 6 Silloin Israelin päämiehet ja kuningas nöyrtyivät ja sanoivat: "Herra on vanhurskas".
\par 7 Kun Herra näki heidän nöyrtyvän, tuli Semajalle tämä Herran sana: "He ovat nöyrtyneet; minä en tuhoa heitä, vaan minä annan heidän hädin tuskin pelastua, eikä minun vihaani vuodateta Jerusalemin päälle Siisakin käden kautta.
\par 8 Kuitenkin heidän on tultava hänen palvelijoikseen, että he tulisivat tietämään, mitä on palvella minua ja mitä on palvella vieraitten maitten valtakuntia."
\par 9 Niin Siisak, Egyptin kuningas, hyökkäsi Jerusalemin kimppuun ja otti Herran temppelin aarteet ja kuninkaan linnan aarteet, otti kaikki tyynni. Hän otti myös kaikki kultakilvet, jotka Salomo oli teettänyt.
\par 10 Kuningas Rehabeam teetti niiden sijaan vaskikilvet ja jätti ne henkivartijain päälliköitten haltuun, jotka vartioivat kuninkaan linnan ovella.
\par 11 Ja niin usein kuin kuningas meni Herran temppeliin, menivät myöskin henkivartijat ja kantoivat niitä ja veivät ne sitten takaisin henkivartijain huoneeseen.
\par 12 Sentähden, että Rehabeam nöyrtyi, kääntyi Herran viha hänestä pois, niin ettei tullut täydellistä tuhoa; ja olivathan asiat Juudassa vielä hyvin.
\par 13 Niin kuningas Rehabeam vahvistui Jerusalemissa ja hallitsi edelleen. Sillä Rehabeam oli neljänkymmenen yhden vuoden vanha tullessaan kuninkaaksi, ja hän hallitsi seitsemäntoista vuotta Jerusalemissa, siinä kaupungissa, jonka Herra oli valinnut kaikista Israelin sukukunnista, sijoittaaksensa nimensä siihen. Hänen äitinsä oli nimeltään Naema, ammonilainen.
\par 14 Ja hän teki sitä, mikä on pahaa, sillä hän ei kiinnittänyt sydäntänsä etsimään Herraa.
\par 15 Rehabeamin vaiheet, sekä aikaisemmat että myöhemmät, ovat kirjoitetut profeetta Semajan ja näkijä Iddon historiaan, sukuluettelojen tapaan. Mutta Rehabeamin ja Jerobeamin väliset taistelut jatkuivat kaiken aikaa.
\par 16 Sitten Rehabeam meni lepoon isiensä tykö, ja hänet haudattiin Daavidin kaupunkiin. Ja hänen poikansa Abia tuli kuninkaaksi hänen sijaansa.

\chapter{13}

\par 1 Kuningas Jerobeamin kahdeksantenatoista hallitusvuotena tuli Abia Juudan kuninkaaksi.
\par 2 Hän hallitsi kolme vuotta Jerusalemissa. Hänen äitinsä nimi oli Mikaja, Uurielin tytär, Gibeasta. Mutta Abia ja Jerobeam olivat sodassa keskenään.
\par 3 Abia alotti sodan urhoollisella sotajoukolla, neljälläsadalla tuhannella valiomiehellä, mutta Jerobeam asettui sotarintaan häntä vastaan kahdeksallasadalla tuhannella valiomiehellä, sotaurholla.
\par 4 Ja Abia nousi Semaraimin vuoren laelle Efraimin vuoristossa ja sanoi: "Kuulkaa minua, Jerobeam ja koko Israel!
\par 5 Täytyyhän teidän tietää, että Herra, Israelin Jumala, on antanut Daavidille ja hänen pojillensa ikuisiksi ajoiksi Israelin kuninkuuden, lujan kuin suolaliitto.
\par 6 Mutta Jerobeam, Nebatin poika, Daavidin pojan Salomon palvelija, nousi ja kapinoitsi herraansa vastaan.
\par 7 Ja hänen luoksensa kokoontui tyhjäntoimittajia, kelvottomia miehiä, ja he pääsivät voitolle Rehabeamista, Salomon pojasta, sillä Rehabeam oli nuori ja arka eikä voinut heitä vastustaa.
\par 8 Ja nyt te luulette voivanne vastustaa Herran kuninkuutta, joka on Daavidin poikien käsissä, koska teitä on suuri joukko ja teillä on kultavasikat, jotka Jerobeam on teettänyt teille jumaliksi.
\par 9 Ettekö te ole karkoittaneet Herran pappeja, Aaronin poikia, ja leeviläisiä, ja itse tehneet itsellenne pappeja, niinkuin muiden maiden kansat tekevät? Ken tahansa tuli papinvirkaan vihittäväksi, mukanaan mullikka ja seitsemän oinasta, hänestä tuli epäjumalan pappi.
\par 10 Mutta meidän Jumalamme on Herra, ja häntä me emme ole hyljänneet. Ja pappeina palvelevat Herraa Aaronin pojat, ja leeviläiset toimittavat palvelustehtäviä.
\par 11 He polttavat Herralle polttouhreja joka aamu ja joka ilta ja hyvänhajuista suitsutusta, ja he latovat leivät päälletysten aitokultaiselle pöydälle ja sytyttävät joka ilta kultaisen seitsenhaaraisen lampun lamppuineen; sillä me hoidamme Herran, meidän Jumalamme, meille antamat tehtävät, mutta te olette hänet hyljänneet.
\par 12 Ja katso, meidän kanssamme, meidän edellämme on Jumala ja ovat hänen pappinsa, ja hälytystorvet toitottamassa teitä vastaan. Te israelilaiset, älkää sotiko Herraa, isienne Jumalaa, vastaan, sillä se ei teille onnistu."
\par 13 Mutta Jerobeam oli pannut väkeä kiertämään heidän taaksensa väijyksiin, niin että toiset olivat Juudan miesten edessä ja toiset väijyksissä heidän takanansa.
\par 14 Kun Juudan miehet kääntyivät, niin katso, heillä oli sota edessä ja takana. Silloin he huusivat Herraa, ja papit puhalsivat torviin.
\par 15 Ja Juudan miehet nostivat sotahuudon; ja kun Juudan miehet nostivat sotahuudon, antoi Jumala Abian ja Juudan voittaa Jerobeamin ja koko Israelin.
\par 16 Ja israelilaiset kääntyivät Juudan miehiä pakoon, ja Jumala antoi heidät näiden käsiin.
\par 17 Ja Abia väkinensä tuotti heille suuren tappion, niin että israelilaisia kaatui viisisataa tuhatta valiomiestä.
\par 18 Näin israelilaiset siihen aikaan nöyryytettiin; mutta Juudan miehet voimistuivat, sillä he turvautuivat Herraan, isiensä Jumalaan.
\par 19 Ja Abia ajoi Jerobeamia takaa ja valloitti häneltä muutamia kaupunkeja: Beetelin ja sen tytärkaupungit, Jesanan ja sen tytärkaupungit ja Efronin ja sen tytärkaupungit.
\par 20 Eikä Jerobeam enää tullut voimiinsa Abian elinpäivinä, vaan Herra löi häntä, niin että hän kuoli.
\par 21 Mutta Abia vahvistui. Ja hän otti itselleen neljätoista vaimoa, ja hänelle syntyi kaksikymmentä kaksi poikaa ja kuusitoista tytärtä.
\par 22 Mitä muuta on kerrottavaa Abiasta, hänen vaelluksestaan ja hänen puheistaan, se on kirjoitettuna profeetta Iddon selityskirjassa.

\chapter{14}

\par 1 Sitten Abia meni lepoon isiensä tykö, ja hänet haudattiin Daavidin kaupunkiin. Ja hänen poikansa Aasa tuli kuninkaaksi hänen sijaansa. Hänen aikanansa oli maassa rauha kymmenen vuotta.
\par 2 Ja Aasa teki sitä, mikä oli hyvää ja oikeata Herran, hänen Jumalansa, silmissä.
\par 3 Hän poisti vieraat alttarit ja uhrikukkulat, murskasi patsaat ja hakkasi maahan asera-karsikot.
\par 4 Ja hän kehoitti Juudaa etsimään Herraa, heidän isiensä Jumalaa, ja noudattamaan lakia ja käskyjä.
\par 5 Ja kaikista Juudan kaupungeista hän poisti uhrikukkulat ja auringonpatsaat, ja valtakunnassa oli rauha hänen aikanansa.
\par 6 Ja hän rakensi varustettuja kaupunkeja Juudaan, koska maassa oli rauha eikä hänellä näinä vuosina ollut mitään sotaa; sillä Herra oli suonut hänen päästä rauhaan.
\par 7 Hän sanoi Juudalle: "Linnoittakaamme nämä kaupungit, ympäröikäämme ne muureilla ja torneilla, ovilla ja salvoilla. Vielä on maa meidän vallassamme, sentähden että me olemme etsineet Herraa, meidän Jumalaamme. Me olemme etsineet häntä, ja hän on suonut meidän päästä rauhaan joka taholla." Niin he rakensivat, ja se heille onnistui.
\par 8 Ja Aasalla oli sotajoukko, joka oli varustettu kilvillä ja keihäillä: Juudasta oli kolmesataa tuhatta miestä ja Benjaminista kaksisataa kahdeksankymmentä tuhatta, jotka kantoivat kilpeä ja jännittivät jousta. Kaikki nämä olivat sotaurhoja.
\par 9 Mutta etiopialainen Serah lähti heitä vastaan sotajoukolla, jossa oli miehiä tuhannen tuhatta ja sotavaunuja kolmesataa; ja hän tuli Maaresaan saakka.
\par 10 Ja Aasa lähti häntä vastaan, ja he asettuivat sotarintaan Sefatan laaksoon, Maaresaan.
\par 11 Ja Aasa huusi Herraa, Jumalaansa, ja sanoi: "Herra, sinä yksin voit auttaa taistelussa voimallisen ja voimattoman välillä. Auta meitä, Herra, meidän Jumalamme, sillä sinuun me turvaudumme ja sinun nimessäsi me olemme tulleet tätä suurta joukkoa vastaan. Herra, sinä olet meidän Jumalamme; älä salli ihmisen päästä voitolle sinusta."
\par 12 Niin Herra antoi Aasan ja Juudan voittaa etiopialaiset, ja etiopialaiset pakenivat.
\par 13 Ja Aasa ja väki, joka oli hänen kanssaan, ajoivat heitä takaa Gerariin saakka. Ja etiopialaisia kaatui niin paljon, ettei heistä jäänyt ketään henkiin, sillä Herra ja hänen sotajoukkonsa tuhosivat heidät. Ja he ottivat paljon saalista,
\par 14 valloittivat kaikki kaupungit Gerarin ympäriltä, sillä Herran kauhu oli vallannut nämä, ja ryöstivät kaikki kaupungit, sillä niissä oli paljon ryöstettävää.
\par 15 Jopa karjamajatkin he valtasivat, veivät pois saaliinaan paljon lampaita ja kameleja ja palasivat Jerusalemiin.

\chapter{15}

\par 1 Niin Jumalan henki tuli Asarjaan, Oodedin poikaan.
\par 2 Hän meni Aasaa vastaan ja sanoi hänelle: "Kuulkaa minua, Aasa ja koko Juuda ja Benjamin. Herra on teidän kanssanne, kun te olette hänen kanssansa; ja jos häntä etsitte, niin te löydätte hänet, mutta jos hylkäätte hänet, niin hän hylkää teidät.
\par 3 Kauan aikaa Israel oli ilman oikeata Jumalaa, ilman opetusta antavaa pappia ja ilman lakia.
\par 4 Mutta kun he ahdistuksessansa palasivat Herran, Israelin Jumalan, tykö ja etsivät häntä, niin he löysivät hänet.
\par 5 Niinä aikoina ei ollut turvallisuutta niillä, jotka lähtivät ja tulivat, vaan suuri hämminki vallitsi kaikkien maakuntain asukasten keskuudessa.
\par 6 Ja kansa törmäsi kansaa vastaan ja kaupunki kaupunkia vastaan, sillä Jumala oli saattanut heidät hämminkiin lähettäen kaikkinaisia ahdistuksia.
\par 7 Mutta te olkaa lujat älkääkä antako kättenne vaipua, sillä teidän työllänne on palkkansa."
\par 8 Kun Aasa kuuli nämä profeetta Oodedin sanat ja ennustuksen, rohkaisi hän mielensä ja toimitti pois iljetykset koko Juudan ja Benjaminin maasta ja niistä kaupungeista, jotka hän oli valloittanut Efraimin vuoristosta, ja uudisti Herran alttarin, joka oli Herran eteisen edessä.
\par 9 Ja hän kokosi koko Juudan ja Benjaminin ja myöskin heidän luonaan asuvat muukalaiset, jotka olivat tulleet Efraimista, Manassesta ja Simeonista, sillä monet olivat siirtyneet Israelista hänen luokseen nähtyänsä, että Herra, hänen Jumalansa, oli hänen kanssaan.
\par 10 Ja he kokoontuivat Jerusalemiin Aasan viidennentoista hallitusvuoden kolmannessa kuussa
\par 11 ja uhrasivat sinä päivänä Herralle saaliista, minkä he olivat tuoneet, seitsemänsataa raavasta ja seitsemäntuhatta lammasta.
\par 12 Ja he tekivät liiton, että etsisivät Herraa, isiensä Jumalaa, kaikesta sydämestään ja kaikesta sielustaan,
\par 13 ja että jokainen, joka ei etsinyt Herraa, Israelin Jumalaa, oli surmattava, olipa hän pieni tai suuri, mies tai nainen.
\par 14 Ja he vannoivat valan Herralle suurella äänellä riemun raikuessa ja torvien ja pasunain pauhatessa.
\par 15 Ja koko Juuda iloitsi siitä valasta, sillä he olivat vannoneet sen kaikesta sydämestään, ja he etsivät Herraa kaikella halullansa ja löysivät hänet; niin Herra soi heidän päästä rauhaan joka taholla.
\par 16 Kuningas Aasa erotti äitinsäkin Maakan kuningattaren arvosta, koska tämä oli pystyttänyt inhotuksen Aseralle; Aasa kukisti inhotuksen, rouhensi ja poltti sen Kidronin laaksossa.
\par 17 Mutta uhrikukkulat eivät hävinneet Israelista. Kuitenkin Aasan sydän oli ehyesti Herralle antautunut, niin kauan kuin hän eli.
\par 18 Ja hän vei Jumalan temppeliin isänsä pyhät lahjat ja omat pyhät lahjansa: hopeata, kultaa ja kalua.
\par 19 Eikä ollut sotaa Aasan kolmanteenkymmenenteen viidenteen hallitusvuoteen asti.

\chapter{16}

\par 1 Aasan kolmantenakymmenentenä kuudentena hallitusvuotena lähti Baesa, Israelin kuningas, Juudaa vastaan ja linnoitti Raaman, estääkseen ketään pääsemästä Aasan, Juudan kuninkaan, luota tai hänen luokseen.
\par 2 Ja Aasa toi hopeata ja kultaa Herran temppelin ja kuninkaan linnan aarrekammioista ja lähetti sen Benhadadille, Aramin kuninkaalle, joka asui Damaskossa, ja käski sanoa:
\par 3 "Onhan liitto meidän välillämme, minun ja sinun, niinkuin oli minun isäni ja sinun isäsi välillä. Katso, minä lähetän sinulle hopeata ja kultaa; mene ja riko liittosi Baesan, Israelin kuninkaan, kanssa, että hän lähtisi pois minun kimpustani."
\par 4 Niin Benhadad kuuli kuningas Aasaa ja lähetti sotajoukkojensa päälliköt Israelin kaupunkeja vastaan, ja he valtasivat Iijonin, Daanin ja Aabel-Maimin sekä kaikki Naftalin kaupunkien varastohuoneet.
\par 5 Kun Baesa kuuli sen, lakkasi hän linnoittamasta Raamaa ja keskeytti työnsä.
\par 6 Mutta kuningas Aasa toi koko Juudan, ja he veivät pois Raamasta kivet ja puut, joilla Baesa oli sitä linnoittanut. Niillä hän linnoitti Geban ja Mispan.
\par 7 Siihen aikaan tuli näkijä Hanani Aasan, Juudan kuninkaan, tykö ja sanoi hänelle: "Koska sinä turvauduit Aramin kuninkaaseen etkä turvautunut Herraan, Jumalaasi, sentähden on Aramin kuninkaan sotajoukko päässyt sinun käsistäsi.
\par 8 Eikö etiopialaisia ja liibyalaisia ollut suuri sotajoukko, eikö heillä ollut hyvin paljon sotavaunuja ja ratsumiehiä? Mutta koska sinä turvauduit Herraan, antoi hän heidät sinun käsiisi.
\par 9 Sillä Herran silmät tarkkaavat kaikkea maata, että hän voimakkaasti auttaisi niitä, jotka ovat ehyellä sydämellä antautuneet hänelle. Tässä sinä teit tyhmästi, sillä tästä lähtien on sinulla yhä oleva sotia."
\par 10 Mutta Aasa vihastui näkijään ja panetti hänet vankilaan, sillä niin vihoissaan hän oli hänelle tästä. Myöskin muutamille muille kansasta Aasa siihen aikaan teki väkivaltaa.
\par 11 Aasan vaiheet, sekä aikaisemmat että myöhemmät, katso, ne ovat kirjoitettuina Juudan ja Israelin kuningasten kirjassa.
\par 12 Ja kolmantenakymmenentenä yhdeksäntenä hallitusvuotenaan Aasa sairastui jaloistaan, ja hänen tautinsa yltyi hyvin kovaksi. Mutta taudissaankaan hän ei etsinyt Herraa, vaan lääkäreitä.
\par 13 Sitten Aasa meni lepoon isiensä tykö ja kuoli neljäntenäkymmenentenä yhdentenä hallitusvuotenaan.
\par 14 Ja hänet haudattiin omaan hautaansa, jonka hän oli hakkauttanut itsellensä Daavidin kaupunkiin; ja hänet laskettiin vuoteelle, joka oli täytetty hajuaineilla ja erilaisilla, voiteeksi sekoitetuilla höysteillä, ja hänen kunniakseen poltettiin ylen runsas kuolinsuitsutus.

\chapter{17}

\par 1 Hänen poikansa Joosafat tuli kuninkaaksi hänen sijaansa. Hän vahvistautui Israelia vastaan.
\par 2 Hän sijoitti sotaväkeä kaikkiin Juudan varustettuihin kaupunkeihin ja asetti maaherroja Juudan maahan ja Efraimin kaupunkeihin, jotka hänen isänsä Aasa oli valloittanut.
\par 3 Ja Herra oli Joosafatin kanssa, sillä hän vaelsi isänsä Daavidin aikaisempia teitä eikä etsinyt baaleja;
\par 4 vaan hän etsi isänsä Jumalaa ja vaelsi hänen käskyjensä mukaan eikä tehnyt, niinkuin Israel teki.
\par 5 Niin Herra vahvisti kuninkuuden hänen käsissään, ja koko Juuda antoi lahjoja Joosafatille, niin että hänelle tuli paljon rikkautta ja kunniaa.
\par 6 Ja kun hänen rohkeutensa kasvoi Herran teillä, poisti hän vielä uhrikukkulatkin ja asera-karsikot Juudasta.
\par 7 Ja kolmantena hallitusvuotenaan hän lähetti ylimmät virkamiehensä Benhailin, Obadjan, Sakarjan, Netanelin ja Miikajan Juudan kaupunkeihin antamaan opetusta,
\par 8 ja heidän kanssansa leeviläiset Semajan, Netanjan, Sebadjan, Asahelin, Semiramotin, Joonatanin, Adonian, Tobian ja Toob-Adonian, leeviläiset; ja näillä oli kanssansa papit Elisama ja Jooram.
\par 9 Nämä opettivat Juudassa, ja heillä oli mukanaan Herran lain kirja; he kiertelivät kaikissa Juudan kaupungeissa ja opettivat kansaa.
\par 10 Ja Herran kauhu valtasi kaikki Juudaa ympäröivien maitten valtakunnat, niin etteivät ne sotineet Joosafatia vastaan.
\par 11 Ja osa filistealaisia toi Joosafatille lahjoja ja hopeata veroksi; myöskin arabialaiset toivat hänelle pikkukarjaa: seitsemäntuhatta seitsemänsataa oinasta ja seitsemäntuhatta seitsemänsataa pukkia.
\par 12 Niin Joosafat tuli yhä mahtavammaksi, jopa ylen mahtavaksi, ja hän rakensi Juudaan linnoja ja varastokaupunkeja.
\par 13 Hänellä oli suuria varastoja Juudan kaupungeissa ja sotilaita, sotaurhoja, Jerusalemissa.
\par 14 Ja tämä oli heidän palvelusvuoronsa heidän perhekuntiensa mukaan: Juudan tuhanten päämiehet olivat: päämies Adna ja hänen kanssaan kolmesataa tuhatta sotaurhoa;
\par 15 hänen rinnallaan päämies Joohanan ja hänen kanssaan kaksisataa kahdeksankymmentä tuhatta;
\par 16 hänen rinnallaan Amasja, Sikrin poika, joka oli vapaaehtoisesti antautunut Herralle, ja hänen kanssaan kaksisataa tuhatta sotaurhoa.
\par 17 Benjaminista: Eljada, sotaurho, ja hänen kanssaan kaksisataa tuhatta jousella ja kilvellä asestettua;
\par 18 hänen rinnallaan Joosabad ja hänen kanssaan sata kahdeksankymmentä tuhatta sotaan varustettua.
\par 19 Nämä palvelivat kuningasta, ja lisäksi ne, jotka kuningas oli sijoittanut varustettuihin kaupunkeihin koko Juudaan.

\chapter{18}

\par 1 Niin Joosafatille tuli paljon rikkautta ja kunniaa, ja hän lankoutui Ahabin kanssa.
\par 2 Ja muutamien vuosien kuluttua hän meni Ahabin luo Samariaan. Ahab teurastutti hänelle ja väelle, joka oli hänen kanssansa, paljon lampaita ja raavaita; ja Ahab yllytti häntä lähtemään sotaan Gileadin Raamotia vastaan.
\par 3 Israelin kuningas Ahab sanoi Juudan kuninkaalle Joosafatille: "Lähdetkö minun kanssani Gileadin Raamotiin?" Hän vastasi hänelle: "Minä niinkuin sinä, minun kansani niinkuin sinun kansasi; minä tulen sinun kanssasi sotaan".
\par 4 Mutta Joosafat sanoi Israelin kuninkaalle: "Kysy kuitenkin ensin, mitä Herra sanoo".
\par 5 Niin Israelin kuningas kokosi profeetat, neljäsataa miestä, ja sanoi heille: "Onko meidän lähdettävä sotaan Gileadin Raamotia vastaan, vai onko minun oltava lähtemättä?" He vastasivat: "Lähde; Jumala antaa sen kuninkaan käsiin".
\par 6 Mutta Joosafat sanoi: "Eikö täällä ole enää ketään muuta Herran profeettaa, jolta voisimme kysyä?"
\par 7 Israelin kuningas vastasi Joosafatille: "On vielä mies, jolta voisimme kysyä Herran mieltä, mutta minä vihaan häntä, sillä hän ei koskaan ennusta minulle hyvää, vaan aina pahaa; se on Miika, Jimlan poika". Joosafat sanoi: "Älköön kuningas niin puhuko".
\par 8 Niin Israelin kuningas kutsui erään hoviherran ja sanoi: "Nouda kiiruusti Miika, Jimlan poika".
\par 9 Mutta Israelin kuningas ja Joosafat, Juudan kuningas, istuivat kumpikin valtaistuimellansa puettuina kuninkaallisiin pukuihinsa; he istuivat puimatantereella Samarian portin ovella, ja kaikki profeetat olivat hurmoksissa heidän edessänsä.
\par 10 Ja Sidkia, Kenaanan poika, teki itsellensä rautasarvet ja sanoi: "Näin sanoo Herra: Näillä sinä pusket aramilaisia, kunnes teet heistä lopun".
\par 11 Ja kaikki profeetat ennustivat samalla tavalla, sanoen: "Mene Gileadin Raamotiin, niin sinä saat voiton; Herra antaa sen kuninkaan käsiin".
\par 12 Ja sanansaattaja, joka oli mennyt kutsumaan Miikaa, puhui hänelle sanoen: "Katso, kaikki profeetat ovat yhdestä suusta luvanneet kuninkaalle hyvää. Olkoon sinun sanasi heidän sanansa kaltainen, ja lupaa sinäkin hyvää."
\par 13 Mutta Miika vastasi: "Niin totta kuin Herra elää, sen minä puhun, minkä minun Jumalani sanoo".
\par 14 Kun hän tuli kuninkaan eteen, sanoi kuningas hänelle: "Miika, onko meidän lähdettävä sotaan Gileadin Raamotiin, vai onko minun oltava lähtemättä?" Hän vastasi hänelle: "Menkää, niin te saatte voiton; heidät annetaan teidän käsiinne".
\par 15 Mutta kuningas sanoi hänelle: "Kuinka monta kertaa minun on vannotettava sinua, ettet puhu minulle muuta kuin totuutta Herran nimessä?"
\par 16 Silloin hän sanoi: "Minä näin koko Israelin hajallaan vuorilla, niinkuin lampaat, joilla ei ole paimenta. Ja Herra sanoi: 'Näillä ei ole isäntää; palatkoot he kukin rauhassa kotiinsa'."
\par 17 Niin Israelin kuningas sanoi Joosafatille: "Enkö minä sanonut sinulle, ettei tämä koskaan ennusta minulle hyvää, vaan aina pahaa?"
\par 18 Mutta hän sanoi: "Kuulkaa siis Herran sana: Minä näin Herran istuvan istuimellansa ja kaiken taivaan joukon seisovan hänen edessään, hänen oikealla ja vasemmalla puolellansa.
\par 19 Ja Herra sanoi: 'Kuka viekoittelisi Ahabin, Israelin kuninkaan, lähtemään sotaan, että hän kaatuisi Gileadin Raamotissa?' Mikä vastasi niin, mikä näin.
\par 20 Silloin tuli henki ja asettui Herran eteen ja sanoi: 'Minä viekoittelen hänet'. Herra kysyi häneltä: 'Miten?'
\par 21 Hän vastasi: 'Minä menen valheen hengeksi kaikkien hänen profeettainsa suuhun'. Silloin Herra sanoi: 'Saat viekoitella, siihen sinä pystyt; mene ja tee niin'.
\par 22 Katso, nyt Herra on pannut valheen hengen näiden sinun profeettaisi suuhun, sillä Herra on päättänyt sinun osaksesi onnettomuuden."
\par 23 Silloin astui esille Sidkia, Kenaanan poika, löi Miikaa poskelle ja sanoi: "Mitä tietä Herran Henki on poistunut minusta puhuakseen sinun kanssasi?"
\par 24 Miika vastasi: "Sen saat nähdä sinä päivänä, jona kuljet huoneesta huoneeseen piiloutuaksesi".
\par 25 Mutta Israelin kuningas sanoi: "Ottakaa Miika ja viekää hänet takaisin Aamonin, kaupungin päällikön, ja Jooaan, kuninkaan pojan, luo.
\par 26 Ja sanokaa: 'Näin sanoo kuningas: Pankaa tämä vankilaan ja elättäkää häntä vaivaisella vedellä ja leivällä, kunnes minä palaan voittajana takaisin'."
\par 27 Miika vastasi: "Jos sinä palaat voittajana takaisin, niin ei Herra ole puhunut minun kauttani". Ja hän sanoi vielä: "Kuulkaa tämä, kaikki kansat".
\par 28 Niin Israelin kuningas ja Joosafat, Juudan kuningas, menivät Gileadin Raamotiin.
\par 29 Ja Israelin kuningas sanoi Joosafatille: "Täytyypä pukeutua tuntemattomaksi, kun käy taisteluun; mutta ole sinä omissa vaatteissasi". Ja Israelin kuningas pukeutui tuntemattomaksi ja kävi taisteluun.
\par 30 Mutta Aramin kuningas oli käskenyt sotavaunujensa päälliköitä sanoen: "Älkää ryhtykö taisteluun kenenkään muun kanssa, olkoon alempi tai ylempi, kuin ainoastaan Israelin kuninkaan kanssa".
\par 31 Kun sotavaunujen päälliköt näkivät Joosafatin, ajattelivat he: "Tuo on Israelin kuningas", ja ympäröivät hänet hyökätäkseen hänen kimppuunsa. Silloin Joosafat huusi, ja Herra auttoi häntä, ja Jumala houkutteli heidät pois hänestä.
\par 32 Kun sotavaunujen päälliköt näkivät, ettei se ollutkaan Israelin kuningas, vetäytyivät he hänestä pois.
\par 33 Mutta eräs mies, joka oli jännittänyt jousensa ja ampui umpimähkään, satutti Israelin kuningasta vyöpanssarin ja rintahaarniskan väliin. Niin tämä sanoi vaunujensa ohjaajalle: "Käännä vaunut ja vie minut pois sotarinnasta, sillä minä olen haavoittunut".
\par 34 Mutta kun taistelu sinä päivänä yltyi yltymistään, jäi Israelin kuningas seisomaan vaunuihinsa, päin aramilaisia, iltaan asti; auringonlaskun aikaan hän kuoli.

\chapter{19}

\par 1 Mutta Joosafat, Juudan kuningas, palasi onnellisesti takaisin kotiinsa Jerusalemiin.
\par 2 Silloin Jeehu, Hananin poika, näkijä, meni kuningas Joosafatia vastaan ja sanoi hänelle: "Oliko sinun autettava jumalatonta, ja rakastatko sinä niitä, jotka vihaavat Herraa? Sentähden on sinun päälläsi Herran viha.
\par 3 Kuitenkin on sinussa löydetty myös hyvää, sillä sinä olet hävittänyt aserat maasta ja kiinnittänyt sydämesi Jumalan etsimiseen."
\par 4 Niin Joosafat jäi Jerusalemiin. Sitten hän meni jälleen kansan keskeen, Beersebasta lähtien aina Efraimin vuoristoon saakka, ja palautti heidät takaisin Herran, heidän isiensä Jumalan, tykö.
\par 5 Ja hän asetti tuomareita maahan, kaikkiin Juudan varustettuihin kaupunkeihin, kaupunki kaupungilta.
\par 6 Ja hän sanoi tuomareille: "Katsokaa, mitä teette, sillä te ette ole tekemässä ihmisten tuomioita, vaan Herran tuomioita, ja hän on teidän kanssanne, kun te tuomitsette.
\par 7 Hallitkoon siis teitä Herran pelko. Ottakaa vaari siitä, mitä teette, sillä Herrassa, meidän Jumalassamme, ei ole vääryyttä eikä puolueellisuutta, eikä hän ota lahjuksia."
\par 8 Myöskin Jerusalemiin Joosafat asetti leeviläisiä, pappeja ja Israelin perhekunta-päämiehiä jakamaan Herran oikeutta ja ratkaisemaan riita-asioita. Ja he palasivat Jerusalemiin.
\par 9 Ja hän käski heitä sanoen: "Tehkää Herraa peljäten, uskollisesti ja ehyellä sydämellä näin:
\par 10 Jokaisessa riita-asiassa, minkä veljenne, jotka asuvat kaupungeissansa, tuovat teidän eteenne, koskipa se murhaa tai lakia, käskyä, säädöksiä tai oikeuksia, varoittakaa heitä, etteivät saattaisi itseänsä vikapäiksi Herran edessä ja ettei viha kohtaisi teitä ja teidän veljiänne. Näin tehkää, ettette tulisi vikapäiksi.
\par 11 Ja katso, ylimmäinen pappi Amarja olkoon teillä esimiehenä kaikissa Herran asioissa ja Sebadja, Ismaelin poika, Juudan heimoruhtinas, kaikissa kuninkaan asioissa, ja olkoot leeviläiset teillä virkamiehinä. Olkaa lujat ja ryhtykää työhön; ja Herra olkoon sen kanssa, joka hyvä on."

\chapter{20}

\par 1 Senjälkeen tulivat mooabilaiset ja ammonilaiset, ja heidän kanssaan myös muita paitsi ammonilaisia, sotimaan Joosafatia vastaan.
\par 2 Niin tultiin ilmoittamaan tästä Joosafatille: "Suuri joukko tulee sinua vastaan meren toiselta puolelta, Aramista, ja katso, he ovat jo Hasason-Taamarissa"; se on Een-Gedissä.
\par 3 Silloin Joosafat peljästyi ja kääntyi kysymään Herralta ja kuulutti paaston koko Juudaan.
\par 4 Niin Juuda kokoontui etsimään apua Herralta; myös kaikista Juudan kaupungeista tultiin etsimään Herraa.
\par 5 Ja Joosafat astui esiin Juudan ja Jerusalemin seurakunnassa, Herran temppelissä, uuden esipihan edessä
\par 6 ja sanoi: "Herra, meidän isiemme Jumala, sinä olet Jumala taivaassa, ja sinä hallitset kaikkia pakanakansain valtakuntia. Sinun kädessäsi on voima ja väkevyys, eikä kukaan kestä sinun edessäsi.
\par 7 Sinä, meidän Jumalamme, karkoitit tämän maan asukkaat kansasi Israelin tieltä ja annoit sen ystäväsi Aabrahamin jälkeläisille ikuisiksi ajoiksi.
\par 8 He asettuivat tänne ja rakensivat täällä sinulle, sinun nimellesi, pyhäkön sanoen:
\par 9 'Jos meitä kohtaa joku onnettomuus, miekka, rangaistustuomio, rutto tai nälänhätä, niin me astumme tämän temppelin eteen ja sinun eteesi, sillä sinun nimesi on tässä temppelissä; ja me huudamme sinua hädässämme, ja sinä kuulet ja autat'.
\par 10 Ja katso, siinä ovat nyt ammonilaiset ja mooabilaiset ja Seirin vuoristolaiset, joiden alueen kautta sinä et antanut israelilaisten kulkea, kun he tulivat Egyptin maasta, vaan nämä kääntyivät heistä pois eivätkä tuhonneet heitä.
\par 11 Katso, nyt he kostavat sen meille: he tulevat karkoittamaan meitä maasta, joka on sinun omasi ja jonka sinä olet antanut meidän omaksemme.
\par 12 Meidän Jumalamme, etkö tuomitse heitä? Sillä me emme mahda mitään tätä suurta joukkoa vastaan, joka hyökkää meidän kimppuumme, emmekä itse tiedä mitä tehdä, vaan sinuun meidän silmämme katsovat."
\par 13 Kaikki Juudan miehet seisoivat siinä Herran edessä pikkulapsineen, vaimoineen ja poikineen.
\par 14 Ja Herran Henki tuli seurakunnan keskellä Jahasieliin, Sakarjan poikaan, joka oli Benajan poika, joka Jegielin poika, joka Mattanjan poika, leeviläisen, joka oli Aasafin jälkeläisiä,
\par 15 ja hän sanoi: "Kuunnelkaa, kaikki te Juudan miehet ja Jerusalemin asukkaat ja sinä kuningas Joosafat. Näin sanoo teille Herra: Älkää peljätkö älkääkä arkailko tätä suurta joukkoa, sillä sota ei ole teidän, vaan Jumalan.
\par 16 Menkää huomenna heitä vastaan. Katso, he nousevat silloin Siisin solaa pitkin, ja te kohtaatte heidät laakson päässä, itäänpäin Jeruelin erämaasta.
\par 17 Mutta silloin ei ole teidän asianne taistella. Astukaa esiin, seisokaa ja katsokaa, kuinka Herra pelastaa teidät, Juuda ja Jerusalem. Älkää peljätkö älkääkä arkailko; menkää huomenna heitä vastaan, ja Herra on oleva teidän kanssanne."
\par 18 Silloin Joosafat kumartui kasvoillensa maahan, ja kaikki Juudan miehet ja Jerusalemin asukkaat lankesivat Herran eteen, rukoilemaan Herraa.
\par 19 Ja ne leeviläiset, jotka olivat Kehatin ja Koorahin jälkeläisiä, nousivat ylistämään Herraa, Israelin Jumalaa, ylen korkealla äänellä.
\par 20 Mutta varhain seuraavana aamuna he menivät Tekoan erämaahan. Ja heidän lähtiessänsä Joosafat astui esiin ja sanoi: "Kuulkaa minua, te Juudan ja Jerusalemin asukkaat. Uskokaa Herraan, Jumalaanne, niin te olette hyvässä turvassa, ja uskokaa hänen profeettojansa, niin te menestytte."
\par 21 Ja neuvoteltuaan kansan kanssa hän asetti veisaajat veisaamaan Herralle ylistysvirsiä pyhässä kaunistuksessa ja kulkemaan aseväen edellä sanoen: "Kiittäkää Herraa, sillä hänen armonsa pysyy iankaikkisesti".
\par 22 Ja juuri kun he alottivat riemuhuudon ja ylistysvirren, antoi Herra väijyjiä tulla ammonilaisten, mooabilaisten ja Seirin vuoristolaisten selkään, jotka olivat hyökänneet Juudan kimppuun; ja heidät voitettiin.
\par 23 Sillä ammonilaiset ja mooabilaiset asettuivat Seirin vuoristolaisia vastaan tuhoamaan ja hävittämään heitä; ja kun he olivat lopettaneet Seirin asukkaat, auttoivat he toisiaan toistensa tuhoamisessa.
\par 24 Kun Juudan miehet tulivat paikalle, josta voi tähystää erämaahan, ja kääntyivät joukkoon päin, niin katso, ruumiita makasi maassa, ei kukaan ollut pelastunut.
\par 25 Niin Joosafat väkineen tuli ryöstämään heiltä saalista, ja he löysivät heidän seastaan paljon sekä tavaraa että ruumiita ja kalliita kaluja. Ja he ottivat itsellensä enemmän kuin saivat kannetuksi; he ryöstivät saalista kolme päivää, sillä sitä oli paljon.
\par 26 Neljäntenä päivänä he kokoontuivat Beraka-laaksoon; sillä siellä he kiittivät Herraa. Siitä on sen paikan nimenä Beraka-laakso vielä tänäkin päivänä.
\par 27 Senjälkeen kaikki Juudan ja Jerusalemin miehet, ja Joosafat heidän etunenässään, kääntyivät iloiten paluumatkalle Jerusalemiin, sillä Herra oli antanut heille ilon heidän vihollisistaan.
\par 28 Ja he tulivat soittaen harpuilla, kanteleilla ja torvilla Jerusalemiin, Herran temppeliin.
\par 29 Ja Jumalan kauhu valtasi kaikkien maitten valtakunnat, kun he kuulivat Herran sotineen Israelin vihollisia vastaan.
\par 30 Sitten Joosafatin valtakunnalla oli rauha; hänen Jumalansa soi hänen päästä rauhaan joka taholla.
\par 31 Niin hallitsi Joosafat Juudaa. Hän oli kolmenkymmenen viiden vuoden vanha tullessaan kuninkaaksi, ja hän hallitsi Jerusalemissa kaksikymmentä viisi vuotta. Hänen äitinsä oli nimeltään Asuba, Silhin tytär.
\par 32 Ja hän vaelsi isänsä Aasan tietä, siltä poikkeamatta, ja teki sitä, mikä on oikein Herran silmissä.
\par 33 Mutta uhrikukkulat eivät hävinneet, eikä kansa vielä ollut kiinnittänyt sydäntänsä isiensä Jumalaan.
\par 34 Mitä muuta on kerrottavaa Joosafatista, sekä hänen aikaisemmista että myöhemmistä vaiheistaan, katso, se on kirjoitettuna Jeehun, Hananin pojan, historiassa, joka on otettu Israelin kuningasten kirjaan.
\par 35 Senjälkeen Joosafat, Juudan kuningas, liittoutui Ahasjan, Israelin kuninkaan, kanssa, joka oli jumalaton menoissaan.
\par 36 Hän liittoutui tämän kanssa rakentaakseen laivoja, joiden oli määrä kulkea Tarsiiseen; ja niin he rakensivat laivoja Esjon-Geberissä.
\par 37 Mutta Elieser, Doodavahun poika, Maaresasta, ennusti Joosafatia vastaan sanoen: "Koska olet liittoutunut Ahasjan kanssa, on Herra särkevä sinun työsi". Niin laivat rikkoutuivat eivätkä kyenneet menemään Tarsiiseen.

\chapter{21}

\par 1 Sitten Joosafat meni lepoon isiensä tykö, ja hänet haudattiin isiensä viereen Daavidin kaupunkiin. Ja hänen poikansa Jooram tuli kuninkaaksi hänen sijaansa.
\par 2 Ja hänellä oli veljiä: Joosafatin pojat Asarja, Jehiel, Sakarja, Asarjahu, Miikael ja Sefatja. Nämä kaikki olivat Joosafatin, Israelin kuninkaan, poikia.
\par 3 Ja heidän isänsä oli antanut heille suuria lahjoja, hopeata, kultaa ja kalleuksia, sekä varustettuja kaupunkeja Juudassa; mutta kuninkuuden hän oli antanut Jooramille, sillä tämä oli esikoinen.
\par 4 Kun Jooram oli noussut isänsä valtaistuimelle ja vahvistunut, tappoi hän miekalla kaikki veljensä, niin myös muutamia Israelin päämiehiä.
\par 5 Jooram oli kolmenkymmenen kahden vuoden vanha tullessaan kuninkaaksi, ja hän hallitsi Jerusalemissa kahdeksan vuotta.
\par 6 Mutta hän vaelsi Israelin kuningasten tietä, niinkuin Ahabin suku oli tehnyt, sillä hänellä oli puolisona Ahabin tytär; ja niin hän teki sitä, mikä on pahaa Herran silmissä.
\par 7 Mutta Herra ei tahtonut tuhota Daavidin sukua, koska hän oli tehnyt liiton Daavidin kanssa ja koska hän oli luvannut antaa hänelle ja hänen pojillensa lampun ainiaaksi.
\par 8 Hänen aikanaan edomilaiset luopuivat Juudan vallanalaisuudesta ja asettivat itsellensä kuninkaan.
\par 9 Niin Jooram lähti sinne päällikköineen ja kaikkine sotavaunuineen. Ja hän nousi yöllä ja voitti edomilaiset, jotka olivat saartaneet hänet, ja sotavaunujen päälliköt.
\par 10 Niin edomilaiset luopuivat Juudan vallanalaisuudesta vapaiksi, aina tähän päivään asti. Siihen aikaan luopui hänen vallanalaisuudestaan myös Libna, koska hän oli hyljännyt Herran, isiensä Jumalan.
\par 11 Hänkin teetti uhrikukkuloita Juudan vuoristoon ja saattoi Jerusalemin asukkaat haureuteen ja vietteli Juudan.
\par 12 Mutta profeetta Elialta tuli hänelle tällainen kirjoitus: "Näin sanoo Herra, sinun isäsi Daavidin Jumala: Koska et ole vaeltanut isäsi Joosafatin teitä etkä Aasan, Juudan kuninkaan, teitä,
\par 13 vaan olet vaeltanut Israelin kuningasten tietä ja saattanut Juudan ja Jerusalemin asukkaat haureuteen, niinkuin Ahabin suku saattoi heidät haureuteen; ja koska myös olet tappanut veljesi, jotka olivat sinun isäsi perhekuntaa ja paremmat kuin sinä,
\par 14 katso, sentähden Herra rankaisee sinun kansaasi, poikiasi, vaimojasi ja kaikkea, mitä sinulla on, kovalla vitsauksella;
\par 15 ja sinä itse olet sairastava vaikeata tautia, sisusvaivaa, kunnes vuoden, parin kuluttua sisuksesi taudin voimasta tunkeutuvat ulos."
\par 16 Niin Herra herätti Jooramia vastaan filistealaisten hengen ja niiden arabialaisten hengen, jotka asuivat etiopialaisten naapureina,
\par 17 ja he menivät Juudaa vastaan, valloittivat sen ja veivät saaliinaan kaiken tavaran, mitä oli kuninkaan linnassa, niin myöskin hänen poikansa ja vaimonsa, niin ettei hänelle jäänyt muuta poikaa kuin Jooahas, hänen nuorin poikansa.
\par 18 Ja kaiken tämän jälkeen Herra rankaisi häntä parantumattomalla sisusvaivalla.
\par 19 Parin vuoden kuluttua, toisen vuoden lopulla, hänen sisuksensa taudin voimasta tunkeutuivat ulos, ja hän kuoli koviin tuskiin. Mutta hänen kansansa ei polttanut hänen kunniakseen kuolinsuitsutusta, niinkuin oli poltettu hänen isiensä kunniaksi.
\par 20 Hän oli kolmenkymmenen kahden vuoden vanha tullessaan kuninkaaksi, ja hän hallitsi Jerusalemissa kahdeksan vuotta. Ja hän meni pois kenenkään kaipaamatta, ja hänet haudattiin Daavidin kaupunkiin, mutta ei kuningasten hautoihin.

\chapter{22}

\par 1 Jerusalemin asukkaat tekivät Ahasjan, hänen nuorimman poikansa, kuninkaaksi hänen sijaansa; sillä kaikki vanhemmat oli tappanut se rosvojoukko, joka arabialaisten kanssa oli tullut leiriin. Niin tuli Ahasja, Jooramin poika, Juudan kuninkaaksi.
\par 2 Ahasja oli kahdenkymmenen kahden vuoden vanha tullessaan kuninkaaksi, ja hän hallitsi Jerusalemissa vuoden. Hänen äitinsä oli nimeltään Atalja, Omrin tytär.
\par 3 Hänkin vaelsi Ahabin suvun teitä, sillä hänen äitinsä oli häntä neuvomassa jumalattomuuteen.
\par 4 Niin hän teki sitä, mikä on pahaa Herran silmissä, samoinkuin Ahabin suku; sillä he olivat hänen isänsä kuoleman jälkeen hänen neuvonantajiaan, hänen turmioksensa.
\par 5 Heidän neuvostansa hän myös lähti Israelin kuninkaan Jooramin, Ahabin pojan, kanssa taistelemaan Hasaelia, Aramin kuningasta, vastaan Gileadin Raamotiin; mutta aramilaiset haavoittivat Jooramin.
\par 6 Niin hän tuli takaisin Jisreeliin parantuakseen, sillä hänessä oli haavat, jotka häneen oli isketty Raamassa hänen taistellessaan Hasaelia, Aramin kuningasta, vastaan. Ja Juudan kuningas Ahasja, Jooramin poika, tuli Jisreeliin katsomaan Jooramia, Ahabin poikaa, koska tämä oli sairaana.
\par 7 Mutta Jumalalta tuli Ahasjan tuhoksi se, että hän meni Jooramin luo. Sillä sinne tultuaan hän meni Jooramin kanssa Jeehua, Nimsin poikaa, vastaan, jonka Herra oli voidellut hävittämään Ahabin sukua.
\par 8 Ja kun Jeehu oli toimeenpanemassa Ahabin suvun tuomiota, tapasi hän Juudan ruhtinaat ja Ahasjan veljenpojat, jotka palvelivat Ahasjaa, ja tappoi heidät.
\par 9 Sitten hän etsi Ahasjaa; ja tämä saatiin kiinni Samariassa, jossa hän piileskeli. Niin hänet vietiin Jeehun eteen ja surmattiin. Ja he hautasivat hänet, sillä he sanoivat: "Hän on Joosafatin poika, hänen, joka kaikesta sydämestänsä etsi Herraa". Eikä Ahasjan perheessä ollut ketään, joka olisi kyennyt hallitsemaan.
\par 10 Kun Atalja, Ahasjan äiti, näki, että hänen poikansa oli kuollut, nousi hän ja tuhosi koko Juudan heimon kuningassuvun.
\par 11 Mutta kuninkaan tytär Joosabat otti surmattavain kuninkaan poikien joukosta salaa Ahasjan pojan Jooaan ja pani hänet imettäjineen makuuhuoneeseen. Näin pappi Joojadan vaimo Joosabat, joka oli kuningas Jooramin tytär ja Ahasjan sisar, sai hänet kätketyksi Ataljalta, niin ettei tämä voinut häntä surmauttaa.
\par 12 Sitten poika oli heidän luonaan Jumalan temppeliin piilotettuna kuusi vuotta, Ataljan hallitessa maata.

\chapter{23}

\par 1 Mutta seitsemäntenä vuotena Joojada rohkaisi mielensä ja liittoutui sadanpäämiesten Asarjan, Jerohamin pojan, Ismaelin, Joohananin pojan, Asarjan, Oobedin pojan, Maasejan, Adajan pojan, ja Elisafatin, Sikrin pojan, kanssa.
\par 2 Nämä kiertelivät Juudassa ja kokosivat leeviläiset kaikista Juudan kaupungeista sekä Israelin perhekunta-päämiehet; ja he tulivat Jerusalemiin.
\par 3 Ja koko seurakunta teki Jumalan temppelissä liiton kuninkaan kanssa. Joojada sanoi heille: "Katso, kuninkaan poika on tuleva kuninkaaksi, niinkuin Herra on puhunut Daavidin pojista.
\par 4 Tehkää näin: Kolmas osa teistä, papeista ja leeviläisistä, joiden on mentävä vartionpitoon sapattina, olkoon ovenvartijoina kynnyksillä,
\par 5 kolmas osa miehittäköön kuninkaan linnan ja kolmas osa Jesod-portin, ja koko kansa Herran temppelin esikartanot.
\par 6 Älköön kukaan muu kuin papit ja palvelusta tekevät leeviläiset menkö Herran temppeliin. He saavat mennä sisään, sillä he ovat pyhät, mutta kaikki muu kansa noudattakoon Herran määräystä.
\par 7 Leeviläiset asettukoot kuninkaan ympärille, kullakin ase kädessä; ja joka tunkeutuu temppeliin, se surmattakoon. Näin olkaa kuninkaan luona, menköön hän ulos tai sisään."
\par 8 Leeviläiset ja koko Juuda tekivät kaiken, mitä pappi Joojada oli käskenyt heidän tehdä. Kukin heistä otti miehensä, sekä ne, joiden oli mentävä vartionpitoon sapattina, että ne, jotka pääsivät vartionpidosta sapattina, sillä pappi Joojada ei ollut vapauttanut osastoja palveluksesta.
\par 9 Ja pappi Joojada antoi sadanpäämiehille keihäät, kilvet ja varustukset, jotka olivat olleet kuningas Daavidin omat ja olivat Jumalan temppelissä.
\par 10 Ja hän asetti koko kansan, kullakin peitsi kädessä, temppelin eteläsivulta aina sen pohjoissivulle saakka, päin alttaria ja temppeliä, kuninkaan ympärille.
\par 11 Sitten he toivat kuninkaan pojan esille, panivat hänen päähänsä kruunun ja antoivat hänelle lain kirjan ja tekivät hänet kuninkaaksi. Ja Joojada ja hänen poikansa voitelivat hänet ja huusivat: "Eläköön kuningas!"
\par 12 Kun Atalja kuuli kansan huudon, sen juostessa ja huutaessa riemuhuutoja kuninkaalle, meni hän kansan luo Herran temppeliin.
\par 13 Hän näki kuninkaan seisovan korkealla paikallansa sisäänkäytävän luona, ja päälliköt ja torvensoittajat kuninkaan luona, ja kaiken maan kansan, joka riemuitsi ja puhalsi torviin, ja veisaajat, jotka soittimillaan johtivat ylistyslaulua. Silloin Atalja repäisi vaatteensa ja huusi: "Kapina, kapina!"
\par 14 Mutta pappi Joojada antoi sadanpäämiesten, sotajoukon johtajien, astua esiin ja sanoi heille: "Viekää hänet pois rivien välitse, ja joka yrittää seurata häntä, se surmattakoon miekalla". Sillä pappi oli sanonut: "Älkää surmatko häntä Herran temppelissä".
\par 15 Niin he kävivät häneen käsiksi, ja kun hän oli tullut kuninkaan linnan Hevosportin sisäänkäytävän luo, surmasivat he hänet siinä.
\par 16 Ja Joojada teki liiton hänen itsensä, koko kansan ja kuninkaan kesken, että he olisivat Herran kansa.
\par 17 Sitten kaikki kansa meni Baalin temppeliin, ja he hävittivät sen. Sen alttarit ja kuvat he murskasivat, ja he tappoivat alttarien edessä Mattanin, Baalin papin.
\par 18 Sitten Joojada asetti päällysmiehiä Herran temppeliin, leeviläisiä pappeja, jotka Daavid oli määrännyt Herran temppeliin, uhraamaan Herran polttouhreja, niinkuin on kirjoitettuna Mooseksen laissa, riemuhuudoin ja lauluin, Daavidin järjestelyn mukaan.
\par 19 Ja hän asetti ovenvartijat Herran temppelin porteille, ettei pääsisi sisään kukaan, joka oli jollakin tavalla saastainen.
\par 20 Ja hän otti mukaansa sadanpäämiehet, ylimykset ja kansan hallitusmiehet sekä kaiken maan kansan ja vei kuninkaan Herran temppelistä, ja he tulivat Yläportin kautta kuninkaan linnaan ja asettivat kuninkaan kuninkaalliselle valtaistuimelle.
\par 21 Ja kaikki maan kansa iloitsi, ja kaupunki pysyi rauhallisena. Mutta Ataljan he surmasivat miekalla.

\chapter{24}

\par 1 Jooas oli seitsemän vuoden vanha tullessaan kuninkaaksi, ja hän hallitsi Jerusalemissa neljäkymmentä vuotta. Hänen äitinsä nimi oli Sibja, Beersebasta.
\par 2 Ja Jooas teki sitä, mikä on oikein Herran silmissä, niin kauan kuin pappi Joojada eli.
\par 3 Ja Joojada otti hänelle kaksi vaimoa, ja hänelle syntyi poikia ja tyttäriä.
\par 4 Senjälkeen Jooas aikoi uudistaa Herran temppelin.
\par 5 Ja hän kutsui kokoon papit ja leeviläiset ja sanoi heille: "Menkää Juudan kaupunkeihin ja kootkaa kaikesta Israelista rahaa Jumalanne temppelin korjaamiseksi, vuodesta vuoteen, ja jouduttakaa tätä asiaa". Mutta leeviläiset eivät jouduttaneet sitä.
\par 6 Silloin kuningas kutsui luokseen ylimmäisen papin Joojadan ja sanoi hänelle: "Miksi et ole pitänyt huolta siitä, että leeviläiset toisivat Juudasta ja Jerusalemista Herran palvelijan Mooseksen määräämän Israelin seurakunnan veron lainmajaa varten?
\par 7 Sillä jumalaton Atalja ja hänen poikansa ovat murtautuneet Jumalan temppeliin; he ovat myös käyttäneet baaleja varten kaikki Herran temppelin pyhät lahjat."
\par 8 Sitten tehtiin kuninkaan käskystä arkku, ja se asetettiin Herran temppelin portin ulkopuolelle.
\par 9 Ja Juudassa ja Jerusalemissa kuulutettiin, että vero, jonka Jumalan palvelija Mooses oli erämaassa määrännyt Israelille, oli tuotava Herralle.
\par 10 Kaikki päämiehet ja kaikki kansa toivat iloiten rahaa ja heittivät arkkuun, kunnes se täyttyi.
\par 11 Aina kun leeviläiset toivat arkun kuninkaan asettamien tarkastusmiesten luo, huomattuaan, että siinä oli paljon rahaa, tulivat kuninkaan kirjuri ja ylimmäisen papin käskyläinen tyhjentämään arkun, ja sitten he kantoivat sen takaisin paikoilleen. Niin he tekivät päivä päivältä ja kokosivat paljon rahaa.
\par 12 Sitten kuningas ja Joojada antoivat sen niille, jotka teettivät työt Herran temppelissä, ja nämä palkkasivat kivenhakkaajia ja puuseppiä uudistamaan Herran temppeliä, niin myös rauta- ja vaskiseppiä korjaamaan Herran temppeliä.
\par 13 Ja työnteettäjät toimivat niin, että työ edistyi heidän käsissään, ja he asettivat Jumalan temppelin entiselleen sen määrämittojen mukaan ja panivat sen kuntoon.
\par 14 Ja kun he olivat päättäneet työnsä, veivät he rahan tähteet kuninkaalle ja Joojadalle; ja niillä teetettiin kaluja Herran temppeliin, jumalanpalvelus- ja uhraamiskaluja, kuppeja sekä kulta- ja hopeakaluja. Ja Herran temppelissä uhrattiin vakituisesti polttouhreja, niin kauan kuin Joojada eli.
\par 15 Mutta Joojada kävi vanhaksi ja sai elämästä kyllänsä, ja hän kuoli. Sadan kolmenkymmenen vuoden vanha hän oli kuollessaan.
\par 16 Ja hänet haudattiin Daavidin kaupunkiin kuningasten joukkoon, sillä hän oli tehnyt sitä, mikä hyvää on, Israelille ja Jumalalle ja hänen temppelilleen.
\par 17 Mutta Joojadan kuoleman jälkeen tulivat Juudan päämiehet ja kumarsivat kuningasta; ja kuningas kuuli heitä.
\par 18 Ja he hylkäsivät Herran, isiensä Jumalan, temppelin ja palvelivat aseroita ja jumalankuvia. Niin viha kohtasi Juudaa ja Jerusalemia tämän heidän rikoksensa tähden.
\par 19 Hän lähetti heidän keskuuteensa profeettoja palauttamaan heitä Herran tykö, ja nämä varoittivat heitä, mutta he eivät kuulleet.
\par 20 Niin Jumalan Henki täytti Sakarjan, pappi Joojadan pojan, ja hän astui kansan eteen ja sanoi heille: "Näin sanoo Jumala: Miksi te rikotte Herran käskyt omaksi onnettomuudeksenne? Koska te olette hyljänneet Herran, hylkää hänkin teidät."
\par 21 Mutta he tekivät salaliiton häntä vastaan ja kivittivät hänet kuoliaaksi kuninkaan käskystä Herran temppelin esipihalla.
\par 22 Sillä kuningas Jooas ei muistanut rakkautta, jota hänen isänsä Joojada oli hänelle osoittanut, vaan tappoi Joojadan pojan. Ja tämä sanoi kuollessaan: "Herra nähköön ja kostakoon".
\par 23 Vuoden vaihteessa kävi aramilaisten sotajoukko Jooaan kimppuun, ja he tunkeutuivat Juudaan ja Jerusalemiin asti ja tuhosivat kansasta kaikki kansan päämiehet. Ja kaiken saaliinsa he lähettivät Damaskon kuninkaalle.
\par 24 Sillä vaikka aramilaisten sotajoukko tuli vähälukuisena, antoi Herra näiden käsiin hyvin suuren sotajoukon, koska he olivat hyljänneet Herran, isiensä Jumalan. Ja niin nämä panivat toimeen rangaistustuomion Jooaalle.
\par 25 Kun he sitten lähtivät häntä ahdistamasta - ja hän jäi heidän lähtiessään hyvin sairaaksi - tekivät hänen palvelijansa salaliiton häntä vastaan pappi Joojadan pojan murhan tähden ja tappoivat hänet hänen vuoteeseensa; ja niin hän kuoli. Ja hänet haudattiin Daavidin kaupunkiin; kuitenkaan ei häntä haudattu kuningasten hautoihin.
\par 26 Ne, jotka tekivät salaliiton häntä vastaan, olivat Saabad, ammonilaisen vaimon Simeatin poika, ja Joosabad, mooabilaisen vaimon Simritin poika.
\par 27 Hänen pojistaan, monista häntä vastaan lausutuista ennustuksista ja Jumalan temppelin uudestaanrakentamisesta, katso, niistä on kirjoitettu Kuningasten kirjan selityskirjaan. Ja hänen poikansa Amasja tuli kuninkaaksi hänen sijaansa.

\chapter{25}

\par 1 Amasja oli kahdenkymmenen viiden vuoden vanha tullessansa kuninkaaksi, ja hän hallitsi Jerusalemissa kaksikymmentä yhdeksän vuotta. Hänen äitinsä oli nimeltään Jooaddan, Jerusalemista.
\par 2 Hän teki sitä, mikä on oikein Herran silmissä, ei kuitenkaan ehyellä sydämellä.
\par 3 Ja kun hänen kuninkuutensa oli vahvistunut, tappoi hän ne palvelijansa, jotka olivat murhanneet kuninkaan, hänen isänsä.
\par 4 Mutta heidän lapsensa hän jätti surmaamatta: hän teki, niinkuin on kirjoitettu laissa, Mooseksen kirjassa, jossa Herra on käskenyt sanoen: "Älkööt isät kuolko lasten tähden älköötkä lapset isien tähden; kukin kuolkoon oman syntinsä tähden".
\par 5 Sitten Amasja kokosi Juudan miehet ja asetti heidät perhekunnittain, tuhannen- ja sadanpäämiesten mukaan, koko Juudan ja Benjaminin. Ja hän piti katselmuksen heistä, kaksikymmenvuotiaista ja sitä vanhemmista, ja huomasi heitä olevan kolmesataa tuhatta sotakelpoista, keihästä ja kilpeä käyttävää valiomiestä.
\par 6 Lisäksi hän palkkasi Israelista sadalla hopeatalentilla satatuhatta sotaurhoa.
\par 7 Mutta Jumalan mies tuli hänen luoksensa ja sanoi: "Kuningas! Älköön Israelin sotajoukko lähtekö sinun kanssasi, sillä Herra ei ole Israelin, kaikkien näiden efraimilaisten, kanssa.
\par 8 Vaan lähde sinä yksin, käy rohkeasti taisteluun; muutoin Jumala kaataa sinut vihollistesi eteen, sillä Jumalalla on valta auttaa ja kaataa." Amasja sanoi Jumalan miehelle:
\par 9 "Mutta miten käy niiden sadan talentin, jotka minä olen antanut Israelin joukolle?" Jumalan mies vastasi: "Herra voi antaa sinulle paljon enemmän, kuin se on".
\par 10 Silloin Amasja erotti sen joukon, joka oli tullut hänen luoksensa Efraimista, palaamaan kotiinsa. Siitä he vihastuivat kovin Juudaan ja menivät kotiinsa vihasta hehkuen.
\par 11 Mutta Amasja rohkaisi mielensä, lähti ja vei väkensä Suolalaaksoon ja surmasi seiriläisiä kymmenen tuhatta.
\par 12 Ja toiset kymmenen tuhatta Juudan miehet ottivat elävinä vangeiksi. Nämä he veivät kallion laelle ja syöksivät heidät kallion laelta alas, niin että he kaikki ruhjoutuivat.
\par 13 Mutta ne, jotka kuuluivat Amasjan palauttamaan joukkoon ja jotka eivät olleet päässeet hänen kanssaan sotaan, hyökkäsivät Juudan kaupunkeihin, Samariaan ja aina Beet-Hooroniin saakka; ja he surmasivat asukkaista kolmetuhatta ja ottivat paljon saalista.
\par 14 Kun Amasja, voitettuaan edomilaiset, tuli takaisin, toi hän mukanaan seiriläisten jumalat ja asetti ne itselleen jumaliksi, ja hän kumarsi niitä ja poltti niille uhreja.
\par 15 Niin Herran viha syttyi Amasjaa kohtaan, ja hän lähetti hänen luokseen profeetan; tämä sanoi hänelle: "Miksi sinä etsit tuon kansan jumalia, jotka eivät voineet pelastaa omaa kansaansa sinun käsistäsi?"
\par 16 Hänen näin hänelle puhuessaan Amasja sanoi hänelle: "Olemmeko me asettaneet sinut kuninkaan neuvonantajaksi? Herkeä jo, muutoin sinut surmataan." Profeetta herkesi ja sanoi: "Minä tiedän, että Jumala on päättänyt sinut tuhota, koska sinä teet näin etkä kuule minun neuvoani".
\par 17 Amasja, Juudan kuningas, lähetti neuvoteltuaan asiasta Jooaalle, Jooahaan pojalle, Jeehun pojanpojalle, Israelin kuninkaalle, tämän sanan: "Tule, otelkaamme keskenämme".
\par 18 Mutta Jooas, Israelin kuningas, lähetti Amasjalle, Juudan kuninkaalle tämän sanan: "Libanonilla kasvava ohdake lähetti setripuulle, joka kasvoi Libanonilla, tämän sanan: 'Anna tyttäresi vaimoksi minun pojalleni'. Mutta metsän eläimet Libanonilla kulkivat ohdakkeen ylitse ja tallasivat sen maahan.
\par 19 Sinä sanot: 'Katso, minä olen voittanut edomilaiset', ja olet käynyt ylpeäksi saatuasi kunniaa. Mutta pysy kotonasi; minkätähden tahdot tuottaa onnettomuuden: sinä itse kukistut ja Juuda sinun kanssasi?"
\par 20 Mutta Amasja ei totellut; sillä tämä tuli Jumalalta: hän tahtoi antaa heidät alttiiksi, koska he olivat etsineet Edomin jumalia.
\par 21 Niin Jooas, Israelin kuningas, lähti liikkeelle, ja he ottelivat keskenään, hän ja Amasja, Juudan kuningas, Beet-Semeksessä, joka on Juudan aluetta.
\par 22 Ja israelilaiset voittivat Juudan miehet, ja nämä pakenivat kukin majallensa.
\par 23 Ja Jooas, Israelin kuningas, otti Juudan kuninkaan Amasjan, Jooaan pojan, Jooahaan pojanpojan, Beet-Semeksessä vangiksi ja toi hänet Jerusalemiin. Sitten hän revitti Jerusalemin muuria Efraimin portista Kulmaporttiin saakka, neljäsataa kyynärää.
\par 24 Ja hän otti kaiken kullan ja hopean sekä kalut, mitä Herran temppelissä, Oobed-Edomin huostassa oli, ja kuninkaan linnan aarteet ynnä panttivankeja ja palasi Samariaan.
\par 25 Mutta Amasja, Jooaan poika, Juudan kuningas, eli Jooaan, Jooahaan pojan, Israelin kuninkaan, kuoleman jälkeen viisitoista vuotta.
\par 26 Mitä muuta on kerrottavaa Amasjan aikaisemmista ja myöhemmistä vaiheista, se on kirjoitettuna Juudan ja Israelin kuningasten kirjassa.
\par 27 Siitä ajasta alkaen, jolloin Amasja luopui Herrasta, punottiin Jerusalemissa salaliittoa häntä vastaan. Hän pakeni Laakiiseen, mutta Laakiiseen lähetettiin miehiä hänen jälkeensä, ja he surmasivat hänet siellä.
\par 28 Hänet nostettiin hevosten selkään ja haudattiin isiensä viereen Juudan kaupunkiin.

\chapter{26}

\par 1 Mutta koko Juudan kansa otti Ussian, joka oli kuudentoista vuoden vanha, ja teki hänet kuninkaaksi hänen isänsä Amasjan sijaan.
\par 2 Hän linnoitti Eelatin ja palautti sen Juudalle, sittenkuin kuningas oli mennyt lepoon isiensä tykö.
\par 3 Ussia oli kuudentoista vuoden vanha tullessaan kuninkaaksi, ja hän hallitsi Jerusalemissa viisikymmentä kaksi vuotta. Hänen äitinsä oli nimeltään Jekolja, Jerusalemista.
\par 4 Ja hän teki sitä, mikä on oikein Herran silmissä, aivan niinkuin hänen isänsä Amasja oli tehnyt.
\par 5 Ja hän etsi Jumalaa, niin kauan kuin Sakarja eli, joka ymmärsi Jumalan näkyjä. Ja niin kauan kuin hän etsi Herraa, antoi Jumala hänen menestyä.
\par 6 Hän lähti sotimaan filistealaisia vastaan ja revitti Gatin, Jabnen ja Asdodin muurit; ja hän rakennutti kaupunkeja Asdodin alueelle ja muualle filistealaisten maahan.
\par 7 Ja Jumala auttoi häntä filistealaisia vastaan ja niitä arabialaisia vastaan, jotka asuivat Guur-Baalissa, sekä meunilaisia vastaan.
\par 8 Ja ammonilaiset maksoivat veroa Ussialle, ja hänen maineensa levisi aina Egyptiin saakka, sillä hän tuli ylen mahtavaksi.
\par 9 Ussia rakennutti Jerusalemiin tornit Kulmaportin ja Laaksoportin päälle ja Kulmauksen päälle ja varusti ne.
\par 10 Hän rakennutti torneja myös erämaahan ja hakkautti paljon vesisäiliöitä, sillä hänellä oli paljon karjaa sekä Alankomaassa että ylätasangolla. Peltomiehiä ja viinitarhureita hänellä oli vuoristossa ja puutarhamailla, sillä hän harrasti maanviljelystä.
\par 11 Ja Ussialla oli sotajoukko, joka teki sotapalvelusta ja lähti sotaan osastoittain, sen lukumäärän mukaan, mikä heitä oli katselmuksessa, jonka kirjuri Jegiel ja päällysmies Maaseja olivat toimittaneet Hananjan, kuninkaallisen päällikön, johdolla.
\par 12 Perhekunta-päämiesten, sotaurhojen, koko lukumäärä oli kaksituhatta kuusisataa.
\par 13 Heidän johdossaan oli kolmensadan seitsemän tuhannen viidensadan miehen suuruinen sotajoukko, joka teki väellä ja voimalla sotapalvelusta auttaakseen kuningasta vihollista vastaan.
\par 14 Ja Ussia varusti heidät, koko sotajoukon, kilvillä, keihäillä, kypäreillä, rintahaarniskoilla, jousilla ja linkokivillä.
\par 15 Ja hän teetti Jerusalemiin taidokkaasti sommiteltuja sotakoneita, jotka asetettiin torneihin ja muurien kulmiin ja joilla ammuttiin nuolia ja suuria kiviä. Niin hänen maineensa levisi kauas, sillä hän sai ihmeteltävästi apua, kunnes hän tuli mahtavaksi.
\par 16 Mutta kun hän oli tullut mahtavaksi, ylpistyi hänen sydämensä, niin että hän teki kelvottoman teon, hän tuli uskottomaksi Herraa, Jumalaansa, kohtaan ja meni Herran temppeliin, suitsutusalttarille, suitsuttamaan.
\par 17 Niin pappi Asarja meni hänen jälkeensä sinne, mukanaan kahdeksankymmentä Herran pappia, rohkeita miehiä.
\par 18 Nämä astuivat kuningas Ussiaa vastaan ja sanoivat hänelle: "Ei ole sinun asiasi, Ussia, suitsuttaa Herralle, vaan pappien, Aaronin poikien, jotka ovat pyhitetyt suitsuttamaan. Mene ulos pyhäköstä, sillä sinä olet ollut uskoton, eikä siitä tule sinulle kunniaa Herralta Jumalalta."
\par 19 Niin Ussia vihastui, ja hän piti juuri suitsutusastiaa kädessään suitsuttaaksensa. Mutta kun hän vihastui pappeihin, puhkesi hänen otsaansa pitali pappien läsnäollessa Herran temppelissä, suitsutusalttarin ääressä.
\par 20 Ja kun ylimmäinen pappi Asarja ja kaikki muut papit kääntyivät hänen puoleensa, niin katso, hänellä oli pitali otsassa. Silloin he äkisti karkoittivat hänet sieltä, ja itsekin hän kiiruusti lähti pois, kun Herra oli häntä lyönyt.
\par 21 Niin kuningas Ussia tuli pitalitautiseksi kuolinpäiväänsä saakka ja asui pitalitautisena eri talossa, sillä hän oli eristetty Herran temppelistä. Hänen poikansa Jootam hallitsi kuninkaan linnaa ja tuomitsi maan kansaa.
\par 22 Mitä muuta on kerrottavaa Ussiasta, hänen aikaisemmista ja myöhemmistä vaiheistaan, sen on profeetta Jesaja, Aamoksen poika, pannut kirjaan.
\par 23 Ja Ussia meni lepoon isiensä tykö, ja hänet haudattiin isiensä viereen vainioon, kuningasten hautojen ääreen; sanottiin näet: "Hän on pitalitautinen". Ja hänen poikansa Jootam tuli kuninkaaksi hänen sijaansa.

\chapter{27}

\par 1 Jootam oli kahdenkymmenen viiden vuoden vanha tullessaan kuninkaaksi, ja hän hallitsi Jerusalemissa kuusitoista vuotta. Hänen äitinsä nimi oli Jerusa, Saadokin tytär.
\par 2 Hän teki sitä, mikä on oikein Herran silmissä, aivan niinkuin hänen isänsä Ussia oli tehnyt, paitsi ettei hän tunkeutunut Herran temppeliin. Mutta kansa yhä vaelsi kelvottomasti.
\par 3 Hän rakennutti Yläportin Herran temppeliin ja rakennutti paljon Oofelin muuria.
\par 4 Hän rakennutti myös kaupunkeja Juudan vuoristoon, ja metsiin hän rakennutti linnoja ja torneja.
\par 5 Hän kävi sotaa ammonilaisten kuningasta vastaan ja voitti heidät; ja ammonilaiset antoivat hänelle sinä vuonna sata talenttia hopeata, kymmenentuhatta koor-mittaa nisuja ja kymmenentuhatta koor-mittaa ohria. Tämän ammonilaiset suorittivat hänelle myöskin toisena ja kolmantena vuotena.
\par 6 Niin Jootam tuli yhä mahtavammaksi, koska hän vaelsi vakaasti Herran, Jumalansa, edessä.
\par 7 Mitä muuta on kerrottavaa Jootamista ja kaikista hänen sodistaan ja hänen vaelluksestansa, katso, se on kirjoitettuna Israelin ja Juudan kuningasten kirjassa.
\par 8 Hän oli kahdenkymmenen viiden vuoden vanha tullessaan kuninkaaksi, ja hän hallitsi Jerusalemissa kuusitoista vuotta.
\par 9 Ja Jootam meni lepoon isiensä tykö, ja hänet haudattiin Daavidin kaupunkiin. Ja hänen poikansa Aahas tuli kuninkaaksi hänen sijaansa.

\chapter{28}

\par 1 Aahas oli kahdenkymmenen vuoden vanha tullessaan kuninkaaksi, ja hän hallitsi Jerusalemissa kuusitoista vuotta. Hän ei tehnyt sitä, mikä on oikein Herran silmissä, niinkuin hänen isänsä Daavid,
\par 2 vaan vaelsi Israelin kuningasten teitä; jopa hän teki valettuja kuvia baaleille.
\par 3 Ja hän poltti uhreja Ben-Hinnomin laaksossa ja poltti poikansa tulessa, niiden kansain kauhistavien tekojen mukaan, jotka Herra oli karkoittanut israelilaisten tieltä.
\par 4 Ja hän teurasti ja poltti uhreja uhrikukkuloilla ja kummuilla ja jokaisen viheriän puun alla.
\par 5 Sentähden Herra, hänen Jumalansa, antoi hänet Aramin kuninkaan käsiin. He voittivat hänet ja ottivat hänen väestään suuren joukon vangiksi ja veivät Damaskoon. Hänet annettiin myöskin Israelin kuninkaan käsiin, ja tämä tuotti hänelle suuren tappion.
\par 6 Pekah, Remaljan poika, surmasi Juudasta satakaksikymmentä tuhatta yhtenä päivänä, kaikki sotakuntoisia miehiä, koska he olivat hyljänneet Herran, isiensä Jumalan.
\par 7 Ja Sikri, efraimilainen urho, tappoi kuninkaan pojan Maasejan, linnan esimiehen Asrikamin ja kuninkaan lähimmän miehen Elkanan.
\par 8 Ja israelilaiset veivät veljiltään vangeiksi vaimoja, poikia ja tyttäriä kaksisataa tuhatta, ryöstivät heiltä myös paljon saalista ja veivät saaliin Samariaan.
\par 9 Siellä oli Herran profeetta nimeltä Ooded; tämä meni sotajoukkoa vastaan, kun se oli tulossa Samariaan, ja sanoi heille: "Katso, Herra, teidän isienne Jumala, on vihastunut Juudaan ja antanut heidät teidän käsiinne, ja te olette tappaneet heitä raivossa, joka ulottuu taivaaseen asti.
\par 10 Ja nyt te ajattelette pakottaa tämän Juudan väen ja Jerusalemin orjiksenne ja orjattariksenne. Eikö teillä jo ole tunnollanne kyllin rikkomuksia Herraa, Jumalaanne, vastaan?
\par 11 Niin kuulkaa nyt minua ja palauttakaa vangit, jotka olette ottaneet veljienne joukosta; sillä Herran vihan hehku on teidän päällänne."
\par 12 Silloin muutamat efraimilaisten päämiehistä, Asarja, Joohananin poika, Berekia, Mesillemotin poika, Hiskia, Sallumin poika, ja Amasa, Hadlain poika, nousivat sotaretkeltä tulevia vastaan
\par 13 ja sanoivat heille: "Älkää tuoko tänne noita vankeja, sillä te saatatte meidät syyllisiksi Herran edessä, kun ajattelette lisätä meidän syntejämme ja syyllisyyttämme. Onhan meidän syyllisyytemme jo suuri, ja vihan hehku on Israelin päällä."
\par 14 Silloin aseväki luopui vangeista ja saaliista päämiesten ja koko seurakunnan edessä.
\par 15 Ja nimeltä mainitut miehet nousivat ja ottivat huostaansa vangit ja antoivat saaliista vaatteita kaikille heidän joukossaan, jotka olivat alasti. He antoivat näille vaatteita ja kenkiä, ruokaa ja juomaa, voitelivat heitä ja toimittivat heille, kaikille uupuneille, aasit ja veivät heidät Jerikoon, Palmukaupunkiin, lähelle heidän veljiänsä. Sitten he palasivat Samariaan.
\par 16 Siihen aikaan kuningas Aahas lähetti sanansaattajat Assurin kuningasten tykö saadaksensa apua.
\par 17 Sillä vielä edomilaisetkin olivat tulleet ja voittaneet Juudan ja ottaneet vankeja.
\par 18 Ja filistealaiset olivat tehneet ryöstöretken Alankomaan ja Juudan Etelämaan kaupunkeihin ja valloittaneet Beet-Semeksen, Aijalonin ja Gederotin, niin myös Sookon ja sen tytärkaupungit, Timran ja sen tytärkaupungit ja Gimson ja sen tytärkaupungit, ja he olivat asettuneet niihin.
\par 19 Sillä Herra nöyryytti Juudaa Aahaan, Israelin kuninkaan, tähden, koska tämä oli harjoittanut kurittomuutta Juudassa ja ollut uskoton Herraa kohtaan.
\par 20 Niin Tillegat-Pilneser, Assurin kuningas, lähti häntä vastaan ja ahdisti häntä eikä tukenut häntä.
\par 21 Sillä vaikka Aahas ryösti Herran temppeliä ja kuninkaan ja päämiesten linnoja ja antoi kaiken Assurin kuninkaalle, ei hänellä ollut siitä apua.
\par 22 Silloinkin kun häntä ahdistettiin, oli hän, kuningas Aahas, edelleen uskoton Herraa kohtaan.
\par 23 Sillä hän uhrasi Damaskon jumalille, jotka olivat voittaneet hänet; hän ajatteli: "Koska Aramin kuningasten jumalat ovat auttaneet heitä, uhraan minäkin niille, että ne minuakin auttaisivat". Mutta ne tulivatkin hänelle ja koko Israelille lankeemukseksi.
\par 24 Ja Aahas kokosi Jumalan temppelin kalut ja hakkasi Jumalan temppelin kalut kappaleiksi. Hän sulki Herran temppelin ovet ja teetti itselleen alttareita Jerusalemin joka kolkkaan.
\par 25 Ja jokaiseen Juudan kaupunkiin hän teetti uhrikukkuloita polttaakseen uhreja muille jumalille; ja niin hän vihoitti Herran, isiensä Jumalan.
\par 26 Mitä muuta on kerrottavaa hänestä ja kaikesta hänen vaelluksestaan, sekä aikaisemmasta että myöhemmästä, katso, se on kirjoitettuna Juudan ja Israelin kuningasten aikakirjassa.
\par 27 Ja Aahas meni lepoon isiensä tykö, ja hänet haudattiin kaupunkiin, Jerusalemiin; sillä häntä ei viety Israelin kuningasten hautoihin. Ja hänen poikansa Hiskia tuli kuninkaaksi hänen sijaansa.

\chapter{29}

\par 1 Hiskia tuli kuninkaaksi kahdenkymmenen viiden vuoden vanhana, ja hän hallitsi Jerusalemissa kaksikymmentä yhdeksän vuotta. Hänen äitinsä oli nimeltään Abia, Sakarjan tytär.
\par 2 Hän teki sitä, mikä on oikein Herran silmissä, aivan niinkuin hänen isänsä Daavid oli tehnyt.
\par 3 Ensimmäisenä hallitusvuotenansa, sen ensimmäisessä kuussa, hän avasi Herran temppelin ovet ja korjasi ne.
\par 4 Sitten hän tuotti papit ja leeviläiset ja kokosi ne idässäpäin olevalle aukealle.
\par 5 Ja hän sanoi heille: "Kuulkaa minua, te leeviläiset! Pyhittäytykää nyt ja pyhittäkää Herran, isienne Jumalan, temppeli ja toimittakaa saastaisuus pois pyhäköstä.
\par 6 Sillä meidän isämme ovat olleet uskottomia ja tehneet sitä, mikä on pahaa Herran, meidän Jumalamme, silmissä, ja hyljänneet hänet. He käänsivät kasvonsa pois Herran asumuksesta ja käänsivät sille selkänsä.
\par 7 He myöskin sulkivat eteisen ovet, sammuttivat lamput, eivät polttaneet suitsuketta eivätkä uhranneet polttouhreja pyhäkössä Israelin Jumalalle.
\par 8 Sentähden Herran viha on kohdannut Juudaa ja Jerusalemia, ja hän on tehnyt heidät kauhuksi, hämmästykseksi ja pilkaksi, niinkuin te omin silmin näette.
\par 9 Katso, meidän isämme ovat kaatuneet miekkaan, ja meidän poikamme, tyttäremme ja vaimomme ovat joutuneet vankeuteen tästä syystä.
\par 10 Nyt minä aion tehdä liiton Herran, Israelin Jumalan, kanssa, että hänen vihansa hehku kääntyisi meistä pois.
\par 11 Älkää siis, lapseni, olko leväperäisiä, sillä teidät Herra on valinnut seisomaan hänen edessänsä ja palvelemaan häntä, olemaan hänen palvelijansa ja suitsuttamaan hänelle."
\par 12 Silloin nousivat leeviläiset: Mahat, Amasain poika, ja Jooel, Asarjan poika, Kehatin jälkeläisistä; Merarin jälkeläisistä Kiis, Abdin poika, ja Asarja, Jehallelelin poika; geersonilaisista Jooah, Simman poika, ja Eeden, Jooahin poika;
\par 13 Elisafanin jälkeläisistä Simri ja Jegiel; Aasafin jälkeläisistä Sakarja ja Mattanja;
\par 14 Heemanin jälkeläisistä Jehiel ja Siimei; ja Jedutunin jälkeläisistä Semaja ja Ussiel.
\par 15 Nämä kokosivat veljensä, pyhittäytyivät ja menivät, niinkuin kuningas oli Herran sanan mukaan käskenyt, puhdistamaan Herran temppeliä.
\par 16 Mutta papit menivät sisälle Herran temppeliin puhdistamaan sitä, ja kaiken saastaisuuden, minkä löysivät Herran temppelistä, he veivät Herran temppelin esipihalle; sieltä leeviläiset ottivat sen ja veivät sen ulos Kidronin laaksoon.
\par 17 He alkoivat pyhittämisen ensimmäisen kuun ensimmäisenä päivänä, ja kuukauden kahdeksantena päivänä he olivat ehtineet Herran eteiseen, ja he pyhittivät Herran temppeliä kahdeksan päivää; ensimmäisen kuun kuudentenatoista päivänä he lopettivat työnsä.
\par 18 Silloin he menivät sisälle kuningas Hiskian tykö ja sanoivat: "Me olemme puhdistaneet koko Herran temppelin, polttouhrialttarin ja kaikki sen kalut ja näkyleipäpöydän ja kaikki sen kalut.
\par 19 Kaikki kalut, jotka kuningas Aahas hallitusaikanansa uskottomuudessaan saastutti, me olemme panneet kuntoon ja pyhittäneet, ja katso, ne ovat Herran alttarin edessä."
\par 20 Kuningas Hiskia kokosi varhain aamulla kaupungin päämiehet ja meni Herran temppeliin.
\par 21 Ja he toivat seitsemän härkää, seitsemän oinasta ja seitsemän karitsaa sekä seitsemän kaurista syntiuhriksi valtakunnan puolesta, pyhäkön puolesta ja Juudan puolesta. Ja hän käski pappi Aaronin poikien, pappien, uhrata ne Herran alttarilla.
\par 22 Sitten he teurastivat raavaat, ja papit ottivat veren ja vihmoivat sen alttarille; ja he teurastivat oinaat ja vihmoivat veren alttarille; sitten he teurastivat karitsat ja vihmoivat veren alttarille.
\par 23 Senjälkeen he toivat syntiuhrikauriit kuninkaan ja seurakunnan eteen, ja nämä laskivat kätensä niiden päälle.
\par 24 Ja papit teurastivat ne ja uhrasivat niiden veren syntiuhrina alttarilla, toimittaen koko Israelille sovituksen; sillä kuningas oli käskenyt uhrata polttouhrin ja syntiuhrin koko Israelin puolesta.
\par 25 Ja hän asetti leeviläiset Herran temppeliin, kymbaalit, harput ja kanteleet käsissä, niinkuin Daavid ja kuninkaan näkijä Gaad ja profeetta Naatan olivat käskeneet; sillä käsky oli Herran antama hänen profeettainsa kautta.
\par 26 Niin leeviläiset seisoivat siinä, Daavidin soittimet käsissä, ja papit, torvet käsissä.
\par 27 Ja Hiskia käski uhrata alttarilla polttouhrin; ja kun uhraaminen alkoi, alkoi myöskin Herran veisu ja torvien soitto Daavidin, Israelin kuninkaan, soittimien johtaessa.
\par 28 Koko seurakunta kumartaen rukoili, veisu kaikui, ja torvet soivat - kaikkea tätä kesti, kunnes polttouhri oli uhrattu.
\par 29 Kun uhraaminen oli päättynyt, polvistuivat kuningas ja kaikki, jotka olivat hänen kanssaan saapuvilla, kumartaen rukoilemaan.
\par 30 Ja kuningas Hiskia ja päämiehet käskivät leeviläisten ylistää Herraa Daavidin ja näkijä Aasafin sanoilla; ja nämä ylistivät häntä iloiten, polvistuivat ja kumartaen rukoilivat.
\par 31 Sitten Hiskia lausui ja sanoi: "Nyt te olette tuoneet täysin käsin lahjoja Herralle; astukaa esille ja tuokaa teurasuhreja ja kiitosuhreja Herran temppeliin". Silloin seurakunta toi teurasuhreja ja kiitosuhreja, ja jokainen, jonka sydän häntä siihen vaati, myöskin polttouhreja.
\par 32 Ja seurakunnan tuomien polttouhrien lukumäärä oli seitsemänkymmentä raavasta, sata oinasta ja kaksisataa karitsaa, nämä kaikki polttouhriksi Herralle.
\par 33 Ja pyhiä lahjoja oli kuusisataa raavasta ja kolmetuhatta lammasta.
\par 34 Mutta pappeja oli niin vähän, etteivät he voineet nylkeä kaikkia polttouhriteuraita; sentähden heidän veljensä leeviläiset auttoivat heitä, kunnes tämä työ oli suoritettu ja kunnes papit olivat pyhittäytyneet, sillä leeviläiset pyrkivät vilpittömämmin kuin papit pyhittäytymään.
\par 35 Myöskin polttouhreja oli paljon ynnä yhteysuhrirasvoja ja polttouhriin kuuluvia juomauhreja.
\par 36 Näin järjestettiin palvelus Herran temppelissä. Ja Hiskia ja kaikki kansa iloitsivat siitä, mitä Jumala oli kansalle valmistanut, sillä se oli tapahtunut äkisti.

\chapter{30}

\par 1 Hiskia lähetti sanansaattajat kaikkeen Israeliin ja Juudaan ja kirjoitti myös kirjeet Efraimiin ja Manasseen, että he tulisivat Herran temppeliin Jerusalemiin viettämään pääsiäistä Herran, Israelin Jumalan, kunniaksi.
\par 2 Ja kuningas, hänen päämiehensä ja kaikki Jerusalemin seurakunta päättivät viettää pääsiäistä toisessa kuussa,
\par 3 sillä he eivät voineet viettää sitä silloin heti, koska pappeja ei ollut pyhittäytynyt riittävää määrää eikä kansa ollut kokoontunut Jerusalemiin.
\par 4 Tämä kelpasi kuninkaalle ja kaikelle seurakunnalle,
\par 5 ja niin he päättivät kuuluttaa koko Israelissa Beersebasta Daaniin saakka, että oli tultava Jerusalemiin viettämään pääsiäistä Herran, Israelin Jumalan, kunniaksi. Sillä sitä ei oltu vietetty joukolla, niinkuin kirjoitettu oli.
\par 6 Juoksijat kulkivat, mukanaan kuninkaan ja hänen päämiestensä kirjeet, halki koko Israelin ja Juudan ja julistivat kuninkaan käskystä: "Te israelilaiset, palatkaa Herran, Aabrahamin, Iisakin ja Israelin Jumalan, tykö, että hän palajaisi niiden tykö, jotka teistä ovat säilyneet ja pelastuneet Assurin kuningasten käsistä.
\par 7 Älkää olko niinkuin teidän isänne ja veljenne, jotka olivat uskottomat Herralle, isiensä Jumalalle, niin että hän antoi teidät häviön omiksi, niinkuin te itse näette.
\par 8 Älkää siis olko niskureita niinkuin teidän isänne; ojentakaa kätenne Herralle ja tulkaa hänen pyhäkköönsä, jonka hän on pyhittänyt ikuisiksi ajoiksi, ja palvelkaa Herraa, teidän Jumalaanne, että hänen vihansa hehku kääntyisi teistä pois.
\par 9 Sillä jos te palajatte Herran tykö, saavat teidän veljenne ja poikanne osakseen laupeuden voittajiltaan ja voivat palata takaisin tähän maahan. Sillä Herra, teidän Jumalanne, on armollinen ja laupias, eikä hän käännä kasvojansa pois teistä, jos te palajatte hänen tykönsä."
\par 10 Ja juoksijat kulkivat kaupungista kaupunkiin Efraimin ja Manassen maassa ja aina Sebuloniin saakka. Mutta heille naurettiin ja heitä pilkattiin.
\par 11 Kuitenkin Asserista, Manassesta ja Sebulonista muutamat nöyrtyivät ja tulivat Jerusalemiin.
\par 12 Myöskin Juudassa vaikutti Jumalan käsi, ja hän antoi heille yksimielisen sydämen tekemään, mitä kuningas ja päämiehet olivat Herran sanan mukaan käskeneet.
\par 13 Niin kokoontui paljon kansaa Jerusalemiin viettämään happamattoman leivän juhlaa toisessa kuussa. Se oli hyvin suuri kokous.
\par 14 Ja he nousivat ja poistivat alttarit Jerusalemista. Myös kaikki suitsutusalttarit he poistivat ja heittivät Kidronin laaksoon.
\par 15 Sitten he teurastivat pääsiäislampaan toisen kuun neljäntenätoista päivänä. Ja papit ja leeviläiset häpesivät ja pyhittäytyivät ja toivat polttouhreja Herran temppeliin.
\par 16 Ja he asettuivat paikoilleen, niinkuin heidän velvollisuutensa oli Jumalan miehen Mooseksen lain mukaan. Papit vihmoivat veren, otettuaan sen leeviläisten käsistä.
\par 17 Sillä seurakunnassa oli monta, jotka eivät olleet pyhittäytyneet; niin leeviläiset huolehtivat pääsiäislammasten teurastamisesta kaikille, jotka eivät olleet puhtaita, ja niiden pyhittämisestä Herralle.
\par 18 Sillä suurin osa kansasta, monet Efraimista ja Manassesta, Isaskarista ja Sebulonista, eivät olleet puhdistautuneet, vaan söivät pääsiäislampaan toisin, kuin kirjoitettu oli. Hiskia oli näet rukoillut heidän puolestaan sanoen: "Herra, joka on hyvä, antakoon anteeksi
\par 19 jokaiselle, joka on kiinnittänyt sydämensä Jumalan, Herran, isiensä Jumalan, etsimiseen, vaikka onkin pyhäkköpuhtautta vailla".
\par 20 Ja Herra kuuli Hiskiaa ja säästi kansaa.
\par 21 Niin israelilaiset, jotka olivat Jerusalemissa, viettivät seitsemän päivää happamattoman leivän juhlaa iloiten suuresti. Ja leeviläiset ja papit ylistivät joka päivä Herraa voimakkailla soittimilla, Herran kunniaksi.
\par 22 Ja Hiskia puhutteli ystävällisesti kaikkia leeviläisiä, jotka taitavasti suorittivat tehtävänsä Herran kunniaksi. Ja he söivät seitsemän päivää juhlauhreja ja uhrasivat yhteysuhreja ja ylistivät Herraa, isiensä Jumalaa.
\par 23 Ja koko seurakunta päätti viettää juhlaa vielä toiset seitsemän päivää; ja niin vietettiin ilojuhlaa vielä seitsemän päivää.
\par 24 Sillä Hiskia, Juudan kuningas, oli antanut seurakunnalle anniksi tuhat härkää ja seitsemän tuhatta lammasta, ja päämiehet olivat antaneet seurakunnalle anniksi tuhat härkää ja kymmenen tuhatta lammasta. Ja paljon pappeja pyhittäytyi.
\par 25 Niin koko Juudan seurakunta iloitsi ja samoin papit ja leeviläiset ja koko Israelista tullut seurakunta, niin myös muukalaiset, jotka olivat tulleet Israelin maasta tai asuivat Juudassa.
\par 26 Ja Jerusalemissa oli ilo suuri, sillä Israelin kuninkaan Salomon, Daavidin pojan, ajoista asti ei sellaista ollut tapahtunut Jerusalemissa.
\par 27 Ja leeviläiset papit nousivat ja siunasivat kansan; ja heidän äänensä tuli kuulluksi, ja heidän rukouksensa tuli Herran pyhään asumukseen, taivaaseen.

\chapter{31}

\par 1 Kun kaikki tämä oli päättynyt, lähtivät kaikki saapuvilla olevat israelilaiset Juudan kaupunkeihin ja murskasivat patsaat, hakkasivat maahan asera-karsikot ja kukistivat uhrikukkulat ja alttarit perinpohjin koko Juudassa, Benjaminissa, Efraimissa ja Manassessa. Sitten kaikki israelilaiset palasivat kukin perintömaallensa, kaupunkeihinsa.
\par 2 Ja Hiskia asetti pappien ja leeviläisten osastot, heidän osastojensa mukaan, kunkin hänen papillisen tai leeviläisen palvelustehtävänsä mukaan, polttouhreja ja yhteysuhreja toimitettaessa palvelemaan, kiittämään ja ylistämään Herran leirin porteissa.
\par 3 Minkä kuningas antoi omaisuudestaan, se käytettiin polttouhreiksi, aamu- ja ehtoopolttouhreiksi sekä sapatteina, uudenkuun päivinä ja juhlina uhrattaviksi polttouhreiksi, niinkuin on kirjoitettuna Herran laissa.
\par 4 Ja hän käski kansan, Jerusalemin asukasten, antaa papeille ja leeviläisille heidän osuutensa, että nämä pitäisivät kiinni Herran laista.
\par 5 Kun tämä sana levisi, antoivat israelilaiset runsaat uutiset jyvistä, viinistä, öljystä ja hunajasta ja kaikesta pellon sadosta; he toivat runsaat kymmenykset kaikesta.
\par 6 Ja ne israelilaiset ja Juudan miehet, jotka asuivat Juudan kaupungeissa, toivat hekin kymmenykset raavaista ja pikkukarjasta sekä kymmenykset Herralle, Jumalallensa, pyhitetyistä pyhistä lahjoista, ja panivat ne eri kasoihin.
\par 7 Kolmannessa kuussa he alkoivat panna kasoihin, ja seitsemännessä kuussa he sen lopettivat.
\par 8 Kun Hiskia ja päämiehet tulivat ja näkivät kasat, kiittivät he Herraa ja hänen kansaansa Israelia.
\par 9 Ja kun Hiskia kysyi papeilta ja leeviläisiltä näistä kasoista,
\par 10 vastasi hänelle ylimmäinen pappi Asarja, joka oli Saadokin sukua, ja sanoi: "Siitä alkaen, kun ruvettiin tuomaan antia Herran temppeliin, me olemme syöneet ja tulleet ravituiksi, ja kuitenkin on jäänyt paljon tähteeksi; sillä Herra on siunannut kansaansa, ja niin on tämä paljous jäänyt tähteeksi."
\par 11 Niin Hiskia käski laittaa kammioita Herran temppeliin. Ja kun ne oli laitettu,
\par 12 tuotiin niihin tunnollisesti anti, kymmenys ja pyhät lahjat. Ja niiden esimiehenä oli leeviläinen Koonanja, ja hänen veljensä Siimei oli häntä lähinnä.
\par 13 Ja Jehiel, Asasja, Nahat, Asael, Jerimot, Joosabat, Eliel, Jismakja, Mahat ja Benaja olivat Koonanjan ja hänen veljensä Siimein käskyläisinä kuningas Hiskian ja Asarjan, Jumalan temppelin esimiehen, määräyksestä.
\par 14 Ja leeviläinen Koore, Jimnan poika, joka oli ovenvartijana idän puolella, hoiti Jumalalle annetut vapaaehtoiset lahjat ja Herralle tulevan annin ja korkeasti-pyhäin lahjain suorituksen.
\par 15 Ja hänen johdossaan olivat Eedem, Minjamin, Jeesua, Semaja, Amarja ja Sekanja, joiden oli pappiskaupungeissa tunnollisesti suoritettava osuudet veljillensä, niin pienille kuin suurille, osastoittain,
\par 16 paitsi sukuluetteloihin merkityille miehenpuolille, kolmivuotiaille ja sitä vanhemmille, kaikille, jotka menivät Herran temppeliin tekemään palvelusta, kunakin päivänä sen päivän palveluksen, hoitamaan tehtäviään, osastoittain.
\par 17 Papit merkittiin sukuluetteloihin perhekunnittain ja leeviläisistä kaksikymmenvuotiaat ja sitä vanhemmat, palvelustehtäviensä mukaan, osastoittain,
\par 18 nimittäin niin, että sukuluetteloihin merkittiin kaikki heidän pienet lapsensa, vaimonsa, poikansa ja tyttärensä, koko heidän joukkonsa, sillä he hoitivat sitä, mikä pyhää oli, tunnollisesti.
\par 19 Aaronin pojilla, papeilla, jotka asuivat kaupunkiensa laidunmailla, oli joka kaupungissa nimeltä mainitut miehet, joiden oli suoritettava osuudet kaikille miehenpuolille pappien joukossa ja kaikille sukuluetteloihin merkityille leeviläisille.
\par 20 Näin Hiskia teki koko Juudassa; hän teki sitä, mikä oli hyvää, oikeata ja totta Herran, hänen Jumalansa, edessä.
\par 21 Ja kaiken, mihin hän ryhtyi etsiessään Jumalaansa, koskipa se palvelusta Jumalan temppelissä tai lakia ja käskyjä, sen hän teki kaikesta sydämestänsä, ja hän menestyi.

\chapter{32}

\par 1 Näiden tapausten ja Hiskian vakaan vaelluksen jälkeen tuli Sanherib, Assurin kuningas, ja hyökkäsi Juudaan, piiritti sen varustettuja kaupunkeja ja aikoi valloittaa ne itsellensä.
\par 2 Kun Hiskia näki Sanheribin tulevan ja olevan aikeessa sotia Jerusalemia vastaan,
\par 3 neuvotteli hän päämiestensä ja urhojensa kanssa tukkiaksensa vesilähteet, jotka olivat kaupungin ulkopuolella; ja he kannattivat häntä siinä.
\par 4 Niin kokoontui paljon kansaa, ja he tukkivat kaikki lähteet ja puron, joka juoksee sen seudun läpi; sillä he sanoivat: "Miksi Assurin kuninkaat tultuansa löytäisivät vettä viljalti?"
\par 5 Ja hän rohkaisi mielensä, rakensi koko revityn osan muuria ja korotti torneja ja teki sen ulkopuolelle toisen muurin, varusti Millon Daavidin kaupungissa ja teetti paljon heittoaseita ja kilpiä.
\par 6 Ja hän asetti sotapäälliköitä kansalle ja kokosi heidät luoksensa aukealle kaupungin portin eteen ja puhui ystävällisesti heille ja sanoi:
\par 7 "Olkaa lujat ja rohkeat, älkää peljätkö älkääkä arkailko Assurin kuningasta ja kaikkea joukkoa, joka on hänen kanssansa; sillä se, joka on meidän kanssamme, on suurempi kuin se, joka on hänen kanssansa.
\par 8 Hänen kanssansa on lihan käsivarsi, mutta meidän kanssamme on Herra, meidän Jumalamme, meitä auttamassa ja meidän sotiamme sotimassa." Ja kansa luotti Hiskian, Juudan kuninkaan, sanoihin.
\par 9 Tämän jälkeen Sanherib, Assurin kuningas, ollessaan itse koko sotavoimansa kanssa Laakiin edustalla, lähetti palvelijoitansa Jerusalemiin Hiskian, Juudan kuninkaan, luo ja kaikkien Jerusalemissa olevien Juudan miesten luo ja käski sanoa:
\par 10 "Näin sanoo Sanherib, Assurin kuningas: Mihin te luotatte, kun jäätte Jerusalemiin saarroksiin?
\par 11 Eikö Hiskia viettele teitä ja saata teitä kuolemaan nälkään ja janoon, kun hän sanoo: Herra, meidän Jumalamme, pelastaa meidät Assurin kuninkaan käsistä?
\par 12 Eikö tämä sama Hiskia ole poistanut hänen uhrikukkuloitansa ja alttareitansa ja sanonut Juudalle ja Jerusalemille näin: 'Yhden ainoan alttarin edessä on teidän kumartaen rukoiltava ja sillä uhreja poltettava'?
\par 13 Ettekö tiedä, mitä minä ja minun isäni olemme tehneet muiden maiden kaikille kansoille? Ovatko näiden maiden kansain jumalat voineet pelastaa maatansa minun käsistäni?
\par 14 Onko kukaan minun isieni tuhoamien kansojen kaikista jumalista voinut pelastaa kansaansa minun käsistäni? Kuinka sitten teidän Jumalanne voisi pelastaa teidät minun käsistäni?
\par 15 Älkää siis antako Hiskian näin pettää ja vietellä itseänne älkääkä uskoko häntä, sillä ei minkään kansan eikä minkään valtakunnan jumala ole voinut pelastaa kansaansa minun käsistäni tai minun isieni käsistä, saati sitten teidän Jumalanne: ei hän pelasta teitä minun käsistäni."
\par 16 Ja hänen palvelijansa puhuivat vielä enemmän Herraa Jumalaa ja hänen palvelijaansa Hiskiaa vastaan.
\par 17 Hän kirjoitti myös kirjeen herjatakseen Herraa, Israelin Jumalaa, ja puhuakseen häntä vastaan; siinä sanottiin näin: "Niinkuin muiden maiden kansain jumalat eivät ole pelastaneet kansojansa minun käsistäni, niin ei Hiskiankaan Jumala ole pelastava kansaansa minun käsistäni".
\par 18 Ja Jerusalemin kansalle, jota oli muurilla, he huusivat kovalla äänellä juudankielellä nostaakseen heissä pelon ja kauhun, että saisivat kaupungin valloitetuksi.
\par 19 Ja he puhuivat Jerusalemin Jumalasta niinkuin muiden maiden kansain jumalista, jotka ovat ihmiskätten tekoa.
\par 20 Mutta kuningas Hiskia ja profeetta Jesaja, Aamoksen poika, rukoilivat tämän tähden ja huusivat taivaan puoleen.
\par 21 Silloin Herra lähetti enkelin, joka tuhosi kaikki sotaurhot, ruhtinaat ja päälliköt Assurin kuninkaan leirissä, niin että hän häpeä kasvoillaan palasi takaisin maahansa. Ja kun hän kerran meni jumalansa temppeliin, kaatoivat hänet siellä miekalla ne, jotka olivat lähteneet hänen omasta ruumiistansa.
\par 22 Näin Herra vapahti Hiskian ja Jerusalemin asukkaat Sanheribin, Assurin kuninkaan, käsistä ja kaikkien käsistä, ja hän johdatti heitä joka taholla.
\par 23 Ja monet toivat Herralle lahjoja Jerusalemiin ja Hiskialle, Juudan kuninkaalle, kalleuksia; ja hän kohosi sen jälkeen kaikkien kansojen silmissä.
\par 24 Niihin aikoihin Hiskia sairastui ja oli kuolemaisillaan. Niin hän rukoili Herraa, ja hän vastasi hänelle ja antoi hänelle tapahtua ihmeen.
\par 25 Mutta Hiskia ei palkinnut hyvää hyvällä, vaan hänen sydämensä ylpistyi; sentähden hänen ja Juudan ja Jerusalemin osaksi tuli viha.
\par 26 Mutta kun Hiskia nöyrtyi sydämensä ylpeydestä, hän ja Jerusalemin asukkaat, ei Herran viha kohdannut heitä Hiskian päivinä.
\par 27 Hiskialle tuli ylen paljon rikkautta ja kunniaa; hän hankki itselleen aarrekammioita hopeata, kultaa, kalliita kiviä, hajuaineita, kilpiä ja kaikkinaisia kallisarvoisia kaluja varten,
\par 28 varastohuoneita jyvä-, viini- ja öljysatoa varten ja talleja kaikenlaisille juhdille ja karjalaumoja tarhoihin.
\par 29 Myös rakennutti hän itsellensä kaupunkeja ja hankki suuret laumat pikkukarjaa ja raavaskarjaa, sillä Jumala oli antanut hänelle ylen suuren omaisuuden.
\par 30 Hiskia myös tukki Giihonin veden yläjuoksun ja johti veden alas, länteenpäin, Daavidin kaupunkiin. Ja Hiskia menestyi kaikessa, mitä hän teki.
\par 31 Niinpä Jumala ainoastaan koetellakseen häntä ja tullakseen tuntemaan kaiken, mitä hänen sydämessään oli, jätti hänet Baabelin ruhtinasten lähettiläiden valtaan, jotka olivat lähetetyt hänen luokseen tiedustelemaan ihmettä, joka oli tapahtunut maassa.
\par 32 Mitä muuta on kerrottavaa Hiskiasta ja hänen hurskaista teoistansa, katso, se on kirjoitettuna profeetta Jesajan, Aamoksen pojan, näyssä, Juudan ja Israelin kuningasten kirjassa.
\par 33 Ja Hiskia meni lepoon isiensä tykö, ja hänet haudattiin siihen, mistä noustaan Daavidin jälkeläisten haudoille. Ja koko Juuda ja Jerusalemin asukkaat osoittivat hänelle kunniaa, kun hän kuoli. Ja hänen poikansa Manasse tuli kuninkaaksi hänen sijaansa.

\chapter{33}

\par 1 Manasse oli kahdentoista vuoden vanha tullessansa kuninkaaksi, ja hän hallitsi Jerusalemissa viisikymmentä viisi vuotta.
\par 2 Hän teki sitä, mikä on pahaa Herran silmissä, niiden kansain kauhistavien tekojen mukaan, jotka Herra oli karkoittanut israelilaisten tieltä.
\par 3 Hän rakensi jälleen uhrikukkulat, jotka hänen isänsä Hiskia oli kukistanut, pystytti alttareja baaleille, teki aseroja ja kumarsi ja palveli kaikkea taivaan joukkoa.
\par 4 Hän rakensi alttareja myös Herran temppeliin, josta Herra oli sanonut: "Jerusalemissa on minun nimeni oleva iankaikkisesti".
\par 5 Hän rakensi alttareja kaikelle taivaan joukolle Herran temppelin molempiin esipihoihin.
\par 6 Myös pani hän poikansa kulkemaan tulen läpi Ben-Hinnomin laaksossa, ennusteli merkeistä, harjoitti noituutta ja velhoutta ja hankki itsellensä vainaja- ja tietäjähenkien manaajia; hän teki paljon sitä, mikä on pahaa Herran silmissä, ja vihoitti hänet.
\par 7 Ja teettämänsä veistetyn kuvapatsaan hän asetti Jumalan temppeliin, josta Jumala oli sanonut Daavidille ja hänen pojallensa Salomolle: "Tähän temppeliin ja Jerusalemiin, jonka minä olen valinnut kaikista Israelin sukukunnista, minä asetan nimeni ikiajoiksi.
\par 8 Enkä minä enää kuljeta Israelin jalkoja pois siitä maasta, jonka olen määrännyt teidän isillenne, jos Israel vain tarkoin noudattaa kaikkea, mitä minä olen käskenyt heidän noudattaa, koko lakia ja käskyjä ja oikeuksia, jotka ovat Mooseksen kautta annetut."
\par 9 Mutta Manasse eksytti Juudan ja Jerusalemin asukkaat tekemään enemmän pahaa, kuin olivat tehneet ne kansat, jotka Herra oli hävittänyt israelilaisten tieltä.
\par 10 Ja Herra puhui Manasselle ja hänen kansallensa, mutta he eivät kuunnelleet.
\par 11 Niin Herra toi Assurin kuninkaan sotapäälliköt heidän kimppuunsa. He ottivat Manassen kiinni koukuilla, kytkivät hänet vaskikahleisiin ja veivät hänet Baabeliin.
\par 12 Mutta ahdingossa ollessaan hän etsi Herran, Jumalansa, mielisuosiota ja nöyrtyi syvästi isiensä Jumalan edessä.
\par 13 Ja kun hän näin rukoili häntä, niin Jumala taipui ja kuuli hänen rukouksensa ja toi hänet takaisin Jerusalemiin, hänen valtakuntaansa. Silloin Manasse tuli tietämään, että Herra on Jumala.
\par 14 Sen jälkeen hän rakennutti Daavidin kaupungin ulomman muurin, länteen päin Giihonista, laaksoon, aina Kalaporttiin saakka, niin että se ympäröi Oofelin, ja hän teki siitä hyvin korkean. Ja hän asetti sotapäälliköitä kaikkiin Juudan varustettuihin kaupunkeihin.
\par 15 Ja hän poisti vieraat jumalat ja kuvapatsaan Herran temppelistä sekä kaikki alttarit, jotka hän oli rakennuttanut Herran temppelin vuorelle ja Jerusalemiin, ja heitätti ne kaupungin ulkopuolelle.
\par 16 Ja hän pani kuntoon Herran alttarin ja uhrasi sillä yhteys- ja kiitosuhreja ja kehoitti Juudaa palvelemaan Herraa, Israelin Jumalaa.
\par 17 Mutta kansa uhrasi edelleen uhrikukkuloilla, kuitenkin ainoastaan Herralle, Jumalallensa.
\par 18 Mitä muuta on kerrottavaa Manassesta, ja kuinka hän rukoili Jumalaansa, ja näkijäin puheista, jotka puhuivat hänelle Herran, Israelin Jumalan, nimessä, katso, se on Israelin kuningasten aikakirjassa.
\par 19 Ja hänen rukouksestaan ja kuinka hän tuli kuulluksi, ja kaikista hänen synneistään ja uskottomuudestaan ja niistä paikoista, joihin hän rakennutti uhrikukkuloita ja pystytti asera-karsikoita ja jumalankuvia, ennenkuin hän nöyrtyi, katso, niistä on kirjoitettu Hoosain aikakirjaan.
\par 20 Ja Manasse meni lepoon isiensä tykö, ja hänet haudattiin linnaansa. Ja hänen poikansa Aamon tuli kuninkaaksi hänen sijaansa.
\par 21 Aamon oli kahdenkymmenen kahden vuoden vanha tullessaan kuninkaaksi, ja hän hallitsi Jerusalemissa kaksi vuotta.
\par 22 Hän teki sitä, mikä on pahaa Herran silmissä, niinkuin hänen isänsä Manasse oli tehnyt. Ja Aamon uhrasi kaikille niille jumalankuville, jotka hänen isänsä Manasse oli teettänyt, ja palveli niitä.
\par 23 Mutta hän ei nöyrtynyt Herran edessä, niinkuin hänen isänsä Manasse oli nöyrtynyt, vaan hän, Aamon, sälytti päällensä suuren syntivelan.
\par 24 Niin hänen palvelijansa tekivät salaliiton häntä vastaan ja tappoivat hänet hänen linnassansa.
\par 25 Mutta maan kansa surmasi kaikki ne, jotka olivat tehneet salaliiton kuningas Aamonia vastaan; ja maan kansa teki hänen poikansa Joosian kuninkaaksi hänen sijaansa.

\chapter{34}

\par 1 Joosia oli kahdeksan vuoden vanha tullessaan kuninkaaksi, ja hän hallitsi Jerusalemissa kolmekymmentä yksi vuotta.
\par 2 Hän teki sitä, mikä on oikein Herran silmissä, ja vaelsi isänsä Daavidin teitä, poikkeamatta oikealle tai vasemmalle.
\par 3 Kahdeksantena hallitusvuotenaan, ollessansa vielä nuorukainen, hän alkoi etsiä isänsä Daavidin Jumalaa; ja kahdentenatoista vuotena hän alkoi puhdistaa Juudaa ja Jerusalemia uhrikukkuloista ja asera-karsikoista sekä veistetyistä ja valetuista jumalankuvista.
\par 4 Baalin alttarit kukistettiin hänen läsnäollessaan, ja niiden yli kohoavat auringonpatsaat hän hakkasi maahan, ja asera-karsikot sekä veistetyt ja valetut jumalankuvat hän murskasi ja rouhensi ja sirotteli niiden haudoille, jotka olivat niille uhranneet.
\par 5 Ja pappien luut hän poltti heidän alttareillaan. Niin hän puhdisti Juudan ja Jerusalemin.
\par 6 Manassen, Efraimin ja Simeonin kaupungeissa aina Naftaliin asti, yltympäri, hän heidän miekoillaan
\par 7 kukisti alttarit, löi palasiksi ja rouhensi asera-karsikot ja jumalankuvat, ja hän hakkasi maahan kaikki auringonpatsaat koko Israelin maasta. Sitten hän palasi Jerusalemiin.
\par 8 Kahdeksantenatoista hallitusvuotenaan, puhdistaessaan maata ja temppeliä, hän lähetti Saafanin, Asaljan pojan, kaupungin päällikön Maasejan ja kansleri Jooahin, Jooahaan pojan, korjaamaan Herran, hänen Jumalansa, temppeliä.
\par 9 Ja he tulivat ylimmäisen papin Hilkian luo ja jättivät Jumalan temppeliin tuodut rahat, joita leeviläiset, ovenvartijat, olivat koonneet Manassesta, Efraimista ja koko muusta Israelista ja koko Juudasta, Benjaminista ja Jerusalemin asukkailta.
\par 10 He antoivat ne työnteettäjille, jotka oli pantu valvomaan töitä Herran temppelissä, ja nämä maksoivat niillä työmiehet, jotka työskentelivät Herran temppelissä, laittoivat ja korjasivat temppeliä;
\par 11 niistä annettiin myös puusepille ja rakentajille, että nämä ostaisivat hakattuja kiviä ja puutavaraa kiinnitysparruiksi ja kattohirsiksi niihin rakennuksiin, jotka Juudan kuninkaat olivat turmelleet.
\par 12 Nämä miehet toimivat siinä työssä luottamusmiehinä. Ja leeviläiset Jahat ja Obadja, Merarin jälkeläiset, ja Sakarja ja Mesullam, Kehatin jälkeläiset, oli pantu valvomaan ja johtamaan heitä, ja myös kaikki leeviläiset, jotka ymmärsivät soittimia.
\par 13 He myöskin valvoivat taakankantajia ja johtivat kaikkia työmiehiä eri töissä. Leeviläisiä oli kirjureina, virkamiehinä ja ovenvartijoina.
\par 14 Kun he olivat viemässä ulos Herran temppeliin tuotuja rahoja, löysi pappi Hilkia Herran lain kirjan, joka oli annettu Mooseksen kautta.
\par 15 Ja Hilkia lausui ja sanoi kirjuri Saafanille: "Minä löysin Herran temppelistä lain kirjan". Ja Hilkia antoi kirjan Saafanille.
\par 16 Saafan vei kirjan kuninkaalle ja teki lisäksi kuninkaalle selon asiasta, sanoen: "Kaikki, mitä annettiin palvelijaisi tehtäväksi, he ovat tehneet:
\par 17 he ovat ottaneet esille rahat, jotka olivat Herran temppelissä, ja antaneet ne työnvalvojille ja työnteettäjille".
\par 18 Ja kirjuri Saafan kertoi kuninkaalle sanoen: "Pappi Hilkia antoi minulle erään kirjan". Ja Saafan luki siitä kuninkaalle.
\par 19 Kun kuningas kuuli lain sanat, repäisi hän vaatteensa.
\par 20 Ja kuningas käski Hilkiaa, Ahikamia, Saafanin poikaa, Abdonia, Miikan poikaa, kirjuri Saafania ja Asajaa, kuninkaan palvelijaa, sanoen:
\par 21 "Menkää ja kysykää minun puolestani ja niiden puolesta, joita on jäljellä Israelista ja Juudasta, neuvoa Herralta tästä löydetystä kirjasta. Sillä suuri on Herran viha, joka on vuodatettu meidän ylitsemme, sentähden että meidän isämme eivät ole noudattaneet Herran sanaa eivätkä tehneet mitään kaikesta siitä, mikä on kirjoitettuna tässä kirjassa."
\par 22 Niin Hilkia ynnä ne muut, jotka kuningas määräsi, menivät naisprofeetta Huldan tykö, joka oli vaatevaraston hoitajan Sallumin, Tokhatin pojan, Hasran pojanpojan, vaimo ja asui Jerusalemissa, toisessa kaupunginosassa. Ja he puhuivat hänelle, niinkuin edellä mainittiin.
\par 23 Niin hän sanoi heille: "Näin sanoo Herra, Israelin Jumala: Sanokaa sille miehelle, joka lähetti teidät minun tyköni:
\par 24 'Näin sanoo Herra: Katso, minä annan onnettomuuden kohdata tätä paikkaa ja sen asukkaita, kaikkien kirousten, jotka on kirjoitettu siihen kirjaan, jota on luettu Juudan kuninkaalle,
\par 25 koska he ovat hyljänneet minut ja polttaneet uhreja muille jumalille ja vihoittaneet minut kaikilla kättensä teoilla; sillä minun vihani on vuodatettu tämän paikan yli, eikä se ole sammuva.
\par 26 Mutta Juudan kuninkaalle, joka on lähettänyt teidät kysymään neuvoa Herralta, sanokaa näin: Näin sanoo Herra, Israelin Jumala, niistä sanoista, jotka sinä olet kuullut:
\par 27 Koska sydämesi on pehminnyt ja sinä olet nöyrtynyt Jumalan edessä, kuullessasi, mitä hän on puhunut tätä paikkaa ja sen asukkaita vastaan, koska sinä olet nöyrtynyt minun edessäni ja reväissyt vaatteesi ja itkenyt minun edessäni, niin minä myös olen kuullut sinua, sanoo Herra.
\par 28 Katso, minä korjaan sinut isiesi tykö, ja sinä saat rauhassa siirtyä hautaasi, ja sinun silmäsi pääsevät näkemästä kaikkea onnettomuutta, minkä minä annan kohdata tätä paikkaa ja sen asukkaita.'" Ja he toivat kuninkaalle tämän vastauksen.
\par 29 Niin kuningas lähetti kokoamaan luoksensa kaikki Juudan ja Jerusalemin vanhimmat.
\par 30 Ja kuningas meni Herran temppeliin, hän ja kaikki Juudan miehet ja Jerusalemin asukkaat, myöskin papit ja leeviläiset, kaikki kansa suurimmasta pienimpään asti, ja hän luki heidän kuultensa kaikki Herran temppelistä löydetyn liitonkirjan sanat.
\par 31 Ja kuningas asettui paikallensa ja teki Herran edessä liiton, että heidän tuli seurata Herraa, noudattaa hänen käskyjänsä, todistuksiansa ja säädöksiänsä kaikesta sydämestään ja kaikesta sielustansa ja täyttää liiton sanat, jotka ovat kirjoitetut siihen kirjaan.
\par 32 Ja hän otti siihen liittoon kaikki, jotka olivat Jerusalemissa ja Benjaminissa. Ja Jerusalemin asukkaat tekivät, niinkuin Jumalan, heidän isiensä Jumalan, liitto vaati.
\par 33 Ja Joosia poisti kaikki kauhistukset israelilaisten kaikista maakunnista ja vaati jokaista, joka Israelissa oli, palvelemaan Herraa, heidän Jumalaansa. Niin kauan kuin hän eli, he eivät luopuneet pois Herrasta, isiensä Jumalasta.

\chapter{35}

\par 1 Sitten Joosia vietti Jerusalemissa pääsiäistä Herran kunniaksi. Pääsiäislammas teurastettiin ensimmäisen kuun neljäntenätoista päivänä.
\par 2 Hän asetti papit heidän palvelustehtäviinsä ja rohkaisi heitä palvelemaan Herran temppelissä.
\par 3 Ja hän sanoi leeviläisille, jotka opettivat kaikkea Israelia ja jotka olivat pyhitetyt Herralle: "Pankaa pyhä arkki temppeliin, jonka Salomo, Daavidin poika, Israelin kuningas, on rakentanut; ei teidän enää tarvitse kantaa sitä olallanne. Palvelkaa nyt Herraa, Jumalaanne, ja hänen kansaansa Israelia.
\par 4 Valmistautukaa perhekunnittain, osastojenne mukaan, Daavidin, Israelin kuninkaan, määräyksen ja hänen poikansa Salomon käskykirjan mukaan,
\par 5 ja asettukaa pyhäkköön veljienne, rahvaan, perhekuntaryhmien mukaan, niin että jokaista ryhmää kohden on yksi leeviläisten perhekuntaosasto.
\par 6 Niin teurastakaa pääsiäislammas ja pyhittäytykää ja valmistakaa se veljillenne, tehdäksenne Herran sanan mukaan, joka on puhuttu Mooseksen kautta."
\par 7 Ja Joosia antoi rahvaalle anniksi pikkukarjaa, karitsoita ja vohlia, kolmekymmentä tuhatta luvultaan, kaikki pääsiäisuhreiksi kaikille saapuvilla oleville, ja kolmetuhatta raavasta, kaikki nämä kuninkaan omaisuutta.
\par 8 Ja hänen päämiehensä antoivat vapaaehtoisesti annin kansalle, papeille ja leeviläisille. Hilkia, Sakarja ja Jehiel, Jumalan temppelin esimiehet, antoivat papeille pääsiäisuhreiksi kaksituhatta kuusisataa karitsaa, niin myös kolmesataa raavasta.
\par 9 Ja Koonanja, Semaja ja hänen veljensä Netanel, sekä Hasabja, Jegiel ja Joosabad, leeviläisten päämiehet, antoivat antina leeviläisille pääsiäisuhreiksi viisituhatta karitsaa ja viisisataa raavasta.
\par 10 Näin järjestettiin jumalanpalvelus. Ja papit asettuivat paikoilleen ja samoin leeviläiset osastoittain, niinkuin kuningas oli käskenyt.
\par 11 Sitten teurastettiin pääsiäislammas, ja papit vihmoivat veren, jonka olivat ottaneet leeviläisten käsistä; ja leeviläiset nylkivät nahan.
\par 12 Mutta he panivat syrjään polttouhrikappaleet, antaakseen ne rahvaalle perhekuntaryhmittäin, uhrattaviksi Herralle, niinkuin on kirjoitettuna Mooseksen kirjassa; samoin he tekivät myös raavaille.
\par 13 Ja he paistoivat tulella pääsiäislampaan säädetyllä tavalla; mutta pyhät lahjat he keittivät padoissa, ruukuissa ja vadeissa ja veivät ne kiiruusti rahvaalle.
\par 14 Sitten he valmistivat itsellensä ja papeille, sillä papit, Aaronin pojat, olivat yöhön saakka uhraamassa polttouhria ja rasvoja; sentähden leeviläiset valmistivat itsellensä ja papeille, Aaronin pojille.
\par 15 Ja veisaajat, Aasafin pojat, olivat paikoillansa, niinkuin Daavid, Aasaf, Heeman ja Jedutun, kuninkaan näkijä, olivat määränneet, niin myös ovenvartijat jokaisella portilla; heidän ei ollut lupa lähteä palveluksestaan, vaan heidän leeviläiset veljensä valmistivat heille.
\par 16 Näin järjestettiin kaikki Herran palvelus sinä päivänä, pääsiäisen vietto ja polttouhrien uhraaminen Herran alttarilla, niinkuin kuningas Joosia oli käskenyt.
\par 17 Näin viettivät saapuvilla olevat israelilaiset sinä aikana pääsiäistä ja happamattoman leivän juhlaa seitsemän päivää.
\par 18 Sellaista pääsiäistä ei oltu vietetty Israelissa profeetta Samuelin ajoista asti; sillä ei kukaan Israelin kuninkaista ollut viettänyt semmoista pääsiäistä, kuin nyt viettivät Joosia, papit ja leeviläiset, koko Juuda ja saapuvilla olevat israelilaiset sekä Jerusalemin asukkaat.
\par 19 Joosian kahdeksantenatoista hallitusvuotena vietettiin tämä pääsiäinen.
\par 20 Kaiken tämän jälkeen, sittenkuin Joosia oli pannut kuntoon temppelin, lähti Neko, Egyptin kuningas, sotimaan Karkemista vastaan, joka on Eufratin varrella; ja Joosia meni häntä vastaan.
\par 21 Niin Neko lähetti sanansaattajat hänen luokseen ja käski sanoa: "Mitä sinulla on tekemistä minun kanssani, Juudan kuningas? Enhän minä nyt tule sinua vastaan, vaan sitä sukua vastaan, joka on sodassa minun kanssani, ja Jumala on käskenyt minua kiiruhtamaan. Jätä rauhaan Jumala, joka on minun kanssani, ettei hän tuhoaisi sinua."
\par 22 Mutta Joosia ei väistänyt häntä, vaan pukeutui tuntemattomaksi taistellakseen häntä vastaan, eikä kuullut Nekon sanoja, jotka kuitenkin tulivat Jumalan suusta. Niin hän meni taistelemaan Megiddon tasangolle.
\par 23 Mutta ampujat ampuivat kuningas Joosiaa; ja kuningas sanoi palvelijoillensa: "Viekää minut pois, sillä minä olen pahasti haavoittunut".
\par 24 Niin hänen palvelijansa siirsivät hänet sotavaunuista ja panivat hänet hänen toisiin vaunuihinsa ja kuljettivat hänet Jerusalemiin. Ja hän kuoli, ja hänet haudattiin isiensä hautoihin. Ja koko Juuda ja Jerusalem surivat Joosiaa.
\par 25 Ja Jeremia sepitti itkuvirren Joosiasta. Ja kaikki laulajat ja laulajattaret ovat itkuvirsissään puhuneet Joosiasta aina tähän päivään asti; ja nämä ovat tulleet yleisiksi Israelissa. Katso, ne ovat kirjoitettuina "Itkuvirsissä".
\par 26 Mitä muuta on kerrottavaa Joosiasta ja hänen hurskaista teoistansa, joita hän teki sen mukaan, kuin on kirjoitettuna Herran laissa,
\par 27 hänen sekä aikaisemmista että myöhemmistä vaiheistaan, katso, se on kirjoitettuna Israelin ja Juudan kuningasten kirjassa.

\chapter{36}

\par 1 Maan kansa otti Joosian pojan Jooahaan ja teki hänet kuninkaaksi Jerusalemiin hänen isänsä jälkeen.
\par 2 Jooahas oli kahdenkymmenen kolmen vuoden vanha tullessaan kuninkaaksi, ja hän hallitsi Jerusalemissa kolme kuukautta.
\par 3 Mutta Egyptin kuningas pani hänet viralta Jerusalemissa ja otti maasta pakkoverona sata talenttia hopeata ja yhden talentin kultaa.
\par 4 Ja Egyptin kuningas teki hänen veljensä Eljakimin Juudan ja Jerusalemin kuninkaaksi ja muutti hänen nimensä Joojakimiksi. Mutta hänen veljensä Jooahaan Neko otti ja vei Egyptiin.
\par 5 Joojakim oli kahdenkymmenen viiden vuoden vanha tullessaan kuninkaaksi, ja hän hallitsi Jerusalemissa yksitoista vuotta. Hän teki sitä, mikä oli pahaa Herran, hänen Jumalansa, silmissä.
\par 6 Ja Nebukadnessar, Baabelin kuningas, lähti häntä vastaan ja kytki hänet vaskikahleisiin viedäkseen hänet Baabeliin.
\par 7 Nebukadnessar vei Herran temppelin kaluja Baabeliin ja pani ne temppeliinsä Baabelissa.
\par 8 Mitä muuta on kerrottavaa Joojakimista ja niistä kauhistuksista, joita hän teki, ja mitä hänessä havaittiin, katso, se on kirjoitettuna Israelin ja Juudan kuningasten kirjassa. Ja hänen poikansa Joojakin tuli kuninkaaksi hänen sijaansa.
\par 9 Joojakin oli kahdeksantoista vuoden vanha tullessansa kuninkaaksi, ja hän hallitsi Jerusalemissa kolme kuukautta ja kymmenen päivää. Hän teki sitä, mikä on pahaa Herran silmissä.
\par 10 Vuoden vaihteessa kuningas Nebukadnessar lähetti noutamaan hänet Baabeliin, hänet ynnä Herran temppelin kallisarvoiset kalut. Ja hän teki hänen veljensä Sidkian Juudan ja Jerusalemin kuninkaaksi.
\par 11 Sidkia oli kahdenkymmenen yhden vuoden vanha tullessaan kuninkaaksi, ja hän hallitsi Jerusalemissa yksitoista vuotta.
\par 12 Hän teki sitä, mikä oli pahaa Herran, hänen Jumalansa, silmissä: hän ei nöyrtynyt profeetta Jeremian edessä, jonka sana tuli Herran suusta.
\par 13 Myöskin hän kapinoi kuningas Nebukadnessaria vastaan, joka kuitenkin oli vannottanut hänet Jumalan kautta. Hän oli niskuri ja paadutti sydämensä, niin ettei hän kääntynyt Herran, Israelin Jumalan, puoleen.
\par 14 Myös kaikki pappien päämiehet ja kansa harjoittivat paljon uskottomuutta jäljittelemällä pakanain kaikkia kauhistuksia, ja he saastuttivat Herran temppelin, jonka hän oli pyhittänyt Jerusalemissa.
\par 15 Ja Herra, heidän isiensä Jumala, lähetti, varhaisesta alkaen, vähän väliä heille varoituksia sanansaattajainsa kautta, sillä hän sääli kansaansa ja asumustansa.
\par 16 Mutta he pilkkasivat Jumalan sanansaattajia ja halveksivat hänen sanaansa ja häpäisivät hänen profeettojansa, kunnes Herran viha hänen kansaansa kohtaan oli kasvanut niin, ettei apua enää ollut.
\par 17 Niin hän toi heidän kimppuunsa kaldealaisten kuninkaan ja surmautti miekalla heidän nuoret miehensä heidän pyhäkössänsä eikä säästänyt nuorukaista eikä neitosta, ei vanhusta eikä harmaapäätä; kaikki hän antoi tämän käsiin.
\par 18 Ja kaikki Jumalan temppelin kalut, sekä suuret että pienet, ja Herran temppelin aarteet sekä kuninkaan ja hänen päämiestensä aarteet, kaikki hän vei Baabeliin.
\par 19 Jumalan temppeli poltettiin, Jerusalemin muurit revittiin, kaikki sen palatsit poltettiin tulella, ja kaikki sen kallisarvoiset esineet hävitettiin.
\par 20 Ja jotka olivat säilyneet miekalta, ne vietiin pakkosiirtolaisuuteen Baabeliin. Ja he olivat hänen ja hänen poikiensa palvelijoina, kunnes Persian valtakunta sai vallan.
\par 21 Ja niin toteutui Herran sana, jonka hän oli puhunut Jeremian suun kautta, kunnes maa oli saanut hyvityksen sapateistaan - niin kauan kuin se oli autiona, se lepäsi - kunnes seitsemänkymmentä vuotta oli kulunut.
\par 22 Mutta Kooreksen, Persian kuninkaan, ensimmäisenä hallitusvuotena herätti Herra, että täyttyisi Herran sana, jonka hän oli puhunut Jeremian suun kautta, Kooreksen, Persian kuninkaan, hengen, niin että tämä koko valtakunnassansa kuulutti ja myös käskykirjassa julistutti näin:
\par 23 "Näin sanoo Koores, Persian kuningas: Kaikki maan valtakunnat on Herra, taivaan Jumala, antanut minulle, ja hän on käskenyt minun rakentaa itsellensä temppelin Jerusalemiin, joka on Juudassa. Kuka vain teidän joukossanne on hänen kansaansa, sen kanssa olkoon Herra, hänen Jumalansa, ja hän menköön sinne."


\end{document}