\begin{document}

\title{Nehemian kirja}


\chapter{1}

\par 1 Nehemian, Hakaljan pojan, kertomus. Kislev-kuussa, kahdentenakymmenentenä vuotena, minun ollessani Suusanin linnassa,
\par 2 tuli Hanani, eräs minun veljistäni, ja muita miehiä Juudasta. Ja minä kyselin heiltä juutalaisista, siitä pelastuneesta joukosta, joka vankeudesta palanneena oli jäljellä, ja Jerusalemista.
\par 3 Ja he sanoivat minulle: "Jotka vankeudesta palanneina ovat jäljellä siinä maakunnassa, ne ovat suuressa kurjuudessa ja häväistyksen alaisina; ja Jerusalemin muuri on revitty maahan, ja sen portit ovat tulella poltetut".
\par 4 Tämän kuultuani minä istuin monta päivää itkien ja murehtien, ja minä paastosin ja rukoilin taivaan Jumalan edessä.
\par 5 Minä sanoin: "Oi Herra, taivaan Jumala, sinä suuri ja peljättävä Jumala, joka pidät liiton ja säilytät laupeuden niille, jotka sinua rakastavat ja noudattavat sinun käskyjäsi,
\par 6 olkoon sinun korvasi tarkkaavainen ja silmäsi avoin kuullaksesi palvelijasi rukoukset, joita minä nyt päivät ja yöt rukoilen sinun edessäsi palvelijaisi, israelilaisten, puolesta, tunnustaen israelilaisten synnit, jotka me olemme tehneet sinua vastaan; sillä minäkin ja minun perhekuntani olemme syntiä tehneet!
\par 7 Me olemme pahoin tehneet sinua vastaan; emme ole noudattaneet käskyjä, säädöksiä ja oikeuksia, jotka sinä annoit palvelijallesi Moosekselle.
\par 8 Muista sana, jonka sinä annoit palvelijallesi Moosekselle, sanoen: 'Jos te tulette uskottomiksi, niin minä hajotan teidät kansojen sekaan;
\par 9 mutta jos te palajatte minun tyköni, noudatatte minun käskyjäni ja seuraatte niitä, niin vaikka teidän karkoitettunne olisivat taivaan ääressä, minä heidät sieltäkin kokoan ja tuon heidät siihen paikkaan, jonka minä olen valinnut nimeni asuinsijaksi'.
\par 10 Ovathan he sinun palvelijasi ja sinun kansasi, jotka sinä vapahdit suurella voimallasi ja väkevällä kädelläsi.
\par 11 Oi Herra, tarkatkoon sinun korvasi palvelijasi rukousta ja niiden palvelijaisi rukousta, jotka tahtovat peljätä sinun nimeäsi! Anna tänä päivänä palvelijasi hankkeen menestyä ja suo hänen saada armo sen miehen edessä." Minä olin näet kuninkaan juomanlaskija.

\chapter{2}

\par 1 Niisan-kuussa, kuningas Artahsastan kahdentenakymmenentenä hallitusvuotena, kun viini oli hänen edessään, otin minä viinin ja annoin sen kuninkaalle. Kun minä en ollut ennen ollut hänen edessään murheellisena,
\par 2 sanoi kuningas minulle: "Minkätähden sinä olet niin murheellisen näköinen? Ethän ole sairas; sinulla on varmaan jokin sydämensuru." Silloin minä peljästyin kovin,
\par 3 mutta minä sanoin kuninkaalle: "Eläköön kuningas iankaikkisesti. Kuinka minä en olisi murheellisen näköinen, kun se kaupunki, jossa minun isieni haudat ovat, on autio ja sen portit tulella poltetut?"
\par 4 Kuningas sanoi minulle: "Mitä sinä siis pyydät?" Niin minä rukoilin taivaan Jumalaa
\par 5 ja sanoin kuninkaalle: "Jos kuningas hyväksi näkee ja jos olet mielistynyt palvelijaasi, niin lähetä minut Juudaan, siihen kaupunkiin, jossa minun isieni haudat ovat, rakentamaan sitä uudestaan".
\par 6 Niin kuningas kysyi minulta, kuningattaren istuessa hänen vieressään: "Kuinka kauan sinun matkasi kestäisi ja milloin voisit palata?" Kun siis kuningas näki hyväksi lähettää minut, ilmoitin minä hänelle määrätyn ajan.
\par 7 Ja minä sanoin kuninkaalle: "Jos kuningas hyväksi näkee, niin annettakoon minun mukaani kirjeet Eufrat-virran tuonpuoleisen maan käskynhaltijoille, että he sallivat minun kulkea sen kautta perille Juudaan asti,
\par 8 samoin kirje Aasafille, kuninkaan puiston vartijalle, että hän antaa minulle hirsiä temppelilinnan porttien kattamista varten sekä kaupungin muuria ja sitä taloa varten, johon menen asumaan". Ja kuningas myönsi minulle sen, koska minun Jumalani hyvä käsi oli minun päälläni.
\par 9 Kun minä sitten tulin Eufrat-virran tämänpuoleisen maan käskynhaltijain luo, annoin minä heille kuninkaan kirjeet. Ja kuningas oli lähettänyt minun kanssani sotapäälliköitä ja ratsumiehiä.
\par 10 Mutta kun hooronilainen Sanballat ja ammonilainen virkamies Tobia sen kuulivat, pahastuivat he kovin siitä, että oli tullut joku, joka harrasti israelilaisten parasta.
\par 11 Ja tultuani Jerusalemiin ja oltuani siellä kolme päivää
\par 12 minä nousin yöllä, ja muutamat miehet minun kanssani, ilmoittamatta kenellekään, mitä Jumalani oli pannut minun sydämeeni ja määrännyt tehtäväksi Jerusalemille. Eikä minulla ollut mukanani muuta juhtaakaan kuin se juhta, jolla minä ratsastin.
\par 13 Ja minä lähdin yöllä Laaksoportista Lohikäärmelähteelle päin ja Lantaportille ja tarkastelin Jerusalemin maahanrevittyjä muureja ja sen tulella poltettuja portteja.
\par 14 Ja minä kuljin eteenpäin Lähdeportille ja Kuninkaanlammikolle; mutta siellä ei ollut juhdalla, jonka selässä minä istuin, tilaa päästä eteenpäin.
\par 15 Silloin minä menin yöllä laaksoa myöten ylöspäin ja tarkastin muuria. Sitten minä palasin taas Laaksoportin kautta ja tulin takaisin.
\par 16 Mutta esimiehet eivät tietäneet, mihin minä olin mennyt ja mitä tein, sillä minä en ollut vielä ilmaissut mitään kenellekään, en juutalaisille, en papeille, ylimyksille, esimiehille enkä muille, joiden oli oltava mukana siinä työssä.
\par 17 Mutta nyt minä sanoin heille: "Te näette, missä kurjuudessa me olemme, kun Jerusalem on autio ja sen portit tulella poltetut. Tulkaa, rakentakaamme uudestaan Jerusalemin muuri, ettemme enää olisi häväistyksen alaiset."
\par 18 Ja minä kerroin heille, kuinka minun Jumalani hyvä käsi oli ollut minun päälläni ja myös mitä kuningas oli minulle sanonut. Niin he sanoivat: "Nouskaamme ja rakentakaamme". Ja he saivat rohkeuden käydä käsiksi tähän hyvään työhön.
\par 19 Mutta kun hooronilainen Sanballat ja ammonilainen virkamies Tobia ja arabialainen Gesem sen kuulivat, pilkkasivat he meitä, osoittivat meille halveksumistaan ja sanoivat: "Mitä te siinä teette? Kapinoitteko te kuningasta vastaan?"
\par 20 Silloin minä vastasin ja sanoin heille: "Taivaan Jumala on antava meille menestystä, ja me, hänen palvelijansa, nousemme ja rakennamme. Mutta teillä ei ole mitään osuutta eikä oikeutta Jerusalemiin, eikä teidän muistonne ole siellä pysyvä."

\chapter{3}

\par 1 Ylimmäinen pappi Eljasib ja hänen veljensä, papit, nousivat ja rakensivat Lammasportin, jonka he pyhittivät ja jonka ovet he asettivat paikoilleen, edelleen Hammea-torniin asti, jonka he pyhittivät, ja edelleen Hananelin-torniin asti.
\par 2 Heistä eteenpäin rakensivat Jerikon miehet; ja näistä eteenpäin rakensi Sakkur, Imrin poika.
\par 3 Ja Kalaportin rakensivat senaalaiset; he kattoivat sen ja asettivat paikoilleen sen ovet, teljet ja salvat.
\par 4 Heistä eteenpäin korjasi muuria Meremot, Uurian poika, joka oli Koosin poika; hänestä eteenpäin korjasi Mesullam, Berekjan poika, joka oli Mesesabelin poika; ja hänestä eteenpäin korjasi Saadok, Baanan poika.
\par 5 Hänestä eteenpäin korjasivat muuria tekoalaiset; mutta heidän ylhäisensä eivät notkistaneet niskaansa Herransa palvelukseen.
\par 6 Vanhan portin korjasivat Joojada, Paaseahin poika, ja Mesullam, Besodjan poika; he kattoivat sen ja asettivat paikoilleen sen ovet, teljet ja salvat.
\par 7 Heistä eteenpäin korjasivat muuria gibeonilainen Melatja ja meeronotilainen Jaadon sekä Gibeonin ja Mispan miehet, jotka olivat Eufrat-virran tämänpuoleisen käskynhaltijan vallan alaisia.
\par 8 Heistä eteenpäin korjasi muuria Ussiel, Harhajan poika, yksi kultasepistä, ja hänestä eteenpäin korjasi Hananja, yksi voiteensekoittajista; he panivat kuntoon Jerusalemia Leveään muuriin saakka.
\par 9 Heistä eteenpäin korjasi muuria Refaja, Huurin poika, Jerusalemin piirin toisen puolen päällikkö.
\par 10 Hänestä eteenpäin korjasi Jedaja, Harumafin poika, oman talonsa kohdalta; hänestä eteenpäin korjasi Hattus, Hasabnejan poika.
\par 11 Toisen osan korjasivat Malkia, Haarimin poika, ja Hassub, Pahat-Mooabin poika, sekä sen lisäksi Uunitornin.
\par 12 Heistä eteenpäin korjasi Sallum, Looheksen poika, Jerusalemin piirin toisen puolen päällikkö, hän ja hänen tyttärensä.
\par 13 Laaksoportin korjasivat Haanun ja Saanoahin asukkaat; he rakensivat sen ja asettivat paikoilleen sen ovet, teljet ja salvat. Sen lisäksi he korjasivat tuhat kyynärää muuria, Lantaporttiin saakka.
\par 14 Lantaportin korjasi Malkia, Reekabin poika, Beet-Keremin piirin päällikkö; hän rakensi sen ja asetti paikoilleen sen ovet, teljet ja salvat.
\par 15 Lähdeportin korjasi Sallum, Kolhoosen poika, Mispan piirin päällikkö; hän rakensi sen, teki siihen katon ja asetti paikoilleen sen ovet, teljet ja salvat. Sen lisäksi hän korjasi Vesijohtolammikon muurin, kuninkaan puutarhan luota, aina niihin portaisiin saakka, jotka laskeutuvat Daavidin kaupungista.
\par 16 Hänen jälkeensä korjasi Nehemia, Asbukin poika, Beet-Suurin piirin toisen puolen päällikkö, aina Daavidin hautojen kohdalle ja Tekolammikolle ja Urhojentalolle saakka.
\par 17 Hänen jälkeensä korjasivat muuria leeviläiset: Rehum, Baanin poika; hänestä eteenpäin korjasi Hasabja, Kegilan piirin toisen puolen päällikkö, piirinsä puolesta.
\par 18 Hänen jälkeensä korjasivat heidän veljensä: Bavvai, Heenadadin poika, joka oli Kegilan piirin toisen puolen päällikkö.
\par 19 Hänestä eteenpäin korjasi Eeser, Jeesuan poika, Mispan päällikkö, toisen osan, siltä kohdalta, mistä noustaan asehuoneeseen, joka on Kulmauksessa.
\par 20 Hänen jälkeensä korjasi Baaruk, Sabbain poika, suurella innolla toisen osan, Kulmauksesta aina ylimmäisen papin Eljasibin talon oveen saakka.
\par 21 Hänen jälkeensä korjasi Meremot, Uurian poika, joka oli Koosin poika, toisen osan, Eljasibin talon ovesta Eljasibin talon päähän.
\par 22 Hänen jälkeensä korjasivat papit, Lakeuden miehet.
\par 23 Heidän jälkeensä korjasivat Benjamin ja Hassub oman talonsa kohdalta; heidän jälkeensä korjasi Asarja, Maasejan poika, joka oli Ananjan poika, talonsa viereltä.
\par 24 Hänen jälkeensä korjasi Binnui, Heenadadin poika, toisen osan, Asarjan talosta aina Kulmaukseen ja kulmaan saakka.
\par 25 Paalal, Uusain poika, korjasi Kulmauksen ja Ylätornin kohdalta, joka ulkonee kuninkaan linnasta vankilan pihaan päin; hänen jälkeensä Pedaja, Paroksen poika
\par 26 - temppelipalvelijat asuivat Oofelilla - itäisen Vesiportin ja ulkonevan tornin kohdalta.
\par 27 Hänen jälkeensä korjasivat tekoalaiset toisen osan, suuren, ulkonevan tornin kohdalta aina Oofelin muuriin saakka.
\par 28 Hevosportin yläpuolelta korjasivat papit, kukin oman talonsa kohdalta.
\par 29 Heidän jälkeensä korjasi Saadok, Immerin poika, oman talonsa kohdalta. Ja hänen jälkeensä korjasi Semaja, Sekanjan poika, Itäportin vartija.
\par 30 Hänen jälkeensä korjasivat Hananja, Selemjan poika, ja Haanun, Saalafin kuudes poika, toisen osan; hänen jälkeensä korjasi Mesullam, Berekjan poika, yliskammionsa kohdalta.
\par 31 Hänen jälkeensä korjasi Malkia, yksi kultasepistä, temppelipalvelijain ja kauppiasten taloon asti, Vartiotornin kohdalta, ja aina Kulmasaliin saakka.
\par 32 Kultasepät ja kauppiaat korjasivat muurin Kulmasalin ja Lammasportin väliltä.

\chapter{4}

\par 1 Kun Sanballat kuuli meidän rakentavan muuria, vihastui hän ja närkästyi kovin. Ja hän pilkkasi juutalaisia
\par 2 ja puhui veljilleen ja Samarian sotaväelle ja sanoi: "Mitä nuo viheliäiset juutalaiset tekevät? Jätetäänkö heidät omiin valtoihinsa? Tulevatko he uhraamaan? Saavatko he työnsä kohta valmiiksi? Voivatko he tehdä eläviksi poltetut kivet tuhkaläjissä?"
\par 3 Ja ammonilainen Tobia seisoi hänen vieressään ja sanoi: "Rakentakoot vain! Kettukin saa heidän kivimuurinsa hajoamaan, jos hyppää sen päälle."
\par 4 Kuule, Jumalamme, kuinka halveksitut me olemme. Käännä heidän häpäisynsä heidän omaan päähänsä. Saata heidät ryöstetyiksi vankeuden maassa.
\par 5 Älä peitä heidän rikkomustansa, älköönkä heidän syntiänsä pyyhittäkö pois sinun kasvojesi edestä, koska he ovat vihoittaneet sinut kohtelemalla noin rakentajia.
\par 6 Mutta me rakensimme muuria; ja koko muuri tuli valmiiksi puoleen korkeuteensa asti, ja siitä kansa sai rohkeutta työhön.
\par 7 Mutta kun Sanballat, Tobia, arabialaiset, ammonilaiset ja asdodilaiset kuulivat, että Jerusalemin muurien korjaus edistyi, niin että niiden aukot alkoivat täyttyä, vihastuivat he kovin.
\par 8 Ja he kaikki tekivät liiton keskenään käydäksensä taisteluun Jerusalemia vastaan ja tehdäkseen siellä häiriötä.
\par 9 Niin me rukoilimme Jumalaamme ja asetimme vartijat heitä vastaan, suojelemaan itseämme heiltä päivin ja öin.
\par 10 Mutta Juuda sanoi: "Taakankantajain voima raukeaa, ja soraa on ylen paljon; me emme jaksa rakentaa muuria".
\par 11 Ja meidän vastustajamme sanoivat: "Ennenkuin he tietävät tai huomaavat, me tulemme heidän keskellensä ja tapamme heidät ja lopetamme työn".
\par 12 Mutta kun heidän läheisyydessään asuvat juutalaiset tulivat ja sanoivat meille joka taholta kymmenenkin kertaa: "Vetäytykää meidän luoksemme",
\par 13 silloin minä asetin kansan muurin taakse, alempiin ja suojattomiin paikkoihin, asetin heidät sukukunnittain asemiinsa miekkoineen, keihäineen ja jousineen.
\par 14 Ja katsastettuani kaiken minä nousin ja sanoin ylimyksille ja esimiehille ja muulle kansalle: "Älkää peljätkö heitä; muistakaa suurta ja peljättävää Herraa, ja taistelkaa veljienne, poikienne ja tyttärienne, vaimojenne ja kotienne puolesta".
\par 15 Sitten kuin vihollisemme olivat kuulleet, että me olimme saaneet tietää asian ja että Jumala oli tehnyt heidän hankkeensa tyhjäksi, palasimme me kaikki takaisin muurille, itsekukin työhönsä.
\par 16 Siitä päivästä alkaen teki toinen puoli minun palvelijoitani työtä, ja toinen puoli oli asestettuna keihäillä, kilvillä, jousilla ja rintahaarniskoilla; ja päämiehet seisoivat kaikkien niiden juutalaisten takana,
\par 17 jotka rakensivat muurilla. Nekin, jotka kantoivat taakkoja ja kuljettivat kuormia, tekivät toisella kädellänsä työtä ja toisessa pitivät keihästä.
\par 18 Ja rakentajilla oli kullakin miekka sidottuna vyöllensä heidän rakentaessaan; ja torvensoittaja seisoi minun vieressäni.
\par 19 Minä sanoin ylimyksille, esimiehille ja muulle kansalle: "Työ on suuri ja laaja, ja me olemme muurilla hajallamme, kaukana toisistamme.
\par 20 Missä kuulette torven soivan, sinne kokoontukaa meidän luoksemme. Meidän Jumalamme sotii meidän puolestamme."
\par 21 Näin me siis teimme työtä, ja toinen puoli väkeä oli asestettuna keihäillä päivänkoitosta siihen asti, kunnes tähdet tulivat näkyviin.
\par 22 Siihen aikaan minä myös sanoin kansalle: "Kukin jääköön palvelijoineen yöksi Jerusalemiin, että he olisivat meidän apunamme yöllä vartioimassa ja päivällä työssä".
\par 23 Emmekä me, en minä, eivät minun veljeni ja palvelijani eivätkä ne vartijat, jotka minua seurasivat, riisuneet vaatteitamme. Kullakin keihäs ja vettä.

\chapter{5}

\par 1 Mutta rahvas ja heidän vaimonsa nostivat suuren huudon juutalaisia veljiänsä vastaan.
\par 2 Muutamat sanoivat: "Meitä, meidän poikiamme ja tyttäriämme on paljon. Meidän täytyy saada viljaa, että meillä olisi mitä syödä pysyäksemme hengissä."
\par 3 Toiset sanoivat: "Peltomme, viinitarhamme ja talomme meidän täytyy pantata saadaksemme viljaa nälkäämme".
\par 4 Toiset sanoivat: "Meidän on täytynyt kuninkaan veroihin lainata rahaa peltojamme ja viinitarhojamme vastaan.
\par 5 Ovathan meidän ruumiimme veljiemme ruumiitten veroiset ja lapsemme heidän lastensa veroiset; ja katso, kuitenkin meidän täytyy antaa poikamme ja tyttäremme orjiksi, ja tyttäriämme on jo annettukin orjuuteen, emmekä me voi sille mitään, koska peltomme ja viinitarhamme ovat toisten käsissä."
\par 6 Kun minä kuulin heidän valituksensa ja nämä puheet, vihastuin minä kovin.
\par 7 Ja harkittuani mielessäni tätä asiaa minä nuhtelin ylimyksiä ja esimiehiä ja sanoin heille: "Tehän kiskotte korkoa, kukin veljeltänne". Sitten minä panin toimeen suuren kokouksen heitä vastaan.
\par 8 Ja minä sanoin heille: "Me olemme, sen mukaan kuin olemme voineet, ostaneet vapaiksi juutalaisia veljiämme, jotka oli myyty pakanoille. Tekö nyt myytte veljiänne, ja täytyykö heidän myydä itsensä meille?" He olivat vaiti eivätkä voineet vastata mitään.
\par 9 Ja minä sanoin: "Ette te siinä tee hyvin. Teidänhän tulisi vaeltaa meidän Jumalamme pelossa jo pakanain, meidän vihollistemme, häväistyksen tähden.
\par 10 Myöskin minä, minun veljeni ja palvelijani olemme lainanneet heille rahaa ja viljaa; luopukaamme tästä saatavasta.
\par 11 Luovuttakaa heille jo tänä päivänä heidän peltonsa, viinitarhansa, öljypuunsa ja talonsa, ja luopukaa rahan korosta sekä viljasta, viinistä ja öljystä, jonka olette heille lainanneet."
\par 12 He vastasivat: "Me luovutamme ne emmekä vaadi heiltä mitään; me teemme, niinkuin olet sanonut". Ja minä kutsuin papit ja vannotin heidät tekemään näin.
\par 13 Minä myös pudistin helmukseni ja sanoin: "Jokaisen, joka ei tätä sanaa täytä, pudistakoon Jumala näin pois hänen talostansa ja vaivannäkönsä hedelmistä; näin hän tulkoon pudistetuksi ja tyhjennetyksi". Ja koko seurakunta sanoi: "Amen", ja ylisti Herraa. Ja kansa teki, niinkuin oli sanottu.
\par 14 Myöskään en minä eivätkä veljeni siitä päivästä asti, jona minut määrättiin olemaan heidän käskynhaltijanaan Juudan maassa, siis kuningas Artahsastan kahdennestakymmenennestä hallitusvuodesta aina hänen kolmanteenkymmenenteen toiseen hallitusvuoteensa saakka, eli kahtenatoista vuotena, syöneet käskynhaltijalle tulevaa ruokaa.
\par 15 Sillä aikaisemmat käskynhaltijat, jotka olivat olleet ennen minua, olivat rasittaneet kansaa ja ottaneet siltä leipää ja viiniä sekä vielä neljäkymmentä hopeasekeliä. Myöskin heidän palvelijansa olivat sortaneet kansaa. Mutta minä en tehnyt niin, sillä minä pelkäsin Jumalaa.
\par 16 Myöskin kävin minä itse käsiksi tämän muurin tekoon. Ja me emme ostaneet mitään peltoa. Ja kaikki minun palvelijani olivat kokoontuneet sinne työhön.
\par 17 Myös sata viisikymmentä miestä, juutalaisia ja esimiehiä, söi minun pöydässäni sekä ne, jotka ympärillämme olevista pakanakansoista tulivat meidän luoksemme.
\par 18 Ja mitä päivittäin valmistettiin ruuaksi, nimittäin härkä ja kuusi valiolammasta sekä lintuja, se valmistettiin minun kustannuksellani; ja joka kymmenes päivä hankittiin kaikenlaisia viinejä viljalti. Mutta siitä huolimatta minä en vaatinut itselleni käskynhaltijalle tulevaa ruokaa, koska työ painoi raskaasti tätä kansaa.
\par 19 Muista, Jumalani, minun hyväkseni kaikki, mitä minä olen tehnyt tämän kansan puolesta.

\chapter{6}

\par 1 Kun Sanballat, Tobia, arabialainen Gesem ja meidän muut vihollisemme kuulivat, että minä olin rakentanut muurin ja ettei siinä enää ollut aukkoa - vaikken minä ollutkaan vielä siihen aikaan asettanut ovia paikoilleen portteihin -
\par 2 niin Sanballat ja Gesem lähettivät minulle sanan: "Tule, kohdatkaamme toisemme Kefirimissä, Oonon laaksossa". He näet ajattelivat tehdä minulle pahaa.
\par 3 Niin minä lähetin heidän luokseen sanansaattajat ja käskin sanoa: "Minulla on suuri työ tekeillä, niin etten voi tulla. Keskeytyisihän työ, jos minä jättäisin sen ja tulisin teidän luoksenne."
\par 4 Ja he lähettivät minulle saman sanan neljä kertaa, mutta minä vastasin heille aina samalla tavalla.
\par 5 Silloin Sanballat lähetti viidennen kerran palvelijansa minun luokseni samassa asiassa, ja tällä oli kädessään avoin kirje.
\par 6 Siihen oli kirjoitettu: "Kansojen kesken huhuillaan, ja myös Gasmu sanoo, että sinä ja juutalaiset suunnittelette kapinaa. Sentähden sinä rakennat muuria, ja itse sinä pyrit heidän kuninkaaksensa - niin kerrotaan.
\par 7 Myöskin olet hankkinut profeettoja julistamaan itsestäsi Jerusalemissa näin: 'Juudassa on kuningas'. Tämän nyt kuningas saa kuulla; tule siis ja neuvotelkaamme keskenämme."
\par 8 Silloin minä lähetin hänelle sanan: "Ei ole tapahtunut mitään semmoista, mistä puhut, vaan sinä olet keksinyt sen omasta päästäsi".
\par 9 He näet kaikki peloittelivat meitä arvellen: "Heidän kätensä herpoavat työssä, ja se jää tekemättä". Mutta vahvista sinä minun käteni.
\par 10 Minä menin Semajan, Delajan pojan, Mehetabelin pojanpojan, taloon, hänen ollessaan eristettynä. Hän sanoi: "Menkäämme yhdessä Jumalan temppeliin, temppelisaliin, ja sulkekaamme temppelisalin ovet. Sillä he tulevat tappamaan sinua; yöllä he tulevat ja tappavat sinut."
\par 11 Mutta minä vastasin: "Pakenisiko minunlaiseni mies? Tahi kuinka voisi minunlaiseni mies mennä temppeliin ja kuitenkin jäädä eloon? Minä en mene."
\par 12 Minä näet ymmärsin, ettei Jumala ollut häntä lähettänyt, vaan että hän oli lausunut minulle sen ennustuksen siitä syystä, että Tobia ja Sanballat olivat palkanneet hänet:
\par 13 hänet oli palkattu siinä tarkoituksessa, että minä peljästyisin ja menettelisin sillä tavalla ja niin tekisin syntiä; siitä he saisivat panettelun aiheen, häväistäkseen minua.
\par 14 Muista, Jumalani, Tobiaa ja Sanballatia näiden heidän tekojensa mukaan, niin myös naisprofeetta Nooadjaa ja muita profeettoja, jotka minua peloittelivat.
\par 15 Muuri valmistui elul-kuun kahdentenakymmenentenä viidentenä päivänä, viidenkymmenen kahden päivän kuluttua.
\par 16 Kun kaikki meidän vihollisemme sen kuulivat, ja kaikki ympärillämme asuvat kansat näkivät sen, havaitsivat he joutuneensa aivan alakynteen; sillä he tunsivat, että tämä työ oli suoritettu meidän Jumalamme avulla.
\par 17 Niinä päivinä meni myös Juudan ylimyksiltä lukuisia kirjeitä Tobialle, ja Tobialta tuli kirjeitä heille.
\par 18 Sillä Juudassa oli monta, jotka olivat valalla liittoutuneet hänen kanssansa; sillä hän oli Sekanjan, Aarahin pojan, vävy, ja hänen poikansa Joohanan oli ottanut vaimokseen Mesullamin, Berekjan pojan, tyttären.
\par 19 He myös puhuivat minulle hyvää hänestä ja veivät minun puheeni hänelle. Kirjeitäkin Tobia lähetti peloitellakseen minua.

\chapter{7}

\par 1 Kun muuri oli rakennettu, asetin minä ovet paikoillensa; ja niiden vartioimisen saivat ovenvartijat, veisaajat ja leeviläiset tehtäväkseen.
\par 2 Ja Jerusalemin päämiehiksi minä asetin veljeni Hananin ja linnanpäällikön Hananjan, sillä hän oli luotettava mies ja pelkäsi Jumalaa enemmän kuin moni muu.
\par 3 Ja minä sanoin heille: "Jerusalemin portteja älköön avattako, ennenkuin aurinko on polttavimmillaan; ja vartijain vielä seisoessa paikoillaan on ovet suljettava ja salvoilla teljettävä. Ja Jerusalemin asukkaita pantakoon vartioimaan, kukin vartiopaikallensa, kukin oman talonsa kohdalle.
\par 4 Kaupunki oli joka suuntaan tilava ja suuri, mutta väkeä siinä oli vähän, ja taloja oli vielä rakentamatta.
\par 5 Niin Jumala antoi minun sydämeeni, että minun oli koottava ylimykset, esimiehet ja kansa sukuluetteloon merkittäviksi. Silloin minä löysin niiden sukuluettelon, jotka ensin olivat tulleet sinne, ja huomasin siihen kirjoitetun:
\par 6 "Nämä ovat tämän maakunnan asukkaat, jotka lähtivät pakkosiirtolaisten vankeudesta, johon Nebukadnessar, Baabelin kuningas, oli heidät vienyt, ja palasivat Jerusalemiin ja Juudaan, kukin kaupunkiinsa,
\par 7 ne, jotka tulivat Serubbaabelin, Jeesuan, Nehemian, Asarjan, Raamian, Nahamanin, Mordokain, Bilsanin, Misperetin, Bigvain, Nehumin ja Baanan kanssa. Israelin kansan miesten lukumäärä oli:
\par 8 Paroksen jälkeläisiä kaksituhatta sata seitsemänkymmentä kaksi;
\par 9 Sefatjan jälkeläisiä kolmesataa seitsemänkymmentä kaksi;
\par 10 Aarahin jälkeläisiä kuusisataa viisikymmentä kaksi;
\par 11 Pahat-Mooabin jälkeläisiä, nimittäin Jeesuan ja Jooabin jälkeläisiä, kaksituhatta kahdeksansataa kahdeksantoista;
\par 12 Eelamin jälkeläisiä tuhat kaksisataa viisikymmentä neljä;
\par 13 Sattun jälkeläisiä kahdeksansataa neljäkymmentä viisi;
\par 14 Sakkain jälkeläisiä seitsemänsataa kuusikymmentä;
\par 15 Binnuin jälkeläisiä kuusisataa neljäkymmentä kahdeksan;
\par 16 Beebain jälkeläisiä kuusisataa kaksikymmentä kahdeksan;
\par 17 Asgadin jälkeläisiä kaksituhatta kolmesataa kaksikymmentä kaksi;
\par 18 Adonikamin jälkeläisiä kuusisataa kuusikymmentä seitsemän;
\par 19 Bigvain jälkeläisiä kaksituhatta kuusikymmentä seitsemän;
\par 20 Aadinin jälkeläisiä kuusisataa viisikymmentä viisi;
\par 21 Aaterin, nimittäin Hiskian, jälkeläisiä yhdeksänkymmentä kahdeksan;
\par 22 Haasumin jälkeläisiä kolmesataa kaksikymmentä kahdeksan;
\par 23 Beesain jälkeläisiä kolmesataa kaksikymmentäneljä;
\par 24 Haarifin jälkeläisiä sata kaksitoista;
\par 25 gibeonilaisia yhdeksänkymmentä viisi;
\par 26 Beetlehemin ja Netofan miehiä sata kahdeksankymmentä kahdeksan;
\par 27 Anatotin miehiä sata kaksikymmentä kahdeksan;
\par 28 Beet-Asmavetin miehiä neljäkymmentä kaksi;
\par 29 Kirjat-Jearimin, Kefiran ja Beerotin miehiä seitsemänsataa neljäkymmentä kolme;
\par 30 Raaman ja Geban miehiä kuusisataa kaksikymmentä yksi;
\par 31 Mikmaan miehiä sata kaksikymmentä kaksi;
\par 32 Beetelin ja Ain miehiä sata kaksikymmentä kolme;
\par 33 toisen Nebon miehiä viisikymmentä kaksi;
\par 34 toisen Eelamin jälkeläisiä tuhat kaksisataa viisikymmentä neljä;
\par 35 Haarimin jälkeläisiä kolmesataa kaksikymmentä;
\par 36 jerikolaisia kolmesataa neljäkymmentä viisi;
\par 37 loodilaisia, haadidilaisia ja oonolaisia seitsemänsataa kaksikymmentä yksi;
\par 38 senaalaisia kolmetuhatta yhdeksänsataa kolmekymmentä.
\par 39 Pappeja oli: Jedajan jälkeläisiä, nimittäin Jesuan sukua, yhdeksänsataa seitsemänkymmentä kolme;
\par 40 Immerin jälkeläisiä tuhat viisikymmentä kaksi;
\par 41 Pashurin jälkeläisiä tuhat kaksisataa neljäkymmentä seitsemän;
\par 42 Haarimin jälkeläisiä tuhat seitsemäntoista.
\par 43 Leeviläisiä oli: Jeesuan ja Kadmielin jälkeläisiä, nimittäin Hoodevan jälkeläisiä, seitsemänkymmentä neljä.
\par 44 Veisaajia oli: Aasafin jälkeläisiä sata neljäkymmentä kahdeksan.
\par 45 Ovenvartijoita oli: Sallumin jälkeläisiä, Aaterin jälkeläisiä, Talmonin jälkeläisiä, Akkubin jälkeläisiä, Hatitan jälkeläisiä, Soobain jälkeläisiä sata kolmekymmentä kahdeksan.
\par 46 Temppelipalvelijoita oli: Siihan jälkeläiset, Hasufan jälkeläiset, Tabbaotin jälkeläiset,
\par 47 Keeroksen jälkeläiset, Siian jälkeläiset, Paadonin jälkeläiset,
\par 48 Lebanan jälkeläiset, Hagaban jälkeläiset, Salmain jälkeläiset,
\par 49 Haananin jälkeläiset, Giddelin jälkeläiset, Gaharin jälkeläiset,
\par 50 Reajan jälkeläiset, Resinin jälkeläiset, Nekodan jälkeläiset,
\par 51 Gassamin jälkeläiset, Ussan jälkeläiset, Paaseahin jälkeläiset,
\par 52 Beesain jälkeläiset, Meunimin jälkeläiset, Nefusesimin jälkeläiset,
\par 53 Bakbukin jälkeläiset, Hakufan jälkeläiset, Harhurin jälkeläiset,
\par 54 Baslutin jälkeläiset, Mehidan jälkeläiset, Harsan jälkeläiset,
\par 55 Barkoksen jälkeläiset, Siiseran jälkeläiset, Taamahin jälkeläiset,
\par 56 Nesiahin jälkeläiset, Hatifan jälkeläiset.
\par 57 Salomon palvelijain jälkeläisiä oli: Sootain jälkeläiset, Sooferetin jälkeläiset, Peridan jälkeläiset,
\par 58 Jaalan jälkeläiset, Darkonin jälkeläiset, Giddelin jälkeläiset,
\par 59 Sefatjan jälkeläiset, Hattilin jälkeläiset, Kooferet-Sebaimin jälkeläiset, Aamonin jälkeläiset.
\par 60 Temppelipalvelijoita ja Salomon palvelijain jälkeläisiä oli kaikkiaan kolmesataa yhdeksänkymmentä kaksi.
\par 61 Nämä ovat ne, jotka lähtivät Teel-Melahista, Teel-Harsasta, Kerub-Addonista ja Immeristä, voimatta ilmoittaa perhekuntaansa ja syntyperäänsä, olivatko israelilaisia:
\par 62 Delajan jälkeläiset, Tobian jälkeläiset, Nekodan jälkeläiset, kuusisataa neljäkymmentä kaksi.
\par 63 Ja pappeja: Habaijan jälkeläiset, Koosin jälkeläiset, Barsillain jälkeläiset, sen, joka oli ottanut itsellensä vaimon gileadilaisen Barsillain tyttäristä ja jota kutsuttiin heidän nimellään.
\par 64 Nämä etsivät sukuluetteloitaan, niitä löytämättä, ja niin heidät julistettiin pappeuteen kelpaamattomiksi.
\par 65 Maaherra kielsi heitä syömästä korkeasti-pyhää, ennenkuin nousisi pappi, joka voi käyttää uurimia ja tummimia.
\par 66 Koko seurakunta yhteenlaskettuna oli neljäkymmentäkaksi tuhatta kolmesataa kuusikymmentä,
\par 67 paitsi heidän palvelijoitansa ja palvelijattariansa, joita oli seitsemäntuhatta kolmesataa kolmekymmentä seitsemän. Lisäksi oli heillä kaksisataa neljäkymmentä viisi mies- ja naisveisaajaa.
\par 68 Kameleja heillä oli neljäsataa kolmekymmentä viisi, aaseja kuusituhatta seitsemänsataa kaksikymmentä.
\par 69 Osa perhekunta-päämiehistä antoi lahjoja rakennustyötä varten. Maaherra antoi rahastoon tuhat dareikkia kultaa, viisikymmentä maljaa ja viisisataa kolmekymmentä papin-ihokasta.
\par 70 Ja muutamat perhekunta-päämiehistä antoivat rakennusrahastoon kaksikymmentä tuhatta dareikkia kultaa ja kaksituhatta kaksisataa miinaa hopeata.
\par 71 Ja muu kansa antoi yhteensä kaksikymmentä tuhatta dareikkia kultaa ja kaksituhatta miinaa hopeata sekä kuusikymmentä seitsemän papin-ihokasta.
\par 72 Ja papit, leeviläiset, ovenvartijat, veisaajat ja osa kansasta sekä temppelipalvelijat, koko Israel, asettuivat kaupunkeihinsa. Ja niin tuli seitsemäs kuukausi, ja israelilaiset olivat jo kaupungeissansa.

\chapter{8}

\par 1 Silloin kokoontui kaikki kansa yhtenä miehenä Vesiportin edustalla olevalle aukealle; ja he pyysivät Esraa, kirjanoppinutta, tuomaan Mooseksen lain kirjan, jonka lain Herra oli antanut Israelille.
\par 2 Niin pappi Esra toi lain seurakunnan eteen, sekä miesten että naisten, kaikkien, jotka voivat ymmärtää, mitä kuulivat. Tämä tapahtui seitsemännen kuun ensimmäisenä päivänä.
\par 3 Ja hän luki sitä Vesiportin edustalla olevalla aukealla päivän koitosta puolipäivään saakka miehille ja naisille, niille, jotka voivat sitä ymmärtää, kaiken kansan kuunnellessa lain kirjan lukemista.
\par 4 Ja Esra, kirjanoppinut, seisoi korkealla puulavalla, joka oli tätä varten tehty. Ja hänen vieressään seisoivat: hänen oikealla puolellaan Mattitja, Sema, Anaja, Uuria, Hilkia ja Maaseja; ja hänen vasemmalla puolellaan Pedaja, Miisael, Malkia, Haasum, Hasbaddana, Sakarja ja Mesullam.
\par 5 Ja Esra avasi kirjan kaiken kansan nähden, sillä hän seisoi ylempänä kaikkea kansaa; ja kun hän avasi sen, nousi kaikki kansa seisomaan.
\par 6 Ja Esra kiitti Herraa, suurta Jumalaa, ja kaikki kansa vastasi, kohottaen kätensä ylös: "Amen, amen"; ja he kumarsivat ja rukoilivat Herraa, heittäytyneinä kasvoilleen maahan.
\par 7 Sitten Jeesua, Baani, Seerebja, Jaamin, Akkub, Sabbetai, Hoodia, Maaseja, Kelita, Asarja, Joosabad, Haanan, Pelaja ja muut leeviläiset opettivat kansalle lakia, kansan seisoessa alallansa.
\par 8 Ja he lukivat Jumalan lain kirjaa kappale kappaleelta ja selittivät sen sisällyksen, niin että luettu ymmärrettiin.
\par 9 Ja Nehemia, maaherra, ja pappi Esra, kirjanoppinut, ja leeviläiset, jotka opettivat kansaa, sanoivat kaikelle kansalle: "Tämä päivä on pyhitetty Herralle, teidän Jumalallenne, älkää murehtiko älkääkä itkekö". Sillä kaikki kansa itki, kun he kuulivat lain sanat.
\par 10 Ja hän sanoi vielä heille: "Menkää ja syökää rasvaisia ruokia ja juokaa makeita juomia ja lähettäkää maistiaisia niille, joilla ei ole mitään valmistettuna, sillä tämä päivä on pyhitetty meidän Herrallemme. Ja älkää olko murheelliset, sillä ilo Herrassa on teidän väkevyytenne."
\par 11 Myöskin leeviläiset rauhoittivat kaikkea kansaa ja sanoivat: "Olkaa hiljaa, sillä tämä päivä on pyhä; älkää olko murheelliset".
\par 12 Ja kaikki kansa meni, söi ja joi, lähetti maistiaisia ja vietti suurta ilojuhlaa; sillä he olivat ymmärtäneet, mitä heille oli julistettu.
\par 13 Seuraavana päivänä kokoontuivat kaiken kansan perhekunta-päämiehet, papit ja leeviläiset Esran, kirjanoppineen, tykö painamaan lain sanoja mieleensä.
\par 14 Niin he huomasivat lakiin kirjoitetun, että Herra oli Mooseksen kautta käskenyt israelilaisia asumaan lehtimajoissa juhlan aikana seitsemännessä kuussa,
\par 15 ja että kaikissa heidän kaupungeissaan ja Jerusalemissa oli julistettava ja kuulutettava näin: "Menkää vuorille ja tuokaa öljypuun lehviä tai metsäöljypuun lehviä sekä myrtin, palmupuun ja muiden tuuheiden puiden lehviä, ja tehkää lehtimajoja, niinkuin on säädetty".
\par 16 Ja kansa meni ja toi niitä ja teki itselleen lehtimajoja kukin katollensa ja pihoihinsa ja Jumalan temppelin esipihoihin sekä Vesiportin aukealle ja Efraimin portin aukealle.
\par 17 Ja koko seurakunta, kaikki vankeudesta palanneet, tekivät lehtimajoja ja asuivat lehtimajoissa. Sillä aina Joosuan, Nuunin pojan, ajoista siihen päivään saakka eivät israelilaiset olleet niin tehneet. Ja vallitsi hyvin suuri ilo.
\par 18 Ja Jumalan lain kirjaa luettiin joka päivä, ensimmäisestä päivästä viimeiseen saakka. Ja he viettivät juhlaa seitsemän päivää, ja kahdeksantena päivänä pidettiin juhlakokous säädetyllä tavalla.

\chapter{9}

\par 1 Mutta saman kuun kahdentenakymmenentenä neljäntenä päivänä israelilaiset kokoontuivat paastoten, säkit yllä ja multaa pään päällä.
\par 2 Ja Israelin heimo eristäytyi kaikista muukalaisista, astui esiin ja tunnusti syntinsä ja isiensä rikkomukset.
\par 3 Sitten he nousivat seisomaan, kukin paikallansa, ja heille luettiin Herran, heidän Jumalansa, lain kirjaa neljännes päivää; ja toisen neljänneksen aikana he tunnustivat syntinsä ja kumartaen rukoilivat Herraa, Jumalaansa.
\par 4 Ja Jeesua, Baani, Kadmiel, Sebanja, Bunni, Seerebja, Baani ja Kenani nousivat leeviläisten korokkeelle ja huusivat suurella äänellä Herraa, Jumalaansa;
\par 5 ja leeviläiset Jeesua, Kadmiel, Baani, Hasabneja, Seerebja, Hoodia, Sebanja ja Petahja sanoivat: "Nouskaa ja kiittäkää Herraa, Jumalaanne, iankaikkisesta iankaikkiseen. Ja kiitettäköön sinun kunniallista nimeäsi, joka on korotettu yli kaiken kiitoksen ja ylistyksen.
\par 6 Sinä yksin olet Herra. Sinä olet tehnyt taivaat ja taivasten taivaat kaikkine joukkoinensa, maan ja kaikki, mitä siinä on, meret ja kaikki, mitä niissä on. Sinä annat elämän niille kaikille, ja taivaan joukot kumartavat sinua.
\par 7 Sinä olet Herra Jumala, joka valitsit Abramin ja veit hänet pois Kaldean Uurista ja annoit hänelle nimen Aabraham.
\par 8 Ja sinä havaitsit hänen sydämensä uskolliseksi sinua kohtaan, ja niin sinä teit hänen kanssaan liiton antaaksesi hänen jälkeläisillensä kanaanilaisten, heettiläisten, amorilaisten, perissiläisten, jebusilaisten ja girgasilaisten maan. Ja lupauksesi sinä olet täyttänyt, sillä sinä olet vanhurskas.
\par 9 Ja sinä näit meidän isiemme kurjuuden Egyptissä ja kuulit heidän huutonsa Kaislameren rannalla.
\par 10 Sinä teit tunnustekoja ja ihmeitä rangaisten faraota ja kaikkia hänen palvelijoitansa ja kaikkea hänen maansa kansaa; sillä sinä tiesit näiden kohdelleen heitä ylimielisesti, ja sinä teit itsellesi nimen, niinkuin se tänäkin päivänä on.
\par 11 Meren sinä halkaisit heidän edessänsä kahtia, niin että he kulkivat meren poikki kuivaa myöten; mutta heidän takaa-ajajansa sinä syöksit syvyyteen, niinkuin kiven valtaviin vesiin.
\par 12 Sinä johdatit heitä päivällä pilvenpatsaasta ja yöllä tulenpatsaasta, valaisten heille tien, jota heidän oli kuljettava.
\par 13 Ja sinä astuit alas Siinain vuorelle ja puhuit heille taivaasta ja annoit heille oikeat tuomiot ja totiset lait, hyvät säädökset ja käskyt.
\par 14 Sinä ilmoitit heille pyhän sapattisi ja annoit heille käskyt, säädökset ja lain palvelijasi Mooseksen kautta.
\par 15 Sinä annoit heille leipää taivaasta heidän nälkäänsä, ja sinä hankit heille vettä kalliosta heidän janoonsa. Ja sinä käskit heidän mennä ottamaan omaksensa maan, jonka sinä olit kättä kohottaen luvannut heille antaa.
\par 16 Mutta he, meidän isämme, olivat ylimielisiä; he olivat niskureita eivätkä totelleet sinun käskyjäsi.
\par 17 He eivät tahtoneet totella eivätkä muistaneet ihmeellisiä tekoja, jotka sinä olit heille tehnyt, vaan olivat niskureita ja valitsivat uppiniskaisuudessaan johtajan palatakseen takaisin orjuuteensa. Mutta sinä olet anteeksiantava Jumala, armahtavainen ja laupias, pitkämielinen ja suuri armossa: sinä et heitä hyljännyt.
\par 18 Vaikka he tekivät itsellensä valetun vasikankuvan ja sanoivat: 'Tämä on sinun jumalasi, joka on johdattanut sinut Egyptistä', ja vaikka he paljon pilkkasivat Jumalaa,
\par 19 niin sinä suuressa laupeudessasi et kuitenkaan hyljännyt heitä erämaassa. Pilvenpatsas ei väistynyt heidän luotansa päivällä, johtamasta heitä tiellä, eikä tulenpatsas yöllä, valaisemasta heille tietä, jota heidän oli kuljettava.
\par 20 Hyvän Henkesi sinä annoit heitä opettamaan, mannaasi et kieltänyt heidän suustansa, ja vettä sinä annoit heille heidän janoonsa.
\par 21 Neljäkymmentä vuotta sinä elätit heitä erämaassa, niin ettei heiltä mitään puuttunut, eivät heidän vaatteensa kuluneet, eivätkä heidän jalkansa ajettuneet.
\par 22 Sinä annoit heidän haltuunsa valtakuntia ja kansoja ja jaoit ne alue alueelta; ja he valloittivat Siihonin maan - Hesbonin kuninkaan maan - ja Oogin, Baasanin kuninkaan, maan.
\par 23 Ja heidän lastensa luvun sinä teit paljoksi kuin taivaan tähdet, ja sinä veit heidät siihen maahan, josta olit antanut heidän isillensä lupauksen, että he saavat mennä ottamaan sen omaksensa.
\par 24 Ja lapset tulivat ja ottivat sen maan omaksensa, ja sinä nöyryytit heidän edessään maan asukkaat, kanaanilaiset, ja annoit nämä heidän käsiinsä, sekä heidän kuninkaansa että sen maan kansat, niin että he tekivät näille, mitä tahtoivat.
\par 25 Ja he valloittivat varustetut kaupungit ja lihavan maan ja ottivat omikseen talot, jotka olivat täynnä kaikkea hyvää, kallioon hakatut vesisäiliöt, viinitarhat, öljypuut ja hedelmäpuita suuret määrät. He söivät ja tulivat ravituiksi ja lihaviksi ja pitivät hyviä päiviä sinun antimiesi runsaudessa.
\par 26 Mutta he niskoittelivat ja kapinoivat sinua vastaan, heittivät sinun lakisi selkänsä taa ja tappoivat sinun profeettasi, jotka heitä varoittivat palauttaakseen heidät sinun luoksesi; ja he pilkkasivat paljon Jumalaa.
\par 27 Sentähden sinä annoit heidät heidän ahdistajainsa käsiin, ja nämä ahdistivat heitä. Mutta kun he ahdinkonsa aikana huusivat sinua, niin sinä taivaasta kuulit ja suuressa laupeudessasi annoit heille vapauttajia, jotka vapauttivat heidät heidän ahdistajainsa käsistä.
\par 28 Mutta rauhaan päästyään he jälleen tekivät sitä, mikä on pahaa sinun edessäsi. Silloin sinä jätit heidät heidän vihollistensa käsiin, niin että nämä vallitsivat heitä. Mutta kun he jälleen huusivat sinua, kuulit sinä taivaasta ja pelastit heidät laupeudessasi monta kertaa.
\par 29 Ja sinä varoitit heitä palauttaaksesi heidät seuraamaan sinun lakiasi, mutta he olivat ylimielisiä eivätkä totelleet sinun käskyjäsi, vaan he rikkoivat sinun oikeutesi - se ihminen, joka ne pitää, on niistä elävä - mutta he käänsivät uppiniskaisina selkänsä ja olivat niskureita eivätkä totelleet.
\par 30 Sinä kärsit heitä monta vuotta ja varoitit heitä Hengelläsi profeettaisi kautta, mutta he eivät ottaneet sitä korviinsa. Niin sinä annoit heidät pakanallisten kansojen käsiin.
\par 31 Mutta sinä suuressa laupeudessasi et tehnyt loppua heistä etkä hyljännyt heitä; sillä sinä olet armahtavainen ja laupias Jumala.
\par 32 Ja nyt, Jumalamme, sinä suuri, väkevä ja peljättävä Jumala, joka pidät liiton ja säilytät laupeuden: älä katso vähäksi kaikkea sitä vaivaa, joka on kohdannut meitä, meidän kuninkaitamme, päämiehiämme, pappejamme, profeettojamme, isiämme ja koko sinun kansaasi, Assurin kuningasten ajoista aina tähän päivään saakka.
\par 33 Sinä olet vanhurskas kaikessa, mikä on meitä kohdannut; sillä sinä olet ollut uskollinen teoissasi, mutta me olemme olleet jumalattomat.
\par 34 Meidän kuninkaamme, päämiehemme, pappimme ja isämme eivät ole seuranneet sinun lakiasi eivätkä tarkanneet sinun käskyjäsi ja säädöksiäsi, jotka sinä olit heille antanut.
\par 35 Ja vaikka he elivät omassa valtakunnassansa ja siinä antimien runsaudessa, jonka sinä olit heille antanut, ja avarassa ja lihavassa maassa, jonka sinä olit heille antanut, eivät he kuitenkaan palvelleet sinua eivätkä kääntyneet pois pahoista teoistansa.
\par 36 Katso, me olemme nyt orjia; siinä maassa, jonka sinä meidän isillemme annoit, että he söisivät sen hedelmiä ja antimia, siinä meidän täytyy orjina olla.
\par 37 Sen runsas sato tulee niille kuninkaille, jotka sinä meidän syntiemme tähden olet pannut meitä hallitsemaan; he vallitsevat meitä ja meidän karjaamme, niinkuin tahtovat, ja me olemme suuressa ahdistuksessa."
\par 38 "Kaiken tämän johdosta me teemme sitoumuksen ja kirjoitamme siihen nimemme; sinetöidyssä asiakirjassa ovat meidän päämiestemme, leeviläistemme ja pappiemme nimet.

\chapter{10}

\par 1 Sinetöidyissä asiakirjoissa ovat nämä nimet: Nehemia, Hakaljan poika, maaherra, ja Sidkia,
\par 2 Seraja, Asarja, Jeremia,
\par 3 Pashur, Amarja, Malkia,
\par 4 Hattus, Sebanja, Malluk,
\par 5 Haarim, Meremot, Obadja,
\par 6 Daniel, Ginneton, Baaruk,
\par 7 Mesullam, Abia, Miijamin,
\par 8 Maasja, Bilgai ja Semaja - nämä ovat pappeja.
\par 9 Leeviläisiä ovat: Jeesua, Asanjan poika, Binnui, Heenadadin jälkeläinen, Kadmiel
\par 10 ja heidän veljensä Sebanja, Hoodia, Kelita, Pelaja, Haanan,
\par 11 Miika, Rehob, Hasabja,
\par 12 Sakkur, Seerebja, Sebanja,
\par 13 Hoodia, Baani ja Beninu.
\par 14 Kansan päämiehiä ovat: Paros, Pahat-Mooab, Eelam, Sattu, Baani,
\par 15 Bunni, Asgad, Beebai,
\par 16 Adonia, Bigvai, Aadin,
\par 17 Aater, Hiskia, Assur,
\par 18 Hoodia, Haasum, Beesai,
\par 19 Haarif, Anatot, Nuubai,
\par 20 Magpias, Mesullam, Heesir,
\par 21 Mesesabel, Saadok, Jaddua,
\par 22 Pelatja, Haanan, Anaja,
\par 23 Hoosea, Hananja, Hassub,
\par 24 Loohes, Pilha, Soobek,
\par 25 Rehum, Hasabna, Maaseja,
\par 26 Ahia, Haanan, Aanan,
\par 27 Malluk, Haarim ja Baana.
\par 28 Ja muu kansa, papit, leeviläiset, ovenvartijat, veisaajat, temppelipalvelijat ja kaikki, jotka ovat eristäytyneet pakanallisista kansoista Jumalan lain puolelle, sekä heidän vaimonsa, poikansa ja tyttärensä, kaikki, jotka pystyvät sen ymmärtämään,
\par 29 liittyvät ylhäisiin veljiinsä, tulevat valalle ja vannovat vaeltavansa Jumalan lain mukaan, joka on annettu Jumalan palvelijan Mooseksen kautta, ja noudattavansa ja seuraavansa kaikkia Herran, meidän Herramme, käskyjä, oikeuksia ja säädöksiä.
\par 30 Me emme anna tyttäriämme maan kansoille emmekä ota heidän tyttäriänsä pojillemme vaimoiksi.
\par 31 Me emme osta sapattina tai pyhäpäivänä maan kansoilta, jos ne tuovat kauppatavaraa tai viljaa mitä tahansa kaupaksi sapattina. Me jätämme joka seitsemäntenä vuotena maan lepäämään ja kaikki saatavat velkomatta.
\par 32 Me sitoudumme suorittamaan vuodessa kolmannes-sekelin palvelusta varten Jumalamme temppelissä:
\par 33 näkyleipiin, jokapäiväiseen ruokauhriin ja jokapäiväiseen polttouhriin, uhreihin sapatteina ja uudenkuun päivinä, juhlauhreihin ja pyhiin lahjoihin, syntiuhreihin sovituksen toimittamiseksi Israelille ja kaikkeen, mikä meidän Jumalamme temppelissä toimitettava on.
\par 34 Me, papit, leeviläiset ja kansa, olemme heittäneet arpaa uhrilahjahalkojen tuomisesta perhekunnittain Jumalamme temppeliin määräaikoina joka vuosi, poltettaviksi Herran, meidän Jumalamme, alttarilla, niinkuin laissa on kirjoitettuna.
\par 35 Me sitoudumme tuomaan maamme uutiset ja kaikkinaisten hedelmäpuiden uutiset joka vuosi Herran temppeliin
\par 36 sekä esikoiset pojistamme ja karjastamme, niinkuin laissa on kirjoitettuna, ja tuomaan raavaittemme ja lampaittemme esikoiset Jumalamme temppeliin, papeille, jotka toimittavat virkaansa meidän Jumalamme temppelissä.
\par 37 Me tuomme parhaat jyvärouheemme ja antimemme, parhaat kaikkinaisten puiden hedelmät, parhaan viinin ja öljyn papeille, Jumalamme temppelin kammioihin, ja maamme kymmenykset leeviläisille. Leeviläiset itse kantavat kymmenykset kaikista kaupungeista, missä meillä on maanviljelystä.
\par 38 Ja papin, Aaronin pojan, tulee olla leeviläisten kanssa, heidän kantaessaan kymmenyksiä, ja leeviläisten tulee viedä kymmenykset kymmenyksistä meidän Jumalamme temppeliin, varastohuoneen kammioihin.
\par 39 Sillä israelilaisten ja leeviläisten on vietävä anti jyvistä, viinistä ja öljystä näihin kammioihin, joissa pyhäkön kalut ja virkaansa toimittavat papit, ovenvartijat ja veisaajat ovat. Me emme laiminlyö Jumalamme temppeliä."

\chapter{11}

\par 1 Ja kansan päämiehet asettuivat Jerusalemiin, mutta muu kansa heitti arpaa saadakseen joka kymmenennen asettumaan Jerusalemiin, pyhään kaupunkiin, yhdeksän kymmenesosan jäädessä asumaan muihin kaupunkeihin.
\par 2 Ja kansa siunasi kaikkia niitä miehiä, jotka vapaaehtoisesti asettuivat Jerusalemiin.
\par 3 Nämä olivat ne maakunnan päämiehet, jotka asettuivat Jerusalemiin ja Juudan kaupunkeihin; he asuivat kukin perintöosallaan, kaupungeissaan, Israel, papit, leeviläiset ja temppelipalvelijat sekä Salomon palvelijain jälkeläiset.
\par 4 Jerusalemissa asuivat seuraavat Juudan ja Benjamin miehet: Juudan miehiä: Ataja, Ussian poika, joka oli Sakarjan poika, joka Amarjan poika, joka Sefatjan poika, joka Mahalalelin poika, Pereksen jälkeläisiä,
\par 5 ja Maaseja, Baarukin poika, joka oli Kolhoosen poika, joka Hasajan poika, joka Adajan poika, joka Joojaribin poika, joka Sakarjan poika, joka siilonilaisen poika;
\par 6 Jerusalemissa asuvia Pereksen jälkeläisiä oli kaikkiaan neljäsataa kuusikymmentä kahdeksan sotakuntoista miestä.
\par 7 Benjaminilaiset olivat nämä: Sallu, Mesullamin poika, joka oli Jooedin poika, joka Pedajan poika, joka Koolajan poika, joka Maasejan poika, joka Iitielin poika, joka Jesajan poika,
\par 8 ja hänen jälkeensä Gabbai ja Sallai, yhdeksänsataa kaksikymmentä kahdeksan.
\par 9 Jooel, Sikrin poika, oli heidän päällysmiehenään ja Juuda, Senuan poika, toisena kaupunginpäällikkönä.
\par 10 Pappeja: Jedaja, Joojaribin poika, Jaakin,
\par 11 Seraja, Hilkian poika, joka oli Mesullamin poika, joka Saadokin poika, joka Merajotin poika, joka Ahitubin poika, Jumalan temppelin esimies,
\par 12 sekä heidän veljensä, jotka toimittivat palvelusta temppelissä, kahdeksansataa kaksikymmentä kaksi miestä; ja Adaja, Jerohamin poika, joka oli Pelaljan poika, joka Amsin poika, joka Sakarjan poika, joka Pashurin poika, joka Malkian poika,
\par 13 sekä hänen veljensä, jotka olivat perhekunta-päämiehiä, kaksisataa neljäkymmentä kaksi miestä; ja Amassai, Asarelin poika, joka oli Ahsain poika, joka Mesillemotin poika, joka Immerin poika,
\par 14 sekä heidän veljensä, jotka olivat sotaurhoja, sata kaksikymmentä kahdeksan miestä. Heidän päällysmiehenään oli Sabdiel, Gedolimin poika.
\par 15 Ja leeviläisiä: Semaja, Hassubin poika, joka oli Asrikamin poika, joka Hasabjan poika, joka Bunnin poika;
\par 16 ja Sabbetai ja Joosabad, jotka valvoivat maallisia toimia Jumalan temppelissä ja olivat leeviläisten päämiehiä,
\par 17 ja Mattanja, Miikan poika, joka oli Sabdin poika, joka Aasafin, ensimmäisen johtajan, poika, joka rukoiltaessa alotti kiitosvirren, ja Bakbukja, hänen veljistään toinen, ja Abda, Sammuan poika, joka oli Gaalalin poika, joka Jedutunin poika.
\par 18 Leeviläisiä oli pyhässä kaupungissa kaikkiaan kaksisataa kahdeksankymmentä neljä.
\par 19 Ovenvartijat olivat Akkub, Talmon sekä heidän veljensä, jotka vartioivat portteja, sata seitsemänkymmentä kaksi.
\par 20 Muut israelilaiset, papit ja leeviläiset asuivat kaikissa muissa Juudan kaupungeissa, kukin perintöosallaan.
\par 21 Temppelipalvelijat asuivat Oofelilla; Siiha ja Gispa valvoivat temppelipalvelijoita.
\par 22 Ja leeviläisten päällysmiehenä Jerusalemissa Jumalan temppelin toimissa oli Ussi, Baanin poika, joka oli Hasabjan poika, joka Mattanjan poika, joka Miikan poika, Aasafin jälkeläisiä, veisaajia.
\par 23 Oli näet heitä koskeva kuninkaan käsky, joka vakuutti veisaajille heidän jokapäiväiset tarpeensa.
\par 24 Ja Petahja, Mesesabelin poika, Serahin, Juudan pojan, jälkeläisiä, oli kuninkaan edusmiehenä kaikissa kansaa koskevissa asioissa.
\par 25 Heidän peltomaillaan olevissa kylissä asui Juudan miehiä: Kirjat-Arbassa ja sen tytärkaupungeissa, Diibonissa ja sen tytärkaupungeissa, Jekabseelissa ja siihen kuuluvissa kylissä,
\par 26 Jeesuassa, Mooladassa, Beet-Peletissä,
\par 27 Hasar-Suualissa, Beersebassa ja sen tytärkaupungeissa,
\par 28 Siklagissa, Mekonassa ja sen tytärkaupungeissa,
\par 29 Een-Rimmonissa, Sorassa, Jarmutissa,
\par 30 Saanoahissa, Adullamissa ja niihin kuuluvissa kylissä, Laakiissa ja sen peltomailla, Asekassa ja sen tytärkaupungeissa; he sijoittuivat siis asumaan Beersebasta aina Hinnomin laaksoon saakka.
\par 31 Benjaminilaiset asuivat, Gebasta alkaen, Mikmaassa, Aijassa, Beetelissä ja sen tytärkaupungeissa,
\par 32 Anatotissa, Noobissa, Ananjassa,
\par 33 Haasorissa, Raamassa, Gittaimissa,
\par 34 Haadidissa, Seboimissa, Neballatissa,
\par 35 Loodissa, Oonossa, Seppäinlaaksossa.
\par 36 Leeviläisistä asui eräitä Juudan osastoja Benjaminissa.

\chapter{12}

\par 1 Nämä olivat ne papit ja leeviläiset, jotka lähtivät Serubbaabelin, Sealtielin pojan, ja Jeesuan kanssa: Seraja, Jeremia, Esra,
\par 2 Amarja, Malluk, Hattus,
\par 3 Sekanja, Rehum, Meremot,
\par 4 Iddo, Ginnetoi, Abia,
\par 5 Miijamin, Maadja, Bilga,
\par 6 Semaja, Joojarib, Jedaja,
\par 7 Sallu, Aamok, Hilkia ja Jedaja. Nämä olivat pappien ja veljiensä päämiehet Jeesuan aikana.
\par 8 Ja leeviläiset olivat: Jeesua, Binnui, Kadmiel, Seerebja, Juuda ja Mattanja, joka yhdessä veljiensä kanssa johti kiitosveisua;
\par 9 ja heidän veljensä Bakbukja ja Unni seisoivat vastapäätä heitä tehtäviään suorittamassa.
\par 10 Jeesualle syntyi Joojakim, Joojakimille syntyi Eljasib, Eljasibille syntyi Joojada,
\par 11 Joojadalle syntyi Joonatan ja Joonatanille syntyi Jaddua.
\par 12 Joojakimin aikana olivat seuraavat papit perhekunta-päämiehinä: Serajalla Meraja, Jeremialla Hananja,
\par 13 Esralla Mesullam, Amarjalla Joohanan,
\par 14 Mallukilla Joonatan, Sebanjalla Joosef,
\par 15 Haarimilla Adna, Merajotilla Helkai,
\par 16 Iddolla Sakarja, Ginnetonilla Mesullam,
\par 17 Abialla Sikri, Minjaminilla, Mooadjalla Piltai,
\par 18 Bilgalla Sammua, Semajalla Joonatan,
\par 19 Joojaribilla Mattenai, Jedajalla Ussi,
\par 20 Sallailla Kallai, Aamokilla Eeber,
\par 21 Hilkialla Hasabja ja Jedajalla Netanel.
\par 22 Leeviläisistä merkittiin Eljasibin, Joojadan, Joohananin ja Jadduan aikana perhekunta-päämiehet muistiin; ja samoin papit persialaisen Daarejaveksen hallitusaikana.
\par 23 Leevin jälkeläisistä ovat perhekunta-päämiehet merkittyinä aikakirjaan Joohananin, Eljasibin pojan, aikaan asti.
\par 24 Ja leeviläisten päämiehet olivat Hasabja, Seerebja ja Jeesua, Kadmielin poika, sekä heidän veljensä, jotka seisoivat heitä vastapäätä ylistämässä ja kiittämässä, niinkuin Jumalan mies Daavid oli käskenyt, osasto osaston vieressä;
\par 25 ja Mattanja, Bakbukja, Obadja, Mesullam, Talmon ja Akkub, ovenvartijat, jotka vartioivat porttien varastohuoneita.
\par 26 Nämä olivat Joojakimin, Jeesuan pojan, Joosadakin pojanpojan, aikalaisia ja käskynhaltija Nehemian ja pappi Esran, kirjanoppineen, aikalaisia.
\par 27 Kun Jerusalemin muuri oli vihittävä, haettiin leeviläiset kaikista heidän asuinpaikoistansa ja tuotiin Jerusalemiin viettämään vihkimäjuhlaa iloiten, kiitosvirsiä ja lauluja laulaen, kymbaaleja, harppuja ja kanteleita soittaen.
\par 28 Niin kokoontuivat veisaajain pojat Lakeudelta, Jerusalemin ympäristöstä, netofalaisten kylistä,
\par 29 Beet-Gilgalista ja Geban ja Asmavetin peltomailta, sillä veisaajat olivat rakentaneet itselleen kyliä Jerusalemin ympäristöön.
\par 30 Papit ja leeviläiset puhdistautuivat ja puhdistivat kansan sekä portit ja muurin.
\par 31 Sitten minä vein Juudan päämiehet rinteelle, muurin yläpuolelle, ja järjestin kaksi suurta laulukuntaa ja juhlakulkuetta muurin yläpuolelle, oikealle kädelle, Lantaportille päin.
\par 32 Ja heidän jäljessään kulkivat Hoosaja ja toinen puoli Juudan päämiehiä
\par 33 sekä Asarja, Esra, Mesullam,
\par 34 Juuda, Benjamin, Semaja ja Jeremia
\par 35 sekä pappien poikia torvineen, ja Sakarja, Joonatanin poika, joka oli Semajan poika, joka Mattanjan poika, joka Miikajan poika, joka Sakkurin poika, joka Aasafin poika,
\par 36 sekä hänen veljensä Semaja, Asarel, Milalai, Gilalai, Maai, Netanel, Juuda ja Hanani Daavidin, Jumalan miehen, soittimia kantaen; ja Esra, kirjanoppinut, kulki heidän edellään.
\par 37 He menivät Lähdeportin ohi ja nousivat suoraan Daavidin kaupungin portaita sinne, mistä noustaan muurille, Daavidin palatsin yläpuolelle, aina Vesiportille asti idässä.
\par 38 Toinen laulukunta kulki vastakkaiseen suuntaan, ja sen jäljessä minä ja toinen puoli kansaa, muurin yläpuolitse, Uunitornin yläpuolitse, aina Leveälle muurille asti,
\par 39 ja Efraimin-portin yläpuolitse ja Vanhanportin, Kalaportin, Hananelin-tornin ja Hammea-tornin ohitse aina Lammasportille saakka; ja he asettuivat Vankilaportin luokse.
\par 40 Sitten molemmat laulukunnat asettuivat Jumalan temppeliin, ja samoin minä ja toinen puoli esimiehiä minun kanssani,
\par 41 sekä papit Eljakim, Maaseja, Minjamin, Miikaja, Eljoenai, Sakarja ja Hananja torvineen,
\par 42 ja Maaseja, Semaja, Eleasar, Ussi, Joohanan, Malkia, Eelam ja Eser. Veisaajat kaiuttivat virsiään, ja Jisrahja oli johtajana.
\par 43 Ja he uhrasivat sinä päivänä suuria uhreja ja iloitsivat, sillä Jumala oli suuresti ilahuttanut heitä; myöskin vaimot ja lapset iloitsivat. Ja ilo kuului Jerusalemista kauas.
\par 44 Sinä päivänä asetettiin miehiä valvomaan varastokammioita, joissa antimet, uutiset ja kymmenykset säilytettiin. Niihin oli koottava kaupunkien peltomailta papeille ja leeviläisille lain mukaan tulevat osuudet. Sillä Juuda iloitsi palvelusta toimittavista papeista ja leeviläisistä.
\par 45 Ja nämä toimittivat tehtävänsä Jumalaansa palvellen ja puhdistuksia suorittaen, samoin veisaajat ja ovenvartijat, niinkuin Daavid ja hänen poikansa Salomo olivat käskeneet.
\par 46 Sillä jo muinoin, Daavidin ja Aasafin aikana, oli veisaajain päämiehiä, ja veisattiin ylistys- ja kiitosvirsiä Jumalalle.
\par 47 Ja koko Israel antoi Serubbaabelin ja Nehemian aikana veisaajille ja ovenvartijoille heidän osuutensa, heidän jokapäiväiset tarpeensa; he pyhittivät lahjoja myöskin leeviläisille, ja leeviläiset pyhittivät lahjoja Aaronin pojille.

\chapter{13}

\par 1 Sinä päivänä luettiin Mooseksen kirjaa kansan kuullen, ja siihen havaittiin kirjoitetun, ettei ammonilainen eikä mooabilainen koskaan pääse Jumalan seurakuntaan,
\par 2 sen tähden että he eivät tulleet leipää ja vettä tuoden israelilaisia vastaan ja koska hän palkkasi heitä vastaan Bileamin kiroamaan heidät, vaikka meidän Jumalamme muuttikin kirouksen siunaukseksi.
\par 3 Kuultuaan lain he erottivat kaiken sekakansan Israelista.
\par 4 Mutta sitä ennen oli pappi Eljasib, Tobian sukulainen, joka oli asetettu meidän Jumalamme temppelin kammionhoitajaksi,
\par 5 sisustanut hänelle suuren kammion, johon ennen oli pantu ruokauhri, suitsuke, kalut ja jyvä-, viini- ja öljykymmenykset, jotka olivat määrätyt leeviläisille, veisaajille ja ovenvartijoille, sekä papeille tuleva anti.
\par 6 Tämän kaiken tapahtuessa en minä ollut Jerusalemissa; sillä Artahsastan, Baabelin kuninkaan, kolmantenakymmenentenä toisena hallitusvuotena minä olin mennyt kuninkaan luo. Mutta jonkun ajan kuluttua minä pyysin kuninkaalta lomaa
\par 7 ja palasin Jerusalemiin. Ja kun minä tulin huomaamaan, minkä pahan Eljasib oli tehnyt Tobian eduksi, sisustaessaan hänelle kammion Jumalan temppelin esipihoissa,
\par 8 pahastuin minä suuresti, ja minä heitin kaikki Tobian huonekalut ulos kammiosta.
\par 9 Ja minä käskin puhdistaa kammiot ja panetin niihin takaisin Jumalan temppelin kalut, ruokauhrin ja suitsukkeen.
\par 10 Vielä minä sain tietää, ettei leeviläisille oltu annettu heidän osuuksiaan, ja niin olivat leeviläiset ja veisaajat, joiden olisi ollut tehtävä palvelusta, vetäytyneet kukin maatilalleen.
\par 11 Minä nuhtelin esimiehiä siitä ja sanoin: "Minkätähden on Jumalan temppeli laiminlyöty?" Ja minä kokosin leeviläiset ja veisaajat ja asetin heidät paikoilleen.
\par 12 Ja koko Juuda toi kymmenykset jyvistä, viinistä ja öljystä varastohuoneisiin.
\par 13 Ja minä asetin Selemjan, papin, ja Saadokin, kirjanoppineen, ja Pedajan, leeviläisen, valvomaan varastohuoneita ja annoin heidän apulaisekseen Haananin, Sakkurin pojan, Mattanjan pojanpojan; sillä heitä pidettiin luotettavina, ja heidän tehtävänään oli toimittaa jako veljiensä kesken.
\par 14 Muista tämän tähden minua, Jumalani, äläkä pyyhi pois minun hurskaita tekojani, jotka minä olen tehnyt Jumalani temppelin ja siinä toimitettavan palveluksen hyväksi.
\par 15 Siihen aikaan minä näin Juudassa niitä, jotka sapattina polkivat viinikuurnaa ja jotka kuljettivat viljaa ja kuormittivat aaseja sillä sekä viinillä, rypäleillä, viikunoilla ja kaikenlaisella muulla kuormatavaralla. Ja he toivat niitä Jerusalemiin sapatinpäivänä; ja minä varoitin heitä, kun he myivät elintarpeita.
\par 16 Sinne oli myös asettunut tyyrolaisia, jotka toivat kalaa ja kaikenlaista kauppatavaraa ja myivät Juudan kansalle sapattina ja Jerusalemissa.
\par 17 Ja minä nuhtelin Juudan ylimyksiä ja sanoin heille: "Kuinka te teette näin pahasti ja rikotte sapatinpäivän?
\par 18 Eikö Jumalamme juuri sen tähden, että teidän isänne näin tekivät, antanut kaiken tämän onnettomuuden kohdata meitä ja tätä kaupunkia? Ja te tuotatte nyt vielä suuremman vihan Israelin ylitse, kun rikotte sapatinpäivän."
\par 19 Ja niin pian kuin oli tullut pimeä ennen sapattia Jerusalemin porteissa, käskin minä sulkea ovet ja kielsin avaamasta niitä ennen kuin sapatin jälkeen. Ja minä asetin palvelijoitani porteille, ettei yhtään kuormaa pääsisi kaupunkiin sapatinpäivänä.
\par 20 Kauppiaat ja kaikenlaisen tavaran myyjät jäivät yöksi Jerusalemin ulkopuolelle, kerran ja toisen.
\par 21 Mutta minä varoitin heitä ja sanoin heille: "Minkätähden te jäätte yöksi muurin edustalle? Jos vielä kerran teette sen, niin minä käyn teihin käsiksi." Sen jälkeen he eivät enää tulleet sapattina.
\par 22 Ja minä käskin leeviläisten puhdistautua ja tulla vartioimaan portteja, että sapatinpäivä pyhitettäisiin. Muista minua, Jumalani, myös tämän tähden ja armahda minua suuressa laupeudessasi.
\par 23 Siihen aikaan minä näin myös juutalaisia, jotka olivat naineet asdodilaisia, ammonilaisia ja mooabilaisia vaimoja.
\par 24 Ja heidän lapsistaan puolet puhuivat asdodinkieltä tai jonkun muun kansan kieltä, eivätkä osanneet puhua juudankieltä.
\par 25 Silloin minä nuhtelin heitä ja kirosin heidät, jopa löin muutamia heistä ja revin heitä parrasta; ja minä vannotin heitä Jumalan kautta: "Älkää antako tyttäriänne heidän pojillensa älkääkä ottako heidän tyttäriään vaimoiksi pojillenne tai itsellenne.
\par 26 Eikö Salomo, Israelin kuningas, tehnyt juuri tuollaista syntiä? Ei yhdelläkään monien kansojen joukossa ollut hänen vertaistaan kuningasta; hän oli Jumalallensa rakas, ja Jumala asetti hänet koko Israelin kuninkaaksi. Kuitenkin muukalaiset vaimot saattoivat hänetkin tekemään syntiä.
\par 27 Ja nytkö meidän täytyy kuulla teistä, että te olette tehneet kaiken tämän suuren pahan ja olleet uskottomat meidän Jumalaamme kohtaan, kun olette naineet muukalaisia vaimoja?"
\par 28 Ja yksi ylimmäisen papin Eljasibin pojan Joojadan pojista oli hooronilaisen Sanballatin vävy; hänet minä karkoitin luotani.
\par 29 Muista heidät, Jumalani, sillä he ovat saastuttaneet pappeuden sekä sekä pappeuden sekä pappeus- ja leeviläisliiton.
\par 30 Niin minä puhdistin heidät kaikesta muukalaisuudesta. Ja minä järjestin virkatehtävät papeille ja leeviläisille, kullekin hänen toimensa mukaan,
\par 31 niin myös halko-uhrilahjan määräaikoina ja uutislahjat. Muista, Jumalani, tämä minun hyväkseni.


\end{document}