\begin{document}

\title{Toinen Samuelin kirja}


\chapter{1}

\par 1 Saulin kuoleman jälkeen, kun Daavid, voitettuaan amalekilaiset, oli palannut takaisin ja ollut Siklagissa pari päivää,
\par 2 tuli kolmantena päivänä eräs mies leiristä Saulin luota, vaatteet reväistyinä ja multaa pään päällä. Ja tullessaan Daavidin luo hän heittäytyi maahan ja osoitti kunnioitusta.
\par 3 Daavid kysyi häneltä: "Mistä tulet?" Hän vastasi hänelle: "Minä olen päässyt pakoon Israelin leiristä".
\par 4 Niin Daavid sanoi hänelle: "Miten on asiat? Kerro minulle." Hän vastasi: "Väki on paennut taistelusta, paljon väkeä on kaatunut ja kuollut; myöskin Saul ja hänen poikansa Joonatan ovat kuolleet".
\par 5 Daavid kysyi nuorelta mieheltä, joka kertoi tämän hänelle: "Mistä tiedät, että Saul ja hänen poikansa Joonatan ovat kuolleet?"
\par 6 Nuori mies, joka oli kertonut hänelle tämän, vastasi: "Minä tulin sattumalta Gilboan vuorelle, ja katso, Saul nojasi keihääseensä, ja sotavaunut ja ratsumiehet ahdistivat häntä.
\par 7 Hän kääntyi taakseen ja nähdessään minut hän huusi minua; minä vastasin: 'Tässä olen'.
\par 8 Niin hän kysyi minulta: 'Kuka olet?' Minä vastasin hänelle: 'Olen amalekilainen'.
\par 9 Sitten hän sanoi minulle: 'Astu luokseni ja surmaa minut, sillä minä olen kangistumassa, vaikka olenkin vielä täysin hengissä'.
\par 10 Niin minä astuin hänen luoksensa ja surmasin hänet, sillä minä käsitin, ettei hän sortumisensa jälkeen jäisi henkiin. Ja minä otin kruunun, joka oli hänen päässänsä, ja rannerenkaan, joka oli hänen käsivarressaan, ja toin ne tänne herralleni."
\par 11 Niin Daavid tarttui vaatteisiinsa ja repäisi ne; samoin tekivät kaikki miehet, jotka olivat hänen kanssansa.
\par 12 Ja he pitivät valittajaisia ja itkivät ja paastosivat iltaan asti Saulin ja hänen poikansa Joonatanin tähden ja Herran kansan tähden ja Israelin heimon tähden, koska he olivat kaatuneet miekkaan.
\par 13 Ja Daavid kysyi nuorelta mieheltä, joka oli hänelle kertonut tämän: "Mistä sinä olet?" Hän vastasi: "Minä olen täällä muukalaisena elävän amalekilaisen miehen poika".
\par 14 Daavid sanoi hänelle: "Kuinka et peljännyt ojentaa kättäsi tuhotaksesi Herran voidellun?"
\par 15 Ja Daavid kutsui yhden palvelijoistaan ja sanoi: "Käy tänne, lyö hänet kuoliaaksi". Ja hän iski hänet kuoliaaksi.
\par 16 Mutta Daavid sanoi hänelle: "Sinun veresi tulkoon oman pääsi päälle, sillä oma suusi on todistanut sinua vastaan, kun sanoit: 'Minä olen surmannut Herran voidellun'".
\par 17 Ja Daavid viritti tämän itkuvirren Saulista ja hänen pojastaan Joonatanista
\par 18 ja käski opettaa sen, "Jousilaulun", Juudan pojille: katso, se on kirjoitettuna "Oikeamielisen kirjassa":
\par 19 "Sinun kaunistuksesi, Israel, on surmattuna kukkuloillasi. Kuinka ovatkaan sankarit kaatuneet!
\par 20 Älkää ilmoittako sitä Gatissa, älkää julistako voitonsanomaa Askelonin kaduilla, etteivät filistealaisten tyttäret iloitsisi, etteivät ympärileikkaamattomain tyttäret riemuitsisi.
\par 21 Te Gilboan vuoret, älköön teille kastetta tulko, älköön sadetta, älköön peltoja, joista anteja uhrataan. Sillä siellä on poisviskattuna sankarien kilpi, Saulin kilpi, öljyllä voitelematonna.
\par 22 Kaatuneitten verta, sankarien ytimiä ei Joonatanin jousi väistänyt, eikä tyhjänä palannut Saulin miekka.
\par 23 Saul ja Joonatan, rakkaat ja armaat, olivat elämässä ja kuolemassa erottamattomat, olivat nopeammat kuin kotkat, leijonia väkevämmät.
\par 24 Te Israelin tyttäret, itkekää Saulia, häntä, joka puetti teidät purppuraan ja koristeihin, joka kiinnitti pukuunne kultahelyjä.
\par 25 Kuinka ovat sankarit taistelussa kaatuneet! Joonatan on surmattuna kukkuloillasi.
\par 26 Minä suren sinua, veljeni Joonatan; sinä olit minulle ylen rakas. Rakkautesi oli minulle ihmeellisempi kuin naisen rakkaus.
\par 27 Kuinka ovatkaan sankarit kaatuneet, kuinka ovat hukkuneet taisteluaseet!"

\chapter{2}

\par 1 Sen jälkeen Daavid kysyi Herralta: "Menenkö minä johonkin Juudan kaupunkiin?" Herra vastasi hänelle: "Mene". Ja Daavid kysyi: "Mihinkä minä menen?" Hän vastasi: "Hebroniin".
\par 2 Niin Daavid meni sinne, mukanaan molemmat vaimonsa, jisreeliläinen Ahinoam ja Abigail, karmelilaisen Naabalin vaimo.
\par 3 Ja Daavid vei sinne myös miehet, jotka olivat hänen kanssaan, kunkin perheinensä; ja he asettuivat Hebronin ympärillä oleviin kaupunkeihin.
\par 4 Ja Juudan miehet tulivat ja voitelivat siellä Daavidin Juudan heimon kuninkaaksi. Kun Daavidille ilmoitettiin, että Gileadin Jaabeksen miehet olivat haudanneet Saulin,
\par 5 lähetti Daavid lähettiläät Gileadin Jaabeksen miesten luo sanomaan heille: "Herra siunatkoon teitä, koska olette osoittaneet herrallenne Saulille laupeutta hautaamalla hänet.
\par 6 Osoittakoon nyt Herra laupeutta ja uskollisuutta teitä kohtaan. Minäkin teen teille hyvää, koska tämän teitte.
\par 7 Rohkaiskaa siis itsenne ja olkaa urhoolliset; teidän herranne Saul tosin on kuollut, mutta Juudan heimo on voidellut kuninkaaksensa minut."
\par 8 Mutta Abner, Neerin poika, Saulin sotapäällikkö, otti Saulin pojan Iisbosetin ja vei hänet Mahanaimiin
\par 9 ja asetti hänet Gileadin, Asurin, Jisreelin, Efraimin, Benjaminin ja koko Israelin kuninkaaksi.
\par 10 Saulin poika Iisboset oli neljänkymmenen vuoden vanha tullessaan Israelin kuninkaaksi ja hallitsi kaksi vuotta. Ainoastaan Juudan heimo seurasi Daavidia.
\par 11 Ja aika, minkä Daavid oli Juudan heimon kuninkaana Hebronissa, oli seitsemän vuotta ja kuusi kuukautta.
\par 12 Abner, Neerin poika, ja Saulin pojan Iisbosetin palvelijat lähtivät Mahanaimista Gibeoniin.
\par 13 Ja Jooab, Serujan poika, ja Daavidin palvelijat lähtivät hekin liikkeelle, ja he kohtasivat toisensa Gibeonin lammikolla; he asettuivat toiset tälle, toiset tuolle puolelle lammikon.
\par 14 Ja Abner sanoi Jooabille: "Nouskoon nuoria miehiä pitämään taistelukisaa meidän edessämme". Jooab sanoi: "Nouskoon vain".
\par 15 Niin heitä nousi ja astui esiin sama määrä: kaksitoista Benjaminin ja Saulin pojan Iisbosetin puolelta sekä kaksitoista Daavidin palvelijaa.
\par 16 Ja he tarttuivat kukin vastustajaansa päähän ja pistivät miekkansa toinen toisensa kylkeen ja kaatuivat yhdessä. Niin sen paikan nimeksi tuli Helkat-Suurim, ja se on Gibeonin luona.
\par 17 Ja sinä päivänä syntyi hyvin kova taistelu; mutta Daavidin palvelijat voittivat Abnerin ja Israelin miehet.
\par 18 Ja siellä oli kolme Serujan poikaa: Jooab, Abisai ja Asael. Ja Asael oli nopeajalkainen kuin gaselli kedolla.
\par 19 Ja Asael ajoi takaa Abneria eikä poikennut oikealle eikä vasemmalle Abnerin jäljestä.
\par 20 Niin Abner käännähti taaksensa ja sanoi: "Sinäkö se olet, Asael?" Hän vastasi: "Minä".
\par 21 Ja Abner sanoi hänelle: "Käänny oikealle tai vasemmalle ja ota joku noista nuorista miehistä kiinni ja riistä siltä sota-asu". Mutta Asael ei tahtonut luopua hänestä.
\par 22 Niin Abner vielä kerran sanoi Asaelille: "Luovu minusta, muutoin lyön sinut maahan. Mutta kuinka minä sitten voisin katsoa sinun veljeäsi Jooabia silmiin?"
\par 23 Mutta kun hän ei tahtonut luopua hänestä, pisti Abner häntä keihään varsipuolella vatsaan, niin että keihäs tuli takaapäin ulos; ja hän kaatui ja kuoli siihen paikkaan. Ja jokainen, joka tuli siihen paikkaan, mihin Asael oli kaatunut ja kuollut, pysähtyi siihen.
\par 24 Mutta Jooab ja Abisai ajoivat Abneria takaa ja tulivat, kun aurinko oli laskenut, Amman kukkulalle, joka on vastapäätä Giiahia, Gibeonin erämaahan päin.
\par 25 Silloin benjaminilaiset kokoontuivat Abnerin taakse yhteen joukkoon ja asettuivat kukkulan laelle.
\par 26 Ja Abner huusi Jooabille ja sanoi: "Pitääkö miekan syödä ainiaan? Etkö sinä ymmärrä, että siitä tulee vain katkeruutta jäljestäpäin? Etkö jo käske väen lakata ajamasta takaa veljiänsä?"
\par 27 Jooab vastasi: "Niin totta kuin Jumala elää: jos sinä et olisi puhunut, niin varmasti väki vasta huomenaamuna olisi lähtenyt pois ajamasta takaa veljiänsä".
\par 28 Ja Jooab puhalsi pasunaan; niin kaikki väki pysähtyi eikä enää ajanut takaa Israelia. Ja he eivät enää taistelleet.
\par 29 Mutta Abner ja hänen miehensä kulkivat koko sen yön Aromaata ja menivät Jordanin yli, kulkivat koko Bitronin läpi ja tulivat Mahanaimiin.
\par 30 Ja Jooab kokosi kaiken kansan, palattuaan ajamasta takaa Abneria. Ja Daavidin palvelijoista puuttui yhdeksäntoista miestä ja Asael.
\par 31 Mutta Daavidin palvelijat olivat lyöneet benjaminilaisia ja Abnerin miehiä kuoliaaksi kolmesataa kuusikymmentä miestä.
\par 32 Ja he ottivat Asaelin ja hautasivat hänet hänen isänsä hautaan Beetlehemiin. Ja Jooab ja hänen miehensä kulkivat koko sen yön ja tulivat päivän valjetessa Hebroniin.

\chapter{3}

\par 1 Sota Saulin suvun ja Daavidin suvun välillä tuli pitkälliseksi. Mutta Daavid vahvistui vahvistumistaan, Saulin suku heikontui heikontumistaan.
\par 2 Ja Daavidille syntyi poikia Hebronissa. Hänen esikoisensa oli Amnon, jonka äiti oli jisreeliläinen Ahinoam.
\par 3 Hänen toinen poikansa oli Kilab, jonka äiti oli Abigail, karmelilaisen Naabalin vaimo; kolmas oli Absalom, Gesurin kuninkaan Talmain tyttären, Maakan, poika.
\par 4 Neljäs oli Adonia, Haggitin poika, ja viides Sefatja, Abitalin poika.
\par 5 Kuudes oli Jitream, Daavidin vaimon Eglan poika. Nämä syntyivät Daavidille Hebronissa.
\par 6 Niin kauan kuin oli sota Saulin suvun ja Daavidin suvun välillä, piti Abner voimakkaasti Saulin suvun puolta.
\par 7 Saulilla oli ollut sivuvaimo, nimeltä Rispa, Aijan tytär. Ja Iisboset sanoi Abnerille: "Miksi olet mennyt minun isäni sivuvaimon tykö?"
\par 8 Abner vihastui kovin Iisbosetin sanoista ja sanoi: "Olenko minä koiranpää, Juudasta kotoisin? Kun minä nytkin osoitan laupeutta isäsi Saulin suvulle, hänen veljilleen ja ystävilleen enkä ole antanut sinun joutua Daavidin käsiin, niin sinä kuitenkin nyt syytät minua rikoksen teosta naiselle.
\par 9 Jumala rangaiskoon Abneria nyt ja vasta, jos minä en tee Daavidille, niinkuin Herra on hänelle vannonut:
\par 10 minä otan pois kuninkuuden Saulin suvulta ja pystytän Daavidille valtaistuimen Israeliin ja Juudaan, Daanista aina Beersebaan asti."
\par 11 Eikä hän enää voinut vastata Abnerille sanaakaan, sillä hän pelkäsi häntä.
\par 12 Mutta Abner lähetti paikalla sanansaattajat Daavidin luo sanomaan: "Kenenkä on maa?" ja vielä näin: "Tee liitto minun kanssani; ja katso, minun käteni on oleva sinun kanssasi ja on kääntävä koko Israelin sinun puolellesi".
\par 13 Hän vastasi: "Hyvä, minä teen liiton sinun kanssasi. Yhtä minä kuitenkin vaadin sinulta: älä astu minun kasvojeni eteen tuomatta Miikalia, Saulin tytärtä, kun tulet astuaksesi minun kasvojeni eteen."
\par 14 Sitten Daavid lähetti sanansaattajat Iisbosetin, Saulin pojan, luo sanomaan: "Anna minulle vaimoni Miikal, jonka minä olen kihlannut itselleni sadalla filistealaisten esinahalla".
\par 15 Iisboset lähetti ottamaan hänet hänen mieheltään Paltielilta, Laiksen pojalta.
\par 16 Ja hänen miehensä lähti hänen mukanaan ja seurasi häntä aina Bahurimiin asti, itkien lakkaamatta. Niin Abner sanoi hänelle: "Mene tiehesi, palaja takaisin". Ja hän palasi takaisin.
\par 17 Abner oli neuvotellut Israelin vanhinten kanssa sanoen: "Jo kauan olette halunneet saada Daavidin kuninkaaksenne.
\par 18 Niin pankaa se nyt toimeen, sillä Herra on sanonut Daavidille näin: 'Palvelijani Daavidin kädellä minä vapautan kansani Israelin filistealaisten käsistä ja kaikkien heidän vihollistensa käsistä'."
\par 19 Abner puhui tämän myös benjaminilaisille, ja Abner meni ja puhui Daavidillekin Hebronissa, mitä mieltä Israel ja koko Benjaminin heimo olivat.
\par 20 Kun Abner tuli Daavidin luo Hebroniin, mukanaan kaksikymmentä miestä, laittoi Daavid pidot Abnerille ja miehille, jotka olivat hänen kanssaan.
\par 21 Ja Abner sanoi Daavidille: "Minä nousen ja menen kokoamaan kaiken Israelin herrani, kuninkaan, luo, että he tekisivät liiton sinun kanssasi ja sinä saisit hallittaviksesi kaikki, joita haluat". Niin Daavid päästi Abnerin menemään, ja hän lähti rauhassa.
\par 22 Katso, silloin Daavidin palvelijat ja Jooab tulivat ryöstöretkeltä ja toivat mukanaan paljon saalista; mutta Abner ei ollut enää Daavidin luona Hebronissa, sillä tämä oli päästänyt hänet menemään, ja hän oli lähtenyt rauhassa.
\par 23 Kun Jooab ja kaikki sotaväki, joka oli hänen kanssaan, tulivat, kerrottiin Jooabille: "Abner, Neerin poika, tuli kuninkaan luo, ja hän päästi hänet menemään, ja hän lähti rauhassa".
\par 24 Silloin Jooab meni kuninkaan luo ja sanoi: "Mitä olet tehnyt? Katso, Abner tuli sinun luoksesi; miksi päästit hänet menemään menojansa?
\par 25 Etkö ymmärrä, että Abner, Neerin poika, tuli pettämään sinua ja urkkimaan lähtöäsi ja tuloasi ja kaikkea, mitä teet?"
\par 26 Sitten Jooab lähti Daavidin luota ja lähetti sanansaattajat Abnerin jälkeen, ja he palauttivat hänet takaisin Boor-Siirasta; eikä Daavid tiennyt siitä mitään.
\par 27 Kun Abner tuli takaisin Hebroniin, vei Jooab hänet syrjään porttiholvin keskelle, muka puhutellakseen häntä kahden kesken, ja pisti häntä siellä vatsaan, niin että hän kuoli - hänen veljensä Asaelin veren kostoksi.
\par 28 Kun Daavid sitten sai sen kuulla, sanoi hän: "Abnerin, Neerin pojan, vereen olen viaton minä ja minun kuninkuuteni Herran edessä iankaikkisesti.
\par 29 Tulkoon se Jooabin pään päälle ja kaiken hänen isänsä perheen päälle, älköötkä Jooabin suvusta loppuko vuotoa tai pitalia sairastavat, kainalosauvoilla kulkijat, miekkaan kaatuvat ja leivän puutteessa olijat."
\par 30 Näin Jooab ja hänen veljensä Abisai tappoivat Abnerin, sentähden että tämä oli surmannut heidän veljensä Asaelin taistelussa Gibeonin luona.
\par 31 Mutta Daavid sanoi Jooabille ja kaikelle väelle, joka oli hänen kanssaan: "Reväiskää vaatteenne ja kääriytykää säkkeihin ja pitäkää valittajaiset Abnerille". Ja kuningas Daavid kulki paarien perässä.
\par 32 Ja he hautasivat Abnerin Hebroniin; ja kuningas korotti äänensä ja itki Abnerin haudalla, ja kaikki kansa itki.
\par 33 Ja kuningas viritti itkuvirren Abnerista ja sanoi: "Pitikö Abnerin kuolla, niinkuin kuolee houkka?
\par 34 Eivät olleet sinun kätesi sidotut eivätkä jalkasi vaskikahleisiin pannut. Sinä kaaduit, niinkuin kaadutaan vilpillisten käden kautta." Niin kaikki kansa itki häntä vielä enemmän.
\par 35 Ja kaikki kansa tuli, kun vielä oli päivä, vaatimaan, että Daavid söisi jotakin; mutta Daavid vannoi ja sanoi: "Jumala rangaiskoon minua nyt ja vasta, jos minä maistan leipää tai muuta, ennenkuin aurinko laskee".
\par 36 Kun kaikki kansa sai kuulla sen, miellytti se heitä, samoinkuin kaikki muu, mitä kuningas teki, oli kaikelle kansalle mieleen.
\par 37 Ja kaikki kansa ja koko Israel ymmärsivät sinä päivänä, että Abnerin, Neerin pojan, surma ei ollut lähtenyt kuninkaasta.
\par 38 Ja kuningas sanoi palvelijoilleen: "Ettekö tiedä, että ruhtinas ja suuri mies on tänä päivänä kaatunut Israelissa?
\par 39 Mutta minä olen vielä heikko, vaikka olen voideltu kuninkaaksi, ja nuo miehet, Serujan pojat, ovat liian väkivaltaiset minun hillitä. Herra maksakoon sille, joka tekee pahaa, hänen pahuutensa mukaan."

\chapter{4}

\par 1 Kun Saulin poika kuuli Abnerin kuolleen Hebronissa, herposivat hänen kätensä, ja koko Israel peljästyi.
\par 2 Ja Saulin pojalla oli partiojoukkojen päämiehinä kaksi miestä, toisen nimi oli Baana ja toisen Reekab, beerotilaisen Rimmonin poikia, benjaminilaisia; sillä Beerotkin luetaan kuuluvaksi Benjaminiin.
\par 3 Mutta beerotilaiset olivat paenneet Gittaimiin ja asuivat siellä muukalaisina, niinkuin asuvat vielä tänäkin päivänä.
\par 4 Joonatanilla, Saulin pojalla, oli poika, joka oli rampa jaloistaan. Hän oli viiden vuoden vanha, kun sanoma Saulista ja Joonatanista tuli Jisreelistä; ja hänen hoitajansa otti hänet ja pakeni. Mutta kun tämä hätääntyneenä pakeni, putosi poika ja tuli ontuvaksi; ja hänen nimensä oli Mefiboset.
\par 5 Beerotilaisen Rimmonin pojat Reekab ja Baana lähtivät matkaan ja tulivat päivän ollessa palavimmillaan Iisbosetin taloon, kun hän oli puolipäivälevollansa.
\par 6 Ja katso, he tulivat sisälle taloon muka ottamaan nisuja, mutta pistivät häntä vatsaan. Ja Reekab ja hänen veljensä Baana pääsivät pakoon.
\par 7 He tulivat niinmuodoin taloon, kun hän lepäsi vuoteellansa makuuhuoneessaan, pistivät hänet kuoliaaksi, löivät häneltä pään poikki ja ottivat sen mukaansa. He kulkivat sitten Aromaan tietä koko yön,
\par 8 toivat Iisbosetin pään Daavidille Hebroniin ja sanoivat kuninkaalle: "Katso, tässä on Iisbosetin, Saulin pojan, sinun vihamiehesi, pää, hänen, joka väijyi sinun henkeäsi. Herra on tänä päivänä kostanut Saulille ja hänen jälkeläisillensä herrani, kuninkaan, puolesta."
\par 9 Niin Daavid vastasi Reekabille ja hänen veljellensä Baanalle, beerotilaisen Rimmonin pojille, ja sanoi heille: "Niin totta kuin Herra elää, joka on vapahtanut minut kaikesta hädästä:
\par 10 sen, joka ilmoitti minulle ja sanoi: 'Katso, Saul on kuollut', luullen tuovansa ilosanoman, minä otatin kiinni ja surmautin Siklagissa antaakseni hänelle sanansaattajan palkan;
\par 11 eikö minun paljoa enemmän nyt, kun jumalattomat miehet ovat murhanneet syyttömän miehen hänen omassa kodissaan, hänen vuoteeseensa, ole vaadittava hänen verensä teidän kädestänne ja hävitettävä teidät maan päältä?"
\par 12 Ja Daavid käski nuorten miesten surmata heidät. Niin he hakkasivat poikki heidän kätensä ja jalkansa ja hirttivät heidät Hebronin lammikon rannalle. Mutta Iisbosetin pään he ottivat ja hautasivat sen Abnerin hautaan Hebroniin.

\chapter{5}

\par 1 Sitten kaikki Israelin sukukunnat tulivat Daavidin luo Hebroniin ja sanoivat näin: "Katso, me olemme sinun luutasi ja lihaasi.
\par 2 Jo kauan sitten, Saulin vielä ollessa kuninkaanamme, sinä saatoit Israelin lähtemään ja tulemaan. Ja sinulle on Herra sanonut: 'Sinä olet kaitseva minun kansaani Israelia, ja sinä olet oleva Israelin ruhtinas'."
\par 3 Ja kaikki Israelin vanhimmat tulivat kuninkaan tykö Hebroniin, ja kuningas Daavid teki heidän kanssaan liiton Hebronissa, Herran edessä; ja sitten he voitelivat Daavidin Israelin kuninkaaksi.
\par 4 Daavid oli kolmenkymmenen vuoden vanha tullessaan kuninkaaksi ja hallitsi neljäkymmentä vuotta.
\par 5 Hebronissa hän hallitsi Juudaa seitsemän vuotta ja kuusi kuukautta, ja Jerusalemissa hän hallitsi koko Israelia ja Juudaa kolmekymmentä kolme vuotta.
\par 6 Ja kuningas meni miehinensä Jerusalemiin jebusilaisia vastaan, jotka asuivat siinä maassa. He sanoivat Daavidille näin: "Tänne sinä et tule, vaan sokeat ja ontuvat karkoittavat sinut sanomalla: 'Ei tule Daavid tänne'".
\par 7 Mutta Daavid valloitti Siionin vuorilinnan, se on Daavidin kaupungin.
\par 8 Ja Daavid sanoi sinä päivänä: "Jokainen, joka surmaa jebusilaisen ja tunkeutuu vesijohdolle asti, hän voittaa ne sokeat ja ontuvat, joita Daavid vihaa". Sentähden on tapana sanoa: "Sokea ja ontuva älköön tulko taloon".
\par 9 Sitten Daavid asettui vuorilinnaan ja kutsui sen Daavidin kaupungiksi. Ja Daavid rakenteli sitä yltympäri, Millosta sisälle päin.
\par 10 Ja Daavid tuli yhä mahtavammaksi, ja Herra, Jumala Sebaot, oli hänen kanssansa.
\par 11 Hiiram, Tyyron kuningas, lähetti sanansaattajat Daavidin luo sekä setripuita, puuseppiä ja kivenhakkaajia; ja nämä rakensivat Daavidille palatsin.
\par 12 Ja Daavid ymmärsi, että Herra oli vahvistanut hänet Israelin kuninkaaksi ja korottanut hänen kuninkuutensa kansansa Israelin tähden.
\par 13 Ja Hebronista tultuaan Daavid otti Jerusalemista vielä useampia sivuvaimoja ja vaimoja. Ja Daavidille syntyi vielä poikia ja tyttäriä.
\par 14 Ja nämä ovat niiden poikien nimet, jotka syntyivät hänelle Jerusalemissa: Sammua, Soobab, Naatan, Salomo,
\par 15 Jibhar, Elisua, Nefeg, Jaafia,
\par 16 Elisama, Eljada ja Elifelet.
\par 17 Mutta kun filistealaiset kuulivat, että Daavid oli voideltu Israelin kuninkaaksi, lähtivät kaikki filistealaiset etsimään Daavidia. Kun Daavid sen kuuli, meni hän alas vuorilinnaan.
\par 18 Kun filistealaiset olivat tulleet ja levittäytyneet Refaimin tasangolle,
\par 19 kysyi Daavid Herralta: "Menenkö minä filistealaisia vastaan? Annatko sinä heidät minun käsiini?" Ja Herra sanoi Daavidille: "Mene, sillä minä annan filistealaiset sinun käsiisi".
\par 20 Niin Daavid tuli Baal-Perasimiin. Ja siellä Daavid voitti heidät ja sanoi: "Herra on murtanut viholliseni minun edessäni, niinkuin vedet murtavat". Siitä sen paikan nimeksi tuli Baal-Perasim.
\par 21 He jättivät siihen epäjumalankuvansa, ja Daavid ja hänen miehensä veivät ne pois.
\par 22 Mutta filistealaiset tulivat vielä kerran ja levittäytyivät Refaimin tasangolle.
\par 23 Niin Daavid kysyi Herralta, ja hän vastasi: "Älä mene heitä vastaan, vaan kierrä heidät takaapäin ja hyökkää heidän kimppuunsa balsamipuiden puolelta.
\par 24 Ja kun kuulet astunnan kahinan balsamipuiden latvoista, niin ryntää nopeasti, sillä silloin Herra on käynyt sinun edelläsi tuhotakseen filistealaisten leirin."

\chapter{6}

\par 1 Daavid kokosi jälleen kaikki valiomiehet Israelista, kolmekymmentä tuhatta miestä.
\par 2 Ja Daavid ja kaikki väki, joka oli hänen kanssaan, nousi ja lähti Juudan Baalasta, tuomaan sieltä Jumalan arkkia, jonka Herra Sebaot oli ottanut nimiinsä, hän, jonka istuinta kerubit kannattavat.
\par 3 Ja he panivat Jumalan arkin uusiin vaunuihin ja veivät sen pois Abinadabin talosta, joka oli mäellä; ja Ussa ja Ahjo, Abinadabin pojat, ohjasivat niitä uusia vaunuja.
\par 4 Niin he veivät Jumalan arkin pois Abinadabin talosta, joka oli mäellä, kulkien Jumalan arkin mukana, ja Ahjo kulki arkin edellä.
\par 5 Ja Daavid ynnä koko Israelin heimo karkeloi kaikin voimin Herran edessä soittaen kaikenkaltaisilla kypressipuisilla soittimilla, kanteleilla, harpuilla, vaskirummuilla, helistimillä ja kymbaaleilla.
\par 6 Mutta kun he tulivat Naakonin puimatantereen luo, ojensi Ussa kätensä Jumalan arkkiin ja tarttui siihen, sillä härät kompastuivat.
\par 7 Silloin Herran viha syttyi Ussaa kohtaan, ja Jumala löi hänet siinä hänen hairahduksensa tähden, niin että hän kuoli siihen, Jumalan arkin ääreen.
\par 8 Mutta Daavid pahastui siitä, että Herra niin oli murtanut Ussan. Siitä se paikka sai nimekseen Peres-Ussa, aina tähän päivään saakka.
\par 9 Ja Daavid pelkäsi sinä päivänä Herraa, niin että hän sanoi: "Kuinka Herran arkki voi tulla minun tyköni?"
\par 10 Eikä Daavid tahtonut siirtää Herran arkkia tykönsä Daavidin kaupunkiin, vaan Daavid toimitti sen syrjään, gatilaisen Oobed-Edomin taloon.
\par 11 Ja Herran arkki jäi kolmeksi kuukaudeksi gatilaisen Oobed-Edomin taloon. Ja Herra siunasi Oobed-Edomia ja koko hänen taloansa.
\par 12 Kun kuningas Daavidille kerrottiin, että Herra oli Jumalan arkin tähden siunannut Oobed-Edomin taloa ja kaikkea, mitä hänellä oli, niin Daavid meni ja toi riemuiten Jumalan arkin Oobed-Edomin talosta Daavidin kaupunkiin.
\par 13 Kun Herran arkin kantajat olivat astuneet kuusi askelta, uhrasi hän härän ja juottovasikan.
\par 14 Ja Daavid hyppi kaikin voimin Herran edessä, ja Daavid oli puettu pellavakasukkaan.
\par 15 Niin Daavid ja koko Israelin heimo toivat Herran arkin riemun raikuessa ja pasunain pauhatessa.
\par 16 Kun Herran arkki tuli Daavidin kaupunkiin, katseli Miikal, Saulin tytär, ikkunasta; ja nähdessään kuningas Daavidin karkeloivan ja hyppivän Herran edessä halveksi hän häntä sydämessänsä.
\par 17 Ja kun he olivat tuoneet Herran arkin ja asettaneet sen paikalleen majaan, jonka Daavid oli sille pystyttänyt, uhrasi Daavid polttouhreja ja yhteysuhreja Herran edessä.
\par 18 Ja kun Daavid oli uhrannut polttouhrin ja yhteysuhrit, siunasi hän kansan Herran Sebaotin nimeen.
\par 19 Ja hän jakoi koko kansalle, kaikelle Israelin joukolle, sekä miehille että naisille, kullekin kakun, kappaleen lihaa ja rypälekakun. Sitten kaikki kansa lähti, kukin kotiinsa.
\par 20 Mutta kun Daavid palasi takaisin tervehtimään perhettänsä, meni Miikal, Saulin tytär, ulos Daavidia vastaan ja sanoi: "Kuinka arvokkaana onkaan Israelin kuningas nyt esiintynyt, kun on tänä päivänä paljastanut itsensä palvelijainsa palvelijattarien silmien edessä, niinkuin kevytmielinen ihminen paljastautuu!"
\par 21 Niin Daavid sanoi Miikalille: "Herran edessä, joka on valinnut minut, sivu sinun isäsi ja koko hänen sukunsa, ja asettanut minut Herran kansan, Israelin, ruhtinaaksi - Herran edessä minä karkeloin;
\par 22 ja tähänkin minä pidän itseni liian vähäisenä ja olen omissa silmissäni halpa; mutta ne palvelijattaret, joista sinä puhuit, tulevat minua kunnioittamaan".
\par 23 Ja Miikal, Saulin tytär, ei saanut lasta kuolinpäiväänsä asti.

\chapter{7}

\par 1 Kerran, kun kuningas istui linnassaan, sittenkuin Herra oli suonut hänen päästä rauhaan kaikilta hänen ympärillään olevilta vihollisilta,
\par 2 sanoi kuningas profeetta Naatanille: "Katso, minä asun setripuisessa linnassa, mutta Jumalan arkki asuu telttakankaan suojassa".
\par 3 Naatan sanoi kuninkaalle: "Tee vain kaikki, mitä mielessäsi on, sillä Herra on sinun kanssasi".
\par 4 Mutta sinä yönä tapahtui, että Naatanille tuli tämä Herran sana:
\par 5 "Mene ja sano minun palvelijalleni Daavidille: Näin sanoo Herra: Sinäkö rakentaisit minulle huoneen asuakseni?
\par 6 Enhän minä ole asunut huoneessa siitä päivästä asti, jona johdatin israelilaiset Egyptistä, tähän päivään saakka, vaan minä olen vaeltanut asuen teltta-asumuksessa.
\par 7 Olenko minä koskaan, missä olenkin vaeltanut kaikkien israelilaisten keskuudessa, millekään Israelin sukukunnalle, jonka olen asettanut kaitsemaan kansaani Israelia, sanonut näin: Miksi te ette ole rakentaneet minulle setripuista huonetta?
\par 8 Sano siis minun palvelijalleni Daavidille: Näin sanoo Herra Sebaot: Minä olen ottanut sinut laitumelta, lammasten jäljestä, kansani Israelin ruhtinaaksi.
\par 9 Ja minä olen ollut sinun kanssasi kaikkialla, missä sinä vaelsit, ja olen hävittänyt kaikki vihollisesi sinun tieltäsi. Ja minä teen sinulle suuren nimen, suurimpien nimien vertaisen maan päällä.
\par 10 Ja minä valmistan sijan kansalleni Israelille ja istutan sen niin, että se asuu paikallansa eikä enää ole levoton, eivätkä vääryyden tekijät sitä enää sorra niinkuin ennen,
\par 11 siitä ajasta saakka, jolloin minä asetin tuomareita kansalleni Israelille; minä annan sinun päästä rauhaan kaikista vihollisistasi. Ja Herra ilmoittaa sinulle, että Herra on tekevä sinulle huoneen.
\par 12 Kun sinun päiväsi ovat päättyneet ja sinä lepäät isiesi tykönä, korotan minä sinun seuraajaksesi jälkeläisesi, joka lähtee sinun ruumiistasi; ja minä vahvistan hänen kuninkuutensa.
\par 13 Hän on rakentava huoneen minun nimelleni, ja minä vahvistan hänen valtaistuimensa ikuisiksi ajoiksi.
\par 14 Minä olen oleva hänen isänsä ja hän minun poikani, niin että, jos hän tekee väärin, minä rankaisen häntä ihmisvitsalla ja niinkuin ihmislapsia lyödään;
\par 15 mutta minun armoni ei poistu hänestä, niinkuin minä poistin sen Saulista, jonka minä poistin sinun tieltäsi.
\par 16 Ja sinun sukusi ja kuninkuutesi pysyvät sinun edessäsi iäti, ja sinun valtaistuimesi on oleva iäti vahva."
\par 17 Aivan näillä sanoilla ja tämän näyn mukaan Naatan puhui Daavidille.
\par 18 Niin kuningas Daavid meni ja asettui Herran eteen ja sanoi: "Mikä olen minä, Herra, Herra, ja mikä on minun sukuni, että olet saattanut minut tähän asti?
\par 19 Mutta tämäkin on ollut vähän sinun silmissäsi, Herra, Herra, ja niin olet sinä puhunut palvelijasi suvulle myöskin kaukaisista asioista, näin opettaen ihmisten tavalla, Herra, Herra.
\par 20 Mitä Daavid enää sinulle puhuisi? Sinähän tunnet palvelijasi, Herra, Herra.
\par 21 Oman sanasi tähden ja mielesi mukaan sinä olet tehnyt sen, että ilmoitit palvelijallesi kaikki nämä suuret asiat.
\par 22 Sentähden olet sinä, Herra Jumala, suuri; sillä kaiken sen mukaan, mitä me olemme korvillamme kuulleet, ei ole sinun vertaistasi, eikä muuta jumalaa ole kuin sinä.
\par 23 Ja missä on maan päällä toista kansaa sinun kansasi Israelin vertaista, jota Jumala itse on käynyt lunastamaan kansaksensa, tehdäkseen sen nimen kuuluisaksi ja tehdäkseen teille, sinun maallesi, suuria ja peljättäviä tekoja kansasi edessä, jonka sinä lunastit itsellesi Egyptistä, pakanain ja heidän jumaliensa vallasta?
\par 24 Ja sinä olet valmistanut itsellesi kansasi Israelin, omaksi kansaksesi ikuisiksi ajoiksi; ja sinä, Herra, olet tullut heidän Jumalaksensa.
\par 25 Niin täytä nyt, Herra Jumala, ikuisiksi ajoiksi se, mitä olet puhunut palvelijastasi ja hänen suvustansa; tee, niinkuin olet puhunut.
\par 26 Niin sinun nimesi tulee suureksi iankaikkisesti, ja se on oleva: Herra Sebaot, Israelin Jumala. Ja palvelijasi Daavidin suku on pysyvä sinun edessäsi.
\par 27 Sillä sinä, Herra Sebaot, Israelin Jumala, olet ilmoittanut palvelijallesi ja sanonut: 'Minä rakennan sinulle huoneen'. Siitä sinun palvelijasi on saanut rohkeuden rukoilla tämän rukouksen sinun kuullaksesi.
\par 28 Ja nyt, Herra, Herra, sinä olet Jumala, ja sinun sanasi ovat todet; ja kun olet luvannut palvelijallesi tämän hyvän,
\par 29 niin suvaitse nyt siunata palvelijasi sukua, että se pysyisi iäti sinun edessäsi. Sillä sinä, Herra, Herra, olet sen luvannut, ja sinun palvelijasi suku on oleva siunattu sinun siunauksellasi iankaikkisesti."

\chapter{8}

\par 1 Sen jälkeen Daavid voitti filistealaiset ja nöyryytti heidät; ja Daavid otti vallan ohjat filistealaisten käsistä.
\par 2 Hän voitti myös mooabilaiset ja mittasi heitä nuoralla, pantuaan heidät makaamaan maahan: aina kaksi nuoranpituutta hän mittasi surmattavaksi ja yhden nuoranpituuden henkiin jätettäväksi. Niin tulivat mooabilaiset Daavidin veronalaisiksi palvelijoiksi.
\par 3 Samoin Daavid voitti Hadadeserin, Rehobin pojan, Sooban kuninkaan, kun tämä oli menossa ulottamaan valtaansa Eufrat-virtaan.
\par 4 Ja Daavid otti häneltä vangiksi tuhat seitsemänsataa ratsumiestä ja kaksikymmentä tuhatta jalkamiestä ja katkoi kaikilta vaunuhevosilta vuohisjänteet; ainoastaan sata vaunuhevosta hän niistä säästi.
\par 5 Ja kun Damaskon aramilaiset tulivat auttamaan Hadadeseria, Sooban kuningasta, voitti Daavid kaksikymmentä kaksi tuhatta aramilaista.
\par 6 Ja Daavid asetti maaherroja Damaskon Aramiin; ja aramilaiset tulivat Daavidin veronalaisiksi palvelijoiksi. Näin Herra antoi Daavidille voiton, mihin tahansa hän meni.
\par 7 Ja Daavid otti ne kultavarustukset, jotka Hadadeserin palvelijoilla oli, ja vei ne Jerusalemiin.
\par 8 Mutta Hadadeserin kaupungeista, Betahista ja Beerotaista, kuningas Daavid otti sangen paljon vaskea.
\par 9 Kun Tooi, Hamatin kuningas, kuuli, että Daavid oli voittanut Hadadeserin koko sotajoukon,
\par 10 lähetti Tooi poikansa Jooramin kuningas Daavidin luo tervehtimään häntä ja onnittelemaan häntä siitä, että hän oli taistellut Hadadeserin kanssa ja voittanut hänet; sillä Hadadeser oli ollut Tooin vastustaja. Ja hänellä oli mukanaan hopea-, kulta- ja vaskikaluja.
\par 11 Nekin kuningas Daavid pyhitti Herralle, samoin kuin oli pyhittänyt sen hopean ja kullan, minkä oli ottanut kaikilta kukistamiltaan kansoilta:
\par 12 Aramilta, Mooabilta, ammonilaisilta, filistealaisilta ja Amalekilta, sekä sen saaliin, minkä hän oli ottanut Hadadeserilta, Rehobin pojalta, Sooban kuninkaalta.
\par 13 Ja Daavid teki nimensä kuuluisaksi, kun hän palasi takaisin, voitettuaan Suolalaaksossa edomilaiset, kahdeksantoista tuhatta miestä.
\par 14 Sen jälkeen hän asetti maaherroja Edomiin, koko Edomiin hän asetti maaherrat, ja kaikki edomilaiset tulivat Daavidin palvelijoiksi. Näin Herra antoi Daavidille voiton, mihin tahansa tämä meni.
\par 15 Ja Daavid hallitsi koko Israelia, ja Daavid teki kaikelle kansallensa sitä, mikä oikeus ja vanhurskaus on.
\par 16 Jooab, Serujan poika, oli sotajoukon ylipäällikkönä, ja Joosafat, Ahiludin poika, oli kanslerina.
\par 17 Saadok, Ahitubin poika, ja Ahimelek, Ebjatarin poika, olivat pappeina, ja Seraja oli kirjurina.
\par 18 Benaja, Joojadan poika, oli kreettien ja pleettien päällikkönä; Daavidin pojat olivat myös pappeina.

\chapter{9}

\par 1 Daavid kysyi: "Onko Saulin suvusta enää jäljellä ketään, jolle minä voisin tehdä laupeuden Joonatanin tähden?"
\par 2 Saulin perheessä oli ollut palvelija nimeltä Siiba, ja hänet he kutsuivat Daavidin eteen. Ja kuningas sanoi hänelle: "Oletko sinä Siiba?" Hän vastasi: "Palvelijasi on".
\par 3 Kuningas kysyi: "Eikö Saulin suvusta ole enää jäljellä ketään, jolle minä voisin tehdä Jumalan laupeuden?" Siiba vastasi kuninkaalle: "Vielä on Joonatanin poika, joka on rampa jaloistaan".
\par 4 Kuningas kysyi häneltä: "Missä hän on?" Siiba vastasi kuninkaalle: "Hän on Maakirin, Ammielin pojan, talossa Loodebarissa".
\par 5 Niin kuningas Daavid lähetti noutamaan hänet Maakirin, Ammielin pojan, talosta Loodebarista.
\par 6 Kun Mefiboset, Saulin pojan Joonatanin poika, tuli Daavidin tykö, lankesi hän kasvoillensa ja osoitti kunnioitusta. Niin Daavid sanoi: "Mefiboset!" Hän vastasi: "Tässä on palvelijasi".
\par 7 Daavid sanoi hänelle: "Älä pelkää, sillä minä teen sinulle laupeuden isäsi Joonatanin tähden, ja minä palautan sinulle isoisäsi Saulin kaiken maaomaisuuden, ja sinä saat aina aterioida minun pöydässäni".
\par 8 Niin tämä kumarsi ja sanoi: "Mikä on palvelijasi, että sinä käännyt minunlaiseni koiranraadon puoleen?"
\par 9 Sitten kuningas kutsui Siiban, Saulin palvelijan, ja sanoi hänelle: "Kaiken, mikä on ollut Saulin ja koko hänen sukunsa omaa, minä annan sinun herrasi pojalle.
\par 10 Ja sinä poikinesi ja palvelijoinesi viljele hänelle sitä maata ja korjaa sato, että sinun herrasi pojalla olisi leipää syödä; kuitenkin saa Mefiboset, sinun herrasi poika, aina aterioida minun pöydässäni." Ja Siiballa oli viisitoista poikaa ja kaksikymmentä palvelijaa.
\par 11 Siiba sanoi kuninkaalle: "Aivan niinkuin herrani, kuningas, käskee palvelijaansa, on palvelijasi tekevä". - "Niin, Mefiboset on syövä minun pöydässäni, niinkuin hän olisi kuninkaan poikia."
\par 12 Ja Mefibosetilla oli pieni poika nimeltä Miika. Ja koko Siiban talonväki oli Mefibosetin palvelijoita.
\par 13 Ja Mefiboset asui Jerusalemissa, koska hän aina söi kuninkaan pöydässä. Ja hän ontui kumpaakin jalkaansa.

\chapter{10}

\par 1 Senjälkeen ammonilaisten kuningas kuoli, ja hänen poikansa Haanun tuli kuninkaaksi hänen sijaansa.
\par 2 Niin Daavid sanoi: "Minä osoitan laupeutta Haanunille, Naahaan pojalle, niinkuin hänen isänsä osoitti laupeutta minulle". Ja Daavid lähetti palvelijoitaan lohduttamaan häntä hänen isänsä kuoleman johdosta. Kun Daavidin palvelijat tulivat ammonilaisten maahan,
\par 3 sanoivat ammonilaisten päämiehet herrallensa Haanunille: "Luuletko sinä, että Daavid tahtoo kunnioittaa sinun isääsi, kun hän lähettää lohduttajia sinun luoksesi? Varmasti on Daavid lähettänyt palvelijansa sinun luoksesi tutkimaan kaupunkia ja vakoilemaan ja hävittämään sitä."
\par 4 Niin Haanun otatti kiinni Daavidin palvelijat, ajatti toisen puolen heidän partaansa ja leikkautti toisen puolen heidän vaatteistaan, peräpuolia myöten, ja päästi heidät sitten menemään.
\par 5 Ja kun se ilmoitettiin Daavidille, lähetti hän sanan heitä vastaan, sillä miehiä oli pahasti häväisty. Ja kuningas käski sanoa: "Jääkää Jerikoon, kunnes partanne on kasvanut, ja tulkaa sitten takaisin".
\par 6 Kun ammonilaiset näkivät joutuneensa Daavidin vihoihin, lähettivät ammonilaiset lähettiläitä ja palkkasivat Beet-Rehobin ja Sooban aramilaisia, kaksikymmentä tuhatta jalkamiestä, ja Maakan kuninkaalta tuhat miestä ja Toobin miehiä kaksitoista tuhatta.
\par 7 Kun Daavid sen kuuli, lähetti hän Jooabin ja koko sotajoukon, kaikki urhot.
\par 8 Niin ammonilaiset lähtivät ja asettuivat sotarintaan portin oven edustalle, mutta Sooban ja Rehobin aramilaiset ja Toobin ja Maakan miehet olivat eri joukkona kedolla.
\par 9 Kun Jooab näki, että häntä uhkasi hyökkäys edestä ja takaa, valitsi hän miehiä kaikista Israelin valiomiehistä ja asettui sotarintaan aramilaisia vastaan.
\par 10 Mutta muun väen hän antoi veljensä Abisain johtoon, ja tämä asettui sotarintaan ammonilaisia vastaan.
\par 11 Ja hän sanoi: "Jos aramilaiset tulevat minulle ylivoimaisiksi, niin tule sinä minun avukseni; jos taas ammonilaiset tulevat sinulle ylivoimaisiksi, niin minä tulen sinua auttamaan.
\par 12 Ole luja, ja pysykäämme lujina kansamme puolesta ja meidän Jumalamme kaupunkien puolesta. Tehköön sitten Herra, minkä hyväksi näkee."
\par 13 Sitten Jooab ja väki, joka oli hänen kanssaan, ryhtyi taisteluun aramilaisia vastaan, ja nämä pakenivat häntä.
\par 14 Ja kun ammonilaiset näkivät aramilaisten pakenevan, pakenivat hekin Abisaita ja menivät kaupunkiin. Sitten Jooab palasi ahdistamasta ammonilaisia ja tuli Jerusalemiin.
\par 15 Kun aramilaiset näkivät, että Israel oli voittanut heidät, kokoontuivat he kaikki,
\par 16 ja Hadareser lähetti nostattamaan niitä aramilaisia, jotka asuivat tuolla puolella Eufrat-virran; ja he tulivat Heelamiin, ja Soobak, Hadareserin sotapäällikkö, johti heitä.
\par 17 Kun se ilmoitettiin Daavidille, kokosi hän kaiken Israelin ja meni Jordanin yli ja tuli Heelamiin; ja aramilaiset asettuivat sotarintaan Daavidia vastaan ja taistelivat hänen kanssaan.
\par 18 Mutta aramilaiset pakenivat Israelia, ja Daavid tappoi aramilaisilta seitsemän sataa vaunuhevosta ja neljäkymmentä tuhatta ratsumiestä, ja heidän sotapäällikkönsä Soobakin hän siellä löi kuoliaaksi.
\par 19 Kun kaikki Hadareserin alaiset kuninkaat näkivät, että Israel oli voittanut heidät, tekivät he Israelin kanssa rauhan ja palvelivat heitä. Sitten aramilaiset eivät enää uskaltaneet auttaa ammonilaisia.

\chapter{11}

\par 1 Vuoden vaihteessa, kuningasten sotaanlähtöaikana, lähetti Daavid Jooabin ja hänen kanssaan palvelijansa ja koko Israelin sotaretkelle; ja he hävittivät ammonilaisten maata ja piirittivät Rabbaa. Mutta Daavid itse jäi Jerusalemiin.
\par 2 Niin tapahtui eräänä iltana, kun Daavid oli noussut vuoteeltansa ja käveli kuninkaan linnan katolla, että hän katolta näki naisen peseytymässä; ja nainen oli näöltään hyvin ihana.
\par 3 Daavid lähetti hankkimaan tietoja naisesta; ja sanottiin: "Se on Batseba, Eliamin tytär, heettiläisen Uurian vaimo".
\par 4 Niin Daavid lähetti sanansaattajat häntä noutamaan, ja hän tuli hänen luoksensa. Ja Daavid makasi naisen kanssa, joka juuri oli puhdistautunut saastaisuudestaan. Sitten tämä palasi kotiinsa.
\par 5 Ja vaimo tuli raskaaksi; ja hän lähetti ilmoittamaan Daavidille: "Minä olen raskaana".
\par 6 Silloin Daavid lähetti Jooabille sanan: "Lähetä minun luokseni heettiläinen Uuria". Ja Jooab lähetti Uurian Daavidin luo.
\par 7 Ja kun Uuria tuli Daavidin luo, kysyi tämä, kuinka Jooab ja väki voivat ja kuinka sota menestyi.
\par 8 Ja Daavid sanoi Uurialle: "Mene kotiisi ja pese jalkasi". Uuria lähti kuninkaan linnasta, ja kuninkaan lahja seurasi häntä.
\par 9 Mutta Uuria panikin maata kuninkaan linnan oven eteen kaikkien muiden herransa palvelijain joukkoon eikä mennyt kotiinsa.
\par 10 Kun Daavidille ilmoitettiin: "Uuria ei ole mennytkään kotiinsa", sanoi Daavid Uurialle: "Tulethan sinä matkalta, mikset mennyt kotiisi?"
\par 11 Mutta Uuria sanoi Daavidille: "Arkki ja Israel ja Juuda asuvat lehtimajoissa, minun herrani Jooab ja herrani palvelijat ovat leiriytyneinä kedolla; menisinkö minä kotiin syömään, juomaan ja makaamaan vaimoni kanssa? Niin totta kuin sinä elät ja sinun sielusi elää: sitä minä en tee."
\par 12 Niin Daavid sanoi Uurialle: "Jää tänne vielä täksi päiväksi, niin minä huomenna päästän sinut menemään". Niin Uuria jäi Jerusalemiin vielä siksi päiväksi. Seuraavana päivänä
\par 13 Daavid kutsui hänet luoksensa syömään ja juomaan ja juotti hänet juovuksiin. Mutta illalla hän meni maata vuoteelleen herransa palvelijain joukkoon eikä mennyt kotiinsa.
\par 14 Seuraavana aamuna Daavid kirjoitti kirjeen Jooabille ja lähetti sen Uurian mukana.
\par 15 Ja kirjeessä hän kirjoitti näin: "Pankaa Uuria eturintaan, kiivaimpaan taisteluun, ja vetäytykää takaisin hänen luotaan, että hänet lyötäisiin kuoliaaksi".
\par 16 Piirittäessään kaupunkia Jooab asetti Uurian siihen paikkaan, missä tiesi urhoollisimpien miesten olevan.
\par 17 Kun sitten kaupungin miehet ryntäsivät ulos ja ryhtyivät taisteluun Jooabin kanssa, niin väkeä, Daavidin palvelijoita, kaatui, ja myöskin heettiläinen Uuria kuoli.
\par 18 Sitten Jooab lähetti ilmoittamaan Daavidille kaikki, mitä taistelussa oli tapahtunut.
\par 19 Ja hän käski sanansaattajaa sanoen: "Kun olet kertonut kuninkaalle kaikki, mitä sodassa on tapahtunut,
\par 20 niin kuningas ehkä kiivastuu ja sanoo sinulle: 'Miksi menitte niin lähelle kaupunkia taistelemaan? Ettekö tienneet, että ne ampuvat muurilta?
\par 21 Kuka surmasi Abimelekin, Jerubbesetin pojan? Eikö nainen heittänyt muurilta jauhinkiviä hänen päällensä, niin että hän kuoli Teebeksessä? Miksi menitte niin lähelle muuria?' Sano silloin: 'Myöskin sinun palvelijasi heettiläinen Uuria on kuollut'."
\par 22 Sanansaattaja lähti, ja tultuaan hän ilmoitti Daavidille kaiken, mitä Jooab oli lähettänyt hänet ilmoittamaan.
\par 23 Ja sanansaattaja sanoi Daavidille: "Ne miehet olivat meille ylivoimaiset ja ryntäsivät meitä vastaan ulos kedolle, mutta me työnsimme heidät takaisin portin ovelle saakka.
\par 24 Silloin ampujat ampuivat muurilta sinun palvelijoitasi, ja kuninkaan palvelijoita kuoli; myöskin palvelijasi heettiläinen Uuria kuoli."
\par 25 Silloin Daavid sanoi sanansaattajalle: "Sano näin Jooabille: 'Älä pane sitä pahaksesi, sillä miekka syö milloin yhden, milloin toisen; taistele urheasti kaupunkia vastaan ja hävitä se'. Rohkaise häntä näin."
\par 26 Kun Uurian vaimo kuuli, että hänen miehensä Uuria oli kuollut, piti hän puolisolleen valittajaiset.
\par 27 Mutta kun suruaika oli ohitse, lähetti Daavid ottamaan hänet linnaansa, ja hän tuli hänen vaimoksensa. Sitten hän synnytti hänelle pojan. Mutta se, minkä Daavid oli tehnyt, oli paha Herran silmissä.

\chapter{12}

\par 1 Ja Herra lähetti Naatanin Daavidin tykö. Kun hän tuli hänen tykönsä, sanoi hän hänelle: "Kaksi miestä oli samassa kaupungissa, toinen rikas ja toinen köyhä.
\par 2 Rikkaalla oli lampaita ja raavaita hyvin paljon.
\par 3 Mutta köyhällä ei ollut muuta kuin yksi ainoa pieni karitsa, jonka hän oli ostanut. Hän elätti sitä, ja se kasvoi hänen luonansa yhdessä hänen lastensa kanssa: se söi hänen leipäpalastansa, joi hänen maljastansa, makasi hänen sylissään ja oli hänelle niinkuin tytär.
\par 4 Niin rikkaalle miehelle tuli vieras. Mutta hän ei raskinut ottaa omia lampaitansa eikä raavaitansa valmistaakseen ruokaa matkamiehelle, joka oli tullut hänen luoksensa; vaan hän otti köyhän miehen karitsan ja valmisti sen miehelle, joka oli tullut hänen luoksensa."
\par 5 Niin Daavid vihastui kovin siihen mieheen ja sanoi Naatanille: "Niin totta kuin Herra elää: mies, joka tämän on tehnyt, on kuoleman oma.
\par 6 Ja karitsa hänen on korvattava nelinkertaisesti, koska hän teki näin ja koska hän ei sääliä tuntenut."
\par 7 Mutta Naatan sanoi Daavidille: "Sinä olet se mies. Näin sanoo Herra, Israelin Jumala: 'Minä olen voidellut sinut Israelin kuninkaaksi ja pelastanut sinut Saulin käsistä.
\par 8 Minä olen antanut sinulle herrasi linnan ja antanut herrasi vaimot sinun syliisi; ja minä olen antanut sinulle Israelin ja Juudan heimot. Ja jos tämä olisi vähän, niin minä antaisin sinulle vielä sekä sitä että tätä.
\par 9 Miksi sinä olet pitänyt halpana Herran sanan ja tehnyt sitä, mikä on pahaa hänen silmissään? Heettiläisen Uurian sinä olet surmannut miekalla, olet tappanut hänet ammonilaisten miekalla, ja hänen vaimonsa sinä olet ottanut vaimoksesi.
\par 10 Sentähden ei miekka ole milloinkaan väistyvä sinun suvustasi, koska olet pitänyt halpana minut ja ottanut vaimoksesi heettiläisen Uurian vaimon.
\par 11 Näin sanoo Herra: Katso, minä nostatan sinulle onnettomuuden sinun omasta perheestäsi, ja silmiesi edessä minä otan sinun vaimosi ja annan heidät toiselle, ja hän on makaava sinun vaimojesi kanssa tämän auringon nähden.
\par 12 Sillä sinä olet tehnyt tekosi salassa, mutta minä teen tämän koko Israelin ja auringon nähden.'"
\par 13 Niin Daavid sanoi Naatanille: "Minä olen tehnyt syntiä Herraa vastaan". Naatan sanoi Daavidille: "Niin on myös Herra antanut sinun syntisi anteeksi; sinä et kuole.
\par 14 Mutta koska sinä tällä teolla olet saattanut Herran viholliset pilkkaamaan häntä, niin se poika, joka sinulle on syntynyt, on kuoleva."
\par 15 Sitten Naatan meni kotiinsa. Ja Herra löi lasta, jonka Uurian vaimo oli Daavidille synnyttänyt, niin että se sairastui vaikeasti.
\par 16 Silloin Daavid etsi Jumalaa pojan tähden, ja Daavid paastosi; ja aina kun hän tuli kotiinsa, makasi hän yötä paljaalla maalla.
\par 17 Niin hänen hovinsa vanhimmat menivät hänen luokseen saadaksensa hänet nousemaan ylös maasta, mutta hän ei tahtonut; eikä hän syönyt mitään heidän kanssansa.
\par 18 Seitsemäntenä päivänä lapsi kuoli. Mutta Daavidin palvelijat eivät uskaltaneet ilmoittaa hänelle, että lapsi oli kuollut, sillä he ajattelivat: "Katso, kun me puhuimme hänelle lapsen vielä eläessä, ei hän kuullut meitä. Kuinka voisimme sitten sanoa hänelle, että lapsi on kuollut? Hän voisi tehdä itselleen pahaa."
\par 19 Mutta kun Daavid näki, että hänen palvelijansa kuiskailivat keskenänsä, ymmärsi hän, että lapsi oli kuollut. Ja Daavid kysyi palvelijoiltansa: "Onko lapsi kuollut?" He vastasivat: "On".
\par 20 Niin Daavid nousi maasta, peseytyi ja voiteli itsensä, muutti vaatteensa, meni Herran huoneeseen ja rukoili. Ja kun hän tuli kotiinsa, pyysi hän ruokaa; ja he tarjosivat, ja hän söi.
\par 21 Mutta hänen palvelijansa sanoivat hänelle: "Miksi teet näin? Lapsen eläessä sinä paastosit ja itkit, mutta lapsen kuoltua sinä nouset ja syöt."
\par 22 Hän vastasi: "Kun lapsi vielä eli, paastosin minä ja itkin, sillä minä ajattelin: Kenties Herra on minulle armollinen, niin että lapsi jää eloon.
\par 23 Mutta kun hän nyt on kuollut, niin mitä minä enää paastoaisin? Enhän minä enää voi palauttaa häntä. Minä menen hänen tykönsä, mutta hän ei enää palaja minun tyköni."
\par 24 Daavid lohdutti vaimoansa Batsebaa ja meni hänen luoksensa ja makasi hänen kanssansa. Ja tämä synnytti pojan, ja hän antoi tälle nimen Salomo, ja Herra rakasti häntä.
\par 25 Ja Daavid laittoi hänet profeetta Naatanin hoitoon, ja tämä kutsui häntä Jedidjaksi Herran tähden.
\par 26 Mutta Jooab ryhtyi taistelemaan ammonilaisten Rabbaa vastaan ja valloitti kuninkaan kaupungin.
\par 27 Ja Jooab lähetti sanansaattajat Daavidin luo ja käski sanoa: "Minä olen ryhtynyt taisteluun Rabbaa vastaan ja olen jo valloittanut vesikaupungin.
\par 28 Niin kokoa nyt jäljellä oleva väki ja asetu leiriin kaupungin edustalle ja valloita se, etten minä sitä valloittaisi ja ottaisi sitä nimiini."
\par 29 Silloin Daavid kokosi kaiken väen ja lähti Rabbaan, ryhtyi taisteluun sitä vastaan ja valloitti sen.
\par 30 Ja hän otti heidän kuninkaansa kruunun hänen päästänsä - se painoi talentin kultaa, ja siinä oli kallis kivi - ja se pantiin Daavidin päähän. Ja hän vei sangen paljon saalista kaupungista.
\par 31 Ja kansan, mitä siellä oli, hän vei pois ja asetti heitä kivisahojen ja rautahakkujen ja rautakirveiden ääreen ja pani heitä tiilentekoon. Näin hän teki kaikille ammonilaisten kaupungeille. Sitten Daavid ja kaikki väki palasi Jerusalemiin.

\chapter{13}

\par 1 Sen jälkeen tapahtui tämä. Daavidin pojalla Absalomilla oli kaunis sisar nimeltä Taamar, ja Daavidin poika Amnon rakastui häneen.
\par 2 Ja Amnon tuli aivan sairaaksi tuskasta sisarensa Taamarin tähden; sillä kun tämä oli neitsyt, näytti Amnonista mahdottomalta tehdä hänelle mitään.
\par 3 Mutta Amnonilla oli ystävä, nimeltä Joonadab, Daavidin veljen Simean poika; ja Joonadab oli hyvin ovela mies.
\par 4 Tämä sanoi hänelle: "Miksi sinä kuihdut noin päivä päivältä, kuninkaan poika? Etkö ilmaise minulle sitä?" Amnon vastasi hänelle: "Minä olen rakastunut veljeni Absalomin sisareen Taamariin".
\par 5 Joonadab sanoi hänelle: "Paneudu vuoteeseesi ja tekeydy sairaaksi. Ja kun isäsi tulee katsomaan sinua, niin sano hänelle: 'Salli minun sisareni Taamarin tulla antamaan minulle jotakin syötävää; jos hän valmistaisi ruuan minun silmieni edessä, että minä näkisin sen, niin minä söisin hänen kädestään'."
\par 6 Niin Amnon paneutui maata ja tekeytyi sairaaksi. Kun kuningas tuli katsomaan häntä, sanoi Amnon kuninkaalle: "Salli minun sisareni Taamarin tulla tekemään kaksi kakkua minun silmieni edessä, niin minä syön hänen kädestään".
\par 7 Silloin Daavid lähetti sanan Taamarille linnaan ja käski sanoa hänelle: "Mene veljesi Amnonin taloon ja valmista hänelle ruokaa".
\par 8 Niin Taamar meni veljensä Amnonin taloon, jossa tämä makasi. Ja hän otti taikinaa, sotki ja teki kakkuja hänen silmiensä edessä ja paistoi kakut,
\par 9 otti pannun ja kaatoi hänen eteensä; mutta tämä ei tahtonut syödä. Ja Amnon sanoi: "Toimittakaa kaikki minun luotani ulos". Niin menivät kaikki hänen luotaan ulos.
\par 10 Sitten Amnon sanoi Taamarille: "Tuo ruoka tänne sisähuoneeseen, syödäkseni sinun kädestäsi". Niin Taamar otti tekemänsä kakut ja toi ne sisähuoneeseen veljellensä Amnonille.
\par 11 Mutta kun hän tarjosi niitä tälle syödä, tarttui tämä häneen ja sanoi hänelle: "Tule, makaa minun kanssani, sisareni!"
\par 12 Mutta hän sanoi: "Pois se, veljeni, älä tee minulle väkivaltaa; sillä semmoista ei saa tehdä Israelissa. Älä tee sellaista häpeällistä tekoa.
\par 13 Mihin minä joutuisin häpeäni kanssa? Ja sinua pidettäisiin Israelissa houkkana. Puhu nyt kuninkaan kanssa; hän ei minua sinulta kiellä."
\par 14 Mutta tämä ei tahtonut kuulla häntä, vaan voitti hänet, teki hänelle väkivaltaa ja makasi hänet.
\par 15 Mutta sitten valtasi Amnonin ylen suuri vastenmielisyys häntä kohtaan: vastenmielisyys häntä kohtaan oli suurempi kuin rakkaus, jolla hän oli häntä rakastanut. Ja Amnon sanoi hänelle: "Nouse, mene tiehesi".
\par 16 Silloin hän sanoi hänelle: "Eikö tämä rikos, että ajat minut pois, ole vielä suurempi kuin se toinen, jonka olet minulle tehnyt?" Mutta hän ei tahtonut kuulla häntä,
\par 17 vaan huusi nuoren miehen, joka häntä palveli, ja sanoi: "Ajakaa ulos minun luotani tämä, ja lukitse ovi hänen jälkeensä".
\par 18 Ja tytöllä oli yllään pitkäliepeinen, hihallinen ihokas; sillä sellaisiin viittoihin olivat kuninkaan tyttäret puetut neitsyenä ollessaan. Ja kun palvelija oli vienyt hänet ulos ja lukinnut oven hänen jälkeensä,
\par 19 sirotteli Taamar tuhkaa päähänsä ja repäisi pitkäliepeisen, hihallisen ihokkaan, joka hänellä oli yllänsä, pani kätensä päänsä päälle ja meni huutaen lakkaamatta.
\par 20 Hänen veljensä Absalom sanoi hänelle: "Onko veljesi Amnon ollut sinun kanssasi? Vaikene nyt, sisareni, onhan hän veljesi. Älä pane tätä asiaa niin sydämellesi." Niin Taamar jäi hyljättynä veljensä Absalomin taloon.
\par 21 Kun kuningas Daavid kuuli kaiken tämän, vihastui hän kovin.
\par 22 Mutta Absalom ei puhunut Amnonille sanaakaan, ei pahaa eikä hyvää, sillä Absalom vihasi Amnonia, sentähden että tämä oli tehnyt väkivaltaa hänen sisarellensa Taamarille.
\par 23 Kahden vuoden kuluttua tapahtui, että Absalomilla oli lammasten keritsiäiset Baal-Haasorissa, joka on Efraimin luona. Ja Absalom kutsui sinne kaikki kuninkaan pojat.
\par 24 Ja Absalom tuli kuninkaan tykö ja sanoi: "Katso, palvelijallasi on lammasten keritsiäiset; jospa kuningaskin palvelijoineen tulisi palvelijansa luo".
\par 25 Mutta kuningas vastasi Absalomille: "Ei niin, poikani, älkäämme tulko kaikki, ettemme olisi sinulle vaivaksi". Ja vaikka hän pyytämällä pyysi häntä, ei hän kuitenkaan tahtonut mennä, vaan hyvästeli hänet.
\par 26 Silloin Absalom sanoi: "Jollet itse tulekaan, salli kuitenkin veljeni Amnonin tulla meidän kanssamme". Kuningas sanoi hänelle: "Miksi juuri hänen olisi mentävä sinun kanssasi?"
\par 27 Mutta kun Absalom pyytämällä pyysi häntä, päästi hän Amnonin ja kaikki muut kuninkaan pojat menemään hänen kanssansa.
\par 28 Ja Absalom käski palvelijoitaan ja sanoi: "Pitäkää silmällä, milloin Amnonin sydän tulee iloiseksi viinistä, ja kun minä sanon teille: 'Surmatkaa Amnon', niin tappakaa hänet älkääkä peljätkö; sillä minähän olen teitä käskenyt. Olkaa lujat ja urhoolliset."
\par 29 Absalomin palvelijat tekivät Amnonille, niinkuin Absalom oli käskenyt. Silloin kaikki kuninkaan pojat nousivat, istuivat kukin muulinsa selkään ja pakenivat.
\par 30 Heidän vielä ollessaan matkalla tuli Daavidille sanoma: "Absalom on surmannut kaikki kuninkaan pojat, niin ettei heistä ole jäänyt jäljelle ainoatakaan".
\par 31 Silloin kuningas nousi, repäisi vaatteensa ja paneutui maahan maata; ja kaikki hänen palvelijansa seisoivat siinä reväistyin vaattein.
\par 32 Niin Joonadab, Daavidin veljen Simean poika, puhkesi puhumaan ja sanoi: "Älköön herrani luulko, että kaikki nuoret miehet, kuninkaan pojat, ovat tapetut; ainoastaan Amnon on kuollut. Sillä ilme Absalomin kasvoilla on tiennyt pahaa siitä päivästä lähtien, jona Amnon teki väkivaltaa hänen sisarellensa Taamarille.
\par 33 Älköön siis herrani, kuningas, panko sydämelleen sitä puhetta, että muka kaikki kuninkaan pojat olisivat kuolleet; sillä ainoastaan Amnon on kuollut."
\par 34 Mutta Absalom pakeni. Kun palvelija, joka oli tähystäjänä, nosti silmänsä ja katseli, niin katso, paljon väkeä tuli tietä, joka oli hänen takanansa, vuoren kuvetta.
\par 35 Silloin Joonadab sanoi kuninkaalle: "Katso, kuninkaan pojat tulevat. On tapahtunut, niinkuin palvelijasi sanoi."
\par 36 Juuri kun hän oli saanut sen sanotuksi, niin katso, kuninkaan pojat tulivat; ja he korottivat äänensä ja itkivät. Myöskin kuningas ja kaikki hänen palvelijansa itkivät hyvin katkerasti.
\par 37 Mutta Absalom oli paennut ja mennyt Gesurin kuninkaan Talmain, Ammihudin pojan, luo. Ja Daavid suri poikaansa kaiken aikaa.
\par 38 Kun Absalom oli paennut ja tullut Gesuriin, jäi hän sinne kolmeksi vuodeksi.
\par 39 Ja kuningas Daavid ikävöi päästä Absalomin luo, sillä hän oli lohduttautunut Amnonin kuolemasta.

\chapter{14}

\par 1 Kun Jooab, Serujan poika, huomasi, että kuninkaan sydän oli kääntynyt Absalomin puoleen,
\par 2 lähetti Jooab noutamaan Tekoasta taitavan vaimon ja sanoi hänelle: "Ole surevinasi ja pukeudu suruvaatteisiin äläkä voitele itseäsi öljyllä, vaan ole niinkuin vaimo, joka jo kauan on surrut vainajaa.
\par 3 Mene sitten kuninkaan eteen ja puhu hänelle näin." Ja Jooab pani sanat hänen suuhunsa.
\par 4 Niin tekoalainen vaimo meni kuninkaan eteen, lankesi kasvoillensa maahan ja osoitti kunnioitusta ja sanoi: "Auta, kuningas!"
\par 5 Kuningas sanoi hänelle: "Mikä sinun on?" Hän vastasi: "Totisesti, minä olen leskivaimo; mieheni on kuollut.
\par 6 Ja sinun palvelijattarellasi oli kaksi poikaa; he tulivat riitaan keskenänsä kedolla, eikä siellä ollut ketään, joka olisi sovittanut heidän välinsä, ja niin toinen löi toisen kuoliaaksi.
\par 7 Ja katso, koko suku on noussut palvelijatartasi vastaan, ja he sanovat: 'Anna tänne veljensä surmaaja, että me otamme häneltä hengen hänen tapetun veljensä hengestä ja niin hävitämme perillisenkin'. Niin he sammuttaisivat kipinänkin, joka minulla vielä on jäljellä, ettei miehestäni jäisi nimeä eikä jälkeläistä maan päälle."
\par 8 Kuningas sanoi vaimolle: "Mene kotiisi, minä annan käskyn sinusta".
\par 9 Mutta tekoalainen vaimo sanoi kuninkaalle: "Herrani, kuningas, tulkoon tämä rikos minun ja minun isäni perheen kannettavaksi, mutta kuningas ja hänen valtaistuimensa olkoon siitä vapaa".
\par 10 Kuningas sanoi: "Tuo minun eteeni se, joka puhuu sinulle niin, ja hän ei enää sinuun koske".
\par 11 Vaimo sanoi: "Muistakoon kuningas Herraa Jumalaansa, ettei verenkostaja saisi tuottaa vielä suurempaa turmiota, ja ettei minun poikaani tuhottaisi". Silloin hän sanoi: "Niin totta kuin Herra elää: ei hiuskarvaakaan sinun poikasi päästä ole putoava maahan".
\par 12 Mutta vaimo sanoi: "Salli palvelijattaresi puhua vielä sananen herralleni, kuninkaalle". Hän vastasi: "Puhu".
\par 13 Vaimo sanoi: "Miksi sinä ajattelet tehdä juuri samoin Jumalan kansaa vastaan, koskapa kuningas ei salli oman hylkäämänsä tulla takaisin? Noin puhuessaanhan kuningas itse joutuu ikäänkuin syylliseksi.
\par 14 Mehän kuolemme ja olemme niinkuin maahan kaadettu vesi, jota ei voi koota takaisin. Mutta Jumala ei ota pois elämää, vaan sitä hän ajattelee, ettei vain hyljätty joutuisi hänestä erotetuksi.
\par 15 Sentähden minä tulin nyt puhumaan tätä herralleni, kuninkaalle, kun kansa sai minut pelkäämään; silloin ajatteli palvelijattaresi: minä puhun kuninkaalle, ehkä kuningas tekee palvelijattarensa sanan mukaan.
\par 16 Niin, kuningas on kuuleva ja pelastava palvelijattarensa sen miehen kourista, joka tahtoo hävittää sekä minut että minun poikani Jumalan perintöosasta.
\par 17 Ja palvelijattaresi ajatteli: minun herrani, kuninkaan, sana on rauhoittava minut. Sillä herrani, kuningas, on Jumalan enkelin kaltainen, niin että hän kuulee, mikä hyvää on ja mikä pahaa. Ja Herra, sinun Jumalasi, olkoon sinun kanssasi."
\par 18 Kuningas vastasi ja sanoi vaimolle: "Älä salaa minulta mitään, mitä minä sinulta kysyn". Vaimo sanoi: "Herrani, kuningas, puhukoon".
\par 19 Kuningas kysyi: "Eikö Jooabin käsi ole mukanasi kaikessa tässä?" Vaimo vastasi ja sanoi: "Niin totta kuin sinun sielusi, minun herrani, kuningas, elää: ei pääse oikeaan eikä vasempaan siitä, mitä herrani, kuningas, puhuu. Niin, sinun palvelijasi Jooab on käskenyt minua tähän ja pannut kaikki nämä sanat palvelijattaresi suuhun.
\par 20 Antaakseen asialle toisen muodon on palvelijasi Jooab näin tehnyt; mutta herrani on viisas niinkuin Jumalan enkeli ja tietää kaiken, mitä maan päällä tapahtuu."
\par 21 Silloin kuningas sanoi Jooabille: "Katso, tämän minä teen: mene ja tuo takaisin nuorukainen Absalom".
\par 22 Niin Jooab lankesi kasvoilleen maahan, osoitti kunnioitusta ja siunasi kuningasta. Ja Jooab sanoi: "Nyt palvelijasi tietää, että minä olen saanut armon sinun silmiesi edessä, herrani, kuningas, koska kuningas tekee palvelijansa sanan mukaan".
\par 23 Sitten Jooab nousi ja meni Gesuriin ja toi Absalomin Jerusalemiin.
\par 24 Mutta kuningas sanoi: "Hän siirtyköön omaan taloonsa, mutta älköön tulko minun kasvojeni eteen". Niin Absalom siirtyi omaan taloonsa, mutta ei tullut kuninkaan kasvojen eteen.
\par 25 Mutta koko Israelissa ei ollut yhtään niin kaunista miestä kuin Absalom, eikä ketään niin ylisteltyä: kantapäästä kiireeseen asti ei ollut hänessä yhtään virheä.
\par 26 Ja kun hän ajatti hiuksensa - aina vuoden kuluttua hän ajatti ne, sillä ne tulivat hänelle niin raskaiksi, että hänen täytyi ne ajattaa - niin painoivat hänen hiuksensa kaksisataa sekeliä kuninkaan painoa.
\par 27 Ja Absalomille syntyi kolme poikaa ja tytär, jonka nimi oli Taamar; hän oli kaunis nainen.
\par 28 Absalom asui kaksi vuotta Jerusalemissa, tulematta kuninkaan kasvojen eteen.
\par 29 Ja Absalom lähetti sanan Jooabille lähettääkseen hänet kuninkaan luo; mutta tämä ei tahtonut tulla hänen luokseen. Ja hän lähetti vielä toisen kerran, mutta hän ei tahtonut tulla.
\par 30 Silloin hän sanoi palvelijoillensa: "Katsokaa, Jooabin maapalsta on minun maapalstani vieressä, ja hänellä on siinä ohraa; menkää ja sytyttäkää se palamaan". Ja Absalomin palvelijat sytyttivät maapalstan palamaan.
\par 31 Niin Jooab nousi ja meni Absalomin luo taloon ja sanoi hänelle: "Miksi sinun palvelijasi ovat sytyttäneet palamaan maapalstan, joka on minun omani?"
\par 32 Absalom vastasi Jooabille: "Katso, minä lähetin sinulle sanan: Tule tänne, niin minä lähetän sinut kuninkaan luo sanomaan: Mitä varten minä olen tullut kotiin Gesurista? Olisi parempi, jos vielä olisin siellä. Nyt minä tahdon tulla kuninkaan kasvojen eteen; ja jos minussa on vääryys, niin surmatkoon hän minut."
\par 33 Silloin Jooab meni kuninkaan tykö ja kertoi hänelle tämän. Ja hän kutsui Absalomin, ja tämä tuli kuninkaan tykö ja kumartui kasvoilleen maahan kuninkaan eteen. Ja kuningas suuteli Absalomia.

\chapter{15}

\par 1 Mutta sitten tapahtui, että Absalom hankki itsellensä vaunut ja hevoset sekä viisikymmentä miestä, jotka juoksivat hänen edellänsä.
\par 2 Ja Absalomilla oli tapana asettua varhain aamulla sen tien viereen, joka vei portille; ja kun joku oli tulossa kuninkaan eteen saamaan oikeutta riita-asiassaan, kutsui Absalom hänet ja kysyi: "Mistä kaupungista sinä olet?" Kun tämä vastasi: "Palvelijasi on siitä tai siitä Israelin sukukunnasta",
\par 3 sanoi Absalom hänelle: "Asiasi on kyllä hyvä ja oikea, mutta kuninkaan luona ei ole ketään, joka sinua kuulisi".
\par 4 Ja Absalom sanoi vielä: "Jospa minut asetettaisiin tuomariksi maahan! Silloin tulisi minun eteeni jokainen, jolla on riita- tai oikeusasia, ja minä antaisin hänelle oikeuden."
\par 5 Ja kun joku lähestyi kumartaaksensa häntä, ojensi hän kätensä, tarttui häneen ja suuteli häntä.
\par 6 Näin Absalom teki kaikille israelilaisille, jotka tulivat kuninkaan eteen saamaan oikeutta. Ja niin Absalom varasti Israelin miesten sydämet.
\par 7 Neljän vuoden kuluttua sanoi Absalom kerran kuninkaalle: "Salli minun mennä Hebroniin täyttämään lupaus, jonka olen Herralle tehnyt.
\par 8 Sillä asuessani Gesurissa, Aramissa, sinun palvelijasi teki lupauksen ja sanoi: 'Jos Herra antaa minun tulla takaisin Jerusalemiin, niin minä palvelen Herraa'."
\par 9 Kuningas sanoi hänelle: "Mene rauhassa". Niin hän nousi ja meni Hebroniin.
\par 10 Mutta Absalom lähetti vakoojia kaikkiin Israelin sukukuntiin ja sanoi: "Kun kuulette pasunan äänen, niin sanokaa: 'Absalom on tullut kuninkaaksi Hebronissa'".
\par 11 Ja Absalomin kanssa oli Jerusalemista kutsuvieraina mennyt kaksisataa miestä, jotka menivät sinne aivan viattomina, tietämättä mitään koko asiasta.
\par 12 Ja Absalom lähetti myös, sillä aikaa kun uhrasi teurasuhrejaan, noutamaan giilolaisen Ahitofelin, Daavidin neuvonantajan, hänen kaupungistaan Giilosta. Ja salaliitto vahvistui, ja yhä enemmän kansaa meni Absalomin puolelle.
\par 13 Niin tultiin Daavidin luo ja ilmoitettiin: "Israelin miesten sydämet ovat kääntyneet Absalomin puolelle".
\par 14 Silloin Daavid sanoi kaikille palvelijoillensa, jotka olivat hänen luonaan Jerusalemissa: "Nouskaa, paetkaamme, sillä meillä ei ole muuta pelastusta Absalomin käsistä. Lähtekää kiiruusti, ettei hän äkkiä saavuttaisi meitä ja saattaisi onnettomuutta meille ja surmaisi kaupungin asukkaita miekan terällä."
\par 15 Kuninkaan palvelijat vastasivat kuninkaalle: "Tehtäköön aivan niinkuin herrani, kuningas, harkitsee parhaaksi. Me olemme sinun palvelijoitasi."
\par 16 Ja kuningas lähti, ja koko hänen hovinsa seurasi häntä; kuitenkin jätti kuningas kymmenen sivuvaimoaan jäljelle vartioimaan linnaa.
\par 17 Niin kuningas lähti, ja kaikki väki seurasi häntä. He pysähtyivät kaukaisimman talon kohdalle
\par 18 kaikkien hänen palvelijainsa kulkiessa hänen ohitsensa. Myös kaikki kreetit ja pleetit sekä kaikki gatilaiset, kuusisataa miestä, jotka olivat tulleet hänen jäljessään Gatista, kulkivat kuninkaan ohitse.
\par 19 Niin kuningas sanoi gatilaiselle Ittaille: "Miksi sinäkin tulet meidän kanssamme? Käänny takaisin ja jää kuninkaan luo, sillä sinä olet vieras, vieläpä maanpaossa kotoasi.
\par 20 Eilen sinä tulit; ottaisinko minä jo tänä päivänä sinut harhailemaan meidän kanssamme matkallamme, kun minä itsekin menen, minne menenkin? Käänny takaisin ja vie veljesi mukanasi. Armo ja uskollisuus sinulle!"
\par 21 Mutta Ittai vastasi kuninkaalle ja sanoi: "Niin totta kuin Herra elää, ja niin totta kuin herrani, kuningas, elää: missä paikassa herrani, kuningas, on, siellä tahtoo myös palvelijasi olla, olkoonpa se elämäksi tai kuolemaksi".
\par 22 Daavid sanoi Ittaille: "Tule sitten ja kulje minun ohitseni". Niin gatilainen Ittai kulki ohitse ja kaikki hänen miehensä ja kaikki vaimot ja lapset, jotka olivat hänen kanssaan.
\par 23 Ja koko maa itki ääneen, kun kaikki väki kulki ohitse. Ja kun kuningas oli kulkemassa Kidronin laakson poikki, kaiken väen kulkiessa erämaan tielle päin,
\par 24 niin katso, siinä oli myös Saadok ynnä hänen kanssaan kaikki leeviläiset, jotka kantoivat Jumalan liitonarkkia. Mutta he laskivat Jumalan arkin maahan, ja Ebjatar uhrasi, kunnes kaikki väki kaupungista oli ehtinyt kulkea ohitse.
\par 25 Niin kuningas sanoi Saadokille: "Vie Jumalan arkki takaisin kaupunkiin. Jos minä saan armon Herran silmien edessä, tuo hän minut takaisin ja antaa minun vielä nähdä hänet ja hänen asumuksensa.
\par 26 Mutta jos hän sanoo näin: 'Minä en ole mielistynyt sinuun', niin katso, hän tehköön minulle, minkä hyväksi näkee."
\par 27 Ja kuningas sanoi pappi Saadokille: "Sinähän olet näkijä; palaa rauhassa kaupunkiin, sinä ja sinun poikasi Ahimaas ja Ebjatarin poika Joonatan, molemmat poikanne, teidän kanssanne.
\par 28 Katsokaa, minä viivyn kahlauspaikoissa erämaassa, kunnes teiltä tulee sana minun tiedokseni."
\par 29 Niin Saadok ja Ebjatar veivät Jumalan arkin takaisin Jerusalemiin ja jäivät sinne.
\par 30 Mutta Daavid kulki itkien Öljymäkeä ylös, pää peitettynä ja paljain jaloin. Ja kaikki väki, mikä oli hänen kanssansa, oli myös peittänyt päänsä, ja he kulkivat itkien lakkaamatta.
\par 31 Ja kun Daavidille kerrottiin, että myös Ahitofel oli salaliittolaisten joukossa Absalomin puolella, sanoi Daavid: "Herra, käännä Ahitofelin neuvo hulluudeksi".
\par 32 Kun sitten Daavid oli tullut vuoren laelle, jossa Jumalaa rukoiltiin, tuli arkilainen Huusai häntä vastaan, ihokas reväistynä ja multaa pään päällä.
\par 33 Daavid sanoi hänelle: "Jos sinä tulet minun kanssani, niin olet minulle vain kuormaksi.
\par 34 Mutta jos palaat kaupunkiin ja sanot Absalomille: 'Minä tahdon olla sinun palvelijasi, kuningas. Minä olen ennen ollut sinun isäsi palvelija, mutta nyt tahdon olla sinun palvelijasi', niin voit olla minulle avuksi tekemällä Ahitofelin neuvon tyhjäksi.
\par 35 Siellähän ovat sinun kanssasi myöskin papit, Saadok ja Ebjatar; kaikki, mitä kuulet kuninkaan linnasta, ilmoita papeille, Saadokille ja Ebjatarille.
\par 36 Katso, heillä on siellä mukanaan myös molemmat poikansa, Saadokilla Ahimaas ja Ebjatarilla Joonatan; näiden kautta voitte lähettää minulle tiedon kaikesta, mitä kuulette."
\par 37 Niin Huusai, Daavidin ystävä, meni kaupunkiin, juuri kun Absalom tuli Jerusalemiin.

\chapter{16}

\par 1 Kun Daavid oli kulkenut vähän matkaa vuoren laelta, tuli häntä vastaan Siiba, Mefibosetin palvelija, mukanaan satuloitu aasipari, jonka selässä oli kaksisataa leipää, sata rusinakakkua, sata hedelmää ja leili viiniä.
\par 2 Niin kuningas sanoi Siiballe: "Mitä sinä näillä teet?" Siiba vastasi: "Aasit ovat kuninkaan perheelle ratsastettaviksi, leipä ja hedelmät palvelijoille syötäviksi ja viini uupuneiden juotavaksi erämaassa".
\par 3 Kuningas sanoi: "Mutta missä sinun herrasi poika on?" Siiba vastasi kuninkaalle: "Hän jäi Jerusalemiin; sillä hän ajatteli: 'Nyt Israelin heimo antaa minulle takaisin isäni kuninkuuden'".
\par 4 Niin kuningas sanoi Siiballe: "Katso, kaikki, mitä Mefibosetilla on, tulee sinun omaksesi". Siiba vastasi: "Minä kumarran; suo minun saada armo sinun silmiesi edessä, herrani, kuningas".
\par 5 Kun kuningas Daavid tuli Bahurimiin, niin katso, sieltä tuli mies, joka oli sukua Saulin perheelle, nimeltä Siimei, Geeran poika; hän tuli ja kiroili tullessaan.
\par 6 Ja hän viskeli kivillä Daavidia ja kaikkia kuningas Daavidin palvelijoita, vaikka kaikki väki ja kaikki urhot olivat hänen oikealla ja vasemmalla puolellaan.
\par 7 Ja kiroillessaan häntä Siimei sanoi näin: "Pois, pois, sinä murhamies, sinä kelvoton!
\par 8 Herra kostaa sinulle kaiken Saulin perheen veren, hänen, jonka sijaan sinä olet tullut kuninkaaksi. Ja Herra antaa nyt kuninkuuden sinun poikasi Absalomin käteen. Ja katso, sinä olet itse joutunut onnettomuuteen, sillä murhamies sinä olet."
\par 9 Niin Abisai, Serujan poika, sanoi kuninkaalle: "Miksi tuo koiranraato saa kiroilla herraani, kuningasta? Anna minun mennä ja lyödä häneltä pää poikki."
\par 10 Mutta kuningas vastasi: "Mitä teillä on minun kanssani tekemistä, te Serujan pojat? Jos hän kiroilee ja jos Herra on käskenyt häntä: 'Kiroile Daavidia', niin kuka voi sanoa: 'Miksi teet niin?'"
\par 11 Ja Daavid sanoi vielä Abisaille ja kaikille palvelijoillensa: "Katso, oma poikani, joka on lähtenyt minun ruumiistani, väijyy henkeäni; miksi ei sitten tämä benjaminilainen? Antakaa hänen olla, kiroilkoon vain; sillä Herra on häntä käskenyt.
\par 12 Ehkä Herra vielä näkee minun kurjuuteni, ja ehkä Herra maksaa minulle hyvällä sen kirouksen, joka tänä päivänä kohtaa minua."
\par 13 Ja Daavid kulki miehinensä tietä myöten, ja Siimei kulki pitkin vuoren kuvetta rinnan hänen kanssansa, kiroili kulkiessaan ja viskeli kiviä ja heitteli soraa.
\par 14 Kun kuningas ja kaikki väki, joka oli hänen kanssansa, oli tullut Ajefimiin, hengähti hän siellä.
\par 15 Mutta Absalom ja kaikki kansa, Israelin miehet, olivat tulleet Jerusalemiin; ja Ahitofel oli hänen kanssansa.
\par 16 Kun arkilainen Huusai, Daavidin ystävä, tuli Absalomin luo, sanoi Huusai Absalomille: "Eläköön kuningas! Eläköön kuningas!"
\par 17 Absalom sanoi Huusaille: "Tällainenko on rakkautesi ystävääsi kohtaan? Miksi et mennyt ystäväsi kanssa?"
\par 18 Huusai vastasi Absalomille: "Ei, vaan kenen Herra ja tämä kansa ja kaikki Israelin miehet ovat valinneet, sen puolella minä olen, ja sen luokse minä jään.
\par 19 Ja toisekseen: ketä minä palvelisin? Enkö hänen poikaansa? Niinkuin olen palvellut sinun isääsi, niin tahdon palvella sinuakin."
\par 20 Sitten Absalom sanoi Ahitofelille: "Antakaa nyt neuvo, mitä meidän on tehtävä".
\par 21 Ahitofel vastasi Absalomille: "Mene isäsi sivuvaimojen tykö, jotka hän on jättänyt linnaa vartioimaan. Silloin koko Israel saa kuulla, että sinä olet saattanut itsesi isäsi vihoihin, ja kaikki, jotka ovat sinun kanssasi, saavat rohkeutta."
\par 22 Niin Absalomille pystytettiin teltta katolle; ja Absalom meni isänsä sivuvaimojen tykö koko Israelin nähden.
\par 23 Ahitofelin antama neuvo oli siihen aikaan niinkuin Jumalalta saatu vastaus: sen arvoinen oli jokainen Ahitofelin neuvo sekä Daavidille että Absalomille.

\chapter{17}

\par 1 Ja Ahitofel sanoi Absalomille: "Salli minun valita kaksitoista tuhatta miestä ja nousta tänä yönä ajamaan takaa Daavidia,
\par 2 niin minä karkaan hänen kimppuunsa, kun hän on väsynyt ja hänen kätensä ovat herpoutuneet, ja saatan hänet kauhun valtaan, niin että kaikki väki, mitä hänen kanssansa on, pakenee. Sitten minä surmaan kuninkaan yksinänsä.
\par 3 Ja minä palautan kaiken kansan sinun luoksesi. Kaikkien palaaminen riippuu siitä miehestä, jota sinä etsit; kaikki kansa saa rauhan."
\par 4 Tämä miellytti Absalomia ja kaikkia Israelin vanhimpia.
\par 5 Kuitenkin sanoi Absalom: "Kutsukaa myös arkilainen Huusai, kuullaksemme, mitä hänellä on sanottavana".
\par 6 Ja kun Huusai tuli Absalomin tykö, sanoi Absalom hänelle: "Niin ja niin puhui Ahitofel. Onko meidän tehtävä hänen sanansa mukaan? Jollei, niin puhu sinä."
\par 7 Huusai vastasi Absalomille: "Ei ole hyvä se neuvo, jonka Ahitofel tällä kertaa on antanut".
\par 8 Ja Huusai sanoi vielä: "Sinä tiedät, että isäsi ja hänen miehensä ovat urhoja ja julmistuneita niinkuin karhu kedolla, jolta poikaset on riistetty; myöskin on isäsi sotilas, joka väkineen ei lepää yöllä.
\par 9 Katso, hän on nyt piiloutunut johonkin onkaloon tai muuhun paikkaan. Jos jo alussa meikäläisiä kaatuisi ja se saataisiin kuulla, niin sanottaisiin: 'Väelle, joka seurasi Absalomia, on tullut tappio'.
\par 10 Silloin urhoollisinkin, jolla on sydän kuin leijonalla, menettäisi rohkeutensa; sillä koko Israel tietää, että sinun isäsi on sankari ja että ne, jotka ovat hänen kanssansa, ovat urhoollisia miehiä.
\par 11 Sentähden on minun neuvoni tämä: Koottakoon sinun luoksesi koko Israel Daanista aina Beersebaan asti, että niitä tulee niin paljon kuin hiekkaa meren rannalla; ja lähde sinä itsekin taisteluun.
\par 12 Kun me sitten karkaamme hänen kimppuunsa, missä ikinä hänet tavataan, niin me painumme hänen päällensä, niinkuin kaste laskeutuu maahan. Eikä hänestä ja kaikista niistä miehistä, jotka ovat hänen kanssaan, jää jäljelle ainoatakaan.
\par 13 Ja jos hän vetäytyisi johonkin kaupunkiin, niin koko Israel asettaisi köydet sen kaupungin ympärille, ja me kiskoisimme sen alas laaksoon, kunnes siitä ei löytyisi kiveäkään."
\par 14 Silloin sanoivat Absalom ja kaikki Israelin miehet: "Arkilaisen Huusain neuvo on parempi kuin Ahitofelin neuvo". Sillä Herra oli säätänyt sen näin, tehdäkseen tyhjäksi Ahitofelin hyvän neuvon, että Herra saattaisi Absalomin onnettomuuteen.
\par 15 Ja Huusai sanoi papeille, Saadokille ja Ebjatarille: "Sen ja sen neuvon on Ahitofel antanut Absalomille ja Israelin vanhimmille, ja sen ja sen neuvon olen minä antanut.
\par 16 Niin lähettäkää nyt pian ja ilmoittakaa Daavidille tämä sana: 'Älä jää yöksi kahlauspaikkoihin erämaahan, vaan mene joen yli, ettei kuningas ja kaikki väki, joka on hänen kanssaan, joutuisi perikatoon'."
\par 17 Ja Joonatan ja Ahimaas olivat asettuneet Roogelin lähteelle, ja eräs palvelijatar kuljetti heille sinne sanaa; sitten he aina menivät ja veivät sanan kuningas Daavidille. Sillä he eivät uskaltaneet näyttäytyä menemällä kaupunkiin.
\par 18 Mutta eräs poikanen näki heidät ja ilmoitti Absalomille. Silloin he molemmat menivät kiiruusti pois ja tulivat erään miehen taloon Bahurimiin. Hänellä oli kaivo pihallaan, ja he laskeutuivat siihen.
\par 19 Ja hänen vaimonsa otti peitteen ja levitti sen kaivon suulle ja sirotti jyviä sen päälle, niin ettei voitu huomata mitään.
\par 20 Kun Absalomin palvelijat tulivat vaimon luo taloon, kysyivät he: "Missä ovat Ahimaas ja Joonatan?" Vaimo vastasi heille: "Tuosta he menivät vesilätäkön yli". He etsivät, ja kun eivät heitä löytäneet, palasivat he takaisin Jerusalemiin.
\par 21 Ja niin pian kuin he olivat menneet, nousivat toiset kaivosta, menivät ja veivät sanan kuningas Daavidille; he sanoivat Daavidille: "Nouskaa ja menkää joutuin veden yli, sillä sen ja sen neuvon on Ahitofel antanut teidän turmioksenne".
\par 22 Silloin Daavid ja kaikki väki, mikä oli hänen kanssaan, nousi, ja he menivät Jordanin yli. Ja kun aamu valkeni, ei ollut ainoatakaan, joka ei olisi tullut Jordanin yli.
\par 23 Mutta kun Ahitofel näki, ettei hänen neuvoansa noudatettu, satuloi hän aasinsa, nousi ja lähti kotiinsa omaan kaupunkiinsa, ja kun hän oli toimittanut talonsa, hirttäytyi hän; niin hän kuoli, ja hänet haudattiin isänsä hautaan.
\par 24 Daavid oli tullut Mahanaimiin, kun Absalom ja kaikki Israelin miehet hänen kanssaan menivät Jordanin yli.
\par 25 Mutta Absalom oli asettanut Amasan Jooabin sijaan sotajoukon ylipäälliköksi. Ja Amasa oli Jitra nimisen jisreeliläisen miehen poika; tämä mies oli yhtynyt Abigailiin, Naahaan tyttäreen, Jooabin äidin Serujan sisareen.
\par 26 Ja Israel ja Absalom leiriytyivät Gileadin maahan.
\par 27 Kun Daavid tuli Mahanaimiin, niin Soobi, Naahaan poika, ammonilaisten Rabbasta, ja Maakir, Ammielin poika, Loodebarista, ja gileadilainen Barsillai, Roogelimista,
\par 28 toivat sinne vuoteita, vateja ja saviastioita, nisuja, ohria, jauhoja ja paahdettuja jyviä, papuja, herneitä
\par 29 sekä hunajaa, voita, lampaita ja juustoja ruuaksi Daavidille ja väelle, mikä oli hänen kanssansa; sillä he ajattelivat: "Väki on nälissään, uuvuksissa ja janoissaan erämaassa".

\chapter{18}

\par 1 Ja Daavid piti mukanaan olevan väen katselmuksen, ja asetti sille tuhannen- ja sadanpäämiehet.
\par 2 Sitten Daavid lähetti väen liikkeelle: yhden kolmanneksen Jooabin johdolla, toisen kolmanneksen Abisain, Serujan pojan, Jooabin veljen, johdolla ja kolmannen kolmanneksen gatilaisen Ittain johdolla. Ja kuningas sanoi kansalle: "Minä lähden myös itse teidän kanssanne".
\par 3 Mutta kansa vastasi: "Älä lähde; sillä jos me pakenemme, ei meistä välitetä, ja jos puolet meistä saa surmansa, ei meistä silloinkaan välitetä, mutta sinä olet niinkuin kymmenen tuhatta meistä. Sentähden on nyt parempi, että sinä olet valmiina tulemaan meille avuksi kaupungista."
\par 4 Niin kuningas sanoi heille: "Minä teen, minkä te hyväksi näette". Ja kuningas asettui portin pieleen, ja kaikki väki lähti liikkeelle sadan ja tuhannen joukkoina.
\par 5 Mutta kuningas käski Jooabia, Abisaita ja Ittaita ja sanoi: "Pidelkää minulle mieliksi hellävaroen nuorukaista Absalomia". Ja kaikki kansa kuuli, kuinka kuningas antoi kaikille päälliköille käskyn Absalomista.
\par 6 Niin väki lähti kentälle Israelia vastaan, ja taistelu tapahtui Efraimin metsässä.
\par 7 Siellä Daavidin palvelijat voittivat Israelin väen, ja siellä oli sinä päivänä suuri mieshukka: kaksikymmentä tuhatta miestä.
\par 8 Ja taistelu levisi koko siihen seutuun; ja metsä söi sinä päivänä enemmän väkeä kuin miekka.
\par 9 Ja Absalom sattui yhteen Daavidin palvelijain kanssa. Absalom ratsasti muulilla; ja kun muuli tuli suuren, tiheäoksaisen tammen alle, tarttui hän päästään tammeen, niin että hän jäi riippumaan taivaan ja maan välille, kun muuli juoksi pois hänen altansa.
\par 10 Sen näki eräs mies ja ilmoitti sen Jooabille ja sanoi: "Katso, minä näin Absalomin riippuvan tammessa".
\par 11 Niin Jooab sanoi miehelle, joka ilmoitti hänelle tämän: "Jos näit sen, miksi et lyönyt häntä siinä maahan? Minun olisi ollut annettava sinulle kymmenen hopeasekeliä ja vyö."
\par 12 Mutta mies vastasi Jooabille: "Vaikka käsiini punnittaisiin tuhat hopeasekeliä, en sittenkään kävisi käsiksi kuninkaan poikaan, sillä kuningas käski meidän kuultemme sinua, Abisaita ja Ittaita sanoen: 'Pitäkää vaari nuorukaisesta Absalomista'.
\par 13 Vai kavaltaisinko minä oman henkeni? Sillä eihän mikään pysy salassa kuninkaalta, ja sinä pysyisit kyllä syrjässä."
\par 14 Niin Jooab sanoi: "Minä en enää kuluta aikaa sinun kanssasi". Sitten hän otti kolme keihästä käteensä ja pisti ne Absalomin rintaan, kun hän vielä eli tammessa.
\par 15 Ja kymmenen nuorta miestä, Jooabin aseenkantajaa, astui Absalomin luo, ja he löivät hänet kuoliaaksi.
\par 16 Sitten Jooab puhalsi pasunaan; ja väki palasi ajamasta takaa Israelia, kun Jooab pysähdytti väen.
\par 17 Ja he ottivat Absalomin ja heittivät hänet metsässä suureen kuoppaan ja pystyttivät sangen suuren kiviroukkion hänen päällensä. Mutta koko Israel pakeni, kukin majallensa.
\par 18 Mutta Absalom oli eläessään hankkinut ja pystyttänyt itsellensä patsaan, joka on Kuninkaanlaaksossa; sillä hän sanoi: "Minulla ei ole poikaa, joka säilyttäisi minun nimeni muiston". Patsaan hän oli kutsunut nimensä mukaan, ja sen nimi on vielä tänäkin päivänä Absalomin muistomerkki.
\par 19 Ja Ahimaas, Saadokin poika, sanoi: "Minä juoksen saattamaan kuninkaalle sen ilosanoman, että Herra on auttanut hänet hänen vihollistensa käsistä oikeuteensa".
\par 20 Mutta Jooab sanoi hänelle: "Tänä päivänä et ole ilosanoman saattaja; jonakin muuna päivänä saattanet ilosanoman, mutta tänä päivänä et saata ilosanomaa, sillä onhan kuninkaan poika kuollut".
\par 21 Sitten Jooab sanoi eräälle etiopialaiselle: "Mene ja ilmoita kuninkaalle, mitä olet nähnyt". Niin etiopialainen kumarsi Jooabille ja lähti juoksemaan.
\par 22 Mutta Ahimaas, Saadokin poika, sanoi taas Jooabille: "Tulkoon mitä tahansa, mutta minä juoksen etiopialaisen jälkeen". Jooab sanoi: "Miksi sinä juoksisit, poikani, eihän sinulle siitä ilosanomansaattajan palkkaa tule?"
\par 23 "Tulkoon mitä tahansa, mutta minä juoksen." Silloin hän sanoi hänelle: "Juokse sitten". Niin Ahimaas lähti juoksemaan Lakeuden tietä ja sivuutti etiopialaisen.
\par 24 Daavid istui molempien porttien välissä. Ja tähystäjä meni portin katolle muurin päälle; kun hän nosti silmänsä ja katseli, niin katso: mies tuli juosten yksinänsä.
\par 25 Tähystäjä huusi ja ilmoitti sen kuninkaalle. Niin kuningas sanoi: "Jos hän on yksin, niin hänellä on ilosanoma". Ja hän tuli yhä lähemmäksi.
\par 26 Sitten tähystäjä näki toisen miehen tulevan juosten; ja hän huusi porttiin sanoen: "Minä näen vielä toisen miehen tulevan juosten yksinänsä". Kuningas sanoi: "Sekin saattaa ilosanomaa".
\par 27 Ja tähystäjä sanoi: "Sen ensimmäisen juoksu näyttää minusta Ahimaasin, Saadokin pojan, juoksulta". Niin kuningas sanoi: "Se on hyvä mies; hän tulee hyviä sanomia tuoden".
\par 28 Ja Ahimaas huusi ja sanoi kuninkaalle: "Rauha!" Sitten hän kumartui kasvoillensa maahan kuninkaan eteen ja sanoi: "Kiitetty olkoon Herra, sinun Jumalasi, joka on antanut sinun käsiisi ne miehet, jotka nostivat kätensä herraani, kuningasta, vastaan".
\par 29 Niin kuningas kysyi: "Voiko nuorukainen Absalom hyvin?" Ahimaas vastasi: "Minä näin suuren väkijoukon, kun kuninkaan palvelija Jooab lähetti minut, palvelijasi; mutta en tiedä, mitä se oli".
\par 30 Kuningas sanoi: "Astu syrjään ja asetu sinne". Niin hän astui syrjään ja jäi seisomaan sinne.
\par 31 Ja katso, etiopialainen tuli perille, ja etiopialainen sanoi: "Ilosanoma herralleni, kuninkaalle: Herra on tänä päivänä auttanut sinut oikeuteesi kaikkien niiden käsistä, jotka ovat nousseet sinua vastaan".
\par 32 Kuningas kysyi etiopialaiselta: "Voiko nuorukainen Absalom hyvin?" Etiopialainen vastasi: "Käyköön herrani kuninkaan vihollisille ja kaikille, jotka nousevat sinua vastaan tehdäksensä sinulle pahaa, niinkuin on käynyt sille nuorukaiselle".
\par 33 Kuningas tuli kovin järkytetyksi, nousi portin päällä olevaan yläsaliin ja itki. Ja mennessänsä hän huusi näin: "Poikani Absalom, minun poikani, oma poikani Absalom! Jospa minä olisin kuollut sinun sijastasi! Absalom, poikani, oma poikani!"

\chapter{19}

\par 1 Jooabille ilmoitettiin: "Katso, kuningas itkee ja suree Absalomia".
\par 2 Ja voitto muuttui sinä päivänä suruksi kaikelle kansalle, koska kansa sinä päivänä kuuli sanottavan, että kuningas oli murheissaan poikansa tähden.
\par 3 Ja kansa tuli sinä päivänä kaupunkiin ikäänkuin varkain, niinkuin tulee varkain väki, joka on häväissyt itsensä pakenemalla taistelusta.
\par 4 Mutta kuningas oli peittänyt kasvonsa, ja hän huusi kovalla äänellä: "Poikani Absalom! Absalom, poikani, oma poikani!"
\par 5 Niin Jooab meni sisälle kuninkaan luo ja sanoi: "Sinä olet tänä päivänä nostanut häpeän kaikkien palvelijaisi kasvoille, vaikka he tänä päivänä ovat pelastaneet sekä sinun oman henkesi että sinun poikiesi, tyttäriesi, vaimojesi ja sivuvaimojesi hengen.
\par 6 Sinähän rakastat niitä, jotka sinua vihaavat, ja vihaat niitä, jotka sinua rakastavat. Sillä nyt sinä olet antanut tietää, ettei sinulla olekaan päälliköitä eikä palvelijoita; - niin, nyt minä tiedän, että jos Absalom olisi elossa ja me kaikki olisimme tänään kuolleet, se olisi sinulle mieleen.
\par 7 Mutta nouse nyt ylös ja mene ulos ja puhuttele ystävällisesti palvelijoitasi. Sillä minä vannon Herran kautta: jollet mene ulos, niin ei totisesti yhtä ainoatakaan miestä jää luoksesi tänä yönä; ja tämä on oleva sinulle suurempi onnettomuus kuin mikään muu, joka on kohdannut sinua nuoruudestasi tähän päivään saakka."
\par 8 Silloin kuningas nousi ja asettui istumaan porttiin. Ja kaikelle kansalle ilmoitettiin ja sanottiin: "Kuningas istuu nyt portissa". Silloin kaikki kansa tuli kuninkaan eteen. Mutta kun Israel oli paennut, kukin majallensa,
\par 9 alkoi kaikki kansa kaikkien Israelin heimojen keskuudessa riidellä ja sanoa: "Kuningas on pelastanut meidät vihollistemme kourista, hän on vapauttanut meidät filistealaisten kourista, ja nyt hän on Absalomia paossa, maasta poissa.
\par 10 Mutta Absalom, jonka me olemme voidelleet kuninkaaksemme, on kuollut taistelussa. Miksi ette nyt ryhdy tuomaan kuningasta takaisin?"
\par 11 Mutta kuningas Daavid lähetti sanan papeille Saadokille ja Ebjatarille ja käski sanoa: "Puhukaa Juudan vanhimmille ja sanokaa: 'Miksi te olette viimeiset tuomaan kuningasta takaisin hänen linnaansa? Sillä se, mitä koko Israel puhuu, on tullut kuninkaan tietoon hänen olinpaikkaansa.
\par 12 Te olette minun veljiäni, minun luutani ja lihaani. Miksi te olisitte viimeiset tuomaan kuningasta takaisin?'
\par 13 Ja Amasalle sanokaa: 'Olethan sinä minun luutani ja lihaani. Jumala rangaiskoon minua nyt ja vasta, jollet sinä kaikkea elinaikaasi ole minun sotapäällikköni Jooabin sijassa.'"
\par 14 Näin hän taivutti kaikkien Juudan miesten sydämet, niin että he yhtenä miehenä lähettivät sanomaan kuninkaalle: "Palaa takaisin, sinä ja kaikki sinun palvelijasi".
\par 15 Niin kuningas palasi takaisin ja tuli Jordanille; mutta Juuda oli tullut Gilgaliin mennäkseen kuningasta vastaan ja tuodakseen kuninkaan Jordanin yli.
\par 16 Myöskin Siimei, Geeran poika, benjaminilainen Bahurimista, lähti kiiruusti Juudan miesten kanssa kuningas Daavidia vastaan.
\par 17 Ja hänen kanssansa oli tuhat miestä Benjaminista sekä Siiba, Saulin perheen palvelija, viidentoista poikansa ja kahdenkymmenen palvelijansa kanssa. Nämä olivat rientäneet Jordanille ennen kuningasta
\par 18 ja menneet yli kahlauspaikasta tuomaan kuninkaan perhettä ja toimittamaan, mitä hän näkisi hyväksi määrätä. Mutta Siimei, Geeran poika, heittäytyi kuninkaan eteen, kun hän oli lähtemässä Jordanin yli,
\par 19 ja sanoi kuninkaalle: "Älköön herrani lukeko minulle pahaa tekoani älköönkä muistelko, kuinka väärin palvelijasi teki sinä päivänä, jona herrani, kuningas, lähti Jerusalemista; älköön kuningas panko sitä sydämellensä.
\par 20 Sillä palvelijasi tuntee, että olen tehnyt synnin; sentähden minä olen tänä päivänä ensimmäisenä koko Joosefin heimosta tullut tänne herraani, kuningasta, vastaan."
\par 21 Niin Abisai, Serujan poika, puuttui puheeseen ja sanoi: "Eikö Siimei ole rangaistava kuolemalla siitä, että hän on kiroillut Herran voideltua?"
\par 22 Mutta Daavid vastasi: "Mitä teillä on minun kanssani tekemistä, te Serujan pojat, kun tulette minulle tänä päivänä kiusaajaksi? Olisiko ketään Israelissa tänä päivänä kuolemalla rangaistava? Minä tiedän tulleeni tänä päivänä Israelin kuninkaaksi."
\par 23 Ja kuningas sanoi Siimeille: "Sinä et kuole". Ja kuningas vannoi sen hänelle.
\par 24 Mefiboset, Saulin pojanpoika, oli myös tullut kuningasta vastaan. Hän ei ollut siistinyt jalkojansa, ei siistinyt partaansa eikä pesettänyt vaatteitansa siitä päivästä, jona kuningas lähti, siihen päivään asti, jona hän rauhassa tuli takaisin.
\par 25 Kun hän nyt tuli Jerusalemista kuningasta vastaan, sanoi kuningas hänelle: "Miksi sinä et tullut minun kanssani, Mefiboset?"
\par 26 Hän vastasi: "Herrani, kuningas, palvelijani petti minut. Sillä sinun palvelijasi sanoi hänelle: 'Satuloi minun aasini, ratsastaakseni sillä ja mennäkseni kuninkaan kanssa' - sillä palvelijasihan on ontuva -
\par 27 mutta hän on panetellut palvelijaasi herralle kuninkaalle. Herrani, kuningas, on kuitenkin niinkuin Jumalan enkeli; tee siis, mitä hyväksi näet.
\par 28 Sillä koko minun isäni suku oli herrani, kuninkaan, edessä kuoleman oma, ja kuitenkin sinä asetit palvelijasi niiden joukkoon, jotka syövät sinun pöydässäsi. Mitäpä minulla on enää muuta oikeutta ja mitä valittamista kuninkaalle?"
\par 29 Kuningas sanoi hänelle: "Mitäs tuosta enää puhut? Minä sanon, että sinä ja Siiba saatte jakaa keskenänne maaomaisuuden."
\par 30 Niin Mefiboset sanoi kuninkaalle: "Ottakoon hän sen vaikka kokonaan, kun kerran herrani, kuningas, on palannut rauhassa kotiinsa".
\par 31 Gileadilainen Barsillai oli myös tullut Roogelimista ja meni kuninkaan kanssa Jordanille saattamaan häntä Jordanin yli.
\par 32 Barsillai oli hyvin vanha, kahdeksankymmenen vuoden vanha. Hän oli elättänyt kuningasta tämän ollessa Mahanaimissa, sillä hän oli hyvin rikas mies.
\par 33 Kuningas sanoi nyt Barsillaille: "Tule sinä minun mukanani, niin minä elätän sinut luonani Jerusalemissa".
\par 34 Mutta Barsillai vastasi kuninkaalle: "Montakopa ikävuotta minulla enää lienee, että menisin kuninkaan kanssa Jerusalemiin:
\par 35 minä olen nyt kahdeksankymmenen vuoden vanha; osaisinko minä enää erottaa hyvää pahasta, tahi maistuisiko palvelijastasi miltään se, mitä hän syö ja mitä juo? Tahi kuuluisiko minusta enää miltään laulajain ja laulajattarien laulu? Miksi palvelijasi vielä tulisi herralleni, kuninkaalle, kuormaksi?
\par 36 Ainoastaan vähän matkaa seuraa palvelijasi kuningasta tuolle puolelle Jordanin. Miksi kuningas antaisi minulle sellaisen korvauksen?
\par 37 Salli palvelijasi palata takaisin, saadakseni kuolla omassa kaupungissani isäni ja äitini haudan ääressä. Mutta katso, tässä on palvelijasi Kimham, menköön hän herrani, kuninkaan, kanssa; ja tee hänelle, mitä hyväksi näet."
\par 38 Niin kuningas sanoi: "Kimham tulkoon minun kanssani, ja minä teen hänelle, mitä sinä hyväksi näet. Ja kaiken, mitä sinä minulta toivot, minä sinulle teen."
\par 39 Sitten kaikki kansa meni Jordanin yli, ja myös kuningas itse meni. Ja kuningas suuteli Barsillaita ja hyvästeli hänet. Sitten tämä palasi takaisin kotiinsa.
\par 40 Kuningas meni Gilgaliin, ja Kimham meni hänen mukanansa, samoin myös kaikki Juudan kansa. Ja he ja puolet Israelin kansasta veivät kuninkaan yli sinne.
\par 41 Ja katso, silloin kaikki muut Israelin miehet tulivat kuninkaan tykö ja sanoivat kuninkaalle: "Miksi ovat meidän veljemme, Juudan miehet, varkain tuoneet Jordanin yli kuninkaan ja hänen perheensä ynnä kaikki Daavidin miehet hänen kanssaan?"
\par 42 Mutta kaikki Juudan miehet vastasivat Israelin miehille: "Onhan kuningas meille läheisempi; miksi vihastut siitä? Olemmeko me syöneet kuningasta, tahi olemmeko vieneet hänet itsellemme?"
\par 43 Israelin miehet vastasivat Juudan miehille ja sanoivat: "Minulla on kymmenen kertaa suurempi osa kuninkaaseen, myös Daavidiin, kuin sinulla. Miksi olet siis ylenkatsonut minua? Ja enkö minä ensin puhunut kuninkaani takaisintuomisesta?" Mutta Juudan miesten sanat olivat vielä ankarammat kuin Israelin miesten sanat.

\chapter{20}

\par 1 Ja siellä oli sattumalta kelvoton mies, nimeltä Seba, Bikrin poika, benjaminilainen. Hän puhalsi pasunaan ja sanoi: "Meillä ei ole mitään osaa Daavidiin eikä perintöosaa Iisain poikaan. Kukin majallensa, Israel!"
\par 2 Silloin kaikki Israelin miehet luopuivat Daavidista ja seurasivat Sebaa, Bikrin poikaa; mutta Juudan miehet riippuivat kiinni kuninkaassansa ja seurasivat häntä Jordanilta aina Jerusalemiin.
\par 3 Niin Daavid tuli linnaansa Jerusalemiin. Ja kuningas otti ne kymmenen sivuvaimoa, jotka hän oli jättänyt palatsia vartioimaan, ja panetti heidät vartioituun taloon; ja hän elätti heitä, mutta ei mennyt heidän luoksensa. Siellä he olivat eristettyinä kuolinpäiväänsä asti, elävän miehen leskinä.
\par 4 Ja kuningas sanoi Amasalle: "Kutsu koolle Juudan miehet kolmen päivän kuluessa ja ole silloin itse paikallasi täällä".
\par 5 Niin Amasa meni kutsumaan koolle Juudaa; mutta hän viipyi ohi sen ajan, jonka hän oli määrännyt.
\par 6 Silloin sanoi Daavid Abisaille: "Nyt Seba, Bikrin poika, tekee meille enemmän pahaa kuin Absalom. Ota sinä herrasi palvelijat ja aja häntä takaa, ettei hän saisi haltuunsa varustettuja kaupunkeja ja repisi meiltä silmiä."
\par 7 Niin Jooabin miehet sekä kreetit ja pleetit ja kaikki urhot lähtivät liikkeelle hänen jäljessään; he lähtivät Jerusalemista ajamaan takaa Sebaa, Bikrin poikaa.
\par 8 Mutta kun he olivat suuren kiven luona, joka on Gibeonissa, tuli Amasa heitä vastaan. Ja Jooab oli puettu takkiinsa, ja sen päällä oli miekanvyö, kiinnitettynä vyötäisille; miekka oli tupessaan, mutta kun hän lähti liikkeelle, niin se putosi.
\par 9 Ja Jooab sanoi Amasalle: "Voitko hyvin, veljeni?" Ja Jooab tarttui oikealla kädellään Amasaa partaan, muka suudellakseen häntä.
\par 10 Ja kun Amasa ei varonut miekkaa, joka Jooabilla oli kädessään, pisti tämä häntä sillä vatsaan, niin että hänen sisälmyksensä valuivat maahan; toista pistoa ei tarvittu: hän kuoli. Sitten Jooab ja hänen veljensä Abisai ajoivat takaa Sebaa, Bikrin poikaa.
\par 11 Mutta eräs Jooabin palvelijoista jäi seisomaan Amasan ääreen ja huusi: "Jokainen, joka on Jooabille myötämielinen ja joka on Daavidin puolella, seuratkoon Jooabia!"
\par 12 Ja Amasa virui verissään keskellä tietä; mutta kun se mies näki, että kaikki väki pysähtyi siihen, vei hän Amasan tieltä syrjään pellolle ja heitti vaatteen hänen päällensä, nähdessään, että kaikki ohikulkijat pysähtyivät.
\par 13 Ja kun hänet oli toimitettu pois tieltä, kulkivat kaikki ohi ja seurasivat Jooabia ajaakseen takaa Sebaa, Bikrin poikaa.
\par 14 Tämä kulki kaikkien Israelin sukukuntien kautta Aabeliin ja Beet-Maakaan, ja kaikki beeriläiset kokoontuivat ja seurasivat häntä sinne saakka.
\par 15 Mutta he tulivat ja piirittivät häntä Aabel-Beet-Maakassa ja loivat kaupunkia vastaan vallin, joka ulottui ulkomuuriin. Ja kaikki Jooabin väki teki hävitystyötä kukistaakseen muurin.
\par 16 Silloin huusi viisas vaimo kaupungista: "Kuulkaa! Kuulkaa! Sanokaa Jooabille, että hän tulee tänne lähelle puhutellakseni häntä".
\par 17 Ja kun tämä tuli häntä lähelle, sanoi vaimo: "Oletko Jooab?" Hän vastasi: "Olen". Vaimo sanoi hänelle: "Kuule palvelijattaresi sanoja". Hän vastasi: "Minä kuulen".
\par 18 Silloin vaimo sanoi: "Muinoin oli tapana sanoa: 'On kysyttävä neuvoa Aabelista', ja niin tuli valmista.
\par 19 Minä olen rauhallisimpia ja uskollisimpia Israelissa, ja sinä koetat ottaa hengen kaupungilta, joka on äiti Israelissa. Miksi sinä tahdot hävittää Herran perintöosan?"
\par 20 Jooab vastasi ja sanoi: "Pois se, pois se! En minä tahdo hävittää enkä tuhota.
\par 21 Niin ei ole asia, vaan eräs mies Efraimin vuoristosta, nimeltä Seba, Bikrin poika, on nostanut kätensä kuningas Daavidia vastaan; luovuttakaa hänet yksin, niin minä lähden pois kaupungin edustalta." Vaimo vastasi Jooabille: "Katso, hänen päänsä heitetään sinulle muurin yli".
\par 22 Sitten vaimo viisaudessaan meni kaiken kansan luo; niin he löivät Sebalta, Bikrin pojalta, pään poikki ja heittivät sen Jooabille. Silloin tämä puhalsi pasunaan, ja he hajaantuivat kaupungin edustalta, kukin majallensa. Mutta Jooab palasi kuninkaan luo Jerusalemiin.
\par 23 Jooab oli Israelin koko sotajoukon ylipäällikkönä, ja Benaja, Joojadan poika, oli kreettien ja pleettien päällikkönä.
\par 24 Adoram oli verotöiden valvojana, ja Joosafat, Ahiludin poika, oli kanslerina.
\par 25 Seja oli kirjurina, ja Saadok ja Ebjatar olivat pappeina.
\par 26 Myöskin jaairilainen Iira oli Daavidilla pappina.

\chapter{21}

\par 1 Daavidin aikana oli nälänhätä, jota kesti kolme vuotta peräkkäin; silloin Daavid etsi Herran kasvoja. Herra vastasi: "Saulin tähden, verivelan alaisen suvun tähden, koska hän surmasi gibeonilaiset".
\par 2 Niin kuningas kutsui gibeonilaiset ja puhutteli heitä. Mutta gibeonilaiset eivät olleet israelilaisia, vaan amorilaisten jätteitä, ja israelilaiset olivat vannoneet heille valan, mutta Saul oli kiivaillessaan israelilaisten ja Juudan puolesta koettanut hävittää heidät.
\par 3 Daavid sanoi gibeonilaisille: "Mitä minun on tehtävä teidän hyväksenne, ja millä minun on teidät sovitettava, että siunaisitte Herran perintöosaa?"
\par 4 Gibeonilaiset vastasivat hänelle: "Ei ole meidän ja Saulin sekä hänen sukunsa välillä kysymys hopeasta eikä kullasta, eikä meillä ole valtaa surmata ketään Israelissa". Hän sanoi: "Mitä sitten vaaditte minua tekemään teidän hyväksenne?"
\par 5 He vastasivat kuninkaalle: "Sen miehen, joka tahtoi meidät lopettaa ja joka aikoi hävittää meidät, niin ettei meillä olisi missä olla koko Israelin alueella,
\par 6 sen miehen jälkeläisistä luovutettakoon meille seitsemän miestä lävistääksemme heidät paaluihin Herralle Saulin, Herran valitun, Gibeassa". Kuningas sanoi: "Minä luovutan".
\par 7 Mefibosetin, Saulin pojan Joonatanin pojan, kuningas kuitenkin säästi Herran valan tähden, joka oli heidän välillänsä, Daavidin ja Joonatanin, Saulin pojan, välillä.
\par 8 Mutta kuningas otti Rispan, Aijan tyttären, kaksi poikaa, Armonin ja Mefibosetin, jotka tämä oli synnyttänyt Saulille, sekä Meerabin, Saulin tyttären, viisi poikaa, jotka tämä oli synnyttänyt Adrielille, meholalaisen Barsillain pojalle,
\par 9 ja luovutti heidät gibeonilaisten käsiin; ja nämä lävistivät heidät paaluihin vuorella Herran edessä, niin että ne seitsemän sortuivat kaikki yhdessä. Näin heidät surmattiin elonleikkuun ensimmäisinä päivinä, ohranleikkuun alussa.
\par 10 Silloin Rispa, Aijan tytär, otti säkin ja levitti sen kalliolle, ja se oli hänen vuoteenaan elonleikkuun alusta siihen asti, kunnes sade taivaasta vuoti heidän päällensä, eikä hän sallinut taivaan lintujen päivällä eikä metsän eläinten yöllä tulla heidän päällensä.
\par 11 Kun Daavidille kerrottiin, mitä Rispa, Aijan tytär, Saulin sivuvaimo, oli tehnyt,
\par 12 meni Daavid ja otti Saulin ja hänen poikansa Joonatanin luut Gileadin Jaabeksen miehiltä, jotka olivat varkain ottaneet ne siltä torilta Beet-Seanista, jolle filistealaiset olivat heidät ripustaneet silloin, kun filistealaiset surmasivat Saulin Gilboassa.
\par 13 Ja hän toi sieltä Saulin ja hänen poikansa Joonatanin luut. Myös paaluihin lävistettyjen luut koottiin
\par 14 ja haudattiin yhdessä Saulin ja hänen poikansa Joonatanin luiden kanssa Benjaminin maahan Seelaan, Saulin isän, Kiisin, hautaan. Tehtiin kaikki, mitä kuningas oli käskenyt. Niin Jumala leppyi maalle.
\par 15 Taas syttyi sota filistealaisten ja Israelin välillä. Ja Daavid lähti palvelijoineen sotimaan filistealaisia vastaan. Ja Daavid väsyi.
\par 16 Niin Jisbi-Benob, Raafan jälkeläisiä, jonka keihäs painoi kolmesataa sekeliä vaskea ja joka oli puettu uusiin tamineihin, aikoi surmata Daavidin.
\par 17 Mutta Abisai, Serujan poika, tuli hänen avukseen ja löi filistealaisen kuoliaaksi. Silloin Daavidin miehet vannottivat hänellä valan sanoen: "Sinä et saa enää lähteä meidän kanssamme taisteluun, ettet sammuttaisi Israelin lamppua".
\par 18 Sen jälkeen oli taas taistelu filistealaisten kanssa Goobissa: silloin huusalainen Sibbekai surmasi Safin, joka myös oli Raafan jälkeläisiä.
\par 19 Taas oli taistelu filistealaisia vastaan Goobissa. Elhanan, Jare-Ooregimin poika, beetlehemiläinen, surmasi gatilaisen Goljatin, jonka peitsen varsi oli niinkuin kangastukki.
\par 20 Taas oli taistelu Gatissa. Siellä oli kookas mies, jolla oli kuusi sormea kummassakin kädessä ja kuusi varvasta kummassakin jalassa, yhteensä kaksikymmentä neljä; hänkin polveutui Raafasta.
\par 21 Ja kun hän häpäisi Israelia, surmasi hänet Joonatan, Daavidin veljen Simean poika.
\par 22 Nämä neljä polveutuivat gatilaisesta Raafasta; he kaatuivat Daavidin ja hänen palvelijainsa käden kautta.

\chapter{22}

\par 1 Ja Daavid puhui Herralle tämän laulun sanat sinä päivänä, jona Herra oli pelastanut hänet kaikkien hänen vihollistensa ja Saulin vallasta. Hän sanoi:
\par 2 "Herra, minun kallioni, linnani ja pelastajani!
\par 3 Jumala, minun vuoreni, jonka turviin minä pakenen, minun kilpeni, autuuteni sarvi, varustukseni ja pakopaikkani, sinä pelastajani, joka pelastat minut väkivallasta!
\par 4 'Ylistetty olkoon Herra' - niin minä huudan, ja vihollisistani minä pelastun.
\par 5 Sillä kuoleman aallot piirittivät minut, turmion virrat peljästyttivät minut.
\par 6 Tuonelan paulat kietoivat minut, kuoleman ansat yllättivät minut.
\par 7 Ahdistuksessani minä rukoilin Herraa, Jumalaani minä rukoilin; ja hän kuuli minun ääneni temppelistänsä, minun huutoni kohosi hänen korviinsa.
\par 8 Silloin maa huojui ja järisi, taivaan perustukset järkkyivät; ne horjuivat, sillä hänen vihansa syttyi.
\par 9 Savu suitsusi hänen sieraimistaan, kuluttava tuli hänen suustansa, palavat hiilet hehkuivat hänestä.
\par 10 Hän notkisti taivaan ja astui alas, synkkä pilvi jalkojensa alla.
\par 11 Hän ajoi kerubin kannattamana ja lensi, hän näkyi tuulen siipien päältä.
\par 12 Ja hän pani pimeyden majaksi ympärillensä, synkät vedet, paksut pilvet.
\par 13 Hohteesta, joka kävi hänen edellänsä, hehkuivat palavat hiilet.
\par 14 Herra jylisi taivaasta, Korkein antoi äänensä kaikua.
\par 15 Hän lennätti nuolia ja hajotti heidät, salamoita, ja kauhistutti heidät.
\par 16 Silloin meren syvyydet tulivat näkyviin, maanpiirin perustukset paljastuivat Herran nuhtelusta, hänen vihansa hengen puuskauksesta.
\par 17 Hän ojensi kätensä korkeudesta ja tarttui minuun, veti minut ylös suurista vesistä.
\par 18 Hän pelasti minut voimallisesta vihollisestani, minun vihamiehistäni, sillä he olivat minua väkevämmät.
\par 19 He hyökkäsivät minun kimppuuni hätäni päivänä, mutta Herra tuli minun tuekseni.
\par 20 Hän toi minut avaraan paikkaan, hän vapautti minut, sillä hän oli mielistynyt minuun.
\par 21 Herra tekee minulle minun vanhurskauteni mukaan; minun kätteni puhtauden mukaan hän minulle maksaa.
\par 22 Sillä minä olen noudattanut Herran teitä enkä ole luopunut pois Jumalastani, jumalattomuuteen.
\par 23 Kaikki hänen oikeutensa ovat minun silmieni edessä, enkä minä poikkea hänen käskyistänsä.
\par 24 Minä olen vilpitön häntä kohtaan ja varon itseni pahoista teoista.
\par 25 Sentähden Herra palkitsee minulle vanhurskauteni mukaan, sen mukaan kuin olen puhdas hänen silmiensä edessä.
\par 26 Hurskasta kohtaan sinä olet hurskas, nuhteetonta sankaria kohtaan nuhteeton;
\par 27 puhdasta kohtaan sinä olet puhdas, mutta kieroa kohtaan nurja.
\par 28 Ja sinä pelastat nöyrän kansan, mutta sinun silmäsi ovat ylpeitä vastaan, sinä alennat heidät.
\par 29 Sillä sinä, Herra, olet minun lamppuni; Herra valaisee minun pimeyteni.
\par 30 Sinun avullasi minä hyökkään rosvojoukkoa vastaan, Jumalani avulla minä ryntään ylitse muurin.
\par 31 Jumalan tie on nuhteeton, Herran sana tulessa koeteltu. Hän on kaikkien kilpi, jotka häneen turvaavat.
\par 32 Sillä kuka muu on Jumala paitsi Herra, ja kuka muu on pelastuksen kallio paitsi meidän Jumalamme?
\par 33 se Jumala, joka on minun vahva turvani ja johdattaa nuhteetonta hänen tiellänsä,
\par 34 tekee hänen jalkansa nopeiksi niinkuin peurat ja asettaa minut kukkuloilleni,
\par 35 joka opettaa minun käteni sotimaan ja käsivarteni vaskijousta jännittämään.
\par 36 Sinä annat minulle pelastuksen kilven; ja kun sinä kuulet minun rukoukseni, teet sinä minut suureksi.
\par 37 Sinä annat minun askeleilleni avaran tilan, ja minun jalkani eivät horju.
\par 38 Minä ajan vihollisiani takaa ja tuhoan heidät enkä palaja, ennenkuin teen heistä lopun.
\par 39 Minä lopetan heidät ja murskaan heidät, niin etteivät enää nouse; he sortuvat minun jalkojeni alle.
\par 40 Sinä vyötät minut voimalla sotaan, sinä painat vastustajani minun alleni.
\par 41 Sinä ajat minun viholliseni pakoon, vihamieheni minä hukutan.
\par 42 He katselevat, mutta pelastajaa ei ole, katsovat Herran puoleen, mutta hän ei heille vastaa.
\par 43 Minä survon heidät maan tomuksi, kadun loaksi minä heidät poljen ja tallaan.
\par 44 Sinä pelastat minut kansani riidoista, sinä varjelet minua, niin että tulen pakanain pääksi; kansat, joita minä en tunne, palvelevat minua.
\par 45 Muukalaiset minua mielistelevät; jo korvan kuulemalta he tottelevat minua.
\par 46 Muukalaiset masentuvat; he tulevat vyöttäytyneinä varustuksistansa.
\par 47 Herra elää! Kiitetty olkoon minun kallioni, ja ylistetty Jumala, minun pelastukseni kallio,
\par 48 Jumala, joka hankkii minulle koston ja laskee kansat minun valtani alle;
\par 49 sinä, joka vapahdat minut vihollisistani ja korotat minut vastustajaini ylitse ja päästät minut väkivaltaisesta miehestä.
\par 50 Sentähden minä ylistän sinua, Herra, kansojen keskuudessa ja veisaan sinun nimesi kiitosta;
\par 51 sinun, joka annat kuninkaallesi suuren avun ja osoitat armoa voidellullesi, Daavidille, ja hänen jälkeläisilleen, iankaikkisesti."

\chapter{23}

\par 1 Nämä olivat Daavidin viimeiset sanat: Näin puhuu Daavid, Iisain poika, näin puhuu korkealle korotettu mies, Jaakobin Jumalan voideltu, ihana Israelin ylistysvirsissä:
\par 2 "Herran Henki on puhunut minulle, ja hänen sanansa on minun kielelläni;
\par 3 Israelin Jumala on sanonut, Israelin kallio on puhunut minulle: 'Joka hallitsee ihmisiä vanhurskaasti, joka hallitsee Jumalan pelossa,
\par 4 hän on niinkuin huomenhohde auringon noustessa pilvettömänä aamuna, kun maa kirkkaassa valossa vihannoi sateen jälkeen'.
\par 5 Eikö minun sukuni ole näin Jumalan edessä? Sillä hän on tehnyt minun kanssani iankaikkisen liiton, kaikin puolin taatun ja vakaan. Hän antaa versoa minulle kaiken autuuden ja kaiken ilon.
\par 6 Mutta kaikki kelvottomat ovat niinkuin poisviskatut orjantappurat, joihin ei käsin tartuta.
\par 7 Ja jos jonkun on koskettava niihin, varustautuu hän raudalla ja keihäänvarrella; sitten ne tulella poltetaan, siinä missä ovat."
\par 8 Nämä ovat Daavidin sankarien nimet: Jooseb-Bassebet, tahkemonilainen, vaunusoturien päällikkö, hän, joka heilutti keihästään kahdeksansadan kaatuneen yli yhdellä kertaa.
\par 9 Hänen jälkeensä Eleasar, Doodin poika, joka oli erään ahohilaisen poika. Hän oli yksi niistä kolmesta urhosta, jotka olivat Daavidin kanssa silloin, kun he häpäisivät filistealaisia, jotka olivat kokoontuneet sinne sotimaan. Israelin miehet vetäytyivät silloin takaisin,
\par 10 mutta hän jäi paikalleen ja surmasi filistealaisia, kunnes hänen kätensä uupui niin, että se kouristui kiinni miekkaan. Ja Herra antoi suuren voiton sinä päivänä; väki kääntyi vain hänen jälkeensä ryöstämään.
\par 11 Hänen jälkeensä Samma, Aagen poika, hararilainen. Kerran olivat filistealaiset kokoontuneet yhteen joukkoon, ja siellä oli peltopalsta, kokonaan hernettä kasvamassa. Ja väki pakeni filistealaisia,
\par 12 mutta hän asettui keskelle palstaa, sai sen pelastetuksi ja voitti filistealaiset; ja niin Herra antoi suuren voiton.
\par 13 Kerran lähti kolme niistä kolmestakymmenestä päälliköstä liikkeelle, ja he tulivat elonleikkuun aikana Daavidin luo Adullamin luolalle. Ja filistealaisten joukko oli leiriytynyt Refaimin tasangolle.
\par 14 Mutta Daavid oli silloin vuorilinnassa, ja filistealaisten vartiosto oli Beetlehemissä.
\par 15 Ja Daavidin rupesi tekemään mieli vettä, ja hän sanoi: "Jospa joku toisi minulle vettä juodakseni Beetlehemin kaivosta, joka on portin edustalla!"
\par 16 Silloin murtautuivat ne kolme urhoa filistealaisten leirin läpi ja ammensivat vettä Beetlehemin kaivosta portin edustalta, kantoivat ja toivat sen Daavidille. Mutta hän ei tahtonut sitä juoda, vaan vuodatti sen juomauhriksi Herralle
\par 17 ja sanoi: "Pois se! Herra varjelkoon minut sitä tekemästä. Onhan se niiden miesten veri, jotka menivät sinne oman henkensä uhalla." Eikä hän tahtonut juoda sitä. Tämän tekivät ne kolme urhoa.
\par 18 Abisai, Jooabin veli, Serujan poika, oli niiden kolmen päällikkö; hän heilutti keihästään kolmensadan kaatuneen yli. Ja hän oli kuulu niiden kolmen joukossa.
\par 19 Hän oli arvossa pidetty niiden kolmen joukossa ja oli heidän päämiehensä, mutta ei hän vetänyt vertoja niille kolmelle.
\par 20 Benaja, Joojadan poika, joka oli urhoollisen ja suurista teoistaan kuuluisan miehen poika, oli kotoisin Kabseelista. Hän surmasi ne kaksi mooabilaista sankaria, ja hän laskeutui alas ja tappoi lumituiskun aikana leijonan kaivoon.
\par 21 Myöskin surmasi hän egyptiläisen miehen, sen uhkean miehen. Egyptiläisellä oli keihäs kädessä, mutta hän meni häntä vastaan ainoastaan sauva kädessä. Ja hän tempasi keihään egyptiläisen kädestä ja tappoi hänet hänen omalla keihäällään.
\par 22 Tällaisia teki Benaja, Joojadan poika. Ja hän oli kuulu niiden kolmen urhon joukossa.
\par 23 Hän oli arvossa pidetty niiden kolmenkymmenen joukossa, mutta ei hän vetänyt vertoja niille kolmelle. Ja Daavid asetti hänet henkivartiostonsa päälliköksi.
\par 24 Niiden kolmenkymmenen joukossa oli Asael, Jooabin veli; Elhanan, Doodon poika, Beetlehemistä;
\par 25 harodilainen Samma; harodilainen Elika;
\par 26 pleettiläinen Heles; tekoalainen Iira, Ikkeksen poika;
\par 27 anatotilainen Abieser; huusalainen Mebunnai;
\par 28 ahohilainen Salmon; netofalainen Maharai;
\par 29 netofalainen Heeleb, Baanan poika; Ittai, Riibain poika, benjaminilaisten Gibeasta;
\par 30 piratonilainen Benaja; Hiddai Nahale-Gaasista;
\par 31 arabalainen Abi-Albon; barhumilainen Asmavet;
\par 32 saalbonilainen Eljahba; Bene-Jaasen; Joonatan;
\par 33 hararilainen Samma; ararilainen Ahiam, Sararin poika;
\par 34 Elifelet, Ahasbain poika, joka oli erään maakatilaisen poika; giilolainen Eliam, Ahitofelin poika;
\par 35 karmelilainen Hesrai; arabilainen Paarai;
\par 36 Jigal, Naatanin poika, Soobasta; gaadilainen Vaani;
\par 37 ammonilainen Selek; beerotilainen Naharai, Jooabin, Serujan pojan, aseenkantaja;
\par 38 jeteriläinen Iira; jeteriläinen Gaareb;
\par 39 heettiläinen Uuria. Kaikkiaan kolmekymmentä seitsemän.

\chapter{24}

\par 1 Mutta Herran viha syttyi taas Israelia kohtaan, niin että hän yllytti Daavidin heitä vastaan, sanoen: "Mene ja laske Israel ja Juuda".
\par 2 Niin kuningas sanoi Jooabille, sotajoukon päällikölle, joka oli hänen luonaan: "Kiertele kaikki Israelin sukukunnat Daanista Beersebaan asti, ja pitäkää kansan katselmus, että minä saisin tietää kansan lukumäärän".
\par 3 Jooab vastasi kuninkaalle: "Herra, sinun Jumalasi, lisätköön kansan, olkoon se kuinka suuri tahansa, satakertaiseksi, ja saakoon herrani, kuningas, nähdä sen omin silmin. Mutta miksi herrani, kuningas, haluaa sellaista?"
\par 4 Kuitenkin kuninkaan sana velvoitti Jooabia ja sotapäällikköjä; niin Jooab ja sotapäälliköt lähtivät kuninkaan luota pitämään Israelin kansan katselmusta.
\par 5 Ja he menivät Jordanin yli ja leiriytyivät Aroeriin, oikealle puolelle jokilaakson keskikohdalla olevaa kaupunkia, Gaadiin ja Jaeseriin päin.
\par 6 Sitten he tulivat Gileadiin ja Tahtim-Hodsin maahan; sitten he tulivat Daan-Jaaniin ja sieltä ympäri Siidoniin.
\par 7 Sitten he tulivat Tyyron linnoitukseen ja kaikkiin hivviläisten ja kanaanilaisten kaupunkeihin ja menivät sieltä Juudan Etelämaahan Beersebaan asti.
\par 8 Ja kun he olivat kierrelleet koko maan, tulivat he yhdeksän kuukauden ja kahdenkymmenen päivän kuluttua takaisin Jerusalemiin.
\par 9 Ja Jooab ilmoitti kuninkaalle kansan lasketun lukumäärän: Israelissa oli kahdeksansataa tuhatta sotakuntoista miekkamiestä, ja Juudassa oli viisisataa tuhatta miestä.
\par 10 Mutta Daavidin omatunto soimasi häntä, sittenkuin hän oli laskenut kansan, ja Daavid sanoi Herralle: "Minä olen tehnyt suuren synnin tehdessäni tämän; mutta anna nyt, Herra, anteeksi palvelijasi rikos, sillä minä olen menetellyt ylen tyhmästi".
\par 11 Ja kun Daavid aamulla nousi, oli profeetta Gaadille, Daavidin näkijälle, tullut tämä Herran sana:
\par 12 "Mene ja puhu Daavidille: Näin sanoo Herra: Kolme ehtoa minä asetan sinun eteesi; valitse yksi niistä, niin minä teen sen sinulle".
\par 13 Niin Gaad meni Daavidin tykö, ilmoitti ja sanoi hänelle: "Tuleeko nälkä seitsemäksi vuodeksi sinun maahasi, vai pakenetko sinä kolme kuukautta vihollisiasi, jotka ajavat sinua takaa, vai tuleeko rutto sinun maahasi kolmeksi päiväksi? Mieti nyt ja katso, mitä minä vastaan hänelle, joka minut lähetti."
\par 14 Daavid vastasi Gaadille: "Minä olen suuressa hädässä. Me tahdomme langeta Herran käsiin, sillä hänen laupeutensa on suuri; ihmisten käsiin minä en tahdo langeta."
\par 15 Niin Herra antoi ruton tulla Israeliin, aamusta alkaen määrättyyn aikaan asti; ja kansaa kuoli Daanista Beersebaan asti seitsemänkymmentä tuhatta miestä.
\par 16 Mutta kun enkeli ojensi kätensä Jerusalemia kohti tuhotaksensa sen, katui Herra sitä pahaa ja sanoi enkelille, joka kansaa tuhosi: "Jo riittää; laske kätesi alas". Ja Herran enkeli oli silloin jebusilaisen Araunan puimatantereen luona.
\par 17 Mutta nähdessään enkelin surmaavan kansaa sanoi Daavid Herralle: "Katso, minä olen tehnyt syntiä, minä olen tehnyt väärin; mutta nämä minun lampaani, mitä he ovat tehneet? Sattukoon sinun kätesi minuun ja minun isäni perheeseen."
\par 18 Ja Gaad tuli Daavidin tykö samana päivänä ja sanoi hänelle: "Mene ja pystytä Herralle alttari jebusilaisen Araunan puimatantereelle".
\par 19 Niin Daavid meni, Gaadin sanan ja Herran käskyn mukaan.
\par 20 Mutta kun Arauna katsahti ulos ja näki kuninkaan palvelijoineen tulevan luoksensa, meni hän ulos ja kumartui kasvoillensa maahan kuninkaan eteen.
\par 21 Ja Arauna sanoi: "Miksi herrani, kuningas, tulee palvelijansa luo?" Daavid vastasi: "Ostamaan sinulta puimatantereen, rakentaakseni Herralle alttarin, että vitsaus taukoaisi kansasta".
\par 22 Arauna sanoi Daavidille: "Herrani, kuningas, ottakoon ja uhratkoon, mitä hyväksi näkee. Katso, tässä on härät polttouhriksi ja puimaäkeet sekä härkien valjaat haloiksi.
\par 23 Tämän kaiken, kuningas, Arauna antaa kuninkaalle." Ja Arauna sanoi kuninkaalle: "Olkoon Herra, sinun Jumalasi, sinulle suosiollinen".
\par 24 Mutta kuningas sanoi Araunalle: "Ei niin, vaan minä ostan ne sinulta täydestä hinnasta, sillä minä en uhraa Herralleni, Jumalalleni, ilmaiseksi saatuja polttouhreja". Niin Daavid osti puimatantereen ja härät viidelläkymmenellä hopeasekelillä.
\par 25 Ja Daavid rakensi sinne alttarin Herralle ja uhrasi polttouhreja ja yhteysuhreja. Niin Herra leppyi maalle, ja vitsaus taukosi Israelista.


\end{document}