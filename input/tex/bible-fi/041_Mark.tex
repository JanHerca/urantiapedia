\begin{document}

\title{Evankeliumi Markuksen mukaan}


\chapter{1}

\par 1 Jeesuksen Kristuksen, Jumalan Pojan, evankeliumin alku.
\par 2 Niinkuin on kirjoitettuna profeetta Esaiaan kirjassa: "Katso, minä lähetän enkelini sinun edelläsi, ja hän on valmistava sinun tiesi".
\par 3 "Huutavan ääni kuuluu erämaassa: 'Valmistakaa Herralle tie, tehkää polut hänelle tasaisiksi'",
\par 4 niin Johannes Kastaja saarnasi erämaassa parannuksen kastetta syntien anteeksisaamiseksi.
\par 5 Ja koko Juudean maa ja kaikki jerusalemilaiset vaelsivat hänen tykönsä, ja hän kastoi heidät Jordanin virrassa, kun he tunnustivat syntinsä.
\par 6 Ja Johanneksella oli puku kamelinkarvoista ja vyötäisillään nahkavyö; ja hän söi heinäsirkkoja ja metsähunajaa.
\par 7 Ja hän saarnasi sanoen: "Minun jälkeeni tulee minua väkevämpi, jonka kengänpaulaa minä en ole kelvollinen maahan kumartuneena päästämään.
\par 8 Minä kastan teidät vedellä, mutta hän kastaa teidät Pyhällä Hengellä."
\par 9 Ja niinä päivinä Jeesus tuli Galilean Nasaretista, ja Johannes kastoi hänet Jordanissa.
\par 10 Ja heti, vedestä noustessaan, hän näki taivasten aukeavan ja Hengen niinkuin kyyhkysen laskeutuvan häneen.
\par 11 Ja taivaista tuli ääni: "Sinä olet minun rakas Poikani; sinuun minä olen mielistynyt".
\par 12 Kohta sen jälkeen Henki ajoi hänet erämaahan.
\par 13 Ja hän oli erämaassa neljäkymmentä päivää, ja saatana kiusasi häntä, ja hän oli petojen seassa; ja enkelit tekivät hänelle palvelusta.
\par 14 Mutta sittenkuin Johannes oli pantu vankeuteen, meni Jeesus Galileaan ja saarnasi Jumalan evankeliumia
\par 15 ja sanoi: "Aika on täyttynyt, ja Jumalan valtakunta on tullut lähelle; tehkää parannus ja uskokaa evankeliumi".
\par 16 Ja kulkiessaan Galilean järven rantaa hän näki Simonin ja Andreaan, Simonin veljen, heittävän verkkoa järveen; sillä he olivat kalastajia.
\par 17 Ja Jeesus sanoi heille: "Seuratkaa minua, niin minä teen teistä ihmisten kalastajia".
\par 18 Kohta he jättivät verkot ja seurasivat häntä.
\par 19 Ja käytyään siitä vähän eteenpäin hän näki Jaakobin, Sebedeuksen pojan, ja Johanneksen, hänen veljensä, heidätkin venheessä, laittamassa verkkojaan kuntoon.
\par 20 Ja kohta hän kutsui heidät; ja he jättivät isänsä, Sebedeuksen, palkkalaisineen venheeseen ja lähtivät seuraamaan häntä.
\par 21 Ja he saapuivat Kapernaumiin; ja hän meni kohta sapattina synagoogaan ja opetti.
\par 22 Ja he olivat hämmästyksissään hänen opetuksestansa, sillä hän opetti heitä niinkuin se, jolla valta on, eikä niinkuin kirjanoppineet.
\par 23 Ja heidän synagoogassaan oli juuri silloin mies, jossa oli saastainen henki; ja se huusi
\par 24 sanoen: "Mitä sinulla on meidän kanssamme tekemistä, Jeesus Nasaretilainen? Oletko tullut meitä tuhoamaan? Minä tunnen sinut, kuka olet, sinä Jumalan Pyhä."
\par 25 Niin Jeesus nuhteli sitä sanoen: "Vaikene ja lähde hänestä".
\par 26 Ja saastainen henki kouristi häntä ja lähti hänestä huutaen suurella äänellä.
\par 27 Ja he hämmästyivät kaikki, niin että kyselivät toisiltaan sanoen: "Mitä tämä on? Uusi, voimallinen oppi! Hän käskee saastaisia henkiäkin, ja ne tottelevat häntä."
\par 28 Ja hänen maineensa levisi kohta koko ympäristöön, kaikkialle Galileaan.
\par 29 Ja tultuaan ulos synagoogasta he menivät kohta Simonin ja Andreaan taloon Jaakobin ja Johanneksen kanssa.
\par 30 Ja Simonin anoppi makasi sairaana kuumeessa, ja kohta he puhuivat hänestä Jeesukselle.
\par 31 Ja hän meni hänen luoksensa ja nosti hänet ylös, tarttuen hänen käteensä; ja kuume lähti hänestä, ja hän palveli heitä.
\par 32 Mutta illan tultua, kun aurinko oli laskenut, tuotiin hänen tykönsä kaikki sairaat ja riivatut,
\par 33 ja koko kaupunki oli koolla oven edessä.
\par 34 Ja hän paransi monta, jotka sairastivat moninaisia tauteja, ja paljon riivaajia hän ajoi ulos eikä sallinut riivaajien puhua, koska ne tunsivat hänet.
\par 35 Ja varhain aamulla, kun vielä oli pimeä, hän nousi, lähti ulos ja meni autioon paikkaan; ja siellä hän rukoili.
\par 36 Mutta Simon ja ne, jotka olivat hänen kanssaan, riensivät hänen jälkeensä;
\par 37 ja löydettyään hänet he sanoivat hänelle: "Kaikki etsivät sinua".
\par 38 Ja hän sanoi heille: "Menkäämme muualle, läheisiin kyliin, että minä sielläkin saarnaisin, sillä sitä varten minä olen tullut".
\par 39 Ja hän meni ja saarnasi heidän synagoogissaan koko Galileassa ja ajoi ulos riivaajat.
\par 40 Ja hänen tykönsä tuli pitalinen mies, rukoili häntä, polvistui ja sanoi hänelle: "Jos tahdot, niin sinä voit minut puhdistaa".
\par 41 Niin Jeesuksen kävi häntä sääliksi, ja ojentaen kätensä hän kosketti häntä ja sanoi hänelle: "Minä tahdon; puhdistu".
\par 42 Ja kohta pitali lähti hänestä, ja hän puhdistui.
\par 43 Ja varoittaen häntä ankarasti hän laski hänet heti menemään
\par 44 ja sanoi hänelle: "Katso, ettet puhu tästä kenellekään mitään, vaan mene ja näytä itsesi papille ja uhraa puhdistumisestasi se, minkä Mooses on säätänyt, todistukseksi heille".
\par 45 Mutta mentyään pois tämä rupesi laajalti julistamaan ja asiasta tietoa levittämään, niin ettei Jeesus enää saattanut julkisesti mennä kaupunkeihin, vaan oleskeli niiden ulkopuolella autioissa paikoissa; ja kaikkialta tultiin hänen tykönsä.

\chapter{2}

\par 1 Ja muutamien päivien perästä hän taas meni Kapernaumiin; ja kun kuultiin hänen olevan kotona,
\par 2 kokoontui paljon väkeä, niin etteivät he enää mahtuneet oven edustallekaan. Ja hän puhui heille sanaa.
\par 3 Ja he tulivat tuoden hänen tykönsä halvattua, jota kantamassa oli neljä miestä.
\par 4 Ja kun he väentungokselta eivät päässeet häntä tuomaan hänen tykönsä, purkivat he katon siltä kohdalta, missä hän oli, ja kaivettuaan aukon laskivat alas vuoteen, jossa halvattu makasi.
\par 5 Kun Jeesus näki heidän uskonsa, sanoi hän halvatulle: "Poikani, sinun syntisi annetaan anteeksi".
\par 6 Mutta siellä istui muutamia kirjanoppineita, ja he ajattelivat sydämessään:
\par 7 "Kuinka tämä näin puhuu? Hän pilkkaa Jumalaa. Kuka voi antaa syntejä anteeksi paitsi Jumala yksin?"
\par 8 Ja heti Jeesus tunsi hengessänsä, että he mielessään niin ajattelivat, ja sanoi heille: "Miksi ajattelette sellaista sydämessänne?
\par 9 Kumpi on helpompaa, sanoako halvatulle: 'Sinun syntisi annetaan anteeksi', vai sanoa: 'Nouse, ota vuoteesi ja käy'?
\par 10 Mutta tietääksenne, että Ihmisen Pojalla on valta maan päällä antaa syntejä anteeksi, niin" - hän sanoi halvatulle -
\par 11 "minä sanon sinulle: nouse, ota vuoteesi ja mene kotiisi."
\par 12 Silloin hän nousi, otti kohta vuoteensa ja meni ulos kaikkien nähden, niin että kaikki hämmästyivät ja ylistivät Jumalaa sanoen: "Tämänkaltaista emme ole ikinä nähneet".
\par 13 Ja taas hän lähti pois ja kulki järven rantaa. Ja kaikki kansa tuli hänen tykönsä, ja hän opetti heitä.
\par 14 Ja ohi kulkiessaan hän näki Leevin, Alfeuksen pojan, istumassa tulliasemalla ja sanoi hänelle: "Seuraa minua". Niin tämä nousi ja seurasi häntä.
\par 15 Ja kun hän aterioi hänen kodissaan, aterioi myös monta publikaania ja syntistä Jeesuksen ja hänen opetuslastensa kanssa; sillä heitä oli paljon häntä seuraamassa.
\par 16 Kun fariseusten kirjanoppineet näkivät, että hän söi syntisten ja publikaanien kanssa, sanoivat he hänen opetuslapsilleen: "Publikaanien ja syntistenkö kanssa hän syö?"
\par 17 Sen kuullessaan Jeesus sanoi heille: "Eivät terveet tarvitse parantajaa, vaan sairaat. En minä ole tullut kutsumaan vanhurskaita, vaan syntisiä."
\par 18 Ja Johanneksen opetuslapset ja fariseukset pitivät paastoa. Niin tultiin ja sanottiin hänelle: "Johanneksen opetuslapset ja fariseusten opetuslapset paastoavat; miksi sinun opetuslapsesi eivät paastoa?"
\par 19 Jeesus sanoi heille: "Eiväthän häävieraat voi paastota silloin, kun ylkä on heidän kanssaan? Niin kauan kuin heillä on ylkä seurassaan, eivät he voi paastota.
\par 20 Mutta päivät tulevat, jolloin ylkä otetaan heiltä pois, ja silloin, sinä päivänä, he paastoavat.
\par 21 Ei kukaan ompele vanuttamattomasta kankaasta paikkaa vanhaan vaippaan; muutoin uusi täytetilkku repii palasen vanhasta vaipasta, ja reikä tulee pahemmaksi.
\par 22 Eikä kukaan laske nuorta viiniä vanhoihin nahkaleileihin; muutoin viini pakahuttaa leilit, ja viini menee hukkaan, ja leilit turmeltuvat; vaan nuori viini on laskettava uusiin leileihin."
\par 23 Ja tapahtui, että hän sapattina kulki viljavainioiden halki, ja hänen opetuslapsensa rupesivat kulkiessaan katkomaan tähkäpäitä.
\par 24 Niin fariseukset sanoivat hänelle: "Katso, miksi he tekevät sapattina sitä, mikä ei ole luvallista?"
\par 25 Hän sanoi heille: "Ettekö ole koskaan lukeneet, mitä Daavid teki, kun hän ja hänen seuralaisensa olivat puutteessa ja heidän oli nälkä,
\par 26 kuinka hän meni Jumalan huoneeseen ylimmäisen papin Abjatarin aikana ja söi näkyleivät, joita ei ollut lupa syödä muiden kuin pappien, ja antoi myös niille, jotka hänen kanssansa olivat?"
\par 27 Ja hän sanoi heille: "Sapatti on asetettu ihmistä varten eikä ihminen sapattia varten.
\par 28 Niin Ihmisen Poika siis on sapatinkin herra."

\chapter{3}

\par 1 Ja hän meni taas synagoogaan, ja siellä oli mies, jonka käsi oli kuivettunut.
\par 2 Ja voidakseen nostaa syytteen häntä vastaan he pitivät häntä silmällä, parantaisiko hän miehen sapattina.
\par 3 Niin hän sanoi miehelle, jonka käsi oli kuivettunut: "Nouse ja astu esille".
\par 4 Ja hän sanoi heille: "Kumpiko on luvallista sapattina: hyvääkö tehdä vai pahaa, pelastaako henki vai tappaa se?" Mutta he olivat vaiti.
\par 5 Silloin hän katsahtaen ympärilleen loi vihassa silmänsä heihin, murheellisena heidän sydämensä paatumuksesta, ja sanoi sille miehelle: "Ojenna kätesi". Ja hän ojensi, ja hänen kätensä tuli jälleen terveeksi.
\par 6 Ja fariseukset lähtivät ulos ja pitivät kohta herodilaisten kanssa neuvoa häntä vastaan, surmataksensa hänet.
\par 7 Mutta Jeesus vetäytyi opetuslapsineen järven rannalle, ja häntä seurasi suuri joukko kansaa Galileasta. Ja Juudeasta
\par 8 ja Jerusalemista ja Idumeasta ja Jordanin tuolta puolen ja Tyyron ja Siidonin ympäristöltä tuli paljon kansaa hänen tykönsä, kun he kuulivat, kuinka suuria tekoja hän teki.
\par 9 Ja hän sanoi opetuslapsillensa, että hänelle oli pidettävä venhe varalla väentungoksen tähden, etteivät he ahdistaisi häntä;
\par 10 sillä hän paransi monta, jonka tähden kaikki, joilla oli vaivoja, tunkeutuivat hänen päälleen koskettaaksensa häntä.
\par 11 Ja kun saastaiset henget näkivät hänet, lankesivat he maahan hänen eteensä ja huusivat sanoen: "Sinä olet Jumalan Poika".
\par 12 Ja hän varoitti ankarasti heitä saattamasta häntä julki.
\par 13 Ja hän nousi vuorelle ja kutsui tykönsä ne, jotka hän itse tahtoi, ja he menivät hänen tykönsä.
\par 14 Niin hän asetti kaksitoista olemaan kanssansa ja lähettääksensä heidät saarnaamaan,
\par 15 ja heillä oli oleva valta ajaa ulos riivaajia.
\par 16 Ja nämä kaksitoista hän asetti: Pietarin - tämän nimen hän antoi Simonille -
\par 17 ja Jaakobin, Sebedeuksen pojan, ja Johanneksen, Jaakobin veljen, joille hän antoi nimen Boanerges, se on: ukkosenjylinän pojat,
\par 18 ja Andreaan ja Filippuksen ja Bartolomeuksen ja Matteuksen ja Tuomaan ja Jaakobin, Alfeuksen pojan, ja Taddeuksen ja Simon Kananeuksen
\par 19 ja Juudas Iskariotin, saman, joka hänet kavalsi.
\par 20 Ja hän tuli kotiin. Ja taas kokoontui kansaa, niin etteivät he päässeet syömäänkään.
\par 21 Kun hänen omaisensa sen kuulivat, menivät he ottamaan häntä huostaansa; sillä he sanoivat: "Hän on poissa suunniltaan".
\par 22 Ja kirjanoppineet, jotka olivat tulleet Jerusalemista, sanoivat: "Hänessä on Beelsebul", ja: "Riivaajien päämiehen voimalla hän ajaa ulos riivaajia".
\par 23 Niin hän kutsui heidät luoksensa ja sanoi heille vertauksilla: "Kuinka saatana voi ajaa ulos saatanan?
\par 24 Ja jos jokin valtakunta riitautuu itsensä kanssa, ei se valtakunta voi pysyä pystyssä.
\par 25 Ja jos jokin talo riitautuu itsensä kanssa, ei se talo voi pysyä pystyssä.
\par 26 Ja jos saatana nousee itseänsä vastaan ja riitautuu itsensä kanssa, ei hän voi pysyä, vaan hänen loppunsa on tullut.
\par 27 Eihän kukaan voi tunkeutua väkevän taloon ja ryöstää hänen tavaraansa, ellei hän ensin sido sitä väkevää; vasta sitten hän ryöstää tyhjäksi hänen talonsa.
\par 28 Totisesti minä sanon teille: kaikki synnit annetaan ihmisten lapsille anteeksi, pilkkaamisetkin, kuinka paljon pilkannevatkin;
\par 29 mutta joka pilkkaa Pyhää Henkeä, se ei saa ikinä anteeksi, vaan on vikapää iankaikkiseen syntiin."
\par 30 Sillä he sanoivat: "Hänessä on saastainen henki".
\par 31 Ja hänen äitinsä ja veljensä tulivat, seisahtuivat ulkopuolelle ja lähettivät hänen luoksensa kutsumaan häntä.
\par 32 Ja kansanjoukko istui hänen ympärillään, ja he sanoivat hänelle: "Katso, sinun äitisi ja veljesi tuolla ulkona kysyvät sinua".
\par 33 Hän vastasi heille ja sanoi: "Kuka on minun äitini, ja ketkä ovat minun veljeni?"
\par 34 Ja katsellen ympärilleen niihin, jotka istuivat hänen ympärillään, hän sanoi: "Katso, minun äitini ja veljeni!
\par 35 Sillä joka tekee Jumalan tahdon, se on minun veljeni ja sisareni ja äitini."

\chapter{4}

\par 1 Ja hän rupesi taas opettamaan järven rannalla. Ja hänen luoksensa kokoontui hyvin paljon kansaa, jonka tähden hän astui venheeseen ja istui siinä järvellä, ja kaikki kansa oli maalla järven rannalla.
\par 2 Ja hän opetti heitä paljon vertauksilla ja sanoi heille opettaessaan:
\par 3 "Kuulkaa! Katso, kylväjä lähti kylvämään.
\par 4 Ja hänen kylväessään osa putosi tien oheen, ja linnut tulivat ja söivät sen.
\par 5 Ja osa putosi kallioperälle, jossa sillä ei ollut paljon maata, ja se nousi kohta oraalle, kun sillä ei ollut syvää maata.
\par 6 Mutta auringon noustua se paahtui, ja kun sillä ei ollut juurta, niin se kuivettui.
\par 7 Ja osa putosi orjantappuroihin; ja orjantappurat nousivat ja tukahuttivat sen, eikä se tehnyt hedelmää.
\par 8 Ja osa putosi hyvään maahan; ja se nousi oraalle, kasvoi ja antoi sadon ja kantoi kolmeenkymmeneen ja kuuteenkymmeneen ja sataan jyvään asti."
\par 9 Ja hän sanoi: "Jolla on korvat kuulla, se kuulkoon".
\par 10 Ja kun hän oli jäänyt yksin, niin ne, jotka olivat hänen ympärillään, ynnä ne kaksitoista kysyivät häneltä näitä vertauksia.
\par 11 Niin hän sanoi heille: "Teille on annettu Jumalan valtakunnan salaisuus, mutta noille ulkopuolella oleville kaikki tulee vertauksissa,
\par 12 että he näkemällä näkisivät, eivätkä huomaisi, ja kuulemalla kuulisivat, eivätkä ymmärtäisi, niin etteivät kääntyisi ja saisi anteeksi".
\par 13 Ja hän sanoi heille: "Ette käsitä tätä vertausta; kuinka sitten voitte ymmärtää kaikki muut vertaukset?
\par 14 Kylväjä kylvää sanan.
\par 15 Mitkä tien oheen putosivat, ovat ne, joihin sana kylvetään, mutta kun he sen kuulevat, niin saatana heti tulee ja ottaa pois heihin kylvetyn sanan.
\par 16 Ja mitkä kallioperälle kylvettiin, ovat niinikään ne, jotka, kun kuulevat sanan, heti ottavat sen ilolla vastaan,
\par 17 mutta heillä ei ole juurta itsessään, vaan he kestävät ainoastaan jonkun aikaa; kun sitten tulee ahdistus tai vaino sanan tähden, niin he kohta lankeavat pois.
\par 18 Ja toisia ovat orjantappuroihin kylvetyt; nämä ovat ne, jotka kuulevat sanan,
\par 19 mutta maailman huolet ja rikkauden viettelys ja muut himot pääsevät valtaan ja tukahuttavat sanan, ja se jää hedelmättömäksi.
\par 20 Ja mitkä hyvään maahan kylvettiin, ovat ne, jotka kuulevat sanan ja ottavat sen vastaan ja kantavat hedelmän, mikä kolmikymmen-, mikä kuusikymmen-, mikä satakertaisen."
\par 21 Ja hän sanoi heille: "Eihän lamppua oteta esiin, pantavaksi vakan alle tai vuoteen alle? Eiköhän lampunjalkaan pantavaksi?
\par 22 Sillä ei mikään ole salattuna muuta varten, kuin että se tulisi ilmi, eikä kätkettynä muuta varten, kuin tullakseen julki.
\par 23 Jos jollakin on korvat kuulla, hän kuulkoon."
\par 24 Ja hän sanoi heille: "Ottakaa vaari siitä, mitä kuulette; millä mitalla te mittaatte, sillä teille mitataan, vieläpä teille lisätäänkin.
\par 25 Sillä sille jolla on, sille annetaan; mutta siltä, jolla ei ole, otetaan pois sekin, mikä hänellä on."
\par 26 Ja hän sanoi: "Niin on Jumalan valtakunta, kuin jos mies kylvää siemenen maahan;
\par 27 ja hän nukkuu, ja hän nousee, öin ja päivin; ja siemen orastaa ja kasvaa, hän ei itse tiedä, miten.
\par 28 Sillä itsestään maa tuottaa viljan: ensin korren, sitten tähkän, sitten täyden jyvän tähkään.
\par 29 Mutta kun hedelmä on kypsynyt, lähettää hän kohta sinne sirpin, sillä elonaika on käsissä."
\par 30 Ja hän sanoi: "Mihin vertaamme Jumalan valtakunnan, eli mitä vertausta siitä käytämme?
\par 31 Se on niinkuin sinapinsiemen, joka, kun se kylvetään maahan, on pienin kaikista siemenistä maan päällä;
\par 32 mutta kun se on kylvetty, niin se nousee ja tulee suurimmaksi kaikista vihanneskasveista ja tekee suuria oksia, niin että taivaan linnut voivat tehdä pesänsä sen varjoon."
\par 33 Monilla tämänkaltaisilla vertauksilla hän puhui heille sanaa, sen mukaan kuin he kykenivät kuulemaan;
\par 34 ja ilman vertausta hän ei puhunut heille. Mutta opetuslapsillensa hän selitti kaikki, kun he olivat yksikseen.
\par 35 Ja sinä päivänä hän illan tultua sanoi heille: "Lähtekäämme yli toiselle rannalle".
\par 36 Niin he laskivat kansan luotaan ja ottivat hänet mukaansa, niinkuin hän venheessä oli; ja muitakin venheitä oli hänen seurassaan.
\par 37 Ja nousi kova myrskytuuli, ja aallot syöksyivät venheeseen, niin että venhe jo täyttyi.
\par 38 Ja itse hän oli peräkeulassa ja nukkui nojaten päänaluseen. Ja he herättivät hänet ja sanoivat hänelle: "Opettaja, etkö välitä siitä, että me hukumme?"
\par 39 Ja herättyään hän nuhteli tuulta ja sanoi järvelle: "Vaikene, ole hiljaa". Niin tuuli asettui, ja tuli aivan tyven.
\par 40 Ja hän sanoi heille: "Miksi olette niin pelkureita? Kuinka teillä ei ole uskoa?"
\par 41 Ja suuri pelko valtasi heidät, ja he sanoivat toisillensa: "Kuka onkaan tämä, kun sekä tuuli että meri häntä tottelevat?"

\chapter{5}

\par 1 Ja he tulivat toiselle puolelle järveä gerasalaisten alueelle.
\par 2 Ja kohta kun hän lähti venheestä, tuli häntä vastaan haudoista mies, joka oli saastaisen hengen vallassa.
\par 3 Hän asusti haudoissa, eikä kukaan enää voinut häntä kahleillakaan sitoa;
\par 4 sillä hän oli monta kertaa ollut sidottuna jalkanuoriin ja kahleisiin, mutta oli särkenyt kahleet ja katkonut jalkanuorat, eikä kukaan kyennyt häntä hillitsemään.
\par 5 Ja hän oleskeli aina, yötä ja päivää, haudoissa ja vuorilla, huutaen ja runnellen itseään kivillä.
\par 6 Kun hän kaukaa näki Jeesuksen, juoksi hän ja kumartui maahan hänen eteensä
\par 7 ja huutaen suurella äänellä sanoi: "Mitä sinulla on minun kanssani tekemistä, Jeesus, Jumalan, Korkeimman, Poika? Minä vannotan sinua Jumalan kautta, älä vaivaa minua."
\par 8 Sillä hän oli sanomaisillaan sille: "Lähde ulos miehestä, sinä saastainen henki".
\par 9 Ja Jeesus kysyi siltä: "Mikä on nimesi?" Niin se sanoi hänelle: "Legio on minun nimeni, sillä meitä on monta".
\par 10 Ja se pyysi pyytämällä häntä, ettei hän lähettäisi niitä pois siitä seudusta.
\par 11 Niin siellä oli lähellä vuorta suuri sikalauma laitumella.
\par 12 Ja ne pyysivät häntä sanoen: "Lähetä meidät sikoihin, että menisimme niihin".
\par 13 Ja hän antoi niille luvan. Niin saastaiset henget lähtivät miehestä ja menivät sikoihin. Silloin lauma, noin kaksituhatta sikaa, syöksyi jyrkännettä alas järveen; ja ne hukkuivat järveen.
\par 14 Ja niiden paimentajat pakenivat ja kertoivat siitä kaupungissa ja maataloissa. Ja kansa lähti katsomaan, mitä oli tapahtunut.
\par 15 Ja he tulivat Jeesuksen luo ja näkivät riivatun, jossa legio oli ollut, istuvan puettuna ja täydessä ymmärryksessään; ja he peljästyivät.
\par 16 Näille kertoivat näkijät, mitä oli tapahtunut riivatulle ja kuinka sikojen oli käynyt.
\par 17 Ja he alkoivat pyytää häntä poistumaan heidän alueeltaan.
\par 18 Ja hänen astuessaan venheeseen se riivattuna ollut pyysi häneltä saada olla hänen kanssaan.
\par 19 Mutta hän ei sitä sallinut, vaan sanoi hänelle: "Mene kotiisi omaistesi luo ja kerro heille, kuinka suuria tekoja Herra on sinulle tehnyt ja kuinka hän on sinua armahtanut".
\par 20 Niin hän lähti ja rupesi Dekapolin alueella julistamaan, kuinka suuria tekoja Jeesus oli hänelle tehnyt; ja kaikki ihmettelivät.
\par 21 Kun Jeesus oli venheellä kulkenut takaisin toiselle puolelle, kokoontui paljon kansaa hänen luoksensa, ja hän oli järven rannalla.
\par 22 Niin tuli muuan synagoogan esimies, nimeltä Jairus, ja lankesi hänet nähdessään hänen jalkojensa juureen,
\par 23 pyysi häntä hartaasti ja sanoi: "Pieni tyttäreni on kuolemaisillaan; tule ja pane kätesi hänen päällensä, että hän tulisi terveeksi ja jäisi eloon".
\par 24 Niin hän lähti hänen kanssansa. Ja häntä seurasi suuri kansan paljous, ja he tunkeutuivat hänen ympärilleen.
\par 25 Ja siellä oli nainen, joka oli sairastanut verenjuoksua kaksitoista vuotta
\par 26 ja paljon kärsinyt monen lääkärin käsissä ja kuluttanut kaiken omaisuutensa saamatta mitään apua, pikemminkin käyden huonommaksi.
\par 27 Tämä oli kuullut Jeesuksesta ja tuli kansanjoukossa takaapäin ja koski hänen vaippaansa;
\par 28 sillä hän sanoi: "Kunhan vain saan koskettaa edes hänen vaatteitaan, niin tulen terveeksi".
\par 29 Ja heti hänen verensä lähde kuivui, ja hän tunsi ruumiissansa, että oli parantunut vaivastaan.
\par 30 Ja heti kun Jeesus itsessään tunsi, että voimaa oli hänestä lähtenyt, kääntyi hän väkijoukossa ja sanoi: "Kuka koski minun vaatteisiini?"
\par 31 Niin hänen opetuslapsensa sanoivat hänelle: "Sinä näet kansanjoukon tungeskelevan ympärilläsi ja sanot: 'Kuka minuun koski?'"
\par 32 Mutta hän katseli ympärilleen nähdäksensä, kuka sen oli tehnyt.
\par 33 Niin nainen pelkäsi ja vapisi, koska hän tiesi, mitä hänelle oli tapahtunut, ja tuli ja lankesi maahan hänen eteensä ja sanoi hänelle koko totuuden.
\par 34 Mutta Jeesus sanoi hänelle: "Tyttäreni, sinun uskosi on tehnyt sinut terveeksi. Mene rauhaan ja ole terve vaivastasi."
\par 35 Hänen vielä puhuessaan tultiin synagoogan esimiehen kotoa sanomaan: "Tyttäresi kuoli; miksi enää opettajaa vaivaat?"
\par 36 Mutta Jeesus ei ottanut kuullakseen, mitä puhuttiin, vaan sanoi synagoogan esimiehelle: "Älä pelkää, usko ainoastaan".
\par 37 Ja hän ei sallinut kenenkään muun seurata mukanansa kuin Pietarin ja Jaakobin ja Johanneksen, Jaakobin veljen.
\par 38 Ja he tulivat synagoogan esimiehen taloon; ja hän näki hälisevän joukon ja ääneensä itkeviä ja vaikeroivia.
\par 39 Ja käydessään sisään hän sanoi heille: "Mitä te hälisette ja itkette? Lapsi ei ole kuollut, vaan nukkuu."
\par 40 Niin he nauroivat häntä. Mutta hän ajoi kaikki ulos ja otti mukaansa lapsen isän ja äidin sekä ne, jotka olivat hänen kanssaan, ja meni sisälle sinne, missä lapsi makasi.
\par 41 Ja hän tarttui lapsen käteen ja sanoi hänelle: "Talita kuum!" Se on käännettynä: Tyttö, minä sanon sinulle, nouse.
\par 42 Ja heti tyttö nousi ja käveli. Sillä hän oli kaksitoistavuotias. Ja he joutuivat suuren hämmästyksen valtaan.
\par 43 Ja hän kielsi ankarasti heitä antamasta kenellekään tietoa tästä ja käski antaa tytölle syötävää.

\chapter{6}

\par 1 Ja hän lähti sieltä ja meni kotikaupunkiinsa, ja hänen opetuslapsensa seurasivat häntä.
\par 2 Ja kun tuli sapatti, rupesi hän opettamaan synagoogassa; ja häntä kuullessaan monet hämmästyivät ja sanoivat: "Mistä tällä on kaikki tämä, ja mikä on se viisaus, joka on hänelle annettu? Ja mitä senkaltaiset voimalliset teot, jotka tapahtuvat hänen kättensä kautta?
\par 3 Eikö tämä ole se rakentaja, Marian poika ja Jaakobin ja Jooseen ja Juudaan ja Simonin veli? Ja eivätkö hänen sisarensa ole täällä meidän parissamme?" Ja he loukkaantuivat häneen.
\par 4 Niin Jeesus sanoi heille: "Ei ole profeetta halveksittu muualla kuin kotikaupungissaan ja sukulaistensa kesken ja kodissaan".
\par 5 Ja hän ei voinut siellä tehdä mitään voimallista tekoa, paitsi että paransi joitakuita sairaita panemalla kätensä heidän päälleen.
\par 6 Ja hän ihmetteli heidän epäuskoansa. Ja hän vaelsi ympäristössä, kulkien kylästä kylään, ja opetti.
\par 7 Ja hän kutsui tykönsä ne kaksitoista ja alkoi lähettää heitä kaksittain ja antoi heille vallan saastaisia henkiä vastaan.
\par 8 Ja hän sääti heille, etteivät saaneet ottaa matkalle muuta kuin ainoastaan sauvan; ei leipää, ei laukkua, ei rahaa vyöhönsä.
\par 9 He saivat kuitenkin sitoa paula-anturat jalkaansa; "mutta älkää pukeko kahta ihokasta yllenne".
\par 10 Ja hän sanoi heille: "Missä tulette taloon, jääkää siihen, kunnes lähdette pois siltä paikkakunnalta.
\par 11 Ja missä paikassa teitä ei oteta vastaan eikä teitä kuulla, sieltä menkää pois ja pudistakaa tomu jalkojenne alta, todistukseksi heille."
\par 12 Niin he lähtivät ja saarnasivat, että oli tehtävä parannus.
\par 13 Ja he ajoivat ulos monta riivaajaa ja voitelivat monta sairasta öljyllä ja paransivat heidät.
\par 14 Ja kuningas Herodes sai kuulla hänestä, sillä hänen nimensä oli tullut tunnetuksi, ja ihmiset sanoivat: "Johannes Kastaja on noussut kuolleista, ja sentähden nämä voimat hänessä vaikuttavat".
\par 15 Mutta toiset sanoivat: "Se on Elias"; toiset taas sanoivat: "Se on profeetta, niinkuin joku muukin profeetoista".
\par 16 Mutta kun Herodes sen kuuli, sanoi hän: "Johannes, jonka minä mestautin, on noussut kuolleista".
\par 17 Sillä hän, Herodes, oli lähettänyt ottamaan kiinni Johanneksen, sitonut ja pannut hänet vankeuteen veljensä Filippuksen vaimon, Herodiaan, tähden. Sillä Herodes oli nainut hänet,
\par 18 ja Johannes oli sanonut Herodekselle: "Sinun ei ole lupa pitää veljesi vaimoa".
\par 19 Ja Herodias piti vihaa häntä vastaan ja tahtoi tappaa hänet, mutta ei voinut.
\par 20 Sillä Herodes pelkäsi Johannesta, koska tiesi hänet vanhurskaaksi ja pyhäksi mieheksi, ja suojeli häntä. Ja kun hän kuunteli häntä, tuli hän epäröivälle mielelle monesta asiasta; ja hän kuunteli häntä mielellään.
\par 21 Niin tuli sopiva päivä, kun Herodes syntymäpäivänään piti pitoja ylimyksilleen ja sotapäälliköille ja Galilean ensimmäisille miehille.
\par 22 Ja Herodiaan tytär tuli sisälle ja tanssi, ja se miellytti Herodesta ja hänen pöytävieraitaan. Niin kuningas sanoi tytölle: "Ano minulta, mitä ikinä tahdot, niin minä annan sinulle".
\par 23 Ja hän vannoi tytölle: "Mitä ikinä minulta anot, sen minä annan sinulle, vaikka puolet valtakuntaani".
\par 24 Niin hän meni ulos ja sanoi äidilleen: "Mitä minä anon?" Tämä sanoi: "Johannes Kastajan päätä".
\par 25 Ja hän meni kohta kiiruusti sisälle kuninkaan luo, pyysi ja sanoi: "Minä tahdon, että nyt heti annat minulle lautasella Johannes Kastajan pään".
\par 26 Silloin kuningas tuli hyvin murheelliseksi, mutta valansa ja pöytävierasten tähden hän ei tahtonut hyljätä hänen pyyntöään.
\par 27 Ja kohta kuningas lähetti henkivartijan ja käski tuoda Johanneksen pään.
\par 28 Niin vartija meni ja löi häneltä pään poikki vankilassa ja toi hänen päänsä lautasella ja antoi sen tytölle, ja tyttö antoi sen äidillensä.
\par 29 Kun hänen opetuslapsensa sen kuulivat, tulivat he ja ottivat hänen ruumiinsa ja panivat sen hautaan.
\par 30 Ja apostolit kokoontuivat Jeesuksen tykö ja kertoivat hänelle kaikki, mitä olivat tehneet ja mitä olivat opettaneet.
\par 31 Niin hän sanoi heille: "Tulkaa te yksinäisyyteen, autioon paikkaan, ja levähtäkää vähän". Sillä tulijoita ja menijöitä oli paljon, ja heillä ei ollut aikaa syödäkään.
\par 32 Ja he lähtivät venheellä autioon paikkaan, yksinäisyyteen.
\par 33 Ja he näkivät heidän lähtevän, ja monet saivat siitä tiedon ja riensivät sinne jalkaisin kaikista kaupungeista ja saapuivat ennen heitä.
\par 34 Ja astuessaan maihin hän näki paljon kansaa, ja hänen kävi heitä sääliksi, koska he olivat niinkuin lampaat, joilla ei ole paimenta, ja hän rupesi opettamaan heille moninaisia.
\par 35 Ja kun päivä jo oli pitkälle kulunut, menivät hänen opetuslapsensa hänen tykönsä ja sanoivat: "Tämä paikka on autio, ja aika on jo myöhäinen;
\par 36 laske heidät luotasi, että he menisivät ympäristöllä oleviin maataloihin ja kyliin ostamaan itsellensä syötävää".
\par 37 Mutta hän vastasi heille ja sanoi: "Antakaa te heille syödä". Niin he sanoivat hänelle: "Lähdemmekö ostamaan leipää kahdellasadalla denarilla antaaksemme heille syödä?"
\par 38 Mutta hän sanoi heille: "Montako leipää teillä on? Menkää katsomaan." Otettuaan siitä selvän he sanoivat: "Viisi, ja kaksi kalaa".
\par 39 Niin hän määräsi heille, että kaikkien oli asetuttava ruokakunnittain vihantaan ruohikkoon.
\par 40 Ja he laskeutuivat ryhmä ryhmän viereen, toisiin sata, toisiin viisikymmentä.
\par 41 Ja hän otti ne viisi leipää ja kaksi kalaa, katsoi ylös taivaaseen ja siunasi ja mursi leivät ja antoi ne opetuslapsilleen kansan eteen pantaviksi; myöskin ne kaksi kalaa hän jakoi kaikille.
\par 42 Ja kaikki söivät ja tulivat ravituiksi.
\par 43 Sitten he keräsivät palaset, kaksitoista täyttä vakallista, ja tähteet kaloista.
\par 44 Ja niitä, jotka olivat syöneet näitä leipiä, oli viisituhatta miestä.
\par 45 Ja kohta hän vaati opetuslapsiansa astumaan venheeseen ja kulkemaan edeltä toiselle rannalle, Beetsaidaan, sillä aikaa kuin hän laski kansan luotansa.
\par 46 Ja sanottuaan heille jäähyväiset hän meni pois vuorelle rukoilemaan.
\par 47 Ja kun ilta tuli, oli venhe keskellä järveä, ja hän oli yksinään maalla.
\par 48 Ja kun hän näki heidän soutaessaan olevan hädässä, sillä tuuli oli heille vastainen, tuli hän neljännen yövartion vaiheilla heidän luoksensa kävellen järven päällä ja aikoi kulkea heidän ohitsensa.
\par 49 Mutta nähdessään hänen kävelevän järven päällä he luulivat häntä aaveeksi ja rupesivat huutamaan;
\par 50 sillä kaikki näkivät hänet ja peljästyivät. Mutta heti hän puhutteli heitä ja sanoi heille: "Olkaa turvallisella mielellä, minä se olen; älkää peljätkö".
\par 51 Ja hän astui venheeseen heidän tykönsä, ja tuuli asettui. Niin he hämmästyivät ylen suuresti sydämessään.
\par 52 Sillä he eivät olleet noista leivistäkään päässeet ymmärrykseen, vaan heidän sydämensä oli paatunut.
\par 53 Ja kuljettuaan yli toiselle rannalle he tulivat Gennesaretiin ja laskivat maihin.
\par 54 Ja heidän noustessaan venheestä kansa heti tunsi hänet;
\par 55 ja he riensivät kiertämään koko sitä paikkakuntaa ja rupesivat vuoteilla kantamaan sairaita sinne, missä kuulivat hänen olevan.
\par 56 Ja missä vain hän meni kyliin tai kaupunkeihin tai maataloihin, asetettiin sairaat aukeille paikoille ja pyydettiin häneltä, että he saisivat koskea edes hänen vaippansa tupsuun. Ja kaikki, jotka koskivat häneen, tulivat terveiksi.

\chapter{7}

\par 1 Ja fariseukset ja muutamat kirjanoppineet, jotka olivat tulleet Jerusalemista, kokoontuivat hänen luoksensa.
\par 2 Ja he näkivät, että muutamat hänen opetuslapsistaan söivät leipää epäpuhtailla, se on pesemättömillä, käsillä.
\par 3 Sillä eivät fariseukset eivätkä ketkään juutalaiset syö, ennenkuin ovat tarkoin pesseet kätensä, noudattaen vanhinten perinnäissääntöä,
\par 4 ja torilta tultuaan he eivät syö, ennenkuin ovat itseään vedellä vihmoneet; ja paljon muuta on, mitä he ovat ottaneet noudattaakseen, niinkuin maljain ja kiviastiain ja vaskiastiain pesemisiä.
\par 5 Niin fariseukset ja kirjanoppineet kysyivät häneltä: "Miksi sinun opetuslapsesi eivät vaella vanhinten perinnäissäännön mukaan, vaan syövät leipää epäpuhtailla käsillä?"
\par 6 Mutta hän sanoi heille: "Oikein Esaias on ennustanut teistä, ulkokullatuista, niinkuin kirjoitettu on: 'Tämä kansa kunnioittaa minua huulillaan, mutta heidän sydämensä on minusta kaukana;
\par 7 mutta turhaan he palvelevat minua opettaen oppeja, jotka ovat ihmiskäskyjä'.
\par 8 Te hylkäätte Jumalan käskyn ja noudatatte ihmisten perinnäissääntöä."
\par 9 Ja hän sanoi heille: "Hyvin te kumoatte Jumalan käskyn noudattaaksenne omaa perinnäissääntöänne.
\par 10 Sillä Mooses on sanonut: 'Kunnioita isääsi ja äitiäsi', ja: 'Joka kiroaa isäänsä tai äitiänsä, sen pitää kuolemalla kuoleman'.
\par 11 Mutta te sanotte, että jos ihminen sanoo isälleen tai äidilleen: 'Se, minkä sinä olisit saava minulta hyväksesi, on korban' - se on uhrilahja --
\par 12 ja niin te ette enää salli hänen antaa mitään avustusta isälleen tai äidilleen.
\par 13 Te teette Jumalan sanan tyhjäksi perinnäissäännöllänne, jonka olette säätäneet. Ja paljon muuta samankaltaista te teette."
\par 14 Ja hän kutsui taas kansan tykönsä ja sanoi heille: "Kuulkaa minua kaikki ja ymmärtäkää:
\par 15 ei mikään, mikä ihmisen ulkopuolelta menee hänen sisäänsä, voi häntä saastuttaa, vaan mikä ihmisestä lähtee ulos, se saastuttaa ihmisen.
\par 16 Jos jollakin on korvat kuulla, hän kuulkoon."
\par 17 Ja kun hän kansasta erottuaan oli mennyt erääseen taloon, kysyivät hänen opetuslapsensa häneltä sitä vertausta.
\par 18 Ja hän sanoi heille: "Niinkö ymmärtämättömiä tekin olette? Ettekö käsitä, ettei mikään, mikä ulkoapäin menee ihmiseen, voi häntä saastuttaa?
\par 19 Sillä se ei mene hänen sydämeensä, vaan vatsaan, ja ulostuu." Näin hän sanoi kaikki ruuat puhtaiksi.
\par 20 Ja hän sanoi: "Mikä ihmisestä lähtee ulos, se saastuttaa ihmisen.
\par 21 Sillä sisästä, ihmisten sydämestä, lähtevät pahat ajatukset, haureudet, varkaudet, murhat,
\par 22 aviorikokset, ahneus, häijyys, petollisuus, irstaus, pahansuonti, jumalanpilkka, ylpeys, mielettömyys.
\par 23 Kaikki tämä paha lähtee sisästä ulos ja saastuttaa ihmisen."
\par 24 Ja hän nousi ja lähti sieltä Tyyron ja Siidonin alueelle. Ja hän meni erääseen taloon eikä tahtonut, että kukaan saisi sitä tietää; mutta hän ei saanut olla salassa,
\par 25 vaan heti kun eräs vaimo, jonka pienessä tyttäressä oli saastainen henki, kuuli hänestä, tuli hän ja lankesi hänen jalkojensa juureen.
\par 26 Ja se vaimo oli kreikatar, syntyään syyrofoinikialainen; ja hän pyysi häntä ajamaan ulos riivaajan hänen tyttärestään.
\par 27 Niin Jeesus sanoi hänelle: "Anna ensin lasten tulla ravituiksi; sillä ei ole soveliasta ottaa lasten leipää ja heittää penikoille".
\par 28 Mutta hän vastasi ja sanoi hänelle: "Niin, Herra; mutta syöväthän penikatkin pöydän alla lasten muruja".
\par 29 Ja hän sanoi vaimolle: "Tämän sanan tähden, mene; riivaaja on lähtenyt sinun tyttärestäsi".
\par 30 Ja vaimo meni kotiinsa ja havaitsi lapsen makaavan vuoteella ja riivaajan lähteneen hänestä.
\par 31 Ja hän lähti jälleen Tyyron alueelta ja kulkien Siidonin kautta tuli Galilean järven ääreen Dekapolin alueen keskitse.
\par 32 Ja hänen tykönsä tuotiin kuuro, joka oli melkein mykkä, ja he pyysivät häntä panemaan kätensä hänen päälleen.
\par 33 Niin hän otti hänet erilleen kansasta, pisti sormensa hänen korviinsa, sylki ja koski hänen kieleensä
\par 34 ja katsahti ylös taivaaseen, huokasi ja sanoi hänelle: "Effata", se on: aukene.
\par 35 Niin hänen korvansa aukenivat, ja hänen kielensä side irtautui, ja hän puhui selkeästi.
\par 36 Ja Jeesus kielsi heitä sitä kenellekään sanomasta; mutta mitä enemmän hän heitä kielsi, sitä enemmän he julistivat.
\par 37 Ja ihmiset hämmästyivät ylenmäärin ja sanoivat: "Hyvin hän on kaikki tehnyt: kuurot hän saa kuulemaan ja mykät puhumaan".

\chapter{8}

\par 1 Niinä päivinä, kun paljon kansaa taas oli koolla eikä heillä ollut mitään syötävää, kutsui hän opetuslapsensa tykönsä ja sanoi heille:
\par 2 "Minun käy sääliksi kansaa, sillä he ovat jo kolme päivää olleet minun tykönäni, eikä heillä ole mitään syötävää.
\par 3 Ja jos minä lasken heidät menemään syömättä kotiinsa, niin he nääntyvät matkalla; sillä muutamat heistä ovat tulleet kaukaa."
\par 4 Niin hänen opetuslapsensa vastasivat hänelle: "Mistä täällä erämaassa kukaan voi saada leipää näiden ravitsemiseksi?"
\par 5 Hän kysyi heiltä: "Montako leipää teillä on?" He sanoivat: "Seitsemän".
\par 6 Silloin hän käski kansan asettua maahan. Ja hän otti ne seitsemän leipää, kiitti, mursi ja antoi opetuslapsillensa, että he panisivat ne kansan eteen. Ja he panivat.
\par 7 Heillä oli myös joitakuita kalasia; ja siunattuaan ne hän käski panna nekin kansan eteen.
\par 8 Niin he söivät ja tulivat ravituiksi. Sitten he keräsivät jääneet palaset, seitsemän vasullista.
\par 9 Ja heitä oli noin neljätuhatta. Ja hän laski heidät luotansa.
\par 10 Ja kohta hän astui opetuslapsineen venheeseen ja meni Dalmanutan seuduille.
\par 11 Ja fariseukset lähtivät sinne ja rupesivat väittelemään hänen kanssaan ja vaativat häneltä merkkiä taivaasta, kiusaten häntä.
\par 12 Niin hän huokasi hengessään ja sanoi: "Miksi tämä sukupolvi vaatii merkkiä? Totisesti minä sanon teille: tälle sukupolvelle ei anneta merkkiä."
\par 13 Ja hän jätti heidät ja astui taas venheeseen ja lähti pois toiselle rannalle.
\par 14 Ja he olivat unhottaneet ottaa mukaansa leipää, eikä heillä ollut muassaan venheessä enempää kuin yksi leipä.
\par 15 Ja hän käski heitä sanoen: "Varokaa ja kavahtakaa fariseusten hapatusta ja Herodeksen hapatusta".
\par 16 Niin he puhuivat keskenään siitä, ettei heillä ollut leipää.
\par 17 Kun Jeesus huomasi sen, sanoi hän heille: "Mitä puhutte siitä, ettei teillä ole leipää? Ettekö vielä käsitä ettekä ymmärrä? Onko teidän sydämenne paatunut?
\par 18 Silmät teillä on, ettekö näe? Ja korvat teillä on, ettekö kuule? Ja ettekö muista:
\par 19 kun minä mursin ne viisi leipää viidelletuhannelle, kuinka monta vakan täyttä palasia te keräsitte?" He sanoivat hänelle: "Kaksitoista".
\par 20 "Ja kun minä mursin ne seitsemän leipää neljälletuhannelle, kuinka monta vasun täyttä palasia te keräsitte?" He sanoivat: "Seitsemän".
\par 21 Niin hän sanoi heille: "Ettekö vieläkään ymmärrä?"
\par 22 Ja he tulivat Beetsaidaan. Ja hänen tykönsä tuotiin sokea, ja he pyysivät, että hän koskisi häneen.
\par 23 Niin hän tarttui sokean käteen, talutti hänet kylän ulkopuolelle, sylki hänen silmiinsä ja pani kätensä hänen päälleen ja kysyi häneltä: "Näetkö mitään?"
\par 24 Tämä katsahti ylös ja sanoi: "Näen ihmiset, sillä minä erotan käveleviä, ne ovat puiden näköisiä".
\par 25 Sitten hän taas pani kätensä hänen silmilleen; ja nyt mies näki tarkkaan ja oli parantunut ja näki kaikki aivan selvästi.
\par 26 Ja hän lähetti hänet hänen kotiinsa sanoen: "Älä edes poikkea kylään".
\par 27 Ja Jeesus lähti opetuslapsinensa Filippuksen Kesarean kyliin. Ja tiellä hän kysyi opetuslapsiltaan ja sanoi heille: "Kenen ihmiset sanovat minun olevan?"
\par 28 He vastasivat hänelle sanoen: "Johannes Kastajan, ja toiset Eliaan, toiset taas jonkun profeetoista".
\par 29 Niin hän kysyi heiltä: "Kenenkä te sanotte minun olevan?" Pietari vastasi ja sanoi hänelle: "Sinä olet Kristus".
\par 30 Ja hän varoitti vakavasti heitä puhumasta kenellekään hänestä.
\par 31 Ja hän alkoi opettaa heille, että Ihmisen Pojan piti kärsimän paljon ja joutuman vanhinten ja ylipappien ja kirjanoppineiden hyljittäväksi ja tuleman tapetuksi, ja kolmen päivän perästä nouseman ylös.
\par 32 Ja tämän hän puhui peittelemättä. Silloin Pietari otti hänet erilleen ja rupesi häntä nuhtelemaan.
\par 33 Mutta hän kääntyi, katsoi opetuslapsiinsa ja nuhteli Pietaria sanoen: "Mene pois minun edestäni, saatana, sillä sinä et ajattele sitä, mikä on Jumalan, vaan sitä, mikä on ihmisten".
\par 34 Ja hän kutsui tykönsä kansan ynnä opetuslapsensa ja sanoi heille: "Jos joku tahtoo minun perässäni kulkea, hän kieltäköön itsensä ja ottakoon ristinsä ja seuratkoon minua.
\par 35 Sillä joka tahtoo pelastaa elämänsä, hän kadottaa sen, mutta joka kadottaa elämänsä minun ja evankeliumin tähden, hän pelastaa sen.
\par 36 Sillä mitä se hyödyttää ihmistä, vaikka hän voittaisi omaksensa koko maailman, mutta saisi vahingon sielullensa?
\par 37 Sillä mitä voi ihminen antaa sielunsa lunnaiksi?
\par 38 Sillä joka häpeää minua ja minun sanojani tässä avionrikkojassa ja syntisessä sukupolvessa, sitä myös Ihmisen Poika on häpeävä, kun hän tulee Isänsä kirkkaudessa pyhien enkelien kanssa."

\chapter{9}

\par 1 Ja hän sanoi heille: "Totisesti minä sanon teille: tässä seisovien joukossa on muutamia, jotka eivät maista kuolemaa, ennenkuin näkevät Jumalan valtakunnan tulevan voimassansa".
\par 2 Ja kuuden päivän kuluttua Jeesus otti mukaansa Pietarin ja Jaakobin ja Johanneksen ja vei heidät erilleen muista korkealle vuorelle, yksinäisyyteen. Ja hänen muotonsa muuttui heidän edessään;
\par 3 ja hänen vaatteensa tulivat hohtaviksi, niin ylen valkoisiksi, ettei kukaan vaatteenvalkaisija maan päällä taida semmoiseksi valkaista.
\par 4 Ja heille ilmestyivät Elias ynnä Mooses, ja nämä puhuivat Jeesuksen kanssa.
\par 5 Niin Pietari rupesi puhumaan ja sanoi Jeesukselle: "Rabbi, meidän on tässä hyvä olla; tehkäämme siis kolme majaa, sinulle yksi ja Moosekselle yksi ja Eliaalle yksi".
\par 6 Sillä hän ei tiennyt, mitä sanoa, koska he olivat peljästyksissään.
\par 7 Ja tuli pilvi, joka peitti heidät varjoonsa, ja pilvestä kuului ääni: "Tämä on minun rakas Poikani; kuulkaa häntä".
\par 8 Ja yhtäkkiä, kun he katsahtivat ympärilleen, eivät he enää nähneet ketään muuta kuin Jeesuksen, joka yksinänsä oli heidän kanssaan.
\par 9 Ja heidän kulkiessaan alas vuorelta hän teroitti heille, etteivät kenellekään kertoisi, mitä olivat nähneet, ennenkuin vasta sitten, kun Ihmisen Poika oli noussut kuolleista.
\par 10 Ja he pitivät mielessään sen sanan ja tutkistelivat keskenään, mitä kuolleista nouseminen oli.
\par 11 Ja he kysyivät häneltä sanoen: "Kirjanoppineethan sanovat, että Eliaan pitää tuleman ensin?"
\par 12 Niin hän sanoi heille: "Elias tosin tulee ensin ja asettaa kaikki kohdalleen. Mutta kuinka sitten on kirjoitettu Ihmisen Pojasta, että hän on paljon kärsivä ja tuleva halveksituksi?
\par 13 Mutta minä sanon teille: Elias onkin tullut, ja he tekivät hänelle, mitä tahtoivat, niinkuin hänestä on kirjoitettu."
\par 14 Ja kun he tulivat opetuslasten luo, näkivät he paljon kansaa heidän ympärillään ja kirjanoppineita väittelemässä heidän kanssaan.
\par 15 Ja kohta kun kaikki kansa hänet näki, hämmästyivät he ja riensivät hänen luoksensa ja tervehtivät häntä.
\par 16 Ja hän kysyi heiltä: "Mitä te väittelette heidän kanssaan?"
\par 17 Silloin vastasi eräs mies kansanjoukosta hänelle: "Opettaja, minä toin sinun tykösi poikani, jossa on mykkä henki.
\par 18 Ja missä vain se käy hänen kimppuunsa, riuhtoo se häntä, ja hänestä lähtee vaahto, ja hän kiristelee hampaitaan; ja hän kuihtuu. Ja minä sanoin sinun opetuslapsillesi, että he ajaisivat sen ulos, mutta he eivät kyenneet."
\par 19 Hän vastasi heille sanoen: "Voi, sinä epäuskoinen sukupolvi, kuinka kauan minun täytyy olla teidän luonanne? Kuinka kauan kärsiä teitä? Tuokaa hänet minun tyköni."
\par 20 Niin he toivat hänet hänen tykönsä. Ja heti kun hän näki Jeesuksen, kouristi henki häntä, ja hän kaatui maahan, kieritteli itseään, ja hänestä lähti vaahto.
\par 21 Ja Jeesus kysyi hänen isältään: "Kuinka kauan aikaa tätä on hänessä ollut?" Niin hän sanoi: "Pienestä pitäen.
\par 22 Ja monesti se on heittänyt hänet milloin tuleen, milloin veteen, tuhotakseen hänet. Mutta jos sinä jotakin voit, niin armahda meitä ja auta meitä."
\par 23 Niin Jeesus sanoi hänelle: "'Jos voit!' Kaikki on mahdollista sille, joka uskoo".
\par 24 Ja heti lapsen isä huusi ja sanoi: "Minä uskon; auta minun epäuskoani".
\par 25 Mutta kun Jeesus näki, että kansaa riensi sinne, nuhteli hän saastaista henkeä ja sanoi sille: "Sinä mykkä ja kuuro henki, minä käsken sinua: lähde ulos hänestä, äläkä enää häneen mene".
\par 26 Niin se huusi ja kouristi häntä kovasti ja lähti ulos. Ja hän kävi ikäänkuin kuolleeksi, niin että monet sanoivat: "Hän kuoli".
\par 27 Mutta Jeesus tarttui hänen käteensä ja nosti hänet ylös. Ja hän nousi.
\par 28 Ja kun Jeesus oli mennyt huoneeseen, niin hänen opetuslapsensa kysyivät häneltä eriksensä: "Miksi emme me voineet ajaa sitä ulos?"
\par 29 Hän sanoi heille: "Tätä lajia ei saa lähtemään ulos muulla kuin rukouksella ja paastolla".
\par 30 Ja he lähtivät sieltä ja kulkivat Galilean läpi; ja hän ei tahtonut, että kukaan saisi sitä tietää.
\par 31 Sillä hän opetti opetuslapsiaan ja sanoi heille: "Ihmisen Poika annetaan ihmisten käsiin, ja he tappavat hänet; ja kun hän on tapettu, nousee hän kolmen päivän perästä ylös".
\par 32 Mutta he eivät käsittäneet sitä puhetta ja pelkäsivät häneltä kysyä.
\par 33 Ja he saapuivat Kapernaumiin. Ja kotiin tultuaan hän kysyi heiltä: "Mistä te tiellä keskustelitte?"
\par 34 Mutta he olivat vaiti; sillä he olivat tiellä keskustelleet toistensa kanssa siitä, kuka oli suurin.
\par 35 Ja hän istuutui, kutsui ne kaksitoista ja sanoi heille: "Jos joku tahtoo olla ensimmäinen, on hänen oltava kaikista viimeinen ja kaikkien palvelija".
\par 36 Ja hän otti lapsen ja asetti sen heidän keskellensä; ja otettuaan sen syliinsä hän sanoi heille:
\par 37 "Joka ottaa tykönsä yhden tämänkaltaisen lapsen minun nimeeni, se ottaa tykönsä minut; ja joka minut ottaa tykönsä, se ei ota tykönsä minua, vaan hänet, joka on minut lähettänyt".
\par 38 Johannes sanoi hänelle: "Opettaja, me näimme erään, joka ei seuraa meitä, sinun nimessäsi ajavan ulos riivaajia; ja me kielsimme häntä, koska hän ei seurannut meitä".
\par 39 Mutta Jeesus sanoi: "Älkää häntä kieltäkö; sillä ei kukaan, joka tekee voimallisen teon minun nimeeni, voi kohta sen jälkeen puhua minusta pahaa.
\par 40 Sillä joka ei ole meitä vastaan, se on meidän puolellamme.
\par 41 Sillä joka antaa teille juodaksenne maljallisen vettä siinä nimessä, että te olette Kristuksen omia, totisesti minä sanon teille: se ei jää palkkaansa vaille.
\par 42 Ja joka viettelee yhden näistä pienistä, jotka uskovat, sen olisi parempi, että myllynkivi olisi pantu hänen kaulaansa ja hänet olisi heitetty mereen.
\par 43 Ja jos sinun kätesi viettelee sinua, hakkaa se poikki. Parempi on sinulle, että käsipuolena menet elämään sisälle, kuin että, molemmat kädet tallella, joudut helvettiin, sammumattomaan tuleen.
\par 45 Ja jos sinun jalkasi viettelee sinua, hakkaa se poikki. Parempi on sinulle, että jalkapuolena menet elämään sisälle, kuin että sinut, molemmat jalat tallella, heitetään helvettiin.
\par 47 Ja jos sinun silmäsi viettelee sinua, heitä se pois. Parempi on sinulle, että silmäpuolena menet sisälle Jumalan valtakuntaan, kuin että sinut, molemmat silmät tallella, heitetään helvettiin,
\par 48 jossa heidän matonsa ei kuole eikä tuli sammu.
\par 49 Sillä jokainen ihminen on tulella suolattava, ja jokainen uhri on suolalla suolattava.
\par 50 Suola on hyvä; mutta jos suola käy suolattomaksi, millä te sen maustatte? Olkoon teillä suola itsessänne, ja pitäkää keskenänne rauha."

\chapter{10}

\par 1 Ja hän nousi sieltä ja tuli Juudean alueelle, kulkien Jordanin toista puolta. Ja taas kokoontui paljon kansaa hänen luoksensa, ja tapansa mukaan hän taas opetti heitä.
\par 2 Ja fariseuksia tuli hänen luoksensa, ja kiusaten häntä he kysyivät häneltä, oliko miehen lupa hyljätä vaimonsa.
\par 3 Hän vastasi ja sanoi heille: "Mitä Mooses on teille säätänyt?"
\par 4 He sanoivat: "Mooses salli kirjoittaa erokirjan ja hyljätä vaimon".
\par 5 Niin Jeesus sanoi heille: "Teidän sydämenne kovuuden tähden hän kirjoitti teille tämän säädöksen.
\par 6 Mutta luomakunnan alusta Jumala 'on luonut heidät mieheksi ja naiseksi.
\par 7 Sentähden mies luopukoon isästänsä ja äidistänsä ja liittyköön vaimoonsa.
\par 8 Ja ne kaksi tulevat yhdeksi lihaksi.' Niin eivät he enää ole kaksi, vaan yksi liha.
\par 9 Minkä siis Jumala on yhdistänyt, sitä älköön ihminen erottako."
\par 10 Ja heidän mentyään huoneeseen opetuslapset taas kysyivät häneltä tätä asiaa.
\par 11 Ja hän sanoi heille: "Joka hylkää vaimonsa ja nai toisen, se tekee huorin häntä vastaan.
\par 12 Ja jos vaimo hylkää miehensä ja menee naimisiin toisen kanssa, niin hän tekee huorin."
\par 13 Ja he toivat hänen tykönsä lapsia, että hän koskisi heihin; mutta opetuslapset nuhtelivat tuojia.
\par 14 Mutta kun Jeesus sen näki, närkästyi hän ja sanoi heille: "Sallikaa lasten tulla minun tyköni, älkääkä estäkö heitä, sillä senkaltaisten on Jumalan valtakunta.
\par 15 Totisesti minä sanon teille: joka ei ota vastaan Jumalan valtakuntaa niinkuin lapsi, se ei pääse sinne sisälle."
\par 16 Ja hän otti heitä syliinsä, pani kätensä heidän päällensä ja siunasi heitä.
\par 17 Ja hänen sieltä tielle mennessään juoksi muuan hänen luoksensa, polvistui hänen eteensä ja kysyi häneltä: "Hyvä opettaja, mitä minun pitää tekemän, että minä iankaikkisen elämän perisin?"
\par 18 Mutta Jeesus sanoi hänelle: "Miksi sanot minua hyväksi? Ei kukaan ole hyvä paitsi Jumala yksin.
\par 19 Käskyt sinä tiedät: 'Älä tapa', 'Älä tee huorin', 'Älä varasta', 'Älä sano väärää todistusta', 'Älä toiselta anasta', 'Kunnioita isääsi ja äitiäsi'."
\par 20 Mutta hän sanoi hänelle: "Opettaja, niitä kaikkia minä olen noudattanut nuoruudestani asti".
\par 21 Niin Jeesus katsoi häneen ja rakasti häntä ja sanoi hänelle: "Yksi sinulta puuttuu: mene, myy kaikki, mitä sinulla on, ja anna köyhille, niin sinulla on oleva aarre taivaassa; ja tule ja seuraa minua".
\par 22 Mutta hän synkistyi siitä puheesta ja meni pois murheellisena, sillä hänellä oli paljon omaisuutta.
\par 23 Silloin Jeesus katsoi ympärilleen ja sanoi opetuslapsillensa: "Kuinka vaikea onkaan niiden, joilla on tavaraa, päästä Jumalan valtakuntaan!"
\par 24 Niin opetuslapset hämmästyivät hänen sanoistaan. Mutta Jeesus rupesi taas puhumaan ja sanoi heille: "Lapset, kuinka vaikea onkaan niiden, jotka luottavat tavaraansa, päästä Jumalan valtakuntaan!
\par 25 Helpompi on kamelin käydä neulansilmän läpi kuin rikkaan päästä Jumalan valtakuntaan."
\par 26 Niin he hämmästyivät yhä enemmän ja sanoivat toisillensa: "Kuka sitten voi pelastua?"
\par 27 Jeesus katsoi heihin ja sanoi: "Ihmisille se on mahdotonta, mutta ei Jumalalle; sillä Jumalalle on kaikki mahdollista".
\par 28 Niin Pietari rupesi puhumaan sanoen hänelle: "Katso, me olemme luopuneet kaikesta ja seuranneet sinua".
\par 29 Jeesus sanoi: "Totisesti minä sanon teille: ei ole ketään, joka minun tähteni ja evankeliumin tähden on luopunut talosta tai veljistä tai sisarista tai äidistä tai isästä tai lapsista tai pelloista,
\par 30 ja joka ei saisi satakertaisesti: nyt tässä ajassa taloja ja veljiä ja sisaria ja äitejä ja lapsia ja peltoja, vainojen keskellä, ja tulevassa maailmassa iankaikkista elämää.
\par 31 Mutta monet ensimmäiset tulevat viimeisiksi ja viimeiset ensimmäisiksi."
\par 32 Ja he olivat matkalla, menossa ylös Jerusalemiin, ja Jeesus kulki heidän edellään; ja heidät valtasi hämmästys, ja ne, jotka seurasivat, olivat peloissaan. Ja hän otti taas tykönsä ne kaksitoista ja rupesi heille puhumaan, mitä hänelle oli tapahtuva:
\par 33 "Katso, me menemme ylös Jerusalemiin, ja Ihmisen Poika annetaan ylipappien ja kirjanoppineitten käsiin, ja he tuomitsevat hänet kuolemaan ja antavat hänet pakanain käsiin;
\par 34 ja ne pilkkaavat häntä ja sylkevät häntä ja ruoskivat häntä ja tappavat hänet; ja kolmen päivän perästä hän on nouseva ylös".
\par 35 Ja Jaakob ja Johannes, Sebedeuksen pojat, menivät hänen luoksensa ja sanoivat hänelle: "Opettaja, me tahtoisimme, että tekisit meille, mitä sinulta anomme".
\par 36 Hän sanoi heille: "Mitä tahdotte, että minä teille tekisin?"
\par 37 Niin he sanoivat hänelle: "Anna meidän istua, toisen oikealla ja toisen vasemmalla puolellasi, sinun kirkkaudessasi".
\par 38 Mutta Jeesus sanoi heille: "Te ette tiedä, mitä anotte. Voitteko juoda sen maljan, jonka minä juon, tahi tulla kastetuiksi sillä kasteella, jolla minut kastetaan?"
\par 39 He sanoivat hänelle: "Voimme". Niin Jeesus sanoi heille: "Sen maljan, jonka minä juon, te tosin juotte, ja sillä kasteella, jolla minut kastetaan, kastetaan teidätkin;
\par 40 mutta minun oikealla tai vasemmalla puolellani istuminen ei ole minun annettavissani, vaan se annetaan niille, joille se on valmistettu".
\par 41 Kun ne kymmenen sen kuulivat, alkoivat he närkästyä Jaakobiin ja Johannekseen.
\par 42 Mutta Jeesus kutsui heidät tykönsä ja sanoi heille: "Te tiedätte, että ne, joita kansojen ruhtinaiksi katsotaan, herroina niitä hallitsevat, ja että kansojen mahtavat käyttävät valtaansa niitä kohtaan.
\par 43 Mutta näin ei ole teidän kesken, vaan joka teidän keskuudessanne tahtoo suureksi tulla, se olkoon teidän palvelijanne;
\par 44 ja joka teidän keskuudessanne tahtoo olla ensimmäinen, se olkoon kaikkien orja.
\par 45 Sillä ei Ihmisen Poikakaan tullut palveltavaksi, vaan palvelemaan ja antamaan henkensä lunnaiksi monen edestä."
\par 46 Ja he tulivat Jerikoon. Ja kun hän vaelsi Jerikosta opetuslastensa ja suuren väkijoukon seuraamana, istui sokea kerjäläinen, Bartimeus, Timeuksen poika, tien vieressä.
\par 47 Ja kun hän kuuli, että se oli Jeesus Nasaretilainen, rupesi hän huutamaan ja sanomaan: "Jeesus, Daavidin poika, armahda minua".
\par 48 Ja monet nuhtelivat häntä saadakseen hänet vaikenemaan. Mutta hän huusi vielä enemmän: "Daavidin poika, armahda minua".
\par 49 Silloin Jeesus seisahtui ja sanoi: "Kutsukaa hänet tänne". Ja he kutsuivat sokean, sanoen hänelle: "Ole turvallisella mielellä, nouse; hän kutsuu sinua".
\par 50 Niin hän heitti vaippansa päältään, kavahti seisomaan ja tuli Jeesuksen tykö.
\par 51 Ja Jeesus puhutteli häntä sanoen: "Mitä tahdot, että minä sinulle tekisin?" Niin sokea sanoi hänelle: "Rabbuuni, että saisin näköni jälleen".
\par 52 Niin Jeesus sanoi hänelle: "Mene, sinun uskosi on sinut pelastanut". Ja kohta hän sai näkönsä ja seurasi häntä tiellä.

\chapter{11}

\par 1 Ja kun he lähestyivät Jerusalemia, tullen Beetfageen ja Betaniaan Öljymäelle, lähetti hän kaksi opetuslastaan
\par 2 ja sanoi heille: "Menkää kylään, joka on edessänne, niin te kohta, kun sinne tulette, löydätte sidottuna varsan, jonka selässä ei yksikään ihminen vielä ole istunut; päästäkää se ja tuokaa tänne.
\par 3 Ja jos joku teille sanoo: 'Miksi te noin teette?', niin sanokaa: 'Herra tarvitsee sitä ja lähettää sen kohta tänne takaisin'."
\par 4 Niin he menivät ja löysivät oven eteen ulos kujalle sidotun varsan ja päästivät sen.
\par 5 Ja muutamat niistä, jotka siellä seisoivat, sanoivat heille: "Mitä te teette, kun päästätte varsan?"
\par 6 Niin he sanoivat heille, niinkuin Jeesus oli käskenyt; ja he antoivat heidän mennä.
\par 7 Ja he toivat varsan Jeesuksen luo ja heittivät vaatteensa sen päälle, ja hän istui sen selkään.
\par 8 Ja monet levittivät vaatteensa tielle, ja toiset lehviä, joita katkoivat kedoilta.
\par 9 Ja ne, jotka kulkivat edellä, ja jotka seurasivat, huusivat: "Hoosianna, siunattu olkoon hän, joka tulee Herran nimeen!
\par 10 Siunattu olkoon isämme Daavidin valtakunta, joka tulee. Hoosianna korkeuksissa!"
\par 11 Ja hän kulki sisälle Jerusalemiin ja meni pyhäkköön; ja katseltuaan kaikkea hän lähti niiden kahdentoista kanssa Betaniaan, sillä aika oli jo myöhäinen.
\par 12 Kun he seuraavana päivänä lähtivät Betaniasta, oli hänen nälkä.
\par 13 Ja kun hän kaukaa näki viikunapuun, jossa oli lehtiä, meni hän katsomaan, löytäisikö ehkä jotakin siitä; mutta tultuaan sen luo hän ei löytänyt muuta kuin lehtiä. Sillä silloin ei ollut viikunain aika.
\par 14 Niin hän puhui ja sanoi sille: "Älköön ikinä enää kukaan sinusta hedelmää syökö". Ja hänen opetuslapsensa kuulivat sen.
\par 15 Ja he tulivat Jerusalemiin. Ja hän meni pyhäkköön ja rupesi ajamaan ulos niitä, jotka myivät ja ostivat pyhäkössä, ja kaatoi kumoon rahanvaihtajain pöydät ja kyyhkysten myyjäin istuimet,
\par 16 eikä sallinut kenenkään kantaa mitään astiaa pyhäkön kautta.
\par 17 Ja hän opetti ja sanoi heille: "Eikö ole kirjoitettu: 'Minun huoneeni on kutsuttava kaikkien kansojen rukoushuoneeksi'? Mutta te olette tehneet siitä ryövärien luolan."
\par 18 Ja ylipapit ja kirjanoppineet kuulivat sen ja miettivät, kuinka saisivat hänet surmatuksi; sillä he pelkäsivät häntä, koska kaikki kansa oli hämmästyksissään hänen opetuksestansa.
\par 19 Illan tultua he menivät kaupungin ulkopuolelle.
\par 20 Ja kun he varhain aamulla kulkivat ohi, näkivät he viikunapuun kuivettuneen juuria myöten.
\par 21 Silloin Pietari muisti Jeesuksen sanat ja sanoi hänelle: "Rabbi, katso, viikunapuu, jonka sinä kirosit, on kuivettunut".
\par 22 Jeesus vastasi ja sanoi heille: "Pitäkää usko Jumalaan.
\par 23 Totisesti minä sanon teille: jos joku sanoisi tälle vuorelle: 'Kohoa ja heittäydy mereen', eikä epäilisi sydämessään, vaan uskoisi sen tapahtuvan, minkä hän sanoo, niin se hänelle tapahtuisi.
\par 24 Sentähden minä sanon teille: kaikki, mitä te rukoilette ja anotte, uskokaa saaneenne, niin se on teille tuleva.
\par 25 Ja kun te seisotte ja rukoilette, niin antakaa anteeksi, jos kenellä teistä on jotakin toistansa vastaan, että myös teidän Isänne, joka on taivaissa, antaisi teille anteeksi teidän rikkomuksenne."
\par 27 Ja he tulivat taas Jerusalemiin. Ja kun hän käveli pyhäkössä, tulivat ylipapit ja kirjanoppineet ja vanhimmat hänen luoksensa
\par 28 ja sanoivat hänelle: "Millä vallalla sinä näitä teet? Tahi kuka sinulle on antanut vallan näitä tehdä?"
\par 29 Mutta Jeesus vastasi heille: "Minä myös teen teille yhden kysymyksen; vastatkaa te minulle, niin minä sanon teille, millä vallalla minä näitä teen.
\par 30 Oliko Johanneksen kaste taivaasta vai ihmisistä? Vastatkaa minulle."
\par 31 Niin he neuvottelivat keskenään sanoen: "Jos sanomme: 'Taivaasta', niin hän sanoo: 'Miksi ette siis uskoneet häntä?'
\par 32 Vai sanommeko: 'Ihmisistä'?" - sitä he kansan tähden pelkäsivät sanoa, sillä kaikkien mielestä Johannes totisesti oli profeetta.
\par 33 Ja he vastasivat ja sanoivat Jeesukselle: "Emme tiedä". Silloin Jeesus sanoi heille: "Niinpä en minäkään sano teille, millä vallalla minä näitä teen".

\chapter{12}

\par 1 Ja hän rupesi puhumaan heille vertauksilla: "Mies istutti viinitarhan ja teki aidan sen ympärille ja kaivoi viinikuurnan ja rakensi tornin; ja hän vuokrasi sen viinitarhureille ja matkusti muille maille.
\par 2 Ja kun aika tuli, lähetti hän palvelijan viinitarhurien luo perimään tarhureilta viinitarhan hedelmiä.
\par 3 Mutta he ottivat hänet kiinni, pieksivät ja lähettivät tyhjin käsin pois.
\par 4 Ja vielä hän lähetti heidän luoksensa toisen palvelijan. Ja häntä he löivät päähän ja häpäisivät.
\par 5 Ja hän lähetti vielä toisen, ja sen he tappoivat; ja samoin useita muita: toisia he pieksivät, toisia tappoivat.
\par 6 Vielä hänellä oli ainoa rakas poikansa. Hänet hän lähetti viimeiseksi heidän luoksensa sanoen: 'Kavahtavat kaiketi minun poikaani'.
\par 7 Mutta viinitarhurit sanoivat toisilleen: 'Tämä on perillinen; tulkaa, tappakaamme hänet, niin perintö jää meille'.
\par 8 Ja he ottivat hänet kiinni, tappoivat ja heittivät hänet ulos viinitarhasta.
\par 9 Mitä nyt viinitarhan herra on tekevä? Hän tulee ja tuhoaa viinitarhurit ja antaa viinitarhan muille.
\par 10 Ettekö ole lukeneet tätä kirjoitusta: 'Se kivi, jonka rakentajat hylkäsivät, on tullut kulmakiveksi;
\par 11 Herralta tämä on tullut ja on ihmeellinen meidän silmissämme'?"
\par 12 Silloin he olisivat tahtoneet ottaa hänet kiinni, mutta pelkäsivät kansaa; sillä he ymmärsivät, että hän oli puhunut tämän vertauksen heistä. Ja he jättivät hänet ja menivät pois.
\par 13 Ja he lähettivät hänen luoksensa muutamia fariseuksia ja herodilaisia kietomaan häntä sanoilla.
\par 14 Nämä tulivat ja sanoivat hänelle: "Opettaja, me tiedämme, että sinä olet totinen etkä välitä kenestäkään, sillä sinä et katso henkilöön, vaan opetat Jumalan tietä totuudessa. Onko luvallista antaa keisarille veroa vai ei? Tuleeko meidän antaa vai ei?"
\par 15 Mutta hän tiesi heidän ulkokultaisuutensa ja sanoi heille: "Miksi kiusaatte minua? Tuokaa denari minun nähdäkseni."
\par 16 Niin he toivat. Ja hän sanoi heille: "Kenen kuva ja päällekirjoitus tämä on?" He vastasivat hänelle: "Keisarin".
\par 17 Jeesus sanoi heille: "Antakaa keisarille, mikä keisarin on, ja Jumalalle, mikä Jumalan on". Ja he ihmettelivät häntä suuresti.
\par 18 Ja hänen luoksensa tuli saddukeuksia, jotka sanovat, ettei ylösnousemusta ole; ja he kysyivät häneltä sanoen:
\par 19 "Opettaja, Mooses on säätänyt meille: 'Jos joltakin kuolee veli, joka jättää jälkeensä vaimon, mutta ei jätä lasta, niin ottakoon hän veljensä lesken ja herättäköön veljellensä siemenen'.
\par 20 Oli seitsemän veljestä. Ensimmäinen otti vaimon, ja kun hän kuoli, ei häneltä jäänyt jälkeläistä.
\par 21 Silloin toinen otti hänet, ja hänkin kuoli jättämättä jälkeläistä. Niin myös kolmas.
\par 22 Samoin kävi kaikille seitsemälle: heiltä ei jäänyt jälkeläistä. Viimeiseksi kaikista vaimokin kuoli.
\par 23 Ylösnousemuksessa siis, kun he nousevat ylös, kenelle heistä hän joutuu vaimoksi, sillä hän oli ollut kaikkien noiden seitsemän vaimona?"
\par 24 Jeesus sanoi heille: "Ettekö te siitä syystä eksy, kun ette tunne kirjoituksia ettekä Jumalan voimaa?
\par 25 Sillä kun kuolleista noustaan, ei naida eikä mennä miehelle; vaan he ovat niinkuin enkelit taivaissa.
\par 26 Mutta mitä siihen tulee, että kuolleet nousevat ylös, ettekö ole lukeneet Mooseksen kirjasta, kertomuksessa orjantappurapensaasta, kuinka Jumala puhui hänelle sanoen: 'Minä olen Aabrahamin Jumala ja Iisakin Jumala ja Jaakobin Jumala'?
\par 27 Ei hän ole kuolleitten Jumala, vaan elävien. Suuresti te eksytte."
\par 28 Silloin tuli hänen luoksensa eräs kirjanoppinut, joka oli kuullut heidän keskustelunsa ja huomannut hänen hyvin vastanneen heille, ja kysyi häneltä: "Mikä on ensimmäinen kaikista käskyistä?"
\par 29 Jeesus vastasi: "Ensimmäinen on tämä: 'Kuule, Israel: Herra, meidän Jumalamme, Herra on yksi ainoa;
\par 30 ja rakasta Herraa, sinun Jumalaasi, kaikesta sydämestäsi ja kaikesta sielustasi ja kaikesta mielestäsi ja kaikesta voimastasi'.
\par 31 Toinen on tämä: 'Rakasta lähimmäistäsi niinkuin itseäsi'. Ei ole mitään käskyä, suurempaa kuin nämä."
\par 32 Niin kirjanoppinut sanoi hänelle: "Oikein sanoit, opettaja, totuuden mukaan, että yksi hän on, ja ettei ketään muuta ole, paitsi hän.
\par 33 Ja rakastaa häntä kaikesta sydämestään ja kaikesta ymmärryksestään ja kaikesta voimastaan, ja rakastaa lähimmäistään niinkuin itseänsä, se on enemmän kuin kaikki polttouhrit ja muut uhrit."
\par 34 Kun Jeesus näki, että hän vastasi ymmärtäväisesti, sanoi hän hänelle: "Sinä et ole kaukana Jumalan valtakunnasta". Eikä kukaan enää rohjennut häneltä kysyä.
\par 35 Ja opettaessaan pyhäkössä Jeesus puhui edelleen ja sanoi: "Kuinka kirjanoppineet sanovat, että Kristus on Daavidin poika?
\par 36 Onhan Daavid itse sanonut Pyhässä Hengessä: 'Herra sanoi minun Herralleni: Istu minun oikealle puolelleni, kunnes minä panen sinun vihollisesi sinun jalkojesi alle'.
\par 37 Daavid itse sanoo häntä Herraksi; kuinka hän sitten on hänen poikansa?" Ja suuri kansanjoukko kuunteli häntä mielellään.
\par 38 Ja opettaessaan hän sanoi: "Varokaa kirjanoppineita, jotka mielellään käyskelevät pitkissä vaipoissa ja haluavat tervehdyksiä toreilla
\par 39 ja etumaisia istuimia synagoogissa ja ensimmäisiä sijoja pidoissa,
\par 40 noita, jotka syövät leskien huoneet ja näön vuoksi pitävät pitkiä rukouksia; he saavat sitä kovemman tuomion".
\par 41 Ja hän istui vastapäätä uhriarkkua ja katseli, kuinka kansa pani rahaa uhriarkkuun. Ja monet rikkaat panivat paljon.
\par 42 Niin tuli köyhä leski ja pani kaksi ropoa, yhteensä muutamia pennejä.
\par 43 Ja hän kutsui opetuslapsensa tykönsä ja sanoi heille: "Totisesti minä sanon teille: tämä köyhä leski pani enemmän kuin kaikki muut, jotka panivat uhriarkkuun.
\par 44 Sillä he kaikki panivat liiastaan, mutta tämä pani puutteestaan kaiken, mitä hänellä oli, koko elämisensä."

\chapter{13}

\par 1 Ja kun hän meni ulos pyhäköstä, sanoi eräs hänen opetuslapsistaan hänelle: "Opettaja, katso, millaiset kivet ja millaiset rakennukset!"
\par 2 Jeesus vastasi hänelle: "Sinä näet nämä suuret rakennukset. Niistä ei ole jäävä kiveä kiven päälle maahan jaottamatta."
\par 3 Ja kun hän istui Öljymäellä, vastapäätä pyhäkköä, kysyivät Pietari ja Jaakob ja Johannes ja Andreas häneltä eriksensä:
\par 4 "Sano meille: milloin tämä tapahtuu, ja mikä on merkki siitä, että kaikki tämä alkaa lopullisesti toteutua?"
\par 5 Niin Jeesus rupesi puhumaan heille: "Katsokaa, ettei kukaan teitä eksytä.
\par 6 Monta tulee minun nimessäni sanoen: 'Minä se olen', ja he eksyttävät monta.
\par 7 Ja kun te kuulette sotien melskettä ja sanomia sodista, älkää peljästykö. Näin täytyy tapahtua, mutta se ei ole vielä loppu.
\par 8 Sillä kansa nousee kansaa vastaan ja valtakunta valtakuntaa vastaan, tulee maanjäristyksiä monin paikoin, tulee nälänhätää. Tämä on synnytystuskien alkua.
\par 9 Mutta pitäkää te vaari itsestänne. Teidät vedetään oikeuksiin, ja teitä piestään synagoogissa, ja teidät asetetaan maaherrain ja kuningasten eteen, minun tähteni, todistukseksi heille.
\par 10 Ja sitä ennen pitää evankeliumi saarnattaman kaikille kansoille.
\par 11 Ja kun he vievät teitä ja vetävät oikeuteen, älkää edeltäpäin huolehtiko siitä, mitä puhuisitte; vaan mitä teille sillä hetkellä annetaan, se puhukaa. Sillä ette te ole puhumassa, vaan Pyhä Henki.
\par 12 Ja veli antaa veljensä kuolemaan ja isä lapsensa, ja lapset nousevat vanhempiansa vastaan ja tappavat heidät.
\par 13 Ja te joudutte kaikkien vihattaviksi minun nimeni tähden; mutta joka vahvana pysyy loppuun asti, se pelastuu.
\par 14 Mutta kun näette hävityksen kauhistuksen seisovan siinä, missä ei tulisi - joka tämän lukee, se tarkatkoon - silloin ne, jotka Juudeassa ovat, paetkoot vuorille;
\par 15 joka on katolla, älköön astuko alas älköönkä menkö sisään noutamaan mitään huoneestansa;
\par 16 ja joka on mennyt pellolle, älköön palatko takaisin noutamaan vaippaansa.
\par 17 Voi raskaita ja imettäväisiä niinä päivinä!
\par 18 Mutta rukoilkaa, ettei se tapahtuisi talvella.
\par 19 Sillä niinä päivinä on oleva ahdistus, jonka kaltaista ei ole ollut hamasta luomakunnan alusta, jonka Jumala on luonut, tähän asti, eikä milloinkaan tule.
\par 20 Ja ellei Herra lyhentäisi niitä päiviä, ei mikään liha pelastuisi; mutta valittujen tähden, jotka hän on valinnut, hän on lyhentänyt ne päivät.
\par 21 Ja jos silloin joku sanoo teille: 'Katso, täällä on Kristus', tai: 'Katso, tuolla', niin älkää uskoko.
\par 22 Sillä vääriä kristuksia ja vääriä profeettoja nousee, ja he tekevät tunnustekoja ja ihmeitä, eksyttääkseen, jos mahdollista, valitut.
\par 23 Mutta olkaa te varuillanne. Minä olen edeltä sanonut teille kaikki.
\par 24 Mutta niinä päivinä, sen ahdistuksen jälkeen, aurinko pimenee, eikä kuu anna valoansa,
\par 25 ja tähdet putoilevat taivaalta, ja voimat, jotka taivaissa ovat, järkkyvät.
\par 26 Ja silloin he näkevät Ihmisen Pojan tulevan pilvissä suurella voimalla ja kirkkaudella.
\par 27 Ja silloin hän lähettää enkelinsä ja kokoaa valittunsa neljältä ilmalta, maan äärestä hamaan taivaan ääreen.
\par 28 Mutta oppikaa viikunapuusta vertaus: Kun sen oksa jo on tuore ja lehdet puhkeavat, niin te tiedätte, että kesä on lähellä.
\par 29 Samoin te myös, kun näette tämän tapahtuvan, tietäkää, että se on lähellä, oven edessä.
\par 30 Totisesti minä sanon teille: tämä sukupolvi ei katoa, ennenkuin nämä kaikki tapahtuvat.
\par 31 Taivas ja maa katoavat, mutta minun sanani eivät katoa.
\par 32 Mutta siitä päivästä tai hetkestä ei tiedä kukaan, eivät enkelit taivaassa, eikä myöskään Poika, vaan ainoastaan Isä.
\par 33 Olkaa varuillanne, valvokaa ja rukoilkaa, sillä ette tiedä, milloin se aika tulee.
\par 34 On niinkuin muille maille matkustaneen miehen: kun hän jätti talonsa ja antoi palvelijoilleen vallan, kullekin oman tehtävänsä, käski hän myös ovenvartijan valvoa.
\par 35 Valvokaa siis, sillä ette tiedä, milloin talon herra tulee, iltamyöhälläkö vai yösydännä vai kukonlaulun aikaan vai varhain aamulla,
\par 36 ettei hän, äkkiarvaamatta tullessaan, tapaisi teitä nukkumasta.
\par 37 Mutta minkä minä teille sanon, sen minä sanon kaikille: valvokaa."

\chapter{14}

\par 1 Niin oli kahden päivän perästä pääsiäinen ja happamattoman leivän juhla, ja ylipapit ja kirjanoppineet miettivät, kuinka saisivat hänet otetuksi kavaluudella kiinni ja tapetuksi.
\par 2 Sillä he sanoivat: "Ei juhlan aikana, ettei syntyisi meteliä kansassa".
\par 3 Ja kun hän oli Betaniassa, pitalisen Simonin asunnossa, tuli hänen aterialla ollessaan nainen, mukanaan alabasteripullo täynnä oikeata, kallista nardusvoidetta. Hän rikkoi alabasteripullon ja vuodatti voiteen hänen päähänsä.
\par 4 Niin oli muutamia, jotka närkästyivät ja sanoivat keskenään: "Mitä varten tämä voiteen haaskaus?
\par 5 Olisihan voinut myydä tämän voiteen enempään kuin kolmeensataan denariin ja antaa ne köyhille." Ja he toruivat häntä.
\par 6 Mutta Jeesus sanoi: "Antakaa hänen olla. Miksi pahoitatte hänen mieltään? Hän teki hyvän työn minulle.
\par 7 Köyhät teillä on aina keskuudessanne, ja milloin tahdotte, voitte heille tehdä hyvää, mutta minua teillä ei ole aina.
\par 8 Hän teki, minkä voi. Edeltäkäsin hän voiteli minun ruumiini hautaamista varten.
\par 9 Totisesti minä sanon teille: missä ikinä kaikessa maailmassa evankeliumia saarnataan, siellä sekin, minkä tämä teki, on mainittava hänen muistoksensa."
\par 10 Ja Juudas Iskariot, yksi niistä kahdestatoista, meni ylipappien luo kavaltaakseen hänet heille.
\par 11 Kun nämä sen kuulivat, ihastuivat he ja lupasivat antaa hänelle rahaa. Ja hän mietti, kuinka hänet sopivassa tilaisuudessa kavaltaisi.
\par 12 Ja ensimmäisenä happamattoman leivän päivänä, kun pääsiäislammas teurastettiin, hänen opetuslapsensa sanoivat hänelle: "Mihin tahdot, että menemme valmistamaan pääsiäislampaan sinun syödäksesi?"
\par 13 Niin hän lähetti kaksi opetuslastaan ja sanoi heille: "Menkää kaupunkiin, niin teitä vastaan tulee mies kantaen vesiastiaa; seuratkaa häntä.
\par 14 Ja mihin hän menee, sen talon isännälle sanokaa: 'Opettaja sanoo: Missä on minun vierashuoneeni, syödäkseni siinä pääsiäislampaan opetuslasteni kanssa?'
\par 15 Ja hän näyttää teille suuren huoneen yläkerrassa, valmiiksi laitetun; valmistakaa meille sinne."
\par 16 Niin opetuslapset lähtivät ja tulivat kaupunkiin ja havaitsivat niin olevan, kuin Jeesus oli heille sanonut, ja valmistivat pääsiäislampaan.
\par 17 Ja kun ehtoo tuli, saapui hän sinne niiden kahdentoista kanssa.
\par 18 Ja kun he olivat aterialla ja söivät, sanoi Jeesus: "Totisesti minä sanon teille: yksi teistä kavaltaa minut, yksi, joka syö minun kanssani".
\par 19 He tulivat murheellisiksi ja rupesivat toinen toisensa perästä sanomaan hänelle: "En kaiketi minä?"
\par 20 Hän sanoi heille: "Yksi teistä kahdestatoista, se, joka kastaa vatiin minun kanssani.
\par 21 Niin, Ihmisen Poika tosin menee pois, niinkuin hänestä on kirjoitettu, mutta voi sitä ihmistä, jonka kautta Ihmisen Poika kavalletaan! Parempi olisi sille ihmiselle, että hän ei olisi syntynyt."
\par 22 Ja heidän syödessään Jeesus otti leivän, siunasi, mursi ja antoi heille ja sanoi: "Ottakaa, tämä on minun ruumiini".
\par 23 Ja hän otti maljan, kiitti ja antoi heille; ja he kaikki joivat siitä.
\par 24 Ja hän sanoi heille: "Tämä on minun vereni, liiton veri, joka vuodatetaan monen edestä.
\par 25 Totisesti minä sanon teille: minä en juo enää viinipuun antia, ennenkuin sinä päivänä, jona juon sitä uutena Jumalan valtakunnassa."
\par 26 Ja veisattuaan kiitosvirren he lähtivät Öljymäelle.
\par 27 Ja Jeesus sanoi heille: "Kaikki te loukkaannutte, sillä kirjoitettu on: 'Minä lyön paimenta, ja lampaat hajotetaan'.
\par 28 Mutta ylösnoustuani minä menen teidän edellänne Galileaan."
\par 29 Niin Pietari sanoi hänelle: "Vaikka kaikki loukkaantuisivat, en kuitenkaan minä".
\par 30 Jeesus sanoi hänelle: "Totisesti minä sanon sinulle: tänään, tänä yönä, ennenkuin kukko kahdesti laulaa, sinä kolmesti minut kiellät".
\par 31 Mutta hän vakuutti vielä lujemmin: "Vaikka minun pitäisi kuolla sinun kanssasi, en sittenkään minä sinua kiellä". Ja samoin sanoivat kaikki muutkin.
\par 32 Ja he tulivat maatilalle, jonka nimi on Getsemane; ja hän sanoi opetuslapsillensa: "Istukaa tässä, sillä aikaa kuin minä rukoilen".
\par 33 Ja hän otti mukaansa Pietarin ja Jaakobin ja Johanneksen; ja hän alkoi kauhistua ja tulla tuskaan.
\par 34 Ja hän sanoi heille: "Minun sieluni on syvästi murheellinen, kuolemaan asti; olkaa tässä ja valvokaa".
\par 35 Ja hän meni vähän edemmäksi, lankesi maahan ja rukoili, että, jos mahdollista, se hetki menisi häneltä ohi,
\par 36 ja sanoi: "Abba, Isä, kaikki on mahdollista sinulle; ota pois minulta tämä malja. Mutta ei, mitä minä tahdon, vaan mitä sinä!"
\par 37 Ja hän tuli ja tapasi heidät nukkumasta ja sanoi Pietarille: "Simon, nukutko? Etkö jaksanut yhtä hetkeä valvoa?
\par 38 Valvokaa ja rukoilkaa, ettette joutuisi kiusaukseen; henki tosin on altis, mutta liha on heikko."
\par 39 Ja taas hän meni pois ja rukoili sanoen samat sanat.
\par 40 Ja tullessaan hän taas tapasi heidät nukkumasta, sillä heidän silmänsä olivat käyneet kovin raukeiksi; ja he eivät tienneet, mitä hänelle vastaisivat.
\par 41 Ja hän tuli kolmannen kerran ja sanoi heille: "Te nukutte vielä ja lepäätte! Jo riittää! Hetki on tullut; katso, Ihmisen Poika annetaan syntisten käsiin.
\par 42 Nouskaa, lähtekäämme; katso, se, joka kavaltaa minut, on lähellä."
\par 43 Ja heti, hänen vielä puhuessaan, Juudas, yksi niistä kahdestatoista, saapui, ja hänen mukanaan joukko ylipappien ja kirjanoppineiden ja vanhinten luota, miekat ja seipäät käsissä.
\par 44 Mutta se, joka hänet kavalsi, oli sopinut heidän kanssaan merkistä, sanoen: "Se, jolle minä suuta annan, hän se on; ottakaa hänet kiinni ja viekää tarkasti vartioituna pois".
\par 45 Ja tultuaan hän kohta astui hänen luoksensa ja sanoi: "Rabbi!" ja antoi hänelle suuta.
\par 46 Silloin he kävivät häneen käsiksi ja ottivat hänet kiinni.
\par 47 Mutta eräs niistä, jotka lähellä seisoivat, veti miekkansa, iski ylimmäisen papin palvelijaa ja sivalsi häneltä pois korvan.
\par 48 Niin Jeesus puhui heille ja sanoi: "Niinkuin ryöväriä vastaan te olette lähteneet minua miekoilla ja seipäillä vangitsemaan.
\par 49 Joka päivä minä olen ollut teidän luonanne opettaen pyhäkössä, ettekä ole ottaneet minua kiinni. Mutta tämä tapahtuu, että kirjoitukset kävisivät toteen."
\par 50 Niin he kaikki jättivät hänet ja pakenivat.
\par 51 Ja häntä seurasi eräs nuorukainen, jolla oli ympärillään liinavaate paljaalle iholle heitettynä; ja he ottivat hänet kiinni.
\par 52 Mutta hän jätti liinavaatteen ja pakeni alastonna.
\par 53 Ja he veivät Jeesuksen ylimmäisen papin eteen, jonne kaikki ylipapit ja vanhimmat ja kirjanoppineet kokoontuivat.
\par 54 Ja Pietari seurasi häntä taampana ylimmäisen papin palatsin esipihaan asti ja istui palvelijain joukkoon ja lämmitteli tulen ääressä.
\par 55 Mutta ylipapit ja koko neuvosto etsivät todistusta Jeesusta vastaan tappaaksensa hänet, mutta eivät löytäneet.
\par 56 Useat kyllä todistivat väärin häntä vastaan, mutta todistukset eivät olleet yhtäpitäviä.
\par 57 Niin muutamat nousivat ja todistivat väärin häntä vastaan, sanoen:
\par 58 "Me olemme kuulleet hänen sanovan: 'Minä hajotan maahan tämän käsillä tehdyn temppelin ja rakennan kolmessa päivässä toisen, joka ei ole käsillä tehty'."
\par 59 Mutta eivät näinkään heidän todistuksensa olleet yhtäpitäviä.
\par 60 Silloin ylimmäinen pappi nousi ja astui esille ja kysyi Jeesukselta sanoen: "Etkö vastaa mitään? Mitä nämä todistavat sinua vastaan?"
\par 61 Mutta hän oli vaiti eikä vastannut mitään. Taas ylimmäinen pappi kysyi häneltä ja sanoi hänelle: "Oletko sinä Kristus, sen Ylistetyn Poika?"
\par 62 Jeesus sanoi: "Olen; ja te saatte nähdä Ihmisen Pojan istuvan Voiman oikealla puolella ja tulevan taivaan pilvissä".
\par 63 Niin ylimmäinen pappi repäisi vaatteensa ja sanoi: "Mitä me enää todistajia tarvitsemme?
\par 64 Te kuulitte hänen pilkkaamisensa. Mitä arvelette?" Niin he kaikki tuomitsivat hänet vikapääksi kuolemaan.
\par 65 Ja muutamat rupesivat sylkemään häntä ja peittivät hänen kasvonsa ja löivät häntä nyrkillä ja sanoivat hänelle: "Profetoi!" Oikeudenpalvelijatkin löivät häntä poskelle.
\par 66 Kun nyt Pietari oli alhaalla esipihassa, tuli sinne muuan ylimmäisen papin palvelijattarista;
\par 67 ja nähdessään Pietarin lämmittelevän hän katsoi häneen ja sanoi: "Sinäkin olit Nasaretilaisen, tuon Jeesuksen, kanssa".
\par 68 Mutta hän kielsi sanoen: "En tiedä enkä käsitä, mitä sanot". Ja hän meni ulos pihalle. Ja kukko lauloi.
\par 69 Ja nähdessään hänet siellä palvelijatar rupesi taas sanomaan lähellä seisoville: "Tämä on yksi niistä".
\par 70 Mutta hän kielsi uudestaan. Ja vähän sen jälkeen lähellä seisovat taas sanoivat Pietarille: "Totisesti, sinä olet yksi niistä, sillä olethan sinä galilealainenkin".
\par 71 Mutta hän rupesi sadattelemaan itseänsä ja vannomaan: "En tunne sitä miestä, josta te puhutte".
\par 72 Ja samassa kukko lauloi toisen kerran. Niin Pietari muisti Jeesuksen sanat, jotka hän oli sanonut hänelle: "Ennenkuin kukko kahdesti laulaa, sinä kolmesti minut kiellät". Ja hän ratkesi itkuun.

\chapter{15}

\par 1 Ja heti aamulla ylipapit ynnä vanhimmat ja kirjanoppineet ja koko neuvosto tekivät päätöksen, sitoivat Jeesuksen, veivät hänet pois ja antoivat Pilatuksen käsiin.
\par 2 Niin Pilatus kysyi häneltä: "Oletko sinä juutalaisten kuningas?" Hän vastasi ja sanoi hänelle: "Sinäpä sen sanot".
\par 3 Ja ylipapit tekivät monta syytöstä häntä vastaan.
\par 4 Niin Pilatus taas kysyi häneltä sanoen: "Etkö vastaa mitään? Katso, kuinka paljon syytöksiä he tekevät sinua vastaan."
\par 5 Mutta Jeesus ei enää vastannut mitään, niin että Pilatus ihmetteli.
\par 6 Mutta juhlan aikana hän tavallisesti päästi heille yhden vangin irti, sen, jota he anoivat.
\par 7 Niin siellä oli eräs mies, jota sanottiin Barabbaaksi, vangittu muiden kapinoitsijain kanssa, jotka kapinassa olivat tehneet murhan.
\par 8 Ja kansa meni sinne ylös ja rupesi pyytämään, että hän tekisi heille, niinkuin hänen tapansa oli.
\par 9 Pilatus vastasi heille sanoen: "Tahdotteko, että päästän teille juutalaisten kuninkaan?"
\par 10 Sillä hän tiesi, että ylipapit kateudesta olivat antaneet hänet hänen käsiinsä.
\par 11 Mutta ylipapit kiihoittivat kansaa anomaan, että hän ennemmin päästäisi heille Barabbaan.
\par 12 Ja Pilatus puhui taas heille ja sanoi: "Mitä minun sitten on tehtävä hänelle, jota te sanotte juutalaisten kuninkaaksi?"
\par 13 Niin he taas huusivat: "Ristiinnaulitse hänet!"
\par 14 Mutta Pilatus sanoi heille: "Mitä pahaa hän sitten on tehnyt?" Mutta he huusivat vielä kovemmin: "Ristiinnaulitse hänet!"
\par 15 Ja kun Pilatus tahtoi tehdä kansalle mieliksi, päästi hän heille Barabbaan, mutta Jeesuksen hän ruoskitti ja luovutti ristiinnaulittavaksi.
\par 16 Niin sotamiehet veivät hänet sisälle linnaan, se on palatsiin, ja kutsuivat siihen koko sotilasjoukon.
\par 17 Ja he pukivat hänen yllensä purppuravaipan, väänsivät orjantappuroista kruunun ja panivat sen hänen päähänsä
\par 18 ja rupesivat tervehtimään häntä: "Terve, juutalaisten kuningas!"
\par 19 Ja he löivät häntä päähän ruovolla, sylkivät häntä ja laskeutuen polvilleen kumarsivat häntä.
\par 20 Ja kun he olivat häntä pilkanneet, riisuivat he häneltä purppuravaipan ja pukivat hänet hänen omiin vaatteisiinsa. Ja he veivät hänet pois, ristiinnaulitakseen hänet.
\par 21 Ja he pakottivat erään ohikulkevan miehen, Simonin, kyreneläisen, joka tuli vainiolta, Aleksanterin ja Rufuksen isän, kantamaan hänen ristiänsä.
\par 22 Ja he veivät hänet paikalle, jonka nimi on Golgata, se on käännettynä: pääkallonpaikka.
\par 23 Ja he tarjosivat hänelle mirhalla sekoitettua viiniä, mutta hän ei sitä ottanut.
\par 24 Ja he ristiinnaulitsivat hänet ja jakoivat keskenään hänen vaatteensa heittäen niistä arpaa, mitä kukin oli saava.
\par 25 Oli kolmas hetki, kun he hänet ristiinnaulitsivat.
\par 26 Ja päällekirjoitukseksi oli merkitty hänen syynsä: "Juutalaisten kuningas".
\par 27 Ja he ristiinnaulitsivat hänen kanssaan kaksi ryöväriä, toisen hänen oikealle ja toisen hänen vasemmalle puolellensa.
\par 29 Ja ne, jotka kulkivat ohitse, herjasivat häntä ja nyökyttivät päätään ja sanoivat: "Voi sinua, joka hajotat maahan temppelin ja kolmessa päivässä sen rakennat!
\par 30 Auta itseäsi ja astu alas ristiltä."
\par 31 Samoin ylipapitkin ynnä kirjanoppineet keskenään pilkkasivat häntä ja sanoivat: "Muita hän on auttanut, itseään ei voi auttaa.
\par 32 Astukoon hän, Kristus, Israelin kuningas, nyt alas ristiltä, että me näkisimme ja uskoisimme." Myöskin ne, jotka olivat ristiinnaulitut hänen kanssaan, herjasivat häntä.
\par 33 Ja kuudennella hetkellä tuli pimeys yli kaiken maan, ja sitä kesti hamaan yhdeksänteen hetkeen.
\par 34 Ja yhdeksännellä hetkellä Jeesus huusi suurella äänellä: "Eeli, Eeli, lama sabaktani?" Se on käännettynä: Jumalani, Jumalani, miksi minut hylkäsit?
\par 35 Sen kuullessaan sanoivat muutamat niistä, jotka siinä seisoivat: "Katso, hän huutaa Eliasta".
\par 36 Ja muuan juoksi ja täytti sienen hapanviinillä, pani sen ruovon päähän ja antoi hänelle juoda sanoen: "Annas, katsokaamme, tuleeko Elias ottamaan hänet alas".
\par 37 Mutta Jeesus huusi suurella äänellä ja antoi henkensä.
\par 38 Ja temppelin esirippu repesi kahtia ylhäältä alas asti.
\par 39 Mutta kun sadanpäämies, joka seisoi häntä vastapäätä, näki hänen näin antavan henkensä, sanoi hän: "Totisesti tämä ihminen oli Jumalan Poika".
\par 40 Ja siellä oli myös naisia taampaa katselemassa. Näiden joukossa olivat Maria Magdaleena ja Maria, Jaakob nuoremman ja Jooseen äiti, ja Salome,
\par 41 jotka hänen ollessaan Galileassa olivat seuranneet ja palvelleet häntä, sekä useita muita, jotka olivat tulleet hänen kanssaan ylös Jerusalemiin.
\par 42 Ja kun jo oli tullut ilta, ja koska oli valmistuspäivä, se on sapatin aattopäivä,
\par 43 tuli Joosef, arimatialainen, arvossapidetty neuvoston jäsen, joka hänkin odotti Jumalan valtakuntaa, rohkaisi mielensä ja meni sisälle Pilatuksen luo ja pyysi Jeesuksen ruumista.
\par 44 Niin Pilatus ihmetteli, oliko hän jo kuollut, ja kutsuttuaan luoksensa sadanpäämiehen kysyi tältä, oliko hän jo kauan sitten kuollut.
\par 45 Ja saatuaan sadanpäämieheltä siitä tiedon hän lahjoitti ruumiin Joosefille.
\par 46 Tämä osti liinavaatteen ja otti hänet alas, kääri hänet liinavaatteeseen ja pani hautaan, joka oli hakattu kallioon, ja vieritti kiven hautakammion ovelle.
\par 47 Ja Maria Magdaleena ja Maria, Jooseen äiti, katselivat, mihin hänet pantiin.

\chapter{16}

\par 1 Ja kun sapatti oli ohi, ostivat Maria Magdaleena ja Maria, Jaakobin äiti, ja Salome hyvänhajuisia yrttejä mennäkseen voitelemaan häntä.
\par 2 Ja viikon ensimmäisenä päivänä he tulivat haudalle ani varhain, auringon noustessa.
\par 3 Ja he sanoivat toisilleen: "Kuka meille vierittää kiven hautakammion ovelta?"
\par 4 Ja katsahtaessaan ylös he näkivät kiven poisvieritetyksi; se oli näet hyvin suuri.
\par 5 Ja mentyään hautakammion sisään he näkivät nuorukaisen istuvan oikealla puolella, puettuna pitkään, valkeaan vaatteeseen; ja he peljästyivät suuresti.
\par 6 Mutta hän sanoi heille: "Älkää peljästykö; te etsitte Jeesusta, Nasaretilaista, joka oli ristiinnaulittu. Hän on noussut ylös; ei hän ole täällä. Katso, tässä on paikka, johon he hänet panivat.
\par 7 Mutta menkää ja sanokaa hänen opetuslapsillensa ja Pietarille: 'Hän menee teidän edellänne Galileaan; siellä te saatte hänet nähdä, niinkuin hän teille sanoi'."
\par 8 Niin he tulivat ulos ja pakenivat haudalta, sillä heidät oli vallannut vavistus ja hämmästys, eivätkä sanoneet kenellekään mitään, sillä he pelkäsivät.
\par 9 Mutta ylösnousemisensa jälkeen hän varhain aamulla viikon ensimmäisenä päivänä ilmestyi ensiksi Maria Magdaleenalle, josta hän oli ajanut ulos seitsemän riivaajaa.
\par 10 Tämä meni ja vei sanan niille, jotka olivat olleet Jeesuksen kanssa ja jotka nyt murehtivat ja itkivät.
\par 11 Mutta kun he kuulivat, että hän eli ja että Maria oli hänet nähnyt, eivät he uskoneet.
\par 12 Ja sen jälkeen hän toisenmuotoisena ilmestyi kahdelle heistä, heidän kävellessään, matkalla maakylään.
\par 13 Hekin menivät ja veivät sanan toisille; mutta nämä eivät uskoneet heitäkään.
\par 14 Vihdoin hän ilmestyi myöskin niille yhdelletoista heidän ollessaan aterialla; ja hän nuhteli heidän epäuskoaan ja heidän sydämensä kovuutta, kun he eivät olleet uskoneet niitä, jotka olivat nähneet hänet ylösnousseeksi.
\par 15 Ja hän sanoi heille: "Menkää kaikkeen maailmaan ja saarnatkaa evankeliumia kaikille luoduille.
\par 16 Joka uskoo ja kastetaan, se pelastuu; mutta joka ei usko, se tuomitaan kadotukseen.
\par 17 Ja nämä merkit seuraavat niitä, jotka uskovat: minun nimessäni he ajavat ulos riivaajia, puhuvat uusilla kielillä,
\par 18 nostavat käsin käärmeitä, ja jos he juovat jotakin kuolettavaa, ei se heitä vahingoita; he panevat kätensä sairasten päälle, ja ne tulevat terveiksi."
\par 19 Kun nyt Herra Jeesus oli puhunut heille, otettiin hänet ylös taivaaseen, ja hän istui Jumalan oikealle puolelle.
\par 20 Mutta he lähtivät ja saarnasivat kaikkialla, ja Herra vaikutti heidän kanssansa ja vahvisti sanan sitä seuraavien merkkien kautta.


\end{document}