\begin{document}

\title{Toinen Mooseksen kirja}


\chapter{1}

\par 1 Nämä ovat Egyptiin tulleiden Israelin poikien nimet; Jaakobin kanssa he olivat itsekukin perheineen tulleet sinne:
\par 2 Ruuben, Simeon, Leevi ja Juuda,
\par 3 Isaskar, Sebulon ja Benjamin,
\par 4 Daan, Naftali, Gaad ja Asser.
\par 5 Ja Jaakobin kupeista lähteneitä oli kaikkiaan seitsemänkymmentä henkeä. Mutta Joosef oli jo ennestään Egyptissä.
\par 6 Ja Joosef kuoli ja kaikki hänen veljensä ynnä koko se sukupolvi.
\par 7 Mutta israelilaiset olivat hedelmälliset ja sikisivät, lisääntyivät ja enenivät hyvin suurilukuisiksi, niin että maa tuli heitä täyteen.
\par 8 Niin Egyptiin tuli uusi kuningas, joka ei Joosefista mitään tiennyt.
\par 9 Tämä sanoi kansallensa: "Katso, israelilaisten kansa on suurempi ja väkevämpi kuin me.
\par 10 Tulkaa, menetelkäämme siis viisaasti heitä kohtaan, että he eivät lisääntyisi eivätkä, jos sota syttyisi, liittyisi hekin vihollisiimme ja sotisi meitä vastaan ja lähtisi maasta pois."
\par 11 Niin heille asetettiin työnjohtajia rasittamaan heitä raskaalla työllä. Ja heidän täytyi rakentaa faraolle varastokaupungit Piitom ja Ramses.
\par 12 Mutta kuta enemmän kansaa rasitettiin, sitä enemmän se lisääntyi, ja sitä enemmän se levisi, niin että israelilaisia ruvettiin pelkäämään.
\par 13 Niin egyptiläiset pitivät israelilaisia orjantyössä väkivalloin
\par 14 ja katkeroittivat heidän elämänsä kovalla laasti- ja tiilityöllä ja kaikenlaisella työllä ulkona kedolla, kaikenlaisella työllä, jota he teettivät heillä väkivalloin.
\par 15 Ja Egyptin kuningas puhui hebrealaisille kätilövaimoille, joista toisen nimi oli Sifra ja toisen Puua;
\par 16 ja hän sanoi: "Kun te autatte hebrealaisia vaimoja heidän synnyttäessänsä, niin tarkastakaa lapsen sukupuoli: jos se on poika, surmatkaa se, mutta jos se on tyttö, jääköön eloon".
\par 17 Mutta kätilövaimot pelkäsivät Jumalaa eivätkä tehneet, niinkuin Egyptin kuningas oli heille sanonut, vaan antoivat poikalasten elää.
\par 18 Niin Egyptin kuningas kutsui kätilövaimot ja sanoi heille: "Miksi te näin teette ja annatte poikalasten elää?"
\par 19 Kätilövaimot vastasivat faraolle: "Hebrealaiset vaimot eivät ole niinkuin egyptiläiset. He ovat voimakkaita; ennenkuin kätilövaimo tulee heidän luoksensa, ovat he jo synnyttäneet."
\par 20 Mutta Jumala salli kätilövaimojen menestyä, ja kansa lisääntyi ja eneni suurilukuiseksi.
\par 21 Ja koska kätilövaimot pelkäsivät Jumalaa, niin hän antoi heille runsaasti perhettä.
\par 22 Niin farao antoi käskyn kaikelle kansallensa, sanoen: "Kaikki poikalapset, jotka syntyvät, heittäkää Niilivirtaan, mutta kaikkien tyttölasten antakaa elää".

\chapter{2}

\par 1 Niin eräs mies, joka oli Leevin sukua, meni ja nai leeviläisen neidon.
\par 2 Ja vaimo tuli raskaaksi ja synnytti pojan. Ja kun hän näki, että se oli ihana lapsi, salasi hän sitä kolme kuukautta.
\par 3 Mutta kun hän ei voinut sitä enää salata, otti hän kaisla-arkun, siveli sen maapihkalla ja piellä, pani lapsen siihen ja laski sen kaislikkoon Niilivirran rantaan.
\par 4 Ja lapsen sisar asettui taammaksi nähdäksensä, mitä hänelle tapahtuisi.
\par 5 Silloin faraon tytär tuli alas peseytymään virrassa, ja hänen seuranaisensa kävelivät virran rannalla; ja kun hän näki arkun kaislikossa, lähetti hän palvelijattarensa ja otatti sen ylös.
\par 6 Ja kun hän avasi sen, näki hän lapsen; ja katso, siinä oli poikanen, joka itki. Niin hänen tuli sitä sääli, ja hän sanoi: "Tämä on hebrealaisten lapsia".
\par 7 Niin lapsen sisar sanoi faraon tyttärelle: "Menenkö kutsumaan sinulle hebrealaisen imettäjän, joka voi imettää sen lapsen sinulle?"
\par 8 Faraon tytär vastasi hänelle: "Mene!" Niin tyttö meni ja kutsui lapsen äidin.
\par 9 Ja faraon tytär sanoi hänelle: "Ota tämä lapsi ja imetä se minulle, niin minä maksan sinulle siitä palkan". Ja vaimo otti lapsen ja imetti sen.
\par 10 Mutta kun lapsi oli kasvanut, toi hän sen faraon tyttärelle, ja tämä otti sen pojaksensa ja antoi hänelle nimen Mooses, sillä hän sanoi: "Minä olen vetänyt hänet ylös vedestä".
\par 11 Ja tapahtui niihin aikoihin, kun Mooses oli kasvanut suureksi, että hän meni veljiensä luo ja näki heidän raskaan työnsä. Ja hän näki egyptiläisen miehen lyövän hebrealaista miestä, erästä hänen veljistään.
\par 12 Silloin hän katseli ympärillensä joka taholle, ja kun hän näki, ettei ketään ollut läheisyydessä, löi hän egyptiläisen kuoliaaksi ja kätki hänet hiekkaan.
\par 13 Ja hän meni toisena päivänä ulos ja näki kaksi hebrealaista miestä tappelemassa keskenään; ja hän sanoi syylliselle: "Miksi lyöt toveriasi?"
\par 14 Tämä vastasi: "Kuka on asettanut sinut meidän päämieheksemme ja tuomariksemme? Aiotko tappaa minutkin, niinkuin tapoit egyptiläisen?" Silloin Mooses peljästyi ja ajatteli: "Se on siis tullut ilmi".
\par 15 Ja kun farao sai kuulla tästä tapahtumasta, etsi hän Moosesta tappaaksensa hänet. Mutta Mooses lähti faraota pakoon ja pysähtyi Midianin maahan ja istahti eräälle kaivolle.
\par 16 Ja Midianin papilla oli seitsemän tytärtä; nämä tulivat vettä ammentamaan ja täyttivät vesikaukalot, juottaakseen isänsä lampaita.
\par 17 Mutta paimenet tulivat ja ajoivat heidät pois. Silloin Mooses nousi ja auttoi heitä ja juotti heidän lampaansa.
\par 18 Ja kun he tulivat isänsä Reguelin luo, kysyi hän: "Kuinka te tänä päivänä niin pian jouduitte?"
\par 19 He vastasivat: "Egyptiläinen mies auttoi meitä paimenten käsistä, ammensipa vielä vettäkin meille ja juotti lampaat".
\par 20 Ja hän sanoi tyttärillensä: "Missä hän on? Miksi te niin jätitte miehen? Kutsukaa hänet aterioimaan meidän kanssamme."
\par 21 Ja Mooses suostui asumaan sen miehen luona, ja hän antoi Moosekselle tyttärensä Sipporan vaimoksi.
\par 22 Tämä synnytti pojan, ja Mooses antoi hänelle nimen Geersom; sillä hän sanoi: "Minä olen muukalainen vieraalla maalla".
\par 23 Ja kun oli kulunut pitkä aika, kuoli Egyptin kuningas. Ja israelilaiset huokailivat orjuuttansa ja valittivat; ja heidän huutonsa heidän orjuutensa tähden nousi Jumalan tykö.
\par 24 Ja Jumala kuuli heidän vaikeroimisensa, ja Jumala muisti liittonsa Aabrahamin, Iisakin ja Jaakobin kanssa.
\par 25 Ja Jumala katsoi israelilaisten puoleen, ja Jumala piti heistä huolta.

\chapter{3}

\par 1 Ja Mooses kaitsi appensa Jetron, Midianin papin, lampaita. Ja kun hän kerran ajoi lampaita erämaan tuolle puolen, tuli hän Jumalan vuoren, Hoorebin, juurelle.
\par 2 Silloin Herran enkeli ilmestyi hänelle tulen liekissä keskellä orjantappurapensasta; ja hän näki, että pensas paloi ilmitulessa, mutta pensas ei kuitenkaan kulunut.
\par 3 Niin Mooses sanoi: "Minä käyn tuonne ja katson tätä suurta näkyä, miksi ei pensas pala poroksi".
\par 4 Kun Herra näki hänen tulevan katsomaan, huusi hän, Jumala, hänelle pensaasta ja sanoi: "Mooses, Mooses!" Hän vastasi: "Tässä olen".
\par 5 Hän sanoi: "Älä tule tänne! Riisu kengät jalastasi, sillä paikka, jossa seisot, on pyhä maa."
\par 6 Ja hän sanoi vielä: "Minä olen sinun isäsi Jumala, Aabrahamin Jumala, Iisakin Jumala ja Jaakobin Jumala". Ja Mooses peitti kasvonsa, sillä hän pelkäsi katsoa Jumalaa.
\par 7 Ja Herra sanoi: "Minä olen nähnyt kansani kurjuuden Egyptissä ja kuullut heidän huutonsa sortajainsa tähden; niin, minä tiedän heidän tuskansa.
\par 8 Sentähden minä olen astunut alas vapauttamaan heidät egyptiläisten kädestä ja johdattamaan heidät siitä maasta hyvään ja tilavaan maahan, maahan, joka vuotaa maitoa ja mettä, sinne, missä kanaanilaiset, heettiläiset, amorilaiset, perissiläiset, hivviläiset ja jebusilaiset asuvat.
\par 9 Ja nyt on israelilaisten huuto tullut minun kuuluviini, ja minä olen myös nähnyt sen sorron, jolla egyptiläiset heitä sortavat.
\par 10 Niin mene nyt, minä lähetän sinut faraon tykö, ja vie minun kansani, israelilaiset, pois Egyptistä."
\par 11 Mutta Mooses sanoi Jumalalle: "Mikä minä olen menemään faraon tykö ja viemään israelilaisia pois Egyptistä?"
\par 12 Hän vastasi: "Minä olen sinun kanssasi; ja tämä olkoon sinulle tunnusmerkkinä, että minä olen sinut lähettänyt: kun olet vienyt kansan pois Egyptistä, niin te palvelette Jumalaa tällä vuorella".
\par 13 Mooses sanoi Jumalalle: "Katso, kun minä menen israelilaisten luo ja sanon heille: 'Teidän isienne Jumala on lähettänyt minut teidän luoksenne', ja kun he kysyvät minulta: 'Mikä hänen nimensä on?' niin mitä minä heille vastaan?"
\par 14 Jumala vastasi Moosekselle: "Minä olen se, joka minä olen". Ja hän sanoi vielä: "Sano israelilaisille näin: 'Minä olen' lähetti minut teidän luoksenne".
\par 15 Ja Jumala sanoi vielä Moosekselle: "Sano israelilaisille näin: Herra, teidän isienne Jumala, Aabrahamin Jumala, Iisakin Jumala ja Jaakobin Jumala, lähetti minut teidän luoksenne; tämä on minun nimeni iankaikkisesti, ja näin minua kutsuttakoon sukupolvesta sukupolveen.
\par 16 Mene ja kokoa Israelin vanhimmat ja sano heille: Herra, teidän isienne Jumala, Aabrahamin, Iisakin ja Jaakobin Jumala, on ilmestynyt minulle ja sanonut: 'Totisesti minä pidän teistä huolen ja pidän silmällä, mitä teille tapahtuu Egyptissä.
\par 17 Ja minä olen päättänyt näin: minä johdatan teidät pois Egyptin kurjuudesta kanaanilaisten, heettiläisten, amorilaisten, perissiläisten, hivviläisten ja jebusilaisten maahan, siihen maahan, joka vuotaa maitoa ja mettä.'
\par 18 Ja he kuulevat sinua. Niin mene sitten, sinä ja Israelin vanhimmat, Egyptin kuninkaan tykö, ja sanokaa hänelle: 'Herra, hebrealaisten Jumala, on kohdannut meitä. Anna siis meidän mennä kolmen päivän matka erämaahan uhraamaan Herralle, Jumalallemme.'
\par 19 Mutta minä tiedän, että Egyptin kuningas ei päästä teitä menemään, ei edes väkevän käden pakolla.
\par 20 Mutta minä ojennan käteni ja lyön Egyptiä kaikenkaltaisilla ihmeilläni, joita minä olen siellä tekevä, ja sitten hän päästää teidät.
\par 21 Ja minä annan tämän kansan päästä egyptiläisten suosioon, niin että te lähtiessänne ette lähde tyhjin käsin.
\par 22 Vaan jokainen vaimo on pyytävä naapuriltaan ja luonaan majailevalta vaimolta hopea- ja kultakaluja ja vaatteita. Niihin te puette poikanne ja tyttärenne ja viette ne saaliina egyptiläisiltä."

\chapter{4}

\par 1 Mooses vastasi ja sanoi: "Katso, he eivät usko minua eivätkä kuule minua, vaan sanovat: 'Ei Herra ole sinulle ilmestynyt'".
\par 2 Herra sanoi hänelle: "Mikä sinulla on kädessäsi?" Hän vastasi: "Sauva".
\par 3 Hän sanoi: "Heitä se maahan". Ja hän heitti sen maahan, ja se muuttui käärmeeksi; ja Mooses pakeni sitä.
\par 4 Mutta Herra sanoi Moosekselle: "Ojenna kätesi ja tartu sen pyrstöön". Niin hän ojensi kätensä ja tarttui siihen, ja se muuttui sauvaksi hänen kädessänsä.
\par 5 - "Siitä he uskovat, että Herra, heidän isiensä Jumala, Aabrahamin Jumala, Iisakin Jumala ja Jaakobin Jumala, on sinulle ilmestynyt".
\par 6 Ja Herra sanoi vielä hänelle: "Pistä kätesi poveesi". Ja hän pisti kätensä poveensa. Ja kun hän veti sen ulos, niin katso, hänen kätensä oli pitalista valkoinen niinkuin lumi.
\par 7 Ja hän sanoi: "Pistä kätesi takaisin poveesi". Ja hän pisti kätensä takaisin poveensa. Ja kun hän veti sen ulos povestansa, niin katso, se oli taas niinkuin hänen muukin ihonsa.
\par 8 Herra sanoi: "Jos he eivät usko sinua eivätkä tottele ensimmäistä tunnustekoa, niin he uskovat toisen tunnusteon.
\par 9 Mutta jos he eivät usko näitäkään kahta tunnustekoa eivätkä kuule sinua, niin ota vettä Niilivirrasta ja kaada kuivalle maalle, niin se vesi, jonka virrasta otat, on muuttuva vereksi kuivalla maalla."
\par 10 Niin Mooses sanoi Herralle: "Oi Herra, minä en ole puhetaitoinen mies; en ole ollut ennen enkä senkään jälkeen, kuin sinä puhuit palvelijallesi; sillä minulla on hidas puhe ja kankea kieli".
\par 11 Ja Herra sanoi hänelle: "Kuka on antanut ihmiselle suun, tahi kuka tekee mykän tai kuuron, näkevän tai sokean? Enkö minä, Herra?
\par 12 Mene siis nyt, minä olen sinun suusi apuna ja opetan sinulle, mitä sinun on puhuttava."
\par 13 Mutta hän sanoi: "Oi Herra, lähetä kuka muu tahansa!"
\par 14 Niin Herra vihastui Moosekseen ja sanoi: "Eikö sinulla ole veljesi Aaron, leeviläinen? Minä tiedän, että hän osaa puhua. Ja katso, hän tuleekin sinua vastaan; ja kun hän näkee sinut, iloitsee hän sydämestänsä.
\par 15 Ja puhu sinä hänelle ja pane sanat hänen suuhunsa. Ja minä olen sinun suusi apuna ja hänen suunsa apuna ja opetan teille, mitä teidän on tehtävä.
\par 16 Ja hän on puhuva sinun puolestasi kansalle; niin hän on oleva sinulla suuna, ja sinä olet oleva hänellä jumalana.
\par 17 Ja ota käteesi tämä sauva, jolla olet tekevä nuo tunnusteot."
\par 18 Niin Mooses tuli takaisin appensa Jetron luo ja sanoi hänelle: "Anna minun mennä takaisin veljieni tykö Egyptiin, katsomaan, ovatko he vielä elossa". Jetro sanoi Moosekselle: "Mene rauhassa".
\par 19 Ja Herra sanoi Moosekselle Midianissa: "Mene takaisin Egyptiin, sillä kaikki ne ovat kuolleet, jotka väijyivät sinun henkeäsi".
\par 20 Niin Mooses otti vaimonsa ja poikansa ja pani heidät aasin selkään ja palasi Egyptin maahan; ja Mooses otti käteensä Jumalan sauvan.
\par 21 Ja Herra sanoi Moosekselle: "Kun tulet takaisin Egyptiin, niin katso, että teet faraon edessä kaikki ne ihmeet, jotka minä olen pannut sinun käteesi. Mutta minä paadutan hänen sydämensä, niin että hän ei päästä kansaa.
\par 22 Sano silloin faraolle: 'Näin sanoo Herra: Israel on minun esikoispoikani;
\par 23 sentähden minä sanon sinulle: Päästä minun poikani palvelemaan minua. Mutta jos kieltäydyt päästämästä häntä, niin katso, minä tapan sinun esikoispoikasi.'"
\par 24 Ja matkan varrella yöpaikassa tapahtui, että Herra kävi hänen kimppuunsa ja tahtoi surmata hänet.
\par 25 Silloin Sippora otti terävän kiven ja leikkasi pois poikansa esinahan, kosketti sillä Moosesta alhaalta ja sanoi: "Sinä olet minun veriylkäni".
\par 26 Niin hän jätti hänet rauhaan. Silloin Sippora sanoi: "Veriylkä ympärileikkauksen kautta".
\par 27 Ja Herra sanoi Aaronille: "Mene Moosesta vastaan erämaahan". Ja hän meni ja kohtasi hänet Jumalan vuorella ja suuteli häntä.
\par 28 Ja Mooses kertoi Aaronille kaikki, mitä Herra oli puhunut lähettäessään hänet, ja kaikki ne tunnusteot, jotka hän oli käskenyt hänen tehdä.
\par 29 Niin Mooses ja Aaron menivät ja kokosivat kaikki israelilaisten vanhimmat.
\par 30 Ja Aaron puhui kaikki, mitä Herra oli Moosekselle sanonut, ja Mooses teki tunnusteot kansan silmien edessä.
\par 31 Ja kansa uskoi. Ja kun he kuulivat, että Herra oli pitänyt huolta israelilaisista ja nähnyt heidän kurjuutensa, kumartuivat he maahan ja rukoilivat.

\chapter{5}

\par 1 Senjälkeen Mooses ja Aaron menivät ja sanoivat faraolle: "Näin sanoo Herra, Israelin Jumala: Päästä minun kansani viettämään minulle juhlaa erämaassa".
\par 2 Mutta farao vastasi: "Kuka on Herra, jota minun pitäisi kuulla ja päästää Israel? Minä en tunne Herraa enkä päästä Israelia."
\par 3 Niin he sanoivat: "Hebrealaisten Jumala on kohdannut meitä. Anna siis meidän mennä kolmen päivän matka erämaahan uhraamaan Herralle, Jumalallemme, ettei hän rankaisisi meitä rutolla tai miekalla."
\par 4 Mutta Egyptin kuningas vastasi heille: "Miksi te, Mooses ja Aaron, pidätätte kansaa työnteosta? Menkää töihinne."
\par 5 Ja farao sanoi vielä: "Katsokaa, liian paljon on muutenkin joutoväkeä maassa, ja te tahdotte saattaa heidät kulkemaan työttöminä".
\par 6 Ja farao antoi sinä päivänä käskyn kansan työnteettäjille ja päällysmiehille, sanoen:
\par 7 "Älkää enää antako kansalle olkia tiilien tekemistä varten niinkuin ennen; he menkööt itse ja kootkoot itselleen oljet.
\par 8 Pankaa kuitenkin heidän tehtäväkseen sama tiilimäärä, jonka he ennenkin ovat tehneet, siitä mitään vähentämättä; sillä he ovat laiskoja, sentähden he huutavat näin: 'Menkäämme uhraamaan Jumalallemme!'
\par 9 Pantakoon miehille raskasta työtä, että heillä olisi siinä tekemistä ja että he eivät kuuntelisi valhepuheita."
\par 10 Niin kansan työnteettäjät ja päällysmiehet menivät ja sanoivat kansalle: "Näin sanoo farao: 'Minä en anna teille enää olkia.
\par 11 Menkää itse ja hankkikaa itsellenne olkia, mistä vain löydätte, mutta työstänne ei vähennetä mitään.'"
\par 12 Niin kansa hajosi pitkin Egyptin maata keräämään pehkuja olkien asemesta.
\par 13 Ja työnteettäjät ahdistivat heitä sanoen: "Tehkää työnne valmiiksi, kunkin päivän määrä päivälleen, niinkuin silloinkin, kun saitte olkia".
\par 14 Ja israelilaisten päällysmiehiä, joita faraon työnteettäjät olivat heille asettaneet, piestiin, ja heille sanottiin: "Miksi ette ole eilen ettekä tänään suorittaneet tähänastista määräänne tiilien teossa?"
\par 15 Niin israelilaisten päällysmiehet menivät ja valittivat faraolle, sanoen: "Minkätähden teet näin palvelijoillesi?
\par 16 Olkia palvelijoillesi ei anneta, mutta kuitenkin sanotaan meille: 'Tehkää tiilet'. Ja katso, palvelijoitasi piestään, vaikka vika on sinun oman väkesi."
\par 17 Mutta hän sanoi: "Te olette laiskoja, laiskoja olette; sentähden te sanotte: 'Menkäämme uhraamaan Herralle'.
\par 18 Menkää vain työhönne. Olkia ei teille anneta, mutta määrätty luku tiiliä on teidän hankittava."
\par 19 Niin israelilaisten päällysmiehet huomasivat joutuneensa ahtaalle, kun heille sanottiin: "Ette saa mitään tiililuvun vähennystä kunkin päivän määrästä".
\par 20 Ja kun he lähtivät faraon luota, kohtasivat he Mooseksen ja Aaronin, jotka odottivat heitä,
\par 21 ja he sanoivat näille: "Herra kostakoon teille ja tuomitkoon teidät; sillä te olette saattaneet meidät faraon ja hänen palvelijainsa vihoihin ja antaneet heidän käteensä miekan, meille surmaksi".
\par 22 Silloin Mooses palasi Herran tykö ja sanoi: "Herra, miksi olet tehnyt niin pahoin tälle kansalle? Miksi olet lähettänyt minut?
\par 23 Sillä siitä asti, kun minä menin faraon tykö puhumaan sinun nimessäsi, hän on tehnyt pahaa tälle kansalle, etkä sinä suinkaan ole kansaasi vapahtanut."

\chapter{6}

\par 1 Mutta Herra vastasi Moosekselle: "Nyt saat nähdä, mitä minä faraolle teen; sillä väkevän käden pakottamana hän on päästävä heidät, ja väkevän käden pakottamana hän on ajava heidät maastansa".
\par 2 Ja Jumala puhui Moosekselle ja sanoi hänelle: "Minä olen Herra.
\par 3 Ja minä olen ilmestynyt Aabrahamille, Iisakille ja Jaakobille 'Jumalana Kaikkivaltiaana', mutta nimelläni 'Herra' en minä ole tehnyt itseäni heille tunnetuksi.
\par 4 Ja minä tein myös liittoni heidän kanssansa, antaakseni heille Kanaanin maan, sen maan, jossa he muukalaisina asuivat.
\par 5 Ja nyt minä olen kuullut israelilaisten vaikeroimisen, kun egyptiläiset pitävät heitä orjantyössä, ja olen muistanut liittoni.
\par 6 Sano sentähden israelilaisille: 'Minä olen Herra, ja minä vien teidät pois egyptiläisten sorron alta ja vapautan teidät orjuudesta ja pelastan teidät ojennetulla käsivarrella ja suurilla rangaistustuomioilla.
\par 7 Niin minä otan teidät kansakseni ja olen teidän Jumalanne; ja te tulette tietämään, että minä olen Herra, teidän Jumalanne, joka vien teidät pois egyptiläisten sorron alta.
\par 8 Ja minä johdatan teidät siihen maahan, jonka minä olen kättä kohottaen luvannut antaa Aabrahamille, Iisakille ja Jaakobille; ja sen minä annan teille omaksi. Minä olen Herra.'"
\par 9 Ja Mooses puhui näin israelilaisille; mutta he eivät kuulleet Moosesta tuskaantumisensa ja raskaan orjuutensa tähden.
\par 10 Sitten Herra puhui Moosekselle ja sanoi:
\par 11 "Mene ja sano faraolle, Egyptin kuninkaalle, että hän päästää israelilaiset maastansa".
\par 12 Mutta Mooses puhui Herran edessä ja sanoi: "Katso, israelilaiset eivät kuulleet minua; kuinka sitten farao kuulisi minua, joka olen huuliltani ympärileikkaamaton?"
\par 13 Mutta Herra puhui Moosekselle ja Aaronille ja käski heitä menemään israelilaisten ja faraon, Egyptin kuninkaan, tykö ja viemään israelilaiset pois Egyptin maasta.
\par 14 Nämä ovat heidän perhekuntainsa päämiehet: Ruubenin, Israelin esikoisen, pojat olivat Hanok ja Pallu, Hesron ja Karmi. Nämä ovat Ruubenin sukukunnat.
\par 15 Simeonin pojat olivat Jemuel, Jaamin, Oohad, Jaakin, Soohar ja Saul, kanaanilaisen vaimon poika. Nämä ovat Simeonin sukukunnat.
\par 16 Ja nämä ovat Leevin poikien nimet heidän polveutumisensa mukaan: Geerson, Kehat ja Merari. Ja Leevin elinaika oli sata kolmekymmentä seitsemän vuotta.
\par 17 Geersonin pojat olivat Libni ja Siimei, sukujensa mukaan.
\par 18 Kehatin pojat olivat Amram, Jishar, Hebron ja Ussiel. Ja Kehatin elinaika oli sata kolmekymmentä kolme vuotta.
\par 19 Merarin pojat olivat Mahli ja Muusi. Nämä ovat Leevin suvut polveutumisensa mukaan.
\par 20 Mutta Amram otti isänsä sisaren Jookebedin vaimokseen, ja tämä synnytti hänelle Aaronin ja Mooseksen. Ja Amramin elinaika oli sata kolmekymmentä seitsemän vuotta.
\par 21 Jisharin pojat olivat Koorah, Nefeg ja Sikri.
\par 22 Ussielin pojat olivat Miisael, Elsafan ja Sitri.
\par 23 Ja Aaron otti vaimoksensa Eliseban, Amminadabin tyttären, Nahsonin sisaren, ja tämä synnytti hänelle Naadabin, Abihun, Eleasarin ja Iitamarin.
\par 24 Koorahin pojat olivat Assir, Elkana ja Abiasaf. Nämä ovat koorahilaisten suvut.
\par 25 Ja Eleasar, Aaronin poika, otti itsellensä vaimon Puutielin tyttäristä, ja tämä synnytti hänelle Piinehaan. Nämä ovat leeviläisten perhekuntien päämiehet heidän sukujensa mukaan.
\par 26 Nämä olivat Aaron ja Mooses, joille Herra sanoi: "Viekää israelilaiset joukkoinensa pois Egyptin maasta".
\par 27 Nämä puhuivat faraolle, Egyptin kuninkaalle, että he aikoivat viedä israelilaiset pois Egyptistä, nimittäin Mooses ja Aaron.
\par 28 Ja niihin aikoihin, kun Herra puhui Moosekselle Egyptin maassa,
\par 29 puhui Herra Moosekselle näin: "Minä olen Herra. Sano faraolle, Egyptin kuninkaalle, kaikki, mitä minä sinulle puhun."
\par 30 Mutta Mooses vastasi Herran edessä: "Katso, minä olen huuliltani ympärileikkaamaton, kuinka siis farao kuulisi minua?"

\chapter{7}

\par 1 Mutta Herra sanoi Moosekselle: "Katso, minä asetan sinut jumalaksi faraolle, ja veljesi Aaron on oleva sinun profeettasi.
\par 2 Puhu kaikki, mitä minä sinun käsken puhua; ja Aaron, sinun veljesi, puhukoon faraolle, että hän päästää israelilaiset maastansa.
\par 3 Mutta minä paadutan faraon sydämen ja teen monta tunnustekoa ja ihmettä Egyptin maassa.
\par 4 Ja farao ei kuule teitä, mutta minä asetan käteni Egyptiä vastaan ja vien pois joukkoni, kansani, israelilaiset, Egyptin maasta, toimittaen suuret rangaistustuomiot.
\par 5 Ja egyptiläiset tulevat tietämään, että minä olen Herra, kun minä ojennan käteni Egyptin yli ja vien pois israelilaiset heidän keskeltänsä."
\par 6 Ja Mooses ja Aaron tekivät, niinkuin Herra oli heitä käskenyt; niin he tekivät.
\par 7 Mutta Mooses oli kahdeksankymmenen vuoden vanha ja Aaron kahdeksankymmenen kolmen vuoden vanha, kun he puhuivat faraon kanssa.
\par 8 Ja Herra puhui Moosekselle ja Aaronille ja sanoi:
\par 9 "Kun farao puhuu teille ja sanoo: 'Tehkää jokin ihmetyö', niin sano sinä Aaronille: 'Ota sauvasi ja heitä se faraon eteen, niin se muuttuu käärmeeksi'".
\par 10 Niin Mooses ja Aaron menivät faraon tykö ja tekivät, niinkuin Herra oli käskenyt. Aaron heitti sauvansa faraon ja hänen palvelijainsa eteen, ja se muuttui käärmeeksi.
\par 11 Ja faraokin kutsui maansa viisaat ja velhot; ja nämä Egyptin tietäjät tekivät samoin taioillansa:
\par 12 he heittivät kukin sauvansa maahan, ja ne muuttuivat käärmeiksi. Mutta Aaronin sauva nieli heidän sauvansa.
\par 13 Ja faraon sydän paatui, eikä hän kuullut heitä, niinkuin Herra oli sanonutkin.
\par 14 Sitten Herra sanoi Moosekselle: "Faraon sydän on kovettunut, hän kieltäytyy päästämästä kansaa.
\par 15 Mene faraon tykö huomenaamuna, kun hän menee veden luo, ja seisahdu hänen tielleen Niilivirran partaalle. Ja ota käteesi se sauva, joka oli muuttunut käärmeeksi,
\par 16 ja sano hänelle: 'Herra, hebrealaisten Jumala, on lähettänyt minut sinun luoksesi ja käskenyt sanoa: Päästä minun kansani palvelemaan minua erämaassa. Mutta katso, sinä et ole totellut tähän asti.
\par 17 Sentähden Herra sanoo näin: Tästä olet tunteva, että minä olen Herra: katso, minä lyön sauvalla, joka on kädessäni, virran veteen, ja se muuttuu vereksi.
\par 18 Ja kalat virrassa kuolevat, ja virta rupeaa haisemaan, niin että egyptiläisiä inhottaa juoda vettä virrasta.'"
\par 19 Ja Herra sanoi Moosekselle: "Sano Aaronille: 'Ota sauvasi ja ojenna kätesi Egyptin vetten yli, sen jokien, kanavien ja lammikkojen yli ja kaikkien niiden paikkojen yli, joihin on vettä kokoontunut, niin ne muuttuvat vereksi. Ja verta on oleva Egyptin maassa kaikkialla, sekä puuastioissa että kiviastioissa.'"
\par 20 Ja Mooses ja Aaron tekivät, niinkuin Herra oli käskenyt. Hän kohotti sauvan ja löi Niilivirran veteen faraon ja hänen palvelijainsa nähden; ja kaikki vesi, joka virrassa oli, muuttui vereksi.
\par 21 Ja kalat virrassa kuolivat, ja virta haisi, niin että egyptiläiset eivät saattaneet juoda vettä virrasta; ja verta oli kaikkialla Egyptin maassa.
\par 22 Mutta Egyptin tietäjät tekivät samoin taioillansa. Ja faraon sydän paatui, eikä hän kuullut heitä, niinkuin Herra oli sanonutkin.
\par 23 Ja farao kääntyi ja palasi kotiinsa eikä välittänyt tästäkään.
\par 24 Mutta kaikki egyptiläiset kaivoivat Niilivirran ympäriltä vettä juodaksensa; sillä he eivät voineet juoda virran vettä.
\par 25 Ja näin oli kulunut seitsemän päivää siitä, kun Herra oli lyönyt Niilivirtaa.

\chapter{8}

\par 1 Sitten Herra sanoi Moosekselle: "Mene faraon luo ja sano hänelle: 'Näin sanoo Herra: Päästä minun kansani palvelemaan minua.
\par 2 Mutta jos kieltäydyt päästämästä heitä, niin katso, minä rankaisen koko sinun maatasi sammakoilla.
\par 3 Ja Niilivirta on vilisevä sammakoita, ja ne nousevat maalle ja tulevat sinun taloosi ja makuuhuoneeseesi ja vuoteeseesi, sekä sinun palvelijaisi taloihin ja kansasi sekaan, sinun leivinuuneihisi ja taikinakaukaloihisi.
\par 4 Jopa sinun ja sinun kansasi ja kaikkien sinun palvelijaisi päälle hyppii sammakoita.'"
\par 5 Ja Herra sanoi Moosekselle: "Sano Aaronille: 'Ojenna kätesi sauvoinensa jokien, kanavien ja lammikkojen yli ja nostata sammakoita Egyptin maahan'".
\par 6 Niin Aaron ojensi kätensä Egyptin vetten yli, ja sammakoita nousi, ja ne peittivät Egyptin maan.
\par 7 Ja tietäjät tekivät samoin taioillansa ja nostattivat sammakoita Egyptin maahan.
\par 8 Niin farao kutsui Mooseksen ja Aaronin ja sanoi: "Rukoilkaa Herraa, että hän ottaisi pois sammakot vaivaamasta minua ja minun kansaani, niin minä päästän kansan uhraamaan Herralle".
\par 9 Mooses sanoi faraolle: "Suvaitse määrätä minulle aika, jonka kuluessa minun on rukoiltava, sinun itsesi, sinun palvelijaisi ja kansasi puolesta, sammakot hävitettäviksi luotasi ja taloistasi, niin että niitä jää ainoastaan Niilivirtaan".
\par 10 Hän vastasi: "Huomiseksi". Niin Mooses sanoi: "Tapahtukoon, niinkuin sanoit, tietääksesi, ettei kukaan ole niinkuin Herra, meidän Jumalamme.
\par 11 Sammakot katoavat luotasi ja taloistasi ja sinun palvelijaisi ja kansasi luota, ja niitä jää ainoastaan Niilivirtaan."
\par 12 Niin Mooses ja Aaron lähtivät faraon luota. Ja Mooses huusi Herran puoleen sammakkojen tähden, jotka hän oli pannut faraon vaivaksi.
\par 13 Ja Herra teki Mooseksen sanan mukaan: sammakot kuolivat huoneista, pihoilta ja kedoilta.
\par 14 Ja he kokosivat niitä läjittäin, ja maa rupesi haisemaan.
\par 15 Mutta kun farao näki päässeensä hengähtämään, kovensi hän sydämensä eikä kuullut heitä, niinkuin Herra oli sanonutkin.
\par 16 Sitten Herra sanoi Moosekselle: "Sano Aaronille: 'Ojenna sauvasi ja lyö maan tomua, niin siitä tulee sääskiä koko Egyptin maahan'".
\par 17 Ja he tekivät niin: Aaron ojensi kätensä ja sauvansa ja löi maan tomua; niin sääsket ahdistivat ihmisiä ja karjaa. Kaikki maan tomu muuttui sääskiksi koko Egyptin maassa.
\par 18 Ja tietäjät tekivät samoin taioillansa saadakseen sääskiä syntymään, mutta he eivät voineet. Ja sääsket ahdistivat ihmisiä ja karjaa.
\par 19 Niin tietäjät sanoivat faraolle: "Tämä on Jumalan sormi". Mutta faraon sydän paatui, eikä hän kuullut heitä, niinkuin Herra oli sanonutkin.
\par 20 Ja Herra sanoi Moosekselle: "Astu huomenaamuna varhain faraon eteen, kun hän menee veden luo, ja sano hänelle: 'Näin sanoo Herra: Päästä minun kansani palvelemaan minua.
\par 21 Sillä jos et päästä minun kansaani, niin katso, minä lähetän paarmoja sinun, sinun palvelijaisi ja sinun kansasi kimppuun ja sinun taloihisi, niin että egyptiläisten talot, jopa se maa, jonka päällä ne ovat, tulevat paarmoja täyteen.
\par 22 Mutta minä erotan sinä päivänä Goosenin maan, jossa minun kansani asuu, ettei sinne paarmoja tule, tietääksesi, että minä olen maan Herra.
\par 23 Näin minä panen pelastuksen erottamaan oman kansani sinun kansastasi. Huomenna on tämä ihme tapahtuva.'"
\par 24 Ja Herra teki niin: paarmoja tuli suuret parvet faraon ja hänen palvelijainsa taloihin; ja paarmat tulivat maan turmioksi koko Egyptin maassa.
\par 25 Niin farao kutsutti Mooseksen ja Aaronin ja sanoi: "Menkää ja uhratkaa Jumalallenne tässä maassa".
\par 26 Mutta Mooses sanoi: "Ei sovi niin tehdä; sillä me uhraamme Herralle, Jumalallemme, sellaista, joka on egyptiläisille kauhistus. Jos me nyt uhraamme egyptiläisten nähden sellaista, joka on heille kauhistus, niin eivätkö he kivitä meitä?
\par 27 Salli meidän mennä kolmen päivän matka erämaahan uhraamaan Herralle, Jumalallemme, niinkuin hän on meille sanonut."
\par 28 Farao sanoi: "Minä päästän teidät uhraamaan Herralle, Jumalallenne, erämaassa; älkää vain menkö kovin kauas. Rukoilkaa minun puolestani."
\par 29 Niin Mooses sanoi: "Katso, kun olen lähtenyt sinun luotasi, rukoilen minä Herraa, ja paarmat häviävät pois huomenna faraolta, hänen palvelijoiltansa ja hänen kansaltaan. Älköön vain farao enää pettäkö, niin ettei hän päästäkään kansaa uhraamaan Herralle."
\par 30 Ja Mooses lähti faraon luota ja rukoili Herraa.
\par 31 Ja Herra teki, niinkuin Mooses oli sanonut: hän vapautti faraon, hänen palvelijansa ja hänen kansansa paarmoista, niin ettei niitä jäänyt ainoatakaan.
\par 32 Mutta farao kovensi sydämensä tälläkin kerralla eikä päästänyt kansaa.

\chapter{9}

\par 1 Sitten Herra sanoi Moosekselle: "Mene faraon tykö ja puhu hänelle: 'Näin sanoo Herra, hebrealaisten Jumala: Päästä minun kansani palvelemaan minua.
\par 2 Sillä jos kieltäydyt päästämästä heitä ja vielä pidätät heitä,
\par 3 niin katso, Herran käsi on lyövä sinun karjaasi, joka on kedolla, hevosia, aaseja, kameleita, nautoja ja lampaita ylen ankaralla ruttotaudilla.
\par 4 Mutta Herra on tekevä erotuksen israelilaisten karjan ja egyptiläisten karjan välillä, niin ettei mitään kuole siitä, mikä on israelilaisten omaa.'"
\par 5 Ja Herra asetti määrätyn ajan ja sanoi: "Huomenna on Herra tekevä niin tässä maassa".
\par 6 Ja seuraavana päivänä Herra teki niin, ja kaikki Egyptin karja kuoli; mutta israelilaisten karjasta ei kuollut ainoatakaan.
\par 7 Ja kun farao lähetti tiedustelemaan, niin katso, israelilaisten karjasta ei ollut kuollut ainoatakaan. Mutta faraon sydän kovettui, eikä hän päästänyt kansaa.
\par 8 Sitten Herra sanoi Moosekselle ja Aaronille: "Ottakaa kahmalonne täyteen pätsin nokea, ja Mooses viskatkoon sen taivasta kohti faraon silmien edessä,
\par 9 niin se muuttuu tomuksi, joka peittää koko Egyptin maan, ja siitä tulee ihmisiin ja karjaan märkäpaiseita kaikkialla Egyptin maassa".
\par 10 Ja he ottivat pätsin nokea ja astuivat faraon eteen, ja Mooses viskasi sen taivasta kohti; niin märkäpaiseita tuli ihmisiin ja karjaan.
\par 11 Eivätkä tietäjätkään voineet pitää puoliaan Moosesta vastaan paiseiden tähden, sillä paiseita oli tietäjissä samoin kuin kaikissa muissakin egyptiläisissä.
\par 12 Mutta Herra paadutti faraon sydämen, niin ettei hän kuullut heitä, niinkuin Herra oli sanonutkin Moosekselle.
\par 13 Sitten Herra sanoi Moosekselle: "Astu huomenaamuna varhain faraon eteen ja sano hänelle: 'Näin sanoo Herra, hebrealaisten Jumala: Päästä minun kansani palvelemaan minua.
\par 14 Muutoin minä tällä kertaa lähetän kaikki vitsaukseni vaivaamaan sinua itseäsi, sinun palvelijoitasi ja kansaasi, tietääksesi, ettei ketään ole minun vertaistani koko maan päällä.
\par 15 Sillä minä olisin jo ojentanut käteni ja lyönyt sinua ja sinun kansaasi ruttotaudilla, niin että olisit kokonaan hävinnyt maan päältä,
\par 16 mutta juuri sitä varten minä olen antanut sinun säilyä, että näyttäisin sinulle voimani ja että minun nimeni julistettaisiin kaiken maan päällä.
\par 17 Jos sinä vielä estät minun kansaani etkä päästä heitä,
\par 18 niin katso, huomenna tähän aikaan minä annan tulla ylen ankaran raesateen, jonka kaltaista ei ole Egyptissä ollut siitä päivästä saakka, jona sen perustus pantiin, aina tähän asti.
\par 19 Lähetä siis nyt saattamaan suojaan karjasi ja kaikki, mitä sinulla on kedolla. Sillä kaikki ihmiset ja karja, jotka ovat kedolla ja joita ei ole korjattu kotiin, joutuvat raesateen alle ja kuolevat.'"
\par 20 Se faraon palvelijoista, joka pelkäsi Herran sanaa, toimitti silloin palvelijansa ja karjansa huoneiden suojaan;
\par 21 mutta joka ei välittänyt Herran sanasta, se jätti palvelijansa ja karjansa kedolle.
\par 22 Ja Herra sanoi Moosekselle: "Ojenna kätesi taivasta kohti, niin raesade kohtaa koko Egyptin maata, ihmisiä, karjaa ja kaikkia kedon kasveja Egyptin maassa".
\par 23 Niin Mooses ojensi sauvansa taivasta kohti, ja Herra antoi jylistä ja lähetti rakeita, ja tulta iski maahan. Näin Herra antoi sataa rakeita yli Egyptin maan.
\par 24 Ja rakeita tuli, ja tulta leimahteli rakeiden keskellä. Raesade oli ylen ankara, niin ettei sellainen ollut kohdannut koko Egyptin maata siitä ajasta saakka, jolloin se tuli asutuksi.
\par 25 Ja rakeet löivät maahan kaikkialla Egyptin maassa kaiken, mitä kedolla oli, sekä ihmiset että karjan; ja rakeet tuhosivat kaikki maan kasvit ja pirstoivat kaikki kedon puut.
\par 26 Ainoastaan Goosenin maata, jossa israelilaiset olivat, ei raesade kohdannut.
\par 27 Niin farao kutsutti Mooseksen ja Aaronin ja sanoi heille: "Minä olen tehnyt syntiä tällä kertaa. Herra on oikeassa, mutta minä ja minun kansani olemme väärässä.
\par 28 Rukoilkaa Herraa, sillä jo on meillä kyllin Jumalan jylinää ja rakeita. Minä päästän teidät, eikä teidän tarvitse enää viipyä."
\par 29 Mooses vastasi hänelle: "Kun lähden kaupungista, niin minä ojennan käteni Herran puoleen, ja jylinä lakkaa eikä rakeita enää tule, tietääksesi, että maa on Herran.
\par 30 Mutta et sinä eivätkä sinun palvelijasi, sen kyllä tiedän, vielä nytkään pelkää Herraa Jumalaa."
\par 31 Niin pellava ja ohra tuhoutuivat, sillä ohra oli tähkällä ja pellava kukalla;
\par 32 mutta nisu ja kolmitahkoinen vehnä eivät turmeltuneet, sillä ne tuleentuvat myöhemmin.
\par 33 Ja Mooses lähti faraon luota, ulos kaupungista, ja ojensi kätensä Herran puoleen; ja jylinä ja rakeet lakkasivat, eikä sade enää vuotanut maahan.
\par 34 Kun farao näki, että sade, rakeet ja jylinä lakkasivat, teki hän yhä edelleen syntiä ja kovensi sydämensä, sekä hän että hänen palvelijansa.
\par 35 Niin faraon sydän paatui, eikä hän päästänyt israelilaisia, niinkuin Herra oli Mooseksen kautta sanonutkin.

\chapter{10}

\par 1 Sitten Herra sanoi Moosekselle: "Mene faraon tykö, sillä minä olen koventanut hänen sydämensä ja hänen palvelijainsa sydämet, että tekisin nämä tunnustekoni heidän keskellänsä
\par 2 ja että sinä kertoisit lapsillesi ja lastesi lapsille, mitä minä olen tehnyt egyptiläisille, ja minun tunnustekoni, jotka olen tehnyt heidän keskellänsä, tietääksenne, että minä olen Herra".
\par 3 Niin Mooses ja Aaron menivät faraon tykö ja sanoivat hänelle: "Näin sanoo Herra, hebrealaisten Jumala: Kuinka kauan sinä kieltäydyt nöyrtymästä minun edessäni? Päästä minun kansani palvelemaan minua.
\par 4 Sillä jos sinä kieltäydyt päästämästä minun kansaani, niin katso, minä annan huomenna tulla sinun maahasi heinäsirkkoja.
\par 5 Ja ne peittävät maan pinnan, niin ettei maata näy, ja ne syövät tähteet siitä, mikä teille rakeilta pelastui ja jäi, ja ne syövät kaikki teidän puunne, jotka kedolla kasvavat.
\par 6 Ja ne täyttävät sinun talosi ja kaikkien sinun palvelijaisi talot ja kaikkien egyptiläisten talot, niin ettei sinun isäsi eivätkä esi-isäsi ole sellaista nähneet siitä päivästä, jona tulivat maailmaan, hamaan tähän päivään asti." Ja hän kääntyi ja lähti pois faraon luota.
\par 7 Mutta faraon palvelijat sanoivat hänelle: "Kuinka kauan tämä mies on oleva meille paulaksi? Päästä miehet palvelemaan Herraa, heidän Jumalaansa. Etkö vieläkään ymmärrä, että Egypti joutuu perikatoon?"
\par 8 Niin Mooses ja Aaron tuotiin takaisin faraon eteen. Ja tämä sanoi heille: "Menkää ja palvelkaa Herraa, Jumalaanne. Vaan ketkä menevät?"
\par 9 Mooses vastasi: "Me menemme, nuoret ja vanhat, me menemme poikinemme ja tyttärinemme, lampainemme ja karjoinemme; sillä meillä on Herran juhla".
\par 10 Niin hän sanoi heille: "Olkoon vain Herra teidän kanssanne, kunhan minä ensin päästän teidät ja teidän vaimonne ja lapsenne. Katsokaa, teillä on paha mielessä.
\par 11 Siitä ei tule mitään. Mutta menkää te, miehet, ja palvelkaa Herraa, sillä sitähän te olette pyytäneetkin." Ja heidät ajettiin pois faraon edestä.
\par 12 Ja Herra sanoi Moosekselle: "Ojenna kätesi Egyptin maan yli ja tuo tänne heinäsirkat; tulkoon niitä Egyptin maahan, ja syökööt ne kaikki maan kasvit, kaikki, mitä rakeilta on jäänyt".
\par 13 Niin Mooses ojensi sauvansa Egyptin maan yli, ja Herra antoi itätuulen puhaltaa yli maan koko sen päivän ja koko yön; kun aamu tuli, toi itätuuli heinäsirkat mukanaan.
\par 14 Ja heinäsirkkoja tuli koko Egyptin maahan, ja ne laskeutuivat ylen suurina laumoina koko Egyptin alueelle; niin paljon ei heinäsirkkoja ollut koskaan sitä ennen tullut eikä sen jälkeen tule.
\par 15 Ne peittivät koko maan pinnan, niin että maa tuli mustaksi; ja ne söivät kaikki maan kasvit ja kaikki puiden hedelmät, jotka olivat rakeilta säilyneet. Niin ei jäänyt mitään vihantaa jäljelle puihin eikä kedon kasveihin koko Egyptin maassa.
\par 16 Silloin farao kutsutti kiiruusti Mooseksen ja Aaronin ja sanoi: "Minä olen rikkonut Herraa, teidän Jumalaanne, ja teitä vastaan.
\par 17 Anna nyt anteeksi minun rikkomukseni vielä tämä kerta, ja rukoilkaa Herraa, Jumalaanne, että hän poistaisi minulta ainakin tämän surman."
\par 18 Niin hän lähti pois faraon luota ja rukoili Herraa.
\par 19 Ja Herra käänsi tuulen hyvin kovaksi länsituuleksi; se vei heinäsirkat mukanaan ja painoi ne Kaislamereen, niin ettei yhtäkään heinäsirkkaa jäänyt koko Egyptin alueelle.
\par 20 Mutta Herra paadutti faraon sydämen, niin ettei hän päästänyt israelilaisia.
\par 21 Sitten Herra sanoi Moosekselle: "Ojenna kätesi taivasta kohti, niin Egyptin maahan tulee sellainen pimeys, että siihen voi käsin tarttua".
\par 22 Ja Mooses ojensi kätensä taivasta kohti, ja koko Egyptin maahan tuli synkeä pimeys kolmeksi päiväksi.
\par 23 Ei kukaan voinut nähdä toistansa, eikä kukaan voinut liikkua paikaltansa kolmeen päivään. Mutta kaikilla israelilaisilla oli valoisata asuinpaikoissansa.
\par 24 Niin farao kutsui Mooseksen ja sanoi: "Menkää ja palvelkaa Herraa; ainoastaan lampaanne ja karjanne jääkööt tänne. Myöskin vaimonne ja lapsenne menkööt teidän mukananne."
\par 25 Mutta Mooses sanoi: "Sinun on annettava mukaamme myös teurasuhrit ja polttouhrit, uhrataksemme niitä Herralle, Jumalallemme.
\par 26 Karjammekin täytyy tulla meidän kanssamme, ei sorkkaakaan saa jäädä, sillä siitä meidän on otettava uhrit palvellaksemme Herraa, Jumalaamme, emmekä itsekään tiedä, ennenkuin tulemme sinne, mitä meidän on uhrattava palvellessamme Herraa."
\par 27 Mutta Herra paadutti faraon sydämen, niin että hän ei tahtonut päästää heitä.
\par 28 Ja farao sanoi hänelle: "Mene pois luotani ja varo, ettet enää tule minun kasvojeni eteen; sillä sinä päivänä, jona tulet minun kasvojeni eteen, sinä olet kuoleva".
\par 29 Mooses vastasi: "Oikein sinä puhuit; minä en tule tämän jälkeen sinun kasvojesi eteen".

\chapter{11}

\par 1 Sitten Herra sanoi Moosekselle: "Vielä yhden vitsauksen minä annan tulla faraolle ja Egyptiin; sen jälkeen hän päästää teidät täältä. Ja kun hän todella päästää teidät, niin hän ajamalla ajaa teidät täältä.
\par 2 Puhu siis nyt kansalle, että he, jokainen mies ja jokainen vaimo, pyytävät lähimmäisiltänsä hopea- ja kultakaluja."
\par 3 Ja Herra antoi kansan päästä egyptiläisten suosioon. Myöskin Mooses oli hyvin arvossapidetty mies Egyptin maassa, sekä faraon palvelijain että kansan silmissä.
\par 4 Ja Mooses sanoi: "Näin sanoo Herra: Puoliyön aikana minä lähden kulkemaan kautta Egyptin maan.
\par 5 Ja kaikki esikoiset Egyptin maassa kuolevat, valtaistuimellansa istuvan faraon esikoisesta käsikiveä vääntävän orjattaren esikoiseen asti, ynnä kaikki karjan esikoiset.
\par 6 Ja koko Egyptin maassa on oleva kova valitus, jonka kaltaista ei ole ollut eikä koskaan tule.
\par 7 Mutta kenellekään israelilaiselle ei koirakaan ole muriseva, ei ihmiselle eikä eläimelle, tietääksesi, että Herra tekee erotuksen egyptiläisten ja Israelin välillä.
\par 8 Ja kaikki nämä sinun palvelijasi tulevat minun luokseni, kumartavat minua ja sanovat: 'Mene pois, sinä ja kaikki kansa, joka sinua seuraa'. Ja sen jälkeen minä menen." Ja niin hän lähti faraon luota vihasta hehkuen.
\par 9 Mutta Herra sanoi Moosekselle: "Farao ei kuule teitä, että paljon minun ihmeitäni tapahtuisi Egyptin maassa".
\par 10 Ja Mooses ja Aaron tekivät kaikki nämä ihmeet faraon edessä; mutta Herra paadutti faraon sydämen, niin ettei hän päästänyt israelilaisia maastansa.

\chapter{12}

\par 1 Ja Herra puhui Moosekselle ja Aaronille Egyptin maassa sanoen:
\par 2 "Tämä kuukausi olkoon teillä kuukausista ensimmäinen; siitä alottakaa vuoden kuukaudet.
\par 3 Puhukaa koko Israelin kansalle ja sanokaa: Tämän kuun kymmenentenä päivänä ottakoon kukin perheenisäntä itsellensä karitsan, yhden karitsan joka perhekuntaa kohti.
\par 4 Mutta jos perhe on liian pieni koko karitsaa syömään, niin ottakoon lähimmän naapurinsa kanssa yhteisen karitsan, henkilöluvun mukaan. Karitsaa kohti laskekaa niin monta, että voivat sen syödä.
\par 5 Ja karitsanne olkoon virheetön, vuoden vanha urospuoli; lampaista tai vuohista se ottakaa.
\par 6 Ja pitäkää se tallella neljänteentoista päivään tätä kuuta; silloin Israelin koko seurakunta teurastakoon sen iltahämärässä.
\par 7 Ja he ottakoot sen verta ja sivelkööt sillä molemmat pihtipielet ja ovenpäällisen niissä taloissa, joissa he sitä syövät.
\par 8 Ja he syökööt lihan samana yönä; tulessa paistettuna, happamattoman leivän ja katkerain yrttien kanssa he sen syökööt.
\par 9 Älkää syökö siitä mitään raakana tai vedessä keitettynä, vaan tulessa paistettuna päineen, jalkoineen ja sisälmyksineen.
\par 10 Älkääkä jättäkö siitä mitään huomenaamuksi; mutta jos jotakin siitä jäisi huomenaamuksi, niin polttakaa se tulessa.
\par 11 Ja syökää se näin: kupeet vyötettyinä, kengät jalassanne ja sauva kädessänne; ja syökää se kiiruusti. Tämä on pääsiäinen Herran kunniaksi.
\par 12 Sillä minä kuljen sinä yönä kautta Egyptin maan ja surmaan kaikki esikoiset Egyptin maassa, sekä ihmiset että eläimet, ja panen toimeen rangaistustuomion, jonka minä olen langettanut kaikista Egyptin jumalista. Minä olen Herra.
\par 13 Ja veri on oleva merkki, teille suojelukseksi, taloissa, joissa olette; sillä kun minä näen veren, niin minä menen teidän ohitsenne, eikä rangaistus ole tuhoava teitä, kun minä rankaisen Egyptin maata.
\par 14 Ja tämä päivä olkoon teille muistopäivä, ja viettäkää sitä Herran juhlana; sukupolvesta sukupolveen viettäkää sitä ikuisena säädöksenä.
\par 15 Seitsemän päivää syökää happamatonta leipää; jo ensimmäisenä päivänä korjatkaa pois hapan taikina taloistanne. Sillä jokainen, joka hapanta syö, ensimmäisestä päivästä seitsemänteen päivään asti, hävitettäköön Israelista.
\par 16 Ensimmäisenä päivänä pitäkää pyhä kokous ja samoin myös seitsemäntenä päivänä pyhä kokous. Mitään työtä älköön tehtäkö niinä päivinä; ainoastaan se, mitä kukin tarvitsee ruuaksi, ainoastaan se valmistettakoon.
\par 17 Ja pitäkää tätä happamattoman leivän juhlaa, sillä juuri sinä päivänä minä vein teidän joukkonne pois Egyptin maasta; sentähden pitäkää tätä päivää, sukupolvesta sukupolveen, ikuisena säädöksenä.
\par 18 Ensimmäisessä kuussa, kuukauden neljäntenätoista päivänä, ehtoolla, syökää happamatonta leipää, ja niin tehkää aina saman kuun yhdennenkolmatta päivän ehtooseen asti.
\par 19 Seitsemään päivään älköön hapanta taikinaa olko teidän taloissanne; sillä jokainen, joka hapanta leipää syö, hävitettäköön Israelin kansasta, olipa hän muukalainen tai maassasyntynyt.
\par 20 Älkää syökö mitään hapanta, vaan syökää happamatonta leipää, missä asuttekin."
\par 21 Ja Mooses kutsui kaikki Israelin vanhimmat ja sanoi heille: "Menkää ja ottakaa lammas kutakin perhekuntaanne kohti ja teurastakaa pääsiäislammas.
\par 22 Ja ottakaa isoppikimppu, kastakaa se vereen, joka on maljassa, ja sivelkää ovenpäällinen ja molemmat pihtipielet sillä verellä, joka on maljassa. Älköönkä kukaan menkö ulos talonsa ovesta ennen aamua.
\par 23 Sillä Herra kulkee rankaisemassa egyptiläisiä; mutta kun Herra näkee veren ovenpäällisessä ja molemmissa pihtipielissä, menee hän sen oven ohi eikä salli tuhoojan tulla teidän taloihinne teitä vitsauksella lyömään.
\par 24 Noudattakaa tätä; se olkoon ikuinen säädös sinulle ja sinun lapsillesi.
\par 25 Ja kun te tulette siihen maahan, jonka Herra on teille antava, niinkuin hän on sanonut, niin noudattakaa näitä menoja.
\par 26 Kun sitten lapsenne kysyvät teiltä: 'Mitä nämä menot merkitsevät?'
\par 27 niin vastatkaa: 'Tämä on pääsiäisuhri Herralle, joka meni israelilaisten talojen ohi Egyptissä, kun hän rankaisi egyptiläisiä, mutta säästi meidän kotimme'." Silloin kansa kumartui maahan ja rukoili.
\par 28 Ja israelilaiset menivät ja tekivät, niinkuin Herra oli Moosekselle ja Aaronille käskyn antanut; niin he tekivät.
\par 29 Ja puoliyön aikana tapahtui, että Herra surmasi kaikki Egyptin maan esikoiset, valtaistuimellaan istuvan faraon esikoisesta vankikuopassa olevan vangin esikoiseen asti, ynnä kaikki karjan esikoiset.
\par 30 Niin farao nousi sinä yönä ja kaikki hänen palvelijansa ynnä kaikki egyptiläiset, ja kova valitus oli Egyptissä; sillä ei ollut yhtään taloa, jossa ei ollut kuollutta.
\par 31 Ja hän kutsutti yöllä Mooseksen ja Aaronin ja sanoi: "Nouskaa ja menkää pois minun kansani joukosta, sekä te että israelilaiset; menkää ja palvelkaa Herraa, niinkuin olette puhuneet.
\par 32 Ottakaa myös lampaanne ja karjanne mukaanne, niinkuin olette puhuneet, ja menkää ja rukoilkaa minullekin siunausta."
\par 33 Ja egyptiläiset ahdistivat kansaa jouduttaaksensa heidän lähtöänsä maasta pois, sillä he ajattelivat: "Me kuolemme kaikki".
\par 34 Ja kansa otti taikinansa, ennenkuin se oli hapannut, ja kantoi taikinakaukalonsa vaippoihin käärittyinä olkapäillään.
\par 35 Ja israelilaiset olivat tehneet Mooseksen sanan mukaan: he olivat pyytäneet egyptiläisiltä hopea- ja kultakaluja sekä vaatteita.
\par 36 Ja Herra oli antanut kansan päästä egyptiläisten suosioon, niin että nämä suostuivat heidän pyyntöönsä; ja niin he veivät saalista egyptiläisiltä.
\par 37 Ja israelilaiset lähtivät liikkeelle ja kulkivat Ramseksesta Sukkotiin; heitä oli noin kuusisataa tuhatta jalkamiestä, paitsi vaimoja ja lapsia.
\par 38 Ja myös paljon sekakansaa meni heidän kanssansa, sekä lampaita ja raavaskarjaa suuret laumat.
\par 39 Ja he leipoivat siitä taikinasta, jonka olivat tuoneet Egyptistä, happamattomia leipiä; sillä se ei ollut hapannut, kun heidät karkoitettiin Egyptistä. He eivät olleet saaneet viipyä, eivätkä he myöskään olleet valmistaneet itsellensä evästä.
\par 40 Mutta aika, jonka israelilaiset olivat asuneet Egyptissä, oli neljäsataa kolmekymmentä vuotta.
\par 41 Kun ne neljäsataa kolmekymmentä vuotta olivat kuluneet, niin juuri sinä päivänä kaikki Herran joukot lähtivät Egyptin maasta.
\par 42 Valvontayö Herran kunniaksi on tämä yö, koska hän silloin vei heidät pois Egyptin maasta; tämän yön kaikki israelilaiset valvokoot Herran kunniaksi, sukupolvesta sukupolveen.
\par 43 Ja Herra sanoi Moosekselle ja Aaronille: "Tämä on säädös pääsiäislampaasta: Kukaan muukalainen älköön siitä syökö;
\par 44 mutta jokainen rahalla ostettu orja syököön siitä, kunhan olet ympärileikannut hänet.
\par 45 Loinen ja päiväpalkkalainen älkööt siitä syökö.
\par 46 Samassa talossa se syötäköön; älköön mitään siitä lihasta vietäkö talosta ulos, älkääkä siitä luuta rikkoko.
\par 47 Koko Israelin seurakunta viettäköön sitä ateriaa.
\par 48 Ja jos joku muukalainen asuu sinun luonasi ja tahtoo viettää pääsiäistä Herran kunniaksi, niin ympärileikkauttakoon kaikki miehenpuolensa ja sitten käyköön sitä viettämään ja olkoon silloin niinkuin maassasyntynyt. Mutta yksikään ympärileikkaamaton älköön siitä syökö.
\par 49 Sama laki olkoon maassasyntyneellä ja muukalaisella, joka asuu teidän keskuudessanne."
\par 50 Ja kaikki israelilaiset tekivät, niinkuin Herra oli Moosekselle ja Aaronille käskyn antanut; niin he tekivät. Ja juuri sinä päivänä Herra vei israelilaiset joukkoinensa pois Egyptin maasta.

\chapter{13}

\par 1 Ja Herra puhui Moosekselle sanoen:
\par 2 "Pyhitä minulle jokainen esikoinen, jokainen, joka israelilaisten seassa, sekä ihmisistä että karjasta, avaa äidinkohdun; se on minun".
\par 3 Ja Mooses sanoi kansalle: "Muistakaa tämä päivä, jona te lähditte Egyptistä, orjuuden pesästä; sillä Herra vei teidät sieltä pois väkevällä kädellä. Sentähden älköön mitään hapanta syötäkö.
\par 4 Tänä päivänä aabib-kuussa te lähditte.
\par 5 Ja kun Herra vie sinut kanaanilaisten, heettiläisten, amorilaisten, hivviläisten ja jebusilaisten maahan, jonka hän sinun isillesi vannotulla valalla on luvannut antaa sinulle, siihen maahan, joka vuotaa maitoa ja mettä, niin vietä tämä jumalanpalvelus tässä kuussa.
\par 6 Syö seitsemänä päivänä happamatonta leipää, ja seitsemäntenä päivänä olkoon Herran juhla.
\par 7 Happamatonta leipää syötäköön näinä seitsemänä päivänä; älköön mitään hapanta olko, älköön olko hapanta taikinaa koko sinun maassasi.
\par 8 Ja kerro pojallesi sinä päivänä ja sano: 'Näin tehdään sen johdosta, mitä Herra minulle teki, kun minä lähdin Egyptistä'.
\par 9 Ja se olkoon merkkinä sinun kädessäsi ja muistutuksena sinun otsallasi, että Herran laki olisi sinun suussasi; sillä Herra vei sinut väkevällä kädellä pois Egyptistä.
\par 10 Sentähden noudata tätä säädöstä määräaikanansa vuodesta vuoteen.
\par 11 Ja kun Herra on vienyt sinut kanaanilaisten maahan, niinkuin hän on sinulle ja sinun isillesi vannonut, ja kun hän on sen sinulle antanut,
\par 12 niin luovuta Herralle kaikki, mikä avaa äidinkohdun; ja kaikki ensiksisyntyneet karjastasi, urospuolet, olkoot Herran.
\par 13 Mutta jokainen aasin ensiksisynnyttämä lunasta lampaalla; mutta jos et sitä lunasta, niin taita siltä niska. Ja lunasta jokainen ihmisen esikoinen poikiesi seassa.
\par 14 Ja kun sinun poikasi vastaisuudessa kysyy sinulta ja sanoo: 'Mitä tämä merkitsee?' niin vastaa hänelle: 'Herra vei meidät väkevällä kädellä pois Egyptistä, orjuuden pesästä.
\par 15 Sillä kun farao paatui ja kieltäytyi päästämästä meitä, surmasi Herra kaikki esikoiset Egyptin maassa, ihmisten esikoisista karjan esikoisiin asti. Sentähden minä uhraan Herralle jokaisen urospuolen, joka avaa äidinkohdun, ja jokaisen esikoisen pojistani minä lunastan.'
\par 16 Ja se olkoon merkkinä sinun kädessäsi ja muistolauseena sinun otsallasi; sillä Herra vei meidät väkevällä kädellä pois Egyptistä."
\par 17 Mutta kun farao oli päästänyt kansan, ei Jumala johdattanut heitä sitä tietä, joka kulki filistealaisten maan kautta, vaikka se oli suorin, sillä Jumala ajatteli, että kansa ehkä katuisi nähdessänsä sodan syttyvän ja palaisi Egyptiin;
\par 18 vaan Jumala antoi kansan poiketa erämaan tietä myöten Kaislamerta kohti. Ja taisteluun valmiina israelilaiset lähtivät Egyptin maasta.
\par 19 Ja Mooses otti mukaansa Joosefin luut; sillä tämä oli vannottanut israelilaisia ja sanonut: "Kun Jumala pitää huolen teistä, viekää silloin minun luuni täältä mukananne".
\par 20 Ja he lähtivät liikkeelle Sukkotista ja leiriytyivät Eetamiin, erämaan reunaan.
\par 21 Ja Herra kulki heidän edellänsä, päivällä pilvenpatsaassa johdattaaksensa heitä tietä myöten ja yöllä tulenpatsaassa valaistaksensa heidän kulkunsa, niin että he voivat vaeltaa sekä päivällä että yöllä.
\par 22 Pilvenpatsas ei poistunut päivällä eikä tulenpatsas yöllä kansan edestä.

\chapter{14}

\par 1 Herra puhui Moosekselle sanoen:
\par 2 "Sano israelilaisille, että he kääntyvät takaisin ja leiriytyvät Pii-Hahirotin kohdalle Migdolin ja meren välille; leiriytykää vastapäätä Baal-Sefonia meren rannalle.
\par 3 Ja farao on ajatteleva, että israelilaiset ovat eksyneet maassa ja että erämaa on saartanut heidät.
\par 4 Ja minä paadutan faraon sydämen, niin että hän ajaa heitä takaa; mutta minä näytän kunniani tuhoamalla faraon ja koko hänen sotajoukkonsa; ja niin egyptiläiset tulevat tietämään, että minä olen Herra." Ja he tekivät niin.
\par 5 Kun Egyptin kuninkaalle ilmoitettiin, että kansa oli paennut, muuttui faraon ja hänen palvelijainsa mieli kansaa kohtaan, ja he sanoivat: "Mitä teimmekään, kun päästimme Israelin meitä palvelemasta!"
\par 6 Ja hän valjastutti hevoset sotavaunujensa eteen ja otti väkensä mukaansa;
\par 7 ja hän otti kuudetsadat valitut sotavaunut sekä kaikki muut Egyptin sotavaunut ynnä vaunusoturit niihin kaikkiin.
\par 8 Sillä Herra paadutti faraon, Egyptin kuninkaan, sydämen, niin että hän lähti ajamaan takaa israelilaisia, vaikka israelilaiset olivat lähteneet matkaan voimallisen käden suojassa.
\par 9 Ja egyptiläiset, kaikki faraon hevoset, sotavaunut ja ratsumiehet ja koko hänen sotajoukkonsa, ajoivat heitä takaa ja saavuttivat heidät leiriytyneinä meren rannalle, Pii-Hahirotin kohdalle, vastapäätä Baal-Sefonia.
\par 10 Ja kun farao oli lähellä, nostivat israelilaiset silmänsä ja näkivät, että egyptiläiset olivat tulossa heidän jäljessänsä. Silloin israelilaiset peljästyivät kovin ja huusivat Herraa.
\par 11 Ja he sanoivat Moosekselle: "Eikö Egyptissä ollut hautoja, kun toit meidät tänne erämaahan kuolemaan? Mitä teit meille, kun johdatit meidät pois Egyptistä!
\par 12 Emmekö sanoneet tätä sinulle Egyptissä? Sanoimmehan: Anna meidän olla rauhassa, että palvelisimme egyptiläisiä. Sillä parempi olisi ollut palvella egyptiläisiä kuin kuolla erämaahan."
\par 13 Mooses vastasi kansalle: "Älkää peljätkö; pysykää paikoillanne, niin te näette, minkä pelastuksen Herra tänä päivänä antaa teille; sillä sellaista, minkä näette egyptiläisille tapahtuvan tänä päivänä, ette koskaan enää tule näkemään.
\par 14 Herra sotii teidän puolestanne, ja te olkaa hiljaa."
\par 15 Ja Herra sanoi Moosekselle: "Miksi huudat minulle? Sano israelilaisille, että he lähtevät liikkeelle.
\par 16 Mutta sinä nosta sauvasi ja ojenna kätesi meren yli ja halkaise se, niin että israelilaiset voivat käydä meren poikki kuivaa myöten.
\par 17 Ja katso, minä paadutan egyptiläisten sydämet, niin että he tulevat heidän perässänsä; minä näytän kunniani tuhoamalla faraon ja koko hänen sotajoukkonsa, hänen sotavaununsa ja ratsumiehensä.
\par 18 Ja egyptiläiset tulevat tietämään, että minä olen Herra, kun minä näytän kunniani tuhoamalla faraon, hänen sotavaununsa ja ratsumiehensä."
\par 19 Niin Jumalan enkeli, joka oli kulkenut Israelin joukon edellä, siirtyi kulkemaan heidän takanansa; ja pilvenpatsas siirtyi heidän edeltänsä ja asettui heidän taaksensa
\par 20 ja tuli egyptiläisten joukon ja Israelin joukon väliin; ja pilvi oli pimeä noille ja valaisi yön näille. Näin koko yönä toinen joukko ei voinut lähestyä toista.
\par 21 Ja Mooses ojensi kätensä meren yli. Niin Herra saattoi vahvalla itätuulella, joka puhalsi koko yön, meren väistymään ja muutti meren kuivaksi maaksi; ja vesi jakautui kahtia.
\par 22 Ja israelilaiset kulkivat meren poikki kuivaa myöten, ja vesi oli heillä muurina sekä oikealla että vasemmalla puolella.
\par 23 Ja egyptiläiset, kaikki faraon hevoset, hänen sotavaununsa ja ratsumiehensä, ajoivat heitä takaa ja tulivat heidän perässänsä keskelle merta.
\par 24 Kun aamuvartio tuli, katsahti Herra egyptiläisten joukkoa tulenpatsaasta ja pilvestä ja saattoi hämminkiin egyptiläisten joukon.
\par 25 Ja hän antoi heidän vaunujensa pyöräin irtautua, niin että heille kävi vaikeaksi päästä eteenpäin. Niin egyptiläiset sanoivat: "Paetkaamme Israelia, sillä Herra sotii heidän puolestansa egyptiläisiä vastaan".
\par 26 Mutta Herra sanoi Moosekselle: "Ojenna kätesi meren yli, että vedet palautuisivat ja peittäisivät egyptiläiset, heidän sotavaununsa ja ratsumiehensä".
\par 27 Niin Mooses ojensi kätensä meren yli, ja aamun koittaessa meri palasi paikoillensa, egyptiläisten paetessa sitä vastaan; ja Herra syöksi egyptiläiset keskelle merta.
\par 28 Ja vedet palasivat ja peittivät sotavaunut ja ratsumiehet, koko faraon sotajoukon, joka oli seurannut heitä mereen; ei yksikään heistä pelastunut.
\par 29 Mutta israelilaiset kulkivat kuivaa myöten meren poikki, ja vesi oli heillä muurina sekä oikealla että vasemmalla puolella.
\par 30 Niin Herra pelasti Israelin sinä päivänä egyptiläisten käsistä, ja Israel näki egyptiläiset kuolleina meren rannalla.
\par 31 Ja Israel näki sen suuren teon, jonka Herra teki tuhotessaan egyptiläiset. Silloin kansa pelkäsi Herraa ja uskoi Herraan ja uskoi hänen palvelijaansa Moosesta.

\chapter{15}

\par 1 Silloin Mooses ja israelilaiset veisasivat Herralle tämän virren; he sanoivat näin: "Minä veisaan Herralle, sillä hän on ylen korkea; hevoset ja miehet hän mereen syöksi.
\par 2 Herra on minun väkevyyteni ja ylistysvirteni, ja hänestä tuli minulle pelastus. Hän on minun Jumalani, ja minä ylistän häntä, hän on minun isäni Jumala, ja minä kunnioitan häntä.
\par 3 Herra on sotasankari, Herra on hänen nimensä.
\par 4 Faraon vaunut ja hänen sotajoukkonsa hän suisti mereen, hänen valitut vaunusoturinsa hukkuivat Kaislamereen.
\par 5 Syvyys peitti heidät; he vajosivat pohjaan niinkuin kivi.
\par 6 Sinun oikea kätesi, Herra, sinä voimassa jalo, sinun oikea kätesi, Herra, murskaa vihollisen.
\par 7 Valtasuuruudessasi sinä kukistat vastustajasi. Sinä päästät vihasi valloilleen, ja se kuluttaa heidät niinkuin korret.
\par 8 Ja sinun vihasi puhalluksesta kasaantuivat vedet, laineet seisahtuivat roukkioiksi, syvyyden aallot hyytyivät keskelle merta.
\par 9 Vihollinen sanoi: 'Minä ajan takaa, minä saavutan heidät, minä jaan saaliin ja tyydytän heissä kostonhimoni; minä paljastan miekkani, minun käteni hävittää heidät'.
\par 10 Sinun tuulesi puhalsi, ja meri peitti heidät; he upposivat valtavesiin niinkuin lyijy.
\par 11 Herra, kuka on sinun vertaisesi jumalien joukossa! Kuka on sinun vertaisesi, sinä pyhyydessä jalo; sinä ylistettävissä teoissa peljättävä, sinä ihmeitten tekijä!
\par 12 Sinä ojensit oikean kätesi, maa nielaisi heidät.
\par 13 Mutta armossasi sinä johdatit lunastamaasi kansaa, sinä veit sen voimallasi pyhään asuntoosi.
\par 14 Kansat kuulivat sen ja vapisivat, tuska valtasi Filistean asukkaat.
\par 15 Silloin peljästyivät Edomin ruhtinaat, Mooabin sankarit valtasi vavistus, kaikki Kanaanin asukkaat menehtyivät pelkoon.
\par 16 Kauhu ja väristys valtasi heidät; sinun käsivartesi väkevyyden tähden he kävivät mykiksi niinkuin kivi. Niin sinun kansasi, Herra, kulkee perille, kulkee perille se kansa, jonka olet itsellesi hankkinut.
\par 17 Sinä viet heidät perille ja istutat heidät vuorelle, joka on sinun perintöosasi, paikkaan, jonka sinä, Herra, olet asunnoksesi valmistanut, pyhäkköösi, Herra, jonka sinun kätesi ovat tehneet.
\par 18 Herra on kuningas aina ja iankaikkisesti."
\par 19 Sillä kun faraon hevoset ja hänen sotavaununsa ja ratsumiehensä menivät mereen, palautti Herra meren vedet heidän päällensä; mutta israelilaiset kulkivat kuivaa myöten meren poikki.
\par 20 Ja naisprofeetta Mirjam, Aaronin sisar, otti vaskirummun käteensä, ja kaikki naiset seurasivat häntä vaskirumpuja lyöden ja karkeloiden.
\par 21 Ja Mirjam viritti heille virren: "Veisatkaa Herralle, sillä hän on ylen korkea, hevoset ja miehet hän mereen syöksi".
\par 22 Sitten Mooses antoi israelilaisten lähteä liikkeelle Kaislameren luota, ja he menivät Suurin erämaahan. Ja he vaelsivat kolme päivää erämaassa löytämättä vettä.
\par 23 Sitten he tulivat Maaraan; mutta he eivät voineet juoda Maaran vettä; sillä se oli karvasta. Sentähden paikka sai nimen Maara.
\par 24 Niin kansa napisi Moosesta vastaan ja sanoi: "Mitä me juomme?"
\par 25 Mutta hän huusi Herran puoleen; ja Herra osoitti hänelle puun, jonka hän heitti veteen, ja vesi tuli makeaksi. Siellä hän antoi kansalle lain ja oikeuden, ja siellä hän koetteli sitä.
\par 26 Hän sanoi: "Jos sinä kuulet Herraa, Jumalaasi, ja teet, mikä on oikein hänen silmissänsä, tarkkaat hänen käskyjänsä ja noudatat kaikkea hänen lakiansa, niin minä en pane sinun kärsittäväksesi yhtäkään niistä vaivoista, jotka olen pannut egyptiläisten kärsittäviksi, sillä minä olen Herra, sinun parantajasi".
\par 27 Sitten he tulivat Eelimiin; siellä oli kaksitoista vesilähdettä ja seitsemänkymmentä palmupuuta. Ja he leiriytyivät siellä veden ääreen.

\chapter{16}

\par 1 Sitten he lähtivät, koko Israelin kansa, liikkeelle Eelimistä ja tulivat Siinin erämaahan, joka on Eelimin ja Siinain välillä, toisen kuun viidentenätoista päivänä siitä, kun he olivat lähteneet Egyptin maasta.
\par 2 Ja koko Israelin kansa napisi Moosesta ja Aaronia vastaan erämaassa;
\par 3 ja israelilaiset sanoivat heille: "Jospa olisimme saaneet surmamme Herran kädestä Egyptin maassa, jossa istuimme lihapatain ääressä ja leipää oli kyllin syödäksemme! Mutta te olette tuoneet meidät tähän erämaahan antaaksenne koko tämän joukon kuolla nälkään."
\par 4 Niin Herra sanoi Moosekselle: "Katso, minä annan sataa teille leipää taivaasta. Ja kansa menköön ja kootkoon kunakin päivänä sen päivän tarpeen. Näin minä koettelen heitä, vaeltavatko he minun lakini mukaan vai eivät.
\par 5 Ja kun he kuudentena päivänä valmistavat sen, mitä ovat tuoneet kotiin, niin sitä on oleva kaksi kertaa niin paljon, kuin mitä he muutoin joka päivä kokoavat."
\par 6 Ja Mooses ja Aaron sanoivat kaikille israelilaisille: "Tänä iltana te tulette tietämään, että Herra on vienyt teidät pois Egyptin maasta,
\par 7 ja huomenaamuna te tulette näkemään Herran kunnian, koska Herra on kuullut teidän napinanne häntä vastaan. Sillä mitä olemme me, että te meitä vastaan napisette?"
\par 8 Ja Mooses sanoi vielä: "Herra antaa teille tänä iltana lihaa syödäksenne ja huomenna leipää yltäkyllin ravinnoksenne, koska Herra on kuullut teidän napinanne, kun olette napisseet häntä vastaan. Sillä mitä olemme me? Ette ole napisseet meitä vastaan, vaan Herraa vastaan."
\par 9 Ja Mooses sanoi Aaronille: "Sano koko Israelin kansalle: Astukaa Herran eteen, sillä hän on kuullut teidän napinanne".
\par 10 Ja kun Aaron oli puhunut tämän koko Israelin seurakunnalle, kääntyivät he erämaahan päin, ja katso, Herran kunnia näkyi pilvessä.
\par 11 Ja Herra puhui Moosekselle sanoen:
\par 12 "Minä olen kuullut israelilaisten napinan. Puhu heille ja sano: Iltahämärässä teillä on oleva lihaa syödäksenne, ja huomenna on teillä leipää yltäkyllin; niin te tulette tietämään, että minä olen Herra, teidän Jumalanne."
\par 13 Ja illalla tuli viiriäisiä, ja ne peittivät leirin, ja aamulla laskeutui kastesumu leirin ympärille.
\par 14 Ja kun kastesumu oli haihtunut, katso, erämaassa oli maan pinnalla jotakin hienoa, suomujen tapaista, jotakin hienoa niinkuin härmä.
\par 15 Kun israelilaiset näkivät sen, kyselivät he toisiltansa: "Mitä tämä on?" Sillä he eivät tienneet, mitä se oli. Ja Mooses sanoi heille: "Tämä on se leipä, jonka Herra on antanut teille syötäväksi.
\par 16 Ja näin on Herra käskenyt: Kootkaa sitä, jokainen tarpeenne mukaan; ottakaa goomer-mitta mieheen perheenne pääluvun mukaan, kukin niin monelle, kuin kullakin majassansa on."
\par 17 Ja israelilaiset tekivät niin, ja he kokosivat, yksi enemmän, toinen vähemmän.
\par 18 Mutta kun he mittasivat sen goomer-mitalla, niin ei sille jäänyt liikaa, joka oli koonnut enemmän, eikä siltä puuttunut, joka oli koonnut vähemmän; jokainen oli koonnut niin paljon, kuin hän tarvitsi.
\par 19 Ja Mooses sanoi heille: "Älköön kukaan jättäkö siitä mitään huomiseksi".
\par 20 Mutta he eivät kuulleet Moosesta, vaan muutamat jättivät siitä jotakin huomiseksi. Niin siihen kasvoi matoja, ja se rupesi haisemaan; ja Mooses vihastui heihin.
\par 21 Ja he kokosivat sitä joka aamu sen verran, kuin kukin tarvitsi; mutta kun aurinko alkoi paahtaa, niin se suli.
\par 22 Mutta kuudentena päivänä he kokosivat sitä leipää kaksinkertaisesti, kaksi goomeria kullekin. Ja kaikki kansan päämiehet tulivat ja ilmoittivat sen Moosekselle.
\par 23 Niin hän sanoi heille: "Tämä tapahtuu Herran sanan mukaan; huomenna on levon päivä, Herralle pyhitetty sapatti. Mitä leivotte, se leipokaa, ja mitä keitätte, se keittäkää; mutta kaikki, mitä tähteeksi jää, pankaa talteen huomiseksi."
\par 24 Ja he panivat sen talteen huomiseksi, niinkuin Mooses oli käskenyt, eikä se ruvennut haisemaan, eikä siihen tullut matoja.
\par 25 Ja Mooses sanoi: "Syökää se tänä päivänä, sillä tänä päivänä on Herran sapatti; tänään ette löydä kedolta mitään.
\par 26 Kuutena päivänä on teidän sitä koottava, mutta seitsemäntenä päivänä on sapatti; silloin ei sitä ole."
\par 27 Kuitenkin muutamat kansasta lähtivät seitsemäntenä päivänä kokoamaan sitä, mutta he eivät löytäneet mitään.
\par 28 Niin Herra sanoi Moosekselle: "Kuinka kauan aiotte kieltäytyä noudattamasta minun käskyjäni ja lakejani?
\par 29 Katsokaa, Herra on antanut teille sapatin; sentähden hän antaa teille kuudentena päivänä kahden päivän leivän. Olkoon jokainen alallaan, älköönkä kukaan lähtekö paikaltansa seitsemäntenä päivänä."
\par 30 Ja kansa lepäsi seitsemäntenä päivänä.
\par 31 Ja Israelin heimo antoi sille nimen manna. Ja se oli valkean korianderinsiemenen kaltaista ja maistui hunajakakulta.
\par 32 Ja Mooses sanoi: "Näin on Herra käskenyt: Ota sitä goomerin täysi säilytettäväksi sukupolvesta sukupolveen, että he näkisivät sen leivän, jolla minä olen ravinnut teitä erämaassa, johdattaessani teitä Egyptin maasta".
\par 33 Ja Mooses sanoi Aaronille: "Ota astia ja pane siihen goomerin täysi mannaa ja aseta se Herran eteen säilytettäväksi sukupolvesta sukupolveen".
\par 34 Ja Aaron asetti sen talteen lain arkin eteen, niinkuin Herra oli Moosekselle käskyn antanut.
\par 35 Ja israelilaiset söivät mannaa neljäkymmentä vuotta, kunnes he tulivat asuttuun maahan; he söivät mannaa siihen asti, kunnes tulivat Kanaanin maan rajalle. -
\par 36 Goomer on kymmenesosa eefaa.

\chapter{17}

\par 1 Sitten kaikki israelilaisten seurakunta lähti liikkeelle Siinin erämaasta ja matkusti levähdyspaikasta toiseen Herran käskyn mukaan. Ja he leiriytyivät Refidimiin; siellä ei ollut vettä kansan juoda.
\par 2 Niin kansa riiteli Moosesta vastaan ja sanoi: "Antakaa meille vettä juoda!" Mooses vastasi heille: "Miksi riitelette minua vastaan? Miksi kiusaatte Herraa?"
\par 3 Mutta kansalla oli siellä jano, ja he napisivat yhä Moosesta vastaan ja sanoivat: "Minkätähden olet tuonut meidät Egyptistä, antaaksesi meidän ja meidän lastemme ja karjamme kuolla janoon?"
\par 4 Niin Mooses huusi Herraa ja sanoi: "Mitä minä teen tälle kansalle? Ei paljon puutu, että he kivittävät minut."
\par 5 Herra vastasi Moosekselle: "Mene kansan edellä ja ota mukaasi muutamia Israelin vanhimpia. Ja ota käteesi sauva, jolla löit Niilivirtaa, ja mene.
\par 6 Katso, minä seison siellä sinun edessäsi kalliolla Hoorebin luona; lyö kallioon, ja siitä on vuotava vettä, niin että kansa saa juoda." Ja Mooses teki niin Israelin vanhimpain nähden.
\par 7 Ja hän antoi sille paikalle nimen Massa ja Meriba sentähden, että israelilaiset siellä riitelivät ja kiusasivat Herraa, sanoen: "Onko Herra meidän keskellämme vai ei?"
\par 8 Sitten tulivat amalekilaiset ja taistelivat Israelia vastaan Refidimissä.
\par 9 Niin Mooses sanoi Joosualle: "Valitse meille miehiä, mene ja taistele huomenna amalekilaisia vastaan. Minä asetun vuoren huipulle, Jumalan sauva kädessäni."
\par 10 Ja Joosua teki, niinkuin Mooses oli hänelle sanonut, ja taisteli amalekilaisia vastaan. Mutta Mooses, Aaron ja Huur nousivat vuoren huipulle.
\par 11 Ja niin kauan kuin Mooses piti kätensä ylhäällä, oli Israel voitolla; mutta kun hän antoi kätensä vaipua, olivat amalekilaiset voitolla.
\par 12 Mutta kun Mooseksen kädet väsyivät, ottivat he kiven ja asettivat sen hänen allensa, ja hän istui sille, ja Aaron ja Huur kannattivat hänen käsiänsä kumpikin puoleltansa. Näin hänen kätensä kestivät vahvoina auringon laskuun asti.
\par 13 Ja Joosua voitti amalekilaiset ja heidän sotaväkensä miekan terällä.
\par 14 Ja Herra sanoi Moosekselle: "Kirjoita tämä kirjaan muistoksi ja teroita se Joosuan mieleen: Minä pyyhin pois amalekilaisten muiston taivaan alta".
\par 15 Ja Mooses rakensi alttarin ja pani sille nimeksi: "Herra on minun lippuni".
\par 16 Ja hän sanoi: "Minä nostan käteni Herran istuinta kohden: Herra sotii amalekilaisia vastaan sukupolvesta sukupolveen".

\chapter{18}

\par 1 Ja Midianin pappi Jetro, Mooseksen appi, sai kuulla kaikki, mitä Jumala oli tehnyt Moosekselle ja kansalleen Israelille, kuinka Herra oli vienyt Israelin pois Egyptistä.
\par 2 Ja Jetro, Mooseksen appi, otti mukaansa Sipporan, Mooseksen vaimon, jonka tämä oli lähettänyt kotiin,
\par 3 ja hänen kaksi poikaansa; näistä oli toisen nimi Geersom, koska Mooses oli sanonut: "Minä olen muukalainen vieraalla maalla",
\par 4 ja toisen nimi Elieser, koska hän oli sanonut: "Isäni Jumala oli minun apuni ja pelasti minut faraon miekasta".
\par 5 Ja Jetro, Mooseksen appi, tuli hänen poikiensa ja hänen vaimonsa kanssa Mooseksen luo erämaahan, jossa tämä oli leiriytynyt Jumalan vuoren juurelle.
\par 6 Ja hän käski sanoa Moosekselle: "Minä, Jetro, sinun appesi, ynnä vaimosi kahden poikansa kanssa olemme tulleet luoksesi".
\par 7 Niin Mooses meni appeansa vastaan, kumarsi ja suuteli häntä. Ja kun he olivat tervehtineet toisiansa, menivät he telttaan.
\par 8 Ja Mooses kertoi apellensa kaikki, mitä Herra oli tehnyt faraolle ja egyptiläisille Israelin tähden, ja kaikki ne vaikeudet, jotka heitä olivat kohdanneet matkalla, ja kuinka Herra oli pelastanut heidät.
\par 9 Ja Jetro iloitsi kaikesta siitä hyvästä, mitä Herra oli tehnyt Israelille, pelastaessaan heidät egyptiläisten kädestä.
\par 10 Ja Jetro sanoi: "Kiitetty olkoon Herra, joka on pelastanut teidät egyptiläisten ja faraon kädestä, Herra, joka on pelastanut kansansa egyptiläisten vallasta.
\par 11 Nyt minä tiedän, että Herra on suurempi kaikkia jumalia; sillä sentähden, että egyptiläiset ylpeilivät, on heidän näin käynyt."
\par 12 Ja Jetro, Mooseksen appi, toimitti polttouhrin ja teurasuhreja Jumalalle; ja Aaron ja kaikki Israelin vanhimmat tulivat aterioimaan Mooseksen apen kanssa Jumalan eteen.
\par 13 Seuraavana päivänä Mooses istui tuomitsemaan kansaa, ja kansa seisoi Mooseksen ympärillä aamusta iltaan asti.
\par 14 Kun Mooseksen appi näki kaiken, mitä hän teki kansalle, sanoi hän: "Mitä tämä puuha on, jota sinulla on kansan kanssa? Miksi sinä istut yksin ja kaikki kansa seisoo ympärilläsi aamusta iltaan asti?"
\par 15 Mooses vastasi apellensa: "Kansa tulee minun luokseni kysymään Jumalalta neuvoa.
\par 16 Kun heillä on jokin riita-asia, tulevat he minun luokseni, ja minä ratkaisen heidän riitansa ja ilmoitan heille Jumalan säädökset ja lait."
\par 17 Niin Mooseksen appi sanoi hänelle: "Siinä sinä et menettele viisaasti.
\par 18 Sinä uuvutat sekä itsesi että tämän kansan, joka on kanssasi; sillä tämä tehtävä on sinulle liian raskas, etkä sinä voi sitä yksinäsi toimittaa.
\par 19 Kuule nyt, mitä minä sanon. Minä neuvon sinua, ja Jumala on oleva sinun kanssasi. Ole sinä kansan edusmies Jumalan edessä, ja saata sinä sen asiat Jumalan eteen. Ja opeta heille säädökset ja lait,
\par 20 ja neuvo heille tie, jota heidän on kuljettava, ja mitä heidän on tehtävä.
\par 21 Mutta valitse koko kansasta kelvollisia ja Jumalaa pelkääväisiä, luotettavia ja väärää voittoa vihaavia miehiä, ja aseta ne heille tuhannen, sadan, viidenkymmenen ja kymmenen päämiehiksi.
\par 22 Nämä tuomitkoot kansaa joka aika. Kaikki suuret asiat he saattakoot sinun tietoosi, mutta kaikki vähäiset asiat ratkaiskoot itse. Huojenna näin jotakin itseltäsi, ja kantakoot he kuormaa sinun kanssasi.
\par 23 Jos näin teet ja Jumala itse sinua näin käskee, niin sinä jaksat sen kestää; ja kaikki tämä kansa saa mennä rauhassa kotiinsa."
\par 24 Ja Mooses noudatti appensa puhetta ja teki kaiken, mitä tämä oli sanonut.
\par 25 Mooses valitsi kelvollisia miehiä koko Israelista ja asetti heidät kansan johtoon, tuhannen, sadan, viidenkymmenen ja kymmenen päämiehiksi.
\par 26 Nämä tuomitsivat kansaa joka aika. Vaikeat asiat he lykkäsivät Moosekselle, mutta vähäiset asiat he ratkaisivat itse.
\par 27 Sitten Mooses saattoi appensa matkaan, ja tämä meni omaan maahansa.

\chapter{19}

\par 1 Kolmantena kuukautena siitä, kun israelilaiset olivat lähteneet Egyptin maasta, juuri samana kuukauden päivänä, he tulivat Siinain erämaahan.
\par 2 Sillä he olivat lähteneet Refidimistä, tulleet Siinain erämaahan ja leiriytyneet erämaahan; ja Israel oli siellä leirissä vuoren kohdalla.
\par 3 Ja Mooses nousi Jumalan tykö, ja Herra huusi häntä vuorelta ja sanoi: "Sano näin Jaakobin heimolle ja ilmoita israelilaisille:
\par 4 'Te olette nähneet, mitä minä olen tehnyt egyptiläisille ja kuinka minä olen kantanut teitä kotkan siivillä ja tuonut teidät luokseni.
\par 5 Jos te nyt kuulette minun ääntäni ja pidätte minun liittoni, niin te olette minun omaisuuteni ennen kaikkia muita kansoja; sillä koko maa on minun.
\par 6 Ja te olette minulle pappisvaltakunta ja pyhä kansa.' Sano nämä sanat israelilaisille."
\par 7 Kun Mooses tuli takaisin, kutsui hän kokoon kansan vanhimmat ja puhui heille kaikki ne sanat, jotka Herra oli hänen käskenyt puhua.
\par 8 Niin koko kansa vastasi yhtenä miehenä ja sanoi: "Kaiken, mitä Herra on puhunut, me teemme". Ja Mooses vei kansan vastauksen Herralle.
\par 9 Niin Herra sanoi Moosekselle: "Katso, minä tulen sinun tykösi paksussa pilvessä, että kansa kuulisi, kun minä puhun sinun kanssasi, ja uskoisi myös sinua ainiaan". Ja Mooses ilmoitti Herralle kansan vastauksen.
\par 10 Niin Herra sanoi Moosekselle: "Mene kansan luo ja pyhitä heidät tänä päivänä ja huomenna, ja peskööt he vaatteensa.
\par 11 Ja olkoot valmiit kolmanneksi päiväksi; sillä kolmantena päivänä Herra astuu koko kansan nähden alas Siinain vuorelle.
\par 12 Ja merkitse raja kansalle yltympäri ja sano: 'Varokaa nousemasta vuorelle tahi koskettamasta sen juureen. Jokainen, joka vuoreen koskee, rangaistakoon kuolemalla.
\par 13 Älköön kenenkään käsi häneen koskeko, vaan hänet kivitettäköön tahi ammuttakoon kuoliaaksi. Olipa eläin tai ihminen, ei se saa jäädä eloon.' Vasta kun pitkä torven puhallus kuuluu, he nouskoot vuorelle."
\par 14 Ja Mooses astui alas vuorelta kansan luo; ja hän pyhitti kansan, ja he pesivät vaatteensa.
\par 15 Ja hän sanoi kansalle: "Olkaa kolmanneksi päiväksi valmiit; älköön kukaan ryhtykö naiseen".
\par 16 Ja kolmantena päivänä, kun aamu oli tullut, alkoi jylistä ja salamoida, ja paksu pilvi laskeutui vuoren ylle, ja kuului ylen kova pasunan ääni, niin että koko kansa, joka oli leirissä, vapisi pelosta.
\par 17 Silloin Mooses vei kansan leiristä Jumalaa vastaan, ja he asettuivat vuoren juurelle.
\par 18 Ja koko Siinain vuori peittyi savuun, kun Herra astui sille alas tulessa, ja siitä nousi savu niinkuin pätsin savu, ja koko vuori vapisi kovasti.
\par 19 Ja pasunan ääni koveni kovenemistaan. Mooses puhui, ja Jumala vastasi hänelle jylinällä.
\par 20 Ja Herra astui alas Siinain vuorelle, vuoren kukkulalle, ja Herra kutsui Mooseksen vuoren kukkulalle; ja Mooses nousi sinne ylös.
\par 21 Niin Herra sanoi Moosekselle: "Astu alas ja varoita kansaa tunkeutumasta lähelle Herraa, häntä nähdäkseen, sillä silloin heistä monta kaatuu.
\par 22 Papitkin, jotka saavat lähestyä Herraa, pyhittäkööt itsensä, ettei Herra heitä tuhoaisi."
\par 23 Mutta Mooses vastasi Herralle: "Kansa ei voi nousta Siinain vuorelle, sillä sinä olet varoittanut meitä ja sanonut: 'Merkitse raja vuoren ympäri ja pyhitä se'."
\par 24 Niin Herra sanoi hänelle: "Astu alas ja tule taas ylös, sinä ja Aaron sinun kanssasi. Mutta papit ja kansa älkööt tunkeutuko ylös, Herraa lähelle, ettei hän heitä tuhoaisi."
\par 25 Ja Mooses astui alas kansan luo ja sanoi heille tämän.

\chapter{20}

\par 1 Ja Jumala puhui kaikki nämä sanat ja sanoi:
\par 2 "Minä olen Herra, sinun Jumalasi, joka vein sinut pois Egyptin maasta, orjuuden pesästä.
\par 3 Älä pidä muita jumalia minun rinnallani.
\par 4 Älä tee itsellesi jumalankuvaa äläkä mitään kuvaa, älä niistä, jotka ovat ylhäällä taivaassa, älä niistä, jotka ovat alhaalla maan päällä, äläkä niistä, jotka ovat vesissä maan alla.
\par 5 Älä kumarra niitä äläkä palvele niitä. Sillä minä, Herra, sinun Jumalasi, olen kiivas Jumala, joka kostan isien pahat teot lapsille kolmanteen ja neljänteen polveen, niille, jotka minua vihaavat;
\par 6 mutta teen laupeuden tuhansille, jotka minua rakastavat ja pitävät minun käskyni.
\par 7 Älä turhaan lausu Herran, sinun Jumalasi, nimeä, sillä Herra ei jätä rankaisematta sitä, joka hänen nimensä turhaan lausuu.
\par 8 Muista pyhittää lepopäivä.
\par 9 Kuusi päivää tee työtä ja toimita kaikki askareesi;
\par 10 mutta seitsemäs päivä on Herran, sinun Jumalasi, sapatti; silloin älä mitään askaretta toimita, älä sinä älköönkä sinun poikasi tai tyttäresi, sinun palvelijasi tai palvelijattaresi tai juhtasi älköönkä muukalaisesi, joka sinun porteissasi on.
\par 11 Sillä kuutena päivänä Herra teki taivaan ja maan ja meren ja kaikki, mitä niissä on, mutta seitsemäntenä päivänä hän lepäsi; sentähden Herra siunasi lepopäivän ja pyhitti sen.
\par 12 Kunnioita isääsi ja äitiäsi, että kauan eläisit siinä maassa, jonka Herra, sinun Jumalasi, sinulle antaa.
\par 13 Älä tapa.
\par 14 Älä tee huorin.
\par 15 Älä varasta.
\par 16 Älä sano väärää todistusta lähimmäisestäsi.
\par 17 Älä himoitse lähimmäisesi huonetta. Älä himoitse lähimmäisesi vaimoa äläkä hänen palvelijaansa, palvelijatartaan, härkäänsä, aasiansa äläkä mitään, mikä on lähimmäisesi omaa."
\par 18 Ja kaikki kansa havaitsi jylinän, tulen leimaukset, pasunan äänen ja vuoren suitsuamisen; ja kun he sen havaitsivat, vapisivat he ja pysyivät taampana.
\par 19 Ja he sanoivat Moosekselle: "Puhu sinä meidän kanssamme, niin me kuulemme. Älköön Jumala puhuko meidän kanssamme, ettemme kuolisi."
\par 20 Mutta Mooses vastasi kansalle: "Älkää peljätkö, sillä Jumala on tullut koettelemaan teitä, että Herran pelko olisi teidän silmäinne edessä ja ettette syntiä tekisi".
\par 21 Ja kansa pysyi taampana, mutta Mooses lähestyi pimeyttä, jossa Jumala oli.
\par 22 Ja Herra sanoi Moosekselle: "Sano israelilaisille näin: Te olette nähneet, että minä olen puhunut teille taivaasta.
\par 23 Älkää tehkö jumalia minun rinnalleni; hopeaisia tai kultaisia jumalia älkää tehkö itsellenne.
\par 24 Tee minulle alttari maasta, ja uhraa sen päällä polttouhrisi ja yhteysuhrisi, lampaasi ja raavaasi. Joka paikassa, mihin minä säädän nimeni muiston, minä tulen sinun tykösi ja siunaan sinua.
\par 25 Mutta jos sinä teet minulle kivialttarin, niin älä rakenna sitä hakatusta kivestä; sillä jos sinä kosket taltallasi kiveen, niin sinä saastutat sen.
\par 26 Äläkä nouse portaita myöten minun alttarilleni, ettei häpysi sen päällä paljastuisi."

\chapter{21}

\par 1 "Nämä ovat ne oikeudet, jotka sinun tulee asettaa heidän eteensä:
\par 2 Jos sinä ostat hebrealaisen orjan, niin hän palvelkoon kuusi vuotta, mutta seitsemäntenä hän pääsköön vapaaksi maksutta.
\par 3 Jos hän on tullut yksinäisenä, niin yksinäisenä lähteköönkin; mutta jos hän oli nainut, niin lähteköön vaimo hänen kanssaan.
\par 4 Jos hänen isäntänsä on antanut hänelle vaimon, ja tämä on synnyttänyt hänelle poikia tai tyttäriä, niin vaimo lapsinensa jääköön isännän omaksi, ja hän lähteköön yksinään.
\par 5 Mutta jos orja vakuuttaa: 'Minä rakastan isäntääni, vaimoani ja lapsiani enkä tahdo päästä vapaaksi',
\par 6 niin hänen isäntänsä vieköön hänet Jumalan eteen ja asettakoon hänet ovea tai pihtipieltä vasten, ja hänen isäntänsä lävistäköön hänen korvansa naskalilla, ja hän olkoon hänen orjansa ainiaan.
\par 7 Jos joku myy tyttärensä orjaksi, älköön tämä pääskö vapaaksi, niinkuin miesorjat pääsevät.
\par 8 Jos hän ei miellytä isäntäänsä, sitten kuin tämä jo on määrännyt hänet itsellensä, niin tämä sallikoon lunastaa hänet pois. Vieraaseen kansaan älköön hänellä olko valtaa häntä myydä, kun hän hänet hylkää.
\par 9 Mutta jos hän määrää hänet pojallensa, niin antakoon hänen nauttia tyttärien oikeutta.
\par 10 Jos hän ottaa itselleen toisen vaimon, niin älköön vähentäkö ensimmäiseltä tämän ravintoa, vaatetusta ja aviollista oikeutta.
\par 11 Jos hän ei tee hänelle näitä kolmea, niin lähteköön vaimo pois maksutta ja rahakorvauksetta.
\par 12 Joka lyö ihmistä, niin että tämä kuolee, se rangaistakoon kuolemalla.
\par 13 Mutta jos hän ei ole tehnyt sitä murha-aikeessa, vaan Jumala on sallinut sen vahingon tapahtua hänen kätensä kautta, niin minä määrään sinulle paikan, johon hän voi paeta.
\par 14 Mutta jos joku menettelee niin rikollisesti lähimmäistänsä kohtaan, että tappaa hänet kavalasti, on sinun otettava hänet minun alttarinikin luota surmattavaksi.
\par 15 Joka lyö isäänsä tai äitiänsä, se rangaistakoon kuolemalla.
\par 16 Joka varastaa ihmisen ja joko myy hänet tahi pitää häntä hallussansa, se rangaistakoon kuolemalla.
\par 17 Joka kiroaa isäänsä tai äitiänsä, se rangaistakoon kuolemalla.
\par 18 Jos miehet riitelevät keskenänsä ja toinen lyö toista kivellä tai nyrkillä, mutta tämä ei kuole, vaan joutuu vuoteen omaksi,
\par 19 niin olkoon, jos hän tointuu ja voi mennä ulos sauvaansa nojaten, lyöjä vapaa rangaistuksesta; korvatkoon ainoastaan hänen sairastamisaikansa ja pitäköön huolta hänen paranemisestaan.
\par 20 Jos joku lyö orjaansa tai orjatartaan sauvalla, niin että tämä kuolee hänen käsiinsä, niin häntä rangaistakoon.
\par 21 Mutta jos se elää päivän tai kaksi, niin älköön lyöjää rangaistako, sillä se on hänen omaa rahaansa.
\par 22 Jos miehet tappelevat keskenänsä ja loukkaavat raskasta vaimoa, niin että hän synnyttää kesken, mutta vahinkoa ei tapahdu, niin sakotettakoon syyllistä vaimon miehen vaatimuksen ja riidanratkaisijain harkinnan mukaan.
\par 23 Mutta jos vahinko tapahtuu, niin annettakoon henki hengestä,
\par 24 silmä silmästä, hammas hampaasta, käsi kädestä, jalka jalasta,
\par 25 palovamma palovammasta, haava haavasta, mustelma mustelmasta.
\par 26 Jos joku lyö orjaansa tai orjatartansa silmään ja turmelee sen, niin päästäköön hänet vapaaksi silmän tähden.
\par 27 Ja jos hän lyö orjaltaan tai orjattareltaan hampaan suusta, niin päästäköön hänet vapaaksi hampaan tähden.
\par 28 Jos härkä puskee miehen tai naisen kuoliaaksi, niin härkä kivitettäköön, älköönkä sen lihaa syötäkö; mutta härän omistaja olkoon vapaa rangaistuksesta.
\par 29 Mutta jos härkä on ennenkin puskenut ja sen isäntää on varoitettu eikä hän ole sitä vartioinut, ja jos se tappaa miehen tai naisen, niin härkä kivitettäköön, ja myös sen isäntä rangaistakoon kuolemalla.
\par 30 Mutta jos hänelle määrätään lunastusmaksu, niin maksakoon henkensä lunnaiksi niin paljon, kuin hänelle määrätään.
\par 31 Jos se puskee pojan tai tytön, niin meneteltäköön saman lain mukaan.
\par 32 Jos härkä puskee orjan tai orjattaren, niin maksakoon sen omistaja pusketun isännälle kolmekymmentä hopeasekeliä, ja härkä kivitettäköön.
\par 33 Jos joku jättää kaivon auki tahi kaivaa kaivon eikä peitä sitä, ja härkä tai aasi putoaa siihen,
\par 34 niin kaivon omistaja korvatkoon isännälle sen rahassa; mutta kuollut eläin olkoon hänen.
\par 35 Jos jonkun härkä puskee toisen härän kuoliaaksi, niin myykööt elävän härän ja jakakoot sen hinnan, ja myös kuolleen jakakoot keskenään.
\par 36 Jos taas oli tunnettua, että se härkä ennenkin oli puskenut eikä sen isäntä ollut sitä vartioinut, niin antakoon härän härästä, mutta kuollut olkoon hänen."

\chapter{22}

\par 1 "Jos joku varastaa härän tai lampaan ja teurastaa tahi myy sen, antakoon viisi raavasta yhdestä härästä ja neljä lammasta yhdestä lampaasta.
\par 2 Jos varas tavataan murtautumasta sisälle ja lyödään kuoliaaksi, ei tappaja ole vereen vikapää.
\par 3 Mutta jos aurinko jo oli noussut, niin tappaja on vereen vikapää. Varas maksakoon korvauksen; mutta jos hänellä ei ole mitään, niin myytäköön hänet varastamansa tavaran korvaukseksi.
\par 4 Jos varastettu eläin, olipa se härkä, aasi tai lammas, tavataan hänen hallustaan elävänä, niin korvatkoon sen kaksinkertaisesti.
\par 5 Jos joku turmelee toiselta pellon tai viinitarhan päästämällä siihen karjansa ja syöttämällä sitä toisen pellossa, antakoon korvaukseksi peltonsa tai viinitarhansa parhaimman kasvun.
\par 6 Jos tuli pääsee irti ja tarttuu orjantappuroihin ja jos kuhilaat tai vilja tai pelto palaa, niin korvatkoon vahingon se, joka on kulovalkean sytyttänyt.
\par 7 Jos joku antaa toiselle rahaa tai tavaraa säilytettäväksi ja se varastetaan tämän talosta, niin varas, jos hänet tavataan, korvatkoon sen kaksinkertaisesti.
\par 8 Mutta jos varasta ei tavata, astukoon talon omistaja Jumalan eteen ja vannokoon, ettei hän ole kädellänsä kajonnut toisen omaan.
\par 9 Jokaisessa anastusasiassa, koskipa se härkää tai aasia tai lammasta tai vaatetta tai mitä tahansa kadonnutta, josta joku sanoo: 'Tämä se on', tulkoon kummankin asia Jumalan eteen; ja se, jonka Jumala tuomitsee syylliseksi, korvatkoon toiselle kaksinkertaisesti.
\par 10 Jos joku antaa toiselle aasin tai härän tai lampaan tai minkä eläimen tahansa säilytettäväksi ja se kuolee tai vahingoittuu tai ryöstetään pois kenenkään näkemättä,
\par 11 niin vala Herran edessä ratkaiskoon heidän välillään, onko toinen kädellänsä kajonnut toisen omaan; omistaja hyväksyköön valan, ja toinen olkoon korvauksesta vapaa.
\par 12 Mutta jos se on häneltä varastettu, korvatkoon sen omistajalle.
\par 13 Jos se on raadeltu, tuokoon sen esiin todistukseksi, eikä hänen tarvitse raadeltua korvata.
\par 14 Jos joku lainaa toiselta elukan ja se vahingoittuu tai kuolee eikä sen omistaja ole saapuvilla, korvatkoon sen.
\par 15 Jos sen omistaja on saapuvilla, ei tarvitse korvausta maksaa; jos se oli vuokralla, olkoon vuokra korvauksena.
\par 16 Jos joku viettelee neitsyen, joka ei ole kihlattu, ja makaa hänet, maksakoon hänestä morsiamenhinnan ja ottakoon hänet vaimokseen.
\par 17 Jos isä kieltäytyy antamasta häntä hänelle, maksakoon mies rahassa morsiamenhinnan niinkuin neitsyestä.
\par 18 Velhonaisen älä salli elää.
\par 19 Jokainen, joka sekaantuu eläimeen, rangaistakoon kuolemalla.
\par 20 Joka uhraa muille jumalille kuin Herralle, ainoalle, olkoon vihitty tuhon omaksi.
\par 21 Älä sorra äläkä ahdista muukalaista, sillä te olette itse olleet muukalaisina Egyptin maassa.
\par 22 Älkää sortako leskeä tai orpoa.
\par 23 Sillä jos sinä sorrat heitä ja he huutavat minua avuksensa, niin minä totisesti kuulen heidän huutonsa,
\par 24 ja minun vihani syttyy, ja minä surmaan teidät miekalla, niin että teidän vaimonne joutuvat leskiksi ja lapsenne orvoiksi.
\par 25 Jos lainaat rahaa jollekin minun kansastani, jollekin köyhälle, joka on sinun luonasi, niin älä menettele koronkiskurin tavoin häntä kohtaan. Älkää panko korkoa hänen maksettavakseen.
\par 26 Jos sinä olet lähimmäiseltäsi ottanut pantiksi vaipan, anna se hänelle takaisin, ennenkuin aurinko laskee;
\par 27 sillä se on hänen ainoa peitteensä, johon hän käärii ruumiinsa. Missä hän muutoin makaisi? Ja jos hän huutaa minua avukseen, kuulen minä häntä, sillä minä olen laupias.
\par 28 Jumalaa älä herjaa, ja kansasi ruhtinasta älä kiroa.
\par 29 Älä viivyttele antamasta antia vilja- ja mehusatosi runsaudesta. Esikoinen pojistasi anna minulle.
\par 30 Samoin tee raavaittesi ja lampaittesi ensiksisyntyneelle. Seitsemän päivää se olkoon emänsä kanssa; kahdeksantena päivänä anna se minulle.
\par 31 Ja te olkaa minulle pyhä kansa. Älkää syökö kedolla raadellun eläimen lihaa, vaan heittäkää se koirille."

\chapter{23}

\par 1 "Älä levitä valheellista huhua, äläkä anna apuasi syylliselle rupeamalla vääräksi todistajaksi.
\par 2 Älä ole joukon mukana tekemässä pahaa, äläkä todista riita-asiassa niin, että taivut joukon mukaan ja väännät oikean vääräksi.
\par 3 Älä ole puolueellinen alhaisen hyväksi hänen asiassansa.
\par 4 Jos tapaat vihollisesi härän tai aasin eksyksissä, niin saata se hänelle takaisin.
\par 5 Jos näet vihamiehesi aasin makaavan kuormansa alla, niin älä jätä häntä auttamatta, vaan auta häntä sitä päästämään.
\par 6 Älä väännä vääräksi keskuudessanne asuvan köyhän oikeutta hänen riita-asiassansa.
\par 7 Pysy erilläsi väärästä asiasta, äläkä surmaa viatonta ja syytöntä, sillä minä en julista syyllistä syyttömäksi.
\par 8 Äläkä ota lahjusta, sillä lahjus sokaisee näkevät ja vääristää syyttömien asiat.
\par 9 Muukalaista älä sorra, sillä te tiedätte muukalaisen mielialan, koska itsekin olette olleet muukalaisina Egyptin maassa.
\par 10 Kuutena vuotena kylvä maasi ja korjaa sen sato.
\par 11 Mutta seitsemäntenä vuotena jätä se korjaamatta ja lepäämään, että kansasi köyhät saisivat siitä syödä; ja mitä jäljelle jää, sen metsän eläimet syökööt. Samoin tee viinitarhallesi ja öljytarhallesi.
\par 12 Kuusi päivää tee työtäsi, mutta lepää seitsemäs päivä, että härkäsi ja aasisi saisivat hengähtää ja orjattaresi poika ynnä muukalainen saisivat virkistyä.
\par 13 Kaikkea, mitä minä olen sanonut teille, noudattakaa. Vierasten jumalien nimiä älkää mainitko, älköön niitä kuuluko teidän huuliltanne.
\par 14 Kolme kertaa vuodessa vietä juhlaa minun kunniakseni.
\par 15 Pidä happamattoman leivän juhla: seitsemänä päivänä syö happamatonta leipää, niinkuin minä olen sinua käskenyt, määrättynä aikana aabib-kuussa, sillä siinä kuussa sinä olet lähtenyt Egyptistä; mutta tyhjin käsin älköön tultako minun kasvojeni eteen.
\par 16 Ja vietä leikkuujuhla, kun leikkaat uutiset viljastasi, jonka olet kylvänyt vainioon, ja korjuujuhla vuoden lopussa, kun korjaat satosi vainiolta.
\par 17 Kolme kertaa vuodessa tulkoon kaikki sinun miesväkesi Herran, Herran, kasvojen eteen.
\par 18 Älä uhraa minun teurasuhrini verta happamen leivän ohella. Ja minun juhlauhrini rasvaa älköön jääkö yön yli seuraavaan aamuun.
\par 19 Parhaat maasi uutisesta tuo Herran, Jumalasi, huoneeseen. Älä keitä vohlaa emänsä maidossa.
\par 20 Katso, minä lähetän enkelin sinun edellesi varjelemaan sinua tiellä ja johdattamaan sinua siihen paikkaan, jonka minä olen valmistanut.
\par 21 Ole varuillasi hänen edessään ja kuule häntä äläkä pahoita hänen mieltänsä. Hän ei jätä teidän rikoksianne rankaisematta, sillä minun nimeni on hänessä.
\par 22 Mutta jos sinä kuulet häntä ja teet kaikki, mitä minä käsken, niin minä olen sinun vihollistesi vihollinen ja vastustajaisi vastustaja.
\par 23 Sillä minun enkelini käy sinun edelläsi ja johdattaa sinut amorilaisten, heettiläisten, perissiläisten, kanaanilaisten, hivviläisten ja jebusilaisten maahan, ja minä hävitän heidät.
\par 24 Älä kumarra heidän jumaliansa, älä palvele niitä äläkä tee, niinkuin he tekevät, vaan kukista ne maahan ja murskaa niiden patsaat.
\par 25 Palvelkaa Herraa, Jumalaanne, niin hän siunaa sinun ruokasi ja juomasi, ja minä pidän puutteen sinusta kaukana.
\par 26 Ei keskensynnyttäjää eikä hedelmätöntä ole sinun maassasi oleva. Ja sinun päiviesi luvun minä teen täydeksi.
\par 27 Minä lähetän kauhuni sinun edelläsi ja saatan hämminkiin kaikki kansat, joiden luo sinä tulet, ja ajan kaikki vihollisesi pakoon sinun edestäsi.
\par 28 Ja minä lähetän herhiläisiä sinun edelläsi karkoittamaan hivviläiset, kanaanilaiset ja heettiläiset sinun tieltäsi.
\par 29 Mutta minä en karkoita heitä sinun tieltäsi yhtenä vuotena, ettei maa tulisi autioksi eivätkä metsän pedot lisääntyisi sinun vahingoksesi;
\par 30 vähitellen minä karkoitan heidät sinun tieltäsi, kunnes olet tullut kyllin lukuisaksi ottamaan haltuusi maan.
\par 31 Ja minä asetan sinun rajasi Kaislamerestä filistealaisten mereen ja erämaasta Eufrat-virtaan asti; sillä minä annan maan asukkaat teidän valtaanne, ja sinä karkoitat heidät tieltäsi.
\par 32 Älä tee liittoa heidän äläkä heidän jumaliensa kanssa.
\par 33 Älkööt he jääkö asumaan sinun maahasi, etteivät saattaisi sinua tekemään syntiä minua vastaan; sillä jos sinä palvelet heidän jumaliansa, on se sinulle paulaksi."

\chapter{24}

\par 1 Ja hän sanoi Moosekselle: "Nouse Herran tykö, sinä ja Aaron, Naadab ja Abihu ynnä seitsemänkymmentä Israelin vanhinta; ja kumartukaa ja rukoilkaa taampana.
\par 2 Mooses yksinään lähestyköön Herraa; muut älkööt lähestykö, ja kansa älköön nousko sinne hänen kanssansa."
\par 3 Ja Mooses tuli ja kertoi kansalle kaikki Herran sanat ja kaikki hänen säädöksensä. Niin koko kansa vastasi yhteen ääneen ja sanoi: "Kaiken, mitä Herra on puhunut, me teemme".
\par 4 Sitten Mooses kirjoitti kaikki Herran sanat. Ja hän nousi varhain seuraavana aamuna ja rakensi alttarin vuoren juurelle sekä pystytti kaksitoista patsasta Israelin kahdentoista sukukunnan mukaan.
\par 5 Ja hän lähetti israelilaisten joukosta nuoria miehiä uhraamaan polttouhreja ja teurastamaan härkiä yhteysuhriksi Herralle.
\par 6 Ja Mooses otti verestä puolet ja pani uhrimaljoihin, ja toisen puolen hän vihmoi alttarille.
\par 7 Ja hän otti liitonkirjan ja luki sen kansan kuullen. Ja he sanoivat: "Kaikkea, mitä Herra on puhunut, me noudatamme ja tottelemme".
\par 8 Niin Mooses otti veren ja vihmoi sitä kansan päälle ja sanoi: "Katso, tämä on sen liiton veri, jonka Herra on tehnyt teidän kanssanne kaikkien näiden sanojen perusteella".
\par 9 Ja Mooses ja Aaron, Naadab ja Abihu ynnä seitsemänkymmentä Israelin vanhinta nousivat vuorelle.
\par 10 Ja he näkivät Israelin Jumalan; ja hänen jalkainsa alla oli alusta, niinkuin safiirikivistä, kirkas kuin itse taivas.
\par 11 Eikä hän kajonnut kädellänsä israelilaisten valittuihin, ja he katselivat Jumalaa, söivät ja joivat.
\par 12 Ja Herra sanoi Moosekselle: "Nouse minun tyköni vuorelle ja ole siellä, niin minä annan sinulle kivitaulut ynnä lain ja käskyt, jotka minä olen kirjoittanut heille opetukseksi".
\par 13 Silloin Mooses ja hänen palvelijansa Joosua lähtivät; ja Mooses nousi Jumalan vuorelle.
\par 14 Mutta vanhimmille hän sanoi: "Odottakaa meitä täällä, kunnes palaamme teidän tykönne. Ja katso, Aaron ja Huur ovat teidän kanssanne; jolla on jotakin asiaa, kääntyköön heidän puoleensa."
\par 15 Kun Mooses oli noussut vuorelle, peitti pilvi vuoren.
\par 16 Ja Herran kirkkaus laskeutui Siinain vuorelle, ja pilvi peitti sen kuusi päivää; ja seitsemäntenä päivänä hän huusi Moosesta pilven keskeltä.
\par 17 Ja Herran kirkkaus vuoren kukkulalla näytti israelilaisten silmissä kuluttavalta tulelta.
\par 18 Ja Mooses meni pilven keskelle ja nousi vuorelle. Ja Mooses oli vuorella neljäkymmentä päivää ja neljäkymmentä yötä.

\chapter{25}

\par 1 Ja Herra puhui Moosekselle sanoen:
\par 2 "Sano israelilaisille, että he kokoavat anteja minulle; jokaiselta, jonka sydän on siihen altis, ottakaa vastaan anti minulle.
\par 3 Ja nämä ovat ne annit, joita teidän on otettava vastaan heiltä: kultaa, hopeata ja vaskea,
\par 4 punasinisiä, purppuranpunaisia ja helakanpunaisia lankoja ja valkoisia pellavalankoja sekä vuohenkarvoja,
\par 5 punaisia oinaannahkoja, sireeninnahkoja, akasiapuuta,
\par 6 öljyä seitsenhaaraista lamppua varten, hajuaineita voiteluöljyä ja hyvänhajuista suitsutusta varten,
\par 7 onyks-kiviä ja muita jalokiviä kasukkaa ja rintakilpeä varten.
\par 8 Ja tehkööt he minulle pyhäkön asuakseni heidän keskellään.
\par 9 Tehkää asumus ja kaikki sen kalusto tarkoin sen kaavan mukaan, jonka minä sinulle näytän.
\par 10 Tehkööt he arkin akasiapuusta, puolenkolmatta kyynärän pituisen, puolentoista kyynärän levyisen ja puolentoista kyynärän korkuisen.
\par 11 Ja päällystä se puhtaalla kullalla, päällystä se sisältä ja ulkoa; ja tee siihen kultareunus yltympäri.
\par 12 Ja vala siihen neljä kultarengasta ja kiinnitä ne sen neljään jalkaan, niin että kaksi rengasta tulee sen toiselle puolelle ja kaksi rengasta sen toiselle puolelle.
\par 13 Ja tee korennot akasiapuusta ja päällystä ne kullalla.
\par 14 Ja pistä renkaisiin arkin sivuille korennot, joilla arkki on kannettava.
\par 15 Korennot jääkööt arkin renkaisiin, älköönkä niitä vedettäkö pois.
\par 16 Ja pane arkkiin laki, jonka minä sinulle annan.
\par 17 Tee myös armoistuin puhtaasta kullasta, puolenkolmatta kyynärän pituinen ja puolentoista kyynärän levyinen.
\par 18 Ja tee kaksi kultakerubia, tee ne kohotakoista tekoa, armoistuimen molempiin päihin.
\par 19 Tee toinen kerubi toiseen päähän ja toinen kerubi toiseen päähän. Tehkää kerubit armoistuimesta kohoaviksi, sen kumpaankin päähän.
\par 20 Ja kerubit levittäkööt siipensä ylöspäin, niin että ne peittävät siivillänsä armoistuimen, ja niiden kasvot olkoot vastakkain; armoistuinta kohti olkoot kerubien kasvot käännetyt.
\par 21 Ja aseta armoistuin arkin päälle ja pane arkkiin laki, jonka minä sinulle annan.
\par 22 Ja siinä minä ilmestyn sinulle ja puhun sinulle armoistuimelta, niiden kahden kerubin välistä, jotka ovat lain arkin päällä, kaiken sen, minkä minä sinun kauttasi israelilaisille säädän.
\par 23 Ja tee pöytä akasiapuusta, kahden kyynärän pituinen, kyynärän levyinen ja puolentoista kyynärän korkuinen.
\par 24 Ja päällystä se puhtaalla kullalla ja tee siihen kultareunus yltympäri.
\par 25 Ja tee sen ympäri kämmenen korkuinen lista ja listan ympäri kultareunus.
\par 26 Ja tee siihen neljä kultarengasta ja kiinnitä renkaat sen neljään kulmaan, kunkin neljän jalan kohdalle.
\par 27 Renkaat olkoot juuri listan alla niiden korentojen pitiminä, joilla pöytä on kannettava.
\par 28 Ja tee ne korennot akasiapuusta ja päällystä ne kullalla, ja niillä on pöytä kannettava.
\par 29 Tee myös siihen vadit ja kupit, kannut ja maljat, joista juomauhrit vuodatetaan; tee ne puhtaasta kullasta.
\par 30 Ja pidä aina minun edessäni pöydällä näkyleivät.
\par 31 Ja tee myös seitsenhaarainen lamppu puhtaasta kullasta. Tee lamppu jalkoineen ja varsineen kohotakoista tekoa; sen kukkakuvut, nuput kukkalehtineen, olkoot samaa kappaletta kuin se.
\par 32 Kuusi haaraa lähteköön lampun sivuista, kolme haaraa toisesta sivusta ja kolme haaraa toisesta sivusta.
\par 33 Toisessa haarassa olkoon kolme mantelinkukan muotoista kukkakupua, nuppua kukkalehtineen, ja toisessa haarassa samoin kolme mantelinkukan muotoista kukkakupua, nuppua kukkalehtineen; näin olkoon jokaisessa kuudessa haarassa, jotka lampusta lähtevät.
\par 34 Mutta lampussa itsessään olkoon neljä mantelinkukan muotoista kukkakupua, nuppua kukkalehtineen.
\par 35 Yksi nuppu olkoon aina jokaisen haaraparin alla niistä kuudesta haarasta, jotka lampusta lähtevät.
\par 36 Nuput ja haarat olkoot samaa kappaletta kuin se; olkoon se kauttaaltaan samaa kohotakoista tekoa, puhdasta kultaa.
\par 37 Ja tee siihen seitsemän lamppua, ja lamput asetettakoon niin, että seitsenhaarainen lamppu heittää valonsa etupuolelle.
\par 38 Ja sen lamppusakset ja karstakupit olkoot puhdasta kultaa.
\par 39 Yhdestä talentista puhdasta kultaa tehtäköön sekä se että kaikki nämä kalut.
\par 40 Ja katso, että teet ne sen kaavan mukaan, joka sinulle niistä vuorella näytettiin."

\chapter{26}

\par 1 "Ja tee asumus kymmenestä telttakankaan kaistasta, jotka ovat valmistetut kerratuista valkoisista pellavalangoista ja punasinisistä, purppuranpunaisista ja helakanpunaisista langoista, ja tee niihin taidokkaasti kudottuja kerubeja.
\par 2 Kunkin kaistan pituus olkoon kaksikymmentäkahdeksan kyynärää ja leveys neljä kyynärää; kaikilla kaistoilla olkoon sama mitta.
\par 3 Viisi kaistaa yhdistettäköön toisiinsa, ja samoin toiset viisi kaistaa yhdistettäköön toisiinsa.
\par 4 Ja tee silmukat punasinisestä langasta ensimmäisen kaistan reunaan, yhdistetyn kappaleen laitaan, ja samoin toisen yhdistetyn kappaleen viimeisen kaistan reunaan.
\par 5 Tee viisikymmentä silmukkaa ensimmäiseen kaistaan, ja tee viisikymmentä silmukkaa vastaavan kaistan laitaan, toiseen yhdistettyyn kappaleeseen, niin että silmukat ovat kohdakkain.
\par 6 Ja tee viisikymmentä kultahakasta ja yhdistä kaistat toisiinsa näillä hakasilla, niin että siitä tulee yhtenäinen asumus.
\par 7 Tee vielä kaistoista, jotka ovat kudotut vuohenkarvoista, teltta asumuksen suojaksi; tee niitä kaistoja yksitoista.
\par 8 Kunkin kaistan pituus olkoon kolmekymmentä kyynärää ja leveys neljä kyynärää; niillä yhdellätoista kaistalla olkoon sama mitta.
\par 9 Liitä yhteen viisi kaistaa erikseen ja kuusi kaistaa erikseen, ja aseta kuudes niistä kaksin kerroin teltan etupuolelle.
\par 10 Ja tee viisikymmentä silmukkaa toisen yhdistetyn kappaleen viimeisen kaistan reunaan ja viisikymmentä silmukkaa toisen yhdistetyn kappaleen vastaavan kaistan reunaan.
\par 11 Tee myös viisikymmentä vaskihakasta ja pistä hakaset silmukkoihin ja liitä teltta yhteen, niin että siitä tulee yhtenäinen.
\par 12 Siitä telttakaistojen liiasta osasta, joka jää riippumaan, jääköön puolet riippumaan asumuksen takasivulle.
\par 13 Ja siitä, mikä telttakaistoissa on liikaa pituutta, riippukoon kyynärän verran asumuksen kummallakin sivulla sitä peittämässä.
\par 14 Ja tee teltalle peite punaisista oinaannahoista ja sen päälle vielä toinen peite sireeninnahoista.
\par 15 Asumuksen laudat tee akasiapuusta, pystyyn asetettaviksi.
\par 16 Jokainen lauta olkoon kymmentä kyynärää pitkä ja puoltatoista kyynärää leveä.
\par 17 Jokaisessa laudassa olkoon kaksi tappia, jotka ovat poikkilistalla yhdistetyt keskenään; tee näin kaikki asumuksen laudat.
\par 18 Ja asumuksen lautoja tee kaksikymmentä lautaa eteläpuolta varten.
\par 19 Ja tee neljäkymmentä hopeajalustaa kahdenkymmenen laudan alle, aina kaksi jalustaa kunkin laudan alle sen kahta tappia varten.
\par 20 Samoin asumuksen toista sivua, pohjoispuolta, varten kaksikymmentä lautaa,
\par 21 ja neljäkymmentä hopeajalustaa, aina kaksi jalustaa kunkin laudan alle.
\par 22 Mutta asumuksen takasivua, länsipuolta, varten tee kuusi lautaa.
\par 23 Ja tee kaksi lautaa asumuksen peränurkkia varten.
\par 24 Ja ne olkoot yhteenliitettyjä kaksoislautoja ja alhaalta alkaen kiinni toisissaan ylös saakka, ensimmäiseen renkaaseen asti; näin tehtäköön ne molemmat ja asetettakoon kumpaankin nurkkaan.
\par 25 Näin tulee olemaan yhteensä kahdeksan lautaa ja niihin kuusitoista hopeajalustaa, aina kaksi jalustaa kunkin laudan alla.
\par 26 Tee myös viisi poikkitankoa akasiapuusta asumuksen toisen sivun lautoja varten,
\par 27 ja viisi poikkitankoa asumuksen toisen sivun lautoja varten, ja viisi poikkitankoa asumuksen takasivun, länsipuolen, lautoja varten.
\par 28 Ja keskimmäinen poikkitanko asetettakoon keskelle lautoja, ja kulkekoon se reunasta reunaan.
\par 29 Ja päällystä laudat kullalla ja tee kullasta niiden renkaat poikkitankojen pitimiksi ja päällystä poikkitangot kullalla.
\par 30 Ja aseta asumus pystyyn sen muotoiseksi, kuin sinulle vuorella näytettiin.
\par 31 Tee vielä esirippu punasinisistä, purppuranpunaisista ja helakanpunaisista langoista ja kerratuista valkoisista pellavalangoista; ja tehtäköön siihen taidokkaasti kudottuja kerubeja.
\par 32 Ja ripusta se neljään akasiapuiseen, kullalla päällystettyyn pylvääseen, joissa on kultakoukut ja jotka seisovat neljällä hopeajalustalla.
\par 33 Ja ripusta esirippu hakasten alle ja vie sinne esiripun sisäpuolelle lain arkki. Ja niin olkoon esirippu teille väliseinänä pyhän ja kaikkeinpyhimmän välillä.
\par 34 Ja aseta armoistuin lain arkin päälle, joka on kaikkeinpyhimmässä.
\par 35 Mutta pöytä sijoita esiripun ulkopuolelle ja seitsenhaarainen lamppu vastapäätä pöytää, asumuksen eteläsivulle; aseta siis pöytä pohjoissivulle.
\par 36 Tee myös teltan oveen uudin, kirjaellen kudottu punasinisistä, purppuranpunaisista ja helakanpunaisista langoista ja kerratuista valkoisista pellavalangoista.
\par 37 Ja tee uudinta varten viisi pylvästä akasiapuusta ja päällystä ne kullalla, mutta niiden koukut olkoot kultaa; ja vala niille viisi vaskijalustaa."

\chapter{27}

\par 1 "Ja tee alttari akasiapuusta; se olkoon neliskulmainen, viittä kyynärää pitkä ja viittä kyynärää leveä sekä kolmea kyynärää korkea.
\par 2 Tee siihen sarvet, sen neljään kulmaan, niin että sarvet ovat samaa kappaletta kuin se. Ja päällystä se vaskella.
\par 3 Tee siihen kuuluvat kattilat tuhan poisviemistä varten sekä lapiot, maljat, haarukat ja hiilipannut. Kaikki sen kalusto tee vaskesta.
\par 4 Ja tee siihen myös verkonkaltainen ristikkokehys vaskesta ja tee verkkoon neljä vaskirengasta, ristikon neljään kulmaan.
\par 5 Ja aseta se alttarin välireunuksen alle, maahan kiinni, niin että verkko ulottuu puolitiehen alttaria.
\par 6 Ja tee alttariin korennot akasiapuusta ja päällystä ne vaskella.
\par 7 Ja korennot pistettäköön renkaisiin, niin että korennot ovat kahden puolen alttaria, sitä kannettaessa.
\par 8 Tee se laudoista, ontoksi. Niinkuin sinulle näytettiin vuorella, niin se tehtäköön.
\par 9 Tee myös asumukselle esipiha. Etelän puolella olkoot esipihan ympärysverhot kerratuista valkoisista pellavalangoista, sadan kyynärän pituiset tätä yhtä sivua varten;
\par 10 ja olkoon niiden pylväitä kaksikymmentä ja näiden vaskijalustoja kaksikymmentä, mutta pylväiden koukut ja niiden koristepienat olkoot hopeata.
\par 11 Samoin myös pohjoisen puolella olkoot ympärysverhot sadan kyynärän pituiset; ja olkoon niiden pylväitä kaksikymmentä ja näiden vaskijalustoja kaksikymmentä, mutta pylväiden koukut ja niiden koristepienat olkoot hopeata.
\par 12 Ja esipihan lännenpuoleisella sivulla olkoot ympärysverhot viidenkymmenen kyynärän pituiset, ja olkoon niiden pylväitä kymmenen ja näiden jalustoja kymmenen.
\par 13 Ja esipihan leveys etupuolella, itään päin, olkoon viisikymmentä kyynärää.
\par 14 Ja olkoot ympärysverhot portin toisella puolella viidentoista kyynärän pituiset, ja olkoon niiden pylväitä kolme ja näiden jalustoja kolme.
\par 15 Samoin olkoot ympärysverhot toisella puolella viidentoista kyynärän pituiset, ja olkoon niiden pylväitä kolme ja näiden jalustoja kolme.
\par 16 Ja esipihan portissa olkoon kahdenkymmenen kyynärän pituinen uudin, kirjaellen kudottu punasinisistä, purppuranpunaisista ja helakanpunaisista langoista ja kerratuista valkoisista pellavalangoista, ja olkoon sen pylväitä neljä ja näiden jalustoja neljä.
\par 17 Kaikissa pylväissä esipihan ympärillä olkoot koristepienat hopeata ja koukut hopeata, mutta jalustat vaskea.
\par 18 Esipihan pituus olkoon sata kyynärää ja leveys viisikymmentä kyynärää; ympärysverho olkoon viittä kyynärää korkea ja kudottu kerratuista valkoisista pellavalangoista, ja jalustat olkoot vaskea.
\par 19 Koko asumuksen kalusto kaikkia siinä tehtäviä töitä varten, samoin kuin kaikki sen vaarnat ja kaikki esipihan vaarnat, olkoot vaskea.
\par 20 Ja käske israelilaisten tuoda sinulle puhdasta, survomalla saatua öljypuun öljyä seitsenhaaraista lamppua varten, että lamput aina voidaan nostaa paikoilleen.
\par 21 Ilmestysmajassa, ulkopuolella esirippua, joka on lain arkin edessä, hoitakoon Aaron poikineen niitä illasta aamuun asti Herran edessä. Tämä olkoon ikuinen säädös, jota israelilaiset noudattakoot sukupolvesta sukupolveen."

\chapter{28}

\par 1 "Ja kutsu eteesi israelilaisten joukosta veljesi Aaron poikineen, että he pappeina palvelisivat minua, Aaron ja hänen poikansa Naadab ja Abihu, Eleasar ja Iitamar.
\par 2 Ja teetä veljellesi Aaronille pyhät vaatteet, kunniaksi ja kaunistukseksi.
\par 3 Ja puhuttele kaikkia taidollisia miehiä, jotka minä olen täyttänyt taidollisuuden hengellä, että he tekevät vaatteet Aaronille, jotta hänet pyhitettäisiin pappina palvelemaan minua.
\par 4 Ja nämä ovat ne vaatteet, jotka heidän on tehtävä: rintakilpi, kasukka, viitta, ruutuinen ihokas, käärelakki ja vyö. Ja he tehkööt sinun veljellesi Aaronille ja hänen pojilleen pyhät vaatteet, että he pappeina palvelisivat minua.
\par 5 Ja he ottakoot tätä varten kultaa sekä punasinisiä, purppuranpunaisia ja helakanpunaisia lankoja ja valkoisia pellavalankoja.
\par 6 Kasukan he tehkööt kullasta sekä punasinisistä, purppuranpunaisista ja helakanpunaisista langoista ja kerratuista valkoisista pellavalangoista, taidokkaasti kutomalla.
\par 7 Siinä olkoon kaksi yhdistettävää olkakappaletta, ja se kiinnitettäköön niihin molemmista päistään.
\par 8 Ja vyö, jolla kasukka kiinnitetään, olkoon tehty samalla tavalla ja samasta kappaleesta kuin se: kullasta sekä punasinisistä, purppuranpunaisista ja helakanpunaisista langoista ja kerratuista valkoisista pellavalangoista.
\par 9 Ota sitten kaksi onyks-kiveä ja kaiverra niihin Israelin poikien nimet,
\par 10 kuusi heidän nimeään toiseen kiveen ja toiset kuusi nimeä toiseen kiveen siinä järjestyksessä, kuin he ovat syntyneet.
\par 11 Niinkuin taidokkaasti hiotaan kiveä, kaiverretaan sinettisormuksia, niin kaiverra Israelin poikien nimet niihin kahteen kiveen. Ja kehystettäköön ne kultapalmikoimilla.
\par 12 Ja pane molemmat kivet kasukan olkakappaleihin kiviksi, jotka johdattavat muistoon israelilaiset; näin Aaron kantakoon heidän nimiänsä molemmilla olkapäillään Herran kasvojen edessä, että heitä muistettaisiin.
\par 13 Ja tee kultapalmikoimia
\par 14 ja kahdet käädyt puhtaasta kullasta; tee ne punomalla, niinkuin punonnaista tehdään, ja kiinnitä punotut käädyt palmikoimiin.
\par 15 Ja jumalanvastausten rintakilpi tee taidokkaasti kutomalla; tee sekin samalla tavalla, kuin kasukka on tehty: tee se kullasta sekä punasinisistä, purppuranpunaisista ja helakanpunaisista langoista ja kerratuista valkoisista pellavalangoista.
\par 16 Se olkoon neliskulmainen ja taskun muotoon tehty, vaaksan pituinen ja vaaksan levyinen.
\par 17 Ja kiinnitä sen pintaan kiviä yliyltään, neljään riviin: ensimmäiseen riviin karneoli, topaasi ja smaragdi;
\par 18 toiseen riviin rubiini, safiiri ja jaspis;
\par 19 kolmanteen riviin hyasintti, akaatti ja ametisti;
\par 20 ja neljänteen riviin krysoliitti, onyks ja berylli. Kultapalmikoimilla kehystettyinä kiinnitettäköön ne paikoilleen.
\par 21 Kiviä olkoon Israelin poikain nimien mukaan kaksitoista, yksi kutakin nimeä kohti; kussakin kivessä olkoon yksi kahdentoista sukukunnan nimistä, kaiverrettuna samalla tavalla, kuin kaiverretaan sinettisormuksia.
\par 22 Ja tee rintakilpeen puhtaasta kullasta käädyt, punotut, niinkuin punonnaista tehdään.
\par 23 Ja tee rintakilpeen kaksi kultarengasta ja kiinnitä molemmat renkaat rintakilven kahteen yläkulmaan.
\par 24 Ja kiinnitä ne molemmat kultapunonnaiset kahteen renkaaseen rintakilven yläkulmiin.
\par 25 Ja kiinnitä molempien punonnaisten toiset kaksi päätä kahteen palmikoimaan ja kiinnitä nämä kasukan olkakappaleihin, sen etupuolelle.
\par 26 Ja tee vielä kaksi kultarengasta ja pane ne rintakilven molempiin alakulmiin, sen sisäpuoliseen, kasukkaa vasten olevaan reunaan.
\par 27 Ja tee vieläkin kaksi kultarengasta ja kiinnitä ne kasukan molempiin olkakappaleihin, niiden alareunaan, etupuolelle, sauman kohdalle, kasukan vyön yläpuolelle.
\par 28 Ja rintakilpi solmittakoon renkaistaan punasinisellä nauhalla kasukan renkaisiin, niin että se on kasukan vyön yläpuolella; näin rintakilpi ei irtaudu kasukasta.
\par 29 Ja niin Aaron kantakoon jumalanvastausten rintakilvessä sydämensä päällä, astuessaan pyhäkköön, Israelin poikain nimet, että heidät alati johdatettaisiin muistoon Herran edessä.
\par 30 Ja pane jumalanvastausten rintakilpeen uurim ja tummim, niin että ne ovat Aaronin sydämen päällä, hänen astuessaan Herran eteen. Ja niin kantakoon Aaron Jumalan vastaukset israelilaisille sydämensä päällä aina Herran edessä ollessaan.
\par 31 Kasukan viitta tee kokonaan punasinisistä langoista;
\par 32 ja sen keskellä olkoon pääntie, ja tämä pääntie ympäröitäköön kudotulla päärmeellä niinkuin haarniskan aukko, ettei se repeäisi.
\par 33 Ja tee sen helmaan granaattiomenia punasinisistä, purppuranpunaisista ja helakanpunaisista langoista, ja kiinnitä ne helmaan ympärinsä ja niiden väliin kultatiukuja yltympäri,
\par 34 vuorotellen kultatiuku ja granaattiomena, viitan helmaan ympärinsä.
\par 35 Ja Aaron pitäköön sen yllään toimittaessaan virkaansa, niin että kuuluu, kun hän menee pyhäkköön Herran eteen ja tulee sieltä ulos, ettei hän kuolisi.
\par 36 Tee myös otsakoriste puhtaasta kullasta ja kaiverra siihen, niinkuin sinettisormusta kaiverretaan, sanat: 'Herralle pyhitetty'.
\par 37 Ja sido se punasinisellä nauhalla, niin että se on kiinni käärelakissa; etupuolella käärelakkia se olkoon.
\par 38 Ja se olkoon Aaronin otsalla, niin että Aaron kantaa kaiken sen, mitä israelilaiset rikkovat uhratessaan pyhiä uhrejansa ja antaessaan pyhiä lahjojansa; ja se olkoon alati hänen otsallaan, että Herran mielisuosio tulisi heidän osaksensa.
\par 39 Kudo myös ruutuinen ihokas valkoisista pellavalangoista, ja tee käärelakki valkoisista pellavalangoista, ja tee vyö kirjokudoksesta.
\par 40 Ja Aaronin pojille tee ihokkaat ja vyöt; ja tee heille päähineet kunniaksi ja kaunistukseksi.
\par 41 Ja pue ne veljesi Aaronin ja hänen poikiensa ylle; ja voitele heidät ja vihi heidät virkaansa ja pyhitä heidät pappeina palvelemaan minua.
\par 42 Ja tee heille pellavakaatiot hävyn peitteeksi; ulottukoot ne lanteilta reisiin asti.
\par 43 Ja Aaron ja hänen poikansa pitäkööt ne yllään mennessänsä ilmestysmajaan, tahi kun he lähestyvät alttaria tehdäkseen palvelusta pyhäkössä, etteivät he joutuisi syynalaisiksi ja kuolisi. Tämä olkoon ikuinen säädös hänelle ja hänen jälkeläisilleen."

\chapter{29}

\par 1 "Tee heille näin, pyhittääksesi heidät pappeina palvelemaan minua: Ota mullikka ja kaksi virheetöntä oinasta
\par 2 ja happamatonta leipää ja öljyyn leivottuja happamattomia kakkuja ja öljyllä voideltuja happamattomia ohukaisia; leivo ne lestyistä nisujauhoista.
\par 3 Ja pane ne samaan koriin ja tuo ne korissa, samalla kertaa kuin tuot mullikan ja kaksi oinasta.
\par 4 Tuo sitten Aaron poikinensa ilmestysmajan ovelle ja pese heidät vedellä.
\par 5 Ja ota vaatteet ja pue Aaronin ylle ihokas ja kasukan viitta ja kasukka ja rintakilpi; ja sido hänen ympärilleen kasukan vyö.
\par 6 Pane myös käärelakki hänen päähänsä ja kiinnitä pyhä otsalehti käärelakkiin.
\par 7 Ja ota voiteluöljyä ja vuodata hänen päähänsä ja voitele hänet.
\par 8 Ja tuo hänen poikansa esille ja pue heidän ylleen ihokkaat.
\par 9 Ja vyötä heidät vyöllä, sekä Aaron että hänen poikansa, ja sido päähineet heidän päähänsä, että pappeus olisi heillä ikuisena säätynä. Vihi näin virkaansa Aaron ja hänen poikansa.
\par 10 Ja tuo mullikka ilmestysmajan eteen, ja Aaron poikineen laskekoon kätensä mullikan pään päälle.
\par 11 Ja teurasta sitten mullikka Herran edessä, ilmestysmajan ovella.
\par 12 Ja ota mullikan verta ja sivele sitä sormellasi alttarin sarviin; mutta kaikki muu veri vuodata alttarin juurelle.
\par 13 Ja ota kaikki sisälmyksiä peittävä rasva ja maksanlisäke ja molemmat munuaiset ynnä niiden päällä oleva rasva, ja polta ne alttarilla.
\par 14 Mutta mullikan liha, nahka ja rapa polta tulessa leirin ulkopuolella. Se on syntiuhri.
\par 15 Ja ota toinen oinaista, ja Aaron poikinensa laskekoon kätensä sen oinaan pään päälle.
\par 16 Teurasta sitten oinas ja ota sen veri ja vihmo se alttarille ympärinsä;
\par 17 ja leikkele oinas määräkappaleiksi ja pese sen sisälmykset ja jalat ja pane ne kappaleiden ja pään päälle.
\par 18 Ja polta koko oinas alttarilla. Se on polttouhri Herralle, se on suloisesti tuoksuva uhri Herralle.
\par 19 Ota sitten toinen oinas, ja Aaron poikinensa laskekoon kätensä oinaan pään päälle.
\par 20 Ja teurasta oinas ja ota sen verta ja sivele Aaronin ja hänen poikiensa oikean korvan lehteen ja oikean käden peukaloon ja heidän oikean jalkansa isoonvarpaaseen, mutta vihmo muu veri alttarille ympärinsä.
\par 21 Ja ota alttarilla olevaa verta ja voiteluöljyä ja pirskoita Aaronin ja hänen vaatteidensa päälle, ja samoin hänen poikiensa ja heidän vaatteidensa päälle. Näin hän tulee pyhäkölle pyhitetyksi, sekä hän itse että hänen vaatteensa, ja samoin hänen poikansa ja hänen poikiensa vaatteet.
\par 22 Ota sitten oinaasta rasva ja rasvahäntä ja sisälmyksiä peittävä rasva ja maksanlisäke ja molemmat munuaiset ynnä niiden päällä oleva rasva ja oikea reisi, sillä tämä on vihkiäisoinas.
\par 23 Ja ota pyöreä leipä ja öljyyn leivottu kakku ja ohukainen happamattomien leipien korista, joka on Herran edessä.
\par 24 Ja pane kaikki nämä Aaronin käsiin ja hänen poikiensa käsiin, että toimitettaisiin niiden heilutus Herran edessä.
\par 25 Ota ne sitten heidän käsistään ja polta alttarilla, polttouhrin päällä, suloiseksi tuoksuksi Herran edessä. Se on Herran uhri.
\par 26 Ja ota rintaliha Aaronin vihkiäisoinaasta ja toimituta sen heilutus Herran edessä; ja se olkoon sinun osasi.
\par 27 Näin sinun on pyhitettävä heilutettu rintaliha ja anniksi annettu reisi, se, mitä on heilutettu ja anniksi annettu Aaronin ja hänen poikiensa vihkiäisoinaasta.
\par 28 Ja ne olkoot Aaronin ja hänen poikiensa ikuinen osuus, israelilaisilta saatu, sillä se on anti. Se olkoon israelilaisten anti heidän yhteysuhreistaan, heidän antinsa Herralle.
\par 29 Aaronin pyhät vaatteet tulkoot hänen pojillensa hänen jälkeensä, että heidät niissä voideltaisiin ja vihittäisiin virkaansa.
\par 30 Seitsemänä päivänä on sen hänen pojistaan, joka tulee papiksi hänen sijaansa, puettava ne ylleen, kun hän menee ilmestysmajaan toimittamaan virkaansa pyhäkössä.
\par 31 Ja ota vihkiäisoinas ja keitä sen liha pyhässä paikassa.
\par 32 Ja Aaron poikinensa syököön ilmestysmajan ovella oinaan lihan ynnä leivän, joka on korissa.
\par 33 He syökööt sen, mitä käytettiin heidän sovittamisekseen, kun heidät vihittiin virkaansa ja pyhitettiin, mutta syrjäinen älköön sitä syökö, sillä se on pyhää.
\par 34 Jos jotakin jää tähteeksi vihkiäislihasta tai leivästä seuraavaan aamuun, polta tähteeksi jäänyt tulessa; älköön sitä syötäkö, sillä se on pyhää.
\par 35 Tee näin Aaronille ja hänen pojillensa, aivan niinkuin minä olen sinua käskenyt. Seitsemän päivää kestäköön heidän vihkimisensä.
\par 36 Ja uhraa joka päivä syntiuhrimullikka sovitukseksi, ja puhdista alttari toimittamalla sen sovitus ja voitele se pyhittääksesi sen.
\par 37 Seitsemänä päivänä toimita alttarin sovitus ja pyhitä se. Näin tulee alttarista korkeasti-pyhä; jokainen, joka alttariin koskee, on pyhäkölle pyhitetty.
\par 38 Ja uhraa alttarilla tämä: kaksi vuodenvanhaa karitsaa joka päivä, ainiaan.
\par 39 Uhraa toinen karitsa aamulla ja toinen karitsa iltahämärässä,
\par 40 ja kumpaakin karitsaa kohti kymmenennes lestyjä jauhoja, sekoitettuna neljännekseen hiin-mittaa survomalla saatua öljyä, ja juomauhriksi neljännes hiin-mittaa viiniä.
\par 41 Ja uhraa toinen karitsa iltahämärässä; uhrattakoon se toimittamalla samankaltainen ruokauhri ja juomauhri kuin aamullakin, suloisesti tuoksuvaksi uhriksi Herralle.
\par 42 Tämä olkoon teillä jokapäiväinen polttouhri sukupolvesta sukupolveen ilmestysmajan ovella Herran edessä, jossa minä ilmestyn teille puhuakseni sinun kanssasi.
\par 43 Siinä minä ilmestyn israelilaisille, ja se on oleva minun kirkkauteni pyhittämä.
\par 44 Ja minä pyhitän ilmestysmajan ja alttarin; ja minä pyhitän Aaronin poikinensa pappeina palvelemaan minua.
\par 45 Ja minä asun israelilaisten keskellä ja olen heidän Jumalansa.
\par 46 Ja he tulevat tietämään, että minä olen Herra, heidän Jumalansa, joka vein heidät pois Egyptin maasta, asuakseni heidän keskellään. Minä olen Herra, heidän Jumalansa."

\chapter{30}

\par 1 "Tee myös alttari suitsutuksen polttamista varten; tee se akasiapuusta.
\par 2 Olkoon se kyynärän pituinen ja kyynärän levyinen, siis neliskulmainen, ja kahta kyynärää korkea; sen sarvet olkoot samaa kappaletta kuin sekin.
\par 3 Ja päällystä se puhtaalla kullalla, sekä sen levy että sivut ympärinsä ja sen sarvet; ja tee kultareunus sen ympäri.
\par 4 Ja tee siihen kaksi kultarengasta; pane ne reunuksen alle, kummallekin sivulle, molempiin sivukappaleihin. Ne olkoot niiden korentojen pitiminä, joilla alttari on kannettava.
\par 5 Ja tee korennotkin akasiapuusta ja päällystä ne kullalla.
\par 6 Ja aseta se lain arkin edessä olevan esiripun eteen, niin että se tulee armoistuimen kohdalle, joka on lain arkin päällä ja jossa minä sinulle ilmestyn.
\par 7 Ja Aaron polttakoon sen päällä hyvänhajuista suitsutusta; joka aamu, kun hän laittaa lamput kuntoon, hän polttakoon sitä.
\par 8 Samoin myös, kun Aaron iltahämärässä nostaa lamput paikoilleen, hän polttakoon sitä. Tämä olkoon teillä jokapäiväinen suitsutusuhri Herran edessä sukupolvesta sukupolveen.
\par 9 Älkää uhratko sen päällä vierasta suitsutusta älkääkä polttouhria tai ruokauhria; älkää myöskään vuodattako juomauhria sen päällä.
\par 10 Ja Aaron toimittakoon kerran vuodessa sen sarvien sovituksen; sovitukseksi uhratun syntiuhrin verellä hän toimittakoon kerran vuodessa sen sovituksen, sukupolvesta sukupolveen. Se on korkeasti-pyhä Herralle."
\par 11 Ja Herra puhui Moosekselle sanoen:
\par 12 "Kun sinä lasket israelilaisten lukumäärän - niiden, joiden on oltava katselmuksessa - niin jokainen heistä suorittakoon, heistä katselmusta pidettäessä, hengestään sovitusmaksun Herralle, ettei mikään rangaistus heitä kohtaisi, heistä katselmusta pidettäessä.
\par 13 Jokainen katselmuksessa oleva antakoon puoli sekeliä, pyhäkkösekelin painon mukaan, kaksikymmentä geeraa laskettuna sekeliin; puoli sekeliä olkoon anti Herralle.
\par 14 Kaikki katselmuksessa olevat, kaksikymmenvuotiaat ja sitä vanhemmat, antakoot tämän annin Herralle.
\par 15 Rikas älköön antako enemmän älköönkä köyhä vähemmän kuin puoli sekeliä, antina Herralle, maksaaksenne sovituksen hengestänne.
\par 16 Ja ota sovitusrahat israelilaisilta ja käytä ne palvelukseen ilmestysmajassa, että israelilaiset johdatettaisiin muistoon Herran edessä teidän henkenne sovitukseksi."
\par 17 Ja Herra puhui Moosekselle sanoen:
\par 18 "Tee myös vaskiallas vaskijalustoineen peseytymistä varten ja aseta se ilmestysmajan ja alttarin välille ja kaada siihen vettä;
\par 19 ja Aaron ja hänen poikansa peskööt siinä kätensä ja jalkansa.
\par 20 Kun he menevät ilmestysmajaan, peseytykööt vedessä, etteivät kuolisi; samoin myös, kun he lähestyvät alttaria ja käyvät toimittamaan virkaansa polttamalla uhrin Herralle.
\par 21 He peskööt kätensä ja jalkansa, etteivät kuolisi. Ja tämä olkoon heille ikuinen säädös, hänelle itselleen ja hänen jälkeläisillensä, sukupolvesta sukupolveen."
\par 22 Ja Herra puhui Moosekselle sanoen:
\par 23 "Ota itsellesi hajuaineita parasta lajia: sulavaa mirhaa viisisataa sekeliä, hyvänhajuista kanelia puolet siitä eli kaksisataa viisikymmentä sekeliä ja hyvänhajuista kalmoruokoa samoin kaksisataa viisikymmentä sekeliä,
\par 24 sitten vielä kassiaa viisisataa sekeliä, pyhäkkösekelin painon mukaan, ja hiin-mitta öljypuun öljyä.
\par 25 Ja tee siitä pyhä voiteluöljy, höystetty voide, jollaista voiteensekoittaja valmistaa; se olkoon pyhä voiteluöljy.
\par 26 Voitele sillä ilmestysmaja, lain arkki
\par 27 ja pöytä kaikkine kaluineen, seitsenhaarainen lamppu kaluineen, niin myös suitsutusalttari,
\par 28 polttouhrialttari kaikkine kaluineen ynnä allas jalustoineen.
\par 29 Ja pyhitä ne, niin että ne tulevat korkeasti-pyhiksi. Jokainen, joka niihin koskee, tulee pyhäksi.
\par 30 Voitele myös Aaron ja hänen poikansa ja pyhitä heidät pappeina palvelemaan minua.
\par 31 Puhu myös israelilaisille ja sano: Tämä olkoon teillä minun pyhä voiteluöljyni sukupolvesta sukupolveen.
\par 32 Kenenkään muun ihmisen ruumiille älköön sitä vuodatettako, älkääkä sen sekoitusta jäljitelkö. Pyhä se on, ja pyhänä se pitäkää.
\par 33 Jokainen, joka valmistaa sellaisen voiteen, ja jokainen, joka sivelee sitä syrjäiseen, hävitettäköön kansastansa."
\par 34 Ja Herra sanoi Moosekselle vielä: "Ota itsellesi hyvänhajuisia aineita, hajupihkaa, simpukankuorta, tuoksukumia, näitä hyvänhajuisia aineita, ja puhdasta suitsuketta, yhtä paljon kutakin lajia,
\par 35 ja tee niistä suitsutus, höystesekoitus, jollaista voiteensekoittaja valmistaa, suolansekainen, puhdas ja pyhä.
\par 36 Ja hienonna osa siitä jauhoksi ja pane sitä lain arkin eteen ilmestysmajaan, jossa minä ilmestyn sinulle. Korkeasti-pyhänä se pitäkää.
\par 37 Älkää valmistako itsellenne mitään muuta suitsutusta tämän sekoituksen mukaan. Pidä tämä Herralle pyhitettynä.
\par 38 Jokainen, joka sellaista tekee nauttiaksensa sen tuoksusta, hävitettäköön kansastansa."

\chapter{31}

\par 1 Ja Herra puhui Moosekselle sanoen:
\par 2 "Katso, minä olen nimeltään kutsunut Besalelin, Uurin pojan, Huurin pojanpojan, Juudan sukukunnasta;
\par 3 ja minä olen täyttänyt hänet Jumalan hengellä, taidollisuudella, ymmärryksellä, tiedolla ja kaikkinaisella kätevyydellä
\par 4 sommittelemaan taidokkaita teoksia ja valmistamaan niitä kullasta, hopeasta ja vaskesta,
\par 5 hiomaan ja kiinnittämään kiviä ja veistämään puuta, tekemään kaikkinaisia töitä.
\par 6 Ja katso, minä olen antanut hänelle apulaiseksi Oholiabin, Ahisamakin pojan, Daanin sukukunnasta, ja olen antanut kaikkien taidollisten sydämeen taidollisuuden tehdä kaikki, mitä minä olen sinun käskenyt teettää:
\par 7 ilmestysmajan, lain arkin, armoistuimen sen päälle, kaiken majan kaluston,
\par 8 pöydän kaluinensa, aitokultaisen seitsenhaaraisen lampun kaikkine kaluinensa, suitsutusalttarin,
\par 9 polttouhrialttarin kaikkine kaluinensa, altaan jalustoineen,
\par 10 virkapuvut ja pappi Aaronin muut pyhät vaatteet sekä hänen poikiensa pappispuvut,
\par 11 voiteluöljyn ja hyvänhajuisen suitsukkeen pyhäkköä varten. He tehkööt kaiken niin, kuin minä olen sinulle käskyn antanut."
\par 12 Ja Herra puhui Moosekselle sanoen:
\par 13 "Puhu israelilaisille ja sano: Pitäkää minun sapattini, sillä se on merkkinä meidän välillämme, minun ja teidän, sukupolvesta sukupolveen, tietääksenne, että minä olen Herra, joka pyhitän teidät.
\par 14 Siis pitäkää sapatti, sillä se on teille pyhä. Joka sen rikkoo, rangaistakoon kuolemalla; sillä kuka ikinä silloin työtä tekee, hävitettäköön kansastansa.
\par 15 Kuusi päivää tehtäköön työtä, mutta seitsemäntenä päivänä on sapatti, levon päivä, Herralle pyhitetty. Kuka ikinä tekee työtä sapatinpäivänä, rangaistakoon kuolemalla.
\par 16 Ja pitäkööt israelilaiset sapatin, niin että he viettävät sapattia sukupolvesta sukupolveen ikuisena liittona.
\par 17 Se on oleva ikuinen merkki minun ja israelilaisten välillä; sillä kuutena päivänä Herra teki taivaan ja maan, mutta seitsemäntenä päivänä hän lepäsi ja hengähti."
\par 18 Ja kun hän oli lakannut puhumasta Mooseksen kanssa Siinain vuorella, antoi hän hänelle kaksi laintaulua, kivitaulua, joitten kirjoitus oli Jumalan sormella kirjoitettu.

\chapter{32}

\par 1 Mutta kun kansa näki, että Mooses viipyi eikä tullut alas vuorelta, kokoontui kansa Aaronin ympärille ja sanoi hänelle: "Nouse, tee meille jumala, joka käy meidän edellämme. Sillä me emme tiedä, mitä on tapahtunut Moosekselle, tälle miehelle, joka johdatti meidät Egyptin maasta."
\par 2 Niin Aaron sanoi heille: "Irroittakaa kultarenkaat, jotka ovat vaimojenne, poikienne ja tyttärienne korvissa, ja tuokaa ne minulle".
\par 3 Ja kaikki kansa irroitti kultarenkaat, jotka heillä oli korvissaan, ja he toivat ne Aaronille;
\par 4 ja hän otti vastaan kullan heidän käsistään, kaavaili sitä piirtimellä ja teki siitä valetun vasikan. Ja he sanoivat: "Tämä on sinun jumalasi, Israel, se, joka on johdattanut sinut Egyptin maasta".
\par 5 Kun Aaron tämän näki, rakensi hän sille alttarin; ja Aaron julisti ja sanoi: "Huomenna on Herran juhla".
\par 6 Ja he nousivat varhain seuraavana päivänä ja uhrasivat polttouhreja ja toivat yhteysuhreja; ja kansa istui syömään ja juomaan, ja sitten he nousivat iloa pitämään.
\par 7 Silloin Herra sanoi Moosekselle: "Mene, astu alas, sillä sinun kansasi, jonka johdatit Egyptin maasta, on turmion tehnyt.
\par 8 Pian he poikkesivat siltä tieltä, jota minä käskin heidän kulkea; he tekivät itselleen valetun vasikan. Sitä he ovat kumartaneet, ja sille he ovat uhranneet ja sanoneet: 'Tämä on sinun jumalasi, Israel, se, joka on johdattanut sinut Egyptin maasta'."
\par 9 Ja Herra sanoi vielä Moosekselle: "Minä näen, että tämä kansa on niskurikansa.
\par 10 Anna minun olla, että vihani leimahtaisi heitä vastaan, hukuttaakseni heidät; mutta sinusta minä teen suuren kansan."
\par 11 Mutta Mooses rukoili armoa Herralta, Jumalaltansa, ja sanoi: "Herra, miksi sinun vihasi syttyy omaa kansaasi vastaan, jonka olet vienyt pois Egyptin maasta suurella voimalla ja väkevällä kädellä?
\par 12 Miksi egyptiläiset saisivat sanoa: 'Heidän onnettomuudekseen hän vei heidät pois, tappaaksensa heidät vuorilla ja hävittääksensä heidät maan päältä'? Käänny vihasi hehkusta ja kadu sitä turmiota, jonka aioit tuottaa kansallesi.
\par 13 Muista palvelijoitasi Aabrahamia, Iisakia ja Israelia, joille olet itse kauttasi vannonut ja sanonut: 'Minä teen teidän jälkeläistenne luvun paljoksi kuin taivaan tähdet; ja koko tämän maan, josta olen puhunut, minä annan teidän jälkeläisillenne, ja he saavat sen ikuiseksi perinnöksi'."
\par 14 Niin Herra katui sitä turmiota, jonka hän oli uhannut tuottaa kansallensa.
\par 15 Ja Mooses kääntyi ja astui alas vuorelta, ja hänen kädessään oli kaksi laintaulua. Ja tauluihin oli kirjoitettu molemmille puolille; etupuolelle ja takapuolelle oli niihin kirjoitettu.
\par 16 Ja taulut olivat Jumalan tekemät, ja kirjoitus oli Jumalan kirjoitusta, joka oli tauluihin kaiverrettu.
\par 17 Kun Joosua kuuli kansan huudon, sen melutessa, sanoi hän Moosekselle: "Sotahuuto kuuluu leiristä".
\par 18 Mutta tämä vastasi: "Se ei ole voittajien huutoa, eikä se ole voitettujen huutoa; minä kuulen laulua".
\par 19 Ja kun Mooses lähestyi leiriä ja näki vasikan ja karkelon, niin hänen vihansa syttyi, ja hän heitti taulut käsistänsä ja murskasi ne vuoren juurella.
\par 20 Senjälkeen hän otti vasikan, jonka he olivat tehneet, poltti sen tulessa ja rouhensi sen hienoksi ja hajotti veteen ja juotti sen israelilaisille.
\par 21 Ja Mooses sanoi Aaronille: "Mitä tämä kansa on tehnyt sinulle, kun olet saattanut heidät näin suureen syntiin?"
\par 22 Aaron vastasi: "Älköön herrani viha syttykö; sinä tiedät itse, että tämä kansa on paha.
\par 23 He sanoivat minulle: 'Tee meille jumala, joka käy meidän edellämme; sillä emme tiedä, mitä on tapahtunut Moosekselle, tälle miehelle, joka johdatti meidät Egyptin maasta'.
\par 24 Niin minä sanoin heille: 'Jolla on kultaa, irroittakoon sen yltänsä'; ja he antoivat sen minulle. Ja minä heitin sen tuleen, ja siitä tuli tämä vasikka."
\par 25 Kun Mooses näki, että kansa oli kurittomuuden vallassa, koska Aaron oli päästänyt heidät kurittomuuden valtaan, vahingoniloksi heidän vihollisillensa,
\par 26 niin Mooses seisahtui leirin portille ja huusi: "Joka on Herran oma, se tulkoon minun luokseni". Silloin kokoontuivat hänen luoksensa kaikki leeviläiset.
\par 27 Ja hän sanoi heille: "Näin sanoo Herra, Israelin Jumala: 'Jokainen sitokoon miekkansa vyölleen. Käykää sitten edestakaisin leirin halki portista porttiin ja tappakaa jokainen, olkoon vaikka oma veli, ystävä tai sukulainen.'"
\par 28 Niin leeviläiset tekivät Mooseksen käskyn mukaan; ja sinä päivänä kaatui kansaa noin kolmetuhatta miestä.
\par 29 Ja Mooses sanoi: "Koska te nyt olette olleet omia poikianne ja veljiänne vastaan, niin vihkiytykää tänä päivänä Herran palvelukseen, että hän tänä päivänä antaisi siunauksen teille".
\par 30 Seuraavana päivänä Mooses sanoi kansalle: "Te olette tehneet suuren synnin. Minä nousen nyt Herran tykö - jos ehkä voisin sovittaa teidän rikoksenne."
\par 31 Ja Mooses palasi Herran tykö ja sanoi: "Voi, tämä kansa on tehnyt suuren synnin! He ovat tehneet itselleen jumalan kullasta.
\par 32 Jospa nyt antaisit heidän rikoksensa anteeksi! Mutta jos et, niin pyyhi minut pois kirjastasi, johon kirjoitat."
\par 33 Mutta Herra vastasi Moosekselle: "Joka on tehnyt syntiä minua vastaan, sen minä pyyhin pois kirjastani.
\par 34 Mene nyt ja johdata kansa siihen paikkaan, josta minä olen sinulle puhunut; katso, minun enkelini käy sinun edelläsi. Mutta kostoni päivänä minä kostan heille heidän rikoksensa."
\par 35 Niin Herra rankaisi kansaa, koska he olivat teettäneet vasikan, jonka Aaron teki.

\chapter{33}

\par 1 Sitten Herra sanoi Moosekselle: "Lähde täältä ja vaella, sinä ja kansa, jonka olet johdattanut Egyptin maasta, siihen maahan, jonka minä olen vannoen luvannut Aabrahamille, Iisakille ja Jaakobille, sanoen: 'Sinun jälkeläisillesi minä annan sen'.
\par 2 Ja minä lähetän enkelin sinun edelläsi ja karkoitan pois kanaanilaiset, amorilaiset, heettiläiset, perissiläiset, hivviläiset ja jebusilaiset,
\par 3 että tulisit siihen maahan, joka vuotaa maitoa ja mettä. Sillä minä en itse vaella sinun kanssasi, koska olet niskurikansa, etten minä sinua tiellä hukuttaisi."
\par 4 Kun kansa kuuli tämän kovan puheen, tulivat he murheellisiksi, eikä yksikään pukenut koristuksiaan yllensä.
\par 5 Ja Herra sanoi Moosekselle: "Sano israelilaisille: Te olette niskurikansa. Jos minä silmänräpäyksenkään vaeltaisin sinun keskelläsi, minä hukuttaisin sinut. Riisu nyt koristuksesi yltäsi, niin minä ajattelen, mitä sinulle tekisin."
\par 6 Niin israelilaiset riisuivat koristuksensa ja olivat Hoorebin vuoren luota lähtien ilman niitä.
\par 7 Mutta Mooses otti majan ja pystytti sen leirin ulkopuolelle, jonkun matkan päähän leiristä, ja kutsui sen ilmestysmajaksi; ja jokaisen, jolla oli kysyttävää Herralta, oli mentävä ilmestysmajalle, leirin ulkopuolelle.
\par 8 Ja kun Mooses lähti majalle, nousi koko kansa, ja kukin asettui majansa ovelle ja katseli Mooseksen jälkeen, kunnes hän oli mennyt majaan.
\par 9 Ja aina kun Mooses meni majaan, laskeutui pilvenpatsas ja seisahtui majan ovelle; ja Herra puhutteli Moosesta.
\par 10 Ja kaikki kansa näki pilvenpatsaan seisovan majan ovella; niin kaikki kansa nousi, ja he kumartuivat itsekukin majansa ovella.
\par 11 Ja Herra puhutteli Moosesta kasvoista kasvoihin, niinkuin mies puhuttelee toista. Sitten Mooses palasi takaisin leiriin; mutta Joosua, Nuunin poika, hänen apumiehensä ja palvelijansa, ei poistunut majasta.
\par 12 Ja Mooses sanoi Herralle: "Katso, sinä sanot minulle: 'Johdata tämä kansa sinne', mutta et ole ilmoittanut minulle, kenen sinä lähetät minun kanssani. Ja kuitenkin sinä sanoit: 'Minä tunnen sinut nimeltäsi, ja sinä olet myös saanut armon minun silmieni edessä'.
\par 13 Jos siis olen saanut armon sinun silmiesi edessä, niin ilmoita minulle tiesi, että tulisin tuntemaan sinut ja tietäisin saaneeni armon sinun silmiesi edessä; ja katso: tämä kansa on sinun kansasi."
\par 14 Hän sanoi: "Pitäisikö minun kasvojeni käymän sinun kanssasi ja minun viemän sinut lepoon?"
\par 15 Hän vastasi hänelle: "Elleivät sinun kasvosi käy meidän kanssamme, niin älä johdata meitä täältä pois.
\par 16 Sillä mistä muutoin tiedetään, että minä olen saanut armon sinun silmiesi edessä, minä ja sinun kansasi, ellei siitä, että sinä käyt meidän kanssamme, niin että me, minä ja sinun kansasi, olemme erikoiset kaikkien kansojen joukossa, jotka maan päällä ovat?"
\par 17 Herra vastasi Moosekselle: "Mitä sinä nyt pyydät, sen minä myös teen; sillä sinä olet saanut armon minun silmieni edessä, ja minä tunnen sinut nimeltäsi".
\par 18 Silloin hän sanoi: "Anna siis minun nähdä sinun kirkkautesi".
\par 19 Hän vastasi: "Minä annan kaiken ihanuuteni käydä sinun ohitsesi ja huudan nimen 'Herra' sinun edessäsi. Ja minä olen armollinen, kenelle olen armollinen, armahdan, ketä armahdan".
\par 20 Ja hän sanoi vielä: "Sinä et voi nähdä minun kasvojani; sillä ei kukaan, joka näkee minut, jää eloon".
\par 21 Sitten Herra sanoi: "Katso, tässä on paikka minun läheisyydessäni; astu tuohon kalliolle.
\par 22 Ja kun minun kirkkauteni kulkee ohitse, asetan minä sinut kallion rotkoon ja peitän sinut kädelläni, kunnes olen kulkenut ohi.
\par 23 Kun minä sitten siirrän pois käteni, näet sinä minun selkäpuoleni; mutta minun kasvojani ei voi kenkään katsoa."

\chapter{34}

\par 1 Ja Herra sanoi Moosekselle: "Veistä itsellesi kaksi kivitaulua, entisten kaltaista, niin minä kirjoitan niihin tauluihin ne sanat, jotka olivat entisissä tauluissa, jotka sinä murskasit.
\par 2 Ole siis huomenaamuksi valmis; ja nouse huomenaamuna Siinain vuorelle ja seiso siellä minua vastassa vuoren huipulla.
\par 3 Mutta älköön kukaan nousko sinne sinun kanssasi, älköönkä ketään näkykö koko vuorella; lampaitakaan ja karjaa älköön käykö laitumella vuoren vaiheilla."
\par 4 Ja Mooses veisti kaksi kivitaulua, entisten kaltaista. Ja varhain seuraavana aamuna Mooses nousi Siinain vuorelle, niinkuin Herra oli häntä käskenyt, ja otti ne kaksi kivitaulua käteensä.
\par 5 Niin Herra astui alas pilvessä, ja Mooses asettui siellä hänen läheisyyteensä ja huusi Herran nimeä.
\par 6 Ja Herra kulki hänen ohitsensa ja huusi: "Herra, Herra on laupias ja armahtavainen Jumala, pitkämielinen ja suuri armossa ja uskollisuudessa,
\par 7 joka pysyy armollisena tuhansille, joka antaa anteeksi pahat teot, rikokset ja synnit, mutta ei kuitenkaan jätä rankaisematta, vaan kostaa isien pahat teot lapsille ja lasten lapsille kolmanteen ja neljänteen polveen".
\par 8 Niin Mooses kumartui nopeasti maahan ja rukoili ja sanoi:
\par 9 "Herra, jos olen saanut armon sinun silmiesi edessä, niin käyköön Herra meidän keskellämme. Sillä tämä on tosin niskurikansa, mutta anna anteeksi meidän pahat tekomme ja syntimme ja ota meidät perintöosaksesi."
\par 10 Hän vastasi: "Katso, minä teen liiton, minä teen kaiken sinun kansasi nähden ihmeellisiä tekoja, joiden kaltaisia ei ole tehty yhdessäkään maassa, ei minkään kansan keskuudessa. Niin koko kansa, jonka keskellä sinä olet, on näkevä Herran teot, sillä peljättävää on se, mitä minä sinulle teen.
\par 11 Noudata, mitä minä tänä päivänä käsken sinun noudattaa. Katso, minä karkoitan sinun tieltäsi amorilaiset, kanaanilaiset, heettiläiset, perissiläiset, hivviläiset ja jebusilaiset.
\par 12 Kavahda, ettet tee liittoa sen maan asukasten kanssa, johon tulet, etteivät he tulisi teidän keskuudessanne ansaksi;
\par 13 vaan kukistakaa heidän alttarinsa ja murskatkaa heidän patsaansa ja hakatkaa maahan heidän asera-karsikkonsa.
\par 14 Älä kumarra muuta jumalaa; sillä Herra on nimeltänsä Kiivas, hän on kiivas Jumala.
\par 15 Älä siis tee liittoa maan asukasten kanssa, ettet, kun he kulkevat haureudessa jumaliensa jäljessä ja uhraavat jumalillensa ja kutsuvat sinua, sinä söisi heidän uhristaan,
\par 16 ja etteivät heidän tyttärensä, kun sinä otat heitä pojillesi vaimoiksi ja kun he kulkevat haureudessa jumaliensa jäljessä, viettelisi sinun poikiasikin haureudessa kulkemaan heidän jumaliensa jäljessä.
\par 17 Älä tee itsellesi valettuja jumalankuvia.
\par 18 Vietä happamattoman leivän juhlaa: seitsemänä päivänä syö happamatonta leipää, niinkuin minä olen sinua käskenyt, määrättynä aikana, aabib-kuussa, sillä aabib-kuussa sinä olet lähtenyt Egyptistä.
\par 19 Kaikki, mikä avaa äidinkohdun, on minun; samoin myös kaikki sinun karjasi urospuolet, raavaittesi ja lampaittesi ensiksisynnyttämät.
\par 20 Mutta aasin ensiksisynnyttämä lunasta lampaalla, mutta jos et sitä lunasta, niin taita siltä niska. Jokainen esikoinen pojistasi lunasta. Ja tyhjin käsin älköön tultako minun kasvojeni eteen.
\par 21 Kuusi päivää tee työtä, mutta lepää seitsemäs päivä; kyntö- ja elonleikkuuaikanakin sinun on levättävä.
\par 22 Ja vietä viikkojuhla, kun leikkaat nisusi uutisen, ja korjuujuhla vuoden vaihteessa.
\par 23 Kolme kertaa vuodessa kaikki sinun miesväkesi tulkoon Herran, sinun Herrasi, Israelin Jumalan, kasvojen eteen.
\par 24 Sillä minä karkoitan kansat sinun tieltäsi ja laajennan sinun alueesi; eikä kukaan ole himoitseva sinun maatasi, kun sinä kolme kertaa vuodessa vaellat tullaksesi Herran, sinun Jumalasi, kasvojen eteen.
\par 25 Älä uhraa minun teurasuhrini verta happamen leivän ohella. Ja pääsiäisjuhlan uhrista älköön mitään jääkö yli yön seuraavaan aamuun.
\par 26 Parhaat maasi uutisesta tuo Herran, sinun Jumalasi, huoneeseen. Älä keitä vohlaa emänsä maidossa."
\par 27 Ja Herra sanoi Moosekselle: "Kirjoita itsellesi nämä sanat, sillä näiden sanojen mukaisesti minä olen tehnyt liiton sinun ja Israelin kanssa".
\par 28 Ja hän oli siellä Herran tykönä neljäkymmentä päivää ja neljäkymmentä yötä syömättä ja juomatta. Ja hän kirjoitti tauluihin liiton sanat, ne kymmenen sanaa.
\par 29 Ja kun Mooses astui alas Siinain vuorelta ja hänellä vuorelta alas astuessaan oli kädessänsä kaksi laintaulua, ei hän tiennyt, että hänen kasvojensa iho oli tullut säteileväksi hänen puhuessaan Herran kanssa.
\par 30 Ja kun Aaron ja kaikki israelilaiset näkivät Mooseksen kasvojen ihon säteilevän, pelkäsivät he lähestyä häntä.
\par 31 Mutta Mooses huusi heitä; niin Aaron ja kaikki kansan päämiehet kääntyivät takaisin hänen luokseen, ja Mooses puhui heille.
\par 32 Sitten kaikki israelilaiset lähestyivät häntä, ja hän käski heidän noudattaa kaikkea, mitä Herra oli puhunut hänelle Siinain vuorella.
\par 33 Ja kun Mooses oli lakannut puhumasta heidän kanssaan, pani hän peitteen kasvoillensa.
\par 34 Mutta niin usein kuin hän meni Herran eteen puhuttelemaan häntä, poisti hän peitteen, siksi kunnes tuli ulos. Ja tultuaan ulos hän puhui israelilaisille, mitä hänen oli käsketty puhua.
\par 35 Ja israelilaiset näkivät joka kerta Mooseksen kasvojen ihon säteilevän; ja Mooses veti peitteen kasvoillensa, siksi kunnes hän jälleen meni puhuttelemaan häntä.

\chapter{35}

\par 1 Ja Mooses kokosi kaiken Israelin kansan ja sanoi heille: "Näin on Herra käskenyt teidän tehdä:
\par 2 Kuusi päivää tehtäköön työtä, mutta seitsemäs päivä olkoon teille pyhä sapatti, levon päivä, Herralle pyhitetty. Kuka ikinä silloin työtä tekee, se surmattakoon.
\par 3 Älkää sytyttäkö tulta sapatin päivänä, missä asuttekin."
\par 4 Ja Mooses sanoi koko Israelin kansalle näin: "Näin on Herra käskenyt ja sanonut:
\par 5 Ottakaa siitä, mitä teillä on, anti Herralle; jokainen, jonka sydän on siihen altis, tuokoon anniksi Herralle kultaa, hopeata ja vaskea,
\par 6 punasinisiä, purppuranpunaisia ja helakanpunaisia lankoja ja valkoisia pellavalankoja sekä vuohenkarvoja,
\par 7 punaisia oinaannahkoja, sireeninnahkoja, akasiapuuta,
\par 8 öljyä seitsenhaaraista lamppua varten, hajuaineita voiteluöljyä ja hyvänhajuista suitsutusta varten,
\par 9 onyks-kiviä ja muita jalokiviä kasukkaa ja rintakilpeä varten.
\par 10 Ja kaikki taidolliset teidän keskuudessanne tulkoot ja tehkööt kaikki, mitä Herra on käskenyt:
\par 11 asumuksen telttoineen ja peitteineen, hakasineen, lautoineen, poikkitankoineen, pylväineen ja jalustoineen,
\par 12 arkin korentoineen, armoistuimen ja sitä verhoavan esiripun,
\par 13 pöydän korentoineen ja kaikkine kaluineen sekä näkyleivät,
\par 14 seitsenhaaraisen lampun kaluineen ja lamppuineen, öljyä seitsenhaaraista lamppua varten,
\par 15 suitsutusalttarin korentoineen, voiteluöljyn ja hyvänhajuisen suitsukkeen, oviuutimen asumuksen ovelle,
\par 16 polttouhrialttarin vaskiristikkoineen, korentoineen ja kaikkine kaluineen, altaan jalustoineen,
\par 17 esipihan ympärysverhot pylväineen ja jalustoineen sekä esipihan porttiin uutimen,
\par 18 asumuksen vaarnat ja esipihan vaarnat köysineen,
\par 19 virkapuvut pyhäkköpalvelusta varten ja pappi Aaronin muut pyhät vaatteet sekä hänen poikiensa pappispuvut."
\par 20 Ja kaikki israelilaisten seurakunta meni pois Mooseksen luota.
\par 21 Sitten he tulivat takaisin, jokainen, jonka sydän häntä siihen vaati, ja jokainen, jonka henki oli siihen altis, ja toivat antinsa Herralle ilmestysmajan valmistamista varten ja kaikkea siinä vietettävää jumalanpalvelusta varten ja pyhiä vaatteita varten.
\par 22 He tulivat, sekä miehet että naiset, jokainen, jonka sydän oli siihen altis, ja toivat solkia, korvarenkaita, sormuksia ja kaulakoristeita, kaikkinaisia kultakaluja - jokainen, joka toi Herralle kultaa heilutusuhriksi.
\par 23 Ja jokainen, jolla oli punasinisiä, purppuranpunaisia ja helakanpunaisia lankoja, valkoisia pellavalankoja, vuohen karvoja, punaisia oinaannahkoja tai sireeninnahkoja, toi niitä.
\par 24 Ja jokainen, joka voi antaa anniksi hopeata ja vaskea, toi annin Herralle. Ja jokainen, jolla oli akasiapuuta, toi sitä kaikkinaisten töiden valmistamista varten.
\par 25 Ja kaikki taitavat naiset kehräsivät omin käsin ja toivat kehräämänsä punasiniset, purppuranpunaiset ja helakanpunaiset langat ja valkoiset pellavalangat;
\par 26 ja kaikki naiset, joiden taidollinen sydän heitä siihen vaati, kehräsivät vuohenkarvoja.
\par 27 Mutta päämiehet toivat onyks-kiviä ja muita jalokiviä kasukkaa ja rintakilpeä varten,
\par 28 ja hajuaineita ja öljyä seitsenhaaraista lamppua varten sekä voiteluöljyksi ja hyvänhajuiseksi suitsutukseksi.
\par 29 Kaikki israelilaiset, miehet ja naiset, joiden sydän oli altis ja vaati heitä tuomaan jotakin niitä töitä varten, joita Herra Mooseksen kautta oli käskenyt tehdä, toivat vapaaehtoisen lahjan Herralle.
\par 30 Ja Mooses sanoi israelilaisille: "Katsokaa, Herra on nimeltään kutsunut Besalelin, Uurin pojan, Huurin pojanpojan, Juudan sukukunnasta,
\par 31 ja on täyttänyt hänet Jumalan hengellä, taidollisuudella, ymmärryksellä, tiedolla ja kaikkinaisella kätevyydellä
\par 32 sommittelemaan taidokkaita teoksia ja valmistamaan niitä kullasta, hopeasta ja vaskesta,
\par 33 hiomaan ja kiinnittämään kiviä ja veistämään puuta, tekemään kaikkinaisia taidokkaita töitä.
\par 34 Hänelle ja Oholiabille, Ahisamakin pojalle, Daanin sukukunnasta, hän on myös antanut kyvyn opettaa muita.
\par 35 Nämä hän on täyttänyt taidollisuudella tekemään kaikkinaisia töitä, jommoisia tekevät seppä, kuvakudosten tekijä, kirjokankaiden kutoja, joka valmistaa punasinisiä, purppuranpunaisia ja helakanpunaisia kankaita ja valkoisia pellavakankaita, sekä muu kankuri, ne, jotka valmistavat kaikkinaisia töitä ja sommittelevat taidokkaita teoksia."

\chapter{36}

\par 1 "Ja Besalel ja Oholiab ynnä kaikki taidolliset, joille Herra on antanut taidollisuuden ja ymmärryksen tietää, miten mikin pyhäkössä tehtävä työ on suoritettava, tehkööt kaikissa kohdin, niinkuin Herra on käskenyt."
\par 2 Niin Mooses kutsui luoksensa Besalelin ja Oholiabin ja kaikki taidolliset, joiden sydämeen Herra oli antanut taidollisuuden, kaikki, joiden sydän vaati heitä ryhtymään työhön suorittaaksensa sen.
\par 3 Ja he ottivat Moosekselta kaikki annit, jotka israelilaiset olivat tuoneet pyhäkössä tehtävien töiden suorittamista varten. Mutta he toivat hänelle vielä joka aamu vapaaehtoisia lahjoja.
\par 4 Niin tulivat kaikki taidolliset, jotka tekivät kaikkinaista työtä pyhäkössä, mies toisensa jälkeen, siitä työstä, jota kukin oli tekemässä,
\par 5 ja sanoivat Moosekselle näin: "Kansa tuo enemmän, kuin mitä tarvitaan sen työn valmistamiseksi, jonka Herra on käskenyt tehdä".
\par 6 Niin Mooses käski kuuluttaa leirissä ja sanoa: "Älköön kukaan, mies tai nainen, enää valmistako mitään anniksi pyhäkköä varten". Ja niin kansa lakkasi tuomasta lahjoja.
\par 7 Sillä kaikkea työtä varten, mikä oli tehtävä, oli jo koottu riittävästi, jopa liiaksikin.
\par 8 Ja niin kaikki taidolliset, jotka siinä työssä olivat, tekivät asumuksen kymmenestä telttakankaan kaistasta, jotka olivat valmistetut kerratuista valkoisista pellavalangoista ja punasinisistä, purppuranpunaisista ja helakanpunaisista langoista; ja niihin tehtiin taidokkaasti kudottuja kerubeja.
\par 9 Kunkin kaistan pituus oli kaksikymmentä kahdeksan kyynärää ja leveys neljä kyynärää; kaikilla kaistoilla oli sama mitta.
\par 10 Ja viisi kaistaa yhdistettiin toisiinsa, ja samoin toiset viisi kaistaa yhdistettiin toisiinsa.
\par 11 Sitten tehtiin silmukat punasinisestä langasta ensimmäisen kaistan reunaan, yhdistetyn kappaleen laitaan, ja samoin toisen yhdistetyn kappaleen viimeisen kaistan reunaan.
\par 12 Viisikymmentä silmukkaa tehtiin ensimmäiseen kaistaan ja viisikymmentä silmukkaa vastaavan kaistan laitaan, toiseen yhdistettyyn kappaleeseen, niin että silmukat olivat kohdakkain.
\par 13 Ja sitten tehtiin viisikymmentä kultahakasta, ja kaistat yhdistettiin toisiinsa näillä hakasilla, niin että siitä tuli yhtenäinen asumus.
\par 14 Vielä tehtiin vuohen karvoista telttakankaan kaistoja asumusta suojaavaksi teltaksi; niitä kaistoja tehtiin yksitoista.
\par 15 Kunkin kaistan pituus oli kolmekymmentä kyynärää ja leveys neljä kyynärää; niillä yhdellätoista kaistalla oli sama mitta.
\par 16 Kaistat liitettiin yhteen, viisi kaistaa erikseen ja kuusi kaistaa erikseen.
\par 17 Ja viisikymmentä silmukkaa tehtiin toisen yhdistetyn kappaleen viimeisen kaistan reunaan ja viisikymmentä silmukkaa toisen yhdistetyn kappaleen vastaavan kaistan reunaan.
\par 18 Sitten tehtiin viisikymmentä vaskihakasta, joilla teltta liitettiin yhteen, niin että siitä tuli yhtenäinen.
\par 19 Vielä tehtiin teltalle peite punaisista oinaannahoista ja sen päälle vielä toinen peite sireeninnahoista.
\par 20 Asumuksen laudat tehtiin akasiapuusta, pystyyn asetettaviksi.
\par 21 Jokainen lauta oli kymmentä kyynärää pitkä ja puoltatoista kyynärää leveä.
\par 22 Jokaisessa laudassa oli kaksi tappia, jotka olivat poikkilistalla yhdistetyt keskenään; näin tehtiin kaikki asumuksen laudat.
\par 23 Ja asumuksen lautoja tehtiin kaksikymmentä lautaa eteläpuolta varten.
\par 24 Ja tehtiin neljäkymmentä hopeajalustaa kahdenkymmenen laudan alle, aina kaksi jalustaa kunkin laudan alle, sen kahta tappia varten.
\par 25 Samoin asumuksen toista sivua, pohjoispuolta, varten tehtiin kaksikymmentä lautaa,
\par 26 ja neljäkymmentä hopeajalustaa, aina kaksi jalustaa kunkin laudan alle.
\par 27 Mutta asumuksen takasivua, länsipuolta, varten tehtiin kuusi lautaa.
\par 28 Ja kaksi lautaa tehtiin asumuksen peränurkkia varten,
\par 29 ja ne olivat yhteenliitettyjä kaksoislautoja ja alhaalta alkaen kiinni toisissaan ylös saakka, ensimmäiseen renkaaseen asti; näin ne molemmat tehtiin kumpaakin nurkkaa varten.
\par 30 Näin tuli olemaan yhteensä kahdeksan lautaa ja niihin kuusitoista hopeajalustaa, aina kaksi jalustaa kunkin laudan alla.
\par 31 Sitten tehtiin myös viisi poikkitankoa akasiapuusta asumuksen toisen sivun lautoja varten
\par 32 ja viisi poikkitankoa asumuksen toisen sivun lautoja varten ja viisi poikkitankoa asumuksen takasivun, länsipuolen, lautoja varten.
\par 33 Ja keskimmäinen poikkitanko tehtiin kulkemaan lautojen keskikohdalta, reunasta reunaan.
\par 34 Ja laudat päällystettiin kullalla, ja niiden renkaat tehtiin kullasta poikkitankojen pitimiksi, ja poikkitangotkin päällystettiin kullalla.
\par 35 Vielä tehtiin esirippu punasinisistä, purppuranpunaisista ja helakanpunaisista langoista ja kerratuista valkoisista pellavalangoista; ja siihen tehtiin taidokkaasti kudottuja kerubeja.
\par 36 Ja siihen tehtiin neljä pylvästä akasiapuusta, ja ne päällystettiin kullalla, ja niiden koukut olivat kultaa; ja niitä varten valettiin neljä hopeajalustaa.
\par 37 Sitten tehtiin myös teltan oveen uudin, kirjaellen kudottu punasinisistä, purppuranpunaisista ja helakanpunaisista langoista ja kerratuista valkoisista pellavalangoista,
\par 38 ja siihen viisi pylvästä koukkuineen; ja niiden päät ja niiden koristepienat päällystettiin kullalla, ja niiden viisi jalustaa tehtiin vaskesta.

\chapter{37}

\par 1 Ja Besalel teki arkin akasiapuusta, puolenkolmatta kyynärän pituisen, puolentoista kyynärän levyisen ja puolentoista kyynärän korkuisen.
\par 2 Ja hän päällysti sen puhtaalla kullalla sisältä ja ulkoa ja teki siihen kultareunuksen yltympäri.
\par 3 Ja hän valoi siihen neljä kultarengasta, sen neljään kulmaan, niin että kaksi rengasta tuli sen toiselle puolelle ja kaksi rengasta sen toiselle puolelle.
\par 4 Ja hän teki korennot akasiapuusta ja päällysti ne kullalla.
\par 5 Ja hän pisti arkin sivuilla oleviin renkaisiin korennot, joilla arkki oli kannettava.
\par 6 Ja hän teki armoistuimen puhtaasta kullasta, puoltakolmatta kyynärää pitkän ja puoltatoista kyynärää leveän.
\par 7 Ja hän teki kaksi kultakerubia, kohotakoista tekoa, armoistuimen molempiin päihin,
\par 8 toisen kerubin toiseen päähän ja toisen kerubin toiseen päähän. Armoistuimesta kohoaviksi hän teki kerubit, sen kumpaankin päähän.
\par 9 Ja kerubit levittivät siipensä ylöspäin, niin että ne peittivät siivillänsä armoistuimen, ja niiden kasvot olivat vastakkain; kerubien kasvot olivat käännetyt armoistuinta kohti.
\par 10 Hän teki myös pöydän akasiapuusta, kahta kyynärää pitkän, kyynärää leveän ja puoltatoista kyynärää korkean.
\par 11 Ja hän päällysti sen puhtaalla kullalla ja teki siihen kultareunuksen yltympäri.
\par 12 Ja hän teki sen ympäri kämmenen korkuisen listan, ja listan ympäri kultareunuksen.
\par 13 Ja hän valoi siihen neljä kultarengasta ja kiinnitti renkaat neljään kulmaan, sen neljän jalan kohdalle.
\par 14 Renkaat olivat juuri listan alla niiden korentojen pitiminä, joilla pöytä oli kannettava.
\par 15 Ja hän teki korennot akasiapuusta ja päällysti ne kullalla; ja niillä oli pöytä kannettava.
\par 16 Hän teki puhtaasta kullasta pöydällä pidettävät astiat: vadit ja kupit, maljat ja kannut, joista juomauhri vuodatetaan.
\par 17 Hän teki myös seitsenhaaraisen lampun puhtaasta kullasta. Kohotakoista tekoa hän teki seitsenhaaraisen lampun jalkoineen ja varsineen; sen kukkakuvut, nuput kukkalehtineen, olivat samaa kappaletta kuin se.
\par 18 Kuusi haaraa lähti lampun sivuista, kolme haaraa toisesta sivusta ja kolme haaraa toisesta sivusta.
\par 19 Toisessa haarassa oli kolme mantelinkukan muotoista kukkakupua, nuppua kukkalehtineen, ja toisessa haarassa samoin kolme mantelinkukan muotoista kukkakupua, nuppua kukkalehtineen; näin oli jokaisessa kuudessa haarassa, jotka lampusta lähtivät.
\par 20 Mutta itse seitsenhaaraisessa lampussa oli neljä mantelinkukan muotoista kukkakupua, nuppua kukkalehtineen.
\par 21 Yksi nuppu oli aina jokaisen haaraparin alla niistä kuudesta haarasta, jotka lampusta lähtivät.
\par 22 Nuput ja haarat olivat samaa kappaletta kuin se; se oli kokonaan samaa kohotakoista tekoa, puhdasta kultaa.
\par 23 Ja hän teki siihen seitsemän lamppua ja siihen kuuluvat lamppusakset ja karstakupit puhtaasta kullasta.
\par 24 Yhdestä talentista puhdasta kultaa hän teki sekä sen että kaikki sen kalut.
\par 25 Hän teki myös suitsutusalttarin akasiapuusta, kyynärän pituisen ja kyynärän levyisen, siis neliskulmaisen, ja kahta kyynärää korkean; sen sarvet olivat samaa kappaletta kuin se.
\par 26 Ja hän päällysti sen puhtaalla kullalla, sekä sen levyn että sivut ympärinsä ja sen sarvet; ja hän teki kultareunuksen sen ympäri.
\par 27 Ja hän teki siihen kaksi kultarengasta reunuksen alle molemmin puolin, molempiin sivuihin, niiden korentojen pitimiksi, joilla alttari oli kannettava.
\par 28 Ja korennotkin hän teki akasiapuusta ja päällysti ne kullalla.
\par 29 Ja hän teki pyhän voiteluöljyn ja puhtaan, hyvänhajuisen suitsukkeen, jollaista voiteensekoittaja valmistaa.

\chapter{38}

\par 1 Ja hän teki polttouhrialttarin akasiapuusta, neliskulmaisen, viittä kyynärää pitkän, viittä kyynärää leveän ja kolmea kyynärää korkean.
\par 2 Ja hän teki siihen sarvet, sen neljään kulmaan, niin että sarvet olivat samaa kappaletta kuin se. Ja hän päällysti sen vaskella.
\par 3 Ja hän teki kaikki alttarin kalut: kattilat, lapiot, maljat, haarukat ja hiilipannut. Kaikki sen kalut hän teki vaskesta.
\par 4 Ja hän teki alttariin verkonkaltaisen ristikkokehyksen vaskesta, alttarin välireunuksen alle, maahan kiinni, niin että se ulottui puolitiehen alttaria.
\par 5 Ja hän valoi neljä rengasta vaskiristikon neljään kulmaan, korentojen pitimiksi.
\par 6 Ja korennot hän teki akasiapuusta ja päällysti ne vaskella.
\par 7 Ja hän pisti alttarin sivuilla oleviin renkaisiin korennot, joilla se oli kannettava. Ja hän teki sen laudoista, ontoksi.
\par 8 Ja hän teki vaskialtaan vaskijalustoineen niiden naisten kuvastimista, jotka toimittivat palvelusta ilmestysmajan ovella.
\par 9 Sitten hän teki esipihan. Etelän puolella olivat esipihan ympärysverhot kerratuista valkoisista pellavalangoista, sadan kyynärän pituiset;
\par 10 niiden pylväitä oli kaksikymmentä ja näiden vaskijalustoja kaksikymmentä, mutta pylväiden koukut ja niiden koristepienat olivat hopeata.
\par 11 Samoin olivat ympärysverhot myös pohjoisen puolella sadan kyynärän pituiset; niiden pylväitä oli kaksikymmentä ja näiden vaskijalustoja kaksikymmentä, mutta pylväiden koukut ja niiden koristepienat olivat hopeata.
\par 12 Mutta lännen puolella olivat ympärysverhot viidenkymmenen kyynärän pituiset; niiden pylväitä oli kymmenen ja näiden jalustoja kymmenen, ja pylväiden koukut ja niiden koristepienat olivat hopeata.
\par 13 Ja etupuolella, itään päin, olivat ympärysverhot viidenkymmenen kyynärän pituiset.
\par 14 Ympärysverhot olivat portin toisella sivulla viidentoista kyynärän pituiset, ja niiden pylväitä oli kolme ja näiden jalustoja kolme.
\par 15 Samoin olivat ympärysverhot toisellakin sivulla, siis esipihan portin kummallakin puolella, viidentoista kyynärän pituiset; niiden pylväitä oli kolme ja näiden jalustoja kolme.
\par 16 Kaikki esipihan ympärysverhot yltympäri olivat kerratuista valkoisista pellavalangoista,
\par 17 ja pylväiden jalustat olivat vaskea, mutta pylväiden koukut ja niiden koristepienat olivat hopeata, ja niiden päiden päällystys oli hopeata; kaikki esipihan pylväät olivat ympäröidyt hopeaisilla koristepienoilla.
\par 18 Ja esipihan portin uudin oli kirjaellen kudottu punasinisistä, purppuranpunaisista ja helakanpunaisista langoista ja kerratuista valkoisista pellavalangoista, kahdenkymmenen kyynärän pituinen ja niinkuin esipihan ympärysverhotkin viiden kyynärän korkuinen, vaatteen leveyden mukaan;
\par 19 ja sen pylväitä oli neljä ja näiden vaskijalustoja neljä, mutta niiden koukut olivat hopeata, ja niiden päiden päällystys ja niiden koristepienat olivat hopeata.
\par 20 Ja kaikki asumuksen ja ympärillä olevan esipihan vaarnat olivat vaskea.
\par 21 Näin paljon lasketaan menneen asumuksen, lain asumuksen, kustannuksiin, joka lasku tehtiin Mooseksen käskyn mukaan leeviläisten toimesta, Iitamarin, pappi Aaronin pojan, johdolla.
\par 22 Ja Besalel, Uurin poika, Huurin pojanpoika, Juudan sukukunnasta, oli tehnyt kaikki, mistä Herra oli Moosekselle käskyn antanut,
\par 23 ja hänellä oli apulaisena Oholiab, Ahisamakin poika, Daanin sukukunnasta, seppä ja kuvakudosten tekijä; tämä valmisti kirjokankaita punasinisistä, purppuranpunaisista ja helakanpunaisista langoista ja valkoisista pellavalangoista.
\par 24 Kultaa, joka käytettiin siihen työhön, kaikkeen työhön pyhäkössä, oli heilutusuhrina tuotu kaikkiaan kaksikymmentä yhdeksän talenttia, seitsemänsataa kolmekymmentä sekeliä, pyhäkkösekelin painon mukaan.
\par 25 Ja hopeata, joka kansalta katselmusta pidettäessä kerääntyi, oli sata talenttia ja tuhat seitsemänsataa seitsemänkymmentä viisi sekeliä, pyhäkkösekelin painon mukaan,
\par 26 puolikas, puoli sekeliä, henkeä kohti, pyhäkkösekelin painon mukaan, jokaiselta katselmuksessa olleelta kaksikymmenvuotiaalta ja sitä vanhemmalta, kaikkiaan kuudeltasadalta kolmelta tuhannelta viideltäsadalta viideltäkymmeneltä.
\par 27 Ja ne sata talenttia hopeata käytettiin pyhäkön jalustain ja esiripun jalustain valamiseen: sata talenttia sataan jalustaan, talentti jalustaa kohti.
\par 28 Ja niistä tuhannesta seitsemästäsadasta seitsemästäkymmenestä viidestä sekelistä tehtiin koukut pylväisiin ja päällystettiin hopealla niiden päät ja tehtiin niihin koristepienat.
\par 29 Ja heilutusuhri-vaskea oli seitsemänkymmentä talenttia, kaksituhatta neljäsataa sekeliä.
\par 30 Siitä tehtiin jalustat ilmestysmajan oveen, vaskialttari ja siihen kuuluva vaskinen ristikkokehys ja kaikki alttarin kalut
\par 31 sekä jalustat ympärillä olevaan esipihaan ja jalustat esipihan porttiin ja kaikki asumuksen vaarnat ja kaikki vaarnat ympärillä olevaan esipihaan.

\chapter{39}

\par 1 Ja punasinisistä, purppuranpunaisista ja helakanpunaisista langoista tehtiin virkapuvut pyhäkköpalvelusta varten; ja samoin tehtiin muut pyhät vaatteet Aaronille, niinkuin Herra oli Moosekselle käskyn antanut.
\par 2 Kasukka tehtiin kullasta sekä punasinisistä, purppuranpunaisista ja helakanpunaisista langoista ja kerratuista valkoisista pellavalangoista.
\par 3 Kultalevyt taottiin ohuiksi ja leikattiin säikeiksi, taidokkaasti kudottaviksi punasinisten, purppuranpunaisten ja helakanpunaisten lankojen ja valkoisten pellavalankojen sekaan.
\par 4 Siihen tehtiin yhdistettävät olkakappaleet; molemmista päistänsä se kiinnitettiin niihin.
\par 5 Ja vyö, joka oli oleva kasukassa sen kiinnittämiseksi, oli tehty samasta kappaleesta kuin se, samalla tavalla kullasta sekä punasinisistä, purppuranpunaisista ja helakanpunaisista langoista ja kerratuista valkoisista pellavalangoista, niinkuin Herra oli Moosekselle käskyn antanut.
\par 6 Sitten valmistettiin kultapalmikoimilla kehystetyt onyks-kivet, joihin oli kaiverrettu Israelin poikien nimet, niinkuin kaiverretaan sinettisormuksia.
\par 7 Ja ne pantiin kasukan olkakappaleihin, että ne kivet johdattaisivat muistoon israelilaiset, niinkuin Herra oli Moosekselle käskyn antanut.
\par 8 Rintakilpi tehtiin taidokkaasti kutomalla, samalla tavalla kuin kasukkakin oli tehty, kullasta sekä punasinisistä, purppuranpunaisista ja helakanpunaisista langoista ja kerratuista valkoisista pellavalangoista.
\par 9 Se oli neliskulmainen ja taskun muotoon tehty; rintakilpi tehtiin vaaksan pituinen ja vaaksan levyinen, taskun muotoon.
\par 10 Ja sen pintaan kiinnitettiin kiviä yliyltään, neljään riviin: ensimmäiseen riviin karneoli, topaasi ja smaragdi;
\par 11 toiseen riviin rubiini, safiiri ja jaspis;
\par 12 kolmanteen riviin hyasintti, akaatti ja ametisti;
\par 13 ja neljänteen riviin krysoliitti, onyks ja berylli. Kultapalmikoimilla kehystettyinä ne kiinnitettiin paikoilleen.
\par 14 Kiviä oli Israelin poikien nimien mukaan kaksitoista, yksi kutakin nimeä kohti; kussakin kivessä oli yksi kahdentoista sukukunnan nimistä, kaiverrettuna samalla tavalla, kuin kaiverretaan sinettisormuksia.
\par 15 Ja rintakilpeen tehtiin puhtaasta kullasta käädyt, punotut, niinkuin punonnaista tehdään.
\par 16 Vielä tehtiin kaksi kultapalmikoimaa ja kaksi kultarengasta, ja molemmat renkaat kiinnitettiin rintakilven molempiin yläkulmiin.
\par 17 Ja molemmat kultapunonnaiset kiinnitettiin kahteen renkaaseen rintakilven yläkulmiin.
\par 18 Ja molempien punonnaisten toiset kaksi päätä kiinnitettiin kahteen palmikoimaan, ja nämä kiinnitettiin kasukan olkakappaleihin, sen etupuolelle.
\par 19 Vielä tehtiin kaksi kultarengasta, ja ne pantiin rintakilven molempiin alakulmiin, sen sisäpuoliseen, kasukkaa vasten olevaan reunaan.
\par 20 Ja vieläkin tehtiin kaksi kultarengasta, ja ne kiinnitettiin kasukan molempiin olkakappaleihin, niiden alareunaan, etupuolelle, sauman kohdalle, kasukan vyön yläpuolelle.
\par 21 Sitten rintakilpi solmittiin renkaistaan punasinisellä nauhalla kasukan renkaisiin, niin että se oli kasukan vyön yläpuolella, eikä siis rintakilpi irtautunut kasukasta - kaikki, niinkuin Herra oli Moosekselle käskyn antanut.
\par 22 Kasukan viitta tehtiin kokonaan kutomalla punasinisistä langoista.
\par 23 Ja pääntie viitan keskellä oli niinkuin haarniskan aukko, ja pääntie ympäröitiin päärmeellä, ettei se repeäisi.
\par 24 Ja viitan helmaan tehtiin granaattiomenia punasinisistä, purppuranpunaisista ja helakanpunaisista kerratuista langoista.
\par 25 Sitten tehtiin tiukuja puhtaasta kullasta, ja nämä tiuvut kiinnitettiin granaattiomenien väliin viitan helmaan ympärinsä, granaattiomenien väliin:
\par 26 vuorotellen tiuku ja granaattiomena viitan helmaan ympärinsä, käytettäviksi virantoimituksessa, niinkuin Herra oli Moosekselle käskyn antanut.
\par 27 Sitten tehtiin Aaronille ja hänen pojilleen ihokkaat, kudotut valkoisista pellavalangoista,
\par 28 käärelakki valkoisista pellavalangoista, koristepäähineet valkoisista pellavalangoista ja pellavakaatiot kerratuista valkoisista pellavalangoista,
\par 29 ja vihdoin vyö kerratuista valkoisista pellavalangoista ja punasinisistä, purppuranpunaisista ja helakanpunaisista langoista, kirjaellen kudottu, niinkuin Herra oli Moosekselle käskyn antanut.
\par 30 Vielä tehtiin otsakoriste, pyhä otsalehti, puhtaasta kullasta ja siihen piirrettiin, niinkuin sinettisormusta kaiverretaan, kirjoitus: "Herralle pyhitetty".
\par 31 Ja siihen kiinnitettiin punasininen nauha sidottavaksi käärelakin yläreunaan, niinkuin Herra oli Moosekselle käskyn antanut.
\par 32 Ja niin valmistui koko työ: asumus, ilmestysmaja. Ja israelilaiset tekivät kaiken, niinkuin Herra oli Moosekselle käskyn antanut; niin he tekivät.
\par 33 Ja he toivat Mooseksen eteen asumuksen, majan kaikkine kaluineen, hakasineen, lautoineen, poikkitankoineen, pylväineen ja jalustoineen,
\par 34 punaisista oinaannahoista tehdyn peitteen ja sireeninnahoista tehdyn peitteen ja verhona olevan esiripun,
\par 35 lain arkin korentoineen, armoistuimen,
\par 36 pöydän kaikkine kaluineen ja näkyleivät,
\par 37 aitokultaisen seitsenhaaraisen lampun riviin pantavine lamppuineen ja kaikkine kaluineen sekä öljyä seitsenhaaraista lamppua varten,
\par 38 kultaisen alttarin, voiteluöljyn ja hyvänhajuisen suitsukkeen, majan oven uutimen,
\par 39 vaskialttarin vaskisine ristikkokehyksineen, sen korennot ja kaikki kalut, altaan jalustoineen,
\par 40 esipihan ympärysverhot pylväineen ja jalustoineen, esipihan portin uutimen sekä sen köydet ja vaarnat, ja kaikki kalut, joita tarvitaan asumuksen, ilmestysmajan, tehtävissä,
\par 41 virkapuvut pyhäkköpalvelusta varten ja pappi Aaronin muut pyhät vaatteet sekä hänen poikiensa pappispuvut.
\par 42 Aivan niinkuin Herra oli Moosekselle käskyn antanut, niin olivat israelilaiset tehneet kaiken sen työn.
\par 43 Ja Mooses katseli kaikkea työtä, ja katso, he olivat tehneet sen niin, kuin Herra oli käskenyt; niin he olivat tehneet. Silloin Mooses siunasi heidät.

\chapter{40}

\par 1 Ja Herra puhui Moosekselle sanoen:
\par 2 "Ensimmäisen kuukauden ensimmäisenä päivänä pystytä asumus, ilmestysmaja.
\par 3 Ja aseta sinne lain arkki ja ripusta esirippu arkin eteen.
\par 4 Ja tuo sinne pöytä ja lado sille näkyleivät päälletysten; ja tuo sinne seitsenhaarainen lamppu ja nosta sen lamput paikoilleen.
\par 5 Ja sijoita kultainen suitsutusalttari lain arkin eteen ja ripusta uudin majan oveen.
\par 6 Ja polttouhrialttari sijoita asumuksen, ilmestysmajan, oven eteen.
\par 7 Ja allas sijoita ilmestysmajan ja alttarin välille ja kaada siihen vettä.
\par 8 Ja tee esipiha yltympäri ja pane uudin esipihan porttiin.
\par 9 Ota sitten voiteluöljyä ja voitele asumus ja kaikki, mitä siinä on, ja pyhitä se kalustoineen, että se tulisi pyhäksi.
\par 10 Ja voitele polttouhrialttari kaikkine kaluineen ja pyhitä alttari, että alttari tulisi korkeasti-pyhäksi.
\par 11 Voitele myös allas jalustoineen ja pyhitä se.
\par 12 Tuo sitten Aaron poikineen ilmestysmajan ovelle ja pese heidät vedellä.
\par 13 Ja pue Aaron pyhiin vaatteisiin ja voitele hänet ja pyhitä hänet, että hän pappina palvelisi minua.
\par 14 Ja tuo myös hänen poikansa ja pue heidän ylleen ihokkaat.
\par 15 Ja voitele heidät, niinkuin sinä voitelit heidän isänsäkin, että he pappeina palvelisivat minua. Niin tämä on oleva heille voitelu ikuiseen pappeuteen, sukupolvesta sukupolveen."
\par 16 Ja Mooses teki kaiken, niinkuin Herra oli häntä käskenyt; niin hän teki.
\par 17 Ja asumus pystytettiin toisen vuoden ensimmäisessä kuussa, kuukauden ensimmäisenä päivänä.
\par 18 Silloin Mooses pystytti asumuksen. Hän asetti paikoilleen sen jalustat, kiinnitti sen laudat, sovitti paikoilleen sen poikkitangot ja pystytti sen pylväät.
\par 19 Ja hän levitti asumuksen ylle teltan ja asetti ylimmäksi sen päälle telttapeitteen, niinkuin Herra oli Moosesta käskenyt.
\par 20 Sitten hän otti lain, pani sen arkkiin ja kiinnitti arkkiin korennot ja asetti armoistuimen arkin päälle.
\par 21 Ja hän toi arkin asumukseen ja sijoitti paikoilleen sitä verhoavan esiripun, ripustaen sen lain arkin eteen, niinkuin Herra oli Moosesta käskenyt.
\par 22 Ja hän asetti pöydän ilmestysmajaan, asumuksen pohjoissivulle, esiripun ulkopuolelle
\par 23 ja latoi sille näkyleivät päälletysten Herran eteen, niinkuin Herra oli Moosesta käskenyt.
\par 24 Ja hän asetti seitsenhaaraisen lampun ilmestysmajaan, vastapäätä pöytää, asumuksen eteläsivulle,
\par 25 ja nosti lamput paikoilleen Herran eteen, niinkuin Herra oli Moosesta käskenyt.
\par 26 Ja hän asetti kultaisen alttarin ilmestysmajaan esiripun eteen
\par 27 ja poltti siinä hyvänhajuista suitsuketta, niinkuin Herra oli Moosesta käskenyt.
\par 28 Ja hän ripusti uutimen asumuksen oveen.
\par 29 Ja polttouhrialttarin hän asetti asumuksen, ilmestysmajan, oven eteen ja uhrasi sen päällä polttouhrin ja ruokauhrin, niinkuin Herra oli Moosesta käskenyt.
\par 30 Ja altaan hän asetti ilmestysmajan ja alttarin välille ja kaatoi siihen vettä peseytymistä varten.
\par 31 Ja Mooses ja Aaron ja hänen poikansa pesivät siinä kätensä ja jalkansa;
\par 32 kun he menivät ilmestysmajaan tahi lähestyivät alttaria, niin he aina peseytyivät, niinkuin Herra oli Moosesta käskenyt.
\par 33 Ja asumuksen ja alttarin ympärille hän laittoi esipihan ja ripusti uutimen esipihan porttiin. Ja niin Mooses päätti sen työn.
\par 34 Sitten pilvi peitti ilmestysmajan, ja Herran kirkkaus täytti asumuksen;
\par 35 eikä Mooses voinut mennä ilmestysmajaan, sillä pilvi oli laskeutunut sen päälle, ja Herran kirkkaus täytti asumuksen.
\par 36 Ja joka kerta, kun pilvi kohosi asumuksen päältä, lähtivät israelilaiset liikkeelle; näin oli koko heidän vaelluksensa ajan.
\par 37 Mutta milloin pilvi ei kohonnut, he eivät lähteneet liikkeelle, ennenkuin sinä päivänä, jona se taas kohosi.
\par 38 Sillä Herran pilvi oli päivällä asumuksen päällä, ja yöllä oli pilvessä tulen hohde kaikkien israelilaisten silmien edessä; näin oli koko heidän vaelluksensa ajan.


\end{document}