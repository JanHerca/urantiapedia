\begin{document}

\title{Kirje efesolaisille}


\chapter{1}

\par 1 Paavali, Jumalan tahdosta Kristuksen Jeesuksen apostoli, Efesossa oleville pyhille ja uskoville Kristuksessa Jeesuksessa.
\par 2 Armo teille ja rauha Jumalalta, meidän Isältämme, ja Herralta Jeesukselta Kristukselta!
\par 3 Ylistetty olkoon meidän Herramme Jeesuksen Kristuksen Jumala ja Isä, joka on siunannut meitä taivaallisissa kaikella hengellisellä siunauksella Kristuksessa,
\par 4 niinkuin hän ennen maailman perustamista oli hänessä valinnut meidät olemaan pyhät ja nuhteettomat hänen edessään, rakkaudessa,
\par 5 edeltäpäin määräten meidät lapseuteen, hänen yhteyteensä Jeesuksen Kristuksen kautta, hänen oman tahtonsa mielisuosion mukaan,
\par 6 sen armonsa kirkkauden kiitokseksi, minkä hän on lahjoittanut meille siinä rakastetussa,
\par 7 jossa meillä on lunastus hänen verensä kautta, rikkomusten anteeksisaaminen, hänen armonsa rikkauden mukaan.
\par 8 Tätä armoa hän on ylenpalttisesti antanut meille kaikkinaiseksi viisaudeksi ja ymmärrykseksi,
\par 9 tehden meille tiettäväksi sen tahtonsa salaisuuden, että hän, päätöksensä mukaan, jonka hän oli nähnyt hyväksi itsessään tehdä -
\par 10 siitä armotaloudesta, minkä hän aikojen täyttyessä aikoi toteuttaa, - oli yhdistävä Kristuksessa yhdeksi kaikki, mitä on taivaissa ja mitä on maan päällä.
\par 11 Hänessä me myös olemme saaneet perintöosan, ollen siihen edeltämäärätyt hänen aivoituksensa mukaan, hänen, joka vaikuttaa kaikki oman tahtonsa päättämän mukaan,
\par 12 että me olisimme hänen kirkkautensa kiitokseksi, me, jotka jo edeltä olimme panneet toivomme Kristukseen.
\par 13 Hänessä on teihinkin, sittenkuin olitte kuulleet totuuden sanan, pelastuksenne evankeliumin, uskoviksi tultuanne pantu luvatun Pyhän Hengen sinetti,
\par 14 sen, joka on meidän perintömme vakuutena, hänen omaisuutensa lunastamiseksi - hänen kirkkautensa kiitokseksi.
\par 15 Sentähden, kun kuulin siitä uskosta, joka teillä on Herrassa Jeesuksessa, ja teidän rakkaudestanne kaikkia pyhiä kohtaan,
\par 16 en minäkään lakkaa kiittämästä teidän tähtenne, kun muistelen teitä rukouksissani,
\par 17 anoen, että meidän Herramme Jeesuksen Kristuksen Jumala, kirkkauden Isä, antaisi teille viisauden ja ilmestyksen Hengen hänen tuntemisessaan
\par 18 ja valaisisi teidän sydämenne silmät, että tietäisitte, mikä on se toivo, johon hän on teidät kutsunut, kuinka suuri hänen perintönsä kirkkaus hänen pyhissään
\par 19 ja mikä hänen voimansa ylenpalttinen suuruus meitä kohtaan, jotka uskomme - sen hänen väkevyytensä voiman vaikutuksen mukaan,
\par 20 jonka hän vaikutti Kristuksessa, kun hän herätti hänet kuolleista ja asetti hänet oikealle puolellensa taivaissa,
\par 21 korkeammalle kaikkea hallitusta ja valtaa ja voimaa ja herrautta ja jokaista nimeä, mikä mainitaan, ei ainoastaan tässä maailmanajassa, vaan myös tulevassa.
\par 22 Ja kaikki hän on asettanut hänen jalkainsa alle ja antanut hänet kaiken pääksi seurakunnalle,
\par 23 joka on hänen ruumiinsa, hänen täyteytensä, joka kaikki kaikissa täyttää.

\chapter{2}

\par 1 Ja Jumala on eläviksi tehnyt teidät, jotka olitte kuolleet rikoksiinne ja synteihinne,
\par 2 joissa te ennen vaelsitte tämän maailman menon mukaan, ilmavallan hallitsijan, sen hengen hallitsijan, mukaan, joka nyt tekee työtään tottelemattomuuden lapsissa,
\par 3 joiden joukossa mekin kaikki ennen vaelsimme lihamme himoissa, noudattaen lihan ja ajatusten mielitekoja, ja olimme luonnostamme vihan lapsia niinkuin muutkin;
\par 4 mutta Jumala, joka on laupeudesta rikas, suuren rakkautensa tähden, jolla hän on meitä rakastanut,
\par 5 on tehnyt meidät, jotka olimme kuolleet rikoksiimme, eläviksi Kristuksen kanssa - armosta te olette pelastetut -
\par 6 ja yhdessä hänen kanssaan herättänyt ja yhdessä hänen kanssaan asettanut meidät taivaallisiin Kristuksessa Jeesuksessa,
\par 7 osoittaakseen tulevina maailmanaikoina armonsa ylenpalttista runsautta, hyvyydessään meitä kohtaan Kristuksessa Jeesuksessa.
\par 8 Sillä armosta te olette pelastetut uskon kautta, ette itsenne kautta - se on Jumalan lahja -
\par 9 ette tekojen kautta, ettei kukaan kerskaisi.
\par 10 Sillä me olemme hänen tekonsa, luodut Kristuksessa Jeesuksessa hyviä töitä varten, jotka Jumala on edeltäpäin valmistanut, että me niissä vaeltaisimme.
\par 11 Muistakaa sentähden, että te ennen, te lihanne puolesta pakanat, jotka olette saaneet ympärileikkaamattomien nimen niiltä, joita, lihaan käsillä tehdyn ympärileikkauksen mukaisesti, sanotaan ympärileikatuiksi -
\par 12 että te siihen aikaan olitte ilman Kristusta, olitte vailla Israelin kansalaisoikeutta ja vieraat lupauksen liitoille, ilman toivoa ja ilman Jumalaa maailmassa;
\par 13 mutta nyt, kun olette Kristuksessa Jeesuksessa, olette te, jotka ennen olitte kaukana, päässeet lähelle Kristuksen veressä.
\par 14 Sillä hän on meidän rauhamme, hän, joka teki molemmat yhdeksi ja purki erottavan väliseinän, nimittäin vihollisuuden,
\par 15 kun hän omassa lihassaan teki tehottomaksi käskyjen lain säädöksinensä, luodakseen itsessänsä nuo kaksi yhdeksi uudeksi ihmiseksi, tehden rauhan,
\par 16 ja yhdessä ruumiissa sovittaakseen molemmat Jumalan kanssa ristin kautta, kuolettaen itsensä kautta vihollisuuden.
\par 17 Ja hän tuli ja julisti rauhaa teille, jotka kaukana olitte, ja rauhaa niille, jotka lähellä olivat;
\par 18 sillä hänen kauttansa on meillä molemmilla pääsy yhdessä Hengessä Isän tykö.
\par 19 Niin ette siis enää ole vieraita ettekä muukalaisia, vaan te olette pyhien kansalaisia ja Jumalan perhettä,
\par 20 apostolien ja profeettain perustukselle rakennettuja, kulmakivenä itse Kristus Jeesus,
\par 21 jossa koko rakennus liittyy yhteen ja kasvaa pyhäksi temppeliksi Herrassa;
\par 22 ja hänessä tekin yhdessä muitten kanssa rakennutte Jumalan asumukseksi Hengessä.

\chapter{3}

\par 1 Sen takia minä, Paavali, teidän, pakanain, tähden Kristuksen Jeesuksen vanki, notkistan polveni -
\par 2 olette kaiketi kuulleet siitä Jumalan armon taloudenhoidosta, mikä on minulle teitä varten annettu,
\par 3 että näet tämä salaisuus on ilmestyksen kautta tehty minulle tiettäväksi, niinkuin olen siitä edellä lyhyesti kirjoittanut;
\par 4 josta te sitä lukiessanne voitte huomata, kuinka perehtynyt minä olen Kristuksen salaisuuteen,
\par 5 jota menneiden sukupolvien aikana ei ole ihmisten lapsille tiettäväksi tehty, niinkuin se nyt Hengessä on ilmoitettu hänen pyhille apostoleilleen ja profeetoille:
\par 6 että näet pakanatkin ovat kanssaperillisiä ja yhtä ruumista ja osallisia lupaukseen Kristuksessa Jeesuksessa evankeliumin kautta,
\par 7 jonka palvelijaksi minä olen tullut Jumalan armon lahjan kautta, joka minulle on annettu hänen voimansa vaikutuksesta.
\par 8 Minulle, kaikista pyhistä halvimmalle, on annettu tämä armo: julistaa pakanoille evankeliumia Kristuksen tutkimattomasta rikkaudesta
\par 9 ja saattaa kaikille ilmeiseksi, mitä on sen salaisuuden taloudenhoito, joka ikuisista ajoista asti on ollut kätkettynä Jumalassa, kaiken Luojassa,
\par 10 että Jumalan moninainen viisaus seurakunnan kautta nyt tulisi taivaallisten hallitusten ja valtojen tietoon
\par 11 sen iankaikkisen aivoituksen mukaisesti, jonka hän oli säätänyt Kristuksessa Jeesuksessa, meidän Herrassamme,
\par 12 jossa meillä, uskon kautta häneen, on uskallus ja luottavainen pääsy Jumalan tykö.
\par 13 Siksi minä pyydän, ettette lannistuisi niiden ahdistusten vuoksi, joita minä teidän tähtenne kärsin, sillä ne ovat teidän kunnianne.
\par 14 Sentähden minä notkistan polveni Isän edessä,
\par 15 josta kaikki, millä isä on, taivaissa ja maan päällä, saa nimensä,
\par 16 että hän kirkkautensa runsauden mukaisesti antaisi teidän, sisällisen ihmisenne puolesta, voimassa vahvistua hänen Henkensä kautta
\par 17 ja Kristuksen asua uskon kautta teidän sydämissänne,
\par 18 niin että te, rakkauteen juurtuneina ja perustuneina, voisitte kaikkien pyhien kanssa käsittää, mikä leveys ja pituus ja korkeus ja syvyys on,
\par 19 ja oppia tuntemaan Kristuksen rakkauden, joka on kaikkea tietoa ylempänä; että tulisitte täyteen Jumalan kaikkea täyteyttä.
\par 20 Mutta hänelle, joka voi tehdä enemmän, monin verroin enemmän kuin kaikki, mitä me anomme tai ymmärrämme, sen voiman mukaan, joka meissä vaikuttaa,
\par 21 hänelle kunnia seurakunnassa ja Kristuksessa Jeesuksessa kautta kaikkien sukupolvien, aina ja iankaikkisesti! Amen.

\chapter{4}

\par 1 Niin kehoitan siis minä, joka olen vankina Herrassa, teitä vaeltamaan, niinkuin saamanne kutsumuksen arvo vaatii,
\par 2 kaikessa nöyryydessä ja hiljaisuudessa ja pitkämielisyydessä kärsien toinen toistanne rakkaudessa
\par 3 ja pyrkien säilyttämään hengen yhteyden rauhan yhdyssiteellä:
\par 4 yksi ruumis ja yksi henki, niinkuin te olette kutsututkin yhteen ja samaan toivoon, jonka te kutsumuksessanne saitte;
\par 5 yksi Herra, yksi usko, yksi kaste;
\par 6 yksi Jumala ja kaikkien Isä, joka on yli kaikkien ja kaikkien kautta ja kaikissa.
\par 7 Mutta itsekullekin meistä on armo annettu Kristuksen lahjan mitan mukaan.
\par 8 Sentähden on sanottu: "Hän astui ylös korkeuteen, hän otti vankeja saaliikseen, hän antoi lahjoja ihmisille".
\par 9 Mutta että hän astui ylös, mitä se on muuta, kuin että hän oli astunut alaskin, maan alimpiin paikkoihin?
\par 10 Hän, joka on astunut alas, on se, joka myös astui ylös, kaikkia taivaita ylemmäksi, täyttääkseen kaikki.
\par 11 Ja hän antoi muutamat apostoleiksi, toiset profeetoiksi, toiset evankelistoiksi, toiset paimeniksi ja opettajiksi,
\par 12 tehdäkseen pyhät täysin valmiiksi palveluksen työhön, Kristuksen ruumiin rakentamiseen,
\par 13 kunnes me kaikki pääsemme yhteyteen uskossa ja Jumalan Pojan tuntemisessa, täyteen miehuuteen, Kristuksen täyteyden täyden iän määrään,
\par 14 ettemme enää olisi alaikäisiä, jotka ajelehtivat ja joita viskellään kaikissa opintuulissa ja ihmisten arpapelissä ja eksytyksen kavalissa juonissa;
\par 15 vaan että me, totuutta noudattaen rakkaudessa, kaikin tavoin kasvaisimme häneen, joka on pää, Kristus,
\par 16 josta koko ruumis, yhteen liitettynä ja koossa pysyen jokaisen jänteensä avulla, kasvaa rakentuakseen rakkaudessa sen voiman määrän mukaan, mikä kullakin osalla on.
\par 17 Sen minä siis sanon ja varoitan Herrassa: älkää enää vaeltako, niinkuin pakanat vaeltavat mielensä turhuudessa,
\par 18 nuo, jotka, pimentyneinä ymmärrykseltään ja vieraantuneina Jumalan elämästä heissä olevan tietämättömyyden tähden ja sydämensä paatumuksen tähden,
\par 19 ovat päästäneet tuntonsa turtumaan ja heittäytyneet irstauden valtaan, harjoittamaan kaikkinaista saastaisuutta, ahneudessa.
\par 20 Mutta näin te ette ole oppineet Kristusta tuntemaan,
\par 21 jos muutoin olette hänestä kuulleet ja hänessä opetusta saaneet, niinkuin totuus on Jeesuksessa:
\par 22 että teidän tulee panna pois vanha ihmisenne, jonka mukaan te ennen vaelsitte ja joka turmelee itsensä petollisia himoja seuraten,
\par 23 ja uudistua mielenne hengeltä
\par 24 ja pukea päällenne uusi ihminen, joka Jumalan mukaan on luotu totuuden vanhurskauteen ja pyhyyteen.
\par 25 Pankaa sentähden pois valhe ja puhukaa totta, kukin lähimmäisensä kanssa, sillä me olemme toinen toisemme jäseniä.
\par 26 "Vihastukaa, mutta älkää syntiä tehkö." Älkää antako auringon laskea vihanne yli,
\par 27 älkääkä antako perkeleelle sijaa.
\par 28 Joka on varastanut, älköön enää varastako, vaan tehköön ennemmin työtä ja toimittakoon käsillään sitä, mikä hyvää on, että hänellä olisi, mitä antaa tarvitsevalle.
\par 29 Mikään rietas puhe älköön suustanne lähtekö, vaan ainoastaan sellainen, mikä on rakentavaista ja tarpeellista ja on mieluista niille, jotka kuulevat.
\par 30 Älkääkä saattako murheelliseksi Jumalan Pyhää Henkeä, joka on teille annettu sinetiksi lunastuksen päivään saakka.
\par 31 Kaikki katkeruus ja kiivastus ja viha ja huuto ja herjaus, kaikki pahuus olkoon kaukana teistä.
\par 32 Olkaa sen sijaan toisianne kohtaan ystävällisiä, hyväsydämisiä, anteeksiantavaisia toinen toisellenne, niinkuin Jumalakin on Kristuksessa teille anteeksi antanut.

\chapter{5}

\par 1 Olkaa siis Jumalan seuraajia, niinkuin rakkaat lapset,
\par 2 ja vaeltakaa rakkaudessa, niinkuin Kristuskin rakasti teitä ja antoi itsensä meidän edestämme lahjaksi ja uhriksi, Jumalalle "suloiseksi tuoksuksi".
\par 3 Mutta haureutta ja minkäänlaista saastaisuutta tai ahneutta älköön edes mainittako teidän keskuudessanne - niinkuin pyhien sopii -
\par 4 älköön myös rivoutta tai tyhmää lorua tai ilvehtimistä, jotka ovat sopimattomia, vaan paremmin kiitosta.
\par 5 Sillä sen te tiedätte ja tunnette, ettei yhdelläkään haureellisella eikä saastaisella eikä ahneella - sillä hän on epäjumalanpalvelija - ole perintöosaa Kristuksen ja Jumalan valtakunnassa.
\par 6 Älköön kukaan pettäkö teitä tyhjillä puheilla, sillä semmoisten tähden kohtaa Jumalan viha tottelemattomuuden lapsia;
\par 7 älkää siis olko niihin osallisia heidän kanssaan.
\par 8 Ennen te olitte pimeys, mutta nyt te olette valkeus Herrassa. Vaeltakaa valkeuden lapsina
\par 9 - sillä kaikkinainen hyvyys ja vanhurskaus ja totuus on valkeuden hedelmä -
\par 10 ja tutkikaa, mikä on otollista Herralle;
\par 11 älköönkä teillä olko mitään osallisuutta pimeyden hedelmättömiin tekoihin, vaan päinvastoin nuhdelkaakin niistä.
\par 12 Sillä häpeällistä on jo sanoakin, mitä he salassa tekevät;
\par 13 mutta tämä kaikki tulee ilmi, kun valkeus sen paljastaa, sillä kaikki, mikä tulee ilmi, on valkeutta.
\par 14 Sentähden sanotaan: "Heräjä sinä, joka nukut, ja nouse kuolleista, niin Kristus sinua valaisee!"
\par 15 Katsokaa siis tarkoin, kuinka vaellatte: ei niinkuin tyhmät, vaan niinkuin viisaat,
\par 16 ja ottakaa vaari oikeasta hetkestä, sillä aika on paha.
\par 17 Älkää sentähden olko mielettömät, vaan ymmärtäkää, mikä Herran tahto on.
\par 18 Älkääkä juopuko viinistä, sillä siitä tulee irstas meno, vaan täyttykää Hengellä,
\par 19 puhuen keskenänne psalmeilla ja kiitosvirsillä ja hengellisillä lauluilla, veisaten ja laulaen sydämessänne Herralle,
\par 20 kiittäen aina Jumalaa ja Isää kaikesta meidän Herramme Jeesuksen Kristuksen nimessä.
\par 21 Ja olkaa toinen toisellenne alamaiset Kristuksen pelossa.
\par 22 Vaimot, olkaa omille miehillenne alamaiset niinkuin Herralle;
\par 23 sillä mies on vaimon pää, niinkuin myös Kristus on seurakunnan pää, hän, ruumiin vapahtaja.
\par 24 Mutta niinkuin seurakunta on Kristukselle alamainen, niin olkoot vaimotkin miehillensä kaikessa alamaiset.
\par 25 Miehet, rakastakaa vaimojanne, niinkuin Kristuskin rakasti seurakuntaa ja antoi itsensä alttiiksi sen edestä,
\par 26 että hän sen pyhittäisi, puhdistaen sen, vedellä pesten, sanan kautta,
\par 27 saadakseen asetetuksi eteensä kirkastettuna seurakunnan, jossa ei olisi tahraa eikä ryppyä eikä mitään muuta sellaista, vaan joka olisi pyhä ja nuhteeton.
\par 28 Samalla tavoin tulee myös miesten rakastaa vaimojansa niinkuin omia ruumiitaan; joka rakastaa vaimoansa, hän rakastaa itseänsä.
\par 29 Sillä eihän kukaan koskaan ole vihannut omaa lihaansa, vaan hän ravitsee ja vaalii sitä, niinkuin Kristuskin seurakuntaa,
\par 30 sillä me olemme hänen ruumiinsa jäseniä.
\par 31 "Sentähden mies luopukoon isästänsä ja äidistänsä ja liittyköön vaimoonsa, ja ne kaksi tulevat yhdeksi lihaksi."
\par 32 Tämä salaisuus on suuri; minä tarkoitan Kristusta ja seurakuntaa.
\par 33 Mutta myös teistä kukin kohdaltaan rakastakoon vaimoaan niinkuin itseänsä; mutta vaimo kunnioittakoon miestänsä.

\chapter{6}

\par 1 Lapset, olkaa vanhemmillenne kuuliaiset Herrassa, sillä se on oikein.
\par 2 "Kunnioita isääsi ja äitiäsi" - tämä on ensimmäinen käsky, jota seuraa lupaus -
\par 3 "että menestyisit ja kauan eläisit maan päällä".
\par 4 Ja te isät, älkää kiihoittako lapsianne vihaan, vaan kasvattakaa heitä Herran kurissa ja nuhteessa.
\par 5 Palvelijat, olkaa kuuliaiset maallisille isännillenne, pelossa ja vavistuksessa, sydämenne yksinkertaisuudessa, niinkuin Kristukselle,
\par 6 ei silmänpalvelijoina, ihmisille mieliksi, vaan Kristuksen palvelijoina, sydämestänne tehden, mitä Jumala tahtoo,
\par 7 hyvällä mielellä palvellen, niinkuin palvelisitte Herraa ettekä ihmisiä,
\par 8 tietäen, että mitä hyvää kukin tekee, sen hän saa takaisin Herralta, olkoonpa orja tai vapaa.
\par 9 Ja te isännät, tehkää samoin heille, jättäkää pois uhkaileminen, sillä tiedättehän, että sekä heidän että teidän Herranne on taivaissa ja ettei hän katso henkilöön.
\par 10 Lopuksi, vahvistukaa Herrassa ja hänen väkevyytensä voimassa.
\par 11 Pukekaa yllenne Jumalan koko sota-asu, voidaksenne kestää perkeleen kavalat juonet.
\par 12 Sillä meillä ei ole taistelu verta ja lihaa vastaan, vaan hallituksia vastaan, valtoja vastaan, tässä pimeydessä hallitsevia maailmanvaltiaita vastaan, pahuuden henkiolentoja vastaan taivaan avaruuksissa.
\par 13 Sentähden ottakaa päällenne Jumalan koko sota-asu, voidaksenne pahana päivänä tehdä vastarintaa ja kaikki suoritettuanne pysyä pystyssä.
\par 14 Seisokaa siis kupeet totuuteen vyötettyinä, ja olkoon pukunanne vanhurskauden haarniska,
\par 15 ja kenkinä jaloissanne alttius rauhan evankeliumille.
\par 16 Kaikessa ottakaa uskon kilpi, jolla voitte sammuttaa kaikki pahan palavat nuolet,
\par 17 ja ottakaa vastaan pelastuksen kypäri ja Hengen miekka, joka on Jumalan sana.
\par 18 Ja tehkää tämä kaikella rukouksella ja anomisella, rukoillen joka aika Hengessä ja sitä varten valvoen kaikessa kestäväisyydessä ja anomisessa kaikkien pyhien puolesta;
\par 19 ja minunkin puolestani, että minulle, kun suuni avaan, annettaisiin oikeat sanat rohkeasti julistaakseni evankeliumin salaisuutta,
\par 20 jonka tähden minä olen lähettiläänä kahleissa, että minä siitä rohkeasti puhuisin, niinkuin minun puhua tulee.
\par 21 Mutta että tekin tietäisitte tilani, kuinka minun on, niin on Tykikus, rakas veljeni ja uskollinen palvelija Herrassa, antava teille siitä kaikesta tiedon.
\par 22 Minä lähetän hänet teidän tykönne juuri sitä varten, että saisitte tietää meidän tilamme ja että hän lohduttaisi teidän sydämiänne.
\par 23 Rauha veljille ja rakkaus, uskon kanssa, Isältä Jumalalta ja Herralta Jeesukselta Kristukselta!
\par 24 Armo olkoon kaikkien kanssa, jotka rakastavat meidän Herraamme Jeesusta Kristusta - katoamattomuudessa.


\end{document}