\begin{document}

\title{Kirje heprealaisille}


\chapter{1}

\par 1 Sittenkuin Jumala muinoin monesti ja monella tapaa oli puhunut isille profeettain kautta,
\par 2 on hän näinä viimeisinä päivinä puhunut meille Pojan kautta, jonka hän on pannut kaiken perilliseksi, jonka kautta hän myös on maailman luonut
\par 3 ja joka, ollen hänen kirkkautensa säteily ja hänen olemuksensa kuva ja kantaen kaikki voimansa sanalla, on, toimitettuaan puhdistuksen synneistä, istunut Majesteetin oikealle puolelle korkeuksissa,
\par 4 tullen enkeleitä niin paljoa korkeammaksi, kuin hänen perimänsä nimi on jalompi kuin heidän.
\par 5 Sillä kenelle enkeleistä hän koskaan on sanonut: "Sinä olet minun Poikani, tänä päivänä minä sinut synnytin"; ja taas: "Minä olen oleva hänen Isänsä, ja hän on oleva minun Poikani"?
\par 6 Ja siitä, kun hän jälleen tuo esikoisensa maailmaan, hän sanoo: "Ja kumartakoot häntä kaikki Jumalan enkelit".
\par 7 Ja enkeleistä hän sanoo: "Hän tekee enkelinsä tuuliksi ja palvelijansa tulen liekiksi";
\par 8 mutta Pojasta: "Jumala, sinun valtaistuimesi pysyy aina ja iankaikkisesti, ja sinun valtakuntasi valtikka on oikeuden valtikka.
\par 9 Sinä rakastit vanhurskautta ja vihasit laittomuutta; sentähden on Jumala, sinun Jumalasi, voidellut sinua iloöljyllä, enemmän kuin sinun osaveljiäsi."
\par 10 Ja: "Sinä, Herra, olet alussa maan perustanut, ja taivaat ovat sinun kättesi tekoja;
\par 11 ne katoavat, mutta sinä pysyt, ja ne vanhenevat kaikki niinkuin vaate,
\par 12 ja niinkuin vaipan sinä ne käärit, niinkuin vaatteen, ja ne muuttuvat; mutta sinä olet sama, eivätkä sinun vuotesi lopu".
\par 13 Kenelle enkeleistä hän koskaan on sanonut: "Istu minun oikealle puolelleni, kunnes minä panen sinun vihollisesi sinun jalkojesi astinlaudaksi"?
\par 14 Eivätkö he kaikki ole palvelevia henkiä, palvelukseen lähetettyjä niitä varten, jotka saavat autuuden periä?

\chapter{2}

\par 1 Sentähden tulee meidän sitä tarkemmin ottaa vaari siitä, mitä olemme kuulleet, ettemme vain kulkeutuisi sen ohitse.
\par 2 Sillä jos enkelien kautta puhuttu sana pysyi lujana, ja jokainen rikkomus ja tottelemattomuus sai oikeudenmukaisen palkkansa,
\par 3 kuinka me voimme päästä pakoon, jos emme välitä tuosta niin suuresta pelastuksesta, jonka Herra alkuaan julisti ja joka niiden vahvistamana, jotka olivat sen kuulleet, saatettiin meille,
\par 4 kun Jumala yhdessä heidän kanssaan todisti tunnusmerkeillä ja ihmeillä ja moninaisilla väkevillä teoilla ja jakamalla Pyhää Henkeä tahtonsa mukaan?
\par 5 Sillä enkelien alle hän ei alistanut tulevaa maailmaa, josta me puhumme.
\par 6 Vaan joku on jossakin paikassa todistanut, sanoen: "Mikä on ihminen, että sinä häntä muistat, tai ihmisen poika, että pidät hänestä huolen?
\par 7 Sinä teit hänet vähäksi aikaa enkeleitä halvemmaksi, kirkkaudella ja kunnialla sinä hänet seppelöitsit, ja sinä panit hänet hallitsemaan kättesi tekoja;
\par 8 asetit kaikki hänen jalkojensa alle." Sillä, asettaessaan kaikki hänen valtansa alle, hän ei jättänyt mitään hänen allensa alistamatta. Mutta nyt emme vielä näe kaikkea hänen valtansa alle asetetuksi.
\par 9 Mutta hänet, joka vähäksi aikaa oli tehty enkeleitä halvemmaksi, Jeesuksen, me näemme hänen kuolemansa kärsimyksen tähden kirkkaudella ja kunnialla seppelöidyksi, että hän Jumalan armosta olisi kaikkien edestä joutunut maistamaan kuolemaa.
\par 10 Sillä hänen, jonka tähden kaikki on ja jonka kautta kaikki on, sopi, saattaessaan paljon lapsia kirkkauteen, kärsimysten kautta tehdä heidän pelastuksensa päämies täydelliseksi.
\par 11 Sillä sekä hän, joka pyhittää, että ne, jotka pyhitetään, ovat kaikki alkuisin yhdestä. Sentähden hän ei häpeä kutsua heitä veljiksi,
\par 12 kun hän sanoo: "Minä julistan sinun nimeäsi veljilleni, ylistän sinua seurakunnan keskellä";
\par 13 ja taas: "Minä panen uskallukseni häneen"; ja taas: "Katso, minä ja lapset, jotka Jumala on minulle antanut!"
\par 14 Koska siis lapsilla on veri ja liha, tuli hänkin niistä yhtäläisellä tavalla osalliseksi, että hän kuoleman kautta kukistaisi sen, jolla oli kuolema vallassaan, se on: perkeleen,
\par 15 ja vapauttaisi kaikki ne, jotka kuoleman pelosta kautta koko elämänsä olivat olleet orjuuden alaisia.
\par 16 Sillä ei hän ota huomaansa enkeleitä, vaan Aabrahamin siemenen hän ottaa huomaansa.
\par 17 Sentähden piti hänen kaikessa tuleman veljiensä kaltaiseksi, että hänestä tulisi laupias ja uskollinen ylimmäinen pappi tehtävissään Jumalan edessä, sovittaakseen kansan synnit.
\par 18 Sillä sentähden, että hän itse on kärsinyt ja ollut kiusattu, voi hän kiusattuja auttaa.

\chapter{3}

\par 1 Sentähden, pyhät veljet, jotka olette taivaallisesta kutsumuksesta osalliset, kiinnittäkää mielenne meidän tunnustuksemme apostoliin ja ylimmäiseen pappiin, Jeesukseen,
\par 2 joka on uskollinen asettajalleen, niinkuin Mooseskin oli "uskollinen koko hänen huoneessansa".
\par 3 Sillä hänen on Moosekseen verraten katsottu ansaitsevan niin paljoa suuremman kirkkauden, kuin huoneen rakentajan kunnia on suurempi kuin huoneen.
\par 4 Sillä jokainen huone on jonkun rakentama, mutta kaiken rakentaja on Jumala.
\par 5 Ja Mooses tosin oli "palvelijana uskollinen koko hänen huoneessansa", todistukseksi siitä, mikä vastedes piti sanottaman,
\par 6 mutta Kristus on uskollinen Poikana, hänen huoneensa haltijana; ja hänen huoneensa olemme me, jos loppuun asti pidämme vahvana toivon rohkeuden ja kerskauksen.
\par 7 Sentähden, niinkuin Pyhä Henki sanoo: "Tänä päivänä, jos te kuulette hänen äänensä,
\par 8 älkää paaduttako sydämiänne, niinkuin teitte katkeroituksessa, kiusauksen päivänä erämaassa,
\par 9 jossa teidän isänne minua kiusasivat ja koettelivat, vaikka olivat nähneet minun tekojani neljäkymmentä vuotta;
\par 10 sentähden minä vihastuin tähän sukupolveen ja sanoin: 'Aina he eksyvät sydämessään'; mutta he eivät oppineet tuntemaan minun teitäni;
\par 11 ja niin minä vihassani vannoin: 'He eivät pääse minun lepooni'."
\par 12 Katsokaa, veljet, ettei vain kenelläkään teistä ole paha, epäuskoinen sydän, niin että hän luopuu elävästä Jumalasta,
\par 13 vaan kehoittakaa toisianne joka päivä, niin kauan kuin sanotaan: "tänä päivänä", ettei teistä kukaan synnin pettämänä paatuisi;
\par 14 sillä me olemme tulleet osallisiksi Kristuksesta, kunhan vain pysymme luottamuksessa, joka meillä alussa oli, vahvoina loppuun asti.
\par 15 Kun sanotaan: "Tänä päivänä, jos te kuulette hänen äänensä, älkää paaduttako sydämiänne, niinkuin teitte katkeroituksessa",
\par 16 ketkä sitten, vaikka kuulivat, katkeroittivat hänet? Eivätkö kaikki, jotka olivat Mooseksen johdolla lähteneet Egyptistä?
\par 17 Mutta keihin hän oli vihastunut neljäkymmentä vuotta? Eikö niihin, jotka olivat syntiä tehneet, joiden ruumiit kaatuivat erämaahan?
\par 18 Ja keille hän vannoi, etteivät he pääse hänen lepoonsa? Eikö tottelemattomille?
\par 19 Ja niin me näemme, että he epäuskon tähden eivät voineet siihen päästä.

\chapter{4}

\par 1 Varokaamme siis, koska lupaus päästä hänen lepoonsa vielä pysyy varmana, ettei vain havaittaisi kenenkään teistä jääneen taipaleelle.
\par 2 Sillä hyvä sanoma on julistettu meille niinkuin heillekin; mutta heidän kuulemansa sana ei heitä hyödyttänyt, koska se ei uskossa sulautunut niihin, jotka sen kuulivat.
\par 3 Sillä me pääsemme lepoon, me, jotka tulimme uskoon, niinkuin hän on sanonut: "Ja niin minä vihassani vannoin: 'He eivät pääse minun lepooni'", vaikka hänen tekonsa olivat valmiina maailman perustamisesta asti.
\par 4 Sillä hän on jossakin sanonut seitsemännestä päivästä näin: "Ja Jumala lepäsi seitsemäntenä päivänä kaikista teoistansa";
\par 5 ja tässä taas: "He eivät pääse minun lepooni".
\par 6 Koska siis varmana pysyy, että muutamat pääsevät siihen, ja ne, joille hyvä sanoma ensin julistettiin, eivät päässeet siihen tottelemattomuuden tähden,
\par 7 niin hän taas määrää päivän, "tämän päivän", sanomalla Daavidin kautta niin pitkän ajan jälkeen, niinkuin ennen on sanottu: "Tänä päivänä, jos te kuulette hänen äänensä, älkää paaduttako sydämiänne".
\par 8 Sillä jos Joosua olisi saattanut heidät lepoon, niin hän ei puhuisi toisesta, senjälkeisestä päivästä.
\par 9 Niin on Jumalan kansalle sapatinlepo varmasti tuleva.
\par 10 Sillä joka on päässyt hänen lepoonsa, on saanut levon teoistaan, hänkin, niinkuin Jumala omista teoistansa.
\par 11 Ahkeroikaamme siis päästä siihen lepoon, ettei kukaan lankeaisi seuraamaan samaa tottelemattomuuden esimerkkiä.
\par 12 Sillä Jumalan sana on elävä ja voimallinen ja terävämpi kuin mikään kaksiteräinen miekka ja tunkee lävitse, kunnes se erottaa sielun ja hengen, nivelet sekä ytimet, ja on sydämen ajatusten ja aivoitusten tuomitsija;
\par 13 eikä mikään luotu ole hänelle näkymätön, vaan kaikki on alastonta ja paljastettua hänen silmäinsä edessä, jolle meidän on tehtävä tili.
\par 14 Kun meillä siis on suuri ylimmäinen pappi, läpi taivasten kulkenut, Jeesus, Jumalan Poika, niin pitäkäämme kiinni tunnustuksesta.
\par 15 Sillä ei meillä ole sellainen ylimmäinen pappi, joka ei voi sääliä meidän heikkouksiamme, vaan joka on ollut kaikessa kiusattu samalla lailla kuin mekin, kuitenkin ilman syntiä.
\par 16 Käykäämme sentähden uskalluksella armon istuimen eteen, että saisimme laupeuden ja löytäisimme armon, avuksemme oikeaan aikaan.

\chapter{5}

\par 1 Sillä jokainen ylimmäinen pappi, ollen ihmisten joukosta otettu, asetetaan ihmisten puolesta toimittamaan sitä, mikä Jumalalle tulee, uhraamaan lahjoja ja uhreja syntien edestä,
\par 2 ja hän voi säälien kohdella tietämättömiä ja eksyviä, koska hän itsekin on heikkouden alainen,
\par 3 ja tämän heikkoutensa tähden hänen täytyy, samoinkuin kansan puolesta, niin itsensäkin puolesta uhrata syntien edestä.
\par 4 Eikä kukaan sitä arvoa itselleen ota, vaan Jumala kutsuu hänet niinkuin Aaroninkin.
\par 5 Niinpä Kristuskaan ei itse korottanut itseänsä ylimmäisen papin kunniaan, vaan hän, joka sanoi hänelle: "Sinä olet minun Poikani, tänä päivänä minä sinut synnytin";
\par 6 niinkuin hän toisessakin paikassa sanoo: "Sinä olet pappi iankaikkisesti Melkisedekin järjestyksen mukaan".
\par 7 Ja lihansa päivinä hän väkevällä huudolla ja kyynelillä uhrasi rukouksia ja anomuksia sille, joka voi hänet kuolemasta pelastaa; ja hänen rukouksensa kuultiin hänen jumalanpelkonsa tähden.
\par 8 Ja niin hän, vaikka oli Poika, oppi siitä, mitä hän kärsi, kuuliaisuuden,
\par 9 ja kun oli täydelliseksi tullut, tuli hän iankaikkisen autuuden aikaansaajaksi kaikille, jotka ovat hänelle kuuliaiset,
\par 10 hän, jota Jumala nimittää "ylimmäiseksi papiksi Melkisedekin järjestyksen mukaan".
\par 11 Tästä meillä on paljon sanottavaa, ja sitä on vaikea selittää, koska olette käyneet hitaiksi kuulemaan.
\par 12 Sillä te, joiden olisi jo aika olla opettajia, olette taas sen tarpeessa, että teille opetetaan Jumalan sanojen ensimmäisiä alkeita; te olette tulleet maitoa tarvitseviksi, ei vahvaa ruokaa.
\par 13 Sillä jokainen, joka vielä nauttii maitoa, on kokematon vanhurskauden sanassa, sillä hän on lapsi;
\par 14 mutta vahva ruoka on täysi-ikäisiä varten, niitä varten, joiden aistit tottumuksesta ovat harjaantuneet erottamaan hyvän pahasta.

\chapter{6}

\par 1 Jättäkäämme sentähden Kristuksen opin alkeet ja pyrkikäämme täydellisyyteen, ryhtymättä taas uudestaan laskemaan perustusta: parannusta kuolleista töistä ja uskoa Jumalaan,
\par 2 oppia kasteista ja kätten päällepanemisesta, kuolleitten ylösnousemisesta ja iankaikkisesta tuomiosta.
\par 3 Ja niin me tahdomme tehdä, jos vain Jumala sallii.
\par 4 Sillä mahdotonta on niitä, jotka kerran ovat valistetut ja taivaallista lahjaa maistaneet ja Pyhästä Hengestä osallisiksi tulleet
\par 5 ja maistaneet Jumalan hyvää sanaa ja tulevan maailmanajan voimia,
\par 6 ja sitten ovat luopuneet - taas uudistaa parannukseen, he kun jälleen itsellensä ristiinnaulitsevat Jumalan Pojan ja häntä julki häpäisevät.
\par 7 Sillä maa, joka särpii sisäänsä sen päälle usein tulevan sateen ja kantaa kasvun hyödyksi niille, joita varten sitä viljelläänkin, saa siunauksen Jumalalta;
\par 8 mutta se, joka tuottaa orjantappuroita ja ohdakkeita, on kelvoton ja lähellä kirousta, ja sen loppu on, että se poltetaan.
\par 9 Mutta teistä, rakkaat, uskomme sitä, mikä on parempaa ja mikä koituu teille pelastukseksi - vaikka puhummekin näin.
\par 10 Sillä Jumala ei ole väärämielinen, niin että hän unhottaisi teidän työnne ja rakkautenne, jota olette osoittaneet hänen nimeänsä kohtaan, kun olette palvelleet pyhiä ja vielä palvelette.
\par 11 Mutta me halajamme sitä, että kukin teistä osoittaa samaa intoa, säilyttääkseen toivon varmuuden loppuun asti,
\par 12 ettette kävisi veltoiksi, vaan että teistä tulisi niiden seuraajia, jotka uskon ja kärsivällisyyden kautta perivät sen, mikä luvattu on.
\par 13 Sillä kun Jumala oli antanut lupauksen Aabrahamille, vannoi hän itse kauttansa, koska hänellä ei ollut ketään suurempaa, kenen kautta vannoa,
\par 14 ja sanoi: "Totisesti, siunaamalla minä sinut siunaan, ja enentämällä minä sinut enennän";
\par 15 ja näin Aabraham, kärsivällisesti odotettuaan, sai, mitä luvattu oli.
\par 16 Sillä ihmiset vannovat suurempansa kautta, ja vala on heille asian vahvistus ja tekee lopun kaikista vastaväitteistä.
\par 17 Sentähden, kun Jumala lupauksen perillisille vielä tehokkaammin tahtoi osoittaa, että hänen päätöksensä on muuttumaton, vakuutti hän sen valalla,
\par 18 että me näistä kahdesta muuttumattomasta asiasta, joissa Jumala ei ole voinut valhetella, saisimme voimallisen kehoituksen, me, jotka olemme paenneet pitämään kiinni edessämme olevasta toivosta.
\par 19 Se toivo meille on ikäänkuin sielun ankkuri, varma ja luja, joka ulottuu esiripun sisäpuolelle asti,
\par 20 jonne Jeesus edelläjuoksijana meidän puolestamme on mennyt, tultuaan ylimmäiseksi papiksi Melkisedekin järjestyksen mukaan, iankaikkisesti.

\chapter{7}

\par 1 Sillä tämä Melkisedek, Saalemin kuningas, Jumalan, Korkeimman, pappi, joka meni Aabrahamia vastaan, hänen palatessaan kuninkaita voittamasta, ja siunasi hänet;
\par 2 jolle Aabraham myös antoi kymmenykset kaikesta ja joka ensiksi, niinkuin hänen nimensäkin merkitsee, on "vanhurskauden kuningas" ja sen lisäksi vielä "Saalemin kuningas", se on "rauhan kuningas";
\par 3 jolla ei ole isää, ei äitiä, ei sukua, ei päivien alkua eikä elämän loppua, mutta joka on Jumalan Poikaan verrattava - hän pysyy pappina ainaisesti.
\par 4 Katsokaa, kuinka suuri hän on, jolle itse kantaisä Aabraham antoi kymmenykset parhaimmasta saaliistaan.
\par 5 Onhan niillä Leevin pojista, jotka saavat pappeuden, käsky lain mukaan ottaa kymmenyksiä kansalta, se on veljiltään, vaikka nämä ovatkin Aabrahamin kupeista lähteneet;
\par 6 mutta hän, jonka sukua ei johdeta heistä, otti kymmenykset Aabrahamilta ja siunasi sen, jolla oli lupaukset.
\par 7 Mutta kieltämätöntä on, että halvempi saa siunauksen paremmaltaan.
\par 8 Ja täällä kuolevaiset ihmiset ottavat kymmenyksiä, mutta siellä se, jonka todistetaan elävän.
\par 9 Ja Aabrahamin kautta, niin sanoakseni, on Leevikin, joka kymmenyksiä ottaa, maksanut kymmenyksiä;
\par 10 sillä hän oli vielä isänsä kupeissa, kun Melkisedek meni tätä vastaan.
\par 11 Jos siis täydellisyys olisi saavutettu leeviläisen pappeuden kautta, sillä tähän on kansa laissa sidottu, miksi sitten oli tarpeen, että nousi toinen pappi Melkisedekin järjestyksen mukaan eikä tullut nimitetyksi Aaronin järjestyksen mukaan?
\par 12 Sillä pappeuden muuttuessa tapahtuu välttämättä myös lain muutos.
\par 13 Sillä se, josta tämä sanotaan, oli toista sukukuntaa, josta ei kukaan ole alttaritointa hoitanut.
\par 14 Onhan tunnettua, että meidän Herramme on noussut Juudasta, jonka sukukunnan pappeudesta Mooses ei ole mitään puhunut.
\par 15 Ja tämä käy vielä paljoa selvemmäksi, kun nousee toinen pappi, Melkisedekin kaltainen,
\par 16 joka ei ole siksi tullut lihallisen käskyn lain mukaan, vaan katoamattoman elämän voimasta.
\par 17 Sillä hänestä todistetaan: "Sinä olet pappi iankaikkisesti Melkisedekin järjestyksen mukaan".
\par 18 Täten kyllä entinen säädös kumotaan, koska se oli voimaton ja hyödytön
\par 19 - sillä laki ei tehnyt mitään täydelliseksi - mutta sijaan tulee parempi toivo, jonka kautta me lähestymme Jumalaa.
\par 20 Ja niinkuin tämä ei tapahtunut ilman valan vannomista - nuo taas ovat papeiksi tulleet ilman heistä vannottua valaa,
\par 21 mutta tämä hänestä vannotulla valalla, sen asettamana, joka hänelle sanoi: "Herra on vannonut eikä ole katuva: 'Sinä olet pappi iankaikkisesti'" -
\par 22 niin on myös se liitto parempi, jonka takaajaksi Jeesus on tullut.
\par 23 Ja noita toisia pappeja on tullut useampia, koska kuolema ei sallinut heidän pysyä;
\par 24 mutta tällä on katoamaton pappeus, sentähden että hän pysyy iankaikkisesti,
\par 25 jonka tähden hän myös voi täydellisesti pelastaa ne, jotka hänen kauttaan Jumalan tykö tulevat, koska hän aina elää rukoillakseen heidän puolestansa.
\par 26 Senkaltainen ylimmäinen pappi meille sopikin: pyhä, viaton, tahraton, syntisistä erotettu ja taivaita korkeammaksi tullut,
\par 27 jonka ei joka päivä ole tarvis, niinkuin ylimmäisten pappien, ensiksi uhrata omien syntiensä edestä ja sitten kansan; sillä tämän hän teki kerta kaikkiaan, uhratessaan itsensä.
\par 28 Sillä laki asettaa ylimmäisiksi papeiksi ihmisiä, jotka ovat heikkoja, mutta valan sana, joka on myöhäisempi kuin laki, asettaa Pojan, iankaikkisesti täydelliseksi tulleen.

\chapter{8}

\par 1 Mutta pääkohta siinä, mistä me puhumme, on tämä: meillä on sellainen ylimmäinen pappi, joka istuu Majesteetin valtaistuimen oikealla puolella taivaissa,
\par 2 tehdäkseen pappispalvelusta kaikkeinpyhimmässä, siinä oikeassa majassa, jonka on rakentanut Herra eikä ihminen.
\par 3 Sillä jokainen ylimmäinen pappi asetetaan uhraamaan lahjoja ja uhreja, jonka tähden on välttämätöntä, että tälläkin on jotakin uhraamista.
\par 4 Jos hän siis olisi maan päällä, ei hän olisikaan pappi, koska jo ovat olemassa ne, jotka lain mukaan esiinkantavat lahjoja,
\par 5 ja jotka palvelevat siinä, mikä on taivaallisten kuva ja varjo, niinkuin ilmoitettiin Moosekselle, kun hänen oli valmistettava maja. Sillä hänelle sanottiin: "Katso, että teet kaikki sen kaavan mukaan, joka sinulle vuorella näytettiin".
\par 6 Mutta tämä taas on saanut niin paljoa jalomman viran, kuin hän on myös paremman liiton välimies, liiton, joka on paremmille lupauksille perustettu.
\par 7 Sillä jos ensimmäinen liitto olisi ollut moitteeton, ei olisi etsitty sijaa toiselle.
\par 8 Sillä moittien heitä hän sanoo: "Katso, päivät tulevat, sanoo Herra, jolloin minä teen Israelin heimon ja Juudan heimon kanssa uuden liiton,
\par 9 en sellaista liittoa kuin se, jonka minä tein heidän isäinsä kanssa silloin, kun minä tartuin heidän käteensä ja vein heidät pois Egyptin maasta. Sillä he eivät pysyneet minun liitossani, ja niin en minäkään heistä huolinut, sanoo Herra.
\par 10 Sillä tämä on se liitto, jonka minä teen Israelin heimon kanssa näiden päivien jälkeen, sanoo Herra: Minä panen lakini heidän mieleensä, ja kirjoitan ne heidän sydämiinsä, ja niin minä olen heidän Jumalansa, ja he ovat minun kansani.
\par 11 Ja silloin ei enää kukaan opeta kansalaistaan eikä veli veljeään sanoen: 'Tunne Herra'; sillä he kaikki, pienimmästä suurimpaan, tuntevat minut.
\par 12 Sillä minä annan anteeksi heidän vääryytensä enkä enää muista heidän syntejänsä."
\par 13 Sanoessaan "uuden" hän on julistanut ensimmäisen liiton vanhentuneeksi; mutta se, mikä vanhenee ja käy iälliseksi, on lähellä häviämistään.

\chapter{9}

\par 1 Olihan tosin ensimmäiselläkin liitolla jumalanpalvelussäännöt ja maallinen pyhäkkö.
\par 2 Sillä maja oli valmistettu niin, että siinä oli etumainen maja, jossa oli sekä lampunjalka että pöytä ja näkyleivät, ja sen nimi on "pyhä".
\par 3 Mutta toisen esiripun takana oli se maja, jonka nimi on "kaikkeinpyhin";
\par 4 siinä oli kultainen suitsutusalttari ja liiton arkki, yltympäri kullalla päällystetty, jossa säilytettiin kultainen mannaa sisältävä astia ja Aaronin viheriöinyt sauva ja liiton taulut,
\par 5 ja arkin päällä kirkkauden kerubit varjostamassa armoistuinta. Mutta näistä nyt ei ole syytä puhua kustakin erikseen.
\par 6 Kun nyt kaikki on näin järjestetty, menevät papit joka aika etumaiseen majaan jumalanpalvelusta toimittamaan,
\par 7 mutta toiseen majaan menee ainoastaan ylimmäinen pappi kerran vuodessa, ei ilman verta, jonka hän uhraa itsensä edestä ja kansan tahattomien syntien edestä.
\par 8 Näin Pyhä Henki osoittaa, että tie kaikkeinpyhimpään vielä on ilmoittamatta, niin kauan kuin etumainen maja vielä seisoo.
\par 9 Tämä on nykyistä aikaa tarkoittava vertauskuva, ja sen mukaisesti uhrataan lahjoja ja uhreja, jotka eivät kykene tekemään täydelliseksi omassatunnossaan sitä, joka jumalanpalvelusta toimittaa,
\par 10 vaan jotka, niinkuin ruuat ja juomat ja erilaiset pesotkin, ovat ainoastaan lihan sääntöjä, jotka ovat voimassa uuden järjestyksen aikaan asti.
\par 11 Mutta kun Kristus tuli tulevaisen hyvän ylimmäiseksi papiksi, niin hän suuremman ja täydellisemmän majan kautta, joka ei ole käsillä tehty, se on: joka ei ole tätä luomakuntaa,
\par 12 meni, ei kauristen ja vasikkain veren kautta, vaan oman verensä kautta kerta kaikkiaan kaikkeinpyhimpään ja sai aikaan iankaikkisen lunastuksen.
\par 13 Sillä jos kauristen ja härkäin veri ja hiehon tuhka, saastaisten päälle vihmottuna, pyhittää lihanpuhtauteen,
\par 14 kuinka paljoa enemmän on Kristuksen veri, hänen, joka iankaikkisen Hengen kautta uhrasi itsensä viattomana Jumalalle, puhdistava meidän omantuntomme kuolleista teoista palvelemaan elävää Jumalaa!
\par 15 Ja sentähden hän on uuden liiton välimies, että, koska hänen kuolemansa on tapahtunut lunastukseksi ensimmäisen liiton aikuisista rikkomuksista, ne, jotka ovat kutsutut, saisivat luvatun iankaikkisen perinnön.
\par 16 Sillä missä on testamentti, siinä on sen tekijän kuolema toteennäytettävä;
\par 17 sillä vasta kuoleman jälkeen testamentti on pitävä, koska se ei milloinkaan ole voimassa tekijänsä eläessä.
\par 18 Sentähden ei myöskään ensimmäistä liittoa verettä vihitty.
\par 19 Sillä kun Mooses oli kaikelle kansalle julkilukenut kaikki käskyt, niinkuin ne laissa kuuluvat, otti hän vasikkain ja kauristen veren ynnä vettä ja purppuravillaa ja isopin ja vihmoi sekä itse kirjan että kaiken kansan,
\par 20 sanoen: "Tämä on sen liiton veri, jonka Jumala on teille säätänyt".
\par 21 Ja samoin hän verellä vihmoi myös majan ja kaikki palvelukseen kuuluvat esineet.
\par 22 Niin puhdistetaan lain mukaan miltei kaikki verellä, ja ilman verenvuodatusta ei tapahdu anteeksiantamista.
\par 23 On siis välttämätöntä, että taivaallisten kuvat tällä tavalla puhdistetaan, mutta että taivaalliset itse puhdistetaan paremmilla uhreilla kuin nämä.
\par 24 Sillä Kristus ei mennyt käsillä tehtyyn kaikkeinpyhimpään, joka vain on sen oikean kuva, vaan itse taivaaseen, nyt ilmestyäkseen Jumalan kasvojen eteen meidän hyväksemme.
\par 25 Eikä hän mennyt uhratakseen itseänsä monta kertaa, niinkuin ylimmäinen pappi joka vuosi menee kaikkeinpyhimpään, vierasta verta mukanaan,
\par 26 sillä muutoin hänen olisi pitänyt kärsimän monta kertaa maailman perustamisesta asti; mutta nyt hän on yhden ainoan kerran maailmanaikojen lopulla ilmestynyt, poistaakseen synnin uhraamalla itsensä.
\par 27 Ja samoinkuin ihmisille on määrätty, että heidän kerran on kuoleminen, mutta senjälkeen tulee tuomio,
\par 28 samoin Kristuskin, kerran uhrattuna ottaakseen pois monien synnit, on toistamiseen ilman syntiä ilmestyvä pelastukseksi niille, jotka häntä odottavat.

\chapter{10}

\par 1 Sillä koska laissa on vain tulevan hyvän varjo, ei itse asiain olemusta, ei se koskaan voi samoilla jokavuotisilla uhreilla, joita he alinomaa kantavat esiin, tehdä niiden tuojia täydellisiksi.
\par 2 Sillä eikö muutoin olisi lakattu niitä uhraamasta, koska näillä, jotka jumalanpalvelustaan toimittavat, kerran puhdistettuina, ei enää olisi ollut mitään tuntoa synneistä?
\par 3 Mutta niissä on jokavuotinen muistutus synneistä.
\par 4 Sillä mahdotonta on, että härkäin ja kauristen veri voi ottaa pois syntejä.
\par 5 Sentähden hän maailmaan tullessaan sanoo: "Uhria ja antia sinä et tahtonut, mutta ruumiin sinä minulle valmistit;
\par 6 polttouhreihin ja syntiuhreihin sinä et mielistynyt.
\par 7 Silloin minä sanoin: 'Katso, minä tulen - kirjakääröön on minusta kirjoitettu - tekemään sinun tahtosi, Jumala'."
\par 8 Kun hän ensin sanoo: "Uhreja ja anteja ja polttouhreja ja syntiuhreja sinä et tahtonut etkä niihin mielistynyt", vaikka niitä lain mukaan uhrataankin,
\par 9 sanoo hän sitten: "Katso, minä tulen tekemään sinun tahtosi". Hän poistaa ensimmäisen, pystyttääkseen toisen.
\par 10 Ja tämän tahdon perusteella me olemme pyhitetyt Jeesuksen Kristuksen ruumiin uhrilla kerta kaikkiaan.
\par 11 Ja kaikki papit seisovat päivä päivältä palvelustaan toimittamassa ja usein uhraamassa, aina samoja uhreja, jotka eivät ikinä voi syntejä poistaa;
\par 12 mutta tämä on, uhrattuaan yhden ainoan uhrin syntien edestä, ainiaaksi istuutunut Jumalan oikealle puolelle,
\par 13 ja odottaa nyt vain, kunnes hänen vihollisensa pannaan hänen jalkojensa astinlaudaksi.
\par 14 Sillä hän on yhdellä ainoalla uhrilla ainiaaksi tehnyt täydellisiksi ne, jotka pyhitetään.
\par 15 Todistaahan sen meille myös Pyhä Henki; sillä sanottuaan:
\par 16 "Tämä on se liitto, jonka minä näiden päivien jälkeen teen heidän kanssaan", sanoo Herra: "Minä panen lakini heidän sydämiinsä ja kirjoitan ne heidän mieleensä";
\par 17 ja: "heidän syntejänsä ja laittomuuksiansa en minä enää muista".
\par 18 Mutta missä nämä ovat anteeksi annetut, siinä ei uhria synnin edestä enää tarvita.
\par 19 Koska meillä siis, veljet, on luja luottamus siihen, että meillä Jeesuksen veren kautta on pääsy kaikkeinpyhimpään,
\par 20 jonka pääsyn hän on vihkinyt meille uudeksi ja eläväksi tieksi, joka käy esiripun, se on hänen lihansa, kautta,
\par 21 ja koska meillä on "suuri pappi, Jumalan huoneen haltija",
\par 22 niin käykäämme esiin totisella sydämellä, täydessä uskon varmuudessa, sydän vihmottuna puhtaaksi pahasta omastatunnosta ja ruumis puhtaalla vedellä pestynä;
\par 23 pysykäämme järkähtämättä toivon tunnustuksessa, sillä hän, joka antoi lupauksen, on uskollinen;
\par 24 ja valvokaamme toinen toistamme rohkaisuksi toisillemme rakkauteen ja hyviin tekoihin;
\par 25 älkäämme jättäkö omaa seurakunnankokoustamme, niinkuin muutamien on tapana, vaan kehoittakaamme toisiamme, sitä enemmän, kuta enemmän näette tuon päivän lähestyvän.
\par 26 Sillä jos me tahallamme teemme syntiä, päästyämme totuuden tuntoon, niin ei ole enää uhria meidän syntiemme edestä,
\par 27 vaan hirmuinen tuomion odotus ja tulen kiivaus, joka on kuluttava vastustajat.
\par 28 Joka hylkää Mooseksen lain, sen pitää armotta kahden tai kolmen todistajan todistuksen nojalla kuoleman:
\par 29 kuinka paljoa ankaramman rangaistuksen luulettekaan sen ansaitsevan, joka tallaa jalkoihinsa Jumalan Pojan ja pitää epäpyhänä liiton veren, jossa hänet on pyhitetty, ja pilkkaa armon Henkeä!
\par 30 Sillä me tunnemme hänet, joka on sanonut: "Minun on kosto, minä olen maksava"; ja vielä: "Herra on tuomitseva kansansa".
\par 31 Hirmuista on langeta elävän Jumalan käsiin.
\par 32 Mutta muistakaa entisiä päiviä, jolloin te, valistetuiksi tultuanne, kestitte monet kärsimysten kilvoitukset,
\par 33 kun te toisaalta olitte häväistysten ja ahdistusten alaisina, kaikkien katseltavina, toisaalta taas tulitte niiden osaveljiksi, joiden kävi samalla tavalla.
\par 34 Sillä vankien kanssa te olette kärsineet ja ilolla pitäneet hyvänänne omaisuutenne ryöstön, tietäen, että teillä on parempi tavara, joka pysyy.
\par 35 Älkää siis heittäkö pois uskallustanne, jonka palkka on suuri.
\par 36 Sillä te tarvitsette kestäväisyyttä, tehdäksenne Jumalan tahdon ja saadaksenne sen, mikä luvattu on.
\par 37 Sillä "vähän, aivan vähän aikaa vielä, niin tulee hän, joka tuleva on, eikä viivyttele;
\par 38 mutta minun vanhurskaani on elävä uskosta, ja jos hän vetäytyy pois, ei minun sieluni mielisty häneen".
\par 39 Mutta me emme ole niitä, jotka vetäytyvät pois omaksi kadotuksekseen, vaan niitä, jotka uskovat sielunsa pelastukseksi.

\chapter{11}

\par 1 Mutta usko on luja luottamus siihen, mitä toivotaan, ojentautuminen sen mukaan, mikä ei näy.
\par 2 Sillä sen kautta saivat vanhat todistuksen.
\par 3 Uskon kautta me ymmärrämme, että maailma on rakennettu Jumalan sanalla, niin että se, mikä nähdään, ei ole syntynyt näkyväisestä.
\par 4 Uskon kautta uhrasi Aabel Jumalalle paremman uhrin kuin Kain, ja uskon kautta hän sai todistuksen, että hän oli vanhurskas, kun Jumala antoi todistuksen hänen uhrilahjoistaan; ja uskonsa kautta hän vielä kuoltuaankin puhuu.
\par 5 Uskon kautta otettiin Eenok pois, näkemättä kuolemaa, "eikä häntä enää ollut, koska Jumala oli ottanut hänet pois". Sillä ennen poisottamistaan hän oli saanut todistuksen, että hän oli otollinen Jumalalle.
\par 6 Mutta ilman uskoa on mahdoton olla otollinen; sillä sen, joka Jumalan tykö tulee, täytyy uskoa, että Jumala on ja että hän palkitsee ne, jotka häntä etsivät.
\par 7 Uskon kautta rakensi Nooa, saatuaan ilmoituksen siitä, mikä ei vielä näkynyt, pyhässä pelossa arkin perhekuntansa pelastukseksi; ja uskonsa kautta hän tuomitsi maailman, ja hänestä tuli sen vanhurskauden perillinen, joka uskosta tulee.
\par 8 Uskon kautta oli Aabraham kuuliainen, kun hänet kutsuttiin lähtemään siihen maahan, jonka hän oli saava perinnöksi, ja hän lähti tietämättä, minne oli saapuva.
\par 9 Uskon kautta hän eli muukalaisena lupauksen maassa niinkuin vieraassa maassa, asuen teltoissa Iisakin ja Jaakobin kanssa, jotka olivat saman lupauksen perillisiä;
\par 10 sillä hän odotti sitä kaupunkia, jolla on perustukset ja jonka rakentaja ja luoja on Jumala.
\par 11 Uskon kautta sai Saarakin voimaa suvun perustamiseen, vieläpä yli-ikäisenä, koska hän piti luotettavana sen, joka oli antanut lupauksen.
\par 12 Sentähden syntyikin yhdestä miehestä, vieläpä kuolettuneesta, niin suuri paljous, kuin on tähtiä taivaalla ja kuin meren rannalla hiekkaa, epälukuisesti.
\par 13 Uskossa nämä kaikki kuolivat eivätkä luvattua saavuttaneet, vaan kaukaa he olivat sen nähneet ja sitä tervehtineet ja tunnustaneet olevansa vieraita ja muukalaisia maan päällä.
\par 14 Sillä jotka näin puhuvat, ilmaisevat etsivänsä isänmaata.
\par 15 Ja jos he olisivat tarkoittaneet sitä maata, josta olivat lähteneet, niin olisihan heillä ollut tilaisuus palata takaisin;
\par 16 mutta nyt he pyrkivät parempaan, se on taivaalliseen. Sentähden Jumala ei heitä häpeä, vaan sallii kutsua itseään heidän Jumalaksensa; sillä hän on valmistanut heille kaupungin.
\par 17 Uskon kautta uhrasi Aabraham, koetukselle pantuna, Iisakin, uhrasi ainoan poikansa, hän, joka oli lupaukset vastaanottanut
\par 18 ja jolle oli sanottu: "Iisakista sinä saat nimellesi jälkeläisen",
\par 19 sillä hän päätti, että Jumala on voimallinen kuolleistakin herättämään; ja sen vertauskuvana hän saikin hänet takaisin.
\par 20 Uskon kautta antoi Iisak Jaakobille ja Eesaulle siunauksen, joka koski tulevaisiakin.
\par 21 Uskon kautta siunasi Jaakob kuollessaan kumpaisenkin Joosefin pojista ja rukoili sauvansa päähän nojaten.
\par 22 Uskon kautta muistutti Joosef loppunsa lähetessä Israelin lasten lähdöstä ja antoi määräyksen luistansa.
\par 23 Uskon kautta pitivät Mooseksen vanhemmat häntä heti hänen syntymänsä jälkeen kätkössä kolme kuukautta, sillä he näkivät, että lapsi oli ihana; eivätkä he peljänneet kuninkaan käskyä.
\par 24 Uskon kautta kieltäytyi Mooses suureksi tultuaan kantamasta faraon tyttären pojan nimeä.
\par 25 Hän otti mieluummin kärsiäkseen vaivaa yhdessä Jumalan kansan kanssa kuin saadakseen synnistä lyhytaikaista nautintoa,
\par 26 katsoen "Kristuksen pilkan" suuremmaksi rikkaudeksi kuin Egyptin aarteet; sillä hän käänsi katseensa palkintoa kohti.
\par 27 Uskon kautta hän jätti Egyptin pelkäämättä kuninkaan vihaa; sillä koska hän ikäänkuin näki sen, joka on näkymätön, niin hän kesti.
\par 28 Uskon kautta hän pani toimeen pääsiäisenvieton ja verensivelyn, ettei esikoisten surmaaja koskisi heihin.
\par 29 Uskon kautta he kulkivat poikki Punaisen meren ikäänkuin kuivalla maalla; jota yrittäessään egyptiläiset hukkuivat.
\par 30 Uskon kautta kaatuivat Jerikon muurit, sittenkuin niiden ympäri oli kuljettu seitsemän päivää.
\par 31 Uskon kautta pelastui portto Raahab joutumasta perikatoon yhdessä uppiniskaisten kanssa, kun oli, rauha mielessään, ottanut vakoojat luoksensa.
\par 32 Ja mitä minä vielä sanoisin? Sillä minulta loppuisi aika, jos kertoisin Gideonista, Baarakista, Simsonista, Jeftasta, Daavidista ja Samuelista ja profeetoista,
\par 33 jotka uskon kautta kukistivat valtakuntia, pitivät vanhurskautta voimassa, saivat kokea lupauksien toteutumista, tukkivat jalopeurain kidat,
\par 34 sammuttivat tulen voiman, pääsivät miekanteriä pakoon, voimistuivat heikkoudesta, tulivat väkeviksi sodassa, ajoivat pakoon muukalaisten sotajoukot.
\par 35 On ollut vaimoja, jotka ylösnousemuksen kautta ovat saaneet kuolleensa takaisin. Toiset ovat antaneet kiduttaa itseään eivätkä ole ottaneet vastaan vapautusta, että saisivat paremman ylösnousemuksen;
\par 36 toiset taas ovat saaneet kokea pilkkaa ja ruoskimista, vieläpä kahleita ja vankeutta;
\par 37 heitä on kivitetty, kiusattu, rikki sahattu, miekalla surmattu; he ovat kierrelleet ympäri lampaannahoissa ja vuohennahoissa, puutteenalaisina, ahdistettuina, pahoinpideltyinä -
\par 38 he, jotka olivat liian hyviä tälle maailmalle -; he ovat harhailleet erämaissa ja vuorilla ja luolissa ja maakuopissa.
\par 39 Ja vaikka nämä kaikki uskon kautta olivat todistuksen saaneet, eivät he kuitenkaan saavuttaneet sitä, mikä oli luvattu;
\par 40 sillä Jumala oli varannut meitä varten jotakin parempaa, etteivät he ilman meitä pääsisi täydellisyyteen.

\chapter{12}

\par 1 Sentähden, kun meillä on näin suuri pilvi todistajia ympärillämme, pankaamme mekin pois kaikki, mikä meitä painaa, ja synti, joka niin helposti meidät kietoo, ja juoskaamme kestävinä edessämme olevassa kilvoituksessa,
\par 2 silmät luotuina uskon alkajaan ja täyttäjään, Jeesukseen, joka hänelle tarjona olevan ilon sijasta kärsi ristin, häpeästä välittämättä, ja istui Jumalan valtaistuimen oikealle puolelle.
\par 3 Ajatelkaa häntä, joka syntisiltä on saanut kärsiä sellaista vastustusta itseänsä kohtaan, ettette väsyisi ja menettäisi toivoanne.
\par 4 Ette vielä ole verille asti tehneet vastarintaa, taistellessanne syntiä vastaan,
\par 5 ja te olette unhottaneet kehoituksen, joka puhuu teille niinkuin lapsille: "Poikani, älä pidä halpana Herran kuritusta, äläkä menetä toivoasi, kun hän sinua nuhtelee;
\par 6 sillä jota Herra rakastaa, sitä hän kurittaa; ja hän ruoskii jokaista lasta, jonka hän ottaa huomaansa".
\par 7 Kuritukseksenne te kärsitte; Jumala kohtelee teitä niinkuin lapsia. Sillä mikä on se lapsi, jota isä ei kurita?
\par 8 Mutta jos te olette ilman kuritusta, josta kaikki ovat osallisiksi tulleet, silloinhan te olette äpäriä ettekä lapsia.
\par 9 Ja vielä: meillä oli ruumiilliset isämme kurittajina, ja heitä me kavahdimme; emmekö paljoa ennemmin olisi alamaiset henkien Isälle, että eläisimme?
\par 10 Sillä nuo kurittivat meitä vain muutamia päiviä varten, oman ymmärryksensä mukaan, mutta tämä kurittaa meitä tosi parhaaksemme, että me pääsisimme osallisiksi hänen pyhyydestään.
\par 11 Mikään kuritus ei tosin sillä kertaa näytä olevan iloksi, vaan murheeksi, mutta jälkeenpäin se antaa vanhurskauden rauhanhedelmän niille, jotka sen kautta ovat harjoitetut.
\par 12 Sentähden: "Ojentakaa hervonneet kätenne ja rauenneet polvenne";
\par 13 ja: "tehkää polut suoriksi jaloillenne", ettei ontuvan jalka nyrjähtäisi, vaan ennemmin parantuisi.
\par 14 Pyrkikää rauhaan kaikkien kanssa ja pyhitykseen, sillä ilman sitä ei kukaan ole näkevä Herraa;
\par 15 ja pitäkää huoli siitä, ettei kukaan jää osattomaksi Jumalan armosta, "ettei mikään katkeruuden juuri pääse kasvamaan ja tekemään häiriötä", ja monet sen kautta tule saastutetuiksi,
\par 16 ja ettei kukaan olisi haureellinen tahi epäpyhä niinkuin Eesau, joka yhdestä ateriasta myi esikoisuutensa.
\par 17 Sillä te tiedätte, että hänet sittemminkin, kun hän tahtoi päästä siunausta perimään, hyljättiin; sillä hän ei löytänyt tilaa peruutukselle, vaikka hän kyynelin sitä pyysi.
\par 18 Sillä te ette ole käyneet sen vuoren tykö, jota voidaan käsin koskea ja joka tulessa palaa, ettekä synkeyden, ette pimeyden, ette myrskyn,
\par 19 ette pasunan kaiun ettekä äänen tykö, joka puhui niin, että ne, jotka sen kuulivat, pyysivät, ettei heille enää puhuttaisi;
\par 20 sillä he eivät voineet kestää tätä käskyä: "Koskettakoon vuorta vaikka eläinkin, se kivitettäköön";
\par 21 ja niin hirmuinen oli se näky, että Mooses sanoi: "Minä olen peljästynyt ja vapisen";
\par 22 vaan te olette käyneet Siionin vuoren tykö ja elävän Jumalan kaupungin, taivaallisen Jerusalemin tykö, ja kymmenien tuhansien enkelien tykö,
\par 23 taivaissa kirjoitettujen esikoisten juhlajoukon ja seurakunnan tykö, ja tuomarin tykö, joka on kaikkien Jumala, ja täydellisiksi tulleitten vanhurskasten henkien tykö,
\par 24 ja uuden liiton välimiehen, Jeesuksen, tykö, ja vihmontaveren tykö, joka puhuu parempaa kuin Aabelin veri.
\par 25 Katsokaa, ettette torju luotanne häntä, joka puhuu; sillä jos nuo, jotka torjuivat luotaan hänet, joka ilmoitti Jumalan tahdon maan päällä, eivät voineet päästä pakoon, niin paljoa vähemmän me, jos käännymme pois hänestä, joka ilmoittaa sen taivaista.
\par 26 Silloin hänen äänensä järkytti maata, mutta nyt hän on luvannut sanoen: "Vielä kerran minä liikutan maan, jopa taivaankin".
\par 27 Mutta tuo "vielä kerran" osoittaa, että ne, mitkä järkkyvät, koska ovat luotuja, tulevat muuttumaan, että ne, jotka eivät järky, pysyisivät.
\par 28 Sentähden, koska me saamme valtakunnan, joka ei järky, olkaamme kiitolliset ja siten palvelkaamme Jumalaa, hänelle mielihyväksi, pyhällä arkuudella ja pelolla;
\par 29 sillä meidän Jumalamme on kuluttavainen tuli.

\chapter{13}

\par 1 Pysyköön veljellinen rakkaus.
\par 2 Älkää unhottako vieraanvaraisuutta; sillä sitä osoittamalla muutamat ovat tietämättään saaneet pitää enkeleitä vierainaan.
\par 3 Muistakaa vankeja, niinkuin olisitte itsekin heidän kanssaan vangittuina; muistakaa pahoinpideltyjä, sillä onhan teillä itsellännekin ruumis.
\par 4 Avioliitto pidettäköön kunniassa kaikkien kesken, ja aviovuode saastuttamatonna; sillä haureelliset ja avionrikkojat Jumala tuomitsee.
\par 5 Älkää olko vaelluksessanne ahneita; tyytykää siihen, mitä teillä on; sillä hän itse on sanonut: "En minä sinua hylkää enkä sinua jätä";
\par 6 niin että me turvallisin mielin sanomme: "Herra on minun auttajani, en minä pelkää; mitä voi ihminen minulle tehdä?"
\par 7 Muistakaa johtajianne, jotka ovat puhuneet teille Jumalan sanaa; katsokaa, kuinka heidän vaelluksensa on päättynyt, ja seuratkaa heidän uskoansa.
\par 8 Jeesus Kristus on sama eilen ja tänään ja iankaikkisesti.
\par 9 Älkää antako monenlaisten ja vieraiden oppien itseänne vietellä; sillä on hyvä, että sydän saa vahvistusta armosta eikä ruuista, joista ne, jotka niitä menoja ovat noudattaneet, eivät ole mitään hyötyneet.
\par 10 Meillä on uhrialttari, josta majassa palvelevilla ei ole valta syödä.
\par 11 Sillä niiden eläinten ruumiit, joiden veren ylimmäinen pappi syntien sovitukseksi kantaa kaikkeinpyhimpään, poltetaan ulkopuolella leirin.
\par 12 Sentähden myös Jeesus, pyhittääkseen omalla verellänsä kansan, kärsi portin ulkopuolella.
\par 13 Niin menkäämme siis hänen tykönsä "ulkopuolelle leirin", hänen pilkkaansa kantaen;
\par 14 sillä ei meillä ole täällä pysyväistä kaupunkia, vaan tulevaista me etsimme.
\par 15 Uhratkaamme siis hänen kauttansa Jumalalle joka aika kiitosuhria, se on: niiden huulten hedelmää, jotka hänen nimeänsä ylistävät.
\par 16 Mutta älkää unhottako tehdä hyvää ja jakaa omastanne, sillä senkaltaisiin uhreihin Jumala mielistyy.
\par 17 Olkaa kuuliaiset johtajillenne ja tottelevaiset, sillä he valvovat teidän sielujanne niinkuin ne, joiden on tehtävä tili, että he voisivat tehdä sitä ilolla eikä huokaillen; sillä se ei ole teille hyödyllistä.
\par 18 Rukoilkaa meidän edestämme; sillä me tiedämme, että meillä on hyvä omatunto, koska tahdomme kaikessa hyvin vaeltaa.
\par 19 Vielä hartaammin kehoitan teitä näin tekemään, että minut sitä pikemmin annettaisiin teille takaisin.
\par 20 Mutta rauhan Jumala, joka on kuolleista nostanut hänet, joka iankaikkisen liiton veren kautta on se suuri lammasten paimen, meidän Herramme Jeesuksen,
\par 21 hän tehköön teidät kykeneviksi kaikkeen hyvään, voidaksenne toteuttaa hänen tahtonsa, ja vaikuttakoon teissä sen, mikä on hänelle otollista, Jeesuksen Kristuksen kautta; hänelle kunnia aina ja iankaikkisesti! Amen.
\par 22 Minä pyydän teitä, veljet: kestäkää tämä kehoituksen sana; sillä lyhykäisesti minä olen teille kirjoittanut.
\par 23 Tietäkää, että veljemme Timoteus on päästetty vapaaksi; ja jos hän pian tulee, saan minä hänen kanssaan nähdä teidät.
\par 24 Sanokaa tervehdys kaikille johtajillenne ja kaikille pyhille. Tervehdyksen lähettävät teille ne, jotka ovat Italiasta.
\par 25 Armo olkoon kaikkien teidän kanssanne.


\end{document}