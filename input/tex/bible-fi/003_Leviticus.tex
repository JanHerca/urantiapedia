\begin{document}

\title{Kolmas Mooseksen kirja}


\chapter{1}

\par 1 Ja Herra kutsui Mooseksen ja puhui hänelle sisältä ilmestysmajasta sanoen:
\par 2 "Puhu israelilaisille ja sano heille: Jos joku teistä tahtoo tuoda uhrilahjan Herralle, niin tuokaa uhrilahjanne karjasta, joko raavaista tai lampaista.
\par 3 Jos hänen uhrilahjansa on raavaspolttouhri, tuokoon virheettömän urospuolen; ilmestysmajan ovelle hän tuokoon sen, että Herran mielisuosio tulisi hänen osaksensa.
\par 4 Ja hän laskekoon kätensä polttouhriteuraan pään päälle; niin se on oleva otollinen ja tuottava hänelle sovituksen.
\par 5 Ja hän teurastakoon mullikan Herran kasvojen edessä, ja papit, Aaronin pojat, tuokoot veren ja vihmokoot veren ympärinsä alttarille, joka on ilmestysmajan ovella.
\par 6 Ja hän nylkeköön polttouhriteuraan ja leikelköön sen määräkappaleiksi.
\par 7 Ja pappi Aaronin pojat tehkööt tulen alttarille ja pankoot halkoja tuleen.
\par 8 Ja papit, Aaronin pojat, asettakoot kappaleet ynnä pään ja rasvan halkojen päälle, jotka ovat tulessa alttarilla.
\par 9 Mutta sisälmykset ja jalat hän pesköön vedessä, ja pappi polttakoon kaiken alttarilla; se on polttouhri, suloisesti tuoksuva uhri Herralle.
\par 10 Ja jos hänen uhrilahjansa, joka on tarkoitettu polttouhriksi, on otettu pikkukarjasta, lampaista tai vuohista, tuokoon virheettömän urospuolen.
\par 11 Ja teurastakoon sen alttarin pohjoissivulla, Herran edessä; ja papit, Aaronin pojat, vihmokoot sen veren alttarille ympärinsä.
\par 12 Ja hän leikelköön sen määräkappaleiksi, ja pappi asettakoon ne ynnä pään ja rasvan halkojen päälle, jotka ovat tulessa alttarilla.
\par 13 Mutta sisälmykset ja jalat hän pesköön vedessä, ja pappi tuokoon sen kaiken ja polttakoon alttarilla; se on polttouhri, suloisesti tuoksuva uhri Herralle.
\par 14 Ja jos hänen uhrilahjansa Herralle on lintupolttouhri, tuokoon uhrilahjanaan metsäkyyhkysiä tai kyyhkysenpoikia.
\par 15 Ja pappi tuokoon sen alttarille, vääntäköön siltä niskat poikki ja polttakoon sen alttarilla, ja puserrettakoon sen veri alttarin seinään.
\par 16 Ja hän ottakoon sen kuvun rapoineen pois ja heittäköön sen alttarin viereen itäpuolelle, tuhkaläjälle.
\par 17 Sitten hän reväisköön sen auki siipien kohdalta, niitä kuitenkaan irroittamatta, ja pappi polttakoon sen alttarilla halkojen päällä, jotka ovat tulessa; se on polttouhri, suloisesti tuoksuva uhri Herralle."

\chapter{2}

\par 1 "Jos joku tahtoo tuoda Herralle lahjaksi ruokauhrin, olkoon hänen uhrilahjanaan lestyjä jauhoja; ja hän vuodattakoon siihen öljyä ja pankoon sen päälle suitsuketta
\par 2 ja vieköön sen papeille, Aaronin pojille, ja pappi ottakoon kourallisen niitä jauhoja ja sitä öljyä ynnä kaiken suitsukkeen ja polttakoon tämän alttariuhriosan alttarilla suloisesti tuoksuvana uhrina Herralle.
\par 3 Mutta se, mikä jää tähteeksi ruokauhrista, olkoon Aaronin ja hänen poikiensa oma; se on korkeasti-pyhää, Herran uhria.
\par 4 Mutta jos tahdot tuoda ruokauhrilahjan uunissa paistetusta, olkoon uhrinasi lestyistä jauhoista valmistetut, öljyyn leivotut happamattomat kakut tai öljyllä voidellut happamattomat ohukaiset.
\par 5 Jos taas uhrilahjasi on ruokauhri, joka leivinlevyllä paistetaan, olkoon se valmistettu lestyistä jauhoista, öljyyn leivottu ja happamaton.
\par 6 Paloittele se palasiksi ja vuodata öljyä sen päälle; se on ruokauhri.
\par 7 Ja jos uhrilahjasi on ruokauhri, joka paistetaan pannussa, valmistettakoon se lestyistä jauhoista ynnä öljystä.
\par 8 Ja vie ruokauhri, joka näin on valmistettu, Herralle; se vietäköön papille, ja hän tuokoon sen alttarin ääreen.
\par 9 Ja pappi erottakoon ruokauhrista alttariuhriosan ja polttakoon sen alttarilla suloisesti tuoksuvana uhrina Herralle.
\par 10 Mutta se, mikä jää tähteeksi ruokauhrista, olkoon Aaronin ja hänen poikiensa oma; se on korkeasti-pyhää, Herran uhria.
\par 11 Mitään ruokauhria, jonka te tuotte Herralle, älköön valmistettako happamesta; älkää polttako uhria Herralle mistään happamesta taikinasta tai hunajasta.
\par 12 Uutisuhrilahjana saatte tuoda niitä Herralle, mutta älkööt ne tulko alttarille suloiseksi tuoksuksi.
\par 13 Ja jokainen ruokauhrilahjasi suolaa suolalla, äläkä anna Jumalasi liitonsuolan puuttua ruokauhristasi; jokaiseen uhrilahjaasi sinun on tuotava suolaa.
\par 14 Mutta jos tahdot tuoda uutisesta ruokauhrin Herralle, niin tuo uutisruokauhrinasi tulessa paahdettuja tähkäpäitä tai survottuja jyviä tuleentumattomasta viljasta.
\par 15 Ja vuodata siihen öljyä ja pane siihen suitsuketta; se on ruokauhri.
\par 16 Ja pappi polttakoon alttariuhriosan survotuista jyvistä ja öljystä ynnä kaiken suitsukkeen uhrina Herralle."

\chapter{3}

\par 1 "Jos hänen uhrilahjansa on yhteysuhri, tuokoon hän, jos hän tuo raavaskarjasta härän tai lehmän, virheettömän eläimen Herran eteen.
\par 2 Ja hän laskekoon kätensä uhriteuraansa pään päälle ja teurastakoon sen ilmestysmajan ovella; ja papit, Aaronin pojat, vihmokoot veren alttarille ympärinsä.
\par 3 Ja hän tuokoon yhteysuhrista uhrina Herralle sisälmyksiä peittävän rasvan ja kaiken sisälmysten rasvan
\par 4 ja molemmat munuaiset ynnä niiden päällä lantiolihaksissa olevan rasvan ja maksanlisäkkeen, joka on irroitettava munuaisten luota.
\par 5 Ja Aaronin pojat polttakoot sen alttarilla polttouhrin päällä, joka on halkojen päällä tulessa, suloisesti tuoksuvana uhrina Herralle.
\par 6 Mutta jos hän tuo uhrilahjanaan Herralle yhteysuhriksi pikkukarjasta urospuolen tai naaraspuolen, niin tuokoon virheettömän eläimen.
\par 7 Jos hän tuo uhrilahjanaan lampaan, niin tuokoon sen Herran eteen
\par 8 ja laskekoon kätensä uhriteuraansa pään päälle ja teurastakoon sen ilmestysmajan ovella, ja Aaronin pojat vihmokoot sen veren alttarille ympärinsä.
\par 9 Ja hän tuokoon yhteysuhrista uhrina Herralle sen rasvan, koko rasvahännän, joka on irroitettava häntänikamista, sisälmyksiä peittävän rasvan ja kaiken sisälmysten rasvan
\par 10 ja molemmat munuaiset ynnä niiden päällä lantiolihaksissa olevan rasvan ja maksanlisäkkeen, joka on irroitettava munuaisten luota.
\par 11 Ja pappi polttakoon sen alttarilla uhriruokana Herralle.
\par 12 Ja jos hänen uhrilahjansa on vuohi, tuokoon sen Herran eteen
\par 13 ja laskekoon kätensä sen pään päälle ja teurastakoon sen ilmestysmajan ovella; ja Aaronin pojat vihmokoot sen veren alttarille ympärinsä.
\par 14 Ja hän tuokoon siitä uhrilahjanaan Herralle uhriksi sisälmyksiä peittävän rasvan ja kaiken sisälmysten rasvan
\par 15 ja molemmat munuaiset ynnä niiden päällä lantiolihaksissa olevan rasvan ja maksanlisäkkeen, joka on irroitettava munuaisten luota.
\par 16 Ja pappi polttakoon ne alttarilla uhriruokana, suloisesti tuoksuvana uhrina; kaikki rasva olkoon Herran oma.
\par 17 Tämä olkoon teille ikuinen säädös sukupolvesta sukupolveen, missä asuttekin: mitään rasvaa tai verta älkää syökö."

\chapter{4}

\par 1 Ja Herra puhui Moosekselle sanoen:
\par 2 "Puhu israelilaisille ja sano: Jos joku erehdyksestä rikkoo jotakuta Herran käskyä vastaan ja tekee jotakin, mitä ei saa tehdä,
\par 3 niin, jos voideltu pappi tekee rikkomuksen ja saattaa kansan vikapääksi, tuokoon rikkomuksensa tähden, jonka hän on tehnyt, virheettömän mullikan Herralle syntiuhriksi.
\par 4 Ja vieköön mullikan ilmestysmajan ovelle Herran eteen ja laskekoon kätensä mullikan pään päälle ja teurastakoon mullikan Herran edessä.
\par 5 Ja ottakoon se voideltu pappi mullikan verta ja vieköön sen ilmestysmajaan,
\par 6 ja pappi kastakoon sormensa vereen ja pirskoittakoon verta seitsemän kertaa Herran edessä, pyhäkön esiripun edessä.
\par 7 Ja pappi sivelköön sitä verta alttarin sarviin, jolla poltetaan hyvänhajuista suitsutusta Herran edessä ja joka on ilmestysmajassa; mullikan kaiken muun veren hän vuodattakoon ilmestysmajan oven edessä olevan polttouhrialttarin juurelle.
\par 8 Ja kaiken syntiuhrimullikan rasvan hän erottakoon siitä pois, sekä sisälmyksiä peittävän rasvan että kaiken sisälmysten rasvan,
\par 9 ja molemmat munuaiset ynnä niiden päällä lantiolihaksissa olevan rasvan ja maksanlisäkkeen, joka on irroitettava munuaisten luota,
\par 10 samalla tavalla kuin se erotetaan pois yhteysuhrihärästä; ja pappi polttakoon ne polttouhrialttarilla.
\par 11 Mutta mullikan nahan ja kaiken lihan ynnä pään ja jalat, sisälmykset ja ravan,
\par 12 koko mullikan, hän vieköön leirin ulkopuolelle puhtaaseen paikkaan, johon tuhka heitetään, ja polttakoon sen halkojen päällä tulessa; siinä paikassa, johon tuhka heitetään, se poltettakoon.
\par 13 Ja jos koko Israelin seurakunta erehdyksestä tekee rikkomuksen ja seurakunta ei siitä tiedä ja he rikkomalla jotakuta Herran käskyä vastaan ovat tehneet sellaista, mitä ei saa tehdä, ja niin joutuneet vikapäiksi,
\par 14 ja jos sitten rikkomus, jonka he ovat tehneet sitä vastaan, tulee tunnetuksi, niin tuokoon seurakunta mullikan syntiuhriksi ja vieköön sen ilmestysmajan eteen.
\par 15 Ja seurakunnan vanhimmat laskekoot kätensä mullikan pään päälle Herran edessä, ja mullikka teurastettakoon Herran edessä.
\par 16 Ja voideltu pappi vieköön mullikan verta ilmestysmajaan.
\par 17 Ja pappi kastakoon sormensa vereen ja pirskoittakoon sitä seitsemän kertaa Herran edessä, esiripun edessä.
\par 18 Ja hän sivelköön sitä verta alttarin sarviin, joka on Herran edessä ilmestysmajassa; kaiken muun veren hän vuodattakoon ilmestysmajan oven edessä olevan polttouhrialttarin juurelle.
\par 19 Ja hän erottakoon siitä pois kaiken rasvan ja polttakoon sen alttarilla
\par 20 ja tehköön tälle mullikalle, niinkuin hän teki syntiuhrimullikalle; samalla tavalla hän sille tehköön. Kun pappi näin on toimittanut heille sovituksen, annetaan heille anteeksi.
\par 21 Ja mullikan hän vieköön leirin ulkopuolelle ja polttakoon sen, niinkuin hän poltti edellisen mullikan; se on seurakunnan syntiuhri.
\par 22 Jos päämies tekee rikkomuksen ja erehdyksestä rikkomalla jotakuta Herran, Jumalansa, käskyä vastaan tekee sellaista, mitä ei saa tehdä, ja niin joutuu vikapääksi,
\par 23 ja hän sitten saa tietää rikkomuksensa, jonka hän on tehnyt, niin tuokoon uhrilahjanaan kauriin, virheettömän urospuolen,
\par 24 ja laskekoon kätensä kauriin pään päälle ja teurastakoon sen siinä paikassa, jossa polttouhriteuraat Herran edessä teurastetaan; se on hänen syntiuhrinsa.
\par 25 Ja pappi ottakoon syntiuhrin verta sormeensa ja sivelköön sitä polttouhrialttarin sarviin; muun veren hän vuodattakoon polttouhrialttarin juurelle.
\par 26 Mutta kaiken sen rasvan hän polttakoon alttarilla, niinkuin yhteysuhrin rasvan. Kun pappi näin on toimittanut hänelle sovituksen hänen rikkomuksestansa, annetaan hänelle anteeksi.
\par 27 Jos joku rahvaasta erehdyksestä rikkoo jotakuta Herran käskyä vastaan tekemällä sellaista, mitä ei saa tehdä, ja niin joutuu vikapääksi,
\par 28 ja hän sitten saa tietää rikkomuksensa, jonka hän on tehnyt, niin tuokoon rikkomuksensa tähden, jonka hän on tehnyt, uhrilahjanaan vuohen, virheettömän naaraspuolen,
\par 29 ja laskekoon kätensä tämän syntiuhriteuraan pään päälle ja teurastakoon sen polttouhripaikalla.
\par 30 Ja pappi ottakoon sen verta sormeensa ja sivelköön sitä polttouhrialttarin sarviin; kaiken muun veren hän vuodattakoon alttarin juurelle.
\par 31 Ja hän irroittakoon kaiken sen rasvan, niinkuin rasva irroitetaan yhteysuhrista, ja pappi polttakoon sen alttarilla suloisena tuoksuna Herralle. Kun pappi näin on toimittanut hänelle sovituksen, annetaan hänelle anteeksi.
\par 32 Mutta jos hän tuo uhrilahjanaan karitsan syntiuhriksi, niin tuokoon virheettömän uuhen
\par 33 ja laskekoon kätensä tämän syntiuhriteuraan pään päälle ja teurastakoon sen syntiuhriksi siinä paikassa, jossa polttouhrieläin teurastetaan.
\par 34 Ja pappi ottakoon syntiuhrin verta sormeensa ja sivelköön polttouhrialttarin sarviin; kaiken muun veren hän vuodattakoon alttarin juurelle.
\par 35 Ja kaiken sen rasvan hän irroittakoon, niinkuin yhteysuhrilampaan rasva irroitetaan, ja pappi polttakoon sen alttarilla Herran uhrin päällä. Kun pappi on toimittanut hänelle sovituksen hänen rikkomuksestansa, jonka hän on tehnyt, annetaan hänelle anteeksi."

\chapter{5}

\par 1 "Jos joku rikkoo siten, että hän, vaikka kuulee vannotuksen ja voisi olla todistajana, joko hän on ollut silminnäkijänä tahi muuten saanut asiasta tietää, ei kuitenkaan ilmoita sitä ja niin joutuu syynalaiseksi;
\par 2 tahi jos joku tietämättään koskee johonkin saastaiseen, mihin tahansa, joko saastaisen metsäeläimen raatoon tai saastaisen kotieläimen raatoon tai saastaisen matelijan raatoon, ja on siten tullut saastaiseksi ja vikapääksi;
\par 3 tahi jos hän tietämättään koskee ihmisen saastaan, olipa se mitä saastaa tahansa, josta tulee saastaiseksi, mutta sitten huomaa sen ja joutuu vikapääksi;
\par 4 tahi jos joku, ajattelemattomasti puhuen huulillansa, tietämättään vannoo tekevänsä jotakin, pahaa tai hyvää - vannoipa mitä hyvänsä, mitä ihminen saattaa ajattelemattomasti vannoa - mutta sitten huomaa sen ja joutuu johonkin sellaiseen vikapääksi,
\par 5 niin, jos hän on vikapää johonkin sellaiseen, tunnustakoon sen, mitä on rikkonut,
\par 6 ja tuokoon hyvityksenä Herralle rikkomuksesta, jonka hän on tehnyt, naaraspuolen pikkukarjasta, uuhen tai vuohen, syntiuhriksi; ja pappi toimittakoon hänelle sovituksen hänen rikkomuksestansa.
\par 7 Mutta jollei hän saa hankituksi sellaista eläintä, niin tuokoon Herralle hyvitykseksi siitä, mitä on rikkonut, kaksi metsäkyyhkystä tai kaksi kyyhkysenpoikaa, toisen syntiuhriksi ja toisen polttouhriksi,
\par 8 ja vieköön ne papille. Tämä uhratkoon ensiksi sen, joka on määrätty syntiuhriksi, vääntäköön siltä niskat poikki, päätä kuitenkaan irroittamatta,
\par 9 ja pirskoittakoon syntiuhrin verta alttarin seinään, ja puserrettakoon se veri, joka jää jäljelle, alttarin juurelle; se on syntiuhri.
\par 10 Ja toisen hän säädetyllä tavalla valmistakoon polttouhriksi. Kun pappi näin on toimittanut hänelle sovituksen hänen rikkomuksestaan, jonka hän on tehnyt, annetaan hänelle anteeksi.
\par 11 Jollei hän saa hankituksi kahta metsäkyyhkystä tai kahta kyyhkysenpoikaa, niin tuokoon uhrilahjanaan sen sovittamiseksi, mitä on rikkonut, kymmenennen osan eefa-mittaa lestyjä jauhoja syntiuhriksi, mutta älköön vuodattako siihen öljyä älköönkä panko sen päälle suitsuketta, sillä se on syntiuhri.
\par 12 Ja vieköön sen papille, ja pappi ottakoon siitä kourallisen, alttariuhriosan, ja polttakoon sen alttarilla Herran uhrin päällä; se on syntiuhri.
\par 13 Kun pappi näin on toimittanut hänelle sovituksen jostakin sellaisesta hänen tekemästään rikkomuksesta, annetaan hänelle anteeksi. Ja papille tulkoon se, mikä ruokauhristakin."
\par 14 Ja Herra puhui Moosekselle sanoen:
\par 15 "Jos joku lankeaa uskottomuuteen ja erehdyksestä rikkoo anastamalla sitä, mikä on Herralle pyhitetty, tuokoon hyvityksenä Herralle vikauhriksi virheettömän oinaan pikkukarjasta, niin monen hopeasekelin arvoisen, pyhäkkösekelin painon mukaan, kuin sinä arvioit.
\par 16 Ja mitä hän on pyhitetystä anastanut itselleen, sen hän korvatkoon ja pankoon siihen lisäksi vielä viidennen osan sen arvosta ja antakoon sen papille. Kun pappi näin on toimittanut hänelle sovituksen uhraamalla vikauhrioinaan, annetaan hänelle anteeksi.
\par 17 Jos joku rikkoo jotakuta Herran käskyä vastaan ja tekee tietämättään sellaista, mitä ei saa tehdä, ja tulee vikapääksi ja joutuu syynalaiseksi,
\par 18 niin tuokoon papille vikauhriksi pikkukarjasta sinun arviosi mukaisen, virheettömän oinaan. Kun pappi on toimittanut hänelle sovituksen siitä erehdyksestä, jonka hän on tietämättään tehnyt, annetaan hänelle anteeksi.
\par 19 Se on vikauhri; sillä hän on tullut vikapääksi Herran edessä."

\chapter{6}

\par 1 Ja Herra puhui Moosekselle sanoen:
\par 2 "Jos joku tekee rikkomuksen ja on uskoton Herraa kohtaan kieltämällä saaneensa lähimmäiseltään, mitä tämä on hänen haltuunsa antanut tai hänen hoitoonsa uskonut, tahi mitä hän itse on väkisin ottanut tai vääryydellä anastanut lähimmäiseltään,
\par 3 tahi jos hän on löytänyt kadotetun esineen ja kieltää sen, tahi vannoo väärin missä asiassa tahansa, jossa ihminen helposti rikkoo,
\par 4 niin hän, jos hän näin on rikkonut ja tullut vikapääksi, antakoon takaisin, mitä hän väkisin on ottanut tai vääryydellä anastanut, tahi mitä hänen haltuunsa oli uskottu, tahi kadotetun esineen, jonka hän oli löytänyt,
\par 5 kaiken, jonka tähden hän oli väärin vannonut; ja korvatkoon sen täyteen määräänsä ja pankoon siihen vielä lisäksi viidennen osan sen arvosta. Sille, jonka oma se on, hän antakoon sen sinä päivänä, jona hän vikauhrinsa toimittaa.
\par 6 Mutta hyvityksenä Herralle hän tuokoon papille vikauhriksi sinun arviosi mukaisen, virheettömän oinaan pikkukarjasta.
\par 7 Kun pappi näin on toimittanut hänelle sovituksen Herran edessä, annetaan hänelle anteeksi kaikki, mitä hän on tehnyt sellaista, josta hän on tullut vikapääksi."
\par 8 Ja Herra puhui Moosekselle sanoen:
\par 9 "Käske Aaronia ja hänen poikiansa ja sano: Tämä on laki polttouhrista. Polttouhri olkoon alttarin liedellä koko yön aamuun asti, ja tulta pidettäköön siten vireillä alttarilla.
\par 10 Ja pappi pukeutukoon pellavapukuunsa ja pukeutukoon pellavakaatioihin peittääkseen häpynsä, ja korjatkoon pois tuhan, joksi tuli on kuluttanut polttouhrin alttarilla, ja pankoon sen alttarin viereen.
\par 11 Ja sitten hän riisukoon vaatteensa ja pukeutukoon toisiin vaatteisiin, ja vieköön tuhan leirin ulkopuolelle puhtaaseen paikkaan.
\par 12 Ja tulta pidettäköön siten vireillä alttarilla älköönkä se koskaan sammuko; ja pappi sytyttäköön alttarilla halot joka aamu ja asettakoon sen päälle polttouhrin ja polttakoon sen päällä yhteysuhrin rasvat.
\par 13 Tuli palakoon aina alttarilla älköönkä koskaan sammuko.
\par 14 Ja tämä on laki ruokauhrista: Aaronin pojat tuokoot sen Herran eteen, alttarin ääreen.
\par 15 Ja pappi erottakoon ruokauhrista kourallisen lestyjä jauhoja ja öljyä ja kaiken suitsukkeen, joka on ruokauhrin päällä, ja polttakoon alttarilla alttariuhriosan suloiseksi tuoksuksi Herralle.
\par 16 Ja mikä siitä jää tähteeksi, sen syököön Aaron poikinensa; happamatonna se syötäköön pyhässä paikassa, ilmestysmajan esipihalla he sen syökööt.
\par 17 Älköön sitä leivottako hapatettuna. Heidän osakseen minä olen sen uhreistani antanut; se on korkeasti-pyhää niinkuin syntiuhri ja vikauhrikin.
\par 18 Jokainen miehenpuoli Aaronin jälkeläisistä syököön siitä. Tämä olkoon heidän ikuinen osuutensa Herran uhreista sukupolvesta sukupolveen. Jokainen, joka niihin koskee, on pyhäkölle pyhitetty."
\par 19 Ja Herra puhui Moosekselle sanoen:
\par 20 "Tämä olkoon Aaronin ja hänen poikiensa uhrilahja, joka heidän on tuotava Herralle voitelupäivänänsä: kymmenes osa eefa-mittaa lestyjä jauhoja jokapäiväiseksi ruokauhriksi, puolet aamulla ja puolet illalla.
\par 21 Leivinlevyllä se valmistettakoon öljyyn leivottuna; tuo se hyvin sotkettuna ja uhraa se palasiksi paloiteltuna ruokauhrina suloiseksi tuoksuksi Herralle.
\par 22 Ja pappi, joka hänen pojistaan on hänen sijaansa voideltu, toimittakoon sen. Se on Herran ikuinen osuus; se poltettakoon kokonaisuhrina.
\par 23 Jokainen papin ruokauhri on kokonaisuhri; älköön sitä syötäkö."
\par 24 Ja Herra puhui Moosekselle sanoen:
\par 25 "Puhu Aaronille ja hänen pojillensa ja sano: Tämä on laki syntiuhrista. Siinä paikassa, missä polttouhri teurastetaan, teurastettakoon myös syntiuhri Herran edessä; ja se on korkeasti-pyhä.
\par 26 Pappi, joka syntiuhrin uhraa, syököön sen; pyhässä paikassa se syötäköön, ilmestysmajan esipihalla.
\par 27 Jokainen, joka sen lihaan koskee, on pyhäkölle pyhitetty, ja jos sen verta on pirskunut vaatteille, niin pese pyhässä paikassa se, mihin sitä on pirskunut.
\par 28 Ja saviastia, jossa se keitetään, rikottakoon; jos se on keitetty vaskiastiassa, hangattakoon astia puhtaaksi ja huuhdottakoon vedellä.
\par 29 Jokainen miehenpuoli papeista saakoon sitä syödä; se on korkeasti-pyhää.
\par 30 Mutta mitään syntiuhria, jonka verta viedään ilmestysmajaan sovituksen toimittamista varten pyhäkössä, älköön syötäkö, vaan se poltettakoon tulessa."

\chapter{7}

\par 1 "Ja tämä on laki vikauhrista. Se on korkeasti-pyhää.
\par 2 Siinä paikassa, missä polttouhri teurastetaan, teurastettakoon myös vikauhri, ja sen veri vihmottakoon alttarille ympärinsä.
\par 3 Ja kaikki sen rasva uhrattakoon, sekä rasvahäntä että sisälmyksiä peittävä rasva,
\par 4 ja molemmat munuaiset ynnä niiden päällä lantiolihaksissa oleva rasva ja maksanlisäke, joka on irroitettava munuaisten luota.
\par 5 Ja pappi polttakoon ne alttarilla uhrina Herralle; se on vikauhri.
\par 6 Jokainen miehenpuoli papeista saakoon sitä syödä. Se syötäköön pyhässä paikassa; se on korkeasti-pyhää.
\par 7 Mitä on säädetty syntiuhrista, koskee vikauhriakin; laki on sama molemmista. Se olkoon sen papin oma, joka sillä sovituksen toimittaa.
\par 8 Ja pappi, joka toimittaa jonkun puolesta polttouhrin, saakoon uhraamansa polttouhriteuraan nahan.
\par 9 Ja jokainen ruokauhri, joka paistetaan uunissa, ja jokainen pannussa tai leivinlevyllä valmistettu, olkoon sen papin oma, joka uhrin toimittaa.
\par 10 Mutta jokainen muu ruokauhri, olipa siihen sekoitettu öljyä tai olipa se kuiva, olkoon kaikkien Aaronin poikien oma, yhden niinkuin toisenkin.
\par 11 Ja tämä on laki yhteysuhrista, joka Herralle tuodaan:
\par 12 Jos joku tuo sen kiitokseksi, niin tuokoon kiitosuhriteuraan lisäksi öljyyn leivottuja happamattomia kakkuja, öljyllä voideltuja happamattomia ohukaisia ja sekoitettuja lestyjä jauhoja öljyyn leivottuina kakkuina.
\par 13 Hapatetusta taikinasta leivottujen kakkujen ohella hän tuokoon tämän uhrilahjansa kiitosuhrina uhratun yhteysuhriteuraan lisäksi.
\par 14 Ja hän tuokoon siitä yhden kutakin uhrilahja-lajia anniksi Herralle; se olkoon sen papin oma, joka vihmoo yhteysuhrin veren.
\par 15 Ja kiitosuhrina uhratun yhteysuhriteuraan liha syötäköön sinä päivänä, jona se on uhrattu, älköönkä mitään siitä jätettäkö seuraavaan aamuun.
\par 16 Mutta jos uhrilahjana tuotu teuras on lupausuhri tai vapaaehtoinen uhri, syötäköön se sinä päivänä, jona se on tuotu; jos kuitenkin jotakin siitä jää tähteeksi, saatakoon se syödä seuraavana päivänä.
\par 17 Mutta mitä uhrilihasta on tähteenä kolmantena päivänä, se poltettakoon tulessa.
\par 18 Jos yhteysuhriteuraan lihaa syödään kolmantena päivänä, ei se ole otollinen eikä sitä lueta tuojan hyväksi, vaan se on saastaista; jokainen, joka sitä syö, joutuu syynalaiseksi.
\par 19 Sitä lihaa, joka on sattunut saastaiseen, mihin tahansa, älköön syötäkö, vaan se poltettakoon tulessa; muuten saakoon jokainen, joka on puhdas, syödä yhteysuhriteuraan lihaa.
\par 20 Mutta jokainen, joka saastaisena ollessaan syö yhteysuhriteuraan lihaa, joka on Herran oma, hävitettäköön kansastansa.
\par 21 Jos joku koskee johonkin saastaiseen, mihin tahansa, joko ihmisen saastaan tahi saastaiseen karjaeläimeen tai mihin inhottavaan saastaan tahansa, ja syö yhteysuhriteuraan lihaa, joka on Herran oma, hänet hävitettäköön kansastansa."
\par 22 Ja Herra puhui Moosekselle sanoen:
\par 23 "Puhu israelilaisille ja sano: Älkää syökö mitään härän, lampaan tai vuohen rasvaa.
\par 24 Itsestään kuolleen tai kuoliaaksi raadellun eläimen rasva käytettäköön kaikkinaisiin tarpeisiin; mutta älkää sitä syökö.
\par 25 Sillä jokainen, joka syö sen eläimen rasvaa, josta tuodaan Herralle uhri, hävitettäköön kansastansa.
\par 26 Älkää myöskään, missä asuttekin, syökö mitään verta, ei lintujen eikä karjaeläinten.
\par 27 Jokainen, joka syö verta, minkälaista hyvänsä, hävitettäköön kansastansa."
\par 28 Ja Herra puhui Moosekselle sanoen:
\par 29 "Puhu israelilaisille ja sano: Joka tuo Herralle yhteysuhrinsa, tuokoon Herralle uhrilahjan tästä yhteysuhristaan.
\par 30 Omin käsin hän tuokoon Herran uhrit; tuokoon rasvan ynnä rintalihan, ja toimitettakoon niiden heilutus Herran edessä.
\par 31 Ja pappi polttakoon rasvan alttarilla, mutta rintaliha olkoon Aaronin ja hänen poikiensa oma.
\par 32 Ja oikea reisi antakaa papille anniksi yhteysuhriteuraistanne.
\par 33 Se Aaronin pojista, joka uhraa yhteysuhrin veren ja rasvan, saakoon oikean reiden osaksensa.
\par 34 Sillä minä olen ottanut heilutus-rintalihan ja anniksi annetun reiden israelilaisilta heidän yhteysuhriteuraistansa ja antanut ne pappi Aaronille ja hänen pojillensa ikuiseksi osuudeksi israelilaisilta."
\par 35 Tämä on Aaronin ja hänen poikiensa osa Herran uhreista, joka heille määrättiin sinä päivänä, jona hän toi heidät pappeina palvelemaan Herraa;
\par 36 sen osan Herra käski sinä päivänä, jona hän heidät voiteli, israelilaisten antaa heille ikuiseksi osuudeksi sukupolvesta sukupolveen.
\par 37 Tämä on se laki polttouhrista, ruokauhrista, syntiuhrista, vikauhrista, vihkiäisuhrista ja yhteysuhrista,
\par 38 jonka lain Herra antoi Moosekselle Siinain vuorella sinä päivänä, jona hän antoi israelilaisille käskyn tuoda uhrilahjansa Herralle Siinain erämaassa.

\chapter{8}

\par 1 Ja Herra puhui Moosekselle sanoen:
\par 2 "Ota Aaron ja hänen poikansa ynnä heidän vaatteensa ja voiteluöljy sekä syntiuhrimullikka ja ne kaksi oinasta ja kori, jossa happamattomat leivät ovat,
\par 3 ja kokoa kaikki seurakunta ilmestysmajan oven eteen".
\par 4 Ja Mooses teki, niinkuin Herra oli häntä käskenyt, ja seurakunta kokoontui ilmestysmajan oven eteen.
\par 5 Ja Mooses sanoi seurakunnalle: "Näin on Herra käskenyt tehdä".
\par 6 Ja Mooses toi Aaronin ja hänen poikansa esille ja pesi heidät vedellä.
\par 7 Sitten hän pani hänen yllensä ihokkaan ja vyötti hänet vyöllä ja puki hänet viittaan ja pani kasukan hänen päälleen ja vyötti hänet kasukan vyöllä ja sitoi sen hänen ympärilleen
\par 8 ja asetti hänen rintaansa rintakilven ja pani kilpeen uurimin ja tummimin
\par 9 ja pani käärelakin hänen päähänsä ja pani käärelakkiin, sen etupuolelle, kultaisen otsakoristeen, pyhän otsalehden, niinkuin Herra oli Moosesta käskenyt.
\par 10 Ja Mooses otti voiteluöljyn ja voiteli asumuksen ja kaikki, mitä siinä oli, ja pyhitti ne;
\par 11 ja hän pirskoitti sitä seitsemän kertaa alttarille ja voiteli alttarin kaikkine kaluineen ynnä altaan jalustoineen pyhittääkseen ne.
\par 12 Ja hän vuodatti voiteluöljyä Aaronin päähän ja voiteli hänet pyhittääkseen hänet.
\par 13 Ja Mooses toi esille Aaronin pojat ja puki heidän yllensä ihokkaat ja vyötti heidät vyöllä ja sitoi päähineet heidän päähänsä, niinkuin Herra oli Moosesta käskenyt.
\par 14 Ja hän toi esille syntiuhrimullikan, ja Aaron ja hänen poikansa laskivat kätensä syntiuhrimullikan pään päälle.
\par 15 Sitten Mooses teurasti sen ja otti sen verta ja siveli sitä sormellansa alttarin sarviin yltympäri ja puhdisti alttarin; ja muun veren hän vuodatti alttarin juurelle ja pyhitti sen toimittamalla sen sovituksen.
\par 16 Ja Mooses otti kaiken sisälmysten päällä olevan rasvan ja maksanlisäkkeen ja molemmat munuaiset rasvoineen ja poltti ne alttarilla.
\par 17 Mutta mullikan nahkoinensa, lihoinensa ja rapoinensa hän poltti tulessa leirin ulkopuolella, niinkuin Herra oli Moosesta käskenyt.
\par 18 Ja hän toi esille polttouhrioinaan, ja Aaron ja hänen poikansa laskivat kätensä oinaan pään päälle.
\par 19 Ja Mooses teurasti sen ja vihmoi veren alttarille ympärinsä
\par 20 ja leikkeli oinaan määräkappaleiksi; sitten Mooses poltti pään ynnä kappaleet ja rasvan.
\par 21 Mutta sisälmykset ja jalat hän pesi vedessä; ja niin Mooses poltti koko oinaan alttarilla. Se oli polttouhri suloiseksi tuoksuksi, uhri Herralle, niinkuin Herra oli Moosekselle käskyn antanut.
\par 22 Ja hän toi esille toisen oinaan, vihkiäisoinaan, ja Aaron ja hänen poikansa laskivat kätensä oinaan pään päälle.
\par 23 Ja Mooses teurasti sen ja otti sen verta ja siveli sitä Aaronin oikean korvan lehteen ja oikean käden peukaloon ja oikean jalan isoonvarpaaseen.
\par 24 Ja Mooses toi Aaronin pojat esille ja siveli verta heidän oikean korvansa lehteen ja oikean kätensä peukaloon ja heidän oikean jalkansa isoonvarpaaseen, mutta vihmoi muun veren alttarille ympärinsä.
\par 25 Ja hän otti rasvan ja rasvahännän ja kaiken sisälmysten päällä olevan rasvan ja maksanlisäkkeen ja molemmat munuaiset rasvoineen ja oikean reiden.
\par 26 Sitten hän otti happamattomien leipien korista, joka oli Herran edessä, yhden happamattoman kakun ja yhden öljyyn leivotun kakun ja yhden ohukaisen ja pani ne rasvojen ja oikean reiden päälle
\par 27 ja pani kaikki nämä Aaronin käsiin ja hänen poikiensa käsiin, ja toimitutti niiden heilutuksen Herran edessä.
\par 28 Sitten Mooses otti ne heidän käsistänsä ja poltti ne alttarilla polttouhrin päällä; tämä oli vihkiäisuhri suloiseksi tuoksuksi, uhri Herralle.
\par 29 Ja Mooses otti rintalihan ja toimitti sen heilutuksen Herran edessä; vihkiäisoinaasta tuli tämä Mooseksen osaksi, niinkuin Herra oli Moosekselle käskyn antanut.
\par 30 Ja Mooses otti voiteluöljyä ja verta, jota oli alttarilla, ja pirskoitti sitä Aaronin ja hänen vaatteidensa päälle ja samoin hänen poikiensa ja heidän vaatteidensa päälle ja pyhitti niin Aaronin ja hänen vaatteensa ynnä hänen poikansa ja heidän vaatteensa.
\par 31 Ja Mooses sanoi Aaronille ja hänen pojillensa: "Keittäkää liha ilmestysmajan ovella ja syökää se siinä ynnä leipä, joka on vihkiäisuhriin kuuluvassa korissa, niinkuin minä olen käskenyt ja sanonut: Aaron poikinensa syököön sen.
\par 32 Mutta mitä jää tähteeksi lihasta ja leivästä, se polttakaa tulessa.
\par 33 Seitsemään päivään älkää lähtekö ilmestysmajan ovelta, älkää ennenkuin teidän vihkimispäivänne ovat kuluneet umpeen, sillä seitsemän päivää kestää teidän vihkimisenne.
\par 34 Niinkuin on tehty tänä päivänä, niin on Herra käskenyt vastakin tehdä toimittaakseen teille sovituksen.
\par 35 Ja olkaa seitsemän päivää ilmestysmajan ovella, päivät ja yöt, ja toimittakaa Herran palvelustehtävät, ettette kuolisi; sillä niin on minulle käsky annettu."
\par 36 Ja Aaron ja hänen poikansa tekivät kaiken, mitä Herra oli Mooseksen kautta käskenyt.

\chapter{9}

\par 1 Ja kahdeksantena päivänä Mooses kutsui Aaronin ja hänen poikansa ja Israelin vanhimmat
\par 2 ja sanoi Aaronille: "Ota itsellesi härkävasikka syntiuhriksi ja oinas polttouhriksi, molemmat virheettömiä, ja tuo ne Herran eteen.
\par 3 Ja puhu israelilaisille ja sano: 'Ottakaa kauris syntiuhriksi ja vasikka ja karitsa, molemmat vuoden vanhoja ja virheettömiä, polttouhriksi
\par 4 ja härkä ja oinas yhteysuhriksi, uhrattaviksi Herran edessä, ynnä ruokauhri, johon on öljyä sekoitettu, sillä tänä päivänä ilmestyy teille Herra'."
\par 5 Ja he toivat, mitä Mooses oli käskenyt, ilmestysmajan edustalle, ja koko seurakunta astui esiin ja asettui Herran eteen.
\par 6 Ja Mooses sanoi: "Näin Herra on käskenyt teidän tehdä, että Herran kirkkaus ilmestyisi teille".
\par 7 Ja Mooses sanoi Aaronille: "Astu alttarin ääreen ja uhraa syntiuhrisi ja polttouhrisi ja toimita itsellesi ja kansalle sovitus ja uhraa sitten kansan uhrilahja ja toimita heille sovitus, niinkuin Herra on käskenyt".
\par 8 Niin Aaron astui alttarin ääreen ja teurasti oman syntiuhrivasikkansa.
\par 9 Ja Aaronin pojat toivat hänelle veren, ja hän kastoi sormensa vereen ja siveli sitä alttarin sarviin, mutta muun veren hän vuodatti alttarin juurelle.
\par 10 Mutta syntiuhriteuraan rasvan ja munuaiset ja maksanlisäkkeen hän poltti alttarilla, niinkuin Herra oli Moosekselle käskyn antanut.
\par 11 Ja lihan ja nahan hän poltti tulessa leirin ulkopuolella.
\par 12 Sitten hän teurasti polttouhrin, ja Aaronin pojat ojensivat hänelle veren, ja hän vihmoi sen alttarille ympärinsä.
\par 13 Ja he ojensivat hänelle polttouhrin kappaleittain ynnä pään, ja hän poltti ne alttarilla.
\par 14 Ja hän pesi sisälmykset ja jalat ja poltti ne polttouhrin päällä alttarilla.
\par 15 Sitten hän toi kansan uhrilahjan ja otti kansan syntiuhrikauriin, teurasti sen ja uhrasi sen syntiuhrina samalla tavalla kuin edellisen.
\par 16 Ja hän toi myös polttouhrin ja uhrasi sen säädetyllä tavalla.
\par 17 Ja hän toi ruokauhrin ja otti siitä kouransa täyden ja poltti sen alttarilla aamu-polttouhrin lisäksi.
\par 18 Sitten hän teurasti härän ja oinaan kansan yhteysuhriksi, ja Aaronin pojat ojensivat hänelle veren, ja hän vihmoi sen alttarille ympärinsä.
\par 19 Mutta härän ja oinaan rasvat, rasvahännän, rasvakalvon, munuaiset ja maksanlisäkkeen,
\par 20 nämä rasvat he panivat rintalihojen päälle, ja hän poltti rasvat alttarilla.
\par 21 Mutta rintalihain ja oikean reiden heilutuksen Aaron toimitti Herran edessä, niinkuin Mooses oli käskenyt.
\par 22 Ja Aaron kohotti kätensä kansaa kohti ja siunasi heidät, ja kun hän oli toimittanut syntiuhrin, polttouhrin ja yhteysuhrin, astui hän alas.
\par 23 Ja Mooses ja Aaron menivät ilmestysmajaan, ja kun he tulivat sieltä ulos, siunasivat he kansan. Silloin Herran kirkkaus ilmestyi kaikelle kansalle.
\par 24 Ja tuli lähti Herran tyköä ja kulutti polttouhrin ja rasvat alttarilta. Ja kaikki kansa näki sen, ja he riemuitsivat ja lankesivat kasvoillensa.

\chapter{10}

\par 1 Ja Aaronin pojat Naadab ja Abihu ottivat kumpikin hiilipannunsa ja virittivät niihin tulen ja panivat suitsuketta sen päälle ja toivat vierasta tulta Herran eteen, vastoin hänen käskyänsä.
\par 2 Silloin lähti tuli Herran tyköä ja kulutti heidät, niin että he kuolivat Herran edessä.
\par 3 Niin Mooses sanoi Aaronille: "Tämä tapahtuu Herran sanan mukaan: Niissä, jotka ovat minua lähellä, minä osoitan pyhyyteni ja kaiken kansan edessä kirkkauteni". - Mutta Aaron oli ääneti.
\par 4 Ja Mooses kutsui Miisaelin ja Elsafanin, Aaronin sedän Ussielin pojat, ja sanoi heille: "Astukaa esiin ja kantakaa sukulaisenne pyhäkön läheisyydestä leirin ulkopuolelle".
\par 5 Ja he astuivat esiin ja kantoivat heidät ihokkaineen leirin ulkopuolelle, niinkuin Mooses oli sanonut.
\par 6 Sitten Mooses sanoi Aaronille ja hänen pojillensa Eleasarille ja Iitamarille: "Älkää päästäkö tukkaanne hajalle älkääkä repikö vaatteitanne, ettette kuolisi ja ettei hänen vihansa kohtaisi koko seurakuntaa; mutta teidän veljenne, koko Israelin heimo, itkekööt tätä paloa, jonka Herra on sytyttänyt.
\par 7 Älkääkä poistuko ilmestysmajan ovelta, ettette kuolisi, sillä Herran voiteluöljy on teissä." Ja he tekivät, niinkuin Mooses oli sanonut.
\par 8 Ja Herra puhui Aaronille sanoen:
\par 9 "Viiniä ja väkijuomaa älkää juoko, älä sinä älköötkä sinun poikasi sinun kanssasi, kun menette ilmestysmajaan, ettette kuolisi. Tämä olkoon teille ikuinen säädös sukupolvesta sukupolveen,
\par 10 tehdäksenne erotuksen pyhän ja epäpyhän, saastaisen ja puhtaan välillä,
\par 11 ja opettaaksenne israelilaisille kaikki ne käskyt, jotka Herra on heille Mooseksen kautta puhunut."
\par 12 Ja Mooses puhui Aaronille ja hänen eloon jääneille pojillensa Eleasarille ja Iitamarille: "Ottakaa se ruokauhri, joka on jäänyt tähteeksi Herran uhreista, ja syökää se happamattomana alttarin ääressä; sillä se on korkeasti-pyhä.
\par 13 Ja syökää se pyhässä paikassa, sillä se on sinun osuutesi ja sinun poikiesi osuus Herran uhreista; niin on minulle käsky annettu.
\par 14 Mutta heilutus-rintaliha ja anniksi annettu reisi syökää puhtaassa paikassa, sinä ja sinun poikasi ja tyttäresi sinun kanssasi; sillä ne ovat sinun osuudeksesi ja sinun poikiesi osuudeksi annetut israelilaisten yhteysuhriteuraista.
\par 15 Anniksi annettu reisi ja heilutus-rintaliha tuotakoon uhrirasvojen kanssa ja toimitettakoon niiden heilutus Herran edessä; ne olkoot sinun ja sinun poikiesi ikuinen osuus, niinkuin Herra on käskenyt."
\par 16 Ja Mooses tiedusteli syntiuhrikaurista, ja katso, se oli poltettu. Silloin hän vihastui Eleasariin ja Iitamariin, Aaronin eloon jääneihin poikiin, ja sanoi:
\par 17 "Miksi ette ole syöneet syntiuhria pyhässä paikassa? Sehän on korkeasti-pyhä, ja hän on antanut sen teille, että poistaisitte kansan syntivelan ja toimittaisitte heille sovituksen Herran edessä.
\par 18 Katso, ei ole sen verta tuotu pyhäkköön sisälle; teidän olisi tullut syödä se pyhäkössä, niinkuin minä olen käskenyt."
\par 19 Mutta Aaron sanoi Moosekselle: "Katso, he ovat tänä päivänä tuoneet syntiuhrinsa ja polttouhrinsa Herran eteen, ja kuitenkin on tämä minua kohdannut; jos minä tänä päivänä söisin syntiuhria, olisikohan se Herralle otollista?"
\par 20 Kun Mooses sen kuuli, tyytyi hän siihen.

\chapter{11}

\par 1 Ja Herra puhui Moosekselle ja Aaronille sanoen heille:
\par 2 "Puhukaa israelilaisille ja sanokaa: Nämä ovat ne eläimet, joita te saatte syödä kaikista nelijalkaisista eläimistä maan päällä:
\par 3 Kaikkia nelijalkaisia eläimiä, joilla on kokonansa halkinaiset sorkat ja jotka märehtivät, te saatte syödä.
\par 4 Näitä älkää kuitenkaan syökö niistä, jotka märehtivät, ja niistä, joilla on sorkat: kamelia, joka kyllä märehtii, mutta jolla ei ole sorkkia: se olkoon teille saastainen;
\par 5 tamaania, joka kyllä märehtii, mutta jolla ei ole sorkkia: se olkoon teille saastainen;
\par 6 jänistä, joka kyllä märehtii, mutta jolla ei ole sorkkia: se olkoon teille saastainen;
\par 7 sikaa, jolla tosin on kokonansa halkinaiset sorkat, mutta joka ei märehdi: se olkoon teille saastainen.
\par 8 Näiden lihaa älkää syökö ja näiden raatoihin älkää koskeko, ne olkoot teille saastaiset.
\par 9 Näitä te saatte syödä kaikista vesieläimistä: kaikkia, joilla on evät ja suomukset ja jotka elävät vedessä, niin hyvin merissä kuin joissa, te saatte syödä.
\par 10 Mutta kaikista niistä, joita vedet vilisevät, kaikista elollisista, jotka elävät vedessä, inhotkaa kaikkia niitä, joilla ei ole eviä eikä suomuksia, elivätpä merissä tai joissa.
\par 11 Niitä inhotkaa; niiden lihaa älkää syökö, ja niiden raadot inhottakoot teitä.
\par 12 Kaikkia niitä vesieläimiä, joilla ei ole eviä eikä suomuksia, inhotkaa.
\par 13 Linnuista taas inhotkaa näitä; älköön niitä syötäkö, vaan olkoot inhottavia: kotka, partakorppikotka ja harmaa korppikotka,
\par 14 haarahaukka ja suohaukkalajit,
\par 15 kaikki kaarnelajit,
\par 16 kamelikurki, pääskynen, kalalokki ja jalohaukkalajit,
\par 17 huuhkaja, kalasääksi ja kissapöllö,
\par 18 sarvipöllö, pelikaani ja likakorppikotka,
\par 19 haikara ja sirriäislajit, harjalintu ja yölepakko.
\par 20 Ja kaikkia siivellisiä pikkueläimiä, jotka liikkuvat neljällä jalalla, inhotkaa.
\par 21 Kuitenkin saatte siivellisistä pikkueläimistä, jotka liikkuvat neljällä jalalla, syödä niitä, joilla jalkojen yläpuolella on kaksi säärtä hypelläksensä niillä maassa.
\par 22 Niistä te saatte syödä seuraavia: heinäsirkkalajeja, solam-sirkkalajeja, hargol-sirkkalajeja ja haagab-sirkkalajeja.
\par 23 Mutta kaikkia muita siivellisiä pikkueläimiä, joilla on neljä jalkaa, inhotkaa.
\par 24 Seuraavista eläimistä te tulette saastaisiksi; jokainen, joka niiden raatoihin koskee, olkoon saastainen iltaan asti.
\par 25 Ja jokainen, joka niiden raatoja kantaa, pesköön vaatteensa ja olkoon saastainen iltaan asti.
\par 26 Kaikki nelijalkaiset eläimet, joilla on sorkat, mutta ei kokonaan halkinaiset, ja jotka eivät märehdi, olkoot teille saastaiset; jokainen, joka niihin koskee, olkoon saastainen.
\par 27 Ja kaikki nelijalkaiset eläimet, jotka käyvät käpälillä, olkoot teille saastaiset; jokainen, joka niiden raatoihin koskee, olkoon saastainen iltaan asti.
\par 28 Ja joka kantaa niiden raatoja, pesköön vaatteensa ja olkoon saastainen iltaan asti; ne olkoot teille saastaiset.
\par 29 Ja nämä olkoot teille saastaiset niistä pikkueläimistä, jotka liikkuvat maassa: myyrä, hiiri ja sisiliskolajit,
\par 30 anaka-eläin, kooah-eläin, letaa-eläin, hoomet-eläin ja kameleontti.
\par 31 Ne olkoot teille saastaiset kaikista pikkueläimistä; jokainen, joka niihin koskee, sittenkuin ne ovat kuolleet, olkoon saastainen iltaan asti.
\par 32 Ja kaikki, minkä päälle joku niistä kuolleena putoaa, on saastaista, olipa se sitten puuastia tai vaatekappale tai nahka tai säkki tai mikä tahansa kalu, jota johonkin tarpeeseen käytetään. Se pantakoon veteen ja olkoon saastainen iltaan asti; sitten se on puhdas.
\par 33 Ja jos joku niistä putoaa saviastiaan, on kaikki, mitä siinä on, saastaista, ja astia rikottakoon.
\par 34 Kaikki ruoka, jota syödään, on saastaista, jos sen astian vettä tulee siihen; ja kaikki juoma, jota juodaan, on jokaisessa sellaisessa astiassa saastaista.
\par 35 Ja kaikki, minkä päälle niiden raato putoaa, tulee saastaiseksi; olkoon se leivinuuni tai liesi, niin se revittäköön maahan; sillä saastaisia ne ovat ja saastaisia ne teille olkoot.
\par 36 Kuitenkin lähde tai kaivo, paikka, johon vettä on kokoontunut, pysyy puhtaana. Mutta joka niiden raatoon koskee, olkoon saastainen.
\par 37 Ja jos niiden raato putoaa kylvösiemenen päälle, minkä hyvänsä, joka kylvetään, pysyy tämä puhtaana.
\par 38 Mutta jos siemen on vedellä kasteltu ja niiden raato putoaa sen päälle, tulee siemen teille saastaiseksi.
\par 39 Ja jos joku eläin, joka on teille ravinnoksi, kuolee, olkoon se, joka sen raatoon koskee, saastainen iltaan asti.
\par 40 Ja joka syö sen raatoa, pesköön vaatteensa ja olkoon saastainen iltaan asti; ja joka kantaa sen raatoa, pesköön vaatteensa ja olkoon saastainen iltaan asti.
\par 41 Kaikki pikkueläimet, jotka liikkuvat maassa, olkoot inhottavia; älköön niitä syötäkö.
\par 42 Kaikista niistä, jotka käyvät vatsallansa, ja kaikista pikkueläimistä, jotka liikkuvat maassa, käyden neljällä tai useammalla jalalla, niistä älkää mitään syökö, sillä ne ovat inhottavia.
\par 43 Älkää saattako itseänne inhottaviksi koskemalla matelevaan pikkueläimeen, mihin tahansa, älkääkä saastuttako itseänne niillä, niin että tulette niistä saastaisiksi.
\par 44 Sillä minä olen Herra, teidän Jumalanne; pyhittäkää siis itsenne ja olkaa pyhät, sillä minä olen pyhä. Niin älkää saastuttako itseänne koskemalla mihinkään pikkueläimeen, joka liikkuu maassa.
\par 45 Sillä minä olen Herra, joka olen johdattanut teidät Egyptin maasta, ja joka olen teidän Jumalanne; olkaa siis pyhät, sillä minä olen pyhä.
\par 46 Tämä on laki nelijalkaisista eläimistä ja linnuista ja kaikista elollisista, joita vedessä vilisee, ja kaikista olennoista, joita maassa liikkuu,
\par 47 että tietäisitte erottaa puhtaat saastaisista ja syötävät eläimet niistä eläimistä, joita ei saa syödä."

\chapter{12}

\par 1 Ja Herra puhui Moosekselle sanoen:
\par 2 "Puhu israelilaisille ja sano: Kun vaimo tulee hedelmälliseksi ja synnyttää poikalapsen, olkoon hän saastainen seitsemän päivää; yhtä monta päivää kuin hänen kuukautisensa kestävät, hän olkoon saastainen.
\par 3 Ja kahdeksantena päivänä ympärileikattakoon pojan esinahan liha.
\par 4 Mutta vaimo pysyköön kotona kolmekymmentä kolme päivää puhdistavan verenvuotonsa aikana; älköön hän koskeko mihinkään, mikä on pyhää, älköönkä tulko pyhäkköön, ennenkuin hänen puhdistuspäivänsä ovat kuluneet umpeen.
\par 5 Mutta jos hän synnyttää tyttölapsen, olkoon hän saastainen kaksi viikkoa, samoinkuin kuukautisissaan, ja pysyköön kotona kuusikymmentä kuusi päivää puhdistavan verenvuotonsa aikana.
\par 6 Mutta kun hänen puhdistuspäivänsä pojan tai tyttären jälkeen ovat kuluneet umpeen, vieköön hän vuoden vanhan karitsan polttouhriksi ja kyyhkysenpojan tai metsäkyyhkysen syntiuhriksi ilmestysmajan oven eteen papille.
\par 7 Ja tämä tuokoon ne Herran eteen ja toimittakoon hänelle sovituksen, niin hän on puhdas verenvuodostansa. Tämä on laki lapsensynnyttäjästä, olipa hän synnyttänyt poika- tai tyttölapsen.
\par 8 Mutta jos hän ei saa hankituksi lammasta, ottakoon kaksi metsäkyyhkystä tai kaksi kyyhkysenpoikaa, toisen polttouhriksi ja toisen syntiuhriksi; ja pappi toimittakoon hänelle sovituksen, niin hän on puhdas."

\chapter{13}

\par 1 Ja Herra puhui Moosekselle ja Aaronille sanoen:
\par 2 "Jos jonkun ihoon tulee nystyrä tai ihottuma tai vaalea pilkku, niinkuin pitalitauti olisi tulemassa hänen ihoonsa, vietäköön hänet pappi Aaronin tai jonkun hänen poikansa, pappien, eteen.
\par 3 Ja jos pappi tarkastaessaan sairasta paikkaa ihossa huomaa, että karvat sairaassa paikassa ovat muuttuneet valkoisiksi ja että se paikka näyttää muuta ihoa matalammalta, on se pitalitautia; ja niin pian kuin pappi on sen huomannut, julistakoon hän hänet saastaiseksi.
\par 4 Ja jos hänen ihossansa on valkea pilkku, mutta se ei näytä muuta ihoa matalammalta eivätkä karvat ole muuttuneet valkoisiksi, sulkekoon pappi sairaan sisälle seitsemäksi päiväksi.
\par 5 Ja seitsemäntenä päivänä pappi tarkastakoon häntä, ja jos sairas paikka hänestä näyttää pysyneen entisellään sairauden leviämättä ihossa, sulkekoon pappi hänet sisälle vielä seitsemäksi päiväksi.
\par 6 Ja seitsemäntenä päivänä pappi uudestaan tarkastakoon häntä, ja jos hän huomaa että sairas paikka on käynyt vaaleaksi ja että sairaus ei ole ihossa levinnyt, julistakoon pappi hänet puhtaaksi, sillä se on ihottumaa; hän pesköön vaatteensa, ja niin hän on puhdas.
\par 7 Mutta jos ihottuma leviää leviämistään hänen ihossansa, sen jälkeen kuin hän oli näyttäytynyt papille tullakseen puhtaaksi julistetuksi, näyttäytyköön uudestaan papille;
\par 8 ja jos pappi huomaa ihottuman levinneen ihossa, julistakoon pappi hänet saastaiseksi; se on pitalia.
\par 9 Jos pitalitauti ilmestyy ihmiseen, vietäköön hänet papin luo.
\par 10 Ja jos pappi tarkastaessaan huomaa ihossa valkoisen nystyrän, jonka kohdalta karvat ovat muuttuneet valkoisiksi, ja liikaa lihaa kasvavan nystyrään,
\par 11 on hänen ihossaan jo vanha pitali, ja sentähden pappi julistakoon hänet saastaiseksi älköönkä enää sulkeko häntä sisälle, sillä hän on saastainen.
\par 12 Ja jos pitali puhkeaa ihossa, niin että se peittää sairaan koko ihon päästä jalkoihin asti, mihin pappi katsokoonkin,
\par 13 ja jos pappi tarkastaessaan huomaa pitalin peittävän koko ihon, niin julistakoon hän sairaan puhtaaksi; hän on kokonansa muuttunut valkoiseksi ja on puhdas.
\par 14 Mutta niin pian kuin häneen ilmestyy liikaa lihaa, on hän saastainen.
\par 15 Ja kun pappi huomaa liian lihan, julistakoon hän hänet saastaiseksi; liika liha on saastaista, se on pitalia.
\par 16 Mutta jos liika liha häviää ja sairas paikka muuttuu valkoiseksi, menköön hän papin luo.
\par 17 Ja jos pappi huomaa, että sairas paikka on muuttunut valkoiseksi, julistakoon pappi sairaan puhtaaksi; hän on puhdas.
\par 18 Ja jos jonkun ihoon tulee paise, joka paranee jälleen,
\par 19 ja paiseen sijalle tulee valkoinen nystyrä tai vaaleanpunainen pilkku, näyttäytyköön hän papille.
\par 20 Ja jos pappi tarkastaessaan huomaa, että se paikka näyttää muuta ihoa matalammalta ja että karvat siinä ovat muuttuneet valkoisiksi, julistakoon pappi hänet saastaiseksi; paiseeseen on puhjennut pitalitauti.
\par 21 Mutta jos pappi tarkastaessaan huomaa, ettei siinä ole valkoisia karvoja ja ettei se paikka ole muuta ihoa matalampi ja että se on käynyt vaaleaksi, sulkekoon pappi hänet sisälle seitsemäksi päiväksi.
\par 22 Ja jos se leviää leviämistään hänen ihossaan, julistakoon pappi hänet saastaiseksi; se on pitalia.
\par 23 Mutta jos pilkku pysyy alallansa eikä leviä, on se paiseen arpi, ja pappi julistakoon hänet puhtaaksi.
\par 24 Tahi jos joku saa palohaavan ihoonsa ja palohaavaan kasvavaan lihaan ilmestyy vaaleanpunainen tai valkoinen pilkku,
\par 25 ja jos pappi tarkastaessaan huomaa karvojen muuttuneen valkoisiksi siinä pilkussa ja sen paikan näyttävän muuta ihoa matalammalta, niin on palohaavaan puhjennut pitali, ja pappi julistakoon hänet saastaiseksi; se on pitalitautia.
\par 26 Mutta jos pappi tarkastaessaan huomaa, ettei pilkussa ole valkoisia karvoja ja ettei se ole muuta ihoa matalampi ja että se on käynyt vaaleaksi, niin sulkekoon pappi hänet sisälle seitsemäksi päiväksi.
\par 27 Ja seitsemäntenä päivänä pappi tarkastakoon häntä; jos se on yhä levinnyt ihossa, julistakoon pappi hänet saastaiseksi; se on pitalitautia.
\par 28 Mutta jos pilkku on pysynyt alallansa eikä ole levinnyt ihossa ja on edelleen vaalennut, on se palohaavasta syntynyt nystyrä, ja pappi julistakoon hänet puhtaaksi; se on palohaavan arpi.
\par 29 Jos mieheen tai naiseen tulee sairas paikka päähän tai partaan,
\par 30 ja jos pappi tarkastaessaan sairasta paikkaa huomaa, että se näyttää muuta ihoa matalammalta ja että siinä on ohuita, kellertäviä karvoja, niin julistakoon pappi hänet saastaiseksi; se on pitalisyyhelmää, se on pää- eli partapitalia.
\par 31 Ja jos pappi tarkastaessaan sairasta paikkaa huomaa, ettei se näytä muuta ihoa matalammalta, mutta ettei siinä ole mustia karvoja, sulkekoon pappi syyhelmää sairastavan sisälle seitsemäksi päiväksi.
\par 32 Ja jos pappi seitsemäntenä päivänä tarkastaessaan sairasta paikkaa huomaa, ettei syyhelmä ole levinnyt ja ettei siinä ole kellertäviä karvoja ja ettei syyhelmä näytä muuta ihoa matalammalta,
\par 33 niin ajakoon sairas hiuksensa tai partansa, kuitenkaan ajamatta sairasta paikkaa, ja pappi sulkekoon syyhelmää sairastavan sisälle vielä seitsemäksi päiväksi.
\par 34 Ja jos pappi seitsemäntenä päivänä tarkastaessaan sairasta paikkaa huomaa, ettei syyhelmä ole levinnyt ihossa ja ettei se näytä muuta ihoa matalammalta, julistakoon pappi hänet puhtaaksi; hän pesköön vaatteensa, ja niin hän on puhdas.
\par 35 Mutta jos syyhelmä leviää leviämistään ihossa, sen jälkeen kuin hänet julistettiin puhtaaksi,
\par 36 ja jos pappi tarkastaessaan häntä huomaa, että syyhelmä on levinnyt ihossa, älköön pappi enää etsikö kellertäviä karvoja; hän on saastainen.
\par 37 Mutta jos syyhelmä hänestä näyttää pysyneen entisellään ja jos siihen on kasvanut mustia karvoja, on syyhelmä parantunut; hän on puhdas, ja pappi julistakoon hänet puhtaaksi.
\par 38 Jos miehen tai naisen ihoon tulee pilkkuja, valkoisia pilkkuja,
\par 39 ja jos pappi tarkastaessaan huomaa vaalenevia, valkoisia pilkkuja heidän ihossaan, on se vaaratonta valkoista ihottumaa, joka on puhjennut ihoon; hän on puhdas.
\par 40 Jos hiukset lähtevät miehen päästä, takaraivosta, on hän takakalju; hän on puhdas.
\par 41 Ja jos hiukset lähtevät päästä, ohimoista, on hän otsakalju; hän on puhdas.
\par 42 Mutta jos vaaleanpunainen pilkku ilmestyy takaraivon tai otsapuolen paljaaseen paikkaan, on se pitalia, joka on puhkeamassa takaraivon tai otsapuolen paljaaseen paikkaan.
\par 43 Ja jos pappi tarkastaessaan häntä huomaa, että nystyrä takaraivon tai otsapuolen paljaassa paikassa on vaaleanpunainen näyttäen samanlaiselta kuin pitali muussa ihossa,
\par 44 niin mies on pitalinen ja saastainen; pappi julistakoon hänet heti saastaiseksi hänen päässänsä olevan sairauden tähden.
\par 45 Joka sairastaa pitalia, hän käyköön rikkirevityissä vaatteissa, tukka hajallaan ja parta peitettynä ja huutakoon: 'Saastainen, saastainen!'
\par 46 Niin kauan kuin sairaus hänessä on, olkoon hän saastainen; saastaisena hän asukoon yksinänsä, hänen asuinsijansa olkoon leirin ulkopuolella.
\par 47 Jos pitalitartuntaa tulee vaatekappaleeseen, olipa vaatekappale villainen tai pellavainen,
\par 48 tai kudottuun tai solmustettuun kankaaseen, olipa se pellavaista tai villaista, tai nahkaan tai mihin nahasta tehtyyn esineeseen tahansa,
\par 49 ja jos tartunnan paikka näyttää vihertävältä tai punertavalta vaatekappaleessa tai nahassa tai kudotussa tai solmustetussa kankaassa tai missä nahasta tehdyssä esineessä tahansa, niin se on pitalia, ja se on näytettävä papille.
\par 50 Ja kun pappi on tarkastanut tartunnan paikan, sulkekoon hän tartutetun sisälle seitsemäksi päiväksi.
\par 51 Ja jos hän seitsemäntenä päivänä tarkastaessaan tartunnan paikkaa huomaa, että tarttuma on levinnyt vaatekappaleessa tai kudotussa tai solmustetussa kankaassa tai nahassa tai missä nahasta tehdyssä esineessä tahansa, on tarttuma pahanlaatuista pitalia; ne ovat saastaiset.
\par 52 Ja hän polttakoon vaatekappaleen tai kudotun tai solmustetun kankaan, olipa se villainen tai pellavainen, tai nahasta tehdyn esineen, olipa se mikä tahansa, jossa tarttuma on, sillä se on pahanlaatuista pitalia; kaikki sellainen poltettakoon tulessa.
\par 53 Mutta jos pappi tarkastaessaan huomaa, ettei tarttuma ole levinnyt vaatekappaleessa tai kudotussa tai solmustetussa kankaassa tai nahasta tehdyssä esineessä, olipa se mikä tahansa,
\par 54 käskeköön pappi pestä esineen, jossa tarttuma on, ja sulkekoon sen sisälle vielä seitsemäksi päiväksi.
\par 55 Ja jos pappi tarkastaessaan tartunnan saanutta esinettä, sen jälkeen kuin se on pesty, huomaa, ettei tarttuma ole muuttunut näöltään, on esine, vaikka tarttuma ei olekaan levinnyt, saastainen, ja se poltettakoon tulessa; se on pitalin syömä, olkoon sitten nurjalta tai oikealta puolelta.
\par 56 Mutta jos pappi tarkastaessaan huomaa, että tartunnan paikka on pesemisen jälkeen vaalennut, reväisköön hän sen pois vaatekappaleesta tai nahasta tai kudotusta tai solmustetusta kankaasta.
\par 57 Ja jos sitten vielä tarttumaa ilmestyy vaatekappaleeseen tai kudottuun tai solmustettuun kankaaseen tai mihin nahasta tehtyyn esineeseen tahansa, on se puhkeavaa pitalia; se esine, jossa tarttuma on, poltettakoon tulessa.
\par 58 Mutta vaatekappale tai kudottu tai solmustettu kangas tai mikä nahasta tehty esine tahansa, josta tarttuma on pesemällä kadonnut, pestäköön vielä kerran, ja niin se on puhdas."
\par 59 Tämä on laki pitalitarttumasta vaatekappaleessa, villaisessa tai pellavaisessa, tai kudotussa tai solmustetussa kankaassa tai missä nahasta tehdyssä esineessä tahansa, laki niiden puhtaiksi tai saastaisiksi julistamisesta.

\chapter{14}

\par 1 Ja Herra puhui Moosekselle sanoen:
\par 2 "Tämä olkoon laki pitalitautisen puhdistamispäivästä. Hänet tuotakoon papin eteen,
\par 3 ja pappi menköön leirin ulkopuolelle; ja jos pappi tarkastaessaan pitalista huomaa, että hän on parantunut pitalitaudista,
\par 4 käskeköön pappi puhdistettavaa varten ottaa kaksi elävää, puhdasta lintua, setripuuta, helakanpunaista lankaa ja isoppikorren.
\par 5 Ja pappi käskeköön teurastaa toisen linnun saviastian päällä, jossa on raikasta vettä.
\par 6 Sitten hän ottakoon sen elävän linnun ynnä setripuun, helakanpunaisen langan ja isoppikorren ja kastakoon ne ynnä elävän linnun sen linnun vereen, joka teurastettiin raikkaan veden päällä,
\par 7 ja pirskoittakoon sitä seitsemän kertaa siihen, joka on pitalista puhdistettava; ja puhdistettuaan hänet hän päästäköön sen elävän linnun lentämään kedolle.
\par 8 Ja puhdistettava pesköön vaatteensa, ajakoon kaikki hiuksensa ja peseytyköön vedessä, niin hän on puhdas. Sen jälkeen hän saa mennä leiriin; asukoon kuitenkin ulkona teltastansa seitsemän päivää.
\par 9 Seitsemäntenä päivänä hän ajakoon kaikki hiukset päästänsä, partansa ja kulmakarvansa: kaikki karvat hän ajakoon; pesköön sitten vaatteensa ja pesköön ruumiinsa vedessä, niin hän on puhdas.
\par 10 Ja kahdeksantena päivänä hän ottakoon kaksi virheetöntä karitsaa ja virheettömän, vuoden vanhan uuhikaritsan sekä kolme kymmenennestä lestyjä jauhoja, joihin on sekoitettu öljyä, ruokauhriksi, ja yhden loog-mitan öljyä.
\par 11 Ja puhdistusta toimittava pappi asettakoon puhdistettavan henkilön ja nämä esineet Herran eteen ilmestysmajan ovelle.
\par 12 Sitten pappi ottakoon toisen karitsan ja tuokoon sen vikauhriksi ynnä loog-mitan öljyä, ja toimittakoon niiden heilutuksen Herran edessä.
\par 13 Ja niin hän teurastakoon karitsan siinä paikassa, jossa syntiuhri- ja polttouhriteuraat teurastetaan, pyhässä paikassa; sillä vikauhri on, niinkuin syntiuhrikin, papin oma, se on korkeasti-pyhä.
\par 14 Ja pappi ottakoon vikauhrin verta ja sivelköön sitä puhdistettavan oikeaan korvanlehteen sekä oikean käden peukaloon ja oikean jalan isoonvarpaaseen.
\par 15 Senjälkeen pappi ottakoon öljyä loog-mitasta ja kaatakoon sitä vasempaan käteensä,
\par 16 ja pappi kastakoon oikean kätensä etusormen öljyyn, jota on hänen vasemmassa kädessään, ja pirskoittakoon öljyä sormellansa seitsemän kertaa Herran edessä.
\par 17 Ja käteensä jäänyttä öljyä pappi sivelköön puhdistettavan oikeaan korvanlehteen sekä oikean käden peukaloon ja oikean jalan isoonvarpaaseen vikauhrin veren päälle.
\par 18 Ja käteensä vielä jääneen öljyn pappi sivelköön puhdistettavan päähän; näin pappi toimittakoon hänelle sovituksen Herran edessä.
\par 19 Sitten pappi uhratkoon syntiuhrin ja toimittakoon puhdistettavalle sovituksen hänen saastaisuudestansa, ja senjälkeen hän teurastakoon polttouhriteuraan.
\par 20 Ja pappi uhratkoon polttouhrin ja ruokauhrin alttarilla, ja kun pappi näin on toimittanut hänelle sovituksen, on hän puhdas.
\par 21 Mutta jos hän on köyhä eikä saa hankituksi niin paljoa, ottakoon vain yhden karitsan vikauhriksi, heilutettavaksi, toimittaakseen itsellensä sovituksen, sekä vain yhden kymmenenneksen lestyjä jauhoja, joihin on sekoitettu öljyä, ruokauhriksi, ja loog-mitan öljyä,
\par 22 sekä kaksi metsäkyyhkystä tai kaksi kyyhkysenpoikaa, mikäli hän on saanut ne hankituksi; toinen olkoon syntiuhriksi ja toinen polttouhriksi.
\par 23 Ja hän vieköön ne papille puhdistustaan varten kahdeksantena päivänä, ilmestysmajan ovelle Herran eteen.
\par 24 Ja pappi ottakoon vikauhrikaritsan ja loog-mitan öljyä, ja pappi toimittakoon niiden heilutuksen Herran edessä,
\par 25 ja vikauhrikaritsa teurastettakoon, ja pappi ottakoon vikauhrin verta ja sivelköön sitä puhdistettavan oikeaan korvanlehteen sekä oikean käden peukaloon ja oikean jalan isoonvarpaaseen.
\par 26 Sitten pappi kaatakoon öljyä vasempaan käteensä,
\par 27 ja pappi pirskoittakoon oikealla etusormellaan öljyä, jota on hänen vasemmassa kädessään, seitsemän kertaa Herran edessä.
\par 28 Ja käteensä jäänyttä öljyä pappi sivelköön puhdistettavan oikeaan korvanlehteen sekä oikean käden peukaloon ja oikean jalan isoonvarpaaseen, siihen paikkaan, missä on vikauhrin verta.
\par 29 Ja käteensä vielä jääneen öljyn pappi sivelköön puhdistettavan päähän toimittaakseen hänelle sovituksen Herran edessä.
\par 30 Ja hän uhratkoon toisen niistä metsäkyyhkysistä tai kyyhkysenpojista, jotka hän on hankkinut,
\par 31 mikäli on saanut ne hankituiksi, toisen syntiuhriksi ja toisen polttouhriksi ruokauhrin ohella. Näin pappi toimittakoon puhdistettavalle sovituksen Herran edessä."
\par 32 Tämä on laki pitalitautia sairastavasta, joka ei saa hankituksi puhdistukseensa sitä, mikä on säädetty.
\par 33 Ja Herra puhui Moosekselle ja Aaronille sanoen:
\par 34 "Kun te tulette Kanaanin maahan, jonka minä annan teille perintömaaksi, ja minä sallin pitalin tarttua johonkin taloon siinä maassa, jonka te saatte perintömaaksenne,
\par 35 niin talon omistaja menköön ja ilmoittakoon sen papille sanoen: 'Minun talooni näyttää ilmestyneen jotakin pitalitarttuman tapaista'.
\par 36 Ja pappi käskeköön tyhjentää talon, ennenkuin menee tarkastamaan tarttumaa, ettei kaikki, mitä talossa on, tulisi saastaiseksi; sitten pappi menköön tarkastamaan taloa.
\par 37 Ja jos hän tarkastaessaan tarttumaa näkee tartunnan paikassa talon seinissä vihertäviä tai punertavia syvennyksiä, jotka näyttävät muuta seinää matalammilta,
\par 38 menköön pappi talosta ulos talon ovelle ja sulkekoon talon seitsemäksi päiväksi.
\par 39 Ja jos pappi palattuaan seitsemäntenä päivänä tarkastaessaan huomaa tarttuman levinneen talon seinissä,
\par 40 käskeköön hän murtaa pois ne kivet, joissa tarttuma on, ja heittää ne ulos kaupungista saastaiseen paikkaan.
\par 41 Ja talo kaavittakoon sisältä ympärinsä puhtaaksi, ja kaavittu savi heitettäköön ulos kaupungista saastaiseen paikkaan.
\par 42 Ja otettakoon toisia kiviä ja pantakoon entisten kivien sijaan, ja otettakoon toista savea ja savettakoon talo sillä.
\par 43 Jos tarttuma taas ilmestyy taloon, sen jälkeen kuin kivet ovat murretut pois ja talo on puhtaaksi kaavittu ja uudestaan savettu,
\par 44 menköön pappi sisälle, ja jos hän tarkastaessaan huomaa tarttuman levinneen talossa, niin on talossa pahanlaatuinen pitali; se on saastainen.
\par 45 Sentähden se talo revittäköön, sekä sen kivet että puuaineet ja kaikki sen savi, ja vietäköön ne ulos kaupungista saastaiseen paikkaan.
\par 46 Ja se, joka on käynyt talossa sinä aikana, jona sen oli oltava suljettuna, olkoon saastainen iltaan asti.
\par 47 Ja se, joka siinä talossa on maannut, pesköön vaatteensa, ja se, joka siinä talossa on syönyt, pesköön vaatteensa.
\par 48 Mutta jos pappi menee sisälle ja tarkastaessaan huomaa, ettei tarttuma ole levinnyt talossa, sen jälkeen kuin talo on savettu, niin julistakoon pappi talon puhtaaksi, sillä tarttuma on hävinnyt.
\par 49 Ja hän ottakoon talon puhdistamiseksi kaksi lintua, setripuuta, helakanpunaista lankaa ja isoppikorren.
\par 50 Ja teurastakoon toisen linnun saviastian päällä, jossa on raikasta vettä.
\par 51 Ja ottakoon setripuun, isoppikorren ja helakanpunaisen langan ja elävän linnun, kastakoon ne teurastetun linnun vereen ja raikkaaseen veteen ja pirskoittakoon sitä taloon seitsemän kertaa.
\par 52 Näin hän puhdistakoon talon linnun verellä ja raikkaalla vedellä, elävällä linnulla, setripuulla, isoppikorrella ja helakanpunaisella langalla.
\par 53 Ja sitten hän päästäköön elävän linnun lentämään ulkopuolelle kaupunkia kedolle. Kun hän näin on toimittanut talolle sovituksen, on se puhdas."
\par 54 Tämä on laki kaikkinaisesta pitalitaudista ja syyhelmästä,
\par 55 pitalista vaatteissa ja taloissa,
\par 56 nystyröistä, ihottumista ja vaaleista pilkuista,
\par 57 neuvoksi siitä, milloin mikin on saastainen, milloin puhdas. Tämä on laki pitalista.

\chapter{15}

\par 1 Ja Herra puhui Moosekselle ja Aaronille sanoen:
\par 2 "Puhukaa israelilaisille ja sanokaa heille: Jos jollakin, kenellä tahansa, on elimestään liman vuoto, on hänen vuotonsa saastainen.
\par 3 Hänen vuotonsa saastaisuus on sellainen, että hän, niin hyvin silloin, kun hänen elimestään vuotaa, kuin silloin, kun hänen elimensä pidättää vuodon, on saastainen.
\par 4 Jokainen vuode, jossa vuotoa sairastava lepää, tulee saastaiseksi, ja jokainen istuin, jolla hän istuu, tulee saastaiseksi.
\par 5 Ja se, joka koskee hänen vuoteeseensa, pesköön vaatteensa ja peseytyköön vedessä ja olkoon saastainen iltaan asti.
\par 6 Ja se, joka istuu istuimelle, jolla vuotoa sairastava on istunut, pesköön vaatteensa ja peseytyköön vedessä ja olkoon saastainen iltaan asti.
\par 7 Ja se, joka koskee vuotoa sairastavan ruumiiseen, pesköön vaatteensa ja peseytyköön vedessä ja olkoon saastainen iltaan asti.
\par 8 Ja jos vuotoa sairastava sylkee puhtaan ihmisen päälle, niin tämä pesköön vaatteensa ja peseytyköön vedessä ja olkoon saastainen iltaan asti.
\par 9 Ja jokainen satula, jossa vuotoa sairastava ratsastaa, tulee saastaiseksi.
\par 10 Ja jokainen, joka koskee mihin hyvänsä, mikä on ollut hänen allaan, olkoon saastainen iltaan asti; ja joka sellaista kantaa, pesköön vaatteensa ja peseytyköön vedessä ja olkoon saastainen iltaan asti.
\par 11 Ja jokainen, johon vuotoa sairastava koskee, ennenkuin on huuhtonut kätensä vedessä, pesköön vaatteensa ja peseytyköön vedessä ja olkoon saastainen iltaan asti.
\par 12 Ja saviastia, johon vuotoa sairastava koskee, rikottakoon; mutta jokainen puuastia huuhdottakoon vedessä.
\par 13 Ja kun vuotoa sairastava tulee puhtaaksi vuodostansa, niin hän laskekoon puhtaaksi-tulemisestaan seitsemän päivää ja sitten pesköön vaatteensa ja pesköön ruumiinsa raikkaassa vedessä, niin hän on puhdas.
\par 14 Ja kahdeksantena päivänä hän ottakoon kaksi metsäkyyhkystä tai kaksi kyyhkysenpoikaa ja tulkoon Herran eteen ilmestysmajan ovelle ja antakoon ne papille.
\par 15 Ja pappi uhratkoon ne, toisen syntiuhriksi ja toisen polttouhriksi. Näin pappi toimittakoon Herran edessä hänelle sovituksen hänen vuodostansa.
\par 16 Kun miehellä on ollut siemenvuoto, pesköön hän koko ruumiinsa vedessä ja olkoon saastainen iltaan asti.
\par 17 Ja jokainen vaatekappale ja jokainen nahka, johon siemenvuotoa on tullut, pestäköön vedessä ja olkoon saastainen iltaan asti.
\par 18 Ja kun mies on maannut naisen kanssa ja vuodattanut siemenensä, peseytykööt he vedessä ja olkoot saastaiset iltaan asti.
\par 19 Ja kun naisella on vuoto, niin että verta vuotaa hänen ruumiistansa, olkoon hän kuukautistilassaan seitsemän päivää, ja jokainen, joka häneen koskee, olkoon saastainen iltaan asti.
\par 20 Ja kaikki, minkä päällä hän lepää kuukautistilansa aikana, tulee saastaiseksi, ja kaikki, minkä päällä hän istuu, tulee saastaiseksi.
\par 21 Ja jokainen, joka hänen vuoteeseensa koskee, pesköön vaatteensa ja peseytyköön vedessä ja olkoon saastainen iltaan asti.
\par 22 Ja jokainen, joka koskee istuimeen, mihin hyvänsä, jolla hän on istunut, pesköön vaatteensa ja peseytyköön vedessä ja olkoon saastainen iltaan asti.
\par 23 Ja jos joku koskee esineeseen, joka on hänen vuoteellaan tai istuimella, jolla hän on istunut, olkoon saastainen iltaan asti.
\par 24 Ja jos mies makaa hänen kanssaan ja hänen kuukautistansa tulee hänen päällensä, olkoon hän saastainen seitsemän päivää, ja jokainen vuode, jossa hän lepää, tulee saastaiseksi.
\par 25 Jos naisen verenvuoto kestää kauan aikaa, vaikka ei ole hänen kuukautisaikansa, tahi jos se jatkuu hänen kuukautisaikansa ohitse, pidettäköön hänet koko vuotonsa ajan saastaisena, niinkuin hänen kuukautisaikanaankin; hän on saastainen.
\par 26 Jokaisesta vuoteesta, jossa hän lepää vuotonsa aikana, olkoon voimassa, mitä on säädetty hänen kuukautisaikana käyttämästään vuoteesta; ja jokainen istuin, jolla hän istuu, tulee saastaiseksi niinkuin hänen kuukautisaikanaankin.
\par 27 Ja jokainen, joka niihin koskee, tulee saastaiseksi; hän pesköön vaatteensa ja peseytyköön vedessä ja olkoon saastainen iltaan asti.
\par 28 Mutta kun hän tulee puhtaaksi vuodostansa, laskekoon seitsemän päivää, ja sitten hän on puhdas.
\par 29 Ja kahdeksantena päivänä hän ottakoon kaksi metsäkyyhkystä tai kaksi kyyhkysenpoikaa ja tuokoon ne papille, ilmestysmajan ovelle.
\par 30 Ja pappi uhratkoon toisen syntiuhriksi ja toisen polttouhriksi. Näin pappi toimittakoon Herran edessä hänelle sovituksen hänen saastaisesta vuodostaan.
\par 31 Näin teidän on varoitettava israelilaisia saastaisuudesta, etteivät he saastaisuudessansa kuolisi, jos saastuttavat minun asumukseni, joka on heidän keskellänsä."
\par 32 Tämä on laki miehestä, joka sairastaa vuotoa, ja miehestä, jolla on siemenvuoto ja joka tulee siitä saastaiseksi,
\par 33 ja naisesta, jolla on kuukautisensa, ja miehestä ja naisesta, jotka sairastavat jotakin vuotoa, sekä miehestä, joka makaa saastaisen naisen kanssa.

\chapter{16}

\par 1 Ja Herra puhui Moosekselle, sen jälkeen kuin kaksi Aaronin poikaa oli kuollut, joita kuolema kohtasi, kun he astuivat Herran eteen.
\par 2 Ja Herra sanoi Moosekselle: "Sano veljellesi Aaronille: älköön hän joka aika menkö pyhimpään, esiripun sisäpuolelle, armoistuimen eteen, joka on arkin päällä, ettei hän kuolisi; sillä minä ilmestyn pilvessä armoistuimen kohdalla.
\par 3 Mutta näin Aaron menköön pyhimpään: hänellä olkoon mukanaan mullikka syntiuhriksi ja oinas polttouhriksi;
\par 4 ja hän pukekoon pyhän pellava-ihokkaan yllensä, peittäköön häpynsä pellavakaatioilla, vyöttäköön itsensä pellavavyöllä ja pankoon pellavaisen käärelakin päähänsä. Nämä ovat pyhät vaatteet; ja hän pesköön ruumiinsa vedessä ja pukekoon ne yllensä.
\par 5 Ja hän ottakoon israelilaisten seurakunnalta kaksi kaurista syntiuhriksi ja yhden oinaan polttouhriksi.
\par 6 Ja Aaron tuokoon oman syntiuhrimullikkansa ja toimittakoon sovituksen itsellensä ja perheellensä.
\par 7 Sitten hän ottakoon ne kaksi kaurista ja asettakoon ne Herran eteen ilmestysmajan ovelle.
\par 8 Ja Aaron heittäköön arpaa niistä kahdesta kauriista: toisen arvan Herralle ja toisen Asaselille.
\par 9 Ja Aaron tuokoon sen kauriin, jonka arpa määräsi Herralle, ja uhratkoon sen syntiuhriksi.
\par 10 Mutta se kauris, jonka arpa määräsi Asaselille, asetettakoon elävänä Herran eteen, että sille toimitettaisiin sovitus ja se sitten päästettäisiin erämaahan Asaselille.
\par 11 Senjälkeen Aaron tuokoon oman syntiuhrimullikkansa ja toimittakoon sovituksen itsellensä ja perheellensä ja teurastakoon oman syntiuhrimullikkansa.
\par 12 Ja hän ottakoon hiilipannun täyteen tulisia hiiliä alttarilta, joka on Herran edessä, ja kahmalonsa täyteen survottua hyvänhajuista suitsuketta ja vieköön ne esiripun sisäpuolelle.
\par 13 Ja hän pankoon suitsukkeen tulen päälle Herran eteen, niin että suitsutus pilvenä peittää lain arkin päällä olevan armoistuimen, ettei hän kuolisi.
\par 14 Ja hän ottakoon mullikan verta ja pirskoittakoon sitä sormellansa armoistuimen etupuolelle; ja armoistuimen eteen hän pirskoittakoon sormellansa verta seitsemän kertaa.
\par 15 Sitten hän teurastakoon kansan syntiuhrikauriin ja vieköön sen verta esiripun sisäpuolelle ja tehköön sen verellä, niinkuin hän teki mullikan verellä: pirskoittakoon sitä armoistuimelle ja armoistuimen eteen.
\par 16 Ja näin hän toimittakoon pyhäkölle sovituksen israelilaisten saastaisuudesta ja heidän rikoksistaan, olivatpa heidän syntinsä minkälaiset tahansa; näin hän tehköön myös ilmestysmajalle, joka on heidän tykönänsä, keskellä heidän saastaisuuttansa.
\par 17 Älköön yhtään ihmistä olko ilmestysmajassa, kun hän tulee toimittamaan sovitusta pyhimmässä, siihen saakka kunnes hän sieltä lähtee ja on toimittanut itsellensä ja perheellensä ja koko Israelin seurakunnalle sovituksen.
\par 18 Sitten hän lähteköön sieltä alttarin ääreen, joka on Herran edessä, ja toimittakoon sille sovituksen; ja hän ottakoon mullikan verta ja kauriin verta ja sivelköön sitä alttarin sarviin yltympäri.
\par 19 Ja hän pirskoittakoon sormellansa sen päälle verta seitsemän kertaa ja puhdistakoon ja pyhittäköön sen israelilaisten saastaisuudesta.
\par 20 Ja kun hän on loppuun toimittanut pyhäkön ja ilmestysmajan ja alttarin sovittamisen, tuokoon hän sen elävän kauriin.
\par 21 Ja Aaron laskekoon molemmat kätensä elävän kauriin pään päälle ja tunnustakoon siinä kaikki israelilaisten pahat teot ja kaikki heidän rikkomuksensa, olipa heillä mitä syntejä tahansa, ja pankoon ne kauriin pään päälle ja lähettäköön sen, sitä varten varatun miehen viemänä, erämaahan.
\par 22 Näin kauris kantakoon kaikki heidän pahat tekonsa autioon seutuun; ja kauris päästettäköön erämaahan.
\par 23 Ja Aaron menköön ilmestysmajaan ja riisukoon yltänsä pellavavaatteet, jotka hän oli pukenut päälleen mennessänsä pyhäkköön, ja jättäköön ne sinne.
\par 24 Ja hän pesköön pyhässä paikassa ruumiinsa vedessä ja pukekoon ylleen omat vaatteensa; sitten hän lähteköön sieltä ja uhratkoon sekä oman polttouhrinsa että kansan polttouhrin ja toimittakoon niin sekä itsellensä että kansalle sovituksen.
\par 25 Ja syntiuhrin rasvan hän polttakoon alttarilla.
\par 26 Ja se, joka päästi kauriin Asaselille, pesköön vaatteensa ja pesköön ruumiinsa vedessä, ja sitten hän tulkoon leiriin.
\par 27 Ja syntiuhrimullikka ja syntiuhrikauris, joiden veri tuotiin sovitukseksi pyhäkköön, vietäköön leirin ulkopuolelle, ja niiden nahka, liha ja rapa poltettakoon tulessa.
\par 28 Ja se, joka ne polttaa, pesköön vaatteensa ja pesköön ruumiinsa vedessä, ja sitten hän tulkoon leiriin.
\par 29 Tämä olkoon teille ikuinen säädös: seitsemännessä kuussa, kuukauden kymmenentenä päivänä, kurittakaa itseänne paastolla älkääkä yhtäkään askaretta toimittako, älköön maassa syntynyt älköönkä muukalainen, joka asuu teidän keskellänne.
\par 30 Sillä sinä päivänä toimitetaan teille sovitus teidän puhdistamiseksenne; kaikista synneistänne te tulette puhtaiksi Herran edessä.
\par 31 Se olkoon teille levon päivä, kurittakaa silloin itseänne paastolla; se olkoon ikuinen säädös.
\par 32 Ja sovituksen toimittakoon se pappi, joka on voideltu ja joka on vihitty papinvirkaan isänsä sijaan; hän pukekoon yllensä pellavavaatteet, pyhät vaatteet,
\par 33 ja toimittakoon sovituksen kaikkeinpyhimmälle ja sovituksen ilmestysmajalle ja alttarille sekä sovituksen myös papeille ja kaikelle seurakunnan kansalle.
\par 34 Tämä olkoon teille ikuinen säädös, että toimitatte israelilaisille sovituksen kaikista heidän synneistänsä kerran vuodessa." Ja hän teki, niinkuin Herra oli Moosekselle käskyn antanut.

\chapter{17}

\par 1 Ja Herra puhui Moosekselle sanoen:
\par 2 "Puhu Aaronille ja hänen pojillensa ja kaikille israelilaisille ja sano heille: Näin on Herra käskenyt sanoen:
\par 3 Kuka ikinä Israelin heimosta teurastaa härän tai karitsan tai vuohen leirissä tai leirin ulkopuolella
\par 4 eikä tuo sitä ilmestysmajan ovelle antaaksensa sen uhrilahjana Herralle Herran asumuksen edessä, hänet pidettäköön vereen vikapäänä; hän on vuodattanut verta, ja se mies hävitettäköön kansastansa.
\par 5 Sentähden israelilaiset tuokoot uhriteuraansa, jotka he ovat teurastaneet kedolla, Herran eteen ilmestysmajan ovelle, papille, ja siellä uhratkoot ne yhteysuhriksi Herralle.
\par 6 Ja pappi vihmokoon veren Herran alttarille ilmestysmajan oven edessä ja polttakoon rasvan suloiseksi tuoksuksi Herralle.
\par 7 Ja älkööt he enää uhratko uhrejansa metsänpeikoille, joiden jäljessä he haureudessa kulkevat. Tämä olkoon heille ikuinen säädös sukupolvesta sukupolveen.
\par 8 Ja sano heille: Kuka ikinä Israelin heimosta tai muukalaisista, jotka asuvat teidän keskellänne, uhraa polttouhrin tai teurasuhrin
\par 9 eikä tuo sitä ilmestysmajan ovelle, Herralle uhrattavaksi, se mies hävitettäköön kansastansa.
\par 10 Kuka ikinä Israelin heimosta tai muukalaisista, jotka asuvat teidän keskellänne, syö verta, mitä tahansa, sitä ihmistä vastaan, joka verta syö, minä käännän kasvoni ja hävitän hänet kansastansa.
\par 11 Sillä lihan sielu on veressä, ja minä olen sen teille antanut alttarille, että se tuottaisi teille sovituksen; sillä veri tuottaa sovituksen, koska sielu on siinä.
\par 12 Sentähden minä olen sanonut israelilaisille: Älköön kukaan teistä syökö verta; älköön myöskään muukalainen, joka asuu teidän keskellänne, syökö verta.
\par 13 Ja kuka ikinä Israelin heimosta tai muukalaisista, jotka asuvat teidän keskellänne, saa pyydystetyksi syötävän metsäeläimen tai linnun, hän vuodattakoon sen veren maahan ja peittäköön multaan.
\par 14 Sillä kaiken lihan sielu on sen veri, jossa sen sielu on; sentähden minä sanon israelilaisille: Älkää syökö minkään lihan verta. Sillä kaiken lihan sielu on sen veri; jokainen, joka sitä syö, hävitettäköön.
\par 15 Ja jokainen, olipa hän maassa syntynyt tai muukalainen, joka syö itsestään kuollutta tai kuoliaaksi raadeltua eläintä, pesköön vaatteensa ja peseytyköön vedessä ja olkoon saastainen iltaan asti; niin hän tulee puhtaaksi.
\par 16 Mutta jos hän ei pese vaatteitansa eikä pese ruumistansa, joutuu hän syynalaiseksi."

\chapter{18}

\par 1 Ja Herra puhui Moosekselle sanoen:
\par 2 "Puhu israelilaisille ja sano heille: Minä olen Herra, teidän Jumalanne.
\par 3 Älkää tehkö, niinkuin tehdään Egyptin maassa, jossa te asuitte, älkääkä tehkö, niinkuin tehdään Kanaanin maassa, johon minä teidät vien; älkää vaeltako heidän tapojensa mukaan.
\par 4 Vaan seuratkaa minun säädöksiäni ja noudattakaa minun käskyjäni vaeltaaksenne niiden mukaan. Minä olen Herra, teidän Jumalanne.
\par 5 Noudattakaa minun käskyjäni ja säädöksiäni, sillä se ihminen, joka ne pitää, on niistä elävä. Minä olen Herra.
\par 6 Älköön kukaan teistä ryhtykö veriheimolaiseensa paljastaakseen hänen häpyänsä. Minä olen Herra.
\par 7 Isäsi häpyä ja äitisi häpyä älä paljasta. Hän on sinun äitisi, älä hänen häpyänsä paljasta.
\par 8 Älä paljasta äitipuolesi häpyä, se on sinun isäsi häpy.
\par 9 Älä paljasta sisaresi häpyä, olipa hän isäsi tytär tai äitisi tytär, kotona tai ulkona syntynyt; älä paljasta heidän häpyänsä.
\par 10 Älä paljasta poikasi tyttären tai tyttäresi tyttären häpyä, sillä se on sinun oma häpysi.
\par 11 Älä paljasta äitipuolesi tyttären häpyä, tyttären, joka on sinun isällesi syntynyt; hän on sinun sisaresi.
\par 12 Älä paljasta isäsi sisaren häpyä; hän on sinun isäsi veriheimolainen.
\par 13 Älä paljasta äitisi sisaren häpyä, sillä hän on sinun äitisi veriheimolainen.
\par 14 Älä paljasta setäsi häpyä, älä ryhdy hänen vaimoonsa; hän on sinun tätisi.
\par 15 Älä paljasta miniäsi häpyä; hän on sinun poikasi vaimo, älä paljasta hänen häpyänsä.
\par 16 Älä paljasta veljesi vaimon häpyä; se on sinun veljesi häpy.
\par 17 Älä paljasta vaimon ja hänen tyttärensä häpyä, äläkä ota vaimoksi hänen poikansa tytärtä tai tyttärensä tytärtä paljastaaksesi hänen häpyänsä, sillä he ovat veriheimolaisia; se on sukurutsausta.
\par 18 Älä ota vaimosi eläessä hänen sisartansa vaimoksesi ja paljasta hänen häpyänsä, sillä se tuo riidan.
\par 19 Älä ryhdy vaimoon paljastaaksesi hänen häpyänsä, kun hän on saastainen kuukautisaikanansa.
\par 20 Älä makaa lähimmäisesi vaimon kanssa, ettet siten itseäsi saastuttaisi.
\par 21 Älä anna lapsiasi poltettaviksi uhrina Molokille, ettet häpäisisi Jumalasi nimeä. Minä olen Herra.
\par 22 Älä makaa miehenpuolen kanssa, niinkuin naisen kanssa maataan; se on kauhistus.
\par 23 Älä sekaannu mihinkään eläimeen, ettet siten itseäsi saastuttaisi. Älköönkä nainen tarjoutuko eläimelle pariutuaksensa sen kanssa; se on iljettävyys.
\par 24 Älkää saastuttako itseänne millään näistä, sillä näillä kaikilla ovat itsensä saastuttaneet ne kansat, jotka minä karkoitan teidän tieltänne.
\par 25 Ja maa tuli saastaiseksi, ja minä kostin sille sen pahat teot, niin että maa oksensi ulos asujamensa.
\par 26 Sentähden noudattakaa minun käskyjäni ja säädöksiäni älkääkä mitään näistä kauhistuksista tehkö, älköön maassa syntynyt älköönkä muukalainen, joka asuu teidän keskellänne -
\par 27 sillä kaikkia näitä kauhistuksia ovat tehneet tämän maan asukkaat, jotka olivat ennen teitä, niin että maa tuli saastaiseksi -
\par 28 ettei maa teitäkin oksentaisi, jos te sen saastutatte, niinkuin se on oksentanut ulos sen kansan, joka oli ennen teitä.
\par 29 Sillä jokainen, joka tekee minkä tahansa näistä kauhistuksista, hävitettäköön kansastansa, kaikki, jotka semmoista tekevät.
\par 30 Sentähden noudattakaa minun määräyksiäni, ettette seuraisi niitä kauhistavia tapoja, joita on seurattu ennen teitä, ettekä saastuttaisi itseänne niillä. Minä olen Herra, teidän Jumalanne."

\chapter{19}

\par 1 Ja Herra puhui Moosekselle sanoen:
\par 2 "Puhu kaikelle israelilaisten seurakunnalle ja sano heille: Olkaa pyhät, sillä minä, Herra, teidän Jumalanne, olen pyhä.
\par 3 Jokainen teistä peljätköön äitiänsä ja isäänsä, ja pitäkää minun sapattini. Minä olen Herra, teidän Jumalanne.
\par 4 Älkää kääntykö epäjumalien puoleen älkääkä tehkö itsellenne valettuja jumalankuvia. Minä olen Herra, teidän Jumalanne.
\par 5 Ja kun te uhraatte yhteysuhria Herralle, uhratkaa se niin, että hänen mielisuosionsa tulee teidän osaksenne.
\par 6 Syökää se samana päivänä, jona te uhraatte, tai seuraavana päivänä; mutta mitä tähteeksi jää kolmanteen päivään, se poltettakoon tulessa.
\par 7 Jos sitä syödään kolmantena päivänä, niin se on saastaista, ei otollista.
\par 8 Ja joka sitä syö, joutuu syynalaiseksi, sillä hän on häväissyt Herran pyhän, ja hänet hävitettäköön kansastansa.
\par 9 Kun te korjaatte eloa maastanne, niin älä leikkaa viljaa pelloltasi reunoja myöten äläkä poimi tähkäpäitä leikkuun jälkeen.
\par 10 Älä myöskään korjaa tyhjäksi viinitarhaasi äläkä poimi varisseita marjoja, vaan jätä ne köyhälle ja muukalaiselle. Minä olen Herra, teidän Jumalanne.
\par 11 Älkää varastako, älkää valhetelko älkääkä lähimmäistänne pettäkö.
\par 12 Älkää vannoko väärin minun nimeeni, ettet häpäisisi Jumalasi nimeä. Minä olen Herra.
\par 13 Älä tee lähimmäisellesi vääryyttä äläkä ota mitään väkisin. Päivämiehesi palkka älköön olko sinun takanasi huomiseen asti.
\par 14 Älä kiroa kuuroa äläkä pane kompastusta sokean eteen, vaan pelkää Jumalaasi. Minä olen Herra.
\par 15 Älkää tehkö vääryyttä tuomitessanne; älä ole puolueellinen köyhän hyväksi äläkä pidä ylhäisen puolta, vaan tuomitse lähimmäisesi oikein.
\par 16 Älä liiku panettelijana kansasi keskellä äläkä vaani lähimmäisesi verta. Minä olen Herra.
\par 17 Älä vihaa veljeäsi sydämessäsi, vaan nuhtele lähimmäistäsi, ettet joutuisi hänen tähtensä syynalaiseksi.
\par 18 Älä kosta äläkä pidä vihaa kansasi lapsia vastaan, vaan rakasta lähimmäistäsi niinkuin itseäsi. Minä olen Herra.
\par 19 Noudattakaa minun käskyjäni. Älä anna karjassasi kahden erilaisen eläimen pariutua, älä kylvä peltoosi kahdenlaista siementä, älköönkä kahdenlaisista langoista kudottua vaatetta tulko yllesi.
\par 20 Jos mies makaa naisen, joka on orjatar ja toisen miehen oma eikä ole lunastettu eikä vapaaksi laskettu, rangaistakoon heitä, ei kuitenkaan kuolemalla, koska nainen ei ollut vapaa.
\par 21 Ja mies tuokoon hyvityksenään Herralle vikauhri-oinaan ilmestysmajan ovelle.
\par 22 Ja kun pappi on toimittanut hänelle vikauhri-oinaalla Herran edessä sovituksen siitä synnistä, jonka hän on tehnyt, annetaan hänelle hänen tekemänsä synti anteeksi.
\par 23 Kun te tulette siihen maahan ja istutatte kaikkinaisia hedelmäpuita, niin jättäkää niiden hedelmä, niiden esinahka, koskemattomaksi; kolmena vuotena pitäkää ne ympärileikkaamattomina, älkää niitä syökö.
\par 24 Mutta neljäntenä vuotena pyhitettäköön kaikki niiden hedelmät ilojuhlassa Herralle,
\par 25 ja vasta viidentenä vuotena syökää niiden hedelmää; näin te lisäätte niiden tuottoa. Minä olen Herra, teidän Jumalanne.
\par 26 Älkää syökö mitään verinensä. Älkää merkeistä ennustelko älkääkä harjoittako noituutta.
\par 27 Älkää keritkö tukkaanne päälaen ympäriltä, äläkä leikkaamalla turmele partasi reunaa.
\par 28 Älkää viileskelkö ihoanne vainajan tähden älkääkä koristelko itseänne ihopiirroksilla. Minä olen Herra.
\par 29 Älä häpäise tytärtäsi antamalla hänen tulla portoksi, ettei maa harjoittaisi haureutta ja tulisi täyteen iljettävyyttä.
\par 30 Pitäkää minun sapattini ja peljätkää minun pyhäkköäni. Minä olen Herra.
\par 31 Älkää kääntykö vainaja- ja tietäjähenkien puoleen; älkää etsikö heitä, ettette tulisi heistä saastutetuiksi. Minä olen Herra, teidän Jumalanne.
\par 32 Nouse harmaapään edessä ja kunnioita vanhusta sekä pelkää Jumalaasi. Minä olen Herra.
\par 33 Kun muukalainen asuu luonasi teidän maassanne, älkää sortako häntä.
\par 34 Muukalainen, joka asuu teidän luonanne, olkoon niinkuin maassa syntynyt teikäläinen. Rakasta häntä niinkuin itseäsi, sillä tekin olitte muukalaisina Egyptin maassa. Minä olen Herra, teidän Jumalanne.
\par 35 Älkää tehkö vääryyttä tuomitessanne älkääkä käyttäessänne pituus-, paino- tai astiamittaa.
\par 36 Olkoon teillä oikea vaaka, oikeat punnukset, oikea eefa-mitta ja oikea hiin-mitta. Minä olen Herra, teidän Jumalanne, joka vein teidät pois Egyptin maasta.
\par 37 Noudattakaa kaikkia minun käskyjäni ja kaikkia minun säädöksiäni ja pitäkää ne. Minä olen Herra."

\chapter{20}

\par 1 Ja Herra puhui Moosekselle sanoen:
\par 2 "Sano israelilaisille: Jos joku, kuka tahansa israelilainen tai muukalainen, joka asuu Israelissa, antaa lapsiansa Molokille, hänet rangaistakoon kuolemalla; maan kansa kivittäköön hänet.
\par 3 Ja minä käännän kasvoni sitä miestä vastaan ja hävitän hänet hänen kansastansa, koska hän on antanut Molokille lapsiansa saastuttaen minun pyhäkköni ja häväisten minun pyhän nimeni.
\par 4 Ja jos maan kansa säästää sitä miestä, joka antaa lapsiansa Molokille, niin etteivät he häntä surmaa,
\par 5 niin minä itse käännän kasvoni sitä miestä vastaan ja hänen sukuansa vastaan; hänet ja kaikki, jotka häntä seuraten ovat lähteneet haureudessa kulkemaan Molokin jäljessä, minä hävitän heidän kansastansa.
\par 6 Ja jos joku kääntyy vainaja- tai tietäjähenkien puoleen, lähtien haureudessa kulkemaan heidän jäljessänsä, sitä ihmistä vastaan minä käännän kasvoni ja hävitän hänet hänen kansastansa.
\par 7 Pyhittäkää siis itsenne ja olkaa pyhät; sillä minä olen Herra, teidän Jumalanne.
\par 8 Ja noudattakaa minun käskyjäni ja pitäkää ne. Minä olen Herra, joka pyhitän teidät.
\par 9 Kuka ikinä kiroaa isäänsä tai äitiänsä, hänet rangaistakoon kuolemalla; isäänsä ja äitiänsä hän on kironnut, hän on verivelan alainen.
\par 10 Jos joku tekee aviorikoksen toisen miehen vaimon kanssa, jos hän tekee aviorikoksen lähimmäisensä vaimon kanssa, niin heidät, sekä avionrikkoja mies että -nainen, rangaistakoon kuolemalla.
\par 11 Jos joku makaa äitipuolensa kanssa, paljastaa hän isänsä hävyn; heidät molemmat rangaistakoon kuolemalla, he ovat verivelan alaiset.
\par 12 Jos joku makaa miniänsä kanssa, rangaistakoon heidät molemmat kuolemalla; he ovat tehneet iljettävyyden, he ovat verivelan alaiset.
\par 13 Jos joku makaa miehenpuolen kanssa, niinkuin naisen kanssa maataan, tekevät he molemmat kauhistuksen; heidät rangaistakoon kuolemalla, he ovat verivelan alaiset.
\par 14 Jos joku ottaa vaimoksi naisen ja hänen äitinsä, on se iljettävyys; sekä hänet että ne molemmat poltettakoon tulessa, ettei mitään iljettävyyttä olisi teidän seassanne.
\par 15 Jos joku sekaantuu eläimeen, rangaistakoon hänet kuolemalla, ja se eläin tappakaa.
\par 16 Ja jos nainen ryhtyy eläimeen, mihin tahansa, pariutuaksensa sen kanssa, niin surmaa sekä nainen että eläin; heidät rangaistakoon kuolemalla, he ovat verivelan alaiset.
\par 17 Jos joku ottaa vaimoksi sisarensa, isänsä tyttären tai äitinsä tyttären, ja näkee hänen häpynsä ja tämä näkee hänen häpynsä, niin se on häpeä, ja heidät hävitettäköön kansalaistensa silmien edessä; hän on paljastanut sisarensa hävyn, hän on joutunut syynalaiseksi.
\par 18 Jos joku makaa kuukautistilassa olevan naisen kanssa ja paljastaa hänen häpynsä, avaa hänen lähteensä, ja nainen paljastaa verensä lähteen, niin heidät molemmat hävitettäköön kansastansa.
\par 19 Älä paljasta äitisi sisaren tai isäsi sisaren häpyä; sillä se, joka niin tekee, on paljastanut veriheimolaisensa; he joutuvat syynalaisiksi.
\par 20 Jos joku makaa setänsä vaimon kanssa, on hän paljastanut setänsä hävyn; he joutuvat syynalaisiksi, lapsettomina he kuolkoot.
\par 21 Jos joku ottaa vaimoksi veljensä vaimon, on se saastaisuus; hän on paljastanut veljensä hävyn, lapsettomiksi he jääkööt.
\par 22 Noudattakaa siis kaikkia minun käskyjäni ja kaikkia minun säädöksiäni ja pitäkää ne, ettei teitä oksentaisi ulos se maa, johon minä vien teidät asumaan.
\par 23 Ja älkää vaeltako sen kansan tapojen mukaan, jonka minä karkoitan teidän tieltänne; sillä kaikkia näitä he ovat tehneet ja nostaneet minun inhoni.
\par 24 Mutta minä sanon teille: Te saatte ottaa omaksenne heidän maansa, sillä minä annan sen teille omaksi, maan, joka vuotaa maitoa ja mettä. Minä olen Herra, teidän Jumalanne, joka olen teidät erottanut muista kansoista.
\par 25 Erottakaa myös puhtaat eläimet saastaisista ja saastaiset linnut puhtaista, ettette saattaisi itseänne inhottaviksi koskemalla nelijalkaisiin eläimiin, lintuihin tai maan matelijoihin, mihin tahansa, jotka minä olen erottanut, että te pitäisitte ne saastaisina.
\par 26 Olkaa siis minulle pyhät, sillä minä, Herra, olen pyhä, ja minä olen teidät erottanut muista kansoista olemaan minun omani.
\par 27 Jos jossakin miehessä tai naisessa on vainaja- tai tietäjähenki, niin heidät rangaistakoon kuolemalla; heidät kivitettäköön, he ovat verivelan alaiset."

\chapter{21}

\par 1 Ja Herra sanoi Moosekselle: "Puhu papeille, Aaronin pojille, ja sano heille: Pappi älköön saattako itseänsä saastaiseksi kenestäkään kuolleesta sukulaisestansa,
\par 2 paitsi lähimmästä veriheimolaisestaan: äidistänsä, isästänsä, pojastansa, tyttärestänsä ja veljestänsä,
\par 3 sekä sisarestansa, jos tämä oli neitsyt ja eli hänen luonaan eikä ollut naitu; hänestä hän saa saastua.
\par 4 Älköön hän, sukunsa päämies, saattako itseänsä saastaiseksi, ettei hän itseänsä häpäisisi.
\par 5 Älkööt he ajako paljaaksi päälakeansa älköötkä leikatko partansa reunaa älköötkä koristelko itseänsä ihopiirroksilla.
\par 6 He olkoot Jumalallensa pyhät älköötkä häväiskö Jumalansa nimeä, sillä he uhraavat Herran uhreja, Jumalansa leipää; sentähden he olkoot pyhät.
\par 7 Älkööt he ottako vaimoksi porttoa tai häväistyä naista älköötkä ottako miehensä hylkäämää vaimoa, sillä pappi on pyhä Jumalallensa.
\par 8 Sinun on pidettävä häntä pyhänä, koska hän uhraa sinun Jumalasi leipää; hän olkoon sinulle pyhä, sillä minä, Herra, joka pyhitän teidät, olen pyhä.
\par 9 Jos papin tytär häpäisee itsensä haureudella, niin hän häpäisee isänsä; hänet poltettakoon tulessa.
\par 10 Se pappi, joka on ylimmäinen veljiensä seassa, jonka pään päälle voiteluöljy on vuodatettu ja joka pyhiin vaatteisiin pukemalla on vihitty virkaansa, älköön päästäkö tukkaansa hajalle älköönkä reväiskö vaatteitansa.
\par 11 Älköön hän myöskään menkö minkään kuolleen luo; isästänsäkään tai äidistänsä hän älköön saattako itseänsä saastaiseksi.
\par 12 Älköön hän lähtekö pyhäköstä, ettei hän häpäisisi Jumalansa pyhäkköä, sillä hänessä on hänen Jumalansa voiteluöljyn vihkimys. Minä olen Herra.
\par 13 Hän ottakoon vaimoksensa neitsyen.
\par 14 Leskeä tai hyljättyä vaimoa tai häväistyä naista tai porttoa hän älköön ottako, vaan neitsyen omasta suvustaan hän ottakoon vaimoksensa,
\par 15 ettei hän häpäisisi jälkeläisiään suvussansa; sillä minä olen Herra, joka hänet pyhitän."
\par 16 Ja Herra puhui Moosekselle sanoen:
\par 17 "Puhu Aaronille ja sano: jos jossakin sinun jälkeläisistäsi, tulevissa sukupolvissa, on joku vamma, älköön hän lähestykö uhraamaan Jumalansa leipää.
\par 18 Älköön lähestykö kukaan, jossa on joku vamma, älköön kukaan sokea tai ontuva, kukaan kasvoiltaan tahi jäseniltään muodoton,
\par 19 jalkansa tai kätensä taittanut,
\par 20 kyttyräselkäinen tai surkastunut, silmävikainen tai ihotautinen tai rupinen tai kuohittu.
\par 21 Pappi Aaronin jälkeläisistä älköön kukaan, jossa on joku vamma, lähestykö uhraamaan Herran uhreja; hänessä on vamma, sentähden hän älköön lähestykö uhraamaan Jumalansa leipää.
\par 22 Kuitenkin hän syököön Jumalansa leipää, sekä korkeasti-pyhää että pyhää.
\par 23 Mutta älköön hän tulko esiripun luo älköönkä lähestykö alttaria, koska hänessä on vamma, ettei hän häpäisisi minun pyhäkköäni; sillä minä olen Herra, joka pyhitän heidät."
\par 24 Ja Mooses sanoi tämän Aaronille ja hänen pojillensa ja kaikille israelilaisille.

\chapter{22}

\par 1 Ja Herra puhui Moosekselle sanoen:
\par 2 "Käske Aaronia ja hänen poikiansa pyhästi pitelemään israelilaisten pyhiä lahjoja, heidän minulle pyhittämiänsä, etteivät he häpäisisi minun pyhää nimeäni. Minä olen Herra.
\par 3 Sano heille polvesta polveen: Jos kuka tahansa teidän jälkeläisistänne saastuneena ollessaan lähestyy niitä pyhiä lahjoja, joita israelilaiset pyhittävät Herralle, hänet hävitettäköön minun kasvojeni edestä.
\par 4 Minä olen Herra. Älköön kukaan Aaronin jälkeläinen, joka on pitalinen tai jolla on vuoto, syökö pyhistä lahjoista, ennenkuin hän on tullut puhtaaksi; ja joka koskee johonkin kuolleesta saastuneeseen tahi jolla on siemenvuoto,
\par 5 tahi joka koskee pikkueläimeen, mihin tahansa, josta saastuu, tai ihmiseen, josta saastuu, olipa sen saastaisuus mitä laatua tahansa,
\par 6 niin jokainen, joka koskee johonkin sellaiseen, olkoon saastainen iltaan asti älköönkä syökö pyhistä lahjoista, ellei hän ole pessyt ruumistaan vedessä.
\par 7 Mutta kun aurinko on laskenut, on hän puhdas, ja sitten hän syököön pyhistä lahjoista, sillä se on hänen ruokansa.
\par 8 Itsestään kuollutta tai kuoliaaksi raadeltua eläintä hän älköön syökö, ettei hän siten saastuisi. Minä olen Herra.
\par 9 He noudattakoot minun määräyksiäni, etteivät joutuisi syynalaisiksi pyhitetyn tähden ja kuolisi, sentähden että häpäisevät pyhitetyn. Minä olen Herra, joka pyhitän heidät.
\par 10 Yksikään syrjäinen älköön syökö pyhästä lahjasta; papin loinen tai päiväpalkkalainen älköön syökö pyhästä lahjasta.
\par 11 Mutta kun pappi ostaa rahallansa orjan, saakoon tämä siitä syödä; samoin hänen kodissaan syntyneet orjat syökööt hänen ruokaansa.
\par 12 Jos papin tytär tulee syrjäisen miehen vaimoksi, älköön hän syökö pyhistä antilahjoista.
\par 13 Mutta jos papin tytär tulee leskeksi tai hyljätyksi ja on lapseton ja hän tulee takaisin isänsä taloon ja asuu siellä niinkuin nuoruudessaan, niin hän syököön isänsä ruokaa; mutta kukaan syrjäinen älköön sitä syökö.
\par 14 Ja jos joku erehdyksessä syö pyhästä lahjasta, korvatkoon papille pyhän lahjan ja pankoon siihen vielä lisäksi viidennen osan sen arvosta.
\par 15 Älkööt papit häväiskö israelilaisten pyhiä lahjoja, joita nämä antavat Herralle,
\par 16 älköötkä saattako heitä syynalaisiksi ja vikapäiksi sallimalla heidän syödä pyhiä lahjojansa; sillä minä olen Herra, joka pyhitän heidät."
\par 17 Ja Herra puhui Moosekselle sanoen:
\par 18 "Puhu Aaronille ja hänen pojillensa ja kaikille israelilaisille ja sano heille: Kuka ikinä Israelin heimosta tai muukalaisista Israelissa tuo polttouhriksi uhrilahjansa, olipa se mikä hyvänsä lupausuhri tai vapaaehtoinen uhri, jonka he tuovat Herralle,
\par 19 niin tuokaa sellainen uhri, että hänen mielisuosionsa tulee teidän osaksenne: virheetön urospuoli raavaista, lampaista tai vuohista.
\par 20 Mutta yhtään eläintä, jossa on vamma, älkää tuoko, sillä siitä hänen mielisuosionsa ei tule teidän osaksenne.
\par 21 Jos joku tuo Herralle yhteysuhrin raavaista tai lampaista, joko lupausta täyttääkseen tai vapaaehtoisena uhrina, olkoon eläin virheetön ollakseen otollinen; älköön siinä olko mitään vammaa.
\par 22 Sokeata tai raajarikkoa tai silvottua tai paiseista tai ihotautista tai rupista älkää Herralle tuoko, älkääkä sellaisia asettako alttarille, uhriksi Herralle.
\par 23 Härän tai lampaan, joka on jäseniltään muodoton tai vaivainen, sinä saat uhrata vapaaehtoisena uhrina, mutta lupausuhriksi se ei ole otollinen.
\par 24 Älkää uhratko Herralle eläintä, joka on kuohittu puristamalla tai musertamalla tai repimällä tai leikkaamalla; älkää tehkö maassanne sellaista.
\par 25 Älkää myöskään muukalaiselta ostako mitään sellaisia eläimiä tuodaksenne niitä Jumalanne ruuaksi, sillä ne ovat turmeltuja, niissä on vamma; niistä ei tule teidän osaksenne mielisuosio."
\par 26 Ja Herra puhui Moosekselle sanoen:
\par 27 "Kun vasikka tai karitsa tai vohla on syntynyt, olkoon se emänsä luona seitsemän päivää; mutta kahdeksannesta päivästä alkaen se on otollinen uhrilahjaksi Herralle.
\par 28 Älkää raavaseläintä tai lammasta ja sen sikiötä teurastako samana päivänä.
\par 29 Kun te uhraatte Herralle kiitosuhrin, uhratkaa se niin, että hänen mielisuosionsa tulee teidän osaksenne.
\par 30 Se syötäköön samana päivänä; älkää jättäkö siitä mitään seuraavaan aamuun. Minä olen Herra.
\par 31 Noudattakaa siis minun käskyjäni ja pitäkää ne. Minä olen Herra.
\par 32 Älkää häväiskö minun pyhää nimeäni: minä tahdon olla pyhä israelilaisten keskellä. Minä olen Herra, joka pyhitän teidät
\par 33 ja joka vein teidät pois Egyptin maasta, ollakseni teidän Jumalanne. Minä olen Herra."

\chapter{23}

\par 1 Ja Herra puhui Moosekselle sanoen:
\par 2 "Puhu israelilaisille ja sano heille: Herran juhla-ajat, joiksi teidän on kuulutettava pyhät kokoukset, minun juhla-aikani, ovat nämä.
\par 3 Kuusi päivää tehtäköön työtä, mutta seitsemäntenä päivänä on sapatti, levon päivä, pyhä kokous; silloin älkää yhtäkään askaretta toimittako, se on Herran sapatti, missä asuttekin.
\par 4 Nämä ovat Herran juhla-ajat, pyhät kokoukset, jotka teidän on kuulutettava määräajallansa:
\par 5 Ensimmäisessä kuussa, kuukauden neljäntenätoista päivänä, iltahämärässä, on pääsiäinen Herran kunniaksi.
\par 6 Ja saman kuukauden viidentenätoista päivänä on happamattoman leivän juhla Herran kunniaksi; syökää happamatonta leipää seitsemän päivää.
\par 7 Ensimmäisenä päivänä olkoon teillä pyhä kokous, silloin älkää yhtäkään arkiaskaretta toimittako.
\par 8 Tuokaa Herralle uhri seitsemänä päivänä. Seitsemäntenä päivänä on pyhä kokous; silloin älkää yhtäkään arkiaskaretta toimittako."
\par 9 Ja Herra puhui Moosekselle sanoen:
\par 10 "Puhu israelilaisille ja sano heille: Kun te tulette siihen maahan, jonka minä teille annan, ja leikkaatte sen viljaa, niin viekää papille viljastanne uutislyhde.
\par 11 Ja hän toimittakoon sen lyhteen heilutuksen Herran edessä, että hänen mielisuosionsa tulisi teidän osaksenne; sapatin jälkeisenä päivänä pappi toimittakoon sen heilutuksen.
\par 12 Ja sinä päivänä, jona teidän lyhteenne heilutus toimitetaan, uhratkaa virheetön, vuoden vanha karitsa polttouhriksi Herralle,
\par 13 ja siihen kuuluvaksi ruokauhriksi kaksi kymmenennestä lestyjä jauhoja, öljyyn sekoitettuna, uhriksi, suloiseksi tuoksuksi Herralle, sekä siihen kuuluvaksi juomauhriksi neljännes hiin-mittaa viiniä.
\par 14 Älkääkä syökö uutisleipää, paahdettuja jyviä tai tuleentumatonta viljaa ennen sitä päivää, jona tuotte Jumalallenne uhrilahjan. Se olkoon teille ikuinen säädös sukupolvesta sukupolveen, missä asuttekin.
\par 15 Sitten laskekaa sapatin jälkeisestä päivästä, siitä päivästä, jona toitte heilutuslyhteen, seitsemän täyttä viikkoa,
\par 16 laskekaa viisikymmentä päivää seitsemännen sapatin jälkeiseen päivään asti; sitten tuokaa Herralle uusi ruokauhri.
\par 17 Sieltä, missä asutte, tuokaa heilutusleiväksi kaksi kakkua, jotka on leivottava happamena kahdesta kymmenenneksestä lestyjä jauhoja, Herralle uutislahjaksi.
\par 18 Ja tuokaa leipänne ohella seitsemän virheetöntä, vuoden vanhaa karitsaa ja mullikka sekä kaksi oinasta; ruoka- ja juomauhreineen ne olkoot polttouhri Herralle, suloisesti tuoksuva uhri Herralle.
\par 19 Ja uhratkaa kauris syntiuhriksi sekä kaksi vuoden vanhaa karitsaa yhteysuhriksi.
\par 20 Ja pappi toimittakoon niiden ja uutisleivän ynnä kahden karitsan heilutuksen Herran edessä. Ne olkoot Herralle pyhät ja papin omat.
\par 21 Ja kuuluttakaa pyhä kokous juuri siksi päiväksi; silloin älkää yhtäkään arkiaskaretta toimittako. Se olkoon teille ikuinen säädös sukupolvesta sukupolveen, missä asuttekin.
\par 22 Kun te korjaatte eloa maastanne, niin älä leikkaa viljaa pelloltasi reunoja myöten äläkä poimi tähkäpäitä leikkuun jälkeen, vaan jätä ne köyhälle ja muukalaiselle. Minä olen Herra, teidän Jumalanne."
\par 23 Ja Herra puhui Moosekselle sanoen:
\par 24 "Puhu israelilaisille ja sano: Seitsemännessä kuussa, kuukauden ensimmäisenä päivänä, pitäkää sapatinlepo, muistojuhla pasunaa soittaen, pyhä kokous.
\par 25 Älkää silloin yhtäkään arkiaskaretta toimittako, vaan tuokaa uhri Herralle."
\par 26 Ja Herra puhui Moosekselle sanoen:
\par 27 "Tämän seitsemännen kuun kymmenentenä päivänä on sovituspäivä; pitäkää silloin pyhä kokous, kurittakaa itseänne paastolla ja tuokaa Herralle uhri.
\par 28 Älkää toimittako yhtäkään askaretta sinä päivänä, sillä se on sovituspäivä, jolloin teille toimitetaan sovitus Herran, teidän Jumalanne, edessä.
\par 29 Sillä jokainen, joka ei sinä päivänä kurita itseänsä paastolla, hävitettäköön kansastansa.
\par 30 Ja jokaisen, joka sinä päivänä toimittaa jotakin askaretta, minä hävitän hänen kansastansa.
\par 31 Älkää silloin yhtäkään askaretta toimittako. Se olkoon teille ikuinen säädös sukupolvesta sukupolveen, missä asuttekin.
\par 32 Se on oleva teille levon päivä, kurittakaa itseänne paastolla. Kuukauden yhdeksäntenä päivänä illalla, illasta iltaan, pitäkää tämä sapatti."
\par 33 Ja Herra puhui Moosekselle sanoen:
\par 34 "Puhu israelilaisille ja sano: Tämän seitsemännen kuun viidentenätoista päivänä on lehtimajanjuhla Herran kunniaksi; se kestää seitsemän päivää.
\par 35 Ensimmäisenä päivänä pidettäköön pyhä kokous; älkää silloin yhtäkään arkiaskaretta toimittako.
\par 36 Seitsemänä päivänä tuokaa uhri Herralle. Kahdeksantena päivänä olkoon teillä pyhä kokous, ja tuokaa uhri Herralle; se on juhlakokous, älkää silloin yhtäkään arkiaskaretta toimittako.
\par 37 Nämä ovat Herran juhla-ajat, joiksi teidän on kuulutettava pyhät kokoukset, tuodaksenne Herralle uhreja: polttouhreja ja ruokauhreja, teurasuhreja ja juomauhreja, kunakin päivänä sen päivän uhrit;
\par 38 ja sitäpaitsi Herran sapatit ja teidän muut lahjanne ja kaikki lupausuhrinne ja kaikki vapaaehtoiset uhrinne, joita te Herralle annatte.
\par 39 Mutta seitsemännen kuukauden viidentenätoista päivänä, kun te olette korjanneet maan sadon, viettäkää Herran juhlaa seitsemän päivää; ensimmäinen päivä on levon päivä, ja kahdeksas päivä on myös levon päivä.
\par 40 Ensimmäisenä päivänä ottakaa ihania hedelmiä puista, palmunoksia ja tuuheiden puiden lehviä sekä pajuja purojen varsilta, ja pitäkää iloa seitsemän päivää Herran, Jumalanne edessä.
\par 41 Ja viettäkää sitä juhlana Herran kunniaksi seitsemän päivää vuodessa. Se olkoon teille ikuinen säädös sukupolvesta sukupolveen; viettäkää se seitsemännessä kuussa.
\par 42 Seitsemän päivää asukaa lehtimajoissa; kaikki Israelissa syntyneet asukoot lehtimajoissa,
\par 43 että teidän jälkeläisenne tietäisivät, kuinka minä annoin israelilaisten asua lehtimajoissa, kun vein heidät pois Egyptin maasta. Minä olen Herra, teidän Jumalanne."
\par 44 Ja Mooses puhui näistä Herran juhla-ajoista israelilaisille.

\chapter{24}

\par 1 Ja Herra puhui Moosekselle sanoen:
\par 2 "Käske israelilaisten tuoda sinulle puhdasta, survomalla saatua öljypuun öljyä seitsenhaaraista lamppua varten, että lamput aina voidaan nostaa paikoilleen.
\par 3 Ulkopuolella esirippua, joka on lain arkin edessä ilmestysmajassa, Aaron hoitakoon sitä niin, että se aina, ehtoosta aamuun asti, palaa Herran edessä. Se olkoon teille ikuinen säädös sukupolvesta sukupolveen.
\par 4 Aitokultaisen seitsenhaaraisen lampun lamppuja hän hoitakoon niin, että ne aina palavat Herran edessä.
\par 5 Ja ota lestyjä jauhoja ja leivo niistä kaksitoista kakkua; joka kakussa olkoon niitä kaksi kymmenennestä.
\par 6 Ja lado ne päälletysten kahteen pinoon, kuusi pinoonsa, aitokultaiselle pöydälle Herran eteen.
\par 7 Ja pane pinojen päälle puhdasta suitsuketta leipien alttariuhriosana uhriksi Herralle.
\par 8 Sapatti sapatilta hän aina asettakoon ne Herran eteen israelilaisten antina; se on ikuinen liitto.
\par 9 Ne olkoot Aaronin ja hänen poikiensa omat, ja he syökööt ne pyhässä paikassa, sillä ne ovat korkeasti-pyhät; ne ovat hänen ikuinen osuutensa Herran uhreista."
\par 10 Erään israelilaisen vaimon poika, jonka isä oli egyptiläinen mies, oli lähtenyt maasta israelilaisten mukana; ja tämän israelilaisen vaimon poika ja muuan israelilainen mies riitelivät keskenänsä leirissä.
\par 11 Ja israelilaisen vaimon poika pilkkasi Herran nimeä ja kirosi sitä. Silloin he toivat hänet Mooseksen eteen. Ja hänen äitinsä nimi oli Selomit, Dibrin tytär, Daanin sukukunnasta.
\par 12 Ja he panivat hänet vankeuteen, kunnes heille ilmoitettaisiin Herran vastaus.
\par 13 Ja Herra puhui Moosekselle sanoen:
\par 14 "Vie kirooja leirin ulkopuolelle, ja kaikki, jotka kuulivat sen, laskekoot kätensä hänen päänsä päälle, ja koko kansa kivittäköön hänet.
\par 15 Ja puhu israelilaisille ja sano: Kuka ikinä Jumalaansa kiroaa, se joutuu syynalaiseksi.
\par 16 Ja joka Herran nimeä pilkkaa, rangaistakoon kuolemalla; koko kansa kivittäköön hänet kuoliaaksi. Olipa se muukalainen tai maassa syntynyt, joka pilkkaa Herran nimeä, hänet surmattakoon.
\par 17 Jos joku lyö kuoliaaksi ihmisen, kenen hyvänsä, hänet rangaistakoon kuolemalla.
\par 18 Mutta joka lyö kuoliaaksi kotieläimen, korvatkoon sen: henki hengestä.
\par 19 Ja joka tuottaa lähimmäisellensä vamman, sille tehtäköön, niinkuin hänkin on tehnyt:
\par 20 ruhje ruhjeesta, silmä silmästä, hammas hampaasta; saman vamman, jonka hän on toiselle tuottanut, saakoon hän itsekin.
\par 21 Joka lyö kuoliaaksi kotieläimen, korvatkoon sen; mutta joka lyö kuoliaaksi ihmisen, se surmattakoon.
\par 22 Sama laki olkoon teillä, niin muukalaisella kuin maassa syntyneelläkin; sillä minä olen Herra, teidän Jumalanne."
\par 23 Ja Mooses puhui näin israelilaisille; ja he veivät kiroojan leirin ulkopuolelle ja kivittivät hänet kuoliaaksi. Näin israelilaiset tekivät, niinkuin Herra oli Moosekselle käskyn antanut.

\chapter{25}

\par 1 Ja Herra puhui Moosekselle Siinain vuorella sanoen:
\par 2 "Puhu israelilaisille ja sano heille: Kun te tulette siihen maahan, jonka minä teille annan, niin maa pitäköön sapattia Herran kunniaksi.
\par 3 Kuutena vuotena kylvä peltosi ja kuutena vuotena leikkaa viinitarhasi ja korjaa niiden sato,
\par 4 mutta seitsemäntenä vuotena olkoon maalla levon aika, sapatti Herran kunniaksi; silloin älä kylvä peltoasi äläkä leikkaa viinitarhaasi.
\par 5 Leikatun viljasi jälkikasvua älä leikkaa, äläkä poimi viinirypäleitä, jotka ovat kasvaneet leikkaamattomissa viinipuissasi. Silloin olkoon maalla levon vuosi.
\par 6 Mutta mitä itsestänsä kasvaa maan levätessä, olkoon teille ruuaksi, sinulle itsellesi, sinun palvelijallesi ja palvelijattarellesi, päiväpalkkalaisellesi ja loisellesi, jotka luonasi asuvat.
\par 7 Ja karjallesi ja metsäeläimille sinun maassasi olkoon kaikki sen sato ruuaksi.
\par 8 Ja laske seitsemän vuosiviikkoa: seitsemän kertaa seitsemän vuotta, niin että seitsemän vuosiviikon aika on neljäkymmentä yhdeksän vuotta.
\par 9 Ja seitsemännessä kuussa, kuukauden kymmenentenä päivänä anna pasunan raikuen soida; sovituspäivänä antakaa pasunan soida koko maassanne.
\par 10 Ja pyhittäkää viideskymmenes vuosi ja julistakaa vapautus maassa kaikille sen asukkaille. Se olkoon teille riemuvuosi; jokainen teistä saa silloin palata perintömaallensa ja sukunsa luo.
\par 11 Riemuvuosi olkoon teille viideskymmenes vuosi; älkää silloin kylväkö älkääkä jälkikasvua leikatko, älkääkä poimiko, mitä leikkaamattomissa viinipuissa on kasvanut.
\par 12 Sillä se on riemuvuosi, se olkoon teille pyhä; pellolta syökää sen sato.
\par 13 Riemuvuotena saa jokainen teistä palata perintömaallensa.
\par 14 Jos myyt jotakin lähimmäisellesi tai ostat jotakin lähimmäiseltäsi, niin älkää tehkö vääryyttä toinen toisellenne.
\par 15 Vaan maksa lähimmäisellesi riemuvuoden jälkeisten vuosien luvun mukaan, satovuosien luvun mukaan hän saakoon maksun sinulta.
\par 16 Kuta useampia vuosia on, maksa sitä suurempi hinta, ja kuta vähemmän vuosia on, maksa sitä pienempi hinta; sillä satomäärän hän myy sinulle.
\par 17 Älköön teistä kukaan tehkö vääryyttä lähimmäisellensä, vaan pelkää Jumalaasi; sillä minä olen Herra, teidän Jumalanne.
\par 18 Sentähden pitäkää minun käskyni ja noudattakaa minun säädöksiäni ja pitäkää ne, niin te saatte turvallisesti asua maassa.
\par 19 Ja maa on antava hedelmänsä, ja teillä on kyllin syötävää, ja te saatte turvallisesti asua siinä.
\par 20 Ja jos te sanotte: 'Mitä me syömme seitsemäntenä vuotena, kun meidän ei ole lupa kylvää eikä korjata satoamme?'
\par 21 niin minä käsken siunaukseni kuudentena vuotena tulemaan teille, niin että se tuottaa teille kolmen vuoden sadon.
\par 22 Ja kun te kylvätte kahdeksantena vuotena, on teillä vielä vanhaa satoa syödä; siihen asti, kunnes yhdeksäntenä vuotena sen sato on saatu, on teillä vanhaa syödä.
\par 23 Älköön maata ainaiseksi myytäkö, sillä maa on minun ja te olette muukalaisia ja vieraita minun luonani.
\par 24 Ja kaikessa siinä maassa, jonka te saatte perintömaaksenne, antakaa oikeus maan sukulunastamiseen.
\par 25 Jos veljesi köyhtyy ja myy perintömaatansa, niin hänen lähin sukulunastajansa tulkoon ja lunastakoon sen, mitä hänen veljensä on myynyt.
\par 26 Mutta jos jollakin ei ole sukulunastajaa, mutta hän itse voi hankkia niin paljon, kuin lunastukseen tarvitaan,
\par 27 niin laskekoon, kuinka monta vuotta on kulunut myymisestä, ja maksakoon jäljellä olevasta ajasta miehelle, jolle hän myi, ja palatkoon perintömaallensa.
\par 28 Mutta jos hän ei voi hankkia niin paljoa, kuin maksamiseen tarvitaan, niin jääköön se, mitä hän on myynyt, ostajan haltuun riemuvuoteen asti. Mutta riemuvuotena se on vapaa, ja hän palatkoon perintömaallensa.
\par 29 Jos joku myy asuinrakennuksen muureilla varustetussa kaupungissa, olkoon hänellä oikeus lunastaa se vuoden kuluessa siitä, kun hän sen myi; sen ajan kestää hänen sukulunastusoikeutensa.
\par 30 Mutta jollei sitä lunasteta, ennenkuin vuosi on kulunut umpeen, niin jääköön talo muureilla varustetussa kaupungissa ainaiseksi ostajalle ja hänen suvullensa, riemuvuotena vapaaksi tulematta.
\par 31 Mutta talot kylissä, joiden ympärillä ei ole muuria, luettakoon peltomaahan; ne olkoot lunastettavissa, ja riemuvuotena ne tulevat vapaiksi.
\par 32 Kuitenkin olkoon leeviläisillä niissä kaupungeissa, jotka ovat heidän perintöomaisuuttaan, ikuinen oikeus talojen lunastamiseen.
\par 33 Jos joku leeviläinen käyttää lunastusoikeutta, niin myyty talo kaupungissa, joka on leeviläistä perintöomaisuutta, palautuu riemuvuotena vapaaksi; sillä talot leeviläiskaupungeissa ovat heidän perintöomaisuuttaan israelilaisten seassa.
\par 34 Ja heidän kaupunkeihinsa kuuluvaa laidunmaata älköön myytäkö, sillä se on heidän ikuista perintöomaisuuttansa.
\par 35 Jos veljesi sinun luonasi köyhtyy eikä jaksa pysyä pystyssä, tue häntä samoinkuin muukalaista tai loista; hän eläköön luonasi.
\par 36 Älä ota korkoa tai voittoa häneltä, vaan pelkää Jumalaasi ja anna veljesi elää luonasi.
\par 37 Älä anna hänelle rahaasi korolle äläkä ota elintarpeista voittoa.
\par 38 Minä olen Herra, teidän Jumalanne, joka vein teidät pois Egyptin maasta antaakseni teille Kanaanin maan ja ollakseni teidän Jumalanne.
\par 39 Jos veljesi sinun luonasi köyhtyy ja myy itsensä sinulle, älä pane häntä orjan työhön,
\par 40 vaan hän olkoon päiväpalkkalaisena ja loisena sinun luonasi; riemuvuoteen asti hän palvelkoon sinua.
\par 41 Silloin hän lähteköön luotasi vapaana, hän ja hänen lapsensa hänen kanssaan, ja palatkoon sukunsa luo ja isiensä perintömaalle.
\par 42 Sillä he ovat minun palvelijoitani, jotka minä vein pois Egyptin maasta; sentähden älköön heitä orjan tavalla myytäkö.
\par 43 Älä hallitse heitä kovuudella, vaan pelkää Jumalaasi.
\par 44 Mutta orjasi ja orjattaresi, jotka hankit itsellesi, ostakaa kansoista, jotka asuvat teidän ympärillänne.
\par 45 Tahi ostakaa ne teidän luonanne asuvien loisten lapsista ja heidän sukulaisistaan, jotka ovat teidän luonanne ja jotka ovat heille syntyneet teidän maassanne; ja ne olkoot teidän perintöomaisuuttanne.
\par 46 Ja jättäkää ne jälkeenne perinnöksi lapsillenne, pysyväksi perintöomaisuudeksi; niitä saatte pitää ainaisesti orjinanne. Mutta veljistänne, israelilaisista, älä ketään kovuudella hallitse.
\par 47 Jos sinun luonasi asuva muukalainen tai loinen tulee varoihin, veljesi köyhtyessä hänen luonaan, niin että hän myy itsensä sinun luonasi asuvalle muukalaiselle tai loiselle tai jollekin muukalaisen suvun jäsenelle,
\par 48 niin olkoon hän, sen jälkeen kuin hän myi itsensä, lunastettavissa; joku hänen veljistään lunastakoon hänet.
\par 49 Tahi lunastakoon hänet hänen setänsä tai setänsä poika tai joku veriheimolainen hänen suvustansa; tahi jos hän tulee varoihin, lunastakoon hän itse itsensä.
\par 50 Ja hänen on laskettava yhdessä ostajansa kanssa, kuinka pitkä aika on riemuvuoteen siitä vuodesta, jolloin hän myi itsensä hänelle; ja hänen myyntihintansa jaettakoon vuosien luvulla; hänen palvelusaikansa laskettakoon saman arvoiseksi kuin päiväpalkkalaisen.
\par 51 Jos vielä on monta vuotta jäljellä, maksakoon hän takaisin lunastushintanaan niitä vastaavan osan siitä summasta, jolla hänet ostettiin.
\par 52 Mutta jos vain vähän vuosia on jäljellä riemuvuoteen, laskekoon ostaja nekin hänelle, ja hän maksakoon lunastushintansa sen mukaan, kuin hänellä on vuosia jäljellä.
\par 53 Vuosikaupalla otetun päiväpalkkalaisen tavalla hän palvelkoon häntä, mutta älköön tämä kovuudella hallitko häntä sinun silmiesi edessä.
\par 54 Mutta jos häntä ei näin lunasteta, tulkoon hän riemuvuotena vapaaksi, hän itse ja hänen lapsensa hänen kanssaan.
\par 55 Sillä israelilaiset ovat minun palvelijoitani; he ovat minun palvelijani, jotka minä vein pois Egyptin maasta. Minä olen Herra, teidän Jumalanne."

\chapter{26}

\par 1 "Älkää tehkö itsellenne epäjumalia älkääkä pystyttäkö itsellenne jumalankuvia tai patsaita, älkää myöskään asettako maahanne kiviä, joissa on kuvia, kumartaaksenne niitä, sillä minä olen Herra, teidän Jumalanne.
\par 2 Pitäkää minun sapattini ja peljätkää minun pyhäkköäni. Minä olen Herra.
\par 3 Jos te vaellatte minun säädöksieni mukaan ja noudatatte minun käskyjäni ja pidätte ne,
\par 4 annan minä teille sateen ajallansa, niin että maa antaa satonsa ja kedon puut kantavat hedelmänsä.
\par 5 Ja puiminen kestää teillä viininkorjuuseen asti, ja viininkorjuu kestää kylvöön asti, ja teillä on kyllin leipää syödäksenne, ja te saatte turvallisesti asua maassanne.
\par 6 Ja minä annan rauhan teidän maallenne, ja te saatte levätä, kenenkään peljättämättä; ja minä hävitän pahat pedot teidän maastanne, eikä miekka käy teidän maanne ylitse.
\par 7 Te ajatte pakoon vihollisenne, ja he kaatuvat miekkaan teidän edessänne.
\par 8 Viisi teistä ajaa pakoon sata, ja sata teistä ajaa pakoon kymmenentuhatta, ja teidän vihollisenne kaatuvat miekkaan teidän edessänne.
\par 9 Ja minä käännyn teidän puoleenne ja teen teidät hedelmällisiksi ja annan teidän lisääntyä ja pidän liittoni teidän kanssanne.
\par 10 Ja te saatte syödä vanhaa, viime vuoden satoa, ja te joudutte viemään vanhan pois uuden tieltä.
\par 11 Ja minä panen asumukseni teidän keskellenne enkä viero teitä.
\par 12 Ja minä vaellan teidän keskellänne ja olen teidän Jumalanne, ja te olette minun kansani.
\par 13 Minä olen Herra, teidän Jumalanne, joka vein teidät pois Egyptin maasta, olemasta heidän orjinansa; ja minä mursin rikki teidän ikeenne puut ja annoin teidän kulkea pää pystyssä.
\par 14 Mutta jos te ette kuule minua ettekä pidä kaikkia näitä käskyjä,
\par 15 vaan hylkäätte halpana minun ohjeeni ja teidän sielunne vieroo minun säädöksiäni, niin ettette pidä kaikkia minun käskyjäni, vaan rikotte minun liittoni,
\par 16 niin minäkin teen teille samoin ja rankaisen teitä hirmuisilla onnettomuuksilla, hivutustaudilla ja kuumeella, jotka sammuttavat teidän silmänne ja näännyttävät sielunne; ja te kylvätte siemenenne turhaan, sillä teidän vihollisenne syövät sen.
\par 17 Ja minä käännän kasvoni teitä vastaan, ja teidän vihollisenne voittavat teidät; teidän vihamiehenne hallitsevat teitä, ja te pakenette, vaikka ei kukaan aja teitä takaa.
\par 18 Jos ette sittenkään kuule minua, niin minä vielä kuritan teitä seitsenkertaisesti teidän syntienne tähden.
\par 19 Ja minä murran teidän ylpeän uhmanne; minä teen teidän taivaanne raudankovaksi ja maanne vasken kaltaiseksi.
\par 20 Ja teidän voimanne kuluu hukkaan, sillä teidän maanne ei anna satoansa eivätkä maan puut anna hedelmäänsä.
\par 21 Ja jos te sittenkin käytte minua vastaan ettekä tahdo kuulla minua, niin minä lyön teitä vielä seitsenkertaisesti, teidän syntienne mukaan.
\par 22 Minä lähetän teidän sekaanne metsän pedot riistämään teiltä lapsenne, raatelemaan karjaanne ja vähentämään teidän lukuanne, niin että teidän tienne tulevat autioiksi.
\par 23 Jos te ette vielä tästäkään minun kurituksestani ota ojentuaksenne, vaan yhä käytte minua vastaan,
\par 24 niin minäkin käyn teitä vastaan ja lyön teitä seitsenkertaisesti teidän syntienne tähden.
\par 25 Minä saatan teidän kimppuunne miekan, joka kostaa rikotun liiton; ja kun te kokoonnutte kaupunkeihinne, niin minä lähetän teidän sekaanne ruttotaudin, ja teidän on antautuminen vihollisen valtaan.
\par 26 Kun minä murran teiltä leivän tuen, niin kymmenen vaimoa paistaa teidän leipänne yhdessä uunissa, ja he tuovat kotiin teidän leipänne vaa'alla punnittuna; ja kun te syötte, ette tule ravituiksi.
\par 27 Ja jos te ette vielä sittenkään kuule minua, vaan yhä käytte minua vastaan,
\par 28 niin minäkin kiivastuksessani käyn teitä vastaan ja kuritan teitä seitsenkertaisesti teidän syntienne tähden.
\par 29 Ja te syötte poikienne lihaa, ja tyttärienne lihaa te syötte.
\par 30 Ja minä kukistan teidän uhrikukkulanne ja hävitän teidän auringonpatsaanne ja panen teidän ruumiinne teidän kivijumalienne ruumiiden päälle, ja minun sieluni inhoaa teitä.
\par 31 Ja minä muutan teidän kaupunkinne raunioiksi ja hävitän teidän pyhäkkönne enkä mielisty teidän uhrienne tuoksuun.
\par 32 Ja minä hävitän maan, niin että teidän vihollisenne, jotka siinä asuvat, siitä tyrmistyvät.
\par 33 Mutta teidät minä hajotan kansojen sekaan ja ajan teitä takaa paljastetulla miekalla, ja teidän maanne tulee autioksi ja kaupunkinne raunioiksi.
\par 34 Silloin maa saa hyvityksen sapateistaan, niin kauan kuin se on autiona ja te olette vihollistenne maassa. Silloin maa lepää ja saa hyvityksen sapateistaan.
\par 35 Niin kauan kuin se on autiona, saa se levätä nauttien sitä lepoa, jota se ei saanut teidän sapatteinanne, asuessanne siinä.
\par 36 Ja niille teistä, jotka jäävät jäljelle, minä annan pelokkaan sydämen heidän vihollistensa maassa, niin että lentävän lehden kahina ajaa heidät pakoon, ja he pakenevat, niinkuin pakenisivat miekkaa, ja kaatuvat, vaikka ei kukaan heitä aja takaa.
\par 37 Ja he kompastuvat toinen toiseensa niinkuin miekkaa paeten, vaikka ei kukaan aja heitä takaa; ja te ette voi pitää puolianne vihollisianne vastaan.
\par 38 Ja te häviätte kansojen sekaan, ja vihollistenne maa nielee teidät.
\par 39 Ja ne teistä, jotka jäävät jäljelle, riutuvat teidän vihollistenne maassa syntivelkansa tähden; ja myös isiensä syntivelan tähden he riutuvat niinkuin nekin.
\par 40 Silloin he tunnustavat syntivelkansa ja isiensä syntivelan, sen että ovat olleet minulle uskottomat ja käyneet minua vastaan,
\par 41 jonka tähden minäkin kävin heitä vastaan ja vein heidät heidän vihollistensa maahan. Silloin heidän ympärileikkaamaton sydämensä nöyrtyy, ja silloin he sovittavat syntivelkansa.
\par 42 Ja silloin minä muistan liittoni Jaakobin kanssa ja muistan liittoni Iisakin kanssa ja liittoni Aabrahamin kanssa, ja minä muistan myös maan.
\par 43 Mutta ensin tulkoon maa tyhjäksi heistä ja saakoon hyvityksen sapateistansa olemalla autiona, niin kauan kuin he ovat poissa; ja he sovittakoot syntivelkansa sentähden, juuri sentähden, että he hylkäsivät minun säädökseni ja että heidän sielunsa vieroi minun käskyjäni.
\par 44 Mutta sittenkään, vaikka he ovat vihollistensa maassa, en minä heitä hylkää enkä viero heitä niin, että lopettaisin heidät ja rikkoisin liittoni heidän kanssansa; sillä minä olen Herra, heidän Jumalansa.
\par 45 Ja minä muistan heidän hyväksensä liiton heidän isiensä kanssa, jotka minä vein pois Egyptin maasta kansojen silmien edessä, ollakseni heidän Jumalansa. Minä olen Herra."
\par 46 Nämä ovat ne käskyt, säädökset ja lait, jotka Herra asetti itsensä ja israelilaisten välille Siinain vuorella, Mooseksen kautta.

\chapter{27}

\par 1 Ja Herra puhui Moosekselle sanoen:
\par 2 "Puhu israelilaisille ja sano heille: Jos jollakin on täytettävänä lupaus Herralle ja sinun on arvioitava luvatun ihmisen arvo,
\par 3 niin arvioitse mies, joka on kahdenkymmenen ja kuudenkymmenen vuoden välillä, viiteenkymmeneen hopeasekeliin pyhäkkösekelin painon mukaan;
\par 4 mutta jos se on nainen, arvioitse hänet kolmeenkymmeneen sekeliin.
\par 5 Viisivuotisesta kaksikymmenvuotiseen asti arvioitse miehenpuoli kahteenkymmeneen sekeliin ja vaimonpuoli kymmeneen sekeliin.
\par 6 Kuukauden vanhasta viisivuotiseen asti arvioitse miehenpuoli viiteen hopeasekeliin ja vaimonpuoli kolmeen hopeasekeliin.
\par 7 Kuusikymmenvuotinen ja sitä vanhempi miehenpuoli arvioitse viiteentoista sekeliin ja vaimonpuoli kymmeneen sekeliin.
\par 8 Jos joku on liian köyhä maksamaan arviohintaa, vietäköön hänet papin eteen, ja pappi arvioikoon hänet; lupauksen tekijän varojen mukaan arvioikoon pappi hänet.
\par 9 Mutta jos lupaus koskee karjaa, josta voidaan tuoda uhrilahja Herralle, olkoon kaikki, mikä siitä annetaan Herralle, pyhää.
\par 10 Sitä älköön vaihdettako älköönkä muutettako; ei hyvää huonoon eikä huonoa hyvään. Jos joku kuitenkin vaihtaa eläimen toiseen, olkoon sekä se että siihen vaihdettu pyhä.
\par 11 Mutta jos se on saastainen eläin, mikä tahansa, joita ei saa tuoda uhrilahjaksi Herralle, niin vietäköön eläin papin eteen;
\par 12 ja pappi arvioikoon sen tarkastaen, onko se hyvä vai huono. Olkoon sillä se arviohinta, jonka pappi määrää.
\par 13 Jos omistaja tahtoo sen lunastaa, antakoon arviohinnan lisäksi viidennen osan siitä.
\par 14 Ja jos joku pyhittää talonsa pyhäksi lahjaksi Herralle, niin pappi arvioikoon sen tarkastaen, onko se hyvä vai huono. Olkoon sillä se arviohinta, jonka pappi määrää.
\par 15 Mutta jos se, joka on talonsa pyhittänyt, tahtoo lunastaa sen, niin antakoon arviosumman lisäksi viidennen osan siitä; ja niin se on hänen omansa.
\par 16 Jos joku pyhittää Herralle kappaleen perintömaatansa, niin arvioi se kylvön suuruuden mukaan; hoomer-mitta ohria vastatkoon viittäkymmentä hopeasekeliä.
\par 17 Jos hän pyhittää peltonsa riemuvuodesta alkaen, jääköön sille arvioimasi täysi hinta.
\par 18 Mutta jos hän pyhittää peltonsa riemuvuoden jälkeen, laskekoon pappi hänelle rahasumman niiden vuosien luvun mukaan, jotka ovat jäljellä seuraavaan riemuvuoteen; ja se vähennettäköön sinun arvioimastasi hinnasta.
\par 19 Ja jos se, joka pellon on pyhittänyt, tahtoo lunastaa sen, niin antakoon arvioimasi hinnan lisäksi viidennen osan siitä; ja niin se jää hänen omakseen.
\par 20 Mutta jos hän ei lunasta peltoa, vaan myy sen toiselle, ei se enää ole lunastettavissa;
\par 21 vaan kun pelto riemuvuotena tulee vapaaksi, olkoon se Herralle pyhitetty, niinkuin vihitty pelto; se joutuu papin omaisuudeksi.
\par 22 Jos joku Herralle pyhittää ostamansa pellon, joka ei ole hänen perintömaatansa,
\par 23 laskekoon pappi hänelle arviohinnan riemuvuoteen asti; ja hän suorittakoon samana päivänä arvioimasi hinnan pyhänä lahjana Herralle.
\par 24 Riemuvuotena pelto palautukoon sille, jolta se on ostettu ja jonka perintömaata se on.
\par 25 Ja kaikki sinun arvioimisesi tapahtukoon pyhäkkösekelin painon mukaan; sekelissä olkoon kaksikymmentä geeraa.
\par 26 Mutta karjan esikoista, joka ensiksi syntyneenä jo on Herran oma, älköön kukaan pyhittäkö; olipa se härkä tai lammas, se on jo Herran oma.
\par 27 Jos se on saastaisia eläimiä, lunastakoon omistaja sen sinun arvioimastasi hinnasta ja pankoon lisäksi viidennen osan sen arvosta; jos sitä ei lunasteta, myytäköön se arvioimaasi hintaan.
\par 28 Mitään vihittyä, mitä hyvänsä, jonka joku on vihkinyt Herralle omaisuudestansa, olipa se ihminen tai eläin tai perintömaa, älköön myytäkö älköönkä lunastettako; sillä kaikki vihitty on korkeasti-pyhää, Herran omaa.
\par 29 Ihmistä, ketä hyvänsä, joka on tuhon omaksi vihitty, älköön lunastettako, vaan hänet surmattakoon.
\par 30 Kaikki maan kymmenykset, sekä pellon viljasta että puiden hedelmistä, ovat Herran omat, pyhä lahja Herralle.
\par 31 Jos joku tahtoo lunastaa jotakin kymmenyksistänsä, antakoon hän siihen lisäksi viidennen osan.
\par 32 Ja kaikki kymmenykset raavaskarjasta ja pikkukarjasta, joka kymmenes eläin kaikesta, mikä kulkee paimenen sauvan alitse, olkoot pyhä lahja Herralle.
\par 33 Älköön kysyttäkö, onko se hyvä vai huono, älköönkä sitä vaihdettako. Jos joku kuitenkin vaihtaa sen, niin olkoon sekä se että siihen vaihdettu pyhä; älköön niitä lunastettako."
\par 34 Nämä ovat ne käskyt, jotka Herra Siinain vuorella antoi Moosekselle israelilaisia varten.


\end{document}