\begin{document}

\title{Ensimmäinen kirje korinttilaisille}


\chapter{1}

\par 1 Paavali, Kristuksen Jeesuksen kutsuttu apostoli Jumalan tahdosta, ja veli Soostenes
\par 2 Korintossa olevalle Jumalan seurakunnalle, Kristuksessa Jeesuksessa pyhitetyille, jotka ovat kutsutut ja pyhät, ynnä kaikille, jotka avuksi huutavat meidän Herramme Jeesuksen Kristuksen nimeä kaikissa paikkakunnissa, niin omissaan kuin meidänkin.
\par 3 Armo teille ja rauha Jumalalta, meidän Isältämme, ja Herralta Jeesukselta Kristukselta!
\par 4 Minä kiitän Jumalaani aina teidän tähtenne siitä Jumalan armosta, joka on annettu teille Kristuksessa Jeesuksessa,
\par 5 että kaikessa olette rikastuneet hänessä, kaikessa puheessa ja kaikessa tiedossa,
\par 6 sen mukaan kuin todistus Kristuksesta on teissä vahvistettu,
\par 7 niin ettei teiltä mitään puutu missään armolahjassa, teidän odottaessanne meidän Herramme Jeesuksen Kristuksen ilmestystä.
\par 8 Hän on myös vahvistava teitä loppuun asti, niin että te olette nuhteettomat meidän Herramme Jeesuksen Kristuksen päivänä.
\par 9 Jumala on uskollinen, hän, jonka kautta te olette kutsutut hänen Poikansa Jeesuksen Kristuksen, meidän Herramme, yhteyteen.
\par 10 Mutta minä kehoitan teitä, veljet, meidän Herramme Jeesuksen Kristuksen nimeen, että kaikki olisitte puheessa yksimieliset ettekä suvaitsisi riitaisuuksia keskuudessanne, vaan pysyisitte sovinnossa ja teillä olisi sama mieli ja sama ajatus.
\par 11 Sillä Kloen perheväeltä olen saanut teistä kuulla, veljeni, että teillä on riitoja keskuudessanne.
\par 12 Tarkoitan sitä, että yksi teistä sanoo: "Minä olen Paavalin puolta", toinen: "Minä Apolloksen", joku taas: "Minä Keefaan", joku vielä: "Minä Kristuksen".
\par 13 Onko Kristus jaettu? Ei kaiketi Paavali ole ristiinnaulittu teidän edestänne? Vai oletteko te kastetut Paavalin nimeen?
\par 14 Minä kiitän Jumalaa, etten ole kastanut teistä ketään muita kuin Krispuksen ja Gaiuksen,
\par 15 niin ettei kukaan saata sanoa, että te olette minun nimeeni kastetut.
\par 16 Kastoinhan tosin Stefanaankin perhekunnan; sitten en tiedä, olenko ketään muuta kastanut.
\par 17 Sillä Kristus ei lähettänyt minua kastamaan, vaan evankeliumia julistamaan - ei puheen viisaudella, ettei Kristuksen risti menisi mitättömäksi.
\par 18 Sillä sana rististä on hullutus niille, jotka kadotukseen joutuvat, mutta meille, jotka pelastumme, se on Jumalan voima.
\par 19 Onhan kirjoitettu: "Minä hävitän viisasten viisauden, ja ymmärtäväisten ymmärryksen minä teen mitättömäksi".
\par 20 Missä ovat viisaat? Missä kirjanoppineet? Missä tämän maailman älyniekat? Eikö Jumala ole tehnyt maailman viisautta hullutukseksi?
\par 21 Sillä kun, Jumalan viisaudesta, maailma ei oppinut viisauden avulla tuntemaan Jumalaa, niin Jumala näki hyväksi saarnauttamansa hullutuksen kautta pelastaa ne, jotka uskovat,
\par 22 koskapa juutalaiset vaativat tunnustekoja ja kreikkalaiset etsivät viisautta,
\par 23 me taas saarnaamme ristiinnaulittua Kristusta, joka on juutalaisille pahennus ja pakanoille hullutus,
\par 24 mutta joka niille, jotka ovat kutsutut, olkootpa juutalaisia tai kreikkalaisia, on Kristus, Jumalan voima ja Jumalan viisaus.
\par 25 Sillä Jumalan hulluus on viisaampi kuin ihmiset, ja Jumalan heikkous on väkevämpi kuin ihmiset.
\par 26 Sillä katsokaa, veljet, omaa kutsumistanne: ei ole monta inhimillisesti viisasta, ei monta mahtavaa, ei monta jalosukuista,
\par 27 vaan sen, mikä on hulluutta maailmalle, sen Jumala valitsi saattaaksensa viisaat häpeään, ja sen, mikä on heikkoa maailmassa, sen Jumala valitsi saattaaksensa sen, mikä väkevää on, häpeään,
\par 28 ja sen, mikä maailmassa on halpasukuista ja halveksittua, sen Jumala valitsi, sen, joka ei mitään ole, tehdäksensä mitättömäksi sen, joka jotakin on,
\par 29 ettei mikään liha voisi kerskata Jumalan edessä.
\par 30 Mutta hänestä on teidän olemisenne Kristuksessa Jeesuksessa, joka on tullut meille viisaudeksi Jumalalta ja vanhurskaudeksi ja pyhitykseksi ja lunastukseksi,
\par 31 että kävisi, niinkuin kirjoitettu on: "Joka kerskaa, sen kerskauksena olkoon Herra".

\chapter{2}

\par 1 Niinpä, kun minä tulin teidän tykönne, veljet, en tullut puheen tai viisauden loistolla teille Jumalan todistusta julistamaan.
\par 2 Sillä minä olin päättänyt olla teidän tykönänne tuntematta mitään muuta paitsi Jeesuksen Kristuksen, ja hänet ristiinnaulittuna.
\par 3 Ja ollessani teidän tykönänne minä olin heikkouden vallassa ja pelossa ja suuressa vavistuksessa,
\par 4 ja minun puheeni ja saarnani ei ollut kiehtovia viisauden sanoja, vaan Hengen ja voiman osoittamista,
\par 5 ettei teidän uskonne perustuisi ihmisten viisauteen, vaan Jumalan voimaan.
\par 6 Kuitenkin me puhumme viisautta täydellisten seurassa, mutta emme tämän maailman viisautta emmekä tämän maailman valtiasten, jotka kukistuvat,
\par 7 vaan me puhumme salattua Jumalan viisautta, sitä kätkettyä, jonka Jumala on edeltämäärännyt ennen maailmanaikoja meidän kirkkaudeksemme,
\par 8 sitä, jota ei kukaan tämän maailman valtiaista ole tuntenut - sillä jos he olisivat sen tunteneet, eivät he olisi kirkkauden Herraa ristiinnaulinneet -
\par 9 vaan, niinkuin kirjoitettu on: "mitä silmä ei ole nähnyt eikä korva kuullut, mikä ei ole ihmisen sydämeen noussut ja minkä Jumala on valmistanut niille, jotka häntä rakastavat".
\par 10 Mutta meille Jumala on sen ilmoittanut Henkensä kautta, sillä Henki tutkii kaikki, Jumalan syvyydetkin.
\par 11 Sillä kuka ihminen tietää, mitä ihmisessä on, paitsi ihmisen henki, joka hänessä on? Samoin ei myös kukaan tiedä, mitä Jumalassa on, paitsi Jumalan Henki.
\par 12 Mutta me emme ole saaneet maailman henkeä, vaan sen Hengen, joka on Jumalasta, että tietäisimme, mitä Jumala on meille lahjoittanut;
\par 13 ja siitä me myös puhumme, emme inhimillisen viisauden opettamilla sanoilla, vaan Hengen opettamilla, selittäen hengelliset hengellisesti.
\par 14 Mutta luonnollinen ihminen ei ota vastaan sitä, mikä Jumalan Hengen on; sillä se on hänelle hullutus, eikä hän voi sitä ymmärtää, koska se on tutkisteltava hengellisesti.
\par 15 Hengellinen ihminen sitä vastoin tutkistelee kaiken, mutta häntä itseään ei kukaan kykene tutkistelemaan.
\par 16 Sillä: "kuka on tullut tuntemaan Herran mielen, niin että voisi neuvoa häntä?" Mutta meillä on Kristuksen mieli.

\chapter{3}

\par 1 Niinpä, veljet, minun ei käynyt puhuminen teille niinkuin hengellisille, vaan niinkuin lihallisille, niinkuin pienille lapsille Kristuksessa.
\par 2 Maitoa minä juotin teille, en antanut ruokaa, sillä sitä ette silloin sietäneet, ettekä vielä nytkään siedä;
\par 3 olettehan vielä lihallisia. Sillä kun keskuudessanne on kateutta ja riitaa, ettekö silloin ole lihallisia ja vaella ihmisten tavoin?
\par 4 Kun toinen sanoo: "Minä olen Paavalin puolta", ja toinen: "Minä olen Apolloksen", ettekö silloin ole niinkuin ihmiset ainakin?
\par 5 Mikä Apollos sitten on? Ja mikä Paavali on? Palvelijoita, joiden kautta te olette tulleet uskoviksi, palvelijoita sen kykynsä mukaan, minkä Herra on heille kullekin antanut.
\par 6 Minä istutin, Apollos kasteli, mutta Jumala on antanut kasvun.
\par 7 Niin ei siis istuttaja ole mitään, eikä kastelijakaan, vaan Jumala, joka kasvun antaa.
\par 8 Mutta istuttaja ja kastelija ovat yhtä; kuitenkin on kumpikin saava oman palkkansa oman työnsä mukaan.
\par 9 Sillä me olemme Jumalan työtovereita; te olette Jumalan viljelysmaa, olette Jumalan rakennus.
\par 10 Sen Jumalan armon mukaan, joka on minulle annettu, minä olen taitavan rakentajan tavoin pannut perustuksen, ja toinen sille rakentaa, mutta katsokoon kukin, kuinka hän sille rakentaa.
\par 11 Sillä muuta perustusta ei kukaan voi panna, kuin mikä pantu on, ja se on Jeesus Kristus.
\par 12 Mutta jos joku rakentaa tälle perustukselle, rakensipa kullasta, hopeasta, jalokivistä, puusta, heinistä tai oljista,
\par 13 niin kunkin teko on tuleva näkyviin; sillä sen on saattava ilmi se päivä, joka tulessa ilmestyy, ja tuli on koetteleva, minkälainen kunkin teko on.
\par 14 Jos jonkun tekemä rakennus kestää, on hän saava palkan;
\par 15 mutta jos jonkun tekemä palaa, joutuu hän vahinkoon; mutta hän itse on pelastuva, kuitenkin ikäänkuin tulen läpi.
\par 16 Ettekö tiedä, että te olette Jumalan temppeli ja että Jumalan Henki asuu teissä?
\par 17 Jos joku turmelee Jumalan temppelin, on Jumala turmeleva hänet; sillä Jumalan temppeli on pyhä, ja sellaisia te olette.
\par 18 Älköön kukaan pettäkö itseään. Jos joku teidän joukossanne luulee olevansa viisas tässä maailmassa, tulkoon hän tyhmäksi, että hänestä tulisi viisas.
\par 19 Sillä tämän maailman viisaus on hullutus Jumalan silmissä. Sillä kirjoitettu on: "Hän vangitsee viisaat heidän viekkauteensa";
\par 20 ja vielä: "Herra tuntee viisasten ajatukset, hän tietää ne turhiksi".
\par 21 Älköön siis kukaan kerskatko ihmisistä; sillä kaikki on teidän,
\par 22 teidän on Paavali ja Apollos ja Keefas, teidän on maailma ja elämä ja kuolema, nykyiset ja tulevaiset, kaikki on teidän.
\par 23 Mutta te olette Kristuksen, ja Kristus on Jumalan.

\chapter{4}

\par 1 Niin pitäköön jokainen meitä Kristuksen käskyläisinä ja Jumalan salaisuuksien huoneenhaltijoina.
\par 2 Sitä tässä huoneenhaltijoilta ennen muuta vaaditaan, että heidät havaitaan uskollisiksi.
\par 3 Mutta siitä minä hyvin vähän välitän, että te minua tuomitsette tai joku inhimillinen oikeus; en minä itsekään tuomitse itseäni,
\par 4 sillä ei minulla ole mitään tunnollani, mutta en minä silti ole vanhurskautettu, vaan minun tuomitsijani on Herra.
\par 5 Älkää sentähden lausuko mitään tuomiota, ennenkuin aika on, ennenkuin Herra tulee, joka myös on saattava valoon pimeyden kätköt ja tuova ilmi sydänten aivoitukset; ja silloin kukin saa kiitoksensa Jumalalta.
\par 6 Tämän olen, veljet, sovittanut itseeni ja Apollokseen, teidän tähtenne, että meistä oppisitte tämän: "Ei yli sen, mikä kirjoitettu on", ettette pöyhkeillen asettuisi mikä minkin puolelle toista vastaan.
\par 7 Sillä kuka antaa sinulle etusijan? Ja mitä sinulla on, jota et ole lahjaksi saanut? Mutta jos olet sen saanut, niin miksi kerskaat, ikäänkuin se ei olisi saatua?
\par 8 Te olette jo ravitut, teistä on tullut jo rikkaita, ilman meitä teistä on tullut kuninkaita! Kunpa teistä olisikin tullut kuninkaita, niin että mekin pääsisimme kuninkaiksi teidän kanssanne!
\par 9 Sillä minusta näyttää, että Jumala on asettanut meidät apostolit vihoviimeisiksi, ikäänkuin kuolemaan tuomituiksi; meistä on tullut kaiken maailman katseltava, sekä enkelien että ihmisten,
\par 10 me olemme houkkia Kristuksen tähden, mutta te älykkäitä Kristuksessa, me olemme heikkoja, mutta te väkeviä; te kunnioitettuja, mutta me halveksittuja.
\par 11 Vielä tänäkin hetkenä me kärsimme sekä nälkää että janoa, olemme alasti, meitä piestään, ja me kuljemme kodittomina,
\par 12 me näemme vaivaa tehden työtä omin käsin. Meitä herjataan, mutta me siunaamme; meitä vainotaan, mutta me kestämme;
\par 13 meitä parjataan, mutta me puhumme leppeästi; meistä on tullut kuin mikäkin maailman tunkio, kaikkien hylkimiä, aina tähän päivään asti.
\par 14 En kirjoita tätä häväistäkseni teitä, vaan niinkuin rakkaita lapsiani neuvoen.
\par 15 Sillä vaikka teillä olisi kymmenentuhatta kasvattajaa Kristuksessa, niin ei teillä kuitenkaan ole monta isää; sillä minä teidät synnytin evankeliumin kautta Kristuksessa Jeesuksessa.
\par 16 Kehoitan siis teitä: olkaa minun seuraajiani.
\par 17 Juuri sentähden minä lähetin teille Timoteuksen, joka on minun rakas ja uskollinen poikani Herrassa; hän on muistuttava teitä minun vaelluksestani Kristuksessa Jeesuksessa, sen mukaan kuin minä kaikkialla, joka seurakunnassa, opetan.
\par 18 Muutamat teistä ovat paisuneet pöyhkeiksi, aivan niinkuin minä en tulisikaan teidän tykönne.
\par 19 Mutta minä tulen pian teidän tykönne, jos Herra tahtoo, ja silloin minä otan selon, en noiden pöyhkeiden sanoista, vaan voimasta.
\par 20 Sillä Jumalan valtakunta ei ole sanoissa, vaan voimassa.
\par 21 Kummanko tahdotte? Tulenko luoksenne vitsa kädessä vaiko rakkaudessa ja sävyisyyden hengessä?

\chapter{5}

\par 1 Yleensä kuuluu, että teidän keskuudessanne harjoitetaan haureutta, jopa semmoista haureutta, jota ei ole pakanainkaan keskuudessa, että eräskin pitää isänsä vaimoa.
\par 2 Ja te olette paisuneet pöyhkeiksi! Eikö teidän pikemminkin olisi pitänyt tulla murheellisiksi, että se, joka on tuommoisen teon tehnyt, poistettaisiin teidän keskuudestanne?
\par 3 Sillä minä, joka tosin ruumiillisesti olen poissa, mutta hengessä kuitenkin läsnä, olen jo, niinkuin läsnäollen, puolestani päättänyt, että se, joka tuommoisen teon on tehnyt, on
\par 4 - sittenkuin olemme, te ja minun henkeni ynnä meidän Herramme Jeesuksen voima, tulleet yhteen -
\par 5 Herran Jeesuksen nimessä hyljättävä saatanan haltuun lihan turmioksi, että hänen henkensä pelastuisi Herran päivänä.
\par 6 Ei ole hyvä, että kerskaatte. Ettekö tiedä, että vähäinen hapatus hapattaa koko taikinan?
\par 7 Peratkaa pois vanha hapatus, että teistä tulisi uusi taikina, niinkuin te olettekin happamattomat; sillä onhan meidän pääsiäislampaamme, Kristus, teurastettu.
\par 8 Viettäkäämme siis juhlaa, ei vanhassa hapatuksessa eikä ilkeyden ja pahuuden hapatuksessa, vaan puhtauden ja totuuden happamattomuudessa.
\par 9 Minä kirjoitin teille kirjeessäni, ettette seurustelisi huorintekijäin kanssa;
\par 10 en tarkoittanut yleensä tämän maailman huorintekijöitä tai ahneita tai anastajia tai epäjumalanpalvelijoita, sillä silloinhan teidän täytyisi lähteä pois maailmasta.
\par 11 Vaan minä kirjoitin teille, että jos joku, jota kutsutaan veljeksi, on huorintekijä tai ahne tai epäjumalanpalvelija tai pilkkaaja tai juomari tai anastaja, te ette seurustelisi ettekä söisikään semmoisen kanssa.
\par 12 Sillä onko minun asiani tuomita niitä, jotka ovat ulkopuolella? Ettekö tekin tuomitse vain niitä, jotka ovat sisäpuolella?
\par 13 Mutta ulkopuolella olevat tuomitsee Jumala. "Poistakaa keskuudestanne se, joka on paha."

\chapter{6}

\par 1 Kuinka rohkenee kukaan teistä, jolla on riita-asia toisen kanssa, käydä oikeutta vääräin edessä? Miksei pyhien edessä?
\par 2 Vai ettekö tiedä, että pyhät tulevat maailman tuomitsemaan? Ja jos te tuomitsette maailman, niin ettekö kelpaa ratkaisemaan aivan vähäpätöisiä asioita?
\par 3 Ettekö tiedä, että me tulemme tuomitsemaan enkeleitä, emmekö sitten maallisia asioita?
\par 4 Jos teillä siis on maallisia asioita ratkaistavina, nekö te asetatte tuomareiksi, jotka ovat halveksittuja seurakunnassa?
\par 5 Teidän häpeäksenne minä tämän sanon. Eikö teidän joukossanne sitten ole yhtäkään viisasta, joka voisi ratkaista veljien välin?
\par 6 Vaan veli käy oikeutta veljen kanssa, vieläpä uskottomain edessä!
\par 7 Teille on jo yleensä vaurioksi, että käräjöitte keskenänne. Miksi ette ennemmin salli tehdä vääryyttä itsellenne? Miksi ette ennemmin anna riistää omaanne?
\par 8 Sen sijaan te itse teette vääryyttä ja riistätte toisen omaa, vieläpä veljien.
\par 9 Vai ettekö tiedä, etteivät väärät saa periä Jumalan valtakuntaa? Älkää eksykö. Eivät huorintekijät, ei epäjumalanpalvelijat, ei avionrikkojat, ei hekumoitsijat eikä miehimykset,
\par 10 eivät varkaat, ei ahneet, ei juomarit, ei pilkkaajat eivätkä anastajat saa periä Jumalan valtakuntaa.
\par 11 Ja tuommoisia te olitte, jotkut teistä; mutta te olette vastaanottaneet peson, te olette pyhitetyt, te olette vanhurskautetut meidän Herramme Jeesuksen Kristuksen nimessä ja meidän Jumalamme Hengessä.
\par 12 Kaikki on minulle luvallista, mutta ei kaikki ole hyödyksi; kaikki on minulle luvallista, mutta minä en saa antaa minkään itseäni vallita.
\par 13 Ruoka on vatsaa varten ja vatsa ruokaa varten; ja Jumala on tekevä lopun niin toisesta kuin toisestakin. Mutta ruumis ei ole haureutta varten, vaan Herraa varten, ja Herra ruumista varten;
\par 14 ja Jumala, joka herätti kuolleista Herran, on herättävä meidätkin voimallansa.
\par 15 Ettekö tiedä, että teidän ruumiinne ovat Kristuksen jäseniä? Ottaisinko siis Kristuksen jäsenet ja tekisin ne porton jäseniksi? Pois se!
\par 16 Vai ettekö tiedä, että joka yhtyy porttoon, tulee yhdeksi ruumiiksi hänen kanssaan? Onhan sanottu: "Ne kaksi tulevat yhdeksi lihaksi".
\par 17 Mutta joka yhtyy Herraan, on yksi henki hänen kanssaan.
\par 18 Paetkaa haureutta. Kaikki muu synti, mitä ikinä ihminen tekee, on ruumiin ulkopuolella; mutta haureuden harjoittaja tekee syntiä omaa ruumistansa vastaan.
\par 19 Vai ettekö tiedä, että teidän ruumiinne on Pyhän Hengen temppeli, joka Henki teissä on ja jonka te olette saaneet Jumalalta, ja ettette ole itsenne omat?
\par 20 Sillä te olette kalliisti ostetut. Kirkastakaa siis Jumala ruumiissanne.

\chapter{7}

\par 1 Mutta mitä siihen tulee, mistä kirjoititte, niin hyvä on miehelle olla naiseen ryhtymättä;
\par 2 mutta haureuden syntien välttämiseksi olkoon kullakin miehellä oma vaimonsa, ja kullakin naisella aviomiehensä.
\par 3 Täyttäköön mies velvollisuutensa vaimoansa kohtaan, ja samoin vaimo miestänsä kohtaan.
\par 4 Vaimon ruumis ei ole hänen omassa, vaan hänen miehensä vallassa; samoin ei miehenkään ruumis ole hänen omassa, vaan vaimon vallassa.
\par 5 Älkää vetäytykö pois toisistanne, paitsi ehkä keskinäisestä sopimuksesta joksikin ajaksi, niin että olisitte vapaat rukoukseen ja sitten taas tulisitte yhteen, ettei saatana teitä kiusaisi teidän hillittömyytenne tähden.
\par 6 Mutta tämän minä sanon myönnytyksenä, en käskynä.
\par 7 Soisin kaikkien ihmisten olevan niinkuin minäkin; mutta kullakin on oma lahjansa Jumalalta, yhdellä yksi, toisella toinen.
\par 8 Naimattomille ja leskille minä taas sanon: heille on hyvä, jos pysyvät sellaisina kuin minäkin;
\par 9 mutta jos eivät voi itseään hillitä, niin menkööt naimisiin; sillä parempi on naida kuin palaa.
\par 10 Mutta naimisissa oleville minä julistan, en kuitenkaan minä, vaan Herra, ettei vaimo saa erota miehestään;
\par 11 mutta jos hän eroaa, niin pysyköön naimatonna tai sopikoon miehensä kanssa; eikä mies saa hyljätä vaimoansa.
\par 12 Mutta muille sanon minä, eikä Herra: jos jollakin veljellä on vaimo, joka ei usko, ja tämä suostuu asumaan hänen kanssaan, niin älköön mies häntä hyljätkö;
\par 13 samoin älköön vaimokaan, jos hänellä on mies, joka ei usko, ja tämä suostuu asumaan hänen kanssaan, hyljätkö miestänsä.
\par 14 Sillä mies, joka ei usko, on pyhitetty vaimonsa kautta, ja vaimo, joka ei usko, on pyhitetty miehensä, uskonveljen, kautta; muutoinhan teidän lapsenne olisivat saastaisia, mutta nyt he ovat pyhiä.
\par 15 Mutta jos se, joka ei usko, eroaa, niin erotkoon; veli ja sisar eivät ole semmoisissa tapauksissa orjuutetut; sillä rauhaan on Jumala teidät kutsunut.
\par 16 Sillä mistä tiedät, vaimo, voitko pelastaa miehesi? Tai mistä tiedät, mies, voitko pelastaa vaimosi?
\par 17 Vaeltakoon vain kukin sen mukaan, kuin Herra on hänelle hänen osansa antanut, ja siinä asemassa, missä hänet Jumala on kutsunut; näin minä säädän kaikissa seurakunnissa.
\par 18 Jos joku on kutsuttu ympärileikattuna, älköön hän pyrkikö ympärileikkaamattomaksi; jos joku on kutsuttu ympärileikkaamatonna, älköön ympärileikkauttako itseään.
\par 19 Ei ympärileikkaus ole mitään, eikä ympärileikkaamattomuus ole mitään, vaan Jumalan käskyjen pitäminen.
\par 20 Pysyköön kukin siinä asemassa, missä hänet on kutsuttu.
\par 21 Jos olet kutsuttu orjana, älä siitä murehdi; mutta vaikka voisitkin päästä vapaaksi, niin ole ennemmin siinä osassasi.
\par 22 Sillä joka orjana on kutsuttu Herrassa, on Herran vapaa; samoin vapaana kutsuttu on Kristuksen orja.
\par 23 Te olette kalliisti ostetut; älkää olko ihmisten orjia.
\par 24 Pysyköön kukin, veljet, Jumalan edessä siinä asemassa, missä hänet on kutsuttu.
\par 25 Mutta neitsyistä minulla ei ole Herran käskyä, vaan minä sanon ajatukseni niinkuin se, joka on Herralta saanut sen laupeuden, että hän on luotettava.
\par 26 Olen siis sitä mieltä, että lähestyvän ahdingon tähden jokaisen on hyvä pysyä entisellään.
\par 27 Jos olet sidottu vaimoon, älä pyydä eroa; jos et ole sidottu vaimoon, älä pyydä itsellesi vaimoa.
\par 28 Mutta jos menetkin naimisiin, et syntiä tee; ja jos neitsyt menee naimisiin, ei hänkään tee syntiä; mutta ne, jotka niin tekevät, joutuvat kärsimään ruumiillista vaivaa, ja siitä minä tahtoisin teidät säästää.
\par 29 Mutta sen minä sanon, veljet: aika on lyhyt; olkoot tästedes nekin, joilla on vaimot, niinkuin ei heillä niitä olisikaan,
\par 30 ja ne, jotka itkevät, niinkuin eivät itkisi, ja ne, jotka iloitsevat, niinkuin eivät iloitsisi, ja ne, jotka ostavat, niinkuin eivät saisi omanansa pitää,
\par 31 ja ne, jotka tätä maailmaa hyödyksensä käyttävät, niinkuin eivät sitä käyttäisi; sillä tämän maailman muoto on katoamassa.
\par 32 Soisin, ettei teillä olisi huolia. Naimaton mies huolehtii siitä, mikä on Herran, kuinka olisi Herralle mieliksi;
\par 33 mutta nainut huolehtii maailmallisista, kuinka olisi vaimolleen mieliksi,
\par 34 ja hänen harrastuksensa käy kahtaanne. Samoin vaimo, jolla ei enää ole miestä, ja neitsyt huolehtivat siitä, mikä on Herran, että olisivat pyhät sekä ruumiin että hengen puolesta; mutta naimisissa oleva huolehtii maailmallisista, kuinka olisi mieliksi miehellensä.
\par 35 Tämän minä sanon teidän omaksi hyödyksenne, en pannakseni kytkyttä kaulaanne, vaan sitä varten, että eläisitte säädyllisesti ja häiriytymättä pysyisitte Herrassa.
\par 36 Mutta jos joku arvelee tekevänsä väärin tytärtänsä kohtaan, joka on täydessä naima-iässä, ja jos kerran sen pitää tapahtua, niin tehköön, niinkuin tahtoo; ei hän syntiä tee: menkööt naimisiin.
\par 37 Joka taas pysyy sydämessään lujana eikä ole minkään pakon alainen, vaan voi noudattaa omaa tahtoansa ja on sydämessään päättänyt pitää tyttärensä naimattomana, se tekee hyvin.
\par 38 Siis, joka naittaa tyttärensä, tekee hyvin, ja joka ei naita, tekee paremmin.
\par 39 Vaimo on sidottu, niin kauan kuin hänen miehensä elää, mutta jos mies kuolee, on hän vapaa menemään naimisiin, kenen kanssa tahtoo, kunhan se vain tapahtuu Herrassa.
\par 40 Mutta hän on onnellisempi, jos pysyy entisellään; se on minun mielipiteeni, ja minä luulen, että minullakin on Jumalan Henki.

\chapter{8}

\par 1 Mitä sitten epäjumalille uhrattuun lihaan tulee, niin tiedämme, että meillä kaikilla on tieto. Tieto paisuttaa, mutta rakkaus rakentaa.
\par 2 Jos joku luulee jotakin tietävänsä, ei hän vielä tiedä, niinkuin tietää tulee;
\par 3 mutta joka rakastaa Jumalaa, sen Jumala tuntee.
\par 4 Mitä nyt epäjumalille uhratun lihan syömiseen tulee, niin tiedämme, ettei maailmassa ole yhtään epäjumalaa ja ettei ole muuta Jumalaa kuin yksi.
\par 5 Sillä vaikka olisikin niin sanottuja jumalia, olipa heitä sitten taivaassa tai maassa, ja niitä on paljon semmoisia jumalia ja herroja,
\par 6 niin on meillä kuitenkin ainoastaan yksi Jumala, Isä, josta kaikki on ja johon me olemme luodut, ja yksi Herra, Jeesus Kristus, jonka kautta kaikki on, niin myös me hänen kauttansa.
\par 7 Mutta ei ole kaikilla tätä tietoa, vaan tottumuksesta epäjumaliin muutamat vielä nytkin syövät uhrilihaa ikäänkuin epäjumalille uhrattuna, ja heidän omatuntonsa, joka on heikko, tahraantuu siitä.
\par 8 Mutta ruoka ei lähennä meitä Jumalaan; jos olemme syömättä, emme siitä vahingoitu; jos syömme, emme siitä hyödy.
\par 9 Katsokaa kuitenkin, ettei tämä vapautenne koidu heikoille loukkaukseksi.
\par 10 Sillä jos joku näkee sinun, jolla on tieto, aterioivan epäjumalan huoneessa, eikö hänen omatuntonsa, kun hän on heikko, vahvistu epäjumalille uhratun syömiseen?
\par 11 Sinun tietosi kautta turmeltuu silloin tuo heikko, sinun veljesi, jonka tähden Kristus on kuollut.
\par 12 Mutta kun te näin teette syntiä veljiä vastaan ja haavoitatte heidän heikkoa omaatuntoaan, niin teette syntiä Kristusta vastaan.
\par 13 Sentähden, jos ruoka on viettelykseksi veljelleni, en minä ikinä enää syö lihaa, etten olisi viettelykseksi veljelleni.

\chapter{9}

\par 1 Enkö minä ole vapaa? Enkö minä ole apostoli? Enkö ole nähnyt Jeesusta, meidän Herraamme? Ettekö te ole minun tekoni Herrassa?
\par 2 Jos en olekaan apostoli muille, olen ainakin teille; sillä te olette minun apostolinvirkani sinetti Herrassa.
\par 3 Tämä on minun puolustukseni niitä vastaan, jotka asettuvat minua tutkimaan.
\par 4 Eikö meillä olisi oikeus saada ruokamme ja juomamme?
\par 5 Eikö meillä olisi oikeus kuljettaa muassamme vaimoa, uskonsisarta, niinkuin muutkin apostolit ja Herran veljet ja Keefas tekevät?
\par 6 Vai ainoastaanko minulla ja Barnabaalla ei ole oikeutta olla ruumiillista työtä tekemättä?
\par 7 Kuka tekee koskaan sotapalvelusta omalla kustannuksellaan? Kuka istuttaa viinitarhan, eikä syö sen hedelmää? Tai kuka kaitsee karjaa, eikä nauti karjansa maitoa?
\par 8 Puhunko tätä vain ihmisten tavalla? Eikö myös laki sano samaa?
\par 9 Onhan Mooseksen laissa kirjoitettuna: "Älä sido puivan härän suuta". Eihän Jumala häristä näin huolta pitäne?
\par 10 Eikö hän sano sitä kaiketikin meidän tähtemme? Meidän tähtemmehän on kirjoitettu, että kyntäjän tulee kyntää toivossa ja puivan puida osansa saamisen toivossa.
\par 11 Jos me olemme kylväneet teille hengellistä hyvää, onko paljon, jos me niitämme teiltä aineellista?
\par 12 Jos muilla on teihin tällainen oikeus, eikö paljoa enemmän meillä? Mutta me emme ole käyttäneet tätä oikeutta, vaan kestämme kaikki, ettemme panisi mitään estettä Kristuksen evankeliumille.
\par 13 Ettekö tiedä, että ne, jotka hoitavat pyhäkön toimia, saavat ravintonsa pyhäköstä, ja jotka ovat asetetut uhrialttarin palvelukseen, saavat osansa silloin kuin alttarikin?
\par 14 Samoin myös Herra on säätänyt, että evankeliumin julistajain tulee saada evankeliumista elatuksensa.
\par 15 Mutta minä en ole ainoatakaan näistä oikeuksistani hyväkseni käyttänyt. Enkä kirjoitakaan tätä siinä tarkoituksessa, että niitä minuun sovitettaisiin, sillä mieluummin minä kuolen. Ei kukaan ole minun kerskaustani tyhjäksi tekevä.
\par 16 Sillä siitä, että julistan evankeliumia, ei minulla ole kerskaamista; minun täytyy se tehdä. Voi minua, ellen evankeliumia julista!
\par 17 Sillä jos vapaasta tahdostani sitä teen, niin minulla on palkka; mutta jos en tee sitä vapaasta tahdostani, niin on huoneenhaltijan toimi kuitenkin minulle uskottu.
\par 18 Mikä siis on minun palkkani? Se, että kun julistan evankeliumia, teen sen ilmaiseksi, niin etten käytä oikeutta, jonka evankeliumi minulle myöntää.
\par 19 Sillä vaikka minä olen riippumaton kaikista, olen tehnyt itseni kaikkien palvelijaksi, voittaakseni niin monta kuin suinkin,
\par 20 ja olen ollut juutalaisille ikäänkuin juutalainen, voittaakseni juutalaisia; lain alaisille ikäänkuin lain alainen, vaikka itse en ole lain alainen, voittaakseni lain alaiset;
\par 21 ilman lakia oleville ikäänkuin olisin ilman lakia - vaikka en ole ilman Jumalan lakia, vaan olen Kristuksen laissa - voittaakseni ne, jotka ovat ilman lakia;
\par 22 heikoille minä olen ollut heikko, voittaakseni heikot; kaikille minä olen ollut kaikkea, pelastaakseni edes muutamia.
\par 23 Mutta kaiken minä teen evankeliumin tähden, että minäkin tulisin siitä osalliseksi.
\par 24 Ettekö tiedä, että jotka kilparadalla juoksevat, ne tosin kaikki juoksevat, mutta yksi saa voittopalkinnon? Juoskaa niinkuin hän, että sen saavuttaisitte.
\par 25 Mutta jokainen kilpailija noudattaa itsensähillitsemistä kaikessa; he saadakseen vain katoavaisen seppeleen, mutta me katoamattoman.
\par 26 Minä en siis juokse umpimähkään, en taistele niinkuin ilmaan hosuen,
\par 27 vaan minä kuritan ruumistani ja masennan sitä, etten minä, joka muille saarnaan, itse ehkä joutuisi hyljättäväksi.

\chapter{10}

\par 1 Sillä minä en tahdo, veljet, pitää teitä tietämättöminä siitä, että isämme olivat kaikki pilven alla ja kulkivat kaikki meren läpi
\par 2 ja saivat kaikki kasteen Moosekseen pilvessä ja meressä
\par 3 ja söivät kaikki samaa hengellistä ruokaa
\par 4 ja joivat kaikki samaa hengellistä juomaa. Sillä he joivat hengellisestä kalliosta, joka heitä seurasi; ja se kallio oli Kristus.
\par 5 Mutta useimpiin heistä Jumala ei mielistynyt, koskapa he hukkuivat erämaassa.
\par 6 Tämä tapahtui varoittavaksi esimerkiksi meille, että me emme pahaa himoitsisi, niinkuin he himoitsivat.
\par 7 Älkää myöskään ruvetko epäjumalanpalvelijoiksi kuten muutamat heistä, niinkuin kirjoitettu on: "Kansa istui syömään ja juomaan, ja he nousivat iloa pitämään".
\par 8 Älkäämmekä harjoittako haureutta, niinkuin muutamat heistä haureutta harjoittivat, ja heitä kaatui yhtenä päivänä kaksikymmentä kolme tuhatta.
\par 9 Älkäämme myöskään kiusatko Herraa, niinkuin muutamat heistä kiusasivat ja saivat käärmeiltä surmansa.
\par 10 Älkääkä napisko, niinkuin muutamat heistä napisivat ja saivat surmansa tuhoojalta.
\par 11 Tämä, mikä tapahtui heille, on esikuvallista ja on kirjoitettu varoitukseksi meille, joille maailmanaikojen loppukausi on tullut.
\par 12 Sentähden, joka luulee seisovansa, katsokoon, ettei lankea.
\par 13 Teitä ei ole kohdannut muu kuin inhimillinen kiusaus; ja Jumala on uskollinen, hän ei salli teitä kiusattavan yli voimienne, vaan salliessaan kiusauksen hän valmistaa myös pääsyn siitä, niin että voitte sen kestää.
\par 14 Sentähden, rakkaani, paetkaa epäjumalanpalvelusta.
\par 15 Minä puhun niinkuin ymmärtäväisille; arvostelkaa itse, mitä minä sanon.
\par 16 Siunauksen malja, jonka me siunaamme, eikö se ole osallisuus Kristuksen vereen? Se leipä, jonka murramme, eikö se ole osallisuus Kristuksen ruumiiseen?
\par 17 Koska leipä on yksi, niin me monet olemme yksi ruumis; sillä me olemme kaikki tuosta yhdestä leivästä osalliset.
\par 18 Katsokaa luonnollista Israelia; eivätkö ne, jotka syövät uhreja, ole alttarista osalliset?
\par 19 Mitä siis sanon? Ettäkö epäjumalanuhri on jotakin, tai että epäjumala on jotakin?
\par 20 Ei, vaan että, mitä pakanat uhraavat, sen he uhraavat riivaajille eivätkä Jumalalle; mutta minä en tahdo, että te tulette osallisiksi riivaajista.
\par 21 Ette voi juoda Herran maljasta ja riivaajien maljasta, ette voi olla osalliset Herran pöydästä ja riivaajien pöydästä.
\par 22 Vai tahdommeko herättää Herran kiivauden? Emme kaiketi me ole häntä väkevämmät?
\par 23 "Kaikki on luvallista", mutta ei kaikki ole hyödyksi; "kaikki on luvallista", mutta ei kaikki rakenna.
\par 24 Älköön kukaan katsoko omaa parastaan, vaan toisen parasta.
\par 25 Syökää kaikkea, mitä lihakaupassa myydään, kyselemättä mitään omantunnon tähden,
\par 26 sillä: "Herran on maa ja kaikki, mitä siinä on".
\par 27 Jos joku, joka ei usko, kutsuu teitä ja te tahdotte mennä hänen luokseen, niin syökää kaikkea, mitä eteenne pannaan, kyselemättä mitään omantunnon tähden.
\par 28 Mutta jos joku sanoo teille: "Tämä on epäjumalille uhrattua", niin jättäkää se syömättä hänen tähtensä, joka sen ilmaisi, ja omantunnon tähden;
\par 29 en tarkoita sinun omaatuntoasi, vaan tuon toisen; sillä miksi minun vapauteni joutuisi toisen omantunnon tuomittavaksi?
\par 30 Jos minä sen kiittäen nautin, miksi minua herjataan siitä, mistä kiitän?
\par 31 Söittepä siis tai joitte tai teittepä mitä hyvänsä, tehkää kaikki Jumalan kunniaksi.
\par 32 Älkää olko pahennukseksi juutalaisille, älkää kreikkalaisille älkääkä Jumalan seurakunnalle,
\par 33 niinkuin minäkin koetan kelvata kaikille kaikessa enkä katso omaa hyötyäni, vaan monien hyötyä, että he pelastuisivat.

\chapter{11}

\par 1 Olkaa minun seuraajiani, niinkuin minä olen Kristuksen seuraaja.
\par 2 Minä kiitän teitä, että minua kaikessa muistatte ja noudatatte minun opetuksiani, niinkuin minä ne teille olen antanut.
\par 3 Mutta minä tahdon, että te tiedätte sen, että Kristus on jokaisen miehen pää ja että mies on vaimon pää ja että Jumala on Kristuksen pää.
\par 4 Jokainen mies, joka rukoilee tai profetoi pää peitettynä, häpäisee päänsä.
\par 5 Mutta jokainen vaimo, joka rukoilee tai profetoi pää peittämätönnä, häpäisee päänsä, sillä se on aivan sama, kuin jos hänen päänsä olisi paljaaksi ajeltu.
\par 6 Sillä jos vaimo ei verhoa päätään, leikkauttakoon hiuksensakin; mutta koska on häpeäksi vaimolle, että hän leikkauttaa tai ajattaa hiuksensa, niin verhotkoon itsensä.
\par 7 Miehen ei tule peittää päätänsä, koska hän on Jumalan kuva ja kunnia; mutta vaimo on miehen kunnia.
\par 8 Sillä mies ei ole alkuisin vaimosta, vaan vaimo miehestä;
\par 9 eikä miestä luotu vaimoa varten, vaan vaimo miestä varten.
\par 10 Sentähden vaimon tulee pitää päässään vallanalaisuuden merkki enkelien tähden.
\par 11 Herrassa ei kuitenkaan ole vaimoa ilman miestä eikä miestä ilman vaimoa.
\par 12 Sillä samoin kuin vaimo on alkuisin miehestä, samoin myös mies on vaimon kautta; mutta kaikki on Jumalasta.
\par 13 Päättäkää itse: sopiiko vaimon rukoilla Jumalaa pää peittämätönnä?
\par 14 Eikö itse luontokin opeta teille, että jos miehellä on pitkät hiukset, se on hänelle häpeäksi;
\par 15 ja että jos vaimolla on pitkät hiukset, se on hänelle kunniaksi? Sillä ovathan hiukset annetut hänelle hunnuksi.
\par 16 Mutta jos joku haluaa väittää vastaan, niin tietäköön, että meillä ei ole sellaista tapaa eikä Jumalan seurakunnilla.
\par 17 Mutta tätä käskiessäni en kiitä sitä, että kokoontumisenne ei tee teitä paremmiksi, vaan pahemmiksi.
\par 18 Sillä ensiksikin olen kuullut, että kun kokoonnutte seurakunnankokoukseen, teillä on riitaisuuksia keskenänne, ja osittain sen uskonkin.
\par 19 Täytyyhän teidän keskuudessanne olla puolueitakin, että kävisi ilmi, ketkä teistä kestävät koetuksen.
\par 20 Kun te näin kokoonnutte yhteen, niin ei se ole Herran aterian nauttimista,
\par 21 sillä syömään ruvettaessa kukin rientää ottamaan eteensä omat ruokansa, ja niin toinen on nälissään ja toinen juovuksissa.
\par 22 Eikö teillä sitten ole muita huoneita, niissä syödäksenne ja juodaksenne? Vai halveksitteko Jumalan seurakuntaa ja tahdotteko häväistä niitä, joilla ei mitään ole? Mitä minun on teille sanominen? Onko minun teitä kiittäminen? Tässä kohden en kiitä.
\par 23 Sillä minä olen saanut Herralta sen, minkä myös olen teille tiedoksi antanut, että Herra Jeesus sinä yönä, jona hänet kavallettiin, otti leivän,
\par 24 kiitti, mursi ja sanoi: "Tämä on minun ruumiini, joka teidän edestänne annetaan; tehkää tämä minun muistokseni".
\par 25 Samoin hän otti myös maljan aterian jälkeen ja sanoi: "Tämä malja on uusi liitto minun veressäni; niin usein kuin te juotte, tehkää se minun muistokseni".
\par 26 Sillä niin usein kuin te syötte tätä leipää ja juotte tämän maljan, te julistatte Herran kuolemaa, siihen asti kuin hän tulee.
\par 27 Sentähden, joka kelvottomasti syö tätä leipää tai juo Herran maljan, hän on oleva vikapää Herran ruumiiseen ja vereen.
\par 28 Koetelkoon siis ihminen itseänsä, ja niin syököön tätä leipää ja juokoon tästä maljasta;
\par 29 sillä joka syö ja juo erottamatta Herran ruumista muusta, syö ja juo tuomioksensa.
\par 30 Sentähden onkin teidän joukossanne paljon heikkoja ja sairaita, ja moni on nukkunut pois.
\par 31 Mutta jos me tutkisimme itseämme, ei meitä tuomittaisi;
\par 32 mutta kun meitä tuomitaan, niin se on meille Herran kuritusta, ettei meitä maailman kanssa kadotukseen tuomittaisi.
\par 33 Sentähden, veljeni, kun kokoonnutte aterioimaan, odottakaa toisianne.
\par 34 Jos kenellä on nälkä, syököön kotonaan, ettette kokoontuisi tuomioksenne. Muista seikoista minä säädän, sitten kuin tulen.

\chapter{12}

\par 1 Mutta mitä hengellisiin lahjoihin tulee, niin en tahdo, veljet, pitää teitä niistä tietämättöminä.
\par 2 Te tiedätte, että kun olitte pakanoita, teitä vietiin mykkien epäjumalien luo, miten vain tahdottiin.
\par 3 Sentähden minä teen teille tiettäväksi, ettei kukaan, joka puhuu Jumalan Hengessä, sano: "Jeesus olkoon kirottu", ja ettei kukaan voi sanoa: "Jeesus olkoon Herra", paitsi Pyhässä Hengessä.
\par 4 Armolahjat ovat moninaiset, mutta Henki on sama;
\par 5 ja seurakuntavirat ovat moninaiset, mutta Herra on sama;
\par 6 ja voimavaikutukset ovat moninaiset, mutta Jumala, joka kaikki kaikissa vaikuttaa, on sama.
\par 7 Mutta kullekin annetaan Hengen ilmoitus yhteiseksi hyödyksi.
\par 8 Niinpä saa Hengen kautta toinen viisauden sanat, toinen tiedon sanat saman Hengen vaikutuksesta;
\par 9 toinen saa uskon samassa Hengessä, toinen taas terveeksitekemisen lahjat siinä yhdessä Hengessä;
\par 10 toinen lahjan tehdä voimallisia tekoja; toinen profetoimisen lahjan, toinen lahjan arvostella henkiä; toinen eri kielillä puhumisen lahjan, toinen taas lahjan selittää kieliä.
\par 11 Mutta kaiken tämän vaikuttaa yksi ja sama Henki, jakaen kullekin erikseen, niinkuin tahtoo.
\par 12 Sillä niinkuin ruumis on yksi ja siinä on monta jäsentä, mutta kaikki ruumiin jäsenet, vaikka niitä on monta, ovat yksi ruumis, niin on Kristuskin;
\par 13 sillä me olemme kaikki yhdessä Hengessä kastetut yhdeksi ruumiiksi, olimmepa juutalaisia tai kreikkalaisia, orjia tai vapaita, ja kaikki olemme saaneet juoda samaa Henkeä.
\par 14 Sillä eihän ruumiskaan ole yksi jäsen, vaan niitä on siinä monta.
\par 15 Jos jalka sanoisi: "Koska en ole käsi, en kuulu ruumiiseen", niin ei se silti olisi ruumiiseen kuulumaton.
\par 16 Ja jos korva sanoisi: "Koska en ole silmä, en kuulu ruumiiseen", niin ei se silti olisi ruumiiseen kuulumaton.
\par 17 Jos koko ruumis olisi silmänä, missä sitten olisi kuulo? Jos taas kokonaan kuulona, missä silloin haisti?
\par 18 Mutta nyt Jumala on asettanut jäsenet, itsekunkin niistä, ruumiiseen, niinkuin hän on tahtonut.
\par 19 Vaan jos ne kaikki olisivat yhtenä jäsenenä, missä sitten ruumis olisi?
\par 20 Mutta nytpä onkin monta jäsentä, ja ainoastaan yksi ruumis.
\par 21 Silmä ei saata sanoa kädelle: "En tarvitse sinua", eikä myöskään pää jaloille: "En tarvitse teitä".
\par 22 Päinvastoin ne ruumiin jäsenet, jotka näyttävät olevan heikompia, ovat välttämättömiä;
\par 23 ja ne ruumiin jäsenet, jotka meistä ovat vähemmän kunniakkaita, me verhoamme sitä kunniallisemmin, ja niitä, joita häpeämme, me sitä häveliäämmin peitämme;
\par 24 mutta ne, joita emme häpeä, eivät sitä tarvitse. Mutta Jumala on liittänyt ruumiin yhteen niin, että antoi halvempiarvoiselle suuremman kunnian,
\par 25 ettei ruumiissa olisi eripuraisuutta, vaan että jäsenet pitäisivät yhtäläistä huolta toinen toisestaan.
\par 26 Ja jos yksi jäsen kärsii, niin kaikki jäsenet kärsivät sen kanssa; tai jos jollekulle jäsenelle annetaan kunnia, niin kaikki jäsenet iloitsevat sen kanssa.
\par 27 Mutta te olette Kristuksen ruumis ja kukin osaltanne hänen jäseniänsä.
\par 28 Niinpä Jumala asetti seurakuntaan ensiksi muutamia apostoleiksi, toisia profeetoiksi, kolmansia opettajiksi, sitten hän antoi voimallisia tekoja, sitten armolahjoja parantaa tauteja, avustaa, hallita, puhua eri kielillä.
\par 29 Eivät suinkaan kaikki ole apostoleja? Eivät kaikki profeettoja? Eivät kaikki opettajia? Eiväthän kaikki tee voimallisia tekoja?
\par 30 Eihän kaikilla ole parantamisen armolahjoja? Eiväthän kaikki puhu kielillä? Eiväthän kaikki kykene niitä selittämään?
\par 31 Pyrkikää osallisiksi parhaimmista armolahjoista. Ja vielä minä osoitan teille tien, verrattoman tien.

\chapter{13}

\par 1 Vaikka minä puhuisin ihmisten ja enkelien kielillä, mutta minulla ei olisi rakkautta, olisin minä vain helisevä vaski tai kilisevä kulkunen.
\par 2 Ja vaikka minulla olisi profetoimisen lahja ja minä tietäisin kaikki salaisuudet ja kaiken tiedon, ja vaikka minulla olisi kaikki usko, niin että voisin vuoria siirtää, mutta minulla ei olisi rakkautta, en minä mitään olisi.
\par 3 Ja vaikka minä jakelisin kaiken omaisuuteni köyhäin ravinnoksi, ja vaikka antaisin ruumiini poltettavaksi, mutta minulla ei olisi rakkautta, ei se minua mitään hyödyttäisi.
\par 4 Rakkaus on pitkämielinen, rakkaus on lempeä; rakkaus ei kadehdi, ei kerskaa, ei pöyhkeile,
\par 5 ei käyttäydy sopimattomasti, ei etsi omaansa, ei katkeroidu, ei muistele kärsimäänsä pahaa,
\par 6 ei iloitse vääryydestä, vaan iloitsee yhdessä totuuden kanssa;
\par 7 kaikki se peittää, kaikki se uskoo, kaikki se toivoo, kaikki se kärsii.
\par 8 Rakkaus ei koskaan häviä; mutta profetoiminen, se katoaa, ja kielillä puhuminen lakkaa, ja tieto katoaa.
\par 9 Sillä tietomme on vajavaista, ja profetoimisemme on vajavaista.
\par 10 Mutta kun tulee se, mikä täydellistä on, katoaa se, mikä on vajavaista.
\par 11 Kun minä olin lapsi, niin minä puhuin kuin lapsi, minulla oli lapsen mieli, ja minä ajattelin kuin lapsi; kun tulin mieheksi, hylkäsin minä sen, mikä lapsen on.
\par 12 Sillä nyt me näemme kuin kuvastimessa, arvoituksen tavoin, mutta silloin kasvoista kasvoihin; nyt minä tunnen vajavaisesti, mutta silloin minä olen tunteva täydellisesti, niinkuin minut itsenikin täydellisesti tunnetaan.
\par 13 Niin pysyvät nyt usko, toivo, rakkaus, nämä kolme; mutta suurin niistä on rakkaus.

\chapter{14}

\par 1 Tavoitelkaa rakkautta ja pyrkikää saamaan hengellisiä lahjoja, mutta varsinkin profetoimisen lahjaa.
\par 2 Sillä kielillä puhuva ei puhu ihmisille, vaan Jumalalle; ei häntä näet kukaan ymmärrä, sillä hän puhuu salaisuuksia hengessä.
\par 3 Mutta profetoiva puhuu ihmisille rakennukseksi ja kehoitukseksi ja lohdutukseksi.
\par 4 Kielillä puhuva rakentaa itseään, mutta profetoiva rakentaa seurakuntaa.
\par 5 Soisin teidän kaikkien puhuvan kielillä, mutta vielä mieluummin soisin teidän profetoivan; sillä profetoiva on suurempi kuin kielillä puhuva, ellei tämä samalla selitä, niin että seurakunta siitä rakentuu.
\par 6 Jos minä nyt, veljet, tulisin luoksenne kielillä puhuen, mitä minä teitä sillä hyödyttäisin, ellen puhuisi teille ilmestyksen tai tiedon tai profetian tai opetuksen sanoja?
\par 7 Niinhän on elottomain soittimienkin laita, huilujen tai kitarain: kuinka tiedetään, mitä huilulla tai kitaralla soitetaan, elleivät ne soi toisistaan erottuvin sävelin?
\par 8 Niinikään, jos pasuna antaa epäselvän äänen, kuka silloin valmistautuu taisteluun?
\par 9 Samoin tekin: jos ette kielellänne saa esiin selvää puhetta, kuinka voidaan sellainen puhe ymmärtää? Tehän puhutte silloin tuuleen.
\par 10 Maailmassa on, kuka tietää, kuinka monta eri kieltä, mutta ei ainoatakaan, jonka äänet eivät ole ymmärrettävissä.
\par 11 Mutta jos en tiedä sanojen merkitystä, olen minä puhujalle muukalainen, ja puhuja on minulle muukalainen.
\par 12 Samoin tekin, koska tavoittelette henkilahjoja, niin pyrkikää seurakunnan rakennukseksi saamaan niitä runsaasti.
\par 13 Sentähden rukoilkoon se, joka kielillä puhuu, että hän taitaisi selittää.
\par 14 Sillä jos minä rukoilen kielillä puhuen, niin minun henkeni kyllä rukoilee, mutta ymmärrykseni on hedelmätön.
\par 15 Kuinka siis on? Minun on rukoiltava hengelläni, mutta minun on rukoiltava myöskin ymmärrykselläni; minun on veisattava kiitosta hengelläni, mutta minun on veisattava myöskin ymmärrykselläni.
\par 16 Sillä jos ylistät Jumalaa hengessä, kuinka oppimattoman paikalla istuva saattaa sanoa "amen" sinun kiitokseesi? Eihän hän ymmärrä, mitä sanot.
\par 17 Sinä kyllä kiität hyvin, mutta toinen ei siitä rakennu.
\par 18 Minä kiitän Jumalaa, että puhun kielillä enemmän kuin teistä kukaan;
\par 19 mutta seurakunnassa tahdon mieluummin puhua viisi sanaa ymmärrykselläni, opettaakseni muitakin, kuin kymmenentuhatta sanaa kielillä.
\par 20 Veljet, älkää olko lapsia ymmärrykseltänne, vaan pahuudessa olkaa lapsia; mutta ymmärrykseltä olkaa täysi-ikäisiä.
\par 21 Laissa on kirjoitettuna: "Vieraskielisten kautta ja muukalaisten huulilla minä olen puhuva tälle kansalle, eivätkä he sittenkään minua kuule, sanoo Herra".
\par 22 Kielet eivät siis ole merkiksi uskoville, vaan niille, jotka eivät usko; mutta profetoiminen ei ole merkiksi uskottomille, vaan uskoville.
\par 23 Jos nyt koko seurakunta kokoontuisi yhteen ja kaikki siellä puhuisivat kielillä ja sinne tulisi opetuksesta tai uskosta osattomia, eivätkö he sanoisi teidän olevan järjiltänne?
\par 24 Mutta jos kaikki profetoisivat ja joku uskosta tai opetuksesta osaton tulisi sisään, niin kaikki paljastaisivat hänet ja kaikki langettaisivat hänestä tuomion,
\par 25 hänen sydämensä salaisuudet tulisivat ilmi, ja niin hän kasvoilleen langeten rukoilisi Jumalaa ja julistaisi, että Jumala totisesti on teissä.
\par 26 Kuinka siis on, veljet? Kun tulette yhteen, on jokaisella jotakin annettavaa: millä on virsi, millä opetus, millä ilmestys, mikä puhuu kielillä, mikä selittää; kaikki tapahtukoon rakennukseksi.
\par 27 Jos kielillä puhutaan, niin puhukoon kullakin kertaa vain kaksi tai enintään kolme, ja yksi kerrallaan, ja yksi selittäköön;
\par 28 mutta jos ei ole selittäjää, niin olkoot vaiti seurakunnassa ja puhukoot itselleen ja Jumalalle.
\par 29 Profeetoista saakoon kaksi tai kolme puhua, ja muut arvostelkoot;
\par 30 mutta jos joku toinen siinä istuva saa ilmestyksen, vaietkoon ensimmäinen.
\par 31 Sillä te saatatte kaikki profetoida, toinen toisenne jälkeen, että kaikki saisivat opetusta ja kaikki kehoitusta.
\par 32 Ja profeettain henget ovat profeetoille alamaiset;
\par 33 sillä ei Jumala ole epäjärjestyksen, vaan rauhan Jumala. Niinkuin kaikissa pyhien seurakunnissa,
\par 34 olkoot vaimot vaiti teidänkin seurakunnankokouksissanne, sillä heidän ei ole lupa puhua, vaan olkoot alamaisia, niinkuin lakikin sanoo.
\par 35 Mutta jos he tahtovat tietoa jostakin, niin kysykööt kotonaan omilta miehiltään, sillä häpeällistä on naisen puhua seurakunnassa.
\par 36 Vai teistäkö Jumalan sana on lähtenyt? Vai ainoastaan teidänkö tykönne se on tullut?
\par 37 Jos joku luulee olevansa profeetta tai hengellinen, niin tietäköön, että mitä minä kirjoitan teille, se on Herran käsky.
\par 38 Mutta jos joku ei sitä ymmärrä, niin olkoon ymmärtämättä.
\par 39 Sentähden, veljeni, harrastakaa profetoimista älkääkä estäkö kielillä puhumasta.
\par 40 Mutta kaikki tapahtukoon säädyllisesti ja järjestyksessä.

\chapter{15}

\par 1 Veljet, minä johdatan teidät tuntemaan sen evankeliumin, jonka minä teille julistin, jonka te myöskin olette ottaneet vastaan ja jossa myös pysytte
\par 2 ja jonka kautta te myös pelastutte, jos pidätte siitä kiinni semmoisena, kuin minä sen teille julistin, ellette turhaan ole uskoneet.
\par 3 Sillä minä annoin teille ennen kaikkea tiedoksi sen, minkä itse olin saanut: että Kristus on kuollut meidän syntiemme tähden, kirjoitusten mukaan,
\par 4 ja että hänet haudattiin ja että hän nousi kuolleista kolmantena päivänä, kirjoitusten mukaan,
\par 5 ja että hän näyttäytyi Keefaalle, sitten niille kahdelletoista.
\par 6 Sen jälkeen hän näyttäytyi yhtä haavaa enemmälle kuin viidellesadalle veljelle, joista useimmat vielä nytkin ovat elossa, mutta muutamat ovat nukkuneet pois.
\par 7 Sen jälkeen hän näyttäytyi Jaakobille, sitten kaikille apostoleille.
\par 8 Mutta kaikkein viimeiseksi hän näyttäytyi minullekin, joka olen ikäänkuin keskensyntynyt.
\par 9 Sillä minä olen apostoleista halvin enkä ole sen arvoinen, että minua apostoliksi kutsutaan, koska olen vainonnut Jumalan seurakuntaa.
\par 10 Mutta Jumalan armosta minä olen se, mikä olen, eikä hänen armonsa minua kohtaan ole ollut turha, vaan enemmän kuin he kaikki minä olen työtä tehnyt, en kuitenkaan minä, vaan Jumalan armo, joka on minun kanssani.
\par 11 Olinpa siis minä tai olivatpa he: näin me saarnaamme, ja näin te olette uskoon tulleet.
\par 12 Mutta jos Kristuksesta saarnataan, että hän on noussut kuolleista, kuinka muutamat teistä saattavat sanoa, ettei kuolleitten ylösnousemusta ole?
\par 13 Vaan jos ei ole kuolleitten ylösnousemusta, ei Kristuskaan ole noussut.
\par 14 Mutta jos Kristus ei ole noussut kuolleista, turha on silloin meidän saarnamme, turha myös teidän uskonne;
\par 15 ja silloin meidät myös havaitaan vääriksi Jumalan todistajiksi, koska olemme todistaneet Jumalaa vastaan, että hän on herättänyt Kristuksen, jota hän ei ole herättänyt, jos kerran kuolleita ei herätetä.
\par 16 Sillä jos kuolleita ei herätetä, ei Kristuskaan ole herätetty.
\par 17 Mutta jos Kristus ei ole herätetty, niin teidän uskonne on turha, ja te olette vielä synneissänne.
\par 18 Ja silloinhan Kristuksessa nukkuneetkin olisivat kadotetut.
\par 19 Jos olemme panneet toivomme Kristukseen ainoastaan tämän elämän ajaksi, niin olemme kaikkia muita ihmisiä surkuteltavammat.
\par 20 Mutta nytpä Kristus on noussut kuolleista, esikoisena kuoloon nukkuneista.
\par 21 Sillä koska kuolema on tullut ihmisen kautta, niin on myöskin kuolleitten ylösnousemus tullut ihmisen kautta.
\par 22 Sillä niinkuin kaikki kuolevat Aadamissa, niin myös kaikki tehdään eläviksi Kristuksessa,
\par 23 mutta jokainen vuorollaan: esikoisena Kristus, sitten Kristuksen omat hänen tulemuksessaan;
\par 24 sitten tulee loppu, kun hän antaa valtakunnan Jumalan ja Isän haltuun, kukistettuaan kaiken hallituksen ja kaiken vallan ja voiman.
\par 25 Sillä hänen pitää hallitseman "siihen asti, kunnes hän on pannut kaikki viholliset jalkojensa alle".
\par 26 Vihollisista viimeisenä kukistetaan kuolema.
\par 27 Sillä: "kaikki hän on alistanut hänen jalkojensa alle". Mutta kun hän sanoo: "kaikki on alistettu", niin ei tietenkään ole alistettu se, joka on alistanut kaiken hänen allensa.
\par 28 Ja kun kaikki on alistettu Pojan valtaan, silloin itse Poikakin alistetaan sen valtaan, joka on alistanut hänen valtaansa kaiken, että Jumala olisi kaikki kaikissa.
\par 29 Mitä muutoin ne, jotka kastattavat itsensä kuolleitten puolesta, sillä saavat aikaan? Jos kuolleet eivät heräjä, miksi nämä sitten kastattavat itsensä heidän puolestaan?
\par 30 Ja miksi mekään antaudumme joka hetki vaaraan?
\par 31 Joka päivä minä olen kuoleman kidassa, niin totta kuin te, veljet, olette minun kerskaukseni Kristuksessa Jeesuksessa, meidän Herrassamme.
\par 32 Jos minä ihmisten tavoin olen taistellut petojen kanssa Efesossa, mitä hyötyä minulle siitä on? Jos kuolleet eivät heräjä, niin: "syökäämme ja juokaamme, sillä huomenna me kuolemme".
\par 33 Älkää eksykö. "Huono seura hyvät tavat turmelee."
\par 34 Raitistukaa oikealla tavalla, älkääkä syntiä tehkö; sillä niitä on, joilla ei ole mitään tietoa Jumalasta. Teidän häpeäksenne minä tämän sanon.
\par 35 Mutta joku ehkä kysyy: "Millä tavoin kuolleet heräjävät, ja millaisessa ruumiissa he tulevat?"
\par 36 Sinä mieletön, se, minkä kylvät, ei virkoa eloon, ellei se ensin kuole!
\par 37 Ja kun kylvät, et kylvä sitä vartta, joka on nouseva, vaan paljaan jyvän, nisun jyvän tai muun minkä tahansa.
\par 38 Mutta Jumala antaa sille varren, sellaisen kuin tahtoo, ja kullekin siemenelle sen oman varren.
\par 39 Ei kaikki liha ole samaa lihaa, vaan toista on ihmisten, toista taas karjan liha, toista on lintujen liha, toista kalojen.
\par 40 Ja on taivaallisia ruumiita ja maallisia ruumiita; mutta toinen on taivaallisten kirkkaus, toinen taas maallisten.
\par 41 Toinen on auringon kirkkaus ja toinen kuun kirkkaus ja toinen tähtien kirkkaus, ja toinen tähti voittaa toisen kirkkaudessa.
\par 42 Niin on myös kuolleitten ylösnousemus: kylvetään katoavaisuudessa, nousee katoamattomuudessa;
\par 43 kylvetään alhaisuudessa, nousee kirkkaudessa; kylvetään heikkoudessa, nousee voimassa;
\par 44 kylvetään sielullinen ruumis, nousee hengellinen ruumis. Jos kerran on sielullinen ruumis, niin on myös hengellinen.
\par 45 Niin on myös kirjoitettu: "Ensimmäisestä ihmisestä, Aadamista, tuli elävä sielu"; viimeisestä Aadamista tuli eläväksitekevä henki.
\par 46 Mutta mikä on hengellistä, se ei ole ensimmäinen, vaan se, mikä on sielullista, on ensimmäinen; sitten on se, mikä on hengellistä.
\par 47 Ensimmäinen ihminen oli maasta, maallinen, toinen ihminen on taivaasta.
\par 48 Minkäkaltainen maallinen oli, senkaltaisia ovat myös maalliset; ja minkäkaltainen taivaallinen on, senkaltaisia ovat myös taivaalliset.
\par 49 Ja niinkuin meissä on ollut maallisen kuva, niin meissä on myös oleva taivaallisen kuva.
\par 50 Mutta tämän minä sanon, veljet, ettei liha ja veri voi periä Jumalan valtakuntaa, eikä katoavaisuus peri katoamattomuutta.
\par 51 Katso, minä sanon teille salaisuuden: emme kaikki kuolemaan nuku, mutta kaikki me muutumme,
\par 52 yhtäkkiä, silmänräpäyksessä, viimeisen pasunan soidessa; sillä pasuna soi, ja kuolleet nousevat katoamattomina, ja me muutumme.
\par 53 Sillä tämän katoavaisen pitää pukeutuman katoamattomuuteen, ja tämän kuolevaisen pitää pukeutuman kuolemattomuuteen.
\par 54 Mutta kun tämä katoavainen pukeutuu katoamattomuuteen ja tämä kuolevainen pukeutuu kuolemattomuuteen, silloin toteutuu se sana, joka on kirjoitettu: "Kuolema on nielty ja voitto saatu".
\par 55 "Kuolema, missä on sinun voittosi? Kuolema, missä on sinun otasi?"
\par 56 Mutta kuoleman ota on synti, ja synnin voima on laki.
\par 57 Mutta kiitos olkoon Jumalan, joka antaa meille voiton meidän Herramme Jeesuksen Kristuksen kautta!
\par 58 Sentähden, rakkaat veljeni, olkaa lujat, järkähtämättömät, aina innokkaat Herran työssä, tietäen, että teidän vaivannäkönne ei ole turha Herrassa.

\chapter{16}

\par 1 Mitä tulee keräykseen pyhiä varten, niin tehkää tekin samalla tavoin, kuin minä olen määrännyt Galatian seurakunnille.
\par 2 Kunkin viikon ensimmäisenä päivänä pankoon jokainen teistä kotonaan jotakin talteen, säästäen menestymisensä mukaan, ettei keräyksiä tehtäisi vasta minun tultuani.
\par 3 Mutta kun olen saapunut teidän tykönne, lähetän minä ne henkilöt, jotka siihen sopiviksi katsotte, kirjeet mukanaan, viemään teidän rakkaudenlahjanne Jerusalemiin;
\par 4 ja jos asia on sen arvoinen, että minunkin on lähteminen, niin saavat he lähteä minun kanssani.
\par 5 Aion nimittäin tulla teidän tykönne kuljettuani läpi Makedonian, sillä minä kuljen Makedonian kautta;
\par 6 mutta teidän luonanne viivyn ehkä jonkin aikaa, kenties vietän talvenkin, että te sitten auttaisitte minua eteenpäin, minne matkustanenkin.
\par 7 Sillä en tahdo nyt vain ohimennen käydä teitä katsomassa, toivon näet saavani viipyä jonkin aikaa teidän luonanne, jos Herra sallii.
\par 8 Mutta Efesossa minä viivyn helluntaihin saakka;
\par 9 sillä minulle on avautunut ovi suureen ja hedelmälliseen työhön, ja vastustajia on paljon.
\par 10 Mutta jos Timoteus saapuu, niin katsokaa, että hän pelotta voi olla teidän luonanne, sillä Herran työtä hän toimittaa niinkuin minäkin.
\par 11 Älköön siis kukaan häntä halveksiko, vaan auttakaa häntä lähtemään rauhassa matkalle, että hän tulisi minun tyköni; sillä minä ja veljet odotamme häntä.
\par 12 Mitä veli Apollokseen tulee, olen hartaasti kehoittanut häntä lähtemään veljien kanssa teidän tykönne; mutta hän ei ole ollenkaan halukas lähtemään nyt, vaan tulee, kun hänelle sopii.
\par 13 Valvokaa, pysykää lujina uskossa, olkaa miehuulliset, olkaa väkevät.
\par 14 Kaikki, mitä teette, tapahtukoon rakkaudessa.
\par 15 Mutta minä kehoitan teitä, veljet: te tunnette Stefanaan perhekunnan ja tiedätte, että se on Akaian ensi hedelmä ja että he ovat antautuneet pyhien palvelukseen.
\par 16 Olkaa tekin kuuliaisia heidän kaltaisilleen ja jokaiselle, joka heidän kanssaan työtä tekee ja vaivaa näkee.
\par 17 Minä iloitsen Stefanaan ja Fortunatuksen ja Akaikuksen tännetulosta, koska he korvaavat minulle teidän poissaolonne;
\par 18 he ovat virkistäneet minun henkeäni niinkuin teidänkin. Antakaa siis täysi tunnustus sellaisille miehille.
\par 19 Aasian seurakunnat tervehtivät teitä. Monet tervehdykset Herrassa lähettävät teille Akylas ja Priska sekä heidän kodissaan kokoontuva seurakunta.
\par 20 Kaikki veljet lähettävät teille tervehdyksen. Tervehtikää toisianne pyhällä suunannolla.
\par 21 Tervehdys minulta, Paavalilta, omakätisesti.
\par 22 Jos joku ei pidä Herraa rakkaana, hän olkoon kirottu! Maran ata.
\par 23 Herran Jeesuksen armo olkoon teidän kanssanne.
\par 24 Minun rakkauteni on teidän kaikkien kanssa Kristuksessa Jeesuksessa.


\end{document}