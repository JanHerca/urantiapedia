\begin{document}

\title{Ensimmäinen Samuelin kirja}


\chapter{1}

\par 1 Raamataim-Soofimissa, Efraimin vuoristossa, oli mies, nimeltä Elkana, Jerohamin poika, joka oli Elihun poika, joka Toohun poika, joka Suufin poika, efratilainen.
\par 2 Hänellä oli kaksi vaimoa; toisen nimi oli Hanna, ja toisen nimi oli Peninna. Ja Peninnalla oli lapsia, mutta Hanna oli lapseton.
\par 3 Tämä mies meni joka vuosi kaupungistansa Siiloon rukoilemaan Herraa Sebaotia ja uhraamaan hänelle. Ja siellä oli kaksi Eelin poikaa, Hofni ja Piinehas, Herran pappeina.
\par 4 Niin Elkana eräänä päivänä uhrasi. Ja hänellä oli tapana antaa vaimollensa Peninnalle ja kaikille tämän pojille ja tyttärille määräosat.
\par 5 Mutta Hannalle hän antoi kahdenkertaisen osan, sillä hän rakasti Hannaa, vaikka Herra oli sulkenut hänen kohtunsa.
\par 6 Ja hänen kilpailijattarensa kiusasi häntä kiusaamistaan suututtaaksensa häntä, koska Herra oli sulkenut hänen kohtunsa.
\par 7 Ja näin tapahtui joka vuosi, niin usein kuin hän meni Herran huoneeseen. Niinpä Peninna nytkin kiusasi häntä, ja hän itki eikä syönyt mitään.
\par 8 Silloin hänen miehensä Elkana sanoi hänelle: "Hanna, mitä itket, miksi et syö ja miksi olet noin apealla mielellä? Enkö minä ole sinulle enempi kuin kymmenen poikaa?"
\par 9 Kun he sitten olivat syöneet ja juoneet Siilossa ja pappi Eeli istui istuimellaan Herran temppelin ovenpielessä, nousi Hanna
\par 10 ja rukoili Herraa mieli murheellisena, itki katkerasti
\par 11 ja teki lupauksen sanoen: "Herra Sebaot, jos sinä katsot palvelijattaresi kurjuutta, muistat minua etkä unhota palvelijatartasi, vaan annat palvelijattarellesi miehisen perillisen, niin minä annan hänet Herralle koko hänen elinajaksensa, eikä partaveitsi ole koskettava hänen päätänsä".
\par 12 Ja kun hän kauan rukoili Herran edessä, tarkkasi Eeli hänen suutansa,
\par 13 sillä Hanna puhui sydämessänsä ja ainoastaan hänen huulensa liikkuivat, mutta ääntä häneltä ei kuulunut; niin Eeli luuli, että hän oli juovuksissa.
\par 14 Ja Eeli sanoi hänelle: "Kuinka kauan sinä siinä olet juovuksissa? Haihduta humalasi."
\par 15 Mutta Hanna vastasi ja sanoi: "Ei, herrani, minä olen murheen raskauttama vaimo; viiniä ja väkijuomaa en ole juonut, vaan minä vuodatin sydämeni Herran eteen.
\par 16 Älä pidä palvelijatartasi kelvottomana naisena, sillä tuskani ja suruni suuruuden tähden minä olen näin kauan puhunut."
\par 17 Eeli vastasi ja sanoi: "Mene rauhaan, Israelin Jumala antakoon sinulle, mitä olet häneltä pyytänyt".
\par 18 Hän sanoi: "Salli palvelijattaresi saada armo sinun silmiesi edessä". Niin vaimo meni pois ja söi eikä enää näyttänyt murheelliselta.
\par 19 Varhain seuraavana aamuna he nousivat ja kumartaen rukoilivat Herran edessä; sitten he palasivat ja tulivat kotiinsa Raamaan. Ja Elkana yhtyi vaimoonsa Hannaan, ja Herra muisti tätä.
\par 20 Ja Hanna tuli raskaaksi ja synnytti vuoden vaihteessa pojan ja antoi hänelle nimen Samuel, sanoen: "Herralta olen minä hänet pyytänyt".
\par 21 Sitten se mies, Elkana, ja koko hänen perheensä lähti uhraamaan Herralle jokavuotista teurasuhria ja lupausuhriaan,
\par 22 mutta Hanna ei lähtenyt, vaan sanoi miehellensä: "Kun poika on vieroitettu, vien minä hänet sinne, niin että hän tulee Herran kasvojen eteen ja saa jäädä sinne ainiaaksi".
\par 23 Hänen miehensä Elkana sanoi hänelle: "Tee, niinkuin hyväksi näet, jää kotiin, kunnes olet hänet vieroittanut; kunhan Herra vain täyttää sanansa". Niin vaimo jäi kotiin ja imetti poikaansa, kunnes hän vieroitti tämän.
\par 24 Mutta kun hän oli hänet vieroittanut, vei hän hänet kanssansa ottaen mukaansa kolme härkää, yhden eefa-mitan jauhoja ja leilin viiniä; niin hän toi hänet Herran huoneeseen Siiloon. Mutta poika oli vielä pieni.
\par 25 Ja teurastettuaan härän he toivat pojan Eelin tykö.
\par 26 Ja Hanna sanoi: "Oi, herrani, niin totta kuin sinä elät, herrani, minä olen se vaimo, joka seisoin tässä sinun luonasi rukoillen Herraa.
\par 27 Tätä poikaa minä rukoilin; Herra antoi minulle, mitä häneltä pyysin.
\par 28 Sentähden myös minä suostun antamaan hänet Herralle: kaikiksi elinpäiviksensä hän olkoon Herralle annettu." Ja Samuel rukoili siellä Herraa.

\chapter{2}

\par 1 Ja Hanna rukoili ja sanoi: "Minun sydämeni riemuitsee Herrassa; minun sarveni kohoaa korkealle Herrassa. Minun suuni on avartunut vihollisiani vastaan, sillä minä iloitsen sinun avustasi.
\par 2 Ei kukaan ole pyhä niinkuin Herra; paitsi sinua ei ole yhtäkään, eikä ole kalliota meidän Jumalamme vertaista.
\par 3 Älkää yhä ylpeitä puhuko, älköön suustanne lähtekö julkeita sanoja; sillä Herra on kaikkitietävä Jumala, hänen edessänsä punnitaan teot.
\par 4 Urhojen jouset ovat särjetyt, voipuneet vyöttäytyvät voimalla.
\par 5 Kylläisinä olleet käyvät palvelukseen leipäpalkoilla, nälkää nähneet eivät enää nälkää näe. Hedelmätön synnyttää seitsemän lasta, lapsirikas kuihtuu.
\par 6 Herra antaa kuoleman ja antaa elämän, hän vie alas tuonelaan ja tuo ylös jälleen.
\par 7 Herra köyhdyttää ja rikastuttaa, hän alentaa ja ylentää.
\par 8 Hän tomusta nostaa halvan, hän loasta korottaa köyhän, pannaksensa heidät ruhtinasten rinnalle ja antaaksensa heidän periä kunniasijat. Sillä maan tukipylväät ovat Herran, hän on asettanut niiden päälle maanpiirin.
\par 9 Hän varjelee hurskastensa jalat, mutta jumalattomat hukkuvat pimeyteen; sillä mies ei omalla voimallaan mitään voi.
\par 10 Jotka taistelevat Herraa vastaan, ne joutuvat kauhun valtaan; hän jylisee taivaasta heidän ylitsensä. Herra tuomitsee maan ääret; hän antaa vallan kuninkaallensa ja korottaa korkealle voideltunsa sarven."
\par 11 Sitten Elkana meni kotiinsa Raamaan; mutta poikanen palveli Herraa pappi Eelin johdolla.
\par 12 Mutta Eelin pojat olivat kelvottomia miehiä; he eivät välittäneet Herrasta
\par 13 eivätkä siitä, mitä papeilla oli oikeus kansalta saada. Niin usein kuin joku uhrasi teurasuhria, tuli papin palvelija lihan kiehuessa, kolmihaarainen haarukka kädessänsä,
\par 14 ja pisti sen kattilaan tai ruukkuun tai pannuun tai pataan, ja kaikki, minkä haarukka toi mukanaan, sen pappi otti itselleen. Niin he tekivät kaikille israelilaisille, jotka tulivat sinne Siiloon.
\par 15 Jo ennenkuin rasva oli poltettu, tuli papin palvelija ja sanoi miehelle, joka uhrasi: "Anna liha paistettavaksi papille, sillä hän ei ota sinulta keitettyä lihaa, vaan raakaa".
\par 16 Kun mies sanoi hänelle: "Ensin poltettakoon rasva; ota sitten itsellesi, mitä haluat", niin hän sanoi: "Ei niin, vaan anna se nyt heti, muutoin minä otan väkisin".
\par 17 Ja nuorten miesten synti oli kovin suuri Herran edessä, koska he pitivät Herran uhrilahjan halpana.
\par 18 Mutta Samuel palveli Herran edessä, puettuna, vaikka oli poikanen, pellavakasukkaan.
\par 19 Ja hänen äitinsä teki hänelle joka vuosi pienen viitan ja toi sen hänelle, kun miehensä kanssa tuli uhraamaan vuotuista teurasuhria.
\par 20 Silloin Eeli aina siunasi Elkanan ja hänen vaimonsa ja sanoi: "Herra antakoon sinulle lapsen tästä vaimosta sen sijaan, jonka hän on suostunut antamaan Herralle"; sitten he menivät kotiinsa.
\par 21 Ja Herra piti Hannasta huolen, niin että hän tuli raskaaksi ja synnytti vielä kolme poikaa ja kaksi tytärtä. Mutta poikanen Samuel kasvoi Herran edessä.
\par 22 Mutta Eeli oli käynyt hyvin vanhaksi; ja kun hän sai kuulla, mitä hänen poikansa tekivät kaikelle Israelille, ja että he makasivat niiden naisten kanssa, jotka toimittivat palvelusta ilmestysmajan ovella,
\par 23 sanoi hän heille: "Miksi te sellaista pahaa teette, mitä minä teistä kuulen kaikelta tältä kansalta?
\par 24 Ei niin, poikani! Ei ole hyvä se huhu, jonka minä kuulen kulkevan Herran kansan seassa.
\par 25 Jos ihminen rikkoo ihmistä vastaan, niin Jumala voi olla hänellä välittäjänä; mutta jos ihminen rikkoo Herraa vastaan, niin kuka voi ruveta hänelle välittäjäksi?" Mutta he eivät kuulleet isäänsä, sillä Herra tahtoi surmata heidät.
\par 26 Mutta poikanen Samuel kasvoi kasvamistaan ja oli sekä Herralle että ihmisille otollinen.
\par 27 Ja Eelin luo tuli Jumalan mies ja sanoi hänelle: "Näin sanoo Herra: Enkö minä ilmestynyt sinun isäsi suvulle, kun he olivat Egyptissä faraon huoneen vallassa?
\par 28 Ja minä valitsin heidät kaikista Israelin sukukunnista olemaan pappeinani, nousemaan minun alttarilleni, polttamaan suitsuketta ja kantamaan kasukkaa minun edessäni; ja minä annoin sinun isäsi suvulle kaikki israelilaisten uhrit.
\par 29 Miksi te häpäisette minun teurasuhrini ja ruokauhrini, jotka minä olen säätänyt asuntooni? Ja miksi sinä kunnioitat poikiasi enemmän kuin minua, niin että te lihotatte itseänne parhaalla osalla jokaisesta minun kansani Israelin uhrilahjasta?
\par 30 Sentähden Herra, Israelin Jumala, sanoo: Minä olen tosin sanonut: sinun sukusi ja sinun isäsi suku saavat vaeltaa minun edessäni iäti. Mutta nyt Herra sanoo: Pois se! Sillä minä kunnioitan niitä, jotka minua kunnioittavat; mutta jotka minut ylenkatsovat, ne tulevat halveksituiksi.
\par 31 Katso, päivät tulevat, jolloin minä katkaisen sinun käsivartesi ja sinun isäsi suvun käsivarren, niin ettei kukaan sinun suvussasi elä vanhaksi.
\par 32 Ja sinä saat nähdä minun asuntoni olevan puutteessa huolimatta kaikesta siitä hyvästä, mikä tulee Israelille. Eikä kukaan sinun suvussasi ole koskaan elävä vanhaksi.
\par 33 Kuitenkaan minä en hävitä kaikkia sinun omaisiasi alttariltani, niin että sammuttaisin sinun silmäsi ja näännyttäisin sielusi; mutta kaikki sinun suvussasi kasvaneet kuolevat miehiksi tultuaan.
\par 34 Ja merkkinä tästä on sinulla oleva se, mitä kahdelle pojallesi, Hofnille ja Piinehaalle, tapahtuu: he kuolevat molemmat samana päivänä.
\par 35 Mutta minä herätän itselleni uskollisen papin, joka tekee minun sydämeni ja mieleni mukaan. Hänelle minä rakennan pysyväisen huoneen, ja hän on vaeltava minun voideltuni edessä kaiken elinaikansa.
\par 36 Ja jokainen, joka jää jäljelle sinun suvustasi, on tuleva kumartamaan häntä saadaksensa hopearahan ja leipäkakun ja sanovat: 'Päästä minutkin johonkin papintehtävään, että saisin palan leipää syödäkseni'."

\chapter{3}

\par 1 Ja poikanen Samuel palveli Herraa Eelin johdolla; mutta Herran sana oli harvinainen siihen aikaan, eivätkä näyt olleet tavallisia.
\par 2 Siihen aikaan tapahtui kerran, kun Eeli, jonka silmiä alkoi hämärtää, niin ettei hän voinut nähdä, makasi sijallansa
\par 3 eikä Jumalan lamppu ollut vielä sammunut ja Samuel makasi Herran temppelissä, jossa Jumalan arkki oli,
\par 4 että Herra kutsui Samuelia. Hän vastasi: "Tässä olen".
\par 5 Ja hän riensi Eelin tykö ja sanoi: "Tässä olen, sinä kutsuit minua". Mutta hän vastasi: "En minä kutsunut; pane jälleen maata". Ja hän meni ja pani maata.
\par 6 Mutta Herra kutsui taas Samuelia; ja Samuel nousi ja meni Eelin tykö ja sanoi: "Tässä olen, sinä kutsuit minua". Mutta tämä vastasi: "En minä kutsunut, poikani; pane jälleen maata".
\par 7 Mutta Samuel ei silloin vielä tuntenut Herraa, eikä Herran sana ollut vielä ilmestynyt hänelle.
\par 8 Ja Herra kutsui Samuelia vielä kolmannen kerran; niin hän nousi ja meni Eelin tykö ja sanoi: "Tässä olen, sinä kutsuit minua". Silloin Eeli ymmärsi, että Herra oli kutsunut poikasta.
\par 9 Ja Eeli sanoi Samuelille: "Mene ja pane maata; ja jos sinua vielä kutsutaan, niin sano: 'Puhu, Herra; palvelijasi kuulee'". Samuel meni ja pani maata sijallensa.
\par 10 Niin Herra tuli ja seisoi ja huusi niinkuin edellisilläkin kerroilla: "Samuel, Samuel!" Samuel vastasi: "Puhu, palvelijasi kuulee".
\par 11 Ja Herra sanoi Samuelille: "Katso, minä teen Israelissa sellaisen teon, että joka sen kuulee, sen molemmat korvat soivat.
\par 12 Sinä päivänä minä annan toteutua Eelille kaiken, mitä olen hänen sukuansa vastaan puhunut, alusta loppuun asti.
\par 13 Minä ilmoitan hänelle, että minä tuomitsen hänen sukunsa ikuisiksi ajoiksi sen rikoksen tähden, kun hän tiesi poikiensa pilkkaavan Jumalaa eikä pitänyt heitä kurissa.
\par 14 Sentähden minä olen vannonut Eelin suvusta: 'Totisesti, Eelin suvun rikosta ei koskaan soviteta, ei teurasuhrilla eikä ruokauhrilla'."
\par 15 Ja Samuel makasi aamuun asti ja avasi sitten Herran huoneen ovet. Eikä Samuel uskaltanut kertoa Eelille sitä ilmestystä.
\par 16 Mutta Eeli kutsui Samuelin ja sanoi: "Samuel, poikani!" Hän vastasi: "Tässä olen".
\par 17 Hän sanoi: "Mikä oli asia, josta hän sinulle puhui? Älä salaa sitä minulta. Jumala rangaiskoon sinua nyt ja vasta, jos salaat minulta mitään siitä, mitä hän sinulle puhui."
\par 18 Niin Samuel kertoi hänelle kaiken eikä salannut häneltä mitään. Mutta hän sanoi: "Hän on Herra; hän tehköön, minkä hyväksi näkee".
\par 19 Ja Samuel kasvoi, ja Herra oli hänen kanssansa eikä antanut yhdenkään sanoistansa varista maahan.
\par 20 Ja koko Israel Daanista Beersebaan asti tiesi, että Samuelille oli uskottu Herran profeetan tehtävä.
\par 21 Ja Herra ilmestyi edelleenkin Siilossa; sillä Herra ilmestyi Samuelille Siilossa Herran sanan kautta. Ja Samuelin sana tuli koko Israelille.

\chapter{4}

\par 1 Israel lähti sotaan filistealaisia vastaan ja leiriytyi Eben-Eserin luo, mutta filistealaiset olivat leiriytyneet Afekiin.
\par 2 Ja filistealaiset asettuivat sotarintaan Israelia vastaan, ja taistelu levisi laajalle, ja filistealaiset voittivat Israelin ja surmasivat taistelukentällä noin neljätuhatta miestä.
\par 3 Kun kansa tuli leiriin, sanoivat Israelin vanhimmat: "Minkätähden Herra antoi tänä päivänä filistealaisten voittaa meidät? Ottakaamme Herran liitonarkki Siilosta luoksemme, tulkoon se keskellemme pelastamaan meidät vihollistemme käsistä."
\par 4 Ja kansa lähetti sanan Siiloon, ja sieltä tuotiin Herran Sebaotin liitonarkki, hänen, jonka valtaistuinta kerubit kannattavat; ja Eelin kaksi poikaa, Hofni ja Piinehas, seurasivat sieltä Jumalan liitonarkkia.
\par 5 Ja kun Herran liitonarkki tuli leiriin, nosti koko Israel suuren riemuhuudon, niin että maa jymisi.
\par 6 Kun filistealaiset kuulivat sen riemuhuudon, sanoivat he: "Mitä tuo suuri riemuhuuto on hebrealaisten leirissä?" Ja he saivat tietää, että leiriin oli tullut Herran arkki.
\par 7 Silloin filistealaiset peljästyivät, sillä he ajattelivat: "Jumala on tullut leiriin". Ja he sanoivat: "Voi meitä! Tällaista ei ole milloinkaan ennen tapahtunut.
\par 8 Voi meitä! Kuka pelastaa meidät tämän voimallisen jumalan kädestä? Tämähän on se jumala, joka löi egyptiläisiä kaikkinaisilla vaivoilla erämaassa.
\par 9 Rohkaiskaa mielenne ja olkaa miehiä, filistealaiset, ettette te joutuisi palvelemaan hebrealaisia, niinkuin he ovat palvelleet teitä. Olkaa miehiä ja taistelkaa."
\par 10 Ja filistealaiset ryhtyivät taisteluun; ja Israel voitettiin, niin että he pakenivat kukin majallensa, ja tappio oli sangen suuri: Israelista kaatui kolmekymmentä tuhatta jalkamiestä.
\par 11 Ja Jumalan arkki ryöstettiin, ja Eelin kaksi poikaa, Hofni ja Piinehas, saivat surmansa.
\par 12 Eräs benjaminilainen mies juoksi sotarinnasta ja tuli samana päivänä Siiloon, takki reväistynä ja multaa pään päällä.
\par 13 Ja kun hän tuli sinne, istui Eeli istuimellaan tien vieressä tähystellen, sillä hänen sydämensä oli levoton Jumalan arkin tähden. Ja kun mies tuli ja kertoi kaupungissa tapahtumasta, puhkesi koko kaupunki valittamaan.
\par 14 Kun Eeli kuuli valitushuudon, kysyi hän: "Mikä meteli tämä on?" Niin mies tuli rientäen sinne ja kertoi sen Eelille.
\par 15 Mutta Eeli oli yhdeksänkymmenen kahdeksan vuoden vanha, ja hänen silmissään oli kaihi, niin ettei hän voinut nähdä.
\par 16 Ja mies sanoi Eelille: "Minä tulen sotarinnasta; olen tänä päivänä paennut sotarinnasta". Niin hän kysyi: "Miten on asiat, poikani?"
\par 17 Sanansaattaja vastasi ja sanoi: "Israel on paennut filistealaisia, ja kansan mieshukka on suuri; myöskin sinun molemmat poikasi, Hofni ja Piinehas, ovat saaneet surmansa, ja Jumalan arkki on ryöstetty".
\par 18 Kun hän mainitsi Jumalan arkin, kaatui Eeli istuimeltaan taapäin portin viereen, taittoi niskansa ja kuoli; sillä hän oli vanha ja raskas mies. Ja hän oli ollut tuomarina Israelissa neljäkymmentä vuotta.
\par 19 Ja hänen miniänsä, Piinehaan vaimo, oli viimeisillänsä raskaana. Kun hän kuuli sanoman Jumalan arkin ryöstöstä ja appensa ja miehensä kuolemasta, vaipui hän maahan ja synnytti, sillä poltot yllättivät hänet.
\par 20 Ja kun hän silloin oli kuolemaisillansa, sanoivat vaimot, jotka seisoivat hänen luonaan: "Älä pelkää, sillä sinä olet synnyttänyt pojan". Mutta hän ei vastannut mitään eikä välittänyt siitä,
\par 21 vaan nimitti pojan Iikabodiksi ja sanoi: "Kunnia on mennyt Israelilta" - hän tarkoitti Jumalan arkin ryöstöä ja appeansa ja miestänsä.
\par 22 Hän sanoi: "Kunnia on mennyt Israelilta", koska Jumalan arkki oli ryöstetty.

\chapter{5}

\par 1 Kun filistealaiset olivat ryöstäneet Jumalan arkin, veivät he sen Eben-Eseristä Asdodiin.
\par 2 Ja filistealaiset ottivat Jumalan arkin ja veivät sen Daagonin temppeliin ja asettivat sen Daagonin rinnalle.
\par 3 Kun asdodilaiset nousivat varhain seuraavana päivänä, niin katso: Daagon makasi kasvoillaan maassa Herran arkin edessä. Mutta he ottivat Daagonin ja asettivat sen takaisin paikoillensa.
\par 4 Kun he taas nousivat varhain seuraavana aamuna, niin katso: Daagon makasi kasvoillaan maassa Herran arkin edessä, mutta Daagonin pää ja molemmat kädet olivat katkaistuina kynnyksellä, ja ainoastaan muu ruumis oli jäljellä.
\par 5 Sentähden eivät Daagonin papit eikä kukaan, joka menee Daagonin temppeliin, astu Daagonin kynnykselle Asdodissa vielä tänäkään päivänä.
\par 6 Mutta Herran käsi painoi asdodilaisia, ja hän tuhosi heitä: hän löi heitä ajoksilla, sekä Asdodia että sen aluetta.
\par 7 Kun Asdodin miehet näkivät, miten oli, sanoivat he: "Älköön Israelin Jumalan arkki jääkö meidän luoksemme, sillä hänen kätensä on käynyt raskaaksi meille ja meidän jumalallemme Daagonille".
\par 8 Ja he lähettivät sanan ja kokosivat kaikki filistealaisten ruhtinaat luoksensa ja sanoivat: "Mitä me teemme Israelin Jumalan arkille?" He vastasivat: "Israelin Jumalan arkki siirtyköön Gatiin". Ja he siirsivät Israelin Jumalan arkin sinne.
\par 9 Mutta kun he olivat siirtäneet sen sinne, kohtasi Herran käsi kaupunkia ja sai aikaan suuren hämmingin: hän löi kaupungin asukkaita, sekä pieniä että suuria, niin että heihin puhkesi ajoksia.
\par 10 Silloin he lähettivät Jumalan arkin Ekroniin. Mutta kun Jumalan arkki tuli Ekroniin, huusivat ekronilaiset sanoen: "He ovat siirtäneet Israelin Jumalan arkin tänne minun luokseni, surmaamaan minut ja minun kansani".
\par 11 Ja he lähettivät sanan ja kokosivat kaikki filistealaisten ruhtinaat ja sanoivat: "Lähettäkää Israelin Jumalan arkki pois, ja palatkoon se kotiinsa älköönkä enää surmatko minua ja minun kansaani". Sillä surma oli saattanut koko kaupungin hämminkiin; Jumalan käsi painoi sitä kovasti.
\par 12 Ne asukkaat, jotka eivät kuolleet, olivat ajoksilla lyödyt, ja huuto nousi kaupungista ylös taivaaseen.

\chapter{6}

\par 1 Herran arkki oli filistealaisten maassa seitsemän kuukautta.
\par 2 Ja filistealaiset kutsuivat papit ja tietäjät ja sanoivat: "Mitä me teemme Herran arkille? Antakaa meidän tietää, millä tavalla me lähettäisimme sen kotiinsa."
\par 3 He vastasivat: "Jos te lähetätte Israelin Jumalan arkin pois, niin älkää lähettäkö häntä tyhjin käsin, vaan antakaa hänelle hyvitys. Silloin te tulette terveiksi ja saatte tietää, minkätähden hänen kätensä ei heltiä teistä."
\par 4 Mutta he kysyivät: "Minkälainen hyvitys meidän on annettava hänelle?" He vastasivat: "Viisi kulta-ajosta ja viisi kultahiirtä, yhtä monta, kuin on filistealaisten ruhtinasta; sillä sama vitsaus on kohdannut kaikkia, myöskin teidän ruhtinaitanne.
\par 5 Tehkää siis ajoksistanne kuvat, samoin hiiristänne, jotka hävittävät maata, ja antakaa kunnia Israelin Jumalalle. Kenties hän silloin hellittää kätensä teistä, teidän jumalistanne ja maastanne.
\par 6 Miksi te paadutatte sydämenne, niinkuin egyptiläiset ja farao paaduttivat sydämensä? Eivätkö he, sittenkuin hän oli näyttänyt heille voimansa, päästäneet israelilaisia menemään?
\par 7 Niin tehkää nyt uudet vaunut ja ottakaa kaksi imettävää lehmää, joiden niskassa ei ole iestä ollut, ja valjastakaa lehmät vaunujen eteen, mutta ajakaa vasikat kotiin, pois niiden jäljestä.
\par 8 Ja ottakaa Herran arkki ja asettakaa se vaunuihin ja pankaa ne kultakalut, jotka annatte hänelle hyvitykseksi, lippaaseen sen viereen, ja lähettäkää se menemään.
\par 9 Ja sitten katsokaa: jos hän menee omaa aluettaan kohti, ylös Beet-Semekseen, niin on hän tuottanut meille tämän suuren onnettomuuden; mutta jos ei, niin tiedämme, ettei hänen kätensä ole meihin koskenut. Se on silloin tapahtunut meille sattumalta."
\par 10 Miehet tekivät niin; he ottivat kaksi imettävää lehmää ja valjastivat ne vaunujen eteen, mutta niiden vasikat he jättivät kotiin.
\par 11 Ja he panivat Herran arkin vaunuihin ynnä lippaan, kultahiiret ja paiseittensa kuvat.
\par 12 Ja lehmät menivät suoraan Beet-Semestä kohti; ne kulkivat koko ajan samaa tietä, poikkeamatta oikealle tai vasemmalle, ja ammuivat lakkaamatta. Ja filistealaisten ruhtinaat seurasivat niitä Beet-Semeksen alueelle asti.
\par 13 Mutta Beet-Semeksessä oltiin leikkaamassa nisua laaksossa. Kun he nostivat silmänsä, näkivät he arkin; ja he ilostuivat nähdessään sen.
\par 14 Vaunut tulivat beet-semekseläisen Joosuan pellolle ja pysähtyivät sinne; ja siellä oli suuri kivi. Ja he hakkasivat vaunut haloiksi ja uhrasivat lehmät polttouhriksi Herralle.
\par 15 Ja leeviläiset nostivat maahan Herran arkin ja sen vieressä olevan lippaan, jossa kultakalut olivat, ja panivat ne sille suurelle kivelle. Ja Beet-Semeksen miehet uhrasivat sinä päivänä Herralle polttouhreja ja teurasuhreja.
\par 16 Katseltuaan tätä ne filistealaisten viisi ruhtinasta palasivat samana päivänä Ekroniin.
\par 17 Nämä ovat ne kultapaiseet, jotka filistealaiset antoivat Herralle hyvitykseksi: yksi Asdodin puolesta, yksi Gassan puolesta, yksi Askelonin puolesta, yksi Gatin puolesta ja yksi Ekronin puolesta.
\par 18 Mutta kultahiiriä oli yhtä monta, kuin oli kaikkiaan filistealaisten kaupunkeja viidellä ruhtinaalla, niin hyvin varustettuja kaupunkeja kuin linnoittamattomia kyliä, suureen Aabel-kiveen asti, jolle he laskivat Herran arkin ja joka on vielä tänäkin päivänä beet-semekseläisen Joosuan pellolla.
\par 19 Mutta Herra surmasi Beet-Semeksen miehiä sentähden, että he olivat katsoneet Herran arkkia; hän surmasi kansasta seitsemänkymmentä miestä, viisikymmentä tuhatta miestä; ja kansa suri sitä, että Herra oli surmannut niin paljon kansaa.
\par 20 Ja Beet-Semeksen miehet sanoivat: "Kuka voi kestää Herran, tämän pyhän Jumalan, edessä? Ja kenen luo hän menee meidän luotamme?"
\par 21 Ja he lähettivät sanansaattajat Kirjat-Jearimin asukasten luo sanomaan: "Filistealaiset ovat tuoneet Herran arkin takaisin; tulkaa tänne ja viekää se luoksenne".

\chapter{7}

\par 1 Niin Kirjat-Jearimin miehet tulivat ja veivät Herran arkin sinne ja toivat sen Abinadabin taloon mäelle. Ja hänen poikansa Eleasarin he pyhittivät pitämään huolta Herran arkista.
\par 2 Ja siitä päivästä, jona arkki oli asettunut Kirjat-Jearimiin, kului pitkä aika, kului kaksikymmentä vuotta; ja koko Israelin heimo huokaili Herran puoleen.
\par 3 Mutta Samuel sanoi koko Israelin heimolle näin: "Jos te kaikesta sydämestänne käännytte Herran puoleen, niin poistakaa keskuudestanne vieraat jumalat ja astartet ja kiinnittäkää sydämenne Herraan ja palvelkaa ainoastaan häntä, niin hän pelastaa teidät filistealaisten käsistä".
\par 4 Niin israelilaiset poistivat baalit ja astartet ja palvelivat ainoastaan Herraa.
\par 5 Ja Samuel sanoi: "Kootkaa koko Israel Mispaan, ja minä rukoilen teidän puolestanne Herraa".
\par 6 Niin he kokoontuivat Mispaan, ammensivat vettä ja vuodattivat sitä Herran eteen sekä paastosivat sen päivän; ja he sanoivat siellä: "Me olemme tehneet syntiä Herraa vastaan". Ja Samuel jakoi israelilaisille oikeutta Mispassa.
\par 7 Kun filistealaiset kuulivat israelilaisten kokoontuneen Mispaan, lähtivät filistealaisten ruhtinaat Israelia vastaan. Kun israelilaiset kuulivat sen, pelkäsivät he filistealaisia.
\par 8 Ja israelilaiset sanoivat Samuelille: "Älä kieltäydy huutamasta meidän puolestamme Herraa, meidän Jumalaamme, että hän pelastaisi meidät filistealaisten käsistä".
\par 9 Niin Samuel otti imevän karitsan ja uhrasi sen polttouhriksi, kokonaisuhriksi, Herralle; ja Samuel huusi Herraa Israelin puolesta, ja Herra kuuli häntä.
\par 10 Sillä kun Samuelin uhratessa polttouhria filistealaiset lähestyivät käydäkseen taisteluun Israelia vastaan, niin sinä päivänä Herra jylisi kovalla ukkosen jylinällä filistealaisia vastaan ja saattoi heidät hämminkiin, niin että Israel voitti heidät.
\par 11 Ja Israelin miehet lähtivät Mispasta ja ajoivat filistealaisia takaa, surmaten heitä aina Beet-Kaarin alapuolelle asti.
\par 12 Ja Samuel otti kiven ja pani sen Mispan ja Seenin välille ja antoi sille nimen Eben-Eser ja sanoi: "Tähän asti on Herra meitä auttanut".
\par 13 Niin filistealaiset lannistettiin, eivätkä he enää tulleet Israelin alueelle. Ja Herran käsi oli filistealaisia vastaan Samuelin koko elinajan.
\par 14 Ja ne kaupungit, jotka filistealaiset olivat ottaneet Israelilta, joutuivat takaisin Israelille, Ekronista Gatiin saakka; ja myös niiden alueet Israel vapautti filistealaisten käsistä. Myöskin Israelin ja amorilaisten välillä oli rauha.
\par 15 Samuel oli tuomarina Israelissa koko elinaikansa.
\par 16 Ja hän teki joka vuosi kiertomatkoja Beeteliin, Gilgaliin ja Mispaan, ja hän jakoi Israelille oikeutta kaikissa näissä paikoissa.
\par 17 Sitten hän palasi jälleen Raamaan, sillä siellä oli hänen kotinsa ja siellä hän jakoi oikeutta Israelille. Ja hän rakensi sinne alttarin Herralle.

\chapter{8}

\par 1 Kun Samuel oli käynyt vanhaksi, pani hän poikansa Israelin tuomareiksi.
\par 2 Hänen esikoisensa nimi oli Jooel, ja hänen toisen poikansa nimi oli Abia; nämä olivat tuomareina Beersebassa.
\par 3 Mutta hänen poikansa eivät vaeltaneet hänen teitänsä, vaan olivat väärän voiton pyytäjiä, ottivat lahjuksia ja vääristivät oikeutta.
\par 4 Niin kaikki Israelin vanhimmat kokoontuivat ja tulivat Samuelin tykö Raamaan
\par 5 ja sanoivat hänelle: "Katso, sinä olet käynyt vanhaksi, eivätkä poikasi vaella sinun teitäsi. Niin aseta nyt meille oikeutta jakamaan kuningas, jollainen kaikilla muillakin kansoilla on."
\par 6 Mutta Samuel pahastui siitä, että he sanoivat: "Anna meille kuningas jakamaan meille oikeutta". Ja Samuel rukoili Herraa.
\par 7 Niin Herra sanoi Samuelille: "Kuule kansan ääntä kaikessa, mitä he sinulle sanovat; sillä sinua he eivät ole pitäneet halpana, vaan minut he ovat pitäneet halpana olemaan heidän kuninkaanansa.
\par 8 Niinkuin he aina siitä päivästä, jona minä johdatin heidät tänne Egyptistä, tähän päivään asti ovat tehneet, kun ovat hyljänneet minut ja palvelleet muita jumalia, aivan niin he tekevät nyt sinullekin.
\par 9 Kuule siis heidän ääntänsä. Kuitenkin varoita heitä vakavasti ja ilmoita heille, mitkä oikeudet on kuninkaalla, joka on heitä hallitseva."
\par 10 Niin Samuel puhui kaikki Herran sanat kansalle, joka häneltä pyysi kuningasta.
\par 11 Hän sanoi: "Nämä oikeudet on kuninkaalla, joka on teitä hallitseva: Teidän poikanne hän ottaa ja panee heidät vaunumiehikseen ja ratsumiehikseen ja vaunujensa edelläjuoksijoiksi.
\par 12 Myöskin hän panee heitä tuhannen- ja viidenkymmenenpäämiehiksi ja kyntämään hänen kyntöjänsä ja leikkaamaan hänen viljaansa sekä tekemään hänelle sotatarpeita ja vaunutarpeita.
\par 13 Ja teidän tyttärenne hän ottaa voiteiden tekijöiksi, keittäjiksi ja leipojiksi.
\par 14 Hän ottaa teidän parhaat peltonne, viinitarhanne ja öljypuunne ja antaa ne palvelijoillensa;
\par 15 ja hän ottaa kymmenykset teidän kylvöstänne ja viinitarhojenne sadosta ja antaa ne hoviherroillensa ja palvelijoillensa.
\par 16 Hän ottaa myös teidän palvelijanne ja palvelijattarenne ja parhaat nuoret miehenne sekä aasinne ja teettää niillä työnsä;
\par 17 ja hän ottaa kymmenykset teidän lampaistanne, ja te tulette hänen palvelijoiksensa.
\par 18 Silloin te huudatte Herraa kuninkaanne tähden, jonka olette itsellenne valinneet, mutta hän ei silloin teitä kuule."
\par 19 Mutta kansa ei tahtonut kuulla Samuelin puhetta, vaan sanoi: "Ei, kuningas meillä pitää olla.
\par 20 Mekin tahdomme olla niinkuin kaikki muut kansat: kuningas jakakoon meille oikeutta ja johtakoon meitä ja käyköön sotiamme."
\par 21 Kun Samuel kuuli kaikki kansan puheet, puhui hän ne Herralle.
\par 22 Ja Herra sanoi Samuelille: "Kuule heidän ääntänsä ja aseta heille kuningas". Niin Samuel sanoi Israelin miehille: "Menkää kukin kaupunkiinne".

\chapter{9}

\par 1 Benjaminissa oli mies, hyvin varakas, nimeltä Kiis, Abielin poika, joka oli Serorin poika, joka Bekoratin poika, joka Afiahin poika, joka erään benjaminilaisen miehen poika.
\par 2 Hänellä oli poika, nimeltä Saul, nuori, kaunis mies. Israelilaisten joukossa ei ollut kauniimpaa miestä kuin hän: hän oli päätänsä pitempi kaikkea kansaa.
\par 3 Ja Kiisiltä, Saulin isältä, olivat aasintammat joutuneet kateisiin, ja Kiis sanoi pojallensa Saulille: "Ota palvelijoista joku mukaasi, nouse ja mene etsimään aasintammoja".
\par 4 Niin hän kulki Efraimin vuoriston halki, ja hän kulki Salisan maan halki; mutta he eivät löytäneet niitä. Ja he kulkivat Saalimin maan halki, eikä niitä ollut sielläkään; sitten he kulkivat Benjaminin maan halki eivätkä löytäneet niitä.
\par 5 Kun he olivat tulleet Suufin maahan, sanoi Saul palvelijallensa, joka hänellä oli mukanaan: "Tule, palatkaamme kotiin; muutoin isäni käy levottomaksi meidän tähtemme ja jättää mielestään aasintammat".
\par 6 Mutta palvelija sanoi hänelle: "Katso, tässä kaupungissa on Jumalan mies, arvossa pidetty mies; kaikki, mitä hän sanoo, se varmasti toteutuu. Menkäämme nyt sinne; kenties hän ilmaisee meille jotakin matkasta, jolla olemme."
\par 7 Niin Saul sanoi palvelijallensa: "Jos menemme sinne, niin mitä me viemme sille miehelle? Sillä leipä on loppunut repuistamme, eikä meillä ole muutakaan lahjaa vietäväksi Jumalan miehelle. Vai onko meillä mitään?"
\par 8 Palvelija vastasi vielä Saulille ja sanoi: "Katso, tässä on minulla neljännes sekeliä hopeata; minä annan sen Jumalan miehelle, että hän ilmaisisi meille jotakin matkastamme".
\par 9 - Muinoin sanottiin Israelissa, kun mentiin kysymään Jumalalta, näin: "Tulkaa, menkäämme näkijän luo". Sillä sitä, jota nyt sanotaan profeetaksi, kutsuttiin muinoin näkijäksi. -
\par 10 Saul vastasi palvelijallensa: "Puheesi on hyvä; tule, menkäämme". Ja he menivät kaupunkiin, jossa Jumalan mies oli.
\par 11 Kun he nousivat rinnettä ylös kaupunkiin, kohtasivat he tyttöjä, jotka olivat menossa vettä ammentamaan. Näiltä he kysyivät: "Onko näkijä täällä?"
\par 12 Nämä vastasivat heille ja sanoivat: "On, katso, hän on tuolla edessäpäin. Riennä, sillä hän on juuri nyt tullut kaupunkiin, koska kansalla tänä päivänä on teurasuhrit uhrikukkulalla.
\par 13 Kun tulette kaupunkiin, tapaatte hänet, ennenkuin hän menee uhrikukkulalle aterioimaan; sillä kansa ei aterioi, ennenkuin hän tulee. Kun hän on siunannut uhrin, sitten vasta kutsuvieraat aterioivat. Menkää siis sinne, sillä juuri nyt te hänet tapaatte."
\par 14 Ja he menivät kaupunkiin. Kun he tulivat kaupunkiin, niin katso, Samuel tuli heitä vastaan menossa uhrikukkulalle.
\par 15 Mutta päivää ennen Saulin tuloa oli Herra ilmoittanut ja sanonut Samuelille:
\par 16 "Huomenna tähän aikaan minä lähetän sinun tykösi miehen Benjaminin maasta. Voitele hänet minun kansani Israelin ruhtinaaksi. Hän on vapauttava minun kansani filistealaisten käsistä. Sillä minä olen katsonut kansani puoleen, sen huuto on tullut minun eteeni."
\par 17 Kun Samuel näki Saulin, ilmoitti Herra hänelle: "Katso, tässä on mies, josta minä olen sinulle sanonut: 'Tämä on vallitseva minun kansaani'".
\par 18 Saul meni Samuelin tykö keskelle porttia ja sanoi: "Sano minulle, missä on näkijän asunto?"
\par 19 Samuel vastasi Saulille ja sanoi: "Minä olen näkijä. Mene minun edelläni uhrikukkulalle. Aterioikaa tänä päivänä minun kanssani; huomenaamuna minä päästän sinut menemään, ja kaikki, mitä sinulla on sydämelläsi, minä selvitän sinulle.
\par 20 Äläkä enää ole huolissasi aasintammoista, jotka ovat olleet sinulta kateissa kolme päivää, sillä ne ovat löytyneet. Ja kenen on kaikki, mitä kallisarvoisinta Israelissa on, jollei sinun ja kaiken sinun isäsi perheen?"
\par 21 Saul vastasi ja sanoi: "Minähän olen benjaminilainen, olen sukukunnasta, joka on Israelin pienimpiä, ja minun sukuni on vähäisin kaikista Benjaminin sukukunnan suvuista. Miksi puhut minulle näin?"
\par 22 Mutta Samuel otti Saulin palvelijoinensa ja vei heidät ruokailuhuoneeseen ja antoi heille ylimmän sijan kutsuvierasten joukossa, joita oli noin kolmekymmentä miestä.
\par 23 Ja Samuel sanoi keittäjälle: "Tuo tänne se kappale, jonka minä sinulle annoin ja josta minä sinulle sanoin: 'Pane se syrjään'".
\par 24 Niin keittäjä otti reiden, ja mitä siihen kuului, ja pani sen Saulin eteen. Ja Samuel sanoi: "Katso, tässä pannaan sinun eteesi se, mikä on varattu; syö siitä. Sillä juuri täksi hetkeksi se säästettiin sinua varten, kun minä sanoin: 'Minä olen kutsunut kansan'." Niin Saul aterioi Samuelin kanssa sinä päivänä.
\par 25 Kun he sitten olivat tulleet uhrikukkulalta alas kaupunkiin, puhui hän Saulin kanssa katolla. Ja he nousivat varhain.
\par 26 Niin Samuel aamun sarastaessa kutsui Saulia katolta ja sanoi: "Nouse, minä lähden saattamaan sinua". Saul nousi, ja hän ja Samuel menivät yhdessä ulos.
\par 27 Kun he tulivat alas kaupungin laitaan, sanoi Samuel Saulille: "Sano palvelijalle, että hän menee meidän edellämme" - ja tämä meni edelle - "mutta pysähdy sinä hetkeksi tähän, niin minä julistan sinulle Jumalan sanan".

\chapter{10}

\par 1 Silloin Samuel otti öljyastian, vuodatti öljyä hänen päähänsä, suuteli häntä ja sanoi: "Katso, Herra on voidellut sinut perintöosansa ruhtinaaksi.
\par 2 Kun sinä tänä päivänä lähdet minun luotani, kohtaat sinä Raakelin haudan luona Selsahissa Benjaminin rajalla kaksi miestä; ne sanovat sinulle: 'Aasintammat, joita olet lähtenyt etsimään, ovat löytyneet; katso, isäsi on heittänyt mielestään aasintammat, kun on levoton teidän tähtenne ja sanoo: Mitä minä voisin tehdä poikani avuksi?'
\par 3 Ja kun menet siitä edemmäksi ja tulet Taaborin tammelle, tulee siellä sinua vastaan kolme miestä menossa Jumalan eteen Beeteliin. Yksi kantaa kolmea vohlaa, toinen kantaa kolmea leipäkakkua, ja kolmas kantaa viinileiliä.
\par 4 Ne tervehtivät sinua ja antavat sinulle kaksi leipää; ota ne heiltä.
\par 5 Senjälkeen sinä tulet Jumalan Gibeaan, jossa filistealaisten maaherrat ovat. Ja tullessasi sinne kaupunkiin sinä kohtaat joukon profeettoja, jotka tulevat alas uhrikukkulalta hurmoksissaan, harppu, vaskirumpu, huilu ja kannel edellänsä.
\par 6 Ja Herran henki tulee sinuun, ja sinäkin joudut hurmoksiin niinkuin hekin; ja sinä muutut toiseksi mieheksi.
\par 7 Ja kun nämä ennusmerkit käyvät toteen, niin tee, mikä tehtäväksesi tulee, sillä Jumala on sinun kanssasi.
\par 8 Mene sitten minun edelläni Gilgaliin, niin minä tulen sinne sinun luoksesi uhraamaan polttouhreja ja yhteysuhreja; odota seitsemän päivää, kunnes minä tulen sinun luoksesi ja ilmoitan sinulle, mitä sinun on tehtävä."
\par 9 Ja kun hän käänsi selkänsä lähteäksensä Samuelin luota, muutti Jumala hänen sydämensä; ja kaikki nämä ennusmerkit kävivät sinä päivänä toteen.
\par 10 Ja kun he tulivat sinne, Gibeaan, niin katso, joukko profeettoja tuli häntä vastaan. Silloin Jumalan henki tuli häneen, ja hänkin joutui hurmoksiin heidän keskellänsä.
\par 11 Kun kaikki, jotka ennestään tunsivat hänet, näkivät hänet hurmoksissa niinkuin profeetatkin, sanoivat he toisillensa: "Mikä Kiisin pojalle on tullut? Onko Saulkin profeettain joukossa?"
\par 12 Mutta eräs sikäläisistä miehistä vastasi ja sanoi: "Kuka sitten heidän isänsä on?" - Niin tuli sananlaskuksi: "Onko Saulkin profeettain joukossa?"
\par 13 Päästyään hurmoksista hän meni uhrikukkulalle.
\par 14 Ja Saulin setä kysyi häneltä ja hänen palvelijaltansa: "Missä te olette käyneet?" Hän vastasi: "Aasintammoja etsimässä. Mutta kun emme niitä missään nähneet, menimme Samuelin tykö."
\par 15 Silloin Saulin setä sanoi: "Kerro minulle, mitä Samuel teille sanoi".
\par 16 Saul vastasi sedällensä: "Hän ilmoitti meille aasintammain löytyneen". Mutta mitä Samuel oli sanonut kuninkuudesta, sitä hän ei hänelle kertonut.
\par 17 Sitten Samuel kutsui kansan koolle Herran eteen Mispaan.
\par 18 Hän sanoi israelilaisille: "Näin sanoo Herra, Israelin Jumala: Minä johdatin Israelin Egyptistä, ja minä vapautin teidät egyptiläisten käsistä sekä kaikkien niiden valtakuntien käsistä, jotka teitä sortivat.
\par 19 Mutta nyt te olette pitäneet halpana Jumalanne, joka on auttanut teidät kaikista onnettomuuksistanne ja ahdingoistanne, ja olette sanoneet hänelle: 'Aseta meille kuningas'. Asettukaa siis Herran eteen sukukunnittain ja suvuittain."
\par 20 Niin Samuel antoi kaikkien Israelin sukukuntien astua esiin; ja arpa osui Benjaminin sukukuntaan.
\par 21 Kun hän sitten antoi Benjaminin sukukunnan astua esiin suvuittain, osui arpa Matrin sukuun; sen jälkeen osui arpa Sauliin, Kiisin poikaan. Mutta kun häntä etsittiin, ei häntä löydetty.
\par 22 Silloin he kysyivät vielä kerran Herralta: "Onko ketään muuta vielä tullut tänne?" Herra vastasi: "Katso, hän on piiloutunut kuormastoon".
\par 23 Niin he juoksivat ja toivat hänet sieltä, ja kun hän asettui kansan keskelle, niin hän oli päätänsä pitempi kaikkea kansaa.
\par 24 Ja Samuel sanoi kaikelle kansalle: "Näettekö, kenen Herra on valinnut; sillä ei ole hänen vertaistansa koko kansan joukossa". Niin koko kansa riemuitsi huutaen: "Eläköön kuningas!"
\par 25 Ja Samuel julisti kansalle kuninkuuden oikeudet, kirjoitti ne kirjaan ja asetti sen Herran eteen. Sitten Samuel päästi kaiken kansan menemään, kunkin kotiinsa.
\par 26 Ja myöskin Saul meni kotiinsa Gibeaan, ja hänen kanssaan meni sotaväki, ne, joiden sydämiä Jumala oli koskettanut.
\par 27 Mutta kelvottomat miehet sanoivat: "Mitä apua meille tästä on?" Ja nämä halveksivat häntä eivätkä tuoneet hänelle mitään lahjoja; mutta hän ei virkkanut mitään.

\chapter{11}

\par 1 Ammonilainen Naahas lähti piirittämään Gileadin Jaabesta. Niin kaikki Jaabeksen miehet sanoivat Naahaalle: "Tee liitto meidän kanssamme, niin me palvelemme sinua".
\par 2 Mutta ammonilainen Naahas vastasi heille: "Sillä ehdolla minä teen liiton teidän kanssanne, että saan puhkaista jokaiselta teiltä oikean silmän; niin minä häpäisen koko Israelin".
\par 3 Jaabeksen vanhimmat sanoivat hänelle: "Jätä meidät rauhaan seitsemäksi päiväksi, niin me lähetämme sanansaattajia kaikkialle Israelin alueelle; jollei kukaan meitä auta, niin me antaudumme sinulle".
\par 4 Niin sanansaattajat tulivat Saulin Gibeaan ja puhuivat asian kansalle. Silloin kaikki kansa korotti äänensä ja itki.
\par 5 Ja katso, Saul tuli käyden härkäin jäljessä pellolta. Ja Saul kysyi: "Mikä kansalla on, kun he itkevät?" Niin he kertoivat hänelle Jaabeksen miesten asian.
\par 6 Kun Saul kuuli tämän asian, tuli Jumalan henki häneen, ja hän vihastui kovin.
\par 7 Ja hän otti härkäparin, paloitteli härät ja lähetti kappaleet kaikkialle Israelin alueelle sanansaattajain mukana ja käski sanoa: "Näin tehdään jokaisen härjille, joka ei seuraa Saulia ja Samuelia". Ja Herran kauhu valtasi kansan, niin että he lähtivät niinkuin yksi mies.
\par 8 Ja hän piti katselmuksen heistä Besekissä; ja israelilaisia oli kolmesataa tuhatta ja Juudan miehiä kolmekymmentä tuhatta.
\par 9 Ja he sanoivat sanansaattajille, jotka olivat tulleet: "Sanokaa näin Gileadin Jaabeksen miehille: 'Huomenna, auringon ollessa polttavimmillaan, te saatte apua'". Niin sanansaattajat tulivat ja ilmoittivat sen Jaabeksen miehille; ja nämä ilostuivat.
\par 10 Niin Jaabeksen miehet sanoivat: "Huomenna me antaudumme teille, ja te saatte tehdä meille, mitä tahdotte".
\par 11 Seuraavana päivänä Saul jakoi kansan kolmeen joukkoon; ja he tunkeutuivat leiriin aamuvartion aikana ja voittivat ammonilaiset, surmaten heitä, kunnes päivä tuli palavimmilleen. Ja jäljelle jääneet hajaantuivat, niin ettei heistä kahta yhteen jäänyt.
\par 12 Silloin kansa sanoi Samuelille: "Ketkä ne olivat, jotka sanoivat: 'Saulko tulisi meidän kuninkaaksemme?' Antakaa tänne ne miehet, surmataksemme heidät."
\par 13 Mutta Saul sanoi: "Tänä päivänä ei ketään surmata, sillä tänä päivänä on Herra antanut voiton Israelille".
\par 14 Ja Samuel sanoi kansalle: "Tulkaa, menkäämme Gilgaliin ja uudistakaamme siellä kuninkuus".
\par 15 Niin kaikki kansa meni Gilgaliin ja teki Saulin kuninkaaksi siellä, Herran edessä Gilgalissa; ja he uhrasivat siellä yhteysuhreja Herran edessä. Ja Saul ja kaikki Israelin miehet iloitsivat siellä suuresti.

\chapter{12}

\par 1 Samuel sanoi kaikelle Israelille: "Katso, minä olen kuullut teidän ääntänne kaikessa, mitä te olette minulta pyytäneet; minä olen asettanut kuninkaan teitä hallitsemaan.
\par 2 Ja nyt on teidän kuninkaanne käyvä teidän edellänne, kun minä olen tullut vanhaksi ja harmaaksi - ovathan jo minun poikani teidän keskellänne. Mutta minä olen käynyt teidän edellänne nuoruudestani tähän päivään asti.
\par 3 Tässä minä olen; todistakaa minua vastaan Herran ja hänen voideltunsa edessä. Keneltä minä olen vienyt härän tai keneltä aasin? Kenelle minä olen tehnyt vääryyttä, kenelle väkivaltaa? Keneltä minä olen lahjuksia ottanut ummistaakseni silmäni hänen hyväksensä? Todistakaa, niin minä korvaan sen teille."
\par 4 He vastasivat: "Et ole tehnyt meille vääryyttä etkä väkivaltaa, et myös ole keneltäkään mitään ottanut".
\par 5 Silloin hän sanoi heille: "Herra on todistaja teitä vastaan, ja hänen voideltunsa on todistaja tänä päivänä, että te ette ole löytäneet mitään minun kädestäni". He vastasivat: "Hän on todistaja".
\par 6 Samuel sanoi kansalle: "Todistaja on Herra, joka kutsui Mooseksen ja Aaronin ja johdatti teidän isänne Egyptin maasta.
\par 7 Astukaa nyt esiin, minä haastan teidät tuomiolle Herran edessä kaikista Herran vanhurskaista teoista, jotka hän on teille ja teidän isillenne tehnyt.
\par 8 Kun Jaakob oli tullut Egyptiin, huusivat teidän isänne Herraa, ja Herra lähetti Mooseksen ja Aaronin viemään teidän isänne pois Egyptistä ja sijoittamaan heidät tähän paikkaan.
\par 9 Mutta he unhottivat Herran, Jumalansa. Silloin hän myi heidät Siiseran, Haasorin sotapäällikön, käsiin ja filistealaisten käsiin ja Mooabin kuninkaan käsiin, ja nämä sotivat heitä vastaan.
\par 10 Niin he huusivat Herraa ja sanoivat: 'Me olemme syntiä tehneet, kun hylkäsimme Herran ja palvelimme baaleja ja astarteja; mutta vapauta nyt meidät vihollistemme käsistä, niin me palvelemme sinua'.
\par 11 Ja Herra lähetti Jerubbaalin, Bedanin, Jeftan ja Samuelin ja vapautti teidät ympärillä asuvain vihollistenne käsistä, niin että saitte asua turvassa.
\par 12 Mutta kun te näitte Naahaan, ammonilaisten kuninkaan, hyökkäävän teitä vastaan, sanoitte te minulle: 'Ei, vaan kuningas hallitkoon meitä', vaikka Herra, teidän Jumalanne, on kuninkaanne.
\par 13 Tässä on nyt kuningas, jonka te valitsitte ja jota anoitte: katso, Herra on antanut teille kuninkaan.
\par 14 Jos te pelkäätte Herraa ja palvelette häntä, kuulette hänen ääntänsä ettekä niskoittele hänen käskyjänsä vastaan, niin te seuraatte Herraa, Jumalaanne, sekä te että kuningas, joka teitä hallitsee.
\par 15 Mutta jollette kuule Herran ääntä, vaan niskoittelette hänen käskyjänsä vastaan, niin Herran käsi on oleva teitä vastaan, niinkuin oli myös teidän isiänne vastaan.
\par 16 Astukaa nyt esiin ja katsokaa tätä suurta tekoa, jonka Herra tekee teidän silmienne edessä.
\par 17 Nythän on nisunleikkuun aika; mutta minä huudan Herraa, että hän antaa ukkosenjylinän ja sateen, että ymmärtäisitte ja näkisitte, kuinka paha Herran silmissä on se, minkä olette tehneet anoessanne itsellenne kuningasta."
\par 18 Ja Samuel huusi Herraa, ja Herra antoi ukkosenjylinän ja sateen sinä päivänä. Ja kaikki kansa pelkäsi suuresti Herraa ja Samuelia.
\par 19 Ja kaikki kansa sanoi Samuelille: "Rukoile palvelijaisi puolesta Herraa, Jumalaasi, ettemme kuolisi; sillä me olemme tehneet kaikkien muiden syntiemme lisäksi senkin pahan, että anoimme itsellemme kuningasta".
\par 20 Samuel sanoi kansalle: "Älkää peljätkö. Te olette tosin tehneet kaiken tämän pahan, mutta älkää kuitenkaan poiketko pois seuraamasta Herraa, vaan palvelkaa Herraa kaikesta sydämestänne.
\par 21 Älkää poiketko seuraamaan turhia jumalia, joista ei ole hyötyä eikä apua, sillä turhia ne ovat.
\par 22 Suuren nimensä tähden Herra ei hylkää kansaansa, koska Herra on tahtonut tehdä teidät omaksi kansaksensa.
\par 23 Ja pois se minusta, että tekisin sen synnin Herraa vastaan, että lakkaisin rukoilemasta teidän puolestanne ja opettamasta teille hyvää ja oikeata tietä.
\par 24 Peljätkää vain Herraa ja palvelkaa häntä uskollisesti kaikesta sydämestänne. Sillä katsokaa, kuinka suuria hän on teille tehnyt.
\par 25 Mutta jos teette pahaa, niin hukutte, sekä te itse että teidän kuninkaanne."

\chapter{13}

\par 1 Saul oli ollut vuoden kuninkaana ja hallitsi Israelia toista vuotta,
\par 2 kun Saul valitsi itsellensä kolmetuhatta miestä Israelista, ja näistä oli kaksituhatta Saulin kanssa Mikmaassa ja Beetelin vuoristossa, ja tuhat Joonatanin kanssa Benjaminin Gibeassa. Mutta muun väen hän oli päästänyt menemään, kunkin majallensa.
\par 3 Mutta Joonatan, Saulin poika, löi kuoliaaksi filistealaisten maaherran, joka asui Gebassa, ja filistealaiset saivat sen kuulla. Silloin Saul puhallutti pasunaan kaikkialla maassa ja käski sanoa: "Hebrealaiset kuulkoot tämän".
\par 4 Ja koko Israel kuuli sen sanoman, että Saul oli lyönyt kuoliaaksi filistealaisten maaherran ja että Israel oli joutunut filistealaisten vihoihin. Niin kansa kutsuttiin koolle Gilgaliin, seuraamaan Saulia.
\par 5 Sillä filistealaiset olivat kokoontuneet sotimaan Israelia vastaan: kolmetkymmenet tuhannet sotavaunut, kuusituhatta ratsumiestä ja muuta väkeä niin paljon kuin hiekkaa meren rannalla; ja he tulivat ylös ja leiriytyivät Mikmaaseen, vastapäätä Beet-Aavenia.
\par 6 Kun Israelin miehet näkivät joutuneensa hätään ja kansaa ahdistettavan, piiloutuivat he luoliin, onkaloihin, kallionrotkoihin, hautoihin ja kaivoihin.
\par 7 Hebrealaisia meni myös Jordanin yli Gaadiin ja Gileadin maahan. Mutta Saul oli vielä Gilgalissa, ja kaikki sotaväki seurasi häntä peloissaan.
\par 8 Kun hän oli odottanut seitsemän päivää, sen ajan, jonka Samuel oli määrännyt, eikä Samuel tullutkaan Gilgaliin, alkoi kansa hajaantua pois hänen luotaan.
\par 9 Silloin Saul sanoi: "Tuokaa minulle polttouhri ja yhteysuhri". Ja hän uhrasi polttouhrin.
\par 10 Mutta juuri kun hän oli saanut polttouhrin uhratuksi, niin katso, Samuel tuli. Ja Saul meni häntä vastaan tervehtimään häntä.
\par 11 Mutta Samuel sanoi: "Mitä olet tehnyt?" Saul vastasi: "Kun näin, että kansa hajaantui pois minun luotani etkä sinä tullut määrättynä aikana, vaikka filistealaiset olivat kokoontuneet Mikmaaseen,
\par 12 niin minä ajattelin: nyt filistealaiset hyökkäävät minua vastaan alas Gilgaliin, enkä minä ole etsinyt Herran mielisuosiota; ja minä rohkaisin itseni ja uhrasin polttouhrin".
\par 13 Samuel sanoi Saulille: "Sinä olet tehnyt tyhmästi. Et ole noudattanut Herran, Jumalasi, käskyä, jonka hän antoi sinulle; muutoin olisi Herra vahvistanut sinun kuninkuutesi Israelissa ikuisiksi ajoiksi.
\par 14 Mutta nyt sinun kuninkuutesi ei ole pysyvä. Herra on etsinyt itselleen mielensä mukaisen miehen, ja hänet on Herra määrännyt kansansa ruhtinaaksi, koska sinä et noudattanut käskyä, minkä Herra sinulle antoi."
\par 15 Sitten Samuel nousi ja meni Gilgalista Benjaminin Gibeaan. Ja Saul piti mukanansa olevan väen katselmuksen: noin kuusisataa miestä.
\par 16 Ja Saul ja hänen poikansa Joonatan ynnä väki, joka oli heidän kanssansa, jäivät Benjaminin Gebaan, mutta filistealaiset olivat leiriytyneet Mikmaaseen.
\par 17 Ja filistealaisten leiristä lähti ryöstöosasto kolmena joukkona: yksi joukko kääntyi Ofran tielle Suualin maahan päin,
\par 18 toinen joukko kääntyi Beet-Hooronin tielle, ja kolmas joukko kääntyi sille tielle, joka vie Seboimin laakson yli kohoavalle alueelle, erämaahan päin.
\par 19 Mutta ei yhtään seppää ollut löydettävissä koko Israelin maasta, sillä filistealaiset ajattelivat, että hebrealaiset muutoin teettäisivät miekkoja tai keihäitä.
\par 20 Ja koko Israelin, joka miehen, oli mentävä filistealaisten luo teroituttamaan vannastansa, kuokkaansa, kirvestänsä tai muuta teräkaluansa,
\par 21 kun vannasten, kuokkien, tadikkojen tai kirvesten terät olivat tylsyneet, tahi kun häränpistimen tutkain oli oikaistava.
\par 22 Niinpä ei taistelupäivänä ollut yhtään miekkaa eikä keihästä kenelläkään siitä väestä, joka oli Saulin ja Joonatanin kanssa; ainoastaan Saulilla ja hänen pojallansa Joonatanilla oli.
\par 23 Mutta filistealaisten vartiosto lähti Mikmaan solatielle.

\chapter{14}

\par 1 Ja tapahtui eräänä päivänä, että Joonatan, Saulin poika, sanoi palvelijalle, joka kantoi hänen aseitansa: "Tule, menkäämme lähelle filistealaisten vartiostoa, joka on tuolla toisella puolella". Mutta hän ei ilmoittanut sitä isällensä.
\par 2 Saul istui silloin granaattiomenapuun juurella Migronissa Gibean rajalla, ja väkeä oli hänellä kanssansa noin kuusisataa miestä.
\par 3 Ja kasukankantajana oli Ahia, Ahitubin, Iikabodin veljen, poika, joka oli Piinehaan poika, joka Eelin poika, sen, joka oli ollut Herran pappina Siilossa. Mutta kansa ei tiennyt, että Joonatan oli lähtenyt.
\par 4 Mutta kummallakin puolen solatietä, jota Joonatan koetti mennä lähelle filistealaisten vartiostoa, oli jyrkkä kallionkieleke; toisen nimi on Booses, toisen Sene.
\par 5 Toinen kallionkieleke on kuin pylväs, pohjoisen puolella, Mikmaaseen päin, toinen on etelän puolella, Gebaan päin.
\par 6 Ja Joonatan sanoi palvelijalle, joka kantoi hänen aseitansa: "Tule, menkäämme lähelle noiden ympärileikkaamattomain vartiostoa; ehkä Herra on tekevä jotakin meidän puolestamme, sillä ei mikään estä Herraa antamasta voittoa harvojen kautta yhtä hyvin kuin monien".
\par 7 Hänen aseenkantajansa vastasi hänelle: "Tee, mitä mielessäsi on. Lähde, minä seuraan sinua mielesi mukaan."
\par 8 Niin Joonatan sanoi: "Katso, me menemme tuonne, lähelle noita miehiä, ja näyttäydymme heille.
\par 9 Jos he sanovat meille näin: 'Olkaa hiljaa, kunnes me tulemme teidän luoksenne', niin me seisomme alallamme emmekä lähde nousemaan heidän luoksensa.
\par 10 Mutta jos he sanovat näin: 'Nouskaa tänne meidän luoksemme', niin me nousemme, sillä silloin Herra on antanut heidät meidän käsiimme; tämä olkoon meillä merkkinä."
\par 11 Kun he sitten molemmat tulivat filistealaisten vartioston näkyviin, sanoivat filistealaiset: "Katso, hebrealaiset tulevat esiin koloistansa, joihin ovat piiloutuneet".
\par 12 Ja vartioston miehet huusivat Joonatanille ja hänen aseenkantajalleen ja sanoivat: "Nouskaa tänne meidän luoksemme, niin me ilmoitamme teille jotakin". Silloin sanoi Joonatan aseenkantajallensa: "Nouse minun perässäni, sillä Herra on antanut heidät Israelin käsiin".
\par 13 Ja Joonatan kiipesi käsin ja jaloin ylöspäin, ja aseenkantaja hänen perässään. Ja he kaatuivat Joonatanin edessä, ja aseenkantaja löi heidät kuoliaaksi hänen jäljessänsä.
\par 14 Tässä ensimmäisessä kahakassa surmasivat Joonatan ja hänen aseenkantajansa noin kaksikymmentä miestä, noin puolen auranalan suuruisella maa-alalla.
\par 15 Silloin syntyi kauhu leirissä ja sen ulkopuolella, koko sotaväessä; myös vartiosto ja ryöstöosasto joutuivat kauhun valtaan. Maakin järisi, ja niin syntyi Jumalan kauhu.
\par 16 Ja Saulin tähystäjät Benjaminin Gibeassa huomasivat lauman hajoavan ja menevän sinne tänne.
\par 17 Niin Saul sanoi väelle, joka oli hänen kanssansa: "Pitäkää katselmus ja katsokaa, kuka on mennyt pois meidän luotamme". Ja kun katselmus pidettiin, huomattiin, etteivät Joonatan ja hänen aseenkantajansa olleet läsnä.
\par 18 Ja Saul sanoi Ahialle: "Tuo tänne Jumalan arkki". Sillä Jumalan arkki oli siihen aikaan israelilaisten hallussa.
\par 19 Mutta Saulin vielä puhutellessa pappia kävi meteli filistealaisten leirissä yhä suuremmaksi. Niin Saul sanoi papille: "Jätä sikseen".
\par 20 Silloin Saul kutsutti koolle kaiken väen, joka oli hänen kanssaan, ja he tulivat taistelupaikalle, ja katso, toisen miekka oli toista vastaan, ja oli mitä suurin hämminki.
\par 21 Myös ne hebrealaiset, jotka ennestään olivat filistealaisten vallassa ja jotka olivat tulleet heidän kanssaan ja olivat leirissä, menivät niiden israelilaisten puolelle, jotka olivat Saulin ja Joonatanin kanssa.
\par 22 Ja kun ne Israelin miehet, jotka olivat piiloutuneet Efraimin vuoristoon, kuulivat filistealaisten pakenevan, yhtyivät hekin kaikki ajamaan heitä takaa taistelussa.
\par 23 Niin Herra antoi Israelille voiton sinä päivänä, ja taistelu levisi Beet-Aavenin toiselle puolelle.
\par 24 Israelin miehet olivat sinä päivänä ylen rasitetut, mutta Saul vannotti kansan ja sanoi: "Kirottu olkoon se mies, joka syö mitään ennen iltaa ja ennen kuin minä olen kostanut vihollisilleni". Ja koko kansa oli ruokaa maistamatta.
\par 25 Ja kun he kaikki tulivat metsään, oli maassa hunajata.
\par 26 Kun kansa tuli kennokakkujen ääreen, niin katso, niistä vuoti hunajata, mutta ei kukaan vienyt kättään suuhun, sillä kansa pelkäsi valaa.
\par 27 Mutta Joonatan ei ollut kuulemassa, kun hänen isänsä vannotti kansan; niin hän ojensi sauvan, joka oli hänen kädessään, pisti sen kärjen kennokakkuun ja vei sitten kätensä suuhunsa, ja silloin hänen silmänsä kirkastuivat.
\par 28 Mutta eräs mies väestä puhkesi puhumaan ja sanoi: "Sinun isäsi vannotti väen ja sanoi: 'Kirottu olkoon se mies, joka tänä päivänä syö mitään'". Ja väki oli näännyksissä.
\par 29 Joonatan vastasi: "Isäni on syössyt maan onnettomuuteen. Katsokaa, kuinka minun silmäni kirkastuivat, kun vähän maistoin tätä hunajata.
\par 30 Jos myös väki olisi tänä päivänä saanut syödä vihollisiltaan ottamaansa saalista, niin eiköhän filistealaisten tappio olisi ollut vieläkin suurempi?"
\par 31 Ja he voittivat sinä päivänä filistealaiset ja ajoivat heitä takaa Mikmaasta Aijaloniin asti; ja väki oli kovin näännyksissä.
\par 32 Sentähden väki syöksyi saaliin kimppuun ja otti lampaita, raavaita ja vasikoita ja teurasti niitä paljaan maan päällä; ja väki söi lihan verinensä.
\par 33 Niin Saulille ilmoitettiin tämä: "Katso, väki tekee syntiä Herraa vastaan, kun syö lihaa verinensä". Hän sanoi: "Te olette menetelleet uskottomasti; vierittäkää nyt tänne minun eteeni suuri kivi".
\par 34 Ja Saul sanoi: "Menkää väen sekaan ja sanokaa heille: 'Tuokaa jokainen härkänne ja lampaanne minun luokseni ja teurastakaa ne täällä. Sitten syökää; älkääkä tehkö syntiä Herraa vastaan syömällä lihaa verinensä.'" Niin väestä jokainen silloin yöllä toi omin käsin härkänsä ja teurasti sen siellä.
\par 35 Ja Saul rakensi alttarin Herralle; tämä oli ensimmäinen alttari, jonka hän Herralle rakensi.
\par 36 Ja Saul sanoi: "Lähtekäämme tänä yönä filistealaisten jälkeen ja ryöstäkäämme heitä, kunnes aamu valkenee, ja älkäämme jättäkö heistä jäljelle ainoatakaan". He vastasivat: "Tee kaikki, mitä hyväksi näet". Mutta pappi sanoi: "Astukaamme tänne Jumalan eteen".
\par 37 Silloin Saul kysyi Jumalalta: "Lähdenkö minä filistealaisten jälkeen? Annatko sinä heidät Israelin käsiin?" Mutta hän ei vastannut hänelle sinä päivänä.
\par 38 Niin Saul sanoi: "Tulkaa tänne, kaikki kansan päämiehet, saadaksenne tietää ja nähdä, mikä synti nyt on tähän syynä.
\par 39 Sillä niin totta kuin Herra elää, hän, joka on antanut Israelille voiton: vaikka syy olisi minun poikani Joonatanin, niin hänen on kuoltava." Mutta ei kukaan koko kansasta vastannut hänelle.
\par 40 Silloin hän sanoi koko Israelille: "Olkaa te toisella puolella, niin minä ja minun poikani Joonatan olemme toisella puolella". Kansa vastasi Saulille: "Tee, niinkuin hyväksi näet".
\par 41 Ja Saul sanoi Herralle, Israelin Jumalalle: "Anna oikea arpa". Niin arpa osui Joonataniin ja Sauliin, ja kansa pääsi vapaaksi.
\par 42 Saul sanoi: "Heittäkää arpaa minun ja minun poikani Joonatanin välillä". Niin arpa osui Joonataniin.
\par 43 Ja Saul sanoi Joonatanille: "Ilmaise minulle, mitä olet tehnyt". Joonatan ilmaisi sen hänelle ja sanoi: "Minä maistoin vähän hunajata sauvan kärjellä, joka oli kädessäni; katso, minä olen valmis kuolemaan".
\par 44 Ja Saul sanoi: "Jumala rangaiskoon minua nyt ja vasta: sinun on kuolemalla kuoltava, Joonatan".
\par 45 Mutta kansa sanoi Saulille: "Onko Joonatanin kuoltava, hänen, joka on hankkinut tämän suuren voiton Israelille? Pois se! Niin totta kuin Herra elää: ei saa hiuskarvakaan pudota maahan hänen päästänsä; sillä Jumalan avulla hän on sen tänä päivänä tehnyt." Näin kansa vapahti Joonatanin kuolemasta.
\par 46 Sitten Saul meni eikä ajanut filistealaisia takaa; ja filistealaiset menivät kotiinsa.
\par 47 Kun nyt Saul oli saanut kuninkuuden Israelissa, soti hän kaikkia vihollisiansa vastaan joka taholla: mooabilaisia, ammonilaisia, edomilaisia, Sooban kuninkaita ja filistealaisia vastaan; ja minne tahansa hän kääntyi, siellä hän rankaisi.
\par 48 Ja hän teki väkeviä tekoja ja voitti amalekilaiset ja vapautti Israelin sen ryöstäjäin käsistä.
\par 49 Saulin pojat olivat Joonatan, Jisvi ja Malkisua; ja hänen kahden tyttärensä nimet olivat: vanhemman nimi Meerab ja nuoremman nimi Miikal.
\par 50 Ja Saulin vaimon nimi oli Ahinoam, Ahimaan tytär. Ja hänen sotapäällikkönsä nimi oli Abner, Neerin, Saulin sedän, poika.
\par 51 Sillä Kiis, Saulin isä, ja Neer, Abnerin isä, olivat Abielin poikia.
\par 52 Mutta filistealaisia vastaan käytiin kiivaasti sotaa, niin kauan kuin Saul eli. Ja kenen vain Saul tapasi urhoollisen ja sotakuntoisen miehen, sen hän otti luoksensa.

\chapter{15}

\par 1 Mutta Samuel sanoi Saulille: "Minut Herra lähetti voitelemaan sinut kansansa Israelin kuninkaaksi. Kuule siis Herran ääntä.
\par 2 Näin sanoo Herra Sebaot: Minä kostan Amalekille sen, mitä hän teki Israelille asettumalla hänen tielleen, kun hän tuli Egyptistä.
\par 3 Mene siis ja voita amalekilaiset, ja vihkikää tuhon omaksi kaikki, mitä heillä on; äläkä säästä heitä, vaan surmaa miehet ja naiset, lapset ja imeväiset, raavaat ja lampaat, kamelit ja aasit."
\par 4 Saul kuulutti kansan kokoon ja piti heistä katselmuksen Telaimissa: kaksisataa tuhatta jalkamiestä ynnä kymmenentuhatta Juudan miestä.
\par 5 Kun Saul tuli lähelle amalekilaisten kaupunkia, asettui hän laaksoon väijyksiin.
\par 6 Ja Saul sanoi keeniläisille: "Tulkaa, erotkaa ja siirtykää pois amalekilaisten joukosta, etten minä hukuttaisi teitä heidän kanssansa; sillä te teitte laupeuden kaikille israelilaisille, kun he tulivat Egyptistä". Niin keeniläiset erosivat amalekilaisista.
\par 7 Ja Saul voitti amalekilaiset ja ajoi heitä takaa Havilasta Suuriin päin, joka on Egyptistä itään päin.
\par 8 Ja hän otti Agagin, Amalekin kuninkaan, elävänä vangiksi, mutta kaiken kansan hän vihki tuhon omaksi miekan terällä.
\par 9 Mutta Saul ja väki säästivät Agagin sekä parhaimmat lampaat, lihavimmat raavaat ja karitsat, kaiken, mikä oli arvokasta; sitä he eivät tahtoneet vihkiä tuhon omaksi. Mutta kaiken karjan, mikä oli halpaa ja arvotonta, he vihkivät tuhon omaksi.
\par 10 Silloin Samuelille tuli tämä Herran sana:
\par 11 "Minä kadun, että tein Saulin kuninkaaksi, sillä hän on kääntynyt pois minusta eikä ole täyttänyt minun käskyjäni". Samuel pahastui, ja hän huusi Herraa kaiken sen yön.
\par 12 Varhain seuraavana aamuna Samuel lähti tapaamaan Saulia. Mutta Samuelille ilmoitettiin: "Saul on mennyt Karmeliin ja pystyttänyt sinne itsellensä muistomerkin; sitten hän on palannut sieltä ja mennyt alas Gilgaliin".
\par 13 Kun Samuel tuli Saulin luo, sanoi Saul hänelle: "Herra siunatkoon sinua! Minä olen täyttänyt Herran käskyn."
\par 14 Mutta Samuel sanoi: "Mitä tämä lammasten määkiminen sitten on, joka kuuluu minun korviini, ja tämä raavaiden ammuminen, jonka kuulen?"
\par 15 Saul vastasi: "Amalekilaisilta ne on tuotu; sillä kansa säästi parhaat lampaat ja raavaat uhrataksensa ne Herralle, sinun Jumalallesi; mutta muut me olemme vihkineet tuhon omiksi".
\par 16 Silloin Samuel sanoi Saulille: "Lopeta; minä ilmoitan sinulle, mitä Herra on minulle tänä yönä puhunut". Hän sanoi hänelle: "Puhu".
\par 17 Samuel sanoi: "Sinä olet Israelin sukukuntien päämies, vaikka oletkin vähäinen omissa silmissäsi: Herra on voidellut sinut Israelin kuninkaaksi.
\par 18 Ja Herra lähetti sinut matkaan ja sanoi: 'Mene ja vihi tuhon omiksi nämä syntiset, amalekilaiset, ja sodi heitä vastaan, kunnes olet tehnyt heistä lopun'.
\par 19 Miksi et kuullut Herran ääntä, vaan syöksyit saaliin kimppuun ja teit sitä, mikä on pahaa Herran silmissä?"
\par 20 Saul vastasi Samuelille: "Minähän olen kuullut Herran ääntä ja tehnyt sen matkan, jolle Herra minut lähetti. Minä olen tuonut tänne Agagin, Amalekin kuninkaan, ja vihkinyt tuhon omiksi amalekilaiset.
\par 21 Mutta väki on ottanut saaliista lampaita ja raavaita, parhaan osan siitä, mikä oli vihitty tuhon omaksi, uhrataksensa ne Herralle, sinun Jumalallesi, Gilgalissa."
\par 22 Silloin Samuel sanoi: "Haluaako Herra polttouhreja ja teurasuhreja yhtä hyvin kuin kuuliaisuutta Herran äänelle? Katso, kuuliaisuus on parempi kuin uhri ja tottelevaisuus parempi kuin oinasten rasva.
\par 23 Sillä tottelemattomuus on taikuuden syntiä, ja niskoittelu on valhetta ja kuin kotijumalain palvelusta. Koska sinä olet hyljännyt Herran sanan, on myös hän hyljännyt sinut, etkä sinä enää saa olla kuninkaana."
\par 24 Saul sanoi Samuelille: "Minä olen tehnyt syntiä, kun olen rikkonut Herran käskyn ja sinun sanasi; sillä minä pelkäsin kansaa ja kuulin heidän ääntänsä.
\par 25 Anna nyt minulle syntini anteeksi ja palaja minun kanssani, niin minä kumartaen rukoilen Herraa."
\par 26 Samuel sanoi Saulille: "Minä en palaja sinun kanssasi; sillä sinä olet hyljännyt Herran sanan, ja Herra on myös hyljännyt sinut, niin että et enää saa olla Israelin kuninkaana".
\par 27 Kun sitten Samuel kääntyi menemään pois, tarttui Saul hänen viittansa liepeeseen, niin että se repesi.
\par 28 Ja Samuel sanoi hänelle: "Herra on tänä päivänä reväissyt Israelin kuninkuuden sinulta ja antanut sen toiselle, joka on sinua parempi.
\par 29 Ja hän, joka on Israelin kunnia, ei valhettele eikä kadu; sillä hän ei ole ihminen, että hän katuisi."
\par 30 Hän vastasi: "Minä olen syntiä tehnyt; mutta osoita minulle kuitenkin se kunnia kansani vanhimpien edessä ja Israelin edessä, että palajat minun kanssani, niin minä kumartaen rukoilisin Herraa, sinun Jumalaasi".
\par 31 Niin Samuel palasi Saulin jäljessä; ja Saul kumartaen rukoili Herraa.
\par 32 Ja Samuel sanoi: "Tuokaa minun eteeni Agag, Amalekin kuningas". Ja Agag tuli iloisesti hänen eteensä; ja Agag sanoi: "Totisesti on katkera kuolema väistynyt pois".
\par 33 Mutta Samuel sanoi: "Niinkuin sinun miekkasi on tehnyt vaimoja lapsettomiksi, niin on myös sinun äitisi tuleva lapsettomaksi ennen muita vaimoja". Ja Samuel hakkasi Agagin kappaleiksi Herran edessä Gilgalissa.
\par 34 Sitten Samuel lähti Raamaan, mutta Saul meni kotiinsa Saulin Gibeaan.
\par 35 Eikä Samuel enää nähnyt Saulia kuolinpäiväänsä asti, ja Samuel suri Saulia. Mutta Herra katui, että oli tehnyt Saulin Israelin kuninkaaksi.

\chapter{16}

\par 1 Ja Herra sanoi Samuelille: "Kuinka kauan sinä suret Saulia? Minähän olen hyljännyt hänet, niin ettei hän enää saa olla Israelin kuninkaana. Täytä sarvesi öljyllä ja mene; minä lähetän sinut beetlehemiläisen Iisain luo, sillä hänen pojistansa minä olen katsonut itselleni kuninkaan."
\par 2 Mutta Samuel sanoi: "Kuinka minä voisin mennä sinne? Jos Saul saa sen kuulla, tappaa hän minut." Herra vastasi: "Ota mukaasi vasikka ja sano: 'Minä tulin uhraamaan Herralle'.
\par 3 Kutsu sitten Iisai uhrille, niin minä ilmoitan sinulle, mitä sinun on tehtävä; ja voitele minulle se, jonka minä sinulle osoitan."
\par 4 Samuel teki, niinkuin Herra oli puhunut. Mutta kun hän tuli Beetlehemiin, tulivat kaupungin vanhimmat vavisten häntä vastaan ja kysyivät: "Tietääkö tulosi rauhaa?"
\par 5 Hän vastasi: "Rauhaa. Minä olen tullut uhraamaan Herralle. Pyhittäytykää ja tulkaa minun kanssani uhrille." Ja hän pyhitti Iisain ja hänen poikansa ja kutsui heidät uhrille.
\par 6 Kun he sitten tulivat ja hän näki Eliabin, ajatteli hän: "Varmaan on nyt tässä Herran edessä hänen voideltunsa".
\par 7 Mutta Herra sanoi Samuelille: "Älä katso hänen näköänsä äläkä kookasta vartaloansa, sillä minä olen hänet hyljännyt. Sillä ei ole, niinkuin ihminen näkee: ihminen näkee ulkomuodon, mutta Herra näkee sydämen."
\par 8 Niin Iisai kutsui Abinadabin ja toi hänet Samuelin eteen. Mutta hän sanoi: "Ei ole Herra tätäkään valinnut".
\par 9 Silloin Iisai toi Samman esille. Mutta hän sanoi: "Ei ole Herra tätäkään valinnut".
\par 10 Niin toi Iisai seitsemän poikaansa Samuelin eteen. Mutta Samuel sanoi Iisaille: "Herra ei ole valinnut ketään näistä".
\par 11 Ja Samuel kysyi Iisailta: "Siinäkö olivat kaikki nuorukaiset?" Hän vastasi: "Vielä on nuorin jäljellä, mutta hän on kaitsemassa lampaita". Niin Samuel sanoi Iisaille: "Lähetä noutamaan hänet, sillä me emme istu ruualle, ennenkuin hän tulee tänne".
\par 12 Silloin tämä noudatti hänet; ja hän oli verevä, kaunissilmäinen ja sorea nähdä. Ja Herra sanoi: "Nouse ja voitele tämä, sillä hän se on".
\par 13 Niin Samuel otti öljysarven ja voiteli hänet hänen veljiensä keskellä. Ja Herran henki tuli Daavidiin, siitä päivästä ja yhä edelleen. Sitten Samuel nousi ja meni Raamaan.
\par 14 Mutta Herran henki poistui Saulista, ja Herran lähettämä paha henki vaivasi häntä.
\par 15 Niin Saulin palvelijat sanoivat hänelle: "Katso, Jumalan lähettämä paha henki vaivaa sinua.
\par 16 Käskeköön vain herramme, niin palvelijasi, jotka ovat edessäsi, etsivät miehen, joka taitaa soittaa kannelta, että hän, kun Jumalan lähettämä paha henki tulee sinuun, soittaisi sitä, ja sinun olisi parempi olla."
\par 17 Silloin Saul sanoi palvelijoillensa: "Katsokaa minulle mies, joka on taitava soittaja, ja tuokaa hänet minun luokseni".
\par 18 Niin eräs nuorista miehistä vastasi ja sanoi: "Katso, minä olen nähnyt beetlehemiläisellä Iisailla pojan, joka taitaa soittaa ja joka on kelpo mies ja sotilas sekä ymmärtäväinen puheiltaan ja komea mies, ja Herra on hänen kanssaan".
\par 19 Niin Saul lähetti sanansaattajat Iisain luo ja käski sanoa: "Lähetä minun luokseni poikasi Daavid, joka kaitsee lampaita".
\par 20 Ja Iisai otti aasin sekä leipää, viinileilin ja vohlan ja lähetti ne poikansa Daavidin mukana Saulille.
\par 21 Ja Daavid tuli Saulin luo ja rupesi palvelemaan häntä; ja hän tuli hänelle hyvin rakkaaksi ja pääsi hänen aseenkantajakseen.
\par 22 Ja Saul lähetti sanan Iisaille ja käski sanoa: "Anna Daavidin jäädä minua palvelemaan, sillä hän on saanut armon minun silmieni edessä".
\par 23 Niin usein kuin Jumalan lähettämä henki tuli Sauliin, otti Daavid kanteleen ja soitti sitä; silloin Saulin oli helpompi ja parempi olla, ja paha henki väistyi hänestä.

\chapter{17}

\par 1 Mutta filistealaiset kokosivat joukkonsa sotaan, ja he kokoontuivat lähelle Sookoa, joka on Juudassa. Ja he leiriytyivät Sookon ja Asekan välille, Efes-Dammimiin.
\par 2 Saul ja Israelin miehet olivat myös kokoontuneet ja leiriytyneet Tammilaaksoon; ja he asettuivat sotarintaan filistealaisia vastaan.
\par 3 Filistealaiset olivat vuorella, joka oli toisella puolella, ja israelilaiset olivat vuorella, joka oli toisella puolella, ja laakso oli heidän välillänsä.
\par 4 Niin filistealaisten joukoista tuli kaksintaistelija nimeltä Goljat, kotoisin Gatista. Hän oli kuuden kyynärän ja vaaksan pituinen.
\par 5 Hänellä oli vaskikypäri päässänsä ja suomuspanssari yllänsä, ja rintahaarniska, joka painoi viisituhatta sekeliä, oli vaskea.
\par 6 Ja hänellä oli säärissään vaskivarukset ja selässään vaskikeihäs.
\par 7 Hänen peitsensä varsi oli niinkuin kangastukki, ja hänen peitsensä kärki, joka painoi kuusisataa sekeliä, oli rautaa. Ja kilvenkantaja kävi hänen edellänsä.
\par 8 Hän astui esiin ja huusi Israelin taisteluriveille sanoen: "Miksi te lähditte sotaan ja asetuitte sotarintaan? Minä olen filistealainen, ja te olette Saulin palvelijoita; valitkaa joukostanne mies, joka tulee tänne minun luokseni.
\par 9 Jos hän kykenee taistelemaan minua vastaan ja surmaa minut, niin me olemme teidän palvelijanne; mutta jos minä voitan ja surmaan hänet, niin te olette meidän palvelijamme ja palvelette meitä."
\par 10 Ja filistealainen sanoi vielä: "Minä olen tänä päivänä häväissyt Israelin taistelurivit. Antakaa tänne mies, niin me taistelemme keskenämme."
\par 11 Kun Saul ja koko Israel kuuli nämä filistealaisen puheet, valtasi heidät kauhu, ja he pelkäsivät suuresti.
\par 12 Mutta Daavid oli sen efratilaisen miehen poika Juudan Beetlehemistä, jonka nimi oli Iisai, ja tällä oli kahdeksan poikaa; hän oli Saulin aikana jo vanha, iäkäs mies.
\par 13 Ja Iisain kolme vanhinta poikaa oli seurannut Saulia sotaan; ja näiden hänen kolmen sotaan lähteneen poikansa nimet olivat: esikoisen Eliab, toisen Abinadab ja kolmannen Samma.
\par 14 Daavid oli nuorin. Ja nuo kolme vanhinta olivat seuranneet Saulia.
\par 15 Mutta Daavid lähti tuontuostakin Saulin luota Beetlehemiin kaitsemaan isänsä lampaita.
\par 16 Ja filistealainen astui esiin haastaen taisteluun joka aamu ja ilta, neljänäkymmenenä päivänä.
\par 17 Ja Iisai sanoi pojallensa Daavidille: "Ota veljillesi eefa-mitta näitä paahdettuja jyviä ja nämä kymmenen leipää, ja vie ne kiiruusti leiriin veljillesi.
\par 18 Ja vie nämä kymmenen juustoa tuhannenpäämiehelle. Tiedustele veljiesi vointia ja hanki varmuus, että lähetys on heille saapunut.
\par 19 Saul ja he ynnä kaikki Israelin miehet ovat Tammilaaksossa sotimassa filistealaisia vastaan."
\par 20 Varhain seuraavana aamuna Daavid jätti lampaat vartijan haltuun, otti kantamuksensa ja lähti matkalle, niinkuin Iisai oli häntä käskenyt. Kun hän tuli leiriin, lähti sotaväki taistelurintaan ja nosti sotahuudon.
\par 21 Ja Israel ja filistealaiset asettuivat sotarintaan toisiansa vastaan.
\par 22 Niin Daavid heitti tavarat selästänsä kuormaston vartijalle ja riensi sotarintaan; ja sinne tultuaan hän tervehti veljiänsä.
\par 23 Kun hän puhutteli heitä, niin katso, kaksintaistelija, Goljat niminen filistealainen, kotoisin Gatista, tuli filistealaisten taisteluriveistä ja puhui niinkuin ennenkin; ja Daavid kuuli sen.
\par 24 Ja kun israelilaiset näkivät tuon miehen, pakenivat he häntä kaikki ja pelkäsivät suuresti.
\par 25 Ja Israelin miehet sanoivat: "Katsokaa miestä, joka tuolla tulee! Hän tulee häpäisemään Israelia. Mutta sen miehen, joka surmaa hänet, kuningas tekee hyvin rikkaaksi ja antaa hänelle tyttärensä ja tekee hänen isänsä perheen veroista vapaaksi Israelissa."
\par 26 Niin Daavid sanoi miehille, jotka seisoivat siinä hänen kanssaan: "Mitä saa se mies, joka surmaa tuon filistealaisen ja poistaa häväistyksen Israelista? Sillä mikä tuo ympärileikkaamaton filistealainen on häpäisemään elävän Jumalan taistelurivejä?"
\par 27 Väki sanoi hänelle saman, minkä ennenkin: sen ja sen saa se mies, joka surmaa hänet.
\par 28 Mutta kun hänen vanhin veljensä Eliab kuuli hänen puhuvan miesten kanssa, vihastui Eliab Daavidiin ja sanoi: "Miksi sinä olet tullut tänne, ja kenelle olet jättänyt sen pienen lammaslauman siellä erämaassa? Minä tunnen sinun julkeutesi ja pahan sisusi; sinä olet tullut tänne katsomaan sotaa."
\par 29 Daavid vastasi: "Mitä minä sitten olen tehnyt? Saaneehan tuon verran kysyä."
\par 30 Ja hän kääntyi hänen luotansa toisen puoleen ja kysyi niinkuin ennenkin; ja väki vastasi hänelle samoin kuin äsken.
\par 31 Mutta tuli tunnetuksi, mitä Daavid oli puhunut; ja se kerrottiin Saulille, ja tämä noudatti hänet luoksensa.
\par 32 Ja Daavid sanoi Saulille: "Älköön kenenkään mieli masentuko hänen tähtensä. Palvelijasi käy taistelemaan tuota filistealaista vastaan."
\par 33 Saul sanoi Daavidille: "Et sinä voi mennä tuota filistealaista vastaan etkä taistella hänen kanssaan, sillä sinä olet nuorukainen, mutta hän on sotilas nuoruudestansa saakka".
\par 34 Mutta Daavid sanoi Saulille: "Palvelijasi on ollut kaitsemassa isänsä lampaita. Jos leijona tai karhu tuli ja vei lampaan laumasta,
\par 35 niin minä hyökkäsin sen jälkeen, löin sen maahan ja tempasin saaliin sen suusta; ja jos se karkasi minua vastaan, niin minä tartuin sen partaan, löin sen maahan ja tapoin sen.
\par 36 Kun palvelijasi on lyönyt maahan sekä leijonan että karhun, niin tuolle ympärileikkaamattomalle filistealaiselle käy niinkuin niille; sillä hän on häväissyt elävän Jumalan taistelurivejä."
\par 37 Ja Daavid sanoi: "Herra, joka on pelastanut minut leijonan ja karhun kynsistä, pelastaa minut myös tämän filistealaisen käsistä". Silloin Saul sanoi Daavidille: "Mene; Herra olkoon sinun kanssasi".
\par 38 Ja Saul puki takkinsa Daavidin ylle ja pani vaskikypärin hänen päähänsä sekä panssarin hänen yllensä.
\par 39 Ja Daavid sitoi hänen miekkansa vyölleen takin päälle ja yritti käydä, sillä hän ei ollut koskaan sellaisia koettanut. Niin Daavid sanoi Saulille: "En minä voi näissä käydä, sillä en ole koskaan tällaisia koettanut". Ja Daavid riisui ne päältänsä.
\par 40 Ja hän otti sauvansa käteensä, valitsi purosta viisi sileätä kiveä, pani ne paimenlaukkuun, joka hänellä oli linkokivisäiliönä, otti lingon käteensä ja meni filistealaista vastaan.
\par 41 Ja filistealainen tuli yhä lähemmäksi Daavidia, ja kilvenkantaja kulki hänen edellänsä.
\par 42 Kun filistealainen katsahti ja näki Daavidin, halveksi hän häntä; sillä hän oli vielä nuorukainen, verevä ja kaunis näöltään.
\par 43 Niin filistealainen sanoi Daavidille: "Olenko minä koira, kun tulet sauva kädessä minua vastaan?" Ja filistealainen kiroili Daavidia vannoen jumaliensa kautta.
\par 44 Sitten filistealainen sanoi Daavidille: "Tule tänne minun luokseni, niin minä annan sinun lihasi taivaan linnuille ja metsän eläimille".
\par 45 Mutta Daavid sanoi filistealaiselle: "Sinä tulet minua vastaan miekan, peitsen ja keihään voimalla, mutta minä tulen sinua vastaan Herran Sebaotin nimeen, Israelin sotajoukon Jumalan, jota sinä olet häväissyt.
\par 46 Tänä päivänä Herra antaa sinut minun käsiini, ja minä surmaan sinut, katkaisen sinulta pään ja annan filistealaisten sotajoukon ruumiit tänä päivänä taivaan linnuille ja metsän pedoille; ja kaikki maat tulevat tietämään, että Israelilla on Jumala.
\par 47 Ja koko tämä suuri joukko tulee tietämään, ettei Herra anna voittoa miekan eikä keihään voimalla; sillä sota on Herran, ja hän antaa teidät meidän käsiimme."
\par 48 Kun filistealainen lähti tulemaan ja lähestyi Daavidia, niin Daavid juoksi nopeasti sotarintaan filistealaista vastaan.
\par 49 Ja Daavid pisti kätensä reppuunsa ja otti sieltä kiven, linkosi ja satutti filistealaista otsaan, niin että kivi upposi hänen otsaansa, ja hän kaatui maahan kasvoillensa.
\par 50 Niin sai Daavid voiton filistealaisesta lingolla ja kivellä ja löi filistealaisen kuoliaaksi, eikä Daavidilla ollut miekkaa kädessään.
\par 51 Sitten Daavid juoksi ja asettui filistealaisen ääreen, tarttui hänen miekkaansa, veti sen tupesta ja tappoi hänet ja löi sillä häneltä pään poikki. Kun filistealaiset näkivät, että heidän sankarinsa oli kuollut, pakenivat he.
\par 52 Mutta Israelin ja Juudan miehet nousivat, nostivat sotahuudon ja ajoivat filistealaisia takaa laakson suulle ja Ekronin porteille saakka; ja filistealaisia kaatui surmattuina Saaraimin tiellä, aina Gatiin ja Ekroniin saakka.
\par 53 Sitten israelilaiset palasivat ajamasta filistealaisia takaa ja ryöstivät heidän leirinsä.
\par 54 Ja Daavid otti filistealaisen pään ja vei sen Jerusalemiin, mutta hänen aseensa hän asetti majaansa.
\par 55 Kun Saul näki Daavidin menevän filistealaista vastaan, kysyi hän sotapäälliköltään Abnerilta: "Kenen poika tuo nuorukainen on, Abner?" Abner vastasi: "Niin totta kuin sinä, kuningas, elät, minä en sitä tiedä".
\par 56 Niin kuningas sanoi: "Kysele sitten, kenen poika tuo nuori mies on".
\par 57 Ja kun Daavid palasi takaisin surmattuaan filistealaisen, otti Abner hänet ja vei hänet Saulin eteen, ja hänellä oli kädessään filistealaisen pää.
\par 58 Niin Saul kysyi häneltä: "Kenen poika sinä olet, nuorukainen?" Daavid vastasi: "Palvelijasi beetlehemiläisen Iisain poika".

\chapter{18}

\par 1 Kun hän oli lakannut puhumasta Saulin kanssa, kiintyi Joonatan kaikesta sielustaan Daavidiin, ja Joonatan rakasti häntä niinkuin omaa sieluansa.
\par 2 Ja Saul otti hänet sinä päivänä luoksensa eikä sallinut hänen enää palata isänsä kotiin.
\par 3 Ja Joonatan teki liiton Daavidin kanssa, sillä hän rakasti häntä niinkuin omaa sieluansa.
\par 4 Ja Joonatan riisui viitan, joka hänellä oli yllänsä, ja antoi sen Daavidille, ja samoin takkinsa, vieläpä miekkansa, jousensa ja vyönsä.
\par 5 Daavid lähti, minne vain Saul hänet lähetti, ja hän menestyi. Niin Saul asetti hänet sotamiesten päälliköksi, ja se oli mieleen kaikelle kansalle ja myös Saulin palvelijoille.
\par 6 Ja kun he olivat tulossa, silloin kun Daavid palasi takaisin surmattuansa filistealaisen, menivät naiset kaikista Israelin kaupungeista laulaen ja karkeloiden kuningas Saulia vastaan, riemuiten, vaskirumpuja ja kymbaaleja lyöden.
\par 7 Ja karkeloivat naiset virittivät laulun ja sanoivat: "Saul voitti tuhat, mutta Daavid kymmenen tuhatta".
\par 8 Silloin Saul vihastui kovin, sillä hän pani sen puheen pahakseen, ja hän sanoi: "Daavidille he antavat kymmenen tuhatta, ja minulle he antavat tuhat; nyt puuttuu häneltä enää vain kuninkuus".
\par 9 Ja Saul katsoi karsain silmin Daavidia siitä päivästä alkaen.
\par 10 Seuraavana päivänä Jumalan lähettämä paha henki valtasi Saulin, niin että hän raivosi kotonansa; mutta Daavid soitteli, niinkuin muulloinkin joka päivä, ja Saulilla oli kädessänsä keihäs.
\par 11 Niin Saul heitti keihään ja ajatteli: "Minä keihästän Daavidin seinään". Mutta Daavid väisti häntä kaksi kertaa.
\par 12 Ja Saul pelkäsi Daavidia, koska Herra oli hänen kanssansa, mutta oli poistunut Saulista.
\par 13 Sentähden Saul toimitti hänet pois luotansa ja teki hänet tuhannenpäämieheksi; ja hän lähti ja tuli väen edellä.
\par 14 Ja Daavid menestyi kaikilla teillään, ja Herra oli hänen kanssansa.
\par 15 Kun Saul näki, että hänellä oli niin suuri menestys, rupesi hän häntä kammomaan.
\par 16 Mutta koko Israel ja Juuda rakasti Daavidia, koska hän lähti ja tuli heidän edellänsä.
\par 17 Ja Saul sanoi Daavidille: "Katso, vanhimman tyttäreni, Meerabin, minä annan sinulle vaimoksi; ole vain urhoollinen ja käy Herran sotia". Sillä Saul ajatteli: "Minun käteni älköön sattuko häneen, vaan sattukoon häneen filistealaisten käsi".
\par 18 Mutta Daavid vastasi Saulille: "Mikä minä olen, mikä on minun elämäni ja mikä on isäni suku Israelissa, että minä tulisin kuninkaan vävyksi?"
\par 19 Kun aika tuli, että Saulin tytär Meerab oli annettava Daavidille, annettiinkin hänet vaimoksi meholalaiselle Adrielille.
\par 20 Mutta Saulin tytär Miikal rakasti Daavidia. Ja kun se ilmoitettiin Saulille, oli se hänelle mieleen.
\par 21 Sillä Saul ajatteli: "Minä annan hänet Daavidille, että hän tulisi hänelle ansaksi ja filistealaisten käsi sattuisi häneen". Niin Saul sanoi Daavidille: "Nyt voit toisen kerran tulla minun vävykseni".
\par 22 Ja Saul käski palvelijoitansa: "Puhukaa salaa Daavidille näin: 'Katso, kuningas on mieltynyt sinuun, ja kaikki hänen palvelijansa rakastavat sinua; sinun on nyt tultava kuninkaan vävyksi'".
\par 23 Niin Saulin palvelijat puhuivat nämä sanat Daavidille. Mutta Daavid sanoi: "Onko teidän mielestänne niin pieni asia tulla kuninkaan vävyksi? Minähän olen köyhä ja halpa mies."
\par 24 Saulin palvelijat ilmoittivat hänelle tämän, sanoen: "Näin on Daavid puhunut".
\par 25 Niin Saul sanoi: "Sanokaa Daavidille näin: 'Kuningas ei halua muuta morsiamenhintaa kuin sata filistealaisten esinahkaa, että kuninkaan vihollisille kostettaisiin'". Sillä Saul ajatteli kaataa Daavidin filistealaisten käden kautta.
\par 26 Kun hänen palvelijansa ilmoittivat tämän Daavidille, miellytti Daavidia tulla näin kuninkaan vävyksi; ja ennenkuin määräaika oli kulunut umpeen,
\par 27 nousi Daavid ja lähti miehineen ja kaatoi filistealaisia kaksisataa miestä. Ja Daavid toi heidän esinahkansa, täyden määrän, kuninkaalle, tullaksensa kuninkaan vävyksi. Silloin Saul antoi tyttärensä Miikalin hänelle vaimoksi.
\par 28 Ja Saul näki ja ymmärsi, että Herra oli Daavidin kanssa, ja Saulin tytär Miikal rakasti häntä.
\par 29 Niin Saul pelkäsi vielä enemmän Daavidia, ja Saulista tuli koko elinajakseen Daavidin vihamies.
\par 30 Mutta filistealaisten ruhtinaat lähtivät sotaan; ja niin usein kuin he lähtivät sotaan, oli Daavidilla suurempi menestys kuin kaikilla muilla Saulin palvelijoilla, niin että hänen nimensä tuli sangen kuuluisaksi.

\chapter{19}

\par 1 Ja Saul puhui pojallensa Joonatanille ja kaikille palvelijoillensa, että Daavid olisi surmattava. Mutta Saulin poika Joonatan oli suuresti mieltynyt Daavidiin.
\par 2 Sentähden Joonatan ilmaisi sen Daavidille ja sanoi: "Minun isäni Saul koettaa saada sinut surmatuksi. Ole siis varuillasi huomenaamuna, kätkeydy ja pysy piilossa.
\par 3 Mutta minä menen ja asetun isäni viereen kedolle, sinne, missä sinä olet, ja puhun sinusta isälleni; ja jos jotakin huomaan, niin minä ilmaisen sen sinulle."
\par 4 Niin Joonatan puhui hyvää Daavidista isällensä Saulille ja sanoi hänelle: "Älköön kuningas rikkoko palvelijaansa Daavidia vastaan, sillä hän ei ole rikkonut sinua vastaan, vaan hänen työnsä ovat olleet sinulle suureksi hyödyksi.
\par 5 Hänhän pani henkensä kämmenelleen ja surmasi sen filistealaisen, ja niin Herra antoi suuren voiton koko Israelille; sinä olet itse nähnyt sen ja iloinnut. Miksi siis rikkoisit viatonta verta vastaan surmaamalla Daavidin syyttömästi?"
\par 6 Saul kuuli Joonatanin ääntä; ja Saul vannoi: "Niin totta kuin Herra elää, häntä ei surmata".
\par 7 Niin Joonatan kutsui Daavidin, ja Joonatan ilmoitti hänelle kaiken tämän. Ja sitten Joonatan vei Daavidin Saulin tykö, ja hän palveli häntä niinkuin ennenkin.
\par 8 Kun sitten sota alkoi uudestaan, lähti Daavid taistelemaan filistealaisia vastaan ja tuotti heille suuren tappion, niin että he pakenivat hänen edestään.
\par 9 Mutta Herran lähettämä paha henki tuli Sauliin, kun hän istui kotonaan keihäs kädessä ja Daavid soitteli.
\par 10 Niin Saul koetti keihästää Daavidin seinään; mutta hän väisti Saulia, ja tämä iski keihään seinään. Daavid pakeni ja pelastui sinä yönä.
\par 11 Niin Saul lähetti miehiä Daavidin kotiin vartioimaan häntä ja surmaamaan hänet aamulla. Mutta Daavidin vaimo Miikal ilmaisi sen hänelle ja sanoi: "Jollet tänä yönä pelasta henkeäsi, niin sinä huomenna olet surman oma".
\par 12 Ja Miikal laski Daavidin alas ikkunasta; ja hän lähti pakoon ja pelastui.
\par 13 Sitten Miikal otti kotijumalan ja asetti sen vuoteeseen, ja levitettyään vuohenkarvoista tehdyn kärpäsverkon sen pään yli hän peitti sen vaatteella.
\par 14 Ja kun Saul lähetti miehiä ottamaan Daavidia, sanoi hän: "Hän on sairas".
\par 15 Niin Saul lähetti miehiä katsomaan Daavidia, sanoen: "Tuokaa hänet vuoteessa tänne minun luokseni surmattavaksi".
\par 16 Mutta kun miehet tulivat sisälle, niin katso, vuoteessa olikin kotijumala, vuohenkarvoista tehty kärpäsverkko pään päällä.
\par 17 Niin Saul sanoi Miikalille: "Miksi sinä olet minut näin pettänyt ja olet päästänyt minun vihamieheni pelastumaan?" Miikal vastasi Saulille: "Hän sanoi minulle: 'Päästä minut, muutoin minä surmaan sinut'".
\par 18 Kun Daavid oli paennut ja pelastunut, meni hän Samuelin luo Raamaan ja kertoi hänelle kaikki, mitä Saul oli hänelle tehnyt. Ja hän ja Samuel menivät Naajotiin ja jäivät sinne.
\par 19 Ja Saulille ilmoitettiin: "Katso, Daavid on Raaman Naajotissa".
\par 20 Niin Saul lähetti miehiä ottamaan Daavidia. Mutta kun he näkivät profeettain joukon hurmoksissa ja Samuelin seisovan johtamassa heitä, tuli Jumalan henki Saulin miehiin, niin että hekin joutuivat hurmoksiin.
\par 21 Kun se ilmoitettiin Saulille, lähetti hän toiset miehet; mutta hekin joutuivat hurmoksiin. Ja Saul lähetti vielä kolmannet miehet, mutta hekin joutuivat hurmoksiin.
\par 22 Silloin hän itse lähti Raamaan. Ja kun hän tuli sen suuren vesisäiliön luo, joka on Seekussa, kysyi hän: "Missä ovat Samuel ja Daavid?" Hänelle vastattiin: "He ovat Raaman Naajotissa".
\par 23 Mutta kun hän oli menossa sinne, Raaman Naajotiin, tuli Jumalan henki häneenkin, niin että hän kulki hurmoksissa, kunnes tuli Raaman Naajotiin.
\par 24 Silloin hänkin riisui vaatteensa ja joutui hurmoksiin Samuelin edessä; ja hän kaatui maahan ja makasi alastonna koko sen päivän ja koko sen yön. Sentähden on tapana sanoa: "Onko Saulkin profeettain joukossa?"

\chapter{20}

\par 1 Mutta Daavid pakeni Raaman Naajotista, tuli ja puhui Joonatanille: "Mitä minä olen tehnyt? Mitä vääryyttä, mitä syntiä minä olen tehnyt sinun isääsi vastaan, kun hän väijyy henkeäni?"
\par 2 Tämä vastasi hänelle: "Pois se! Et sinä kuole. Katso, isäni ei tee mitään, suurta eikä pientä, ilmoittamatta minulle. Miksi sitten isäni salaisi minulta tämän? Ei ole niin."
\par 3 Mutta Daavid vielä vakuutti vannoen ja sanoi: "Isäsi tietää, että minä olen saanut armon sinun silmiesi edessä; sentähden hän ajattelee: 'Älköön Joonatan saako tietää tätä, ettei hän tulisi murheelliseksi'. Mutta niin totta kuin Herra elää ja niin totta kuin sinä itse elät: on vain askel minun ja kuoleman välillä."
\par 4 Niin Joonatan sanoi Daavidille: "Mitä sinä vain lausut toivovasi, sen minä sinulle teen".
\par 5 Daavid sanoi Joonatanille: "Katso, huomenna on uusikuu, ja minun pitäisi istua aterioimassa kuninkaan kanssa; mutta salli minun mennä ja kätkeytyä ulos kedolle ylihuomeniltaan asti.
\par 6 Jos isäsi kaipaa minua, niin sano: 'Daavid pyysi minulta, että saisi käväistä kaupungissaan Beetlehemissä, sillä koko suvulla on siellä jokavuotiset teurasuhrit'.
\par 7 Ja jos hän sanoo: 'Hyvä on', niin voi palvelijasi olla rauhassa. Mutta jos hän vihastuu, niin tiedä, että hänellä on paha mielessä.
\par 8 Tee siis laupeus palvelijallesi, koska olet ottanut palvelijasi Herran liittoon kanssasi. Mutta jos minussa on vääryys, niin surmaa sinä minut; sillä miksi sinä veisit minut isäsi eteen?"
\par 9 Joonatan sanoi: "Pois se! Jos huomaan, että minun isälläni on paha mielessä sinua vastaan, niin minä sen varmasti sinulle ilmoitan."
\par 10 Mutta Daavid sanoi Joonatanille: "Kunpa sitten joku ilmoittaisi minulle, antaako isäsi sinulle kovan vastauksen".
\par 11 Joonatan sanoi Daavidille: "Tule, menkäämme ulos kedolle". Niin he menivät molemmat ulos kedolle.
\par 12 Ja Joonatan sanoi Daavidille: "Minä vakuutan Herran, Israelin Jumalan, kautta, että koetan huomenna tai ylihuomenna tähän aikaan päästä selville isästäni; ja jos Daavidin asiat ovat hyvin, niin minä varmasti lähetän siitä tiedon ja ilmoitan sen sinulle.
\par 13 Herra rangaiskoon Joonatania nyt ja vasta, jollen minä, jos isäni mielii tehdä sinulle pahaa, ilmoita sitä sinulle ja päästä sinua menemään rauhassa. Herra olkoon sinun kanssasi, niinkuin hän on ollut minun isäni kanssa.
\par 14 Etkö sinäkin, jos minä silloin vielä elän, etkö sinäkin tee Herran laupeutta minulle, niin ettei minun tarvitse kuolla?
\par 15 Ethän koskaan kiellä laupeuttasi minun suvultani, et silloinkaan, kun Herra hävittää Daavidin vihamiehet kaikki tyynni maan päältä?"
\par 16 Niin Joonatan teki liiton Daavidin suvun kanssa. Ja Herra vaati koston Daavidin vihamiehiltä.
\par 17 Ja Joonatan vannotti vielä Daavidin heidän keskinäisen rakkautensa kautta, sillä hän rakasti häntä niinkuin omaa sieluansa.
\par 18 Ja Joonatan sanoi hänelle: "Huomenna on uusikuu, ja sinua kaivataan, kun paikkasi on tyhjä.
\par 19 Mutta ylihuomenna mene nopeasti siihen paikkaan, johon kätkeydyit sinä päivänä, jona se teko tapahtui, ja istuudu Eselin kiven ääreen.
\par 20 Niin minä ammun kolme nuolta sen laitaa kohti, niinkuin ampuisin maaliin.
\par 21 Sitten minä lähetän palvelijan sanoen: 'Mene ja hae nuolet'. Jos minä silloin sanon palvelijalle: 'Katso, nuolet ovat takanasi, tännempänä, ota ne', niin tule kotiin, sillä silloin voit olla rauhassa eikä mitään ole tekeillä; niin totta kuin Herra elää.
\par 22 Mutta jos minä sanon sille nuorelle miehelle näin: 'Katso, nuolet ovat edessäsi, sinnempänä', niin lähde, sillä Herra lähettää sinut pois.
\par 23 Ja mitä minä ja sinä olemme keskenämme puhuneet, sen todistaja meidän välillämme, minun ja sinun, on Herra iankaikkisesti."
\par 24 Niin Daavid kätkeytyi kedolle. Ja kun uusikuu tuli, istuutui kuningas aterialle.
\par 25 Kuningas istui tavallisella istuimellansa, istuimella seinän vieressä. Ja Joonatan nousi seisomaan, ja Abner istuutui Saulin viereen. Mutta Daavidin paikka oli tyhjä.
\par 26 Saul ei kuitenkaan sanonut mitään sinä päivänä, sillä hän ajatteli: "Jotakin on hänelle tapahtunut: hän ei liene puhdas; varmaankaan hän ei ole puhdas".
\par 27 Mutta kun Daavidin paikka oli tyhjä seuraavanakin päivänä, toisena uudenkuun päivänä, sanoi Saul pojallensa Joonatanille: "Miksi ei Iisain poika ole eilen eikä tänään tullut aterialle?"
\par 28 Joonatan vastasi Saulille: "Daavid pyysi minulta, että saisi mennä Beetlehemiin;
\par 29 hän sanoi: 'Salli minun mennä, sillä meillä on suvun teurasuhrit kaupungissa, ja veljeni itse on käskenyt minut sinne; jos minä siis olen saanut armon sinun silmiesi edessä, niin päästä minut katsomaan veljiäni'. Sentähden hän ei ole tullut kuninkaan pöytään."
\par 30 Silloin Saul vihastui Joonataniin ja sanoi hänelle: "Sinä säädyttömän naisen poika! Tiesinhän minä, että sinä olet mieltynyt Iisain poikaan, häpeäksi itsellesi ja häpeäksi äitisi hävylle.
\par 31 Sillä niin kauan kuin Iisain poika elää maan päällä, et sinä eikä sinun kuninkuutesi ole turvassa. Lähetä nyt noutamaan hänet minun luokseni, sillä hän on kuoleman oma."
\par 32 Joonatan vastasi isälleen Saulille ja sanoi hänelle: "Miksi hänet on surmattava? Mitä hän on tehnyt?"
\par 33 Silloin Saul heitti keihään häntä kohti surmataksensa hänet. Niin Joonatan ymmärsi, että hänen isänsä oli päättänyt tappaa Daavidin.
\par 34 Ja Joonatan nousi pöydästä, vihasta hehkuen, eikä syönyt mitään toisena uudenkuun päivänä, sillä hän oli murheissaan Daavidin tähden, koska hänen isänsä oli häväissyt tätä.
\par 35 Seuraavana aamuna Joonatan lähti ulos kedolle, niinkuin hän oli sopinut Daavidin kanssa, ja hänellä oli mukanaan pieni poikanen.
\par 36 Ja hän sanoi pojalle: "Juokse noutamaan nuolet, jotka minä ammun". Ja pojan juostessa hän ampui nuolen hänen ylitsensä.
\par 37 Ja kun poika oli tulemassa siihen paikkaan, missä Joonatanin ampuma nuoli oli, huusi Joonatan pojalle ja sanoi: "Nuoli on edessäsi, sinnempänä".
\par 38 Ja Joonatan huusi vielä pojalle: "Riennä nopeasti, älä seisahtele!" Niin Joonatanin poikanen otti nuolen ja tuli herransa luo.
\par 39 Eikä poika tiennyt asiasta mitään; ainoastaan Joonatan ja Daavid tiesivät sen.
\par 40 Ja Joonatan antoi aseensa pojalle, joka hänellä oli mukanaan, ja sanoi hänelle: "Mene ja vie nämä kaupunkiin".
\par 41 Ja kun poika oli mennyt, nousi Daavid ylös etelän puolelta; ja hän lankesi kasvoilleen maahan ja kumartui kolme kertaa; ja he suutelivat toisiansa ja itkivät yhdessä; Daavid itki hillittömästi.
\par 42 Ja Joonatan sanoi Daavidille: "Mene rauhassa. Niin on, kuin me molemmat olemme vannoneet Herran nimeen ja sanoneet: 'Herra on todistaja meidän välillämme, minun ja sinun, ja minun jälkeläisteni ja sinun jälkeläistesi välillä iankaikkisesti'."
\par 43 Sitten Daavid nousi ja lähti, mutta Joonatan palasi kaupunkiin.

\chapter{21}

\par 1 Ja Daavid tuli pappi Ahimelekin luo Noobiin. Mutta Ahimelek tuli vavisten Daavidia vastaan ja kysyi häneltä: "Miksi sinä tulet yksin, eikä ole ketään sinun kanssasi?"
\par 2 Daavid vastasi pappi Ahimelekille: "Kuningas on käskenyt minut asialle ja sanoi minulle: 'Älköön kukaan saako tietää mitään siitä asiasta, jolle minä sinut lähetän, ja siitä käskystä, jonka minä olen sinulle antanut'. Ja palvelijat minä olen määrännyt siihen ja siihen paikkaan.
\par 3 Anna nyt minulle, mitä sinulla on, viisi leipää, tahi mitä muuta löytyy."
\par 4 Pappi vastasi Daavidille ja sanoi: "Tavallista leipää minulla ei ole; ainoastaan pyhää leipää on, jos vain palvelijat ovat karttaneet naisia".
\par 5 Daavid vastasi papille ja sanoi hänelle: "Totisesti olivat naiset eristettyinä meistä niinkuin ennenkin, milloin minä olen retkelle lähtenyt; ovatpa palvelijain reputkin aina olleet pyhät, vaikka retki on ollutkin tavallinen retki. Sitä enemmän leipä repuissa nyt pysyy pyhänä."
\par 6 Niin pappi antoi hänelle pyhää leipää; sillä siellä ei ollut muuta leipää kuin näkyleipiä, jotka oli siirretty pois Herran edestä, että sinne samana päivänä, jona ne oli otettu pois, pantaisiin tuoreet leivät.
\par 7 Mutta siellä oli sinä päivänä eristettynä Herran edessä eräs Saulin palvelija, nimeltä Dooeg, edomilainen, Saulin paimenten päämies.
\par 8 Ja Daavid kysyi vielä Ahimelekiltä: "Eikö sinulla ole täällä yhtään keihästä tai miekkaa? Sillä minä en ottanut mukaani, en miekkaani enkä muita aseitani, koska kuninkaan asia oli niin kiireellinen."
\par 9 Pappi vastasi: "On, sen filistealaisen Goljatin miekka, jonka sinä surmasit Tammilaaksossa; katso, se on vaippaan käärittynä tuolla kasukan takana. Jos tahdot ottaa itsellesi sen, niin ota; sillä muuta kuin se ei täällä ole." Daavid sanoi: "Ei ole toista sen vertaista; anna se minulle".
\par 10 Sitten Daavid nousi ja pakeni sinä päivänä Saulia ja tuli Aakiin, Gatin kuninkaan, luo.
\par 11 Mutta Aakiin palvelijat sanoivat hänelle: "Eikö tämä ole Daavid, sen maan kuningas? Eivätkö he karkeloiden virittäneet hänelle tätä laulua: 'Saul voitti tuhat, mutta Daavid kymmenentuhatta'?"
\par 12 Daavid pani mieleensä nämä sanat ja pelkäsi suuresti Aakista, Gatin kuningasta.
\par 13 Niin hän tekeytyi mielipuoleksi heidän silmiensä edessä ja raivosi heidän käsissään ja piirusteli portin oviin ja valutti sylkeä partaansa.
\par 14 Silloin Aakis sanoi palvelijoilleen: "Näettehän, että mies on hullu. Miksi toitte hänet minun luokseni?
\par 15 Puuttuuko minulta hulluja, kun toitte tämän hulluttelemaan minun eteeni? Tällainenko saisi tulla minun hoviini?"

\chapter{22}

\par 1 Niin Daavid lähti sieltä ja pääsi pakoon Adullamin luolaan. Ja kun hänen veljensä ja koko hänen isänsä perhe kuulivat sen, tulivat he sinne hänen luoksensa.
\par 2 Ja hänen luoksensa kokoontui kaikenlaista ahdingossa olevaa, velkaantunutta ja katkeroitunutta väkeä, ja hän rupesi heidän päämiehekseen. Näin liittyi häneen noin neljäsataa miestä.
\par 3 Sieltä Daavid meni Mooabin Mispeen. Ja hän sanoi Mooabin kuninkaalle: "Salli minun isäni ja äitini tulla asumaan teidän luoksenne, kunnes saan tietää, mitä Jumala minulle tekee".
\par 4 Ja hän vei heidät Mooabin kuninkaan eteen; ja he jäivät tämän luo kaikeksi aikaa, minkä Daavid oli vuorilinnassa.
\par 5 Mutta profeetta Gaad sanoi Daavidille: "Älä jää vuorilinnaan; lähde täältä ja mene Juudan maahan". Niin Daavid lähti ja tuli Jaar-Heretiin.
\par 6 Ja Saul kuuli, että oli saatu vihiä Daavidista ja niistä miehistä, jotka olivat hänen kanssaan; ja Saul istui Gibeassa tamariskipuun alla kummulla, keihäs kädessä, ja kaikki hänen palvelijansa seisoivat hänen luonaan.
\par 7 Niin Saul sanoi palvelijoilleen, jotka seisoivat hänen luonaan: "Kuulkaa, benjaminilaiset! Antaako Iisain poika teillekin kaikille pellot ja viinitarhat, ja tekeekö hän teidät kaikki tuhannen- ja sadanpäämiehiksi?
\par 8 Tehän olette kaikki salaliitossa minua vastaan, eikä kukaan ole ilmaissut minulle, että oma poikani on tehnyt liiton Iisain pojan kanssa. Ei kukaan teistä välitä niin paljoa minusta, että olisi ilmaissut sen minulle. Sillä minun poikani on yllyttänyt palvelijani väijymään minua, niinkuin nyt tapahtuu."
\par 9 Niin edomilainen Dooeg, joka seisoi siinä Saulin palvelijain kanssa, vastasi ja sanoi: "Minä näin Iisain pojan tulevan Noobiin, Ahimelekin, Ahitubin pojan, luo.
\par 10 Tämä kysyi hänelle neuvoa Herralta ja evästi hänet ja antoi hänelle filistealaisen Goljatin miekan."
\par 11 Silloin kuningas lähetti kutsumaan pappi Ahimelekin, Ahitubin pojan, ja koko hänen isänsä suvun, Noobin papit. Ja he tulivat kaikki kuninkaan luo.
\par 12 Niin Saul sanoi: "Kuule, Ahitubin poika!" Hän vastasi: "Tässä olen, Herrani".
\par 13 Saul sanoi hänelle: "Miksi te olette salaliitossa minua vastaan, sinä ja Iisain poika, koska sinä olet antanut hänelle leipää ja miekan ja kysynyt hänelle Herralta neuvoa, että hän ryhtyisi väijymään minua, niinkuin nyt tapahtuu?"
\par 14 Ahimelek vastasi kuninkaalle ja sanoi: "Kuka on kaikkien sinun palvelijaisi joukossa niin uskottu kuin Daavid, joka on kuninkaan vävy ja sinun henkivartiostosi päällikkö ja on korkeassa arvossa sinun hovissasi?
\par 15 Enhän minä nyt ensi kertaa kysynyt hänelle Jumalalta neuvoa. Pois se! Älköön kuningas lukeko mitään viaksi minulle, palvelijalleen, älköönkä kaikelle minun isäni suvulle, sillä palvelijasi ei tietänyt kaikesta tästä vähän vähääkään."
\par 16 Mutta kuningas sanoi: "Sinun on kuolemalla kuoltava, Ahimelek, sinun ja koko sinun isäsi suvun".
\par 17 Ja kuningas sanoi henkivartijoille, jotka seisoivat hänen luonaan: "Käykää tänne ja surmatkaa Herran papit, sillä hekin pitävät Daavidin puolta; ja vaikka tiesivät hänen paenneen, eivät he ilmoittaneet sitä minulle". Mutta kuninkaan palvelijat eivät tahtoneet ojentaa kättään lyömään kuoliaaksi Herran pappeja.
\par 18 Niin kuningas sanoi Dooegille: "Käy sinä tänne ja lyö papit kuoliaaksi". Ja edomilainen Dooeg astui esiin ja löi papit kuoliaaksi; ja hän tappoi sinä päivänä kahdeksankymmentä viisi pellavakasukkaa kantavaa miestä.
\par 19 Myös pappiskaupungin Noobin asukkaat hän surmasi miekan terällä, sekä miehet että naiset, sekä lapset että imeväiset; jopa raavaat, aasit ja lampaatkin hän tappoi miekan terällä.
\par 20 Ainoastaan yksi Ahimelekin, Ahitubin pojan, poika, nimeltä Ebjatar, pelastui ja pakeni Daavidin luo.
\par 21 Ja Ebjatar kertoi Daavidille, että Saul oli tappanut Herran papit.
\par 22 Niin Daavid sanoi Ebjatarille: "Minä ymmärsin jo silloin, että edomilainen Dooeg, kun hän oli siellä, oli ilmaiseva sen Saulille. Minä olen syypää koko sinun isäsi suvun, joka hengen, kuolemaan.
\par 23 Jää minun luokseni, älä pelkää; sillä joka väijyy sinun henkeäsi, se väijyy minun henkeäni. Minun luonani sinä olet turvassa."

\chapter{23}

\par 1 Daavidille ilmoitettiin: "Katso, filistealaiset ovat taistelemassa Kegilaa vastaan ja ryöstävät puimatantereita".
\par 2 Niin Daavid kysyi Herralta: "Menenkö minä ja lyönkö nämä filistealaiset maahan?" Herra vastasi Daavidille: "Mene, lyö filistealaiset maahan ja pelasta Kegila".
\par 3 Mutta Daavidin miehet sanoivat hänelle: "Mehän elämme pelossa jo täällä Juudassa. Menisimmekö vielä Kegilaan filistealaisten taistelurivejä vastaan?"
\par 4 Niin Daavid kysyi taas Herralta, ja Herra vastasi hänelle ja sanoi: "Nouse ja mene alas Kegilaan; sillä minä annan filistealaiset sinun käsiisi".
\par 5 Ja Daavid meni miehinensä Kegilaan ja ryhtyi taisteluun filistealaisia vastaan ja kuljetti heidän karjansa pois ja tuotti heille suuren tappion. Näin Daavid pelasti Kegilan asukkaat.
\par 6 Ja kun Ebjatar, Ahimelekin poika, pakeni Daavidin luo Kegilaan, seurasi kasukka hänen mukanaan.
\par 7 Saulille ilmoitettiin, että Daavid oli mennyt Kegilaan. Niin Saul sanoi: "Jumala on jättänyt hänet minun käsiini, sillä hän on itse sulkenut itsensä sisälle, kun on mennyt kaupunkiin, jossa on ovet ja salvat".
\par 8 Sitten Saul kuulutti kaiken kansan sotaan, menemään alas Kegilaan, piirittämään Daavidia ja hänen miehiänsä.
\par 9 Kun Daavid sai tietää, että Saul hankki hänelle tuhoa, sanoi hän pappi Ebjatarille: "Tuo tänne kasukka".
\par 10 Ja Daavid sanoi: "Herra, Israelin Jumala! Sinun palvelijasi on kuullut, että Saul aikoo tulla Kegilaan hävittämään kaupungin minun tähteni.
\par 11 Luovuttavatko Kegilan miehet minut hänen käsiinsä? Tuleeko Saul, niinkuin palvelijasi on kuullut? Herra, Israelin Jumala, ilmoita se palvelijallesi." Herra vastasi: "Tulee".
\par 12 Ja Daavid kysyi vielä: "Luovuttavatko Kegilan miehet minut ja minun mieheni Saulin käsiin?" Herra vastasi: "Luovuttavat".
\par 13 Niin Daavid nousi miehinensä, joita oli noin kuusisataa miestä, ja he lähtivät Kegilasta ja kuljeskelivat paikasta toiseen. Mutta kun Saulille kerrottiin, että Daavid oli päässyt pakoon Kegilasta, luopui hän retkestänsä.
\par 14 Ja Daavid oleskeli erämaassa vuorten huipuilla; hän oleskeli Siifin erämaan vuoristossa. Ja Saul etsi häntä kaiken aikaa, mutta Jumala ei antanut häntä tämän käsiin.
\par 15 Daavid huomasi, että Saul oli lähtenyt väijymään hänen henkeänsä, ja Daavid oli Hooreksessa, Siifin erämaassa.
\par 16 Silloin Saulin poika Joonatan nousi ja meni Daavidin luo Hoorekseen ja rohkaisi häntä Jumalassa.
\par 17 Hän sanoi hänelle: "Älä pelkää; sillä minun isäni Saulin käsi ei ole tapaava sinua, vaan sinusta tulee Israelin kuningas, ja minä tulen olemaan lähinnä sinua. Myös minun isäni Saul tietää kyllä sen."
\par 18 Sitten he molemmat tekivät liiton Herran edessä. Ja Daavid jäi Hoorekseen, mutta Joonatan palasi kotiinsa.
\par 19 Mutta siifiläisiä meni Saulin luo Gibeaan, ja he sanoivat: "Daavid piileksii meillä päin, vuorten huipuilla Hooreksessa, Gibeat-Hakilassa, joka on kallioerämaasta etelään.
\par 20 Niin tule nyt sinne, kuningas, milloin vain haluat tulla; me kyllä luovutamme hänet kuninkaan käsiin."
\par 21 Saul sanoi: "Herra siunatkoon teitä, kun te säälitte minua.
\par 22 Menkää, pitäkää häntä vielä silmällä, tiedustelkaa ja katsokaa, missä paikassa hänen jalkansa liikkuu ja kuka hänet on siellä nähnyt; sillä minulle on sanottu, että hän on hyvin kavala.
\par 23 Katsokaa ja tiedustelkaa kaikki piilopaikat, joihin hän voi piiloutua. Palatkaa sitten minun luokseni ja tuokaa varmat tiedot, niin minä lähden teidän kanssanne. Ja jos hän on tässä maassa, niin minä etsin hänet kaikkien Juudan sukujen seasta."
\par 24 Ja he nousivat ja menivät Siifiin Saulin edellä. Mutta Daavid miehineen oli Maaonin erämaassa, aromaassa, kallioerämaasta etelään.
\par 25 Kun Saul miehineen meni etsimään Daavidia, ilmoitettiin se tälle, ja hän laskeutui eräälle kalliolle ja jäi Maaonin erämaahan. Kun Saul kuuli sen, lähti hän ajamaan Daavidia takaa Maaonin erämaahan.
\par 26 Ja Saul kulki vuoren toista puolta ja Daavid miehineen vuoren toista puolta. Ja kun Daavid riensi pakoon Saulin tieltä ja Saul miehinensä oli kiertämässä Daavidia ja hänen miehiänsä, ottaaksensa heidät kiinni,
\par 27 tuli sanansaattaja Saulin luo ja sanoi: "Tule kiiruusti, sillä filistealaiset ovat hyökänneet maahan".
\par 28 Silloin Saul lakkasi ajamasta takaa Daavidia ja lähti filistealaisia vastaan. Siitä se paikka sai nimekseen Mahlekotkallio.

\chapter{24}

\par 1 Mutta Daavid lähti sieltä ja oleskeli Een-Gedin vuorten huipuilla.
\par 2 Ja kun Saul palasi ajamasta filistealaisia takaa, ilmoitettiin hänelle: "Katso, Daavid on Een-Gedin erämaassa".
\par 3 Niin Saul otti kolmetuhatta valiomiestä koko Israelista ja meni etsimään Daavidia ja hänen miehiään Kauriskallioiden itäpuolelta.
\par 4 Ja kun hän tuli tien varrella oleville karjatarhoille, oli siellä luola; ja Saul meni luolaan tarpeelleen. Mutta Daavid ja hänen miehensä istuivat luolan perällä.
\par 5 Niin Daavidin miehet sanoivat hänelle: "Katso, tämä on se päivä, josta Herra on sanonut sinulle: 'Minä annan vihamiehesi sinun käsiisi, tehdäksesi hänelle, mitä hyväksi näet'". Ja Daavid nousi ja leikkasi salaa kappaleen Saulin viitan liepeestä.
\par 6 Mutta sen jälkeen Daavidin omatunto soimasi häntä siitä, että hän oli leikannut Saulin viitan lievettä.
\par 7 Ja hän sanoi miehillensä: "Pois se! Herra varjelkoon minua tekemästä sitä herralleni, Herran voidellulle, että satuttaisin käteni häneen; sillä hän on Herran voideltu".
\par 8 Ja Daavid kovisti miehiänsä eikä sallinut heidän hyökätä Saulin kimppuun. Niin Saul nousi luolasta ja meni matkaansa.
\par 9 Senjälkeen Daavid myös nousi, lähti luolasta ja huusi Saulin jälkeen: "Minun herrani, kuningas!" Kun Saul katsahti taaksensa, kumartui Daavid kasvoilleen maahan ja osoitti kunnioitusta.
\par 10 Ja Daavid sanoi Saulille: "Miksi sinä kuuntelet ihmisten puhetta, jotka sanovat: 'Daavid hankkii sinulle onnettomuutta'?
\par 11 Katso, näethän nyt omin silmin, kuinka Herra tänä päivänä antoi sinut minun käsiini luolassa, mutta minä säästin sinut, vaikka minua neuvottiin tappamaan sinut; sillä minä ajattelin: 'Minä en satuta kättäni herraani; sillä hän on Herran voideltu'.
\par 12 Katso itse, isäni, katso tätä viittasi liepeen kappaletta, joka on minun kädessäni. Kun minä leikkasin sen sinun viittasi liepeestä enkä sinua tappanut, niin ymmärrä siitä ja näe, ettei minulla ole tekeillä mitään pahaa tai rikollista ja etten minä ole sinua vastaan rikkonut, vaikka sinä vainoat minua ottaaksesi minulta hengen.
\par 13 Herra tuomitkoon meidän välillämme, minun ja sinun, ja Herra kostakoon sinulle minun puolestani, mutta minun käteni ei sinuun satu.
\par 14 On, niinkuin vanha sananlasku sanoo: 'Jumalattomuus tulee jumalattomista'; mutta minun käteni ei sinuun satu.
\par 15 Kenen jälkeen Israelin kuningas on lähtenyt? Ketä sinä ajat takaa? Koiranraatoa, yhtä kirppua!
\par 16 Herra olkoon tuomari ja tuomitkoon meidän välillämme, minun ja sinun; hän nähköön ja ajakoon minun asiani ja auttakoon minut sinun käsistäsi oikeuteeni."
\par 17 Kun Daavid oli päättänyt tämän puheensa Saulille, sanoi Saul: "Eikö se ole sinun äänesi, poikani Daavid?" Ja Saul korotti äänensä ja itki.
\par 18 Ja hän sanoi Daavidille: "Sinä olet minua vanhurskaampi, sillä sinä olet tehnyt minulle hyvää, vaikka minä olen tehnyt sinulle pahaa.
\par 19 Sinä olet tänä päivänä osoittanut hyvyyttäsi minua kohtaan, kun et tappanut minua, vaikka Herra oli antanut minut sinun käsiisi.
\par 20 Sillä kun joku kohtaa vihollisensa, päästääkö hän hänet menemään rauhassa? Herra palkitkoon sinulle runsaasti sen, mitä olet tänä päivänä tehnyt minulle.
\par 21 Ja katso, nyt minä tiedän, että sinä tulet kuninkaaksi ja että Israelin kuninkuus on pysyvä sinun kädessäsi.
\par 22 Niin vanno nyt minulle Herran kautta, ettet hukuta minun jälkeläisiäni etkä hävitä minun nimeäni isäni suvusta."
\par 23 Ja Daavid vannoi Saulille. Sitten Saul meni kotiinsa; mutta Daavid ja hänen miehensä nousivat vuorilinnaan.

\chapter{25}

\par 1 Ja Samuel kuoli, ja koko Israel kokoontui pitämään hänen valittajaisiaan, ja he hautasivat hänet hänen kotipaikkaansa Raamaan. Ja Daavid nousi ja meni Paaranin erämaahan.
\par 2 Ja Maaonissa oli mies, jolla oli karjataloutensa Karmelissa, ja se mies oli hyvin rikas: hänellä oli kolmetuhatta lammasta ja tuhat vuohta. Ja hän oli keritsemässä lampaitansa Karmelissa.
\par 3 Miehen nimi oli Naabal, ja hänen vaimonsa nimi oli Abigail. Vaimo oli hyvin ymmärtäväinen ja vartaloltaan kaunis, mutta mies oli tyly ja menoissaan raaka; hän oli kaalebilainen.
\par 4 Ja kun Daavid kuuli erämaassa, että Naabal keritsi lampaitansa,
\par 5 lähetti hän sinne kymmenen nuorta miestä, ja Daavid sanoi nuorille miehille: "Menkää Karmeliin, ja kun tulette Naabalin luo, niin tervehtikää häntä minun nimessäni
\par 6 ja sanokaa sille eläjälle: 'Sinä elät rauhassa; rauhassa elää myös sinun perheesi ja kaikki, mitä sinulla on.
\par 7 Minä olen nyt kuullut, että sinulla on lammasten keritsiäiset. Sinun paimenesi ovat olleet meidän läheisyydessämme; me emme ole heitä loukanneet, eikä heiltä ole mitään hävinnyt koko sinä aikana, minkä ovat olleet Karmelissa.
\par 8 Kysy nuorilta miehiltäsi, niin he sanovat sen sinulle. Saakoot siis nämä miehet armon sinun silmiesi edessä, sillä mehän olemme tulleet juhlapäivänä. Anna sentähden palvelijoillesi ja pojallesi Daavidille sitä, mitä sinulla on käsillä.'"
\par 9 Kun Daavidin miehet tulivat sinne, puhuivat he Daavidin nimessä Naabalille tämän kaiken ja jäivät hiljaa odottamaan.
\par 10 Mutta Naabal vastasi Daavidin palvelijoille ja sanoi: "Kuka Daavid on, kuka on Iisain poika? Tätä nykyä on paljon orjia, jotka karkaavat isäntäinsä luota.
\par 11 Ottaisinko minä ruokani ja juomani ja teuraani, jotka olen teurastanut keritsiäisiini, ja antaisin ne miehille, jotka ovat kotoisin ties mistä?"
\par 12 Niin Daavidin miehet kääntyivät ja menivät pois, ja tultuaan takaisin he kertoivat hänelle tämän kaiken.
\par 13 Niin Daavid sanoi miehillensä: "Jokainen sitokoon miekkansa vyölleen". Ja jokainen sitoi miekkansa vyölleen; ja Daavid itsekin sitoi miekkansa vyölleen. Ja noin neljäsataa miestä lähti seuraamaan Daavidia, mutta kaksisataa jäi kuormaston luo.
\par 14 Mutta eräs nuori mies toi sanan Abigailille, Naabalin vaimolle, ja sanoi: "Katso, Daavid on lähettänyt sanansaattajia erämaasta tervehtimään meidän isäntäämme, mutta hän vain haukkui heitä.
\par 15 Ne miehet ovat kuitenkin olleet meille varsin hyviä: he eivät loukanneet meitä, eikä meiltä mitään hävinnyt koko sinä aikana, minkä kuljeskelimme heidän läheisyydessään ollessamme kedolla.
\par 16 He olivat muurina meidän ympärillämme yöllä ja päivällä koko sen ajan, minkä oleskelimme heidän läheisyydessään paimentaessamme lampaita.
\par 17 Niin ajattele nyt sinä ja katso, mitä voit tehdä, sillä onnettomuus uhkaa meidän isäntäämme ja koko hänen taloansa. Hän itse on kelvoton mies, niin ettei hänelle auta puhua."
\par 18 Niin Abigail otti joutuin kaksisataa leipää, kaksi leiliä viiniä, viisi lampaanpaistia, viisi sea-mittaa paahdettuja jyviä, sata rusinakakkua ja kaksisataa viikunakakkua ja pani ne aasien selkään.
\par 19 Ja hän sanoi palvelijoillensa: "Menkää minun edelläni, minä tulen jäljessänne". Mutta miehellensä Naabalille hän ei sitä ilmoittanut.
\par 20 Kun hän sitten ratsasti aasilla alas mäkeä vuoren suojassa, niin katso, Daavid miehinensä tuli toista mäkeä alas häntä vastaan, niin että hän joutui kohtaamaan heidät.
\par 21 Mutta Daavid oli sanonut: "Mitään saamatta minä olen suojellut kaikkea, mitä tällä miehellä oli erämaassa, niin ettei mitään ole hävinnyt kaikesta, mitä hänellä oli. Hän on palkinnut minulle hyvän pahalla.
\par 22 Jumala rangaiskoon Daavidin vihamiehiä nyt ja vasta: totisesti, minä en jätä kaikesta hänen väestänsä huomenaamuksi henkiin yhtään miehenpuolta."
\par 23 Mutta kun Abigail näki Daavidin, laskeutui hän kiiruusti aasin selästä maahan, lankesi kasvoilleen Daavidin eteen ja kumartui maahan.
\par 24 Ja kun hän oli langennut hänen jalkainsa juureen, sanoi hän: "Minun on syy, herrani. Mutta salli palvelijattaresi puhua sinulle ja kuule palvelijattaresi sanoja.
\par 25 Älköön herrani välittäkö mitään tuosta kelvottomasta miehestä, Naabalista, sillä nimi on miestä myöten. Naabal on hänen nimensä, ja houkkamainen hän on. Mutta minä, sinun palvelijattaresi, en ole nähnyt niitä miehiä, jotka sinä, herrani, lähetit.
\par 26 Ja nyt, herrani, niin totta kuin Herra elää ja niin totta kuin sinä itse elät, jonka Herra on estänyt joutumasta verivelan alaiseksi ja auttamasta itseäsi omalla kädelläsi: käyköön nyt sinun vihamiehillesi ja kaikille, jotka hankkivat onnettomuutta minun herralleni, niinkuin Naabalille.
\par 27 Ja nyt annettakoon tämä tervehdyslahja, jonka palvelijattaresi on herralleni tuonut, niille miehille, jotka seuraavat herraani.
\par 28 Anna palvelijattaresi rikkomus anteeksi. Sillä Herra on rakentava minun herralleni pysyväisen huoneen, koska minun herrani käy Herran sotia; eikä mitään pahaa löydetä sinussa koko elinaikanasi.
\par 29 Ja jos joku nousee vainoamaan sinua ja väijymään sinun henkeäsi, niin herrani henki on tallella elävien kukkarossa Herran, sinun Jumalasi, tykönä; mutta sinun vihollistesi hengen hän lingolla linkoaa pois.
\par 30 Ja kun Herra tekee minun herralleni kaiken hyvän, josta hän on sinulle puhunut, ja määrää sinut Israelin ruhtinaaksi,
\par 31 niin ei sinua kaada eikä herrani tuntoa vaivaa se, että olisit aiheettomasti vuodattanut verta ja että herrani olisi itse auttanut itseänsä. Mutta kun Herra on tekevä hyvää minun herralleni, niin muista palvelijatartasi."
\par 32 Niin Daavid sanoi Abigailille: "Kiitetty olkoon Herra, Israelin Jumala, joka tänä päivänä lähetti sinut minua vastaan.
\par 33 Ja siunattu olkoon sinun ymmärtäväisyytesi, ja siunattu ole sinä itse, joka tänä päivänä estit minut joutumasta verivelan alaiseksi ja auttamasta itseäni omalla kädelläni.
\par 34 Mutta niin totta kuin Herra, Israelin Jumala, elää, hän, joka on pidättänyt minut tekemästä sinulle pahaa: jollet sinä joutuin olisi tullut minua vastaan, niin totisesti ei Naabalin väestä olisi huomenna aamun valjetessa ollut jäljellä yhtään miehenpuolta."
\par 35 Sitten Daavid otti häneltä, mitä hän oli hänelle tuonut, ja sanoi hänelle: "Mene rauhassa kotiisi. Katso, minä olen kuullut sinua ja tehnyt sinulle mieliksi."
\par 36 Kun Abigail sitten tuli Naabalin luo, oli tällä talossansa pidot, niinkuin kuninkaan pidot; ja Naabalin sydän oli iloinen, ja hän oli kovin juovuksissa. Niin hän ei kertonut Naabalille mitään, ennenkuin aamu valkeni.
\par 37 Mutta seuraavana aamuna, kun Naabalista humala oli haihtunut, kertoi hänen vaimonsa hänelle, mitä oli tapahtunut. Silloin hänen sydämensä kuoleutui hänen povessaan, ja hän ikäänkuin kivettyi.
\par 38 Ja noin kymmenen päivän kuluttua Herra löi Naabalia, niin että hän kuoli.
\par 39 Kun Daavid kuuli, että Naabal oli kuollut, sanoi hän: "Kiitetty olkoon Herra, joka on kostanut Naabalille minun häväistykseni ja pidättänyt palvelijansa pahasta ja kääntänyt Naabalin pahuuden hänen oman päänsä päälle". Senjälkeen Daavid lähetti sanomaan Abigailille, että hän halusi ottaa hänet vaimokseen.
\par 40 Ja kun Daavidin palvelijat tulivat Abigailin luo Karmeliin, puhuivat he hänelle ja sanoivat: "Daavid lähetti meidät sinun luoksesi ottamaan sinut hänelle vaimoksi".
\par 41 Niin hän nousi ylös, kumartui kasvoilleen maahan ja sanoi: "Katso, palvelijattaresi on valmis rupeamaan orjattareksi ja pesemään herrani palvelijain jalat".
\par 42 Sitten Abigail nousi kiiruusti ylös ja istuutui aasin selkään, samoin hänen viisi palvelijatartansa, jotka olivat hänellä seuralaisina. Ja niin hän seurasi Daavidin sanansaattajia ja tuli hänen vaimoksensa.
\par 43 Daavid oli nainut myös Ahinoamin Jisreelistä, niin että he molemmat tulivat hänen vaimoikseen.
\par 44 Mutta Saul oli antanut tyttärensä Miikalin, Daavidin vaimon, Paltille, Laiksen pojalle, joka oli kotoisin Gallimista.

\chapter{26}

\par 1 Siifiläiset tulivat Saulin luo Gibeaan ja sanoivat: "Daavid piileksii Gibeat-Hakilassa, kallioerämaan puolella".
\par 2 Niin Saul nousi ja meni Siifin erämaahan, ja hänen kanssansa kolmetuhatta Israelin valiomiestä, etsimään Daavidia Siifin erämaasta.
\par 3 Ja Saul leiriytyi Gibeat-Hakilaan, joka on kallioerämaan puolella tien varrella. Mutta Daavid oleskeli erämaassa. Ja kun Daavid huomasi, että Saul oli tullut hänen jäljessään erämaahan,
\par 4 lähetti hän vakoojia ja sai varman tiedon Saulin tulosta.
\par 5 Niin Daavid nousi ja meni siihen paikkaan, johon Saul oli leiriytynyt; ja Daavid näki, missä paikassa Saul ja hänen sotapäällikkönsä Abner, Neerin poika, makasivat. Saul näet makasi leirissä, ja väki oli asettunut hänen ympärilleen.
\par 6 Ja Daavid puhui heettiläiselle Ahimelekille ja Abisaille, Serujan pojalle, Jooabin veljelle, ja sanoi: "Kuka lähtee minun kanssani Saulin tykö leiriin?" Abisai vastasi: "Minä lähden sinun kanssasi".
\par 7 Niin Daavid ja Abisai menivät yöllä väen luo ja näkivät, että Saul makasi ja nukkui leirissä, ja hänen keihäänsä oli pistettynä maahan hänen päänpohjiinsa; mutta Abner ja väki makasivat hänen ympärillänsä.
\par 8 Ja Abisai sanoi Daavidille: "Jumala on tänä päivänä antanut vihamiehesi sinun käsiisi. Salli nyt minun keihästää hänet maahan; yksi isku vain, toista ei tarvita."
\par 9 Mutta Daavid vastasi Abisaille: "Älä surmaa häntä; sillä kuka on rankaisematta satuttanut kätensä Herran voideltuun?"
\par 10 Ja Daavid sanoi vielä: "Niin totta kuin Herra elää: Herra itse lyö hänet, tahi hänen kuolinpäivänsä tulee, tahi hän menee sotaan ja tuhoutuu siellä.
\par 11 Mutta minä en satuta kättäni Herran voideltuun. Pois se, Herra varjelkoon! Vaan ota nyt keihäs hänen päänpohjistaan ja vesiastia, ja sitten menkäämme."
\par 12 Ja Daavid otti keihään ja vesiastian Saulin päänpohjista, ja sitten he menivät tiehensä. Eikä kukaan nähnyt tai huomannut sitä, eikä kukaan herännyt, vaan he nukkuivat kaikki; sillä Herran lähettämä raskas uni oli vallannut heidät.
\par 13 Kun sitten Daavid oli tullut toiselle puolelle, asettui hän kauas vuoren laelle, pitkän välimatkan päähän.
\par 14 Ja Daavid huusi väelle ja Abnerille, Neerin pojalle, ja sanoi: "Etkö vastaa, Abner?" Abner vastasi ja sanoi: "Kuka sinä olet, joka niin huudat kuninkaalle?"
\par 15 Daavid sanoi Abnerille: "Sinähän olet mies, eikä sinun vertaistasi ole Israelissa. Miksi et sitten vartioinut herraasi, kuningasta? Sillä joku kansasta tuli surmaamaan kuningasta, sinun herraasi.
\par 16 Ei ole hyvin tehty, mitä sinä olet tehnyt. Niin totta kuin Herra elää: te olette kuoleman omat, koska ette vartioineet herraanne, Herran voideltua. Katsohan nyt: missä ovat kuninkaan keihäs ja vesiastia, jotka olivat hänen päänpohjissaan?"
\par 17 Niin Saul tunsi Daavidin äänen ja sanoi: "Sehän on sinun äänesi, poikani Daavid". Daavid vastasi: "Niin on, herrani, kuningas".
\par 18 Ja hän sanoi vielä: "Miksi herrani vainoaa palvelijaansa? Mitä minä olen tehnyt, tahi mitä pahaa minulla on tekeillä?
\par 19 Kuulkoon nyt herrani, kuningas, mitä hänen palvelijansa sanoo: Jos Herra on yllyttänyt sinut minua vastaan, niin saakoon hän tuntea ruokauhrin hajua; mutta jos sen ovat tehneet ihmiset, niin olkoot he kirotut Herran edessä, koska tänä päivänä karkoittavat minut pois Herran perintöosasta, sanoen: 'Mene ja palvele muita jumalia'.
\par 20 Älköön kuitenkaan minun vereni vuotako maahan siellä kaukana Herran kasvoista. Niin, Israelin kuningas on lähtenyt liikkeelle etsimään yhtä kirppua, niinkuin ajetaan peltokanaa vuorilla."
\par 21 Niin Saul sanoi: "Minä olen tehnyt syntiä. Tule takaisin, poikani Daavid; sillä minä en enää tee sinulle pahaa, koska minun henkeni tänä päivänä on ollut kallis sinun silmissäsi. Katso, minä olen tehnyt tyhmästi ja erehtynyt kovin pahoin."
\par 22 Daavid vastasi ja sanoi: "Tässä on kuninkaan keihäs; tulkoon joku miehistä tänne ottamaan sen.
\par 23 Herra palkitsee jokaiselle hänen vanhurskautensa ja uskollisuutensa. Sillä Herra antoi sinut tänä päivänä minun käsiini, mutta minä en tahtonut satuttaa kättäni Herran voideltuun.
\par 24 Katso, niinkuin sinun henkesi tänä päivänä on ollut suuriarvoinen minun silmissäni, niin on myös minun henkeni oleva suuriarvoinen Herran silmissä, ja hän on pelastava minut kaikesta hädästä."
\par 25 Saul sanoi Daavidille: "Siunattu ole sinä, poikani Daavid! Mihin sinä ryhdyt, siihen sinä pystyt." Sitten Daavid meni pois, ja Saul palasi kotiinsa.

\chapter{27}

\par 1 Mutta Daavid ajatteli mielessään: "Kerran Saulin käsi kuitenkin tuhoaa minut. Minulla ei ole muuta neuvoa kuin pelastautua filistealaisten maahan. Silloin Saul huomaa turhaksi enää etsiä minua kaikkialta Israelin alueelta, ja niin minä pelastun hänen käsistänsä."
\par 2 Niin Daavid nousi ja meni, hän ja ne kuusisataa miestä, jotka olivat hänen kanssaan, Aakiin, Maaokin pojan, Gatin kuninkaan, luo.
\par 3 Ja Daavid asettui miehinensä Aakiin luo Gatiin. Jokaisella oli perheensä mukanaan, Daavidilla molemmat vaimonsa, jisreeliläinen Ahinoam ja karmelilainen Abigail, Naabalin vaimo.
\par 4 Kun sitten Saulille ilmoitettiin, että Daavid oli paennut Gatiin, ei hän enää etsinyt häntä.
\par 5 Ja Daavid sanoi Aakiille: "Jos minä olen saanut armon sinun silmiesi edessä, niin annettakoon minulle paikka jossakin maaseutukaupungissa, asuakseni siellä. Sillä miksi sinun palvelijasi asuisi kuninkaan kaupungissa sinun luonasi?"
\par 6 Niin Aakis antoi hänelle sinä päivänä Siklagin. Sentähden on Siklag Juudan kuningasten oma vielä tänä päivänä.
\par 7 Aika, jonka Daavid asui filistealaisten maassa, oli vuosi ja neljä kuukautta.
\par 8 Daavid lähti miehinensä, ja he tekivät ryöstöretkiä gesurilaisten, geresiläisten ja amalekilaisten maahan. Sillä nämä asuivat siinä maassa, joka muinoin ulottui Suuriin päin ja Egyptin maahan asti.
\par 9 Ja hävittäessään maata Daavid ei jättänyt yhtään miestä eikä naista henkiin, otti lampaat, raavaat, aasit, kamelit ja vaatteet ja palasi sitten takaisin ja tuli Aakiin luo.
\par 10 Ja kun Aakis kysyi: "Ette kai ole tänään tehneet ryöstöretkeä?" vastasi Daavid: "Kyllä, Juudan Etelämaahan", tahi: "Jerahmeelilaisten Etelämaahan", tahi: "Keeniläisten Etelämaahan".
\par 11 Daavid ei jättänyt yhtään miestä eikä naista henkiin, vietäväksi Gatiin, sillä hän ajatteli: "Ne voisivat kertoa meistä ja sanoa: 'Näin on Daavid tehnyt, tämä on ollut hänen tapansa kaiken aikaa, minkä hän asui filistealaisten maassa'".
\par 12 Ja Aakis luotti Daavidiin ja ajatteli: "Hän on saattanut itsensä kansansa Israelin vihoihin ja tulee olemaan minun palvelijani ainiaan".

\chapter{28}

\par 1 Siihen aikaan filistealaiset kokosivat joukkonsa sotaretkelle Israelia vastaan. Ja Aakis sanoi Daavidille: "Tiedä, että sinun on miehinesi lähdettävä minun kanssani sotajoukon mukana".
\par 2 Daavid vastasi Aakiille: "Hyvä! Sinä tulet tietämään, mitä palvelijasi saa aikaan." Aakis sanoi Daavidille: "Siispä minä asetan sinut oman pääni vartijaksi koko täksi aikaa".
\par 3 Samuel oli kuollut, ja koko Israel oli pitänyt hänelle valittajaiset; ja he olivat haudanneet hänet hänen kaupunkiinsa Raamaan. Ja Saul oli toimittanut pois maasta vainaja- ja tietäjähenkien manaajat.
\par 4 Ja filistealaiset kokoontuivat, ja he tulivat ja leiriytyivät Suunemiin. Niin Saulkin kokosi koko Israelin, ja he leiriytyivät Gilboaan.
\par 5 Mutta kun Saul näki filistealaisten leirin, peljästyi hän, ja hänen sydämensä vapisi kovin.
\par 6 Ja Saul kysyi Herralta, mutta Herra ei vastannut hänelle, ei unissa, ei uurimin eikä profeettain kautta.
\par 7 Niin Saul sanoi palvelijoillensa: "Etsikää minulle joku vaimo, jolla on vallassaan vainajahenki, niin minä menen hänen luoksensa ja kysyn häneltä". Hänen palvelijansa vastasivat hänelle: "Katso, Een-Doorissa on vaimo, jolla on vallassaan vainajahenki".
\par 8 Silloin Saul teki itsensä tuntemattomaksi, pukeutui toisiin vaatteisiin ja lähti matkalle, mukanansa kaksi miestä. He tulivat yöllä vaimon luo, ja hän sanoi: "Ennusta minulle vainajahengen avulla ja nostata minulle se, jonka minä sinulle sanon".
\par 9 Mutta vaimo vastasi hänelle: "Sinä tiedät kyllä, mitä Saul on tehnyt, kuinka hän on hävittänyt vainaja- ja tietäjähenkien manaajat maasta. Miksi virität minulle ansan tuottaaksesi minulle kuoleman?"
\par 10 Niin Saul vannoi hänelle Herran kautta ja sanoi: "Niin totta kuin Herra elää: tästä asiasta ei sinulle tule mitään syytä".
\par 11 Vaimo kysyi: "Kenen minä nostatan sinulle?" Hän vastasi: "Nostata minulle Samuel".
\par 12 Mutta kun vaimo näki Samuelin, huudahti hän kovalla äänellä. Ja vaimo sanoi Saulille: "Miksi olet pettänyt minut? Sinähän olet Saul."
\par 13 Kuningas sanoi hänelle: "Älä pelkää. Mitä sinä näet?" Vaimo vastasi Saulille: "Minä näen jumal'olennon nousevan maasta".
\par 14 Hän kysyi häneltä: "Minkä näköinen hän on?" Vaimo vastasi: "Vanha mies nousee ylös, viittaan verhoutuneena". Niin Saul ymmärsi, että se oli Samuel, ja kumartui kasvoilleen maahan ja osoitti kunnioitusta.
\par 15 Ja Samuel sanoi Saulille: "Miksi sinä olet häirinnyt minua ja nostattanut minut?" Saul vastasi: "Minä olen suuressa hädässä: filistealaiset sotivat minua vastaan, ja Jumala on poistunut minusta eikä vastaa minulle enää, ei profeettain kautta eikä unissa. Niin minä kutsutin sinut, että ilmoittaisit minulle, mitä minun on tehtävä."
\par 16 Mutta Samuel vastasi: "Miksi sinä minulta kysyt, kun Herra on poistunut sinusta ja tullut vihamieheksesi?
\par 17 Herra on tehnyt sen, minkä hän on minun kauttani puhunut: Herra on reväissyt kuninkuuden sinun kädestäsi ja on antanut sen toiselle, Daavidille.
\par 18 Kun sinä et kuullut Herran ääntä etkä tehnyt Amalekille, mitä hänen hehkuva vihansa vaati, sentähden Herra on tänä päivänä tehnyt sinulle tämän.
\par 19 Herra antaa myöskin Israelin yhdessä sinun kanssasi filistealaisten käsiin, ja huomenna olet sinä poikinesi minun tykönäni; myös Israelin leirin Herra antaa filistealaisten käsiin."
\par 20 Niin Saul siinä tuokiossa kaatui pitkin koko pituuttaan maahan, sillä hän peljästyi kovin Samuelin puheesta; myös olivat häneltä voimat lopussa, sillä vuorokauteen hän ei ollut syönyt mitään.
\par 21 Mutta vaimo meni Saulin luo, ja kun hän näki, että tämä oli kauhun vallassa, sanoi hän hänelle: "Katso, palvelijattaresi kuuli sinua, minä panin henkeni kämmenelleni ja tottelin käskyä, jonka sinä minulle lausuit.
\par 22 Niin kuule nyt sinäkin palvelijatartasi ja salli minun asettaa sinun eteesi palanen leipää, ja syö, että olisit voimissasi, kun lähdet matkalle."
\par 23 Hän epäsi ja sanoi: "Minä en syö". Mutta kun hänen palvelijansa yhdessä vaimon kanssa pyytämällä pyysivät häntä, kuuli hän heitä; ja hän nousi maasta ja istui vuoteelle.
\par 24 Ja vaimolla oli juottovasikka kotonaan; sen hän teurasti joutuin. Sitten hän otti jauhoja, sotki ne ja leipoi niistä happamattomia leipiä.
\par 25 Ja hän toi ruuan Saulin ja hänen palvelijainsa eteen, ja he söivät. Sitten he nousivat ja lähtivät samana yönä.

\chapter{29}

\par 1 Filistealaiset kokosivat kaikki joukkonsa Afekiin, ja israelilaiset olivat leiriytyneet lähteelle, joka on Jisreelissä.
\par 2 Ja kun filistealaisten ruhtinaat tulivat satoineen ja tuhansineen, ja Daavid miehineen tuli viimeiseksi Aakiin kanssa,
\par 3 sanoivat filistealaisten päälliköt: "Mitä nuo hebrealaiset täällä tekevät?" Aakis vastasi filistealaisten päälliköille: "Tämähän on Daavid, Saulin, Israelin kuninkaan, palvelija, joka on ollut minun luonani toista vuotta, enkä minä ole havainnut hänessä mitään siitä päivästä alkaen, jona hän siirtyi minun luokseni, tähän päivään asti".
\par 4 Niin filistealaisten päälliköt vihastuivat häneen; ja filistealaisten päälliköt sanoivat hänelle: "Toimita tämä mies takaisin; palatkoon hän siihen paikkaan, jonka sinä olet hänelle määrännyt. Älköön hän tulko meidän kanssamme taisteluun, ettei hän taistelussa tulisi meidän vastustajaksemme. Sillä miten hän voisi hankkia itselleen Herransa suosion paremmin kuin näiden miesten päillä?
\par 5 Eikö tämä ole se Daavid, jolle he karkeloiden virittivät tämän laulun: 'Saul voitti tuhat, mutta Daavid kymmenentuhatta'?"
\par 6 Niin Aakis kutsui Daavidin ja sanoi hänelle: "Niin totta kuin Herra elää: sinä olet rehellinen, ja minulle olisi mieleen, että olisit täällä leirissä, lähtisit ja tulisit minun kanssani; sillä minä en ole havainnut mitään pahaa sinussa siitä päivästä alkaen, jona tulit minun luokseni, tähän päivään asti. Mutta ruhtinaille sinä et ole mieluinen.
\par 7 Lähde siis takaisin ja mene rauhassa, ettet tekisi mitään, joka ei olisi filistealaisten ruhtinaille mieleen."
\par 8 Daavid sanoi Aakiille: "Mitä minä olen tehnyt, ja mitä olet havainnut palvelijassasi siitä päivästä alkaen, jona minä tulin sinun palvelukseesi, tähän päivään asti, koska en saa tulla taistelemaan herrani, kuninkaan, vihollisia vastaan?"
\par 9 Aakis vastasi ja sanoi Daavidille: "Minä tiedän, että sinä olet minulle mieluinen niinkuin Jumalan enkeli; mutta filistealaisten päälliköt sanovat: 'Hän ei saa lähteä meidän kanssamme taisteluun'.
\par 10 Nouse siis huomenaamuna varhain, sinä ja sinun herrasi palvelijat, jotka ovat tulleet kanssasi, ja lähtekää matkaan huomenaamuna varhain, kun päivä valkenee."
\par 11 Niin Daavid miehinensä lähti varhain seuraavana aamuna matkalle palatakseen filistealaisten maahan. Mutta filistealaiset menivät Jisreeliin.

\chapter{30}

\par 1 Kun Daavid miehinensä kolmantena päivänä tuli Siklagiin, olivat amalekilaiset tehneet ryöstöretken Etelämaahan ja Siklagiin, ja he olivat vallanneet Siklagin ja polttaneet sen tulella.
\par 2 Naiset, mitä siellä oli, sekä pienet että suuret, he olivat ottaneet vangiksi, surmaamatta ketään; he olivat vieneet ne pois ja menneet matkoihinsa.
\par 3 Kun Daavid miehinensä tuli kaupunkiin, niin katso, se oli tulella poltettu, ja heidän vaimonsa, poikansa ja tyttärensä oli otettu vangiksi.
\par 4 Silloin Daavid ja väki, joka oli hänen kanssaan, korottivat äänensä ja itkivät, kunnes eivät enää jaksaneet itkeä.
\par 5 Vangiksi oli otettu myös Daavidin molemmat vaimot, jisreeliläinen Ahinoam ja Abigail, karmelilaisen Naabalin vaimo.
\par 6 Ja Daavid joutui suureen hätään, sillä kansa aikoi kivittää hänet: niin katkeroitunut oli koko kansa, kukin poikiensa ja tyttäriensä tähden. Mutta Daavid rohkaisi mielensä Herrassa, Jumalassansa.
\par 7 Ja Daavid sanoi pappi Ebjatarille, Ahimelekin pojalle: "Tuo minulle kasukka". Niin Ebjatar toi kasukan Daavidille.
\par 8 Ja Daavid kysyi Herralta: "Ajanko takaa tuota rosvojoukkoa? Saavutanko minä sen?" Hän vastasi hänelle: "Aja, sillä sinä saavutat sen ja pelastat pelastettavat".
\par 9 Niin Daavid lähti, hän ja ne kuusisataa miestä, jotka olivat hänen kanssaan, ja he tulivat Besorin purolle; siihen pysähtyivät ne, jotka olivat jääneet muista jälkeen.
\par 10 Mutta Daavid jatkoi takaa-ajoa neljänsadan miehen kanssa; sillä niitä, jotka väsyneinä pysähtyivät eivätkä menneet Besorin puron poikki, oli kaksisataa miestä.
\par 11 Ja he tapasivat kedolla egyptiläisen miehen ja toivat hänet Daavidin luo. He antoivat hänelle leipää syödä ja vettä juoda,
\par 12 ja he antoivat vielä hänelle viikunakakun ja kaksi rusinakakkua syödä. Silloin hän virkosi henkiin; sillä hän ei ollut syönyt eikä juonut kolmeen vuorokauteen.
\par 13 Ja Daavid kysyi häneltä: "Kenen olet miehiä ja mistä tulet?" Hän vastasi: "Minä olen egyptiläinen nuorukainen, erään amalekilaisen miehen palvelija; mutta minun herrani jätti minut, sillä minä sairastuin kolme päivää sitten.
\par 14 Me teimme ryöstöretken kreettien, Juudan ja Kaalebin Etelämaahan ja poltimme Siklagin tulella."
\par 15 Daavid sanoi hänelle: "Vietkö minut tuon rosvojoukon luo?" Hän vastasi: "Vanno minulle Jumalan kautta, ettet surmaa minua etkä luovuta minua herrani käsiin, niin minä vien sinut sen rosvojoukon luo".
\par 16 Ja hän vei hänet sinne; ja katso, heitä oli hajallaan maassa kaikkialla syömässä, juomassa ja juhlimassa kaikella sillä suurella saaliilla, jonka olivat ottaneet filistealaisten maasta ja Juudan maasta.
\par 17 Niin Daavid kaatoi heitä aamuhämärästä iltaan asti; eikä heistä pelastunut kuin neljäsataa nuorta miestä, jotka nousivat kamelien selkään ja pakenivat.
\par 18 Ja Daavid pelasti kaikki, mitä amalekilaiset olivat ottaneet; myös molemmat vaimonsa Daavid pelasti.
\par 19 Ei ketään puuttunut, ei pientä eikä suurta, ei kenenkään poikaa eikä kenenkään tytärtä eikä saalista tai muuta, mitä he olivat ottaneet itsellensä; kaiken Daavid toi takaisin.
\par 20 Daavid otti myös kaikki lampaat ja raavaat, ja niitä ajettiin muun karjan edellä ja huudettiin: "Tämä on Daavidin saalis!"
\par 21 Ja kun Daavid tuli niiden kahdensadan miehen luo, jotka olivat väsyneet, jaksamatta seurata Daavidia, ja jotka oli jätetty Besorin purolle, tulivat he Daavidia vastaan ja sitä väkeä, joka oli hänen kanssansa; ja kun Daavid väkineen lähestyi, tervehti hän heitä.
\par 22 Mutta kaikenlaiset huonot ja kelvottomat miehet niiden joukosta, jotka olivat kulkeneet Daavidin mukana, rupesivat sanomaan: "Koska nämä eivät ole kulkeneet meidän mukanamme, emme me anna heille mitään saaliista, jonka pelastimme, paitsi kunkin vaimon ja lapset; viekööt ne ja menkööt matkoihinsa".
\par 23 Mutta Daavid sanoi: "Älkää tehkö niin, veljeni, sen jälkeen, mitä Herra on meille suonut, kun hän varjeli meidät ja antoi meidän käsiimme rosvojoukon, joka hyökkäsi meidän kimppuumme.
\par 24 Ja kukapa teitä kuulisi tässä asiassa? Sillä taisteluun menevällä ja kuormaston luo jäävällä on oleva yhtäläinen osuus; heidän on jaettava tasan."
\par 25 Ja sillensä asia jäi silloin ja sen jälkeenkin: hän teki sen säännöksi ja tavaksi Israelissa, aina tähän päivään asti.
\par 26 Kun Daavid tuli Siklagiin, lähetti hän osan saaliista Juudan vanhimmille, ystävillensä, sanoen: "Tässä on teille tervehdyslahja siitä saaliista, joka otettiin Herran vihollisilta" -
\par 27 lahja niille, jotka olivat Beetelissä, Etelämaan Raamotissa, Jattirissa,
\par 28 Aroerissa, Sifmotissa, Estemoassa,
\par 29 Raakalissa, jerahmeelilaisten kaupungeissa, keeniläisten kaupungeissa,
\par 30 Hormassa, Boor-Aasanissa, Atakissa
\par 31 ja Hebronissa, ja samoin lahja kaikille muille paikkakunnille, joissa Daavid miehinensä oli kulkenut.

\chapter{31}

\par 1 Mutta filistealaiset taistelivat Israelia vastaan. Ja Israelin miehet pakenivat filistealaisia, ja heitä kaatui surmattuina Gilboan vuorella.
\par 2 Ja filistealaiset pääsivät Saulin ja hänen poikiensa kintereille, ja filistealaiset surmasivat Joonatanin, Abinadabin ja Malkisuan, Saulin pojat.
\par 3 Ja kun taistelu kiihtyi ankaraksi Saulia vastaan ja jousimiehet keksivät hänet, joutui hän suureen hätään jousimiesten ahdistaessa.
\par 4 Ja Saul sanoi aseenkantajallensa: "Paljasta miekkasi ja lävistä sillä minut, etteivät nuo ympärileikkaamattomat tulisi lävistämään minua ja pitämään minua pilkkanaan". Mutta hänen aseenkantajansa ei tahtonut, sillä hän pelkäsi kovin. Niin Saul itse otti miekan ja heittäytyi siihen.
\par 5 Mutta kun hänen aseenkantajansa näki, että Saul oli kuollut, heittäytyi hänkin miekkaansa ja kuoli hänen kanssansa.
\par 6 Niin kuolivat sinä päivänä yhdessä Saul, hänen kolme poikaansa ja hänen aseenkantajansa ynnä kaikki hänen miehensä.
\par 7 Ja kun Israelin miehet, jotka asuivat tuolla puolella laakson ja tuolla puolella Jordanin, huomasivat, että Israelin miehet olivat paenneet ja että Saul poikinensa oli kuollut, jättivät he kaupungit ja pakenivat, ja filistealaiset tulivat ja asettuivat niihin.
\par 8 Seuraavana päivänä filistealaiset tulivat ryöstämään surmattuja ja löysivät Saulin ja hänen kolme poikaansa kaatuneina Gilboan vuorella.
\par 9 Niin he löivät häneltä pään poikki, ryöstivät häneltä aseet ja lähettivät ne ympäri filistealaisten maata, julistaaksensa voitonsanomaa epäjumaliensa temppeleissä ja kansan seassa.
\par 10 Ja he asettivat hänen aseensa Astarten temppeliin, mutta hänen ruumiinsa he kiinnittivät Beet-Seanin muurille.
\par 11 Kun Gileadin Jaabeksen asukkaat kuulivat, mitä filistealaiset olivat tehneet Saulille,
\par 12 nousivat he, kaikki sotakuntoiset miehet, kulkivat kaiken yötä ja ottivat Saulin ja hänen poikiensa ruumiit alas Beet-Seanin muurilta, menivät Jaabekseen ja polttivat ne siellä.
\par 13 Sitten he ottivat heidän luunsa, hautasivat ne tamariskipuun alle Jaabekseen ja paastosivat seitsemän päivää.


\end{document}