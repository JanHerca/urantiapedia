\begin{document}

\title{Toinen kirje tessalonikalaisille}


\chapter{1}

\par 1 Paavali ja Silvanus ja Timoteus tessalonikalaisten seurakunnalle Jumalassa, meidän Isässämme, ja Herrassa Jeesuksessa Kristuksessa.
\par 2 Armo teille ja rauha Isältä Jumalalta ja Herralta Jeesukselta Kristukselta!
\par 3 Me olemme velvolliset aina kiittämään Jumalaa teidän tähtenne, veljet, niinkuin oikein onkin, koska teidän uskonne runsaasti kasvaa ja keskinäinen rakkautenne lisääntyy itsekussakin, kaikissa teissä,
\par 4 niin että me itsekin Jumalan seurakunnissa kerskaamme teistä, teidän kärsivällisyydestänne ja uskostanne kaikissa vainoissanne ja ahdistuksissa, joita teillä on kestettävänä
\par 5 ja jotka ovat osoituksena Jumalan vanhurskaasta tuomiosta, että teidät katsottaisiin arvollisiksi Jumalan valtakuntaan, jonka tähden kärsittekin,
\par 6 koskapa Jumala katsoo oikeaksi kostaa ahdistuksella niille, jotka teitä ahdistavat,
\par 7 ja antaa teille, joita ahdistetaan, levon yhdessä meidän kanssamme, kun Herra Jeesus ilmestyy taivaasta voimansa enkelien kanssa
\par 8 tulen liekissä ja kostaa niille, jotka eivät tunne Jumalaa eivätkä ole kuuliaisia meidän Herramme Jeesuksen evankeliumille.
\par 9 Heitä kohtaa silloin rangaistukseksi iankaikkinen kadotus Herran kasvoista ja hänen voimansa kirkkaudesta,
\par 10 kun hän sinä päivänä tulee, että hän kirkastuisi pyhissänsä ja olisi ihmeteltävä kaikissa uskovissa, sillä te olette uskoneet meidän todistuksemme.
\par 11 Sitä varten me aina rukoilemmekin teidän puolestanne, että meidän Jumalamme katsoisi teidät kutsumisensa arvoisiksi ja voimallisesti saattaisi täydelliseksi kaiken teidän halunne hyvään ja teidän uskonne teot,
\par 12 että meidän Herramme Jeesuksen nimi teissä kirkastuisi ja te hänessä, meidän Jumalamme ja Herran Jeesuksen Kristuksen armon mukaan.

\chapter{2}

\par 1 Mutta mitä tulee meidän Herramme Jeesuksen Kristuksen tulemukseen ja meidän kokoontumiseemme hänen tykönsä, niin me pyydämme teitä, veljet,
\par 2 ettette anna minkään hengen ettekä sanan ettekä minkään muka meidän lähettämämme kirjeen heti järkyttää itseänne, niin että menetätte mielenne maltin, ettekä anna niiden itseänne peljästyttää, ikäänkuin Herran päivä jo olisi käsissä.
\par 3 Älkää antako kenenkään vietellä itseänne millään tavalla. Sillä se päivä ei tule, ennenkuin luopumus ensin tapahtuu ja laittomuuden ihminen ilmestyy, kadotuksen lapsi,
\par 4 tuo vastustaja, joka korottaa itsensä yli kaiken, mitä jumalaksi tai jumaloitavaksi kutsutaan, niin että hän asettuu Jumalan temppeliin ja julistaa olevansa Jumala.
\par 5 Ettekö muista, että minä, kun vielä olin teidän tykönänne, sanoin tämän teille?
\par 6 Ja nyt te tiedätte, mikä pidättää, niin että hän vasta ajallansa ilmestyy.
\par 7 Sillä laittomuuden salaisuus on jo vaikuttamassa; jahka vain tulee tieltä poistetuksi se, joka nyt vielä pidättää,
\par 8 niin silloin ilmestyy tuo laiton, jonka Herra Jeesus on surmaava suunsa henkäyksellä ja tuhoava tulemuksensa ilmestyksellä,
\par 9 tuo, jonka tulemus tapahtuu saatanan vaikutuksesta valheen kaikella voimalla ja tunnusteoilla ja ihmeillä
\par 10 ja kaikilla vääryyden viettelyksillä niille, jotka joutuvat kadotukseen, sentähden etteivät ottaneet vastaan rakkautta totuuteen, voidaksensa pelastua.
\par 11 Ja sentähden Jumala lähettää heille väkevän eksytyksen, niin että he uskovat valheen,
\par 12 että kaikki ne tuomittaisiin, jotka eivät ole uskoneet totuutta, vaan mielistyneet vääryyteen.
\par 13 Mutta me olemme velvolliset aina kiittämään Jumalaa teidän tähtenne, veljet, te Herran rakastetut, sentähden että Jumala alusta alkaen valitsi teidät pelastukseen Hengen pyhityksessä ja uskossa totuuteen.
\par 14 Siihen hän on myös kutsunut teidät meidän evankeliumimme kautta, omistamaan meidän Herramme Jeesuksen Kristuksen kirkkauden.
\par 15 Niin seisokaa siis, veljet, lujina ja pitäkää kiinni niistä opetuksista, joita olette oppineet joko meidän puheestamme tai kirjeestämme.
\par 16 Ja meidän Herramme Jeesus Kristus itse ja Jumala, meidän Isämme, joka on rakastanut meitä ja armossa antanut meille iankaikkisen lohdutuksen ja hyvän toivon,
\par 17 lohduttakoon teidän sydämiänne ja vahvistakoon teitä kaikessa hyvässä työssä ja puheessa.

\chapter{3}

\par 1 Sitten vielä, veljet, rukoilkaa meidän edestämme, että Herran sana nopeasti leviäisi ja tulisi kirkastetuksi muuallakin niinkuin teidän keskuudessanne,
\par 2 ja että me pelastuisimme nurjista ja häijyistä ihmisistä; sillä usko ei ole joka miehen.
\par 3 Mutta Herra on uskollinen, ja hän on vahvistava teitä ja varjeleva teidät pahasta.
\par 4 Ja me luotamme teihin Herrassa, että te sekä nyt että vasta teette, mitä me käskemme.
\par 5 Ja Herra ohjatkoon teidän sydämenne Jumalan rakkauteen ja Kristuksen kärsivällisyyteen.
\par 6 Mutta Herran Jeesuksen Kristuksen nimessä me käskemme teitä, veljet, vetäytymään pois jokaisesta veljestä, joka vaeltaa kurittomasti eikä sen opetuksen mukaan, jonka olette meiltä saaneet.
\par 7 Tiedättehän itse, kuinka meidän jälkiämme on seurattava, sillä me emme ole olleet kurittomia teidän keskuudessanne,
\par 8 emmekä ilmaiseksi syöneet kenenkään leipää, vaan työssä ja vaivassa me ahkeroitsimme yöt ja päivät, ettemme olisi kenellekään teistä rasitukseksi;
\par 9 ei niin, ettei meillä olisi siihen valtaa, vaan me tahdomme olla teille esikuvaksi, että te kulkisitte meidän jälkiämme.
\par 10 Sillä jo silloin, kun olimme teidän tykönänne, me sääsimme teille, että kuka ei tahdo työtä tehdä, ei hänen syömänkään pidä.
\par 11 Sillä me olemme kuulleet, että muutamat teidän keskuudessanne vaeltavat kurittomasti, eivät tee työtä, vaan puuhailevat sellaisessa, mikä ei heille kuulu.
\par 12 Semmoisia me käskemme ja kehoitamme Herrassa Jeesuksessa Kristuksessa, tekemään työtä hiljaisuudessa ja syömään omaa leipäänsä.
\par 13 Mutta te, veljet, älkää väsykö tekemästä sitä, mikä hyvää on.
\par 14 Mutta jos kuka ei tottele sitä, mitä me tässä kirjeessä olemme sanoneet, niin merkitkää hänet älkääkä seurustelko hänen kanssaan, että hän häpeäisi.
\par 15 Älkää kuitenkaan pitäkö häntä vihollisena, vaan neuvokaa niinkuin veljeä.
\par 16 Mutta itse rauhan Herra antakoon teille rauhan, aina ja kaikella tavalla. Herra olkoon kaikkien teidän kanssanne.
\par 17 Tervehdys minulta, Paavalilta, omakätisesti. Tämä on merkkinä jokaisessa kirjeessäni; näin minä kirjoitan.
\par 18 Meidän Herramme Jeesuksen Kristuksen armo olkoon kaikkien teidän kanssanne.


\end{document}