\begin{document}

\title{Habakukin kirja}


\chapter{1}

\par 1 Ennustus, jonka profeetta Habakuk näki.
\par 2 Kuinka kauan, Herra, minun täytyy apua huutaa, ja sinä et kuule, parkua sinulle: "Väkivaltaa!" ja sinä et auta?
\par 3 Minkätähden sinä annat minun nähdä vääryyttä ja itse katselet turmiota? Minun edessäni on hävitys ja väkivalta; on syntynyt riita, ja on noussut tora.
\par 4 Sentähden on laki heikko, ja oikeus ei tule milloinkaan voimaan. Sillä jumalaton saartaa vanhurskaan; sentähden oikeus vääristetään.
\par 5 Katsokaa kansojen joukkoon, katselkaa ja kauhistukaa, sillä minä teen teidän päivinänne teon, jota ette uskoisi, jos siitä kerrottaisiin.
\par 6 Sillä katso, minä nostan kaldealaiset, tuiman ja rajun kansan, joka kulkee maata lavealti ja ottaa omaksensa asuinsijat, jotka eivät ole sen.
\par 7 Se on hirmuinen ja peljättävä, siitä itsestään tulee sen oikeus ja sen korkeus.
\par 8 Sen hevoset ovat nopeammat kuin pantterit, ne juoksevat kiivaammin kuin sudet illoin. Sen ratsumiehet kiidättävät - kaukaa tulevat sen ratsumiehet, ne lentävät, niinkuin kotka syöksyy syönnökselleen.
\par 9 Se tulee kaikkinensa väkivallan tekoon, heidän kasvojensa suunta on suoraan eteenpäin; se kokoaa vankeja kuin hietaa.
\par 10 Pilkkanaan pitää se kuninkaat ja naurunansa ruhtinaat. Se nauraa kaikille varustuksille, kasaa kokoon hiekkaa ja valloittaa ne.
\par 11 Sitten se tuulena kiitää ja hyökkää, mutta joutuu syynalaiseksi, tuo, jolla oma voimansa on jumalana.
\par 12 Etkö sinä ole ikiajoista asti Herra, minun pyhä Jumalani? Me emme kuole! Sinä, Herra, olet pannut sen tuomioksi; sinä, kallio, olet asettanut sen kuritukseksi.
\par 13 Sinun silmäsi ovat puhtaat, niin ettet voi katsoa pahaa etkä saata katsella turmiota. Minkätähden sinä katselet uskottomia, olet vaiti, kun jumalaton nielee hurskaampansa,
\par 14 teet ihmiset samankaltaisiksi kuin meren kalat, kuin matelevaiset, joilla ei ole hallitsijaa?
\par 15 Se nostaa heidät kaikki ylös koukulla, vetää heidät pyydyksessään ja kokoaa heidät verkkoonsa. Sentähden se iloitsee ja riemuitsee.
\par 16 Sentähden se uhraa pyydykselleen ja polttaa uhreja verkollensa, sillä niitten turvin sen osa on rasvainen, sen ruoka lihava.
\par 17 Saako se sentähden tyhjentää pyydyksensä, aina surmata kansoja säälimättä?

\chapter{2}

\par 1 Minä seison vartiopaikallani, asetun varustukseen ja tähystän, nähdäkseni, mitä hän minulle puhuu, mitä hän valitukseeni vastaa.
\par 2 Ja Herra vastasi minulle ja sanoi: "Kirjoita näky ja piirrä selvästi tauluihin, niin että sen voi juostessa lukea".
\par 3 Sillä näky odottaa vielä aikaansa, mutta se rientää määränsä päähän, eikä se petä. Jos se viipyy, odota sitä; sillä varmasti se toteutuu, eikä se myöhästy.
\par 4 Katso, sen kansan sielu on kopea eikä ole suora; mutta vanhurskas on elävä uskostansa.
\par 5 Ja vielä: viini on petollista, urho on röyhkeä; mutta hän ei ole pysyväinen, hän, joka levittää kitansa kuin tuonela ja on täyttymätön kuin kuolema, kokoaa itselleen kaikki kansakunnat ja kerää itselleen kaikki kansat.
\par 6 Mutta ne kaikki virittävät hänestä pilkkalaulun, ivarunon, arvoituslaulun; he sanovat: "Voi sitä, joka hankkii paljon omaisuutta, mikä ei ole hänen - ja kuinkahan pitkäksi aikaa? - ja kasaa päällensä pantteja!
\par 7 Eivätkö nouse äkisti sinun vaivaajasi, eivätkö heräjä sinun ravistajasi? Ja sinä joudut niiden ryöstettäväksi.
\par 8 Sillä niinkuin sinä olet riistänyt monia kansakuntia, niin riistävät sinua kaikki jäljellejääneet kansat ihmisveren tähden ja väkivallan tähden, joka on tehty maalle, kaupungille ja kaikille sen asukkaille.
\par 9 Voi sitä, joka kiskoo väärää voittoa huoneensa hyväksi laittaakseen pesänsä korkealle, pelastuakseen kovan onnen kourista!
\par 10 Sinä olet neuvoillasi tuottanut häpeätä huoneellesi, tuhoamalla paljon kansoja, ja olet oman henkesi rikkonut.
\par 11 Sillä kivi seinästä on huutava ja orsi katosta on vastaava.
\par 12 Voi sitä, joka rakentaa kaupungin verivelkojen varaan ja perustaa linnan vääryyden varaan!
\par 13 Katso, eikö tule se Herralta Sebaotilta, että kansat näkevät vaivaa tulen hyväksi ja kansakunnat väsyttävät itseänsä tyhjän hyväksi?
\par 14 Sillä maa on oleva täynnä Herran kunnian tuntemusta, niinkuin vedet peittävät meren.
\par 15 Voi sitä, joka juottaa lähimmäistänsä, joka sekoittaa juoman sinun vihallasi ja juovuttaa hänet nähdäkseen hänen alastomuutensa!
\par 16 Sinä ravitset itsesi häpeällä, et kunnialla: niinpä juo sinäkin ja paljasta ympärileikkaamattomuutesi. Sinun kohdallesi on kiertyvä Herran oikean käden malja ja häväistys kunnian sijaan.
\par 17 Sillä sinut on peittävä Libanonille tekemäsi väkivalta ja eläimiä kohdannut hävitys, joka on ne säikyttänyt, ihmisveren tähden ja väkivallan tähden, joka on tehty maalle, kaupungille ja kaikille sen asukkaille.
\par 18 Mitä hyötyä on veistetystä kuvasta, että sen valmistaja veistää sen, ja valetusta kuvasta ja valheen opettajasta, että sen valmistaja siihen luottaa ja tekee mykkiä epäjumalia?
\par 19 Voi sitä, joka sanoo puulle: 'Heräjä!' ja mykälle kivelle: 'Nouse!' Sekö voisi opettaa? Katso, se on silattu kullalla ja hopealla, ja henkeä siinä ei ole, ei ensinkään.
\par 20 Mutta Herra on pyhässä temppelissänsä, hänen edessään vaietkoon kaikki maa."

\chapter{3}

\par 1 Profeetta Habakukin rukous; virren tapaan.
\par 2 Herra, minä olen kuullut sinulta sanoman, ja olen peljästynyt. Herra, herätä eloon tekosi jo kesken vuotten, tee se jo kesken vuotten tiettäväksi. Muista vihassasi laupeutta.
\par 3 Jumala tulee Teemanista, Pyhä Paaranin vuorelta. Sela. Hänen valtasuuruutensa peittää taivaat, hänen ylistystänsä on maa täynnä.
\par 4 Hänen hohteensa on kuin aurinko, hänestä käyvät säteet joka taholle; se on hänen voimansa verho.
\par 5 Hänen edellänsä käy rutto ja polttotauti tulee hänen jäljessänsä.
\par 6 Hän seisahtuu ja mittaa maan, hän katsahtaa ja saa kansat vapisemaan. Ikivuoret särkyvät, ikuiset kukkulat vaipuvat, hänen polkunsa ovat iankaikkiset.
\par 7 Vaivan alaisina minä näen Kuusanin majat, Midianin maan telttavaatteet vapisevat.
\par 8 Virtoihinko Herra on vihastunut? Kohtaako sinun vihasi virtoja, merta sinun kiivastuksesi, koska ajat hevosillasi, pelastuksesi vaunuilla?
\par 9 Paljas, paljastettu on sinun jousesi: valat, vitsat, sana. Sela. Virroilla sinä halkaiset maan.
\par 10 Vuoret näkevät sinut ja järkkyvät, rankkasade purkaa vettä, syvyys antaa äänensä, kohottaa kätensä korkealle.
\par 11 Aurinko ja kuu astuvat majaansa sinun kiitävien nuoltesi valossa, sinun keihääsi salaman hohteessa.
\par 12 Kiivastuksessa sinä astut maata, puit kansoja vihassa.
\par 13 Sinä olet lähtenyt auttamaan kansaasi, auttamaan voideltuasi. Sinä murskaat pään jumalattoman huoneesta, paljastat perustukset kaulan tasalle. Sela.
\par 14 Sinä lävistät heidän omilla keihäillään pään heidän johtajiltansa, jotka hyökkäävät hajottamaan minua - se on heidän ilonsa - aivan kuin pääsisivät syömään kurjaa salassa.
\par 15 Sinä ajat hevosillasi merta, paljojen vetten kuohua.
\par 16 Minä kuulin tämän, ja minun ruumiini vapisee, minun huuleni värisevät huudosta, mätä menee minun luihini, minä seison paikallani ja tutisen, kun minun täytyy hiljaa odottaa ahdistuksen päivää, jolloin kansan kimppuun käy hyökkääjä.
\par 17 Sillä ei viikunapuu kukoista, eikä viiniköynnöksissä ole rypäleitä; öljypuun sato pettää, eivätkä pellot tuota syötävää. Lampaat ovat kadonneet tarhasta, eikä ole karjaa vajoissa.
\par 18 Mutta minä riemuitsen Herrassa, iloitsen autuuteni Jumalassa.
\par 19 Herra, Herra on minun voimani. Hän tekee minun jalkani nopsiksi niinkuin peurat ja antaa minun käydä kukkuloillani. Veisuunjohtajalle; minun kielisoittimillani.


\end{document}