\begin{document}

\title{Danielin kirja}


\chapter{1}

\par 1 Joojakimin, Juudan kuninkaan, kolmantena hallitusvuotena tuli Nebukadnessar, Baabelin kuningas, Jerusalemin edustalle ja piiritti sitä.
\par 2 Ja Herra antoi hänen käsiinsä Joojakimin, Juudan kuninkaan, sekä osan Jumalan huoneen astioita, ja hän vei ne Sinearin maahan, jumalansa huoneeseen. Ja hän vei astiat jumalansa aarrekammioon.
\par 3 Ja kuningas käski ylimmäisen hoviherransa Aspenaan tuoda israelilaisia poikia, jotka olivat kuninkaallista sukua tai ylimysperheistä,
\par 4 nuorukaisia, joissa ei ollut mitään virheä ja jotka olivat kaunismuotoisia, jotka kykenivät käsittämään kaikkea viisautta, olivat terävä-älyisiä ja hyväoppisia ja jotka olisivat kelvollisia palvelemaan kuninkaan hovissa; heille hänen tuli opettaa kaldealaisten kirjoitusta ja kieltä.
\par 5 Ja kuningas määräsi heille jokapäiväiseksi ravinnoksi ruokaa kuninkaan pöydästä ja viiniä, jota hän itse joi. Niin heitä oli kasvatettava kolme vuotta, ja niiden kuluttua heidän tuli astua kuninkaan palvelukseen.
\par 6 Heidän joukossaan olivat juutalaiset pojat Daniel, Hananja, Miisael ja Asarja.
\par 7 Ja hoviherrain päällikkö pani heille nimet: Danielille hän pani nimen Beltsassar, Hananjalle nimen Sadrak, Miisaelille nimen Meesak ja Asarjalle nimen Abednego.
\par 8 Mutta Daniel päätti lujasti olla saastuttamatta itseään kuninkaan pöydän ruualla ja viinillä, jota tämä joi, ja anoi hoviherrain päälliköltä, ettei hänen tarvitsisi itseään saastuttaa.
\par 9 Ja Jumala salli Danielin saada suosion ja armon hoviherrain päällikön edessä.
\par 10 Mutta hoviherrain päällikkö sanoi Danielille: "Minä pelkään, että jos herrani, kuningas, joka on määrännyt teidän ruokanne ja juomanne, huomaa teidän kasvonne laihemmiksi kuin muiden ikäistenne nuorukaisten, niin te saatatte minun pääni vaaraan kuninkaan edessä".
\par 11 Silloin Daniel sanoi katsastajalle, jonka hoviherrain päällikkö oli määrännyt pitämään silmällä Danielia, Hananjaa, Miisaelia ja Asarjaa:
\par 12 "Koettele palvelijoitasi kymmenen päivää, ja annettakoon meille vihannesruokaa syödäksemme ja vettä juodaksemme.
\par 13 Sitten tarkastettakoon sinun edessäsi, miltä me näytämme ja miltä näyttävät ne nuorukaiset, jotka syövät kuninkaan pöydän ruokaa; ja tee sitten palvelijoillesi sen mukaan, mitä silloin havaitset."
\par 14 Ja hän kuuli heitä tässä asiassa ja koetteli heitä kymmenen päivää.
\par 15 Mutta kymmenen päivän kuluttua havaittiin heidät muodoltaan kauniimmiksi ja ruumiiltaan lihavammiksi kuin yksikään niistä nuorukaisista, jotka söivät kuninkaan pöydän ruokaa.
\par 16 Ja niin katsastaja jätti pois heille määrätyn ruuan ja heidän juotavansa viinin ja antoi heille vihannesruokaa.
\par 17 Ja Jumala antoi näille neljälle nuorukaiselle taidon käsittää kaikki kirjoitukset ja kaiken viisauden; ja Daniel ymmärsi myös kaikkinaiset näyt ja unet.
\par 18 Kun sitten ne päivät olivat kuluneet, joiden jälkeen kuningas oli käskenyt tuoda heidät esiin, niin hoviherrain päällikkö toi heidät Nebukadnessarin eteen.
\par 19 Kun nyt kuningas keskusteli heidän kanssaan, ei ollut heidän joukossansa yhtäkään Danielin, Hananjan, Miisaelin ja Asarjan vertaista. Niin he tulivat kuninkaan palvelijoiksi.
\par 20 Ja kaikissa viisautta ja ymmärrystä vaativissa asioissa, joita kuningas heiltä kyseli, hän havaitsi heidät kymmentä vertaa etevämmiksi kuin kaikki tietäjät ja noidat, mitä koko hänen valtakunnassaan oli.
\par 21 Ja Daniel oli siellä kuningas Kooreksen ensimmäiseen vuoteen asti.

\chapter{2}

\par 1 Nebukadnessarin toisena hallitusvuotena näki Nebukadnessar unia, ja hänen mielensä oli levoton eikä hän enää saanut unta.
\par 2 Silloin kuningas käski kutsua tietäjät ja noidat, velhot ja kaldealaiset ilmoittamaan kuninkaalle hänen unensa; ja he tulivat ja astuivat kuninkaan eteen.
\par 3 Ja kuningas sanoi heille: "Minä olen nähnyt unen, ja mieleni on levoton, kunnes saan tietää sen unen.
\par 4 Silloin kaldealaiset puhuivat kuninkaalle araminkielellä: "Eläköön kuningas iankaikkisesti! Sano uni palvelijoillesi, niin me ilmoitamme sen selityksen."
\par 5 Kuningas vastasi ja sanoi kaldealaisille: "Tämä minun sanani on peruuttamaton: Ellette ilmoita minulle unta ja sen selitystä, niin teidät hakataan kappaleiksi ja teidän talonne tehdään soraläjiksi.
\par 6 Mutta jos te ilmoitatte minulle unen ja sen selityksen, niin te saatte minulta lahjoja ja antimia ja suuren kunnian. Sentähden ilmoittakaa minulle uni ja sen selitys."
\par 7 He vastasivat toistamiseen ja sanoivat: "Kuningas sanokoon unen palvelijoilleen, niin me ilmoitamme sen selityksen".
\par 8 Kuningas vastasi ja sanoi: "Minä huomaan selvästi, että te vain koetatte voittaa aikaa, koska näette peruuttamattomaksi tämän minun sanani,
\par 9 että ellette ilmoita minulle unta, on teillä edessä vain yksi tuomio. Sillä te olette sopineet keskenänne, että puhutte minun edessäni valheellista ja turmiollista puhetta toivoen, että aika muuttuu. Sentähden sanokaa minulle uni; silloin minä tiedän, että te osaatte ilmoittaa minulle selityksen siihen."
\par 10 Kaldealaiset vastasivat kuninkaan edessä ja sanoivat: "Ei ole maan päällä ihmistä, joka kykenisi selittämään sen, mitä kuningas sanoi. Eikä yksikään suuri ja voimallinen kuningas ole koskaan vaatinut tämänkaltaista asiaa keneltäkään tietäjältä, noidalta tai kaldealaiselta.
\par 11 Sillä asia, jota kuningas vaatii, on vaikea, eikä ole ketään, joka voisi sen kuninkaalle selittää, paitsi jumalat, joiden asuinsija ei ole ihmisten tykönä."
\par 12 Tästä kuningas suuttui ja julmistui kovin ja käski tappaa kaikki Baabelin viisaat.
\par 13 Kun tästä oli käsky annettu ja viisaat piti tapettaman, etsittiin Danielia ja hänen tovereitaan tapettaviksi.
\par 14 Silloin Daniel antoi viisaan ja taitavan vastauksen Arjokille, kuninkaan henkivartioväen päällikölle, joka oli lähtenyt tappamaan Baabelin viisaita.
\par 15 Hän vastasi ja sanoi Arjokille, kuninkaan päällikölle: "Miksi on kuningas antanut niin ankaran käskyn?" Silloin Arjok kertoi Danielille asian.
\par 16 Niin Daniel meni palatsiin ja pyysi kuningasta antamaan hänelle aikaa, että hän saisi ilmoittaa kuninkaalle selityksen.
\par 17 Sitten Daniel meni kotiinsa ja kertoi asian tovereillensa Hananjalle, Miisaelille ja Asarjalle
\par 18 kehoittaen heitä rukoilemaan armoa taivaan Jumalalta tämän salatun asian tähden, ettei Danielia ja hänen tovereitansa tapettaisi muiden Baabelin viisaitten kanssa.
\par 19 Silloin salaisuus ilmoitettiin Danielille yöllisessä näyssä. Niin Daniel kiitti taivaan Jumalaa.
\par 20 Daniel lausui ja sanoi: "Olkoon Jumalan nimi kiitetty iankaikkisesta iankaikkiseen, sillä hänen on viisaus ja voima.
\par 21 Hän muuttaa ajat ja hetket, hän syöksee kuninkaat vallasta ja korottaa kuninkaat valtaan, hän antaa viisaille viisauden ja taidollisille ymmärryksen.
\par 22 Hän paljastaa syvät ja salatut asiat, hän tietää, mitä pimeydessä on, ja valkeus asuu hänen tykönänsä.
\par 23 Sinua, minun isieni Jumala, minä kiitän ja ylistän siitä, että olet antanut minulle viisauden ja voiman ja että nyt annoit minun tietää, mitä me sinulta rukoilimme. Sillä sinä annoit meidän tietää kuninkaan asian."
\par 24 Sitten Daniel meni Arjokin tykö, jolla oli kuninkaan määräys tappaa Baabelin viisaat. Hän meni ja sanoi hänelle näin: "Älä tapa Baabelin viisaita; vie minut kuninkaan eteen, niin minä ilmoitan kuninkaalle selityksen".
\par 25 Silloin Arjok vei kiiruusti Danielin kuninkaan eteen ja sanoi tälle näin: "Minä olen löytänyt juutalaisten pakkosiirtolaisten joukosta miehen, joka ilmoittaa kuninkaalle selityksen".
\par 26 Kuningas vastasi ja sanoi Danielille, jonka nimenä oli Beltsassar: "Voitko sinä ilmoittaa minulle unen, jonka minä näin, ja sen selityksen?"
\par 27 Daniel vastasi kuninkaalle ja sanoi: "Salaisuutta, jonka kuningas tahtoo tietää, eivät viisaat, noidat, tietäjät eivätkä tähtienselittäjät voi ilmoittaa kuninkaalle.
\par 28 Mutta on Jumala taivaassa; hän paljastaa salaisuudet ja ilmoittaa kuningas Nebukadnessarille, mitä on tapahtuva aikojen lopussa. Tämä on sinun unesi, sinun pääsi näky, joka sinulla oli vuoteessasi.
\par 29 Kun sinä, kuningas, olit vuoteessasi, nousi mieleesi ajatus, mitä tämän jälkeen on tapahtuva; ja hän, joka paljastaa salaisuudet, ilmoitti sinulle, mitä tapahtuva on.
\par 30 Ja tämä salaisuus on paljastettu minulle, ei oman viisauteni voimasta, ikäänkuin minulla olisi sitä enemmän kuin kenelläkään muulla ihmisellä, vaan sentähden että selitys ilmoitettaisiin kuninkaalle ja sinä saisit selville sydämesi ajatukset.
\par 31 Sinä näit, kuningas, katso, oli iso kuvapatsas. Se kuvapatsas oli suuri, ja sen kirkkaus oli ylenpalttinen. Se seisoi sinun edessäsi, ja se oli hirvittävä nähdä.
\par 32 Kuvan pää oli parasta kultaa, sen rinta ja käsivarret hopeata, sen vatsa ja lanteet vaskea.
\par 33 Sen sääret olivat rautaa, sen jalat osaksi rautaa, osaksi savea.
\par 34 Sinun sitä katsellessasi irtautui kivilohkare - ei ihmiskäden voimasta - ja iski kuvapatsasta jalkoihin, jotka olivat rautaa ja savea, ja murskasi ne.
\par 35 Silloin musertuivat yhdellä haavaa rauta, savi, vaski, hopea ja kulta, ja niiden kävi kuin akanain kesäisillä puimatantereilla: tuuli vei ne, eikä niistä löydetty jälkeäkään. Mutta kivestä, joka oli kuvapatsaan murskannut, tuli suuri vuori, ja se täytti koko maan.
\par 36 Tämä oli se uni, ja nyt sanomme kuninkaalle sen selityksen.
\par 37 Sinä, kuningas, olet kuningasten kuningas, jolle taivaan Jumala on antanut vallan, voiman, väkevyyden ja kunnian
\par 38 ja jonka käteen hän on antanut ihmiset, missä ikinä heitä asuu, ja kedon eläimet ja taivaan linnut, asettaen sinut kaikkien niiden valtiaaksi: sinä olet se kultainen pää.
\par 39 Mutta sinun jälkeesi nousee toinen valtakunta, joka on halvempi kuin sinun; ja sitten kolmas valtakunta, joka on vaskea ja joka hallitsee kaikkea maata.
\par 40 Ja neljäs valtakunta on tuleva, luja kuin rauta; niinkuin rauta musertaa ja särkee kaiken, niinkuin rauta murskaa, niin se on musertava ja murskaava ne kaikki.
\par 41 Ja että sinä näit jalkain ja varvasten olevan osittain savenvalajan savea, osittain rautaa, se merkitsee, että se on oleva hajanainen valtakunta; kuitenkin on siinä oleva raudan lujuutta, niinkuin sinä näit rautaa olevan saven seassa.
\par 42 Ja että jalkojen varpaat olivat osaksi rautaa, osaksi savea, se merkitsee, että osa sitä valtakuntaa on oleva luja, osa sitä on oleva hauras.
\par 43 Että sinä näit rautaa olevan saven seassa, se merkitsee, että vaikka ne sekaantuvat toisiinsa ihmissiemenellä, ne eivät yhdisty toinen toiseensa, niinkuin ei rautakaan sekaannu saveen.
\par 44 Mutta niiden kuningasten päivinä on taivaan Jumala pystyttävä valtakunnan, joka on kukistumaton iankaikkisesti ja jonka valtaa ei toiselle kansalle anneta. Se on musertava kaikki ne muut valtakunnat ja tekevä niistä lopun, mutta se itse on pysyvä iankaikkisesti,
\par 45 niinkuin sinä näit, että kivilohkare irtautui vuoresta - ei ihmiskäden voimasta - ja murskasi raudan, vasken, saven, hopean ja kullan. Suuri Jumala on ilmoittanut kuninkaalle, mitä tämän jälkeen on tapahtuva. Ja uni on tosi ja sen selitys luotettava."
\par 46 Silloin kuningas Nebukadnessar lankesi kasvoillensa ja kumartui maahan Danielin edessä ja käski uhrata hänelle ruokauhria ja suitsutusta.
\par 47 Kuningas vastasi Danielille ja sanoi: "Totisesti on teidän Jumalanne jumalien Jumala ja kuningasten herra ja se, joka paljastaa salaisuudet; sillä sinä olet voinut paljastaa tämän salaisuuden".
\par 48 Sitten kuningas korotti Danielin ja antoi hänelle paljon suuria lahjoja ja asetti hänet koko Baabelin maakunnan herraksi ja kaikkien Baabelin viisaitten ylimmäiseksi päämieheksi.
\par 49 Ja Danielin anomuksesta kuningas antoi Sadrakin, Meesakin ja Abednegon hoitoon Baabelin maakunnan hallinnon. Mutta Daniel jäi kuninkaan hoviin.

\chapter{3}

\par 1 Kuningas Nebukadnessar teetti kultaisen kuvapatsaan, jonka korkeus oli kuusikymmentä kyynärää ja leveys kuusi kyynärää, ja pystytti sen Duuran lakeudelle Baabelin maakuntaan.
\par 2 Ja kuningas Nebukadnessar lähetti kokoamaan satraapit, maaherrat, käskynhaltijat, neuvonantajat, aarteiden hoitajat, lainoppineet, tuomarit ja kaikki muut maakuntain virkamiehet, että he tulisivat sen kuvapatsaan vihkiäisiin, jonka kuningas Nebukadnessar oli pystyttänyt.
\par 3 Silloin kokoontuivat satraapit, maaherrat, käskynhaltijat, neuvonantajat, aarteiden hoitajat, lainoppineet, tuomarit ja kaikki muut maakuntain virkamiehet sen kuvapatsaan vihkiäisiin, jonka kuningas Nebukadnessar oli pystyttänyt, ja he asettuivat sen kuvapatsaan eteen, jonka Nebukadnessar oli pystyttänyt.
\par 4 Ja kuuluttaja huusi voimallisesti: "Teille, kansat, sukukunnat ja kielet, julistetaan:
\par 5 heti kun te kuulette torvien, huilujen, kitarain, harppujen, psalttarien, säkkipillien ja kaikkinaisten muiden soittimien äänen, langetkaa maahan ja kumartaen rukoilkaa kultaista kuvapatsasta, jonka kuningas Nebukadnessar on pystyttänyt.
\par 6 Mutta joka ei lankea maahan ja kumarra, se heitetään sillä hetkellä tuliseen pätsiin."
\par 7 Sentähden, heti kun kaikki kansat kuulivat torvien, huilujen, kitarain, harppujen, psalttarien, säkkipillien ja kaikkinaisten muiden soittimien äänen, lankesivat kaikki kansat, sukukunnat ja kielet maahan ja kumartaen rukoilivat kultaista kuvapatsasta, jonka kuningas Nebukadnessar oli pystyttänyt.
\par 8 Silloin astui heti kaldealaisia miehiä esiin syyttämään juutalaisia.
\par 9 He lausuivat ja sanoivat kuningas Nebukadnessarille: "Kuningas eläköön iankaikkisesti!
\par 10 Sinä, kuningas, olet antanut käskyn, että jokainen, joka kuulee torvien, huilujen, kitarain, harppujen, psalttarien, säkkipillien ja kaikkinaisten muiden soittimien äänen, langetkoon maahan ja kumartaen rukoilkoon kultaista kuvapatsasta,
\par 11 ja että joka ei lankea maahan ja kumarra, se heitetään tuliseen pätsiin.
\par 12 On juutalaisia miehiä, joiden hoitoon sinä olet antanut Baabelin maakunnan hallinnon, Sadrak, Meesak ja Abednego: nämä miehet eivät välitä sinun käskystäsi, kuningas; he eivät palvele sinun jumaliasi eivätkä kumartaen rukoile kultaista kuvapatsasta, jonka sinä olet pystyttänyt."
\par 13 Silloin Nebukadnessar vihan vimmassa käski tuoda Sadrakin, Meesakin ja Abednegon. Kun nämä miehet oli tuotu kuninkaan eteen,
\par 14 lausui Nebukadnessar ja sanoi heille: "Aivanko tahallanne te, Sadrak, Meesak ja Abednego, ette tahdo palvella minun jumalaani ettekä kumartaen rukoilla kultaista kuvapatsasta, jonka minä olen pystyttänyt?
\par 15 Nyt, jos te siinä silmänräpäyksessä, kun kuulette torvien, huilujen, kitarain, harppujen, psalttarien, säkkipillien ja kaikkinaisten muiden soittimien äänen, olette valmiit lankeamaan maahan ja kumartaen rukoilemaan kuvapatsasta, jonka minä olen teettänyt, niin hyvä! Mutta ellette kumarra, niin teidät heti paikalla heitetään tuliseen pätsiin; ja kuka on se jumala, joka pelastaa teidät minun kädestäni?"
\par 16 Sadrak, Meesak ja Abednego vastasivat ja sanoivat kuninkaalle: "Nebukadnessar! Ei ole tarpeellista meidän vastata sinulle tähän sanaakaan.
\par 17 Jos niin käy, voi meidän Jumalamme kyllä pelastaa meidät tulisesta pätsistä, ja hän pelastaa myös sinun kädestäsi, kuningas.
\par 18 Ja vaikka ei pelastaisikaan, niin tiedä se, kuningas, että me emme palvele sinun jumaliasi emmekä kumartaen rukoile kultaista kuvapatsasta, jonka sinä olet pystyttänyt."
\par 19 Silloin Nebukadnessar tuli kiukkua täyteen Sadrakia, Meesakia ja Abednegoa kohtaan ja hänen hahmonsa muuttui. Ja hän käski ja sanoi, että pätsi oli kuumennettava seitsemän kertaa kuumemmaksi kuin tavallisesti.
\par 20 Ja hän käski sotajoukkonsa väkevimpien miesten sitoa Sadrakin, Meesakin ja Abednegon ja heittää heidät tuliseen pätsiin.
\par 21 Silloin nämä sidottiin vaippoineen, takkeineen, päähineineen ja muine vaatteineen ja heitettiin tuliseen pätsiin.
\par 22 Koska nyt kuninkaan sana oli niin ankara ja pätsi niin kovin kuumennettu, tappoi tulen liekki ne miehet, jotka veivät ylös Sadrakin, Meesakin ja Abednegon.
\par 23 Mutta nämä kolme miestä, Sadrak, Meesak ja Abednego, suistuivat sidottuina tuliseen pätsiin.
\par 24 Silloin kuningas Nebukadnessar hämmästyi ja nousi kiiruusti ylös ja lausui hallitusmiehilleen sanoen: "Emmekö heittäneet kolme miestä sidottuina tuleen?" He vastasivat ja sanoivat kuninkaalle: "Totisesti, kuningas!"
\par 25 Hän vastasi ja sanoi: "Katso, minä näen neljä miestä kävelevän vapaina tulessa, eivätkä he ole vahingoittuneet, ja neljäs on näöltänsä niinkuin jumalan poika".
\par 26 Silloin Nebukadnessar meni tulisen pätsin aukolle, lausui ja sanoi: "Sadrak, Meesak ja Abednego, te korkeimman Jumalan palvelijat, astukaa ulos ja tulkaa tänne". Sadrak, Meesak ja Abednego astuivat silloin ulos tulesta.
\par 27 Satraapit, maaherrat, käskynhaltijat ja kuninkaan hallitusmiehet kokoontuivat ja näkivät, ettei tuli ollut voinut mitään näiden miesten ruumiille, etteivät heidän päänsä hiukset olleet kärventyneet eivätkä heidän vaatteensa vioittuneet; eikä heissä tuntunut tulen käryä.
\par 28 Silloin Nebukadnessar lausui ja sanoi: "Kiitetty olkoon Sadrakin, Meesakin ja Abednegon Jumala, joka lähetti enkelinsä ja pelasti palvelijansa, jotka häneen turvasivat eivätkä totelleet kuninkaan käskyä, vaan antoivat ruumiinsa alttiiksi, ennemmin kuin palvelivat kumartaen rukoilivat muuta jumalaa kuin omaa Jumalaansa.
\par 29 Ja minä annan käskyn, että jokainen, olkoon hän mitä kansaa, kansakuntaa ja kieltä tahansa, joka puhuu pilkaten Sadrakin, Meesakin ja Abednegon Jumalasta, hakattakoon kappaleiksi, ja hänen talonsa tehtäköön soraläjäksi; sillä ei ole muuta jumalaa, joka niin voi pelastaa kuin tämä."
\par 30 Sitten kuningas asetti Sadrakin, Meesakin ja Abednegon suureen valtaan Baabelin maakunnassa.
\par 31 "Kuningas Nebukadnessar kaikille kansoille, kansakunnille ja kielille, jotka asuvat kaiken maan päällä: suuri olkoon teidän rauhanne!
\par 32 Minä olen nähnyt hyväksi ilmoittaa ne tunnusteot ja ihmeet, jotka korkein Jumala on minulle tehnyt.
\par 33 Kuinka suuret ovat hänen tunnustekonsa ja kuinka voimalliset hänen ihmeensä! Hänen valtakuntansa on iankaikkinen valtakunta ja hänen valtansa pysyy suvusta sukuun."

\chapter{4}

\par 1 "Minä Nebukadnessar elin rauhassa huoneessani ja onnellisena palatsissani.
\par 2 Minä näin unen, ja se peljästytti minut; ja unikuvat, joita minulla oli vuoteessani, minun pääni näyt, kauhistuttivat minut.
\par 3 Niin minä annoin käskyn tuoda eteeni kaikki Baabelin viisaat, että he ilmoittaisivat minulle unen selityksen.
\par 4 Silloin tulivat tietäjät, noidat, kaldealaiset ja tähtienselittäjät, ja minä kerroin heille unen, mutta he eivät voineet ilmoittaa minulle sen selitystä.
\par 5 Mutta viimein tuli minun eteeni Daniel, jonka nimi on minun jumalani nimen mukaan Beltsassar ja jossa on pyhien jumalien henki; ja minä kerroin hänelle unen:
\par 6 'Beltsassar, sinä tietäjäin päämies, jossa minä tiedän olevan pyhien jumalien hengen ja jolle mikään salaisuus ei ole liian vaikea! Sano, mitä olivat ne näyt, jotka minä unessani näin, ja mikä on sen selitys.
\par 7 Nämä olivat minun pääni näyt, jotka minulla oli vuoteessani. Minä näin: Katso, oli puu keskellä maata, ja sen korkeus oli suuri.
\par 8 Puu kasvoi ja vahvistui, niin että sen latva ulottui taivaaseen ja se näkyi kaiken maan ääriin.
\par 9 Sen lehvät olivat kauniit ja sen hedelmät suuret, ja siinä oli ravintoa kaikille. Sen alla etsivät varjoa kedon eläimet, ja sen oksilla asuivat taivaan linnut, ja kaikki liha sai siitä ravintonsa.
\par 10 Minä näin pääni näyissä, joita minulla oli vuoteessani: katso, pyhä enkeli astui alas taivaasta.
\par 11 Hän huusi voimallisesti ja sanoi näin: hakatkaa puu poikki ja karsikaa sen oksat, riipikää sen lehvät ja hajottakaa sen hedelmät. Paetkoot eläimet sen alta ja linnut sen oksilta.
\par 12 Mutta sen kanto juurineen jättäkää maahan, rauta- ja vaskikahleissa kedon ruohikkoon. Taivaan kasteesta hän kastukoon, ja niinkuin eläinten olkoon hänen osansa maan ruoho.
\par 13 Hänen sydämensä muutetaan, niin ettei se ole ihmisen sydän, ja hänelle annetaan eläimen sydän. Ja niin kulukoon häneltä seitsemän aikaa.
\par 14 Tämä on säädetty enkelien päätöksellä, ja niin ovat tästä asiasta pyhät sanoneet, että elävät tietäisivät Korkeimman hallitsevan ihmisten valtakuntaa ja antavan sen, kenelle hän tahtoo, ja asettavan sen päämieheksi ihmisistä halvimman.'
\par 15 Tämän unen näin minä, kuningas Nebukadnessar. Ja sinä, Beltsassar, sano sen selitys, koska ei yksikään minun valtakuntani viisaista voi minulle ilmoittaa sen selitystä. Mutta sinä sen voit, sillä sinussa on pyhien jumalien henki."
\par 16 Silloin Daniel, jonka nimenä oli Beltsassar, hämmästyi hetkeksi, ja hänen ajatuksensa peljästyttivät häntä. Kuningas lausui ja sanoi: "Beltsassar, älköön uni ja sen selitys sinua peljästyttäkö". Beltsassar vastasi ja sanoi: "Herrani, koskekoon uni sinun vihollisiasi ja sen selitys sinun vastustajiasi.
\par 17 Puu, jonka sinä näit ja joka kasvoi ja vahvistui, niin että sen latva ulottui taivaaseen ja se näkyi kaikkeen maahan,
\par 18 jonka lehvät olivat kauniit ja hedelmät suuret ja jossa oli ravintoa kaikille, jonka alla kedon eläimet asuivat ja jonka oksilla taivaan linnut oleskelivat,
\par 19 - se puu olet sinä, kuningas, joka olet kasvanut ja vahvistunut; sinun suuruutesi on kasvanut ja ulottuu taivaaseen ja sinun valtasi maan ääriin.
\par 20 Ja että kuningas näki pyhän enkelin astuvan alas taivaasta ja sanovan: 'Hakatkaa puu poikki ja turmelkaa se, mutta jättäkää sen kanto juurineen maahan, rauta- ja vaskikahleissa kedon ruohikkoon; taivaan kasteesta hän kastukoon, ja niinkuin kedon eläinten olkoon hänen osansa, kunnes häneltä on kulunut seitsemän aikaa',
\par 21 sen selitys, oi kuningas, ja Ylimmäisen päätös, joka on kohdannut minun herraani, kuningasta, on tämä:
\par 22 Sinut ajetaan pois ihmisten seasta, ja kedon eläinten parissa on sinun asuinpaikkasi oleva; sinä joudut syömään ruohoa niinkuin raavaat, ja sinä olet kastuva taivaan kasteesta; ja niin on sinulta kuluva seitsemän aikaa, kunnes tulet tuntemaan, että Korkein hallitsee ihmisten valtakuntaa ja antaa sen, kenelle hän tahtoo.
\par 23 Mutta että käskettiin jättää maahan puun kanto juurineen, se tietää, että sinun valtakuntasi pysytetään sinulla, ja sinä saat sen, niin pian kuin tulet tuntemaan, että valta on taivaan.
\par 24 Sentähden, kuningas, kelvatkoon sinulle minun neuvoni: kirvoita synnit itsestäsi almuilla ja pahat tekosi vaivaisia armahtamalla; ehkäpä silloin onnesi kestäisi."
\par 25 Tämä kaikki kohtasi kuningas Nebukadnessaria.
\par 26 Kaksitoista kuukautta tämän jälkeen, kun kuningas oli kävelemässä kuninkaallisen palatsinsa katolla Baabelissa,
\par 27 hän puhkesi puhumaan sanoen: "Eikö tämä ole se suuri Baabel, jonka minä väkevällä voimallani olen rakentanut kuninkaalliseksi linnaksi, valtasuuruuteni kunniaksi!"
\par 28 Vielä oli sana kuninkaan suussa, kun taivaasta tuli ääni: "Sinulle, kuningas Nebukadnessar, julistetaan: Sinun valtakuntasi on otettu sinulta pois.
\par 29 Sinut ajetaan pois ihmisten seasta, ja kedon eläinten parissa on sinun asuinpaikkasi oleva; sinä joudut syömään ruohoa niinkuin raavaat, ja niin on sinulta kuluva seitsemän aikaa, kunnes tulet tuntemaan, että Korkein hallitsee ihmisten valtakuntaa ja antaa sen, kenelle hän tahtoo."
\par 30 Sillä hetkellä se sana toteutui Nebukadnessarissa: hänet ajettiin pois ihmisten seasta, ja hän söi ruohoa niinkuin raavaat, ja hänen ruumiinsa kastui taivaan kasteesta, kunnes hänen hiuksensa kasvoivat pitkiksi kuin kotkan sulat ja hänen kyntensä kuin lintujen kynnet.
\par 31 "Mutta sen ajan kuluttua minä, Nebukadnessar, nostin silmäni taivasta kohti, ja minun järkeni palasi. Ja minä kiitin Korkeinta, minä ylistin ja kunnioitin häntä, joka elää iankaikkisesti, jonka hallitus on iankaikkinen hallitus ja jonka valtakunta pysyy suvusta sukuun.
\par 32 Kaikki maan asukkaat ovat tyhjänveroiset; ja hän tekee, niinkuin hän tahtoo, taivaan joukoille ja maan asukkaille, eikä ole sitä, joka pidättää hänen kätensä ja sanoo hänelle: 'Mitäs teet?'
\par 33 Siihen aikaan palasi minun järkeni, ja palasi minun valtasuuruuteni ja loistoni, minun valtakuntani kunniaksi. Ja minun hallitusmieheni ja ylimykseni etsivät minut, ja minut pantiin jälleen hallitsemaan valtakuntaani, ja minun valtani lisääntyi ylenpalttisesti.
\par 34 Nyt minä, Nebukadnessar, kiitän, ylistän ja kunnioitan taivaan kuningasta; sillä kaikki hänen työnsä ovat totiset ja hänen tiensä oikeat. Ja hän voi nöyryyttää ne, jotka ylpeydessä vaeltavat."

\chapter{5}

\par 1 Kuningas Belsassar laittoi suuret pidot tuhannelle ylimyksellensä, ja hän joi viiniä näiden tuhannen edessä.
\par 2 Kun viini oli makeimmillaan, käski Belsassar tuoda ne kulta- ja hopea-astiat, jotka hänen isänsä Nebukadnessar oli ottanut Jerusalemin temppelistä, että kuningas ja hänen ylimyksensä, hänen puolisonsa ja sivuvaimonsa joisivat niistä.
\par 3 Silloin tuotiin ne kulta-astiat, jotka oli otettu temppelistä, Jumalan huoneesta, Jerusalemista, ja niistä joivat kuningas ja hänen ylimyksensä, hänen puolisonsa ja sivuvaimonsa.
\par 4 He joivat viiniä ja ylistivät kultaisia ja hopeisia, vaskisia, rautaisia, puisia ja kivisiä jumalia.
\par 5 Sillä hetkellä ilmestyivät ihmiskäden sormet ja kirjoittivat kuninkaan palatsin kalkitulle seinälle, vastapäätä lampunjalkaa, ja kuningas näki käden, joka kirjoitti.
\par 6 Silloin kuninkaan kasvot kalpenivat, ja hänen ajatuksensa peljästyttivät hänet; hänen lanteittensa nivelet herposivat, ja hänen polvensa tutisivat.
\par 7 Kuningas huusi kovalla äänellä ja käski tuoda noidat, kaldealaiset ja tähtienselittäjät. Kuningas lausui ja sanoi Baabelin tietäjille: "Kuka ikinä voi lukea tämän kirjoituksen ja ilmoittaa minulle sen selityksen, hänet puetaan purppuraan, ja hänen kaulaansa pannaan kultakäädyt, ja hän on oleva yksi valtakunnan kolmesta valtamiehestä".
\par 8 Silloin tulivat kaikki kuninkaan viisaat, mutta he eivät voineet lukea kirjoitusta eivätkä ilmoittaa kuninkaalle sen selitystä.
\par 9 Kuningas Belsassar peljästyi silloin suuresti, ja hänen kasvonsa kalpenivat, ja hänen ylimyksensä tyrmistyivät.
\par 10 Kuninkaan äiti tuli kuninkaan ja hänen ylimystensä puheen tähden pitohuoneeseen. Kuninkaan äiti lausui ja sanoi: "Eläköön kuningas iankaikkisesti! Älkööt sinun ajatuksesi peljättäkö sinua, älköötkä kasvosi kalvetko.
\par 11 Sinun valtakunnassasi on mies, jossa on pyhien jumalien henki ja jolla sinun isäsi päivinä havaittiin olevan valistus ja ymmärrys ynnä viisaus, samankaltainen kuin jumalien; hänet asetti isäsi, kuningas Nebukadnessar, tietäjäin, noitien, kaldealaisten ja tähtienselittäjäin päämieheksi - sinun isäsi, kuningas -
\par 12 sentähden, että erinomainen henki ja tieto ynnä myös taito selittää unia, arvata arvoituksia ja ratkaista ongelmia havaittiin juuri hänessä, Danielissa, jolle kuningas oli antanut nimen Beltsassar. Kutsuttakoon nyt Daniel, niin hän ilmoittaa selityksen."
\par 13 Silloin Daniel tuotiin kuninkaan eteen. Kuningas lausui ja sanoi Danielille: "Oletko sinä Daniel, joka on niitä juutalaisia pakkosiirtolaisia, mitkä minun isäni, kuningas, on tuonut Juudasta?
\par 14 Minä olen kuullut sinusta, että sinussa on jumalien henki ja että sinussa on havaittu valistus, taito ja erinomainen viisaus.
\par 15 Ja nyt tuotiin minun eteeni viisaat ja noidat lukemaan tätä kirjoitusta ja ilmoittamaan minulle sen selitys, mutta he eivät voineet sen selitystä ilmoittaa.
\par 16 Mutta sinun minä olen kuullut voivan antaa selityksiä ja ratkaista ongelmia. Nyt siis, jos voit lukea kirjoituksen ja ilmoittaa minulle sen selityksen, niin sinut puetaan purppuraan, ja kultakäädyt pannaan sinun kaulaasi, ja sinä olet oleva yksi valtakunnan kolmesta valtamiehestä."
\par 17 Silloin Daniel vastasi ja sanoi kuninkaalle: "Lahjasi pidä itse, ja antimesi anna toiselle. Mutta kirjoituksen minä luen kuninkaalle ja ilmoitan hänelle sen selityksen.
\par 18 Sinä kuningas! Korkein Jumala antoi sinun isällesi Nebukadnessarille kuninkuuden, voiman, kunnian ja valtasuuruuden.
\par 19 Ja sen voiman tähden, jonka hän oli hänelle antanut, vapisivat kaikki kansat, kansakunnat ja kielet ja pelkäsivät hänen edessään. Hän tappoi, kenen hän tahtoi, hän jätti henkiin, kenen hän tahtoi, hän ylensi, kenen hän tahtoi, hän alensi, kenen hän tahtoi.
\par 20 Mutta kun hänen sydämensä paisui ja hänen henkensä kävi korskeaksi ja ylpeäksi, syöstiin hänet kuninkaalliselta valtaistuimeltansa, ja hänen kunniansa otettiin häneltä pois.
\par 21 Hänet ajettiin pois ihmisten seasta, ja hänen sydämensä tuli eläinten sydämen kaltaiseksi; hänen asuntonsa oli villiaasien parissa, ja hän joutui syömään ruohoa niinkuin raavaat; hänen ruumiinsa kastui taivaan kasteesta, kunnes hän tuli tuntemaan, että korkein Jumala hallitsee ihmisten valtakuntaa ja asettaa sen päämieheksi, kenen hän tahtoo.
\par 22 Mutta sinä, hänen poikansa Belsassar, et ole nöyryyttänyt sydäntäsi, vaikka tämän kaiken tiesit;
\par 23 vaan sinä olet korottanut itsesi taivaan Herraa vastaan: hänen huoneensa astiat on tuotu sinun eteesi, ja sinä ja sinun ylimyksesi, sinun puolisosi ja sivuvaimosi olette juoneet niistä viiniä, ja sinä olet ylistänyt hopeisia ja kultaisia, vaskisia, rautaisia, puisia ja kivisiä jumalia, jotka eivät näe, eivät kuule eivätkä mitään tiedä. Mutta sitä Jumalaa, jonka kädessä on sinun henkesi ja kaikki sinun tiesi, sinä et ole kunnioittanut.
\par 24 Sentähden on tämä käsi lähetetty hänen tyköänsä ja tämä kirjoitus kirjoitettu.
\par 25 Ja tämä on kirjoitus, joka on tuonne kirjoitettu: 'Mene, mene, tekel, ufarsin'.
\par 26 Ja tämä on sen selitys: mene merkitsee: Jumala on laskenut sinun valtakuntasi luvun ja on tehnyt siitä lopun.
\par 27 Tekel: sinut on vaa'alla punnittu ja köykäiseksi havaittu.
\par 28 Peres: sinun valtakuntasi on pirstottu ja annettu meedialaisille ja persialaisille."
\par 29 Silloin Belsassar antoi käskyn, että Daniel oli puettava purppuraan ja kultakäädyt pantava hänen kaulaansa ja julistettava, että hän oli oleva yksi valtakunnan kolmesta valtamiehestä.
\par 30 Samana yönä tapettiin Belsassar, kaldealaisten kuningas,
\par 31 ja Daarejaves, meedialainen, sai haltuunsa valtakunnan ollessaan noin kuudenkymmenen kahden vuoden ikäinen.

\chapter{6}

\par 1 Daarejaves näki hyväksi asettaa valtakuntaansa sata kaksikymmentä satraappia, että heitä olisi kaikkialla valtakunnassa.
\par 2 Ja heitä ylempänä oli kolme valtaherraa, joista Daniel oli yksi. Heille tuli satraappien tehdä tili, ettei kuningas kärsisi vahinkoa.
\par 3 Mutta Daniel oli etevämpi muita valtaherroja sekä satraappeja, sillä hänessä oli erinomainen henki; ja kuningas aikoi asettaa hänet koko valtakunnan päämieheksi.
\par 4 Silloin toiset valtaherrat sekä satraapit etsivät Danielia vastaan syytä valtakunnan hallinnossa. Mutta he eivät voineet löytää mitään syytä eikä rikkomusta, sillä hän oli uskollinen, eikä hänestä laiminlyöntiä eikä rikkomusta löydetty.
\par 5 Silloin nämä miehet sanoivat: "Me emme löydä tuossa Danielissa mitään syytä - ellemme löydä sitä hänen jumalanpalveluksessaan".
\par 6 Senjälkeen nämä valtaherrat ja satraapit riensivät kiiruusti kuninkaan tykö ja sanoivat hänelle näin: "Kuningas Daarejaves eläköön iankaikkisesti!
\par 7 Kaikki kuninkaalliset valtaherrat, maaherrat, satraapit, hallitusmiehet ja käskynhaltijat ovat keskenänsä neuvotelleet, että olisi annettava kuninkaallinen julistus ja vahvistettava kielto, että kuka ikinä kolmenkymmenen päivän kuluessa rukoilee jotakin yhdeltäkään jumalalta tai ihmiseltä, paitsi sinulta, kuningas, se heitettäköön jalopeurain luolaan.
\par 8 Nyt, kuningas! Säädä kielto ja kirjoita kirjoitus, jota meedialaisten ja persialaisten peruuttamattoman lain mukaan ei voida muuttaa."
\par 9 Niin kuningas Daarejaves kirjoitti kirjoituksen ja kiellon.
\par 10 Niin pian kuin Daniel oli saanut tietää, että kirjoitus oli kirjoitettu, meni hän taloonsa, jonka yläsalin ikkunat olivat avatut Jerusalemiin päin. Ja hän lankesi kolmena hetkenä päivässä polvillensa, rukoili ja kiitti Jumalaansa, aivan niinkuin hän ennenkin oli tehnyt.
\par 11 Silloin nuo miehet riensivät kiiruusti sinne ja tapasivat Danielin rukoilemasta ja avuksi huutamasta Jumalaansa.
\par 12 Sitten he astuivat kuninkaan eteen ja muistuttivat kuninkaan kiellosta: "Etkö ole kirjoittanut kieltoa, että kuka ikinä kolmenkymmenen päivän kuluessa rukoilee jotakin yhdeltäkään jumalalta tai ihmiseltä, paitsi sinulta, kuningas, se heitettäköön jalopeurain luolaan?" Kuningas vastasi ja sanoi: "Kielto on luja, meedialaisten ja persialaisten peruuttamattoman lain mukaan".
\par 13 Silloin he vastasivat ja sanoivat kuninkaalle: "Daniel, joka on juutalaisia pakkosiirtolaisia, ei välitä sinusta, kuningas, eikä kiellosta, jonka olet kirjoittanut, vaan kolmena hetkenä päivässä hän toimittaa rukouksensa".
\par 14 Mutta kun kuningas sen kuuli, tuli hän sangen murheelliseksi, ja hän mietti, miten voisi pelastaa Danielin; auringon laskuun asti hän vaivasi itseään vapauttaaksensa hänet.
\par 15 Silloin ne miehet riensivät kiiruusti kuninkaan tykö ja sanoivat kuninkaalle: "Tiedä, kuningas, meedialaisten ja persialaisten laki on, ettei mitään kuninkaan vahvistamaa kieltoa tai julistusta peruuteta".
\par 16 Silloin kuningas käski tuoda Danielin ja heittää hänet jalopeurain luolaan. Ja kuningas puhui Danielille ja sanoi: "Sinun Jumalasi, jota sinä lakkaamatta palvelet, pelastakoon sinut". Sitten tuotiin kivi ja pantiin luolan suulle,
\par 17 ja kuningas sinetöi sen omallaan ja ylimystensä sineteillä, ettei Danielin asiassa muutosta tapahtuisi.
\par 18 Sitten kuningas meni palatsiinsa ja vietti yönsä paastoten eikä sallinut tuoda eteensä naisia, ja hänen unensa pakeni häneltä.
\par 19 Aamun sarastaessa kuningas sitten nousi ja meni kiiruusti jalopeurain luolalle.
\par 20 Ja lähestyessään luolaa hän huusi Danielille murheellisella äänellä, ja kuningas lausui ja sanoi Danielille: "Daniel, sinä elävän Jumalan palvelija, onko sinun Jumalasi, jota lakkaamatta palvelet, voinut pelastaa sinut jalopeuroilta?"
\par 21 Silloin Daniel vastasi kuninkaalle: "Kuningas eläköön iankaikkisesti!
\par 22 Minun Jumalani on lähettänyt enkelinsä ja sulkenut jalopeurain kidat, niin etteivät ne ole minua vahingoittaneet, sillä minut on havaittu nuhteettomaksi hänen edessänsä, enkä minä ole sinuakaan vastaan, kuningas, rikosta tehnyt."
\par 23 Silloin kuningas ihastui suuresti ja käski ottaa Danielin ylös luolasta. Ja kun Daniel oli otettu ylös luolasta, ei hänessä havaittu mitään vammaa; sillä hän oli turvannut Jumalaansa.
\par 24 Ja kuningas käski tuoda ne miehet, jotka olivat syyttäneet Danielia, ja heittää heidät lapsineen ja vaimoineen jalopeurain luolaan; eivätkä he ehtineet luolan pohjaan, ennenkuin jalopeurat hyökkäsivät heidän kimppuunsa ja murskasivat kaikki heidän luunsa.
\par 25 Sitten kuningas Daarejaves kirjoitti kaikille kansoille, kansakunnille ja kielille, mitä koko maan päällä asuu: "Suuri olkoon teidän rauhanne!
\par 26 Minä olen antanut käskyn, että minun valtakuntani koko valtapiirissä vavistakoon ja peljättäköön Danielin Jumalaa. Sillä hän on elävä Jumala ja pysyy iankaikkisesti. Hänen valtakuntansa ei häviä, eikä hänen herrautensa lopu.
\par 27 Hän pelastaa ja vapahtaa, hän tekee tunnustekoja ja ihmeitä taivaassa ja maan päällä, hän, joka pelasti Danielin jalopeurain kynsistä."
\par 28 Ja tämä Daniel oli korkeassa arvossa ja kunniassa Daarejaveksen valtakunnassa sekä persialaisen Kooreksen valtakunnassa.

\chapter{7}

\par 1 Belsassarin, Baabelin kuninkaan, ensimmäisenä hallitusvuotena Daniel näki unen, päänsä näyn, vuoteessansa. Sitten hän kirjoitti tämän unen.
\par 2 Kertomuksen alku on tämä: Daniel lausui ja sanoi: Minä näin yöllä näyssäni, ja katso, taivaan neljä tuulta kuohutti suurta merta.
\par 3 Ja merestä nousi neljä suurta petoa, kukin erilainen kuin toinen.
\par 4 Ensimmäinen oli kuin leijona, mutta sillä oli kotkan siivet. Minun sitä katsellessani reväistiin siltä siivet, ja se nostettiin maasta pystyyn ja asetettiin kahdelle jalalle niinkuin ihminen, ja sille annettiin ihmisen sydän.
\par 5 Ja katso, oli toinen peto, joka oli karhun näköinen. Se nostettiin toiselle kyljellensä, ja sillä oli suussa kolme kylkiluuta, hammasten välissä; ja sille sanottiin näin: "Nouse ja syö paljon lihaa".
\par 6 Tämän jälkeen minä näin, ja katso, oli taas toinen peto, pantterin kaltainen, ja sen selässä oli neljä linnunsiipeä. Sillä pedolla oli neljä päätä, ja sille annettiin valta.
\par 7 Sen jälkeen minä näin yöllisessä näyssäni, ja katso, oli neljäs peto, kauhea, hirmuinen ja ylen väkevä; sillä oli suuret rautaiset hampaat, ja se söi ja murskasi ja tallasi tähteet jalkoihinsa. Se oli erilainen kuin kaikki edelliset pedot, ja sillä oli kymmenen sarvea.
\par 8 Minä tarkkasin sarvea, ja katso, eräs muu pieni sarvi puhkesi niiden välistä, ja kolme edellisistä sarvista reväistiin pois sen edestä. Ja katso, sillä sarvella oli silmät kuin ihmisen silmät, ja suu, joka herjaten puhui.
\par 9 Minun sitä katsellessani valtaistuimet asetettiin, ja Vanhaikäinen istuutui. Hänen vaatteensa olivat valkeat kuin lumi ja hänen päänsä hiukset kuin puhdas villa. Hänen valtaistuimensa oli tulen liekkejä, ja sen pyörät olivat palavaa tulta.
\par 10 Tulivirta vuoti, se kävi ulos hänestä; tuhannen tuhatta palveli häntä, ja kymmenen tuhatta kertaa kymmenen tuhatta seisoi hänen edessänsä. Oikeus istui tuomiolle, ja kirjat avattiin.
\par 11 Minä katselin, ja silloin, niiden herjaavien sanojen tähden, joita sarvi puhui, minun katsellessani peto tapettiin, ja sen ruumis hävitettiin ja heitettiin tuleen palamaan.
\par 12 Ja muiltakin pedoilta otettiin valta pois; niiden elämän pituus oli määrätty aikaa ja hetkeä myöten.
\par 13 Minä näin yöllisessä näyssä, ja katso, taivaan pilvissä tuli Ihmisen Pojan kaltainen; ja hän saapui Vanhaikäisen tykö, ja hänet saatettiin tämän eteen.
\par 14 Ja hänelle annettiin valta, kunnia ja valtakunta, ja kaikki kansat, kansakunnat ja kielet palvelivat häntä. Hänen valtansa on iankaikkinen valta, joka ei lakkaa, ja hänen valtakuntansa on valtakunta, joka ei häviä.
\par 15 Minä, Daniel, tunsin henkeni tulevan murheelliseksi ruumiissani, ja minun näkemäni näyt peljättivät minua.
\par 16 Minä lähestyin yhtä siellä seisovista ja pyysin häneltä varmaa tietoa kaikista näistä asioista. Niin hän vastasi minulle ja ilmoitti minulle niiden selityksen:
\par 17 "Nuo suuret pedot, joita on neljä, ovat neljä kuningasta, jotka nousevat maasta.
\par 18 Mutta Korkeimman pyhät saavat valtakunnan ja omistavat valtakunnan iankaikkisesti - iankaikkisesta iankaikkiseen."
\par 19 Senjälkeen minä tahdoin saada varmuuden neljännestä pedosta, joka oli erilainen kuin kaikki muut ja ylen hirmuinen; jolla oli rautaiset hampaat ja vaskiset kynnet, joka söi ja murskasi ja tallasi tähteet jalkoihinsa;
\par 20 sekä pedon pään kymmenestä sarvesta ynnä siitä sarvesta, joka puhkesi ja jonka edestä kolme putosi; jolla sarvella oli silmät ja herjauksia puhuva suu ja joka näytti suuremmalta kuin ne muut;
\par 21 se sarvi, jonka minä näin sotivan pyhiä vastaan ja voittavan heidät,
\par 22 siihen asti kunnes Vanhaikäinen tuli ja oikeus annettiin Korkeimman pyhille ja aika joutui ja pyhät saivat omaksensa valtakunnan.
\par 23 Hän vastasi näin: "Neljäs peto on neljäs valtakunta, joka syntyy maan päälle, erilainen kuin kaikki muut valtakunnat. Se syö kaiken maan ja tallaa ja murskaa sen.
\par 24 Ja ne kymmenen sarvea ovat kymmenen kuningasta, jotka nousevat siitä valtakunnasta. Ja heidän jälkeensä nousee eräs muu, ja hän on erilainen kuin edelliset, ja hän kukistaa kolme kuningasta.
\par 25 Hän puhuu sanoja Korkeinta vastaan ja hävittää Korkeimman pyhiä. Hän pyrkii muuttamaan ajat ja lain, ja ne annetaan hänen käteensä ajaksi ja kahdeksi ajaksi ja puoleksi ajaksi.
\par 26 Sitten oikeus istuu tuomiolle, ja hänen valtansa otetaan pois ja hävitetään ja tuhotaan loppuun asti.
\par 27 Ja valtakunta ja valta ja valtakuntien voima kaiken taivaan alla annetaan Korkeimman pyhien kansalle. Hänen valtakuntansa on iankaikkinen valtakunta, ja kaikki vallat palvelevat häntä ja ovat hänelle alamaiset."
\par 28 Tähän loppuu kertomus. Minua, Danielia, peljättivät minun ajatukseni suuresti, ja minun kasvoni kalpenivat, ja minä kätkin asian sydämeeni.

\chapter{8}

\par 1 Kuningas Belsassarin kolmantena hallitusvuotena näin minä, Daniel, näyn, senjälkeen kuin minulla jo ennen oli ollut näky.
\par 2 Kun minä näyssä katselin, havaitsin minä olevani Suusanin linnassa, Eelamin maakunnassa; ja kun minä näyssä katselin, olin minä Uulai-joen rannalla.
\par 3 Minä nostin silmäni ja katsoin. Ja katso, oinas seisoi päin jokea, ja sillä oli kaksi sarvea; ja sarvet olivat korkeat, ja toinen oli toista korkeampi; ja korkeampi puhkesi esiin myöhemmin.
\par 4 Minä näin oinaan puskevan länteen, pohjoiseen ja etelään päin, eikä yksikään eläin kestänyt sen edessä, eikä kukaan voinut pelastaa sen vallasta. Se teki, mitä tahtoi; ja se tuli suureksi.
\par 5 Sitten minä tarkkasin, ja katso: tuli kauris päivän laskun puolelta, kulki koko maan ylitse eikä maata koskettanut; ja kauriilla oli keskellä otsaa uhkea sarvi.
\par 6 Ja se tuli aivan sen kaksisarvisen oinaan luokse, jonka minä olin nähnyt seisovan päin jokea, ja karkasi sen kimppuun vihansa väessä.
\par 7 Ja minä näin sen käyvän kiinni oinaaseen ja kiukuissaan puskevan oinasta ja murskaavan sen molemmat sarvet. Eikä oinaalla ollut voimaa kestää sen edessä, vaan kauris heitti sen maahan ja tallasi sitä; eikä ollut ketään, joka olisi voinut pelastaa oinaan sen vallasta.
\par 8 Ja kauris tuli ylen suureksi; mutta kun se oli väkevimmillään, särkyi suuri sarvi, ja sen sijalle kasvoi neljä uhkeata sarvea, taivaan neljää tuulta kohti.
\par 9 Ja yhdestä niistä puhkesi esiin sarvi, alussa vähäpätöinen. Se kasvoi suuresti etelään päin ja itään päin ja Ihanaan maahan päin.
\par 10 Ja se kasvoi taivaan sotajoukkoon asti ja pudotti maahan osan siitä sotajoukosta ja tähdistä ja tallasi niitä.
\par 11 Hän ylpeili sotajoukon ruhtinastakin vastaan, ja tältä otettiin pois jokapäiväinen uhri, ja hänen pyhäkkönsä paikka kukistettiin.
\par 12 Myös sotajoukko jokapäiväisen uhrin lisäksi annettiin rikollisesti alttiiksi tuholle. Se sarvi heitti totuuden maahan, ja mitä se teki, siinä se menestyi.
\par 13 Sitten minä kuulin yhden pyhän puhuvan, ja toinen pyhä sanoi sille, joka puhui: "Kuinka pitkää aikaa tarkoittaa näky jokapäiväisestä uhrista ja kauhistavasta rikoksesta: pyhäkön ja sotajoukon alttiiksi antamisesta tallattavaksi?"
\par 14 Ja hän sanoi minulle: "Kahtatuhatta kolmeasataa iltaa ja aamua; sitten pyhäkkö asetetaan jälleen oikeuteensa".
\par 15 Kun minä, Daniel, olin nähnyt tämän näyn ja koetin sitä ymmärtää, niin katso, minun edessäni seisoi miehen muotoinen olento.
\par 16 Ja minä kuulin ihmisen äänen Uulain keskeltä, ja se huusi ja sanoi: "Gabriel, selitä tälle se näky!"
\par 17 Silloin hän tuli aivan lähelle sitä paikkaa, jossa minä seisoin; ja hänen tullessansa minut valtasi pelko, ja minä lankesin kasvoilleni. Ja hän sanoi minulle: "Tarkkaa, ihmislapsi, sillä näky tarkoittaa lopun aikaa".
\par 18 Ja kun hän puhui minulle, olin minä horroksissa, kasvot maata vasten; mutta hän tarttui minuun ja nosti minut seisomaan.
\par 19 Sitten hän sanoi: "Katso, minä ilmoitan sinulle, mitä on tapahtuva viimeisenä vihan aikana; sillä lopun aikaa tämä tarkoittaa.
\par 20 Kaksisarvinen oinas, jonka sinä näit, on: Meedian ja Persian kuninkaat.
\par 21 Ja kauris on Jaavanin kuningas, ja suuri sarvi, joka sillä oli keskellä otsaa, on ensimmäinen kuningas.
\par 22 Ja että se särkyi ja neljä nousi sen sijalle, se on: neljä valtakuntaa nousee siitä kansasta, ei kuitenkaan niin väkevää kuin hän.
\par 23 Ja heidän valtansa lopulla, kun luopiot ovat täyttäneet syntiensä mitan, nousee kuningas, kasvoilta röyhkeä ja juonissa taitava.
\par 24 Ja väkevä on hänen voimansa, vaikka ei tosin hänen omasta voimastaan, ja ihmeellisen paljon hän saa aikaan hävitystä; ja hän menestyy siinä, mitä hän tekee, ja hän tuottaa turmion väkeville ja pyhien kansalle.
\par 25 Ja hänen oveluutensa tähden onnistuu petos hänen kädessään. Hän hautoo suuria sydämessään, ja keskellä rauhaa hän tuottaa turmion monille. Ruhtinasten ruhtinastakin vastaan hän nousee, mutta ilman ihmiskättä hänet muserretaan.
\par 26 Ja näky illoista ja aamuista, josta oli puhe, on tosi. Mutta sinä lukitse näky, sillä se tarkoittaa kaukaista aikaa."
\par 27 Ja minä, Daniel, olin raukea ja sairastin jonkin aikaa. Sitten minä nousin ja toimitin palvelusta kuninkaan tykönä; ja minä olin hämmästyksissäni näyn tähden enkä sitä ymmärtänyt.

\chapter{9}

\par 1 Daarejaveksen, Ahasveroksen pojan, ensimmäisenä hallitusvuotena, hänen, joka oli meedialaista sukua ja oli tullut kaldealaisten valtakunnan kuninkaaksi -
\par 2 hänen ensimmäisenä hallitusvuotenaan, minä, Daniel, kirjoituksista huomasin vuosien luvun, josta Herran sana oli tullut profeetta Jeremialle, että Jerusalem oli oleva raunioina seitsemänkymmentä vuotta.
\par 3 Ja minä käänsin kasvoni Herran Jumalan puoleen hartaassa rukouksessa ja anomisessa, paastossa, säkissä ja tuhassa.
\par 4 Minä rukoilin Herraa, Jumalaani, tunnustin ja sanoin: "Oi Herra, sinä suuri ja peljättävä Jumala, joka pidät liiton ja säilytät laupeuden niille, jotka sinua rakastavat ja noudattavat sinun käskyjäsi.
\par 5 Me olemme syntiä tehneet, olemme väärin tehneet, olleet jumalattomat ja uppiniskaiset; me olemme poikenneet pois sinun käskyistäsi ja oikeuksistasi
\par 6 emmekä ole kuulleet sinun palvelijoitasi, profeettoja, jotka puhuivat sinun nimessäsi kuninkaillemme, ruhtinaillemme ja isillemme ja kaikelle maan kansalle.
\par 7 Sinun, Herra, on vanhurskaus, mutta meidän on häpeä, niinkuin se on tänä päivänä Juudan miesten ja Jerusalemin asukasten ja koko Israelin, läheisten ja kaukaisten, kaikissa maissa, joihin sinä olet heidät karkoittanut heidän uskottomuutensa tähden, jota he ovat sinulle osoittaneet.
\par 8 Herra, meidän on häpeä, meidän kuninkaittemme, ruhtinaittemme ja isiemme, koska me olemme tehneet syntiä sinua vastaan.
\par 9 Herran, meidän Jumalamme, on armo ja anteeksiantamus, sillä me olemme olleet hänelle uppiniskaiset;
\par 10 me emme ole kuulleet Herran, meidän Jumalamme, ääntä emmekä vaeltaneet hänen laissansa, jonka hän asetti meidän eteemme palvelijainsa, profeettain, kautta.
\par 11 Vaan koko Israel rikkoi sinun lakisi ja poikkesi pois eikä kuullut sinun ääntäsi, ja niin vuodatettiin meidän päällemme se kirous ja vala, joka on kirjoitettu Mooseksen, Jumalan palvelijan, laissa; sillä me olimme tehneet syntiä häntä vastaan.
\par 12 Ja hän toteutti sanansa, jonka hän on puhunut meitä ja meidän tuomareitamme vastaan, jotka meitä tuomitsivat, ja antoi meidän päällemme tulla niin suuren onnettomuuden, ettei senkaltaista ole tapahtunut koko taivaan alla, kuin Jerusalemissa tapahtui.
\par 13 Niinkuin on kirjoitettu Mooseksen laissa, niin tuli kaikki tämä onnettomuus meidän päällemme. Mutta me emme koettaneet lepyttää Herraa, meidän Jumalaamme, niin että olisimme kääntyneet pois synneistämme ja ottaneet vaarin sinun totuudestasi.
\par 14 Sentähden Herra valvoi ja antoi tämän onnettomuuden tulla meidän päällemme; sillä Herra, meidän Jumalamme, on vanhurskas kaikissa töissänsä, jotka hän tekee, mutta me emme ole kuulleet hänen ääntänsä.
\par 15 Ja nyt, Herra, meidän Jumalamme, joka toit kansasi pois Egyptin maasta väkevällä kädellä ja teit itsellesi nimen, niinkuin se vielä tänä päivänä on: me olemme syntiä tehneet ja olleet jumalattomat.
\par 16 Herra, kaiken vanhurskautesi tähden, kääntyköön sinun vihasi ja kiivastuksesi pois sinun kaupungistasi Jerusalemista, sinun pyhästä vuorestasi; sillä meidän syntiemme tähden ja meidän isäimme pahojen tekojen tähden on Jerusalem ja sinun kansasi tullut kaikkien häväistäväksi, jotka meidän ympärillämme ovat.
\par 17 Ja nyt, meidän Jumalamme, kuule palvelijasi rukous ja hänen anomisensa ja valista kasvosi pyhäkkösi ylitse, joka on autiona; Herran tähden.
\par 18 Minun Jumalani, kallista korvasi ja kuule, avaa silmäsi ja katso meidän hävitystämme ja sitä kaupunkia, joka on otettu sinun nimiisi; sillä me annamme rukouksiemme langeta sinun eteesi, emme omaan vanhurskauteemme, vaan sinun suureen armoosi luottaen.
\par 19 Herra, kuule, Herra, anna anteeksi, Herra, huomaa ja tee tekosi itsesi tähden, älä viivyttele, minun Jumalani; sillä sinun kaupunkisi ja sinun kansasi ovat sinun nimiisi otetut."
\par 20 Ja vielä minä puhuin ja rukoilin ja tunnustin syntini ja kansani Israelin synnit ja annoin rukoukseni langeta Herran, Jumalani, eteen minun Jumalani pyhän vuoren puolesta.
\par 21 Ja kun minä vielä puhuin rukouksessa, tuli se mies, Gabriel, jonka minä olin nähnyt ennen näyssä, kiiruusti kiitäen minun tyköni ehtoouhrin aikana.
\par 22 Ja hän opetti minua, puhui minulle ja sanoi: "Daniel, nyt minä olen lähtenyt neuvomaan sinua ymmärrykseen.
\par 23 Kun sinä aloit rukoilla, lähti liikkeelle sana, ja minä olen tullut sitä ilmoittamaan; sillä sinä olet otollinen. Käsitä siis se sana ja ymmärrä näky.
\par 24 Seitsemänkymmentä viikkoa on säädetty sinun kansallesi ja pyhälle kaupungillesi; silloin luopumus päättyy, ja synti sinetillä lukitaan, ja pahat teot sovitetaan, ja iankaikkinen vanhurskaus tuodaan, ja näky ja profeetta sinetillä vahvistetaan, ja kaikkeinpyhin voidellaan.
\par 25 Ja tiedä ja käsitä: siitä ajasta, jolloin tuli se sana, että Jerusalem on jälleen rakennettava, voideltuun, ruhtinaaseen, asti, on kuluva seitsemän vuosiviikkoa; ja kuusikymmentäkaksi vuosiviikkoa, niin se jälleen rakennetaan toreinensa ja vallihautoinensa, mutta keskellä ahtaita aikoja.
\par 26 Ja kuudenkymmenen kahden vuosiviikon mentyä tuhotaan voideltu, eikä häneltä jää ketään. Ja kaupungin ja pyhäkön hävittää hyökkäävän ruhtinaan väki, mutta hän itse saa loppunsa tulvassa. Ja loppuun asti on oleva sota: hävitys on säädetty.
\par 27 Ja hän tekee liiton raskaaksi monille yhden vuosiviikon ajaksi, ja puoleksi vuosiviikoksi hän lakkauttaa teurasuhrin ja ruokauhrin; ja hävittäjä tulee kauhistuksen siivillä. Tämä loppuu vasta, kun säädetty tuomio vuodatetaan hävittäjän ylitse."

\chapter{10}

\par 1 Persian kuninkaan Kooreksen kolmantena hallitusvuotena ilmoitettiin sana Danielille, jota kutsuttiin Beltsassariksi; ja se sana on totuus ja merkitsee suurta vaivaa. Ja hän ymmärsi sanan, ja hän käsitti näyn.
\par 2 Niinä päivinä minä, Daniel, murehdin kolmen viikon päivät.
\par 3 Herkullista ruokaa minä en syönyt, ei liha eikä viini tullut minun suuhuni, enkä minä voidellut itseäni öljyllä, ennenkuin kolmen viikon päivät olivat loppuun kuluneet.
\par 4 Ensimmäisen kuun kahdentenakymmenentenä neljäntenä päivänä minä olin suuren virran, Hiddekelin, rannalla.
\par 5 Minä nostin silmäni ja näin, ja katso: oli eräs mies, puettuna pellavavaatteisiin ja kupeet vyötettyinä Uufaan kullalla.
\par 6 Hänen ruumiinsa oli kuin krysoliitti, hänen kasvonsa olivat kuin salaman leimaus, hänen silmänsä kuin tulisoihdut, hänen käsivartensa ja jalkansa kuin kiiltävän vasken välke; ja hänen sanojensa ääni oli kuin suuren kansanjoukon pauhina.
\par 7 Ja minä, Daniel, yksin näin sen näyn, mutta miehet, jotka olivat minun kanssani, eivät näkyä nähneet; kuitenkin valtasi heidät suuri pelko, ja he pakenivat ja lymysivät.
\par 8 Ja minä jäin yksin. Ja kun minä näin tämän suuren näyn, meni minulta kaikki voima; minun verevä muotoni muuttui kaamean näköiseksi, eikä minussa ollut voimaa mihinkään.
\par 9 Ja minä kuulin hänen sanainsa äänen; ja kuullessani hänen sanainsa äänen minä vaivuin horroksiin kasvoilleni, kasvot maata vasten.
\par 10 Ja katso, käsi kosketti minua ja ravisti minut hereille, polvieni ja kätteni varaan.
\par 11 Ja hän sanoi minulle: "Daniel, sinä otollinen mies, ota vaari niistä sanoista, jotka minä sinulle puhun, ja nouse seisomaan, sillä minut on nyt lähetetty sinun tykösi". Ja hänen puhuessansa minulle tämän sanan minä nousin vavisten.
\par 12 Ja hän sanoi minulle: "Älä pelkää, Daniel, sillä ensimmäisestä päivästä asti, jona sinä taivutit sydämesi ymmärrykseen ja nöyryyteen Jumalasi edessä, ovat sinun sanasi tulleet kuulluiksi; ja sinun sanojesi tähden minä olen tullut.
\par 13 Persian valtakunnan enkeliruhtinas seisoi vastustamassa minua kaksikymmentäyksi päivää, mutta katso, Miikael, yksi ensimmäisistä enkeliruhtinaista, tuli minun avukseni, sillä minä olin jäänyt yksin sinne, Persian kuningasten tykö.
\par 14 Ja minä tulin opettamaan sinulle, mitä on tapahtuva sinun kansallesi päivien lopulla; sillä vielä tämäkin näky koskee niitä päiviä."
\par 15 Ja hänen näitä minulle puhuessansa minä käänsin kasvoni maahan päin ja olin ääneti.
\par 16 Ja katso, olento, ihmislasten muotoinen, kosketti minun huuliani; silloin minä avasin suuni ja puhuin ja sanoin edessäni seisovalle: "Herrani, nähdessäni näyn valtasi minut tuska, ja minulta meni kaikki voima.
\par 17 Ja kuinka voi herrani palvelija, tällainen kuin minä, puhutella sellaista, kuin minun herrani on? Sillä siitä asti ei minussa ole voimaa, tuskin enää henkeäkään."
\par 18 Silloin kosketti minua jälleen se ihmisen muotoinen ja vahvisti minua.
\par 19 Ja hän sanoi: "Älä pelkää, sinä otollinen mies, rauha olkoon sinulle. Vahvistu! Vahvistu!" Ja hänen puhuessaan minulle minä vahvistuin ja sanoin: "Puhukoon herrani, sillä sinä olet minua vahvistanut".
\par 20 Ja hän sanoi: "Tiedätkö, mitä varten minä olen tullut sinun tykösi? Nyt minä käyn jälleen sotimaan Persian enkeliruhtinasta vastaan, ja kun minä olen päässyt hänestä, niin katso: tulee Jaavanin enkeliruhtinas.
\par 21 Mutta minä ilmoitan sinulle, mitä on kirjoitettuna totuuden kirjassa. Eikä ole ketään muuta vahvistamassa minua heitä vastaan paitsi teidän enkeliruhtinaanne Miikael.

\chapter{11}

\par 1 Ja minä seisoin meedialaisen Daarejaveksen ensimmäisenä hallitusvuotena häntä vahvistamassa ja suojelemassa."
\par 2 Ja nyt minä ilmoitan sinulle totuuden: Katso, vielä nousee kolme kuningasta Persiassa, ja neljäs rikastuu kaikkia muita rikkaammaksi. Ja kun hän on vahvistunut rikkaudessaan, panee hän kaiken liikkeelle Jaavanin valtakuntaa vastaan.
\par 3 Sitten nousee sankarikuningas; hän hallitsee suurella vallalla ja tekee, mitä tahtoo.
\par 4 Mutta juuri kun hän on noussut, hajoaa hänen valtakuntansa ja jakautuu neljään taivaan tuuleen. Se ei joudu hänen jälkeläisilleen eikä ole niin mahtava kuin hänen hallitessaan. Sillä hänen valtakuntansa kukistuu ja joutuu muille, ei heille.
\par 5 Ja Etelän kuningas on voimistuva sekä yksi hänen ruhtinaistaan; tämä on voimistuva vielä enemmän kuin hän, ja hänen valtansa on oleva suuri valta.
\par 6 Ja vuosien kuluttua he tekevät keskenään liiton, ja Etelän kuninkaan tytär menee Pohjan kuninkaan tykö saadakseen aikaan sopimuksen. Mutta häneltä menee hänen käsivartensa voima; eikä pysy Pohjan kuningas, ei hänen käsivartensa. Ja tytär itse annetaan alttiiksi ja ne, jotka olivat hänet tuoneet, ja hänen isänsä ja se, joka häntä aikoinaan auttoi.
\par 7 Sitten hänen juurtensa vesoista nousee eräs hänen sijaansa ja tulee sotajoukkoa vastaan, tulee Pohjan kuninkaan linnoitukseen ja tekee heille mielensä mukaan ja on väkevä.
\par 8 Myöskin heidän jumalansa ja valetut kuvansa ja kallisarvoiset astiansa, hopeat ja kullat hän vie saaliinansa Egyptiin; sitten hän muutamia vuosia pysyy Pohjan kuninkaasta erillään.
\par 9 Tämä hyökkää Etelän kuninkaan valtakuntaan, mutta palajaa takaisin maahansa.
\par 10 Ja hänen poikansa varustautuvat ja kokoavat suuret sotavoimat. Ja hän hyökkää ja kuohuu ja tulvii, hän tulee toistamiseen ja tunkeutuu hänen linnoitukseensa asti.
\par 11 Silloin Etelän kuningas kiukustuu ja lähtee sotimaan häntä vastaan, Pohjan kuningasta vastaan. Tämä nostattaa suuren joukon, mutta se joukko joutuu hänen valtaansa.
\par 12 Ja kun se joukko on raivattu pois, paisuu hänen sydämensä. Hän kaataa kymmeniä tuhansia, mutta ei ole kyllin vahva.
\par 13 Pohjan kuningas nostattaa jälleen joukon, entistä suuremman, ja muutaman ajan, muutaman vuoden kuluttua hyökkää suurella sotavoimalla ja runsailla varustuksilla.
\par 14 Niinä aikoina monet nousevat Etelän kuningasta vastaan; ja sinun omasta kansastasi nousee väkivallan miehiä, että näky kävisi toteen, mutta he itse lankeavat.
\par 15 Ja Pohjan kuningas hyökkää ja luo vallin ja valloittaa varustetun kaupungin. Eivät kestä Etelän käsivarret, ei sen valioväki, ei ole sillä voimaa seisoa vastaan.
\par 16 Ja hän, joka hyökkää sitä vastaan, tekee, mitä tahtoo, eikä kukaan voi seisoa häntä vastaan. Hän asettuu Ihanaan maahan, ja hävitys tulee hänen kätensä kautta.
\par 17 Ja nyt hän aikoo hyökätä valtakuntansa koko voimalla; mutta sopimus on hänellä mielessä, ja hän saa sen aikaan. Hän antaa hänelle yhden tyttäristään, tälle turmioksi. Mutta siitä ei tule pysyväistä eikä ole hänelle etua.
\par 18 Sitten hän kääntyy rantamaita vastaan ja ottaa valtaansa monet. Mutta eräs sotapäällikkö tekee hänen herjauksistaan lopun ja kostaa hänelle hänen herjauksensa.
\par 19 Silloin hän kääntyy oman maansa linnoituksiin, mutta kompastuu ja kaatuu, eikä häntä enää ole.
\par 20 Ja hänen sijaansa nousee eräs, joka antaa veronvaatijan käydä läpi valtakunnan ihanimman maan. Mutta muutamien päivien kuluttua hänet tuhotaan, ei kuitenkaan vihan väellä eikä sodalla.
\par 21 Ja hänen sijaansa nousee kelvoton, joka ei ollut saapa kuninkaan arvoa. Hän tulee keskellä rauhaa ja anastaa juonilla kuninkuuden.
\par 22 Ja sotajoukkojen tulva huuhtoutuu pois hänen edestänsä ja menee murskaksi, niin myös liiton ruhtinas.
\par 23 Siitä saakka kun liittoudutaan hänen kanssansa, hän harjoittaa petosta. Hän lähtee liikkeelle ja saa ylivallan vähällä väellä.
\par 24 Keskellä rauhaa hän hyökkää maakunnan lihavimpiin seutuihin ja tekee, mitä eivät hänen isänsä eivätkä hänen isiensä isät olleet tehneet: ryöstösaalista ja tavaraa hän jakelee omilleen; ja linnoituksia vastaan hän hankitsee juoniansa, säädettyyn aikaan asti.
\par 25 Ja hän panee liikkeelle voimansa ja rohkeutensa Etelän kuningasta vastaan, hyökäten suurella sotajoukolla. Mutta Etelän kuningas varustautuu sotaan suurella ja ylen väkevällä sotajoukolla. Hän ei kuitenkaan kestä, sillä häntä vastaan hankitaan juonia.
\par 26 Ja ne, jotka syövät hänen pöydästään, tuhoavat hänet, ja hänen sotajoukkonsa huuhdotaan pois, ja on paljon kaatuneita ja haavoitettuja.
\par 27 Ja kumpaisellakin kuninkaalla on paha mielessä toistansa vastaan. Samassa pöydässä he puhuvat valhetta; mutta se ei onnistu; sillä säädetty aika ei ole vielä lopussa.
\par 28 Silloin hän palaa maahansa paljoine tavaroineen, miettien hankkeita pyhää liittoa vastaan; hän toteuttaa ne ja palaa maahansa.
\par 29 Määräaikana hän hyökkää jälleen Etelämaahan, mutta tällä viimeisellä retkellä ei käy niinkuin ensimmäisellä.
\par 30 Häntä vastaan hyökkäävät kittiläisten laivat, ja hän menettää rohkeutensa, kääntyy takaisin ja purkaa kiukkunsa pyhää liittoa vastaan. Kotiin palattuaan hän suo huomiota niille, jotka hylkäävät pyhän liiton.
\par 31 Hänen lähettämänsä sotajoukot nousevat ja häväisevät pyhäkön linnoituksineen, poistavat jokapäiväisen uhrin ja asettavat sinne hävityksen kauhistuksen.
\par 32 Ja liitonrikkojat hän viettelee luopumukseen houkutuksillaan, mutta niitten joukko, jotka tuntevat Jumalansa, pysyy lujana ja tekee tehtävänsä.
\par 33 Ja taidolliset kansan seassa opettavat monta, mutta heitä sorretaan miekalla, tulella, vankeudella ja ryöstöllä, jonkun aikaa.
\par 34 Ja keskellä sortoa heille suodaan pieni menestys, ja monet liittyvät heihin teeskennellen.
\par 35 Ja taidollisista jotkut kompastuvat, että heidän joukkonsa koeteltaisiin, seulottaisiin ja puhdistettaisiin lopun ajaksi, sillä vielä kestää, ennenkuin määräaika on.
\par 36 Ja kuningas tekee, mitä hän tahtoo, ja korottaa itsensä ja uhittelee jokaista jumalaa, itse jumalien Jumalaa vastaan hän puhuu kauheita. Ja hän menestyy, kunnes vihan aika on lopussa; sillä mikä on säädetty, se tapahtuu.
\par 37 Hän ei välitä isäinsä jumalista, ei naisten lempijumalasta, eikä hän välitä mistään muustakaan jumalasta, sillä hän uhittelee niitä kaikkia.
\par 38 Mutta sen sijaan hän kunnioittaa linnoitusten jumalaa. Sitä jumalaa, jota hänen isänsä eivät tunteneet, hän kunnioittaa kullalla ja hopealla, kalliilla kivillä ja muilla kalleuksilla.
\par 39 Ja tätä hän tekee vahvoille linnoituksille - hän vieraine jumalineen. Niille, jotka hän omikseen tuntee, hän osoittaa suurta kunniaa ja panee heidät monien hallitsijaksi ja jakaa heille maata palkaksi.
\par 40 Mutta lopun ajalla Etelän kuningas iskee yhteen hänen kanssansa. Ja Pohjan kuningas käy tämän kimppuun vaunuilla ja ratsuilla ja monilla laivoilla, hyökkää hänen maihinsa, tulvana leviten niiden ylitse.
\par 41 Hän hyökkää myös Ihanaan maahan, ja monta kaatuu. Mutta hänen kädestänsä pelastuvat nämä: Edom ja Mooab ja ammonilaisten pääosa.
\par 42 Ja hän ojentaa kätensä maita kohti; Egyptin maa ei ole säästyvä.
\par 43 Hän valtaa kulta- ja hopea-aarteet ja kaikki Egyptin kalleudet, ja liibyalaiset ja etiopialaiset liittyvät häntä seuraamaan.
\par 44 Mutta sanomat idästä ja pohjoisesta säikähdyttävät häntä, ja hän lähtee täynnä kiukkua hävittämään monia ja vihkimään heitä tuhon omiksi.
\par 45 Hän pystyttää hovitelttansa meren ja pyhäkön ihanan vuoren välille. Mutta hänen loppunsa tulee, eikä häntä kukaan auta.

\chapter{12}

\par 1 Siihen aikaan nousee Miikael, se suuri enkeliruhtinas, joka seisoo sinun kansasi lasten suojana. Ja se on oleva ahdistuksen aika, jonka kaltaista ei ole ollut siitä saakka, kuin kansoja on ollut, hamaan siihen aikaan asti. Mutta siihen aikaan pelastetaan sinun kansasi, kaikki, jotka kirjaan kirjoitetut ovat.
\par 2 Ja monet maan tomussa makaavista heräjävät, toiset iankaikkiseen elämään, toiset häpeään ja iankaikkiseen kauhistukseen.
\par 3 Ja taidolliset loistavat, niinkuin taivaanvahvuus loistaa, ja ne, jotka monta vanhurskauteen saattavat, niinkuin tähdet, aina ja iankaikkisesti.
\par 4 Mutta sinä, Daniel, lukitse nämä sanat ja sinetöi tämä kirja lopun aikaan asti. Monet sitä tutkivat, ja ymmärrys lisääntyy.
\par 5 Ja minä, Daniel, näin, ja katso, siellä seisoi kaksi muuta, toinen virran tällä rannalla, toinen virran tuolla rannalla.
\par 6 Ja toinen sanoi pellavapukuiselle miehelle, joka oli virran vetten yläpuolella: "Kuinka kauan on vielä näitten ihmeellisten asiain loppuun?"
\par 7 Ja minä kuuntelin pellavapukuista miestä, joka oli virran vetten yläpuolella, ja hän nosti oikean ja vasemman kätensä taivasta kohti ja vannoi hänen kauttansa, joka elää iankaikkisesti: "Siihen on vielä aika, kaksi aikaa ja puoli aikaa. Ja kun pyhän kansan yhden osan hajotus on loppunut, silloin nämä kaikki täyttyvät."
\par 8 Ja minä kuulin, mutta en ymmärtänyt, ja minä sanoin: "Herrani, mikä on oleva näitten päätös?"
\par 9 Niin hän sanoi: "Mene, Daniel, sillä ne sanat pysyvät lukittuina ja sinetöityinä lopun aikaan asti.
\par 10 Monet puhdistetaan, kirkastetaan ja koetellaan, mutta jumalattomat pysyvät jumalattomina, eikä yksikään jumalaton ymmärrä tätä, mutta taidolliset ymmärtävät.
\par 11 Ja siitä ajasta, jolloin jokapäiväinen uhri poistetaan ja hävityksen kauhistus asetetaan, on oleva tuhat kaksisataa yhdeksänkymmentä päivää.
\par 12 Autuas se, joka odottaa ja saavuttaa tuhat kolmesataa kolmekymmentä viisi päivää.
\par 13 Mutta sinä, mene, siksi kunnes loppu tulee; ja lepää, ja nouse osaasi päivien lopussa."


\end{document}