\begin{document}

\title{Evankeliumi Matteuksen mukaan}


\chapter{1}

\par 1 Jeesuksen Kristuksen, Daavidin pojan, Aabrahamin pojan, syntykirja.
\par 2 Aabrahamille syntyi Iisak, Iisakille syntyi Jaakob, Jaakobille syntyi Juuda ja tämän veljet;
\par 3 Juudalle syntyi Faares ja Sera Taamarista, Faareelle syntyi Esrom, Esromille syntyi Aram;
\par 4 Aramille syntyi Aminadab, Aminadabille syntyi Nahasson, Nahassonille syntyi Salmon;
\par 5 Salmonille syntyi Booas Raahabista, Booaalle syntyi Oobed Ruutista, Oobedille syntyi Iisai;
\par 6 Iisaille syntyi Daavid, kuningas. Daavidille syntyi Salomo Uurian vaimosta;
\par 7 Salomolle syntyi Rehabeam, Rehabeamille syntyi Abia, Abialle syntyi Aasa;
\par 8 Aasalle syntyi Joosafat, Joosafatille syntyi Jooram, Jooramille syntyi Ussia;
\par 9 Ussialle syntyi Jootam, Jootamille syntyi Aahas, Aahaalle syntyi Hiskia;
\par 10 Hiskialle syntyi Manasse, Manasselle syntyi Aamon, Aamonille syntyi Joosia;
\par 11 Joosialle syntyi Jekonja ja tämän veljet Babyloniin siirtämisen aikoina.
\par 12 Babyloniin siirtämisen jälkeen Jekonjalle syntyi Sealtiel, Sealtielille syntyi Serubbaabel;
\par 13 Serubbaabelille syntyi Abiud, Abiudille syntyi Eljakim, Eljakimille syntyi Asor;
\par 14 Asorille syntyi Saadok, Saadokille syntyi Akim, Akimille syntyi Eliud;
\par 15 Eliudille syntyi Eleasar, Eleasarille syntyi Mattan, Mattanille syntyi Jaakob;
\par 16 Jaakobille syntyi Joosef, Marian mies, hänen, josta syntyi Jeesus, jota kutsutaan Kristukseksi.
\par 17 Näin on sukupolvia Aabrahamista Daavidiin kaikkiaan neljätoista polvea, ja Daavidista Babyloniin siirtämiseen neljätoista polvea, ja Babyloniin siirtämisestä Kristukseen asti neljätoista polvea.
\par 18 Jeesuksen Kristuksen syntyminen oli näin. Kun hänen äitinsä Maria oli kihlattu Joosefille, huomattiin hänen ennen heidän yhteenmenoaan olevan raskaana Pyhästä Hengestä.
\par 19 Mutta kun Joosef, hänen miehensä, oli hurskas, ja koska hän ei tahtonut saattaa häntä häpeään, aikoi hän salaisesti hyljätä hänet.
\par 20 Mutta kun hän tätä ajatteli, niin katso, hänelle ilmestyi unessa Herran enkeli, joka sanoi: "Joosef, Daavidin poika, älä pelkää ottaa tykösi Mariaa, vaimoasi; sillä se, mikä hänessä on siinnyt, on Pyhästä Hengestä.
\par 21 Ja hän on synnyttävä pojan, ja sinun on annettava hänelle nimi Jeesus, sillä hän on vapahtava kansansa heidän synneistänsä."
\par 22 Tämä kaikki on tapahtunut, että kävisi toteen, minkä Herra on puhunut profeetan kautta, joka sanoo:
\par 23 "Katso, neitsyt tulee raskaaksi ja synnyttää pojan, ja tälle on annettava nimi Immanuel", mikä käännettynä on: Jumala meidän kanssamme.
\par 24 Herättyään unesta Joosef teki, niinkuin Herran enkeli oli käskenyt hänen tehdä, ja otti vaimonsa tykönsä
\par 25 eikä yhtynyt häneen, ennenkuin hän oli synnyttänyt pojan. Ja hän antoi hänelle nimen Jeesus.

\chapter{2}

\par 1 Kun Jeesus oli syntynyt Juudean Beetlehemissä kuningas Herodeksen aikana, niin katso, tietäjiä tuli itäisiltä mailta Jerusalemiin,
\par 2 ja he sanoivat: "Missä on se äsken syntynyt juutalaisten kuningas? Sillä me näimme hänen tähtensä itäisillä mailla ja olemme tulleet häntä kumartamaan."
\par 3 Kun kuningas Herodes sen kuuli, hämmästyi hän ja koko Jerusalem hänen kanssaan.
\par 4 Ja hän kokosi kaikki kansan ylipapit ja kirjanoppineet ja kyseli heiltä, missä Kristus oli syntyvä.
\par 5 He sanoivat hänelle: "Juudean Beetlehemissä; sillä näin on kirjoitettu profeetan kautta:
\par 6 'Ja sinä Beetlehem, sinä Juudan seutu, et suinkaan ole vähäisin Juudan ruhtinasten joukossa, sillä sinusta on lähtevä hallitsija, joka kaitsee minun kansaani Israelia'."
\par 7 Silloin Herodes kutsui salaa tietäjät tykönsä ja tutkiskeli heiltä tarkoin, mihin aikaan tähti oli ilmestynyt.
\par 8 Ja hän lähetti heidät Beetlehemiin sanoen: "Menkää ja tiedustelkaa tarkasti lasta; ja kun sen löydätte, niin ilmoittakaa minulle, että minäkin tulisin häntä kumartamaan".
\par 9 Kuultuaan kuninkaan sanat he lähtivät matkalle; ja katso, tähti, jonka he olivat itäisillä mailla nähneet, kulki heidän edellään, kunnes se tuli sen paikan päälle, jossa lapsi oli, ja pysähtyi siihen.
\par 10 Nähdessään tähden he ihastuivat ylen suuresti.
\par 11 Niin he menivät huoneeseen ja näkivät lapsen ynnä Marian, hänen äitinsä. Ja he lankesivat maahan ja kumarsivat häntä, avasivat aarteensa ja antoivat hänelle lahjoja: kultaa ja suitsuketta ja mirhaa.
\par 12 Ja Jumala kielsi heitä unessa Herodeksen tykö palaamasta, ja he menivät toista tietä takaisin omaan maahansa.
\par 13 Mutta kun he olivat menneet, niin katso, Herran enkeli ilmestyi Joosefille unessa ja sanoi: "Nouse, ota lapsi ja hänen äitinsä ja pakene Egyptiin, ja ole siellä siihen asti, kuin minä sinulle sanon; sillä Herodes on etsivä lasta surmatakseen hänet".
\par 14 Niin hän nousi, otti yöllä lapsen ja hänen äitinsä ja lähti Egyptiin.
\par 15 Ja hän oli siellä Herodeksen kuolemaan asti; että kävisi toteen, minkä Herra on puhunut profeetan kautta, joka sanoo: "Egyptistä minä kutsuin poikani".
\par 16 Silloin Herodes, nähtyään, että tietäjät olivat hänet pettäneet, vihastui kovin ja lähetti tappamaan kaikki poikalapset Beetlehemistä ja koko sen ympäristöstä, kaksivuotiaat ja nuoremmat, sen mukaan kuin hän oli aikaa tietäjiltä tarkoin tiedustellut.
\par 17 Silloin kävi toteen, mikä on puhuttu profeetta Jeremiaan kautta, joka sanoo:
\par 18 "Ääni kuuluu Raamasta, itku ja iso parku; Raakel itkee lapsiansa eikä lohdutuksesta huoli, kun heitä ei enää ole".
\par 19 Mutta kun Herodes oli kuollut, niin katso, Herran enkeli ilmestyi unessa Joosefille Egyptissä
\par 20 ja sanoi: "Nouse, ota lapsi ja hänen äitinsä ja mene Israelin maahan, sillä ne ovat kuolleet, jotka väijyivät lapsen henkeä".
\par 21 Niin hän nousi, otti lapsen ja hänen äitinsä ja meni Israelin maahan.
\par 22 Mutta kun hän kuuli, että Arkelaus hallitsi Juudeaa isänsä Herodeksen jälkeen, niin hän pelkäsi mennä sinne. Ja hän sai unessa Jumalalta käskyn ja lähti Galilean alueelle.
\par 23 Ja sinne tultuaan hän asettui asumaan kaupunkiin, jonka nimi on Nasaret; että kävisi toteen, mikä profeettain kautta on puhuttu: "Hän on kutsuttava Nasaretilaiseksi".

\chapter{3}

\par 1 Niinä päivinä tuli Johannes Kastaja ja saarnasi Juudean erämaassa
\par 2 ja sanoi: "Tehkää parannus, sillä taivasten valtakunta on tullut lähelle".
\par 3 Sillä hän on se, josta profeetta Esaias puhuu sanoen: "Huutavan ääni kuuluu erämaassa: 'Valmistakaa Herralle tie, tehkää polut hänelle tasaisiksi'."
\par 4 Ja Johanneksella oli puku kamelinkarvoista ja vyötäisillään nahkavyö; ja hänen ruokanaan oli heinäsirkat ja metsähunaja.
\par 5 Silloin vaelsi hänen tykönsä Jerusalem ja koko Juudea ja kaikki Jordanin ympäristö,
\par 6 ja hän kastoi heidät Jordanin virrassa, kun he tunnustivat syntinsä.
\par 7 Mutta nähdessään paljon fariseuksia ja saddukeuksia tulevan kasteelle hän sanoi heille: "Te kyykäärmeitten sikiöt, kuka on neuvonut teitä pakenemaan tulevaista vihaa?
\par 8 Tehkää sentähden parannuksen soveliaita hedelmiä,
\par 9 älkääkä luulko saattavanne sanoa mielessänne: 'Onhan meillä isänä Aabraham'; sillä minä sanon teille, että Jumala voi näistä kivistä herättää lapsia Aabrahamille.
\par 10 Jo on kirves pantu puitten juurelle; jokainen puu, joka ei tee hyvää hedelmää, siis hakataan pois ja heitetään tuleen.
\par 11 Minä kastan teidät vedellä parannukseen, mutta se, joka minun jäljessäni tulee, on minua väkevämpi, jonka kenkiäkään minä en ole kelvollinen kantamaan; hän kastaa teidät Pyhällä Hengellä ja tulella.
\par 12 Hänellä on viskimensä kädessään, ja hän puhdistaa puimatanterensa ja kokoaa nisunsa aittaan, mutta ruumenet hän polttaa sammumattomassa tulessa."
\par 13 Silloin Jeesus tuli Galileasta Jordanille Johanneksen tykö hänen kastettavakseen.
\par 14 Mutta tämä esteli häntä sanoen: "Minun tarvitsee saada sinulta kaste, ja sinä tulet minun tyköni!"
\par 15 Jeesus vastasi ja sanoi hänelle: "Salli nyt; sillä näin meidän sopii täyttää kaikki vanhurskaus". Silloin hän salli sen hänelle.
\par 16 Kun Jeesus oli kastettu, nousi hän kohta vedestä, ja katso, taivaat aukenivat, ja hän näki Jumalan Hengen tulevan alas niinkuin kyyhkysen ja laskeutuvan hänen päällensä.
\par 17 Ja katso, taivaista kuului ääni, joka sanoi: "Tämä on minun rakas Poikani, johon minä olen mielistynyt".

\chapter{4}

\par 1 Sitten Henki vei Jeesuksen ylös erämaahan perkeleen kiusattavaksi.
\par 2 Ja kun Jeesus oli paastonnut neljäkymmentä päivää ja neljäkymmentä yötä, tuli hänen lopulta nälkä.
\par 3 Silloin kiusaaja tuli hänen luoksensa ja sanoi hänelle: "Jos sinä olet Jumalan Poika, niin käske näiden kivien muuttua leiviksi".
\par 4 Mutta hän vastasi ja sanoi: "Kirjoitettu on: 'Ei ihminen elä ainoastaan leivästä, vaan jokaisesta sanasta, joka Jumalan suusta lähtee'."
\par 5 Silloin perkele otti hänet kanssansa pyhään kaupunkiin ja asetti hänet pyhäkön harjalle
\par 6 ja sanoi hänelle: "Jos sinä olet Jumalan Poika, niin heittäydy tästä alas; sillä kirjoitettu on: 'Hän antaa enkeleilleen käskyn sinusta', ja: 'He kantavat sinua käsillänsä, ettet jalkaasi kiveen loukkaisi'."
\par 7 Jeesus sanoi hänelle: "Taas on kirjoitettu: 'Älä kiusaa Herraa, sinun Jumalaasi'."
\par 8 Taas perkele otti hänet kanssansa sangen korkealle vuorelle ja näytti hänelle kaikki maailman valtakunnat ja niiden loiston
\par 9 ja sanoi hänelle: "Tämän kaiken minä annan sinulle, jos lankeat maahan ja kumarrat minua".
\par 10 Silloin Jeesus sanoi hänelle: "Mene pois, saatana; sillä kirjoitettu on: 'Herraa, sinun Jumalaasi, pitää sinun kumartaman ja häntä ainoata palveleman'."
\par 11 Silloin perkele jätti hänet; ja katso, enkeleitä tuli hänen tykönsä, ja he tekivät hänelle palvelusta.
\par 12 Mutta kun Jeesus kuuli, että Johannes oli pantu vankeuteen, poistui hän Galileaan.
\par 13 Ja hän jätti Nasaretin ja meni asumaan Kapernaumiin, joka on meren rannalla, Sebulonin ja Naftalin alueella;
\par 14 että kävisi toteen, mikä on puhuttu profeetta Esaiaan kautta, joka sanoo:
\par 15 "Sebulonin maa ja Naftalin maa, meren tie, Jordanin tuonpuoleinen maa, pakanain Galilea -
\par 16 kansa, joka pimeydessä istui, näki suuren valkeuden, ja jotka istuivat kuoleman maassa ja varjossa, niille koitti valkeus".
\par 17 Siitä lähtien Jeesus rupesi saarnaamaan ja sanomaan: "Tehkää parannus, sillä taivasten valtakunta on tullut lähelle".
\par 18 Ja kulkiessaan Galilean järven rantaa hän näki kaksi veljestä, Simonin, jota kutsutaan Pietariksi, ja Andreaan, hänen veljensä, heittämässä verkkoa järveen; sillä he olivat kalastajia.
\par 19 Ja hän sanoi heille: "Seuratkaa minua, niin minä teen teistä ihmisten kalastajia".
\par 20 Niin he jättivät kohta verkot ja seurasivat häntä.
\par 21 Ja käydessään siitä eteenpäin hän näki toiset kaksi veljestä, Jaakobin, Sebedeuksen pojan, ja Johanneksen, hänen veljensä, venheessä isänsä Sebedeuksen kanssa laittamassa verkkojaan kuntoon; ja hän kutsui heidät.
\par 22 Niin he jättivät kohta venheen ja isänsä ja seurasivat häntä.
\par 23 Ja hän kierteli kautta koko Galilean ja opetti heidän synagoogissaan ja saarnasi valtakunnan evankeliumia ja paransi kaikkinaisia tauteja ja kaikkinaista raihnautta, mitä kansassa oli.
\par 24 Ja maine hänestä levisi koko Syyriaan, ja hänen luoksensa tuotiin kaikki sairastavaiset, monenlaisten tautien ja vaivojen rasittamat, riivatut, kuunvaihetautiset ja halvatut; ja hän paransi heidät.
\par 25 Ja häntä seurasi suuri kansan paljous Galileasta ja Dekapolista ja Jerusalemista ja Juudeasta ja Jordanin tuolta puolen.

\chapter{5}

\par 1 Kun hän näki kansanjoukot, nousi hän vuorelle; ja kun hän oli istuutunut, tulivat hänen opetuslapsensa hänen tykönsä.
\par 2 Niin hän avasi suunsa ja opetti heitä ja sanoi:
\par 3 "Autuaita ovat hengellisesti köyhät, sillä heidän on taivasten valtakunta.
\par 4 Autuaita ovat murheelliset, sillä he saavat lohdutuksen.
\par 5 Autuaita ovat hiljaiset, sillä he saavat maan periä.
\par 6 Autuaita ovat ne, jotka isoavat ja janoavat vanhurskautta, sillä heidät ravitaan.
\par 7 Autuaita ovat laupiaat, sillä he saavat laupeuden.
\par 8 Autuaita ovat puhdassydämiset, sillä he saavat nähdä Jumalan.
\par 9 Autuaita ovat rauhantekijät, sillä heidät pitää Jumalan lapsiksi kutsuttaman.
\par 10 Autuaita ovat ne, joita vanhurskauden tähden vainotaan, sillä heidän on taivasten valtakunta.
\par 11 Autuaita olette te, kun ihmiset minun tähteni teitä solvaavat ja vainoavat ja valhetellen puhuvat teistä kaikkinaista pahaa.
\par 12 Iloitkaa ja riemuitkaa, sillä teidän palkkanne on suuri taivaissa. Sillä samoin he vainosivat profeettoja, jotka olivat ennen teitä.
\par 13 Te olette maan suola; mutta jos suola käy mauttomaksi, millä se saadaan suolaiseksi? Se ei enää kelpaa mihinkään muuhun kuin pois heitettäväksi ja ihmisten tallattavaksi.
\par 14 Te olette maailman valkeus. Ei voi ylhäällä vuorella oleva kaupunki olla kätkössä;
\par 15 eikä lamppua sytytetä ja panna vakan alle, vaan lampunjalkaan, ja niin se loistaa kaikille huoneessa oleville.
\par 16 Niin loistakoon teidän valonne ihmisten edessä, että he näkisivät teidän hyvät tekonne ja ylistäisivät teidän Isäänne, joka on taivaissa.
\par 17 Älkää luulko, että minä olen tullut lakia tai profeettoja kumoamaan; en minä ole tullut kumoamaan, vaan täyttämään.
\par 18 Sillä totisesti minä sanon teille: kunnes taivas ja maa katoavat, ei laista katoa pieninkään kirjain, ei ainoakaan piirto, ennenkuin kaikki on tapahtunut.
\par 19 Sentähden, joka purkaa yhdenkään näistä pienimmistä käskyistä ja sillä tavalla opettaa ihmisiä, se pitää pienimmäksi taivasten valtakunnassa kutsuttaman; mutta joka niitä noudattaa ja niin opettaa, se pitää kutsuttaman suureksi taivasten valtakunnassa.
\par 20 Sillä minä sanon teille: ellei teidän vanhurskautenne ole paljoa suurempi kuin kirjanoppineiden ja fariseusten, niin te ette pääse taivasten valtakuntaan.
\par 21 Te olette kuulleet sanotuksi vanhoille: 'Älä tapa', ja: 'Joka tappaa, se on ansainnut oikeuden tuomion'.
\par 22 Mutta minä sanon teille: jokainen, joka vihastuu veljeensä, on ansainnut oikeuden tuomion; ja joka sanoo veljelleen: 'Sinä tyhjänpäiväinen', on ansainnut suuren neuvoston tuomion; ja joka sanoo: 'Sinä hullu', on ansainnut helvetin tulen.
\par 23 Sentähden, jos tuot lahjaasi alttarille ja siellä muistat, että veljelläsi on jotakin sinua vastaan,
\par 24 niin jätä lahjasi siihen alttarin eteen, ja käy ensin sopimassa veljesi kanssa, ja tule sitten uhraamaan lahjasi.
\par 25 Suostu pian sopimaan riitapuolesi kanssa, niin kauan kuin vielä olet hänen kanssaan tiellä, ettei riitapuolesi vetäisi sinua tuomarin eteen ja tuomari antaisi sinua oikeudenpalvelijalle, ja ettei sinua pantaisi vankeuteen.
\par 26 Totisesti minä sanon sinulle: sieltä et pääse, ennenkuin maksat viimeisenkin rovon.
\par 27 Te olette kuulleet sanotuksi: 'Älä tee huorin'.
\par 28 Mutta minä sanon teille: jokainen, joka katsoo naista himoiten häntä, on jo sydämessään tehnyt huorin hänen kanssansa.
\par 29 Jos sinun oikea silmäsi viettelee sinua, repäise se pois ja heitä luotasi; sillä parempi on sinulle, että yksi jäsenistäsi joutuu hukkaan, kuin että koko ruumiisi heitetään helvettiin.
\par 30 Ja jos sinun oikea kätesi viettelee sinua, hakkaa se poikki ja heitä luotasi; sillä parempi on sinulle, että yksi jäsenistäsi joutuu hukkaan, kuin että koko ruumiisi menee helvettiin.
\par 31 On sanottu: 'Joka hylkää vaimonsa, antakoon hänelle erokirjan'.
\par 32 Mutta minä sanon teille: jokainen, joka hylkää vaimonsa muun kuin huoruuden tähden, saattaa hänet tekemään huorin, ja joka nai hyljätyn, tekee huorin.
\par 33 Vielä olette kuulleet sanotuksi vanhoille: 'Älä vanno väärin', ja: 'Täytä Herralle valasi'.
\par 34 Mutta minä sanon teille: älkää ensinkään vannoko, älkää taivaan kautta, sillä se on Jumalan valtaistuin,
\par 35 älkääkä maan kautta, sillä se on hänen jalkojensa astinlauta, älkää myöskään Jerusalemin kautta, sillä se on suuren Kuninkaan kaupunki;
\par 36 äläkä vanno pääsi kautta, sillä et sinä voi yhtäkään hiusta tehdä valkeaksi etkä mustaksi;
\par 37 vaan olkoon teidän puheenne: 'On, on', tahi: 'ei, ei'. Mitä siihen lisätään, se on pahasta.
\par 38 Te olette kuulleet sanotuksi: 'Silmä silmästä ja hammas hampaasta'.
\par 39 Mutta minä sanon teille: älkää tehkö pahalle vastarintaa; vaan jos joku lyö sinua oikealle poskelle, käännä hänelle toinenkin;
\par 40 ja jos joku tahtoo sinun kanssasi käydä oikeutta ja ottaa ihokkaasi, anna hänen saada vaippasikin;
\par 41 ja jos joku pakottaa sinua yhden virstan matkalle, kulje hänen kanssaan kaksi.
\par 42 Anna sille, joka sinulta anoo, äläkä käännä selkääsi sille, joka sinulta lainaa pyytää.
\par 43 Te olette kuulleet sanotuksi: 'Rakasta lähimmäistäsi ja vihaa vihollistasi'.
\par 44 Mutta minä sanon teille: rakastakaa vihollisianne ja rukoilkaa niiden puolesta, jotka teitä vainoavat,
\par 45 että olisitte Isänne lapsia, joka on taivaissa; sillä hän antaa aurinkonsa koittaa niin pahoille kuin hyvillekin, ja antaa sataa niin väärille kuin vanhurskaillekin.
\par 46 Sillä jos te rakastatte niitä, jotka teitä rakastavat, mikä palkka teille siitä on tuleva? Eivätkö publikaanitkin tee samoin?
\par 47 Ja jos te osoitatte ystävällisyyttä ainoastaan veljillenne, mitä erinomaista te siinä teette? Eivätkö pakanatkin tee samoin?
\par 48 Olkaa siis te täydelliset, niinkuin teidän taivaallinen Isänne täydellinen on."

\chapter{6}

\par 1 "Kavahtakaa, ettette harjoita vanhurskauttanne ihmisten nähden, että he teitä katselisivat; muutoin ette saa palkkaa Isältänne, joka on taivaissa.
\par 2 Sentähden, kun annat almuja, älä soitata torvea edelläsi, niinkuin ulkokullatut tekevät synagoogissa ja kaduilla saadakseen ylistystä ihmisiltä. Totisesti minä sanon teille: he ovat saaneet palkkansa.
\par 3 Vaan kun sinä almua annat, älköön vasen kätesi tietäkö, mitä oikea kätesi tekee,
\par 4 että almusi olisi salassa; ja sinun Isäsi, joka salassa näkee, maksaa sinulle.
\par 5 Ja kun rukoilette, älkää olko niinkuin ulkokullatut; sillä he mielellään seisovat ja rukoilevat synagoogissa ja katujen kulmissa, että ihmiset heidät näkisivät. Totisesti minä sanon teille: he ovat saaneet palkkansa.
\par 6 Vaan sinä, kun rukoilet, mene kammioosi ja sulje ovesi ja rukoile Isääsi, joka on salassa; ja sinun Isäsi, joka salassa näkee, maksaa sinulle.
\par 7 Ja kun rukoilette, niin älkää tyhjiä hokeko niinkuin pakanat, jotka luulevat, että heitä heidän monisanaisuutensa tähden kuullaan.
\par 8 Älkää siis olko heidän kaltaisiaan; sillä teidän Isänne kyllä tietää, mitä te tarvitsette, ennenkuin häneltä anottekaan.
\par 9 Rukoilkaa siis te näin: Isä meidän, joka olet taivaissa! Pyhitetty olkoon sinun nimesi;
\par 10 tulkoon sinun valtakuntasi; tapahtukoon sinun tahtosi myös maan päällä niinkuin taivaassa;
\par 11 anna meille tänä päivänä meidän jokapäiväinen leipämme;
\par 12 ja anna meille meidän velkamme anteeksi, niinkuin mekin annamme anteeksi meidän velallisillemme;
\par 13 äläkä saata meitä kiusaukseen; vaan päästä meidät pahasta, [sillä sinun on valtakunta ja voima ja kunnia iankaikkisesti. Amen.]
\par 14 Sillä jos te annatte anteeksi ihmisille heidän rikkomuksensa, niin teidän taivaallinen Isänne myös antaa teille anteeksi;
\par 15 mutta jos te ette anna ihmisille anteeksi, niin ei myöskään teidän Isänne anna anteeksi teidän rikkomuksianne.
\par 16 Ja kun paastoatte, älkää olko synkännäköisiä niinkuin ulkokullatut; sillä he tekevät kasvonsa surkeiksi, että ihmiset näkisivät heidän paastoavan. Totisesti minä sanon teille: he ovat saaneet palkkansa.
\par 17 Vaan kun sinä paastoat, niin voitele pääsi ja pese kasvosi,
\par 18 etteivät paastoamistasi näkisi ihmiset, vaan sinun Isäsi, joka on salassa; ja sinun Isäsi, joka salassa näkee, maksaa sinulle.
\par 19 Älkää kootko itsellenne aarteita maan päälle, missä koi ja ruoste raiskaa ja missä varkaat murtautuvat sisään ja varastavat.
\par 20 Vaan kootkaa itsellenne aarteita taivaaseen, missä ei koi eikä ruoste raiskaa ja missä eivät varkaat murtaudu sisään eivätkä varasta.
\par 21 Sillä missä sinun aarteesi on, siellä on myös sinun sydämesi.
\par 22 Silmä on ruumiin lamppu. Jos siis silmäsi on terve, niin koko sinun ruumiisi on valaistu.
\par 23 Mutta jos silmäsi on viallinen, niin koko ruumiisi on pimeä. Jos siis se valo, joka sinussa on, on pimeyttä, kuinka suuri onkaan pimeys!
\par 24 Ei kukaan voi palvella kahta herraa; sillä hän on joko tätä vihaava ja toista rakastava, taikka tähän liittyvä ja toista halveksiva. Ette voi palvella Jumalaa ja mammonaa.
\par 25 Sentähden minä sanon teille: älkää murehtiko hengestänne, mitä söisitte tai mitä joisitte, älkääkä ruumiistanne, mitä päällenne pukisitte. Eikö henki ole enemmän kuin ruoka ja ruumis enemmän kuin vaatteet?
\par 26 Katsokaa taivaan lintuja: eivät ne kylvä eivätkä leikkaa eivätkä kokoa aittoihin, ja teidän taivaallinen Isänne ruokkii ne. Ettekö te ole paljoa suurempiarvoiset kuin ne?
\par 27 Ja kuka teistä voi murehtimisellaan lisätä ikäänsä kyynäränkään vertaa?
\par 28 Ja mitä te murehditte vaatteista? Katselkaa kedon kukkia, kuinka ne kasvavat; eivät ne työtä tee eivätkä kehrää.
\par 29 Kuitenkin minä sanon teille: ei Salomo kaikessa loistossansa ollut niin vaatetettu kuin yksi niistä.
\par 30 Jos siis Jumala näin vaatettaa kedon ruohon, joka tänään kasvaa ja huomenna uuniin heitetään, eikö paljoa ennemmin teitä, te vähäuskoiset?
\par 31 Älkää siis murehtiko sanoen: 'Mitä me syömme?' tahi: 'Mitä me juomme?' tahi: 'Millä me itsemme vaatetamme?'
\par 32 Sillä tätä kaikkea pakanat tavoittelevat. Teidän taivaallinen Isänne kyllä tietää teidän kaikkea tätä tarvitsevan.
\par 33 Vaan etsikää ensin Jumalan valtakuntaa ja hänen vanhurskauttansa, niin myös kaikki tämä teille annetaan.
\par 34 Älkää siis murehtiko huomisesta päivästä, sillä huominen päivä pitää murheen itsestään. Riittää kullekin päivälle oma vaivansa."

\chapter{7}

\par 1 "Älkää tuomitko, ettei teitä tuomittaisi;
\par 2 sillä millä tuomiolla te tuomitsette, sillä teidät tuomitaan; ja millä mitalla te mittaatte, sillä teille mitataan.
\par 3 Kuinka näet rikan, joka on veljesi silmässä, mutta et huomaa malkaa omassa silmässäsi?
\par 4 Taikka kuinka saatat sanoa veljellesi: 'Annas, minä otan rikan silmästäsi', ja katso, malka on omassa silmässäsi?
\par 5 Sinä ulkokullattu, ota ensin malka omasta silmästäsi, ja sitten sinä näet ottaa rikan veljesi silmästä.
\par 6 Älkää antako pyhää koirille, älkääkä heittäkö helmiänne sikojen eteen, etteivät ne tallaisi niitä jalkoihinsa ja kääntyisi ja repisi teitä.
\par 7 Anokaa, niin teille annetaan; etsikää, niin te löydätte; kolkuttakaa, niin teille avataan.
\par 8 Sillä jokainen anova saa, ja etsivä löytää, ja kolkuttavalle avataan.
\par 9 Vai kuka teistä on se ihminen, joka antaa pojallensa kiven, kun tämä pyytää häneltä leipää,
\par 10 taikka, kun hän pyytää kalaa, antaa hänelle käärmeen?
\par 11 Jos siis te, jotka olette pahoja, osaatte antaa lapsillenne hyviä lahjoja, kuinka paljoa ennemmin teidän Isänne, joka on taivaissa, antaa sitä, mikä hyvää on, niille, jotka sitä häneltä anovat!
\par 12 Sentähden, kaikki, mitä te tahdotte ihmisten teille tekevän, tehkää myös te samoin heille; sillä tämä on laki ja profeetat.
\par 13 Menkää ahtaasta portista sisälle. Sillä se portti on avara ja tie lavea, joka vie kadotukseen, ja monta on, jotka siitä sisälle menevät;
\par 14 mutta se portti on ahdas ja tie kaita, joka vie elämään, ja harvat ovat ne, jotka sen löytävät.
\par 15 Kavahtakaa vääriä profeettoja, jotka tulevat teidän luoksenne lammastenvaatteissa, mutta sisältä ovat raatelevaisia susia.
\par 16 Heidän hedelmistään te tunnette heidät. Eihän orjantappuroista koota viinirypäleitä eikä ohdakkeista viikunoita?
\par 17 Näin jokainen hyvä puu tekee hyviä hedelmiä, mutta huono puu tekee pahoja hedelmiä.
\par 18 Ei saata hyvä puu kasvaa pahoja hedelmiä eikä huono puu kasvaa hyviä hedelmiä.
\par 19 Jokainen puu, joka ei tee hyvää hedelmää, hakataan pois ja heitetään tuleen.
\par 20 Niin te siis tunnette heidät heidän hedelmistään.
\par 21 Ei jokainen, joka sanoo minulle: 'Herra, Herra!', pääse taivasten valtakuntaan, vaan se, joka tekee minun taivaallisen Isäni tahdon.
\par 22 Moni sanoo minulle sinä päivänä: 'Herra, Herra, emmekö me sinun nimesi kautta ennustaneet ja sinun nimesi kautta ajaneet ulos riivaajia ja sinun nimesi kautta tehneet monta voimallista tekoa?'
\par 23 Ja silloin minä lausun heille julki: 'Minä en ole koskaan teitä tuntenut; menkää pois minun tyköäni, te laittomuuden tekijät'.
\par 24 Sentähden on jokainen, joka kuulee nämä minun sanani ja tekee niiden mukaan, verrattava ymmärtäväiseen mieheen, joka huoneensa kalliolle rakensi.
\par 25 Ja rankkasade lankesi, ja virrat tulvivat, ja tuulet puhalsivat ja syöksyivät sitä huonetta vastaan, mutta se ei sortunut, sillä se oli kalliolle perustettu.
\par 26 Ja jokainen, joka kuulee nämä minun sanani eikä tee niiden mukaan, on verrattava tyhmään mieheen, joka huoneensa hiekalle rakensi.
\par 27 Ja rankkasade lankesi, ja virrat tulvivat, ja tuulet puhalsivat ja syöksähtivät sitä huonetta vastaan, ja se sortui, ja sen sortuminen oli suuri."
\par 28 Ja kun Jeesus lopetti nämä puheet, olivat kansanjoukot hämmästyksissään hänen opetuksestansa,
\par 29 sillä hän opetti heitä niinkuin se, jolla valta on, eikä niinkuin heidän kirjanoppineensa.

\chapter{8}

\par 1 Kun hän astui alas vuorelta, seurasi häntä suuri kansan paljous.
\par 2 Ja katso, tuli pitalinen mies ja kumartui maahan hänen eteensä ja sanoi: "Herra, jos tahdot, niin sinä voit minut puhdistaa".
\par 3 Niin hän ojensi kätensä, kosketti häntä ja sanoi: "Minä tahdon; puhdistu". Ja kohta hän puhdistui pitalistaan.
\par 4 Ja Jeesus sanoi hänelle: "Katso, ettet puhu tästä kenellekään; vaan mene ja näytä itsesi papille, ja uhraa lahja, jonka Mooses on säätänyt, todistukseksi heille".
\par 5 Ja kun hän saapui Kapernaumiin, tuli hänen tykönsä sadanpäämies ja rukoili häntä
\par 6 ja sanoi: "Herra, minun palvelijani makaa kotona halvattuna ja on kovissa vaivoissa".
\par 7 Hän sanoi hänelle: "Minä tulen ja parannan hänet".
\par 8 Mutta sadanpäämies vastasi ja sanoi: "Herra, en minä ole sen arvoinen, että tulisit minun kattoni alle; vaan sano ainoastaan sana, niin minun palvelijani paranee.
\par 9 Sillä minä itsekin olen toisen vallan alainen, ja minulla on sotamiehiä käskyni alaisina, ja minä sanon tälle: 'Mene', ja hän menee, ja toiselle: 'Tule', ja hän tulee, ja palvelijalleni: 'Tee tämä', ja hän tekee."
\par 10 Tämän kuultuaan Jeesus ihmetteli ja sanoi niille, jotka häntä seurasivat: "Totisesti minä sanon teille: en ole kenelläkään Israelissa löytänyt näin suurta uskoa.
\par 11 Ja minä sanon teille: monet tulevat idästä ja lännestä ja aterioitsevat Aabrahamin ja Iisakin ja Jaakobin kanssa taivasten valtakunnassa;
\par 12 mutta valtakunnan lapset heitetään ulos pimeyteen; siellä on oleva itku ja hammasten kiristys."
\par 13 Ja Jeesus sanoi sadanpäämiehelle: "Mene. Niinkuin sinä uskot, niin sinulle tapahtukoon." Ja palvelija parani sillä hetkellä.
\par 14 Kun Jeesus tuli Pietarin kotiin, näki hän hänen anoppinsa makaavan sairaana kuumeessa.
\par 15 Niin hän koski tämän käteen, ja kuume lähti hänestä; ja hän nousi ja palveli häntä.
\par 16 Mutta illan tultua tuotiin hänen tykönsä monta riivattua. Ja hän ajoi henget ulos sanalla, ja kaikki sairaat hän paransi;
\par 17 että kävisi toteen, mikä on puhuttu profeetta Esaiaan kautta, joka sanoo: "Hän otti päällensä meidän sairautemme ja kantoi meidän tautimme".
\par 18 Mutta kun Jeesus näki paljon kansaa ympärillään, käski hän lähteä toiselle rannalle.
\par 19 Ja eräs kirjanoppinut tuli ja sanoi hänelle: "Opettaja, minä seuraan sinua, mihin ikinä menet".
\par 20 Niin Jeesus sanoi hänelle: "Ketuilla on luolat ja taivaan linnuilla pesät, mutta Ihmisen Pojalla ei ole, mihin hän päänsä kallistaisi".
\par 21 Ja eräs toinen hänen opetuslapsistaan sanoi hänelle: "Herra, salli minun ensin käydä hautaamassa isäni".
\par 22 Mutta Jeesus sanoi hänelle: "Seuraa sinä minua, ja anna kuolleitten haudata kuolleensa".
\par 23 Ja hän astui venheeseen, ja hänen opetuslapsensa seurasivat häntä.
\par 24 Ja katso, järvellä nousi kova myrsky, niin että venhe peittyi aaltoihin; mutta hän nukkui.
\par 25 Niin he menivät ja herättivät hänet sanoen: "Herra, auta, me hukumme".
\par 26 Hän sanoi heille: "Te vähäuskoiset, miksi olette pelkureita?" Silloin hän nousi ja nuhteli tuulia ja järveä, ja tuli aivan tyven.
\par 27 Ja ihmiset ihmettelivät ja sanoivat: "Millainen tämä on, kun sekä tuulet että meri häntä tottelevat?"
\par 28 Kun hän tuli toiselle rannalle, gadaralaisten alueelle, tuli häntä vastaan kaksi riivattua, jotka olivat haudoista lähteneet ja olivat kovin raivoisia, niin ettei kukaan voinut sitä tietä kulkea.
\par 29 Ja katso, he huusivat sanoen: "Mitä sinulla on meidän kanssamme tekemistä, sinä Jumalan Poika? Oletko tullut tänne vaivaamaan meitä ennen aikaa?"
\par 30 Ja etäällä heistä kävi suuri sikalauma laitumella.
\par 31 Niin riivaajahenget pyysivät häntä sanoen: "Jos ajat meidät pois, niin lähetä meidät sikalaumaan".
\par 32 Ja hän sanoi niille: "Menkää". Silloin ne lähtivät heistä ja menivät sikoihin. Ja katso, koko lauma syöksyi jyrkännettä alas järveen ja hukkui veteen.
\par 33 Mutta paimentajat pakenivat; ja he menivät kaupunkiin ja ilmoittivat kaikki, myöskin sen, miten riivattujen oli käynyt.
\par 34 Ja katso, koko kaupunki lähti Jeesusta vastaan; ja kun he hänet näkivät, pyysivät he häntä menemään pois heidän alueeltaan.

\chapter{9}

\par 1 Ja hän astui venheeseen, meni jälleen toiselle rannalle ja tuli omaan kaupunkiinsa.
\par 2 Ja katso, hänen tykönsä tuotiin halvattu mies, joka makasi vuoteella. Kun Jeesus näki heidän uskonsa, sanoi hän halvatulle: "Poikani, ole turvallisella mielellä; sinun syntisi annetaan sinulle anteeksi".
\par 3 Ja katso, muutamat kirjanoppineista sanoivat mielessään: "Tämä pilkkaa Jumalaa".
\par 4 Mutta Jeesus ymmärsi heidän ajatuksensa ja sanoi: "Miksi te ajattelette pahaa sydämessänne?
\par 5 Sillä kumpi on helpompaa, sanoako: 'Sinun syntisi annetaan sinulle anteeksi', vai sanoa: 'Nouse ja käy'?
\par 6 Mutta tietääksenne, että Ihmisen Pojalla on valta maan päällä antaa syntejä anteeksi, niin" - hän sanoi halvatulle - "nouse, ota vuoteesi ja mene kotiisi."
\par 7 Ja hän nousi ja lähti kotiinsa.
\par 8 Mutta kun kansanjoukot sen näkivät, peljästyivät he ja ylistivät Jumalaa, joka oli antanut senkaltaisen vallan ihmisille.
\par 9 Ja sieltä kulkiessaan ohi Jeesus näki miehen, jonka nimi oli Matteus, istumassa tulliasemalla ja sanoi hänelle: "Seuraa minua". Niin tämä nousi ja seurasi häntä.
\par 10 Ja kun hän aterioi hänen kodissaan, niin katso, tuli monta publikaania ja syntistä, ja he aterioivat Jeesuksen ja hänen opetuslastensa kanssa.
\par 11 Ja kun fariseukset sen näkivät, sanoivat he hänen opetuslapsilleen: "Miksi teidän opettajanne syö publikaanien ja syntisten kanssa?"
\par 12 Mutta kun Jeesus sen kuuli, sanoi hän: "Eivät terveet tarvitse parantajaa, vaan sairaat.
\par 13 Mutta menkää ja oppikaa, mitä tämä on: 'Laupeutta minä tahdon enkä uhria'. Sillä en minä ole tullut kutsumaan vanhurskaita, vaan syntisiä."
\par 14 Silloin Johanneksen opetuslapset tulivat hänen tykönsä ja sanoivat: "Me ja fariseukset paastoamme paljon; miksi sinun opetuslapsesi eivät paastoa?"
\par 15 Niin Jeesus sanoi heille: "Eiväthän häävieraat voi surra, niinkauan kuin ylkä on heidän kanssaan? Mutta päivät tulevat, jolloin ylkä otetaan heiltä pois, ja silloin he paastoavat.
\par 16 Ei kukaan pane vanuttamattomasta kankaasta paikkaa vanhaan vaippaan, sillä semmoinen täytetilkku repii palasen vaipasta, ja reikä tulee pahemmaksi.
\par 17 Eikä nuorta viiniä lasketa vanhoihin nahkaleileihin; muutoin leilit pakahtuvat, ja viini juoksee maahan, ja leilit turmeltuvat; vaan nuori viini lasketaan uusiin leileihin, ja niin molemmat säilyvät."
\par 18 Kun hän tätä heille puhui, niin katso, eräs päämies tuli ja kumartui maahan hänen eteensä ja sanoi: "Minun tyttäreni kuoli juuri ikään, mutta tule ja pane kätesi hänen päällensä, niin hän virkoaa eloon".
\par 19 Niin Jeesus nousi ja seurasi häntä opetuslapsinensa.
\par 20 Ja katso, nainen, joka oli sairastanut verenjuoksua kaksitoista vuotta, tuli takaapäin ja kosketti hänen vaippansa tupsua.
\par 21 Sillä hän sanoi itsekseen: "Jos vain saan koskettaa hänen vaippaansa, niin minä tulen terveeksi".
\par 22 Silloin Jeesus kääntyi, näki hänet ja sanoi: "Tyttäreni, ole turvallisella mielellä; sinun uskosi on tehnyt sinut terveeksi". Ja sillä hetkellä nainen tuli terveeksi.
\par 23 Ja kun Jeesus tuli päämiehen taloon ja näki huilunsoittajat ja hälisevän väkijoukon,
\par 24 sanoi hän: "Menkää pois, sillä tyttö ei ole kuollut, vaan nukkuu". Niin he nauroivat häntä.
\par 25 Mutta kun väkijoukko oli ajettu ulos, meni hän sisälle ja tarttui hänen käteensä; ja tyttö nousi.
\par 26 Ja sanoma tästä levisi koko siihen maahan.
\par 27 Ja kun Jeesus kulki sieltä, seurasi häntä kaksi sokeaa huutaen ja sanoen: "Daavidin poika, armahda meitä".
\par 28 Ja hänen mentyänsä huoneeseen sokeat tulivat hänen tykönsä; ja Jeesus sanoi heille: "Uskotteko, että minä voin sen tehdä?" He sanoivat hänelle: "Uskomme, Herra".
\par 29 Silloin hän kosketti heidän silmiänsä ja sanoi: "Tapahtukoon teille uskonne mukaan".
\par 30 Ja heidän silmänsä aukenivat. Ja Jeesus varoitti heitä vakavasti sanoen: "Katsokaa, ettei kukaan saa tästä tietää".
\par 31 Mutta he menivät pois ja levittivät sanomaa hänestä koko siihen maahan.
\par 32 Ja katso, näiden lähdettyä tuotiin hänen tykönsä mykkä mies, joka oli riivattu.
\par 33 Ja kun riivaaja oli ajettu ulos, niin mykkä puhui, ja kansa ihmetteli sanoen: "Tällaista ei ole Israelissa ikinä nähty".
\par 34 Mutta fariseukset sanoivat: "Riivaajain päämiehen voimalla hän ajaa ulos riivaajia".
\par 35 Ja Jeesus vaelsi kaikki kaupungit ja kylät ja opetti heidän synagoogissaan ja saarnasi valtakunnan evankeliumia ja paransi kaikkinaisia tauteja ja kaikkinaista raihnautta.
\par 36 Ja nähdessään kansanjoukot hänen tuli heitä sääli, kun he olivat nääntyneet ja hyljätyt niinkuin lampaat, joilla ei ole paimenta.
\par 37 Silloin hän sanoi opetuslapsillensa: "Eloa on paljon, mutta työmiehiä vähän.
\par 38 Rukoilkaa siis elon Herraa, että hän lähettäisi työmiehiä elonkorjuuseensa."

\chapter{10}

\par 1 Ja hän kutsui tykönsä ne kaksitoista opetuslastaan ja antoi heille vallan ajaa ulos saastaisia henkiä ja parantaa kaikkinaisia tauteja ja kaikkinaista raihnautta.
\par 2 Ja nämä ovat niiden kahdentoista apostolin nimet: ensimmäinen oli Simon, jota kutsuttiin Pietariksi, ja Andreas, hänen veljensä, sitten Jaakob Sebedeuksen poika, ja Johannes, hänen veljensä,
\par 3 Filippus ja Bartolomeus, Tuomas ja Matteus, publikaani, Jaakob, Alfeuksen poika, ja Lebbeus, lisänimeltä Taddeus,
\par 4 Simon Kananeus ja Juudas Iskariot, sama, joka hänet kavalsi.
\par 5 Nämä kaksitoista Jeesus lähetti ja käski heitä sanoen: "Älköön tienne viekö pakanain luokse, älkääkä menkö mihinkään samarialaisten kaupunkiin,
\par 6 vaan menkää ennemmin Israelin huoneen kadonneitten lammasten tykö.
\par 7 Ja missä kuljette, saarnatkaa ja sanokaa: 'Taivasten valtakunta on tullut lähelle'.
\par 8 Parantakaa sairaita, herättäkää kuolleita, puhdistakaa pitalisia, ajakaa ulos riivaajia. Lahjaksi olette saaneet, lahjaksi antakaa.
\par 9 Älkää varustako itsellenne kultaa, älkää hopeata älkääkä vaskea vyöhönne,
\par 10 älkää laukkua matkalle, älkää kahta ihokasta, älkää kenkiä, älkääkä sauvaa; sillä työmies on ruokansa ansainnut.
\par 11 Ja mihin kaupunkiin tai kylään te tulettekin, tiedustelkaa, kuka siellä on arvollinen, ja jääkää hänen luokseen, kunnes sieltä lähdette.
\par 12 Ja tullessanne taloon tervehtikää sitä.
\par 13 Ja jos talo on arvollinen, tulkoon sille teidän rauhanne; mutta jos se ei ole arvollinen, palatkoon teidän rauhanne teille takaisin.
\par 14 Ja missä teitä ei oteta vastaan eikä teidän sanojanne kuulla, lähtekää pois siitä talosta tai siitä kaupungista ja pudistakaa tomu jaloistanne.
\par 15 Totisesti minä sanon teille: Sodoman ja Gomorran maan on tuomiopäivänä oleva helpompi kuin sen kaupungin.
\par 16 Katso, minä lähetän teidät niinkuin lampaat susien keskelle; olkaa siis älykkäät kuin käärmeet ja viattomat kuin kyyhkyset.
\par 17 Kavahtakaa ihmisiä, sillä he vetävät teidät oikeuksiin, ja synagoogissaan he teitä ruoskivat;
\par 18 ja teidät viedään maaherrain ja kuningasten eteen minun tähteni, todistukseksi heille ja pakanoille.
\par 19 Mutta kun he vetävät teitä oikeuteen, älkää huolehtiko siitä, miten tahi mitä puhuisitte, sillä teille annetaan sillä hetkellä, mitä teidän on puhuminen.
\par 20 Sillä ette te itse puhu, vaan teidän Isänne Henki puhuu teissä.
\par 21 Ja veli antaa veljensä kuolemaan ja isä lapsensa, ja lapset nousevat vanhempiansa vastaan ja tappavat heidät.
\par 22 Ja te joudutte kaikkien vihattaviksi minun nimeni tähden; mutta joka vahvana pysyy loppuun asti, se pelastuu.
\par 23 Ja kun teitä vainotaan yhdessä kaupungissa, paetkaa toiseen; sillä totisesti minä sanon teille: te ette ehdi loppuun käydä Israelin kaupunkeja, ennenkuin Ihmisen Poika tulee.
\par 24 Ei ole opetuslapsi opettajaansa parempi, eikä palvelija parempi isäntäänsä.
\par 25 Opetuslapselle riittää, että hänelle käy niinkuin hänen opettajalleen, ja palvelijalle, että hänelle käy niinkuin hänen isännälleen. Jos he perheenisäntää ovat sanoneet Beelsebuliksi, kuinka paljoa enemmän hänen perheväkeään!
\par 26 Älkää siis peljätkö heitä. Sillä ei ole mitään peitettyä, mitä ei tule paljastetuksi, eikä mitään salattua, mikä ei tule tunnetuksi.
\par 27 Minkä minä sanon teille pimeässä, se puhukaa päivän valossa. Ja minkä kuulette kuiskattavan korvaanne, se julistakaa katoilta.
\par 28 Älkääkä peljätkö niitä, jotka tappavat ruumiin, mutta eivät voi tappaa sielua; vaan ennemmin peljätkää häntä, joka voi sekä sielun että ruumiin hukuttaa helvettiin.
\par 29 Eikö kahta varpusta myydä yhteen ropoon? Eikä yksikään niistä putoa maahan teidän Isänne sallimatta.
\par 30 Ovatpa teidän päänne hiuksetkin kaikki luetut.
\par 31 Älkää siis peljätkö; te olette suurempiarvoiset kuin monta varpusta.
\par 32 Sentähden, jokaisen, joka tunnustaa minut ihmisten edessä, minäkin tunnustan Isäni edessä, joka on taivaissa.
\par 33 Mutta joka kieltää minut ihmisten edessä, sen minäkin kiellän Isäni edessä, joka on taivaissa.
\par 34 Älkää luulko, että minä olen tullut tuomaan rauhaa maan päälle; en ole tullut tuomaan rauhaa, vaan miekan.
\par 35 Sillä minä olen tullut 'nostamaan pojan riitaan isäänsä vastaan ja tyttären äitiänsä vastaan ja miniän anoppiansa vastaan;
\par 36 ja ihmisen vihamiehiksi tulevat hänen omat perhekuntalaisensa'.
\par 37 Joka rakastaa isäänsä taikka äitiänsä enemmän kuin minua, se ei ole minulle sovelias; ja joka rakastaa poikaansa taikka tytärtänsä enemmän kuin minua, se ei ole minulle sovelias;
\par 38 ja joka ei ota ristiänsä ja seuraa minua, se ei ole minulle sovelias.
\par 39 Joka löytää elämänsä, kadottaa sen; ja joka kadottaa elämänsä minun tähteni, hän löytää sen.
\par 40 Joka ottaa tykönsä teidät, se ottaa tykönsä minut; ja joka ottaa minut tykönsä, ottaa tykönsä hänet, joka on minut lähettänyt.
\par 41 Joka profeetan ottaa tykönsä profeetan nimen tähden, saa profeetan palkan; ja joka vanhurskaan ottaa tykönsä vanhurskaan nimen tähden, saa vanhurskaan palkan.
\par 42 Ja kuka hyvänsä antaa yhdelle näistä pienistä maljallisen kylmää vettä, hänen juodaksensa, opetuslapsen nimen tähden, totisesti minä sanon teille: hän ei jää palkkaansa vaille."

\chapter{11}

\par 1 Ja kun Jeesus oli antanut kahdelletoista opetuslapselleen kaikki nämä käskyt, niin hän lähti sieltä opettamaan ja saarnaamaan heidän kaupunkeihinsa.
\par 2 Mutta kun Johannes vankilassa ollessaan kuuli Kristuksen teot, lähetti hän opetuslapsiansa
\par 3 sanomaan hänelle: "Oletko sinä se tuleva, vai pitääkö meidän toista odottaman?"
\par 4 Niin Jeesus vastasi ja sanoi heille: "Menkää ja kertokaa Johannekselle, mitä kuulette ja näette:
\par 5 sokeat saavat näkönsä, ja rammat kävelevät, pitaliset puhdistuvat, ja kuurot kuulevat, ja kuolleet herätetään, ja köyhille julistetaan evankeliumia.
\par 6 Ja autuas on se, joka ei loukkaannu minuun."
\par 7 Kun he olivat menneet, rupesi Jeesus puhumaan kansalle Johanneksesta: "Mitä te lähditte erämaahan katselemaan? Ruokoako, jota tuuli huojuttaa?
\par 8 Vai mitä lähditte katsomaan? Ihmistäkö, hienoihin vaatteisiin puettua? Katso, hienopukuiset ovat kuningasten kartanoissa.
\par 9 Vai mitä te lähditte? Profeettaako katsomaan? Totisesti, minä sanon teille: hän on enemmän kuin profeetta.
\par 10 Tämä on se, josta on kirjoitettu: 'Katso, minä lähetän enkelini sinun edelläsi, ja hän on valmistava tiesi sinun eteesi'.
\par 11 Totisesti minä sanon teille: ei ole vaimoista syntyneitten joukosta noussut suurempaa kuin Johannes Kastaja; mutta vähäisin taivasten valtakunnassa on suurempi kuin hän.
\par 12 Mutta Johannes Kastajan päivistä tähän asti hyökätään taivasten valtakuntaa vastaan, ja hyökkääjät tempaavat sen itselleen.
\par 13 Sillä kaikki profeetat ja laki ovat ennustaneet Johannekseen asti;
\par 14 ja jos tahdotte ottaa vastaan: hän on Elias, joka oli tuleva.
\par 15 Jolla on korvat, se kuulkoon.
\par 16 Mutta mihin minä vertaan tämän sukupolven? Se on lasten kaltainen, jotka istuvat toreilla ja huutavat toisilleen
\par 17 sanoen: 'Me soitimme teille huilua, ja te ette karkeloineet; me veisasimme itkuvirsiä, ja te ette valittaneet'.
\par 18 Sillä Johannes tuli, hän ei syö eikä juo, ja he sanovat: 'Hänessä on riivaaja'.
\par 19 Ihmisen Poika tuli, hän syö ja juo, ja he sanovat: 'Katso syömäriä ja viininjuojaa, publikaanien ja syntisten ystävää!' Ja viisaus on oikeaksi näytetty teoissansa."
\par 20 Sitten hän rupesi nuhtelemaan niitä kaupunkeja, joissa useimmat hänen voimalliset tekonsa olivat tapahtuneet, siitä, etteivät ne olleet tehneet parannusta:
\par 21 "Voi sinua, Korasin! Voi sinua, Beetsaida! Sillä jos ne voimalliset teot, jotka ovat tapahtuneet teissä, olisivat tapahtuneet Tyyrossa ja Siidonissa, niin nämä jo aikaa sitten olisivat säkissä ja tuhassa tehneet parannuksen.
\par 22 Mutta minä sanon teille: Tyyron ja Siidonin on tuomiopäivänä oleva helpompi kuin teidän.
\par 23 Ja sinä, Kapernaum, korotetaankohan sinut hamaan taivaaseen? Hamaan tuonelaan on sinun astuttava alas. Sillä jos ne voimalliset teot, jotka ovat tapahtuneet sinussa, olisivat tapahtuneet Sodomassa, niin se seisoisi vielä tänäkin päivänä.
\par 24 Mutta minä sanon teille: Sodoman maan on tuomiopäivänä oleva helpompi kuin sinun."
\par 25 Siihen aikaan Jeesus johtui puhumaan sanoen: "Minä ylistän sinua, Isä, taivaan ja maan Herra, että olet salannut nämä viisailta ja ymmärtäväisiltä ja ilmoittanut ne lapsenmielisille.
\par 26 Niin, Isä, sillä näin on sinulle hyväksi näkynyt.
\par 27 Kaikki on minun Isäni antanut minun haltuuni, eikä kukaan muu tunne Poikaa kuin Isä, eikä Isää tunne kukaan muu kuin Poika ja se, kenelle Poika tahtoo hänet ilmoittaa.
\par 28 Tulkaa minun tyköni, kaikki työtätekeväiset ja raskautetut, niin minä annan teille levon.
\par 29 Ottakaa minun ikeeni päällenne ja oppikaa minusta, sillä minä olen hiljainen ja nöyrä sydämeltä; niin te löydätte levon sielullenne.
\par 30 Sillä minun ikeeni on sovelias, ja minun kuormani on keveä."

\chapter{12}

\par 1 Siihen aikaan Jeesus kulki sapattina viljavainioiden halki; ja hänen opetuslastensa oli nälkä, ja he rupesivat katkomaan tähkäpäitä ja syömään.
\par 2 Mutta kun fariseukset sen näkivät, sanoivat he hänelle: "Katso, sinun opetuslapsesi tekevät, mitä ei ole lupa tehdä sapattina".
\par 3 Niin hän sanoi heille: "Ettekö ole lukeneet, mitä Daavid teki, kun hänen ja hänen seuralaistensa oli nälkä,
\par 4 kuinka hän meni Jumalan huoneeseen, ja kuinka he söivät näkyleivät, joita ei hänen eikä hänen seuralaistensa ollut lupa syödä, vaan ainoastaan pappien?
\par 5 Tai ettekö ole lukeneet laista, että papit sapattina pyhäkössä rikkovat sapatin ja ovat kuitenkin syyttömät?
\par 6 Mutta minä sanon teille: tässä on se, joka on pyhäkköä suurempi.
\par 7 Mutta jos tietäisitte, mitä tämä on: 'Laupeutta minä tahdon enkä uhria', niin te ette tuomitsisi syyttömiä.
\par 8 Sillä Ihmisen Poika on sapatin herra."
\par 9 Ja hän lähti sieltä ja tuli heidän synagoogaansa.
\par 10 Ja katso, siellä oli mies, jonka käsi oli kuivettunut. Niin he kysyivät häneltä sanoen: "Onko luvallista sapattina parantaa?" voidaksensa nostaa syytteen häntä vastaan.
\par 11 Niin hän sanoi heille: "Kuka teistä on se mies, joka ei, jos hänen ainoa lampaansa putoaa sapattina kuoppaan, tartu siihen ja nosta sitä ylös?
\par 12 Kuinka paljon suurempiarvoinen onkaan ihminen kuin lammas! Sentähden on lupa tehdä sapattina hyvää."
\par 13 Sitten hän sanoi miehelle: "Ojenna kätesi!" Ja hän ojensi; ja se tuli entiselleen, terveeksi niinkuin toinenkin.
\par 14 Niin fariseukset lähtivät ulos ja pitivät neuvoa häntä vastaan, surmataksensa hänet.
\par 15 Mutta kun Jeesus huomasi sen, väistyi hän sieltä pois. Ja monet seurasivat häntä, ja hän paransi heidät kaikki,
\par 16 ja hän varoitti vakavasti heitä saattamasta häntä julki;
\par 17 että kävisi toteen, mikä on puhuttu profeetta Esaiaan kautta, joka sanoo:
\par 18 "Katso, minun palvelijani, jonka minä olen valinnut, minun rakkaani, johon minun sieluni on mielistynyt; minä panen Henkeni häneen, ja hän on saattava oikeuden sanomaa pakanoille.
\par 19 Ei hän riitele eikä huuda, ei hänen ääntänsä kuule kukaan kaduilla.
\par 20 Särjettyä ruokoa hän ei muserra, ja suitsevaista kynttilänsydäntä hän ei sammuta, kunnes hän saattaa oikeuden voittoon.
\par 21 Ja hänen nimeensä pakanat panevat toivonsa."
\par 22 Silloin tuotiin hänen tykönsä riivattu mies, joka oli sokea ja mykkä, ja hän paransi hänet, niin että mykkä puhui ja näki.
\par 23 Ja kaikki kansa hämmästyi ja sanoi: "Eiköhän tämä ole Daavidin poika?"
\par 24 Mutta kun fariseukset sen kuulivat, sanoivat he: "Tämä ei aja riivaajia ulos kenenkään muun kuin Beelsebulin, riivaajain päämiehen, voimalla".
\par 25 Mutta hän tiesi heidän ajatuksensa ja sanoi heille: "Jokainen valtakunta, joka riitautuu itsensä kanssa, joutuu autioksi, eikä mikään kaupunki tai talo, joka riitautuu itsensä kanssa, pysy pystyssä.
\par 26 Jos nyt saatana ajaa ulos saatanan, niin hän on riitautunut itsensä kanssa; kuinka siis hänen valtakuntansa pysyy pystyssä?
\par 27 Ja jos minä Beelsebulin voimalla ajan ulos riivaajia, kenenkä voimalla sitten teidän lapsenne ajavat niitä ulos? Niinpä he tulevat olemaan teidän tuomarinne.
\par 28 Mutta jos minä Jumalan Hengen voimalla ajan ulos riivaajia, niin on Jumalan valtakunta tullut teidän tykönne.
\par 29 Taikka kuinka voi kukaan tunkeutua väkevän taloon ja ryöstää hänen tavaroitansa, ellei hän ensin sido sitä väkevää? Vasta sitten hän ryöstää tyhjäksi hänen talonsa.
\par 30 Joka ei ole minun kanssani, se on minua vastaan; ja joka ei minun kanssani kokoa, se hajottaa.
\par 31 Sentähden minä sanon teille: jokainen synti ja pilkka annetaan ihmisille anteeksi, mutta Hengen pilkkaamista ei anteeksi anneta.
\par 32 Ja jos joku sanoo sanan Ihmisen Poikaa vastaan, niin hänelle annetaan anteeksi; mutta jos joku sanoo jotakin Pyhää Henkeä vastaan, niin hänelle ei anteeksi anneta, ei tässä maailmassa eikä tulevassa.
\par 33 Joko tehkää puu hyväksi ja sen hedelmä hyväksi, tahi tehkää puu huonoksi ja sen hedelmä huonoksi; sillä hedelmästä puu tunnetaan.
\par 34 Te kyykäärmeitten sikiöt, kuinka te saattaisitte hyvää puhua, kun itse olette pahoja? Sillä sydämen kyllyydestä suu puhuu.
\par 35 Hyvä ihminen tuo hyvän runsaudesta esille hyvää, ja paha ihminen tuo pahan runsaudesta esille pahaa.
\par 36 Mutta minä sanon teille: jokaisesta turhasta sanasta, minkä ihmiset puhuvat, pitää heidän tekemän tili tuomiopäivänä.
\par 37 Sillä sanoistasi sinut julistetaan vanhurskaaksi, ja sanoistasi sinut tuomitaan syylliseksi."
\par 38 Silloin muutamat kirjanoppineista ja fariseuksista vastasivat hänelle sanoen: "Opettaja, me tahdomme nähdä sinulta merkin".
\par 39 Mutta hän vastasi heille ja sanoi: "Tämä paha ja avionrikkoja sukupolvi tavoittelee merkkiä, mutta sille ei anneta muuta merkkiä kuin profeetta Joonaan merkki.
\par 40 Sillä niinkuin Joonas oli meripedon vatsassa kolme päivää ja kolme yötä, niin on myös Ihmisen Poika oleva maan povessa kolme päivää ja kolme yötä.
\par 41 Niiniven miehet nousevat tuomiolle yhdessä tämän sukupolven kanssa ja tulevat sille tuomioksi; sillä he tekivät parannuksen Joonaan saarnan vaikutuksesta, ja katso, tässä on enempi kuin Joonas.
\par 42 Etelän kuningatar on heräjävä tuomiolle tämän sukupolven kanssa ja tuleva sille tuomioksi; sillä hän tuli maan ääristä kuulemaan Salomon viisautta, ja katso, tässä on enempi kuin Salomo.
\par 43 Kun saastainen henki lähtee ihmisestä, kuljeksii se autioita paikkoja ja etsii lepoa, eikä löydä.
\par 44 Silloin se sanoo: 'Minä palaan huoneeseeni, josta lähdin'. Ja kun se tulee, tapaa se huoneen tyhjänä ja lakaistuna ja kaunistettuna.
\par 45 Silloin se menee ja ottaa mukaansa seitsemän muuta henkeä, pahempaa kuin se itse, ja ne tulevat sisään ja asuvat siellä. Ja sen ihmisen viimeiset tulevat pahemmiksi kuin ensimmäiset. Niin käy myös tälle pahalle sukupolvelle."
\par 46 Hänen vielä puhuessaan kansalle, katso, hänen äitinsä ja veljensä seisoivat ulkona, tahtoen häntä puhutella.
\par 47 Niin joku sanoi hänelle: "Katso, sinun äitisi ja veljesi seisovat ulkona ja tahtovat sinua puhutella".
\par 48 Mutta hän vastasi ja sanoi sille, joka sen hänelle ilmoitti: "Kuka on minun äitini, ja ketkä ovat minun veljeni?"
\par 49 Ja hän ojensi kätensä opetuslastensa puoleen ja sanoi: "Katso, minun äitini ja veljeni!
\par 50 Sillä jokainen, joka tekee minun taivaallisen Isäni tahdon, on minun veljeni ja sisareni ja äitini."

\chapter{13}

\par 1 Sinä päivänä Jeesus lähti asunnostaan ja istui järven rannalle.
\par 2 Ja hänen tykönsä kokoontui paljon kansaa, jonka tähden hän astui venheeseen ja istuutui, ja kaikki kansa seisoi rannalla.
\par 3 Ja hän puhui heille paljon vertauksilla ja sanoi: "Katso, kylväjä meni kylvämään.
\par 4 Ja hänen kylväessään putosivat muutamat siemenet tien oheen, ja linnut tulivat ja söivät ne.
\par 5 Toiset putosivat kallioperälle, jossa niillä ei ollut paljon maata, ja ne nousivat kohta oraalle, kun niillä ei ollut syvää maata.
\par 6 Mutta auringon noustua ne paahtuivat, ja kun niillä ei ollut juurta, niin ne kuivettuivat.
\par 7 Toiset taas putosivat orjantappuroihin, ja orjantappurat nousivat ja tukahuttivat ne.
\par 8 Ja toiset putosivat hyvään maahan ja antoivat sadon, mitkä sata, mitkä kuusikymmentä, mitkä kolmekymmentä jyvää.
\par 9 Jolla on korvat, se kuulkoon."
\par 10 Niin hänen opetuslapsensa tulivat ja sanoivat hänelle: "Minkätähden sinä puhut heille vertauksilla?"
\par 11 Hän vastasi ja sanoi: "Sentähden, että teidän on annettu tuntea taivasten valtakunnan salaisuudet, mutta heidän ei ole annettu.
\par 12 Sillä sille, jolla on, annetaan, ja hänellä on oleva yltäkyllin; mutta siltä, jolla ei ole, otetaan pois sekin, mikä hänellä on.
\par 13 Sentähden minä puhun heille vertauksilla, että he näkevin silmin eivät näe ja kuulevin korvin eivät kuule, eivätkä ymmärrä.
\par 14 Ja heissä käy toteen Esaiaan ennustus, joka sanoo: 'Kuulemalla kuulkaa, älkääkä ymmärtäkö, ja näkemällä nähkää, älkääkä käsittäkö.
\par 15 Sillä paatunut on tämän kansan sydän, ja korvillaan he työläästi kuulevat, ja silmänsä he ovat ummistaneet, etteivät he näkisi silmillään, eivät kuulisi korvillaan, eivät ymmärtäisi sydämellään eivätkä kääntyisi ja etten minä heitä parantaisi.'
\par 16 Mutta autuaat ovat teidän silmänne, koska ne näkevät, ja teidän korvanne, koska ne kuulevat.
\par 17 Sillä totisesti minä sanon teille: monet profeetat ja vanhurskaat ovat halunneet nähdä, mitä te näette, eivätkä ole nähneet, ja kuulla, mitä te kuulette, eivätkä ole kuulleet.
\par 18 Kuulkaa siis te vertaus kylväjästä:
\par 19 Kun joku kuulee valtakunnan sanan eikä ymmärrä, niin tulee paha ja tempaa pois sen, mikä hänen sydämeensä kylvettiin. Tämä on se, mikä kylvettiin tien oheen.
\par 20 Mikä kallioperälle kylvettiin, on se, joka kuulee sanan ja heti ottaa sen ilolla vastaan;
\par 21 mutta hänellä ei ole juurta itsessään, vaan hän kestää ainoastaan jonkun aikaa, ja kun tulee ahdistus tai vaino sanan tähden, niin hän heti lankeaa pois.
\par 22 Mikä taas orjantappuroihin kylvettiin, on se, joka kuulee sanan, mutta tämän maailman huoli ja rikkauden viettelys tukahuttavat sanan, ja hän jää hedelmättömäksi.
\par 23 Mutta mikä hyvään maahan kylvettiin, on se, joka kuulee sanan ja ymmärtää sen ja myös tuottaa hedelmän ja tekee, mikä sata jyvää, mikä kuusikymmentä, mikä kolmekymmentä."
\par 24 Toisen vertauksen hän puhui heille sanoen: "Taivasten valtakunta on verrattava mieheen, joka kylvi hyvän siemenen peltoonsa.
\par 25 Mutta ihmisten nukkuessa hänen vihamiehensä tuli ja kylvi lustetta nisun sekaan ja meni pois.
\par 26 Ja kun laiho kasvoi ja teki hedelmää, silloin lustekin tuli näkyviin.
\par 27 Niin perheenisännän palvelijat tulivat ja sanoivat hänelle: 'Herra, etkö kylvänyt peltoosi hyvää siementä? Mistä siihen sitten on tullut lustetta?'
\par 28 Hän sanoi heille: 'Sen on vihamies tehnyt'. Niin palvelijat sanoivat hänelle: 'Tahdotko, että menemme ja kokoamme sen?'
\par 29 Mutta hän sanoi: 'En, ettette lustetta kootessanne nyhtäisi sen mukana nisuakin.
\par 30 Antakaa molempain kasvaa yhdessä elonleikkuuseen asti; ja elonaikana minä sanon leikkuumiehille: Kootkaa ensin luste ja sitokaa se kimppuihin poltettavaksi, mutta nisu korjatkaa minun aittaani.'"
\par 31 Vielä toisen vertauksen hän puhui heille sanoen: "Taivasten valtakunta on sinapinsiemenen kaltainen, jonka mies otti ja kylvi peltoonsa.
\par 32 Se on kaikista siemenistä pienin, mutta kun se on kasvanut, on se suurin vihanneskasveista ja tulee puuksi, niin että taivaan linnut tulevat ja tekevät pesänsä sen oksille."
\par 33 Taas toisen vertauksen hän puhui heille: "Taivasten valtakunta on hapatuksen kaltainen, jonka nainen otti ja sekoitti kolmeen vakalliseen jauhoja, kunnes kaikki happani".
\par 34 Tämän kaiken Jeesus puhui kansalle vertauksilla, ja ilman vertausta hän ei puhunut heille mitään;
\par 35 että kävisi toteen, mikä on puhuttu profeetan kautta, joka sanoo: "Minä avaan suuni vertauksiin, minä tuon ilmi sen, mikä on ollut salassa maailman perustamisesta asti".
\par 36 Sitten hän laski luotaan kansanjoukot ja meni asuntoonsa. Ja hänen opetuslapsensa tulivat hänen tykönsä ja sanoivat: "Selitä meille vertaus pellon lusteesta".
\par 37 Niin hän vastasi ja sanoi: "Hyvän siemenen kylväjä on Ihmisen Poika.
\par 38 Pelto on maailma; hyvä siemen ovat valtakunnan lapset, mutta lusteet ovat pahan lapset.
\par 39 Vihamies, joka ne kylvi, on perkele; elonaika on maailman loppu, ja leikkuumiehet ovat enkelit.
\par 40 Niinkuin lusteet kootaan ja tulessa poltetaan, niin on tapahtuva maailman lopussa.
\par 41 Ihmisen Poika lähettää enkelinsä, ja he kokoavat hänen valtakunnastaan kaikki, jotka ovat pahennukseksi ja jotka tekevät laittomuutta,
\par 42 ja heittävät heidät tuliseen pätsiin; siellä on oleva itku ja hammasten kiristys.
\par 43 Silloin vanhurskaat loistavat Isänsä valtakunnassa niinkuin aurinko. Jolla on korvat, se kuulkoon.
\par 44 Taivasten valtakunta on peltoon kätketyn aarteen kaltainen, jonka mies löysi ja kätki; ja siitä iloissaan hän meni ja myi kaikki, mitä hänellä oli, ja osti sen pellon.
\par 45 Vielä taivasten valtakunta on kuin kauppias, joka etsi kalliita helmiä,
\par 46 ja löydettyään yhden kallisarvoisen helmen hän meni ja myi kaikki, mitä hänellä oli, ja osti sen.
\par 47 Vielä taivasten valtakunta on nuotan kaltainen, joka heitettiin mereen ja kokosi kaikkinaisia kaloja.
\par 48 Ja kun se tuli täyteen, vetivät he sen rannalle, istuutuivat ja kokosivat hyvät astioihin, mutta kelvottomat he viskasivat pois.
\par 49 Näin on käyvä maailman lopussa; enkelit lähtevät ja erottavat pahat vanhurskaista
\par 50 ja heittävät heidät tuliseen pätsiin; siellä on oleva itku ja hammasten kiristys.
\par 51 Oletteko ymmärtäneet tämän kaiken?" He vastasivat hänelle: "Olemme".
\par 52 Ja hän sanoi heille: "Niin on jokainen kirjanoppinut, joka on tullut taivasten valtakunnan opetuslapseksi, perheenisännän kaltainen, joka tuo aarrekammiostaan esille uutta ja vanhaa".
\par 53 Ja kun Jeesus oli lopettanut nämä vertaukset, lähti hän sieltä.
\par 54 Ja hän tuli kotikaupunkiinsa ja opetti heitä heidän synagoogassaan, niin että he hämmästyivät ja sanoivat: "Mistä hänellä on tämä viisaus ja nämä voimalliset teot?
\par 55 Eikö tämä ole se rakentajan poika? Eikö hänen äitinsä ole nimeltään Maria ja hänen veljensä Jaakob ja Joosef ja Simon ja Juudas?
\par 56 Ja eivätkö hänen sisarensa ole kaikki meidän parissamme? Mistä sitten hänellä on tämä kaikki?"
\par 57 Ja he loukkaantuivat häneen. Mutta Jeesus sanoi heille: "Ei ole profeetta halveksittu muualla kuin kotikaupungissaan ja kodissaan".
\par 58 Ja heidän epäuskonsa tähden hän ei tehnyt siellä monta voimallista tekoa.

\chapter{14}

\par 1 Siihen aikaan neljännysruhtinas Herodes kuuli maineen Jeesuksesta.
\par 2 Ja hän sanoi palvelijoilleen: "Se on Johannes Kastaja; hän on noussut kuolleista, ja sentähden nämä voimat hänessä vaikuttavat".
\par 3 Sillä Herodes oli ottanut Johanneksen kiinni ja sitonut hänet ja pannut vankeuteen veljensä Filippuksen vaimon, Herodiaan, tähden.
\par 4 Sillä Johannes oli sanonut hänelle: "Sinun ei ole lupa pitää häntä".
\par 5 Ja Herodes olisi tahtonut tappaa Johanneksen, mutta pelkäsi kansaa, sillä he pitivät häntä profeettana.
\par 6 Mutta kun Herodeksen syntymäpäivä tuli, tanssi Herodiaan tytär heidän edessään, ja se miellytti Herodesta;
\par 7 sentähden hän valalla vannoen lupasi antaa hänelle, mitä ikinä hän anoisi.
\par 8 Niin hän äitinsä yllytyksestä sanoi: "Anna tuoda minulle tänne lautasella Johannes Kastajan pää".
\par 9 Silloin kuningas tuli murheelliseksi, mutta valansa ja pöytävierasten tähden hän käski antaa sen.
\par 10 Ja hän lähetti lyömään Johannekselta pään poikki vankilassa.
\par 11 Ja hänen päänsä tuotiin lautasella ja annettiin tytölle; ja tämä vei sen äidilleen.
\par 12 Ja hänen opetuslapsensa tulivat ja ottivat hänen ruumiinsa ja hautasivat hänet; ja he menivät ja ilmoittivat asian Jeesukselle.
\par 13 Kun Jeesus sen kuuli, lähti hän sieltä venheellä autioon paikkaan, yksinäisyyteen. Ja tämän kuultuaan kansa meni jalkaisin kaupungeista hänen jälkeensä.
\par 14 Ja astuessaan maihin Jeesus näki paljon kansaa, ja hänen kävi heitä sääliksi, ja hän paransi heidän sairaansa.
\par 15 Mutta kun ilta tuli, menivät hänen opetuslapsensa hänen tykönsä ja sanoivat: "Tämä paikka on autio, ja päivä on jo pitkälle kulunut; laske siis kansa luotasi, että he menisivät kyliin ostamaan itsellensä ruokaa".
\par 16 Mutta Jeesus sanoi heille: "Ei heidän tarvitse mennä pois; antakaa te heille syödä".
\par 17 He sanoivat hänelle: "Meillä ei ole täällä muuta kuin viisi leipää ja kaksi kalaa".
\par 18 Mutta hän sanoi: "Tuokaa ne tänne minulle".
\par 19 Ja hän käski kansan asettua ruohikkoon, otti ne viisi leipää ja kaksi kalaa, katsoi ylös taivaaseen ja siunasi, mursi ja antoi leivät opetuslapsillensa, ja opetuslapset antoivat kansalle.
\par 20 Ja kaikki söivät ja tulivat ravituiksi. Sitten he keräsivät jääneet palaset, kaksitoista täyttä vakallista.
\par 21 Ja niitä, jotka aterioivat, oli noin viisituhatta miestä, paitsi naisia ja lapsia.
\par 22 Ja kohta hän vaati opetuslapsiansa astumaan venheeseen ja kulkemaan edeltä toiselle rannalle, sillä aikaa kuin hän laski kansan luotansa.
\par 23 Ja laskettuaan kansan hän nousi vuorelle yksinäisyyteen, rukoilemaan. Ja kun ilta tuli, oli hän siellä yksinänsä.
\par 24 Mutta venhe oli jo monen vakomitan päässä maasta, aaltojen ahdistamana, sillä tuuli oli vastainen.
\par 25 Ja neljännellä yövartiolla Jeesus tuli heidän tykönsä kävellen järven päällä.
\par 26 Kun opetuslapset näkivät hänen kävelevän järven päällä, peljästyivät he ja sanoivat: "Se on aave", ja huusivat pelosta.
\par 27 Mutta Jeesus puhutteli heitä kohta ja sanoi: "Olkaa turvallisella mielellä, minä se olen; älkää peljätkö".
\par 28 Pietari vastasi hänelle ja sanoi: "Jos se olet sinä, Herra, niin käske minun tulla tykösi vettä myöten".
\par 29 Hän sanoi: "Tule". Ja Pietari astui ulos venheestä ja käveli vetten päällä mennäkseen Jeesuksen tykö.
\par 30 Mutta nähdessään, kuinka tuuli, hän peljästyi ja rupesi vajoamaan ja huusi sanoen: "Herra, auta minua".
\par 31 Niin Jeesus kohta ojensi kätensä, tarttui häneen ja sanoi hänelle: "Sinä vähäuskoinen, miksi epäilit?"
\par 32 Ja kun he olivat astuneet venheeseen, asettui tuuli.
\par 33 Niin venheessä-olijat kumarsivat häntä ja sanoivat: "Totisesti sinä olet Jumalan Poika".
\par 34 Ja kuljettuaan yli he tulivat maihin Gennesaretiin.
\par 35 Ja kun sen paikkakunnan miehet tunsivat hänet, lähettivät he sanan kaikkeen ympäristöön, ja hänen tykönsä tuotiin kaikki sairaat.
\par 36 Ja he pyysivät häneltä, että vain saisivat koskea hänen vaippansa tupsuun; ja kaikki, jotka koskivat, paranivat.

\chapter{15}

\par 1 Silloin tuli fariseuksia ja kirjanoppineita Jerusalemista Jeesuksen luo, ja he sanoivat:
\par 2 "Miksi sinun opetuslapsesi rikkovat vanhinten perinnäissääntöä? Sillä he eivät pese käsiään ruvetessaan aterialle."
\par 3 Mutta hän vastasi ja sanoi heille: "Miksi te itse rikotte Jumalan käskyn perinnäissääntönne tähden?
\par 4 Sillä Jumala on sanonut: 'Kunnioita isääsi ja äitiäsi', ja: 'Joka kiroaa isäänsä tai äitiänsä, sen pitää kuolemalla kuoleman'.
\par 5 Mutta te sanotte: Joka sanoo isälleen tai äidilleen: 'Se, minkä sinä olisit ollut minulta saapa, on annettu uhrilahjaksi', sen ei tarvitse kunnioittaa isäänsä eikä äitiänsä.
\par 6 Ja niin te olette tehneet Jumalan sanan tyhjäksi perinnäissääntönne tähden.
\par 7 Te ulkokullatut, oikein teistä Esaias ennusti, sanoen:
\par 8 'Tämä kansa kunnioittaa minua huulillaan, mutta heidän sydämensä on minusta kaukana,
\par 9 mutta turhaan he palvelevat minua opettaen oppeja, jotka ovat ihmiskäskyjä'."
\par 10 Ja hän kutsui kansan tykönsä ja sanoi heille: "Kuulkaa ja ymmärtäkää.
\par 11 Ei saastuta ihmistä se, mikä menee suusta sisään; vaan mikä suusta käy ulos, se saastuttaa ihmisen."
\par 12 Silloin opetuslapset tulivat ja sanoivat hänelle: "Tiedätkö, että fariseukset loukkaantuivat kuullessaan tuon puheen?"
\par 13 Mutta hän vastasi ja sanoi: "Jokainen istutus, jota minun taivaallinen Isäni ei ole istuttanut, on juurineen revittävä pois.
\par 14 Älkää heistä välittäkö: he ovat sokeita sokeain taluttajia; mutta jos sokea sokeaa taluttaa, niin he molemmat kuoppaan lankeavat."
\par 15 Niin Pietari vastasi ja sanoi hänelle: "Selitä meille tämä vertaus".
\par 16 Mutta Jeesus sanoi: "Vieläkö tekin olette ymmärtämättömiä?
\par 17 Ettekö käsitä, että kaikki, mikä käy suusta sisään, menee vatsaan ja ulostuu?
\par 18 Mutta mikä käy suusta ulos, se tulee sydämestä, ja se saastuttaa ihmisen.
\par 19 Sillä sydämestä lähtevät pahat ajatukset, murhat, aviorikokset, haureudet, varkaudet, väärät todistukset, jumalanpilkkaamiset.
\par 20 Nämä ihmisen saastuttavat; mutta pesemättömin käsin syöminen ei saastuta ihmistä."
\par 21 Ja Jeesus lähti sieltä ja vetäytyi Tyyron ja Siidonin tienoille.
\par 22 Ja katso, kanaanilainen vaimo tuli niiltä seuduilta ja huusi sanoen: "Herra, Daavidin poika, armahda minua. Riivaaja vaivaa kauheasti minun tytärtäni."
\par 23 Mutta hän ei vastannut hänelle sanaakaan. Niin hänen opetuslapsensa tulivat ja rukoilivat häntä sanoen: "Päästä hänet menemään, sillä hän huutaa meidän jälkeemme".
\par 24 Hän vastasi ja sanoi: "Minua ei ole lähetetty muitten kuin Israelin huoneen kadonneitten lammasten tykö".
\par 25 Mutta vaimo tuli ja kumarsi häntä ja sanoi: "Herra, auta minua".
\par 26 Mutta hän vastasi ja sanoi: "Ei ole soveliasta ottaa lasten leipää ja heittää penikoille".
\par 27 Mutta vaimo sanoi: "Niin, Herra; mutta syöväthän penikatkin niitä muruja, jotka heidän herrainsa pöydältä putoavat".
\par 28 Silloin Jeesus vastasi ja sanoi hänelle: "Oi vaimo, suuri on sinun uskosi, tapahtukoon sinulle, niinkuin tahdot". Ja hänen tyttärensä oli siitä hetkestä terve.
\par 29 Ja Jeesus lähti sieltä ja tuli Galilean järven rannalle; ja hän nousi vuorelle ja istui sinne.
\par 30 Ja hänen tykönsä tuli paljon kansaa, ja he toivat mukanaan rampoja, raajarikkoja, sokeita, mykkiä ja paljon muita, ja laskivat heidät hänen jalkojensa juureen; ja hän paransi heidät,
\par 31 niin että kansa ihmetteli nähdessään mykkäin puhuvan, raajarikkojen olevan terveitä, rampojen kävelevän ja sokeain näkevän; ja he ylistivät Israelin Jumalaa.
\par 32 Ja Jeesus kutsui opetuslapsensa tykönsä ja sanoi: "Minun käy sääliksi kansaa, sillä he ovat jo kolme päivää olleet minun tykönäni, eikä heillä ole mitään syötävää; enkä minä tahdo laskea heitä syömättä menemään, etteivät nääntyisi matkalla".
\par 33 Niin opetuslapset sanoivat hänelle: "Mistä me saamme täällä erämaassa niin paljon leipää, että voimme ravita noin suuren kansanjoukon?"
\par 34 Jeesus sanoi heille: "Montako leipää teillä on?" He sanoivat: "Seitsemän, ja muutamia kalasia".
\par 35 Niin hän käski kansan asettua maahan.
\par 36 Ja hän otti ne seitsemän leipää ja kalat, kiitti, mursi ja antoi opetuslapsillensa, ja opetuslapset antoivat kansalle.
\par 37 Ja kaikki söivät ja tulivat ravituiksi. Sitten he keräsivät jääneet palaset, seitsemän täyttä vasullista.
\par 38 Ja niitä, jotka aterioivat, oli neljätuhatta miestä, paitsi naisia ja lapsia.
\par 39 Ja laskettuaan kansan tyköänsä hän astui venheeseen ja meni Magadanin alueelle.

\chapter{16}

\par 1 Ja fariseukset ja saddukeukset tulivat hänen luoksensa ja kiusasivat häntä pyytäen häntä näyttämään heille merkin taivaasta.
\par 2 Mutta hän vastasi ja sanoi heille: "Kun ilta tulee, sanotte te: 'Tulee selkeä ilma, sillä taivas ruskottaa',
\par 3 ja aamulla: 'Tänään tulee rajuilma, sillä taivas ruskottaa ja on synkkä'. Taivaan muodon te osaatte arvioida, mutta aikain merkkejä ette osaa.
\par 4 Tämä paha ja avionrikkoja sukupolvi tavoittelee merkkiä, mutta sille ei anneta muuta merkkiä kuin Joonaan merkki." Ja hän jätti heidät ja meni pois.
\par 5 Kun opetuslapset saapuivat toiselle rannalle, olivat he unhottaneet ottaa leipää mukaansa.
\par 6 Ja Jeesus sanoi heille: "Varokaa ja kavahtakaa fariseusten ja saddukeusten hapatusta".
\par 7 Niin he puhuivat keskenään sanoen: "Emme ottaneet leipää mukaamme".
\par 8 Mutta kun Jeesus sen huomasi, sanoi hän: "Te vähäuskoiset, mitä puhutte keskenänne siitä, ettei teillä ole leipää mukananne?
\par 9 Ettekö vielä käsitä? Ja ettekö muista niitä viittä leipää viidelletuhannelle ja kuinka monta vakallista otitte talteen,
\par 10 ettekä niitä seitsemää leipää neljälletuhannelle ja kuinka monta vasullista otitte talteen?
\par 11 Kuinka te ette käsitä, etten minä puhunut teille leivästä? Vaan kavahtakaa fariseusten ja saddukeusten hapatusta."
\par 12 Silloin he ymmärsivät, ettei hän käskenyt kavahtamaan leivän hapatusta, vaan fariseusten ja saddukeusten oppia.
\par 13 Kun Jeesus tuli Filippuksen Kesarean tienoille, kysyi hän opetuslapsiltaan sanoen: "Kenen ihmiset sanovat Ihmisen Pojan olevan?"
\par 14 Niin he sanoivat: "Muutamat Johannes Kastajan, toiset Eliaan, toiset taas Jeremiaan tahi jonkun muun profeetoista".
\par 15 Hän sanoi heille: "Kenenkä te sanotte minun olevan?"
\par 16 Simon Pietari vastasi ja sanoi: "Sinä olet Kristus, elävän Jumalan Poika".
\par 17 Jeesus vastasi ja sanoi hänelle: "Autuas olet sinä, Simon, Joonaan poika, sillä ei liha eikä veri ole sitä sinulle ilmoittanut, vaan minun Isäni, joka on taivaissa.
\par 18 Ja minä sanon sinulle: sinä olet Pietari, ja tälle kalliolle minä rakennan seurakuntani, ja tuonelan portit eivät sitä voita.
\par 19 Minä olen antava sinulle taivasten valtakunnan avaimet, ja minkä sinä sidot maan päällä, se on oleva sidottu taivaissa, ja minkä sinä päästät maan päällä, se on oleva päästetty taivaissa."
\par 20 Silloin hän varoitti vakavasti opetuslapsiaan kenellekään sanomasta, että hän on Kristus.
\par 21 Siitä lähtien Jeesus alkoi ilmoittaa opetuslapsilleen, että hänen piti menemän Jerusalemiin ja kärsimän paljon vanhimmilta ja ylipapeilta ja kirjanoppineilta ja tuleman tapetuksi ja kolmantena päivänä nouseman ylös.
\par 22 Silloin Pietari otti hänet erilleen ja rupesi nuhtelemaan häntä sanoen: "Jumala varjelkoon, Herra, älköön se sinulle tapahtuko".
\par 23 Mutta hän kääntyi ja sanoi Pietarille: "Mene pois minun edestäni, saatana; sinä olet minulle pahennukseksi, sillä sinä et ajattele sitä, mikä on Jumalan, vaan sitä, mikä on ihmisten".
\par 24 Silloin Jeesus sanoi opetuslapsillensa: "Jos joku tahtoo minun perässäni kulkea, hän kieltäköön itsensä ja ottakoon ristinsä ja seuratkoon minua.
\par 25 Sillä joka tahtoo pelastaa elämänsä, hän kadottaa sen, mutta joka kadottaa elämänsä minun tähteni, hän löytää sen.
\par 26 Sillä mitä se hyödyttää ihmistä, vaikka hän voittaisi omaksensa koko maailman, mutta saisi sielullensa vahingon? Taikka mitä voi ihminen antaa sielunsa lunnaiksi?
\par 27 Sillä Ihmisen Poika on tuleva Isänsä kirkkaudessa enkeliensä kanssa, ja silloin hän maksaa kullekin hänen tekojensa mukaan.
\par 28 Totisesti minä sanon teille: tässä seisovien joukossa on muutamia, jotka eivät maista kuolemaa, ennenkuin näkevät Ihmisen Pojan tulevan kuninkuudessaan."

\chapter{17}

\par 1 Ja kuuden päivän kuluttua Jeesus otti mukaansa Pietarin sekä Jaakobin ja hänen veljensä Johanneksen ja vei heidät korkealle vuorelle, yksinäisyyteen.
\par 2 Ja hänen muotonsa muuttui heidän edessään, ja hänen kasvonsa loistivat niinkuin aurinko, ja hänen vaatteensa tulivat valkoisiksi niinkuin valo.
\par 3 Ja katso, heille ilmestyivät Mooses ja Elias, jotka puhuivat hänen kanssansa.
\par 4 Niin Pietari rupesi puhumaan ja sanoi Jeesukselle: "Herra, meidän on tässä hyvä olla; jos tahdot, niin minä teen tähän kolme majaa, sinulle yhden ja Moosekselle yhden ja Eliaalle yhden".
\par 5 Hänen vielä puhuessaan, katso, heidät varjosi valoisa pilvi; ja katso, pilvestä kuului ääni, joka sanoi: "Tämä on minun rakas Poikani, johon minä olen mielistynyt; kuulkaa häntä".
\par 6 Kun opetuslapset sen kuulivat, lankesivat he kasvoilleen ja peljästyivät kovin.
\par 7 Niin Jeesus tuli heidän tykönsä, koski heihin ja sanoi: "Nouskaa, älkääkä peljätkö".
\par 8 Ja kun he nostivat silmänsä, eivät he nähneet ketään muuta kuin Jeesuksen yksinään.
\par 9 Ja heidän kulkiessaan alas vuorelta Jeesus varoitti heitä sanoen: "Älkää kenellekään kertoko tätä näkyä, ennenkuin Ihmisen Poika on noussut kuolleista".
\par 10 Ja hänen opetuslapsensa kysyivät häneltä sanoen: "Miksi sitten kirjanoppineet sanovat, että Eliaan pitää tulla ensin?"
\par 11 Jeesus vastasi ja sanoi: "Elias tosin tulee ja asettaa kaikki kohdalleen.
\par 12 Mutta minä sanon teille, että Elias on jo tullut. Mutta he eivät tunteneet häntä, vaan tekivät hänelle, mitä tahtoivat. Samoin myös Ihmisen Poika saa kärsiä heiltä."
\par 13 Silloin opetuslapset ymmärsivät hänen puhuneen heille Johannes Kastajasta.
\par 14 Ja kun he saapuivat kansan luo, tuli hänen tykönsä muuan mies, polvistui hänen eteensä
\par 15 ja sanoi: "Herra, armahda minun poikaani, sillä hän on kuunvaihetautinen ja kärsii kovin; usein hän kaatuu, milloin tuleen, milloin veteen.
\par 16 Ja minä toin hänet sinun opetuslastesi tykö, mutta he eivät voineet häntä parantaa."
\par 17 Niin Jeesus vastasi ja sanoi: "Voi sinä epäuskoinen ja nurja sukupolvi, kuinka kauan minun täytyy olla teidän kanssanne? Kuinka kauan kärsiä teitä? Tuokaa hänet tänne minun tyköni."
\par 18 Ja Jeesus nuhteli riivaajaa, ja se lähti pojasta, ja poika oli siitä hetkestä terve.
\par 19 Sitten opetuslapset menivät Jeesuksen tykö eriksensä ja sanoivat: "Miksi emme me voineet ajaa sitä ulos?"
\par 20 Niin hän sanoi heille: "Teidän epäuskonne tähden; sillä totisesti minä sanon teille: jos teillä olisi uskoa sinapinsiemenenkään verran, niin te voisitte sanoa tälle vuorelle: 'Siirry täältä tuonne', ja se siirtyisi, eikä mikään olisi teille mahdotonta".
\par 21 []
\par 22 Ja kun he yhdessä vaelsivat Galileassa, sanoi Jeesus heille: "Ihmisen Poika annetaan ihmisten käsiin,
\par 23 ja he tappavat hänet, ja kolmantena päivänä hän nousee ylös". Ja he tulivat kovin murheellisiksi.
\par 24 Ja kun he saapuivat Kapernaumiin, tulivat temppeliveron kantajat Pietarin luo ja sanoivat: "Eikö teidän opettajanne maksa temppeliveroa?"
\par 25 Hän sanoi: "Maksaa". Ja kun hän tuli huoneeseen, kysyi Jeesus häneltä ensi sanaksi: "Mitä arvelet, Simon? Keiltä maan kuninkaat ottavat tullia tai veroa? Lapsiltaanko vai vierailta?"
\par 26 Ja kun hän vastasi: "Vierailta", sanoi Jeesus hänelle: "Lapset ovat siis vapaat.
\par 27 Mutta ettemme heitä loukkaisi, niin mene ja heitä onki järveen. Ota sitten ensiksi saamasi kala, ja kun avaat sen suun, löydät hopearahan. Ota se ja anna heille minun puolestani ja omasta puolestasi."

\chapter{18}

\par 1 Sillä hetkellä opetuslapset tulivat Jeesuksen tykö ja sanoivat: "Kuka on suurin taivasten valtakunnassa?"
\par 2 Niin hän kutsui tykönsä lapsen, asetti sen heidän keskellensä
\par 3 ja sanoi: "Totisesti minä sanon teille: ellette käänny ja tule lasten kaltaisiksi, ette pääse taivasten valtakuntaan.
\par 4 Sentähden, joka nöyrtyy tämän lapsen kaltaiseksi, se on suurin taivasten valtakunnassa.
\par 5 Ja joka ottaa tykönsä yhden tämänkaltaisen lapsen minun nimeeni, se ottaa tykönsä minut.
\par 6 Mutta joka viettelee yhden näistä pienistä, jotka uskovat minuun, sen olisi parempi, että myllynkivi ripustettaisiin hänen kaulaansa ja hänet upotettaisiin meren syvyyteen.
\par 7 Voi maailmaa viettelysten tähden! Viettelysten täytyy kyllä tulla; mutta voi sitä ihmistä, jonka kautta viettelys tulee!
\par 8 Mutta jos sinun kätesi tai jalkasi viettelee sinua, hakkaa se poikki ja heitä luotasi; parempi on sinulle, että käsipuolena tai jalkapuolena pääset elämään sisälle, kuin että sinut, molemmat kädet tai molemmat jalat tallella, heitetään iankaikkiseen tuleen.
\par 9 Ja jos sinun silmäsi viettelee sinua, repäise se pois ja heitä luotasi; parempi on sinun silmäpuolena mennä elämään sisälle, kuin että sinut, molemmat silmät tallella, heitetään helvetin tuleen.
\par 10 Katsokaa, ettette halveksu yhtäkään näistä pienistä; sillä minä sanon teille, että heidän enkelinsä taivaissa näkevät aina minun Isäni kasvot, joka on taivaissa.
\par 11 []
\par 12 Mitä arvelette? Jos jollakin ihmisellä on sata lammasta ja yksi niistä eksyy, eikö hän jätä niitä yhdeksääkymmentä yhdeksää vuorille ja mene etsimään eksynyttä?
\par 13 Ja jos hän sen löytää, totisesti minä sanon teille: hän iloitsee enemmän siitä kuin niistä yhdeksästäkymmenestä yhdeksästä, jotka eivät olleet eksyneet.
\par 14 Niin ei myöskään teidän taivaallisen Isänne tahto ole, että yksikään näistä pienistä joutuisi kadotukseen.
\par 15 Mutta jos veljesi rikkoo sinua vastaan, niin mene ja nuhtele häntä kahdenkesken; jos hän sinua kuulee, niin olet voittanut veljesi.
\par 16 Mutta jos hän ei sinua kuule, niin ota vielä yksi tai kaksi kanssasi, 'että jokainen asia vahvistettaisiin kahden tai kolmen todistajan sanalla'.
\par 17 Mutta jos hän ei kuule heitä, niin ilmoita seurakunnalle. Mutta jos hän ei seurakuntaakaan kuule, niin olkoon hän sinulle, niinkuin olisi pakana ja publikaani.
\par 18 Totisesti minä sanon teille: kaikki, minkä te sidotte maan päällä, on oleva sidottu taivaassa, ja kaikki, minkä te päästätte maan päällä, on oleva päästetty taivaassa.
\par 19 Vielä minä sanon teille: jos kaksi teistä maan päällä keskenään sopii mistä asiasta tahansa, että he sitä anovat, niin he saavat sen minun Isältäni, joka on taivaissa.
\par 20 Sillä missä kaksi tahi kolme on kokoontunut minun nimeeni, siinä minä olen heidän keskellänsä."
\par 21 Silloin Pietari meni hänen tykönsä ja sanoi hänelle: "Herra, kuinka monta kertaa minun on annettava anteeksi veljelleni, joka rikkoo minua vastaan? Ihanko seitsemän kertaa?"
\par 22 Jeesus vastasi hänelle: "Minä sanon sinulle: ei seitsemän kertaa, vaan seitsemänkymmentä kertaa seitsemän.
\par 23 Sentähden taivasten valtakunta on verrattava kuninkaaseen, joka vaati palvelijoiltansa tiliä.
\par 24 Ja kun hän rupesi tilintekoon, tuotiin hänen eteensä eräs, joka oli hänelle velkaa kymmenentuhatta leiviskää.
\par 25 Mutta kun tällä ei ollut, millä maksaa, niin hänen herransa määräsi myytäväksi hänet ja hänen vaimonsa ja lapsensa ja kaikki, mitä hänellä oli, ja velan maksettavaksi.
\par 26 Silloin palvelija lankesi maahan ja rukoili häntä sanoen: 'Ole pitkämielinen minua kohtaan, niin minä maksan sinulle kaikki'.
\par 27 Niin herran kävi sääliksi sitä palvelijaa, ja hän päästi hänet ja antoi hänelle velan anteeksi.
\par 28 Mutta mentyään ulos se palvelija tapasi erään kanssapalvelijoistaan, joka oli hänelle velkaa sata denaria; ja hän tarttui häneen, kuristi häntä kurkusta ja sanoi: 'Maksa, minkä olet velkaa'.
\par 29 Niin hänen kanssapalvelijansa lankesi maahan ja pyysi häntä sanoen: 'Ole pitkämielinen minua kohtaan, niin minä maksan sinulle'.
\par 30 Mutta hän ei tahtonut, vaan meni ja heitti hänet vankeuteen, kunnes hän maksaisi velkansa.
\par 31 Kun nyt hänen kanssapalvelijansa näkivät, mitä tapahtui, tulivat he kovin murheellisiksi ja menivät ja ilmoittivat herrallensa kaiken, mitä oli tapahtunut.
\par 32 Silloin hänen herransa kutsui hänet eteensä ja sanoi hänelle: 'Sinä paha palvelija! Minä annoin sinulle anteeksi kaiken sen velan, koska sitä minulta pyysit;
\par 33 eikö sinunkin olisi pitänyt armahtaa kanssapalvelijaasi, niinkuin minäkin sinua armahdin?'
\par 34 Ja hänen herransa vihastui ja antoi hänet vanginvartijan käsiin, kunnes hän maksaisi kaiken, minkä oli hänelle velkaa.
\par 35 Näin myös minun taivaallinen Isäni tekee teille, ellette anna kukin veljellenne sydämestänne anteeksi."

\chapter{19}

\par 1 Ja kun Jeesus oli lopettanut nämä puheet, lähti hän Galileasta ja kulki Jordanin tuota puolta Juudean alueelle.
\par 2 Ja suuri kansan paljous seurasi häntä, ja hän paransi heitä siellä.
\par 3 Ja fariseuksia tuli hänen luoksensa, ja he kiusasivat häntä sanoen: "Onko miehen lupa hyljätä vaimonsa mistä syystä tahansa?"
\par 4 Hän vastasi ja sanoi: "Ettekö ole lukeneet, että Luoja jo alussa 'loi heidät mieheksi ja naiseksi'
\par 5 ja sanoi: 'Sentähden mies luopukoon isästänsä ja äidistänsä ja liittyköön vaimoonsa, ja ne kaksi tulevat yhdeksi lihaksi'?
\par 6 Niin eivät he enää ole kaksi, vaan yksi liha. Minkä siis Jumala on yhdistänyt, sitä älköön ihminen erottako."
\par 7 He sanoivat hänelle: "Miksi sitten Mooses käski antaa erokirjan ja hyljätä hänet?"
\par 8 Hän sanoi heille: "Teidän sydämenne kovuuden tähden Mooses salli teidän hyljätä vaimonne, mutta alusta ei niin ollut.
\par 9 Mutta minä sanon teille: joka hylkää vaimonsa muun kuin huoruuden tähden ja nai toisen, se tekee huorin; ja joka nai hyljätyn, se tekee huorin."
\par 10 Opetuslapset sanoivat hänelle: "Jos miehen on näin laita vaimoonsa nähden, niin ei ole hyvä naida".
\par 11 Mutta hän sanoi heille: "Ei tämä sana kaikkiin sovellu, vaan ainoastaan niihin, joille se on suotu.
\par 12 Sillä on niitä, jotka syntymästään, äitinsä kohdusta saakka, ovat avioon kelpaamattomia, ja on niitä, jotka ihmiset ovat tehneet avioon kelpaamattomiksi, ja niitä, jotka taivasten valtakunnan tähden ovat tehneet itsensä avioon kelpaamattomiksi. Joka voi sen itseensä sovittaa, se sovittakoon."
\par 13 Silloin tuotiin hänen tykönsä lapsia, että hän panisi kätensä heidän päälleen ja rukoilisi; mutta opetuslapset nuhtelivat tuojia.
\par 14 Niin Jeesus sanoi: "Antakaa lasten olla, älkääkä estäkö heitä tulemasta minun tyköni, sillä senkaltaisten on taivasten valtakunta".
\par 15 Ja hän pani kätensä heidän päälleen ja lähti sieltä pois.
\par 16 Ja katso, eräs mies tuli ja sanoi hänelle: "Opettaja, mitä hyvää minun pitää tekemän, että minä saisin iankaikkisen elämän?"
\par 17 Niin hän sanoi hänelle: "Miksi kysyt minulta, mikä on hyvää? On ainoastaan yksi, joka on hyvä. Mutta jos tahdot päästä elämään sisälle, niin pidä käskyt."
\par 18 Hän sanoi hänelle: "Mitkä?" Jeesus sanoi: "Nämä: 'Älä tapa', 'Älä tee huorin', 'Älä varasta', 'Älä sano väärää todistusta',
\par 19 'Kunnioita isääsi ja äitiäsi', ja: 'Rakasta lähimmäistäsi niinkuin itseäsi'".
\par 20 Nuorukainen sanoi hänelle: "Kaikkia niitä minä olen noudattanut; mitä minulta vielä puuttuu?"
\par 21 Jeesus sanoi hänelle: "Jos tahdot olla täydellinen, niin mene, myy, mitä sinulla on, ja anna köyhille, niin sinulla on oleva aarre taivaissa; ja tule ja seuraa minua".
\par 22 Mutta kun nuorukainen kuuli tämän sanan, meni hän pois murheellisena, sillä hänellä oli paljon omaisuutta.
\par 23 Silloin Jeesus sanoi opetuslapsillensa: "Totisesti minä sanon teille: rikkaan on vaikea päästä taivasten valtakuntaan.
\par 24 Ja vielä minä sanon teille: helpompi on kamelin käydä neulansilmän läpi kuin rikkaan päästä Jumalan valtakuntaan."
\par 25 Kun opetuslapset sen kuulivat, hämmästyivät he kovin ja sanoivat: "Kuka sitten voi pelastua?"
\par 26 Niin Jeesus katsoi heihin ja sanoi heille: "Ihmisille se on mahdotonta, mutta Jumalalle on kaikki mahdollista".
\par 27 Silloin Pietari vastasi ja sanoi hänelle: "Katso, me olemme luopuneet kaikesta ja seuranneet sinua; mitä me siitä saamme?"
\par 28 Niin Jeesus sanoi heille: "Totisesti minä sanon teille: siinä uudestisyntymisessä, jolloin Ihmisen Poika istuu kirkkautensa valtaistuimella, saatte tekin, jotka olette minua seuranneet, istua kahdellatoista valtaistuimella ja tuomita Israelin kahtatoista sukukuntaa.
\par 29 Ja jokainen, joka on luopunut taloista tai veljistä tai sisarista tai isästä tai äidistä tai lapsista tai pelloista minun nimeni tähden, on saava monin verroin takaisin ja perivä iankaikkisen elämän.
\par 30 Mutta monet ensimmäiset tulevat viimeisiksi, ja monet viimeiset ensimmäisiksi."

\chapter{20}

\par 1 "Sillä taivasten valtakunta on perheenisännän kaltainen, joka varhain aamulla lähti ulos palkkaamaan työmiehiä viinitarhaansa.
\par 2 Ja kun hän oli sopinut työmiesten kanssa denarista päivältä, lähetti hän heidät viinitarhaansa.
\par 3 Ja hän lähti ulos kolmannen hetken vaiheilla ja näki toisia seisomassa torilla joutilaina;
\par 4 ja hän sanoi heille: 'Menkää tekin minun viinitarhaani, ja mikä kohtuus on, sen minä annan teille'.
\par 5 Niin he menivät. Taas hän lähti ulos kuudennen ja yhdeksännen hetken vaiheilla ja teki samoin.
\par 6 Ja kun hän lähti ulos yhdennentoista hetken vaiheilla, tapasi hän vielä toisia siellä seisomassa; ja hän sanoi heille: 'Miksi seisotte täällä kaiken päivää joutilaina?'
\par 7 He sanoivat hänelle: 'Kun ei kukaan ole meitä palkannut'. Hän sanoi heille: 'Menkää tekin minun viinitarhaani'.
\par 8 Mutta kun ilta tuli, sanoi viinitarhan herra tilansa hoitajalle: 'Kutsu työmiehet ja maksa heille palkka, viimeisistä alkaen ensimmäisiin asti'.
\par 9 Kun nyt tulivat ne, jotka olivat saapuneet yhdennentoista hetken vaiheilla, saivat he kukin denarin.
\par 10 Ja kun ensimmäiset tulivat, luulivat he saavansa enemmän; mutta hekin saivat kukin denarin.
\par 11 Kun he sen saivat, napisivat he perheen isäntää vastaan
\par 12 ja sanoivat: 'Nämä viimeiset ovat tehneet työtä vain yhden hetken, ja sinä teit heidät meidän vertaisiksemme, jotka olemme kantaneet päivän kuorman ja helteen'.
\par 13 Niin hän vastasi eräälle heistä ja sanoi: 'Ystäväni, en minä tee sinulle vääryyttä; etkö sopinut minun kanssani denarista?
\par 14 Ota omasi ja mene. Mutta minä tahdon tälle viimeiselle antaa saman verran kuin sinullekin.
\par 15 Enkö saa tehdä omallani, mitä tahdon? Vai onko silmäsi nurja sentähden, että minä olen hyvä?'
\par 16 Näin viimeiset tulevat ensimmäisiksi ja ensimmäiset viimeisiksi."
\par 17 Ja kun Jeesus lähti kulkemaan ylös Jerusalemiin, otti hän ne kaksitoista erilleen ja sanoi matkalla heille:
\par 18 "Katso, me menemme ylös Jerusalemiin, ja Ihmisen Poika annetaan ylipappien ja kirjanoppineitten käsiin, ja he tuomitsevat hänet kuolemaan
\par 19 ja antavat hänet pakanoille pilkattavaksi ja ruoskittavaksi ja ristiinnaulittavaksi, ja kolmantena päivänä hän on nouseva ylös".
\par 20 Silloin Sebedeuksen poikain äiti tuli poikineen hänen tykönsä ja kumarsi häntä, aikoen anoa häneltä jotakin.
\par 21 Niin hän sanoi vaimolle: "Mitä tahdot?" Tämä sanoi hänelle: "Sano, että nämä minun kaksi poikaani saavat istua, toinen sinun oikealla ja toinen vasemmalla puolellasi, sinun valtakunnassasi".
\par 22 Mutta Jeesus vastasi ja sanoi: "Te ette tiedä, mitä anotte. Voitteko juoda sen maljan, jonka minä olen juova?" He sanoivat hänelle: "Voimme".
\par 23 Hän sanoi heille: "Minun maljani te tosin juotte, mutta minun oikealla ja vasemmalla puolellani istuminen ei ole minun annettavissani, vaan se annetaan niille, joille minun Isäni on sen valmistanut".
\par 24 Kun ne kymmenen sen kuulivat, närkästyivät he näihin kahteen veljekseen.
\par 25 Mutta Jeesus kutsui heidät tykönsä ja sanoi: "Te tiedätte, että kansojen ruhtinaat herroina niitä hallitsevat, ja että mahtavat käyttävät valtaansa niitä kohtaan.
\par 26 Näin älköön olko teillä keskenänne, vaan joka teidän keskuudessanne tahtoo suureksi tulla, se olkoon teidän palvelijanne,
\par 27 ja joka teidän keskuudessanne tahtoo olla ensimmäinen, se olkoon teidän orjanne;
\par 28 niinkuin ei Ihmisen Poikakaan tullut palveltavaksi, vaan palvelemaan ja antamaan henkensä lunnaiksi monen edestä."
\par 29 Ja heidän lähtiessään Jerikosta seurasi häntä suuri kansan paljous.
\par 30 Ja katso, kaksi sokeaa istui tien vieressä; ja kun he kuulivat, että Jeesus kulki ohitse, huusivat he sanoen: "Herra, Daavidin poika, armahda meitä".
\par 31 Niin kansa nuhteli heitä saadakseen heidät vaikenemaan; mutta he huusivat sitä enemmän sanoen: "Herra, Daavidin poika, armahda meitä".
\par 32 Silloin Jeesus seisahtui ja kutsui heidät tykönsä ja sanoi: "Mitä tahdotte, että minä teille tekisin?"
\par 33 He sanoivat hänelle: "Herra, että meidän silmämme aukenisivat".
\par 34 Niin Jeesuksen tuli heitä sääli, ja hän kosketti heidän silmiänsä, ja kohta he saivat näkönsä ja seurasivat häntä.

\chapter{21}

\par 1 Ja kun he lähestyivät Jerusalemia ja saapuivat Beetfageen, Öljymäelle, silloin Jeesus lähetti kaksi opetuslasta
\par 2 ja sanoi heille: "Menkää kylään, joka on edessänne, niin te kohta löydätte aasintamman sidottuna ja varsan sen kanssa; päästäkää ne ja tuokaa minulle.
\par 3 Ja jos joku teille jotakin sanoo, niin vastatkaa: 'Herra tarvitsee niitä'; ja kohta hän lähettää ne."
\par 4 Mutta tämä tapahtui, että kävisi toteen, mikä on puhuttu profeetan kautta, joka sanoo:
\par 5 "Sanokaa tytär Siionille: 'Katso, sinun kuninkaasi tulee sinulle hiljaisena ja ratsastaen aasilla, ikeenalaisen aasin varsalla'."
\par 6 Niin opetuslapset menivät ja tekivät, niinkuin Jeesus oli heitä käskenyt,
\par 7 ja toivat aasintamman varsoineen ja panivat niiden selkään vaatteensa, ja hän istuutui niiden päälle.
\par 8 Ja suurin osa kansasta levitti vaatteensa tielle, ja toiset karsivat oksia puista ja hajottivat tielle.
\par 9 Ja kansanjoukot, jotka kulkivat hänen edellään ja jotka seurasivat, huusivat sanoen: "Hoosianna Daavidin pojalle! Siunattu olkoon hän, joka tulee Herran nimeen. Hoosianna korkeuksissa!"
\par 10 Ja kun hän tuli Jerusalemiin, joutui koko kaupunki liikkeelle ja sanoi: "Kuka tämä on?"
\par 11 Niin kansa sanoi: "Tämä on se profeetta, Jeesus, Galilean Nasaretista".
\par 12 Ja Jeesus meni pyhäkköön; ja hän ajoi ulos kaikki, jotka myivät ja ostivat pyhäkössä, ja kaatoi kumoon rahanvaihtajain pöydät ja kyyhkysten myyjäin istuimet.
\par 13 Ja hän sanoi heille: "Kirjoitettu on: 'Minun huoneeni pitää kutsuttaman rukoushuoneeksi', mutta te teette siitä ryövärien luolan."
\par 14 Ja hänen tykönsä pyhäkössä tuli sokeita ja rampoja, ja hän paransi heidät.
\par 15 Mutta kun ylipapit ja kirjanoppineet näkivät ne ihmeet, joita hän teki, ja lapset, jotka huusivat pyhäkössä ja sanoivat: "Hoosianna Daavidin pojalle", niin he närkästyivät
\par 16 ja sanoivat hänelle: "Kuuletko, mitä nämä sanovat?" Niin Jeesus sanoi heille: "Kuulen; ettekö ole koskaan lukeneet: 'Lasten ja imeväisten suusta sinä olet valmistanut itsellesi kiitoksen'?"
\par 17 Ja hän jätti heidät ja meni ulos kaupungista Betaniaan ja oli siellä yötä.
\par 18 Kun hän varhain aamulla palasi kaupunkiin, oli hänen nälkä.
\par 19 Ja nähdessään tien vieressä viikunapuun hän meni sen luo, mutta ei löytänyt siitä muuta kuin pelkkiä lehtiä; ja hän sanoi sille: "Älköön sinusta ikinä enää hedelmää kasvako". Ja kohta viikunapuu kuivettui.
\par 20 Kun opetuslapset tämän näkivät, ihmettelivät he ja sanoivat: "Kuinka viikunapuu niin äkisti kuivettui?"
\par 21 Jeesus vastasi ja sanoi heille: "Totisesti minä sanon teille: jos teillä olisi uskoa ettekä epäilisi, niin ette ainoastaan voisi tehdä sitä, mikä viikunapuussa tapahtui, vaan vieläpä, jos sanoisitte tälle vuorelle: 'Kohoa ja heittäydy mereen', niin se tapahtuisi.
\par 22 Ja kaiken, mitä te anotte rukouksessa, uskoen, te saatte."
\par 23 Ja kun hän oli mennyt pyhäkköön, tulivat hänen opettaessaan ylipapit ja kansan vanhimmat hänen luoksensa ja sanoivat: "Millä vallalla sinä näitä teet? Ja kuka sinulle on antanut tämän vallan?"
\par 24 Jeesus vastasi ja sanoi heille: "Minä myös teen teille yhden kysymyksen; jos te minulle siihen vastaatte, niin minäkin sanon teille, millä vallalla minä näitä teen.
\par 25 Mistä Johanneksen kaste oli? Taivaastako vai ihmisistä?" Niin he neuvottelivat keskenänsä sanoen: "Jos sanomme: 'Taivaasta', niin hän sanoo meille: 'Miksi ette siis uskoneet häntä?'
\par 26 Mutta jos sanomme: 'Ihmisistä', niin meidän täytyy peljätä kansaa, sillä kaikki pitävät Johannesta profeettana."
\par 27 Ja he vastasivat Jeesukselle ja sanoivat: "Emme tiedä". Niin hänkin sanoi heille: "Niinpä en minäkään sano teille, millä vallalla minä näitä teen.
\par 28 Mutta miten teistä on? Miehellä oli kaksi poikaa; ja hän meni ensimmäisen luo ja sanoi: 'Poikani, mene tänään tekemään työtä minun viinitarhaani'.
\par 29 Tämä vastasi ja sanoi: 'En tahdo'; mutta jäljestäpäin hän katui ja meni.
\par 30 Niin hän meni toisen luo ja sanoi samoin. Tämä taas vastasi ja sanoi: 'Minä menen, herra', mutta ei mennytkään.
\par 31 Kumpi näistä kahdesta teki isänsä tahdon?" He sanoivat: "Ensimmäinen". Jeesus sanoi heille: "Totisesti minä sanon teille: publikaanit ja portot menevät ennen teitä Jumalan valtakuntaan.
\par 32 Sillä Johannes tuli teidän tykönne vanhurskauden tietä, ja te ette uskoneet häntä, mutta publikaanit ja portot uskoivat häntä; ja vaikka te sen näitte, ette jäljestäpäinkään katuneet, niin että olisitte häntä uskoneet.
\par 33 Kuulkaa toinen vertaus: Oli perheenisäntä, joka istutti viinitarhan ja teki aidan sen ympärille ja kaivoi siihen viinikuurnan ja rakensi tornin; ja hän vuokrasi sen viinitarhureille ja matkusti muille maille.
\par 34 Ja kun hedelmäin aika lähestyi, lähetti hän palvelijoitansa viinitarhurien luokse perimään hänelle tulevat hedelmät.
\par 35 Mutta viinitarhurit ottivat kiinni hänen palvelijansa; minkä he pieksivät, minkä tappoivat, minkä kivittivät.
\par 36 Vielä hän lähetti toisia palvelijoita, useampia kuin ensimmäiset; ja näille he tekivät samoin.
\par 37 Mutta viimein hän lähetti heidän luokseen poikansa sanoen: 'Minun poikaani he kavahtavat'.
\par 38 Mutta kun viinitarhurit näkivät pojan, sanoivat he keskenänsä: 'Tämä on perillinen; tulkaa, tappakaamme hänet, niin me saamme hänen perintönsä'.
\par 39 Ja he ottivat hänet kiinni ja heittivät ulos viinitarhasta ja tappoivat.
\par 40 Kun viinitarhan herra tulee, mitä hän tekee noille viinitarhureille?"
\par 41 He sanoivat hänelle: "Nuo pahat hän pahoin tuhoaa ja vuokraa viinitarhan toisille viinitarhureille, jotka antavat hänelle hedelmät ajallansa".
\par 42 Jeesus sanoi heille: "Ettekö ole koskaan lukeneet kirjoituksista: 'Se kivi, jonka rakentajat hylkäsivät, on tullut kulmakiveksi; Herralta tämä on tullut ja on ihmeellinen meidän silmissämme'?
\par 43 Sentähden minä sanon teille: Jumalan valtakunta otetaan teiltä pois ja annetaan kansalle, joka tekee sen hedelmiä.
\par 44 Ja joka tähän kiveen kaatuu, se ruhjoutuu, mutta jonka päälle se kaatuu, sen se murskaa."
\par 45 Kun ylipapit ja fariseukset kuulivat nämä hänen vertauksensa, ymmärsivät he, että hän puhui heistä.
\par 46 Ja he olisivat tahtoneet ottaa hänet kiinni, mutta pelkäsivät kansaa, koska se piti häntä profeettana.

\chapter{22}

\par 1 Ja Jeesus rupesi taas puhumaan heille vertauksilla ja sanoi:
\par 2 "Taivasten valtakunta on verrattava kuninkaaseen, joka laittoi häät pojallensa.
\par 3 Ja hän lähetti palvelijansa kutsumaan häihin kutsuvieraita, mutta nämä eivät tahtoneet tulla.
\par 4 Vielä hän lähetti toisia palvelijoita lausuen: 'Sanokaa kutsutuille: Katso, minä olen valmistanut ateriani, minun härkäni ja syöttilääni ovat teurastetut, ja kaikki on valmiina; tulkaa häihin'.
\par 5 Mutta he eivät siitä välittäneet, vaan menivät pois, mikä pellolleen, mikä kaupoilleen;
\par 6 ja toiset ottivat kiinni hänen palvelijansa, pitelivät pahoin ja tappoivat.
\par 7 Mutta kuningas vihastui ja lähetti sotajoukkonsa ja tuhosi nuo murhamiehet ja poltti heidän kaupunkinsa.
\par 8 Sitten hän sanoi palvelijoillensa: 'Häät ovat valmistetut, mutta kutsutut eivät olleet arvollisia.
\par 9 Menkää siis teiden risteyksiin ja kutsukaa häihin, keitä tapaatte.'
\par 10 Ja palvelijat menivät ulos teille ja kokosivat kaikki, keitä vain tapasivat, sekä pahat että hyvät, ja häähuone tuli täyteen pöytävieraita.
\par 11 Mutta kun kuningas meni katsomaan pöytävieraita, näki hän siellä miehen, joka ei ollut puettu häävaatteisiin.
\par 12 Ja hän sanoi hänelle: 'Ystävä, kuinka sinä olet tullut tänne sisälle, vaikka sinulla ei ole häävaatteita?' Mutta hän jäi sanattomaksi.
\par 13 Silloin kuningas sanoi palvelijoille: 'Sitokaa hänen jalkansa ja kätensä ja heittäkää hänet ulos pimeyteen'. Siellä on oleva itku ja hammasten kiristys.
\par 14 Sillä monet ovat kutsutut, mutta harvat valitut."
\par 15 Silloin fariseukset menivät ja neuvottelivat, kuinka saisivat hänet sanoissa solmituksi.
\par 16 Ja he lähettivät hänen luoksensa opetuslapsensa herodilaisten kanssa sanomaan: "Opettaja, me tiedämme, että sinä olet totinen ja opetat Jumalan tietä totuudessa, kenestäkään välittämättä, sillä sinä et katso henkilöön.
\par 17 Sano siis meille: miten arvelet? Onko luvallista antaa keisarille veroa vai ei?"
\par 18 Mutta Jeesus ymmärsi heidän pahuutensa ja sanoi: "Miksi kiusaatte minua, te ulkokullatut?
\par 19 Näyttäkää minulle veroraha." Niin he toivat hänelle denarin.
\par 20 Hän sanoi heille: "Kenen kuva ja päällekirjoitus tämä on?"
\par 21 He vastasivat: "Keisarin". Silloin hän sanoi heille: "Antakaa siis keisarille, mikä keisarin on, ja Jumalalle, mikä Jumalan on".
\par 22 Kun he sen kuulivat, ihmettelivät he ja jättivät hänet ja menivät pois.
\par 23 Sinä päivänä tuli hänen luoksensa saddukeuksia, jotka sanovat, ettei ylösnousemusta ole, ja he kysyivät häneltä
\par 24 sanoen: "Opettaja, Mooses on sanonut: 'Jos joku kuolee lapsetonna, niin hänen veljensä naikoon hänen vaimonsa ja herättäköön siemenen veljelleen'.
\par 25 Keskuudessamme oli seitsemän veljestä. Ensimmäinen otti vaimon ja kuoli; ja koska hänellä ei ollut jälkeläistä, jätti hän vaimonsa veljelleen.
\par 26 Niin myös toinen ja kolmas, ja samoin kaikki seitsemän.
\par 27 Viimeiseksi kaikista kuoli vaimo.
\par 28 Kenen vaimo noista seitsemästä hän siis ylösnousemuksessa on oleva? Sillä kaikkien vaimona hän on ollut."
\par 29 Jeesus vastasi ja sanoi heille: "Te eksytte, koska te ette tunne kirjoituksia ettekä Jumalan voimaa.
\par 30 Sillä ylösnousemuksessa ei naida eikä mennä miehelle; vaan he ovat niinkuin enkelit taivaassa.
\par 31 Mutta mitä kuolleitten ylösnousemukseen tulee, ettekö ole lukeneet, mitä Jumala on puhunut teille, sanoen:
\par 32 'Minä olen Aabrahamin Jumala ja Iisakin Jumala ja Jaakobin Jumala'? Ei hän ole kuolleitten Jumala, vaan elävien."
\par 33 Ja sen kuullessaan kansa hämmästyi hänen oppiansa.
\par 34 Mutta kun fariseukset kuulivat, että hän oli tukkinut saddukeuksilta suun, kokoontuivat he yhteen;
\par 35 ja eräs heistä, joka oli lainoppinut, kysyi häneltä kiusaten:
\par 36 "Opettaja, mikä on suurin käsky laissa?"
\par 37 Niin Jeesus sanoi hänelle: "'Rakasta Herraa, sinun Jumalaasi, kaikesta sydämestäsi ja kaikesta sielustasi ja kaikesta mielestäsi'.
\par 38 Tämä on suurin ja ensimmäinen käsky.
\par 39 Toinen, tämän vertainen, on: 'Rakasta lähimmäistäsi niinkuin itseäsi'.
\par 40 Näissä kahdessa käskyssä riippuu kaikki laki ja profeetat."
\par 41 Ja fariseusten ollessa koolla Jeesus kysyi heiltä
\par 42 sanoen: "Mitä arvelette Kristuksesta? Kenen poika hän on?" He sanoivat hänelle: "Daavidin".
\par 43 Hän sanoi heille: "Kuinka sitten Daavid Hengessä kutsuu häntä Herraksi, sanoen:
\par 44 'Herra sanoi minun Herralleni: Istu minun oikealle puolelleni, kunnes minä panen sinun vihollisesi sinun jalkojesi alle'.
\par 45 Jos siis Daavid kutsuu häntä Herraksi, kuinka hän on hänen poikansa?"
\par 46 Ja kukaan ei voinut vastata hänelle sanaakaan; eikä siitä päivästä lähtien yksikään enää rohjennut kysyä häneltä mitään.

\chapter{23}

\par 1 Silloin Jeesus puhui kansalle ja opetuslapsilleen
\par 2 sanoen: "Mooseksen istuimella istuvat kirjanoppineet ja fariseukset.
\par 3 Sentähden, kaikki, mitä he sanovat teille, se tehkää ja pitäkää; mutta heidän tekojensa mukaan älkää tehkö, sillä he sanovat, mutta eivät tee.
\par 4 He sitovat kokoon raskaita ja vaikeasti kannettavia taakkoja ja panevat ne ihmisten hartioille, mutta itse he eivät tahdo niitä sormellaankaan liikuttaa.
\par 5 Ja kaikki tekonsa he tekevät sitä varten, että ihmiset heitä katselisivat. He tekevät raamatunlausekotelonsa leveiksi ja vaippansa tupsut suuriksi
\par 6 ja rakastavat ensimmäistä sijaa pidoissa ja etumaisia istuimia synagoogissa,
\par 7 ja tahtovat mielellään, että heitä tervehditään toreilla, ja että ihmiset kutsuvat heitä nimellä 'rabbi'.
\par 8 Mutta te älkää antako kutsua itseänne rabbiksi, sillä yksi on teidän opettajanne, ja te olette kaikki veljiä.
\par 9 Ja isäksenne älkää kutsuko ketään maan päällä, sillä yksi on teidän Isänne, hän, joka on taivaissa.
\par 10 Älkääkä antako kutsua itseänne mestareiksi, sillä yksi on teidän mestarinne, Kristus.
\par 11 Vaan joka teistä on suurin, se olkoon teidän palvelijanne.
\par 12 Mutta joka itsensä ylentää, se alennetaan; ja joka itsensä alentaa, se ylennetään.
\par 13 Mutta voi teitä, kirjanoppineet ja fariseukset, te ulkokullatut, kun suljette taivasten valtakunnan ihmisiltä! Sillä itse te ette mene sisälle, ettekä salli meneväisten sisälle mennä.
\par 14 []
\par 15 Voi teitä, kirjanoppineet ja fariseukset, te ulkokullatut, kun te kierrätte meret ja mantereet tehdäksenne yhden käännynnäisen; ja kun joku on siksi tullut, niin teette hänestä helvetin lapsen, kahta vertaa pahemman, kuin te itse olette!
\par 16 Voi teitä, te sokeat taluttajat, jotka sanotte: 'Jos joku vannoo temppelin kautta, niin se ei ole mitään; mutta jos joku vannoo temppelin kullan kautta, niin hän on valaansa sidottu'!
\par 17 Te tyhmät ja sokeat! Kumpi on suurempi, kultako vai temppeli, joka kullan pyhittää?
\par 18 Ja: 'Jos joku vannoo alttarin kautta, niin se ei ole mitään; mutta jos joku vannoo sen päällä olevan uhrilahjan kautta, niin hän on valaansa sidottu'.
\par 19 Te sokeat! Kumpi on suurempi, uhrilahjako vai alttari, joka uhrilahjan pyhittää?
\par 20 Sentähden, joka vannoo alttarin kautta, vannoo sen kautta ja kaiken kautta, mitä sen päällä on.
\par 21 Ja joka vannoo temppelin kautta, vannoo sen kautta ja hänen kauttansa, joka siinä asuu.
\par 22 Ja joka vannoo taivaan kautta, vannoo Jumalan valtaistuimen kautta ja hänen kauttansa, joka sillä istuu.
\par 23 Voi teitä, kirjanoppineet ja fariseukset, te ulkokullatut, kun te annatte kymmenykset mintuista ja tilleistä ja kuminoista, mutta jätätte sikseen sen, mikä laissa on tärkeintä: oikeuden ja laupeuden ja uskollisuuden! Näitä tulisi noudattaa, eikä noitakaan sikseen jättää.
\par 24 Te sokeat taluttajat, jotka siivilöitte hyttysen, mutta nielette kamelin!
\par 25 Voi teitä, kirjanoppineet ja fariseukset, te ulkokullatut, kun te puhdistatte maljan ja vadin ulkopuolen, mutta sisältä ne ovat täynnä ryöstöä ja hillittömyyttä!
\par 26 Sinä sokea fariseus, puhdista ensin maljan sisus, että sen ulkopuolikin tulisi puhtaaksi!
\par 27 Voi teitä, kirjanoppineet ja fariseukset, te ulkokullatut, kun te olette valkeiksi kalkittujen hautojen kaltaisia: ulkoa ne kyllä näyttävät kauniilta, mutta ovat sisältä täynnä kuolleitten luita ja kaikkea saastaa!
\par 28 Samoin tekin ulkoa kyllä näytätte ihmisten silmissä hurskailta, mutta sisältä te olette täynnä ulkokultaisuutta ja laittomuutta.
\par 29 Voi teitä, kirjanoppineet ja fariseukset, te ulkokullatut, kun te rakennatte profeettain hautoja ja kaunistatte vanhurskasten hautakammioita,
\par 30 ja sanotte: 'Jos me olisimme eläneet isäimme päivinä, emme olisi olleet osallisia heidän kanssaan profeettain vereen'!
\par 31 Niin te siis todistatte itsestänne, että olette niiden lapsia, jotka tappoivat profeetat.
\par 32 Täyttäkää siis te isäinne mitta.
\par 33 Te käärmeet, te kyykäärmeitten sikiöt, kuinka te pääsisitte helvetin tuomiota pakoon?
\par 34 Sentähden, katso, minä lähetän teidän tykönne profeettoja ja viisaita ja kirjanoppineita. Muutamat heistä te tapatte ja ristiinnaulitsette, ja toisia heistä te ruoskitte synagoogissanne ja vainoatte kaupungista kaupunkiin;
\par 35 että teidän päällenne tulisi kaikki se vanhurskas veri, joka maan päällä on vuodatettu vanhurskaan Aabelin verestä Sakariaan, Barakiaan pojan, vereen asti, jonka te tapoitte temppelin ja alttarin välillä.
\par 36 Totisesti minä sanon teille: tämä kaikki on tuleva tämän sukupolven päälle.
\par 37 Jerusalem, Jerusalem, sinä, joka tapat profeetat ja kivität ne, jotka ovat sinun tykösi lähetetyt, kuinka usein minä olenkaan tahtonut koota sinun lapsesi, niinkuin kana kokoaa poikansa siipiensä alle! Mutta te ette ole tahtoneet.
\par 38 Katso, 'teidän huoneenne on jäävä hyljätyksi'.
\par 39 Sillä minä sanon teille: tästedes te ette näe minua, ennenkuin sanotte: 'Siunattu olkoon hän, joka tulee Herran nimeen'."

\chapter{24}

\par 1 Ja Jeesus lähti ulos pyhäköstä ja meni pois; ja hänen opetuslapsensa tulivat hänen tykönsä näyttämään hänelle pyhäkön rakennuksia.
\par 2 Niin hän vastasi ja sanoi heille: "Ettekö näe näitä kaikkia? Totisesti minä sanon teille: tähän ei ole jäävä kiveä kiven päälle, maahan jaottamatta."
\par 3 Ja kun hän istui Öljymäellä, tulivat opetuslapset erikseen hänen tykönsä ja sanoivat: "Sano meille: milloin se tapahtuu, ja mikä on sinun tulemuksesi ja maailman lopun merkki?"
\par 4 Silloin Jeesus vastasi ja sanoi heille: "Katsokaa, ettei kukaan teitä eksytä.
\par 5 Sillä monta tulee minun nimessäni sanoen: 'Minä olen Kristus', ja he eksyttävät monta.
\par 6 Ja te saatte kuulla sotien melskettä ja sanomia sodista; katsokaa, ettette peljästy. Sillä näin täytyy tapahtua, mutta tämä ei ole vielä loppu.
\par 7 Sillä kansa nousee kansaa vastaan ja valtakunta valtakuntaa vastaan, ja nälänhätää ja maanjäristyksiä tulee monin paikoin.
\par 8 Mutta kaikki tämä on synnytystuskien alkua.
\par 9 Silloin teidät annetaan vaivaan, ja teitä tapetaan, ja te joudutte kaikkien kansojen vihattaviksi minun nimeni tähden.
\par 10 Ja silloin monet lankeavat pois, ja he antavat toisensa alttiiksi ja vihaavat toinen toistaan.
\par 11 Ja monta väärää profeettaa nousee, ja he eksyttävät monta.
\par 12 Ja sentähden, että laittomuus pääsee valtaan, kylmenee useimpien rakkaus.
\par 13 Mutta joka vahvana pysyy loppuun asti, se pelastuu.
\par 14 Ja tämä valtakunnan evankeliumi pitää saarnattaman kaikessa maailmassa, todistukseksi kaikille kansoille; ja sitten tulee loppu.
\par 15 Kun te siis näette hävityksen kauhistuksen, josta on puhuttu profeetta Danielin kautta, seisovan pyhässä paikassa - joka tämän lukee, se tarkatkoon -
\par 16 silloin ne, jotka Juudeassa ovat, paetkoot vuorille;
\par 17 joka on katolla, älköön astuko alas noutamaan, mitä hänen huoneessansa on,
\par 18 ja joka on pellolla, älköön palatko takaisin noutamaan vaippaansa.
\par 19 Voi raskaita ja imettäväisiä niinä päivinä!
\par 20 Mutta rukoilkaa, ettei teidän pakonne tapahtuisi talvella eikä sapattina.
\par 21 Sillä silloin on oleva suuri ahdistus, jonka kaltaista ei ole ollut maailman alusta hamaan tähän asti eikä milloinkaan tule.
\par 22 Ja ellei niitä päiviä olisi lyhennetty, ei mikään liha pelastuisi; mutta valittujen tähden ne päivät lyhennetään.
\par 23 Jos silloin joku sanoo teille: 'Katso, täällä on Kristus', tahi: 'Tuolla', niin älkää uskoko.
\par 24 Sillä vääriä kristuksia ja vääriä profeettoja nousee, ja he tekevät suuria tunnustekoja ja ihmeitä, niin että eksyttävät, jos mahdollista, valitutkin.
\par 25 Katso, minä olen sen teille edeltä sanonut.
\par 26 Sentähden, jos teille sanotaan: 'Katso, hän on erämaassa', niin älkää menkö sinne, tahi: 'Katso, hän on kammiossa', niin älkää uskoko.
\par 27 Sillä niinkuin salama leimahtaa idästä ja näkyy hamaan länteen, niin on oleva Ihmisen Pojan tulemus.
\par 28 Missä raato on, sinne kotkat kokoontuvat.
\par 29 Mutta kohta niiden päivien ahdistuksen jälkeen aurinko pimenee, eikä kuu anna valoansa, ja tähdet putoavat taivaalta, ja taivaitten voimat järkkyvät.
\par 30 Ja silloin Ihmisen Pojan merkki näkyy taivaalla, ja silloin kaikki maan sukukunnat parkuvat; ja he näkevät Ihmisen Pojan tulevan taivaan pilvien päällä suurella voimalla ja kirkkaudella.
\par 31 Ja hän lähettää enkelinsä suuren pasunan pauhatessa, ja he kokoavat hänen valittunsa neljältä ilmalta, taivasten ääristä hamaan toisiin ääriin.
\par 32 Mutta oppikaa viikunapuusta vertaus: kun sen oksa jo on tuore ja lehdet puhkeavat, niin te tiedätte, että kesä on lähellä.
\par 33 Samoin te myös, kun näette tämän kaiken, tietäkää, että se on lähellä, oven edessä.
\par 34 Totisesti minä sanon teille: tämä sukupolvi ei katoa, ennenkuin kaikki nämä tapahtuvat.
\par 35 Taivas ja maa katoavat, mutta minun sanani eivät koskaan katoa.
\par 36 Mutta siitä päivästä ja hetkestä ei tiedä kukaan, eivät taivasten enkelit, eikä myöskään Poika, vaan Isä yksin.
\par 37 Sillä niinkuin oli Nooan päivinä, niin on Ihmisen Pojan tulemus oleva.
\par 38 Sillä niinkuin ihmiset olivat niinä päivinä ennen vedenpaisumusta: söivät ja joivat, naivat ja naittivat, aina siihen päivään asti, jona Nooa meni arkkiin,
\par 39 eivätkä tienneet, ennenkuin vedenpaisumus tuli ja vei heidät kaikki; niin on myös Ihmisen Pojan tulemus oleva.
\par 40 Silloin on kaksi miestä pellolla; toinen korjataan talteen, ja toinen jätetään.
\par 41 Kaksi naista on jauhamassa käsikivillä; toinen korjataan talteen, ja toinen jätetään.
\par 42 Valvokaa siis, sillä ette tiedä, minä päivänä teidän Herranne tulee.
\par 43 Mutta se tietäkää: jos perheenisäntä tietäisi, millä yövartiolla varas tulee, totta hän valvoisi, eikä sallisi taloonsa murtauduttavan.
\par 44 Sentähden olkaa tekin valmiit, sillä sinä hetkenä, jona ette luule, Ihmisen Poika tulee.
\par 45 Kuka siis on se uskollinen ja ymmärtäväinen palvelija, jonka hänen herransa on asettanut pitämään huolta palvelusväestään, antamaan heille ruokaa ajallansa?
\par 46 Autuas se palvelija, jonka hänen herransa tullessaan havaitsee näin tekevän!
\par 47 Totisesti minä sanon teille: hän asettaa hänet kaiken omaisuutensa hoitajaksi.
\par 48 Mutta jos paha palvelija sanoo sydämessään: 'Minun herrani viipyy',
\par 49 ja rupeaa lyömään kanssapalvelijoitaan ja syö ja juo juopuneiden kanssa,
\par 50 niin sen palvelijan herra tulee päivänä, jona hän ei odota, ja hetkenä, jota hän ei arvaa,
\par 51 ja hakkaa hänet kappaleiksi ja määrää hänelle saman osan kuin ulkokullatuille. Siellä on oleva itku ja hammasten kiristys."

\chapter{25}

\par 1 "Silloin on taivasten valtakunta oleva kymmenen neitsyen kaltainen, jotka ottivat lamppunsa ja lähtivät ylkää vastaan.
\par 2 Mutta viisi heistä oli tyhmää ja viisi ymmärtäväistä.
\par 3 Tyhmät ottivat lamppunsa, mutta eivät ottaneet öljyä mukaansa.
\par 4 Mutta ymmärtäväiset ottivat öljyä astioihinsa ynnä lamppunsa.
\par 5 Yljän viipyessä tuli heille kaikille uni, ja he nukkuivat.
\par 6 Mutta yösydännä kuului huuto: 'Katso, ylkä tulee! Menkää häntä vastaan.'
\par 7 Silloin kaikki nämä neitsyet nousivat ja laittoivat lamppunsa kuntoon.
\par 8 Ja tyhmät sanoivat ymmärtäväisille: 'Antakaa meille öljyänne, sillä meidän lamppumme sammuvat'.
\par 9 Mutta ymmärtäväiset vastasivat ja sanoivat: 'Emme voi, se ei riitä meille ja teille. Menkää ennemmin myyjäin luo ostamaan itsellenne.'
\par 10 Mutta heidän lähdettyään ostamaan ylkä tuli; ja ne, jotka olivat valmiit, menivät hänen kanssansa häihin, ja ovi suljettiin.
\par 11 Ja myöhemmin toisetkin neitsyet tulivat ja sanoivat: 'Herra, Herra, avaa meille!'
\par 12 Mutta hän vastasi ja sanoi: 'Totisesti minä sanon teille: minä en tunne teitä'.
\par 13 Valvokaa siis, sillä ette tiedä päivää ettekä hetkeä.
\par 14 Sillä tapahtuu, niinkuin tapahtui, kun mies matkusti muille maille: hän kutsui palvelijansa ja uskoi heille omaisuutensa;
\par 15 yhdelle hän antoi viisi leiviskää, toiselle kaksi ja kolmannelle yhden, kullekin hänen kykynsä mukaan, ja lähti muille maille.
\par 16 Se, joka oli saanut viisi leiviskää, meni kohta ja asioitsi niillä ja voitti toiset viisi leiviskää.
\par 17 Samoin kaksi leiviskää saanut voitti toiset kaksi.
\par 18 Mutta yhden leiviskän saanut meni pois ja kaivoi kuopan maahan ja kätki siihen herransa rahan.
\par 19 Pitkän ajan kuluttua näiden palvelijain herra palasi ja ryhtyi tilintekoon heidän kanssansa.
\par 20 Silloin tuli se, joka oli saanut viisi leiviskää, ja toi toiset viisi leiviskää ja sanoi: 'Herra, viisi leiviskää sinä minulle uskoit, katso, toiset viisi leiviskää minä olen voittanut'.
\par 21 Hänen herransa sanoi hänelle: 'Hyvä on, sinä hyvä ja uskollinen palvelija. Vähässä sinä olet ollut uskollinen, minä panen sinut paljon haltijaksi. Mene herrasi iloon.'
\par 22 Myös se, joka oli saanut kaksi leiviskää, tuli ja sanoi: 'Herra, kaksi leiviskää sinä minulle uskoit, katso, toiset kaksi leiviskää minä olen voittanut'.
\par 23 Hänen herransa sanoi hänelle: 'Hyvä on, sinä hyvä ja uskollinen palvelija. Vähässä sinä olet ollut uskollinen, minä panen sinut paljon haltijaksi. Mene herrasi iloon.'
\par 24 Sitten myös se, joka oli saanut yhden leiviskän, tuli ja sanoi: 'Herra, minä tiesin sinut kovaksi mieheksi; sinä leikkaat sieltä, mihin et ole kylvänyt, ja kokoat sieltä, missä et ole eloa viskannut.
\par 25 Ja peloissani minä menin ja kätkin sinun leiviskäsi maahan; katso, tässä on omasi.'
\par 26 Mutta hänen herransa vastasi ja sanoi hänelle: 'Sinä paha ja laiska palvelija! Sinä tiesit minun leikkaavan sieltä, mihin en ole kylvänyt, ja kokoavan sieltä, missä en ole viskannut.
\par 27 Sinun olisi siis pitänyt jättää minun rahani rahanvaihtajille, niin minä tultuani olisin saanut omani takaisin korkoineen.
\par 28 Ottakaa sentähden leiviskä häneltä pois ja antakaa sille, jolla on kymmenen leiviskää.
\par 29 Sillä jokaiselle, jolla on, annetaan, ja hänellä on oleva yltäkyllin; mutta jolla ei ole, siltä otetaan pois sekin, mikä hänellä on.
\par 30 Ja heittäkää tuo kelvoton palvelija ulos pimeyteen; siellä on oleva itku ja hammasten kiristys.'
\par 31 Mutta kun Ihmisen Poika tulee kirkkaudessaan ja kaikki enkelit hänen kanssaan, silloin hän istuu kirkkautensa valtaistuimelle.
\par 32 Ja hänen eteensä kootaan kaikki kansat, ja hän erottaa toiset toisista, niinkuin paimen erottaa lampaat vuohista.
\par 33 Ja hän asettaa lampaat oikealle puolelleen, mutta vuohet vasemmalle.
\par 34 Silloin Kuningas sanoo oikealla puolellaan oleville: 'Tulkaa, minun Isäni siunatut, ja omistakaa se valtakunta, joka on ollut teille valmistettuna maailman perustamisesta asti.
\par 35 Sillä minun oli nälkä, ja te annoitte minulle syödä; minun oli jano, ja te annoitte minulle juoda; minä olin outo, ja te otitte minut huoneeseenne;
\par 36 minä olin alaston, ja te vaatetitte minut; minä sairastin, ja te kävitte minua katsomassa; minä olin vankeudessa, ja te tulitte minun tyköni.'
\par 37 Silloin vanhurskaat vastaavat hänelle sanoen: 'Herra, milloin me näimme sinut nälkäisenä ja ruokimme sinua, tai janoisena ja annoimme sinulle juoda?
\par 38 Ja milloin me näimme sinut outona ja otimme sinut huoneeseemme, tai alastonna ja vaatetimme sinut?
\par 39 Ja milloin me näimme sinun sairastavan tai olevan vankeudessa ja tulimme sinun tykösi?'
\par 40 Niin Kuningas vastaa ja sanoo heille: 'Totisesti minä sanon teille: kaikki, mitä olette tehneet yhdelle näistä minun vähimmistä veljistäni, sen te olette tehneet minulle'.
\par 41 Sitten hän myös sanoo vasemmalla puolellaan oleville: 'Menkää pois minun tyköäni, te kirotut, siihen iankaikkiseen tuleen, joka on valmistettu perkeleelle ja hänen enkeleillensä.
\par 42 Sillä minun oli nälkä, ja te ette antaneet minulle syödä; minun oli jano, ja te ette antaneet minulle juoda;
\par 43 minä olin outo, ja te ette ottaneet minua huoneeseenne; minä olin alaston, ja te ette vaatettaneet minua; sairaana ja vankeudessa, ja te ette käyneet minua katsomassa.'
\par 44 Silloin hekin vastaavat sanoen: 'Herra, milloin me näimme sinut nälkäisenä tai janoisena tai outona tai alastonna tai sairaana tai vankeudessa, emmekä sinua palvelleet?'
\par 45 Silloin hän vastaa heille ja sanoo: 'Totisesti minä sanon teille: kaiken, minkä olette jättäneet tekemättä yhdelle näistä vähimmistä, sen te olette jättäneet tekemättä minulle'.
\par 46 Ja nämä menevät pois iankaikkiseen rangaistukseen, mutta vanhurskaat iankaikkiseen elämään."

\chapter{26}

\par 1 Ja kun Jeesus oli lopettanut kaikki nämä puheet, sanoi hän opetuslapsillensa:
\par 2 "Te tiedätte, että kahden päivän perästä on pääsiäinen; silloin Ihmisen Poika annetaan ristiinnaulittavaksi".
\par 3 Silloin ylipapit ja kansan vanhimmat kokoontuivat Kaifas nimisen ylimmäisen papin palatsiin
\par 4 ja neuvottelivat, kuinka ottaisivat Jeesuksen kiinni kavaluudella ja tappaisivat hänet.
\par 5 Mutta he sanoivat: "Ei juhlan aikana, ettei syntyisi meteliä kansassa".
\par 6 Kun Jeesus oli Betaniassa pitalisen Simonin asunnossa,
\par 7 tuli hänen luoksensa nainen, mukanaan alabasteripullo täynnä kallisarvoista voidetta, minkä hän vuodatti Jeesuksen päähän hänen ollessaan aterialla.
\par 8 Mutta sen nähdessään hänen opetuslapsensa närkästyivät ja sanoivat: "Mitä varten tämä haaskaus?
\par 9 Olisihan sen voinut myydä kalliista hinnasta ja antaa rahat köyhille."
\par 10 Kun Jeesus sen huomasi, sanoi hän heille: "Miksi pahoitatte tämän naisen mieltä? Sillä hän teki hyvän työn minulle.
\par 11 Köyhät teillä on aina keskuudessanne, mutta minua teillä ei ole aina.
\par 12 Sillä kun hän valoi tämän voiteen minun ruumiilleni, teki hän sen minun hautaamistani varten.
\par 13 Totisesti minä sanon teille: missä ikinä kaikessa maailmassa tätä evankeliumia saarnataan, siellä sekin, minkä hän teki, on mainittava hänen muistoksensa."
\par 14 Silloin meni yksi niistä kahdestatoista, nimeltä Juudas Iskariot, ylipappien luo
\par 15 ja sanoi: "Mitä tahdotte antaa minulle, niin minä saatan hänet teidän käsiinne?" Ja he maksoivat hänelle kolmekymmentä hopearahaa.
\par 16 Ja siitä hetkestä hän etsi sopivaa tilaisuutta kavaltaaksensa hänet.
\par 17 Mutta ensimmäisenä happamattoman leivän päivänä opetuslapset tulivat Jeesuksen tykö ja sanoivat: "Mihin tahdot, että valmistamme pääsiäislampaan sinun syödäksesi?"
\par 18 Niin hän sanoi: "Menkää kaupunkiin sen ja sen luo ja sanokaa hänelle: 'Opettaja sanoo: Minun aikani on lähellä; sinun luonasi minä syön pääsiäisaterian opetuslasteni kanssa'."
\par 19 Ja opetuslapset tekivät, niinkuin Jeesus oli heitä käskenyt, ja valmistivat pääsiäislampaan.
\par 20 Ja kun ehtoo tuli, asettui hän aterialle kahdentoista opetuslapsensa kanssa.
\par 21 Ja heidän syödessään hän sanoi: "Totisesti minä sanon teille: yksi teistä kavaltaa minut".
\par 22 Silloin he tulivat kovin murheellisiksi ja rupesivat toinen toisensa perästä sanomaan hänelle: "Herra, en kai minä ole se?"
\par 23 Hän vastasi ja sanoi: "Joka minun kanssani pisti kätensä vatiin, se kavaltaa minut.
\par 24 Ihmisen Poika tosin menee pois, niinkuin hänestä on kirjoitettu, mutta voi sitä ihmistä, jonka kautta Ihmisen Poika kavalletaan! Parempi olisi sille ihmiselle, että hän ei olisi syntynyt."
\par 25 Niin Juudas, joka hänet kavalsi, vastasi ja sanoi: "Rabbi, en kai minä ole se?" Hän sanoi hänelle: "Sinäpä sen sanoit".
\par 26 Ja heidän syödessään Jeesus otti leivän, siunasi, mursi ja antoi opetuslapsillensa ja sanoi: "Ottakaa ja syökää; tämä on minun ruumiini".
\par 27 Ja hän otti maljan, kiitti ja antoi heille ja sanoi: "Juokaa tästä kaikki;
\par 28 sillä tämä on minun vereni, liiton veri, joka monen edestä vuodatetaan syntien anteeksiantamiseksi.
\par 29 Ja minä sanon teille: tästedes minä en juo tätä viinipuun antia, ennenkuin sinä päivänä, jona juon sitä uutena teidän kanssanne Isäni valtakunnassa."
\par 30 Ja veisattuaan kiitosvirren he lähtivät Öljymäelle.
\par 31 Silloin Jeesus sanoi heille: "Tänä yönä te kaikki loukkaannutte minuun; sillä kirjoitettu on: 'Minä lyön paimenta, ja lauman lampaat hajotetaan'.
\par 32 Mutta ylösnoustuani minä menen teidän edellänne Galileaan."
\par 33 Niin Pietari vastasi ja sanoi hänelle: "Vaikka kaikki loukkaantuisivat sinuun, niin minä en koskaan loukkaannu".
\par 34 Jeesus sanoi hänelle: "Totisesti minä sanon sinulle: tänä yönä, ennenkuin kukko laulaa, sinä kolmesti minut kiellät".
\par 35 Pietari sanoi hänelle: "Vaikka minun pitäisi kuolla sinun kanssasi, en sittenkään minä sinua kiellä". Samoin sanoivat myös kaikki muut opetuslapset.
\par 36 Sitten Jeesus tuli heidän kanssaan Getsemane nimiselle maatilalle; ja hän sanoi opetuslapsillensa: "Istukaa tässä, sillä aikaa kuin minä menen ja rukoilen tuolla".
\par 37 Ja hän otti mukaansa Pietarin ja ne kaksi Sebedeuksen poikaa; ja hän alkoi murehtia ja tulla tuskaan.
\par 38 Silloin hän sanoi heille: "Minun sieluni on syvästi murheellinen, kuolemaan asti; olkaa tässä ja valvokaa minun kanssani".
\par 39 Ja hän meni vähän edemmäksi, lankesi kasvoilleen ja rukoili sanoen: "Isäni, jos mahdollista on, niin menköön minulta pois tämä malja; ei kuitenkaan niinkuin minä tahdon, vaan niinkuin sinä".
\par 40 Ja hän tuli opetuslasten tykö ja tapasi heidät nukkumasta ja sanoi Pietarille: "Niin ette siis jaksaneet yhtä hetkeä valvoa minun kanssani!
\par 41 Valvokaa ja rukoilkaa, ettette joutuisi kiusaukseen; henki tosin on altis, mutta liha on heikko."
\par 42 Taas hän meni pois toisen kerran ja rukoili sanoen: "Isäni, jos tämä malja ei voi mennä minun ohitseni, minun sitä juomattani, niin tapahtukoon sinun tahtosi".
\par 43 Ja tullessaan hän taas tapasi heidät nukkumasta, sillä heidän silmänsä olivat käyneet raukeiksi.
\par 44 Ja hän jätti heidät, meni taas ja rukoili kolmannen kerran ja sanoi samat sanat uudestaan.
\par 45 Sitten hän tuli opetuslasten tykö ja sanoi heille: "Te nukutte vielä ja lepäätte! Katso, hetki on lähellä, ja Ihmisen Poika annetaan syntisten käsiin.
\par 46 Nouskaa, lähtekäämme; katso, se on lähellä, joka minut kavaltaa."
\par 47 Ja katso, hänen vielä puhuessaan tuli Juudas, yksi niistä kahdestatoista, ja hänen kanssaan lukuisa joukko ylipappien ja kansan vanhinten luota miekat ja seipäät käsissä.
\par 48 Ja se, joka hänet kavalsi, oli antanut heille merkin sanoen: "Se, jolle minä suuta annan, hän se on; ottakaa hänet kiinni".
\par 49 Ja kohta hän meni Jeesuksen luo ja sanoi: "Terve, rabbi!" ja antoi hänelle suuta.
\par 50 Niin Jeesus sanoi hänelle: "Ystäväni, mitä varten sinä tänne tulit?" Silloin he tulivat Jeesuksen luo, kävivät häneen käsiksi ja ottivat hänet kiinni.
\par 51 Ja katso, eräs niistä, jotka olivat Jeesuksen kanssa, ojensi kätensä, veti miekkansa ja iski ylimmäisen papin palvelijaa ja sivalsi häneltä pois korvan.
\par 52 Silloin Jeesus sanoi hänelle: "Pistä miekkasi tuppeen; sillä kaikki, jotka miekkaan tarttuvat, ne miekkaan hukkuvat.
\par 53 Vai luuletko, etten voisi rukoilla Isääni, niin että hän lähettäisi heti minulle enemmän kuin kaksitoista legionaa enkeleitä?
\par 54 Mutta kuinka silloin kävisivät toteen kirjoitukset, jotka sanovat, että näin pitää tapahtuman?"
\par 55 Sillä hetkellä Jeesus sanoi joukolle: "Niinkuin ryöväriä vastaan te olette lähteneet minua miekoilla ja seipäillä vangitsemaan. Joka päivä minä olen istunut pyhäkössä opettamassa, ettekä ole ottaneet minua kiinni.
\par 56 Mutta tämä kaikki on tapahtunut, että profeettain kirjoitukset kävisivät toteen." Silloin kaikki opetuslapset jättivät hänet ja pakenivat.
\par 57 Mutta Jeesuksen kiinniottajat veivät hänet ylimmäisen papin, Kaifaan, eteen, jonne kirjanoppineet ja vanhimmat olivat kokoontuneet.
\par 58 Ja Pietari seurasi häntä taampana ylimmäisen papin esipihaan asti; ja hän meni sinne ja istui palvelijain joukkoon nähdäksensä, kuinka lopulta kävisi.
\par 59 Mutta ylipapit ja koko neuvosto etsivät väärää todistusta Jeesusta vastaan tappaaksensa hänet,
\par 60 mutta eivät löytäneet, vaikka monta väärää todistajaa oli tullut esille. Mutta vihdoin tuli kaksi,
\par 61 ja he sanoivat: "Tämä on sanonut: 'Minä voin hajottaa maahan Jumalan temppelin ja kolmessa päivässä sen rakentaa'."
\par 62 Silloin ylimmäinen pappi nousi ja sanoi hänelle: "Etkö vastaa mitään siihen, mitä nämä todistavat sinua vastaan?"
\par 63 Mutta Jeesus oli vaiti. Niin ylimmäinen pappi sanoi hänelle: "Minä vannotan sinua elävän Jumalan kautta, että sanot meille, oletko sinä Kristus, Jumalan Poika".
\par 64 Jeesus sanoi hänelle: "Sinäpä sen sanoit. Mutta minä sanon teille: tästedes te saatte nähdä Ihmisen Pojan istuvan Voiman oikealla puolella ja tulevan taivaan pilvien päällä."
\par 65 Silloin ylimmäinen pappi repäisi vaatteensa ja sanoi: "Hän on pilkannut Jumalaa. Mitä me enää todistajia tarvitsemme? Katso, nyt kuulitte hänen pilkkaamisensa.
\par 66 Miten teistä on?" He vastasivat sanoen: "Hän on vikapää kuolemaan".
\par 67 Silloin he sylkivät häntä silmille ja löivät häntä nyrkillä kasvoihin; ja toiset sivalsivat häntä poskelle
\par 68 ja sanoivat: "Profetoi meille, Kristus, kuka se on, joka sinua löi".
\par 69 Mutta Pietari istui ulkopuolella esipihassa. Ja hänen luoksensa tuli muuan palvelijatar ja sanoi: "Sinäkin olit Jeesuksen, galilealaisen, seurassa".
\par 70 Mutta hän kielsi kaikkien kuullen ja sanoi: "En ymmärrä, mitä sanot".
\par 71 Ja kun hän oli mennyt ulos portille, näki hänet toinen nainen ja sanoi sielläoleville: "Tämäkin oli Jeesuksen, Nasaretilaisen, seurassa".
\par 72 Ja taas hän kielsi valalla vannoen: "En tunne sitä miestä".
\par 73 Vähän sen jälkeen tulivat ne, jotka siinä seisoivat, ja sanoivat Pietarille: "Totisesti, sinä myös olet yksi heistä, sillä kielimurteesikin ilmaisee sinut".
\par 74 Silloin hän rupesi sadattelemaan ja vannomaan: "En tunne sitä miestä". Ja samassa lauloi kukko.
\par 75 Niin Pietari muisti Jeesuksen sanat, jotka hän oli sanonut: "Ennenkuin kukko laulaa, sinä kolmesti minut kiellät". Ja hän meni ulos ja itki katkerasti.

\chapter{27}

\par 1 Mutta aamun koittaessa kaikki ylipapit ja kansan vanhimmat pitivät neuvoa Jeesusta vastaan tappaaksensa hänet;
\par 2 ja he sitoivat hänet ja veivät pois ja antoivat hänet maaherran, Pilatuksen, käsiin.
\par 3 Kun Juudas, hänen kavaltajansa, näki, että hänet oli tuomittu, silloin hän katui ja toi takaisin ne kolmekymmentä hopearahaa ylipapeille ja vanhimmille
\par 4 ja sanoi: "Minä tein synnin, kun kavalsin viattoman veren". Mutta he sanoivat: "Mitä se meihin koskee? Katso itse eteesi."
\par 5 Ja hän viskasi hopearahat temppeliin, lähti sieltä, meni pois ja hirttäytyi.
\par 6 Niin ylipapit ottivat hopearahat ja sanoivat: "Ei ole luvallista panna näitä temppelirahastoon, koska ne ovat veren hinta".
\par 7 Ja neuvoteltuaan he ostivat niillä savenvalajan pellon muukalaisten hautausmaaksi.
\par 8 Sentähden sitä peltoa vielä tänäkin päivänä kutsutaan Veripelloksi.
\par 9 Silloin kävi toteen, mikä on puhuttu profeetta Jeremiaan kautta, joka sanoo: "Ja he ottivat ne kolmekymmentä hopearahaa, hinnan siitä arvioidusta miehestä, jonka he olivat israelilaisten puolesta arvioineet,
\par 10 ja antoivat ne savenvalajan pellosta, niinkuin Herra oli minun kauttani käskenyt".
\par 11 Mutta Jeesus seisoi maaherran edessä. Ja maaherra kysyi häneltä sanoen: "Oletko sinä juutalaisten kuningas?" Niin Jeesus sanoi: "Sinäpä sen sanot".
\par 12 Ja kun ylipapit ja vanhimmat häntä syyttivät, ei hän mitään vastannut.
\par 13 Silloin Pilatus sanoi hänelle: "Etkö kuule, kuinka paljon he todistavat sinua vastaan?"
\par 14 Mutta hän ei vastannut yhteenkään hänen kysymykseensä, niin että maaherra suuresti ihmetteli.
\par 15 Oli tapana, että maaherra juhlan aikana päästi kansalle yhden vangin irti, kenenkä he tahtoivat.
\par 16 Ja heillä oli silloin kuuluisa vanki, jota sanottiin Barabbaaksi.
\par 17 Kun he nyt olivat koolla, sanoi Pilatus heille: "Kummanko tahdotte, että minä teille päästän, Barabbaanko vai Jeesuksen, jota sanotaan Kristukseksi?"
\par 18 Sillä hän tiesi, että he kateudesta olivat antaneet hänet hänen käsiinsä.
\par 19 Mutta kun hän istui tuomarinistuimella, lähetti hänen vaimonsa hänelle sanan: "Älä puutu siihen vanhurskaaseen mieheen, sillä minä olen tänä yönä unessa paljon kärsinyt hänen tähtensä".
\par 20 Mutta ylipapit ja vanhimmat yllyttivät kansaa anomaan Barabbasta, mutta surmauttamaan Jeesuksen.
\par 21 Ja maaherra puhui heille ja sanoi: "Kummanko näistä kahdesta tahdotte, että minä teille päästän?" Niin he sanoivat: "Barabbaan".
\par 22 Pilatus sanoi heille: "Mitä minun sitten on tehtävä Jeesukselle, jota sanotaan Kristukseksi?" He sanoivat kaikki: "Ristiinnaulittakoon!"
\par 23 Niin maaherra sanoi: "Mitä pahaa hän sitten on tehnyt?" Mutta he huusivat vielä kovemmin sanoen: "Ristiinnaulittakoon!"
\par 24 Ja kun Pilatus näki, ettei mikään auttanut, vaan että meteli yhä yltyi, otti hän vettä ja pesi kätensä kansan nähden ja sanoi: "Viaton olen minä tämän miehen vereen. Katsokaa itse eteenne."
\par 25 Niin kaikki kansa vastasi ja sanoi: "Tulkoon hänen verensä meidän päällemme ja meidän lastemme päälle".
\par 26 Silloin hän päästi heille Barabbaan, mutta Jeesuksen hän ruoskitti ja luovutti ristiinnaulittavaksi.
\par 27 Silloin maaherran sotamiehet veivät Jeesuksen mukanaan palatsiin ja keräsivät hänen ympärilleen koko sotilasjoukon.
\par 28 Ja he riisuivat hänet ja panivat hänen päällensä tulipunaisen vaipan
\par 29 ja väänsivät orjantappuroista kruunun, panivat sen hänen päähänsä ja ruovon hänen oikeaan käteensä, polvistuivat hänen eteensä ja pilkkasivat häntä ja sanoivat: "Terve, juutalaisten kuningas!"
\par 30 Ja he sylkivät häntä, ottivat ruovon ja löivät häntä päähän.
\par 31 Ja kun he olivat häntä pilkanneet, riisuivat he häneltä vaipan, pukivat hänet hänen omiin vaatteisiinsa ja veivät hänet pois ristiinnaulittavaksi.
\par 32 Ja matkalla he tapasivat kyreneläisen miehen, jonka nimi oli Simon. Hänet he pakottivat kantamaan hänen ristiänsä.
\par 33 Ja tultuaan paikalle, jota sanotaan Golgataksi - se on: pääkallon paikaksi -
\par 34 he tarjosivat hänelle juotavaksi katkeralla nesteellä sekoitettua viiniä; mutta maistettuaan hän ei tahtonut sitä juoda.
\par 35 Ja kun he olivat hänet ristiinnaulinneet, jakoivat he keskenään hänen vaatteensa heittäen niistä arpaa.
\par 36 Sitten he istuutuivat ja vartioivat häntä siellä.
\par 37 Ja he olivat panneet hänen päänsä yläpuolelle hänen syynsä julki, näin kirjoitettuna: "Tämä on Jeesus, juutalaisten kuningas".
\par 38 Silloin ristiinnaulittiin hänen kanssansa kaksi ryöväriä, toinen oikealle ja toinen vasemmalle puolelle.
\par 39 Ja ne, jotka kulkivat ohitse, herjasivat häntä, nyökyttivät päätänsä
\par 40 ja sanoivat: "Sinä, joka hajotat maahan temppelin ja kolmessa päivässä sen rakennat, auta itseäsi, jos olet Jumalan Poika, ja astu alas ristiltä".
\par 41 Samoin ylipapit ja kirjanoppineet ja vanhimmat pilkkasivat häntä ja sanoivat:
\par 42 "Muita hän on auttanut, itseään ei voi auttaa. Onhan hän Israelin kuningas; astukoon nyt alas ristiltä, niin me uskomme häneen.
\par 43 Hän on luottanut Jumalaan; vapahtakoon nyt Jumala hänet, jos on häneen mielistynyt; sillä hän on sanonut: 'Minä olen Jumalan Poika'."
\par 44 Ja samalla tavalla herjasivat häntä ryöväritkin, jotka olivat ristiinnaulitut hänen kanssansa.
\par 45 Mutta kuudennesta hetkestä alkaen tuli pimeys yli kaiken maan, ja sitä kesti hamaan yhdeksänteen hetkeen.
\par 46 Ja yhdeksännen hetken vaiheilla Jeesus huusi suurella äänellä sanoen: "Eeli, Eeli, lama sabaktani?" Se on: Jumalani, Jumalani, miksi minut hylkäsit?
\par 47 Sen kuullessaan sanoivat muutamat niistä, jotka siinä seisoivat: "Hän huutaa Eliasta".
\par 48 Ja kohta muuan heistä juoksi ja otti sienen, täytti sen hapanviinillä, pani sen ruovon päähän ja antoi hänelle juoda.
\par 49 Mutta muut sanoivat: "Annas, katsokaamme, tuleeko Elias häntä pelastamaan".
\par 50 Niin Jeesus taas huusi suurella äänellä ja antoi henkensä.
\par 51 Ja katso, temppelin esirippu repesi kahtia ylhäältä alas asti, ja maa järisi, ja kalliot halkesivat,
\par 52 ja haudat aukenivat, ja monta nukkuneiden pyhien ruumista nousi ylös.
\par 53 Ja he lähtivät haudoistaan ja tulivat hänen ylösnousemisensa jälkeen pyhään kaupunkiin ja ilmestyivät monelle.
\par 54 Mutta kun sadanpäämies ja ne, jotka hänen kanssaan vartioitsivat Jeesusta, näkivät maanjäristyksen ja mitä muuta tapahtui, peljästyivät he suuresti ja sanoivat: "Totisesti tämä oli Jumalan Poika".
\par 55 Ja siellä oli monta naista, jotka olivat Galileasta seuranneet Jeesusta ja palvelleet häntä; he seisoivat taampana katselemassa.
\par 56 Heidän joukossaan oli Maria Magdaleena ja Maria, Jaakobin ja Joosefin äiti, ja Sebedeuksen poikain äiti.
\par 57 Mutta illan tultua saapui rikas mies, Arimatiasta kotoisin, nimeltä Joosef, joka hänkin oli Jeesuksen opetuslapsi.
\par 58 Hän meni Pilatuksen luo ja pyysi Jeesuksen ruumista. Silloin Pilatus käski antaa sen hänelle.
\par 59 Ja Joosef otti ruumiin, kääri sen puhtaaseen liinavaatteeseen
\par 60 ja pani sen uuteen hautakammioonsa, jonka oli hakkauttanut kallioon. Ja hän vieritti suuren kiven hautakammion ovelle ja lähti pois.
\par 61 Ja siellä olivat Maria Magdaleena ja toinen Maria, jotka istuivat vastapäätä hautaa.
\par 62 Seuraavana päivänä, joka oli valmistuspäivän jälkeinen, ylipapit ja fariseukset kokoontuivat Pilatuksen luo
\par 63 ja sanoivat: "Herra, me muistamme sen villitsijän vielä eläessään sanoneen: 'Kolmen päivän kuluttua minä nousen ylös'.
\par 64 Käske siis tarkasti vartioida hautaa kolmanteen päivään asti, etteivät hänen opetuslapsensa tulisi ja varastaisi häntä ja sanoisi kansalle: 'Hän nousi kuolleista', ja niin viimeinen villitys olisi pahempi kuin ensimmäinen."
\par 65 Niin Pilatus sanoi heille: "Tuossa on vartijaväkeä, menkää, vartioikaa niin hyvin kuin taidatte".
\par 66 Niin he menivät ja turvasivat haudan lukitsemalla kiven sinetillä ja asettamalla vartijat.

\chapter{28}

\par 1 Ja kun sapatti oli päättynyt ja viikon ensimmäisen päivän aamu koitti, tulivat Maria Magdaleena ja se toinen Maria katsomaan hautaa.
\par 2 Ja katso, tapahtui suuri maanjäristys, sillä Herran enkeli astui alas taivaasta, tuli ja vieritti kiven pois ja istui sille.
\par 3 Hän oli näöltään niinkuin salama, ja hänen vaatteensa olivat valkeat kuin lumi.
\par 4 Ja häntä peljästyen vartijat vapisivat ja kävivät ikäänkuin kuolleiksi.
\par 5 Mutta enkeli puhutteli naisia ja sanoi heille: "Älkää te peljätkö; sillä minä tiedän teidän etsivän Jeesusta, joka oli ristiinnaulittu.
\par 6 Ei hän ole täällä, sillä hän on noussut ylös, niinkuin hän sanoi. Tulkaa, katsokaa paikkaa, jossa hän on maannut.
\par 7 Ja menkää kiiruusti ja sanokaa hänen opetuslapsillensa, että hän on noussut kuolleista. Ja katso, hän menee teidän edellänne Galileaan; siellä te saatte hänet nähdä. Katso, minä olen sen teille sanonut."
\par 8 Ja he menivät kiiruusti haudalta peloissaan ja suuresti iloiten ja juoksivat viemään sanaa hänen opetuslapsillensa.
\par 9 Mutta katso, Jeesus tuli heitä vastaan ja sanoi: "Terve teille!" Ja he menivät hänen tykönsä, syleilivät hänen jalkojaan ja kumartaen rukoilivat häntä.
\par 10 Silloin Jeesus sanoi heille: "Älkää peljätkö; menkää ja viekää sana minun veljilleni, että he menisivät Galileaan: siellä he saavat minut nähdä".
\par 11 Mutta heidän mennessään, katso, muutamat vartijaväestä tulivat kaupunkiin ja ilmoittivat ylipapeille kaikki, mitä oli tapahtunut.
\par 12 Ja nämä kokoontuivat vanhinten kanssa ja pitivät neuvoa ja antoivat sotamiehille runsaat rahat
\par 13 ja sanoivat: "Sanokaa, että hänen opetuslapsensa tulivat yöllä ja veivät hänet varkain meidän nukkuessamme.
\par 14 Ja jos tämä tulee maaherran korviin, niin me lepytämme hänet ja toimitamme niin, että saatte olla huoletta."
\par 15 Niin he ottivat rahat ja tekivät, niinkuin heitä oli neuvottu. Ja tätä puhetta on levitetty juutalaisten kesken, ja sitä kerrotaan vielä tänäkin päivänä.
\par 16 Ja ne yksitoista opetuslasta vaelsivat Galileaan sille vuorelle, jonne Jeesus oli käskenyt heidän mennä.
\par 17 Ja kun he näkivät hänet, niin he kumartaen rukoilivat häntä, mutta muutamat epäilivät.
\par 18 Ja Jeesus tuli heidän tykönsä ja puhui heille ja sanoi: "Minulle on annettu kaikki valta taivaassa ja maan päällä.
\par 19 Menkää siis ja tehkää kaikki kansat minun opetuslapsikseni, kastamalla heitä Isän ja Pojan ja Pyhän Hengen nimeen
\par 20 ja opettamalla heitä pitämään kaikki, mitä minä olen käskenyt teidän pitää. Ja katso, minä olen teidän kanssanne joka päivä maailman loppuun asti."


\end{document}