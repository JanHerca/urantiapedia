\begin{document}

\title{Evankeliumi Luukkaan mukaan}


\chapter{1}

\par 1 Koska monet ovat ryhtyneet tekemään kertomusta meidän keskuudessamme tosiksi tunnetuista tapahtumista,
\par 2 sen mukaisesti kuin meille ovat kertoneet ne, jotka alusta asti ovat omin silmin ne nähneet ja olleet sanan palvelijoita,
\par 3 niin olen minäkin, tarkkaan tutkittuani alusta alkaen kaikki, päättänyt kirjoittaa ne järjestyksessään sinulle, korkea-arvoinen Teofilus,
\par 4 että oppisit tuntemaan, kuinka varmat ne asiat ovat, jotka sinulle on opetettu.
\par 5 Herodeksen, Juudean kuninkaan, aikana oli pappi, nimeltä Sakarias, Abian osastoa. Ja hänen vaimonsa oli Aaronin tyttäriä, ja tämän nimi oli Elisabet.
\par 6 He olivat molemmat hurskaita Jumalan edessä, vaeltaen kaikissa Herran käskyissä ja säädöksissä nuhteettomina.
\par 7 Mutta heillä ei ollut lasta, sillä Elisabet oli hedelmätön; ja he olivat molemmat tulleet iällisiksi.
\par 8 Niin tapahtui, kun hänen osastonsa palvelusvuoro tuli ja hän toimitti papillisia tehtäviä Jumalan edessä,
\par 9 että hän tavanmukaisessa pappistehtävien arpomisessa sai osaksensa mennä Herran temppeliin suitsuttamaan.
\par 10 Ja kaikki kansa oli suitsuttamisen aikana ulkopuolella rukoilemassa.
\par 11 Silloin ilmestyi hänelle Herran enkeli seisoen suitsutusalttarin oikealla puolella.
\par 12 Ja hänet nähdessään Sakarias hämmästyi, ja hänet valtasi pelko.
\par 13 Mutta enkeli sanoi hänelle: "Älä pelkää, Sakarias; sillä sinun rukouksesi on kuultu, ja vaimosi Elisabet on synnyttävä sinulle pojan, ja sinun on annettava hänelle nimi Johannes.
\par 14 Ja hän on oleva sinulle iloksi ja riemuksi, ja monet iloitsevat hänen syntymisestään.
\par 15 Sillä hän on oleva suuri Herran edessä; viiniä ja väkijuomaa hän ei juo, ja hän on oleva täytetty Pyhällä Hengellä hamasta äitinsä kohdusta.
\par 16 Ja hän kääntää monta Israelin lapsista Herran, heidän Jumalansa, tykö.
\par 17 Ja hän käy hänen edellään Eliaan hengessä ja voimassa, kääntääksensä isien sydämet lasten puoleen ja tottelemattomat vanhurskasten mielenlaatuun, näin Herralle toimittaaksensa valmistetun kansan."
\par 18 Niin Sakarias sanoi enkelille: "Kuinka minä tämän käsittäisin? Sillä minä olen vanha, ja minun vaimoni on iälliseksi tullut."
\par 19 Enkeli vastasi ja sanoi hänelle: "Minä olen Gabriel, joka seison Jumalan edessä, ja minä olen lähetetty puhumaan sinulle ja julistamaan sinulle tämän ilosanoman.
\par 20 Ja katso, sinä tulet mykäksi etkä kykene mitään puhumaan siihen päivään saakka, jona tämä tapahtuu, sentähden ettet uskonut minun sanojani, jotka käyvät aikanansa toteen."
\par 21 Ja kansa oli odottamassa Sakariasta, ja he ihmettelivät, että hän niin kauan viipyi temppelissä.
\par 22 Mutta ulos tullessaan hän ei kyennyt puhumaan heille; silloin he ymmärsivät, että hän oli nähnyt näyn temppelissä. Ja hän viittoi heille ja jäi mykäksi.
\par 23 Ja kun hänen virkatoimensa päivät olivat päättyneet, meni hän kotiinsa.
\par 24 Ja niiden päiväin perästä Elisabet, hänen vaimonsa, tuli raskaaksi ja pysytteli salassa viisi kuukautta, sanoen:
\par 25 "Näin on Herra tehnyt minulle niinä päivinä, jolloin hän katsoi minun puoleeni poistaaksensa minusta ihmisten ylenkatseen".
\par 26 Kuudentena kuukautena sen jälkeen Jumala lähetti enkeli Gabrielin Galilean kaupunkiin, jonka nimi on Nasaret,
\par 27 neitsyen tykö, joka oli kihlattu Joosef nimiselle miehelle Daavidin suvusta; ja neitsyen nimi oli Maria.
\par 28 Ja tullessaan sisälle hänen tykönsä enkeli sanoi: "Terve, armoitettu! Herra olkoon sinun kanssasi."
\par 29 Mutta hän hämmästyi suuresti siitä puheesta ja mietti, mitä tämä tervehdys mahtoi tarkoittaa.
\par 30 Niin enkeli sanoi hänelle: "Älä pelkää, Maria; sillä sinä olet saanut armon Jumalan edessä.
\par 31 Ja katso, sinä tulet raskaaksi ja synnytät pojan, ja sinun on annettava hänelle nimi Jeesus.
\par 32 Hän on oleva suuri, ja hänet pitää kutsuttaman Korkeimman Pojaksi, ja Herra Jumala antaa hänelle Daavidin, hänen isänsä, valtaistuimen,
\par 33 ja hän on oleva Jaakobin huoneen kuningas iankaikkisesti, ja hänen valtakunnallansa ei pidä loppua oleman."
\par 34 Niin Maria sanoi enkelille: "Kuinka tämä voi tapahtua, kun minä en miehestä mitään tiedä?"
\par 35 Enkeli vastasi ja sanoi hänelle: "Pyhä Henki tulee sinun päällesi, ja Korkeimman voima varjoaa sinut; sentähden myös se pyhä, mikä syntyy, pitää kutsuttaman Jumalan Pojaksi.
\par 36 Ja katso, sinun sukulaisesi Elisabet, hänkin kantaa kohdussaan poikaa vanhalla iällään, ja tämä on kuudes kuukausi hänellä, jota sanottiin hedelmättömäksi;
\par 37 sillä Jumalalle ei mikään ole mahdotonta."
\par 38 Silloin Maria sanoi: "Katso, minä olen Herran palvelijatar; tapahtukoon minulle sinun sanasi mukaan". Ja enkeli lähti hänen tyköänsä.
\par 39 Niinä päivinä Maria nousi ja kulki kiiruusti vuorimaahan erääseen Juudan kaupunkiin
\par 40 ja meni Sakariaan kotiin ja tervehti Elisabetia.
\par 41 Ja kun Elisabet kuuli Marian tervehdyksen, hypähti lapsi hänen kohdussansa; ja Elisabet täytettiin Pyhällä Hengellä.
\par 42 Ja hän puhkesi puhumaan suurella äänellä ja sanoi: "Siunattu sinä vaimojen joukossa, ja siunattu sinun kohtusi hedelmä!
\par 43 Ja kuinka minulle tapahtuu tämä, että minun Herrani äiti tulee minun tyköni?
\par 44 Sillä katso, kun sinun tervehdyksesi ääni tuli minun korviini, hypähti lapsi ilosta minun kohdussani.
\par 45 Ja autuas se, joka uskoi, sillä se sana on täyttyvä, mikä hänelle on tullut Herralta!"
\par 46 Ja Maria sanoi: "Minun sieluni suuresti ylistää Herraa,
\par 47 ja minun henkeni riemuitsee Jumalasta, vapahtajastani;
\par 48 sillä hän on katsonut palvelijattarensa alhaisuuteen. Katso, tästedes kaikki sukupolvet ylistävät minua autuaaksi.
\par 49 Sillä Voimallinen on tehnyt minulle suuria, ja hänen nimensä on pyhä,
\par 50 ja hänen laupeutensa pysyy polvesta polveen niille, jotka häntä pelkäävät.
\par 51 Hän on osoittanut voimansa käsivarrellaan; hän on hajottanut ne, joilla oli ylpeät ajatukset sydämessään.
\par 52 Hän on kukistanut valtiaat valtaistuimilta ja korottanut alhaiset.
\par 53 Nälkäiset hän on täyttänyt hyvyyksillä, ja rikkaat hän on lähettänyt tyhjinä pois.
\par 54 Hän on ottanut huomaansa palvelijansa Israelin, muistaaksensa laupeuttaan
\par 55 Aabrahamia ja hänen siementänsä kohtaan iankaikkisesti, niinkuin hän on meidän isillemme puhunut."
\par 56 Ja Maria oli hänen tykönänsä noin kolme kuukautta ja palasi jälleen kotiinsa.
\par 57 Mutta Elisabetin synnyttämisen aika tuli; ja hän synnytti pojan.
\par 58 Ja kun hänen naapurinsa ja sukulaisensa kuulivat, että Herra oli tehnyt hänelle suuren laupeuden, iloitsivat he hänen kanssansa.
\par 59 Ja kahdeksantena päivänä he tulivat ympärileikkaamaan lasta ja tahtoivat antaa hänelle hänen isänsä mukaan nimen Sakarias.
\par 60 Mutta hänen äitinsä vastasi ja sanoi: "Ei suinkaan, vaan hänen nimensä on oleva Johannes".
\par 61 Niin he sanoivat hänelle: "Eihän sinun suvussasi ole ketään, jolla on se nimi".
\par 62 Ja he kysyivät viittomalla lapsen isältä, minkä nimen hän tahtoi hänelle annettavaksi.
\par 63 Niin hän pyysi taulun ja kirjoitti siihen nämä sanat: "Johannes on hänen nimensä". Ja kaikki ihmettelivät.
\par 64 Ja kohta hänen suunsa aukeni, ja hänen kielensä vapautui, ja hän puhui kiittäen Jumalaa.
\par 65 Ja tuli pelko kaikille heidän ympärillään asuvaisille, ja koko Juudean vuorimaassa puhuttiin kaikista näistä tapahtumista;
\par 66 ja kaikki, jotka niistä kuulivat, panivat ne mieleensä ja sanoivat: "Mikähän tästä lapsesta tulee?" Sillä Herran käsi oli hänen kanssansa.
\par 67 Ja Sakarias, hänen isänsä, täytettiin Pyhällä Hengellä, ja hän ennusti sanoen:
\par 68 "Kiitetty olkoon Herra, Israelin Jumala, sillä hän on katsonut kansansa puoleen ja valmistanut sille lunastuksen
\par 69 ja kohottanut meille pelastuksen sarven palvelijansa Daavidin huoneesta
\par 70 - niinkuin hän on puhunut hamasta ikiajoista pyhäin profeettainsa suun kautta -
\par 71 pelastukseksi vihollisistamme ja kaikkien niiden kädestä, jotka meitä vihaavat,
\par 72 tehdäkseen laupeuden meidän isillemme ja muistaakseen pyhän liittonsa,
\par 73 sen valan, jonka hän vannoi Aabrahamille, meidän isällemme;
\par 74 suodakseen meidän, vapahdettuina vihollistemme kädestä, pelkäämättä palvella häntä
\par 75 pyhyydessä ja vanhurskaudessa hänen edessään kaikkina elinpäivinämme.
\par 76 Ja sinä, lapsukainen, olet kutsuttava Korkeimman profeetaksi, sillä sinä olet käyvä Herran edellä valmistaaksesi hänen teitään,
\par 77 antaaksesi hänen kansalleen pelastuksen tuntemisen heidän syntiensä anteeksisaamisessa,
\par 78 meidän Jumalamme sydämellisen laupeuden tähden, jonka kautta meidän puoleemme katsoo aamun koitto korkeudesta,
\par 79 loistaen meille, jotka istumme pimeydessä ja kuoleman varjossa, ja ohjaten meidän jalkamme rauhan tielle."
\par 80 Ja lapsi kasvoi ja vahvistui hengessä. Ja hän oli erämaassa siihen päivään asti, jona hän oli astuva Israelin eteen.

\chapter{2}

\par 1 Ja tapahtui niinä päivinä, että keisari Augustukselta kävi käsky, että kaikki maailma oli verolle pantava.
\par 2 Tämä verollepano oli ensimmäinen ja tapahtui Kyreniuksen ollessa Syyrian maaherrana.
\par 3 Ja kaikki menivät verolle pantaviksi, kukin omaan kaupunkiinsa.
\par 4 Niin Joosefkin lähti Galileasta, Nasaretin kaupungista, ylös Juudeaan, Daavidin kaupunkiin, jonka nimi on Beetlehem, hän kun oli Daavidin huonetta ja sukua,
\par 5 verolle pantavaksi Marian, kihlattunsa, kanssa, joka oli raskaana.
\par 6 Niin tapahtui heidän siellä ollessaan, että Marian synnyttämisen aika tuli.
\par 7 Ja hän synnytti pojan, esikoisensa, ja kapaloi hänet ja pani hänet seimeen, koska heille ei ollut sijaa majatalossa.
\par 8 Ja sillä seudulla oli paimenia kedolla vartioimassa yöllä laumaansa.
\par 9 Niin heidän edessään seisoi Herran enkeli, ja Herran kirkkaus loisti heidän ympärillään, ja he peljästyivät suuresti.
\par 10 Mutta enkeli sanoi heille: "Älkää peljätkö; sillä katso, minä ilmoitan teille suuren ilon, joka on tuleva kaikelle kansalle:
\par 11 teille on tänä päivänä syntynyt Vapahtaja, joka on Kristus, Herra, Daavidin kaupungissa.
\par 12 Ja tämä on teille merkkinä: te löydätte lapsen kapaloituna ja seimessä makaamassa."
\par 13 Ja yhtäkkiä oli enkelin kanssa suuri joukko taivaallista sotaväkeä, ja he ylistivät Jumalaa ja sanoivat:
\par 14 "Kunnia Jumalalle korkeuksissa, ja maassa rauha ihmisten kesken, joita kohtaan hänellä on hyvä tahto!"
\par 15 Ja kun enkelit olivat menneet paimenten luota taivaaseen, niin nämä puhuivat toisillensa: "Menkäämme nyt Beetlehemiin katsomaan sitä, mikä on tapahtunut ja minkä Herra meille ilmoitti".
\par 16 Ja he menivät kiiruhtaen ja löysivät Marian ja Joosefin ja lapsen, joka makasi seimessä.
\par 17 Ja kun he tämän olivat nähneet, ilmoittivat he sen sanan, joka oli puhuttu heille tästä lapsesta.
\par 18 Ja kaikki, jotka sen kuulivat, ihmettelivät sitä, mitä paimenet heille puhuivat.
\par 19 Mutta Maria kätki kaikki nämä sanat ja tutkisteli niitä sydämessänsä.
\par 20 Ja paimenet palasivat kiittäen ja ylistäen Jumalaa kaikesta, minkä olivat kuulleet ja nähneet, sen mukaan kuin heille oli puhuttu.
\par 21 Kun sitten kahdeksan päivää oli kulunut ja lapsi oli ympärileikattava, annettiin hänelle nimi Jeesus, jonka enkeli oli hänelle antanut, ennenkuin hän sikisi äitinsä kohdussa.
\par 22 Ja kun heidän puhdistuspäivänsä, Mooseksen lain mukaan, olivat täyttyneet, veivät he hänet ylös Jerusalemiin asettaaksensa hänet Herran eteen
\par 23 - niinkuin on kirjoitettuna Herran laissa: "Jokainen miehenpuoli, joka avaa äidinkohdun, luettakoon Herralle pyhitetyksi" -
\par 24 ja uhrataksensa, niinkuin Herran laissa on säädetty, parin metsäkyyhkysiä tai kaksi kyyhkysenpoikaa.
\par 25 Ja katso, Jerusalemissa oli mies, nimeltä Simeon; hän oli hurskas ja jumalinen mies, joka odotti Israelin lohdutusta, ja Pyhä Henki oli hänen päällänsä.
\par 26 Ja Pyhä Henki oli hänelle ilmoittanut, ettei hän ollut näkevä kuolemaa, ennenkuin oli nähnyt Herran Voidellun.
\par 27 Ja hän tuli Hengen vaikutuksesta pyhäkköön. Ja kun vanhemmat toivat Jeesus-lasta sisälle tehdäkseen hänelle, niinkuin tapa oli lain mukaan,
\par 28 otti hänkin hänet syliinsä ja kiitti Jumalaa ja sanoi:
\par 29 "Herra, nyt sinä lasket palvelijasi rauhaan menemään, sanasi mukaan;
\par 30 sillä minun silmäni ovat nähneet sinun autuutesi,
\par 31 jonka sinä olet valmistanut kaikkien kansojen nähdä,
\par 32 valkeudeksi, joka on ilmestyvä pakanoille, ja kirkkaudeksi kansallesi Israelille".
\par 33 Ja hänen isänsä ja äitinsä ihmettelivät sitä, mitä hänestä sanottiin.
\par 34 Ja Simeon siunasi heitä ja sanoi Marialle, hänen äidilleen: "Katso, tämä on pantu lankeemukseksi ja nousemukseksi monelle Israelissa ja merkiksi, jota vastaan sanotaan
\par 35 - ja myös sinun sielusi lävitse on miekka käyvä - että monen sydämen ajatukset tulisivat ilmi".
\par 36 Ja oli naisprofeetta, Hanna, Fanuelin tytär, Asserin sukukuntaa. Hän oli jo tullut iälliseksi. Mentyään neitsyenä naimisiin hän oli elänyt miehensä kanssa seitsemän vuotta,
\par 37 ja oli nyt leski, kahdeksankymmenen neljän vuoden ikäinen. Hän ei poistunut pyhäköstä, vaan palveli siellä Jumalaa paastoilla ja rukouksilla yötä ja päivää.
\par 38 Ja juuri sillä hetkellä hän tuli siihen, ylisti Jumalaa ja puhui lapsesta kaikille, jotka odottivat Jerusalemin lunastusta.
\par 39 Ja täytettyään kaiken, mikä Herran lain mukaan oli tehtävä, he palasivat Galileaan, kaupunkiinsa Nasaretiin.
\par 40 Ja lapsi kasvoi ja vahvistui ja täyttyi viisaudella, ja Jumalan armo oli hänen päällänsä.
\par 41 Ja hänen vanhempansa matkustivat joka vuosi Jerusalemiin pääsiäisjuhlille.
\par 42 Hänen ollessaan kaksitoistavuotias he niinikään vaelsivat ylös sinne juhlan tavan mukaan.
\par 43 Ja kun ne päivät olivat kuluneet ja he lähtivät kotiin, jäi poikanen Jeesus Jerusalemiin, eivätkä hänen vanhempansa sitä huomanneet.
\par 44 He luulivat hänen olevan matkaseurueessa ja kulkivat päivänmatkan ja etsivät häntä sukulaisten ja tuttavien joukosta;
\par 45 mutta kun eivät löytäneet, palasivat he Jerusalemiin etsien häntä.
\par 46 Ja kolmen päivän kuluttua he löysivät hänet pyhäköstä, jossa hän istui opettajain keskellä kuunnellen heitä ja kysellen heiltä.
\par 47 Ja kaikki, jotka häntä kuulivat, ihmettelivät hänen ymmärrystänsä ja vastauksiansa.
\par 48 Ja hänet nähdessään hänen vanhempansa hämmästyivät, ja hänen äitinsä sanoi hänelle: "Poikani, miksi meille näin teit? Katso, sinun isäsi ja minä olemme huolestuneina etsineet sinua."
\par 49 Niin hän sanoi heille: "Mitä te minua etsitte? Ettekö tienneet, että minun pitää niissä oleman, mitkä minun Isäni ovat?"
\par 50 Mutta he eivät ymmärtäneet sitä sanaa, jonka hän heille puhui.
\par 51 Ja hän lähti heidän kanssansa ja tuli Nasaretiin ja oli heille alamainen. Ja hänen äitinsä kätki kaikki nämä sanat sydämeensä.
\par 52 Ja Jeesus varttui viisaudessa ja iässä ja armossa Jumalan ja ihmisten edessä.

\chapter{3}

\par 1 Viidentenätoista keisari Tiberiuksen hallitusvuotena, kun Pontius Pilatus oli Juudean maaherrana ja Herodes Galilean neljännysruhtinaana ja hänen veljensä Filippus Iturean ja Trakonitiinmaan neljännysruhtinaana ja Lysanias Abilenen neljännysruhtinaana,
\par 2 siihen aikaan kun Hannas oli ylimmäisenä pappina, ynnä myös Kaifas, tuli Jumalan sana Johannekselle, Sakariaan pojalle, erämaassa.
\par 3 Ja hän vaelsi kaikissa seuduissa Jordanin varrella ja saarnasi parannuksen kastetta syntien anteeksisaamiseksi,
\par 4 niinkuin on kirjoitettuna profeetta Esaiaan sanojen kirjassa: "Huutavan ääni kuuluu erämaassa: 'Valmistakaa Herralle tie, tehkää polut hänelle tasaisiksi'.
\par 5 Kaikki laaksot täytettäköön, ja kaikki vuoret ja kukkulat alennettakoon, ja mutkat tulkoot suoriksi ja koleikot tasaisiksi teiksi,
\par 6 ja kaikki liha on näkevä Jumalan autuuden."
\par 7 Niin hän sanoi kansalle, joka vaelsi hänen kastettavakseen: "Te kyykäärmeitten sikiöt, kuka on teitä neuvonut pakenemaan tulevaista vihaa?
\par 8 Tehkää sentähden parannuksen soveliaita hedelmiä, älkääkä ruvetko sanomaan mielessänne: 'Onhan meillä isänä Aabraham', sillä minä sanon teille, että Jumala voi näistä kivistä herättää lapsia Aabrahamille.
\par 9 Jo on myös kirves pantu puitten juurelle; jokainen puu, joka ei tee hyvää hedelmää, siis hakataan pois ja heitetään tuleen."
\par 10 Ja kansa kysyi häneltä sanoen: "Mitä meidän siis pitää tekemän?"
\par 11 Hän vastasi ja sanoi heille: "Jolla on kaksi ihokasta, antakoon toisen sille, joka on ilman; ja jolla on ruokaa, tehköön samoin".
\par 12 Niin tuli myös publikaaneja kastettaviksi, ja he sanoivat hänelle: "Opettaja, mitä meidän pitää tekemän?"
\par 13 Hän sanoi heille: "Älkää vaatiko enempää, kuin mikä teille on säädetty".
\par 14 Myös sotamiehet kysyivät häneltä sanoen: "Mitäs meidän pitää tekemän?" Ja hän sanoi heille: "Älkää kiskoko keneltäkään älkääkä kiristäkö, vaan tyytykää palkkaanne".
\par 15 Mutta kun kansa yhä odotti ja kaikki ajattelivat sydämessään Johanneksesta, eikö hän itse ehkä ollut Kristus,
\par 16 niin Johannes vastasi kaikille sanoen: "Minä kastan teidät vedellä, mutta on tuleva minua väkevämpi, jonka kengänpaulaakaan minä en ole kelvollinen päästämään; hän kastaa teidät Pyhällä Hengellä ja tulella.
\par 17 Hänellä on viskimensä kädessään, ja hän puhdistaa puimatanterensa ja kokoaa nisut aittaansa, mutta ruumenet hän polttaa sammumattomassa tulessa."
\par 18 Antaen myös monia muita kehoituksia hän julisti kansalle evankeliumia.
\par 19 Mutta kun Herodes, neljännysruhtinas, sai häneltä nuhteita veljensä vaimon, Herodiaan, tähden ja kaiken sen pahan tähden, mitä hän, Herodes, oli tehnyt,
\par 20 niin hän kaiken muun lisäksi teki senkin, että sulki Johanneksen vankeuteen.
\par 21 Kun siis kaikkea kansaa kastettiin ja myöskin Jeesus oli saanut kasteen ja rukoili, niin tapahtui, että taivas aukeni
\par 22 ja Pyhä Henki laskeutui hänen päällensä ruumiillisessa muodossa, niinkuin kyyhkynen, ja taivaasta tuli ääni: "Sinä olet minun rakas Poikani; sinuun minä olen mielistynyt".
\par 23 Ja hän, Jeesus, oli alottaessaan vaikutuksensa noin kolmenkymmenen vuoden vanha, ja oli, niinkuin luultiin, Joosefin poika. Joosef oli Eelin poika,
\par 24 Eeli Mattatin, tämä Leevin, tämä Melkin, tämä Jannain, tämä Joosefin,
\par 25 tämä Mattatiaan, tämä Aamoksen, tämä Naahumin, tämä Eslin, tämä Naggain,
\par 26 tämä Maahatin, tämä Mattatiaan, tämä Semeinin, tämä Joosekin, tämä Joodan,
\par 27 tämä Johananin, tämä Reesan, tämä Serubbaabelin, tämä Sealtielin, tämä Neerin,
\par 28 tämä Melkin, tämä Addin, tämä Koosamin, tämä Elmadamin, tämä Eerin,
\par 29 tämä Jeesuksen, tämä Elieserin, tämä Joorimin, tämä Mattatin, tämä Leevin,
\par 30 tämä Simeonin, tämä Juudan, tämä Joosefin, tämä Joonamin, tämä Eliakimin,
\par 31 tämä Melean, tämä Mennan, tämä Mattatan, tämä Naatanin, tämä Daavidin,
\par 32 tämä Iisain, tämä Oobedin, tämä Booaan, tämä Saalan, tämä Nahassonin,
\par 33 tämä Aminadabin, tämä Adminin, tämä Arnin, tämä Esromin, tämä Faareen, tämä Juudan,
\par 34 tämä Jaakobin, tämä Iisakin, tämä Aabrahamin, tämä Taaran, tämä Naahorin,
\par 35 tämä Serukin, tämä Ragaun, tämä Faalekin, tämä Eberin, tämä Saalan,
\par 36 tämä Kainamin, tämä Arfaksadin, tämä Seemin, tämä Nooan, tämä Laamekin,
\par 37 tämä Metusalan, tämä Eenokin, tämä Jaaretin, tämä Mahalalelin, tämä Keenanin,
\par 38 tämä Enoksen, tämä Seetin, tämä Aadamin, tämä Jumalan.

\chapter{4}

\par 1 Sitten Jeesus täynnä Pyhää Henkeä palasi Jordanilta; ja Henki kuljetti häntä erämaassa,
\par 2 ja perkele kiusasi häntä neljäkymmentä päivää. Eikä hän syönyt mitään niinä päivinä, mutta kun ne olivat päättyneet, tuli hänen nälkä.
\par 3 Niin perkele sanoi hänelle: "Jos sinä olet Jumalan Poika, niin sano tälle kivelle, että se muuttuu leiväksi".
\par 4 Jeesus vastasi hänelle: "Kirjoitettu on: 'Ei ihminen elä ainoastaan leivästä'."
\par 5 Ja perkele vei hänet korkealle vuorelle ja näytti hänelle yhdessä tuokiossa kaikki maailman valtakunnat
\par 6 ja sanoi hänelle: "Sinulle minä annan kaiken tämän valtapiirin ja sen loiston, sillä minun haltuuni se on annettu, ja minä annan sen, kenelle tahdon.
\par 7 Jos sinä siis kumarrut minun eteeni, niin tämä kaikki on oleva sinun."
\par 8 Jeesus vastasi ja sanoi hänelle: "Kirjoitettu on: 'Sinun pitää kumartaman Herraa, sinun Jumalaasi, ja häntä ainoata palveleman'."
\par 9 Niin hän vei hänet Jerusalemiin ja asetti hänet pyhäkön harjalle ja sanoi hänelle: "Jos sinä olet Jumalan Poika, niin heittäydy tästä alas;
\par 10 sillä kirjoitettu on: 'Hän antaa enkeleilleen käskyn sinusta, että he varjelevat sinua',
\par 11 ja: 'He kantavat sinua käsillänsä, ettet jalkaasi kiveen loukkaisi'."
\par 12 Jeesus vastasi ja sanoi hänelle: "Sanottu on: 'Älä kiusaa Herraa, sinun Jumalaasi'."
\par 13 Ja kun oli kaiken kiusattavansa kiusannut, poistui perkele hänen luotaan ajaksi.
\par 14 Ja Jeesus palasi Hengen voimassa Galileaan; ja sanoma hänestä levisi kaikkiin ympärillä oleviin seutuihin.
\par 15 Ja hän opetti heidän synagoogissaan, ja kaikki ylistivät häntä.
\par 16 Ja hän saapui Nasaretiin, jossa hänet oli kasvatettu, ja meni tapansa mukaan sapatinpäivänä synagoogaan ja nousi lukemaan.
\par 17 Niin hänelle annettiin profeetta Esaiaan kirja, ja kun hän avasi kirjan, löysi hän sen paikan, jossa oli kirjoitettuna:
\par 18 "Herran Henki on minun päälläni, sillä hän on voidellut minut julistamaan evankeliumia köyhille; hän on lähettänyt minut saarnaamaan vangituille vapautusta ja sokeille näkönsä saamista, päästämään sorretut vapauteen,
\par 19 saarnaamaan Herran otollista vuotta".
\par 20 Ja käärittyään kirjan kokoon hän antoi sen palvelijalle ja istuutui; ja kaikkien synagoogassa olevien silmät olivat häneen kiinnitetyt.
\par 21 Niin hän rupesi puhumaan heille: "Tänä päivänä tämä kirjoitus on käynyt toteen teidän korvainne kuullen".
\par 22 Ja kaikki lausuivat hänestä hyvän todistuksen ja ihmettelivät niitä armon sanoja, jotka hänen suustansa lähtivät; ja he sanoivat: "Eikö tämä ole Joosefin poika?"
\par 23 Niin hän sanoi heille: "Kaiketi aiotte sanoa minulle tämän sananlaskun: 'Parantaja, paranna itsesi'; 'tee täälläkin, kotikaupungissasi, niitä suuria tekoja, joita olemme kuulleet tapahtuneen Kapernaumissa'."
\par 24 Ja hän sanoi: "Totisesti minä sanon teille: ei kukaan profeetta ole otollinen kotikaupungissaan.
\par 25 Minä sanon teille totuudessa: monta leskeä oli Eliaan aikana Israelissa, kun taivas oli suljettuna kolme vuotta ja kuusi kuukautta ja suuri nälkä tuli kaikkeen maahan,
\par 26 eikä Eliasta lähetetty kenenkään tykö heistä, vaan ainoastaan leskivaimon tykö Siidonin-maan Sareptaan.
\par 27 Ja monta pitalista miestä oli Israelissa profeetta Elisan aikana, eikä kukaan heistä tullut puhdistetuksi, vaan ainoastaan Naiman, syyrialainen."
\par 28 Tämän kuullessaan kaikki, jotka olivat synagoogassa, tulivat kiukkua täyteen
\par 29 ja nousivat ja ajoivat hänet ulos kaupungista ja veivät hänet sen vuoren jyrkänteelle asti, jolle heidän kaupunkinsa oli rakennettu, syöstäkseen hänet alas.
\par 30 Mutta hän lähti pois käyden heidän keskitsensä.
\par 31 Ja hän meni alas Kapernaumiin, Galilean kaupunkiin, ja opetti kansaa sapattina.
\par 32 Ja he olivat hämmästyksissään hänen opetuksestansa, sillä hänen puheessansa oli voima.
\par 33 Ja synagoogassa oli mies, jossa oli saastaisen riivaajan henki. Tämä huusi suurella äänellä:
\par 34 "Voi, mitä sinulla on meidän kanssamme tekemistä, Jeesus Nasaretilainen? Oletko tullut meitä tuhoamaan? Minä tunnen sinut, kuka olet, sinä Jumalan Pyhä."
\par 35 Niin Jeesus nuhteli häntä sanoen: "Vaikene ja lähde hänestä". Ja riivaaja viskasi hänet maahan heidän keskelleen ja lähti hänestä, häntä ollenkaan vahingoittamatta.
\par 36 Ja heidät kaikki valtasi hämmästys, ja he puhuivat keskenään sanoen: "Mitä tämä puhe on, sillä hän käskee vallalla ja voimalla saastaisia henkiä, ja ne lähtevät ulos?"
\par 37 Ja maine hänestä levisi kaikkialle ympäristön seutuihin.
\par 38 Niin hän nousi ja meni synagoogasta Simonin kotiin. Ja Simonin anoppi sairasti kovaa kuumetta, ja he rukoilivat Jeesusta hänen puolestansa.
\par 39 Niin hän meni ja kumartui hänen ylitsensä ja nuhteli kuumetta, ja se lähti hänestä; ja heti vaimo nousi ja palveli heitä.
\par 40 Auringon laskiessa kaikki, joilla oli sairaita, mikä missäkin taudissa, veivät ne hänen tykönsä. Ja hän pani kätensä heidän itsekunkin päälle ja paransi heidät.
\par 41 Myös lähtivät riivaajat ulos monesta, huutaen ja sanoen: "Sinä olet Jumalan Poika!" Mutta hän nuhteli niitä eikä sallinut niiden puhua, koska ne tiesivät hänen olevan Kristuksen.
\par 42 Ja päivän tultua hän lähti pois ja meni autioon paikkaan; mutta kansa etsi häntä, ja saavuttuaan hänen luokseen he pidättelivät häntä lähtemästä heidän luotansa.
\par 43 Mutta hän sanoi heille: "Minun tulee muillekin kaupungeille julistaa Jumalan valtakunnan evankeliumia, sillä sitä varten minä olen lähetetty".
\par 44 Ja hän saarnasi Galilean synagoogissa.

\chapter{5}

\par 1 Kun kansa tunkeutui hänen ympärilleen kuulemaan Jumalan sanaa ja hän seisoi Gennesaretin järven rannalla,
\par 2 niin hän näki järven rannassa kaksi venhettä; mutta kalastajat olivat niistä lähteneet ja huuhtoivat verkkojaan.
\par 3 Ja hän astui toiseen niistä, joka oli Simonin, ja pyysi häntä viemään sen vähän matkaa maasta; ja hän istui ja opetti kansaa venheestä.
\par 4 Mutta puhumasta lakattuaan hän sanoi Simonille: "Vie venhe syvälle ja heittäkää verkkonne apajalle".
\par 5 Niin Simon vastasi ja sanoi hänelle: "Mestari, koko yön me olemme tehneet työtä emmekä ole mitään saaneet; mutta sinun käskystäsi minä heitän verkot".
\par 6 Ja sen tehtyään he saivat kierretyksi suuren joukon kaloja, ja heidän verkkonsa repeilivät.
\par 7 Niin he viittasivat toisessa venheessä oleville tovereilleen, että nämä tulisivat auttamaan heitä; ja he tulivat. Ja he täyttivät molemmat venheet, niin että ne olivat uppoamaisillaan.
\par 8 Kun Simon Pietari sen näki, lankesi hän Jeesuksen polvien eteen ja sanoi: "Mene pois minun tyköäni, Herra, sillä minä olen syntinen ihminen".
\par 9 Sillä kalansaaliin tähden, jonka he olivat saaneet, oli hämmästys vallannut hänet ja kaikki ne, jotka olivat hänen kanssaan,
\par 10 ja samoin myös Simonin kalastuskumppanit, Jaakobin ja Johanneksen, Sebedeuksen pojat. Mutta Jeesus sanoi Simonille: "Älä pelkää, tästedes sinä saat saaliiksi ihmisiä".
\par 11 Ja he veivät venheet maihin, jättivät kaikki ja seurasivat häntä.
\par 12 Ja kun hän oli eräässä kaupungissa, niin katso, siellä oli mies, yltänsä pitalissa. Ja nähdessään Jeesuksen hän lankesi kasvoilleen ja rukoili häntä sanoen: "Herra, jos tahdot, niin sinä voit minut puhdistaa".
\par 13 Niin hän ojensi kätensä, kosketti häntä ja sanoi: "Minä tahdon; puhdistu". Ja kohta pitali lähti hänestä.
\par 14 Ja hän kielsi häntä siitä kenellekään puhumasta ja sanoi: "Mene, näytä itsesi papille, ja uhraa puhdistumisestasi, niinkuin Mooses on säätänyt, todistukseksi heille".
\par 15 Mutta sanoma hänestä levisi vielä enemmän; ja paljon kansaa kokoontui kuulemaan häntä ja parantuakseen vaivoistansa.
\par 16 Mutta hän vetäytyi pois ja oleskeli erämaassa ja rukoili.
\par 17 Ja eräänä päivänä, kun hän opetti, istui siinä fariseuksia ja lainopettajia, joita oli tullut kaikista Galilean ja Juudean kylistä ja Jerusalemista; ja Herran voima vaikutti, niin että hän paransi sairaat.
\par 18 Ja katso, muutamat miehet kantoivat vuoteella miestä, joka oli halvattu; ja he koettivat viedä hänet sisään ja asettaa Jeesuksen eteen.
\par 19 Ja kun he väentungokselta eivät saaneet viedyksi häntä sisään muuta tietä, nousivat he katolle ja laskivat hänet vuoteineen tiilikaton läpi heidän keskellensä Jeesuksen eteen.
\par 20 Ja nähdessään heidän uskonsa hän sanoi: "Ihminen, sinun syntisi ovat sinulle anteeksi annetut".
\par 21 Niin kirjanoppineet ja fariseukset rupesivat ajattelemaan ja sanomaan: "Kuka tämä on, joka puhuu Jumalan pilkkaa? Kuka voi antaa syntejä anteeksi, paitsi Jumala yksin?"
\par 22 Mutta kun Jeesus tiesi heidän ajatuksensa, vastasi hän ja sanoi heille: "Mitä te ajattelette sydämessänne?
\par 23 Kumpi on helpompaa, sanoako: 'Sinun syntisi ovat sinulle anteeksi annetut', vai sanoa: 'Nouse ja käy'?
\par 24 Mutta tietääksenne, että Ihmisen Pojalla on valta maan päällä antaa syntejä anteeksi," - hän sanoi halvatulle - "minä sanon sinulle: nouse, ota vuoteesi ja mene kotiisi."
\par 25 Ja kohta hän nousi heidän nähtensä, otti vuoteen, jolla oli maannut, ja lähti kotiinsa ylistäen Jumalaa.
\par 26 Ja heidät kaikki valtasi hämmästys, ja he ylistivät Jumalaa; ja pelkoa täynnä he sanoivat: "Me olemme tänään nähneet ihmeellisiä".
\par 27 Ja sen jälkeen hän lähti sieltä ja näki tulliasemalla istumassa publikaanin, jonka nimi oli Leevi, ja sanoi hänelle: "Seuraa minua".
\par 28 Niin tämä jätti kaikki, nousi ja seurasi häntä.
\par 29 Ja Leevi valmisti hänelle suuret pidot kodissaan; ja siellä oli suuri joukko publikaaneja ja muita aterioimassa heidän kanssaan.
\par 30 Niin fariseukset ja heidän kirjanoppineensa napisivat hänen opetuslapsiansa vastaan ja sanoivat: "Miksi te syötte ja juotte publikaanien ja syntisten kanssa?"
\par 31 Jeesus vastasi ja sanoi heille: "Eivät terveet tarvitse parantajaa, vaan sairaat.
\par 32 En minä ole tullut kutsumaan vanhurskaita, vaan syntisiä parannukseen."
\par 33 Niin he sanoivat hänelle: "Johanneksen opetuslapset paastoavat usein ja pitävät rukouksia, samoin fariseustenkin, mutta sinun opetuslapsesi syövät ja juovat".
\par 34 Jeesus sanoi heille: "Ettehän voi vaatia häävieraita paastoamaan silloin, kun ylkä on heidän kanssaan?
\par 35 Mutta päivät tulevat, jolloin ylkä otetaan heiltä pois, ja silloin, niinä päivinä, he paastoavat."
\par 36 Ja hän sanoi heille myös vertauksen: "Ei kukaan leikkaa uudesta vaipasta paikkaa ja pane vanhaan vaippaan; muutoin hän rikkoo uuden vaipan, eikä uudesta vaipasta otettu paikka vanhaan sovi.
\par 37 Eikä kukaan laske nuorta viiniä vanhoihin nahkaleileihin; muutoin nuori viini pakahuttaa leilit, ja viini juoksee maahan, ja leilit turmeltuvat.
\par 38 Vaan nuori viini on laskettava uusiin leileihin.
\par 39 Eikä kukaan, joka juo vanhaa viiniä, halua nuorta, sillä hän sanoo: 'Vanha on hyvää'."

\chapter{6}

\par 1 Niin tapahtui eräänä sapattina, että hän kulki viljavainioiden halki, ja hänen opetuslapsensa katkoivat tähkäpäitä, hiersivät niitä käsissään ja söivät.
\par 2 Silloin muutamat fariseuksista sanoivat: "Miksi teette, mitä ei ole lupa tehdä sapattina?"
\par 3 Mutta Jeesus vastasi heille ja sanoi: "Ettekö ole lukeneet, mitä Daavid teki, kun hänen ja hänen seuralaistensa oli nälkä,
\par 4 kuinka hän meni Jumalan huoneeseen, otti näkyleivät ja söi ja antoi seuralaisilleenkin, vaikkei niitä ollut lupa syödä muiden kuin ainoastaan pappien?"
\par 5 Ja hän sanoi heille: "Ihmisen Poika on sapatin herra".
\par 6 Ja toisena sapattina hän meni synagoogaan ja opetti; ja siellä oli mies, jonka oikea käsi oli kuivettunut.
\par 7 Ja keksiäkseen jotakin, mistä häntä syyttäisivät, kirjanoppineet ja fariseukset pitivät häntä silmällä, parantaisiko hän sapattina.
\par 8 Mutta hän tiesi heidän ajatuksensa ja sanoi miehelle, jonka käsi oli kuivettunut: "Nouse ja astu esille". Ja hän nousi ja astui esille.
\par 9 Niin Jeesus sanoi heille: "Minä kysyn teiltä: kumpi on luvallista sapattina: hyvääkö tehdä, vai tehdä pahaa, pelastaako henki, vai hukuttaa?"
\par 10 Ja hän katsoi ympärilleen heihin kaikkiin ja sanoi miehelle: "Ojenna kätesi". Mies teki niin, ja hänen kätensä tuli jälleen terveeksi.
\par 11 Mutta he vimmastuivat kovin ja puhelivat keskenään, mitä heidän olisi tehtävä Jeesukselle.
\par 12 Niin tapahtui niinä päivinä, että hän lähti vuorelle rukoilemaan; ja hän oli siellä kaiken yötä rukoillen Jumalaa.
\par 13 Ja päivän tultua hän kutsui tykönsä opetuslapsensa ja valitsi heistä kaksitoista, joille hän myös antoi apostolin nimen:
\par 14 Simonin, jolle hän myös antoi nimen Pietari, ja Andreaan, hänen veljensä, ja Jaakobin ja Johanneksen, ja Filippuksen ja Bartolomeuksen,
\par 15 ja Matteuksen ja Tuomaan, ja Jaakobin, Alfeuksen pojan, ja Simonin, jota kutsuttiin Kiivailijaksi,
\par 16 ja Juudaan, Jaakobin pojan, sekä Juudas Iskariotin, josta tuli kavaltaja.
\par 17 Ja hän astui alas heidän kanssaan ja seisahtui lakealle paikalle; ja siellä oli suuri joukko hänen opetuslapsiaan ja paljon kansaa kaikesta Juudeasta ja Jerusalemista ja Tyyron ja Siidonin rantamaasta. Nämä olivat saapuneet kuulemaan häntä ja parantuakseen taudeistansa.
\par 18 Ja myös ne, jotka olivat saastaisten henkien vaivaamia, tulivat terveiksi.
\par 19 Ja kaikki kansa tahtoi päästä koskettamaan häntä, koska hänestä lähti voima, joka paransi kaikki.
\par 20 Ja hän nosti silmänsä opetuslastensa puoleen ja sanoi: "Autuaita olette te, köyhät, sillä teidän on Jumalan valtakunta.
\par 21 Autuaita te, jotka nyt isoatte, sillä teidät ravitaan! Autuaita te, jotka nyt itkette, sillä te saatte nauraa!
\par 22 Autuaita olette te, kun ihmiset vihaavat teitä ja erottavat teidät yhteydestään ja herjaavat teitä ja pyyhkivät pois teidän nimenne ikäänkuin jonkin pahan - Ihmisen Pojan tähden.
\par 23 Iloitkaa sinä päivänä, riemuun ratketkaa; sillä katso, teidän palkkanne on suuri taivaassa; sillä näin tekivät heidän isänsä profeetoille.
\par 24 Mutta voi teitä, te rikkaat, sillä te olette jo saaneet lohdutuksenne!
\par 25 Voi teitä, jotka nyt olette kylläiset, sillä teidän on oleva nälkä! Voi teitä, jotka nyt nauratte, sillä te saatte murehtia ja itkeä!
\par 26 Voi teitä, kun kaikki ihmiset puhuvat teistä hyvää! Sillä niin tekivät heidän isänsä väärille profeetoille.
\par 27 Mutta teille, jotka kuulette, minä sanon: rakastakaa vihollisianne, tehkää hyvää niille, jotka teitä vihaavat,
\par 28 siunatkaa niitä, jotka teitä kiroavat, rukoilkaa niiden edestä, jotka teitä parjaavat.
\par 29 Jos joku lyö sinua poskelle, tarjoa hänelle toinenkin, ja jos joku ottaa sinulta vaipan, älä häneltä kiellä ihokastasikaan.
\par 30 Anna jokaiselle, joka sinulta anoo, äläkä vaadi takaisin siltä, joka sinun omaasi ottaa.
\par 31 Ja niinkuin te tahdotte ihmisten teille tekevän, niin tehkää tekin heille.
\par 32 Ja jos te rakastatte niitä, jotka teitä rakastavat, mitä kiitosta teille siitä tulee? Rakastavathan syntisetkin niitä, jotka heitä rakastavat.
\par 33 Ja jos teette hyvää niille, jotka teille hyvää tekevät, mitä kiitosta teille siitä tulee? Niinhän syntisetkin tekevät.
\par 34 Ja jos te lainaatte niille, joilta toivotte saavanne takaisin, mitä kiitosta teille siitä tulee? Syntisetkin lainaavat syntisille saadakseen saman verran takaisin.
\par 35 Vaan rakastakaa vihollisianne ja tehkää hyvää ja lainatkaa, toivomatta saavanne mitään takaisin; niin teidän palkkanne on oleva suuri, ja te tulette Korkeimman lapsiksi, sillä hän on hyvä kiittämättömille ja pahoille.
\par 36 Olkaa armahtavaiset, niinkuin teidän Isänne on armahtavainen.
\par 37 Älkääkä tuomiko, niin ei teitäkään tuomita; älkää kadotustuomiota lausuko, niin ei teillekään kadotustuomiota lausuta. Antakaa anteeksi, niin teillekin anteeksi annetaan.
\par 38 Antakaa, niin teille annetaan. Hyvä mitta, sullottu, pudistettu ja kukkurainen, annetaan teidän helmaanne; sillä millä mitalla te mittaatte, sillä mitataan teille takaisin."
\par 39 Hän sanoi heille myös vertauksen: "Eihän sokea voi sokeaa taluttaa? Eivätkö molemmat lankea kuoppaan?
\par 40 Ei ole opetuslapsi opettajaansa parempi; täysin oppineena jokainen on oleva niinkuin hänen opettajansa.
\par 41 Kuinka näet rikan, joka on veljesi silmässä, mutta et huomaa malkaa omassa silmässäsi?
\par 42 Kuinka saatat sanoa veljellesi: 'Veljeni, annas, minä otan pois rikan, joka on silmässäsi', sinä, joka et näe malkaa omassa silmässäsi? Sinä ulkokullattu, ota ensin malka omasta silmästäsi, sitten sinä näet ottaa pois rikan, joka on veljesi silmässä.
\par 43 Sillä ei ole hyvää puuta, joka tekee huonon hedelmän, eikä taas huonoa puuta, joka tekee hyvän hedelmän;
\par 44 sillä jokainen puu tunnetaan hedelmästään. Eihän viikunoita koota orjantappuroista, eikä viinirypäleitä korjata orjanruusupensaasta.
\par 45 Hyvä ihminen tuo sydämensä hyvän runsaudesta esiin hyvää, ja paha tuo pahastansa esiin pahaa; sillä sydämen kyllyydestä suu puhuu.
\par 46 Miksi te huudatte minulle: 'Herra, Herra!' ettekä tee, mitä minä sanon?
\par 47 Jokainen, joka tulee minun tyköni ja kuulee minun sanani ja tekee niiden mukaan - minä osoitan teille, kenen kaltainen hän on.
\par 48 Hän on miehen kaltainen, joka huonetta rakentaessaan kaivoi syvään ja laski perustuksen kalliolle; kun sitten tulva tuli, syöksähti virta sitä huonetta vastaan, mutta ei voinut sitä horjuttaa, sillä se oli hyvästi rakennettu.
\par 49 Mutta joka kuulee eikä tee, se on miehen kaltainen, joka perustusta panematta rakensi huoneensa maan pinnalle; ja virta syöksähti sitä vastaan, ja heti se sortui, ja sen huoneen kukistuminen oli suuri."

\chapter{7}

\par 1 Kun hän oli kansan kuullen kaikki nämä puheensa puhunut, meni hän Kapernaumiin.
\par 2 Ja eräällä sadanpäämiehellä oli palvelija, joka sairasti ja oli kuolemaisillaan ja jota hän piti suuressa arvossa.
\par 3 Ja kuultuaan Jeesuksesta hän lähetti juutalaisten vanhimpia hänen tykönsä ja pyysi, että hän tulisi parantamaan hänen palvelijansa.
\par 4 Kun nämä saapuivat Jeesuksen tykö, pyysivät he häntä hartaasti ja sanoivat: "Hän ansaitsee, että teet hänelle tämän;
\par 5 sillä hän rakastaa meidän kansaamme, ja hän on rakentanut meille synagoogan".
\par 6 Niin Jeesus lähti heidän kanssansa. Mutta kun hän ei enää ollut kaukana talosta, lähetti sadanpäämies ystäviänsä sanomaan hänelle: "Herra, älä vaivaa itseäsi, sillä en minä ole sen arvoinen, että tulisit minun kattoni alle;
\par 7 sentähden en katsonutkaan itseäni arvolliseksi tulemaan sinun luoksesi; vaan sano sana, niin minun palvelijani paranee.
\par 8 Sillä minä itsekin olen toisen vallan alaiseksi asetettu, ja minulla on sotamiehiä käskettävinäni, ja minä sanon tälle: 'Mene', ja hän menee, ja toiselle: 'Tule', ja hän tulee, ja palvelijalleni: 'Tee tämä', ja hän tekee."
\par 9 Tämän kuultuaan Jeesus ihmetteli häntä, kääntyi ja sanoi kansalle, joka häntä seurasi: "Minä sanon teille: en ole Israelissakaan löytänyt näin suurta uskoa."
\par 10 Ja taloon palatessaan lähettiläät tapasivat palvelijan terveenä.
\par 11 Sen jälkeen hän vaelsi Nain nimiseen kaupunkiin, ja hänen kanssaan vaelsivat hänen opetuslapsensa ynnä suuri kansanjoukko.
\par 12 Kun hän nyt lähestyi kaupungin porttia, katso, silloin kannettiin ulos kuollutta, äitinsä ainokaista poikaa. Ja äiti oli leski, ja hänen kanssaan kulki paljon kaupungin kansaa.
\par 13 Ja hänet nähdessään Herra armahti häntä ja sanoi hänelle: "Älä itke".
\par 14 Ja hän meni ja kosketti paareja; niin kantajat seisahtuivat. Ja hän sanoi: "Nuorukainen, minä sanon sinulle: nouse."
\par 15 Niin kuollut nousi istualleen ja rupesi puhumaan. Ja hän antoi hänet hänen äidillensä.
\par 16 Ja heidät kaikki valtasi pelko, ja he ylistivät Jumalaa sanoen: "Suuri profeetta on noussut meidän keskellemme", ja: "Jumala on katsonut kansansa puoleen".
\par 17 Ja tämä puhe hänestä levisi koko Juudeaan ja kaikkiin ympärillä oleviin seutuihin.
\par 18 Ja Johannekselle kertoivat hänen opetuslapsensa tästä kaikesta.
\par 19 Niin Johannes kutsui luoksensa opetuslapsistaan kaksi ja lähetti heidät Herran tykö kysymään: "Oletko sinä se tuleva, vai pitääkö meidän toista odottaman?"
\par 20 Miehet saapuivat hänen tykönsä ja sanoivat: "Johannes Kastaja on lähettänyt meidät sinun tykösi ja kysyy: 'Oletko sinä se tuleva, vai pitääkö meidän toista odottaman?'"
\par 21 Sillä hetkellä hän juuri paransi useita taudeista ja vitsauksista ja pahoista hengistä, ja monelle sokealle hän antoi näön.
\par 22 Niin hän vastasi ja sanoi heille: "Menkää ja kertokaa Johannekselle, mitä olette nähneet ja kuulleet: sokeat saavat näkönsä, rammat kävelevät, pitaliset puhdistuvat, kuurot kuulevat, kuolleet herätetään, köyhille julistetaan evankeliumia.
\par 23 Ja autuas on se, joka ei loukkaannu minuun."
\par 24 Kun Johanneksen lähettiläät olivat menneet, rupesi hän puhumaan kansalle Johanneksesta: "Mitä te lähditte erämaahan katselemaan? Ruokoako, jota tuuli huojuttaa?
\par 25 Vai mitä lähditte katsomaan? Ihmistäkö, hienoihin vaatteisiin puettua? Katso, ne jotka koreissa vaatteissa käyvät ja herkullisesti elävät, ne ovat kuningasten linnoissa.
\par 26 Vai mitä lähditte katsomaan? Profeettaako? Totisesti, minä sanon teille: hän on enemmän kuin profeetta.
\par 27 Tämä on se, josta on kirjoitettu: 'Katso, minä lähetän enkelini sinun edelläsi, ja hän on valmistava tiesi sinun eteesi'.
\par 28 Minä sanon teille: ei ole vaimoista syntyneitten joukossa yhtäkään suurempaa kuin Johannes; mutta vähäisin Jumalan valtakunnassa on suurempi kuin hän.
\par 29 Ja kaikki kansa, joka häntä kuuli, publikaanitkin, tunnustivat Jumalan vanhurskaaksi ja antoivat kastaa itsensä Johanneksen kasteella.
\par 30 Mutta fariseukset ja lainoppineet tekivät turhaksi Jumalan aivoituksen heitä kohtaan eivätkä ottaneet Johannekselta kastetta.
\par 31 Mihin minä siis vertaan tämän sukupolven ihmiset, ja kenen kaltaisia he ovat?
\par 32 He ovat lasten kaltaisia, jotka istuvat torilla ja huutavat toisilleen ja sanovat: 'Me soitimme teille huilua, ja te ette karkeloineet; me veisasimme itkuvirsiä, ja te ette itkeneet'.
\par 33 Sillä Johannes Kastaja on tullut, ei syö leipää eikä juo viiniä, ja te sanotte: 'Hänessä on riivaaja'.
\par 34 Ihmisen Poika on tullut, syö ja juo, ja te sanotte: 'Katso syömäriä ja viininjuojaa, publikaanien ja syntisten ystävää!'
\par 35 Ja viisaus on kaikkien lastensa puolelta oikeaksi näytetty."
\par 36 Niin eräs fariseuksista pyysi häntä ruualle kanssaan; ja hän meni fariseuksen taloon ja asettui aterialle.
\par 37 Ja katso, siinä kaupungissa oli nainen, joka eli syntisesti; ja kun hän sai tietää, että Jeesus oli aterialla fariseuksen talossa, toi hän alabasteripullon täynnä hajuvoidetta
\par 38 ja asettui hänen taakseen hänen jalkojensa kohdalle, itki ja rupesi kastelemaan hänen jalkojansa kyynelillään ja kuivasi ne päänsä hiuksilla ja suuteli hänen jalkojaan ja voiteli ne hajuvoiteella.
\par 39 Mutta kun fariseus, joka oli hänet kutsunut, sen näki, ajatteli hän mielessään näin: "Jos tämä olisi profeetta, tietäisi hän, mikä ja millainen tuo nainen on, joka häneen koskee: että hän on syntinen."
\par 40 Niin Jeesus vastasi ja sanoi hänelle: "Simon, minulla on jotakin sanomista sinulle". Hän virkkoi: "Opettaja, sano". -
\par 41 "Lainanantajalla oli kaksi velallista; toinen oli velkaa viisisataa denaria, toinen viisikymmentä.
\par 42 Kun heillä ei ollut, millä maksaa, antoi hän molemmille velan anteeksi. Kumpi heistä siis rakastaa häntä enemmän?"
\par 43 Simon vastasi ja sanoi: "Minun mielestäni se, jolle hän antoi enemmän anteeksi". Hän sanoi hänelle: "Oikein sinä ratkaisit".
\par 44 Ja naiseen kääntyen hän sanoi Simonille: "Näetkö tämän naisen? Minä tulin sinun taloosi; et sinä antanut vettä minun jaloilleni, mutta tämä kasteli kyynelillään minun jalkani ja kuivasi ne hiuksillaan.
\par 45 Et sinä antanut minulle suudelmaa, mutta tämä ei ole lakannut suutelemasta minun jalkojani siitä asti, kuin tulin sisään.
\par 46 Et sinä voidellut öljyllä minun päätäni, mutta tämä voiteli hajuvoiteella minun jalkani.
\par 47 Sentähden minä sanon sinulle: tämän paljot synnit ovat anteeksi annetut: hänhän näet rakasti paljon; mutta jolle vähän anteeksi annetaan, se rakastaa vähän."
\par 48 Sitten hän sanoi naiselle: "Sinun syntisi ovat anteeksi annetut".
\par 49 Niin ateriakumppanit rupesivat ajattelemaan mielessänsä: "Kuka tämä on, joka synnitkin anteeksi antaa?"
\par 50 Mutta hän sanoi naiselle: "Sinun uskosi on sinut pelastanut; mene rauhaan".

\chapter{8}

\par 1 Ja sen jälkeen hän vaelsi kaupungista kaupunkiin ja kylästä kylään ja saarnasi ja julisti Jumalan valtakunnan evankeliumia; ja ne kaksitoista olivat hänen kanssansa,
\par 2 niin myös muutamia naisia, jotka olivat parannetut pahoista hengistä ja taudeista: Maria, Magdaleenaksi kutsuttu, josta seitsemän riivaajaa oli lähtenyt ulos,
\par 3 ja Johanna, Herodeksen taloudenhoitajan Kuusaan vaimo, ja Susanna ja useita muita, jotka palvelivat heitä varoillansa.
\par 4 Kun paljon kansaa kokoontui ja ihmisiä kulki joka kaupungista hänen tykönsä, puhui hän vertauksella:
\par 5 "Kylväjä meni kylvämään siementänsä. Ja hänen kylväessään putosi osa tien oheen ja tallautui, ja taivaan linnut söivät sen.
\par 6 Ja osa putosi kalliolle, ja oraalle noustuaan se kuivettui, kun sillä ei ollut kosteutta.
\par 7 Ja osa putosi orjantappurain sekaan, ja orjantappurat kasvoivat mukana ja tukahuttivat sen.
\par 8 Ja osa putosi hyvään maahan, kasvoi ja teki satakertaisen hedelmän." Tämän sanottuaan hän lausui suurella äänellä: "Jolla on korvat kuulla, se kuulkoon".
\par 9 Niin hänen opetuslapsensa kysyivät häneltä, mitä tämä vertaus merkitsi.
\par 10 Hän sanoi: "Teidän on annettu tuntea Jumalan valtakunnan salaisuudet, mutta muille ne esitetään vertauksissa, että he, vaikka näkevät, eivät näkisi, ja vaikka kuulevat, eivät ymmärtäisi.
\par 11 Vertaus on tämä: siemen on Jumalan sana.
\par 12 Mitkä tien oheen putosivat, ovat ne, jotka kuulevat, mutta sitten perkele tulee ja ottaa sanan pois heidän sydämestään, etteivät he uskoisi ja pelastuisi.
\par 13 Ja mitkä kalliolle putosivat, ovat ne, jotka kuullessaan sanan ottavat sen ilolla vastaan, mutta joilla ei ole juurta: ainoastaan ajaksi he uskovat ja kiusauksen hetkellä luopuvat.
\par 14 Mikä taas orjantappuroihin putosi, ne ovat ne, jotka kuulevat, mutta vaeltaessaan tukehtuvat tämän elämän huoliin, rikkauteen ja hekumoihin, eivätkä tuota kypsää hedelmää.
\par 15 Mutta mikä hyvään maahan putosi, ne ovat ne, jotka sanan kuultuansa säilyttävät sen vilpittömässä ja hyvässä sydämessä ja tuottavat hedelmän kärsivällisyydessä.
\par 16 Ei kukaan joka sytyttää lampun, peitä sitä astialla tai pane vuoteen alle, vaan panee sen lampunjalkaan, että sisälletulijat näkisivät valon.
\par 17 Sillä ei ole mitään salattua, mikä ei tule ilmi, eikä kätkettyä, mikä ei tule tunnetuksi ja joudu päivän valoon.
\par 18 Katsokaa siis, miten kuulette; sillä sille, jolla on, annetaan, mutta siltä, jolla ei ole, otetaan pois sekin, minkä hän luulee itsellään olevan."
\par 19 Ja hänen äitinsä ja veljensä tulivat häntä tapaamaan, mutta eivät väentungokselta päässeet hänen tykönsä.
\par 20 Niin hänelle ilmoitettiin: "Sinun äitisi ja veljesi seisovat ulkona ja tahtovat nähdä sinua".
\par 21 Mutta hän vastasi ja sanoi heille: "Minun äitini ja veljeni ovat nämä, jotka kuulevat Jumalan sanan ja sen mukaan tekevät".
\par 22 Niin tapahtui eräänä päivänä, että hän astui opetuslapsinensa venheeseen ja sanoi heille: "Menkäämme järven tuolle puolen". Ja he lähtivät vesille.
\par 23 Ja heidän purjehtiessaan hän nukkui. Mutta alas järvelle syöksyi myrskytuuli, ja venhe tuli vettä täyteen, ja he olivat vaarassa.
\par 24 Niin he menivät ja herättivät hänet sanoen: "Mestari, mestari, me hukumme!" Ja herättyään hän nuhteli tuulta ja veden aallokkoa; ja ne asettuivat, ja tuli tyven.
\par 25 Ja hän sanoi heille: "Missä on teidän uskonne?" Mutta pelko oli vallannut heidät, ja he ihmettelivät, sanoen toisilleen: "Kuka onkaan tämä, kun hän käskee sekä tuulia että vettä, ja ne tottelevat häntä?"
\par 26 Ja he purjehtivat gerasalaisten alueelle, joka on vastapäätä Galileaa.
\par 27 Ja kun hän oli noussut maihin, tuli häntä vastaan kaupungista mies, jossa oli riivaajia; ja hän ei ollut pitkään aikaan pukenut vaatteita ylleen eikä asunut huoneessa, vaan haudoissa.
\par 28 Kun hän näki Jeesuksen, parkaisi hän ja lankesi maahan hänen eteensä ja huusi suurella äänellä: "Mitä sinulla on minun kanssani tekemistä, Jeesus, Jumalan, Korkeimman, Poika? Minä rukoilen sinua: älä minua vaivaa."
\par 29 Sillä hän oli käskemäisillään saastaista henkeä menemään ulos siitä miehestä. Sillä pitkät ajat se oli temponut häntä mukaansa; hänet oli sidottu kahleisiin ja jalkanuoriin, ja häntä oli vartioitu, mutta hän oli katkaissut siteet ja kulkeutunut riivaajan ajamana erämaihin.
\par 30 Niin Jeesus kysyi siltä sanoen: "Mikä on nimesi?" Hän vastasi: "Legio"; sillä monta riivaajaa oli mennyt häneen.
\par 31 Ja ne pyysivät häntä, ettei hän käskisi heidän mennä syvyyteen.
\par 32 Niin siellä oli vuorella suuri sikalauma laitumella; ja ne pyysivät häntä, että hän antaisi heille luvan mennä sikoihin. Ja hän antoi niille luvan.
\par 33 Niin riivaajat lähtivät ulos miehestä ja menivät sikoihin. Silloin lauma syöksyi jyrkännettä alas järveen ja hukkui.
\par 34 Mutta nähtyään, mitä oli tapahtunut, paimentajat pakenivat ja kertoivat siitä kaupungissa ja maataloissa.
\par 35 Niin kansa lähti katsomaan, mitä oli tapahtunut, ja he tulivat Jeesuksen luo ja tapasivat miehen, josta riivaajat olivat lähteneet, istumassa Jeesuksen jalkojen juuressa puettuna ja täydessä ymmärryksessä; ja he peljästyivät.
\par 36 Mutta silminnäkijät kertoivat heille, kuinka riivattu oli tullut terveeksi.
\par 37 Ja koko gerasalaisten seutukunnan kansa pyysi häntä poistumaan heidän luotansa, sillä suuri pelko oli vallannut heidät; niin hän astui venheeseen ja palasi takaisin.
\par 38 Ja mies, josta riivaajat olivat lähteneet, pyysi häneltä saada olla hänen kanssaan. Mutta Jeesus lähetti hänet luotansa sanoen:
\par 39 "Palaja kotiisi ja kerro, kuinka suuria töitä Jumala on sinulle tehnyt". Ja hän meni ja julisti kaikkialla kaupungissa, kuinka suuria töitä Jeesus oli hänelle tehnyt.
\par 40 Kun Jeesus palasi, oli kansa häntä vastassa; sillä kaikki odottivat häntä.
\par 41 Ja katso, silloin tuli mies, nimeltä Jairus, joka oli synagoogan esimies. Ja hän lankesi Jeesuksen jalkojen juureen ja pyysi häntä tulemaan kotiinsa,
\par 42 sillä hänellä oli tytär, ainoa lapsi, noin kaksitoistavuotias, ja se oli kuolemaisillaan. Mutta hänen sinne mennessään väentungos ahdisti häntä.
\par 43 Ja siellä oli nainen, joka kaksitoista vuotta oli sairastanut verenjuoksua ja lääkäreille kuluttanut kaiken omaisuutensa, eikä kukaan ollut voinut häntä parantaa.
\par 44 Tämä lähestyi takaapäin ja kosketti hänen vaippansa tupsua, ja heti hänen verenjuoksunsa asettui.
\par 45 Ja Jeesus sanoi: "Kuka minuun koski?" Mutta kun kaikki kielsivät, sanoi Pietari ja ne, jotka olivat hänen kanssaan: "Mestari, väentungos ahdistaa ja pusertaa sinua".
\par 46 Mutta Jeesus sanoi: "Joku minuun koski; sillä minä tunsin, että voimaa lähti minusta".
\par 47 Kun nainen näki, ettei hän pysynyt salassa, tuli hän vavisten, lankesi hänen eteensä ja ilmoitti kaiken kansan kuullen, mistä syystä hän oli koskenut häneen ja kuinka hän oli kohta tullut terveeksi.
\par 48 Niin hän sanoi hänelle: "Tyttäreni, uskosi on sinut pelastanut; mene rauhaan".
\par 49 Hänen vielä puhuessaan tuli joku synagoogan esimiehen kotoa ja sanoi: "Tyttäresi on kuollut; älä enää opettajaa vaivaa".
\par 50 Mutta sen kuultuaan Jeesus sanoi hänelle: "Älä pelkää; usko ainoastaan, niin hän paranee".
\par 51 Ja kun hän tuli taloon, ei hän sallinut kenenkään muun käydä sisälle kanssaan kuin Pietarin ja Johanneksen ja Jaakobin sekä lapsen isän ja äidin.
\par 52 Ja kaikki itkivät ja vaikeroivat tyttöä. Mutta Jeesus sanoi: "Älkää itkekö, sillä hän ei ole kuollut, vaan nukkuu".
\par 53 Niin he nauroivat häntä, tietäen tytön kuolleeksi.
\par 54 Mutta hän tarttui hänen käteensä ja huusi sanoen: "Lapsi, nouse!"
\par 55 Niin hänen henkensä palasi, ja hän nousi heti ylös; ja Jeesus käski antaa hänelle syötävää.
\par 56 Ja hänen vanhempansa hämmästyivät; mutta Jeesus kielsi heitä kenellekään sanomasta, mitä oli tapahtunut.

\chapter{9}

\par 1 Niin hän kutsui kokoon ne kaksitoista ja antoi heille voiman ja vallan kaikkia riivaajia vastaan ja voiman parantaa tauteja.
\par 2 Ja hän lähetti heidät julistamaan Jumalan valtakuntaa ja parantamaan sairaita.
\par 3 Ja hän sanoi heille: "Älkää ottako mitään matkalle, ei sauvaa, ei laukkua, ei leipää, ei rahaa, älköön myös kenelläkään olko kahta ihokasta.
\par 4 Ja mihin taloon tulette, siihen jääkää, ja siitä lähtekää matkallenne.
\par 5 Ja missä eivät ota teitä vastaan, siitä kaupungista lähtekää pois, ja pudistakaa tomu jaloistanne, todistukseksi heitä vastaan."
\par 6 Niin he lähtivät ja kulkivat kylästä kylään julistaen evankeliumia ja parantaen sairaita kaikkialla.
\par 7 Ja neljännysruhtinas Herodes sai kuulla kaikki, mitä tapahtui, eikä tiennyt, mitä ajatella; sillä muutamat sanoivat: "Johannes on noussut kuolleista",
\par 8 mutta toiset: "Elias on ilmestynyt", toiset taas: "Joku vanhoista profeetoista on noussut ylös".
\par 9 Ja Herodes sanoi: "Johanneksen minä olen mestauttanut; mutta kuka tämä on, josta minä tuollaista kuulen?" Ja hän etsi tilaisuutta saadakseen nähdä hänet.
\par 10 Ja apostolit palasivat ja kertoivat Jeesukselle kaikki, mitä olivat tehneet. Niin hän otti heidät mukaansa ja vetäytyi yksinäisyyteen lähelle Beetsaida nimistä kaupunkia.
\par 11 Mutta kun kansa sai sen tietää, seurasivat he häntä; ja hän otti heidät vastaan ja puhui heille Jumalan valtakunnasta ja teki terveiksi ne, jotka parantamista tarvitsivat.
\par 12 Ja päivä alkoi laskea. Niin ne kaksitoista tulivat ja sanoivat hänelle: "Laske kansa luotasi, että he menisivät ympärillä oleviin kyliin ja maataloihin majoittumaan ja saamaan ravintoa, sillä täällä me olemme autiossa paikassa".
\par 13 Mutta hän sanoi heille: "Antakaa te heille syödä". Niin he sanoivat: "Meillä ei ole enempää kuin viisi leipää ja kaksi kalaa, ellemme lähde ostamaan ruokaa kaikelle tälle kansalle".
\par 14 Sillä heitä oli noin viisituhatta miestä. Niin hän sanoi opetuslapsilleen: "Asettakaa heidät aterioimaan ruokakunnittain, noin viisikymmentä kuhunkin".
\par 15 Ja he tekivät niin ja asettivat kaikki aterioimaan.
\par 16 Niin hän otti ne viisi leipää ja kaksi kalaa, katsoi ylös taivaaseen ja siunasi ne ja mursi ja antoi opetuslapsilleen kansan eteen pantaviksi.
\par 17 Ja kaikki söivät ja tulivat ravituiksi; ja heiltä jääneet tähteet kerättiin, kaksitoista vakallista palasia.
\par 18 Ja tapahtui, kun hän oli yksinäisessä paikassa rukoilemassa ja hänen opetuslapsensa olivat hänen kanssaan, että hän kysyi heiltä ja sanoi: "Kenen kansa sanoo minun olevan?"
\par 19 He vastasivat sanoen: "Johannes Kastajan, mutta toiset Eliaan, toiset taas sanovat, että joku vanhoista profeetoista on noussut ylös".
\par 20 Niin hän sanoi heille: "Kenenkä te sanotte minun olevan?" Pietari vastasi ja sanoi: "Sinä olet Jumalan Kristus".
\par 21 Niin hän vakavasti varoittaen kielsi heitä kenellekään tästä puhumasta
\par 22 ja sanoi: "Ihmisen Pojan pitää kärsimän paljon ja joutuman vanhinten ja ylipappien ja kirjanoppineiden hyljittäväksi ja tuleman tapetuksi ja kolmantena päivänä nouseman ylös".
\par 23 Ja hän sanoi kaikille: "Jos joku tahtoo minun perässäni kulkea, hän kieltäköön itsensä ja ottakoon joka päivä ristinsä ja seuratkoon minua.
\par 24 Sillä joka tahtoo pelastaa elämänsä, hän kadottaa sen, mutta joka kadottaa elämänsä minun tähteni, hän pelastaa sen.
\par 25 Sillä mitä se hyödyttää ihmistä, vaikka hän voittaisi omaksensa koko maailman, mutta saattaisi itsensä kadotukseen tai turmioon?
\par 26 Sillä joka häpeää minua ja minun sanojani, sitä Ihmisen Poika on häpeävä, kun hän tulee omassa ja Isänsä ja pyhäin enkelien kirkkaudessa.
\par 27 Totisesti minä sanon teille: tässä seisovien joukossa on muutamia, jotka eivät maista kuolemaa, ennenkuin näkevät Jumalan valtakunnan."
\par 28 Noin kahdeksan päivää sen jälkeen kuin hän oli tämän puhunut, hän otti mukaansa Pietarin ja Johanneksen ja Jaakobin ja nousi vuorelle rukoilemaan.
\par 29 Ja hänen rukoillessaan hänen kasvojensa näkö muuttui, ja hänen vaatteensa tulivat säteilevän valkoisiksi.
\par 30 Ja katso, hänen kanssaan puhui kaksi miestä, ja ne olivat Mooses ja Elias.
\par 31 He näkyivät kirkkaudessa ja puhuivat hänen poismenostansa, jonka hän oli saattava täytäntöön Jerusalemissa.
\par 32 Mutta Pietari ja ne, jotka olivat hänen kanssansa, olivat unen raskauttamia; mutta kun he siitä heräsivät, näkivät he hänen kirkkautensa ja ne kaksi miestä, jotka seisoivat hänen luonansa.
\par 33 Ja kun nämä olivat eroamassa hänestä, sanoi Pietari Jeesukselle: "Mestari, meidän on tässä hyvä olla; tehkäämme kolme majaa, sinulle yksi ja Moosekselle yksi ja Elialle yksi". Mutta hän ei tiennyt, mitä sanoi.
\par 34 Ja hänen tätä sanoessaan tuli pilvi ja peitti heidät varjoonsa; ja he peljästyivät joutuessaan pilveen.
\par 35 Ja pilvestä kuului ääni, joka sanoi: "Tämä on minun Poikani, se valittu; kuulkaa häntä".
\par 36 Ja äänen kuuluessa he huomasivat Jeesuksen olevan yksin. Ja he olivat siitä vaiti eivätkä niinä päivinä ilmoittaneet kenellekään mitään siitä, mitä olivat nähneet.
\par 37 Kun he seuraavana päivänä menivät alas vuorelta, tuli paljon kansaa häntä vastaan.
\par 38 Ja katso, kansanjoukosta huusi eräs mies sanoen: "Opettaja, minä rukoilen sinua, katsahda minun poikani puoleen, sillä hän on minun ainokaiseni;
\par 39 ja katso, hänen kimppuunsa käy henki, ja heti hän parkaisee, ja se kouristaa häntä, niin että vaahto lähtee; ja vaivoin se hänestä poistuu, runnellen häntä.
\par 40 Ja minä pyysin sinun opetuslapsiasi ajamaan sitä ulos, mutta he eivät voineet."
\par 41 Jeesus vastasi ja sanoi: "Voi, sinä epäuskoinen ja nurja sukupolvi; kuinka kauan minun täytyy olla teidän luonanne ja kärsiä teitä? Tuo poikasi tänne."
\par 42 Ja vielä pojan tullessakin riivaaja repi häntä ja kouristi kovin. Mutta Jeesus nuhteli saastaista henkeä ja paransi pojan ja antoi hänet takaisin hänen isällensä.
\par 43 Ja kaikki hämmästyivät Jumalan valtasuuruutta. Mutta kun kaikki ihmettelivät kaikkea sitä, mitä Jeesus teki, sanoi hän opetuslapsillensa:
\par 44 "Ottakaa korviinne nämä sanat: Ihmisen Poika annetaan ihmisten käsiin".
\par 45 Mutta he eivät käsittäneet tätä puhetta, ja se oli heiltä peitetty, niin etteivät he sitä ymmärtäneet, ja he pelkäsivät kysyä häneltä, mitä se puhe oli.
\par 46 Ja heidän mieleensä tuli ajatus, kuka heistä mahtoi olla suurin.
\par 47 Mutta kun Jeesus tiesi heidän sydämensä ajatuksen, otti hän lapsen ja asetti sen viereensä
\par 48 ja sanoi heille: "Joka ottaa tykönsä tämän lapsen minun nimeeni, se ottaa tykönsä minut; ja joka ottaa tykönsä minut, ottaa tykönsä hänet, joka on minut lähettänyt. Sillä joka teistä kaikista on pienin, se on suuri."
\par 49 Silloin Johannes rupesi puhumaan ja sanoi: "Mestari, me näimme erään miehen sinun nimessäsi ajavan ulos riivaajia, ja me kielsimme häntä, koska hän ei seuraa meidän mukanamme".
\par 50 Mutta Jeesus sanoi hänelle: "Älkää kieltäkö; sillä joka ei ole teitä vastaan, se on teidän puolellanne".
\par 51 Ja kun hänen ylösottamisensa aika oli täyttymässä, käänsi hän kasvonsa Jerusalemia kohti, vaeltaaksensa sinne.
\par 52 Ja hän lähetti edellänsä sanansaattajia; ja he lähtivät matkalle ja menivät erääseen samarialaisten kylään valmistaakseen hänelle majaa.
\par 53 Mutta siellä ei otettu häntä vastaan, koska hän oli vaeltamassa kohti Jerusalemia.
\par 54 Kun hänen opetuslapsensa Jaakob ja Johannes sen näkivät, sanoivat he: "Herra, tahdotko, niin sanomme, että tuli taivaasta tulkoon alas ja hävittäköön heidät?"
\par 55 Mutta hän kääntyi ja nuhteli heitä.
\par 56 Ja he vaelsivat toiseen kylään.
\par 57 Ja heidän tietä vaeltaessaan eräs mies sanoi hänelle: "Minä seuraan sinua, mihin ikinä menet".
\par 58 Niin Jeesus sanoi hänelle: "Ketuilla on luolat ja taivaan linnuilla pesät, mutta Ihmisen Pojalla ei ole, mihin hän päänsä kallistaisi".
\par 59 Toiselle hän sanoi: "Seuraa minua". Mutta tämä sanoi: "Herra, salli minun ensin käydä hautaamassa isäni".
\par 60 Mutta Jeesus sanoi hänelle: "Anna kuolleitten haudata kuolleensa, mutta mene sinä ja julista Jumalan valtakuntaa".
\par 61 Vielä eräs toinen sanoi: "Minä tahdon seurata sinua, Herra; mutta salli minun ensin käydä ottamassa jäähyväiset kotiväeltäni".
\par 62 Mutta Jeesus sanoi hänelle: "Ei kukaan, joka laskee kätensä auraan ja katsoo taaksensa, ole sovelias Jumalan valtakuntaan".

\chapter{10}

\par 1 Sen jälkeen Herra valitsi seitsemänkymmentä muuta ja lähetti heidät kaksittain edellänsä jokaiseen kaupunkiin ja paikkaan, jonne hän itse aikoi mennä.
\par 2 Ja hän sanoi heille: "Eloa on paljon, mutta työmiehiä vähän. Rukoilkaa siis elon Herraa, että hän lähettäisi työmiehiä elonkorjuuseensa.
\par 3 Menkää; katso, minä lähetän teidät niinkuin lampaat susien keskelle.
\par 4 Älkää ottako mukaanne rahakukkaroa, älkää laukkua, älkää kenkiä, älkääkä tervehtikö ketään tiellä.
\par 5 Kun tulette johonkin taloon, niin sanokaa ensiksi: 'Rauha tälle talolle!'
\par 6 Ja jos siellä on rauhan lapsi, niin teidän rauhanne on lepäävä hänen päällänsä; mutta jos ei ole, niin se palajaa teille.
\par 7 Ja olkaa siinä talossa ja syökää ja juokaa, mitä heillä on tarjota, sillä työmies on palkkansa ansainnut. Älkää siirtykö talosta taloon.
\par 8 Ja mihin kaupunkiin te tulettekin, missä teidät otetaan vastaan, syökää, mitä eteenne pannaan,
\par 9 ja parantakaa sairaat siellä ja sanokaa heille: 'Jumalan valtakunta on tullut teitä lähelle'.
\par 10 Mutta kun tulette kaupunkiin, jossa teitä ei oteta vastaan, niin menkää sen kaduille ja sanokaa:
\par 11 'Tomunkin, joka teidän kaupungistanne on jalkoihimme tarttunut, me pudistamme teille takaisin; mutta se tietäkää, että Jumalan valtakunta on tullut lähelle'.
\par 12 Minä sanon teille: Sodoman on oleva sinä päivänä helpompi kuin sen kaupungin.
\par 13 Voi sinua, Korasin! Voi sinua, Beetsaida! Sillä jos ne voimalliset teot, jotka teissä ovat tapahtuneet, olisivat tapahtuneet Tyyrossa ja Siidonissa, niin nämä jo aikaa sitten olisivat säkissä ja tuhassa istuen tehneet parannuksen.
\par 14 Mutta Tyyron ja Siidonin on oleva tuomiolla helpompi kuin teidän.
\par 15 Ja sinä, Kapernaum, korotetaankohan sinut hamaan taivaaseen? Hamaan tuonelaan on sinun astuttava alas.
\par 16 Joka kuulee teitä, se kuulee minua, ja joka hylkää teidät, hylkää minut; mutta joka minut hylkää, hylkää hänet, joka on minut lähettänyt."
\par 17 Niin ne seitsemänkymmentä palasivat iloiten ja sanoivat: "Herra, riivaajatkin ovat meille alamaiset sinun nimesi tähden".
\par 18 Silloin hän sanoi heille: "Minä näin saatanan lankeavan taivaasta niinkuin salaman.
\par 19 Katso, minä olen antanut teille vallan tallata käärmeitä ja skorpioneja ja kaikkea vihollisen voimaa, eikä mikään ole teitä vahingoittava.
\par 20 Älkää kuitenkaan siitä iloitko, että henget ovat teille alamaiset, vaan iloitkaa siitä, että teidän nimenne ovat kirjoitettuina taivaissa."
\par 21 Sillä hetkellä hän riemuitsi Pyhässä Hengessä ja sanoi: "Minä ylistän sinua, Isä, taivaan ja maan Herra, että olet salannut nämä viisailta ja ymmärtäväisiltä ja ilmoittanut ne lapsenmielisille. Niin, Isä, sillä näin on sinulle hyväksi näkynyt.
\par 22 Kaikki on minun Isäni antanut minun haltuuni, eikä kukaan muu tunne, kuka Poika on, kuin Isä; eikä kukaan muu tunne, kuka Isä on, kuin Poika ja se, kenelle Poika tahtoo hänet ilmoittaa."
\par 23 Ja hän kääntyi opetuslapsiinsa erikseen ja sanoi: "Autuaat ovat ne silmät, jotka näkevät, mitä te näette.
\par 24 Sillä minä sanon teille: monet profeetat ja kuninkaat ovat tahtoneet nähdä, mitä te näette, eivätkä ole nähneet, ja kuulla, mitä te kuulette, eivätkä ole kuulleet."
\par 25 Ja katso, eräs lainoppinut nousi ja kysyi kiusaten häntä: "Opettaja, mitä minun pitää tekemän, että minä iankaikkisen elämän perisin?"
\par 26 Niin hän sanoi hänelle: "Mitä laissa on kirjoitettuna? Kuinkas luet?"
\par 27 Hän vastasi ja sanoi: "Rakasta Herraa, sinun Jumalaasi, kaikesta sydämestäsi ja kaikesta sielustasi ja kaikesta voimastasi ja kaikesta mielestäsi, ja lähimmäistäsi niinkuin itseäsi".
\par 28 Hän sanoi hänelle: "Oikein vastasit; tee se, niin sinä saat elää".
\par 29 Mutta hän tahtoi näyttää olevansa vanhurskas ja sanoi Jeesukselle: "Kuka sitten on minun lähimmäiseni?"
\par 30 Jeesus vastasi ja sanoi: "Eräs mies vaelsi Jerusalemista alas Jerikoon ja joutui ryövärien käsiin, jotka riisuivat hänet alasti ja löivät haavoille ja menivät pois jättäen hänet puolikuolleeksi.
\par 31 Niin vaelsi sattumalta eräs pappi sitä tietä ja näki hänet ja meni ohitse.
\par 32 Samoin leeviläinenkin: kun hän tuli sille paikalle ja näki hänet, meni hän ohitse.
\par 33 Mutta kun eräs samarialainen, joka matkusti sitä tietä, tuli hänen kohdalleen ja näki hänet, niin hän armahti häntä.
\par 34 Ja hän meni hänen luokseen ja sitoi hänen haavansa ja vuodatti niihin öljyä ja viiniä, pani hänet juhtansa selkään ja vei hänet majataloon ja hoiti häntä.
\par 35 Ja seuraavana aamuna hän otti esiin kaksi denaria ja antoi majatalon isännälle ja sanoi: 'Hoida häntä, ja mitä sinulta lisää kuluu, sen minä palatessani sinulle maksan'.
\par 36 Kuka näistä kolmesta sinun mielestäsi osoitti olevansa sen lähimmäinen, joka oli joutunut ryövärien käsiin?"
\par 37 Hän sanoi: "Se, joka osoitti hänelle laupeutta". Niin Jeesus sanoi hänelle: "Mene ja tee sinä samoin".
\par 38 Ja heidän vaeltaessaan hän meni muutamaan kylään. Niin eräs nainen, nimeltä Martta, otti hänet kotiinsa.
\par 39 Ja hänellä oli sisar, Maria niminen, joka asettui istumaan Herran jalkojen juureen ja kuunteli hänen puhettansa.
\par 40 Mutta Martta puuhasi monissa palvelustoimissa; ja hän tuli ja sanoi: "Herra, etkö sinä välitä mitään siitä, että sisareni on jättänyt minut yksinäni palvelemaan? Sano siis hänelle, että hän minua auttaisi."
\par 41 Herra vastasi ja sanoi hänelle: "Martta, Martta, moninaisista sinä huolehdit ja hätäilet,
\par 42 mutta tarpeellisia on vähän, tahi yksi ainoa. Maria on valinnut hyvän osan, jota ei häneltä oteta pois."

\chapter{11}

\par 1 Ja kun hän oli eräässä paikassa rukoilemassa ja oli lakannut, sanoi eräs hänen opetuslapsistansa hänelle: "Herra, opeta meitä rukoilemaan, niinkuin Johanneskin opetti opetuslapsiansa".
\par 2 Niin hän sanoi heille: "Kun rukoilette, sanokaa: Isä, pyhitetty olkoon sinun nimesi; tulkoon sinun valtakuntasi; (tapahtukoon sinun tahtosi myös maan päällä niinkuin taivaassa;)
\par 3 anna meille joka päivä meidän jokapäiväinen leipämme;
\par 4 ja anna meille meidän syntimme anteeksi, sillä mekin annamme anteeksi jokaiselle velallisellemme; äläkä saata meitä kiusaukseen; (vaan päästä meidät pahasta)."
\par 5 Ja hän sanoi heille: "Jos jollakin teistä on ystävä ja hän menee hänen luoksensa yösydännä ja sanoo hänelle: 'Ystäväni, lainaa minulle kolme leipää,
\par 6 sillä eräs ystäväni on matkallaan tullut minun luokseni, eikä minulla ole, mitä panna hänen eteensä';
\par 7 ja toinen sisältä vastaa ja sanoo: 'Älä minua vaivaa; ovi on jo suljettu, ja lapseni ovat minun kanssani vuoteessa; en minä voi nousta antamaan sinulle' -
\par 8 minä sanon teille: vaikka hän ei nousekaan antamaan hänelle sentähden, että hän on hänen ystävänsä, nousee hän kuitenkin sentähden, että toinen ei hellitä, ja antaa hänelle niin paljon, kuin hän tarvitsee.
\par 9 Niinpä minäkin sanon teille: anokaa, niin teille annetaan; etsikää, niin te löydätte; kolkuttakaa, niin teille avataan.
\par 10 Sillä jokainen anova saa, ja etsivä löytää, ja kolkuttavalle avataan.
\par 11 Ja kuka teistä on se isä, joka poikansa häneltä pyytäessä kalaa antaa hänelle kalan sijasta käärmeen,
\par 12 taikka joka hänen pyytäessään munaa antaa hänelle skorpionin?
\par 13 Jos siis te, jotka olette pahoja, osaatte antaa lapsillenne hyviä lahjoja, kuinka paljoa ennemmin taivaallinen Isä antaa Pyhän Hengen niille, jotka sitä häneltä anovat!"
\par 14 Ja hän ajoi ulos riivaajan, ja se oli mykkä; ja kun riivaaja oli lähtenyt, niin tapahtui, että mykkä mies puhui; ja kansa ihmetteli.
\par 15 Mutta muutamat heistä sanoivat: "Beelsebulin, riivaajain päämiehen, voimalla hän ajaa ulos riivaajia".
\par 16 Toiset taas kiusasivat häntä ja pyysivät häneltä merkkiä taivaasta.
\par 17 Mutta hän tiesi heidän ajatuksensa ja sanoi heille: "Jokainen valtakunta, joka riitautuu itsensä kanssa, joutuu autioksi, ja talo kaatuu talon päälle.
\par 18 Jos nyt saatanakin on riitautunut itsensä kanssa, kuinka hänen valtakuntansa pysyy pystyssä? Tehän sanotte minun Beelsebulin voimalla ajavan ulos riivaajia.
\par 19 Mutta jos minä Beelsebulin voimalla ajan ulos riivaajia, kenenkä voimalla sitten teidän lapsenne ajavat niitä ulos? Sentähden he tulevat olemaan teidän tuomarinne.
\par 20 Mutta jos minä Jumalan sormella ajan ulos riivaajia, niin onhan Jumalan valtakunta tullut teidän tykönne.
\par 21 Kun väkevä aseellisena vartioitsee kartanoaan, on hänen omaisuutensa turvassa.
\par 22 Mutta kun häntä väkevämpi karkaa hänen päällensä ja voittaa hänet, ottaa hän häneltä kaikki aseet, joihin hän luotti, ja jakaa häneltä riistämänsä saaliin.
\par 23 Joka ei ole minun kanssani, se on minua vastaan, ja joka ei minun kanssani kokoa, se hajottaa.
\par 24 Kun saastainen henki lähtee ihmisestä, kuljeksii se autioita paikkoja ja etsii lepoa; ja kun ei löydä, sanoo se: 'Minä palaan huoneeseeni, josta lähdin'.
\par 25 Ja kun se tulee, tapaa se sen lakaistuna ja kaunistettuna.
\par 26 Silloin se menee ja ottaa mukaansa seitsemän muuta henkeä, pahempaa kuin se itse, ja ne tulevat sisään ja asuvat siellä. Ja sen ihmisen viimeiset tulevat pahemmiksi kuin ensimmäiset."
\par 27 Niin hänen tätä puhuessaan eräs nainen kansanjoukosta korotti äänensä ja sanoi hänelle: "Autuas on se kohtu, joka on kantanut sinut, ja ne rinnat, joita olet imenyt".
\par 28 Mutta hän sanoi: "Niin, autuaat ovat ne, jotka kuulevat Jumalan sanan ja sitä noudattavat".
\par 29 Kun kansaa yhä kokoontui, rupesi hän puhumaan: "Tämä sukupolvi on paha sukupolvi: se tavoittelee merkkiä, mutta sille ei anneta muuta merkkiä kuin Joonaan merkki.
\par 30 Sillä niinkuin Joonas tuli niiniveläisille merkiksi, niin Ihmisen Poikakin on oleva merkkinä tälle sukupolvelle.
\par 31 Etelän kuningatar on heräjävä tuomiolle yhdessä tämän sukupolven miesten kanssa ja tuleva heille tuomioksi; sillä hän tuli maan ääristä kuulemaan Salomon viisautta, ja katso, tässä on enempi kuin Salomo.
\par 32 Niiniven miehet nousevat tuomiolle yhdessä tämän sukupolven kanssa ja tulevat sille tuomioksi; sillä he tekivät parannuksen Joonaan saarnan vaikutuksesta, ja katso, tässä on enempi kuin Joonas.
\par 33 Ei kukaan, joka sytyttää lampun, pane sitä kätköön eikä vakan alle, vaan panee sen lampunjalkaan, että sisälletulijat näkisivät valon.
\par 34 Sinun silmäsi on ruumiin lamppu. Kun silmäsi on terve, on koko sinun ruumiisi valaistu, mutta kun se on viallinen, on myös sinun ruumiisi pimeä.
\par 35 Katso siis, ettei valo, joka sinussa on, ole pimeyttä.
\par 36 Jos siis koko sinun ruumiisi on valoisa eikä miltään osaltaan pimeä, on se oleva kokonaan valoisa, niinkuin lampun valaistessa sinua kirkkaalla loisteellaan."
\par 37 Hänen näin puhuessaan pyysi eräs fariseus häntä luoksensa aterioimaan; niin hän meni sinne ja asettui aterialle.
\par 38 Mutta kun fariseus näki, ettei hän peseytynyt ennen ateriaa, ihmetteli hän.
\par 39 Silloin Herra sanoi hänelle: "Kyllä te, fariseukset, puhdistatte maljan ja vadin ulkopuolen, mutta sisäpuoli teissä on täynnä ryöstöä ja pahuutta.
\par 40 Te mielettömät, eikö se, joka on tehnyt ulkopuolen, ole tehnyt sisäpuoltakin?
\par 41 Mutta antakaa almuksi se, mikä sisällä on; katso, silloin kaikki on teille puhdasta.
\par 42 Mutta voi teitä, te fariseukset, kun te annatte kymmenykset mintuista ja ruuduista ja kaikenlaisista vihanneksista, mutta sivuutatte oikeuden ja Jumalan rakkauden! Näitä olisi tullut noudattaa eikä noitakaan laiminlyödä.
\par 43 Voi teitä, te fariseukset, kun te rakastatte etumaista istuinta synagoogissa ja tervehdyksiä toreilla!
\par 44 Voi teitä, kun te olette niinkuin merkittömät haudat, joiden päällitse ihmiset kävelevät niistä tietämättä!"
\par 45 Silloin eräs lainoppineista rupesi puhumaan ja sanoi hänelle: "Opettaja, kun noin puhut, niin sinä häpäiset myös meitä".
\par 46 Mutta hän sanoi: "Voi teitäkin, te lainoppineet, kun te sälytätte ihmisten päälle vaikeasti kannettavia taakkoja ettekä itse sormellannekaan koske niihin taakkoihin!
\par 47 Voi teitä, kun te rakennatte profeettain hautakammioita, ja teidän isänne ovat heidät tappaneet!
\par 48 Näin te siis olette isäinne tekojen todistajia ja suostutte niihin: sillä he tappoivat profeetat, ja te rakennatte niille hautakammioita.
\par 49 Sentähden Jumalan viisaus sanookin: 'Minä lähetän heille profeettoja ja apostoleja, ja muutamat niistä he tappavat ja toisia vainoavat,
\par 50 että tältä sukukunnalta vaadittaisiin kaikkien profeettain veri, mikä on vuodatettu maailman perustamisesta asti,
\par 51 hamasta Aabelin verestä Sakariaan vereen saakka, hänen, joka surmattiin alttarin ja temppelin välillä'. Niin, minä sanon teille, se pitää tältä sukukunnalta vaadittaman.
\par 52 Voi teitä, te lainoppineet, kun te olette vieneet tiedon avaimen! Itse te ette ole menneet sisälle, ja sisälle meneviä te olette estäneet."
\par 53 Ja hänen sieltä lähtiessään kirjanoppineet ja fariseukset rupesivat kovasti ahdistamaan häntä ja urkkimaan häneltä moninaisia,
\par 54 väijyen, miten saisivat hänet hänen sanoistaan ansaan.

\chapter{12}

\par 1 Kun sillä välin kansaa oli kokoontunut tuhatmäärin, niin että he polkivat toisiaan, rupesi hän puhumaan opetuslapsillensa: "Ennen kaikkea kavahtakaa fariseusten hapatusta, se on: ulkokultaisuutta.
\par 2 Ei ole mitään peitettyä, mikä ei tule paljastetuksi, eikä mitään salattua, mikä ei tule tunnetuksi.
\par 3 Sentähden, kaikki, mitä te pimeässä sanotte, joutuu päivänvalossa kuultavaksi, ja mitä korvaan puhutte kammioissa, se katoilta julistetaan.
\par 4 Mutta minä sanon teille, ystävilleni: älkää peljätkö niitä, jotka tappavat ruumiin, eivätkä sen jälkeen voi mitään enempää tehdä.
\par 5 Vaan minä osoitan teille, ketä teidän on pelkääminen: peljätkää häntä, jolla on valta tapettuansa syöstä helvettiin. Niin, minä sanon teille, häntä te peljätkää.
\par 6 Eikö viittä varpusta myydä kahteen ropoon? Eikä Jumala ole yhtäkään niistä unhottanut.
\par 7 Ovatpa teidän päänne hiuksetkin kaikki luetut. Älkää peljätkö; te olette suurempiarvoiset kuin monta varpusta.
\par 8 Mutta minä sanon teille: jokaisen, joka tunnustaa minut ihmisten edessä, myös Ihmisen Poika tunnustaa Jumalan enkelien edessä.
\par 9 Mutta joka kieltää minut ihmisten edessä, se kielletään Jumalan enkelien edessä.
\par 10 Ja jokaiselle, joka sanoo sanan Ihmisen Poikaa vastaan, annetaan anteeksi; mutta sille, joka Pyhää Henkeä pilkkaa, ei anteeksi anneta.
\par 11 Mutta kun he vievät teitä synagoogain ja hallitusten ja esivaltojen eteen, älkää huolehtiko siitä, miten tai mitä vastaisitte puolestanne tahi mitä sanoisitte;
\par 12 sillä Pyhä Henki opettaa teille sillä hetkellä, mitä teidän on sanottava."
\par 13 Niin muuan mies kansanjoukosta sanoi hänelle: "Opettaja, sano minun veljelleni, että hän jakaisi kanssani perinnön".
\par 14 Mutta hän vastasi hänelle: "Ihminen, kuka on minut asettanut teille tuomariksi tai jakomieheksi?"
\par 15 Ja hän sanoi heille: "Katsokaa eteenne ja kavahtakaa kaikkea ahneutta, sillä ei ihmisen elämä riipu hänen omaisuudestaan, vaikka sitä ylenpalttisesti olisi".
\par 16 Ja hän puhui heille vertauksen sanoen: "Rikkaan miehen maa kasvoi hyvin.
\par 17 Niin hän mietti mielessään ja sanoi: 'Mitä minä teen, kun ei minulla ole, mihin viljani kokoaisin?'
\par 18 Ja hän sanoi: 'Tämän minä teen: minä revin maahan aittani ja rakennan suuremmat ja kokoan niihin kaiken eloni ja hyvyyteni;
\par 19 ja sanon sielulleni: sielu, sinulla on paljon hyvää tallessa moneksi vuodeksi; nauti lepoa, syö, juo ja iloitse'.
\par 20 Mutta Jumala sanoi hänelle: 'Sinä mieletön, tänä yönä sinun sielusi vaaditaan sinulta pois; kenelle sitten joutuu se, minkä sinä olet hankkinut?'
\par 21 Näin käy sen, joka kokoaa aarteita itselleen, mutta jolla ei ole rikkautta Jumalan tykönä."
\par 22 Ja hän sanoi opetuslapsillensa: "Sentähden minä sanon teille: älkää murehtiko hengestänne, mitä söisitte, älkääkä ruumiistanne, mitä päällenne pukisitte.
\par 23 Sillä henki on enemmän kuin ruoka, ja ruumis enemmän kuin vaatteet.
\par 24 Katselkaa kaarneita: eivät ne kylvä eivätkä leikkaa, eikä niillä ole säilytyshuonetta eikä aittaa; ja Jumala ruokkii ne. Kuinka paljoa suurempiarvoiset te olette kuin linnut!
\par 25 Ja kuka teistä voi murehtimisellaan lisätä ikäänsä kyynäränkään vertaa?
\par 26 Jos siis ette voi sitäkään, mikä vähintä on, mitä te murehditte muusta?
\par 27 Katselkaa kukkia, kuinka ne kasvavat: eivät ne työtä tee eivätkä kehrää. Kuitenkin minä sanon teille: ei Salomo kaikessa loistossansa ollut niin vaatetettu kuin yksi niistä.
\par 28 Jos siis Jumala näin vaatettaa kedon ruohon, joka tänään kasvaa ja huomenna uuniin heitetään, kuinka paljoa ennemmin teidät, te vähäuskoiset!
\par 29 Älkää siis tekään etsikö, mitä söisitte ja mitä joisitte, älkääkä korkeita tavoitelko.
\par 30 Sillä näitä kaikkia maailman pakanakansat tavoittelevat; mutta teidän Isänne kyllä tietää teidän näitä tarvitsevan.
\par 31 Vaan etsikää Jumalan valtakuntaa, niin myös nämä teille annetaan sen ohessa.
\par 32 Älä pelkää, sinä piskuinen lauma; sillä teidän Isänne on nähnyt hyväksi antaa teille valtakunnan.
\par 33 Myykää, mitä teillä on, ja antakaa almuja; hankkikaa itsellenne kulumattomat kukkarot, loppumaton aarre taivaisiin, mihin ei varas ulotu ja missä koi ei turmele.
\par 34 Sillä missä teidän aarteenne on, siellä on myös teidän sydämenne.
\par 35 Olkoot teidän kupeenne vyötetyt ja lamppunne palamassa;
\par 36 ja olkaa te niiden ihmisten kaltaiset, jotka herraansa odottavat, milloin hän palajaa häistä, että he hänen tullessaan ja kolkuttaessaan heti avaisivat hänelle.
\par 37 Autuaat ne palvelijat, jotka heidän herransa tullessaan tapaa valvomasta! Totisesti minä sanon teille: hän vyöttäytyy ja asettaa heidät aterioimaan ja menee ja palvelee heitä.
\par 38 Ja jos hän tulee toisella yövartiolla tai kolmannella ja havaitsee heidän näin tekevän, niin autuaat ovat ne palvelijat.
\par 39 Mutta se tietäkää: jos perheenisäntä tietäisi, millä hetkellä varas tulee, hän ei sallisi taloonsa murtauduttavan.
\par 40 Niin olkaa tekin valmiit, sillä sinä hetkenä, jona ette luule, Ihmisen Poika tulee."
\par 41 Niin Pietari sanoi: "Herra, meistäkö sinä sanot tämän vertauksen vai myös kaikista muista?"
\par 42 Ja Herra sanoi: "Kuka siis on se uskollinen ja ymmärtäväinen huoneenhaltija, jonka hänen herransa asettaa pitämään huolta hänen palvelusväestään, antamaan heille ajallaan heidän ruokaosansa?
\par 43 Autuas se palvelija, jonka hänen herransa tullessaan havaitsee näin tekevän!
\par 44 Totisesti minä sanon teille: hän asettaa hänet kaiken omaisuutensa hoitajaksi.
\par 45 Mutta jos palvelija sanoo sydämessään: 'Herrani tulo viivästyy', ja rupeaa lyömään palvelijoita ja palvelijattaria sekä syömään ja juomaan ja päihdyttämään itseänsä,
\par 46 niin sen palvelijan herra tulee päivänä, jona hän ei odota, ja hetkenä, jota hän ei arvaa, ja hakkaa hänet kappaleiksi ja määrää hänelle saman osan kuin uskottomille.
\par 47 Ja sitä palvelijaa, joka tiesi herransa tahdon, mutta ei tehnyt valmistuksia eikä toiminut hänen tahtonsa mukaan, rangaistaan monilla lyönneillä.
\par 48 Sitä taas, joka ei tiennyt, mutta teki semmoista, mikä lyöntejä ansaitsee, rangaistaan vain muutamilla lyönneillä. Sillä jokaiselta, jolle on paljon annettu, myös paljon vaaditaan; ja jolle on paljon uskottu, siltä sitä enemmän kysytään.
\par 49 Tulta minä olen tullut heittämään maan päälle; ja kuinka minä tahtoisinkaan, että se jo olisi syttynyt!
\par 50 Mutta minä olen kasteella kastettava, ja kuinka minä olenkaan ahdistettu, kunnes se on täytetty!
\par 51 Luuletteko, että minä olen tullut tuomaan maan päälle rauhaa? Ei, sanon minä teille, vaan eripuraisuutta.
\par 52 Sillä tästedes riitautuu viisi samassa talossa keskenään, kolme joutuu riitaan kahta vastaan ja kaksi kolmea vastaan,
\par 53 isä poikaansa vastaan ja poika isäänsä vastaan, äiti tytärtänsä vastaan ja tytär äitiänsä vastaan, anoppi miniäänsä vastaan ja miniä anoppiansa vastaan."
\par 54 Ja hän sanoi myöskin kansalle: "Kun näette pilven nousevan lännestä, sanotte kohta: 'Tulee sade'; ja niin tuleekin.
\par 55 Ja kun näette etelätuulen puhaltavan, sanotte: 'Tulee helle'; ja niin tuleekin.
\par 56 Te ulkokullatut, maan ja taivaan muodon te osaatte arvioida; mutta kuinka ette arvioitse tätä aikaa?
\par 57 Miksi ette jo itsestänne päätä, mikä oikeata on?
\par 58 Kun kuljet riitapuolesi kanssa hallitusmiehen eteen, niin tee matkalla voitavasi päästäksesi hänestä sovussa eroon, ettei hän raastaisi sinua tuomarin eteen ja tuomari antaisi sinua oikeudenpalvelijalle, ja ettei oikeudenpalvelija heittäisi sinua vankeuteen.
\par 59 Minä sanon sinulle: sieltä et pääse, ennenkuin maksat viimeisenkin rovon."

\chapter{13}

\par 1 Samaan aikaan oli saapuvilla muutamia, jotka kertoivat hänelle niistä galilealaisista, joiden veren Pilatus oli sekoittanut heidän uhriensa vereen.
\par 2 Niin Jeesus vastasi ja sanoi heille: "Luuletteko, että nämä galilealaiset olivat syntisemmät kuin kaikki muut galilealaiset, koska he saivat kärsiä tämän?
\par 3 Eivät olleet, sanon minä teille, mutta ellette tee parannusta, niin samoin te kaikki hukutte.
\par 4 Taikka ne kahdeksantoista, jotka saivat surmansa, kun torni Siloassa kaatui heidän päällensä, luuletteko, että he olivat syyllisemmät kuin kaikki muut ihmiset, jotka Jerusalemissa asuvat?
\par 5 Eivät olleet, sanon minä teille, mutta ellette tee parannusta, niin samoin te kaikki hukutte."
\par 6 Ja hän puhui tämän vertauksen: "Eräällä miehellä oli viikunapuu istutettuna viinitarhassaan; ja hän tuli etsimään hedelmää siitä, mutta ei löytänyt.
\par 7 Niin hän sanoi viinitarhurille: 'Katso, kolmena vuotena minä olen käynyt etsimässä hedelmää tästä viikunapuusta, mutta en ole löytänyt. Hakkaa se pois; mitä varten se vielä maata laihduttaa?'
\par 8 Mutta tämä vastasi ja sanoi hänelle: 'Herra, anna sen olla vielä tämä vuosi; sillä aikaa minä kuokin ja lannoitan maan sen ympäriltä.
\par 9 Ehkä se ensi vuonna tekee hedelmää; mutta jos ei, niin hakkaa se pois'."
\par 10 Ja hän oli opettamassa eräässä synagoogassa sapattina.
\par 11 Ja katso, siellä oli nainen, jossa oli ollut heikkouden henki kahdeksantoista vuotta, ja hän oli koukistunut ja täydelleen kykenemätön oikaisemaan itseänsä.
\par 12 Hänet nähdessään Jeesus kutsui hänet luoksensa ja sanoi hänelle: "Nainen, sinä olet päässyt heikkoudestasi",
\par 13 ja pani kätensä hänen päälleen. Ja heti hän oikaisi itsensä suoraksi ja ylisti Jumalaa.
\par 14 Mutta synagoogan esimies, joka närkästyi siitä, että Jeesus paransi sapattina, rupesi puhumaan ja sanoi kansalle: "Kuusi päivää on, joina tulee työtä tehdä; tulkaa siis niinä päivinä parannuttamaan itseänne, älkääkä sapatinpäivänä".
\par 15 Mutta Herra vastasi hänelle ja sanoi: "Te ulkokullatut, eikö jokainen teistä sapattina päästä härkäänsä tai aasiansa seimestä ja vie sitä juomaan?
\par 16 Ja tätä naista, joka on Aabrahamin tytär ja jota saatana on pitänyt sidottuna, katso, jo kahdeksantoista vuotta, tätäkö ei olisi pitänyt päästää siitä siteestä sapatinpäivänä?"
\par 17 Ja hänen näin sanoessaan kaikki hänen vastustajansa häpesivät, ja kaikki kansa iloitsi kaikista niistä ihmeellisistä teoista, joita hän teki.
\par 18 Niin hän sanoi: "Minkä kaltainen on Jumalan valtakunta, ja mihin minä sen vertaisin?
\par 19 Se on sinapinsiemenen kaltainen, jonka mies otti ja kylvi puutarhaansa; ja se kasvoi, ja siitä tuli puu, ja taivaan linnut tekivät pesänsä sen oksille."
\par 20 Ja taas hän sanoi: "Mihin minä vertaisin Jumalan valtakunnan?
\par 21 Se on hapatuksen kaltainen, jonka nainen otti ja sekoitti kolmeen vakalliseen jauhoja, kunnes kaikki happani."
\par 22 Ja hän vaelsi kaupungista kaupunkiin ja kylästä kylään ja opetti, kulkien Jerusalemia kohti.
\par 23 Ja joku kysyi häneltä: "Herra, onko niitä vähän, jotka pelastuvat? Niin hän sanoi heille:
\par 24 "Kilvoitelkaa päästäksenne sisälle ahtaasta ovesta, sillä monet, sanon minä teille, koettavat päästä sisälle, mutta eivät voi.
\par 25 Sen jälkeen kuin perheenisäntä on noussut ja sulkenut oven ja te rupeatte seisomaan ulkona ja kolkuttamaan ovea sanoen: 'Herra, avaa meille', vastaa hän ja sanoo teille: 'En minä tunne teitä enkä tiedä, mistä te olette'.
\par 26 Silloin te rupeatte sanomaan: 'Mehän söimme ja joimme sinun seurassasi, ja meidän kaduillamme sinä opetit'.
\par 27 Mutta hän on lausuva: 'Minä sanon teille: en tiedä, mistä te olette. Menkää pois minun tyköäni, kaikki te vääryyden tekijät.'
\par 28 Siellä on oleva itku ja hammasten kiristys, kun näette Aabrahamin ja Iisakin ja Jaakobin ja kaikkien profeettain olevan Jumalan valtakunnassa, mutta huomaatte itsenne heitetyiksi ulos.
\par 29 Ja tulijoita saapuu idästä ja lännestä ja pohjoisesta ja etelästä, ja he aterioitsevat Jumalan valtakunnassa.
\par 30 Ja katso, on viimeisiä, jotka tulevat ensimmäisiksi, ja on ensimmäisiä, jotka tulevat viimeisiksi."
\par 31 Samalla hetkellä tuli hänen luoksensa muutamia fariseuksia, ja he sanoivat hänelle: "Lähde ja mene täältä pois, sillä Herodes tahtoo tappaa sinut".
\par 32 Niin hän sanoi heille: "Menkää ja sanokaa sille ketulle: 'Katso, minä ajan ulos riivaajia ja parannan sairaita tänään ja huomenna, ja kolmantena päivänä minä pääsen määräni päähän'.
\par 33 Kuitenkin minun pitää vaeltaman tänään ja huomenna ja ylihuomenna, sillä ei sovi, että profeetta saa surmansa muualla kuin Jerusalemissa.
\par 34 Jerusalem, Jerusalem, sinä, joka tapat profeetat ja kivität ne, jotka ovat sinun tykösi lähetetyt, kuinka usein minä olenkaan tahtonut koota sinun lapsesi, niinkuin kana kokoaa poikansa siipiensä alle! Mutta te ette ole tahtoneet.
\par 35 Katso, 'teidän huoneenne on jäävä hyljätyksi'. Mutta minä sanon teille: te ette näe minua, ennenkuin se aika tulee, jolloin te sanotte: 'Siunattu olkoon hän, joka tulee Herran nimeen'."

\chapter{14}

\par 1 Ja kun hän sapattina tuli erään fariseusten johtomiehen taloon aterialle, pitivät he häntä silmällä.
\par 2 Ja katso, siellä oli vesitautinen mies hänen edessään.
\par 3 Niin Jeesus rupesi puhumaan lainoppineille ja fariseuksille ja sanoi: "Onko luvallista parantaa sapattina, vai eikö?" Mutta he olivat vaiti.
\par 4 Ja hän koski mieheen, paransi hänet ja laski menemään.
\par 5 Ja hän sanoi heille: "Jos joltakin teistä putoaa poika tai härkä kaivoon, eikö hän heti vedä sitä ylös sapatinpäivänäkin?"
\par 6 Eivätkä he kyenneet vastaamaan tähän.
\par 7 Ja huomatessaan, kuinka kutsutut valitsivat itselleen ensimmäisiä sijoja, hän puhui heille vertauksen ja sanoi heille:
\par 8 "Kun joku on kutsunut sinut häihin, älä asetu aterioimaan ensimmäiselle sijalle; sillä, jos hän on kutsunut jonkun sinua arvollisemman,
\par 9 niin hän, joka on sinut ja hänet kutsunut, ehkä tulee ja sanoo sinulle: 'Anna tälle sija', ja silloin sinä saat häveten siirtyä viimeiselle paikalle.
\par 10 Vaan kun olet kutsuttu, mene ja asetu viimeiselle sijalle, ja niin on se, joka on sinut kutsunut, sisään tullessaan sanova sinulle: 'Ystäväni, astu ylemmäksi'. Silloin tulee sinulle kunnia kaikkien pöytäkumppaniesi edessä.
\par 11 Sillä jokainen, joka itsensä ylentää, alennetaan, ja joka itsensä alentaa, se ylennetään."
\par 12 Ja hän sanoi myös sille, joka oli hänet kutsunut: "Kun laitat päivälliset tai illalliset, älä kutsu ystäviäsi, älä veljiäsi, älä sukulaisiasi äläkä rikkaita naapureita, etteivät hekin vuorostaan kutsuisi sinua, ja ettet sinä siten saisi maksua.
\par 13 Vaan kun laitat pidot, kutsu köyhiä, raajarikkoja, rampoja, sokeita;
\par 14 niin sinä olet oleva autuas, koska he eivät voi maksaa sinulle; sillä sinulle maksetaan vanhurskasten ylösnousemuksessa."
\par 15 Tämän kuullessaan eräs pöytäkumppaneista sanoi hänelle: "Autuas se, joka saa olla aterialla Jumalan valtakunnassa!"
\par 16 Niin hän sanoi hänelle: "Eräs mies laittoi suuret illalliset ja kutsui monta.
\par 17 Ja illallisajan tullessa hän lähetti palvelijansa sanomaan kutsutuille: 'Tulkaa, sillä kaikki on jo valmiina'.
\par 18 Mutta he rupesivat kaikki järjestään estelemään. Ensimmäinen sanoi hänelle: 'Minä ostin pellon, ja minun täytyy lähteä sitä katsomaan; pyydän sinua, pidä minut estettynä'.
\par 19 Toinen sanoi: 'Minä ostin viisi paria härkiä ja menen niitä koettelemaan; pyydän sinua, pidä minut estettynä'.
\par 20 Vielä toinen sanoi: 'Minä otin vaimon, ja sentähden en voi tulla'.
\par 21 Ja palvelija tuli takaisin ja ilmoitti herralleen tämän. Silloin isäntä vihastui ja sanoi palvelijalleen: 'Mene kiiruusti kaupungin kaduille ja kujille ja tuo köyhät ja raajarikot, sokeat ja rammat tänne sisälle'.
\par 22 Ja palvelija sanoi: 'Herra, on tehty, minkä käskit, ja vielä on tilaa'.
\par 23 Niin Herra sanoi palvelijalle: 'Mene teille ja aitovierille ja pakota heitä tulemaan sisälle, että minun taloni täyttyisi;
\par 24 sillä minä sanon teille, ettei yksikään niistä miehistä, jotka olivat kutsutut, ole maistava minun illallisiani'."
\par 25 Ja hänen mukanaan kulki paljon kansaa; ja hän kääntyi ja sanoi heille:
\par 26 "Jos joku tulee minun tyköni eikä vihaa isäänsä ja äitiänsä ja vaimoaan ja lapsiaan ja veljiään ja sisariaan, vieläpä omaa elämäänsäkin, hän ei voi olla minun opetuslapseni.
\par 27 Ja joka ei kanna ristiänsä ja seuraa minua, se ei voi olla minun opetuslapseni.
\par 28 Sillä jos joku teistä tahtoo rakentaa tornin, eikö hän ensin istu laskemaan kustannuksia, nähdäkseen, onko hänellä varoja rakentaa se valmiiksi,
\par 29 etteivät, kun hän on pannut perustuksen, mutta ei kykene saamaan rakennusta valmiiksi, kaikki, jotka sen näkevät, rupeaisi pilkkaamaan häntä
\par 30 sanoen: 'Tuo mies ryhtyi rakentamaan, mutta ei kyennyt saamaan valmiiksi'?
\par 31 Tahi jos joku kuningas tahtoo lähteä sotimaan toista kuningasta vastaan, eikö hän ensin istu ja pidä neuvoa, kykeneekö hän kymmenellä tuhannella kohtaamaan sitä, joka tulee häntä vastaan kahdellakymmenellä tuhannella?
\par 32 Ja ellei kykene, niin hän, toisen vielä ollessa kaukana, lähettää hänen luoksensa lähettiläät hieromaan rauhaa.
\par 33 Niin ei myös teistä yksikään, joka ei luovu kaikesta, mitä hänellä on, voi olla minun opetuslapseni.
\par 34 Suola on hyvä; mutta jos suolakin käy mauttomaksi, millä se saadaan suolaiseksi?
\par 35 Ei se kelpaa maahan eikä lantaan; se heitetään pois. Jolla on korvat kuulla, se kuulkoon!"

\chapter{15}

\par 1 Ja kaikki publikaanit ja syntiset tulivat hänen tykönsä kuulemaan häntä.
\par 2 Mutta fariseukset ja kirjanoppineet nurisivat ja sanoivat: "Tämä ottaa vastaan syntisiä ja syö heidän kanssaan".
\par 3 Niin hän puhui heille tämän vertauksen sanoen:
\par 4 "Jos jollakin teistä on sata lammasta ja hän kadottaa yhden niistä, eikö hän jätä niitä yhdeksääkymmentä yhdeksää erämaahan ja mene etsimään kadonnutta, kunnes hän sen löytää?
\par 5 Ja löydettyään hän panee sen hartioillensa iloiten.
\par 6 Ja kun hän tulee kotiin, kutsuu hän kokoon ystävänsä ja naapurinsa ja sanoo heille: 'Iloitkaa minun kanssani, sillä minä löysin lampaani, joka oli kadonnut'.
\par 7 Minä sanon teille: samoin on ilo taivaassa suurempi yhdestä syntisestä, joka tekee parannuksen, kuin yhdeksästäkymmenestä yhdeksästä vanhurskaasta, jotka eivät parannusta tarvitse.
\par 8 Tahi jos jollakin naisella on kymmenen hopearahaa ja hän kadottaa yhden niistä, eikö hän sytytä lamppua ja lakaise huonetta ja etsi visusti, kunnes hän sen löytää?
\par 9 Ja löydettyään hän kutsuu kokoon ystävättärensä ja naapurinaiset ja sanoo: 'Iloitkaa minun kanssani, sillä minä löysin rahan, jonka olin kadottanut'.
\par 10 Niin myös, sanon minä teille, on ilo Jumalan enkeleillä yhdestä syntisestä, joka tekee parannuksen."
\par 11 Vielä hän sanoi: "Eräällä miehellä oli kaksi poikaa.
\par 12 Ja nuorempi heistä sanoi isälleen: 'Isä, anna minulle se osa tavaroista, mikä minulle on tuleva'. Niin hän jakoi heille omaisuutensa.
\par 13 Eikä kulunut montakaan päivää, niin nuorempi poika kokosi kaiken omansa ja matkusti pois kaukaiseen maahan; ja siellä hän hävitti tavaransa eläen irstaasti.
\par 14 Mutta kun hän oli kaikki tuhlannut, tuli kova nälkä koko siihen maahan, ja hän alkoi kärsiä puutetta.
\par 15 Ja hän meni ja yhtyi erääseen sen maan kansalaiseen, ja tämä lähetti hänet tiluksilleen kaitsemaan sikoja.
\par 16 Ja hän halusi täyttää vatsansa niillä palkohedelmillä, joita siat söivät, mutta niitäkään ei kukaan hänelle antanut.
\par 17 Niin hän meni itseensä ja sanoi: 'Kuinka monella minun isäni palkkalaisella on yltäkyllin leipää, mutta minä kuolen täällä nälkään!
\par 18 Minä nousen ja menen isäni tykö ja sanon hänelle: Isä, minä olen tehnyt syntiä taivasta vastaan ja sinun edessäsi
\par 19 enkä enää ansaitse, että minua sinun pojaksesi kutsutaan; tee minut yhdeksi palkkalaisistasi.'
\par 20 Ja hän nousi ja meni isänsä tykö. Mutta kun hän vielä oli kaukana, näki hänen isänsä hänet ja armahti häntä, juoksi häntä vastaan ja lankesi hänen kaulaansa ja suuteli häntä hellästi.
\par 21 Mutta poika sanoi hänelle: 'Isä, minä olen tehnyt syntiä taivasta vastaan ja sinun edessäsi enkä enää ansaitse, että minua sinun pojaksesi kutsutaan'.
\par 22 Silloin isä sanoi palvelijoilleen: 'Tuokaa pian parhaat vaatteet ja pukekaa hänet niihin, ja pankaa sormus hänen sormeensa ja kengät hänen jalkaansa;
\par 23 ja noutakaa syötetty vasikka ja teurastakaa. Ja syökäämme ja pitäkäämme iloa,
\par 24 sillä tämä minun poikani oli kuollut ja virkosi eloon, hän oli kadonnut ja on jälleen löytynyt.' Ja he rupesivat iloa pitämään.
\par 25 Mutta hänen vanhempi poikansa oli pellolla. Ja kun hän tuli ja lähestyi kotia, kuuli hän laulun ja karkelon.
\par 26 Ja hän kutsui luoksensa yhden palvelijoista ja tiedusteli, mitä se oli.
\par 27 Tämä sanoi hänelle: 'Sinun veljesi on tullut, ja isäsi teurastutti syötetyn vasikan, kun sai hänet terveenä takaisin'.
\par 28 Niin hän vihastui eikä tahtonut mennä sisälle; mutta hänen isänsä tuli ulos ja puhutteli häntä leppeästi.
\par 29 Mutta hän vastasi ja sanoi isälleen: 'Katso, niin monta vuotta minä olen sinua palvellut enkä ole milloinkaan sinun käskyäsi laiminlyönyt, ja kuitenkaan et ole minulle koskaan antanut vohlaakaan, pitääkseni iloa ystävieni kanssa.
\par 30 Mutta kun tämä sinun poikasi tuli, joka on tuhlannut sinun omaisuutesi porttojen kanssa, niin hänelle sinä teurastit syötetyn vasikan.'
\par 31 Niin hän sanoi hänelle: 'Poikani, sinä olet aina minun tykönäni, ja kaikki, mikä on minun omaani, on sinun.
\par 32 Mutta pitihän nyt riemuita ja iloita, sillä tämä sinun veljesi oli kuollut ja virkosi eloon, hän oli kadonnut ja on jälleen löytynyt.'"

\chapter{16}

\par 1 Ja hän puhui myös opetuslapsilleen: "Oli rikas mies, jolla oli huoneenhaltija, ja hänelle kanneltiin, että tämä hävitti hänen omaisuuttansa.
\par 2 Ja hän kutsui hänet eteensä ja sanoi hänelle: 'Mitä minä kuulenkaan sinusta? Tee tili huoneenhallituksestasi; sillä sinä et saa enää minun huonettani hallita.'
\par 3 Niin huoneenhaltija sanoi mielessään: 'Mitä minä teen, kun isäntäni ottaa minulta pois huoneenhallituksen? Kaivaa minä en jaksa, kerjuuta häpeän.
\par 4 Minä tiedän, mitä teen, että ottaisivat minut taloihinsa, kun minut pannaan pois huoneenhallituksesta.'
\par 5 Ja hän kutsui luoksensa jokaisen herransa velallisista ja sanoi ensimmäiselle: 'Paljonko sinä olet velkaa minun herralleni?'
\par 6 Tämä sanoi: 'Sata astiaa öljyä'. Niin hän sanoi hänelle: 'Tässä on velkakirjasi, istu ja kirjoita pian viisikymmentä'.
\par 7 Sitten hän sanoi toiselle: 'Entä sinä, paljonko sinä olet velkaa?' Tämä sanoi: 'Sata tynnyriä nisuja'. Hän sanoi hänelle: 'Tässä on velkakirjasi, kirjoita kahdeksankymmentä'.
\par 8 Ja herra kehui väärää huoneenhaltijaa siitä, että hän oli menetellyt ovelasti. Sillä tämän maailman lapset ovat omaa sukukuntaansa kohtaan ovelampia kuin valkeuden lapset.
\par 9 Ja minä sanon teille: tehkää itsellenne ystäviä väärällä mammonalla, että he, kun se loppuu, ottaisivat teidät iäisiin majoihin.
\par 10 Joka vähimmässä on uskollinen, on paljossakin uskollinen, ja joka vähimmässä on väärä, on paljossakin väärä.
\par 11 Jos siis ette ole olleet uskolliset väärässä mammonassa, kuka teille uskoo sitä, mikä oikeata on?
\par 12 Ja jos ette ole olleet uskolliset siinä, mikä on toisen omaa, kuka teille antaa sitä, mikä teidän omaanne on?
\par 13 Ei kukaan palvelija voi palvella kahta herraa; sillä hän on joko tätä vihaava ja toista rakastava, taikka tähän liittyvä ja toista halveksiva. Ette voi palvella Jumalaa ja mammonaa."
\par 14 Tämän kaiken kuulivat fariseukset, jotka olivat rahanahneita, ja he ivasivat häntä.
\par 15 Ja hän sanoi heille: "Te juuri olette ne, jotka teette itsenne vanhurskaiksi ihmisten edessä, mutta Jumala tuntee teidän sydämenne; sillä mikä ihmisten kesken on korkeata, se on Jumalan edessä kauhistus.
\par 16 Laki ja profeetat olivat Johannekseen asti; siitä lähtien julistetaan Jumalan valtakuntaa, ja jokainen tunkeutuu sinne väkisin.
\par 17 Mutta ennemmin taivas ja maa katoavat, kuin yksikään lain piirto häviää.
\par 18 Jokainen, joka hylkää vaimonsa ja nai toisen, tekee huorin; ja joka nai miehensä hylkäämän, tekee huorin.
\par 19 Oli rikas mies, joka pukeutui purppuraan ja hienoihin pellavavaatteisiin ja eli joka päivä ilossa loisteliaasti.
\par 20 Mutta eräs köyhä, nimeltä Lasarus, makasi hänen ovensa edessä täynnä paiseita
\par 21 ja halusi ravita itseään niillä muruilla, jotka putosivat rikkaan pöydältä. Ja koiratkin tulivat ja nuolivat hänen paiseitansa.
\par 22 Niin tapahtui, että köyhä kuoli, ja enkelit veivät hänet Aabrahamin helmaan. Ja rikaskin kuoli, ja hänet haudattiin.
\par 23 Ja kun hän nosti silmänsä tuonelassa, vaivoissa ollessaan, näki hän kaukana Aabrahamin ja Lasaruksen hänen helmassaan.
\par 24 Ja hän huusi sanoen: 'Isä Aabraham, armahda minua ja lähetä Lasarus kastamaan sormensa pää veteen ja jäähdyttämään minun kieltäni, sillä minulla on kova tuska tässä liekissä!'
\par 25 Mutta Aabraham sanoi: 'Poikani, muista, että sinä eläessäsi sait hyväsi, ja Lasarus samoin sai pahaa; mutta nyt hän täällä saa lohdutusta, sinä taas kärsit tuskaa.
\par 26 Ja kaiken tämän lisäksi on meidän välillemme ja teidän vahvistettu suuri juopa, että ne, jotka tahtovat mennä täältä teidän luoksenne, eivät voisi, eivätkä ne, jotka siellä ovat, pääsisi yli meidän luoksemme.'
\par 27 Hän sanoi: 'Niin minä siis rukoilen sinua, isä, että lähetät hänet isäni taloon
\par 28 - sillä minulla on viisi veljeä - todistamaan heille, etteivät hekin joutuisi tähän vaivan paikkaan'.
\par 29 Mutta Aabraham sanoi: 'Heillä on Mooses ja profeetat; kuulkoot niitä'.
\par 30 Niin hän sanoi: 'Ei, isä Aabraham; vaan jos joku kuolleista menisi heidän tykönsä, niin he tekisivät parannuksen'.
\par 31 Mutta Aabraham sanoi hänelle: 'Jos he eivät kuule Moosesta ja profeettoja, niin eivät he usko, vaikka joku kuolleistakin nousisi ylös'."

\chapter{17}

\par 1 Ja hän sanoi opetuslapsillensa: "Mahdotonta on, että viettelykset jäisivät tulematta; mutta voi sitä, jonka kautta ne tulevat!
\par 2 Hänen olisi parempi, että myllynkivi pantaisiin hänen kaulaansa ja hänet heitettäisiin mereen, kuin että hän viettelee yhden näistä pienistä.
\par 3 Pitäkää itsestänne vaari! Jos sinun veljesi tekee syntiä, niin nuhtele häntä, ja jos hän katuu, anna hänelle anteeksi.
\par 4 Ja jos hän seitsemän kertaa päivässä tekee syntiä sinua vastaan ja seitsemän kertaa kääntyy sinun puoleesi ja sanoo: 'Minä kadun', niin anna hänelle anteeksi."
\par 5 Ja apostolit sanoivat Herralle: "Lisää meille uskoa".
\par 6 Niin Herra sanoi: "Jos teillä olisi uskoa sinapinsiemenenkään verran, niin te voisitte sanoa tälle silkkiäispuulle: 'Nouse juurinesi ja istuta itsesi mereen', ja se tottelisi teitä.
\par 7 Jos jollakin teistä on palvelija kyntämässä tai paimentamassa, sanooko hän tälle tämän tullessa pellolta: 'Käy heti aterialle'?
\par 8 Eikö hän pikemminkin sano hänelle: 'Valmista minulle ateria, vyöttäydy ja palvele minua, sillä aikaa kuin minä syön ja juon; ja sitten syö ja juo sinä'?
\par 9 Ei kaiketi hän kiitä palvelijaa siitä, että tämä teki, mitä oli käsketty?
\par 10 Niin myös te, kun olette tehneet kaiken, mitä teidän on käsketty tehdä, sanokaa: 'Me olemme ansiottomia palvelijoita; olemme tehneet vain sen, minkä olimme velvolliset tekemään'."
\par 11 Ja kun hän oli matkalla Jerusalemiin, kulki hän Samarian ja Galilean välistä rajaa.
\par 12 Ja hänen mennessään erääseen kylään kohtasi häntä kymmenen pitalista miestä, jotka jäivät seisomaan loitommaksi;
\par 13 ja he korottivat äänensä ja sanoivat: "Jeesus, mestari, armahda meitä!"
\par 14 Ja heidät nähdessään hän sanoi heille: "Menkää ja näyttäkää itsenne papeille". Ja tapahtui heidän mennessään, että he puhdistuivat.
\par 15 Mutta yksi heistä, kun näki olevansa parannettu, palasi takaisin ja ylisti Jumalaa suurella äänellä
\par 16 ja lankesi kasvoilleen hänen jalkojensa juureen ja kiitti häntä; ja se mies oli samarialainen.
\par 17 Niin Jeesus vastasi ja sanoi: "Eivätkö kaikki kymmenen puhdistuneet? Missä ne yhdeksän ovat?
\par 18 Eikö ollut muita, jotka olisivat palanneet Jumalaa ylistämään, kuin tämä muukalainen?"
\par 19 Ja hän sanoi hänelle: "Nouse ja mene; sinun uskosi on sinut pelastanut".
\par 20 Ja kun fariseukset kysyivät häneltä, milloin Jumalan valtakunta oli tuleva, vastasi hän heille ja sanoi: "Ei Jumalan valtakunta tule nähtävällä tavalla,
\par 21 eikä voida sanoa: 'Katso, täällä se on', tahi: 'Tuolla'; sillä katso, Jumalan valtakunta on sisällisesti teissä".
\par 22 Ja hän sanoi opetuslapsillensa: "Tulee aika, jolloin te halajaisitte nähdä edes yhtä Ihmisen Pojan päivää, mutta ette saa nähdä.
\par 23 Ja teille sanotaan: 'Katso, tuolla hän on!' 'Katso, täällä!' Älkää menkö sinne älkääkä juosko perässä.
\par 24 Sillä niinkuin salaman leimaus loistaa taivaan äärestä taivaan ääreen, niin on Ihmisen Poika päivänänsä oleva.
\par 25 Mutta sitä ennen pitää hänen kärsimän paljon ja joutuman tämän sukupolven hyljittäväksi.
\par 26 Ja niinkuin kävi Nooan päivinä, niin käy myöskin Ihmisen Pojan päivinä:
\par 27 he söivät, joivat, naivat ja menivät miehelle, aina siihen päivään asti, jona Nooa meni arkkiin; ja vedenpaisumus tuli ja hukutti heidät kaikki.
\par 28 Niin myös, samoin kuin kävi Lootin päivinä: he söivät, joivat, ostivat, myivät, istuttivat ja rakensivat,
\par 29 mutta sinä päivänä, jona Loot lähti Sodomasta, satoi tulta ja tulikiveä taivaasta, ja se hukutti heidät kaikki,
\par 30 samoin käy sinä päivänä, jona Ihmisen Poika ilmestyy.
\par 31 Sinä päivänä älköön se, joka katolla on ja jolla on tavaransa huoneessa, astuko alas niitä noutamaan; ja älköön myös se, joka pellolla on, palatko takaisin.
\par 32 Muistakaa Lootin vaimoa!
\par 33 Joka tahtoo tallettaa elämänsä itselleen, hän kadottaa sen; mutta joka sen kadottaa, pelastaa sen.
\par 34 Minä sanon teille: sinä yönä on kaksi miestä yhdellä vuoteella; toinen korjataan talteen, ja toinen jätetään.
\par 35 Kaksi naista jauhaa yhdessä; toinen korjataan talteen, mutta toinen jätetään."
\par 36 []
\par 37 Ja he vastasivat ja sanoivat hänelle: "Missä, Herra?" Niin hän sanoi heille: "Missä raato on, sinne myös kotkat kokoontuvat".

\chapter{18}

\par 1 Ja hän puhui heille vertauksen siitä, että heidän tuli aina rukoilla eikä väsyä.
\par 2 Hän sanoi: "Eräässä kaupungissa oli tuomari, joka ei peljännyt Jumalaa eikä hävennyt ihmisiä.
\par 3 Ja siinä kaupungissa oli leskivaimo, joka vähän väliä tuli hänen luoksensa ja sanoi: 'Auta minut oikeuteeni riitapuoltani vastaan'.
\par 4 Mutta pitkään aikaan hän ei tahtonut. Vaan sitten hän sanoi mielessään: 'Vaikka en pelkää Jumalaa enkä häpeä ihmisiä,
\par 5 niin kuitenkin, koska tämä leski tuottaa minulle vaivaa, minä autan hänet oikeuteensa, ettei hän lopulta tulisi ja kävisi minun silmilleni'."
\par 6 Niin Herra sanoi: "Kuulkaa, mitä tuo väärä tuomari sanoo!
\par 7 Eikö sitten Jumala toimittaisi oikeutta valituillensa, jotka häntä yötä päivää avuksi huutavat, ja viivyttäisikö hän heiltä apuansa?
\par 8 Minä sanon teille: hän toimittaa heille oikeuden pian. Kuitenkin, kun Ihmisen Poika tulee, löytäneekö hän uskoa maan päältä?"
\par 9 Niin hän puhui vielä muutamille, jotka luottivat itseensä, luullen olevansa vanhurskaita, ja ylenkatsoivat muita, tämän vertauksen:
\par 10 "Kaksi miestä meni ylös pyhäkköön rukoilemaan, toinen fariseus ja toinen publikaani.
\par 11 Fariseus seisoi ja rukoili itsekseen näin: 'Jumala, minä kiitän sinua, etten minä ole niinkuin muut ihmiset, riistäjät, väärämieliset, huorintekijät, enkä myöskään niinkuin tuo publikaani.
\par 12 Minä paastoan kahdesti viikossa; minä annan kymmenykset kaikista tuloistani.'
\par 13 Mutta publikaani seisoi taampana eikä edes tahtonut nostaa silmiään taivasta kohti, vaan löi rintaansa ja sanoi: 'Jumala, ole minulle syntiselle armollinen'.
\par 14 Minä sanon teille: tämä meni kotiinsa vanhurskaampana kuin se toinen; sillä jokainen, joka itsensä ylentää, alennetaan, mutta joka itsensä alentaa, se ylennetään."
\par 15 Ja he toivat hänen tykönsä myös pieniä lapsia, että hän koskisi heihin; mutta sen nähdessään opetuslapset nuhtelivat tuojia.
\par 16 Mutta Jeesus kutsui lapset tykönsä ja sanoi: "Sallikaa lasten tulla minun tyköni älkääkä estäkö heitä, sillä senkaltaisten on Jumalan valtakunta.
\par 17 Totisesti minä sanon teille: joka ei ota vastaan Jumalan valtakuntaa niinkuin lapsi, se ei pääse sinne sisälle."
\par 18 Ja eräs hallitusmies kysyi häneltä sanoen: "Hyvä opettaja, mitä minun pitää tekemän, että minä iankaikkisen elämän perisin?"
\par 19 Jeesus sanoi hänelle: "Miksi sanot minua hyväksi? Ei kukaan ole hyvä, paitsi Jumala yksin.
\par 20 Käskyt sinä tiedät: 'Älä tee huorin', 'Älä tapa', 'Älä varasta', 'Älä sano väärää todistusta', 'Kunnioita isääsi ja äitiäsi'."
\par 21 Mutta hän sanoi: "Tätä kaikkea minä olen noudattanut nuoruudestani asti".
\par 22 Kun Jeesus sen kuuli, sanoi hän hänelle: "Yksi sinulta vielä puuttuu: myy kaikki, mitä sinulla on, ja jakele köyhille, niin sinulla on oleva aarre taivaissa; ja tule ja seuraa minua".
\par 23 Mutta tämän kuullessaan hän tuli kovin murheelliseksi, sillä hän oli sangen rikas.
\par 24 Kun Jeesus näki hänen olevan murheissaan, sanoi hän: "Kuinka vaikea onkaan niiden, joilla on tavaraa, päästä Jumalan valtakuntaan!
\par 25 Helpompi on kamelin käydä neulansilmän läpi kuin rikkaan päästä Jumalan valtakuntaan."
\par 26 Niin ne, jotka sen kuulivat, sanoivat: "Kuka sitten voi pelastua?"
\par 27 Mutta hän sanoi: "Mikä ihmisille on mahdotonta, se on Jumalalle mahdollista".
\par 28 Silloin Pietari sanoi: "Katso, me olemme luopuneet siitä, mitä meillä oli, ja seuranneet sinua".
\par 29 Niin hän sanoi heille: "Totisesti minä sanon teille: ei ole ketään, joka Jumalan valtakunnan tähden on luopunut talosta tai vaimosta tai veljistä tai vanhemmista tai lapsista,
\par 30 ja joka ei saisi monin verroin takaisin tässä ajassa, ja tulevassa maailmassa iankaikkista elämää".
\par 31 Ja hän otti tykönsä ne kaksitoista ja sanoi heille: "Katso, me menemme ylös Jerusalemiin, ja kaikki on täysin toteutuva, mitä profeettain kautta on kirjoitettu Ihmisen Pojasta.
\par 32 Sillä hänet annetaan pakanain käsiin, ja häntä pilkataan ja häväistään ja syljetään;
\par 33 ja ruoskittuaan he tappavat hänet, ja kolmantena päivänä hän nousee ylös."
\par 34 Mutta he eivät ymmärtäneet tästä mitään, ja tämä puhe oli heiltä niin salattu, etteivät he käsittäneet, mitä sanottiin.
\par 35 Ja hänen lähestyessään Jerikoa eräs sokea istui tien vieressä kerjäten.
\par 36 Ja kuullessaan, että siitä kulki kansaa ohi, hän kyseli, mitä se oli.
\par 37 He ilmoittivat hänelle Jeesuksen, Nasaretilaisen, menevän ohitse.
\par 38 Niin hän huusi sanoen: "Jeesus, Daavidin poika, armahda minua!"
\par 39 Ja edelläkulkijat nuhtelivat häntä saadakseen hänet vaikenemaan; mutta hän huusi vielä enemmän: "Daavidin poika, armahda minua!"
\par 40 Silloin Jeesus seisahtui ja käski taluttaa hänet tykönsä. Ja hänen tultuaan lähelle Jeesus kysyi häneltä:
\par 41 "Mitä tahdot, että minä sinulle tekisin?" Hän sanoi: "Herra, että saisin näköni jälleen".
\par 42 Niin Jeesus sanoi hänelle: "Saa näkösi; sinun uskosi on sinut pelastanut".
\par 43 Ja heti hän sai näkönsä ja seurasi häntä ylistäen Jumalaa. Ja sen nähdessään kaikki kansa kiitti Jumalaa.

\chapter{19}

\par 1 Ja hän tuli Jerikon kaupunkiin ja kulki sen läpi.
\par 2 Ja katso, siellä oli mies, nimeltä Sakkeus; ja hän oli publikaanien päämies ja oli rikas.
\par 3 Ja hän koetti saada nähdä Jeesusta, kuka hän oli, mutta ei voinut kansalta, kun oli varreltansa vähäinen.
\par 4 Niin hän juoksi edelle ja nousi metsäviikunapuuhun nähdäkseen hänet, sillä Jeesus oli kulkeva siitä ohitse.
\par 5 Ja tultuaan sille paikalle Jeesus katsahti ylös ja sanoi hänelle: "Sakkeus, tule nopeasti alas, sillä tänään minun pitää oleman sinun huoneessasi".
\par 6 Ja hän tuli nopeasti alas ja otti hänet iloiten vastaan.
\par 7 Ja sen nähdessään kaikki nurisivat sanoen: "Syntisen miehen luokse hän meni majailemaan".
\par 8 Mutta Sakkeus astui esiin ja sanoi Herralle: "Katso, Herra, puolet omaisuudestani minä annan köyhille, ja jos joltakulta olen jotakin petoksella ottanut, niin annan nelinkertaisesti takaisin".
\par 9 Niin Jeesus sanoi hänestä: "Tänään on pelastus tullut tälle huoneelle, koska hänkin on Aabrahamin poika;
\par 10 sillä Ihmisen Poika on tullut etsimään ja pelastamaan sitä, mikä kadonnut on".
\par 11 Ja heidän tätä kuunnellessaan hän puhui vielä vertauksen, koska hän oli lähellä Jerusalemia ja he luulivat, että Jumalan valtakunta oli kohta ilmestyvä.
\par 12 Hän sanoi näin: "Eräs jalosukuinen mies lähti matkalle kaukaiseen maahan saadakseen itsellensä kuninkuuden ja sitten palatakseen.
\par 13 Ja hän kutsui luoksensa kymmenen palvelijaansa, antoi heille kymmenen leiviskää ja sanoi heille: 'Asioikaa näillä, kunnes minä tulen'.
\par 14 Mutta hänen kansalaisensa vihasivat häntä ja lähettivät lähettiläät hänen jälkeensä sanomaan: 'Emme tahdo tätä kuninkaaksemme'.
\par 15 Ja saatuansa kuninkuuden ja palattuansa hän käski kutsua eteensä ne palvelijat, joille hän oli antanut rahat, saadakseen tietää, mitä kukin oli asioimisellaan ansainnut.
\par 16 Niin ensimmäinen tuli esiin ja sanoi: 'Herra, sinun leiviskäsi on tuottanut kymmenen leiviskää'.
\par 17 Ja hän sanoi hänelle: 'Hyvä on, sinä hyvä palvelija; koska vähimmässä olet ollut uskollinen, niin saat vallita kymmentä kaupunkia'.
\par 18 Ja toinen tuli ja sanoi: 'Herra, sinun leiviskäsi on tuottanut viisi leiviskää'.
\par 19 Niin hän sanoi tällekin: 'Sinä, vallitse sinä viittä kaupunkia'.
\par 20 Vielä tuli yksi ja sanoi: 'Herra, katso, tässä on sinun leiviskäsi, jota olen säilyttänyt liinasessa.
\par 21 Sillä minä pelkäsin sinua, koska olet ankara mies: sinä otat, mitä et ole talteen pannut, ja leikkaat, mitä et ole kylvänyt.'
\par 22 Hän sanoi hänelle: 'Oman sanasi mukaan minä sinut tuomitsen, sinä paha palvelija. Sinä tiesit minut ankaraksi mieheksi, joka otan, mitä en ole talteen pannut, ja leikkaan, mitä en ole kylvänyt;
\par 23 miksi et siis antanut rahojani rahanvaihtajan pöytään, että minä tultuani olisin saanut periä ne korkoineen?'
\par 24 Ja hän sanoi vieressä seisoville: 'Ottakaa häneltä pois se leiviskä ja antakaa sille, jolla on kymmenen leiviskää'.
\par 25 - Niin he sanoivat hänelle: 'Herra, hänellä on jo kymmenen leiviskää'. -
\par 26 'Minä sanon teille: jokaiselle, jolla on, annetaan; mutta jolla ei ole, siltä otetaan pois sekin, mikä hänellä on.
\par 27 Mutta viholliseni, jotka eivät tahtoneet minua kuninkaaksensa, tuokaa tänne ja teloittakaa minun edessäni.'"
\par 28 Ja tämän sanottuaan hän kulki edellä vaeltaen ylös Jerusalemiin.
\par 29 Ja tapahtui, kun hän tuli lähelle Beetfagea ja Betaniaa, sille vuorelle, jota kutsutaan Öljymäeksi, että hän lähetti kaksi opetuslastaan
\par 30 sanoen: "Menkää edessä olevaan kylään, niin sinne tullessanne te löydätte sidottuna varsan, jonka selässä ei vielä yksikään ihminen ole istunut; päästäkää se ja tuokaa tänne.
\par 31 Ja jos joku kysyy teiltä: 'Miksi te sen päästätte?' niin sanokaa näin: 'Herra tarvitsee sitä'."
\par 32 Ja lähetetyt menivät ja havaitsivat niin olevan, kuin hän oli heille sanonut.
\par 33 Ja heidän päästäessään varsaa sen omistajat sanoivat heille: "Miksi te päästätte varsan?"
\par 34 Niin he sanoivat: "Herra tarvitsee sitä".
\par 35 Ja he veivät sen Jeesuksen luo ja heittivät vaatteensa varsan selkään ja istuttivat Jeesuksen niiden päälle.
\par 36 Ja hänen kulkiessaan kansa levitti vaatteensa tielle.
\par 37 Ja kun hän jo oli lähellä, laskeutuen Öljymäen rinnettä, rupesi koko opetuslasten joukko iloiten kiittämään Jumalaa suurella äänellä kaikista voimallisista teoista, jotka he olivat nähneet,
\par 38 sanoen: "Siunattu olkoon hän, joka tulee, Kuningas Herran nimessä; rauha taivaassa ja kunnia korkeuksissa!"
\par 39 Ja muutamat fariseukset kansanjoukosta sanoivat hänelle: "Opettaja, nuhtele opetuslapsiasi".
\par 40 Mutta hän vastasi ja sanoi: "Minä sanon teille: jos nämä olisivat vaiti, niin kivet huutaisivat".
\par 41 Ja kun hän tuli lähemmäksi ja näki kaupungin, itki hän sitä
\par 42 ja sanoi: "Jospa tietäisit sinäkin tänä päivänä, mikä rauhaasi sopii! Mutta nyt se on sinun silmiltäsi salattu.
\par 43 Sillä sinulle tulevat ne päivät, jolloin sinun vihollisesi sinut vallilla saartavat ja piirittävät sinut ja ahdistavat sinua joka puolelta;
\par 44 ja he kukistavat sinut maan tasalle ja surmaavat lapsesi, jotka sinussa ovat, eivätkä jätä sinuun kiveä kiven päälle, sentähden ettet etsikkoaikaasi tuntenut."
\par 45 Ja hän meni pyhäkköön ja rupesi ajamaan myyjiä ulos
\par 46 ja sanoi heille: "Kirjoitettu on: 'Minun huoneeni on oleva rukoushuone', mutta te olette tehneet siitä ryövärien luolan."
\par 47 Ja hän opetti joka päivä pyhäkössä. Mutta ylipapit ja kirjanoppineet sekä kansan ensimmäiset miettivät, miten saisivat hänet surmatuksi;
\par 48 mutta he eivät keksineet, mitä tekisivät, sillä kaikki kansa riippui hänessä ja kuunteli häntä.

\chapter{20}

\par 1 Ja tapahtui eräänä päivänä, kun hän opetti kansaa pyhäkössä ja julisti evankeliumia, että ylipapit ja kirjanoppineet astuivat yhdessä vanhinten kanssa esiin
\par 2 ja puhuivat hänelle sanoen: "Sano meille, millä vallalla sinä näitä teet, tahi kuka on se, joka on antanut sinulle tämän vallan?"
\par 3 Hän vastasi ja sanoi heille: "Minä myös teen teille kysymyksen; sanokaa minulle:
\par 4 oliko Johanneksen kaste taivaasta vai ihmisistä?"
\par 5 Niin he neuvottelivat keskenänsä sanoen: "Jos sanomme: 'Taivaasta', niin hän sanoo: 'Miksi ette siis uskoneet häntä?'
\par 6 Mutta jos sanomme: 'Ihmisistä', niin kaikki kansa kivittää meidät, sillä se uskoo vahvasti, että Johannes oli profeetta."
\par 7 Ja he vastasivat, etteivät tienneet, mistä se oli.
\par 8 Niin Jeesus sanoi heille: "Niinpä en minäkään sano teille, millä vallalla minä näitä teen".
\par 9 Ja hän rupesi puhumaan kansalle ja puhui tämän vertauksen: "Mies istutti viinitarhan ja vuokrasi sen viinitarhureille ja matkusti muille maille kauaksi aikaa.
\par 10 Ja ajan tullen hän lähetti palvelijan viinitarhurien luokse, että he antaisivat tälle osan viinitarhan hedelmistä; mutta viinitarhurit pieksivät hänet ja lähettivät tyhjin käsin pois.
\par 11 Ja hän lähetti vielä toisen palvelijan; mutta hänetkin he pieksivät ja häpäisivät ja lähettivät tyhjin käsin pois.
\par 12 Ja hän lähetti vielä kolmannen; mutta tämänkin he haavoittivat ja heittivät ulos.
\par 13 Niin viinitarhan herra sanoi: 'Mitä minä teen? Minä lähetän rakkaan poikani; häntä he kaiketi kavahtavat.'
\par 14 Mutta kun viinitarhurit näkivät hänet, neuvottelivat he keskenään ja sanoivat: 'Tämä on perillinen; tappakaamme hänet, että perintö tulisi meidän omaksemme'.
\par 15 Ja he heittivät hänet ulos viinitarhasta ja tappoivat. Mitä nyt viinitarhan herra on tekevä heille?
\par 16 Hän tulee ja tuhoaa nämä viinitarhurit ja antaa viinitarhan muille." Niin he sen kuullessaan sanoivat: "Pois se!"
\par 17 Mutta hän katsahti heihin ja sanoi: "Mitä siis on tämä kirjoitus: 'Se kivi, jonka rakentajat hylkäsivät, on tullut kulmakiveksi'?
\par 18 Jokainen, joka kaatuu siihen kiveen, ruhjoutuu, mutta jonka päälle se kaatuu, sen se murskaa."
\par 19 Ja kirjanoppineet ja ylipapit tahtoivat ottaa hänet sillä hetkellä kiinni, mutta he pelkäsivät kansaa; sillä he ymmärsivät, että hän oli puhunut sen vertauksen heistä.
\par 20 Ja he vartioivat häntä ja lähettivät hänen luokseen hurskaiksi tekeytyviä urkkijoita, saadakseen hänet kiinni jostakin sanasta, niin että voisivat antaa hänet esivallalle ja maaherran käsiin.
\par 21 Ja ne kysyivät häneltä sanoen: "Opettaja, me tiedämme, että sinä puhut ja opetat oikein etkä katso henkilöön, vaan opetat Jumalan tietä totuudessa.
\par 22 Onko meidän lupa antaa keisarille veroa vai eikö?"
\par 23 Mutta hän havaitsi heidän kavaluutensa ja sanoi heille:
\par 24 "Näyttäkää minulle denari. Kenen kuva ja päällekirjoitus siinä on?" He vastasivat: "Keisarin".
\par 25 Niin hän sanoi heille: "Antakaa siis keisarille, mikä keisarin on, ja Jumalalle, mikä Jumalan on".
\par 26 Ja he eivät kyenneet saamaan häntä hänen puheestaan kiinni kansan edessä; ja he ihmettelivät hänen vastaustaan ja vaikenivat.
\par 27 Niin astui esiin muutamia saddukeuksia, jotka väittävät, ettei ylösnousemusta ole, ja he kysyivät häneltä
\par 28 sanoen: "Opettaja, Mooses on säätänyt meille: 'Jos joltakin kuolee veli, jolla on vaimo, mutta ei ole lapsia, niin ottakoon hän veljensä vaimon ja herättäköön siemenen veljelleen'.
\par 29 Nyt oli seitsemän veljestä. Ensimmäinen otti vaimon ja kuoli lapsetonna.
\par 30 Niin toinen otti sen vaimon,
\par 31 ja sitten kolmas, ja samoin kaikki seitsemän; ja he kuolivat jättämättä lapsia.
\par 32 Viimeiseksi vaimokin kuoli.
\par 33 Kenelle heistä siis tämä vaimo ylösnousemuksessa joutuu vaimoksi? Sillä kaikkien seitsemän vaimona hän oli ollut."
\par 34 Niin Jeesus sanoi heille: "Tämän maailmanajan lapset naivat ja menevät miehelle.
\par 35 Mutta ne, jotka on arvollisiksi nähty pääsemään toiseen maailmaan ja ylösnousemukseen kuolleista, eivät nai eivätkä mene miehelle.
\par 36 Sillä he eivät enää voi kuolla, kun ovat enkelien kaltaisia; ja he ovat Jumalan lapsia, koska ovat ylösnousemuksen lapsia.
\par 37 Mutta että kuolleet nousevat ylös, sen Mooseskin on osoittanut kertomuksessa orjantappurapensaasta, kun hän sanoo Herraa Aabrahamin Jumalaksi ja Iisakin Jumalaksi ja Jaakobin Jumalaksi.
\par 38 Mutta hän ei ole kuolleitten Jumala, vaan elävien; sillä kaikki hänelle elävät."
\par 39 Niin muutamat kirjanoppineista vastasivat ja sanoivat: "Opettaja, oikein sinä sanoit".
\par 40 Ja he eivät enää rohjenneet kysyä häneltä mitään.
\par 41 Niin hän sanoi heille: "Kuinka he voivat sanoa, että Kristus on Daavidin poika?
\par 42 Sanoohan Daavid itse psalmien kirjassa: 'Herra sanoi minun Herralleni: Istu minun oikealle puolelleni,
\par 43 kunnes minä panen sinun vihollisesi sinun jalkojesi astinlaudaksi.'
\par 44 Daavid siis kutsuu häntä Herraksi; kuinka hän sitten on hänen poikansa?"
\par 45 Ja kaiken kansan kuullen hän sanoi opetuslapsillensa:
\par 46 "Kavahtakaa kirjanoppineita, jotka mielellään käyskelevät pitkissä vaipoissa ja haluavat tervehdyksiä toreilla ja etumaisia istuimia synagoogissa ja ensimmäisiä sijoja pidoissa,
\par 47 noita, jotka syövät leskien huoneet ja näön vuoksi pitävät pitkiä rukouksia; he saavat sitä kovemman tuomion".

\chapter{21}

\par 1 Ja hän katsahti ja näki rikkaiden panevan lahjoja uhriarkkuun.
\par 2 Niin hän näki myös köyhän lesken panevan siihen kaksi ropoa.
\par 3 Silloin hän sanoi: "Totisesti minä sanon teille: tämä köyhä leski pani enemmän kuin kaikki muut.
\par 4 Sillä kaikki nuo panivat lahjansa liiastaan, mutta tämä pani puutteestaan, koko elämisensä, mikä hänellä oli."
\par 5 Ja kun muutamat puhuivat pyhäköstä, kuinka se oli kauniilla kivillä ja temppelilahjoilla kaunistettu, sanoi hän:
\par 6 "Päivät tulevat, jolloin tästä, mitä katselette, ei ole jäävä kiveä kiven päälle, maahan jaottamatta".
\par 7 Niin he kysyivät häneltä sanoen: "Opettaja, milloin tämä sitten tapahtuu? Ja mikä on oleva merkki tämän tulemisesta?"
\par 8 Niin hän sanoi: "Katsokaa, ettei teitä eksytetä. Sillä monta tulee minun nimessäni sanoen: 'Minä olen se', ja: 'Aika on lähellä'. Mutta älkää menkö heidän perässään.
\par 9 Ja kun kuulette sotien ja kapinain melskettä, älkää peljästykö. Sillä näitten täytyy ensin tapahtua, mutta loppu ei tule vielä heti."
\par 10 Sitten hän sanoi heille: "Kansa nousee kansaa vastaan ja valtakunta valtakuntaa vastaan,
\par 11 ja tulee suuria maanjäristyksiä, tulee ruttoa ja nälänhätää monin paikoin, ja taivaalla on oleva peljättäviä näkyjä ja suuria merkkejä.
\par 12 Mutta ennen tätä kaikkea he käyvät teihin käsiksi ja vainoavat teitä ja vetävät teidät synagoogiin ja heittävät vankiloihin ja vievät teidät kuningasten ja maaherrain eteen minun nimeni tähden.
\par 13 Ja näin te joudutte todistamaan.
\par 14 Pankaa siis sydämellenne, ettette edeltäpäin huolehdi, miten te vastaatte puolestanne.
\par 15 Sillä minä annan teille suun ja viisauden, jota vastaan eivät ketkään teidän vastustajanne kykene asettumaan tai väittämään.
\par 16 Omat vanhemmatkin ja veljet ja sukulaiset ja ystävät antavat teidät alttiiksi; ja muutamia teistä tapetaan,
\par 17 ja te joudutte kaikkien vihattaviksi minun nimeni tähden.
\par 18 Mutta ei hiuskarvaakaan teidän päästänne katoa.
\par 19 Kestäväisyydellänne te voitatte omaksenne elämän.
\par 20 Mutta kun te näette Jerusalemin sotajoukkojen ympäröimänä, silloin tietäkää, että sen hävitys on lähellä.
\par 21 Silloin ne, jotka Juudeassa ovat, paetkoot vuorille, ja jotka ovat kaupungissa, lähtekööt sieltä pois, ja jotka maalla ovat, älkööt sinne menkö.
\par 22 Sillä ne ovat koston päiviä, että kaikki täyttyisi, mikä kirjoitettu on.
\par 23 Voi raskaita ja imettäväisiä niinä päivinä! Sillä suuri hätä on oleva maan päällä ja viha tätä kansaa vastaan;
\par 24 ja he kaatuvat miekan terään, heidät viedään vangeiksi kaikkien kansojen sekaan, ja Jerusalem on oleva pakanain tallattavana, kunnes pakanain ajat täyttyvät.
\par 25 Ja on oleva merkit auringossa ja kuussa ja tähdissä, ja ahdistus kansoilla maan päällä ja epätoivo, kun meri ja aallot pauhaavat.
\par 26 Ja ihmiset menehtyvät peljätessään ja odottaessaan sitä, mikä maanpiiriä kohtaa; sillä taivaitten voimat järkkyvät.
\par 27 Ja silloin he näkevät Ihmisen Pojan tulevan pilvessä suurella voimalla ja kirkkaudella.
\par 28 Mutta kun nämä alkavat tapahtua, niin rohkaiskaa itsenne ja nostakaa päänne, sillä teidän vapautuksenne on lähellä."
\par 29 Ja hän puhui heille vertauksen: "Katsokaa viikunapuuta ja kaikkia puita.
\par 30 Kun ne jo puhkeavat lehteen, niin siitä te näette ja itsestänne ymmärrätte, että kesä jo on lähellä.
\par 31 Samoin te myös, kun näette tämän tapahtuvan, tietäkää, että Jumalan valtakunta on lähellä.
\par 32 Totisesti minä sanon teille: tämä sukupolvi ei katoa, ennenkuin kaikki tapahtuu.
\par 33 Taivas ja maa katoavat, mutta minun sanani eivät katoa.
\par 34 Mutta pitäkää vaari itsestänne, ettei teidän sydämiänne raskauta päihtymys ja juoppous eikä elatuksen murheet, niin että se päivä yllättää teidät äkkiarvaamatta
\par 35 niinkuin paula; sillä se on saavuttava kaikki, jotka koko maan päällä asuvat.
\par 36 Valvokaa siis joka aika ja rukoilkaa, että saisitte voimaa paetaksenne tätä kaikkea, mikä tuleva on, ja seisoaksenne Ihmisen Pojan edessä."
\par 37 Ja hän opetti päivät pyhäkössä, mutta öiksi hän lähti pois ja vietti ne vuorella, jota kutsutaan Öljymäeksi.
\par 38 Ja kaikki kansa tuli varhain aamuisin hänen tykönsä pyhäkköön kuulemaan häntä.

\chapter{22}

\par 1 Mutta happamattoman leivän juhla, jota pääsiäiseksi sanotaan, oli lähellä.
\par 2 Ja ylipapit ja kirjanoppineet miettivät, kuinka saisivat hänet surmatuksi; sillä he pelkäsivät kansaa.
\par 3 Niin saatana meni Juudaaseen, jota kutsuttiin Iskariotiksi ja joka oli yksi niistä kahdestatoista.
\par 4 Ja tämä meni ja puhui ylipappien ja pyhäkön vartioston päällikköjen kanssa, miten hän saattaisi hänet heidän käsiinsä.
\par 5 Ja he ihastuivat ja sitoutuivat antamaan hänelle rahaa.
\par 6 Ja hän lupautui ja etsi sopivaa tilaisuutta kavaltaakseen hänet heille ilman melua.
\par 7 Niin tuli se happamattoman leivän päivistä, jona pääsiäislammas oli teurastettava.
\par 8 Ja hän lähetti Pietarin ja Johanneksen sanoen: "Menkää ja valmistakaa meille pääsiäislammas syödäksemme".
\par 9 Niin he kysyivät häneltä: "Mihin tahdot, että valmistamme sen?"
\par 10 Hän vastasi heille: "Katso, saapuessanne kaupunkiin tulee teitä vastaan mies kantaen vesiastiaa; seuratkaa häntä siihen taloon, johon hän menee,
\par 11 ja sanokaa talon isännälle: 'Opettaja sanoo sinulle: Missä on vierashuone, syödäkseni siinä pääsiäislampaan opetuslasteni kanssa?'
\par 12 Niin hän näyttää teille suuren, aterioiville varustetun huoneen yläkerrassa; sinne valmistakaa."
\par 13 Ja he menivät ja havaitsivat niin olevan, kuin Jeesus oli heille sanonut, ja valmistivat pääsiäislampaan.
\par 14 Ja kun hetki tuli, asettui hän aterialle ja apostolit hänen kanssansa.
\par 15 Ja hän sanoi heille: "Minä olen halajamalla halannut syödä tämän pääsiäislampaan teidän kanssanne, ennenkuin minä kärsin;
\par 16 sillä minä sanon teille, etten minä sitä enää syö, ennenkuin sen täyttymys tapahtuu Jumalan valtakunnassa".
\par 17 Ja hän otti maljan, kiitti ja sanoi: "Ottakaa tämä ja jakakaa keskenänne.
\par 18 Sillä minä sanon teille: tästedes minä en juo viinipuun antia, ennenkuin Jumalan valtakunta tulee."
\par 19 Ja hän otti leivän, kiitti, mursi ja antoi heille ja sanoi: "Tämä on minun ruumiini, joka teidän edestänne annetaan. Tehkää se minun muistokseni."
\par 20 Samoin myös maljan, aterian jälkeen, ja sanoi: "Tämä malja on uusi liitto minun veressäni, joka teidän edestänne vuodatetaan.
\par 21 Mutta, katso, minun kavaltajani käsi on minun kanssani pöydällä.
\par 22 Sillä Ihmisen Poika tosin menee pois, niinkuin säädetty on; mutta voi sitä ihmistä, jonka kautta hänet kavalletaan!"
\par 23 Ja he rupesivat keskenänsä kyselemään, kuka heistä mahtoi olla se, joka oli tämän tekevä.
\par 24 Ja heidän välillään syntyi myös kiista siitä, kuka heistä oli katsottava suurimmaksi.
\par 25 Niin hän sanoi heille: "Kansojen kuninkaat herroina niitä hallitsevat, ja niiden valtiaita sanotaan hyväntekijöiksi.
\par 26 Mutta älkää te niin; vaan joka teidän keskuudessanne on suurin, se olkoon niinkuin nuorin, ja johtaja niinkuin se, joka palvelee.
\par 27 Sillä kumpi on suurempi, sekö, joka aterioi, vai se, joka palvelee? Eikö se, joka aterioi? Mutta minä olen teidän keskellänne niinkuin se, joka palvelee.
\par 28 Mutta te olette pysyneet minun kanssani minun kiusauksissani;
\par 29 ja minä säädän teille, niinkuin minun Isäni on minulle säätänyt, kuninkaallisen vallan,
\par 30 niin että te saatte syödä ja juoda minun pöydässäni minun valtakunnassani ja istua valtaistuimilla ja tuomita Israelin kahtatoista sukukuntaa.
\par 31 Simon, Simon, katso, saatana on tavoitellut teitä valtaansa, seuloakseen teitä niinkuin nisuja;
\par 32 mutta minä olen rukoillut sinun puolestasi, ettei sinun uskosi raukeaisi tyhjään. Ja kun sinä kerran palajat, niin vahvista veljiäsi."
\par 33 Niin Simon sanoi hänelle: "Herra, sinun kanssasi minä olen valmis menemään sekä vankeuteen että kuolemaan".
\par 34 Mutta hän sanoi: "Minä sanon sinulle, Pietari: ei laula tänään kukko, ennenkuin sinä kolmesti kiellät tuntevasi minua."
\par 35 Ja hän sanoi heille: "Kun minä lähetin teidät ilman rahakukkaroa ja laukkua ja kenkiä, puuttuiko teiltä mitään?" He vastasivat: "Ei mitään".
\par 36 Niin hän sanoi heille: "Mutta nyt, jolla on kukkaro, ottakoon sen mukaansa; niin myös laukun. Ja jolla ei ole, myyköön vaippansa ja ostakoon miekan.
\par 37 Sillä minä sanon teille, että minussa pitää käymän toteen tämän, mikä kirjoitettu on: 'Ja hänet luettiin pahantekijäin joukkoon'. Sillä se, mikä minusta on sanottu, on täyttynyt."
\par 38 Niin he sanoivat: "Herra, katso, tässä on kaksi miekkaa". Mutta hän vastasi heille: "Riittää".
\par 39 Ja hän meni ulos ja lähti tapansa mukaan Öljymäelle, ja hänen opetuslapsensa seurasivat häntä.
\par 40 Ja tultuaan siihen paikkaan hän sanoi heille: "Rukoilkaa, ettette joutuisi kiusaukseen".
\par 41 Ja hän vetäytyi heistä noin kivenheiton päähän, laskeutui polvilleen ja rukoili
\par 42 sanoen: "Isä, jos sinä tahdot, niin ota pois minulta tämä malja; älköön kuitenkaan tapahtuko minun tahtoni, vaan sinun".
\par 43 Niin hänelle ilmestyi taivaasta enkeli, joka vahvisti häntä.
\par 44 Ja kun hän oli suuressa tuskassa, rukoili hän yhä hartaammin. Ja hänen hikensä oli niinkuin veripisarat, jotka putosivat maahan.
\par 45 Ja kun hän nousi rukoilemasta ja meni opetuslastensa tykö, tapasi hän heidät murheen tähden nukkumasta.
\par 46 Niin hän sanoi heille: "Miksi te nukutte? Nouskaa ja rukoilkaa, ettette joutuisi kiusaukseen."
\par 47 Ja katso, hänen vielä puhuessaan tuli joukko kansaa, ja yksi niistä kahdestatoista, se, jonka nimi oli Juudas, kulki heidän edellään. Ja hän tuli Jeesuksen luo antamaan hänelle suuta.
\par 48 Mutta Jeesus sanoi hänelle: "Juudas, suunantamisellako sinä Ihmisen Pojan kavallat?"
\par 49 Kun nyt ne, jotka olivat hänen ympärillään, näkivät, mitä oli tulossa, sanoivat he: "Herra, iskemmekö miekalla?"
\par 50 Ja eräs heistä iski ylimmäisen papin palvelijaa ja sivalsi häneltä pois oikean korvan.
\par 51 Mutta Jeesus vastasi sanoen: "Sallikaa vielä tämäkin". Ja hän koski hänen korvaansa ja paransi hänet.
\par 52 Niin Jeesus sanoi ylipapeille ja pyhäkön vartioston päälliköille ja vanhimmille, jotka olivat tulleet häntä vastaan: "Niinkuin ryöväriä vastaan te olette lähteneet miekat ja seipäät käsissä.
\par 53 Minä olen joka päivä ollut teidän kanssanne pyhäkössä, ettekä ole ojentaneet käsiänne minua vastaan. Mutta tämä on teidän hetkenne ja pimeyden valta."
\par 54 Niin he ottivat hänet kiinni ja kuljettivat pois ja veivät hänet ylimmäisen papin taloon. Ja Pietari seurasi taampana.
\par 55 Ja he virittivät valkean keskelle esipihaa ja asettuivat yhdessä istumaan, ja Pietari istui heidän joukkoonsa.
\par 56 Niin eräs palvelijatar, nähdessään hänen istuvan tulen ääressä, katseli häntä kiinteästi ja sanoi: "Tämäkin oli hänen kanssaan".
\par 57 Mutta hän kielsi sanoen: "Nainen, en tunne häntä".
\par 58 Ja hetkisen perästä näki hänet toinen, eräs mies, ja sanoi: "Sinäkin olet yksi niistä". Mutta Pietari sanoi: "Mies, en ole".
\par 59 Ja noin yhden hetken kuluttua vakuutti vielä toinen sanoen: "Totisesti, tämä oli myös hänen kanssaan; sillä onhan hän galilealainenkin".
\par 60 Mutta Pietari sanoi: "En ymmärrä, mies, mitä sanot". Ja samassa, hänen vielä puhuessaan, lauloi kukko.
\par 61 Ja Herra kääntyi ja katsoi Pietariin; ja Pietari muisti Herran sanat, kuinka hän oli hänelle sanonut: "Ennenkuin kukko tänään laulaa, sinä kolmesti minut kiellät".
\par 62 Ja hän meni ulos ja itki katkerasti.
\par 63 Ja miehet, jotka pitivät Jeesusta kiinni, pilkkasivat häntä ja pieksivät häntä.
\par 64 Ja he peittivät hänen kasvonsa ja kysyivät häneltä sanoen: "Profetoi, kuka se on, joka sinua löi!"
\par 65 Ja paljon muita herjaussanoja he puhuivat häntä vastaan.
\par 66 Ja päivän valjetessa kansan vanhimmat ja ylipapit ja kirjanoppineet kokoontuivat ja veivät hänet neuvostonsa eteen
\par 67 ja sanoivat: "Jos sinä olet Kristus, niin sano se meille". Hän vastasi heille: "Jos minä teille sanon, niin te ette usko;
\par 68 ja jos kysyn, ette vastaa.
\par 69 Mutta tästedes Ihmisen Poika on istuva Jumalan voiman oikealla puolella."
\par 70 Silloin he kaikki sanoivat: "Sinä siis olet Jumalan Poika?" Hän vastasi heille: "Tepä sen sanotte, että minä olen".
\par 71 Niin he sanoivat: "Mitä me enää todistusta tarvitsemme? Sillä me itse olemme kuulleet sen hänen omasta suustansa."

\chapter{23}

\par 1 Ja he nousivat, koko joukko, ja veivät hänet Pilatuksen eteen.
\par 2 Ja he alkoivat syyttää häntä sanoen: "Tämän me olemme havainneet villitsevän kansaamme, kieltävän antamasta veroja keisarille ja sanovan itseään Kristukseksi, kuninkaaksi".
\par 3 Niin Pilatus kysyi häneltä sanoen: "Oletko sinä juutalaisten kuningas?" Hän vastasi hänelle ja sanoi: "Sinäpä sen sanot".
\par 4 Pilatus sanoi ylipapeille ja kansalle: "En minä löydä mitään syytä tässä miehessä".
\par 5 Mutta he ahdistivat yhä enemmän ja sanoivat: "Hän yllyttää kansaa opettaen kaikkialla Juudeassa, Galileasta alkaen tänne asti".
\par 6 Mutta kun Pilatus sen kuuli, kysyi hän, oliko mies galilealainen.
\par 7 Ja saatuaan tietää hänen olevan Herodeksen hallintoalueelta hän lähetti hänet Herodeksen eteen, joka hänkin niinä päivinä oli Jerusalemissa.
\par 8 Kun Herodes näki Jeesuksen, ihastui hän suuresti; sillä hän oli jo kauan aikaa halunnut nähdä häntä, koska oli kuullut hänestä, ja hän toivoi saavansa nähdä häneltä jonkin ihmeen.
\par 9 Ja hän teki Jeesukselle monia kysymyksiä; mutta tämä ei vastannut hänelle mitään.
\par 10 Ja ylipapit ja kirjanoppineet seisoivat siinä ja syyttivät häntä kiivaasti.
\par 11 Mutta Herodes joukkoineen kohteli häntä halveksivasti ja pilkkasi häntä; ja puetettuaan hänet loistavaan pukuun hän lähetti hänet takaisin Pilatuksen eteen.
\par 12 Ja Herodes ja Pilatus tulivat sinä päivänä ystäviksi keskenään; he olivat näet ennen olleet toistensa vihamiehiä.
\par 13 Niin Pilatus kutsui kokoon ylipapit ja hallitusmiehet ja kansan
\par 14 ja sanoi heille: "Te olette tuoneet minulle tämän miehen kansan yllyttäjänä; ja katso, minä olen teidän läsnäollessanne häntä tutkinut enkä ole havainnut tätä miestä syylliseksi mihinkään, mistä te häntä syytätte,
\par 15 eikä Herodeskaan, sillä hän lähetti hänet takaisin meille. Ja katso, hän ei ole tehnyt mitään, mikä ansaitsee kuoleman.
\par 16 Kuritettuani häntä minä siis hänet päästän."
\par 17 []
\par 18 Niin he huusivat kaikki yhdessä, sanoen: "Vie pois tämä, mutta päästä meille Barabbas!"
\par 19 Tämä oli heitetty vankeuteen kaupungissa tehdystä kapinasta sekä murhasta.
\par 20 Niin Pilatus taas puhui heille, koska hän tahtoi päästää Jeesuksen irti.
\par 21 Mutta he huusivat vastaan ja sanoivat: "Ristiinnaulitse, ristiinnaulitse hänet!"
\par 22 Niin hän puhui heille kolmannen kerran: "Mitä pahaa hän sitten on tehnyt? En ole havainnut hänessä mitään, mistä hän ansaitsisi kuoleman. Kuritettuani häntä minä siis hänet päästän."
\par 23 Mutta he ahdistivat häntä suurilla huudoilla, vaatien Jeesusta ristiinnaulittavaksi; ja heidän huutonsa pääsivät voitolle.
\par 24 Niin Pilatus tuomitsi heidän vaatimuksensa täytettäväksi.
\par 25 Ja hän päästi irti sen, joka kapinasta ja murhasta oli vankeuteen heitetty ja jota he vaativat, mutta Jeesuksen hän antoi alttiiksi heidän mielivallallensa.
\par 26 Ja viedessään häntä pois he saivat käsiinsä Simonin, erään kyreneläisen, joka tuli vainiolta; ja hänen olalleen he panivat ristin kannettavaksi Jeesuksen jäljessä.
\par 27 Ja häntä seurasi suuri joukko kansaa, myös naisia, jotka valittivat ja itkivät häntä.
\par 28 Niin Jeesus kääntyi heihin ja sanoi: "Jerusalemin tyttäret, älkää minua itkekö, vaan itkekää itseänne ja lapsianne.
\par 29 Sillä katso, päivät tulevat, jolloin sanotaan: 'Autuaita ovat hedelmättömät ja ne kohdut, jotka eivät ole synnyttäneet, ja rinnat, jotka eivät ole imettäneet'.
\par 30 Silloin ruvetaan sanomaan vuorille: 'Langetkaa meidän päällemme', ja kukkuloille: 'Peittäkää meidät'.
\par 31 Sillä jos tämä tehdään tuoreelle puulle, mitä sitten kuivalle tapahtuu?"
\par 32 Myös kaksi muuta, kaksi pahantekijää, vietiin hänen kanssaan surmattaviksi.
\par 33 Ja kun saavuttiin paikalle, jota sanotaan Pääkallonpaikaksi, niin siellä he ristiinnaulitsivat hänet sekä pahantekijät, toisen oikealle ja toisen vasemmalle puolelle.
\par 34 Mutta Jeesus sanoi: "Isä, anna heille anteeksi, sillä he eivät tiedä, mitä he tekevät". Ja he jakoivat keskenään hänen vaatteensa ja heittivät niistä arpaa.
\par 35 Ja kansa seisoi ja katseli. Ja hallitusmiehetkin ivasivat häntä ja sanoivat: "Muita hän on auttanut; auttakoon itseänsä, jos hän on Jumalan Kristus, se valittu".
\par 36 Myös sotamiehet pilkkasivat häntä, menivät hänen luoksensa ja tarjosivat hänelle hapanviiniä
\par 37 ja sanoivat: "Jos sinä olet juutalaisten kuningas, niin auta itseäsi".
\par 38 Oli myös hänen päänsä päällä kirjoitus: "Tämä on juutalaisten kuningas".
\par 39 Niin toinen pahantekijöistä, jotka siinä riippuivat, herjasi häntä: "Etkö sinä ole Kristus? Auta itseäsi ja meitä."
\par 40 Mutta toinen vastasi ja nuhteli häntä sanoen: "Etkö sinä edes pelkää Jumalaa, sinä, joka olet saman rangaistuksen alainen?
\par 41 Me tosin kärsimme oikeuden mukaan, sillä me saamme, mitä meidän tekomme ansaitsevat; mutta tämä ei ole mitään pahaa tehnyt."
\par 42 Ja hän sanoi: "Jeesus, muista minua, kun tulet valtakuntaasi".
\par 43 Niin Jeesus sanoi hänelle: "Totisesti minä sanon sinulle: tänä päivänä pitää sinun oleman minun kanssani paratiisissa."
\par 44 Ja oli jo noin kuudes hetki. Niin yli kaiken maan tuli pimeys, jota kesti hamaan yhdeksänteen hetkeen,
\par 45 sillä aurinko oli pimentynyt. Ja temppelin esirippu repesi keskeltä kahtia.
\par 46 Ja Jeesus huusi suurella äänellä ja sanoi: "Isä, sinun käsiisi minä annan henkeni". Ja sen sanottuaan hän antoi henkensä.
\par 47 Mutta kun sadanpäämies näki, mitä tapahtui, kunnioitti hän Jumalaa ja sanoi: "Totisesti, tämä oli vanhurskas mies".
\par 48 Ja kun kaikki kansa, ne, jotka olivat kokoontuneet tätä katselemaan, näkivät, mitä tapahtui, löivät he rintoihinsa ja palasivat kukin kotiinsa.
\par 49 Mutta kaikki hänen tuttavansa seisoivat taampana, myöskin naiset, jotka olivat seuranneet häntä Galileasta, ja katselivat tätä.
\par 50 Ja katso, oli neuvoston jäsen, nimeltä Joosef, hyvä ja hurskas mies, joka ei ollut suostunut heidän neuvoonsa ja tekoonsa.
\par 51 Tämä oli kotoisin juutalaisten kaupungista Arimatiasta, ja hän odotti Jumalan valtakuntaa.
\par 52 Hän meni Pilatuksen luo ja pyysi Jeesuksen ruumista.
\par 53 Ja otettuaan sen alas hän kääri sen liinavaatteeseen. Ja hän pani hänet hautaan, joka oli hakattu kallioon ja johon ei oltu vielä ketään pantu.
\par 54 Ja silloin oli valmistuspäivä, ja sapatti oli alkamaisillaan.
\par 55 Ja naiset, jotka olivat tulleet hänen kanssaan Galileasta, seurasivat jäljessä ja katselivat hautaa ja kuinka hänen ruumiinsa sinne pantiin.
\par 56 Ja palattuaan kotiinsa he valmistivat hyvänhajuisia yrttejä ja voiteita; mutta sapatin he viettivät hiljaisuudessa lain käskyn mukaan.

\chapter{24}

\par 1 Mutta viikon ensimmäisenä päivänä ani varhain he tulivat haudalle, tuoden mukanaan valmistamansa hyvänhajuiset yrtit.
\par 2 Ja he havaitsivat kiven vieritetyksi pois haudalta.
\par 3 Niin he menivät sisään, mutta eivät löytäneet Herran Jeesuksen ruumista.
\par 4 Ja kun he olivat tästä ymmällä, niin katso, kaksi miestä seisoi heidän edessään säteilevissä vaatteissa.
\par 5 Ja he peljästyivät ja kumartuivat kasvoillensa maahan. Niin miehet sanoivat heille: "Miksi te etsitte elävää kuolleitten joukosta?
\par 6 Ei hän ole täällä, hän on noussut ylös. Muistakaa, kuinka hän puhui teille vielä ollessaan Galileassa,
\par 7 sanoen: 'Ihmisen Poika pitää annettaman syntisten ihmisten käsiin ja ristiinnaulittaman, ja hänen pitää kolmantena päivänä nouseman ylös'."
\par 8 Niin he muistivat hänen sanansa.
\par 9 Ja he palasivat haudalta ja veivät sanan tästä kaikesta niille yhdelletoista ja kaikille muille.
\par 10 Ja ne, jotka kertoivat tämän apostoleille, olivat Maria Magdaleena ja Johanna ja Maria, Jaakobin äiti, ja muut naiset heidän kanssansa.
\par 11 Mutta näiden puheet näyttivät heistä turhilta, eivätkä he uskoneet heitä.
\par 12 Mutta Pietari nousi ja juoksi haudalle; ja kun hän kurkisti sisään, näki hän siellä ainoastaan käärinliinat. Ja hän meni pois ihmetellen itsekseen sitä, mikä oli tapahtunut.
\par 13 Ja katso, kaksi heistä kulki sinä päivänä Emmaus nimiseen kylään, joka on kuudenkymmenen vakomitan päässä Jerusalemista.
\par 14 Ja he puhelivat keskenään kaikesta tästä, mikä oli tapahtunut.
\par 15 Ja heidän keskustellessaan ja tutkistellessaan tapahtui, että Jeesus itse lähestyi heitä ja kulki heidän kanssansa.
\par 16 Mutta heidän silmänsä olivat pimitetyt, niin etteivät he tunteneet häntä.
\par 17 Ja hän sanoi heille: "Mistä te siinä kävellessänne puhutte keskenänne?" Niin he seisahtuivat murheellisina muodoltansa.
\par 18 Ja toinen heistä, nimeltä Kleopas, vastasi ja sanoi hänelle: "Oletko sinä ainoa muukalainen Jerusalemissa, joka et tiedä, mitä siellä näinä päivinä on tapahtunut?"
\par 19 Hän sanoi heille: "Mitä?" Niin he sanoivat hänelle: "Sitä, mikä tapahtui Jeesukselle, Nasaretilaiselle, joka oli profeetta, voimallinen teossa ja sanassa Jumalan ja kaiken kansan edessä,
\par 20 kuinka meidän ylipappimme ja hallitusmiehemme antoivat hänet tuomittavaksi kuolemaan ja ristiinnaulitsivat hänet.
\par 21 Mutta me toivoimme hänen olevan sen, joka oli lunastava Israelin. Ja onhan kaiken tämän lisäksi nyt jo kolmas päivä siitä, kuin nämä tapahtuivat.
\par 22 Ovatpa vielä muutamat naiset joukostamme saattaneet meidät hämmästyksiin. He kävivät aamulla varhain haudalla
\par 23 eivätkä löytäneet hänen ruumistaan, ja tulivat ja sanoivat myös nähneensä enkelinäyn, ja enkelit olivat sanoneet hänen elävän.
\par 24 Ja muutamat niistä, jotka olivat meidän kanssamme, menivät haudalle ja havaitsivat niin olevan, kuin naiset olivat sanoneet; mutta häntä he eivät nähneet."
\par 25 Niin hän sanoi heille: "Oi, te ymmärtämättömät ja hitaat sydämeltä uskomaan kaikkea sitä, minkä profeetat ovat puhuneet!
\par 26 Eikö Kristuksen pitänyt tätä kärsimän ja sitten menemän kirkkauteensa?"
\par 27 Ja hän alkoi Mooseksesta ja kaikista profeetoista ja selitti heille, mitä hänestä oli kaikissa kirjoituksissa sanottu.
\par 28 Ja kun he lähestyivät kylää, johon olivat menossa, niin hän oli aikovinaan kulkea edemmäksi.
\par 29 Mutta he vaativat häntä sanoen: "Jää meidän luoksemme, sillä ilta joutuu ja päivä on jo laskemassa". Ja hän meni sisään ja jäi heidän luoksensa.
\par 30 Ja tapahtui, kun hän oli aterialla heidän kanssaan, että hän otti leivän, siunasi, mursi ja antoi heille.
\par 31 Silloin heidän silmänsä aukenivat, ja he tunsivat hänet. Ja hän katosi heidän näkyvistään.
\par 32 Ja he sanoivat toisillensa: "Eikö sydämemme ollut meissä palava, kun hän puhui meille tiellä ja selitti meille kirjoitukset?"
\par 33 Ja he nousivat sillä hetkellä ja palasivat Jerusalemiin ja tapasivat ne yksitoista kokoontuneina ja ne, jotka olivat heidän kanssansa.
\par 34 Ja nämä sanoivat: "Herra on totisesti noussut ylös ja on ilmestynyt Simonille".
\par 35 Ja itse he kertoivat, mitä oli tapahtunut tiellä ja kuinka he olivat hänet tunteneet, kun hän mursi leivän.
\par 36 Mutta heidän tätä puhuessaan Jeesus itse seisoi heidän keskellään ja sanoi heille: "Rauha teille!"
\par 37 Niin heidät valtasi säikähdys ja pelko, ja he luulivat näkevänsä hengen.
\par 38 Mutta hän sanoi heille: "Miksi olette hämmästyneet, ja miksi nousee sellaisia ajatuksia teidän sydämeenne?
\par 39 Katsokaa minun käsiäni ja jalkojani ja nähkää, että minä itse tässä olen. Kosketelkaa minua ja katsokaa, sillä ei hengellä ole lihaa eikä luita, niinkuin te näette minulla olevan."
\par 40 Ja tämän sanottuaan hän näytti heille kätensä ja jalkansa.
\par 41 Mutta kun he eivät vielä uskoneet, ilon tähden, vaan ihmettelivät, sanoi hän heille: "Onko teillä täällä jotakin syötävää?"
\par 42 Niin he antoivat hänelle palasen paistettua kalaa.
\par 43 Ja hän otti ja söi heidän nähtensä.
\par 44 Ja hän sanoi heille: "Tätä tarkoittivat minun sanani, kun minä puhuin teille ollessani vielä teidän kanssanne, että kaiken pitää käymän toteen, mikä minusta on kirjoitettu Mooseksen laissa ja profeetoissa ja psalmeissa".
\par 45 Silloin hän avasi heidän ymmärryksensä käsittämään kirjoitukset.
\par 46 Ja hän sanoi heille: "Niin on kirjoitettu, että Kristus oli kärsivä ja kolmantena päivänä nouseva kuolleista,
\par 47 ja että parannusta syntien anteeksisaamiseksi on saarnattava hänen nimessänsä kaikille kansoille, alkaen Jerusalemista.
\par 48 Te olette tämän todistajat.
\par 49 Ja katso, minä lähetän teille sen, jonka minun Isäni on luvannut; mutta te pysykää tässä kaupungissa, kunnes teidän päällenne puetaan voima korkeudesta."
\par 50 Sitten hän vei heidät pois, lähes Betaniaan asti, ja nosti kätensä ja siunasi heidät.
\par 51 Ja tapahtui, että hän siunatessaan heitä erkani heistä, ja hänet otettiin ylös taivaaseen.
\par 52 Ja he kumarsivat häntä ja palasivat Jerusalemiin suuresti iloiten.
\par 53 Ja he olivat alati pyhäkössä ja ylistivät Jumalaa.


\end{document}