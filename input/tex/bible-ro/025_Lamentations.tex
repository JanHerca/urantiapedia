\begin{document}

\title{Plângerile lui Ieremia}


\chapter{1}

\par 1 O, cum a rămas pustie cetatea cea cu mult popor! Cum a ajuns ca o văduvă cea mai de frunte dintre neamuri; doamna cetăților a ajuns birnică.
\par 2 Noaptea plânge întruna cu lacrimi pe obraz și dintre toți câți o iubeau, nici unul n-o mai mângâie; toți prietenii au devenit dușmani.
\par 3 Iuda s-a dus în robie, la suferință și la muncă grea; sălășluiește printre neamuri și nu-și află odihnă. Toți asupritorii lui l-au prins la strâmtorare.
\par 4 Toate căile Sionului sunt pline de jale și nimeni nu mai vine la sărbătoare. Toate porțile (cetății) sunt pustii, preoții ei suspină; fecioarele sunt deznădăjduite și ea este plină de amar.
\par 5 Vrăjmașii ei sunt biruitori, dușmanii ei sunt cu voie bună; căci Domnul a umilit-o din pricina multelor ei păcate, iar feciorii ei au plecat în robie înaintea asupritorului.
\par 6 Așa și-a irosit fiica Sionului toată strălucirea! Căpeteniile ei sunt asemenea cerbilor care nu află pășune și fug sleiți de puteri dinaintea urmăritorului.
\par 7 Ierusalimul își aduce aminte de zilele ticăloșiei lui și ale rătăcirii lui, de toate strălucirile pe care le-a avut în străvechile vremuri. Acum însă, când poporul a căzut în mâna vrăjmașului și când nimeni nu-i poate veni în ajutor, dușmanii lui se uită la el și râd de prăbușirea lui.
\par 8 Ierusalimul a păcătuit de moarte, pentru aceasta a ajuns de spaimă; toți cei ce-l cinsteau nu-l mai iau în seamă, căci au văzut goliciunea lui, iar el suspină și își întoarce fața.
\par 9 Necurăția lui e lipită de poala hainelor lui căci la sfârșitul lui el nu s-a gândit. El s-a prăbușit în chip uluitor și n-are pe nimeni să-l mângâie! "Vezi, Doamne, necazul meu, căci vrăjmașul biruiește".
\par 10 Dușmanii au întins mâna spre toate vistieriile lui. El a văzut neamuri intrând în templul său, neamuri cărora le-ai dat poruncă: "Să nu intre în obștea ta!"
\par 11 Tot poporul Tău suspină căutând pâine, și își dau odoarele lor pentru mâncare, ca să-și țină viața. Vezi, Doamne, și ia aminte cum am ajuns de ocară!
\par 12 O, voi trecătorilor, priviți și vedeți dacă este vreo durere ca aceea care mă copleșește și cu care Domnul m-a umplut de necaz în ziua întăririi mâniei Lui.
\par 13 Foc a trimis de sus peste oasele mele și m-a smerit, picioarelor mele le-a întins cursă, și m-a făcut să dau înapoi; pustiitu-m-a cu totul, iar eu toată ziua bolesc.
\par 14 Jugul păcatelor mele mi-a fost legat de gât de către mâna Lui; strânse ca într-un mănunchi, ele atârnă de grumazul meu; El a făcut să se destrame puterea mea și m-a dat în mâna celor cărora nu puteam să mă împotrivesc.
\par 15 Domnul a spulberat pe toți voinicii din mijlocul meu, El a chemat oaste împotriva mea, ca să sfărâme pe voinicii mei. Stăpânul a toate a strivit ca în teasc pe fecioara, fiica lui Iuda.
\par 16 Pentru aceasta eu plâng mereu, din ochii mei izvorăsc lacrimi, căci departe de mine este Mângâietorul, Cel ce-mi îmbărbăta inima. Feciorii mei cu toții au fost dați pieirii, căci dușmanul a avut biruință.
\par 17 Sionul întinde mâinile sale și nimeni nu-l mângâie! Domnul a dat poruncă tuturor vrăjmașilor lui Iacov ca să-l împresoare. Ajuns-a Ierusalimul înaintea ochilor lor ca un lucru spurcat.
\par 18 Drept este Domnul, căci împotriva poruncilor Lui m-am răzvrătit. Luați aminte, voi, toate popoarele, și vedeți necazul meu: fecioarele mele și flăcăii mei au fost duși în robie.
\par 19 Strigat-am către iubiții mei, dar ei m-au înșelat; preoții mei și bătrânii mei au pierit în cetate, când căutau hrană ca să-și țină viața.
\par 20 Vezi, Doamne, cât sunt de strâmtorat, lăuntrul meu arde! Inima mea se zbuciumă în trupul meu, pentru că m-am răzvrătit foarte. Afară sabia seceră pe feciorii mei, iar înăuntru, moartea.
\par 21 Toți aud suspinul meu, dar nimeni nu mă mângâie! toți dușmanii, aflând de nenorocirea mea, se bucură că ai făcut așa. Să vină peste ei ziua pe care ai făgăduit-o și să ajungă și ei ca mine!
\par 22 Toată fărădelegea lor să vină înaintea Ta și să le faci lor precum mi-ai făcut mie, pentru toate păcatele mele! Căci suspinele mele sunt fără de număr, iar inima mea bolește!

\chapter{2}

\par 1 O, cum a acoperit cu nori Domnul întru mânia Lui pe fiica Sionului! Din cer a aruncat pe pământ măreția lui Israel și în ziua mâniei Sale nu și-a adus aminte de așternutul picioarelor Sale.
\par 2 Domnul a nimicit fără milă toate sălașele lui Iacov; întru întărâtarea urgiei Lui a doborât la pământ întăriturile fiicei lui Iuda; le-a făcut una cu pământul, a pângărit regatul și căpeteniile lui.
\par 3 Întru aprinderea mâniei Lui a zdrobit toată puterea lui Israel; înaintea dușmanului și-a tras dreapta înapoi. El a aprins pe Iacov cu un foc arzător care prăpădește de jur împrejur.
\par 4 El a încordat arcul Său ca un dușman, dreapta Sa a stat gata ca a unui vrăjmaș și a ucis tot ce desfăta ochiul în cortul fiicei Sionului; vărsat-a ca un foc mânia Lui.
\par 5 Stăpânul S-a arătat ca un dușman nimicind Israelul; i-a dărâmat toate palatele și cetățile întărite și asupra fiicei lui Iuda a adus multă supărare.
\par 6 Prăbușit-a la pământ ca pe o dumbravă cortul lui, stricat-a locul de sărbătoare. Domnul a făcut să se uite zilele de odihnă în Sion, disprețuind, în văpaia mâniei Lui, pe rege și pe preot.
\par 7 Disprețuit-a Domnul jertfelnicul Său și S-a îndepărtat de locașul Său cel sfânt; dat-a zidurile palatelor Sale în mâna dușmanilor care au strigat în templul Domnului ca în zilele de sărbătoare.
\par 8 Găsit-a Domnul cu cale să surpe zidurile fiicei Sionului; întins-a funia și n-a tras înapoi mâna Sa, până nu le-a nimicit. El a întins jalea peste ziduri și întărituri, ce stau laolaltă dărăpănate.
\par 9 Porțile lui s-au afundat în pământ, zăvoarele lor El le-a sfărâmat; regele lui și căpeteniile sunt pribegi printre neamuri. Lege nu mai au, chiar și profeții nu mai primesc vedenii de la Domnul.
\par 10 Stau la pământ și tac bătrânii fiicei Sionului; pe capul lor și-au pus țărână și s-au încins cu sac; la pământ își pleacă fecioarele Ierusalimului capetele lor.
\par 11 Ochii mei se sfârșesc de plâns, lăuntrul meu arde ca văpaia, măruntaiele mele fierb și fierea mi se varsă pe pământ din pricina zdrobirii fiicei neamului meu, când copiii și pruncii stau sleiți de putere în piețele cetății,
\par 12 Zicând mamei lor: "Unde este pâine, unde este vin?" Ei cad sleiți de putere ca doborâți de sabie pe piețele cetății și își dau sufletul la sânul maicii lor.
\par 13 Cu cine te voi asemăna, cu cine aș putea să te pun alături, o, fiică a Ierusalimului! Cu cine te-aș pune față în față, ca să te pot mângâia, o, fecioară, fiică a Sionului? Căci nețărmurită ca marea este năruirea ta! Cine ar putea să te tămăduiască?
\par 14 Profeții tăi au avut pentru tine vedenii zadarnice și arătări și nu ți-au dat pe față fărădelegea ta, ca să-ți schimbe calea ta, ci ți-au arătat vedenii înșelătoare și aducătoare de pieire.
\par 15 Bat spre tine din palme toți cei ce trec pe cale, fluieră și clatină din cap pentru fiica Ierusalimului: "Aceasta este, oare, cetatea pe care o numeau cununa frumuseții, bucuria a tot pământul?"
\par 16 Către tine toți vrăjmașii tăi cască gura lor, fluieră, scrâșnesc din dinți, zicând: "Am nimicit-o. Da, aceasta este ziua pe care noi o așteptam; am aflat-o și o vedem".
\par 17 Împlinit-a Domnul hotărârea Sa, adus-a la îndeplinire cuvântul Său, spus din zilele străvechi; prăbușit-a fără milă, bucurat-a pe vrăjmașul tău, înălțat-a puterea apăsătorilor tăi.
\par 18 Strigă către Domnul, tu fecioară, fiică a Sionului! Să curgă lacrimile tale ca un șuvoi, zi și noapte; nu înceta, ochiul tău să nu zăbovească!
\par 19 Scoală, jelește-te în timpul nopții, la început do strajă; varsă-ți inima ta ca apa înaintea feței Celui Atotstăpânitor. Către El ridică mâna ta pentru viața pruncilor tăi, care se prăpădesc de foame la colțul tuturor ulițelor.
\par 20 Vezi, o, Doamne, și privește cui ai făcut aceasta! Să mănânce femeile rodul pântecelui lor, copiii pe care îi poartă în brațe? Să fie uciși în templu, Doamne, preotul și profetul?
\par 21 Stau culcați la pământ pe ulițe tânăr și bătrân. Fecioarele și flăcăii mei de sabie au căzut; Tu i-ai ucis în ziua mâniei Tale, jertfitu-i-ai fără de milă.
\par 22 Chemat-ai ca la sărbătoare pe toți cei ce au sălaș în jurul meu. Și în ziua mâniei Domnului n-a scăpat, nici n-a rămas vreunul; pe cei care i-am purtat în brațe și i-am făcut mari, mi i-a nimicit dușmanul.

\chapter{3}

\par 1 Eu sunt omul care am văzut nenorocirea sub varga aprinderii Lui.
\par 2 El m-a purtat și m-a dus în întuneric și în beznă.
\par 3 Da, împotriva mea întoarce și iar întoarce în toată vremea mâna Sa.
\par 4 Mistuit-a trupul meu și pielea mea, zdrobit-a oasele mele;
\par 5 A ridicat zid împotriva mea și m-a înconjurat de venin și de zbucium,
\par 6 Mutându-mă în împărăția morii, ca pe morții cei din veac.
\par 7 M-a împrejmuit cu zid și n-am pe unde să ies, îngreuiat-a lanțurile mele;
\par 8 Chiar când strig și răcnesc, rugăciunea mea nu se aude;
\par 9 El a astupat cărările mele cu piatră și a întortochiat potecile mele.
\par 10 El a ajuns pentru mine ca un urs la pândă, ca un leu în ascunzătoare.
\par 11 Rătăcit-a căile mele, m-a sfâșiat și m-a pustiit;
\par 12 A încordat arcul Său și m-a așezat ca țintă săgeții Sale,
\par 13 Trimițând în rărunchii mei pe fiii tolbei Sale.
\par 14 Făcutu-m-am de râs față de poporul meu, cântecul lor de batjocură în fiecare zi.
\par 15 El m-a săturat de amărăciuni, adăpatu-m-a cu pelin.
\par 16 A zdrobit de piatră dinții mei și m-a afundat în cenușă.
\par 17 Tu ai răpit pacea sufletului meu, uitat-am fericirea
\par 18 ți am zis: "S-a dus puterea vieții mele și nădejdea mea în Domnul".
\par 19 Adu-ți aminte de nevoia și necazul meu, de pelin și otravă!
\par 20 Să-ți aduci aminte că împovărat este în mine sufletul meu.
\par 21 Aceasta voi pune-o la inimă, de aceea voi nădăjdui:
\par 22 Milele Domnului nu s-au sfârșit, milostivirile Lui nu încetează.
\par 23 În fiecare dimineață sunt altele, credincioșia Ta este mare!
\par 24 "Partea mea este Domnul, a zis sufletul meu, de aceea voi nădăjdui în El".
\par 25 Bun este Domnul cu cei ce se încred în El, pentru omul care Îl caută.
\par 26 Bine este să aștepți în tăcere ajutorul Domnului.
\par 27 Bine este omului să poarte un jug din tinerețile lui.
\par 28 Să stea la o parte în tăcere, dacă Domnul îi dă poruncă!
\par 29 Să atingă pulberea cu buzele lui; poate mai este nădejde!
\par 30 Să dea obrazul lui spre lovire și să se sature de ocară!
\par 31 Căci Domnul nu aruncă pe oameni pentru totdeauna;
\par 32 Ci El pedepsește și are milă după mulțimea milelor Lui.
\par 33 Că nu de bună voie umilește și pedepsește pe fiii oamenilor.
\par 34 Când călcăm în picioare pe toți robii pământului,
\par 35 Când călcăm dreptatea omului înaintea feței Celui Preaînalt,
\par 36 Când nu dăm dreptate cuiva în pricina lui, oare Stăpânul a toate nu vede?
\par 37 Cine este Cel ce a grăit și s-a făcut, fără numai Domnul, Care a poruncit?
\par 38 Nu iese oare din gura Celui Preaînalt binele și răul?
\par 39 De ce suspină omul toată viața, fiecare pentru păcatul lui?
\par 40 Să cercetăm căile noastre, luând aminte și întorcându-ne la Domnul!
\par 41 Să ridicăm inimile și mâinile noastre la Domnul din cer!
\par 42 Noi am păcătuit și ne-am răzvrătit și Tu ne-ai iertat.
\par 43 Tu Te-ai învesmântat cu mânie și ne-ai urmărit; Tu ai ucis fără milă;
\par 44 Tu Te-ai ascuns în nori, ca să nu străbată rugăciunea la Tine;
\par 45 Tu ai făcut din mine o măturătură și un gunoi, în mijlocul popoarelor.
\par 46 Toți dușmanii noștri au deschis gura împotriva noastră;
\par 47 De spaimă și de groapă am avut parte, de pustiire și de ruină.
\par 48 Șuvoaie de apă lăcrimează ochiul meu, din pricina prăpădului fiicei poporului meu.
\par 49 Ochiul meu varsă lacrimi fără încetare, căci nu este ușurare,
\par 50 Până să se uite în jos și să privească Domnul din ceruri.
\par 51 Ochiul meu mă doare din pricina fiicelor cetății mele.
\par 52 Ca pe o pasăre m-au vânat fără cuvânt vrăjmașii mei,
\par 53 Au vrut să nimicească în groapă viața mea, și au aruncat cu pietre în mine.
\par 54 Ape năvăleau peste capul meu și cugetam: "Sunt pierdut!"
\par 55 Chemat-am numele Tău, Doamne, din groapa cea mai dedesubt.
\par 56 Tu ai auzit glasul meu: "Nu astupa urechea la suspinul și strigătul meu".
\par 57 Tu erai aproape în ziua când Te-am strigat și ai zis: "Nu-ți fie frică!"
\par 58 O, Doamne, Tu ai judecat pricina mea, Tu ai izbăvit viața mea!
\par 59 Văzut-ai, Doamne, apăsarea mea, ajută-mi și-mi fă dreptate!
\par 60 Tu ai văzut toată răzbunarea lor, toate uneltirile lor împotriva mea;
\par 61 O, Doamne, Tu ai auzit ocările lor, toate chibzuielile lor împotriva mea,
\par 62 Graiurile potrivnicilor mei și gândul lor ascuns împotriva mea.
\par 63 Privește: de stau sau de se scoală, eu sunt de râsul lor!
\par 64 Răsplătește-le, Doamne, după faptele mâinilor lor,
\par 65 Dă-le învârtoșare inimii, blestemul Tău să fie pentru ei!
\par 66 Urmărește-i cu mânie și nimicește-i sub cerurile Tale, Doamne!

\chapter{4}

\par 1 O, cum s-a întunecat aurul, și cel mai curat aur și-a schimbat fața; pietrele nestemate vărsate au fost la colțul tuturor ulițelor!
\par 2 Feciorii Sionului, cei mai de seamă altădată, cântăriți cu aur, cum au ajuns să fie socotiți ca vasele de lut, lucru de mână de olar!
\par 3 Chiar și șacalii își dau sânul, ca puii lor să sugă, dar fiica poporului meu ajuns-a crudă, ca struții în pustiu.
\par 4 Din pricina setei lipitu-s-a limba sugaciului de cerul gurii lui; copiii cer pâine, dar nimeni nu le-o întinde.
\par 5 Cei care mâncau odinioară mâncăruri alese cad de foame pe ulițe; cei care au fost crescuți în purpură stau trântiți în gunoi.
\par 6 Vina fiicei poporului meu a fost mai mare decât a Sodomei, prăbușită într-o clipă, nu de mână omenească.
\par 7 Căpeteniile ei erau mai strălucitoare decât zăpada, mai albe decât laptele; trupul lor era mai roșu decât mărgeanul, ca safirul era înfățișarea lor.
\par 8 Chipul lor a ajuns mai negru decât funinginea, pe ulițe nu-i poți cunoaște; pielea lor s-a zbârcit pe oase, s-a uscat ca o așchie de lemn.
\par 9 Mai fericiți au fost cei care au căzut de sabie, decât cei morți de foame, care se prăpădesc încet, doborâți de lipsa roadelor de pe câmp.
\par 10 Femeile, deși miloase, au fiert cu mâinile lor copiii și i-au mâncat în vremea căderii fiicei poporului meu.
\par 11 Sfârșit-a Domnul mânia, vărsat-a pe deplin urgia aprinderii Lui; și în Sion a aprins un fac care l-a mistuit.
\par 12 Nici n-ar fi putut să creadă regii pământului și toți locuitorii lumii că vrăjmașul și apăsătorul ar putea să intre pe porțile Ierusalimului!
\par 13 Dar s-a întâmplat, din pricina păcatelor profeților (mincinoși) și a fărădelegilor preoților, care au vărsat în mijlocul lui sângele celor drepți.
\par 14 Pe ulițe rătăceau pătați de sânge, și nimeni nu se atingea de hainele lor.
\par 15 "Păziți-vă! Un necurat!" Striga lumea după ei. "Fugiți, la o parte, nu-i atingeți!" Și dacă mai voiesc să rătăcească undeva - se zicea printre neamuri - n-ar trebui să rămână aici!
\par 16 Fața plină de mânie a Domnului i-a risipit pe ei. Pe preoți nimeni nu-i mai lua în seamă, de bătrâni nu se îndura.
\par 17 Și ochii noștri se sting de supărare, așteptând zadarnic un ajutor! Din turnul nostru ne-am uitat departe spre un popor al cărui ajutor nu vine.
\par 18 Și pândeau pașii noștri ca să nu umblăm prin piețele noastre. Sfârșitul nostru se apropia, sosise!
\par 19 Prigonitorii noștri erau mai iuți decât vulturii de pe cer; umblau după noi prin munți, ne pândeau în pustiu.
\par 20 Suflarea vieții noastre, unsul Domnului, a fost prins în groapa lor - acela despre care noi ziceam: "La umbra lui vom viețui printre popoare".
\par 21 Bucură-te și te veselește, fiica Edomului, tu care locuiești în pământul Uț; și la tine va veni cupa; vei bea și te vei lăsa goală.
\par 22 Fărădelegea ta, o, fiică a Sionului, s-a sfârșit; la fel robia. Dar ție îți cercetează păcatele, o, fiică a Edomului, și dă pe față fărădelegile tale!

\chapter{5}

\par 1 Adu-ți aminte, Doamne, de cele întâmplate, și vezi ocara noastră!
\par 2 Moștenirea și casele noastre au căzut în mâna celor străini, de alt neam.
\par 3 Am ajuns orfani, fără de tată, mamele noastre sunt văduve.
\par 4 Bem apa noastră cu bani, lemnele noastre le primim cu plată.
\par 5 Pe grumajii noștri stau prigonitorii și, deși n-avem puteri, nu ne dau răgaz.
\par 6 Întindem mâna către Egipt și Asiria ca să ne sature de pâine.
\par 7 Părinții noștri au greșit și nu mai sunt, dar noi purtăm fărădelegile lor.
\par 8 Slugi ne stăpânesc și nimeni nu vine să ne scoată din mina lor.
\par 9 Cu primejdia vieții noastre ne agonisim pâinea, în fața sabiei care ne amenință în pustiu.
\par 10 Pielea noastră s-a înnegrit ca un cuptor de văpaia foametei.
\par 11 Ei au batjocorit femeile în Sion, fecioarele din cetățile lui Iuda.
\par 12 Căpeteniile au fost spânzurate de mâna lor, fețele bătrânilor nu au mai fost luate în seamă.
\par 13 Flăcăii au învârtit la râșniță și tinerii s-au poticnit cărând lemne.
\par 14 Bătrânii nu mai stau la poartă, cei tineri nu mai cântă din alăute.
\par 15 S-a dus veselia inimii noastre, jocul nostru s-a schimbat în plâns.
\par 16 Căzut-a cununa de pe capul nostru; vai nouă, că am păcătuit!
\par 17 Pentru aceasta inima noastră tânjește și ochii s-au întunecat.
\par 18 Muntele Sionului a rămas pustiu și pe el se plimbă vulpile.
\par 19 Tu, Doamne, împărățești în veci și scaunul Tău în neam de neam!
\par 20 Pentru ce vrei să ne uiți, să ne părăsești atât de multă vreme?
\par 21 Întoarce-Te către noi și ne vom întoarce; înnoiește zilele noastre ca în vremea cea de demult!
\par 22 Sau Tu ne-ai urgisit și Te-ai mâniat pe noi, fără măsură?


\end{document}