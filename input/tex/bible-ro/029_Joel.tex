\begin{document}

\title{Ioel}


\chapter{1}

\par 1 Cuvântul Domnului care a fost către Ioil, fiul lui Petuel.
\par 2 Ascultați acestea, voi bătrânilor, luați aminte, voi toți locuitorii țării! Oare s-a mai întâmplat așa în vremea voastră sau în zilele părinților voștri?
\par 3 Povestiți-le feciorilor voștri, iar ei feciorilor lor, iar aceia neamului care va veni după ei!
\par 4 Ceea ce a rămas de la lăcustele mici au mâncat lăcustele mari, și ceea ce a rămas de la cele mari au mâncat cele zburătoare, și ceea ce a rămas de la cele zburătoare au prăpădit stolurile de lăcuste.
\par 5 Deșteptați-vă, bețivilor, și plângeți! Și voi, băutorilor de vin, tânguiți-vă pentru vinul cel nou, căci vi s-a luat de la gură.
\par 6 Căci un popor a năvălit în țara mea, puternic și fără de număr; dinții lui ca dinții de leu, iar fălcile, ca fălcile de leoaică.
\par 7 Pustiit-a via mea, iar smochinul meu mi l-a făcut bucăți; l-a jupuit de coajă și l-a trântit la pământ. Mlădițele de vită au rămas albe.
\par 8 Tânguiește-te ca o fecioară încinsă cu sac după bărbatul din tinerețea ei!
\par 9 Prinosul și jertfa cu turnare nu mai sunt în templul Domnului! Preoții, slujitorii Domnului, sunt în mare jale.
\par 10 Câmpul a rămas pustiu, țarina este plină de jale; grâul nu mai este, mustul ajuns-a de ocară, iar untdelemn n-a rămas chiar nici un strop!
\par 11 Plugarii sunt zăpăciți, stăpânii de vii se tânguiesc pentru grâu și pentru orz, căci secerișul din țarini este pierdut.
\par 12 Vița de vie este fără vlagă, smochinul tânjește, rodiile de asemenea și finicii, merii și toți copacii de pe câmp s-au uscat; ba mai mult, bucuria ajuns-a ocară pentru fiii oamenilor.
\par 13 Încingeți-vă și vă tânguiți, voi preoților, izbucniți în bocete, voi slujitori ai altarului! Intrați în templu și petreceți noaptea în sac de jale, voi slujitori ai lui Dumnezeu, căci prinosul și jertfa cu turnare au fost îndepărtate din templul Dumnezeului vostru!
\par 14 Postiți post sfânt, strângeți obște de prăznuire, adunați pe bătrâni, pe toți locuitorii țării în templul Dumnezeului vostru și strigați către Domnul rugându-L:
\par 15 "O, ce zi! Căci aproape este ziua Domnului și vine ca o pustiire de la Cel Atotputernic.
\par 16 Oare nu ni s-a luat hrana de sub ochii noștri, bucuria și veselia din templul Dumnezeului nostru?
\par 17 Semințele s-au uscat sub bulgării din brazdă, grânarele sunt goale, hambarele sunt fără nimic în ele, căci nu mai este grâu.
\par 18 Cum gem vitele! Cirezile de boi mugesc, căci nu află nicăieri pășune; chiar și turmele de oi sunt în mare lipsă.
\par 19 Către Tine, Doamne, strig! Căci focul a mistuit toate pășunile pustiului și văpaia lui a dogorit toți copacii de pe câmp.
\par 20 Și fiarele câmpului zbiară către tine, căci pâraiele de apă au secat și focul a mistuit pășunile stepei".

\chapter{2}

\par 1 Sunați din trâmbiță în Sion, strigați din răsputeri în muntele cel sfânt al Meu, ca să se cutremure toți locuitorii țării! Căci vine ziua Domnului; iată ea este aproape;
\par 2 O zi de întuneric și de beznă, zi cu nori și cu negură deasă. Precum zorile se revarsă pe deasupra munților, tot așa dă năvală un popor numeros și puternic, cum n-a mai fost niciodată și cum nu va mai fi după el până în anii vremurilor celor mai îndepărtate.
\par 3 Înaintea lui merge mistuind focul, iar după el arde văpaia. Pământul este înaintea lui ca grădina raiului, iar după trecerea lui, pustiu înfricoșător, căci nimic nu scapă din fața lui.
\par 4 Cum sunt caii, așa le este chipul lor; și ca sprinteni călăreți, așa aleargă.
\par 5 Se aud ca un duruit de care de război, care se avântă în goană pe creștetul munților, ca pârâitul flăcărilor care mistuie o miriște, ca o oștire puternică așezată în linie de bătaie.
\par 6 Înaintea lui popoarele tremură de spaimă, toate fețele ard ca văpaia.
\par 7 Aleargă ca niște viteji; ca războinicii încercați se avântă peste ziduri; om după om își urmează calea, fără ca vreunul să se rătăcească.
\par 8 Nimeni nu se îmbrâncește cu cel de alături, fiecare își vede de drumul lui; printre săgeți își croiesc cale, fără ca nici unul să rupă rândul.
\par 9 Dau năvală în cetate, aleargă pe deasupra zidurilor, pătrund în case și intră pe ferestre ca furii.
\par 10 Înaintea lor tremură pământul, cerul se cutremură, soarele și luna se întunecă, iar stelele își pierd strălucirea lor.
\par 11 Dar Domnul Își face auzit glasul în fruntea oștirilor Sale, căci întinsă foarte este tabăra Lui și puternic este cel ce împlinește poruncile Lui. Ziua Domnului este mare și înfricoșătoare foarte și cine va putea sta împotriva ei?
\par 12 "Și acum, zice Domnul, întoarceți-vă la Mine din toată inima voastră, cu postiri, cu plâns și cu tânguire".
\par 13 Sfâșiați inimile și nu hainele voastre, și întoarceți-vă către Domnul Dumnezeul vostru, căci El este milostiv și îndurat, încet la mânie și mult Milostiv și-I pare rău de răul pe care l-a trimis asupra voastră.
\par 14 O, de v-ați întoarce și v-ați pocăi, ar rămâne de pe urma voastră o binecuvântare, un prinos și o jertfă cu turnare pentru Domnul Dumnezeul vostru!
\par 15 Sunați din trâmbiță în Sion, gătiți postiri sfinte, prăznuiți sărbătoarea cea pentru toți!
\par 16 Adunați poporul, vestiți o adunare sfântă, strângeți laolaltă pe bătrâni, aduceți copiii și pruncii care sug la sân; să iasă mirele din cămara lui și mireasa din iatacul ei!
\par 17 Între tindă și altar să plângă preoții, slujitorii Domnului, și să zică: "Milostivește-Te, Doamne, către poporul Tău și nu face de ocară moștenirea Ta ca să-și bată joc de ea neamurile!" Pentru ce să se spună printre neamuri: "Unde este Dumnezeul lor?"
\par 18 Dar Domnul este plin de zel pentru țara Sa și Se îndură de poporul Său.
\par 19 Pentru aceasta a răspuns Domnul către poporul Său, grăind: "Iată, Eu vă voi trimite grâu, must și untdelemn și vă voi sătura și nu vă voi mai face de ocară printre neamuri!
\par 20 Prăpădul dușmanului de la miazănoapte îl voi depărta de la voi și-l voi izgoni înspre un ținut uscat și pustiu; capătul lui spre marea cea de la răsărit, iar sfârșitul la marea cea dinspre apus; și se va ridica din el duhoare și miros de stârv va porni din el, căci a săvârșit lucruri mari".
\par 21 Nu te teme, tu țară, bucură-te și te veselește, căci Domnul a făcut lucruri mari!
\par 22 Nu vă temeți nici voi dobitoacele câmpului, căci au înverzit pășunile pustiului; căci pomii dau roadele lor; smochinul și vița de vie vor fi plini de roade.
\par 23 Și voi locuitori ai Sionului, bucurați-vă și vă veseliți în Domnul Dumnezeul vostru, căci El v-a dat pe Învățătorul dreptății; și v-a mai trimis și ploaie, ploaie timpurie și târzie, ca odinioară.
\par 24 Și ariile se vor umple de grâu; iar teascurile vor da peste margini de must și de untdelemn.
\par 25 Și vă voi da ani de belșug în locul anilor în care au mâncat lăcustele mici, cele mari, cele zburătoare și stolurile de lăcuste, marea Mea oștire pe care am trimis-o împotriva voastră.
\par 26 Și veți mânca din destul și vă veți sătura și veți preaslăvi numele Domnului Dumnezeului vostru, Care a făcut cu voi lucruri minunate. Și poporul Meu nu se va rușina în veci de veci!
\par 27 Atunci vă veți da seama că Eu sunt în mijlocul lui Israel și că Eu sunt Domnul Dumnezeul vostru și nu este altul, iar poporul Meu nu va mai fi niciodată de ocară!
\par 28 Dar după aceea, vărsa-voi Duhul Meu peste tot trupul, și fiii și fiicele voastre vor profeți, bătrânii voștri visuri vor visa iar tinerii voștri vedenii vor vedea.
\par 29 Chiar și peste robi și peste roabe voi vărsa Duhul Meu.
\par 30 Și vă voi arăta semne minunate în cer și pe pământ: sânge, foc și stâlpi de fum;
\par 31 Soarele se va întuneca și luna va fi roșie ca sângele, înainte de venirea zilei celei mari și înfricoșătoare a Domnului.
\par 32 Și oricine va chema numele Domnului se va izbăvi, căci în muntele Sionului și în Ierusalim va fi mântuirea, precum a zis Domnul; și între cei mântuiți, numai cei ce cheamă pe Domnul.

\chapter{3}

\par 1 Căci iată în zilele acelea și în vremea aceea, când voi întoarce pe Iuda și pe Ierusalim din robie,
\par 2 Aduna-voi toate popoarele și le voi coborî în valea lui Iosafat și Mă voi judeca acolo cu ele pentru poporul Meu și pentru moștenirea Mea Israel, pe care au împrăștiat-o între neamuri și țara Mea au împărțit-o în bucăți.
\par 3 Căci ei au aruncat sorți asupra poporului Meu, au dat copilul pentru o desfrânată și fata tânără au vândut-o pentru vin și au băut prețul ei.
\par 4 Ce-Mi căutați ceartă voi, Tirule și Sidonule, împreună cu toate ținuturile Filisteii? Oare vreți să vă răzbunați împotriva Mea? Dacă vreți să vă răzbunați asupra Mea, îndată voi face să cadă răzbunarea asupra capetelor voastre.
\par 5 Voi ați luat aurul și argintul Meu, precum și odoarele Mele, și le-ați dus în palatele voastre.
\par 6 Pe fiii lui Iuda și pe cei ai Ierusalimului i-ați vândut Grecilor, ca să-i depărtați din țara lor.
\par 7 Iată Eu îi voi stârpi din ținutul unde i-ați vândut și voi întoarce fapta voastră asupra capului vostru.
\par 8 Și voi vinde pe fiii și pe fiicele voastre feciorilor lui Iuda, iar ei îi vor vinde Sabeenilor, popor tare departe, căci așa a grăit Domnul.
\par 9 Dați neamurilor de știre lucrul acesta: Pregătiți-vă de război! Înflăcărați vitejii! Toți bărbații buni de luptă să dea fuga și să se apropie.
\par 10 Faceți din brăzdarele voastre săbii și din secerile voastre lănci! Cel slab să zică: "Eu sunt viteaz!"
\par 11 Alergați în grabă și veniți, voi, toate popoarele din jur, și adunați-vă aici! - Acolo, coboară, Doamne, pe vitejii Tăi!
\par 12 Să se trezească toate neamurile și să vină în valea lui Iosafat, căci acolo voi așeza scaun de judecată pentru toate popoarele din jur.
\par 13 Aduceți seceri, căci holda este coaptă, veniți, coborâți-vă, căci teascul este plin, albiile dau peste margini; fărădelegile lor n-au seamăn.
\par 14 Mulțimi și iar mulțimi în Valea Judecății. Căci aproape este ziua Domnului în Valea judecății!
\par 15 Soarele și luna se întunecă și stelele își pierd lumina lor.
\par 16 Din Sion Domnul va striga puternic și din Ierusalim va slobozi tunetul Său: pământul și cerurile se vor cutremura atunci.
\par 17 Atunci veți cunoaște că Eu sunt Domnul Dumnezeul vostru, Care sălășluiește în Sion, în muntele cel sfânt al Meu; Ierusalimul va fi altarul Meu, iar străinii nu vor mai trece pe acolo.
\par 18 Și în vremea aceea, din munți va curge must, văile vor fi pline de lapte, toate pâraiele din Iuda vor șerpui umplute de apă, iar din templul Domnului va ieși un izvor oare va uda valea Șitim.
\par 19 Egiptul va fi pustiit și Edomul se va preface în ținut nelocuit, din pricina silniciilor împotriva fiilor lui Iuda, fiindcă au vărsat sânge nevinovat în pământul lor.
\par 20 Dar Iuda va fi locuit în veci și de-a pururi și Ierusalimul din neam în neam.
\par 21 Voi răzbuna sângele pe care nu l-am răzbunat; și Domnul va locui Sionul în veac.


\end{document}