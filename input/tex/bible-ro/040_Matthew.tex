\begin{document}

\title{Matei}


\chapter{1}

\par 1 Cartea neamului lui Iisus Hristos, fiul lui David, fiul lui Avraam.
\par 2 Avraam a născut pe Isaac; Isaac a născut pe Iacov; Iacov a născut pe Iuda și frații lui;
\par 3 Iuda a născut pe Fares și pe Zara, din Tamar; Fares a născut pe Esrom; Esrom a născut pe Aram;
\par 4 Aram a născut pe Aminadav; Aminadav a născut pe Naason; Naason a născut pe Salmon;
\par 5 Salmon a născut pe Booz, din Rahav; Booz a născut pe Iobed, din Rut; Iobed a născut pe Iesei;
\par 6 Iesei a născut pe David regele; David a născut pe Solomon din femeia lui Urie;
\par 7 Solomon a născut pe Roboam; Roboam a născut pe Abia; Abia a născut pe Asa;
\par 8 Asa a născut pe Iosafat; Iosafat a născut pe Ioram; Ioram a născut pe Ozia;
\par 9 Ozia a născut pe Ioatam; Ioatam a născut pe Ahaz; Ahaz a născut pe Iezechia;
\par 10 Iezechia a născut pe Manase; Manase a născut pe Amon; Amon a născut pe Iosia;
\par 11 Iosia a născut pe Iehonia și pe frații lui, la strămutarea în Babilon;
\par 12 După strămutarea în Babilon, Iehonia a născut pe Salatiel; Salatiel a născut pe Zorobabel;
\par 13 Zorobabel a născut pe Abiud; Abiud a născut pe Eliachim; Eliachim a născut pe Azor;
\par 14 Azor a născut pe Sadoc; Sadoc a născut pe Achim; Achim a născut pe Eliud;
\par 15 Eliud a născut pe Eleazar; Eleazar a născut pe Matan; Matan a născut pe Iacov;
\par 16 Iacov a născut pe Iosif, logodnicul Mariei, din care S-a născut Iisus, Care se cheamă Hristos.
\par 17 Așadar, toate neamurile de la Avraam până la David sunt paisprezece; și de la David până la strămutarea în Babilon sunt paisprezece; și de la strămutarea în Babilon până la Hristos sunt paisprezece neamuri.
\par 18 Iar nașterea lui Iisus Hristos așa a fost: Maria, mama Lui, fiind logodită cu Iosif, fără să fi fost ei înainte împreună, s-a aflat având în pântece de la Duhul Sfânt.
\par 19 Iosif, logodnicul ei, drept fiind și nevrând s-o vădească, a voit s-o lase în ascuns.
\par 20 Și cugetând el acestea, iată îngerul Domnului i s-a arătat în vis, grăind: Iosife, fiul lui David, nu te teme a lua pe Maria, logodnica ta, că ce s-a zămislit într-însa este de la Duhul Sfânt.
\par 21 Ea va naște Fiu și vei chema numele Lui: Iisus, căci El va mântui poporul Său de păcatele lor.
\par 22 Acestea toate s-au făcut ca să se împlinească ceea ce s-a zis de Domnul prin proorocul care zice:
\par 23 "Iată, Fecioara va avea în pântece și va naște Fiu și vor chema numele Lui Emanuel, care se tâlcuiește: Cu noi este Dumnezeu".
\par 24 Și deșteptându-se din somn, Iosif a făcut așa precum i-a poruncit îngerul Domnului și a luat la el pe logodnica sa.
\par 25 Și fără să fi cunoscut-o pe ea Iosif, Maria a născut pe Fiul său Cel Unul-Născut, Căruia I-a pus numele Iisus.

\chapter{2}

\par 1 Iar dacă S-a născut Iisus în Betleemul Iudeii, în zilele lui Irod regele, iată magii de la Răsărit au venit în Ierusalim, întrebând:
\par 2 Unde este regele Iudeilor, Cel ce S-a născut? Căci am văzut la Răsărit steaua Lui și am venit să ne închinăm Lui.
\par 3 Și auzind, regele Irod s-a tulburat și tot Ierusalimul împreună cu el.
\par 4 Și adunând pe toți arhiereii și cărturarii poporului, căuta să afle de la ei: Unde este să Se nască Hristos?
\par 5 Iar ei i-au zis: În Betleemul Iudeii, că așa este scris de proorocul:
\par 6 "Și tu, Betleeme, pământul lui Iuda, nu ești nicidecum cel mai mic între căpeteniile lui Iuda, căci din tine va ieși Conducătorul care va paște pe poporul Meu Israel".
\par 7 Atunci Irod chemând în ascuns pe magi, a aflat de la ei lămurit în ce vreme s-a arătat steaua.
\par 8 Și trimițându-i la Betleem, le-a zis: Mergeți și cercetați cu de-amănuntul despre Prunc și, dacă Îl veți afla, vestiți-mi și mie, ca, venind și eu, să mă închin Lui.
\par 9 Iar ei, ascultând pe rege, au plecat și iată, steaua pe care o văzuseră în Răsărit mergea înaintea lor, până ce a venit și a stat deasupra, unde era Pruncul.
\par 10 Și văzând ei steaua, s-au bucurat cu bucurie mare foarte.
\par 11 Și intrând în casă, au văzut pe Prunc împreună cu Maria, mama Lui, și căzând la pământ, s-au închinat Lui; și deschizând vistieriile lor, I-au adus Lui daruri: aur, tămâie și smirnă.
\par 12 Iar luând înștiințare în vis să nu se mai întoarcă la Irod, pe altă cale s-au dus în țara lor.
\par 13 După plecarea magilor, iată îngerul Domnului se arată în vis lui Iosif, zicând: Scoală-te, ia Pruncul și pe mama Lui și fugi în Egipt și stai acolo până ce-ți voi spune, fiindcă Irod are să caute Pruncul ca să-L ucidă.
\par 14 Și sculându-se, a luat, noaptea, Pruncul și pe mama Lui și a plecat în Egipt.
\par 15 Și au stat acolo până la moartea lui Irod, ca să se împlinească cuvântul spus de Domnul, prin proorocul: "Din Egipt am chemat pe Fiul Meu".
\par 16 Iar când Irod a văzut că a fost amăgit de magi, s-a mâniat foarte și, trimițând a ucis pe toți pruncii care erau în Betleem și în toate hotarele lui, de doi ani și mai jos, după timpul pe care îl aflase de la magi.
\par 17 Atunci s-a împlinit ceea ce se spusese prin Ieremia proorocul:
\par 18 "Glas în Rama s-a auzit, plângere și tânguire multă; Rahela își plânge copiii și nu voiește să fie mângâiată pentru că nu sunt".
\par 19 După moartea lui Irod, iată că îngerul Domnului s-a arătat în vis lui Iosif în Egipt,
\par 20 Și i-a zis: Scoală-te, ia Pruncul și pe mama Lui și mergi în pământul lui Israel, căci au murit cei ce căutau să ia sufletul Pruncului.
\par 21 Iosif, sculându-se, a luat Pruncul și pe mama Lui și a venit în pământul lui Israel.
\par 22 Și auzind că domnește Arhelau în Iudeea, în locul lui Irod, tatăl său, s-a temut să meargă acolo și, luând poruncă, în vis, s-a dus în părțile Galileii.
\par 23 Și venind a locuit în orașul numit Nazaret, ca să se împlinească ceea ce s-a spus prin prooroci, că Nazarinean Se va chema.

\chapter{3}

\par 1 În zilele acelea, a venit Ioan Botezătorul și propovăduia în pustia Iudeii,
\par 2 Spunând: Pocăiți-vă că s-a apropiat împărăția cerurilor.
\par 3 El este acela despre care a zis proorocul Isaia: "Glasul celui ce strigă în pustie: Pregătiți calea Domnului, drepte faceți cărările Lui".
\par 4 Iar Ioan avea îmbrăcămintea lui din păr de cămilă, și cingătoare de piele împrejurul mijlocului, iar hrana era lăcuste și miere sălbatică.
\par 5 Atunci a ieșit la el Ierusalimul și toată Iudeea și toată împrejurimea Iordanului.
\par 6 Și erau botezați de către el în râul Iordan, mărturisindu-și păcatele.
\par 7 Dar văzând Ioan pe mulți din farisei și saduchei venind la botez, le-a zis: Pui de vipere, cine v-a arătat să fugiți de mânia ce va să fie?
\par 8 Faceți deci roadă, vrednică de pocăință,
\par 9 Și să nu credeți că puteți zice în voi înșivă: Părinte avem pe Avraam, căci vă spun că Dumnezeu poate și din pietrele acestea să ridice fii lui Avraam.
\par 10 Iată securea stă la rădăcina pomilor și tot pomul care nu face roadă bună se taie și se aruncă în foc.
\par 11 Eu unul vă botez cu apă spre pocăință, dar Cel ce vine după mine este mai puternic decât mine; Lui nu sunt vrednic să-I duc încălțămintea; Acesta vă va boteza cu Duh Sfânt și cu foc.
\par 12 El are lopata în mână și va curăța aria Sa și va aduna grâul în jitniță, iar pleava o va arde cu foc nestins.
\par 13 În acest timp a venit Iisus din Galileea, la Iordan, către Ioan, ca să se boteze de către el.
\par 14 Ioan însă Îl oprea, zicând: Eu am trebuință să fiu botezat de Tine, și Tu vii la mine?
\par 15 Și răspunzând, Iisus a zis către el: Lasă acum, că așa se cuvine nouă să împlinim toată dreptatea. Atunci L-a lăsat.
\par 16 Iar botezându-se Iisus, când ieșea din apă, îndată cerurile s-au deschis și Duhul lui Dumnezeu s-a văzut pogorându-se ca un porumbel și venind peste El.
\par 17 Și iată glas din ceruri zicând: "Acesta este Fiul Meu cel iubit întru Care am binevoit".

\chapter{4}

\par 1 Atunci Iisus a fost dus de Duhul în pustiu, ca să fie ispitit de către diavolul.
\par 2 Și după ce a postit patruzeci de zile și patruzeci de nopți, la urmă a flămânzit.
\par 3 Și apropiindu-se, ispititorul a zis către El: De ești Tu Fiul lui Dumnezeu, zi ca pietrele acestea să se facă pâini.
\par 4 Iar El, răspunzând, a zis: Scris este: "Nu numai cu pâine va trăi omul, ci cu tot cuvântul care iese din gura lui Dumnezeu".
\par 5 Atunci diavolul L-a dus pe aripa în sfânta cetate, L-a pus pe aripa templului,
\par 6 Și I-a zis: Dacă Tu ești Fiul lui Dumnezeu, aruncă-Te jos, că scris este: "Îngerilor Săi va porunci pentru Tine și Te vor ridica pe mâini, ca nu cumva să izbești de piatră piciorul Tău".
\par 7 Iisus i-a răspuns: Iarăși este scris: "Să nu ispitești pe Domnul Dumnezeul tău".
\par 8 Din nou diavolul L-a dus pe un munte foarte înalt și I-a arătat toate împărățiile lumii și slava lor.
\par 9 Și I-a zis Lui: Acestea toate Ți le voi da Ție, dacă vei cădea înaintea mea și Te vei închina mie.
\par 10 Atunci Iisus i-a zis: Piei, satano, căci scris este: "Domnului Dumnezeului tău să te închini și Lui singur să-I slujești".
\par 11 Atunci L-a lăsat diavolul și iată îngerii, venind la El, Îi slujeau.
\par 12 Și Iisus, auzind că Ioan a fost întemnițat, a plecat în Galileea.
\par 13 Și părăsind Nazaretul, a venit de a locuit în Capernaum, lângă mare, în hotarele lui Zabulon și Neftali,
\par 14 Ca să se împlinească ce s-a zis prin Isaia proorocul care zice:
\par 15 "Pământul lui Zabulon și pământul lui Neftali spre mare, dincolo de Iordan, Galileea neamurilor;
\par 16 Poporul care stătea în întuneric a văzut lumină mare și celor ce ședeau în latura și în umbra morții lumină le-a răsărit".
\par 17 De atunci a început Iisus să propovăduiască și să spună: Pocăiți-vă, căci s-a apropiat împărăția cerurilor.
\par 18 Pe când umbla pe lângă Marea Galileii, a văzut pe doi frați, pe Simon ce se numește Petru și pe Andrei, fratele lui, care aruncau mreaja în mare, căci erau pescari.
\par 19 Și le-a zis: Veniți după Mine și vă voi face pescari de oameni.
\par 20 Iar ei, îndată lăsând mrejele, au mers după El.
\par 21 Și de acolo, mergând mai departe, a văzut alți doi frați, pe Iacov al lui Zevedeu și pe Ioan fratele lui, în corabie cu Zevedeu, tatăl lor, dregându-și mrejele și i-a chemat.
\par 22 Iar ei îndată, lăsând corabia și pe tatăl lor, au mers după El.
\par 23 Și a străbătut Iisus toată Galileea, învățând în sinagogile lor și propovăduind Evanghelia împărăției și tămăduind toată boala și toată neputința în popor.
\par 24 Și s-a dus vestea despre El în toată Siria, și aduceau la El pe toți cei ce se aflau în suferințe, fiind cuprinși de multe feluri de boli și de chinuri, pe demonizați, pe lunatici, pe slăbănogi, și El îi vindeca.
\par 25 Și mulțimi multe mergeau după El, din Galileea, din Decapole, din Ierusalim, din Iudeea și de dincolo de Iordan.

\chapter{5}

\par 1 Văzând mulțimile, Iisus S-a suit în munte, și așezându-se, ucenicii Lui au venit la El.
\par 2 Și deschizându-și gura, îi învăța zicând:
\par 3 Fericiți cei săraci cu duhul, că a lor este împărăția cerurilor.
\par 4 Fericiți cei ce plâng, că aceia se vor mângâia.
\par 5 Fericiți cei blânzi, că aceia vor moșteni pământul.
\par 6 Fericiți cei ce flămânzesc și însetează de dreptate, că aceia se vor sătura.
\par 7 Fericiți cei milostivi, că aceia se vor milui.
\par 8 Fericiți cei curați cu inima, că aceia vor vedea pe Dumnezeu.
\par 9 Fericiți făcătorii de pace, că aceia fiii lui Dumnezeu se vor chema.
\par 10 Fericiți cei prigoniți pentru dreptate, că a lor este împărăția cerurilor.
\par 11 Fericiți veți fi voi când vă vor ocărî și vă vor prigoni și vor zice tot cuvântul rău împotriva voastră, mințind din pricina Mea.
\par 12 Bucurați-vă și vă veseliți, că plata voastră multă este în ceruri, că așa au prigonit pe proorocii cei dinainte de voi.
\par 13 Voi sunteți sarea pământului; dacă sarea se va strica, cu ce se va săra? De nimic nu mai e bună decât să fie aruncată afară și călcată în picioare de oameni.
\par 14 Voi sunteți lumina lumii; nu poate o cetate aflată pe vârf de munte să se ascundă.
\par 15 Nici nu aprind făclie și o pun sub obroc, ci în sfeșnic, și luminează tuturor celor din casă.
\par 16 Așa să lumineze lumina voastră înaintea oamenilor, așa încât să vadă faptele voastre cele bune și să slăvească pe Tatăl vostru Cel din ceruri.
\par 17 Să nu socotiți că am venit să stric Legea sau proorocii; n-am venit să stric, ci să împlinesc.
\par 18 Căci adevărat zic vouă: Înainte de a trece cerul și pământul, o iotă sau o cirtă din Lege nu va trece, până ce se vor face toate.
\par 19 Deci, cel ce va strica una din aceste porunci, foarte mici, și va învăța așa pe oameni, foarte mic se va chema în împărăția cerurilor; iar cel ce va face și va învăța, acesta mare se va chema în împărăția cerurilor.
\par 20 Căci zic vouă: Că de nu va prisosi dreptatea voastră mai mult decât a cărturarilor și a fariseilor, nu veți intra în împărăția cerurilor.
\par 21 Ați auzit că s-a zis celor de demult: "Să nu ucizi"; iar cine va ucide, vrednic va fi de osândă.
\par 22 Eu însă vă spun vouă: Că oricine se mânie pe fratele său vrednic va fi de osândă; și cine va zice fratelui său: netrebnicule, vrednic va fi de judecata sinedriului; iar cine va zice: nebunule, vrednic va fi de gheena focului.
\par 23 Deci, dacă îți vei aduce darul tău la altar și acolo îți vei aduce aminte că fratele tău are ceva împotriva ta,
\par 24 Lasă darul tău acolo, înaintea altarului, și mergi întâi și împacă-te cu fratele tău și apoi, venind, adu darul tău.
\par 25 Împacă-te cu pârâșul tău degrabă, până ești cu el pe cale, ca nu cumva pârâșul să te dea judecătorului, și judecătorul slujitorului și să fii aruncat în temniță.
\par 26 Adevărat grăiesc ție: Nu vei ieși de acolo, până ce nu vei fi dat cel din urmă ban.
\par 27 Ați auzit că s-a zis celor de demult: "Să nu săvârșești adulter".
\par 28 Eu însă vă spun vouă: Că oricine se uită la femeie, poftind-o, a și săvârșit adulter cu ea în inima lui.
\par 29 Iar dacă ochiul tău cel drept te smintește pe tine, scoate-l și aruncă-l de la tine, căci mai de folos îți este să piară unul din mădularele tale, decât tot trupul să fie aruncat în gheenă.
\par 30 Și dacă mâna ta cea dreaptă te smintește pe tine, taie-o și o aruncă de la tine, căci mai de folos îți este să piară unul din mădularele tale, decât tot trupul tău să fie aruncat în gheenă.
\par 31 S-a zis iarăși: "Cine va lăsa pe femeia sa, să-i dea carte de despărțire".
\par 32 Eu însă vă spun vouă: Că oricine va lăsa pe femeia sa, în afară de pricină de desfrânare, o face să săvârșească adulter, și cine va lua pe cea lăsată săvârșește adulter.
\par 33 Ați auzit ce s-a zis celor de demult: "Să nu juri strâmb, ci să ții înaintea Domnului jurămintele tale".
\par 34 Eu însă vă spun vouă: Să nu vă jurați nicidecum nici pe cer, fiindcă este tronul lui Dumnezeu,
\par 35 Nici pe pământ, fiindcă este așternut al picioarelor Lui, nici pe Ierusalim, fiindcă este cetate a marelui Împărat,
\par 36 Nici pe capul tău să nu te juri, fiindcă nu poți să faci un fir de păr alb sau negru,
\par 37 Ci cuvântul vostru să fie: Ceea ce este da, da; și ceea ce este nu, nu; iar ce e mai mult decât acestea, de la cel rău este.
\par 38 Ați auzit că s-a zis: "Ochi pentru ochi și dinte pentru dinte".
\par 39 Eu însă vă spun vouă: Nu vă împotriviți celui rău; iar cui te lovește peste obrazul drept, întoarce-i și pe celălalt.
\par 40 Celui ce voiește să se judece cu tine și să-ți ia haina, lasă-i și cămașa.
\par 41 Iar de te va sili cineva să mergi o milă, mergi cu el două.
\par 42 Celui care cere de la tine, dă-i; și de la cel ce voiește să se împrumute de la tine, nu întoarce fața ta.
\par 43 Ați auzit că s-a zis: "Să iubești pe aproapele tău și să urăști pe vrăjmașul tău".
\par 44 Iar Eu zic vouă: Iubiți pe vrăjmașii voștri, binecuvântați pe cei ce vă blestemă, faceți bine celor ce vă urăsc și rugați-vă pentru cei ce vă vatămă și vă prigonesc,
\par 45 Ca să fiți fiii Tatălui vostru Celui din ceruri, că El face să răsară soarele și peste cei răi și peste cei buni și trimite ploaie peste cei drepți și peste cei nedrepți.
\par 46 Căci dacă iubiți pe cei ce vă iubesc, ce răsplată veți avea? Au nu fac și vameșii același lucru?
\par 47 Și dacă îmbrățișați numai pe frații voștri, ce faceți mai mult? Au nu fac și neamurile același lucru?
\par 48 Fiți, dar, voi desăvârșiți, precum Tatăl vostru Cel ceresc desăvârșit este.

\chapter{6}

\par 1 Luați aminte ca faptele dreptății voastre să nu le faceți înaintea oamenilor ca să fiți văzuți de ei; altfel nu veți avea plată de la Tatăl vostru Cel din ceruri.
\par 2 Deci, când faci milostenie, nu trâmbița înaintea ta, cum fac fățarnicii în sinagogi și pe ulițe, ca să fie slăviți de oameni; adevărat grăiesc vouă: și-au luat plata lor.
\par 3 Tu însă, când faci milostenie, să nu știe stânga ta ce face dreapta ta,
\par 4 Ca milostenia ta să fie într-ascuns și Tatăl tău, Care vede în ascuns, îți va răsplăti ție.
\par 5 Iar când vă rugați, nu fiți ca fățarnicii cărora le place, prin sinagogi și prin colțurile ulițelor, stând în picioare, să se roage, ca să se arate oamenilor; adevărat grăiesc vouă: și-au luat plata lor.
\par 6 Tu însă, când te rogi, intră în cămara ta și, închizând ușa, roagă-te Tatălui tău, Care este în ascuns, și Tatăl tău, Care este în ascuns, îți va răsplăti ție.
\par 7 Când vă rugați, nu spuneți multe ca neamurile, că ele cred că în multa lor vorbărie vor fi ascultate.
\par 8 Deci nu vă asemănați lor, că știe Tatăl vostru de cele ce aveți trebuință mai înainte ca să cereți voi de la El.
\par 9 Deci voi așa să vă rugați: Tatăl nostru, Care ești în ceruri, sfințească-se numele Tău;
\par 10 Vie împărăția Ta; facă-se voia Ta, precum în cer și pe pământ.
\par 11 Pâinea noastră cea spre ființă dă-ne-o nouă astăzi;
\par 12 Și ne iartă nouă greșealele noastre, precum și noi iertăm greșiților noștri;
\par 13 Și nu ne duce pe noi în ispită, ci ne izbăvește de cel rău. Că a Ta este împărăția și puterea și slava în veci. Amin!
\par 14 Că de veți ierta oamenilor greșealele lor, ierta-va și vouă Tatăl vostru Cel ceresc;
\par 15 Iar de nu veți ierta oamenilor greșealele lor, nici Tatăl vostru nu vă va ierta greșealele voastre.
\par 16 Când postiți, nu fiți triști ca fățarnicii; că ei își smolesc fețele, ca să se arate oamenilor că postesc. Adevărat grăiesc vouă, și-au luat plata lor.
\par 17 Tu însă, când postești, unge capul tău și fața ta o spală,
\par 18 Ca să nu te arăți oamenilor că postești, ci Tatălui tău care este în ascuns, și Tatăl tău, Care vede în ascuns, îți va răsplăti ție.
\par 19 Nu vă adunați comori pe pământ, unde molia și rugina le strică și unde furii le sapă și le fură.
\par 20 Ci adunați-vă comori în cer, unde nici molia, nici rugina nu le strică, unde furii nu le sapă și nu le fură.
\par 21 Căci unde este comoara ta, acolo va fi și inima ta.
\par 22 Luminătorul trupului este ochiul; de va fi ochiul tău curat, tot trupul tău va fi luminat.
\par 23 Iar de va fi ochiul tău rău, tot trupul tău va fi întunecat. Deci, dacă lumina care e în tine este întuneric, dar întunericul cu cât mai mult!
\par 24 Nimeni nu poate să slujească la doi domni, căci sau pe unul îl va urî și pe celălalt îl va iubi, sau de unul se va lipi și pe celălalt îl va disprețui; nu puteți să slujiți lui Dumnezeu și lui mamona.
\par 25 De aceea zic vouă: Nu vă îngrijiți pentru sufletul vostru ce veți mânca, nici pentru trupul vostru cu ce vă veți îmbrăca; au nu este sufletul mai mult decât hrana și trupul decât îmbrăcămintea?
\par 26 Priviți la păsările cerului, că nu seamănă, nici nu seceră, nici nu adună în jitnițe, și Tatăl vostru Cel ceresc le hrănește. Oare nu sunteți voi cu mult mai presus decât ele?
\par 27 Și cine dintre voi, îngrijindu-se poate să adauge staturii sale un cot?
\par 28 Iar de îmbrăcăminte de ce vă îngrijiți? Luați seama la crinii câmpului cum cresc: nu se ostenesc, nici nu torc.
\par 29 Și vă spun vouă că nici Solomon, în toată mărirea lui, nu s-a îmbrăcat ca unul dintre aceștia.
\par 30 Iar dacă iarba câmpului, care astăzi este și mâine se aruncă în cuptor, Dumnezeu astfel o îmbracă, oare nu cu mult mai mult pe voi, puțin credincioșilor?
\par 31 Deci, nu duceți grijă, spunând: Ce vom mânca, ori ce vom bea, ori cu ce ne vom îmbrăca?
\par 32 Că după toate acestea se străduiesc neamurile; știe doar Tatăl vostru Cel ceresc că aveți nevoie de ele.
\par 33 Căutați mai întâi împărăția lui Dumnezeu și dreptatea Lui și toate acestea se vor adăuga vouă.
\par 34 Nu vă îngrijiți de ziua de mâine, căci ziua de mâine se va îngriji de ale sale. Ajunge zilei răutatea ei.

\chapter{7}

\par 1 Nu judecați, ca să nu fiți judecați.
\par 2 Căci cu judecata cu care judecați, veți fi judecați, și cu măsura cu care măsurați, vi se va măsura.
\par 3 De ce vezi paiul din ochiul fratelui tău, și bârna din ochiul tău nu o iei în seamă?
\par 4 Sau cum vei zice fratelui tău: Lasă să scot paiul din ochiul tău și iată bârna este în ochiul tău?
\par 5 Fățarnice, scoate întâi bârna din ochiul tău și atunci vei vedea să scoți paiul din ochiul fratelui tău.
\par 6 Nu dați cele sfinte câinilor, nici nu aruncați mărgăritarele voastre înaintea porcilor, ca nu cumva să le calce în picioare și, întorcându-se, să vă sfâșie pe voi.
\par 7 Cereți și vi se va da; căutați și veți afla; bateți și vi se va deschide.
\par 8 Că oricine cere ia, cel care caută află, și celui ce bate i se va deschide.
\par 9 Sau cine este omul acela între voi care, de va cere fiul său pâine, oare el îi va da piatră?
\par 10 Sau de-i va cere pește, oare el îi va da șarpe?
\par 11 Deci, dacă voi, răi fiind, știți să dați daruri bune fiilor voștri, cu cât mai mult Tatăl vostru Cel din ceruri va da cele bune celor care cer de la El?
\par 12 Ci toate câte voiți să vă facă vouă oamenii, asemenea și voi faceți lor, că aceasta este Legea și proorocii.
\par 13 Intrați prin poarta cea strâmtă, că largă este poarta și lată este calea care duce la pieire și mulți sunt cei care o află.
\par 14 Și strâmtă este poarta și îngustă este calea care duce la viață și puțini sunt care o află.
\par 15 Feriți-vă de proorocii mincinoși, care vin la voi în haine de oi, iar pe dinăuntru sunt lupi răpitori.
\par 16 După roadele lor îi veți cunoaște. Au doară culeg oamenii struguri din spini sau smochine din mărăcini?
\par 17 Așa că orice pom bun face roade bune, iar pomul rău face roade rele.
\par 18 Nu poate pom bun să facă roade rele, nici pom rău să facă roade bune.
\par 19 Iar orice pom care nu face roadă bună se taie și se aruncă în foc.
\par 20 De aceea, după roadele lor îi veți cunoaște.
\par 21 Nu oricine Îmi zice: Doamne, Doamne, va intra în împărăția cerurilor, ci cel ce face voia Tatălui Meu Celui din ceruri.
\par 22 Mulți Îmi vor zice în ziua aceea: Doamne, Doamne, au nu în numele Tău am proorocit și nu în numele Tău am scos demoni și nu în numele Tău minuni multe am făcut?
\par 23 Și atunci voi mărturisi lor: Niciodată nu v-am cunoscut pe voi. Depărtați-vă de la Mine cei ce lucrați fărădelegea.
\par 24 De aceea, oricine aude aceste cuvinte ale Mele și la îndeplinește asemăna-se-va bărbatului înțelept care a clădit casa lui pe stâncă.
\par 25 A căzut ploaia, au venit râurile mari, au suflat vânturile și au bătut în casa aceea, dar ea n-a căzut, fiindcă era întemeiată pe stâncă.
\par 26 Iar oricine aude aceste cuvinte ale Mele și nu le îndeplinește, asemăna-se-va bărbatului nechibzuit care și-a clădit casa pe nisip.
\par 27 Și a căzut ploaia și au venit râurile mari și au suflat vânturile și au izbit casa aceea, și a căzut. Și căderea ei a fost mare.
\par 28 Iar când Iisus a sfârșit cuvintele acestea, mulțimile erau uimite de învățătura Lui.
\par 29 Că îi învăța pe ei ca unul care are putere, iar nu cum îi învățau cărturarii lor.

\chapter{8}

\par 1 Și coborându-Se El din munte, mulțimi multe au mers după El.
\par 2 Și iată un lepros, apropiindu-se, I se închina, zicând: Doamne, dacă voiești, poți să mă curățești.
\par 3 Și Iisus, întinzând mâna, S-a atins de el, zicând: Voiesc, curățește-te. Și îndată s-a curățit lepra lui.
\par 4 Și i-a zis Iisus: Vezi, nu spune nimănui, ci mergi, arată-te preotului și adu darul pe care l-a rânduit Moise, spre mărturie lor.
\par 5 Pe când intra în Capernaum, s-a apropiat de El un sutaș, rugându-L,
\par 6 Și zicând: Doamne, sluga mea zace în casă, slăbănog, chinuindu-se cumplit.
\par 7 Și i-a zis Iisus: Venind, îl voi vindeca.
\par 8 Dar sutașul, răspunzând, I-a zis: Doamne, nu sunt vrednic să intri sub acoperișul meu, ci numai zi cu cuvântul și se va vindeca sluga mea.
\par 9 Că și eu sunt om sub stăpânirea altora și am sub mine ostași și-i spun acestuia: Du-te, și se duce; și celuilalt: Vino, și vine; și slugii mele: Fă aceasta, și face.
\par 10 Auzind, Iisus S-a minunat și a zis celor ce veneau după El: Adevărat grăiesc vouă: la nimeni, în Israel, n-am găsit atâta credință.
\par 11 Și zic vouă că mulți de la răsărit și de la apus vor veni și vor sta la masă cu Avraam, cu Isaac și cu Iacov în împărăția cerurilor.
\par 12 Iar fiii împărăției vor fi aruncați în întunericul cel mai din afară; acolo va fi plângerea și scrâșnirea dinților.
\par 13 Și a zis Iisus sutașului: Du-te, fie ție după cum ai crezut. Și s-a însănătoșit sluga lui în ceasul acela.
\par 14 Și venind Iisus în casa lui Petru, a văzut pe soacra acestuia zăcând, prinsă de friguri.
\par 15 Și S-a atins de mâna ei, și au lăsat-o frigurile și s-a sculat și Îi slujea Lui.
\par 16 Și făcându-se seară, au adus la El mulți demonizați și a scos duhurile cu cuvântul și pe toți cei bolnavi i-a vindecat,
\par 17 Ca să se împlinească ceea ce s-a spus prin Isaia proorocul, care zice: "Acesta neputințele noastre a luat și bolile noastre le-a purtat".
\par 18 Și văzând Iisus mulțime împrejurul Lui, a poruncit ucenicilor să treacă de cealaltă parte a mării.
\par 19 Și apropiindu-se un cărturar, i-a zis: Învățătorule, Te voi urma oriunde vei merge.
\par 20 Dar Iisus i-a răspuns: Vulpile au vizuini și păsările cerului cuiburi; Fiul Omului însă nu are unde să-Și plece capul.
\par 21 Un altul dintre ucenici I-a zis: Doamne, dă-mi voie întâi să mă duc și să îngrop pe tatăl meu.
\par 22 Iar Iisus i-a zis: Vino după Mine și lasă morții să-și îngroape morții lor.
\par 23 Intrând El în corabie, ucenicii Lui L-au urmat.
\par 24 Și, iată, furtună mare s-a ridicat pe mare, încât corabia se acoperea de valuri; iar El dormea.
\par 25 Și venind ucenicii la El, L-au deșteptat zicând: Doamne, mântuiește-ne, că pierim.
\par 26 Iisus le-a zis: De ce vă este frică, puțin credincioșilor? S-a sculat atunci, a certat vânturile și marea și s-a făcut liniște deplină.
\par 27 Iar oamenii s-au mirat, zicând: Cine este Acesta că și vânturile și marea ascultă de El?
\par 28 Și trecând El dincolo, în ținutul Gadarenilor, L-au întâmpinat doi demonizați, care ieșeau din morminte, foarte cumpliți, încât nimeni nu putea să treacă pe calea aceea.
\par 29 Și iată, au început să strige și să zică: Ce ai Tu cu noi, Iisuse, Fiul lui Dumnezeu? Ai venit aici mai înainte de vreme ca să ne chinuiești?
\par 30 Departe de ei era o turmă mare de porci, păscând.
\par 31 Iar demonii Îl rugau, zicând: Dacă ne scoți afară, trimite-ne în turma de porci.
\par 32 Și El le-a zis: Duceți-vă. Iar ei, ieșind, s-au dus în turma de porci. Și iată, toată turma s-a aruncat de pe țărm în mare și a pierit în apă.
\par 33 Iar păzitorii au fugit și, ducându-se în cetate, au spus toate cele întâmplate cu demonizații.
\par 34 Și iată toată cetatea a ieșit în întâmpinarea lui Iisus și, văzându-L, L-au rugat să treacă din hotarele lor.

\chapter{9}

\par 1 Intrând în corabie, Iisus a trecut și a venit în cetatea Sa.
\par 2 Și iată, I-au adus un slăbănog zăcând pe pat. Și Iisus, văzând credința lor, a zis slăbănogului: Îndrăznește, fiule! Iertate sunt păcatele tale!
\par 3 Dar unii dintre cărturari ziceau în sine: Acesta hulește.
\par 4 Și Iisus, știind gândurile lor, le-a zis: Pentru ce cugetați rele în inimile voastre?
\par 5 Căci ce este mai lesne a zice: Iertate sunt păcatele tale, sau a zice: Scoală-te și umblă?
\par 6 Dar ca să știți că putere are Fiul Omului pe pământ a ierta păcatele, a zis slăbănogului: Scoală-te, ia-ți patul și mergi la casa ta.
\par 7 Și, sculându-se, s-a dus la casa sa.
\par 8 Iar mulțimile văzând acestea, s-au înspăimântat și au slăvit pe Dumnezeu, Cel care dă oamenilor asemenea putere.
\par 9 Și plecând Iisus de acolo, a văzut un om care ședea la vamă, cu numele Matei, și i-a zis acestuia: Vino după Mine. Și sculându-se, a mers după El.
\par 10 Și pe când ședea El la masă, în casă, iată mulți vameși și păcătoși au venit și au șezut la masă împreună cu Iisus și cu ucenicii Lui.
\par 11 Și văzând fariseii, au zis ucenicilor: Pentru ce mănâncă Învățătorul vostru cu vameșii și cu păcătoșii?
\par 12 Și auzind El, a zis: Nu cei sănătoși au nevoie de doctor, ci cei bolnavi.
\par 13 Dar mergând, învățați ce înseamnă: Milă voiesc, iar nu jertfă; că n-am venit să chem pe drepți, ci pe păcătoși la pocăință.
\par 14 Atunci au venit la El ucenicii lui Ioan, zicând: Pentru ce noi și fariseii postim mult, iar ucenicii Tăi nu postesc?
\par 15 Și Iisus le-a zis: Pot oare, fiii nunții să fie triști câtă vreme mirele este cu ei? Dar vor veni zile când mirele va fi luat de la ei și atunci vor posti.
\par 16 Nimeni nu pune un petic de postav nou la o haină veche, căci peticul acesta, ca umplutură, trage din haină și se face o ruptură și mai rea.
\par 17 Nici nu pun oamenii vin nou în burdufuri vechi; alminterea burdufurile crapă: vinul se varsă și burdufurile se strică; ci pun vin nou în burdufuri noi și amândouă se păstrează împreună.
\par 18 Pe când le spunea acestea, iată un dregător, venind, I s-a închinat, zicând: Fiica mea a murit de curând dar, venind, pune mâna Ta peste ea și va fi vie.
\par 19 Atunci Iisus, sculându-Se, a mers după el împreună cu ucenicii.
\par 20 Și iată o femeie cu scurgere de sânge de doisprezece ani, apropiindu-se de El pe la spate, s-a atins de poala hainei Lui.
\par 21 Căci zicea în gândul ei: Numai să mă ating de haina Lui și mă voi face sănătoasă;
\par 22 Iar Iisus, întorcându-Se și văzând-o, i-a zis: Îndrăznește, fiică, credința ta te-a mântuit. Și s-a tămăduit femeia din ceasul acela.
\par 23 Iisus, venind la casa dregătorului și văzând pe cântăreții din flaut și mulțimea tulburată, a zis:
\par 24 Depărtați-vă, căci copila n-a murit, ci doarme. Dar ei râdeau de El.
\par 25 Iar după ce mulțimea a fost scoasă afară, intrând, a luat-o de mână, și copila s-a sculat.
\par 26 Și a ieșit vestea aceasta în tot ținutul acela.
\par 27 Plecând Iisus de acolo, doi orbi se țineau după El strigând și zicând: Miluiește-ne pe noi, Fiule al lui David.
\par 28 După ce a intrat în casă, au venit la El orbii și Iisus i-a întrebat: Credeți că pot să fac Eu aceasta? Zis-au Lui: Da, Doamne!
\par 29 Atunci S-a atins de ochii lor, zicând: După credința voastră, fie vouă!
\par 30 Și s-au deschis ochii lor. Iar Iisus le-a poruncit cu asprime, zicând: Vedeți, nimeni să nu știe.
\par 31 Iar ei, ieșind, L-au vestit în tot ținutul acela.
\par 32 Și plecând ei, iată au adus la El un om mut, având demon.
\par 33 Și fiind scos demonul, mutul a grăit. Iar mulțimile se minunau zicând: Niciodată nu s-a arătat așa în Israel.
\par 34 Dar fariseii ziceau: Cu domnul demonilor scoate pe demoni.
\par 35 Și Iisus străbătea toate cetățile și satele, învățând în sinagogile lor, propovăduind Evanghelia împărăției și vindecând toată boala și toată neputința în popor.
\par 36 Și văzând mulțimile, I s-a făcut milă de ele că erau necăjite și rătăcite ca niște oi care n-au păstor.
\par 37 Atunci a zis ucenicilor Lui: Secerișul e mult, dar lucrătorii sunt puțini.
\par 38 Rugați, deci, pe Domnul secerișului, ca să scoată lurcători la secerișul Său.

\chapter{10}

\par 1 Chemând la Sine pe cei doisprezece ucenici ai Săi, le-a dat lor putere asupra duhurilor celor necurate, ca să le scoată și să tămăduiască orice boală și orice neputință.
\par 2 Numele celor doisprezece apostoli sunt acestea: Întâi Simon, cel numit Petru, și Andrei, fratele lui; Iacov al lui Zevedeu și Ioan fratele lui;
\par 3 Filip și Vartolomeu, Toma și Matei vameșul, Iacov al lui Alfeu și Levi ce se zice Tadeu;
\par 4 Simon Cananeul și Iuda Iscarioteanul, cel care L-a vândut.
\par 5 Pe acești doisprezece i-a trimis Iisus, poruncindu-le lor și zicând: În calea păgânilor să nu mergeți, și în vreo cetate de samarineni să nu intrați;
\par 6 Ci mai degrabă mergeți către oile cele pierdute ale casei lui Israel.
\par 7 Și mergând, propovăduiți, zicând: S-a apropiat împărăția cerurilor.
\par 8 Tămăduiți pe cei neputincioși, înviați pe cei morți, curățiți pe cei leproși, pe demoni scoateți-i; în dar ați luat, în dar să dați.
\par 9 Să nu aveți nici aur, nici arginți, nici bani în cingătorile voastre;
\par 10 Nici traistă pe drum, nici două haine, nici încălțăminte, nici toiag; că vrednic este lucrătorul de hrana sa.
\par 11 În orice cetate sau sat veți intra, cercetați cine este în el vrednic și acolo rămâneți până ce veți ieși.
\par 12 Și intrând în casă, urați-i, zicând: "Pace casei acesteia".
\par 13 Și dacă este casa aceea vrednică, vină pacea voastră peste ea. Iar de nu este vrednică, pacea voastră întoarcă-se la voi.
\par 14 Cine nu vă va primi pe voi, nici nu va asculta cuvintele voastre, ieșind din casa sau din cetatea aceea, scuturați praful de pe picioarele voastre.
\par 15 Adevărat grăiesc vouă, mai ușor va fi pământului Sodomei și Gomorei, în ziua judecății, decât cetății aceleia.
\par 16 Iată Eu vă trimit pe voi ca pe niște oi în mijlocul lupilor; fiți dar înțelepți ca șerpii și nevinovați ca porumbeii.
\par 17 Feriți-vă de oameni, căci vă vor da pe mâna sinedriștilor și în sinagogile lor vă vor bate cu biciul.
\par 18 La dregători și la regi veți fi duși pentru Mine, spre mărturie lor și păgânilor.
\par 19 Iar când vă vor da pe voi în mâna lor, nu vă îngrijiți cum sau ce veți vorbi, căci se va da vouă în ceasul acela ce să vorbiți;
\par 20 Fiindcă nu voi sunteți care vorbiți, ci Duhul Tatălui vostru este care grăiește întru voi.
\par 21 Va da frate pe frate la moarte și tată pe fiu și se vor scula copiii împotriva părinților și-i vor ucide.
\par 22 Și veți fi urâți de toți pentru numele Meu; iar cel ce va răbda până în sfârșit, acela se va mântui.
\par 23 Când vă urmăresc pe voi în cetatea aceasta, fugiți în cealaltă; adevărat grăiesc vouă: nu veți sfârși cetățile lui Israel, până ce va veni Fiul Omului.
\par 24 Nu este ucenic mai presus de învățătorul său, nici slugă mai presus de stăpânul său.
\par 25 Destul este ucenicului să fie ca învățătorul și slugii ca stăpânul. Dacă pe stăpânul casei l-au numit Beelzebul, cu cât mai mult pe casnicii lui?
\par 26 Deci nu vă temeți de ei, căci nimic nu este acoperit care să nu iasă la iveală și nimic ascuns care să nu ajungă cunoscut.
\par 27 Ceea ce vă grăiesc la întuneric, spuneți la lumină și ceea ce auziți la ureche, propovăduiți de pe case.
\par 28 Nu vă temeți de cei ce ucid trupul, iar sufletul nu pot să-l ucidă; temeți-vă mai curând de acela care poate și sufletul și trupul să le piardă în gheena.
\par 29 Au nu se vând două vrăbii pe un ban? Și nici una din ele nu va cădea pe pământ fără știrea Tatălui vostru.
\par 30 La voi însă și perii capului, toți sunt numărați.
\par 31 Așadar nu vă temeți; voi sunteți cu mult mai de preț decât păsările.
\par 32 Oricine va mărturisi pentru Mine înaintea oamenilor, mărturisi-voi și Eu pentru el înaintea Tatălui Meu, Care este în ceruri.
\par 33 Iar de cel ce se va lepăda de Mine înaintea oamenilor și Eu Mă voi lepăda de el înaintea Tatălui Meu, Care este în ceruri.
\par 34 Nu socotiți că am venit să aduc pace pe pământ; n-am venit să aduc pace, ci sabie.
\par 35 Căci am venit să despart pe fiu de tatăl său, pe fiică de mama sa, pe noră de soacra sa.
\par 36 Și dușmanii omului (vor fi) casnicii lui.
\par 37 Cel ce iubește pe tată ori pe mamă mai mult decât pe Mine nu este vrednic de Mine; cel ce iubește pe fiu ori pe fiică mai mult decât pe Mine nu este vrednic de Mine.
\par 38 Și cel ce nu-și ia crucea și nu-Mi urmează Mie nu este vrednic de Mine.
\par 39 Cine ține la sufletul lui îl va pierde, iar cine-și pierde sufletul lui pentru Mine îl va găsi.
\par 40 Cine vă primește pe voi pe Mine Mă primește, și cine Mă primește pe Mine primește pe Cel ce M-a trimis pe Mine.
\par 41 Cine primește prooroc în nume de prooroc plată de prooroc va lua, și cine primește pe un drept în nume de drept răsplata dreptului va lua.
\par 42 Și cel ce va da de băut unuia dintre aceștia mici numai un pahar cu apă rece, în nume de ucenic, adevărat grăiesc vouă: nu va pierde plata sa.

\chapter{11}

\par 1 Sfârșind Iisus de dat aceste învățături celor doisprezece ucenici ai Săi, a trecut de acolo ca să învețe și să propovăduiască mai departe prin cetățile lor.
\par 2 Și auzind Ioan, în închisoare, despre faptele lui Hristos, și trimițând pe doi dintre ucenicii săi, au zis Lui:
\par 3 Tu ești Cel ce vine, sau să așteptăm pe altul?
\par 4 Și Iisus, răspunzând, le-a zis: Mergeți și spuneți lui Ioan cele ce auziți și vedeți:
\par 5 Orbii își capătă vederea și șchiopii umblă, leproșii se curățesc și surzii aud, morții înviază și săracilor li se binevestește.
\par 6 Și fericit este acela care nu se va sminti întru Mine.
\par 7 După plecarea acestora, Iisus a început să vorbească mulțimilor despre Ioan: Ce-ați ieșit să vedeți în pustie? Au trestie clătinată de vânt?
\par 8 Dar de ce ați ieșit? Să vedeți un om îmbrăcat în haine moi? Iată, cei ce poartă haine moi sunt în casele regilor.
\par 9 Atunci de ce-ați ieșit? Să vedeți un prooroc? Da, zic vouă, și mai mult decât un prooroc.
\par 10 Că el este acela despre care s-a scris: "Iată Eu trimit, înaintea feței Tale, pe îngerul Meu, care va pregăti calea Ta, înaintea Ta".
\par 11 Adevărat zic vouă: Nu s-a ridicat între cei născuți din femei unul mai mare decât Ioan Botezătorul; totuși cel mai mic în împărăția cerurilor este mai mare decât el.
\par 12 Din zilele lui Ioan Botezătorul până acum împărăția cerurilor se ia prin străduință și cei ce se silesc pun mâna pe ea.
\par 13 Toți proorocii și Legea au proorocit până la Ioan.
\par 14 Și dacă voiți să înțelegeți, el este Ilie, cel ce va să vină.
\par 15 Cine are urechi de auzit să audă.
\par 16 Dar cu cine voi asemăna neamul acesta? Este asemenea copiilor care șed în piețe și strigă către alții,
\par 17 Zicând: V-am cântat din fluier și n-ați jucat; v-am cântat de jale și nu v-ați tânguit.
\par 18 Căci a venit Ioan, nici mâncând, nici bând, și spun: Are demon.
\par 19 A venit Fiul Omului, mâncând și bând și spun: Iată om mâncăcios și băutor de vin, prieten al vameșilor și al păcătoșilor. Dar înțelepciunea s-a dovedit dreaptă din faptele ei.
\par 20 Atunci a început Iisus să mustre cetățile în care se făcuseră cele mai multe minuni ale Sale, căci nu s-au pocăit.
\par 21 Vai ție, Horazine, vai ție, Betsaida, că dacă în Tir și în Sidon s-ar fi făcut minunile ce s-au făcut în voi, de mult, în sac și în cenușă, s-ar fi pocăit.
\par 22 Dar zic vouă: Tirului și Sidonului le va fi mai ușor în ziua judecății, decât vouă.
\par 23 Și tu, Capernaume: N-ai fost înălțat până la cer? Până la iad te vei coborî. Căci de s-ar fi făcut în Sodoma minunile ce s-au făcut în tine, ar fi rămas până astăzi.
\par 24 Dar zic vouă că pământului Sodomei îi va fi mai ușor în ziua judecății decât ție.
\par 25 În vremea aceea, răspunzând, Iisus a zis: Te slăvesc pe Tine, Părinte, Doamne al cerului și al pământului, căci ai ascuns acestea de cei înțelepți și pricepuți și le-ai descoperit pruncilor.
\par 26 Da, Părinte, căci așa a fost bunăvoirea înaintea Ta.
\par 27 Toate Mi-au fost date de către Tatăl Meu și nimeni nu cunoaște pe Fiul, decât numai Tatăl, nici pe Tatăl nu-L cunoaște nimeni, decât numai Fiul și cel căruia va voi Fiul să-i descopere.
\par 28 Veniți la Mine toți cei osteniți și împovărați și Eu vă voi odihni pe voi.
\par 29 Luați jugul Meu asupra voastră și învățați-vă de la Mine, că sunt blând și smerit cu inima și veți găsi odihnă sufletelor voastre.
\par 30 Căci jugul Meu e bun și povara Mea este ușoară.

\chapter{12}

\par 1 În vremea aceea, mergea Iisus, într-o zi de sâmbătă, printre semănături, iar ucenicii Lui au flămânzit și au început să smulgă spice și să mănânce.
\par 2 Văzând aceasta, fariseii au zis Lui: Iată, ucenicii Tăi fac ceea ce nu se cuvine să facă sâmbăta.
\par 3 Iar El le-a zis: Au n-ați citit ce-a făcut David când a flămânzit, el și cei ce erau cu el?
\par 4 Cum a intrat în casa Domnului și a mâncat pâinile punerii înainte, care nu se cuveneau lui să le mănânce, nici celor ce erau cu el, ci numai preoților?
\par 5 Sau n-ați citit în Lege că preoții, sâmbăta, în templu, calcă sâmbăta și sunt fără de vină?
\par 6 Ci grăiesc vouă că mai mare decât templul este aici.
\par 7 Dacă știați ce înseamnă: Milă voiesc iar nu jertfă, n-ați fi osândit pe cei nevinovați.
\par 8 Că Domn este și al sâmbetei Fiul Omului.
\par 9 Și trecând de acolo, a venit în sinagoga lor.
\par 10 Și iată un om având mâna uscată. Și L-au întrebat, zicând: Cade-se, oare, a vindeca sâmbăta? Ca să-L învinuiască.
\par 11 El le-a zis: Cine va fi între voi omul care va avea o oaie și, de va cădea ea sâmbăta în groapă, nu o va apuca și o va scoate?
\par 12 Cu cât se deosebește omul de oaie! De aceea se cade a face bine sâmbăta.
\par 13 Atunci i-a zis omului: Întinde mâna ta. El a întins-o și s-a făcut sănătoasă ca și cealaltă.
\par 14 Și ieșind, fariseii s-au sfătuit împotriva Lui cum să-L piardă.
\par 15 Iisus însă, cunoscându-i, S-a dus de acolo. Și mulți au venit după El și i-a vindecat pe toți.
\par 16 Dar le-a poruncit ca să nu-L dea în vileag,
\par 17 Ca să se împlinească ceea ce s-a spus prin Isaia proorocul, care zice:
\par 18 "Iată Fiul Meu pe Care L-am ales, iubitul Meu întru Care a binevoit sufletul Meu; pune-voi Duhul Meu peste El și judecată neamurilor va vesti.
\par 19 Nu se va certa, nici nu va striga, nu va auzi nimeni, pe ulițe, glasul Lui.
\par 20 Trestie strivită nu va frânge și feștilă fumegândă nu va stinge, până ce nu va scoate, spre biruință, judecata.
\par 21 Și în numele Lui vor nădăjdui neamurile."
\par 22 Atunci au adus la El pe un demonizat, orb și mut, și l-a vindecat, încât cel orb și mut vorbea și vedea.
\par 23 Mulțimile toate se mirau zicând: Nu este, oare, Acesta, Fiul lui David?
\par 24 Fariseii însă, auzind, ziceau: Acesta nu scoate pe demoni decât cu Beelzebul, căpetenia demonilor.
\par 25 Cunoscând gândurile lor, Iisus le-a zis: Orice împărăție care se dezbină în sine se pustiește, orice cetate sau casă care se dezbină în sine nu va dăinui.
\par 26 Dacă satana scoate pe satana, s-a dezbinat în sine; dar atunci cum va dăinui împărăția lui?
\par 27 Și dacă Eu scot pe demoni cu Beelzebul, feciorii voștri cu cine îi scot? De aceea ei vă vor fi judecători.
\par 28 Iar dacă Eu cu Duhul lui Dumnezeu scot pe demoni, iată a ajuns la voi împărăția lui Dumnezeu.
\par 29 Cum poate cineva să intre în casa celui tare și să-i jefuiască lucrurile, dacă nu va lega întâi pe cel tare și pe urmă să-i prade casa?
\par 30 Cine nu este cu Mine este împotriva Mea și cine nu adună cu Mine risipește.
\par 31 De aceea vă zic: Orice păcat și orice hulă se va ierta oamenilor, dar hula împotriva Duhului nu se va ierta.
\par 32 Celui care va zice cuvânt împotriva Fiului Omului, se va ierta lui; dar celui care va zice împotriva Duhului Sfânt, nu i se va ierta lui, nici în veacul acesta, nici în cel ce va să fie.
\par 33 Ori spuneți că pomul este bun și rodul lui e bun, ori spuneți că pomul e rău și rodul lui e rău, căci după roadă se cunoaște pomul.
\par 34 Pui de vipere, cum puteți să grăiți cele bune, odată ce sunteți răi? Căci din prisosul inimii grăiește gura.
\par 35 Omul cel bun din comoara lui cea bună scoate afară cele bune, pe când omul cel rău, din comoara lui cea rea scoate afară cele rele.
\par 36 Vă spun că pentru orice cuvânt deșert, pe care-l vor rosti, oamenii vor da socoteală în ziua judecății.
\par 37 Căci din cuvintele tale vei fi găsit drept, și din cuvintele tale vei fi osândit.
\par 38 Atunci I-au răspuns unii dintre cărturari și farisei, zicând: Învățătorule, voim să vedem de la Tine un semn.
\par 39 Iar El, răspunzând, le-a zis: Neam viclean și desfrânat cere semn, dar semn nu i se va da, decât semnul lui Iona proorocul.
\par 40 Că precum a fost Iona în pântecele chitului trei zile și trei nopți, așa va fi și Fiul Omului în inima pământului trei zile și trei nopți.
\par 41 Bărbații din Ninive se vor scula la judecată cu neamul acesta și-l vor osândi, că s-au pocăit la propovăduirea lui Iona; iată aici este mai mult decât Iona.
\par 42 Regina de la miazăzi se va scula la judecată cu neamul acesta și-l va osândi, căci a venit de la marginile pământului ca să asculte înțelepciunea lui Solomon, și iată aici este mai mult decât Solomon.
\par 43 Și când duhul necurat a ieșit din om, umblă prin locuri fără apă, căutând odihnă și nu găsește.
\par 44 Atunci zice: Mă voi întoarce la casa mea de unde am ieșit; și venind, o află golită, măturată și împodobită.
\par 45 Atunci se duce și ia cu sine alte șapte duhuri mai rele decât el și, intrând, sălășluiesc aici și se fac cele de pe urmă ale omului aceluia mai rele decât cele dintâi. Așa va fi și cu acest neam viclean.
\par 46 Și încă vorbind El mulțimilor, iată mama și frații Lui stăteau afară, căutând să vorbească cu El.
\par 47 Cineva I-a zis: Iată mama Ta și frații Tăi stau afară, căutând să-Ți vorbească.
\par 48 Iar El i-a zis: Cine este mama Mea și cine sunt frații Mei?
\par 49 Și, întinzând mâna către ucenicii Săi, a zis: Iată mama Mea și frații Mei.
\par 50 Că oricine va face voia Tatălui Meu Celui din ceruri, acela îmi este frate și soră și mamă.

\chapter{13}

\par 1 În ziua aceea, ieșind Iisus din casă, ședea lângă mare.
\par 2 Și s-au adunat la El mulțimi multe, încât intrând în corabie ședea în ea și toată mulțimea sta pe țărm.
\par 3 Și le-a grăit lor multe, în pilde, zicând: Iată a ieșit semănătorul să semene.
\par 4 Și pe când semăna, unele semințe au căzut lângă drum și au venit păsările și le-au mâncat.
\par 5 Altele au căzut pe loc pietros, unde n-aveau pământ mult și îndată au răsărit, că n-aveau pământ adânc;
\par 6 Iar când s-a ivit soarele, s-au pălit de arșiță și, neavând rădăcină, s-au uscat.
\par 7 Altele au căzut între spini, dar spinii au crescut și le-au înăbușit.
\par 8 Altele au căzut pe pământ bun și au dat rod: una o sută, alta șaizeci, alta treizeci.
\par 9 Cine are urechi de auzit să audă.
\par 10 Și ucenicii, apropiindu-se de El, I-au zis: De ce le vorbești lor în pilde?
\par 11 Iar El, răspunzând, le-a zis: Pentru că vouă vi s-a dat să cunoașteți tainele împărăției cerurilor, pe când acestora nu li s-a dat.
\par 12 Căci celui ce are i se va da și-i va prisosi, iar de la cel ce nu are, și ce are i se va lua.
\par 13 De aceea le vorbesc în pilde, că, văzând, nu văd și, auzind, nu aud, nici nu înțeleg.
\par 14 Și se împlinește cu ei proorocia lui Isaia, care zice: "Cu urechile veți auzi, dar nu veți înțelege, și cu ochii vă veți uita, dar nu veți vedea".
\par 15 Căci inima acestui popor s-a învârtoșat și cu urechile aude greu și ochii lui s-au închis, ca nu cumva să vadă cu ochii și să audă cu urechile și cu inima să înțeleagă și să se întoarcă, și Eu să-i tămăduiesc pe ei.
\par 16 Dar fericiți sunt ochii voștri că văd și urechile voastre că aud.
\par 17 Căci adevărat grăiesc vouă că mulți prooroci și drepți au dorit să vadă cele ce priviți voi, și n-au văzut, și să audă cele ce auziți voi, și n-au auzit.
\par 18 Voi, deci, ascultați pilda semănătorului:
\par 19 De la oricine aude cuvântul împărăției și nu-l înțelege, vine cel viclean și răpește ce s-a semănat în inima lui; aceasta este sămânța semănată lângă drum.
\par 20 Cea semănată pe loc pietros este cel care aude cuvântul și îndată îl primește cu bucurie,
\par 21 Dar nu are rădăcină în sine, ci ține până la o vreme și, întâmplându-se strâmtorare sau prigoană pentru cuvânt, îndată se smintește.
\par 22 Cea semănată în spini este cel care aude cuvântul, dar grija acestei lumi și înșelăciunea avuției înăbușă cuvântul și îl face neroditor.
\par 23 Iar sămânța semănată în pământ bun este cel care aude cuvântul și-l înțelege, deci care aduce rod și face: unul o sută, altul șaizeci, altul treizeci.
\par 24 Altă pildă le-a pus lor înainte, zicând: Asemenea este împărăția cerurilor omului care a semănat sămânță bună în țarina sa.
\par 25 Dar pe când oamenii dormeau, a venit vrăjmașul lui, a semănat neghină printre grâu și s-a dus.
\par 26 Iar dacă a crescut paiul și a făcut rod, atunci s-a arătat și neghina.
\par 27 Venind slugile stăpânului casei, i-au zis: Doamne, n-ai semănat tu, oare, sămânță bună în țarina ta? De unde dar are neghină?
\par 28 Iar el le-a răspuns: Un om vrăjmaș a făcut aceasta. Slugile i-au zis: Voiești deci să ne ducem și s-o plivim?
\par 29 El însă a zis: Nu, ca nu cumva, plivind neghina, să smulgeți odată cu ea și grâul.
\par 30 Lăsați să crească împreună și grâul și neghina, până la seceriș, și la vremea secerișului voi zice secerătorilor: Pliviți întâi neghina și legați-o în snopi ca s-o ardem, iar grâul adunați-l în jitnița mea.
\par 31 O altă pildă le-a pus înainte, zicând: Împărăția cerurilor este asemenea grăuntelui de muștar, pe care, luându-l, omul l-a semănat în țarina sa,
\par 32 Și care este mai mic decât toate semințele, dar când a crescut este mai mare decât toate legumele și se face pom, încât vin păsările cerului și se sălășluiesc în ramurile lui.
\par 33 Altă pildă le-a spus lor: Asemenea este împărăția cerurilor aluatului pe care, luându-l, o femeie l-a ascuns în trei măsuri de făină, până ce s-a dospit toată.
\par 34 Toate acestea le-a vorbit Iisus mulțimilor în pilde, și fără pildă nu le grăia nimic,
\par 35 Ca să se împlinească ce s-a spus prin proorocul care zice: "Deschide-voi în pilde gura Mea, spune-voi cele ascunse de la întemeierea lumii".
\par 36 După aceea, lăsând mulțimile, a venit în casă, iar ucenicii Lui s-au apropiat de El, zicând: Lămurește-ne nouă pilda cu neghina din țarină.
\par 37 El, răspunzând, le-a zis: Cel ce seamănă sămânța cea bună este Fiul Omului.
\par 38 Țarina este lumea; sămânța cea bună sunt fiii împărăției; iar neghina sunt fiii celui rău.
\par 39 Dușmanul care a semănat-o este diavolul; secerișul este sfârșitul lumii, iar secerătorii sunt îngerii.
\par 40 Și, după cum se alege neghina și se arde în foc, așa va fi la sfârșitul veacului.
\par 41 Trimite-va Fiul Omului pe îngerii Săi, vor culege din împărăția Lui toate smintelile și pe cei ce fac fărădelegea,
\par 42 Și-i vor arunca pe ei în cuptorul cu foc; acolo va fi plângerea și scrâșnirea dinților.
\par 43 Atunci cei drepți vor străluci ca soarele în împărăția Tatălui lor. Cel ce are urechi de auzit să audă.
\par 44 Asemenea este împărăția cerurilor cu o comoară ascunsă în țarină, pe care, găsind-o un om, a ascuns-o, și de bucuria ei se duce și vinde tot ce are și cumpără țarina aceea.
\par 45 Iarăși asemenea este împărăția cerurilor cu un neguțător care caută mărgăritare bune.
\par 46 Și aflând un mărgăritar de mult preț, s-a dus, a vândut toate câte avea și l-a cumpărat.
\par 47 Asemenea este iarăși împărăția cerurilor cu un năvod aruncat în mare și care adună tot felul de pești.
\par 48 Iar când s-a umplut, l-au tras pescarii la mal și, șezând, au ales în vase pe cei buni, iar pe cei răi i-au aruncat afară.
\par 49 Așa va fi la sfârșitul veacului: vor ieși îngerii și vor despărți pe cei răi din mijlocul celor drepți.
\par 50 Și îi vor arunca în cuptorul cel de foc; acolo va fi plângerea și scrâșnirea dinților.
\par 51 Înțeles-ați toate acestea? Zis-au Lui: Da, Doamne.
\par 52 Iar El le-a zis: De aceea, orice cărturar cu învățătură despre împărăția cerurilor este asemenea unui om gospodar, care scoate din vistieria sa noi și vechi.
\par 53 Iar după ce Iisus a sfârșit aceste pilde, a trecut de acolo.
\par 54 Și venind în patria Sa, îi învăța pe ei în sinagoga lor, încât ei erau uimiți și ziceau: De unde are El înțelepciunea aceasta și puterile?
\par 55 Au nu este Acesta fiul teslarului? Au nu se numește mama Lui Maria și frații (verii) Lui: Iacov și Iosif și Simon și Iuda?
\par 56 Și surorile (verișoarele) Lui au nu sunt toate la noi? Deci, de unde are El toate acestea?
\par 57 Și se sminteau întru El. Iar Iisus le-a zis: Nu este prooroc disprețuit decât în patria lui și în casa lui.
\par 58 Și n-a făcut acolo multe minuni, din pricina necredinței lor.

\chapter{14}

\par 1 În vremea aceea, a auzit tetrarhul Irod de vestea ce se dusese despre Iisus.
\par 2 Și a zis slujitorilor săi: Acesta este Ioan Botezătorul; el s-a sculat din morți și de aceea se fac minuni prin el.
\par 3 Căci Irod, prinzând pe Ioan, l-a legat și l-a pus în temniță, pentru Irodiada, femeia lui Filip, fratele său.
\par 4 Căci Ioan îi zicea lui: Nu ți se cuvine s-o ai de soție.
\par 5 Și voind să-l ucidă, s-a temut de mulțime, că-l socotea pe el ca prooroc.
\par 6 Iar prăznuind Irod ziua lui de naștere, fiica Irodiadei a jucat în mijloc și i-a plăcut lui Irod.
\par 7 De aceea, cu jurământ i-a făgăduit să-i dea orice va cere.
\par 8 Iar ea, îndemnată fiind de mama sa, a zis: Dă-mi, aici pe tipsie, capul lui Ioan Botezătorul.
\par 9 Și regele s-a întristat, dar, pentru jurământ și pentru cei care ședeau cu el la masă, a poruncit să i se dea.
\par 10 Și a trimis și a tăiat capul lui Ioan, în temniță.
\par 11 Și capul lui a fost adus pe tipsie și a fost dat fetei, iar ea l-a dus mamei sale.
\par 12 Și, venind ucenicii lui, au luat trupul lui și l-au înmormântat și s-au dus să dea de știre lui Iisus.
\par 13 Iar Iisus, auzind, S-a dus de acolo singur, cu corabia, în loc pustiu dar, aflând, mulțimile au venit după El, pe jos, din cetăți.
\par 14 Și ieșind, a văzut mulțime mare și I S-a făcut milă de ei și a vindecat pe bolnavii lor.
\par 15 Iar când s-a făcut seară, ucenicii au venit la El și I-au zis: locul este pustiu și vremea iată a trecut; deci, dă drumul mulțimilor ca să se ducă în sate, să-și cumpere mâncare.
\par 16 Iisus însă le-a răspuns: N-au trebuință să se ducă; dați-le voi să mănânce.
\par 17 Iar ei I-au zis: Nu avem aici decât cinci pâini și doi pești.
\par 18 Și El a zis: Aduceți-Mi-le aici.
\par 19 Și poruncind să se așeze mulțimile pe iarbă și luând cele cinci pâini și cei doi pești și privind la cer, a binecuvântat și, frângând, a dat ucenicilor pâinile, iar ucenicii mulțimilor.
\par 20 Și au mâncat toți și s-au săturat și au strâns rămășițele de fărâmituri, douăsprezece coșuri pline.
\par 21 Iar cei ce mâncaseră erau ca la cinci mii de bărbați, afară de femei și de copii.
\par 22 Și îndată Iisus a silit pe ucenici să intre în corabie și să treacă înaintea Lui, pe țărmul celălalt, până ce El va da drumul mulțimilor.
\par 23 Iar dând drumul mulțimilor, S-a suit în munte, ca să Se roage singur. Și, făcându-se seară, era singur acolo.
\par 24 Iar corabia era acum la multe stadii departe de pământ, fiind învăluită de valuri, căci vântul era împotrivă.
\par 25 Iar la a patra strajă din noapte, a venit la ei Iisus, umblând pe mare.
\par 26 Văzându-L umblând pe mare, ucenicii s-au înspăimântat, zicând că e nălucă și de frică au strigat.
\par 27 Dar El le-a vorbit îndată, zicându-le: Îndrăzniți, Eu sunt; nu vă temeți!
\par 28 Iar Petru, răspunzând, a zis: Doamne, dacă ești Tu, poruncește să vin la Tine pe apă.
\par 29 El i-a zis: Vino. Iar Petru, coborându-se din corabie, a mers pe apă și a venit către Iisus.
\par 30 Dar văzând vântul, s-a temut și, începând să se scufunde, a strigat, zicând: Doamne, scapă-mă!
\par 31 Iar Iisus, întinzând îndată mâna, l-a apucat și a zis: Puțin credinciosule, pentru ce te-ai îndoit?
\par 32 Și suindu-se ei în corabie, s-a potolit vântul.
\par 33 Iar cei din corabie I s-au închinat, zicând: Cu adevărat Tu ești Fiul lui Dumnezeu.
\par 34 Și, trecând dincolo, au venit în pământul Ghenizaretului.
\par 35 Și, cunoscându-L, oamenii locului aceluia au trimis în tot acel ținut și au adus la El pe toți bolnavii.
\par 36 Și-L rugau ca numai să se atingă de poala hainei Lui; și câți se atingeau se vindecau.

\chapter{15}

\par 1 Atunci au venit din Ierusalim, la Iisus, fariseii și cărturarii, zicând:
\par 2 Pentru ce ucenicii Tăi calcă datina bătrânilor? Căci nu-și spală mâinile când mănâncă pâine.
\par 3 Iar El, răspunzând, le-a zis: De ce și voi călcați porunca lui Dumnezeu pentru datina voastră?
\par 4 Căci Dumnezeu a zis: Cinstește pe tatăl tău și pe mama ta, iar cine va blestema pe tată sau pe mamă, cu moarte să se sfârșească.
\par 5 Voi însă spuneți: Cel care va zice tatălui său sau mamei sale: Cu ce te-aș fi putut ajuta este dăruit lui Dumnezeu,
\par 6 Acela nu va cinsti pe tatăl său sau pe mama sa; și ați desființat cuvântul lui Dumnezeu pentru datina voastră.
\par 7 Fățarnicilor, bine a proorocit despre voi Isaia, când a zis:
\par 8 "Poporul acesta Mă cinstește cu buzele, dar inima lor este departe de Mine.
\par 9 Și zadarnic Mă cinstesc ei, învățând învățături ce sunt porunci ale oamenilor".
\par 10 Și chemând la Sine mulțimile, le-a zis: Ascultați și înțelegeți:
\par 11 Nu ceea ce intră pe gură spurcă pe om, ci ceea ce iese din gură, aceea spurcă pe om.
\par 12 Atunci, apropiindu-se, ucenicii I-au zis: Știi că fariseii, auzind cuvântul, s-au scandalizat?
\par 13 Iar El, răspunzând, a zis: Orice răsad pe care nu l-a sădit Tatăl Meu cel ceresc, va fi smuls din rădăcină.
\par 14 Lăsații pe ei; sunt călăuze oarbe, orbilor; și dacă orb pe orb va călăuzi, amândoi vor cădea în groapă.
\par 15 Și Petru, răspunzând, I-a zis: Lămurește-ne nouă pilda aceasta.
\par 16 El a zis: Acum și voi sunteți nepricepuți?
\par 17 Nu înțelegeți că tot ce intră în gură se duce în pântece și se aruncă afară?
\par 18 Iar cele ce ies din gură pornesc din inimă și acelea spurcă pe om.
\par 19 Căci din inimă ies: gânduri rele, ucideri, adultere, desfrânări, furtișaguri, mărturii mincinoase, hule.
\par 20 Acestea sunt care spurcă pe om, dar a mânca cu mâini nespălate nu spurcă pe om.
\par 21 Și ieșind de acolo, a plecat Iisus în părțile Tirului și ale Sidonului.
\par 22 Și iată o femeie cananeiancă, din acele ținuturi, ieșind striga, zicând: Miluiește-mă, Doamne, Fiul lui David! Fiica mea este rău chinuită de demon.
\par 23 El însă nu i-a răspuns nici un cuvânt; și apropiindu-se, ucenicii Lui Îl rugau, zicând: Slobozește-o, că strigă în urma noastră.
\par 24 Iar El, răspunzând, a zis: Nu sunt trimis decât către oile cele pierdute ale casei lui Israel.
\par 25 Iar ea, venind, s-a închinat Lui, zicând: Doamne, ajută-mă.
\par 26 El însă, răspunzând, i-a zis: Nu este bine să iei pâinea copiilor și s-o arunci câinilor.
\par 27 Dar ea a zis: Da, Doamne, dar și câinii mănâncă din fărâmiturile care cad de la masa stăpânilor lor.
\par 28 Atunci, răspunzând, Iisus i-a zis: O, femeie, mare este credința ta; fie ție după cum voiești. Și s-a tămăduit fiica ei în ceasul acela.
\par 29 Și trecând Iisus de acolo, a venit lângă Marea Galileii și, suindu-Se în munte, a șezut acolo.
\par 30 Și mulțimi multe au venit la El, având cu ei șchiopi, orbi, muți, ciungi, și mulți alții și i-au pus la picioarele Lui, iar El i-a vindecat.
\par 31 Încât mulțimea se minuna văzând pe muți vorbind, pe ciungi sănătoși, pe șchiopi umblând și pe orbi văzând, și slăveau pe Dumnezeul lui Israel.
\par 32 Iar Iisus, chemând la Sine pe ucenicii Săi, a zis: Milă îmi este de mulțime, că iată sunt trei zile de când așteaptă lângă Mine și n-au ce să mănânce; și să-i slobozesc flămânzi nu voiesc, ca să nu se istovească pe drum.
\par 33 Și ucenicii I-au zis: De unde să avem noi, în pustie, atâtea pâini, cât să săturăm atâta mulțime?
\par 34 Și Iisus i-a întrebat: Câte pâini aveți? Ei au răspuns: Șapte și puțini peștișori.
\par 35 Și poruncind mulțimii să șadă jos pe pământ,
\par 36 A luat cele șapte pâini și pești și, mulțumind, a frânt și a dat ucenicilor, iar ucenicii mulțimilor.
\par 37 Și au mâncat toți și s-au săturat și au luat șapte coșuri pline, cu rămășițe de fărâmituri.
\par 38 Iar cei ce au mâncat erau ca la patru mii de bărbați, afară de copii și de femei.
\par 39 După aceea a dat drumul mulțimilor, S-a suit în corabie și S-a dus în ținutul Magdala.

\chapter{16}

\par 1 Și apropiindu-se fariseii și saducheii și ispitindu-L, I-au cerut să le arate semn din cer.
\par 2 Iar El, răspunzând, le-a zis: Când se face seară, ziceți: Mâine va fi timp frumos, pentru că e cerul roșu.
\par 3 Iar dimineața ziceți: Astăzi va fi furtună, pentru că cerul este roșu-posomorât. Fățarnicilor, fața cerului știți s-o judecați, dar semnele vremilor nu puteți!
\par 4 Neam viclean și adulter cere semn și semn nu se va da lui, decât numai semnul lui Iona. Și lăsându-i, a plecat.
\par 5 Și venind ucenicii pe celălalt țărm, au uitat să ia pâini.
\par 6 Iar Iisus le-a zis: Luați aminte și feriți-vă de aluatul fariseilor și al saducheilor.
\par 7 Iar ei cugetau în sinea lor, zicând: Aceasta, pentru că n-am luat pâine.
\par 8 Dar Iisus, cunoscându-le gândul, a zis: Ce cugetați în voi înșivă, puțin credincioșilor, că n-ați luat pâine?
\par 9 Tot nu înțelegeți, nici nu vă aduceți aminte de cele cinci pâini, la cei cinci mii de oameni, și câte coșuri ați luat?
\par 10 Nici de cele șapte pâini, la cei patru mii de oameni, și câte coșuri ați luat?
\par 11 Cum nu înțelegeți că nu despre pâini v-am zis? Ci feriți-vă de aluatul fariseilor și al saducheilor.
\par 12 Atunci au înțeles că nu le-a spus să se ferească de aluatul pâinii, ci de învățătura fariseilor și a saducheilor.
\par 13 Și venind Iisus în părțile Cezareii lui Filip, îi întreba pe ucenicii Săi, zicând: Cine zic oamenii că sunt Eu, Fiul Omului?
\par 14 Iar ei au răspuns: Unii, Ioan Botezătorul, alții Ilie, alții Ieremia sau unul dintre prooroci.
\par 15 Și le-a zis: Dar voi cine ziceți că sunt?
\par 16 Răspunzând Simon Petru a zis: Tu ești Hristosul, Fiul lui Dumnezeu Celui viu.
\par 17 Iar Iisus, răspunzând, i-a zis: Fericit ești Simone, fiul lui Iona, că nu trup și sânge ți-au descoperit ție aceasta, ci Tatăl Meu, Cel din ceruri.
\par 18 Și Eu îți zic ție, că tu ești Petru și pe această piatră voi zidi Biserica Mea și porțile iadului nu o vor birui.
\par 19 Și îți voi da cheile împărăției cerurilor și orice vei lega pe pământ va fi legat și în ceruri, și orice vei dezlega pe pământ va fi dezlegat și în ceruri.
\par 20 Atunci a poruncit ucenicilor Lui să nu spună nimănui că El este Hristosul.
\par 21 De atunci a început Iisus să le arate ucenicilor Lui că El trebuie să meargă la Ierusalim și să pătimească multe de la bătrâni și de la arhierei și de la cărturari și să fie ucis, și a treia zi să învieze.
\par 22 Și Petru, luându-L la o parte, a început să-L dojenească, zicându-I: Fie-Ți milă de Tine să nu Ți se întâmple Ție aceasta.
\par 23 Iar El, întorcându-se, a zis lui Petru: Mergi înapoia Mea, satano! Sminteală Îmi ești; că nu cugeți cele ale lui Dumnezeu, ci cele ale oamenilor.
\par 24 Atunci Iisus a zis ucenicilor Săi: Dacă vrea cineva să vină după Mine, să se lepede de sine, să-și ia crucea și să-Mi urmeze Mie.
\par 25 Că cine va voi să-și scape sufletul îl va pierde; iar cine își va pierde sufletul pentru Mine îl va afla.
\par 26 Pentru că ce-i va folosi omului, dacă va câștiga lumea întreagă, iar sufletul său îl va pierde? Sau ce va da omul în schimb pentru sufletul său?
\par 27 Căci Fiul Omului va să vină întru slava Tatălui Său, cu îngerii Săi; și atunci va răsplăti fiecăruia după faptele sale.
\par 28 Adevărat grăiesc vouă: Sunt unii din cei ce stau aici care nu vor gusta moartea până ce nu vor vedea pe Fiul Omului, venind în împărăția Sa.

\chapter{17}

\par 1 Și după șase zile, Iisus a luat cu Sine pe Petru și pe Iacov și pe Ioan, fratele lui, și i-a dus într-un munte înalt, de o parte.
\par 2 Și S-a schimbat la față, înaintea lor, și a strălucit fața Lui ca soarele, iar veșmintele Lui s-au făcut albe ca lumina.
\par 3 Și iată, Moise și Ilie s-au arătat lor, vorbind cu El.
\par 4 Și, răspunzând, Petru a zis lui Iisus: Doamne, bine este să fim noi aici; dacă voiești, voi face aici trei colibe: Ție una, și lui Moise una, și lui Ilie una.
\par 5 Vorbind el încă, iată un nor luminos i-a umbrit pe ei, și iată glas din nor zicând: "Acesta este Fiul Meu Cel iubit, în Care am binevoit; pe Acesta ascultați-L".
\par 6 Și, auzind, ucenicii au căzut cu fața la pământ și s-au spăimântat foarte.
\par 7 Și Iisus S-a apropiat de ei, și, atingându-i, le-a zis: Sculați-vă și nu vă temeți.
\par 8 Și, ridicându-și ochii, nu au văzut pe nimeni, decât numai pe Iisus singur.
\par 9 Și pe când se coborau din munte, Iisus le-a poruncit, zicând: Nimănui să nu spuneți ceea ce ați văzut, până când Fiul Omului Se va scula din morți.
\par 10 Și ucenicii L-au întrebat, zicând: Pentru ce dar zic cărturarii că trebuie să vină mai întâi Ilie?
\par 11 Iar El, răspunzând, a zis: Ilie într-adevăr va veni și va așeza la loc toate.
\par 12 Eu însă vă spun vouă că Ilie a și venit, dar ei nu l-au cunoscut, ci au făcut cu el câte au voit; așa și Fiul Omului va pătimi de la ei.
\par 13 Atunci au înțeles ucenicii că Iisus le-a vorbit despre Ioan Botezătorul.
\par 14 Și mergând ei spre mulțime, s-a apropiat de El un om, căzându-I în genunchi,
\par 15 Și zicând: Doamne, miluiește pe fiul meu că este lunatic și pătimește rău, căci adesea cade în foc și adesea în apă.
\par 16 Și l-am dus la ucenicii Tăi și n-au putut să-l vindece.
\par 17 Iar Iisus, răspunzând, a zis: O, neam necredincios și îndărătnic, până când voi fi cu voi? Până când vă voi suferi pe voi? Aduceți-l aici la Mine.
\par 18 Și Iisus l-a certat și demonul a ieșit din el și copilul s-a vindecat din ceasul acela.
\par 19 Atunci, apropiindu-se ucenicii de Iisus, I-au zis de o parte: De ce noi n-am putut să-l scoatem?
\par 20 Iar Iisus le-a răspuns: Pentru puțina voastră credință. Căci adevărat grăiesc vouă: Dacă veți avea credință în voi cât un grăunte de muștar, veți zice muntelui acestuia: Mută-te de aici dincolo, și se va muta; și nimic nu va fi vouă cu neputință.
\par 21 Dar acest neam de demoni nu iese decât numai cu rugăciune și cu post.
\par 22 Pe când străbăteau Galileea, Iisus le-a spus: Fiul Omului va să fie dat în mâinile oamenilor.
\par 23 Și-L vor omorî, dar a treia zi va învia. Și ei s-au întristat foarte!
\par 24 Venind ei în Capernaum, s-au apropiat de Petru cei ce strâng darea (pentru Templu) și i-au zis: Învățătorul vostru nu plătește darea?
\par 25 Ba, da! - a zis el. Dar intrând în casă, Iisus i-a luat înainte, zicând: Ce ți se pare, Simone? Regii pământului de la cine iau dări sau bir? De la fiii lor sau de la străini?
\par 26 El I-a zis: De la străini. Iisus i-a zis: Așadar, fiii sunt scutiți.
\par 27 Ci ca să nu-i smintim pe ei, mergând la mare, aruncă undița și peștele care va ieși întâi, ia-l, și, deschizându-i gura, vei găsi un statir (un ban de argint). Ia-l și dă-l lor pentru Mine și pentru tine.

\chapter{18}

\par 1 În ceasul acela, s-au apropiat ucenicii de Iisus și I-au zis: Cine, oare, este mai mare în împărăția cerurilor?
\par 2 Și chemând la Sine un prunc, l-a pus în mijlocul lor,
\par 3 Și a zis: Adevărat zic vouă: De nu vă veți întoarce și nu veți fi precum pruncii, nu veți intra în împărăția cerurilor.
\par 4 Deci cine se va smeri pe sine ca pruncul acesta, acela este cel mai mare în împărăția cerurilor.
\par 5 Și cine va primi un prunc ca acesta în numele Meu, pe Mine Mă primește.
\par 6 Iar cine va sminti pe unul dintr-aceștia mici care cred în Mine, mai bine i-ar fi lui să i se atârne de gât o piatră de moară și să fie afundat în adâncul mării.
\par 7 Vai lumii, din pricina smintelilor! Că smintelile trebuie să vină, dar vai omului aceluia prin care vine sminteala.
\par 8 Iar dacă mâna ta sau piciorul tău te smintește, taie-l și aruncă-l de la tine, că este bine pentru tine să intri în viață ciung sau șchiop, decât, având amândouă mâinile sau amândouă picioarele, să fii aruncat în focul cel veșnic.
\par 9 Și dacă ochiul tău te smintește, scoate-l și aruncă-l de la tine, că mai bine este pentru tine să intri în viață cu un singur ochi, decât, având amândoi ochii, să fii aruncat în gheena focului.
\par 10 Vedeți să nu disprețuiți pe vreunul din aceștia mici, că zic vouă: Că îngerii lor, în ceruri, pururea văd fața Tatălui Meu, Care este în ceruri.
\par 11 Căci Fiul Omului a venit să caute și să mântuiască pe cel pierdut.
\par 12 Ce vi se pare? Dacă un om ar avea o sută de oi și una din ele s-ar rătăci, nu va lăsa, oare, în munți pe cele nouăzeci și nouă și ducându-se va căuta pe cea rătăcită?
\par 13 Și dacă s-ar întâmpla s-o găsească, adevăr grăiesc vouă că se bucură de ea mai mult decât de cele nouăzeci și nouă, care nu s-au rătăcit.
\par 14 Astfel nu este vrere înaintea Tatălui vostru, Cel din ceruri, ca să piară vreunul dintr-aceștia mici.
\par 15 De-ți va greși ție fratele tău, mergi, mustră-l pe el între tine și el singur. Și de te va asculta, ai câștigat pe fratele tău.
\par 16 Iar de nu te va asculta, ia cu tine încă unul sau doi, ca din gura a doi sau trei martori să se statornicească tot cuvântul.
\par 17 Și de nu-i va asculta pe ei, spune-l Bisericii; iar de nu va asculta nici de Biserică, să-ți fie ție ca un păgân și vameș.
\par 18 Adevărat grăiesc vouă: Oricâte veți lega pe pământ, vor fi legate și în cer, și oricâte veți dezlega pe pământ, vor fi dezlegate și în cer.
\par 19 Iarăși grăiesc vouă că, dacă doi dintre voi se vor învoi pe pământ în privința unui lucru pe care îl vor cere, se va da lor de către Tatăl Meu, Care este în ceruri.
\par 20 Că unde sunt doi sau trei, adunați în numele Meu, acolo sunt și Eu în mijlocul lor.
\par 21 Atunci Petru, apropiindu-se de El, I-a zis: Doamne, de câte ori va greși față de mine fratele meu și-i voi ierta lui? Oare până de șapte ori?
\par 22 Zis-a lui Iisus: Nu zic ție până de șapte ori, ci până de șaptezeci de ori câte șapte.
\par 23 De aceea, asemănatu-s-a împărăția cerurilor omului împărat care a voit să se socotească cu slugile sale.
\par 24 Și, începând să se socotească cu ele, i s-a adus un datornic cu zece mii de talanți.
\par 25 Dar neavând el cu ce să plătească, stăpânul său a poruncit să fie vândut el și femeia și copii și pe toate câte le are, ca să se plătească.
\par 26 Deci, căzându-i în genunchi, sluga aceea i se închina, zicând: Doamne, îngăduiește-mă și-ți voi plăti ție tot.
\par 27 Iar stăpânul slugii aceleia, milostivindu-se de el, i-a dat drumul și i-a iertat și datoria.
\par 28 Dar, ieșind, sluga aceea a găsit pe unul dintre cei ce slujeau cu el și care-i datora o sută de dinari. Și punând mâna pe el, îl sugruma zicând: Plătește-mi ce ești dator.
\par 29 Deci, căzând cel ce era slugă ca și el, îl ruga zicând: Îngăduiește-mă și îți voi plăti.
\par 30 Iar el nu voia, ci, mergând, l-a aruncat în închisoare, până ce va plăti datoria.
\par 31 Iar celelalte slugi, văzând deci cele petrecute, s-au întristat foarte și, venind, au spus stăpânului toate cele întâmplate.
\par 32 Atunci, chemându-l stăpânul său îi zise: Slugă vicleană, toată datoria aceea ți-am iertat-o, fiindcă m-ai rugat.
\par 33 Nu se cădea, oare, ca și tu să ai milă de cel împreună slugă cu tine, precum și eu am avut milă de tine?
\par 34 Și mâniindu-se stăpânul lui, l-a dat pe mâna chinuitorilor, până ce-i va plăti toată datoria.
\par 35 Tot așa și Tatăl Meu cel ceresc vă va face vouă, dacă nu veți ierta - fiecare fratelui său - din inimile voastre.

\chapter{19}

\par 1 Iar după ce Iisus a sfârșit cuvintele acestea, a plecat din Galileea și a venit în hotarele Iudeii, dincolo de Iordan.
\par 2 Și au mers după El mulțimi multe și i-a vindecat pe ei acolo.
\par 3 Și s-au apropiat de El fariseii, ispitindu-L și zicând: Se cuvine, oare, omului să-și lase femeia sa, pentru orice pricină?
\par 4 Răspunzând, El a zis: N-ați citit că Cel ce i-a făcut de la început i-a făcut bărbat și femeie?
\par 5 Și a zis: Pentru aceea va lăsa omul pe tatăl său și pe mama sa și se va lipi de femeia sa și vor fi amândoi un trup.
\par 6 Așa încât nu mai sunt doi, ci un trup. Deci, ce a împreunat Dumnezeu omul să nu despartă.
\par 7 Ei I-au zis Lui: Pentru ce, dar, Moise a poruncit să-i dea carte de despărțire și să o lase?
\par 8 El le-a zis: Pentru învârtoșarea inimii voastre, v-a dat voie Moise să lăsați pe femeile voastre, dar din început nu a fost așa.
\par 9 Iar Eu zic vouă că oricine va lăsa pe femeia sa, în afară de pricină de desfrânare, și se va însura cu alta, săvârșește adulter; și cine s-a însurat cu cea lăsată săvârșește adulter.
\par 10 Ucenicii I-au zis: Dacă astfel este pricina omului cu femeia, nu este de folos să se însoare.
\par 11 Iar El le-a zis: Nu toți pricep cuvântul acesta, ci aceia cărora le este dat.
\par 12 Că sunt fameni care s-au născut așa din pântecele mamei lor; sunt fameni pe care oamenii i-au făcut fameni, și sunt fameni care s-au făcut fameni pe ei înșiși, pentru împărăția cerurilor. Cine poate înțelege să înțeleagă.
\par 13 Atunci I s-au adus copii, ca să-și pună mâinile peste ei și să Se roage; dar ucenicii îi certau.
\par 14 Iar Iisus a zis: Lăsați copiii și nu-i opriți să vină la Mine, că a unora ca aceștia este împărăția cerurilor.
\par 15 Și punându-Și mâinile peste ei, S-a dus de acolo.
\par 16 Și, iată, venind un tânăr la El, I-a zis: Bunule Învățător, ce bine să fac, ca să am viața veșnică?
\par 17 Iar El a zis: De ce-Mi zici bun? Nimeni nu este bun decât numai Unul Dumnezeu. Iar de vrei să intri în viață, păzește poruncile.
\par 18 El I-a zis: Care? Iar Iisus a zis: Să nu ucizi, să nu săvârșești adulter, să nu furi, să nu mărturisești strâmb;
\par 19 Cinstește pe tatăl tău și pe mama ta și să iubești pe aproapele tău ca pe tine însuți.
\par 20 Zis-a lui tânărul: Toate acestea le-am păzit din copilăria mea. Ce-mi mai lipsește?
\par 21 Iisus i-a zis: Dacă voiești să fii desăvârșit, du-te, vinde averea ta, dă-o săracilor și vei avea comoară în cer; după aceea, vino și urmează-Mi.
\par 22 Ci, auzind cuvântul acesta, tânărul a plecat întristat, căci avea multe avuții.
\par 23 Iar Iisus a zis ucenicilor Săi: Adevărat zic vouă că un bogat cu greu va intra în împărăția cerurilor.
\par 24 Și iarăși zic vouă că mai lesne este să treacă cămila prin urechile acului, decât să intre un bogat în împărăția lui Dumnezeu.
\par 25 Auzind, ucenicii s-au uimit foarte, zicând: Dar cine poate să se mântuiască?
\par 26 Dar Iisus, privind la ei, le-a zis: La oameni aceasta e cu neputință, la Dumnezeu însă toate sunt cu putință.
\par 27 Atunci Petru, răspunzând, I-a zis: Iată noi am lăsat toate și Ți-am urmat Ție. Cu noi oare ce va fi?
\par 28 Iar Iisus le-a zis: Adevărat zic vouă că voi cei ce Mi-ați urmat Mie, la înnoirea lumii, când Fiul Omului va ședea pe tronul slavei Sale, veți ședea și voi pe douăsprezece tronuri, judecând cele douăsprezece seminții ale lui Israel.
\par 29 Și oricine a lăsat case sau frați, sau surori, sau tată, sau mamă, sau femeie, sau copii, sau țarine, pentru numele Meu, înmulțit va lua înapoi și va moșteni viața veșnică.
\par 30 Și mulți dintâi vor fi pe urmă, și cei de pe urmă vor fi întâi.

\chapter{20}

\par 1 Căci împărăția cerurilor este asemenea unui om stăpân de casă, care a ieșit dis-de-dimineață să tocmească lucrători pentru via sa.
\par 2 Și învoindu-se cu lucrătorii cu un dinar pe zi, i-a trimis în via sa.
\par 3 Și ieșind pe la ceasul al treilea, a văzut pe alții stând în piață fără lucru.
\par 4 Și le-a zis acelora: Mergeți și voi în vie, și ce va fi cu dreptul, vă voi da.
\par 5 Iar ei s-au dus. Ieșind iarăși pe la ceasul al șaselea și al nouălea, a făcut tot așa.
\par 6 Ieșind pe la ceasul al unsprezecelea, a găsit pe alții, stând fără lucru, și le-a zis: De ce ați stat aici toată ziua fără lucru?
\par 7 Zis-au lui: Fiindcă nimeni nu ne-a tocmit. Zis-a lor: Duceți-vă și voi în vie și ce va fi cu dreptul veți lua.
\par 8 Făcându-se seară, stăpânul viei a zis către îngrijitorul său: Cheamă pe lucrători și dă-le plata, începând de cei din urmă până la cei dintâi.
\par 9 Venind cei din ceasul al unsprezecelea, au luat câte un dinar.
\par 10 Și venind cei dintâi, au socotit că vor lua mai mult, dar au luat și ei tot câte un dinar.
\par 11 Și după ce au luat, cârteau împotriva stăpânului casei,
\par 12 Zicând: Aceștia de pe urmă au făcut un ceas și i-ai pus deopotrivă cu noi, care am dus greutatea zilei și arșița.
\par 13 Iar el, răspunzând, a zis unuia dintre ei: Prietene, nu-ți fac nedreptate. Oare nu te-ai învoit cu mine un dinar?
\par 14 Ia ce este al tău și pleacă. Voiesc să dau acestuia de pe urmă ca și ție.
\par 15 Au nu mi se cuvine mie să fac ce voiesc cu ale mele? Sau ochiul tău este rău, pentru că eu sunt bun?
\par 16 Astfel vor fi cei de pe urmă întâi și cei dintâi pe urmă, că mulți sunt chemați, dar puțini aleși.
\par 17 Și suindu-Se la Ierusalim, Iisus a luat de o parte pe cei doisprezece ucenici și le-a spus lor, pe cale:
\par 18 Iată ne suim la Ierusalim și Fiul Omului va fi dat pe mâna arhiereilor și a cărturarilor, și-L vor osândi la moarte;
\par 19 Și Îl vor da pe mâna păgânilor, ca să-L batjocorească și să-L răstignească, dar a treia zi va învia.
\par 20 Atunci a venit la El mama fiilor lui Zevedeu, împreună cu fiii ei, închinându-se și cerând ceva de la El.
\par 21 Iar El a zis ei: Ce voiești? Ea a zis Lui: Zi ca să șadă acești doi fii ai mei, unul de-a dreapta 1i altul de-a stânga Ta, întru împărăția Ta.
\par 22 Dar Iisus, răspunzând, a zis: Nu știți ce cereți. Puteți, oare, să beți paharul pe care-l voi bea Eu și cu botezul cu care Eu Mă botez să vă botezați? Ei I-au zis: Putem.
\par 23 Și El a zis lor: Paharul Meu veți bea și cu botezul cu care Eu Mă botez vă veți boteza, dar a ședea de-a dreapta și de-a stânga Mea nu este al Meu a da, ci se va da celor pentru care s-a pregătit de către Tatăl Meu.
\par 24 Și auzind cei zece s-au mâniat pe cei doi frați.
\par 25 Dar Iisus, chemându-i la Sine, a zis: Știți că ocârmuitorii neamurilor domnesc peste ele și cei mari le stăpânesc.
\par 26 Nu tot așa va fi între voi, ci care între voi va vrea să fie mare să fie slujitorul vostru.
\par 27 Și care între voi va vrea să fie întâiul să vă fie vouă slugă,
\par 28 După cum și Fiul Omului n-a venit să I se slujească, ci ca să slujească El și să-Și dea sufletul răscumpărare pentru mulți.
\par 29 Și plecând ei din Ierihon, mulțime mare venea în urma Lui.
\par 30 Și iată doi orbi, care ședeau lângă drum, auzind că trece Iisus, au strigat, zicând: Miluiește-ne pe noi, Doamne, Fiul lui David!
\par 31 Dar mulțimea îi certa ca să tacă; ei însă și mai tare strigau, zicând: Miluiește-ne pe noi, Doamne, Fiul lui David.
\par 32 Și Iisus, stând, i-a chemat și le-a zis: Ce voiți să vă fac?
\par 33 Zis-au Lui: Doamne, să se deschidă ochii noștri.
\par 34 Și făcându-I-se milă, Iisus S-a atins de ochii lor, și îndată au văzut și I-au urmat Lui.

\chapter{21}

\par 1 Iar când s-au apropiat de Ierusalim și au venit la Betfaghe la Muntele Măslinilor, atunci Iisus a trimis pe doi ucenici,
\par 2 Zicându-le: Mergeți în satul care este înaintea voastră și îndată veți găsi o asină legată și un mânz cu ea; dezlegați-o și aduceți-o la Mine.
\par 3 Și dacă vă va zice cineva ceva, veți spune că-I trebuie Domnului; și le va trimite îndată.
\par 4 Iar acestea toate s-au făcut, ca să se împlinească ceea ce s-a spus prin proorocul, care zice:
\par 5 "Spuneți fiicei Sionului: Iată Împăratul tău vine la tine blând și șezând pe asină, pe mânz, fiul celei de sub jug".
\par 6 Mergând deci ucenicii și făcând după cum le-a poruncit Iisus,
\par 7 Au adus asina și mânzul și deasupra lor și-au pus veșmintele, iar El a șezut peste ele.
\par 8 Și cei mai mulți din mulțime își așterneau hainele pe cale, iar alții tăiau ramuri din copaci și le așterneau pe cale,
\par 9 Iar mulțimile care mergeau înaintea Lui și care veneau după El strigau zicând: Osana Fiului lui David; binecuvântat este Cel ce vine întru numele Domnului! Osana întru cei de sus!
\par 10 Și intrând El în Ierusalim, toată cetatea s-a cutremurat, zicând: Cine este Acesta?
\par 11 Iar mulțimile răspundeau: Acesta este Iisus, proorocul din Nazaretul Galileii.
\par 12 Și a intrat Iisus în templu și a alungat pe toți cei ce vindeau și cumpărau în templu și a răsturnat mesele schimbătorilor de bani și scaunele celor care vindeau porumbei.
\par 13 Și a zis lor: Scris este: "Casa Mea, casă de rugăciune se va chema, iar voi o faceți peșteră de tâlhari!"
\par 14 Și au venit la El, în templu, orbi și șchiopi și i-a făcut sănătoși.
\par 15 Și văzând arhiereii și cărturarii minunile pe care le făcuse și pe copiii care strigau în templu și ziceau: Osana Fiului lui David, s-au mâniat,
\par 16 Și I-au zis: Auzi ce zic aceștia? Iar Iisus le-a zis: Da. Au niciodată n-ați citit că din gura copiilor și a celor ce sug Ți-ai pregătit laudă?
\par 17 Și lăsându-i, a ieșit afară din cetate la Betania, și noaptea a rămas acolo.
\par 18 Dimineața, a doua zi, pe când se întorcea în cetate, a flămânzit;
\par 19 Și văzând un smochin lângă cale, S-a dus la el, dar n-a găsit nimic în el decât numai frunze, și a zis lui: De acum înainte să nu mai fie rod din tine în veac! Și smochinul s-a uscat îndată.
\par 20 Văzând aceasta, ucenicii s-au minunat, zicând: Cum s-a uscat smochinul îndată?
\par 21 Iar Iisus, răspunzând, le-a zis: Adevărat grăiesc vouă: Dacă veți avea credință și nu vă veți îndoi, veți face nu numai ce s-a făcut cu smochinul, ci și muntelui acestuia de veți zice: Ridică-te și aruncă-te în mare, va fi așa.
\par 22 Și toate câte veți cere, rugându-vă cu credință, veți primi.
\par 23 Iar după ce a intrat în templu, s-au apropiat de El, pe când învăța, arhiereii și bătrânii poporului și au zis: Cu ce putere faci acestea? Și cine Ți-a dat puterea aceasta?
\par 24 Răspunzând, Iisus le-a zis: Vă voi întreba și Eu pe voi un cuvânt, pe care, de Mi-l veți spune, și Eu vă voi spune vouă cu ce putere fac acestea:
\par 25 Botezul lui Ioan de unde a fost? Din cer sau de la oameni? Iar ei cugetau întru sine, zicând: De vom zice: Din cer, ne va spune: De ce, dar, n-ați crezut lui?
\par 26 Iar de vom zice: De la oameni, ne temem de popor, fiindcă toți îl socotesc pe Ioan de prooroc.
\par 27 Și răspunzând ei lui Iisus, au zis: Nu știm. Zis-a lor și El: Nici Eu nu vă spun cu ce putere fac acestea.
\par 28 Dar ce vi se pare? Un om avea doi fii. Și, ducându-se la cel dintâi, i-a zis: Fiule, du-te astăzi și lucrează în via mea.
\par 29 Iar el, răspunzând, a zis: Mă duc, Doamne, și nu s-a dus.
\par 30 Mergând la al doilea, i-a zis tot așa; acesta, răspunzând, a zis: Nu vreau, apoi căindu-se, s-a dus.
\par 31 Care dintr-aceștia doi a făcut voia Tatălui? Zis-au Lui: Cel de-al doilea. Zis-a lor Iisus: Adevărat grăiesc vouă că vameșii și desfrânatele merg înaintea voastră în împărăția lui Dumnezeu.
\par 32 Căci a venit Ioan la voi în calea dreptății și n-ați crezut în el, ci vameșii și desfrânatele au crezut, iar voi ați văzut și nu v-ați căit nici după aceea, ca să credeți în el.
\par 33 Ascultați altă pildă: Era un om oarecare stăpân al casei sale, care a sădit vie. A împrejmuit-o cu gard, a săpat în ea teasc, a clădit un turn și a dat-o lucrătorilor, iar el s-a dus departe.
\par 34 Când a sosit timpul roadelor, a trimis pe slugile sale la lucrători, ca să-i ia roadele.
\par 35 Dar lucrătorii, punând mâna pe slugi, pe una au bătut-o, pe alta au omorât-o, iar pe alta au ucis-o cu pietre.
\par 36 Din nou a trimis alte slugi, mai multe decât cele dintâi, și au făcut cu ele tot așa.
\par 37 La urmă, a trimis la ei pe fiul său zicând: Se vor rușina de fiul meu.
\par 38 Iar lucrătorii viei, văzând pe fiul, au zis între ei: Acesta este moștenitorul; veniți să-l omorâm și să avem noi moștenirea lui.
\par 39 Și, punând mâna pe el, l-au scos afară din vie și l-au ucis.
\par 40 Deci, când va veni stăpânul viei, ce va face acelor lucrători?
\par 41 I-au răspuns: Pe acești răi, cu rău îi va pierde, iar via o va da altor lucrători, care vor da roadele la timpul lor.
\par 42 Zis-a lor Iisus: Au n-ați citit niciodată în Scripturi: "Piatra pe care au nesocotit-o ziditorii, aceasta a ajuns să fie în capul unghiului. De la Domnul a fost aceasta și este lucru minunat în ochii noștri"?
\par 43 De aceea vă spun că împărăția lui Dumnezeu se va lua de la voi și se va da neamului care va face roadele ei.
\par 44 Cine va cădea pe piatra aceasta se va sfărâma, iar pe cine va cădea îl va strivi.
\par 45 Iar arhiereii și fariseii, ascultând pildele Lui, au înțeles că despre ei vorbește.
\par 46 Și căutând să-L prindă, s-au temut de popor pentru că Îl socotea prooroc.

\chapter{22}

\par 1 Și, răspunzând, Iisus a vorbit iarăși în pilde, zicându-le:
\par 2 Împărăția cerurilor asemănatu-s-a omului împărat care a făcut nuntă fiului său.
\par 3 Și a trimis pe slugile sale ca să cheme pe cei poftiți la nuntă, dar ei n-au voit să vină.
\par 4 Iarăși a trimis alte slugi, zicând: Spuneți celor chemați: Iată, am pregătit ospățul meu; juncii mei și cele îngrășate s-au junghiat și toate sunt gata. Veniți la nuntă.
\par 5 Dar ei, fără să țină seama, s-au dus: unul la țarina sa, altul la neguțătoria lui;
\par 6 Iar ceilalți, punând mâna pe slugile lui, le-au batjocorit și le-au ucis.
\par 7 Și auzind împăratul de acestea, s-a umplut de mânie, și trimițând oștile sale, a nimicit pe ucigașii aceia și cetății lor i-au dat foc.
\par 8 Apoi a zis către slugile sale: Nunta este gata, dar cei poftiți n-au fost vrednici.
\par 9 Mergeți deci la răspântiile drumurilor și pe câți veți găsi, chemați-i la nuntă.
\par 10 Și ieșind slugile acelea la drumuri, au adunat pe toți câți i-au găsit, și răi și buni, și s-a umplut casa nunții cu oaspeți.
\par 11 Iar intrând împăratul ca să privească pe oaspeți, a văzut acolo un om care nu era îmbrăcat în haină de nuntă,
\par 12 Și i-a zis: Prietene, cum ai intrat aici fără haină de nuntă? El însă a tăcut.
\par 13 Atunci împăratul a zis slugilor: Legați-l de picioare și de mâini și aruncați-l în întunericul cel mai din afară. Acolo va fi plângerea și scrâșnirea dinților.
\par 14 Căci mulți sunt chemați, dar puțini aleși.
\par 15 Atunci s-au dus fariseii și au ținut sfat ca să-L prindă pe El în cuvânt.
\par 16 Și au trimis la El pe ucenicii lor, împreună cu irodianii, zicând: Învățătorule, știm că ești omul adevărului și întru adevăr înveți calea lui Dumnezeu și nu-Ți pasă de nimeni, pentru că nu cauți la fața oamenilor.
\par 17 Spune-ne deci nouă: Ce Ți se pare? Se cuvine să dăm dajdie Cezarului sau nu?
\par 18 Iar Iisus, cunoscând viclenia lor, le-a răspuns: Ce Mă ispitiți, fățarnicilor?
\par 19 Arătați-Mi banul de dajdie. Iar ei I-au adus un dinar.
\par 20 Iisus le-a zis: Al cui e chipul acesta și inscripția de pe el?
\par 21 Răspuns-au ei: Ale Cezarului. Atunci a zis lor: Dați deci Cezarului cele ce sunt ale Cezarului și lui Dumnezeu cele ce sunt ale lui Dumnezeu.
\par 22 Auzind aceasta, s-au minunat și, lăsându-L, s-au dus.
\par 23 În ziua aceea, s-au apropiat de El saducheii, cei ce zic că nu este înviere, și L-au întrebat,
\par 24 Zicând: Învățătorule, Moise a zis: Dacă cineva moare neavând copii, fratele lui să ia de soție pe cea văduvă și să ridice urmași fratelui său.
\par 25 Deci erau, la noi, șapte frați; și cel dintâi s-a însurat și a murit și, neavând urmaș, a lăsat pe femeia sa fratelui său.
\par 26 Asemenea și al doilea și al treilea, până la al șaptelea.
\par 27 În urma tuturor a murit și femeia.
\par 28 La înviere, deci, a cărui dintre cei șapte va fi femeia? Căci toți au avut-o de soție.
\par 29 Răspunzând, Iisus le-a zis: Vă rătăciți neștiind Scripturile, nici puterea lui Dumnezeu.
\par 30 Căci la înviere, nici nu se însoară, nici nu se mărită, ci sunt ca îngerii lui Dumnezeu în cer.
\par 31 Iar despre învierea morților, au n-ați citit ce vi s-a spus vouă de Dumnezeu, zicând:
\par 32 "Eu sunt Dumnezeul lui Avraam și Dumnezeul lui Isaac și Dumnezeul lui Iacov"? Nu este Dumnezeul morților, ci al viilor.
\par 33 Iar mulțimile, ascultându-L, erau uimite de învățătura Lui.
\par 34 Și auzind fariseii că a închis gura saducheilor, s-au adunat laolaltă.
\par 35 Unul dintre ei, învățător de Lege, ispitindu-L pe Iisus, L-a întrebat:
\par 36 Învățătorule, care poruncă este mai mare în Lege?
\par 37 El i-a răspuns: Să iubești pe Domnul Dumnezeul tău, cu toată inima ta, cu tot sufletul tău și cu tot cugetul tău.
\par 38 Aceasta este marea și întâia poruncă.
\par 39 Iar a doua, la fel ca aceasta: Să iubești pe aproapele tău ca pe tine însuți.
\par 40 În aceste două porunci se cuprind toată Legea și proorocii.
\par 41 Și fiind adunați fariseii, i-a întrebat Iisus,
\par 42 Zicând: Ce vi se pare despre Hristos? Al cui Fiu este? Zis-au Lui: Al lui David.
\par 43 Zis-a lor: Cum deci David, în duh, Îl numește pe El Domn? - zicând:
\par 44 "Zis-a Domnul Domnului meu: Șezi de-a dreapta Mea, până ce voi pune pe vrăjmașii Tăi așternut picioarelor Tale".
\par 45 Deci dacă David Îl numește pe El domn, cum este fiu al lui?
\par 46 Și nimeni nu putea să-I răspundă cuvânt și nici n-a mai îndrăznit cineva, din ziua aceea, să-L mai întrebe.

\chapter{23}

\par 1 Atunci a vorbit Iisus mulțimilor și ucenicilor Săi,
\par 2 Zicând: Cărturarii și fariseii au șezut în scaunul lui Moise;
\par 3 Deci toate câte vă vor zice vouă, faceți-le și păziți-le; dar după faptele lor nu faceți, că ei zic, dar nu fac.
\par 4 Că leagă sarcini grele și cu anevoie de purtat și le pun pe umerii oamenilor, iar ei nici cu degetul nu voiesc să le miște.
\par 5 Toate faptele lor le fac ca să fie priviți de oameni; căci își lățesc filacteriile și își măresc ciucurii de pe poale.
\par 6 Și le place să stea în capul mesei la ospețe și în băncile dintâi, în sinagogi,
\par 7 Și să li se plece lumea în piețe și să fie numiți de oameni: Rabi.
\par 8 Voi însă să nu vă numiți rabi, că unul este Învățătorul vostru: Hristos, iar voi toți sunteți frați.
\par 9 Și tată al vostru să nu numiți pe pământ, că Tatăl vostru unul este, Cel din ceruri.
\par 10 Nici învățători să nu vă numiți, că Învățătorul vostru este unul: Hristos.
\par 11 Și care este mai mare între voi să fie slujitorul vostru.
\par 12 Cine se va înălța pe sine se va smeri, și cine se va smeri pe sine se va înălța.
\par 13 Vai vouă, cărturarilor și fariseilor fățarnici! Că închideți împărăția cerurilor înaintea oamenilor; că voi nu intrați, și nici pe cei ce vor să intre nu-i lăsați.
\par 14 Vai vouă, cărturarilor și fariseilor fățarnici! Că mâncați casele văduvelor și cu fățărnicie vă rugați îndelung; pentru aceasta mai multă osândă veți lua.
\par 15 Vai vouă, cărturarilor și fariseilor fățarnici! Că înconjurați marea și uscatul ca să faceți un ucenic, și dacă l-ați făcut, îl faceți fiu al gheenei și îndoit decât voi.
\par 16 Vai vouă, călăuze oarbe, care ziceți: Cel ce se va jura pe templu nu este cu nimic legat, dar cel ce se va jura pe aurul templului este legat.
\par 17 Nebuni și orbi! Ce este mai mare, aurul sau templul care sfințește aurul?
\par 18 Ziceți iar: Cel ce se va jura pe altar cu nimic nu este legat, dar cel ce se va jura pe darul ce este deasupra altarului este legat.
\par 19 Nebuni și orbi! Ce este mai mare, darul sau altarul care sfințește darul?
\par 20 Deci, cel ce se jură pe altar se jură pe el și pe toate câte sunt deasupra lui.
\par 21 Deci cel ce se jură pe templu se jură pe el și pe Cel care locuiește în el.
\par 22 Cel ce se jură pe cer se jură pe tronul lui Dumnezeu și pe Cel ce șade pe el.
\par 23 Vai vouă, cărturarilor și fariseilor fățarnici! Că dați zeciuială din izmă, din mărar și din chimen, dar ați lăsat părțile mai grele ale Legii: judecata, mila și credința; pe acestea trebuia să le faceți și pe acelea să nu le lăsați
\par 24 Călăuze oarbe care strecurați țânțarul și înghițiți cămila!
\par 25 Vai vouă, cărturarilor și fariseilor fățarnici! Că voi curățiți partea din afară a paharului și a blidului, iar înăuntru sunt pline de răpire și de lăcomie.
\par 26 Fariseule orb! Curăță întâi partea dinăuntru a paharului și a blidului, ca să fie curată și cea din afară.
\par 27 Vai vouă, cărturarilor și fariseilor fățarnici! Că semănați cu mormintele cele văruite, care pe din afară se arată frumoase, înăuntru însă sunt pline de oase de morți și de toată necurăția.
\par 28 Așa și voi, pe din afară vă arătați drepți oamenilor, înăuntru însă sunteți plini de fățărnicie și de fărădelege.
\par 29 Vai vouă, cărturarilor și fariseilor fățarnici! Că zidiți mormintele proorocilor și împodobiți pe ale drepților,
\par 30 Și ziceți: De am fi fost noi în zilele părinților noștri, n-am fi fost părtași cu ei la vărsarea sângelui proorocilor.
\par 31 Astfel, dar, mărturisiți voi înșivă că sunteți fii ai celor ce au ucis pe prooroci.
\par 32 Dar voi întreceți măsura părinților voștri!
\par 33 Șerpi, pui de vipere, cum veți scăpa de osânda gheenei?
\par 34 De aceea, iată Eu trimit la voi prooroci și înțelepți și cărturari; dintre ei veți ucide și veți răstigni; dintre ei veți biciui în sinagogi și-i veți urmări din cetate în cetate,
\par 35 Ca să cadă asupra voastră tot sângele drepților răspândit pe pământ, de la sângele dreptului Abel, până la sângele lui Zaharia, fiul lui Varahia, pe care l-ați ucis între templu și altar.
\par 36 Adevărat grăiesc vouă, vor veni acestea toate asupra acestui neam.
\par 37 Ierusalime, Ierusalime, care omori pe prooroci și cu pietre ucizi pe cei trimiși la tine; de câte ori am voit să adun pe fiii tăi, după cum adună pasărea puii săi sub aripi, dar nu ați voit.
\par 38 Iată, casa voastră vi se lasă pustie;
\par 39 Căci vă zic vouă: De acum nu Mă veți mai vedea, până când nu veți zice: Binecuvântat este Cel ce vine întru numele Domnului.

\chapter{24}

\par 1 Și ieșind Iisus din templu, S-a dus și s-au apropiat de el ucenicii Lui, ca să-I arate clădirile templului.
\par 2 Iar El, răspunzând, le-a zis: Vedeți toate acestea? Adevărat grăiesc vouă: Nu va rămâne aici piatră pe piatră, care să nu se risipească.
\par 3 Și șezând El pe Muntele Măslinilor, au venit la El ucenicii, de o parte, zicând: Spune nouă când vor fi acestea și care este semnul venirii Tale și al sfârșitului veacului?
\par 4 Răspunzând, Iisus le-a zis: Vedeți să nu vă amăgească cineva.
\par 5 Căci mulți vor veni în numele Meu, zicând: Eu sunt Hristos, și pe mulți îi vor amăgi.
\par 6 Și veți auzi de războaie și de zvonuri de războaie; luați seama să nu vă speriați, căci trebuie să fie toate, dar încă nu este sfârșitul.
\par 7 Căci se va ridica neam peste neam și împărăție peste împărăție și va fi foamete și ciumă și cutremure pe alocuri.
\par 8 Dar toate acestea sunt începutul durerilor.
\par 9 Atunci vă vor da pe voi spre asuprire și vă vor ucide și veți fi urâți de toate neamurile pentru numele Meu.
\par 10 Atunci mulți se vor sminti și se vor vinde unii pe alții; și se vor urî unii pe alții.
\par 11 Și mulți prooroci mincinoși se vor scula și vor amăgi pe mulți.
\par 12 Iar din pricina înmulțirii fărădelegii, iubirea multora se va răci.
\par 13 Dar cel ce va răbda până sfârșit, acela se va mântui.
\par 14 Și se va propovădui această Evanghelie a împărăției în toată lumea spre mărturie la toate neamurile; și atunci va veni sfârșitul.
\par 15 Deci, când veți vedea urâciunea pustiirii ce s-a zis prin Daniel proorocul, stând în locul cel sfânt - cine citește să înțeleagă -
\par 16 Atunci cei din Iudeea să fugă în munți.
\par 17 Cel ce va fi pe casă să nu se coboare, ca să-și ia lucrurile din casă.
\par 18 Iar cel ce va fi în țarină să nu se întoarcă înapoi, ca să-și ia haina.
\par 19 Vai de cele însărcinate și de cele ce vor alăpta în zilele acelea!
\par 20 Rugați-vă ca să nu fie fuga voastră iarna, nici sâmbăta.
\par 21 Căci va fi atunci strâmtorare mare, cum n-a fost de la începutul lumii până acum și nici nu va mai fi.
\par 22 Și de nu s-ar fi scurtat acele zile, n-ar mai scăpa nici un trup, dar pentru cei aleși se vor scurta acele zile.
\par 23 Atunci, de vă va zice cineva: Iată, Mesia este aici sau dincolo, să nu-l credeți.
\par 24 Căci se vor ridica hristoși mincinoși și prooroci mincinoși și vor da semne mari și chiar minuni, ca să amăgească, de va fi cu putință, și pe cei aleși.
\par 25 Iată, v-am spus de mai înainte.
\par 26 Deci, de vă vor zice vouă: Iată este în pustie, să nu ieșiți; iată este în cămări, să nu credeți.
\par 27 Căci precum fulgerul iese de la răsărit și se arată până la apus, așa va fi și venirea Fiului Omului.
\par 28 Căci unde va fi stârvul, acolo se vor aduna vulturii.
\par 29 Iar îndată după strâmtorarea acelor zile, soarele se va întuneca și luna nu va mai da lumina ei, iar stelele vor cădea din cer și puterile cerurilor se vor zgudui.
\par 30 Atunci se va arăta pe cer semnul Fiului Omului și vor plânge toate neamurile pământului și vor vedea pe Fiul Omului venind pe norii cerului, cu putere și cu slavă multă.
\par 31 Și va trimite pe îngerii Săi, cu sunet mare de trâmbiță, și vor aduna pe cei aleși ai Lui din cele patru vânturi, de la marginile cerurilor până la celelalte margini.
\par 32 Învățați de la smochin pilda: Când mlădița lui se face fragedă și odrăslește frunze, cunoașteți că vara e aproape.
\par 33 Asemenea și voi, când veți vedea toate acestea, să știți că este aproape, la uși.
\par 34 Adevărat grăiesc vouă că nu va trece neamul acesta, până ce nu vor fi toate acestea.
\par 35 Cerul și pământul vor trece, dar cuvintele Mele nu vor trece.
\par 36 Iar de ziua și de ceasul acela nimeni nu știe, nici îngerii din ceruri, nici Fiul, ci numai Tatăl.
\par 37 Și precum a fost în zilele lui Noe, așa va fi venirea Fiului Omului.
\par 38 Căci precum în zilele acelea dinainte de potop, oamenii mâncau și beau, se însurau și se măritau, până în ziua când a intrat Noe în corabie,
\par 39 Și n-au știut până ce a venit potopul și i-a luat pe toți, la fel va fi și venirea Fiului Omului.
\par 40 Atunci, din doi care vor fi în țarină, unul se va lua și altul se va lăsa.
\par 41 Din două care vor măcina la moară, una se va lua și alta se va lăsa.
\par 42 Privegheați deci, că nu știți în care zi vine Domnul vostru.
\par 43 Aceea cunoașteți, că de-ar ști stăpânul casei la ce strajă din noapte vine furul, ar priveghea și n-ar lăsa să i se spargă casa.
\par 44 De aceea și voi fiți gata, că în ceasul în care nu gândiți Fiul Omului va veni.
\par 45 Cine, oare, este sluga credincioasă și înțeleaptă pe care a pus-o stăpânul peste slugile sale, ca să le dea hrană la timp?
\par 46 Fericită este sluga aceea, pe care venind stăpânul său, o va afla făcând așa.
\par 47 Adevărat zic vouă că peste toate avuțiile sale o va pune.
\par 48 Iar dacă acea slugă, rea fiind, va zice în inima sa: Stăpânul meu întârzie,
\par 49 Și va începe să bată pe cei ce slujesc împreună cu el, să mănânce și să bea cu bețivii,
\par 50 Veni-va stăpânul slugii aceleia în ziua când nu se așteaptă și în ceasul pe care nu-l cunoaște,
\par 51 Și o va tăia din dregătorie și partea ei o va pune cu fățarnicii. Acolo va fi plângerea și scrâșnirea dinților.

\chapter{25}

\par 1 Împărăția cerurilor se va asemăna cu zece fecioare, care luând candelele lor, au ieșit în întâmpinarea mirelui.
\par 2 Cinci însă dintre ele erau fără minte, iar cinci înțelepte.
\par 3 Căci cele fără de minte, luând candelele, n-au luat cu sine untdelemn.
\par 4 Iar cele înțelepte au luat untdelemn în vase, odată cu candelele lor.
\par 5 Dar mirele întârziind, au ațipit toate și au adormit.
\par 6 Iar la miezul nopții s-a făcut strigare: Iată, mirele vine! Ieșiți întru întâmpinarea lui!
\par 7 Atunci s-au deșteptat toate acele fecioare și au împodobit candelele lor.
\par 8 Și cele fără de minte au zis către cele înțelepte: Dați-ne din untdelemnul vostru, că se sting candelele noastre.
\par 9 Dar cele înțelepte le-au răspuns, zicând: Nu, ca nu cumva să nu ne ajungă nici nouă și nici vouă. Mai bine mergeți la cei ce vând și cumpărați pentru voi.
\par 10 Deci plecând ele ca să cumpere, a venit mirele și cele ce erau gata au intrat cu el la nuntă și ușa s-a închis.
\par 11 Iar mai pe urmă, au sosit și celelalte fecioare, zicând: Doamne, Doamne, deschide-ne nouă.
\par 12 Iar el, răspunzând, a zis: Adevărat zic vouă: Nu vă cunosc pe voi.
\par 13 Drept aceea, privegheați, că nu știți ziua, nici ceasul când vine Fiul Omului.
\par 14 Și mai este ca un om care, plecând departe, și-a chemat slugile și le-a dat pe mână avuția sa.
\par 15 Unuia i-a dat cinci talanți, altuia doi, altuia unul, fiecăruia după puterea lui și a plecat.
\par 16 Îndată, mergând, cel ce luase cinci talanți a lucrat cu ei și a câștigat alți cinci talanți.
\par 17 De asemenea și cel cu doi a câștigat alți doi.
\par 18 Iar cel ce luase un talant s-a dus, a săpat o groapă în pământ și a ascuns argintul stăpânului său.
\par 19 După multă vreme a venit și stăpânul acelor slugi și a făcut socoteala cu ele.
\par 20 Și apropiindu-se cel care luase cinci talanți, a adus alți cinci talanți, zicând: Doamne, cinci talanți mi-ai dat, iată alți cinci talanți am câștigat cu ei.
\par 21 Zis-a lui stăpânul: Bine, slugă bună și credincioasă, peste puține ai fost credincioasă, peste multe te voi pune; intră întru bucuria domnului tău.
\par 22 Apropiindu-se și cel cu doi talanți, a zis: Doamne, doi talanți mi-ai dat, iată alți doi talanți am câștigat cu ei.
\par 23 Zis-a lui stăpânul: Bine, slugă bună și credincioasă, peste puține ai fost credincioasă, peste multe te voi pune; intră întru bucuria domnului tău.
\par 24 Apropiindu-se apoi și cel care primise un talant, a zis: Doamne, te-am știut că ești om aspru, care seceri unde n-ai semănat și aduni de unde n-ai împrăștiat.
\par 25 Și temându-mă, m-am dus de am ascuns talantul tău în pământ; iată ai ce este al tău.
\par 26 Și răspunzând stăpânul său i-a zis: Slugă vicleană și leneșă, știai că secer de unde n-am semănat și adun de unde n-am împrăștiat?
\par 27 Se cuvenea deci ca tu să pui banii mei la zarafi, și eu, venind, aș fi luat ce este al meu cu dobândă.
\par 28 Luați deci de la el talantul și dați-l celui ce are zece talanți.
\par 29 Căci tot celui ce are i se va da și-i va prisosi, iar de la cel ce n-are și ce are i se va lua.
\par 30 Iar pe sluga netrebnică aruncați-o întru întunericul cel mai din afară. Acolo va fi plângerea și scrâșnirea dinților.
\par 31 Când va veni Fiul Omului întru slava Sa, și toți sfinții îngeri cu El, atunci va ședea pe tronul slavei Sale.
\par 32 Și se vor aduna înaintea Lui toate neamurile și-i va despărți pe unii de alții, precum desparte păstorul oile de capre.
\par 33 Și va pune oile de-a dreapta Sa, iar caprele de-a stânga.
\par 34 Atunci va zice Împăratul celor de-a dreapta Lui: Veniți, binecuvântații Tatălui Meu, moșteniți împărăția cea pregătită vouă de la întemeierea lumii.
\par 35 Căci flămând am fost și Mi-ați dat să mănânc; însetat am fost și Mi-ați dat să beau; străin am fost și M-ați primit;
\par 36 Gol am fost și M-ați îmbrăcat; bolnav am fost și M-ați cercetat; în temniță am fost și ați venit la Mine.
\par 37 Atunci drepții Îi vor răspunde, zicând: Doamne, când Te-am văzut flămând și Te-am hrănit? Sau însetat și Ți-am dat să bei?
\par 38 Sau când Te-am văzut străin și Te-am primit, sau gol și Te-am îmbrăcat?
\par 39 Sau când Te-am văzut bolnav sau în temniță și am venit la Tine?
\par 40 Iar Împăratul, răspunzând, va zice către ei: Adevărat zic vouă, întrucât ați făcut unuia dintr-acești frați ai Mei, prea mici, Mie Mi-ați făcut.
\par 41 Atunci va zice și celor de-a stânga: Duceți-vă de la Mine, blestemaților, în focul cel veșnic, care este gătit diavolului și îngerilor lui.
\par 42 Căci flămând am fost și nu Mi-ați dat să mănânc; însetat am fost și nu Mi-ați dat să beau;
\par 43 Străin am fost și nu M-ați primit; gol, și nu M-ați îmbrăcat; bolnav și în temniță, și nu M-ați cercetat.
\par 44 Atunci vor răspunde și ei, zicând: Doamne, când Te-am văzut flămând, sau însetat, sau străin, sau gol, sau bolnav, sau în temniță și nu Ți-am slujit?
\par 45 El însă le va răspunde, zicând: Adevărat zic vouă: Întrucât nu ați făcut unuia dintre acești prea mici, nici Mie nu Mi-ați făcut.
\par 46 Și vor merge aceștia la osândă veșnică, iar drepții la viață veșnică.

\chapter{26}

\par 1 Iar după ce a sfârșit toate aceste cuvinte, a zis Iisus către ucenicii Săi:
\par 2 Știți că peste două zile va fi Paștile și Fiul Omului va fi dat să fie răstignit.
\par 3 Atunci arhiereii și bătrânii poporului s-au adunat în curtea arhiereului, care se numea Caiafa.
\par 4 Și împreună s-au sfătuit ca să prindă pe Iisus, cu vicleșug, și să-L ucidă.
\par 5 Dar ziceau: Nu în ziua praznicului, ca să nu se facă tulburare în popor.
\par 6 Fiind Iisus în Betania, în casa lui Simon Leprosul,
\par 7 S-a apropiat de El o femeie, având un alabastru cu mir de mare preț, și l-a turnat pe capul Lui, pe când ședea la masă.
\par 8 Și văzând ucenicii, s-au mâniat și au zis: De ce risipa aceasta?
\par 9 Căci mirul acesta se putea vinde scump, iar banii să se dea săracilor.
\par 10 Dar Iisus, cunoscând gândul lor, le-a zis: Pentru ce faceți supărare femeii? Căci lucru bun a făcut ea față de Mine.
\par 11 Căci pe săraci totdeauna îi aveți cu voi, dar pe Mine nu Mă aveți totdeauna;
\par 12 Că ea, turnând mirul acesta pe trupul Meu, a făcut-o spre îngroparea Mea.
\par 13 Adevărat zic vouă: Oriunde se va propovădui Evanghelia aceasta, în toată lumea, se va spune și ce-a făcut ea, spre pomenirea ei.
\par 14 Atunci unul din cei doisprezece, numit Iuda Iscarioteanul, ducându-se la arhierei,
\par 15 A zis: Ce voiți să-mi dați și eu Îl voi da în mâinile voastre? Iar ei i-au dat treizeci de arginți.
\par 16 Și de atunci căuta un prilej potrivit ca să-L dea în mâinile lor.
\par 17 În cea dintâi zi a Azimelor, au venit ucenicii la Iisus și L-au întrebat: Unde voiești să-Ți pregătim să mănânci Paștile?
\par 18 Iar El a zis: Mergeți în cetate, la cutare și spuneți-i: Învățătorul zice: Timpul Meu este aproape; la tine vreau să fac Paștile cu ucenicii Mei.
\par 19 Și ucenicii au făcut precum le-a poruncit Iisus și au pregătit Paștile.
\par 20 Iar când s-a făcut seară, a șezut la masă cu cei doisprezece ucenici.
\par 21 Și pe când mâncau, Iisus a zis: Adevărat grăiesc vouă, că unul dintre voi Mă va vinde.
\par 22 Și ei, întristându-se foarte, au început să-I zică fiecare: Nu cumva eu sunt, Doamne?
\par 23 Iar El, răspunzând, a zis: Cel ce a întins cu Mine mâna în blid, acela Mă va vinde.
\par 24 Fiul Omului merge precum este scris despre El. Vai, însă, acelui om prin care Fiul Omului se vinde! Bine era de omul acela dacă nu se năștea.
\par 25 Și Iuda, cel ce L-a vândut, răspunzând a zis: Nu cumva sunt eu, Învățătorule? Răspuns-a lui: Tu ai zis.
\par 26 Iar pe când mâncau ei, Iisus, luând pâine și binecuvântând, a frânt și, dând ucenicilor, a zis: Luați, mâncați, acesta este trupul Meu.
\par 27 Și luând paharul și mulțumind, le-a dat, zicând: Beți dintru acesta toți,
\par 28 Că acesta este Sângele Meu, al Legii celei noi, care pentru mulți se varsă spre iertarea păcatelor.
\par 29 Și vă spun vouă că nu voi mai bea de acum din acest rod al viței până în ziua aceea când îl voi bea cu voi, nou, întru împărăția Tatălui Meu.
\par 30 Și după ce au cântat laude, au ieșit la Muntele Măslinilor.
\par 31 Atunci Iisus le-a zis: Voi toți vă veți sminti întru Mine în noaptea aceasta căci scris este: "Bate-voi păstorul și se vor risipi oile turmei".
\par 32 Dar după învierea Mea voi merge mai înainte de voi în Galileea.
\par 33 Iar Petru, răspunzând, I-a zis: Dacă toți se vor sminti întru Tine, eu niciodată nu mă voi sminti.
\par 34 Zis-a Iisus lui: Adevărat zic ție că în noaptea aceasta, mai înainte de a cânta cocoșul, de trei ori te vei lepăda de Mine.
\par 35 Petru i-a zis: Și de ar fi să mor împreună cu Tine, nu mă voi lepăda de Tine. Și toți ucenicii au zis la fel.
\par 36 Atunci Iisus a mers împreună cu ei la un loc ce se cheamă Ghetsimani și a zis ucenicilor: Ședeți aici, până ce Mă voi duce acolo și Mă voi ruga.
\par 37 Și luând cu Sine pe Petru și pe cei doi fii ai lui Zevedeu, a început a Se întrista și a Se mâhni.
\par 38 Atunci le-a zis: Întristat este sufletul Meu până la moarte. Rămâneți aici și privegheați împreună cu Mine.
\par 39 Și mergând puțin mai înainte, a căzut cu fața la pământ, rugându-Se și zicând: Părintele Meu, de este cu putință, treacă de la Mine paharul acesta! Însă nu precum voiesc Eu, ci precum Tu voiești.
\par 40 Și a venit la ucenici și i-a găsit dormind și i-a zis lui Petru: Așa, n-ați putut un ceas să privegheați cu Mine!
\par 41 Privegheați și vă rugați, ca să nu intrați în ispită. Căci duhul este osârduitor, dar trupul este neputincios.
\par 42 Iarăși ducându-se, a doua oară, s-a rugat, zicând: Părintele Meu, dacă nu este cu putință să treacă acest pahar, ca să nu-l beau, facă-se voia Ta.
\par 43 Și venind iarăși, i-a aflat dormind, căci ochii lor erau îngreuiați.
\par 44 Și lăsându-i, S-a dus iarăși și a treia oară S-a rugat, același cuvânt zicând.
\par 45 Atunci a venit la ucenici și le-a zis: Dormiți de acum și vă odihniți! Iată s-a apropiat ceasul și Fiul Omului va fi dat în mâinile păcătoșilor.
\par 46 Sculați-vă să mergem, iată s-a apropiat cel ce M-a vândut.
\par 47 Și pe când vorbea încă, iată a sosit Iuda, unul dintre cei doisprezece, și împreună cu el mulțime multă, cu săbii și cu ciomege, de la arhierei și de la bătrânii poporului.
\par 48 Iar vânzătorul le-a dat semn, zicând: Pe care-L voi săruta, Acela este: puneți mâna pe El.
\par 49 Și îndată, apropiindu-se de Iisus, a zis: Bucură-Te, Învățătorule! Și L-a sărutat.
\par 50 Iar Iisus i-a zis: Prietene, pentru ce ai venit? Atunci ei, apropiindu-se, au pus mâinile pe Iisus și L-au prins.
\par 51 Și iată, unul dintre cei ce erau cu Iisus, întinzând mâna, a tras sabia și, lovind pe sluga arhiereului, i-a tăiat urechea.
\par 52 Atunci Iisus i-a zis: Întoarce sabia ta la locul ei, că toți cei ce scot sabia, de sabie vor pieri.
\par 53 Sau ți se pare că nu pot să rog pe Tatăl Meu și să-Mi trimită acum mai mult de douăsprezece legiuni de îngeri?
\par 54 Dar cum se vor împlini Scripturile, că așa trebuie să fie?
\par 55 În ceasul acela, a zis Iisus mulțimilor: Ca la un tâlhar ați ieșit cu săbii și cu ciomege, ca să Mă prindeți. În fiecare zi ședeam în templu și învățam și n-ați pus mâna pe Mine.
\par 56 Dar toate acestea s-au făcut ca să se împlinească Scripturile proorocilor. Atunci toți ucenicii, lăsându-L, au fugit.
\par 57 Iar cei care au prins pe Iisus L-au dus la Caiafa arhiereul, unde erau adunați cărturarii și bătrânii.
\par 58 Iar Petru Îl urma de departe până a ajuns la curtea arhiereului și, intrând înăuntru, ședea cu slugile, ca să vadă sfârșitul.
\par 59 Iar arhiereii, bătrânii și tot sinedriul căutau mărturie mincinoasă împotriva lui Iisus, ca să-L omoare.
\par 60 Și n-au găsit, deși veniseră mulți martori mincinoși. Mai pe urmă însă au venit doi și au spus:
\par 61 Acesta a zis: Pot să dărâm templul lui Dumnezeu și în trei zile să-l clădesc.
\par 62 Și, sculându-se, arhiereul I-a zis: Nu răspunzi nimic la ceea ce mărturisesc aceștia împotriva Ta?
\par 63 Dar Iisus tăcea. Și arhiereul I-a zis: Te jur pe Dumnezeul cel viu, să ne spui nouă de ești Tu Hristosul, Fiul lui Dumnezeu.
\par 64 Iisus i-a răspuns: Tu ai zis. Și vă spun încă: De acum veți vedea pe Fiul Omului șezând de-a dreapta puterii și venind pe norii cerului.
\par 65 Atunci arhiereul și-a sfâșiat hainele, zicând: A hulit! Ce ne mai trebuie martori? Iată acum ați auzit hula Lui.
\par 66 Ce vi se pare? Iar ei, răspunzând, au zis: Este vinovat de moarte.
\par 67 Și au scuipat în obrazul Lui, bătându-L cu pumnii, iar unii Îi dădeau palme,
\par 68 Zicând: Proorocește-ne, Hristoase, cine este cel ce Te-a lovit.
\par 69 Iar Petru ședea afară, în curte. Și o slujnică s-a apropiat de el, zicând: Și tu erai cu Iisus Galileianul.
\par 70 Dar el s-a lepădat înaintea tuturor, zicând: Nu știu ce zici.
\par 71 Și ieșind el la poartă, l-a văzut alta și a zis celor de acolo: Și acesta era cu Iisus Nazarineanul.
\par 72 Și iarăși s-a lepădat cu jurământ: Nu cunosc pe omul acesta.
\par 73 Iar după puțin, apropiindu-se cei ce stăteau acolo au zis lui Petru: Cu adevărat și tu ești dintre ei, căci și graiul te vădește.
\par 74 Atunci el a început a se blestema și a se jura: Nu cunosc pe omul acesta. Și îndată a cântat cocoșul.
\par 75 Și Petru și-a adus aminte de cuvântul lui Iisus, care zisese: Mai înainte de a cânta cocoșul, de trei ori te vei lepăda de Mine. Și ieșind afară, a plâns cu amar.

\chapter{27}

\par 1 Iar făcându-se dimineață, toți arhiereii și bătrânii poporului au ținut sfat împotriva lui Iisus, ca să-L omoare.
\par 2 Și, legându-L, L-au dus și L-au predat dregătorului Ponțiu Pilat.
\par 3 Atunci Iuda, cel ce L-a vândut, văzând că a fost osândit, s-a căit și a adus înapoi arhiereilor și bătrânilor cei treizeci de arginți,
\par 4 Zicând: Am greșit vânzând sânge nevinovat. Ei i-au zis: Ce ne privește pe noi? Tu vei vedea.
\par 5 Și el, aruncând arginții în templu, a plecat și, ducându-se, s-a spânzurat.
\par 6 Iar arhiereii, luând banii, au zis: Nu se cuvine să-i punem în vistieria templului, deoarece sunt preț de sânge.
\par 7 Și ținând ei sfat, au cumpărat cu ei Țarina Olarului, pentru îngroparea străinilor.
\par 8 Pentru aceea s-a numit țarina aceea Țarina Sângelui, până în ziua de astăzi.
\par 9 Atunci s-a împlinit cuvântul spus de Ieremia proorocul, care zice: "Și au luat cei treizeci de arginți, prețul celui prețuit, pe care l-au prețuit fiii lui Israel,
\par 10 Și i-au dat pe Țarina Olarului după cum mi-a spus mie Domnul".
\par 11 Iar Iisus stătea înaintea dregătorului. Și L-a întrebat dregătorul, zicând: Tu ești regele iudeilor? Iar Iisus i-a răspuns: Tu zici.
\par 12 Și la învinuirile aduse Lui de către arhierei și bătrâni, nu răspundea nimic.
\par 13 Atunci I-a zis Pilat: Nu auzi câte mărturisesc ei împotriva Ta?
\par 14 Și nu i-a răspuns lui nici un cuvânt, încât dregătorul se mira foarte.
\par 15 La sărbătoarea Paștilor, dregătorul avea obiceiul să elibereze mulțimii un întemnițat pe care-l voiau.
\par 16 Și aveau atunci un vinovat vestit, care se numea Baraba.
\par 17 Deci adunați fiind ei, Pilat le-a zis: Pe cine voiți să vi-l eliberez, pe Baraba sau pe Iisus, care se zice Hristos?
\par 18 Că știa că din invidie L-au dat în mâna lui.
\par 19 Și pe când stătea Pilat în scaunul de judecată, femeia lui i-a trimis acest cuvânt: Nimic să nu-I faci Dreptului acestuia, că mult am suferit azi, în vis, pentru El.
\par 20 Însă arhiereii și bătrânii au ațâțat mulțimile ca să ceară pe Baraba, iar pe Iisus să-L piardă.
\par 21 Iar dregătorul, răspunzând, le-a zis: Pe cine din cei doi voiți să vă eliberez? Iar ei au răspuns: Pe Baraba.
\par 22 Și Pilat le-a zis: Dar ce voi face cu Iisus, ce se cheamă Hristos? Toți au răspuns: Să fie răstignit!
\par 23 A zis iarăși Pilat: Dar ce rău a făcut? Ei însă mai tare strigau și ziceau: Să fie răstignit!
\par 24 Și văzând Pilat că nimic nu folosește, ci mai mare tulburare se face, luând apă și-a spălat mâinile înaintea mulțimii, zicând: Nevinovat sunt de sângele Dreptului acestuia. Voi veți vedea.
\par 25 Iar tot poporul a răspuns și a zis: Sângele Lui asupra noastră și asupra copiilor noștri!
\par 26 Atunci le-a eliberat pe Baraba, iar pe Iisus L-a biciuit și L-a dat să fie răstignit.
\par 27 Atunci ostașii dregătorului, ducând ei pe Iisus în pretoriu, au adunat în jurul Lui toată cohorta,
\par 28 Și dezbrăcându-L de toate hainele Lui, I-au pus o hlamidă roșie.
\par 29 Și împletind o cunună de spini, I-au pus-o pe cap și în mâna Lui cea dreaptă trestie; și, îngenunchind înaintea lui își băteau joc de El, zicând: Bucură-Te, regele iudeilor!
\par 30 Și scuipând asupra Lui, au luat trestia și-L băteau peste cap.
\par 31 Iar după ce L-au batjocorit, L-au dezbrăcat de hlamidă, L-au îmbrăcat cu hainele Lui și L-au dus să-L răstignească.
\par 32 Și ieșind, au găsit pe un om din Cirene, cu numele Simon; pe acesta l-au silit să ducă crucea Lui.
\par 33 Și venind la locul numit Golgota, care înseamnă: Locul Căpățânii,
\par 34 I-au dat să bea vin amestecat cu fiere; și, gustând, nu a voit să bea.
\par 35 Iar după ce L-au răstignit, au împărțit hainele Lui, aruncând sorți, ca să se împlinească ceea ce s-a zis de proorocul: "Împărțit-au hainele Mele între ei, iar pentru cămașa Mea au aruncat sorți".
\par 36 Și ostașii, șezând, Îl păzeau acolo.
\par 37 Și deasupra capului au pus vina Lui scrisă: Acesta este Iisus, regele iudeilor.
\par 38 Atunci au fost răstigniți împreună cu El doi tâlhari, unul de-a dreapta și altul de-a stânga.
\par 39 Iar trecătorii Îl huleau, clătinându-și capetele,
\par 40 Și zicând: Tu, Cel ce dărâmi templul și în trei zile îl zidești, mântuiește-Te pe Tine Însuți! Dacă ești Fiul lui Dumnezeu, coboară-Te de pe cruce!
\par 41 Asemenea și arhiereii, bătându-și joc de El, cu cărturarii și cu bătrânii, ziceau:
\par 42 Pe alții i-a mântuit, iar pe Sine nu poate să Se mântuiască! Dacă este regele lui Israel, să Se coboare acum de pe cruce, și vom crede în El.
\par 43 S-a încrezut în Dumnezeu: Să-L scape acum, dacă-L vrea pe El! Căci a zis: Sunt Fiul lui Dumnezeu.
\par 44 În același chip Îl ocărau și tâlharii cei împreună-răstigniți cu El.
\par 45 Iar de la ceasul al șaselea, s-a făcut întuneric peste tot pământul, până la ceasul al nouălea.
\par 46 Iar în ceasul al nouălea a strigat Iisus cu glas mare, zicând: Eli, Eli, lama sabahtani? adică: Dumnezeul Meu, Dumnezeul Meu, pentru ce M-ai părăsit?
\par 47 Iar unii dintre cei ce stăteau acolo, auzind ziceau: Pe Ilie îl strigă Acesta.
\par 48 Și unul dintre ei, alergând îndată și luând un burete, și umplându-l de oțet și punându-l într-o trestie, Îi da să bea.
\par 49 Iar ceilalți ziceau: Lasă, să vedem dacă vine Ilie să-L mântuiască.
\par 50 Iar Iisus, strigând iarăși cu glas mare, Și-a dat duhul.
\par 51 Și iată, catapeteasma templului s-a sfâșiat în două de sus până jos, și pământul s-a cutremurat și pietrele s-au despicat;
\par 52 Mormintele s-au deschis și multe trupuri ale sfinților adormiți s-au sculat.
\par 53 Și ieșind din morminte, după învierea Lui, au intrat în cetatea sfântă și s-au arătat multora.
\par 54 Iar sutașul și cei ce împreună cu el păzeau pe Iisus, văzând cutremurul și cele întâmplate, s-au înfricoșat foarte, zicând: Cu adevărat, Fiul lui Dumnezeu era Acesta!
\par 55 Și erau acolo multe femei, privind de departe, care urmaseră din Galileea pe Iisus, slujindu-I,
\par 56 Între care era Maria Magdalena și Maria, mama lui Iacov și a lui Iosi, și mama fiilor lui Zevedeu.
\par 57 Iar făcându-se seară, a venit un om bogat din Arimateea, cu numele Iosif, care și el era un ucenic al lui Iisus.
\par 58 Acesta, ducându-se la Pilat, a cerut trupul lui Iisus. Atunci Pilat a poruncit să i se dea.
\par 59 Și Iosif, luând trupul, l-a înfășurat în giulgiu curat de in,
\par 60 Și l-a pus în mormântul nou al său, pe care-l săpase în stâncă, și, prăvălind o piatră mare la ușa mormântului, s-a dus.
\par 61 Iar acolo era Maria Magdalena și cealaltă Marie, șezând în fața mormântului.
\par 62 Iar a doua zi, care este după vineri, s-au adunat arhiereii și fariseii la Pilat,
\par 63 Zicând: Doamne, ne-am adus aminte că amăgitorul Acela a spus, fiind încă în viață: După trei zile Mă voi scula.
\par 64 Deci, poruncește ca mormântul să fie păzit până a treia zi, ca nu cumva ucenicii Lui să vină și să-L fure și să spună poporului: S-a sculat din morți. Și va fi rătăcirea de pe urmă mai rea decât cea dintâi.
\par 65 Pilat le-a zis: Aveți strajă; mergeți și întăriți cum știți.
\par 66 Iar ei, ducându-se, au întărit mormântul cu strajă, pecetluind piatra.

\chapter{28}

\par 1 După ce a trecut sâmbăta, când se lumina de ziua întâi a săptămânii (Duminică), au venit Maria Magdalena și cealaltă Marie, ca să vadă mormântul.
\par 2 Și iată s-a făcut cutremur mare, că îngerul Domnului, coborând din cer și venind, a prăvălit piatra și ședea deasupra ei.
\par 3 Și înfățișarea lui era ca fulgerul și îmbrăcămintea lui albă ca zăpada.
\par 4 Și de frica lui s-au cutremurat cei ce păzeau și s-au făcut ca morți.
\par 5 Iar îngerul, răspunzând, a zis femeilor: Nu vă temeți, că știu că pe Iisus cel răstignit Îl căutați.
\par 6 Nu este aici; căci S-a sculat precum a zis; veniți de vedeți locul unde a zăcut.
\par 7 Și degrabă mergând, spuneți ucenicilor Lui că S-a sculat din morți și iată va merge înaintea voastră în Galileea; acolo Îl veți vedea. Iată v-am spus vouă.
\par 8 Iar plecând ele în grabă de la mormânt, cu frică și cu bucurie mare au alergat să vestească ucenicilor Lui.
\par 9 Dar când mergeau ele să vestească ucenicilor, iată Iisus le-a întâmpinat, zicând: Bucurați-vă! Iar ele, apropiindu-se, au cuprins picioarele Lui și I s-au închinat.
\par 10 Atunci Iisus le-a zis: Nu vă temeți. Duceți-vă și vestiți fraților Mei, ca să meargă în Galileea, și acolo Mă vor vedea.
\par 11 Și plecând ele, iată unii din strajă, venind în cetate, au vestit arhiereilor toate cele întâmplate.
\par 12 Și, adunându-se ei împreună cu bătrânii și ținând sfat, au dat bani mulți ostașilor,
\par 13 Zicând: Spuneți că ucenicii Lui, venind noaptea, L-au furat, pe când noi dormeam;
\par 14 Și de se va auzi aceasta la dregătorul, noi îl vom îndupleca și pe voi fără grijă vă vom face.
\par 15 Iar ei, luând arginții, au făcut precum au fost învățați. Și s-a răspândit cuvântul acesta între Iudei, până în ziua de azi.
\par 16 Iar cei unsprezece ucenici au mers în Galileea, la muntele unde le poruncise lor Iisus.
\par 17 Și văzându-L, I s-au închinat, ei care se îndoiseră.
\par 18 Și apropiindu-Se Iisus, le-a vorbit lor, zicând: Datu-Mi-s-a toată puterea, în cer și pe pământ.
\par 19 Drept aceea, mergând, învățați toate neamurile, botezându-le în numele Tatălui și al Fiului și al Sfântului Duh,
\par 20 Învățându-le să păzească toate câte v-am poruncit vouă, și iată Eu cu voi sunt în toate zilele, până la sfârșitul veacului. Amin.


\end{document}