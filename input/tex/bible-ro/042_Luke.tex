\begin{document}

\title{Luca}


\chapter{1}

\par 1 Deoarece mulți s-au încercat să alcătuiască o istorisire despre faptele deplin adeverite între noi,
\par 2 Așa cum ni le-au lăsat cei ce le-au văzut de la început și au fost slujitori ai Cuvântului,
\par 3 Am găsit și eu cu cale, preaputernice Teofile, după ce am urmărit toate cu de-amănuntul de la început, să ți le scriu pe rând,
\par 4 Ca să te încredințezi despre temenicia învățăturii pe care ai primit-o.
\par 5 Era în zilele lui Irod, regele Iudeii, un preot cu numele Zaharia din ceata preoțească a lui Abia, iar femeia lui era din fiicele lui Aaron și se numea Elisabeta.
\par 6 Și erau amândoi drepți înaintea lui Dumnezeu, umblând fără prihană în toate poruncile și rânduielile Domnului.
\par 7 Dar nu aveau nici un copil, deoarece Elisabeta era stearpă și amândoi erau înaintați în zilele lor.
\par 8 Și pe când Zaharia slujea înaintea lui Dumnezeu, în rândul săptămânii sale,
\par 9 A ieșit la sorți, după obiceiul preoției, să tămâieze intrând în templul Domnului.
\par 10 Iar toată mulțimea poporului, în ceasul tămâierii, era afară și se ruga.
\par 11 Și i s-a arătat îngerul Domnului, stând de-a dreapta altarului tămâierii.
\par 12 Și văzându-l, Zaharia s-a tulburat și frică a căzut peste el.
\par 13 Iar îngerul a zis către el: Nu te teme, Zaharia, pentru că rugăciunea ta a fost ascultată și Elisabeta, femeia ta, îți va naște un fiu și-l vei numi Ioan.
\par 14 Și bucurie și veselie vei avea și, de nașterea lui, mulți se vor bucura.
\par 15 Căci va fi mare înaintea Domnului; nu va bea vin, nici altă băutură amețitoare și încă din pântecele mamei sale se va umple de Duhul Sfânt.
\par 16 Și pe mulți din fiii lui Israel îi va întoarce la Domnul Dumnezeul lor.
\par 17 Și va merge înaintea Lui cu duhul și puterea lui Ilie, ca să întoarcă inimile părinților spre copii și pe cei neascultători la înțelepciunea drepților, ca să gătească Domnului un popor pregătit.
\par 18 Și a zis Zaharia către înger: După ce voi cunoaște aceasta? Căci eu sunt bătrân și femeia mea înaintată în zilele ei.
\par 19 Și îngerul, răspunzând, i-a zis: Eu sunt Gavriil, cel ce stă înaintea lui Dumnezeu. Și am fost trimis să grăiesc către tine și să-ți binevestesc acestea.
\par 20 Și iată vei fi mut și nu vei putea să vorbești până în ziua când vor fi acestea, pentru că n-ai crezut în cuvintele mele, care se vor împlini la timpul lor.
\par 21 Și poporul aștepta pe Zaharia și se mira că întârzie în templu.
\par 22 Și ieșind, nu putea să vorbească. Și ei au înțeles că a văzut vedenie în templu; și el le făcea semne și a rămas mut.
\par 23 Și când s-au împlinit zilele slujirii lui la templu, s-a dus la casa sa.
\par 24 Iar după aceste zile, Elisabeta, femeia lui, a zămislit și cinci luni s-a tăinuit pe sine, zicând:
\par 25 Că așa mi-a făcut mie Domnul în zilele în care a socotit să ridice dintre oameni ocara mea.
\par 26 Iar în a șasea lună a fost trimis îngerul Gavriil de la Dumnezeu, într-o cetate din Galileea, al cărei nume era Nazaret,
\par 27 Către o fecioară logodită cu un bărbat care se chema Iosif, din casa lui David; iar numele fecioarei era Maria.
\par 28 Și intrând îngerul la ea, a zis: Bucură-te, ceea ce ești plină de har, Domnul este cu tine. Binecuvântată ești tu între femei.
\par 29 Iar ea, văzându-l, s-a tulburat de cuvântul lui și cugeta în sine: Ce fel de închinăciune poate să fie aceasta?
\par 30 Și îngerul i-a zis: Nu te teme, Marie, căci ai aflat har la Dumnezeu.
\par 31 Și iată vei lua în pântece și vei naște fiu și vei chema numele lui Iisus.
\par 32 Acesta va fi mare și Fiul Celui Preaînalt se va chema și Domnul Dumnezeu Îi va da Lui tronul lui David, părintele Său.
\par 33 Și va împărăți peste casa lui Iacov în veci și împărăția Lui nu va avea sfârșit.
\par 34 Și a zis Maria către înger: Cum va fi aceasta, de vreme ce eu nu știu de bărbat?
\par 35 Și răspunzând, îngerul i-a zis: Duhul Sfânt Se va pogorî peste tine și puterea Celui Preaînalt te va umbri; pentru aceea și Sfântul care Se va naște din tine, Fiul lui Dumnezeu se va chema.
\par 36 Și iată Elisabeta, rudenia ta, a zămislit și ea fiu la bătrânețea ei și aceasta este a șasea lună pentru ea, cea numită stearpă.
\par 37 Că la Dumnezeu nimic nu este cu neputință.
\par 38 Și a zis Maria: Iată roaba Domnului. Fie mie după cuvântul tău! Și îngerul a plecat de la ea.
\par 39 Și în acele zile, sculându-se Maria, s-a dus în grabă în ținutul muntos, într-o cetate a seminției lui Iuda.
\par 40 Și a intrat în casa lui Zaharia și a salutat pe Elisabeta.
\par 41 Iar când a auzit Elisabeta salutarea Mariei, pruncul a săltat în pântecele ei și Elisabeta s-a umplut de Duh Sfânt,
\par 42 Și cu glas mare a strigat și a zis: Binecuvântată ești tu între femei și binecuvântat este rodul pântecelui tău.
\par 43 Și de unde mie aceasta, ca să vină la mine Maica Domnului meu?
\par 44 Că iată, cum veni la urechile mele glasul salutării tale, pruncul a săltat de bucurie în pântecele meu.
\par 45 Și fericită este aceea care a crezut că se vor împlini cele spuse ei de la Domnul.
\par 46 Și a zis Maria: Mărește sufletul meu pe Domnul.
\par 47 Și s-a bucurat duhul meu de Dumnezeu, Mântuitorul meu,
\par 48 Că a căutat spre smerenia roabei Sale. Că, iată, de acum mă vor ferici toate neamurile.
\par 49 Că mi-a făcut mie mărire Cel Puternic și sfânt este numele Lui.
\par 50 Și mila Lui în neam și în neam spre cei ce se tem de El.
\par 51 Făcut-a tărie cu brațul Său, risipit-a pe cei mândri în cugetul inimii lor.
\par 52 Coborât-a pe cei puternici de pe tronuri și a înălțat pe cei smeriți,
\par 53 Pe cei flămânzi i-a umplut de bunătăți și pe cei bogați i-a scos afară deșerți.
\par 54 A sprijinit pe Israel, slujitorul Său, ca să-Și aducă aminte de mila Sa,
\par 55 Precum a grăit către părinții noștri, lui Avraam și seminției lui, în veac.
\par 56 Și a rămas Maria împreună cu ea ca la trei luni; și s-a înapoiat la casa sa.
\par 57 Și după ce s-a împlinit vremea să nască, Elisabeta a născut un fiu.
\par 58 Și au auzit vecinii și rudele ei că Domnul a mărit mila Sa față de ea și se bucurau împreună cu ea.
\par 59 Iar când a fost în ziua a opta, au venit să taie împrejur pruncul și-l numeau Zaharia, după numele tatălui său.
\par 60 Și răspunzând, mama lui a zis: Nu! Ci se va chema Ioan.
\par 61 Și au zis către ea: Nimeni din rudenia ta nu se cheamă cu numele acesta.
\par 62 Și au făcut semn tatălui său cum ar vrea el să fie numit.
\par 63 Și cerând o tăbliță, el a scris, zicând: Ioan este numele lui. Și toți s-au mirat.
\par 64 Și îndată i s-a deschis gura și limba și vorbea, binecuvântând pe Dumnezeu.
\par 65 Și frica i-a cuprins pe toți care locuiau împrejurul lor; și în tot ținutul muntos al Iudeii s-au vestit toate aceste cuvinte.
\par 66 Și toți care le auzeau le puneau la inimă, zicând: Ce va fi, oare, acest copil? Căci mâna Domnului era cu el.
\par 67 Și Zaharia, tatăl lui, s-a umplut de Duh Sfânt și a proorocit, zicând:
\par 68 Binecuvântat este Domnul Dumnezeul lui Israel, că a cercetat și a făcut răscumpărare poporului Său;
\par 69 Și ne-a ridicat putere de mântuire în casa lui David, slujitorul Său,
\par 70 Precum a grăit prin gura sfinților Săi prooroci din veac;
\par 71 Mântuire de vrăjmașii noștri și din mâna tuturor celor ce ne urăsc pe noi.
\par 72 Și să facă milă cu părinții noștri, ca ei să-și aducă aminte de legământul Său cel sfânt;
\par 73 De jurământul cu care S-a jurat către Avraam, părintele nostru,
\par 74 Ca, fiind izbăviți din mâna vrăjmașilor, să ne dea nouă fără frică,
\par 75 Să-I slujim în sfințenie și în dreptate, înaintea feței Sale, în toate zilele vieții noastre.
\par 76 Iar tu, pruncule, prooroc al Celui Preaînalt te vei chema, că vei merge înaintea feței Domnului, ca să gătești căile Lui,
\par 77 Să dai poporului Său cunoștința mântuirii întru iertarea păcatelor lor,
\par 78 Prin milostivirea milei Dumnezeului nostru, cu care ne-a cercetat pe noi Răsăritul cel de Sus,
\par 79 Ca să lumineze pe cei care șed în întuneric și în umbra morții și să îndrepte picioarele noastre pe calea păcii.
\par 80 Iar copilul creștea și se întărea cu duhul. Și a fost în pustie până în ziua arătării lui către Israel.

\chapter{2}

\par 1 În zilele acelea a ieșit poruncă de la Cezarul August să se înscrie toată lumea.
\par 2 Această înscriere s-a făcut întâi pe când Quirinius ocârmuia Siria.
\par 3 Și se duceau toți să se înscrie, fiecare în cetatea sa.
\par 4 Și s-a suit și Iosif din Galileea, din cetatea Nazaret, în Iudeea, în cetatea lui David care se numește Betleem, pentru că el era din casa și din neamul lui David.
\par 5 Ca să se înscrie împreună cu Maria, cea logodită cu el, care era însărcinată.
\par 6 Dar pe când erau ei acolo, s-au împlinit zilele ca ea să nască,
\par 7 Și a născut pe Fiul său, Cel Unul-Născut și L-a înfășat și L-a culcat în iesle, căci nu mai era loc de găzduire pentru ei.
\par 8 Și în ținutul acela erau păstori, stând pe câmp și făcând de strajă noaptea împrejurul turmei lor.
\par 9 Și iată îngerul Domnului a stătut lângă ei și slava Domnului a strălucit împrejurul lor, și ei s-au înfricoșat cu frică mare.
\par 10 Dar îngerul le-a zis: Nu vă temeți. Căci, iată, vă binevestesc vouă bucurie mare, care va fi pentru tot poporul.
\par 11 Că vi s-a născut azi Mântuitor, Care este Hristos Domnul, în cetatea lui David.
\par 12 Și acesta va fi semnul: Veți găsi un prunc înfășat, culcat în iesle.
\par 13 Și deodată s-a văzut, împreună cu îngerul, mulțime de oaste cerească, lăudând pe Dumnezeu și zicând:
\par 14 Slavă întru cei de sus lui Dumnezeu și pe pământ pace, între oameni bunăvoire!
\par 15 Iar după ce îngerii au plecat de la ei, la cer, păstorii vorbeau unii către alții: Să mergem dar până la Betleem, să vedem cuvântul acesta ce s-a făcut și pe care Domnul ni l-a făcut cunoscut.
\par 16 Și, grăbindu-se, au venit și au aflat pe Maria și pe Iosif și pe Prunc, culcat în iesle.
\par 17 Și văzându-L, au vestit cuvântul grăit lor despre acest Copil.
\par 18 Și toți câți auzeau se mirau de cele spuse lor de către păstori.
\par 19 Iar Maria păstra toate aceste cuvinte, punându-le în inima sa.
\par 20 Și s-au întors păstorii, slăvind și lăudând pe Dumnezeu, pentru toate câte auziseră și văzuseră precum li se spusese.
\par 21 Și când s-au împlinit opt zile, ca să-L taie împrejur, I-au pus numele Iisus, cum a fost numit de înger, mai înainte de a se zămisli în pântece.
\par 22 Și când s-au împlinit zilele curățirii lor, după legea lui Moise, L-au adus pe Prunc la Ierusalim, ca să-L pună înaintea Domnului.
\par 23 Precum este scris în Legea Domnului, că orice întâi-născut de parte bărbătească să fie închinat Domnului.
\par 24 Și să dea jertfă, precum s-a zis în Legea Domnului, o pereche de turturele sau doi pui de porumbel.
\par 25 Și iată era un om în Ierusalim, cu numele Simeon; și omul acesta era drept și temător de Dumnezeu, așteptând mângâierea lui Israel, și Duhul Sfânt era asupra lui.
\par 26 Și lui i se vestise de către Duhul Sfânt că nu va vedea moartea până ce nu va vedea pe Hristosul Domnului.
\par 27 Și din îndemnul Duhului a venit la templu; și când părinții au adus înăuntru pe Pruncul Iisus, ca să facă pentru El după obiceiul Legii,
\par 28 El L-a primit în brațele sale și a binecuvântat pe Dumnezeu și a zis:
\par 29 Acum slobozește pe robul Tău, după cuvântul Tău, în pace,
\par 30 Că ochii mei văzură mântuirea Ta,
\par 31 Pe care ai gătit-o înaintea feței tuturor popoarelor,
\par 32 Lumină spre descoperirea neamurilor și slavă poporului Tău Israel.
\par 33 Iar Iosif și mama Lui se mirau de ceea ce se vorbea despre Prunc.
\par 34 Și i-a binecuvântat Simeon și a zis către Maria, mama Lui: Iată, Acesta este pus spre căderea și spre ridicarea multora din Israel și ca un semn care va stârni împotriviri.
\par 35 Și prin sufletul tău va trece sabie, ca să se descopere gândurile din multe inimi.
\par 36 Și era și Ana proorocița, fiica lui Fanuel, din seminția lui Așer, ajunsă la adânci bătrânețe și care trăise cu bărbatul ei șapte ani de la fecioria sa.
\par 37 Și ea era văduvă, în vârstă de optzeci și patru de ani, și nu se depărta de templu, slujind noaptea și ziua în post și în rugăciuni.
\par 38 Și venind ea în acel ceas, lăuda pe Dumnezeu și vorbea despre Prunc tuturor celor ce așteptau mântuire în Ierusalim.
\par 39 După ce au săvârșit toate, s-au întors în Galileea, în cetatea lor Nazaret.
\par 40 Iar Copilul creștea și Se întărea cu duhul, umplându-Se de înțelepciune și harul lui Dumnezeu era asupra Lui.
\par 41 Și părinții Lui, în fiecare an, se duceau de sărbătoarea Paștilor, la Ierusalim.
\par 42 Iar când a fost El de doisprezece ani, s-au suit la Ierusalim, după obiceiul sărbătorii.
\par 43 Și sfârșindu-se zilele, pe când se întorceau ei, Copilul Iisus a rămas în Ierusalim și părinții Lui nu știau.
\par 44 Și socotind că este în ceata călătorilor de drum, au venit cale de o zi, căutându-L printre rude și printre cunoscuți.
\par 45 Și, negăsindu-L, s-au întors la Ierusalim, căutându-L.
\par 46 Iar după trei zile L-au aflat în templu, șezând în mijlocul învățătorilor, ascultându-i și întrebându-i.
\par 47 Și toți care Îl auzeau se minunau de priceperea și de răspunsurile Lui.
\par 48 Și văzându-L, rămaseră uimiți, iar mama Lui a zis către El: Fiule, de ce ne-ai făcut nouă așa? Iată, tatăl Tău și eu Te-am căutat îngrijorați.
\par 49 Și El a zis către ei: De ce era să Mă căutați? Oare, nu știați că în cele ale Tatălui Meu trebuie să fiu?
\par 50 Dar ei n-au înțeles cuvântul pe care l-a spus lor.
\par 51 Și a coborât cu ei și a venit în Nazaret și le era supus. Iar mama Lui păstra în inima ei toate aceste cuvinte.
\par 52 Și Iisus sporea cu înțelepciunea și cu vârsta și cu harul la Dumnezeu și la oameni.

\chapter{3}

\par 1 În al cincisprezecelea an al domniei Cezarului Tiberiu, pe când Ponțiu Pilat era procuratorul Iudeii, Irod, tetrarh al Galileii, Filip, fratele său, tetrarh al Itureii și al ținutului Trahonitidei, iar Lisanias, tetrarh al Abilenei,
\par 2 În zilele arhiereilor Anna și Caiafa, a fost cuvântul lui Dumnezeu către Ioan, fiul lui Zaharia, în pustie.
\par 3 Și a venit el în toată împrejurimea Iordanului, propovăduind botezul pocăinței, spre iertarea păcatelor.
\par 4 Precum este scris în cartea cuvintelor lui Isaia proorocul: "Este glasul celui ce strigă în pustie: Gătiți calea Domnului, drepte faceți cărările Lui.
\par 5 Orice vale se va umple și orice munte și orice deal se va pleca; căile cele strâmbe se vor face drepte și cele colțuroase, drumuri netede.
\par 6 Și toată făptura va vedea mântuirea lui Dumnezeu".
\par 7 Deci zicea Ioan mulțimilor care veneau să se boteze de el: Pui de vipere, cine v-a arătat să fugiți de mânia ce va să fie?
\par 8 Faceți, dar, roade vrednice de pocăință și nu începeți a zice în voi înșivă: Avem tată pe Avraam, căci vă spun că Dumnezeu poate și din pietrele acestea să ridice fii lui Avraam.
\par 9 Acum securea stă la rădăcina pomilor; deci orice pom care nu face roadă bună se taie și se aruncă în foc.
\par 10 Și mulțimile îl întrebau, zicând: Ce să facem deci?
\par 11 Răspunzând, Ioan le zicea: Cel ce are două haine să dea celui ce nu are și cel ce are bucate să facă asemenea.
\par 12 Și au venit și vameșii să se boteze și i-au spus: Învățătorule, noi ce să facem?
\par 13 El le-a răspuns: Nu faceți nimic mai mult peste ce vă este rânduit.
\par 14 Și îl întrebau și ostașii, zicând: Dar noi ce să facem? Și le-a zis: Să nu asupriți pe nimeni, nici să învinuiți pe nedrept, și să fiți mulțumiți cu solda voastră.
\par 15 Iar poporul fiind în așteptare și întrebându-se toți despre Ioan în cugetele lor: Nu cumva el este Hristosul?
\par 16 A răspuns Ioan tuturor, zicând: Eu vă botez cu apă, dar vine Cel ce este mai tare decât mine, Căruia nu sunt vrednic să-I dezleg cureaua încălțămintelor. El vă va boteza cu Duh Sfânt și cu foc,
\par 17 A Cărui lopată este în mâna Lui, ca să curețe aria și să adune grâul în jitnița Sa, iar pleava o va arde cu foc nestins.
\par 18 Încă și alte multe îndemnând, propovăduia poporului vestea cea bună.
\par 19 Iar Irod tetrarhul, mustrat fiind de el pentru Irodiada, femeia lui Filip, fratele său, și pentru toate relele pe care le-a făcut Irod,
\par 20 A adăugat la toate și aceasta, încât a închis pe Ioan în temniță.
\par 21 Și după ce s-a botezat tot poporul, botezându-Se și Iisus și rugându-Se, s-a deschis cerul,
\par 22 Și S-a coborât Duhul Sfânt peste El, în chip trupesc, ca un porumbel, și s-a făcut glas din cer: Tu ești Fiul Meu cel iubit, întru Tine am binevoit.
\par 23 Și Iisus Însuși era ca de treizeci de ani când a început (să propovăduiască), fiind, precum se socotea, fiu al lui Iosif, care era fiul lui Eli,
\par 24 Fiul lui Matat, fiul lui Levi, fiul lui Melhi, fiul lui Ianai, fiul lui Iosif,
\par 25 Fiul lui Matatia, fiul lui Amos, fiul lui Naum, fiul lui Esli, fiul lui Nagai,
\par 26 Fiul lui Iosua, fiul lui Matatia, fiul lui Semein, fiul lui Ioseh, fiul lui Ioda,
\par 27 Fiul lui Ioanan, fiul lui Resa, fiul lui Zorobabel, fiul lui Salatiel, fiul lui Neri,
\par 28 Fiul lui Melhi, fiul lui Adi, fiul lui Cosam, fiul lui Elmadam, fiul lui Er,
\par 29 Fiul lui Iosua, fiul lui Eliezer, fiul lui Lorim, fiul lui Matat, fiul lui Levi,
\par 30 Fiul lui Simeon, fiul lui Iuda, fiul lui Iosif, fiul lui Ionam, fiul lui Eliachim,
\par 31 Fiul lui Melea, fiul lui Mena, fiul lui Matata, fiul lui Natan, fiul lui David,
\par 32 Fiul lui Iesei, fiul lui Iobed, fiul lui Booz, fiul lui Sala, fiul lui Naason,
\par 33 Fiul Aminadav, fiul lui Admin, fiul lui Arni, fiul lui Esrom, fiul lui Fares, fiul lui Iuda.
\par 34 Fiul lui Iacov, fiul lui Isaac, fiul lui Avraam, fiul lui Tara, fiul lui Nahor,
\par 35 Fiul lui Serug, fiul lui Ragav, fiul lui Falec, fiul lui Eber, fiul lui Sala,
\par 36 Fiul lui Cainam, fiul lui Arfaxad, fiul lui Sim, fiul lui Noe, fiul lui Lameh,
\par 37 Fiul lui Matusala, fiul lui Enoh, fiul Iaret, fiul lui Maleleil, fiul lui Cainam,
\par 38 Fiul lui Enos, fiul lui Set, fiul lui Adam, fiul lui Dumnezeu.

\chapter{4}

\par 1 Iar Iisus, plin de Duhul Sfânt, S-a întors de la Iordan și a fost dus de Duhul în pustie,
\par 2 Timp de patruzeci de zile, fiind ispitit de diavolul. Și în aceste zile nu a mâncat nimic; și, sfârșindu-se ele, a flămânzit.
\par 3 Și I-a spus diavolul: Dacă ești Fiul lui Dumnezeu, zi acestei pietre să se facă pâine.
\par 4 Și a răspuns Iisus către el: Scris este că nu numai cu pâine va trăi omul, ci cu orice cuvânt al lui Dumnezeu.
\par 5 Și suindu-L diavolul pe un munte înalt, I-a arătat într-o clipă toate împărățiile lumii.
\par 6 Și I-a zis diavolul: Ție îți voi da toată stăpânirea aceasta și strălucirea lor, căci mi-a fost dată mie și eu o dau cui voiesc;
\par 7 Deci dacă Tu Te vei închina înaintea mea, toată va fi a Ta.
\par 8 Și răspunzând, Iisus i-a zis: Mergi înapoia Mea, satano, căci scris este: "Domnului Dumnezeului tău să te închini și numai Lui Unuia să-I slujești".
\par 9 Și L-a dus în Ierusalim și L-a așezat pe aripa templului și I-a zis: Dacă ești Fiul lui Dumnezeu, aruncă-Te de aici jos;
\par 10 Căci scris este: "Că îngerilor Săi va porunci pentru Tine, ca să Te păzească";
\par 11 Și te vor ridica pe mâini, ca nu cumva să lovești de piatră piciorul Tău.
\par 12 Și răspunzând, Iisus i-a zis: S-a spus: "Să nu ispitești pe Domnul Dumnezeul tău".
\par 13 Și diavolul, sfârșind toată ispita, s-a îndepărtat de la El, până la o vreme.
\par 14 Și S-a întors Iisus în puterea Duhului în Galileea și a ieșit vestea despre El în toată împrejurimea.
\par 15 Și învăța în sinagogile lor, slăvit fiind de toți.
\par 16 Și a venit în Nazaret, unde fusese crescut, și, după obiceiul Său, a intrat în ziua sâmbetei în sinagogă și S-a sculat să citească.
\par 17 Și I s-a dat cartea proorocului Isaia. Și, deschizând El cartea, a găsit locul unde era scris:
\par 18 "Duhul Domnului este peste Mine, pentru care M-a uns să binevestesc săracilor; M-a trimis să vindec pe cei zdrobiți cu inima; să propovăduiesc robilor dezrobirea și celor orbi vederea; să slobozesc pe cei apăsați,
\par 19 Și să vestesc anul plăcut Domnului".
\par 20 Și închizând cartea și dând-o slujitorului, a șezut, iar ochii tuturor erau ațintiți asupra Lui.
\par 21 Și El a început a zice către ei: Astăzi s-a împlinit Scriptura aceasta în urechile voastre.
\par 22 Și toți Îl încuviințau și se mirau de cuvintele harului care ieșeau din gura Lui și ziceau: Nu este, oare, Acesta fiul lui Iosif?
\par 23 Și El le-a zis: Cu adevărat Îmi veți spune această pildă: Doctore, vindecă-te pe tine însuți! Câte am auzit că s-au făcut în Capernaum, fă și aici în patria Ta.
\par 24 Și le-a zis: Adevărat zic vouă că nici un prooroc nu este bine primit în patria sa.
\par 25 Și adevărat vă spun că multe văduve erau în zilele lui Ilie, în Israel, când s-a închis cerul trei ani și șase luni, încât a fost foamete mare peste tot pământul.
\par 26 Și la nici una dintre ele n-a fost trimis Ilie, decât la Sarepta Sidonului, la o femeie văduvă.
\par 27 Și mulți leproși erau în Israel în zilele proorocului Elisei, dar nici unul dintre ei nu s-a curățat, decât Neeman Sirianul.
\par 28 Și toți, în sinagogă, auzind acestea, s-au umplut de mânie.
\par 29 Și sculându-se, L-au scos afară din cetate și L-au dus pe sprânceana muntelui, pe care era zidită cetatea lor, ca să-L arunce în prăpastie;
\par 30 Iar El, trecând prin mijlocul lor, S-a dus.
\par 31 Și S-a coborât la Capernaum, cetate a Galileii, și îi învăța sâmbăta.
\par 32 Și erau uimiți de învățătura Lui, căci cuvântul Lui era cu putere.
\par 33 Iar în sinagogă era un om, având duh de demon necurat, și a strigat cu glas tare:
\par 34 Lasă! Ce ai cu noi, Iisuse Nazarinene? Ai venit ca să ne pierzi? Te știm cine ești: Sfântul lui Dumnezeu.
\par 35 Și l-a certat Iisus, zicând: Taci și ieși din el. Iar demonul, aruncându-l în mijlocul sinagogii, a ieșit din el, cu nimic vătămându-l.
\par 36 Și frică li s-a făcut tuturor și spuneau unii către alții, zicând: Ce este acest cuvânt? Că poruncește duhurilor necurate, cu stăpânire și cu putere, și ele ies.
\par 37 Și a ieșit vestea despre El în tot locul din împrejurimi.
\par 38 Și sculându-Se din sinagogă, a intrat în casa lui Simon. Iar soacra lui Simon era prinsă de friguri rele și L-au rugat pentru ea.
\par 39 Și El, plecându-Se asupra ei, a certat frigurile și frigurile au lăsat-o. Iar ea, îndată sculându-se, le slujea;
\par 40 Dar apunând soarele, toți câți aveau bolnavi de felurite boli îi aduceau la El; iar El, punându-Și mâinile pe fiecare dintre ei, îi făcea sănătoși.
\par 41 Din mulți ieșeau și demoni, care strigau și ziceau: Tu ești Fiul lui Dumnezeu. Dar El, certându-i, nu-i lăsa să vorbească acestea, că știau că El este Hristosul.
\par 42 Iar făcându-se ziuă, a ieșit și S-a dus într-un loc pustiu; și mulțimile Îl căutau și au venit până la El, și-L țineau ca să nu plece de la ei.
\par 43 Și El a zis către ei: Trebuie să binevestesc împărăția lui Dumnezeu și altor cetăți, fiindcă pentru aceasta am fost trimis.
\par 44 Și propovăduia în sinagogile Galileii.

\chapter{5}

\par 1 Pe când mulțimea Îl îmbulzea, ca să asculte cuvântul lui Dumnezeu, și El ședea lângă lacul Ghenizaret,
\par 2 A văzut două corăbii oprite lângă țărm, iar pescarii, coborând din ele, spălau mrejele.
\par 3 Și urcându-Se într-una din corăbii care era a lui Simon, l-a rugat s-o depărteze puțin de la uscat. Și șezând în corabie, învăța, din ea, mulțimile.
\par 4 Iar când a încetat de a vorbi, i-a zis lui Simon: Mână la adânc, și lăsați în jos mrejele voastre, ca să pescuiți.
\par 5 Și, răspunzând, Simon a zis: Învățătorule, toată noaptea ne-am trudit și nimic nu am prins, dar, după cuvântul Tău, voi arunca mrejele.
\par 6 Și făcând ei aceasta, au prins mulțime mare de pește, că li se rupeau mrejele.
\par 7 Și au făcut semn celor care erau în cealaltă corabie, să vină să le ajute. Și au venit și au umplut amândouă corăbiile, încât erau gata să se afunde,
\par 8 Iar Simon Petru, văzând aceasta, a căzut la genunchii lui Iisus, zicând: Ieși de la mine, Doamne, că sunt om păcătos.
\par 9 Căci spaima îl cuprinsese pe el și pe toți cei ce erau cu el, pentru pescuitul peștilor pe care îi prinseseră.
\par 10 Tot așa și pe Iacov și pe Ioan, fiii lui Zevedeu, care erau împreună cu Simon. Și a zis Iisus către Simon: Nu te teme; de acum înainte vei fi pescar de oameni.
\par 11 Și trăgând corăbiile la țărm, au lăsat totul și au mers după El.
\par 12 Și pe când erau într-una din cetăți, iată un om plin de lepră; văzând pe Iisus, a căzut cu fața la pământ și I s-a rugat zicând: Doamne, dacă voiești, poți să mă curățești.
\par 13 Și întinzând El mâna, S-a atins de lepros, zicând: Voiesc, fii curățat! Și îndată s-a dus lepra de pe el.
\par 14 Iar Iisus i-a poruncit să nu spună nimănui, ci, mergând, arată-te preotului și, pentru curățirea ta, du jertfa, precum a orânduit Moise, spre mărturie lor.
\par 15 Dar și mai mult străbătea vorba despre El și mulțimi multe se adunau, ca să asculte și să se vindece de bolile lor.
\par 16 Iar El Se retrăgea în locuri pustii și Se ruga.
\par 17 Și într-una din zile Iisus învăța și de față ședeau farisei și învățători ai Legii, veniți din toate satele Galileii, din Iudeea și din Ierusalim. Și puterea Domnului se arăta în tămăduiri.
\par 18 Și iată niște bărbați aduceau pe pat un om care era slăbănog și căutau să-l ducă înăuntru și să-l pună înaintea Lui;
\par 19 Dar negăsind pe unde să-l ducă, din pricina mulțimii, s-au suit pe acoperiș și, printre cărămizi, l-au lăsat cu patul în mijloc, înaintea lui Iisus.
\par 20 Și văzând credința lor, El le-a zis: Omule, iertate îți sunt păcatele tale.
\par 21 Iar fariseii și cărturarii au început să cârtească, zicând: Cine este Acesta care grăiește hule? Cine poate să ierte păcatele decât unul Dumnezeu?
\par 22 Iar Iisus, cunoscând gândurile lor, răspunzând a zis către ei: Ce cugetați în inimile voastre?
\par 23 Ce este mai ușor? A zice: Iertate sunt păcatele tale, sau a zice: Scoală și umblă?
\par 24 Iar ca să știți că Fiul Omului are pe pământ putere să ierte păcatele, a zis slăbănogului: Ție îți zic: Scoală-te, ia patul tău și mergi la casa ta.
\par 25 Și îndată, ridicându-se înaintea lor, luând patul pe care zăcuse, s-a dus la casa sa, slăvind pe Dumnezeu.
\par 26 Și uimire i-a cuprins pe toți și slăveau pe Dumnezeu și, plini de frică, ziceau: Am văzut astăzi lucruri minunate.
\par 27 Și după aceasta a ieșit și a văzut un vameș, cu numele Levi, care ședea la vamă, și i-a zis: Vino după Mine.
\par 28 Și, lăsând toate, el s-a sculat și a mers după El.
\par 29 Și I-a făcut Levi un ospăț mare în casa sa. Și era mulțime multă de vameși și de alții care ședeau cu ei la masă.
\par 30 Dar fariseii și cărturarii lor murmurau către ucenicii Lui, zicând: De ce mâncați și beți împreună cu vameșii și cu păcătoșii?
\par 31 Și Iisus, răspunzând, a zis către ei: N-au trebuință de doctor cei sănătoși, ci cei bolnavi.
\par 32 N-am venit să chem pe drepți, ci pe păcătoși la pocăință.
\par 33 Iar ei au zis către El: Ucenicii lui Ioan postesc adesea și fac rugăciuni, de asemenea și ai fariseilor, iar ai Tăi mănâncă și beau.
\par 34 Iar Iisus a zis către ei: Puteți, oare, să faceți pe fiii nunții să postească, cât timp Mirele este cu ei?
\par 35 Dar vor veni zile când Mirele se va lua de la ei; atunci vor posti în acele zile.
\par 36 Le-a spus lor și o pildă: Nimeni, rupând petic de la haină nouă, nu-l pune la haină veche, altfel rupe haina cea nouă, iar peticul luat din ea nu se potrivește la cea veche.
\par 37 Și nimeni nu pune vin nou în burdufuri vechi, altfel vinul nou va sparge burdufurile; și se varsă și vinul și se strică și burdufurile.
\par 38 Ci vinul nou trebuie pus în burdufuri noi și împreună se vor păstra.
\par 39 Și nimeni, bând vin vechi, nu voiește de cel nou, căci zice: E mai bun cel vechi.

\chapter{6}

\par 1 Într-o sâmbătă, a doua după Paști, Iisus mergea prin semănături și ucenicii Lui smulgeau spice, le frecau cu mâinile și mâncau.
\par 2 Dar unii dintre farisei au zis: De ce faceți ce nu se cade a face sâmbăta?
\par 3 Și Iisus, răspunzând, a zis către ei: Oare n-ați citit ce a făcut David, când a flămânzit el și cei ce erau cu el?
\par 4 Cum a intrat în casa lui Dumnezeu și a luat pâinile punerii înainte și a mâncat și a dat și însoțitorilor săi, din ele, pe care nu se cuvine să le mănânce decât numai preoții?
\par 5 Și le zicea: Fiul Omului este Domn și al sâmbetei.
\par 6 Iar în altă sâmbătă, a intrat El în sinagogă și învăța. Și era acolo un om a cărui mână dreaptă era uscată.
\par 7 Dar cărturarii și fariseii Îl pândeau de-l va vindeca sâmbăta, ca să-I găsească vină.
\par 8 Însă El știa gândurile lor și a zis omului care avea mâna uscată: Scoală-te și stai la mijloc. El s-a sculat și a stat.
\par 9 Atunci Iisus a zis către ei: Vă întreb pe voi, ce se cade sâmbăta: a face bine sau a face rău? A scăpa un suflet sau a-l pierde?
\par 10 Și privind împrejur pe toți aceștia, i-a zis: Întinde mâna ta. Iar el a făcut așa și mâna lui s-a făcut sănătoasă, ca și cealaltă.
\par 11 Ei însă s-au umplut de mânie și vorbeau unii cu alții ce să facă cu Iisus.
\par 12 Și în zilele acelea, Iisus a ieșit la munte ca să Se roage și a petrecut noaptea în rugăciune către Dumnezeu.
\par 13 Și când s-a făcut ziuă, a chemat la Sine pe ucenicii Săi și a ales dintre ei doisprezece, pe care i-a numit Apostoli.
\par 14 Pe Simon, căruia i-a zis Petru, și pe Andrei, fratele lui, și pe Iacov, și pe Ioan, și pe Filip, și pe Vartolomeu,
\par 15 Și pe Matei, și pe Toma, și pe Iacov al lui Alfeu și pe Simon numit Zilotul,
\par 16 Și pe Iuda al lui Iacov și pe Iuda Iscarioteanul, care s-a făcut trădător.
\par 17 Și coborând împreună cu ei, a stat în loc șes, El și mulțime multă de ucenici ai Săi și mulțime mare de popor din toată Iudeea, din Ierusalim și de pe țărmul Tirului și al Sidonului, care veniseră ca să-L asculte și să se vindece de bolile lor.
\par 18 Și cei chinuiți de duhuri necurate se vindecau.
\par 19 Și toată mulțimea căuta să se atingă de El că putere ieșea din El și-i vindeca pe toți.
\par 20 Și El, ridicându-Și ochii spre ucenicii Săi, zicea: Fericiți voi cei săraci, că a voastră este împărăția lui Dumnezeu.
\par 21 Fericiți voi care flămânziți acum, că vă veți sătura. Fericiți cei ce plângeți acum, că veți râde.
\par 22 Fericiți veți fi când oamenii vă vor urî pe voi și vă vor izgoni dintre ei, și vă vor batjocori și vor lepăda numele voastre ca rău din pricina Fiului Omului.
\par 23 Bucurați-vă în ziua aceea și vă veseliți, că, iată, plata voastră multă este în cer; pentru că tot așa făceau proorocilor părinții lor.
\par 24 Dar vai vouă bogaților, că vă luați pe pământ mângâierea voastră.
\par 25 Vai vouă celor ce sunteți sătui acum, că veți flămânzi. Vai vouă celor ce astăzi râdeți, că veți plânge și vă veți tângui.
\par 26 Vai vouă când toți oamenii vă vor vorbi de bine. Căci tot așa făceau proorocilor mincinoși părinții lor.
\par 27 Iar vouă celor ce ascultați vă spun: Iubiți pe vrăjmașii voștri, faceți bine celor ce vă urăsc pe voi;
\par 28 Binecuvântați pe cei ce vă blestemă, rugați-vă pentru cei ce vă fac necazuri.
\par 29 Celui ce te lovește peste obraz, întoarce-i și pe celălalt; pe cel ce-ți ia haina, nu-l împiedica să-ți ia și cămașa;
\par 30 Oricui îți cere, dă-i; și de la cel care ia lucrurile tale, nu cere înapoi.
\par 31 Și precum voiți să vă facă vouă oamenii, faceți-le și voi asemenea;
\par 32 Și dacă iubiți pe cei ce vă iubesc, ce răsplată puteți avea? Căci și păcătoșii iubesc pe cei ce îi iubesc pe ei.
\par 33 Și dacă faceți bine celor ce vă fac vouă bine, ce mulțumire puteți avea? Că și păcătoșii același lucru fac.
\par 34 Și dacă dați împrumut celor de la care nădăjduiți să luați înapoi, ce mulțumire puteți avea? Că și păcătoșii dau cu împrumut păcătoșilor, ca să primească înapoi întocmai.
\par 35 Ci iubiți pe vrăjmașii voștri și faceți bine și dați cu împrumut, fără să nădăjduiți nimic în schimb, și răsplata voastră va fi multă și veți fi fiii Celui Preaînalt, că El este bun cu cei nemulțumitori și răi.
\par 36 Fiți milostivi, precum și Tatăl vostru este milostiv.
\par 37 Nu judecați și nu veți fi judecați; nu osândiți și nu veți fi osândiți; iertați și veți fi iertați.
\par 38 Dați și se va da. Turna-vor în sânul vostru o măsură bună, îndesată, clătinată și cu vârf, căci cu ce măsură veți măsura, cu aceeași vi se va măsura.
\par 39 Și le-a spus și pildă: Poate orb pe orb să călăuzească? Nu vor cădea amândoi în groapă?
\par 40 Nu este ucenic mai presus decât învățătorul său; dar orice ucenic desăvârșit va fi ca învățătorul său.
\par 41 De ce vezi paiul din ochiul fratelui tău, iar bârna din ochiul tău nu o iei în seamă?
\par 42 Sau cum poți să zici fratelui tău: Frate, lasă să scot paiul din ochiul tău, nevăzând bârna care este în ochiul tău? Fățarnice, scoate mai întâi bârna din ochiul tău și atunci vei vedea să scoți paiul din ochiul fratelui tău.
\par 43 Căci nu este pom bun care să facă roade rele și, iarăși, nici pom rău care să facă roade bune.
\par 44 Căci fiecare pom se cunoaște după roadele lui. Că nu se adună smochine din mărăcini și nici nu se culeg struguri din spini.
\par 45 Omul bun, din vistieria cea bună a inimii sale, scoate cele bune, pe când omul rău, din vistieria cea rea a inimii lui, scoate cele rele. Căci din prisosul inimii grăiește gura lui.
\par 46 Și pentru ce Mă chemați: Doamne, Doamne, și nu faceți ce vă spun?
\par 47 Oricine vine la Mine și aude cuvintele Mele și le face, vă voi arăta cu cine se aseamănă:
\par 48 Asemenea este unui om care, zidindu-și casă, a săpat, a adâncit și i-a pus temelia pe piatră, și venind apele mari și puhoiul izbind în casa aceea, n-a putut s-o clintească, fiindcă era bine clădită pe piatră.
\par 49 Iar cel ce aude, dar nu face, este asemenea omului care și-a zidit casa pe pământ fără temelie, și izbind în ea puhoiul de ape, îndată a căzut și prăbușirea acelei case a fost mare.

\chapter{7}

\par 1 Și după ce a sfârșit toate aceste cuvinte ale Sale în auzul poporului, a intrat în Capernaum.
\par 2 Iar sluga unui sutaș, care era la el în cinste, fiind bolnavă era să moară.
\par 3 Și auzind despre Iisus, a trimis la El bătrâni ai iudeilor, rugându-L să vină și să vindece pe sluga lui.
\par 4 Iar ei, venind la Iisus, L-au rugat stăruitor, zicând: Vrednic este să-i faci lui aceasta,
\par 5 Căci iubește neamul nostru și el ne-a zidit sinagoga.
\par 6 Iar Iisus mergea cu ei. Și nefiind El acum departe de casă, a trimis la El prieteni, zicându-I: Doamne, nu Te osteni, că nu sunt vrednic ca să intri sub acoperământul meu.
\par 7 De aceea nici pe mine nu m-am socotit vrednic să vin la Tine. Ci spune cu cuvântul și se va vindeca sluga mea.
\par 8 Căci și eu sunt om pus sub stăpânire, având sub mine ostași, și zic acestuia: Du-te, și se duce, și altuia: Vino, și vine, și slugii mele: Fă aceasta, și face.
\par 9 Iar Iisus, auzind acestea, S-a minunat de el și, întorcându-Se, a zis mulțimii care venea după El: Zic vouă că nici în Israel n-am aflat atâta credință;
\par 10 Și întorcându-se cei trimiși acasă, au găsit sluga sănătoasă.
\par 11 Și după aceea, S-a dus într-o cetate numită Nain și cu El împreună mergeau ucenicii Lui și multă mulțime.
\par 12 Iar când S-a apropiat de poarta cetății, iată scoteau un mort, singurul copil al mamei sale, și ea era văduvă, și mulțime mare din cetate era cu ea.
\par 13 Și, văzând-o Domnul, I s-a făcut milă de ea și i-a zis: Nu plânge!
\par 14 Și apropiindu-Se, S-a atins de sicriu, iar cei ce-l duceau s-au oprit. Și a zis: Tinere, ție îți zic, scoală-te.
\par 15 Și s-a ridicat mortul și a început să vorbească, și l-a dat mamei lui.
\par 16 Și frică i-a cuprins pe toți și slăveau pe Dumnezeu, zicând: Prooroc mare s-a ridicat între noi și Dumnezeu a cercetat pe poporul Său.
\par 17 Și a ieșit cuvântul acesta despre El în toată Iudeea și în toată împrejurimea.
\par 18 Și au vestit lui Ioan ucenicii lui de toate acestea.
\par 19 Și chemând la sine pe doi dintre ucenicii săi, Ioan i-a trimis către Domnul, zicând: Tu ești Cel ce va să vină sau să așteptăm pe altul?
\par 20 Și ajungând la El, bărbații au zis: Ioan Botezătorul ne-a trimis la Tine, zicând: Tu ești Cel ce va să vină sau să așteptăm pe altul?
\par 21 Și în acel ceas El a vindecat pe mulți de boli și de răni și de duhuri rele și multor orbi le-a dăruit vederea.
\par 22 Și răspunzând, le-a zis: Mergeți și spuneți lui Ioan cele ce ați văzut și cele ce ați auzit: Orbii văd, șchiopii umblă, leproșii se curățesc, surzii aud, morții înviază și săracilor li se binevestește.
\par 23 Și fericit este acela care nu se va sminti întru Mine.
\par 24 Iar, după ce trimișii lui Ioan au plecat, El a început să vorbească mulțimilor despre Ioan: Ce ați ieșit să priviți, în pustie? Oare trestie clătinată de vânt?
\par 25 Dar ce ați ieșit să vedeți? Oare om îmbrăcat în haine moi? Iată, cei ce poartă haine scumpe și petrec în desfătare sunt în casele regilor.
\par 26 Dar ce-ați ieșit să vedeți? Oare prooroc? Da! Zic vouă: și mai mult decât un prooroc.
\par 27 Acesta este cel despre care s-a scris: "Iată trimit înaintea feței Tale pe îngerul Meu care va găti calea Ta, înaintea Ta".
\par 28 Zic vouă: Între cei născuți din femei, nimeni nu este mai mare decât Ioan; dar cel mai mic în împărăția lui Dumnezeu este mai mare decât el.
\par 29 Și tot poporul auzind, și vameșii s-au încredințat de dreptatea lui Dumnezeu, botezându-se cu botezul lui Ioan.
\par 30 Iar fariseii și învățătorii de lege au călcat voia lui Dumnezeu în ei înșiși, nebotezându-se de el.
\par 31 Cu cine voi asemăna pe oamenii acestui neam? Și cu cine sunt ei asemenea?
\par 32 Sunt asemenea copiilor care șed în piață și strigă unii către alții, zicând: V-am cântat din fluier și n-ați jucat; v-am cântat de jale și n-ați plâns.
\par 33 Căci a venit Ioan Botezătorul, nemâncând pâine și negustând vin, și ziceți: Are demon!
\par 34 A venit și Fiul Omului, mâncând și bând, și ziceți: Iată un om mâncăcios și băutor de vin, prieten al vameșilor și al păcătoșilor!
\par 35 Dar Înțelepciunea a fost găsită dreaptă de către toți fiii ei.
\par 36 Unul din farisei L-a rugat pe Iisus să mănânce cu el. Și intrând în casa fariseului, a șezut la masă.
\par 37 Și iată era în cetate o femeie păcătoasă și, aflând că șade la masă, în casa fariseului, a adus un alabastru cu mir.
\par 38 Și, stând la spate, lângă picioarele Lui, plângând, a început să ude cu lacrimi picioarele Lui, și cu părul capului ei le ștergea. Și săruta picioarele Lui și le ungea cu mir.
\par 39 Și văzând, fariseul, care-L chemase, a zis în sine: Acesta, de-ar fi prooroc, ar ști cine e și ce fel e femeia care se atinge de El, că este păcătoasă.
\par 40 Și răspunzând, Iisus a zis către el: Simone, am să-ți spun ceva. Învățătorule, spune, zise el.
\par 41 Un cămătar avea doi datornici. Unul era dator cu cinci sute de dinari, iar celălalt cu cincizeci.
\par 42 Dar, neavând ei cu ce să plătească, i-a iertat pe amândoi. Deci, care dintre ei îl va iubi mai mult?
\par 43 Simon, răspunzând, a zis: Socotesc că acela căruia i-a iertat mai mult. Iar El i-a zis: Drept ai judecat.
\par 44 Și întorcându-se către femeie, a zis lui Simon: Vezi pe femeia aceasta? Am intrat în casa ta și apă pe picioare nu Mi-ai dat; ea însă, cu lacrimi, Mi-a udat picioarele și le-a șters cu părul ei.
\par 45 Sărutare nu Mi-ai dat; ea însă de când am intrat, n-a încetat să-Mi sărute picioarele.
\par 46 Cu untdelemn capul Meu nu l-ai uns; ea însă cu mir Mi-a uns picioarele.
\par 47 De aceea îți zic: Iertate sunt păcatele ei cele multe, căci mult a iubit. Iar cui se iartă puțin, puțin iubește.
\par 48 Și a zis ei: Iertate îți sunt păcatele.
\par 49 Și au început cei ce ședeau împreună la masă să zică în sine: Cine este Acesta care iartă și păcatele?
\par 50 Iar către femeie a zis: Credința ta te-a mântuit; mergi în pace.

\chapter{8}

\par 1 Și după aceea Iisus umbla prin cetăți și prin sate, propovăduind și binevestind împărăția lui Dumnezeu, și cei doisprezece erau cu El;
\par 2 Și unele femei care fuseseră vindecate de duhuri rele și de boli: Maria, numită Magdalena, din care ieșiseră șapte demoni,
\par 3 Și Ioana, femeia lui Huza, un iconom al lui Irod, și Suzana și multe altele care le slujeau din avutul lor.
\par 4 Și adunându-se mulțime multă și venind de prin cetăți la El, a zis în pildă:
\par 5 Ieșit-a semănătorul să semene sămânța sa. Și semănând el, una a căzut lângă drum și a fost călcată cu picioarele și păsările cerului au mâncat-o.
\par 6 Și alta a căzut pe piatră, și, răsărind, s-a uscat, pentru că nu avea umezeală.
\par 7 Și alta a căzut între spini și spinii, crescând cu ea, au înăbușit-o.
\par 8 Și alta a căzut pe pământul cel bun și, crescând, a făcut rod însutit. Acestea zicând, striga: Cine are urechi de auzit să audă.
\par 9 Și ucenicii Lui Îl întrebau: Ce înseamnă pilda aceasta?
\par 10 El a zis: Vouă vă este dat să cunoașteți tainele împărăției lui Dumnezeu, iar celorlalți în pilde, ca, văzând, să nu vadă și, auzind, să nu înțeleagă.
\par 11 Iar pilda aceasta înseamnă: Sămânța este cuvântul lui Dumnezeu.
\par 12 Iar cea de lângă drum sunt cei care aud, apoi vine diavolul și ia cuvântul din inima lor, ca nu cumva, crezând, să se mântuiască.
\par 13 Iar cea de pe piatră sunt aceia care, auzind cuvântul îl primesc cu bucurie, dar aceștia nu au rădăcină; ei cred până la o vreme, iar la vreme de încercare se leapădă.
\par 14 Cea căzută între spini sunt cei ce aud cuvântul, dar umblând cu grijile și cu bogăția și cu plăcerile vieții, se înăbușă și nu rodesc.
\par 15 Iar cea de pe pământ bun sunt cei ce, cu inimă curată și bună, aud cuvântul, îl păstrează și rodesc întru răbdare.
\par 16 Nimeni, aprinzând făclia, n-o ascunde sub un vas, sau n-o pune sub pat, ci o așează în sfeșnic, pentru ca cei ce intră să vadă lumina.
\par 17 Căci nu este nimic ascuns, care să nu se dea pe față și nimic tainic, care să nu se cunoască și să nu vină la arătare.
\par 18 Luați seama deci cum auziți: Celui ce are i se va da; iar de la cel ce nu are, și ce i se pare că are se va lua de la el.
\par 19 Și au venit la El mama Lui și frații; dar nu puteau să se apropie de El din pricina mulțimii.
\par 20 Și I s-a vestit: Mama Ta și frații Tăi stau afară și voiesc să Te vadă.
\par 21 Iar El, răspunzând, a zis către ei: Mama mea și frații Mei sunt aceștia care ascultă cuvântul lui Dumnezeu și-l îndeplinesc.
\par 22 Și într-una din zile a intrat în corabie cu ucenicii Săi și a zis către ei: Să trecem de cealaltă parte a lacului. Și au plecat.
\par 23 Dar, pe când ei vâsleau, El a adormit. Și s-a lăsat pe lac o furtună de vânt, și corabia se umplea de apă și erau în primejdie.
\par 24 Și, apropiindu-se, L-au deșteptat, zicând: Învățătorule, Învățătorule, pierim. Iar El, sculându-Se, a certat vântul și valul apei și ele au încetat și s-a făcut liniște.
\par 25 Și le-a zis: Unde este credința voastră? Iar ei, temându-se, s-au mirat, zicând unii către alții: Oare cine este Acesta, că poruncește și vânturilor și apei, și-L ascultă?
\par 26 Și au ajuns cu corabia în ținutul Gerghesenilor, care este în fața Galileii.
\par 27 Și ieșind pe uscat, L-a întâmpinat un bărbat din cetate, care avea demon și care de multă vreme nu mai punea haină pe el și în casă nu mai locuia, ci prin morminte.
\par 28 Și văzând pe Iisus, strigând, a căzut înaintea Lui și cu glas mare a zis: Ce ai cu mine, Iisuse, Fiul lui Dumnezeu Celui Preaînalt? Rogu-Te, nu mă chinui.
\par 29 Căci poruncea duhului necurat să iasă din om, pentru că de mulți ani îl stăpânea, și era legat în lanțuri și în obezi, păzindu-l, dar el, sfărâmând legăturile, era mânat de demon, în pustie.
\par 30 Și l-a întrebat Iisus, zicând: Care-ți este numele? Iar el a zis: Legiune. Căci demoni mulți intraseră în el.
\par 31 Și-L rugau pe El să nu le poruncească să meargă în adânc.
\par 32 Și era acolo o turmă mare de porci, care pășteau pe munte. Și L-au rugat să le îngăduie să intre în ei; și le-a îngăduit.
\par 33 Și, ieșind demonii din om, au intrat în porci, iar turma s-a aruncat de pe țărm în lac și s-a înecat.
\par 34 Iar păzitorii văzând ce s-a întâmplat, au fugit și au vestit în cetate și prin sate.
\par 35 Și au ieșit să vadă ce s-a întâmplat și au venit la Iisus și au găsit pe omul din care ieșiseră demonii, îmbrăcat și întreg la minte, șezând jos, la picioarele lui Iisus și s-au înfricoșat.
\par 36 Și cei ce văzuseră le-au spus cum a fost izbăvit demonizatul.
\par 37 Și L-a rugat pe El toată mulțimea din ținutul Gerghesenilor să plece de la ei, căci erau cuprinși de frică mare. Iar El, intrând în corabie, S-a înapoiat.
\par 38 Iar bărbatul din care ieșiseră demonii Îl ruga să rămână cu El. Iisus însă i-a dat drumul zicând:
\par 39 Întoarce-te în casa ta și spune cât bine ți-a făcut ție Dumnezeu. Și a plecat, vestind în toată cetatea câte îi făcuse Iisus.
\par 40 Și când s-a întors Iisus, L-a primit mulțimea, căci toți Îl așteptau.
\par 41 Și iată a venit un bărbat, al cărui nume era Iair și care era mai-marele sinagogii. Și căzând la picioarele lui Iisus, Îl ruga să intre în casa Lui,
\par 42 Căci avea numai o fiică, ca de doisprezece ani, și ea era pe moarte. Și, pe când se ducea El, mulțimile Îl împresurau.
\par 43 Și o femeie, care de doisprezece ani avea scurgere de sânge și cheltuise cu doctorii toată averea ei, și de nici unul nu putuse să fie vindecată,
\par 44 Apropiindu-se pe la spate, s-a atins de poala hainei Lui și îndată s-a oprit curgerea sângelui ei.
\par 45 Și a zis Iisus: Cine este cel ce s-a atins de Mine? Dar toți tăgăduind, Petru și ceilalți care erau cu El, au zis: Învățătorule, mulțimile Te îmbulzesc și Te strâmtorează și Tu zici: Cine este cel ce s-a atins de mine?
\par 46 Iar Iisus a zis: S-a atins de Mine cineva. Căci am simțit o putere care a ieșit din Mine.
\par 47 Și, femeia, văzându-se vădită, a venit tremurând și, căzând înaintea Lui, a spus de față cu tot poporul din ce cauză s-a atins de El și cum s-a tămăduit îndată.
\par 48 Iar El i-a zis: Îndrăznește, fiică, credința ta te-a mântuit. Mergi în pace.
\par 49 Și încă vorbind El, a venit cineva de la mai-marele sinagogii, zicând: A murit fiica ta. Nu mai supăra pe Învățătorul.
\par 50 Dar Iisus, auzind, i-a răspuns: Nu te teme; crede numai și se va izbăvi.
\par 51 Și venind în casă n-a lăsat pe nimeni să intre cu El, decât numai pe Petru și pe Ioan și pe Iacov și pe tatăl copilei și pe mamă.
\par 52 Și toți plângeau și se tânguiau pentru ea. Iar El a zis: Nu plângeți; n-a murit, ci doarme.
\par 53 Și râdeau de El, știind că a murit.
\par 54 Iar El, scoțând pe toți afară și apucând-o de mână, a strigat, zicând: Copilă, scoală-te!
\par 55 Și duhul ei s-a întors și a înviat îndată; și a poruncit El să i se dea să mănânce.
\par 56 Și au rămas uimiți părinții ei. Iar El le-a poruncit să nu spună nimănui ce s-a întâmplat.

\chapter{9}

\par 1 Și chemând pe cei doisprezece ucenici ai Săi, le-a dat putere și stăpânire peste toți demonii și să vindece bolile.
\par 2 Și i-a trimis să propovăduiască împărăția lui Dumnezeu și să vindece pe cei bolnavi.
\par 3 Și a zis către ei: Să nu luați nimic pe drum, nici toiag, nici traistă, nici pâine, nici bani și nici să nu aveți câte două haine.
\par 4 Și în orice casă veți intra, acolo să rămâneți și de acolo să plecați.
\par 5 Și câți nu vă vor primi, ieșind din acea cetate scuturați praful de pe picioarele voastre, spre mărturie împotriva lor.
\par 6 Iar ei, plecând, mergeau prin sate binevestind și vindecând pretutindeni.
\par 7 Și a auzit Irod tetrarhul toate cele făcute și era nedumerit, că se zicea de către unii că Ioan s-a sculat din morți;
\par 8 Iar de unii că Ilie s-a arătat, iar de alții, că un prooroc dintre cei vechi a înviat.
\par 9 Iar Irod a zis: Lui Ioan eu i-am tăiat capul. Cine este dar Acesta despre care aud asemenea lucruri? Și căuta să-l vadă.
\par 10 Și, întorcându-se apostolii, I-au spus toate câte au făcut. Și, luându-i cu Sine, S-a dus de o parte într-un loc pustiu, aproape de cetatea numită Betsaida.
\par 11 Iar mulțimile, aflând, au mers după El și El, primindu-le, le vorbea despre împărăția lui Dumnezeu, iar pe cei care aveau trebuință de vindecare îi făcea sănătoși.
\par 12 Dar ziua a început să se plece spre seară. Și, venind la El, cei doisprezece I-au spus: Dă drumul mulțimii să se ducă prin satele și prin sătulețele dimprejur, ca să poposească și să-și găsească mâncare, că aici suntem în loc pustiu.
\par 13 Iar El a zis către ei: Dați-le voi să mănânce. Iar ei au zis: Nu avem mai mult de cinci pâini și doi pești, afară numai dacă, ducându-ne noi, vom cumpăra merinde pentru tot poporul acesta.
\par 14 Căci erau ca la cinci mii de bărbați. Dar El a zis către ucenicii Săi: Așezați-i jos, în cete de câte cincizeci de inși.
\par 15 Și au făcut așa și i-au așezat pe toți.
\par 16 Iar Iisus, luând cele cinci pâini și cei doi pești și privind la cer, le-a binecuvântat, a frânt și a dat ucenicilor, ca să pună mulțimii înainte.
\par 17 Și au mâncat și s-au săturat toți și au luat ceea ce le-a rămas, douăsprezece coșuri de fărâmituri.
\par 18 Și când Se ruga El singur, erau cu El ucenicii, și i-a întrebat, zicând: Cine zic mulțimile că sunt Eu?
\par 19 Iar ei, răspunzând, au zis: Ioan Botezătorul, iar alții Ilie, iar alții că a înviat un prooroc din cei vechi.
\par 20 Și El le-a zis: Dar voi cine ziceți că sunt Eu? Iar Petru, răspunzând, a zis: Hristosul lui Dumnezeu.
\par 21 Iar El, certându-i, le-a poruncit să nu spună nimănui aceasta,
\par 22 Zicând că Fiul Omului trebuie să pătimească multe și să fie defăimat de către bătrâni și de către arhierei și de către cărturari și să fie omorât, iar a treia zi să învieze.
\par 23 Și zicea către toți: Dacă voiește cineva să vină după Mine, să se lepede de sine, să-și ia crucea în fiecare zi și să-Mi urmeze Mie;
\par 24 Căci cine va voi să-și scape sufletul îl va pierde; iar cine-și va pierde sufletul pentru Mine, acela îl va mântui.
\par 25 Că ce folosește omului dacă va câștiga lumea toată, iar pe sine se va pierde sau se va păgubi?
\par 26 Căci de cel ce se va rușina de Mine și de cuvintele Mele, de acesta și Fiul Omului se va rușina, când va veni întru slava Sa și a Tatălui și a sfinților îngeri.
\par 27 Cu adevărat însă spun vouă: Sunt unii, dintre cei ce stau aici, care nu vor gusta moartea, până ce nu vor vedea împărăția lui Dumnezeu.
\par 28 Iar după cuvintele acestea, ca la opt zile, luând cu Sine pe Petru și pe Ioan și pe Iacov, S-a suit pe munte ca să Se roage.
\par 29 Și pe când se ruga El, chipul feței Sale s-a făcut altul și îmbrăcămintea Lui albă strălucind.
\par 30 Și iată doi bărbați vorbeau cu El, care erau Moise și Ilie,
\par 31 Și care, arătându-se întru slavă, vorbeau despre sfârșitul Lui, pe care avea să-l împlinească în Ierusalim.
\par 32 Iar Petru și cei ce erau cu el erau îngreuiați de somn; și deșteptându-se, au văzut slava Lui și pe cei doi bărbați stând cu El.
\par 33 Și când s-au despărțit ei de El, Petru a zis către Iisus: Învățătorule, bine este ca noi să fim aici și să facem trei colibe: una Ție, una lui Moise și una lui Ilie, neștiind ce spune.
\par 34 Și, pe când vorbea el acestea, s-a făcut un nor și i-a umbrit; și ei s-au spăimântat când au intrat în nor.
\par 35 Și glas s-a făcut din nor, zicând: Acesta este Fiul Meu cel ales, de acesta să ascultați!
\par 36 Și când a trecut glasul, S-a aflat Iisus singur. Și ei au tăcut și nimănui n-au spus nimic, în zilele acelea, din cele ce văzuseră.
\par 37 În ziua următoare, când s-au coborât din munte, L-a întâmpinat mulțime multă.
\par 38 Și iată un bărbat din mulțime a strigat, zicând: Învățătorule, rogu-mă Ție, caută spre fiul meu, că îl am numai pe el;
\par 39 Și iată un duh îl apucă și îndată strigă și-l zguduie cu spume și abia pleacă de la el, după ce l-a zdrobit.
\par 40 Și m-am rugat de ucenicii Tăi ca să-l alunge, și n-au putut.
\par 41 Iar Iisus, răspunzând, a zis: O, neam necredincios și îndărătnic! Până când voi fi cu voi și vă voi suferi? Adu aici pe fiul tău.
\par 42 Și, apropiindu-se el, demonul l-a aruncat la pământ și l-a zguduit. Iar Iisus a certat pe duhul cel necurat și a vindecat pe copil și l-a dat tatălui lui.
\par 43 Iar toți au rămas uimiți de mărirea lui Dumnezeu. Și mirându-se toți de toate câte făcea, a zis către ucenicii Săi:
\par 44 Puneți în urechile voastre cuvintele acestea: Căci Fiul Omului va fi dat în mâinile oamenilor.
\par 45 Iar ei nu înțelegeau cuvântul acesta, căci era ascuns pentru ei ca să nu-l priceapă și se temeau să-L întrebe despre acest cuvânt.
\par 46 Și a intrat gând în inima lor: Cine dintre ei ar fi mai mare?
\par 47 Iar Iisus, cunoscând cugetul inimii lor, a luat un copil, l-a pus lângă Sine,
\par 48 Și le-a zis: Oricine va primi pruncul acesta, în numele Meu, pe Mine Mă primește; iar oricine Mă va primi pe Mine, primește pe Cel ce M-a trimis pe Mine. Căci cel ce este mai mic între voi toți, acesta este mare.
\par 49 Iar Ioan, răspunzând, a zis: Învățătorule, am văzut pe unul care, în numele Tău, scoate demoni și l-am oprit, pentru că nu-Ți urmează împreună cu noi.
\par 50 Iar Iisus a zis către el: Nu-l opriți; căci cine nu este împotriva voastră este pentru voi.
\par 51 Și când s-au împlinit zilele înălțării Sale, El S-a hotărât să meargă la Ierusalim.
\par 52 Și a trimis vestitori înaintea Lui. Și ei, mergând, au intrat într-un sat de samarineni, ca să facă pregătiri pentru El.
\par 53 Dar ei nu L-au primit, pentru că El se îndrepta spre Ierusalim.
\par 54 Și văzând aceasta, ucenicii Iacov și Ioan I-au zis: Doamne, vrei să zicem să se coboare foc din cer și să-i mistuie, cum a făcut și Ilie?
\par 55 Iar El, întorcându-Se, i-a certat și le-a zis: Nu știți, oare, fiii cărui duh sunteți? Căci Fiul Omului n-a venit ca să piardă sufletele oamenilor, ci ca să le mântuiască.
\par 56 Și s-au dus în alt sat.
\par 57 Și pe când mergeau ei pe cale, zis-a unul către El: Te voi însoți, oriunde Te vei duce.
\par 58 Și i-a zis Iisus: Vulpile au vizuini și păsările cerului cuiburi; dar Fiul Omului n-are unde să-Și plece capul.
\par 59 Și a zis către altul: urmează-Mi. Iar el a zis: Doamne, dă-mi voie întâi să merg să îngrop pe tatăl meu.
\par 60 Iar El i-a zis: Lasă morții să-și îngroape morții lor, iar tu mergi de vestește împărăția lui Dumnezeu.
\par 61 Dar altul a zis: Îți voi urma, Doamne, dar întâi îngăduie-mi ca să rânduiesc cele din casa mea.
\par 62 Iar Iisus a zis către el: Nimeni care pune mâna pe plug și se uită îndărăt nu este potrivit pentru împărăția lui Dumnezeu.

\chapter{10}

\par 1 Iar după acestea, Domnul a ales alți șaptezeci (și doi) și i-a trimis câte doi înaintea feței Sale, în fiecare cetate și loc, unde Însuși avea să vină.
\par 2 Și zicea către ei: Secerișul este mult, dar lucrătorii sunt puțini; rugați deci pe Domnul secerișului, ca să scoată lucrători la secerișul Său.
\par 3 Mergeți; iată, Eu vă trimit ca pe niște miei în mijlocul lupilor.
\par 4 Nu purtați pungă, nici traistă, nici încălțăminte; și pe nimeni să nu salutați pe cale.
\par 5 Iar în orice casă veți intra, întâi ziceți: Pace casei acesteia.
\par 6 Și de va fi acolo un fiu al păcii, pacea voastră se va odihni peste el, iar de nu, se va întoarce la voi.
\par 7 Și în această casă rămâneți, mâncând și bând cele ce vă vor da, căci vrednic este lucrătorul de plata sa. Nu vă mutați din casă în casă.
\par 8 Și în orice cetate veți intra și vă vor primi, mâncați cele ce vă vor pune înainte.
\par 9 Și vindecați pe bolnavii din ea și ziceți-le: S-a apropiat de voi împărăția lui Dumnezeu.
\par 10 Și în orice cetate veți intra și nu vă vor primi, ieșind în piețele ei, ziceți:
\par 11 Și praful care s-a lipit de picioarele noastre din cetatea noastră vi-l scuturăm vouă. Dar aceasta să știți, că s-a apropiat împărăția lui Dumnezeu.
\par 12 Zic vouă: Că mai ușor va fi Sodomei în ziua aceea, decât cetății aceleia.
\par 13 Vai ție, Horazine! Vai ție, Betsaido! Căci dacă în Tir și în Sidon s-ar fi făcut minunile care s-au făcut la voi, de mult s-ar fi pocăit, stând în sac și în cenușă.
\par 14 Dar Tirului și Sidonului mai ușor le va fi la judecată, decât vouă.
\par 15 Și tu, Capernaume, nu ai fost înălțat, oare, până la cer? Până la iad vei fi coborât!
\par 16 Cel ce vă ascultă pe voi pe Mine Mă ascultă, și cel ce se leapădă de voi se leapădă de Mine; iar cine se leapădă de Mine se leapădă de Cel ce M-a trimis pe Mine.
\par 17 Și s-au întors cei șaptezeci (și doi) cu bucurie, zicând: Doamne, și demonii ni se supun în numele Tău.
\par 18 Și le-a zis: Am văzut pe satana ca un fulger căzând din cer.
\par 19 Iată, v-am dat putere să călcați peste șerpi și peste scorpii, și peste toată puterea vrăjmașului, și nimic nu vă va vătăma.
\par 20 Dar nu vă bucurați de aceasta, că duhurile vi se pleacă, ci vă bucurați că numele voastre sunt scrise în ceruri.
\par 21 În acesta ceas, El S-a bucurat în Duhul Sfânt și a zis: Te slăvesc pe Tine, Părinte, Doamne al cerului și al pământului, că ai ascuns acestea de cei înțelepți și de cei pricepuți și le-ai descoperit pruncilor. Așa, Părinte, căci așa a fost înaintea Ta, bunăvoința Ta.
\par 22 Toate Mi-au fost date de către Tatăl Meu și nimeni nu cunoaște cine este Fiul, decât numai Tatăl, și cine este Tatăl, decât numai Fiul și căruia voiește Fiul să-i descopere.
\par 23 Și întorcându-Se către ucenici, de o parte a zis: Fericiți sunt ochii care văd cele ce vedeți voi!
\par 24 Căci zic vouă: Mulți prooroci și regi au voit să vadă ceea ce vedeți voi, dar n-au văzut, și să audă ceea ce auziți, dar n-au auzit.
\par 25 Și iată, un învățător de lege s-a ridicat, ispitindu-L și zicând: Învățătorule, ce să fac ca să moștenesc viața de veci?
\par 26 Iar Iisus a zis către el: Ce este scris în Lege? Cum citești?
\par 27 Iar el, răspunzând, a zis: Să iubești pe Domnul Dumnezeul tău din toată inima ta și din tot sufletul tău și din toată puterea ta și din tot cugetul tău, iar pe aproapele tău ca pe tine însuți.
\par 28 Iar El i-a zis: Drept ai răspuns, fă aceasta și vei trăi.
\par 29 Dar el, voind să se îndrepteze pe sine, a zis către Iisus: Și cine este aproapele meu?
\par 30 Iar Iisus, răspunzând, a zis: Un om cobora de la Ierusalim la Ierihon, și a căzut între tâlhari, care, după ce l-au dezbrăcat și l-au rănit, au plecat, lăsându-l aproape mort.
\par 31 Din întâmplare un preot cobora pe calea aceea și, văzându-l, a trecut pe alături.
\par 32 De asemenea și un levit, ajungând în acel loc și văzând, a trecut pe alături.
\par 33 Iar un samarinean, mergând pe cale, a venit la el și, văzându-l, i s-a făcut milă,
\par 34 Și, apropiindu-se, i-a legat rănile, turnând pe ele untdelemn și vin, și, punându-l pe dobitocul său, l-a dus la o casă de oaspeți și a purtat grijă de el.
\par 35 Iar a doua zi, scoțând doi dinari i-a dat gazdei și i-a zis: Ai grijă de el și, ce vei mai cheltui, eu, când mă voi întoarce, îți voi da.
\par 36 Care din acești trei ți se pare că a fost aproapele celui căzut între tâlhari?
\par 37 Iar el a zis: Cel care a făcut milă cu el. Și Iisus i-a zis: Mergi și fă și tu asemenea.
\par 38 Și pe când mergeau ei, El a intrat într-un sat, iar o femeie, cu numele Marta, L-a primit în casa ei.
\par 39 Și ea avea o soră ce se numea Maria, care, așezându-se la picioarele Domnului, asculta cuvântul Lui.
\par 40 Iar Marta se silea cu multă slujire și, apropiindu-se, a zis: Doamne, au nu socotești că sora mea m-a lăsat singură să slujesc? Spune-i deci să-mi ajute.
\par 41 Și, răspunzând, Domnul i-a zis: Marto, Marto, te îngrijești și pentru multe te silești;
\par 42 Dar un lucru trebuie: căci Maria partea bună și-a ales, care nu se va lua de la ea.

\chapter{11}

\par 1 Și pe când Se ruga Iisus într-un loc, când a încetat, unul dintre ucenicii Lui I-a zis: Doamne, învață-ne să ne rugăm, cum a învățat și Ioan pe ucenicii lui.
\par 2 Și le-a zis: Când vă rugați, ziceți: Tatăl nostru, Care ești în ceruri, sfințească-se numele Tău. Vie împărăția Ta. Facă-se voia Ta, precum în cer așa și pe pământ.
\par 3 Pâinea noastră cea spre ființă, dă-ne-o nouă în fiecare zi.
\par 4 Și ne iartă nouă păcatele noastre, căci și noi înșine iertăm tuturor celor ce ne greșesc nouă. Și nu ne duce pe noi în ispită, ci ne izbăvește de cel rău.
\par 5 Și a zis către ei: Cine dintre voi, având un prieten și se va duce la el în miez de noapte și-i va zice: Prietene, împrumută-mi trei pâini,
\par 6 Că a venit, din cale, un prieten la mine și n-am ce să-i pun înainte,
\par 7 Iar acela, răspunzând dinăuntru, să-i zică: Nu mă da de osteneală. Acum ușa e încuiată și copiii mei sunt în pat cu mine. Nu pot să mă scol să-ți dau.
\par 8 Zic vouă: Chiar dacă, sculându-se, nu i-ar da pentru că-i este prieten, dar, pentru îndrăzneala lui, sculându-se, îi va da cât îi trebuie.
\par 9 Și Eu zic vouă: Cereți și vi se va da; căutați și veți afla; bateți și vi se va deschide.
\par 10 Că oricine cere ia; și cel ce caută găsește, și celui ce bate i se va deschide.
\par 11 Și care tată dintre voi, dacă îi va cere fiul pâine, oare, îi va da piatră? Sau dacă îi va cere pește, oare îi va da, în loc de pește, șarpe?
\par 12 Sau dacă-i va cere un ou, îi va da scorpie?
\par 13 Deci dacă voi, răi fiind, știți să dați fiilor voștri daruri bune, cu cât mai mult Tatăl vostru Cel din ceruri va da Duh Sfânt celor care îl cer de la El!
\par 14 Și a scos un demon, și acela era mut. Și când a ieșit demonul, mutul a vorbit, iar mulțimile s-au minunat.
\par 15 Iar unii dintre ei au zis: Cu Beelzebul, căpetenia demonilor, scoate pe demoni.
\par 16 Iar alții, ispitindu-L, cereau de la El semn din cer.
\par 17 Dar El, cunoscând gândurile lor, le-a zis: Orice împărăție, dezbinându-se în sine, se pustiește și casă peste casă cade.
\par 18 Și dacă satana s-a dezbinat în sine, cum va mai sta împărăția lui? Fiindcă ziceți că Eu scot pe demoni cu Beelzebul.
\par 19 Iar dacă Eu scot demonii cu Beelzebul, fiii voștri cu cine îi scot? De aceea ei vă vor fi judecători.
\par 20 Iar dacă Eu, cu degetul lui Dumnezeu, scot pe demoni iată a ajuns la voi împărăția lui Dumnezeu.
\par 21 Când cel tare și înarmat fiind își păzește curtea, avuțiile lui sunt în pace.
\par 22 Dar când unul mai tare decât el vine asupra lui și-l înfrânge, îi ia toate armele pe care se bizuia, iar prăzile de la el le împarte.
\par 23 Cel ce nu este cu Mine este împotriva Mea; și cel ce nu adună cu Mine risipește.
\par 24 Când duhul cel necurat iese din om, umblă prin locuri fără apă, căutând odihnă, și, negăsind, zice: Mă voi întoarce la casa mea, de unde am ieșit.
\par 25 Și, venind, o află măturată și împodobită.
\par 26 Atunci merge și ia cu el alte șapte duhuri mai rele decât el și, intrând, locuiește acolo; și se fac cele de pe urmă ale omului aceluia mai rele decât cele dintâi.
\par 27 Și când zicea El acestea, o femeie din mulțime, ridicând glasul, I-a zis: Fericit este pântecele care Te-a purtat și fericiți sunt sânii pe care i-ai supt!
\par 28 Iar El a zis: Așa este, dar fericiți sunt cei ce ascultă cuvântul lui Dumnezeu și-l păzesc.
\par 29 Iar îngrămădindu-se mulțimile, El a început a zice: Neamul acesta este un neam viclean; cere semn dar semn nu i se va da decât semnul proorocului Iona.
\par 30 Căci precum a fost Iona un semn pentru Niniviteni așa va fi și Fiul Omului semn pentru acest neam.
\par 31 Regina de la miazăzi se va ridica la judecată cu bărbații neamului acestuia și-i va osândi, pentru că a venit de la marginile pământului, ca să asculte înțelepciunea lui Solomon; și, iată, mai mult decât Solomon este aici.
\par 32 Bărbații din Ninive se vor scula la judecată cu neamul acesta și-l vor osândi, pentru că s-au pocăit la propovăduirea lui Iona; și, iată, mai mult decât Iona este aici.
\par 33 Nimeni, aprinzând făclie, nu o pune în loc ascuns, nici sub obroc, ci în sfeșnic, ca aceia care intră să vadă lumina.
\par 34 Luminătorul trupului este ochiul tău. Când ochiul tău este curat, atunci tot trupul tău e luminat; dar când ochiul tău e rău, atunci și trupul tău e întunecat.
\par 35 Ia seama deci ca lumina din tine să nu fie întuneric.
\par 36 Așadar, dacă tot trupul tău e luminat, neavând nici o parte întunecată, luminat va fi în întregime, ca și când te luminează făclia cu strălucirea ei.
\par 37 Și pe când Iisus vorbea, un fariseu Îl ruga să prânzească la el; și, intrând, a șezut la masă.
\par 38 Iar fariseul s-a mirat văzând că El nu S-a spălat înainte de masă.
\par 39 Și Domnul a zis către el: Acum, voi fariseilor, curățiți partea din afară a paharului și a blidului, dar lăuntrul vostru este plin de răpire și de viclenie.
\par 40 Nebunilor! Oare, cel ce a făcut partea din afară n-a făcut și partea dinăuntru?
\par 41 Dați mai întâi milostenie cele ce sunt înlăuntrul vostru și, iată, toate vă vor fi curate.
\par 42 Dar vai vouă, fariseilor! Că dați zeciuială din izmă și din untariță și din toate legumele și lăsați la o parte dreptatea și iubirea de Dumnezeu; pe acestea se cuvenea să le faceți și pe acelea să nu le lăsați.
\par 43 Vai vouă, fariseilor! Că iubiți scaunele din față în sinagogi și în închinăciunile din piețe.
\par 44 Vai vouă, cărturarilor și fariseilor fățarnici! Că sunteți ca mormintele ce nu se văd, și oamenii, care umblă peste ele, nu le știu.
\par 45 Și răspunzând, unul dintre învățătorii de Lege I-a zis: Învățătorule, acestea zicând, ne mustri și pe noi!
\par 46 Iar El a zis: Vai și vouă, învățătorilor de Lege! Că împovărați pe oameni cu sarcini anevoie de purtat, iar voi nu atingeți sarcinile nici cel puțin cu un deget.
\par 47 Vai vouă! Că zidiți mormintele proorocilor pe care părinții voștri i-au ucis.
\par 48 Așadar, mărturisiți și încuviințați faptele părinților voștri, pentru că ei i-au ucis, iar voi le clădiți mormintele.
\par 49 De aceea și înțelepciunea lui Dumnezeu a zis: "Voi trimite la ei prooroci și apostoli și dintre ei vor ucide și vor prigoni";
\par 50 Ca să se ceară de la neamul acesta sângele tuturor proorocilor, care s-a vărsat de la facerea lumii,
\par 51 De la sângele lui Abel până la sângele lui Zaharia, care a pierit între altar și templu. Adevărat vă spun: Se va cere de la neamul acesta.
\par 52 Vai vouă, învățătorilor de Lege! Că ați luat cheia cunoștinței; voi înșivă n-ați intrat, iar pe cei ce voiau să intre i-ați împiedecat.
\par 53 Iar ieșind El de acolo, cărturarii și fariseii au început să-L urască groaznic și să-L silească să vorbească despre multe,
\par 54 Pândindu-L și căutând să prindă ceva din gura Lui, ca să-I găsească vină.

\chapter{12}

\par 1 Și în același timp, adunându-se mulțime mii și mii de oameni, încât se călcau unii pe alții, Iisus a început să vorbească întâi către ucenicii Săi: Feriți-vă de aluatul fariseilor, care este fățărnicia.
\par 2 Că nimic nu este acoperit care să nu se descopere și nimic ascuns care să nu se cunoască.
\par 3 De aceea, câte ați spus la întuneric se vor auzi la lumină; și ceea ce ați vorbit la ureche, în odăi, se va vesti de pe acoperișuri.
\par 4 Dar vă spun vouă, prietenii Mei: Nu vă temeți de cei care ucid trupul și după aceasta n-au ce să mai facă.
\par 5 Vă voi arăta însă de cine să vă temeți: Temeți-vă de acela care, după ce a ucis, are putere să arunce în gheena; da, vă zic vouă, de acela să vă temeți.
\par 6 Nu se vând oare cinci vrăbii cu doi bani? Și nici una dintre ele nu este uitată înaintea lui Dumnezeu.
\par 7 Ci și perii capului vostru, toți sunt numărați. Nu vă temeți; voi sunteți mai de preț decât multe vrăbii.
\par 8 Și zic vouă: Oricine va mărturisi pentru Mine înaintea oamenilor, și Fiul Omului va mărturisi pentru el înaintea îngerilor lui Dumnezeu.
\par 9 Iar cel ce se va lepăda de Mine înaintea oamenilor, lepădat va fi înaintea îngerilor lui Dumnezeu.
\par 10 Oricui va spune vreun cuvânt împotriva Fiului Omului, i se va ierta; dar celui ce va huli împotriva Duhului Sfânt, nu i se va ierta.
\par 11 Iar când vă vor duce în sinagogi și la dregători și la stăpâniri nu vă îngrijiți cum sau ce veți răspunde, sau ce veți zice,
\par 12 Că Duhul Sfânt vă va învăța chiar în ceasul acela, ce trebuie să spuneți.
\par 13 Zis-a Lui cineva din mulțime: Învățătorule, zi fratelui meu să împartă cu mine moștenirea.
\par 14 Iar El i-a zis: Omule, cine M-a pus pe Mine judecător sau împărțitor peste voi?
\par 15 Și a zis către ei: Vedeți și păziți-vă de toată lăcomia, căci viața cuiva nu stă în prisosul avuțiilor sale.
\par 16 Și le-a spus lor această pildă, zicând: Unui om bogat i-a rodit din belșug țarina.
\par 17 Și el cugeta în sine, zicând: Ce voi face, că n-am unde să adun roadele mele?
\par 18 Și a zis: Aceasta voi face: Voi strica jitnițele mele și mai mari le voi zidi și voi strânge acolo tot grâul și bunătățile mele;
\par 19 Și voi zice sufletului meu: Suflete, ai multe bunătăți strânse pentru mulți ani; odihnește-te, mănâncă, bea, veselește-te.
\par 20 Iar Dumnezeu i-a zis: Nebune! În această noapte vor cere de la tine sufletul tău. Și cele ce ai pregătit ale cui vor fi?
\par 21 Așa se întâmplă cu cel ce-și adună comori sieși și nu se îmbogățește în Dumnezeu.
\par 22 Și a zis către ucenicii Săi: De aceea zic vouă: Nu vă îngrijiți pentru viața voastră ce veți mânca, nici pentru trupul vostru cu ce vă veți îmbrăca.
\par 23 Viața este mai mult decât hrana și trupul mai mult decât îmbrăcămintea.
\par 24 Priviți la corbi, că nici nu seamănă, nici nu seceră; ei n-au cămară, nici jitniță, și Dumnezeu îi hrănește. Cu cât mai de preț sunteți voi decât păsările!
\par 25 Și cine dintre voi, îngrijindu-se, poate să adauge staturii sale un cot?
\par 26 Deci dacă nu puteți să faceți nici cel mai mic lucru, de ce vă îngrijiți de celelalte?
\par 27 Priviți la crini cum cresc: Nu torc, nici nu țes. Și zic vouă că nici Solomon, în toată mărirea lui, nu s-a îmbrăcat ca unul dintre aceștia.
\par 28 Iar dacă iarba care este azi pe câmp, iar mâine se aruncă în cuptor, Dumnezeu așa o îmbracă, cu cât mai mult pe voi, puțin credincioșilor!
\par 29 Și voi să nu căutați ce veți mânca sau ce veți bea și nu fiți îngrijorați.
\par 30 Căci toate acestea păgânii lumii le caută; dar Tatăl vostru știe că aveți nevoie de acestea;
\par 31 Căutați mai întâi împărăția Lui. Și toate acestea se vor adăuga vouă.
\par 32 Nu te teme, turmă mică, pentru că Tatăl vostru a binevoit să vă dea vouă împărăția.
\par 33 Vindeți averile voastre și dați milostenie; faceți-vă pungi care nu se învechesc, comoară neîmpuținată în ceruri, unde fur nu se apropie, nici molie nu o strică.
\par 34 Căci unde este comoara voastră, acolo este inima voastră.
\par 35 Să fie mijloacele voastre încinse și făcliile voastre aprinse.
\par 36 Și voi fiți asemenea oamenilor care așteaptă pe stăpânul lor când se întoarce de la nuntă, ca, venind, și bătând, îndată să-i deschidă.
\par 37 Fericite sunt slugile acelea pe care, venind, stăpânul le va afla priveghind. Adevărat zic vouă că se va încinge și le va pune la masă și, apropiindu-se le va sluji.
\par 38 Fie că va veni la straja a doua, fie că va veni la straja a treia, și le va găsi așa, fericite sunt acelea.
\par 39 Iar aceasta să știți că, de ar ști stăpânul casei în care ceas vine furul, ar veghea și n-ar lăsa să i se spargă casa.
\par 40 Deci și voi fiți gata, că în ceasul în care nu gândiți Fiul Omului va veni.
\par 41 Și a zis Petru: Doamne, către noi spui pilda aceasta sau și către toți?
\par 42 Și a zis Domnul: Cine este iconomul credincios și înțelept pe care stăpânul îl va pune peste slugile sale, ca să le dea, la vreme, partea lor de grâu?
\par 43 Fericită este sluga aceea pe care, venind stăpânul, o va găsi făcând așa.
\par 44 Adevărat vă spun că o va pune peste toate avuțiile sale.
\par 45 Iar de va zice sluga aceea în inima sa: Stăpânul meu zăbovește să vină, și va începe să bată pe slugi și pe slujnice, și să mănânce, și să bea și să se îmbete,
\par 46 Veni-va stăpânul slugii aceleia în ziua în care ea nu se așteaptă și în ceasul în care ea nu știe și o va tăia în două, iar partea ei va pune-o cu cei necredincioși.
\par 47 Iar sluga aceea care a știut voia stăpânului și nu s-a pregătit, nici n-a făcut după voia lui, va fi bătută mult.
\par 48 Și cea care n-a știut, dar a făcut lucruri vrednice de bătaie, va fi bătută puțin. Și oricui i s-a dat mult, mult i se va cere, și cui i s-a încredințat mult, mai mult i se va cere.
\par 49 Foc am venit să arunc pe pământ și cât aș vrea să fie acum aprins!
\par 50 Și cu botez am a Mă boteza, și câtă nerăbdare am până ce se va îndeplini!
\par 51 Vi se pare că am venit să dau pace pe pământ? Vă spun că nu, ci dezbinare.
\par 52 Căci de acum înainte cinci dintr-o casă vor fi dezbinați: trei împotriva a doi și doi împotriva a trei.
\par 53 Dezbinați vor fi: tatăl împotriva fiului și fiul împotriva tatălui, mama împotriva fiicei și fiica împotriva mamei, soacra împotriva nurorii sale și nora împotriva soacrei.
\par 54 Și zicea mulțimilor: Când vedeți un nor ridicându-se dinspre apus, îndată ziceți că vine ploaie mare; și așa este.
\par 55 Iar când suflă vântul de la miazăzi, ziceți că va fi arșiță, și așa este.
\par 56 Fățarnicilor! Fața pământului și a cerului știți să o deosebiți, dar vremea aceasta cum de nu o deosebiți?
\par 57 De ce, dar, de la voi înșivă nu judecați ce este drept?
\par 58 Și când mergi cu pârâșul tău la dregător, dă-ți silința să te scapi de el pe cale, ca nu cumva să te târască la judecător, și judecătorul să te dea în mâna temnicerului, iar temnicerul să te arunce în temniță.
\par 59 Zic ție: Nu vei ieși de acolo, până ce nu vei plăti și cel din urmă ban.

\chapter{13}

\par 1 Și erau de față în acel timp unii care-I vesteau despre galileienii al căror sânge Pilat l-a amestecat cu jertfele lor.
\par 2 Și El, răspunzând, le-a zis: Credeți, oare, că acești galileieni au fost ei mai păcătoși decât toți galileienii, fiindcă au suferit aceasta?
\par 3 Nu! zic vouă; dar dacă nu vă veți pocăi, toți veți pieri la fel.
\par 4 Sau acei optsprezece inși, peste care s-a surpat turnul în Siloam și i-a ucis, gândiți, oare, că ei au fost mai păcătoși decât toți oamenii care locuiau în Ierusalim?
\par 5 Nu! zic vouă; dar de nu vă veți pocăi, toți veți pieri la fel.
\par 6 Și le-a spus pilda aceasta: Cineva avea un smochin, sădit în via sa și a venit să caute rod în el, dar n-a găsit.
\par 7 Și a zis către vier: Iată trei ani sunt de când vin și caut rod în smochinul acesta și nu găsesc. Taie-l; de ce să ocupe locul în zadar?
\par 8 Iar el, răspunzând, a zis: Doamne, lasă-l și anul acesta, până ce îl voi săpa împrejur și voi pune gunoi.
\par 9 Poate va face rod în viitor; iar de nu, îl vei tăia.
\par 10 Și învăța Iisus într-una din sinagogi sâmbăta.
\par 11 Și iată o femeie care avea de optsprezece ani un duh de neputință și care era gârbovă, de nu putea să se ridice în sus nicidecum;
\par 12 Iar Iisus, văzând-o, a chemat-o și i-a zis: Femeie, ești dezlegată de neputința ta.
\par 13 Și Și-a pus mâinile asupra ei, și ea îndată s-a îndreptat și slăvea pe Dumnezeu.
\par 14 Iar mai-marele sinagogii, mâniindu-se că Iisus a vindecat-o sâmbăta, răspunzând, zicea mulțimii: Șase zile sunt în care trebuie să se lucreze; venind deci într-acestea, vindecați-vă, dar nu în ziua sâmbetei!
\par 15 Iar Domnul i-a răspuns și a zis: Fățarnicilor! Fiecare dintre voi nu dezleagă, oare, sâmbăta boul său, sau asinul de la iesle, și nu-l duce să-l adape?
\par 16 Dar aceasta, fiică a lui Avraam fiind, pe care a legat-o satana, iată de optsprezece ani, nu se cuvenea, oare, să fie dezlegată de legătura aceasta, în ziua sâmbetei?
\par 17 Și zicând El acestea, s-au rușinat toți care erau împotriva Lui, și toată mulțimea se bucura de faptele strălucite săvârșite de El.
\par 18 Deci zicea: Cu ce este asemenea împărăția lui Dumnezeu și cu ce o voi asemăna?
\par 19 Asemenea este grăuntelui de muștar pe care, luându-l, un om l-a aruncat în grădina sa, și a crescut și s-a făcut copac, iar păsările cerului s-au sălășluit în ramurile lui.
\par 20 Și iarăși a zis: Cu ce voi asemăna împărăția lui Dumnezeu?
\par 21 Asemenea este aluatului pe care, luându-l, femeia l-a ascuns în trei măsuri de făină, până ce s-a dospit totul.
\par 22 Și mergea El prin cetăți și prin sate, învățând și călătorind spre Ierusalim.
\par 23 Și I-a zis cineva: Doamne, puțini sunt, oare, cei ce se mântuiesc? Iar El le-a zis:
\par 24 Siliți-vă să intrați prin poarta cea strâmtă, că mulți, zic vouă, vor căuta să intre și nu vor putea.
\par 25 După ce se va scula stăpânul casei și va încuia ușa și veți începe să stați afară și să bateți la ușă, zicând: Doamne, deschide-ne! - și el, răspunzând, vă va zice: Nu vă știu de unde sunteți,
\par 26 Atunci voi veți începe să ziceți: Am mâncat înaintea ta și am băut și în piețele noastre ai învățat.
\par 27 Și el vă va zice: Vă spun: Nu știu de unde sunteți. Depărtați-vă de la mine toți lucrătorii nedreptății.
\par 28 Acolo va fi plângerea și scrâșnirea dinților, când veți vedea pe Avraam și pe Isaac și pe Iacov și pe toți proorocii în Împărăția lui Dumnezeu, iar pe voi aruncați afară.
\par 29 Și vor veni alții de la răsărit și de la apus, de la miazănoapte și de la miazăzi și vor ședea la masă în împărăția lui Dumnezeu.
\par 30 Și iată, sunt unii de pe urmă care vor fi întâi, și sunt alții întâi care vor fi pe urmă.
\par 31 În ceasul acela au venit la El unii din farisei, zicându-I: Ieși și du-Te de aici, că Irod vrea să Te ucidă.
\par 32 Și El le-a zis: Mergând, spuneți vulpii acesteia: Iată, alung demoni și fac vindecări, astăzi și mâine, iar a treia zi voi sfârși.
\par 33 Însă și astăzi și mâine și în ziua următoare merg, fiindcă nu este cu putință să piară prooroc afară din Ierusalim.
\par 34 Ierusalime, Ierusalime, care omori pe prooroci și ucizi cu pietre pe cei trimiși la tine, de câte ori am voit să adun pe fiii tăi, cum adună pasărea puii săi sub aripi, dar n-ați voit.
\par 35 Iată vi se lasă casa voastră pustie, că adevărat grăiesc vouă. Nu Mă veți mai vedea până ce va veni vremea când veți zice: Binecuvântat este Cel ce vine întru numele Domnului!

\chapter{14}

\par 1 Și când a intrat El în casa unuia dintre căpeteniile fariseilor sâmbăta, ca să mănânce, și ei Îl pândeau,
\par 2 Iată un om bolnav de idropică era înaintea Lui.
\par 3 Și, răspunzând, Iisus a zis către învățătorii de lege și către farisei, spunând: Cuvine-se a vindeca sâmbăta ori nu?
\par 4 Ei însă au tăcut. Și luându-l, l-a vindecat și i-a dat drumul.
\par 5 Și către ei a zis: Care dintre voi, de-i cădea fiul sau boul în fântână nu-l va scoate îndată în ziua sâmbetei?
\par 6 Și n-au putut să-i răspundă la acestea.
\par 7 Și luând seama cum își alegeau la masă cele dintâi locuri, a spus celor chemați o pildă, zicând între ei:
\par 8 Când vei fi chemat de cineva la nuntă, nu te așeza în locul cel dintâi, ca nu cumva să fie chemat de el altul mai de cinste decât tine.
\par 9 Și venind cel care te-a chemat pe tine și pe el, îți va zice: Dă acestuia locul. Și atunci, cu rușine, te vei duce să te așezi pe locul cel mai de pe urmă.
\par 10 Ci, când vei fi chemat, mergând așează-te în cel din urmă loc, ca atunci când va veni cel ce te-a chemat, el să-ți zică: Prietene, mută-te mai sus. Atunci vei avea cinstea în fața tuturor celor care vor ședea împreună cu tine.
\par 11 Căci, oricine se înalță pe sine se va smeri, iar cel ce se smerește pe sine se va înălța.
\par 12 Zis-a și celui ce-L chemase: Când faci prânz sau cină, nu chema pe prietenii tăi, nici pe frații tăi, nici pe rudele tale, nici vecinii bogați, ca nu cumva să te cheme și ei, la rândul lor, pe tine, și să-ți fie ca răsplată.
\par 13 Ci, când faci un ospăț, cheamă pe săraci, pe neputincioși, pe șchiopi, pe orbi,
\par 14 Și fericit vei fi că nu pot să-ți răsplătească. Căci ți se va răsplăti la învierea drepților.
\par 15 Și auzind acestea, unul dintre cei ce ședeau cu El la masă I-a zis: Fericit este cel ce va prânzi în împărăția lui Dumnezeu!
\par 16 Iar El i-a zis: Un om oarecare a făcut cină mare și a chemat pe mulți;
\par 17 Și a trimis la ceasul cinei pe sluga sa ca să spună celor chemați: Veniți, că iată toate sunt gata.
\par 18 Și au început unul câte unul, să-și ceară iertare. Cel dintâi i-a zis: Țarină am cumpărat și am nevoie să ies ca s-o văd; te rog iartă-mă.
\par 19 Și altul a zis: Cinci perechi de boi am cumpărat și mă duc să-i încerc; te rog iartă-mă.
\par 20 Al treilea a zis: Femeie mi-am luat și de aceea nu pot veni.
\par 21 Și întorcându-se, sluga a spus stăpânului său acestea. Atunci, mâniindu-se, stăpânul casei a zis: Ieși îndată în piețele și ulițele cetății, și pe săraci, și pe neputincioși, și pe orbi, și pe șchiopi adu-i aici.
\par 22 Și a zis sluga: Doamne, s-a făcut precum ai poruncit și tot mai este loc.
\par 23 Și a zis stăpânul către slugă: Ieși la drumuri și la garduri și silește să intre, ca să mi se umple casa,
\par 24 Căci zic vouă: Nici unul din bărbații aceia care au fost chemați nu va gusta din cina mea.
\par 25 Și mergeau cu El mulțimi multe; și întorcându-Se, a zis către ele:
\par 26 Dacă vine cineva la Mine și nu urăște pe tatăl său și pe mamă și pe femeie și pe copii și pe frați și pe surori, chiar și sufletul său însuși, nu poate să fie ucenicul Meu.
\par 27 Și cel ce nu-și poartă crucea sa și nu vine după Mine nu poate să fie ucenicul Meu.
\par 28 Că cine dintre voi vrând să zidească un turn nu stă mai întâi și-și face socoteala cheltuielii, dacă are cu ce să-l isprăvească?
\par 29 Ca nu cumva, punându-i temelia și neputând să-l termine, toți cei care vor vedea să înceapă a-l lua în râs,
\par 30 Zicând: Acest om a început să zidească, dar n-a putut isprăvi.
\par 31 Sau care rege, plecând să se bată în război cu alt rege, nu va sta întâi să se sfătuiască, dacă va putea să întâmpine cu zece mii pe cel care vine împotriva lui cu douăzeci de mii?
\par 32 Iar de nu, încă fiind el departe, îi trimite solie și se roagă de pace.
\par 33 Așadar oricine dintre voi care nu se leapădă de tot ce are nu poate să fie ucenicul Meu.
\par 34 Bună este sarea, dar dacă și sarea se va strica, cu ce va fi dreasă?
\par 35 Nici în pământ, nici în gunoi, nu este de folos, ci o aruncă afară. Cine are urechi de auzit să audă.

\chapter{15}

\par 1 Și se apropiau de El toți vameșii și păcătoșii, ca să-L asculte.
\par 2 Și fariseii și cărturarii cârteau, zicând: Acesta primește la Sine pe păcătoși și mănâncă cu ei.
\par 3 Și a zis către ei pilda aceasta, spunând:
\par 4 Care om dintre voi, având o sută de oi și pierzând din ele una, nu lasă pe cele nouăzeci și nouă în pustie și se duce după cea pierdută, până ce o găsește?
\par 5 Și găsind-o, o pune pe umerii săi, bucurându-se;
\par 6 Și sosind acasă, cheamă prietenii și vecinii, zicându-le: Bucurați-vă cu mine, că am găsit oaia cea pierdută.
\par 7 Zic vouă: Că așa și în cer va fi mai multă bucurie pentru un păcătos care se pocăiește, decât pentru nouăzeci și nouă de drepți, care n-au nevoie de pocăință.
\par 8 Sau care femeie, având zece drahme, dacă pierde o drahmă, nu aprinde lumina și nu mătură casa și nu caută cu grijă până ce o găsește?
\par 9 Și găsind-o, cheamă prietenele și vecinele sale, spunându-le: Bucurați-vă cu mine, căci am găsit drahma pe care o pierdusem.
\par 10 Zic vouă, așa se face bucurie îngerilor lui Dumnezeu pentru un păcătos care se pocăiește.
\par 11 Și a zis: Un om avea doi fii.
\par 12 Și a zis cel mai tânăr dintre ei tatălui său: Tată, dă-mi partea ce mi se cuvine din avere. Și el le-a împărțit averea.
\par 13 Și nu după multe zile, adunând toate, fiul cel mai tânăr s-a dus într-o țară depărtată și acolo și-a risipit averea, trăind în desfrânări.
\par 14 Și după ce a cheltuit totul, s-a făcut foamete mare în țara aceea, și el a început să ducă lipsă.
\par 15 Și ducându-se, s-a alipit el de unul din locuitorii acelei țări, și acesta l-a trimis la țarinile sale să păzească porcii.
\par 16 Și dorea să-și sature pântecele din roșcovele pe care le mâncau porcii, însă nimeni nu-i dădea.
\par 17 Dar, venindu-și în sine, a zis: Câți argați ai tatălui meu sunt îndestulați de pâine, iar eu pier aici de foame!
\par 18 Sculându-mă, mă voi duce la tatăl meu și-i voi spune: Tată, am greșit la cer și înaintea ta;
\par 19 Nu mai sunt vrednic să mă numesc fiul tău. Fă-mă ca pe unul din argații tăi.
\par 20 Și, sculându-se, a venit la tatăl său. Și încă departe fiind el, l-a văzut tatăl său și i s-a făcut milă și, alergând, a căzut pe grumazul lui și l-a sărutat.
\par 21 Și i-a zis fiul: Tată, am greșit la cer și înaintea ta și nu mai sunt vrednic să mă numesc fiul tău.
\par 22 Și a zis tatăl către slugile sale: Aduceți degrabă haina lui cea dintâi și-l îmbrăcați și dați inel în mâna lui și încălțăminte în picioarele lui;
\par 23 Și aduceți vițelul cel îngrășat și-l înjunghiați și, mâncând, să ne veselim;
\par 24 Căci acest fiu al meu mort era și a înviat, pierdut era și s-a aflat. Și au început să se veselească.
\par 25 Iar fiul cel mare era la țarină. Și când a venit și s-a apropiat de casă, a auzit cântece și jocuri.
\par 26 Și, chemând la sine pe una dintre slugi, a întrebat ce înseamnă acestea.
\par 27 Iar ea i-a răspuns: Fratele tău a venit, și tatăl tău a înjunghiat vițelul cel îngrășat, pentru că l-a primit sănătos.
\par 28 Și el s-a mâniat și nu voia să intre; dar tatăl lui, ieșind, îl ruga.
\par 29 Însă el, răspunzând, a zis tatălui său: Iată, atâția ani îți slujesc și niciodată n-am călcat porunca ta. Și mie niciodată nu mi-ai dat un ied, ca să mă veselesc cu prietenii mei.
\par 30 Dar când a venit acest fiu al tău, care ți-a mâncat averea cu desfrânatele, ai înjunghiat pentru el vițelul cel îngrășat.
\par 31 Tatăl însă i-a zis: Fiule, tu totdeauna ești cu mine și toate ale mele ale tale sunt.
\par 32 Trebuia însă să ne veselim și să ne bucurăm, căci fratele tău acesta mort era și a înviat, pierdut era și s-a aflat.

\chapter{16}

\par 1 Și zicea și către ucenicii Săi: Era un om bogat care avea un iconom și acesta a fost pârât lui că-i risipește avuțiile.
\par 2 Și chemându-l, i-a zis: Ce este aceasta ce aud despre tine? Dă-mi socoteala de iconomia ta, căci nu mai poți să fii iconom.
\par 3 Iar iconomul a zis în sine: Ce voi face că stăpânul meu ia iconomia de la mine? Să sap, nu pot; să cerșesc, mi-e rușine.
\par 4 Știu ce voi face, ca să mă primească în casele lor, când voi fi scos din iconomie.
\par 5 Și chemând la sine, unul câte unul, pe datornicii stăpânului său, a zis celui dintâi: Cât ești dator stăpânului meu?
\par 6 Iar el a zis: O sută de măsuri de untdelemn. Iconomul i-a zis: Ia-ți zapisul și, șezând, scrie degrabă cincizeci.
\par 7 După aceea a zis altuia: Dar tu, cât ești dator? El i-a spus: O sută de măsuri de grâu. Zis-a iconomul: Ia-ți zapisul și scrie optzeci.
\par 8 Și a lăudat stăpânul pe iconomul cel nedrept, căci a lucrat înțelepțește. Căci fiii veacului acestuia sunt mai înțelepți în neamul lor decât fiii luminii.
\par 9 Și Eu zic vouă: Faceți-vă prieteni cu bogăția nedreaptă, ca atunci, când veți părăsi viața, să vă primească ei în corturile cele veșnice.
\par 10 Cel ce este credincios în foarte puțin și în mult este credincios; și cel ce e nedrept în foarte puțin și în mult este nedrept.
\par 11 Deci dacă n-ați fost credincioși în bogăția nedreaptă, cine vă va încredința pe cea adevărată?
\par 12 Și dacă în ceea ce este străin nu ați fost credincioși, cine vă va da ce este al vostru?
\par 13 Nici o slugă nu poate să slujească la doi stăpâni. Fiindcă sau pe unul îl va urî și pe celălalt îl va iubi, sau de unul se va ține și pe celălalt îl va disprețui. Nu puteți să slujiți lui Dumnezeu și lui mamona.
\par 14 Toate acestea le auzeau și fariseii, care erau iubitori de argint și-L luau în bătaie de joc.
\par 15 Și El le-a zis: Voi sunteți cei ce vă faceți pe voi drepți înaintea oamenilor, dar Dumnezeu cunoaște inimile voastre; căci ceea ce la oameni este înalt, urâciune este înaintea lui Dumnezeu.
\par 16 Legea și proorocii au fost până la Ioan; de atunci împărăția lui Dumnezeu se binevestește și fiecare se silește spre ea.
\par 17 Dar mai lesne e să treacă cerul și pământul, decât să cadă din Lege un corn de literă.
\par 18 Oricine-și lasă femeia sa și ia pe alta săvârșește adulter; și cel ce ia pe cea lăsată de bărbat săvârșește adulter.
\par 19 Era un om bogat care se îmbrăca în porfiră și în vison, veselindu-se în toate zilele în chip strălucit.
\par 20 Iar un sărac, anume Lazăr, zăcea înaintea porții lui, plin de bube,
\par 21 Poftind să se sature din cele ce cădeau de la masa bogatului; dar și câinii venind, lingeau bubele lui.
\par 22 Și a murit săracul și a fost dus de către îngeri în sânul lui Avraam. A murit și bogatul și a fost înmormântat.
\par 23 Și în iad, ridicându-și ochii, fiind în chinuri, el a văzut de departe pe Avraam și pe Lazăr în sânul lui.
\par 24 Și el, strigând, a zis: Părinte Avraame, fie-ți milă de mine și trimite pe Lazăr să-și ude vârful degetului în apă și să-mi răcorească limba, căci mă chinuiesc în această văpaie.
\par 25 Dar Avraam a zis: Fiule, adu-ți aminte că ai primit cele bune ale tale în viața ta, și Lazăr, asemenea, pe cele rele; iar acum aici el se mângâie, iar tu te chinuiești.
\par 26 Și peste toate acestea, între noi și voi s-a întărit prăpastie mare, ca cei care voiesc să treacă de aici la voi să nu poată, nici cei de acolo să treacă la noi.
\par 27 Iar el a zis: Rogu-te, dar, părinte, să-l trimiți în casa tatălui meu,
\par 28 Căci am cinci frați, să le spună lor acestea, ca să nu vină și ei în acest loc de chin.
\par 29 Și i-a zis Avraam: Au pe Moise și pe prooroci; să asculte de ei.
\par 30 Iar el a zis: Nu, părinte Avraam, ci, dacă cineva dintre morți se va duce la ei, se vor pocăi.
\par 31 Și i-a zis Avraam: Dacă nu ascultă de Moise și de prooroci, nu vor crede nici dacă ar învia cineva dintre morți.

\chapter{17}

\par 1 Și a zis către ucenicii Săi: Cu neputință este să nu vină smintelile, dar vai aceluia prin care ele vin!
\par 2 Mai de folos i-ar fi dacă i s-ar lega de gât o piatră de moară și ar fi aruncat în mare, decât să smintească pe unul din aceștia mici.
\par 3 Luați aminte la voi înșivă. De-ți va greși fratele tău, dojenește-l și dacă se va pocăi, iartă-l.
\par 4 Și chiar dacă îți va greși de șapte ori într-o zi și de șapte ori se va întoarce către tine, zicând: Mă căiesc, iartă-l.
\par 5 Și au zis apostolii către Domnul: Sporește-ne credința.
\par 6 Iar Domnul a zis: De ați avea credință cât un grăunte de muștar, ați zice acestui sicomor: Dezrădăcinează-te și te sădește în mare, și vă va asculta.
\par 7 Cine dintre voi, având o slugă la arat sau la păscut turme, îi va zice când se întoarce din țarină: Vino îndată și șezi la masă?
\par 8 Oare, nu-i va zice: Pregătește-mi ca să cinez și, încingându-te, slujește-mi, până ce voi mânca și voi bea și după aceea vei mânca și vei bea și tu?
\par 9 Va mulțumi, oare, slugii că a făcut cele poruncite? Cred că nu.
\par 10 Așa și voi, când veți face toate cele poruncite vouă, să ziceți: Suntem slugi netrebnice, pentru că am făcut ceea ce eram datori să facem.
\par 11 Iar pe când Iisus mergea spre Ierusalim și trecea prin mijlocul Samariei și al Galileii,
\par 12 Intrând într-un sat, L-au întâmpinat zece leproși care stăteau departe,
\par 13 Și care au ridicat glasul și au zis: Iisuse, Învățătorule, fie-Ți milă de noi!
\par 14 Și văzându-i, El le-a zis: Duceți-vă și vă arătați preoților. Dar, pe când ei se duceau, s-au curățit.
\par 15 Iar unul dintre ei, văzând că s-a vindecat, s-a întors cu glas mare slăvind pe Dumnezeu.
\par 16 Și a căzut cu fața la pământ la picioarele lui Iisus, mulțumindu-I. Și acela era samarinean.
\par 17 Și răspunzând, Iisus a zis: Au nu zece s-au curățit? Dar cei nouă unde sunt?
\par 18 Nu s-a găsit să se întoarcă să dea slavă lui Dumnezeu decât numai acesta, care este de alt neam?
\par 19 Și i-a zis: Scoală-te și du-te; credința ta te-a mântuit.
\par 20 Și fiind întrebat de farisei când va veni împărăția lui Dumnezeu, le-a răspuns și a zis: Împărăția lui Dumnezeu nu va veni în chip văzut.
\par 21 Și nici nu vor zice: Iat-o aici sau acolo. Căci, iată, împărăția lui Dumnezeu este înăuntrul vostru.
\par 22 Zis-a către ucenici: Veni-vor zile când veți dori să vedeți una din zilele Fiului Omului, și nu veți vedea.
\par 23 Și vor zice vouă: Iată este acolo; iată, aici; nu vă duceți și nu vă luați după ei.
\par 24 Căci după cum fulgerul, fulgerând dintr-o parte de sub cer, luminează până la cealaltă parte de sub cer, așa va fi și Fiul Omului în ziua Sa.
\par 25 Dar mai întâi El trebuie să sufere multe și să fie lepădat de neamul acesta.
\par 26 Și precum a fost în zilele lui Noe, tot așa va fi și în zilele Fiului Omului:
\par 27 Mâncau, beau, se însurau, se măritau până în ziua când a intrat Noe în corabie și a venit potopul și i-a nimicit pe toți.
\par 28 Tot așa precum a fost în zilele lui Lot: mâncau, beau, cumpărau, vindeau, sădeau, și zideau,
\par 29 Iar în ziua în care a ieșit Lot din Sodoma a plouat din cer foc și pucioasă și i-a nimicit pe toți,
\par 30 La fel va fi în ziua în care se va arăta Fiul Omului.
\par 31 În ziua aceea, cel care va fi pe acoperișul casei, și lucrurile lui în casă, să nu se coboare ca să le ia; de asemenea, cel ce va fi în țarină să nu se întoarcă înapoi.
\par 32 Aduceți-vă aminte de femeia lui Lot.
\par 33 Cine va căuta să-și scape sufletul, îl va pierde; iar cine îl va pierde, acela îl va dobândi.
\par 34 Zic vouă: În noaptea aceea vor fi doi într-un pat; unul va fi luat, iar celălalt va fi lăsat.
\par 35 Două vor măcina împreună; una va fi luată și alta va fi lăsată.
\par 36 Doi vor fi în ogor; unul se va lua altul se va lăsa.
\par 37 Și răspunzând, ucenicii I-au zis: Unde, Doamne? Iar El le-a zis: Unde va fi stârvul, acolo se vor aduna vulturii.

\chapter{18}

\par 1 Și le spunea o pildă cum trebuie să se roage totdeauna și să nu-și piardă nădejdea,
\par 2 Zicând: Într-o cetate era un judecător care de Dumnezeu nu se temea și de om nu se rușina.
\par 3 Și era, în cetatea aceea, o văduvă, care venea la el, zicând: Fă-mi dreptate față de potrivnicul meu.
\par 4 Și un timp n-a voit, dar după acestea a zis întru sine: Deși de Dumnezeu nu mă tem și de om nu mă rușinez,
\par 5 Totuși, fiindcă văduva aceasta îmi face supărare, îi voi face dreptate, ca să nu vină mereu să mă supere.
\par 6 Și a zis Domnul: Auziți ce spune judecătorul cel nedrept?
\par 7 Dar Dumnezeu, oare, nu va face dreptate aleșilor Săi care strigă către El ziua și noaptea și pentru care El rabdă îndelung?
\par 8 Zic vouă că le va face dreptate în curând. Dar Fiul Omului, când va veni, va găsi, oare, credință pe pământ?
\par 9 Către unii care se credeau că sunt drepți și priveau cu dispreț pe ceilalți, a zis pilda aceasta:
\par 10 Doi oameni s-au suit la templu, ca să se roage: unul fariseu și celălalt vameș.
\par 11 Fariseul, stând, așa se ruga în sine: Dumnezeule, Îți mulțumesc că nu sunt ca ceilalți oameni, răpitori, nedrepți, adulteri, sau ca și acest vameș.
\par 12 Postesc de două ori pe săptămână, dau zeciuială din toate câte câștig.
\par 13 Iar vameșul, departe stând, nu voia nici ochii să-și ridice către cer, ci-și bătea pieptul, zicând: Dumnezeule, fii milostiv mie, păcătosului.
\par 14 Zic vouă că acesta s-a coborât mai îndreptat la casa sa, decât acela. Fiindcă oricine se înalță pe sine se va smeri, iar cel ce se smerește pe sine se va înălța.
\par 15 Și aduceau la El și pruncii, ca să Se atingă de ei. Iar ucenicii, văzând, îi certau.
\par 16 Iar Iisus i-a chemat la Sine, zicând: Lăsați copii să vină la Mine și nu-i opriți, căci împărăția lui Dumnezeu este a unora ca aceștia.
\par 17 Adevărat grăiesc vouă: Cine nu va primi împărăția lui Dumnezeu ca un prunc nu va intra în ea.
\par 18 Și L-a întrebat un dregător, zicând: Bunule Învățător, ce să fac ca să moștenesc viața de veci?
\par 19 Iar Iisus i-a zis: Pentru ce Mă numești bun? Nimeni nu este bun, decât unul Dumnezeu.
\par 20 Știi poruncile: Să nu săvârșești adulter, să nu ucizi, să nu furi, să nu mărturisești strâmb, cinstește pe tatăl tău și pe mama ta.
\par 21 Iar el a zis: Toate acestea le-am păzit din tinerețile mele.
\par 22 Auzind Iisus i-a zis: Încă una îți lipsește: Vinde toate câte ai și le împarte săracilor și vei avea comoară în ceruri; și vino de urmează Mie.
\par 23 Iar el, auzind acestea, s-a întristat, căci era foarte bogat.
\par 24 Și văzându-l întristat, Iisus a zis: Cât de greu vor intra cei ce au averi în împărăția lui Dumnezeu!
\par 25 Că mai lesne este a trece cămila prin urechile acului decât să intre bogatul în împărăția lui Dumnezeu.
\par 26 Zis-au cei ce ascultau: Și cine poate să se mântuiască?
\par 27 Iar El a zis: Cele ce sunt cu neputință la oameni sunt cu putință la Dumnezeu.
\par 28 Iar Petru a zis: Iată, noi, lăsând toate ale noastre, am urmat Ție.
\par 29 Și El le-a zis: Adevărat grăiesc vouă: Nu este nici unul care a lăsat casă, sau femeie, sau frați, sau părinți, sau copii, pentru împărăția lui Dumnezeu,
\par 30 Și să nu ia cu mult mai mult în vremea aceasta, iar în veacul ce va să vină, viață veșnică.
\par 31 Și luând la Sine pe cei doisprezece, a zis către ei: Iată ne suim la Ierusalim și se vor împlini toate cele scrise prin prooroci despre Fiul Omului.
\par 32 Căci va fi dat păgânilor și va fi batjocorit și va fi ocărât și scuipat.
\par 33 Și, după ce Îl vor biciui, Îl vor ucide; iar a treia zi va învia.
\par 34 Și ei n-au înțeles nimic din acestea, căci cuvântul acesta era ascuns pentru ei și nu înțelegeau cele spuse.
\par 35 Și când S-a apropiat Iisus de Ierihon, un orb ședea lângă drum, cerșind.
\par 36 Și, auzind el mulțimea care trecea, întreba ce e aceasta.
\par 37 Și i-au spus că trece Iisus Nazarineanul.
\par 38 Și el a strigat, zicând: Iisuse, Fiul lui David, fie-Ți milă de mine!
\par 39 Și cei care mergeau înainte îl certau ca să tacă, iar el cu mult mai mult striga: Fiule al lui David, fie-Ți milă de mine!
\par 40 Și oprindu-Se, Iisus a poruncit să-l aducă la El; și apropiindu-se, l-a întrebat:
\par 41 Ce voiești să-ți fac? Iar el a zis: Doamne, să văd!
\par 42 Și Iisus i-a zis: Vezi! Credința ta te-a mântuit.
\par 43 Și îndată a văzut și mergea după El, slăvind pe Dumnezeu. Și tot poporul, care văzuse, a dat laudă lui Dumnezeu.

\chapter{19}

\par 1 Și intrând, trecea prin Ierihon.
\par 2 Și iată un bărbat, cu numele Zaheu, și acesta era mai-marele vameșilor și era bogat.
\par 3 Și căuta să vadă cine este Iisus, dar nu putea de mulțime, pentru că era mic de statură.
\par 4 Și alergând el înainte, s-a suit într-un sicomor, ca să-L vadă, căci pe acolo avea să treacă.
\par 5 Și când a sosit la locul acela, Iisus, privind în sus, a zis către el: Zahee, coboară-te degrabă, căci astăzi în casa ta trebuie să rămân.
\par 6 Și a coborât degrabă și L-a primit, bucurându-se.
\par 7 Și văzând, toți murmurau, zicând că a intrat să găzduiască la un om păcătos.
\par 8 Iar Zaheu, stând, a zis către Domnul: Iată, jumătate din averea mea, Doamne, o dau săracilor și, dacă am năpăstuit pe cineva cu ceva, întorc împătrit.
\par 9 Și a zis către el Iisus: Astăzi s-a făcut mântuire casei acesteia, căci și acesta este fiu al lui Avraam.
\par 10 Căci Fiul Omului a venit să caute și să mântuiască pe cel pierdut.
\par 11 Și ascultând ei acestea, Iisus, adăugând, le-a spus o pildă, fiindcă El era aproape de Ierusalim, iar ei credeau că împărăția lui Dumnezeu se va arăta îndată.
\par 12 Deci a zis: Un om de neam mare s-a dus într-o țară îndepărtată, ca să-și ia domnie și să se întoarcă.
\par 13 Și chemând zece slugi ale sale, le-a dat zece mine și a zis către ele: Neguțătoriți cu ele până ce voi veni!
\par 14 Dar cetățenii lui îl urau și au trimis solie în urma lui, zicând: Nu voim ca acesta să domnească peste noi.
\par 15 Și când s-a întors el, după ce luase domnia, a zis să fie chemate slugile acelea, cărora le dăduse banii, ca să știe cine ce a neguțătorit.
\par 16 Și a venit cea dintâi, zicând: Doamne, mina ta a adus câștig zece mine.
\par 17 Și i-a zis stăpânul: Bine slugă bună, fiindcă întru puțin ai fost credincioasă, să ai stăpânire peste zece cetăți.
\par 18 Și a venit a doua, zicând: Mina ta, stăpâne, a mai adus cinci mine.
\par 19 Iar el a zis și acesteia: Să ai și tu stăpânire peste cinci cetăți.
\par 20 A venit și cealaltă, zicând: Doamne, iată mina ta, pe care am păstrat-o într-un ștergar,
\par 21 Că mă temeam de tine, pentru că ești om aspru: iei ce nu ai pus și seceri ce n-ai semănat.
\par 22 Zis-a lui stăpânul: Din cuvintele tale te voi judeca, slugă vicleană. Ai știut că sunt om aspru: iau ce nu am pus și secer ce nu am semănat;
\par 23 Pentru ce deci n-ai dat banul meu schimbătorilor de bani? Și eu, venind, l-aș fi luat cu dobândă.
\par 24 Și a zis celor ce stăteau de față: Luați de la el mina și dați-o celui ce are zece mine.
\par 25 Și ei au zis lui: Doamne, acela are zece mine.
\par 26 Zic vouă: Că oricui are i se va da, iar de la cel ce nu are și ceea ce are i se va lua.
\par 27 Iar pe acei vrăjmași ai mei, care n-au voit să domnesc peste ei, aduceți-i aici și tăiați-i în fața mea.
\par 28 Și zicând acestea, mergea înainte, suindu-Se la Ierusalim.
\par 29 Iar când S-a apropiat de Betfaghe și de Betania, către muntele care se zice Muntele Măslinilor, a trimis pe doi dintre ucenici,
\par 30 Zicând: Mergeți în satul dinaintea voastră și, intrând în el, veți găsi un mânz legat pe care nimeni dintre oameni n-a șezut vreodată. Și, dezlegându-l, aduceți-l.
\par 31 Și dacă vă va întreba cineva: Pentru ce-l dezlegați?, veți zice așa: Pentru că Domnul are trebuință de el.
\par 32 Și, plecând, cei trimiși au găsit precum le-a spus.
\par 33 Pe când aceștia dezlegau mânzul, au zis stăpânii lui către ei: De ce dezlegați mânzul?
\par 34 Iar ei au răspuns: Pentru că are trebuință de el Domnul.
\par 35 Și i-au adus la Iisus și, aruncându-și hainele lor pe mânz, l-au ajutat pe Iisus să urce pe el.
\par 36 Iar pe când mergea El, așterneau hainele lor pe cale.
\par 37 Și apropiindu-se de poalele Muntelui Măslinilor, toată mulțimea ucenicilor, bucurându-se, a început să laude pe Dumnezeu, cu glas tare, pentru toate minunile pe care le văzuse,
\par 38 Zicând: Binecuvântat este Împăratul care vine întru numele Domnului! Pace în cer și slavă întru cei de sus.
\par 39 Dar unii farisei din mulțime au zis către El: Învățătorule, ceartă-ți ucenicii.
\par 40 Și El, răspunzând, a zis: Zic vouă: Dacă vor tăcea aceștia, pietrele vor striga.
\par 41 Și când S-a apropiat, văzând cetatea, a plâns pentru ea, zicând:
\par 42 Dacă ai fi cunoscut și tu, în ziua aceasta, cele ce sunt spre pacea ta! Dar acum ascunse sunt de ochii tăi.
\par 43 Căci vor veni zile peste tine, când dușmanii tăi vor săpa șanț în jurul tău și te vor împresura și te vor strâmtora din toate părțile.
\par 44 Și te vor face una cu pământul, și pe fiii tăi care sunt în tine, și nu vor lăsa în tine piatră pe piatră pentru că nu ai cunoscut vremea cercetării tale.
\par 45 Și intrând în templu, a început să scoată pe cei ce vindeau și cumpărau în el.
\par 46 Zicându-le: Scris este: "Și va fi casa Mea casă de rugăciune"; dar voi ați făcut din ea peșteră de tâlhari.
\par 47 Și era în fiecare zi în templu și învăța. Dar arhiereii și cărturarii și fruntașii poporului căutau să-L piardă.
\par 48 Și nu găseau ce să-I facă, căci tot poporul se ținea după El, ascultându-L.

\chapter{20}

\par 1 Și într-una din zile, pe când Iisus învăța poporul în templu și binevestea, au venit arhiereii și cărturarii, împreună cu bătrânii,
\par 2 Și, vorbind, au zis către El: Spune nouă, cu ce putere faci acestea, sau cine este Cel ce Ți-a dat această putere?
\par 3 Iar El, răspunzând, a zis către ei: Vă voi întreba și Eu pe voi un cuvânt, și spuneți-Mi:
\par 4 Botezul lui Ioan era din cer sau de la oameni?
\par 5 Și ei cugetau în sinea lor, zicând: Dacă vom spune: Din cer, va zice: Pentru ce n-ați crezut în el?
\par 6 Iar dacă vom zice: De la oameni, tot poporul ne va ucide cu pietre, căci este încredințat că Ioan a fost prooroc.
\par 7 Și au răspuns că nu știu de unde.
\par 8 Și Iisus le-a zis: Nici Eu nu vă spun vouă cu ce putere fac acestea.
\par 9 Și a început să spună către popor pilda aceasta: Un om a sădit vie și a dat-o lucrătorilor și a plecat departe pentru multă vreme.
\par 10 Și la timpul potrivit, a trimis la lucrători o slugă ca să-i dea din rodul viei. Lucrătorii însă, bătând-o, au trimis-o fără nimic.
\par 11 Și a trimis apoi altă slugă, dar ei, bătând-o și pe aceea și batjocorind-o, au trimis-o fără nimic.
\par 12 Și a trimis apoi pe a treia; iar ei, rănind-o și pe aceea, au alungat-o.
\par 13 Și stăpânul viei a zis: Ce voi face? Voi trimite pe fiul meu cel iubit; poate se vor rușina de el.
\par 14 Iar lucrătorii, văzându-l, s-au vorbit între ei, zicând: Acesta este moștenitorul; să-l omorâm ca moștenirea să fie a noastră.
\par 15 Și scoțându-l afară din vie, l-au ucis. Ce va face, deci, acestora, stăpânul viei?
\par 16 Va veni și va pierde pe lucrătorii aceia, iar via o va da altora. Iar ei auzind, au zis: Să nu se întâmple!
\par 17 El însă, privind la ei, a zis: Ce înseamnă, deci, scriptura aceasta: "Piatra pe care n-au luat-o în seamă ziditorii, aceasta a ajuns în capul unghiului"?
\par 18 Oricine va cădea pe această piatră va fi sfărâmat, iar pe cine va cădea ea îl va zdrobi.
\par 19 Iar cărturarii și arhiereii căutau să pună mâna pe El, în ceasul acela, dar s-au temut de popor. Căci ei au înțeles că Iisus spusese pilda aceasta pentru ei.
\par 20 Și pândindu-L, I-au trimis iscoade, care se prefăceau că sunt drepți, ca să-L prindă în cuvânt și să-L dea stăpânirii și puterii dregătorului.
\par 21 Și L-au întrebat, zicând: Învățătorule, știm că vorbești și înveți drept și nu cauți la fața omului, ci cu adevărat înveți calea lui Dumnezeu:
\par 22 Se cuvine ca noi să dăm dajdie Cezarului sau nu?
\par 23 Dar Iisus, cunoscând vicleșugul lor, a zis către ei: De ce Mă ispitiți?
\par 24 Arătați-mi un dinar. Al cui chip și scriere are pe el? Iar ei au zis: Ale Cezarului.
\par 25 Și El a zis către ei: Așadar, dați cele ce sunt ale Cezarului, Cezarului și cele ce sunt ale lui Dumnezeu, lui Dumnezeu.
\par 26 Și nu L-au putut prinde în cuvânt înaintea poporului și, mirându-se de cuvântul Lui, au tăcut.
\par 27 Și apropiindu-se unii dintre saducheii care zic că nu este înviere, L-au întrebat:
\par 28 Zicând: Învățătorule, Moise a scris pentru noi: Dacă moare fratele cuiva, având femeie, și el n-a avut copii, să ia fratele lui pe femeie și să ridice urmaș fratelui său.
\par 29 Erau deci șapte frați. Și cel dintâi, luându-și femeie, a murit fără de copii.
\par 30 Și a luat-o al doilea, și a murit și el fără copii.
\par 31 A luat-o și al treilea; și tot așa toți șapte n-au lăsat copii și au murit.
\par 32 La urmă a murit și femeia.
\par 33 Deci femeia, la înviere, a căruia dintre ei va fi soție, căci toți șapte au avut-o de soție?
\par 34 Și le-a zis lor Iisus: Fiii veacului acestuia se însoară și se mărită;
\par 35 Iar cei ce se vor învrednici să dobândească veacul acela și învierea cea din morți, nici nu se însoară, nici nu se mărită.
\par 36 Căci nici să moară nu mai pot, căci sunt la fel cu îngerii și sunt fii ai lui Dumnezeu, fiind fii ai învierii.
\par 37 Iar că morții înviază a arătat chiar Moise la rug, când numește Domn pe Dumnezeul lui Avraam, și Dumnezeul lui Isaac, și Dumnezeul lui Iacov.
\par 38 Dumnezeu deci nu este Dumnezeu al morților, ci al viilor, căci toți trăiesc în El.
\par 39 Iar unii dintre cărturari, răspunzând, au zis: Învățătorule, bine ai zis.
\par 40 Și nu mai cutezau să-L întrebe nimic.
\par 41 Iar El i-a întrebat: Cum se zice, dar, că Hristos este Fiul lui David?
\par 42 Căci însuși David spune în Cartea Psalmilor: "Zis-a Domnul Domnului meu: Șezi de-a dreapta Mea,
\par 43 Până ce voi pune pe vrăjmașii Tăi așternut picioarelor Tale".
\par 44 Deci David Îl numește Domn; și cum este fiu al lui?
\par 45 Și ascultând tot poporul, a zis ucenicilor:
\par 46 Păziți-vă de cărturari, cărora le place să se plimbe în haine lungi, care iubesc plecăciunile în piețe și scaunele cele dintâi în sinagogi și locurile cele dintâi la ospețe,
\par 47 Mâncând casele văduvelor și de ochii lumii rugându-se îndelung; aceștia vor lua mai mare osândă.

\chapter{21}

\par 1 Și privind, a văzut pe cei bogați, aruncând darurile lor în vistieria templului.
\par 2 Și a văzut și pe o văduvă săracă, aruncând acolo doi bani.
\par 3 Și a zis: Adevărat vă spun că această văduvă săracă a aruncat mai mult decât toți.
\par 4 Căci toți aceștia din prisosul lor au aruncat la daruri, aceasta însă din sărăcia ei a aruncat tot ce avea pentru viață.
\par 5 Iar unii vorbind despre templu că este împodobit cu pietre frumoase și cu podoabe, El a zis:
\par 6 Vor veni zile când, din cele ce vedeți, nu va rămâne piatră peste piatră care să nu se risipească.
\par 7 Și ei L-au întrebat, zicând: Învățătorule, când oare, vor fi acestea? Și care este semnul când au să fie acestea?
\par 8 Iar El a zis: Vedeți să nu fiți amăgiți, căci mulți vor veni în numele Meu, zicând: Eu sunt, și vremea s-a apropiat. Nu mergeți după ei.
\par 9 Iar când veți auzi de războaie și de răzmerițe, să nu vă înspăimântați; căci acestea trebuie să fie întâi, dar sfârșitul nu va fi curând.
\par 10 Atunci le-a zis: Se va ridica neam peste neam și împărăție peste împărăție.
\par 11 Și vor fi cutremure mari și, pe alocurea, foamete și ciumă și spaime și semne mari din cer vor fi.
\par 12 Dar, mai înainte de toate acestea, își vor pune mâinile pe voi și vă vor prigoni, dându-vă în sinagogi și în temnițe, ducându-vă la împărați și la dregători, pentru numele Meu.
\par 13 Și va fi vouă spre mărturie.
\par 14 Puneți deci în inimile voastre să nu gândiți de mai înainte ce veți răspunde;
\par 15 Căci Eu vă voi da gură și înțelepciune, căreia nu-i vor putea sta împotrivă, nici să-i răspundă toți potrivnicii voștri.
\par 16 Și veți fi dați și de părinți și de frați și de neamuri și de prieteni, și vor ucide dintre voi.
\par 17 Și veți fi urâți de toți pentru numele Meu.
\par 18 Și păr din capul vostru nu va pieri.
\par 19 Prin răbdarea voastră veți dobândi sufletele voastre.
\par 20 Iar când veți vedea Ierusalimul înconjurat de oști, atunci să știți că s-a apropiat pustiirea lui.
\par 21 Atunci cei din Iudeea să fugă la munți și cei din mijlocul lui să iasă din el și cei de prin țarină să nu intre în el.
\par 22 Căci acestea sunt zilele răzbunării, ca să se împlinească toate cele scrise.
\par 23 Dar vai celor care vor avea în pântece și celor care vor alăpta în acele zile. Căci va fi în țară mare strâmtorare și mânie împotriva acestui popor.
\par 24 Și vor cădea de ascuțișul săbiei și vor fi duși robi la toate neamurile, și Ierusalimul va fi călcat în picioare de neamuri, până ce se vor împlini vremurile neamurilor.
\par 25 Și vor fi semne în soare, în lună și în stele, iar pe pământ spaimă întru neamuri și nedumerire din pricina vuietului mării și al valurilor.
\par 26 Iar oamenii vor muri de frică și de așteptarea celor ce au să vină peste lume, căci puterile cerurilor se vor clătina.
\par 27 Și atunci vor vedea pe Fiul Omului venind pe nori cu putere și cu slavă multă.
\par 28 Iar când vor începe să fie acestea, prindeți curaj și ridicați capetele voastre, pentru că răscumpărarea voastră se apropie.
\par 29 Și le-a spus o pildă: Vedeți smochinul și toți copacii:
\par 30 Când înfrunzesc aceștia, văzându-i, de la voi înșivă știți că vara este aproape.
\par 31 Așa și voi, când veți vedea făcându-se acestea, să știți că aproape este împărăția lui Dumnezeu.
\par 32 Adevărat grăiesc vouă că nu va trece neamul acesta până ce nu vor fi toate acestea.
\par 33 Cerul și pământul vor trece, dar cuvintele Mele nu vor trece.
\par 34 Luați seama la voi înșivă, să nu se îngreuieze inimile voastre de mâncare și de băutură și de grijile vieții, și ziua aceea să vine peste voi fără de veste,
\par 35 Ca o cursă; căci va veni peste toți cei ce locuiesc pe fața întregului pământ.
\par 36 Privegheați dar în toată vremea rugându-vă, ca să vă întăriți să scăpați de toate acestea care au să vină și să stați înaintea Fiului Omului.
\par 37 Și ziua era în templu și învăța, iar noaptea, ieșind, o petrecea pe muntele ce se cheamă al Măslinilor.
\par 38 Și tot poporul venea dis-de-dimineață la El în templu, ca să-L asculte.

\chapter{22}

\par 1 Și se apropia sărbătoarea Azimelor, care se chema Paști.
\par 2 Și arhiereii și cărturarii căutau cum să-L omoare; căci se temeau de popor.
\par 3 Și a intrat satana în Iuda, cel numit Iscarioteanul, care era din numărul celor doisprezece.
\par 4 Și, ducându-se, el a vorbit cu arhiereii și cu căpeteniile oastei, cum să-L dea în mâinile lor.
\par 5 Și ei s-au bucurat și s-au învoit să-i dea bani.
\par 6 Și el a primit și căuta prilej să-L dea lor, fără știrea mulțimii.
\par 7 Și a sosit ziua Azimelor, în care trebuia să se jertfească Paștile.
\par 8 Și a trimis pe Petru și pe Ioan, zicând: Mergeți și ne pregătiți Paștile, ca să mâncăm.
\par 9 Iar ei I-au zis: Unde voiești să pregătim?
\par 10 Iar El le-a zis: Iată, când veți intra în cetate, vă va întâmpina un om ducând un urcior cu apă; mergeți după el în casa în care va intra.
\par 11 Și spuneți stăpânului casei: Învățătorul îți zice: Unde este încăperea în care să mănânc Paștile cu ucenicii mei?
\par 12 Și acela vă va arăta un foișor mare, așternut; acolo să pregătiți.
\par 13 Iar, ei, ducându-se, au aflat precum le spusese și au pregătit Paștile.
\par 14 Și când a fost ceasul, S-a așezat la masă, și apostolii împreună cu El.
\par 15 Și a zis către ei: Cu dor am dorit să mănânc cu voi acest Paști, mai înainte de patima Mea,
\par 16 Căci zic vouă că de acum nu-l voi mai mânca, până când nu va fi desăvârșit în împărăția lui Dumnezeu.
\par 17 Și luând paharul, mulțumind, a zis: Luați acesta și împărțiți-l între voi;
\par 18 Că zic vouă: Nu voi mai bea de acum din rodul viței, până ce nu va veni împărăția lui Dumnezeu.
\par 19 Și luând pâinea, mulțumind, a frânt și le-a dat lor, zicând: Acesta este Trupul Meu care se dă pentru voi; aceasta să faceți spre pomenirea Mea.
\par 20 Asemenea și paharul, după ce au cinat, zicând: Acest pahar este Legea cea nouă, întru Sângele Meu, care se varsă pentru voi.
\par 21 Dar iată, mâna celui ce Mă vinde este cu Mine la masă.
\par 22 Și Fiul Omului merge precum a fost orânduit, dar vai omului aceluia prin care este vândut!
\par 23 Iar ei au început să se întrebe, unul pe altul, cine dintre ei ar fi acela, care avea să facă aceasta?
\par 24 Și s-a iscat între ei și neînțelegere: cine dintre ei se pare că e mai mare?
\par 25 Iar El le-a zis: Regii neamurilor domnesc peste ele și se numesc binefăcători.
\par 26 Dar între voi să nu fie astfel, ci cel mai mare dintre voi să fie ca cel mai tânăr, și căpetenia ca acela care slujește.
\par 27 Căci cine este mai mare: cel care stă la masă, sau cel care slujește? Oare, nu cel ce stă la masă? Iar Eu, în mijlocul vostru, sunt ca unul ce slujește.
\par 28 Și voi sunteți aceia care ați rămas cu Mine în încercările Mele.
\par 29 Și Eu vă rânduiesc vouă împărăție, precum Mi-a rânduit Mie Tatăl Meu,
\par 30 Ca să mâncați și să beți la masa Mea, în împărăția Mea și să ședeți pe tronuri, judecând cele douăsprezece seminții ale lui Israel.
\par 31 Și a zis Domnul: Simone, Simone, iată satana v-a cerut să vă cearnă ca pe grâu;
\par 32 Iar Eu M-am rugat pentru tine să nu piară credința ta. Și tu, oarecând, întorcându-te, întărește pe frații tăi.
\par 33 Iar el I-a zis: Doamne, cu Tine sunt gata să merg și în temniță și la moarte.
\par 34 Iar Iisus i-a zis: Zic ție, Petre, nu va cânta astăzi cocoșul, până ce de trei ori te vei lepăda de Mine, că nu Mă cunoști.
\par 35 Și le-a zis: Când v-am trimis pe voi fără pungă, fără traistă și fără încălțăminte, ați avut lipsă de ceva? Iar ei au zis: De nimic.
\par 36 Și El le-a zis: Acum însă cel ce are pungă să o ia, tot așa și traista, și cel ce nu are sabie să-și vândă haina și să-și cumpere.
\par 37 Căci vă spun că trebuie să se împlinească întru Mine Scriptura aceasta: "Și cu cei fără de lege s-a socotit", căci cele despre Mine au ajuns la sfârșit.
\par 38 Iar ei au zis: Doamne, iată aici două săbii. Zis-a lor: Sunt de ajuns.
\par 39 Și, ieșind, s-a dus după obicei în Muntele Măslinilor, și ucenicii l-au urmat.
\par 40 Și când a sosit în acest loc, le-a zis: Rugați-vă, ca să nu intrați în ispită.
\par 41 Și El S-a depărtat de ei ca la o aruncătură de piatră, și îngenunchind, Se ruga.
\par 42 Zicând: Părinte, de voiești, treacă de la Mine acest pahar. Dar nu voia Mea, ci voia Ta să se facă.
\par 43 Iar un înger din cer s-a arătat Lui și-L întărea.
\par 44 Iar El, fiind în chin de moarte, mai stăruitor Se ruga. Și sudoarea Lui s-a făcut ca picături de sânge care picurau pe pământ.
\par 45 Și, ridicându-Se din rugăciune, a venit la ucenicii Lui și i-a aflat adormiți de întristare.
\par 46 Și le-a zis: De ce dormiți? Sculați-vă și vă rugați, ca să nu intrați în ispită.
\par 47 Și vorbind El, iată o mulțime și cel ce se numea Iuda, unul dintre cei doisprezece, venea în fruntea lor. Și s-a apropiat de Iisus, ca să-L sărute.
\par 48 Iar Iisus i-a zis: Iuda, cu sărutare vinzi pe Fiul Omului?
\par 49 Iar cei din preajma Lui, văzând ce avea să se întâmple, au zis: Doamne, dacă vom lovi cu sabia?
\par 50 Și unul dintre ei a lovit pe sluga arhiereului și i-a tăiat urechea dreaptă.
\par 51 Dar Iisus, răspunzând, a zis: Lăsați, până aici. Și atingându-Se de urechea lui l-a vindecat
\par 52 Și către arhiereii, către căpeteniile templului și către bătrânii care veniseră asupra Lui, Iisus a zis: Ca la un tâlhar ați ieșit, cu săbii și cu toiege.
\par 53 În toate zilele fiind cu voi în templu, n-ați întins mâinile asupra Mea. Dar acesta este ceasul vostru și stăpânirea întunericului.
\par 54 Și, prinzându-L, L-au dus și L-au băgat în casa arhiereului. Iar Petru Îl urma de departe.
\par 55 Și, aprinzând ei foc în mijlocul curții și șezând împreună, a șezut și Petru în mijlocul lor.
\par 56 Și o slujnică, văzându-l șezând la foc, și uitându-se bine la el, a zis: Și acesta era cu El.
\par 57 Iar el s-a lepădat, zicând: Femeie, nu-L cunosc.
\par 58 Și după puțin timp, văzându-l un altul, i-a zis: Și tu ești dintre ei. Petru însă a zis: Omule, nu sunt.
\par 59 Iar când a trecut ca un ceas, un altul susținea zicând: Cu adevărat și acesta era cu El, căci este galileian.
\par 60 Și Petru a zis: Omule, nu știu ce spui. Și îndată, încă vorbind el, a cântat cocoșul.
\par 61 Și întorcându-Se, Domnul a privit spre Petru; și Petru și-a adus aminte de cuvântul Domnului, cum îi zisese că, mai înainte de a cânta cocoșul astăzi, tu te vei lepăda de Mine de trei ori.
\par 62 Și ieșind afară, Petru a plâns cu amar.
\par 63 Iar bărbații care Îl păzeau pe Iisus, Îl batjocoreau, bătându-L.
\par 64 Și acoperindu-I fața, Îl întrebau, zicând: Proorocește cine este cel ce Te-a lovit?
\par 65 Și hulindu-L, multe altele spuneau împotriva Lui.
\par 66 Și când s-a făcut ziuă, s-au adunat bătrânii poporului, arhiereii și cărturarii și L-au dus pe El în sinedriul lor.
\par 67 Zicând: Spune nouă dacă ești Tu Hristosul. Și El le-a zis: Dacă vă voi spune, nu veți crede;
\par 68 Iar dacă vă voi întreba, nu-Mi veți răspunde.
\par 69 De acum însă Fiul Omului va ședea de-a dreapta puterii lui Dumnezeu.
\par 70 Iar ei au zis toți: Așadar, Tu ești Fiul lui Dumnezeu? Și El a zis către ei: Voi ziceți că Eu sunt.
\par 71 Și ei au zis: Ce ne mai trebuie mărturii, căci noi înșine am auzit din gura Lui?

\chapter{23}

\par 1 Și sculându-se toată mulțimea acestora, L-au dus înaintea lui Pilat.
\par 2 Și au început să-L pârască, zicând: Pe Acesta L-am găsit răzvrătind neamul nostru și împiedicând să dăm dajdie Cezarului și zicând că El este Hristos rege.
\par 3 Iar Pilat L-a întrebat, zicând: Tu ești regele iudeilor? Iar El, răspunzând, a zis: Tu zici.
\par 4 Și Pilat a zis către arhierei și către mulțimi: Nu găsesc nici o vină în Omul acesta.
\par 5 Dar ei stăruiau, zicând că întărâtă poporul, învățând prin toată Iudeea, începând din Galileea până aici.
\par 6 Și Pilat auzind, a întrebat dacă omul este galileian.
\par 7 Și aflând că este sub stăpânirea lui Irod, l-a trimis la Irod, care era și el în Ierusalim în acele zile.
\par 8 Iar Irod, văzând pe Iisus, s-a bucurat foarte, că de multă vreme dorea să-L cunoască pentru că auzise despre El, și nădăjduia să vadă vreo minune săvârșită de El.
\par 9 Și L-a întrebat Irod multe lucruri, dar El nu i-a răspuns nimic.
\par 10 Și arhiereii și cărturarii erau de față, învinuindu-L foarte tare.
\par 11 Iar Irod, împreună cu ostașii săi, batjocorindu-L și luându-L în râs, L-a îmbrăcat cu o haină strălucitoare și L-a trimis iarăși la Pilat.
\par 12 Și în ziua aceea, Irod și Pilat s-au făcut prieteni unul cu altul, căci mai înainte erau în dușmănie între ei.
\par 13 Iar Pilat, chemând arhiereii și căpeteniile și poporul,
\par 14 A zis către ei: Ați adus la mine pe Omul acesta, ca pe un răzvrătitor al poporului; dar iată eu, cercetându-L în fața voastră, nici o vină n-am găsit în acest Om, din cele ce aduceți împotriva Lui.
\par 15 Și nici Irod n-a găsit, căci L-a trimis iarăși la noi. Și iată, El n-a săvârșit nimic vrednic de moarte.
\par 16 Deci, pedepsindu-L, Îl voi elibera.
\par 17 Și trebuia, la praznic, să le elibereze un vinovat.
\par 18 Dar ei, cu toții, au strigat, zicând: Ia-L pe Acesta și eliberează-ne pe Baraba,
\par 19 Care era aruncat în temniță pentru o răscoală făcută în cetate și pentru omor.
\par 20 Și iarăși le-a vorbit Pilat, voind să le elibereze pe Iisus.
\par 21 Dar ei strigau, zicând: Răstignește-L! Răstignește-L!
\par 22 Iar el a zis a treia oară către ei: Ce rău a săvârșit Acesta? Nici o vină de moarte nu am aflat întru El. Deci, pedepsindu-L, Îl voi elibera.
\par 23 Dar ei stăruiau, cerând cu strigăte mari ca El să fie răstignit, și strigătele lor au biruit.
\par 24 Deci Pilat a hotărât să se împlinească cererea lor.
\par 25 Și le-a eliberat pe cel aruncat în temniță pentru răscoală și ucidere, pe care îl cereau ei, iar pe Iisus L-a dat în voia lor.
\par 26 Și pe când Îl duceau, oprind pe un oarecare Simon Cirineul, care venea din țarină, i-au pus crucea, ca s-o ducă în urma lui Iisus.
\par 27 Iar după El venea mulțime multă de popor și de femei, care se băteau în piept și Îl plângeau.
\par 28 Și întorcându-Se către ele, Iisus le-a zis: Fiice ale Ierusalimului, nu Mă plângeți pe Mine, ci pe voi plângeți-vă și pe copiii voștri.
\par 29 Căci iată, vin zile în care vor zice: Fericite sunt cele sterpe și pântecele care n-au născut și sânii care n-au alăptat!
\par 30 Atunci vor începe să spună munților: Cădeți peste noi; și dealurilor: Acoperiți-ne.
\par 31 Căci dacă fac acestea cu lemnul verde, cu cel uscat ce va fi?
\par 32 Și erau duși și alții, doi făcători de rele, ca să-i omoare împreună cu El.
\par 33 Și când au ajuns la locul ce se cheamă al Căpățânii, L-au răstignit acolo pe El și pe făcătorii de rele, unul de-a dreapta și unul de-a stânga.
\par 34 Iar Iisus zicea: Părinte, iartă-le lor, că nu știu ce fac. Și împărțind hainele Lui, au aruncat sorți.
\par 35 Și sta poporul privind, iar căpeteniile își băteau joc de El, zicând: Pe alții i-a mântuit; să Se mântuiască și pe Sine Însuși, dacă El este Hristosul, alesul lui Dumnezeu.
\par 36 Și Îl luau în râs și ostașii care se apropiau, aducându-I oțet.
\par 37 Și zicând: Dacă Tu ești regele iudeilor, mântuiește-Te pe Tine Însuți!
\par 38 Și deasupra Lui era scris cu litere grecești, latinești și evreiești: Acesta este regele iudeilor.
\par 39 Iar unul dintre făcătorii de rele răstigniți, Îl hulea zicând: Nu ești Tu Hristosul? Mântuiește-Te pe Tine Însuți și pe noi.
\par 40 Și celălalt, răspunzând, îl certa, zicând: Nu te temi tu de Dumnezeu, că ești în aceeași osândă?
\par 41 Și noi pe drept, căci noi primim cele cuvenite după faptele noastre; Acesta însă n-a făcut nici un rău.
\par 42 Și zicea lui Iisus: Pomenește-mă, Doamne, când vei veni în împărăția Ta.
\par 43 Și Iisus i-a zis: Adevărat grăiesc ție, astăzi vei fi cu Mine în rai.
\par 44 Și era acum ca la ceasul al șaselea și întuneric s-a făcut peste tot pământul până la ceasul al nouălea.
\par 45 Când soarele s-a întunecat; iar catapeteasma templului s-a sfâșiat pe la mijloc.
\par 46 Și Iisus, strigând cu glas tare, a zis: Părinte, în mâinile Tale încredințez duhul Meu. Și acestea zicând, Și-a dat duhul.
\par 47 Iar sutașul, văzând cele ce s-au făcut, a slăvit pe Dumnezeu, zicând: Cu adevărat, Omul Acesta drept a fost.
\par 48 Și toate mulțimile care veniseră la această priveliște, văzând cele întâmplate, se întorceau bătându-și pieptul.
\par 49 Și toți cunoscuții Lui, și femeile care Îl însoțiseră din Galileea, stăteau departe, privind acestea.
\par 50 Și iată un bărbat cu numele Iosif, sfetnic fiind, bărbat bun și drept,
\par 51 - Acesta nu se învoise cu sfatul și cu fapta lor. El era din Arimateea, cetate a iudeilor, așteptând împărăția lui Dumnezeu.
\par 52 Acesta, venind la Pilat, a cerut trupul lui Iisus.
\par 53 Și coborându-L, L-a înfășurat în giulgiu de in și L-a pus într-un mormânt săpat în piatră, în care nimeni, niciodată, nu mai fusese pus.
\par 54 Și ziua aceea era vineri, și se lumina spre sâmbătă.
\par 55 Și urmându-I femeile, care veniseră cu El din Galileea, au privit mormântul și cum a fost pus trupul Lui.
\par 56 Și, întorcându-se, au pregătit miresme și miruri; iar sâmbătă s-au odihnit, după Lege.

\chapter{24}

\par 1 Iar în prima zi după sâmbătă, foarte de dimineață, au venit ele la mormânt, aducând miresmele pe care le pregătiseră.
\par 2 Și au găsit piatra răsturnată de pe mormânt.
\par 3 Și intrând, nu au găsit trupul Domnului Iisus.
\par 4 Și fiind ele încă nedumerite de aceasta, iată doi bărbați au stat înaintea lor, în veșminte strălucitoare.
\par 5 Și, înfricoșându-se ele și plecându-și fețele la pământ, au zis aceia către ele: De ce căutați pe Cel viu între cei morți?
\par 6 Nu este aici, ci S-a sculat. Aduceți-vă aminte cum v-a vorbit, fiind încă în Galileea,
\par 7 Zicând că Fiul Omului trebuie să fie dat în mâinile oamenilor păcătoși și să fie răstignit, iar a treia zi să învieze.
\par 8 Și ele și-au adus aminte de cuvântul Lui.
\par 9 Și întorcându-se de la mormânt, au vestit toate acestea celor unsprezece și tuturor celorlalți.
\par 10 Iar ele erau: Maria Magdalena, și Ioana și Maria lui Iacov și celelalte împreună cu ele, care ziceau către apostoli acestea.
\par 11 Și cuvintele acestea au părut înaintea lor ca o aiurare și nu le-au crezut.
\par 12 Și Petru, sculându-se, a alergat la mormânt și, plecându-se, a văzut giulgiurile singure zăcând. Și a plecat, mirându-se în sine de ceea ce se întâmplase.
\par 13 Și iată, doi dintre ei mergeau în aceeași zi la un sat care era departe de Ierusalim, ca la șaizeci de stadii, al cărui nume era Emaus.
\par 14 Și aceia vorbeau între ei despre toate întâmplările acestea.
\par 15 Și pe când vorbeau și se întrebau între ei. și Iisus Însuși, apropiindu-Se, mergea împreună cu ei.
\par 16 Dar ochii lor erau ținuți ca să nu-L cunoască.
\par 17 Și El a zis către ei: Ce sunt cuvintele acestea pe care le schimbați unul cu altul în drumul vostru? Iar ei s-au oprit, cuprinși de întristare.
\par 18 Răspunzând, unul cu numele Cleopa a zis către El: Tu singur ești străin în Ierusalim și nu știi cele ce s-au întâmplat în el în zilele acestea?
\par 19 El le-a zis: Care? Iar ei I-au răspuns: Cele despre Iisus Nazarineanul, Care era prooroc puternic în faptă și în cuvânt înaintea lui Dumnezeu și a întregului popor.
\par 20 Cum L-au osândit la moarte și L-au răstignit arhiereii și mai-marii noștri;
\par 21 Iar noi nădăjduiam că El este Cel ce avea să izbăvească pe Israel; și, cu toate acestea, astăzi este a treia zi de când s-au petrecut acestea.
\par 22 Dar și niște femei de ale noastre ne-au spăimântat ducându-se dis-de-dimineață la mormânt,
\par 23 Și, negăsind trupul Lui, au venit zicând că au văzut arătare de îngeri, care le-au spus că El este viu.
\par 24 Iar unii dintre noi s-au dus la mormânt și au găsit așa precum spuseseră femeile, dar pe El nu L-au văzut.
\par 25 Și El a zis către ei: O, nepricepuților și zăbavnici cu inima ca să credeți toate câte au spus proorocii!
\par 26 Nu trebuia oare, ca Hristos să pătimească acestea și să intre în slava Sa?
\par 27 Și începând de la Moise și de la toți proorocii, le-a tâlcuit lor, din toate Scripturile cele despre El.
\par 28 Și s-au apropiat de satul unde se duceau, iar El se făcea că merge mai departe.
\par 29 Dar ei Îl rugau stăruitor, zicând: Rămâi cu noi că este spre seară și s-a plecat ziua. Și a intrat să rămână cu ei.
\par 30 Și, când a stat împreună cu ei la masă, luând El pâinea, a binecuvântat și, frângând, le-a dat lor.
\par 31 Și s-au deschis ochii lor și L-au cunoscut; și El s-a făcut nevăzut de ei.
\par 32 Și au zis unul către altul: Oare, nu ardea în noi inima noastră, când ne vorbea pe cale și când ne tâlcuia Scripturile?
\par 33 Și, în ceasul acela sculându-se, s-au întors la Ierusalim și au găsit adunați pe cei unsprezece și pe cei ce erau împreună cu ei,
\par 34 Care ziceau că a înviat cu adevărat Domnul și S-a arătat lui Simon.
\par 35 Și ei au istorisit cele petrecute pe cale și cum a fost cunoscut de ei la frângerea pâinii.
\par 36 Și pe când vorbeau ei acestea, El a stat în mijlocul lor și le-a zis: Pace vouă.
\par 37 Iar ei, înspăimântându-se și înfricoșându-se, credeau că văd duh.
\par 38 Și Iisus le-a zis: De ce sunteți tulburați și pentru ce se ridică astfel de gânduri în inima voastră?
\par 39 Vedeți mâinile Mele și picioarele Mele, că Eu Însumi sunt; pipăiți-Mă și vedeți, că duhul nu are carne și oase, precum Mă vedeți pe Mine că am.
\par 40 Și zicând acestea, le-a arătat mâinile și picioarele Sale.
\par 41 Iar ei încă necrezând de bucurie și minunându-se, El le-a zis: Aveți aici ceva de mâncare?
\par 42 Iar ei i-au dat o bucată de pește fript și dintr-un fagure de miere.
\par 43 Și luând, a mâncat înaintea lor.
\par 44 Și le-a zis: Acestea sunt cuvintele pe care le-am grăit către voi fiind încă împreună cu voi, că trebuie să se împlinească toate cele scrise despre Mine în Legea lui Moise, în prooroci și în psalmi.
\par 45 Atunci le-a deschis mintea ca să priceapă Scripturile.
\par 46 Și le-a spus că așa este scris și așa trebuie să pătimească Hristos și așa să învieze din morți a treia zi.
\par 47 Și să se propovăduiască în numele Său pocăința spre iertarea păcatelor la toate neamurile, începând de la Ierusalim.
\par 48 Voi sunteți martorii acestora.
\par 49 Și iată, Eu trimit peste voi făgăduința Tatălui Meu; voi însă ședeți în cetate, până ce vă veți îmbrăca cu putere de sus.
\par 50 Și i-a dus afară până spre Betania și, ridicându-Și mâinile, i-a binecuvântat.
\par 51 Și pe când îi binecuvânta, S-a despărțit de ei și S-a înălțat la cer.
\par 52 Iar ei, închinându-se Lui, s-au întors în Ierusalim cu bucurie mare.
\par 53 Și erau în toată vremea în templu, lăudând și binecuvântând pe Dumnezeu. Amin.


\end{document}