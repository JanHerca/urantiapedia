\begin{document}

\title{2 Samuel}


\chapter{1}

\par 1 După moartea lui Saul, când David se întorcea din războiul ce-l purtase împotriva Amaleciților și după ce s-a oprit două zile în Țiclag,
\par 2 Iată că a treia zi a venit un om din tabăra lui Saul. Acesta avea pe el hainele rupte și pe cap țărână și după ce a ajuns la David, a căzut la pământ și i s-a închinat.
\par 3 Iar David i-a zis: "De unde vii tu?" "Am fugit din tabăra lui Israel", răspunse acela.
\par 4 "Spune-mi, a zis David, ce s-a întâmplat?" "Poporul, a zis omul, a fugit din luptă și mulțime din popor a căzut și a murit, și a murit și Saul și fiul său Ionatan".
\par 5 "Cum știi că Saul și fiul său Ionatan au murit?" a întrebat David pe tânărul care-i adusese vestea.
\par 6 "Din întâmplare, m-am dus pe muntele Ghelboa, a zis omul ce-i grăia, și iată Saul stătea sprijinit pe sulița sa, iar carele și călăreții îl ajungeau.
\par 7 Atunci el s-a întors și, văzându-mă pe mine, m-a strigat și eu am răspuns: "Iată-mă!"
\par 8 și el mi-a zis: "Cine ești tu?" și eu i-am răspuns: "Sunt un amalecit!"
\par 9 Atunci el mi-a zis: "Apropie-te de mine și mă ucide, că durerea morții m-a cuprins și sufletul meu este încă tot în mine!"
\par 10 Și eu m-am apropiat și l-am ucis, căci știam că n-are să mai trăiască după căderea sa. Apoi am luat cununa regească de pe capul lui, brățara ce era la mâna lui și le-am adus aici la stăpânul meu".
\par 11 Atunci a apucat David hainele sale și le-a rupt; asemenea și oamenii cei ce erau cu el și-au rupt hainele lor,
\par 12 Și au plâns și s-au tânguit și au postit până seara după Saul și după fiul său Ionatan, după poporul Domnului și după casa lui Israel, care căzuseră de sabie.
\par 13 Apoi David a zis către omul care-i spusese acestea: "De unde ești tu?" "Eu, a zis acela, sunt fiul unui străin amalecit".
\par 14 Și David a zis: "Cum nu te-ai temut tu să-li ridici mâna, ca să ucizi pe unsul Domnului?"
\par 15 Apoi chemând David pe unul din slujitori, i-a zis: "Vino de-l ucide!" și l-a ucis.
\par 16 Iar David a zis către el: "Sângele tău să fie pe capul tău, căci buzele tale au mărturisit împotriva ta, când ai zis: Eu am ucis pe unsul Domnului".
\par 17 Și a plâns David pe Saul și pe Ionatan, fiul lui, prin această cântare de jale,
\par 18 Pe care a poruncit să o învețe fiii lui Iuda, după cum este scrisă în cartea dreptului, unde se zice:
\par 19 "Podoaba ta, Israele, a fost doborâtă pe înălțimile tale! Cum au căzut vitejii!
\par 20 Nu vestiți în Gat și nu dați de știre pe ulițele Ascalonului, ca să nu se bucure fiicele Filistenilor, ca să nu prăznuiască fiicele celor netăiați împrejur!
\par 21 Munților Ghelboa, să nu mai cadă pe voi nici rouă, nici ploaie; și să nu mai fie pe voi țarini roditoare, căci acolo a fost doborât scutul celor războinici, scutul lui Saul, ca și cum n-ar fi fost uns cu untdelemn sfințit.
\par 22 Arcul lui Ionatan nu se întorcea fără sânge de răniți, fără grăsimea celor puternici, nici sabia lui Saul nu se învârtea zadarnic!
\par 23 Saul și Ionatan, cei iubiți și uniți în viața lor, nici la moarte nu s-au despărțit. Fost-au mai iuți decât vulturii și mai puternici decât leii!
\par 24 Fiicele lui Israel, plângeți pe Saul, cel ce v-a îmbrăcat în purpură cu podoabe și v-a pus pe haine podoabe de aur!
\par 25 Cum au căzut vitejii în toiul luptei! Ucis a fost Ionatan pe înălțimile tale, Israele!
\par 26 Frate Ionatane, întristat sunt după tine, căci tu mi-ai fost foarte scump și iubirea ta a fost pentru mine mai presus de iubirea femeiască!
\par 27 Cum au căzut cei viteji! Pierit-a arma de război!"

\chapter{2}

\par 1 După aceasta David a întrebat pe Domnul și a zis: "Să mă duc, oare, în vreuna din cetățile lui Iuda?" "Du-te!", i-a zis Domnul. "Unde să mă duc?", a întrebat iarăși David. "În Hebron", i s-a răspuns.
\par 2 Și s-a dus acolo David și cele două femei ale sale, Ahinoam izreeliteanca și Abigail carmeliteanca, fosta femeie a lui Nabal.
\par 3 Și pe oamenii cei ce fuseseră cu el i-a adus David pe fiecare cu familia lui și s-a așezat în cetatea Hebron.
\par 4 Atunci au venit bărbații lui Iuda și au uns acolo pe David rege pentru casa lui Iuda, spunându-i-se lui David că locuitorii din Iabeșul Galaadului au îngropat pe Saul.
\par 5 Atunci a trimis David soli la locuitorii din Iabeșul Galaadului, ca să le spună: "Binecuvântați sunteți voi de Domnul, pentru că ați arătat milă lui Saul, domnul vostru și unsul Domnului, și l-ați îngropat pe el și pe fiul lui, Ionatan.
\par 6 Să vă răsplătească dar Domnul cu milă și cu credincioșie, și eu vă voi face bine, pentru că ați făcut aceasta.
\par 7 Să se întărească mâinile voastre și să fiți curajoși, căci stăpânul vostru Saul a murit și casa lui luda m-a uns pe mine rege peste voi.
\par 8 Dar Abner, fiul lui Ner, căpetenia oștirilor lui Saul, a luat pe Ișboșet, fiul lui Saul, și l-a dus din tabăra lui la Mahanaim
\par 9 Și l-a făcut rege peste Galaad, Așer, Izreel, Efrem, Veniamin și peste tot Israelul.
\par 10 Ișboșet însă, fiul lui Saul, era ca de patruzeci de ani când s-a făcut rege peste Israel și a domnit doi ani; iar cu David a rămas numai casa lui Iuda.
\par 11 Tot timpul cât a domnit David în Hebron peste casa lui Iuda, au fost șapte ani și șase luni.
\par 12 Atunci Abner, fiul lui Ner, și slugile lui Ișboșet, fiul lui Saul, au ieșit din Mahanaim la Ghibeon și a ieșit și Ioab, fiul Țeruiei, cu slugile lui David și s-au întâlnit la iazul Ghibeonului.
\par 13 Și s-au așezat unii de o parte de iaz, iar alții de cealaltă parte a iazului.
\par 14 Și a zis Abner către Ioab: "Să se scoale tinerii și să joace înaintea noastră!" "Să se scoale!", a zis Ioab.
\par 15 Și s-au sculat și s-au dus un număr de doisprezece Veniamineni din partea lui Ișboșet, fiul lui Saul, și doisprezece dintre slugile lui David.
\par 16 Și s-au apucat de cap unul pe altul și și-au înfipt sabia unul altuia în coastă și au căzut împreună. Și s-a numit locul acela Helcat-Hațurim, (Locul Săbiilor), care se află în Ghibeon.
\par 17 Apoi s-a dat în ziua aceea bătălia cea mai crâncenă și Abner cu oamenii lui Israel au fost bătuți de slugile lui David.
\par 18 Acolo se aflau trei feciori ai Țeruiei: Ioab, Abișai și Asael. Asael însă era sprinten de picior, ca o căprioară de câmp.
\par 19 Și a alergat Asael după Abner și l-a urmărit, fără să se abată nici la dreapta, nici la stânga de pe urmele lui.
\par 20 Și uitându-se Abner înapoi, a zis: "Asael, tu ești oare?" "Eu!" a zis acesta.
\par 21 "Abate-te la dreapta sau la stânga, a zis Abner, și alege-ți unul din oameni și ia armele lui!"
\par 22 Dar Asael nu a voit să se lase de el. Și a zis iarăși Abner către Asael: "Lasă-te de mine, ca să nu te dobor la pământ. Atunci cu ce obraz mă voi arăta înaintea lui Ioab, fratele tău? Ce este aceasta? întoarce-te la fratele tău!"
\par 23 Dar acela nu a voit să se lase. Atunci Abner, întorcându-și lancea, l-a lovit în pântece și lancea a trecut printr-însul și el a căzut chiar acolo și a murit pe loc. Și toți cei ce treceau pe la locul unde căzuse și murise Asael se opreau.
\par 24 Ioab și Abișai încă urmăreau pe Abner. Soarele asfințise când au sosit aceștia la dealul Amma ce vine în fața Ghiahului, pe calea ce duce spre pustiul Ghibeonului.
\par 25 Și atunci, adunându-se Veniaminenii împrejurul lui Abner, au alcătuit o oștire și au stat pe vârful unui deal,
\par 26 De unde a strigat Abner către Ioab și a zis: "Oare mereu va sfâșia sabia? Dare nu știi tu că urmările au să fie dureroase? Până când nu vei zice oamenilor să înceteze de a mai urmări pe frații lor?"
\par 27 "Viu este Dumnezeu, a zis Ioab, dacă tu ne-ai fi grăit altfel, încă de dimineață ar fi încetat oamenii mei de a mai urmări pe frații lor".
\par 28 Apoi a sunat Ioab din trâmbiță și tot poporul s-a oprit și mai mult n-au mai urmărit pe Israeliți, încetând lupta.
\par 29 Abner însă și oamenii lui au mers pe șes toată noaptea aceea și au trecut Iordanul și, străbătând tot Bitronul, au venit la Mahanaim.
\par 30 Ioab însă s-a întors din urmărirea lui Abner și a adunat tot poporul și au lipsit de la număr, dintre oamenii lui David, nouăsprezece inși, precum și Asael.
\par 31 Iar slugile lui David, lovind pe Veniamineni și pe oamenii lui Abner, au căzut din aceștia trei sute șaizeci de oameni.
\par 32 Și luând pe Asael, l-au îngropat în mormântul tatălui său, ce se află în Betleem. Apoi Ioab cu oamenii săi au mers toată noaptea și în revărsatul zorilor au ajuns la Hebron.

\chapter{3}

\par 1 Și a ținut multă vreme dușmănia între casa lui Saul și casa lui David. David însă se întărea mereu, iar casa lui Saul slăbea din ce în ce mai mult.
\par 2 Lui David i s-au născut șase fii în Hebron. Întâiul său născut a fost Amnon din Ahinoam, izreeliteanca.
\par 3 Al doilea fiu al lui a fost Chileab din Abigail carmeliteanca, fosta femeie a lui Nabal. Al treilea a fost Abesalom, fiul Maachei, fiica lui Talmai, regele Gheșurului.
\par 4 Al patrulea a fost Adonia, fiul Haghitei. Al cincilea a fost Șefatia, fiul Abitalei.
\par 5 Iar al șaselea a fost Itream din Egla, femeia lui David. Aceștia i s-au născut lui David în Hebron.
\par 6 Pe când era dușmănie între casa lui Saul și casa lui David, Abner ținea cu casa lui Saul.
\par 7 Saul avusese o concubină cu numele Rițpa, fiica lui Aia. Abner a intrat la ea, iar Ișboșet a zis către Abner: "La ce ai intrat tu la concubina tatălui meu?"
\par 8 Abner însă, mâniindu-se strașnic de vorbele lui Ișboșet, a zis: "Au doară eu sunt cap de câine? Eu, împotriva casei lui Iuda, am arătat acum milă casei lui Saul, tatăl tău, fraților lui și prietenilor lui, și nu te-am dat în mâinile lui David; iar tu îmi găsești acum vină pentru o femeie?
\par 9 Așa și așa să facă Dumnezeu lui Abner și încă și mar mult să-i facă! Precum s-a jurat Domnul lui David, tocmai așa voi și face în ziua aceasta:
\par 10 Voi lua domnia de la casa lui Saul și voi pune tronul lui David peste cața lui Israel și peste Iuda, de la Dan până la Beer-Șeba".
\par 11 Și n-a putut Ișboșet să răspundă lui Abner nimic, căci se temea de el.
\par 12 Abner însă a trimis din partea sa vestitori la David în Hebron, unde se afla el, să-i zică: "Al cui este pământul acesta?", și să-i mai zică încă: "Încheie legământ cu mine și mâna mea va fi cu tine, ca să întoarcă la tine pe tot poporul lui Israel!"
\par 13 Iar David a răspuns: "Bine, voi încheia legământ cu tine; dar te rog un lucru anume: nu vei vedea fața mea, dacă nu vei aduce cu tine și pe Micol, fiica lui Saul, când vei veni să te vezi cu mine".
\par 14 Apoi a trimis David soli la Ișboșet, fiul lui Saul, să-i zică: "Dă-mi pe femeia mea, Micol, pe care am luat-o de femeie pentru o sută de prepuțuri filistene".
\par 15 Și a trimis Ișboșet și a luat-o de la bărbatul ei, de la Paltiel, fiul lui Laiș;
\par 16 Și s-a dus cu ea și bărbatul ei și a petrecut-o cu plângere până la Bahurim. Dar Abner a zis către el: "Du-te înapoi!" Și acela s-a întors.
\par 17 Atunci s-a întors Abner către bătrânii lui Israel și a zis: "Voi și ieri și alaltăieri ați dorit ca David să fie rege peste voi;
\par 18 Acum faceți aceasta, căci Domnul a zis către David: Prin mâna robului Meu David voi izbăvi poporul Meu Israel din mâna Filistenilor și din mâna tuturor vrăjmașilor lui".
\par 19 La fel a grăit Abner și Veniaminenilor. S-a dus apoi Abner la Hebron, ca să spună lui David tot ce dorea Israel și toată casa lui Veniamin.
\par 20 Și a venit Abner la David în Hebron, și cu el au venit și douăzeci de oameni, și a făcut David ospăț pentru Abner și pentru însoțitorii lui.
\par 21 Abner a zis lui David: "Eu mă voi scula și mă voi duce și voi aduna la regele, stăpânul meu, tot poporul lui Israel ca să facă legământ cu tine și vei fi rege peste toți, după cum dorește sufletul tău". Iar David a dat drumul lui Abner și s-a dus cu pace.
\par 22 Și iată slugile lui David cu Ioab au venit de la bătălie și au adus cu ei pradă multă. Dar Abner nu mai era cu David în Hebron, căci David îi dăduse drumul și se dusese cu pace.
\par 23 Și când Ioab și toată oștirea lui au venit, i s-a spus lui Ioab: "Abner, fiul lui Ner, a venit la rege și acesta i-a dat drumul de s-a dus cu pace".
\par 24 Atunci a venit Ioab la rege și a zis: "Ce ai făcut? Iată a venit la tine Abner; de ce n-ai dat drumul să plece?
\par 25 Tu știi pe Abner, fiul lui Ner; el a venit să te înșele și să afle pe unde intri și pe unde ieși și să cunoască tot ceea ce faci tu".
\par 26 Ieșind apoi Ioab de la David, a trimis oameni după Abner și l-au întors aceștia de la fântâna Sira, fără știrea lui David.
\par 27 Și când Abner s-a întors la Hebron, Ioab l-a băgat pe poartă înăuntru, ca și cum ar fi vrut să vorbească cu el în taină, și acolo l-a lovit în pântece. Și a murit Abner pentru sângele lui Asael, fratele lui Ioab.
\par 28 Auzind în urmă David de aceasta, a zis: "Nevinovat sunt eu și regatul meu în veac înaintea Domnului de sângele lui Abner, fiul lui Ner; cadă el pe capul lui Ioab și peste toată casa tatălui său;
\par 29 Niciodată să nu lipsească din casa lui Ioab cei ce pătimesc de scurgere, cei leproși, cei ce merg în cârji, cei omorâți de sabie și cei lipsiți de pâine".
\par 30 Ioab însă și fratele său Abișai uciseseră pe Abner, pentru că acesta omorâse pe fratele lor Asael în lupta de la Ghibeon.
\par 31 David însă a zis către Ioab și către toți oamenii care erau cu el: "Rupeți-vă hainele și vă încingeți cu sac și jeliți pe Abner!"
\par 32 Apoi regele David a mers după sicriul lui și, când a fost îngropat Abner în Hebron, regele a plâns tare la mormântul lui Abner și a plâns și tot poporul.
\par 33 Și a zis regele, când plângea pe Abner:
\par 34 "Cum să moară Abner ca un rău? Mâinile nu ți-au fost legate, nici picioarele nu-ți erau încătușate, ci ai căzut ca cei doborâți de tâlhari!"
\par 35 Atunci tot poporul a început să plângă încă și mai tare după el. Și a venit tot poporul să aducă lui David pâine, când încă era ziuă; însă David s-a jurat, zicând: "Așa și așa să facă Dumnezeu cu mine și încă și mai mult să facă, de voi gusta pâine sau altceva înainte de asfințitul soarelui".
\par 36 Și a aflat de aceasta tot poporul și i-a plăcut aceasta, cum plăcea întregului popor tot ceea ce făcea regele.
\par 37 Și a aflat în ziua aceea tot poporul și tot Israelul că nu din pricina regelui s-a săvârșit uciderea lui Abner, fiul lui Ner.
\par 38 Și a zis regele către slugile sale: "Știți voi oare că astăzi a căzut în Israel un bărbat și o căpetenie mare?
\par 39 Eu astăzi sunt încă slab, deși sunt uns rege; iar oamenii aceștia, fiii Țeruiei, sunt mai tari decât mine. Să răsplătească deci Domnul celui ce face rău după răutatea lui!"

\chapter{4}

\par 1 Auzind Ișboșet, fiul lui Saul, că a murit Abner în Hebron, i-au slăbit mâinile și tot Israelul s-a tulburat.
\par 2 Ișboșet, fiul lui Saul, avea două căpetenii de oștire; numele unuia era Baana și numele celuilalt era Rechab, feciorii lui Rimon Beeroteanul, din urmașii lui Veniamin, căci și Beerotul se socotea al lui Veniamin.
\par 3 Și au fugit Beerotenii la Ghitaim rămânând acolo ca străini până azi.
\par 4 De la Ionatan, fiul lui Saul, rămăsese ura fiu șchiop. Acesta era de cinci ani, când a venit din Israel vestea despre moartea lui Saul și a lui Ionatan, iar doica lui l-a luat și a fugit. Dar pe când fugea ea grăbită, el a căzut și a rămas șchiop. Numele lui era Mefiboșet.
\par 5 Atunci au plecat Rechab și Baana, fiii lui Rimon Beeroteanul, și au venit în casa lui Ișboșet chiar în arșița zilei; acesta însă dormea de amiază în patul său.
\par 6 Iar portarului casei, care curățise grâu, îi venise somn și adormise. Atunci Rechab și Baana, fratele său, au intrat în casă, ca și-cum ar fi vrut să ia grâu, și l-au lovit pe Ișboșet în stomac și apoi au fugit.
\par 7 Și când intraseră ei în casă, Ișboșet dormea în patul său, în odaia sa de dormit, și ei l-au lovit și l-au omorât și i-au tăiat capul și au luat capul lui cu ei și au mers prin câmpie toată noaptea
\par 8 Și au adus capul lui Ișboșet la David, în Hebron, și au zis către rege: "Iată capul lui Ișboșet, fiul lui Saul, dușmanul tău, care a căutat sufletul tău. Acum Domnul a răzbunat pe domnul meu regele, împotriva lui Saul, vrăjmașul tău, și împotriva urmașilor lui".
\par 9 Și răspunzând David lui Rechab și lui Baana, fratele lui, feciorii lui Rimon Beeroteanul, le-a zis: "Viu este Domnul, Care a izbăvit sufletul meu din tot necazul,
\par 10 Că dacă pe cel ce mi-a adus vestea și a zis: "Iată a murit Saul și Ionatan", și se socotea pe sine vestitor de bucurie, eu, în loc să-l răsplătesc, l-am prins și l-am ucis în Țiclag,
\par 11 Apoi acum, când niște oameni netrebnici au ucis un om nevinovat în casa lui și în patul lui, oare nu voi cere sângele lui din mâinile voastre și nu vă voi stârpi de pe pământ?"
\par 12 Și a poruncit David slugilor și i-au ucis, și le-au tăiat mâinile și picioarele și le-au spânzurat deasupra iazului din Hebron. Iar capul lui Ișboșet l-au luat și l-au îngropat în mormântul lui Abner, în Hebron.

\chapter{5}

\par 1 Atunci au venit toate triburile lui Israel la David în Hebron și au zis:
\par 2 "Iată, noi suntem oasele tale și carnea ta. Încă de pe când Saul domnea peste noi, tu ai povățuit pe Israel și Domnul a zis către tine: Tu vei paște pe poporul Meu Israel și tu vei fi povățuitorul lui Israel".
\par 3 Au venit toți bătrânii lui Israel la rege în Hebron și a încheiat cu ei regele David legământ în Hebron, înaintea Domnului; și au uns pe David rege peste tot Israelul.
\par 4 David însă era ca de treizeci de ani când s-a făcut rege și a domnit patruzeci de ani.
\par 5 în Hebron a domnit peste Iuda șapte ani și șase luni, iar în Ierusalim a domnit treizeci și trei de ani peste tot Israelul și peste Iuda.
\par 6 Atunci au pornit regele și oamenii lui la Ierusalim, împotriva Iebuseilor, locuitorii țării aceleia. Dar aceștia au zis către David: "Nu vei intra aici, căci te vor goni orbii și șchiopii care zic: David nu va intra aici!"
\par 7 David însă a luat cetatea Sionului; aceasta este cetatea lui David.
\par 8 Și a zis David în ziua aceea: "Tot cel ce va ucide pe Iebusei să lovească cu lancea și pe șchiopii și pe orbii care urăsc sufletul lui David". De aceea se și zice: "Orbul și șchiopul nu vor intra în casa Domnului!"
\par 9 Atunci s-a mutat David în cetate și a numit-o cetatea lui David; și a făcut întărituri de jur împrejur, de la Milo și până înăuntru.
\par 10 Și a propășit David și s-a înălțat, și Domnul Dumnezeul Savaot era cu el.
\par 11 În vremea aceea a trimis Hiram, regele Tirului, soli la David și lemn de cedru, tâmplari, pietrari și zidari, și aceștia au făcut casă lui David.
\par 12 Și a înțeles David că Domnul l-a întărit rege peste Israel și a înălțat regatul său din pricina poporului său Israel.
\par 13 Și și-a mai luat David femei și concubine din Ierusalim, după ce a venit din Hebron.
\par 14 Și i s-au mai născut lui David fii și fiice. Iată și numele celor ce i s-au născut în Ierusalim: Șamua și Șobab, Natan și Solomon;
\par 15 Ibhar și Elișua, Nefeg și Iafia;
\par 16 Elișama, Eliada și Elifelet; Samae, Iesivat, Natan, Galamaan, Ievaar, Teisus, Elfalat, Naged, Nafec, Ionatan, Leasamis, Baalimat și Elifaat.
\par 17 Iar dacă au auzit Filistenii că David a fost uns rege peste Israel, s-au ridicat Filistenii cu toții să caute pe David. Auzind însă David, s-a dus în cetate;
\par 18 Iar Filistenii au venit și s-au așezat în valea Refaim.
\par 19 Și a întrebat David pe Domnul, zicând: "Să mă duc oare împotriva Filistenilor? Îi vei da Tu, oare, în mâinile mele?" "Du-te, a zis Domnul către David, căci Eu voi da pe Filisteni în mâinile tale!"
\par 20 Atunci s-a dus David la Baal-Perațim și i-a lovit acolo și a zis David: "Domnul a măturat pe vrăjmașii mei dinaintea mea, ca și cum i-ar fi luat apa". Și de aceea s-a și dat locului aceluia numele de Baal-Perațim.
\par 21 Filistenii însă și-au lăsat acolo idolii lor, iar David și oamenii săi i-au luat și a poruncit să-i ardă cu foc.
\par 22 Dar Filistenii au năvălit iarăși și s-au așezat în valea Refaim.
\par 23 David a întrebat din nou pe Domnul, zicând: "Să mă duc oare împotriva Filistenilor și îi vei da Tu oare în mâinile mele?" "Să nu ieși înaintea lor, i-a răspuns El, ci i-ai pe la spate și înaintează spre ei dinspre dumbrava murelor;
\par 24 Și când vei auzi un zgomot, ca și cum ar veni pe vârful arborilor dumbrăvilor, atunci să pornești, căci atunci a pornit Domnul înaintea ta, ca să lovească oștirea Filistenilor".
\par 25 Și a făcut David cum i-a poruncit Domnul și a lovit pe Filisteni de la Ghibeon până la Ghezer.

\chapter{6}

\par 1 După aceea a adunat David din nou pe toți aleșii săi din Israel, ca la treizeci de mii.
\par 2 Și, David, cu tot poporul care era cu el, a pornit la Baalat în Iuda, ca să aducă chivotul Domnului, asupra căruia este chemat numele Domnului Savaot Cel ce șade pe heruvimi.
\par 3 Și punând chivotul Domnului într-un car nou, l-au scos din casa lui Aminadab; iar fiii lui Aminadab, Uza și Ahio, duceau carul cel nou.
\par 4 Și l-au adus cu chivotul Domnului din casa lui Aminadab cea de pe deal și Ahio mergea înaintea chivotului Domnului.
\par 5 Iar David și toți fiii lui Israel cântau înaintea Domnului din tot felul de instrumente muzicale de lemn de chiparos, din harpe, din psaltire, din timpane, din fluiere și din chimvale.
\par 6 Când însă au ajuns la aria lui Nachon, Uza și-a întins mâinile sale spre chivotul Domnului ca să-l sprijine, și s-a apucat de el, căci boii erau gata să-l răstoarne.
\par 7 Domnul însă s-a mâniat pe Uza și l-a lovit Dumnezeu chiar acolo pentru îndrăzneala lui și a murit el acolo lingă chivotul Domnului.
\par 8 Atunci s-a întristat David că a lovit Domnul pe Uza și locul acesta și până astăzi se cheamă Perei-Uza.
\par 9 Și s-a temut David în ziua aceea de Domnul și a zis: "Cum va intra chivotul Domnului la mine?"
\par 10 Și n-a voit David să ducă chivotul Domnului la sine, în cetatea lui David, ci l-au întors în casa lui Obed-Edom Gateanul.
\par 11 Și a rămas chivotul Domnului în casa lui Obed-Edom Gateanul trei luni și a binecuvântat Domnul pe Obed-Edom și toată casa lui.
\par 12 Iar când i s-a spus regelui David și i s-a zis: "Domnul a binecuvântat casa lui Obed-Edom și toate câte erau ale lui pentru chivotul Domnului", atunci s-a dus David și a adus cu alai chivotul Domnului din casa lui Obed-Edom în cetatea lui David.
\par 13 Dar când cei ce duceau chivotul Domnului făceau câte șase pași, el aducea jertfă un vițel și un berbec.
\par 14 Și David dănțuia cât putea înaintea Domnului și era îmbrăcat cu efod de in.
\par 15 Așa a adus David și tot poporul chivotul Domnului cu strigăte și cu sunete de trâmbiță.
\par 16 Iar când a intrat chivotul Domnului în cetatea lui David, Micol, fiica lui Saul, se uita pe fereastră și, văzând pe regele David sărind și jucând înaintea Domnului, l-a disprețuit în inima sa.
\par 17 Și au dus chivotul Domnului și l-au pus la locul lui, în mijlocul cortului, pe care-l făcuse pentru el David; apoi David a adus arderi de tot înaintea Domnului și jertfe de împăcare.
\par 18 Iar dacă a isprăvit David de adus arderile de tot și jertfele de împăcare, a binecuvântat poporul în numele Domnului Savaot.
\par 19 Și a împărțit la tot poporul, la toată mulțimea lui Israel de la Dan până la Beer-Șeba, fiecăruia, atât bărbaților cât și femeilor, câte o pâine și câte o bucată de carne friptă și câte o turtă. Și s-a dus tot poporul, mergând fiecare la casa sa.
\par 20 Dar când s-a întors David ca să binecuvânteze casa sa, atunci Micol, fiica lui Saul, i-a ieșit întru întâmpinare și i-a zis: "Câtă cinste și-a făcut azi regele lui Israel, descoperindu-se înaintea ochilor roabelor și robilor săi, cum se descoperă un om de nimic".
\par 21 "Înaintea Domnului voi juca, a zis David către Micol; binecuvântat este Domnul, Cel ce m-a ales pe mine în locul tatălui tău și a casei lui întregi, întărindu-mă cârmuitor al poporului Domnului, Israel; cânta-voi și voi juca înaintea Domnului.
\par 22 Și încă și mai mult mă voi înjosi și voi fi încă și mai mic în ochii tăi, iar înaintea slujnicilor de care grăiești tu, voi fi în cinste".
\par 23 Și Micol, fiica lui Saul, n-a avut copii până în ziua morții ei.

\chapter{7}

\par 1 Pe când regele trăia în casa sa și-l liniștise Domnul dinspre toți vrăjmașii săi de primprejur,
\par 2 A zis regele către proorocul Natan: "Iată, eu locuiesc în casă de cedru, iar chivotul Domnului stă în cort".
\par 3 "Tot ce ai la inimă, a zis Natan către rege, mergi și fă, căci Domnul este cu tine!"
\par 4 Și chiar în noaptea aceea a fost cuvântul Domnului către Natan, zicând:
\par 5 "Mergi și spune robului Meu David: Așa grăiește Domnul: Tu oare ai să-Mi zidești casă pentru locuința Mea,
\par 6 Când Eu n-am locuit în casă din timpul în care am scos pe fiii lui Israel din Egipt și până astăzi, ci am trecut din cort în cort?
\par 7 Pe oriunde am umblat cu toți fiii lui Israel, am spus Eu, oare, măcar o vorbă cuiva din seminții, căruia i-am încredințat să păstorească poporul Meu Israel și am zis Eu oare: Pentru ce nu-Mi faceți casă de cedru?
\par 8 Și acum așa să zici robului Meu David: Așa zice Domnul Savaot: Te-am luat de la stână, de la oi, ca să fii povățuitorul poporului Meu Israel;
\par 9 Am fost cu tine pretutindeni; oriunde ai umblat, am stârpit pe toți vrăjmașii tăi dinaintea feței tale și am făcut numele tău mare, ca numele celor mari de pe pământ.
\par 10 Voi tocmi loc pentru poporul Meu, pentru Israel, îl voi înrădăcina și va trăi el în pace la locul său și mai mult nu se va mai neliniști; oamenii necredincioși nu-l vor mai strâmtora, ca mai înainte,
\par 11 Pe vremea când puneam judecători peste poporul Meu Israel. Ba te voi liniști și pe tine dinspre vrăjmașii tăi.
\par 12 Iată Domnul îți vestește că-ți va întări casa, iar când se vor împlini zilele tale și vei răposa cu părinții tăi, atunci voi ridica după tine pe urmașul tău, care va răsări din coapsele tale și voi întări stăpânirea sa.
\par 13 Acela va zidi casă numelui Meu și Eu voi întări scaunul domniei lui în veci.
\par 14 Eu voi fi aceluia tată, iar el Îmi va fi fiu; de va greși, îl voi pedepsi Eu cu toiagul bărbaților și cu loviturile fiilor oamenilor,
\par 15 Dar mila Mea nu o voi lua de la el cum am luat-o de la Saul, pe care l-am lepădat înaintea feței tale.
\par 16 Casa ta va fi neclintită, regatul tău va rămâne veșnic înaintea ta și tronul tău va sta în veci".
\par 17 Toate cuvintele acestea și toată vedenia aceasta le-a spus Natan lui David.
\par 18 Atunci s-a dus regele David și, stând înaintea feței Domnului, a zis: "Cine sunt eu, Doamne Dumnezeul meu, și ce este casa mea, de m-ai mărit așa?
\par 19 Ba încă aceasta s-a părut lucru mic în ochii Tăi, Doamne Dumnezeul meu, și ai mai vestit încă și de viitorul casei robului Tău! Este aceasta, oare lucru omenesc, Doamne Dumnezeul Meu?
\par 20 Ce mai poate să-ți spună David? Tu știi pe robul Tău, Doamne Dumnezeule!
\par 21 Pentru cuvântul Tău și după inima Ta faci aceasta, descoperind toată mărirea aceasta robului Tău.
\par 22 În toate ești mare, Doamne Dumnezeule, căci nu este asemenea ție și nu este Dumnezeu afară de Tine, după toate câte am auzit noi cu urechile noastre.
\par 23 Cine este asemenea poporului Tău Israel, singurul popor de pe pământ, pentru care a venit Dumnezeu, ca să și-I câștige de popor, să-Și preaslăvească numele Lui și să săvârșească lucruri mari și minunate, înaintea poporului Tău, pe care Tu ți l-ai câștigat de la Egipteni, izgonind popoarele și zeii lor?
\par 24 Și Tu ți-ai întărit pe poporul Tău Israel, ca popor al Tău pe veci, și Tu; Doamne, Te-ai făcut Dumnezeul lui.
\par 25 Și acum, Doamne Dumnezeule, întărește pe veci cuvântul pe care l-ai rostit despre robul Tău și despre casa lui și împlinește ceea ce i-ai sortit,
\par 26 Ca să preaînalțe numele Tău în veci și să se zică: Domnul Savaot este Dumnezeu peste Israel. Casa robului Tău David să fie tare înaintea feței Tale.
\par 27 De vreme ce Tu, Doamne Savaot, Dumnezeul lui Israel, ai descoperit robului Tău, zicând: "Îți voi face casă", apoi robul Tău și-a gătit inima sa, ca să se roage ție cu această rugăciune.
\par 28 Deci, Doamne Dumnezeul meu, Tu ești Dumnezeu și cuvintele Tale sunt neschimbate și Tu ai vestit robului Tău un astfel de bine.
\par 29 Începe acum și binecuvântează casa robului Tău, ca să fie ea veșnic  înaintea feței Tale, căci Tu, Doamne Dumnezeule, ai vestit aceasta, și, prin binecuvântarea Ta, se va face casa robului Tău binecuvântată, ca să fie înaintea Ta în veci".

\chapter{8}

\par 1 După aceasta David a lovit pe Filisteni și i-a supus și a luat David Meteg-Haama din mâna Filistenilor.
\par 2 Apoi a bătut și pe Moabiți și i-a măsurat cu funia, punându-i la pământ; și a măsurat două funii spre ucidere, și o funie spre cruțare și lăsare în viață. Atunci au ajuns Moabiții robi lui David și birnici.
\par 3 Apoi a bătut David pe Hadad-Ezer, fiul lui Rehob, regele din Țoba, pe când acesta mergea ca să-și întemeieze din nou domnia sa la râul Eufratului;
\par 4 Și a luat David de la el o mie șapte sute de călăreți și douăzeci de mii de pedestrași și a tăiat David vinele la toți caii de la care, lăsând pentru sine din ei numai pentru o sută de care.
\par 5 Atunci au venit Sirienii din Damasc în ajutor lui Hadad-Ezer, regele Țobei; însă David a omorât douăzeci și două de mii de Sirieni.
\par 6 Și a pus David oști de pază în Siria Damascului, iar Sirienii au ajuns robi și birnici lui David. Domnul însă a păzit pe David pretutindeni unde s-a dus.
\par 7 Atunci a luat David scuturile cele de aur care s-au găsit la robii lui Hadad-Ezer și le-a dus la Ierusalim.
\par 8 Pe acestea le-a luat apoi Șișac, regele Egiptului, în timpul năvălirii lui asupra Ierusalimului, în zilele lui Roboam, fiul lui Solomon. Iar din Tebah și Beritai, cetățile lui Hadad-Ezer, regele David a luat foarte multă aramă.
\par 9 Auzind Tou, regele Hamatului, că David a bătut toată oștirea lui Hadad-Ezer,
\par 10 A trimis pe Hadoram, fiul său, la regele David să-l salute și să-i mulțumească, pentru că s-a războit cu Hadad-Ezer și l-a biruit. Căci Hadad-Ezer se afla în război cu Tou. Iar în mâinile lui Hadoram se aflau vase de argint, de aur și de aramă.
\par 11 Pe acestea încă le-a hărăzit David Domnului, împreună cu aurul Și argintul pe care îl afierosise din cele luate de la toate popoarele supuse: de la Sirieni, Filisteni și Amaleciți și din prada de la Hadad-Ezer, fiul lui Rehob, regele Țobei.
\par 12 Astfel și-a făcut David nume, întorcându-se de la înfrângerea celor optsprezece mii de Sirieni din Valea Sărată.
\par 13 Apoi a pus oștiri de pază în Idumeea; în toată Idumeea a pus oștiri de pază și toți Idumeii au ajuns robii lui David.
\par 14 Iar Domnul a păzit pe David pretutindeni pe unde a fost.
\par 15 Și a domnit David peste tot Israelul, făcând judecată și dreptate în tot poporul său.
\par 16 Ioab, fiul lui Țeruia, era căpetenia oștirii, iar Iosafat, fiul lui Ahilud, era cronicar.
\par 17 Țadoc, fiul lui Ahitub, și Ahimelec, fiul lui Abiatar, au fost preoți, iar Seraia a fost dregător.
\par 18 Benaia, fiul lui Iehoiada, a fost căpetenie peste Cheretieni și Peletieni, iar fiii lui David erau cei dintâi la curte.

\chapter{9}

\par 1 "N-a mai rămas, oare, cineva din casa lui Saul? - zise David. Eu i-aș arăta milă din pricina lui Ionatan".
\par 2 În casa lui Saul însă fusese un rob, cu numele Țiba. Pe acesta l-au chemat la David și i-a zis regele: "Tu ești Țiba?" "Eu, robul tău", a răspuns acesta.
\par 3 "Nu cumva mai este cineva din casa lui Saul? a întrebat regele, că i-aș arăta mila lui Dumnezeu". "Ba este, fiul lui Ionatan, cel șchiop de picioare", a zis Țiba către rege.
\par 4 Iar regele zise: "Unde este?" "Iată, a răspuns Țiba regelui, el este în casa lui Machir, fiul lui Amiel, din Lodebar".
\par 5 Și a trimis regele David de l-au luat de la casa lui Machir, fiul lui Amiel, din Lodebar.
\par 6 Și a venit Mefiboșet, fiul lui Ionatan, la David și, căzând cu fața la pământ, s-a închinat regelui. Și a zis regele: "Mefiboșet!" "Da, robul tău!" a răspuns acesta.
\par 7 "Nu te teme, a zis regele David, că eu îți voi arăta milă pentru tatăl tău, Ionatan, și-ți voi întoarce toate țarinile lui Saul, bunicul tău, și tu vei mânca totdeauna pâine la masa mea".
\par 8 Atunci s-a închinat Mefiboșet șl a zis: "Ce este robul tău, de ai căutat tu la un asemenea câine mort, cum sunt eu?"
\par 9 Regele însă a chemat pe Țiba, sluga lui Saul, și i-a zis: "Toate câte au fost ale lui Saul și ale întregii lui case le dau fiului stăpânului tău:
\par 10 Deci lucrează pentru el pământul, tu cu fiii tăi și cu robii tăi, și strânge roadele lui, ca fiul stăpânului tău să aibă pâine de hrană. Mefiboșet, fiul stăpânului tău, va mânca totdeauna la masa mea".
\par 11 Țiba avea cincisprezece feciori și douăzeci de robi. Și a zis Țiba către rege: "Tot ce poruncește regele, stăpânul meu, robului său, robul tău va îndeplini".
\par 12 Și mânca Mefiboșet la masa lui David, ca unul din copiii regelui. Mefiboșet avea un copil mic cu numele Micha. Și toți cei ce trăiau în casa lui Țiba erau slugile lui Mefiboșet.
\par 13 Iar Mefiboșet era olog. Trăia în Ierusalim și mânca totdeauna la masa regelui.

\chapter{10}

\par 1 Trecând câtăva vreme, a murit regele Amoniților, iar în locul lui s-a făcut rege Hanun, fiul lui.
\par 2 Atunci a zis David: "Voi arăta milă lui Hanun, fiul lui Nahaș, pentru binefacerea ce mi-a arătat tatăl său". Apoi a trimis David pe slugile sale să mângâie pe Hanun de moartea tatălui său. Și au venit slugile lui David în țara Amoniților.
\par 3 Însă căpeteniile Amoniților au zis către Hanun, domnul lor: "Socotiți, oare, că David din dragoste către tatăl tău a trimis mângâietori la tine? Nu cumva a trimis David slugile sale la tine ca să iscodească cetatea și să vadă ce este în ea și apoi s-o dărâme?
\par 4 Atunci a luat Hanun pe slugile lui David și a ras fiecăruia jumătate de barbă și le-a tăiat hainele pe jumătate, până la șolduri, și apoi le-a dat drumul.
\par 5 Când i s-a spus aceasta lui David, acesta a trimis înaintea lor, deoarece erau foarte batjocoriți. Și a poruncit regele să li se spună: "Rămâneți în Ierihon până vă vor crește bărbile și atunci vă veți întoarce".
\par 6 Amoniții insă, văzând că s-au făcut nesuferiți înaintea lui David, au trimis să tocmească cu plată pe Sirienii din Bet-Rehov și pe Sirienii din Țoba, douăzeci de mii de pedestrași, de la regele Amalecit din Maacha o mie de oameni și din Iștov douăsprezece mii de oameni.
\par 7 Când a auzit de aceasta, David a trimis pe Ioab cu toată oștirea de viteji.
\par 8 Și ieșind, Amoniții s-au așezat în rânduri de luptă la poartă, iar Sirienii din Țoba, din Rehov, din Iștov și din Maacha au stat deoparte în câmp.
\par 9 Văzând Ioab că oștirea dușmană era așezată împotriva lui și înainte și în urmă, a ales oștenii cei mai de seamă ai lui Israel și i-a pus în rânduri de luptă împotriva Sirienilor.
\par 10 Iar cealaltă parte de oameni a încredințat-o lui Abișai, fratele său, ca să-i pună în rânduri de luptă împotriva Amaleciților:
\par 11 Apoi a zis Ioab: "Dacă Sirienii mă vor birui pe mine, tu să mă ajuți, iar dacă Amoniții te vor birui pe tine, îți voi veni eu în ajutor.
\par 12 Fii curajos și să ne ținem cu bărbăție pentru poporul nostru și pentru cetățile Dumnezeului nostru, și Domnul va face ce va binevoi".
\par 13 După aceea a întrat Ioab și poporul ce era cu el în luptă cu Sirienii, dar aceștia au fugit de el.
\par 14 Amoniții, văzând că Sirienii pleacă, au fugit și ei de Abișai și s-au dus în cetate. Întorcându-se Ioab de la Amoniți, a intrat în Ierusalim.
\par 15 Sirienii însă, văzând că au fost biruiți de Israeliți, s-au adunat la un loc.
\par 16 Și a trimis Hadad-Ezer de au chemat pe Sirienii cei de peste râul Eufrat și aceștia au venit la Helam, iar Sovac, căpetenia oștirii lui Hadad-Ezer, îi conducea.
\par 17 Când s-a spus de aceasta lui David, acesta a adunat pe toți Israeliții și, trecând Iordanul, a venit la Helam. Sirienii s-au așezat împotriva lui David și s-au bătut cu el.
\par 18 Dar au fugit Sirienii de Israeliți și David a nimicit Sirienilor șapte sute de care și patruzeci de mii de călăreți; ba a lovit și pe căpetenia Sovac, care a și murit acolo.
\par 19 Și când regii supuși lui Hadad-Ezer au văzut că sunt învinși de Israeliți, au încheiat pace cu Israeliții și s-au supus acestora. Iar Sirienii s-au temut să mai dea ajutor Amoniților.

\chapter{11}

\par 1 Peste un an, pe vremea când regii pornesc la război, David a trimis pe Ioab și slugile sale cu el și pe toți Israeliții și aceștia au lovit pe Amoniți și au împresurat Raba;
\par 2 Dar David a rămas în Ierusalim. Odată, spre seară, sculându-se David din pat și plimbându-se pe acoperișul casei domnești, a văzut de pe acoperiș o femeie scăldându-se, și femeia aceasta era foarte frumoasă.
\par 3 Atunci a trimis David să se cerceteze cine este acea femeie. Și i s-a spus că este Batșeba, fiica lui Eliam, femeia lui Urie Heteul.
\par 4 Apoi David a trimis slugile să o ia; ea a venit la el și el s-a culcat cu ea. Iar dacă s-a curățit ea de necurăția ei, s-a întors la casa sa.
\par 5 Femeia aceasta a rămas însărcinată și a trimis de s-a vestit lui David, zicând: "Eu sunt însărcinată".
\par 6 Atunci a trimis David să se zică lui Ioab: "Trimite la mine pe Urie Heteul!" Și a trimis Ioab pe Urie la David.
\par 7 Venind Urie la David, acesta l-a întrebat de sănătatea lui Ioab, de starea poporului și de mersul războiului.
\par 8 Apoi a zis David către Urie: "Du-te acasă și-ți spală picioarele!" Ieșind Urie din casa regelui, în urma lui i s-a trimis un dar de la masa regelui.
\par 9 Dar Urie a dormit la poarta casei regelui cu toate slugile stăpânului său și nu s-a dus la casa sa.
\par 10 Și i s-a spus lui David, zicând: "Urie nu s-a dus la casa sa". "Iată, a zis David către Urie, tu ai venit de pe drum, de ce nu te-ai dus la casa ta?"
\par 11 Iar Urie a zis: "Chivotul Domnului și Israel și Iuda sunt în corturi; stăpânul meu Ioab și robii domnului meu sunt în tabără, iar eu să mă duc la casa mea să mănânc, să beau și să mă culc cu femeia mea? Mă jur pe viața ta și pe viața sufletului tău că nu voi face aceasta".
\par 12 "Rămâi aici și ziua aceasta, a zis David lui Urie, iar mâine îți voi da drumul". Și a rămas Urie în Ierusalim în ziua aceea până a doua zi.
\par 13 Și l-a chemat David și a mâncat Urie înaintea lui și a băut și David i-a arătat cinste. Dar seara Urie s-a dus să se culce în patul său cu robii stăpânului său, iar la casa sa nu s-a dus.
\par 14 Dimineața David a scris scrisoare lui Ioab și a trimis-o pe Urie.
\par 15 În scrisoarea aceea el scria așa: "Puneți pe Urie unde va fi luptă mai crâncenă și retrageți-vă de la el, ca să fie lovit și ucis".
\par 16 De aceea, când Ioab a împresurat cetatea, a pus pe Urie într-un astfel de loc, de care știa că este apărat de oameni viteji.
\par 17 Și au ieșit oamenii din cetate și s-au luptat cu Ioab și au căzut câțiva din popor, din slugile lui David. Acolo a fost ucis și Urie Heteul.
\par 18 Atunci a trimis Ioab să se facă cunoscut lui David tot mersul luptei.
\par 19 Și a poruncit trimisului și i-a zis: "După ce vei povesti regelui tot mersul luptei,
\par 20 Și vei vedea că regele se mânie și-ți zice: "De ce v-ați apropiat să vă luptați așa aproape de cetate? Nu știați voi că de pe zidurile cetății pot să arunce în voi?
\par 21 Cine oare a ucis pe Abimelec, fiul lui Ierubaal? Au nu o femeie, care a aruncat în el de pe zid o bucată de râșniță și l-a lovit și el a murit în Tebeț? De ce v-ați apropiat așa tare de cetate?" Atunci să-i zici: "Și robul tău Urie Heteul a fost lovit și a murit".
\par 22 S-a dus deci trimisul lui Ioab la rege în Ierusalim și, ajungând, a povestit lui David despre toate, pentru care fusese trimis de Ioab și de tot mersul luptei. Și s-a mâniat David pe Ioab și a zis trimisului: "De ce v-ați apropiat așa tare de cetate să vă luptați? Nu știați voi oare că vă pot lovi de pe zidurile cetății? Cine a ucis pe Abimelec, fiul lui Ierubaal? Oare nu o femeie care a aruncat în el de pe zid cu o bucată de râșnită, și a murit în Tebeț? De ce v-ați apropiat așa tare de zid?"
\par 23 Atunci trimisul a spus lui David: "Acei oameni ne-au răpus pe noi și au ieșit asupra noastră în câmp, dar noi i-am alungat până la poartă.
\par 24 Atunci au început a săgeta arcașii de pe ziduri asupra robilor tăi și au murit câțiva din robii regelui; și a murit de asemenea și robul tău Urie Heteul".
\par 25 Atunci David a zis: "Așa să spui lui Ioab: Să nu te tulbure lucrul acesta, căci sabia o dată mănâncă pe unul, altă dată mănâncă pe altul. Întețește lupta împotriva cetății și dărâm-o. Așa să-l încurajezi".
\par 26 Și auzind femeia lui Urie că a murit Urie, bărbatul ei, a plâns după el.
\par 27 Iar dacă s-a isprăvit vremea plângerii, a trimis David și a luat-o în casa sa și ea a ajuns femeia lui și i-a născut un fiu. Fapta aceasta, pe care a făcut-o David, a fost rea înaintea Domnului.

\chapter{12}

\par 1 Atunci a trimis Domnul pe Natan proorocul la David și a venit acela la el și i-a zis: "Erau într-o cetate doi oameni: unul bogat și altul sărac.
\par 2 Cel bogat avea foarte multe vite mari și mărunte,
\par 3 Iar cel sărac n-avea decât o singură oiță, pe care el o cumpărase de mică și o hrănise și ea crescuse cu copiii lui. Din pâinea lui mâncase și ea și se adăpase din ulcica lui, la sânul lui dormise și era pentru el ca o fiică.
\par 4 Dar iată că a venit la bogat un călător, și gazda nu s-a îndurat să ia din oile sale sau din vitele sale, ca să gătească cină pentru călătorul care venise la el, ci a luat oița săracului și a gătit-o pe aceea pentru omul care venise la el".
\par 5 Atunci s-a mâniat David cumplit asupra acelui om și a zis către Natan: "Precum este adevărat că Domnul este viu, tot așa este de adevărat că omul care a făcut aceasta este vrednic de moarte;
\par 6 Pentru oaie el trebuie să întoarcă împătrit, pentru că a făcut una ca aceasta și pentru că n-a avut milă".
\par 7 Atunci Natan a zis către David: "Tu ești omul care a făcut aceasta. Așa zice Domnul Dumnezeul lui Israel: Eu te-am uns rege pentru Israel și Eu te-am izbăvit din mâna lui Saul,
\par 8 Ți-am dat casa domnului tău și femeile domnului tău la sânul tău; ți-am dat ție casa lui Israel și a lui Iuda și, dacă aceasta este puțin pentru tine, ți-aș mai adăuga.
\par 9 Pentru ce însă ai nesocotit tu cuvântul Domnului, făcând rău înaintea ochilor Lui? Pe Urie Heteul tu l-ai lovit cu sabia, pe femeia lui ți-ai luat-o de soție, iar pe el l-ai ucis cu sabia Amoniților.
\par 10 Deci nu se va depărta sabia de deasupra casei tale în veac, pentru că tu M-ai nesocotit pe Mine și ai luat pe femeia lui Urie Heteul, ca să-ți fie nevastă.
\par 11 Așa zice Domnul: Iată Eu voi ridica asupra ta rău chiar din casa ta și voi lua pe femeile tale înaintea ochilor tăi și le voi da aproapelui tău și se va culca acela cu femeile tale în văzul soarelui acestuia.
\par 12 Tu ai făcut pe ascuns, iar Eu voi face aceasta înaintea a tot Israelul și înaintea soarelui". "Am păcătuit înaintea Domnului", a zis David către Natan.
\par 13 "Și Domnul a ridicat păcatul de deasupra ta, a zis Natan, și tu nu vei muri.
\par 14 Dar fiindcă tu prin această faptă ai dat vrăjmașilor Domnului pricină să-L hulească, de aceea va muri fiul ce ți se va naște".
\par 15 Apoi s-a dus Natan la casa sa, iar Domnul a lovit copilul pe care i-l născuse lui David femeia lui Urie și acela s-a îmbolnăvit.
\par 16 Și s-a rugat David Domnului pentru copil, a postit și, ducându-se deoparte, a petrecut noaptea întins pe pământ.
\par 17 Atunci au intrat la el bătrânii casei lui ca să-l ridice de la pământ, dar el n-î voit și nici n-a mâncat pâine cu ei.
\par 18 După șapte zile a murit copilul și slugile lui David se temeau să-i spună că a murit copilul. Căci ei își ziceau: "Când copilul era încă viu și noi îl mângâiam, el nu ne băga în seamă; cum să-i spunem acum: A murit copilul? Ar putea să facă vreun rău".
\par 19 Dar văzând David că slugile sale șoptesc între ele, a priceput că a murit copilul și le-a întrebat: "A murit copilul?" "A murit", i s-a răspuns.
\par 20 Atunci David s-a sculat de la pământ, s-a spălat, s-a uns și și-a schimbat hainele și s-a dus în casa Domnului și s-a rugat. Întorcându-se apoi acasă, a cerut să i se dea pâine și a mâncat.
\par 21 Și i-au zis slugile: "Ce va să zică aceasta? Când copilul era încă în viață, ai postit, ai plâns și n-ai dormit; iar după ce copilul a murit, te-ai sculat, ai mâncat și ai băut?"
\par 22 "Câtă vreme copilul era viu, a zis David, am postit și am plâns, căci socoteam: Cine știe, poate mă va milui Domnul și va trăi copilul.
\par 23 Dar acum el a murit; de ce să mai postesc? Îl mai pot eu, oare, întoarce? Eu mă voi duce la el, iar el nu se va mai întoarce la mine".
\par 24 Și a mângâiat David pe Batșeba, femeia sa, a intrat la ea, s-a culcat cu ea și ea a zămislit și a mai născut un fiu și i-a pus numele Solomon. Domnul l-a iubit pe acesta,
\par 25 Și a trimis pe proorocul Natan, și acesta i-a pus numele Iedida, adică iubitul Domnului, cum îi spusese Domnul.
\par 26 Ioab însă lupta împotriva cetății Amoniților, Raba, și aproape luase cetatea domnească.
\par 27 Atunci a trimis Ioab la David să i se spună: "Am tăbărât asupra cetății Raba și am luat cetatea prin apă.
\par 28 Adună acum celălalt popor și vino asupra cetății și o ia; căci de o voi lua eu, atunci se va slăvi numele meu".
\par 29 Atunci a adunat David tot poporul și s-a dus asupra cetății Raba, s-a luptat împotriva ei și a luat-o.
\par 30 Și a luat David de pe capul regelui ei coroana, care era de un talant de aur și cu pietre scumpe, și a pus-o pe capul său; a luat și foarte multă pradă din cetate.
\par 31 Iar pe poporul care se afla în ea l-a scos și l-a pus sub fierăstrău și sub grapă de fier și sub securi de fier și i-a aruncat în cuptoarele de ars cărămidă. Așa a făcut el cu toate cetățile Amoniților. După aceea David s-a întors la Ierusalim cu tot poporul.

\chapter{13}

\par 1 După aceea s-au petrecut următoarele: Abesalom, fiul lui David, avea o soră frumoasă, cu numele Tamara. Pe aceasta o iubea Amnon, alt fiu al lui David.
\par 2 Și s-a chinuit Amnon până într-atâta, că s-a îmbolnăvit din pricina surorii sale Tamara, căci aceasta era fecioară și lui Amnon i se părea greu să-i facă ceva.
\par 3 Avea însă Amnon un prieten, anume Ionadab, fiul lui Șama, fratele lui David.
\par 4 Ionadab era om foarte șiret. Acesta i-a zis: "Fiul regelui, de ce slăbești tu așa pe fiecare zi? Spune mie!" "Iubesc pe Tamara, sora lui Abesalom, fratele meu", a zis Amnon către el.
\par 5 "Culcă-te în patul tău, i-a zis Ionadab, și te fă bolnav; iar când tatăl tău va veni să te cerceteze, să-i zici: Lasă să vină Tamara, sora mea, să mă întărească cu hrană, pregătind mâncare înaintea ochilor mei, ca să văd și să mănânc din mâinile ei!"
\par 6 Și s-a culcat Amnon și s-a făcut bolnav și a venit regele să-l cerceteze. Atunci Amnon a zis către rege: "Lasă pe Tamara, sora mea, să vină și să coacă înaintea ochilor mei o turtă sau două și să mănânc din mâinile ei".
\par 7 Și a trimis David la Tamara acasă să-i spună: "Du-te acasă la Amnon, fratele tău, și-i fă de mâncare!"
\par 8 și s-a dus ea acasă la fratele său Amnon; acesta sta culcat. Și a luat ea făină, a frământat-o, a făcut înaintea ochilor lui turte și le-a copt;
\par 9 Apoi a luat tigaia și a pus-o înaintea lui, dar el n-a vrut să mănânce. și a zis Amnon: "Să iasă toți de la mine!"
\par 10 Și au ieșit de la el toți oamenii. Apoi Amnon a zis către Tamara: "Du mâncarea în odaia cea din fund și voi mânca acolo din mâinile tale". Atunci a luat Tamara turtele ce le gătise și le-a dus lui Amnon, fratele său, în odaia cea din fund.
\par 11 Dar când le-a pus înaintea lui ca să mănânce, el a apucat-o și i-a zis: "Vino și te culcă cu mine, sora mea!"
\par 12 "Nu, frate, a zis ea, nu mă necinsti, căci aceasta nu se face în Israel; nu face ticăloșia aceasta!
\par 13 Căci unde mă voi duce eu cu necinstea mea? Și tu vei fi în Israel cu unul din cei fără de minte. Vorbește cu regele și el nu se va împotrivi să mă dea după tine".
\par 14 El însă n-a vrut să asculte cuvintele ei, ci a silit-o și s-a culcat cu ea și a necinstit-o.
\par 15 După aceea a urât-o Amnon cu ura cea mai mare, așa încât ura cu care a urât-o el era mai mare decât iubirea pe care o avusese către ea. Și i-a zis ei Amnon: "Scoală și pleacă!"
\par 16 "Ba nu, frate, i-a zis Tamara, a mă alunga este un rău și încă și mai mare decât cel dintâi, pe care mi l-ai făcut tu mie".
\par 17 Dar el n-a vrut să o asculte, ci a chemat pe omul său care-l slujea și i-a zis: "Alungă pe aceasta de la mine afară și încuie ușa după ea".
\par 18 Ea însă era îmbrăcată cu haină pestriță, căci astfel de haine purtau pe deasupra fetele regelui care erau fecioare. Și a scos-o sluga afară și a încuiat ușa după ea.
\par 19 Iar Tamara și-a presărat cenușă pe capul său și-a rupt haina cea pestriță, cu care era îmbrăcată și, punându-și mâinile pe cap, mergea așa și striga.
\par 20 Atunci a zis către ea Abesalom, fratele ei: "Nu cumva Amnon, fratele tău, a umblat cu tine? Dar taci acum, sora mea, căci el este fratele tău; nu-ți zdrobi inima pentru fapta aceasta". Și a șezut Tamara părăsită în casa lui Abesalom, fratele său.
\par 21 Și a auzit regele David de toate cele întâmplate și s-a mâniat foarte tare, dar n-a stricat inima lui Amnon, fiul său, căci îl iubea, pentru că era întâiul său născut.
\par 22 Abesalom însă nu grăia cu Amnon nici bine, nici rău, căci Abesalom ura pe Amnon, pentru că acesta necinstise pe Tamara, sora sa.
\par 23 Iar după doi ani, pe vremea când tundeau oile lui Abesalom în Baal-Hațor, care se află în Efraim, a chemat Abesalom pe toți fiii regelui.
\par 24 și venind Abesalom la rege, a zis: "Iată acum este tunsul oilor la robul tău; deci să meargă regele și slugile sale la robul tău!"
\par 25 Regele însă a zis către Abesalom: "Ba nu, fiul meu, nu vom merge cu toții, ca să nu te împovărăm". Abesalom însă l-a rugat cu mare stăruință, dar el n-a vrut să se ducă, ci l-a binecuvântat.
\par 26 Atunci Abesalom a zis către el: "Dacă nu, să meargă cu noi măcar Amnon, fratele meu". "De ce să meargă el cu tine?" a zis regele.
\par 27 Dar stăruind Abesalom, regele a dat drumul lui Amnon și la toți fiii regelui să se ducă cu el. Și a făcut Abesalom ospăț, ca ospățul unui rege.
\par 28 Și a mai poruncit Abesalom slugilor sale, zicând: "Luați seama, că îndată ce inima lui Amnon se va veseli de vin și când eu voi zice: Loviți pe Amnon, să-l ucideți și să nu vă temeți; eu vă poruncesc aceasta, să fiți curajoși și viteji!"
\par 29 Și au făcut slugile lui Abesalom cu Amnon cum le poruncise Abesalom. Atunci s-au sculat fiii regelui cu toții și, încălecând fiecare pe catârul său, au fugit.
\par 30 Și încă pe cale fiind ei, a ajuns la David vestea că Abesalom a omorât pe coti fiii regelui și n-a mai rămas nici unul din ei.
\par 31 Atunci s-a sculat regele și și-a rupt hainele sale și s-a aruncat la pământ; și toate slugile sale, care stăteau înaintea sa, și-au rupt veșmintele lor.
\par 32 Atunci Ionadab, fiul lui Șama, fratele lui David, a zis: "Să nu creadă regele, stăpânul meu, că au omorât pe toți băieții, fiii regelui; numai singur Amnon a murit, căci Abesalom avea aceasta în gând încă din ziua când Amnon a necinstit pe sora sa, Tamara.
\par 33 Deci, regele, stăpânul meu, să nu se tulbure cu gândul că ar fi murit toți fiii regelui, căci numai Amnon singur a murit".
\par 34 Atunci a fugit Abesalom, iar omul de strajă, ridicându-și ochii săi, a privit și iată popor mult venea pe drumul de pe coasta muntelui. Și venind straja, a dat de veste regelui, zicând: "Am văzut oameni pe drumul Bahurim, de pe coasta muntelui".
\par 35 Atunci Ionadab a zis către rege: "Iată vin fiii regelui; cum a zis robul tău așa și este".
\par 36 Și cum a sfârșit el vorbele acestea, iată au sosit și fiii regelui și au ridicat strigăt și au plâns. Și a plâns și regele însuși și toate slugile lui plângere mare.
\par 37 Iar Abesalom a fugit și s-a dus la Talmai, fiul lui Amihud, regele Gheșurului. Și a plâns regele David după fiul său în toate zilele.
\par 38 Iar Abesalom, fugind și ducându-se în Gheșur, a stat acolo trei ani.
\par 39 Dar regele David nu s-a apucat să urmărească pe Abesalom, căci se mângâiase de moartea lui Amnon.

\chapter{14}

\par 1 Cunoscând Ioab, fiul Țeruiei, că inima regelui s-a întors spre Abesalom,
\par 2 A trimis Ioab la Tecoa și a luat de acolo o femeie înțeleaptă și i-a zis: "Fă-te că ești bocitoare, îmbracă-te cu haine de jale, nu te unge cu untdelemn și fii ca a femeie care a plâns zile multe după un mort;
\par 3 Și du-te la rege și zi către el așa și așa". Și i-a spus Ioab ce anume să zică.
\par 4 Venind deci femeia cea din Tecoa la rege și căzând cu fața la pământ, s-a închinat și a zis: "Ajutor, o, rege, ajutor!"
\par 5 "Ce ai?" a zis regele către ea. "Sunt de mult văduvă, a zis ea, căci mi-a murit bărbatul.
\par 6 Și avea roaba ta doi feciori și aceștia s-au sfădit în țarină; și, neavând cine-i despărți, a lovit unul din ei pe celălalt și l-a omorât.
\par 7 Și iată s-au sculat toate rudele asupra roabei tale și zic: "Dă-ne pe ucigașul fratelui său să-l omorâm pentru sufletul fratelui său pe care l-a pierdut el și vom pierde chiar și pe moștenitorul lui". Și așa vor să stingă ei și cea din urmă scânteie a mea, ca să nu mai lase bărbatului meu nici nume, nici urmași pe fața pământului".
\par 8 "Mergi în pace la casa ta, a zis regele către femeie, căci voi da poruncă pentru tine".
\par 9 Dar femeia cea din Tecoa a zis către rege: "O, rege, stăpânul meu, asupra mea să fie vina și asupra casei tatălui meu; iar regele și tronul lui este nevinovat".
\par 10 "pe cel ce va fi împotriva ta, a zis regele, să-l aduci la mine și mai mult nu te va mai atinge".
\par 11 "Poruncește, o, rege, în numele Domnului Dumnezeului tău, a zis ea, ca să nu se înmulțească răzbunătorii sângelui și să nu piardă pe fiul meu". "Viu este Domnul, a zis regele, nici un fir de păr de al fiului tău nu va cădea pe pământ!"
\par 12 "Îngăduie roabei tale, a zis femeia, să mai spun o vorbă regelui, stăpânului meu".
\par 13 "Spune", a zis el. "Pentru ce cugeți tu așa împotriva poporului Domnului?, a zis femeia. Rostind cuvântul acesta, regele s-a osândit pe sine însuși, pentru că nu aduce înapoi pe izgonitul său.
\par 14 Noi vom muri și vom fi ca apa vărsată pe pământ, care nu se mai poate aduna. Dumnezeu însă nu voiește să piardă sufletul și se gândește cum ar face să nu lepede de la Sine nici pe cel înlăturat.
\par 15 Și acum eu am venit să spun regelui, stăpânul meu, cuvintele acestea, pentru că poporul mă sperie și roaba ta a zis: "Am să grăiesc eu cu regele, să văd nu va face el după cuvântul roabei sale?
\par 16 De bună seamă regele va asculta și va izbăvi pe roaba sa din mâna oamenilor care voiesc să mă piardă împreună cu fiul meu din moștenirea lui Dumnezeu.
\par 17 Și roaba ta a zis: Să fie cuvântul regelui, stăpânul meu, spre mângâierea mea, căci regele, stăpânul meu, este ca îngerul lui Dumnezeu și poate ca să asculte și bune și rele, și Domnul Dumnezeul tău va fi cu tine".
\par 18 Și răspunzând, regele a zis către femeie: "Să nu ascunzi de mine ceea ce am să te întreb!" "Grăiește, a zis femeia, o, rege, stăpânul meu!"
\par 19 "Nu cumva este mâna lui Ioab în tot ce spui tu?" a zis regele. Iar femeia i-a răspuns și a zis: "Să trăiască sufletul tău, o, rege! Nu pot să mă abat nici la dreapta, nici la stânga de la ceea ce a zis regele, stăpânul meu. Adevărat, robul tău Ioab mi-a poruncit și el a pus în gura roabei tale toate cuvintele acestea;
\par 20 Și tot robul tău m-a învățat ca prin pildă să dau lucrului această înfățișare. Dar regele, stăpânul meu, este înțelept, cum este înțelept îngerul lui Dumnezeu, ca să cunoască tot ce este pe pământ".
\par 21 Atunci regele a zis către Ioab: "Iată, fac lucrul acesta; du-te dar și adu înapoi pe băiatul Abesalom".
\par 22 Atunci Ioab a căzut cu fața la pământ și s-a închinat și a binecuvântat pe rege, zicând: "Acum robul tău cunoaște că a aflat bunăvoință înaintea ochilor tăi, o, rege, stăpânul meu, de vreme ce regele a făcut cum a zis robul tău".
\par 23 Și sculându-se, Ioab s-a dus în Gheșur și a adus pe Abesalom la Ierusalim.
\par 24 Și a zis regele: "Să se întoarcă la casa sa, dar fața mea nu o va vedea". Și s-a întors Abesalom la casa sa; dar fața regelui n-a văzut-o.
\par 25 În tot Israelul nu era bărbat așa de frumos ca Abesalom și așa de lăudat ca el; din tălpile picioarelor și până în creștetul capului nu avea nici o meteahnă.
\par 26 Când își tundea capul său - și și-l tundea în fiecare an, pentru că-l îngreuia - părul de pe capul lui cântărea două sute de sicli, după cântarul regesc.
\par 27 Și i s-au născut lui Abesalom trei băieți și o fată, anume Tamara. Aceasta a fost o femeie frumoasă la chip și a ajuns soția lui Roboam, fiul lui Solomon și i-a născut pe Abia.
\par 28 Abesalom a rămas în Ierusalim doi ani, dar fața regelui n-a văzut-o.
\par 29 Și a trimis Abesalom după Ioab, ca să-l trimită la rege, dar acesta n-a vrut să vină la el. și a trimis și a doua oară și acesta tot n-a vrut să vină.
\par 30 Atunci a zis Abesalom slugilor sale: "Vedeți voi partea de țarină a lui Ioab, care este lângă a mea și unde el are semănat orz? Duceți-vă și-i dați foc!" Și au ars slugile lui Abesalom acea parte de țarină cu foc. Deci venind slugile lui Ioab la. acesta cu hainele rupte, au zis: "Slugile lui Abesalom au ars ogorul tău cu foc".
\par 31 Atunci s-a sculat Ioab și a venit la Abesalom acasă și i-a zis: "Pentru ce slugile tale au ars cu foc ogorul meu?"
\par 32 Iar Abesalom a zis: "Iată, eu am trimis la tine și am zis: Vino încoace, ca să te trimit la rege să-i zici: Pentru ce am venit din Gheșur? Mai bine-mi era să fi rămas acolo. Vreau să văd fața regelui. De sunt vinovat, atunci ucide-mă".
\par 33 Și s-a dus Ioab la rege și i-a spus aceasta. Și a chemat regele pe Abesalom și a venit acesta la rege și, căzând cu fața sa la pământ înaintea regelui, i s-a închinat, iar regele a sărutat. pe Abesalom.

\chapter{15}

\par 1 După aceasta Abesalom și-a înjghebat care și cai și cincizeci de bărbați, care mergeau înaintea sa.
\par 2 Și se scula Abesalom dis-de-dimineață, se oprea la poartă lângă cale, și, când venea cineva la rege să se judece pentru vreo pricină, Abesalom îl chema la sine și-l întreba: "Din ce cetate ești tu?" Și când acela îi răspundea: "Robul tău este din cutare trib al lui Israel",
\par 3 Atunci Abesalom îi zicea: "Iată pricina ta este bună și dreaptă, dar la rege n-are cine să te asculte".
\par 4 Și mai zicea Abesalom: "O, de m-ar pune pe mine judecător în țara aceasta, ar veni la mine oricine ar avea neînțelegeri și judecată și eu l-aș judeca drept".
\par 5 Și de se apropia cineva să i se închine, el își întindea mâna și-l îmbrățișa și-l săruta.
\par 6 Așa se purta Abesalom cu tot israelitul care venea pentru judecată la rege și a intrat Abesalom la inima Israeliților.
\par 7 După patruzeci de ani de domnie a lui David, a zis Abesalom către rege: "Mă duc la Hebron să-mi împlinesc o făgăduință, pe care am făcut-o Domnului,
\par 8 Căci eu, robul tău, pe când trăiam la Gheșur în Siria, am făcut făgăduința aceasta: De mă va întoarce Domnul la Ierusalim, voi aduce jertfă Domnului".
\par 9 Și i-a zis regele: "Du-te cu pace!" Și el s-a sculat și s-a dus la Hebron.
\par 10 Atunci a trimis Abesalom șapte fete la toate triburile lui Israel, zicând: "Când veți auzi sunetul cornului, să ziceți: Abesalom s-a făcut rege în Hebron".
\par 11 Și s-au dus cu Abesalom două sute de oameni din Ierusalim, care fuseseră poftiți de el, dar s-au dus din nevinovăție, neștiind ce este la mijloc.
\par 12 În timpul jertfei, Abesalom a trimis și a chemat pe Ahitofel Ghiloneanul, sfetnicul lui David, din cetatea lui, Ghilo. Și s-a făcut răzvrătire mare și curgea poporul și se înmulțea împrejurul lui Abesalom.
\par 13 Deci a venit un vestitor la David și a zis: "Inima Israeliților a înclinat în partea lui Abesalom".
\par 14 Iar David a zis către toate slugile sale, care erau cu el în Ierusalim: "Sculați-vă să fugim, căci nu vom scăpa de Abesalom. Grăbiți-vă să plecăm, ca să nu ne ajungă și să ne prindă, ca să nu aducă necaz asupra noastră și să strice cetățile cu sabia".
\par 15 Și slugile regelui au zis către rege: "La tot ce va voi regele, stăpânul nostru, noi slugile tale suntem gata".
\par 16 Și a ieșit regele pe jos și după el a mers toată casa lui. Regele însă a lăsat zece femei din concubinele sale, ca să păzească casa.
\par 17 Și au plecat regele și tot poporul pe jos și s-au oprit la Bet-Merhac.
\par 18 Toate slugile lui mergeau pe lângă el, iar toți Cheretienii și toți Peletienii și toți Gateienii, ca la șase sute de oameni, care veniseră împreună cu el din Gat, mergeau înaintea regelui.
\par 19 "Pentru ce mergi și tu cu noi? a zis regele către Itai din Gat. Întoarce-te și rămâi cu regele, căci tu ești străin și ai venit aici din țara ta.
\par 20 Ieri ai venit și astăzi să te silesc să mergi cu noi? Eu mă duc unde se va întâmpla. Întoarce-te și întoarce și pe frații tăi cu tine. Domnul să facă milă și dreptate cu tine".
\par 21 "Precum e adevărat că Domnul este viu, a răspuns Itai regelui, și precum este viu regele, stăpânul meu, tot așa este de adevărat că oriunde va fi regele, stăpânul meu, la viață și la moarte, acolo va fi și robul tău".
\par 22 "Atunci, a zis regele David către Itai, vino și umblă cu mine". Și s-a dus Itai din Gat și toți oamenii lui și toți copiii care erau cu el.
\par 23 Și a plâns toată țara cu glas mare și tot poporul a trecut pârâul Chedron și a trecut și regele pârâul Chedron și s-a dus tot poporul cu regele pe calea spre pustiu.
\par 24 Și iată era acolo și Țadoc preotul, împreună cu toți Leviții care duceau chivotul legământului Domnului din Betar și au pus acolo chivotul lui Dumnezeu; iar Abiatar a stat pe un loc înalt până ce a ieșit tot poporul din cetate.
\par 25 "Întoarce chivotul lui Dumnezeu în cetate, a zis regele către Țadoc, ca să stea la locul lui. De voi afla milă în ochii Domnului, mă va întoarce și-mi va da să-L văd pe El și locașul Lui.
\par 26 Iar dacă El îmi va zice: "Nu mai este bunăvoința Mea cu tine", atunci iată-mă, facă cu mine ce va binevoi".
\par 27 "Înțelegi tu? a mai zis regele către Țadoc preotul, întoarce-te cu pace în cetate, cu Ahimaaț, fiul tău și cu Ionatan, fiul lui Abiatar, amândoi fiii voștri.
\par 28 Să știți, eu am să rămân în câmpia din pustiu până îmi va veni veste de la voi".
\par 29 Atunci Țadoc și Abiatar au întors chivotul lui Dumnezeu în Ierusalim și au rămas acolo.
\par 30 Iar David s-a dus în Muntele Eleonului și, mergând, plângea; capul îi era acoperit și picioarele desculțe. Și toți oamenii care erau cu el își acoperiseră fiecare capul și mergeau plângând.
\par 31 Atunci s-a spus lui David: "Și Ahitofel este în numărul răzvrătiților cu Abesalom". "Doamne, Dumnezeul meu, a zis atunci David, risipește planurile lui Ahitofel!"
\par 32 Iar când David a ajuns pe vârful muntelui, unde s-a închinat lui Dumnezeu, iată a venit în întâmpinarea lui Hușai Archianul, cel mai bun prieten al lui David. Acesta avea haina sfâșiată și pe cap cenușă.
\par 33 "De vei merge cu mine, i-a zis David, îmi vei fi o povară.
\par 34 Iar de te vei întoarce în cetate și vei zice lui Abesalom: "Rege, frații tăi au trecut; a trecut și regele, tatăl tău, și acum eu sunt robul tău; iasă-mă cu viață. Până acum am fast robul tatălui tău, iar acum sunt robul tău". Atunci vei strica planurile lui Ahitofel cele împotriva mea.
\par 35 Iată este acolo cu tine Țadoc și Abiatar preoții și tot cuvântul ce vei auzi la casa regelui să-l spui preoților Țadoc și Abiatar.
\par 36 Acolo sunt și cei doi fii ai lor: Ahimaaț, fiul lui Țadoc și Ionatan, fiul lui Abiatar. Prin aceștia să trimiteți la mine orice veste veți auzi".
\par 37 Și a venit Hușai, prietenul lui David, în cetate. Abesalom însă intra atunci în Ierusalim.

\chapter{16}

\par 1 După ce David a trecut puțin de vârful muntelui, iată îl întâmpină Țiba, sluga lui Mefiboșet, cu o pereche de asini încărcați, pe care se aflau două sute de pâini, o sută de legături de stafide, o sută de legături de smochine și un burduf de vin.
\par 2 Și regele a zis către Țiba: "Ce sunt acestea?" "Asinii, a răspuns Țiba, sunt pentru rege, ca să umble, pâinile și fructele pentru hrana oamenilor, iar vinul, ca să bea cei ce vor slăbi în pustie".
\par 3 "Unde este fiul stăpânului tău?" a întrebat regele. "A rămas în Ierusalim", a răspuns Țiba regelui; căci a zis: "Acum casa lui Israel îmi va întoarce mie domnia tatălui meu".
\par 4 "Ale tale să fie toate câte are Mefiboșet", a zis regele către Țiba. "Să aflu milă în ochii domnului meu, regele", a răspuns Țiba, închinându-se.
\par 5 Iar când a ajuns regele David la Bahurim, ieșea de acolo un om din neamul casei lui Saul, cu numele de Șimei, fiul lui Ghera. El mergea și blestema,
\par 6 Aruncând cu pietre asupra lui David și asupra tuturor robilor lui David; Iar poporul tot și toți oamenii de luptă erau la dreapta și la stânga regelui.
\par 7 "Pleacă, pleacă, ucigașule și nelegiuitule, zicea Șimei, blestemând pe rege.
\par 8 Domnul a întors asupra ta tot sângele casei lui Saul, în locul căruia te-ai făcut tu rege și a dat Domnul domnia în mâinile lui Abesalom; fiul tău; și iată tu ești în necaz, pentru că ești băutor de sânge".
\par 9 "Pentru ce acest câine leșinat blesteamă pe domnul meu, rege?, a zis Abișai, fiul Țeruiei. Mă duc să-i iau capul".
\par 10 "Fiii Țeruiei, a zis regele, ce ne privește aceasta pe mine și pe voi? Lăsați-l să blesteme, căci Domnul i-a poruncit să blesteme pe David. Cine deci poate să-i zică: "De ce faci tu așa?"
\par 11 Apoi David a mai zis lui Abișai și tuturor slugilor sale: "Iată, dacă fiul; meu, care a ieșit din coapsele mele, caută sufletul meu, cu atât mai vârtos fiul unui veniaminean. Lăsați-l să blesteme, căci Domnul i-a poruncit.
\par 12 Poate va căuta Domnul la umilirea mea și-mi va răsplăti cu bine pentru acest blestem al lui".
\par 13 Și s-a dus David și oamenii lui în drumul lor, iar Șimei mergea pe coasta muntelui în preajma lui, mergea și blestema, aruncând spre el cu pietre și cu praf.
\par 14 Apoi, ajungând regele și tot poporul ce era cu el la Aiefim, s-a odihnit acolo.
\par 15 Abesalom însă și tot poporul lui Israel au venit în Ierusalim și împreună cu ei a venit și Ahitofel.
\par 16 Și când Hușai Archianul, prietenul lui David, a venit la Abesalom și i-a zis: "Trăiască regele!"
\par 17 Abesalom a zis către Hușai: "Așa dragoste ai tu către prietenul tău? De ce nu te-ai dus și tu cu prietenul tău?"
\par 18 "Nu, a zis Hușai către Abesalom, eu urmez pe acela pe care l-a ales Domnul și acest popor și tot Israelul; cu acela sunt eu și cu acela rămân.
\par 19 Și apoi cui am să slujesc? Oare nu fiului său? Cum am slujit tatălui tău, așa am să-ți slujesc și ție".
\par 20 "Dați-mi sfat, a zis Abesalom către Ahitofel, ce să facem!"
\par 21 "Intră la concubinele tatălui tău, a răspuns Ahitofel, pe care le-a lăsat el să păzească casa sa; și vor auzi toți Israeliții că tu ai ajuns să fii urât de tatăl tău și se vor întări mâinile tuturor celor ce sunt cu tine".
\par 22 Atunci au întins pentru Abesalom un cort pe acoperișul casei. Și a intrat Abesalom la concubinele tatălui său, înaintea ochilor a tot Israelul.
\par 23 Iar sfaturile lui Ahitofel, pe care le dădea el, se socoteau atunci ca și cum ar fi cerut cineva povață de la Dumnezeu. Așa fusese orice sfat al lui Ahitofel atât pentru David, cât și pentru Abesalom.

\chapter{17}

\par 1 "Eu, a zis Ahitofel către Abesalom, am să aleg douăsprezece mii de oameni și mă voi ridica să mă duc noaptea asta în urmărirea lui David;
\par 2 Și voi năvăli asupra lui când va fi ostenit și cu mâinile slăbănogite, și-l voi umple de groază și toți oamenii care sunt cu el se vor împrăștia și voi ucide numai pe rege,
\par 3 Iar pe oameni îi voi întoarce pe toți la tine. Și când nu va mai fi unul, al cărui suflet îl cauți tu, atunci tot poporul va fi în pace".
\par 4 Și a plăcut vorba aceasta lui Abesalom și tuturor bătrânilor lui Israel.
\par 5 "Chemați pe Hușai Archianul, a zis Abesalom, să auzim ce zice el".
\par 6 Atunci a venit Hușai la Abesalom și Abesalom i-a zis: "Iată ce zice Ahitofel; să facem oare cum zice el? Iar dacă nu, spune-mi tu!"
\par 7 "De data aceasta, a zis Hușai către Abesalom, nu este bun sfatul pe care l-a dat Ahitofel".
\par 8 "Tu cunoști pe tatăl tău și pe oamenii lui, urmă mai departe Hușai. Ei sunt viteji și foarte îndârjiți, ca ursoaica pustiului când i se răpesc puii și ca vierul sălbatic din câmp. Și apoi tatăl tău este om războinic; el nu stă să rămână cu poporul.
\par 9 Iată acum ei, de bună seamă, se ascunde în vreo peșteră sau în alt loc; și de cade cineva la cel dintâi atac asupra lor, se va auzi și se va zice: "Au fost înfrânți oamenii care au urmat lui Abesalom".
\par 10 Atunci până și cel mai viteaz care are inima ca de leu va cădea cu duhul; căci la tot Israelul este cunoscut cât de viteaz este tatăl tău și cât de curajoși sunt cei ce se află cu el.
\par 11 De aceea eu vă sfătuiesc: Să se adune la tine tot Israelul de la Dan până la Beerșeba, la număr tocmai ca nisipul mării, și să mergi tu însuți în mijlocul lor.
\par 12 Atunci vom merge asupra lui, în orice loc s-ar afla, și vom năvăli asupra lui, cum cade roua pe pământ, și nu-i va mai rămâne pe lângă el nici un om din cei ce se află cu el.
\par 13 Iar de va intra în vreo cetate, atunci tot Israelul va aduce frânghii la cetatea aceea și o vom târî în râu, încât nu va rămâne din ea nici pietricică".
\par 14 "Sfatul lui Hușai Archianul, au zis atunci Abesalom și tot Israelul, e mai bun decât sfatul lui. Ahitofel". Așa a judecat Domnul să strice sfatul cel mai bun al lui Ahitofel, ca să aducă Domnul pieirea asupra lui Abesalom.
\par 15 Apoi Hușai a zis către preoții Țadoc și Abiatar: "Așa și așa a sfătuit Ahitofel pe Abesalom și pe bătrânii lui Israel, iar eu i-am sfătuit așa și așa".
\par 16 Deci trimiteți acum repede să spună lui David așa: Tu, noaptea asta, să nu rămâi pe câmp în pustiu, ci să treci mai repede, ca să nu piară regele și oamenii care sunt cu el".
\par 17 În vremea aceasta Ionatan și Ahimaaț stăteau la En-Roghel. Deci s-a dus o slujnică și le-a spus acestora, iar aceștia s-au dus și au vestit pe regele David, căci ei nu se puteau arăta în cetate.
\par 18 Dar i-a zărit un tânăr și a spus lui Abesalom. Ei însă au plecat amândoi repede și s-au dus la Bahurim, în casa unui om, în ograda căruia se afla o fântână, și s-au ascuns în ea;
\par 19 Iar femeia omului a luat o pătură și a întins-o pe gura fântânii, a pus pe ea niște grâu pisat, așa încât nu se vedea nimic.
\par 20 Și venind slujitorii lui Abesalom în casă la femeie, i-au zis: "Unde sunt Ahimaaț și Ionatan?" "Au trecut pârâul", le-a răspuns femeia. Și i-au căutat oamenii lui Abesalom, dar nu i-au găsit și s-au întors la Ierusalim.
\par 21 Iar dacă au plecat aceștia, ei au ieșit din fântână și s-au dus de au spus lui David: "Ridicați-vă și treceți repede apa, căci acestea a sfătuit Ahitofel împotriva voastră".
\par 22 Atunci s-a ridicat David și toți oamenii care erau cu el și au trecut Iordanul și până la ziuă n-a rămas niciunul care să nu fi trecut.
\par 23 Ahitofel însă, văzând că planul său n-a fost urmat, a pus șaua pe asin, a plecat și s-a dus la casa sa, în cetatea sa, și și-a făcut testamentul în folosul casei sale, apoi s-a spânzurat și a murit și a fost înmormântat în cetatea tatălui său.
\par 24 David a venit după aceea la Mahanaim, iar Abesalom a trecut Iordanul și tot Israelul era cu el.
\par 25 În locul lui Ioab, Abesalom a pus peste oștire pe Amasa. Amasa era fiul unui om cu numele Itra, din Israel, care intrase la Abigail, fiica lui Nahaș, sora Țeruiei, mama lui Ioab.
\par 26 Israel cu Abesalom și-au așezat tabăra în ținutul Galaad.
\par 27 Când David a venit la Mahanaim, Șobi, fiul lui Nahaș, din "Raba Amoniților, Machir, fiul lui Amiel din Lodebar, și Barzilai Galaaditeanul, din Roghelim,
\par 28 Au adus zece paturi pregătite, zece talere, vase de lut, grâu, orz, făină, grăunțe prăjite, bob, linte și pâine;
\par 29 Miere, unt, oi și brânză de vaci și le-au dat lui David și oamenilor care erau cu el, căci zicea: "Poporul este flămând și ostenit și a suferit de sete în pustie".

\chapter{18}

\par 1 Atunci a numărat David pe oamenii care erau cu el și a pus căpetenii peste sute și căpetenii peste mii.
\par 2 Și a trimis David pe oamenii săi: a treia parte sub comanda lui Ioab, a treia parte sub comanda lui Abișai, fiul Țeruiei, fratele lui Ioab, și a treia parte sub comanda lui Itai Gateul. Și a zis regele către oameni: "Și eu însumi voi merge cu voi".
\par 3 "Să nu mergi, i-au zis oamenii, că noi chiar de vom fugi, nu se va ține seamă de aceasta; și chiar de ar muri jumătate din noi, de asemenea nu se va ține seamă; iar tu singur ești cât noi, zece mii. Deci pentru noi este mai bine ca tu să ne dai ajutor din cetate!"
\par 4 "Ce vi se pare că este bine, a răspuns regele, aceea voi face". Atunci a stat regele la poartă și a ieșit tot poporul rânduit pe sute și pe mii.
\par 5 Apoi regele a dat poruncă lui Ioab, lui Abișai și lui Itai și le-a zis: "Să-mi cruțați pe băiatul Abesalom!" Și tot poporul a auzit cum a poruncit regele tuturor căpeteniilor pentru Abesalom.
\par 6 Au ieșit deci oamenii la câmp în întâmpinarea Israeliților, și s-a dat bătălia în pădurea lui Efraim.
\par 7 Acolo a fost în ziua aceea bătălie mare; poporul israelit a fost înfrânt de robii lui David, căzând uciși douăzeci de mii de oameni.
\par 8 Lupta s-a întins în tot ținutul acela și pădurea a mâncat în ziua aceea mai mulți oameni decât a doborât sabia.
\par 9 Când s-a întâlnit Abesalom cu oamenii lui David, era călare pe un catâr. Când catârul a fugit cu el pe sub crăcile unui stejar mare, părul lui Abesalom s-a încurcat în crengile stejarului și el a rămas spânzurat în văzduh, iar catârul de sub el s-a dus înainte.
\par 10 Atunci cineva a văzut aceasta și a spus lui Ioab, zicând: "Iată, am văzut pe Abesalom spânzurat de un stejar".
\par 11 "Dacă l-ai văzut, a zis Ioab către omul care-i adusese vestea, de ce nu l-ai doborât acolo, la pământ? Ți-aș fi dat zece sicli de argint și o cingătoare!"
\par 12 "De mi-ai fi pus în mână și o mie de sicli de argint, a răspuns acela lui Ioab, nici atunci nu mi-aș fi ridicat mâna asupra fiului regelui, căci în auzul nostru ți-a poruncit regele ție și lui Abișai și lui Itai și a zis: "Cruțați-mi pe băiatul Abesalom!"
\par 13 Și dacă eu însumi aș fi făcut altfel cu primejdia vieții mele, aceasta nu s-ar fi putut ascunde de rege și tu singur te-ai fi ridicat asupra mea".
\par 14 "N-am la ce să mai zăbovesc cu tine", a zis Ioab. Apoi a luat în mână trei săgeți și le-a înfipt în inima lui Abesalom, care era încă viu în crengile stejarului.
\par 15 Apoi au împresurat pe Abesalom zece tineri care duceau armele lui Ioab și, lovindu-l, l-au ucis.
\par 16 După aceea Ioab a sunat din trâmbiță și s-au întors oamenii de la urmărirea lui Israel, căci Ioab a cruțat poporul.
\par 17 Apoi au luat pe Abesalom și l-au aruncat acolo în pădure într-o groapă adâncă și au aruncat deasupra lui o grămadă mare de pietre. Iar Israeliții s-au împrăștiat cu toții, ducându-se fiecare la casa sa.
\par 18 Abesalom însă își făcuse un monument încă de pe când trăia, în Valea Regelui; căci își zisese: "Eu n-am fiu, ca să mi se păstreze amintirea!" și a dat monumentului numele său, așa că și astăzi se numește el: Monumentul lui Abesalom.
\par 19 Iar Ahimaaț, fiul lui Țadoc, a zis către Ioab: "Mă duc să vestesc pe regele că Domnul prin judecata Sa l-a izbăvit din mâinile dușmanilor lui".
\par 20 "Astăzi, a răspuns Ioab, n-ai să fii un bun vestitor. Îl vei vesti în altă zi, iar nu astăzi, căci a murit fiul regelui".
\par 21 "Du-te, a zis apoi Ioab către Hușai, du-te și spune regelui ceea ce ai văzut". Hușai, închinându-se înaintea lui Ioab, s-a dus numaidecât,
\par 22 Iar Ahimaaț, fiul lui Țadoc, a zis stăruitor către Ioab: "Fie ce-o fi, dar eu mă duc cu Hușai". "De ce să te duci, fiul meu? a zis Ioab. Nu duci o veste bună!"
\par 23 "Fie și așa, a răspuns Ahimaaț, dar eu tot mă duc". "Du-te", i-a zis Ioab. Și a apucat Ahimaaț la fugă pe un drum mai de-a dreptul și a întrecut pe Hușai.
\par 24 David ședea atunci între cele două porți; iar straja se suise pe acoperișul porții, pe zid, și, ridicându-și ochii, a văzut pe cel ce venea și a strigat și a zis către rege: "Iată un om vine în fugă".
\par 25 "Dacă este numai unul, a zis regele, atunci ne aduce o veste". Omul însă se apropia din ce în ce mai mult.
\par 26 Atunci straja a mai văzut un om venind în fugă; și a strigat straja la portar și a zis: "Iată că mai aleargă un om". "Și acela este un vestitor", a zis regele. "Eu văd, a zis straja, că mersul omului dinainte seamănă cu mersul lui Ahimaaț, fiul lui Țadoc".
\par 27 "Acesta este om bun, a zis regele, și vine cu veste bună!"
\par 28 Și a strigat Ahimaaț și a zis către rege: "Pace!" Apoi s-a închinat regelui cu fața până la pământ și a zis: "Binecuvântat este Domnul Dumnezeul tău, Care a dat în mâinile noastre pe oamenii care își ridicaseră mâinile lor împotriva regelui, stăpânul meu!"
\par 29 "Dar băiatul Abesalom este sănătos?" a întrebat regele. "Am văzut tulburare mare acolo când Ioab, robul regelui, a trimis pe robul tău, a zis Ahimaaț, dar eu nu știu ce era acolo".
\par 30 "Treci și rămâi aici", a zis regele. Și Ahimaaț a trecut și a stat acolo.
\par 31 Atunci a sosit și Hușai. Și Hușai a zis către rege: "Veste bună aduc regelui, stăpânul meu! Domnul ți-a făcut astăzi dreptate, izbăvindu-te din mâna tuturor celor ce s-au ridicat împotriva ta".
\par 32 "Băiatul Abesalom este oare el  sănătos?", a întrebat regele pe Hușai. "Întâmple-se dușmanilor regelui, stăpânul meu, a răspuns Hușai, și tuturor celor ce au uneltit rele împotriva ta, ce i s-a întâmplat lui!"
\par 33 Atunci regele s-a tulburat și s-a dus în foișorul de deasupra porții și a plâns, iar când se ducea, zicea: "O, fiul meu Abesalom, Abesalom, fiul meu! Mai bine muream eu în locul tău! Abesalom, Abesalom, fiul meu!"

\chapter{19}

\par 1 Atunci i s-a spus lui Ioab: "Iată regele plânge după Abesalom".
\par 2 Astfel biruința din ziua aceea s-a prefăcut în plângere pentru tot poporul, căci poporul a auzit chiar în ziua aceea și zicea că regele este întristat după fiul său.
\par 3 Atunci a intrat poporul în cetate pe furiș, cum se furișează oamenii rușinați care au luat-o la fugă în timpul luptei.
\par 4 Regele însă, acoperindu-și fața, striga tare: "Abesalom, Abesalom, fiul meu!"
\par 5 Dar venind Ioab la rege în casă, a zis: "Tu astăzi ai umplut de rușine pe toate slugile tale care au izbăvit acum viața ta și viața fiilor și fiicelor tale, viața femeilor și concubinelor tale.
\par 6 Tu iubești pe cei ce te urăsc și pe cei ce te iubesc îi urăști; căci ai arătat astăzi că pentru tine sunt nimic și căpeteniile și slugile; astăzi am aflat eu că de ar fi rămas Abesalom cu viață, iar noi am fi murit cu toții, aceasta ți-ar fi fost mai plăcut.
\par 7 Deci, scoală și ieși de grăiește după inima slugilor tale. Căci mă jur pe Domnul că dacă nu ieși, în noaptea aceasta nu-ți va mai rămâne nici un om. Și aceasta va fi pentru tine cea mai mare din toate nenorocirile care au venit asupra ta din tinerețea ta și până acum!"
\par 8 Atunci s-a sculat regele și a șezut la poartă, iar poporului întreg s-a vestit că regele stă la poartă. și a venit tot poporul în fața regelui la poartă, iar Israeliții au fugit pe la vetrele lor.
\par 9 Tot poporul din toate triburile lui Israel vorbea și zicea: "Regele David ne-a izbăvit din mâinile vrăjmașilor noștri și ne-a scăpat din mâinile Filistenilor, iar acum el însuși a fugit din țara aceasta, din regatul său, de frica lui Abesalom.
\par 10 Abesalom insă, pe care noi l-am uns rege pentru noi, a murit în război. Pentru ce dar întârziem noi acum a aduce înapoi pe rege?" Și aceste vorbe au străbătut tot Israelul și au ajuns și până la rege.
\par 11 Atunci regele David a trimis să se spună preoților Țadoc și Abiatar: "Spuneți bătrânilor lui luda: Pentru ce voiți să fiți cei din urmă în a aduce înapoi pe rege, când cuvintele a tot Israelul au ajuns până la rege, în casa lui?
\par 12 Voi sunteți frații mei; oasele mele și carnea mea voi sunteți. Pentru ce dar voiți să fiți cei din urmă în a aduce pe rege înapoi?
\par 13 Iar lui Amasa să-i ziceți: "Nu ești tu oare osul meu și carnea mea? Așa și așa să-mi facă mie Dumnezeu și încă și mai rău să-mi facă, dacă tu nu ai să fii căpetenia oștirii mele pentru totdeauna în locul lui Ioab!"
\par 14 Și așa a înduplecat regele inima tuturor Iudeilor ca a unui singur om; iar aceștia au trimis la rege să-i spună: "Întoarce-te tu însuți și toate slugile tale!
\par 15 Și s-a întors regele și a venit la Iordan, iar Iudeii au venit la Ghilgal, ca să întâmpine pe rege și să-l treacă Iordanul.
\par 16 Atunci Șimei, fiul lui Ghera, un veniaminean din Bahurim, s-a grăbit să iasă cu Iudeii în întâmpinarea regelui David.
\par 17 Acesta avea cu el o mie de oameni veniamineni, și pe Țiba, sluga casei lui Saul, cu cei cincisprezece fii ai săi și cu douăzeci de robi ai săi. Aceștia au trecut Iordanul înaintea regelui și au pregătit pentru rege trecerea Iordanului.
\par 18 Când însă au pornit luntrea, ca să aducă pe rege și casa lui, ca să-i slujească, atunci Șimei, fiul lui Ghera, a căzut cu fața la pământ înaintea regelui, îndată ce acesta a trecut Iordanul,
\par 19 Și a zis către rege: "Domnul meu, să nu-mi socotești ca o nelegiuire și să nu pomenești ceea ce ți-a greșit robul tău în ziua aceea când regele, stăpânul meu, a ieșit din Ierusalim și să nu iei în seamă, o, rege, aceasta!
\par 20 Căci robul tău știe că a greșit. și iată eu acum am venit cel dintâi din toată casa lui Iosif, ca să ies în întâmpinarea regelui, stăpânul meu".
\par 21 "Se poate oare, a zis Abișai, fiul Țeruiei, ca Șimei să nu moară pentru că a blestemat pe unsul Domnului?"
\par 22 "Fiii Țeruiei, ce este între mine și voi? a zis David, și pentru ce vă împotriviți voi astăzi mie? E timpul oare astăzi să se ucidă cineva în Israel? Nu văd eu, oare, acum că sunt rege peste Israel?"
\par 23 Iar lui Șimei i-a zis regele: "Nu, tu nu vei muri!" și i s-a jurat regele.
\par 24 Mefiboșet, fiul lui Ionatan, fiul lui Saul, a ieșit în întâmpinarea regelui. Acesta, din ziua în care ieșise regele și până în ziua când se întorsese cu pace, nu-și mai spălase picioarele, nici nu-și mai tăiase unghiile, nici nu-și îngrijise barba și nici hainele nu și le mai spălase.
\par 25 Și când a ieșit din Ierusalim în întâmpinarea regelui, regele i-a zis: "Mefiboșet, pentru ce n-ai mers și tu cu mine?"
\par 26 "Stăpânul meu, rege, a răspuns acela, sluga mea m-a înșelat, căci eu, robul tău, am zis: "Voi pune șaua pe asin și voi încăleca și mă voi duce c; regele, deoarece robul tău este olog.
\par 27 Iar el a clevetit pe robul tău înaintea regelui, stăpânul meu. Dar regele, stăpânul meu, este ca un înger al lui Dumnezeu. Fă ce binevoiești.
\par 28 Deși toată casa tatălui meu a fost vinovată de moarte înaintea domnului meu, regele, totuși tu ai pus pe robul tău printre cei ce mănâncă la masa ta. Ce drept am eu oare să mă jeluiesc înaintea regelui?"
\par 29 "De ce grăiești tu toate acestea? i-a zis regele. Eu am zis ca tu și Țiba să împărțiți între voi țarina!"
\par 30 "Să ia chiar toată țarina, a răspuns Mefiboșet. Bine că s-a întors cu pace domnul meu, regele, la casa sa!"
\par 31 Atunci a venit și Barzilai Galaaditeanul din Roghelim și a trecut Iordanul cu regele, ca să-l petreacă pe acesta până peste Iordan.
\par 32 Barzilai însă era foarte bătrân, ca de optzeci de ani, și el ospătase pe rege în timpul șederii lui la Mahanaim, pentru că era om foarte bogat.
\par 33 "Hai cu mine, a zis regele către Barzilai, și te voi ospăta și eu în Ierusalim".
\par 34 "Mult oare mi-a mai rămas de trăit, a răspuns Barzilai, ca să merg cu regele la Ierusalim?
\par 35 Eu am acum optzeci de ani. Mai pot eu oare osebi binele de rău? Și va afla oare robul tău gustul celor ce va mânca și va bea? Sau voi fi eu în stare să aud glasul cântărelilor și cântărețelor? La ce dar să fie robul tău o povară pentru domnul meu, regele?
\par 36 Robul tău va mai merge puțin dincolo de Iordan cu regele. Pentru ce să-mi răsplătească regele cu așa milă?
\par 37 Dă voie robului tău să se întoarcă, ca să moară în cetatea sa, lângă mormântul tatălui meu și al mamei mele. Dar iată fiul meu Chimham, robul tău! Să meargă el cu domnul meu, regele, și fă cu el ceea ce binevoiești!
\par 38 "Să meargă cu mine Chimham, a zis regele, și voi face cu el ce vrei și orice vei voi tu de la mine voi face pentru tine!"
\par 39 Și a trecut tot poporul Iordanul, de asemenea și regele. Apoi a sărutat regele pe Barzilai și l-a binecuvântat și acesta s-a întors la casa sa.
\par 40 După aceea regele s-a îndreptat spre Ghilgal și s-a dus cu el și Chimham, și tot poporul lui Iuda și jumătate din poporul lui Israel a petrecut pe rege.
\par 41 Dar iată tot Israelul a venit la rege și a zis către el: "Pentru ce bărbații lui Iuda, frații noștri, te-au răpit și au trecut pe rege și casa lui și pe toți oamenii lui David peste Iordan?"
\par 42 "Pentru aceea, că regele este mai apropiat de noi, au răspuns Israeliților toți bărbații lui Iuda. Și de ce să vă supărați voi pentru aceasta? Am mâncat noi, oare, ceva de la rege, sau am primit daruri de la el, sau ne-a scutit de dări?"
\par 43 Și Israeliții au răspuns bărbaților lui Iuda: "Noi suntem zece părți din rege și noi suntem încă și întâi născuți față de voi. Pentru ce dar ne-ați disprețuit? Oare nu noi trebuia să spunem cel dintâi cuvânt pentru întoarcerea regelui?" Dar cuvântul bărbaților lui Iuda a fost mai puternic decât cuvântul Israeliților.

\chapter{20}

\par 1 Din întâmplare, se afla acolo un om netrebnic, anume Șeba, fiul lui Bicri veniamineanul. Acesta a sunat din trâmbiță și a zis: "Noi n-avem nici o împărtășire cu David și nici o legătură cu fiul lui Iesei! Fiecare la cortul său, Israele!"
\par 2 Atunci s-au despărțit de David toți Israeliții și s-au dus după Șeba, fiul lui Bicri. Iudeii însă au rămas de partea regelui lor, de la Iordan, până la Ierusalim.
\par 3 Ajungând apoi David la casa sa în Ierusalim, a luat regele pe cele zece concubine pe care le lăsase să aibă în grijă casa și le-a pus într-o casă deosebită, sub supraveghere și le purta de grijă, dar nu se ducea la ele. Și au trăit ele acolo până la moartea lor, ca văduve.
\par 4 Apoi a zis David către Amasa: "În timp de trei zile cheamă pe Iudei la mine și să vii și tu aici!"
\par 5 Și s-a dus Amasa să cheme pe Iudei, dar a zăbovit mai multă vreme decât i se dăduse.
\par 6 Atunci David a zis lui Abișai: "Acum Șeba, fiul lui Bicri, are să ne facă mai mult rău decât Abesalom. Ia tu slugile stăpânului tău și urmărește-l, ca să nu-și găsească cetăți întărite și să nu scape din ochii noștri!"
\par 7 Și a plecat Abișai, urmat de oamenii lui Ioab, de Cheretieni și Peletieni și de toți oamenii buni de luptă din Ierusalim, să urmărească pe Șeba, fiul lui Bicri.
\par 8 Dar pe când erau ei aproape de piatra cea mare, de lângă Ghibeon, s-a întâlnit cu ei Amasa. Ioab era îmbrăcat cu hainele sale ostășești și încins cu sabia care atârna la șold în teaca ei, în care intra și ieșea foarte ușor.
\par 9 "Ești sănătos, frate?" a zis Ioab către Amasa. Apoi a apucat Ioab pe Amasa cu mâna de barbă, ca să-l sărute.
\par 10 Amasa însă nu s-a ferit de sabia care era în mâna lui Ioab și acesta l-a lovit cu ea în pântece și i-a vărsat măruntaiele jos și nu i-a mai dat altă lovitură. Și Amasa a murit. Apoi Ioab cu fratele său Abișai au alergat după Șeba, fiul lui Bicri.
\par 11 Unul din oamenii lui Ioab însă a rămas lângă Amasa și striga: "Cine vrea pe Ioab și cine este pentru David să urmeze pe Ioab!"
\par 12 Amasa însă zăcea mort în sânge, în mijlocul drumului. Dar văzând omul acela al lui Ioab că tot poporul se oprește la trupul lui Amasa, l-a târât pe acesta din drum în câmp și a aruncat peste el o haină, deoarece vedea că orice trecător se apropia de el.
\par 13 Iar după ce a fost târât din drum, tot poporul lui Israel s-a dus după Ioab, să urmărească pe Șeba, fiul lui Bicri.
\par 14 Ioab însă a trecut prin toate triburile israelite până la Abel și Bet-Maaca și prin tot ținutul Berim și toți locuitorii cetăților s-au adunat și au mers după el.
\par 15 Venind apoi, au împresurat pe Șeba în Abel și în Bet-Maaca și au ridicat un val împrejurul cetății; și apropiindu-se de zid, toți cei ce erau cu Ioab se sileau să spargă zidul.
\par 16 Atunci o femeie înțeleaptă a strigat de pe zidul cetății: "Ascultați, ascultați, spuneți lui Ioab să vină încoace, că am să-i vorbesc!"
\par 17 Apropiindu-se Ioab, femeia a zis: "Tu ești Ioab?" "Eu!" a răspuns Ioab. "Ascultă vorba roabei tale!" a zis, femeia. "Ascult!" a răspuns Ioab.
\par 18 "Odinioară era obiceiul, a adăugat ea, să se spună: Să se întrebe în Abel și în Dan, dacă se mai păstrează ceea ce au hotărât credincioșii din Israel
\par 19 Eu sunt din cetățile pașnice și credincioase ale lui Israel, și tu voiești să strici o cetate și încă mama cetăților lui Israel. De ce să strici tu moștenirea Domnului?"
\par 20 "Departe, departe de mine gândul de a strica sau de a dărâma! a răspuns Ioab.
\par 21 Lucrul nu era așa, ci un om din munții lui Efraim, anume Șeba, fiul lui Bicri, și-a ridicat mâna asupra regelui David. Dați-mi-l numai pe el singur și mă voi depărta de cetate". "Iată, a zis femeia către Ioab, capul lui îți va fi aruncat peste zid!"
\par 22 Și s-a dus femeia cu vorba ei înțeleaptă la tot poporul și a spus la tot poporul ca să taie capul lui Șeba, fiul lui Bicri. Și au tăiat capul lui Șeba, fiul lui Bicri, și l-au aruncat lui Ioab. Atunci Ioab a sunat din trâmbiță și toți oamenii s-au depărtat de cetate și s-au dus pe la casele lor. Iar Ioab s-a întors în Ierusalim la rege.
\par 23 Și era Ioab peste toată oștirea Israeliților, iar Benaia, fiul lui Iehoiada, era peste Cheretieni și Peletieni.
\par 24 Adoram era peste dări; Iosafat, fiul lui Ahilud, era cronicar;
\par 25 Siva era secretar; Țadoc și Abiatar erau preoți.
\par 26 De asemenea și Ira din Iair era sfetnic apropiat al lui David.

\chapter{21}

\par 1 În zilele lui David a fost foamete în țară trei ani, unul după altul. Și a întrebat David pe Domnul și Domnul a zis: "Aceasta este din pricina lui Saul și a casei lui cea însetată de sânge, pentru că el a ucis pe Ghibeoniți".
\par 2 Atunci regele a chemat pe Ghibeoniți și a vorbit eu ei. Ghibeoniții însă nu erau din fiii lui Israel, ci rămășițe din Amorei, și Israeliții le făcuseră jurământ, dar Saul a voit să-i piardă din râvnă pentru fiii lui Israel și ai lui Iuda.
\par 3 "Ce să fac pentru voi, a zis David către Ghibeoniți, și cu ce să vă împac, ca să binecuvântați moștenirea Domnului?"
\par 4 "Nouă nu ne trebuie nici aur, nici argint de la Saul sau de la casa lui, au răspuns Ghibeoniții, și nici nu trebuie să se ucidă cineva din Israel". "Atunci ce voiți să fac eu pentru voi?" a întrebat din nou David.
\par 5 "Fiindcă acel om, au răspuns ei regelui, ne-a măcelărit și a vrut să ne stârpească astfel ca să nu mai fim nici unul în ținuturile lui Israel,
\par 6 De aceea, dă-ne din urmașii acelui om șapte bărbați și noi îi vom spânzura, înaintea Domnului în Ghibeea lui Saul, alesul Domnului". "Vă voi da!", a zis regele.
\par 7 Dar regele a cruțat pe Mefiboșet, fiul lui Ionatan, fiul lui Saul, pentru jurământul pe numele Domnului, care era între ei, între David și Ionatan, fiul lui Saul.
\par 8 A luat însă regele pe cei doi fii pe care Rițpa, fiica lui Aia, îi născuse lui Saul, Armoni și Mefiboșet, și pe cei cinci fii pe care Merob, fiica lui Saul, îi născuse lui Adriel, fiul lui Barzilai din Mehola,
\par 9 și i-a dat în mâinile Ghibeoniților; iar aceștia i-au spânzurat, pe munte, înaintea Domnului; și au pierit toți șapte împreună. Ei au fost uciși în cele dintâi zile ale secerișului orzului.
\par 10 Atunci Rițpa, fiica lui Aia, a luat sac și l-a întins pe stâncă în chip de cort și a șezut de la începutul secerișului până a dat Dumnezeu ploaie din cer și n-a lăsat să se atingă de ei ziua păsările cerului și noaptea fiarele câmpului.
\par 11 Și i s-a spus lui David ce a făcut Rițpa; fiica lui Aia, concubina lui Saul.
\par 12 Atunci David s-a dus și a luat oasele lui Saul și oasele lui Ionatan, fiul lui, de la locuitorii din Iabeșul Galaadului, pe care aceștia le luaseră pe ascuns din piața Bet-Sanului, unde fuseseră spânzurate de Filisteni, când Filistenii uciseseră pe Saul pe muntele Ghilboa.
\par 13 Și a strămutat el de acolo oasele lui Saul și oasele lui Ionatan, fiul lui, și a adunat și oasele celor ce fuseseră spânzurați;
\par 14 Și a îngropat oasele lui Saul și ale lui Ionatan, fiul lui, și oasele celor spânzurați în ținutul lui Veniamin, la Țela, în mormântul lui Chiș, tatăl lui Saul. Și s-a făcut tot ce a poruncit regele. ți după aceea S-a milostivit Domnul asupra țării.
\par 15 Dar din nou s-a pornit război între Filisteni și Israeliți. Și a ieșit David împreună cu slugile lui și au luptat cu Filistenii. David însă a obosit.
\par 16 Atunci Ișbi-Benob, unul din urmașii lui Rafa, a cărui suliță era de trei sute de sicli de aramă, și care era încins cu sabie nouă, a vrut să lovească pe David;
\par 17 Dar i-a ajutat lui David Abișai, fiul Țeruiei, și a scăpat Abișai pe David și a lovit pe filistean și l-a omorât. Atunci oamenii lui David s-au jurat și au zis: "Tu să nu mai ieși cu noi la război, ca să nu se stingă făclia în Israel".
\par 18 După aceea a fost din nou război cu Filistenii la Gob. Atunci Sibecai Hușatitul a ucis pe Saf, unul din urmașii lui Rafa.
\par 19 Și a mai fost o altă bătălie la Gob cu Filistenii. Atunci Elhanan, fiul lui Iaare Oreghim din Betleem, a ucis pe Goliat din Gat, a cărui coadă de suliță era ca un sul de la războiul de țesut.
\par 20 Și a mai fost încă o bătălie în Gat. Și era acolo un om înalt, care avea câte șase degete la mână și la picioare, în total deci douăzeci și patru, tot din urmașii lui Rafa.
\par 21 Acesta hulea pe Israeliți, dar l-a ucis Ionatan, fiul lui Șama, fratele lui David.
\par 22 Toți acești patru oameni erau din neamul lui Rafa, din Gat, și au căzut de mâna lui David și a slugilor lui.

\chapter{22}

\par 1 Atunci a cântat David cântarea aceasta Domnului, în ziua când Domnul l-a izbăvit de toți vrăjmașii lui și din mâinile lui Saul, și a zis:
\par 2 "Domnul este întărirea mea, scăparea mea și izbăvitorul meu.
\par 3 Dumnezeu este stânca mea cea de scăpare, Scutul meu și puterea cea mântuitoare, Adăpostul meu cel tare și scăparea mea! Mântuitorul meu, din necaz m-ai izbăvit!
\par 4 Pe Domnul Cel vrednic de laudă L-am chemat Și de vrăjmașii mei am fost izbăvit.
\par 5 Valurile morții mă-mpresuraseră Și eram potopit de șuvoaiele răutății;
\par 6 Lanțurile iadului mă încătușaseră și eram prins în lanțurile morții.
\par 7 Dar în necazul meu am chemat pe Domnul, Strigăt am înălțat către Dumnezeul meu Și El mi-a auzit glasul din locașul Său Și strigătul meu a ajuns la urechile Lui.
\par 8 Atunci s-a clătinat și s-a cutremurat pământul, Și temeliile cerurilor s-au zguduit, Pentru că Se mâniase Domnul!
\par 9 Fum ieșea din nările Lui, Din gura Lui ieșea foc mistuitor Și cărbuni aprinși țâșneau.
\par 10 Plecat-a cerurile și S-a coborât Și sub picioarele Lui era negură deasă.
\par 11 Șezut-a pe heruvimi și a zburat, zburat-a pe aripile vântului!
\par 12 Din negură Și-a făcut adăpost Și cort împrejurul Său; Cu ape întunecoase și cu nori negri era înfășurat.
\par 13 De strălucirea ce-I mergea înainte Se împrăștiau norii, Aruncând grindină și cărbuni de foc.
\par 14 Atunci Domnul a tunat în ceruri Și Cel Preaînalt a făcut să răsuna glasul Său;
\par 15 Slobozit-a săgețile Sale și a împrăștiat pe vrăjmași, Înmulțitu-Și-a fulgerele și i-a pus pe fugă.
\par 16 De certarea Ta, Doamne, De suflarea nărilor Tale, S-au arătat fundurile mării Și temeliile lumii s-au descoperit.
\par 17 Tinzându-Și El mâna de sus, m-a apucat Și m-a scos din ape adânci;
\par 18 Izbăvitu-m-a de vrăjmașul meu cel puternic, De cei ce mă urau și erau mai tari ca mine:
\par 19 Aceștia mă împresuraseră în ziua necazului, Dar Domnul a fost ajutorul meu;
\par 20 El m-a scos le loc larg, izbăvitu-m-a, pentru că m-a iubit!
\par 21 Domnul mi-a dat după nevinovăția mea, După curăția mâinilor mele mi-s plătit.
\par 22 Căci am urmat căile Domnului, Și înaintea Dumnezeului meu m-am pocăit.
\par 23 Toate poruncile Lui le-am avut înaintea mea Și de la legea Lui nu m-am abătut.
\par 24 Fără prihană am fost înaintea Lui Și de nedreptate m-am păzit.
\par 25 Deci Domnul mi-a plătit după nevinovăția mea Și după curăția mâinilor mele pe care a cunoscut-o.
\par 26 Căci Tu, Doamne, cu cel bun Te arăți bun, Cu omul drept Te porți cu dreptate,
\par 27 Cu cel cuvios, Cuvios ești, Iar cu cel îndărătnic, Te porți după îndărătnicia lui.
\par 28 Tu pe poporul smerit îl izbăvești, Iar ochii cei mândri îi smerești.
\par 29 Tu ești lumina mea, Doamne! Doamne, luminează întunericul meu!
\par 30 Cu Tine mă voi arunca spre oștirea înarmată, Cu Dumnezeul meu voi doborî zidul.
\par 31 Căile Domnului sunt desăvârșite, Cuvântul Domnului e lămurit prin foc. Și scut este El tuturor celor ce nădăjduiesc în El;
\par 32 Căci cine este Dumnezeu, dacă nu Domnul? Și cine este apărător, dacă nu Dumnezeul nostru?
\par 33 Dumnezeu este Cel ce mă încinge cu putere și calea cea dreaptă-mi arată;
\par 34 Care dă picioarelor mele sprinteneala cerbului, Și la locurile cele înalte mă așează;
\par 35 Care-mi deprinde mâinile mele la război Și brațele mele să întindă arzul de aramă.
\par 36 Doamne, Tu mă aperi cu scutul Tău izbăvitor, Cu dreapta Ta mă sprijini și înalți cu bunătatea Ta,
\par 37 Tu lărgești calea sub pașii mei Și picioarele mele nu se poticnesc.
\par 38 Urmări-voi pe vrăjmașii mei și-i voi prinde. Nu mă voi întoarce până nu-i stârpesc.
\par 39 Îi voi lovi și nu se vor putea ține. Cădea-vor sub picioarele mele.
\par 40 Tu mă încingi cu putere pentru război Și faci să cadă potrivnicii sub mine
\par 41 Pe vrăjmașii mei Tu-i pui pe fugă înaintea mea și pe cei ce mă urăsc îi spulberi.
\par 42 Striga-vor, dar nimeni nu-i va scăpa, Pe Domnul vor striga, dar nu-i va auzi;
\par 43 Ca praful ce-l ia vântul îi voi pisa și-i voi călca în picioare ca tina ulițelor.
\par 44 Tu mă izbăvești de răzvrătirea poporului Și în fruntea neamurilor mă pui.
\par 45 Poporul pe care nu-l cunoșteam îmi slujește Și dintr-un cuvânt mi se supune.
\par 46 Fiii celor de alt neam pălesc de frică și tremură în zidurile cetății lor.
\par 47 Viu este Domnul și binecuvântat este numele Lui! Lăudat fie Dumnezeu, Cel ce mă izbăvește,
\par 48 Dumnezeu, Cel ce mă răzbună, Care pune popoarele sub stăpânirea mea Și de vrăjmașii mei mă izbăvește.
\par 49 Doamne, Tu mă înalți peste vrăjmașii mei Și de omul nedrept mă izbăvești.
\par 50 De aceea Te voi lăuda printre popoare și voi da slavă numelui Tău,
\par 51 Cel ce dă biruință strălucită regelui Său Și face milă cu unsul Său, David, Și cu urmașii lui din veac în veac".

\chapter{23}

\par 1 Iată cele din urmă cuvinte ale lui David: "Cuvintele lui David, fiul lui Iesei, Graiurile bărbatului suspus, Ale unsului lui Dumnezeu, celui din Iacov, Ale dulcelui cântăreț din Israel:
\par 2 "Duhul  Domnului grăiește prin mine, Și cuvântul Lui este pe limba mea.
\par 3 Dumnezeul lui Israel a vorbit. Tăria lui Israel mi-a zis: Cel ce domnește între oameni cu dreptate, Cel ce stăpânește cu temere de Dumnezeu
\par 4 E ca lumina dimineții când răsare soarele, E ca dimineața fără nori, Ca razele după ploaie ce fac să răsară iarba din pământ.
\par 5 Nu este așa oare, casa mea la Dumnezeu? Căci legământ veșnic a încheiat El cu mine, Așezământ pe veci și neschimbat, Și El face să răsară toată voia și nădejdea mea!
\par 6 Iar cei răi sunt ca spinii aruncați, Care nu se pot lua cu mâna;
\par 7 Ci se iau cu fierul sau cu coada unei suliți, Și-n răspântii de foc se ard".
\par 8 Iată acum și numele vitejilor lui David: Ioșeb-Bașebet Tașchemonitul, unul dintre marile căpetenii. Acesta și-a ridicat lancea sa asupra a opt sute de oameni deodată și i-a ucis.
\par 9 După el vine Eleazar, fiul lui Dodo, fiul lui Ahohi, unul din cei trei viteji care împreună cu David au înfruntat pe Filisteni, când se adunaseră la război și când Israeliții se retrăseseră pe înălțimi.
\par 10 Acesta s-a ridicat atunci și a lovit în Filisteni până ce i-a obosit și i s-a lipit mâna de sabia sa. În ziua aceea a dat Domnul biruință mare și poporul s-a dus după el, numai ca să adune pe cei uciși.
\par 11 După el vine Șama, fiul lui Aghe, din Harar. Când Filistenii s-au adunat la Lehi, unde era o țarină semănată cu linte, și când poporul a fugit de Filisteni,
\par 12 Acesta a rămas în țarină și a apărat-o, bătând pe Filisteni. Atunci Domnul a dat biruință mare.
\par 13 Aceștia trei mai însemnați din cele treizeci de căpetenii s-au dus și au intrat în vremea secerișului la David în peștera Adulam, când cetele Filistenilor tăbărâseră în valea Refaim.
\par 14 David se afla într-un loc întărit, iar o ceată de Filisteni era în Betleem.
\par 15 Și fiindu-i sete lui David, a zis: "Cine mă va adăpa cu apă din fântâna Betleemului cea de la poartă?"
\par 16 Atunci acești trei viteji au străbătut prin tabăra Filistenilor, au scos apă din fântâna Betleemului cea de la poartă și au adus-o lui David; dar el n-a vrut s-o bea, ci a vărsat-o înaintea Domnului, zicând:
\par 17 "Să mă ferească Dumnezeu să fac una ca aceasta! Aceasta nu este, oare, sângele oamenilor, care și-au pus viața în primejdie?" Și n-a vrut să bea. Iată ce-au făcut acești trei viteji.
\par 18 Abișai, fratele lui Ioab, fiul Țeruiei, era întâiul între alți trei. El a ucis cu sulița sa trei sute de oameni și era în mare cinste între acești trei.
\par 19 Între acești trei el era cel mai de seamă și căpetenie, dar cu cei trei de mai sus nu se asemăna.
\par 20 Benaia, fiul lui Iehoiada, un om din Cabțeel, viteaz și vestit prin fapte mari, a ucis pe cei doi fii ai lui Ariel Moabitul și s-a coborât într-o groapă și a ucis un leu pe vreme de iarnă.
\par 21 Tot el a ucis un egiptean de o statură falnică. Egipteanul avea suliță în mână, iar el s-a dus la acela cu un băț, i-a smucit sulița și l-a ucis cu ea.
\par 22 Iată ce a făcut Benaia, fiul lui Iehoiada, și era mult prețuit de cei trei.
\par 23 El era cel mai de seamă între cei treizeci, dar cu cei trei de mai sus nu se asemăna. Și David l-a primit între cei mai de aproape sfetnici ai săi.
\par 24 Asael, fratele lui Ioab, este din numărul celor treizeci, Elhanan, fiul lui Dodo, din Betleem.
\par 25 Șama Haroditeanul; Elica Haroditeanul;
\par 26 Heleț Paltianul; Ira, fiul lui Icheș, din Tecoa;
\par 27 Abiezer din Anatot; Mebunai din Hușa.
\par 28 Țalmon Ahohitul; Maharai din Netof,
\par 29 Heleb, fiul lui Baana, din Netof; Itai, fiul lui Ribai, din Ghibeea lui Veniamin;
\par 30 Benaia din Piraton; Hidai din Nahale-Gaaș;
\par 31 Abi-Baal din Araba; Azmavet din Bahurim;
\par 32 Eliahba din Șaalbon; Ionatan unul din fiii lui Iașen;
\par 33 Șama din Harar; Ahiam, fiul lui Șarar, din Arar;
\par 34 Elifelet, fiul lui Ahasbai, fiul unui Maacatean; Eliam, fiul lui Ahitofel, din Ghilo;
\par 35 Hețrai din Carmel; Paarai din Arba;
\par 36 Igal, fiul lui Natan, din Țoba; Bani din Gad.
\par 37 Țelec Amonitul; Naharai din Beerot, armașul lui Ioab, fiul Țeruiei.
\par 38 Ira din Ieter; Gareb din Ieter;
\par 39 Urie Heteul. De toți treizeci și șapte.

\chapter{24}

\par 1 Mânia Domnului s-a aprins iarăși asupra Israeliților, pentru că cineva din ei îndemnase pe David, zicând: "Mergi de numără pe Israel și pe Iuda!"
\par 2 Și a zis regele către Ioab, căpetenia oștirii care era cu el: "Cutreieră toate triburile din Israel și ale lui Iuda, de la Dan până la Beer-Șeba, numără poporul, ca să știu numărul oamenilor".
\par 3 "Domnul Dumnezeul tău, a răspuns Ioab, să înmulțească poporul încă pe atâta pe cât este, ba încă și de o sută de ori atâta și ochii domnului meu, regele, să vadă. Și pentru ce domnul meu, regele, voiește acest lucru?"
\par 4 Dar cuvântul regelui, dat lui Ioab și căpeteniilor oștirii, a biruit. Și s-a dus Ioab cu căpeteniile oștirii de la rege să numere poporul lui Israel.
\par 5 Trecând Iordanul, au poposit la Aroer, în partea dreaptă a cetății care este în mijlocul văii Gad, aproape de Iazer.
\par 6 De acolo au mers în Galaad și în pământul Tahtim-Hodși, de unde au venit la Dan-Iaan,
\par 7 Și ocolind Sidonul, au mers la cetatea Tir și prin toate cetățile Heveilor și Canaaneilor și au ieșit în partea de miazăzi a Iudei, la Beer-Șeba.
\par 8 Apoi au străbătut toată țara aceasta și după nouă luni și douăzeci de zile au ajuns la Ierusalim.
\par 9 Și a dat Ioab regelui cartea cu numărătoarea poporului, din care se vedea că Israeliții erau opt sute de mii de bărbați vârstnici, buni de război, iar cei din Iuda cinci sute de mii.
\par 10 Atunci s-a cutremurat inima lui David după ce a numărat poporul. Și a zis David către Domnul: "Greu am păcătuit eu, făcând așa, și acum mă rog înaintea Ta, Doamne, iartă păcatul robului Tău, căci m-am purtat peste măsură de nebunește!"
\par 11 A doua zi dimineața s-a sculat David. Fusese însă cuvântul Domnului către Gad proorocul, ca să spună lui David viitorul, zicându-i:
\par 12 "Mergi și spune lui David: Așa zice Domnul: Îți arăt trei pedepse: alege-ți una din ele, să vină asupra ta".
\par 13 Și a venit Gad la David și i-a vestit, zicându-i: "Alege-ți ce vrei: foamete în țara ta șapte ani; să fugi trei luni de vrăjmașii tăi și ei să te urmărească; sau timp de trei zile să fie ciumă în țara ta? Chibzuiește și hotărăște; ce să spun Celui ce m-a trimis",
\par 14 "E tare greu, a răspuns David lui Gad, dar să cad mai bine în mâna Domnului, căci mila Lui este mare; numai în mâinile oamenilor să nu cad!" Și și-a ales David ciuma în vremea secerișului grâului.
\par 15 Și a trimis Domnul ciumă asupra lui Israel de dimineață până la vremea hotărâtă. Și a început molima în popor și au murit de la Dan până la Beer-Șeba șaptezeci de mii de oameni.
\par 16 Și și-a întins îngerul Domnului mâna asupra Ierusalimului, ca să-l pustiiască, dar I S-a făcut milă Domnului și a zis îngerului care ucidea poporul: "Destul! Oprește-i acum mâna!" Îngerul Domnului se afla atunci la aria lui Aravna Iebuseul.
\par 17 Și văzând David pe îngerul care lovea poporul, a zis către Domnul: "Iată eu am păcătuit! Fărădelegea am făcut-o eu. Dar aceste oi ce-au făcut? Deci îndreaptă-ți mâna Ta asupra mea și asupra casei tatălui meu!"
\par 18 În ziua aceea Gad a venit la David și a zis: "Mergi de ridică jertfelnic Domnului în aria lui Aravna Iebuseul".
\par 19 Și s-a dus David, cum îi zisese proorocul Gad și cum poruncise Domnul.
\par 20 și privind Aravna, a văzut pe rege și slugile lui venind la el; și a ieșit Aravna și s-a închinat regelui cu fața până la pământ.
\par 21 "La ce a venit domnul meu, regele, la robul tău?" a întrebat Aravna. "Ca să cumpăr de la tine aria, a răspuns David, și să fac acolo jertfelnic Domnului, pentru ca să înceteze moartea în popor"
\par 22 "Să ia domnul meu, regele, și să aducă jertfă Domnului ce voiește, a zis Aravna către David. Iată boii pentru ardere de tot, iar carele și jugurile boilor vor sluji de lemne.
\par 23 Toate acestea le dăruiesc regelui. Domnul Dumnezeul tău să te binecuvânteze!" a adăugat Aravna.
\par 24 "Ba nu, a zis regele către Aravna, eu am să-ți plătesc prețul și nu voi aduce Domnului Dumnezeului meu arderi de tot, lucruri luate în darn. Și a cumpărat David aria și boii cu cincizeci de sicli de argint.
\par 25 Și a ridicat David acolo jertfelnic Domnului și a adus arderi de tot și jertfe de împăcare. Și S-a milostivit Domnul asupra țării și a încetat moartea în Israel.


\end{document}