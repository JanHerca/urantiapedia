\begin{document}

\title{Romani}


\chapter{1}

\par 1 Pavel, rob al lui Iisus Hristos, chemat de El apostol, rânduit pentru vestirea Evangheliei lui Dumnezeu,
\par 2 Pe care a făgăduit-o mai înainte, prin proorocii Săi, în Sfintele Scripturi,
\par 3 Despre Fiul Său, Cel născut din sămânța lui David, după trup,
\par 4 Care a fost rânduit Fiu al lui Dumnezeu întru putere, după Duhul sfințeniei, prin învierea Lui din morți, Iisus Hristos, Domnul nostru,
\par 5 Prin Care am primit har și apostolie, ca să aduc, în numele Său, la ascultarea credinței, toate neamurile,
\par 6 Întru care sunteți și voi chemați ai lui Iisus Hristos:
\par 7 Tuturor celor ce sunteți în Roma, iubiți de Dumnezeu, chemați și sfinți: har vouă și pace de la Dumnezeu, Tatăl nostru, și de la Domnul Iisus Hristos!
\par 8 Mulțumesc, întâi Dumnezeului meu, prin Iisus Hristos, pentru voi toți, fiindcă credința voastră se vestește în toată lumea.
\par 9 Căci martor îmi este Dumnezeu, Căruia Îi slujesc cu duhul meu, întru Evanghelia Fiului Său, că neîncetat fac pomenire despre voi,
\par 10 Cerând totdeauna în rugăciunile mele ca să am cumva, prin voința Lui, vreodată, bun prilej ca să vin la voi.
\par 11 Pentru că doresc mult să vă văd ca să vă împărtășesc vreun dar duhovnicesc, spre întărirea voastră.
\par 12 Și aceasta ca să mă mângâi împreună cu voi prin credința noastră laolaltă, a voastră și a mea.
\par 13 Fraților, nu vreau ca voi să nu știți că, de multe ori, mi-am pus în gând să vin la voi, dar am fost până acum împiedicat, ca să am și între voi vreo roadă, ca și la celelalte neamuri.
\par 14 Dator sunt și elinilor și barbarilor și învățaților și neînvățaților;
\par 15 Astfel, cât despre mine, sunt bucuros să vă vestesc Evanghelia și vouă, celor din Roma.
\par 16 Căci nu mă rușinez de Evanghelia lui Hristos, pentru că este putere a lui Dumnezeu spre mântuirea a tot celui care crede, iudeului întâi, și elinului.
\par 17 Căci dreptatea lui Dumnezeu se descoperă în ea din credință spre credință, precum este scris: "Iar dreptul din credință va fi viu".
\par 18 Căci mânia lui Dumnezeu se descoperă din cer peste toată fărădelegea și peste toată nedreptatea oamenilor care țin nedreptatea drept adevăr.
\par 19 Pentru că ceea ce se poate cunoaște despre Dumnezeu este cunoscut de către ei; fiindcă Dumnezeu le-a arătat lor.
\par 20 Cele nevăzute ale Lui se văd de la facerea lumii, înțelegându-se din făpturi, adică veșnica Lui putere și dumnezeire, așa ca ei să fie fără cuvânt de apărare,
\par 21 Pentru că, cunoscând pe Dumnezeu, nu L-au slăvit ca pe Dumnezeu, nici nu I-au mulțumit, ci s-au rătăcit în gândurile lor și inima lor cea nesocotită s-a întunecat.
\par 22 Zicând că sunt înțelepți, au ajuns nebuni.
\par 23 Și au schimbat slava lui Dumnezeu Celui nestricăcios cu asemănarea chipului omului celui stricăcios și al păsărilor și al celor cu patru picioare și al târâtoarelor.
\par 24 De aceea Dumnezeu i-a dat necurăției, după poftele inimilor lor, ca să-și pângărească trupurile lor între ei,
\par 25 Ca unii care au schimbat adevărul lui Dumnezeu în minciună și s-au închinat și au slujit făpturii, în locul Făcătorului, Care este binecuvântat în veci, amin!
\par 26 Pentru aceea, Dumnezeu i-a dat unor patimi de ocară, căci și femeile lor au schimbat fireasca rânduială cu cea împotriva firii;
\par 27 Asemenea și bărbații lăsând rânduiala cea după fire a părții femeiești, s-au aprins în pofta lor unii pentru alții, bărbați cu bărbați, săvârșind rușinea și luând cu ei răsplata cuvenită rătăcirii lor.
\par 28 Și precum n-au încercat să aibă pe Dumnezeu în cunoștință, așa și Dumnezeu i-a lăsat la mintea lor fără judecată, să facă cele ce nu se cuvine.
\par 29 Plini fiind de toată nedreptatea, de desfrânare, de viclenie, de lăcomie, de răutate; plini de pizmă, de ucidere, de ceartă, de înșelăciune, de purtări rele, bârfitori,
\par 30 Grăitori de rău, urâtori de Dumnezeu, ocărâtori, semeți, trufași, lăudăroși, născocitori de rele, nesupuși părinților,
\par 31 Neînțelepți, călcători de cuvânt, fără dragoste, fără milă;
\par 32 Aceștia, deși au cunoscut dreapta orânduire a lui Dumnezeu, că cei ce fac unele ca acestea sunt vrednici de moarte, nu numai că fac ei acestea, ci le și încuviințează celor care le fac.

\chapter{2}

\par 1 Pentru aceea, oricine ai fi, o, omule, care judeci, ești fără cuvânt de răspuns, căci, în ceea ce judeci pe altul, pe tine însuți te osândești, căci același lucruri faci și tu care judeci.
\par 2 Și noi știm că judecata lui Dumnezeu este după adevăr, față de cei ce fac unele ca acestea.
\par 3 Și socotești tu, oare, omule, care judeci pe cei ce fac unele ca acestea, dar le faci și tu, că tu vei scăpa de judecata lui Dumnezeu?
\par 4 Sau disprețuiești tu bogăția bunătății Lui și a îngăduinței și a îndelungii Lui răbdări, neștiind că bunătatea lui Dumnezeu te îndeamnă la pocăință?
\par 5 Dar după învârtoșarea ta și după inima ta nepocăită, îți aduni mânie în ziua mâniei și a arătării dreptei judecăți a lui Dumnezeu,
\par 6 Care va răsplăti fiecăruia după faptele lui:
\par 7 Viață veșnică celor ce, prin stăruință în faptă bună, caută mărire, cinste și nestricăciune,
\par 8 Iar iubitorilor de ceartă, care nu se supun adevărului, ci se supun nedreptății: mânie și furie.
\par 9 Necaz și strâmtorare peste sufletul oricărui om care săvârșește răul, al iudeului mai întâi, și al elinului;
\par 10 Dar mărire, cinste și pace oricui face binele: iudeului mai întâi, și elinului.
\par 11 Căci nu este părtinire la Dumnezeu!
\par 12 Câți, deci, fără lege, au păcătuit, fără lege vor și pieri; iar câți au păcătuit în lege, prin lege vor fi judecați.
\par 13 Fiindcă nu cei ce aud legea sunt drepți la Dumnezeu, ci cei ce împlinesc legea vor fi îndreptați.
\par 14 Căci, când păgânii care nu au lege, din fire fac ale legii, aceștia, neavând lege, își sunt loruși lege,
\par 15 Ceea ce arată fapta legii scrisă în inimile lor, prin mărturia conștiinței lor și prin judecățile lor, care îi învinovățesc sau îi și apără,
\par 16 În ziua în care Dumnezeu va judeca, prin Iisus Hristos, după Evanghelia mea, cele ascunse ale oamenilor.
\par 17 Dar dacă tu te numești iudeu și te reazimi pe lege și te lauzi cu Dumnezeu,
\par 18 Și cunoști voia Lui și știi să încuviințezi cele bune, fiind învățat din lege,
\par 19 Și ești încredințat că tu ești călăuză orbilor, lumină celor ce sunt în întuneric,
\par 20 Povățuitor celor fără de minte, învățător celor nevârstnici, având în lege dreptarul cunoștiinței și al adevărului,
\par 21 Deci tu, cel care înveți pe altul, pe tine însuți nu te înveți? Tu cel care propovăduiești: Să nu furi - și tu furi?
\par 22 Tu, cel care zici: Să nu săvârșești adulter, săvârșești adulter? Tu cel care urăști idolii, furi cele sfinte?
\par 23 Tu, care te lauzi cu legea, Îl necinstești pe Dumnezeu, prin călcarea legii?
\par 24 "Căci numele lui Dumnezeu, din pricina voastră, este hulit între neamuri", precum este scris.
\par 25 Căci tăierea împrejur folosește, dacă păzești legea; dacă însă ești călcător de lege, tăierea ta împrejur s-a făcut netăiere împrejur.
\par 26 Deci dacă cel netăiat împrejur păzește hotărârile legii, netăierea lui împrejur nu va fi, oare, socotită ca tăiere împrejur?
\par 27 Iar el - din fire netăiat împrejur, dar împlinitor al legii - nu te va judeca, oare, pe tine, care, prin litera legii și prin tăierea împrejur, ești călcător de lege?
\par 28 Pentru că nu cel ce se arată pe din afară e iudeu, nici cea arătată pe dinafară în trup, este tăiere împrejur;
\par 29 Ci este iudeu cel întru ascuns, iar tăierea împrejur este aceea a inimii, în duh, nu în literă; a cărui laudă nu vine de la oameni, ci de la Dumnezeu.

\chapter{3}

\par 1 Care este deci întâietatea iudeului și folosul tăierii împrejur?
\par 2 Este mare în toate privințele. Întâi, pentru că lor li s-au încredințat cuvintele lui Dumnezeu.
\par 3 Căci ce este dacă unii n-au crezut? Oare necredința lor va nimici credincioșia lui Dumnezeu?
\par 4 Nicidecum! Ci Dumnezeu se vădește în adevărul Său, pe când tot omul întru minciună, precum este scris: "Drept ești Tu întru cuvintele Tale și biruitor când vei judeca Tu".
\par 5 Iar dacă nedreptatea noastră învederează dreptatea lui Dumnezeu, ce vom zice? Nu cumva este nedrept Dumnezeu care aduce mânia? - Ca om vorbesc.
\par 6 Nicidecum! Căci atunci cum va judeca Dumnezeu lumea?
\par 7 Căci dacă adevărul lui Dumnezeu, prin minciuna mea, a prisosit spre slava Lui, pentru ce dar mai sunt și eu judecat ca păcătos?
\par 8 Și de ce n-am face cele rele, ca să vină cele bune, precum suntem huliți și precum spun unii că zicem noi? Osânda aceasta este dreaptă.
\par 9 Dar ce? Avem noi vreo precădere? Nicidecum. Căci am învinuit mai înainte și pe iudei, și pe elini, că toți sunt sub păcat,
\par 10 După cum este scris: "Nu este drept nici unul;
\par 11 Nu este cel ce înțelege, nu este cel ce caută pe Dumnezeu.
\par 12 Toți s-au abătut, împreună, netrebnici s-au făcut. Nu este cine să facă binele, nici măcar unul nu este.
\par 13 Mormânt deschis este gâtlejul lor; viclenii vorbit-au cu limbile lor; venin de viperă este sub buzele lor;
\par 14 Gura lor e plină de blestem și amărăciune;
\par 15 Iuți sunt picioarele lor să verse sânge;
\par 16 Pustiire și nenorocire sunt în drumurile lor;
\par 17 Și calea păcii ei nu au cunoscut-o;
\par 18 Nu este frică de Dumnezeu înaintea ochilor lor".
\par 19 Dar știm că cele câte zice Legea le spune celor care sunt sub Lege, ca orice gură să fie închisă și ca toată lumea să fie vinovată înaintea lui Dumnezeu.
\par 20 Pentru că din faptele Legii nici un om nu se va îndrepta înaintea Lui, căci prin Lege vine cunoștința păcatului.
\par 21 Dar acum, în afară de Lege, s-a arătat dreptatea lui Dumnezeu, fiind mărturisită de Lege și de prooroci.
\par 22 Dar dreptatea lui Dumnezeu vine prin credința în Iisus Hristos, pentru toți și peste toți cei ce cred, căci nu este deosebire.
\par 23 Fiindcă toți au păcătuit și sunt lipsiți de slava lui Dumnezeu;
\par 24 Îndreptându-se în dar cu harul Lui, prin răscumpărarea cea în Hristos Iisus.
\par 25 Pe Care Dumnezeu L-a rânduit (jertfă de) ispășire, prin credința în sângele Lui, ca să-Și arate dreptatea Sa, pentru iertarea păcatelor celor mai înainte făcute,
\par 26 Întru îngăduința lui Dumnezeu - ca să-Și arate dreptatea Sa, în vremea de acum, spre a fi El Însuși drept, și îndreptând pe cel ce trăiește din credința în Iisus.
\par 27 Deci, unde este pricina de laudă? A fost înlăturată. Prin care Lege? Prin Legea faptelor? Nu, ci prin Legea credinței.
\par 28 Căci socotim că prin credință se va îndrepta omul, fără faptele Legii.
\par 29 Oare Dumnezeu este numai al iudeilor? Nu este El și Dumnezeul păgânilor? Da, și al păgânilor.
\par 30 Fiindcă este un singur Dumnezeu, Care va îndrepta din credință pe cei tăiați împrejur și, prin credință, pe cei netăiați împrejur.
\par 31 Desființăm deci noi Legea prin credință? Nicidecum! Dimpotrivă, întărim Legea.

\chapter{4}

\par 1 Deci, ce vom zice că a dobândit după trup strămoșul nostru Avraam?
\par 2 Căci dacă Avraam s-a îndreptat din fapte, are de ce să se laude, dar nu înaintea lui Dumnezeu.
\par 3 Căci, ce spune Scriptura? Și "Avraam a crezut lui Dumnezeu și i s-a socotit lui ca dreptate".
\par 4 Celui care face fapte, nu i se socotește plata după har, ci după datorie;
\par 5 Iar celui care nu face fapte, ci crede în Cel ce îndreptează pe cel păcătos, credința lui i se socotește ca dreptate.
\par 6 Precum și David vorbește despre fericirea omului căruia Dumnezeu îi socotește dreptatea fără fapte:
\par 7 "Fericiți aceia, cărora li s-au iertat fărădelegile și ale căror păcate li s-au acoperit!
\par 8 Fericit bărbatul căruia Domnul nu-i va socoti păcatul".
\par 9 Deci fericirea aceasta este ea numai pentru cei tăiați împrejur sau și pentru cei netăiați împrejur? Căci zicem: "I s-a socotit lui Avraam credința ca dreptate".
\par 10 Dar cum i s-a socotit? Când era tăiat împrejur sau când era netăiat împrejur? Nu când era tăiat împrejur, ci când era netăiat împrejur.
\par 11 Iar semnul tăierii împrejur l-a primit ca pecete a dreptății pentru credința lui din vremea netăierii împrejur, ca să fie el părinte al tuturor celor ce cred, netăiați împrejur, pentru a li se socoti și lor (credința) ca dreptate,
\par 12 Și părinte al celor tăiați împrejur. Dar nu numai al celor care sunt tăiați împrejur, ci și care umblă pe urmele credinței pe care o avea părintele nostru Avraam, pe când era netăiat împrejur.
\par 13 Pentru că Avraam și seminția lui nu prin lege au primit făgăduința că vor moșteni lumea, ci prin dreptatea cea din credință.
\par 14 Căci dacă moștenitorii sunt cei ce au legea, atunci credința a ajuns zadarnică, iar făgăduința s-a desființat,
\par 15 Căci legea pricinuiește mâine; dar unde nu este lege, nu este nici călcare de lege.
\par 16 De aceea (moștenirea făgăduită) este din credință, ca să fie din har și ca făgăduința să rămână sigură pentru toți urmașii, nu numai pentru toți cei ce se țin de lege, ci și pentru cei ce se țin de credința lui Avraam, care este părinte al nostru al tuturor,
\par 17 Precum este scris: "Te-am pus părinte al multor neamuri", în fața Celui în Care a crezut, a lui Dumnezeu, Care înviază morții și cheamă la ființă cele ce încă nu sunt;
\par 18 Împotriva oricărei nădejdi, Avraam a crezut cu nădejde că el va fi părinte al multor neamuri, după cum i s-a spus: "Așa va fi seminția ta";
\par 19 Și neslăbind în credință, nu s-a uitat la trupul său amorțit - căci era aproape de o sută de ani - și nici la amorțirea pântecelui Sarrei;
\par 20 Și nu s-a îndoit, prin necredință, de făgăduința lui Dumnezeu, ci s-a întărit în credință, dând slavă lui Dumnezeu,
\par 21 Și fiind încredințat că ceea ce i-a făgăduit are putere să și facă.
\par 22 De acea, credința lui i s-a socotit ca dreptate.
\par 23 Și nu s-a scris numai pentru el că i s-a socotit ca dreptate,
\par 24 Ci se va socoti și pentru noi, cei care credem în Cel ce a înviat din morți pe Iisus, Domnul nostru,
\par 25 Care S-a dat pentru păcatele noastre și a înviat pentru îndreptarea noastră.

\chapter{5}

\par 1 Deci fiind îndreptați din credință, avem pace cu Dumnezeu, prin Domnul nostru Iisus Hristos,
\par 2 Prin Care am avut și apropiere, prin credință, la harul acesta, în care stăm, și ne lăudăm întru nădejdea slavei lui Dumnezeu.
\par 3 Și nu numai atât, ci ne lăudăm și în suferințe, bine știind că suferința aduce răbdare,
\par 4 Și răbdarea încercare, și încercarea nădejde
\par 5 Iar nădejdea nu rușinează pentru că iubirea lui Dumnezeu s-a vărsat în inimile noastre, prin Duhul Sfânt, Cel dăruit nouă.
\par 6 Căci Hristos, încă fiind noi neputincioși, la timpul hotărât a murit pentru cei necredincioși.
\par 7 Căci cu greu va muri cineva pentru un drept; dar pentru cel bun poate se hotărăște cineva să moară.
\par 8 Dar Dumnezeu Își arată dragostea Lui față de noi prin aceea că, pentru noi, Hristos a murit când noi eram încă păcătoși.
\par 9 Cu atât mai vârtos, deci, acum, fiind îndreptați prin sângele Lui, ne vom izbăvi prin El de mânie.
\par 10 Căci dacă, pe când eram vrăjmași, ne-am împăcat cu Dumnezeu, prin moartea Fiului Său, cu atât mai mult, împăcați fiind, ne vom mântui prin viața Lui.
\par 11 Și nu numai atât, ci și ne lăudăm în Dumnezeu prin Domnul nostru Iisus Hristos, prin Care am primit acum împăcarea.
\par 12 De aceea, precum printr-un om a intrat păcatul în lume și prin păcat moartea, așa și moartea a trecut la toți oamenii, pentru că toți au păcătuit în el.
\par 13 Căci, până la lege, păcatul era în lume, dar păcatul nu se socotește când nu este lege.
\par 14 Ci a împărățit moartea de la Adam până la Moise și peste cei ce nu păcătuiseră, după asemănarea greșelii lui Adam, care este chip al Celui ce avea să vină.
\par 15 Dar nu este cu greșeala cum este cu harul, căci dacă prin greșeala unuia cei mulți au murit, cu mult mai mult harul lui Dumnezeu și darul Lui au prisosit asupra celor mulți, prin harul unui singur om, Iisus Hristos.
\par 16 Și ce aduce darul nu seamănă cu ce a adus acel unul care a păcătuit; căci judecata dintr-unul duce la osândire, iar harul din multe greșeli duce la îndreptare.
\par 17 Căci, dacă prin greșeala unuia moartea a împărățit printr-unul, cu mult mai mult cei ce primesc prisosința harului și a darului dreptății vor împărăți în viață prin Unul Iisus Hristos.
\par 18 Așadar, precum prin greșeala unuia a venit osânda pentru toți oamenii, așa și prin îndreptarea adusă de Unul a venit, pentru toți oamenii, îndreptarea care dă viață;
\par 19 Căci precum prin neascultarea unui om s-au făcut păcătoși cei mulți, tot așa prin ascultarea unuia se vor face drepți cei mulți.
\par 20 Iar Legea a intrat și ea ca se înmulțească greșeala; iar unde s-a înmulțit păcatul, a prisosit harul;
\par 21 Pentru că precum a împărățit păcatul prin moarte, așa și harul să împărățească prin dreptate, spre viața veșnică, prin Iisus Hristos, Domnul nostru.

\chapter{6}

\par 1 Ce vom zice deci? Rămâne-vom, oare, în păcat, ca să se înmulțească harul?
\par 2 Nicidecum! Noi care am murit păcatului, cum vom mai trăi în păcat?
\par 3 Au nu știți că toți câți în Hristos Iisus ne-am botezat, întru moartea Lui ne-am botezat?
\par 4 Deci ne-am îngropat cu El, în moarte, prin botez, pentru ca, precum Hristos a înviat din morți, prin slava Tatălui, așa să umblăm și noi întru înnoirea vieții;
\par 5 Căci dacă am fost altoiți pe El prin asemănarea morții Lui, atunci vom fi părtași și ai învierii Lui,
\par 6 Cunoscând aceasta, că omul nostru cel vechi a fost răstignit împreună cu El, ca să se nimicească trupul păcatului, pentru a nu mai fi robi ai păcatului.
\par 7 Căci Cel care a murit a fost curățit de păcat.
\par 8 Iar dacă am murit împreună cu Hristos, credem că vom și viețui împreună cu El,
\par 9 Știind că Hristos, înviat din morți, nu mai moare. Moarta nu mai are stăpânire asupra Lui.
\par 10 Căci ce a murit, a murit păcatului o dată pentru totdeauna, iar ce trăiește, trăiește lui Dumnezeu.
\par 11 Așa și voi, socotiți-vă că sunteți morți păcatului, dar vii pentru Dumnezeu, în Hristos Iisus, Domnul nostru.
\par 12 Deci să nu împărățească păcatul în trupul vostru cel muritor, ca să vă supuneți poftelor lui;
\par 13 Nici să nu puneți mădularele voastre ca arme ale nedreptății în slujba păcatului, ci, înfățișați-vă pe voi lui Dumnezeu, ca vii, sculați din morți, și mădularele voastre ca arme ale dreptății lui Dumnezeu.
\par 14 Căci păcatul nu va avea stăpânire asupra voastră, fiindcă nu sunteți sub lege, ci sub har.
\par 15 Oare, atunci să păcătuim fiindcă nu suntem sub lege, ci sub har? Nicidecum!
\par 16 Au nu știți că celui ce vă dați spre ascultare robi, sunteți robi aceluia căruia vă supuneți: fie ai păcatului spre moarte, fie ai ascultării spre dreptate?
\par 17 Mulțumim însă lui Dumnezeu, că (deși) erați robi ai păcatului, v-ați supus din toată inima dreptarului învățăturii căreia ați fost încredințați,
\par 18 Și izbăvindu-vă de păcat, v-ați făcut robi ai dreptății.
\par 19 Omenește vorbesc, pentru slăbiciunea trupului vostru. - Căci precum ați făcut mădularele voastre roabe necurăției și fărădelegii, spre fărădelege, tot așa faceți acum mădularele voastre roabe dreptății, spre sfințire.
\par 20 Căci atunci, când erați robi ai păcatului, erați liberi față de dreptate.
\par 21 Deci ce roadă aveați atunci? Roade de care acum vă e rușine; pentru că sfârșitul acelora este moartea.
\par 22 Dar acum, izbăviți fiind de păcat și robi făcându-vă lui Dumnezeu, aveți roada voastră spre sfințire, iar sfârșitul, viață veșnică.
\par 23 Pentru că plata păcatului este moartea, iar harul lui Dumnezeu, viața veșnică, în Hristos Iisus, Domnul nostru.

\chapter{7}

\par 1 Oare nu știți, fraților - căci celor ce cunosc Legea vorbesc - că Legea are putere asupra omului, atâta timp cât el trăiește?
\par 2 Căci femeia măritată e legată, prin lege, de bărbatul său atâta timp cât el trăiește; iar dacă i-a murit bărbatul, este dezlegată de legea bărbatului.
\par 3 Deci, trăindu-i bărbatul, se va numi adulteră dacă va fi cu alt bărbat; iar dacă i-a murit bărbatul este liberă față de lege, ca să nu fie adulteră, luând un alt bărbat.
\par 4 Așa că, frații mei, și voi ați murit Legii, prin trupul lui Hristos, spre a fi ai altuia, ai Celui ce a înviat din morți, ca să aducem roade lui Dumnezeu.
\par 5 Căci pe când eram în trup, patimile păcatelor, care erau prin Lege, lucrau în mădularele noastre, ca să aducem roade morții;
\par 6 Dar acum ne-am desfăcut de Lege, murind aceluia în care eram ținuți robi, ca noi să slujim întru înnoirea Duhului, iar nu după slova cea veche.
\par 7 Ce vom zice deci? Au doară Legea este păcat? Nicidecum. Dar eu n-am cunoscut păcatul,  decât prin Lege. Căci n-aș fi știut pofta, dacă Legea n-ar fi zis: Să nu poftești!
\par 8 Dar păcatul, luând pricină prin poruncă, a lucrat în mine tot felul de pofte. Căci fără lege, păcatul era mort.
\par 9 Iar eu cândva trăiam fără lege, dar după ce a venit porunca, păcatul a prins viață;
\par 10 Iar eu am murit! Și porunca, dată spre viață, mi s-a aflat a fi spre moarte.
\par 11 Pentru că păcatul, luând îndemn prin poruncă, m-a înșelat și m-a ucis prin ea.
\par 12 Deci, Legea e sfântă și porunca e sfântă și dreaptă și bună.
\par 13 Atunci, ce era bun s-a făcut pentru mine pricina morții? Nicidecum! Ci păcatul, ca să se arate păcat, mi-a adus moartea, prin ceea ce a fost bun, pentru ca păcatul, prin poruncă, să fie peste măsură de păcătos.
\par 14 Căci știm că Legea e duhovnicească; dar eu sunt trupesc, vândut sub păcat.
\par 15 Pentru că ceea ce fac nu știu; căci nu săvârșesc ceea ce voiesc, ci fac ceea ce urăsc.
\par 16 Iar dacă fac ceea ce nu voiesc, recunosc că Legea este bună.
\par 17 Dar acum nu eu fac acestea, ci păcatul care locuiește în mine.
\par 18 Fiindcă știu că nu locuiește în mine, adică în trupul meu, ce este bun. Căci a voi se află în mine, dar a face binele nu aflu;
\par 19 Căci nu fac binele pe care îl voiesc, ci răul pe care nu-l voiesc, pe acela îl săvârșesc.
\par 20 Iar dacă fac ceea ce nu voiesc eu, nu eu fac aceasta, ci păcatul care locuiește în mine.
\par 21 Găsesc deci în mine, care voiesc să fac bine, legea că răul este legat de mine.
\par 22 Că, după omul cel lăuntric, mă bucur de legea lui Dumnezeu;
\par 23 Dar văd în mădularele mele o altă lege, luptându-se împotriva legii minții mele și făcându-mă rob legii păcatului, care este în mădularele mele.
\par 24 Om nenorocit ce sunt! Cine mă va izbăvi de trupul morții acesteia?
\par 25 Mulțumesc lui Dumnezeu, prin Iisus Hristos, Domnul nostru! Deci, dar, eu însumi, cu mintea mea, slujesc legii lui Dumnezeu, iar cu trupul, legii păcatului.

\chapter{8}

\par 1 Drept aceea nici o osândă nu este acum asupra celor ce sunt în Hristos Iisus.
\par 2 Căci legea duhului vieții în Hristos Iisus m-a eliberat de legea păcatului și a morții,
\par 3 Pentru că ceea ce era cu neputință Legii - fiind slabă prin trup - a săvârșit Dumnezeu, trimițând pe Fiul Său întru asemănarea trupului păcatului și pentru păcat a osândit păcatul în trup,
\par 4 Pentru ca îndreptarea din Lege să se împlinească în noi, care nu umblăm după trup, ci după duh.
\par 5 Căci cei ce sunt după trup cugetă cele ale trupului, iar cei ce sunt după Duh, cele ale Duhului.
\par 6 Căci dorința cărnii este moarte dar dorința Duhului este viață și pace;
\par 7 Fiindcă dorința cărnii este vrăjmășie împotriva lui Dumnezeu, căci nu se supune legii lui Dumnezeu, că nici nu poate.
\par 8 Iar cei ce sunt în carne nu pot să placă lui Dumnezeu.
\par 9 Dar voi nu sunteți în carne, ci în Duh, dacă Duhul lui Dumnezeu locuiește în voi. Iar dacă cineva nu are Duhul lui Hristos, acela nu este al Lui.
\par 10 Iar dacă Hristos este în voi, trupul este mort pentru păcat; iar Duhul, viață pentru dreptate,
\par 11 Iar dacă Duhul Celui ce a înviat pe Iisus din morți locuiește în voi, Cel ce a înviat pe Hristos Iisus din morți va face vii și trupurile voastre cele muritoare, prin Duhul Său care locuiește în voi.
\par 12 Drept aceea, fraților, nu suntem datori trupului, ca să viețuim după trup.
\par 13 Căci dacă viețuiți după trup, veți muri, iar dacă ucideți, cu Duhul, faptele trupului, veți fi vii.
\par 14 Căci câți sunt mânați de Duhul lui Dumnezeu sunt fii ai lui Dumnezeu.
\par 15 Pentru că n-ați primit iarăși un duh al robiei, spre temere, ci ați primit Duhul înfierii, prin care strigăm: Avva! Părinte!
\par 16 Duhul însuși mărturisește împreună cu duhul nostru că suntem fii ai lui Dumnezeu.
\par 17 Și dacă suntem fii, suntem și moștenitori - moștenitori ai lui Dumnezeu și împreună-moștenitori cu Hristos, dacă pătimim împreună cu El, ca împreună cu El să ne și preamărim.
\par 18 Căci socotesc că pătimirile vremii de acum nu sunt vrednice de mărirea care ni se va descoperi.
\par 19 Pentru că făptura așteaptă cu nerăbdare descoperirea fiilor lui Dumnezeu.
\par 20 Căci făptura a fost supusă deșertăciunii - nu din voia ei, ci din cauza aceluia care a supus-o - cu nădejde,
\par 21 Pentru că și făptura însăși se va izbăvi din robia stricăciunii, ca să fie părtașă la libertatea măririi fiilor lui Dumnezeu.
\par 22 Căci știm că toată făptura împreună suspină și împreună are dureri până acum.
\par 23 Și nu numai atât, ci și noi, care avem pârga Duhului, și noi înșine suspinăm în noi, așteptând înfierea, răscumpărarea trupului nostru.
\par 24 Căci prin nădejde ne-am mântuit; dar nădejdea care se vede nu mai e nădejde. Cum ar nădăjdui cineva ceea ce vede?
\par 25 Iar dacă nădăjduim ceea ce nu vedem, așteptăm prin răbdare.
\par 26 De asemenea și Duhul vine în ajutor slăbiciunii noastre, căci noi nu știm să ne rugăm cum trebuie, ci Însuși Duhul Se roagă pentru noi cu suspine negrăite.
\par 27 Iar Cel ce cercetează inimile știe care este dorința Duhului, căci după Dumnezeu El Se roagă pentru sfinți.
\par 28 Și știm că Dumnezeu toate le lucrează spre binele celor ce iubesc pe Dumnezeu, al celor care sunt chemați după voia Lui;
\par 29 Căci pe cei pe care i-a cunoscut mai înainte, mai înainte i-a și hotărât să fie asemenea chipului Fiului Său, ca El să fie întâi născut între mulți frați.
\par 30 Iar pe care i-a hotărât mai înainte, pe aceștia i-a și chemat; și pe care i-a chemat, pe aceștia i-a și îndreptat; iar pe care i-a îndreptat, pe aceștia i-a și mărit.
\par 31 Ce vom zice deci la acestea? Dacă Dumnezeu e pentru noi, cine este împotriva noastră?
\par 32 El, Care pe Însuși Fiul Său nu L-a cruțat, ci L-a dat morții, pentru noi toți, cum nu ne va da, oare, toate împreună cu El?
\par 33 Cine va ridica pâră împotriva aleșilor lui Dumnezeu? Dumnezeu este Cel ce îndreptează;
\par 34 Cine este Cel ce osândește? Hristos, Cel ce a murit, și mai ales Cel ce a înviat, Care și este de-a dreapta lui Dumnezeu, Care mijlocește pentru noi!
\par 35 Cine ne va despărți pe noi de iubirea lui Hristos? Necazul, sau strâmtorarea, sau prigoana, sau foametea, sau lipsa de îmbrăcăminte, sau primejdia, sau sabia?
\par 36 Precum este scris: "Pentru Tine suntem omorâți toată ziua, socotiți am fost ca niște oi de junghiere".
\par 37 Dar în toate acestea suntem mai mult decât biruitori, prin Acela Care ne-a iubit.
\par 38 Căci sunt încredințat că nici moartea, nici viața, nici îngerii, nici stăpânirile, nici cele de acum, nici cele ce vor fi, nici puterile,
\par 39 Nici înălțimea, nici adâncul și nici o altă făptură nu va putea să ne despartă pe noi de dragostea lui Dumnezeu, cea întru Hristos Iisus, Domnul nostru.

\chapter{9}

\par 1 Spun adevărul în Hristos, nu mint, martor fiindu-mi conștiința mea în Duhul Sfânt,
\par 2 Că mare îmi este întristarea și necurmată durerea inimii.
\par 3 Căci aș fi dorit să fiu eu însumi anatema de la Hristos pentru frații mei, cei de un neam cu mine, după trup,
\par 4 Care sunt israeliți, ale cărora sunt înfierea și slava și legămintele și Legea și închinarea și făgăduințele,
\par 5 Ai cărora sunt părinții și din care după trup este Hristos, Cel ce este peste toate Dumnezeu, binecuvântat în veci. Amin!
\par 6 Dar nu așa că ar fi căzut cuvântul lui Dumnezeu: căci nu toți cei din Israel sunt și israeliți;
\par 7 Nici pentru că sunt urmașii lui Avraam, sunt toți fii, ci "întru Isaac, a zis, se vor chema ție urmași",
\par 8 Adică: Nu copiii trupului sunt copii ai lui Dumnezeu, ci fiii făgăduinței se socotesc urmași.
\par 9 Căci al făgăduinței este cuvântul acesta: "(La anul) pe vremea aceasta voi veni și Sara va avea un fiu".
\par 10 Dar nu numai ea, ci și Rebeca, având copii gemeni dintr-unul, Isaac, părintele nostru;
\par 11 Și nefiind ei încă născuți și nefăcând ei ceva bun sau rău, ca să rămână voia lui Dumnezeu cea după alegere, nu din fapte, ci de la Cel care cheamă,
\par 12 I s-a zis ei că "cel mai mare va sluji celui mai mic",
\par 13 Precum este scris: "Pe Iacov l-am iubit, iar pe Isav l-am urât".
\par 14 Ce vom zice dar? Nu cumva la Dumnezeu este nedreptate? Nicidecum!
\par 15 Căci grăiește către Moise: "Voi milui pe cine vreau să-l miluiesc și Mă voi îndura de cine vreau să Mă îndur".
\par 16 Deci, dar, nu este nici de la cel care voiește, nici de la cel ce aleargă, ci de la Dumnezeu care miluiește.
\par 17 Căci Scriptura zice lui Faraon: "Pentru aceasta chiar te-am ridicat, ca să arăt în tine puterea Mea și ca numele Meu să se vestească în tot pământul".
\par 18 Deci, dar, Dumnezeu pe cine voiește îl miluiește, iar pe cine voiește îl împietrește.
\par 19 Îmi vei zice deci: De ce mai dojenește? Căci voinței Lui cine i-a stat împotrivă?
\par 20 Dar, omule, tu cine ești care răspunzi împotriva lui Dumnezeu? Oare făptura va zice Celui ce a făcut-o: De ce m-ai făcut așa?
\par 21 Sau nu are olarul putere peste lutul lui, ca din aceeași frământătură să facă un vas de cinste, iar altul de necinste?
\par 22 Și ce este dacă Dumnezeu, voind să-Și arate mânia și să facă cunoscută puterea Sa, a suferit cu multă răbdare vasele mâniei Sale, gătite spre pierire,
\par 23 Și ca să facă cunoscută bogăția slavei Sale către vasele milei, pe care mai dinainte le-a gătit spre slavă?
\par 24 Adică pe noi, pe care ne-a și chemat, nu numai dintre iudei, ci și dintre păgâni,
\par 25 Precum zice El și la Osea: "Chema-voi poporul Meu pe cel ce nu este poporul Meu, și iubită pe cea care nu era iubită;
\par 26 Și va fi în locul unde li s-a zis lor: Nu voi sunteți poporul Meu - acolo se vor chema fii ai Dumnezeului Celui viu".
\par 27 Iar Isaia strigă pentru Israel : "Dacă numărul fiilor lui Israel ar fi ca nisipul mării, rămășița se va mântui.
\par 28 Pentru că împlinind și scurtând, Domnul va îndeplini, pe pământ, cuvântul Său".
\par 29 Și precum a proorocit Isaia: "Dacă Domnul Savaot nu ne-ar fi lăsat nouă urmași, am fi ajuns ca Sodoma și ne-am fi asemănat cu Gomora".
\par 30 Ce vom zice, deci? Că neamurile care nu căutau dreptatea au dobândit dreptatea, însă dreptatea din credință;
\par 31 Iar Israel, urmărind legea dreptății, n-a ajuns la legea dreptății.
\par 32 Pentru ce? Pentru că nu o căutau din credință, ci ca din faptele Legii. S-au poticnit de piatra poticnirii,
\par 33 Precum este scris: "Iată pun în Sion piatră de poticnire și piatră de sminteală; și tot cel ce crede în El nu se va rușina".

\chapter{10}

\par 1 Fraților, bunăvoința inimii mele și rugăciunea mea către Dumnezeu, pentru Israel, este spre mântuire.
\par 2 Căci le mărturisesc că au râvnă pentru Dumnezeu, dar sunt fără cunoștință.
\par 3 Deoarece, necunoscând dreptatea lui Dumnezeu și căutând să statornicească dreptatea lor, dreptății lui Dumnezeu ei nu s-au supus.
\par 4 Căci sfârșitul Legii este Hristos, spre dreptate tot celui ce crede.
\par 5 Căci Moise scrie despre dreptatea care vine din lege, că: "Omul care o va îndeplini va trăi prin ea".
\par 6 Iar dreptatea din credință grăiește așa: "Să nu zici în inima ta: Cine se va sui la cer?", ca adică să coboare pe Hristos!
\par 7 Sau: "Cine se va coborî întru adânc?", ca să ridice pe Hristos din morți!
\par 8 Dar ce zice Scriptura? "Aproape este de tine cuvântul, în gura ta și în inima ta", - adică cuvântul credinței pe care-l propovăduim.
\par 9 Că de vei mărturisi cu gura ta că Iisus este Domnul și vei crede în inima ta că Dumnezeu L-a înviat pe El din morți, te vei mântui.
\par 10 Căci cu inima se crede spre dreptate, iar cu gura se mărturisește spre mântuire.
\par 11 Căci zice Scriptura: "Tot cel ce crede în El nu se va rușina".
\par 12 Căci nu este deosebire între iudeu și elin, pentru că Același este Domnul tuturor, Care îmbogățește pe toți cei ce-L cheamă pe El.
\par 13 Căci: "Oricine va chema numele Domnului se va mântui".
\par 14 Dar cum vor chema numele Aceluia în Care încă n-au crezut? Și cum vor crede în Acela de Care n-au auzit? Și cum vor auzi, fără propovăduitor?
\par 15 Și cum vor propovădui, de nu vor fi trimiși? Precum este scris: "Cât de frumoase sunt picioarele celor ce vestesc pacea, ale celor ce vestesc cele bune!"
\par 16 Dar nu toți s-au supus Evangheliei, căci Isaia zice: "Doamne, cine a crezut celor auzite de la noi?"
\par 17 Prin urmare, credința este din auzire, iar auzirea prin cuvântul lui Hristos.
\par 18 Dar întreb: Oare n-au auzit? Dimpotrivă: "În tot pământul a ieșit vestirea lor și la marginile lumii cuvintele lor".
\par 19 Dar zic: Nu cumva Israel n-a înțeles? Moise spune cel dintâi: "Voi întărâta râvna voastră prin cel ce nu este poporul (Meu) și voi ațâța mânia voastră cu un popor nepriceput".
\par 20 Isaia îndrăznește și zice: "Am fost aflat de cei ce nu Mă căutau și M-am făcut arătat celor ce nu întrebau de Mine".
\par 21 Dar către Israel zice: "Toată ziua întins-am mâinile Mele către un popor neascultător și împotrivă grăitor".

\chapter{11}

\par 1 Întreb deci: Oare lepădat-a Dumnezeu pe poporul Său? Nicidecum! Căci și eu sunt israelit, din urmașii lui Avraam, din seminția lui Veniamin.
\par 2 Nu a lepădat Dumnezeu pe poporul Său, pe care mai înainte l-a cunoscut. Nu știți, oare, ce zice Scriptura despre Ilie? Cum se roagă el împotriva lui Israel, zicând:
\par 3 "Doamne, pe proorocii Tăi i-au omorât, jertfelnicele Tale le-au surpat și eu am rămas singur și ei caută să-mi ia sufletul!".
\par 4 Dar ce-i spune dumnezeiescul răspuns? "Mi-am pus deoparte șapte mii de bărbați, care nu și-au plecat genunchiul înaintea lui Baal".
\par 5 Deci tot așa și în vremea de acum este o rămășiță aleasă prin har.
\par 6 Iar dacă este prin har, nu mai este din fapte; altfel harul nu mai este har. Iar dacă este din fapte, nu mai este har, altfel fapta nu mai este faptă.
\par 7 Ce este deci? Nu tot Israelul a dobândit ceea ce căuta; ci cei aleși au dobândit, iar ceilalți s-au împietrit,
\par 8 Precum este scris: "Dumnezeu le-a dat duh de amorțire, ochi ca să nu vadă și urechi ca să nu audă până în ziua de azi".
\par 9 Iar David zice: "Facă-se masa lor cursă și laț și sminteală și răsplătire lor!
\par 10 Întunce-se ochii lor ca să nu vadă și spinarea lor încovoaie-o pentru totdeauna!"
\par 11 Deci, întreb: S-a poticnit, oare, ca să cadă? Nicidecum! Și prin căderea lor, neamurilor le-a venit mântuirea, ca Israel să-și întărâte râvna față de ele.
\par 12 Dar dacă greșeala lor a fost bogăție lumii și micșorarea lor bogăție neamurilor, cu cât mai mult întreg numărul lor!
\par 13 Căci v-o spun vouă, neamurilor: Întru cât sunt eu, deci, apostol al neamurilor, slăvesc slujirea mea,
\par 14 Doar voi izbuti să ațâț râvna celor din neamul meu și să mântuiesc pe unii dintre ei.
\par 15 Căci dacă înlăturarea lor a adus împăcarea lumii, ce va fi primirea lor la loc, dacă nu o înviere din morți?
\par 16 Iar dacă este pârga (de făină) sfântă, și frământătura este sfântă; și dacă rădăcina este sfântă, și ramurile sunt.
\par 17 Iar dacă unele din ramuri au fost tăiate, și tu, care erai măslin sălbatic, ai fost altoit printre cele rămase, și părtaș te-ai făcut rădăcinii și grăsimii măslinului,
\par 18 Nu te mândri față de ramuri; iar dacă te mândrești, nu tu porți rădăcina, ci rădăcina pe tine.
\par 19 Dar vei zice: Au fost tăiate ramurile, ca să fiu altoit eu.
\par 20 Bine! Din cauza necredinței au fost tăiate, iar tu stai prin credință. Nu te îngâmfa, ci teme-te;
\par 21 Căci dacă Dumnezeu n-a cruțat ramurile firești, nici pe tine nu te va cruța.
\par 22 Vezi deci bunătatea și asprimea lui Dumnezeu: Asprimea Lui către cei ce au căzut în bunătatea Lui către tine, dacă vei stărui în această bunătate; altfel și tu vei fi tăiat.
\par 23 Dar și aceia, de nu vor stărui în necredință, vor fi altoiți; căci puternic este Dumnezeu să-i altoiască iarăși.
\par 24 Căci dacă tu ai fost tăiat din măslinul cel din fire sălbatic și împotriva firii ai fost altoit în măslin bun, cu atât mai vârtos aceștia, care sunt după fire, vor fi altoiți în însuși măslinul lor.
\par 25 Pentru că nu voiesc, fraților, ca voi să nu știți taina aceasta, ca să nu vă socotiți pe voi înșivă înțelepți; că împietrirea s-a făcut lui Israel în parte, până ce va intra tot numărul neamurilor.
\par 26 Și astfel întregul Israel se va mântui, precum este scris: "Din Sion va veni Izbăvitorul și va îndepărta nelegiuirile de la Iacov;
\par 27 Și acesta este legământul Meu cu ei, când voi ridica păcatele lor".
\par 28 Cât privește Evanghelia, ei sunt vrăjmași din pricina voastră, dar cu privire la alegere ei sunt iubiți, din cauza părinților.
\par 29 Căci darurile și chemarea lui Dumnezeu nu se pot lua înapoi.
\par 30 După cum voi, cândva, n-ați ascultat de Dumnezeu, dar acum ați fost miluiți prin neascultarea acestora,
\par 31 Tot așa și aceștia n-au ascultat acum, ca, prin mila către voi, să fie miluiți și ei acum.
\par 32 Căci Dumnezeu i-a închis pe toți în neascultare, pentru ca pe toți să-i miluiască.
\par 33 O, adâncul bogăției și al înțelepciunii și al științei lui Dumnezeu! Cât sunt de necercetate judecățile Lui și cât sunt de nepătrunse căile Lui!
\par 34 Căci cine a cunoscut gândul Domnului sau cine a fost sfetnicul Lui?
\par 35 Sau cine mai înainte I-a dat Lui și va lua înapoi de la El?
\par 36 Pentru că de la El și prin El și întru El sunt toate. A Lui să fie mărirea în veci. Amin!

\chapter{12}

\par 1 Vă îndemn, deci, fraților, pentru îndurările lui Dumnezeu, să înfățișați trupurile voastre ca pe o jertfă vie, sfântă, bine plăcută lui Dumnezeu, ca închinarea voastră cea duhovnicească,
\par 2 Și să nu vă potriviți cu acest veac, ci să vă schimbați prin înnoirea minții, ca să deosebiți care este voia lui Dumnezeu, ce este bun și plăcut și desăvârșit.
\par 3 Căci, prin harul ce mi s-a dat, spun fiecăruia din voi să nu cugete despre sine mai mult decât trebuie să cugete, ci să cugete fiecare spre a fi înțelept, precum Dumnezeu i-a împărțit măsura credinței.
\par 4 Ci precum într-un singur trup avem multe mădulare și mădularele nu au toate aceeași lucrare,
\par 5 Așa și noi, cei mulți, un trup suntem în Hristos și fiecare suntem mădulare unii altora;
\par 6 Dar avem felurite daruri, după harul ce ni s-a dat. Dacă avem proorocie, să proorocim după  măsura credinței;
\par 7 Dacă avem slujbă, să stăruim în slujbă; dacă unul învață, să se sârguiască în învățătură;
\par 8 Dacă îndeamnă, să fie la îndemnare; dacă împarte altora, să împartă cu firească nevinovăție; dacă stă în frunte, să fie cu tragere de inimă; dacă miluiește, să miluiască cu voie bună!
\par 9 Dragostea să fie nefățarnică. Urâți răul, alipiți-vă de bine.
\par 10 În iubire frățească, unii pe alții iubiți-vă; în cinste, unii altora dați-vă întâietate.
\par 11 La sârguință, nu pregetați; cu duhul fiți fierbinți; Domnului slujiți.
\par 12 Bucurați-vă în nădejde; în suferință fiți răbdători; la rugăciune stăruiți.
\par 13 Faceți-vă părtași la trebuințele sfinților, iubirea de străini urmând.
\par 14 Binecuvântați pe cei ce vă prigonesc, binecuvântați-i și nu-i blestemați.
\par 15 Bucurați-vă cu cei ce se bucură; plângeți cu cei ce plâng.
\par 16 Cugetați același lucru unii pentru alții; nu cugetați la cele înalte, ci lăsați-vă duși de spre cele smerite. Nu vă socotiți voi înșivă înțelepți.
\par 17 Nu răsplătiți nimănui răul cu rău. Purtați grijă de cele bune înaintea tuturor oamenilor.
\par 18 Dacă se poate, pe cât stă în puterea voastră, trăiți în bună pace cu toți oamenii.
\par 19 Nu vă răzbunați singuri, iubiților, ci lăsați loc mâniei (lui Dumnezeu), căci scris este: "A Mea este răzbunarea; Eu voi răsplăti, zice Domnul".
\par 20 Deci, dacă vrăjmașul tău este flămând, dă-i de mâncare; dacă îi este sete, dă-i să bea, căci, făcând acestea, vei grămădi cărbuni de foc pe capul lui.
\par 21 Nu te lăsa biruit de rău, ci biruiește răul cu binele.

\chapter{13}

\par 1 Tot sufletul să se supună înaltelor stăpâniri, căci nu este stăpânire decât de la Dumnezeu; iar cele ce sunt, de Dumnezeu sunt rânduite.
\par 2 Pentru aceea, cel ce se împotrivește stăpânirii se împotrivește rânduielii lui Dumnezeu. Iar cel ce se împotrivesc își vor lua osândă.
\par 3 Căci dregătorii nu sunt frică pentru fapta bună, ci pentru cea rea. Voiești, deci, să nu-ți fie frică de stăpânire? Fă binele și vei avea laudă de la ea.
\par 4 Căci ea este slujitoare a lui Dumnezeu spre binele tău. Iar dacă faci rău, teme-te; căci nu în zadar poartă sabia; pentru că ea este slujitoare a lui Dumnezeu și răzbunătoare a mâniei Lui, asupra celui ce săvârșește răul.
\par 5 De aceea este nevoie să vă supuneți, nu numai pentru mânie, ci și pentru conștiință.
\par 6 Că pentru aceasta plătiți și dări. Căci (dregătorii) sunt slujitorii lui Dumnezeu, stăruind în această slujire neîncetat.
\par 7 Dați deci tuturor cele ce sunteți datori: celui cu darea, darea; celui cu vama, vamă; celui cu teama, teamă; celui cu cinstea, cinste.
\par 8 Nimănui cu nimic nu fiți datori, decât cu iubirea unuia față de altul; că cel care iubește pe aproapele a împlinit legea.
\par 9 Pentru că (poruncile): Să nu săvârșești adulter; să nu ucizi; să nu furi; să nu mărturisești strâmb; să nu poftești... și orice altă poruncă ar mai fi se cuprind în acest cuvânt: Să iubești pe aproapele tău ca pe tine însuți.
\par 10 Iubirea nu face rău aproapelui; iubirea este deci împlinirea legii.
\par 11 Și aceasta, fiindcă știți în ce timp ne găsim, căci este chiar ceasul să vă treziți din somn; căci acum mântuirea este mai aproape de noi, decât atunci când am crezut.
\par 12 Noaptea e pe sfârșite; ziua este aproape. Să lepădăm dar lucrurile întunericului și să ne îmbrăcăm cu armele luminii.
\par 13 Să umblăm cuviincios, ca ziua: nu în ospețe și în beții, nu în desfrânări și în fapte de rușine, nu în ceartă și în pizmă;
\par 14 Ci îmbrăcați-vă în Domnul Iisus Hristos și grija de trup să nu o faceți spre pofte.

\chapter{14}

\par 1 Primiți-l pe cel slab în credință fără să-i judecați gândurile.
\par 2 Unul crede să mănânce de toate; cel slab însă mănâncă legume.
\par 3 Cel ce mănâncă să nu disprețuiască pe cel ce nu mănâncă; iar cel ce nu mănâncă să nu osândească pe cel ce mănâncă, fiindcă Dumnezeu l-a primit.
\par 4 Cine ești tu, ca să judeci pe sluga altuia? Pentru stăpânul său stă sau cade. Dar va sta, căci Domnul are putere ca să-l facă să stea.
\par 5 Unul deosebește o zi de alta, iar altul judecă toate zilele la fel. Fiecare să fie deplin încredințat în mintea lui.
\par 6 Cel ce ține ziua, o ține pentru Domnul; și cel ce nu ține ziua, nu o ține pentru Domnul. Și cel ce mănâncă pentru Domnul mănâncă, căci mulțumește lui Dumnezeu; și cel ce nu mănâncă pentru Domnul nu mănâncă, și mulțumește lui Dumnezeu.
\par 7 Căci nimeni dintre noi nu trăiește pentru sine și nimeni nu moare pentru sine.
\par 8 Că dacă trăim, pentru Domnul trăim, și dacă murim, pentru Domnul murim. Deci și dacă trăim, și dacă murim, ai Domnului suntem.
\par 9 Căci pentru aceasta a murit și a înviat Hristos, ca să stăpânească și peste morți și peste vii.
\par 10 Dar tu, de ce judeci pe fratele tău? Sau și tu, de ce disprețuiești pe fratele tău? Căci toți ne vom înfățișa înaintea judecății lui Dumnezeu.
\par 11 Căci scris este: "Viu sunt Eu! - zice Domnul - Tot genunchiul să Mi se plece și toată limba  să dea slavă lui Dumnezeu".
\par 12 Deci, dar, fiecare din voi va da seama despre sine lui Dumnezeu.
\par 13 Deci să nu ne mai judecăm unii pe alții, ci mai degrabă judecați aceasta: Să nu dați fratelui prilej de poticnire sau de sminteală.
\par 14 Știu și sunt încredințat în Domnul Iisus că nimic nu este întinat prin sine, decât numai pentru cel care gândește că e ceva întinat; pentru acela întinat este.
\par 15 Dar dacă, pentru mâncare, fratele tău se mâhnește, nu mai umbli potrivit iubirii. Nu pierde, cu mâncarea ta, pe acela pentru care a murit Hristos.
\par 16 Nu lăsați ca bunul vostru să fie defăimat.
\par 17 Căci împărăția lui Dumnezeu nu este mâncare și băutură, ci dreptate și pace și bucurie în Duhul Sfânt.
\par 18 Iar cel ce slujește lui Hristos, în aceasta este plăcut lui Dumnezeu și cinstit de oameni.
\par 19 Drept aceea să urmărim cele ale păcii și cele ale zidirii unuia de către altul.
\par 20 Nu strica, pentru mâncare, lucrul lui Dumnezeu. Toate sunt curate, dar rău este pentru omul care mănâncă spre poticnire.
\par 21 Bine este să nu mănânci carne, nici să bei vin, nici să faci ceva de care fratele tău se poticnește, se smintește sau slăbește (în credință).
\par 22 Credința pe care o ai, s-o ai pentru tine însuți, înaintea lui Dumnezeu. Fericit este cel ce nu se judecă pe sine în ceea ce aprobă!
\par 23 Iar cel ce se îndoiește, dacă va mânca, se osândește, fiindcă n-a fost din credință. Și tot ce nu este din credință este păcat.

\chapter{15}

\par 1 Datori suntem noi cei tari să purtăm slăbiciunile celor neputincioși și să nu căutăm plăcerea noastră.
\par 2 Ci fiecare dintre noi să caute să placă aproapelui său, la ce este bine, spre zidire.
\par 3 Că și Hristos n-a căutat plăcerea Sa, ci, precum este scris: "Ocările celor ce Te ocărăsc pe Tine, au căzut asupra Mea".
\par 4 Căci toate câte s-au scris mai înainte, s-au scris spre învățătura noastră, ca prin răbdarea și mângâierea, care vin din Scripturi, să avem nădejde.
\par 5 Iar Dumnezeul răbdării și al mângâierii să vă dea vouă a gândi la fel unii pentru alții, după Iisus Hristos,
\par 6 Pentru ca toți laolaltă și cu o singură gură să slăviți pe Dumnezeu și Tatăl Domnului nostru Iisus Hristos.
\par 7 De aceea, primiți-vă unii pe alții, precum și Hristos v-a primit pe voi, spre slava lui Dumnezeu.
\par 8 Căci spun: Că Hristos S-a făcut slujitor al tăierii împrejur pentru adevărul lui Dumnezeu, ca să întărească făgăduințele date părinților,
\par 9 Iar neamurile să slăvească pe Dumnezeu pentru mila Lui, precum este scris: "Pentru aceasta Te voi lăuda între neamuri și voi cânta numele Tău".
\par 10 Și iarăși zice Scriptura: "Veseliți-vă, neamuri, cu poporul Lui".
\par 11 Și iarăși: "Lăudați pe Domnul toate neamurile; lăudați-L pe El toate popoarele".
\par 12 Și iarăși Isaia zice: "Și Se va arăta rădăcina lui Iesei, Cel care Se ridică să domnească peste neamuri; întru Acela neamurile vor nădăjdui".
\par 13 Iar Dumnezeul nădejdii să vă umple pe voi de toată bucuria și pacea în credință, ca să prisosească nădejdea voastră, prin puterea Duhului Sfânt.
\par 14 Și, frații mei, sunt încredințat eu însumi despre voi, că și voi sunteți plini de bunătate, plini de toată cunoștința, putând să vă povățuiți unii pe alții.
\par 15 Și v-am scris, fraților, mai cu îndrăzneală, în parte, ca să vă amintesc despre harul ce mi-a fost dat de Dumnezeu,
\par 16 Ca să fiu slujitor al lui Iisus Hristos la neamuri, slujind Evanghelia lui Dumnezeu, pentru ca prinosul neamurilor, fiind sfințit în Duhul Sfânt, să fie bine primit.
\par 17 Așadar, în Hristos Iisus am laudă, în cele către Dumnezeu.
\par 18 Căci nu voi cuteza să spun ceva din cele ce n-a săvârșit Hristos prin mine, spre ascultarea neamurilor, prin cuvânt și prin faptă,
\par 19 Prin puterea semnelor și a minunilor, prin puterea Duhului Sfânt, așa încât de la Ierusalim și din ținuturile de primprejur până la Iliria, am împlinit propovăduirea Evangheliei lui Hristos,
\par 20 Râvnind astfel să binevestesc acolo unde Hristos nu fusese numit, ca să nu zidesc pe temelie străină,
\par 21 Ci precum este scris: "Cărora nu li s-a vestit despre El, aceia Îl vor vedea; și cei ce n-au auzit Îl vor înțelege".
\par 22 De aceea am și fost împiedicat, de multe ori, ca să vin la voi.
\par 23 Dar acum, nemaiavând loc în aceste ținuturi și având dorința de mulți ani să vin la voi,
\par 24 Când mă voi duce în Spania, voi veni la voi. Căci nădăjduiesc să vă văd în trecere și, de către voi, să fiu însoțit până acolo, după ce mă voi bucura întâi, în parte, de voi.
\par 25 Acum însă mă duc la Ierusalim, ca să slujesc sfinților.
\par 26 Căci Macedonia și Ahaia au binevoit să facă o strângere de ajutoare pentru săracii dintre sfinții de la Ierusalim.
\par 27 Căci ei au binevoit și sunt datori față de ei. Căci dacă neamurile s-au împărtășit de cele duhovnicești ale lor, datori sunt și ei să le slujească în cele trupești.
\par 28 Săvârșind deci aceasta și încredințându-le roada aceasta, voi trece pe la voi, în Spania.
\par 29 Și știu că, venind la voi, voi veni cu deplinătatea binecuvântării lui Hristos.
\par 30 Dar vă îndemn, fraților, pentru Domnul nostru Iisus Hristos și pentru iubirea Duhului Sfânt, ca împreună cu mine să luptați în rugăciuni către Dumnezeu pentru mine,
\par 31 Ca să scap de necredincioșii din Iudeea și ca ajutorul meu la Ierusalim să fie bine primit de către sfinți,
\par 32 Ca să vin la voi cu bucurie prin voia lui Dumnezeu și să-mi găsesc liniștea împreună cu voi.
\par 33 Iar Dumnezeul păcii să fie cu voi cu toți. Amin!

\chapter{16}

\par 1 Și vă încredințez pe Febe, sora noastră, care este diaconiță a Bisericii din Chenhrea,
\par 2 Ca s-o primiți în Domnul, cu vrednicia cuvenită sfinților și să-i fiți de ajutor la orice ar avea nevoie de ajutorul vostru. Căci și ea a ajutat pe mulți și pe mine însumi.
\par 3 Îmbrățișați pe Priscila și Acvila, împreună-lucrători cu mine în Hristos Iisus,
\par 4 Care și-au pus grumazul lor pentru viața mea și cărora nu numai eu le mulțumesc, ci și toate Bisericile dintre neamuri,
\par 5 Și Biserica din casa lor.  Îmbrățișați pe Epenet, iubitul meu, care este pârga Asiei, în Hristos.
\par 6 Îmbrățișați pe Maria care s-a ostenit mult pentru voi.
\par 7 Îmbrățișați pe Andronic și pe Iunias, cei de un neam cu mine și împreună închiși cu mine, care sunt vestiți între apostoli și care înaintea mea au fost în Hristos.
\par 8 Îmbrățișați pe Ampliat, iubitul meu în Domnul.
\par 9 Îmbrățișați pe Urban, împreună-lucrător cu mine în Hristos, și pe Stahis, iubitul meu.
\par 10 Îmbrățișați pe Apelles, cel încercat în Hristos. Îmbrățișați pe cei ce sunt din casa lui Aristobul.
\par 11 Îmbrățișați pe Irodion, cel de un neam cu mine. Îmbrățișați pe cei din casa lui Narcis, care sunt în Domnul.
\par 12 Îmbrățișați pe Trifena și pe Trifosa, care s-au ostenit în Domnul. Îmbrățișați pe iubita Persida, care mult s-a ostenit în Domnul.
\par 13 Îmbrățișați pe Ruf, cel ales întru Domnul, și pe mama lui, care este și a mea.
\par 14 Îmbrățișați pe Asincrit, pe Flegon, pe Hermes, pe Patrova, pe Hermas și pe frații care sunt împreună cu ei.
\par 15 Îmbrățișați pe Filolog și pe Iulia, pe Nereu și pe sora lui, pe Olimpian și pe toți sfinții care sunt împreună cu ei.
\par 16 Îmbrățișați-vă unii pe alții cu sărutare sfântă. Vă îmbrățișează pe voi toate Bisericile lui Hristos.
\par 17 Și vă îndemn, fraților, să vă păziți de cei ce fac dezbinări și sminteli împotriva învățăturii pe care ați primit-o. Depărtați-vă de ei.
\par 18 Căci unii ca aceștia nu slujesc Domnului nostru Iisus Hristos, ci pântecelui lor, și prin vorbele lor frumoase și măgulitoare, înșeală inimile celor fără de răutate.
\par 19 Căci ascultarea voastră este cunoscută de toți. Mă bucur deci de voi și voiesc să fiți înțelepți spre bine și nevinovați la rău.
\par 20 Iar Dumnezeul păcii va zdrobi repede sub picioarele voastre pe satana. Harul Domnului nostru Iisus Hristos cu voi!
\par 21 Vă îmbrățișează Timotei, cel împreună-lucrător cu mine, și Luciu și Iason și Sosipatru, cei de un neam cu mine,
\par 22 Vă îmbrățișez în Domnul eu, Tertius, care am scris epistola.
\par 23 Vă îmbrățișează Gaius, gazda mea și a toată Biserica. Vă îmbrățișează Erast, vistiernicul cetății, și fratele Cvartus.
\par 24 Harul Domnului nostru Iisus Hristos să fie cu voi cu toți. Amin!
\par 25 Iar celui ce poate să vă întărească după Evanghelia mea și după propovăduirea lui Iisus Hristos, potrivit cu descoperirea tainei celei ascunse din timpuri veșnice,
\par 26 Iar acum arătată prin Scripturile proorocilor, după porunca veșnicului Dumnezeu și cunoscută la toate neamurile spre ascultarea credinței,
\par 27 Unuia înțeleptului Dumnezeu, prin Iisus Hristos, fie slava în vecii vecilor. Amin!


\end{document}