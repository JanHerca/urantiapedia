\begin{document}

\title{Apocalipsa}


\chapter{1}

\par 1 Descoperirea lui Iisus Hristos, pe care I-a dat-o Dumnezeu, ca să arate robilor Săi cele ce trebuie să se petreacă în curând, iar El, prin trimiterea îngerului Său, a destăinuit-o robului Său Ioan,
\par 2 Care a mărturisit cuvântul lui Dumnezeu și mărturia lui Iisus Hristos, câte a văzut.
\par 3 Fericit este cel ce citește și cei ce ascultă cuvintele proorociei și păstrează cele scrise în aceasta! Căci vremea este aproape.
\par 4 Ioan, celor șapte Biserici, care sunt în Asia: Har vouă și pace de la Cel ce este și Cel ce era și Cel ce vine și de la cele șapte duhuri, care sunt înaintea scaunului Lui,
\par 5 Și de la Iisus Hristos, Martorul cel credincios, Cel întâi născut din morți, și Domnul împăraților pământului. Lui, Care ne iubește și ne-a dezlegat pe noi din păcatele noastre, prin sângele Său,
\par 6 Și ne-a făcut pe noi împărăție, preoți ai lui Dumnezeu și Tatăl Său, Lui fie slava și puterea, în vecii vecilor. Amin!
\par 7 Iată, El vine cu norii și orice ochi Îl va vedea și-L vor vedea și cei ce L-au împuns și se vor jeli, din pricina Lui, toate semințiile pământului. Așa. Amin.
\par 8 Eu sunt Alfa și Omega, zice Domnul Dumnezeu, Cel ce este, Cel ce era și Cel ce vine, Atotțiitorul.
\par 9 Eu Ioan, fratele vostru și împreună cu voi părtaș la suferința și la împărăția și la răbdarea în Iisus, fost-a în insula ce se cheamă Patmos, pentru cuvântul lui Dumnezeu și pentru mărturia lui Iisus.
\par 10 Am fost în duh în zi de duminică și am auzit, în urma mea, glas mare de trâmbiță,
\par 11 Care zicea: Ceea ce vezi scrie în carte și trimite celor șapte Biserici: la Efes, și la Smirna, și la Pergam, și la Tiatira, și la Sardes, și la Filadelfia, și la Laodiceea.
\par 12 Și m-am întors să văd al cui este glasul care vorbea cu mine și, întorcându-mă, am văzut șapte sfeșnice de aur.
\par 13 Și în mijlocul sfeșnicelor pe Cineva asemenea Fiului Omului, îmbrăcat în veșmânt lung până la picioare și încins pe sub sân cu un brâu de aur.
\par 14 Capul Lui și părul Lui erau albe ca lâna albă și ca zăpada, și ochii Lui, ca para focului.
\par 15 Picioarele Lui erau asemenea aramei arse în cuptor, iar glasul Lui era ca un vuiet de ape multe;
\par 16 În mâna Lui cea dreaptă avea șapte stele; și din gura Lui ieșea o sabie ascuțită cu două tăișuri, iar fața Lui era ca soarele, când strălucește în puterea lui.
\par 17 Și când L-am văzut, am căzut la picioarele Lui ca un mort. Și El a pus mâna dreaptă peste mine, zicând: Nu te teme! Eu sunt Cel dintâi și Cel de pe urmă,
\par 18 Și Cel ce sunt viu. Am fost mort, și, iată, sunt viu, în vecii vecilor, și am cheile morții și ale iadului.
\par 19 Scrie, deci, cele ce ai văzut și cele ce sunt și cele ce au să fie după acestea.
\par 20 Taina celor șapte stele, pe care le-ai văzut în dreapta Mea, și a celor șapte sfeșnice de aur este: Cele șapte stele sunt îngerii celor șapte Biserici, iar sfeșnicele cele șapte sunt șapte Biserici.

\chapter{2}

\par 1 Scrie îngerului Bisericii din Efes: Acestea zice Cel ce ține cele șapte stele în dreapta Sa, Cel care umblă în mijlocul celor șapte sfeșnice de aur:
\par 2 Știu faptele tale și osteneala ta și răbdarea ta și cum că nu poți suferi pe cei răi și ai cercat pe cei ce se zic pe sine apostoli și nu sunt și i-ai aflat mincinoși;
\par 3 Și stărui în răbdare și ai suferit pentru numele Meu și nu ai obosit.
\par 4 Dar am împotriva ta faptul că ai părăsit dragostea ta cea dintâi.
\par 5 Drept aceea, adu-ți aminte de unde ai căzut și te pocăiește și fă faptele de mai înainte; iar de nu, vin la tine curând și voi mișca sfeșnicul tău din locul lui, dacă nu te vei pocăi.
\par 6 Ai însă partea bună că urăști faptele nicolaiților, pe care le urăsc și Eu.
\par 7 Cine are urechi să audă ceea ce Duhul zice Bisericilor: Celui ce va birui îi voi da să mănânce din pomul vieții, care este în raiul lui Dumnezeu.
\par 8 Iar îngerului Bisericii din Smirna, scrie-i: Acestea zice Cel dintâi și Cel de pe urmă, Cel care a murit și a înviat:
\par 9 Știu necazul tău și sărăcia ta, dar ești bogat, și hula din partea celor ce zic despre ei înșiși că sunt iudei și nu sunt, ci sinagogă a satanei.
\par 10 Nu te teme de cele ce ai să pătimești. Că iată diavolul va să arunce dintre voi în temniță, ca să fiți ispitiți, și veți avea necaz zece zile. Fii credincios până la moarte și îți voi da cununa vieții.
\par 11 Cine are urechi să audă ceea ce Duhul zice Bisericilor: Cel ce biruiește nu va fi vătămat de moartea cea de-a doua.
\par 12 Iar îngerului Bisericii din Pergam scrie-i: Acestea zice Cel ce are sabia ascuțită de amândouă părțile:
\par 13 Știu unde sălășluiești: unde este scaunul satanei; și ții numele Meu și n-ai tăgăduit credința Mea, în zilele lui Antipa, martorul Meu cel credincios, care a fost ucis la voi, unde locuiește satana.
\par 14 Dar am împotriva ta câteva lucruri, că ai acolo pe unii care țin învățătura lui Balaam, cel ce învăța pe Balac să pună piatră de poticneală înaintea fiilor lui Israel, ca să mănânce care jertfită idolilor și să se dea desfrânării.
\par 15 Astfel ai și tu pe unii care, de asemenea, țin învățătura nicolaiților.
\par 16 Pocăiește-te deci, iar de nu, vin la tine curând și voi face cu ei război, cu sabia gurii Mele.
\par 17 Cine are urechi să audă ceea ce Duhul zice Bisericilor: Biruitorului îi voi da din mana cea ascunsă și-i voi da lui o pietricică albă și pe pietricică scris un nume nou, pe care nimeni nu-l știe, decât primitorul.
\par 18 Iar îngerului Bisericii din Tiatira scrie-i: Acestea zice Fiul lui Dumnezeu, ai Cărui ochi sunt ca para focului și picioarele asemenea aramei strălucitoare:
\par 19 Știu faptele tale și dragostea și credința și slujirea și răbdarea ta și știu că faptele tale cele de pe urmă sunt mai multe decât cele dintâi.
\par 20 Dar am împotriva ta faptul că lași pe femeia Izabela, care se zice pe sine proorociță, de învață și amăgește pe robii Mei, ca să facă desfrânări și să mănânce cele jertfite idolilor.
\par 21 Și i-am dat timp să se pocăiască și nu voiește să se pocăiască de desfrânarea ei.
\par 22 Iată, o arunc pe ea bolnavă la pat și pe cei ce se desfrânează cu ea, în mare strâmtorare, dacă nu se vor pocăi de faptele lor.
\par 23 Și pe fiii ei cu moarte îi voi ucide și vor cunoaște toate Bisericile că Eu sunt Cel care cercetez rănunchii și inimile și voi da vouă, fiecăruia, după faptele voastre.
\par 24 Iar vouă și celorlalți din Tiatira câți nu au învățătura aceasta, ca unii care n-au cunoscut adâncurile satanei, după cum spun ei, vă zic: nu pun peste voi altă greutate.
\par 25 Însă, ceea ce aveți, țineți până voi veni.
\par 26 Și celui ce biruiește și celui ce păzește până la capăt faptele Mele, îi voi da lui stăpânire peste neamuri.
\par 27 Și le va păstori pe ele cu toiag de fier și ca pe vasele olarului le va sfărâma, precum și eu am luat putere de la Tatăl Meu.
\par 28 Și-i voi da lui steaua cea de dimineață.
\par 29 Cine are urechi să audă ceea ce Duhul zice Bisericilor.

\chapter{3}

\par 1 Iar îngerului Bisericii din Sardes scrie-i: Acestea zice Cel ce are cele șapte duhuri ale lui Dumnezeu și cele șapte stele: Știu faptele tale, că ai nume, că trăiești, dar ești mort.
\par 2 Priveghează și întărește ce a mai rămas și era să moară. Căci n-am găsit faptele tale depline înaintea Dumnezeului Meu.
\par 3 Drept aceea, adu-ți aminte cum ai primit și ai auzit și păstrează și te pocăiește. Iar de nu vei priveghea, voi veni ca un fur și nu vei ști în care ceas voi veni asupra ta.
\par 4 Dar ai câțiva oameni în Sardes, care nu și-au mânjit hainele lor, ci ei vor umbla cu Mine îmbrăcați în veșminte albe, fiindcă sunt vrednici.
\par 5 Cel ce biruiește va fi astfel îmbrăcat în veșminte albe și nu voi șterge deloc numele lui din cartea vieții și voi mărturisi numele lui înaintea părintelui Meu și înaintea îngerilor Lui.
\par 6 Cel ce are urechi să audă ceea ce Duhul zice Bisericilor.
\par 7 Iar îngerului Bisericii din Filadelfia scrie-i: Acestea zice Cel Sfânt, Cel Adevărat, Cel ce are cheia lui David, Cel ce deschide și nimeni nu va închide și închide și nimeni nu va deschide:
\par 8 Știu faptele tale; iată, am lăsat înaintea ta o ușă deschisă, pe care nimeni nu poate să o închidă, fiindcă, deși ai putere mică, tu ai păzit cuvântul Meu și nu ai tăgăduit numele Meu.
\par 9 Iată, îți dau din sinagoga satanei, dintre cei care se zic pe sine că sunt iudei și nu sunt, ci mint; iată, îi voi face să vină și să se închine înaintea picioarelor tale și vor cunoaște că te-am iubit.
\par 10 Pentru că ai păzit cuvântul răbdării Mele, și Eu te voi păzi pe tine de ceasul ispitei ce va să vină peste toată lumea, ca să încerce pe cei ce locuiesc pe pământ.
\par 11 Vin curând; ține ce ai, ca nimeni să nu ia cununa ta.
\par 12 Pe cel ce biruiește îl voi face stâlp în templul Dumnezeului Meu și afară nu va mai ieși și voi scrie pe el numele Dumnezeului Meu și numele cetății Dumnezeului Meu, - al noului Ierusalim, care se pogoară din cer, de la Dumnezeul Meu - și numele Meu cel nou.
\par 13 Cine are urechi să audă ceea ce Duhul zice Bisericilor.
\par 14 Iar îngerului Bisericii din Laodiceea scrie-i: Acestea zice Cel ce este Amin, martorul cel credincios și adevărat, începutul zidirii lui Dumnezeu:
\par 15 Știu faptele tale; că nu ești nici rece, nici fierbinte. O, de ai fi rece sau fierbinte!
\par 16 Astfel, fiindcă ești căldicel - nici fierbinte, nici rece - am să te vărs din gura Mea.
\par 17 Fiindcă tu zici: Sunt bogat și m-am îmbogățit și de nimic nu am nevoie! Și nu știi că tu ești cel ticălos și vrednic de plâns, și sărac și orb și gol!
\par 18 Te sfătuiesc să cumperi de la Mine aur lămurit în foc, ca să te îmbogățești, și veșminte albe ca să te îmbraci și să nu se dea pe față rușinea goliciunii tale, și alifie de ochi ca să-ți ungi ochii și să vezi.
\par 19 Eu pe câți îi iubesc îi mustru și îi pedepsesc; sârguiește dar și te pocăiește.
\par 20 Iată, stau la ușă și bat; de va auzi cineva glasul Meu și va deschide ușa, voi intra la el și voi cina cu el și el cu Mine.
\par 21 Celui ce biruiește îi voi da să șadă cu Mine pe scaunul Meu, precum și Eu am biruit și am șezut cu Tatăl Meu pe scaunul Lui.
\par 22 Cine are urechi să audă ceea ce Duhul zice Bisericilor.

\chapter{4}

\par 1 După acestea, m-am uitat și iată o ușă era deschisă în cer și glasul cel dintâi - glasul ca de trâmbiță, pe care l-am auzit vorbind cu mine - mi-a zis: Suie-te aici și îți voi arăta cele ce trebuie să fie după acestea.
\par 2 Îndată am fost în duh; și iată un tron era în cer și pe tron ședea Cineva.
\par 3 Și Cel ce ședea semăna la vedere cu piatra de iasp și de sardiu, iar de jur împrejurul tronului era un curcubeu, cu înfățișarea smaraldului.
\par 4 Și douăzeci și patru de scaune înconjurau tronul și pe scaune douăzeci și patru de bătrâni, șezând, îmbrăcați în haine albe și purtând pe capetele lor cununi de aur.
\par 5 Și din tron ieșeau fulgere și glasuri și tunete; și șapte făclii de foc ardeau înaintea tronului, care sunt cele șapte duhuri ale lui Dumnezeu,
\par 6 Și înaintea tronului, ca o mare de sticlă, asemenea cu cristalul. Iar în mijlocul tronului și împrejurul tronului patru ființe, pline de ochi, dinainte și dinapoi.
\par 7 Și ființa cea dintâi era asemenea leului, a doua ființă asemenea vițelului, a treia ființă avea față de om, iar a patra ființă era asemenea vulturului care zboară.
\par 8 Și cele patru ființe, având fiecare din ele câte șase aripi, sunt pline de ochi, de jur împrejur și pe dinăuntru, și odihnă nu au, ziua și noaptea, zicând: Sfânt, Sfânt, Sfânt, Domnul Dumnezeu, Atotțiitorul, Cel ce era și Cel ce este și Cel ce vine.
\par 9 Și când cele patru ființe dădeau slavă, cinste și mulțumită Celui ce șade pe tron, Celui ce este viu în vecii vecilor,
\par 10 Atunci cei douăzeci și patru de bătrâni, căzând înaintea Celui ce ședea pe tron, se închinau Celui ce este viu în vecii vecilor și aruncau cununile lor înaintea tronului, zicând:
\par 11 Vrednic ești, Doamne și Dumnezeul nostru, să primești slava și cinstea și puterea, căci Tu ai zidit toate lucrurile și prin voința Ta ele erau și s-au făcut.

\chapter{5}

\par 1 Am văzut apoi, în mâna dreaptă a Celui ce ședea pe tron, o carte scrisă înăuntru și pe dos, pecetluită cu șapte peceți.
\par 2 Și am văzut un înger puternic, care striga cu glas mare: Cine este vrednic să deschidă cartea și să desfacă toate pecețile ei?
\par 3 Dar nimeni în cer, nici pe pământ, nici sub pământ nu putea să deschidă cartea, nici să se uite în ea.
\par 4 Și am plâns mult, fiindcă nimeni n-a fost găsit vrednic să deschidă cartea, nici să se uite în ea.
\par 5 Și unul dintre bătrâni mi-a zis: Nu plânge. Iată, a biruit leul din seminția lui Iuda, rădăcina lui David, ca să deschidă cartea și cele șapte peceți ale ei.
\par 6 Și am văzut, la mijloc, între tron și cele patru ființe și în mijlocul bătrânilor, stând un Miel, ca înjunghiat, și care avea șapte coarne și șapte ochi, care sunt cele șapte duhuri ale lui Dumnezeu, trimise în tot pământul.
\par 7 Și a venit și a luat cartea, din dreapta Celui ce ședea pe tron.
\par 8 Și când a luat cartea, cele patru ființe și cei douăzeci și patru de bătrâni au căzut înaintea Mielului, având fiecare alăută și cupe de aur pline cu tămâie care sunt rugăciunile sfinților.
\par 9 Și cântau o cântare nouă, zicând: Vrednic ești să iei cartea și să deschizi pecețile ei, fiindcă ai fost înjunghiat și ai răscumpărat lui Dumnezeu, cu sângele Tău, oameni din toată seminția și limba și poporul și neamul;
\par 10 Și I-ai făcut Dumnezeului nostru împărăție și preoți, și vor împărăți pe pământ.
\par 11 Și am văzut și am auzit glas de îngeri mulți, de jur împrejurul tronului și al ființelor și al bătrânilor, și era numărul lor zeci de mii de zeci de mii și mii de mii,
\par 12 Zicând cu glas mare: Vrednic este Mielul cel înjunghiat ca să ia puterea și bogăția și înțelepciunea și tăria și cinstea și slava și binecuvântarea.
\par 13 Și toată făptura care este în cer și pe pământ și sub pământ și în mare și toate câte sunt în acestea le-am auzit, zicând: Celui ce șade pe tron și Mielului fie binecuvântarea și cinstea și slava și puterea, în vecii vecilor!
\par 14 Și cele patru ființe ziceau: Amin! Iar bătrânii căzură și se închinară.

\chapter{6}

\par 1 Și am văzut când Mielul a deschis pe cea dintâi din cele șapte peceți, și am auzit pe una din cele patru ființe zicând cu glas ca de tunet: Vino și vezi.
\par 2 Și m-am uitat și iată un cal alb și cel care ședea pe el avea un arc; și i s-a dat lui cunună și a pornit ca un biruitor ca să biruiască.
\par 3 Și când a deschis pecetea a doua, am auzit, zicând, pe a doua ființă: Vino și vezi.
\par 4 Și a ieșit alt cal, roșu ca focul; și celui ce ședea pe el i s-a dat să ia pacea de pe pământ, ca oamenii să se junghie între ei; și o sabie mare i s-a dat.
\par 5 Și când a deschis pecetea a treia, am auzit pe a treia ființă, zicând: Vino și vezi. Și m-am uitat și iată un cal negru și cel care ședea pe el avea un cântar în mâna lui.
\par 6 Și am auzit, în mijlocul celor patru ființe, ca un glas care zicea: Măsura de grâu un dinar, și trei măsuri de orz un dinar. Dar de untdelemn și de vin să nu te atingi.
\par 7 Și când a deschis pecetea a patra, am auzit glasul ființei a patra, zicând: Vino și vezi.
\par 8 Și m-am uitat și iată un cal galben-vânăt și numele celui ce ședea pe el era: Moartea; și iadul se ținea după el; și li s-a dat lor putere peste a patra parte a pământului, ca să ucidă cu sabie și cu foamete, și cu moarte și cu fiarele de pe pământ.
\par 9 Și când a deschis pecetea a cincea, am văzut, sub jertfelnic, sufletele celor înjunghiați pentru cuvântul lui Dumnezeu și pentru mărturia pe care au dat-o.
\par 10 Și strigau cu glas mare și ziceau: Până când, Stăpâne sfinte și adevărate, nu vei judeca și nu vei răzbuna sângele nostru, față de cei ce locuiesc pe pământ?
\par 11 Și fiecăruia dintre ei i s-a dat câte un veșmânt alb și li s-a spus ca să stea în tihnă, încă puțină vreme, până când vor împlini numărul și cei împreună-slujitori cu ei și frații lor, cei ce aveau să fie omorâți ca și ei.
\par 12 Și m-am uitat când a deschis pecetea a șasea și s-a făcut cutremur mare, soarele s-a făcut negru ca un sac de păr și luna întreagă s-a făcut ca sângele,
\par 13 Și stelele cerului au căzut pe pământ, precum smochinul își leapădă smochinele sale verzi, când este zguduit de vijelie.
\par 14 Iar cerul s-a dat în lături, ca o carte de piele pe care o faci sul și toți munții și toate insulele s-au mișcat din locurile lor.
\par 15 Și împărații pământului și domnii și căpeteniile oștilor și bogații și cei puternici și toți robii și toți slobozii s-au ascuns în peșteri și în stâncile munților,
\par 16 Strigând munților și stâncilor: Cădeți peste noi și ne ascundeți pe noi de fața Celui ce șade pe tron și de mânia Mielului;
\par 17 Că a venit ziua cea mare a mâniei lor, și cine are putere ca să stea pe loc?

\chapter{7}

\par 1 După aceasta am văzut patru îngeri, stând la cele patru unghiuri ale pământului, ținând cele patru vânturi ale pământului, ca să nu sufle vânt pe pământ, nici peste mare, nici peste vreun copac.
\par 2 Și am văzut un alt înger care se ridica de la Răsăritul Soarelui și avea pecetea Viului Dumnezeu. Îngerul a strigat cu glas puternic către cei patru îngeri, cărora li s-a dat să vatăme pământul și marea,
\par 3 Zicând: Nu vătămați pământul, nici marea, nici copacii, până ce nu vom pecetlui, pe frunțile lor, pe robii Dumnezeului nostru.
\par 4 Și am auzit numărul celor pecetluiți: o sută patruzeci și patru de mii de pecetluiți, din toate semințiile fiilor lui Israel:
\par 5 Din seminția lui Iuda, douăsprezece mii de pecetluiți; din seminția lui Ruben, douăsprezece mii; din seminția lui Gad, douăsprezece mii;
\par 6 Din seminția lui Așer, douăsprezece mii; din seminția lui Neftali, douăsprezece mii; din seminția lui Manase, douăsprezece mii;
\par 7 Din seminția lui Simeon, douăsprezece mii; din seminția lui Levi, douăsprezece mii; din seminția lui Isahar, douăsprezece mii;
\par 8 Din seminția lui Zabulon, douăsprezece mii; din seminția lui Iosif, douăsprezece mii; din seminția Veniamin, douăsprezece mii de pecetluiți.
\par 9 După acestea, m-am uitat și iată mulțime multă, pe care nimeni nu putea s-o numere, din tot neamul și semințiile și popoarele și limbile, stând înaintea tronului și înaintea Mielului, îmbrăcați în veșminte albe și având în mână ramuri de finic.
\par 10 Și mulțimea striga cu glas mare, zicând: Mântuirea este de la Dumnezeul nostru, Care șade pe tron, și de la Mielul.
\par 11 Și toți îngerii stăteau împrejurul tronului bătrânilor și al celor patru ființe, și au căzut înaintea tronului pe fețele lor și s-au închinat lui Dumnezeu,
\par 12 Zicând: Amin! Binecuvântarea și slava și înțelepciunea și mulțumirea și cinstea și puterea și tăria fie Dumnezeului nostru, în vecii vecilor. Amin!
\par 13 Iar unul dintre bătrâni a deschis gura și mi-a zis: Aceștia care sunt îmbrăcați în veșminte albe, cine sunt și de unde au venit?
\par 14 Și i-am zis: Doamne, Tu știi. El mi-a răspuns: Aceștia sunt cei ce vin din strâmtorarea cea mare și și-au spălat veșmintele lor și le-au făcut albe în sângele Mielului.
\par 15 Pentru aceea sunt înaintea tronului lui Dumnezeu, și Îi slujesc ziua și noaptea, în templul Lui, și Cel ce șade pe tron îi va adăposti în cortul Său.
\par 16 Și nu vor mai flămânzi, nici nu vor mai înseta, nici nu va mai cădea soarele peste ei și nici o arșiță;
\par 17 Căci Mielul, Cel ce stă în mijlocul tronului, îi va paște pe ei și-i va duce la izvoarele apelor vieții și Dumnezeu va șterge orice lacrimă din ochii lor.

\chapter{8}

\par 1 Și când Mielul a deschis pecetea a șaptea, s-a făcut tăcere în cer, ca la o jumătate de ceas.
\par 2 Și am văzut pe cei șapte îngeri, care stau înaintea lui Dumnezeu și li s-a dat lor șapte trâmbițe.
\par 3 Și a venit un alt înger și a stat la altar, având cădelniță de aur, și i s-a dat lui tămâie multă, ca s-o aducă împreună cu rugăciunile tuturor sfinților, pe altarul de aur dinaintea tronului.
\par 4 Și fumul tămâiei s-a suit, din mâna îngerului, înaintea lui Dumnezeu, împreună cu rugăciunile sfinților.
\par 5 Și îngerul a luat cădelnița și a umplut-o din focul altarului și a aruncat pe pământ; și s-au pornit tunete și glasuri și fulgere și cutremur.
\par 6 Iar cei șapte îngeri, care aveau cele șapte trâmbițe, s-au gătit ca să trâmbițeze.
\par 7 Și a trâmbițat întâiul înger, și s-a pornit grindină și foc amestecat cu sânge și au căzut pe pământ; și a ars din pământ a treia parte, și a ars din copaci a treia parte, iar iarba verde a ars de tot.
\par 8 A trâmbițat, apoi, al doilea înger, și ca un munte mare arzând în flăcări s-a prăbușit în mare și a treia parte din mare s-a prefăcut în sânge;
\par 9 Și a pierit a treia parte din făpturile cu viață în ele, care sunt în mare, și a treia parte din corăbii s-a sfărâmat.
\par 10 Și a trâmbițat al treilea înger, și a căzut din cer o stea uriașă, arzând ca o făclie, și a căzut peste izvoarele apelor.
\par 11 Și numele stelei se cheamă Absintos. Și a treia parte din ape s-a făcut ca pelinul și mulți dintre oameni au murit din pricina apelor, pentru că se făcuseră amare.
\par 12 Și a trâmbițat al patrulea înger; și a fost lovită a treia parte din soare, și a treia parte din lună, și a treia parte din stele, ca să fie întunecată a treia parte a lor și ziua să-și piardă din lumină a treia parte, și noaptea tot așa.
\par 13 Și am văzut și am auzit un vultur, care zbura spre înaltul cerului și striga cu glas mare: Vai, vai, vai celor ce locuiesc pe pământ, din pricina celorlalte glasuri ale trâmbiței celor trei îngeri, care sunt gata să trâmbițeze!

\chapter{9}

\par 1 Și a trâmbițat al cincilea înger, și am văzut o stea căzută din cer pe pământ și i s-a dat cheia fântânii adâncului.
\par 2 Și a deschis fântâna adâncului și fum s-a ridicat din fântână, ca fumul unui cuptor mare, și soarele și văzduhul s-au întunecat de fumul fântânii.
\par 3 Și din fum au ieșit lăcuste pe pământ și li s-a dat lor putere precum au putere scorpiile pământului.
\par 4 Și li s-a poruncit să nu vatăme iarba pământului și nici o verdeață și nici un copac, fără numai pe oamenii care nu au pecetea lui Dumnezeu pe frunțile lor.
\par 5 Și nu li s-a dat ca să-i omoare, ci ca să fie chinuiți cinci luni; și chinul lor este la fel cu chinul scorpiei, când a înțepat pe om.
\par 6 Și în zilele acelea vor căuta oamenii moartea și nu o vor afla și vor dori să moară; moartea însă va fugi de ei.
\par 7 Iar înfățișarea lăcustelor era asemenea unor cai pregătiți de război. Pe capete aveau cununi ca de aur, și fețele lor erau ca niște fețe de oameni.
\par 8 Și aveau păr ca părul de femei și dinții lor erau ca dinții leilor.
\par 9 Și aveau platoșe ca platoșele de fier, iar vuietul aripilor era la fel cu vuietul unei mulțimi de care și de cai, care aleargă la luptă.
\par 10 Și aveau cozi și bolduri asemenea scorpiilor; și puterea lor e în cozile lor, ca să vatăme pe oameni cinci luni.
\par 11 Și au ca împărat al lor pe îngerul adâncului, al cărui nume, în evreiește, este Abaddon, iar în elinește are numele Apollion.
\par 12 Întâiul "vai" a trecut; iată vine încă un "vai" și încă unul, după acestea.
\par 13 Și a trâmbițat al șaselea înger. Și am auzit un glas, din cele patru cornuri ale altarului de aur, care este înaintea lui Dumnezeu,
\par 14 Zicând către îngerul al șaselea, cel ce avea trâmbița: Dezleagă pe cei patru îngeri care sunt legați la râul cel mare, Eufratul.
\par 15 Și au fost dezlegați cei patru îngeri, care erau gătiți spre ceasul și ziua și luna și anul acela, ca să omoare a treia parte din oameni.
\par 16 Și numărul oștilor era de douăzeci de mii de ori câte zece mii de călăreți, căci am auzit numărul lor.
\par 17 Și așa am văzut, în vedenie, caii și pe cei ce ședeau pe ei, având platoșe ca de foc și de iachint și de pucioasă; iar capetele cailor semănau cu capetele leilor și din gurile lor ieșea foc și fum și pucioasă.
\par 18 De aceste trei plăgi: de focul și de fumul și de pucioasa, care ieșea din gurile lor, a fost ucisă a treia parte din oameni.
\par 19 Pentru că puterea cailor este în gura lor și în cozile lor; căci cozile lor sunt asemenea șerpilor, având capete, și cu acestea vatămă.
\par 20 Dar ceilalți oameni care nu au murit de plăgile acestea, nu s-au pocăit de faptele mâinilor lor, ca să nu se mai închine idolilor de aur și de argint și de aramă și de piatră și de lemn, care nu pot nici să vadă, nici să audă, nici să umble.
\par 21 Și nu s-au pocăit de uciderile lor, nici de fermecătoriile lor, nici de desfrânarea lor, nici de furtișagurile lor.

\chapter{10}

\par 1 Și am văzut alt înger puternic, pogorându-se din cer, învăluit într-un nor și pe capul lui era curcubeul, iar fața lui strălucea ca soarele și picioarele lui erau ca niște stâlpi de foc,
\par 2 Și în mână avea o carte mică, deschisă. Și a pus piciorul lui cel drept pe mare, iar pe cel stâng pe pământ,
\par 3 Și a strigat cu glas puternic, precum răcnește leul. Iar când a strigat, cele șapte tunete au slobozit glasurile lor.
\par 4 Și când au vorbit cele șapte tunete, voiam să scriu, dar am auzit o voce care zicea din cer: Pecetluiește cele ce au spus cele șapte tunete și nu le scrie.
\par 5 Iar îngerul pe care l-am văzut stând pe mare și pe pământ, și-a ridicat mâna dreaptă către cer,
\par 6 Și s-a jurat pe Cel ce este viu în vecii vecilor, Care a făcut cerul și cele ce sunt în cer și pământul și cele ce sunt pe pământ și marea și cele ce sunt în mare, că timp nu va mai fi,
\par 7 Ci, în zilele când va grăi al șaptelea înger - când va fi să trâmbițeze - atunci va fi săvârșită taina lui Dumnezeu, precum bine a vestit robilor Săi, proorocilor.
\par 8 Iar glasul din cer, pe care-l auzisem, iarăși a vorbit cu mine, zicând: Mergi de ia cartea cea deschisă din mâna îngerului, care stă pe mare și pe pământ.
\par 9 Și m-am dus la înger și i-am zis să-mi dea cartea. Și mi-a răspuns: Ia-o și mănânc-o și va amărî pântecele tău, dar în gura ta va fi dulce ca mierea.
\par 10 Atunci am luat cartea din mâna îngerului și am mâncat-o; și era în gura mea dulce ca mierea, dar, după ce-am mâncat-o pântecele meu s-a amărât.
\par 11 Și apoi mi-a zis: Tu trebuie să proorocești, încă o dată, la popoare și la neamuri și la limbi și la mulți împărați.

\chapter{11}

\par 1 Apoi mi-au dat o trestie, asemenea unui toiag, zicând: Scoală-te și măsoară templul lui Dumnezeu și altarul și pe cei ce se închină în el.
\par 2 Iar curtea cea din afară a templului, scoate-o din socoteală și n-o măsura, pentru că a fost dată neamurilor, care vor călca în picioare cetatea sfântă patruzeci și două de luni.
\par 3 Și voi da putere celor doi martori ai mei și vor prooroci, îmbrăcați în sac, o mie două sute și șaizeci de zile.
\par 4 Aceștia sunt cei doi măslini și cele două sfeșnice care stau înaintea Domnului pământului.
\par 5 Și dacă voiește cineva să-i vatăme, foc iese din gura lor și mistuiește pe vrăjmașii lor; și dacă ar voi cineva să-i vatăme, acela trebuie ucis.
\par 6 Aceștia au putere să închidă cerul, ca ploaia să nu plouă în zilele proorociei lor, și putere au peste ape să le schimbe în sânge și să bată pământul cu orice fel de urgie, ori de câte ori vor voi.
\par 7 Iar când vor isprăvi cu mărturia lor, fiara care se ridică din adânc va face război cu ei, și-i va birui și-i va omorî.
\par 8 Și trupurile lor vor zăcea pe ulițele cetății celei mari, care se cheamă, duhovnicește, Sodoma și Egipt, unde a fost răstignit și Domnul lor.
\par 9 Și din popoare, din seminții, din limbi și din neamuri vor privi la trupurile lor trei zile și jumătate și nu vor îngădui ca ele să fie puse în mormânt.
\par 10 Iar locuitorii de pe pământ se vor bucura de moartea lor și vor fi în veselie și își vor trimite daruri unul altuia, pentru că acești doi prooroci au chinuit pe locuitorii de pe pământ.
\par 11 Și după cele trei zile și jumătate, duh de viață de la Dumnezeu a intrat în ei și s-au ridicat pe picioarele lor și frică mare a căzut peste cei ce se uitau la ei.
\par 12 Și din cer au auzit glas puternic, zicându-le: Suiți-vă aici! Și s-au suit la cer, în nori, și au privit la ei dușmanii lor.
\par 13 Și în ceasul acela s-a făcut cutremur mare și a zecea parte din cetate s-a prăbușit și au pierit în cutremur șapte mii de oameni, iar ceilalți s-au înfricoșat și au dat slavă Dumnezeului cerului.
\par 14 Al doilea "vai" a trecut; al treilea "vai", iată, vine degrabă.
\par 15 Și a trâmbițat al șaptelea înger și s-au pornit, în cer, glasuri puternice care ziceau: Împărăția lumii a ajuns a Domnului nostru și a Hristosului Său și va împărăți în vecii vecilor.
\par 16 Și cei douăzeci și patru de bătrâni, care șed înaintea lui Dumnezeu pe scaunele lor, au căzut cu fețele la pământ și s-au închinat lui Dumnezeu,
\par 17 Zicând: Mulțumim Ție, Doamne Dumnezeule, Atotțiitorule, Cel ce ești și Cel ce erai și Cel ce vii, că ai luat puterea Ta cea mare și împărățești.
\par 18 Și neamurile s-au mâniat, dar a venit mânia Ta și vremea celor morți, ca să fie judecați, și să răsplătești pe robii Tăi, pe prooroci și pe sfinți și pe cei ce se tem de numele Tău, pe cei mici și pe cei mari, și să pierzi pe cei ce prăpădesc pământul.
\par 19 Și s-a deschis templul lui Dumnezeu, cel din cer, și s-a văzut în templul Lui chivotul legământului Său, și au fost fulgere și vuiete și tunete și cutremur și grindină mare.

\chapter{12}

\par 1 Și s-a arătat din cer un semn mare: o femeie înveșmântată cu soarele și luna era sub picioarele ei și pe cap purta cunună din douăsprezece stele.
\par 2 Și era însărcinată și striga, chinuindu-se și muncindu-se ca să nască.
\par 3 Și alt semn s-a arătat în cer: iată un balaur mare, roșu, având șapte capete și zece coarne, și pe capetele lui, șapte cununi împărătești.
\par 4 Iar coada lui târa a treia parte din stelele cerului și le-a aruncat pe pământ. Și balaurul stătu înaintea femeii, care era să nască, pentru ca să înghită copilul, când se va naște.
\par 5 Și a născut un copil de parte bărbătească, care avea să păstorească toate neamurile cu toiag de fier. Și copilul ei fu răpit la Dumnezeu și la tronul Lui,
\par 6 Iar femeia a fugit în pustie, unde are loc gătit de Dumnezeu, ca să o hrănească pe ea, acolo, o mie două sute și șaizeci de zile.
\par 7 Și s-a făcut război în cer: Mihail și îngerii lui au pornit război cu balaurul. Și se războia și balaurul și îngerii lui.
\par 8 Și n-a izbutit el, nici nu s-a mai găsit pentru ei loc în cer.
\par 9 Și a fost aruncat balaurul cel mare, șarpele de demult, care se cheamă diavol și satana, cel ce înșeală pe toată lumea, aruncat a fost pe pământ și îngerii lui au fost aruncați cu el.
\par 10 Și am auzit glas mare, în cer, zicând: Acum s-a făcut mântuirea și puterea și împărăția Dumnezeului nostru și stăpânirea Hristosului Său, căci aruncat a fost pârâșul fraților noștri, cel ce îi pâra pe ei înaintea Dumnezeului nostru, ziua și noaptea.
\par 11 Și ei l-au biruit prin sângele Mielului și prin cuvântul mărturiei lor și nu și-au iubit sufletul lor, până la moarte.
\par 12 Pentru aceasta, bucurați-vă ceruri și cei ce locuiți în ele. Vai vouă, pământule și mare, fiindcă diavolul a coborât la voi având mânie mare, căci știe că timpul lui e scurt.
\par 13 Iar când a văzut balaurul că a fost aruncat pe pământ, a prigonit pe femeia care născuse pruncul.
\par 14 Și femeii i s-au dat cele două aripi ale marelui vultur, ca să zboare în pustie, la locul ei, unde e hrănită acolo o vreme și vremuri și jumătate de vreme, departe de fața șarpelui.
\par 15 Și șarpele a aruncat din gura lui, după femeie, apă ca un râu ca s-o ia apa.
\par 16 Și pământul i-a venit femeii într-ajutor, căci pământul și-a deschis gura sa și a înghițit râul pe care-l aruncase balaurul, din gură.
\par 17 Și balaurul s-a aprins de mânie asupra femeii și a pornit să facă război cu ceilalți din seminția ei, care păzesc poruncile lui Dumnezeu și țin mărturia lui Iisus.

\chapter{13}

\par 1 Și a stat pe nisipul mării. Și am văzut ridicându-se din mare o fiară, care avea zece coarne și șapte capete și pe coarnele ei zece cununi împărătești și pe capetele ei: nume de hulă.
\par 2 Și fiara pe care am văzut-o era asemenea leopardului, picioarele ei erau ca ale ursului, iar gura ei ca o gură de leu. Și balaurul i-a dat ei puterea lui și scaunul lui și stăpânire mare.
\par 3 Și unul din capetele fiarei era ca înjunghiat de moarte, dar rana ei cea de moarte fu vindecată și tot pământul s-a minunat mergând după fiară.
\par 4 Și s-au închinat balaurului, fiindcă i-a dat fiarei stăpânirea; și s-au închinat fiarei, zicând: Cine este asemenea fiarei și cine poate să se lupte cu ea?
\par 5 Și i s-a dat ei gură să grăiască semeții și hule și i s-a dat putere să lucreze timp de patruzeci și două de luni.
\par 6 Și și-a deschis gura sa spre hula lui Dumnezeu, ca să hulească numele Lui și cortul Lui și pe cei ce locuiesc în cer.
\par 7 Și i s-a dat să facă război cu sfinții și să-i biruiască și i s-a dat ei stăpânire peste toată seminția și poporul și limba și neamul.
\par 8 Și i se vor închina ei toți cei ce locuiesc pe pământ, ale căror nume nu sunt scrise, de la întemeierea lumii, în cartea vieții Mielului celui înjunghiat.
\par 9 Dacă are cineva urechi - să audă!
\par 10 Cine duce în robie de robie are parte; cine cu sabia va ucide trebuie să fie ucis de sabie. Aici este răbdarea și credința sfinților.
\par 11 Și am văzut o altă fiară, ridicându-se din pământ, și avea două coarne asemenea mielului, dar grăia ca un balaur
\par 12 Și toată stăpânirea celei dintâi fiare ea o pune în lucrare, în fața ei. Și face pământul și pe locuitorii de pe el să se închine fiarei celei dintâi, a cărei rană de moarte fusese vindecată.
\par 13 Și face semne mari, încât și foc face să se pogoare din cer, pe pământ, înaintea oamenilor,
\par 14 Și amăgește pe cei ce locuiesc pe pământ prin semnele ce i s-au dat să facă înaintea fiarei, zicând celor ce locuiesc pe pământ să facă un chip fiarei care a fost rănită cu sabia și a rămas în viață.
\par 15 Și i s-a dat ei să insufle duh chipului fiarei, ca chipul fiarei să și grăiască și să omoare pe toți câți nu se vor închina chipului fiarei.
\par 16 Și ea îi silește pe toți, pe cei mici și pe cei mari, și pe cei bogați și pe cei săraci, și pe cei slobozi și pe cei robi, ca să-și pună semn pe mâna lor cea dreaptă sau pe frunte.
\par 17 Încât nimeni să nu poată cumpăra sau vinde, decât numai cel ce are semnul, adică numele fiarei, sau numărul numelui fiarei.
\par 18 Aici este înțelepciunea. Cine are pricepere să socotească numărul fiarei; căci este număr de om. Și numărul ei este șase sute șaizeci și șase.

\chapter{14}

\par 1 Și m-am uitat și iată Mielul stătea pe muntele Sion și cu El o sută patruzeci și patru de mii, care aveau numele Lui și numele Tatălui Lui, scris pe frunțile lor.
\par 2 Atunci am auzit un glas din cer, ca un vuiet de ape multe și ca bubuitul unui tunet puternic, iar glasul pe care l-am auzit ca glasul celor ce cântă cu alăutele lor.
\par 3 Și cântau o cântare nouă, înaintea tronului și înaintea celor patru ființe și înaintea bătrânilor; și nimeni nu putea să învețe cântarea decât numai cei o sută patruzeci și patru de mii, care fuseseră răscumpărați de pe pământ.
\par 4 Aceștia sunt care nu s-au întinat cu femei, căci sunt feciorelnici. Aceștia sunt care merg după Miel ori unde se va duce. Aceștia au fost răscumpărați dintre oameni, pârgă lui Dumnezeu și Mielului.
\par 5 Iar în gura lor nu s-a aflat minciună, fiindcă sunt fără prihană.
\par 6 Și am văzut apoi alt înger, care zbura prin mijlocul cerului, având să binevestească Evanghelia veșnică celor ce locuiesc pe pământ și la tot neamul și seminția și limba și poporul,
\par 7 Zicând cu glas puternic: Temeți-vă de Dumnezeu și dați Lui slavă, că a venit ceasul judecății Lui, și vă închinați Celui ce a făcut cerul și pământul și marea și izvoarele apelor.
\par 8 Și un al doilea înger a venit, zicând: A căzut, a căzut Babilonul, cetatea cea mare, care a adăpat toate neamurile din vinul furiei desfrânării sale.
\par 9 Și al treilea înger a venit după ei, strigând cu glas puternic: Cine se închină fiarei și chipului ei și primește semnul ei pe fruntea lui, sau pe mâna lui,
\par 10 Va bea și el din vinul aprinderii lui Dumnezeu, turnat neamestecat, în potirul mâniei Sale, și se va chinui în foc și în pucioasă, înaintea sfinților îngeri și înaintea Mielului.
\par 11 Și fumul chinului lor se siue în vecii vecilor. Și nu au odihnă nici ziua nici noaptea cei ce se închină fiarei și chipului ei și oricine primește semnul numelui ei.
\par 12 Aici este răbdarea sfinților, care păzesc poruncile lui Dumnezeu și credința lui Iisus.
\par 13 Și am auzit un glas din cer, zicând: Scrie: Fericiți cei morți, cei ce acum mor întru Domnul! Da, grăiește Duhul, odihnească-se de ostenelile lor, căci faptele lor vin cu ei,
\par 14 Și am privit și iată un nor alb și Cel ce ședea pe nor era asemenea Fiului Omului, având pe cap cunună de aur și în mână seceră ascuțită.
\par 15 Și iată un alt înger a ieșit din templu, strigând cu glas mare Celui ce ședea pe nor: Trimite secera și seceră, că a venit ceasul de secerat, fiindcă s-a copt secerișul pământului.
\par 16 Și Cel ce ședea pe nor a aruncat pe pământ secera lui și pământul a fost secerat.
\par 17 Și un alt înger a ieșit din templul cel ceresc, având și el un cuțitaș ascuțit.
\par 18 Și încă un înger a ieșit din altar, având putere asupra focului, și a strigat cu glas mare celui care avea cuțitașul ascuțit, zicând: Trimite cuțitașul tău cel ascuțit și culege ciorchinii viei pământului, căci s-au copt.
\par 19 Și îngerul a aruncat, pe pământ, cuțitașul lui și a cules via pământului și strugurii i-a aruncat în teascul cel mare al mâniei lui Dumnezeu.
\par 20 Și teascul a fost călcat afară din cetate și a ieșit sânge din teasc, până la zăbalele cailor, pe o întindere de o mie șase sute de stadii.

\chapter{15}

\par 1 Am văzut, apoi, în cer, alt semn, mare și minunat: șapte îngeri având șapte pedepse - cele de pe urmă - căci cu ele s-a sfârșit mânia lui Dumnezeu.
\par 2 Și am văzut ca o mare de cristal, amestecată cu foc, și pe biruitorii fiarei și ai chipului ei și ai numărului numelui ei, stând în picioare pe marea de cristal și având alăutele lui Dumnezeu.
\par 3 Și ei cântau cântarea lui Moise, robul lui Dumnezeu, și cântarea Mielului, zicând: Mari și minunate sunt lucrurile Tale, Doamne Dumnezeule, Atotțiitorule! Drepte și adevărate sunt căile Tale, Împărate al neamurilor!
\par 4 Cine nu se va teme de Tine, Doamne, și nu va slăvi numele Tău? Că tu singur ești sfânt și toate neamurile vor veni și se vor închina înaintea Ta, pentru că judecățile Tale s-au făcut cunoscute.
\par 5 Și după aceasta, m-am uitat și s-a deschis templul cortului mărturiei din cer.
\par 6 Și au ieșit din templu cei șapte îngeri cu cele șapte pedepse, îmbrăcați în veșmânt de in curat, luminos, și încinși, pe la piept, cu cingători de aur.
\par 7 Și una din cele patru făpturi dădu celor șapte îngeri cele șapte cupe de aur pline de mânia lui Dumnezeu, Cel ce este viu în vecii vecilor.
\par 8 Iar templul se umplu de fum, din slava lui Dumnezeu și din puterea Lui, și nimeni nu putea să intre în templu, până ce se vor sfârși cele șapte urgii ale celor șapte îngeri.

\chapter{16}

\par 1 Și am auzit glas mare, din templu, zicând celor șapte îngeri: Duceți-vă și vărsați pe pământ cele șapte cupe ale mâniei lui Dumnezeu.
\par 2 Și s-a dus cel dintâi și a vărsat cupa lui pe pământ. Și o bubă rea și ucigătoare s-a ivit pe oamenii care aveau semnul fiarei și care se închinau chipului fiarei.
\par 3 Și al doilea înger a vărsat cupa lui în mare, și marea s-a prefăcut în sânge ca de mort, și orice suflare de viață a murit, din cele ce sunt în mare.
\par 4 Iar cel de al treilea a vărsat cupa lui în râuri și în izvoarele apelor și s-au prefăcut în sânge.
\par 5 Și am auzit pe îngerul apelor, zicând: Drept ești Tu, Cel ce ești și Cel ce erai, Cel Sfânt, că ai judecat acestea:
\par 6 Fiindcă au vărsat sângele sfinților și al proorocilor, tot sânge le-ai dat să bea. Vrednici sunt!
\par 7 Și am auzit din altar, grăind: Da, Doamne Dumnezeule, Atotțiitorule, adevărate și drepte sunt judecățile Tale!
\par 8 Și al patrulea înger a vărsat cupa lui în soare și i s-a dat să dogorească pe oameni cu focul lui.
\par 9 Și oamenii au fost dogoriți cu mare arșiță și au hulit numele lui Dumnezeu, Care are putere peste urgiile acestea, și nu s-au pocăit ca să-I dea slavă.
\par 10 Și al cincilea înger a vărsat cupa lui pe scaunul fiarei și în împărăția ei s-a făcut întuneric și oamenii își mușcau limbile de durere.
\par 11 Și au hulit pe Dumnezeul cerului din pricina durerilor și a bubelor lor, dar de faptele lor nu s-au pocăit.
\par 12 Și al șaselea înger a vărsat cupa lui în râul cel mare Eufrat și apele lui au secat, ca să fie gătită calea împăraților de la Răsăritul Soarelui.
\par 13 Și am văzut ieșind din gura balaurului și din gura fiarei și din gura proorocului celui mincinos trei duhuri necurate ca niște broaște.
\par 14 Căci sunt duhuri diavolești, făcătoare de semne și care se duc la împărații lumii întregi, să-i adune la războiul zilei celei mari a lui Dumnezeu, Atotțiitorul.
\par 15 Iată, vin ca un fur. Fericit este cel ce priveghează și păstrează veșmintele sale, ca să nu umble gol și să se vadă rușinea lui!
\par 16 Și i-au strâns la locul ce se cheamă evreiește Harmaghedon.
\par 17 Și al șaptelea înger a vărsat cupa lui în văzduh și glas mare a ieșit din templul cerului, de la tron, strigând: S-a făcut!
\par 18 Și s-au pornit fulgere și vuiete și tunete și s-a făcut cutremur mare, așa cum nu a fost, de când este omul pe pământ, un cutremur atât de puternic.
\par 19 Și cetatea cea mare s-a rupt în trei părți și cetățile neamurilor s-au prăbușit, și Babilonul cel mare a fost pomenit înaintea lui Dumnezeu, ca să-i dea paharul vinului aprinderii mâniei Lui.
\par 20 Și toate insulele pieriră și munții nu se mai aflară.
\par 21 Și grindină mare, cât talantul, se prăvăli din cer peste oameni. Și oamenii huliră pe Dumnezeu, din pricina pedepsei cu grindină, căci urgia ei era foarte mare.

\chapter{17}

\par 1 Și a venit unul din cei șapte îngeri, care aveau cele șapte cupe, și a grăit către mine, zicând: Vino să-ți arăt judecata desfrânatei celei mari, care șade pe ape multe,
\par 2 Cu care s-au desfrânat împărații pământului și cei ce locuiesc pe pământ s-au îmbătat de vinul desfrânării ei.
\par 3 Și m-a dus, în duh, în pustie. Și am văzut o femeie șezând pe o fiară roșie, plină de nume de hulă, având șapte capete și zece coarne.
\par 4 Și femeia era îmbrăcată în purpură și în stofă stacojie și împodobită cu aur și cu pietre scumpe și cu mărgăritare, având în mână un pahar de aur, plin de urâciunile și de necurățiile desfrânării ei.
\par 5 Iar pe fruntea ei scris nume tainic: Babilonul cel mare, mama desfrânatelor și a urâciunilor pământului.
\par 6 Și am văzut o femeie, beată de sângele sfinților și de sângele mucenicilor lui Iisus, și văzând-o, m-am mirat cu mirare mare.
\par 7 Și îngerul mi-a zis: De ce te miri? Eu îți voi spune taina femeii și a fiarei care o poartă și care are cele șapte capete și cele zece coarne.
\par 8 Fiara pe care ai văzut-o era și nu este și va să se ridice din adânc și să meargă spre pieire. Și se vor mira cei ce locuiesc pe pământ ale căror nume nu sunt scrise de la întemeierea lumii în cartea vieții, văzând pe fiară că era și nu este, dar se va arăta.
\par 9 Aici trebuie minte care are înțelepciune. Cele șapte capete sunt șapte munți deasupra cărora șade femeia.
\par 10 Dar sunt și șapte împărați: cinci au căzut, unul mai este, celălalt încă nu a venit, iar când va veni are de stat puțină vreme.
\par 11 Și fiara care era și nu mai este - este al optulea împărat și este dintre cei șapte și merge spre pieire.
\par 12 Și cele zece coarne pe care le-ai văzut sunt zece împărați, care încă n-au luat împărăția, dar care vor lua stăpânire de împărați, un ceas, împreună cu fiara.
\par 13 Aceștia au un singur cuget și puterea și stăpânirea lor o dau fiarei.
\par 14 Ei vor porni război împotriva Mielului, dar Mielul îi va birui, pentru că este Domnul domnilor și Împăratul împăraților și vor birui și cei împreună cu El - chemați și aleși și credincioși.
\par 15 Și mi-a zis: Apele pe care le-ai văzut și deasupra cărora șade desfrânata, sunt popoare și gloate și neamuri și limbi.
\par 16 Și cele zece coarne pe care le-ai văzut și fiara vor urî pe desfrânată și o vor face pustie și goală și carnea ei o vor mânca și pe ea o vor arde în foc.
\par 17 Căci Dumnezeu a pus în inimile lor să facă voia Lui și să se întâlnească într-un gând și să dea fiarei împărăția lor, până se vor împlini cuvintele lui Dumnezeu.
\par 18 Iar femeia pe care ai văzut-o este cetatea cea mare care are stăpânire peste împărații pământului.

\chapter{18}

\par 1 După acestea, am văzut un alt înger, pogorându-se din cer, având putere mare, și pământul s-a luminat de slava lui,
\par 2 Și a strigat cu glas puternic și a zis: A căzut! A căzut Babilonul cel mare și a ajuns locaș demonilor, închisoare tuturor duhurilor necurate, și închisoare tuturor păsărilor spurcate și urâte.
\par 3 Pentru că din vinul aprinderii desfrânării ei au băut toate neamurile și împărații pământului s-au desfrânat cu ea și neguțătorii lumii din mulțimea desfătărilor ei s-au îmbogățit.
\par 4 Și am auzit un alt glas din cer, zicând: Ieșiți din ea, poporul meu, ca să nu vă faceți părtași la păcatele ei și să nu fiți loviți de pedepsele sortite ei;
\par 5 Fiindcă păcatele ei au ajuns până la cer și Dumnezeu Și-a adus aminte de nedreptățile ei.
\par 6 Dați-i înapoi, precum v-a dat și ea și, după faptele ei, cu măsură îndoită, îndoit măsurați-i; în paharul în care v-a turnat, turnați-i de două ori.
\par 7 Pe cât s-a mărit pe sine și a fost în desfătări, tot pe atâta dați-i chin și plângere. Fiindcă în inima ei zice: Șed ca împărăteasă și văduvă nu sunt și jale nu voi vedea nicidecum!
\par 8 Pentru aceea într-o singură zi vor veni pedepsele peste ea: moarte și tânguire și foamete și focul va arde-o de tot, căci puternic este Domnul Dumnezeu, Cel ce o judecă.
\par 9 Iar împărații pământului, care s-au desfrânat cu ea și s-au dezmierdat cu ea, se vor jeli și se vor bate în piept pentru ea, când vor vedea fumul focului în care arde,
\par 10 Stând departe de frica chipurilor ei, și zicând: Vai! Vai! Cetatea cea mare, Babilonul, cetatea cea tare, că într-un ceas a venit judecata ta!
\par 11 Și neguțătorii lumii plâng și se tânguiesc asupra ei, căci nimeni nu mai cumpără marfa lor,
\par 12 Marfă de aur și de argint, pietre prețioase și mărgăritare, vison și porfiră, mătase și stofă stacojie, tot felul de lemn bine mirositor și tot felul de lucruri de fildeș, de lemn de mare preț și marfă de aramă și de fier și de marmură,
\par 13 Și scorțișoară și balsam și mirodenii și mir și tămâie și vin și untdelemn și făină de grâu curat și grâu și vite și oi și cai și căruțe și trupuri și suflete de oameni.
\par 14 Și roadele cele dorite de sufletul tău s-au dus de la tine și toate cele grase și strălucite au pierit de la tine și niciodată nu le vor mai găsi.
\par 15 Iar neguțătorii de aceste lucruri, care s-au îmbogățit de pe urma ei, vor sta departe, de frica chinurilor ei, plângând și tânguindu-se,
\par 16 Și zicând: Vai! Vai! Cetatea cea mare, cea înveșmântată în vison și în porfiră și în stofă stacojie și împodobită cu aur și cu pietre scumpe și cu mărgăritare! Că într-un ceas s-a pustiit atâta bogăție!
\par 17 Și toți cârmacii și toți cei ce plutesc pe mare și corăbierii și toți câți lucrează pe mare stăteau departe,
\par 18 Și strigau, uitându-se la fumul focului în care ardea, zicând: Care cetate era asemenea cu cetatea cea mare!
\par 19 Și își puneau țărână pe capetele lor și strigau plângând și tânguindu-se și zicând: Vai! Vai! Cetatea cea mare, în care s-au îmbogățit din comorile ei toți cei ce țin corăbii pe mare, că într-un ceas s-a pustiit!
\par 20 Veselește-te de ea, cerule și voi sfinților, și voi apostolilor, și voi proorocilor, pentru că Dumnezeu a pronunțat judecata voastră asupra ei.
\par 21 Și un înger puternic a ridicat o piatră, mare cât o piatră de moară, și a aruncat-o în mare, zicând: Cu astfel de repeziciune va fi aruncat Babilonul, cetatea cea mare, și nu se va mai afla.
\par 22 Și glasul celor ce cântă din chitară și din gură și din flaut și din trâmbiță nu se va mai auzi de acum în tine și nici un meșteșugar de orice fel de meșteșug nu se va mai afla în tine și huruit de mori nu se va mai auzi în tine niciodată!
\par 23 Și niciodată lumina de lampă nu se va mai ivi în tine; și glasul de mire și mireasă nu se vor mai auzi în tine niciodată, pentru că neguțătorii tăi erau stăpânitorii lumii și pentru că toate neamurile s-au rătăcit cu fermecătoria ta.
\par 24 Și s-a găsit în ea sânge de prooroci și de sfinți și sângele tuturor celor înjunghiați pe pământ.

\chapter{19}

\par 1 După acestea, am auzit, în cer, ca un glas puternic de mulțime multă zicând: Aliluia! Mântuirea și slava și puterea sunt ale Dumnezeului nostru!
\par 2 Pentru că adevărate și drepte sunt judecățile Lui! Pentru că a judecat pe desfrânata cea mare, care a stricat pământul cu desfrânarea ei, și a răzbunat sângele robilor Săi, din mâna ei!
\par 3 Și a doua oară au zis: Aliluia! Și fumul focului în care arde ea se ridică în vecii vecilor.
\par 4 Iar cei douăzeci și patru de bătrâni și cele patru ființe au căzut și s-au închinat lui Dumnezeu, Cel ce șade pe tron, zicând: Amin! Aliluia!
\par 5 Și un glas a ieșit din tron, zicând: Lăudați pe Dumnezeul nostru toate slugile Lui, cei ce vă temeți de El, mici și mari.
\par 6 Și am auzit ca un glas de mulțime multă și ca un vuiet de ape multe și ca un bubuit de tunete puternice, zicând: Aliluia! pentru că Domnul Dumnezeul nostru, Atotțiitorul, împărățește.
\par 7 Să ne bucurăm și să ne veselim și să-I dăm slavă, căci a venit nunta Mielului și mireasa Lui s-a pregătit,
\par 8 Și i s-a dat ei să se înveșmânteze cu vison curat, luminos, căci visonul sunt faptele cele drepte ale sfinților.
\par 9 Și mi-a zis: Scrie: Fericiți cei chemați la cina nunții Mielului! Și mi-a zis: Acestea sunt adevăratele cuvinte ale lui Dumnezeu.
\par 10 Și am căzut înaintea picioarelor lui, ca să mă închin lui. Și el mi-a zis: Vezi să nu faci aceasta! Sunt împreună-slujitor cu tine și cu frații tăi, care au mărturia lui Iisus. Lui Dumnezeu închină-te, căci mărturia lui Iisus este duhul proorociei.
\par 11 Și am văzut cerul deschis și iată un cal alb, și Cel ce ședea pe el se numește Credincios și Adevărat și judecă și se războiește întru dreptate.
\par 12 Iar ochii Lui sunt ca para focului și pe capul Lui sunt cununi multe și are nume scris pe care nimeni nu-l înțelege decât numai El.
\par 13 Și este îmbrăcat în veșmânt stropit cu sânge și numele Lui se cheamă: Cuvântul lui Dumnezeu.
\par 14 Și oștile din cer veneau după El, călare pe cai albi, purtând veșminte de vison alb, curat.
\par 15 Iar din gura Lui ieșea sabie ascuțită, ca să lovească neamurile cu ea. Și El îi va păstori cu toiag de fier și va călca teascul vinului aprinderii mâniei lui Dumnezeu, Atotțiitorul.
\par 16 Și pe haina Lui și pe coapsa Lui are nume scris: Împăratul împăraților și Domnul domnilor.
\par 17 Și am văzut un înger stând în soare; și a strigat cu glas puternic, grăind tuturor păsărilor care zboară spre înaltul cerului: Veniți și vă adunați la ospățul cel mare al lui Dumnezeu,
\par 18 Ca să mâncați trupuri de împărați și trupuri de căpetenii de oști și trupurile celor puternici, și trupurile cailor și ale călăreților lor, și trupurile tuturor celor slobozi și celor robi, și ale celor mici și celor mari.
\par 19 Și am văzut fiara și pe împărații pământului, și oștirile lor adunate, ca să facă război ce Cel ce șade pe cal și cu oștirea Lui.
\par 20 Și fiara a fost răpusă și, cu ea, proorocul cel mincinos, cel ce făcea înaintea ei semnele cu care amăgea pe cei ce au purtat semnul fiarei și pe cei ce s-au închinat chipului ei. Amândoi au fost aruncați de vii în iezerul de foc unde arde pucioasă.
\par 21 Iar ceilalți au fost uciși cu sabia care iese din gura Celui ce șade pe cal, și toate păsările s-au săturat din trupurile lor.

\chapter{20}

\par 1 Și am văzut un înger, pogorându-se din cer, având cheia adâncului și un lanț mare în mâna lui.
\par 2 Și a prins pe balaur, șarpele cel vechi, care este diavolul și satana, și l-a legat pe mii de ani,
\par 3 Și l-a aruncat în adânc și l-a închis și a pecetluit deasupra lui, ca să nu mai amăgească neamurile, până ce se vor sfârși miile de ani. După aceea, trebuie să fie dezlegat câtăva vreme.
\par 4 Și am văzut tronuri și celor ce ședeau pe ele li s-a dat să facă judecată. Și am văzut sufletele celor tăiați pentru mărturia lui Iisus și pentru cuvântul lui Dumnezeu, care nu s-au închinat fiarei, nici chipului ei, și nu au primit semnul ei pe fruntea și pe mâna lor. Și ei au înviat și au împărățit cu Hristos mii de ani.
\par 5 Iar ceilalți morți nu înviază până ce nu se vor sfârși miile de ani. Aceasta este învierea cea dintâi.
\par 6 Fericit și sfânt este cel ce are parte de învierea cea dintâi. Peste aceștia moartea cea de a doua nu are putere, ci vor fi preoți ai lui Dumnezeu și ai lui Hristos și vor împărăți cu El mii de ani.
\par 7 Și către sfârșitul miilor de ani, satana va fi dezlegat din închisoarea lui,
\par 8 Și va ieși să amăgească neamurile, care sunt în cele patru unghiuri ale pământului, pe Gog și pe Magog, și să le adune la război; iar numărul lor este ca nisipul mării.
\par 9 Și s-au suit pe fața pământului, și au înconjurat tabăra sfinților și cetatea cea iubită. Dar s-a pogorât foc din cer și i-a mistuit.
\par 10 Și diavolul, care-i amăgise, a fost aruncat în iezerul de foc și de pucioasă, unde este și fiara și proorocul mincinos, și vor fi chinuiți acolo, zi și noapte, în vecii vecilor.
\par 11 Și am văzut, iar, un tron mare alb și pe Cel ce ședea pe el, iar dinaintea feței Lui pământul și cerul au fugit și loc nu s-a mai găsit pentru ele.
\par 12 Și am văzut pe morți, pe cei mari și pe cei mici, stând înaintea tronului și cărțile au fost deschise; și o altă carte a fost deschisă, care este cartea vieții; și morții au fost judecați din cele scrise în cărți, potrivit cu faptele lor.
\par 13 Și marea a dat pe morții cei din ea și moartea și iadul au dat pe morții lor, și judecați au fost, fiecare după faptele sale.
\par 14 Și moartea și iadul au fost aruncate în râul de foc. Aceasta e moartea cea de a doua: iezerul cel de foc.
\par 15 Iar cine n-a fost aflat scris în cartea vieții, a fost aruncat în iezerul de foc.

\chapter{21}

\par 1 Și am văzut cer nou și pământ nou. Căci cerul cel dintâi și pământul cel dintâi au trecut; și marea nu mai este.
\par 2 Și am văzut cetatea sfântă, noul Ierusalim, pogorându-se din cer de la Dumnezeu, gătită ca o mireasă, împodobită pentru mirele ei.
\par 3 Și am auzit, din tron, un glas puternic care zicea: Iată, cortul lui Dumnezeu este cu oamenii și El va sălășlui cu ei și ei vor fi poporul Lui și însuși Dumnezeu va fi cu ei.
\par 4 Și va șterge orice lacrimă din ochii lor și moarte nu va mai fi; nici plângere, nici strigăt, nici durere nu vor mai fi, căci cele dintâi au trecut.
\par 5 Și Cel ce ședea pe tron a grăit: Iată, noi le facem pe toate. Și a zis: Scrie, fiindcă aceste cuvinte sunt vrednice de crezare și adevărate.
\par 6 Și iar mi-a zis: Făcutu-s-a! Eu sunt Alfa și Omega, începutul și sfârșitul. Celui ce însetează îi voi da să bea, în dar, din izvorul apei vieții.
\par 7 Cel ce va birui va moșteni acestea și-i voi fi lui Dumnezeu și el Îmi va fi Mie fiu
\par 8 Iar partea celor fricoși și necredincioși și spurcați și ucigași și desfrânați și fermecători și închinători de idoli și a tuturor celor mincinoși este în iezerul care arde, cu foc și cu pucioasă, care este moartea a doua.
\par 9 Și a venit unul din cei șapte îngeri, care aveau cele șapte cupe pline cu cele din urmă șapte pedepse, și a grăit către mine zicând: Vino să-ți arăt pe mireasa, femeia Mielului.
\par 10 Și m-a dus pe mine, în duh, într-un munte mare și înalt și mi-a arătat cetatea cea sfântă, Ierusalimul, pogorându-se din cer, de la Dumnezeu,
\par 11 Având slava lui Dumnezeu. Lumina ei era asemenea cu cea a pietrei de mare preț, ca piatra de iaspis, limpede cum e cristalul.
\par 12 Și avea zid mare și înalt și avea douăsprezece porți, iar la porți douăsprezece îngeri și nume scrise deasupra, care sunt numele celor douăsprezece seminții ale fiilor lui Israel.
\par 13 Spre răsărit trei porți și spre miazănoapte trei porți și spre miazăzi trei porți și spre apus trei porți.
\par 14 Iar zidul cetății avea douăsprezece pietre de temelie și în ele douăsprezece nume, ale celor douăsprezece apostoli ai Mielului.
\par 15 Și cel ce vorbea cu mine avea pentru măsurat o trestie de aur, ca să măsoare cetatea și porțile ei și zidul ei.
\par 16 Și cetatea este în patru colțuri și lungimea ei este tot atâta cât și lățimea. Și a măsurat cetatea cu trestia: douăsprezece mii de stadii. Lungimea și lărgimea și înălțimea ei sunt deopotrivă.
\par 17 Și a măsurat și zidul ei: o sută patruzeci și patru de coți, după măsura omenească, care este și a îngerului.
\par 18 Și zidăria zidului ei este de iaspis, iar cetatea este din aur curat, ca sticla cea curată.
\par 19 Temeliile zidului cetății sunt împodobite cu tot felul de pietre scumpe: întâia piatră de temelie este de iaspis, a doua din safir, a treia din halcedon, a patra de smarald,
\par 20 A cincea de sardonix, a șasea de sardiu, a șaptea de hrisolit, a opta de beril, a noua de topaz, a zecea de hrisopras, a unsprezecea de iachint, a douăsprezecea de ametist.
\par 21 Iar cele douăsprezece porți sunt douăsprezece mărgăritare; fiecare din porți este dintr-un mărgăritar. Și piața cetății este de aur curat, și străvezie ca sticla.
\par 22 Și templu n-am văzut în ea, pentru că Domnul Dumnezeu, Atotțiitorul, și Mielul este templul ei.
\par 23 Și cetatea nu are trebuință de soare, nici de lună, ca să o lumineze, căci slava lui Dumnezeu a luminat-o și făclia ei este Mielul.
\par 24 Și neamurile vor umbla în lumina ei, iar împărații pământului vor aduce la ea mărirea lor.
\par 25 Și porțile cetății nu se vor mai închide ziua, căci noaptea nu va mai fi acolo.
\par 26 Și vor aduce în ea slava și cinstea neamurilor.
\par 27 Și în cetate nu va intra nimic pângărit și nimeni care e dedat cu spurcăciunea și cu minciuna, ci numai cei scriși în Cartea vieții Mielului.

\chapter{22}

\par 1 Și mi-a arătat, apoi, râul și apa vieții, limpede cum e cristalul și care izvorăște din tronul lui Dumnezeu și al Mielului,
\par 2 Și în mijlocul pieței din cetate, de o parte și de alta a râului, crește pomul vieții, făcând rod de douăsprezece ori pe an, în fiecare lună dându-și rodul; și frunzele pomului sunt spre tămăduirea neamurilor.
\par 3 Nici un blestem nu va mai fi. Și tronul lui Dumnezeu și al Mielului va fi în ea și slugile Lui Îi vor sluji Lui.
\par 4 Și vor vedea fața Lui și numele Lui va fi pe frunțile lor.
\par 5 Și noapte nu va mai fi; și nu au trebuință de lumina lămpii sau de lumina soarelui, pentru că Domnul Dumnezeu le va fi lor lumină și vor împărăți în vecii vecilor.
\par 6 Și îngerul mi-a zis: Aceste cuvinte sunt vrednice de crezare și adevărate și Domnul, Dumnezeul duhurilor proorocilor, a trimis pe îngerul Său să arate robilor Săi cele ce trebuie să se întâmple în curând.
\par 7 Și iată vin curând. Fericit cel ce păzește cuvintele proorociei acestei cărți!
\par 8 Și eu, Ioan, sunt cel ce am văzut și am auzit acestea, iar când am auzit și am văzut, am căzut să mă închin înaintea picioarelor îngerului care mi-a arătat acestea.
\par 9 Și el mi-a zis: Vezi să nu faci aceasta! Căci sunt împreună-slujitor cu tine și cu frații tăi, proorocii, și cu cei ce păstrează cuvintele cărții acesteia. Lui Dumnezeu închină-te!
\par 10 Apoi mi-a zis: Să nu pecetluiești cuvintele proorociei acestei cărți, căci vremea este aproape.
\par 11 Cine e nedrept, să nedreptățească înainte. Cine e spurcat, să se spurce încă. Cine este drept, să facă dreptate mai departe. Cine este sfânt, să se sfințească încă.
\par 12 Iată, vin curând și plata Mea este cu Mine, ca să dau fiecăruia, după cum este fapta lui.
\par 13 Eu sunt Alfa și Omega, cel dintâi și cel de pe urmă, începutul și sfârșitul.
\par 14 Fericiți cei ce spală veșmintele lor ca să aibă stăpânire peste pomul vieții și prin porți să intre în cetate!
\par 15 Afară câinii și vrăjitorii și desfrânații și ucigașii și închinătorii de idoli și toți cei ce lucrează și iubesc minciuna!
\par 16 Eu, Iisus, am trimis pe îngerul Meu ca să mărturisească vouă acestea, cu privire la Biserici. Eu sunt rădăcina și odrasla lui David, steaua care strălucește dimineața.
\par 17 Și Duhul și mireasa zic: Vino. Și cel ce aude să zică: Vino. Și cel însetat să vină, cel ce dorește să ia în dar apa vieții.
\par 18 Și eu mărturisesc oricui ascultă cuvintele proorociei acestei cărți: De va mai adăuga cineva ceva la ele, Dumnezeu va trimite asupra lui pedepsele ce sunt scrise în cartea aceasta;
\par 19 Iar de va scoate cineva din cuvintele cărții acestei proorocii, Dumnezeu va scoate partea lui din pomul vieții și din cetatea sfântă și de la cele scrise în cartea aceasta.
\par 20 Cel ce mărturisește acestea zice: Da, vin curând. Amin! Vino, Doamne Iisuse!
\par 21 Harul Domnului Iisus Hristos, cu voi cu toți! Amin.


\end{document}