\begin{document}

\title{Iuda}


\chapter{1}

\par 1 Iuda, rob al lui Iisus Hristos și frate al lui Iacov, celor ce sunt chemați, iubiți în Dumnezeu-Tatăl și păstrați pentru Iisus Hristos:
\par 2 Milă vouă și pace și iubirea să se înmulțească!
\par 3 Iubiților, punând toată râvna să vă scriu despre mântuirea cea de obște, simțit-am nevoie să vă scriu și să vă îndemn ca să luptați pentru credința dată sfinților, odată pentru totdeauna.
\par 4 Căci s-au strecurat printre voi unii oameni nelegiuiți, care de mai înainte au fost rânduiți spre această osândă, schimbând ei harul Dumnezeului nostru în desfrânare, și care tăgăduiesc pe singurul nostru Stăpân și Domn, pe Iisus Hristos.
\par 5 Voiesc dar să vă aduc aminte vouă celor ce ați știut odată toate acestea că Domnul, după ce a izbăvit pe poporul Său din pământul Egiptului, a pierdut, după aceea, pe cei ce n-au crezut.
\par 6 Iar pe îngerii care nu și-au păzit vrednicia, ci au părăsit locașul lor, i-a pus la păstrare sub întuneric, în lanțuri veșnice, spre judecata zilei celei mari.
\par 7 Tot așa, Sodoma și Gomora și cetățile dimprejurul lor care, în același chip ca acestea, s-au dat la desfrânare și au umblat după trup străin, stau înainte ca pildă, suferind pedeapsa focului celui veșnic.
\par 8 Asemenea deci și aceștia, visând, pângăresc trupul, leapădă stăpânirea și hulesc măririle (cerești).
\par 9 Dar Mihail Arhanghelul, când se împotrivea diavolului, certându-se cu el pentru trupul lui Moise, n-a îndrăznit să aducă judecată de hulă, ci a zis: "Să te certe pe tine Domnul!"
\par 10 Aceștia însă defaimă cele ce nu cunosc, iar cele ce, - ca dobitoacele necuvântătoare, - știu din fire, într-acestea își găsesc pieirea.
\par 11 Vai lor! Că au umblat în calea lui Cain și, pentru plată, s-au dat cu totul în rătăcirea lui Balaam și au pierit ca în răzvrătirea lui Core.
\par 12 Aceștia sunt ca niște pete de necurăție la mesele voastre obștești, ospătând fără sfială împreună cu voi, îmbuibându-se pe ei înșiși, nori fără apă, purtați de vânturi, pomi tomnatici fără roade, de două ori uscați și dezrădăcinați,
\par 13 Valuri sălbatice ale mării, care își spumegă rușinea lor, stele rătăcitoare, cărora întunericul întunericului li se păstrează în veșnicie.
\par 14 Dar și Enoh, al șaptelea de la Adam, a proorocit despre aceștia, zicând: Iată, a venit Domnul cu zecile de mii de sfinți ai Lui,
\par 15 Ca să facă judecată împotriva tuturor și să mustre pe toți nelegiuiții de toate faptele nelegiuirii lor, în care au făcut fărădelege, și de toate cuvintele de ocară pe care ei, păcătoși, netemători de Dumnezeu, le-au rostit împotriva Lui.
\par 16 Aceștia sunt cârtitori, nemulțumiți cu starea lor, umblând după poftele lor și gura lor grăiește lucruri trufașe, deși, pentru folos, dau unor fețe mare cinste.
\par 17 Voi, însă, iubiților, aduceți-vă aminte de cuvintele zise mai dinainte de către apostolii Domnului nostru Iisus Hristos,
\par 18 Că ei vă spuneau: În vremea de pe urmă vor fi batjocoritori, umblând potrivit cu poftele lor nelegiuite.
\par 19 Aceștia sunt cei ce fac dezbinări, (oameni) firești, care nu au Duhul.
\par 20 Dar voi, iubiților, zidiți-vă pe voi înșivă, întru a voastră prea sfântă credință, rugându-vă în Duhul Sfânt.
\par 21 Păziți-vă întru dragostea lui Dumnezeu și așteptați mila Domnului nostru Iisus Hristos, spre viață veșnică.
\par 22 Și pe unii, șovăitori, mustrați-i,
\par 23 Pe alții, smulgându-i din foc, mântuiți-i; de alții, însă, fie-vă milă cu frică, urând și cămașa spurcată de pe trupul lor.
\par 24 Iar Celui ce poate să vă păzească pe voi de orice cădere și să vă pună înaintea slavei Lui, neprihăniți cu bucurie mare,
\par 25 Singurului Dumnezeu, Mântuitorul nostru, prin Iisus Hristos, Domnul nostru, slavă, preamărire, putere și stăpânire, mai înainte de tot veacul și acum și întru toți vecii. Amin!


\end{document}