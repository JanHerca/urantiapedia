\begin{document}

\title{1 Petru}


\chapter{1}

\par 1 Petru, apostol al lui Iisus Hristos, către cei ce trăiesc împrăștiați printre străini, în Pont, în Galatia, în Capadocia, în Asia și în Bitinia,
\par 2 Aleși după cea mai dinainte știință a lui Dumnezeu-Tatăl, și prin sfințirea de către Duhul, spre ascultare și stropirea cu sângele lui Iisus Hristos: har vouă și pacea să se înmulțească!
\par 3 Binecuvântat fie Dumnezeu și Tatăl Domnului nostru Iisus Hristos, Care, după mare mila Sa, prin învierea lui Iisus Hristos din morți, ne-a născut din nou, spre nădejde vie,
\par 4 Spre moștenire nestricăcioasă și neîntinată și neveștejită, păstrată în ceruri pentru voi,
\par 5 Cei ce sunteți păziți cu puterea lui Dumnezeu, prin credință, spre mântuire, gata să se dea pe față în vremea de apoi.
\par 6 Întru aceasta vă bucurați, măcar că acum ar trebui să fiți triști, încercați fiind de multe feluri de ispite pentru puțină vreme,
\par 7 Pentru ca credința voastră încercată, mult mai de preț decât aurul cel pieritor, dar lămurit prin foc, să fie găsită spre laudă și spre slavă și spre cinste, la arătarea lui Iisus Hristos.
\par 8 Pe El, fără să-L fi văzut, Îl iubiți; întru El, deși acum nu-L vedeți, voi credeți și vă bucurați cu bucurie negrăită și preamărită,
\par 9 Dobândind răsplata credinței voastre, mântuirea sufletelor.
\par 10 Această mântuire au căutat-o cu stăruință și au cercetat-o cu de-amănuntul proorocii, care au proorocit despre harul ce avea să vină la voi.
\par 11 Cercetând în care și în ce fel de vreme le arăta Duhul lui Hristos, Care era întru ei, când le mărturisea de mai înainte despre patimile lui Hristos și despre măririle cele de după ele,
\par 12 Lor le-a fost descoperit că nu pentru ei înșiși, ci pentru voi slujeau ei aceste lucruri, care acum vi s-au vestit prin cei ce, întru Duhul Sfânt trimis din cer, v-au propovăduit Evanghelia, spre care și îngerii doresc să privească.
\par 13 Pentru aceea, încingând mijloacele cugetului vostru, trezindu-vă, nădăjduiți desăvârșit în harul care vi se va da vouă, la arătarea lui Iisus Hristos.
\par 14 Ca fii ascultători, nu vă potriviți poftelor de mai înainte, din vremea neștiinței voastre,
\par 15 Ci, după Sfântul Care v-a chemat pe voi, fiți și voi înșivă sfinți în toată petrecerea vieții.
\par 16 Că scris este: "Fiți sfinți, pentru că Eu sunt Sfânt".
\par 17 Și dacă chemați Tată pe Cel ce judecă cu nepărtinire, după lucrul fiecăruia, petreceți în frică zilele vremelniciei voastre,
\par 18 Știind că nu cu lucruri stricăcioase, cu argint sau cu aur, ați fost răscumpărați din viața voastră deșartă, lăsată de la părinți,
\par 19 Ci cu scumpul sânge al lui Hristos, ca al unui miel nevinovat și neprihănit,
\par 20 Care a fost cunoscut mai dinainte de întemeierea lumii, dar Care S-a arătat, în anii cei mai de pe urmă, pentru voi,
\par 21 Cei ce prin El ați crezut în Dumnezeu, Care L-a înviat pe El din morți, și I-a dat Lui slavă, ca să vă fie credința și nădejdea voastră în Dumnezeu.
\par 22 Curățindu-vă sufletele prin ascultarea de adevăr, spre nefățarnică iubire de frați, iubiți-vă unul pe altul, din toată inima, cu toată stăruința,
\par 23 Fiind născuți din nou nu din sămânță stricăcioasă, ci din nestricăcioasă, prin cuvântul lui Dumnezeu cel viu și care rămâne în veac.
\par 24 Pentru că tot trupul este ca iarba și toată slava lui ca floarea ierbii: uscatu-s-a iarba și floarea a căzut,
\par 25 Iar cuvântul Domnului rămâne în veac. Și acesta este cuvântul, care vi s-a binevestit.

\chapter{2}

\par 1 Deci, lepădând toată răutatea și tot vicleșugul și fățărniciile și pizmele și toate clevetirile,
\par 2 Ca niște prunci de curând născuți, să doriți laptele cel duhovnicesc și neprefăcut, ca prin el să creșteți spre mântuire,
\par 3 De vreme ce ați gustat și ați văzut că bun este Domnul.
\par 4 Apropiați-vă de El, piatra cea vie, de oameni într-adevăr neluată în seamă, dar la Dumnezeu aleasă și de preț;
\par 5 Și voi înșivă, ca pietre vii, zidiți-vă drept casă duhovnicească, preoție sfântă, ca să aduceți jertfe duhovnicești, bine-plăcute lui Dumnezeu, prin Iisus Hristos;
\par 6 Pentru că scris este în Scriptură: "Iată, pun în Sion Piatra din capul unghiului, aleasă, de mare preț, și cel ce va crede în ea nu se va rușina".
\par 7 Pentru voi, deci, care credeți, (Piatra) este cinstea; iar pentru cei ce nu cred, piatra pe care n-au băgat-o în seamă ziditorii, aceasta a ajuns să fie în capul unghiului,
\par 8 Și piatră de poticnire și stâncă de sminteală, de care se poticnesc, fiindcă n-au dat ascultare cuvântului, spre care au și fost puși.
\par 9 Iar voi sunteți seminție aleasă, preoție împărătească, neam sfânt, popor agonisit de Dumnezeu, ca să vestiți în lume bunătățile Celui ce v-a chemat din întuneric, la lumina Sa cea minunată,
\par 10 Voi care odinioară nu erați popor, iar acum sunteți poporul lui Dumnezeu; voi care odinioară n-aveați parte de milă, iar acum sunteți miluiți.
\par 11 Iubiților, vă îndemn ca pe niște străini ce sunteți și călători aici pe pământ, să vă feriți de poftele cele trupești care se războiesc împotriva sufletului.
\par 12 Purtați-vă cu cinste între neamuri, ca în ceea ce ei acum vă bârfesc ca pe niște făcători de rele, privind ei mai de aproape faptele voastre cele bune, să preamărească pe Dumnezeu, în ziua când îi va cerceta.
\par 13 Supuneți-vă, pentru Domnul, oricărei orânduiri omenești, fie împăratului, ca înalt stăpânitor,
\par 14 Fie dregătorilor, ca unora ce sunt trimiși de el, spre pedepsirea făcătorilor de rele și spre lauda făcătorilor de bine;
\par 15 Căci așa este voia lui Dumnezeu, ca voi, prin faptele voastre cele bune, să închideți gura oamenilor fără minte și fără cunoștință.
\par 16 Trăiți ca oamenii liberi, dar nu ca și cum ați avea libertatea drept acoperământ al răutății, ci ca robi ai lui Dumnezeu.
\par 17 Dați tuturor cinste, iubiți frăția, temeți-vă de Dumnezeu, cinstiți pe împărat.
\par 18 Slugilor, supuneți-vă stăpânilor voștri, cu toată frica, nu numai celor buni și blânzi, ci și celor urâcioși.
\par 19 Căci aceasta este plăcut lui Dumnezeu, să sufere cineva întristări, pe nedrept, cu gândul la El.
\par 20 Căci, ce laudă este dacă, pentru greșeală, primiți bătaie întru răbdare? Iar dacă, pentru binele făcut, veți pătimi și veți răbda, aceasta este plăcut lui Dumnezeu.
\par 21 Căci spre aceasta ați fost chemați, că și Hristos a pătimit pentru voi, lăsându-vă pildă, ca să pășiți pe urmele Lui,
\par 22 Care n-a săvârșit nici un păcat, nici s-a aflat vicleșug în gura Lui,
\par 23 Și Care, ocărât fiind, nu răspundea cu ocară; suferind, nu amenința, ci Se lăsa în știrea Celui ce judecă cu dreptate.
\par 24 El a purtat păcatele noastre, în trupul Său, pe lemn, pentru ca noi, murind față de păcate, să viețuim dreptății: cu a Cărui rană v-ați vindecat.
\par 25 Căci erați ca niște oi rătăcite, dar v-ați întors acum la Păstorul și la Păzitorul sufletelor voastre.

\chapter{3}

\par 1 Asemenea și voi, femeilor, supuneți-vă bărbaților voștri, așa încât, chiar dacă sunt unii care nu se pleacă cuvântului, să fie câștigați, fără propovăduire, prin purtarea femeilor lor,
\par 2 Văzând de aproape viața voastră curată și plină de sfială.
\par 3 Podoaba voastră să nu fie cea din afară: împletirea părului, podoabele de aur și îmbrăcarea hainelor scumpe,
\par 4 Ci să fie omul cel tainic al inimii, întru nestricăcioasa podoabă a duhului blând și liniștit, care este de mare preț înaintea lui Dumnezeu.
\par 5 Că așa se împodobeau, odinioară, și sfintele femei, care nădăjduiau în Dumnezeu, supunându-se bărbaților lor,
\par 6 Precum Sarra asculta de Avraam și-l numea pe el domn, ale cărei fiice sunteți, dacă faceți ce e bine și nu vă temeți de nimic.
\par 7 Voi, bărbaților, de asemenea, trăiți înțelepțește cu femeile voastre, ca fiind făpturi mai slabe, și faceți-le parte de cinste, ca unora care, împreună cu voi, sunt moștenitoare ale harului vieții, așa încât rugăciunile voastre să nu fie împiedicate.
\par 8 În sfârșit, fiți toți într-un gând, împreună-pătimitori, iubitori de frați, milostivi, smeriți.
\par 9 Nu răsplătiți răul cu rău sau ocara cu ocară, ci, dimpotrivă, binecuvântați, căci spre aceasta ați fost chemați, ca să moșteniți binecuvântarea.
\par 10 Cel ce voiește să iubească viața și să vadă zile bune să-și oprească limba de la rău și buzele sale să nu grăiască vicleșug;
\par 11 Să se ferească de rău și să facă bine; să caute pacea și s-o urmeze;
\par 12 Căci ochii Domnului sunt peste cei drepți și urechile Lui spre rugăciunile lor, iar fața Domnului este împotriva celor ce fac rele.
\par 13 Și cine vă va face vouă rău, dacă sunteți plini de râvnă pentru bine?
\par 14 Dar de veți și pătimi pentru dreptate, fericiți veți fi. Iar de frica lor să nu vă temeți, nici să vă tulburați.
\par 15 Ci pe Domnul, pe Hristos, să-L sfințiți în inimile voastre și să fiți gata totdeauna să răspundeți oricui vă cere socoteală despre nădejdea voastră,
\par 16 Dar cu blândețe și cu frică, având cuget curat, ca, tocmai în ceea ce sunteți clevetiți, să iasă de rușine cei ce grăiesc de rău purtarea voastră cea bună întru Hristos.
\par 17 Căci e mai bine, dacă așa este voia lui Dumnezeu, să pătimiți făcând cele bune, decât făcând cele rele.
\par 18 Pentru că și Hristos a suferit odată moartea pentru păcatele noastre, El cel drept pentru cei nedrepți, ca să ne aducă pe noi la Dumnezeu, omorât fiind cu trupul, dar viu făcut cu duhul,
\par 19 Cu care S-a coborât și a propovăduit și duhurilor ținute în închisoare,
\par 20 Care fuseseră neascultătoare altădată, când îndelunga-răbdare a lui Dumnezeu aștepta, în zilele lui Noe, și se pregătea corabia în care puține suflete, adică opt, s-au mântuit prin apă.
\par 21 Iar această mântuire prin apă închipuia botezul, care vă mântuiește astăzi și pe voi, nu ca ștergere a necurăției trupului, ci ca deschiderea cugetului bun către Dumnezeu, prin învierea lui Iisus Hristos,
\par 22 Care, după ce S-a suit la cer, este de-a dreapta lui Dumnezeu, și se supun Lui îngerii și stăpâniile și puterile.

\chapter{4}

\par 1 Așadar, fiindcă Hristos a pătimit cu trupul, înarmați-vă și voi cu gândul acesta: că cine a suferit cu trupul a isprăvit cu păcatul,
\par 2 Ca să nu mai trăiască timpul ce mai are de trăit în trup după poftele oamenilor, ci după voia lui Dumnezeu.
\par 3 Destul este că, în vremurile trecute, ați făcut cu desăvârșire voia neamurilor, umblând în desfrânări, în pofte, în beții, în ospețe fără măsură, în petreceri cu vin mult și în neiertate slujiri idolești.
\par 4 De aceea ei se miră că voi nu mai alergați cu ei în aceeași revărsare a desfrâului și vă hulesc.
\par 5 Ei își vor da seama înaintea Celui ce este gata să judece viii și morții.
\par 6 Că spre aceasta s-a binevestit morților, ca să fie judecați ca oameni, după trup, dar să vieze, după Dumnezeu cu duhul.
\par 7 Iar sfârșitul tuturor s-a apropiat; fiți dar cu mintea întreagă și privegheați în rugăciuni.
\par 8 Dar mai presus de toate, țineți din răsputeri la dragostea dintre voi, pentru că dragostea acoperă mulțime de păcate.
\par 9 Fiți, între voi, iubitori de străini, fără cârtire.
\par 10 După darul pe care l-a primit fiecare, slujiți unii altora, ca niște buni iconomi ai harului celui de multe feluri al lui Dumnezeu.
\par 11 Dacă vorbește cineva, cuvintele lui să fie ca ale lui Dumnezeu; dacă slujește cineva, slujba lui să fie ca din puterea pe care o dă Dumnezeu, pentru ca întru toate Dumnezeu să se slăvească prin Iisus Hristos, Căruia Îi este slava și stăpânirea în vecii vecilor. Amin.
\par 12 Iubiților, nu vă mirați de focul aprins între voi spre ispitire, ca și cum vi s-ar întâmpla ceva străin,
\par 13 Ci, întrucât sunteți părtași la suferințele lui Hristos, bucurați-vă, pentru ca și la arătarea slavei Lui să vă bucurați cu bucurie mare.
\par 14 De sunteți ocărâți pentru numele lui Hristos, fericiți sunteți, căci Duhul slavei și al lui Dumnezeu Se odihnește peste voi; de către unii El se hulește, iar de voi se preaslăvește.
\par 15 Nimeni dintre voi să nu sufere ca ucigaș, sau fur, sau făcător de rele, sau ca un râvnitor de lucruri străine.
\par 16 Iar de suferă ca creștin, să nu se rușineze, ci să preamărească pe Dumnezeu, pentru numele acesta.
\par 17 Căci vremea este ca să înceapă judecata de la casa lui Dumnezeu; și dacă începe întâi de la noi, care va fi sfârșitul celor care nu ascultă de Evanghelia lui Dumnezeu?
\par 18 Și dacă dreptul abia se mântuiește, ce va fi cu cel necredincios și păcătos?
\par 19 Pentru aceea, și cei ce suferă, după voia lui Dumnezeu, să-și încredințeze Lui, credinciosului Ziditor, sufletele lor, săvârșind fapte bune.

\chapter{5}

\par 1 Pe preoții cei dintre voi îi rog ca unul ce sunt împreună-preot și martor al patimilor lui Hristos și părtaș al slavei celei ce va să se descopere:
\par 2 Păstoriți turma lui Dumnezeu, dată în paza voastră, cercetând-o, nu cu silnicie, ci cu voie bună, după Dumnezeu, nu pentru câștig urât, ci din dragoste;
\par 3 Nu ca și cum ați fi stăpâni peste Biserici, ci pilde făcându-vă turmei.
\par 4 Iar când Se va arăta Mai-marele păstorilor, veți lua cununa cea neveștejită a măririi.
\par 5 Tot așa și voi, fiilor duhovnicești, supuneți-vă preoților; și toți, unii față de alții, îmbrăcați-vă întru smerenie, pentru că Dumnezeu celor mândri le stă împotrivă, iar celor smeriți le dă har.
\par 6 Deci, smeriți-vă sub mâna cea tare a lui Dumnezeu, ca El să vă înalțe la timpul cuvenit.
\par 7 Lăsați-I Lui toată grija voastră, căci El are grijă de voi.
\par 8 Fiți treji, privegheați. Potrivnicul vostru, diavolul, umblă, răcnind ca un leu, căutând pe cine să înghită,
\par 9 Căruia stați împotrivă, tari în credință, știind că aceleași suferințe îndură și frații voștri în lume.
\par 10 Iar Dumnezeul a tot harul, Care v-a chemat la slava Sa cea veșnică, întru Hristos Iisus, El însuși, după ce veți suferi puțină vreme, vă va duce la desăvârșire, vă va întări, vă va împuternici, vă va face neclintiți.
\par 11 A Lui fie slava și puterea în vecii vecilor. Amin!
\par 12 V-am scris aceste puține lucruri, prin Silvan, pe care îl socotesc frate credincios, ca să vă îndemn și să vă mărturisesc că adevăratul har al lui Dumnezeu este acesta, în care stați.
\par 13 Biserica cea aleasă din Babilon și Marcu, fiul meu, vă îmbrățișează.
\par 14 Îmbrățișați-vă unul pe altul cu sărutarea dragostei. Pace vouă tuturor, celor întru Hristos Iisus. Amin.


\end{document}