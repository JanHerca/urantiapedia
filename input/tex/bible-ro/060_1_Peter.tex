\begin{document}

\title{1 Petru}


\chapter{1}

\par 1 Petru, apostol al lui Iisus Hristos, catre cei ce traiesc împra?tia?i printre straini, în Pont, în Galatia, în Capadocia, în Asia ?i în Bitinia,
\par 2 Ale?i dupa cea mai dinainte ?tiin?a a lui Dumnezeu-Tatal, ?i prin sfin?irea de catre Duhul, spre ascultare ?i stropirea cu sângele lui Iisus Hristos: har voua ?i pacea sa se înmul?easca!
\par 3 Binecuvântat fie Dumnezeu ?i Tatal Domnului nostru Iisus Hristos, Care, dupa mare mila Sa, prin învierea lui Iisus Hristos din mor?i, ne-a nascut din nou, spre nadejde vie,
\par 4 Spre mo?tenire nestricacioasa ?i neîntinata ?i neve?tejita, pastrata în ceruri pentru voi,
\par 5 Cei ce sunte?i pazi?i cu puterea lui Dumnezeu, prin credin?a, spre mântuire, gata sa se dea pe fa?a în vremea de apoi.
\par 6 Întru aceasta va bucura?i, macar ca acum ar trebui sa fi?i tri?ti, încerca?i fiind de multe feluri de ispite pentru pu?ina vreme,
\par 7 Pentru ca credin?a voastra încercata, mult mai de pre? decât aurul cel pieritor, dar lamurit prin foc, sa fie gasita spre lauda ?i spre slava ?i spre cinste, la aratarea lui Iisus Hristos.
\par 8 Pe El, fara sa-L fi vazut, Îl iubi?i; întru El, de?i acum nu-L vede?i, voi crede?i ?i va bucura?i cu bucurie negraita ?i preamarita,
\par 9 Dobândind rasplata credin?ei voastre, mântuirea sufletelor.
\par 10 Aceasta mântuire au cautat-o cu staruin?a ?i au cercetat-o cu de-amanuntul proorocii, care au proorocit despre harul ce avea sa vina la voi.
\par 11 Cercetând în care ?i în ce fel de vreme le arata Duhul lui Hristos, Care era întru ei, când le marturisea de mai înainte despre patimile lui Hristos ?i despre maririle cele de dupa ele,
\par 12 Lor le-a fost descoperit ca nu pentru ei în?i?i, ci pentru voi slujeau ei aceste lucruri, care acum vi s-au vestit prin cei ce, întru Duhul Sfânt trimis din cer, v-au propovaduit Evanghelia, spre care ?i îngerii doresc sa priveasca.
\par 13 Pentru aceea, încingând mijloacele cugetului vostru, trezindu-va, nadajdui?i desavâr?it în harul care vi se va da voua, la aratarea lui Iisus Hristos.
\par 14 Ca fii ascultatori, nu va potrivi?i poftelor de mai înainte, din vremea ne?tiin?ei voastre,
\par 15 Ci, dupa Sfântul Care v-a chemat pe voi, fi?i ?i voi în?iva sfin?i în toata petrecerea vie?ii.
\par 16 Ca scris este: "Fi?i sfin?i, pentru ca Eu sunt Sfânt".
\par 17 ?i daca chema?i Tata pe Cel ce judeca cu nepartinire, dupa lucrul fiecaruia, petrece?i în frica zilele vremelniciei voastre,
\par 18 ?tiind ca nu cu lucruri stricacioase, cu argint sau cu aur, a?i fost rascumpara?i din via?a voastra de?arta, lasata de la parin?i,
\par 19 Ci cu scumpul sânge al lui Hristos, ca al unui miel nevinovat ?i neprihanit,
\par 20 Care a fost cunoscut mai dinainte de întemeierea lumii, dar Care S-a aratat, în anii cei mai de pe urma, pentru voi,
\par 21 Cei ce prin El a?i crezut în Dumnezeu, Care L-a înviat pe El din mor?i, ?i I-a dat Lui slava, ca sa va fie credin?a ?i nadejdea voastra în Dumnezeu.
\par 22 Cura?indu-va sufletele prin ascultarea de adevar, spre nefa?arnica iubire de fra?i, iubi?i-va unul pe altul, din toata inima, cu toata staruin?a,
\par 23 Fiind nascu?i din nou nu din samân?a stricacioasa, ci din nestricacioasa, prin cuvântul lui Dumnezeu cel viu ?i care ramâne în veac.
\par 24 Pentru ca tot trupul este ca iarba ?i toata slava lui ca floarea ierbii: uscatu-s-a iarba ?i floarea a cazut,
\par 25 Iar cuvântul Domnului ramâne în veac. ?i acesta este cuvântul, care vi s-a binevestit.

\chapter{2}

\par 1 Deci, lepadând toata rautatea ?i tot vicle?ugul ?i fa?arniciile ?i pizmele ?i toate clevetirile,
\par 2 Ca ni?te prunci de curând nascu?i, sa dori?i laptele cel duhovnicesc ?i neprefacut, ca prin el sa cre?te?i spre mântuire,
\par 3 De vreme ce a?i gustat ?i a?i vazut ca bun este Domnul.
\par 4 Apropia?i-va de El, piatra cea vie, de oameni într-adevar neluata în seama, dar la Dumnezeu aleasa ?i de pre?;
\par 5 ?i voi în?iva, ca pietre vii, zidi?i-va drept casa duhovniceasca, preo?ie sfânta, ca sa aduce?i jertfe duhovnice?ti, bine-placute lui Dumnezeu, prin Iisus Hristos;
\par 6 Pentru ca scris este în Scriptura: "Iata, pun în Sion Piatra din capul unghiului, aleasa, de mare pre?, ?i cel ce va crede în ea nu se va ru?ina".
\par 7 Pentru voi, deci, care crede?i, (Piatra) este cinstea; iar pentru cei ce nu cred, piatra pe care n-au bagat-o în seama ziditorii, aceasta a ajuns sa fie în capul unghiului,
\par 8 ?i piatra de poticnire ?i stânca de sminteala, de care se poticnesc, fiindca n-au dat ascultare cuvântului, spre care au ?i fost pu?i.
\par 9 Iar voi sunte?i semin?ie aleasa, preo?ie împarateasca, neam sfânt, popor agonisit de Dumnezeu, ca sa vesti?i în lume bunata?ile Celui ce v-a chemat din întuneric, la lumina Sa cea minunata,
\par 10 Voi care odinioara nu era?i popor, iar acum sunte?i poporul lui Dumnezeu; voi care odinioara n-avea?i parte de mila, iar acum sunte?i milui?i.
\par 11 Iubi?ilor, va îndemn ca pe ni?te straini ce sunte?i ?i calatori aici pe pamânt, sa va feri?i de poftele cele trupe?ti care se razboiesc împotriva sufletului.
\par 12 Purta?i-va cu cinste între neamuri, ca în ceea ce ei acum va bârfesc ca pe ni?te facatori de rele, privind ei mai de aproape faptele voastre cele bune, sa preamareasca pe Dumnezeu, în ziua când îi va cerceta.
\par 13 Supune?i-va, pentru Domnul, oricarei orânduiri omene?ti, fie împaratului, ca înalt stapânitor,
\par 14 Fie dregatorilor, ca unora ce sunt trimi?i de el, spre pedepsirea facatorilor de rele ?i spre lauda facatorilor de bine;
\par 15 Caci a?a este voia lui Dumnezeu, ca voi, prin faptele voastre cele bune, sa închide?i gura oamenilor fara minte ?i fara cuno?tin?a.
\par 16 Trai?i ca oamenii liberi, dar nu ca ?i cum a?i avea libertatea drept acoperamânt al rauta?ii, ci ca robi ai lui Dumnezeu.
\par 17 Da?i tuturor cinste, iubi?i fra?ia, teme?i-va de Dumnezeu, cinsti?i pe împarat.
\par 18 Slugilor, supune?i-va stapânilor vo?tri, cu toata frica, nu numai celor buni ?i blânzi, ci ?i celor urâcio?i.
\par 19 Caci aceasta este placut lui Dumnezeu, sa sufere cineva întristari, pe nedrept, cu gândul la El.
\par 20 Caci, ce lauda este daca, pentru gre?eala, primi?i bataie întru rabdare? Iar daca, pentru binele facut, ve?i patimi ?i ve?i rabda, aceasta este placut lui Dumnezeu.
\par 21 Caci spre aceasta a?i fost chema?i, ca ?i Hristos a patimit pentru voi, lasându-va pilda, ca sa pa?i?i pe urmele Lui,
\par 22 Care n-a savâr?it nici un pacat, nici s-a aflat vicle?ug în gura Lui,
\par 23 ?i Care, ocarât fiind, nu raspundea cu ocara; suferind, nu amenin?a, ci Se lasa în ?tirea Celui ce judeca cu dreptate.
\par 24 El a purtat pacatele noastre, în trupul Sau, pe lemn, pentru ca noi, murind fa?a de pacate, sa vie?uim drepta?ii: cu a Carui rana v-a?i vindecat.
\par 25 Caci era?i ca ni?te oi ratacite, dar v-a?i întors acum la Pastorul ?i la Pazitorul sufletelor voastre.

\chapter{3}

\par 1 Asemenea ?i voi, femeilor, supune?i-va barba?ilor vo?tri, a?a încât, chiar daca sunt unii care nu se pleaca cuvântului, sa fie câ?tiga?i, fara propovaduire, prin purtarea femeilor lor,
\par 2 Vazând de aproape via?a voastra curata ?i plina de sfiala.
\par 3 Podoaba voastra sa nu fie cea din afara: împletirea parului, podoabele de aur ?i îmbracarea hainelor scumpe,
\par 4 Ci sa fie omul cel tainic al inimii, întru nestricacioasa podoaba a duhului blând ?i lini?tit, care este de mare pre? înaintea lui Dumnezeu.
\par 5 Ca a?a se împodobeau, odinioara, ?i sfintele femei, care nadajduiau în Dumnezeu, supunându-se barba?ilor lor,
\par 6 Precum Sarra asculta de Avraam ?i-l numea pe el domn, ale carei fiice sunte?i, daca face?i ce e bine ?i nu va teme?i de nimic.
\par 7 Voi, barba?ilor, de asemenea, trai?i în?elep?e?te cu femeile voastre, ca fiind fapturi mai slabe, ?i face?i-le parte de cinste, ca unora care, împreuna cu voi, sunt mo?tenitoare ale harului vie?ii, a?a încât rugaciunile voastre sa nu fie împiedicate.
\par 8 În sfâr?it, fi?i to?i într-un gând, împreuna-patimitori, iubitori de fra?i, milostivi, smeri?i.
\par 9 Nu rasplati?i raul cu rau sau ocara cu ocara, ci, dimpotriva, binecuvânta?i, caci spre aceasta a?i fost chema?i, ca sa mo?teni?i binecuvântarea.
\par 10 Cel ce voie?te sa iubeasca via?a ?i sa vada zile bune sa-?i opreasca limba de la rau ?i buzele sale sa nu graiasca vicle?ug;
\par 11 Sa se fereasca de rau ?i sa faca bine; sa caute pacea ?i s-o urmeze;
\par 12 Caci ochii Domnului sunt peste cei drep?i ?i urechile Lui spre rugaciunile lor, iar fa?a Domnului este împotriva celor ce fac rele.
\par 13 ?i cine va va face voua rau, daca sunte?i plini de râvna pentru bine?
\par 14 Dar de ve?i ?i patimi pentru dreptate, ferici?i ve?i fi. Iar de frica lor sa nu va teme?i, nici sa va tulbura?i.
\par 15 Ci pe Domnul, pe Hristos, sa-L sfin?i?i în inimile voastre ?i sa fi?i gata totdeauna sa raspunde?i oricui va cere socoteala despre nadejdea voastra,
\par 16 Dar cu blânde?e ?i cu frica, având cuget curat, ca, tocmai în ceea ce sunte?i cleveti?i, sa iasa de ru?ine cei ce graiesc de rau purtarea voastra cea buna întru Hristos.
\par 17 Caci e mai bine, daca a?a este voia lui Dumnezeu, sa patimi?i facând cele bune, decât facând cele rele.
\par 18 Pentru ca ?i Hristos a suferit odata moartea pentru pacatele noastre, El cel drept pentru cei nedrep?i, ca sa ne aduca pe noi la Dumnezeu, omorât fiind cu trupul, dar viu facut cu duhul,
\par 19 Cu care S-a coborât ?i a propovaduit ?i duhurilor ?inute în închisoare,
\par 20 Care fusesera neascultatoare altadata, când îndelunga-rabdare a lui Dumnezeu a?tepta, în zilele lui Noe, ?i se pregatea corabia în care pu?ine suflete, adica opt, s-au mântuit prin apa.
\par 21 Iar aceasta mântuire prin apa închipuia botezul, care va mântuie?te astazi ?i pe voi, nu ca ?tergere a necura?iei trupului, ci ca deschiderea cugetului bun catre Dumnezeu, prin învierea lui Iisus Hristos,
\par 22 Care, dupa ce S-a suit la cer, este de-a dreapta lui Dumnezeu, ?i se supun Lui îngerii ?i stapâniile ?i puterile.

\chapter{4}

\par 1 A?adar, fiindca Hristos a patimit cu trupul, înarma?i-va ?i voi cu gândul acesta: ca cine a suferit cu trupul a ispravit cu pacatul,
\par 2 Ca sa nu mai traiasca timpul ce mai are de trait în trup dupa poftele oamenilor, ci dupa voia lui Dumnezeu.
\par 3 Destul este ca, în vremurile trecute, a?i facut cu desavâr?ire voia neamurilor, umblând în desfrânari, în pofte, în be?ii, în ospe?e fara masura, în petreceri cu vin mult ?i în neiertate slujiri idole?ti.
\par 4 De aceea ei se mira ca voi nu mai alerga?i cu ei în aceea?i revarsare a desfrâului ?i va hulesc.
\par 5 Ei î?i vor da seama înaintea Celui ce este gata sa judece viii ?i mor?ii.
\par 6 Ca spre aceasta s-a binevestit mor?ilor, ca sa fie judeca?i ca oameni, dupa trup, dar sa vieze, dupa Dumnezeu cu duhul.
\par 7 Iar sfâr?itul tuturor s-a apropiat; fi?i dar cu mintea întreaga ?i priveghea?i în rugaciuni.
\par 8 Dar mai presus de toate, ?ine?i din rasputeri la dragostea dintre voi, pentru ca dragostea acopera mul?ime de pacate.
\par 9 Fi?i, între voi, iubitori de straini, fara cârtire.
\par 10 Dupa darul pe care l-a primit fiecare, sluji?i unii altora, ca ni?te buni iconomi ai harului celui de multe feluri al lui Dumnezeu.
\par 11 Daca vorbe?te cineva, cuvintele lui sa fie ca ale lui Dumnezeu; daca sluje?te cineva, slujba lui sa fie ca din puterea pe care o da Dumnezeu, pentru ca întru toate Dumnezeu sa se slaveasca prin Iisus Hristos, Caruia Îi este slava ?i stapânirea în vecii vecilor. Amin.
\par 12 Iubi?ilor, nu va mira?i de focul aprins între voi spre ispitire, ca ?i cum vi s-ar întâmpla ceva strain,
\par 13 Ci, întrucât sunte?i parta?i la suferin?ele lui Hristos, bucura?i-va, pentru ca ?i la aratarea slavei Lui sa va bucura?i cu bucurie mare.
\par 14 De sunte?i ocarâ?i pentru numele lui Hristos, ferici?i sunte?i, caci Duhul slavei ?i al lui Dumnezeu Se odihne?te peste voi; de catre unii El se hule?te, iar de voi se preaslave?te.
\par 15 Nimeni dintre voi sa nu sufere ca uciga?, sau fur, sau facator de rele, sau ca un râvnitor de lucruri straine.
\par 16 Iar de sufera ca cre?tin, sa nu se ru?ineze, ci sa preamareasca pe Dumnezeu, pentru numele acesta.
\par 17 Caci vremea este ca sa înceapa judecata de la casa lui Dumnezeu; ?i daca începe întâi de la noi, care va fi sfâr?itul celor care nu asculta de Evanghelia lui Dumnezeu?
\par 18 ?i daca dreptul abia se mântuie?te, ce va fi cu cel necredincios ?i pacatos?
\par 19 Pentru aceea, ?i cei ce sufera, dupa voia lui Dumnezeu, sa-?i încredin?eze Lui, credinciosului Ziditor, sufletele lor, savâr?ind fapte bune.

\chapter{5}

\par 1 Pe preo?ii cei dintre voi îi rog ca unul ce sunt împreuna-preot ?i martor al patimilor lui Hristos ?i parta? al slavei celei ce va sa se descopere:
\par 2 Pastori?i turma lui Dumnezeu, data în paza voastra, cercetând-o, nu cu silnicie, ci cu voie buna, dupa Dumnezeu, nu pentru câ?tig urât, ci din dragoste;
\par 3 Nu ca ?i cum a?i fi stapâni peste Biserici, ci pilde facându-va turmei.
\par 4 Iar când Se va arata Mai-marele pastorilor, ve?i lua cununa cea neve?tejita a maririi.
\par 5 Tot a?a ?i voi, fiilor duhovnice?ti, supune?i-va preo?ilor; ?i to?i, unii fa?a de al?ii, îmbraca?i-va întru smerenie, pentru ca Dumnezeu celor mândri le sta împotriva, iar celor smeri?i le da har.
\par 6 Deci, smeri?i-va sub mâna cea tare a lui Dumnezeu, ca El sa va înal?e la timpul cuvenit.
\par 7 Lasa?i-I Lui toata grija voastra, caci El are grija de voi.
\par 8 Fi?i treji, priveghea?i. Potrivnicul vostru, diavolul, umbla, racnind ca un leu, cautând pe cine sa înghita,
\par 9 Caruia sta?i împotriva, tari în credin?a, ?tiind ca acelea?i suferin?e îndura ?i fra?ii vo?tri în lume.
\par 10 Iar Dumnezeul a tot harul, Care v-a chemat la slava Sa cea ve?nica, întru Hristos Iisus, El însu?i, dupa ce ve?i suferi pu?ina vreme, va va duce la desavâr?ire, va va întari, va va împuternici, va va face neclinti?i.
\par 11 A Lui fie slava ?i puterea în vecii vecilor. Amin!
\par 12 V-am scris aceste pu?ine lucruri, prin Silvan, pe care îl socotesc frate credincios, ca sa va îndemn ?i sa va marturisesc ca adevaratul har al lui Dumnezeu este acesta, în care sta?i.
\par 13 Biserica cea aleasa din Babilon ?i Marcu, fiul meu, va îmbra?i?eaza.
\par 14 Îmbra?i?a?i-va unul pe altul cu sarutarea dragostei. Pace voua tuturor, celor întru Hristos Iisus. Amin.


\end{document}