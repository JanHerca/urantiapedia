\begin{document}

\title{Exod}


\chapter{1}

\par 1 Numele fiilor lui Israel, care au intrat în Egipt împreună cu Iacov, tatăl lor, aducând fiecare toată casa sa, sunt acestea:
\par 2 Ruben, Simeon, Levi și Iuda;
\par 3 Isahar, Zabulon și Veniamin;
\par 4 Dan, Neftali, Gad și Așer.
\par 5 Sufletele însă ieșite din Iacov erau de toate șaptezeci și cinci, iar Iosif era de mai înainte în Egipt.
\par 6 Dar au murit și Iosif și toți frații lui și toți cei de pe vremea lor.
\par 7 Iar fiii lui Israel s-au născut în număr mare și s-au înmulțit, au crescut și s-au întărit foarte, foarte tare, și s-a umplut țara de ei.
\par 8 Dar s-a ridicat alt rege peste Egipt, care nu cunoscuse pe Iosif.
\par 9 Acesta a zis către poporul său: "Iată, neamul fiilor lui Israel e mulțime mare și e mai tare decât noi.
\par 10 Veniți dar să-i împilăm, ca să nu se mai înmulțească și ca nu cumva la vreme de război să se unească cu vrăjmașii noștri și, bătându-ne, să iasă din țara noastră!"
\par 11 De aceea au pus peste ei supraveghetori de lucrări, ca să-i împileze cu munci grele. Atunci a zidit Israel cetăți tari lui Faraon: Pitom și Ramses, care serveau lui Faraon ca hambare, și cetatea On sau Iliopolis.
\par 12 Însă cu cât îi împilau mai mult, cu atât mai mult se înmulțeau și se întăreau foarte, foarte tare, așa că Egiptenii se îngrozeau de fiii lui Israel.
\par 13 De aceea Egiptenii sileau încă și mai strașnic la muncă pe fiii lui Israel
\par 14 Și le făceau viața amară prin munci grele, la lut, la cărămidă și la tot felul de lucru de câmp și prin alte felurite munci, la care-i sileau cu strășnicie.
\par 15 Ba, regele Egiptului a poruncit moașelor evreiești, care se numeau: una Șifra și alta Pua,
\par 16 Și le-a zis: "Când moșiți la evreice, să luați seama când nasc: de va fi băiat, să-l omorâți, iar de va fi fată, să o cruțați!"
\par 17 Moașele însă s-au temut de Dumnezeu și n-au făcut cum le poruncise regele Egiptului, ci au lăsat și pe băieți să trăiască.
\par 18 Atunci a chemat regele Egiptului pe moașe și le-a zis: "pentru ce ați făcut așa și ați lăsat să trăiască și copiii de parte bărbătească?"
\par 19 Iar moașele au răspuns lui Faraon: "Femeile evreice nu sunt ca egiptencele, ci ele sunt voinice și nasc până nu vin moașele la ele".
\par 20 De aceea Dumnezeu a făcut bine moașelor, iar poporul lui Israel se înmulțea și se întărea mereu.
\par 21 Și fiindcă moașele se temeau de Dumnezeu, de aceea El le-a întărit neamul.
\par 22 Atunci Faraon a poruncit la tot poporul său și a zis: "Tot copilul de parte bărbătească, ce se va naște Evreilor, să-l aruncați în Nil, iar fetele să le lăsați să trăiască toate!"

\chapter{2}

\par 1 Un om oarecare, din seminția lui Levi, și-a luat femeie din fetele lui Levi.
\par 2 Femeia aceea a luat în pântece și a născut un băiat și, văzând că e frumos, l-a ascuns vreme de trei luni.
\par 3 Dar, fiindcă nu putea să-l mai dosească, a luat mama lui un coș de papură și l-a uns cu catran și cu smoală și punând copilul în el l-a așezat în păpuriș, la marginea râului.
\par 4 Iar sora copilului pândea de departe ca să vadă ce are să i se întâmple.
\par 5 Atunci s-a pogorât fata lui Faraon la râu să se scalde, și roabele ei o însoțiră pe malul râului. Și văzând coșul în păpuriș, ea a trimis pe una din roabele sale să-l aducă.
\par 6 Și, deschizându-l, a văzut copilul: era un băiat care plângea. Atunci i s-a făcut milă de el fetei lui Faraon și a zis: "Acesta este dintre copiii Evreilor".
\par 7 Iar sora copilului a zis către fata lui Faraon: "Voiești să mă duc să-ți chem o doică dintre evreice, ca să alăpteze copilul?"
\par 8 Fata lui Faraon i-a zis: "Du-te!" Și s-a dus copila și a chemat pe mama pruncului.
\par 9 Atunci fata lui Faraon i-a zis: "Ia-mi copilul acesta și mi-l alăptează, că eu am să-ți plătesc! " și a luat femeia copilul și l-a alăptat.
\par 10 După ce a crescut copilul, doica l-a dus la fata lui Faraon și i-a fost ca fiu și i-a pus numele Moise, pentru că își zicea: "Din apă l-am scos!"
\par 11 Iar după multă vreme, când se făcuse mare, Moise a ieșit la fiii lui Israel, frații săi, și a văzut muncile lor cele grele. Cu prilejul acesta a văzut el pe un egiptean că bătea pe un evreu dintre fiii lui Israel, frații săi;
\par 12 Și căutând încoace și încolo și nevăzând pe nimeni, el a ucis pe egiptean și l-a ascuns în nisip.
\par 13 Apoi ieșind iarăși a doua zi, a văzut doi evrei certându-se și a zis asupritorului: "pentru ce bați pe aproapele tău?"
\par 14 Acela însă i-a răspuns: "Cine te-a pus căpetenie și judecător peste noi? Nu cumva vrei să mă omori și pe mine, cum ai omorât ieri pe egipteanul acela?" Și s-a spăimântat Moise și a zis: "Cu adevărat s-a vădit fapta aceasta!"
\par 15 Iar dacă a aflat Faraon de fapta aceasta, el a voit să ucidă pe Moise. Moise însă a fugit de la fața lui Faraon și s-a dus în țara Madian; și sosind în țara Madian, s-a oprit la o fântână.
\par 16 Preotul din Madian însă avea șapte fete, care pășteau oile tatălui lor. Și venind acestea au scos apă și au umplut adăpătorile, ca să adape oile tatălui lor.
\par 17 Dar păstorii venind, le-au alungat. Atunci s-a sculat Moise și le-a apărat, le-a scos apă și le-a adăpat oile.
\par 18 Mergând ele la tatăl lor Raguel, acesta le-a zis: "Cum de ați venit astăzi așa de curând?"
\par 19 Iar ele au zis: "Un egiptean oarecare ne-a apărat de păstori, ne-a scos apă și ne-a adăpat oile noastre!"
\par 20 Zis-a acela către fiicele sale: "Dar unde este acela? De ce l-ați lăsat? Chemați-l și dați-i să mănânce pâine!"
\par 21 Și a rămas Moise la omul acela și i-a dat pe fiica sa Sefora de soție.
\par 22 Aceasta, luând în pântece, a născut un fiu și i-a pus Moise numele Gherșon, zicând: "Am ajuns pribeag în țară străină". Și luând iarăși în pântece, femeia a născut alt fiu și i-a pus numele Eliezer, pentru că și-a zis: "Dumnezeul tatălui meu mi-a fost ajutor și m-a scăpat din mâna lui Faraon".
\par 23 Apoi, după trecere de vreme multă, a murit regele Egiptului, de care fugise Moise. Fiii lui Israel însă gemeau sub povara muncilor și strigau și strigarea lor din muncă s-a suit până la Dumnezeu.
\par 24 Auzind suspinele lor, Dumnezeu Și-a adus aminte de legământul Său pe care îl făcuse cu Avraam, cu Isaac și cu Iacov.
\par 25 De aceea a căutat Dumnezeu spre fiii lui Israel și S-a gândit la ei.

\chapter{3}

\par 1 În vremea aceea, Moise păștea oile lui Ietro, preotul din Madian, socrul său. Și depărtându-se odată cu turma în pustie, a ajuns până la muntele lui Dumnezeu, la Horeb;
\par 2 Iar acolo i S-a arătat îngerul Domnului într-o pară de foc, ce ieșea dintr-un rug; și a văzut că rugul ardea, dar nu se mistuia.
\par 3 Atunci Moise și-a zis: "Mă duc să văd această arătare minunată: că rugul nu se mistuiește".
\par 4 Iar dacă a văzut Domnul că se apropie să privească, a strigat la el Domnul din rug și a zis: "Moise! Moise!". Și el a răspuns: "Iată-mă, Doamne!"
\par 5 Și Domnul a zis: "Nu te apropia aici! Ci scoate-ți încălțămintea din picioarele tale, că locul pe care calci este pământ sfânt!"
\par 6 Apoi i-a zis iarăși: "Eu sunt Dumnezeul tatălui tău, Dumnezeul lui Avraam și Dumnezeul lui Isaac și Dumnezeul lui Iacov!" Și și-a acoperit Moise fața sa, că se temea să privească pe Dumnezeu.
\par 7 Zis-a Domnul către Moise: "Am văzut necazul poporului Meu în Egipt și strigarea lui de sub apăsători am auzit și durerea lui o știu.
\par 8 M-am pogorât dar să-l izbăvesc din mâna Egiptenilor, să-l scot din țara aceasta și să-l duc într-un pământ roditor și larg, în țara unde curge miere și lapte, în ținutul Canaaneilor, al Heteilor, al Amoreilor, al Ferezeilor, al Ghergheseilor, al Heveilor și al Iebuseilor.
\par 9 Iată dar că strigarea fiilor lui Israel a ajuns acum până la Mine și am văzut chinurile lor, cu care-i pedepsesc Egiptenii.
\par 10 Vino dar să te trimit la Faraon, regele Egiptului, ca să scoți pe fiii lui Israel, poporul Meu, din țara Egiptului!"
\par 11 Atunci a zis Moise către Dumnezeu: "Cine sunt eu, ca să mă duc la Faraon, regele Egiptului, și să scot pe fiii lui Israel din țara Egiptului?"
\par 12 Iar Dumnezeu i-a zis: "Eu voi fi cu tine și acesta îți va fi semnul că te trimit Eu: când vei scoate pe poporul Meu din țara Egiptului, vă veți închina lui Dumnezeu în muntele acesta!"
\par 13 Zis-a iarăși Moise către Dumnezeu: "Iată, eu mă voi duce la fiii lui Israel și le voi zice: Dumnezeul părinților voștri m-a trimis la voi... Dar de-mi vor zice: Cum Îl cheamă, ce să le spun?"
\par 14 Atunci Dumnezeu a răspuns lui Moise: "Eu sunt Cel ce sunt". Apoi i-a zis: "Așa să spui fiilor lui Israel: Cel ce este m-a trimis la voi!"
\par 15 Apoi a zis Dumnezeu iarăși către Moise: "Așa să spui fiilor lui Israel: "Domnul Dumnezeul părinților noștri, Dumnezeul lui Avraam, Dumnezeul lui Isaac și Dumnezeul lui Iacov m-a trimis la voi. Acesta este numele Meu pe veci; aceasta este pomenirea Mea din neam în neam!"
\par 16 Mergând dar, adună pe bătrânii fiilor lui Israel și le spune: "Domnul Dumnezeul părinților noștri, Dumnezeul lui Avraam, Dumnezeul lui Isaac și Dumnezeul lui Iacov mi S-a arătat și a zis: V-am cercetat de aproape și am văzut câte vi se întâmplă în Egipt!"
\par 17 Și mi-a mai zis: "Vă voi scoate din împilarea Egiptului și vă voi duce în pământul Canaaneilor, al Heteilor, al Amoreilor, al Ferezeilor, al Ghergheseilor, al Heveilor și al Iebuseilor, în pământul unde curge miere și lapte".
\par 18 Iar ei vor asculta glasul tău. Atunci vei intra tu și bătrânii lui Israel la Faraon, regele Egiptului, și-i veți zice: "Domnul Dumnezeul Evreilor ne-a chemat. Lasă-ne dar să mergem în pustie, cale de trei zile, ca să aducem jertfă Dumnezeului nostru".
\par 19 Eu însă știu că Faraon, regele Egiptului, nu are să vă lase să plecați, până nu îl voi sili Eu cu mână tare.
\par 20 Voi întinde deci mâna Mea și voi lovi Egiptul cu toate minunile, pe care le voi face în mijlocul lui, și după aceea vă va lăsa.
\par 21 Voi da poporului acestuia trecere înaintea Egiptenilor și când veți ieși, nu veți ieși cu mâinile goale,
\par 22 Ci fiecare femeie va cere la vecina sa și de la cea care stă cu ea în casă vase de argint, lucruri de aur și haine și veți împodobi cu ele pe fiii voștri și pe fetele voastre și veți prăda pe Egipteni".

\chapter{4}

\par 1 Și răspunzând, Moise a zis: "Dar de nu mă vor crede și nu vor asculta de glasul meu, ci vor zice: "Nu ți S-a arătat Domnul!", ce să le spun?"
\par 2 Zis-a Domnul către el: "Ce ai în mână?" Și el a răspuns: "Un toiag!"
\par 3 "Aruncă-l jos!" îi zise Domnul. Și a aruncat Moise toiagul jos și s-a făcut toiagul șarpe și a fugit Moise de el.
\par 4 Și a zis Domnul către Moise: "Întinde mâna și-l apucă de coadă!" Și și-a întins Moise mâna și l-a apucat de coadă și s-a făcut toiag în mâna lui.
\par 5 Apoi a zis Domnul: "Așa să faci înaintea lor, ca să te creadă că ți S-a arătat Dumnezeul părinților lor, Dumnezeul lui Avraam, Dumnezeul lui Isaac și Dumnezeul lui Iacov!"
\par 6 Zis-a Domnul iarăși: "Bagă-ți mâna în sân!" Și când a scos-o din sân, iată mâna lui era albă ca zăpada de lepră.
\par 7 Și i-a zis din nou Domnul: "Bagă-ți iarăși mina în sân!" Și și-a băgat Moise mâna în sân; și când a scos-o din sân, iată, era iar curată, ca tot trupul său.
\par 8 "Dacă nu te vor crede și nu vor asculta glasul semnului întâi, te vor crede la săvârșirea semnului al doilea.
\par 9 Iar de nu te vor crede nici după amândouă semnele și nu vor asculta glasul tău, atunci să iei apă din fluviu și s-o verși pe uscat; apa luată din râu se va face pe usca: sânge".
\par 10 Atunci Moise a zis către Domnul: "O, Doamne, eu nu sunt om îndemânatic la vorbă, ci grăiesc cu anevoie și sunt gângav; și aceasta nu de ieri de alaltăieri, nici de când ai început Tu a grăi cu robul Tău; gura mea și limba mea sunt anevoioase".
\par 11 Dumnezeu însă a zis către Moise: "Cine a dat omului gură și cine face pe om mut, sau surd, sau cu vedere, sau orb? Oare nu Eu, Domnul Dumnezeu?
\par 12 Mergi dar: Eu voi deschide gura ta și te voi învăța ce să grăiești".
\par 13 Zis-a Moise: "Rogu-mă, Doamne, trimite pe altul, pe care vei vrea să-l trimiți!"
\par 14 Atunci, aprinzându-se mânia Domnului asupra lui Moise, a zis: "Nu ai tu, oare, pe fratele tău Aaron levitul? Știu că el poate să vorbească în locul tău. Iată el te va întâmpina și, când te va vedea, se va bucura în inima sa.
\par 15 Tu-i vei grăi lui și îi vei pune în gură cuvintele Mele, iar Eu voi deschide gura ta și gura lui și vă voi învăța ce să faceți.
\par 16 Va grăi el poporului, în locul tău, vorbind pentru tine, iar tu îi vei fi grăitor din partea lui Dumnezeu.
\par 17 Toiagul acesta, care a fost prefăcut în șarpe, ia-l în mâna ta, căci cu el ai să faci minuni".
\par 18 Deci, a plecat Moise de acolo și s-a întors la Ietro, socrul său, și a zis către el: "Mă duc înapoi la frații mei, care sunt în Egipt, ca să văd de mai trăiesc". Iar Ietro i-a zis: "Mergi în pace!"
\par 19 După atât de multe zile a murit regele Egiptului care prigonise pe Moise, și Domnul a grăit aceasta lui Moise în pământul Madian: "Scoală și întoarce-te în Egipt, că au murit toți cei ce căutau sufletul tău!"
\par 20 Luând atunci femeia și copiii, Moise i-a pus pe asini și s-a întors în Egipt. Și a luat Moise în mâna sa și toiagul cel de la Dumnezeu.
\par 21 Și a zis Domnul către Moise: "Când vei merge și vei ajunge în țara Egiptului, caută să faci înaintea lui Faraon toate minunile ce ți-am poruncit. Eu însă voi învârtoșa inima lui și nu va da drumul poporului.
\par 22 Dar tu să zici lui Faraon: Așa zice Domnul Dumnezeul Evreilor: Israel este fiul Meu, întâi-născutul Meu.
\par 23 Îți zic dar: Lasă pe fiul Meu să Mi se închine; iar de nu-l vei lăsa, iată, îți voi ucide pe fiul tău cel întâi-născut".
\par 24 Însă, la un popas de noapte, pe cale, l-a întâmpinat îngerul Domnului și a încercat să-l omoare.
\par 25 Dar Sefora, luând un cuțit de piatră, a tăiat împrejur pe fiul său și atingând picioarele lui Moise a zis; "Tu-mi ești un soț crud!"
\par 26 Și S-a dus Domnul de la el; iar ea, din pricina acestei tăieri împrejur, i-a zis lui Moise: "Soț crud!"
\par 27 Atunci a zis Domnul către Aaron: "Mergi în întâmpinarea lui Moise în pustie!" Și s-a dus acesta și s-a întâlnit cu el în muntele lui Dumnezeu și s-au sărutat amândoi.
\par 28 Atunci a spus Moise lui Aaron toate cuvintele Domnului, pe care i le poruncise, și toate semnele ce-i încredințase să facă.
\par 29 După aceea s-au dus Moise și Aaron și au adunat pe toți bătrânii fiilor lui Israel
\par 30 Și le-a spus Aaron toate cuvintele pe care le grăise Domnul lui Moise, și a făcut Moise semne înaintea poporului;
\par 31 Și poporul a crezut și s-a bucurat că a cercetat Domnul pe fiii lui Israel și a văzut necazurile lor și, plecându-se, s-au închinat.

\chapter{5}

\par 1 După aceea Moise și Aaron au intrat la Faraon și au zis către dânsul: "Așa grăiește Domnul Dumnezeul lui Israel: Lasă pe poporul Meu, ca să-Mi facă sărbătoare în pustie!"
\par 2 Faraon însă a zis: "Cine este acela Domnul, ca să-I ascult glasul și să dau drumul fiilor lui Israel? Nu-L cunosc pe Domnul și nu voi da drumul lui Israel!"
\par 3 Zis-au ei către dânsul: "Dumnezeul Evreilor ne-a chemat; lasă-ne să mergem în pustie cale de trei zile, ca să aducem jertfă Domnului Dumnezeului nostru, ca să nu pierim de ciumă sau de sabie!"
\par 4 Iar regele Egiptului le-a zis: "Moise și Aaron, pentru ce-mi stingheriți poporul de la lucru? Duceți-vă fiecare din voi la treburile voastre!"
\par 5 Apoi Faraon a zis iar: "Iată acum s-a înmulțit poporul acesta în țară și voi îl întrerupeți de la lucru".
\par 6 Și chiar în ziua aceea a poruncit Faraon căpeteniilor și slujbașilor poporului și le-a zis:
\par 7 "De acum înainte să nu mai dați poporului paie pentru facerea cărămizii, ca ieri și ca alaltăieri, ci să se ducă ei să-și adune paie.
\par 8 Dar cărămizi să facă tot atâtea câte făceau în fiecare zi; să-i siliți și să nu le împuținați munca; fiindcă sunt fără treabă și de aceea strigă și zic: Haidem să aducem jertfă Dumnezeului nostru!
\par 9 Să fie dar împovărați de lucru oamenii aceștia și să se îndeletnicească cu acestea, iar nu să se îndeletnicească cu vorbe mincinoase!"
\par 10 Și au ieșit căpeteniile lor și slujbașii poporului și au zis către popor: "Așa zice Faraon: Nu vă mai dau paie.
\par 11 Mergeți voi înșivă și vă adunați paie de unde veți găsi, dar din lucrul vostru nu vi se va scădea nimic!"
\par 12 Atunci s-a risipit poporul în tot Egiptul, ca să strângă trestie în loc de paie.
\par 13 Iar slujbașii îi sileau, zicând: "Împliniți-vă lucrul dat pentru fiecare zi, ca atunci când vi se dădea paie".
\par 14 Iar pe slujbașii puși peste ei de căpeteniile lui Faraon îi băteau, zicând: "Pentru ce n-ați făcut și astăzi numărul vostru de cărămizi, ca ieri și ca alaltăieri?"
\par 15 Atunci s-au dus slujbașii fiilor lui Israel și au strigat către Faraon, zicând: "Pentru ce faci așa cu robii tăi?
\par 16 Paie nu se dau robilor tăi, dar ne zic: Faceți cărămidă. Și robii tăi sunt bătuți și poporul tău e mereu vinovat".
\par 17 Iar el le-a zis: "Sunteți leneși și de aceea ziceți: Haidem să aducem jertfă Dumnezeului nostru.
\par 18 Acum duceți-vă dar și munciți! Paie nu vi se vor da, dar numărul de cărămizi rânduit să-l faceți!"
\par 19 Deci au văzut slujbașii fiilor lui Israel pacostea căzută peste ei, când le spunea: "Numărul de cărămizi lucrate pentru fiecare zi nu vi se va împuțina".
\par 20 Și, ieșind de la Faraon, s-au întâlnit ei cu Moise și cu Aaron, care veneau înaintea lor
\par 21 Și au zis către ei: "Să vă vadă și să vă judece Dumnezeu, că ne-ați făcut urâți înaintea lui Faraon și a slujitorilor lui și le-ați dat sabie la mână, ca să ne omoare".
\par 22 Atunci Moise s-a întors la Domnul și a zis: "Doamne, pentru ce ai adus necazul acesta asupra acestui popor și la ce m-ai trimis pe mine?
\par 23 Căci de când am mers eu la Faraon și i-am grăit în numele Tău, el a început să necăjească și mai rău pe poporul acesta, iar de izbăvit Tu nu l-ai izbăvit pe poporul Tău".

\chapter{6}

\par 1 Atunci a zis Domnul către Moise: "Acum ai să vezi ce am să fac lui Faraon, că sub lucrarea mâinii Mele celei tari el îi va lăsa și sub lucrarea brațului Meu celui puternic el îi va alunga din pământul său".
\par 2 Apoi a grăit Domnul cu Moise și a zis către el: "Eu sunt Domnul
\par 3 Și M-am arătat lui Avraam, lui Isaac și lui Iacov ca Dumnezeu Atotputernic, iar cu numele Meu de Domnul nu M-am făcut cunoscut lor.
\par 4 Mai mult: am făcut legământ cu ei, ca să le dau pământul Canaan, pământul pribegiei lor, în care rătăceau ei.
\par 5 Și, în sfârșit, am auzit suspinul fiilor lui Israel, pe care îi țin Egiptenii în robie, și Mi-am adus aminte de legământul Meu cu voi.
\par 6 Mergi dar de vorbește fiilor lui Israel și le spune: Eu sunt Domnul și am să vă scot de la munca cea grea a Egiptenilor și am să vă izbăvesc din robia lor; am să vă izbăvesc cu braț înalt și cu pedepse mari;
\par 7 Am să vă primesc să-Mi fiți popor, iar Eu să vă fiu Dumnezeu și voi veți cunoaște că Eu sunt Domnul Dumnezeul vostru, Care v-a scos din pământul Egiptului și de sub munca apăsătoare a Egiptenilor.
\par 8 Apoi am să vă duc în pământul acela pentru care Mi-am ridicat mâna să-l dau lui Avraam, lui Isaac și lui Iacov, și pe care am să-l dau vouă în stăpânire, căci Eu sunt Domnul!"
\par 9 Și a grăit Moise așa fiilor lui Israel; dar ei n-au ascultat pe Moise din pricina deznădejdii și a greutății muncilor lor.
\par 10 Și iarăși a grăit Domnul cu Moise și i-a zis:
\par 11 "Intră și spune lui Faraon, regele Egiptului, ca să lase pe fiii lui Israel să iasă din țara lui!"
\par 12 Dar Moise a grăit înaintea Domnului și a zis: "Iată, fiii lui Israel nu mă ascultă. Cum, dar, mă va asculta Faraon? Și apoi eu sunt și gângav".
\par 13 Domnul însă a grăit lui Moise și Aaron și le-a poruncit să spună lui Faraon, regele Egiptului, să dea drumul fiilor lui Israel din țara Egiptului.
\par 14 Iată acum începătorii familiilor strămoșești: Fiii lui Ruben, întâi-născutul lui Israel: Enoh și Falu, Hețron și Carmi. Acestea sunt familiile lui Ruben.
\par 15 Fiii lui Simeon: Iemuel și Iamin, Ohad și Iachin, Țohar și Saul, fiii canaaneencii. Acestea sunt familiile lui Simeon.
\par 16 Iar numele fiilor lui Levi, înșirați cum s-au născut, sunt acestea: Gherșon, Cahat și Merari. Iar anii vieții lui Levi au fost o sută treizeci și șapte.
\par 17 Fiii lui Gherșon: Libni și Șimei, cu familiile lor.
\par 18 Fiii lui Cahat: Amram, Ițhar, Hebron și Uziel. Iar anii vieții lui Cahat au fost o sută treizeci și trei de ani.
\par 19 Fiii lui Merari: Mahli și Muși. Acesta este neamul lui Levi, după familiile lui.
\par 20 Amram însă a luat de femeie pe Iochebed, fata unchiului său, și aceasta i-a născut pe Aaron și pe Moise, precum și pe Mariam, sora lor. Iar anii vieții lui Amram au fost o sută treizeci și șapte de ani.
\par 21 Fiii lui Ițhar: Core, Nefeg și Zicri.
\par 22 Fiii lui Uziel: Misael, Elțafan și Sitri.
\par 23 Iar Aaron și-a luat de soție pe Elisaveta, fata lui Aminadab și sora lui Naason; aceasta i-a născut pe Nadab și pe Abiud, pe Eleazar și pe Itamar.
\par 24 Fiii lui Core: Asir, Elcana și Abiasaf. Acestea sunt familiile lui Core.
\par 25 Eleazar, fiul lui Aaron, și-a luat de femeie pe una din fiicele lui Putiel și aceasta i-a născut pe Finees. Aceștia sunt începătorii familiilor strămoșești ale leviților.
\par 26 Acesta este acel Aaron și acel Moise, cărora Dumnezeu le-a zis: "Scoateți pe fiii lui Israel din pământul Egiptului cu oștirea lor!"
\par 27 Și aceștia au spus lui Faraon, regele Egiptului, să dea drumul fiilor lui Israel din pământul Egiptului. Acesta este acel Aaron și acesta este acel Moise
\par 28 Din timpul când a grăit Domnul cu Moise în țara Egiptului
\par 29 Și i-a zis lui: "Eu sunt Domnul!" Spune lui Faraon, regele Egiptului, câte-ți vorbesc Eu!"
\par 30 Iar Moise a răspuns Domnului: "Iată eu sunt greoi la vorbă. Cum dar mă va asculta Faraon?"

\chapter{7}

\par 1 Răspuns-a Domnul lui Moise și i-a zis: "Iată, Eu fac din tine un dumnezeu pentru Faraon, iar Aaron, fratele tău, îți va fi prooroc.
\par 2 Tu dar vei grăi lui Aaron toate câte îți voi porunci, iar Aaron, fratele tău, va spune lui Faraon, ca să lase pe fiii lui Israel să iasă din pământul lui.
\par 3 Eu însă voi învârtoșa inima lui Faraon și voi arăta mulțimea semnelor Mele și a minunilor Mele în pământul Egiptului.
\par 4 Faraon nu vă va asculta, dar Eu îmi voi pune mâna asupra Egiptului și voi scoate oștirile Mele, pe poporul Meu, pe fiii lui Israel din pământul Egiptului, cu mare izbândă.
\par 5 Când voi întinde mâna Mea asupra Egiptului și voi scoate pe fiii lui Israel din mijlocul lui, atunci vor cunoaște toți Egiptenii că Eu sunt Domnul".
\par 6 Moise și Aaron s-au supus; cum le-a poruncit Domnul, așa au făcut.
\par 7 Când au început a grăi lui Faraon, Moise era de optzeci de ani, iar Aaron, fratele lui, de optzeci și trei de ani.
\par 8 Și a grăit Domnul cu Moise și cu Aaron și a zis:
\par 9 "Dacă Faraon vă va zice: Dă-ne vreun semn sau vreo minune, atunci tu să zici fratelui tău Aaron: Ia toiagul și-l aruncă jos înaintea lui Faraon și înaintea slujitorilor lui, și se va face șarpe".
\par 10 S-au dus deci Moise și Aaron la Faraon și la slujitorii lui și au făcut cum le poruncise Domnul: a aruncat Aaron toiagul său înaintea lui Faraon și înaintea slujitorilor lui și s-a făcut șarpe.
\par 11 Atunci a chemat și Faraon pe înțelepții Egiptului și pe vrăjitori și au făcut și vrăjitorii Egiptenilor asemenea lucru cu vrăjile lor:
\par 12 Fiecare din ei și-a aruncat toiagul și s-a făcut șarpe. Dar toiagul lui Aaron a înghițit toiegele lor.
\par 13 De aceea s-a învârtoșat inima lui Faraon și nu i-a ascultat, după cum spusese Domnul.
\par 14 Zis-a Domnul către Moise: "Inima lui Faraon se îndărătnicește și nu lasă poporul.
\par 15 Du-te dar la Faraon dimineață; iată el are să iasă la apă, iar tu să stai în calea lui, pe malul râului, și toiagul acesta, care s-a prefăcut în șarpe, să-l iei în mâna ta;
\par 16 Și să zici lui Faraon: Domnul, Dumnezeul Evreilor m-a trimis la tine să-ți spun: Lasă pe poporul Meu să-Mi facă slujbă în pustie; și iată până acum nu M-ai ascultat.
\par 17 Așa zice Domnul: Din aceasta vei cunoaște că Eu sunt Domnul: iată, cu acest toiag, care e în mâna mea, voi lovi apa din râu și se va preface în sânge;
\par 18 Peștele din râu va muri, râul se va împuți și Egiptenii nu vor putea bea apă din râu".
\par 19 Și a mai zis Domnul către Moise: "Să zici lui Aaron, fratele tău: Ia toiagul în mână și întinde-ți mâna asupra apelor Egiptului: asupra râurilor lui, asupra lacurilor lui și asupra oricărei adunări de apă; și se vor preface în sânge și va fi sânge în toată țara Egiptului, în vasele de lemn și în cele de piatră!"
\par 20 Și au făcut Moise și Aaron cum le-a poruncit Domnul: a ridicat Aaron toiagul său și a lovit apa râului, înaintea ochilor lui Faraon și înaintea ochilor slujitorilor lui, și toată apa din râu s-a prefăcut în sânge.
\par 21 Atunci peștele din râu a murit, râul s-a împuțit și Egiptenii nu puteau să bea apă din râu; și era sânge în toată țara Egiptului.
\par 22 Și au făcut așa și magii Egipteni cu vrăjile lor. De aceea s-a învârtoșat inima lui Faraon și nu i-a ascultat, precum le spusese Domnul.
\par 23 Întorcându-se, Faraon a intrat în casa sa și nu a pus la inimă aceasta.
\par 24 Atunci au săpat toți Egiptenii în preajma râului, ca să găsească să bea apă, căci din râu nu puteau să bea apă.
\par 25 Se împliniseră șapte zile de când lovise Domnul apa.

\chapter{8}

\par 1 Atunci a zis Domnul către Moise: "Intră la Faraon și-i zi: Așa grăiește Domnul: Lasă pe poporul Meu, ca să-Mi slujească.
\par 2 Iar de nu vei vrea să-l lași, iată Eu voi lovi toate ținuturile tale cu broaște.
\par 3 Râul va mișuna de broaște și, ieșind, acestea se vor sui în casele tale, în dormitoarele tale, pe paturile tale, în casele slujitorilor tăi și ale poporului tău, în cuptoarele tale și în aluaturile tale;
\par 4 Pe tine, pe poporul tău și pe toate slugile tale se vor sui broaște".
\par 5 Și a mai zis Domnul către Moise: "Spune lui Aaron, fratele tău: întinde toiagul cu mâna ta spre râuri, spre lacuri și spre bălți și fă să iasă broaște în pământul Egiptului!"
\par 6 Și și-a întins Aaron mâna sa asupra apelor Egiptului și ele au scos broaște; și au ieșit broaște și au acoperit pământul Egiptului.
\par 7 Dar au făcut asemenea și vrăjitorii Egiptenilor cu vrăjile lor și au scos broaște în pământul Egiptului.
\par 8 Atunci a chemat Faraon pe Moise și pe Aaron și a zis: "Rugați-vă pentru mine Domnului să depărteze broaștele de la mine și de la poporul meu și voi lăsa poporul lui Israel să jertfească Domnului!"
\par 9 Moise însă a zis către Faraon: "Sorocește-mi însuți când să mă rog pentru tine, pentru slugile tale și pentru poporul tău, ca să piară broaștele de la tine, de la poporul tău și din casele voastre și să rămână numai în râu",
\par 10 Iar el a zis: "Mâine". Zis-a Moise: "Va fi cum ai zis, ca să știi că nu este altul ca Domnul Dumnezeul nostru.
\par 11 Se vor depărta broaștele de la tine, din casele tale, din țarine, de la slugile tale și de la poporul tău și numai în râu vor rămâne".
\par 12 Ieșind deci Moise și Aaron de la Faraon, a strigat Moise către Domnul ca să piară broaștele pe care le trimisese împotriva lui Faraon.
\par 13 Și a făcut Domnul după cuvântul lui Moise și au murit broaștele de prin case, de prin curți și ds prin țarini;
\par 14 Și le-au adunat grămezi, grămezi, și s-a împuțit pământul.
\par 15 Văzând însă că s-a făcut ușurare, Faraon și-a învârtoșat inima și nu i-a ascultat, după cum spusese Domnul.
\par 16 Atunci a zis Domnul către Moise: "Spune lui Aaron: Întinde-ți toiagul tău cu mâna și lovește țărâna pământului și vor fi țânțari pe oameni, pe vite, pe Faraon și în casa lui și pe slugile lui; toată țărâna pământului se va face țânțari în tot pământul Egiptului".
\par 17 Și au făcut ei așa: Aaron și-a întins toiagul cu mâna sa și a lovit țărâna pământului și s-au ivit țânțari pe oameni și pe vite. Toată țărâna pământului s-a făcut țânțari în tot pământul Egiptului.
\par 18 Au încercat atunci și magii cu vrăjile lor să facă țânțari, dar n-au putut. Și au rămas țânțari pe oameni și pe vite.
\par 19 Și au zis magii către Faraon: "Acesta e degetul lui Dumnezeu!" Dar inima lui Faraon s-a învârtoșat și nu i-a ascultat, după cum spusese Domnul.
\par 20 Zis-a Domnul către Moise: "Scoală mâine de dimineață și ieși înaintea lui Faraon în vremea când el are să iasă la apă, iar tu să-i zici: Așa grăiește Domnul: Lasă pe poporul Meu ca să-Mi slujească în pustie!
\par 21 Dacă însă nu vei lăsa pe poporul Meu, iată Eu voi trimite asupra ta, asupra slujitorilor tăi, asupra poporului tău și asupra caselor voastre tăuni și se vor umple casele Egiptenilor de tăuni și pământul pe care trăiesc ei.
\par 22 Și voi osebi în ziua aceea pământul Goșen în care locuiește poporul Meu, că acolo nu vor fi tăuni ca să știi că Eu sunt Domnul, în mijlocul acestei țări.
\par 23 Voi face deosebire între poporul Meu și poporul tău și chiar mâine va fi semnul acesta pe pământ".
\par 24 Și a făcut Domnul așa și a venit mulțime de tăuni în casa lui Faraon, în casele slujitorilor lui și în tot pământul Egiptului, încât s-a pustiit țara de tăuni.
\par 25 Atunci a chemat Faraon pe Moise și pe Aaron și a zis: "Mergeți și aduceți jertfă Domnului Dumnezeului vostru în țară!"
\par 26 Moise însă a zis: "Nu este cu putință să se facă așa, pentru că cele ce aducem noi jertfă Domnului Dumnezeului nostru sunt urâciune înaintea Egiptenilor. Și de vom jertfi noi înaintea Egiptenilor cele ce sunt urâciune pentru ei, nu ne vor ucide, oare, cu pietre?
\par 27 De aceea ne vom duce în pustie cale de trei zile și vom aduce acolo jertfă Domnului Dumnezeului nostru, după cum ne va zice Domnul".
\par 28 Zis-a Faraon: "Eu vă voi lăsa să aduceți jertfă Domnului Dumnezeului vostru, în pustie, dar să nu vă duceți departe. Rugați-vă dar Domnului pentru mine!"
\par 29 Iar Moise a zis: "Iată, cum voi ieși de la tine, mă voi ruga Domnului Dumnezeu și mâine se vor îndepărta tăunii de la Faraon, de la slujitorii lui și de la poporul lui, dar Faraon să înceteze a mai înșela, nedând drumul poporului să aducă jertfă Domnului!"
\par 30 Și ieșind Moise de la Faraon, s-a rugat lui Dumnezeu.
\par 31 Și a făcut Domnul după cum zisese Moise: a îndepărtat tăunii de la Faraon, de la slujitorii lui și de la poporul lui și n-a mai rămas nici unul.
\par 32 Dar Faraon și-a învârtoșat inima și de data aceasta și n-a lăsat poporul să se ducă.

\chapter{9}

\par 1 Atunci a zis Domnul către Moise: "Intră la Faraon și-i spune: Acestea zice Domnul Dumnezeul Evreilor: Lasă pe poporul Meu să-Mi slujească!
\par 2 Iar de nu vei vrea să lași pe poporul Meu, ci-l vei mai ține,
\par 3 Iată, mâna Domnului va fi peste vitele tale cele de la câmp: peste cai, peste asini, peste cămile, peste boi și oi și va fi moarte foarte mare.
\par 4 Dar va face Domnul osebire în ziua aceea între vitele Israeliților și vitele Egiptenilor: din toate vitele fiilor lui Israel nu va muri nici una".
\par 5 Și a pus Domnul soroc și a zis: "Mâine va face Domnul aceasta în țara aceasta!"
\par 6 Și a doua zi a făcut Domnul aceasta și au murit toate vitele Egiptenilor, iar din vitele fiilor lui Israel n-a murit nici una.
\par 7 Atunci a trimis Faraon să afle și iată din toate vitele fiilor lui Israel nu murise nici una. Dar inima lui Faraon s-a învârtoșat și nu a lăsat poporul să se ducă.
\par 8 Iarăși a grăit Domnul cu Moise și cu Aaron și a zis: "Luați-vă câte o mână plină de cenușă din cuptor și s-o arunce Moise spre cer înaintea lui Faraon și a slujitorilor lui.
\par 9 Și se va stârni pulbere în tot pământul Egiptului și vor fi pe oameni și pe vite răni și bășici usturătoare în toată țara Egiptului".
\par 10 Deci, au luat ei cenușă din cuptor, au mers înaintea lui Faraon, a aruncat-o Moise spre cer și s-au făcut bube cu puroi pe oameni și pe vite;
\par 11 Și magii n-au putut sta împotriva lui Moise din pricina rănilor, pentru că erau bube pe ei și în tot Egiptul.
\par 12 Dar Domnul a învârtoșat inima lui Faraon și nu i-a ascultat, cum zisese Domnul lui Moise.
\par 13 Zis-a Domnul către Moise: "Să te scoli mâine de dimineață, să te înfățișezi lui Faraon și să-i zici: Așa grăiește Domnul Dumnezeul Evreilor: Lasă pe poporul Meu ca să-Mi slujească,
\par 14 Fiindcă de data aceasta voi trimite toate pedepsele Mele împotriva ta, a slugilor tale și a poporului tău, ca să vezi că nu este altul asemenea Mie în tot pământul.
\par 15 De Mi-aș fi întins mâna și te-aș fi lovit pe tine și pe poporul tău cu ciumă, tu ai fi fost șters de pe fața pământului;
\par 16 Dar te-am cruțat, ca să-Mi arăt puterea Mea și ca să se vestească numele Meu în tot pământul,
\par 17 Și tu tot te mai împotrivești poporului Meu și nu-l lași.
\par 18 Iată, Eu voi ploua mâine, pe vremea asta, grindină foarte multă, cum n-a mai fost în Egipt de la întemeierea lui și până în ziua aceasta.
\par 19 Trimite dar acum să adune turmele tale și toate câte ai la câmp, că asupra tuturor oamenilor și vitelor, care vor fi în țarină și nu vor intra în casă, va cădea grindină și vor muri".
\par 20 Acei dintre robii lui Faraon, care s-au temut de Domnul, au adunat în grabă pe oamenii și turmele lor acasă,
\par 21 Iar cei ce n-au luat aminte la cuvântul Domnului, aceia și-au lăsat slugile și vitele lor în câmp.
\par 22 Și a zis Domnul către Moise: "Întinde mâna ta spre cer și va cădea grindină peste tot pământul Egiptului: peste oameni, peste turme și peste toată iarba câmpului din pământul Egiptului!"
\par 23 Atunci și-a întins Moise mâna spre cer și a slobozit Domnul tunete, grindină și foc pe pământ; și a plouat Domnul grindină în pământul Egiptului.
\par 24 Aceasta a fost o grindină foarte mare și printre grindină ardea foc, cum nu mai fusese în tot pământul Egiptului, de când se așezaseră oamenii pe el.
\par 25 Grindina aceasta a bătut în tot pământul Egiptului, tot ce era pe câmp, oameni și dobitoace; toată iarba câmpului a bătut-o grindina și toți pomii de pe câmp i-a rupt grindina.
\par 26 Numai în ținutul Goșen, unde trăiau fiii lui Israel, n-a fost grindină.
\par 27 Atunci trimițând, Faraon a chemat pe Moise și pe Aaron și a zis către ei: "Acum văd că am păcătuit! Domnul este drept, iar eu și poporul meu suntem vinovați.
\par 28 Rugați-vă Domnului pentru mine, să înceteze tunetele, grindina și focul pe pământ și vă voi lăsa și, mai mult, nu vă voi împiedica!"
\par 29 Iar Moise i-a zis: "Îndată ce voi ieși din oraș, voi întinde mâna mea spre cer, către Domnul, și vor înceta tunetele; nu va mai fi nici grindină, nici ploaie, ca să cunoști că al Domnului este pământul.
\par 30 Dar știu că tu și slujitorii tăi nu vă temeți încă de Domnul Dumnezeu".
\par 31 Atunci inul și orzul s-au stricat, pentru că orzul era înspicat și inul în floare.
\par 32 Iar grâul și ovăzul nu s-au stricat, pentru că acestea erau mai târzii.
\par 33 Ieșind deci Moise de la Faraon și din cetate și întinzându-și mâinile către Domnul, au încetat tunetele și grindina și s-a oprit ploaia.
\par 34 Văzând însă că au încetat ploaia, grindina și tunetele, păcătuit-a Faraon înainte și și-a învârtoșat inima și el și slugile sale.
\par 35 Și învârtoșată fiind inima lui Faraon și a slugilor lui, el n-a lăsat pe fiii lui Israel să plece cum poruncise Dumnezeu prin mâna lui Moise.

\chapter{10}

\par 1 Atunci a grăit iarăși Domnul cu Moise și a zis: "Intră la Faraon, că i-am învârtoșat inima lui și a slugilor lui, ca să arăt între ei pe rând aceste semne ale Mele;
\par 2 Ca să istorisiți în auzul fiilor voștri și al fiilor fiilor voștri câte am făcut în Egipt și semnele Mele, pe care le-am arătat într-însul, și ca să cunoașteți că Eu sunt Domnul!"
\par 3 Și a intrat Moise și Aaron la Faraon și i-au zis: "Așa grăiește Domnul Dumnezeul Evreilor: Până când nu vei vrea să te smerești înaintea Mea? Lasă pe poporul Meu, ca să-Mi slujească!
\par 4 Iar de nu vei lăsa pe poporul Meu, iată mâine, pe vremea asta, voi aduce lăcuste multe în toate hotarele tale;
\par 5 Și vor acoperi ele fața pământului, încât pământul nu se va putea vedea, și vor mânca tot ce a mai rămas la voi, pe pământ, nestricat de grindină; toți pomii ce cresc prin câmpiile voastre;
\par 6 Vor umplea casele tale, casele tuturor slugilor tale și toate casele în tot pământul Egiptenilor, cum n-au văzut părinții tăi, nici părinții părinților tăi de când trăiesc ei pe pământ și până în ziua de astăzi". Apoi s-a întors Moise și a ieșit de la Faraon.
\par 7 Atunci dregătorii lui Faraon au zis către acesta: "Oare mult are să ne chinuiască omul acesta? Dă drumul oamenilor acestora, ca să facă slujbă Dumnezeului lor! Sau vrei să vezi Egiptul pierind?"
\par 8 Și ei au întors pe Moise și pe Aaron la Faraon; iar Faraon a zis către ei: "Duceți-vă și faceți slujbă Domnului Dumnezeului vostru! Dar cine sunt cei care trebuie să meargă?"
\par 9 Răspuns-a Moise: "Vom merge cu cei tineri și cu cei bătrâni ai noștri, cu fiii noștri, cu fiicele noastre, cu oile noastre și cu boii noștri, căci e sărbătoarea Domnului Dumnezeului nostru".
\par 10 Faraon însă le-a zis: "Fie așa! Dumnezeu cu voi! Eu sunt gata să vă dau drumul. Dar la ce să vă duceți cu copiii? Se vede că aveți gând rău.
\par 11 Nu! Duceți-vă numai bărbații și faceți slujbă Domnului, cum ați cerut!" Și au fost dați afară de la Faraon.
\par 12 Atunci a zis Domnul către Moise: "Întinde-ți mâna ta asupra pământului Egiptului și vor năvăli lăcustele asupra pământului Egiptului și vor mânca toată iarba pământului, toate roadele pomilor și tot ce a rămas nestricat de grindină".
\par 13 Deci și-a ridicat Moise toiagul său asupra pământului Egiptului și a adus Domnul asupra pământului acestuia vânt de la răsărit toată ziua aceea și toată noaptea și, când s-a făcut ziuă, vântul de la răsărit a adus lăcuste.
\par 14 Și au năvălit ele în tot pământul Egiptului, s-au așezat în toate ținuturile Egiptului mulțime multă; asemenea lăcuste n-au mai fost și nu vor mai fi.
\par 15 Și au acoperit ele toată țara, cât nu se mai vedea pământul; și au mâncat toată iarba pământului și toate roadele pomilor, care nu fuseseră stricate de grindină; și n-a rămas nici un fir de verdeață, nici în arbori, nici în iarba câmpului în tot pământul Egiptului.
\par 16 Atunci Faraon a chemat în grabă pe Moise și pe Aaron și le-a zis: "Greșit-am înaintea Domnului Dumnezeului vostru și înaintea voastră!
\par 17 Iertați-mi acum încă o dată greșeala mea și vă rugați Domnului Dumnezeului vostru să abată în orice chip de la mine prăpădul acesta!"
\par 18 Și ieșind de la Faraon, Moise s-a rugat lui Dumnezeu,
\par 19 Și Domnul a stârnit vânt puternic de la apus și acesta a dus lăcustele și le-a aruncat în Marea Roșie și n-a rămas nici o lăcustă în tot pământul Egiptului.
\par 20 Dar Domnul a învârtoșat inima lui Faraon și acesta n-a dat drumul fiilor lui Israel.
\par 21 Atunci a zis Domnul către Moise: "Întinde mâna ta spre cer și se va face întuneric în pământul Egiptului, încât să-l pipăi cu mâna".
\par 22 Și și-a întins Moise mâna sa spre cer și s-a făcut întuneric beznă trei zile în tot pământul Egiptului,
\par 23 De nu se vedea om cu om, și nimeni nu s-a urnit de la locul său trei zile. Iar la fiii lui Israel a fost lumină peste tot în locuințele lor.
\par 24 Atunci a chemat Faraon pe Moise și pe Aaron și le-a zis: "Duceți-vă și faceți slujbă Domnului Dumnezeului vostru, dar să rămână aici vitele voastre mărunte și mari, iar copiii să meargă cu voi".
\par 25 Moise însă a zis: "Ba nu, ci dă-ne vite pentru jertfele și arderile de tot ce avem să aducem Domnului Dumnezeului nostru.
\par 26 Deci, să meargă cu noi și turmele noastre și să nu rămână nici un picior, căci din ele avem să luăm ca să aducem jertfă Domnului Dumnezeului nostru; dar, până nu vom ajunge acolo, nu știm ce avem să aducem jertfă Domnului Dumnezeului nostru".
\par 27 Domnul a învârtoșat inima lui Faraon și el n-a vrut să le dea drumul,
\par 28 Ci a zis Faraon către Moise: "Du-te de aici! Dar bagă de seamă să nu te mai arăți în fața mea, căci în ziua când vei vedea fața mea, vei muri".
\par 29 Răspuns-a Moise: "Cum ai zis, așa va fi. Mai mult nu voi mai vedea fața ta!"

\chapter{11}

\par 1 După aceea a zis Domnul către Moise: "Încă o plagă voi mai aduce asupra lui Faraon și asupra Egiptului și după aceea vă vor da drumul de aici. Dar când vă vor da drumul, cu grăbire vă vor alunga de aici.
\par 2 Spune dar poporului în taină, ca fiecare bărbat de la vecinul său și fiecare femeie de la vecina ei să ceară împrumut vase de argint și vase de aur și haine".
\par 3 Și a dat Domnul poporului Său trecere înaintea Egiptenilor și aceștia le-au împrumutat cele cerute. Dar și Moise ajunsese mare foarte în pământul Egiptului, înaintea lui Faraon și a slujitorilor lui Faraon și a tot poporul.
\par 4 Și a zis Moise: "Așa grăiește Domnul: La miezul nopții voi trece prin Egipt
\par 5 Și va muri tot întâiul născut în pământul Egiptului, de la întâiul născut al lui Faraon, care urmează să șadă pe tronul său, până la întâiul născut al roabei de la râșniță și până la întâiul născut al dobitoacelor.
\par 6 Și va fi plângere mare în tot pământul Egiptului, cum n-a mai fost și cum nu va mai fi.
\par 7 Iar la toți fiii lui Israel nici câine nu va lătra, nici la om, nici la dobitoc, ca să cunoașteți ce deosebire face Domnul între Egipteni și Israeliți.
\par 8 Și se vor pogorî toți acești slujitori ai tăi la mine și, închinându-se mie, vor zice: Ieși împreună cu tot poporul tău, pe care-l povățuiești tu. Și după aceea voi și ieși". Și a ieșit Moise de la Faraon înfierbântat de mânie.
\par 9 Apoi a zis Domnul către Moise: "Nu vă va asculta nici acum Faraon, ca să se înmulțească semnele Mele și minunile Mele în pământul Egiptului!"
\par 10 A făcut deci Moise și Aaron toate semnele și minunile acestea înaintea lui Faraon. Dar Domnul a învârtoșat inima lui Faraon și el n-a ascultat să lase pe Israel să iasă din pământul său.

\chapter{12}

\par 1 Apoi a grăit Domnul cu Moise și Aaron în pământul Egiptului și le-a zis:
\par 2 "Luna aceasta să vă fie începutul lunilor, să vă fie întâia între lunile anului.
\par 3 Vorbește deci la toată obștea fiilor lui Israel și le spune: În ziua a zecea a lunii acesteia să-și ia fiecare din capii de familie un miel; câte un miel de familie să luați fiecare.
\par 4 Iar dacă vor fi puțini în familie, încât să nu fie deajuns ca să poată mânca mielul, să ia cu sine de la vecinul cel mai aproape de dânsul un număr de suflete: numărați-vă la un miel atâția cât pot să-l mănânce.
\par 5 Mielul să vă fie de un an, parte bărbătească și fără meteahnă, și să luați sau un miel, sau un ied,
\par 6 Să-l țineți până în ziua a paisprezecea a lunii acesteia și atunci toată adunarea obștii fiilor lui Israel să-l junghie către seară.
\par 7 Să ia din sângele lui și să ungă amândoi ușorii și pragul cel de sus al ușii casei unde au să-l mănânce.
\par 8 Și să mănânce în noaptea aceea carnea lui friptă la foc; dar s-o mănânce cu azimă și cu ierburi amare.
\par 9 Dar să nu-l mâncați nefript deajuns sau fiert în apă, ci să mâncați totul fript bine pe foc, și capul cu picioarele și măruntaiele.
\par 10 Să nu lăsați din el pe a doua zi și oasele lui să nu le zdrobiți. Ceea ce va rămâne pe a doua zi să ardeți în foc.
\par 11 Să-l mâncați însă așa: să aveți coapsele încinse, încălțămintea în picioare și toiegele în mâinile voastre; și să-l mâncați cu grabă, căci este Paștile Domnului.
\par 12 În noaptea aceea voi trece peste pământul Egiptului și voi lovi pe tot întâiul născut în pământul Egiptului, al oamenilor și al dobitoacelor, și voi face judecată asupra tuturor dumnezeilor în pământul Egiptului, căci Eu sunt Domnul.
\par 13 Iar la voi sângele va fi semn pe casele în care vă veți afla: voi vedea sângele și vă voi ocoli și nu va fi între voi rană omorâtoare, când voi lovi pământul Egiptului.
\par 14 Ziua aceea să fie spre pomenire și să prăznuiți într-însa sărbătoarea Domnului, din neam în neam; ca așezare veșnică s-o prăznuiți.
\par 15 Șapte zile să mâncați azime; din ziua întâi să depărtați din casele voastre dospitura, căci cine va mânca dospit din ziua întâi până în ziua a șaptea, sufletul aceluia se va stârpi din Israel.
\par 16 În ziua întâi să aveți adunare sfântă, în ziua a șaptea iar adunare sfântă; și în acele zile să nu faceți nici un fel de lucru decât numai cele ce trebuie fiecăruia de mâncat, numai acelea să vi le faceți.
\par 17 Păziți sărbătoarea azimilor, că în ziua aceea am scos taberele voastre din pământul Egiptului; păziți ziua aceasta în neamul vostru ca așezământ veșnic.
\par 18 Începând din seara zilei a paisprezecea a lunii întâi și până în seara zilei a douăzeci și una a aceleiași luni, să mâncați pâine nedospită.
\par 19 Șapte zile să nu se afle dospitură în casele voastre; tot cel care va mânca dospit, sufletul acela se va stârpi din obștea lui Israel, fie străin sau băștinaș al pământului aceluia.
\par 20 Tot ce e dospit să nu mâncați, ci în toate așezările voastre să mâncați azimă".
\par 21 Apoi a chemat Moise pe toți bătrânii fiilor lui Israel și le-a zis: "Mergeri și vă luați miei după familiile voastre și junghiați Paștile.
\par 22 După aceea să luați un mănunchi de isop și, muindu-l în sângele strâns de la miel într-un vas, să ungeți pragul de sus și amândoi ușorii ușii cu sângele cel din vas, iar voi să nu ieșiți nici unul din casă până dimineața;
\par 23 Căci are să treacă Domnul să lovească Egiptul; și văzând sângele de pe pragul de sus și de pe cei doi ușori, Domnul va trece pe lângă ușă și nu va îngădui pierzătorului să intre în casele voastre, ca să vă lovească.
\par 24 Păziți acestea ca un așezământ veșnic pentru voi și pentru copiii voștri.
\par 25 Iar după ce veți intra în pământul pe care Domnul îl va da vouă, cum a zis, să păziți rânduiala aceasta.
\par 26 Și când vă vor zice copiii voștri: Ce înseamnă rânduiala aceasta?
\par 27 Să le spuneți: Aceasta este jertfa ce o aducem de Paști Domnului, Care în Egipt a trecut pe lângă casele fiilor lui Israel, când a lovit Egiptul, iar casele noastre le-a izbăvit". Și s-a plecat poporul și s-a închinat.
\par 28 Au mers deci fiii lui Israel și au făcut toate cum poruncise Domnul lui Moise și Aaron; așa au făcut.
\par 29 Iar la miezul nopții a lovit Domnul pe toți întâi-născuții în pământul Egiptului, de la întâi-născutul lui Faraon, care ședea pe tron, până la întâi-născutul robului, care sta în închisoare, și pe toți întâi-născuții dobitoacelor.
\par 30 Și s-a sculat noaptea Faraon însuși, toate slugile lui și toți Egiptenii, și s-a făcut bocet mare în toată țara Egiptului, căci nu era casă unde să nu fie mort.
\par 31 În aceeași noapte a chemat Faraon pe Moise și pe Aaron și le-a zis: "Sculați-vă și ieșiți din pământul poporului meu! Și voi și fiii lui Israel! Și duceți-vă de faceți slujbă Domnului Dumnezeului vostru, precum ați zis.
\par 32 Luați cu voi și oile și boii voștri, cum ați cerut, și vă duceți și mă binecuvântați și pe mine!"
\par 33 Și sileau Egiptenii pe poporul evreu să iasă degrabă din țara aceea, căci ziceau: "Pierim cu toții!"
\par 34 Atunci poporul a luat pe umeri aluatul său până a nu se dospi, cu covețile învelite în hainele lor.
\par 35 Și făcând fiii lui Israel cum le poruncise Moise, ei au cerut de la Egipteni vase de argint și de aur și haine;
\par 36 Iar Domnul a dat poporului Său trecere înaintea Egiptenilor, ca să-i dea tot ce a cerut. Și astfel au fost prădați Egiptenii.
\par 37 Fiii lui Israel au plecat din Ramses spre Sucot, ca fa șase sute de mii de bărbați pedeștri, afară de copii.
\par 38 Și a mai ieșit împreună cu ei mulțime de oameni de felurite neamuri, și oi, și boi, și turme foarte mari.
\par 39 Iar din aluatul ce l-au scos din Egipt au copt azime, că nu se dospise încă, pentru că i-au scos Egiptenii și nu putuseră zăbovi nici măcar să-și facă de mâncare pentru drum.
\par 40 Timpul însă, cât fiii lui Israel și părinții lor au trăit în Egipt și în țara Canaan, a fost de patru sute treizeci de ani.
\par 41 Iar după trecerea celor patru sute treizeci de ani a ieșit toată oștirea Domnului din pământul Egiptului, noaptea.
\par 42 Aceasta a fost noaptea de priveghere a Domnului pentru scoaterea lor din țara Egiptului și pe această noapte de priveghere pentru Domnul o vor păzi toți fiii lui Israel din neam în neam.
\par 43 După aceea a zis Domnul către Moise și Aaron: "Rânduiala Paștelui este aceasta: Nimeni din cei de alt neam să nu mănânce din el.
\par 44 Dar tot robul cumpărat cu bani și tăiat împrejur să mănânce din el.
\par 45 Străinul și simbriașul așijderea să nu mănânce din el.
\par 46 Să se mănânce în aceeași casă; să nu lăsați pe a doua zi; carnea să nu o scoateți afară din casă și oasele să nu le zdrobiți.
\par 47 Să-l prăznuiască toată obștea fiilor lui Israel.
\par 48 Iar de va veni la voi vreun străin să facă Paștile Domnului, să tai împrejur pe toți cei de parte bărbătească ai lui și numai atunci să-l săvârșească și va fi ca și locuitorul de baștină al țării; dar tot cel netăiat împrejur să nu mănânce din el.
\par 49 O lege să fie și pentru băștinaș și pentru străinul ce se va așeza la voi!"
\par 50 Și au făcut fiii lui Israel cum poruncise Domnul lui Moise și Aaron; așa au făcut.
\par 51 Deci, în ziua aceea a scos Domnul pe fiii lui Israel din țara Egiptului, cu oștirea lor.

\chapter{13}

\par 1 În vremea aceea a vorbit Domnul cu Moise și i-a zis:
\par 2 "Să-Mi sfințești pe tot întâiul născut, pe tot cel ce se naște întâi la fiii lui Israel, de la om până la dobitoc, că este al Meu!"
\par 3 Iar Moise a zis către popor: "Să vă aduceți aminte de ziua aceasta, în care ați ieșit din pământul Egiptului, din casa robiei, căci cu mână tare v-a scos Domnul de acolo și să nu mâncați dospit;
\par 4 Că astăzi ieșiți voi, în luna Aviv.
\par 5 Iar când te va duce Domnul Dumnezeul tău în țara Canaaneilor, a Heteilor, a Amoreilor, a Heveilor, a Iebuseilor, a Ghergheseilor și a Ferezeilor, pentru care S-a jurat El părinților tăi să-ți dea țara unde curge miere și lapte, să faci slujba aceasta în această lună.
\par 6 Șapte zile să mănânci azime, iar în ziua a șaptea este sărbătoarea Domnului:
\par 7 Azime să mâncați șapte zile și să nu se găsească la tine pâine dospită, nici aluat dospit în toate hotarele tale.
\par 8 În ziua aceea să spui fiului tău și să zici: Acestea sunt pentru cele ce a făcut Domnul cu mine, când am ieșit din Egipt.
\par 9 Să fie acestea ca un semn pe mâna ta și aducere aminte înaintea ochilor tăi, pentru ca legea Domnului să fie în gura ta, căci cu mână tare te-a scos Domnul Dumnezeu din Egipt.
\par 10 Să păziți dar legea aceasta din an în an, la vremea hotărâtă.
\par 11 Și când te va duce Domnul Dumnezeul tău în țara Canaanului, cum S-a jurat ție și părinților tăi, și ti-o va da ție,
\par 12 Atunci să osebești Domnului pe tot cel de parte bărbătească de la oameni, care se naște întâi; și pe tot cel de parte bărbătească, care se va naște întâi din turmele sau de la vitele ce vei avea, să-l închini Domnului.
\par 13 Pe tot întâi-născutul de la asină să-l răscumperi cu un miel; iar de nu-l vei răscumpăra, îi vei frânge gâtul; să răscumperi și pe tot întâi-născutul din oameni în neamul tău.
\par 14 Când însă te va întreba după aceea fiul tău și va zice: Ce înseamnă aceasta?, să-i spui: Cu mână puternică ne-a scos Domnul din pământul Egiptului, din casa robiei.
\par 15 Că atunci când se îndărătnicea Faraon să ne dea drumul, Domnul a omorât pe toți întâi-născuții în pământul Egiptului, de la întâi-născutul oamenilor până la întâi-născutul dobitoacelor. De aceea jertfesc eu Domnului pe tot întâi-născutul de parte bărbătească și pe tot întâi-născutul din fiii mei îl răscumpăr.
\par 16 Să fie dar aceasta ca un semn la mâna ta și ca o tăbliță deasupra ochilor tăi, căci cu mână tare ne-a scos Domnul din Egipt!"
\par 17 Iar după ce Faraon a dat drumul poporului, Dumnezeu nu l-a dus pe calea cea către pământul Filistenilor, care era mai scurtă; căci a zis Dumnezeu: "Nu cumva poporul, văzând război, să-i pară rău și să se întoarcă în Egipt".
\par 18 Ci a dus Dumnezeu poporul împrejur, pe calea pustiului, către Marea Roșie. Și fiii lui Israel au ieșit în bună rânduială din pământul Egiptului.
\par 19 Atunci a luat Moise cu sine oasele lui Iosif; căci Iosif legase pe fiii lui Israel cu jurământ, zicând: "Are să vă cerceteze Dumnezeu și atunci să luați cu voi și oasele mele de aici!"
\par 20 Fiii lui Israel au pornit apoi din Sucot și și-au așezat tabăra la Etam, la capătul pustiului.
\par 21 Iar Domnul mergea înaintea lor: ziua în stâlp de nor, arătându-le calea, iar noaptea în stâlp de foc, luminându-le, ca să poată merge și ziua și noaptea.
\par 22 Și n-a lipsit stâlpul de nor ziua, nici stâlpul de foc noaptea dinaintea poporului.

\chapter{14}

\par 1 Atunci a grăit Domnul cu Moise și a zis:
\par 2 "Spune fiilor lui Israel să se întoarcă și să-și așeze tabăra în fața Pi-Hahirotului, între Migdal și mare, în preajma lui Baal-Țefon. Acolo, în preajma lui, lângă mare, să tăbărâți.
\par 3 Că Faraon va zice către poporul său: Fiii aceștia ai lui Israel s-au rătăcit în pământul acesta și i-a închis pustiul.
\par 4 Iar Eu voi învârtoșa inima lui Faraon și va alerga după ei. Și-Mi voi arăta slava Mea asupra lui Faraon și asupra a toată oștirea lui; și vor cunoaște toți Egiptenii că Eu sunt Domnul!" Și au făcut așa.
\par 5 Atunci s-a dat de știre regelui Egiptului că poporul evreu a fugit. și s-a întors inima lui Faraon și a slujitorilor lui asupra poporului acestuia și ei au zis: "Ce am făcut noi? Cum de am lăsat pe fiii lui Israel să se ducă și să nu ne mai robească nouă?"
\par 6 A înhămat deci Faraon carele sale de război și a luat poporul său cu sine:
\par 7 A luat cu sine șase sute de căruțe alese și toată călărimea Egiptului și căpeteniile lor.
\par 8 Iar Domnul a învârtoșat inima lui Faraon, regele Egiptului, și a slujitorilor lui, și a alergat acesta după fiii lui Israel; dar fiii lui Israel ieșiseră sub mână înaltă.
\par 9 Și au alergat după ei Egiptenii cu toți caii și carele lui Faraon, cu călăreții și cu toată oștirea lui și i-au ajuns când poposiseră ei la mare, lângă Pi-Hahirot, în fața lui Baal-Țefon.
\par 10 Dar când s-a apropiat Faraon și când s-au uitat fiii lui Israel înapoi și au văzut că Egiptenii vin după ei, s-au spăimântat foarte tare fiii lui Israel și au strigat către Domnul;
\par 11 Și au zis către Moise: "Oare nu erau morminte în țara Egiptului, de ce ne-ai adus să murim în pustie? Ce ai făcut tu cu noi, scoțându-ne din Egipt?
\par 12 Nu ți-am spus noi, oare, de aceasta în Egipt, când ți-am zis: Lasă-ne să robim Egiptenilor, că e mai bine să fim robi Egiptenilor decât să murim în pustia aceasta?"
\par 13 Moise însă a zis către popor: "Nu vă temeți! Stați și veți vedea minunea cea de la Domnul, pe care vă va face-o El astăzi, căci pe Egiptenii pe care îi vedeți astăzi nu-i veți mai vedea niciodată.
\par 14 Domnul are să Se lupte pentru voi, iar voi fiți liniștiți!"
\par 15 Atunci a zis Domnul către Moise: "Ce strigi către Mine? Spune fiilor lui Israel să pornească,
\par 16 Iar tu ridică-n toiagul și-ți întinde mâna asupra mării și o desparte și vor trece fiii lui Israel prin mijlocul mării, ca pe uscat.
\par 17 Iată, Eu voi învârtoșa inima lui Faraon și a tuturor Egiptenilor, ca să meargă pe urmele lor. Și-Mi voi arăta slava Mea asupra lui Faraon și asupra a toată oștirea lui, asupra carelor lui și asupra călăreților lui.
\par 18 Și vor cunoaște toți Egiptenii că Eu sunt Domnul, când Îmi voi arăta slava Mea asupra lui Faraon, asupra carelor lui și asupra călăreților lui".
\par 19 Atunci s-a ridicat îngerul Domnului, care mergea înaintea taberei fiilor lui Israel, și s-a mutat în urma lor; și s-a ridicat stâlpul cel de nor dinaintea lor și a stat în urma lor.
\par 20 Astfel a trecut el și a stat între tabăra Egiptenilor și tabăra fiilor lui Israel; și era negură și întuneric pentru unii, iar pentru ceilalți lumină, noaptea, și toată noaptea nu s-au apropiat unii de alții.
\par 21 Iar Moise și-a întins mâna sa asupra mării și a alungat Domnul marea toată noaptea cu vânt puternic de la răsărit și s-a făcut marea uscat, că s-au despărțit apele.
\par 22 Și au intrat fiii lui Israel prin mijlocul mării, mergând ca pe uscat, iar apele le erau perete, la dreapta și la stânga lor.
\par 23 Iar Egiptenii urmărindu-i, au intrat după ei în mijlocul mării toți caii lui Faraon, carele și călăreții lui.
\par 24 Dar în straja dimineții a căutat Domnul din stâlpul cel de foc și din nor spre tabăra Egiptenilor și a umplut tabăra Egiptenilor de spaimă.
\par 25 Și a făcut să sară roțile de la carele lor, încât cu anevoie mergeau carele. Atunci au zis Egiptenii: "Să fugim de la fața lui Israel, că Domnul se luptă pentru ei cu Egiptenii!"
\par 26 Iar Domnul a zis către Moise: "Întindeți mâna asupra mării, ca să se întoarcă apele asupra Egiptenilor, asupra carelor lor și asupra călăreților lor".
\par 27 Și și-a întins Moise mâna asupra mării și spre ziuă s-a întors apa la locul ei, iar Egiptenii fugeau împotriva apei. Și așa a înecat Dumnezeu pe Egipteni în mijlocul mării.
\par 28 Iar apele s-au tras la loc și au acoperit carele și călăreții întregii oștiri a lui Faraon, care intrase după Israeliți în mare, și nu a rămas nici unul dintre ei.
\par 29 Fiii lui Israel însă au trecut prin mare ca pe uscat și apa le-a fost perete la dreapta și stânga lor.
\par 30 Așa a izbăvit Domnul în ziua aceea pe Israeliți din mâinile Egiptenilor; și au văzut fiii lui Israel pe Egipteni morți pe malurile mării.
\par 31 Văzut-a Israel mâna cea tare pe care a întins-o Domnul asupra Egiptenilor, și s-a temut poporul de Domnul și a crezut în Domnul și în Moise, sluga Lui.

\chapter{15}

\par 1 Atunci Moise și fiii lui Israel au cântat Domnului cântarea aceasta și au zis: "Să cântăm Domnului, căci cu slavă S-a preaslăvit! Pe cal și pe călăreț în mare i-a aruncat!
\par 2 Tăria mea și mărirea mea este Domnul, căci El m-a izbăvit. Acesta este Dumnezeul meu și-L voi preaslăvi, Dumnezeul părintelui meu și-L voi preaînălța!
\par 3 Domnul este viteaz în luptă; Domnul este numele Lui.
\par 4 Carele lui Faraon și oștirea lui în mare le-a aruncat; Pe căpeteniile cele de seamă ale lui, Marea Roșie le-a înghițit,
\par 5 Adâncul le-a acoperit, În fundul mării ca o piatră s-au pogorât.
\par 6 Dreapta Ta, Doamne, și-a arătat tăria. Mâna Ta cea dreaptă, Doamne, pe vrăjmași i-a sfărâmat.
\par 7 Cu mulțimea slavei Tale ai surpat pe cei potrivnici. Trimis-ai mânia Ta Și i-a mistuit ca pe niște paie.
\par 8 La suflarea nărilor Tale s-a despărțit apa, Strânsu-s-au la un loc apele ca un perete Și s-au închegat valurile în inima mării.
\par 9 Vrăjmașul zicea: "Alerga-voi după ei și-i voi ajunge; Pradă voi împărți și-mi voi sătura sufletul de răzbunare; Voi scoate sabia și mâna mea îi va stârpi".
\par 10 Dar ai trimis Tu duhul Tău Și marea i-a înghițit; Afundatu-s-au ca plumbul În apele cele mari.
\par 11 Doamne, cine este asemenea ție între dumnezei? Cine este asemenea ție preaslăvit în sfințenie, Minunat întru slavă Și făcător de minuni?
\par 12 Întins-ai dreapta Ta Și i-a înghițit pământul!
\par 13 Călăuzit-ai cu mila Ta acest popor și l-ai izbăvit; Tu îl povățuiești cu puterea Ta, Spre locașul sfințeniei Tale.
\par 14 Auzit-au neamurile și s-au cutremurat, Frică a cuprins pe cei din Filisteia.
\par 15 Atunci s-au spăimântat căpeteniile Edomului, Pe conducătorii Moabului cutremur i-a cuprins; Și toți câți trăiesc în Canaan și-au pierdut cumpătul.
\par 16 Frică și groază va cădea peste ei. Și de măreția brațului Tău, Ca pietrele vor încremeni, Până va trece poporul Tău, Doamne, Până va trece poporul Tău acesta, pe care l-ai câștigat Tu.
\par 17 Tu îl vei duce și-l vei sădi în muntele moștenirii Tale, În locul ce ți l-ai făcut sălășluire, Doamne, În locașul sfânt cel zidit de mâinile Tale, Doamne!
\par 18 Împărăți-va Domnul în veac și în veacul veacului.
\par 19 Căci caii lui Faraon cu carele și călăreții lui au intrat în mare. Întors-a Domnul asupra lor apele mării, Iar fiii lui Israel au trecut prin mare, ca pe uscat!"
\par 20 Atunci a luat Mariam proorocița, sora lui Aaron, timpanul în mâna sa, și au ieșit după dânsa toate femeile cu timpane și dănțuind.
\par 21 Și răspundea Mariam înaintea lor: "Să cântăm Domnului, căci cu slavă S-a preaslăvit! Pe cal și pe călăreț în mare i-a aruncat!"
\par 22 Apoi a ridicat Moise pe fiii lui Israel de la Marea Roșie și i-a dus în pustia Șur și au mers trei zile prin pustie și n-au găsit apă.
\par 23 Au ajuns apoi la Mara, dar n-au putut să bea apă nici din Mara, că era amară, pentru care s-a și numit locul acela Mara.
\par 24 De aceea cârtea poporul împotriva lui Moise și zicea: "Ce să bem?"
\par 25 Atunci Moise a strigat către Domnul și Domnul i-a arătat un lemn; și l-a aruncat în apă și s-a îndulcit apa. Acolo a pus Domnul poporului Său rânduieli și porunci și acolo l-a încercat și i-a zis:
\par 26 "De vei asculta cu luare-aminte glasul Domnului Dumnezeului tău și vei face lucruri drepte înaintea Lui și de vei lua aminte la poruncile Lui și vei păzi legile Lui, nu voi aduce asupra ta nici una din bolile pe care le-am adus asupra Egiptenilor, că Eu sunt Domnul Dumnezeul tău Care te vindecă".
\par 27 Apoi au venit în Elim. Și erau acolo douăsprezece izvoare de apă și șaptezeci de pomi de finic. Și au tăbărât acolo lângă apă.

\chapter{16}

\par 1 Plecând apoi din Elim, a venit toată obștea fiilor lui Israel în pustia Sin, câre este între Elim și între Sinai, în ziua a cincisprezecea a lunii a doua, după ieșirea din Egipt.
\par 2 În pustia aceasta toată obștea fiilor lui Israel a cârtit împotriva lui Moise și Aaron.
\par 3 Și au zis către ei fiii lui Israel: "Mai bine muream bătuți de Domnul în pământul Egiptului, când ședeam împrejurul căldărilor cu carne și mâncam pâine de ne săturam! Dar voi ne-ați adus în pustia aceasta, ca toată obștea aceasta să moară de foame".
\par 4 Domnul însă a zis către Moise: "Iată Eu le voi ploua pâine din cer. Să iasă dar poporul și să adune în fiecare zi cât trebuie pentru o zi, ca să-l încerc dacă va umbla sau nu după legea Mea.
\par 5 Iar în ziua a șasea să adune de două ori mai mult decât adunau în celelalte zile, pentru o zi".
\par 6 Atunci au zis Moise și Aaron către toată adunarea fiilor lui Israel: "Diseară veți cunoaște că Domnul v-a scos din pământul Egiptului.
\par 7 Și dimineață veți vedea slava Domnului, că El a auzit cârtirea voastră împotriva lui Dumnezeu; iar noi ce suntem de cârtiți împotriva noastră?"
\par 8 Și a mai zis Moise: "Când Domnul vă va da diseară carne să mâncați și dimineață pâine să vă săturați, din aceea veri afla că a auzit Domnul cârtirea ce ați ridicat asupra Lui. Căci noi ce suntem? Cârtirea voastră nu este împotriva noastră, ci împotriva lui Dumnezeu".
\par 9 Apoi a zis Moise către Aaron: "Spune la toată adunarea fiilor lui Israel: Apropiați-vă înaintea lui Dumnezeu, că a auzit cârtirea voastră!"
\par 10 Iar când vorbea Aaron către toată adunarea fiilor lui Israel, au căutat ei spre pustie și iată slava Domnului s-a arătat în nor.
\par 11 Și a grăit Domnul cu Moise și a zis:
\par 12 "Am auzit cârtirea fiilor lui Israel. Spune-le dar: Diseară carne veți mânca, iar dimineață vă veți sătura de pâine și veți cunoaște că Eu, Domnul, sunt Dumnezeul vostru".
\par 13 Iar, dacă s-a făcut seară, au venit prepelițe și au acoperit tabăra, iar dimineața, după ce s-a luat roua dimprejurul taberei,
\par 14 Iată, se afla pe fața pustiei ceva mărunt, ca niște grăunțe, și albicios, ca grindina pe pământ.
\par 15 Și văzând fiii lui Israel, au zis unii către alții: "Ce e asta?" Că nu știau ce e. Iar Moise le-a zis: "Aceasta e pâinea pe care v-o dă Dumnezeu să o mâncați.
\par 16 Iată ce a poruncit Domnul: Adunați fiecare cât să vă ajungă de mâncat; câte un omer de om, după numărul sufletelor voastre; fiecare câți are în cort, atâtea omere să adune!"
\par 17 Și au făcut așa fiii lui Israel; au adunat unii mai mult, alții mai puțin;
\par 18 Dar măsurând cu omerul, nici celui ce adunase mult n-a prisosit, nici celui ce adunase puțin n-a lipsit, ci fiecare, cât era deajuns la cei ce erau cu sine, atât a adunat.
\par 19 Zis-a iarăși Moise către ei: "Nimeni să nu lase din aceasta pe a doua zi".
\par 20 Dar ei n-au ascultat pe Moise, ci unii au lăsat din aceasta pe a doua zi; dar a făcut viermi și s-a stricat. Și s-a mâniat pe ei Moise.
\par 21 Fiecare aduna mană dimineața cât îi trebuia pentru mâncat în ziua aceea, căci, dacă se înfierbânta soarele, ceea ce rămânea se topea.
\par 22 Iar în ziua a șasea adunară de două ori mai multă: câte două omere de fiecare. Și au venit toate căpeteniile adunării să-l înștiințeze pe Moise.
\par 23 Iar Moise le-a zis: "Iată ce a zis Domnul: Mâine e odihnă, odihna cea sfântă în cinstea Domnului; ce trebuie copt coaceți, ce trebuie fiert, fierbeți astăzi, și ce va rămâne, păstrați pe a doua zi!"
\par 24 Și au lăsat din acestea până dimineața, după cum le poruncise Moise, și nu s-au stricat nici n-au făcut viermi.
\par 25 Apoi a zis Moise: "Mâncați aceasta astăzi, că astăzi este odihna în cinstea Domnului și nu veți găsi de aceasta astăzi pe câmp.
\par 26 Șase zile să adunați, iar ziua a șaptea este zi de odihnă și nu veți afla din ea în această zi".
\par 27 Dar unii din popor au ieșit să adune și în ziua a șaptea și n-au găsit.
\par 28 Atunci Domnul a zis către Moise: "Până când nu veți voi să ascultați de poruncile Mele și de învățăturile Mele?
\par 29 Vedeți că Domnul v-a dat ziua aceasta de odihnă și de aceea vă dă El în ziua a șasea și pâine pentru două zile; rămâneți fiecare în casele voastre și nimeni să nu iasă de la locul său în ziua a șaptea".
\par 30 Și s-a odihnit poporul în ziua a șaptea.
\par 31 Casa lui Israel i-a pus numele mană și aceasta era albă, ca sămânța de coriandru, iar la gust ca turta cu miere.
\par 32 După aceea Moise a zis: "Iată ce poruncește Domnul: Umpleți cu mană un omer, ca să se păstreze în viitor urmașilor voștri, ca să vadă pâinea cu care v-am hrănit Eu în pustie, după ce v-am scos din țara Egiptului".
\par 33 Iar către Aaron a zis Moise: "Ia un vas de aur și toarnă în el un omer plin cu mană și pune-l înaintea Domnului, ca să se păstreze în viitor pentru urmașii voștri!"
\par 34 Și l-a pus Aaron înaintea chivotului mărturiei, ca să se păstreze, cum poruncise Domnul lui Moise.
\par 35 Iar fiii lui Israel au mâncat mană patruzeci de ani, până ce au ajuns în țară locuită; până ce au ajuns în hotarele pământului Canaan au mâncat mană.
\par 36 Iar omerul este a zecea parte dintr-o efă.

\chapter{17}

\par 1 După aceea a plecat la drum toată obștea fiilor lui Israel din pustia Sin, după porunca Domnului, și a tăbărât la Rafidim, unde poporul nu avea apă de băut.
\par 2 Și poporul căuta ceartă lui Moise, zicând: "Dă-ne apă să bem!" Iar Moise le-a zis: "De ce mă bănuiți și de ce ispitiți pe Domnul?"
\par 3 Atunci poporul, apăsat de sete, cârtea împotriva lui Moise și zicea: "Ce este aceasta? Ne-ai scos din Egipt ca să ne omori cu sete pe noi, pe copiii noștri și turmele noastre?"
\par 4 Iar Moise a strigat către Domnul și a zis: "Ce să fac cu poporul acesta? Căci puțin lipsește ca să mă ucidă cu pietre".
\par 5 Zis-a Domnul către Moise: "Treci pe dinaintea poporului acestuia, dar ia cu tine câțiva din bătrânii lui Israel; ia în mână și toiagul cu care ai lovit Nilul și du-te.
\par 6 Iată Eu voi sta înaintea ta acolo la stânca din Horeb, iar tu vei lovi în stâncă și va curge din ea apă și va bea poporul". Și a făcut Moise așa înaintea bătrânilor lui Israel.
\par 7 De aceea s-a pus locului aceluia numele: Masa și Meriba, pentru că acolo cârtiseră fiii lui Israel și pentru că ispitiseră pe Domnul, zicând: "Este, oare, Domnul în mijlocul nostru sau nu?"
\par 8 Atunci au venit Amaleciții să se bată cu Israeliții la Rafidim.
\par 9 Iar Moise a zis către Iosua: "Alege-fi bărbați voinici și du-te de te luptă cu Amaleciții! Iar eu mă voi sui mâine în vârful muntelui și toiagul lui Dumnezeu va fi în mâna mea".
\par 10 A făcut deci Iosua cum îi zisese Moise și s-a dus să bată pe Amaleciți; iar Moise cu Aaron și Or s-au suit în vârful muntelui.
\par 11 Când își ridica Moise mâinile, biruia Israel; iar când își lăsa el mâinile, biruiau Amaleciții.
\par 12 Dar obosind mâinile lui Moise, au luat o piatră și au pus-o lângă el și a șezut Moise pe piatră; iar Aaron și Or îi sprijineau mâinile, unul de o parte și altul de altă parte. Și au stat mâinile lui ridicate până la asfințitul soarelui.
\par 13 Și a zdrobit Iosua pe Amalec și tot poporul lui cu ascuțișul sabiei.
\par 14 Atunci a zis Domnul către Moise: "Scrie acestea în carte spre pomenire și spune lui Iosua că voi șterge cu totul pomenirea lui Amalec de sub cer!"
\par 15 Atunci a făcut Moise un jertfelnic Domnului și i-a pus numele: "Domnul este scăparea mea!"
\par 16 Căci zicea: "Pentru că mi-au fost mâinile ridicate spre scaunul Domnului, de aceea va bate Domnul pe Amalec din neam în neam!"

\chapter{18}

\par 1 Auzind însă Ietro, preotul din Madian, socrul lui Moise, de toate câte făcuse Dumnezeu pentru Moise și pentru Israel, poporul Său, când a scos Domnul pe Israel din Egipt,
\par 2 A luat Ietro, socrul lui Moise, pe Sefora, femeia lui Moise, care fusese trimisă înainte de acesta acasă,
\par 3 Și pe cei doi fii ai ei, din care unul se chema Gherșom, pentru că Moise își zisese: "Rătăcit sunt eu în pământ străin",
\par 4 Iar pe altul îl chema Eliezer, pentru că-și zisese el: "Dumnezeul părinților mei mi-a fost ajutor și m-a scăpat de sabia lui Faraon!"
\par 5 Și a venit Ietro, socrul lui Moise, cu fiii acestuia și cu femeia lui, la Moise în pustie, unde-și așezase el tabăra, la muntele lui Dumnezeu.
\par 6 Atunci el a trimis vorbă lui Moise, zicând: "Iată, eu, Ietro, socrul tău, și femeia ta și cei doi fii ai ei împreună cu ea venim la tine".
\par 7 Deci a ieșit Moise în întâmpinarea socrului său, s-a plecat înaintea lui și l-a sărutat. Iar după ce s-au binecuvântat unul pe altul, au intrat în cort.
\par 8 Apoi a povestit Moise socrului său toate câte a făcut Domnul cu Faraon și cu toți Egiptenii pentru Israel, toate suferințele ce le-au întâlnit ei în cale și cum i-a izbăvit Domnul din mâinile lui Faraon și din mâinile Egiptenilor.
\par 9 Iar Ietro s-a bucurat de toate binefacerile ce a arătat Domnul lui Israel, când l-a izbăvit din mâna Egiptenilor și din mâna lui Faraon.
\par 10 și a zis Ietro: "Binecuvântat este Domnul, Care v-a izbăvit din mâinile Egiptenilor și din mâna tui Faraon, Cel ce a izbăvit pe poporul acesta din stăpânirea Egiptenilor.
\par 11 Acum am cunoscut și eu că Domnul este mare peste toți dumnezeii, pentru că a smerit pe aceștia".
\par 12 Apoi Ietro, socrul lui Moise, a adus lui Dumnezeu ardere de tot și jertfă. și au venit Aaron și toți bătrânii lui Israel să mănânce pâine cu socrul lui Moise înaintea lui Dumnezeu.
\par 13 Iar a doua zi a șezut Moise să judece poporul și a stat poporul înaintea lui Moise de dimineață până seara.
\par 14 Văzând Ietro, socrul lui Moise, tot ceea ce făcea el cu poporul, i-a zis: "Ce faci tu cu poporul? De ce stai tu singur și tot poporul tău stă înaintea ta de dimineață până seara?"
\par 15 Iar Moise a zis către socrul său: "Poporul vine la mine să ceară judecată de la Dumnezeu.
\par 16 Când se ivesc între ei neînțelegeri, vin la mine și judec pe fiecare și-i învăț poruncile lui Dumnezeu și legile Lui".
\par 17 Iar socrul lui Moise a zis către acesta: "Ceea ce faci, nu faci bine.
\par 18 Căci te vei prăpădi și tu, și poporul acesta, care este cu tine. E grea pentru tine sarcina aceasta și nu o vei putea împlini singur.
\par 19 Acum dar ascultă-mă pe mine: Am să-ți dau un sfat și Dumnezeu să fie cu tine! Fii tu pentru popor mijlocitor înaintea lui Dumnezeu și înfățișează la Dumnezeu nevoile lui.
\par 20 Învață-i poruncile și legile Lui; arată-le calea Lui, pe care trebuie să meargă, și faptele ce trebuie să facă.
\par 21 Iar mai departe alege-ți din tot poporul oameni drepți și cu frica lui Dumnezeu; oameni drepți, care urăsc lăcomia, și-i pune căpetenii peste mii, căpetenii peste sute, căpetenii peste cincizeci, căpetenii peste zeci.
\par 22 Aceștia să judece poporul în toată vremea: pricinile grele să le aducă la tine, iar pe cele mici să le judece ei toate. Ușurează-ți povara și ei să o poarte împreună cu tine!
\par 23 De vei face lucrul acesta și te va întări și Dumnezeu cu porunci, vei putea să faci față, și tot poporul acesta va ajunge cu pace la locul său".
\par 24 Și a ascultat Moise glasul socrului său și a făcut toate câte i-a zis.
\par 25 A ales deci Moise din tot Israelul oameni destoinici și i-a pus căpetenii în popor: peste mii, peste sute, peste cincizeci, peste zeci.
\par 26 Și judecau aceștia poporul în toată vremea; toate pricinile grele le aduceau la Moise, iar pe cele mai ușoare le judecau ei toate.
\par 27 După aceea a petrecut Moise pe socrul său și acesta s-a dus în țara lui.

\chapter{19}

\par 1 Iar în luna a treia de la ieșirea fiilor lui Israel din pământul Egiptului, chiar în ziua de lună plină, au ajuns în pustia Sinai.
\par 2 Plecase deci Israel de la Rafidim și ajungând în pustia Sinai, au tăbărât acolo în pustie, în fața muntelui.
\par 3 Apoi s-a suit Moise în munte, la Dumnezeu; și l-a strigat Domnul din vârful muntelui și i-a zis: "Grăiește casei lui Iacov și vestește fiilor lui Israel așa:
\par 4 "Ați văzut ce am făcut Egiptenilor și cum v-am luat pe aripi de vultur și v-am adus la Mine.
\par 5 Deci, de veți asculta glasul Meu și de veți păzi legământul Meu, dintre toate neamurile Îmi veți fi popor ales că al Meu este tot pământul;
\par 6 Îmi veți fi împărăție preoțească și neam sfânt!" Acestea sunt cuvintele pe care le vei spune fiilor lui Israel".
\par 7 Și venind, Moise a chemat pe bătrânii poporului și le-a spus toate cuvintele acestea pe care le poruncise Domnul.
\par 8 Atunci tot poporul, răspunzând într-un glas, a zis: "Toate câte a zis Domnul vom face și vom fi ascultători!" Și a dus Moise cuvintele poporului la Domnul.
\par 9 Iar Domnul a zis către Moise: "Iată voi veni la tine în stâlp de nor des, ca să audă poporul că Eu grăiesc cu tine, și să te creadă pururea". Iar Moise a spus Domnului cuvintele poporului.
\par 10 Zis-a Domnul către Moise: "Pogoară-te de grăiește poporului să se țină curat astăzi și mâine, și să-și spele hainele,
\par 11 Ca să fie gata pentru poimâine, căci poimâine Se va pogorî Domnul înaintea ochilor a tot poporul pe Muntele Sinai.
\par 12 Să-i tragi poporului hotar împrejurul muntelui și să-i spui: Păziți-vă de a vă sui în munte și de a vă atinge de ceva din el, că tot cel ce se va atinge de munte va muri.
\par 13 Nici cu mâna să nu se atingă de el, că va fi ucis cu pietre sau se va săgeta cu săgeata; nu va rămâne în viață, fie om, fie dobitoc. Iar dacă se vor îndepărta tunetele și trâmbițele și norul de pe munte, se vor putea sui în munte".
\par 14 Pogorându-se deci Moise din munte la popor, el a sfințit poporul și, spălându-și ei hainele,
\par 15 Le-a zis Moise: "Să fiți gata pentru poimâine și de femei să nu vă atingeți!"
\par 16 Iar a treia zi, când s-a făcut ziuă, erau tunete și fulgere și nor des pe Muntele Sinai și sunet de trâmbițe foarte puternic. Și s-a cutremurat tot poporul în tabără.
\par 17 Atunci a scos Moise poporul din tabără în întâmpinarea lui Dumnezeu și au stat la poalele muntelui.
\par 18 Iar Muntele Sinai fumega tot, că Se pogorâse Dumnezeu pe el în foc; și se ridica de pe el fum, ca fumul dintr-un cuptor, și tot muntele se cutremura puternic.
\par 19 De asemenea și sunetul trâmbiței se auzea din ce în ce mai tare; și Moise grăia, iar Dumnezeu îi răspundea cu glas.
\par 20 Deci, fiind pogorât Domnul pe Muntele Sinai, pe vârful muntelui, a chemat Domnul pe Moise în vârful muntelui și s-a suit Moise acolo.
\par 21 Atunci a zis Domnul către Moise: "Pogoară-te și oprește poporul, ca să nu năvălească spre Domnul, să vadă slava Lui, că vor cădea mulți dintre ei.
\par 22 Iar preoții, care se apropie de Domnul Dumnezeu, să se sfințească, ca nu cumva să-i lovească Domnul".
\par 23 Zis-a Moise către Domnul: "Nu se poate ca poporul să se suie pe Muntele Sinai, pentru că Tu ne-ai oprit din vreme, și ai zis: Trage hotar împrejurul muntelui și-l sfințește!"
\par 24 Iar Domnul i-a răspuns: "Du-te și te pogoară și apoi te vei sui împreună cu Aaron; iar preoții și poporul să nu îndrăznească a se sui la Domnul, ca să nu-i lovească Domnul".
\par 25 Și s-a pogorât Moise la popor și i-a spus toate.

\chapter{20}

\par 1 Atunci a rostit Domnul înaintea lui Moise toate cuvintele acestea și a zis:
\par 2 "Eu sunt Domnul Dumnezeul tău, Care te-a scos din pământul Egiptului și din casa robiei.
\par 3 Să nu ai alți dumnezei afară de Mine!
\par 4 Să nu-ți faci chip cioplit și nici un fel de asemănare a nici unui lucru din câte sunt în cer, sus, și din câte sunt pe pământ, jos, și din câte sunt în apele de sub pământ!
\par 5 Să nu te închini lor, nici să le slujești, că Eu, Domnul Dumnezeul tău, sunt un Dumnezeu zelos, care pedepsesc pe copii pentru vina părinților ce Mă urăsc pe Mine, până la al treilea și al patrulea neam,
\par 6 Și Mă milostivesc până la al miilea neam către cei ce Mă iubesc și păzesc poruncile Mele.
\par 7 Să nu iei numele Domnului Dumnezeului tău în deșert, că nu va lăsa Domnul nepedepsit pe cel ce ia în deșert numele Lui.
\par 8 Adu-ți aminte de ziua odihnei, ca să o sfințești.
\par 9 Lucrează șase zile și-ți fă în acelea toate treburile tale,
\par 10 Iar ziua a șaptea este odihna Domnului Dumnezeului tău: să nu faci în acea zi nici un lucru: nici tu, nici fiul tău, nici fiica ta, nici sluga ta, nici slujnica ta, nici boul tău, nici asinul tău, nici orice dobitoc al tău, nici străinul care rămâne la tine,
\par 11 Că în șase zile a făcut Domnul cerul și pământul, marea și toate cele ce sunt într-însele, iar în ziua a șaptea S-a odihnit. De aceea a binecuvântat Domnul ziua a șaptea și a sfințit-o.
\par 12 Cinstește pe tatăl tău și pe mama ta, ca să-ți fie bine și să trăiești ani mulți pe pământul pe care Domnul Dumnezeul tău ți-l va da ție.
\par 13 Să nu ucizi!
\par 14 Să nu fii desfrânat!
\par 15 Să nu furi!
\par 16 Să nu mărturisești strâmb împotriva aproapelui tău!
\par 17 Să nu dorești casa aproapelui tău; să nu dorești femeia aproapelui tău, nici ogorul lui, nici sluga lui, nici slujnica lui, nici boul lui, nici asinul lui și nici unul din dobitoacele lui și nimic din câte are aproapele tău!"
\par 18 Și tot poporul a auzit fulgerele și tunetele și sunetul trâmbițelor, și a văzut muntele fumegând; și văzând, tot poporul s-a dat înapoi și a stat departe, temându-se.
\par 19 Apoi a zis către Moise: "Vorbește tu cu noi și vom asculta, dar Dumnezeu să nu grăiască cu noi, ca să nu murim".
\par 20 Zis-a Moise către popor: "Cutezați, că Dumnezeu a venit la voi, să vă pună la încercare pentru ca frica Lui să fie în voi, ca să nu greșiți".
\par 21 Și a stat tot poporul departe, iar Moise s-a apropiat de întunericul unde era Dumnezeu.
\par 22 Atunci Domnul a zis către Moise: "Așa să vorbești casei lui Iacov și așa să vestești fiilor lui Israel: Ați văzut că am grăit cu voi din cer!
\par 23 Să nu vă faceți dumnezei de argint și nici dumnezei de aur să nu vă faceți.
\par 24 Să-Mi faci jertfelnic de pământ și să aduci pe el arderile de tot ale tale, jertfele de izbăvire, oile și boii tăi. În tot locul unde voi pune pomenirea numelui Meu, acolo voi veni la tine, ca să te binecuvântez.
\par 25 Iar de-Mi vei face jertfelnic de piatră, să nu-l faci de piatră cioplită; că de vei pune dalta ta pe ea, o vei spurca.
\par 26 Și să nu te sui pe trepte la jertfelnicul Meu, ca să nu se descopere acolo goliciunea ta!"

\chapter{21}

\par 1 "Iată acum legiuirile pe care tu le vei pune în vedere lor:
\par 2 De vei cumpăra rob evreu, el să-ți lucreze șase ani, iar în anul al șaptelea să iasă slobod, în dar.
\par 3 Dacă acela a venit în casa ta singur, singur să iasă; iar de a venit cu femeie, să iasă cu el și femeia lui.
\par 4 Dacă însă îi va fi dat stăpânul femeie și aceasta va fi născut fii sau fiice, atunci femeia și copiii ei vor fi ai stăpânului lui, iar el va ieși singur.
\par 5 Iar dacă robul va zice: Îmi iubesc stăpânul, femeia și copiii și nu voi să mă liberez,
\par 6 Atunci să-l aducă stăpânul lui la judecători și, după ce l-a apropiat de ușă sau la ușori, să-i găurească stăpânul urechea cu o sulă, și-l va robi în veci.
\par 7 Dacă cineva își va vinde fiica roabă, ea nu va ieși cum ies roabele.
\par 8 Dacă ea nu va plăcea stăpânului său, care și-a ales-o, să-i îngăduie a se răscumpăra, dar el nu va avea voie s-o vândă la familie străină, după ce i-a fost necredincios.
\par 9 Dacă a logodit-o cu fiul său, atunci să se poarte cu ea după dreptul fiicelor.
\par 10 Iar dacă va mai lua și pe alta, atunci ea să nu fie lipsită de hrană, de îmbrăcăminte și de traiul cu bărbatul său.
\par 11 Iar dacă el nu-i va face aceste trei lucruri, să iasă de la dânsul în dar, fără răscumpărare.
\par 12 De va lovi cineva pe un om și acela va muri, să fie dat morții.
\par 13 Iar de nu-l va fi lovit cu voință și i-a căzut sub mână din îngăduirea lui Dumnezeu, îți voi hotărî un loc, unde să fugă ucigașul.
\par 14 Dacă însă va ucide cineva pe aproapele său cu bună știință și cu vicleșug și va fugi la altarul Meu, și de la altarul Meu să-l iei și să-l omori.
\par 15 Cel ce va bate pe tată sau pe mamă să fie omorât.
\par 16 Cel ce va fura un om din fiii lui Israel și, făcându-l rob, îl va vinde, sau se va găsi în mâinile lui, acela să fie omorât.
\par 17 Cel ce va grăi de rău pe tatăl său sau pe mama sa, acela să fie omorât.
\par 18 De se vor sfădi doi oameni și unul va lovi pe celălalt cu o piatră, sau cu pumnul, și acela nu va muri, ci va cădea la pat,
\par 19 De se va scula și va ieși din casă cu ajutorul cârjei, cel ce l-a lovit nu va fi vinovat de moarte, ci va plăti numai împiedicarea aceluia de la muncă și vindecarea lui.
\par 20 Iar de va lovi cineva pe robul său sau pe slujnica sa cu toiagul, și ei vor muri sub mâna lui, aceia trebuie să fie răzbunați;
\par 21 Iar de vor mai trăi o zi sau două, ei nu trebuie răzbunați, că sunt plătiți cu argintul stăpânului lor.
\par 22 De se vor bate doi oameni și vor lovi o femeie însărcinată și aceasta va lepăda copilul său fără altă vătămare, să se supună cel vinovat la despăgubirea ce o va cere bărbatul acelei femei și el va trebui să plătească potrivit cu hotărârea judecătorilor.
\par 23 Iar de va fi și altă vătămare, atunci să plătească suflet pentru suflet,
\par 24 Ochi pentru ochi, dinte pentru dinte, mână pentru mână, picior pentru picior,
\par 25 Arsură pentru arsură, rană pentru rană, vânătaie pentru vânătaie.
\par 26 Iar de va lovi cineva pe robul său în ochi iar pe slujnica sa o va lovi în ochi și ea îl va pierde, să-l lase liber ca despăgubire pentru ochi.
\par 27 Și de va pricinui căderea unui dinte al robului său sau al roabei sale, să le dea drumul pentru acel dinte.
\par 28 Dacă un bou va împunge de moarte bărbat sau femeie, boul să fie ucis cu pietre și carnea lui să nu se mănânce, iar stăpânul boului să fie nevinovat.
\par 29 Iar dacă boul a fost împungător cu o zi sau cu două sau cu trei înainte, și stăpânul lui, fiind vestit despre aceasta, nu l-a închis și boul a ucis bărbat sau femeie, boul să fie ucis cu pietre și stăpânul lui să fie dat morții.
\par 30 Dacă însă i se va pune stăpânului preț de răscumpărare, pentru sufletul său, ce va fi pus asupra lui aceea va și plăti.
\par 31 Tot după această lege să se urmeze, de va împunge boul băiat sau fată.
\par 32 Iar de va împunge boul rob sau roabă, să se plătească stăpânului acestora treizeci de sicli de argint, iar boul să fie ucis cu pietre.
\par 33 De va săpa cineva o fântână sau va descoperi o fântână și nu o va acoperi și va cădea în ea un bou sau un asin,
\par 34 Stăpânul fântânii trebuie să plătească argint stăpânului lor, iar boul sau asinul să fie al lui.
\par 35 Iar dacă boul cuiva va împunge boul altuia și va muri, să se vândă boul cel viu și prețul să-l împartă pe din două; de asemenea și pe cel ucis să-l împartă pe din două.
\par 36 Iar de s-a știut că boul a fost împungător de multă vreme, dar stăpânul lui, fiind înștiințat despre aceasta, nu l-a păzit, atunci acesta trebuie să plătească bou pentru bou, iar cel ucis să fie al lui".

\chapter{22}

\par 1 "De va fura cineva un bou sau o oaie și le va junghia, sau le va vinde, să plătească cinci boi pentru un bou și patru oi pentru oaie!
\par 2 Dacă furul va fi prins spărgând și va fi lovit încât să moară, cel ce l-a lovit nu va fi vinovat de moartea lui.
\par 3 Iar de se va face aceasta după ce a răsărit soarele, va fi vinovat și pentru ucidere va fi ucis. Cel ce a furat va trebui să plătească tot și de nu are cu ce, să fie vândut el pentru plata celor furate.
\par 4 Iar de se va prinde furul și cele furate se vor găsi la el vii, fie bou, oaie sau asin, să plătească îndoit.
\par 5 De va pricinui cineva pagubă într-o țarină sau vie, lăsând vitele să pască, stricând țarina altuia, să plătească din țarina sa potrivit cu stricăciunea; iar de a păscut toată țarina, să plătească despăgubire cu ce are mai bun în țarina sa și cu ce are mai bun în via sa.
\par 6 De va izbucni foc și va cuprinde spini și, întinzându-se, va arde clăi, sau snopi, sau holdă, să plătească despăgubire îndoit cel ce a aprins focul.
\par 7 De va da cineva vecinului său argint sau lucruri să le păstreze și acelea vor fi furate din casa acestui om, de se va găsi furul, să le plătească îndoit;
\par 8 Iar de nu se va găsi furul, să vină stăpânul casei înaintea judecătorilor și să jure că nu și-a întins mâna asupra lucrului aproapelui său.
\par 9 Pentru tot lucrul care s-ar putea fura: bou sau asin, oaie sau haină, sau orice lucru pierdut, despre care va zice cineva: "Acesta este al meu!" pricina amândurora trebuie să fie adusă înaintea judecătorilor, și cel ce va fi osândit de judecători să plătească aproapelui său îndoit.
\par 10 De va da cineva spre pază aproapelui său asin sau bou, sau oaie, sau alt dobitoc și va muri, sau va fi vătămat, sau luat fără să știe cineva,
\par 11 Să facă amândoi jurământ înaintea Domnului, că cel ce a luat pe seama sa nu și-a întins mâna asupra lucrului aproapelui său, și așa stăpânul trebuie să primească jurământul, iar celălalt nu va avea să-l despăgubească;
\par 12 Iar de se va fura de la el, să plătească stăpânului despăgubire.
\par 13 Dacă însă va fi sfâșiat de fiară, să-i aducă ceea ce a rămas ca mărturie și nu va plăti despăgubire pentru vita sfâșiată.
\par 14 De va împrumuta cineva de la aproapele său vită și aceea se va vătăma sau va pieri, și stăpânul ei nu va fi cu ea, să o plătească;
\par 15 Iar dacă stăpânul ei a fost cu ea, să nu o plătească. Iar dacă a fost închiriată cu bani, se va socoti pentru chiria aceea.
\par 16 De va amăgi cineva o fată nelogodită și se va culca cu ea, să o înzestreze și să o ia de soție;
\par 17 Iar dacă tatăl ei se va feri și nu va voi să o dea lui de femeie, atunci el să plătească tatălui fetei bani câți se cer pentru înzestrarea fetelor.
\par 18 Pe vrăjitori să nu-i lăsați să trăiască!
\par 19 Tot cel ce se împreună cu dobitoc să fie omorât.
\par 20 Cel ce jertfește la alți dumnezei, afară de Domnul, să se piardă.
\par 21 Pe străin să nu-l strâmtorezi, nici să-l apeși, căci și voi ați fost străini în pământul Egiptului.
\par 22 La nici o văduvă și la nici un orfan să nu le faceți rău!
\par 23 Iar de le veți face rău și vor striga către Mine, voi auzi plângerea lor,
\par 24 Și se va aprinde mânia Mea și vă voi ucide cu sabia și vor fi femeile voastre văduve și copiii voștri orfani.
\par 25 De vei împrumuta bani fratelui sărac din poporul Meu, să nu-l strâmtorezi și să nu-i pui camătă.
\par 26 De vei lua zălog haina aproapelui tău, să i-l întorci până la asfințitul soarelui,
\par 27 Căci aceasta este învelitoarea lui, aceasta este singura îmbrăcăminte pentru goliciunea sa. fn ce va dormi el? Deci de va striga către Mine, îl voi auzi, pentru că sunt milostiv.
\par 28 Pe judecători să nu-i grăiești de rău și pe căpetenia poporului tău să nu o hulești!
\par 29 Nu întârzia a-Mi aduce pârga ariei tale și a teascului tău; pe cel întâi-născut din fiii tăi să Mi-l dai Mic!
\par 30 Asemenea să faci cu boul tău, cu oaia ta și cu asinul tău: șapte zile să fie ei la mama lor, iar în ziua a opta să Mi le dai Mie!
\par 31 Să-Mi fiți popor sfânt; să nu mâncați carnea dobitocului sfâșiat de fiară în câmp, ci s-o aruncați la câini!"

\chapter{23}

\par 1 "Să nu iei aminte la zvon deșert; să nu te unești cu cel nedrept, ca să fii martor mincinos!
\par 2 Să nu te iei după cei mai mulți, ca să faci rău; și la judecată să nu urmezi celor mai mulți, ca să te abați de la dreptate;
\par 3 Nici săracului să nu-i fii părtinitor la judecată!
\par 4 De vei întâlni boul dușmanului tău sau asinul lui rătăcit, să-l întorci și să I-l duci!
\par 5 De vei vedea asinul vrăjmașului tău căzut sub povară, să nu-l treci cu vederea, ci să-l ridici împreună cu el.
\par 6 Să nu judeci strâmb pricina săracului tău!
\par 7 De orice cuvânt mincinos să te ferești; să nu ucizi pe cel nevinovat și drept, căci Eu nu voi ierta pe nelegiuit.
\par 8 Daruri să nu primești, căci darurile orbesc ochii celor ce văd și strâmbă pricinile cele drepte.
\par 9 Pe străin să nu-l obijduiești, nici să nu-l strâmtorezi, căci voi știți cum e sufletul pribeagului, că și voi ați fost pribegi în țara Egiptului.
\par 10 Șase ani să semeni țarina ta și să aduni roadele ei,
\par 11 Iar în al șaptelea, las-o să se odihnească; și se vor hrăni săracii poporului tău, iar rămășițele le vor mânca fiarele câmpului. Așa să faci și cu via ta și cu măslinii tăi.
\par 12 În șase zile să-ți faci treburile tale, iar în ziua a șaptea să te odihnești, ca să se odihnească și boul tău și asinul tău și ca să răsufle fiul roabei tale și străinul care e cu tine.
\par 13 Păziți toate câte v-am spus și numele altor dumnezei să nu le pomeniți, nici să se audă ele din gura voastră.
\par 14 De trei ori în an să-Mi prăznuiești:
\par 15 Să ții sărbătoarea azimelor. șapte zile să mănânci azime în timpul lunii lui Aviv cum ți-am poruncit, "ci în acea lună ai ieșit din Egipt; să nu te înfățișezi înaintea Mea cu mâna goală.
\par 16 Să ții apoi sărbătoarea secerișului și a strângerii celor dintâi roade ale tale, pe care le-ai semănat în țarina ta, și sărbătoarea strângerii roadelor toamna, când aduni de pe câmp munca ta.
\par 17 De trei ori pe an să se înfățișeze înaintea Domnului Dumnezeului tău toți cei de parte bărbătească ai tăi.
\par 18 Când voi alunga neamurile de la fața ta și voi lărgi hotarele tale, să nu torni sângele jertfei tale pe dospit, nici grăsimea de la jertfa Mea cea de la sărbători să nu rămână pe a doua zi.
\par 19 Pârga. din roadele țarinii tale să o aduci în casa Domnului Dumnezeului tău! Să nu fierbi iedul în laptele mamei lui!
\par 20 Iată Eu trimit înaintea ta pe îngerul Meu, ca să te păzească în cale și să te ducă la pământul acela pe care l-am pregătit pentru tine.
\par 21 Ia aminte la tine însuți; să-l asculți și să nu-i fi necredincios, că nu te va ierta, pentru că numele Meu este în el.
\par 22 De vei asculta cu luare aminte glasul său și vei face toate câte îți poruncesc și de vei păzi legământul Meu, Îmi veți fi popor ales dintre toate neamurile, că al Meu este tot pământul, iar voi Îmi veți fi preoție împărătească și neam sfânt. Spune cuvintele acestea fiilor lui Israel: De veți asculta cu luare aminte glasul îngerului Meu și veți împlini toate câte vă voi spune, voi fi vrăjmaș vrăjmașilor tăi și potrivnicilor tăi le voi fi potrivnic.
\par 23 Când va merge înaintea ta îngerul Meu, povățuitorul tău, și te va duce la Amorei, la Hetei, la Ferezei, ia Canaanei, la Gherghesei, la Hevei și la Iebusei, și-i voi stârpi pe aceștia de la fața voastră,
\par 24 Atunci să nu te închini la dumnezeii lor, nici să le slujești, nici să faci după faptele acelora, ci să-i zdrobești de tot și să strici stâlpii lor.
\par 25 Să slujești numai Domnului Dumnezeului tău și El va binecuvânta pâinea ta, vinul tău, apa ta și voi abate bolile de la voi.
\par 26 În fața ta nu va fi femeie care să nască înainte de vreme sau stearpă; și voi umple numărul zilelor tale.
\par 27 Groază voi trimite înaintea ta și voi îngrozi de tot pe poporul asupra căruia veți merge și voi pune pe fugă pe toți vrăjmașii tăi.
\par 28 Trimite-voi înaintea ta viespi și vor alunga de la fața voastră pe Amorei, pe Hevei, pe Iebusei, pe Canaanei și pe Hetei.
\par 29 Dar nu-i voi alunga de la fața voastră într-un an, ca să nu se pustiiască pământul și ca să nu se înmulțească asupra ta fiarele sălbatice;
\par 30 Ci-i voi alunga încetul cu încetul, până ce vă veți înmulți și veți lua în stăpânire pământul acela.
\par 31 Întinde-voi hotarele tale de la Marea Roșie până la Marea Filistenilor și de la pustie până la râul cel mare al Eufratului, căci voi da în mâinile voastre pe locuitorii pământului acestuia și-i voi alunga de la fața ta.
\par 32 Să nu vă amestecați și să nu faceți legământ cu ei, nici cu dumnezeii lor.
\par 33 Să nu locuiască ei în țara voastră, ca să nu vă facă să păcătuiți împotriva Mea; că de veți sluji dumnezeilor lor, aceștia vor fi cursă pentru voi".

\chapter{24}

\par 1 Apoi a zis Dumnezeu iarăși către Moise: "Suie-te la Domnul, tu și Aaron, Nadab, Abiud și șaptezeci dintre bătrânii lui Israel și vă închinați Domnului de departe.
\par 2 Numai Moise singur să se apropie de Domnul, iar ceilalți să nu se apropie; poporul de asemenea să nu se suie cu el!"
\par 3 A venit deci Moise și a spus poporului toate cuvintele Domnului și legile. Atunci a răspuns tot poporul într-un glas și a zis: "Toate cuvintele pe care le-a grăit Domnul le vom face și le vom asculta!"
\par 4 Iar Moise a scris toate cuvintele Domnului. Și el, sculându-se dis-de-dimineață, a zidit jertfelnic sub munte și a pus doisprezece stâlpi, după cele douăsprezece seminții ale lui Israel.
\par 5 A trimis apoi tineri dintre fiii lui Israel, de au adus aceștia arderi de tot și au jertfit vitei, ca jertfă de izbăvire Domnului Dumnezeu.
\par 6 Atunci Moise, luând jumătate din sânge, l-a turnat într-un vas, iar cu cealaltă jumătate de sânge a stropit jertfelnicul.
\par 7 După aceea, luând cartea legământului, a citit în auzul poporului; iar ei au zis: "Toate câte a grăit Domnul le vom face și le vom asculta!"
\par 8 După aceea, luând Moise sângele, a stropit poporul, zicând: "Acesta este sângele legământului, pe care l-a încheiat Domnul cu voi, după toate cuvintele acestea".
\par 9 Apoi s-a suit Moise și Aaron, Nadab, Abiud și șaptezeci dintre bătrânii lui Israel
\par 10 Și au văzut locul unde stătea Dumnezeul lui Israel; sub picioarele Lui era ceva, ce semăna cu un lucru de safir, curat și limpede ca seninul cerului.
\par 11 Dar El n-a întins mâna Sa împotriva aleșilor lui Israel, iar ei au văzut pe Dumnezeu, apoi au mâncat și au băut.
\par 12 Și a zis Domnul către Moise: "Suie-te la Mine în munte și fii acolo, că am să-ți dau table de piatră, legea și poruncile, pe care le-am scris Eu pentru învățătura lor!"
\par 13 Atunci, sculându-se Moise împreună cu Iosua, slujitorul său, s-a suit în muntele Domnului;
\par 14 Iar bătrânilor le-a zis: "Rămâneți aici până ne vom întoarce la voi. Iată Aaron și Or sunt cu voi; de va avea cineva pricină, să vină la ei".
\par 15 S-a suit deci Moise și Iosua în munte și un nor a acoperit muntele.
\par 16 Slava Domnului s-a pogorât pe Muntele Sinai și l-a acoperit norul șase zile, iar în ziua a șaptea a strigat Domnul pe Moise din mijlocul norului.
\par 17 Chipul slavei Domnului de pe vârful muntelui era în ochii fiilor lui Israel, ca un foc mistuitor.
\par 18 Și s-a suit Moise pe munte și a intrat în mijlocul norului; și a stat Moise pe munte patruzeci de zile și patruzeci de nopți.

\chapter{25}

\par 1 Atunci a grăit Dumnezeu cu Moise și a zis:
\par 2 "Spune fiilor lui Israel să-Mi aducă prinoase: de la tot omul, pe care-l lasă inima să dea, să primești prinoase pentru Mine.
\par 3 Iar prinoasele ce vei primi de la ei sunt acestea: aur, argint și aramă;
\par 4 Mătase violetă, purpurie și stacojie, în și păr de capră.
\par 5 Piei de berbec vopsite roșu, piei de vițel de mare și lemn de salcâm;
\par 6 Untdelemn pentru candele, aromate pentru mirul de uns și pentru miresmele de tămâiere;
\par 7 Piatră de sardiu și pietre de pus la efod și la hoșen.
\par 8 Din acestea să-Mi faci locaș sfânt și voi locui în mijlocul lor.
\par 9 Cortul și toate vasele și obiectele lui să le faci după modelul ce-ți voi arăta Eu; așa să le faci!
\par 10 Chivotul legii să-l faci din lemn de salcâm: lung de doi coți și jumătate, larg de un cot și jumătate și înalt de un cot și jumătate.
\par 11 Să-l fereci cu aur curat, și pe dinăuntru și pe din afară. Sus, împrejurul lui, să-i faci cunună împletită de aur.
\par 12 Apoi să torni pentru el patru inele de aur și să le prinzi în cele patru colțuri de jos ale lui: două inele pe o latură și două inele pe cealaltă latură.
\par 13 Să faci pârghii din lemn de salcâm și să le îmbraci cu aur.
\par 14 Și să vâri pârghiile prin inelele de pe laturile chivotului, încât cu ajutorul lor să se poarte chivotul.
\par 15 Pârghiile să fie necontenit în inelele chivotului.
\par 16 Iar în chivot să pui legea, pe care ji-o voi da.
\par 17 Să faci și capac la chivot, de aur curat, lung de doi coți și jumătate, și lat de un cot și jumătate.
\par 18 Apoi să faci doi heruvimi de aur; și să-i faci ca dintr-o bucată, ca și cum ar răsări din cele două capete ale capacului;
\par 19 Să pui un heruvim la un capăt și un heruvim la celălalt capăt al capacului.
\par 20 Și heruvimii să-i faci ca și cum ar ieși din capac. Heruvimii aceștia să fie cu aripile întinse pe deasupra capacului, acoperind cu aripile lor capacul, iar fețele să și le aibă unul spre altul; spre capac să fie fețele heruvimilor.
\par 21 Apoi să pui acest capac deasupra la chivot, iar în chivot să pui legea ce îți voi da.
\par 22 Acolo, între cei doi heruvimi de deasupra chivotului legii, Mă voi descoperi ție și îți voi grăi de toate, câte am a porunci prin tine fiilor lui Israel.
\par 23 Să faci apoi masă din lemn de salcâm: lungă de doi coți, lată de un cot, înaltă de un cot și jumătate.
\par 24 S-o îmbraci cu aur curat și să-i faci împrejur cunună de aur, împletită.
\par 25 Să mai faci împrejurul ei pervaz înalt de o palmă și împrejurul pervazului să faci cunună de aur.
\par 26 Să mai faci patru inele de aur și să prinzi cele patru inele sub cunună, în cele patru colțuri de la picioarele mesei.
\par 27 Inelele să fie în pervaz ca niște torți pentru pârghii, ca să se poarte cu ele masa.
\par 28 Iar pârghiile să le faci din lemn de salcâm, să le fereci cu aur curat și cu ele se va purta masa.
\par 29 Apoi să faci pentru ea talere, cădelnițe, pahare și cupe, ca să torni cu ele; acestea să le faci din aur curat.
\par 30 Iar pe masă să pui pâinile punerii înainte, care se vor afla pururea înaintea Mea.
\par 31 Să faci sfeșnic din aur curat. Sfeșnicul să-l faci bătut din ciocan: fusul, brațele, cupele, nodurile și florile lui să fie dintr-o bucată.
\par 32 Șase brațe să iasă pe cele două laturi ale lui: trei brațe ale sfeșnicului să fie pe o latură a lui și trei brațe ale sfeșnicului să fie pe cealaltă latură.
\par 33 El va avea la un braț trei cupe în forma florii de migdal, cu nodurile și florile lor, și la alt braț va avea trei cupe în forma florii de migdal, cu nodurile și florile lor. Așa vor avea toate cele șase brațe, ce ies din fusul sfeșnicului.
\par 34 Iar pe fusul sfeșnicului să fie patru cupe în forma florii de migdal, cu nodurile și florile lor:
\par 35 Un nod sub două brațe, un alt nod sub alte două brațe și un al treilea nod sub cele din urmă două brațe, iar în vârful fusului sfeșnicului să fie încă o cupă în forma florii de migdal cu nodul și floarea ei.
\par 36 Nodurile și ramurile acestea să fie dintr-o bucată cu sfeșnicul. El trebuie să fie lucrat tot cu ciocanul, dintr-o singură bucată de aur curat.
\par 37 Să-i faci șapte candele și să pui în el candelele acestea, ca să lumineze latura din fața lui.
\par 38 Să-i faci mucări și tăvițe de aur curat.
\par 39 Dintr-un talant de aur curat să se facă toate obiectele acestea.
\par 40 Vezi să faci acestea toate după modelul ce ți s-a arătat în munte".

\chapter{26}

\par 1 "Cortul însă să-l faci din zece covoare de in răsucit și de mătase violetă, stacojie și vișinie; în țesătura lor să faci chipuri de heruvimi alese cu iscusință.
\par 2 Lungimea fiecărui covor să fie de douăzeci și opt de coți, și lățimea fiecărui covor să fie de patru coți: toate covoarele să aibă aceeași măsură.
\par 3 Cinci covoare se vor uni la un loc și celelalte cinci iar se vor uni la un loc.
\par 4 Apoi să faci cheotori de mătase violetă, pe marginea covorului din capătul jumătății întâi de acoperiș, și tot asemenea cheotori să faci pe marginea covorului din urmă de la cealaltă jumătate de acoperiș.
\par 5 Cincizeci de cheotori să faci la un covor și pe marginea covorului ce are a se uni cu el să faci tot cincizeci de cheotori. Cheotorile acestea să răspundă unele cu altele.
\par 6 Să faci cincizeci de copci din aur și cu copcile acestea să unești cele două jumătăți de acoperiș, și așa va fi acoperișul cortului o singură bucată.
\par 7 Apoi să faci covoare din păr de capră, ca să acoperi cortul. Unsprezece covoare de acestea să faci.
\par 8 Lungimea unui covor să fie de treizeci de coți, și lățimea unui covor să fie de patru coți; cele unsprezece covoare să aibă toate aceeași măsură.
\par 9 Să unești între ele cinci covoare și celelalte șase covoare iar să le unești între ele. Jumătatea din al șaselea covor să o îndoi în partea de dinainte a cortului.
\par 10 Să faci cincizeci de cheotori pe marginea covorului din capătul unei jumătăți, ca să se poată uni cu cealaltă jumătate; și alte cincizeci de cheotori să faci la marginea celeilalte jumătăți de acoperiș, care trebuie unită cu cea dintâi.
\par 11 Să faci apoi cincizeci de copci din aramă și să vâri copcile acestea în cheotori și să unești cele două jumătăți de acoperiș, ca să fie unul singur.
\par 12 Iar prisosul de acoperiș, o jumătate de covor, care prisosește de la acoperișul cortului, să atârne în partea dindărăt a cortului.
\par 13 Partea însă din lungimea acoperișului, care prisosește de o parte și de alta a cortului, să atârne peste pereții cortului, de o parte un cot și de alta un cot, ca să-i apere.
\par 14 După aceea să faci cortului un acoperiș de piei roșii de berbec și încă un acoperiș de piei de vițel de mare pe deasupra.
\par 15 Să faci apoi pentru cort scânduri din lemn de salcâm, ca să stea în picioare.
\par 16 Fiecare scândură să o faci lungă de zece coți și lată de un cot și jumătate să fie fiecare scândură.
\par 17 O scândură să aibă două cepuri la capăt, unul în dreptul altuia. Așa să faci la toate scândurile cortului.
\par 18 Și scânduri de acestea pentru cort să faci douăzeci, pentru latura dinspre miazăzi.
\par 19 Sub aceste douăzeci de scânduri să faci patruzeci de postamente de argint: câte două postamente sub o scândură, pentru cele două cepuri ale ei și două postamente pentru altă scândură, pentru cele două cepuri ale ei.
\par 20 Douăzeci de scânduri să faci pentru cealaltă latură, dinspre miazănoapte.
\par 21 Și pentru acestea să faci patruzeci de postamente de argint, câte două postamente sub o scândură și două postamente pentru altă scândură;
\par 22 Iar pentru partea dindărăt a cortului, care vine spre asfințit, să faci șase scânduri.
\par 23 Și două scânduri să faci pentru unghiurile cortului din partea dindărăt a lui.
\par 24 Acestea să fie de două ori mai groase și sus unite prin câte un inel. Așa trebuie să fie amândouă la fel pentru amândouă unghiurile.
\par 25 Și așa vor fi opt scânduri în partea dindărăt a cortului și pentru cele șaisprezece postamente de argint, câte două postamente sub fiecare scândură, pentru cele două cepuri ale ei.
\par 26 Să faci apoi pârghii din lemn de salcâm: cinci pârghii pentru scândurile de pe o latură a cortului,
\par 27 Cinci pârghii pentru scândurile de pe cealaltă latură a cortului și cinci pârghii pentru scândurile din partea de la fundul cortului, care vine spre asfințit.
\par 28 Iar pârghia din mijloc va trece prin scânduri de la un capăt la celălalt al cortului.
\par 29 Scândurile să le îmbraci cu aur; inelele, prin care se vâră pârghiile, să le faci de aur și să îmbraci cu aur și pârghiile.
\par 30 Și așa să înjghebezi cortul după modelul care ti s-a arătat în acest munte.
\par 31 Să faci o perdea de in răsucit și de mătase violetă, stacojie și vișinie, răsucită, iar în țesătura ei să aibă chipuri de heruvimi alese cu iscusință;
\par 32 Și s-o atârni cu verigi de aur pe patru stâlpi din lemn de salcâm, îmbrăcați cu aur și așezați pe patru postamente de argint.
\par 33 După ce vei prinde perdeaua în copci, să aduci acolo după perdea chivotul legii și perdeaua vă va despărți astfel sfânta de sfânta sfintelor.
\par 34 La chivotul legii din sfânta sfintelor să pui capacul.
\par 35 Iar dincoace, în afară de perdea, să pui masa și în fața mesei să pui sfeșnicul; în partea de miazăzi a cortului să-l pui. Masa însă s-o pui în partea de miazănoapte a cortului.
\par 36 Apoi să faci o perdea la ușa cortului, de mătase violetă, stacojie și vișinie, răsucită și de in răsucit, cu flori alese în țesătura ei.
\par 37 Pentru perdeaua aceasta să faci cinci stâlpi din lemn de salcâm și să-i îmbraci cu aur. La ei să faci verigi de aur și să torni pentru ei cinci postamente de aramă".

\chapter{27}

\par 1 "Să faci un jertfelnic din lemn de salcâm, lung de cinci coți, lat de cinci coți. Jertfelnicul să fie în patru colțuri și înălțimea lui de trei coți.
\par 2 În cele patru colțuri ale lui să faci coarne. Coarnele să fie ca răsărite din el și să-l îmbraci cu aramă.
\par 3 Apoi să-i faci căldări pentru pus cenușa, lopățele, lighene, furculițe și clește. Toate uneltele acestea să le faci de aramă.
\par 4 Să faci jertfelnicului o împletitură, un fel de cămașă din sârmă de aramă, și la împletitura aceasta să faci, în cele patru colțuri ale ei, patru inele de aramă.
\par 5 Să îmbraci cu cămașa aceasta partea de jos a jertfelnicului, ca să fie împletitura până la jumătatea jertfelnicului.
\par 6 Să faci pentru jertfelnic pârghii, din lemn de salcâm, și să le îmbraci cu aramă.
\par 7 Drugii aceștia să-i vâri prin inele, pe o parte și pe alta a jertfelnicului, ca să poată fi purtat.
\par 8 Iar jertfelnicul să-l faci de scânduri, gol înăuntru. După cum ți s-a arătat în munte, așa să-l faci.
\par 9 Cortului să-i faci curte. Pe partea dinspre miazăzi, perdelele să fie de in răsucit, lungi de o sută de coti numai pe partea aceasta.
\par 10 Pentru ele să faci douăzeci de stâlpi și pentru ei douăzeci de postamente de aramă; cârligele la stâlpi și verigile lor să fie de argint.
\par 11 Tot așa și pe latura dinspre miazănoapte să fie perdele de o sută de coți în lungime și la ele douăzeci de stâlpi, iar sub ei douăzeci de postamente de aramă; cârligele și verigile stâlpilor să fie de argint.
\par 12 În latul curții, pe partea dinspre asfințit, să fie perdelele lungi de cincizeci de coți și la ele zece stâlpi și la stâlpi zece postamente.
\par 13 Tot de cincizeci de coți să fie perdelele din latul curții în partea dinainte, cea dinspre răsărit; și la ele zece stâlpi și sub ei zece postamente.
\par 14 Din aceștia, cincisprezece coți la un capăt al laturii să fie perdelele, cu trei stâlpi ai lor și cu trei postamente,
\par 15 Și la celălalt capăt cincisprezece coți să fie perdele la fel, cu trei stâlpi ai lor și cu trei postamente.
\par 16 Iar la mijloc, poarta curții, largă da douăzeci de coți, să aibă perdele de lână violetă, stacojie și vișinie, răsucită și de in răsucit, cu patru stâlpi și patru postamente.
\par 17 Toți stâlpii curții împrejur să fie ferecați cu argint și uniți cu legături de argint, iar postamentele lor să fie de aramă.
\par 18 Așadar lungimea curții să fie de o sută de coți, lățimea peste tot de cincizeci de coți, înălțimea de cinci coți, perdelele să fie de in răsucit, iar postamentele stâlpilor de aramă.
\par 19 Toate lucrurile, toate uneltele și toți țărușii curții să fie de aramă.
\par 20 Să poruncești fiilor lui Israel să-ți aducă untdelemn curat pentru luminat, stors din măsline, ca să ardă sfeșnicul în toată vremea, în cortul adunării, în fața perdelei celei de dinaintea chivotului legii.
\par 21 Sfeșnicul îl va aprinde Aaron și fiii lui, de seara până dimineața, înaintea Domnului. Aceasta e lege veșnică pentru fiii lui Israel din neam în neam".

\chapter{28}

\par 1 "Să iei la tine pe Aaron, fratele tău, și pe fiii lui, ca dintre fiii lui Israel să-Mi fie preoți Aaron și fiii lui Aaron: Nadab, Abiud, Eleazar și Itamar.
\par 2 Să faci lui Aaron, fratele tău, veșminte sfințite, spre cinste și podoabă.
\par 3 Să spui dar, la toți cei iscusiți, pe care i-am umplut de duhul înțelepciunii și al priceperii, să facă lui Aaron veșminte sfințite pentru ziua sfințirii lui, cu care să-Mi slujească.
\par 4 Iată dar veșmintele ce trebuie să facă: hoșen, efod, meil, hiton, chidar și cingătoare. Acestea sunt veșmintele sfințite, ce trebuie să facă ei lui Aaron, fratele tău, și fiilor lui, ca să-Mi slujească ei ca preoți.
\par 5 Pentru acestea vor lua aur curat și mătăsuri violete, purpurii și stacojii și în subțire
\par 6 Și vor face efod lucrat cu iscusință din fire de aur, de mătase violetă, stacojie și vișinie, răsucită și de in răsucit.
\par 7 Acesta va fi din două bucăți: una pe piept și alta pe spate, unite pe umeri cu două încheietori.
\par 8 Cingătoarea efodului, care vine peste el, să fie lucrată la fel cu el, din fire de aur curat, de mătase violetă, stacojie și vișinie, răsucită și de in răsucit
\par 9 Apoi să iei două pietre, amândouă pietrele să fie de smarald, și să sapi pe ele numele fiilor lui Israel:
\par 10 Șase nume pe o piatră și celelalte șase nume pe cealaltă piatră, după rânduiala în care s-au născut ei.
\par 11 Cum fac săpătorii în piatră, care sapă peceți, așa să fie săpătura pe cele două pietre cu numele fiilor lui Israel și să așezi pietrele în cuibulețe de aur curat.
\par 12 Aceste două pietre să le pui încheietori la efod. Pietrele acestea vor fi spre pomenirea fiilor lui Israel și Aaron va purta numele fiilor lui Israel, spre pomenire înaintea Domnului, pe amândoi umerii săi.
\par 13 Să faci cuibulețe de aur curat.
\par 14 Apoi să faci două lănțișoare tot de aur curat; acestea să le faci, răsucite ca sfoara; și să prinzi lănțișoarele cele răsucite de cuibulețele de la încheietorile efodului, în partea de dinainte.
\par 15 Să faci hoșenul judecății, lucrat cu iscusință, la fel cu efodul: din fire de aur, de mătase violetă, stacojie, vișinie și de in răsucit.
\par 16 Acesta să fie îndoit, în patru colțuri, lung de o palmă și lat de o palmă.
\par 17 Pe el să așezi o înfloritură de pietre scumpe, înșirate în patru rânduri. Un rând de pietre să fie: un sardeon, un topaz și un smarald; acesta e rândul întâi.
\par 18 În rândul al doilea: un rubin, un safir și un diamant;
\par 19 În rândul al treilea: un opal, o agată și un ametist;
\par 20 Și în rândul al patrulea: un hrisolit, un onix și un iaspis. Acestea trebuie să fie așezate după rânduiala lor în cuibulețe de aur.
\par 21 Pietrele acestea trebuie să fie în număr de douăsprezece, după numărul numelor celor doisprezece fii ai lui Israel, înșirate pe cele două pietre de pe umeri, după numele lor și după rânduiala în care s au născut ei. Pe fiecare trebuie să sapi, ca pe pecete, câte un nume din numărul celor douăsprezece seminții.
\par 22 Apoi să faci pentru hoșen lănțișoare de aur curat, lucrat răsucit, ca sfoara.
\par 23 Să mai faci pentru hoșen două verigi de aur și aceste două verigi de aur să le prinzi de cele două colțuri de sus ale hoșenului;
\par 24 Să introduci cele două lănțișoare împletite de aur în cele două verigi din cele două colțuri ale hoșenului
\par 25 Și să prinzi celelalte două capete ale lănțișoarelor de cuibulețele efodului de pe umeri, în partea de dinainte.
\par 26 Și să mai faci două verigi de aur și să le prinzi de colțurile de jos ale hoșenului, care cad pe cingătoarea efodului.
\par 27 Apoi să mai faci încă două verigi de aur și să le prinzi de cele două margini de jos ale efodului, pe partea de dinainte, deasupra cingătorii efodului
\par 28 Și să prinzi verigile hoșenului de verigile efodului cu un șnur de mătase albastră, ca să stea peste cingătoarea efodului și ca hoșenul să nu se miște de pe efod.
\par 29 Și va purta Aaron, când va intra în cortul adunării, numele fiilor lui Israel pe hoșenul judecății, la inima sa, spre veșnică pomenire înaintea Domnului.
\par 30 În hoșenul judecății să pui Urim și Tumim; și vor fi acestea la inima lui Aaron, când va intra el în cortul adunării să se înfățișeze înaintea Domnului. Astfel va purta Aaron pururea la inima sa judecata fiilor lui Israel, înaintea Domnului.
\par 31 Să faci apoi meilul de sub efod tot de mătase vișinie.
\par 32 Acesta va avea la mijloc, sus, o deschizătură pentru cap și deschizătura să aibă împrejur un guler țesut ca platoșa, ca să nu se rupă.
\par 33 Iar pe la poale îi vei face de jur împrejur ciucuri tot de mătase violetă, stacojie, vișinie și de in răsucit;
\par 34 Și printre ciucuri vei pune clopoței de aur de jur împrejur așa: un ciucure și un clopoțel de aur, un ciucure și un clopoțel de aur.
\par 35 Și acesta va fi pe Aaron în timpul slujbei, când va intra în cortul sfânt, înaintea Domnului, și când va ieși, ca să se audă sunetul clopoțeilor și să nu moară.
\par 36 Să faci după aceea o tăbliță șlefuită, de aur curat, și să sapi pe ea, cum se sapă pe pecete, cuvintele: "Sfințenia Domnului",
\par 37 Și s-o prinzi cu șnur de mătase violetă de chidar, așa ca să vină în partea de dinainte a chidarului.
\par 38 Aceasta va fi pe fruntea lui Aaron și Aaron va purta pe fruntea sa neajunsurile prinoaselor afierosite de fiii lui Israel și ale tuturor darurilor aduse de ei; ea va fi pururea pe fruntea lui, pentru a atrage bunăvoința Domnului spre ei.
\par 39 Hitonul să-l faci de in și tot de in să faci și mitra, iar cingătoarea să o faci brodată cu mătase de felurite culori.
\par 40 Să faci de asemenea și fiilor lui Aaron hitoane și cingători; și să le faci și turbane pentru cinste și podoabă.
\par 41 Să îmbraci cu acestea pe fratele tău Aaron și împreună cu el și pe fiii lui, să-i ungi, să-i întărești în slujbele lor și să-i sfințești, ca să-Mi fie preoți.
\par 42 Să le faci pantaloni de in, de la brâu până sub genunchi, ca să-și acopere goliciunea trupului lor;
\par 43 Aaron și fiii lui să se îmbrace când vor intra în cortul adunării sau când se vor apropia de jertfelnic, în sfânta, ca să slujească, pentru a nu-și atrage păcat asupra lor și să moară. Aceasta să fie lege veșnică pentru el și pentru urmașii lui e

\chapter{29}

\par 1 "Iată ce trebuie să săvârșești asupra lor, când îi vei sfinți să-Mi fie preoți: să iei un vițel din cireadă, doi berbeci fără meteahnă,
\par 2 Pâini nedospite, azime frământate cu untdelemn și turte nedospite, unse cu untdelemn. Acestea să le faci din făină de grâu aleasă.
\par 3 Să le pui într-un paner și să le aduci în paner la cortul adunării o dată cu vițelul și cu berbecii.
\par 4 Apoi să aduci pe Aaron și pe fiii lui la intrarea cortului adunării și să-i speli cu apă.
\par 5 Și, luând veșmintele sfinte, să îmbraci pe Aaron, fratele tău, cu hitonul și cu meilul, cu efodul și cu hoșenul, și să-l încingi peste efod;
\par 6 Să-i pui pe cap mitra, iar la mitră să prinzi diadema sfințeniei.
\par 7 Apoi să iei untdelemn de ungere și să-i torni pe cap și să-l ungi.
\par 8 După aceea să aduci și pe fiii lui și să-i îmbraci cu hitoane;
\par 9 Să-i încingi cu brâie și să le pui turbanele; și-Mi vor fi preoți în veac. Așa vei sfinți tu pe Aaron și pe fiii lui.
\par 10 Să aduci apoi vițelul înaintea cortului adunării și să-și pună Aaron și fiii lui mâinile pe capul vițelului, înaintea Domnului, la ușa cortului adunării.
\par 11 Să junghii vițelul înaintea Domnului, la ușa cortului adunării.
\par 12 Să iei din sângele vițelului și să pui cu degetul tău pe coarnele jertfelnicului, iar celălalt sânge să-l torni tot la temelia jertfelnicului.
\par 13 Apoi să iei toată grăsimea cea de pe măruntaie, seul de pe ficat, amândoi rărunchii și grăsimea de pe ei și să le arzi pe jertfelnic.
\par 14 Iar carnea vițelului, pielea lui și necurățeniile lui să le arzi cu foc afară din tabără. Căci e jertfă pentru păcat.
\par 15 După aceea să iei un berbec și să-și pună Aaron și fiii lui mâinile pe capul berbecului;
\par 16 Să junghii berbecul și, luând sângele lui, să stropești jertfelnicul de jur împrejur.
\par 17 Să tai apoi berbecul în bucăți, să speli cu apă măruntaiele și picioarele și să le pui pe bucăți lângă căpățâna lui.
\par 18 Și să arzi berbecul tot pe jertfelnic. Aceasta este ardere de tot Domnului, jertfă Domnului, mireasmă plăcută înaintea Lui.
\par 19 Să iei și celălalt berbec, să-și pună Aaron și fiii lui mâinile pe capul berbecului și să-l junghii;
\par 20 Să iei din sângele lui și să pui pe vârful urechii drepte a lui Aaron, pe vârful degetului mare al mâinii drepte, pe vârful degetului mare al piciorului drept și pe vârful urechilor drepte ale fiilor lui și pe vârful degetelor mari ale mâinilor drepte ale lor și pe vârful degetelor mari ale picioarelor drepte ale lor. Și să stropești cu sânge jertfelnicul pe toate părțile.
\par 21 Să iei din sângele de pe jertfelnic și din untdelemnul de ungere și să stropești asupra lui Aaron și asupra veșmintelor lui, asupra fiilor lui și asupra veșmintelor fiilor lui, și se va sfinți el și veșmintele lui, fiii lui și veșmintele fiilor lui. Iar celălalt sânge al berbecului să-l torni lângă altar împrejur.
\par 22 Apoi să iei din berbec grăsimea și coada lui, grăsimea ce acoperă măruntaiele, seul de pe ficat, amândoi rărunchii și grăsimea de pe ei și șoldul drept; pentru că acesta este berbecul pentru sfințirea în preot;
\par 23 Să mai iei o pâine, din cele cu untdelemn, o turtă cu untdelemn și o azimă din panerul ce este pus înaintea Domnului;
\par 24 Să le pui toate pe brațele lui Aaron și pe brațele fiilor lui, ca să le aducă, legănându-le, înaintea Domnului.
\par 25 Apoi să iei acestea din mâinile lor și să le arzi pe jertfelnic, ardere de tot, spre bună mireasmă înaintea Domnului; aceasta este jertfă Domnului.
\par 26 Să iei pieptul berbecului, care este pentru sfințirea lui Aaron, și să-l duci înaintea Domnului legănându-l; aceasta va fi partea ta.
\par 27 Să sfințești pieptul legănat și spata legănată, care au fost ridicate înaintea Domnului, din berbecul cel pentru sfințirea lui Aaron și a fiilor lui.
\par 28 Și acestea să fie prin lege veșnică pentru Aaron și fiii lui din cele ce aduc fiii lui Israel, căci acesta e dar ridicat din cele ce vor aduce fiii lui Israel ca jertfă de pace, darul ridicat Domnului.
\par 29 Veșmintele sfinte cele pentru Aaron să fie, după el, ale fiilor săi, și să fie unși și sfințiți, îmbrăcați cu ele.
\par 30 Marele preot dintre fiii lui, care-i va urma și care va intra în cortul adunării, ca să slujească în locul cel sfânt, se va îmbrăca cu ele șapte zile.
\par 31 Să iei apoi berbecul cel pentru sfințire și să fierbi carnea lui în locul cel sfânt;
\par 32 Și să mănânce Aaron și fiii lui carnea berbecului acestuia și pâinile cele din paner, la ușa cortului adunării,
\par 33 Că prin aceasta s-a făcut curățirea lor pentru a fi sfințiți și pentru a li se încredința preoția; nimeni altul să nu mănânce, că sunt sfințite.
\par 34 Iar de va rămâne din această carne de sfințire și din pâini pe a doua zi, să arzi rămășițele cu foc și să nu se mănânce, că este lucru sfințit.
\par 35 Deci așa să faci cu Aaron și cu fiii lui, după cum ti-am poruncit: șapte zile să tină sfințirea lor.
\par 36 Vițelul cel de jertfă pentru păcat să-l aduci în fiecare zi pentru curățire; jertfa pentru păcat s-o săvârșești pe jertfelnic pentru curățirea lui și să-l ungi pentru sfințirea lui.
\par 37 Șapte zile să cureți astfel jertfelnicul și să-l sfințești și va fi jertfelnicul sfințenie mare; tot ce se va atinge de el se va sfinți.
\par 38 Iată ce vei aduce tu pe jertfelnic: doi miei de un an fără meteahnă vei aduce în fiecare zi, jertfă necontenită:
\par 39 Un miel să-l aduci dimineața și celălalt miel să-l aduci seara.
\par 40 A zecea parte dintr-o efă de făină de grâu, frământată cu a patra parte dintr-un hin de untdelemn curat, iar pentru turnare, a patra parte de hin de vin, pentru un miel.
\par 41 Al doilea miel să-l aduci seara cu dar de făină, ca și pe cel de dimineață, și cu aceeași turnare de vin; să-l aduci jertfă Domnului întru miros cu bună mireasmă.
\par 42 Aceasta va fi jertfă necontenită în neamul vostru, la ușile cortului mărturiei, unde Mă voi arăta pe Mine Însumi vouă ca să grăiesc cu tine.
\par 43 Acolo Mă voi pogorî Eu Însumi la fiii lui Israel și se va sfinți locul acesta de slava Mea.
\par 44 Voi sfinți cortul adunării și jertfelnicul; pe Aaron și pe fiii lui de asemenea îi voi sfinți, ca să-Mi fie preoți.
\par 45 Și voi locui în mijlocul fiilor lui Israel și le voi fi Dumnezeu;
\par 46 Și vor cunoaște că Eu, Domnul, sunt Dumnezeul lor, Cel ce i-am scos din pământul Egiptului, ca să locuiesc în mijlocul lor și să le fiu Dumnezeu!"

\chapter{30}

\par 1 "Să faci de asemenea un jertfelnic de tămâiere, din lemn de salcâm.
\par 2 Dar să-l faci pătrat, lung de un cot și lat de un cot și înalt de doi coli; coarnele lui să fie din el.
\par 3 Să îmbraci cu aur curat partea lui de sus, pereții împrejur și coarnele lui; și să-i faci împrejur o cunună de aur împletită.
\par 4 Sub cununa lui împletită să-i faci două inele de aur curat și să le pui la două colțuri pe două laturi ale lui. Acestea să fie de băgat pârghiile pentru a-l purta cu ele.
\par 5 Pârghiile să i le faci din lemn de salcâm și să le îmbraci cu aur.
\par 6 Jertfelnicul să-l așezi în fața perdelei, care este dinaintea chivotului legii, unde am să Mă arăt Eu ție.
\par 7 Pe el Aaron va arde tămâie mirositoare în fiecare dimineață, când pregătește candelele.
\par 8 Când va aprinde Aaron seara candelele, iar va arde miresme. Această tămâiere neîntreruptă se va face pururea înaintea Domnului din neam în neam.
\par 9 Să nu aduceți pe el nici o ardere de tămâie străină, nici ardere de tot, nici dar de pâine, nici turnare să nu turnați pe el.
\par 10 Aaron va săvârși jertfa de curățire peste coarnele lui o dată pe an; cu sânge din jertfa de curățire cea pentru păcat îl va curăți el o dată pe an; în neamul vostru aceasta este mare sfințenie înaintea Domnului".
\par 11 Apoi a grăit Domnul cu Moise și a zis:
\par 12 "Când vei face numărătoarea fiilor lui Israel, la cercetarea lor, să dea fiecare Domnului răscumpărare ca să nu vină nici o nenorocire asupra lor în timpul număratului.
\par 13 Cel ce intră la numărătoare să dea jumătate de siclu, după siclul sfânt, care are douăzeci de ghere; deci darul Domnului va fi o jumătate de siclu.
\par 14 Tot cel ce intră la numărătoare, de la douăzeci de ani în sus, să aducă această dare Domnului.
\par 15 Să nu dea bogatul mai mult, nici săracul mai puțin de jumătate de siclu dar Domnului, pentru răscumpărarea sufletului.
\par 16 Să iei argintul de răscumpărare de la fiii lui Israel și să-l dai la trebuințele cortului adunării și va fi pentru fiii lui Israel spre pomenire înaintea Domnului, ca să cruțe sufletele voastre".
\par 17 Și iarăși a grăit Domnul cu Moise și a zis:
\par 18 "Să faci o baie de aramă, cu postament de aramă, pentru spălat; s-o pui între cortul adunării și jertfelnic și să torni într-însa apă.
\par 19 Aaron și fiii lui își vor spăla în ea mâinile și picioarele lor cu apă.
\par 20 Când trebuie să intre ei în cortul mărturiei, să se spele cu apă de aceasta, ca să nu moară; și când trebuie să se apropie de jertfelnic, ca să slujească și ca să aducă ardere de tot Domnului, să-și spele mâinile lor și picioarele lor cu apă, ca să nu moară.
\par 21 Și va fi aceasta rânduială veșnică pentru el și pentru urmașii lui, din neam în neam".
\par 22 Apoi a grăit Domnul cu Moise și a zis:
\par 23 "Să iei din cele mai bune mirodenii: cinci sute sicli de smirnă aleasă; jumătate din aceasta, adică două sute cincizeci sicli de scorțișoară mirositoare; două sute cincizeci sicli trestie mirositoare;
\par 24 Cinci sute sicli casie, după siclul sfânt, și untdelemn de măsline un hin,
\par 25 Și să faci din acestea mir pentru ungerea sfântă, mir alcătuit după meșteșugul făcătorilor de aromate; acesta va fi mirul pentru sfânta ungere.
\par 26 Să ungi cu el cortul adunării, chivotul legii și toate lucrurile din cort,
\par 27 Masa și toate cele de pe ea, sfeșnicul și toate lucrurile lui, jertfelnicul tămâierii,
\par 28 Jertfelnicul arderii de tot și toate lucrurile lui și baia și postamentul ei.
\par 29 Și să le sfințești pe acestea și va fi sfințenie mare; tot ce se va atinge de ele se va sfinți.
\par 30 Să ungi de asemenea și pe Aaron și pe fiii lui și să-i sfințești, ca să-Mi fie preoți.
\par 31 Iar fiilor lui Israel să le spui: Acesta va fi pentru voi mirul sfintei ungeri, în numele Meu, în neamul vostru.
\par 32 Trupurile celorlalți oameni să nu le ungi cu el și după chipul alcătuirii lui să nu vă faceți pentru voi mir la fel. Acesta este lucru sfânt și sfânt trebuie să fie și pentru voi.
\par 33 Cine își va face ceva asemănător lui, sau cine se va unge cu el din cei ce nu trebuie să se ungă, acela se va stârpi din poporul său".
\par 34 Apoi a zis Domnul către Moise: "Ia-ți mirodenii: stacte, oniha, halvan mirositor și tămâie curată, din toate aceeași măsură,
\par 35 Și fă din ele, cu ajutorul meșteșugului făcătorilor de aromate, un amestec de tămâiat, cu adaos de sare, curat și sfânt;
\par 36 Pisează-l mărunt și-l pune înaintea chivotului legii, în cortul adunării, unde am să Mă arăt ție. Aceasta va fi pentru voi sfințenie mare.
\par 37 Tămâie, alcătuită în felul acesta, să nu vă faceți pentru voi: sfințenie să vă fie ea pentru Domnul.
\par 38 Cine își va face asemenea amestec, ca să afume cu el, sufletul acela se va stârpi din poporul său".

\chapter{31}

\par 1 După aceea a grăit Domnul cu Moise și a zis:
\par 2 "Iată, Eu am rânduit anume pe Bețaleel, fiul lui Uri, fiul lui Or, din seminția lui Iuda,
\par 3 Și l-am umplut de duh dumnezeiesc, de înțelepciune, de pricepere, de știință și de iscusință la tot lucrul,
\par 4 Ca să facă lucruri de aur, de argint și de aramă, de mătase violetă, stacojie și vișinie, și de in răsucit,
\par 5 Să șlefuiască pietre scumpe pentru podoabe și să sape în lemn tot felul de lucruri.
\par 6 Și iată, i-am dat ca ajutor pe Oholiab, fiul lui Ahisamac, din seminția lui Dan, și am pus înțelepciune în mintea oricărui om iscusit, ca să facă toate câte ți-am poruncit:
\par 7 Cortul adunării, chivotul legii, capacul cel de deasupra lui și toate lucrurile cortului;
\par 8 Masa și toate vasele ei; sfeșnicul cel de aur curat cu toate obiectele lui și jertfelnicul tămâierii;
\par 9 Jertfelnicul pentru arderile de tot cu toate obiectele lui; baia și postamentul ei;
\par 10 țesăturile pentru înveliș, veșmintele sfințite pentru Aaron preotul și veșmintele de slujbă pentru fiii lui;
\par 11 Mirul pentru ungere și aromatele mirositoare pentru locașul cel sfânt; toate le vor face ei așa, cum ti-am poruncit Eu ție".
\par 12 Și a mai vorbit Domnul cu Moise și a zis:
\par 13 "Spune fiilor lui Israel așa: Băgați de seamă să păziți zilele Mele de odihnă, căci acestea sunt semn între Mine și voi din neam în neam, ca să știți că Eu sunt Domnul, Cel ce vă sfințește.
\par 14 Păziți deci ziua de odihnă, căci ea este sfântă pentru voi. Cel ce o va întina, acela va fi omorât; tot cel ce va face într-însa vreo lucrare, sufletul acela va fi stârpit din poporul Meu;
\par 15 Șase zile să lucreze, iar ziua a șaptea este zi de odihnă, închinată Domnului; tot cel ce va munci în ziua odihnei va fi omorât.
\par 16 Să păzească deci fiii lui Israel ziua odihnei, prăznuind ziua odihnei din neam în neam, ca un legământ veșnic.
\par 17 Acesta este semn veșnic între Mine și fiii lui Israel, pentru că în șase zile a făcut Domnul cerul și pământul, iar în ziua a șaptea a încetat și S-a odihnit".
\par 18 După ce a încetat Dumnezeu de a grăi cu Moise, pe Muntele Sinai, i-a dat cele două table ale legii, table de piatră, scrise cu degetul lui Dumnezeu.

\chapter{32}

\par 1 Văzând însă poporul că Moise întârzie a se pogorî din munte, s-a adunat la Aaron și i-a zis: "Scoală și ne fă dumnezei, care să meargă înaintea noastră, căci cu omul acesta, cu Moise, care ne-a scos din țara Egiptului, nu știm ce s-a întâmplat".
\par 2 Iar Aaron le-a zis: "Scoateți cerceii de aur din urechile femeilor voastre, ale feciorilor voștri și ale fetelor voastre și-i aduceți la mine".
\par 3 Atunci tot poporul a scos cerceii cei de aur din urechile alor săi și i-a adus la Aaron.
\par 4 Luându-i din mâinile lor, i-a turnat în tipar și a făcut din ei un vițel turnat și l-a cioplit cu dalta. Iar ei au zis: "Iată, Israele, dumnezeul tău, care te-a scos din țara Egiptului!
\par 5 Văzând aceasta, Aaron a zidit înaintea lui un jertfelnic; și a strigat Aaron și a zis: "Mâine este sărbătoarea Domnului!"
\par 6 A doua zi s-au sculat ei de dimineață și au adus arderi de tot și jertfe de pace; apoi a șezut poporul de a mâncat și a băut și pe urmă s-a sculat și a jucat.
\par 7 Atunci a zis Domnul către Moise: "Grăbește de te pogoară de aici, căci poporul tău, pe care l-ai scos din țara Egiptului, s-a răzvrătit.
\par 8 Curând s-au abătut de la calea pe care le-am poruncit-o, și-au făcut un vițel turnat și s-au închinat la el, aducându-i jertfe și zicând: "Iată, Israele, dumnezeul tău, care te-a scos din țara Egiptului!"
\par 9 Și a mai zis Domnul către Moise: "Eu Mă uit la poporul acesta și văd că este popor tare de cerbice;
\par 10 Lasă-Mă dar acum să se aprindă mânia Mea asupra lor, să-i pierd și să fac din tine un popor mare!"
\par 11 Moise însă a rugat pe Domnul Dumnezeul său și a zis: "Să nu se aprindă, Doamne, mânia Ta asupra poporului Tău, pe care l-ai scos din țara Egiptului cu putere mare și cu brațul Tău cel înalt,
\par 12 Ca nu cumva să zică Egiptenii: I-a dus la pieire, ca să-i ucidă în munți și să-i șteargă de pe fala pământului. Întoarce-Ți iuțimea mâniei Tale, milostivește-Te și nu căuta la răutatea poporului Tău.
\par 13 Adu-ți aminte de Avraam, de Isaac și de Iacov, robii Tăi, cărora Te-ai jurat Tu pe Tine Însuți, zicând: Voi înmulți foarte tare neamul vostru, ca stelele cerului; și tot pământul acesta, de care v-am vorbit, îl voi da urmașilor voștri și-l vor stăpâni în veci!"
\par 14 Atunci a abătut Domnul pieirea ce zisese s-o aducă asupra poporului Său.
\par 15 După aceea Moise, întorcându-se; s-a pogorât din munte, cu cele două table ale legii în mână, scrise pe amândouă părțile lor - pe o parte și pe alta erau scrise.
\par 16 Tablele acestea erau lucrul lui Dumnezeu și scrierea era scrierea lui Dumnezeu, săpată pe table.
\par 17 Atunci, auzind Iosua glasul poporului răsunând, a zis către Moise: "În tabără se aud strigăte de război".
\par 18 Iar Moise a zis: "Acesta nu este glas de biruitori, nici glas de biruiți; ci eu aud glas de oameni beli".
\par 19 Iar după ce s-a apropiat de tabără, el a văzut vițelul și jocurile și, aprinzându-se de mânie, a aruncat din mâinile sale cele două table și le-a sfărâmat sub munte.
\par 20 Apoi luând vițelul, pe care-l făcuseră ei, l-a ars în foc, l-a făcut pulbere și, presărându-l în apă, a dat-o să o bea fiii lui Israel.
\par 21 După aceea a zis către Aaron: "Ce ti-a făcut poporul acesta, de l-ai vârât într-un păcat așa de mare?"
\par 22 Iar Aaron a răspuns lui Moise: "Să nu se aprindă mânia domnului meu! Tu știi pe poporul acesta că e răzvrătitor.
\par 23 Căci ei mi-au zis: Fă-ne dumnezei, care să meargă înaintea noastră, căci cu omul acesta, cu Moise, care ne-a scos din țara Egiptului, nu știm ce s-a întâmplat.
\par 24 Atunci eu le-am zis: Cine are aur să-l scoată. Și ei l-au scos și mi l-au dat mie și eu l-am aruncat în foc și a ieșit acest vițel".
\par 25 Moise, văzând că poporul acesta e neînfrânat, căci Aaron îngăduise să ajungă neînfrânat și de râs înaintea dușmanilor lui,
\par 26 A stat la intrarea taberei și a zis: "Cine este pentru Domnul să vină la mine!" Și s-au adunat la el toți fiii lui Levi.
\par 27 Iar Moise le-a zis: "Așa zice Domnul Dumnezeul lui Israel: Să-și încingă fiecare din voi sabia sa la șold și străbătând tabăra de la o intrare până la cealaltă, înainte și înapoi, să ucidă fiecare pe fratele său, pe prietenul său și pe aproapele său".
\par 28 Și au făcut fiii lui Levi după cuvântul lui Moise. În ziua aceea au căzut din popor ca la trei mii de oameni.
\par 29 Căci Moise le zisese fiilor lui Levi: "Afierosiți-vă astăzi mâinile voastre Domnului, fiecare prin fiul său sau prin fratele său, ca să vă trimită El astăzi binecuvântare!"
\par 30 Iar a doua zi a zis Moise către popor: "Ați făcut păcat mare; mă voi sui acum la Domnul să văd nu cumva voi șterge păcatul vostru".
\par 31 Și s-a întors Moise la Domnul și a zis: "O, Doamne, poporul acesta a săvârșit păcat mare, făcându-și dumnezeu de aur.
\par 32 Rogu-mă acum, de vrei să le ierți păcatul acesta, iartă-i; iar de nu, șterge-mă și pe mine din cartea Ta, în care m-ai scris!"
\par 33 Zis-a Domnul către Moise: "Pe acela care a greșit înaintea Mea îl voi șterge din cartea Mea.
\par 34 Iar acum mergi și du poporul acesta la locul unde ți-am zis. Iată îngerul Meu va merge înaintea ta și în ziua cercetării Mele voi pedepsi păcatul lor".
\par 35 Astfel a lovit Domnul poporul, pentru vițelul ce își făcuse, pe care-l turnase Aaron.

\chapter{33}

\par 1 Apoi a zis Domnul către Moise: "Du-te de aici tu și poporul tău, pe care l-ai scos din pământul Egiptului, și suiți-vă în pământul, pentru care M-am jurat lui Avraam, lui Isaac și lui Iacov, zicând: Urmașilor voștri îl voi da.
\par 2 Eu voi trimite înaintea ta pe îngerul Meu și va izgoni pe Canaanei, pe Amorei, pe Hetei, pe Ferezei, pe Gherghesei, pe Hevei și pe Iebusei,
\par 3 Și te voi duce în țara unde curge lapte și miere. Dar Eu nu voi merge în mijlocul vostru, ca să nu vă pierd pe cale, pentru că sunteți popor îndărătnic!
\par 4 Auzind însă acest cuvânt grozav, poporul a plâns cu jale și nimeni n-a mai pus pe sine podoabele sale.
\par 5 Căci Domnul zisese lui Moise: "Spune fiilor lui Israel: Voi sunteți popor îndărătnic. De voi merge Eu în mijlocul vostru, într-o clipeală vă voi pierde. Dezbrăcați acum de pe voi hainele voastre cele frumoase și podoabele voastre și voi vedea ce voi face cu voi".
\par 6 Atunci fiii lui Israel au dezbrăcat de pe ei podoabele lor și hainele cele frumoase când au plecat de la Muntele Horeb.
\par 7 Iar Moise, luându-și cortul, l-a întins afară din tabără, departe de ea, și-l numi cortul adunării; și tot cel ce căuta pe Domnul venea la cortul adunării, care se afla afară din tabără.
\par 8 Și când se îndrepta Moise spre cort, tot poporul se scula și sta fiecare la ușa cortului său și se uita după Moise, până ce intra el în cort.
\par 9 Iar după ce intra Moise în cort, se pogora un stâlp de nor și se oprea la intrarea cortului și Domnul grăia cu Moise.
\par 10 Și vedea tot poporul stâlpul cel de nor, care stătea la ușa cortului, și se scula tot poporul și se închina fiecare din ușa cortului său.
\par 11 Domnul însă grăia cu Moise față către față, cum ar grăi cineva cu prietenul său. După aceea Moise se întorcea în tabără; iar tânărul său slujitor Iosua, fiul lui Navi, nu părăsea cortul.
\par 12 Atunci a zis Moise către Domnul: "Iată, Tu îmi spui: Du pe poporul acesta, dar nu mi-ai descoperit pe cine ai să trimiți cu mine, deși mi-ai spus: Te cunosc pe nume și ai aflat bunăvoință înaintea ochilor Mei.
\par 13 Deci, de am aflat bunăvoință în ochii Tăi, arată-Te să Te văd, ca să cunosc și să aflu bunăvoință în ochii Tăi și că acest neam e poporul Tău".
\par 14 Și a zis Domnul către el: "Eu Însumi voi merge înaintea Ta și Te voi duce la odihnă!"
\par 15 Zis-a Moise către Domnul: "Dacă nu mergi Tu Însuți cu noi, atunci să nu ne scoți de aici;
\par 16 Căci prin ce se va cunoaște cu adevărat că eu și poporul Tău am aflat bunăvoință înaintea Ta? Au nu prin aceea ca Tu să fii însoțitorul nostru? Atunci eu și poporul Tău vom fi cei mai slăviți dintre toate popoarele de pe pământ".
\par 17 Și a zis Domnul către Moise: "Voi face și ceea ce zici tu, pentru că tu ai aflat bunăvoință înaintea Mea și te cunosc pe tine mai mult decât pe toți".
\par 18 Și Moise a zis: "Arată-mi slava Ta! "
\par 19 Zis-a Domnul către Moise: "Eu voi trece pe dinaintea ta toată slava Mea, voi rosti numele lui Iahve înaintea ta și pe cel ce va fi de miluit îl voi milui și cine va fi vrednic de îndurare, de acela Mă voi îndura".
\par 20 Apoi a adăugat: "Fața Mea însă nu vei putea s-o vezi, că nu poate vedea omul fața Mea și să trăiască".
\par 21 Și iarăși a zis Domnul: "Iată aici la Mine un loc: șezi pe stânca aceasta;
\par 22 Când va trece slava Mea, te voi ascunde în scobitura stâncii și voi pune mâna Mea peste tine până voi trece;
\par 23 Iar când voi ridica mâna Mea, tu vei vedea spatele Meu, iar fața Mea nu o vei vedea!"

\chapter{34}

\par 1 Zis-a Domnul către Moise: "Cioplește două table de piatră, ca cele dintâi, și suie-te la Mine în munte și voi scrie pe aceste table cuvintele care au fost scrise pe tablele cele dintâi, pe care le-ai sfărâmat.
\par 2 Să fii gata dis-de-dimineață și dimineață să te sui în Muntele Sinai și să stai înaintea Mea acolo pe vârful muntelui.
\par 3 Dar nimeni să nu se suie cu tine, nici să se arate în tot muntele: nici oi, nici vite mari să nu pască împrejurul acestui munte".
\par 4 Deci a cioplit Moise două table de piatră, asemenea cu cele dintâi, și, sculându-se dis-de-dimineață, a luat Moise în mâini cele două table de piatră și s-a suit în Muntele Sinai, cum îi poruncise Domnul.
\par 5 Atunci S-a pogorât Domnul în nor, a stat acolo și a rostit numele lui Iahve.
\par 6 Și Domnul, trecând pe dinaintea lui, a zis: "Iahve, Iahve, Dumnezeu, iubitor da oameni, milostiv, îndelung-răbdător, plin de îndurare și de dreptate,
\par 7 Care păzește adevărul și arată milă la mii de neamuri; Care iartă vina și răzvrătirea și păcatul, dar nu lasă nepedepsit pe cel ce păcătuiește; Care pentru păcatele părinților pedepsește pe copii și pe copiii copiilor până la al treilea și al patrulea neam!"
\par 8 Atunci a căzut Moise îndată la pământ și s-a închinat lui Dumnezeu,
\par 9 Zicând: "De am aflat bunăvoință în ochii Tăi, Stăpâne, să meargă Stăpânul în mijlocul nostru, căci poporul acesta e îndărătnic; dar, iartă nelegiuirile noastre și păcatele și ne fă moștenirea Ta!"
\par 10 Domnul însă a zis către Moise: "Iată, Eu închei legământ înaintea a tot poporul tău: Voi face lucruri slăvite, cum n-au fost în tot pământul și la toate popoarele; și tot poporul în mijlocul căruia te vei afla tu, va vedea lucrurile Domnului, căci înfricoșător va fi ceea ce voi face pentru tine.
\par 11 Păstrează ceea ce îți poruncesc Eu acum: Iată Eu voi izgoni de la fața ta pe Amorei, pe Canaanei, pe Hetei, pe Ferezei, pe Hevei, pe Gherghesei și pe Iebusei.
\par 12 Ferește-te să intri în legătură cu locuitorii țării aceleia, în care ai să intri, ca să nu fie ei o cursă între voi.
\par 13 Jertfelnicele lor să le stricați, stâlpii lor să-i sfărâmați; să tăiați dumbrăvile lor cele sfințite și dumnezeii lor cei ciopliți să-i ardeți în foc,
\par 14 Căci tu nu trebuie să te închini la alt dumnezeu, fără numai Domnului Dumnezeu, pentru că numele Lui este "Zelosul"; Dumnezeu este zelos.
\par 15 Nu cumva să intri în legătură cu locuitorii țării aceleia, pentru că ei, urmând după dumnezeii lor și aducând jertfe dumnezeilor lor, te vor pofti și pe tine să guști din jertfa lor.
\par 16 Și vei lua fetele lor soții pentru fiii tăi și fetele tale le vei mărita după feciorii lor; și vor merge fetele tale după dumnezeii lor și fiii tăi vor merge după dumnezeii lor.
\par 17 Să nu-ți faci dumnezei turnați.
\par 18 Sărbătoarea azimelor să o păzești: șapte zile, cum ți-am poruncit Eu, să mănânci azime, la vremea rânduită în luna Aviv, căci în luna Aviv ai ieșit tu din Egipt.
\par 19 Tot întâiul născut de parte bărbătească este al Meu; asemenea și tot întâiul născut al vacii și tot întâiul născut al oii.
\par 20 Iar întâiul născut al asinei să-l răscumperi cu un miel, iar de nu-l vei răscumpăra, să-i frângi gâtul. Toți întâii născuți din fiii tăi să-i răscumperi și nimeni să nu se înfățișeze înaintea Mea cu mâna goală.
\par 21 Șase zile lucrează, iar în ziua a șaptea să te odihnești; chiar în vremea semănatului și a secerișului să te odihnești.
\par 22 Să ții și sărbătoarea săptămânilor, sărbătoarea pârgei, la secerișul grâului, și sărbătoarea strângerii roadelor, la sfârșitul toamnei.
\par 23 De trei ori pe an să se înfățișeze înaintea Domnului Dumnezeului lui Israel toți cei de parte bărbătească ai tăi,
\par 24 Căci când voi goni popoarele de la fața ta și voi lărgi hotarele tale, nimeni nu va pofti ogorul tău, de te vei sui să te înfățișezi înaintea Domnului Dumnezeului tău de trei ori pe an.
\par 25 Să nu torni sângele jertfei Mele pe pâine dospită și jertfa de la sărbătoarea Paștilor să nu rămână până a doua zi.
\par 26 Cele dintâi roade ale țarinii tale să le aduci în casa Domnului Dumnezeului tău. Să nu fierbi iedul în laptele mamei sale".
\par 27 Și a mai zis Domnul către Moise: "Scrie-ți cuvintele acestea, căci pe cuvintele acestea închei Eu legământ cu tine și cu Israel!"
\par 28 Moise a stat acolo la Domnul patruzeci de zile și patruzeci de nopți; și nici pâine n-a mâncat, nici apă n-a băut. Și a scris Moise pe table cuvintele legământului: cele zece porunci.
\par 29 Iar când se pogora Moise din Muntele Sinai, având în mâini cele două table ale legii, el nu știa că fața sa strălucea, pentru că grăise Dumnezeu cu el.
\par 30 Deci Aaron și toți fiii lui Israel, văzând pe Moise că are fața strălucitoare, s-au temut să se apropie de el.
\par 31 Atunci i-a chemat Moise și au venit la el Aaron și toate căpeteniile obștei și Moise a grăit cu ei.
\par 32 După aceasta s-au apropiat de el toți fiii lui Israel și el le-a poruncit tot ce-i grăise Domnul în Muntele Sinai.
\par 33 Iar după ce a încetat de a grăi cu ei, Moise și-a acoperit fața cu un văl.
\par 34 Când însă intra el înaintea Domnului, ca să vorbească cu El, atunci își ridica vălul până când ieșea; iar la ieșire spunea fiilor lui Israel cele ce i se porunciseră de către Domnul.
\par 35 Și vedeau fiii lui Israel că fața lui Moise strălucea și Moise își punea iar vălul peste fața sa, până când intra din nou să vorbească cu Domnul.

\chapter{35}

\par 1 Atunci a adunat Moise toată obștea fiilor lui Israel și le-a zis: "Iată ce a poruncit Domnul să faceți:
\par 2 șase zile să lucrați, iar ziua a șaptea să fie sfântă pentru voi, zi de odihnă, odihna Domnului; tot cel ce va lucra în ziua aceea va fi omorât.
\par 3 În ziua odihnei să nu faceți foc în toate locașurile voastre. Eu sunt Domnul!"
\par 4 Apoi a grăit Moise la toată obștea fiilor lui Israel și a zis: "Iată ce a mai poruncit Domnul să vă spun:
\par 5 Aduceți din ale voastre, daruri Domnului; fiecare să aducă Domnului daruri cât îl lasă inima: aur, argint și aramă;
\par 6 Mătase violetă, stacojie și vișinie, vison răsucit și păr de capră;
\par 7 Piei de berbec vopsite roșu, piei de vițel de mare și lemn de salcâm;
\par 8 Untdelemn pentru sfeșnic și aromate pentru mirul de uns și pentru făcut miresme de tămâie,
\par 9 Piatră de sardiu, pietre pentru prins la efod și la hoșen.
\par 10 Tot cel cu minte înțeleaptă dintre voi să vină și să facă toate câte a poruncit Domnul:
\par 11 Cortul și acoperămintele lui, acoperișul lui cel de deasupra, verigile și scândurile lui, pârghiile, stâlpii și postamentele lor;
\par 12 Chivotul legii și pârghiile lui, capacul lui și perdeaua despărțitoare, pietre de smarald, tămâie și mir de ungere;
\par 13 Masa cu pârghiile și toate uneltele ei și pâinile pentru punerea înainte,
\par 14 Sfeșnicul pentru luminat cu toate obiectele lui, candelele lui și untdelemnul de ars;
\par 15 Jertfelnicul tămâierii și pârghiile lui și miresme pentru tămâiere,
\par 16 Jertfelnicul pentru arderile de tot, împletitura de sârmă pentru el, pârghiile lui și toate cele de trebuință pentru el; baia și postamentul ei,
\par 17 Perdelele curții, stâlpul ei cu postamentele lor și perdeaua de la intrarea în curte,
\par 18 Țărușii cortului; țărușii curții și frânghiile lor,
\par 19 Veșmintele sfinte pentru făcut slujba în locașul sfânt; veșminte sfinte pentru Aaron preotul și veșminte pentru fiii lui, pentru slujba preoției".
\par 20 După aceea, plecând toată obștea fiilor lui Israel de la Moise,
\par 21 A adus fiecare cât l-a lăsat inima sa și cât l-a îndemnat cugetul să aducă dar Domnului, pentru facerea cortului adunării și a tuturor lucrurilor lui și pentru toate veșmintele sfinte.
\par 22 Și veneau bărbații cu femeile și fiecare, după cum îl lăsa inima, aducea verigi, cercei, inele, brățări și tot felul de lucruri de aur; cum voia fiecare să aducă Domnului daruri de aur.
\par 23 Fiecare din cei ce aveau mătase violetă, stacojie și vișinie, în și păr de capră, piei de berbec vopsite roșu și vânăt, le aducea.
\par 24 Fiecare din cei ce puteau să aducă în dar argint sau aramă, aducea din acestea Domnului; și fiecare din cei ce aveau lemn de salcâm, aducea pentru toate cele de trebuință;
\par 25 Toate femeile cu minte iscusită torceau cu mâinile lor și aduceau tort, mătase violetă, stacojie și vișinie și în.
\par 26 Toate femeile, pe care le trăgea inima și știau să toarcă, torceau păr de capră;
\par 27 Iar căpeteniile aduceau pietre de smarald și pietre scumpe de pus la efod și la hoșen,
\par 28 Precum și miresme, untdelemn pentru candelabru, mir de ungere și miresme de tămâiere.
\par 29 Deci tot bărbatul și femeia din fiii lui Israel, pe care i-a tras inima să aducă pentru toate lucrurile ce poruncise Domnul prin Moise să se facă, au adus dar de bună voie Domnului.
\par 30 Apoi a zis Moise către fiii lui Israel: "Iată Domnul a chemat anume pe Bețaleel, fiul lui Uri al lui Or, din seminția lui Iuda,
\par 31 Și l-a umplut de duhul dumnezeiesc al înțelepciunii, al priceperii, al științei și a toată iscusința,
\par 32 Ca să lucreze țesături iscusite, să facă lucruri de aur, de argint și de aramă;
\par 33 Să cioplească pietrele scumpe pentru încrustat, să sape în lemn și să facă tot felul de lucruri iscusite.
\par 34 Și priceperea de a învăța pe alții a pus-o în inima lui, în a lui și a lui Oholiab, fiul lui Ahisamac, din seminția lui Dan.
\par 35 A umplut inima acestora de înțelepciune, ca să facă pentru locașul sfânt orice lucru de săpător și de țesător iscusit, de cusător pe pânză de mătase violetă, stacojie și vișinie și de in, și de țesător în stare de a face orice lucru și a născoci țesături iscusite".

\chapter{36}

\par 1 Și Bețaleel, Oholiab și toți cei cu minte iscusită, cărora Domnul le dăduse înțelepciune și pricepere, ca să știe să facă tot felul de lucruri trebuitoare la locașul cel sfânt, vor trebui să facă după cum poruncise Domnul.
\par 2 Iar Moise a chemat pe Bețaleel, pe Oholiab și pe toți cei cu minte iscusită, cărora le dăduse Domnul iscusință și pe toți cei ce-i trăgea inima să vină la lucru de bună voie, ca să ajute la acestea.
\par 3 Și au luat ei de la Moise toate prinoasele, pe care le aduseseră fiii lui Israel pentru toate cele trebuincioase locașului sfânt, ca să le lucreze. Atunci tot se mai aduceau încă la el daruri de bună voie în fiecare dimineață.
\par 4 Deci toți cei cu minte iscusită, care împlineau tot felul de lucrări la locașul sfânt, au venit fiecare, de la lucru cu care se îndeletnicea,
\par 5 Și ei au spus lui Moise, zicând: "Poporul aduce mult mai mult decât trebuie pentru lucrurile ce a poruncit Domnul să se facă".
\par 6 Atunci a poruncit Moise și s-a strigat în tabără, că nici bărbat, nici femeie să nu mai facă nimic pentru dăruit la locașul sfânt. Și a încetat poporul de a mai aduce.
\par 7 Căci material adunat era destul pentru toate lucrurile ce trebuiau făcute, ba mai și prisosea.
\par 8 Atunci toți cei cu minte iscusită, care se îndeletniceau cu facerea locașului sfânt, au făcut pentru cort zece covoare de in răsucit și de mătase violetă, stacojie și vișinie; și în țesătura lor au făcut chipuri de heruvimi, alese cu iscusință.
\par 9 Lungimea unui covor era de douăzeci și opt de coți și lățimea unui covor era de patru coți. Toate covoarele aveau aceeași măsură.
\par 10 Cinci covoare au fost prinse unul de altul și celelalte cinci iar au fost prinse unul de altul; și așa s-au făcut două jumătăți de acoperiș.
\par 11 Apoi au făcut cheotori de mătase violetă pe marginea covorului din marginea jumătății întâi de acoperiș, unde trebuia unită cu jumătatea a doua; de asemenea au făcut și pe marginea jumătății a doua, unde aceasta trebuia unită cu cea dintâi;
\par 12 Cincizeci de cheotori au făcut la o jumătate de acoperământ și cincizeci de cheotori au făcut la cealaltă și cheotorile acestea erau unele în dreptul altora.
\par 13 După aceea au făcut cincizeci de copci de aur și cu copcile acestea au unit cele două jumătăți de acoperiș una cu alta și s-a făcut un acoperiș întreg al cortului.
\par 14 Apoi au făcut covoare de păr de capră pentru acoperit cortul peste cele de mai sus. Unsprezece covoare de acestea au făcut.
\par 15 Lungimea unui covor era de treizeci de coti, iar lățimea de patru coti; și cele unsprezece covoare aveau toate aceeași măsură.
\par 16 Și au unit cinci covoare la un loc și pe celelalte șase covoare iar le-au unit la un loc.
\par 17 Apoi au făcut cincizeci de cheotori pe marginea covorului celui din marginea unei jumătăți, unde aceasta trebuia să se unească cu cealaltă jumătate, iar cincizeci de cheotori le-au făcut pe marginea covorului din marginea celeilalte jumătăți, care trebuia să se unească cu cea dintâi.
\par 18 Și au făcut cincizeci de copci de aramă ca să unească covoarele spre a se face un singur acoperiș de cort.
\par 19 Apoi au mai făcut pentru cort un acoperiș de piei de berbec vopsite în roșu și un acoperiș, pe deasupra, de piei vinete.
\par 20 După aceea au făcut pentru cort scânduri din lemn de salcâm de pus în picioare.
\par 21 Fiecare scândură era lungă de zece coți și lată de un cot și jumătate.
\par 22 Fiecare scândură avea două cepuri, așezate unul în dreptul celuilalt.
\par 23 Așa au făcut toate scândurile cortului și anume: douăzeci de scânduri pentru latura de miazăzi;
\par 24 Și sub aceste douăzeci de scânduri au făcut patruzeci de postamente de argint, câte două postamente la fiecare scândură, pentru cele două cepuri ale ei.
\par 25 Pentru latura a doua, dinspre miazănoapte, au făcut alte douăzeci de scânduri
\par 26 Și patruzeci de postamente de argint, câte două postamente de fiecare scândură, pentru cele două cepuri ale ei;
\par 27 Iar pentru partea dindărăt a cortului, dinspre asfințit, au făcut șase scânduri.
\par 28 Au mai făcut două scânduri pentru unghiurile de la fundul cortului.
\par 29 Acestea erau prinse jos și sus prin câte un inel.
\par 30 și așa, cu cele două scânduri de la cele două colțuri erau la partea dindărăt a cortului opt scânduri, iar postamente de argint șaisprezece, câte două postamente sub fiecare scândură.
\par 31 Apoi au făcut cinci pârghii din lemn de salcâm pentru scândurile de pe o latură a cortului,
\par 32 Și cinci pârghii pentru scândurile de pe cealaltă latură;
\par 33 Iar pârghia din mijloc au făcut-o așa, ca să treacă prin scânduri, de la un capăt la celălalt al peretelui.
\par 34 Scândurile le-au îmbrăcat cu aur; inelele, prin care se vârau pârghiile, le-au făcut de aur și tot cu aur au îmbrăcat și pârghiile.
\par 35 După aceea au făcut o perdea de mătase violetă, stacojie și vișinie și de in răsucit și pe ea au făcut chipuri de heruvimi cu iscusință alese, ca să o pună între sfânta și sfânta sfintelor.
\par 36 Pentru ea au făcut patru stâlpi, din lemn de salcâm, i-au îmbrăcat cu aur, le-au făcut cârlige de aur și sub ei au turnat patru postamente de argint.
\par 37 Apoi la ușa cortului au făcut o perdea de mătase violetă, stacojie și vișinie și de in răsucit, cu alesături.
\par 38 Pentru ea au făcut cinci stâlpi cu cârligele lor și i-au îmbrăcat cu aur, turnând pentru ei cinci postamente de aramă.

\chapter{37}

\par 1 După aceea Bețaleel a făcut chivotul din lemn de salcâm, lung de doi coți și jumătate, larg de un cot și jumătate și înalt tot de un cot și jumătate;
\par 2 L-a îmbrăcat cu aur curat pe dinăuntru și pe din afară, iar împrejur i-a făcut o cunună de aur.
\par 3 A turnat pentru el patru inele de aur, pentru cele patru colțuri de jos ale lui: două inele pe o latură și două inele pe cealaltă latură.
\par 4 A făcut două pârghii de lemn de salcâm, le-a îmbrăcat cu aur,
\par 5 Și le-a vârât prin inelele de pe laturile chivotului, ca să poarte chivotul cu ele.
\par 6 A făcut apoi capacul chivotului de aur curat: lung de doi coți și jumătate și lat de un cot și jumătate.
\par 7 A făcut de asemenea doi heruvimi de aur, lucrați din ciocan, pentru cele două capete ale capacului,
\par 8 Și i-a așezat unul la un capăt și altul la celălalt capăt al capacului. Heruvimii aceștia erau făcuți ca ieșind din capac la cele două capete ale lui.
\par 9 Cei doi heruvimi își întindeau aripile unul spre altul, umbrind capacul, iar fețele lor erau îndreptate una către alta, privind spre capac.
\par 10 A făcut apoi masa din lemn de salcâm, lungă de doi coți, lată de un cot și înaltă de un cot și jumătate.
\par 11 A îmbrăcat-o cu aur curat și împrejur i-a făcut o cunună de aur.
\par 12 I-a făcut de asemenea un pervaz împrejur înalt de o palmă, iar împrejurul pervazului a făcut cunună împletită de aur.
\par 13 A turnat pentru ea patru inele de aur și a prins aceste inele în cele patru colțuri de la cele patru picioare ale ei.
\par 14 Inelele erau prinse de pervaz și prin ele se petreceau două pârghii pentru purtat masa.
\par 15 Pârghiile de purtat masa le-a făcut din lemn de salcâm și le-a îmbrăcat cu aur.
\par 16 A făcut vase trebuitoare pentru masă: talere, cădelnițe, linguri și cupe pentru turnat, toate de aur curat.
\par 17 După aceea a făcut un sfeșnic de aur curat și sfeșnicul acesta l-a lucrat din ciocan. Fusul lui, brațele lui, cupele lui, nodurile lui și florile lui erau toate dintr-o bucată.
\par 18 Din laturile lui ieșeau șase brațe: trei brațe ale sfeșnicului ieșeau dintr-o latură a lui și trei brațe ale sfeșnicului ieșeau din cealaltă latură a lui;
\par 19 Un braț avea trei cupe în forma florii de migdal, cu nodurile și florile lor; alt braț avea trei cupe tot în forma florii de migdal, cu nodurile și florile lor; așa aveau toate cele șase brațe, care ieșeau din laturile sfeșnicului.
\par 20 Iar pe fusul sfeșnicului erau patru cupe în forma florii de migdal, cu nodurile și florile lor.
\par 21 Cele șase brațe, care ieșeau din el, aveau: un nod sub primele două brațe, un nod sub alte două brațe și un nod sub ultimele două brațe.
\par 22 Nodurile și ramurile de pe ele erau una cu fusul. Sfeșnicul întreg era lucrat din ciocan, dintr-o singură bucată de aur curat.
\par 23 Apoi i-a făcut șapte candele, mucări și tăvițe de aur curat.
\par 24 Dintr-un talant de aur curat au făcut sfeșnicul cu toate cele necesare lui.
\par 25 A făcut apoi jertfelnicul tămâierii, din lemn de salcâm, lung de un cot, lat de un cot, adică pătrat și înalt de doi coți; coarnele lui erau din el;
\par 26 Și l-a îmbrăcat cu aur curat pe deasupra, pe laturile lui de jur împrejur și pe coarnele lui, iar împrejur i-a făcut cunună de aur.
\par 27 Sub pervazul lui, la două din colțurile lui, a prins două inele de aur, și le-a prins de o parte și de alta a lui, ca să se petreacă prin ele pârghiile de purtat.
\par 28 Pârghiile le-au făcut din lemn de salcâm și le-au îmbrăcat cu aur.
\par 29 A făcut de asemenea mir pentru sfânta ungere și tămâie mirositoare, curate și cu iscusință alcătuite de pregătitorii de aromate.

\chapter{38}

\par 1 După aceea a făcut jertfelnicul pentru arderile de tot, din lemn de salcâm, lung de cinci coți, lat de cinci coți, adică cu fața pătrată și înalt de trei coți.
\par 2 I-a făcut patru coarne, ce ieșeau din el, în cele patru colțuri ale lui, și l-a îmbrăcat cu aramă.
\par 3 A făcut apoi toate lucrurile trebuitoare jertfelnicului: oale, lopățele, cupe, furculițe și vase pentru cărbuni; toate obiectele acestea le-a făcut din aramă.
\par 4 A mai făcut pentru jertfelnic o cămașă, un fel de împletitură de aramă, care îmbrăca partea lui de jos până la jumătatea lui.
\par 5 A turnat apoi patru verigi de aramă pentru cele patru colțuri ale împletiturii celei de aramă, pentru petrecut prin ele pârghiile de purtat;
\par 6 Iar pârghiile le-au făcut din lemn de salcâm și le-a îmbrăcat cu aramă.
\par 7 Pârghiile se petreceau prin inelele din laturile jertfelnicului, ca să poată fi purtat cu ajutorul lor. Jertfelnicul l-a făcut din scânduri, gol înăuntru.
\par 8 A făcut apoi baia de aramă și postamentul ei tot de aramă, cu chipuri iscusit lucrate care împodobeau intrarea cortului adunării.
\par 9 După aceea a făcut curtea. Spre miazăzi curtea avea perdele de in răsucit, lungi de o sută de coți,
\par 10 Și la ele douăzeci de stâlpi cu douăzeci de postamente de aramă sub ei și cu cârligele și legătorile lor de argint;
\par 11 Pe latura de miazănoapte a făcut perdele lungi de o sută de coți și la ele douăzeci de stâlpi, cu douăzeci de postamente de aramă sub ei, cu cârligele și legătorile lor de argint.
\par 12 În partea dinspre asfințit a făcut perdele lungi de cincizeci de coți și pentru ele zece stâlpi, cu zece postamente de aramă și cu cârligele și legăturile lor de argint;
\par 13 Iar în partea de dinainte, dinspre răsărit, a făcut perdele lungi de cincizeci de coți,
\par 14 Și anume: de o parte a porții curții cincisprezece coți de perdele și la ele trei stâlpi cu trei postamente de aramă;
\par 15 De cealaltă parte a porții curții cincisprezece coți de perdele și la ele trei stâlpi cu trei postamente de aramă.
\par 16 Toate perdelele pe toate laturile curții erau de in răsucit.
\par 17 Iar stâlpii aveau postamentele de aramă, cârligele și legătorile lor de argint, vârfurile îmbrăcate în argint și toți erau uniți între ei prin legători de argint.
\par 18 Iar perdeaua pentru poarta curții a făcut-o din lână violetă, stacojie și vișinie și din in răsucit, cu alesături, lungă de douăzeci de coți și înaltă de cinci coți, pe toată întinderea, ca și perdelele curții.
\par 19 Pentru ea a făcut patru stâlpi cu patru postamente de aramă sub ei, cu cârligele și legătorile de argint ș cu vârfurile îmbrăcate în argint.
\par 20 Toți țărușii împrejurul cortului și curții erau de aramă.
\par 21 Iată acum și socoteala lucrurilor ce s-au întrebuințat la cortul adunării, care s-a făcut după porunca lui Moise, prin leviți, sub supravegherea lui Itamar, fiul preotului Aaron.
\par 22 Toate însă, câte a poruncit Domnul lui Moise, s-au lucrat de Bețaleel, fiul lui Uri al lui Or, din seminția lui Iuda,
\par 23 Ajutat de Oholiab, fiul lui Ahisamac, din seminția lui Dan, săpător în piatră și țesător iscusit și cusător pe pânze de in și de mătase violetă, stacojie și vișinie.
\par 24 Tot aurul întrebuințat la cort și la toate lucrurile lui a fost douăzeci și nouă de talanți și șapte sute treizeci sicli de aur, ce s-au adus în dar, socotit după siclul sfânt.
\par 25 Iar argintul, ce s-a adus dar de la cei numărați ai obștei, a fost o sută de talanți și o mie șapte sute șaptezeci și cinci de sicli, socotit după siclul sfânt.
\par 26 Argintul acesta s-a luat de la șase sute trei mii cinci sute cincizeci de oameni, în vârstă de la douăzeci de ani în sus, trecuți prin numărătoare, câte o jumătate de siclu de cap, socotit după siclul sfânt.
\par 27 O sută de talanți de argint s-au întrebuințat la turnarea postamentelor scândurilor cortului și a postamentelor stâlpilor perdelelor lui: o sută de postamente din o sută de talanți, câte un talant la postament;
\par 28 Iar din o mie șapte sute șaptezeci și cinci de sicli au făcut cârligele de la stâlpii curții, au îmbrăcat vârfurile lor și au făcut pentru ei legători.
\par 29 Aramă, adusă în dar, a fost: trei sute șaptezeci de talanți și două mii patru sute de sicli.
\par 30 Din ea au făcut postamente pentru stâlpii de la intrarea cortului adunării, jertfelnicul cel de aramă, cămașa de aramă a lui și toate uneltele jertfelnicului;
\par 31 Postamentele pentru toți stâlpii curții, postamentele pentru stâlpii de la intrarea curții, toți țărușii cortului și toți țărușii dimprejurul curții.

\chapter{39}

\par 1 Iar din mătase violetă, stacojie și vișinie au făcut veșminte de slujbă, pentru slujit în locașul sfânt, și au mai făcut veșminte sfinte pentru Aaron, cum poruncise Domnul lui Moise.
\par 2 Au făcut efodul din fire de aur, din mătase violetă, stacojie și vișinie și din in răsucit.
\par 3 Și anume: au desfăcut aurul în foi și au tăiat fire, pe care le-au țesut cu iscusință printre firele de mătase violetă, stacojie și vișinie și de in răsucit, lucru iscusit.
\par 4 I-au făcut încheietori de încheiat pe umeri și au unit amândouă părțile lui.
\par 5 Brâul efodului, care vine peste el, la fel cu el, l-au făcut din fire de aur, din mătase violetă, stacojie și vișinie și din in răsucit, cum poruncise Domnul lui Moise.
\par 6 Au lucrat apoi două  pietre de smarald, așezându-le în cuibulețe de aur și săpând pe ele numele fiilor lui Israel, cum se sapă pe pecete,
\par 7 Și le-au pus la încheieturile efodului, pe umeri, întru pomenirea fiilor lui Israel, cum poruncise Domnul lui Moise.
\par 8 Au făcut apoi hoșenul, lucrare iscusită, la fel cu efodul, din fire de aur și din mătase violetă, stacojie și vișinie și din in răsucit.
\par 9 Hoșenul l-au făcut dublu, în patru colțuri, lung de o palmă și lat de o palmă.
\par 10 Și au pus pe el pietre scumpe, așezate în patru rânduri: într-un rând un sardeon, un topaz și un smarald - rândul întâi;
\par 11 În rândul al doilea: un rubin, un safir și un diamant;
\par 12 În rândul al treilea: un opal, o agată și un ametist;
\par 13 Și în rândul al patrulea: un hrisolit, un onix și un iaspis. Ele erau așezate în cuibulețe de aur.
\par 14 Pietrele acestea erau în număr de douăsprezece, după numărul fiilor lui Israel, și pe fiecare din ele era săpat, ca pe pecete, câte un nume, din cele ale celor douăsprezece seminții.
\par 15 La hoșen au făcut apoi lănțișoare groase de aur curat și lucrate răsucit, ca sfoara;
\par 16 Au mai făcut două rozete și două verigi de aur și au prins cele două verigi de cele două colțuri de sus ale hoșenului;
\par 17 Și au agățat două capete ale lănțișoarelor de cele două verigi din colturile hoșenului,
\par 18 Iar celelalte două capete ale celor două lănțișoare le-au agățat de cele două rozete și le-au prins pe acestea de încheieturile efodului, pe fața acestuia.
\par 19 După aceea au mai făcut încă două verigi de aur și le-au prins de celelalte două colțuri ale hoșenului pe cealaltă parte dinspre efod;
\par 20 Și au mai făcut și alte două verigi de aur și le-au prins de cele două încheieturi ale efodului, dedesubt, pe fața lui, unde se unesc, mai sus de încingătoarea efodului.
\par 21 Și au legat hoșenul cu verigile lui de verigile efodului cu un șnur de mătase violetă, ca să stea deasupra încingătorii efodului și ca să nu cadă hoșenul de pe efod, cum poruncise Domnul lui Moise.
\par 22 Iar meilul care vine sub efod, l-au făcut din purpură țesută violet.
\par 23 Acesta avea în partea de sus o deschizătură și împrejurul acestei deschizături avea un guler, țesut ca o platoșă, ca să nu se rupă.
\par 24 Meilului i-au făcut pe la poale ciucuri de mătase violetă, stacojie și vișinie și de in răsucit;
\par 25 I-au mai făcut și clopoței de aur curat și au pus clopoței printre ciucurii de la poalele meilului de jur împrejur;
\par 26 Și i-au așezat pe la poalele meilului de slujbă așa: un clopoțel și un ciucure, un clopoțel și un ciucure, cum poruncise Domnul lui Moise.
\par 27 Au făcut apoi pentru Aaron și pentru fiii lui hitoane țesute din in,
\par 28 Chidare de in, turbane tot de in și pantaloni de in răsucit;
\par 29 Și cingătoare din in răsucit și de mătase violetă, stacojie și vișinie, țesută cu alesături, cum poruncise Domnul lui Moise.
\par 30 După aceea au făcut o tăbliță de aur curat, diadema sfințeniei, și au săpat pe ea, ca pe pecete, cuvintele: "Sfințenia Domnului".
\par 31 Și au prins de ea un șnur de mătase violetă, ca s-o lege peste chidar, cum poruncise Domnul lui Moise.
\par 32 Așa s-au sfârșit toate lucrările de la cortul adunării. Și au făcut fiii lui Israel toate; cum poruncise Domnul lui Moise așa au făcut.
\par 33 Apoi au adus la Moise: cortul, acoperămintele și toate cele de trebuință ale lui, cârligele lui, scândurile lui, pârghiile lui, stâlpii lui și postamentele lui;
\par 34 Acoperișurile cele cu piei roșii de berbec și acoperișurile cele de piei vinete și perdeaua din mijloc;
\par 35 Chivotul legii, capacul lui și pârghiile;
\par 36 Masa cu toate cele de trebuință pentru ea și pâinile de pus înainte;
\par 37 Sfeșnicul cel de aur curat, candelele lui, candele puse în el la locul lor, și toate cele trebuincioase pentru el și untdelemn de ars;
\par 38 Jertfelnicul cel de aur, mirul pentru ungere, miresme pentru tămâiere și perdeaua de la intrarea cortului;
\par 39 Jertfelnicul cel de aramă, cămașa lui cea de aramă, pârghiile lui și toate cele trebuitoare pentru el, baia și postamentul ei;
\par 40 Perdelele curții, stâlpii ei și postamentele lor, perdelele de la intrarea curții, frânghiile, țărușii și toate lucrurile trebuitoare la slujbă în cortul adunării;
\par 41 Veșmintele de slujit în cort, veșmintele sfinte ale preotului Aaron și veșmintele de slujbă pentru fiii lui.
\par 42 Toate aceste lucruri le făcuseră fiii lui Israel așa cum poruncise Domnul lui Moise.
\par 43 Și privi Moise toată lucrarea și iată ei o făcuseră așa cum poruncise Domnul și Moise i-a binecuvântat.

\chapter{40}

\par 1 Apoi iarăși a grăit Domnul cu Moise și a zis:
\par 2 "În ziua întâi a lunii întâi, să așezi cortul adunării.
\par 3 Și să pui într-însul chivotul legii și dinaintea chivotului să atârni perdeaua;
\par 4 Apoi să aduci înăuntru masa și să așezi pe ea toate lucrurile ei; și sfeșnicul să-l duci înăuntru și să aprinzi într-însul candelele lui.
\par 5 Să așezi jertfelnicul cel de aur pentru tămâiere înaintea chivotului legii și să atârni perdeaua la intrare în cortul adunării.
\par 6 Apoi să așezi jertfelnicul arderilor de tot înaintea intrării în cortul adunării.
\par 7 Iar între cortul adunării și jertfelnic să așezi baia și să torni în ea apă;
\par 8 Și împrejurul cortului să așezi împrejmuirea curții și la intrarea curții să atârni perdeaua.
\par 9 După aceea să iei mir de ungere și să ungi cortul și toate cele din el și să-l sfințești pe el și toate lucrurile lui și va fi sfânt;
\par 10 Să ungi jertfelnicul arderilor de tot și toate lucrurile lui și să sfințești jertfelnicul și va fi sfințenie mare;
\par 11 Să ungi apoi baia și postamentul ei și să o sfințești.
\par 12 Apoi să aduci pe Aaron și pe fiii lui la ușa cortului adunării și să-i speli cu apă;
\par 13 Să îmbraci pe Aaron în sfintele veșminte și să-l ungi și să-l sfințești, ca să-mi fie preot.
\par 14 Să aduci și pe fiii lui, să-i îmbraci cu hitoane,
\par 15 Și să-i ungi, cum ai uns pe tatăl lor, ca să-Mi fie preoți, și această ungere îi va sfinți preoți pentru totdeauna în neamul lor!"
\par 16 și a făcut Moise tot; cum i-a poruncit Domnul așa a făcut:
\par 17 În luna întâi a anului al doilea de la ieșirea lor din Egipt, în ziua întâi a lunii a fost așezat cortul.
\par 18 Și a așezat Moise cortul, a pus postamentele lui, scândurile lui, pârghiile lui și stâlpii lui;
\par 19 A întins deasupra cortului acoperămintele și peste aceste acoperăminte a pus acoperișul, cum poruncise Domnul lui Moise.
\par 20 Apoi a luat și a pus legea în chivot, a petrecut pârghiile prin inelele chivotului și a pus deasupra, la chivot, capacul;
\par 21 A dus apoi chivotul în cort, a atârnat perdeaua și a închis chivotul legii, precum poruncise Domnul lui Moise.
\par 22 După aceea a pus masa în cortul adunării, în partea de miazănoapte a cortului, în afară de perdea,
\par 23 Și a așezat pe ea pâinile punerii înaintea Domnului, cum poruncise Dumnezeu lui Moise.
\par 24 Sfeșnicul l-a așezat în cortul adunării, în fața mesei, în partea de miazăzi a cortului,
\par 25 Și a aprins candelele lui înaintea Domnului, precum poruncise Domnul lui Moise.
\par 26 A așezat jertfelnicul cel de aur în cortul adunării, înaintea perdelei,
\par 27 Și a aprins pe el tămâie mirositoare, cum poruncise Domnul lui Moise;
\par 28 A atârnat perdeaua la ușa cortului;
\par 29 Iar jertfelnicul arderilor de tot 1-a așezat la intrarea în cortul adunării și a pus pe el arderi de tot și prinoase de pâine, cum poruncise Domnul lui Moise.
\par 30 Apoi a așezat baia între cortul adunării și jertfelnic și a turnat în ea apă pentru spălat;
\par 31 Moise, Aaron și fiii lui trebuia să-și spele din ea mâinile și picioarele;
\par 32 Când intrau ei în cortul adunării sau când se apropiau de jertfelnic ca să slujească, se spălau din ea cum poruncise Domnul lui Moise.
\par 33 După aceea au pus împrejmuirea curții împrejurul cortului și a jertfelnicului și a atârnat perdeaua la intrarea curții. Și așa a isprăvit Moise lucrările.
\par 34 Atunci un nor a acoperit cortul adunării și locașul s-a umplut de slava Domnului;
\par 35 Și Moise n-a putut să intre în cortul adunării, pentru că-l cuprinsese pe acesta norul și slava Domnului umpluse locașul.
\par 36 În tot timpul călătoriei fiilor lui Israel, când se ridica norul de pe cort, atunci plecau la drum,
\par 37 Iar de nu se ridica norul, nici ei nu plecau la drum până nu se ridica;
\par 38 Pentru că în tot timpul călătoriei, ziua stătea peste cort norul Domnului, iar noaptea se afla peste el foc, înaintea ochilor întregii case a lui Israel.


\end{document}