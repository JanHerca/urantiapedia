\begin{document}

\title{Cântarea cântărilor}


\chapter{1}

\par 1 Cântarea cântărilor, a lui Solomon.
\par 2 Sărută-mă cu sărutările gurii tale, că sărutările tale sunt mai bune ca vinul.
\par 3 Miresmele tale sunt balsam mirositor, mir vărsat este numele tău; de aceea fecioarele te iubesc.
\par 4 Răpește-mă, ia-mă cu tine! Hai să fugim! - Regele m-a dus în cămările sale: ne vom veseli și ne vom bucura de tine. Îți vom preamări dragostea mai mult decât vinul. Cine te iubește, după dreptate te iubește!
\par 5 Neagră sunt, fete din Ierusalim, dar frumoasă, ca sălașurile lui Chedar, ca și corturile lui Solomon.
\par 6 Nu vă uitați că sunt negricioasă că doar soarele m-a ars. Fiii maicii mele s-au mâniat, trimisu-m-au să le păzesc viile, dar via mea nu mi-am păzit-o!
\par 7 Spune-mi dar, iubitul meu, unde-ți paști tu oile? Unde poposești tu la amiază? De ce oare să rătăcesc zadarnic pe la turmele tovarășilor tăi?
\par 8 Dacă nu știi unde, tu cea mai frumoasă între femei, ține atunci mereu urmele oilor, paște-ți mieii în preajma colibelor, iezii în preajma ciobanilor!
\par 9 Cu un telegar înhămat la carul lui Faraon te aseamăn eu, draga mea!
\par 10 Frumoși se văd obrajii tăi, așezați intre cercei și gâtul tău împodobit cu șire de mărgăritare.
\par 11 Făuri-vom pentru tine lănțișoare aurite, cu picături și crestături de argint.
\par 12 Cât regele a stat la masă, nardul meu a revărsat mireasmă.
\par 13 Periniță de mirt este iubitul meu, care se ascunde între sânii mei.
\par 14 Iubitul meu e ciorchinele de ienupăr, din Enghedi, de la vii cules.
\par 15 Cât de frumoasă ești tu, draga mea! Cât de frumoasă ești! Ochi de porumbiță sunt ochii tăi.
\par 16 Cât de frumos ești dragul meu! Și cît de drăgălaș ești tu! Pajiștea de iarbă verde ne este al nostru pat.
\par 17 Cedrii ne sunt acoperiș sălășluirii și adăpost ne sunt chiparoșii.

\chapter{2}

\par 1 Eu sunt narcisul din câmpie, sunt crinul de prin vâlcele.
\par 2 Cum este crinul între spini, așa este draga mea între fete.
\par 3 Cum este mărul intre copaci, așa este dragul meu printre flăcăi. Să stau la umbra mărului îmi place, dulce este rodul lui în gura mea!
\par 4 El m-a dus în casa de ospăț și sus drept steag era iubirea.
\par 5 Întăriți-mă cu vin, cu mere răcoriți-mă, că sunt bolnavă de iubire.
\par 6 Stânga sa este sub cap la mine și cu dreapta lui mă cuprinde.
\par 7 Vă jur, fete din Ierusalim, pe cerboaice, pe gazelele din câmp, nu treziți pe draga mea; până nu-i va fi ei voia!
\par 8 Auzi glasul celui drag! Iată-l vine, săltând peste coline, trecând din munte-n munte.
\par 9 Ca o gazelă e iubitul meu sau e ca un pui de cerb; iată-l la noi pe prispă, iată-l privește pe fereastră, printre gratii iată-l se uită.
\par 10 Și începe să-mi vorbească: Scoală, draga mea, și vino!
\par 11 Iarna a trecut, ploaia a încetat.
\par 12 Flori pe câmp s-au arătat și a sosit vremea cântării, în țarină glas de turturea se aude.
\par 13 Smochinii își dezvelesc mugurii și florile de vie văzduhul parfumează. Scoală, draga mea, și vino!
\par 14 Porumbița mea, ce-n crăpături de stâncă, la loc prăpăstios te-ascunzi, arată-ți fața ta! Lasă-mă să-ți aud glasul! Că glasul tău e dulce și fața ta plăcută.
\par 15 Prindeți vulpile, prindeți puii lor, ele ne strică viile, că via noastră e acum în floare.
\par 16 Iubitul meu este al meu și eu sunt a lui. El printre crini își paște mieii.
\par 17 Până nu se răcorește ziua, până nu se-ntinde umbra serii, vino, dragul meu, săltând ca o căprioară, ca un pui de cerb, peste munții ce ne despart.

\chapter{3}

\par 1 Noaptea-n pat l-am căutat pe dragul sufletului meu, l-am căutat, dar, iată, nu l-am mai aflat.
\par 2 Scula-mă-voi, mi-am zis, și-n târg voi alerga, pe ulițe, prin piețe, amănunțit voi căuta pe dragul sufletului meu. L-am căutat, nu l-am mai aflat.
\par 3 Întâlnitu-m-am cu paznicii, cei ce târgul străjuiesc; "N-ați văzut, zic eu, pe dragul sufletului meu?"
\par 4 Dar abia m-am despărțit de ei și iată, eu l-am găsit, pe cel iubit; apucatu-l-am atunci și nu l-am mai lăsat, până nu l-am dus la mama mea, până nu l-am dus în casa ei.
\par 5 Vă jur, fete din Ierusalim, pe cerboaice, pe gazelele din câmp, nu treziți pe draga mea, până nu-i va fi ei voia!
\par 6 Cine este aceea care se ridică din pustiu, ca un stâlp de fum, par-c-ar arde smirnă și tămâie, par-c-ar arde miresme iscusit gătite?
\par 7 Iat-o, este ea, lectica lui Solomon, înconjurată de șaizeci de voinici, viteji falnici din Israel.
\par 8 Toți sunt înarmați, la război deprinși. Fiecare poartă sabie la șold, pentru orice întâmplare și frică de noapte.
\par 9 Regele Solomon și-a făcut tron de nuntă din lemn de cedru din Liban.
\par 10 Stâlpii lui sunt de argint, pereții de aur curat, patul e de purpură, iar acoperișul este de scumpe alesături; darul dragostei alese a fetelor din Ierusalim.
\par 11 Ieșiți, fetele Sionului, priviți pe Solomon încoronat, cum a lui maică l-a încununat, în ziua sărbătoririi nunții lui, în ziua bucuriei inimii lui!

\chapter{4}

\par 1 Cât de frumoasă ești tu, draga mea, cît de frumoasă ești! Ochi de porumbiță ai, umbriri de negrele-ți sprâncene, părul tău turmă de capre pare, ce din munți, din Galaad coboară.
\par 2 Dinții tăi par turmă de oi tunse, ce ies din scăldătoare făcând două șiruri strânse și neavând nici o știrbitură.
\par 3 Cordeluțe purpurii sunt ale tale buze și gura ta-i încântătoare. Două jumătăți de rodii par obrajii tăi sub vălul tău cel străveziu.
\par 4 Gâtul tău e turnul lui David, menit să fie casă de arme: mii de scuturi atârnă acolo și tot scuturi de viteji.
\par 5 Cei doi sâni ai tăi par doi pui de căprioară, doi iezi care pasc printre crini.
\par 6 Până nu se răcorește ziua, până nu se-ntinde umbra serii, voi veni la tine, colină de mirt, voi veni la tine, munte de tămâie.
\par 7 Cât de frumoasă ești tu, draga mea, și fără nici o pată.
\par 8 Vino din Liban, mireasa mea, vino din Liban cu mine! Degrabă coboară din Amana, din Senir și din Hermon, din culcușul leilor și din munți cu leoparzi!
\par 9 Sora mea, mireasa mea, tu mi-ai robit inima numai c-o privire a ta și cu colanu-ți de la sân.
\par 10 Cât de dulce, când dezmierzi, ești tu sora mea mireasă; și mai dulce decât vinul este mângâierea ta. și mireasma ta plăcută este mai presus de orice mir.
\par 11 Ale tale buze miere izvorăsc, iubito, miere curge, lapte curge, de sub limba ta; mirosul îmbrăcămintei tale e mireasmă de Liban.
\par 12 Ești grădină încuiată, sora mea, mireasa mea, fântână acoperită și izvor pecetluit.
\par 13 Vlăstarele tale clădesc un paradis de rodii cu fructe dulci și minunate, având pe margini arbuști care revarsă miresme:
\par 14 Nard, șofran și scorțișoară cu trestie mirositoare, cu felurime de copaci, ce tămâie lăcrimează, cu mirt și cu aloe și cu arbuști mirositori.
\par 15 În grădină-i o fântână, un izvor de apă vie și pâraie din Libar.
\par 16 Scoală vânt de miazănoapte, vino vânt de miazăzi, suflați prin grădina mea și miresmele-i stârniți; iar iubitul meu să vină, în grădina sa să intre și din roadele ei scumpe să culeagă, să mănânce.

\chapter{5}

\par 1 Venit-am în grădina mea, sora mea, mireasa mea! Strâns-am miruri aromate, miere am mâncat din faguri, vin și lapte am băut. Mâncați și beți, prieteni, fiți beți de dragoste, iubiții mei!
\par 2 De dormit dormeam, dar inima-mi veghea. Auzi glasul celui drag! El la ușă bătând zice: Deschide-mi, surioară, deschide-mi, iubita mea, porumbița mea, curata mea, capul îmi este plin de rouă și părul ud de vlaga nopții.
\par 3 Haina eu mi-am dezbrăcat, cum s-o-mbrac eu iar? Picioarele mi le-am spălat, cum să le murdăresc eu iar?
\par 4 Iubitul mâna pe fereastră a întins și inima mi-a tresărit.
\par 5 Iute să-i deschid m-am ridicat, din mână mir mi-a picurat, mir din degete mi-a curs pe închizătoarea ușii.
\par 6 Celui drag eu i-am deschis, dar iubitul meu plecase; sufletu-mi încremenise, când cel drag mie-mi vorbise; iată eu l-am căutat, dar de-aflat nu l-am aflat; pe nume l-am tot strigat, dar răspuns nu mi s-a dat.
\par 7 Întâlnitu-m-au străjerii, cei ce târgul străjuiesc, m-au izbit și m-au rănit și vălul mi l-au luat, cei ce zidul îl păzesc.
\par 8 Fete din Ierusalim, vă jur: De-ntâlniți pe dragul meu, ce să-i spuneți oare lui? Că-s bolnavă de iubire.
\par 9 Ce are iubitul tău mai mult ca alții, o tu, cea mai frumoasă-ntre femei? Cu cît iubitul tău e mai ales ca alți iubiți, ca să ne rogi așa cu jurământ?
\par 10 Iubitul meu e alb și rumen, și între zeci de mii este întâiul.
\par 11 Capul lui, aur curat; părul lui, păr ondulat, negru-nchis, pană de corb.
\par 12 Ochii lui sunt porumbei, ce în lapte trupu-și scaldă, la izvor stând mulțumiți.
\par 13 Trandafir mirositor sunt obrajii lui, strat de ierburi aromate. Iar buzele lui, la fel cu crinii roșii, în mir mirositor sunt scăldate.
\par 14 Braiele-i sunt drugi de aur cu topaze împodobite; pieptul lui e scut de fildeș cu safire ferecat.
\par 15 Stâlpi de marmură sunt picioarele lui, pe temei de aur așezate. Înfățișarea lui e ca Libanul și e măreț ca cedrul.
\par 16 Gura lui e negrăit de dulce și totul este în el fermecător; iată cum este al meu iubit, fiice din Ierusalim, iată cum este al meu mire!

\chapter{6}

\par 1 Unde s-a dus iubitul tău, cea mai frumoasă-ntre femei? Unde-a plecat al tău iubit, ca să-l cutăm și noi cu tine?
\par 2 Iubitul meu în grădina lui s-a dus, în straturi d-aromate pline, să-și pască turma acolo și crini frumoși s-adune.
\par 3 Eu a iubitului meu sunt și el este al meu, el printre crini își paște iezii.
\par 4 Frumoasă ești, iubita mea, frumoasă ești ca Tirța și ca Ierusalimul dragă, dar ca și oastea în război temută.
\par 5 Întoarce-ți ochii de la mine, că ei de tot mă scot din fire.
\par 6 Părul tău turme de capre pare, ce din munți, din Galaad coboară. Dinții tăi par turmă de oi tunse, ce din scăldătoare ies, făcând două șiruri strânse și neavând nici o știrbitură.
\par 7 Două jumătăți de rodii par obrajii tăi, sub vălul tău cel străveziu.
\par 8 Solomon are șaizeci de regine și optzeci de concubine, iar fecioare socoteala cine le-o mai ține!
\par 9 Dar ea e numai una, porumbița mea, curata mea; una-i ea la a ei mamă, singură născută în casă. Fetele când au văzut-o, laude i-au înălțat, iar reginele și concubinele osanale i-au cântat.
\par 10 Cine-i aceasta, ziceau ele, care ca zarea strălucește și ca luna-i de frumoasă, ca soarele-i de luminoasă și ca oastea de război temută?
\par 11 La grădina nucilor m-am dus, ca să văd verdeața văii, dacă a dat vita de vie și dacă merii au înflorit.
\par 12 Și nu știu cum s-a petrecut, că a mea inimă m-a dus la oștirea de război a viteazului meu neam.
\par 13 Întoarce-te, Sulamita! Întoarce-te, fața să ți-o privim! Ce priviți la Sulamita, ca la hora din Mahanaim?

\chapter{7}

\par 1 Cât de frumoase sunt, domniță, picioarele tale în sandale! Rotundă-i coapsa ta, ca un colan, de meșter iscusit lucrat.
\par 2 Sânul tău e cupă rotunjită, pururea de vin tămâios plină; trupul tău e snop de grâu, încins frumos cu crini din câmp.
\par 3 Cei doi sâni ai tăi par doi pui de căprioară, par doi pui gemeni ai unei gazele.
\par 4 Gâtul tău e stâlp de fildeș; ochii tăi sunt parcă iezerele din Heșbon, de la poarta Bat-Rabim. Nasul tău este ca turnul din Liban, ce privește spre Damasc.
\par 5 Capul tău este măreț cum e Carmelul, iar părul ți-e de purpură; cu ale lui mândre șuvițe ții un rege în robie.
\par 6 Cât de frumoasă ești și atrăgătoare, prin drăgălășia ta, iubito!
\par 7 Ca finicul ești de zveltă și sânii tăi par struguri atârnați în vie.
\par 8 În finic eu m-aș sui - ziceam eu - și de-ale lui crengi m-aș apuca, sânii tăi mi-ar fi drept struguri, suflul gurii tale ca mirosul de mere.
\par 9 Sărutarea ta mai dulce-ar fi ca vinul, ce-ar curge din belșug spre-al tău iubit, ale lui buze-nflăcărate potolind.
\par 10 Eu sunt a lui, a celui drag. El dorul meu îl poartă.
\par 11 Hai, iubitul meu, la câmp, hai la țară să petrecem!
\par 12 Mâine hai la vie să vedem dacă a dat vița de vie, merii de-au înmugurit, de-s aproape de-nflorit. Și acolo îți voi da dezmierdările mele.
\par 13 Mandragorele miresme varsă și la noi acasă sunt multe fructe vechi și noi pe care, iubitul meu, pentru tine le-am păstrat.

\chapter{8}

\par 1 O, de mi-ai fi fost tu fratele meu și să fi supt la sinul mamei mele, atunci pe uliță de te-ntâlneam, cu drag, prelung te sărutam și nimeni nu-ndrăznea osândei să mă dea.
\par 2 Te-aș fi luat și-n casa mamei te-aș fi dus, în casa celei ce m-a născut; tu graiuri dulci mi-ai fi spus și eu cu drag ți-aș fi dat vin bun și must de rodii.
\par 3 Stânga sa este sub capul meu și cu dreapta-i mă cuprinde.
\par 4 Fete din Ierusalim, vă jur pe cerboaicele și gazele din câmp, nu treziți pe draga mea până nu-i va fi ei voia!
\par 5 Cine se înalță din pustiu, sprijinită de-al său drag? Sub mărul acesta am trezit iubirea ta, aici unde te-a născut și ți-a dat lumina zilei mama ta.
\par 6 Ca pecete pe sânul tău mă poartă, poartă-mă pe mâna ta ca pe o brățară! Că iubirea ca moartea e de tare și ca iadul de grozavă este gelozia. Săgețile ei sunt săgeți de foc și flacăra ei ca fulgerul din cer.
\par 7 Marea nu poate stinge dragostea, nici râurile s-o potolească; de-ar da cineva pentru iubire toate comorile casei sale, cu dispreț ar fi respins acela.
\par 8 Avem o mică surioară, care sâni nu are încă. Ce-am face cu sora noastră, când ea ar fi pețită?
\par 9 Zid de piatră de-ar fi ea, coroană de argint i-am face; iar ușă dac-ar fi, cu lemn de cedru am căptuși-o.
\par 10 Zid sunt eu acum și sânii mei sunt turnuri; drept aceea în ochii lui eu am aflat pacea.
\par 11 Solomon avea o vie pe coasta Baal-Hamon; el a dat via lucrătorilor, s-o lucreze și să-i dea fiecare la rod o mie de sicli de argint.
\par 12 Via mea este la mine acasă; mia de sicli să fie a ta, Solomoane, și două sute numai pentru cei ce păzesc roadele ei!
\par 13 O, tu, ce în grădini sălășluiești, prietenii vor să-ți asculte glasul; fă-mă să-l aud și eu cu ei!
\par 14 Fugi degrab, iubitul meu, sprinten ca o căprioară fii, fii ca puiul cel de cerb, peste munții cei îmbălsămați!


\end{document}