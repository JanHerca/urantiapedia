\begin{document}

\title{Iov}


\chapter{1}

\par 1 Era odată în ținutul Uz un om pe care îl chema Iov și acest om era fără prihană și drept; se temea de Dumnezeu și se ferea de ce este rău.
\par 2 Și i s-au născut șapte feciori și trei fete.
\par 3 El avea șapte mii de oi, trei mii de cămile, cinci sute de perechi de boi și cinci sute de asini și mulțime mare de slugi. Și omul acesta era cel mai de seamă dintre toți răsăritenii.
\par 4 Feciorii lui se duceau unul la altul și făceau ospețe în casele lor, fiecare la ziua lui, și trimiteau să cheme pe surorile lor ca să mănânce și să bea cu ei.
\par 5 Și apoi, când isprăveau zilele petrecerii lor, Iov chema și sfințea pe feciorii săi și se scula dis-de-dimineață și aducea arderi de tot, după numărul lor al tuturor, căci Iov zicea: "Se poate ca feciorii mei să fi păcătuit și să fi cugetat cu păcat împotriva lui Dumnezeu". Și așa făcea Iov mereu.
\par 6 Dar într-o zi îngerii lui Dumnezeu s-au înfățișat înaintea Domnului și Satan a venit și el printre ei.
\par 7 Atunci Domnul a zis către Satan: "De unde vii?" Iar Satan a răspuns Domnului și a zis: "Am dat târcoale pe pământ și m-am plimbat în sus și în jos".
\par 8 Și Domnul a zis către Satan: "Te-ai uitat la robul Meu Iov, că nu este nici unul ca el pe pământ fără prihană și drept și temător de Dumnezeu și care să se ferească de ce este rău?"
\par 9 Dar Satan a răspuns Domnului și a zis: "Ore degeaba se teme Iov de Dumnezeu?
\par 10 N-ai făcut Tu gard în jurul lui și în jurul casei lui și în jurul a tot ce este al lui, în toate părțile și ai binecuvântat lucrul mâinilor lui și turmele lui au umplut pământul?
\par 11 Dar ia întinde mâna Ta și atinge-Te de tot ce este al lui, să vedem dacă nu Te va blestema în față!"
\par 12 Atunci Domnul a zis către Satan: "Iată, tot ce are el este în puterea ta; numai asupra lui să nu întinzi mâna ta". Și Satan a pierit din fața lui Dumnezeu.
\par 13 Și într-o zi, când feciorii și fetele lui Iov mâncau și beau vin în casa fratelui lor mai mare,
\par 14 A sosit un vestitor la Iov și i-a spus: "Boii erau la arătură și asinele pășteau pe lângă ei;
\par 15 Atunci Sabeenii au năvălit asupra lor, au pus mâna pe vite, și pe robi i-au trecut prin ascuțișul sabiei. Și am scăpat numai eu singur și am venit să-ți dau de veste!"
\par 16 Nu a sfârșit vorba bine și altul a sosit și a spus: "Focul lui Dumnezeu a căzut din cer și a ars oile tale și pe robii tăi și i-a mistuit. Și am scăpat numai eu singur și am venit să-ți dau de veste!"
\par 17 Nu a sfârșit vorba bine și altul a sosit și a spus: "Caldeii, împărțiți în trei cete, au dat năvală peste cămilele tale și le-au ridicat și pe robi i-au trecut prin ascuțișul sabiei. Și am scăpat numai eu singur și am venit să-ți dau de veste!"
\par 18 Nu sfârșise vorba bine și altul a sosit și a spus: "Feciorii tăi și fetele tale mâncau și beau vin în casa fratelui lor mai mare,
\par 19 Și iată că un vânt puternic s-a stârnit dinspre pustiu și a izbit în cele patru colțuri ale casei și casa s-a prăbușit peste tineri și ei au murit. Și am scăpat numai eu singur și am venit să-ți dau de veste".
\par 20 Atunci Iov s-a sculat și-a sfâșiat veșmântul, s-a ras pe cap și, căzând la pământ, s-a închinat,
\par 21 Și a rostit: "Gol am ieșit din pântecele mamei mele și gol mă voi întoarce în pământ! Domnul a dat, Domnul a luat; fie numele Domnului binecuvântat!"
\par 22 Și întru toate acestea, Iov nu a păcătuit și nu a rostit nici un cuvânt de hulă împotriva lui Dumnezeu.

\chapter{2}

\par 1 Și iarăși au venit într-o zi îngerii lui Dumnezeu să se înfățișeze înaintea Domnului și Satan a venit și el printre ei să se înfățișeze înaintea Domnului.
\par 2 Și Domnul a zis către Satan: "De unde vii?" Iar Satan a răspuns Domnului și a zis: "Am dat târcoale pe pământ și m-am plimbat în sus și în jos".
\par 3 Și Domnul a zis către Satan: "Ai luat tu seama la robul Meu Iov? Că nu este nici unul ca el pe pământ, fără prihană și drept și temător de Dumnezeu și care să se ferească de ce este rău. El se ține cu putere în statornicia lui și tu M-ai întărâtat pe nedrept împotriva lui ca să-l prăpădesc".
\par 4 Dar Satan a răspuns Domnului și a zis: "Cojoc pentru cojoc! Că tot ce are omul dă pentru viața lui.
\par 5 Dar ia întinde-ți mâna și atinge-Te de osul și de carnea lui! Să vedem dacă nu Te va blestema în față!"
\par 6 Și Domnul a zis către Satan: "Îl dau în puterea ta! Numai nu te atinge de viața lui".
\par 7 Atunci Satan a plecat dinaintea Domnului și a lovit pe Iov cu lepră, din tălpile picioarelor până în creștetul capului.
\par 8 Și a luat Iov un ciob ca să se scarpine și ședea pe gunoi, afară din oraș.
\par 9 Atunci nevasta lui a zis către el: "Te ții mereu în statornicia ta? Blesteamă pe Dumnezeu și mori!
\par 10 Dar Iov i-a răspuns: "Vorbești cum ar vorbi una din femeile nebune! Ce? Dacă am primit de la Dumnezeu cele bune, nu vom primi oare și pe cele rele?" Și în toate acestea, Iov n-a păcătuit de loc cu buzele sale.
\par 11 Iar trei prieteni ai lui Iov au aflat despre toate aceste nenorociri care dăduseră peste el și au venit fiecare din țara lui și ei erau: Elifaz din Teman, Bildad din Șuah și Țofar din Naamah. Ei se înțeleseseră împreună să vină să împărtășească durerea lui și să-l mângâie.
\par 12 Și când ei și-au ridicat ochii de departe nu l-au mai recunoscut. Atunci au slobozit glasurile lor, s-au tânguit și și-au sfâșiat fiecare veșmântul și și-au presărat capul cu țărână.
\par 13 Apoi au șezut pe pământ, lângă el, șapte zile și șapte nopți, fără să-i spună nici un cuvânt, căci vedeau cât este de mare durerea lui.

\chapter{3}

\par 1 După aceea, Iov a deschis gura sa și a blestemat ziua în care s-a născut.
\par 2 Și Iov a vorbit și a zis:
\par 3 "Piară ziua în care m-am născut și noaptea care a zis: un prunc de parte bărbătească s-a zămislit!
\par 4 Ziua aceea să se facă întuneric și Domnului din cer să nu-I pese de ea și lumina să n-o mai lumineze.
\par 5 Bezna și umbra morții s-o cotropească, norii s-o învăluiască și toate negurile s-o înspăimânte!
\par 6 Întunericul să cuprindă noaptea aceea și să nu mai fie pusă în zilele anului și în socoteala lunilor să nu mai intre!
\par 7 Pustie să rămână noaptea aceea și nici o bucurie să nu pătrundă în ea!
\par 8 Blestemată să fie de către cei ce blesteamă zilele, de către cei ce știu să descânte Leviatanul.
\par 9 Să se întunece stelele revărsatului zorilor ei; să aștepte lumina și nimic să nu vină și să nu mai vadă genele aurorei,
\par 10 Pentru că n-a închis pântecele care m-a zămislit și n-a ascuns durerea dinaintea ochilor mei.
\par 11 De ce n-am murit când eram în sânul mamei mele? Și nu mi-am dat duhul, ieșind din pântecele ei?
\par 12 De ce m-au primit cei doi genunchi și de ce cei doi sâni mi-au dat să sug?
\par 13 Căci acum aș sta culcat și liniștit, aș dormi și m-aș odihni,
\par 14 Cu împărații și cu dregătorii pământului, care și-au zidit morminte în singurătate,
\par 15 Sau cu domnitorii care umplu de aur și de argint casele lor.
\par 16 Sau de ce n-am fost o stârpitură aruncată și ascunsă, ca acei prunci care n-au apucat să vadă lumina?
\par 17 Acolo cei nelegiuiți se astâmpără și cei împovărați se odihnesc.
\par 18 Acolo cei ce poartă lanțuri ajung la liman de pace și nu mai aud glasul paznicului.
\par 19 Mic și mare acolo sunt tot una și robul a scăpat de stăpânul său.
\par 20 Pentru ce dă Dumnezeu lumina vieții celui nenorocit și zile celor cu sufletul amărât;
\par 21 Celor ce așteaptă moartea, și ea nu vine, și care scormonesc după ea mai mult ca după o comoară;
\par 22 Celor ce se bucură cu bucurie mare și sunt plini de fericire, fiindcă au găsit un mormânt;
\par 23 Celui care nu știe încotro să meargă și pe care îl îngrădește Dumnezeu de jur-împrejur?
\par 24 Gemetele mele sunt pâinea mea și vaietele mele curg ca apa.
\par 25 De ceea ce mă tem, aceea mi se întâmplă și de ceea ce mi-e frică tocmai de aceea am parte.
\par 26 N-am nici tihnă, nici odihnă, nu-mi găsesc nici o pace și zbuciumul mă stăpânește".

\chapter{4}

\par 1 Atunci Elifaz din Teman a deschis gura și a zis:
\par 2 "Să-ți vorbim ori să nu-ți vorbim? Necazul tău e crâncen! Dar cine ar putea să-și înăbușe cuvintele?
\par 3 Iată, tu dădeai învățătură multora și întăreai multe mâini slăbite.
\par 4 Cuvintele tale au ținut în sus pe cei ce erau să cadă și tu ai întărit genunchii care se clătinau.
\par 5 Acum când ți-a venit și ție rândul, ești la strâmtorare și ți-ai pierdut firea; acum când lovitura te-a ajuns, te-ai spăimântat!
\par 6 Frica ta de Dumnezeu nu-ți dă încredere și desăvârșirea căilor tale nu-ți dă nădejde?
\par 7 Ia adu-ți aminte, care nevinovat s-a prăpădit și unde le-a pierit urma celor drepți în fala lui Dumnezeu?
\par 8 După cum am văzut eu, numai cei ce ară nelegiuirea și seamănă răutatea au parte de asemenea roade.
\par 9 La porunca Domnului, ei vor pieri, de suflarea mâniei Lui se vor stinge.
\par 10 Răcnetul leului și glasul leopardului, precum și dinții puilor de lei se sfărâmă.
\par 11 Leul bătrân moare că nu mai are ce mânca și puii leoaicei se risipesc.
\par 12 O șoaptă a răzbit până la mine și urechea mea a prins ceva din ea.
\par 13 În spaimele care vin din nălucirile nopții, atunci când somn adânc se lasă peste oameni,
\par 14 Cutremur m-a apucat și fiori mi-au scuturat toate oasele.
\par 15 Atunci un duh a trecut prin fața mea; tot părul mi s-a zbârlit de groază.
\par 16 A stat drept în picioare, dar n-am știut cine este; o umbră este înaintea ochilor mei, și aud o voce ușoară care zice:
\par 17 "Un om poate să fie drept în fața lui Dumnezeu? O făptură omenească este ea curată înaintea Celui ce a zidit-o?
\par 18 Dacă El nu se încrede în slujitorii Săi și dacă găsește vină chiar îngerilor Săi,
\par 19 Cu cât mai vârtos celor ce locuiesc în locuințe de lut, a căror obârșie este în țărână și pe care îi strivește ca pe o molie.
\par 20 De dimineață până seara sunt zdrobiți, pier pe vecie fără să-i scape nimeni.
\par 21 Mor, dar nu mor de prea multă înțelepciune".

\chapter{5}

\par 1 Strigă acum, dacă o fi cineva care să-ți răspundă. Către care din sfinții îngeri te vei îndrepta?
\par 2 Mânia ucide pe cel fără de minte, iar aprinderea omoară pe cel rătăcit.
\par 3 Am văzut pe nebun prinzând rădăcină și pe loc am blestemat sălașul lui:
\par 4 Să se depărteze copiii lui de orice izbăvire și să fie călcați în picioare la poartă și nimeni să nu le vină într-ajutor.
\par 5 Secerișul lui să-l mănânce flămânzii și să-l ducă cu ei în ascunzișuri și toată averea lui s-o soarbă însetații!
\par 6 Pentru că nelegiuirea nu iese din pământ și necazul nu răsare din pulbere,
\par 7 Ci omul își naște singur suferința, precum vulturii se ridică în aer, prin puterile lor;
\par 8 Dar eu alerg la Dumnezeu și Lui Îi arăt necazul meu.
\par 9 El face lucruri mari și nepătrunse, lucruri minunate și fără număr.
\par 10 El dă ploaie pe pământ și trimite apă pe câmpii.
\par 11 El înalță pe cei smeriți și izbăvește pe cei necăjiți.
\par 12 El destramă planurile celor vicleni și cu mâinile lor nu pot să izbândească.
\par 13 El prinde pe înțelepți în istețimea lor și sfatul celor înșelători iese prost.
\par 14 Ziua în amiaza mare dau de întuneric și umblă ca pe timp de noapte în ceas de zi.
\par 15 Dar Dumnezeu scapă din gura lor pe omul dosădit și din mâna celui puternic pe cel sărac.
\par 16 Astfel, sărmanul prinde nădejde și nedreptatea își închide gura.
\par 17 Fericit este omul pe care Dumnezeu îl mustră! Și să nu disprețuiești certarea Celui Atotputernic.
\par 18 Căci El rănește și El leagă rana, El lovește și mâinile Lui tămăduiesc.
\par 19 De șase ori din nevoi te va scoate, iar a șaptea oară, răul te va ocoli.
\par 20 În timp de foamete, te va scăpa de la moarte și în bătălie din primejdia sabiei.
\par 21 Vei fi la adăpost de biciul bârfelii și nu te vei teme de prăpăd când va veni.
\par 22 Îți vei râde de pustiire și de foamete și nu-ți va păsa de fiarele pământului.
\par 23 Căci vei avea legământ cu pietrele câmpului și fiara sălbatică va trăi cu tine în pace.
\par 24 Tu vei ști cortul tău la adăpost și când îți vei cerceta locuința nu vei avea dezamăgire.
\par 25 Vei ști că urmașii tăi sunt numeroși și că odraslele tale sunt multe ca iarba pământului.
\par 26 Sosi-vei la mormânt, la adânci bătrânețe, ca o șiră de grâu strânsă la vremea ei.
\par 27 Iată ceea ce am cercetat și așa este. Ascultă și învață spre folosul tău".

\chapter{6}

\par 1 Atunci Iov a răspuns și a grăit:
\par 2 "O, dacă durerea mea s-ar cântări și nenorocirea mea ar fi pusă la cântar!
\par 3 Și fiindcă este mai grea decât nisipul mărilor, de aceea cuvintele mele sunt bâlbâite!
\par 4 Pentru că săgețile Celui Atotputernic stau înfipte în mine și duhul meu bea veninul lor, de aceea spaimele lui Dumnezeu vin cete-cete împotriva mea.
\par 5 Zbiară măgarul sălbatic când e lângă pășunea verde? Mugește boul când stă lângă nutreț?
\par 6 Poți să mănânci ce n-are sare fără sare? Are vreun gust albușul oului?
\par 7 Sufletul meu n-a voit să se atingă de ele; inima mea s-a dezgustat de pâinea mea.
\par 8 Cine îmi va dărui îndeplinirea rugăciunii mele și va face ca Dumnezeu să-mi dea ce aștept,
\par 9 Și să primească să mă sfărâme și să-și întindă mâna și să mă nimicească!
\par 10 Dar va fi încă o mângâiere pentru mine și voi tresălta, deși împovărat de dureri nemiloase, fiindcă n-am ascuns poruncile Celui Sfânt.
\par 11 Ce putere mai am ca să aștept și ce viitor mai am ca să prelungesc viața mea?
\par 12 Tăria mea este tăria pietrelor? Trupul meu este oare de aramă?
\par 13 Aș putea găsi vreun sprijin în mine și tot ajutorul n-a fugit, oare, departe de mine?
\par 14 Celui ce este în suferință i se cuvine mila prietenului său, dar el uită teama de Cel Atotputernic.
\par 15 Frații mei s-au arătat înșelători ca un puhoi, ca albia puhoaielor repezi.
\par 16 Erau acoperite de ghiață, zăpada stătea grămadă peste ele;
\par 17 Dar cum se topește zăpada, ele și seacă și, cum se încălzește, ele se usucă pe loc.
\par 18 Caravanele se abat din drumul lor, ele înaintează în pustiu și se rătăcesc.
\par 19 Caravanele din Tema așteaptă, convoaiele din Saba nădăjduiesc în ele.
\par 20 Și sunt uimiți că au avut încredere; când sosesc lângă uscatele puhoaie sunt uimiți.
\par 21 Așa ați fost și voi acum pentru mine: Vă scutură spaima și vă este frică!
\par 22 Nu cumva v-am zis: Dați-mi de pomană și împărțiți din averile voastre pentru mine?
\par 23 Sau scăpați-mă din mâna unui dușman sau răscumpărați-mă din mâna tiranilor?
\par 24 Fiți învățătorii mei și eu voi tăcea; lămuriți-mă unde este păcatul meu!
\par 25 Cât de îmbietoare sunt cuvintele întregimii sufletești! Dar ce judecă judecata care vine de la voi?
\par 26 Cugetați voi să faceți judecata vorbelor? Ducă-se în vânt cuvintele unui deznădăjduit!
\par 27 Voi năpăstuiți pe orfan, voi împovărați pe prietenul vostru.
\par 28 Și acum întrebați și vă întoarceți către mine și în fața voastră nu voi spune minciună!
\par 29 Cercetați din nou! Nu este nici o viclenie! Cercetați din nou! Dreptatea mea este mereu aici!
\par 30 Este oare vreo strâmbătate pe limba mea și cerul gurii mele nu va deosebi el ce este rău și ce este amar?

\chapter{7}

\par 1 Oare omul pe pământ nu este ca într-o slujbă ostășească și zilele lui nu sunt ca zilele unui simbriaș?
\par 2 El este asemenea robului care suspină după umbră, asemenea năimitului care-și așteaptă simbria.
\par 3 Astfel și eu am avut parte de luni de durere, și mi-au fost date nopți de suferință.
\par 4 Dacă mă culc, zic: Când va veni ziua? Dacă mă scol, mă întreb: Când va veni seara? Și sunt năpădit de fel de fel de arătări până la asfințit.
\par 5 Trupul meu e plin de păduchi și de solzi de murdărie; pielea mea crapă și se zbârcește.
\par 6 Zilele mele au fost mai grabnice ca suveica și s-au isprăvit, fiindcă firul s-a terminat.
\par 7 Adu-ți aminte, Doamne, că viața mea e o suflare, că ochiul meu nu va mai vedea fericirea.
\par 8 Ochiul celui ce mă vedea nu mă va mai zări; ochii Tăi mă vor căuta, dar eu nu voi mai fi.
\par 9 Negura se risipește, piere, tot astfel cel ce coboară în iad nu mai vine înapoi.
\par 10 Nu se mai înapoiază în casa sa și locuința sa nu-l mai cunoaște.
\par 11 Drept aceea nu voi pune strajă gurii mele, ci voi vorbi întru deznădejdea duhului meu și mă voi plânge întru amărăciunea inimii mele.
\par 12 Sunt eu, oare, oceanul sau balaurul din ocean, ca să pui să mă păzească?
\par 13 Când gândesc: Patul meu mă va odihni, culcușul meu îmi va alina durerile,
\par 14 Atunci Tu mă spăimântezi cu vise și mă îngrozești cu năluciri.
\par 15 Pentru aceea, sufletul meu ar vrea mai bine ștreangul, mai bine moartea decât aceste chinuri.
\par 16 Mă isprăvesc, nu voi trăi în veac; lasă-mă, căci zilele mele sunt o suflare.
\par 17 Ce este omul, ca să-ți bați capul cu el și ca să-i dai luarea Ta aminte?
\par 18 De ce îl cercetezi în fiecare dimineață și de ce îl urmărești în orice clipă?
\par 19 Când vei înceta să mă privești? Când îmi vei da răgaz să-mi înghit saliva?
\par 20 Dacă am greșit, ce ți-am făcut ție, Păzitorule de oameni? De ce m-ai luat țintă pentru săgețile Tale și de ce ți-am ajuns povară?
\par 21 De ce nu îngădui greșeala mea și nu lași să treacă fărădelegea mea? Degrabă mă voi culca în țărână; mă vei căuta, dar nu mă vei mai găsi".

\chapter{8}

\par 1 Atunci Bildad din Șuah a răspuns și a zis:
\par 2 "Până când vei tot vorbi astfel de lucruri și cuvintele din gura ta vor izvorî ca vijelia?
\par 3 Dumnezeu o să încovoaie ce e drept? Cel Atotputernic o să strâmbe El dreptatea?
\par 4 Dacă feciorii tăi au păcătuit față de El, El i-a lăsat să se prăbușească sub povara nelegiuirii lor.
\par 5 Dar tu dacă vii la Dumnezeu, dacă te rogi de Cel Atotputernic,
\par 6 Dacă ești nevinovat și fără pată, atunci de bună seamă că va veghea asupra ta și va clădi la loc casa dreptății tale
\par 7 Și starea ta cea veche va fi nimica toată, atât de mult vei fi deasupra în starea ta cea nouă.
\par 8 Întreabă pe cei care au fost înaintea noastră și ia aminte la cele trăite și pățite de părinți.
\par 9 Căci noi suntem de ieri și nu știm nimic, căci zilele noastre pe pământ nu sunt decât o umbră.
\par 10 Ei îți vor da învățătură, ei îți vor grăi și din inima lor îți vor cuvânta unele ca acestea:
\par 11 Oare papura crește fără baltă și rogozul fără umezeală?
\par 12 Pe când încă este în floare și nu este tăiat, el se usucă, mai înainte decât orice buruiană.
\par 13 Tot așa se întâmplă cu toți aceia care uită pe Dumnezeu și așa se veștejește nădejdea celui nelegiuit.
\par 14 Încrederea lui e spulberată și bizuința lui este o pânză de păianjen.
\par 15 Se sprijină pe casa sa, dar ea nu se ține; se agață de ea, dar casa se prăvale.
\par 16 Stă plin de suc în fața soarelui și în grădina unde este își întinde vlăstarii;
\par 17 Rădăcinile lui se împletesc cu pietrele și se înfig în adâncul stâncilor.
\par 18 Dacă îl smulgi din loc, locul îl tăgăduiește: Nu te-am văzut niciodată!
\par 19 Iată-l acum putred, pe cărare, și din pământ răsar alți vlăstari.
\par 20 Dumnezeu nu disprețuiește pe cel desăvârșit și nu ia de mână pe răufăcători.
\par 21 Gura ta va fi plină încă o dată de râsete și buzele tale de veselie.
\par 22 Cei ce te urăsc se vor înveșmânta în rușine și cortul celor răi va pieri!"

\chapter{9}

\par 1 Atunci Iov a răspuns și a zis:
\par 2 "Știu bine că așa este; căci cum ar putea un om să fie drept înaintea lui Dumnezeu?
\par 3 Dacă ar fi să se certe cu El, din o mie de lucruri n-ar putea să-I răspundă nici la unul singur.
\par 4 A Lui este înțelepciunea și atotputernicia; cine ar putea să-I stea împotrivă și să rămână teafăr?
\par 5 El mișcă munții din loc fără ca ei să prindă de veste că El i-a răsturnat în mânia Lui.
\par 6 El zguduie pământul din temelia lui, așa încât stâlpii lui se clatină.
\par 7 El poruncește soarelui și soarele nu se mai ridică. El pune pecetea Lui asupra stelelor.
\par 8 El singur este Cel ce întinde cerurile și umblă pe valurile mării.
\par 9 El a zidit Carul mare, Ralița, Pleiadele și cămările stelelor de miazăzi.
\par 10 El a făcut lucruri mari și nepătrunse și minuni fără de număr.
\par 11 Iată, dacă trece pe lângă mine, eu nu-L văd, și dacă se strecoară, eu nu-I prind de veste.
\par 12 Dacă ia și ridică, cine va putea să-L oprească și cine-I va zice: Ce ai făcut?
\par 13 Dumnezeu nu-și înfrânează mânia Sa și sub El se încovoaie toți slujitorii mândriei.
\par 14 Și eu atunci cum o să-I răspund și ce cuvinte o să aleg?
\par 15 Chiar dacă aș avea dreptate, nu-I voi răspunde, ci mă voi ruga judecătorului.
\par 16 Chiar dacă m-ar asculta, când Îl chem, tot n-aș putea să cred că ascultă glasul meu,
\par 17 Căci El mă sfărâmă ca sub furtună și înmulțește fără cuvânt rănile mele.
\par 18 El nu-mi dă răgaz să răsuflu și mă adapă cu amărăciune.
\par 19 Dacă este vorba de putere, El este Cel puternic. Dacă este vorba de judecată, cine mă va apăra?
\par 20 Oricâtă dreptate aș avea, gura mea mă va osândi și dacă sunt fără prihană ea mă scoate vinovat.
\par 21 Sunt oare desăvârșit? Eu singur nu mă cunosc pe mine și viața mea o disprețuiesc.
\par 22 Pentru aceea am zis: Tot una este! El nimicește pe cel desăvârșit și pe cel viclean.
\par 23 Dacă o nenorocire aduce moartea deodată, ce-I pasă Lui de deznădejdea celor fără de vină?
\par 24 Dacă o țară a încăput pe mâna unui om viclean, El acoperă fața judecătorilor Săi. Și dacă nu El, cine atunci?
\par 25 Zilele mele sunt mai grabnice decât un aducător de vești și au fugit fără să vadă fericirea.
\par 26 S-au strecurat ca niște bărci de papură, ca un vultur care se năpustește asupra prăzii sale.
\par 27 Dacă zic: Vreau să-mi uit suferința, să-mi schimb înfățișarea și să fiu voios,
\par 28 Sunt năpădit de teama chinurilor mele, știind bine că Tu nu mă vei scoate nevinovat.
\par 29 Dacă sunt vinovat, de ce să mă mai trudesc zadarnic?
\par 30 Dacă m-aș spăla cu zăpadă și mi-aș curăți mâinile cu leșie,
\par 31 Atunci Tu tot m-ai cufunda în noroi, încât și veșmintele mele s-ar scârbi de mine.
\par 32 Căci Dumnezeu nu este un om ca mine, ca să stau cu El de vorbă și ca să mergem împreună la judecată.
\par 33 Între noi nu se află un al treilea care să-și pună mâna peste noi amândoi
\par 34 Și care să depărteze varga Sa de deasupra capului meu, așa încât groaza Lui să nu mă mai tulbure;
\par 35 Atunci aș vorbi și nu m-aș mai teme de El. Dar nu este așa și eu sunt singur cu mine însumi.

\chapter{10}

\par 1 Sufletul meu este dezgustat de viața mea. Voi lăsa să curgă slobodă tânguirea mea și voi vorbi întru suferința sufletului meu.
\par 2 Voi spune către Domnul: Nu mă osândi; lămurește-mă, să știu pentru ce Te cerți cu mine.
\par 3 Care e folosul Tău, când ești aprig și disprețuiești făptura mâinilor Tale și ești surâzător la sfatul celor răi?
\par 4 Ai Tu ochi materiali și vezi lucrurile precum le vede omul?
\par 5 Zilele Tale sunt oare ca zilele omului și anii Tăi ca anii omenești,
\par 6 Ca să cercetezi fărădelegea mea și să cauți păcatul meu,
\par 7 Când știi bine că nu sunt vinovat și că nimeni nu mă poate scăpa din mâna Ta?
\par 8 Mâinile Tale m-au făcut și m-au zidit și apoi Tu mă nimicești în întregime.
\par 9 Adu-ți aminte că m-ai făcut din pământ și că mă vei întoarce în țărână.
\par 10 Nu m-ai turnat oare ca pe lapte și nu m-ai închegat ca pe caș?
\par 11 M-ai îmbrăcat în piele și în carne, m-ai țesut din oase și din vine;
\par 12 Apoi mi-ai dat viață, și bunăvoința Ta și purtarea Ta de grijă au ținut vie suflarea mea,
\par 13 Și ceea ce Tu țineai ascuns în inima Ta, iată știu acum gândul Tău:
\par 14 Dacă păcătuiesc, Tu mă supraveghezi și nu mă dezvinovățești de greșeala mea.
\par 15 Dacă sunt vinovat este amar de mine și dacă sunt drept nu cutez să ridic capul, ca unul ce sunt sătul de ocară și sunt apăsat de necazuri.
\par 16 Și astfel fără vlagă cum sunt, Tu mă vânezi ca un leu și din nou Te arăți minunat față de mine.
\par 17 Tu înnoiești dușmănia Ta împotriva mea. Tu sporești mânia Ta asupră-mi ca niște oștiri primenite care se luptă cu mine.
\par 18 De ce m-ai scos din sânul mamei mele? Aș fi murit și nici un ochi nu m-ar fi văzut!
\par 19 Aș fi fost ca unul care n-a fost niciodată și din pântecele mamei mele aș fi trecut în mormânt.
\par 20 Nu sunt oare zilele mele destul de puține? Dă-Te la o parte ca să pot să-mi vin puțin în fire,
\par 21 Mai înainte ca să plec spre a nu mă mai întoarce din ținutul întunericului și al umbrelor morții,
\par 22 Țara de întuneric și neorânduială unde lumina e totuna cu bezna".

\chapter{11}

\par 1 Atunci Țofar, din Naamah, a luat cuvântul și a vorbit:
\par 2 "Cel ce înșiră atâtea vorbe să nu primească nici un răspuns și tocmai vorbărețul să aibă dreptate?
\par 3 Toate câte le-ai spus îi vor face pe oameni să tacă și vei râde de ei, fără ca nimeni să te înfrunte?
\par 4 Fiindcă tu zici: Credința mea este curată și în ochii Tăi n-am nici o vină.
\par 5 Dar cine va face pe Dumnezeu să vorbească, să Își deschidă buzele spre tine,
\par 6 Și să-ți destăinuiască tainele înțelepciunii? (căci ele sunt cu anevoie de înțeles); atunci de-abia vei ști că Dumnezeu îți cere socoteală de greșeala ta.
\par 7 Descoperi-vei tu care este firea lui Dumnezeu? Urca-vei tu până la desăvârșirea Celui Atotputernic?
\par 8 Ea este mai înaltă decât cerurile. Și ce vei face tu? Ea este mai adâncă decât împărăția morții. Cum vei pătrunde-o tu?
\par 9 Măsura ei este mai lungă decât pământul și mai lată decât marea.
\par 10 Dacă trece cu vederea, dacă ține ascuns, dacă dă pe față, cine poate să-L oprească?
\par 11 El cunoaște pe cei ce trăiesc din înșelăciune, El vede nedreptatea și o ține în seamă;
\par 12 Astfel deci un om fără minte câștigă înțelepciune, precum puiul de asin ajunge asin mare.
\par 13 Cât despre tine, dacă inima ta e credincioasă și dacă întinzi mâinile către El,
\par 14 Și depărtezi de mâna ta fărădelegea ei și nu rabzi să locuiască nedreptatea în corturile tale,
\par 15 Atunci vei ridica fruntea ta fără pată pe ea, vei fi puternic și fără frică.
\par 16 Fiindcă vei uita necazul tău de azi și-ți vei aduce aminte de el numai ca de niște ape, care au fost și au trecut.
\par 17 Și viața ta va înflori mai mândră decât miezul zilei, iar întunericul se va face revărsat de zori.
\par 18 Atunci tu vei fi la adăpost, căci vei fi plin de nădejde, te vei simți ocrotit și te vei culca fără grijă.
\par 19 Te vei întinde fără ca să te strâmtoreze nimeni și mulți vor răsfăța obrazul tău.
\par 20 Dar ochii nelegiuiților tânjesc și loc de scăpare nu au, iar nădejdea este când își vor da sufletul".

\chapter{12}

\par 1 Atunci Iov a răspuns și a zis:
\par 2 "Cu adevărat numai voi sunteți înțelepți și înțelepciunea va muri o dată cu voi.
\par 3 Dar și eu am minte ca voi și nu sunt mai prejos decât voi și cine nu cunoaște lucrurile pe care mi le-ați spus?
\par 4 Eu am ajuns pricină de batjocură pentru prietenul meu, eu care chem pe Dumnezeu și căruia El răspunde: cel drept, cel fără vină e pricină de râs.
\par 5 Să disprețuim nenorocirea (gândesc cei fericiți); încă o lovitură celor ce se poticnesc.
\par 6 Foarte liniștite stau și sunt corturile jefuitorilor și cei ce mânie pe Dumnezeu sunt plini de încredere, ca unii care au făcut din pumnul lor un dumnezeu.
\par 7 Dar ia întreabă dobitoacele și te vor învăța, și păsările cerului, și te vor lămuri;
\par 8 Sau vorbește cu pământul, și-ți va da învățătură și peștii mării îți vor istorisi cu de-amănuntul.
\par 9 Cine nu cunoaște din toate acestea că mâna Domnului a făcut aceste lucruri?
\par 10 În mâna Lui El ține viața a tot ce trăiește și suflarea întregii omeniri.
\par 11 Urechea nu deosebește ea cuvintele tot așa, precum cerul gurii deosebește mâncarea?
\par 12 Oare nu la bătrâni sălășluiește înțelepciunea și priceperea nu merge mână în mână cu vârsta înaintată?
\par 13 La Dumnezeu se află înțelepciunea și puterea; sfatul și pătrunderea sunt ale Lui.
\par 14 Ceea ce dărâmă El, nimeni nu mai zidește la loc și pe cine-l închide, nimeni nu poate să-l mai deschidă.
\par 15 Dacă oprește apele pe loc, ele scad și pier; dacă le dă drumul, ele răstoarnă lumea;
\par 16 Tăria și înțelepciunea sunt la El. El este stăpân și peste rătăcit și peste cel ce-l face să rătăcească.
\par 17 El gonește pe sfetnici în picioarele goale și pe judecători îi aruncă pradă nebuniei.
\par 18 El destramă puterea împăraților și pune cingătoare de frânghie în jurul coapselor lor.
\par 19 El gonește pe preoți în picioarele goale și dă peste cap pe cei puternici.
\par 20 El taie vorba celor meșteri la cuvânt și ia mintea celor bătrâni.
\par 21 El face de ocară pe cei mari și slăbește încingătoarea celor voinici.
\par 22 El scoate din întuneric lucrurile ascunse și aduce la lumină ceea ce era acoperit de umbră.
\par 23 El sporește neamurile și apoi le pierde, El le lasă să se întindă și apoi le strâmtorează.
\par 24 El scoate din minți pe căpeteniile popoarelor și îi lasă să rătăcească în singurătăți fără cărări.
\par 25 Acolo ei orbecăiesc în întuneric, fără nici o lumină, căci Dumnezeu îi lasă să se împleticească aidoma celui ce s-a îmbătat.

\chapter{13}

\par 1 De bună seamă, ochiul meu a văzut toate acestea, urechea mea le-a auzit și le-a înțeles.
\par 2 Ceea ce știți voi, știu și eu și nu sunt deloc mai prejos decât voi.
\par 3 Dar eu vreau să vorbesc cu Cel Atotputernic, vreau să-mi apăr pricina înaintea lui Dumnezeu.
\par 4 Căci voi sunteți niște născocitori ai minciunii, sunteți cu toții niște doctori neputincioși!
\par 5 Ce bine ar fi fost dacă ați fi tăcut! Câtă înțelepciune ar fi fost din partea voastră!
\par 6 Ascultați acum apărarea mea și băgați de seamă la rostirea buzelor mele.
\par 7 Oare de dragul lui Dumnezeu spuneți voi lucruri strâmbe și spre apărarea Lui croiți minciuni?
\par 8 Voiți să țineți cu El și să fiți apărătorii Lui?
\par 9 Și dacă ar fi ca să vă cerceteze, ar fi bine de voi, sau vreți să-L înșelați cum înșelați un om?
\par 10 Desigur El vă va osândi dacă în ascuns vreți să fiți părtinitori cu El.
\par 11 Măreția Lui oare nu vă înfricoșează și groaza Lui nu va cădea oare peste voi?
\par 12 Rostirile voastre au tăria cenușei. Răspunsurile voastre se prefac în noroi.
\par 13 Închideți gura în fața mea și eu voi vorbi, orice ar fi să se întâmple.
\par 14 Drept aceea îmi voi lua în dinți carnea mea și viața mea o pun în mâna mea.
\par 15 Dacă o fi să mă ucidă, nu voi tremura, dar voi descurca în fața Sa firele pricinei mele,
\par 16 Și chiar aceasta îmi va fi mie izbândă, fiindcă un nelegiuit nu se înfățișează înaintea Lui.
\par 17 Ascultați cu luare-aminte cuvintele mele și ceea ce vă voi spune să vă rămână în urechi.
\par 18 Iată am pus la cale o judecată, și știu că eu sunt cel ce am dreptate.
\par 19 Are cineva ceva de spus împotriva mea? Atunci voi amuți degrabă și voi aștepta moartea.
\par 20 Numai scutește-mă de două lucruri, și nu mă voi ascunde de fața Ta.
\par 21 Depărtează mâna Ta de deasupră-mi și nu mă mai tulbura cu groaza Ta.
\par 22 Apoi cheamă-mă și eu iți voi răspunde, sau lasă-mă să vorbesc eu și Tu să-mi dai răspuns.
\par 23 Câte greșeli și câte păcate am făcut? Dă-mi pe față călcarea mea de lege și păcatul meu.
\par 24 De ce ascunzi fața Ta și mă iei drept un dușman al Tău?
\par 25 Vrei oare, să înspăimânți o frunză pe care o bate vântul? Vrei să Te îndârjești împotriva unui pai uscat?
\par 26 De ce să scrii împotriva mea aceste hotărâri amare? De ce să-mi scoți ochii cu greșelile tinereții?
\par 27 De ce să-mi vâri picioarele în butuci și să pândești toți pașii mei și toate urmele mele?
\par 28 Când Tu știi că trupul meu se nimicește ca un putregai și ca o haină mâncată de molii!

\chapter{14}

\par 1 Omul născut din femeie are puține zile de trăit, dar se satură de necazuri.
\par 2 Ca și floarea, el crește și se veștejește și ca umbra el fuge și e fără durată.
\par 3 Și asupra lui privești și pe mine Tu mă silești să vin la judecată cu Tine.
\par 4 Cine ar putea să scoată ceva curat din ceea ce este necurat? Nimeni!
\par 5 Deoarece zilele lui sunt măsurate și știi socoteala lunilor lui și i-ai pus un hotar peste care nu va trece.
\par 6 Întoarce-ți privirea de la el, ca să aibă puțin răgaz, să se poată bucura ca simbriașul la sfârșitul zilei (de muncă).
\par 7 Un copac, de pildă, tot are nădejde, căci dacă-l tai, el crește din nou și vlăstarii nu-i vor lipsi.
\par 8 Dacă rădăcina lui îmbătrânește în pământ și dacă trunchiul lui putrezește,
\par 9 Când dă de apă înverzește din nou și se acoperă cu ramuri ca și cum ar fi atunci sădit.
\par 10 Dar omul când moare rămâne nimicit; când omul își dă sufletul, unde mai este el?
\par 11 Apele mărilor pot să dispară, fluviile pot să scadă și să sece.
\par 12 La fel și omul se culcă și nu se mai scoală; și cât vor sta cerurile, el nu se mai deșteaptă și nu se mai trezește din somnul lui.
\par 13 O, de m-ai ascunde în împărăția morților, ca să mă ții acolo până când va trece mânia Ta, și de mi-ai soroci o vreme, când iarăși să-ți aduci aminte de mine!
\par 14 Dacă omul a murit o dată, fi-va el iarăși viu? Toate zilele robiei mele aș aștepta până ce vor veni să mă schimbe.
\par 15 Atunci Tu mă vei chema și eu Îți voi răspunde; Tu vei cere înapoi lucrul mâinilor Tale.
\par 16 Pe când astăzi Tu numeri pașii mei; atunci Tu nu vei mai lua seama la păcatul meu.
\par 17 Nelegiuirea mea ar fi pecetluită ca într-un sac și greșeala mea ai spăla-o și ai face-o albă.
\par 18 Și precum muntele se dărâmă și se preface în nisip și precum stânca e rostogolită din locul ei,
\par 19 Și precum apele mănâncă pietrele și valurile lor acoperă pământul, tot așa Tu sfărâmi nădejdea omului.
\par 20 Tu Te ridici uriaș împotriva lui, și el se nimicește; Tu schimbi înfățișarea lui și-l trimiți de la Tine.
\par 21 Dacă feciorii lui ajung la mare cinste, el nu mai știe; dacă au ajuns de râsul lumii, el nu-i mai vede.
\par 22 Carnea lui e în întristare mare numai pentru el. Sufletul lui numai pentru el e cuprins de jale".

\chapter{15}

\par 1 Atunci Elifaz, din Teman, a răspuns și a zis:
\par 2 "Este oare cinstit pentru înțelept să răspundă cu cuvinte ușuratice și să-și umple pieptul cu suflarea vântului de răsărit?
\par 3 I se cuvine lui să judece cu vorbe seci și prin cuvântări care n-au nici o noimă?
\par 4 Tu mergi atât de departe, încât desființezi cucernicia și nesocotești rugăciunea înaintea lui Dumnezeu.
\par 5 Nelegiuirea ta insuflă gura ta și tu împrumuți vorbirea ta de la cei vicleni.
\par 6 Chiar gura ta te osândește și nu eu, chiar buzele tale sunt martore împotriva ta.
\par 7 Nu cumva ești tu cel dintâi om care s-a născut? Venit-ai tu pe lume mai înainte decât munții?
\par 8 Ai stat tu de sfat cu Dumnezeu și te-ai făcut tu stăpân pe toată înțelepciunea?
\par 9 Ce știi tu pe care să nu-l știm și noi? Ce pricepi tu și noi nu pricepem?
\par 10 Printre noi se află oameni vechi de zile, bătrâni mai în vârstă decât tatăl tău.
\par 11 Ți se pare puțin lucru mângâierile în numele lui Dumnezeu și cuvintele spuse cu blândețe?
\par 12 De ce te lași târât de inima ta și de ce privești așa trufaș cu ochii tăi?
\par 13 De ce întorci spre Dumnezeu mânia ta și dai drumul din gura ta la astfel de cuvântări?
\par 14 Ce este omul ca să se creadă curat, și cel născut din femeie, ca să se creadă neprihănit?
\par 15 Dacă Dumnezeu nu are încredere în sfinții Săi și dacă cerurile nu sunt destul de curate înaintea ochilor Săi,
\par 16 Cu atât mai puțin o făptură urâcioasă și stricată cum este omul cel ce bea nedreptatea ca apa.
\par 17 Vreau să-ți dau o învățătură, ascultă-mă; și ceea ce am văzut vreau să-ți aduc la cunoștință;
\par 18 Ceea ce înțelepții au vestit fără să ascundă nimic, precum au auzit de la părinții lor,
\par 19 Atunci când țara le-a fost dată numai lor și nici un străin nu se așezase încă printre ei.
\par 20 Nelegiuitul se chinuiește în toate zilele vieții sale și de-a lungul anilor hărăziți celui tiran.
\par 21 Glasuri îngrozitoare fac larmă în urechile lui; în mijlocul păcii, i se pare că un ucigaș se năpustește asupra lui.
\par 22 El nu mai nădăjduiește să mai iasă din întuneric și își simte capul mereu sub sabie.
\par 23 Se și vede aruncat de mâncare vulturilor, fiindcă știe că prăpădul lui este fără întârziere.
\par 24 Ziua întunericul îl înspăimântă. Zbuciumul și tulburarea îl strâng la mijloc și se aruncă asupra-i gata de împresurare,
\par 25 Fiindcă a îndrăznit să-și îndrepte mâna împotriva lui Dumnezeu și să facă pe viteazul față de Cel Atotputernic;
\par 26 Fiindcă a îndrăznit să năvălească împotriva Lui cu gâtul întins și la adăpostul scuturilor sale groase și rotunde.
\par 27 Chipul lui i se ascundea în grăsime și osânza stătea grea pe coapsele lui,
\par 28 Și sălășluia în cetăți pustiite, în case în care nu mai stătea nimeni, fiindcă amenințau să se prăbușească.
\par 29 Nu va aduna bogăție și ce are nu va ține mult, iar umbra lui nu se va lungi pe pământ.
\par 30 Nu va mai putea să iasă din întuneric. Focul va mistui ramurile sale și vijelia va mătura florile lui;
\par 31 Să nu se creadă în minciună, fiindcă știm că e deșertăciune.
\par 32 Vrejul său se va ofili mai înainte de vreme și mlădița sa nu va da muguri verzi.
\par 33 Ca vița mănată, va lăsa să cadă rodul său și la fel ca măslinul va împrăștia florile sale.
\par 34 Fiindcă ceata celui rău la inimă va fi lăsată stearpă și focul mistuie corturile cu bogății de jaf.
\par 35 Ei zămislesc răutatea și nasc nelegiuirea, dar cu aceasta pântecele lor dospește înșelăciunea".

\chapter{16}

\par 1 Atunci Iov a răspuns și a grăit:
\par 2 "Am auzit mereu astfel de lucruri; sunteți toți niște jalnici mângâietori.
\par 3 Când se vor sfârși aceste vorbe goale și ce te chinuiește ca să răspunzi?
\par 4 Și eu aș vorbi așa ca voi, dacă sufletul vostru ar fi în locul sufletului meu; aș putea să spun multe cuvinte împotriva voastră și să dau din cap în privința voastră.
\par 5 V-aș mângâia numai cu gura și cu mișcarea buzelor mele v-aș aduce ușurare.
\par 6 Dar, dacă vorbesc, durerea mea nu se liniștește și dacă tac din gură, durerea mea nu se depărtează de la mine.
\par 7 În ceasul de față, Dumnezeu mi-a luat toată vlaga; toată ticăloșia mea mă împresoară, Doamne!
\par 8 M-ai acoperit cu zbârcituri, care toate mărturisesc împotriva mea; neputința mea ea însăși mă dă de gol și bârfitorul stă împotriva mea.
\par 9 El mă sfâșie în furia Lui și se poartă cu mine dușmănos, scrâșnește din dinți împotriva mea; dușmanul meu aruncă asupră-mi săgețile ochilor săi;
\par 10 Deschis-au gura lor împotriva mea, în batjocură m-au lovit peste obraji. Toți grămadă se înghesuie împotriva mea.
\par 11 Dumnezeu mă dă pe mâna unui păgân. El mă aruncă pradă celor răi.
\par 12 Mi-era destul de bine, dar El m-a sfărâmat. M-a luat de ceafă și m-a făcut praf și a aruncat asupră-mi toate săgețile Sale;
\par 13 În jurul meu se învârtesc săgețile Sale; El îmi străpunge rărunchii fără milă; El varsă pe pământ fierea mea.
\par 14 El mă dărâmă bucată cu bucată și năvălește asupra mea ca un războinic.
\par 15 Am cusut un sac pe trupul meu și am vârât în țărână capul meu.
\par 16 Chipul meu s-a înroșit de plânset și umbra morții s-a sălășluit în pleoapele mele;
\par 17 Și cu toate acestea, în mâinile mele nu este nici o silnicie și rugăciunea mea este curată!
\par 18 Pământule, nu ascunde sângele meu și să nu fie nici un loc nestrăbătut de bocetele mele.
\par 19 Iar acum martorul meu este în ceruri și cel ce dă pentru mine bună mărturie este sus în locurile înalte.
\par 20 Prietenii mei își bat joc de mine, dar ochiul meu varsă lacrimi înaintea lui Dumnezeu.
\par 21 O, de-ar fi îngăduit omului să stea de vorbă cu Dumnezeu, cum stă de vorbă un am "u prietenul său!
\par 22 Căci acești puțini ani se vor scurge și voi apuca pe un drum de pe care nu mă voi mai întoarce.

\chapter{17}

\par 1 Sufletul meu e dărăpănat, zilele mele se sting, mormântul mă așteaptă.
\par 2 Sunt împresurat de batjocoritori și ochii mei trebuie să privească spre ocările lor.
\par 3 Dă-mi acum chezășia Ta lângă Tine, altfel cine ar vrea să răspundă pentru mine?
\par 4 Pentru că Tu ai luat priceperea din inima lor, de aceea Tu nu-i vei ridica.
\par 5 Sunt unii care fac ospăț cu prietenii, atunci când acasă ochii copiilor se sting de foame.
\par 6 Am ajuns de poveste între oameni; sunt acela pe care-l scuipi în față.
\par 7 Ochii mei s-au întunecat de supărare, mădularele mele s-au subțiat ca umbra.
\par 8 Oamenii cei drepți stau înmărmuriți și cel nevinovat se răscoală împotriva celui nelegiuit.
\par 9 Cel ce este drept se ține însă de calea sa și cine este cu mâinile curate e din ce în ce mai tare.
\par 10 Cât despre voi ceilalți, voi toți dați înapoi și veniți aici, căci nu voi găsi printre voi nici un înțelept.
\par 11 Zilele mele s-au scurs, socotințele mele s-au sfărâmat și la fel dorințele inimii mele.
\par 12 Din noapte ei vor să facă zi și spun că lumina este mai aproape decât întunericul.
\par 13 Mai pot să nădăjduiesc? Împărăția morții este casa mea, culcușul meu l-am întins în inima întunericului.
\par 14 Am zis mormântului: Tu ești tatăl meu; am zis viermilor: voi sunteți mama și surorile mele!
\par 15 Atunci unde mai este nădejdea mea și cine a văzut pe undeva norocul meu?
\par 16 El s-a rostogolit până în fundul iadului și împreună cu mine se va cufunda în țărână".

\chapter{18}

\par 1 Atunci Bildad din Șuah a început să vorbească și a zis:
\par 2 "Când vei ajunge odată la capătul unor astfel de vorbe? Vino-ți în fire și apoi vom vorbi.
\par 3 Pentru ce suntem socotiți ca niște dobitoace? De ce să trecem în ochii tăi drept vite cornute?
\par 4 Nu cumva pentru tine care te sfâșii în mânia ta, o să ajungă pământul să se pustiiască și stâncile să se mute din locul lor?
\par 5 Firește, lumina nelegiuitului se stinge și flacăra focului lui nu mai strălucește.
\par 6 Lumina se întunecă în cortul lui și candela de deasupra lui se isprăvește.
\par 7 Pașii lui, altădată vânjoși, se îngustează și propriul lui sfat acum îl poticnește.
\par 8 El dă cu picioarele în laț și se plimbă în rețeaua de sfori.
\par 9 Capcana l-a prins de călcâi și lațul s-a încolăcit pe el.
\par 10 Cursa care trebuia să-l prindă este ascunsă în pământ și prinzătoarea stă în poteca lui;
\par 11 Spaimele dau peste el din toate părțile și se țin de el pas cu pas.
\par 12 Lângă bunătățile lui el moare de foame și nenorocirea lui stă gata lângă el.
\par 13 Boala mușcă din trupul lui. Primul născut al morții roade mădularele lui.
\par 14 Din cortul unde stătea la adăpost este scos afară și târât înaintea groaznicului împărat.
\par 15 Nimeni din ai lui nu mai sălășluiește în cortul lui, care nu mai este al lui. Pe locuința lui plouă cu pucioasă.
\par 16 Rădăcinile lui se usucă în pământ, iar ramurile lui se veștejesc în aer.
\par 17 Pomenirea lui se șterge de pe pământ și în toată lumea numele i-a pierit.
\par 18 De la lumină i-au dat brânci în întuneric și de pretutindeni e scos afară.
\par 19 Nu lasă nici urmași, nici sămânță în poporul său și nimeni nu mai trăiește după el prin locurile prin care a locuit.
\par 20 Cei din apus s-au mirat foarte de soarta lui și cei din răsărit au simțit fiori în ei.
\par 21 Aceasta rămâne din sălașurile celui nelegiuit și iată locul celui ce n-a cunoscut pe Dumnezeu".

\chapter{19}

\par 1 Atunci Iov a început să vorbească și a zis:
\par 2 "Câtă vreme veți întrista voi sufletul meu și mă veți zdrobi cu cuvântările voastre?
\par 3 Iată a zecea oară de când mă batjocoriți. Nu vă este rușine că vă purtați așa?
\par 4 Chiar dacă ar fi adevărat că am păcătuit, greșeala mea este pe capul meu.
\par 5 Iar dacă voi vă faceți tari și mari împotriva mea și-mi scoateți ochii cu ticăloșia mea,
\par 6 Să știi că Dumnezeu este Cel ce mă urmărește și că El m-a învăluit cu lațul Său.
\par 7 Dacă strig de multa-apăsare, nu primesc nici un răspuns; țip în gura mare, dar nimeni nu-mi face dreptate.
\par 8 El a astupat calea mea, ca să nu mai trec pe ea, și a acoperit cu beznă toate drumurile mele.
\par 9 M-a dezbrăcat de mărirea mea și mi-a smuls cununa de pe cap.
\par 10 M-a dărâmat de jur împrejur și sunt în ceasul morții și nădejdea mea a scos-o din rădăcină ca pe un copac.
\par 11 Aprins-a împotriva mea mânia Sa și m-a luat drept dușmanul Său.
\par 12 Hoardele Sale sosesc grămadă, își fac drum până la mine și pun tabără de jur împrejurul cortului meu.
\par 13 A depărtat pe frații mei de lângă mine și cunoscuții mei își întorc fața când mă văd.
\par 14 Rudele mele au pierit, casnicii mei au uitat de mine.
\par 15 Cei ce locuiau împreună cu mine și slujnicele mele se uită la mine ca la un străin; sunt în ochii lor ca unul venit din altă țară.
\par 16 Chem pe sluga mea și nu-mi răspunde, măcar că o rog cu gura mea.
\par 17 Suflarea mea a ajuns nesuferită pentru femeia mea și am ajuns să miros greu pentru fiii cei născuți din coapsele mele.
\par 18 Până și copiii îmi arată dispreț; când mă scol, vorbesc pe seama mea.
\par 19 Toți sfetnicii mei cei mai de aproape mă urgisesc și aceia pe care îi iubeam s-au întors împotriva mea.
\par 20 Oasele mele ies afară din piele și nu mi-au mai rămas tefere decât gingiile.
\par 21 Milă fie-vă de mine, aveți milă de mine, o, voi, prietenii mei, căci mâna lui Dumnezeu m-a lovit!
\par 22 De ce mă prigoniți cu urgia lui Dumnezeu și nu vă mai săturați de carnea mea?
\par 23 Cit aș vrea ca vorbele mele să fie scrise, cât aș vrea să fie săpate pe aramă.
\par 24 Să fie săpate pe veci, cu un condei de fier și de plumb, într-o stâncă!
\par 25 Dar eu știu că Răscumpărătorul meu este viu ți că El, în ziua cea de pe urmă, va ridica iar din pulbere această piele a mea ce se destramă.
\par 26 Și afară din trupul meu voi vedea pe Dumnezeu.
\par 27 Pe El Îl voi vedea și ochii mei Îl vor privi, nu ai altuia. Și de dorul acesta măruntaiele mele tânjesc în mine.
\par 28 Iar dacă ziceți: Cum îl vom urmări și ce pricină de proces vom găsi noi în el?
\par 29 Temeți-vă de sabie, pentru voi înșivă, când mânia va izbucni împotriva greșelii voastre. Și atunci veți învăța că este o judecată!"

\chapter{20}

\par 1 Și Țofar din Naamah a început să vorbească și a zis:
\par 2 "Cugetul meu mă împinge să vorbesc, din pricina frământării pe care o simt în mine.
\par 3 Am auzit o învățătură care mă scoate din sărite și atunci o pornire vijelioasă, din duhul meu, mă face să răspund.
\par 4 Nu știi tu oare că de mult de tot, din zilele când omul a fost așezat pe pământ,
\par 5 Desfătarea celor fără de lege ține foarte puțin și bucuria fățarnicului nu stă decât o clipă?
\par 6 Chiar dacă statura lui s-ar înălța până la ceruri și cu capul s-ar atinge de nori,
\par 7 El totuși va pieri ca o nălucă, pe vecie, și cei ce îl vedeau vor întreba: Ce s-a făcut?
\par 8 Zboară ca un vis și nu mai dai de el, e măturat ca o vedenie de noapte.
\par 9 Ochiul, care îl privea, nu-l mai vede și locul unde se găsea, nu-l mai zărește.
\par 10 Feciorii lui vor trebui să cerșească mila celor săraci și mâinile lui vor da înapoi ce a luat cu forța.
\par 11 Oasele lui sunt încă pline de vlaga tinereții, dar ea se va culca cu el în pulbere;
\par 12 Dacă răutatea este dulce în gura lui, el o ascunde sub limba lui.
\par 13 Dacă o tine în gură și nu o scuipă, dacă o mestecă în cerul gurii,
\par 14 Totuși hrana lui în măruntaiele lui se strică și se face în intestinele lui venin de năpârcă.
\par 15 Averea, pe care a înghițit-o, acum o varsă; Dumnezeu i-o dă afară din pântece.
\par 16 Venin de șarpe otrăvitor sugea. Limba de năpârcă îl omoară!
\par 17 El nu va mai vedea pâraiele de proaspăt untdelemn, valurile de miere și de smântână.
\par 18 Dă îndărăt ce-a câștigat și nu se mai folosește de câștig și de roadele negustoriei sale nu se mai bucură.
\par 19 Pentru că a asuprit fără milă pe săraci și a furat o casă, în loc să o zidească.
\par 20 El nu va cunoaște pacea lăuntrică și el nu va scăpa nimic din toate câte prețuiește.
\par 21 Nimic nu scapă de lăcomia lui, de aceea înflorirea lui nu va ține deloc.
\par 22 Când bogăția lui va fi la culme, tulburarea îl va apuca deodată și toate loviturile nenorocirii vor cădea în capul lui.
\par 23 Când va fi gata să-și sature pântecele, Dumnezeu va dezlănțui asupră-i urgia mâniei Sale și va ploua cu săgeți peste el.
\par 24 Dacă va scăpa de platoșa de fier, îl va străpunge arcul de aramă.
\par 25 O săgeată îi iese din spate, o alta i s-a înfipt în ficați și spaimele morții îl sfârșesc.
\par 26 Toată neagra pieire amenință comorile pe care le-a adunat; un foc care arde neaprins îl mistuie pe oricine va mai rămâne din cortul lui.
\par 27 Cerurile vor dezveli fărădelegea lui și pământul i se va ridica împotrivă.
\par 28 Năpraznică revărsare de ape va mătura casa lui și apele vor curge în ziua dumnezeieștii mânii.
\par 29 Aceasta este partea hărăzită de Dumnezeu omului nelegiuit, aceasta este moștenirea pe care el o primește de la Domnul".

\chapter{21}

\par 1 Atunci Iov a vorbit încă o dată și a zis:
\par 2 "Ascultați cu luare-aminte cuvântul meu și aici să se oprească mângâierile voastre.
\par 3 Îngăduiți-mi să vorbesc și eu, și după ce voi vorbi, atunci poți să-ți bați joc.
\par 4 Oare plângerea mea se înalță împotriva unui om? Și atunci răbdarea mea cum n-o să fie pe sfârșite?
\par 5 Uitați-vă la mine și mirați-vă foarte și puneți mâna la gură.
\par 6 Căci, când mă gândesc, mă apucă groaza și toată carnea de pe mine tremură.
\par 7 Pentru ce ticăloșii au viață, ajung la adânci bătrânețe și sporesc în putere?
\par 8 Urmașii lor se ridică voinici în fața lor și odraslele lor dăinuiesc sub ochii lor.
\par 9 Casele lor stau nevătămate, fără teamă și varga lui Dumnezeu nu stă deasupra lor.
\par 10 Taurii sunt plini de vlagă și prăsitori, juncanele lor fată și nu leapădă.
\par 11 Copiii lor zburdă ca oile și odraslele lor dănțuiesc împrejur.
\par 12 Ei cântă din tobă și din harfă și se desfată la sunetele flautului.
\par 13 Își isprăvesc zilele în fericire și coboară cu pace în împărăția morții.
\par 14 Și tocmai ei ziceau lui Dumnezeu: "În lături de la noi! Nu vrem deloc să cunoaștem căile Tale!
\par 15 Cine este Cel Atotputernic ca să-I slujim Lui și ce folos vom avea să-I înălțăm rugăciuni?"
\par 16 N-ai zice, oare, că fericirea lor e în mâna lor? Sfatul celor răi nu este totdeauna departe de Dumnezeu?
\par 17 De câte ori se stinge candela nelegiuiților și nenorocirea dă năvală peste ei? De câte ori Dumnezeu nimicește cu mânia Sa pe cei răufăcători,
\par 18 Ca să fie ei ca paiul în bătaia vântului și ca pleava pe care o răsucește vârtejul?
\par 19 Dumnezeu, vei zice, păstrează pentru copiii lui răsplata fărădelegii lui. Dar să-l pedepsească pe el însuși, ca să se învețe.
\par 20 Să-și vadă cu ochii nenorocirea și să se adape din mânia Celui Atotputernic!
\par 21 Fiindcă ce-i mai pasă de casa lui, după moartea lui, când numărul lunilor lui a fost retezat?
\par 22 Dar nu cumva Îi vom da noi învățătură lui Dumnezeu, Lui care stă și judecă pe cei de sus?
\par 23 Unul moare, în plinătatea puterii sale, când este înconjurat de fericire și de pace,
\par 24 Când gălețile îi sunt pline de lapte și oasele pe care le suge, pline cu măduvă.
\par 25 Altul moare, cu sufletul copleșit de amărăciune, fără să fi gustat vreo fericire.
\par 26 Și unul și altul se culcă în țărână și viermii îi cotropesc.
\par 27 Știu prea bine gândurile voastre și socotințele pe care vi le făuriți în privința mea.
\par 28 Voi ziceți în mintea voastră: Unde este casa asupritorului și unde este cortul în care locuiau nelegiuiții?
\par 29 N-ați întrebat oare pe cei ce trec pe drum și n-ați recunoscut dreptatea spuselor lor?
\par 30 Anume cum că în ziua nenorocirii cel rău este cruțat și că în ceasul mâniei el scapă?
\par 31 Cine îl mustră în față pentru purtarea lui și cine-i întoarce cu aceeași măsură faptele pe care le-a făcut?
\par 32 Iar când este dus la locul de odihnă, din stâlpul de la căpătâi el parcă stă de strajă.
\par 33 Bulgării pământului îi sunt ușori; în convoi pe urma lui înaintează toată lumea, și înaintea lui o mulțime nenumărată.
\par 34 Atunci ce sunt deșartele mângâieri pe care mi le dați? Din toate cuvintele voastre nu rămâne decât înșelăciune".

\chapter{22}

\par 1 Elifaz din Teman a răspuns atunci și a zis:
\par 2 "Poate omul să fie de vreun folos lui Dumnezeu? Nu, fiindcă înțeleptul își este de folos lui însuși.
\par 3 Ce are Cel Atotputernic dacă tu ești fără prihană? Și care este câștigul Lui, dacă drumurile tale sunt fără vină?
\par 4 Oare El te pedepsește pentru cucernicia ta și pentru ea intră cu tine în judecată?
\par 5 Nu, dimpotrivă, fiindcă răutatea ta este mare și fărădelegile tale sunt fără hotar!
\par 6 Căci tu fără dreptate luai zăloage de la frații tăi și smulgeai veșmântul de pe oameni și-i lăsai goi.
\par 7 Tu nu dădeai să bea celui însetat și nu dădeai să mănânce celui flămând;
\par 8 Cel cu pumnul tare cotropește pământul și cel cu trecere îl ia, în stăpânire.
\par 9 Goneai de la pragul tău pe văduve cu mâinile goale și brațele celor orfani tu le sfărâmai.
\par 10 Acesta este cuvântul pentru care lațuri te înconjoară și spaimele te-au apucat dintr-o dată.
\par 11 Lumina s-a stins pentru tine și nu mai vezi și o apă revărsată te-a dat la fund.
\par 12 Dumnezeu nu este El oare mai presus de ceruri? Privește în sus spre stele cât de sus sunt ele!
\par 13 Tu ai zis: Ce știe Dumnezeu! Judecă El oare prin umbră?
\par 14 Norii sunt ca o perdea în fața Lui și El nu poate să vadă; El se plimbă numai de jur împrejurul cerurilor.
\par 15 Voiești tu să urmezi pe străvechea cale pe care au bătătorit-o oamenii cei fără de lege?
\par 16 Cei ce au fost măturați înainte de vreme, când un fluviu s-a rostogolit peste temeliile lor,
\par 17 Și ei ziceau lui Dumnezeu: "În lături de la noi! Și ce poate să ne facă Cel Atotputernic?"
\par 18 Dar tocmai El umpluse casele lor de bunătăți, însă sfatul celor răi rămânea departe de Dumnezeu.
\par 19 Cei drepți se uită și se bucură, iar cel nevinovat râde de ei.
\par 20 Iată, avuția lor a nimicit-o și focul a mistuit toată strânsura lor!
\par 21 Împacă-te cu Dumnezeu și cazi la pace. Atunci bine va fi de tine.
\par 22 Primește, te rog, învățătură din gura Lui și pune la inimă cuvintele Lui;
\par 23 Dacă te întorci la Cel puternic și te smerești, dacă depărtezi nedreptatea de cortul tău,
\par 24 Atunci aurul tău îl vei prețui drept țărână și comorile Ofirului drept pietricele,
\par 25 Pentru că Cel Atotputernic va fi pentru tine sloi de aur și grămezi de argint.
\par 26 Atunci tu te vei desfăta întru Cel Atotputernic și ridica-vei fața ta către Dumnezeu.
\par 27 Tu vei chema numele Lui și El te va auzi și tu vei împlini juruințele tale.
\par 28 Când te vei hotărî să faci un lucru, lucrul îl vei izbuti și lumina va străluci pe toate drumurile tale,
\par 29 Fiindcă Dumnezeu smerește pe mândri și mândria, și mântuiește pe acela care-și pleacă ochii în pământ.
\par 30 El izbăvește pe cel nevinovat și tu la fel vei scăpa, când mâinile tale vor fi curate".

\chapter{23}

\par 1 Dar Iov iarăși a vorbit și a zis:
\par 2 "Și de data aceasta plângerea mea este luată tot ca răzvrătire și totuși mâna mea de-abia înăbușe suspinele mele.
\par 3 O, dacă aș ști unde să-L găsesc! Dacă aș putea să ajung la palatul Lui!
\par 4 Atunci aș dezvălui înaintea Lui pricina mea și aș umple gura mea cu învinuiri.
\par 5 Aș ști atunci cuvintele cu care mi-ar răspunde și aș înțelege rostul spuselor Lui.
\par 6 Și-ar dezlănțui El oare toată puterea în cearta Lui cu mine? Nu, El ar sta și m-ar asculta.
\par 7 El ar lua aminte la omul drept care vorbește în fala Lui și astfel aș fi iertat pe vecie de Judecătorul meu.
\par 8 Căci iată, dacă o iau spre răsărit, El nu este acolo; dacă o iau spre apus, nu-L zăresc!
\par 9 L-am căutat spre miazănoapte și n-am dat de El, m-am întors către miazăzi și nici aici nu L-am văzut!
\par 10 Dar El cunoaște și umbletul meu și starea mea pe loc și dacă ar fi să mă treacă prin cuptor de foc, voi ieși din cuptor curat ca aurul.
\par 11 M-am ținut cu pasul meu după pasul Lui, am păzit calea Lui și nu m-am abătut din ea.
\par 12 De la porunca buzelor Sale nu m-am depărtat, la sânul meu am ținut ascunse cuvintele gurii Sale.
\par 13 Dar hotărârea Lui este luată și cine-L va împiedica? Căci ceea ce sufletul Său a poftit, aceea va și face.
\par 14 Fiindcă El aduce la îndeplinire hotărârea Sa și alte foarte multe lucruri la fel, care sunt în gândul Său.
\par 15 Iată pentru ce sunt înspăimântat în fața Lui. Mă gândesc și mi-e teamă de El.
\par 16 Dumnezeu a slăbit inima mea și Cel Atotputernic m-a îngrozit.
\par 17 Și n-am tăcut din pricina întunericului și din pricina nopții care a învăluit fața mea.

\chapter{24}

\par 1 De ce, pentru Cel Atotputernic, vremurile răsplătirilor sunt ascunse și cei ce-L cunosc n-au văzut zilele Sale de judecător?
\par 2 Viclenii mută hotarele țarinilor, fură turma de oi cu cioban cu tot.
\par 3 Duc la ei acasă asinul copiilor orfani și iau zălog boul văduvei.
\par 4 Dau la o parte de pe cale pe cei săraci din țară, iar pe toți nenorociții din țară îi silesc să se ascundă.
\par 5 Aceștia la fel cu asinii sălbatici din pustie ies pe furiș să-și caute de mâncare și, după ce lucrează până seara, tot n-au pâine pentru copii.
\par 6 Ei seceră noaptea pe câmp, ei culeg via nelegiuitului;
\par 7 Petrec noaptea goi, fiindcă n-au cu ce să se învelească, pentru că n-au veșmânt să se apere de frig.
\par 8 Ploaia repede din munți îi udă până la piele și în loc de adăpost strâng în brațe stâncile.
\par 9 Cei dintâi smulg pe orfan de la țâță și iau zălog haina săracului.
\par 10 Și săracii umblă goi, fără îmbrăcăminte și, istoviți de foame, duc în spinare snopii.
\par 11 La teascul bogatului, ei storc untdelemnul, ei calcă jghiaburile cu struguri și tânjesc de sete.
\par 12 În cetate, muribunzii se vaită și sufletul celor răniți cere ajutor; dar Dumnezeu n-aude rugăciunea lor!
\par 13 Mai sunt răzvrătiți împotriva zilei, care nu cunosc cărările ei și nu rămân în potecile ei.
\par 14 Ucigașul se scoală dis-de-dimineață, ucide pe cel sărac și nevoiaș și jefuiește.
\par 15 Ochii celui desfrânat pândesc amurgul zilei și el își zice: Nu mă vede nici țipenie de om, și își pune o mahramă pe față.
\par 16 Tâlharul, acoperit de întuneric, sparge casele și intră în ele, căci el le-a pus semn de cu ziuă,
\par 17 Iar când vine dimineața, parcă ar fi pentru ei umbra morții. Când zorii strălucesc, toate spaimele morții dau peste ei.
\par 18 Nelegiuitul plutește ușor ca pe fața apelor, dar pe pământ partea lui este plină de blestem și fericirea nu va călca niciodată via lui.
\par 19 Precum seceta și arșița sorb apele zăpezilor topite, tot astfel soarbe locuința morților pe păcătoși.
\par 20 Pântecele mamei lor l-au uitat, viermii se desfătează din el, nimeni nu-l mai ține minte și astfel nelegiuirea lor s-a frânt ca un copac.
\par 21 Ei chinuiau pe femeia stearpă și fără de copii, ei s-au purtat aprig cu femeia văduvă.
\par 22 Dar Cel ce, prin puterea Lui, strunește pe cei puternici, se ridică răzbunător și toți aceștia nu se mai țin stăpâni pe viața lor.
\par 23 El îi lasă să se sprijine cu bună încredințare, dar ochii Lui erau asupra căilor lor.
\par 24 Se ridicaseră, dar acum nu mai sunt, s-au așternut ca nalba, când o cosești și ca spicul ierbii s-au veștejit.
\par 25 Dacă ziceți că nu este așa, cine îmi va dovedi că am mințit și cine va spulbera cuvântul meu?"

\chapter{25}

\par 1 Atunci Bildad din Șuah a început să vorbească și a zis:
\par 2 "A Lui este stăpânirea, a Lui este puterea înfricoșătoare! Și El sălășluiește pacea în locurile preaînalte.
\par 3 Cine poate să numere oștile Sale? Și peste cine nu se ridică paza Lui?
\par 4 Cum ar putea un om să fie fără de prihană înaintea lui Dumnezeu, sau cum ar putea să fie curat cel ce se naște din femeie?
\par 5 Iată nici luna nu strălucește destul în ochii Lui și nici stelele nu sunt de tot curate, pentru El!
\par 6 Cu cât mai puțin omul, care nu este decât putreziciune, cu atât mai puțin născutul din om, care nu este decât un vierme!"

\chapter{26}

\par 1 Atunci Iov a răspuns și a zis:
\par 2 "În ce chip ajuți tu pe cel ce este fără de putere și sprijini brațul care a slăbit?
\par 3 Cum știi tu să sfătuiești pe cel lipsit de înțelepciune și ce belșug de știință ai dat pe față?
\par 4 Către cine ai îndreptat tu cuvintele tale și al cui duh grăia prin gura ta?
\par 5 Înaintea lui Dumnezeu, umbrele răposaților tremură sub pământ, iar apele și vietățile din ape se înspăimântă.
\par 6 Împărăția morților este goală înaintea Lui și adâncul este fără acoperiș.
\par 7 El întinde miazănoaptea peste genune; El spânzură pământul pe nimic.
\par 8 El închide apele în norii Săi și norii nu se rup sub greutatea apelor.
\par 9 El acoperă fața lunii pline, desfășurând asupra ei norii Săi.
\par 10 El a tras un cerc pe suprafața apelor, până la hotarul dintre lumină și întuneric.
\par 11 Stâlpii cerului se clatină și se înspăimântă la mustrarea Lui.
\par 12 Cu puterea Lui El a despicat marea și cu înțelepciunea Lui a sfărâmat furia ei.
\par 13 Suflarea Lui înseninează cerurile și mâna Lui străpunge șarpele fugar!
\par 14 Și dacă acestea sunt marginile din afară ale înfăptuirilor Sale, cât de puțin lucru este ceea ce străbate până la noi! Dar tunetul puterii Sale, cine ar putea să-l înțeleagă?"

\chapter{27}

\par 1 Dar Iov a mers mai departe cu pildele lui și a zis:
\par 2 "Viu este Dumnezeu Care a dat la o parte dreptatea mea! Viu este Cel Atotputernic Care a împovărat sufletul meu!
\par 3 Câtă vreme duhul meu va fi întreg în mine și suflarea lui Dumnezeu în pieptul meu,
\par 4 Buzele mele nu vor rosti nici un neadevăr și limba mea nu va grăi nici o minciună!
\par 5 Departe de mine gândul să vă dau dreptate! Până când o fi să-mi dau duhul nu mă voi lepăda de nevinovăția mea.
\par 6 Țin cu tărie la dreptatea mea și nu voi lăsa-o să-mi scape; inima mea nu se rușinează de zilele pe care le-am trăit.
\par 7 Dușmanul meu să aibă partea nelegiuitului și cel ce este împotriva mea să aibă partea celui ce lucrează nedreptatea!
\par 8 Care este nădejdea unui înrăit, când el se roagă și își ridică sufletul către Dumnezeu?
\par 9 Aude oare Dumnezeu strigarea lui, când dă peste el vreo nenorocire?
\par 10 Este oare Cel Atotputernic desfătarea lui? Cheamă el în toată vremea numele lui Dumnezeu?
\par 11 Voiesc să vă învăț căile lui Dumnezeu și ceea ce este în gândul Celui Atotputernic nu vreau să vă ascund.
\par 12 Și dacă voi toți ați dovedit-o (ca și mine), atunci pentru ce vorbiți în zadar?
\par 13 Iată partea pe care Dumnezeu o păstrează celui rău și moștenirea pe care asupritorii vor primi-o de la Cel Atotputernic.
\par 14 Dacă fiii săi sunt numeroși, sunt pentru tăișul sabiei și odraslele lui nu au atâta pâine cât să se sature:
\par 15 Câți mai scapă dintre ai lui vor muri de ciumă și văduvele lor nu-i vor jeli.
\par 16 Dacă adună bani mulți ca nisipul și grămădește veșminte multe ca noroiul,
\par 17 Poate să le grămădească, dar cu ele se va îmbrăca un om fără prihană și de toți banii lui va avea parte unul cu inima curată.
\par 18 Casa pe care și-a zidit-o este casa unei molii și ca o colibă pe care și-o face un pândar.
\par 19 Se culcă bogat, dar nu se mai culcă a doua oară; deschide ochii și nu mai este.
\par 20 Spaimele l-au ajuns ziua în amiaza mare; în puterea nopții, un vârtej l-a smuls.
\par 21 Vântul de la răsărit l-a spulberat și se duce; din locul de unde era îl spulberă.
\par 22 Dumnezeu îl împovărează fără milă și înaintea mâinii care îl pedepsește el caută să fugă.
\par 23 Oamenii bat din mâini la priveliștea aceasta și cu fluierături îl alungă de peste tot.

\chapter{28}

\par 1 Argintul are zăcămintele lui de obârșie și aurul are locul lui de unde-l scoți și-l lămurești.
\par 2 Din pământ scoatem fierul și din stânca topită scoatem arama.
\par 3 Omul a pus hotare întunericului și cercetează până în cele mai depărtate adâncuri, sfredelind piatra ascunsă în umbră și în beznă.
\par 4 Un popor străin a săpat cărări pe sub pământ, uitate de piciorul celor de deasupra și departe de oameni; scormonitorii se spânzură pe funii și se clatină încoace și în colo.
\par 5 Și deasupra este pământul din care iese pâinea, dar pe dedesubt este răvășit ca de foc.
\par 6 Aici pietrele lui sunt de safir, dincoace sunt puzderii de aur,
\par 7 Cărări pe care nu le-a cunoscut pasărea de pradă și pe care ochiul vulturului nu și le-a însemnat.
\par 8 Fiarele sălbatice nu le-au călcat niciodată, niciodată leul nu s-a strecurat pe aici.
\par 9 Dar omul a ajuns cu mâna lui la aceste stânci de cremene și munții i-a răsturnat din temelie.
\par 10 El a săpat șanțuri în stânci și nimic de preț nu scapă privirii lui.
\par 11 El a răscolit izvoarele apelor și tot ce era în adâncime a scos afară la lumină.
\par 12 Dar înțelepciunea de unde izvorăște ea și care este locul de obârșie al priceperii?
\par 13 Pământeanul nu cunoaște calea către ea, căci ea nu se găsește pe meleagurile celor vii.
\par 14 Adâncul a grăit: Ea nu se află în sânul meu! Și marea a spus la fel: Ea nu este la mine!
\par 15 Mintea cea înaltă nu poate fi schimbată cu bulgări de aur și argintul nu-l cântărești ca s-o plătești.
\par 16 Ea nu poate să fie prețuită nici cu aurul Ofirului, nici cu prețioasa cornalină, nici cu pietre de safir!
\par 17 Cu ea alături nu pot să stea nici aurul, nici cristalul și cu un vas din aurul cel mai curat nu se poate schimba ea.
\par 18 Despre mărgean și despre diamant, nici să mai pomenim, iar agonisirea înțelepciunii întrece cu mult pe aceea a mărgăritarelor.
\par 19 Topazele Etiopiei nu stau în cumpănă cu ea și cu aurul cel mai curat nu vei plăti-o niciodată!
\par 20 Și această înțelepciune de unde vine ea și care este sălașul priceperii?
\par 21 Ea a fost ascunsă de ochii oricărei făpturi vii; ea a fost tăinuită și de pasărea cerului.
\par 22 Adâncul și moartea au zis: Noi am auzit vorbindu-se de ea.
\par 23 Dumnezeu îi cunoaște drumul și numai El este Cel ce știe locuința ei.
\par 24 Când El privea până la marginile pământului și îmbrățișa cu ochii tot ce se află sub ceruri,
\par 25 Ca să dea vântului cumpănă și să chibzuiască legea apelor,
\par 26 Când El statornicea ploilor un făgaș și o cale bubuitului tunetului,
\par 27 Atunci El a văzut înțelepciunea ș i a cântărit-o, atunci a pus-o în lumină și i-a măsurat adâncimea.
\par 28 După aceea Dumnezeu a zis omului: Iată, frica de Dumnezeu, aceasta este înțelepciunea, iar în depărtarea de cel rău stă priceperea".

\chapter{29}

\par 1 Apoi Iov a mers mai departe cu pildele sale și a zis:
\par 2 "O, dacă aș fi încă o dată ca în lunile de mai înainte, ca în zilele când Dumnezeu mă ocrotea,
\par 3 Ca atunci când El ținea strălucitoare deasupra capului meu candela Sa și, luminat de ea, eu străbăteam prin întuneric!
\par 4 De ce nu sunt încă o dată ca în zilele toamnei mele, când Dumnezeu ținea parte cortului meu,
\par 5 Când Cel Atotputernic era încă cu mine și împrejurul meu stăteau feciorii mei,
\par 6 Iar picioarele mele se scăldau în lapte și stânca aspră izvora pentru mine pâraie de untdelemn?
\par 7 Atunci când ieșeam la poarta de sus a cetății și așezam în piață scaunul meu,
\par 8 Tineretul, văzându-mă, se ascundea cu sfială, iar cei bătrâni se ridicau în picioare și rămâneau așa.
\par 9 Fruntașii poporului își opreau cuvântările și își puneau mâna la gură.
\par 10 Glasul căpeteniilor scădea și limba lor se lipea de cerul gurii.
\par 11 Căci urechea care mă auzea mă fericea și ochiul care mă vedea îmi dădea mare mărturie.
\par 12 Fiindcă scăpam de pieire pe cel sărman care striga după ajutor și pe orfanul fără sprijin.
\par 13 Binecuvântările celui ce era gata să piară veneau asupră-mi și umpleam de bucurie inima văduvei.
\par 14 Mă îmbrăcam întru dreptate, ca într-un veșmânt și judecata mea cea dreaptă era mantia mea și turbanul meu.
\par 15 Eram ochii celui orb și piciorul celui șchiop;
\par 16 Eram tatăl celor neputincioși și cercetam cu sârguință pricinile care îmi erau necunoscute.
\par 17 Sfărâmam fălcile nelegiuitului și smulgeam prada din dinții lui.
\par 18 Și îmi ziceam: Voi adormi în cuibul meu și ca pasărea Phoenix voi înmulți zilele mele.
\par 19 Rădăcina mea se va răsfira pe lângă apă și roua se va lăsa, noaptea, peste ramurile mele.
\par 20 Slava mea va întineri neîncetat și arcul meu se va înnoi în mâna mea.
\par 21 Oamenii mă ascultau și stăteau fără grai și așteptau să audă sfatul meu.
\par 22 După ce le vorbeam eu, ei nu mai spuneau nimic și cuvântul meu cădea asupra lor picătură cu picătură.
\par 23 Mă așteptau precum aștepți ploaia și căscau gura lor, ca pentru bura de primăvară.
\par 24 Dacă le surâdeam, nu-și credeau ochilor și surâsul meu nu-l lăsau să se piardă.
\par 25 Le arătam care este dreapta cale și stăteam mereu în fruntea lor, stăteam ca un împărat, între ostașii săi și, oriunde-i duceam, ei veneau după mine.

\chapter{30}

\par 1 Iar acum am ajuns de batjocură pentru cei mai tineri decât mine și pe ai căror părinți îi prețuiam prea puțin, ca să-i pun alături cu câinii turmelor mele.
\par 2 Ce aș fi făcut cu puterea brațelor lor, odată ce vlaga lor se dusese toată?
\par 3 Din pricina sărăciei și a foametei înspăimântătoare, ei mânțcau rădăcini din locuri uscate și mama lor era câmpia pustie și jalnică.
\par 4 Ei culegeau ierburi de prin mărăcini și pâinea lor era rădăcina de ienupăr.
\par 5 Erau goniți din mijlocul oamenilor și după ei lumea urla ca după niște hoți.
\par 6 Drept aceea, au ajuns să se aciueze pe marginea șuvoaielor, prin găurile pământului și prin văgăunile stâncilor.
\par 7 Zbiară prin hățișuri, stau grămadă pe sub scaieți.
\par 8 Neam de oameni ticăloși, neam de oameni fără nume, ei erau gunoaiele pe care le arunci din țară!
\par 9 și astăzi, iată că sunt cântecul lor, am ajuns basmul lor.
\par 10 Le e groază de mine, s-au depărtat de mine și pentru obrazul meu n-au făcut economie cu scuipatul lor!
\par 11 Cel ce și-a deznodat ștreangul robiei mă asuprește și tot așa cel ce și-a scos zăbalele din gură.
\par 12 În dreapta mea se ridică martori potrivnici mie, în cursa lor au prins picioarele mele și și-au croit drumuri împotrivă-mi.
\par 13 Au dărâmat poteca mea, cu gând ca să mă piardă, ei se suie încoace și nimeni nu le este stavilă.
\par 14 Ca printr-o spărtură largă, ei dau iureș și în dărâmături se tăvălesc.
\par 15 Mulțimea spaimelor s-a întors asupra mea, slava mea au gonit-o ca vântul și izbăvirea mea a trecut ca un nor.
\par 16 Și acum sufletul meu se topește în mine, zile de amărăciune mă cuprind.
\par 17 Noaptea oasele mele sunt ca sfredelite și nervii mei nu știu de odihnă.
\par 18 Cu o putere năpraznică, Dumnezeu mă tine de haină și mă strânge de gât ca gulerul cămășii.
\par 19 Mi-a dat brânci în noroi și am ajuns să fiu la fel cu praful și cu cenușa.
\par 20 Strig către Tine și nu-mi răspunzi, stau în picioare și Tu nu mă vezi.
\par 21 Tu Te-ai făcut asupritorul meu și cu toată puterea brațului Tău mă prigonești.
\par 22 Tu mă ridici deasupra vântului și mă pui pe el călare și apoi mă nimicești cu iureșul furtunii.
\par 23 Știu foarte bine că Tu mă duci spre moarte și la locul de întâlnire al tuturor muritorilor.
\par 24 Totuși împotriva sărmanului nu ridicam mâna mea, când striga către mine, în nenorocirea lui.
\par 25 N-am plâns oare și eu împreună cu cel care-și ducea viața greu? Sufletul meu n-avea milă de cel sărman?
\par 26 Mă așteptam la fericire și iată că a venit nenorocirea; așteptam lumina și a venit întunericul.
\par 27 Măruntaiele mele au fiert în clocote fără încetare; zile de jale grea mi-au sosit înainte.
\par 28 Am umblat înnegrit la față, dar nu de soare; m-am ridicat în adunare și am strigat.
\par 29 Am ajuns frate cu șacalii, am ajuns tovarăș cu struții.
\par 30 Pielea s-a făcut pe mine neagră și oasele mele sunt arse de friguri.
\par 31 Astfel harfa mea a ajuns instrument tânguirii și flautul meu glasul bocitoarelor.

\chapter{31}

\par 1 Făcusem legământ cu ochii mei și asupra unei fecioare nu-i ridicam.
\par 2 Și care este partea pe care Dumnezeu o trimite din ceruri și ce câștig hărăzește, din înălțime, Cel Atotputernic?
\par 3 Nefericirea nu este ea oare pentru cel nedrept și nenorocirea pentru făptuitorii fărădelegii?
\par 4 Nu vede, oare, Dumnezeu căile mele și nu numără El toți pașii mei?
\par 5 Umblat-am oare întru minciună și picioarele mele au zorit spre înșelăciune?
\par 6 Să mă cântărească în cumpăna dreptății și Dumnezeu să cunoască neprihănirea mea.
\par 7 Dacă pașii mei s-au abătut de la calea cea dreaptă și inima mea a fost târâtă de ochii mei, iar de mâinile mele s-a lipit vreo murdărie,
\par 8 Atunci altul să mănânce ceea ce eu semăn și vlăstarii mei să fie scoși din rădăcină!
\par 9 Dacă inima mea a fost amăgită de vreo femeie și am stat de pândă la ușa aproapelui meu,
\par 10 Atunci nevasta mea să învârtească la râșniță pentru altul și alții să aibă parte de ea.
\par 11 Căci aceasta ar fi o urâciune, o nelegiuire vrednică de pedeapsa judecătorilor,
\par 12 Un foc care mistuie până la iad și care nimicește toată strânsura mea;
\par 13 Dacă aș fi nesocotit dreptul slugii sau al slujnicei mele, în socotelile lor cu mine,
\par 14 Ce mă voi face eu; când Dumnezeu se va ridica și ce răspuns ti voi da, când va lua procesul în cercetare?
\par 15 Cel ce m-a făcut pe mine în pântecele mamei mele nu l-a făcut și pe robul meu? Nu este, oare, El singur Care ne-a alcătuit în pântece?
\par 16 Datu-m-am, oare, în lături, când săracul dorea ceva și lăsat-am să se stingă de plânsete ochii văduvelor?
\par 17 Mâncam, oare, singur bucata mea de pâine și orfanului nu-i dădeam din ea?
\par 18 Dimpotrivă, din tinerețile mele, am crescut pe orfan ca un tată și de la naștere, am călăuzit pe văduvă.
\par 19 Dacă vedeam un nenorocit fără haină și vreun sărac care n-avea cămașă pe el,
\par 20 Nu mă binecuvântau coapsele lui și nu-l încălzea lâna mieilor mei?
\par 21 Dacă am repezit mâna mea împotriva vreunui orfan, fiindcă vedeam că am sprijinitori la masa judecății,
\par 22 Atunci să cadă umărul meu din încheietură și brațul meu să se dezlege de osul celălalt!
\par 23 Dar eu mă temeam de pedeapsa lui Dumnezeu și înaintea măreției Lui nu puteam să stau.
\par 24 Mi-am pus eu încrederea în aur sau am zis aurului lămurit: Tu ești nădejdea mea?
\par 25 Ori eram fericit peste măsură, că aveam atâta avere și că mâna mea agonisise mult?
\par 26 Ori când vedeam soarele în strălucirea lui și luna înaintând cu măreție,
\par 27 A fost inima mea amăgită în taină și am dus eu mâna la gură, ca s-o sărut?
\par 28 Și aceasta ar fi fost o mare fărădelege, fiindcă aș fi tăgăduit pe Dumnezeul cel Preaînalt.
\par 29 M-am bucurat eu de nenorocirea dușmanului meu și am tresăltat când vreo răutate dăduse peste el?
\par 30 Eu n-am îngăduit limbii mele să greșească și să ceară moartea dușmanului, blestemându-l.
\par 31 Oamenii care țineau de casa mea ziceau: "Unde s-ar găsi vreunul care să nu se fi săturat la masa lui?"
\par 32 Străinul nu petrecea noaptea niciodată afară; porțile mele le deschideam călătorului.
\par 33 Acoperit-am eu, ca lumea cealaltă, păcatele mele, ascunzând, în sânul meu, greșeala făptuită,
\par 34 Pentru că, adică, mă temeam de zarva cetății și mă înspăimânta disprețul cetățenilor și atunci rămâneam fără glas și nu mai mă arătam în poartă?
\par 35 O, cine, îmi va da pe cineva care să mă asculte? Iată aici iscălitura mea! Cel Atotputernic să-mi răspundă! Iar învinuirea scrisă de potrivnicii mei,
\par 36 Voi purta-o pe umărul meu, voi înnoda-o în jurul capului meu, ca o cunună.
\par 37 Îi voi da socoteală de toți pașii mei, ca un principe mă voi înfățișa înaintea Lui.
\par 38 Nu cumva ogoarele mele cer răzbunare împotriva mea și brazdele lor sunt prididite de lacrimi?
\par 39 Nu cumva m-am înfruptat din roadele lor și n-am plătit și am făcut pe vechii lor stăpâni să se plângă de mine?
\par 40 Dacă ar fi așa, atunci să crească pe ele pălămidă în loc de grâu și neghină în loc de orz!" Aici cuvintele lui Iov se termină.

\chapter{32}

\par 1 Astfel, acești trei bărbați nu mai răspunseră nimic lui Iov, pentru că el se socotea fără vină.
\par 2 Atunci se aprinse de mânie Elihu, fiul lui Baracheel din Buz, din familia lui Ram. Și mânia lui se aprinse împotriva lui Iov, fiindcă el pretindea că este drept înaintea lui Dumnezeu,
\par 3 Și iarăși se aprinse mânia lui împotriva celor trei prieteni ai lui Iov, fiindcă ei nu găseau nici un răspuns și totuși osândeau pe Iov.
\par 4 Elihu însă așteptase pe când ei vorbeau cu Iov, fiindcă ei erau mai în vârstă decât Elihu.
\par 5 Dar când a văzut el că nu mai este nici un răspuns în gura celor trei oameni, atunci s-a aprins mânia lui.
\par 6 Și așa Elihu, fiul lui Baracheel din Buz, a început a vorbi și a zis: "Eu sunt tânăr și voi sunteți bătrâni, de aceea m-am sfiit și m-am temut să vă dau pe față gândul meu.
\par 7 Mi-am zis: vârsta trebuie să vorbească și mulțimea anilor să ne învețe înțelepciunea.
\par 8 Dar duhul din om și suflarea Celui Atotputernic dau priceperea.
\par 9 Nu cei bătrâni sunt înțelepți și nici moșnegii nu sunt cei ce înțeleg totdeauna dreptatea.
\par 10 Drept aceea am zis: Luați aminte la mine, voi arăta și eu ce știu.
\par 11 Iată că am așteptat cuvintele voastre, am stat cu urechea ațintită la judecățile voastre, pe când voi vă căutați ce aveați de spus.
\par 12 Am stat cu ochii ațintiți asupra voastră și iată că nici unul n-a convins pe Iov, nici unul n-a răsturnat cuvintele lui;
\par 13 De aceea să nu ziceți: Noi am găsit înțelepciunea și Dumnezeu ne dă învățătura, iar nu un om.
\par 14 Astfel, nu voi pune înainte niște cuvinte ca acestea și nu-i voi răspunde cu temeiurile voastre.
\par 15 Ei au fost opăriți, n-au mai răspuns nimic, cuvintele le-au fugit din gură
\par 16 Și eu am așteptat! Dar pentru că ei nu mai vorbesc, fiindcă au stat pe loc și nu mai răspund,
\par 17 Voi zice și eu ceva din partea mea, voi arăta și eu știința mea,
\par 18 Căci sunt plin de cuvinte până în gât și duhul meu lăuntric îmi dă zor.
\par 19 Iată că cugetul meu în mine este ca un vin care n-are pe unde să răsufle, ca un vin care sparge niște burdufuri noi.
\par 20 Voi vorbi deci ca să mă ușurez, voi deschide gura mea și nu-l voi lăsa pe Iov fără răspuns.
\par 21 Nu voi lua partea nimănui și nu voi măguli pe nimeni,
\par 22 Căci nu mă pricep să lingușesc, altfel într-o clipeală m-ar smulge Ziditorul meu.

\chapter{33}

\par 1 Drept aceea, Iov, te rog, ascultă cuvintele mele și ia aminte la toate cuvintele mele.
\par 2 Iată că am deschis gura mea și limba mea grăiește.
\par 3 Inima mea va scoate la iveală cuvinte de învățătură, buzele mele se vor rosti cu limpezime,
\par 4 Duhul lui Dumnezeu este Cel ce m-a făcut și suflarea Celui Atotputernic este dătătoarea vieții mele.
\par 5 Dacă poți, răspunde-mi, apără-ți pricina înaintea mea, fii tare!
\par 6 Înaintea lui Dumnezeu eu sunt la fel cu tine și eu ca și tine am fost frământat din lut,
\par 7 De aceea frica de mine să nu te tulbure, nici mâna mea să nu atârne greu asupra ta.
\par 8 Tu ai spus în auzul meu și eu am auzit rostul vorbelor tale spunând așa:
\par 9 "Eu sunt curat și fără nici o vină, eu sunt fără prihană și n-am nici o greșeală;
\par 10 Dar iată că Dumnezeu caută pricină de ură împotriva mea și mă socotește ca un vrăjmaș al Lui.
\par 11 El pune picioarele mele în butuci și pândește toți pașii mei!"
\par 12 Dar aici îți voi răspunde că tu n-ai dreptate, fiindcă Dumnezeu este mai mare decât omul.
\par 13 De ce grăiești împotriva Lui, fiindcă El nu dă nimănui socoteală de toate câte face?
\par 14 Vezi că Dumnezeu vorbește când într-un fel, când într-alt fel, dar omul nu ia aminte.
\par 15 Și anume, El vorbește în vis, în vedeniile nopții, atunci când somnul se lasă peste oameni și când ei dorm în așternutul lor.
\par 16 Atunci El dă înștiințări oamenilor și-i cutremură cu arătările Sale.
\par 17 Ca să întoarcă pe om de la cele rele și să-l ferească de mândrie
\par 18 Ca să-i ferească sufletul de prăpastie și viața lui de calea mormântului;
\par 19 De aceea, prin durere, omul este mustrat în patul lui și oasele lui sunt zguduite de un cutremur neîntrerupt.
\par 20 Pofta lui este dezgustată de mâncare și inima lui nu mai poftește nici cele mai bune bucate.
\par 21 Carnea după el se prăpădește și piere și oasele lui, până acum nevăzute, îi ies prin piele.
\par 22 Sufletul lui vine încet, încet spre prăpastie și viața lui spre împărăția morților.
\par 23 Dacă atunci se află un înger lângă el, un mijlocitor între vii, care să-i arate omului calea datoriei,
\par 24 Dumnezeu Se milostivește de el și zice îngerului: "Izbăvește-l ca să nu cadă în prăpastie; am găsit pentru sufletul lui prețul de răscumpărare!"
\par 25 Atunci trupul lui înflorește ca în tinerețe și el vine înapoi la zilele de la începutul vieții sale.
\par 26 El se roagă lui Dumnezeu și Dumnezeu îi arată bunătatea Sa și-i îngăduie să vadă fața Sa cu mare bucurie și astfel îi dă omului iertarea Sa.
\par 27 Atunci omul privește peste semenii săi și zice: "Păcătuisem și călcasem dreptatea, dar n-am fost pedepsit după faptele mele.
\par 28 Căci El a izbăvit sufletul meu ca să nu treacă prin strâmtorile morții și ochii mei văd încă lumina".
\par 29 Iată toate acestea le face Dumnezeu de două ori, de trei ori cu omul,
\par 30 Ca să-i scoată sufletul din pieire și ca să-l lumineze cu lumina celor vii.
\par 31 Ia aminte Iov, ascultă-mă pe mine, taci și eu voi vorbi!
\par 32 Dacă ai ceva de spus dă-mi răspuns, vorbește, căci dorința mea este să-ți dau dreptate.
\par 33 Iar dacă nu, ascultă la mine: ține-ți gura și te voi învăța care este înțelepciunea".

\chapter{34}

\par 1 Elihu a vorbit mai departe și a zis:
\par 2 "Ascultați, înțelepților, cuvintele mele și voi, învățaților, ațintiți-vă urechile,
\par 3 Fiindcă urechea deosebește cuvintele, precum cerul gurii gustă rnâncarea.
\par 4 Să cercetăm între noi ce este drept, să știm între noi ceea ce este bine,
\par 5 Fiindcă Iov a zis: "Eu sunt drept, dar Dumnezeu nu-mi dă dreptate!
\par 6 Deși nevinovat, trec drept mincinos; rana mea este nevindecată, deși eu nu am nici o greșeală".
\par 7 Cine mai este ca Iov, care să bea batjocura, cum ar bea apa?
\par 8 Care să se însoțească cu cei care fac nedreptate și să meargă în pas cu făcătorii de rele?
\par 9 Căci Iov a zis: "Omul n-are nici un folos, dacă se străduiește să fie plăcut lui Dumnezeu".
\par 10 Dar voi oameni de inimă, ascultați-mă! Departe este de Dumnezeu răutatea, departe este de El nedreptatea!
\par 11 Căci Dumnezeu întoarce omului după faptele lui și se poartă cu fiecare după purtarea lui.
\par 12 Cu adevărat, Dumnezeu nu făptuiește răul și Cel Atotputernic nu strâmbă dreptatea.
\par 13 Cine i-a încredințat cârmuirea pământului și cine i-a dat în grijă această lume întreagă?
\par 14 Dacă Dumnezeu n-ar cugeta decât la Sine Însuși și dacă ar lua înapoi la Sine duhul Său și suflarea Sa,
\par 15 Toate făpturile ar pieri deodată și omul s-ar întoarce în țărână.
\par 16 Dacă ai minte, ascultă aceasta, pleacă urechea la cuvintele mele.
\par 17 Unul care prigonește dreptatea ar putea oare să domnească? Și vei osândi tu pe Cel mare și drept?
\par 18 El, Care strigă împăraților: Netrebnicilor! Și celor mai mari de pe pământ: Nelegiuiților!
\par 19 El nu caută la fala celor mari și nu face deosebire între bogat și sărac, pentru că toți sunt lucrarea mâinilor Sale.
\par 20 Într-o clipită ei mor și se duc; în miez de noapte, un popor se zbuciumă și fără greutate prăbușește pe tiran.
\par 21 Pentru că ochii Domnului supraveghează cărările omului și vede toți pașii lui.
\par 22 Pentru El nu este nici întuneric, nici umbră, unde să se poată ascunde cei ce lucrează nelegiuirea.
\par 23 Dumnezeu n-are nevoie să privească multă vreme pe cineva, ca să-l tragă înaintea judecății Sale.
\par 24 El zdrobește pe puternici, fără lungă cercetare și pune pe alții în locul lor.
\par 25 De vreme ce El cunoaște faptele lor, El îi răstoarnă în fapt de noapte și-i zdrobește.
\par 26 Ca pe niște nelegiuiți ce sunt, El îi lovește de fală cu foarte mulți privitori,
\par 27 Fiindcă s-au dat la o parte din preajma Sa și n-au voit să priceapă cărările Sale
\par 28 Și au făcut să urce până la Domnul strigătul celui sărac și să răsune în urechile Sale plânsul celor nenorociți.
\par 29 Dacă Domnul se odihnește, cine poate să-L smulgă din odihna Lui și dacă Își acoperă fața, cine poate să-L mai zărească? Dar El stă și supraveghează și pe popoare și pe oameni,
\par 30 Ca unul Care nu voiește stăpânirea celor nelegiuiți, nici poticnirea popoarelor.
\par 31 Dacă un fățarnic zice lui Dumnezeu: "Am fost târât la păcat și nu voi mai face ce este rău,
\par 32 Ceea ce nu știu, Tu învață-mă; dacă am săvârșit vreo nedreptate nu voi porni iar de la capăt!"
\par 33 Crezi tu, după părerea ta, că Dumnezeu îi va face în schimb tot așa? Fiindcă ai fost disprețuitor, fiindcă te faci tu judecător în locul meu, spune-mi atunci ce știi tu?
\par 34 Oamenii în toată firea vor zice și tot așa orice om cuminte care mă ascultă:
\par 35 Iov nu vorbește după dreapta învățătură și cuvintele lui nu sunt după sfânta dreptate.
\par 36 Însă Iov trebuie cercetat până la capăt cu privire la acele răspunsuri vrednice de niște nelegiuiți.
\par 37 El a sporit păcatul său; aici între noi el pune la îndoială greșeala lui și îngrămădește vorbele sale împotriva lui Dumnezeu".

\chapter{35}

\par 1 Elihu a vorbit mai departe și a zis:
\par 2 "Crezi tu că ai dreptate și socotești că te-ai limpezit înaintea lui Dumnezeu,
\par 3 Când zici: "Ce folosesc, ce câștig am eu, că nu păcătuiesc?"
\par 4 Iată ce-ți voi răspunde și ție și prietenilor tăi:
\par 5 Privește cerurile și îndreaptă intr-acolo ochii; uită-te la nori, cât sunt ei de sus, față de tine!
\par 6 Dacă păcătuiești, ce rău îi faci lui Dumnezeu și dacă păcatele tale sunt numeroase, ce-I strică Lui?
\par 7 Dacă ești drept, ce dar Îi faci sau ce primește El din mâna ta?
\par 8 Răutatea ta poate să strice unui om ca și tine, dreptatea ta să folosească celui ce este ca și tine născut din om.
\par 9 Ei strigă atunci când împilarea a trecut orice margini, ei răcnesc în mâinile celor puternici.
\par 10 Dar ei nu întreabă: Unde este Dumnezeu Cel ce ne-a făcut, El Care dăruiește nopții cântări de veselie?
\par 11 El Care ne dă mai multă înțelepciune decât dobitoacelor pământului și mai multă pricepere decât păsărilor cerului?
\par 12 Să tot strige ei atunci, căci Dumnezeu nu răspunde, din pricina trufașei împilări a celor răi.
\par 13 Zadarnică le este truda; Dumnezeu nu aude și Cel Atotputernic nu ia aminte.
\par 14 Cu atât mai puțin, când tu zici că nu știi de unde să-L iei, că tu ești cu El în judecată și că-L tot aștepți să vină.
\par 15 Ba, încă atunci când tu spui că mânia Lui nu pedepsește și că El nu prea știe limpede ce este aceea nelegiuire!
\par 16 Da, Iov își deschide gura zadarnic și, neștiind ce spune, înmulțește cuvintele fără rost.

\chapter{36}

\par 1 Elihu a mers mai departe și a grăit:
\par 2 "Așteaptă o clipă și vei învăța și altele, căci sunt încă temeiuri și cuvinte de partea lui Dumnezeu.
\par 3 Voi porni cu știința mea de departe și voi dovedi dreptatea Ziditorului meu.
\par 4 Căci cu adevărat ceea ce-ți spun eu nu este minciună și cal ce stă lângă tine este unul desăvârșit în cunoștință.
\par 5 Firește, Dumnezeu este prea puternic, dar nu leapădă pe nimeni; El este prea puternic prin înălțimea înțelepciunii Sale.
\par 6 El nu lasă pe nelegiuit să propășească și celor nenorociți le face dreptate.
\par 7 El nu despoaie pe cei drepți de dreptatea lor, iar cu împărații la fel: îi pune în jețuri împărătești și-i așază să domnească de-a pururi. Dar ei se umflă de trufie.
\par 8 Și atunci iată-i ferecați cu lanțuri și iată-i prinși cu funiile mâhnirii.
\par 9 După aceea, Dumnezeu le dezvăluie fapta pe care au făcut-o și nelegiuirea în care au căzut, anume că s-au trufit.
\par 10 Dar El le face această destăinuire ca să ia aminte și le dă poruncă să se întoarcă de la răutatea lor;
\par 11 Dacă dau ascultare și vin la supunere, ei își isprăvesc zilele lor în fericire și anii lor în desfătări;
\par 12 Iar dacă sunt neascultători, atunci trec prin strâmtorile morții și se sting nepricepuți și orbi.
\par 13 Nelegiuiții se mânie; ei nu se roagă lui Dumnezeu, când sunt puși în lanțuri.
\par 14 Unii ca aceștia se sting de tineri și viața lor se veștejește în floare.
\par 15 Dar pe cel nenorocit Dumnezeu îl scapă prin nenorocirea lui și prin suferință Dumnezeu îi dă învățătură.
\par 16 Tot așa și pe tine te va scoate din strânsoarea durerii, ca să te pună la loc larg, unde nu mai este nici o stinghereală și unde masa ta va fi încărcată cu mâncări grase și alese.
\par 17 Dacă tu ai fost pedepsit cu strășnicie, ca un nelegiuit, tu scoate din pedeapsă puterea dreptății;
\par 18 Certarea Lui să nu te împingă la mânie împotriva Lui și mulțimea bătăii să nu te scoată din calea cugetului drept.
\par 19 Era oare să pună Dumnezeu vreun preț pe bogățiile tale? Nu! Nici pe aur, nici pe toate mijloacele puterii pământești.
\par 20 Nu pofti noaptea (depărtării de Dumnezeu), căci în ea, popoare întregi au fost smulse din locul lor.
\par 21 Ia seama, nu te duce la nedreptate, căci ea este adevărata cauză a suferinței.
\par 22 Da, Dumnezeu este nespus de mare prin puterea Lui! Cine poate să învețe ca El?
\par 23 Cine I-a dat învățătură cum să se poarte? Și cine poate să-I spună: "Aceasta ai făcut-o rău?"
\par 24 Adu-ți aminte și preamărește opera Lui, pe care o cântă, în laudele lor, oamenii;
\par 25 Orice om o privește, măcar că o îmbrățișează cu ochiul, numai de departe.
\par 26 Cât este de mare Dumnezeu! Dar noi nu putem să-L înțelegem și numărul anilor Săi nu se poate socoti.
\par 27 El atrage picăturile de apă, El le preface în aburi și dă ploaia.
\par 28 Iar norii o trec prin sita lor și o varsă picături peste mulțimile omenești.
\par 29 Cine poate să priceapă cum se desfășoară norii și cum bubuie tunetul în cortul lui?
\par 30 Iată că El a rostogolit aburii Săi și a acoperit adâncimile mării.
\par 31 Prin el Domnul hrănește popoarele și le dă belșug de mâncare.
\par 32 El ridică fulgerul, cu amândouă mâinile și-l trimite să lovească la țintă.
\par 33 El dă din vreme de veste ciobanului și oilor, care simt din aer apropierea vijeliei.

\chapter{37}

\par 1 Și din pricina aceasta inima mea se zbuciumă și se zbate din locul ei.
\par 2 Ascultați bubuitul glasului Său Și tunetul care iese din gura Sa.
\par 3 Peste toată întinderea cerului El azvârle fulgerul Său și fulgerul Său ajunge până la marginile pământului.
\par 4 În urma fulgerului, vine un muget prelung. El tună cu glasul Lui zguduitor, El nu mai împiedică fulgerele cât timp glasul Lui răsună.
\par 5 Dumnezeu cu tunetul Său săvârșește minuni, El face lucruri mari pe care noi nu putem să le pricepem.
\par 6 El poruncește zăpezii: "Cazi pe pământ ", și ploilor îmbelșugate: "Stăruiți cu putere!"
\par 7 Pe fiecare om El pune a Sa pecetie, pentru ca toți oamenii să recunoască puterea Lui.
\par 8 Fiarele sălbatice se dau înapoi în culcușurile lor și rămân ascunse în vizuinile lor.
\par 9 Vijelia vine de la miazăzi și frigul vine de la miazănoapte.
\par 10 La suflarea lui Dumnezeu se încheagă ghiața și întinderea apelor se face sloi.
\par 11 El umple norii cu apă și din întunecimea furtunii sloboade fulgerele.
\par 12 Iar norii, învârtindu-se în cercuri, aleargă după planurile Sale, astfel că îndeplinesc tot ce le poruncește, în lungul și în latul lumii Sale pământești.
\par 13 Și Domnul îi trimite: aici ca o bătaie pentru pământ, dincolo ca o milostivire a voinței Sale.
\par 14 Iov, ia aminte la aceste lucruri, stai locului și te uită la minunile lui Dumnezeu!
\par 15 Înțelegi tu cum cârmuiește Dumnezeu norii Săi și în ce fel poate norul să sloboadă fulgerul pe pământ?
\par 16 Înțelegi tu plutirea norilor, minuni ale Aceluia a Cărui știință este desăvârșită?
\par 17 Tu, care te aprinzi în veșmintele tale, când pământul se odihnește sub vântul arzător din miazăzi,
\par 18 Poți să întinzi la fel cu El boltitura cerului, ca o oglindă turnată din metal?
\par 19 Spune-mi și mie ce vom putea să grăim cu El? Ce vorbă vom începe noi cu El, astfel întunecați la minte precum suntem?
\par 20 Acum, când eu vorbesc, cine-I dă de veste ce zic eu? Când a vorbit cineva ceva, El o știe fiindcă I-a Spus altul?
\par 21 Oamenii nu pot să privească prealuminosul soare, care strălucește pe cer, acum după ce vântul a împrăștiat norii.
\par 22 Acum lumină biruitoare se revarsă din norii de la miazănoapte și măreția Domnului robește și cutremură inima.
\par 23 Pe Cel Atotputernic nu putem să-L ajungem cu priceperea noastră. El este atotînalt în putere și bogat în judecată și nu calcă niciodată dreptatea în picioare.
\par 24 Pentru aceea oamenii se tem de El și I se închină; El însă nu-și pogoară privirile asupra nici unuia dintre cei ce se cred pe sine înțelepți".

\chapter{38}

\par 1 Atunci Dumnezeu i-a răspuns lui Iov, din sânul vijeliei, și i-a zis:
\par 2 "Cine este cel ce pune pronia sub obroc, prin cuvinte fără înțelepciune?
\par 3 Încinge-ți deci coapsele ca un viteaz și Eu te voi întreba și tu Îmi vei da lămuriri!
\par 4 Unde erai tu, când am întemeiat pământul? Spune-Mi, dacă știi să spui.
\par 5 Știi tu cine a hotărât măsurile pământului sau cine a întins deasupra lui lanțul de măsurat?
\par 6 În ce au fost întărite temeliile lui sau cine a pus piatra lui cea din capul unghiului,
\par 7 Atunci când stelele dimineții cântau laolaltă și toți îngerii lui Dumnezeu Mă sărbătoreau?
\par 8 Cine a închis marea cu porți, când ea ieșea năvalnică, din sânul firii,
\par 9 Și când i-am dat ca veșmânt negura și norii drept scutece,
\par 10 Apoi i-am hotărnicit hotarul Meu și i-am pus porți și zăvoare
\par 11 Și am zis: Până aici vei veni și mai departe nu te vei întinde, aici se va sfărâma trufia valurilor tale?
\par 12 Ai poruncit tu dimineții, vreodată în viața ta, și i-ai arătat aurorei care este locul ei,
\par 13 Ca să apuce pământul de colțuri și să scuture pe nelegiuiți de pe pământ?
\par 14 În revărsatul zorilor, pământul se face roșu ca roșiile peceți și ia culori ca de veșmânt.
\par 15 Cei răi rămân fără noaptea (prielnică lor) și brațul ridicat este frânt.
\par 16 Ai fost tu până la izvoarele mării sau te-ai plimbat pe fundul prăpastiei?
\par 17 Ți s-au arătat oare porțile morții și porțile umbrei le-ai văzut?
\par 18 Ai cugetat oare la întinderea pământului? Spune, știi toate acestea?
\par 19 Care drum duce la palatul luminii și care este locul întunericului,
\par 20 Ca să știi să-l călăuzești în cuprinsul lui și să poți să nimerești potecile care duc la el acasă?
\par 21 Tu știi bine, căci atunci erai născut și numărul zilelor tale e foarte mare.
\par 22 Ai ajuns tu la cămările zăpezii? Ai văzut tu cămările grindinei,
\par 23 Pe care le țin deoparte pentru vremuri de strâmtorare, pentru zilele de bătălie și de război?
\par 24 Unde se risipesc aburii și se răspândește pe pământ vântul de la răsărit?
\par 25 Cine a săpat albie puhoaielor cerului și cine a croit drum bubuitului tunetului,
\par 26 Ca să plouă pe un pământ nelocuit și pe o pustietate unde nu se află ființă omenească
\par 27 Și să adape ținuturile sterpe și uscate și să scoată pajiște de iarbă din întinderea pleșuvă?
\par 28 Are ploaia un tată? Cine a zămislit stropii de rouă?
\par 29 Din sânul cui a ieșit gheața? Și cine este cel ce naște promoroaca din cer?
\par 30 Apele se încheagă și se întăresc ca piatra și fala mării se face sloi!
\par 31 Poți tu să legi cataramele Pleiadelor sau să deznozi lanțurile Orionului?
\par 32 Poți tu să scoți la vreme cununa Zodiacului și vei fi tu cârmaci Carului Mare și stelelor lui?
\par 33 Cunoști tu legile cerului și poți tu să faci să fie pe pământ ceea ce este scris în ele?
\par 34 Poți tu să ridici până la nori glasul tău ca să se sloboadă peste tine potopul ploilor?
\par 35 Ești tu în stare să azvârli fulgerele și ele să plece și să-ți spună: Iată-ne?
\par 36 Cine a pus atâta înțelepciune în pasărea ibis sau cine i-a dat pricepere cocoșului?
\par 37 Cine poate să țină cu destoinicie socoteala norilor și să verse pe pământ burdufurile cerului,
\par 38 Ca să se adune pulberea și să se întărească, iar bulgării de pământ să se lipească laolaltă?
\par 39 Tu ești cel ce aduci pradă leoaicei și potolești foamea puilor de leu,
\par 40 Când s-au ascuns în vizuini sau stau și pândesc ascunși în hățișuri?
\par 41 Cine are grijă de mâncarea corbului, când puii lui croncănesc la Dumnezeu, de foame, și zboară încoace și încolo după hrană?

\chapter{39}

\par 1 Știi tu când nasc caprele sălbatice? Ai băgat de seamă care este vremea cerboaicelor?
\par 2 Numeri tu lunile sarcinii lor și știi tu când le vine ceasul să nască?
\par 3 Ele îngenunchiază, fată puii și scapă de durerile lor,
\par 4 Iar puii lor prind putere, se fac mari pe câmp, pornesc și nu se mai întorc spre mamele lor.
\par 5 Cine a lăsat slobod asinul sălbatic și l-a dezlegat de la iesle?
\par 6 I-am dat pustiul ca să-l locuiască și pământul sărat i l-am hărăzit ocol;
\par 7 El își bate joc de zarva orașelor; el nu aude strigătele nici unui stăpân;
\par 8 El străbate munții, locul său de pășune, și umblă după orișice verdeață.
\par 9 Va voi bivolul sălbatic să se bage la tine slugă și să petreacă noaptea lingă ieslele tale?
\par 10 Poți tu să-l legi cu funia de gât și să tragă grapa după tine, peste arătură?
\par 11 Poți să te încrezi în el, fiindcă este atât de tare, și să-i lași în grijă munca ta?
\par 12 Te bizui tu pe el, că mai vine înapoi să-ți aducă roadele la aria ta?
\par 13 Aripile struțului sunt negrăit de agere; struțul are pene preafrumoase ți mândru penaj.
\par 14 Când își lasă ouăle pe pământ și le lasă să se clocească în nisipul fierbinte,
\par 15 El uită că oarecine poate să le calce cu piciorul și că vreo fiară sălbatică poate să le strivească.
\par 16 Struțul e hain cu puii săi, ca și cum n-ar fi ai lui, și nu-i pasă deloc de truda sa zadarnică.
\par 17 Vezi că Dumnezeu nu l-a înzestrat cu pricepere și pătrundere.
\par 18 Când se scoală însă și pornește, face de ocară și pe cal și pe călăreț.
\par 19 Tu ești cel ce dai putere calului? Tu i-ai împodobit gâtul cu falnica lui coamă?
\par 20 Tu l-ai învățat să sară ușor, ca o lăcustă? Nechezatul lui viteaz insuflă spaimă!
\par 21 El bate pământul cu copita și mândru de puterea lui pornește spre taberele înarmate;
\par 22 El își bate joc de primejdie și n-are nici o teamă și nu se dă înapoi dinaintea sabiei.
\par 23 La oblânc îi sună tolba cu săgeți; fulgere aruncă sulița și lancea.
\par 24 De aprindere, de nerăbdare, el mănâncă, gonind, pământul și, când a sunat trâmbița, nu mai are astâmpăr.
\par 25 La chemarea trâmbiței, pare că zice: Haide! Și de departe soarbe cu nările bătălia, tunetul poruncilor căpeteniilor și strigătele războinice.
\par 26 Oare, prin deșteptăciunea ta s-a îmbrăcat în pene șoimul și își întinde aripile ca niște seceri, spre miazăzi?
\par 27 Nu cumva vulturul se ridică în înălțime din porunca ta și își așază cuibul pe vârfuri neajunse?
\par 28 El își face sălașul în stânci și acolo petrece noaptea - pe un vârf de stâncă și pe vreo înălțime prăpăstioasă.
\par 29 Acolo el stă și își pândește prada; ochii săi străpung depărtările,
\par 30 Puii săi beau sângele prăzii și unde sunt hoiturile, acolo se adună vulturii".

\chapter{40}

\par 1 Și Domnul a vorbit mai departe cu Iov și i-a zis:
\par 2 "Cel ce s-a apucat la ceartă cu Cel Atotputernic se va da oare bătut? Cel ce judecă pe Dumnezeu va răspunde ceva?"
\par 3 Și Iov a răspuns Domnului zicând:
\par 4 "Dacă am fost ușuratic, ce răspuns să-ți mai dau? Voi pune mâna mea pe gura mea.
\par 5 Am vorbit o dată, dar încă o dată nu voi mai vorbi; de două ori și nu voi lua-o iar de la început".
\par 6 Atunci Domnul a vorbit cu Iov, din mijlocul furtunii și a zis:
\par 7 "Încinge-ți coapsele ca un viteaz și te voi întreba și Îmi vei da lămuriri.
\par 8 Poți tu cu adevărat să găsești cusur judecății Mele? Și Mă vei osândi pe Mine, ca să-ți faci dreptate?
\par 9 Este brațul tău ca brațul lui Dumnezeu? Și glasul tău este, oare, tunet, precum este glasul Lui?
\par 10 Atunci împodobește-te cu măreție și cu semeție, îmbracă-te cu strălucire și cu cinste!
\par 11 Revarsă puhoaiele mâniei tale și doboară cu o privire pe cel trufaș!
\par 12 Vezi de toți semeții și smerește-i și calcă în picioare, fără zăbavă, pe toți cei răi!
\par 13 Ascunde-i pe toți grămadă, în pământ, și îi înmormântează.
\par 14 Și atunci Eu Însumi te voi preamări, pentru toate câte ai izbândit cu dreapta ta.
\par 15 Ia privește acum înaintea ta, hipopotamul; și el ca și tine este făptura Mea; el paște iarbă ca boul.
\par 16 Vezi ce putere are în coapsele lui și ce tărie are în mușchii de pe pântece.
\par 17 Coada lui e dârză ca lemnul cedrului și vinele de pe pulpele lui stau ca niște noduri.
\par 18 Oasele lui sunt ca niște țevi de aramă și mădularele ca niște drugi de fier.
\par 19 El este fruntea făpturilor lui Dumnezeu și făcut să fie cel mai mare peste celelalte dobitoace.
\par 20 Munții îi dau hrană și toate fiarele sălbatice sunt îngrozite când îl văd.
\par 21 El se culcă sub florile de lotus, în ocolul trestiilor și al bălții.
\par 22 Frunzele de lotus îi fac umbră și sălciile bălții îl împrejmuiesc.
\par 23 Dacă fluviul vine mare, fără de veste, el nu se sinchisește; el stă liniștit pe loc, chiar când ar fi ca Iordanul să-i urce năvalnic până la gură.
\par 24 Cine poate să-l privească? Cine poate să-i străpungă nasul cu un laț?

\chapter{41}

\par 1 Poți tu să prinzi leviatanul cu undița, ori să-i legi limba cu o sfoară?
\par 2 Vei putea tu să-i vâri în nas o trestie sau să-i găurești falca cu cârligul?
\par 3 Îți va face el multe rugăminți și îți va spune el lucruri drăgălașe?
\par 4 Ori va face cu tine legământ și-l vei lua la tine rob pe toată viața?
\par 5 Te vei juca tu cu el, cum te joci cu o pasăre, sau îl vei lega tu ca să-ți înveselești fetele?
\par 6 Pescarii întovărășiți vor putea să-l scoată de vânzare și negustorii să-l vândă cu bucata?
\par 7 Vei putea tu să-i găurești pielea cu săgeți și capul cu cârligul pescăresc?
\par 8 Ridică-ți numai mâna împotriva lui și vei pomeni de o asemenea luptă și nu o vei mai începe niciodată!
\par 9 Iată, este o deșertăciune să mai nădăjduiești în izbândă; numai înfățișarea lui și te dă la pământ.
\par 10 Cine este atât de nechibzuit încât să-l întărâte? Cine va îndrăzni să dea piept cu Mine?
\par 11 Cine M-a îndatorat cu ceva, ca să fiu acum dator să-i dau înapoi? Tot ce se află sub ceruri este al Meu.
\par 12 Cât despre leviatan, voi vorbi despre mădularele lui și despre tăria lui și despre frumoasa lui întocmire.
\par 13 Cine a ridicat pulpana din față a veșmântului lui și cine poate pătrunde în căptușeala armurei lui?
\par 14 Cine a deschis vreodată porțile gurii lui? Zimții lui sunt îngrozitori!
\par 15 Spinarea lui este ca un șirag de scuturi, pe care le-ai fi întărit și pecetluit puternic.
\par 16 Ele sunt strânse unul într-altul atât de tare, că nici vântul nu pătrunde printre ele.
\par 17 Fiecare e lipit de cel de lângă el și se țin așa și nu se mai despart.
\par 18 Din strănutul lui scapără lumină și ochii lui sunt roșii ca pleoapele zorilor.
\par 19 Din gura lui ies parcă niște torțe aprinse și izbucnesc valuri de scântei.
\par 20 Din nările lui iese fum, ca dintr-o căldare pusă la foc și care fierbe.
\par 21 Răsuflarea lui este de cărbuni aprinși și din gura lui țâșnesc flăcări.
\par 22 Puterea lui e adunată în grumazul lui și înaintea lui țâșnește groaza.
\par 23 Carnea lui e îndesată; oricât ai apăsa în ea nu se lasă.
\par 24 Inima lui este tare ca piatra, tare ca piatra râșniței, cea de dedesubt.
\par 25 De măreția lui se tem și valurile; valurile mării se dau înapoi din fața lui.
\par 26 Să-l atingi cu sabia nu folosești nimic; nici cu lancea, nici cu săgeata, nici cu toporul.
\par 27 Fierul pentru pielea lui este ca paiul, iar arama ca lemnul putred.
\par 28 Săgeata nu-l pune pe fugă și pietrele din praștie cad pe el ca niște pleavă.
\par 29 O săgeată pentru el este un pai în vânt și își bate joc de vâjâitul unei lănci ce zboară.
\par 30 Sub pântecele lui sunt niște solzi ascuțiți; când dă prin noroi, pare că dă cu grapa.
\par 31 Când se afundă, apa fierbe ca într-o căldare; el preface marea într-un cazan de fiert mirodenii.
\par 32 El lasă în urmă o dâră luminoasă și adâncul pare un cap cu plete albe.
\par 33 Pe pământ el nu-și află perechea și e făcut să nu cunoască frica.
\par 34 El se uită de sus la toți câți sunt puternici și este împărat peste toate fiarele sălbatice".

\chapter{42}

\par 1 Și Iov a răspuns Domnului zicând:
\par 2 "Știu că poți să faci orice și că nu este nici un gând care să nu ajungă pentru Tine faptă.
\par 3 Cine cutează, ai zis Tu, să bârfească planurile Mele, din lipsă de înțelepciune? Cu adevărat, am vorbit fără să înțeleg despre lucruri prea minunate pentru mine și nu știam.
\par 4 Ascultă - ai spus Tu iar - și Eu voi vorbi, te voi întreba și tu Îmi vei da lămuriri.
\par 5 Din spusele unora și altora auzisem despre Tine, dar acum ochiul meu Te-a văzut.
\par 6 Pentru aceea, mă urgisesc eu pe mine însumi și mă pocăiesc în praf și în cenușă".
\par 7 Iar după ce Domnul a rostit aceste cuvinte către Iov, a grăit către Elifaz din Teman: "Mânia Mea arde împotriva ta și împotriva celor doi prieteni ai tăi, pentru că n-ați vorbit de Mine așa de drept, precum a vorbit robul Meu Iov.
\par 8 Acum deci luați șapte viței și șapte berbeci și duceți-vă la robul Meu Iov și aduceți-le, pentru voi, ardere de tot; iar robul Meu Iov să se roage pentru voi; din dragoste pentru el, voi fi îngăduitor, ca să nu Mă port cu voi după nebunia voastră, întrucât n-ați vorbit despre Mine așa de drept cum a vorbit robul Meu Iov".
\par 9 Astfel Elifaz din Teman, Bildad din Șuah și Țofar din Naamat, s-au dus și au făcut cum le spusese Domnul, și Domnul a primit rugăciunea lui Iov.
\par 10 Și Domnul l-a pus pe Iov iarăși în starea lui de la început, după ce s-a rugat pentru prieteni, și i-a întors îndoit tot ce avusese mai înainte.
\par 11 Și toți frații și toate surorile și toți prietenii lui de altă dată au venit să-l cerceteze, au mâncat pâine în casa lui, l-au compătimit, l-au mângâiat de toate nenorocirile pe care le slobozise Domnul asupra lui și fiecare i-a dat câte un chesita și câte un inel de aur.
\par 12 Și Dumnezeu a binecuvântat sfârșitul vieții lui Iov mai bogat decât începutul ei, și el a strâns paisprezece mii de oi, șase mii de cămile, o mie de perechi de boi și o mie de asine.
\par 13 Și a avut șapte fii și trei fiice.
\par 14 Celei dintâi i-a pus numele Iemima, celei de a doua, Cheția și celei de a treia, Cheren-Hapuc.
\par 15 Iar în toată țara nu se găseau femei atât de frumoase ca fetele lui Iov, și tatăl lor le-a făcut părtașe la moștenire, lângă frații lor.
\par 16 Și Iov a mai trăit după aceea o sută patruzeci de ani și a văzut pe fiii săi și pe fiii fiilor săi, până la al patrulea neam.
\par 17 Și Iov a murit bătrân și încărcat de zile.


\end{document}