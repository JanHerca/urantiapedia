\begin{document}

\title{Iov}


\chapter{1}

\par 1 Era odata în ?inutul Uz un om pe care îl chema Iov ?i acest om era fara prihana ?i drept; se temea de Dumnezeu ?i se ferea de ce este rau.
\par 2 ?i i s-au nascut ?apte feciori ?i trei fete.
\par 3 El avea ?apte mii de oi, trei mii de camile, cinci sute de perechi de boi ?i cinci sute de asini ?i mul?ime mare de slugi. ?i omul acesta era cel mai de seama dintre to?i rasaritenii.
\par 4 Feciorii lui se duceau unul la altul ?i faceau ospe?e în casele lor, fiecare la ziua lui, ?i trimiteau sa cheme pe surorile lor ca sa manânce ?i sa bea cu ei.
\par 5 ?i apoi, când ispraveau zilele petrecerii lor, Iov chema ?i sfin?ea pe feciorii sai ?i se scula dis-de-diminea?a ?i aducea arderi de tot, dupa numarul lor al tuturor, caci Iov zicea: "Se poate ca feciorii mei sa fi pacatuit ?i sa fi cugetat cu pacat împotriva lui Dumnezeu". ?i a?a facea Iov mereu.
\par 6 Dar într-o zi îngerii lui Dumnezeu s-au înfa?i?at înaintea Domnului ?i Satan a venit ?i el printre ei.
\par 7 Atunci Domnul a zis catre Satan: "De unde vii?" Iar Satan a raspuns Domnului ?i a zis: "Am dat târcoale pe pamânt ?i m-am plimbat în sus ?i în jos".
\par 8 ?i Domnul a zis catre Satan: "Te-ai uitat la robul Meu Iov, ca nu este nici unul ca el pe pamânt fara prihana ?i drept ?i temator de Dumnezeu ?i care sa se fereasca de ce este rau?"
\par 9 Dar Satan a raspuns Domnului ?i a zis: "Ore degeaba se teme Iov de Dumnezeu?
\par 10 N-ai facut Tu gard în jurul lui ?i în jurul casei lui ?i în jurul a tot ce este al lui, în toate par?ile ?i ai binecuvântat lucrul mâinilor lui ?i turmele lui au umplut pamântul?
\par 11 Dar ia întinde mâna Ta ?i atinge-Te de tot ce este al lui, sa vedem daca nu Te va blestema în fa?a!"
\par 12 Atunci Domnul a zis catre Satan: "Iata, tot ce are el este în puterea ta; numai asupra lui sa nu întinzi mâna ta". ?i Satan a pierit din fa?a lui Dumnezeu.
\par 13 ?i într-o zi, când feciorii ?i fetele lui Iov mâncau ?i beau vin în casa fratelui lor mai mare,
\par 14 A sosit un vestitor la Iov ?i i-a spus: "Boii erau la aratura ?i asinele pa?teau pe lânga ei;
\par 15 Atunci Sabeenii au navalit asupra lor, au pus mâna pe vite, ?i pe robi i-au trecut prin ascu?i?ul sabiei. ?i am scapat numai eu singur ?i am venit sa-?i dau de veste!"
\par 16 Nu a sfâr?it vorba bine ?i altul a sosit ?i a spus: "Focul lui Dumnezeu a cazut din cer ?i a ars oile tale ?i pe robii tai ?i i-a mistuit. ?i am scapat numai eu singur ?i am venit sa-?i dau de veste!"
\par 17 Nu a sfâr?it vorba bine ?i altul a sosit ?i a spus: "Caldeii, împar?i?i în trei cete, au dat navala peste camilele tale ?i le-au ridicat ?i pe robi i-au trecut prin ascu?i?ul sabiei. ?i am scapat numai eu singur ?i am venit sa-?i dau de veste!"
\par 18 Nu sfâr?ise vorba bine ?i altul a sosit ?i a spus: "Feciorii tai ?i fetele tale mâncau ?i beau vin în casa fratelui lor mai mare,
\par 19 ?i iata ca un vânt puternic s-a stârnit dinspre pustiu ?i a izbit în cele patru col?uri ale casei ?i casa s-a prabu?it peste tineri ?i ei au murit. ?i am scapat numai eu singur ?i am venit sa-?i dau de veste".
\par 20 Atunci Iov s-a sculat ?i-a sfâ?iat ve?mântul, s-a ras pe cap ?i, cazând la pamânt, s-a închinat,
\par 21 ?i a rostit: "Gol am ie?it din pântecele mamei mele ?i gol ma voi întoarce în pamânt! Domnul a dat, Domnul a luat; fie numele Domnului binecuvântat!"
\par 22 ?i întru toate acestea, Iov nu a pacatuit ?i nu a rostit nici un cuvânt de hula împotriva lui Dumnezeu.

\chapter{2}

\par 1 ?i iara?i au venit într-o zi îngerii lui Dumnezeu sa se înfa?i?eze înaintea Domnului ?i Satan a venit ?i el printre ei sa se înfa?i?eze înaintea Domnului.
\par 2 ?i Domnul a zis catre Satan: "De unde vii?" Iar Satan a raspuns Domnului ?i a zis: "Am dat târcoale pe pamânt ?i m-am plimbat în sus ?i în jos".
\par 3 ?i Domnul a zis catre Satan: "Ai luat tu seama la robul Meu Iov? Ca nu este nici unul ca el pe pamânt, fara prihana ?i drept ?i temator de Dumnezeu ?i care sa se fereasca de ce este rau. El se ?ine cu putere în statornicia lui ?i tu M-ai întarâtat pe nedrept împotriva lui ca sa-l prapadesc".
\par 4 Dar Satan a raspuns Domnului ?i a zis: "Cojoc pentru cojoc! Ca tot ce are omul da pentru via?a lui.
\par 5 Dar ia întinde-?i mâna ?i atinge-Te de osul ?i de carnea lui! Sa vedem daca nu Te va blestema în fa?a!"
\par 6 ?i Domnul a zis catre Satan: "Îl dau în puterea ta! Numai nu te atinge de via?a lui".
\par 7 Atunci Satan a plecat dinaintea Domnului ?i a lovit pe Iov cu lepra, din talpile picioarelor pâna în cre?tetul capului.
\par 8 ?i a luat Iov un ciob ca sa se scarpine ?i ?edea pe gunoi, afara din ora?.
\par 9 Atunci nevasta lui a zis catre el: "Te ?ii mereu în statornicia ta? Blesteama pe Dumnezeu ?i mori!
\par 10 Dar Iov i-a raspuns: "Vorbe?ti cum ar vorbi una din femeile nebune! Ce? Daca am primit de la Dumnezeu cele bune, nu vom primi oare ?i pe cele rele?" ?i în toate acestea, Iov n-a pacatuit de loc cu buzele sale.
\par 11 Iar trei prieteni ai lui Iov au aflat despre toate aceste nenorociri care dadusera peste el ?i au venit fiecare din ?ara lui ?i ei erau: Elifaz din Teman, Bildad din ?uah ?i ?ofar din Naamah. Ei se în?elesesera împreuna sa vina sa împarta?easca durerea lui ?i sa-l mângâie.
\par 12 ?i când ei ?i-au ridicat ochii de departe nu l-au mai recunoscut. Atunci au slobozit glasurile lor, s-au tânguit ?i ?i-au sfâ?iat fiecare ve?mântul ?i ?i-au presarat capul cu ?arâna.
\par 13 Apoi au ?ezut pe pamânt, lânga el, ?apte zile ?i ?apte nop?i, fara sa-i spuna nici un cuvânt, caci vedeau cât este de mare durerea lui.

\chapter{3}

\par 1 Dupa aceea, Iov a deschis gura sa ?i a blestemat ziua în care s-a nascut.
\par 2 ?i Iov a vorbit ?i a zis:
\par 3 "Piara ziua în care m-am nascut ?i noaptea care a zis: un prunc de parte barbateasca s-a zamislit!
\par 4 Ziua aceea sa se faca întuneric ?i Domnului din cer sa nu-I pese de ea ?i lumina sa n-o mai lumineze.
\par 5 Bezna ?i umbra mor?ii s-o cotropeasca, norii s-o învaluiasca ?i toate negurile s-o înspaimânte!
\par 6 Întunericul sa cuprinda noaptea aceea ?i sa nu mai fie pusa în zilele anului ?i în socoteala lunilor sa nu mai intre!
\par 7 Pustie sa ramâna noaptea aceea ?i nici o bucurie sa nu patrunda în ea!
\par 8 Blestemata sa fie de catre cei ce blesteama zilele, de catre cei ce ?tiu sa descânte Leviatanul.
\par 9 Sa se întunece stelele revarsatului zorilor ei; sa a?tepte lumina ?i nimic sa nu vina ?i sa nu mai vada genele aurorei,
\par 10 Pentru ca n-a închis pântecele care m-a zamislit ?i n-a ascuns durerea dinaintea ochilor mei.
\par 11 De ce n-am murit când eram în sânul mamei mele? ?i nu mi-am dat duhul, ie?ind din pântecele ei?
\par 12 De ce m-au primit cei doi genunchi ?i de ce cei doi sâni mi-au dat sa sug?
\par 13 Caci acum a? sta culcat ?i lini?tit, a? dormi ?i m-a? odihni,
\par 14 Cu împara?ii ?i cu dregatorii pamântului, care ?i-au zidit morminte în singuratate,
\par 15 Sau cu domnitorii care umplu de aur ?i de argint casele lor.
\par 16 Sau de ce n-am fost o stârpitura aruncata ?i ascunsa, ca acei prunci care n-au apucat sa vada lumina?
\par 17 Acolo cei nelegiui?i se astâmpara ?i cei împovara?i se odihnesc.
\par 18 Acolo cei ce poarta lan?uri ajung la liman de pace ?i nu mai aud glasul paznicului.
\par 19 Mic ?i mare acolo sunt tot una ?i robul a scapat de stapânul sau.
\par 20 Pentru ce da Dumnezeu lumina vie?ii celui nenorocit ?i zile celor cu sufletul amarât;
\par 21 Celor ce a?teapta moartea, ?i ea nu vine, ?i care scormonesc dupa ea mai mult ca dupa o comoara;
\par 22 Celor ce se bucura cu bucurie mare ?i sunt plini de fericire, fiindca au gasit un mormânt;
\par 23 Celui care nu ?tie încotro sa mearga ?i pe care îl îngrade?te Dumnezeu de jur-împrejur?
\par 24 Gemetele mele sunt pâinea mea ?i vaietele mele curg ca apa.
\par 25 De ceea ce ma tem, aceea mi se întâmpla ?i de ceea ce mi-e frica tocmai de aceea am parte.
\par 26 N-am nici tihna, nici odihna, nu-mi gasesc nici o pace ?i zbuciumul ma stapâne?te".

\chapter{4}

\par 1 Atunci Elifaz din Teman a deschis gura ?i a zis:
\par 2 "Sa-?i vorbim ori sa nu-?i vorbim? Necazul tau e crâncen! Dar cine ar putea sa-?i înabu?e cuvintele?
\par 3 Iata, tu dadeai înva?atura multora ?i întareai multe mâini slabite.
\par 4 Cuvintele tale au ?inut în sus pe cei ce erau sa cada ?i tu ai întarit genunchii care se clatinau.
\par 5 Acum când ?i-a venit ?i ?ie rândul, e?ti la strâmtorare ?i ?i-ai pierdut firea; acum când lovitura te-a ajuns, te-ai spaimântat!
\par 6 Frica ta de Dumnezeu nu-?i da încredere ?i desavâr?irea cailor tale nu-?i da nadejde?
\par 7 Ia adu-?i aminte, care nevinovat s-a prapadit ?i unde le-a pierit urma celor drep?i în fala lui Dumnezeu?
\par 8 Dupa cum am vazut eu, numai cei ce ara nelegiuirea ?i seamana rautatea au parte de asemenea roade.
\par 9 La porunca Domnului, ei vor pieri, de suflarea mâniei Lui se vor stinge.
\par 10 Racnetul leului ?i glasul leopardului, precum ?i din?ii puilor de lei se sfarâma.
\par 11 Leul batrân moare ca nu mai are ce mânca ?i puii leoaicei se risipesc.
\par 12 O ?oapta a razbit pâna la mine ?i urechea mea a prins ceva din ea.
\par 13 În spaimele care vin din nalucirile nop?ii, atunci când somn adânc se lasa peste oameni,
\par 14 Cutremur m-a apucat ?i fiori mi-au scuturat toate oasele.
\par 15 Atunci un duh a trecut prin fa?a mea; tot parul mi s-a zbârlit de groaza.
\par 16 A stat drept în picioare, dar n-am ?tiut cine este; o umbra este înaintea ochilor mei, ?i aud o voce u?oara care zice:
\par 17 "Un om poate sa fie drept în fa?a lui Dumnezeu? O faptura omeneasca este ea curata înaintea Celui ce a zidit-o?
\par 18 Daca El nu se încrede în slujitorii Sai ?i daca gase?te vina chiar îngerilor Sai,
\par 19 Cu cât mai vârtos celor ce locuiesc în locuin?e de lut, a caror obâr?ie este în ?arâna ?i pe care îi strive?te ca pe o molie.
\par 20 De diminea?a pâna seara sunt zdrobi?i, pier pe vecie fara sa-i scape nimeni.
\par 21 Mor, dar nu mor de prea multa în?elepciune".

\chapter{5}

\par 1 Striga acum, daca o fi cineva care sa-?i raspunda. Catre care din sfin?ii îngeri te vei îndrepta?
\par 2 Mânia ucide pe cel fara de minte, iar aprinderea omoara pe cel ratacit.
\par 3 Am vazut pe nebun prinzând radacina ?i pe loc am blestemat sala?ul lui:
\par 4 Sa se departeze copiii lui de orice izbavire ?i sa fie calca?i în picioare la poarta ?i nimeni sa nu le vina într-ajutor.
\par 5 Seceri?ul lui sa-l manânce flamânzii ?i sa-l duca cu ei în ascunzi?uri ?i toata averea lui s-o soarba înseta?ii!
\par 6 Pentru ca nelegiuirea nu iese din pamânt ?i necazul nu rasare din pulbere,
\par 7 Ci omul î?i na?te singur suferin?a, precum vulturii se ridica în aer, prin puterile lor;
\par 8 Dar eu alerg la Dumnezeu ?i Lui Îi arat necazul meu.
\par 9 El face lucruri mari ?i nepatrunse, lucruri minunate ?i fara numar.
\par 10 El da ploaie pe pamânt ?i trimite apa pe câmpii.
\par 11 El înal?a pe cei smeri?i ?i izbave?te pe cei necaji?i.
\par 12 El destrama planurile celor vicleni ?i cu mâinile lor nu pot sa izbândeasca.
\par 13 El prinde pe în?elep?i în iste?imea lor ?i sfatul celor în?elatori iese prost.
\par 14 Ziua în amiaza mare dau de întuneric ?i umbla ca pe timp de noapte în ceas de zi.
\par 15 Dar Dumnezeu scapa din gura lor pe omul dosadit ?i din mâna celui puternic pe cel sarac.
\par 16 Astfel, sarmanul prinde nadejde ?i nedreptatea î?i închide gura.
\par 17 Fericit este omul pe care Dumnezeu îl mustra! ?i sa nu dispre?uie?ti certarea Celui Atotputernic.
\par 18 Caci El rane?te ?i El leaga rana, El love?te ?i mâinile Lui tamaduiesc.
\par 19 De ?ase ori din nevoi te va scoate, iar a ?aptea oara, raul te va ocoli.
\par 20 În timp de foamete, te va scapa de la moarte ?i în batalie din primejdia sabiei.
\par 21 Vei fi la adapost de biciul bârfelii ?i nu te vei teme de prapad când va veni.
\par 22 Î?i vei râde de pustiire ?i de foamete ?i nu-?i va pasa de fiarele pamântului.
\par 23 Caci vei avea legamânt cu pietrele câmpului ?i fiara salbatica va trai cu tine în pace.
\par 24 Tu vei ?ti cortul tau la adapost ?i când î?i vei cerceta locuin?a nu vei avea dezamagire.
\par 25 Vei ?ti ca urma?ii tai sunt numero?i ?i ca odraslele tale sunt multe ca iarba pamântului.
\par 26 Sosi-vei la mormânt, la adânci batrâne?e, ca o ?ira de grâu strânsa la vremea ei.
\par 27 Iata ceea ce am cercetat ?i a?a este. Asculta ?i înva?a spre folosul tau".

\chapter{6}

\par 1 Atunci Iov a raspuns ?i a grait:
\par 2 "O, daca durerea mea s-ar cântari ?i nenorocirea mea ar fi pusa la cântar!
\par 3 ?i fiindca este mai grea decât nisipul marilor, de aceea cuvintele mele sunt bâlbâite!
\par 4 Pentru ca sage?ile Celui Atotputernic stau înfipte în mine ?i duhul meu bea veninul lor, de aceea spaimele lui Dumnezeu vin cete-cete împotriva mea.
\par 5 Zbiara magarul salbatic când e lânga pa?unea verde? Muge?te boul când sta lânga nutre??
\par 6 Po?i sa manânci ce n-are sare fara sare? Are vreun gust albu?ul oului?
\par 7 Sufletul meu n-a voit sa se atinga de ele; inima mea s-a dezgustat de pâinea mea.
\par 8 Cine îmi va  darui îndeplinirea rugaciunii mele ?i va face ca Dumnezeu sa-mi dea ce a?tept,
\par 9 ?i sa primeasca sa ma sfarâme ?i sa-?i întinda mâna ?i sa ma nimiceasca!
\par 10 Dar va fi înca o mângâiere pentru mine ?i voi tresalta, de?i împovarat de dureri nemiloase, fiindca n-am ascuns poruncile Celui Sfânt.
\par 11 Ce putere mai am ca sa a?tept ?i ce viitor mai am ca sa prelungesc via?a mea?
\par 12 Taria mea este taria pietrelor? Trupul meu este oare de arama?
\par 13 A? putea gasi vreun sprijin în mine ?i tot ajutorul n-a fugit, oare, departe de mine?
\par 14 Celui ce este în suferin?a i se cuvine mila prietenului sau, dar el uita teama de Cel Atotputernic.
\par 15 Fra?ii mei s-au aratat în?elatori ca un puhoi, ca albia puhoaielor repezi.
\par 16 Erau acoperite de ghia?a, zapada statea gramada peste ele;
\par 17 Dar cum se tope?te zapada, ele ?i seaca ?i, cum se încalze?te, ele se usuca pe loc.
\par 18 Caravanele se abat din drumul lor, ele înainteaza în pustiu ?i se ratacesc.
\par 19 Caravanele din Tema a?teapta, convoaiele din Saba nadajduiesc în ele.
\par 20 ?i sunt uimi?i ca au avut încredere; când sosesc lânga uscatele puhoaie sunt uimi?i.
\par 21 A?a a?i fost ?i voi acum pentru mine: Va scutura spaima ?i va este frica!
\par 22 Nu cumva v-am zis: Da?i-mi de pomana ?i împar?i?i din averile voastre pentru mine?
\par 23 Sau scapa?i-ma din mâna unui du?man sau rascumpara?i-ma din mâna tiranilor?
\par 24 Fi?i înva?atorii mei ?i eu voi tacea; lamuri?i-ma unde este pacatul meu!
\par 25 Cât de îmbietoare sunt cuvintele întregimii suflete?ti! Dar ce judeca judecata care vine de la voi?
\par 26 Cugeta?i voi sa face?i judecata vorbelor? Duca-se în vânt cuvintele unui deznadajduit!
\par 27 Voi napastui?i pe orfan, voi împovara?i pe prietenul vostru.
\par 28 ?i acum întreba?i ?i va întoarce?i catre mine ?i în fa?a voastra nu voi spune minciuna!
\par 29 Cerceta?i din nou! Nu este nici o viclenie! Cerceta?i din nou! Dreptatea mea este mereu aici!
\par 30 Este oare vreo strâmbatate pe limba mea ?i cerul gurii mele nu va deosebi el ce este rau ?i ce este amar?

\chapter{7}

\par 1 Oare omul pe pamânt nu este ca într-o slujba osta?easca ?i zilele lui nu sunt ca zilele unui simbria??
\par 2 El este asemenea robului care suspina dupa umbra, asemenea naimitului care-?i a?teapta simbria.
\par 3 Astfel ?i eu am avut parte de luni de durere, ?i mi-au fost date nop?i de suferin?a.
\par 4 Daca ma culc, zic: Când va veni ziua? Daca ma scol, ma întreb: Când va veni seara? ?i sunt napadit de fel de fel de aratari pâna la asfin?it.
\par 5 Trupul meu e plin de paduchi ?i de solzi de murdarie; pielea mea crapa ?i se zbârce?te.
\par 6 Zilele mele au fost mai grabnice ca suveica ?i s-au ispravit, fiindca firul s-a terminat.
\par 7 Adu-?i aminte, Doamne, ca via?a mea e o suflare, ca ochiul meu nu va mai vedea fericirea.
\par 8 Ochiul celui ce ma vedea nu ma va mai zari; ochii Tai ma vor cauta, dar eu nu voi mai fi.
\par 9 Negura se risipe?te, piere, tot astfel cel ce coboara în iad nu mai vine înapoi.
\par 10 Nu se mai înapoiaza în casa sa ?i locuin?a sa nu-l mai cunoa?te.
\par 11 Drept aceea nu voi pune straja gurii mele, ci voi vorbi întru deznadejdea duhului meu ?i ma voi plânge întru amaraciunea inimii mele.
\par 12 Sunt eu, oare, oceanul sau balaurul din ocean, ca sa pui sa ma pazeasca?
\par 13 Când gândesc: Patul meu ma va odihni, culcu?ul meu îmi va alina durerile,
\par 14 Atunci Tu ma spaimântezi cu vise ?i ma îngroze?ti cu naluciri.
\par 15 Pentru aceea, sufletul meu ar vrea mai bine ?treangul, mai bine moartea decât aceste chinuri.
\par 16 Ma ispravesc, nu voi trai în veac; lasa-ma, caci zilele mele sunt o suflare.
\par 17 Ce este omul, ca sa-?i ba?i capul cu el ?i ca sa-i dai luarea Ta aminte?
\par 18 De ce îl cercetezi în fiecare diminea?a ?i de ce îl urmare?ti în orice clipa?
\par 19 Când vei înceta sa ma prive?ti? Când îmi vei da ragaz sa-mi înghit saliva?
\par 20 Daca am gre?it, ce ?i-am facut ?ie, Pazitorule de oameni? De ce m-ai luat ?inta pentru sage?ile Tale ?i de ce ?i-am ajuns povara?
\par 21 De ce nu îngadui gre?eala mea ?i nu la?i sa treaca faradelegea mea? Degraba ma voi culca în ?arâna; ma vei cauta, dar nu ma vei mai gasi".

\chapter{8}

\par 1 Atunci Bildad din ?uah a raspuns ?i a zis:
\par 2 "Pâna când vei tot vorbi astfel de lucruri ?i cuvintele din gura ta vor izvorî ca vijelia?
\par 3 Dumnezeu o sa încovoaie ce e drept? Cel Atotputernic o sa strâmbe El dreptatea?
\par 4 Daca feciorii tai au pacatuit fa?a de El, El i-a lasat sa se prabu?easca sub povara nelegiuirii lor.
\par 5 Dar tu daca vii la Dumnezeu, daca te rogi de Cel Atotputernic,
\par 6 Daca e?ti nevinovat ?i fara pata, atunci de buna seama ca va veghea asupra ta ?i va cladi la loc casa drepta?ii tale
\par 7 ?i starea ta cea veche va fi nimica toata, atât de mult vei fi deasupra în starea ta cea noua.
\par 8 Întreaba pe cei care au fost înaintea noastra ?i ia aminte la cele traite ?i pa?ite de parin?i.
\par 9 Caci noi suntem de ieri ?i nu ?tim nimic, caci zilele noastre pe pamânt nu sunt decât o umbra.
\par 10 Ei î?i vor da înva?atura, ei î?i vor grai ?i din inima lor î?i vor cuvânta unele ca acestea:
\par 11 Oare papura cre?te fara balta ?i rogozul fara umezeala?
\par 12 Pe când înca este în floare ?i nu este taiat, el se usuca, mai înainte decât orice buruiana.
\par 13 Tot a?a se întâmpla cu to?i aceia care uita pe Dumnezeu ?i a?a se ve?teje?te nadejdea celui nelegiuit.
\par 14 Încrederea lui e spulberata ?i bizuin?a lui este o pânza de paianjen.
\par 15 Se sprijina pe casa sa, dar ea nu se ?ine; se aga?a de ea, dar casa se pravale.
\par 16 Sta plin de suc în fa?a soarelui ?i în gradina unde este î?i întinde vlastarii;
\par 17 Radacinile lui se împletesc cu pietrele ?i se înfig în adâncul stâncilor.
\par 18 Daca îl smulgi din loc, locul îl tagaduie?te: Nu te-am vazut niciodata!
\par 19 Iata-l acum putred, pe carare, ?i din pamânt rasar al?i vlastari.
\par 20 Dumnezeu nu dispre?uie?te pe cel desavâr?it ?i nu ia de mâna pe raufacatori.
\par 21 Gura ta va fi plina înca o data de râsete ?i buzele tale de veselie.
\par 22 Cei ce te urasc se vor înve?mânta în ru?ine ?i cortul celor rai va pieri!"

\chapter{9}

\par 1 Atunci Iov a raspuns ?i a zis:
\par 2 "?tiu bine ca a?a este; caci cum ar putea un om sa fie drept înaintea lui Dumnezeu?
\par 3 Daca ar fi sa se certe cu El, din o mie de lucruri n-ar putea sa-I raspunda nici la unul singur.
\par 4 A Lui este în?elepciunea ?i atotputernicia; cine ar putea sa-I stea împotriva ?i sa ramâna teafar?
\par 5 El mi?ca mun?ii din loc fara ca ei sa prinda de veste ca El i-a rasturnat în mânia Lui.
\par 6 El zguduie pamântul din temelia lui, a?a încât stâlpii lui se clatina.
\par 7 El porunce?te soarelui ?i soarele nu se mai ridica. El pune pecetea Lui asupra stelelor.
\par 8 El singur este Cel ce întinde cerurile ?i umbla pe valurile marii.
\par 9 El a zidit Carul mare, Rali?a, Pleiadele ?i camarile stelelor de miazazi.
\par 10 El a facut lucruri mari ?i nepatrunse ?i minuni fara de numar.
\par 11 Iata, daca trece pe lânga mine, eu nu-L vad, ?i daca se strecoara, eu nu-I prind de veste.
\par 12 Daca ia ?i ridica, cine va putea sa-L opreasca ?i cine-I va zice: Ce ai facut?
\par 13 Dumnezeu nu-?i înfrâneaza mânia Sa ?i sub El se încovoaie to?i slujitorii mândriei.
\par 14 ?i eu atunci cum o sa-I raspund ?i ce cuvinte o sa aleg?
\par 15 Chiar daca a? avea dreptate, nu-I voi raspunde, ci ma voi ruga judecatorului.
\par 16 Chiar daca m-ar asculta, când Îl chem, tot n-a? putea sa cred ca asculta glasul meu,
\par 17 Caci El ma sfarâma ca sub furtuna ?i înmul?e?te fara cuvânt ranile mele.
\par 18 El nu-mi da ragaz sa rasuflu ?i ma adapa cu amaraciune.
\par 19 Daca este vorba de putere, El este Cel puternic. Daca este vorba de judecata, cine ma va apara?
\par 20 Oricâta dreptate a? avea, gura mea ma va osândi ?i daca sunt fara prihana ea ma scoate vinovat.
\par 21 Sunt oare desavâr?it? Eu singur nu ma cunosc pe mine ?i via?a mea o dispre?uiesc.
\par 22 Pentru aceea am zis: Tot una este! El nimice?te pe cel desavâr?it ?i pe cel viclean.
\par 23 Daca o nenorocire aduce moartea deodata, ce-I pasa Lui de deznadejdea celor fara de vina?
\par 24 Daca o ?ara a încaput pe mâna unui om viclean, El acopera fa?a judecatorilor Sai. ?i daca nu El, cine atunci?
\par 25 Zilele mele sunt mai grabnice decât un aducator de ve?ti ?i au fugit fara sa vada fericirea.
\par 26 S-au strecurat ca ni?te barci de papura, ca un vultur care se napuste?te asupra prazii sale.
\par 27 Daca zic: Vreau sa-mi uit suferin?a, sa-mi schimb înfa?i?area ?i sa fiu voios,
\par 28 Sunt napadit de teama chinurilor mele, ?tiind bine ca Tu nu ma vei scoate nevinovat.
\par 29 Daca sunt vinovat, de ce sa ma mai trudesc zadarnic?
\par 30 Daca m-a? spala cu zapada ?i mi-a? cura?i mâinile cu le?ie,
\par 31 Atunci Tu tot m-ai cufunda în noroi, încât ?i ve?mintele mele s-ar scârbi de mine.
\par 32 Caci Dumnezeu nu este un om ca mine, ca sa stau cu El de vorba ?i ca sa mergem împreuna la judecata.
\par 33 Între noi nu se afla un al treilea care sa-?i puna mâna peste noi amândoi
\par 34 ?i care sa departeze varga Sa de deasupra capului meu, a?a încât groaza Lui sa nu ma mai tulbure;
\par 35 Atunci a? vorbi ?i nu m-a? mai teme de El. Dar nu este a?a ?i eu sunt singur cu mine însumi.

\chapter{10}

\par 1 Sufletul meu este dezgustat de via?a mea. Voi lasa sa curga sloboda tânguirea mea ?i voi vorbi întru suferin?a sufletului meu.
\par 2 Voi spune catre Domnul: Nu ma osândi; lamure?te-ma, sa ?tiu pentru ce Te cer?i cu mine.
\par 3 Care e folosul Tau, când e?ti aprig ?i dispre?uie?ti faptura mâinilor Tale ?i e?ti surâzator la sfatul celor rai?
\par 4 Ai Tu ochi materiali ?i vezi lucrurile precum le vede omul?
\par 5 Zilele Tale sunt oare ca zilele omului ?i anii Tai ca anii omene?ti,
\par 6 Ca sa cercetezi faradelegea mea ?i sa cau?i pacatul meu,
\par 7 Când ?tii bine ca nu sunt vinovat ?i ca nimeni nu ma poate scapa din mâna Ta?
\par 8 Mâinile Tale m-au facut ?i m-au zidit ?i apoi Tu ma nimice?ti în întregime.
\par 9 Adu-?i aminte ca m-ai facut din pamânt ?i ca ma vei întoarce în ?arâna.
\par 10 Nu m-ai turnat oare ca pe lapte ?i nu m-ai închegat ca pe ca??
\par 11 M-ai îmbracat în piele ?i în carne, m-ai ?esut din oase ?i din vine;
\par 12 Apoi mi-ai dat via?a, ?i bunavoin?a Ta ?i purtarea Ta de grija au ?inut vie suflarea mea,
\par 13 ?i ceea ce Tu ?ineai ascuns în inima Ta, iata ?tiu acum gândul Tau:
\par 14 Daca pacatuiesc, Tu ma supraveghezi ?i nu ma dezvinova?e?ti de gre?eala mea.
\par 15 Daca sunt vinovat este amar de mine ?i daca sunt drept nu cutez sa ridic capul, ca unul ce sunt satul de ocara ?i sunt apasat de necazuri.
\par 16 ?i astfel fara vlaga cum sunt, Tu ma vânezi ca un leu ?i din nou Te ara?i minunat fa?a de mine.
\par 17 Tu înnoie?ti du?mania Ta împotriva mea. Tu spore?ti mânia Ta asupra-mi ca ni?te o?tiri primenite care se lupta cu mine.
\par 18 De ce m-ai scos din sânul mamei mele? A? fi murit ?i nici un ochi nu m-ar fi vazut!
\par 19 A? fi fost ca unul care n-a fost niciodata ?i din pântecele mamei mele a? fi trecut în mormânt.
\par 20 Nu sunt oare zilele mele destul de pu?ine? Da-Te la o parte ca sa pot sa-mi vin pu?in în fire,
\par 21 Mai înainte ca sa plec spre a nu ma mai întoarce din ?inutul întunericului ?i al umbrelor mor?ii,
\par 22 ?ara de întuneric ?i neorânduiala unde lumina e totuna cu bezna".

\chapter{11}

\par 1 Atunci ?ofar, din Naamah, a luat cuvântul ?i a vorbit:
\par 2 "Cel ce în?ira atâtea vorbe sa nu primeasca nici un raspuns ?i tocmai vorbare?ul sa aiba dreptate?
\par 3 Toate câte le-ai spus îi vor face pe oameni sa taca ?i vei râde de ei, fara ca nimeni sa te înfrunte?
\par 4 Fiindca tu zici: Credin?a mea este curata ?i în ochii Tai n-am nici o vina.
\par 5 Dar cine va face pe Dumnezeu sa vorbeasca, sa Î?i deschida buzele spre tine,
\par 6 ?i sa-?i destainuiasca tainele în?elepciunii? (caci ele sunt cu anevoie de în?eles); atunci de-abia vei ?ti ca Dumnezeu î?i cere socoteala de gre?eala ta.
\par 7 Descoperi-vei tu care este firea lui Dumnezeu? Urca-vei tu pâna la desavâr?irea Celui Atotputernic?
\par 8 Ea este mai înalta decât cerurile. ?i ce vei face tu? Ea este mai adânca decât împara?ia mor?ii. Cum vei patrunde-o tu?
\par 9 Masura ei este mai lunga decât pamântul ?i mai lata decât marea.
\par 10 Daca trece cu vederea, daca ?ine ascuns, daca da pe fa?a, cine poate sa-L opreasca?
\par 11 El cunoa?te pe cei ce traiesc din în?elaciune, El vede nedreptatea ?i o ?ine în seama;
\par 12 Astfel deci un om fara minte câ?tiga în?elepciune, precum puiul de asin ajunge asin mare.
\par 13 Cât despre tine, daca inima ta e credincioasa ?i daca întinzi mâinile catre El,
\par 14 ?i departezi de mâna ta faradelegea ei ?i nu rabzi sa locuiasca nedreptatea în corturile tale,
\par 15 Atunci vei ridica fruntea ta fara pata pe ea, vei fi puternic ?i fara frica.
\par 16 Fiindca vei uita necazul tau de azi ?i-?i vei aduce aminte de el numai ca de ni?te ape, care au fost ?i au trecut.
\par 17 ?i via?a ta va înflori mai mândra decât miezul zilei, iar întunericul se va face revarsat de zori.
\par 18 Atunci tu vei fi la adapost, caci vei fi plin de nadejde, te vei sim?i ocrotit ?i te vei culca fara grija.
\par 19 Te vei întinde fara ca sa te strâmtoreze nimeni ?i mul?i vor rasfa?a obrazul tau.
\par 20 Dar ochii nelegiui?ilor tânjesc ?i loc de scapare nu au, iar nadejdea este când î?i vor da sufletul".

\chapter{12}

\par 1 Atunci Iov a raspuns ?i a zis:
\par 2 "Cu adevarat numai voi sunte?i în?elep?i ?i în?elepciunea va muri o data cu voi.
\par 3 Dar ?i eu am minte ca voi ?i nu sunt mai prejos decât voi ?i cine nu cunoa?te lucrurile pe care mi le-a?i spus?
\par 4 Eu am ajuns pricina de batjocura pentru prietenul meu, eu care chem pe Dumnezeu ?i caruia El raspunde: cel drept, cel fara vina e pricina de râs.
\par 5 Sa dispre?uim nenorocirea (gândesc cei ferici?i); înca o lovitura celor ce se poticnesc.
\par 6 Foarte lini?tite stau ?i sunt corturile jefuitorilor ?i cei ce mânie pe Dumnezeu sunt plini de încredere, ca unii care au facut din pumnul lor un dumnezeu.
\par 7 Dar ia întreaba dobitoacele ?i te vor înva?a, ?i pasarile cerului, ?i te vor lamuri;
\par 8 Sau vorbe?te cu pamântul, ?i-?i va da înva?atura ?i pe?tii marii î?i vor istorisi cu de-amanuntul.
\par 9 Cine nu cunoa?te din toate acestea ca mâna Domnului a facut aceste lucruri?
\par 10 În mâna Lui El ?ine via?a a tot ce traie?te ?i suflarea întregii omeniri.
\par 11 Urechea nu deosebe?te ea cuvintele tot a?a, precum cerul gurii deosebe?te mâncarea?
\par 12 Oare nu la batrâni sala?luie?te în?elepciunea ?i priceperea nu merge mâna în mâna cu vârsta înaintata?
\par 13 La Dumnezeu se afla în?elepciunea ?i puterea; sfatul ?i patrunderea sunt ale Lui.
\par 14 Ceea ce darâma El, nimeni nu mai zide?te la loc ?i pe cine-l închide, nimeni nu poate sa-l mai deschida.
\par 15 Daca opre?te apele pe loc, ele scad ?i pier; daca le da drumul, ele rastoarna lumea;
\par 16 Taria ?i în?elepciunea sunt la El. El este stapân ?i peste ratacit ?i peste cel ce-l face sa rataceasca.
\par 17 El gone?te pe sfetnici în picioarele goale ?i pe judecatori îi arunca prada nebuniei.
\par 18 El destrama puterea împara?ilor ?i pune cingatoare de frânghie în jurul coapselor lor.
\par 19 El gone?te pe preo?i în picioarele goale ?i da peste cap pe cei puternici.
\par 20 El taie vorba celor me?teri la cuvânt ?i ia mintea celor batrâni.
\par 21 El face de ocara pe cei mari ?i slabe?te încingatoarea celor voinici.
\par 22 El scoate din întuneric lucrurile ascunse ?i aduce la lumina ceea ce era acoperit de umbra.
\par 23 El spore?te neamurile ?i apoi le pierde, El le lasa sa se întinda ?i apoi le strâmtoreaza.
\par 24 El scoate din min?i pe capeteniile popoarelor ?i îi lasa sa rataceasca în singurata?i fara carari.
\par 25 Acolo ei orbecaiesc în întuneric, fara nici o lumina, caci Dumnezeu îi lasa sa se împleticeasca aidoma celui ce s-a îmbatat.

\chapter{13}

\par 1 De buna seama, ochiul meu a vazut toate acestea, urechea mea le-a auzit ?i le-a în?eles.
\par 2 Ceea ce ?ti?i voi, ?tiu ?i eu ?i nu sunt deloc mai prejos decât voi.
\par 3 Dar eu vreau sa vorbesc cu Cel Atotputernic, vreau sa-mi apar pricina înaintea lui Dumnezeu.
\par 4 Caci voi sunte?i ni?te nascocitori ai minciunii, sunte?i cu to?ii ni?te doctori neputincio?i!
\par 5 Ce bine ar fi fost daca a?i fi tacut! Câta în?elepciune ar fi fost din partea voastra!
\par 6 Asculta?i acum apararea mea ?i baga?i de seama la rostirea buzelor mele.
\par 7 Oare de dragul lui Dumnezeu spune?i voi lucruri strâmbe ?i spre apararea Lui croi?i minciuni?
\par 8 Voi?i sa ?ine?i cu El ?i sa fi?i aparatorii Lui?
\par 9 ?i daca ar fi ca sa va cerceteze, ar fi bine de voi, sau vre?i sa-L în?ela?i cum în?ela?i un om?
\par 10 Desigur El va va osândi daca în ascuns vre?i sa fi?i partinitori cu El.
\par 11 Mare?ia Lui oare nu va înfrico?eaza ?i groaza Lui nu va cadea oare peste voi?
\par 12 Rostirile voastre au taria cenu?ei. Raspunsurile voastre se prefac în noroi.
\par 13 Închide?i gura în fa?a mea ?i eu voi vorbi, orice ar fi sa se întâmple.
\par 14 Drept aceea îmi voi lua în din?i carnea mea ?i via?a mea o pun în mâna mea.
\par 15 Daca o fi sa ma ucida, nu voi tremura, dar voi descurca în fa?a Sa firele pricinei mele,
\par 16 ?i chiar aceasta îmi va fi mie izbânda, fiindca un nelegiuit nu se înfa?i?eaza înaintea Lui.
\par 17 Asculta?i cu luare-aminte cuvintele mele ?i ceea ce va voi spune sa va ramâna în urechi.
\par 18 Iata am pus la cale o judecata, ?i ?tiu ca eu sunt cel ce am dreptate.
\par 19 Are cineva ceva de spus împotriva mea? Atunci voi amu?i degraba ?i voi a?tepta moartea.
\par 20 Numai scute?te-ma de doua lucruri, ?i nu ma voi ascunde de fa?a Ta.
\par 21 Departeaza mâna Ta de deasupra-mi ?i nu ma mai tulbura cu groaza Ta.
\par 22 Apoi cheama-ma ?i eu i?i voi raspunde, sau lasa-ma sa vorbesc eu ?i Tu sa-mi dai raspuns.
\par 23 Câte gre?eli ?i câte pacate am facut? Da-mi pe fa?a calcarea mea de lege ?i pacatul meu.
\par 24 De ce ascunzi fa?a Ta ?i ma iei drept un du?man al Tau?
\par 25 Vrei oare, sa înspaimân?i o frunza pe care o bate vântul? Vrei sa Te îndârje?ti împotriva unui pai uscat?
\par 26 De ce sa scrii împotriva mea aceste hotarâri amare? De ce sa-mi sco?i ochii cu gre?elile tinere?ii?
\par 27 De ce sa-mi vâri picioarele în butuci ?i sa pânde?ti to?i pa?ii mei ?i toate urmele mele?
\par 28 Când Tu ?tii ca trupul meu se nimice?te ca un putregai ?i ca o haina mâncata de molii!

\chapter{14}

\par 1 Omul nascut din femeie are pu?ine zile de trait, dar se satura de necazuri.
\par 2 Ca ?i floarea, el cre?te ?i se ve?teje?te ?i ca umbra el fuge ?i e fara durata.
\par 3 ?i asupra lui prive?ti ?i pe mine Tu ma sile?ti sa vin la judecata cu Tine.
\par 4 Cine ar putea sa scoata ceva curat din ceea ce este necurat? Nimeni!
\par 5 Deoarece zilele lui sunt masurate ?i ?tii socoteala lunilor lui ?i i-ai pus un hotar peste care nu va trece.
\par 6 Întoarce-?i privirea de la el, ca sa aiba pu?in ragaz, sa se poata bucura ca simbria?ul la sfâr?itul zilei (de munca).
\par 7 Un copac, de pilda, tot are nadejde, caci daca-l tai, el cre?te din nou ?i vlastarii nu-i vor lipsi.
\par 8 Daca radacina lui îmbatrâne?te în pamânt ?i daca trunchiul lui putreze?te,
\par 9 Când da de apa înverze?te din nou ?i se acopera cu ramuri ca ?i cum ar fi atunci sadit.
\par 10 Dar omul când moare ramâne nimicit; când omul î?i da sufletul, unde mai este el?
\par 11 Apele marilor pot sa dispara, fluviile pot sa scada ?i sa sece.
\par 12 La fel ?i omul se culca ?i nu se mai scoala; ?i cât vor sta cerurile, el nu se mai de?teapta ?i nu se mai treze?te din somnul lui.
\par 13 O, de m-ai ascunde în împara?ia mor?ilor, ca sa ma ?ii acolo pâna când va trece mânia Ta, ?i de mi-ai soroci o vreme, când iara?i sa-?i aduci aminte de mine!
\par 14 Daca omul a murit o data, fi-va el iara?i viu? Toate zilele robiei mele a? a?tepta pâna ce vor veni sa ma schimbe.
\par 15 Atunci Tu ma vei chema ?i eu Î?i voi raspunde; Tu vei cere înapoi lucrul mâinilor Tale.
\par 16 Pe când astazi Tu numeri pa?ii mei; atunci Tu nu vei mai lua seama la pacatul meu.
\par 17 Nelegiuirea mea ar fi pecetluita ca într-un sac ?i gre?eala mea ai spala-o ?i ai face-o alba.
\par 18 ?i precum muntele se darâma ?i se preface în nisip ?i precum stânca e rostogolita din locul ei,
\par 19 ?i precum apele manânca pietrele ?i valurile lor acopera pamântul, tot a?a Tu sfarâmi nadejdea omului.
\par 20 Tu Te ridici uria? împotriva lui, ?i el se nimice?te; Tu schimbi înfa?i?area lui ?i-l trimi?i de la Tine.
\par 21 Daca feciorii lui ajung la mare cinste, el nu mai ?tie; daca au ajuns de râsul lumii, el nu-i mai vede.
\par 22 Carnea lui e în întristare mare numai pentru el. Sufletul lui numai pentru el e cuprins de jale".

\chapter{15}

\par 1 Atunci Elifaz, din Teman, a raspuns ?i a zis:
\par 2 "Este oare cinstit pentru în?elept sa raspunda cu cuvinte u?uratice ?i sa-?i umple pieptul cu suflarea vântului de rasarit?
\par 3 I se cuvine lui sa judece cu vorbe seci ?i prin cuvântari care n-au nici o noima?
\par 4 Tu mergi atât de departe, încât desfiin?ezi cucernicia ?i nesocote?ti rugaciunea înaintea lui Dumnezeu.
\par 5 Nelegiuirea ta insufla gura ta ?i tu împrumu?i vorbirea ta de la cei vicleni.
\par 6 Chiar gura ta te osânde?te ?i nu eu, chiar buzele tale sunt martore împotriva ta.
\par 7 Nu cumva e?ti tu cel dintâi om care s-a nascut? Venit-ai tu pe lume mai înainte decât mun?ii?
\par 8 Ai stat tu de sfat cu Dumnezeu ?i te-ai facut tu stapân pe toata în?elepciunea?
\par 9 Ce ?tii tu pe care sa nu-l ?tim ?i noi? Ce pricepi tu ?i noi nu pricepem?
\par 10 Printre noi se afla oameni vechi de zile, batrâni mai în vârsta decât tatal tau.
\par 11 ?i se pare pu?in lucru mângâierile în numele lui Dumnezeu ?i cuvintele spuse cu blânde?e?
\par 12 De ce te la?i târât de inima ta ?i de ce prive?ti a?a trufa? cu ochii tai?
\par 13 De ce întorci spre Dumnezeu mânia ta ?i dai drumul din gura ta la astfel de cuvântari?
\par 14 Ce este omul ca sa se creada curat, ?i cel nascut din femeie, ca sa se creada neprihanit?
\par 15 Daca Dumnezeu nu are încredere în sfin?ii Sai ?i daca cerurile nu sunt destul de curate înaintea ochilor Sai,
\par 16 Cu atât mai pu?in o faptura urâcioasa ?i stricata cum este omul cel ce bea nedreptatea ca apa.
\par 17 Vreau sa-?i dau o înva?atura, asculta-ma; ?i ceea ce am vazut vreau sa-?i aduc la cuno?tin?a;
\par 18 Ceea ce în?elep?ii au vestit fara sa ascunda nimic, precum au auzit de la parin?ii lor,
\par 19 Atunci când ?ara le-a fost data numai lor ?i nici un strain nu se a?ezase înca printre ei.
\par 20 Nelegiuitul se chinuie?te în toate zilele vie?ii sale ?i de-a lungul anilor harazi?i celui tiran.
\par 21 Glasuri îngrozitoare fac larma în urechile lui; în mijlocul pacii, i se pare ca un uciga? se napuste?te asupra lui.
\par 22 El nu mai nadajduie?te sa mai iasa din întuneric ?i î?i simte capul mereu sub sabie.
\par 23 Se ?i vede aruncat de mâncare vulturilor, fiindca ?tie ca prapadul lui este fara întârziere.
\par 24 Ziua întunericul îl înspaimânta. Zbuciumul ?i tulburarea îl strâng la mijloc ?i se arunca asupra-i gata de împresurare,
\par 25 Fiindca a îndraznit sa-?i îndrepte mâna împotriva lui Dumnezeu ?i sa faca pe viteazul fa?a de Cel Atotputernic;
\par 26 Fiindca a îndraznit sa navaleasca împotriva Lui cu gâtul întins ?i la adapostul scuturilor sale groase ?i rotunde.
\par 27 Chipul lui i se ascundea în grasime ?i osânza statea grea pe coapsele lui,
\par 28 ?i sala?luia în ceta?i pustiite, în case în care nu mai statea nimeni, fiindca amenin?au sa se prabu?easca.
\par 29 Nu va aduna boga?ie ?i ce are nu va ?ine mult, iar umbra lui nu se va lungi pe pamânt.
\par 30 Nu va mai putea sa iasa din întuneric. Focul va mistui ramurile sale ?i vijelia va matura florile lui;
\par 31 Sa nu se creada în minciuna, fiindca ?tim ca e de?ertaciune.
\par 32 Vrejul sau se va ofili mai înainte de vreme ?i mladi?a sa nu va da muguri verzi.
\par 33 Ca vi?a manata, va lasa sa cada rodul sau ?i la fel ca maslinul va împra?tia florile sale.
\par 34 Fiindca ceata celui rau la inima va fi lasata stearpa ?i focul mistuie corturile cu boga?ii de jaf.
\par 35 Ei zamislesc rautatea ?i nasc nelegiuirea, dar cu aceasta pântecele lor dospe?te în?elaciunea".

\chapter{16}

\par 1 Atunci Iov a raspuns ?i a grait:
\par 2 "Am auzit mereu astfel de lucruri; sunte?i to?i ni?te jalnici mângâietori.
\par 3 Când se vor sfâr?i aceste vorbe goale ?i ce te chinuie?te ca sa raspunzi?
\par 4 ?i eu a? vorbi a?a ca voi, daca sufletul vostru ar fi în locul sufletului meu; a? putea sa spun multe cuvinte împotriva voastra ?i sa dau din cap în privin?a voastra.
\par 5 V-a? mângâia numai cu gura ?i cu mi?carea buzelor mele v-a? aduce u?urare.
\par 6 Dar, daca vorbesc, durerea mea nu se lini?te?te ?i daca tac din gura, durerea mea nu se departeaza de la mine.
\par 7 În ceasul de fa?a, Dumnezeu mi-a luat toata vlaga; toata ticalo?ia mea ma împresoara, Doamne!
\par 8 M-ai acoperit cu zbârcituri, care toate marturisesc împotriva mea; neputin?a mea ea însa?i ma da de gol ?i bârfitorul sta împotriva mea.
\par 9 El ma sfâ?ie în furia Lui ?i se poarta cu mine du?manos, scrâ?ne?te din din?i împotriva mea; du?manul meu arunca asupra-mi sage?ile ochilor sai;
\par 10 Deschis-au gura lor împotriva mea, în batjocura m-au lovit peste obraji. To?i gramada se înghesuie împotriva mea.
\par 11 Dumnezeu ma da pe mâna unui pagân. El ma arunca prada celor rai.
\par 12 Mi-era destul de bine, dar El m-a sfarâmat. M-a luat de ceafa ?i m-a facut praf ?i a aruncat asupra-mi toate sage?ile Sale;
\par 13 În jurul meu se învârtesc sage?ile Sale; El îmi strapunge rarunchii fara mila; El varsa pe pamânt fierea mea.
\par 14 El ma darâma bucata cu bucata ?i navale?te asupra mea ca un razboinic.
\par 15 Am cusut un sac pe trupul meu ?i am vârât în ?arâna capul meu.
\par 16 Chipul meu s-a înro?it de plânset ?i umbra mor?ii s-a sala?luit în pleoapele mele;
\par 17 ?i cu toate acestea, în mâinile mele nu este nici o silnicie ?i rugaciunea mea este curata!
\par 18 Pamântule, nu ascunde sângele meu ?i sa nu fie nici un loc nestrabatut de bocetele mele.
\par 19 Iar acum martorul meu este în ceruri ?i cel ce da pentru mine buna marturie este sus în locurile înalte.
\par 20 Prietenii mei î?i bat joc de mine, dar ochiul meu varsa lacrimi înaintea lui Dumnezeu.
\par 21 O, de-ar fi îngaduit omului sa stea de vorba cu Dumnezeu, cum sta de vorba un am "u prietenul sau!
\par 22 Caci ace?ti pu?ini ani se vor scurge ?i voi apuca pe un drum de pe care nu ma voi mai întoarce.

\chapter{17}

\par 1 Sufletul meu e darapanat, zilele mele se sting, mormântul ma a?teapta.
\par 2 Sunt împresurat de batjocoritori ?i ochii mei trebuie sa priveasca spre ocarile lor.
\par 3 Da-mi acum cheza?ia Ta lânga Tine, altfel cine ar vrea sa raspunda pentru mine?
\par 4 Pentru ca Tu ai luat priceperea din inima lor, de aceea Tu nu-i vei ridica.
\par 5 Sunt unii care fac ospa? cu prietenii, atunci când acasa ochii copiilor se sting de foame.
\par 6 Am ajuns de poveste între oameni; sunt acela pe care-l scuipi în fa?a.
\par 7 Ochii mei s-au întunecat de suparare, madularele mele s-au sub?iat ca umbra.
\par 8 Oamenii cei drep?i stau înmarmuri?i ?i cel nevinovat se rascoala împotriva celui nelegiuit.
\par 9 Cel ce este drept se ?ine însa de calea sa ?i cine este cu mâinile curate e din ce în ce mai tare.
\par 10 Cât despre voi ceilal?i, voi to?i da?i înapoi ?i veni?i aici, caci nu voi gasi printre voi nici un în?elept.
\par 11 Zilele mele s-au scurs, socotin?ele mele s-au sfarâmat ?i la fel dorin?ele inimii mele.
\par 12 Din noapte ei vor sa faca zi ?i spun ca lumina este mai aproape decât întunericul.
\par 13 Mai pot sa nadajduiesc? Împara?ia mor?ii este casa mea, culcu?ul meu l-am întins în inima întunericului.
\par 14 Am zis mormântului: Tu e?ti tatal meu; am zis viermilor: voi sunte?i mama ?i surorile mele!
\par 15 Atunci unde mai este nadejdea mea ?i cine a vazut pe undeva norocul meu?
\par 16 El s-a rostogolit pâna în fundul iadului ?i împreuna cu mine se va cufunda în ?arâna".

\chapter{18}

\par 1 Atunci Bildad din ?uah a început sa vorbeasca ?i a zis:
\par 2 "Când vei ajunge odata la capatul unor astfel de vorbe? Vino-?i în fire ?i apoi vom vorbi.
\par 3 Pentru ce suntem socoti?i ca ni?te dobitoace? De ce sa trecem în ochii tai drept vite cornute?
\par 4 Nu cumva pentru tine care te sfâ?ii în mânia ta, o sa ajunga pamântul sa se pustiiasca ?i stâncile sa se mute din locul lor?
\par 5 Fire?te, lumina nelegiuitului se stinge ?i flacara focului lui nu mai straluce?te.
\par 6 Lumina se întuneca în cortul lui ?i candela de deasupra lui se isprave?te.
\par 7 Pa?ii lui, altadata vânjo?i, se îngusteaza ?i propriul lui sfat acum îl poticne?te.
\par 8 El da cu picioarele în la? ?i se plimba în re?eaua de sfori.
\par 9 Capcana l-a prins de calcâi ?i la?ul s-a încolacit pe el.
\par 10 Cursa care trebuia sa-l prinda este ascunsa în pamânt ?i prinzatoarea sta în poteca lui;
\par 11 Spaimele dau peste el din toate par?ile ?i se ?in de el pas cu pas.
\par 12 Lânga bunata?ile lui el moare de foame ?i nenorocirea lui sta gata lânga el.
\par 13 Boala mu?ca din trupul lui. Primul nascut al mor?ii roade madularele lui.
\par 14 Din cortul unde statea la adapost este scos afara ?i târât înaintea groaznicului împarat.
\par 15 Nimeni din ai lui nu mai sala?luie?te în cortul lui, care nu mai este al lui. Pe locuin?a lui ploua cu pucioasa.
\par 16 Radacinile lui se usuca în pamânt, iar ramurile lui se ve?tejesc în aer.
\par 17 Pomenirea lui se ?terge de pe pamânt ?i în toata lumea numele i-a pierit.
\par 18 De la lumina i-au dat brânci în întuneric ?i de pretutindeni e scos afara.
\par 19 Nu lasa nici urma?i, nici samân?a în poporul sau ?i nimeni nu mai traie?te dupa el prin locurile prin care a locuit.
\par 20 Cei din apus s-au mirat foarte de soarta lui ?i cei din rasarit au sim?it fiori în ei.
\par 21 Aceasta ramâne din sala?urile celui nelegiuit ?i iata locul celui ce n-a cunoscut pe Dumnezeu".

\chapter{19}

\par 1 Atunci Iov a început sa vorbeasca ?i a zis:
\par 2 "Câta vreme ve?i întrista voi sufletul meu ?i ma ve?i zdrobi cu cuvântarile voastre?
\par 3 Iata a zecea oara de când ma batjocori?i. Nu va este ru?ine ca va purta?i a?a?
\par 4 Chiar daca ar fi adevarat ca am pacatuit, gre?eala mea este pe capul meu.
\par 5 Iar daca voi va face?i tari ?i mari împotriva mea ?i-mi scoate?i ochii cu ticalo?ia mea,
\par 6 Sa ?tii ca Dumnezeu este Cel ce ma urmare?te ?i ca El m-a învaluit cu la?ul Sau.
\par 7 Daca strig de multa-apasare, nu primesc nici un raspuns; ?ip în gura mare, dar nimeni nu-mi face dreptate.
\par 8 El a astupat calea mea, ca sa nu mai trec pe ea, ?i a acoperit cu bezna toate drumurile mele.
\par 9 M-a dezbracat de marirea mea ?i mi-a smuls cununa de pe cap.
\par 10 M-a darâmat de jur împrejur ?i sunt în ceasul mor?ii ?i nadejdea mea a scos-o din radacina ca pe un copac.
\par 11 Aprins-a împotriva mea mânia Sa ?i m-a luat drept du?manul Sau.
\par 12 Hoardele Sale sosesc gramada, î?i fac drum pâna la mine ?i pun tabara de jur împrejurul cortului meu.
\par 13 A departat pe fra?ii mei de lânga mine ?i cunoscu?ii mei î?i întorc fa?a când ma vad.
\par 14 Rudele mele au pierit, casnicii mei au uitat de mine.
\par 15 Cei ce locuiau împreuna cu mine ?i slujnicele mele se uita la mine ca la un strain; sunt în ochii lor ca unul venit din alta ?ara.
\par 16 Chem pe sluga mea ?i nu-mi raspunde, macar ca o rog cu gura mea.
\par 17 Suflarea mea a ajuns nesuferita pentru femeia mea ?i am ajuns sa miros greu pentru fiii cei nascu?i din coapsele mele.
\par 18 Pâna ?i copiii îmi arata dispre?; când ma scol, vorbesc pe seama mea.
\par 19 To?i sfetnicii mei cei mai de aproape ma urgisesc ?i aceia pe care îi iubeam s-au întors împotriva mea.
\par 20 Oasele mele ies afara din piele ?i nu mi-au mai ramas tefere decât gingiile.
\par 21 Mila fie-va de mine, ave?i mila de mine, o, voi, prietenii mei, caci mâna lui Dumnezeu m-a lovit!
\par 22 De ce ma prigoni?i cu urgia lui Dumnezeu ?i nu va mai satura?i de carnea mea?
\par 23 Cit a? vrea ca vorbele mele sa fie scrise, cât a? vrea sa fie sapate pe arama.
\par 24 Sa fie sapate pe veci, cu un condei de fier ?i de plumb, într-o stânca!
\par 25 Dar eu ?tiu ca Rascumparatorul meu este viu ?i ca El, în ziua cea de pe urma, va ridica iar din pulbere aceasta piele a mea ce se destrama.
\par 26 ?i afara din trupul meu voi vedea pe Dumnezeu.
\par 27 Pe El Îl voi vedea ?i ochii mei Îl vor privi, nu ai altuia. ?i de dorul acesta maruntaiele mele tânjesc în mine.
\par 28 Iar daca zice?i: Cum îl vom urmari ?i ce pricina de proces vom gasi noi în el?
\par 29 Teme?i-va de sabie, pentru voi în?iva, când mânia va izbucni împotriva gre?elii voastre. ?i atunci ve?i înva?a ca este o judecata!"

\chapter{20}

\par 1 ?i ?ofar din Naamah a început sa vorbeasca ?i a zis:
\par 2 "Cugetul meu ma împinge sa vorbesc, din pricina framântarii pe care o simt în mine.
\par 3 Am auzit o înva?atura care ma scoate din sarite ?i atunci o pornire vijelioasa, din duhul meu, ma face sa raspund.
\par 4 Nu ?tii tu oare ca de mult de tot, din zilele când omul a fost a?ezat pe pamânt,
\par 5 Desfatarea celor fara de lege ?ine foarte pu?in ?i bucuria fa?arnicului nu sta decât o clipa?
\par 6 Chiar daca statura lui s-ar înal?a pâna la ceruri ?i cu capul s-ar atinge de nori,
\par 7 El totu?i va pieri ca o naluca, pe vecie, ?i cei ce îl vedeau vor întreba: Ce s-a facut?
\par 8 Zboara ca un vis ?i nu mai dai de el, e maturat ca o vedenie de noapte.
\par 9 Ochiul, care îl privea, nu-l mai vede ?i locul unde se gasea, nu-l mai zare?te.
\par 10 Feciorii lui vor trebui sa cer?easca mila celor saraci ?i mâinile lui vor da înapoi ce a luat cu for?a.
\par 11 Oasele lui sunt înca pline de vlaga tinere?ii, dar ea se va culca cu el în pulbere;
\par 12 Daca rautatea este dulce în gura lui, el o ascunde sub limba lui.
\par 13 Daca o tine în gura ?i nu o scuipa, daca o mesteca în cerul gurii,
\par 14 Totu?i hrana lui în maruntaiele lui se strica ?i se face în intestinele lui venin de napârca.
\par 15 Averea, pe care a înghi?it-o, acum o varsa; Dumnezeu i-o da afara din pântece.
\par 16 Venin de ?arpe otravitor sugea. Limba de napârca îl omoara!
\par 17 El nu va mai vedea pâraiele de proaspat untdelemn, valurile de miere ?i de smântâna.
\par 18 Da îndarat ce-a câ?tigat ?i nu se mai folose?te de câ?tig ?i de roadele negustoriei sale nu se mai bucura.
\par 19 Pentru ca a asuprit fara mila pe saraci ?i a furat o casa, în loc sa o zideasca.
\par 20 El nu va cunoa?te pacea launtrica ?i el nu va scapa nimic din toate câte pre?uie?te.
\par 21 Nimic nu scapa de lacomia lui, de aceea înflorirea lui nu va ?ine deloc.
\par 22 Când boga?ia lui va fi la culme, tulburarea îl va apuca deodata ?i toate loviturile nenorocirii vor cadea în capul lui.
\par 23 Când va fi gata sa-?i sature pântecele, Dumnezeu va dezlan?ui asupra-i urgia mâniei Sale ?i va ploua cu sage?i peste el.
\par 24 Daca va scapa de plato?a de fier, îl va strapunge arcul de arama.
\par 25 O sageata îi iese din spate, o alta i s-a înfipt în fica?i ?i spaimele mor?ii îl sfâr?esc.
\par 26 Toata neagra pieire amenin?a comorile pe care le-a adunat; un foc care arde neaprins îl mistuie pe oricine va mai ramâne din cortul lui.
\par 27 Cerurile vor dezveli faradelegea lui ?i pamântul i se va ridica împotriva.
\par 28 Napraznica revarsare de ape va matura casa lui ?i apele vor curge în ziua dumnezeie?tii mânii.
\par 29 Aceasta este partea harazita de Dumnezeu omului nelegiuit, aceasta este mo?tenirea pe care el o prime?te de la Domnul".

\chapter{21}

\par 1 Atunci Iov a vorbit înca o data ?i a zis:
\par 2 "Asculta?i cu luare-aminte cuvântul meu ?i aici sa se opreasca mângâierile voastre.
\par 3 Îngadui?i-mi sa vorbesc ?i eu, ?i dupa ce voi vorbi, atunci po?i sa-?i ba?i joc.
\par 4 Oare plângerea mea se înal?a împotriva unui om? ?i atunci rabdarea mea cum n-o sa fie pe sfâr?ite?
\par 5 Uita?i-va la mine ?i mira?i-va foarte ?i pune?i mâna la gura.
\par 6 Caci, când ma gândesc, ma apuca groaza ?i toata carnea de pe mine tremura.
\par 7 Pentru ce ticalo?ii au via?a, ajung la adânci batrâne?e ?i sporesc în putere?
\par 8 Urma?ii lor se ridica voinici în fa?a lor ?i odraslele lor dainuiesc sub ochii lor.
\par 9 Casele lor stau nevatamate, fara teama ?i varga lui Dumnezeu nu sta deasupra lor.
\par 10 Taurii sunt plini de vlaga ?i prasitori, juncanele lor fata ?i nu leapada.
\par 11 Copiii lor zburda ca oile ?i odraslele lor dan?uiesc împrejur.
\par 12 Ei cânta din toba ?i din harfa ?i se desfata la sunetele flautului.
\par 13 Î?i ispravesc zilele în fericire ?i coboara cu pace în împara?ia mor?ii.
\par 14 ?i tocmai ei ziceau lui Dumnezeu: "În laturi de la noi! Nu vrem deloc sa cunoa?tem caile Tale!
\par 15 Cine este Cel Atotputernic ca sa-I slujim Lui ?i ce folos vom avea sa-I înal?am rugaciuni?"
\par 16 N-ai zice, oare, ca fericirea lor e în mâna lor? Sfatul celor rai nu este totdeauna departe de Dumnezeu?
\par 17 De câte ori se stinge candela nelegiui?ilor ?i nenorocirea da navala peste ei? De câte ori Dumnezeu nimice?te cu mânia Sa pe cei raufacatori,
\par 18 Ca sa fie ei ca paiul în bataia vântului ?i ca pleava pe care o rasuce?te vârtejul?
\par 19 Dumnezeu, vei zice, pastreaza pentru copiii lui rasplata faradelegii lui. Dar sa-l pedepseasca pe el însu?i, ca sa se înve?e.
\par 20 Sa-?i vada cu ochii nenorocirea ?i sa se adape din mânia Celui Atotputernic!
\par 21 Fiindca ce-i mai pasa de casa lui, dupa moartea lui, când numarul lunilor lui a fost retezat?
\par 22 Dar nu cumva Îi vom da noi înva?atura lui Dumnezeu, Lui care sta ?i judeca pe cei de sus?
\par 23 Unul moare, în plinatatea puterii sale, când este înconjurat de fericire ?i de pace,
\par 24 Când gale?ile îi sunt pline de lapte ?i oasele pe care le suge, pline cu maduva.
\par 25 Altul moare, cu sufletul cople?it de amaraciune, fara sa fi gustat vreo fericire.
\par 26 ?i unul ?i altul se culca în ?arâna ?i viermii îi cotropesc.
\par 27 ?tiu prea bine gândurile voastre ?i socotin?ele pe care vi le fauri?i în privin?a mea.
\par 28 Voi zice?i în mintea voastra: Unde este casa asupritorului ?i unde este cortul în care locuiau nelegiui?ii?
\par 29 N-a?i întrebat oare pe cei ce trec pe drum ?i n-a?i recunoscut dreptatea spuselor lor?
\par 30 Anume cum ca în ziua nenorocirii cel rau este cru?at ?i ca în ceasul mâniei el scapa?
\par 31 Cine îl mustra în fa?a pentru purtarea lui ?i cine-i întoarce cu aceea?i masura faptele pe care le-a facut?
\par 32 Iar când este dus la locul de odihna, din stâlpul de la capatâi el parca sta de straja.
\par 33 Bulgarii pamântului îi sunt u?ori; în convoi pe urma lui înainteaza toata lumea, ?i înaintea lui o mul?ime nenumarata.
\par 34 Atunci ce sunt de?artele mângâieri pe care mi le da?i? Din toate cuvintele voastre nu ramâne decât în?elaciune".

\chapter{22}

\par 1 Elifaz din Teman a raspuns atunci ?i a zis:
\par 2 "Poate omul sa fie de vreun folos lui Dumnezeu? Nu, fiindca în?eleptul î?i este de folos lui însu?i.
\par 3 Ce are Cel Atotputernic daca tu e?ti fara prihana? ?i care este câ?tigul Lui, daca drumurile tale sunt fara vina?
\par 4 Oare El te pedepse?te pentru cucernicia ta ?i pentru ea intra cu tine în judecata?
\par 5 Nu, dimpotriva, fiindca rautatea ta este mare ?i faradelegile tale sunt fara hotar!
\par 6 Caci tu fara dreptate luai zaloage de la fra?ii tai ?i smulgeai ve?mântul de pe oameni ?i-i lasai goi.
\par 7 Tu nu dadeai sa bea celui însetat ?i nu dadeai sa manânce celui flamând;
\par 8 Cel cu pumnul tare cotrope?te pamântul ?i cel cu trecere îl ia, în stapânire.
\par 9 Goneai de la pragul tau pe vaduve cu mâinile goale ?i bra?ele celor orfani tu le sfarâmai.
\par 10 Acesta este cuvântul pentru care la?uri te înconjoara ?i spaimele te-au apucat dintr-o data.
\par 11 Lumina s-a stins pentru tine ?i nu mai vezi ?i o apa revarsata te-a dat la fund.
\par 12 Dumnezeu nu este El oare mai presus de ceruri? Prive?te în sus spre stele cât de sus sunt ele!
\par 13 Tu ai zis: Ce ?tie Dumnezeu! Judeca El oare prin umbra?
\par 14 Norii sunt ca o perdea în fa?a Lui ?i El nu poate sa vada; El se plimba numai de jur împrejurul cerurilor.
\par 15 Voie?ti tu sa urmezi pe stravechea cale pe care au batatorit-o oamenii cei fara de lege?
\par 16 Cei ce au fost matura?i înainte de vreme, când un fluviu s-a rostogolit peste temeliile lor,
\par 17 ?i ei ziceau lui Dumnezeu: "În laturi de la noi! ?i ce poate sa ne faca Cel Atotputernic?"
\par 18 Dar tocmai El umpluse casele lor de bunata?i, însa sfatul celor rai ramânea departe de Dumnezeu.
\par 19 Cei drep?i se uita ?i se bucura, iar cel nevinovat râde de ei.
\par 20 Iata, avu?ia lor a nimicit-o ?i focul a mistuit toata strânsura lor!
\par 21 Împaca-te cu Dumnezeu ?i cazi la pace. Atunci bine va fi de tine.
\par 22 Prime?te, te rog, înva?atura din gura Lui ?i pune la inima cuvintele Lui;
\par 23 Daca te întorci la Cel puternic ?i te smere?ti, daca departezi nedreptatea de cortul tau,
\par 24 Atunci aurul tau îl vei pre?ui drept ?arâna ?i comorile Ofirului drept pietricele,
\par 25 Pentru ca Cel Atotputernic va fi pentru tine sloi de aur ?i gramezi de argint.
\par 26 Atunci tu te vei desfata întru Cel Atotputernic ?i ridica-vei fa?a ta catre Dumnezeu.
\par 27 Tu vei chema numele Lui ?i El te va auzi ?i tu vei împlini juruin?ele tale.
\par 28 Când te vei hotarî sa faci un lucru, lucrul îl vei izbuti ?i lumina va straluci pe toate drumurile tale,
\par 29 Fiindca Dumnezeu smere?te pe mândri ?i mândria, ?i mântuie?te pe acela care-?i pleaca ochii în pamânt.
\par 30 El izbave?te pe cel nevinovat ?i tu la fel vei scapa, când mâinile tale vor fi curate".

\chapter{23}

\par 1 Dar Iov iara?i a vorbit ?i a zis:
\par 2 "?i de data aceasta plângerea mea este luata tot ca razvratire ?i totu?i mâna mea de-abia înabu?e suspinele mele.
\par 3 O, daca a? ?ti unde sa-L gasesc! Daca a? putea sa ajung la palatul Lui!
\par 4 Atunci a? dezvalui înaintea Lui pricina mea ?i a? umple gura mea cu învinuiri.
\par 5 A? ?ti atunci cuvintele cu care mi-ar raspunde ?i a? în?elege rostul spuselor Lui.
\par 6 ?i-ar dezlan?ui El oare toata puterea în cearta Lui cu mine? Nu, El ar sta ?i m-ar asculta.
\par 7 El ar lua aminte la omul drept care vorbe?te în fala Lui ?i astfel a? fi iertat pe vecie de Judecatorul meu.
\par 8 Caci iata, daca o iau spre rasarit, El nu este acolo; daca o iau spre apus, nu-L zaresc!
\par 9 L-am cautat spre miazanoapte ?i n-am dat de El, m-am întors catre miazazi ?i nici aici nu L-am vazut!
\par 10 Dar El cunoa?te ?i umbletul meu ?i starea mea pe loc ?i daca ar fi sa ma treaca prin cuptor de foc, voi ie?i din cuptor curat ca aurul.
\par 11 M-am ?inut cu pasul meu dupa pasul Lui, am pazit calea Lui ?i nu m-am abatut din ea.
\par 12 De la porunca buzelor Sale nu m-am departat, la sânul meu am ?inut ascunse cuvintele gurii Sale.
\par 13 Dar hotarârea Lui este luata ?i cine-L va împiedica? Caci ceea ce sufletul Sau a poftit, aceea va ?i face.
\par 14 Fiindca El aduce la îndeplinire hotarârea Sa ?i alte foarte multe lucruri la fel, care sunt în gândul Sau.
\par 15 Iata pentru ce sunt înspaimântat în fa?a Lui. Ma gândesc ?i mi-e teama de El.
\par 16 Dumnezeu a slabit inima mea ?i Cel Atotputernic m-a îngrozit.
\par 17 ?i n-am tacut din pricina întunericului ?i din pricina nop?ii care a învaluit fa?a mea.

\chapter{24}

\par 1 De ce, pentru Cel Atotputernic, vremurile rasplatirilor sunt ascunse ?i cei ce-L cunosc n-au vazut zilele Sale de judecator?
\par 2 Viclenii muta hotarele ?arinilor, fura turma de oi cu cioban cu tot.
\par 3 Duc la ei acasa asinul copiilor orfani ?i iau zalog boul vaduvei.
\par 4 Dau la o parte de pe cale pe cei saraci din ?ara, iar pe to?i nenoroci?ii din ?ara îi silesc sa se ascunda.
\par 5 Ace?tia la fel cu asinii salbatici din pustie ies pe furi? sa-?i caute de mâncare ?i, dupa ce lucreaza pâna seara, tot n-au pâine pentru copii.
\par 6 Ei secera noaptea pe câmp, ei culeg via nelegiuitului;
\par 7 Petrec noaptea goi, fiindca n-au cu ce sa se înveleasca, pentru ca n-au ve?mânt sa se apere de frig.
\par 8 Ploaia repede din mun?i îi uda pâna la piele ?i în loc de adapost strâng în bra?e stâncile.
\par 9 Cei dintâi smulg pe orfan de la ?â?a ?i iau zalog haina saracului.
\par 10 ?i saracii umbla goi, fara îmbracaminte ?i, istovi?i de foame, duc în spinare snopii.
\par 11 La teascul bogatului, ei storc untdelemnul, ei calca jghiaburile cu struguri ?i tânjesc de sete.
\par 12 În cetate, muribunzii se vaita ?i sufletul celor rani?i cere ajutor; dar Dumnezeu n-aude rugaciunea lor!
\par 13 Mai sunt razvrati?i împotriva zilei, care nu cunosc cararile ei ?i nu ramân în potecile ei.
\par 14 Uciga?ul se scoala dis-de-diminea?a, ucide pe cel sarac ?i nevoia? ?i jefuie?te.
\par 15 Ochii celui desfrânat pândesc amurgul zilei ?i el î?i zice: Nu ma vede nici ?ipenie de om, ?i î?i pune o mahrama pe fa?a.
\par 16 Tâlharul, acoperit de întuneric, sparge casele ?i intra în ele, caci el le-a pus semn de cu ziua,
\par 17 Iar când vine diminea?a, parca ar fi pentru ei umbra mor?ii. Când zorii stralucesc, toate spaimele mor?ii dau peste ei.
\par 18 Nelegiuitul plute?te u?or ca pe fa?a apelor, dar pe pamânt partea lui este plina de blestem ?i fericirea nu va calca niciodata via lui.
\par 19 Precum seceta ?i ar?i?a sorb apele zapezilor topite, tot astfel soarbe locuin?a mor?ilor pe pacato?i.
\par 20 Pântecele mamei lor l-au uitat, viermii se desfateaza din el, nimeni nu-l mai ?ine minte ?i astfel nelegiuirea lor s-a frânt ca un copac.
\par 21 Ei chinuiau pe femeia stearpa ?i fara de copii, ei s-au purtat aprig cu femeia vaduva.
\par 22 Dar Cel ce, prin puterea Lui, strune?te pe cei puternici, se ridica razbunator ?i to?i ace?tia nu se mai ?in stapâni pe via?a lor.
\par 23 El îi lasa sa se sprijine cu buna încredin?are, dar ochii Lui erau asupra cailor lor.
\par 24 Se ridicasera, dar acum nu mai sunt, s-au a?ternut ca nalba, când o cose?ti ?i ca spicul ierbii s-au ve?tejit.
\par 25 Daca zice?i ca nu este a?a, cine îmi va dovedi ca am min?it ?i cine va spulbera cuvântul meu?"

\chapter{25}

\par 1 Atunci Bildad din ?uah a început sa vorbeasca ?i a zis:
\par 2 "A Lui este stapânirea, a Lui este puterea înfrico?atoare! ?i El sala?luie?te pacea în locurile preaînalte.
\par 3 Cine poate sa numere o?tile Sale? ?i peste cine nu se ridica paza Lui?
\par 4 Cum ar putea un om sa fie fara de prihana înaintea lui Dumnezeu, sau cum ar putea sa fie curat cel ce se na?te din femeie?
\par 5 Iata nici luna nu straluce?te destul în ochii Lui ?i nici stelele nu sunt de tot curate, pentru El!
\par 6 Cu cât mai pu?in omul, care nu este decât putreziciune, cu atât mai pu?in nascutul din om, care nu este decât un vierme!"

\chapter{26}

\par 1 Atunci Iov a raspuns ?i a zis:
\par 2 "În ce chip aju?i tu pe cel ce este fara de putere ?i sprijini bra?ul care a slabit?
\par 3 Cum ?tii tu sa sfatuie?ti pe cel lipsit de în?elepciune ?i ce bel?ug de ?tiin?a ai dat pe fa?a?
\par 4 Catre cine ai îndreptat tu cuvintele tale ?i al cui duh graia prin gura ta?
\par 5 Înaintea lui Dumnezeu, umbrele raposa?ilor tremura sub pamânt, iar apele ?i vieta?ile din ape se înspaimânta.
\par 6 Împara?ia mor?ilor este goala înaintea Lui ?i adâncul este fara acoperi?.
\par 7 El întinde miazanoaptea peste genune; El spânzura pamântul pe nimic.
\par 8 El închide apele în norii Sai ?i norii nu se rup sub greutatea apelor.
\par 9 El acopera fa?a lunii pline, desfa?urând asupra ei norii Sai.
\par 10 El a tras un cerc pe suprafa?a apelor, pâna la hotarul dintre lumina ?i întuneric.
\par 11 Stâlpii cerului se clatina ?i se înspaimânta la mustrarea Lui.
\par 12 Cu puterea Lui El a despicat marea ?i cu în?elepciunea Lui a sfarâmat furia ei.
\par 13 Suflarea Lui însenineaza cerurile ?i mâna Lui strapunge ?arpele fugar!
\par 14 ?i daca acestea sunt marginile din afara ale înfaptuirilor Sale, cât de pu?in lucru este ceea ce strabate pâna la noi! Dar tunetul puterii Sale, cine ar putea sa-l în?eleaga?"

\chapter{27}

\par 1 Dar Iov a mers mai departe cu pildele lui ?i a zis:
\par 2 "Viu este Dumnezeu Care a dat la o parte dreptatea mea! Viu este Cel Atotputernic Care a împovarat sufletul meu!
\par 3 Câta vreme duhul meu va fi întreg în mine ?i suflarea lui Dumnezeu în pieptul meu,
\par 4 Buzele mele nu vor rosti nici un neadevar ?i limba mea nu va grai nici o minciuna!
\par 5 Departe de mine gândul sa va dau dreptate! Pâna când o fi sa-mi dau duhul nu ma voi lepada de nevinova?ia mea.
\par 6 ?in cu tarie la dreptatea mea ?i nu voi lasa-o sa-mi scape; inima mea nu se ru?ineaza de zilele pe care le-am trait.
\par 7 Du?manul meu sa aiba partea nelegiuitului ?i cel ce este împotriva mea sa aiba partea celui ce lucreaza nedreptatea!
\par 8 Care este nadejdea unui înrait, când el se roaga ?i î?i ridica sufletul catre Dumnezeu?
\par 9 Aude oare Dumnezeu strigarea lui, când da peste el vreo nenorocire?
\par 10 Este oare Cel Atotputernic desfatarea lui? Cheama el în toata vremea numele lui Dumnezeu?
\par 11 Voiesc sa va înva? caile lui Dumnezeu ?i ceea ce este în gândul Celui Atotputernic nu vreau sa va ascund.
\par 12 ?i daca voi to?i a?i dovedit-o (ca ?i mine), atunci pentru ce vorbi?i în zadar?
\par 13 Iata partea pe care Dumnezeu o pastreaza celui rau ?i mo?tenirea pe care asupritorii vor primi-o de la Cel Atotputernic.
\par 14 Daca fiii sai sunt numero?i, sunt pentru tai?ul sabiei ?i odraslele lui nu au atâta pâine cât sa se sature:
\par 15 Câ?i mai scapa dintre ai lui vor muri de ciuma ?i vaduvele lor nu-i vor jeli.
\par 16 Daca aduna bani mul?i ca nisipul ?i gramade?te ve?minte multe ca noroiul,
\par 17 Poate sa le gramadeasca, dar cu ele se va îmbraca un om fara prihana ?i de to?i banii lui va avea parte unul cu inima curata.
\par 18 Casa pe care ?i-a zidit-o este casa unei molii ?i ca o coliba pe care ?i-o face un pândar.
\par 19 Se culca bogat, dar nu se mai culca a doua oara; deschide ochii ?i nu mai este.
\par 20 Spaimele l-au ajuns ziua în amiaza mare; în puterea nop?ii, un vârtej l-a smuls.
\par 21 Vântul de la rasarit l-a spulberat ?i se duce; din locul de unde era îl spulbera.
\par 22 Dumnezeu îl împovareaza fara mila ?i înaintea mâinii care îl pedepse?te el cauta sa fuga.
\par 23 Oamenii bat din mâini la priveli?tea aceasta ?i cu fluieraturi îl alunga de peste tot.

\chapter{28}

\par 1 Argintul are zacamintele lui de obâr?ie ?i aurul are locul lui de unde-l sco?i ?i-l lamure?ti.
\par 2 Din pamânt scoatem fierul ?i din stânca topita scoatem arama.
\par 3 Omul a pus hotare întunericului ?i cerceteaza pâna în cele mai departate adâncuri, sfredelind piatra ascunsa în umbra ?i în bezna.
\par 4 Un popor strain a sapat carari pe sub pamânt, uitate de piciorul celor de deasupra ?i departe de oameni; scormonitorii se spânzura pe funii ?i se clatina încoace ?i în colo.
\par 5 ?i deasupra este pamântul din care iese pâinea, dar pe dedesubt este rava?it ca de foc.
\par 6 Aici pietrele lui sunt de safir, dincoace sunt puzderii de aur,
\par 7 Carari pe care nu le-a cunoscut pasarea de prada ?i pe care ochiul vulturului nu ?i le-a însemnat.
\par 8 Fiarele salbatice nu le-au calcat niciodata, niciodata leul nu s-a strecurat pe aici.
\par 9 Dar omul a ajuns cu mâna lui la aceste stânci de cremene ?i mun?ii i-a rasturnat din temelie.
\par 10 El a sapat ?an?uri în stânci ?i nimic de pre? nu scapa privirii lui.
\par 11 El a rascolit izvoarele apelor ?i tot ce era în adâncime a scos afara la lumina.
\par 12 Dar în?elepciunea de unde izvora?te ea ?i care este locul de obâr?ie al priceperii?
\par 13 Pamânteanul nu cunoa?te calea catre ea, caci ea nu se gase?te pe meleagurile celor vii.
\par 14 Adâncul a grait: Ea nu se afla în sânul meu! ?i marea a spus la fel: Ea nu este la mine!
\par 15 Mintea cea înalta nu poate fi schimbata cu bulgari de aur ?i argintul nu-l cântare?ti ca s-o plate?ti.
\par 16 Ea nu poate sa fie pre?uita nici cu aurul Ofirului, nici cu pre?ioasa cornalina, nici cu pietre de safir!
\par 17 Cu ea alaturi nu pot sa stea nici aurul, nici cristalul ?i cu un vas din aurul cel mai curat nu se poate schimba ea.
\par 18 Despre margean ?i despre diamant, nici sa mai pomenim, iar agonisirea în?elepciunii întrece cu mult pe aceea a margaritarelor.
\par 19 Topazele Etiopiei nu stau în cumpana cu ea ?i cu aurul cel mai curat nu vei plati-o niciodata!
\par 20 ?i aceasta în?elepciune de unde vine ea ?i care este sala?ul priceperii?
\par 21 Ea a fost ascunsa de ochii oricarei fapturi vii; ea a fost tainuita ?i de pasarea cerului.
\par 22 Adâncul ?i moartea au zis: Noi am auzit vorbindu-se de ea.
\par 23 Dumnezeu îi cunoa?te drumul ?i numai El este Cel ce ?tie locuin?a ei.
\par 24 Când El privea pâna la marginile pamântului ?i îmbra?i?a cu ochii tot ce se afla sub ceruri,
\par 25 Ca sa dea vântului cumpana ?i sa chibzuiasca legea apelor,
\par 26 Când El statornicea ploilor un faga? ?i o cale bubuitului tunetului,
\par 27 Atunci El a vazut în?elepciunea ? i a cântarit-o, atunci a pus-o în lumina ?i i-a masurat adâncimea.
\par 28 Dupa aceea Dumnezeu a zis omului: Iata, frica de Dumnezeu, aceasta este în?elepciunea, iar în departarea de cel rau sta priceperea".

\chapter{29}

\par 1 Apoi Iov a mers mai departe cu pildele sale ?i a zis:
\par 2 "O, daca a? fi înca o data ca în lunile de mai înainte, ca în zilele când Dumnezeu ma ocrotea,
\par 3 Ca atunci când El ?inea stralucitoare deasupra capului meu candela Sa ?i, luminat de ea, eu strabateam prin întuneric!
\par 4 De ce nu sunt înca o data ca în zilele toamnei mele, când Dumnezeu ?inea parte cortului meu,
\par 5 Când Cel Atotputernic era înca cu mine ?i împrejurul meu stateau feciorii mei,
\par 6 Iar picioarele mele se scaldau în lapte ?i stânca aspra izvora pentru mine pâraie de untdelemn?
\par 7 Atunci când ie?eam la poarta de sus a ceta?ii ?i a?ezam în pia?a scaunul meu,
\par 8 Tineretul, vazându-ma, se ascundea cu sfiala, iar cei batrâni se ridicau în picioare ?i ramâneau a?a.
\par 9 Frunta?ii poporului î?i opreau cuvântarile ?i î?i puneau mâna la gura.
\par 10 Glasul capeteniilor scadea ?i limba lor se lipea de cerul gurii.
\par 11 Caci urechea care ma auzea ma fericea ?i ochiul care ma vedea îmi dadea mare marturie.
\par 12 Fiindca scapam de pieire pe cel sarman care striga dupa ajutor ?i pe orfanul fara sprijin.
\par 13 Binecuvântarile celui ce era gata sa piara veneau asupra-mi ?i umpleam de bucurie inima vaduvei.
\par 14 Ma îmbracam întru dreptate, ca într-un ve?mânt ?i judecata mea cea dreapta era mantia mea ?i turbanul meu.
\par 15 Eram ochii celui orb ?i piciorul celui ?chiop;
\par 16 Eram tatal celor neputincio?i ?i cercetam cu sârguin?a pricinile care îmi erau necunoscute.
\par 17 Sfarâmam falcile nelegiuitului ?i smulgeam prada din din?ii lui.
\par 18 ?i îmi ziceam: Voi adormi în cuibul meu ?i ca pasarea Phoenix voi înmul?i zilele mele.
\par 19 Radacina mea se va rasfira pe lânga apa ?i roua se va lasa, noaptea, peste ramurile mele.
\par 20 Slava mea va întineri neîncetat ?i arcul meu se va înnoi în mâna mea.
\par 21 Oamenii ma ascultau ?i stateau fara grai ?i a?teptau sa auda sfatul meu.
\par 22 Dupa ce le vorbeam eu, ei nu mai spuneau nimic ?i cuvântul meu cadea asupra lor picatura cu picatura.
\par 23 Ma a?teptau precum a?tep?i ploaia ?i cascau gura lor, ca pentru bura de primavara.
\par 24 Daca le surâdeam, nu-?i credeau ochilor ?i surâsul meu nu-l lasau sa se piarda.
\par 25 Le aratam care este dreapta cale ?i stateam mereu în fruntea lor, stateam ca un împarat, între osta?ii sai ?i, oriunde-i duceam, ei veneau dupa mine.

\chapter{30}

\par 1 Iar acum am ajuns de batjocura pentru cei mai tineri decât mine ?i pe ai caror parin?i îi pre?uiam prea pu?in, ca sa-i pun alaturi cu câinii turmelor mele.
\par 2 Ce a? fi facut cu puterea bra?elor lor, odata ce vlaga lor se dusese toata?
\par 3 Din pricina saraciei ?i a foametei înspaimântatoare, ei mân?cau radacini din locuri uscate ?i mama lor era câmpia pustie ?i jalnica.
\par 4 Ei culegeau ierburi de prin maracini ?i pâinea lor era radacina de ienupar.
\par 5 Erau goni?i din mijlocul oamenilor ?i dupa ei lumea urla ca dupa ni?te ho?i.
\par 6 Drept aceea, au ajuns sa se aciueze pe marginea ?uvoaielor, prin gaurile pamântului ?i prin vagaunile stâncilor.
\par 7 Zbiara prin ha?i?uri, stau gramada pe sub scaie?i.
\par 8 Neam de oameni ticalo?i, neam de oameni fara nume, ei erau gunoaiele pe care le arunci din ?ara!
\par 9 ?i astazi, iata ca sunt cântecul lor, am ajuns basmul lor.
\par 10 Le e groaza de mine, s-au departat de mine ?i pentru obrazul meu n-au facut economie cu scuipatul lor!
\par 11 Cel ce ?i-a deznodat ?treangul robiei ma asupre?te ?i tot a?a cel ce ?i-a scos zabalele din gura.
\par 12 În dreapta mea se ridica martori potrivnici mie, în cursa lor au prins picioarele mele ?i ?i-au croit drumuri împotriva-mi.
\par 13 Au darâmat poteca mea, cu gând ca sa ma piarda, ei se suie încoace ?i nimeni nu le este stavila.
\par 14 Ca printr-o spartura larga, ei dau iure? ?i în darâmaturi se tavalesc.
\par 15 Mul?imea spaimelor s-a întors asupra mea, slava mea au gonit-o ca vântul ?i izbavirea mea a trecut ca un nor.
\par 16 ?i acum sufletul meu se tope?te în mine, zile de amaraciune ma cuprind.
\par 17 Noaptea oasele mele sunt ca sfredelite ?i nervii mei nu ?tiu de odihna.
\par 18 Cu o putere napraznica, Dumnezeu ma tine de haina ?i ma strânge de gât ca gulerul cama?ii.
\par 19 Mi-a dat brânci în noroi ?i am ajuns sa fiu la fel cu praful ?i cu cenu?a.
\par 20 Strig catre Tine ?i nu-mi raspunzi, stau în picioare ?i Tu nu ma vezi.
\par 21 Tu Te-ai facut asupritorul meu ?i cu toata puterea bra?ului Tau ma prigone?ti.
\par 22 Tu ma ridici deasupra vântului ?i ma pui pe el calare ?i apoi ma nimice?ti cu iure?ul furtunii.
\par 23 ?tiu foarte bine ca Tu ma duci spre moarte ?i la locul de întâlnire al tuturor muritorilor.
\par 24 Totu?i împotriva sarmanului nu ridicam mâna mea, când striga catre mine, în nenorocirea lui.
\par 25 N-am plâns oare ?i eu împreuna cu cel care-?i ducea via?a greu? Sufletul meu n-avea mila de cel sarman?
\par 26 Ma a?teptam la fericire ?i iata ca a venit nenorocirea; a?teptam lumina ?i a venit întunericul.
\par 27 Maruntaiele mele au fiert în clocote fara încetare; zile de jale grea mi-au sosit înainte.
\par 28 Am umblat înnegrit la fa?a, dar nu de soare; m-am ridicat în adunare ?i am strigat.
\par 29 Am ajuns frate cu ?acalii, am ajuns tovara? cu stru?ii.
\par 30 Pielea s-a facut pe mine neagra ?i oasele mele sunt arse de friguri.
\par 31 Astfel harfa mea a ajuns instrument tânguirii ?i flautul meu glasul bocitoarelor.

\chapter{31}

\par 1 Facusem legamânt cu ochii mei ?i asupra unei fecioare nu-i ridicam.
\par 2 ?i care este partea pe care Dumnezeu o trimite din ceruri ?i ce câ?tig haraze?te, din înal?ime, Cel Atotputernic?
\par 3 Nefericirea nu este ea oare pentru cel nedrept ?i nenorocirea pentru faptuitorii faradelegii?
\par 4 Nu vede, oare, Dumnezeu caile mele ?i nu numara El to?i pa?ii mei?
\par 5 Umblat-am oare întru minciuna ?i picioarele mele au zorit spre în?elaciune?
\par 6 Sa ma cântareasca în cumpana drepta?ii ?i Dumnezeu sa cunoasca neprihanirea mea.
\par 7 Daca pa?ii mei s-au abatut de la calea cea dreapta ?i inima mea a fost târâta de ochii mei, iar de mâinile mele s-a lipit vreo murdarie,
\par 8 Atunci altul sa manânce ceea ce eu seman ?i vlastarii mei sa fie sco?i din radacina!
\par 9 Daca inima mea a fost amagita de vreo femeie ?i am stat de pânda la u?a aproapelui meu,
\par 10 Atunci nevasta mea sa învârteasca la râ?ni?a pentru altul ?i al?ii sa aiba parte de ea.
\par 11 Caci aceasta ar fi o urâciune, o nelegiuire vrednica de pedeapsa judecatorilor,
\par 12 Un foc care mistuie pâna la iad ?i care nimice?te toata strânsura mea;
\par 13 Daca a? fi nesocotit dreptul slugii sau al slujnicei mele, în socotelile lor cu mine,
\par 14 Ce ma voi face eu; când Dumnezeu se va ridica ?i ce raspuns ti voi da, când va lua procesul în cercetare?
\par 15 Cel ce m-a facut pe mine în pântecele mamei mele nu l-a facut ?i pe robul meu? Nu este, oare, El singur Care ne-a alcatuit în pântece?
\par 16 Datu-m-am, oare, în laturi, când saracul dorea ceva ?i lasat-am sa se stinga de plânsete ochii vaduvelor?
\par 17 Mâncam, oare, singur bucata mea de pâine ?i orfanului nu-i dadeam din ea?
\par 18 Dimpotriva, din tinere?ile mele, am crescut pe orfan ca un tata ?i de la na?tere, am calauzit pe vaduva.
\par 19 Daca vedeam un nenorocit fara haina ?i vreun sarac care n-avea cama?a pe el,
\par 20 Nu ma binecuvântau coapsele lui ?i nu-l încalzea lâna mieilor mei?
\par 21 Daca am repezit mâna mea împotriva vreunui orfan, fiindca vedeam ca am sprijinitori la masa judeca?ii,
\par 22 Atunci sa cada umarul meu din încheietura ?i bra?ul meu sa se dezlege de osul celalalt!
\par 23 Dar eu ma temeam de pedeapsa lui Dumnezeu ?i înaintea mare?iei Lui nu puteam sa stau.
\par 24 Mi-am pus eu încrederea în aur sau am zis aurului lamurit: Tu e?ti nadejdea mea?
\par 25 Ori eram fericit peste masura, ca aveam atâta avere ?i ca mâna mea agonisise mult?
\par 26 Ori când vedeam soarele în stralucirea lui ?i luna înaintând cu mare?ie,
\par 27 A fost inima mea amagita în taina ?i am dus eu mâna la gura, ca s-o sarut?
\par 28 ?i aceasta ar fi fost o mare faradelege, fiindca a? fi tagaduit pe Dumnezeul cel Preaînalt.
\par 29 M-am bucurat eu de nenorocirea du?manului meu ?i am tresaltat când vreo rautate daduse peste el?
\par 30 Eu n-am îngaduit limbii mele sa gre?easca ?i sa ceara moartea du?manului, blestemându-l.
\par 31 Oamenii care ?ineau de casa mea ziceau: "Unde s-ar gasi vreunul care sa nu se fi saturat la masa lui?"
\par 32 Strainul nu petrecea noaptea niciodata afara; por?ile mele le deschideam calatorului.
\par 33 Acoperit-am eu, ca lumea cealalta, pacatele mele, ascunzând, în sânul meu, gre?eala faptuita,
\par 34 Pentru ca, adica, ma temeam de zarva ceta?ii ?i ma înspaimânta dispre?ul ceta?enilor ?i atunci ramâneam fara glas ?i nu mai ma aratam în poarta?
\par 35 O, cine, îmi va da pe cineva care sa ma asculte? Iata aici iscalitura mea! Cel Atotputernic sa-mi raspunda! Iar învinuirea scrisa de potrivnicii mei,
\par 36 Voi purta-o pe umarul meu, voi înnoda-o în jurul capului meu, ca o cununa.
\par 37 Îi voi da socoteala de to?i pa?ii mei, ca un principe ma voi înfa?i?a înaintea Lui.
\par 38 Nu cumva ogoarele mele cer razbunare împotriva mea ?i brazdele lor sunt prididite de lacrimi?
\par 39 Nu cumva m-am înfruptat din roadele lor ?i n-am platit ?i am facut pe vechii lor stapâni sa se plânga de mine?
\par 40 Daca ar fi a?a, atunci sa creasca pe ele palamida în loc de grâu ?i neghina în loc de orz!" Aici cuvintele lui Iov se termina.

\chapter{32}

\par 1 Astfel, ace?ti trei barba?i nu mai raspunsera nimic lui Iov, pentru ca el se socotea fara vina.
\par 2 Atunci se aprinse de mânie Elihu, fiul lui Baracheel din Buz, din familia lui Ram. ?i mânia lui se aprinse împotriva lui Iov, fiindca el pretindea ca este drept înaintea lui Dumnezeu,
\par 3 ?i iara?i se aprinse mânia lui împotriva celor trei prieteni ai lui Iov, fiindca ei nu gaseau nici un raspuns ?i totu?i osândeau pe Iov.
\par 4 Elihu însa a?teptase pe când ei vorbeau cu Iov, fiindca ei erau mai în vârsta decât Elihu.
\par 5 Dar când a vazut el ca nu mai este nici un raspuns în gura celor trei oameni, atunci s-a aprins mânia lui.
\par 6 ?i a?a Elihu, fiul lui Baracheel din Buz, a început a vorbi ?i a zis: "Eu sunt tânar ?i voi sunte?i batrâni, de aceea m-am sfiit ?i m-am temut sa va dau pe fa?a gândul meu.
\par 7 Mi-am zis: vârsta trebuie sa vorbeasca ?i mul?imea anilor sa ne înve?e în?elepciunea.
\par 8 Dar duhul din om ?i suflarea Celui Atotputernic dau priceperea.
\par 9 Nu cei batrâni sunt în?elep?i ?i nici mo?negii nu sunt cei ce în?eleg totdeauna dreptatea.
\par 10 Drept aceea am zis: Lua?i aminte la mine, voi arata ?i eu ce ?tiu.
\par 11 Iata ca am a?teptat cuvintele voastre, am stat cu urechea a?intita la judeca?ile voastre, pe când voi va cauta?i ce avea?i de spus.
\par 12 Am stat cu ochii a?inti?i asupra voastra ?i iata ca nici unul n-a convins pe Iov, nici unul n-a rasturnat cuvintele lui;
\par 13 De aceea sa nu zice?i: Noi am gasit în?elepciunea ?i Dumnezeu ne da înva?atura, iar nu un om.
\par 14 Astfel, nu voi pune înainte ni?te cuvinte ca acestea ?i nu-i voi raspunde cu temeiurile voastre.
\par 15 Ei au fost opari?i, n-au mai raspuns nimic, cuvintele le-au fugit din gura
\par 16 ?i eu am a?teptat! Dar pentru ca ei nu mai vorbesc, fiindca au stat pe loc ?i nu mai raspund,
\par 17 Voi zice ?i eu ceva din partea mea, voi arata ?i eu ?tiin?a mea,
\par 18 Caci sunt plin de cuvinte pâna în gât ?i duhul meu launtric îmi da zor.
\par 19 Iata ca cugetul meu în mine este ca un vin care n-are pe unde sa rasufle, ca un vin care sparge ni?te burdufuri noi.
\par 20 Voi vorbi deci ca sa ma u?urez, voi deschide gura mea ?i nu-l voi lasa pe Iov fara raspuns.
\par 21 Nu voi lua partea nimanui ?i nu voi maguli pe nimeni,
\par 22 Caci nu ma pricep sa lingu?esc, altfel într-o clipeala m-ar smulge Ziditorul meu.

\chapter{33}

\par 1 Drept aceea, Iov, te rog, asculta cuvintele mele ?i ia aminte la toate cuvintele mele.
\par 2 Iata ca am deschis gura mea ?i limba mea graie?te.
\par 3 Inima mea va scoate la iveala cuvinte de înva?atura, buzele mele se vor rosti cu limpezime,
\par 4 Duhul lui Dumnezeu este Cel ce m-a facut ?i suflarea Celui Atotputernic este datatoarea vie?ii mele.
\par 5 Daca po?i, raspunde-mi, apara-?i pricina înaintea mea, fii tare!
\par 6 Înaintea lui Dumnezeu eu sunt la fel cu tine ?i eu ca ?i tine am fost framântat din lut,
\par 7 De aceea frica de mine sa nu te tulbure, nici mâna mea sa nu atârne greu asupra ta.
\par 8 Tu ai spus în auzul meu ?i eu am auzit rostul vorbelor tale spunând a?a:
\par 9 "Eu sunt curat ?i fara nici o vina, eu sunt fara prihana ?i n-am nici o gre?eala;
\par 10 Dar iata ca Dumnezeu cauta pricina de ura împotriva mea ?i ma socote?te ca un vrajma? al Lui.
\par 11 El pune picioarele mele în butuci ?i pânde?te to?i pa?ii mei!"
\par 12 Dar aici î?i voi raspunde ca tu n-ai dreptate, fiindca Dumnezeu este mai mare decât omul.
\par 13 De ce graie?ti împotriva Lui, fiindca El nu da nimanui socoteala de toate câte face?
\par 14 Vezi ca Dumnezeu vorbe?te când într-un fel, când într-alt fel, dar omul nu ia aminte.
\par 15 ?i anume, El vorbe?te în vis, în vedeniile nop?ii, atunci când somnul se lasa peste oameni ?i când ei dorm în a?ternutul lor.
\par 16 Atunci El da în?tiin?ari oamenilor ?i-i cutremura cu aratarile Sale.
\par 17 Ca sa întoarca pe om de la cele rele ?i sa-l fereasca de mândrie
\par 18 Ca sa-i fereasca sufletul de prapastie ?i via?a lui de calea mormântului;
\par 19 De aceea, prin durere, omul este mustrat în patul lui ?i oasele lui sunt zguduite de un cutremur neîntrerupt.
\par 20 Pofta lui este dezgustata de mâncare ?i inima lui nu mai pofte?te nici cele mai bune bucate.
\par 21 Carnea dupa el se prapade?te ?i piere ?i oasele lui, pâna acum nevazute, îi ies prin piele.
\par 22 Sufletul lui vine încet, încet spre prapastie ?i via?a lui spre împara?ia mor?ilor.
\par 23 Daca atunci se afla un înger lânga el, un mijlocitor între vii, care sa-i arate omului calea datoriei,
\par 24 Dumnezeu Se milostive?te de el ?i zice îngerului: "Izbave?te-l ca sa nu cada în prapastie; am gasit pentru sufletul lui pre?ul de rascumparare!"
\par 25 Atunci trupul lui înflore?te ca în tinere?e ?i el vine înapoi la zilele de la începutul vie?ii sale.
\par 26 El se roaga lui Dumnezeu ?i Dumnezeu îi arata bunatatea Sa ?i-i îngaduie sa vada fa?a Sa cu mare bucurie ?i astfel îi da omului iertarea Sa.
\par 27 Atunci omul prive?te peste semenii sai ?i zice: "Pacatuisem ?i calcasem dreptatea, dar n-am fost pedepsit dupa faptele mele.
\par 28 Caci El a izbavit sufletul meu ca sa nu treaca prin strâmtorile mor?ii ?i ochii mei vad înca lumina".
\par 29 Iata toate acestea le face Dumnezeu de doua ori, de trei ori cu omul,
\par 30 Ca sa-i scoata sufletul din pieire ?i ca sa-l lumineze cu lumina celor vii.
\par 31 Ia aminte Iov, asculta-ma pe mine, taci ?i eu voi vorbi!
\par 32 Daca ai ceva de spus da-mi raspuns, vorbe?te, caci dorin?a mea este sa-?i dau dreptate.
\par 33 Iar daca nu, asculta la mine: ?ine-?i gura ?i te voi înva?a care este în?elepciunea".

\chapter{34}

\par 1 Elihu a vorbit mai departe ?i a zis:
\par 2 "Asculta?i, în?elep?ilor, cuvintele mele ?i voi, înva?a?ilor, a?inti?i-va urechile,
\par 3 Fiindca urechea deosebe?te cuvintele, precum cerul gurii gusta rnâncarea.
\par 4 Sa cercetam între noi ce este drept, sa ?tim între noi ceea ce este bine,
\par 5 Fiindca Iov a zis: "Eu sunt drept, dar Dumnezeu nu-mi da dreptate!
\par 6 De?i nevinovat, trec drept mincinos; rana mea este nevindecata, de?i eu nu am nici o gre?eala".
\par 7 Cine mai este ca Iov, care sa bea batjocura, cum ar bea apa?
\par 8 Care sa se înso?easca cu cei care fac nedreptate ?i sa mearga în pas cu facatorii de rele?
\par 9 Caci Iov a zis: "Omul n-are nici un folos, daca se straduie?te sa fie placut lui Dumnezeu".
\par 10 Dar voi oameni de inima, asculta?i-ma! Departe este de Dumnezeu rautatea, departe este de El nedreptatea!
\par 11 Caci Dumnezeu întoarce omului dupa faptele lui ?i se poarta cu fiecare dupa purtarea lui.
\par 12 Cu adevarat, Dumnezeu nu faptuie?te raul ?i Cel Atotputernic nu strâmba dreptatea.
\par 13 Cine i-a încredin?at cârmuirea pamântului ?i cine i-a dat în grija aceasta lume întreaga?
\par 14 Daca Dumnezeu n-ar cugeta decât la Sine Însu?i ?i daca ar lua înapoi la Sine duhul Sau ?i suflarea Sa,
\par 15 Toate fapturile ar pieri deodata ?i omul s-ar întoarce în ?arâna.
\par 16 Daca ai minte, asculta aceasta, pleaca urechea la cuvintele mele.
\par 17 Unul care prigone?te dreptatea ar putea oare sa domneasca? ?i vei osândi tu pe Cel mare ?i drept?
\par 18 El, Care striga împara?ilor: Netrebnicilor! ?i celor mai mari de pe pamânt: Nelegiui?ilor!
\par 19 El nu cauta la fala celor mari ?i nu face deosebire între bogat ?i sarac, pentru ca to?i sunt lucrarea mâinilor Sale.
\par 20 Într-o clipita ei mor ?i se duc; în miez de noapte, un popor se zbuciuma ?i fara greutate prabu?e?te pe tiran.
\par 21 Pentru ca ochii Domnului supravegheaza cararile omului ?i vede to?i pa?ii lui.
\par 22 Pentru El nu este nici întuneric, nici umbra, unde sa se poata ascunde cei ce lucreaza nelegiuirea.
\par 23 Dumnezeu n-are nevoie sa priveasca multa vreme pe cineva, ca sa-l traga înaintea judeca?ii Sale.
\par 24 El zdrobe?te pe puternici, fara lunga cercetare ?i pune pe al?ii în locul lor.
\par 25 De vreme ce El cunoa?te faptele lor, El îi rastoarna în fapt de noapte ?i-i zdrobe?te.
\par 26 Ca pe ni?te nelegiui?i ce sunt, El îi love?te de fala cu foarte mul?i privitori,
\par 27 Fiindca s-au dat la o parte din preajma Sa ?i n-au voit sa priceapa cararile Sale
\par 28 ?i au facut sa urce pâna la Domnul strigatul celui sarac ?i sa rasune în urechile Sale plânsul celor nenoroci?i.
\par 29 Daca Domnul se odihne?te, cine poate sa-L smulga din odihna Lui ?i daca Î?i acopera fa?a, cine poate sa-L mai zareasca? Dar El sta ?i supravegheaza ?i pe popoare ?i pe oameni,
\par 30 Ca unul Care nu voie?te stapânirea celor nelegiui?i, nici poticnirea popoarelor.
\par 31 Daca un fa?arnic zice lui Dumnezeu: "Am fost târât la pacat ?i nu voi mai face ce este rau,
\par 32 Ceea ce nu ?tiu, Tu înva?a-ma; daca am savâr?it vreo nedreptate nu voi porni iar de la capat!"
\par 33 Crezi tu, dupa parerea ta, ca Dumnezeu îi va face în schimb tot a?a? Fiindca ai fost dispre?uitor, fiindca te faci tu judecator în locul meu, spune-mi atunci ce ?tii tu?
\par 34 Oamenii în toata firea vor zice ?i tot a?a orice om cuminte care ma asculta:
\par 35 Iov nu vorbe?te dupa dreapta înva?atura ?i cuvintele lui nu sunt dupa sfânta dreptate.
\par 36 Însa Iov trebuie cercetat pâna la capat cu privire la acele raspunsuri vrednice de ni?te nelegiui?i.
\par 37 El a sporit pacatul sau; aici între noi el pune la îndoiala gre?eala lui ?i îngramade?te vorbele sale împotriva lui Dumnezeu".

\chapter{35}

\par 1 Elihu a vorbit mai departe ?i a zis:
\par 2 "Crezi tu ca ai dreptate ?i socote?ti ca te-ai limpezit înaintea lui Dumnezeu,
\par 3 Când zici: "Ce folosesc, ce câ?tig am eu, ca nu pacatuiesc?"
\par 4 Iata ce-?i voi raspunde ?i ?ie ?i prietenilor tai:
\par 5 Prive?te cerurile ?i îndreapta intr-acolo ochii; uita-te la nori, cât sunt ei de sus, fa?a de tine!
\par 6 Daca pacatuie?ti, ce rau îi faci lui Dumnezeu ?i daca pacatele tale sunt numeroase, ce-I strica Lui?
\par 7 Daca e?ti drept, ce dar Îi faci sau ce prime?te El din mâna ta?
\par 8 Rautatea ta poate sa strice unui om ca ?i tine, dreptatea ta sa foloseasca celui ce este ca ?i tine nascut din om.
\par 9 Ei striga atunci când împilarea a trecut orice margini, ei racnesc în mâinile celor puternici.
\par 10 Dar ei nu întreaba: Unde este Dumnezeu Cel ce ne-a facut, El Care daruie?te nop?ii cântari de veselie?
\par 11 El Care ne da mai multa în?elepciune decât dobitoacelor pamântului ?i mai multa pricepere decât pasarilor cerului?
\par 12 Sa tot strige ei atunci, caci Dumnezeu nu raspunde, din pricina trufa?ei împilari a celor rai.
\par 13 Zadarnica le este truda; Dumnezeu nu aude ?i Cel Atotputernic nu ia aminte.
\par 14 Cu atât mai pu?in, când tu zici ca nu ?tii de unde sa-L iei, ca tu e?ti cu El în judecata ?i ca-L tot a?tep?i sa vina.
\par 15 Ba, înca atunci când tu spui ca mânia Lui nu pedepse?te ?i ca El nu prea ?tie limpede ce este aceea nelegiuire!
\par 16 Da, Iov î?i deschide gura zadarnic ?i, ne?tiind ce spune, înmul?e?te cuvintele fara rost.

\chapter{36}

\par 1 Elihu a mers mai departe ?i a grait:
\par 2 "A?teapta o clipa ?i vei înva?a ?i altele, caci sunt înca temeiuri ?i cuvinte de partea lui Dumnezeu.
\par 3 Voi porni cu ?tiin?a mea de departe ?i voi dovedi dreptatea Ziditorului meu.
\par 4 Caci cu adevarat ceea ce-?i spun eu nu este minciuna ?i cal ce sta lânga tine este unul desavâr?it în cuno?tin?a.
\par 5 Fire?te, Dumnezeu este prea puternic, dar nu leapada pe nimeni; El este prea puternic prin înal?imea în?elepciunii Sale.
\par 6 El nu lasa pe nelegiuit sa propa?easca ?i celor nenoroci?i le face dreptate.
\par 7 El nu despoaie pe cei drep?i de dreptatea lor, iar cu împara?ii la fel: îi pune în je?uri împarate?ti ?i-i a?aza sa domneasca de-a pururi. Dar ei se umfla de trufie.
\par 8 ?i atunci iata-i fereca?i cu lan?uri ?i iata-i prin?i cu funiile mâhnirii.
\par 9 Dupa aceea, Dumnezeu le dezvaluie fapta pe care au facut-o ?i nelegiuirea în care au cazut, anume ca s-au trufit.
\par 10 Dar El le face aceasta destainuire ca sa ia aminte ?i le da porunca sa se întoarca de la rautatea lor;
\par 11 Daca dau ascultare ?i vin la supunere, ei î?i ispravesc zilele lor în fericire ?i anii lor în desfatari;
\par 12 Iar daca sunt neascultatori, atunci trec prin strâmtorile mor?ii ?i se sting nepricepu?i ?i orbi.
\par 13 Nelegiui?ii se mânie; ei nu se roaga lui Dumnezeu, când sunt pu?i în lan?uri.
\par 14 Unii ca ace?tia se sting de tineri ?i via?a lor se ve?teje?te în floare.
\par 15 Dar pe cel nenorocit Dumnezeu îl scapa prin nenorocirea lui ?i prin suferin?a Dumnezeu îi da înva?atura.
\par 16 Tot a?a ?i pe tine te va scoate din strânsoarea durerii, ca sa te puna la loc larg, unde nu mai este nici o stinghereala ?i unde masa ta va fi încarcata cu mâncari grase ?i alese.
\par 17 Daca tu ai fost pedepsit cu stra?nicie, ca un nelegiuit, tu scoate din pedeapsa puterea drepta?ii;
\par 18 Certarea Lui sa nu te împinga la mânie împotriva Lui ?i mul?imea bataii sa nu te scoata din calea cugetului drept.
\par 19 Era oare sa puna Dumnezeu vreun pre? pe boga?iile tale? Nu! Nici pe aur, nici pe toate mijloacele puterii pamânte?ti.
\par 20 Nu pofti noaptea (departarii de Dumnezeu), caci în ea, popoare întregi au fost smulse din locul lor.
\par 21 Ia seama, nu te duce la nedreptate, caci ea este adevarata cauza a suferin?ei.
\par 22 Da, Dumnezeu este nespus de mare prin puterea Lui! Cine poate sa înve?e ca El?
\par 23 Cine I-a dat înva?atura cum sa se poarte? ?i cine poate sa-I spuna: "Aceasta ai facut-o rau?"
\par 24 Adu-?i aminte ?i preamare?te opera Lui, pe care o cânta, în laudele lor, oamenii;
\par 25 Orice om o prive?te, macar ca o îmbra?i?eaza cu ochiul, numai de departe.
\par 26 Cât este de mare Dumnezeu! Dar noi nu putem sa-L în?elegem ?i numarul anilor Sai nu se poate socoti.
\par 27 El atrage picaturile de apa, El le preface în aburi ?i da ploaia.
\par 28 Iar norii o trec prin sita lor ?i o varsa picaturi peste mul?imile omene?ti.
\par 29 Cine poate sa priceapa cum se desfa?oara norii ?i cum bubuie tunetul în cortul lui?
\par 30 Iata ca El a rostogolit aburii Sai ?i a acoperit adâncimile marii.
\par 31 Prin el Domnul hrane?te popoarele ?i le da bel?ug de mâncare.
\par 32 El ridica fulgerul, cu amândoua mâinile ?i-l trimite sa loveasca la ?inta.
\par 33 El da din vreme de veste ciobanului ?i oilor, care simt din aer apropierea vijeliei.

\chapter{37}

\par 1 ?i din pricina aceasta inima mea se zbuciuma ?i se zbate din locul ei.
\par 2 Asculta?i bubuitul glasului Sau ?i tunetul care iese din gura Sa.
\par 3 Peste toata întinderea cerului El azvârle fulgerul Sau ?i fulgerul Sau ajunge pâna la marginile pamântului.
\par 4 În urma fulgerului, vine un muget prelung. El tuna cu glasul Lui zguduitor, El nu mai împiedica fulgerele cât timp glasul Lui rasuna.
\par 5 Dumnezeu cu tunetul Sau savâr?e?te minuni, El face lucruri mari pe care noi nu putem sa le pricepem.
\par 6 El porunce?te zapezii: "Cazi pe pamânt ", ?i ploilor îmbel?ugate: "Starui?i cu putere!"
\par 7 Pe fiecare om El pune a Sa pecetie, pentru ca to?i oamenii sa recunoasca puterea Lui.
\par 8 Fiarele salbatice se dau înapoi în culcu?urile lor ?i ramân ascunse în vizuinile lor.
\par 9 Vijelia vine de la miazazi ?i frigul vine de la miazanoapte.
\par 10 La suflarea lui Dumnezeu se încheaga ghia?a ?i întinderea apelor se face sloi.
\par 11 El umple norii cu apa ?i din întunecimea furtunii sloboade fulgerele.
\par 12 Iar norii, învârtindu-se în cercuri, alearga dupa planurile Sale, astfel ca îndeplinesc tot ce le porunce?te, în lungul ?i în latul lumii Sale pamânte?ti.
\par 13 ?i Domnul îi trimite: aici ca o bataie pentru pamânt, dincolo ca o milostivire a voin?ei Sale.
\par 14 Iov, ia aminte la aceste lucruri, stai locului ?i te uita la minunile lui Dumnezeu!
\par 15 În?elegi tu cum cârmuie?te Dumnezeu norii Sai ?i în ce fel poate norul sa sloboada fulgerul pe pamânt?
\par 16 În?elegi tu plutirea norilor, minuni ale Aceluia a Carui ?tiin?a este desavâr?ita?
\par 17 Tu, care te aprinzi în ve?mintele tale, când pamântul se odihne?te sub vântul arzator din miazazi,
\par 18 Po?i sa întinzi la fel cu El boltitura cerului, ca o oglinda turnata din metal?
\par 19 Spune-mi ?i mie ce vom putea sa graim cu El? Ce vorba vom începe noi cu El, astfel întuneca?i la minte precum suntem?
\par 20 Acum, când eu vorbesc, cine-I da de veste ce zic eu? Când a vorbit cineva ceva, El o ?tie fiindca I-a Spus altul?
\par 21 Oamenii nu pot sa priveasca prealuminosul soare, care straluce?te pe cer, acum dupa ce vântul a împra?tiat norii.
\par 22 Acum lumina biruitoare se revarsa din norii de la miazanoapte ?i mare?ia Domnului robe?te ?i cutremura inima.
\par 23 Pe Cel Atotputernic nu putem sa-L ajungem cu priceperea noastra. El este atotînalt în putere ?i bogat în judecata ?i nu calca niciodata dreptatea în picioare.
\par 24 Pentru aceea oamenii se tem de El ?i I se închina; El însa nu-?i pogoara privirile asupra nici unuia dintre cei ce se cred pe sine în?elep?i".

\chapter{38}

\par 1 Atunci Dumnezeu i-a raspuns lui Iov, din sânul vijeliei, ?i i-a zis:
\par 2 "Cine este cel ce pune pronia sub obroc, prin cuvinte fara în?elepciune?
\par 3 Încinge-?i deci coapsele ca un viteaz ?i Eu te voi întreba ?i tu Îmi vei da lamuriri!
\par 4 Unde erai tu, când am întemeiat pamântul? Spune-Mi, daca ?tii sa spui.
\par 5 ?tii tu cine a hotarât masurile pamântului sau cine a întins deasupra lui lan?ul de masurat?
\par 6 În ce au fost întarite temeliile lui sau cine a pus piatra lui cea din capul unghiului,
\par 7 Atunci când stelele dimine?ii cântau laolalta ?i to?i îngerii lui Dumnezeu Ma sarbatoreau?
\par 8 Cine a închis marea cu por?i, când ea ie?ea navalnica, din sânul firii,
\par 9 ?i când i-am dat ca ve?mânt negura ?i norii drept scutece,
\par 10 Apoi i-am hotarnicit hotarul Meu ?i i-am pus por?i ?i zavoare
\par 11 ?i am zis: Pâna aici vei veni ?i mai departe nu te vei întinde, aici se va sfarâma trufia valurilor tale?
\par 12 Ai poruncit tu dimine?ii, vreodata în via?a ta, ?i i-ai aratat aurorei care este locul ei,
\par 13 Ca sa apuce pamântul de col?uri ?i sa scuture pe nelegiui?i de pe pamânt?
\par 14 În revarsatul zorilor, pamântul se face ro?u ca ro?iile pece?i ?i ia culori ca de ve?mânt.
\par 15 Cei rai ramân fara noaptea (prielnica lor) ?i bra?ul ridicat este frânt.
\par 16 Ai fost tu pâna la izvoarele marii sau te-ai plimbat pe fundul prapastiei?
\par 17 ?i s-au aratat oare por?ile mor?ii ?i por?ile umbrei le-ai vazut?
\par 18 Ai cugetat oare la întinderea pamântului? Spune, ?tii toate acestea?
\par 19 Care drum duce la palatul luminii ?i care este locul întunericului,
\par 20 Ca sa ?tii sa-l calauze?ti în cuprinsul lui ?i sa po?i sa nimere?ti potecile care duc la el acasa?
\par 21 Tu ?tii bine, caci atunci erai nascut ?i numarul zilelor tale e foarte mare.
\par 22 Ai ajuns tu la camarile zapezii? Ai vazut tu camarile grindinei,
\par 23 Pe care le ?in deoparte pentru vremuri de strâmtorare, pentru zilele de batalie ?i de razboi?
\par 24 Unde se risipesc aburii ?i se raspânde?te pe pamânt vântul de la rasarit?
\par 25 Cine a sapat albie puhoaielor cerului ?i cine a croit drum bubuitului tunetului,
\par 26 Ca sa ploua pe un pamânt nelocuit ?i pe o pustietate unde nu se afla fiin?a omeneasca
\par 27 ?i sa adape ?inuturile sterpe ?i uscate ?i sa scoata paji?te de iarba din întinderea ple?uva?
\par 28 Are ploaia un tata? Cine a zamislit stropii de roua?
\par 29 Din sânul cui a ie?it ghea?a? ?i cine este cel ce na?te promoroaca din cer?
\par 30 Apele se încheaga ?i se întaresc ca piatra ?i fala marii se face sloi!
\par 31 Po?i tu sa legi cataramele Pleiadelor sau sa deznozi lan?urile Orionului?
\par 32 Po?i tu sa sco?i la vreme cununa Zodiacului ?i vei fi tu cârmaci Carului Mare ?i stelelor lui?
\par 33 Cuno?ti tu legile cerului ?i po?i tu sa faci sa fie pe pamânt ceea ce este scris în ele?
\par 34 Po?i tu sa ridici pâna la nori glasul tau ca sa se sloboada peste tine potopul ploilor?
\par 35 E?ti tu în stare sa azvârli fulgerele ?i ele sa plece ?i sa-?i spuna: Iata-ne?
\par 36 Cine a pus atâta în?elepciune în pasarea ibis sau cine i-a dat pricepere coco?ului?
\par 37 Cine poate sa ?ina cu destoinicie socoteala norilor ?i sa verse pe pamânt burdufurile cerului,
\par 38 Ca sa se adune pulberea ?i sa se întareasca, iar bulgarii de pamânt sa se lipeasca laolalta?
\par 39 Tu e?ti cel ce aduci prada leoaicei ?i potole?ti foamea puilor de leu,
\par 40 Când s-au ascuns în vizuini sau stau ?i pândesc ascun?i în ha?i?uri?
\par 41 Cine are grija de mâncarea corbului, când puii lui croncanesc la Dumnezeu, de foame, ?i zboara încoace ?i încolo dupa hrana?

\chapter{39}

\par 1 ?tii tu când nasc caprele salbatice? Ai bagat de seama care este vremea cerboaicelor?
\par 2 Numeri tu lunile sarcinii lor ?i ?tii tu când le vine ceasul sa nasca?
\par 3 Ele îngenunchiaza, fata puii ?i scapa de durerile lor,
\par 4 Iar puii lor prind putere, se fac mari pe câmp, pornesc ?i nu se mai întorc spre mamele lor.
\par 5 Cine a lasat slobod asinul salbatic ?i l-a dezlegat de la iesle?
\par 6 I-am dat pustiul ca sa-l locuiasca ?i pamântul sarat i l-am harazit ocol;
\par 7 El î?i bate joc de zarva ora?elor; el nu aude strigatele nici unui stapân;
\par 8 El strabate mun?ii, locul sau de pa?une, ?i umbla dupa ori?ice verdea?a.
\par 9 Va voi bivolul salbatic sa se bage la tine sluga ?i sa petreaca noaptea linga ieslele tale?
\par 10 Po?i tu sa-l legi cu funia de gât ?i sa traga grapa dupa tine, peste aratura?
\par 11 Po?i sa te încrezi în el, fiindca este atât de tare, ?i sa-i la?i în grija munca ta?
\par 12 Te bizui tu pe el, ca mai vine înapoi sa-?i aduca roadele la aria ta?
\par 13 Aripile stru?ului sunt negrait de agere; stru?ul are pene preafrumoase ?i mândru penaj.
\par 14 Când î?i lasa ouale pe pamânt ?i le lasa sa se cloceasca în nisipul fierbinte,
\par 15 El uita ca oarecine poate sa le calce cu piciorul ?i ca vreo fiara salbatica poate sa le striveasca.
\par 16 Stru?ul e hain cu puii sai, ca ?i cum n-ar fi ai lui, ?i nu-i pasa deloc de truda sa zadarnica.
\par 17 Vezi ca Dumnezeu nu l-a înzestrat cu pricepere ?i patrundere.
\par 18 Când se scoala însa ?i porne?te, face de ocara ?i pe cal ?i pe calare?.
\par 19 Tu e?ti cel ce dai putere calului? Tu i-ai împodobit gâtul cu falnica lui coama?
\par 20 Tu l-ai înva?at sa sara u?or, ca o lacusta? Nechezatul lui viteaz insufla spaima!
\par 21 El bate pamântul cu copita ?i mândru de puterea lui porne?te spre taberele înarmate;
\par 22 El î?i bate joc de primejdie ?i n-are nici o teama ?i nu se da înapoi dinaintea sabiei.
\par 23 La oblânc îi suna tolba cu sage?i; fulgere arunca suli?a ?i lancea.
\par 24 De aprindere, de nerabdare, el manânca, gonind, pamântul ?i, când a sunat trâmbi?a, nu mai are astâmpar.
\par 25 La chemarea trâmbi?ei, pare ca zice: Haide! ?i de departe soarbe cu narile batalia, tunetul poruncilor capeteniilor ?i strigatele razboinice.
\par 26 Oare, prin de?teptaciunea ta s-a îmbracat în pene ?oimul ?i î?i întinde aripile ca ni?te seceri, spre miazazi?
\par 27 Nu cumva vulturul se ridica în înal?ime din porunca ta ?i î?i a?aza cuibul pe vârfuri neajunse?
\par 28 El î?i face sala?ul în stânci ?i acolo petrece noaptea - pe un vârf de stânca ?i pe vreo înal?ime prapastioasa.
\par 29 Acolo el sta ?i î?i pânde?te prada; ochii sai strapung departarile,
\par 30 Puii sai beau sângele prazii ?i unde sunt hoiturile, acolo se aduna vulturii".

\chapter{40}

\par 1 ?i Domnul a vorbit mai departe cu Iov ?i i-a zis:
\par 2 "Cel ce s-a apucat la cearta cu Cel Atotputernic se va da oare batut? Cel ce judeca pe Dumnezeu va raspunde ceva?"
\par 3 ?i Iov a raspuns Domnului zicând:
\par 4 "Daca am fost u?uratic, ce raspuns sa-?i mai dau? Voi pune mâna mea pe gura mea.
\par 5 Am vorbit o data, dar înca o data nu voi mai vorbi; de doua ori ?i nu voi lua-o iar de la început".
\par 6 Atunci Domnul a vorbit cu Iov, din mijlocul furtunii ?i a zis:
\par 7 "Încinge-?i coapsele ca un viteaz ?i te voi întreba ?i Îmi vei da lamuriri.
\par 8 Po?i tu cu adevarat sa gase?ti cusur judeca?ii Mele? ?i Ma vei osândi pe Mine, ca sa-?i faci dreptate?
\par 9 Este bra?ul tau ca bra?ul lui Dumnezeu? ?i glasul tau este, oare, tunet, precum este glasul Lui?
\par 10 Atunci împodobe?te-te cu mare?ie ?i cu seme?ie, îmbraca-te cu stralucire ?i cu cinste!
\par 11 Revarsa puhoaiele mâniei tale ?i doboara cu o privire pe cel trufa?!
\par 12 Vezi de to?i seme?ii ?i smere?te-i ?i calca în picioare, fara zabava, pe to?i cei rai!
\par 13 Ascunde-i pe to?i gramada, în pamânt, ?i îi înmormânteaza.
\par 14 ?i atunci Eu Însumi te voi preamari, pentru toate câte ai izbândit cu dreapta ta.
\par 15 Ia prive?te acum înaintea ta, hipopotamul; ?i el ca ?i tine este faptura Mea; el pa?te iarba ca boul.
\par 16 Vezi ce putere are în coapsele lui ?i ce tarie are în mu?chii de pe pântece.
\par 17 Coada lui e dârza ca lemnul cedrului ?i vinele de pe pulpele lui stau ca ni?te noduri.
\par 18 Oasele lui sunt ca ni?te ?evi de arama ?i madularele ca ni?te drugi de fier.
\par 19 El este fruntea fapturilor lui Dumnezeu ?i facut sa fie cel mai mare peste celelalte dobitoace.
\par 20 Mun?ii îi dau hrana ?i toate fiarele salbatice sunt îngrozite când îl vad.
\par 21 El se culca sub florile de lotus, în ocolul trestiilor ?i al bal?ii.
\par 22 Frunzele de lotus îi fac umbra ?i salciile bal?ii îl împrejmuiesc.
\par 23 Daca fluviul vine mare, fara de veste, el nu se sinchise?te; el sta lini?tit pe loc, chiar când ar fi ca Iordanul sa-i urce navalnic pâna la gura.
\par 24 Cine poate sa-l priveasca? Cine poate sa-i strapunga nasul cu un la??

\chapter{41}

\par 1 Po?i tu sa prinzi leviatanul cu undi?a, ori sa-i legi limba cu o sfoara?
\par 2 Vei putea tu sa-i vâri în nas o trestie sau sa-i gaure?ti falca cu cârligul?
\par 3 Î?i va face el multe rugamin?i ?i î?i va spune el lucruri dragala?e?
\par 4 Ori va face cu tine legamânt ?i-l vei lua la tine rob pe toata via?a?
\par 5 Te vei juca tu cu el, cum te joci cu o pasare, sau îl vei lega tu ca sa-?i învesele?ti fetele?
\par 6 Pescarii întovara?i?i vor putea sa-l scoata de vânzare ?i negustorii sa-l vânda cu bucata?
\par 7 Vei putea tu sa-i gaure?ti pielea cu sage?i ?i capul cu cârligul pescaresc?
\par 8 Ridica-?i numai mâna împotriva lui ?i vei pomeni de o asemenea lupta ?i nu o vei mai începe niciodata!
\par 9 Iata, este o de?ertaciune sa mai nadajduie?ti în izbânda; numai înfa?i?area lui ?i te da la pamânt.
\par 10 Cine este atât de nechibzuit încât sa-l întarâte? Cine va îndrazni sa dea piept cu Mine?
\par 11 Cine M-a îndatorat cu ceva, ca sa fiu acum dator sa-i dau înapoi? Tot ce se afla sub ceruri este al Meu.
\par 12 Cât despre leviatan, voi vorbi despre madularele lui ?i despre taria lui ?i despre frumoasa lui întocmire.
\par 13 Cine a ridicat pulpana din fa?a a ve?mântului lui ?i cine poate patrunde în captu?eala armurei lui?
\par 14 Cine a deschis vreodata por?ile gurii lui? Zim?ii lui sunt îngrozitori!
\par 15 Spinarea lui este ca un ?irag de scuturi, pe care le-ai fi întarit ?i pecetluit puternic.
\par 16 Ele sunt strânse unul într-altul atât de tare, ca nici vântul nu patrunde printre ele.
\par 17 Fiecare e lipit de cel de lânga el ?i se ?in a?a ?i nu se mai despart.
\par 18 Din stranutul lui scapara lumina ?i ochii lui sunt ro?ii ca pleoapele zorilor.
\par 19 Din gura lui ies parca ni?te tor?e aprinse ?i izbucnesc valuri de scântei.
\par 20 Din narile lui iese fum, ca dintr-o caldare pusa la foc ?i care fierbe.
\par 21 Rasuflarea lui este de carbuni aprin?i ?i din gura lui ?â?nesc flacari.
\par 22 Puterea lui e adunata în grumazul lui ?i înaintea lui ?â?ne?te groaza.
\par 23 Carnea lui e îndesata; oricât ai apasa în ea nu se lasa.
\par 24 Inima lui este tare ca piatra, tare ca piatra râ?ni?ei, cea de dedesubt.
\par 25 De mare?ia lui se tem ?i valurile; valurile marii se dau înapoi din fa?a lui.
\par 26 Sa-l atingi cu sabia nu folose?ti nimic; nici cu lancea, nici cu sageata, nici cu toporul.
\par 27 Fierul pentru pielea lui este ca paiul, iar arama ca lemnul putred.
\par 28 Sageata nu-l pune pe fuga ?i pietrele din pra?tie cad pe el ca ni?te pleava.
\par 29 O sageata pentru el este un pai în vânt ?i î?i bate joc de vâjâitul unei lanci ce zboara.
\par 30 Sub pântecele lui sunt ni?te solzi ascu?i?i; când da prin noroi, pare ca da cu grapa.
\par 31 Când se afunda, apa fierbe ca într-o caldare; el preface marea într-un cazan de fiert mirodenii.
\par 32 El lasa în urma o dâra luminoasa ?i adâncul pare un cap cu plete albe.
\par 33 Pe pamânt el nu-?i afla perechea ?i e facut sa nu cunoasca frica.
\par 34 El se uita de sus la to?i câ?i sunt puternici ?i este împarat peste toate fiarele salbatice".

\chapter{42}

\par 1 ?i Iov a raspuns Domnului zicând:
\par 2 "?tiu ca po?i sa faci orice ?i ca nu este nici un gând care sa nu ajunga pentru Tine fapta.
\par 3 Cine cuteaza, ai zis Tu, sa bârfeasca planurile Mele, din lipsa de în?elepciune? Cu adevarat, am vorbit fara sa în?eleg despre lucruri prea minunate pentru mine ?i nu ?tiam.
\par 4 Asculta - ai spus Tu iar - ?i Eu voi vorbi, te voi întreba ?i tu Îmi vei da lamuriri.
\par 5 Din spusele unora ?i altora auzisem despre Tine, dar acum ochiul meu Te-a vazut.
\par 6 Pentru aceea, ma urgisesc eu pe mine însumi ?i ma pocaiesc în praf ?i în cenu?a".
\par 7 Iar dupa ce Domnul a rostit aceste cuvinte catre Iov, a grait catre Elifaz din Teman: "Mânia Mea arde împotriva ta ?i împotriva celor doi prieteni ai tai, pentru ca n-a?i vorbit de Mine a?a de drept, precum a vorbit robul Meu Iov.
\par 8 Acum deci lua?i ?apte vi?ei ?i ?apte berbeci ?i duce?i-va la robul Meu Iov ?i aduce?i-le, pentru voi, ardere de tot; iar robul Meu Iov sa se roage pentru voi; din dragoste pentru el, voi fi îngaduitor, ca sa nu Ma port cu voi dupa nebunia voastra, întrucât n-a?i vorbit despre Mine a?a de drept cum a vorbit robul Meu Iov".
\par 9 Astfel Elifaz din Teman, Bildad din ?uah ?i ?ofar din Naamat, s-au dus ?i au facut cum le spusese Domnul, ?i Domnul a primit rugaciunea lui Iov.
\par 10 ?i Domnul l-a pus pe Iov iara?i în starea lui de la început, dupa ce s-a rugat pentru prieteni, ?i i-a întors îndoit tot ce avusese mai înainte.
\par 11 ?i to?i fra?ii ?i toate surorile ?i to?i prietenii lui de alta data au venit sa-l cerceteze, au mâncat pâine în casa lui, l-au compatimit, l-au mângâiat de toate nenorocirile pe care le slobozise Domnul asupra lui ?i fiecare i-a dat câte un chesita ?i câte un inel de aur.
\par 12 ?i Dumnezeu a binecuvântat sfâr?itul vie?ii lui Iov mai bogat decât începutul ei, ?i el a strâns paisprezece mii de oi, ?ase mii de camile, o mie de perechi de boi ?i o mie de asine.
\par 13 ?i a avut ?apte fii ?i trei fiice.
\par 14 Celei dintâi i-a pus numele Iemima, celei de a doua, Che?ia ?i celei de a treia, Cheren-Hapuc.
\par 15 Iar în toata ?ara nu se gaseau femei atât de frumoase ca fetele lui Iov, ?i tatal lor le-a facut parta?e la mo?tenire, lânga fra?ii lor.
\par 16 ?i Iov a mai trait dupa aceea o suta patruzeci de ani ?i a vazut pe fiii sai ?i pe fiii fiilor sai, pâna la al patrulea neam.
\par 17 ?i Iov a murit batrân ?i încarcat de zile.


\end{document}