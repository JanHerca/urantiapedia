\begin{document}

\title{Evrei}


\chapter{1}

\par 1 După ce Dumnezeu odinioară, în multe rânduri și în multe chipuri, a vorbit părinților noștri prin prooroci,
\par 2 În zilele acestea mai de pe urmă ne-a grăit nouă prin Fiul, pe Care L-a pus moștenitor a toate și prin Care a făcut și veacurile;
\par 3 Care, fiind strălucirea slavei și chipul ființei Lui și Care ține toate cu cuvântul puterii Sale, după ce a săvârșit, prin El însuși, curățirea păcatelor noastre, a șezut de-a dreapta slavei, întru cele prea înalte,
\par 4 Făcându-Se cu atât mai presus de îngeri, cu cât a moștenit un nume mai deosebit decât ei.
\par 5 Căci căruia dintre îngeri i-a zis Dumnezeu vreodată: "Fiul Meu ești Tu, Eu astăzi Te-am născut"; și iarăși: "Eu Îi voi fi Lui Tată și El Îmi va fi Mie Fiu"?
\par 6 Și iarăși, când aduce în lume pe Cel întâi născut, El zice: "Și să se închine Lui toți îngerii lui Dumnezeu".
\par 7 Și de îngeri zice: "Cel ce face pe îngerii Săi duhuri și pe slujitorii Săi pară de foc";
\par 8 Iar către Fiul: "Tronul Tău, Dumnezeule, în veacul veacului; și toiagul dreptății este toiagul împărăției Tale.
\par 9 Iubit-ai dreptatea și ai urât fărădelegea; pentru aceea Te-a uns pe Tine, Dumnezeule, Dumnezeul Tău cu untdelemnul bucuriei, mai mult decât pe părtașii Tăi".
\par 10 Și: "Întru început Tu, Doamne, pământul l-ai întemeiat și cerurile sunt lucrul mâinilor Tale;
\par 11 Ele vor pieri, dar Tu rămâi, și toate ca o haină se vor învechi;
\par 12 Și ca un pe un veșmânt le vei strânge și ca o haină vor fi schimbate. Dar Tu același ești și anii Tăi nu se vor sfârși".
\par 13 Și căruia dintre îngeri a zis Dumnezeu vreodată: "Șezi de-a dreapta Mea până când voi pune pe vrăjmașii tăi așternut picioarelor Tale"?
\par 14 Îngerii oare nu sunt toți duhuri slujitoare, trimise ca să slujească, pentru cei ce vor fi moștenitorii mântuirii?

\chapter{2}

\par 1 Pentru aceea se cuvine ca noi să luăm aminte cu atât mai mult la cele auzite, ca nu cumva să ne pierdem.
\par 2 Căci, dacă s-a adeverit cuvântul grăit prin îngeri și orice călcare de poruncă și orice neascultare și-a primit dreapta răsplătire,
\par 3 Cum vom scăpa noi, dacă vom fi nepăsători la astfel de mântuire care, luând obârșie din propovăduirea Domnului, ne-a fost adeverită de cei ce au ascultat-o,
\par 4 Împreună mărturisind și Dumnezeu cu semne și cu minuni și cu multe feluri de puteri și cu darurile Duhului Sfânt, împărțite după a Sa voință?
\par 5 Pentru că nu îngerilor a supus Dumnezeu lumea viitoare, despre care vorbim.
\par 6 Iar cineva a mărturisit undeva, zicând: "Ce este omul, că-l pomenești pe el, sau fiul omului, că-l cercetezi pe el?
\par 7 L-ai micșorat pe el cu puțin față de îngeri; și cu mărire și cu cinste l-ai încununat și l-ai pus peste lucrurile mâinilor Tale.
\par 8 Toate le-ai supus sub picioarele lui". Dar prin faptul că a supus lui toate (înțelegem) că nimic nu i-a lăsat nesupus. Acum însă, încă nu vedem cum toate i-au fost supuse.
\par 9 Ci pe Cel micșorat cu puțin față de îngeri, pe Iisus, Îl vedem încununat cu slavă și cu cinste, din pricina morții pe care a suferit-o, astfel că, prin harul lui Dumnezeu, El a gustat moartea pentru fiecare om.
\par 10 Căci ducând pe mulți fii la mărire, I se cădea Aceluia, pentru Care sunt toate și prin Care sunt toate, ca să desăvârșească prin pătimire pe Începătorul mântuirii lor.
\par 11 Pentru că și Cel ce sfințește și cei ce se sfințesc, dintr-Unul sunt toți; de aceea nu se rușinează să-i numească pe ei frați,
\par 12 Zicând: "Spune-voi fraților mei numele Tău. În mijlocul Bisericii Te voi lăuda".
\par 13 Și iarăși: "Eu voi fi încrezător în El"; și iarăși: "Iată Eu și pruncii pe care Mi i-a dat Dumnezeu".
\par 14 Deci, de vreme ce pruncii s-au făcut părtași sângelui și trupului, în același fel și El S-a împărtășit de acestea, ca să surpe prin moartea Sa pe cel ce are stăpânirea morții, adică pe diavolul,
\par 15 Și să izbăvească pe acei pe care frica morții îi ținea în robie toată viața.
\par 16 Căci, într-adevăr, nu a luat firea îngerilor, ci sămânța lui Avraam a luat.
\par 17 Pentru aceea, dator era întru toate să Se asemene fraților, ca să fie milostiv și credincios arhiereu în cele către Dumnezeu, pentru curățirea păcatelor poporului.
\par 18 Căci prin ceea ce a pătimit, fiind El însuși ispitit, poate și celor ce se ispitesc să le ajute.

\chapter{3}

\par 1 Pentru aceea, frați sfinți, părtași chemării cerești, luați aminte la Apostolul și Arhiereul mărturisirii noastre, la Iisus Hristos,
\par 2 Care credincios a fost Celui ce L-a rânduit, precum și Moise în toată casa Lui.
\par 3 Pentru că Acesta (Iisus) S-a învrednicit de mai multă slavă decât Moise, după cum are mai multă cinste decât casa cel ce a zidit-o.
\par 4 Căci orice casă e zidită de către cineva, iar Ziditorul a toate este Dumnezeu.
\par 5 Moise a fost credincios în toată casa Domnului, ca o slugă, spre mărturia celor ce erau să fie descoperite în viitor,
\par 6 Iar Hristos a fost credincios ca Fiu peste casa Sa. Și casa Lui suntem noi, numai dacă ținem până la sfârșit cu neclintire, îndrăzneala mărturisirii și lauda nădejdii noastre.
\par 7 De aceea, precum zice Duhul Sfânt: "Dacă veți auzi astăzi glasul Lui,
\par 8 Nu vă învârtoșați inimile voastre, ca la răzvrătire în ziua ispitirii din pustie,
\par 9 Unde M-au ispitit părinții voștri, M-au încercat, și au văzut faptele Mele, timp de patruzeci de ani.
\par 10 De aceea M-am mâniat pe neamul acesta și am zis: Pururea ei rătăcesc cu inima și căile Mele nu le-au cunoscut,
\par 11 Că M-am jurat în mânia Mea: "Nu vor intra întru odihna Mea!".
\par 12 Luați seama, fraților, să nu fie cumva, în vreunul din voi, o inimă vicleană a necredinței, ca să vă depărteze de la Dumnezeul cel viu.
\par 13 Ci îndemnați-vă unii pe alții, în fiecare zi, până ce putem să zicem: astăzi! ca nimeni dintre voi să nu se învârtoșeze cu înșelăciunea păcatului;
\par 14 Căci ne-am făcut părtași ai lui Hristos, numai dacă vom păstra temeinic, până la urmă, începutul stării noastre întru El,
\par 15 De vreme ce se zice: Dacă veți auzi astăzi glasul Lui, nu învârtoșați inimile voastre, ca la răzvrătire.
\par 16 Cine sunt cei care, auzind, s-au răzvrătit? Oare nu toți care au ieșit din Egipt, prin Moise?
\par 17 Și împotriva cui a ținut mâine timp de patruzeci de ani? Au nu împotriva celor ce au păcătuit, ale căror oase au căzut în pustie?
\par 18 Și cui S-a jurat că nu vor intra întru odihna Sa, decât numai celor ce au fost neascultători?
\par 19 Vedem dar că n-au putut să intre, din pricina necredinței lor.

\chapter{4}

\par 1 Să ne temem, deci, ca nu cumva, câtă vreme ni se lasă făgăduința să intrăm în odihna Lui, să pară că a rămas pe urmă cineva dintre voi.
\par 2 Pentru că și nouă ni s-a binevestit ca și acelora, dar cuvântul propovăduirii nu le-a fost lor de folos, nefiind unit cu credința la cei care l-au auzit.
\par 3 Pe când noi, fiindcă am crezut, intrăm în odihnă, precum s-a zis: "M-am jurat întru mânia Mea: nu vor intra întru odihna Mea", măcar că lucrurile erau săvârșite de la întemeierea lumii.
\par 4 Căci undeva, despre ziua a șaptea, a zis astfel: "Și S-a odihnit Dumnezeu în ziua a șaptea de toate lucrurile Sale".
\par 5 Și în același loc, zice iarăși: "Nu vor intra întru odihna Mea!".
\par 6 Deci, de vreme ce rămâne ca unii să intre în odihnă, iar aceia cărora mai dinainte li s-a binevestit, pentru nesupunerea lor, n-au intrat,
\par 7 Dumnezeu hotărăște din nou o zi, astăzi rostind prin gura lui David, după atâta vreme, precum s-a zis mai sus: "Dacă veți auzi astăzi glasul Lui, nu învârtoșați inimile voastre".
\par 8 Căci dacă Iosua le-ar fi adus odihnă, Dumnezeu n-ar mai fi vorbit, după acestea, de o altă zi de odihnă.
\par 9 Drept aceea, s-a lăsat altă sărbătoare de odihnă poporului lui Dumnezeu.
\par 10 Pentru că cine a intrat în odihna lui Dumnezeu s-a odihnit și el de lucrurile lui, precum Dumnezeu de ale Sale.
\par 11 Să ne silim, deci, ca să intrăm în acea odihnă, ca nimeni să nu cadă în aceeași pildă a neascultării,
\par 12 Căci cuvântul lui Dumnezeu e viu și lucrător și mai ascuțit decât orice sabie cu două tăișuri, și pătrunde până la despărțitura sufletului și duhului, dintre încheieturi și măduvă, și destoinic este să judece simțirile și cugetările inimii,
\par 13 Și nu este nici o făptură ascunsă înaintea Lui, ci toate sunt goale și descoperite, pentru ochii Celui în fața Căruia noi vom da socoteală.
\par 14 Drept aceea, având Arhiereu mare, Care a străbătut cerurile, pe Iisus, Fiul lui Dumnezeu, să ținem cu tărie mărturisirea.
\par 15 Că nu avem Arhiereu care să nu poată suferi cu noi în slăbiciunile noastre, ci ispitit întru toate după asemănarea noastră, afară de păcat.
\par 16 Să ne apropiem, deci, cu încredere de tronul harului, ca să luăm milă și să aflăm har, spre ajutor, la timp potrivit.

\chapter{5}

\par 1 Căci orice arhiereu, fiind luat dintre oameni, este pus pentru oameni, spre cele către Dumnezeu, ca să aducă daruri și jertfe pentru păcate;
\par 2 El poate să fie îngăduitor cu cei neștiutori și rătăciți, de vreme ce și el este cuprins de slăbiciune.
\par 3 Din această pricină dator este, precum pentru popor, așa și pentru sine să jertfească pentru păcate.
\par 4 Și nimeni nu-și ia singur cinstea aceasta, ci dacă este chemat de Dumnezeu după cum și Aaron.
\par 5 Așa și Hristos nu S-a preaslăvit pe Sine însuși, ca să Se facă arhiereu, ci Cel ce a grăit către El: "Fiul Meu ești Tu, Eu astăzi Te-am născut".
\par 6 În alt loc se zice: "Tu ești Preot în veac după rânduiala lui Melchisedec".
\par 7 El, în zilele trupului Său, a adus, cu strigăt și cu lacrimi, cereri și rugăciuni către Cel ce putea să-L mântuiască din moarte și auzit a fost pentru evlavia Sa,
\par 8 Și deși era Fiu, a învățat ascultarea din cele ce a pătimit,
\par 9 Și desăvârșindu-Se, S-a făcut tuturor celor ce-L ascultă pricină de mântuire veșnică.
\par 10 Iar de Dumnezeu a fost numit: Arhiereu după rânduiala lui Melchisedec.
\par 11 În privința aceasta avem mult de vorbit și lucruri grele de tâlcuit, de vreme ce v-ați făcut greoi la auzit.
\par 12 Căci voi, care de multă vreme s-ar fi cuvenit să fiți învățători, aveți iarăși trebuință ca cineva să vă învețe cele dintâi începuturi ale cuvintelor lui Dumnezeu și ați ajuns să aveți nevoie de lapte, nu de hrană tare.
\par 13 Pentru că oricine se hrănește cu lapte este nepriceput în cuvântul dreptății, de vreme ce este prunc.
\par 14 Iar hrana tare este pentru cei desăvârșiți, care au prin obișnuință simțurile învățate să deosebească binele și răul.

\chapter{6}

\par 1 De aceea, lăsând cuvântul de început despre Hristos, să ne ridicăm spre ceea ce este desăvârșit, fără să mai punem din nou temelia învățăturii despre pocăința de faptele moarte și despre credința în Dumnezeu,
\par 2 A învățăturii despre botezuri, despre punerea mâinilor, despre învierea morților și despre judecata veșnică.
\par 3 Și aceasta vom face-o cu voia lui Dumnezeu.
\par 4 Căci este cu neputință pentru cei ce s-au luminat odată și au gustat darul cel ceresc și părtași s-au făcut Duhului Sfânt,
\par 5 Și au gustat cuvântul cel bun al lui Dumnezeu și puterile veacului viitor,
\par 6 Cu neputință este pentru ei, dacă au căzut, să se înnoiască iarăși spre pocăință, fiindcă ei răstignesc loruși, a doua oară, pe Fiul lui Dumnezeu și-L fac de batjocură.
\par 7 Țarina, când absoarbe ploaia ce se coboară adeseori asupra ei și rodește iarba folositoare celor pentru care a fost muncită, primește binecuvântarea de la Dumnezeu;
\par 8 Dar dacă aduce spini și ciulini, se face netrebnică și blestemul îi stă aproape iar la urmă focul o așteaptă.
\par 9 Despre voi, iubiților, deși vorbim astfel, suntem încredințați de lucruri mai bune și aducătoare de mântuire.
\par 10 Căci Dumnezeu nu este nedrept, ca să uite lucrul vostru și dragostea pe care ați arătat-o pentru numele Lui, voi, care ați slujit și slujiți sfinților.
\par 11 Dorind dar, ca fiecare dintre voi să arate aceeași râvnă spre adeverirea nădejdii, până la sfârșit,
\par 12 Ca să nu fiți greoi, ci următori ai celor ce, prin credință și îndelungă-răbdare, moștenesc făgăduințele.
\par 13 Căci Dumnezeu, când a dat făgăduință lui Avraam, de vreme ce n-avea pe nimeni mai mare, pe care să Se jure, S-a jurat pe Sine însuși,
\par 14 Zicând: "Cu adevărat, binecuvântând te voi binecuvânta, și înmulțind te voi înmulți".
\par 15 Și așa, având Avraam îndelungă-răbdare, a dobândit făgăduința.
\par 16 Pentru că oamenii se jură pe cel ce e mai mare și jurământul e la ei o chezășie și sfârșitul oricărei neînțelegeri.
\par 17 În aceasta, Dumnezeu voind să arate și mai mult, moștenitorilor făgăduinței, nestrămutarea hotărârii Sale, a pus la mijloc jurământul:
\par 18 Ca prin două fapte nestrămutate - făgăduința și jurământul - în care e cu neputință ca Dumnezeu să fi mințit, noi, cei ce căutăm scăpare, să avem îndemn puternic ca să ținem nădejdea pusă înainte,
\par 19 Pe care o avem ca o ancoră a sufletului, neclintită și tare, intrând dincolo de catapeteasmă,
\par 20 Unde Iisus a intrat pentru noi ca înaintemergător, fiind făcut Arhiereu în veac, după rânduiala lui Melchisedec.

\chapter{7}

\par 1 Căci acest Melchisedec, rege al Salemului, preot al lui Dumnezeu cel Preaînalt, care a întâmpinat pe Avraam, pe când se întorcea de la nimicirea regilor și l-a binecuvântat,
\par 2 Căruia Avraam i-a dat și zeciuială din toate, se tâlcuiește mai întâi: rege al dreptății, apoi și rege al Salemului, adică rege al păcii,
\par 3 Fără tată, fără mamă, fără spiță de neam, neavând nici început al zilelor, nici sfârșit al vieții, ci, asemănat fiind Fiului lui Dumnezeu, el rămâne preot pururea.
\par 4 Vedeți, dar, cât de mare e acesta, căruia chiar patriarhul Avraam i-a dat zeciuială din prada de război.
\par 5 Și cei dintre fiii lui Levi, care primesc preoția, au poruncă după lege, ca să ia zeciuială de la popor, adică de la frații lor, măcar că și aceștia au ieșit din coapsele lui Avraam;
\par 6 Iar Melchisedec, care nu-și trage neamul din ei, a primit zeciuială de la Avraam și pe Avraam, care avea făgăduințele, l-a binecuvântat.
\par 7 Fără de nici o îndoială, cel mai mic ia binecuvântare de la cel mai mare.
\par 8 Și aici iau zeciuială niște oameni muritori, pe când dincolo, unul care e dovedit că este viu.
\par 9 Și ca să spun așa, prin Avraam, a dat zeciuială și Levi, cel ce lua zeciuială,
\par 10 Fiindcă el era încă în coapsele lui Avraam, când l-a întâmpinat Melchisedec.
\par 11 Dacă deci desăvârșirea ar fi fost prin preoția Leviților (căci legea s-a dat poporului pe temeiul preoției lor), ce nevoie mai era să se ridice un alt preot după rânduiala lui Melchisedec, și să nu se zică după rânduiala lui Aaron?
\par 12 Iar dacă preoția s-a schimbat urmează numaidecât și schimbarea Legii.
\par 13 Căci Acela, despre Care se spun acestea, își ia obârșia dintr-o altă seminție, de unde nimeni n-a slujit altarului,
\par 14 Știut fiind că Domnul nostru a răsărit din Iuda, iar despre seminția acestora, cu privire la preoți, Moise n-a vorbit nimic.
\par 15 Apoi este lucru și mai lămurit că, dacă se ridică un alt preot după asemănarea lui Melchisedec,
\par 16 El s-a făcut nu după legea unei porunci trupești, ci cu puterea unei vieți nepieritoare,
\par 17 Căci se mărturisește: "Tu ești Preot în veac, după rânduiala lui Melchisedec".
\par 18 Astfel, porunca dată întâi se desființează, pentru neputința și nefolosul ei;
\par 19 Căci Legea n-a desăvârșit nimic, iar în locul ei își face cale o nădejde mai bună, prin care ne apropiem de Dumnezeu.
\par 20 Ci încă a fost la mijloc și un jurământ, căci pe când aceia s-au făcut preoți fără de jurământ,
\par 21 El S-a făcut cu jurământul Celui ce I-a grăit: "Juratu-S-a Domnul și nu Se va căi: Tu ești Preot în veac, după rânduiala lui Melchisedec".
\par 22 Cu aceasta, Iisus S-a făcut chezașul unui mai bun testament.
\par 23 Apoi acolo s-a ridicat un șir de preoți, fiindcă moartea îi împiedica să dăinuiască.
\par 24 Aici însă, Iisus, prin aceea că rămâne în veac, are o preoție netrecătoare (veșnică).
\par 25 Pentru aceasta, și poate să mântuiască desăvârșit pe cei ce se apropie prin El de Dumnezeu, căci pururea e viu ca să mijlocească pentru ei.
\par 26 Un astfel de Arhiereu se cuvenea să avem: sfânt, fără de răutate, fără de pată, osebit de cei păcătoși, și fiind mai presus decât cerurile.
\par 27 El nu are nevoie să aducă zilnic jertfe, ca arhiereii: întâi pentru păcatele lor, apoi pentru ale poporului, căci El a făcut aceasta o dată pentru totdeauna, aducându-Se jertfă pe Sine însuși.
\par 28 Căci Legea pune ca arhierei oameni care au slăbiciune, pe când cuvântul jurământului, venit în urma Legii, pune pe Fiul, desăvârșit în veacul veacului.

\chapter{8}

\par 1 Lucru de căpetenie din cele spuse este că avem astfel de Arhiereu care a șezut de-a dreapta tronului slavei în ceruri,
\par 2 Slujitor Altarului și Cortului celui adevărat, pe care l-a înfipt Dumnezeu și nu omul.
\par 3 Apoi, orice arhiereu este pus ca să aducă daruri și jertfe; de aceea trebuincios era ca și acest Arhiereu să fi avut ceva ce să aducă.
\par 4 Dacă ar fi pe pământ, nici n-ar fi preot, fiindcă aici sunt aceia care aduc darurile potrivit Legii,
\par 5 Care slujesc închipuirii și umbrei celor cerești, precum a primit poruncă Moise, când era să facă cortul: "Ia seama, zice Domnul, să faci toate după chipul ce ți-a fost arătat în munte".
\par 6 Acum însă, Arhiereul nostru a dobândit o slujire cu atât mai osebită, cu cât este și Mijlocitorul unui testament mai bun, ca unul care este întemeiat pe mai bune făgăduințe.
\par 7 Căci dacă (testamentul) cel dintâi ar fi fost fără de prihană, nu s-ar mai fi căutat loc pentru al doilea;
\par 8 Ci Dumnezeu îi mustră și le zice: "Iată vin zile, zice Domnul, când voi face, cu casa lui Israel și cu casa lui Iuda, testament nou,
\par 9 Nu ca testamentul pe care l-am făcut cu părinții lor, în ziua când i-am apucat de mână ca să-i scot din pământul Egiptului; căci ei n-au rămas în testamentul Meu, de aceea și Eu i-am părăsit - zice Domnul.
\par 10 Că acesta e testamentul pe care îl voi face cu casa lui Israel, după acele zile, zice Domnul: Pune-voi legile Mele în cugetul lor și în inima lor le voi scrie, și voi fi lor Dumnezeu și ei vor fi poporul Meu.
\par 11 Și nu va mai învăța fiecare pe vecinul său și fiecare pe fratele său zicând: Cunoaște pe Domnul! - căci toți Mă vor cunoaște, de la cel mai mic până la cel mai mare al lor;
\par 12 Căci voi fi milostiv cu nedreptățile lor și de păcatele lor nu-Mi voi mai aduce aminte".
\par 13 Și zicând: "Nou", Domnul a învechit pe cel dintâi. Iar ce se învechește și îmbătrânește, aproape este de pieire.

\chapter{9}

\par 1 Deci și cei dintâi (Așezământ) avea orânduieli pentru slujba dumnezeiască și un altar pământesc,
\par 2 Căci s-a pregătit cortul mărturiei. În el se aflau, mai întâi, sfeșnicul și masa și pâinile punerii înainte; partea aceasta se numește Sfânta.
\par 3 Apoi, după catapeteasma a doua, era cortul numit Sfânta Sfintelor,
\par 4 Având altarul tămâierii de aur și chivotul Așezământului ferecat peste tot cu aur, în care era năstrapa de aur, care avea mana, era toiagul lui Aaron ce odrăslise și tablele Legii.
\par 5 Deasupra chivotului erau heruvimii slavei, care umbreau altarul împăcării; despre acestea nu putem acum să vorbim cu de-amănuntul.
\par 6 Astfel fiind întocmite aceste încăperi, preoții intrau totdeauna în cortul cel dintâi, săvârșind slujbele dumnezeiești;
\par 7 În cel de-al doilea însă numai arhiereul, o dată pe an, și nu fără de sânge, pe care îl aducea pentru sine însuși și pentru greșealele poporului.
\par 8 Prin aceasta, Duhul Sfânt ne lămurește că drumul către Sfânta Sfintelor nu era să fie arătat, câtă vreme cortul întâi mai sta în picioare,
\par 9 Care era o pildă pentru timpul de față și însemna că darurile și jertfele ce se aduceau n-aveau putere să desăvârșească cugetul închinătorului.
\par 10 Acestea erau numai legiuiri pământești - despre mâncăruri, despre băuturi, despre felurite spălări - și erau porunci până la vremea îndreptării.
\par 11 Iar Hristos, venind Arhiereu al bunătăților celor viitoare, a trecut prin cortul cel mai mare și mai desăvârșit, nu făcut de mână, adică nu din zidirea aceasta;
\par 12 El a intrat o dată pentru totdeauna în Sfânta Sfintelor, nu cu sânge de țapi și de viței, ci cu însuși sângele Său, și a dobândit o veșnică răscumpărare.
\par 13 Căci dacă sângele țapilor și al taurilor și cenușa junincii, stropind pe cei spurcați, îi sfințește spre curățirea trupului,
\par 14 Cu cât mai mult sângele lui Hristos, Care, prin Duhul cel veșnic, S-a adus lui Dumnezeu pe Sine, jertfă fără de prihană, va curăți cugetul vostru de faptele cele moarte, ca să slujiți Dumnezeului celui viu?
\par 15 Și pentru aceasta El este Mijlocitorul unui nou testament, ca prin moartea suferită spre răscumpărarea greșealelor de sub întâiul testament, cei chemați să ia făgăduința moștenirii veșnice.
\par 16 Căci unde este testament, trebuie neapărat să fie vorba despre moartea celui ce a făcut testamentul.
\par 17 Un testament ajunge temeinic după moarte, fiindcă nu are nici o putere, câtă vreme trăiește cel ce l-a făcut.
\par 18 De aceea, nici cel dintâi n-a fost sfințit fără sânge.
\par 19 Într-adevăr Moise, după ce a rostit față cu tot poporul toate poruncile din Lege, luând sângele cel de viței și de țapi, cu apă și cu lână roșie și cu isop, a stropit și cartea și pe tot poporul,
\par 20 Și a zis: "Acesta este sângele testamentului pe care l-a poruncit vouă Dumnezeu".
\par 21 Și a stropit, de asemenea, cu sânge, cortul și toate vasele pentru slujbă.
\par 22 După Lege, aproape toate se curățesc cu sânge, și fără vărsare de sânge nu se dă iertare.
\par 23 Trebuie dar ca chipurile celor din ceruri să fie curățite prin acestea, iar cele cerești înseși cu jertfe mai bune decât acestea.
\par 24 Căci Hristos n-a intrat într-o Sfântă a Sfintelor făcută de mâini - închipuirea celei adevărate - ci chiar în cer, ca să Se înfățișeze pentru noi înaintea lui Dumnezeu;
\par 25 Iar nu ca să Se aducă pe Sine însuși jertfă de mai multe ori - ca arhiereul care intră în Sfânta Sfintelor cu sânge străin, în fiecare an.
\par 26 Altfel, ar fi trebuit să pătimească de mai multe ori, de la întemeierea lumii; ci acum, la sfârșitul veacurilor, S-a arătat o dată, spre ștergerea păcatului, prin jertfa Sa.
\par 27 Și precum este rânduit oamenilor o dată să moară, iar după aceea să fie judecata,
\par 28 Tot așa și Hristos, după ce a fost adus o dată jertfă, ca să ridice păcatele multora, a doua oară fără de păcat Se va arăta celor ce cu stăruință Îl așteaptă spre mântuire.

\chapter{10}

\par 1 În adevăr, Legea având umbra bunurilor viitoare, iar nu însuși chipul lucrurilor, nu poate niciodată - cu aceleași jertfe, aduse neîncetat în fiecare an - să facă desăvârșiți pe cei ce se apropie.
\par 2 Altfel, n-ar fi încetat oare jertfele aduse, dacă cei ce săvârșesc slujba dumnezeiască, fiind o dată curățiți, n-ar mai avea nici o conștiință a păcatelor?
\par 3 Ci prin ele, an de an, se face amintirea păcatelor.
\par 4 Pentru că este cu neputință ca sângele de tauri și de țapi să înlăture păcatele.
\par 5 Drept aceea, intrând în lume, zice: "Jertfă și prinos n-ai voit, dar mi-ai întocmit trup.
\par 6 Arderi de tot și jertfe pentru păcat nu ți-au plăcut;
\par 7 Atunci am zis: Iată vin, în sulul cărții este scris despre mine, să fac voia Ta, Dumnezeule".
\par 8 Zicând mai sus că: "Jertfă și prinoase și arderile de tot și jertfele pentru păcat n-ai voit, nici nu Ți-au plăcut", care se aduc după Lege,
\par 9 Atunci a zis: "Iată vin ca să fac voia Ta, Dumnezeule". El desființează deci pe cei dintâi ca să statornicească pe al doilea.
\par 10 Întru această voință suntem sfințiți, prin jertfa trupului lui Iisus Hristos, o dată pentru totdeauna.
\par 11 Și orice preot stă și slujește în fiecare zi și aceleași jertfe aduce de multe ori, ca unele care niciodată nu pot să înlăture păcatele.
\par 12 Acesta dimpotrivă, aducând o singură jertfă pentru păcate, a șezut în vecii vecilor, de-a dreapta lui Dumnezeu,
\par 13 Și așteaptă până ce vrăjmașii Lui vor fi puși așternut picioarelor Lui.
\par 14 Căci printr-o singură jertfă adusă, a adus la veșnică desăvârșire pe cei ce se sfințesc;
\par 15 Dar și Duhul cel Sfânt ne mărturisește aceasta, fiindcă după ce a zis:
\par 16 "Acesta este așezământul pe care îl voi întocmi cu ei, după acele zile - zice Domnul: Da-voi legile Mele în inimile lor și le voi scrie în cugetele lor".
\par 17 Și adaugă: "Iar de păcatele lor și de fărădelegile lor nu-Mi voi mai aduce aminte".
\par 18 Unde este dar iertarea acestora, nu mai este jertfă pentru păcate.
\par 19 Drept aceea, fraților, având îndrăzneală, să intrăm în Sfânta Sfintelor, prin sângele lui Iisus,
\par 20 Pe calea cea nouă și vie pe care pentru noi a înnoit-o, prin catapeteasmă, adică prin trupul Său,
\par 21 Și având mare preot peste casa lui Dumnezeu,
\par 22 Să ne apropiem cu inimă curată, întru plinătatea credinței, curățindu-ne prin stropire inimile de orice cuget rău, și spălându-ne trupul în apă curată,
\par 23 Să ținem mărturisirea nădejdii cu neclintire, pentru că credincios este Cel ce a făgăduit,
\par 24 Și să luăm seama unul altuia, ca să ne îndemnăm la dragoste și la fapte bune,
\par 25 Fără să părăsim Biserica noastră, precum le este obiceiul unora, ci îndemnători făcându-ne, cu atât mai mult, cu cât vedeți că se apropie ziua aceea.
\par 26 Căci dacă păcătuim de voia noastră, după ce am luat cunoștiință despre adevăr, nu ne mai rămâne, pentru păcate, nici o jertfă,
\par 27 Ci o înfricoșată așteptare a judecății și iuțimea focului care va mistui pe cei potrivnici.
\par 28 Călcând cineva Legea lui Moise, e ucis fără de milă, pe cuvântul a doi sau trei martori;
\par 29 Gândiți-vă: cu cât mai aspră fi-va pedeapsa cuvenită celui ce a călcat în picioare pe Fiul lui Dumnezeu, și a nesocotit sângele testamentului cu care s-a sfințit, și a batjocorit duhul harului.
\par 30 Căci cunoaștem pe Cel ce a zis: "A Mea este răzbunarea; Eu voi răsplăti". Și iarăși: "Domnul va judeca pe poporul Său".
\par 31 Înfricoșător lucru este să cădem în mâinile Dumnezeului celui viu.
\par 32 Aduceți-vă, dar, aminte mai întâi de zilele în care, după ce ați fost luminați, ați răbdat luptă grea de suferințe,
\par 33 Parte făcându-vă priveliște cu ocările și cu necazurile îndurate, parte suferind împreună cu cei ce treceau prin unele ca acestea,
\par 34 Căci ați avut milă de cei închiși, iar răpirea averilor voastre ați primit-o cu bucurie, bine știind că voi aveți o mai bună și statornică avere.
\par 35 Nu lepădați dar încrederea voastră, care are mare răsplătire.
\par 36 Căci aveți nevoie de răbdare ca, făcând voia lui Dumnezeu, să dobândiți făgăduința.
\par 37 "Căci mai este puțin timp, prea puțin, și Cel ce e să vină, va veni și nu va întârzia;
\par 38 Iar dreptul din credință va fi viu; și de se va îndoi cineva, nu va binevoi sufletul Meu întru el".
\par 39 Noi nu suntem (fii) ai îndoielii spre pieire, ci ai credinței spre dobândirea sufletului.

\chapter{11}

\par 1 Iar credința este încredințarea celor nădăjduite, dovedirea lucrurilor celor nevăzute.
\par 2 Prin ea, cei din vechime au dat buna lor mărturie.
\par 3 Prin credință înțelegem că s-au întemeiat veacurile prin cuvântul lui Dumnezeu, de s-au făcut din nimic cele ce se văd.
\par 4 Prin credință, Abel a adus lui Dumnezeu mai bună jertfă decât Cain, pentru care a luat mărturie că este drept, mărturisind Dumnezeu despre darurile lui; și prin credință grăiește și azi, deși a murit.
\par 5 Prin credință, Enoh a fost luat de pe pământ ca să nu vadă moartea, și nu s-a mai aflat, pentru că Dumnezeu îl strămutase, căci mai înainte de a-l strămuta, el a avut mărturie că a bine-plăcut lui Dumnezeu.
\par 6 Fără credință, dar, nu este cu putință să fim plăcuți lui Dumnezeu, căci cine se apropie de Dumnezeu trebuie să creadă că El este și că Se face răsplătitor celor care Îl caută.
\par 7 Prin credință, luând Noe înștiințare de la Dumnezeu despre cele ce nu se vedeau încă, a gătit, cu evlavie, o corabie spre mântuirea casei sale; prin credință el a osândit lumea și dreptății celei din credință s-a făcut moștenitor.
\par 8 Prin credință, Avraam, când a fost chemat, a ascultat și a ieșit la locul pe care era să-l ia spre moștenire și a ieșit neștiind încotro merge.
\par 9 Prin credință, a locuit vremelnic în pământul făgăduinței, ca într-un pământ străin, locuind în corturi cu Isaac și cu Iacov, cei dimpreună moștenitori ai aceleiași făgăduințe;
\par 10 Căci aștepta cetatea cu temelii puternice, al cărei meșter și lucrător este Dumnezeu.
\par 11 Prin credință, și Sara însăși a primit putere să zămislească fiu, deși trecuse de vârsta cuvenită, pentru că ea L-a socotit credincios pe Cel ce făgăduise.
\par 12 Pentru aceea, dintr-un singur om, și acela ca și mort, s-au născut atâția urmași - mulți "ca stelele cerului și ca nisipul cel fără de număr de pe țărmul mării".
\par 13 Toți aceștia au murit întru credință, fără să primească făgăduințele, ci văzându-le de departe și iubindu-le cu dor și mărturisind că pe pământ ei sunt străini și călători.
\par 14 Iar cei ce grăiesc unele ca acestea dovedesc că ei își caută lor patrie.
\par 15 Într-adevăr, dacă ar fi avut în minte pe aceea din care ieșiseră, aveau vreme să se întoarcă.
\par 16 Dar acum ei doresc una mai bună, adică pe cea cerească. Pentru aceea Dumnezeu nu Se rușinează de ei ca să Se numească Dumnezeul lor, căci le-a gătit lor cetate.
\par 17 Prin credință, Avraam, când a fost încercat, a adus pe Isaac (jertfă). Cel ce primise făgăduințele aducea jertfă pe fiul său unul născut!
\par 18 Către el grăise Dumnezeu: "Că în Isaac ți se va chema ție urmaș".
\par 19 Dar Avraam a socotit că Dumnezeu este puternic să-l învieze și din morți; drept aceea l-a dobândit înapoi ca un fel de pildă (a învierii) Lui.
\par 20 Prin credința despre cele viitoare a binecuvântat Isaac pe Iacov și pe Esau.
\par 21 Prin credință Iacov, când a fost să moară, a binecuvântat pe fiecare din fiii lui Iosif și s-a închinat, rezemându-se pe vârful toiagului său.
\par 22 Prin credință Iosif, la sfârșitul vieții, a pomenit despre ieșirea fiilor lui Israel și a dat porunci cu privire la oasele sale.
\par 23 Prin credință, când s-a născut Moise, a fost ascuns de părinții lui trei luni, căci l-au văzut prunc frumos și nu s-au temut de porunca regelui.
\par 24 Prin credință, Moise, când s-a făcut mare, n-a vrut să fie numit fiul fiicei lui Faraon,
\par 25 Ci a ales mai bine să pătimească cu poporul lui Dumnezeu, decât să aibă dulceața cea trecătoare a păcatului,
\par 26 Socotind că batjocorirea pentru Hristos este mai mare bogăție decât comorile Egiptului, fiindcă se uita la răsplătire.
\par 27 Prin credință, a părăsit Egiptul, fără să se teamă de urgia regelui, căci a rămas neclintit, ca cel care vede pe Cel nevăzut.
\par 28 Prin credință, a rânduit Paștile și stropirea cu sânge, ca îngerul nimicitor să nu se atingă de cei întâi-născuți ai lor.
\par 29 Prin credință au trecut israeliții Marea Roșie, ca pe uscat, pe care egiptenii, încercând și ei s-o treacă, s-au înecat.
\par 30 Prin credință, zidurile Ierihonului au căzut, după ce au fost înconjurate șapte zile.
\par 31 Prin credință Rahav, desfrânata, fiindcă primise cu pace iscoadele, n-a pierit împreună cu cei neascultători.
\par 32 Și ce voi mai zice? Căci timpul nu-mi va ajunge, ca să vorbesc de Ghedeon, de Barac, de Samson, de Ieftae, de David, de Samuel și de prooroci,
\par 33 Care prin credință, au biruit împărății, au făcut dreptate, au dobândit făgăduințele, au astupat gurile leilor,
\par 34 Au stins puterea focului, au scăpat de ascuțișul sabiei, s-au împuternicit, din slabi ce erau s-au făcut tari în război, au întors taberele vrăjmașilor pe fugă;
\par 35 Unele femei și-au luat pe morții lor înviați. Iar alții au fost chinuiți, neprimind izbăvirea, ca să dobândească mai bună înviere;
\par 36 Alții au suferit batjocură și bici, ba chiar lanțuri și închisoare;
\par 37 Au fost uciși cu pietre, au fost puși la cazne, au fost tăiați cu fierăstrăul, au murit uciși cu sabia, au pribegit în piei de oaie și în piei de capră, lipsiți, strâmtorați, rău primiți.
\par 38 Ei, de care lumea nu era vrednică, au rătăcit în pustii, și în munți, și în peșteri, și în crăpăturile pământului.
\par 39 Și toți aceștia, mărturisiți fiind prin credință, n-au primit făgăduința,
\par 40 Pentru că Dumnezeu rânduise pentru noi ceva mai bun, ca ei să nu ia fără noi desăvârșirea.

\chapter{12}

\par 1 De aceea și noi, având împrejurul nostru atâta nor de mărturii, să lepădăm orice povară și păcatul ce grabnic ne împresoară și să alergăm cu stăruință în lupta care ne stă înainte.
\par 2 Cu ochii ațintiți asupra lui Iisus, începătorul și plinitorul credinței, Care, pentru bucuria pusă înainte-I, a suferit crucea, n-a ținut seama de ocara ei și a șezut de-a dreapta tronului lui Dumnezeu.
\par 3 Luați aminte, dar, la Cel ce a răbdat de la păcătoși, asupra Sa, o atât de mare împotrivire, ca să nu vă lăsați osteniți, slăbind în sufletele voastre.
\par 4 În lupta voastră cu păcatul, nu v-ați împotrivit încă până la sânge.
\par 5 Și ați uitat îndemnul care vă grăiește ca unor fii: "Fiul meu, nu disprețui certarea Domnului, nici nu te descuraja, când ești mustrat de El.
\par 6 Căci pe cine îl iubește Domnul îl ceartă, și biciuiește pe tot fiul pe care îl primește".
\par 7 Răbdați spre înțelepțire, Dumnezeu se poartă cu voi ca față de fii. Căci care este fiul pe care tatăl său nu-l pedepsește?
\par 8 Iar dacă sunteți fără de certare, de care toți au parte, atunci sunteți fii nelegitimi și nu fii adevărați.
\par 9 Apoi dacă am avut pe părinții noștri după trup, care să ne certe, și ne sfiam de ei, oare nu ne vom supune cu atât mai vârtos Tatălui duhurilor, ca să avem viață?
\par 10 Pentru că ei, precum găseau cu cale, ne pedepseau pentru puține zile, iar Acesta, spre folosul nostru, ca să ne împărtășim de sfințenia Lui.
\par 11 Orice mustrare, la început, nu pare că e de bucurie, ci de întristare, dar mai pe urmă dă celor încercați cu ea roada pașnică a dreptății.
\par 12 Pentru aceea, "îndreptați mâinile cele ostenite și genunchii cei slăbănogiți.
\par 13 Faceți cărări drepte pentru picioarele voastre", așa încât cine este șchiop să nu se abată, ci mai vârtos să se vindece.
\par 14 Căutați pacea cu toți și sfințenia, fără de care nimeni nu va vedea pe Domnul,
\par 15 Veghind cu luare aminte ca nimeni să nu rămână lipsit de harul lui Dumnezeu și ca nu cumva, odrăslind vreo pricină de amărăciune, să vă tulbure, și prin ea mulți să se molipsească.
\par 16 Și să nu fie vreunul desfrânat sau întinat ca Esau, care pentru o mâncare și-a vândut dreptul de întâi născut.
\par 17 Știți că mai pe urmă, când a dorit să moștenească binecuvântarea, nu a fost luat în seamă, căci, deși cu lacrimi a căutat, n-a mai avut cum să schimbe hotărârea.
\par 18 Căci voi nu v-ați apropiat nici de muntele ce putea fi pipăit, nici de focul care ardea cu flacără, nici de nor, nici de beznă, nici de vijelie,
\par 19 Nici de glasul trâmbiței, nici de răsunetul cuvintelor despre care cei ce îl auzeau s-au rugat să nu li se mai grăiască,
\par 20 Deoarece nu puteau să sufere porunca: "Chiar dacă și fiară de s-ar atinge de munte, să fie ucisă cu pietre, sau să fie străpunsă cu săgeata",
\par 21 Și atât de înfricoșătoare era arătarea, încât Moise a zis: "Sunt înspăimântat și mă cutremur!".
\par 22 Ci v-ați apropiat de muntele Sion și de cetatea Dumnezeului celui viu, de Ierusalimul cel ceresc și de zeci de mii de îngeri, în adunare sărbătorească,
\par 23 Și de Biserica celor întâi născuți, care sunt scriși în ceruri și de Dumnezeu, Judecătorul tuturor, și de duhurile drepților celor desăvârșiți,
\par 24 Și de Iisus, Mijlocitorul noului testament, și de sângele stropirii care grăiește mai bine decât al lui Abel.
\par 25 Luați seama să nu vă lepădați de Cel care vorbește. Căci dacă aceia n-au scăpat de pedeapsă, nevoind să asculte pe cel ce le grăia pe pământ, cu atât mai mult noi - îndepărtându-ne de Cel ce ne grăiește din ceruri -
\par 26 Al Cărui glas, odinioară, a zguduit pământul, iar acum, vorbind, a făgăduit: "Încă o dată voi clătina nu numai pământul, ci și cerul".
\par 27 Iar prin aceea că zice: "Încă o dată" arată schimbarea celor clătinate, ca a unor lucruri făcute, ca să rămână cele neclintite.
\par 28 De aceea, fiindcă primim o împărăție neclintită, să fim mulțumitori, și așa să-I aducem lui Dumnezeu închinare plăcută, cu evlavie și cu sfială.
\par 29 Căci "Dumnezeul nostru este și foc mistuitor".

\chapter{13}

\par 1 Rămâneți întru dragostea frățească.
\par 2 Primirea de oaspeți să n-o uitați căci prin aceasta unii, fără ca să știe, au primit în gazdă, îngeri.
\par 3 Aduceți-vă aminte de cei închiși, ca și cum ați fi închiși cu ei; aduceți-vă aminte de cei ce îndură rele, întrucât și voi sunteți în trup.
\par 4 Cinstită să fie nunta întru toate și patul nespurcat. Iar pe desfrânați îi va judeca Dumnezeu.
\par 5 Feriți-vă de iubirea de argint și îndestulați-vă cu cele ce aveți, căci însuși Dumnezeu a zis: "Nu te voi lăsa, nici nu te voi părăsi".
\par 6 Pentru aceea, având bună îndrăzneală, să zicem: "Domnul este într-ajutorul meu; nu mă voi teme! Ce-mi va face mie omul?".
\par 7 Aduceți-vă aminte de mai-marii voștri, care v-au grăit vouă cuvântul lui Dumnezeu; priviți cu luare aminte cum și-au încheiat viața și urmați-le credința.
\par 8 Iisus Hristos, ieri și azi și în veci, este același.
\par 9 Nu vă lăsați furați de învățăturile străine cele de multe feluri; căci bine este să vă întăriți prin har inima voastră, nu cu mâncăruri, de la care n-au avut nici un folos cei ce au umblat cu ele.
\par 10 Avem altar, de la care nu au dreptul să mănânce cei ce slujesc cortului.
\par 11 Într-adevăr, trupurile dobitoacelor - al căror sânge e adus de arhiereu, pentru împăcare, în Sfânta Sfintelor - sunt arse afară din tabără.
\par 12 Pentru aceea și Iisus, ca să sfințească poporul cu sângele Său, a pătimit în afara porții.
\par 13 Deci dar să ieșim la El, afară din tabără, luând asupra noastră ocara Lui.
\par 14 Căci nu avem aici cetate stătătoare, ci o căutăm pe aceea ce va să fie.
\par 15 Așadar, prin El să aducem pururea lui Dumnezeu jertfă de laudă, adică rodul buzelor, care preaslăvesc numele Lui.
\par 16 Iar facerea de bine și întrajutorarea nu le dați uitării; căci astfel de jertfe sunt bine plăcute lui Dumnezeu.
\par 17 Ascultați pe mai-marii voștri și vă supuneți lor, fiindcă ei priveghează pentru sufletele voastre, având să dea de ele seamă, ca să facă aceasta cu bucurie și nu suspinând, căci aceasta nu v-ar fi de folos.
\par 18 Rugați-vă pentru noi; căci suntem încredințați că avem un cuget bun, dorind ca întru toate cu cinste să trăim.
\par 19 Și mai mult vă rog să faceți aceasta, ca să vă fiu dat înapoi mai curând.
\par 20 Iar Dumnezeul păcii, Cel ce, prin sângele unui testament veșnic, a sculat din morți pe Păstorul cel mare al oilor, pe Domnul nostru Iisus,
\par 21 Să vă întărească în orice lucru bun, ca să faceți voia Lui, și să lucreze în noi ceea ce este bine plăcut în fața Lui, prin Iisus Hristos, Căruia fie slava în vecii vecilor. Amin!
\par 22 Și vă rog, fraților, să îngăduiți acest cuvânt de îndemn, căci vi l-am scris pe scurt.
\par 23 Să știți că fratele Timotei este slobod. Dacă vine mai degrabă, vă voi vedea împreună cu el.
\par 24 Îmbrățișați pe toți mai-marii voștri și pe toți sfinții. Vă îmbrățișează cei din Italia.
\par 25 Harul fie cu voi cu toți! Amin.


\end{document}