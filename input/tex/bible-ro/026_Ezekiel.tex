\begin{document}

\title{Ezechiel}


\chapter{1}

\par 1 În anul al treizecilea, în ziua a cincea a lunii a patra, când mă aflam între robi, la râul Chebar, mi s-au deschis cerurile și am văzut niște vedenii dumnezeiești.
\par 2 În ziua a cincea a lunii a patra, în anul al cincilea ai robirii regelui Ioiachim,
\par 3 A fost cuvântul Domnului către mine, preotul Iezechiel, fiul lui Buzi, preotul, la râul Chebar, în țara Caldeilor. Acolo a fost peste mine mâna Domnului.
\par 4 Eu priveam și iată venea dinspre miazănoapte un vânt vijelios, un nor mare și un val de foc, care răspândea în toate părțile raze strălucitoare; iar în mijlocul focului strălucea ca un metal în văpaie.
\par 5 Și în mijloc am văzut ceva ca patru fiare, a căror înfățișare semăna cu chipul omenesc.
\par 6 Fiecare din ele avea patru fețe și fiecare din ele avea patru aripi.
\par 7 Picioarele lor erau drepte, iar copitele picioarelor lor erau cum sunt copitele picioarelor de vițel și scânteiau ca arama strălucitoare, iar aripile lor erau sprintene.
\par 8 De cele patru părți ele aveau sub aripi mâini de om și toate patru își aveau fețele lor și aripile lor.
\par 9 Aripile lor se atingeau una de alta, și când mergeau, fiarele nu se întorceau, ci fiecare mergea drept înainte.
\par 10 Fețele lor? - Toate patru aveau câte o față de om înainte, toate patru aveau câte o față de leu la dreapta, toate patru aveau câte o față de bou la stânga și toate patru mai aveau și câte o față de vultur în spate.
\par 11 Fețele lor și aripile lor erau despărțite în partea de sus, și, la fiecare, două din aripi erau întinse, iar două le acopereau trupul.
\par 12 Fiecare fiară mergea drept înainte și mergea încotro îi dădea duhul să meargă și în mersul său nu se întorcea.
\par 13 Înfățișarea acestor fiare se asemăna cu înfățișarea cărbunilor aprinși, cu înfățișarea unor făclii aprinse; printre fiare curgea foc, iar din foc țâșneau raze și fulgere.
\par 14 Fiarele alergau înainte și înapoi iute ca fulgerul.
\par 15 Când mă uitam eu la fiare, iată am văzut jos, lângă aceste fiare, câte o roată la fiecare din cele patru fețe ale lor.
\par 16 Aceste roți, după înfățișarea lor, parcă erau de crisolit, iar după făptură toate aveau aceeași înfățișare. Și după alcătuirea și după făptura lor ele parcă erau vârâte una în alta.
\par 17 Ele înaintau în toate cele patru părți, și în timpul mersului nu se întorceau.
\par 18 Obezile lor formau un cerc larg și de o înălțime înfricoșătoare și aceste obezi la toate patru erau pline de ochi de jur împrejur.
\par 19 Când mergeau fiarele, mergeau și roțile de lângă ele, și când se ridicau fiarele de la pământ, se ridicau și roțile.
\par 20 Ele mergeau încotro le da duhul să meargă și roțile se ridicau împreună, căci duh de viață era și în roți.
\par 21 Când mergeau acelea, mergeau și acestea, și când acelea se opreau, se opreau și acestea; iar când acelea se ridicau de la pământ, atunci împreună cu ele se ridicau și roțile, pentru că duh de viață era și în roți.
\par 22 Deasupra capetelor fiarelor se vedea un fel de boltă, întinsă sus, deasupra capetelor lor, care semăna cu cristalul cel mai curat;
\par 23 Iar sub bolta aceasta erau întinse aripile fiarelor una spre alta, și fiecare fiară mai avea câte două aripi, care le acopereau trupurile;
\par 24 Când mergeau fiarele, auzeam fâlfâitul aripilor lor, ca un vuiet de ape mari, ca glasul Celui Atotputernic zgomot strașnic, ca zgomotul dintr-un lagăr ostășesc; iar când ele se opreau, își lăsau aripile în jos.
\par 25 După ce fiarele se opreau și își lăsau aripile în jos, zgomotul se auzea încă sub bolta ce se întindea deasupra capetelor lor.
\par 26 Pe bolta de deasupra capetelor fiarelor era ceva care semăna cu un tron și la înfățișare era ca piatra de safir; iar sus pe acest tron era ca un chip de om.
\par 27 Și am mai văzut ceva, ca un metal înroșit în foc, ca niște foc, sub care se afla acel chip de om și care lumina împrejur; de la coapsele acelui chip de om în sus și de la coapsele chipului aceluia în jos se vedea un fel de foc, un fel de lumină strălucitoare care-l împresura de jur împrejur.
\par 28 Cum este curcubeul ce se află pe cer la vreme de ploaie, așa era înfățișarea acelei lumini strălucitoare care-l înconjura. Astfel era chipul slavei Domnului. Și când am văzut eu aceasta, am căzut cu fața la pământ.

\chapter{2}

\par 1 Atunci am auzit glasul Unuia care mi-a zis: "Fiul omului, scoală în picioare, că am să-ți vorbesc!"
\par 2 Și cum mi-a zis Acela vorbele acestea, a intrat Duhul în mine și m-a ridicat în picioare, și am ascultat pe Cel ce-mi vorbea.
\par 3 Acela mi-a zis: "Fiul omului, am să te trimit la fiii lui Israel, la acești oameni neascultători, care s-au răzvrătit împotriva Mea; ei, ca și părinții lor, păcătuiesc înaintea Mea până în ziua de astăzi.
\par 4 Acești fii sunt nerușinați și cu inima împietrită; la ei te trimit Eu, ca să le zici: Așa grăiește Domnul!
\par 5 Ori te-or asculta, ori nu te-or asculta - căci sunt un neam îndărătnic - să știe însă că este între ei un prooroc.
\par 6 Dar tu, fiul omului, să nu te temi de ei și de cuvintele lor să nu te sperii, deși ei vor fi pentru tine spini și ciulini, și ai să trăiești între ei, ca între scorpii; să nu te temi de cuvintele lor și de fața lor să nu te sperii, deși sunt un neam îndărătnic,
\par 7 Ci să le spui cuvintele Mele, ori te-or asculta, ori nu te-or asculta, căci sunt un neam de răzvrătiți.
\par 8 Tu însă, fiul omului, ascultă ce voiesc să-ți spun! Nu fi îndărătnic, ca acest neam de răzvrătiți! Deschide-ți gura și mănâncă ceea ce am să-ți dau!"
\par 9 Și privind eu, am văzut o mână întinsă spre mine și în ea o hârtie strânsă sul;
\par 10 Și a desfășurat-o înaintea mea, și am văzut că era scrisă și pe o parte și pe alta: plângere, tânguire și jale era scris în ea.

\chapter{3}

\par 1 Apoi mi-a zis: "Fiul omului, mănâncă ceea ce ai dinainte, mănâncă această hârtie și mergi de grăiește casei lui Israel!"
\par 2 Atunci eu mi-am deschis gura și Acela mi-a dat să mănânc cartea aceea,
\par 3 Și mi-a zis: "Fiul omului, hrănește-ți pântecele și-ți satură lăuntrul tău cu această carte pe care ți-o dau Eu!" Și eu am mâncat-o și era în gura mea dulce ca mierea.
\par 4 Apoi Acela mi-a zis: "Fiul omului, scoală și mergi la casa lui Israel și le spune cuvintele Mele;
\par 5 Căci nu ești trimis la un popor cu grai necunoscut și cu limbă neînțeleasă, ci către casa lui Israel;
\par 6 Nici la mai multe popoare cu grai necunoscut și cu limbă neînțeleasă, ale căror cuvinte nu le-ai pricepe. Și chiar la unele ca acestea de te-aș trimite, ele tot te-ar asculta;
\par 7 Casa lui Israel însă nu va vrea să te asculte, pentru că nu vrea să Mă asculte pe Mine, că toată casa lui Israel are fruntea încruntată și inima împietrită.
\par 8 Pentru aceasta voi împietri chipul tău ca și chipurile lor și fruntea ta întocmai ca frunțile lor;
\par 9 Voi face fruntea ta ca diamantul, mai tare decât stânca. Să nu te temi de ei și de fața lor să nu te sperii, căci ei sunt un neam de răzvrătiți.
\par 10 Fiul omului - îmi zice iar Acela, - primește în inima ta și ascultă cu urechile tale toate cuvintele ce am să-ți vorbesc.
\par 11 Scoală și mergi la cei duși în robie, la fiii poporului tău și, ori te-or asculta, ori nu te-or asculta, tu grăiește-le și le spune: Așa zice Domnul Dumnezeu".
\par 12 Atunci m-a ridicat Duhul și am auzit la spatele meu un glas mare ca de tunet care zicea: "Binecuvântată fie slava Domnului în locul unde sălășluiește El!"
\par 13 Și am mai auzit zgomotul fiarelor care băteau din aripi și huruitul roților de lângă ele și bubuit puternic de tunet.
\par 14 Și Duhul m-a luat și m-a dus. Și am mers eu amărât și cu sufletul întristat, dar mâna Domnului lucra puternic asupra mea.
\par 15 Apoi am sosit în Tel-Aviv, la robii care locuiau aproape de râul Chebar, și m-am oprit acolo unde trăiau ei și am stat între ei șapte zile în uimire.
\par 16 Iar după ce s-au împlinit cele șapte zile, a fost către mine cuvântul Domnului și mi-a zis:
\par 17 "Fiul omului! Iată, te-am pus străjer casei lui Israel; vei asculta deci cuvântul ce-Mi va ieși din gură și-l vei vesti ca din partea Mea.
\par 18 De voi zice celui rău: Vei muri! și tu nu-l vei înștiința, nici nu-i vei grăi, pentru a abate pe cel rău de la calea lui cea rea, ca să trăiască, cel rău va pieri în nelegiuirea sa și Eu voi cere sângele lui din mâna ta.
\par 19 Iar dacă tu vei înștiința pe cel rău și el nu se va întoarce de la răutatea lui și de la calea sa cea rea, acela va pieri de păcatul său, iar tu îți vei mântui sufletul tău.
\par 20 De asemenea, când cel drept se va abate de la dreptatea sa și va săvârși răul când voi pune înaintea lui o cursă și va muri, dacă tu nu l-ai înștiințat, acela va muri pentru păcatul său, iar faptele lui de dreptate, pe care le-a făcut el, nu i se vor pomeni, și Eu voi cere sângele lui din mâinile tale.
\par 21 Iar dacă tu vei înștiința pe cel drept să nu păcătuiască și el nu va păcătui, atunci va fi și el viu, pentru că a fost înștiințat și îți vei mântui și tu sufletul tău".
\par 22 Apoi a fost iarăși acolo mâna Domnului peste mine și mi-a zis Domnul: "Scoală-te și ieși la câmp, că am să-ți vorbesc acolo".
\par 23 Atunci m-am sculat, am ieșit la câmp, și iată mi s-a arătat acolo slava Domnului, pe care o văzusem la râul Chebar, și am căzut cu fața la pământ.
\par 24 Dar a intrat în mine Duhul și m-a ridicat în picioare, iar Domnul mi-a grăit și mi-a zis: "Mergi și te închide în casa ta!
\par 25 Fiul omului, iată se vor pune asupra ta frânghii cu care vei fi legat, ca să nu mai ieși în mijlocul lor.
\par 26 Și limba ta o voi lipi de cerul gurii tale, ca să fii mut și să nu-i mai poți mustra, că aceștia sunt un neam răzvrătit.
\par 27 Dar când îți voi grăi, voi deschide gura ta și tu le vei zice: Așa grăiește Domnul Dumnezeu! Cine va vrea să asculte, să asculte, și cine nu va vrea să asculte, să nu asculte, căci sunt un neam îndărătnic".

\chapter{4}

\par 1 "Și tu, fiul omului, ia-ți o cărămidă și pune-o înaintea ta și sapă pe ea o cetate, Ierusalimul.
\par 2 Apoi rânduiește împresurare împotriva ei, ridică întărituri și val și rânduiește tabără împotriva ei; de jur împrejurul ei așază berbeci de spart zidul.
\par 3 Apoi ia o tablă de fier și o pune ca un zid de fier între tine și cetate și întoarce-ți fața spre ea, ca și cum ea ar fi împresurată, iar tu cel care o împresori. Aceasta va fi un semn pentru casa lui Israel.
\par 4 După aceea să te culci pe partea stângă, punând pe ea nelegiuirile casei lui Israel, și vei purta nelegiuirile ei atâtea zile, cât vei sta culcat pe partea stângă;
\par 5 Că Eu am să-ți număr atâtea zile câți ani au ținut nelegiuirile ei, adică ai să porți tu nelegiuirile casei lui Israel trei sute nouăzeci de zile.
\par 6 Iar după ce vei împlini aceste zile, să te culci pe partea dreaptă și vei purta nelegiuirile casei lui Iuda timp de patruzeci de zile, câte o zi de fiecare an ce ți-am hotărât Eu.
\par 7 Să-ți îndrepți fața ta și mâna ta cea dreaptă dezgolită spre Ierusalimul împresurat și să proorocești împotriva lui.
\par 8 Și iată, Eu voi pune pe tine legături, ca să nu te poți întoarce de pe o parte pe alta, până nu vei împlini zilele împresurării tale.
\par 9 Ia-ți grâu și orz, bob și linte, mei și orzoaică, toarnă-le într-un vas și-ți fă din ele atâtea pâini, câte zile ai să șezi culcat pe partea stângă; că trei sute nouăzeci de zile ai să mănânci din ele.
\par 10 Hrana ta, cu care te vei hrăni, să o mănânci cu măsură, câte douăzeci de sicli pe zi, dar s-o mănânci din timp în timp.
\par 11 Și apa să o bei cu măsură, câte a șasea parte de hin pe zi și tot din timp în timp.
\par 12 Pâinile le vei găti ca turtele de orz și le vei coace înaintea ochilor lor cu necurățenie de om".
\par 13 Apoi Domnul a mai zis: "Așa își vor mânca fiii lui Israel pâinea lor, necurată, printre popoarele la care îi voi izgoni".
\par 14 Atunci am zis: "O, Doamne Dumnezeule, sufletul meu niciodată nu s-a întinat; din tinerețile mele și până acum niciodată n-am mâncat dintr-un animal mort sau sfâșiat și carne necurată n-a intrat în gura mea".
\par 15 Iar El mi-a răspuns: "Iată, Eu îți dau, în loc de necurățenie de om, baligă de bou și pe ea vei găti pâinea ta".
\par 16 Apoi mi-a zis: "Fiul omului, iată voi trimite foamete în Ierusalim și oamenii vor mânca pâinea cu măsură și cu întristare și tot cu măsură și amărăciune vor bea și apă,
\par 17 Căci va fi la ei lipsă de pâine și de apă; se vor uita cu groază unul la altul și vor pieri pentru nelegiuirile lor".

\chapter{5}

\par 1 "Iar tu, fiul omului, ia-ți un cuțit tăios, ia-ți un brici de bărbierit și-l trece pe capul tău și pe barba ta; apoi ia o cumpănă cu talgere și împarte părul în trei părți:
\par 2 O treime să o arzi cu foc în mijlocul cetății, după ce se vor împlini zilele împresurării; o altă treime s-o iei și s-o toci cu cuțitul în împrejurimile cetății; iar cealaltă treime s-o spulberi în vânt, și eu voi trage sabia în urma lor.
\par 3 Să oprești însă din părul acela puține fire și să le legi în poala hainei tale.
\par 4 Apoi să iei câteva fire, să le arunci în foc și să le arzi. Din acestea va izbucni foc împotriva întregii case a lui Israel.
\par 5 Apoi vei zice către casa lui Israel: Așa grăiește Domnul Dumnezeu: Acesta este Ierusalimul, pe care Eu l-am pus în mijlocul neamurilor și al țărilor dimprejur.
\par 6 Dar el s-a răzvrătit împotriva hotărârilor Mele mai mult decât neamurile și împotriva legilor Mele mai mult decât țările care îl înconjoară, căci el a lepădat legile Mele și poruncile Mele nu le-a urmat.
\par 7 De aceea așa zice Domnul Dumnezeu: Pentru că tu ai înmulțit nelegiuirile mai mult decât neamurile dimprejurul tău și poruncile Mele nu le urmezi, nici nu împlinești legile Mele, ba nici după legile neamurilor celor dimprejurul tău nu te porți,
\par 8 De aceea așa zice Domnul Dumnezeu: Iată, Eu vin împotriva ta și în mijlocul tău voi rosti osânda ta, în fala neamurilor.
\par 9 Pentru toate spurcăciunile tale îți voi face ceea ce niciodată n-am făcut până acum și nici nu voi mai face niciodată.
\par 10 De aceea părinții își vor mânca pe copii în mijlocul tău și copiii își vor mânca părinții lor; voi aduce asupra ta osândă și toate rămășițele tale le voi spulbera în toate vânturile,
\par 11 Pentru că tu ai spurcat locașul Meu cel sfânt cu toți idolii tăi și cu toate ticăloșiile tale, de aceea zice Domnul Dumnezeu: Precum este adevărat că Eu sunt viu, tot așa este de adevărat că te voi micșora, ochiul Meu nu te va cruța și nici nu te va milui.
\par 12 O treime din locuitorii tăi vor muri de ciumă și vor pieri de foame în mijlocul tău; o treime din ei vor cădea de sabie în împrejurimile tale; și cealaltă treime o voi împrăștia în toate vânturile și voi trage sabia în urma lor.
\par 13 Așa-Mi voi împlini mânia, Îmi voi potoli urgia Mea cu ei și Mă voi răzbuna; și când se va săvârși urgia Mea asupra lor, vor cunoaște că Eu, Domnul, am grăit în râvna Mea.
\par 14 Pustietate te voi face și de ocară între popoarele cele dimprejurul tău și în fața tuturor trecătorilor.
\par 15 Vei fi de râs și de batjocură, de pildă și de groază la popoarele cele dimprejurul tău, când voi săvârși asupra ta judecățile Mele cu mânie, cu urgie și cu pedepse aspre: Eu, Domnul, am zis aceasta.
\par 16 Când voi slobozi împotriva voastră săgețile cele ucigătoare ale foametei, care vor semăna moartea, când le voi slobozi spre pieirea voastră și voi întări foametea între voi și voi zdrobi tot paiul de grâu,
\par 17 Eu voi trimite împotriva voastră foametea și fiare rele care te vor lipsi de copii; ciumă și sânge vor trece peste tine și sabie voi aduce împotriva ta: Eu, Domnul, grăiesc acestea".

\chapter{6}

\par 1 Fost-a către mine cuvântul Domnului și mi-a zis:
\par 2 "Fiul omului, întoarce-ți fața ta spre munții lui Israel, proorocește împotriva lor și zi:
\par 3 Munți ai lui Israel, ascultați cuvântul Domnului Dumnezeu! Așa grăiește Domnul Dumnezeu munților și dealurilor, șesurilor și văilor: Iată, Eu voi aduce asupra voastră sabie și voi dărâma înălțimile voastre;
\par 4 Altarele voastre vor fi stricate, idolii ridicați de voi în cinstea soarelui vor fi sfărâmați și pe oamenii voștri îi voi face să cadă morți înaintea idolilor voștri;
\par 5 Trupurile fiilor lui Israel le voi pune înaintea idolilor lor și oasele lor le voi risipi împrejurul altarelor lor.
\par 6 În toate așezările voastre cele locuite, cetățile vor fi nimicite și locurile înalte ale voastre pustiite, altarele voastre vor fi stricate și pustiite, idolii voștri vor fi sfărâmați și nimiciți, stâlpii voștri închinați soarelui vor fi sfărâmați și lucrările voastre vor fi ruinate.
\par 7 Cei uciși vor cădea între voi și veți cunoaște că Eu sunt Domnul.
\par 8 Dar Eu voi lăsa o rămășiță din voi, care va scăpa de sabie printre celelalte neamuri, când veți fi risipiți prin țări.
\par 9 Aceia din voi, care vor scăpa, își vor aduce aminte de Mine printre neamurile unde vor fi duși în robie, pentru că voi umili inima lor cea desfrânată, care s-a abătut de la Mine, și ochii lor, care s-au desfrânat cu idolii, se vor scârbi de ei înșiși, din pricina acelor rele pe care le-au făcut și a tuturor ticăloșiilor lor;
\par 10 Și vor cunoaște că Eu sunt Domnul și că nu în zadar i-am amenințat că le voi trimite toate aceste rele".
\par 11 Așa grăiește Domnul Dumnezeu: "Lovește palmă de palmă, bate cu piciorul și zi: Vai de casa lui Israel, că are să cadă de sabie, de foamete și de ciumă, pentru toate nelegiuirile sale!
\par 12 Cel ce va fi departe va muri de ciumă; cel ce va fi aproape va muri de sabie, iar cel ce va scăpa de acestea și va rămâne va muri de foame. Așa-Mi voi potoli mânia Mea împotriva lor.
\par 13 Când ucișii vor zăcea printre idolii lor și împrejurul altarelor lor, pe tot locul înalt și pe toate vârfurile de munte, sub tot pomul verde, sub tot stejarul umbros și în tot locul, unde aduceau tămâie mirositoare tuturor idolilor lor, veți cunoaște că Eu sunt Domnul.
\par 14 Și-Mi voi întinde mâna asupra lor și voi face țara peste tot unde locuiesc ei mai pustie și mai deșartă decât pustiul Diblat, și vor cunoaște că Eu sunt Domnul".

\chapter{7}

\par 1 Fost-a către mine cuvântul Domnului și mi-a zis:
\par 2 "Și tu, fiul omului, spune: Așa grăiește Domnul Dumnezeul țării lui Israel: Vine, vine sfârșitul asupra celor patru laturi ale pământului.
\par 3 Iată îți vine sfârșitul! Trimite-voi împotriva ta mânia Mea și te voi judeca după căile tale și după toate ticăloșiile tale te voi pedepsi.
\par 4 Ochiul Meu nu te va cruța, nici nu te va milui, ci-ți voi răsplăti după căile tale, ticăloșiile tale vor fi peste tine și vei cunoaște că Eu sunt Domnul".
\par 5 Așa grăiește Domnul Dumnezeu: "Iată vine nenorocire peste nenorocire:
\par 6 Sfârșitul! A venit sfârșitul! E lângă tine! Iată-l a sosit!
\par 7 A sosit nenorocire asupra ta, locuitor al țării! Vine vremea, vine ziua tulburării, iar nu a chiuiturilor de bucurie prin munți.
\par 8 Iată vin acum îndată să vărs asupra ta urgia Mea, mânia Mea să o potolesc asupra ta și te voi judeca după căile tale și după toate ticăloșiile tale te voi pedepsi.
\par 9 Ochiul Meu nu te va cruța, nici nu te va milui, ci îți vor plăti după căile tale; ticăloșiile tale vor fi peste tine și vei cunoaște că Eu sunt Domnul Care lovește.
\par 10 Iată ziua! Iat-o că vine! Rândul tău a venit! Vine urgia! Toiagul înflorește! Trufia se deschide,
\par 11 Și se ridică silnicia ca să slujească de toiag pentru răutate; nimic nu va rămâne din ei, nici din bogățiile lor, nici din petrecerile lor, nici din strălucirea lor!
\par 12 Vine vremea, a sosit ziua! Cel ce cumpără sămânță să nu se bucure și cel ce vinde să nu plângă, căci mânia vine peste toată mulțimea lor!
\par 13 Căci cel ce vinde nu se va mai întoarce la cele vândute, măcar de ar și rămânea printre cei vii, căci proorocia îndreptată împotriva a toată mulțimea lor nu va fi schimbată și nici nu-și va întări viața sa prin nelegiuirile sale.
\par 14 Sună din trâmbițe și totul e gata, dar nimeni nu va merge la război, pentru că mânia Mea este împotriva a toată mulțimea lor.
\par 15 Afară va fi sabie, iar înăuntru ciumă și foamete. Cel din câmp va muri de sabie și pe cel din cetate îl va mânca ciuma și foametea.
\par 16 Iar care vor fugi, vor scăpa și vor fi prin munți ca niște porumbei rătăciți. Toți vor geme, fiecare pentru nelegiuirea sa.
\par 17 La toți le vor tremura mâinile, și picioarele tuturor se vor muia ca apa.
\par 18 Atunci cu sac se vor îmbrăca și groază îi va cuprinde; toți vor avea rușinea pe fețe și pe cap pleșuvia.
\par 19 Argintul și-l vor arunca pe ulițe și vor disprețui aurul; argintul lor și aurul lor nu-i vor putea scăpa în ziua urgiei Domnului; cu acestea ei nu vor putea să-și sature sufletele, nici să-și umple pântecele lor, pentru că acestea au fost pricină a nelegiuirilor lor.
\par 20 Prin găteli frumoase ei le-au prefăcut pe acestea în mândrie și tot din acestea au făcut ei chipurile cele rușinoase ale idolilor lor. De aceea le voi face pe acestea necurate pentru ei.
\par 21 Și le voi da pradă în mâinile străinilor și de jaf nelegiuiților pământului, care le vor spurca.
\par 22 Îmi voi întoarce fața de la ei și jefuitorii vor întina locașul Meu cel sfânt, pentru că vor intra în el și-l vor spurca.
\par 23 Pregătește lanțuri, că țara aceasta e mânjită de nelegiuiți cu sânge, iar cetatea e plină de silnicii.
\par 24 De aceea vai aduce pe cei mai răi dintre neamuri ca să pună stăpânire pe casele lor. Voi pune capăt trufiei celor puternici și cele sfinte ale lor vor fi întinate.
\par 25 Iată vine pieirea și var căuta pacea, dar nu o vor găsi.
\par 26 Va veni nenorocire peste nenorocire și zvon peste zvon, și oamenii vor cere vedenii de la prooroc; dar preotului îi va lipsi cunoștința legii și bătrânului sfatul.
\par 27 Regele va plânge, căpetenia va fi cuprinsă de groază, iar mâinile poporului țării vor tremura. Mă voi purta cu ei după purtările lor și după judecățile lor îi voi judeca și vor cunoaște că Eu sunt Domnul".

\chapter{8}

\par 1 În anul al șaselea de la robirea regelui Ioiachim, în cinci ale lunii a șasea, pe când ședeam eu în casa mea și bătrânii lui Iuda ședeau înaintea mea, s-a lăsat peste mine mâna Domnului Dumnezeu.
\par 2 Și privind eu, am văzut un chip ca de om, de foc parcă; și parcă de la brâul lui în jos era foc, iar de la brâul lui în sus era o strălucire, ca de metal în văpaie.
\par 3 Și a întins, parcă, acela un fel de mână, și m-a apucat de părul capului meu și m-a ridicat Duhul între pământ și cer, și m-a dus, în vedenii dumnezeiești, la Ierusalim, la intrarea porții dinăuntru, îndreptată spre miazănoapte, unde era așezat idolul geloziei care stârnește gelozia.
\par 4 Și iată acolo era slava Dumnezeului lui Israel asemenea aceleia pe care o văzusem eu în câmp.
\par 5 Atunci mi-a zis Domnul: "Fiul omului, ridică-ți ochii spre miazănoapte!" Și mi-am ridicat ochii spre miazănoapte și iată acel idol al geloziei era la ușa dinspre miazănoapte a altarului, la intrare.
\par 6 Și mi-a zis Domnul: "Fiul omului, vezi ce fac ei? Vezi tu ce urâciuni mari face casa lui Israel aici, ca să Mă îndepărtez de locașul Meu cel sfânt? Dar întoarce-te și urâciuni și mai mari vei vedea!"
\par 7 Apoi m-a dus pe poartă în curte și privind, am văzut o spărtură în perete.
\par 8 Și mi-a zis Domnul: "Fiul omului, sapă în perete!" Și am săpat în perete și iată am dat de un fel de ușă.
\par 9 Și mi-a zis Domnul: "Intră și vezi urâciunile cele dezgustătoare pe care le fac aceștia aici".
\par 10 Și am intrat și am privit și iată erau acolo tot felul de chipuri de târâtoare, de animale necurate și de tot felul de idoli de ai casei lui Israel, zugrăviți pe pereți de jur împrejur.
\par 11 Înaintea lor stăteau șaptezeci de bărbați din bătrânii casei lui Israel, având în mijloc pe Iaazania, fiul lui Șafan; fiecare din ei avea în mâini câte o cădelniță și un nor gros de fum de tămâie se ridica în sus.
\par 12 Și mi-a zis Domnul: "Fiul omului, vezi ce fac bătrânii casei lui Israel la întuneric, stând fiecare în cămara sa plină de chipuri? Că își zic: Domnul nu ne vede! A părăsit țara Sa".
\par 13 Apoi mi-a zis Domnul: "Întoarce-te și vei vedea urâciuni încă și mai mari, pe care le fac ei".
\par 14 Și m-a dus la ușa cea dinspre miazănoapte a templului Domnului, și iată acolo ședeau niște femei, care plângeau pe Tamuz.
\par 15 Și mi-a zis Domnul: "Vezi, fiul omului? Întoarce-te și vei vedea încă și mai mari urâciuni!"
\par 16 Apoi m-a dus în curtea cea dinăuntru a templului Domnului și iată la ușa templului Domnului, între pridvor și jertfelnic, stăteau vreo douăzeci și cinci de oameni cu spatele spre templul Domnului, iar cu fețele spre răsărit și se închinau spre răsărit la soare.
\par 17 Și mi-a zis Domnul: "Vezi, fiul omului? Nu i-a ajuns casei lui Iuda să-și facă astfel de urâciuni, ca acele pe care le fac aceștia aici, ci au umplut și țara de necredință, îndoit mâniindu-Mă. Iată ei apropie ramuri de nările lor.
\par 18 De aceea și Eu voi lucra cu urgie; ochiul Meu nu-i va cruța, și Eu nu Mă voi îndura. Chiar de ar striga ei cu glas mare la urechile Mele, nu-i voi auzi".

\chapter{9}

\par 1 Apoi a răsunat la urechile mele un glas mare și a zis: "Apropiați-vă pedepsitorii cetății, având fiecare în mână unealta de nimicire!"
\par 2 Și iată dinspre poarta de sus, care dă spre miazănoapte, veneau șase bărbați, având fiecare în mână unealta sa ucigătoare; și între ei se afla unul, îmbrăcat cu haină de in, care avea la brâu unelte de scris. Aceștia, venind, s-au oprit lângă jertfelnicul cel de aramă.
\par 3 Atunci slava Dumnezeului lui Israel s-a pogorât de pe heruvimul pe care se afla la pragul casei. Și a chemat Domnul pe omul cel îmbrăcat în haină de in, care avea la brâu unelte de scris,
\par 4 Și i-a zis Domnul: Treci prin mijlocul cetății, prin Ierusalim, și însemnează cu semnul crucii (litera "tau" care în alfabetul vechi grec avea forma unei cruci) pe frunte, pe oamenii care gem și care plâng din cauza multor ticăloșii care se săvârșesc în mijlocul lui".
\par 5 Iar celorlalți le-a zis în auzul meu: "Mergeți după el prin cetate și loviți! Să nu aveți nici o milă și ochiul vostru să fie necruțător!
\par 6 Ucideți și nimiciți pe bătrâni, tineri, fecioare, copii, femei, dar să nu vă atingeți de nici un om, care are pe frunte semnul "+"! Și să începeți cu locul Meu cel sfânt!" Și au început ei cu bătrânii, care erau înaintea templului.
\par 7 Apoi le-a zis: "Întinați templul, umpleți curțile cu uciși și ieșiți!" Și au ieșit și au început să ucidă prin cetate.
\par 8 Și după ce i-au ucis, iar eu am rămas, am căzut cu fața la pământ și, strigând, am zis: "O, Doamne, Dumnezeule, vei pierde oare tot ce a mai rămas din Israel, vărsându-Ți mânia asupra Ierusalimului?"
\par 9 Iar El mi-a răspuns: "Nelegiuirea casei lui Israel și a lui Iuda este mare, foarte mare și țara aceasta e mânjită cu sânge, iar cetatea e plină de nedreptate, că ei zic: "A părăsit Domnul țara aceasta, Domnul nu mai vede!
\par 10 De aceea ochiul Meu nu-i va cruța și Eu nu Mă voi îndura și voi întoarce purtarea lor asupra capului lor".
\par 11 Și iată omul cel îmbrăcat cu haină de in, care avea la brâu uneltele de scris, a dat răspuns și a zis: "Am făcut cum mi-ai poruncit".

\chapter{10}

\par 1 Privind eu atunci, am văzut pe bolta, care era deasupra capetelor heruvimilor, ceva asemănător la înfățișare cu un tron de rege, ca piatra de safir.
\par 2 Și a zis Domnul către omul cel îmbrăcat în haina de in: "Intră între roțile cele de sub heruvimi și umple-ți pumnii de cărbuni aprinși, pe care-i vei lua dintre heruvimi, și-i aruncă asupra cetății!" Și el a intrat acolo înaintea ochilor mei;
\par 3 Și când a intrat omul acela, heruvimii stăteau în partea dreaptă a casei și un nor umplea curtea cea dinăuntru.
\par 4 Atunci s-a ridicat slava Domnului de pe heruvimi spre pragul templului și templul s-a umplut de nori, iar curtea s-a umplut de strălucirea slavei Domnului.
\par 5 Freamătul aripilor heruvimilor se auzea până și în curtea de afară, ca glasul când vorbește Dumnezeu Atotputernicul.
\par 6 Când a dat Domnul poruncă omului celui îmbrăcat cu haină de in și i-a zis: "Ia foc dintre roțile cele dintre heruvimi", și când a intrat el și a stat lângă roți,
\par 7 Atunci unul din heruvimi și-a întins mâna în focul cel dintre heruvimi, a luat și a dat în pumni celui îmbrăcat în haină de in, iar el, luându-l, a ieșit.
\par 8 Și la heruvimi sub aripi se vedeau un fel de mâini omenești.
\par 9 Căutând eu atunci, am văzut lângă heruvimi patru roți, câte o roată lângă fiecare heruvim; roțile acestea erau la vedere ca și crisolitul.
\par 10 După făptură erau toate la fel și parcă erau vârâte una în alta.
\par 11 Când ele mergeau, mergeau în toate patru părțile, și în vremea mersului nu se întorceau, ci încotro era îndreptat capul heruvimului, într-acolo mergeau, și în vremea mersului nu se întorceau.
\par 12 Tot trupul heruvimilor, spatele lor, mâinile lor și aripile lor erau pline de ochi; asemenea și roțile, toate cele patru roți de jur împrejur...
\par 13 Roților acestora, după cum am auzit eu, li s-a zis: "Galgal" (Vijelie).
\par 14 Fiecare din ființele acestea avea patru fețe: fața întâi era față de heruvim, fața a doua era față de om, cea de a treia era față de leu și cea de a patra era față de vultur.
\par 15 Atunci heruvimii s-au ridicat; ei erau aceleași fiare pe care le văzusem la râul Chebar.
\par 16 Când mergeau heruvimii, mergeau și roțile pe lângă ei, iar când heruvimii își ridicau aripile ca să se ridice de la pământ, nici roțile nu se despărțeau, ci erau împreună cu ei.
\par 17 Când aceia stăteau, stăteau și acestea; iar când se ridicau aceia, și acestea se ridicau, pentru că duhul fiarelor era și în ele.
\par 18 Atunci slava Domnului s-a dus de la prag și s-a așezat pe heruvimi.
\par 19 Iar heruvimii, întinzând aripile, s-au ridicat de la pământ înaintea ochilor mei și s-au dus însoțiți de roți și s-au oprit la poarta cea dinspre răsărit a templului Domnului, și slava Dumnezeului lui Israel era deasupra lor.
\par 20 Aceștia erau aceleași fiare, pe care le văzusem eu la picioarele Dumnezeului lui Israel, la râul Chebar.
\par 21 Fiecare avea câte patru fețe și fiecare avea câte patru aripi, iar sub aripi aveau un fel de mâini omenești.
\par 22 Chipul fețelor și înfățișarea lor era ca acelea pe care le văzusem la râul Chebar, ba și ei înșiși erau aceiași. Fiecare mergea în partea înspre care era cu fața.

\chapter{11}

\par 1 Atunci m-a ridicat Duhul și m-a dus la poarta de răsărit a templului Domnului, care este spre răsărit. Și iată în poartă, la intrare, erau douăzeci și cinci de oameni și în mijlocul lor am văzut pe Iaazania, fiul lui Azur, și pe Pelatia, fiul lui Benaia, căpeteniile poporului.
\par 2 Și mi-a zis Domnul: "Fiul omului, iată oamenii care gândesc nelegiuirea și care dau sfaturi rele în cetatea aceasta,
\par 3 Zicând: "Încă n-a venit vremea să ne zidim case; cetatea este cazanul, iar noi carnea".
\par 4 De aceea proorocește, fiul omului, proorocește împotriva lor!"
\par 5 Atunci S-a pogorât peste mine Duhul Domnului și mi-a zis: "Spune: Așa grăiește Domnul: Casa lui Israel, cele ce ziceți voi și cele ce vă vin în minte le știu.
\par 6 Mulți ați ucis voi în cetatea aceasta și ați umplut ulițele ei de trupuri.
\par 7 De aceea așa zice Domnul Dumnezeu: Ucișii pe care i-ați îngropat în ea sunt carnea, iar ea e căldarea. Pe voi însă vă voi scoate din ea.
\par 8 Voi vă temeți de sabie, și sabie voi aduce asupra voastră, zice Domnul Dumnezeu.
\par 9 Vă voi scoate din ea, vă voi da pe mâna străinilor și vă voi judeca.
\par 10 De sabie veți cădea; la hotarele lui Israel vă voi judeca, și veți cunoaște că Eu sunt Domnul.
\par 11 Cetatea aceasta nu va fi pentru voi cazan, nici voi nu veți fi pentru ea carne; la hotarele lui Israel vă voi judeca.
\par 12 Și veți cunoaște că Eu sunt Domnul, că nu v-ați purtat după poruncile Mele, nici legile Mele nu le-ați împlinit, ci v-aii purtat după obiceiurile neamurilor care vă înconjoară".
\par 13 Pe când prooroceam eu, Pelatia, fiul lui Benaia, a murit. Atunci eu am căzut cu fața la pământ și am strigat cu glas mare, zicând: "O, Doamne Dumnezeule, oare voiești să pierzi ce a mai rămas lui Israel?"
\par 14 Atunci a fost către mine cuvântul Domnului și mi-a zis:
\par 15 "Fiul omului, aceștia sunt frații tăi, frații tăi de un sânge cu tine, și toată casa lui Israel, cărora locuitorii Ierusalimului le zic: "Depărtați-vă de Domnul! Țara asta nouă ni s-a dat de moștenire".
\par 16 La aceasta să zici: Așa zice Domnul Dumnezeu: Deși i-am depărtat printre popoare și deși i-am risipit prin țări, totuși voi fi pentru ei un locaș sfânt în acele țări unde îi voi risipi.
\par 17 După aceea zi: Așa grăiește Domnul Dumnezeu: Vă voi aduna de prin popoare și vă voi întoarce de prin țările unde sunteți risipiți și vă voi da pământul lui Israel.
\par 18 Atunci vor veni acolo și vor depărta din el toate urâciunile și toți idolii.
\par 19 Și le voi da aceeași inimă și duh nou voi pune în ei; voi scoate din trupul lor inima cea de piatră și le voi da inimă de carne,
\par 20 Ca să urmeze poruncile Mele și legile să le păzească și să le împlinească; vor fi poporul Meu, iar Eu le voi fi Dumnezeu.
\par 21 Celor a căror inimă este legată de idolii lor și de urâciunile lor le voi cere socoteală de purtarea lor", zice Domnul Dumnezeu.
\par 22 Atunci heruvimii și-au întins aripile și roțile erau lângă ei, iar slava Dumnezeului lui Israel, sus, deasupra lor.
\par 23 Apoi slava Domnului s-a ridicat din mijlocul cetății și a stat deasupra muntelui, care se află spre răsărit de cetate.
\par 24 Atunci iar m-a luat Duhul și m-a dus în Caldeea, la cei robiți, cuprins de vedeniile ce le aveam prin Duhul lui Dumnezeu. Și vedenia pe care o văzusem s-a depărtat de la mine.
\par 25 Și am spus celor ce au fost duși în robie toate cuvintele Domnului pe care mi le descoperise El.

\chapter{12}

\par 1 Fost-a către mine cuvântul Domnului și mi-a zis:
\par 2 "Fiul omului, trăiești în mijlocul unui neam răzvrătit. Aceștia au ochi ca să vadă, dar nu văd; au urechi ca să audă, dar nu aud, pentru că sunt un neam de răzvrătiți.
\par 3 Tu însă, fiul omului, fă-ți toate cele de trebuință pentru pribegie și pleacă ziua, sub ochii lor, și pribegește sub ochii lor din loc în loc, poate vor înțelege că sunt un neam de răzvrătiți.
\par 4 Lucrurile tale să le scoți cum se scot lucrurile la vreme de pribegie, ziua, în ochii lor, iar tu pleacă seara, în ochii lor, cum se pleacă în surghiun.
\par 5 În ochii lor fă-ți spărtură în zid pe unde vei ieși.
\par 6 Pune-ți lucrurile pe umăr în ochii lor și ieși pe întuneric, și fața să și-o acoperi, ca să nu vezi pământul, căci te-am pus semn casei lui Israel".
\par 7 Și am făcut cum mi se poruncise: lucrurile mele, ca pe niște lucruri trebuincioase la vreme de pribegie, le-am scos ziua, iar seara mi-am săpat cu mâna o spărtură în zid și pe întuneric mi-am scos povara și am luat-o pe umăr înaintea ochilor lor.
\par 8 Apoi a fost către mine cuvântul Domnului dimineața și mi-a zis:
\par 9 "Fiul omului, casa lui Israel, neam de răzvrătiți, nu te-a întrebat ce faci?
\par 10 Spune-le: Așa grăiește Domnul Dumnezeu: Aceasta este o prevestire pentru cârmuitorul Ierusalimului și pentru toată casa lui Israel care se află acolo.
\par 11 Și să le mai spui: Eu sunt un semn pentru voi; ceea ce fac eu, aceea li se va face și lor: în pribegie și în robie vor merge.
\par 12 Și căpetenia, care se află între ei, va lua povara pe umeri și pe întuneric va ieși prin zidul pe care îl vor străpunge ca să-l scoată afară. El își va acoperi fața, ca să nu vadă cu ochii săi pământul acesta.
\par 13 Dar voi întinde mreaja Mea în calea lui și-l voi prinde în lațul Meu, și-l voi duce la Babilon, în țara Caldeilor, dar el nu o va vedea și va muri acolo,
\par 14 Iar pe toți cei dimprejurul lui, apărătorii lui și toată oștirea lui, o voi împrăștia în toate vânturile și în urma lor voi trage sabia Mea.
\par 15 Și vor cunoaște că Eu sunt Domnul, când îi voi risipi printre popoare și-i voi împrăștia pe fața pământului.
\par 16 Dar un mic număr din ei voi cruța de sabie, de ciumă și de foamete ca să povestească popoarelor, la care vor merge, despre toate ticăloșiile lor, și să știe și ele că Eu sunt Domnul".
\par 17 Fost-a către mine cuvântul Domnului și mi-a zis:
\par 18 "Fiul omului, tremurând să-ți mănânci pâinea ta și apa ta s-o bei amărât și necăjit!
\par 19 Spune poporului țării: Așa grăiește Domnul Dumnezeu despre locuitorii Ierusalimului și despre țara lui Israel: Cu întristare își vor mânca pâinea lor și apa lor și-o vor bea cu groază, pentru că țara lor va fi lipsită de orice belșug pentru nedreptățile tuturor celor ce o locuiesc.
\par 20 Orașele cele locuite ale lor vor fi dărâmate și țara pustiită și vor ști că Eu sunt Domnul".
\par 21 Fost-a cuvântul Domnului către mine și mi-a zis:
\par 22 "Fiul omului, ce înseamnă zicătoarea care este la voi, în țara lui Israel: "Zilele se prelungesc și orice vedenie proorocească a pierit?"
\par 23 De aceea spune-le: Așa grăiește Domnul Dumnezeu: Nimici-voi această zicătoare și o asemenea zicătoare nu se va mai auzi în țara lui Israel. Spune-le însă că aproape este vremea și toată vedenia proorocească se va împlini.
\par 24 Că nu va mai rămâne zadarnică nici o vedenie proorocească și nici o proorocie nu va mai fi mincinoasă în casa lui Israel.
\par 25 Căci Eu, Eu Domnul grăiesc, și cuvântul care-l spun Eu se va împlini și nu se va schimba. Neam de răzvrătiți, în zilele voastre am rostit cuvântul și-l voi împlini", zice Domnul Dumnezeu.
\par 26 Și iarăși a fost cuvântul Domnului către mine și mi-a zis:
\par 27 "Fiul omului, iată casa lui Israel zice: "Vedenia proorocească, pe care a văzut-o acesta, se va împlini după multă vreme, pentru că el proorocește pentru niște timpuri depărtate".
\par 28 De aceea spune-le: "Așa grăiește Domnul Dumnezeu: Nici unul din cuvintele Mele nu va fi amânat, ci cuvântul pe care-l voi rosti se va împlini", zice Domnul Dumnezeu.

\chapter{13}

\par 1 Fost-a cuvântul Domnului către mine și mi-a zis:
\par 2 "Fiul omului, rostește profeție împotriva proorocilor lui Israel, care prevestesc, și zi celor ce proorocesc după îndemnul inimii lor: Ascultați cuvântul Domnului!
\par 3 Așa grăiește Domnul Dumnezeu: Vai de proorocii cei mincinoși, care urmează duhul lor și nu văd nimic!
\par 4 Proorocii tăi, Israele, sunt ca vulpile din ruine.
\par 5 La spărturile zidurilor nu se suie, nici nu apără cu zid casa lui Israel, ca să stea tare la luptă în ziua Domnului.
\par 6 Vedeniile lor sunt deșarte și prevestirile lor mincinoase; ei zic: "Domnul a spus", dar Domnul nu i-a trimis și ei încredințează că se va împlini cuvântul lor.
\par 7 Domnul întreabă: Vedeniile ce le-ați văzut nu sunt ele, oare, deșarte și prevestirile ce le-ați rostit nu sunt ele, oare, mincinoase? Voi ziceți: Domnul a spus, dar Eu n-am spus.
\par 8 De aceea așa zice Domnul Dumnezeu: Pentru că spuneți lucruri deșarte și pentru că vedeniile voastre sunt minciuni, iată Eu vin împotriva voastră, zice Domnul Dumnezeu.
\par 9 Mâna Mea va fi împotriva acestor prooroci, care văd lucruri deșarte și prevestesc minciuni; în sfatul poporului Meu ei nu vor fi, nici nu se vor înscrie în cartea casei lui Israel și vor ști că Eu sunt Domnul Dumnezeu.
\par 10 Pentru că duc poporul Meu în rătăcire, spunând: "Pace", atunci când nu e pace; și pentru că atunci când el face zid, ei îl tencuiesc cu ipsos.
\par 11 Spune celor ce tencuiesc zidul că acesta va cădea. Ploaie potopitoare se va vărsa, pietre de grindină vor cădea, și puhoi vijelios va doborî zidul.
\par 12 Iată el va cădea; atunci nu vă vor întreba oare: "Unde este tencuiala cu care l-ați acoperit?"
\par 13 De aceea așa zice Domnul Dumnezeu: Iată, Eu, în mânia Mea, voi slobozi furtună mare, ploaie potopitoare voi vărsa în urgia Mea și în mânia Mea va cădea grindină pustiitoare.
\par 14 Și voi dărâma zidul pe care voi l-ați acoperit cu ipsos și-l voi doborî la pământ; temeliile lui se vor dezgoli și el va cădea; o dată cu el, veți pieri și voi și veți ști că Eu sunt Domnul.
\par 15 Așa îmi voi potoli mânia Mea împotriva zidului și împotriva celor ce l-au acoperit cu ipsos și vă voi zice: Nu mai este zidul, nici cei ce l-au tencuit;
\par 16 Nu mai sunt proorocii lui Israel care au proorocit Ierusalimului și i-au prevestit vedenii de pace, atunci când nu era pace, zice Domnul Dumnezeu.
\par 17 Iar tu, fiul omului, îndreaptă-ți fața spre fiicele poporului tău, care proorocesc după inima lor; rostește împotriva lor proorocie,
\par 18 Și le spune: Așa grăiește Domnul Dumnezeu: Vai de cele ce cos pernițe fermecate pentru subsuori și fac marame pentru capul celor de orice statură, pentru a vâna sufletele! Au doar vânând sufletele poporului Meu, vă veți mântui sufletele voastre?
\par 19 Voi Mă necinstiți înaintea poporului Meu pentru o mână de orz și pentru o bucată de pâine, ucigând sufletele care nu trebuie să moară, cruțând viața sufletelor care nu trebuie să trăiască și amăgind astfel pe poporul Meu, care ascultă minciună.
\par 20 De aceea așa grăiește Domnul Dumnezeu: Iată Eu sunt împotriva pernițelor voastre fermecate, cu care voi prindeți sufletele în laț ca pe păsări; vi le voi smulge de la subsuori și voi da drumul sufletelor, pe care voi le prindeți în laț.
\par 21 Voi sfâșia maramele voastre și voi izbăvi poporul Meu din mâinile voastre, ca să nu mai fie pradă în mâinile voastre și veți ști că Eu sunt Domnul.
\par 22 Pentru că întristați prin minciuni inima dreptului, pe care Eu n-am voit să o întristez, și pentru că sprijiniți brațul celui nelegiuit, ca acesta să nu se întoarcă de la calea lui cea rea spre a-și păstra viața.
\par 23 De aceea nu veți mai avea vedenii deșarte și în viitor nu veți mai rosti prevestiri; voi izbăvi poporul Meu din mâinile voastre și veți cunoaște că Eu sunt Domnul".

\chapter{14}

\par 1 Atunci au venit la mine câțiva din bătrânii lui Israel și au șezut înaintea mea.
\par 2 Și a fost către mine cuvântul Domnului și a zis:
\par 3 "Fiul omului, acești bărbați își poartă idolii în inimă și își au ochii ațintiți spre ceea ce i-a făcut să cadă în nedreptăți. Pot Eu oare să le răspund?
\par 4 De aceea grăiește cu ei și le spune: Așa grăiește Domnul Dumnezeu: Dacă cineva din casa lui Israel, care poartă în inima sa idolii săi și își are privirile ațintite spre ceea ce l-a făcut să cadă în nedreptăți, va veni să întrebe pe prooroc, oare aș putea Eu, Domnul, să-i dau răspuns, din cauza mulțimii idolilor lui?
\par 5 Să înțeleagă dar casa lui Israel în inima sa, căci ei cu toții au ajuns străini de Mine, prin idolii lor.
\par 6 De aceea spune casei lui Israel: Așa grăiește Domnul Dumnezeu: Abateți-vă și vă întoarceți de la idolii voștri și de la toate urâciunile voastre întoarceți-vă fața.
\par 7 Că dacă cineva din casa lui Israel și din străinii care trăiesc în Israel s-a depărtat de la Mine, îngăduind idolii în inima sa și ațintindu-și ochii spre ceea ce l-a făcut să cadă în nedreptăți, va veni la prooroc ca să Mă întrebe prin el, îi voi răspunde Eu oare?
\par 8 Voi îndrepta fața Mea împotriva omului aceluia, îl voi face să fie semn și pildă și-l voi pierde din poporul Meu și veți cunoaște că Eu sunt Domnul.
\par 9 Iar dacă proorocul va amăgi și va spune cuvânt, ca și cum Eu, Domnul, i l-aș fi spus, atunci Eu îmi voi întinde mâna și-l voi stârpi din poporul Meu Israel.
\par 10 Și așa își vor lua toți pedeapsa pentru nelegiuirea lor; cum va fi pedeapsa celui ce întreabă, așa va fi și pedeapsa celui ce proorocește,
\par 11 Ca în viitor casa lui Israel să nu se mai abată de la Mine și ca să nu se mai întineze cu tot felul de nelegiuiri; ca să fie poporul Meu, iar Eu să fiu Dumnezeul lor", zice Domnul Dumnezeu.
\par 12 Fost-a cuvântul Domnului către mine și mi-a zis:
\par 13 "Fiul omului, dacă vreo țară ar păcătui înaintea Mea, abătându-se în chip nelegiuit de la Mine, și Eu aș întinde mâna Mea asupra ei, aș zdrobi în ea tot spicul de grâu și aș trimite asupra ei foametea și aș începe să pierd în ea pe oameni și dobitoace;
\par 14 Dacă s-ar afla acolo cei trei bărbați: Noe, Daniel și Iov, apoi aceștia, prin dreptatea lor, și-ar scăpa numai viața lor, zice Domnul Dumnezeu.
\par 15 Sau dacă aș trimite asupra acestui pământ fiare rele, care l-ar văduvi de popor, și dacă țel din pricina fiarelor ar ajunge pustiu și de nestrăbătut,
\par 16 Atunci acești trei bărbați, aflându-se în el, precum este de adevărat că Eu sunt viu, zice Domnul, tot așa este de adevărat că ei n-ar scăpa nici pe fii, nici pe fiice, ci numai ei singuri ar scăpa, iar țara ar ajunge pustie.
\par 17 Sau dacă aș aduce în țara aceasta sabie și aș zice: "Sabie, străbate țara" și aș începe a pierde acolo pe oameni și animale,
\par 18 Atunci acești trei bărbați, aflându-se în ea, precum este adevărat că Eu sunt viu, tot așa este de adevărat, zice Domnul, că ei n-ar scăpa nici pe fii, nici pe fiice, ci numai ei singuri ar scăpa.
\par 19 Sau dacă Eu aș trimite ciuma în țara aceasta și aș revărsa asupra ei urgia Mea în vărsare de sânge, ca să pierd din ea și pe oameni și pe animale,
\par 20 Apoi Noe, Daniel și Iov, aflându-se în ea, precum este adevărat că Eu sunt viu, zice Domnul, tot așa este de adevărat că n-ar scăpa nici fii, nici fiice; prin dreptatea lor ei și-ar scăpa numai viața lor".
\par 21 Că așa zice Domnul Dumnezeu: "Chiar de aș trimite aceste patru pedepse grozave ale Mele: sabia, foametea, fiarele sălbatice și ciuma împotriva Ierusalimului, ca să stârpesc din el oamenii și dobitoacele,
\par 22 Și atunci va rămâne în ei o rămășiță de fii și fiice, care vor fi scoși de acolo. Iată ei vor veni la voi și voi veți vedea purtarea lor și faptele dor și vă veți mângâia de nenorocirea pe care Eu am adus-o asupra Ierusalimului și de toate câte am adus asupra lui.
\par 23 Ei vă vor mângâia, când veți vedea purtarea lor și faptele lor, și veți cunoaște că nu în zadar am făcut Eu toate câte am făcut în el", zice Domnul Dumnezeu.

\chapter{15}

\par 1 Și a fost iarăși cuvântul Domnului către mine:
\par 2 "Fiul omului, ce întâietate are lemnul de viță de vie față de oricare alt lemn, și coarda de viță de vie între arborii din pădure?
\par 3 Se ia oare din el vreo bucățică pentru vreun lucru? Se ia oare din el măcar pentru un cui, ca să atârni în el un lucru oarecare?
\par 4 Iată el se dă focului spre ardere; amândouă capetele lui le mistuie focul, și mijlocul arde și el. Va mai fi el bun de ceva?
\par 5 Nici când era întreg nu era bun la ceva, cu atât mai mult nu va fi acum, când l-a mistuit focul; acum când a ars se mai poate face ceva cu el?
\par 6 De aceea așa zice Domnul Dumnezeu: Precum lemnul de viță de vie dintre arborii pădurii l-am dat focului ca să-l ardă, așa voi da și pe locuitorii Ierusalimului.
\par 7 Îmi voi întoarce fața Mea împotriva lor. Dintr-un foc au scăpat, dar focul îi va mistui, și veți ști că Eu sunt Domnul, când îmi voi întoarce fața împotriva lor.
\par 8 Voi face din țară un pustiu, pentru că ei Mi-au fost necredincioși", zice Domnul Dumnezeu.

\chapter{16}

\par 1 Fost-a cuvântul Domnului către mine:
\par 2 "Fiul omului, spune Ierusalimului urâciunile lui,
\par 3 Și-i spune: Așa grăiește Domnul Dumnezeu către fiica Ierusalimului: Obârșia ta și patria ta e țara Canaan; tatăl tău e amoreu și mama ta e hetită.
\par 4 La nașterea ta, în ziua în care te-ai născut, nu ti s-a tăiat buricul și cu apă n-ai fost spălată pentru curățire și cu sare n-ai fost sărată, nici cu scutece înfășată.
\par 5 Ochiul nimănui nu s-a înduioșat spre tine, ca din milă de tine să-ți fi făcut vreuna din acestea; ci ai fost aruncată în câmp din dispreț către viața ta, în ziua nașterii tale.
\par 6 Și am trecut Eu pe lângă tine și te-am văzut zbătându-te în sângele tău și ți-am zis: "Trăiește în sângele tău!" Așa ți-am zis: "Trăiește în sângele tău!"
\par 7 Și te-am înmulțit ca pe iarba câmpului; ai crescut, te-ai făcut mare și ai ajuns la o frumusețe desăvârșită; și s-a ridicat pieptul și ți-a crescut părul; dar erai goală, de tot goală.
\par 8 Atunci am trecut Eu pe lângă tine și te-am văzut, și iată aceea era vremea ta, vremea iubirii. Atunci mi-am întins Eu poala mantiei Mele peste tine și am acoperit goliciunea ta, și-am făcut un jurământ, am făcut un legământ cu tine, zice Domnul Dumnezeu și tu ai fost a Mea.
\par 9 Apoi te-am spălat cu apă, am curățit de pe tine sângele tău și te-am uns cu untdelemn.
\par 10 Ți-am dat veșminte brodate, încălțăminte de piele fină, o legătură de vison pentru cap și o mantie de mătase.
\par 11 Te-am gătit cu podoabe și ți-am pus brățări la mâini și salbe la gât.
\par 12 Ți-am dat inel în nas și cercei în urechi și pe cap ți-am pus o coroană minunată.
\par 13 Așa ai fost împodobită cu aur și cu argint și îmbrăcămintea ta era de vison, de mătase și de țesături brodate; te-ai hrănit cu pâine din cea mai bună făină de grâu, cu miere și untdelemn, și erai foarte frumoasă și ai ajuns la vrednicia de regină.
\par 14 Ai fost renumită printre neamuri pentru frumusețea ta, pentru că ea era desăvârșită datorită strălucirii Mele cu care te-am îmbrăcat", zice Domnul Dumnezeu.
\par 15 Dar tu te-ai încrezut în frumusețea ta și, folosindu-te de renumele tău, ai început să te desfrânezi; și-ai cheltuit desfrânarea ta cu tot trecătorul, dându-te pe tine lui.
\par 16 Ai luat din hainele tale, ca să-ți faci locuri înalte în culori felurite, și te-ai desfrânat acolo, cum niciodată nu s-a întâmplat și nici nu va mai fi.
\par 17 Ai luat lucrurile tale de găteală, făcute din aurul Meu și din argintul Meu, pe care ți le-am dat Eu, și ți-ai făcut chipuri de bărbat și te-ai desfrânat cu ele.
\par 18 Ai luat hainele tale cele brodate și i-ai îmbrăcat pe ei cu ele; ai pus înaintea lor uleiul și tămâia Mea.
\par 19 Și pâinea Mea, pe care Eu ți-o dădeam ție; făina de grâu, uleiul și mierea, pe care Eu ți le dădeam ție, tu le puneai înaintea lor spre miros de bună mireasmă; iată ce s-a întâmplat, zice Domnul Dumnezeu.
\par 20 Ai luat pe fiii tăi și pe fiicele tale pe care Mi i-ai născut Mie și i-ai adus lor jertfă spre mâncare. Dar puțin te-ai desfrânat tu oare?
\par 21 Ba tu și pe fiii Mei i-ai junghiat și i-ai dat lor, trecându-i prin foc.
\par 22 Pe lângă toate urâciunile și desfrânările tale tu nu ți-ai adus aminte de zilele tinereții tale, când erai goală, cu totul goală, zbătându-te și aruncată în sângele tău.
\par 23 După toate nelegiuirile tale, vai, vai de tine, zice Domnul Dumnezeu.
\par 24 Că ți-ai făcut case de desfrânare, ai așezat locuri înalte în fiecare piață;
\par 25 La răspântia fiecărui drum ți-ai făcut locuri înalte, ți-ai batjocorit frumusețea ta și ți-ai arătat picioarele înaintea fiecărui trecător și ți-ai înmulțit desfrânările.
\par 26 Te-ai desfrânat cu fiii Egiptului, vecinii tăi, oameni înalți la statură, și ți-ai înmulțit desfrânările, mâniindu-Mă pe Mine.
\par 27 Iată, Mi-am întins asupra ta mâna Mea, am împuținat cele menite pentru tine și te-am lăsat pradă fetelor Filistenilor, dușmancele tale, care s-au rușinat de purtarea ta cea nelegiuită.
\par 28 Tu te-ai desfrânat cu Asirienii și nu te-ai săturat; te-ai desfrânat cu ei, dar nu te-ai mulțumit cu atât,
\par 29 Ci ai înmulțit desfrânările tale din pământul Canaan până în pământul Caldeii, dar nici cu atât nu te-ai mulțumit.
\par 30 Cât de obosită trebuie să fie inima ta, zice Domnul Dumnezeu, după ce ai făcut toate acestea, ca o desfrânată nestăpânită!
\par 31 Când ți-ai făcut case de desfrânare la fiecare răspântie de drum și ți-ai făcut locuri înalte în fiecare piață, nu erai ca o desfrânată, pentru că respingeai darurile,
\par 32 Ci ca o femeie adulteră, care, în locul bărbatului său, primește pe alții.
\par 33 Tuturor desfrânatelor se dau daruri; tu însă dădeai însăți daruri amanților tăi și îi cumpărai, ca să vină aceștia din toate părțile la tine și să se desfrâneze cu tine.
\par 34 La tine desfrânările tale se petreceau în alt fel decât se întâmplă cu femeile; nu umblau bărbații după tine, ci tu dădeai daruri, iar ție nu ți se dădeau daruri, și deci tu te-ai purtat cu totul altfel decât altele.
\par 35 De aceea ascultă, desfrânato, cuvântul Domnului:
\par 36 Așa zice Domnul Dumnezeu: Pentru că tu ți-ai vărsat astfel banii tăi și pentru că în desfrânările tale și s-a descoperit goliciunea ta înaintea amanților tăi și înaintea tuturor oamenilor tăi nerușinați și pentru sângele fiilor tăi, pe care tu i-ai dat lor,
\par 37 Pentru toate acestea iată Eu voi aduna pe toți amanții tăi cu care te-ai desfrânat tu și pe care i-ai iubit și pe toți aceia pe care i-ai urât, și-i voi aduna pretutindeni împotriva ta și voi descoperi înaintea lor goliciunea ta și vor vedea toată rușinea ta.
\par 38 Te voi judeca cum se judecă femeile adultere și cele ce varsă sânge și te voi preda urgiei și pismei;
\par 39 Te voi da în mâinile acelora și ei vor dărâma casele tale de desfrânare, vor risipi locurile tale înalte, vor rupe de pe tine hainele tale, vor lua podoabele tale și te vor lăsa goală, de tot goală.
\par 40 Voi strânge împotriva ta adunare și te vor ucide cu pietre și cu sabie te vor tăia.
\par 41 Vor arde casele tale cu foc și te vor judeca înaintea ochilor multor femei; așa voi pune capăt desfrânării tale și nu vei mai face daruri.
\par 42 Îmi voi potoli cu tine urgia Mea și Mă voi liniști și nu Mă vai mai mânia.
\par 43 Pentru că tu nu ți-ai adus aminte de zilele tinereții tale și de toate cu câte M-ai mâniat, iată și Eu voi întoarce purtarea ta asupra capului tău, zice Domnul Dumnezeu, ca tu să nu te mai dedal desfrâului cu toți idolii tăi.
\par 44 Iată, tot cel ce grăiește în pilde poate să zică de tine: "Cum e mama, așa e și fiica!"
\par 45 Tu ești cu adevărat fiica mamei tale, care și-a lepădat bărbatul și copiii; tu ești cu adevărat sora surorilor tale, care și-au lepădat bărbații și copiii lor. Mama voastră este hetită și tatăl vostru amoreu.
\par 46 Iar sora ta cea mai mare este Samaria, care trăiește cu fiicele sale în stânga ta. Iar sora ta cea mai mică, aceea care trăiește la dreapta ta, este Sodoma cu fiicele ei.
\par 47 Tu însă n-ai mers nici măcar pe căile lor și nici măcar la urâciunile lor nu te-ai mărginit; aceasta și s-a părut puțin; tu te-ai arătat mai stricată decât ele în toate căile tale.
\par 48 Viu sunt Eu, zice Domnul Dumnezeu; Sodoma, sora ta, n-a făcut nici ea, nici fiicele ei ce-ai făcut tu și fiicele tale.
\par 49 Iată care au fost fărădelegile Sodomei, sora ta, și ale fiicelor ei: mândria, îmbuibarea și trândăvia; iar mâinile săracului și ale celui nevoiaș nu le-au sprijinit.
\par 50 Ele s-au mândrit și au făcut urâciune înaintea Mea; de aceea le-am și nimicit, cum ai văzut.
\par 51 Iar Samaria n-a păcătuit nici pe jumătate din ce ai păcătuit tu. Tu le-ai întrecut în urâciuni și prin urâciunile tale, pe care le-ai făcut tu, surorile tale s-au dovedit mai drepte decât tine.
\par 52 Poartă-ți dar rușinea ta și tu, ceea ce osândeai pe surorile tale; față de păcatele tale, cu care tu te-ai batjocorit, acelea sunt mai drepte decât tine. Roșește dar de rușine și tu și du-ți batjocura ta, pentru că ai îndreptățit astfel pe surorile tale.
\par 53 Dar Eu voi aduce înapoi pe prinșii lor de război, pe prinșii de război ai Sodomei și ai fiicelor ei, pe prinșii de război ai Samariei și ai fiilor ei și pe prinșii tăi de război în mijlocul lor,
\par 54 Ca să-ți porți rușinea ta și să te rușinezi de tot ceea ce ai făcut, servind ca mângâiere pentru ele.
\par 55 Surorile tale, Sodoma, și fiicele ei, se vor întoarce la starea lor de mai înainte; Samaria și fiicele ei se vor întoarce la starea lor de mai înainte; și tu și fiicele tale vă veți întoarce la starea voastră de altădată.
\par 56 De sora ta, Sodoma, nici pomenire nu a fost pe buzele tale în zilele trufiei tale,
\par 57 Până nu se descoperise goliciunea ta, ca în zilele batjocurei ce ți-a venit din partea fiicelor Siriei și a tuturor celor ce o înconjoară, din partea fiicelor Filistenilor, care din toate părțile se uitau la tine cu dispreț.
\par 58 Tu suferi din pricina desfrâului tău și din pricina ticăloșiei tale, zice Domnul Dumnezeu;
\par 59 Că așa grăiește Domnul Dumnezeu: Mă voi purta cu tine, cum te-ai purtat tu, disprețuind jurământul prin ruperea legământului.
\par 60 Dar Eu Îmi voi aduce aminte de legământul Meu încheiat cu tine în zilele tinereții tale și voi înnoi cu tine un așezământ veșnic.
\par 61 Și tu îți vei aduce aminte de căile tale și-ți va fi rușine când vei începe să primești la tine pe surorile tale, cele mai mari decât tine și pe cele mai mici decât tine și când Eu le voi da ție ca fiice, însă nu după așezământul tău.
\par 62 Voi înnoi așezământul Meu cu tine și vei cunoaște că Eu sunt Domnul,
\par 63 Ca să-ți aduci aminte și să te rușinezi, ca pe viitor să nu poți nici gura să-ți deschizi de rușine, când iți voi ierta ceea ce ai făcut", zice Domnul Dumnezeu.

\chapter{17}

\par 1 Fost-a iarăși cuvântul Domnului către mine și mi-a zis:
\par 2 "Fiul omului, spune casei lui Israel o ghicitoare, spune-i o pildă
\par 3 Și zi: Așa grăiește Domnul Dumnezeu: Un vultur mare eu aripi mari, cu pene lungi, pufos și pestriț a zburat în Liban și a frânt vârful unui cedru.
\par 4 A rupt pe cel mai de sus din lăstarii lui cei tineri și l-a adus în țara Canaanului și l-a pus într-o cetate de negustori;
\par 5 A luat apoi pe unul din lăstarii cedrului și l-a pus într-un pământ roditor, l-a sădit lângă o apă mare, ca pe o salcie.
\par 6 Și lăstarul a crescut și s-a făcut un butuc de viță întins, dar nu înalt, ale cărui ramuri se întorceau spre vultur, iar rădăcinile îi erau sub el; el s-a făcut viță, a dat lăstari și a făcut coarde.
\par 7 Și a mai fost un alt vultur cu aripi mari și pufos. Și iată acest butuc de viță s-a întins spre el cu rădăcinile și cu coardele sale, ca să-l ude acela mai mult decât era udat în locul unde fusese sădit.
\par 8 El era sădit într-o țarină bună, la niște ape mari, încât își putea întinde vițele ca să aducă rod și să ajungă un minunat butuc de vie.
\par 9 Spune dar: Așa grăiește Domnul Dumnezeu: Spori-va el oare? Nu cumva i se vor smulge rădăcinile și i se vor rupe vițele, încât să se usuce? Toți lăstarii tineri, care au crescut din el se vor usca. Și se va smulge din rădăcinile lui nu cu putere mare, nici cu oameni mulți.
\par 10 Iată, deși a fost sădit, îi va merge oare, bine? Oare nu se va usca îndată ce se va atinge de el vântul de răsărit? Se va usca pe mușuroiul pe care a fost sădit".
\par 11 Fost-a cuvântul Domnului către mine și mi-a zis:
\par 12 "Spune neamului de răzvrătiți: Oare nu știți ce înseamnă acestea? Spune: Iată a venit regele Babilonului la Ierusalim și a luat pe rege și pe căpeteniile lui și i-a dus cu sine în Babilon.
\par 13 A luat pe unul din neamul regesc și a încheiat cu acela legământ, l-a legat cu jurământ, iar pe puternicii țării i-a luat cu sine,
\par 14 Ca regatul să rămână ascultător și fără trufie, ca să păzească legământul și să-s fie credincios.
\par 15 Dar acela s-a depărtat de el, trimițându-și soli în Egipt ca să li se dea cai și oameni mulți. Va izbuti el oare și va scăpa el, cel ce a făcut una ca aceasta? El a călcat învoiala; va scăpa el oare?
\par 16 Precum este adevărat că Eu sunt viu, zice Domnul Dumnezeu, tot așa este adevărat că acolo unde șade regele care l-a făcut pe el rege, adică la Babilon, va muri, pentru că a nesocotit jurământul pe care l-a dat aceluia și a călcat legământul încheiat cu el.
\par 17 Faraon cu puterea cea mare și cu mulțimea poporului nu va face nimic pentru el în acest război, când se va ridica val și se vor zidi turnuri de împresurare spre pieirea a multe suflete.
\par 18 El a nesocotit jurământul, ca să calce legământul, și iată ți-a dat mâna și a făcut toate acestea și nu va scăpa.
\par 19 De aceea, așa zice Domnul Dumnezeu: Precum este adevărat că Eu sunt viu, tot așa este de adevărat că jurământul Meu, pe care l-a nesocotit, și legământul Meu, pe care l-a călcat, le voi întoarce asupra capului său;
\par 20 Voi arunca asupra lui mreaja Mea și va fi prins în lațurile Mele; îl voi aduce la Babilon și acolo Mă voi judeca cu el pentru necredincioșia lui față de Mine.
\par 21 Iar toți fugarii lui din toate taberele lui vor cădea de sabie, și cei rămași vor fi împrăștiați în toate vânturile și veți cunoaște că Eu, Domnul, am zis acestea".
\par 22 Așa zice Domnul Dumnezeu: "Voi lua din vârful cedrului celui înalt un lăstar și-l voi sădi; din creștetul lui voi rupe lăstarul cel plăpând și-l voi sădi pe un munte înalt și măreț.
\par 23 Pe muntele cel înalt al lui Israel îl voi sădi și va slobozi ramuri, va aduce rod, și se va face un cedru falnic și vor locui în el tot felul de păsări; tot felul de înaripate vor locui în umbra ramurilor lui.
\par 24 Și vor cunoaște toți pomii câmpului că Eu, Domnul, pomul înalt îl fac mic și pe cel mic îl fac înalt; pomul verde îl usuc și pomul uscat îl înverzesc. Eu, Domnul, am spus acestea".

\chapter{18}

\par 1 Și a mai fost cuvântul Domnului către mine și mi-a zis:
\par 2 "Pentru ce spuneți voi în țara lui Israel pilda aceasta și ziceți: Părinții au mâncat aguridă și copiilor li s-au strepezit dinții?
\par 3 Precum este adevărat că Eu sunt viu, zice Domnul Dumnezeu, tot așa este de adevărat că pe viitor nu se va mai grăi pilda aceasta lui Israel.
\par 4 Că iată toate sufletele sunt ale Mele; cum este al Meu sufletul tatălui, tot așa și sufletul fiului; sufletul care a greșit va muri.
\par 5 De este cineva drept și face judecată și dreptate;
\par 6 De nu mănâncă jertfit în munte și spre idolii casei lui Israel nu-și întoarce ochii săi; femeia aproapelui său nu o necinstește și de femeie nu se apropie în timpul perioadei ei de necurăție;
\par 7 Pe nimeni nu strâmtorează și datornicului îi întoarce zălogul, furt nu face, celui flămând îi dă din pâinea sa și pe cel gol îl îmbracă cu haină;
\par 8 Banii săi cu camătă nu-i dă și camătă nu ia; de la nedreptate mâinile și le stăpânește și judecata dintre un om și altul o face cu dreptate;
\par 9 De se poartă după poruncile Mele și legile Mele cu credincioșie le păzește, acela este drept și fără îndoială viu va fi, zice Domnul Dumnezeu.
\par 10 Dar de i s-a născut fiu hoț, care varsă sânge sau face ceva de felul acesta,
\par 11 Și care nu urmează calea tatălui său, ci mănâncă cele jertfite în munți, necinstește femeia aproapelui său;
\par 12 Pe sărac și pe lipsit îl apasă, răpește avutul altuia și zălogul nu-l întoarce; își ridică ochii la idoli și face ticăloșii;
\par 13 Banii și-i dă cu dobândă și ia camătă; unul ca acesta va trăi oare? Nu! De va face asemenea ticăloșii nu va trăi, ci sigur va muri și sângele lui va fi asupra lui.
\par 14 Iar de i s-a născut un fiu, care, văzând păcatele, văzând toate câte le-a făcut tatăl său, el se păzește și nu face nimic asemenea;
\par 15 În munți nu mănâncă jertfe idolești, nu-și ridică ochii spre idolii cei mincinoși ai casei lui Israel și femeia aproapelui său nu o necinstește;
\par 16 Pe nimeni nu apasă, zălog nu ia și avutul altuia nu-l risipește; celui flămând îi dă din pâinea sa, cu haina sa îmbracă pe cel gol;
\par 17 Pe sărac nu asuprește, nu ia nici camătă, nici dobândă; poruncile Mele le păzește și se poartă după legile Mele; acest om nu va muri pentru nedreptățile părintelui său, ci în veci va trăi.
\par 18 Iar tatăl său, pentru că a apăsat pe alții, a răpit ceea ce era al fratelui său și a făcut în poporul său ceea ce nu era îngăduit, iată va muri pentru nedreptatea sa.
\par 19 Dar veți zice: Pentru ce fiul să nu poarte nedreptatea tatălui său? Pentru că fiul a făcut ceea ce era drept și legiuit și toate legile Mele le-a ținut și le-a împlinit; de aceea va trăi.
\par 20 Sufletul care păcătuiește va muri. Fiul nu va purta nedreptatea tatălui, și tatăl nu va purta nedreptatea fiului. Celui drept i se va socoti dreptatea sa, iar celui rău, răutatea sa.
\par 21 Dar dacă cel rău se întoarce de la nelegiuirile sale pe care le-a făcut și păzește toate legile Mele și face ceea ce e bun și drept, el va trăi și nu va muri.
\par 22 Nu se vor pomeni deloc nelegiuirile pe care el le va fi făcut, ci va trăi pentru dreptatea pe care va fi făcut-o.
\par 23 Oare voiesc Eu moartea păcătosului, zice Domnul Dumnezeu - și nu mai degrabă să se întoarcă de la căile sale și să fie viu?
\par 24 Dar și dreptul, dacă se va abate de la dreptatea sa și se va purta cu nedreptate și va face toate acele ticăloșii pe care le face nelegiuitul, va fi el oare viu? Toate faptele lui bune, pe care le va fi făcut, nu se vor pomeni, ci pentru nelegiuirea sa, pe care va fi făcut-o, și pentru păcatele sale, pe care le-a săvârșit, va muri.
\par 25 Dar voi ziceți: "Calea Domnului nu este dreaptă". Ascultați, casa lui Israel: Oare calea Mea nu este dreaptă, sau nu sunt drepte căile voastre?
\par 26 Dacă cel drept se abate de la dreptatea sa și face nelegiuire și din pricina aceasta moare, apoi el moare pentru nelegiuirea sa, pe care a făcut-o.
\par 27 Și cel nelegiuit, dacă se întoarce de la nelegiuirea sa, pe care a făcut-o și face judecată și dreptate, își întoarce sufletul său la viață;
\par 28 Căci el a văzut și s-a întors de la toate nelegiuirile sale, pe care le-a făcut; de aceea va fi viu și nu va muri.
\par 29 Însă casa lui Israel zice: "Calea Domnului nu este dreaptă". Casa lui Israel, oare calea Mea nu este dreaptă, sau nu sunt drepte căile voastre?
\par 30 De aceea vă voi judeca pe voi din casa lui Israel, pe fiecare după căile sale, zice Domnul Dumnezeu; pocăiți-vă și vă întoarceți de la toate nelegiuirile voastre, ca necredința să nu vă fie piedică.
\par 31 Lepădați de la voi toate păcatele voastre cu care ați greșit și vă faceți o inimă nouă și un duh nou. De ce să muriți voi, casa lui Israel?
\par 32 Căci Eu nu voiesc moartea păcătosului, zice Domnul Dumnezeu; întoarceți-vă deci și trăiți!"

\chapter{19}

\par 1 Iar tu, fiul omului, fă o tânguire pentru căpeteniile lui Israel și spune:
\par 2 "Ce este mama ta? O leoaică! Ea a dormit între lei și între puii de leu și-a alăptat puii săi.
\par 3 A crescut pe unul din puii săi, care s-a făcut leu, și a învățat să sfâșie prada, și a mâncat oameni.
\par 4 Și auzind popoarele de dânsul, a fost prins în groapa lor și l-au adus în lanțuri în țara Egiptului.
\par 5 Așteptând leoaica puțin timp și văzând că așteptarea sa este zadarnică, a luat pe altul din puii săi și l-a făcut leu.
\par 6 Acesta, făcându-se leu, a început să umble printre lei și a învățat să sfâșie prada și a mâncat oameni;
\par 7 A atacat casele lor și le-a pustiit cetățile; și de larma răgetului lui, țara și locuitorii ei s-au îngrozit și s-a pustiit țara și toate satele ei.
\par 8 Atunci s-au sculat împotriva lui neamurile din ținuturile vecine, și-au întins cursele lor împotriva lui și el a fost prins în groapa lor;
\par 9 Atunci l-au pus în lanțuri într-o cușcă și l-au dus la regele Babilonului; apoi l-au vârât într-o cetate, ca să nu se mai audă glasul lui prin munții lui Israel.
\par 10 Mama ta a fost ca o viță de vie, răsădită lângă apă; rămuroasă și roditoare a fost ea din pricina belșugului de apă.
\par 11 Și ea avea coarde tari, pentru sceptre regale, și și-a ridicat trunchiul său sus printre ramuri dese și atrăgea privirile cu înălțimea ei și mulțimea ramurilor ei.
\par 12 Dar a fost ruptă cu mânie și aruncată la pământ, vântul de răsărit a uscat rodul ei; ramurile ei cele puternice au fost rupte, s-au uscat și le-a mistuit focul.
\par 13 Iar acum ea a fost sădită în pustiu, într-un pământ sec și însetat.
\par 14 Din trunchiul ce poartă ramurile ei a ieșit foc, a mistuit rodul ei și nu au mai rămas în ea coarde puternice pentru sceptrele regale. Aceasta este cântare de jale și de jale va rămâne".

\chapter{20}

\par 1 În ziua a zecea a lunii a cincea din anul al șaptelea al robiei lui Ioiachim, au venit niște bărbați dintre bătrânii lui Israel să întrebe pe Domnul și au șezut înaintea feței mele.
\par 2 Atunci a fost cuvântul Domnului către mine și mi-a zis:
\par 3 "Fiul omului, vorbește cu bătrânii lui Israel și spune-le: Așa zice Domnul Dumnezeu: Ați venit să Mă întrebați? Precum este adevărat că Eu sunt viu, tot așa este de adevărat că nu voi da răspuns, zice Domnul Dumnezeu.
\par 4 Vrei să-i judeci, fiul omului? Dacă vrei să-i judeci, spune-le ticăloșiile părinților lor.
\par 5 Și spune-le: Așa zice Domnul Dumnezeu: În ziua în care am ales pe Israel și, ridicându-Mi mâna, M-am jurat seminției casei lui Iacov și M-am descoperit lor în țara Egiptului și, ridicându-Mi mâna, le-am zis: Eu sunt Domnul Dumnezeul vostru;
\par 6 În ziua aceea, ridicându-Mi mina, M-am jurat lor să-i scot din țara Egiptului și să-i duc în țara pe care o alesesem pentru ei, în care curge lapte și miere, cea mai frumoasă dintre toate țările.
\par 7 Și le-am zis: Lepădați fiecare ticăloșiile de la ochii voștri și nu vă mai întinați cu idolii Egiptului, că Eu sunt Domnul Dumnezeul vostru!
\par 8 Dar ei s-au răzvrătit împotriva Mea și n-au vrut să Mă asculte; nimeni n-a lepădat ticăloșiile de la ochii săi și idolii Egiptului nu i-a părăsit. Atunci am zis: Voi vărsa asupra lor mânia Mea și urgia Mea o voi deșerta peste ei în mijlocul țării Egiptului.
\par 9 Însă, dacă i-am scos din țara Egiptului, am făcut aceasta pentru ca numele Meu să nu fie pângărit în ochii neamurilor printre care se aflau ei și înaintea cărora Mă descoperisem, ca să-i scot din țara Egiptului.
\par 10 Scoțându-i din țara Egiptului, i-am adus în pustiu.
\par 11 Și le-am dat legile Mele, le-am arătat rânduielile Mele, prin care omul, care le va ține, va trăi.
\par 12 De asemenea le-am dat și zilele Mele de odihnă, ca să fie semn între Mine și ei, ca să cunoască ei că Eu sunt Domnul, Sfințitorul lor.
\par 13 Însă casa lui Israel s-a răzvrătit împotriva Mea în pustiu; după legile Mele n-a umblat și a lepădat rânduielile Mele, pe care omul trebuie să le împlinească, ca să trăiască prin ele, și zilele Mele de odihnă le-au călcat. Atunci Eu am gândit să revărs asupra lor mânia Mea în pustiu, ca să-i pierd.
\par 14 Dar Eu am făcut ca numele Meu să nu fie pângărit în ochii neamurilor, la vederea cărora îi scosesem.
\par 15 Ba ridicându-Mi chiar mâna asupra lor în pustiu, M-am jurat că nu-i voi duce în țara ce rânduisem, în care curge lapte și miere și este cea mai frumoasă din toate țările,
\par 16 Pentru că ei lepădaseră așezămintele Mele și după poruncile Mele nu se purtau, ci călcau zilele Mele de odihnă, că inima lor era îndreptată spre idolii lor.
\par 17 Dar ochiul Meu nu s-a îndurat să-i piardă și nu i-am stârpit în pustiu.
\par 18 Am zis către fiii lor în pustiu: Nu vă purtați după rânduielile părinților voștri și obiceiurile lor să nu le păziți, nici să nu vă întinați cu idolii lor.
\par 19 Eu sunt Domnul Dumnezeul vostru; purtați-vă după poruncile Mele și hotărârile Mele păziți-le și le împliniți.
\par 20 Cinstiți zilele Mele de odihnă, ca să fie semn între Mine și voi, ca să știți că Eu sunt Domnul Dumnezeul vostru.
\par 21 Dar și fiii s-au răzvrătit împotriva Mea; nici ei nu s-au purtat după rânduielile Mele și legile Mele nu le-au păzit; n-au împlinit ceea ce omul trebuie să împlinească, ca să fie viu; au călcat zilele Mele de odihnă și atunci am gândit să vărs peste ei urgia Mea și să sting mânia Mea împotriva lor în pustiu;
\par 22 Dar iarăși Mi-am tras mâna înapoi și am făcut ca numele Meu să nu fie pângărit în ochii neamurilor, la vederea cărora îi scosesem.
\par 23 De asemenea Mi-am ridicat mâna în pustiu și M-am jurat să-i risipesc printre popoare și să-i împrăștii prin țări,
\par 24 Pentru că nu împliniseră rânduielile Mele, legile Mele le lepădaseră și călcaseră zilele Mele de odihnă, că ochii lor erau îndreptați spre idolii părinților lor.
\par 25 Ba încă le-am dat și legi care nu erau bune și rânduieli prin care ei nu puteau trăi;
\par 26 Și i-am întinat prin ofrandele lor, făcându-i să jertfească pe toți întâi-născuții lor, pentru a-i pedepsi ca să știe că Eu sunt Domnul.
\par 27 De aceea, fiul omului, vorbește casei lui Israel și le spune: Așa zice Domnul Dumnezeu: Iată cu ce M-au mai hulit încă părinții voștri, purtându-se cu necredincioșie față de Mine:
\par 28 Eu i-am adus în țara pe care cu jurământ făgăduisem să le-o dau, iar ei, punându-și ochii pe toată colina înaltă și pe tot arborele umbros, au început să junghie acolo jertfele lor și au pus acolo prinoasele lor și miresmele lor de tămâiere cele jignitoare pentru Mine și au săvârșit acolo jertfele lor cu turnare.
\par 29 Atunci le-am zis: Ce este locul înalt unde vă duceți voi? Și ei l-au numit cu numele Bama (înălțime) până în ziua de astăzi.
\par 30 Pentru aceea, zi casei lui Israel: Acestea zice Domnul Dumnezeu: Nu vă întinați oare după pilda părinților voștri și oare nu vă desfrânați după ticăloșiile lor?
\par 31 Adunându-vă darurile și trecând copiii voștri prin foc, vă pângăriți cu toți idolii voștri până în ziua de astăzi. Și mai voiți încă să Mă întrebați, casa lui Israel? Precum este adevărat că Eu sunt viu, zice Domnul Dumnezeu, tot așa este de adevărat că nu vă voi da răspuns.
\par 32 Ceea ce vă frământă mintea nu se va împlini nicidecum. Voi ziceți: Vom sluji lemnului și pietrei, ca neamurile, ca triburile din țările străine.
\par 33 Cum este adevărat că Eu sunt viu, tot așa este de adevărat că, zice Domnul Dumnezeu, cu mână tare, cu braț ridicat și cu vărsarea urgiei Mele voi domni peste voi.
\par 34 Vă voi scoate din mijlocul popoarelor și vă voi aduna din țările străine unde ați fost împrăștiați și vă voi aduna cu mână tare, cu braț ridicat și cu vărsarea mâniei,
\par 35 Și vă voi aduce în pustiul popoarelor și acolo Mă voi judeca cu voi, față către față.
\par 36 Cum M-am judecat cu părinții voștri în pustiul din țara Egiptului, așa Mă voi judeca și cu voi, zice Domnul Dumnezeu.
\par 37 Vă voi trece sub toiag și vă voi vârî în legăturile așezământului.
\par 38 Voi alege din voi pe răzvrătiți, pe cei nesupuși Mie; îi voi scoate din țara unde sălășluiesc ei, dar în țara lui Israel nu vor intra și veți ști că Eu sunt Domnul.
\par 39 Iar voi, casa lui Israel - așa zice Domnul Dumnezeu - duceți-vă fiecare la idolii săi și le slujiți, dacă nu Mă ascultați pe Mine, dar nu mai pângăriți numele Meu cel sfânt cu darurile voastre și cu idolii voștri.
\par 40 Pentru că pe muntele Meu cel sfânt, pe muntele cel înalt al lui Israel, - zice Domnul Dumnezeu - acolo îmi va sluji toată casa lui Israel, toată, oricâtă ar fi ea pe pământ; acolo îi voi primi cu bunăvoință și acolo voi cere prinoasele voastre și pârgile voastre cu toate cele sfinte ale voastre.
\par 41 Vă voi primi ca pe niște tămâie mirositoare, când vă voi scoate din mijlocul popoarelor și vă voi aduna din țările străine, unde ați fost împrăștiați și Mă voi sfinți întru voi înaintea ochilor neamurilor.
\par 42 Și veți ști atunci că Eu sunt Domnul, când vă voi duce în țara lui Israel, în țara pe care am jurat să o dau părinților voștri, ridicându-Mi mâna.
\par 43 Vă veți aminti acolo de căile voastre și de toate faptele voastre, cu care v-ați întinat și vă veți dezgusta singuri de toate nelegiuirile voastre, pe care le-ați făcut.
\par 44 Veți ști că Eu sunt Domnul, când voi face cu voi după numele Meu, nu după căile voastre cele rele, nici după faptele voastre cele stricate, casa lui Israel", zice Domnul Dumnezeu.
\par 45 Și a fost iar cuvântul Domnului către mine:
\par 46 "Fiul omului, întoarce-ți fața spre Teman și rostește cuvântul tău spre miazăzi și proorocește împotriva pădurii din Negheb.
\par 47 Și zi pădurii celei de la miazăzi: Ascultă cuvântul Domnului: Asa zice Domnul Dumnezeu: Iată Eu voi aprinde foc în tine și va arde în tine tot copacul verde și tot lemnul uscat; și flacăra vâlvâitoare nu se va stinge și toată fața va fi arsă de ea de la Negheb până la miazănoapte.
\par 48 Și va vedea tot trupul că Eu, Domnul, am aprins focul și nu se va stinge".
\par 49 Și am zis: "0, Doamne, Dumnezeule, ei zic despre mine: "Nu cumva ne spune el o pildă?"

\chapter{21}

\par 1 Fost-a iarăși cuvântul Domnului către mine și mi-a zis:
\par 2 "Fiul omului, întoarce-ți fața spre Ierusalim și vorbește împotriva locașului lor cel sfânt și proorocește împotriva țării lui Israel.
\par 3 Și zi pământului lui Israel: Așa grăiește Domnul Dumnezeu: Iată, Eu sunt asupra ta și-Mi voi trage sabia din teaca ei și voi stârpi din tine pe cel drept și pe cel necredincios.
\par 4 Și ca să pierd din tine pe cel drept și pe cel necredincios, sabia Mea va ieși din teaca sa spre a lovi tot trupul de la miazăzi până la miazănoapte.
\par 5 Atunci va cunoaște tot trupul că Eu, Domnul, Mi-am scos sabia din teaca ei și nu se va mai întoarce.
\par 6 Tu, fiul omului, suspină atât încât să se frângă șalele tale, suspină cu amărăciune înaintea ochilor lor.
\par 7 Iar când îți vor zice: "Pentru ce suspini?" Spune-le: Pentru vestea ce vine... și toată inima se va tulbura, toate mâinile vor slăbi și toți genunchii vor tremura ca frunza. Iat-o vine și este aproape, zice Domnul Dumnezeu.
\par 8 Și a mai fost cuvântul Domnului către mine și mi-a zis:
\par 9 "Fiul omului, proorocește și zi: Așa grăiește Domnul Dumnezeu: "Sabia este ascuțită și oțelită.
\par 10 E ascuțită ca să junghie mai mult; e oțelită ca să scânteieze ca fulgerul. Ne vom bucura, oare, că sabia fiului meu disprețuiește tot lemnul?
\par 11 Eu am dat-o la oțelit, ca să se ia în mână, și acum această sabie e ascuțită și oțelită, gata să fie dată în mâna ucigătorului.
\par 12 Fiul omului, suspină și te vaită, că ea e trasă asupra poporului Meu, asupra tuturor căpeteniilor lui Israel; aceștia vor fi dați sub sabie cu poporul Meu, de aceea lovește-te cu mâinile peste coapse,
\par 13 Căci ea e și încercată acum; și ce este de mirare dacă ea va nesocoti și sceptrul? Nici acesta nu va rămâne, zice Domnul Dumnezeu.
\par 14 Dar tu, fiul omului, proorocește și lovește palmele una de alta și loviturile sabiei se vor îndoi și se vor întrei; aceasta este sabia măcelului, sabia cumplitului măcel, sabia care trebuie să-i urmărească,
\par 15 Pentru a arunca groază în inimi și a înmulți jertfele tot mai mult. Vai! La toate porțile lor voi pune sabie grozavă și ascuțită spre junghiere, care scânteiază ca fulgerul.
\par 16 Sabie, pregătește-te! Ia-o la dreapta și apucă la stânga, încotro vrei să-ți întorci fața ta!
\par 17 Iar Eu voi bate din palme și-Mi voi potoli mânia. Eu, Domnul, grăiesc acestea".
\par 18 Fost-a cuvântul Domnului către mine și mi-a zis:
\par 19 "Tu, fiul omului, închipuiește-ți două drumuri pe care trebuie să treacă sabia regelui Babilonului; acestea amândouă trebuie să plece din aceeași țară; apoi închipuiește-ți o mână, închipuiește-o în cetatea Babilonului, de unde pleacă drumurile.
\par 20 Închipuiește drumul pe care sabia trebuie să vină împotriva cetății Raba, a fiilor lui Amon și împotriva lui Iuda, împotriva Ierusalimului celui întărit,
\par 21 Pentru că regele Babilonului s-a oprit la o răspântie, unde încep două drumuri, și stă să ghicească: scutură săgețile, întreabă terafimii și cercetează ficatul.
\par 22 Sorțul din dreapta are scris pe el: "spre Ierusalim", încât el trebuie să îndrepte berbecii, să îndemne la ucidere și să scoată strigăte de război; să așeze berbecii împotriva porților, să ridice valuri și să facă turnuri de împresurare.
\par 23 Acest sorț s-a părut neadevărat în ochii acelora care au făcut jurăminte mincinoase; dar regele Babilonului, aducându-și aminte de această necredință a lor, a hotărât să ia Ierusalimul.
\par 24 De aceea așa zice Domnul Dumnezeu: De vreme ce voi vă aduceți aminte de nelegiuirea voastră, făcând ca fărădelegile voastre să fie vădite și scoțând la iveală păcatele voastre în toate faptele voastre; de vreme ce voi singuri vă aduceți aminte de acestea, veți fi prinși cu mâna.
\par 25 Și ție, căpetenie nelegiuită și rea a lui Israel, căreia și-a venit ziua acum, când nelegiuirea ta a ajuns la culme,
\par 26 Așa zice Domnul Dumnezeu: Diadema se va scoate, cununa va fi ridicată, lucrurile se vor schimba; cele smerite se vor înălța și cele înalte se vor smeri;
\par 27 O voi lepăda, o voi lepăda, o voi lepăda și nu va mai fi până va veni acela căruia se cuvine și o voi da lui.
\par 28 Iar tu, fiul omului, proorocește și spune: Așa grăiește Domnul Dumnezeu despre fiii lui Amon și despre ocara lor. Și le spune: Sabia, sabia este trasă pentru junghiere, este oțelită pentru măcel, ca să scânteieze ca fulgerul.
\par 29 În mijlocul acelor vedenii deșarte și prezicerilor tale mincinoase, ea te va face să cazi printre trupurile necredincioșilor, cărora le-a venit ziua atunci când nedreptatea și-a ajuns culmea.
\par 30 O voi întoarce oare în teaca ei? La locul unde ai fost făcut, în țara unde te-ai născut, te voi judeca;
\par 31 Voi vărsa asupra ta mânia Mea; voi sufla asupra ta focul urgiei Mele și te voi da în mâna oamenilor barbari, făuritori ai distrugerii.
\par 32 Mâncare focului vei fi; sângele tău va curge în mijlocul țării și nici nu se va mai pomeni de tine, că Eu, Domnul, am spus acestea".

\chapter{22}

\par 1 Fost-a cuvântul Domnului către mine:
\par 2 "Tu, fiul omului, voiești oare să judeci, să judeci cetatea sângelui? Spune-i toate urâciunile ei,
\par 3 Și-i spune: Așa zice Domnul Dumnezeu: O, cetate, care verși sânge în mijlocul tău, ca să-ți vină vremea, și care-ți faci idoli, ca să te întinezi!
\par 4 Cu sângele, pe care l-ai vărsat, te-ai făcut vinovată și cu idolii, pe care i-ai făcut, te-ai spurcat și ți-ai apropiat zilele tale și ai ajuns la sfârșitul anilor tăi. De aceea te voi da popoarelor spre batjocură și tuturor țărilor spre bătaie de joc.
\par 5 Cei de aproape și cei de departe ai tăi își vor bate joc de tine, care ți-ai întinat numele și ești plină de neorânduială.
\par 6 Iată cei ce povățuiesc în Israel, fiecare după măsura puterilor sale, au fost întru tine ca să verse sânge.
\par 7 La tine, tatăl și mama sunt disprețuiți, la tine străinul este chinuit, la tine orfanul și văduva sunt asupriți.
\par 8 Cele sfinte ale tale, tu nu le cinstești și zilele Mele de odihnă le calci.
\par 9 Clevetitori sunt destui în tine, ca să verse sânge; în munții tăi se mănâncă jertfe idolești și în mijlocul tău se fac urâciuni.
\par 10 La tine se descoperă goliciunea tatălui, la tine se siluiește femeia în vremea perioadei ei de necurăție.
\par 11 Unul face ticăloșie cu femeia vecinului său, altul se spurcă cu nora sa; unul siluiește pe sora sa, fiica tatălui său.
\par 12 În tine se ia mită, ca să se verse sânge; tu iei dobândă și camătă și cu silnicie apuci câștig de la fratele tău, iar pe Mine M-ai uitat, zice Domnul Dumnezeu.
\par 13 Și iată, Eu Mi-am lovit palmele Mele una de alta, văzând lăcomia ta, care se vede la tine, și vărsarea de sânge, care se săvârșește în mijlocul tău.
\par 14 Va suferi oare inima ta și vor fi oare tari mâinile tale în acele zile, când voi lucra împotriva ta? Eu, Domnul, am zis și voi face.
\par 15 Te voi împrăștia printre popoare și te voi vântura prin țări, voi pune capăt urâciunilor tale cele din tine.
\par 16 Singură te vei face disprețuită înaintea ochilor popoarelor și vei cunoaște că Eu sunt Domnul".
\par 17 Și a fost cuvântul Domnului către mine:
\par 18 "Fiul omului, casa lui Israel Mi s-a făcut zgură; toți sunt plumb și fier și cositor în cuptor, au ajuns ca niște zgură de argint.
\par 19 De aceea, așa zice Domnul Dumnezeu: De vreme ce toți v-ați făcut zgură, de aceea iată Eu vă voi aduna în Ierusalim.
\par 20 După cum se pune în cuptor la un loc argint și aramă și fier și cositor și plumb, ca să aprindă asupra lor foc și să le topească,
\par 21 Vă voi aduna și voi aprinde asupra voastră focul mâniei Mele și vă voi topi în mijlocul cetății.
\par 22 Cum se topește argintul în cuptor, așa vă veți topi și voi în mijlocul cetății, și veți ști că Eu, Domnul, am vărsat asupra voastră urgia Mea".
\par 23 Fost-a cuvântul Domnului către mine și mi-a zis:
\par 24 "Fiul omului, spune Ierusalimului: Ești o țară necurățită și neudată de ploaie în ziua mâniei.
\par 25 Mai-marii lui urzesc în el intrigi; ca niște lei ce mugesc și sfâșie prada, așa mănâncă ei sufletele; strâng avuții și lucruri prețioase și sporesc numărul văduvelor.
\par 26 Preoții lui calcă legea Mea și pângăresc lucrurile sfinte ale Mele; nu osebesc ce este sfânt de ce nu este sfânt și nu fac deosebire între curat și necurat; de la zilele Mele de odihnă și-au întors ochii, și Eu sunt înjosit de către aceștia.
\par 27 Căpeteniile lui în sânul lui sunt ca niște lupi care sfâșie prada; ei varsă sânge și ucid sufletele, ca să-și sature lăcomia.
\par 28 Proorocii lui tencuiesc toate cu ipsos, cu vedenii deșarte și cu prevestiri mincinoase, zicând: "Așa grăiește Domnul Dumnezeu", când Domnul nu le grăiește nimic.
\par 29 Iar poporul săvârșește silnicii și jaf; asuprește pe sărac și pe cel în necaz, chinuiește pe străin fără nici un drept.
\par 30 Am căutat printre ei să găsesc un om ca să se poarte cu dreptate și să stea înaintea feței Mele, pentru țara aceasta, ca să nu o pierd, și nu am găsit.
\par 31 Deci voi vărsa asupra lor mânia Mea; cu focul urgiei Mele îi voi pierde și purtarea lor o voi întoarce asupra capului lor", zice Domnul Dumnezeu.

\chapter{23}

\par 1 Și a fost iarăși cuvântul Domnului către mine și mi-a zis:
\par 2 "Fiul omului, au fost odată două femei, fiicele unei mame;
\par 3 Și acestea s-au desfrânat în tinerețea lor, s-au desfrânat în Egipt. Acolo sânul lor a fost atins și trupul lor feciorelnic acolo a fost pângărit.
\par 4 Numele lor erau: al celei mai mari Ohola și al surorii ei Oholiba. Ele erau ale Mele și au născut fii și fiice. Ohola este Samaria și Oholiba este Ierusalimul.
\par 5 Ohola a început să-Mi fie necredincioasă și s-a aprins după amanții ei, după Asirieni, vecinii săi,
\par 6 Dregători și căpetenii de cetăți, toți tineri și frumoși, îmbrăcați în purpură și călăreți iscusiți.
\par 7 Ea și-a făcut plăcerile cu aceia, care erau fruntașii Asirienilor, și cu toți aceia după care înnebunea, spurcându-se cu toți idolii lor.
\par 8 Dar ea n-a încetat a se desfrâna și cu Egiptenii, căci aceștia dormiseră cu ea în tinerețile ei, atinseseră sânul ei fecioresc și își vărsaseră desfrânarea lor peste dânsa.
\par 9 De aceea am dat-o și în mâna amanților ei, în mâinile Asirienilor, după care înnebunea.
\par 10 Aceștia au dezvelit rușinea ei și au luat pe fiii ei și pe fiicele ei, iar pe ea au ucis-o cu sabia. Și a ajuns ea de ocară între femei, după ce a fost osândită.
\par 11 Sora ei, Oholiba, a văzut aceasta și a fost și mai stricată în poftele sale și desfrânările ei au întrecut pe ale surorii sale.
\par 12 Ea s-a aprins după fiii lui Asur, vecinii ei, după dregători și căpetenii de cetăți, toți tineri aleși, îmbrăcați frumos și călăreți iscusiți.
\par 13 Am văzut deci că s-a pângărit și aceasta; ca și sora sa, urmând amândouă aceeași cale.
\par 14 Dar aceasta a mers și mai departe cu desfrânarea, pentru că văzând zugrăvite pe pereți chipuri de bărbați, chipuri de caldei, zugrăviți cu vopsele,
\par 15 Încinși peste mijloc cu centuri și pe cap cu turbane largi, având înfățișarea de mari viteji, după felul Babilonenilor, a căror patrie este țara Caldeilor,
\par 16 Ea s-a aprins după ei la cea dintâi privire a lor și a trimis soli la ei, în Caldeea.
\par 17 Și au venit fiii Babilonului la aceasta, în palatul ei de desfrânare, și au pângărit-o cu desfrânările lor și ea s-a pângărit cu ei și apoi sufletul său s-a depărtat de ei.
\par 18 Iar când ea s-a dedat pe față la desfrânările sale și și-a descoperit goliciunea sa, atunci s-a depărtat inima Mea și de ea, cum se depărtase inima Mea și de sora ei;
\par 19 Căci ea a sporit desfrânările sale, aducându-și aminte de zilele tinereții sale, când se desfrânase în țara Egiptului;
\par 20 Și s-a aprins după amanții ei, cei cu trup ca de măgar și cu înfierbântarea ca de armăsar.
\par 21 Și așa ți-ai adus tu aminte de desfrânarea din tinerețile tale, când Egiptenii îți strângeau sânii feciorelnici ai pieptului tău.
\par 22 De aceea, Oholiba, așa zice Domnul Dumnezeu: Iată, Eu voi ațâța împotriva ta pe amanții tăi de. care s-a dezgustat sufletul tău și-i voi aduce împotriva ta din toate părțile.
\par 23 Voi aduce pe fiii Babilonului și pe toți Caldeii: din Pecod, din Șoa și din Coa și, împreună cu ei, pe toți Asirienii, tineri frumoși, căpetenii de provincii, căpetenii de cetăți, slujitori și oameni vestiți; toți călăreți iscusiți.
\par 24 Și vor veni împotriva ta cu arme, cu cai și cu căruțe și cu mulțime de popor și te vor înconjura din toate părțile, cu sulițe, cu săbii și cu scuturi; și te voi da lor spre judecată, și cu judecata lor te vor judeca.
\par 25 Și voi întoarce zelul Meu împotriva ta, și ei se vor purta cu tine cu urgie; îți vor tăia nasul și urechile, iar celelalte ale tale vor cădea de sabie. Lua-vor pe fiii și pe fiicele tale, iar celelalte ale tale de foc vor fi mâncate.
\par 26 Vor lua de pe tine hainele tale, vor smulge gătelile tale,
\par 27 Și voi pune capăt desfrânărilor tale și destrăbălărilor tale aduse din țara Egiptului, iar tu nu-ți vei mai întoarce ochii spre ei și de Egipt nu-ți vei mai aminti,
\par 28 Că așa zice Domnul Dumnezeu: Iată, te dau în mâinile acelora pe care i-ai urât, în mâinile acelora de la care s-a întors sufletul tău.
\par 29 Aceia se vor purta cu tine crud și-ți vor lua tot ce ai dobândit cu osteneală; te vor lăsa goală, cu totul goală, și descoperită va fi goliciunea ta cea rușinoasă și desfrânarea ta și pângărirea.
\par 30 Acestea ți se vor face pentru desfrânarea ta cu popoarele, cu idolii cu care te-ai pângărit.
\par 31 Ai mers pe calea surorii tale și de aceea îți voi da în mână și paharul ei.
\par 32 Așa zice Domnul Dumnezeu: Vei bea paharul surorii tale, cel adânc și larg, și vei ajunge de râs și de batjocură, căci este mare paharul acesta.
\par 33 Cuprinsă vei fi de beție și de amărăciune, că acesta este paharul groazei și al pustiirii, paharul surorii tale, Samaria.
\par 34 Și tu îl vei bea, îl vei goli și cioburile lui le vei roade și-ți vei sfâșia pieptul; că Eu am spus aceasta, zice Domnul Dumnezeu.
\par 35 De aceea, așa zice Domnul Dumnezeu: Pentru că M-ai uitat și te-ai întors de la Mine, de aceea suferă și tu pentru nelegiuirea ta și pentru desfrânarea ta".
\par 36 Zis-a Domnul către mine: "Fiul omului, vrei tu să judeci pe Ohola și pe Oholiba? Spune-le ticăloșiile lor!
\par 37 Căci ele s-au desfrânat și pe mâinile lor au sânge; s-au desfrânat cu idolii lor, și pe fiii lor, care Mi i-au născut, i-au trecut prin foc, ca să le fie de mâncare.
\par 38 Iată ce Mi-au mai făcut ele: Au pângărit locașul Meu cel sfânt în aceeași zi și zilele Mele de odihnă le-au călcat;
\par 39 Pentru că atunci când ele au junghiat copiii lor pentru idolii lor, în aceeași zi au venit în locașul Meu cel sfânt ca să-l pângărească; iată cum s-au purtat ele în locașul Meu!
\par 40 Afară de aceasta, ele umblau după oameni veniți de departe; trimiteau la ei soli și ei veneau. Pentru ei te-ai îmbăiat, ți-ai făcut ochii cu dresuri și te-ai gătit cu podoabe.
\par 41 Te-ai așezat pe pat luxos, în fața căruia era gătită o masă pe care tu ai pus tămâia Mea și untdelemnul Meu.
\par 42 Și răsunau strigătele de veselie ale mulțimii nepăsătoare, din pricina gloatei de oameni aduși din pustiu; și ei au pus brățări la mâinile femeilor și coroane mărețe pe capetele lor.
\par 43 Atunci am zis despre cea îmbătrânită în desfrânări: Acum vor sfârși desfrânările ei o dată cu ea.
\par 44 Și au venit la ea ca la o desfrânată; așa veneau la aceste femei desfrânate, adică la Ohola și la Oholiba.
\par 45 Dar bărbați înțelepți le vor judeca și le or osândi cu osânda desfrânatelor și cu osânda celor ce varsă sânge, pentru că ele sunt niște desfrânate și mâinile lor sunt însângerate.
\par 46 Căci așa zice Domnul Dumnezeu: Să se adune împotriva lor mulțimea și să le dea urgiei și jafului.
\par 47 Mulțimea le va ucide cu pietre și le va tăia cu săbiile; vor omorî pe fiii lor și pe fiicele lor și casele lor le vor arde cu foc.
\par 48 Așa voi pune capăt desfrânării în țara aceasta și toate femeile vor lua învățătură și nu vor mai face lucruri rușinoase ca ele.
\par 49 Nelegiuirea voastră va cădea asupra voastră, și veți purta greutatea păcatelor voastre de închinare la idoli, și veți ști că Eu sunt Domnul Dumnezeu".

\chapter{24}

\par 1 Fost-a cuvântul Domnului către mine, în anul al nouălea, în luna a zecea, în ziua a zecea a lunii, și mi-a zis:
\par 2 "Fiul omului, scrie-ți numele acestei zile, anume al acestei zile, că chiar în ziua aceasta regele Babilonului va păși spre Ierusalim.
\par 3 Rostește dar pentru neamul de răzvrătiți o pildă și le spune: Așa zice Domnul Dumnezeu: Pune un cazan și toarnă în el apă;
\par 4 Pune în el bucăți de carne, tot bucăți din cele mai bune, șolduri și spete, și-l umple cu cele mai bune oase;
\par 5 Ia ceea ce este mai bun din turmă, pune lemne dedesubt, fierbe-l în clocote, așa ca să fiarbă și oasele din el.
\par 6 De aceea, așa zice Domnul Dumnezeu: Vai de cetatea sângelui! Vai de căldarea în care este rugină și de pe care nu se mai ia rugina! Aruncați bucată cu bucată din el și nu alegeți prin sorț.
\par 7 Că sângele pe care ea l-a vărsat este în mijlocul ei. Ea l-a lăsat pe stâncă goală; nu l-a vărsat pe pământ, unde s-ar fi putut acoperi cu țărână.
\par 8 Pentru a-Mi ațâța mânia, pentru a Mă răzbuna, am lăsat sângele ei pe stâncă goală, ca să nu fie acoperit".
\par 9 De aceea așa zice Domnul Dumnezeu: "Vai de cetatea sângelui! Că voi aprinde un foc mare.
\par 10 Adu lemne, aprinde focul, fierbe carnea; lasă să se îngroașe tot și oasele să se ardă.
\par 11 Pune căldarea goală pe cărbuni, ca să se încălzească și ca arama ei să se înfierbânte și să se topească murdăria din ea și toată rugina de pe ea să se ducă.
\par 12 Munca va fi grea, dar rugina cea multă de pe ea nu se va duce; și în foc rugina va rămâne pe ea.
\par 13 Murdăria ta e atât de grozavă, că oricât te voi curăți, tu tot necurată vei fi; de acum înainte nu te vei mai curări de murdăria ta, până ce nu-Mi voi potoli urgia Mea asupra ta.
\par 14 Eu sunt Domnul, și Eu spun că vor veni acestea și le voi împlini, nu le voi schimba, nici nu voi cruța ceva; după purtările și după faptele tale te voi judeca", zice Domnul Dumnezeu.
\par 15 Fost-a cuvântul Domnului către mine și mi-a zis:
\par 16 "Fiul omului, iată Eu printr-o lovitură îți voi lua mângâierea ochilor tăi; dar tu nu te tângui și nu plânge, și nici lacrimile să nu-ți curgă;
\par 17 Suspină în ascuns și plângere pentru morți să nu faci; pune-ți turbanul pe cap și încălțăminte în picioare; barba să nu ți-o acoperi și pâine de la străini să nu mănânci".
\par 18 După ce dimineața am grăit eu poporului cuvântul Domnului, seara mi-a murit femeia și a doua zi am făcut precum mi se poruncise.
\par 19 Poporul însă mi-a zis: "Pentru ce nu ne spui ce însemnătate au pentru noi cele ce le faci tu?"
\par 20 Și le-am răspuns: "A fost cuvântul Domnului către mine și mi-a zis:
\par 21 Spune casei lui Israel: Așa grăiește Domnul Dumnezeu: Iată locașul Meu cel sfânt, mândria puterilor voastre, bucuria ochilor voștri și dragostea sufletelor voastre, îl voi da spre pângărire, iax fiii și fiicele voastre, pe care i-ați părăsit, vor cădea de sabie;
\par 22 Și veți face și voi ceea ce fac eu: bărbile nu le veți acoperi și pâine de la străini nu veți mânca;
\par 23 Turbanele voastre le veți avea pe cap și încălțămintele voastre în picioare; nu vă veți tângui, nici nu vă veți plânge, ci vă veți stinge pentru păcatele voastre și veți suspina unul către altul.
\par 24 Iezechiel va fi semn pentru voi: ceea ce el a făcut, veți face și voi întocmai; și când se vor împlini acestea, veți ști că Eu sunt Domnul Dumnezeu.
\par 25 Și tu, fiul omului, în ziua când le voi fi luat puterea lor, mândria lor falnică, bucuria ochilor lor, desfătările sufletului lor, fiii și fiicele lor,
\par 26 În ziua aceea va veni la tine cel scăpat de acolo, care îți va aduce știrea;
\par 27 În ziua aceea se va deschide gura ta către cel scăpat și vei grăi și nu vei mai fi mut și vei fi semn pentru ei, și ei vor afla că Eu sunt Domnul".

\chapter{25}

\par 1 Fost-a cuvântul Domnului către mine și mi-a zis:
\par 2 "Fiul omului, întoarce-ți fața spre fiii lui Amon și grăiește împotriva lor.
\par 3 Spune fiilor lui Amon: Ascultați cuvântul Domnului Dumnezeu: Așa zice Domnul Dumnezeu: Deoarece tu faci: Aha! Aha! împotriva locașului Meu cel sfânt când el e pângărit și când țara lui Israel e pustiită și casa lui Iuda dusă în robie,
\par 4 De aceea, iată Eu te voi da de moștenire fiilor Răsăritului, care își vor face sălașele la tine și vor așeza la tine corturile și vor mânca fructele și laptele tău.
\par 5 Voi face din Raba un staul de cămile și din cetățile lui Amon o stână de oi, și veți ști că Eu sunt Domnul.
\par 6 Că așa zice Domnul Dumnezeu: Pentru că ai bătut din palme și ai bătut din picioare și din suflet te-ai bucurat cu tot disprețul tău pentru țara lui Israel,
\par 7 De aceea iată Eu Îmi voi întinde mâna împotriva ta și te voi da popoarelor spre jefuire și te voi stârpi din numărul popoarelor și din numărul țărilor te voi șterge; te voi zdrobi și vei ști că Eu sunt Domnul".
\par 8 Așa zice Domnul Dumnezeu: "Pentru că Moab și Seir zic: Iată și casa lui Iuda este ca toate neamurile;
\par 9 De aceea iată Eu, începând de la cetăți, de la toate cetățile lui de pe hotar, de la Bet-Ieșimot, Baal-Meon și Chiriataim, podoabele țării, voi descoperi colinele Moabului și voi nimici cetățile lui în toată întinderea lui.
\par 10 Și-l voi deschide și-l voi da în stăpânire fiilor Răsăritului împreună cu țara fiilor lui Amon, ca să nu se mai amintească despre fiii lui Amon printre neamuri.
\par 11 Voi săvârși astfel judecăți împotriva lui Moab, și ei vor ști că Eu sunt Domnul".
\par 12 Așa zice Domnul Dumnezeu: "Pentru că Edom s-a răzbunat cumplit împotriva casei lui Iuda și a păcătuit greu, săvârșind răzbunare asupra ei,
\par 13 De aceea, așa zice Domnul Dumnezeu: Îmi voi întinde mâna împotriva Edomului și voi pierde din el pe oameni și pe animale și îl voi face pustietate; de la Teman până la Dedan toți vor cădea de sabie.
\par 14 Răzbunarea Mea împotriva Edomului o voi săvârși prin mâna poporului Meu Israel; el va lucra în Edom după mânia Mea și după urgia Mea și vor ști Edomiții ce este răzbunarea Mea", zice Domnul Dumnezeu.
\par 15 Așa grăiește Domnul Dumnezeu: "Pentru că Filistenii s-au purtat răzbunători și s-au răzbunat cu ură în suflet și cu o veșnică dușmănie de moarte,
\par 16 De aceea așa zice Domnul Dumnezeu: "Iată Eu Îmi voi întinde mâna împotriva Filistenilor și voi pierde pe Cheretieni și restul locuitorilor de pe țărmul mării îl voi stârpi.
\par 17 Voi săvârși împotriva lor cumplite răzbunări, pedepse grele, și vor ști că Eu sunt Domnul, când voi săvârși asupra lor răzbunarea Mea".

\chapter{26}

\par 1 În anul al unsprezecelea, în ziua întâi a lunii întâi, a fost cuvântul Domnului către mine:
\par 2 "Fiul omului, pentru că Tirul face împotriva Ierusalimului: Aha! Aha! și zice: Iată el - poarta popoarelor - este dărâmat; acum aleargă la mine; eu mă umplu, iar el se pustiește,
\par 3 De aceea, așa zice Domnul Dumnezeu: Iată sunt împotriva ta, Tirule, și voi ridica împotriva ta popoare multe, cum își ridică marea valurile sale.
\par 4 Voi sfărâma zidurile Tirului și turnurile lui le voi dărâma; voi mătura praful din el și-l voi face stâncă goală.
\par 5 Loc de uscat mrejele va fi el la mare, pentru că a zis acestea, grăiește Domnul Dumnezeu, și va fi prada neamurilor.
\par 6 Iar fiicele lui, care sunt pe pământ, vor fi ucise cu sabia și vor ști că Eu sunt Domnul".
\par 7 Că așa zice Domnul Dumnezeu: "Iată Eu voi aduce împotriva Tirului de la miazănoapte pe Nabucodonosor, regele Babilonului, regele regilor, cu cai, cu care și cu călăreți, cu oștire și cu mulțime de neamuri.
\par 8 Pe fiicele tale cele din câmpie el le va ucide cu sabia și va ridica împotriva ta turnuri de împresurare, va face val împrejurul tău și va pune împotriva ta scuturile.
\par 9 Spre zidurile tale va împinge berbecii de spart ziduri și turnurile tale le va dărâma cu topoarele.
\par 10 De mulțimea cailor lui vei fi acoperit de praf și de zgomotul călăreților, al carelor și al roților se vor cutremura zidurile tale, când va intra el pe porțile tale, cum se intră într-o cetate sfărâmată.
\par 11 Cu copitele cailor săi va călca el toate ulițele tale, pe poporul tău îl va ucide cu sabie, iar puternicele tale columne le va răsturna la pământ.
\par 12 Vor jefui bogăția ta și mărfurile tale le vor fura, vor dărâma zidurile tale și frumoasele tale case le vor strica, și pietrele tale și arborii tăi și pământul tău le vor arunca în apă.
\par 13 Voi curma zgomotul cântecelor tale și sunet de chitară nu se va mai auzi la tine.
\par 14 Te voi face stâncă goală și loc de uscat mrejele vei fi; nu vei mai fi zidit din nou, căci Domnul a spus acestea", zice Domnul Dumnezeu.
\par 15 Așa zice Domnul Dumnezeu Tirului: "De zgomotul căderii tale și de geamătul răniților tăi, când se va face măcelul în tine, nu se vor cutremura oare insulele?
\par 16 Toți stăpânitorii mării se vor cobori de pe tronurile lor, își vor scoate purpurile și își vor dezbrăca hainele lor cele brodate; cu groază se vor îmbrăca, vor ședea la pământ și vor tremura fără încetare și vor fi umiliți din pricina ta.
\par 17 Vor ridica plângere împotriva ta și-ți vor zice: "Cum ai pierit tu, cel locuit de stăpânitorii mărilor, cetate vestită, care erai tare pe mare și tu și locuitorii tăi, care aduceai groază asupra tuturor celor care locuiau pe uscat!
\par 18 Acum, în ziua căderii tale, s-au cutremurat insulele; insulele de pe mare sunt îngrozite de sfârșitul tău".
\par 19 Că așa zice Domnul Dumnezeu: "Când te voi face cetate pustie, asemenea cetăților nelocuite, când voi ridica împotriva ta adâncul și te vor acoperi apele cele mari,
\par 20 Atunci te voi coborî cu cei ce se coboară în mormânt, la poporul de odinioară și te voi așeza în adâncurile pământului, în pustietăți veșnice, cu cei ce s-au dus în mormânt, ca să nu mai fii locuit și să nu mai dăinuiești în țara celor vii.
\par 21 Groază te voi face și nu vei mai fi; te vor căuta și nu te vor mai găsi în veci", zice Domnul Dumnezeu.

\chapter{27}

\par 1 Fost-a cuvântul Domnului către mine și mi-a zis:
\par 2 "Și tu, fiul omului, ridică plângere împotriva Tirului,
\par 3 Și zi către el: O, tu, cel ce ești așezat la marginea mării și faci negoț cu popoarele a nenumărate insule, așa grăiește Domnul Dumnezeu: Tirule, tu zici: "Eu sunt o corabie de desăvârșită frumusețe!"
\par 4 Ținutul tău este în largul mării; cei ce te-au zidit te-au făcut minunat de frumos.
\par 5 Toate acoperișurile corăbiilor tale le-ai făcut din chiparos de Senir și cedru de Liban s-a adus, ca să-ți facă ție catarge.
\par 6 Vâslele tale s-au făcut de stejar din Vasan; băncile și le-au făcut din lemn de cimșir, împodobite cu fildeș din insulele Chitim;
\par 7 Pânzele tale, din vison de Egipt brodat, îți slujeau ca steag. Porfira violetă și stacojie din insulele Elișa alcătuiau acoperământul tău.
\par 8 Locuitorii Sidonului și ai Arvadului erau vâslașii tăi; și cei mai iscusiți ai tăi, Tirule, erau cârmaci.
\par 9 Bătrânii din Ghebal și meșterii lui erau la tine, ca să-ți repare stricăciunile. Toate corăbiile mării și corăbiile lor erau la tine, ca să facă negoțul tău.
\par 10 Perși, Lidieni și Libieni se aflau în oștirea ta și erau oamenii tăi de război; atârnau în tine scuturile și coifurile lor.
\par 11 Fiii Arvadului împreună cu oștirea ta stăteau împrejur pe zidurile tale și în turnurile tale se aflau oameni viteji; aceștia își atârnau tolbele sus pe zidurile tale și desăvârșeau frumusețea ta.
\par 12 Cei din Tarsis făceau negoț cu tine pentru tot felul de bogății și veneau la târgul tău cu argint, cu fier, cu cositor și plumb.
\par 13 Iavan, Tubal și Meșec făceau negoț cu tine, dând în schimb, pe mărfurile tale, suflete omenești și vase de aramă, în piețele tale.
\par 14 Cei din casa Togarma aduceau la târgul tău cai și căruțe.
\par 15 Fiii lui Dedan făceau negoț cu tine; insule multe luau mărfurile tale și-ți plăteau cu fildeș și abanos.
\par 16 Pentru mulțimea mărfurilor tale făcea negoț cu tine Siria și venea la târgul tău cu smaralde, cu purpură, cu țesături alese, cu în subțire, cu mărgean și cu rubine.
\par 17 Iuda și ținuturile lui Israel făceau negoț cu tine și pe mărfurile tale dădeau grâu de Minit și turte, miere, ulei și balsam.
\par 18 Damascul făcea negoț cu tine și, pentru mulțimea multă de lucruri și pentru toate bunătățile ce aveai tu din belșug, îți aducea vin de Helbon și lână albă.
\par 19 Vedan și Iavan din Uzal îți plăteau pe mărfurile tale fier lucrat; casie și trestie mirositoare și se aduceau în schimb.
\par 20 Dedan făcea negoț cu tine cu pături pentru pus pe cai.
\par 21 Arabia și toate căpeteniile din Chedar făceau negoț cu tine; miei, berbeci și țapi îți dădeau în schimb pentru mărfurile tale.
\par 22 Neguțătorii din Șeba și Rama făceau negoț cu tine, dând în schimb tot felul de aromate alese, felurite pietre scumpe și aur, pentru mărfurile tale.
\par 23 Haran, Cane și Eden, neguțătorii din Șeba, Asiria și Chilmad făceau negoț cu tine.
\par 24 Aceștia făceau negoț cu tine cu haine scumpe, cu mantii de purpură violetă și brodată, stofe țesute cu felurite culori, funii împletite și tari, puse în lăzi de cedru, aduse pe piețele tale.
\par 25 Corăbiile Tarsisului erau caravanele tale pentru negoțul tău și prin acestea ai ajuns tu bogat și foarte slăvit pe mare.
\par 26 Vâslașii tăi te-au făcut să călătorești pe apele cele mari, dar un vânt de la răsărit te va sfărâma în mijlocul mărilor.
\par 27 Bogăția ta și mărfurile tale, corăbierii și cârmacii tăi, cei ce dreg crăpăturile corăbiilor, cei ce fac schimb de mărfuri cu tine, toți ostașii care se află în tine se vor prăbuși în inima mărilor în ziua căderii tale.
\par 28 De strigătul cârmacilor tăi se vor cutremura împrejurimile.
\par 29 Se vor coborî din corăbiile lor toți vâslașii, corăbierii și toți cârmacii mării vor sta pe uscat;
\par 30 Vor plânge pentru tine cu mare glas, presărându-și capetele lor cu cenușă și tăvălindu-se în pulbere;
\par 31 Își vor tunde pentru tine părul până la piele, cu sac se vor îmbrăca și vor plânge după tine cu plângere mare de durerea inimii;
\par 32 Și în durerea lor vor cânta cântare de jale pentru tine și te vor baci așa: "Cine a fost ca Tirul, ca această cetate dărâmată în mare!"
\par 33 Când veneau mărfurile tale de pe mări, tu săturai popoare multe; prin mulțimea bogăției tale și prin negoțul tău îmbogățeai pe regii pământului;
\par 34 Iar când ai fost sfărâmat de mări în adâncul apelor, mărfurile tale și tot ce se grămădea în tine au căzut împreună cu tine.
\par 35 Toți locuitorii insulelor s-au îngrozit de tine și regii lor s-au cutremurat și s-au schimbat la față.
\par 36 Negustorii popoarelor fluieră asupra ta. Tu ai ajuns o groază, ești nimicit pentru totdeauna!

\chapter{28}

\par 1 Fost-a cuvântul Domnului către mine și mi-a zis:
\par 2 "Fiul omului, spune celui ce domnește în Tir: Așa zice Domnul Dumnezeu: Inima ta s-a înălțat și a zis: "Sunt un dumnezeu și stau pe scaunul lui Dumnezeu în inima mărilor, dar tu, deși nu ești Dumnezeu, ci om, îți închipui în inima ta că ești la fel cu Dumnezeu;
\par 3 Iată, tu îți închipui că ești mai înțelept decât Daniel și nu sunt taine ascunse pentru tine;
\par 4 Prin înțelepciunea ta și cu mintea ta ți-ai agonisit bogăție și ai adunat în vistieriile tale argint și aur;
\par 5 Prin înțelepciunea ta cea mare, prin ajutorul negoțului tău, ți-ai sporit bogăția și mintea ta s-a îngâmfat cu bogăția ta;
\par 6 De aceea, așa zice Domnul Dumnezeu: Pentru că tu te-ai asemănat cu Dumnezeu,
\par 7 Iată, Eu voi aduce împotriva ta pe străinii cei mai răi din toate popoarele, și aceia își vor scoate sabia împotriva frumoasei tale înțelepciuni și vor întina strălucirea ta;
\par 8 În mormânt te voi coborî și vei muri în inima mărilor de moartea celor uciși.
\par 9 Spune-vei oare înaintea ucigașului tău: "Eu sunt un dumnezeu", când tu ești un om în mâna celui care te ucide, iar nu Dumnezeu?
\par 10 Vei muri de mâna străinilor, de moartea celor netăiați împrejur, căci Eu am spus aceasta", zice Domnul Dumnezeu.
\par 11 Și a fost cuvântul Domnului către mine și mi-a zis:
\par 12 "Fiul omului, plânge pe regele Tirului și-i spune: Așa zice Domnul Dumnezeu: Tu erai pecetea desăvârșiri, deplinătatea înțelepciunii și cununa frumuseții.
\par 13 Tu te aflai în Eden, în grădina lui Dumnezeu; hainele tale erau împodobite cu tot felul de pietre scumpe: cu rubine, topaze și diamante, cu crisolit, onix și iaspis, cu safir, smarald, carbuncul și aur; toate erau pregătite și așezate cu iscusință în cuibulețe și puse pe tine în ziua în care ai fost făcut.
\par 14 Tu erai heruvimul pus ca să ocrotești; te așezasem pe muntele cel sfânt al lui Dumnezeu, și umblai prin mijlocul pietrelor celor de foc.
\par 15 Fost-ai fără prihană în căile tale din ziua facerii tale și până s-a încuibat în tine nelegiuirea.
\par 16 Din pricina întinderii negoțului tău, lăuntrul tău s-a umplut de nedreptate și ai păcătuit, și Eu te-am izgonit pe tine, heruvim ocrotitor, din pietrele cele scânteietoare și te-am aruncat din muntele lui Dumnezeu, ca pe un necurat.
\par 17 Din pricina frumuseții tale s-a îngâmfat inima ta, și pentru trufia ta ți-ai pierdut înțelepciunea. De aceea te-am aruncat la pământ și te voi da înaintea regilor spre batjocură.
\par 18 Prin mulțimea nelegiuirilor tale, săvârșite în negoțul tău nedrept, ți-ai pângărit altarele tale; și Eu voi scoate din mijlocul tău foc, care te va și mistui; și te voi preface în cenușă pe pământ înaintea ochilor tuturor celor ce te văd.
\par 19 Toți cei ce te cunosc între popoare se vor mira de tine, vei ajunge o groază și în veci nu vei mai fi".
\par 20 Fost-a către mine cuvântul Domnului și mi-a zis:
\par 21 "Fiul omului, întoarce-ți fața spre Sidon, proorocește împotriva lui și spune:
\par 22 Așa grăiește Domnul Dumnezeu: Iată Eu sunt împotriva ta, Sidoane; Mă voi preaslăvi în mijlocul tău și se va ști că Eu sunt Domnul, când te voi judeca și-Mi voi arăta sfințenia Mea în mijlocul tău.
\par 23 Voi trimite împotriva ta ciumă și vărsare de sânge pe ulițele tale și vor cădea uciși în mijlocul tău de sabia care te va lovi din toate părțile, și vor ști toți că Eu sunt Domnul.
\par 24 Și nu vei mai fi pentru casa lui Israel spin care rănește și ciulin care sfâșie printre cei ce o înconjoară și o urăsc, și vor ști toți că Eu sunt Domnul Dumnezeu".
\par 25 Așa zice Domnul Dumnezeu: "Când voi aduna casa lui Israel din mijlocul popoarelor unde este împrăștiată și voi arăta prin aceasta sfințenia Mea în ochii neamurilor și când va locui ea în pământul său, pe care l-am dat robului Meu Iacov,
\par 26 Ei vor locui acolo în siguranță, își vor face case, vor sădi vii. Când voi face judecăți asupra tuturor celor dimprejur care îi disprețuiesc, vor ști că Eu sunt Domnul Dumnezeul lor".

\chapter{29}

\par 1 În ziua a douăsprezecea a lunii a zecea din anul al zecelea după robirea lui Ioiachim, a fost cuvântul Domnului către mine și mi-a zis:
\par 2 "Fiul omului, întoarce-ți fața spre Faraon, regele Egiptului, și proorocește împotriva lui și a tot Egiptul,
\par 3 Și grăiește și zi: Așa zice Domnul Dumnezeu: Iată Eu sunt împotriva ta, Faraoane, rege al Egiptului, crocodilul cel mare, care stai lungit între râurile tale și zici: "Al meu este râul și eu l-am făcut pentru mine".
\par 4 Eu însă voi înfige cârligul în fălcile tale și de solzii tăi voi lipi peștii râurilor tale și te voi târî afară din râurile tale, cu tot peștele râurilor tale, care s-a lipit de solzii tăi;
\par 5 Și te voi arunca în pustiu pe tine și tot peștele din râurile tale și vei cădea în câmpia goală și nu te vor lua, nici te vor ridica; te voi da mâncare fiarelor pământului și păsărilor cerului.
\par 6 Și vor ști toți locuitorii Egiptului că Eu sunt Domnul; pentru că ei au fost pentru casa lui Israel toiag de trestie.
\par 7 Când ei te prindeau, tu te sfărâmai în mâinile lor și tu le sfâșiai toată mâna; iar când ei se sprijineau de tine, tu te rupeai și le zdruncinai coapsele.
\par 8 De aceea, așa grăiește Domnul Dumnezeu: Iată, Eu voi aduce împotriva ta sabie și voi nimici oamenii și vitele din tine;
\par 9 Și va ajunge pământul Egiptului pustiu și deșert și vor afla că Eu sunt Domnul. Pentru că el zice: "Al meu este râul și eu l-am făcut".
\par 10 De aceea, iată Eu vin împotriva fluviilor tale și voi face pământul Egiptului pustiu între pustiurile ce se întind de la Migdol până la Siena și până în hotarele Etiopiei.
\par 11 Picior de om nu va trece prin el, nici picior de vită nu va trece prin el și nu va locui nimic în el patruzeci de ani.
\par 12 Voi face pământul Egiptului o pustietate între țările pustiite; și cetățile lui între cetățile pustiite vor fi pustii patruzeci de ani și pe Egipteni îi voi împrăștia printre neamuri și-i voi risipi prin țări".
\par 13 Așa zice Domnul Dumnezeu: "După trecerea celor patruzeci de ani, voi aduna pe Egipteni dintre neamurile printre care au fost împrăștiați,
\par 14 Și voi aduce înapoi pe prinșii de război ai Egiptului și-i voi așeza iarăși în țara Patros, în pământul nașterii lor, și vor fi acolo un regat slab.
\par 15 Va fi mai slab decât celelalte regate și nu se vor mai înălța peste popoare; îl voi mai micșora, ca să nu mai domnească peste popoare;
\par 16 Și nu va mai fi de acum înainte pentru casa lui Israel pricină de încredere, ci îi va aduce aminte de nelegiuirea ei, că s-a dat de partea Egiptului, și vor cunoaște că Eu sunt Domnul".
\par 17 În ziua întâi a lunii întâi din anul al douăzeci și șaptelea de la robirea lui Ioiachim, a fost cuvântul Domnului către mine și mi-a zis:
\par 18 "Fiul omului, Nabucodonosor, regele Babilonului, și-a obosit oștirile sale printr-o muncă grea împotriva Tirului; toate capetele s-au pleșuvit și toți umerii sunt răniți; dar nici pentru el, nici pentru oștirile lui nu este nici o răsplată de la Tir pentru lucrarea pe care a făcut-o împotriva lui.
\par 19 De aceea, așa zice Domnul Dumnezeu: Iată, Eu dau lui Nabucodonosor, regele Babilonului, țara Egiptului, ca să prade bogăția lui și să facă jaf în el; aceasta va fi răsplata oștirilor lui.
\par 20 Ca răsplată pentru lucrarea pe care a făcut-o Tirului, Eu îi dau țara Egiptului, pentru că acest lucru l-a făcut el pentru Mine, zice Domnul Dumnezeu.
\par 21 În ziua aceea voi face să crească cornul casei lui Israel și ție-ți voi deschide gura în mijlocul lor și vor ști că Eu sunt Domnul".

\chapter{30}

\par 1 Fost-a cuvântul Domnului către mine și mi-a zis:
\par 2 "Fiul omului, proorocește și zi: Așa grăiește Domnul Dumnezeu: Plângeți! O, ce zi!
\par 3 Că se apropie ziua, se apropie ziua Domnului, ziua cea întunecată! Vine vremea neamurilor.
\par 4 Atunci se va duce sabia în Egipt și groază se va lăți în Etiopia; vor cădea în Egipt cei loviți și bogățiile lui se vor lua și vor fi dărâmate temeliile lui.
\par 5 Etiopia, Libia, Lidia, străini de toate neamurile, Cub și fiii țării așezământului vor cădea împreună cu ei de sabie.
\par 6 Așa zice Domnul: "Vor cădea sprijinitorii Egiptului și îngâmfarea puterii lui se va prăbuși; de la Migdol și până la Siena vor cădea în el de sabie; așa zice Domnul Dumnezeu.
\par 7 Și va fi el pustiu între pustiuri și cetățile lor vor face parte dintre cetățile pustiite.
\par 8 Și vor ști că Eu sunt Domnul, când voi trimite foc asupra Egiptului și toți sprijinitorii lui vor fi zdrobiți.
\par 9 În ziua aceea vor merge vestitori de la Mine pe corăbii, ca să îngrozească pe Etiopienii cei fără de grijă, și se va întinde groaza la ei, ca în ziua Egiptului, căci iat-o că vine".
\par 10 Așa zice Domnul Dumnezeu: "Voi pune capăt mulțimii Egiptului prin mâna lui Nabucodonosor, regele Babilonului.
\par 11 El și, împreună cu el, poporul lui, cel mai strașnic dintre popoare, vor fi aduși pentru pieirea acestei țări; și își vor scoate săbiile lor împotriva Egiptului și vor umple țara de uciși.
\par 12 Fluviile lor le voi usca și voi da pământul în mâinile celor răi; prin mâna străinilor voi pustii țara și tot ce este în ea. Eu, Domnul, am spus acestea".
\par 13 Așa zice Domnul Dumnezeu: "Pierde-voi idolii și pe dumnezeii cei mincinoși din Nof (Memfis). Nu va mai fi prinț în țara Egiptului și voi răspândi groaza în țara Egiptului.
\par 14 Voi pustii Patrosul și foc voi trimite asupra Țoanului și voi rosti judecată asupra lui No (Teba).
\par 15 Vărsa-voi urgia Mea peste Sin, cetatea Egiptului și mulțimea oamenilor din No o voi pierde.
\par 16 Voi trimite foc asupra Egiptului; cutremura-se-va Sin și No se va prăbuși, iar asupra Nofului vor năvăli vrăjmașii în ziua cea mare.
\par 17 Tinerii din On și din Bubastis vor cădea de sabie, iar ceilalți se vor duce în robie.
\par 18 Și în Tahpanhes se va întuneca ziua, când voi zdrobi acolo jugul Egiptului și se va curma puterea cea mândră a lui. Un nor îl va acoperi și fiicele lui vor fi duse în robie.
\par 19 Așa voi face Eu judecată împotriva Egiptului și vor ști că Eu sunt Domnul".
\par 20 În anul al unsprezecelea, în luna întâi, în ziua a șaptea a lunii, a fost cuvântul Domnului către mine:
\par 21 "Fiul omului, Eu am și zdrobit un braț al lui Faraon, regele Egiptului, și iată, nimeni nu l-a legat ca să se vindece și nu l-a înfășurat cu legături ca să capete putere pentru a mânui sabia".
\par 22 De aceea, așa zice Domnul Dumnezeu: Iată, Eu sunt împotriva lui Faraon, regele Egiptului, și voi zdrobi brațele lui, pe cel sănătos și pe cel zdrobit, încât sabia va cădea din mâinile lui.
\par 23 Voi împrăștia pe Egipteni printre popoare și-i voi vântura prin țări.
\par 24 Iar brațele regelui Babilonului le voi întări și-i voi da sabia Mea în mână, iar brațele lui Faraon le voi zdrobi și el, rănit cumplit, va geme înaintea lui.
\par 25 Întări-voi brațele regelui Babilonului, iar brațele lui Faraon vor cădea fără putere; și vor ști că Eu sunt Domnul când voi da sabia Mea în mâinile regelui Babilonului, și acesta o va întinde asupra Egiptului.
\par 26 Voi împrăștia pe Egipteni printre neamuri; îi voi risipi prin țări și vor ști că Eu sunt Domnul".

\chapter{31}

\par 1 În anul al unsprezecelea, în ziua întâi a lunii a treia, a fost cuvântul Domnului către mine:
\par 2 "Fiul omului, spune lui Faraon, regele Egiptului, și poporului lui: Cu cine te asemeni tu în mărirea ta?
\par 3 Iată Asiria era un cedru în Liban, cu ramuri frumoase, cu frunziș umbros și cu trunchi înalt; vârful lui se ridicase până la nori.
\par 4 Apele îl făcuseră să crească, adâncul îl ridicase și râurile acestuia înconjurau locul unde fusese sădit și trimiteau apele lor la toți arborii câmpului.
\par 5 De aceea înălțimea lui întrecuse pe toți arborii câmpului și avea pe dânsul mulți lăstari și ramurile lui se înmulțiseră; lăstarii lui se făcuseră înalți, pentru că avuseseră apă multă la creșterea lor.
\par 6 în lăstarii lui își împletiseră cuiburi tot felul de păsări de-ale cerului; sub ramurile lui își scoteau puii tot felul de fiare de ale pământului, iar la umbra lui trăiau numeroase și felurite popoare.
\par 7 El era frumos prin înălțimea trunchiului său, și prin lungimea ramurilor sale, căci rădăcinile sale se aflau lângă niște ape mari.
\par 8 Cedrii din grădina lui Dumnezeu nu-l umbreau, chiparoșii nu se puteau asemăna cu crengile lui și castanii nu erau la mărime ca ramurile lui; nici un copac din grădina lui Dumnezeu nu se asemăna cu el în frumusețe.
\par 9 Eu îl împodobisem cu mulțimea ramurilor lui, încât toți arborii Edenului din grădina lui Dumnezeu îl pizmuiau.
\par 10 De aceea, așa a zis Domnul Dumnezeu: Pentru că s-a făcut înalt la statură și creștetul lui și l-a înălțat până la nori și inima lui s-a îngâmfat cu înălțimea lui,
\par 11 De aceea l-am dat în mâna regelui neamurilor, și acesta s-a purtat cu el după răutatea lui; pentru nelegiuirea lui l-am lepădat.
\par 12 Tăiatu-l-au străinii cei mai răi dintre popoare și l-au prăvălit peste munți; ramurile lui au căzut prin toate văile, iar lăstarii lui s-au frânt prin toate văgăunile pământului și de sub umbra lui au fugit toate popoarele pământului și l-au părăsit.
\par 13 Pe dărâmăturile lui s-au așezat toate păsările cerului și în lăstarii lui erau toate fiarele câmpului.
\par 14 Aceasta s-a făcut pentru ca nici unul dintre arborii de pe lângă ape să nu se îngâmfe cu statura sa înaltă și să nu-și înalțe vârful până la nori; ca toți stejarii ce se adapă din ape să nu mai vadă înălțimea lor, căci toți vor fi dați morții, în latura cea de dedesubt a pământului, împreună cu fiii oamenilor, care s-au coborât în mormânt.
\par 15 Așa zice Domnul Dumnezeu: "În ziua aceea, când s-a coborât el în locuința morților, în semn de jale, am închis peste el adâncul; am oprit râurile lui și apele cele mari au secat; am întunecat Libanul pentru el și toți arborii câmpului s-au uscat din cauza lui.
\par 16 La vuietul căderii lui am făcut să se cutremure neamurile, când l-am prăbușit în locuința morților, la cei ce se coborâseră în mormânt, și s-au bucurat în latura cea de dedesubt toți arborii Edenului, cei mai aleși și mai buni ai Libanului, toți cei adăpați cu apă;
\par 17 Și aceștia s-au coborât cu el în locuința morților la cei uciși de sabie, care erau brațul lui și trăiau în umbra lui, printre neamuri".
\par 18 Deci, cu care din arborii Edenului te-ai asemănat tu în strălucire și măreție? Acum însă la rând cu arborii Edenului vei fi prăbușit în adânc, vei. zăcea în mijlocul celor netăiați împrejur, cu cei uciși de sabie. Iată pe Faraon și toată mulțimea supușilor săi", zice Domnul.

\chapter{32}

\par 1 În anul al doisprezecelea de la robirea lui Ioiachim, în luna a douăsprezecea, în ziua întâi a acestei luni, a fost către mine cuvântul Domnului și mi-a zis:
\par 2 "Fiul omului, ridică plângere asupra lui Faraon, regele Egiptului, și spune-i: Leu al neamurilor, iată-te nimicit. Tu erai ca un crocodil în ape, tu suflai din nările tale, tu tulburai apele cu picioarele tale și întărâtai valurile lor.
\par 3 Așa zice Domnul Dumnezeu: Cu putere voi arunca asupra ta mreaja Mea în adunarea a multor popoare și ele te vor târî afară cu mreaja Mea.
\par 4 Te voi arunca pe uscat, în câmp deschis te voi arunca și se vor așeza pe tine toate păsările cerului și se vor sătura cu tine toate fiarele pământului.
\par 5 Voi împărți cărnurile tale pe munți și văile le voi umple cu stârvul tău.
\par 6 țara în care înoți tu o voi umple cu sângele tău până în munți și văgăunile lor vor fi umplute cu tine.
\par 7 Când te vei stinge, voi acoperi cerurile și stelele lor le voi întuneca; soarele îl voi acoperi cu nor și luna nu va mai lumina cu lumina sa.
\par 8 Toate stelele care luminează pe cer le voi întuneca deasupra ta și asupra țării tale voi aduce negură, zice Domnul Dumnezeu.
\par 9 Voi umple de tulburare inima multor popoare, când voi vesti căderea ta la popoarele de prin țările pe care tu nu le-ai cunoscut.
\par 10 Voi umple prin tine de groază multe popoare și regii lor se vor cutremura de frică prin tine, când voi flutura sabia Mea înaintea lor, și în orice clipă va tremura fiecare pentru sufletul său în ziua căderii tale.
\par 11 Căci așa zice Domnul Dumnezeu: Sabia regelui Babilonului va veni asupra ta;
\par 12 Cu sabia războinicilor voi face să cadă mulțimea supușilor tăi; aceștia sunt cei mai cruzi dintre popoare, vor zdrobi mândria Egiptului, și toată mulțimea poporului lui va pieri.
\par 13 Voi nimici toate viețuitoarele din apele tale cele mari și mai mult nu le va mai tulbura picior de om, nici copită de dobitoc nu le va tulbura.
\par 14 Atunci voi potoli apele; voi face să curgă ca untdelemnul fluviile lui, zice Domnul Dumnezeu.
\par 15 Când voi face din țara Egiptului un pustiu, când țara va fi jefuită de tot ce are și când voi lovi pe toți cei ce trăiesc în ea, atunci vor ști că Eu sunt Domnul.
\par 16 Iată cântarea de jale pe care o vor striga fiicele neamurilor. Ele o vor striga asupra Egiptului și asupra întregului său popor. Ele vor striga această cântare de jale", zice Domnul Dumnezeu.
\par 17 În anul al doisprezecelea, în ziua a cincisprezecea a lunii întâi, a fost cuvântul Domnului către mine:
\par 18 "Fiul omului, plângi pentru mulțimea poporului Egiptului și fă-o să coboare, ea și fiicele neamurilor strălucite, în adâncurile pământului, cu cei ce s-au coborât în mormânt.
\par 19 Pe cine întreci tu! Coboară-te și zaci cu cei netăiați împrejur!
\par 20 Aceia au căzut printre cei uciși de sabie; și el este dat sabiei; târâți-l pe el și toate mulțimile lui!
\par 21 În mijlocul locuinței morților se va vorbi de el și de cei ce-l sprijineau, cei mai vestiți dintre eroi; aceia au căzut și zac printre cei netăiați împrejur, răpuși de sabie.
\par 22 Acolo este Asiria și oamenii ei de război, împrejurul mormântului ei, toți uciși, căzuți sub sabie.
\par 23 Mormintele lor sunt așezate chiar în fundul adâncului și armata ei împrejurul mormântului ei; toți uciși, căzuți sub sabie, ei care împrăștiau groaza în pământul celor vii.
\par 24 Acolo este Elam cu toată armata lui împrejurul mormântului său; toți uciși, căzuți sub sabie și netăiați împrejur s-au coborât în adâncuri, ei care împrăștiaseră groază pe pământul celor vii, iar acum își poartă rușinea cu cei ce s-au coborât în mormânt.
\par 25 În mijlocul ucișilor l-am culcat împreună cu toată mulțimea oamenilor lui și mormintele lor sunt împrejurul lui; toți acești netăiați împrejur sunt uciși de sabie; și după cum au împrăștiat groază pe pământul celor vii, așa își poartă rușinea lor alături de cei ce s-au coborât în mormânt și sunt așezați printre cei uciși.
\par 26 Acolo sunt Meșec și Tubal cu toată mulțimea lor de popor și mormintele lor sunt împrejurul lor; toți acești netăiați împrejur au murit de sabie, pentru că au împrăștiat groază pe pământul celor vii.
\par 27 Nu trebuia oare să zacă și ei printre vitejii căzuți dintre cei netăiați împrejur, care cu armele lor de război s-au coborât în locuința morților și săbiile lor și le-au pus sub cap și fărădelegea lor a rămas pe oasele lor, pentru că ei, ca niște puternici, au fost o spaimă pe pământul celor vii?
\par 28 Și tu, Faraoane, vei fi zdrobit în mijlocul celor netăiați împrejur și vei zăcea împreună cu cei uciși de sabie.
\par 29 Acolo sunt Edom și regii lui și toate căpeteniile lui, care cu toată vitejia lor au fost așezați printre cei căzuți de sabie și zac cu cei netăiați împrejur, care s-au coborât în mormânt.
\par 30 Acolo sunt stăpânitorii de la miazănoapte și toți Sidonienii, care s-au coborât acolo cu cei uciși și, fiind rușinați în puternicia lor cu care au împrăștiat groază, zac cu cei netăiați împrejur, care au fost uciși cu sabia și își poartă rușinea cu cei coborâți în mormânt.
\par 31 Faraon îi va vedea și se va mângâia la vederea acestei întregi mulțimi, ucisă de sabie Faraon și toată armata lui, zice Domnul.
\par 32 Căci voi împrăștia frica Mea peste pământul celor vii și Faraon cu toată mulțimea lui va fi pus printre cei netăiați împrejur cu cei căzuți de sabie, zice Domnul Dumnezeu.

\chapter{33}

\par 1 Fost-a cuvântul Domnului către mine și mi-a zis:
\par 2 "Fiul omului, rostește cuvânt către fiii poporului tău și le spune: De voi aduce sabie asupra unei țări și poporul țării aceleia va lua din mijlocul său un om și îl va pune străjer,
\par 3 Și el, văzând sabia venind împotriva țării, va trâmbița din trâmbiță și va vesti poporul;
\par 4 De va auzi cineva sunetul trâmbiței, dar nu se va păzi, când va veni sabia și-l va prinde, sângele aceluia va fi asupra capului său.
\par 5 Pentru că a auzit glasul trâmbiței și nu s-a păzit, sângele lui va fi asupra lui; iar cel ce se va păzi își va scăpa viața sa.
\par 6 Dacă însă străjerul a văzut sabia venind și nu a sunat din trâmbiță și poporul n-a fost vestit și va veni sabia și va ridica viața cuiva, acela a fost răpit pentru păcatele lui, dar sângele lui îl voi cere din mâna străjerului.
\par 7 Și pe tine, fiul omului, te-am pus Eu străjer casei lui Israel și tu vei auzi cuvânt din gura Mea și îl vei vesti din partea Mea.
\par 8 Când Eu voi zice păcătosului: "Păcătosule, vei muri", și tu nu-i vei grăi nimic, ca să vestești pe păcătos să se abată de la calea lui, atunci păcătosul acela va muri pentru păcatele sale, iar sângele lui îl voi cere din mâna ta.
\par 9 Iar dacă tu ai vestit pe păcătos să se abată de la calea lui și să se întoarcă de la ea, și el nu s-a abătut de la calea lui, atunci el va muri pentru păcatele lui, iar tu ți-ai scăpat viața.
\par 10 Și tu, fiul omului, spune casei lui Israel: Voi ziceți așa: "Nelegiuirile noastre și păcatele noastre sunt asupra noastră și ne stingem în ele; cum vom putea dar să trăim?"
\par 11 Spune-le: Precum este adevărat că Eu sunt viu, tot așa este de adevărat că Eu nu voiesc moartea. păcătosului, ci ca păcătosul să se întoarcă de la calea sa și să fie viu. Întoarceți-vă, întoarceți-vă de la căile voastre cele rele! Pentru ce să muriți voi, casa lui Israel?
\par 12 Și tu, fiul omului, spune fiilor poporului tău: Dreptatea dreptului nu-l va scăpa în ziua păcătuirii lui și nelegiuitul nu va cădea pentru nelegiuirea sa în ziua întoarcerii sale de la nelegiuirea sa, precum nici dreptul în ziua păcătuirii sale nu va putea rămâne cu viață pentru dreptatea sa.
\par 13 Când voi zice dreptului că va fi viu, iar el se va încrede în dreptatea sa și va face nedreptate, atunci nu se va mai pomeni toată dreptatea lui, ci el va muri pentru tot răul pe care l-a făcut.
\par 14 Și când voi zice păcătosului: "Vei muri", dar el se va întoarce de la păcatele sale și va face judecată și dreptate,
\par 15 Dacă acest păcătos va înapoia zălogul, pentru cele răpite va despăgubi, va umbla după legile vieții, nefăcând nimic rău, atunci el va fi viu și nu va muri.
\par 16 Nici unul din păcatele sale, pe care le-a făcut, nu i se vor pomeni și, pentru că a început a face dreptate și judecată, va fi viu.
\par 17 Fiii poporului tău zic: "Calea Domnului nu este dreaptă!" Dar nedreaptă este calea lor.
\par 18 Dacă dreptul se va abate de la dreptatea sa și va începe să facă nelegiuire, va muri pentru aceasta.
\par 19 De asemenea, dacă nelegiuitul s-a întors de la nelegiuirea sa și a început să facă judecată și dreptate, pentru aceasta el va trăi.
\par 20 Voi însă ziceți: "Calea Domnului este nedreaptă". Eu vă voi judeca pe voi, casa lui Israel, și voi judeca pe fiecare după purtările lui".
\par 21 În anul al doisprezecelea după robirea noastră, în ziua a cincea a lunii a zecea, a venit la mine unul din cei scăpați din Ierusalim și mi-a spus: "Cetatea este dărâmată!"
\par 22 Mâna Domnului a fost peste mine, seara, încă înainte de a veni acest fugar; iar dimineața, când a venit acesta la mine, Domnul îmi deschisese gura și nu mai eram mut, ei mi se deschisese gura.
\par 23 Și a fost cuvântul Domnului către mine și mi-a zis:
\par 24 Fiul omului, cei ce trăiesc în locurile pustiite din țara lui Israel zic: "Avraam a fost unul și a primit în stăpânire țara aceasta, iar noi suntem mulți; deci nouă ne este dată în stăpânire țara aceasta.
\par 25 De aceea spune-le: "Așa grăiește Domnul Dumnezeu: Voi mâncați mâncare cu sânge; vă ridicați ochii spre idolii voștri și vărsați sânge și apoi voiți să stăpâniți țara?
\par 26 Voi vă rezemați pe sabia voastră, faceți ticăloșii, vă pângăriți femeile unii altora și apoi voiți să stăpâniți țara?
\par 27 Iată ce să le spui: Așa grăiește Domnul Dumnezeu: Precum este adevărat că Eu sunt viu, tot așa de adevărat este că cei ce locuiesc în locurile pustiite vor cădea de sabie; iar cel ce se află în câmp, pe acela îl voi da fiarelor spre mâncare, iar cei din cetăți și din peșteri vor muri de ciumă.
\par 28 Și voi face din țară un pustiu și o singurătate, trufia puterii ei va înceta, și munții lui Israel se vor pustii, încât nici un trecător nu va mai trece prin ei.
\par 29 Și vor cunoaște că Eu sunt Domnul, când voi face țara pustietatea pustietăților pentru toate ticăloșiile pe care le-au făcut ei.
\par 30 Iar despre tine, fiul omului, fiii poporului tău grăiesc pe la ziduri și pe la ușile caselor și zice unul către altul și frate către frate: "Mergeți de vedeți ce cuvânt a ieșit de la Domnul!"
\par 31 Și ei vin la tine, ca la o adunare de petrecere; poporul Meu se așază înaintea ta și ascultă cuvintele tale. Dar nu le împlinește; căci ei cu gura lor fac din acestea o petrecere, iar inima lor e târâtă după poftele lor.
\par 32 Iată că tu ești pentru ei un cântăreț plăcut, cu glas frumos și care cântă bine din instrumentul său; ei ascultă cuvintele tale, dar nimeni nu le împlinește,
\par 33 Iar când aceste lucruri vor veni, și iată că ele vin, atunci vor ști că în mijlocul lor era un prooroc".

\chapter{34}

\par 1 Fost-a cuvântul Domnului către mine și mi-a zis:
\par 2 "Fiul omului, proorocește împotriva păstorilor lui Israel, proorocește și le spune: Așa grăiește Domnul Dumnezeu: Vai de păstorii lui Israel, care s-au păstorit pe ei înșiși! Păstorii nu trebuia ei oare să păstorească turma?
\par 3 Dar voi ați mâncat grăsimea și cu lâna v-ați îmbrăcat; oile cele grase le-ați junghiat, iar turma n-ați păscut-o.
\par 4 Pe cele slabe nu le-ați întărit; oaia bolnavă n-ați lecuit-o și pe cea rănită n-ați legat-o; pe cea rătăcită n-ați întors-o și pe cea pierdută n-ați căutat-o, ci le-ați stăpânit cu asprime și cruzime.
\par 5 Și ele, neavând păstor, s-au risipit și, risipindu-se, au ajuns mâncarea tuturor fiarelor câmpului.
\par 6 De aceea rătăcesc oile Mele prin toți munții și pe tot dealul înalt; împrăștiatu-s-au oile Mele peste toată fața pământului și nimeni nu îngrijește de ele și nimeni nu le caută.
\par 7 De aceea, ascultați păstori, cuvântul Domnului:
\par 8 Precum este adevărat că Eu sunt viu, zice Domnul Dumnezeu, tot așa este de adevărat că voi face dreptate; pentru că oile Mele au fost lăsate pradă și fără păstor, oile Mele au ajuns mâncarea tuturor fiarelor câmpului, iar păstorii Mei n-au purtat grijă de oile Mele, ci păstorii s-au păscut pe ei înșiși și oile Mele nu le-au păscut.
\par 9 De aceea ascultați, păstori, cuvântul Domnului.
\par 10 Asa zice Domnul Dumnezeu: Iată Eu vin la păstori; le voi cere înapoi oile Mele din mâna lor și îi voi împiedica să nu mai pască oile Mele și nu se vor mai paște păstorii pe ei înșiși și voi smulge oile Mele din gura lor și ele nu vor mai fi pentru ei o pradă de sfâșiat.
\par 11 Căci așa zice Domnul Dumnezeu: Iată Eu Însumi voi purta grijă de oile Mele și le voi cerceta.
\par 12 Cum cercetează păstorul turma sa în ziua când se află în mijlocul turmei sale risipite, așa voi cerceta și Eu oile Mele și le voi aduna din toate locurile, unde au fost ele risipite în ziua cea cețoasă și întunecată.
\par 13 Le voi face să iasă din mijlocul popoarelor, le voi aduna din diferite țări și le voi aduce în țara lor și le voi paște prin munții lui Israel, pe lângă cursurile de apă și prin toate locurile de locuit ale țării acesteia.
\par 14 Le voi paște în pășune bună și staulul va fi pe munții cei înalți ai lui Israel; acolo se vor odihni ele, în staul bun și vor paște în pășune grasă în munții lui Israel.
\par 15 Eu voi paște oile Mele și Eu le voi odihni, zice Domnul Dumnezeu.
\par 16 Oaia pierdută și rătăcită o voi întoarce la staul, pe cea rănită o voi lega și pe cea bolnavă o voi întări, iar pe cea grasă și tare o voi păzi și voi păstori cu dreptate.
\par 17 Iar despre voi, oile Mele, așa zice Domnul Dumnezeu: Iată voi face judecată între oaie și oaie, între berbec și țap.
\par 18 Oare nu vă ajunge că pașteți în pășune bună, iar ce rămâne călcați cu picioarele voastre și că beți apă curată, iar pe cea care rămâne o tulburați cu picioarele voastre,
\par 19 Asa că oile Mele sunt nevoite să se hrănească cu ceea ce este călcat de picioarele voastre și să bea ceea ce este tulburat de picioarele voastre?"
\par 20 De aceea așa le zice Domnul Dumnezeu: "Iată Eu Însumi voi face judecată între oaia grasă și oaia slabă.
\par 21 Deoarece voi izbiți cu umărul, cu șoldul și cu coarnele voastre, împungeți pe toate oile bolnăvicioase, până când le scoateți afară,
\par 22 Eu voi veni să scap oile Mele, ca să nu mai fie pradă și voi judeca între oaie și oaie.
\par 23 Voi pune peste ele un singur păstor, care le va paște; voi pune pe robul Meu David; el le va paște și el va fi păstorul lor.
\par 24 Iar Eu, Domnul, le voi fi Dumnezeu, iar robul Meu David va fi prinț între ei. Eu, Domnul, am grăit acestea.
\par 25 Voi încheia cu acela legământul păcii și voi depărta din țară fiarele sălbatice, încât oile Mele să trăiască în siguranță în pustiu și să doarmă în pădure.
\par 26 Voi dărui lor și împrejurimilor muntelui Meu binecuvântare și ploaie le voi trimite la vreme; ploi de binecuvântare vor fi acestea.
\par 27 Pomul din câmp își va da rodul său și pământul își va da roadele sale și oile Mele vor fi în siguranță pe pământul lor și vor ști că Eu sunt Domnul, când voi sfărâma cătușele jugului lor și le voi scăpa din mâinile celor ce le-au robit.
\par 28 Nu vor mai fi ele pradă popoarelor și fiarele câmpului nu le vor mai sfâșia; ele vor trăi în siguranță și nimeni nu le va mai tulbura.
\par 29 Voi face acolo sădire vestită și nu vor mai pieri de foame pe pământ, nici nu vor mai suferi ocară de la popoare.
\par 30 Și vor ști că Eu, Domnul Dumnezeul lor, sunt cu ele, iar ele, casa lui Israel, sunt poporul Meu, zice Domnul Dumnezeu.
\par 31 Și voi, oile Mele, sunteți turma pe care o pasc, iar Eu sunt Dumnezeul vostru, zice Domnul Dumnezeu.

\chapter{35}

\par 1 Fost-a cuvântul Domnului către mine și mi-a zis:
\par 2 "Fiul omului, întoarce-ți fața spre muntele Seir și proorocește împotriva lui,
\par 3 Și spune-i: Așa grăiește Domnul Dumnezeu: Iată Eu sunt împotriva ta, munte Seir, și-Mi voi întinde mâna împotriva ta și te voi face pustiu și nelocuit.
\par 4 Cetățile tale le voi preface în ruine și tu însuți vei fi pustiit și vei ști că Eu sunt Domnul;
\par 5 Fiindcă ai dușmănie veșnică și ai dat pe fiii lui Israel în mâna sabiei în timpul necredinței lor, în vremea pieirii desăvârșite,
\par 6 De aceea, precum este adevărat că Eu sunt viu, zice Domnul Dumnezeu, tot așa este de adevărat că te voi umple de sânge și sângele te va urmări; și pentru că tu n-ai urât vărsarea de sânge, de aceea sângele te va și urmări.
\par 7 Voi face muntele Seir o singurătate și un pustiu; voi nimici pe oricine străbate țara.
\par 8 Și voi umple înălțimile lui de ucișii lui. Pe dealurile tale, în văile tale și în toate văgăunile tale vor cădea răpuși de sabie.
\par 9 Te voi face pustiu veșnic și în cetățile tale nu vor mai trăi oameni, și veți ști că Eu sunt Domnul.
\par 10 De vreme ce tu ai zis: "Aceste două popoare și aceste două țări vor fi ale mele și le voi stăpâni", cu toate că Domnul era acolo,
\par 11 De aceea precum este adevărat că Eu sunt viu, zice Domnul Dumnezeu, tot așa este de adevărat că Mă voi purta cu tine după măsura urii tale și a pizmei tale, pe care le-ai arătat către ele, și Mă voi face cunoscut lor, când te voi judeca.
\par 12 Atunci vei ști că Eu, Domnul, am auzit toate hulele tale pe care le-ai rostit împotriva munților lui Israel, zicând: "S-au pustiit și ne sunt dați nouă spre mâncare!"
\par 13 Auzit-am că v-ați lăudat înaintea Mea cu limba voastră și ați înmulțit vorbele voastre împotriva Mea.
\par 14 Așa zice Domnul Dumnezeu: Când tot pământul se va bucura, pe tine te voi face pustiu.
\par 15 Cum te-ai bucurat tu, că partea casei lui Israel s-a pustiit, așa voi face și cu tine: pustiit vei fi, munte Seir, și împreună cu tine și tot Edomul și vor ști că Eu sunt Domnul".

\chapter{36}

\par 1 "Și tu, fiul omului, proorocește asupra munților lui Israel și spune: "Munții lui Israel, ascultați cuvântul Domnului.
\par 2 Așa zice Domnul Dumnezeu: Deoarece vrăjmașul grăiește despre voi și zice: "Aha, și înălțimile cele veșnice au ajuns moștenirea noastră",
\par 3 De aceea proorocește și spune: Așa grăiește Domnul Dumnezeu: Pentru că vă pustiesc, și anume pentru că vă pustiesc și vă înghit din toate părțile, ca să ajungeți moștenirea celorlalte popoare și ați ajuns clevetirea și ocara oamenilor,
\par 4 De aceea, munți ai lui Israel, ascultați cuvântul Domnului Dumnezeu: Așa grăiește Domnul Dumnezeu către munți și dealuri., către văi și vâlcele, către ruinele pustii și către cetățile părăsite, care au ajuns pradă și ocară celorlalte popoare de primprejur;
\par 5 De aceea, așa zice Domnul Dumnezeu: În focul zelului Meu am rostit cuvânt împotriva celorlalte popoare și împotriva întregului Edom, care au socotit țara Mea ca moștenire a lor, și s-au bucurat din toată inima lor și cu tot disprețul sufletului lor, ca să-i jefuiască roadele.
\par 6 De aceea rostește proorocie asupra țării lui Israel și spune munților și dealurilor, văilor și vâlcelelor: Așa zice Domnul Dumnezeu: Iată Eu am rostit aceasta în zelul urgiei Mele, pentru că voi purtați asupra voastră ocara neamurilor.
\par 7 De aceea, așa zice Domnul Dumnezeu: Ridicatu-Mi-am mâna cu jurământ, că popoarele care sunt împrejurul vostru vor purta ele singure rușinea lor.
\par 8 Iar voi, munții lui Israel, veți întinde ramurile voastre și veți aduce roadele voastre poporului Meu Israel; că se apropie venirea lui.
\par 9 Căci iată Eu Mă întorc spre voi și veți fi lucrați și semănați.
\par 10 Și voi așeza pe voi mulțime de oameni, toată casa lui Israel. Cetățile vor fi locuite și ruinele zidite din nou.
\par 11 Voi înmulți la voi oamenii și dobitoacele; se vor prăsi acestea și se vor înmulți și vă voi face să fiți locuiți, ca și mai înainte și vă voi face bine mai mult decât altă dată și veți ști că Eu sunt Domnul.
\par 12 Voi aduce pe voi oameni, pe poporul Meu Israel și ei te vor stăpâni pe tine, țară, și tu vei fi moștenirea lor și nu-i vei mai lipsi de copiii lor".
\par 13 Așa zice Domnul Dumnezeu: "Pentru că se zice despre voi: "Tu ești o rară care mănânci oameni și lipsești neamul tău de copiii săi",
\par 14 De aceea nu vei mai mânca pe oameni și pe poporul tău nu-l vei mai lipsi de copiii săi, zice Domnul Dumnezeu.
\par 15 Și nu vei mai auzi batjocuri de la popoare și hulă de la neamuri nu vei mai purta pe obrazul tău; pe poporul tău de acum înainte nu-l vei mai lipsi de capii", zice Domnul Dumnezeu.
\par 16 Fost-a cuvântul Domnului către mine și mi-a zis:
\par 17 "Fiul omului, când casa lui Israel trăia în țara sa, au pângărit-o cu purtarea lor și cu faptele lor; căile lor erau înaintea feței Mele ca necurățenia femeii în timpul regulei ei;
\par 18 Eu am vărsat asupra lor mânia Mea pentru sângele pe care l-au vărsat în țară și pentru că au întinat-o cu idolii lor.
\par 19 I-am risipit printre neamuri și au fost împrăștiați prin țările străine; după purtările lor și după faptele lor i-am judecat.
\par 20 Și au mers la neamurile la care s-au dus și au necinstit numele Meu cel sfânt, încât se zicea despre ei: "Acesta este poporul Domnului, care a ieșit din țara sa".
\par 21 Am luat aminte la numele Meu cel sfânt, pe care l-a necinstit casa lui Israel printre popoarele la care s-a dus.
\par 22 Și de aceea spune casei lui Israel: Așa grăiește Domnul Dumnezeu: Aceasta o fac nu pentru voi, casa lui Israel, ci pentru numele Meu cel sfânt pe care l-ați necinstit voi printre neamurile la care ați mers.
\par 23 Voi sfinți numele Meu cel mare care a fost necinstit la neamurile printre care l-ați necinstit voi, și vor ști neamurile că Eu sunt Domnul, zice Domnul Dumnezeu, când Mă voi sfinți în voi, înaintea ochilor lor.
\par 24 De aceea vă voi scoate dintre neamuri și din toate țările vă voi aduna și vă voi aduce în pământul vostru.
\par 25 Și vă voi stropi cu apă curată și vă veți curăți de toate întinăciunile voastre și de toți idolii voștri vă voi curăți.
\par 26 Vă voi da inimă nouă și duh nou vă voi da; voi lua din trupul vostru inima cea de piatră și vă voi da inimă de carne.
\par 27 Pune-voi înăuntrul vostru Duhul Meu și voi face ca să umblați după legile Mele și să păziți și să urmați rânduielile Mele.
\par 28 Veți locui în țara pe care am dat-o părinților voștri și veți fi poporul Meu și Eu voi fi Dumnezeul vostru.
\par 29 Vă voi scăpa de toate necurățiile voastre și voi chema pâinea și o voi înmulți și nu vă voi lăsa să suferiți de foame.
\par 30 Voi înmulți fructele în pom și roadele în câmp, ca să nu mai suferiți de acum înainte ocara neamurilor din pricina foametei.
\par 31 Atunci vă veți aduce aminte de purtările voastre cele rele și de faptele voastre care n-au fost bune și vă veți scârbi de voi înșivă pentru nelegiuirile voastre și pentru ticăloșiile voastre.
\par 32 Știut să vă fie, că nu pentru voi, zice Domnul Dumnezeu, voi face aceasta. Roșiți și vă rușinați de căile voastre, casa lui Israel!"
\par 33 Așa zice Domnul Dumnezeu: "În ziua aceea, când vă voi curăți de toate fărădelegile voastre și voi face să fie cetățile locuite, când așezările dărâmate vor fi iarăși zidite,
\par 34 Și pământul cel pustiit, care în ochii oricărui trecător era o pustietate, va fi lucrat,
\par 35 Atunci se va zice: "Acest pământ, altă dată pustiit, s-a făcut ca grădina Edenului și cetățile acestea pustiite și dărâmate sunt iarăși întărituri locuite".
\par 36 Și neamurile care vor rămâne împrejurul vostru, vor ști că Eu, Domnul, zidesc din nou cele ruinate și sădesc cele pustiite. Eu, Domnul, am zis și fac".
\par 37 Așa grăiește Domnul Dumnezeu: "Iată încă și pentru aceasta voi lăsa casa lui Israel să Mă caute; îi voi înmulți pe oamenii săi ca pe o turmă.
\par 38 Cum sunt de multe oile de jertfă în Ierusalim, în timpul sărbătorilor, așa vor fi pline de oameni cetățile pustiite, și vor ști că Eu sunt Domnul".

\chapter{37}

\par 1 Fost-a mâna Domnului peste mine și m-a dus Domnul cu Duhul și m-a așezat în mijlocul unui câmp plin de oase omenești,
\par 2 Și m-a purtat împrejurul lor; dar iată oasele acestea erau foarte multe pe fața pământului și uscate de tot.
\par 3 Și mi-a zis Domnul: "Fiul omului, vor învia, oasele acestea?" Iar eu am zis: "Dumnezeule, numai Tu știi aceasta".
\par 4 Domnul însă mi-a zis: "Proorocește asupra oaselor acestora și le spune: Oase uscate, ascultați cuvântul Domnului!
\par 5 Așa grăiește Domnul Dumnezeu oaselor acestora: Iată Eu voi face să intre în voi duh și veți învia.
\par 6 Voi pune pe voi vine și carne va crește pe voi; vă voi acoperi cu piele, voi face să intre în voi duh și veți învia și veți ști că Eu sunt Domnul".
\par 7 Proorocit-am deci cum mi se poruncise. și când am proorocit, iată s-a făcut un vuiet și o mișcare și oasele au început să se apropie, fiecare os la încheietura sa.
\par 8 Și am privit și eu și iată erau pe ele vine și crescuse carne și pielea le acoperea pe deasupra, iar duh nu era în ele.
\par 9 Atunci mi-a zis Domnul: "Fiul omului, proorocește duhului, proorocește și spune duhului: Așa grăiește Domnul Dumnezeu: Duhule, vino din cele patru vânturi și suflă peste morții aceștia și vor învia!"
\par 10 Deci am proorocit eu, cum mi se poruncise, și a intrat în ei duhul și au, înviat și mulțime multă foarte de oameni s-au ridicat pe picioarele lor.
\par 11 Și mi-a zis iarăși Domnul: "Fiul omului, oasele acestea sunt toată casa lui Israel. Iată ei zic: "S-au uscat oasele noastre și nădejdea noastră a pierit; suntem smulși din rădăcină".
\par 12 De aceea proorocește și le spune: Așa grăiește Domnul Dumnezeu: Iată, Eu voi deschide mormintele voastre și vă voi scoate pe voi, poporul Meu, din mormintele voastre și vă voi duce în țara lui Israel.
\par 13 Astfel veți ști că Eu sunt Domnul, când voi deschide mormintele voastre și vă voi scoate pe voi, poporul Meu, din mormintele voastre.
\par 14 Și voi pune în voi Duhul Meu și veți învia și vă voi așeza în țara voastră și veți ști că Eu, Domnul, am zis aceasta și am făcut", zice Domnul.
\par 15 Fost-a iarăși către mine cuvântul Domnului și mi-a zis:
\par 16 "Iar tu, fiul omului, ia-ți un toiag și scrie pe el: "Lui Iuda și fiilor lui Israel, care sunt uniți cu el". și să mai iei un toiag și să scrii pe el: "Lui Iosif". Acesta este toiagul lui Efraim și a toată casa lui Israel, care este unită cu el.
\par 17 Apoi să le apropii unul de altul încât ele să fie în mâna ta ca un singur toiag.
\par 18 Iar când te vor întreba fiii poporului tău: "Nu ne vei tălmăci oare și nouă ce înseamnă ceea ce ai în mână?
\par 19 Tu să le spui: Așa grăiește Domnul Dumnezeu: Iată Eu voi lua toiagul lui Iosif, care este în mâna lui Efraim și a semințiilor lui Israel unite cu el și le voi împreuna cu toiagul lui Iuda și voi face din ele un singur toiag și vor fi în mâna Mea una.
\par 20 Când însă amândouă toiegele pe care vei scrie vor fi în mâna ta înaintea ochilor lor,
\par 21 Atunci să le spui: Așa grăiește Domnul Dumnezeu: Iată, Eu voi lua pe fiii lui Israel din mijlocul neamurilor, printre care se află, îi voi aduna din toate părțile și-i voi aduce în țara lor;
\par 22 Iar în țara aceasta, pe munții lui Israel, îi voi face un singur neam și un singur rege va fi peste toți; nu vor mai fi două neamuri și în viitor nu se vor mai împărți în două regate;
\par 23 Nu se vor mai pângări cu idolii lor, cu urâciunile lor și cu toate păcatele lor. și voi izbăvi de toate fărădelegile pe care le-au săvârșit, îi voi curăți și vor fi poporul Meu, iar Eu voi fi Dumnezeul lor.
\par 24 Iar robul Meu David va fi rege peste ei și păstorul lor al tuturor, și ei se vor purta după cum cer poruncile Mele și legile Mele le vor păzi și le vor împlini.
\par 25 Vor locui țara pe care am dat-o Eu robului Meu Iacov, unde au trăit părinții lor; acolo vor locui ei și copiii lor în veci; iar robul Meu David va fi peste ei rege în veac.
\par 26 Voi încheia cu ei un legământ al păcii, legământ veșnic voi avea cu ei. Voi pune rânduială la ei, îi voi înmulți și voi așeza în mijlocul lor locașul Meu pe veci.
\par 27 Fi-va locașul Meu la ei și voi fi Dumnezeul lor, iar ei vor fi poporul Meu.
\par 28 Atunci vor ști popoarele că Eu sunt Domnul Care sfințește pe Israel, când locașul Meu va fi veșnic în mijlocul lor".

\chapter{38}

\par 1 Fost-a cuvântul Domnului către mine și mi-a zis:
\par 2 "Fiul omului, întoarce-îi fața spre Gog din țara lui Magog, regele lui Roș, al lui Meșec și al lui Tubal; proorocește împotriva lor,
\par 3 Și spune: Așa grăiește Domnul Dumnezeu: Iată, Eu sunt împotriva ta, Gog, rege al lui Roș și al lui Meșec și al lui Tubal!
\par 4 Te voi prinde, voi pune zăbale în fălcile tale și te voi scoate pe tine și toată oștirea ta, caii și toți călăreții strălucit îmbrăcați, ceată mare cu platoșe și cu scuturi, toți înarmați cu săbii;
\par 5 Și cu ei voi scoate pe Perși, pe Etiopieni și pe Libieni, toți cu scuturi și coifuri;
\par 6 Și pe Gomer cu toate oștirile lui; casa lui Togarma din hotarele de la miazănoapte, cu toate oștirile lui și voi mai scoate și alte multe popoare cu tine.
\par 7 Gătește-te și fii gata, tu ai toată mulțimea ta strânsă împrejurul tău, și fii căpetenia lor.
\par 8 După zile multe tu vei primi porunci. În anii de pe urmă vei veni în țara izbăvită de sabie, ai cărei locuitori au fost adunați dintr-o mulțime de popoare, în munții lui Israel, care au fost mult timp pustiiți. De când au fost despărțiți de celelalte popoare, ei locuiesc toți în siguranță.
\par 9 Și tu te vei ridica, cum se ridică furtuna și te vei duce ca norul, ca să acoperi țara, tu și toată oastea ta și multe popoare împreună cu tine.
\par 10 Așa zice Domnul Dumnezeu: n ziua aceea îți vor veni gânduri în mintea ta și vei face planuri rele,
\par 11 și vei zice: "Mă voi ridica împotriva unei țări fără apărare, voi merge împotriva oamenilor pașnici care trăiesc în siguranță, căci aceia toți trăiesc în cetăți fără ziduri și n-au nici porți, nici zăvoare,
\par 12 Ca să fac jaf și să iau pradă, punând mâna pe ruinele locuite din nou și pe poporul cel adunat din mijlocul neamurilor, care crește turme și strânge averi și care locuiește în mijlocul pământului".
\par 13 Șeba, Dedan și negustorii Tarsisului cu toți puii de lei ai lor vor zice: "Ai venit tu oare ca să faci jaf, ai adunat taberile tale, ca să faci pradă, să iei argint și aur, să ridici dobitoace și avere și să apuci pradă mare?"
\par 14 De aceea, rostește proorocie, fiul omului, și spune lui Gog: Așa grăiește Domnul Dumnezeu: Nu este așa oare că în ziua când poporul Meu Israel va trăi în siguranță, tu vei porni la drum?
\par 15 Și vei pleca de la locul tău, din hotarele de la miazănoapte, tu și multe popoare împreună cu tine, toți călări pe cai, tabără mare și oștire nenumărată?
\par 16 Și te vei ridica împotriva poporului Meu, împotriva lui Israel, ca un nor care acoperă pământul; aceasta va fi în zilele cele de pe urmă când te voi aduce împotriva țării Mele, ca popoarele să Mă cunoască pe Mine, când Eu voi fi arătat sfințenia Mea înaintea ochilor lor, asupra ta, o, Gog!
\par 17 Așa zice Domnul Dumnezeu: "Nu ești tu, oare, același despre care am grăit Eu în zilele cele de demult prin robii Mei, proorocii lui Israel, care au proorocit în vremurile acelea că te voi aduce împotriva lor?
\par 18 Și în ziua aceea, când Gog va veni împotriva țării lui Israel, zice Domnul Dumnezeu, mânia Mea se va aprinde pe fața Mea.
\par 19 Și în zelul Meu, în văpaia urgiei Mele am zis: Cu adevărat în ziua aceea va fi un mare cutremur în țara lui Israel.
\par 20 Atunci vor tremura înaintea Mea peștii mării și păsările cerului, fiarele câmpului și toate târâtoarele care se târăsc pe pământ și toți oamenii care sunt pe fața pământului; și se vor prăbuși munții, stâncile se vor dărâma și toate zidurile vor cădea la pământ.
\par 21 Prin toți munții Mei voi chema sabia împotriva lui, zice Domnul Dumnezeu; sabia fiecărui om va fi împotriva fratelui său.
\par 22 Și îl voi pedepsi cu ciumă și vărsare de sânge; voi vărsa asupra lui și a taberilor lui și asupra multor popoare care sunt cu el, ploaie potopitoare și grindină de piatră, foc și pucioasă;
\par 23 Voi arăta slava Mea și sfințenia Mea și Mă voi arăta înaintea ochilor multor popoare și vor ști că Eu sunt Domnul".

\chapter{39}

\par 1 "Iar tu, fiul omului, rostește proorocie împotriva lui Gog și spune: Așa grăiește Domnul Dumnezeu: Iată, Eu sunt împotriva ta, Gog, prințul lui Roș, al lui Meșec și al lui `Tubal!
\par 2 Te voi ademeni și te voi trage, te voi scoate din hotarele de la miazănoapte și te voi aduce în munții lui Israel.
\par 3 Acolo voi scoate aurul tău din mâna stângă a ta și voi arunca săgețile tale din mâna dreaptă a ta.
\par 4 Cădea-vei în munții lui Israel, tu și toate oștile tale și popoarele cele ce sunt cu tine; și te voi da spre mâncare la tot felul de păsări de pradă și fiarelor câmpului.
\par 5 Cădea-vei în câmp deschis, căci Eu am spus acestea", zice Domnul Dumnezeu.
\par 6 "Și voi trimite foc în pământul lui Magog și asupra locuitorilor insulelor, care trăiesc fără grijă și vor ști că Eu sunt Domnul.
\par 7 Voi arăta numele Meu cel sfânt poporului Meu Israel și nu voi mai lăsa de acum să se necinstească sfânt numele Meu și vor ști neamurile că Eu sunt Domnul cel sfânt în Israel.
\par 8 Iată, aceasta va veni și se va împlini, zice Domnul Dumnezeu; aceasta este ziua aceea de care am grăit Eu.
\par 9 Atunci locuitorii cetăților lui Israel vor ieși și vor aprinde foc, vor arde armele, scuturile, pavezele, arcurile, săgețile; lăncile și sulițele; șapte ani le vor arde.
\par 10 Nu vor aduce lemne din câmp, nici nu var tăia din pădure, ci vor arde numai arme; vor jefui pe jefuitorii lor și vor prăda pe prădătorii for, zice Domnul Dumnezeu.
\par 11 În ziua aceea voi da lui Gog loc de mormânt, în Israel, valea trecătorilor, la răsărit de Marea Moartă și mormântul acela va împiedica pe trecători; acolo vor îngropa pe Gog și toată tabăra lui și vor numi valea aceea Valea taberei lui Gog.
\par 12 Șapte luni îi va îngropa casa lui Israel, ca să curețe țara.
\par 13 Tot poporul țării îi va îngropa și va fi vestită la ei ziua în care Mă voi preaslăvi, zice Domnul Dumnezeu;
\par 14 Apoi se vor ridica oameni care să cutreiere necontenit țara și cu ajutorul trecătorilor să îngroape pe cei ce au rămas pe fața pământului, ca să curețe țara; iar după trecerea a șapte luni, vor începe să facă cercetări.
\par 15 Și când cineva din cei ce cutreieră țara va vedea os de om, va pune semn lângă el până ce groparii îl vor îngropa în Valea taberei lui Gog.
\par 16 Numele cetății va fi Hamona (cimitir). Și așa vor curăți ei țara".
\par 17 "Așa zice Domnul Dumnezeu: Iar tu, fiul omului, spune la tot felul de păsări și tuturor fiarelor câmpului: "Adunați-vă și mergeți din toate părțile, adunați-vă la jertfa Mea, pe care o voi junghia Eu pentru voi, la jertfa cea mare din munții lui Israel și veți mânca acolo carne și veți bea sânge.
\par 18 Carnea războinicilor o veți mânca și veți bea sângele căpeteniilor pământului, al berbecilor, al mieilor, al țapilor, al vițeilor și al tuturor celor îngrășați din Vasan;
\par 19 Veți mânca grăsime până vă veți sătura și veți bea sânge până vă veți îmbăta din jertfa Mea, pe care o voi junghia pentru voi.
\par 20 și vă veți sătura la masa Mea de cai și de călăreți, de războinici și de tot felul de ostași, zice Domnul Dumnezeu.
\par 21 Voi arăta slava Mea între neamuri, și toate neamurile vor vedea judecata Mea, pe care o voi săvârși Eu, și mâna Mea, pe care o voi pune asupra lor.
\par 22 Atunci va ști casa lui Israel că Eu sunt Domnul Dumnezeul lor, din ziua de astăzi înainte.
\par 23 Popoarele de asemenea vor cunoaște că neamul lui Israel a fost dus în robie pentru nedreptatea lui; pentru că ei s-au purtat cu necredincioșie înaintea Mea, am ascuns Eu fața și i-am dat pe mâna vrăjmașilor lor și au căzut ei cu toții de sabie;
\par 24 Pentru necurățiile lor și pentru fărădelegile lor am făcut Eu aceasta cu ei și Mi-am ascuns Eu fața de la dânșii.
\par 25 De aceea așa zice Domnul Dumnezeu: Acum voi întoarce prizonierii lui Iacov, Mă voi îndura de toată casa lui Israel și voi fi zelos pentru numele Meu cel sfânt.
\par 26 Ei vor uita ocara lor și toate nelegiuirile lor pe care le-au făcut înaintea Mea, când vor trăi în țara lor în siguranță și nimeni nu-i va tulbura.
\par 27 Când îi voi întoarce dintre popoare și-i voi aduna dintre țările vrăjmașilor lor și-Mi voi arăta în ei sfințenia Mea înaintea ochilor a multor neamuri.
\par 28 Atunci vor ști că Eu sunt Domnul Dumnezeul lor, când, după ce i-am risipit printre popoare, iarăși îi voi aduna în țara lor și nu voi mai lăsa acolo nici unul din ei.
\par 29 Și nu voi mai ascunde de ei fața Mea pentru că voi revărsa duhul Meu asupra casei lui Israel, zice Domnul Dumnezeu.

\chapter{40}

\par 1 În anul al douăzeci și cincilea după robirea noastră, la începutul anului, în ziua a zecea a lunii, la paisprezece ani după dărâmarea cetății Ierusalimului, tocmai în ziua aceea a fost mâna Domnului peste mine și m-a dus în țara lui Israel.
\par 2 Dar am fost dus acolo în niște vedenii dumnezeiești și am fost așezat pe un munte foarte înalt. Pe acest munte, pe partea lui de miazăzi, era un fel de ziduri de cetate.
\par 3 Am fost dus acolo și iată era un om, a cărui înfățișare era ca înfățișarea aramei strălucitoare, el avea în mână o sfoară de in și o prăjină de măsurat și stătea la poartă.
\par 4 Omul acela mi-a zis: "Fiul omului, privește cu ochii tăi, ascultă cu urechile tale ș ia aminte la toate câte am să-ți arăt, căci de aceea ai fost tu adus aici, ca să-ți arăt acestea. Să vestești casei lui Israel tot ce vei vedea".
\par 5 Iată, un zid înconjura templul pe dinafară de jur împrejur și în mâna omului aceluia era o prăjină de măsurat, lungă de șase coți, socotind cotul cât lungimea mâinii de la cot în jos, cu palmă cu tot. Omul acela a măsurat zidul și era gros de o prăjină și înalt tot de o prăjină.
\par 6 Apoi a mers la poarta cea cu fața spre răsărit, se urcă pe cele șapte trepte ale ei și găsi terasa ei lată de o prăjină și terasa cea dinăuntru lată tot de o prăjină.
\par 7 Fiecare din odăile laterale avea lungimea de o prăjină și lățimea tot de o prăjină, iar tinda dintre odăi era de cinci coți.
\par 8 Apoi a măsurat pridvorul porții dinăuntru, și era de o prăjină.
\par 9 Iar pridvorul celălalt a avut la măsurătoare opt coți și stâlpii câte doi coți. Acest pridvor era la poartă, înăuntru, spre templu.
\par 10 Odăi de pază la porțile dinspre răsărit erau trei de o parte și trei de cealaltă parte; tustrele aveau aceeași măsură și aceeași măsură aveau și stâlpii de o parte și de cealaltă.
\par 11 A măsurat apoi deschizătura porții și a găsit zece coți lățime și treisprezece coți lungime.
\par 12 Dinaintea odăilor de pază era o prispă de un cot și la cele de dincolo o prispă tot de un cot. Odăile de dincoace aveau șase coți și tot șase coți aveau și odăile de dincolo.
\par 13 Apoi a măsurat el fața porții, de la acoperișul unei camere până la acoperișul celeilalte, douăzeci și cinci de coți în lățime. Ușile camerelor erau față în față.
\par 14 La măsurarea pridvorului a măsurat douăzeci de coți; dinaintea pridvorului era o curte, în fața porții.
\par 15 De la fața de dinafară a porții până la fața ei dinăuntru erau cincizeci de coți.
\par 16 Odăile de pază aveau ferestre cu gratii; asemenea ferestre erau și printre stâlpi, spre poartă de jur împrejur. Iar pe stâlpi erau săpate ramuri de finic.
\par 17 Apoi m-a dus omul acela în curtea cea din afară șl iată acolo erau camere și împrejurul curții era făcut caldarâm de piatră. Pe acel caldarâm erau treizeci de camere.
\par 18 Caldarâmul acesta era pe laturile porții, e răspunzând lungimii lor. Acest caldarâm era mai jos.
\par 19 A măsurat apoi lățimea, de la poarta de jos până la marginea de afară a curții lăuntrice, o sută de coți.
\par 20 Apoi m-a dus spre miazănoapte și iată, era și acolo o poartă la curtea cea de la margine, care dădea spre miazănoapte și a măsurat clădirea porții cât e de lungă și de lată;
\par 21 Camerele de pe laturile ei și prispele ei, trei de o parte și trei de alta, stâlpii ei erau de măsura celor de la poarta cea dinspre răsărit; lungimea clădirii porții, cincizeci de coți și lățimea, douăzeci și cinci de coți.
\par 22 Ferestrele ei, prispele ei și palmierii ei erau ca și la poarta care dădea spre răsărit; la ea duc șapte trepte și înaintea ei are pridvor.
\par 23 Dinaintea ei, în curtea cea dinăuntru, este o poartă care dă spre miazănoapte, ca și cea care dă spre răsărit. Și a măsurat de la poarta curții de la margine până la poarta curții dinăuntru și a găsit o sută de coți.
\par 24 După aceea m-a dus spre miazăzi, unde era poarta de miazăzi; și a măsurat-o pe ea, stâlpii și pridvorul; și aveau aceeași măsură.
\par 25 Ferestrele camerelor și ale pridvorului erau la fel cu ferestrele clădirilor celorlalte două porți; lungimea era de cincizeci de coți și lățimea de douăzeci și cinci de coți.
\par 26 Scara dinaintea ei era cu șapte trepte și avea înaintea ei pridvor; și palmierii de podoabă erau unul pe un stâlp și altul pe alt stâlp de la intrare.
\par 27 Și în fala ei se afla poarta curții celei dinăuntru. A măsurat de la poarta de miazăzi până la poarta curții celei dinăuntru o sută de coți.
\par 28 Apoi m-a dus pe poarta de miazăzi în curtea cea dinăuntru; și a măsurat el poarta cea dinspre miazăzi și a găsit aceeași măsură.
\par 29 Camerele ei de pe laturi, stâlpii ei și pridvorul ei aveau aceeași măsură. Împrejur, la camere și la pridvor avea ferestre; lungimea era de cincizeci de coți și lățimea de douăzeci și cinci de coji.
\par 30 Împrejur avea coridor lung de douăzeci și cinci de coți și lat de cinci.
\par 31 Spre curtea cea de la margine, avea și pridvor; pe stâlpii ei erau palmieri săpați, iar scara dinaintea ei era cu opt trepte.
\par 32 Apoi m-a dus la poarta cea dinspre răsărit a curții celei dinăuntru și a măsurat-o și a găsit aceeași măsură.
\par 33 Camerele ei cele de pe laturi, stâlpii ei și pridvorul ei erau de aceeași măsură. Împrejur avea ferestre la camere și la pridvor. Lungimea ei era de cincizeci de coți și lățimea de douăzeci și cinci de coți.
\par 34 Pridvorul ei era spre curtea cea de la margine și avea palmieri săpați pe stâlpii ei de o parte și de alta a intrării; iar scara ei avea opt trepte.
\par 35 După aceea m-a dus la poarta cea dinspre miazănoapte și a măsurat-o și a găsit aceeași măsură.
\par 36 Ea avea camere pe de lături, stâlpi, pridvor și ferestre pe din afară; în lungime avea cincizeci de coli, iar în lățime douăzeci și cinci de coli.
\par 37 Pridvorul ei era spre curtea cea de la margine și palmieri avea și pe unul și pe altul din stâlpii de la intrare; iar scara ei avea opt trepte.
\par 38 Mai era o cameră care se deschidea spre pridvorul porții; acolo se spăla jertfa arderii de tot.
\par 39 Iar în pridvorul porții erau două mese de o parte a intrării și două mese de cealaltă parte, pe care se tăiau jertfele arderii de tot, jertfele pentru păcat și jertfele pentru vină.
\par 40 Pe latura din afară a pridvorului, spre răsărit, aproape de intrarea porții celei dinspre miazănoapte, se aflau două mese și pe cealaltă latură a pridvorului, spre apus, se aflau iar două mese.
\par 41 Așadar erau patru mese de o parte și patru mese de cealaltă parte; în total opt mese, pe care se tăiau jertfele.
\par 42 Patru mese, pentru pregătirea arderilor de tot, erau de piatră cioplită, lungi de un cot și jumătate și late de un cot și jumătate și înalte de un cot. Pe ele se puneau uneltele de junghiat, arderile de tot și jertfele celelalte.
\par 43 Pe margine, de jur împrejur mesele aveau un pervaz din ele, înalt de un lat de mână; iar deasupra meselor era acoperământul, ca să le apere de ploaie și de căldură.
\par 44 În curtea cea dinăuntru, în partea din afară a clădirilor porților, erau două camere pentru cântăreți: una pe latura clădirii porții dinspre miazănoapte, cu fața spre miazăzi, iar cealaltă pe latura clădirii porții celei de miazăzi, cu fața spre miazănoapte.
\par 45 Și mi-a zis bărbatul acela: Camera aceasta, cu fața spre miazăzi, este pentru preoți care veghează la paza templului;
\par 46 Iar camera cea cu fața spre miazănoapte este pentru preoții care fac slujba la altar; aceștia sunt fiii lui Țadoc, singurii dintre fiii lui Levi care se apropie de Domnul ca să-I slujească.
\par 47 Apoi a măsurat curtea și a găsit o sută de coți în lungime și o sută de coți în lățime; ea era în patru colțuri, iar în fața templului se ridica altarul.
\par 48 Apoi m-a dus în pridvorul templului, a măsurat stâlpi pridvorului și a găsit cinci coți de o parte și cinci coți de cealaltă parte, de la țâțânile ușilor până în pereți trei coți de o parte și trei coți de altă parte.
\par 49 Lungimea pridvorului era de unsprezece coți și lățimea de douăzeci de coți. La el suia o scară cu zece trepte. Și lângă stâlpi erau coloane: una de o parte și alta de altă parte a intrării.

\chapter{41}

\par 1 După aceea m-a dus în templu, a măsurat stâlpii și a găsit șase coți în lățime de o parte și șase coți de cealaltă parte; aceasta era lărgimea cortului adunării.
\par 2 Lărgimea ușii era de zece coți și de amândouă părțile ușii câte cinci coți. A măsurat apoi lungimea templului și a găsit-o de patruzeci de coți, iar lățimea de douăzeci de coti.
\par 3 A mers înăuntrul templului și a măsurat stâlpii de la ușă și i-a găsit de doi coți, iar ușa de șase coți; de la țâțânile ușii până în perete a găsit șapte coți de o parte și șapte coți de cealaltă parte.
\par 4 A măsurat locașul și a găsit douăzeci de coți în lungime, douăzeci de coți în lățime și mi-a zis: "Aceasta este Sfânta Sfintelor".
\par 5 Apoi a măsurat peretele templului și l-a găsit gros de șase coți; lățimea camerelor de pe laturile templului de jur împrejur a găsit-o de patru coți.
\par 6 În jurul templului sunt trei rânduri de câte treizeci de camere, cameră lângă cameră. Ele intră într-un zid, care este făcut împrejurul templului anume pentru aceste camere, ca ele să fie întărite, dar de peretele templului nu se ating.
\par 7 Camerele cu cât sunt mai sus, cu atât sunt mai încăpătoare, subțiindu-se peretele. Din camerele de jos te urci la cele din mijloc și de la cele din mijloc la cele de sus; suișul este învârtit, căci te sui pe o scară în chipul melcului.
\par 8 Și am văzut un caldarâm înalt împrejurul templului, care slujea de temelie pentru camerele laterale, care avea lățimea de o prăjină întreagă, adică de sase coti mari.
\par 9 Grosimea zidului camerelor laterale, care ieșeau în afară, era de cinci coți, iar lângă camerele laterale era un loc gol.
\par 10 Locul gol dintre camerele laterale ale templului și dintre camerele dimprejurul curții templului are o lățime de douăzeci de coți de jur împrejur.
\par 11 Ușile camerelor laterale dădeau într-un loc deschis, o ușă în partea de miazănoapte, iar altă ușă în partea de miazăzi; iar lățimea locului deschis era de cinci coți.
\par 12 Clădirea de dinaintea locului deschis din partea de apus avea o lățime de șaptezeci do coți; zidul acestei clădiri era de cinci coți de jur împrejur, iar lungimea ei era de nouăzeci de coți.
\par 13 A măsurat templul; el avea o sută de coți în lungime; locul liber, clădirea de la apus și zidurile lui de asemenea aveau o lungime de o sută de coti.
\par 14 Lățimea feței templului și curtea din partea dinspre răsărit era de o sută de coți.
\par 15 Apoi a măsurat lungimea clădirii din fața locului liber din spatele templului cu camerele laterale de o parte și de alta a lui și avea o sută de coți.
\par 16 Ușorii ușilor și ai ferestrelor cu gratii, camerele laterale din cele trei caturi, pe jos și de jos până Ia ferestre, de jur împrejur erau căptușite cu lemn. Ferestrele erau închise.
\par 17 Până la înălțimea ușilor, tot peretele, atât la despărțitura cea mai dinăuntru, cât și la cea mai din afară, de jur împrejur, înăuntru și pe afară, era împodobit cu chipuri săpate cu anumită măsură;
\par 18 Și anume, erau săpați heruvimi și palmieri, astfel: între fiecare doi heruvimi un palmier și fiecare heruvim avea două fețe.
\par 19 Heruvimul într-o parte avea o față de om, întoarsă spre un palmier, și în cealaltă parte avea o față de leu, întoarsă spre alt palmier. Așa erau făcute chipuri în tot templul și împrejur.
\par 20 De jos până la înălțimea ușilor erau sculptați heruvimi și palmieri ca și pe pereții templului.
\par 21 În templu, ușorii ușilor erau în patru muchii. Iar în fața Sfintei Sfintelor se vedea un fel de altar de lemn.
\par 22 Altarul era de lemn, înalt de trei coți și lung de doi coți; și coarnele lui și postamentul lui și pereții lui erau de lemn. Și mi-a zis bărbatul acela: Aceasta este masa cea de dinaintea Domnului.
\par 23 Sfânta Sfintelor avea două uși și Sfânta avea două uși.
\par 24 Fiecare din acele două uși avea câte două canaturi, care se deschideau într-o parte și în alta, căci două canaturi erau la o ușă și două la cealaltă ușă.
\par 25 Și pe ele, pe ușile templului, erau sculptați heruvimi și palmieri, ca și pe pereți. Iar în fața pridvorului, afară, era o pardoseală de lemn.
\par 26 Pe o latură și pe alta a pridvorului erau ferestre cu gratii și chipuri de palmieri; asemenea erau și în camerele laterale și pe căptușeala de lemn.

\chapter{42}

\par 1 După aceea m-a scos spre curtea cea din afară spre miazănoapte și m-a dus la camerele din fața curții, din fața clădirii, spre miazănoapte,
\par 2 La acel loc care este spre poarta de miazănoapte a curții dinăuntru și care are în lungime o sută de coți, iar în lățime cincizeci de coți;
\par 3 Adică în dreptul locului de douăzeci de coți al curții dinăuntrul și în dreptul caldarâmului curții din afară, unde era o galerie cu trei rânduri în fața altei galerii cu trei rânduri.
\par 4 Pe dinaintea camerelor era un loc de trecere de zece coți lat și de o sută de coți lung. Ușile erau spre miazănoapte.
\par 5 Camerele cele de sus erau mai strâmte decât cele de jos și cele de la mijloc ale clădirii, pentru că galeriile le răpeau o parte din întinderea lor.
\par 6 Ele aveau trei caturi, dar nu aveau stâlpi ca în curte. De aceea, plecând de la pământ, camerele de sus erau mai strâmte decât cele de jos și decât cele de la mijloc.
\par 7 Zidul din afară, paralel cu camerele, dinspre curtea cea din afară, din fața camerelor, avea în lungime cincizeci de coți,
\par 8 Pentru că și camerele dinspre curtea cea din afară aveau o lungime tot numai de cincizeci de coți. Și aceste două clădiri care erau în dreptul templului aveau o sută de coți.
\par 9 Iar de jos, intrarea la aceste camere era dinspre răsărit, cum veneai din curtea cea din afară.
\par 10 Se aflau de asemenea camere și în lungul zidului curții dinăuntru, din partea dinspre miazăzi, în fața curții și a clădirii templului;
\par 11 Dinaintea lor era un loc de trecere întocmai ca și la camerele cele dinspre miazănoapte și avea aceeași lungime ca și acelea și aceeași lățime; toate ieșirile lor, întocmirea lor și ușile lor erau la fel;
\par 12 Tot așa era și cu ușile camerelor de la miazăzi. Apoi era o ușă de la capătul locului de trecere, ce mergea de-a lungul zidului drept spre răsărit.
\par 13 Și mi-a zis bărbatul acela: "Camerele dinspre miazănoapte și camerele dinspre miazăzi, care sunt în fața curții, sunt camere sfinte, în care preoții care se apropie de Domnul mănâncă cele mai sfinte jertfe; tot acolo pun ei cele mai sfinte jertfe și prinosul de pâine, jertfă pentru păcat și jertfă pentru vină, căci acesta este loc sfânt.
\par 14 Când preoții intră acolo, nu se cuvine să iasă din acest loc sfânt în curtea cea din afară, până nu lasă acolo hainele lor cu care au fost îmbrăcați la slujbă, că acestea sunt sfințite; ei trebuie să se îmbrace cu alte haine și numai după aceea să iasă la popor.
\par 15 După ce a isprăvit el de măsurat templul și curțile cele dinăuntrul zidurilor, m-a scos pe poarta dinspre răsărit și a început să măsoare împrejur.
\par 16 Și a măsurat latura cea dinspre răsărit cu prăjina de măsurat și a găsit în total cinci sute de coți.
\par 17 Pe latura de miazănoapte cu aceeași prăjină a măsurat cinci sute de coți.
\par 18 Pe latura de miazăzi a măsurat cu prăjina de măsurat tot cinci sute de coți.
\par 19 Apoi, apucând pe latura de apus, a măsurat cu prăjina de măsurat cinci sute de coți.
\par 20 A măsurat în cele patru laturi zidul de jur împrejurul locașului sfânt; lungimea era de cinci sute de coți și lățimea de cinci sute de coți; zidul acesta despărțea locul sfânt de cel ce nu este sfânt.

\chapter{43}

\par 1 Apoi m-a dus la poartă, la poarta dinspre răsărit.
\par 2 Și iată slava Dumnezeului lui Israel venea dinspre răsărit; glasul Lui era ca glasul de ape multe și pământul strălucea de slava Lui.
\par 3 Vedenia aceasta era ca aceea pe care o văzusem mai înainte, tocmai ca aceea pe care o văzusem când am venit să vestesc pieirea cetății; vedenia aceasta era asemenea vedeniei pe care o văzusem la râul Chebar. Atunci am căzut cu fața la pământ.
\par 4 Și slava Domnului a intrat în templu pe poarta care este cu fața spre răsărit.
\par 5 Și m-a ridicat Duhul și m-a dus în curtea cea dinăuntru și iată slava Domnului umplea tot templul.
\par 6 Și am auzit pe Cineva Care-mi grăia din templu, iar bărbatul acela de mai înainte stătea lângă mine.
\par 7 și mi-a zis glasul: "Fiul omului, acesta este locul tronului Meu și locul pe care-Mi pun tălpile picioarelor Mele, unde voi locui veșnic în mijlocul fiilor lui Israel; casa lui Israel nu va mai întina numele Meu cel sfânt, nici ea, nici regii ei, prin desfrânările lor, prin cadavrele regilor lor, cu locurile lor înalte.
\par 8 El își puneau pragurile lor lângă pragurile Mele și țâțânile ușilor lor lângă țâțânile ușilor Mele, încât un singur perete era între Mine și ei, și întinau numele Meu cel sfânt cu urâciunile lor pe care le făceau, și de aceea i-am pierdut întru mânia Mea.
\par 9 Iar acum ei vor depărta de la Mine desfrânările lor și trupurile moarte ale regilor lor și Eu voi locui în mijlocul lor în veci.
\par 10 Iar tu, fiul omului, descrie casei lui Israel acest templu, ca să se rușineze ei de fărădelegile lor și să-i măsoare planul.
\par 11 Dacă ei se vor rușina de toate acelea câte au făcut, atunci să le arăți chipul templului și planul lui, ieșirile lui și intrările lui și forma lui și toate întocmirile lui, toate formele lui și toate legile lui; pune toate acestea în scris înaintea ochilor lor, ca ei să vadă forma lui și toate întocmirile lui și să le urmeze întocmai.
\par 12 Iată acum legea templului: pe vârful muntelui tot locul care îl înconjoară este locul cel mai sfânt. Aceasta este legea templului.
\par 13 Iată măsurile jertfelnicului în coți, socotind drept cot brațul de la cot în jos, împreună cu palma: soclul de jos al lui era înalt de un cot și lat tot de un cot; iar brâul, care-l încingea pe margine, avea o palmă în lățime. Acesta era soclul jertfelnicului.
\par 14 Pe soclu, care era la pământ, se înălța un fundament mic, care mergea ca un fel de prispă de jur împrejurul jertfelnicului, înaltă de doi coți și lată de un cot. Pe fundamentul mic se înălța fundamentul mare, care iarăși încingea jertfelnicul ca o prispă înaltă de patru coți și lată de un cot.
\par 15 Pe fundamentul mare se înălța însuși jertfelnicul, înalt de patru coți; și din jertfelnic se ridicau patru coarne.
\par 16 Jertfelnicul avea doisprezece coți înălțime și doisprezece coți în lungime. El era în patru colțuri, având toate cele patru laturi ale sale deopotrivă de lungi.
\par 17 Iar fața soclului jertfelnicului era de paisprezece coți în lungime și de paisprezece coți înălțime și împrejurul ei avea un brâu de o jumătate de cot. Soclul de pus împrejur era lat de un cot, iar scara de suit la jertfelnic era în partea de răsărit.
\par 18 Apoi mi-a zis bărbatul acela: "Fiul omului, așa grăiește Domnul Dumnezeu: Iată rânduielile jertfelnicului pentru ziua când va fi el făcut, ca să se aducă pe el arderi de tot și ca să fie stropit cu sânge.
\par 19 Preoților din tribul lui Levi, care sunt din neamul lui Sadoc și care se apropie de Mine ca să-Mi slujească, dă-le, zice Domnul Dumnezeu, un vițel din cireada de boi ca jertfă pentru păcat;
\par 20 Și să iei sângele lui și să stropești cele patru coarne ale jertfelnicului și pe cele patru colțuri ale soclului lui și brâul cel dimprejur și astfel să-l cureți și să-l sfințești.
\par 21 Apoi ia vițelul cel de jertfă pentru păcat și arde-l la locul rânduit al templului, dar afară din locul cel sfânt.
\par 22 Iar a doua zi, ca jertfă pentru păcat, să aduci din turma de capre un țap fără meteahnă și să cureți jertfelnicul tot așa, cum l-ați curățit cu vițelul.
\par 23 Iar după ce vei isprăvi curățirea, ia din cireada de boi un vițel fără meteahnă și din turma de oi un berbec fără meteahnă,
\par 24 Și-i adu înaintea Domnului; preoții să arunce asupra lor sare și să-i înalțe ardere de tot Domnului.
\par 25 Șapte zile se aduce jertfă pentru păcat câte un țap pe zi; de asemenea să se aducă jertfă câte un vițel din cireada de boi și câte un berbec din turma de oi, fără meteahnă.
\par 26 Șapte zile să facă ispășire pentru jertfelnic, să-l curețe și să-l sfințească.
\par 27 Iar după sfârșitul acestor zile, în ziua a opta și mai departe, preoții vor înălța, pe jertfelnic, arderile de tot ale voastre și jertfele de împăcare și Eu Mă voi milostivi spre voi", zice Domnul Dumnezeu.

\chapter{44}

\par 1 Apoi m-a dus bărbatul acela înapoi la poarta cea din afară a templului, spre răsărit, și aceasta era închisă.
\par 2 Și mi-a zis Domnul: "Poarta aceasta va fi închisă, nu se va deschide și nici un om nu va intra pe ea, căci Domnul Dumnezeul lui Israel a intrat pe ea. De aceea va fi închisă.
\par 3 Cit privește pe rege, el se va așeza acolo, ca să mănânce pâine înaintea Domnului; pe calea porții va intra și pe aceeași cale va ieși".
\par 4 După aceea m-a dus pe calea porții de la miazănoapte, în fața templului, și am privit, și iată slava Domnului umplea templul Domnului, și am căzut cu fața la pământ.
\par 5 Și mi-a zis Domnul: "Fiul omului, pleacă-ți inima ta, privește cu ochii tăi și ascultă cu urechile tale toate câte-ți voi grăi despre toate așezămintele templului Domnului și despre toate legile ei. Uită-te cu băgare de seamă la intrarea templului și la toate ieșirile din locașul cel sfânt.
\par 6 Și spune casei celei răzvrătite a lui Israel: Așa grăiește Domnul Dumnezeu: Destul vouă, casa lui Israel, cu toate urâciunile voastre;
\par 7 Că ați băgat înăuntru fii străini, netăiați împrejur la inimă și netăiați împrejur la trup, ca să stea în locașul Meu cel sfânt și să spurce templul Meu; ați adus pâinea Mea, grăsimea și sângele și ați stricat legământul Meu cu tot felul de urâciuni de ale voastre.
\par 8 Voi n-ați făcut slujba Mea în templu, ci i-ați pus pe ei să îndeplinească slujba voastră în templul Meu în locul vostru".
\par 9 Așa zice Domnul Dumnezeu: "Nici un fiu străin, netăiat împrejur ia inimă și netăiat împrejur la trup, nu trebuie să intre în locașul Meu cel sfânt, nici chiar acel fiu care locuiește în mijlocul fiilor lui Israel.
\par 10 Chiar și leviții, care s-au depărtat de Mine în timpul rătăcirii lui Israel pentru a-și urma idolii lor, își vor purta greutatea păcatului lor.
\par 11 Ei vor sluji în templul Meu ca străjeri la porțile templului și făcând slujba templului; ei vor junghia pentru popor arderi de tot și alte jertfe și vor sta înaintea lui ca să-i slujească.
\par 12 Pentru că ei au slujit înaintea idolilor lui și au fost pentru casa lui Israel sminteală și au dus-o la necredință, Mi-am ridicat mâna împotriva lor, zice Domnul Dumnezeu, și își vor lua pedeapsa pentru vinovăția lor.
\par 13 Ei nu se vor apropia de Mine ca să slujească înaintea Mea; nu se vor apropia de lucrurile Mele cele sfinte, nici de Sfânta Sfintelor, ci vor purta asupra lor necinstea și urâciunile lor, pe care le-au făcut.
\par 14 Îi voi face străjeri la templu, să facă slujba lui, și tot ce trebuie făcut la el.
\par 15 Iar preoții din seminția lui Levi, fiii lui Sadoc, care în vremea abaterii de la Mine a fiilor lui Israel au îndeplinit slujirea Mea în locașul Meu cel sfânt, aceia se vor apropia de Mine, ca să-Mi slujească și vor sta înaintea feței Mele, ca să-Mi aducă grăsime și sânge, zice Domnul Dumnezeu.
\par 16 Ei vor intra în locașul Meu cel sfânt și se vor apropia de masa Mea, ca să-Mi slujească; ei vor îndeplini slujirea Mea.
\par 17 Când vor veni la poarta curții dinăuntru, atunci se vor îmbrăca în haine de in, iar haine de lână nu trebuie să aibă pe ei în timpul slujbei lor în porțile curții dinăuntru și în templu.
\par 18 Turbanele de pe capetele lor trebuie să fie tot de in; hainele cele de pe dedesubt de pe coapsele lor să fie de asemenea de in. Ei nu trebuie să se încingă, ca să nu transpire.
\par 19 Iar când va trebui să iasă în curtea cea de la margine, la popor, atunci trebuie să dezbrace hainele lor cu care au slujit și să le lase în camerele cele sfințite și să se îmbrace "u alte haine, ca să nu se atingă de popor cu hainele lor cele sfințite.
\par 20 Capetele lor nu trebuie să și le radă, dar nici părul să nu-și lase să crească, ci să-și tundă neapărat capul.
\par 21 Vin nu trebuie să bea nici un preot când are să intre în curtea cea dinăuntru;
\par 22 Nici văduvă, nici despărțită de bărbat nu trebuie să ia ei de femeie, ci pot să ia numai fată din neamul casei lui Israel și văduvă care a rămas în văduvie după moartea unui preot.
\par 23 Ei trebuie să învețe pe poporul Meu a deosebi ce este sfânt de ce nu este sfânt și să le lămurească ce este curat și ce este necurat.
\par 24 În pricinile nehotărâte, ei trebuie să ia parte la judecată și vor judeca după așezămintele Mele și legile Mele vor păzi și toate rânduielile Mele cele pentru sărbătorile Mele și pentru zilele Mele de odihnă le vor păzi cu sfințenie.
\par 25 De omul mort nimeni din ei nu trebuie să se apropie, ca să nu se facă necurat; numai pentru tată și pentru mamă, pentru fiu și pentru fiică, pentru frate și soră nemăritată pot să se facă necurați.
\par 26 După curățirea acestuia, trebuie să i se mai socotească încă șapte zile.
\par 27 Și în ziua aceea, când trebuie să se apropie de cele sfinte în curtea cea dinăuntru, ca să slujească în templu, trebuie să aducă jertfă pentru păcat, zice Domnul Dumnezeu.
\par 28 Iar cât privește partea lor de moșie, apoi Eu sunt partea lor; și moșie nu li se va da întru Israel, căci Eu sunt moșia lor.
\par 29 Ei vor mânca din prinosul de pâine, din jertfa pentru păcat și din jertfa pentru vină. Și tot ce este afierosit în Israel al lor este.
\par 30 Pârga din toate roadele voastre și din tot felul de prinoase, ori din ce s-ar alcătui prinoasele voastre, este a preoților. Pârga din cele treierate ale voastre să o dați preotului, ca să odihnească binecuvântarea asupra casei tale.
\par 31 Nici un fel de mortăciuni și nimic sfâșiat de fiară, nici de păsări, nici de dobitoace nu trebuie să mănânce preoții.

\chapter{45}

\par 1 Când veți împărți pământul în părți prin sorți, atunci să osebiți o parte sfântă a Domnului de douăzeci și cinci de mii de coți în lungime și douăzeci de mii în lățime, ca să fie sfânt acest loc în toate hotarele lui de jur împrejur.
\par 2 Din el va merge la locașul sfânt o bucată în patru colțuri de cinci sute de coți pe cinci sute de coți și împrejurul lui o fâșie de loc lată de cincizeci de coti.
\par 3 De la locul pe care va fi locașul cel sfânt, Sfânta Sfintelor, vei măsura cele douăzeci și cinci de mii de coți în lungime și zece mii de coți în lățime;
\par 4 Această parte sfântă de pământ va fi a preoților care slujesc locașului sfânt și care se apropie să slujească Domnului; acesta va fi pentru ei loc de case și pentru locașul sfânt.
\par 5 Douăzeci și cinci de mii de coți în lungime și zece mii de coți în lățime va fi bucata de pământ a Leviților, slujitorii templului, ca moșie a lor cu cetăți de locuit.
\par 6 În stăpânirea cetății să dați cinci mii de coți în lățime și douăzeci și cinci de mii în lungime, în fața locului sfânt, care este osebit pentru Domnul. Acesta trebuie să fie al întregii oase a lui Israel.
\par 7 Și regelui să-i dați parte de pământ, de o parte și de alta a locului sfânt care este osebit Domnului și a locului cetății, adică o parte la răsărit, în partea de răsărit a celor două porțiuni și o parte la asfințit, în partea de asfințit a celor două porțiuni. În lungime se vor întinde ca una din acele părți de la hotarul de apus până la hotarul de răsărit al țării.
\par 8 Acesta este pământul lui, moșia lui în Israel, ca regii Mei de acum să nu mai strâmtoreze poporul Meu, și ca să lase pământul casei lui Israel după triburile ei.
\par 9 Așa zice Domnul Dumnezeu: "Destul, regi ai lui Israel! Lăsați nedreptățile și împilările și faceți judecată și dreptate! Încetați de a mai asupri pe poporul Meu, zice Domnul Dumnezeu.
\par 10 Să aveți cântar drept și efă dreaptă și bat drept.
\par 11 Efa și batul trebuie să fie măsuri deopotrivă de mari, încât într-un bat să încapă a zecea parte dintr-un homer și într-o efă să încapă a zecea parte dintr-un homer. Mărimea lor trebuie măsurată cu homerul.
\par 12 Siclul să aibă douăzeci de ghere. Mina va fi de douăzeci de sicli, de douăzeci și cinci de sicli și de cincisprezece sicli.
\par 13 lată ofranda ce trebuie să dați regelui: a șasea parte de efă din homerul de grâu și a șasea parte de efă din homerul de orz.
\par 14 Hotărârea pentru untdelemn: dintr-o coră de untdelemn veți da a zecea parte dintr-un bat; zece baturi fac un homer, căci homerul are zece baturi;
\par 15 Veți da o oaie dintr-o turmă de două sute de oi, din pășunile cele mănoase ale lui Israel. Toate acestea le veți da pentru prinosul de pâine și ardere de tot și jertfă de împăcare spre curățirea voastră, zice Domnul Dumnezeu.
\par 16 Tot poporul țării este îndatorat să dea aceste prinoase regelui lui Israel,
\par 17 Iar în sarcina regelui vor fi arderile de tot, prinosul de pâine și turnările la sărbători, la lună nouă, la ziua de odihnă și la toate prăznuirile casei lui Israel. El va trebui să aducă jertfă pentru păcat, prinos de pâine, ardere de tot și jertfă de împăcare pentru ispășirea casei lui Israel".
\par 18 Asa zice Domnul Dumnezeu: "În ziua întâi a lunii întâi ia din cireada de boi un junc fără meteahnă și curăță locașul sfânt.
\par 19 Preotul să ia din sângele acestei jertfe pentru păcat și să stropească cu el ușorii ușii templului, cele patru laturi ala jertfelnicului și ușorii porților curții celei dinăuntru.
\par 20 Același lucru să-l faci și în ziua a șaptea a lunii, pentru cei ce au greșit cu știință și din neștiință și așa să cureți templul.
\par 21 În ziua a paisprezecea a lunii întâi, trebuie să prăznuiți Paștile, sărbătoare de șapte zile, când trebuie să se mănânce azime.
\par 22 În această zi regele va aduce pentru sâne și pentru tot poporul țării un vițel ca jertfă pentru păcat.
\par 23 Și în cele șapte zile ale sărbătorii el trebuie să aducă ardere de tot Domnului în fiecare zi câte șapte viței și câte șapte berbeci fără meteahnă, iar ca jertfă pentru păcat să aducă în fiecare zi câte un țap din turma de capre.
\par 24 Prinos de pâine trebuie să aducă el câte o efă de fiecare vițel și câte o efă de fiecare berbec și câte un hin de untdelemn la efă.
\par 25 În ziua a cincisprezecea a lunii a șaptea, la sărbătoarea corturilor, timp de șapte zile, el trebuie să aducă la fel: aceeași jertfă pentru păcat, aceeași ardere de tot și tot atâta prinos de pâine și tot atâta untdelemn".

\chapter{46}

\par 1 Așa zice Domnul Dumnezeu: "Poarta curții celei dinăuntru, care dă spre răsărit, trebuie să fie încuiată în timpul celor șapte zile de lucru, iar în ziua odihnei ea trebuie să fie deschisă, și în ziua de lună nouă trebuie să fie deschisă.
\par 2 Regele va trece prin pridvorul din afară al porții acesteia și va sta la ușorul acestei porți; iar preoții vor săvârși arderea de tot a lui, și jertfa lui cea de împăcare; și el din pragul porții se va închina Domnului și va ieși, iar poarta va rămâne neîncuiată până seara.
\par 3 Poporul țării se va închina înaintea Domnului, dinaintea porții, în ziua odihnei și la lună nouă.
\par 4 Arderea de tot pe care regele trebuie să o aducă Domnului în ziua odihnei, trebuie să fie de șase miei fără meteahnă și un berbec fără meteahnă,
\par 5 Prinosul de pâine de o efă de făină, cu berbecul și cu mieii, cât îi va da mâna, iar untdelemnul, un hin la efă.
\par 6 În ziua de lună nouă se va aduce de el un junc fără meteahnă din cireada de boi și de asemenea șase miei și un berbec fără meteahnă.
\par 7 Prinos de pâine va aduce o efă cu vițelul și o efă cu berbecul, iar cu mieii, cât îi va da mâna, untdelemn, câte un hin la efă.
\par 8 Când vine regele, trebuie să intre prin pridvorul porții celei dinăuntru și tot pe acolo să iasă.
\par 9 Iar când va veni poporul țării înaintea feței Domnului, la sărbători, atunci, intrând pe poarta de miazănoapte pentru închinare, trebuie să iasă pe poarta de miazăzi, iar când intră pe poarta de miazăzi, trebuie să iasă pe poarta de miazănoapte; el nu trebuie să iasă tot pe acea poartă pe care a intrat, ci trebuie să iasă pe cea din fața aceleia.
\par 10 Regele trebuie să fie în mijlocul lor; când intră ei, intră și el, și când ies ei, iese și el.
\par 11 În zilele de sărbătoare și în zilele de bucurie, prinosul de pâine din partea lui trebuie să fie de câte o efă, cu vițelul și berbecul, iar cu mieii, cât îi va da mina, iar untdelemn, câte un hin la efă.
\par 12 Iar dacă regele, din evlavie, vrea să aducă ardere de tot Domnului, trebuie să i se deschidă porțile cete de la răsărit și el va săvârși arderea sa de tot și jertfa sa de mulțumire tot așa cum a săvârșit-o în ziua odihnei, și după aceea va ieși, iar după ieșirea lui poarta se va închide.
\par 13 În fiecare zi să aduci Domnului ardere de tot un miel de un an, fără meteahnă; în fiecare dimineață să-l aduci.
\par 14 Iar ca prinos de pâine să adaugi la el în fiecare dimineață a șasea parte de efă și untdelemn a treia parte din hin, ca să amesteci făina. Aceasta este o lege veșnică despre prinosul de pâine ce trebuie să se aducă Domnului totdeauna.
\par 15 Să se aducă ardere de tot un miel și prinos de pâine și untdelemn necontenit în fiecare dimineață.
\par 16 Așa zice Domnul Dumnezeu: "Dacă regele va da vreunuia din fiii săi dar, acest dar trebuie să treacă moștenire și la fiii aceluia. Ei îl vor stăpâni ca pe o moștenire.
\par 17 Iar dacă el dă din moștenirea sa cuiva din robii săi dar, acesta va fi al aceluia numai până la anul jubileu și atunci se va întoarce la rege. Moștenirea lui poate trece numai la fiii lui.
\par 18 Dar regele nu poate lua din partea de moștenire a poporului, strâmtorându-l în moștenirea lui. El numai din moștenirea sa poate împărți copiilor săi, ca nimeni din poporul Meu să nu fie izgonit din moștenirea iuim.
\par 19 Apoi m-a dus bărbatul acela pe calea porții celei de miazăzi la camerele sfințite ale preoților, dinspre miazănoapte, și iată era un loc în fund, spre apus.
\par 20 Și mi-a zis: "Acesta este locul unde preoții trebuie să fiarbă jertfa cea pentru vină și jertfa cea pentru păcat; unde trebuie să coacă pâinile din prinoase, fără să le scoată în curtea cea din afară, pentru sfințirea poporului".
\par 21 După aceea m-a scos în curtea cea din afară și m-a dus în cele patru colțuri ale curții și iată în fiecare colț al curții era încă o curte.
\par 22 În toate patru colțurile curții erau curți acoperite, de patruzeci de coți în lungime și de treizeci în lățime; curțile din toate patru colțurile aveau o singură măsură.
\par 23 Și împrejurul celor patru curți erau ziduri, iar pe lângă pereți de jur împrejur erau vetre pentru gătit mâncare.
\par 24 Și mi-a zis bărbatul acela: "Iată bucătăriile în care slujitorii templului fierb jertfele poporului".

\chapter{47}

\par 1 Apoi m-a dus înapoi la ușa templului și iată de sub pragul templului curgea o apă spre răsărit; pentru că templul era cu fața spre răsărit și apa curgea de sub partea dreaptă a templului, pe partea de miazăzi a jertfelnicului.
\par 2 M-a scos spoi pe partea cea de la miazănoapte și m-a dus pe din afară, împrejur, la poarta care dă spre răsărit și iată apa curgea pe partea dreaptă.
\par 3 Când bărbatul acela mergea spre răsărit, ținea în mână sfoara și a măsurat o mie de coți; și m-a dus prin apă și apa era până la glezne.
\par 4 A mai măsurat apoi o mie de coți Și m-a dus prin apă, și apa era până la genunchi. Și a mai măsurat încă o mie de con și apa era până la brâu.
\par 5 Și a mai măsurat încă o mie de coji și era un râu pe care nu-l puteam trece, căci apele crescuseră; erau ape de trecut înotând, un fluviu care nu se putea trece.
\par 6 Atunci mi-a zis bărbatul acela: "Ai văzut, fiul omului?" Și el m-a dus înapoi la malul râului.
\par 7 Și cârd am venit înapoi, iată pe malurile râului erau mulți arbori pe o parte și pe alta.
\par 8 și mi-a zis acela: "Această apă curge în partea de răsărit a țării, se va coborî în șes și va intra în mare, și apele ei se vor face sănătoase.
\par 9 Toată vietatea care mișună acolo pe unde va trece râul, va trăi. Pește va fi foarte mult, pentru că va intra acolo apa aceasta și apele din mare se vor face sănătoase; unde va intra râul acesta, toate vor trăi acolo.
\par 10 Lângă el vor sta pescarii de la En-Gaddi până la En-Eglaim și își vor arunca mrejele. Vor fi pești de tot soiul, ca cei din Marea cea Mare (Mediterană) și foarte numeroși.
\par 11 Mlaștinile lui și lacurile lui, care nu se vor face sănătoase, vor rămâne pentru sare.
\par 12 La râu, pe amândouă laturile lui, vor crește tot felul de arbori care dau hrană. Frunzele lor nu se vor veșteji și fructele din ei nu se vor mai isprăvi. n fiecare lună se vor coace fructe noi, pentru că apa pentru ele vine din locul cel sfânt; fructele lor se vor întrebuința ca hrană, iar frunzele la leacuri".
\par 13 Așa zice Domnul Dumnezeu: "Iată hotarele pământului pe care-l veți împărți ca moștenire celor douăsprezece seminții ale lui Israel: Iosif va avea două părți.
\par 14 Și veți stăpâni din el toți deopotrivă, pentru că, ridicându-Mi mâna, M-am jurat să-l dau părinților voștri, de aceea va fi pământul acesta moștenirea voastră.
\par 15 Iată care sunt hotarele țării acesteia: În partea de miazănoapte de la Marea cea Mare, drumul Hetlonului până la intrarea Hamatului: Țedad,
\par 16 Berot, Sibraim care e între hotarul Damascului și al Hamatului, Hațer-Hațicon, spre hotarul Hauranului.
\par 17 Și va fi hotarul de la mare până la Hațar-Enon în hotarul Damascului, având la miazănoapte ținutul Hamat. Aceasta este latura de miazănoapte.
\par 18 Hotarul de răsărit să-l trageți printre Hauran și Damasc, printre Galaad și țara lui Israel, pe Iordan, de la hotarul de miazănoapte până la marea de răsărit spre Tamar. Aceasta este latura de răsărit.
\par 19 Iar latura de miazăzi începe spre miazăzi de la Tamar și merge până la apele Meriba la Cadeș, apoi urmează cursul râului până la Marea cea Mare. Aceasta este latura de miazăzi.
\par 20 Iar hotarul de la apus este Marea cea Mare, de la hotarul de miazăzi până în dreptul Hamatului, Aceasta este latura de la apus.
\par 21 Și să vă împărțiți între voi țara aceasta, după semințiile lui Israel.
\par 22 Și s-o împărțiți prin sorți ca moștenire a voastră și a străinilor care vor trăi la voi, care au născut copii între voi; și ei trebuie să se socotească între fiii lui Israel deopotrivă cu locuitorii băștinași. Cu voi să primească și ei la sorți moștenirea lor intre semințiile lui Israel.
\par 23 În care seminție va trăi străinul, în aceea i se va da și moștenire", zice Domnul Dumnezeu.

\chapter{48}

\par 1 Iată numele semințiilor: la capătul de miazănoapte al țării, pe calea care duce de la Hetlon la Hamat, Hațar-Enon la hotarul Damascului, pe hotarul de miazănoapte, care este spre Hamat, pe toată întinderea aceasta de la răsărit până la mare este o singură parte, a lui Dan.
\par 2 Lângă hotarul lui Dan, de la hotarul de răsărit până la cel de apus, este partea lui Așer.
\par 3 Lângă hotarul lui Așer, de la hotarul de răsărit până la cel de apus, este partea lui Neftali.
\par 4 Lângă hotarul lui Neftali, de la hotarul de răsărit până la cel de apus, este partea lui Manase.
\par 5 Lângă hotarul lui Manase, de la hotarul de răsărit până la cel de la apus, este partea lui Efraim.
\par 6 Lângă hotarul lui Efraim, de la hotarul de răsărit până la cel de apus, este partea lui Ruben.
\par 7 Lângă hotarul lui Ruben, de la hotarul de răsărit până la cel de apus, este partea lui Iuda.
\par 8 Iar lângă hotarul lui Iuda, de la hotarul de răsărit până la cel de apus, este partea sfântă, în lățime de douăzeci și cinci de mii de coți, iar în lungime deopotrivă cu celelalte părți, de la marginea de răsărit ța țării până la cea de apus. În mijlocul acestei părți va fi templul.
\par 9 Partea pe care o veți osebi Domnului va fi lungă de douăzeci și cinci de mii de coți, iar lățimea de zece mii de coți.
\par 10 Și această parte sfântă trebuie să fie a preoților: la miazănoapte douăzeci și cinci de mii de coți, la apus în lățime zece mii de coți, spre răsărit în lățime zece mii și spre miazăzi în lungime douăzeci și cinci de mii, iar la mijloc va fi locașul sfânt al Domnului.
\par 11 Aceasta să o afierosiți preoților din fiii lui Sadoc, care au stat în slujba Mea și care, în vremea rătăcirii fiilor lui Israel, nu s-au abătut de la Mine, cum s-au abătut ceilalți leviți.
\par 12 Această parte sfântă de pământ, va fi a lor, sfințenie mare, la hotarul leviților.
\par 13 Leviții vor primi lângă partea preoților o parte de douăzeci și cinci de mii de coți în lungime și zece mii de coți în lățime: toată partea douăzeci și cinci de mii în lungime și zece mii în lățime.
\par 14 Și din această parte ei nu pot nici să vândă, nici să schimbe; pârga pământului nu poate fi trecută altora, pentru că ea este lucru sfânt, al Domnului.
\par 15 Iar următoarele cinci mii în lățime și douăzeci și cinci de mii în lungime se dau pentru cetate, pentru întrebuințarea obștească, pentru locuințe și pentru piețe; cetatea va fi în mijloc.
\par 16 Iată măsurile ei: latura de miazănoapte patru mii cinci sute de coți, latura de miazăzi patru mii cinci sute de coți, latura de răsărit patru mii cinci sute de coți și latura de la apus de patru mii cinci sute de coți.
\par 17 Iar pășunea împrejurul cetății va fi două sute cincizeci de coți la miază-noapte, la răsărit două sute cincizeci, la miazăzi două sute cincizeci și la apus două sute cincizeci.
\par 18 Iar ce rămâne din lungime în linie cu porțiunea sfântă - zece mii spre răsărit și zece mii spre apus - va forma un venit pentru hrana muncitorilor din cetate.
\par 19 În cetate pot să lucreze lucrători din toate semințiile lui Israel.
\par 20 Toată partea aceasta, un pătrat de douăzeci și cinci de mii de coți în lungime și douăzeci și cinci de mii de coți în lățime, va fi partea sfântă și va cuprinde și pământul cetății.
\par 21 Ce va rămâne va fi al regelui de o parte și de alta a porțiunii sfinte și a porțiunii cetății, ce va trece peste cele douăzeci și cinci de mii de coli spre răsărit până la hotarul țării și ce va trece peste cele douăzeci și cinci de mii de coți spre apus până la hotarul țării. Aceasta va fi partea regelui, așa încât partea sfântă și locașul sfânt vor fi la mijloc.
\par 22 Și ceea ce rămâne de la stăpânirea leviților și de la stăpânirea orașului, la mijloc, între hotarul lui Iuda și hotarul lui Veniamin, va fi al regelui.
\par 23 Iată acum celelalte seminții: alături de partea leviților, de la hotarul de răsărit al țării până la cel de apus, este partea lui Veniamin.
\par 24 Lângă hotarul de miazăzi al lui Veniamin, de la hotarul de răsărit al țării până la cel de apus, este partea lui Simeon.
\par 25 Lângă hotarul de miazăzi al lui Simeon, de la hotarul de răsărit al țării până la cel de apus, este partea lui Isahar.
\par 26 Lângă hotarul de miazăzi al lui Isahar, de la hotarul de răsărit al țării până la cel de apus, este partea lui Zabulon.
\par 27 Lângă hotarul de miazăzi al lui Zabulon, de la hotarul de răsărit al țării până la cel de apus, este partea lui Gad.
\par 28 Iar pe hotarul de miazăzi al lui Gad merge și hotarul de miazăzi al jării de la Tamar spre apele Meriba de la Cadeș, de-a lungul râului acestuia până la Marea cea Mare.
\par 29 Aceasta este țara pe care o veți împărți prin sorți semințiilor lui Israel, și acestea sunt părțile semințiilor, zice Domnul Dumnezeu.
\par 30 Iată acum care sunt porțile cetății. Porțile cetății vor purta numele semințiilor lui Israel.
\par 31 Latura de miazănoapte va avea patru mii cinci sute de coli și va avea trei porți: una, poarta lui Ruben, una, poarta lui Iuda și una, poarta lui Levi.
\par 32 Latura de răsărit va avea patru mii cinci sute de coți și trei porți: una, poarta lui Iosif, una, poarta lui Veniamin și una, poarta lui Dan.
\par 33 Latura de miazăzi va avea patru mii cinci sute de coți și trei porți: una, poarta lui Simeon, una, poarta lui Isahar și una, poarta lui Zabulon.
\par 34 Și latura dinspre mare va avea patru mii cinci sute de coli și trei porți: una, poarta lui Gad, una, poarta lui Așer și una, poarta lui Neftali.
\par 35 De jur împrejurul cetății sunt optsprezece mii de coți. Iar din ziua aceea înainte numele cetății va fi: Domnul este acolo.


\end{document}