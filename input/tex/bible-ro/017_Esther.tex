\begin{document}

\title{Estera}


\chapter{1}

\par 1 În anul al doilea al domniei lui Artaxerxe cel Mare, în ziua întâi a lunii Nisan, Mardoheu, fiul lui Iair, fiul lui Șimei, fiul lui Chiș, iudeu din seminția lui Veniamin, om mare, care trăia în cetatea Suza și slujea la curtea regelui, a visat un vis. Mardoheu era din robii pe care Nabucodonosor, regele Babilonului, îi luase în robie din Ierusalim cu Iehonia, regele Iudei. Iar visul lui a fost acesta: "S-a făcut parcă zgomot mare pe pământ, tunet, cutremur și tulburare. Și deodată au ieșit doi balauri mari, gata să se lupte unul cu altul, și urletul lor era cumplit. După urletul lor, toate popoarele s-au pregătit de război, ca să lovească pe poporul drepților. Și iată a venit pe pământ zi de negură și întuneric, de necaz și strâmtorare, de mare durere și tulburare. Atunci tot poporul drepților s-a tulburat, temându-se de răul lor; s-a pregătit să piară și a început să strige către Domnul. La strigătul lor, a ieșit parcă dintr-un izvor mic un râu mare cu apă multă, a strălucit lumină și soare, s-au înălțat cei smeriți și au nimicit pe trufași". Deșteptându-se Mardoheu după acest vis, care închipuia ce avea să facă Dumnezeu, păstra acest vis în inima sa și dorea să-l priceapă în toată întregimea lui până în seară. Și a rămas Mardoheu la curte cu Gavata și cu Tara, doi eunuci ai regelui care păzeau curtea; atunci a auzit ce vorbeau ei, a aflat planul și a descoperit că ei se pregăteau să pună mâna pe regele Artaxerxe și i-a spus regelui. Regele a cercetat pe cei doi eunuci și după ce ei au mărturisit, au fost spânzurați. Regele a scris întâmplarea aceasta spre aducere aminte și a scris-o și Mardoheu. Atunci a poruncit regele lui Mardoheu să slujească la curte și i-a dat și daruri pentru aceasta. Dar pe lângă rege era atunci vestit Aman. fiul lui Hamadata din țara Agag. Acesta se silea să facă rău lui Mardoheu și poporului său pentru cei doi eunuci ai regelui. După întâmplările acestea, în zilele lui Artaxerxe, care domnea peste o sută douăzeci și șapte de țări, de la India până în Etiopia,
\par 2 În vremea când regele Artaxerxe își avea scaunul său domnesc în cetatea Suza,
\par 3 În anul al treilea al domniei lui, a dat el ospăț pentru dregătorii săi și pentru cei ce-i slujeau lui, pentru căpeteniile mai înalte ale oștirii Persiei și Mediei și pentru guvernatorii țărilor sale,
\par 4 Arătându-și marea bogăție a regatului său și strălucirea deosebită a măririi sale, în curgere de mai multă vreme, anume timp de o sută optzeci de zile.
\par 5 După sfârșitul acestor zile, regele a făcut și pentru poporul său care se afla în capitala Suza, de la mic până la mare, ospăț de șapte zile, în grădina curții sale domnești,
\par 6 Împodobită cu covoare de mătase albă și de purpură violetă, atârnate pe frânghii de in și de mătase, trecute prin verigi de argint, întărite în stâlpi de marmură.
\par 7 Divanuri de aur și de argint erau așezate pe pardoseală de porfiră, de marmură albă, de sidef și marmură neagră.
\par 8 Băuturile se turnau în vase de aur și în cupe felurite, în valoare de treizeci de mii de talanți, iar vin din care bea însuși regele a fost mult, după bogăția și dărnicia regelui. Băutura se consuma cuviincios și fără silă, că regele poruncise tuturor cârmuitorilor din casa sa să facă fiecăruia după voia lui.
\par 9 Regina Vasti a făcut și ea ospăț, pentru femei, în casa domnească a regelui Artaxerxe.
\par 10 În ziua a șaptea, când inima regelui s-a înveselit de vin, acesta a zis către Mehuman, Bizeta, Harbona, Bigta, Abgata, Zetar și Carcas, cei șapte eunuci care slujeau înaintea feței regelui Artaxerxe,
\par 11 Să aducă pe regina Vasti înaintea feței regelui cu coroana regească pe cap, ca să arate popoarelor și căpeteniilor frumusețea ei, pentru că era foarte frumoasă.
\par 12 Dar regina Vasti n-a voit să vină după porunca regelui ce i s-a trimis prin eunuci.
\par 13 Din pricina aceasta s-a mâniat regele tare și s-a aprins într-însul urgie mare și a zis către înțelepții care știau datinile din trecut, (căci așa era rânduiala ca orice pricină a regelui să se spună înaintea celor ce cunoșteau legile și dreptul):
\par 14 "Cum să se aplice legea față de regina Vasti, pentru că ea nu a împlinit porunca regelui Artaxerxe, ce i s-a trimis prin eunuci?"
\par 15 Și erau în apropierea regelui: Carșena, Șetar, Admata, Tarșiș, Mereș, Marsena și Memucan, șapte căpetenii ale Mediei și Persiei, care puteau vedea fața regelui și țineau locurile cele dintâi în regat.
\par 16 Atunci a răspuns Memucan înaintea regelui și a căpeteniilor: "Regina Vasti s-a făcut vinovată nu numai înaintea regelui, ci și înaintea tuturor dregătorilor și înaintea tuturor popoarelor care sânt în toate țările regelui Artaxerxe,
\par 17 Pentru că purtarea reginei Vasti va ajunge la știrea tuturor femeilor, și vor nesocoti și acelea pe bărbații lor și vor zice: Regele Artaxerxe a poruncit să fie adusă regina Vasti înaintea feței sale și ea n-a venit.
\par 18 De acum soțiile dregătorilor din Persia și Media, care vor auzi de purtarea reginei, vor zice la fel dregătorilor regelui, și va fi destul dispreț și mânie.
\par 19 Deci, dacă binevoiește regele, să iasă de la el hotărâre regească și să se scrie în legile Persiei și Mediei, cu neschimbare, că regina Vasti nu va mai intra înaintea regelui Artaxerxe, iar vrednicia de regină a ei regele o va da alteia mai vrednică decât ea.
\par 20 Când se va auzi de această hotărâre a regelui, care se va împrăștia în toată împărăția lui cât este ea de mare, atunci toate femeile își vor cinsti bărbații de la mic până la mare".
\par 21 Și cuvântul a plăcut regelui și dregătorilor, și a făcut după sfatul lui Memucan.
\par 22 Regele a trimis în toate țările scrisori, scrise pentru fiecare țară cu scrisul ei și pentru fiecare popor în limba lui, ca fiecare bărbat să fie stăpân în casa sa. Aceasta s-a adus la cunoștința fiecăruia în limba părintească a fiecăruia.

\chapter{2}

\par 1 După întâmplarea aceasta, când s-a potolit mânia regelui Artaxerxe, și-a adus el aminte de regina Vasti, de ceea ce făcuse ea și de ce se hotărâse împotriva ei,
\par 2 Iar oamenii regelui, care-i slujeau lui, au zis: "Să se caute pentru rege fete tinere și frumoase.
\par 3 Să rânduiască dar regele dregători în toate țările regatului său, care să adune toate fetele tinere și frumoase la față în capitala Suza, în casa femeilor, sub supravegherea lui Hegai, eunucul regelui, păzitorul femeilor, și să le dea săpun și celelalte de trebuință pentru spălat.
\par 4 Și fata care va plăcea ochilor regelui să fie regină în locul Vastei". Cuvântul acesta a plăcut regelui, și el a făcut așa.
\par 5 În vremea aceea era în capitala Suza un iudeu cu numele Mardoheu, fiul lui Iair, fiul lui Șimei, fiul lui Chiș, din seminția lui Veniamin.
\par 6 Acesta fusese adus din Ierusalim împreună cu robii luați cu Iehonia, regele lui Iuda, pe care-i luase Nabucodonosor, regele Babilonului.
\par 7 El creștea pe Hadasa, adică pe Estera, fiica unchiului său, deoarece ea nu avea nici tată, nici mamă. Fata aceasta era mândră la înfățișare și frumoasă la chip și, după moartea tatălui ei și a mamei ei, o luase Mardoheu ca fiică.
\par 8 Când s-a făcut cunoscută porunca regelui și hotărârea lui și când au fost adunate multe fete în capitala Suza, sub supravegherea lui Hegai, atunci a fost luată și Estera în casa regelui sub supravegherea lui Hegai, păzitorul femeilor.
\par 9 Fata aceasta a plăcut ochilor lui și și-a atras bunăvoința lui. De aceea s-a grăbit el să-i dea cele de spălat și tot ce i se cuvenea și a pus pe lângă ea șapte fete vrednice de a fi cu ea și au așezat-o pe ea și fetele ei în cea mai bună încăpere din casa femeilor.
\par 10 Estera însă n-a spus nimic nici de poporul său, nici de neamurile sale, pentru că Mardoheu îi dăduse poruncă să nu spună.
\par 11 Mardoheu venea în fiecare zi în curtea casei femeilor, ca să afle știri despre sănătatea Esterei și cele ce se petreceau cu ea.
\par 12 Când venea fiecărei fete vremea să intre la regele Artaxerxe, după ce timp de douăsprezece luni se săvârșeau asupra ei cele rânduite pentru femei, (căci atât timp ținea vremea curățirii lor: șase luni cu miruri și șase luni cu aromate și alte unsori femeiești),
\par 13 Atunci fata intra la rege și orice ar fi cerut, i se dădea pentru a merge din casa femeilor în casa regelui.
\par 14 Seara ea intra și dimineața se ducea în a doua casă a femeilor sub supravegherea lui Șaașgaz, eunuc al regelui, păzitorul concubinelor, și nu mai intra după aceea la rege decât doar dacă ar fi voit-o regele și ar fi fost chemată anume.
\par 15 Când a venit vremea Esterei, fiica lui Abihail, unchiul lui Mardoheu, care o luase la sine ca fiică, să meargă la rege, atunci ea nu a cerut nimic, decât numai ceea ce-i zisese Hegai, eunucul regelui, păzitorul femeilor. Ea a aflat trecere înaintea tuturor celor ce o vedeau.
\par 16 Estera a fost luată la regele Artaxerxe și dusă în casa lui domnească în luna a zecea, adică în luna Tebet, în al șaptelea an al domniei lui.
\par 17 Și a iubit regele pe Estera mai mult decât pe toate femeile, și ea a dobândit bunăvoința lui și trecere mai mult decât toate femeile, și el a pus coroana domnească pe capul ei și a făcut-o regină în locul Vastei.
\par 18 Apoi a făcut regele ospăț mare pentru toți dregătorii săi și pentru cei ce-i slujeau lui. Acesta a fost ospăț pentru Estera; regele a mai făcut mari ușurări țărilor și a împărțit daruri cu dărnicie de rege.
\par 19 Iar când s-au adunat fetele a doua oară și Mardoheu ședea la poarta regelui,
\par 20 Estera nu spusese nimic despre neamul său, așa cum îi poruncise Mardoheu, pentru că Estera împlinea cuvântul lui Mardoheu, ca și când era sub ocrotirea lui.
\par 21 Într-o zi, când Mardoheu ședea la poarta regelui, doi eunuci ai regelui, Bigtan și Tereș, căpeteniile paznicilor lui, s-au aprins de mânie, că se dăduse întâietate lui Mardoheu, și căutau să pună mâna pe regele Artaxerxe.
\par 22 Aflând acest lucru, Mardoheu i-a spus reginei Estera, iar Estera i-a spus regelui în numele lui Mardoheu.
\par 23 Fapta s-a cercetat și s-a găsit adevărată. Amândoi eunucii au fost spânzurați în curte. Această binefacere a lui Mardoheu a fost scrisă în cartea amintirilor zilnice ale regelui.

\chapter{3}

\par 1 După aceea a ridicat regele Artaxerxe pe Aman, fiul lui Hamadata, din țara Agag, și a pus scaunul lui mai presus decât al tuturor căpeteniilor pe care le avea.
\par 2 Toți cei ce slujeau regelui, care erau la poarta regelui, se închinau și se aruncau cu fețele la pământ înaintea lui Aman, căci așa poruncise regele; iar Mardoheu nu se închina și nu-și pleca fața la pământ.
\par 3 De aceea cei ce slujeau regelui și care erau la poartă, ziceau lui Mardoheu: "De ce calci tu porunca regelui?"
\par 4 Și cum ei ziceau aceasta în fiecare zi, iar el nu-i asculta, i-au spus lui Aman, ca să vadă de va stărui Mardoheu în purtarea sa, căci el le spusese că este iudeu.
\par 5 Când a văzut Aman că Mardoheu nu se închină și nu-și pleacă fruntea până la pământ înaintea sa, s-a umplut de mânie.
\par 6 Dar i s-a părut prea puțin lucru să pună mâna numai pe Mardoheu. Și fiindcă i se spusese din ce popor e Mardoheu, Aman și-a pus în gând să ucidă pe toți Iudeii care se aflau în tot regatul lui Artaxerxe, întrucât erau poporul lui Mardoheu.
\par 7 Au făcut deci sfat în luna întâi, adică în luna Nisan, în anul al doisprezecelea al domniei lui Artaxerxe, și au aruncat Pur, adică sorți, de față cu Aman, ca să vadă în ce lună și în ce zi să fie ucis dintr-o dată poporul lui Mardoheu, și a căzut sorțul pe luna a douăsprezecea, adică pe luna lui Adar.
\par 8 În vremea aceasta a spus Aman regelui Artaxerxe: "Este un popor risipit și împrăștiat printre popoare, prin toate țările regatului tău. Legile lui sânt deosebite de legile tuturor popoarelor, legilor regelui nu se supun și regele nu se cuvine să-l lase așa.
\par 9 Dacă binevoiește regele, atunci să se hotărască în scris să fie uciși și eu voi cântări zece mii de talanți de argint și voi da în mâna vistiernicilor, ca să-i verse în vistieria regelui".
\par 10 Atunci și-a scos regele inelul său din mâna sa și l-a dat lui Aman, fiul lui Hamadata, din țara Agag, ca să întărească decretul cel împotriva Iudeilor,
\par 11 Zicând lui Aman: "Îți dau ție acest argint și poporul; fă cu el ce vrei!"
\par 12 Atunci au fost chemați scriitorii regelui în a treisprezecea zi a lunii întâi, și s-a scris, cum poruncise Aman, către satrapii regatului și către căpetenia fiecărei din cele o sută douăzeci și șapte de țări, de la țările Indiei și până la Etiopia, și către căpeteniile fiecărui popor; și s-a scris fiecărei țări cu scrierea ei și fiecărui popor în limba lui și toate s-au scris în numele regelui Artaxerxe și au fost întărite cu inelul lui.
\par 13 Scrisorile s-au trimis prin ștafete în toate țările regelui, ca să ucidă, să piardă și să nimicească pe toți Iudeii, mic și mare, copii și femei, într-o singură zi, și anume în a treisprezecea zi a lunii a douăsprezecea, adică în luna Adar, iar averile lor să le jefuiască. Iată cuprinsul acelei scrisori: "Marele rege Artaxerxe, celor ce cârmuiesc de la India până la Etiopia peste o sută douăzeci și șapte de țări și căpeteniilor și slujitorilor de sub conducerea lor acestea scrie: Domnind peste multe popoare și stăpânind toată lumea, eu, fără să fiu îngâmfat de putere, ci cârmuind pururea cu blândețe și cu liniște, am voit să fac viața supușilor pururea netulburată, păzind regatul meu în pace și ușor de străbătut până la hotarele lui, statornicind pacea de toți dorită. Dar când eu am întrebat pe sfetnici, cum am putea aduce aceasta la îndeplinire, atunci Aman, care e vestit la noi prin înțelepciune și se bucură neclintit de bunăvoința noastră și care a dovedit cea mai deplină credincioșie, pentru care a dobândit cinstea de a ședea în al doilea loc după rege, ne-a arătat că prin toate neamurile lumii s-a amestecat un popor vrăjmaș, potrivnic legilor tuturor popoarelor, care necontenit nesocotește poruncile regelui, ca să nu se poată întemeia cârmuirea noastră fără meteahnă. Aflând deci că numai singur acest popor se împotrivește pururea oricărui om, că duce un fel de viață străină de legi și, împotrivindu-se pururea lucrărilor noastre, face cele mai mari nelegiuiri, ca regatul nostru să nu ajungă a fi bine întocmit, am poruncit cele arătate vouă în scrisorile lui Aman, care este pus de noi peste lucruri și ca un al doilea părinte al nostru, ca să-i stârpiți pe toți, cu femei și cu copii, prin sabie cumplită, fără nici o milă și cruțare, în treisprezece ale lunii a douăsprezecea, adică în luna lui Adar, a acestui an, ca astfel acești oameni vrăjmași și astăzi, ca și în trecut, fiind într-o singură zi aruncați cu sila în iad, să nu ne mai împiedice în viitor de a duce viață pașnică și netulburată până în sfârșit.
\par 14 Cuprinsul acestei porunci să se dea în fiecare țară, ca lege cu putere pentru toate popoarele, ca ele să fie gata pentru ziua aceea".
\par 15 Ștafetele au zburat repede cu porunca regelui. Porunca s-a făcut cunoscută și în capitala Suza. Regele și Aman ședeau și beau, iar cetatea Suza era în fierbere.

\chapter{4}

\par 1 Când Mardoheu a aflat tot ce se făcuse, și-a rupt hainele sale, și-a pus pe sine sac și cenușă și, ieșind în mijlocul cetății, a ridicat strigăt amarnic.
\par 2 Ajungând însă până la poarta regelui, s-a oprit, deoarece nu putea să intre pe poarta regelui îmbrăcat cu sac și cu cenușă pe cap.
\par 3 Tot așa și în fiecare țară și loc, unde ajungea porunca regelui și scrisoarea lui, era tânguire mare printre Iudei, post, plângere și bocet. Sacul și cenușa erau așternutul multora.
\par 4 Atunci au venit slujitorii Esterei, eunucii și slujnicele ei și i-au spus; și regina s-a tulburat strașnic. Apoi a trimis haine lui Mardoheu, ca să se îmbrace cu ele și să lepede de pe el sacul. Dar el n-a voit.
\par 5 De aceea a chemat Estera pe Hatac, unul din eunucii regelui pe care acesta îl pusese să fie pe lângă ea, și l-a trimis la Mardoheu, ca să afle de ce și pentru ce sânt acestea.
\par 6 Mergând Hatac la Mardoheu, în piața orașului cea din fața curții regelui,
\par 7 I-a spus Mardoheu toate câte se întâmplase și de numărul arginților ce făgăduise Aman să verse în vistieria domnească pentru Iudei ca să-i stârpească.
\par 8 Îi dădu de asemenea și o copie de pe porunca domnească, cuprinsă în hrisovul dat în Suza pentru stârpirea lor, ca să-l arate Esterei și să-i dea de veste despre toate. Pe lângă aceasta o sfătuia să meargă la rege și să-l roage de iertare pentru poporul ei, aducându-și aminte de zilele sale cele smerite, când era crescută sub mâna lui, a lui Mardoheu, pentru că Aman, cel al doilea după rege, a osândit pe Iudei la moarte; să strige de asemenea către Domnul, ca să-i izbăvească pe ei de la moarte.
\par 9 Deci a venit Hatac și a spus Esterei vorbele lui Mardoheu.
\par 10 Atunci a vorbit Estera cu Hatac și l-a trimis să spună lui Mardoheu:
\par 11 "Toți cei ce slujesc regelui și popoarele din țările regelui știu că tot cel ce va intra la rege înăuntrul curții, bărbat sau femeie, fără să fie chemat, ia o singură osândă: moartea. Numai acela spre care va întinde regele sceptrul său de aur scapă cu viață. Eu însă n-am fost chemată la rege de mai bine de treizeci de zile".
\par 12 Aceste cuvinte ale Esterei au fost spuse lui Mardoheu.
\par 13 Iar Mardoheu a răspuns Esterei următoarele: "Să nu socoți că ai să scapi tu singură în casa regelui, dintre toți Iudeii.
\par 14 Dacă tu vei tăcea în vremea aceasta, atunci izbăvirea și eliberarea vor veni pentru Iudei din altă parte, iar tu și casa tatălui tău veți pieri. Și cine știe dacă tu n-ai ajuns la vrednicia de regină tocmai pentru vremile acestea?"
\par 15 Atunci Estera a răspuns lui Mardoheu:
\par 16 "Mergi, adună pe toți Iudeii din Suza și postiți pentru mine; să nu mâncați și să nu beți trei zile, nici ziua, nici noaptea și voi posti și eu cu slujnicile mele și apoi mă voi duce la rege, deși aceasta este împotriva legii și de va fi să pier, voi pieri".
\par 17 Atunci s-a dus Mardoheu și a făcut cum îi poruncise Estera. El s-a rugat Domnului, pomenind toate lucrurile Domnului și zicând: "Doamne, Doamne, Împărate atotțiitorule, toți sunt în puterea Ta și nu este cine să se împotrivească Ție când vei voi să izbăvești pe Israel. Tu ai făcut cerul și pământul și toate cele minunate de sub cer. Tu ești Domnul tuturor și nu este cine să se împotrivească Ție, Doamne! Tu toate le știi Doamne, Tu știi că eu nu din mândrie, nici din trufie, nici ca să jignesc nu m-am închinat lui Aman cel mândru, căci eu cu plăcere m-aș fi apucat să sărut tălpile picioarelor lui pentru izbăvirea lui Israel. Dar eu am făcut aceasta ca să nu dau slavă oamenilor mai presus de slava lui Dumnezeu și nu m-am închinat nimănui, decât numai Ție, Domnului meu și nici nu voi face aceasta din mândrie. Și acum, Doamne Dumnezeule, Împărate, Dumnezeul lui Avraam, cruță pe poporul Tău, căci se pune la cale pieirea noastră și voiesc să piardă moștenirea Ta cea dintru început. Nu trece cu vederea partea Ta pe care ai răscumpărat-o pentru Tine din țara Egiptului. Auzi rugăciunea mea și Te milostivește spre moștenirea Ta; întoarce plânsul nostru în veselie ca, vii fiind noi, să lăudăm numele Tău, Doamne, și nu astupa gura celor ce Te preamăresc pe Tine". Și toți Israeliții au strigat din toate puterile lor că moartea era înaintea ochilor lor. Și a alergat și Estera la Domnul, cuprinsă de groaza morții și, dezbrăcându-se de hainele slavei sale, s-a îmbrăcat în haine de deznădejde și de jale, iar în locul unsorilor celor scumpe, cu cenușă și cu țărână și-a presărat capul său; și-a smerit cumplit trupul său și tot locul împodobit altădată l-a umplut de păr smuls din capul său și, rugându-se Domnului, Dumnezeului lui Israel, a zis: "Domnul meu, numai Tu singur ești Împăratul nostru; ajută-mi mie celei singuratice și fără ajutor afară de Tine, că pieirea mea e aproape! Eu am auzit, Doamne, de la tatăl meu, în neamul meu părintesc, că Tu ți-ai ales pe Israel din toate popoarele și pe părinții noștri din toți strămoșii lor, ca să fie moștenirea Ta veșnică și ai făcut pentru ei toate câte ai zis. Acum noi am greșit înaintea Ta și Tu ne-ai dat în mâinile vrăjmașilor noștri, pentru că am lăudat pe dumnezeii lor; drept ești Tu, Doamne! Dar ei acum nu se mai mulțumesc cu robia noastră amară, ci și-au dat mâna cu idolii lor, ca să răstoarne poruncile gurii Tale și să stârpească moștenirea Ta și să astupe gura celor ce Te slăvesc pe Tine și să stingă slava casei Tale și jertfelnicul Tău, să dezlege gura popoarelor pentru a preaslăvi pe dumnezeii lor cei mincinoși și pentru ca regele cel pământesc să fie admirat totdeauna. Nu da, Doamne, sceptrul Tău dumnezeilor celor ce nu sânt, ca să nu se bucure vrăjmașii de căderea noastră, ci întoarce uneltirea lor asupra lor înșiși, iar pe uneltitorul împotriva noastră dă-l rușinii. Adu-ți aminte, Doamne, arată-Te nouă în vremea necazului nostru și-mi dă mie curaj. Împărate al dumnezeilor și Stăpâne a toată stăpânirea, dăruiește gurii mele cuvânt cu trecere înaintea leului acestuia și umple inima lui de ură către cel ce ne prigonește pe noi, spre pieirea lui și a celor de un gând cu el. Iar pe noi ne izbăvește cu mâna Ta și-mi ajută mie celei singure, care nu am alt ajutor decât pe Tine, Doamne; Tu ai știință de toate și cunoști că eu urăsc slava celor fără de lege și mi-e silă de patul celor netăiați împrejur și de tot cel de alt neam. De asemenea cunoști nevoia mea, că nu pot suferi semnul mândriei mele care se află pe capul meu în zilele când mă arăt, mi-e silă de el, ca de o haină întinată cu sânge și nici nu-l port când sânt singură. Roaba Ta n-a mâncat din masa lui Aman, nici n-a prețuit ospățul regesc; vin jertfit la idoli n-am băut, nici nu s-a veselit roaba Ta din vremea schimbării soartei mele și până acum decât numai de Tine, Doamne Dumnezeul lui Avraam. Dumnezeule, Cel ce ai putere peste toate, auzi glasul celor fără de nădejde și ne izbăvește din mâinile uneltitorilor de rele, scăpându-mă din frica mea".

\chapter{5}

\par 1 După trei zile de rugăciune Estera și-a dezbrăcat hainele cele de jale și s-a îmbrăcat în cele de regină și, făcându-se strălucită și chemând pe Dumnezeul cel atoatevăzător și mântuitor, a luat cu ea două slujnice: una de care se sprijinea oarecum din alintare, iar alta care, urmându-i, îi ținea hainele. Ea era minunată, în culmea frumuseții sale și fața sa era veselă ca și cum ar fi fost plină de iubire, iar inima ei era apăsată de frică. S-a oprit în curtea dinăuntru a casei regelui, la ușa regelui. Regele ședea atunci pe tronul său domnesc, în casa domnească, chiar în dreptul ușii, îmbrăcat în toate hainele măririi sale, tot în aur și în pietre scumpe, dar cumplit de posomorât. Când regele a văzut pe regina Estera stând afară, aceasta a aflat milă în ochii lui. Întorcându-și fața înflăcărată de slavă, el a privit cu mânie strașnică. Atunci regina a căzut cu duhul, s-a schimbat la față din pricina slăbiciunii ce i-a venit și s-a aplecat pe capul slujnicei care o însoțea. Dar Dumnezeu a schimbat duhul regelui în blândețe și, sculându-se cu grăbire de pe tronul său, a cuprins-o în brațele sale, până ea și-a venit în fire. Apoi a mângâiat-o cu vorbe bune, zicându-i: "Ce ai Estera? Eu sânt fratele tău! Liniștește-te, că nu vei muri, căci stăpânirea ne este comună. Apropie-te!"
\par 2 Apoi regele și-a întins spre Estera sceptrul său cel de aur, care era în mâna sa. Atunci Estera s-a apropiat și s-a atins de vârful sceptrului, iar regele i-a pus sceptrul pe grumazul ei și a sărutat-o, zicându-i: "Vorbește-mi!" Și ea a zis: "Stăpânul meu, eu am văzut în tine parcă pe îngerul lui Dumnezeu și s-a tulburat inima mea de frică în fața slavei tale, că minunat ești stăpâne și fața ta este plină de har". Și când vorbea ea, a căzut din pricina slăbiciunii și regele s-a tulburat și toate slugile lui o mângâiau.
\par 3 Apoi regele i-a zis: "Ce voiești, regină Estera și care-ți este cererea? Chiar și jumătate din regat ți se va da".
\par 4 Și a zis Estera: "Eu am acum zi de sărbătoare. De binevoiește regele, să vină cu Aman astăzi la ospățul pe care i l-am pregătit eu".
\par 5 Iar regele a zis: "Mergi degrabă după Aman, ca să se facă după cuvântul Esterei". Și a venit regele cu Aman la ospățul ce-l pregătise Estera.
\par 6 La băutură regele a zis către Estera: "Ce dorință ai? Ea ți se va îndeplini. Care este cererea ta? Ea ți se va împlini chiar și până la jumătate din regatul meu!"
\par 7 Atunci Estera a răspuns și a zis: "Iată dorința și cererea mea:
\par 8 De am aflat bunăvoință în ochii regelui și de binevoiește el să împlinească dorința mea și să-mi împlinească cererea, atunci să vină regele cu Aman mâine la ospățul ce-l voi pregăti, și eu voi da răspuns după porunca regelui".
\par 9 În ziua aceea a ieșit Aman vesel și voios. Dar când Aman a văzut pe Mardoheu la poarta regelui și că acesta nu s-a sculat măcar de la locul său înaintea lui, s-a umplut de mânie asupra lui Mardoheu.
\par 10 Cu toate acestea Aman s-a stăpânit. Ajungând acasă, a trimis să cheme pe prietenii săi și pe Zereș, femeia sa.
\par 11 Și le-a povestit Aman despre bogăția sa cea mare, despre mulțimea fiilor săi și despre felul cum l-a mărit pe el regele și cum l-a înălțat peste căpeteniile și slujitorii săi.
\par 12 "Ba și regina Estera, a zis mai departe Aman, pe nimeni afară de mine n-a chemat cu regele la ospățul pe care l-a pregătit. Chiar și mâine sânt chemat la ea la ospăț.
\par 13 Dar toate acestea nu mă mulțumesc câtă vreme văd pe iudeul Mardoheu șezând la poarta domnească".
\par 14 La acestea Zereș, soția sa, și prietenii săi i-au zis: "Să se pregătească spânzurătoare înaltă de cincizeci de coți și mâine dimineață cere regelui să fie spânzurat Mardoheu și apoi vei merge voios la ospăț cu regele". Acest cuvânt a plăcut lui Aman și a pus să se pregătească spânzurătoarea.

\chapter{6}

\par 1 În noaptea aceea Domnul a depărtat somnul de la rege și acesta a poruncit slugii să-i aducă Cronica însemnărilor zilnice și ele au fost citite înaintea regelui.
\par 2 Acolo se afla scris ce descoperise Mardoheu regelui cu privire la Bigtan și Tereș, cei doi eunuci ai regelui, păzitorii pragului, care uneltiseră să pună mâna pe regele Artaxerxe.
\par 3 Atunci a zis regele: "Ce cinste și răsplată s-a dat lui Mardoheu pentru aceasta?" Iar oamenii regelui, care-i slujeau, au zis: "Nu i s-a făcut nimic".
\par 4 Când întreba regele de binefacerile lui Mardoheu, a venit Aman în curte, iar regele a zis: "Cine este în curte?" Aman însă venise în curtea de afară a casei regelui să vorbească cu regele, ca să fie spânzurat Mardoheu în spânzurătoarea pe care i-o pregătise el.
\par 5 Și oamenii regelui au zis: "Iată Aman stă în curte". Regele a zis: "Să intre! "
\par 6 Aman a intrat, iar regele i-a zis: "Ce să se facă omului pe care regele vrea să-l cinstească?>> Aman socotea în inima sa: "Pe cine altul voiește regele să cinstească, dacă nu pe mine?"
\par 7 Și a zis Aman: "Pentru omul pe care vrea să-l cinstească regele,
\par 8 Să se aducă îmbrăcăminte regească și calul pe care călărește regele și coroană de rege;
\par 9 Apoi să se dea hainele, coroana și calul unuia dintre cei dintâi dregători ai regelui, ca să îmbrace pe omul acela, pe care vrea regele să-l cinstească și să-l plimbe călare pe cal prin piața cetății și să se strige înaintea lui: Așa se face omului, pe care vrea regele să-l cinstească!"
\par 10 Atunci regele a zis lui Aman: "Bine ai zis! Ia repede haine și cal, cum ai spus, și fă așa iudeului Mardoheu, care șade la poarta regelui. Să nu lași nimic din toate câte ai zis".
\par 11 Și a luat Aman haine și cal, a îmbrăcat pe Mardoheu și l-a plimbat călare pe cal prin piața cetății, strigând înaintea lui: "Așa se face omului, pe care regele vrea să-l cinstească!"
\par 12 Apoi Mardoheu s-a întors la poarta regelui, iar Aman s-a dus repede acasă trist și cu capul în pământ.
\par 13 și a spus Aman soției sale, Zereș și tuturor prietenilor săi toate câte se petrecuseră cu el. Iar înțelepții lui și Zereș, femeia sa, i-au zis: "Dacă Mardoheu, din pricina căruia a început căderea ta, e din neamul iudeilor, atunci tu nu vei putea face nimic împotriva lui, ci vei cădea sigur înaintea lui, căci cu el este Dumnezeul cel viu>.
\par 14 Și încă grăind ei cu el, au venit eunucii regelui și-l grăbeau pe Aman să meargă la ospățul pe care-l pregătise Estera.

\chapter{7}

\par 1 În ziua aceea regele cu Aman au venit să prânzească la regina Estera.
\par 2 În timpul ospățului, regele a zis iarăși către Estera: "Care este dorința ta, regină Estera? Că ea îți va fi împlinită. Și care este rugămintea ta, că ți se va împlini chiar și până la jumătate din regatul meu".
\par 3 Regina Estera a răspuns: "De am aflat bunăvoință în ochii tăi, o, rege, și dacă binevoiește regele, atunci să ni se dăruiască viață mie și poporului meu, după ruga mea.
\par 4 Căci suntem vânduți noi, eu și poporul meu, spre ucidere, spre nimicire și pieire. De am fi vânduți ca să fim robi eu aș fi tăcut, deși vrăjmașul nu ar fi acoperit paguba regelui".
\par 5 Și regele Artaxerxe a răspuns: "Cine este acela și unde este omul care a îndrăznit a gândi să facă așa?
\par 6 Iar Estera a răspuns: "Asupritorul Și vrăjmașul este răutăciosul Aman!" și s-a cutremurat Aman înaintea regelui și a reginei.
\par 7 Și s-a sculat regele de la ospăț plin de mânie și s-a dus în grădina palatului; iar Aman a rămas să roage pe regina Estera, pentru viața sa, căci vedea bine că regele hotărâse pieirea lui.
\par 8 Când s-a întors regele din grădina palatului, Aman tocmai se aruncase pe patul pe care se afla Estera. Și a zis regele: "Vrei încă să și siluiești pe regină aici, în casa mea?" Acest cuvânt, ieșit din gura regelui, acoperi de tulburare fața lui Aman.
\par 9 Atunci Harbona, unul din eunucii regelui, a zis: "Iată și spânzurătoarea pe care a pregătit-o Aman pentru Mardoheu, care a grăit de bine pe rege, stă la casa lui Aman, înaltă de cincizeci de coți". Iar regele a zis: "Spânzurați-l acolo!"
\par 10 Și au spânzurat pe Aman în spânzurătoarea pregătită de el. Numai așa s-a potolit mânia regelui.

\chapter{8}

\par 1 În ziua aceea, regele Artaxerxe a dat reginei Estera casa lui Aman, vrăjmașul Iudeilor; iar Mardoheu a fost chemat de rege, căci Estera îi spusese că el este rudă cu ea.
\par 2 Și regele și-a scos inelul pe care-l luase de la Aman, și l-a dat lui Mardoheu; iar Estera a pus pe Mardoheu ispravnic peste casa lui Aman.
\par 3 Apoi Estera a vorbit din nou înaintea regelui, a căzut la picioarele lui, a plâns și l-a rugat să abată răutatea lui Aman Agaghitul și uneltirea lui pe care el o îndreptase împotriva Iudeilor.
\par 4 Atunci regele și-a întins sceptrul său cel de aur către Estera, și s-a ridicat Estera și a stat înaintea feței regelui,
\par 5 Și a zis: "De binevoiește regele și de am aflat eu trecere înaintea feței lui; de este drept lucrul acesta înaintea regelui și de plac eu ochilor lui, atunci să se scrie, ca să fie revocate scrisorile cele trimise după uneltirile lui Aman, fiul lui Hamadata, din țara Agag, pentru uciderea Iudeilor în toate părțile regatului;
\par 6 Căci cum aș putea eu să privesc nenorocirea care ar atinge pe poporul meu și cum aș putea să văd pieirea neamului meu?"
\par 7 Regele Artaxerxe a zis către Estera și Mardoheu: "Casa lui Aman am dat-o Esterei, iar pe el l-am spânzurat, pentru că și-a întins mâna sa asupra Iudeilor.
\par 8 Scrieți și voi despre Iudei ce vă place, în numele regelui și întăriți cu inelul regelui, căci scrisoarea scrisă în numele regelui și întărită cu inelul regelui nu se poate schimba".
\par 9 Atunci au fost chemați scriitorii regelui, în luna a treia, adică în luna Sivan, în ziua de douăzeci și trei ale lunii, și cum a poruncit Mardoheu, așa s-a scris Iudeilor și satrapilor, guvernatorilor țărilor și cârmuitorilor ținuturilor lor, de la India până la Etiopia, din cele o sută douăzeci și șapte țări; și s-a scris fiecărei țări cu scrierea ei și fiecărui popor în limba lui, și Iudeilor le-a scris cu literele lor și în limba lor.
\par 10 Mardoheu a scris în numele regelui Artaxerxe, scrisorile le-a pecetluit cu inelul regelui și le-a trimis prin curieri călări, pe caii din hergheliile regelui,
\par 11 Spunând că regele îngăduie iudeilor din fiecare cetate să se adune ca să-și apere viața lor, să bată, să ucidă și să piardă pe toți cei puternici din popor și din țară, care ar vrea să-i atace, cu femeile și cu copiii lor, iar averea lor să o jefuiască.
\par 12 Aceasta să se facă în toate țările lui Artaxerxe, într-o singură zi, în a treisprezecea zi a lunii a douăsprezecea, adică a lunii Adar. Cuprinsul acestei scrisori este următorul: "Marele rege Artaxerxe, satrapilor celor o sută douăzeci și șapte de rări de la India și până la Etiopia, guvernatorilor de provincii și tuturor supușilor, credincioșilor lui, salutare. Mulți din cei răsplătiți cu cinste prin nemărginita bunătate a binefacerilor s-au trufit peste măsură și nu numai supușilor noștri caută să le facă rău, ci, neputându-și sătura mândria, încearcă să urzească uneltiri și împotriva binefăcătorilor lor, pierzând nu numai simțul recunoștinței omenești, ci, plini de trufie nebună, caută în chip nelegiuit să scape și de judecata lui Dumnezeu, Cel ce pururea toate le vede. Dar adesea și mulți, fiind îmbrăcați cu putere, ca să rânduiască lucrurile prietenilor ce s-au încrezut în ei, prin încredințările lor îi fac vinovați de vărsare de sânge nevinovat și-i supun la primejdii de neînlăturat, amăgind gândul bun și neprihănit al stăpânilor prin vorbărie vicleană și mincinoasă. Aceasta se poate vedea atât din povestirile mai vechi, cum am spus, precum și din faptele săvârșite în chip nelegiuit înaintea ochilor voștri de răutatea celor ce stăpânesc cu nevrednicie. De aceea ne îngrijim pentru viitor, ca să întocmim noi o împărăție netulburată pentru toți oamenii din lume, neîngăduind înșelătoriile, și pricinile ce ni se înfățișează judecându-le cu toată luarea aminte cuvenită. Așa Aman, fiul lui Hamadata Macedoneanul, cu adevărat străin de sângele persan și foarte departe de bunătatea noastră, fiind primit oaspete la noi, s-a învrednicit de bunăvoința pe care noi o avem către orice popor, până într-atâta, încât a fost numit părintele nostru și cinstit de toți, înfățișându-se ca a doua persoană după tronul regal. Dar fiind de o mândrie nemăsurată, a uneltit să ne lipsească pe noi de putere și de viață, iar pe salvatorul și pururea binefăcătorul nostru Mardoheu și pe nevinovata părtașă la regalitatea noastră, Estera, cu tot poporul lor, se silea prin felurite măsuri viclene să-i piardă. Astfel socotea el să ne lase fără oameni, iar împărăția persană să o dea Macedonenilor. Noi însă pe Iudei, osândiți de acest făcător de rele, la moarte, îi găsim nu răufăcători, ci oameni care trăiesc după cele mai drepte legiuiri și fii ai Dumnezeului celui viu, mare și preaînalt, Care ne-a dăruit nouă și strămoșilor noștri împărăție preafericită. De aceea bine veți face, de nu veți aduce la îndeplinire scrisorile trimise de Aman al lui Hamadata, căci el, făcând aceasta, a fost spânzurat la porțile Suzei cu toată casa, după cuvântul lui Dumnezeu, Celui ce stăpânește peste toți, Care i-a dat lui curând osânda cuvenită. Iar o copie a acestei scrisori puneți-o la vedere în tot locul, să lăsați pe Iudei să se folosească de legile lor și să-i ajutați, ca pe cei ce se vor scula asupra lor la vreme de necaz să-i poată răpune în ziua de treisprezece ale lunii a douăsprezecea, Adar, chiar în ziua aceea. Căci Dumnezeu, Cel ce stăpânește peste toți, în loc să piardă pe poporul cel ales, i-a rânduit lui această bucurie. Și voi, Iudeilor, între sărbătorile voastre vestite, sărbătoriți și această zi însemnată cu toată veselia, ca să fie de acum și să rămână în viitor pentru voi și pentru Perșii binevoitori amintirea izbăvirii voastre, iar pentru dușmanii voștri să fie amintirea pieirii lor. Fiecare cetate, sau ținut îndeobște, care nu se va conforma, se va pustii fără cruțare cu sabie și foc și va ajunge nu numai nelocuită de oameni totdeauna, ci dezgustătoare pentru fiare și păsări.
\par 13 Copii ale acestei scrisori să se dea în fiecare țară, ca lege, astfel ca Iudeii să fie gata pentru ziua aceea să se răzbune pe vrăjmașii lor".
\par 14 Curieri, călări pe cai iuți din hergheliile regelui, au alergat repede și cu mare grabă, cu porunca regelui. Porunca a fost vestită și în cetatea Suza.
\par 15 Mardoheu a ieșit de la rege în veșminte regești de culoare purpurie și albă și cu cunună mare de aur, iar cetatea Suza s-a bucurat și s-a veselit.
\par 16 La Iudei a fost atunci lumină în case, bucurie, veselie și mare prăznuire.
\par 17 De asemenea în toate țările, prin toate cetățile și în toate locurile unde ajunsese scrisoarea cu porunca regelui, a fost bucurie, veselie, ospețe și prăznuire la Iudei. Chiar și dintre popoarele țării mulți se făcură Iudei, pentru că-i cuprinsese frica de Iudei.

\chapter{9}

\par 1 În luna a douăsprezecea, adică în luna Adar, în a treisprezecea zi, când a sosit timpul aducerii la îndeplinire a poruncii regelui și a decretului lui; în ziua aceea, când nădăjduiau dușmanii Iudeilor să-i biruiască, s-a dovedit dimpotrivă că Iudeii și-au arătat puterea asupra dușmanilor lor;
\par 2 Atunci Iudeii s-au adunat în toate cetățile lor, prin toate țările lui Artaxerxe, ca să pună mâna pe cei ce le doreau răul și nimeni n-a putut sta împotriva feței lor, pentru că frica de Iudei apăsa asupra tuturor popoarelor.
\par 3 Toate căpeteniile țărilor, satrapii, guvernatorii și slujitorii regelui au sprijinit pe Iudei, de teama lui Mardoheu,
\par 4 Căci Mardoheu era mare în casa regelui și renumele lui se lățise în toate țările, deoarece omul acesta, Mardoheu, devenise tot mai puternic.
\par 5 Atunci au stârpit Iudeii pe toți dușmanii lor, ucigând cu sabia, omorând, pierzând și făcând cu dușmanii lor după voia lor.
\par 6 În cetatea Suza, au ucis Iudeii cinci sute de oameni.
\par 7 Între aceștia au ucis și pe Parșandata, pe Dalfon, pe Aspata,
\par 8 Pe Porata, pe Adalia, pe Aridata,
\par 9 Pe Parmașta, pe Arisai, pe Aridai și pe Iezata,
\par 10 Adică pe cei zece feciori ai lui Aman, fiul lui Hamadata, vrăjmașul Iudeilor; dar ei nu i-au prădat de averile lor.
\par 11 În acea zi s-a adus la cunoștință regelui numărul celor uciși în cetatea Suza.
\par 12 Atunci regele a zis către regina Estera: "În cetatea Suza, au ucis Iudeii cinci sute de oameni și pe cei zece fii ai lui Aman. Ce vor fi făcut ei în celelalte țări ale regelui? Care este dorința ta? Ea ți se va împlini. Și ce dorință mai ai? Că ea îți va fi împlinită".
\par 13 Estera a răspuns: "De binevoiește regele, să se îngăduie Iudeilor celor din Suza să facă același lucru și mâine, pe care l-au făcut astăzi, iar pe cei zece fii ai lui Aman să-i spânzure".
\par 14 Și a poruncit regele să se facă așa; s-a dat poruncă în Suza și pe cei zece fii ai lui Aman i-au spânzurat.
\par 15 Și s-au adunat Iudeii cei din Suza și în ziua a paisprezecea a lunii lui Adar și au omorât trei sute de oameni, dar la jaf nu și-au întins mâinile.
\par 16 Iar ceilalți Iudei care se aflau în țările regelui s-au adunat ca să-și apere viața lor și să fie netulburați de vrăjmașii lor. Aceștia au omorât din dușmanii lor șaptezeci și cinci de mii, iar la jaf nu și-au întins mâinile.
\par 17 Aceasta s-a petrecut în treisprezece ale lunii lui Adar. În ziua de paisprezece a aceleiași luni s-au liniștit și au făcut-o zi de ospăț și de veselie.
\par 18 Iudeii însă, care se aflau în Suza, s-au adunat în ziua de treisprezece și de paisprezece ale lunii lui Adar, iar în cincisprezece ale ei s-au liniștit și au făcut-o zi de ospăț și de veselie.
\par 19 De aceea Iudeii din provincie, care locuiesc în sate neîntărite, petrec ziua de paisprezece ale lunii Adar în veselie și ospețe, ca zi de sărbătoare, trimițându-și daruri unii altora; iar cei ce trăiesc în orașe petrec și ziua de cincisprezece ale lunii Adar în mare veselie, trimițând daruri vecinilor.
\par 20 După aceea Mardoheu a scris toate întâmplările acestea și a trimis scrisori tuturor Iudeilor care erau în țările regelui Artaxerxe, la cei de aproape și la cei de departe,
\par 21 Ca să sărbătorească acele zile bune în fiecare an, în ziua de paisprezece și de cincisprezece ale lunii Adar,
\par 22 Întrucât acestea sânt zilele în care Iudeii au fost lăsați în pace de vrăjmașii lor și întrucât aceasta este luna în care întristarea lor s-a prefăcut în bucurie și tânguirea în zi de sărbătoare. Să facă dar din ele zile de petrecere și de veselie, trimițându-și unii altora daruri și dând milostenie la săraci.
\par 23 Și au primit Iudeii cele ce le scrisese Mardoheu:
\par 24 Cum Aman, fiul lui Hamadata, din țara Agag, vrăjmașul tuturor Iudeilor, se gândise să-i ucidă și aruncase Pur, adică sori, pentru pierderea lor;
\par 25 Cum Estera a străbătut până la rege și cum regele a poruncit prin scrisoare nouă, ca uneltirea cea rea a lui Aman, pe care o plănuise el asupra Iudeilor, să se întoarcă asupra capului lui și să-l spânzure pe el și pe fiii lui.
\par 26 De aceea s-au și numit aceste zile Purim, de la numirea Pur. Deci, potrivit cu toate cuvintele scrisorii și cu toate cele ce ei văzuseră și cu cele ce se petrecuseră la ei,
\par 27 Iudeii au stabilit și au primit pentru ei și pentru urmașii lor și pentru cei ce li se vor alătura, ca să sărbătorească fără abatere aceste două zile în fiecare an după rânduială,
\par 28 Și ca zilele acestea ale Purimului să fie pomenite și prăznuite din neam în neam, în fiecare familie, la Iudei, în fiecare țară și în fiecare cetate.
\par 29 Regina Estera, fiica lui Abihail și Mardoheu Iudeul au scris a doua oară în chip stăruitor cele ce au făcut ei, ca să întărească scrisoarea despre Purim,
\par 30 Și au trimis scrisori tuturor Iudeilor din cele o sută douăzeci și șapte de țări ale regatului lui Artaxerxe cu cuvinte de pace și de credincioșie,
\par 31 Ca ei să păzească cu tărie aceste zile ale Purimului la vremea lor, precum le statorniciseră Mardoheu Iudeul și regina Estera, pentru ei și pentru urmașii lor cu post și tânguire.
\par 32 Și porunca Esterei întări așezarea sărbătorii Purimului scriind-o în carte.

\chapter{10}

\par 1 După aceea regele Artaxerxe puse bir pe țări și pe insulele mării.
\par 2 De altfel toate faptele slăvite și atotputernicia lui, cum și arătarea amănunțită a măririi lui Mardoheu cu care-l cinstise regele, sânt scrise în cartea Cronicilor regilor Mediei și Persiei.
\par 3 De asemenea e arătat acolo că iudeul Mardoheu era al doilea după regele Artaxerxe, mare înaintea Iudeilor și iubit între mulțimile fraților săi, căci căuta binele poporului său și vorbea în folosul neamului lui. Și zicea Mardoheu: "De la Dumnezeu a fost aceasta, că eu mi-am adus aminte de visul pe care l-am visat despre aceste întâmplări, că n-a rămas neîmplinit nimic din el. Izvorul cel mic s-a făcut râu mare și a fost lumină, soare și mulțime de apă; acest râu este Estera, pe care și-a luat-o de femeie regele și a făcut-o regină. Iar cei doi balauri sânt eu și Aman. Popoarele sânt cei ce s-au unit să stârpească numele Iudeilor; iar poporul meu sânt Israeliții, care au strigat către Dumnezeu și au fost izbăviți. A izbăvit Dumnezeu pe poporul Său și ne-a izbăvit Domnul din toate relele acestea. Săvârșit-a Dumnezeu semne și minuni mari, cum n-au fost printre neamuri. Așa a rânduit Dumnezeu doi sorți: unul pentru poporul lui Dumnezeu, iar celălalt pentru neamuri. Acești doi sorii au ieșit în ceasul, la vremea și în ziua judecății înaintea lui Dumnezeu și a tuturor neamurilor. Atunci și-a adus aminte Domnul de poporul Său și a dat dreptate moștenirii Sale. Aceste zile ale lunii lui Adar, adică a paisprezecea și a cincisprezecea ale acestei luni, se vor prăznui veșnic cu alai, cu bucurie și veselie înaintea lui Dumnezeu în poporul Său Israel. în anul al patrulea al domniei lui Ptolomeu și a Cleopatrei, Dositei, care se spune că a fost preot și levit și Ptolomeu, fiul său, au adus în Alexandria această scrisoare despre Purim. Această scrisoare se spune că a tâlcuit-o Lisimah, fiul lui Ptolomeu, care fusese la Ierusalim.


\end{document}