\begin{document}

\title{Estera}


\chapter{1}

\par 1 În anul al doilea al domniei lui Artaxerxe cel Mare, în ziua întâi a lunii Nisan, Mardoheu, fiul lui Iair, fiul lui ?imei, fiul lui Chi?, iudeu din semin?ia lui Veniamin, om mare, care traia în cetatea Suza ?i slujea la curtea regelui, a visat un vis. Mardoheu era din robii pe care Nabucodonosor, regele Babilonului, îi luase în robie din Ierusalim cu Iehonia, regele Iudei. Iar visul lui a fost acesta: "S-a facut parca zgomot mare pe pamânt, tunet, cutremur ?i tulburare. ?i deodata au ie?it doi balauri mari, gata sa se lupte unul cu altul, ?i urletul lor era cumplit. Dupa urletul lor, toate popoarele s-au pregatit de razboi, ca sa loveasca pe poporul drep?ilor. ?i iata a venit pe pamânt zi de negura ?i întuneric, de necaz ?i strâmtorare, de mare durere ?i tulburare. Atunci tot poporul drep?ilor s-a tulburat, temându-se de raul lor; s-a pregatit sa piara ?i a început sa strige catre Domnul. La strigatul lor, a ie?it parca dintr-un izvor mic un râu mare cu apa multa, a stralucit lumina ?i soare, s-au înal?at cei smeri?i ?i au nimicit pe trufa?i". De?teptându-se Mardoheu dupa acest vis, care închipuia ce avea sa faca Dumnezeu, pastra acest vis în inima sa ?i dorea sa-l priceapa în toata întregimea lui pâna în seara. ?i a ramas Mardoheu la curte cu Gavata ?i cu Tara, doi eunuci ai regelui care pazeau curtea; atunci a auzit ce vorbeau ei, a aflat planul ?i a descoperit ca ei se pregateau sa puna mâna pe regele Artaxerxe ?i i-a spus regelui. Regele a cercetat pe cei doi eunuci ?i dupa ce ei au marturisit, au fost spânzura?i. Regele a scris întâmplarea aceasta spre aducere aminte ?i a scris-o ?i Mardoheu. Atunci a poruncit regele lui Mardoheu sa slujeasca la curte ?i i-a dat ?i daruri pentru aceasta. Dar pe lânga rege era atunci vestit Aman. fiul lui Hamadata din ?ara Agag. Acesta se silea sa faca rau lui Mardoheu ?i poporului sau pentru cei doi eunuci ai regelui. Dupa întâmplarile acestea, în zilele lui Artaxerxe, care domnea peste o suta douazeci ?i ?apte de ?ari, de la India pâna în Etiopia,
\par 2 În vremea când regele Artaxerxe î?i avea scaunul sau domnesc în cetatea Suza,
\par 3 În anul al treilea al domniei lui, a dat el ospa? pentru dregatorii sai ?i pentru cei ce-i slujeau lui, pentru capeteniile mai înalte ale o?tirii Persiei ?i Mediei ?i pentru guvernatorii ?arilor sale,
\par 4 Aratându-?i marea boga?ie a regatului sau ?i stralucirea deosebita a maririi sale, în curgere de mai multa vreme, anume timp de o suta optzeci de zile.
\par 5 Dupa sfâr?itul acestor zile, regele a facut ?i pentru poporul sau care se afla în capitala Suza, de la mic pâna la mare, ospa? de ?apte zile, în gradina cur?ii sale domne?ti,
\par 6 Împodobita cu covoare de matase alba ?i de purpura violeta, atârnate pe frânghii de in ?i de matase, trecute prin verigi de argint, întarite în stâlpi de marmura.
\par 7 Divanuri de aur ?i de argint erau a?ezate pe pardoseala de porfira, de marmura alba, de sidef ?i marmura neagra.
\par 8 Bauturile se turnau în vase de aur ?i în cupe felurite, în valoare de treizeci de mii de talan?i, iar vin din care bea însu?i regele a fost mult, dupa boga?ia ?i darnicia regelui. Bautura se consuma cuviincios ?i fara sila, ca regele poruncise tuturor cârmuitorilor din casa sa sa faca fiecaruia dupa voia lui.
\par 9 Regina Vasti a facut ?i ea ospa?, pentru femei, în casa domneasca a regelui Artaxerxe.
\par 10 În ziua a ?aptea, când inima regelui s-a înveselit de vin, acesta a zis catre Mehuman, Bizeta, Harbona, Bigta, Abgata, Zetar ?i Carcas, cei ?apte eunuci care slujeau înaintea fe?ei regelui Artaxerxe,
\par 11 Sa aduca pe regina Vasti înaintea fe?ei regelui cu coroana regeasca pe cap, ca sa arate popoarelor ?i capeteniilor frumuse?ea ei, pentru ca era foarte frumoasa.
\par 12 Dar regina Vasti n-a voit sa vina dupa porunca regelui ce i s-a trimis prin eunuci.
\par 13 Din pricina aceasta s-a mâniat regele tare ?i s-a aprins într-însul urgie mare ?i a zis catre în?elep?ii care ?tiau datinile din trecut, (caci a?a era rânduiala ca orice pricina a regelui sa se spuna înaintea celor ce cuno?teau legile ?i dreptul):
\par 14 "Cum sa se aplice legea fa?a de regina Vasti, pentru ca ea nu a împlinit porunca regelui Artaxerxe, ce i s-a trimis prin eunuci?"
\par 15 ?i erau în apropierea regelui: Car?ena, ?etar, Admata, Tar?i?, Mere?, Marsena ?i Memucan, ?apte capetenii ale Mediei ?i Persiei, care puteau vedea fa?a regelui ?i ?ineau locurile cele dintâi în regat.
\par 16 Atunci a raspuns Memucan înaintea regelui ?i a capeteniilor: "Regina Vasti s-a facut vinovata nu numai înaintea regelui, ci ?i înaintea tuturor dregatorilor ?i înaintea tuturor popoarelor care sânt în toate ?arile regelui Artaxerxe,
\par 17 Pentru ca purtarea reginei Vasti va ajunge la ?tirea tuturor femeilor, ?i vor nesocoti ?i acelea pe barba?ii lor ?i vor zice: Regele Artaxerxe a poruncit sa fie adusa regina Vasti înaintea fe?ei sale ?i ea n-a venit.
\par 18 De acum so?iile dregatorilor din Persia ?i Media, care vor auzi de purtarea reginei, vor zice la fel dregatorilor regelui, ?i va fi destul dispre? ?i mânie.
\par 19 Deci, daca binevoie?te regele, sa iasa de la el hotarâre regeasca ?i sa se scrie în legile Persiei ?i Mediei, cu neschimbare, ca regina Vasti nu va mai intra înaintea regelui Artaxerxe, iar vrednicia de regina a ei regele o va da alteia mai vrednica decât ea.
\par 20 Când se va auzi de aceasta hotarâre a regelui, care se va împra?tia în toata împara?ia lui cât este ea de mare, atunci toate femeile î?i vor cinsti barba?ii de la mic pâna la mare".
\par 21 ?i cuvântul a placut regelui ?i dregatorilor, ?i a facut dupa sfatul lui Memucan.
\par 22 Regele a trimis în toate ?arile scrisori, scrise pentru fiecare ?ara cu scrisul ei ?i pentru fiecare popor în limba lui, ca fiecare barbat sa fie stapân în casa sa. Aceasta s-a adus la cuno?tin?a fiecaruia în limba parinteasca a fiecaruia.

\chapter{2}

\par 1 Dupa întâmplarea aceasta, când s-a potolit mânia regelui Artaxerxe, ?i-a adus el aminte de regina Vasti, de ceea ce facuse ea ?i de ce se hotarâse împotriva ei,
\par 2 Iar oamenii regelui, care-i slujeau lui, au zis: "Sa se caute pentru rege fete tinere ?i frumoase.
\par 3 Sa rânduiasca dar regele dregatori în toate ?arile regatului sau, care sa adune toate fetele tinere ?i frumoase la fa?a în capitala Suza, în casa femeilor, sub supravegherea lui Hegai, eunucul regelui, pazitorul femeilor, ?i sa le dea sapun ?i celelalte de trebuin?a pentru spalat.
\par 4 ?i fata care va placea ochilor regelui sa fie regina în locul Vastei". Cuvântul acesta a placut regelui, ?i el a facut a?a.
\par 5 În vremea aceea era în capitala Suza un iudeu cu numele Mardoheu, fiul lui Iair, fiul lui ?imei, fiul lui Chi?, din semin?ia lui Veniamin.
\par 6 Acesta fusese adus din Ierusalim împreuna cu robii lua?i cu Iehonia, regele lui Iuda, pe care-i luase Nabucodonosor, regele Babilonului.
\par 7 El cre?tea pe Hadasa, adica pe Estera, fiica unchiului sau, deoarece ea nu avea nici tata, nici mama. Fata aceasta era mândra la înfa?i?are ?i frumoasa la chip ?i, dupa moartea tatalui ei ?i a mamei ei, o luase Mardoheu ca fiica.
\par 8 Când s-a facut cunoscuta porunca regelui ?i hotarârea lui ?i când au fost adunate multe fete în capitala Suza, sub supravegherea lui Hegai, atunci a fost luata ?i Estera în casa regelui sub supravegherea lui Hegai, pazitorul femeilor.
\par 9 Fata aceasta a placut ochilor lui ?i ?i-a atras bunavoin?a lui. De aceea s-a grabit el sa-i dea cele de spalat ?i tot ce i se cuvenea ?i a pus pe lânga ea ?apte fete vrednice de a fi cu ea ?i au a?ezat-o pe ea ?i fetele ei în cea mai buna încapere din casa femeilor.
\par 10 Estera însa n-a spus nimic nici de poporul sau, nici de neamurile sale, pentru ca Mardoheu îi daduse porunca sa nu spuna.
\par 11 Mardoheu venea în fiecare zi în curtea casei femeilor, ca sa afle ?tiri despre sanatatea Esterei ?i cele ce se petreceau cu ea.
\par 12 Când venea fiecarei fete vremea sa intre la regele Artaxerxe, dupa ce timp de douasprezece luni se savâr?eau asupra ei cele rânduite pentru femei, (caci atât timp ?inea vremea cura?irii lor: ?ase luni cu miruri ?i ?ase luni cu aromate ?i alte unsori femeie?ti),
\par 13 Atunci fata intra la rege ?i orice ar fi cerut, i se dadea pentru a merge din casa femeilor în casa regelui.
\par 14 Seara ea intra ?i diminea?a se ducea în a doua casa a femeilor sub supravegherea lui ?aa?gaz, eunuc al regelui, pazitorul concubinelor, ?i nu mai intra dupa aceea la rege decât doar daca ar fi voit-o regele ?i ar fi fost chemata anume.
\par 15 Când a venit vremea Esterei, fiica lui Abihail, unchiul lui Mardoheu, care o luase la sine ca fiica, sa mearga la rege, atunci ea nu a cerut nimic, decât numai ceea ce-i zisese Hegai, eunucul regelui, pazitorul femeilor. Ea a aflat trecere înaintea tuturor celor ce o vedeau.
\par 16 Estera a fost luata la regele Artaxerxe ?i dusa în casa lui domneasca în luna a zecea, adica în luna Tebet, în al ?aptelea an al domniei lui.
\par 17 ?i a iubit regele pe Estera mai mult decât pe toate femeile, ?i ea a dobândit bunavoin?a lui ?i trecere mai mult decât toate femeile, ?i el a pus coroana domneasca pe capul ei ?i a facut-o regina în locul Vastei.
\par 18 Apoi a facut regele ospa? mare pentru to?i dregatorii sai ?i pentru cei ce-i slujeau lui. Acesta a fost ospa? pentru Estera; regele a mai facut mari u?urari ?arilor ?i a împar?it daruri cu darnicie de rege.
\par 19 Iar când s-au adunat fetele a doua oara ?i Mardoheu ?edea la poarta regelui,
\par 20 Estera nu spusese nimic despre neamul sau, a?a cum îi poruncise Mardoheu, pentru ca Estera împlinea cuvântul lui Mardoheu, ca ?i când era sub ocrotirea lui.
\par 21 Într-o zi, când Mardoheu ?edea la poarta regelui, doi eunuci ai regelui, Bigtan ?i Tere?, capeteniile paznicilor lui, s-au aprins de mânie, ca se daduse întâietate lui Mardoheu, ?i cautau sa puna mâna pe regele Artaxerxe.
\par 22 Aflând acest lucru, Mardoheu i-a spus reginei Estera, iar Estera i-a spus regelui în numele lui Mardoheu.
\par 23 Fapta s-a cercetat ?i s-a gasit adevarata. Amândoi eunucii au fost spânzura?i în curte. Aceasta binefacere a lui Mardoheu a fost scrisa în cartea amintirilor zilnice ale regelui.

\chapter{3}

\par 1 Dupa aceea a ridicat regele Artaxerxe pe Aman, fiul lui Hamadata, din ?ara Agag, ?i a pus scaunul lui mai presus decât al tuturor capeteniilor pe care le avea.
\par 2 To?i cei ce slujeau regelui, care erau la poarta regelui, se închinau ?i se aruncau cu fe?ele la pamânt înaintea lui Aman, caci a?a poruncise regele; iar Mardoheu nu se închina ?i nu-?i pleca fa?a la pamânt.
\par 3 De aceea cei ce slujeau regelui ?i care erau la poarta, ziceau lui Mardoheu: "De ce calci tu porunca regelui?"
\par 4 ?i cum ei ziceau aceasta în fiecare zi, iar el nu-i asculta, i-au spus lui Aman, ca sa vada de va starui Mardoheu în purtarea sa, caci el le spusese ca este iudeu.
\par 5 Când a vazut Aman ca Mardoheu nu se închina ?i nu-?i pleaca fruntea pâna la pamânt înaintea sa, s-a umplut de mânie.
\par 6 Dar i s-a parut prea pu?in lucru sa puna mâna numai pe Mardoheu. ?i fiindca i se spusese din ce popor e Mardoheu, Aman ?i-a pus în gând sa ucida pe to?i Iudeii care se aflau în tot regatul lui Artaxerxe, întrucât erau poporul lui Mardoheu.
\par 7 Au facut deci sfat în luna întâi, adica în luna Nisan, în anul al doisprezecelea al domniei lui Artaxerxe, ?i au aruncat Pur, adica sor?i, de fa?a cu Aman, ca sa vada în ce luna ?i în ce zi sa fie ucis dintr-o data poporul lui Mardoheu, ?i a cazut sor?ul pe luna a douasprezecea, adica pe luna lui Adar.
\par 8 În vremea aceasta a spus Aman regelui Artaxerxe: "Este un popor risipit ?i împra?tiat printre popoare, prin toate ?arile regatului tau. Legile lui sânt deosebite de legile tuturor popoarelor, legilor regelui nu se supun ?i regele nu se cuvine sa-l lase a?a.
\par 9 Daca binevoie?te regele, atunci sa se hotarasca în scris sa fie uci?i ?i eu voi cântari zece mii de talan?i de argint ?i voi da în mâna vistiernicilor, ca sa-i verse în vistieria regelui".
\par 10 Atunci ?i-a scos regele inelul sau din mâna sa ?i l-a dat lui Aman, fiul lui Hamadata, din ?ara Agag, ca sa întareasca decretul cel împotriva Iudeilor,
\par 11 Zicând lui Aman: "Î?i dau ?ie acest argint ?i poporul; fa cu el ce vrei!"
\par 12 Atunci au fost chema?i scriitorii regelui în a treisprezecea zi a lunii întâi, ?i s-a scris, cum poruncise Aman, catre satrapii regatului ?i catre capetenia fiecarei din cele o suta douazeci ?i ?apte de ?ari, de la ?arile Indiei ?i pâna la Etiopia, ?i catre capeteniile fiecarui popor; ?i s-a scris fiecarei ?ari cu scrierea ei ?i fiecarui popor în limba lui ?i toate s-au scris în numele regelui Artaxerxe ?i au fost întarite cu inelul lui.
\par 13 Scrisorile s-au trimis prin ?tafete în toate ?arile regelui, ca sa ucida, sa piarda ?i sa nimiceasca pe to?i Iudeii, mic ?i mare, copii ?i femei, într-o singura zi, ?i anume în a treisprezecea zi a lunii a douasprezecea, adica în luna Adar, iar averile lor sa le jefuiasca. Iata cuprinsul acelei scrisori: "Marele rege Artaxerxe, celor ce cârmuiesc de la India pâna la Etiopia peste o suta douazeci ?i ?apte de ?ari ?i capeteniilor ?i slujitorilor de sub conducerea lor acestea scrie: Domnind peste multe popoare ?i stapânind toata lumea, eu, fara sa fiu îngâmfat de putere, ci cârmuind pururea cu blânde?e ?i cu lini?te, am voit sa fac via?a supu?ilor pururea netulburata, pazind regatul meu în pace ?i u?or de strabatut pâna la hotarele lui, statornicind pacea de to?i dorita. Dar când eu am întrebat pe sfetnici, cum am putea aduce aceasta la îndeplinire, atunci Aman, care e vestit la noi prin în?elepciune ?i se bucura neclintit de bunavoin?a noastra ?i care a dovedit cea mai deplina credincio?ie, pentru care a dobândit cinstea de a ?edea în al doilea loc dupa rege, ne-a aratat ca prin toate neamurile lumii s-a amestecat un popor vrajma?, potrivnic legilor tuturor popoarelor, care necontenit nesocote?te poruncile regelui, ca sa nu se poata întemeia cârmuirea noastra fara meteahna. Aflând deci ca numai singur acest popor se împotrive?te pururea oricarui om, ca duce un fel de via?a straina de legi ?i, împotrivindu-se pururea lucrarilor noastre, face cele mai mari nelegiuiri, ca regatul nostru sa nu ajunga a fi bine întocmit, am poruncit cele aratate voua în scrisorile lui Aman, care este pus de noi peste lucruri ?i ca un al doilea parinte al nostru, ca sa-i stârpi?i pe to?i, cu femei ?i cu copii, prin sabie cumplita, fara nici o mila ?i cru?are, în treisprezece ale lunii a douasprezecea, adica în luna lui Adar, a acestui an, ca astfel ace?ti oameni vrajma?i ?i astazi, ca ?i în trecut, fiind într-o singura zi arunca?i cu sila în iad, sa nu ne mai împiedice în viitor de a duce via?a pa?nica ?i netulburata pâna în sfâr?it.
\par 14 Cuprinsul acestei porunci sa se dea în fiecare ?ara, ca lege cu putere pentru toate popoarele, ca ele sa fie gata pentru ziua aceea".
\par 15 ?tafetele au zburat repede cu porunca regelui. Porunca s-a facut cunoscuta ?i în capitala Suza. Regele ?i Aman ?edeau ?i beau, iar cetatea Suza era în fierbere.

\chapter{4}

\par 1 Când Mardoheu a aflat tot ce se facuse, ?i-a rupt hainele sale, ?i-a pus pe sine sac ?i cenu?a ?i, ie?ind în mijlocul ceta?ii, a ridicat strigat amarnic.
\par 2 Ajungând însa pâna la poarta regelui, s-a oprit, deoarece nu putea sa intre pe poarta regelui îmbracat cu sac ?i cu cenu?a pe cap.
\par 3 Tot a?a ?i în fiecare ?ara ?i loc, unde ajungea porunca regelui ?i scrisoarea lui, era tânguire mare printre Iudei, post, plângere ?i bocet. Sacul ?i cenu?a erau a?ternutul multora.
\par 4 Atunci au venit slujitorii Esterei, eunucii ?i slujnicele ei ?i i-au spus; ?i regina s-a tulburat stra?nic. Apoi a trimis haine lui Mardoheu, ca sa se îmbrace cu ele ?i sa lepede de pe el sacul. Dar el n-a voit.
\par 5 De aceea a chemat Estera pe Hatac, unul din eunucii regelui pe care acesta îl pusese sa fie pe lânga ea, ?i l-a trimis la Mardoheu, ca sa afle de ce ?i pentru ce sânt acestea.
\par 6 Mergând Hatac la Mardoheu, în pia?a ora?ului cea din fa?a cur?ii regelui,
\par 7 I-a spus Mardoheu toate câte se întâmplase ?i de numarul argin?ilor ce fagaduise Aman sa verse în vistieria domneasca pentru Iudei ca sa-i stârpeasca.
\par 8 Îi dadu de asemenea ?i o copie de pe porunca domneasca, cuprinsa în hrisovul dat în Suza pentru stârpirea lor, ca sa-l arate Esterei ?i sa-i dea de veste despre toate. Pe lânga aceasta o sfatuia sa mearga la rege ?i sa-l roage de iertare pentru poporul ei, aducându-?i aminte de zilele sale cele smerite, când era crescuta sub mâna lui, a lui Mardoheu, pentru ca Aman, cel al doilea dupa rege, a osândit pe Iudei la moarte; sa strige de asemenea catre Domnul, ca sa-i izbaveasca pe ei de la moarte.
\par 9 Deci a venit Hatac ?i a spus Esterei vorbele lui Mardoheu.
\par 10 Atunci a vorbit Estera cu Hatac ?i l-a trimis sa spuna lui Mardoheu:
\par 11 "To?i cei ce slujesc regelui ?i popoarele din ?arile regelui ?tiu ca tot cel ce va intra la rege înauntrul cur?ii, barbat sau femeie, fara sa fie chemat, ia o singura osânda: moartea. Numai acela spre care va întinde regele sceptrul sau de aur scapa cu via?a. Eu însa n-am fost chemata la rege de mai bine de treizeci de zile".
\par 12 Aceste cuvinte ale Esterei au fost spuse lui Mardoheu.
\par 13 Iar Mardoheu a raspuns Esterei urmatoarele: "Sa nu soco?i ca ai sa scapi tu singura în casa regelui, dintre to?i Iudeii.
\par 14 Daca tu vei tacea în vremea aceasta, atunci izbavirea ?i eliberarea vor veni pentru Iudei din alta parte, iar tu ?i casa tatalui tau ve?i pieri. ?i cine ?tie daca tu n-ai ajuns la vrednicia de regina tocmai pentru vremile acestea?"
\par 15 Atunci Estera a raspuns lui Mardoheu:
\par 16 "Mergi, aduna pe to?i Iudeii din Suza ?i posti?i pentru mine; sa nu mânca?i ?i sa nu be?i trei zile, nici ziua, nici noaptea ?i voi posti ?i eu cu slujnicile mele ?i apoi ma voi duce la rege, de?i aceasta este împotriva legii ?i de va fi sa pier, voi pieri".
\par 17 Atunci s-a dus Mardoheu ?i a facut cum îi poruncise Estera. El s-a rugat Domnului, pomenind toate lucrurile Domnului ?i zicând: "Doamne, Doamne, Împarate atot?iitorule, to?i sunt în puterea Ta ?i nu este cine sa se împotriveasca ?ie când vei voi sa izbave?ti pe Israel. Tu ai facut cerul ?i pamântul ?i toate cele minunate de sub cer. Tu e?ti Domnul tuturor ?i nu este cine sa se împotriveasca ?ie, Doamne! Tu toate le ?tii Doamne, Tu ?tii ca eu nu din mândrie, nici din trufie, nici ca sa jignesc nu m-am închinat lui Aman cel mândru, caci eu cu placere m-a? fi apucat sa sarut talpile picioarelor lui pentru izbavirea lui Israel. Dar eu am facut aceasta ca sa nu dau slava oamenilor mai presus de slava lui Dumnezeu ?i nu m-am închinat nimanui, decât numai ?ie, Domnului meu ?i nici nu voi face aceasta din mândrie. ?i acum, Doamne Dumnezeule, Împarate, Dumnezeul lui Avraam, cru?a pe poporul Tau, caci se pune la cale pieirea noastra ?i voiesc sa piarda mo?tenirea Ta cea dintru început. Nu trece cu vederea partea Ta pe care ai rascumparat-o pentru Tine din ?ara Egiptului. Auzi rugaciunea mea ?i Te milostive?te spre mo?tenirea Ta; întoarce plânsul nostru în veselie ca, vii fiind noi, sa laudam numele Tau, Doamne, ?i nu astupa gura celor ce Te preamaresc pe Tine". ?i to?i Israeli?ii au strigat din toate puterile lor ca moartea era înaintea ochilor lor. ?i a alergat ?i Estera la Domnul, cuprinsa de groaza mor?ii ?i, dezbracându-se de hainele slavei sale, s-a îmbracat în haine de deznadejde ?i de jale, iar în locul unsorilor celor scumpe, cu cenu?a ?i cu ?arâna ?i-a presarat capul sau; ?i-a smerit cumplit trupul sau ?i tot locul împodobit altadata l-a umplut de par smuls din capul sau ?i, rugându-se Domnului, Dumnezeului lui Israel, a zis: "Domnul meu, numai Tu singur e?ti Împaratul nostru; ajuta-mi mie celei singuratice ?i fara ajutor afara de Tine, ca pieirea mea e aproape! Eu am auzit, Doamne, de la tatal meu, în neamul meu parintesc, ca Tu ?i-ai ales pe Israel din toate popoarele ?i pe parin?ii no?tri din to?i stramo?ii lor, ca sa fie mo?tenirea Ta ve?nica ?i ai facut pentru ei toate câte ai zis. Acum noi am gre?it înaintea Ta ?i Tu ne-ai dat în mâinile vrajma?ilor no?tri, pentru ca am laudat pe dumnezeii lor; drept e?ti Tu, Doamne! Dar ei acum nu se mai mul?umesc cu robia noastra amara, ci ?i-au dat mâna cu idolii lor, ca sa rastoarne poruncile gurii Tale ?i sa stârpeasca mo?tenirea Ta ?i sa astupe gura celor ce Te slavesc pe Tine ?i sa stinga slava casei Tale ?i jertfelnicul Tau, sa dezlege gura popoarelor pentru a preaslavi pe dumnezeii lor cei mincino?i ?i pentru ca regele cel pamântesc sa fie admirat totdeauna. Nu da, Doamne, sceptrul Tau dumnezeilor celor ce nu sânt, ca sa nu se bucure vrajma?ii de caderea noastra, ci întoarce uneltirea lor asupra lor în?i?i, iar pe uneltitorul împotriva noastra da-l ru?inii. Adu-?i aminte, Doamne, arata-Te noua în vremea necazului nostru ?i-mi da mie curaj. Împarate al dumnezeilor ?i Stapâne a toata stapânirea, daruie?te gurii mele cuvânt cu trecere înaintea leului acestuia ?i umple inima lui de ura catre cel ce ne prigone?te pe noi, spre pieirea lui ?i a celor de un gând cu el. Iar pe noi ne izbave?te cu mâna Ta ?i-mi ajuta mie celei singure, care nu am alt ajutor decât pe Tine, Doamne; Tu ai ?tiin?a de toate ?i cuno?ti ca eu urasc slava celor fara de lege ?i mi-e sila de patul celor netaia?i împrejur ?i de tot cel de alt neam. De asemenea cuno?ti nevoia mea, ca nu pot suferi semnul mândriei mele care se afla pe capul meu în zilele când ma arat, mi-e sila de el, ca de o haina întinata cu sânge ?i nici nu-l port când sânt singura. Roaba Ta n-a mâncat din masa lui Aman, nici n-a pre?uit ospa?ul regesc; vin jertfit la idoli n-am baut, nici nu s-a veselit roaba Ta din vremea schimbarii soartei mele ?i pâna acum decât numai de Tine, Doamne Dumnezeul lui Avraam. Dumnezeule, Cel ce ai putere peste toate, auzi glasul celor fara de nadejde ?i ne izbave?te din mâinile uneltitorilor de rele, scapându-ma din frica mea".

\chapter{5}

\par 1 Dupa trei zile de rugaciune Estera ?i-a dezbracat hainele cele de jale ?i s-a îmbracat în cele de regina ?i, facându-se stralucita ?i chemând pe Dumnezeul cel atoatevazator ?i mântuitor, a luat cu ea doua slujnice: una de care se sprijinea oarecum din alintare, iar alta care, urmându-i, îi ?inea hainele. Ea era minunata, în culmea frumuse?ii sale ?i fa?a sa era vesela ca ?i cum ar fi fost plina de iubire, iar inima ei era apasata de frica. S-a oprit în curtea dinauntru a casei regelui, la u?a regelui. Regele ?edea atunci pe tronul sau domnesc, în casa domneasca, chiar în dreptul u?ii, îmbracat în toate hainele maririi sale, tot în aur ?i în pietre scumpe, dar cumplit de posomorât. Când regele a vazut pe regina Estera stând afara, aceasta a aflat mila în ochii lui. Întorcându-?i fa?a înflacarata de slava, el a privit cu mânie stra?nica. Atunci regina a cazut cu duhul, s-a schimbat la fa?a din pricina slabiciunii ce i-a venit ?i s-a aplecat pe capul slujnicei care o înso?ea. Dar Dumnezeu a schimbat duhul regelui în blânde?e ?i, sculându-se cu grabire de pe tronul sau, a cuprins-o în bra?ele sale, pâna ea ?i-a venit în fire. Apoi a mângâiat-o cu vorbe bune, zicându-i: "Ce ai Estera? Eu sânt fratele tau! Lini?te?te-te, ca nu vei muri, caci stapânirea ne este comuna. Apropie-te!"
\par 2 Apoi regele ?i-a întins spre Estera sceptrul sau cel de aur, care era în mâna sa. Atunci Estera s-a apropiat ?i s-a atins de vârful sceptrului, iar regele i-a pus sceptrul pe grumazul ei ?i a sarutat-o, zicându-i: "Vorbe?te-mi!" ?i ea a zis: "Stapânul meu, eu am vazut în tine parca pe îngerul lui Dumnezeu ?i s-a tulburat inima mea de frica în fa?a slavei tale, ca minunat e?ti stapâne ?i fa?a ta este plina de har". ?i când vorbea ea, a cazut din pricina slabiciunii ?i regele s-a tulburat ?i toate slugile lui o mângâiau.
\par 3 Apoi regele i-a zis: "Ce voie?ti, regina Estera ?i care-?i este cererea? Chiar ?i jumatate din regat ?i se va da".
\par 4 ?i a zis Estera: "Eu am acum zi de sarbatoare. De binevoie?te regele, sa vina cu Aman astazi la ospa?ul pe care i l-am pregatit eu".
\par 5 Iar regele a zis: "Mergi degraba dupa Aman, ca sa se faca dupa cuvântul Esterei". ?i a venit regele cu Aman la ospa?ul ce-l pregatise Estera.
\par 6 La bautura regele a zis catre Estera: "Ce dorin?a ai? Ea ?i se va îndeplini. Care este cererea ta? Ea ?i se va împlini chiar ?i pâna la jumatate din regatul meu!"
\par 7 Atunci Estera a raspuns ?i a zis: "Iata dorin?a ?i cererea mea:
\par 8 De am aflat bunavoin?a în ochii regelui ?i de binevoie?te el sa împlineasca dorin?a mea ?i sa-mi împlineasca cererea, atunci sa vina regele cu Aman mâine la ospa?ul ce-l voi pregati, ?i eu voi da raspuns dupa porunca regelui".
\par 9 În ziua aceea a ie?it Aman vesel ?i voios. Dar când Aman a vazut pe Mardoheu la poarta regelui ?i ca acesta nu s-a sculat macar de la locul sau înaintea lui, s-a umplut de mânie asupra lui Mardoheu.
\par 10 Cu toate acestea Aman s-a stapânit. Ajungând acasa, a trimis sa cheme pe prietenii sai ?i pe Zere?, femeia sa.
\par 11 ?i le-a povestit Aman despre boga?ia sa cea mare, despre mul?imea fiilor sai ?i despre felul cum l-a marit pe el regele ?i cum l-a înal?at peste capeteniile ?i slujitorii sai.
\par 12 "Ba ?i regina Estera, a zis mai departe Aman, pe nimeni afara de mine n-a chemat cu regele la ospa?ul pe care l-a pregatit. Chiar ?i mâine sânt chemat la ea la ospa?.
\par 13 Dar toate acestea nu ma mul?umesc câta vreme vad pe iudeul Mardoheu ?ezând la poarta domneasca".
\par 14 La acestea Zere?, so?ia sa, ?i prietenii sai i-au zis: "Sa se pregateasca spânzuratoare înalta de cincizeci de co?i ?i mâine diminea?a cere regelui sa fie spânzurat Mardoheu ?i apoi vei merge voios la ospa? cu regele". Acest cuvânt a placut lui Aman ?i a pus sa se pregateasca spânzuratoarea.

\chapter{6}

\par 1 În noaptea aceea Domnul a departat somnul de la rege ?i acesta a poruncit slugii sa-i aduca Cronica însemnarilor zilnice ?i ele au fost citite înaintea regelui.
\par 2 Acolo se afla scris ce descoperise Mardoheu regelui cu privire la Bigtan ?i Tere?, cei doi eunuci ai regelui, pazitorii pragului, care uneltisera sa puna mâna pe regele Artaxerxe.
\par 3 Atunci a zis regele: "Ce cinste ?i rasplata s-a dat lui Mardoheu pentru aceasta?" Iar oamenii regelui, care-i slujeau, au zis: "Nu i s-a facut nimic".
\par 4 Când întreba regele de binefacerile lui Mardoheu, a venit Aman în curte, iar regele a zis: "Cine este în curte?" Aman însa venise în curtea de afara a casei regelui sa vorbeasca cu regele, ca sa fie spânzurat Mardoheu în spânzuratoarea pe care i-o pregatise el.
\par 5 ?i oamenii regelui au zis: "Iata Aman sta în curte". Regele a zis: "Sa intre! "
\par 6 Aman a intrat, iar regele i-a zis: "Ce sa se faca omului pe care regele vrea sa-l cinsteasca?>> Aman socotea în inima sa: "Pe cine altul voie?te regele sa cinsteasca, daca nu pe mine?"
\par 7 ?i a zis Aman: "Pentru omul pe care vrea sa-l cinsteasca regele,
\par 8 Sa se aduca îmbracaminte regeasca ?i calul pe care calare?te regele ?i coroana de rege;
\par 9 Apoi sa se dea hainele, coroana ?i calul unuia dintre cei dintâi dregatori ai regelui, ca sa îmbrace pe omul acela, pe care vrea regele sa-l cinsteasca ?i sa-l plimbe calare pe cal prin pia?a ceta?ii ?i sa se strige înaintea lui: A?a se face omului, pe care vrea regele sa-l cinsteasca!"
\par 10 Atunci regele a zis lui Aman: "Bine ai zis! Ia repede haine ?i cal, cum ai spus, ?i fa a?a iudeului Mardoheu, care ?ade la poarta regelui. Sa nu la?i nimic din toate câte ai zis".
\par 11 ?i a luat Aman haine ?i cal, a îmbracat pe Mardoheu ?i l-a plimbat calare pe cal prin pia?a ceta?ii, strigând înaintea lui: "A?a se face omului, pe care regele vrea sa-l cinsteasca!"
\par 12 Apoi Mardoheu s-a întors la poarta regelui, iar Aman s-a dus repede acasa trist ?i cu capul în pamânt.
\par 13 ?i a spus Aman so?iei sale, Zere? ?i tuturor prietenilor sai toate câte se petrecusera cu el. Iar în?elep?ii lui ?i Zere?, femeia sa, i-au zis: "Daca Mardoheu, din pricina caruia a început caderea ta, e din neamul iudeilor, atunci tu nu vei putea face nimic împotriva lui, ci vei cadea sigur înaintea lui, caci cu el este Dumnezeul cel viu>.
\par 14 ?i înca graind ei cu el, au venit eunucii regelui ?i-l grabeau pe Aman sa mearga la ospa?ul pe care-l pregatise Estera.

\chapter{7}

\par 1 În ziua aceea regele cu Aman au venit sa prânzeasca la regina Estera.
\par 2 În timpul ospa?ului, regele a zis iara?i catre Estera: "Care este dorin?a ta, regina Estera? Ca ea î?i va fi împlinita. ?i care este rugamintea ta, ca ?i se va împlini chiar ?i pâna la jumatate din regatul meu".
\par 3 Regina Estera a raspuns: "De am aflat bunavoin?a în ochii tai, o, rege, ?i daca binevoie?te regele, atunci sa ni se daruiasca via?a mie ?i poporului meu, dupa ruga mea.
\par 4 Caci suntem vându?i noi, eu ?i poporul meu, spre ucidere, spre nimicire ?i pieire. De am fi vându?i ca sa fim robi eu a? fi tacut, de?i vrajma?ul nu ar fi acoperit paguba regelui".
\par 5 ?i regele Artaxerxe a raspuns: "Cine este acela ?i unde este omul care a îndraznit a gândi sa faca a?a?
\par 6 Iar Estera a raspuns: "Asupritorul ?i vrajma?ul este rautaciosul Aman!" ?i s-a cutremurat Aman înaintea regelui ?i a reginei.
\par 7 ?i s-a sculat regele de la ospa? plin de mânie ?i s-a dus în gradina palatului; iar Aman a ramas sa roage pe regina Estera, pentru via?a sa, caci vedea bine ca regele hotarâse pieirea lui.
\par 8 Când s-a întors regele din gradina palatului, Aman tocmai se aruncase pe patul pe care se afla Estera. ?i a zis regele: "Vrei înca sa ?i siluie?ti pe regina aici, în casa mea?" Acest cuvânt, ie?it din gura regelui, acoperi de tulburare fa?a lui Aman.
\par 9 Atunci Harbona, unul din eunucii regelui, a zis: "Iata ?i spânzuratoarea pe care a pregatit-o Aman pentru Mardoheu, care a grait de bine pe rege, sta la casa lui Aman, înalta de cincizeci de co?i". Iar regele a zis: "Spânzura?i-l acolo!"
\par 10 ?i au spânzurat pe Aman în spânzuratoarea pregatita de el. Numai a?a s-a potolit mânia regelui.

\chapter{8}

\par 1 În ziua aceea, regele Artaxerxe a dat reginei Estera casa lui Aman, vrajma?ul Iudeilor; iar Mardoheu a fost chemat de rege, caci Estera îi spusese ca el este ruda cu ea.
\par 2 ?i regele ?i-a scos inelul pe care-l luase de la Aman, ?i l-a dat lui Mardoheu; iar Estera a pus pe Mardoheu ispravnic peste casa lui Aman.
\par 3 Apoi Estera a vorbit din nou înaintea regelui, a cazut la picioarele lui, a plâns ?i l-a rugat sa abata rautatea lui Aman Agaghitul ?i uneltirea lui pe care el o îndreptase împotriva Iudeilor.
\par 4 Atunci regele ?i-a întins sceptrul sau cel de aur catre Estera, ?i s-a ridicat Estera ?i a stat înaintea fe?ei regelui,
\par 5 ?i a zis: "De binevoie?te regele ?i de am aflat eu trecere înaintea fe?ei lui; de este drept lucrul acesta înaintea regelui ?i de plac eu ochilor lui, atunci sa se scrie, ca sa fie revocate scrisorile cele trimise dupa uneltirile lui Aman, fiul lui Hamadata, din ?ara Agag, pentru uciderea Iudeilor în toate par?ile regatului;
\par 6 Caci cum a? putea eu sa privesc nenorocirea care ar atinge pe poporul meu ?i cum a? putea sa vad pieirea neamului meu?"
\par 7 Regele Artaxerxe a zis catre Estera ?i Mardoheu: "Casa lui Aman am dat-o Esterei, iar pe el l-am spânzurat, pentru ca ?i-a întins mâna sa asupra Iudeilor.
\par 8 Scrie?i ?i voi despre Iudei ce va place, în numele regelui ?i întari?i cu inelul regelui, caci scrisoarea scrisa în numele regelui ?i întarita cu inelul regelui nu se poate schimba".
\par 9 Atunci au fost chema?i scriitorii regelui, în luna a treia, adica în luna Sivan, în ziua de douazeci ?i trei ale lunii, ?i cum a poruncit Mardoheu, a?a s-a scris Iudeilor ?i satrapilor, guvernatorilor ?arilor ?i cârmuitorilor ?inuturilor lor, de la India pâna la Etiopia, din cele o suta douazeci ?i ?apte ?ari; ?i s-a scris fiecarei ?ari cu scrierea ei ?i fiecarui popor în limba lui, ?i Iudeilor le-a scris cu literele lor ?i în limba lor.
\par 10 Mardoheu a scris în numele regelui Artaxerxe, scrisorile le-a pecetluit cu inelul regelui ?i le-a trimis prin curieri calari, pe caii din hergheliile regelui,
\par 11 Spunând ca regele îngaduie iudeilor din fiecare cetate sa se adune ca sa-?i apere via?a lor, sa bata, sa ucida ?i sa piarda pe to?i cei puternici din popor ?i din ?ara, care ar vrea sa-i atace, cu femeile ?i cu copiii lor, iar averea lor sa o jefuiasca.
\par 12 Aceasta sa se faca în toate ?arile lui Artaxerxe, într-o singura zi, în a treisprezecea zi a lunii a douasprezecea, adica a lunii Adar. Cuprinsul acestei scrisori este urmatorul: "Marele rege Artaxerxe, satrapilor celor o suta douazeci ?i ?apte de rari de la India ?i pâna la Etiopia, guvernatorilor de provincii ?i tuturor supu?ilor, credincio?ilor lui, salutare. Mul?i din cei rasplati?i cu cinste prin nemarginita bunatate a binefacerilor s-au trufit peste masura ?i nu numai supu?ilor no?tri cauta sa le faca rau, ci, neputându-?i satura mândria, încearca sa urzeasca uneltiri ?i împotriva binefacatorilor lor, pierzând nu numai sim?ul recuno?tin?ei omene?ti, ci, plini de trufie nebuna, cauta în chip nelegiuit sa scape ?i de judecata lui Dumnezeu, Cel ce pururea toate le vede. Dar adesea ?i mul?i, fiind îmbraca?i cu putere, ca sa rânduiasca lucrurile prietenilor ce s-au încrezut în ei, prin încredin?arile lor îi fac vinova?i de varsare de sânge nevinovat ?i-i supun la primejdii de neînlaturat, amagind gândul bun ?i neprihanit al stapânilor prin vorbarie vicleana ?i mincinoasa. Aceasta se poate vedea atât din povestirile mai vechi, cum am spus, precum ?i din faptele savâr?ite în chip nelegiuit înaintea ochilor vo?tri de rautatea celor ce stapânesc cu nevrednicie. De aceea ne îngrijim pentru viitor, ca sa întocmim noi o împara?ie netulburata pentru to?i oamenii din lume, neîngaduind în?elatoriile, ?i pricinile ce ni se înfa?i?eaza judecându-le cu toata luarea aminte cuvenita. A?a Aman, fiul lui Hamadata Macedoneanul, cu adevarat strain de sângele persan ?i foarte departe de bunatatea noastra, fiind primit oaspete la noi, s-a învrednicit de bunavoin?a pe care noi o avem catre orice popor, pâna într-atâta, încât a fost numit parintele nostru ?i cinstit de to?i, înfa?i?ându-se ca a doua persoana dupa tronul regal. Dar fiind de o mândrie nemasurata, a uneltit sa ne lipseasca pe noi de putere ?i de via?a, iar pe salvatorul ?i pururea binefacatorul nostru Mardoheu ?i pe nevinovata parta?a la regalitatea noastra, Estera, cu tot poporul lor, se silea prin felurite masuri viclene sa-i piarda. Astfel socotea el sa ne lase fara oameni, iar împara?ia persana sa o dea Macedonenilor. Noi însa pe Iudei, osândi?i de acest facator de rele, la moarte, îi gasim nu raufacatori, ci oameni care traiesc dupa cele mai drepte legiuiri ?i fii ai Dumnezeului celui viu, mare ?i preaînalt, Care ne-a daruit noua ?i stramo?ilor no?tri împara?ie preafericita. De aceea bine ve?i face, de nu ve?i aduce la îndeplinire scrisorile trimise de Aman al lui Hamadata, caci el, facând aceasta, a fost spânzurat la por?ile Suzei cu toata casa, dupa cuvântul lui Dumnezeu, Celui ce stapâne?te peste to?i, Care i-a dat lui curând osânda cuvenita. Iar o copie a acestei scrisori pune?i-o la vedere în tot locul, sa lasa?i pe Iudei sa se foloseasca de legile lor ?i sa-i ajuta?i, ca pe cei ce se vor scula asupra lor la vreme de necaz sa-i poata rapune în ziua de treisprezece ale lunii a douasprezecea, Adar, chiar în ziua aceea. Caci Dumnezeu, Cel ce stapâne?te peste to?i, în loc sa piarda pe poporul cel ales, i-a rânduit lui aceasta bucurie. ?i voi, Iudeilor, între sarbatorile voastre vestite, sarbatori?i ?i aceasta zi însemnata cu toata veselia, ca sa fie de acum ?i sa ramâna în viitor pentru voi ?i pentru Per?ii binevoitori amintirea izbavirii voastre, iar pentru du?manii vo?tri sa fie amintirea pieirii lor. Fiecare cetate, sau ?inut îndeob?te, care nu se va conforma, se va pustii fara cru?are cu sabie ?i foc ?i va ajunge nu numai nelocuita de oameni totdeauna, ci dezgustatoare pentru fiare ?i pasari.
\par 13 Copii ale acestei scrisori sa se dea în fiecare ?ara, ca lege, astfel ca Iudeii sa fie gata pentru ziua aceea sa se razbune pe vrajma?ii lor".
\par 14 Curieri, calari pe cai iu?i din hergheliile regelui, au alergat repede ?i cu mare graba, cu porunca regelui. Porunca a fost vestita ?i în cetatea Suza.
\par 15 Mardoheu a ie?it de la rege în ve?minte rege?ti de culoare purpurie ?i alba ?i cu cununa mare de aur, iar cetatea Suza s-a bucurat ?i s-a veselit.
\par 16 La Iudei a fost atunci lumina în case, bucurie, veselie ?i mare praznuire.
\par 17 De asemenea în toate ?arile, prin toate ceta?ile ?i în toate locurile unde ajunsese scrisoarea cu porunca regelui, a fost bucurie, veselie, ospe?e ?i praznuire la Iudei. Chiar ?i dintre popoarele ?arii mul?i se facura Iudei, pentru ca-i cuprinsese frica de Iudei.

\chapter{9}

\par 1 În luna a douasprezecea, adica în luna Adar, în a treisprezecea zi, când a sosit timpul aducerii la îndeplinire a poruncii regelui ?i a decretului lui; în ziua aceea, când nadajduiau du?manii Iudeilor sa-i biruiasca, s-a dovedit dimpotriva ca Iudeii ?i-au aratat puterea asupra du?manilor lor;
\par 2 Atunci Iudeii s-au adunat în toate ceta?ile lor, prin toate ?arile lui Artaxerxe, ca sa puna mâna pe cei ce le doreau raul ?i nimeni n-a putut sta împotriva fe?ei lor, pentru ca frica de Iudei apasa asupra tuturor popoarelor.
\par 3 Toate capeteniile ?arilor, satrapii, guvernatorii ?i slujitorii regelui au sprijinit pe Iudei, de teama lui Mardoheu,
\par 4 Caci Mardoheu era mare în casa regelui ?i renumele lui se la?ise în toate ?arile, deoarece omul acesta, Mardoheu, devenise tot mai puternic.
\par 5 Atunci au stârpit Iudeii pe to?i du?manii lor, ucigând cu sabia, omorând, pierzând ?i facând cu du?manii lor dupa voia lor.
\par 6 În cetatea Suza, au ucis Iudeii cinci sute de oameni.
\par 7 Între ace?tia au ucis ?i pe Par?andata, pe Dalfon, pe Aspata,
\par 8 Pe Porata, pe Adalia, pe Aridata,
\par 9 Pe Parma?ta, pe Arisai, pe Aridai ?i pe Iezata,
\par 10 Adica pe cei zece feciori ai lui Aman, fiul lui Hamadata, vrajma?ul Iudeilor; dar ei nu i-au pradat de averile lor.
\par 11 În acea zi s-a adus la cuno?tin?a regelui numarul celor uci?i în cetatea Suza.
\par 12 Atunci regele a zis catre regina Estera: "În cetatea Suza, au ucis Iudeii cinci sute de oameni ?i pe cei zece fii ai lui Aman. Ce vor fi facut ei în celelalte ?ari ale regelui? Care este dorin?a ta? Ea ?i se va împlini. ?i ce dorin?a mai ai? Ca ea î?i va fi împlinita".
\par 13 Estera a raspuns: "De binevoie?te regele, sa se îngaduie Iudeilor celor din Suza sa faca acela?i lucru ?i mâine, pe care l-au facut astazi, iar pe cei zece fii ai lui Aman sa-i spânzure".
\par 14 ?i a poruncit regele sa se faca a?a; s-a dat porunca în Suza ?i pe cei zece fii ai lui Aman i-au spânzurat.
\par 15 ?i s-au adunat Iudeii cei din Suza ?i în ziua a paisprezecea a lunii lui Adar ?i au omorât trei sute de oameni, dar la jaf nu ?i-au întins mâinile.
\par 16 Iar ceilal?i Iudei care se aflau în ?arile regelui s-au adunat ca sa-?i apere via?a lor ?i sa fie netulbura?i de vrajma?ii lor. Ace?tia au omorât din du?manii lor ?aptezeci ?i cinci de mii, iar la jaf nu ?i-au întins mâinile.
\par 17 Aceasta s-a petrecut în treisprezece ale lunii lui Adar. În ziua de paisprezece a aceleia?i luni s-au lini?tit ?i au facut-o zi de ospa? ?i de veselie.
\par 18 Iudeii însa, care se aflau în Suza, s-au adunat în ziua de treisprezece ?i de paisprezece ale lunii lui Adar, iar în cincisprezece ale ei s-au lini?tit ?i au facut-o zi de ospa? ?i de veselie.
\par 19 De aceea Iudeii din provincie, care locuiesc în sate neîntarite, petrec ziua de paisprezece ale lunii Adar în veselie ?i ospe?e, ca zi de sarbatoare, trimi?ându-?i daruri unii altora; iar cei ce traiesc în ora?e petrec ?i ziua de cincisprezece ale lunii Adar în mare veselie, trimi?ând daruri vecinilor.
\par 20 Dupa aceea Mardoheu a scris toate întâmplarile acestea ?i a trimis scrisori tuturor Iudeilor care erau în ?arile regelui Artaxerxe, la cei de aproape ?i la cei de departe,
\par 21 Ca sa sarbatoreasca acele zile bune în fiecare an, în ziua de paisprezece ?i de cincisprezece ale lunii Adar,
\par 22 Întrucât acestea sânt zilele în care Iudeii au fost lasa?i în pace de vrajma?ii lor ?i întrucât aceasta este luna în care întristarea lor s-a prefacut în bucurie ?i tânguirea în zi de sarbatoare. Sa faca dar din ele zile de petrecere ?i de veselie, trimi?ându-?i unii altora daruri ?i dând milostenie la saraci.
\par 23 ?i au primit Iudeii cele ce le scrisese Mardoheu:
\par 24 Cum Aman, fiul lui Hamadata, din ?ara Agag, vrajma?ul tuturor Iudeilor, se gândise sa-i ucida ?i aruncase Pur, adica sori, pentru pierderea lor;
\par 25 Cum Estera a strabatut pâna la rege ?i cum regele a poruncit prin scrisoare noua, ca uneltirea cea rea a lui Aman, pe care o planuise el asupra Iudeilor, sa se întoarca asupra capului lui ?i sa-l spânzure pe el ?i pe fiii lui.
\par 26 De aceea s-au ?i numit aceste zile Purim, de la numirea Pur. Deci, potrivit cu toate cuvintele scrisorii ?i cu toate cele ce ei vazusera ?i cu cele ce se petrecusera la ei,
\par 27 Iudeii au stabilit ?i au primit pentru ei ?i pentru urma?ii lor ?i pentru cei ce li se vor alatura, ca sa sarbatoreasca fara abatere aceste doua zile în fiecare an dupa rânduiala,
\par 28 ?i ca zilele acestea ale Purimului sa fie pomenite ?i praznuite din neam în neam, în fiecare familie, la Iudei, în fiecare ?ara ?i în fiecare cetate.
\par 29 Regina Estera, fiica lui Abihail ?i Mardoheu Iudeul au scris a doua oara în chip staruitor cele ce au facut ei, ca sa întareasca scrisoarea despre Purim,
\par 30 ?i au trimis scrisori tuturor Iudeilor din cele o suta douazeci ?i ?apte de ?ari ale regatului lui Artaxerxe cu cuvinte de pace ?i de credincio?ie,
\par 31 Ca ei sa pazeasca cu tarie aceste zile ale Purimului la vremea lor, precum le statornicisera Mardoheu Iudeul ?i regina Estera, pentru ei ?i pentru urma?ii lor cu post ?i tânguire.
\par 32 ?i porunca Esterei întari a?ezarea sarbatorii Purimului scriind-o în carte.

\chapter{10}

\par 1 Dupa aceea regele Artaxerxe puse bir pe ?ari ?i pe insulele marii.
\par 2 De altfel toate faptele slavite ?i atotputernicia lui, cum ?i aratarea amanun?ita a maririi lui Mardoheu cu care-l cinstise regele, sânt scrise în cartea Cronicilor regilor Mediei ?i Persiei.
\par 3 De asemenea e aratat acolo ca iudeul Mardoheu era al doilea dupa regele Artaxerxe, mare înaintea Iudeilor ?i iubit între mul?imile fra?ilor sai, caci cauta binele poporului sau ?i vorbea în folosul neamului lui. ?i zicea Mardoheu: "De la Dumnezeu a fost aceasta, ca eu mi-am adus aminte de visul pe care l-am visat despre aceste întâmplari, ca n-a ramas neîmplinit nimic din el. Izvorul cel mic s-a facut râu mare ?i a fost lumina, soare ?i mul?ime de apa; acest râu este Estera, pe care ?i-a luat-o de femeie regele ?i a facut-o regina. Iar cei doi balauri sânt eu ?i Aman. Popoarele sânt cei ce s-au unit sa stârpeasca numele Iudeilor; iar poporul meu sânt Israeli?ii, care au strigat catre Dumnezeu ?i au fost izbavi?i. A izbavit Dumnezeu pe poporul Sau ?i ne-a izbavit Domnul din toate relele acestea. Savâr?it-a Dumnezeu semne ?i minuni mari, cum n-au fost printre neamuri. A?a a rânduit Dumnezeu doi sor?i: unul pentru poporul lui Dumnezeu, iar celalalt pentru neamuri. Ace?ti doi sorii au ie?it în ceasul, la vremea ?i în ziua judeca?ii înaintea lui Dumnezeu ?i a tuturor neamurilor. Atunci ?i-a adus aminte Domnul de poporul Sau ?i a dat dreptate mo?tenirii Sale. Aceste zile ale lunii lui Adar, adica a paisprezecea ?i a cincisprezecea ale acestei luni, se vor praznui ve?nic cu alai, cu bucurie ?i veselie înaintea lui Dumnezeu în poporul Sau Israel. în anul al patrulea al domniei lui Ptolomeu ?i a Cleopatrei, Dositei, care se spune ca a fost preot ?i levit ?i Ptolomeu, fiul sau, au adus în Alexandria aceasta scrisoare despre Purim. Aceasta scrisoare se spune ca a tâlcuit-o Lisimah, fiul lui Ptolomeu, care fusese la Ierusalim.


\end{document}