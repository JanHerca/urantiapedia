\begin{document}

\title{1 Timotei}


\chapter{1}

\par 1 Pavel, apostol al lui Iisus Hristos, după porunca lui Dumnezeu, Mântuitorul nostru, și a lui Iisus Hristos, nădejdea noastră,
\par 2 Lui Timotei, adevărat fiu în credință: Har, milă, pace, de la Dumnezeu-Tatăl și de la Hristos Iisus, Domnul nostru.
\par 3 Când am plecat în Macedonia, te-am îndemnat să rămâi în Efes, ca să poruncești unora să nu învețe o altă învățătură,
\par 4 Nici să ia aminte la basme și la nesfârșite înșirări de neamuri, care aduc mai degrabă certuri, decât lucrarea mântuitoare a lui Dumnezeu, cea întru credință;
\par 5 Iar ținta poruncii este dragostea din inimă curată, din cuget bun și din credință nefățarnică,
\par 6 De la care unii rătăcind s-au întors spre deșartă vorbire,
\par 7 Voind să fie învățători ai Legii, dar neînțelegând nici cele ce spun, nici cele pentru care dau adeverire.
\par 8 Noi știm că legea este bună, dacă se folosește cineva de ea potrivit legii;
\par 9 Știind aceasta, că legea nu este pusă pentru cel drept, ci pentru cei fără de lege și răzvrătiți, pentru necredincioși și păcătoși, pentru necuvioși și spurcați, pentru ucigașii de tată și ucigașii de mamă, pentru omorâtorii de oameni,
\par 10 Pentru desfrânați, pentru sodomiți, pentru vânzătorii de oameni, pentru mincinoși, pentru cei care jură strâmb și pentru tot ce stă împotriva sănătoase,
\par 11 După Evanghelia slavei fericitului Dumnezeu, cea încredințată mie.
\par 12 Mulțumesc Celui ce m-a întărit, lui Hristos Iisus, Domnul nostru, că m-a socotit credincios și m-a pus să-I slujesc,
\par 13 Pe mine, care mai înainte huleam, prigoneam și batjocoream. Totuși am fost miluit, căci în necredința mea, am lucrat din neștiință.
\par 14 Și a prisosit foarte harul Domnului nostru, împreună cu credința și cu dragostea cea întru Hristos Iisus.
\par 15 Vrednic de credință și de toată primirea e cuvântul că Iisus Hristos a venit în lume ca să mântuiască pe cei păcătoși, dintre care cel dintâi sunt eu.
\par 16 Și tocmai pentru aceea am fost miluit, ca Iisus Hristos să arate mai întâi în mine toată îndelunga Sa răbdare, ca pildă celor ce vor crede în El, spre viața veșnică.
\par 17 Iar Împăratul veacurilor, Celui nestricăcios, nevăzutului, singurului Dumnezeu fie cinste și slavă în vecii vecilor. Amin!
\par 18 Această poruncă îți încredințez, fiule Timotei, ca potrivit proorociilor făcute mai înainte asupra ta, să te lupți lupta cea bună, după cuvântul lor,
\par 19 Având credință și cuget bun, pe care unii, lepădându-le, au căzut din credință;
\par 20 Dintre aceștia sunt Imeneu și Alexandru, pe care i-am dat satanei, ca să se învețe să nu hulească.

\chapter{2}

\par 1 Vă îndemn deci, înainte de toate, să faceți cereri, rugăciuni, mijlociri, mulțumiri, pentru toți oamenii,
\par 2 Pentru împărați și pentru toți care sunt în înalte dregătorii, ca să petrecem viață pașnică și liniștită întru toată cuvioșia și buna-cuviință,
\par 3 Că acesta este lucru bun și primit înaintea lui Dumnezeu, Mântuitorul nostru,
\par 4 Care voiește ca toți oamenii să se mântuiască și la cunoștința adevărului să vină.
\par 5 Căci unul este Dumnezeu, unul este și Mijlocitorul între Dumnezeu și oameni: omul Hristos Iisus,
\par 6 Care S-a dat pe Sine preț de răscumpărare pentru toți, mărturia adusă la timpul său.
\par 7 Spre aceasta am fost pus propovăduitor și apostol (adevăr grăiesc în Hristos, nu mint) - învățător neamurilor, în credință și adevăr.
\par 8 Vreau deci ca bărbații să se roage în tot locul, ridicând mâini sfinte, fără de mânie și fără șovăire.
\par 9 Asemenea și femeile, în îmbrăcăminte cuviincioasă, făcându-și lor podoabă din sfială și din cumințenie, nu din păr împletit și din aur, sau din mărgăritare, sau din veșminte de mult preț;
\par 10 Ci, din fapte bune, precum se cuvine unor femei temătoare de Dumnezeu.
\par 11 Femeia să se învețe în liniște, cu toată ascultarea.
\par 12 Nu îngăduiesc femeii nici să învețe pe altul, nici să stăpânească pe bărbat, ci să stea liniștită.
\par 13 Căci Adam a fost zidit întâi, apoi Eva.
\par 14 Și nu Adam a fost amăgit, ci femeia, amăgită fiind, s-a făcut călcătoare de poruncă.
\par 15 Dar ea se va mântui prin naștere de fii, dacă va stărui, cu înțelepciune, în credință, în iubire și în sfințenie.

\chapter{3}

\par 1 Vrednic de crezare, este cuvântul: de poftește cineva episcopie, bun lucru dorește.
\par 2 Se cuvine, dar, ca episcopul să fie fără de prihană, bărbat al unei singure femei, veghetor, înțelept, cuviincios, iubitor de străini, destoinic să învețe pe alții,
\par 3 Nebețiv, nedeprins să bată, neagonisitor de câștig urât, ci blând, pașnic, neiubitor de argint,
\par 4 Bine chivernisind casa lui, având copii ascultători, cu toată bună-cuviința;
\par 5 Căci dacă nu știe cineva să-și rânduiască propria lui casă, cum va purta grijă de Biserica lui Dumnezeu?
\par 6 Episcopul să nu fie de curând botezat, ca nu cumva, trufindu-se, să cadă în osânda diavolului.
\par 7 Dar el trebuie să aibă și mărturie bună de la cei din afară, ca să nu cadă în ocară și în cursa diavolului.
\par 8 Diaconii, de asemenea, trebuie să fie cucernici, nu vorbind în două feluri, nu dedați la vin mult, neagonisitori de câștig urât,
\par 9 Păstrând taina credinței în cuget curat.
\par 10 Dar și aceștia să fie mai întâi puși la încercare, apoi, dacă se dovedesc fără prihană, să fie diaconiți.
\par 11 Femeile (lor) de asemenea să fie cuviincioase, neclevetitoare, cumpătate, credincioase întru toate.
\par 12 Diaconii să fie bărbați ai unei singure femei, să-și chivernisească bine casele și pe copiii lor.
\par 13 Căci cei ce slujesc bine, rang bun dobândesc și mult curaj în credința cea întru Hristos Iisus.
\par 14 Îți scriu aceasta nădăjduind că voi veni la tine fără întârziere;
\par 15 Ca să știi, dacă zăbovesc, cum trebuie să petreci în casa lui Dumnezeu, care este Biserica Dumnezeului celui viu, stâlp și temelie a adevărului.
\par 16 Și cu adevărat, mare este taina dreptei credințe: Dumnezeu S-a arătat în trup, S-a îndreptat în Duhul, a fost văzut de îngeri, S-a propovăduit între neamuri, a fost crezut în lume, S-a înălțat întru slavă.

\chapter{4}

\par 1 Dar Duhul grăiește lămurit că, în vremurile cele de apoi, unii se vor depărta de la credință, luând aminte la duhurile cele înșelătoare și la învățăturile demonilor,
\par 2 Prin fățărnicia unor mincinoși, care sunt înfierați în cugetul lor.
\par 3 Aceștia opresc de la căsătorie și de la unele bucate, pe care Dumnezeu le-a făcut, spre gustare cu mulțumire, pentru cei credincioși și pentru cei ce au cunoscut adevărul,
\par 4 Pentru că orice făptură a lui Dumnezeu este bună și nimic nu este de lepădat, dacă se ia cu mulțumire;
\par 5 Căci se sfințește prin cuvântul lui Dumnezeu și prin rugăciune.
\par 6 Punându-le înaintea fraților acestea, vei fi bun slujitor al lui Hristos Iisus, hrănindu-te cu cuvintele credinței și ale bunei învățături căreia ai urmat;
\par 7 Iar de basmele cele lumești și băbești, ferește-te și deprinde-te cu dreapta credință.
\par 8 Căci deprinderea trupească la puțin folosește, dar dreapta credință spre toate este de folos, având făgăduința vieții de acum și a celei ce va să vină.
\par 9 Vrednic de credință este acest cuvânt și vrednic de toată primirea,
\par 10 Fiindcă pentru aceasta ne și ostenim și suntem ocărâți și ne luptăm, căci ne-am pus nădejdea în Dumnezeul cel viu, Care este Mântuitorul tuturor oamenilor, mai ales al credincioșilor.
\par 11 Acestea să le poruncești și să-i înveți.
\par 12 Nimeni să nu disprețuiască tinerețile tale, ci fă-te pildă credincioșilor cu cuvântul, cu purtarea, cu dragostea, cu duhul, cu credința, cu curăția.
\par 13 Până voi veni eu, ia aminte la citit, la îndemnat, la învățătură.
\par 14 Nu fi nepăsător față de harul care este întru tine, care ți s-a dat prin proorocie, cu punerea mâinilor mai-marilor preoților.
\par 15 Cugetă la acestea, ține-te de acestea, ca propășirea ta să fie vădită tuturor.
\par 16 Ia aminte la tine însuți și la învățătură; stăruie în acestea, căci, făcând aceasta, și pe tine te vei mântui și pe cei care te ascultă.

\chapter{5}

\par 1 Pe cel bătrân să nu-l înfrunți, ci să-l îndemni ca pe un părinte; pe cei tineri, ca pe frați.
\par 2 Pe femeile bătrâne îndeamnă-le ca pe niște mame, pe cele tinere ca pe surori, în toată curăția.
\par 3 Pe văduve cinstește-le, dar pe cele cu adevărat văduve.
\par 4 Dacă vreo văduvă are copii sau nepoți, aceștia să se învețe să cinstească mai întâi casa lor și să dea răsplătire părinților, pentru că lucrul acesta este bun și primit înaintea lui Dumnezeu.
\par 5 Cea cu adevărat văduvă și rămasă singură are nădejdea în Dumnezeu și stăruiește în cereri și în rugăciuni, noaptea și ziua.
\par 6 Iar cea care trăiește în desfătări, deși e vie, e moartă.
\par 7 Și acestea poruncește-le, ca ele să fie fără de prihană.
\par 8 Dacă însă cineva nu poartă grijă de ai săi și mai ales de casnicii săi, s-a lepădat de credință și este mai rău decât un necredincios.
\par 9 Să fie înscrisă între văduve cea care nu are mai puțin de șaizeci de ani și a fost femeia unui singur bărbat;
\par 10 Dacă are mărturie de fapte bune: dacă a crescut copii, dacă a fost primitoare de străini, dacă a spălat picioarele sfinților, dacă a venit în ajutorul celor strâmtorați, dacă s-a ținut stăruitor de tot ce este lucru bun.
\par 11 Iar de văduvele tinere ferește-te. Căci, atunci când poftele le îndepărtează de Hristos, vor să se mărite.
\par 12 Și își agonisesc osândă, fiindcă și-au călcat credința cea dintâi.
\par 13 Dar în același timp se învață să fie leneșe, cutreierând casele, și nu numai leneșe, ci și guralive și iscoditoare, grăind cele ce nu se cuvin.
\par 14 Vreau deci ca văduvele tinere să se mărite, să aibă copii, să-și vadă de case, și să nu dea potrivnicului nici un prilej de ocară.
\par 15 Căci unele s-au și abătut, ca să se ducă după satana.
\par 16 Dacă vreun credincios sau vreo credincioasă are în casă văduve, să aibă grija lor, ca Biserica să nu fie împovărată, ci să poată ajuta pe cele cu adevărat văduve.
\par 17 Preoții, care își țin bine dregătoria, să se învrednicească de îndoită cinste, mai ales cei care se ostenesc cu cuvântul și cu învățătura.
\par 18 Pentru că Scriptura zice: "Să nu legi gura boului care treieră", și: "Vrednic este lucrătorul de plata sa".
\par 19 Pâră împotriva preotului să nu primești, fără numai de la doi sau trei martori.
\par 20 Pe cei ce păcătuiesc mustră-i de față cu toți, ca și ceilalți să aibă teamă.
\par 21 Te îndemn stăruitor înaintea lui Dumnezeu și a lui Iisus Hristos și a îngerilor aleși, ca să păzești acestea, fără a lua o hotărâre dinainte, nefăcând nimic cu părtinire.
\par 22 Nu-ți pune mâinile degrabă pe nimeni, nici nu te face părtaș la păcatele altora. Păstrează-te curat.
\par 23 De acum nu bea numai apă, ci folosește puțin vin, pentru stomacul tău și pentru desele tale slăbiciuni.
\par 24 Păcatele unor oameni sunt vădite, mergând înaintea lor la judecată, ale altora însă vin în urma lor.
\par 25 Tot așa și faptele cele bune sunt vădite, și cele ce nu sunt altfel nu se pot ascunde.

\chapter{6}

\par 1 Cei ce se găsesc sub jugul robiei să socotească pe stăpânii lor vrednici de toată cinstea, ca să nu fie hulite numele și învățătura lui Dumnezeu.
\par 2 Iar cei ce au stăpâni credincioși să nu-i disprețuiască, sub cuvânt că sunt frați; ci mai mult să-i slujească, fiindcă primitorii bunei lor slujiri sunt credincioși și iubiți. Acestea învață-i și poruncește-le.
\par 3 Iar de învață cineva altă învățătură și nu se ține de cuvintele cele sănătoase ale Domnului nostru Iisus Hristos și de învățătura cea după dreapta credință,
\par 4 Acela e un îngâmfat, care nu știe nimic, suferind de boala discuțiilor și a certurilor de cuvinte, din care pornesc: ceartă, pizmă, defăimări, bănuieli viclene,
\par 5 Gâlcevi necurmate ale oamenilor stricați la minte și lipsiți de adevăr, care socotesc că evlavia este un mijloc de câștig. Depărtează-te de unii ca aceștia.
\par 6 Și, în adevăr, evlavia este mare câștig, dar atunci când ea se îndestulează cu ce are.
\par 7 Pentru că noi n-am adus nimic în lume, tot așa cum nici nu putem să scoatem ceva din ea afară;
\par 8 Ci, având hrană și îmbrăcăminte cu acestea vom fi îndestulați.
\par 9 Cei ce vor să se îmbogățească, dimpotrivă, cad în ispită și în cursă și în multe pofte nebunești și vătămătoare, ca unele care cufundă pe oameni în ruină și în pierzare.
\par 10 Că iubirea de argint este rădăcina tuturor relelor și cei ce au poftit-o cu înfocare au rătăcit de la credință, și s-au străpuns cu multe dureri.
\par 11 Dar tu, o, omule al lui Dumnezeu, fugi de acestea și urmează dreptatea, evlavia, credința, dragostea, răbdarea, blândețea.
\par 12 Luptă-te lupta cea bună a credinței, cucerește viața veșnică la care ai fost chemat și pentru care ai dat bună mărturie înaintea multor martori.
\par 13 Îți poruncesc înaintea lui Dumnezeu, Cel ce aduce toate la viață, și înaintea lui Iisus Hristos, Cel ce, în fața lui Ponțiu Pilat, a mărturisit mărturisirea cea bună:
\par 14 Să păzești porunca fără pată, fără vină, până la arătarea Domnului nostru Iisus Hristos,
\par 15 Pe care, la timpul cuvenit, o va arăta fericitul și singurul Stăpânitor, Împăratul împăraților și Domnul domnilor,
\par 16 Cel ce singur are nemurire și locuiește întru lumină neapropiată; pe Care nu L-a văzut nimeni dintre oameni, nici nu poate să-L vadă; a Căruia este cinstea și puterea veșnică! Amin.
\par 17 Celor bogați în veacul de acum poruncește-le să nu se semețească, nici să-și pună nădejdea în bogăția cea nestatornică, ci în Dumnezeul cel viu, Care ne dă cu belșug toate, spre îndulcirea noastră,
\par 18 Să facă ce e bine, să se înavuțească în fapte bune, să fie darnici, să fie cu inimă largă,
\par 19 Agonisindu-și lor bună temelie în veacul viitor, ca să dobândească, cu adevărat, viața veșnică.
\par 20 O, Timotei, păzește comoara ce ți s-a încredințat, depărtându-te de vorbirile deșarte și lumești și de împotrivirile științei mincinoase,
\par 21 Pe care unii, mărturisind-o, au rătăcit de la credință. Harul fie cu tine! Amin.


\end{document}