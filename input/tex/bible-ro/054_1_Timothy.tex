\begin{document}

\title{1 Timothy}

1Ti 1:1  Pavel, apostol al lui Iisus Hristos, dupa porunca lui Dumnezeu, Mântuitorul nostru, ?i a lui Iisus Hristos, nadejdea noastra,
1Ti 1:2  Lui Timotei, adevarat fiu în credin?a: Har, mila, pace, de la Dumnezeu-Tatal ?i de la Hristos Iisus, Domnul nostru.
1Ti 1:3  Când am plecat în Macedonia, te-am îndemnat sa ramâi în Efes, ca sa porunce?ti unora sa nu înve?e o alta înva?atura,
1Ti 1:4  Nici sa ia aminte la basme ?i la nesfâr?ite în?irari de neamuri, care aduc mai degraba certuri, decât lucrarea mântuitoare a lui Dumnezeu, cea întru credin?a;
1Ti 1:5  Iar ?inta poruncii este dragostea din inima curata, din cuget bun ?i din credin?a nefa?arnica,
1Ti 1:6  De la care unii ratacind s-au întors spre de?arta vorbire,
1Ti 1:7  Voind sa fie înva?atori ai Legii, dar neîn?elegând nici cele ce spun, nici cele pentru care dau adeverire.
1Ti 1:8  Noi ?tim ca legea este buna, daca se folose?te cineva de ea potrivit legii;
1Ti 1:9  ?tiind aceasta, ca legea nu este pusa pentru cel drept, ci pentru cei fara de lege ?i razvrati?i, pentru necredincio?i ?i pacato?i, pentru necuvio?i ?i spurca?i, pentru uciga?ii de tata ?i uciga?ii de mama, pentru omorâtorii de oameni,
1Ti 1:10  Pentru desfrâna?i, pentru sodomi?i, pentru vânzatorii de oameni, pentru mincino?i, pentru cei care jura strâmb ?i pentru tot ce sta împotriva sanatoase,
1Ti 1:11  Dupa Evanghelia slavei fericitului Dumnezeu, cea încredin?ata mie.
1Ti 1:12  Mul?umesc Celui ce m-a întarit, lui Hristos Iisus, Domnul nostru, ca m-a socotit credincios ?i m-a pus sa-I slujesc,
1Ti 1:13  Pe mine, care mai înainte huleam, prigoneam ?i batjocoream. Totu?i am fost miluit, caci în necredin?a mea, am lucrat din ne?tiin?a.
1Ti 1:14  ?i a prisosit foarte harul Domnului nostru, împreuna cu credin?a ?i cu dragostea cea întru Hristos Iisus.
1Ti 1:15  Vrednic de credin?a ?i de toata primirea e cuvântul ca Iisus Hristos a venit în lume ca sa mântuiasca pe cei pacato?i, dintre care cel dintâi sunt eu.
1Ti 1:16  ?i tocmai pentru aceea am fost miluit, ca Iisus Hristos sa arate mai întâi în mine toata îndelunga Sa rabdare, ca pilda celor ce vor crede în El, spre via?a ve?nica.
1Ti 1:17  Iar Împaratul veacurilor, Celui nestricacios, nevazutului, singurului Dumnezeu fie cinste ?i slava în vecii vecilor. Amin!
1Ti 1:18  Aceasta porunca î?i încredin?ez, fiule Timotei, ca potrivit proorociilor facute mai înainte asupra ta, sa te lup?i lupta cea buna, dupa cuvântul lor,
1Ti 1:19  Având credin?a ?i cuget bun, pe care unii, lepadându-le, au cazut din credin?a;
1Ti 1:20  Dintre ace?tia sunt Imeneu ?i Alexandru, pe care i-am dat satanei, ca sa se înve?e sa nu huleasca.
1Ti 2:1  Va îndemn deci, înainte de toate, sa face?i cereri, rugaciuni, mijlociri, mul?umiri, pentru to?i oamenii,
1Ti 2:2  Pentru împara?i ?i pentru to?i care sunt în înalte dregatorii, ca sa petrecem via?a pa?nica ?i lini?tita întru toata cuvio?ia ?i buna-cuviin?a,
1Ti 2:3  Ca acesta este lucru bun ?i primit înaintea lui Dumnezeu, Mântuitorul nostru,
1Ti 2:4  Care voie?te ca to?i oamenii sa se mântuiasca ?i la cuno?tin?a adevarului sa vina.
1Ti 2:5  Caci unul este Dumnezeu, unul este ?i Mijlocitorul între Dumnezeu ?i oameni: omul Hristos Iisus,
1Ti 2:6  Care S-a dat pe Sine pre? de rascumparare pentru to?i, marturia adusa la timpul sau.
1Ti 2:7  Spre aceasta am fost pus propovaduitor ?i apostol (adevar graiesc în Hristos, nu mint) - înva?ator neamurilor, în credin?a ?i adevar.
1Ti 2:8  Vreau deci ca barba?ii sa se roage în tot locul, ridicând mâini sfinte, fara de mânie ?i fara ?ovaire.
1Ti 2:9  Asemenea ?i femeile, în îmbracaminte cuviincioasa, facându-?i lor podoaba din sfiala ?i din cumin?enie, nu din par împletit ?i din aur, sau din margaritare, sau din ve?minte de mult pre?;
1Ti 2:10  Ci, din fapte bune, precum se cuvine unor femei tematoare de Dumnezeu.
1Ti 2:11  Femeia sa se înve?e în lini?te, cu toata ascultarea.
1Ti 2:12  Nu îngaduiesc femeii nici sa înve?e pe altul, nici sa stapâneasca pe barbat, ci sa stea lini?tita.
1Ti 2:13  Caci Adam a fost zidit întâi, apoi Eva.
1Ti 2:14  ?i nu Adam a fost amagit, ci femeia, amagita fiind, s-a facut calcatoare de porunca.
1Ti 2:15  Dar ea se va mântui prin na?tere de fii, daca va starui, cu în?elepciune, în credin?a, în iubire ?i în sfin?enie.
1Ti 3:1  Vrednic de crezare, este cuvântul: de pofte?te cineva episcopie, bun lucru dore?te.
1Ti 3:2  Se cuvine, dar, ca episcopul sa fie fara de prihana, barbat al unei singure femei, veghetor, în?elept, cuviincios, iubitor de straini, destoinic sa înve?e pe al?ii,
1Ti 3:3  Nebe?iv, nedeprins sa bata, neagonisitor de câ?tig urât, ci blând, pa?nic, neiubitor de argint,
1Ti 3:4  Bine chivernisind casa lui, având copii ascultatori, cu toata buna-cuviin?a;
1Ti 3:5  Caci daca nu ?tie cineva sa-?i rânduiasca propria lui casa, cum va purta grija de Biserica lui Dumnezeu?
1Ti 3:6  Episcopul sa nu fie de curând botezat, ca nu cumva, trufindu-se, sa cada în osânda diavolului.
1Ti 3:7  Dar el trebuie sa aiba ?i marturie buna de la cei din afara, ca sa nu cada în ocara ?i în cursa diavolului.
1Ti 3:8  Diaconii, de asemenea, trebuie sa fie cucernici, nu vorbind în doua feluri, nu deda?i la vin mult, neagonisitori de câ?tig urât,
1Ti 3:9  Pastrând taina credin?ei în cuget curat.
1Ti 3:10  Dar ?i ace?tia sa fie mai întâi pu?i la încercare, apoi, daca se dovedesc fara prihana, sa fie diaconi?i.
1Ti 3:11  Femeile (lor) de asemenea sa fie cuviincioase, neclevetitoare, cumpatate, credincioase întru toate.
1Ti 3:12  Diaconii sa fie barba?i ai unei singure femei, sa-?i chiverniseasca bine casele ?i pe copiii lor.
1Ti 3:13  Caci cei ce slujesc bine, rang bun dobândesc ?i mult curaj în credin?a cea întru Hristos Iisus.
1Ti 3:14  Î?i scriu aceasta nadajduind ca voi veni la tine fara întârziere;
1Ti 3:15  Ca sa ?tii, daca zabovesc, cum trebuie sa petreci în casa lui Dumnezeu, care este Biserica Dumnezeului celui viu, stâlp ?i temelie a adevarului.
1Ti 3:16  ?i cu adevarat, mare este taina dreptei credin?e: Dumnezeu S-a aratat în trup, S-a îndreptat în Duhul, a fost vazut de îngeri, S-a propovaduit între neamuri, a fost crezut în lume, S-a înal?at întru slava.
1Ti 4:1  Dar Duhul graie?te lamurit ca, în vremurile cele de apoi, unii se vor departa de la credin?a, luând aminte la duhurile cele în?elatoare ?i la înva?aturile demonilor,
1Ti 4:2  Prin fa?arnicia unor mincino?i, care sunt înfiera?i în cugetul lor.
1Ti 4:3  Ace?tia opresc de la casatorie ?i de la unele bucate, pe care Dumnezeu le-a facut, spre gustare cu mul?umire, pentru cei credincio?i ?i pentru cei ce au cunoscut adevarul,
1Ti 4:4  Pentru ca orice faptura a lui Dumnezeu este buna ?i nimic nu este de lepadat, daca se ia cu mul?umire;
1Ti 4:5  Caci se sfin?e?te prin cuvântul lui Dumnezeu ?i prin rugaciune.
1Ti 4:6  Punându-le înaintea fra?ilor acestea, vei fi bun slujitor al lui Hristos Iisus, hranindu-te cu cuvintele credin?ei ?i ale bunei înva?aturi careia ai urmat;
1Ti 4:7  Iar de basmele cele lume?ti ?i babe?ti, fere?te-te ?i deprinde-te cu dreapta credin?a.
1Ti 4:8  Caci deprinderea trupeasca la pu?in folose?te, dar dreapta credin?a spre toate este de folos, având fagaduin?a vie?ii de acum ?i a celei ce va sa vina.
1Ti 4:9  Vrednic de credin?a este acest cuvânt ?i vrednic de toata primirea,
1Ti 4:10  Fiindca pentru aceasta ne ?i ostenim ?i suntem ocarâ?i ?i ne luptam, caci ne-am pus nadejdea în Dumnezeul cel viu, Care este Mântuitorul tuturor oamenilor, mai ales al credincio?ilor.
1Ti 4:11  Acestea sa le porunce?ti ?i sa-i înve?i.
1Ti 4:12  Nimeni sa nu dispre?uiasca tinere?ile tale, ci fa-te pilda credincio?ilor cu cuvântul, cu purtarea, cu dragostea, cu duhul, cu credin?a, cu cura?ia.
1Ti 4:13  Pâna voi veni eu, ia aminte la citit, la îndemnat, la înva?atura.
1Ti 4:14  Nu fi nepasator fa?a de harul care este întru tine, care ?i s-a dat prin proorocie, cu punerea mâinilor mai-marilor preo?ilor.
1Ti 4:15  Cugeta la acestea, ?ine-te de acestea, ca propa?irea ta sa fie vadita tuturor.
1Ti 4:16  Ia aminte la tine însu?i ?i la înva?atura; staruie în acestea, caci, facând aceasta, ?i pe tine te vei mântui ?i pe cei care te asculta.
1Ti 5:1  Pe cel batrân sa nu-l înfrun?i, ci sa-l îndemni ca pe un parinte; pe cei tineri, ca pe fra?i.
1Ti 5:2  Pe femeile batrâne îndeamna-le ca pe ni?te mame, pe cele tinere ca pe surori, în toata cura?ia.
1Ti 5:3  Pe vaduve cinste?te-le, dar pe cele cu adevarat vaduve.
1Ti 5:4  Daca vreo vaduva are copii sau nepo?i, ace?tia sa se înve?e sa cinsteasca mai întâi casa lor ?i sa dea rasplatire parin?ilor, pentru ca lucrul acesta este bun ?i primit înaintea lui Dumnezeu.
1Ti 5:5  Cea cu adevarat vaduva ?i ramasa singura are nadejdea în Dumnezeu ?i staruie?te în cereri ?i în rugaciuni, noaptea ?i ziua.
1Ti 5:6  Iar cea care traie?te în desfatari, de?i e vie, e moarta.
1Ti 5:7  ?i acestea porunce?te-le, ca ele sa fie fara de prihana.
1Ti 5:8  Daca însa cineva nu poarta grija de ai sai ?i mai ales de casnicii sai, s-a lepadat de credin?a ?i este mai rau decât un necredincios.
1Ti 5:9  Sa fie înscrisa între vaduve cea care nu are mai pu?in de ?aizeci de ani ?i a fost femeia unui singur barbat;
1Ti 5:10  Daca are marturie de fapte bune: daca a crescut copii, daca a fost primitoare de straini, daca a spalat picioarele sfin?ilor, daca a venit în ajutorul celor strâmtora?i, daca s-a ?inut staruitor de tot ce este lucru bun.
1Ti 5:11  Iar de vaduvele tinere fere?te-te. Caci, atunci când poftele le îndeparteaza de Hristos, vor sa se marite.
1Ti 5:12  ?i î?i agonisesc osânda, fiindca ?i-au calcat credin?a cea dintâi.
1Ti 5:13  Dar în acela?i timp se înva?a sa fie lene?e, cutreierând casele, ?i nu numai lene?e, ci ?i guralive ?i iscoditoare, graind cele ce nu se cuvin.
1Ti 5:14  Vreau deci ca vaduvele tinere sa se marite, sa aiba copii, sa-?i vada de case, ?i sa nu dea potrivnicului nici un prilej de ocara.
1Ti 5:15  Caci unele s-au ?i abatut, ca sa se duca dupa satana.
1Ti 5:16  Daca vreun credincios sau vreo credincioasa are în casa vaduve, sa aiba grija lor, ca Biserica sa nu fie împovarata, ci sa poata ajuta pe cele cu adevarat vaduve.
1Ti 5:17  Preo?ii, care î?i ?in bine dregatoria, sa se învredniceasca de îndoita cinste, mai ales cei care se ostenesc cu cuvântul ?i cu înva?atura.
1Ti 5:18  Pentru ca Scriptura zice: "Sa nu legi gura boului care treiera", ?i: "Vrednic este lucratorul de plata sa".
1Ti 5:19  Pâra împotriva preotului sa nu prime?ti, fara numai de la doi sau trei martori.
1Ti 5:20  Pe cei ce pacatuiesc mustra-i de fa?a cu to?i, ca ?i ceilal?i sa aiba teama.
1Ti 5:21  Te îndemn staruitor înaintea lui Dumnezeu ?i a lui Iisus Hristos ?i a îngerilor ale?i, ca sa paze?ti acestea, fara a lua o hotarâre dinainte, nefacând nimic cu partinire.
1Ti 5:22  Nu-?i pune mâinile degraba pe nimeni, nici nu te face parta? la pacatele altora. Pastreaza-te curat.
1Ti 5:23  De acum nu bea numai apa, ci folose?te pu?in vin, pentru stomacul tau ?i pentru desele tale slabiciuni.
1Ti 5:24  Pacatele unor oameni sunt vadite, mergând înaintea lor la judecata, ale altora însa vin în urma lor.
1Ti 5:25  Tot a?a ?i faptele cele bune sunt vadite, ?i cele ce nu sunt altfel nu se pot ascunde.
1Ti 6:1  Cei ce se gasesc sub jugul robiei sa socoteasca pe stapânii lor vrednici de toata cinstea, ca sa nu fie hulite numele ?i înva?atura lui Dumnezeu.
1Ti 6:2  Iar cei ce au stapâni credincio?i sa nu-i dispre?uiasca, sub cuvânt ca sunt fra?i; ci mai mult sa-i slujeasca, fiindca primitorii bunei lor slujiri sunt credincio?i ?i iubi?i. Acestea înva?a-i ?i porunce?te-le.
1Ti 6:3  Iar de înva?a cineva alta înva?atura ?i nu se ?ine de cuvintele cele sanatoase ale Domnului nostru Iisus Hristos ?i de înva?atura cea dupa dreapta credin?a,
1Ti 6:4  Acela e un îngâmfat, care nu ?tie nimic, suferind de boala discu?iilor ?i a certurilor de cuvinte, din care pornesc: cearta, pizma, defaimari, banuieli viclene,
1Ti 6:5  Gâlcevi necurmate ale oamenilor strica?i la minte ?i lipsi?i de adevar, care socotesc ca evlavia este un mijloc de câ?tig. Departeaza-te de unii ca ace?tia.
1Ti 6:6  ?i, în adevar, evlavia este mare câ?tig, dar atunci când ea se îndestuleaza cu ce are.
1Ti 6:7  Pentru ca noi n-am adus nimic în lume, tot a?a cum nici nu putem sa scoatem ceva din ea afara;
1Ti 6:8  Ci, având hrana ?i îmbracaminte cu acestea vom fi îndestula?i.
1Ti 6:9  Cei ce vor sa se îmboga?easca, dimpotriva, cad în ispita ?i în cursa ?i în multe pofte nebune?ti ?i vatamatoare, ca unele care cufunda pe oameni în ruina ?i în pierzare.
1Ti 6:10  Ca iubirea de argint este radacina tuturor relelor ?i cei ce au poftit-o cu înfocare au ratacit de la credin?a, ?i s-au strapuns cu multe dureri.
1Ti 6:11  Dar tu, o, omule al lui Dumnezeu, fugi de acestea ?i urmeaza dreptatea, evlavia, credin?a, dragostea, rabdarea, blânde?ea.
1Ti 6:12  Lupta-te lupta cea buna a credin?ei, cucere?te via?a ve?nica la care ai fost chemat ?i pentru care ai dat buna marturie înaintea multor martori.
1Ti 6:13  Î?i poruncesc înaintea lui Dumnezeu, Cel ce aduce toate la via?a, ?i înaintea lui Iisus Hristos, Cel ce, în fa?a lui Pon?iu Pilat, a marturisit marturisirea cea buna:
1Ti 6:14  Sa paze?ti porunca fara pata, fara vina, pâna la aratarea Domnului nostru Iisus Hristos,
1Ti 6:15  Pe care, la timpul cuvenit, o va arata fericitul ?i singurul Stapânitor, Împaratul împara?ilor ?i Domnul domnilor,
1Ti 6:16  Cel ce singur are nemurire ?i locuie?te întru lumina neapropiata; pe Care nu L-a vazut nimeni dintre oameni, nici nu poate sa-L vada; a Caruia este cinstea ?i puterea ve?nica! Amin.
1Ti 6:17  Celor boga?i în veacul de acum porunce?te-le sa nu se seme?easca, nici sa-?i puna nadejdea în boga?ia cea nestatornica, ci în Dumnezeul cel viu, Care ne da cu bel?ug toate, spre îndulcirea noastra,
1Ti 6:18  Sa faca ce e bine, sa se înavu?easca în fapte bune, sa fie darnici, sa fie cu inima larga,
1Ti 6:19  Agonisindu-?i lor buna temelie în veacul viitor, ca sa dobândeasca, cu adevarat, via?a ve?nica.
1Ti 6:20  O, Timotei, paze?te comoara ce ?i s-a încredin?at, departându-te de vorbirile de?arte ?i lume?ti ?i de împotrivirile ?tiin?ei mincinoase,
1Ti 6:21  Pe care unii, marturisind-o, au ratacit de la credin?a. Harul fie cu tine! Amin.


\end{document}