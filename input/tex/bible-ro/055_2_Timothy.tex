\begin{document}

\title{2 Timotei}


\chapter{1}

\par 1 Pavel, apostolul lui Hristos Iisus, prin voia lui Dumnezeu, după făgăduința vieții care este în Hristos Iisus,
\par 2 Lui Timotei, iubitului fiu: Har, milă, pace de la Dumnezeu-Tatăl și de la Hristos Iisus, Domnul nostru!
\par 3 Mulțumesc lui Dumnezeu, Căruia Îi slujesc din strămoși, cu cuget curat, că te pomenesc neîncetat, zi și noapte, în rugăciunile mele.
\par 4 Și pentru că îmi aduc aminte de lacrimile tale, doresc mult să te văd, ca să mă umplu de bucurie;
\par 5 Îmi aduc iarăși aminte de credința ta neprefăcută, care, precum s-a sălășluit întâi în bunica ta Loida și în mama Eunichi, tot așa, sunt încredințat, că și întru tine.
\par 6 Din această pricină, îți amintesc să aprinzi și mai mult din nou harul lui Dumnezeu, care este în tine, prin punerea mâinilor mele.
\par 7 Căci Dumnezeu nu ne-a dat duhul temerii, ci al puterii și al dragostei și al înțelepciunii.
\par 8 Deci, nu te rușina de a mărturisi pe Domnul nostru, nici de mine, cel pus în lanțuri pentru El, ci pătimește împreună cu mine pentru Evanghelie după puterea de la Dumnezeu.
\par 9 El ne-a mântuit și ne-a chemat cu chemare sfântă, nu după faptele noastre, ci după a Sa hotărâre și după harul ce ne-a fost dat în Hristos Iisus, mai înainte de începutul veacurilor,
\par 10 Iar acum s-a dat pe față prin arătarea Mântuitorului nostru Iisus Hristos, Cel ce a nimicit moartea și a adus la lumină viața și nemurirea, prin Evanghelie.
\par 11 Spre aceasta am fost pus eu propovăduitor și apostol și învățător al neamurilor.
\par 12 Din această pricină și sufăr toate acestea, dar nu mă rușinez, că știu în cine am crezut și sunt încredințat că puternic este să păzească comoara ce mi-a încredințat, până în ziua aceea.
\par 13 Ține dreptarul cuvintelor sănătoase pe care le-ai auzit de la mine, cu credința și cu iubirea ce este în Hristos Iisus.
\par 14 Comoara cea bună ce ți s-a încredințat, păzește-o cu ajutorul Sfântului Duh, Care sălășluiește întru noi.
\par 15 Tu știi că toți cei din Asia s-au lepădat de mine, între care Fighel și Ermoghen.
\par 16 Domnul să aibă milă de casa lui Onisifor, căci de multe ori m-a însuflețit și de lanțurile mele nu s-a rușinat,
\par 17 Ci venind în Roma, cu multă osârdie m-a căutat și m-a găsit.
\par 18 Să-i dea Domnul ca, în ziua aceea, el să afle milă de la Domnul. Și cât de mult mi-a slujit el în Efes, tu știi prea bine.

\chapter{2}

\par 1 Tu, deci, fiul meu, întărește-te în harul care e în Hristos Iisus,
\par 2 Și cele ce ai auzit de la mine, cu mulți martori de față, acestea le încredințează la oameni credincioși, care vor fi destoinici să învețe și pe alții.
\par 3 Suferă împreună cu mine, ca un bun ostaș al lui Hristos Iisus.
\par 4 Nici un ostaș nu se încurcă cu treburile vieții, ca să fie pe plac celui care strânge oaste.
\par 5 Iar când se luptă cineva, la jocuri, nu ia cununa, dacă nu s-a luptat după regulile jocului.
\par 6 Cuvine-se ca plugarul ce se ostenește să mănânce el mai întâi din roade.
\par 7 Înțelege cele ce-ți grăiesc, căci Domnul îți va da pricepere în toate.
\par 8 Adu-ți aminte de Iisus Hristos, Care a înviat din morți, din neamul lui David, după Evanghelia mea,
\par 9 Pentru Care sufăr până și lanțuri ca un făcător de rele, dar cuvântul lui Dumnezeu nu se leagă.
\par 10 De aceea toate le rabd, pentru cei aleși, ca și ei să aibă parte de mântuirea care este întru Hristos Iisus și de slava veșnică.
\par 11 Vrednic de crezare este cuvântul: căci dacă am murit împreună cu El, vom și învia împreună cu El.
\par 12 Dacă rămânem întru El, vom și împărăți împreună cu El; de-L vom tăgădui, și El ne va tăgădui pe noi.
\par 13 Dacă nu-I suntem credincioși, El rămâne credincios, căci nu poate să Se tăgăduiască pe Sine însuși.
\par 14 Amintește-le acestea și îndeamnă stăruitor înaintea lui Dumnezeu să nu se certe pe cuvinte, ceea ce la nimic nu folosește, decât la pierzarea ascultătorilor.
\par 15 Silește-te să te arăți încercat, înaintea lui Dumnezeu lucrător cu fața curată, drept învățând cuvântul adevărului.
\par 16 Iar de deșartele vorbiri lumești ferește-te, căci ele vor spori nelegiuirea tot mai mult.
\par 17 Cuvântul lor va roade ca o cangrenă. Dintre ei sunt Imeneu și Filet,
\par 18 Care au rătăcit de la adevăr, zicând că învierea s-a și petrecut, și răstoarnă credința unora.
\par 19 Dar temelia cea tare a lui Dumnezeu stă neclintită, având pecetea aceasta: "Cunoscut-a Domnul pe cei ce sunt ai Săi"; și "să se depărteze de la nedreptate oricine cheamă numele Domnului".
\par 20 Iar într-o casă mare nu sunt numai vase de aur și de argint, ci și de lemn și de lut; și unele sunt spre cinste, iar altele spre necinste.
\par 21 Deci, de se va curăți cineva pe sine de acestea, va fi vas de cinste, sfințit, de bună trebuință stăpânului, potrivit pentru tot lucrul bun.
\par 22 Fugi de poftele tinereților și urmează dreptatea, credința, dragostea, pacea cu cei ce cheamă pe Domnul din inimă curată.
\par 23 Ferește-te de întrebările nebunești, știind că dau prilej de ceartă.
\par 24 Un slujitor al Domnului nu trebuie să se certe, ci să fie blând față cu toți, destoinic să dea învățătură, îngăduitor,
\par 25 Certând cu blândețe pe cei ce stau împotrivă, ca doar le va da Dumnezeu pocăință spre cunoașterea adevărului,
\par 26 Și ei să scape din cursa diavolului, de care sunt prinși, pentru a-i face voia.

\chapter{3}

\par 1 Și aceasta să știi că, în zilele din urmă, vor veni vremuri grele;
\par 2 Că vor fi oameni iubitori de sine, iubitori de arginți, lăudăroși, trufași, hulitori, neascultători de părinți, nemulțumitori, fără cucernicie,
\par 3 Lipsiți de dragoste, neînduplecați, clevetitori, neînfrânați, cruzi, neiubitori de bine,
\par 4 Trădători, necuviincioși, îngâmfați, iubitori de desfătări mai mult decât iubitori de Dumnezeu,
\par 5 Având înfățișarea adevăratei credințe, dar tăgăduind puterea ei. Depărtează-te și de aceștia.
\par 6 Căci dintre aceștia sunt cei ce se vâră prin case și robesc femeiuști împovărate de păcate și purtate de multe feluri de pofte,
\par 7 Mereu învățând și neputând niciodată să ajungă la cunoașterea adevărului.
\par 8 După cum Iannes și Iambres s-au împotrivit lui Moise, așa și aceștia stau împotriva adevărului, oameni stricați la minte și netrebnici pentru credință.
\par 9 Dar nu vor merge mai departe, pentru că nebunia lor va fi vădită tuturor, precum a fost și a acelora.
\par 10 Tu însă mi-ai urmat în învățătură, în purtare, în năzuință, în credință, în îndelungă răbdare, în dragoste, în stăruință,
\par 11 În prigonirile și suferințele care mi s-au făcut în Antiohia, în Iconiu, în Listra; câte prigoniri am răbdat! și din toate m-a izbăvit Domnul.
\par 12 Și toți care voiesc să trăiască cucernic în Hristos Iisus vor fi prigoniți.
\par 13 Iar oamenii răi și amăgitori vor merge spre tot mai rău, rătăcind pe alții și rătăciți fiind ei înșiși.
\par 14 Tu însă rămâi în cele ce ai învățat și de care ești încredințat, deoarece știi de la cine le-ai învățat,
\par 15 Și fiindcă de mic copil cunoști Sfintele Scripturi, care pot să te înțelepțească spre mântuire, prin credința cea întru Hristos Iisus.
\par 16 Toată Scriptura este insuflată de Dumnezeu și de folos spre învățătură, spre mustrare, spre îndreptare, spre înțelepțirea cea întru dreptate,
\par 17 Astfel ca omul lui Dumnezeu să fie desăvârșit, bine pregătit pentru orice lucru bun.

\chapter{4}

\par 1 Eu te îndemn deci stăruitor în fața lui Dumnezeu și a lui Hristos Iisus, Care va să judece viii și morții, la arătarea Lui și în împărăția Lui;
\par 2 Propovăduiește cuvântul, stăruiește cu timp și fără de timp, mustră, ceartă, îndeamnă, cu toată îndelunga-răbdare și învățătura.
\par 3 Căci va veni o vreme când nu vor mai suferi învățătura sănătoasă, ci - dornici să-și desfăteze auzul - își vor grămădi învățători după poftele lor,
\par 4 Și își vor întoarce auzul de la adevăr și se vor abate către basme.
\par 5 Tu fii treaz în toate, suferă răul, fă lucru de evanghelist, slujba ta fă-o deplin!
\par 6 Că eu de-acum mă jertfesc și vremea despărțirii mele s-a apropiat.
\par 7 Lupta cea bună m-am luptat, călătoria am săvârșit, credința am păzit.
\par 8 De acum mi s-a gătit cununa dreptății, pe care Domnul îmi va da-o în ziua aceea, El, Dreptul Judecător, și nu numai mie, ci și tuturor celor ce au iubit arătarea Lui.
\par 9 Silește-te să vii curând la mine,
\par 10 Că Dimas, iubind veacul de acum, m-a lăsat și s-a dus la Tesalonic, Crescent în Galatia, Tit în Dalmația;
\par 11 Numai Luca este cu mine. Ia pe Marcu și adu-l cu tine, căci îmi este de folos în slujire.
\par 12 Pe Tihic l-am trimis la Efes.
\par 13 Când vei veni, adu-mi felonul pe care l-am lăsat în Troada, la Carp, precum și cărțile, mai ales pergamentele.
\par 14 Alexandru arămarul mi-a făcut multe rele; Domnul să-i răsplătească după faptele lui.
\par 15 Păzește-te și tu de el, căci s-a împotrivit foarte mult cuvântărilor noastre.
\par 16 La întâia mea apărare, nimeni nu mi-a venit într-ajutor, ci toți m-au părăsit. Să nu li se țină în socoteală!
\par 17 Dar Domnul mi-a stat într-ajutor și m-a întărit, pentru ca, prin mine, Evanghelia să fie pe deplin vestită și s-o audă toate neamurile; iar eu am fost izbăvit din gura leului.
\par 18 Domnul mă va izbăvi de orice lucru rău și mă va mântui, în împărăția Sa cerească. Lui fie slava în vecii vecilor. Amin!
\par 19 Îmbrățișează pe Priscila și pe Acvila și casa lui Onisifor.
\par 20 Erast a rămas în Corint; pe Trofim l-am lăsat în Milet, fiind bolnav.
\par 21 Silește-te să vii mai înainte de începutul iernii. Te îmbrățișează Eubul și Pudențiu și Linos și Claudia și frații toți.
\par 22 Domnul Iisus Hristos să fie cu duhul tău! Harul fie cu voi! Amin.


\end{document}