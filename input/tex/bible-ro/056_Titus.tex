\begin{document}

\title{Tit}


\chapter{1}

\par 1 Pavel, robul lui Dumnezeu și apostol al lui Iisus Hristos, după credința aleșilor lui Dumnezeu și după cunoștința adevărului cel întocmai cu dreapta credință,
\par 2 - Întru nădejdea vieții veșnice, pe care a făgăduit-o mai înainte de anii veacurilor Dumnezeu, Care nu minte,
\par 3 Și Care, la timpul cuvenit, Și-a făcut cunoscut cuvântul Său, prin propovăduirea încredințată mie, după porunca Mântuitorului nostru Dumnezeu -
\par 4 Lui Tit, adevăratul fiu după credința cea de obște: Har, milă și pace, de la Dumnezeu-Tatăl și de la Domnul Iisus Hristos, Mântuitorul nostru.
\par 5 Pentru aceasta te-am lăsat în Creta, ca să îndreptezi cele ce mai lipsesc și să așezi preoți prin cetăți, precum ți-am rânduit:
\par 6 De este cineva fără de prihană, bărbat al unei femei, având fii credincioși, nu sub învinuire de desfrânare sau neascultători.
\par 7 Căci se cuvine ca episcopul să fie fără de prihană, ca un iconom al lui Dumnezeu, neîngâmfat, nu grabnic la mânie, nu dat la băutură, pașnic, nepoftitor de câștig urât,
\par 8 Ci iubitor de străini, iubitor de bine, înțelept, drept, cuvios, cumpătat,
\par 9 Ținându-se de cuvântul cel credincios al învățăturii, ca să fie destoinic și să îndemne la învățătura cea sănătoasă și să mustre pe cei potrivnici.
\par 10 Pentru că mulți sunt răzvrătiți, grăitori în deșert și înșelători, mai ales cei din tăierea împrejur,
\par 11 Cărora trebuie să li se închidă gura ca unora care răzvrătesc case întregi, învățând, pentru câștig urât, cele ce nu se cuvin.
\par 12 Unul dintre ei, chiar un prooroc al lor, a rostit: Cretanii sunt pururea mincinoși, fiare rele, pântece leneșe.
\par 13 Mărturia aceasta este adevărată; pentru care pricină, mustră-i cu asprime, ca să fie sănătoși în credință,
\par 14 Și să nu dea ascultare basmelor iudaicești și poruncilor unor oameni, care se întorc de la adevăr.
\par 15 Toate sunt curate pentru cei curați; iar pentru cei întinați și necredincioși nimeni nu este curat, ci li s-au întinat lor și mintea și cugetul.
\par 16 Ei mărturisesc că Îl cunosc pe Dumnezeu, dar cu faptele lor Îl tăgăduiesc, urâcioși fiind, nesupuși, și la orice lucru bun, netrebnici.

\chapter{2}

\par 1 Dar tu grăiește cele ce se cuvin învățăturii sănătoase.
\par 2 Bătrânii să fie treji, cinstiți, întregi la minte, sănătoși în credință, în dragoste, în răbdare;
\par 3 Bătrânele de asemenea să aibă, în înfățișare, sfințită cuviință, să fie neclevetitoare, nerobite de vin mult, să învețe de bine,
\par 4 Ca să înțelepțească pe cele tinere să-și iubească bărbații, să-și iubească copiii,
\par 5 Și să fie cumpătate, curate, gospodine, bune, plecate bărbaților lor, ca să nu fie defăimat cuvântul lui Dumnezeu.
\par 6 Îndeamnă, de asemenea, pe cei tineri să fie cumpătați.
\par 7 Întru toate arată-te pe tine pildă de fapte bune, dovedind în învățătură neschimbare, cuviință,
\par 8 Cuvânt sănătos și fără prihană, pentru ca cel potrivnic să se rușineze, neavând de zis nimic rău despre noi.
\par 9 Slugile să se supună stăpânilor lor, întru toate, ca să fie bine-plăcute, neîntorcându-le vorba,
\par 10 Să nu dosească ceva, ci să le arate toată buna credință, ca să facă de cinste întru toate învățătura Mântuitorului nostru Dumnezeu.
\par 11 Căci harul mântuitor al lui Dumnezeu s-a arătat tuturor oamenilor,
\par 12 Învățându-ne pe noi să lepădăm fărădelegea și poftele lumești și, în veacul de acum, să trăim cu înțelepciune, cu dreptate și cu cucernicie;
\par 13 Și să așteptăm fericita nădejde și arătarea slavei marelui Dumnezeu și Mântuitorului nostru Hristos Iisus,
\par 14 Care S-a dat pe Sine pentru noi, ca să ne izbăvească de toată fărădelegea și să-Și curățească Lui popor ales, râvnitor de fapte bune.
\par 15 Acestea grăiește, îndeamnă și mustră cu toată tăria. Nimeni să nu te disprețuiască.

\chapter{3}

\par 1 Adu-le aminte să se supună stăpânirilor și dregătorilor, să asculte, să fie gata la orice lucru bun,
\par 2 Să nu defaime pe nimeni, să fie pașnici, să fie îngăduitori, arătând întreaga blândețe față de toți oamenii.
\par 3 Căci și noi eram altădată fără de minte, neascultători, amăgiți, slujind poftelor și multor feluri de desfătări, petrecând viața în răutate și pizmuire, urâți fiind și urându-ne unul pe altul;
\par 4 Iar când bunătatea și iubirea de oameni a Mântuitorului nostru Dumnezeu s-au arătat,
\par 5 El ne-a mântuit, nu din faptele cele întru dreptate, săvârșite de noi, ci după a Lui îndurare, prin baia nașterii celei de a doua și prin înnoirea Duhului Sfânt,
\par 6 Pe Care L-a vărsat peste noi, din belșug, prin Iisus Hristos, Mântuitorul nostru,
\par 7 Ca îndreptându-ne prin harul Lui, să ne facem, după nădejde, moștenitorii vieții celei veșnice.
\par 8 Vrednic de crezare este cuvântul, și voiesc să adeverești acestea cu tărie, pentru ca acei ce au crezut în Dumnezeu să aibă grijă să fie în frunte la fapte bune. Că acestea sunt cele bune și de folos oamenilor.
\par 9 Iar de întrebările nebunești și de înșirări de neamuri și de certuri și de sfădirile pentru lege, ferește-te, căci sunt nefolositoare și deșarte.
\par 10 De omul eretic, după întâia și a doua mustrare, depărtează-te,
\par 11 Știind că unul ca acesta s-a abătut și a căzut în păcat, fiind singur de sine osândit.
\par 12 Când voi trimite pe Artemas la tine sau pe Tihic, sârguiește-te să vii la mine la Nicopole, căci acolo m-am hotărât să iernez.
\par 13 Pe Zenas, cunoscătorul de lege, și pe Apollo trimite-i mai înainte, cu bună grijă, ca nimic să nu le lipsească.
\par 14 Să învețe și ai noștri să poarte grijă de lucrurile bune, spre treburile cele de neapărată nevoie, ca ei să nu fie fără de roadă.
\par 15 Te îmbrățișează toți care sunt cu mine. Îmbrățișează pe cei ce ne iubesc întru credință. Harul fie cu voi cu toți! Amin.


\end{document}