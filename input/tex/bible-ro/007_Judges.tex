\begin{document}

\title{Judecători}


\chapter{1}

\par 1 Dupa moartea lui Iosua, au întrebat fiii lui Israel pe Domnul, zicând: "Cine din noi sa mearga mai întâi asupra Canaaneilor, ca sa se lupte cu ei?"
\par 2 ?i a zis Domnul: "Iuda sa mearga ?i Eu voi da ?ara în mâna lui!"
\par 3 Iuda a zis catre Simeon, fratele sau: "Intra cu mine în ?inutul care mi-a cazut la sor?i ?i sa ne luptam cu Canaaneii ?i voi intra ?i eu cu tine în ?inutul care ?i-a cazut la sor?i". ?i s-a dus Simeon cu el.
\par 4 Atunci s-a dus Iuda cu Simeon ?i Domnul a dat pe Canaanei ?i pe Ferezei în mâinile lor ?i au ucis ei din aceia zece mii de oameni în Bezec.
\par 5 ?i s-au întâlnit ei în Bezec cu Adoni-Bezec ?i s-au batut cu dânsul ?i au rapus pe Canaanei ?i pe Ferezei,
\par 6 Iar Adoni-Bezec a fugit; dar ei au alergat dupa el ?i l-au prins ?i i-au taiat degetele cele mari de la mâini ?i de la picioare.
\par 7 Atunci Adoni-Bezec a zis: "?aptezeci de regi cu degetele cele mari taiate de la mâinile ?i de la picioarele lor adunau farâmituri sub masa mea; cum am facut eu, a?a mi-a platit ?i mie Dumnezeu. ?i l-au dus în Ierusalim ?i a murit acolo.
\par 8 Apoi au mers fiii lui Iuda cu razboi asupra Ierusalimului ?i l-au luat ?i l-au trecut prin sabie, iar cetatea au dat-o focului.
\par 9 Dupa aceea s-au dus fiii lui Iuda sa se lupte cu Canaaneii care traiau în mun?i, în ?ara de miazazi ?i în par?ile de jos.
\par 10 ?i a mers Iuda asupra Canaaneilor care locuiau în Hebron; iar numele Hebronului era mai înainte Chiriat-Arba; ?i a batut per ?e?ai, pe Ahiman ?i pe Talmai, din neamul lui Enac.
\par 11 Iar de acolo s-au dus împotriva locuitorilor Debirului al carui nume era mai înainte Chiriat-Sefer.
\par 12 Atunci Caleb a zis: "Cine va lovi Chiriat-Seferul ?i-l va lua, aceluia îi voi da pe Acsa, fiica mea, de femeie".
\par 13 ?i l-a luat Otniel, fiul lui Chenaz, fratele mai mic al lui Caleb, ?i acesta i-a dat de so?ie pe Acsa, fiica sa.
\par 14 ?i când a vrut ea sa plece, Otniel a înva?at-o sa ceara la tatal ei o ?arina ?i ea s-a coborât de pe asin. ?i i-a zis Caleb: "Ce vrei?"
\par 15 Iar Acsa a zis catre el: "Da-mi binecuvântare; tu mi-ai dat pamântul de la miazazi, da-mi ?i izvoarele de apa". ?i i-a dat Caleb, dupa dorin?a ei, izvoarele cele de sus ?i izvoarele cele de jos.
\par 16 Iar fiii lui Hobab cheneul, socrul lui Moise, au mers din cetatea Palmierilor cu fiii lui Iuda în pustiul lui Iuda din Negheb, care este la miazazi de Arad ?i, venind, s-au a?ezat între Amaleci?i.
\par 17 Apoi a mers Iuda cu Simeon, fratele sau ?i a lovit pe Canaaneii care locuiau în ?efat ?i i-au dat pieirii; de aceea s-a numit cetatea aceea Horma.
\par 18 ?i a mai luat Iuda ?i Gaza cu împrejurimile ei, Ascalonul cu împrejurimile lui ?i Ecronul cu ?inutul lui.
\par 19 ?i a fost Domnul cu Iuda ?i acesta a luat în stapânire partea muntoasa; dar pe locuitorii din vale nu i-a putut alunga, pentru ca ei aveau care de fier.
\par 20 ?i au dat lui Caleb Hebronul, cum zisese Moise, ?i a primit el acolo de mo?tenire cele trei ceta?i ale fiilor lui Enac ?i a alungat pe cei trei fii ai lui Enac.
\par 21 Iar fiii lui Veniamin n-au alungat pe Iebuseii care locuiau în Ierusalim; ?i traiesc Iebusei cu fiii lui Veniamin în Ierusalim pâna în ziua de astazi.
\par 22 S-au suit de asemenea ?i fiii lui Iosif asupra Betelului ?i Domnul a fost cu ei,
\par 23 ?i s-au oprit ei ?i au cercetat Betelul; iar numele ceta?ii acesteia era mai înainte Luz.
\par 24 ?i au vazut cei de straja un om ie?ind din cetate ?i l-au prins ?i i-au zis: "Arata-ne intrarea ceta?ii ?i vom avea mila de tine".
\par 25 ?i le-a aratat intrarea ceta?ii ?i ei au lovit cetatea cu sabia, iar omului aceluia ?i la tot neamul lui i-au dat drumul.
\par 26 ?i s-a dus omul acela în pamântul Heteilor ?i a zidit acolo o cetate ?i i-a pus numele Luz. ?i acesta este numele ei pâna în ziua de astazi.
\par 27 Manase de asemenea n-a alungat pe locuitorii Bet-?eanului, care e Schitopole, ?i ai ceta?ilor supuse lui, nici pe ai Taanacului ?i ai ceta?ilor supuse lui, nici pe locuitorii din Dor ?i ai ceta?ilor supuse lui, nici pe locuitorii Ibleamului ?i ai ceta?ilor supuse lui, nici pe locuitorii Meghidonului ?i ai ceta?ilor supuse lui; ?i au ramas Canaaneii sa traiasca în pamântul acesta.
\par 28 Când Israel a ajuns puternic, atunci a facut pe Canaanei birnici, dar de izgonit nu i-a izgonit.
\par 29 Nici Efraim n-a izgonit pe Canaaneii care locuiau în Ghezer, ?i au trait Canaaneii în mijlocul lor în Ghezer ?i le-au platit bir.
\par 30 Zabulon înca n-a izgonit pe locuitorii Chitronului ?i pe locuitorii Nahalolului; ?i au locuit Canaaneii în mijlocul lor ?i le-au platit bir.
\par 31 Nici A?er n-a izgonit pe locuitorii din Aco care-i plateau bir, nici pe locuitorii din Dor, nici pe locuitorii din Sidon ?i Mahaleb, din Aczib, din Helba, din Afec ?i din Rehob.
\par 32 ?i a locuit A?er între Canaanei, locuitorii rarii aceleia, caci nu i-a izgonit.
\par 33 Nici Neftali n-a izgonit pe locuitorii din Bet-?eme?, nici pe locuitorii din Bet-Anat, ?i a trait între Canaanei, locuitorii jarii aceleia. Iar locuitorii din Bet-?eme? ?i din Bet-Anat erau birnicii lui.
\par 34 Dar Amoreii au împins pe fiii lui Dan în mun?i ?i nu i-a lasat sa se coboare în vale.
\par 35 ?i au ramas Amoreii sa locuiasca în Har-Heres, în Aialon ?i ?aalbim; dar mâna fiilor lui Iosif a rapus pe Amorei ?i au ajuns ace?tia birnicii lor.
\par 36 Iar hotarele Amoreilor se întindeau pâna dincolo de înal?imea Acrabimului ?i de Sela.

\chapter{2}

\par 1 Atunci s-a suit un înger al Domnului din Ghilgal catre Bochim, catre Betel ?i catre casa lui Israel ?i le-a zis: "A?a graie?te Domnul:
\par 2 Eu v-am scos din Egipt ?i v-am bagat în ?ara pentru care M-am jurat parin?ilor vo?tri sa v-o dau ?i am zis: Nu voi rupe în veac legamântul Meu cu voi; voi însa sa nu intra?i în legatura cu locuitorii ?arii acesteia; dumnezeilor lor sa nu va închina?i, idolii lor sa-i sfarâma?i ?i jertfelnicele lor sa le darâma?i. Dar voi n-a?i ascultat glasul Meu. Pentru ce a?i facut aceasta?
\par 3 De aceea va zic: Nu Ma voi apuca sa stramut pe locuitorii ace?tia pe care Eu am voit sa-i alung, nu-i voi izgoni de la voi ?i ei va vor fi la?, iar dumnezeii lor vor fi pentru voi mreaja".
\par 4 Când a spus îngerul Domnului cuvintele acestea tuturor fiilor lui Israel, atunci poporul a ridicat strigare mare ?i a plâns.
\par 5 De aceea s-a ?i numit locul acela Bochim. ?i au adus ei acolo jertfa Domnului.
\par 6 Dupa ce a dat Iosua drumul poporului ?i s-au dus fiii lui Israel fiecare la casa sa ?i fiecare la mo?ia sa, ca sa ia ?ara de mo?tenire,
\par 7 Atunci poporul a slujit Domnului în toate zilele lui Iosua ?i în toate zilele batrânilor, a caror via?a s-a prelungit dupa Iosua ?i care vazusera toate lucrurile cele mari ale Domnului, pe care le facuse El cu Israel.
\par 8 Iosua, fiul lui Navi, sluga Domnului, a murit fiind de o suta zece ani,
\par 9 ?i l-au îngropat în cuprinsul mo?tenirii sale, la Timnat-Heres, în muntele lui Efraim, spre miazanoapte de muntele Gaa?;
\par 10 ?i dupa ce tot rândul acela de oameni a trecut la parin?ii lor ?i când s-a ridicat în locul lor alt rând de oameni, care nu cuno?teau pe Domnul ?i lucrurile Sale pe care le facuse cu Israel,
\par 11 Atunci fiii lui Israel au început a face rele înaintea ochilor Domnului ?i s-au apucat sa slujeasca baalilor;
\par 12 Au parasit pe Domnul Dumnezeul parin?ilor lor, Care îi scosese din pamântul Egiptului ?i s-au întors la al?i dumnezei, catre dumnezeii popoarelor dimprejurul lor ?i au început sa se închine acelora ?i au mâniat pe Domnul;
\par 13 Au parasit pe Domnul ?i au început a se închina lui Baal ?i Astartelor.
\par 14 De aceea s-a aprins mânia Domnului asupra lui Israel ?i l-a dat în mâinile jefuitorilor care i-au jefuit; i-a dat în mâinile vrajma?ilor dimprejurul lor ?i n-au mai putut sa se împotriveasca vrajma?ilor lor.
\par 15 Ori încotro apucau, mâna Domnului pretutindeni era împotriva lor, ca sa faca rau, cum le graise Domnul ?i cum li Se jurase. ?i erau foarte strâmtora?i.
\par 16 Atunci le-a ridicat Domnul judecatori care i-au izbavit din mâinile jefuitorilor lor.
\par 17 Dar nici pe judecatori nu-i ascultau ei, ci se purtau desfrânat mergând pe urmele altor dumnezei ?i se închinau acelora ?i mâniau pe Domnul; u?or se abateau de la calea pe care umblasera parin?ii lor care se supusesera poruncilor Domnului. Ei însa nu faceau a?a.
\par 18 Când le ridica lor Domnul judecatori, atunci Însu?i Domnul era cu judecatorul ?i-i izbavia pe ei de vrajma?ii lor în toate zilele judecatorului.
\par 19 Dar cum murea judecatorul, ei iara?i faceau ?i mai rau decât parin?ii lor, abatându-se la al?i dumnezei, slujind acelora ?i închinându-se lor, nu se lasau de lucrurile lor ?i nu se abateau de la calea lor cea rea.
\par 20 ?i se aprindea mânia Domnului asupra lui Israel ?i zicea: "Pentru ca poporul acesta calca poruncile Mele, pe care Eu le-am a?ezat cu parin?ii lor ?i nu asculta glasul Meu,
\par 21 De aceea nici Eu nu voi mai izgoni de la ei nici unul din acele popoare pe care le-a lasat Iosua, fiul lui Navi, în ?ara, când a murit,
\par 22 Ca sa ispiteasca prin ele pe Israel ?i sa vada de va ?ine el calea Domnului ?i de va umbla pe ea, cum s-au ?inut parin?ii lor, sau nu".
\par 23 ?i a lasat Domnul pe popoarele acestea ?i nu le-a alungat îndata, nici nu le-a dat în mâinile lui Iosua.

\chapter{3}

\par 1 Iata popoarele acelea pe care le-a lasat Domnul ca sa ispiteasca prin ele pe Israel ?i pe to?i aceia care nu cuno?teau toate razboaiele Canaanului;
\par 2 Pe care le lasase numai pentru ca genera?iile viitoare de oameni ale fiilor lui Israel sa ?tie ?i sa înve?e razboiul pe care nu-l cunoscusera mai înainte:
\par 3 Cinci stapânitori Filisteni, to?i Canaaneii, Sidonienii ?i Heveii care locuiau în mun?ii Libanului de la muntele Baal-Hermon pâna la intrarea Hamatului.
\par 4 Acestea fusesera lasate ca sa se încerce prin ele Israeli?ii ?i sa se afle daca se supun ei poruncilor Domnului, pe care le-a dat El parin?ilor lor prin Moise.
\par 5 ?i au trait fiii lui Israel între Canaanei, Hetei, Amorei, Ferezei, Hevei, Gherghesei ?i Iebusei,
\par 6 ?i ?i-au luat femei din fetele acelora ?i pe fetele lor le-au dat dupa feciorii acelora ?i au slujit dumnezeilor lor.
\par 7 Deci au facut rele înaintea ochilor Domnului fiii lui Israel ?i au uitat pe Domnul Dumnezeul lor ?i au slujit baalilor ?i astartelor.
\par 8 Atunci s-a aprins mânia Domnului asupra lui Israel ?i i-a dat în mâinile lui Cu?an-Ri?eataim, regele Mesopotamiei, ?i fiii lui Israel au robit lui Cu?an-Ri?eataim opt ani.
\par 9 Dupa aceea au strigat fiii lui Israel catre Domnul ?i a ridicat Domnul un izbavitor pentru fiii lui Israel, care i-a izbavit ?i anume pe Otniel, fiul lui Chenaz, fratele mai mic al lui Caleb.
\par 10 ?i a fost Duhul Domnului peste acesta ?i a fost el judecator lui Israel. Acesta a ie?it la razboi împotriva lui Cu?an-Ri?eataim ?i Domnul a dat în mâinile lui pe Cu?an-Ri?eataim, regele Mesopotamiei, ?i a apasat mâna lui pe Cu?an-Ri?eataim.
\par 11 Dupa aceea s-a odihnit ?ara patruzeci de ani ?i apoi a murit Otniel, fiul lui Chenaz.
\par 12 Apoi fiii lui Israel iara?i s-au apucat sa faca rele înaintea ochilor Domnului ?i a întarit Domnul pe Eglon, regele Moabului, împotriva Israeli?ilor, pentru ca ei faceau rele înaintea ochilor Domnului.
\par 13 ?i a adunat acela la sine pe to?i Moabi?ii ?i Amaleci?ii ?i a plecat sa loveasca pe Israel; ?i au luat cetatea Palmierilor.
\par 14 Iar fiii lui Israel au slujit lui Eglon, regele Moabului, optsprezece ani.
\par 15 Atunci au strigat fiii lui Israel catre Domnul ?i Domnul le-a ridicat ca izbavitor pe Aod, fiul lui Ghera, din neamul lui Veniamin, care era stângaci. ?i au trimis fiii lui Israel prin el daruri lui Eglon, regele Moabului.
\par 16 Aod ?i-a facut sabie cu doua ascu?i?uri, lunga de un cot ?i a încins-o sub mantaua sa la ?oldul drept
\par 17 ?i a mers cu daruri la Eglon, regele Moabului. Eglon însa era om foarte gras.
\par 18 Dupa ce a înfa?i?at Aod toate darurile, a dat drumul oamenilor care adusesera darurile,
\par 19 Iar el însu?i, întorcându-se de la idolii de lânga Ghilgal, a zis regelui: "Cuvânt în taina am a-?i spune, o rege!" Iar el a zis: "Mai încet!" Atunci au ie?it de la dânsul to?i cei ce stateau pe lânga el.
\par 20 ?i a intrat Aod la dânsul; caci el ?edea într-un foi?or racoros, pe care îl avea acolo la o parte. ?i a zis Aod: "Eu, o rege, am catre tine un cuvânt al lui Dumnezeu". Atunci Eglon s-a sculat de pe scaun înaintea lui.
\par 21 ?i când s-a sculat el, Aod ?i-a întins mâna sa stânga ?i a scos sabia de la coapsa sa dreapta ?i a împlântat-o în pântecele lui,
\par 22 A?a încât a intrat dupa ascu?i?ul sabiei ?i mânerul ?i grasimea a acoperit rana pe unde intrase sabia, caci Aod n-a scos-o din pântecele lui.
\par 23 Apoi Aod a ie?it în tinda, tragând dupa sine u?a foi?orului ?i încuind-o.
\par 24 Iar dupa ce a ie?it el, au venit slugile lui Eglon ?i, vazând u?a foi?orului încuiata, au zis: "Se vede ca el î?i acopera picioarele În camera de vara".
\par 25 ?i au a?teptat ei destula vreme; dar vazând ca nu mai deschide nimeni u?a foi?orului, au adus o cheie ?i au deschis ?i iata stapânul lor zacea mort la pamânt.
\par 26 Pâna sa se dumireasca aceia, Aod a plecat ?i nimeni nu se mai gândea la el; a trecut pe lânga idoli ?i a scapat în Seira.
\par 27 Iar dupa ce a venit în ?ara lui Israel, Aod a trâmbi?at din trâmbi?a pe muntele Efraim ?i s-au coborât la dânsul fiii lui Israel din muntele Efraim ?i el mergea înaintea lor.
\par 28 ?i ?a zis el catre dân?ii: "Veni?i dupa mine, ca a dat Domnul pe vrajma?ii no?tri Moabi?i în mâinile voastre". ?i s-au dus dupa dânsul ?i au apucat vadul Iordanului spre Moab ?i nu au lasat pe nimeni sa treaca.
\par 29 ?i au ucis atunci din Moabi?i pâna la zece mii de oameni, to?i sanato?i ?i voinici, încât nimeni n-a scapat.
\par 30 A?a au fost supu?i în ziua aceea Moabi?ii înaintea lui Israel ?i s-a lini?tit ?ara lui optzeci de ani. Iar Aod a fost judecatorul lor pâna la moartea sa.
\par 31 Dupa dânsul a fost judecator ?amgar, fiul lui Anat, care a ucis ?ase sute de Filisteni cu un ba?, cu bold de mânat boii, ?i acesta a izbavit de asemenea pe Israel.

\chapter{4}

\par 1 Dupa ce a murit Aod, fiii lui Israel au început iar sa faca rele înaintea ochilor Domnului.
\par 2 ?i Domnul i-a dat în mâinile lui Iabin, regele Canaanului, care domnea în Ha?or. Acesta avea capetenie peste o?tire pe Sisera care locuia în Haro?et-Goim.
\par 3 ?i au strigat fiii lui Israel catre Domnul; caci Iabin avea noua sute de care de fier ?i a apasat cumplit pe fiii lui Israel douazeci de ani.
\par 4 În vremea aceea era judecator în Israel Debora-prooroci?a, so?ia lui Lapidot.
\par 5 Aceasta locuia sub palmierul Deborei, între Rama ?i Betel, pe muntele Efraim ?i veneau acolo la ea fiii lui Israel sa se judece.
\par 6 ?i a trimis Debora de a chemat pe Barac, fiul lui Abinoam, din Chede?ul Neftalimului ?i i-a zis: "Domnul Dumnezeul lui Israel î?i porunce?te: Du-te ?i te suie pe muntele Tabor ?i ia cu tine zece mii de oameni din fiii lui Neftali ?i din fiii lui Zabulon;
\par 7 Iar Eu voi aduce la tine, la pârâul Chi?on, pe Sisera, capetenia o?tirilor lui Iabin ?i carele lui ?i oastea lui cea multa ?i-l voi da în mâinile tale".
\par 8 Iar Barac a zis catre dânsa: "De mergi tu cu mine, ma voi duce; iar de nu mergi cu mine, eu nu ma voi duce. Caci eu nu ?tiu ziua când are sa trimita Domnul pe îngerul Sau în ajutorul meu".
\par 9 Atunci ea a zis catre el: "De mers voi merge cu tine, dar sa ?tii ca nu va mai fi slava ta în calea aceasta în care mergi; ci în mâna unei femei va da Domnul pe Sisera". ?i s-a sculat Debora ?i s-a dus cu Barac la Chede?.
\par 10 ?i a chemat Barac pe Zabulonieni ?i Neftalimieni la Chede? ?i s-au dus dupa dânsul zece mii de oameni ?i s-a dus ?i Debora cu dân?ii.
\par 11 Atunci Heber Cheneul s-a despar?it de Chenei, fiii lui Hobab, rudenia lui Moise, ?i ?i-a întins cortul sau la dumbrava din ?aanaim, aproape de Chede?.
\par 12 ?i i s-a spus lui Sisera ca Barac, fiul lui Abinoam, s-a suit pe muntele Taborului.
\par 13 Atunci Sisera a adunat toate carele sale, noua sute de care de fier, ?i tot poporul pe care-l avea ?i a venit din Haro?et-Goim la râul Chi?on.
\par 14 Iar Debora a zis catre Barac: "Scoala, ca aceasta este ziua aceea în care Domnul are sa dea pe Sisera în mâinile tale. Însu?i Domnul are sa mearga înaintea ta". ?i s-a coborât Barac din muntele Taborului ?i dupa el ?i cei zece mii de oameni.
\par 15 Atunci Domnul a pus pe fuga pe Sisera ?i toate carele lui ?i toata tabara lui prin sabia lui Barac; ?i s-a coborât Sisera din carul sau ?i a fugit pe jos.
\par 16 Iar Barac a urmarit carele lui ?i tabara lui pâna la Haro?et-Goim ?i a cazut toata o?tirea lui Sisera de sabie ?i nimeni n-a ramas.
\par 17 Sisera însa a fugit pe jos în cortul Iaelei, femeia lui Heber Cheneul; caci între Iabin, regele Ha?orului, ?i casa lui Heber Cheneul era pace.
\par 18 ?i a ie?it Iaela în întâmpinarea lui Sisera ?i i-a zis: "Intra, domnul meu intra la mine, nu te teme!" ?i el a intrat la ea în cort ?i ea l-a acoperit cu haina sa.
\par 19 ?i a zis Sisera catre ea: "Da-mi putina apa sa beau, ca mi-e sete! ?i ea a dezlegat un burduf cu lapte ?i l-a adapat ?i iar l-a acoperit.
\par 20 Apoi Sisera i-a zis: "Stai la u?a cortului ?i de va veni cineva sa te întrebe ?i va zice: Nu este aici cineva? Tu sa zici: Nu!"
\par 21 Apoi Iaela, femeia lui Heber, a luat un ?aru? de la cort ?i un ciocan în mâna sa ?i s-a apropiat de el înceti?or ?i i-a înfipt ?aru?ul în tâmpla lui, a?a încât l-a pironit la pamânt, caci el dormea, fiind obosit; ?i a?a a murit.
\par 22 ?i iata Barac alerga în urmarirea lui Sisera. Atunci Iaela a ie?it în întâmpinarea lui ?i i-a zis: "Intra ?i-?i voi arata pe omul pe care tu îl cau?i". ?i el a intrat ?i iata Sisera zacea mort cu ?aru?ul în tâmpla.
\par 23 ?i a supus Domnul Dumnezeu în ziua aceea pe Iabin, regele Canaanului, în fa?a fiilor lui Israel.
\par 24 ?i s-a întarit mâna fiilor lui Israel din ce în ce mai mult asupra lui Iabin, regele Canaanului, pâna ce au stârpit pe Iabin, regele Canaaneilor,

\chapter{5}

\par 1 Atunci Debora ?i Barac, fiul lui Abinoam, au cântat cântarea aceasta:
\par 2 "Când se arata judecatori în Israel, poporul merge de bunavoie la razboi; Binecuvânta?i pe Domnul!
\par 3 Asculta?i dar, regilor! Capetenii, lua?i aminte! Caci cântare voi cânta Domnului. Cânta-voi Domnului Dumnezeului lui Israel:
\par 4 Când ie?eai Tu, Doamne, din Seir, Când treceai Tu prin câmpiile Edomului, Pamântul se cutremura ?i cerurile se topeau, Norii picurau picuri de ploaie,
\par 5 Mun?ii se naruiau înaintea Domnului, Ca ?i acest Sinai, la vederea Dumnezeului lui Israel.
\par 6 În zilele lui ?amgar, fiul lui Anat, Pe vremea Iaelei, drumurile erau pustii; Calatorii umblau atunci pe poteci ascunse.
\par 7 Satele în Israel erau pustii... pustii... Pâna m-am sculat eu, Debora, Pâna m-am ridicat eu, mama în Israel.
\par 8 Pe atunci Israel î?i alesese dumnezei noi, De aceea razboiul batea la por?i; Dar nu se vedea nici scut, nici lance în mâini, La cei patruzeci de mii din Israel.
\par 9 Inima îmi e la capeteniile lui Israel, La cei ce plecau din popor de voie la razboi. Binecuvânta?i pe Domnul!
\par 10 Aceia calareau pe asine murge, Sau ?edeau în caru?e, sub coviltire de scoar?e scumpe ?i mergeau pe drum cântând.
\par 11 ?i în rândurile o?tirii, ce tabara la fântâni, ?i acolo rasuna lauda Domnului, Lauda capeteniilor lui Israel. Atunci poporul Domnului ie?ea la por?i.
\par 12 De?teapta-te, Debora, de?teapta-te! De?teapta-te, de?teapta-te ?i cânta! Scoala ?i tu, Barac! Scoala fiul lui Abinoam ?i ia în robie pe cei ce te-au robit!
\par 13 Atunci poporul Domnului s-a trezit. Rama?i?a lui s-a strâns cu cei viteji.
\par 14 Din Efraim au purces capetenii în vale la Chi?on. Fratele tau Veniamin a fost printre o?tenii tai. Din Machir au venit capetenii ?i din Zabulon cârmuitori de o?tire.
\par 15 Principii din Israel au fost lânga Debora. Isahar, credincios lui Barac, Se îndrepta în urma lui spre vale. La pâraiele lui Ruben Sunt grele cumpene suflete?ti!
\par 16 Pentru ce ai ramas tu în mijlocul staulelor? Ca sa ascul?i behaitul turmelor? La pâraiele lui Ruben, Sunt grele cumpene suflete?ti!
\par 17 Galaadul ?ade lini?tit dincolo de Iordan. ?i Dan de ce sta el la corabiile sale? A?er sta pe malul marii ?i se odihne?te în limanurile sale.
\par 18 Zabulon este un popor ce înfrunta moartea, ?i este gata a-?i da via?a în lupta. Nu mai puf în ca el e Neftali, Care locuie?te podi?urile înalte.
\par 19 Atunci au venit regi sa se razboiasca, Razboitu-s-au atunci regii Canaanului La Taanac, pe apa Meghidonului, Dar n-au luat prada, nici argint, nici bani.
\par 20 Stelele de sus s-au luptat atunci, Din mersul lor s-au razboit cu Sisera.
\par 21 Pârâul Chi?on, pârâu stravechi! Pârâul Chi?on i-a maturat. Suflete al meu, calca-i în picioare!
\par 22 Atunci copitele cailor în ropot loveau pamântul, Fugeau vitejii lor sa-?i frânga gâtul.
\par 23 Blestem ceta?ii Meroz, zice îngerul Domnului! Blestem, blestem celor ce locuiesc în ea! Ca n-au venit în ajutorul Domnului, În ajutorul Domnului cu cei viteji.
\par 24 Binecuvântata sa fie între femei Iaela, femeia lui Heber Cheneul! Binecuvântata fie între femeile ?arii.
\par 25 Sisera i-a cerut apa; ea i-a dat lapte; în cupa scumpa i-a dat cel mai bun lapte.
\par 26 Cu stânga a apucat ?aru?ul, Iar cu dreapta ciocan greu de lucrator. Cu ciocanul a zdrobit lui Sisera capul, Cu ?aru?ul i-a strapuns tâmpla.
\par 27 Atunci a cazut îndata la picioarele ei ?i acolo a ramas. Cazut-a la picioarele ei ?i nu s-a mai sculat. Unde a cazut, acolo a ramas zdrobit.
\par 28 Pe fereastra printre gratii prive?te, Se uita mama lui Sisera ?i striga: "De ce nu mai vin oare carele lui  Oare de ce zabovesc ele a?a de mult?"
\par 29 Cea mai priceputa din femeile ei zice ?i singura raspunde la întrebarea sa:
\par 30 "Se vede ca au gasit ?i împart prada: O fata sau doua de cap de om, Haine pestri?e prada pentru Sisera, Prada de haine pestri?e cu aur cusute; Doua, trei ?aluri vargate, cusute cu aur, Pentru grumajii viteazului".
\par 31 A?a sa piara to?i vrajma?ii Tai, Doamne! Iar cei ce Te iubesc sa fie ca soarele Când rasare în toata stralucirea lui". Dupa aceea ?ara s-a bucurat de pace patruzeci de ani.

\chapter{6}

\par 1 Fiii lui Israel au început iara?i sa faca rele înaintea Domnului ?i Domnul i-a dat în mâinile Madiani?ilor pentru ?apte ani.
\par 2 ?i mâna Madiani?ilor era grea pentru Israel, ?i fiii lui Israel ?i-au facut, de raul Madiani?ilor, ascunzatori în mun?i ?i pe?teri ?i stânci greu de patruns.
\par 3 Când Israel semana, veneau Madiani?ii, Amaleci?ii ?i locuitorii din pustie la el
\par 4 ?i stateau la ei în corturi, mâncând roadele pamântului pâna la Gaza, ?i nu lasau pentru hrana lui Israel nici oaie, nici bou, nici asin.
\par 5 Caci ei veneau cu vitele ?i cu corturile lor ?i veneau mul?i ca lacustele; ei ?i camilele lor erau fara numar ?i cutreierau ?ara lui Israel ?i o pustiau.
\par 6 ?i Israel a saracit cumplit din pricina Madiani?ilor ?i a strigat catre Domnul.
\par 7 ?i când au strigat fiii lui Israel catre Domnul împotriva Madiani?ilor,
\par 8 A trimis Domnul prooroc la fiii lui Israel ?i le-a zis: "A?a graie?te Domnul Dumnezeul lui Israel: Eu v-am scos din ?ara Egiptului, Eu v-am scos din casa robiei;
\par 9 Eu v-am scapat din mâinile Egiptenilor ?i din mâinile tuturor celor ce va apasau, i-am  alungat de la voi ?i ?ara lor am dat-o voua,
\par 10 ?i v-am spus: Eu sunt Domnul Dumnezeul vostru; sa nu cinsti?i pe dumnezeii Amoreilor, în ?ara carora trai?i. Dar voi n-a?i ascultat glasul Meu".
\par 11 Atunci a venit îngerul Domnului ?i a ?ezut în Ofra sub un stejar, care era al lui Ioa?, tatal lui Abiezer; ?i fiul sau Ghedeon treiera atunci grâul în arie, ca sa-l ascunda de Madiani?i.
\par 12 ?i i s-a aratat îngerul Domnului ?i i-a zis: "Domnul este cu tine, voinicule!"
\par 13 Iar Ghedeon i-a zis: "Domnul meu, daca Domnul e cu noi, pentru ce ne-au ajuns pe noi toate necazurile acestea? ?i unde sunt oare toate minunile Lui de care ne-au istorisit noua parin?ii no?tri când ne spuneau: Din Egipt ne-a scos pe noi Domnul. Acum însa ne-a parasit Domnul ?i ne-a dat în mâinile Madiani?ilor".
\par 14 ?i cautând Domnul spre el, a zis: "Mergi cu aceasta putere a ta ?i izbave?te pe Israel din mâinile Madiani?ilor. Iata, Eu te trimit!"
\par 15 Atunci Ghedeon a zis: "Doamne, cum sa izbavesc eu pe Israel? Iata neamul meu este cel mai sarac din semin?ia lui Manase, iar eu sunt cel mai mic în casa tatalui meu". Domnul însa i-a zis:
\par 16 "Eu voi fi cu tine ?i tu vei bate pe Madiani?i, ca pe un singur om".
\par 17 A zis Ghedeon catre Dânsul: "De am aflat eu trecere în ochii Tai, arata-mi un semn, ca sa-mi dovede?ti cele ce-mi vorbe?ti:
\par 18 Sa nu Te duci de aici, pâna nu ma voi întoarce la Tine ?i-mi voi aduce darul meu ?i ?i-l voi da". ?i Domnul a zis: "Voi sta pâna te vei întoarce".
\par 19 ?i s-a dus Ghedeon ?i a gatit un ied ?i azime din o efa de faina; carnea a pus-o într-un co?, iar zeama a turnat-o într-o oala ?i a dus-o la El sub stejar ?i I-a pus-o înainte.
\par 20 ?i a zis catre dânsul îngerul Domnului: "Ia carnea ?i azimile ?i pune-le pe piatra aceasta ?i toarna zeama peste ele". ?i a facut Ghedeon a?a.
\par 21 Atunci îngerul Domnului, întinzându-?i vârful toiagului ce-l avea în mâna sa, s-a atins de carne ?i de azime; ?i a ie?it foc din piatra ?i a mistuit carnea ?i azimile; ?i îngerul Domnului s-a facut nevazut de la ochii lui.
\par 22 ?i a cunoscut Ghedeon ca acesta este îngerul Domnului, ?i a zis Ghedeon: "Vai de mine, Stapâne Doamne, ca am vazut pe îngerul Domnului fa?a catre fa?a!"
\par 23 Zis-a Domnul: "Pace ?ie. Nu te teme, caci nu vei muri!"
\par 24 ?i a facut acolo Ghedeon un jertfelnic Domnului ?i l-a numit "Iahve-?alom". ?i se afla acesta ?i astazi în Ofra lui Abiezer.
\par 25 În noaptea aceea i-a zis Domnul: "Ia un vi?el din cireada tatalui tau ?i un taur de ?apte ani ?i sfarâma jertfelnicul lui Baal pe care-l are tatal tau ?i taie copacul cel sfânt de lânga el;
\par 26 ?i zide?te un jertfelnic în cinstea Domnului Dumnezeului tau, Care ?i S-a aratat pe vârful stâncii acesteia; apoi ia taurul ?i-l adu ardere de tot pe lemnele copacului pe care ai sa-l tai".
\par 27 Atunci Ghedeon a luat zece oameni dintre slugile sale ?i a facut cum îi graise Domnul. ?i fiindca se temea de casnicii tatalui sau ?i de oamenii din cetate sa faca acestea ziua, le-a facut noaptea.
\par 28 ?i când s-au sculat diminea?a locuitorii ceta?ii, au vazut jertfelnicul lui Baal darâmat ?i copacul cel de lânga el taiat ?i taurul adus ardere de tot pe jertfelnicul cel nou.
\par 29 ?i ziceau unii catre al?ii: "Cine oare a facut acestea?" Iar dupa ce au cercetat ?i au întrebat, au zis: "Ghedeon, fiul lui Ioa?, a facut acestea!"
\par 30 Atunci au zis locuitorii ceta?ii catre Ioa?: "Scoate pe fiul tau, ca trebuie sa moara, pentru ca a darâmat jertfelnicul lui Baal ?i a taiat copacul cel de lânga el".
\par 31 Iar Ioa? a zis celor ce venisera la dânsul: "Voi oare vre?i sa trece?i de partea lui Baal? Vreri voi oare sa-l apara?i? Cine va trece de partea lui acela va fi dat mor?ii, chiar în diminea?a aceasta; daca el este dumnezeu, sa se apere singur pe sine pentru ca i s-a stricat jertfelnicul".
\par 32 Din acea zi au început a numi pe Ghedeon, Ierubaal, pentru ca ziceau: "Sa se judece singur Baal cu dânsul, pentru ca i-a stricat jertfelnicul".
\par 33 În timpul acesta to?i Madiani?ii, Amaleci?ii ?i locuitorii Rasaritului s-au adunat împreuna, au trecut râul ?i ?i-au a?ezat tabara în valea Izreel.
\par 34 Atunci a cuprins Duhul Domnului pe Ghedeon ?i a trâmbi?at acesta din trâmbi?a ?i a fost chemata familia lui Abiezer sa mearga cu dânsul.
\par 35 Apoi s-au trimis soli prin toata semin?ia lui Manase ?i aceasta a raspuns ca merge cu dânsul. ?i tot a?a s-au trimis soli la A?er, la Zabulon ?i la Neftali ?i au venit ?i ace?tia în întâmpinarea lor.
\par 36 Atunci a zis Ghedeon catre Dumnezeu: "De vrei sa izbave?ti pe Israel prin mâna mea, cum zici,
\par 37 Apoi iata eu întind aici în arie lâna ce am tuns; ?i de va fi roua numai pe lâna, iar încolo peste tot locul uscaciune, atunci voi ?ti ca vei izbavi pe Israel prin mâna mea, cum ai zis".
\par 38 ?i s-a facut a?a; ?i a doua zi s-a sculat Ghedeon dis-de-diminea?a ?i s-a apucat sa stoarca lâna ?i a stors roua din lâna un vas plin de apa.
\par 39 Apoi iara?i a zis Ghedeon catre Domnul: "Sa nu Te mânii pe mine, daca am sa mai zic o data ?i daca am sa mai fac numai o încercare cu lâna: sa fie uscata numai lâna, iar peste tot locul sa fie roua".
\par 40 ?i a facut a?a Dumnezeu în noaptea aceea: a fost uscaciune numai pe lâna, iar peste tot locul a fost roua.

\chapter{7}

\par 1 Atunci s-a sculat Ierubaal, adica Ghedeon, ?i tot poporul care era cu dânsul dis-de-diminea?a ?i au tabarât la En-Harod, iar tabara Madiani?ilor era spre miazanoapte de dânsul pe colina More cea din ?es.
\par 2 Iar Domnul a zis catre Ghedeon: "E prea mult popor cu tine; nu voi putea Eu sa dau pe Madian în mâinile lor, ca sa nu se mândreasca Israel înaintea Mea ?i sa nu zica: Mâna mea m-a izbavit!
\par 3 De aceea graie?te în auzul poporului ?i zi: Cine este fricos ?i se teme, acela sa se întoarca ?i sa se duca înapoi din Muntele Galaad". ?i s-au întors din popor douazeci ?i doua de mii ?i au ramas zece mii.
\par 4 Apoi a zis Domnul catre Ghedeon: "Tot e prea mult popor; du-l la apa; acolo ?i-l voi alege. ?i de care voi zice sa mearga cu tine, acela sa mearga cu tine, iar de care î?i voi zice ca nu trebuie sa mearga cu tine, acela sa nu mearga".
\par 5 ?i a dus el poporul la apa, iar Domnul a zis catre Ghedeon: "Cine va limpai apa cu limba din pumni, cum limpaie câinele, pe acela sa-l pui deoparte; de asemenea sa pui deoparte ?i pe to?i aceia care-?i vor pleca genunchii ?i vor bea apa".
\par 6 ?i a fost numarul celor ce au limpait cu limba lor din pumni trei sute de oameni; iar tot celalalt popor s-a plecat pe genunchii sai sa bea apa.
\par 7 Atunci a zis Domnul catre Ghedeon: "Cu cei trei sute care au limpait am sa va izbavesc Eu ?i am sa dau pe Madiani?i în mâinile voastre, iar tot poporul celalalt sa se duca fiecare la locul sau".
\par 8 ?i au luat de la popor merindele ?i trâmbi?ele; apoi a dat Ghedeon drumul tuturor Israeli?ilor pe la corturi ?i a oprit la sine pe cei trei sute de oameni, iar tabara Madiani?ilor era din jos de el, în vale.
\par 9 În noaptea aceea i-a zis Domnul: "Scoala ?i te coboara la tabara, ca Eu o voi da în mâinile tale.
\par 10 Daca însa te temi sa te duci singur, atunci du-te la tabara tu ?i Pura, sluga ta,
\par 11 ?i ai sa auzi ce se graie?te ?i atunci au sa se îmbarbateze mâinile tale ?i ai sa te duci în tabara". ?i s-a dus el ?i Pura, sluga sa, pâna la cele dintâi straji ale taberei.
\par 12 Iar Madiani?ii ?i Amaleci?ii ?i to?i locuitorii Rasaritului se a?ezasera în vale atât de mul?i, ca lacustele; camilele nu mai aveau numar ?i erau multe, ca nisipul de pe malurile marii.
\par 13 Ghedeon veni. ?i iata unul povestea altuia un vis ?i zicea: "Am visat parca o pâine rotunda de orz, ce se rostogolea prin tabara madianita ?i, ajungând la un cort, a izbit în el a?a de tare, încât el a cazut, s-a rasturnat ?i s-a desfacut".
\par 14 Celalalt i-a raspuns: "Aceasta nu este alta decât sabia lui Ghedeon, fiul lui Ioa? israelitul; Dumnezeu a dat în mâna lui pe Madiani?i ?i toata tabara".
\par 15 Auzind povestirea visului ?i talmacirea lui, Ghedeon s-a închinat Domnului ?i s-a întors în tabara israelita, zicând: "Scula?i! Domnul a dat tabara Madiani?ilor în mâinile noastre".
\par 16 Apoi a împar?it pe cei trei sute de oameni în trei cete ?i le-a dat la to?i în mâini trâmbi?e ?i oale goale ?i în oale faclii.
\par 17 ?i le-a zis: "Sa va uita?i la mine ?i sa face?i ce voi face eu; iata eu ma duc la tabara ?i ce voi face eu, sa face?i ?i voi.
\par 18 Când eu ?i cei cu mine vom trâmbi?a, sa trâmbi?a?i ?i voi din trâmbi?ele voastre împrejurul întregii tabere ?i sa striga?i: Sabia Domnului ?i a lui Ghedeon!"
\par 19 ?i s-a apropiat de tabara Ghedeon ?i cu el o suta de oameni, pe la începutul strajii de mijloc a nop?ii, ?i au de?teptat strajile ?i au trâmbi?at din trâmbi?e ?i au sfarâmat oalele pe care le aveau în mâini.
\par 20 ?i au trâmbi?at tustreile cete din trâmbi?e ?i au spart oalele ?i ?ineau în mâna stânga faclia, iar în mâna dreapta trâmbi?ele ?i trâmbi?au din trâmbi?e ?i strigau: "Sabia Domnului ?i a lui Ghedeon!"
\par 21 ?i stateau fiecare la locul sau împrejurul taberei ?i au început cei din tabara a alerga în toata tabara ?i a striga ?i au luat-o la fuga.
\par 22 Pe când cei trei sute de oameni sunau din trâmbi?e, în toata tabara a întors Domnul sabia unora asupra altora, ?i a fugit tabara catre ?erera pâna la Bet?ita ?i pâna la hotarele lui Abelmehola, aproape de Tabat.
\par 23 Atunci au fost chema?i Israeli?ii din semin?iile lui Neftali ?i A?er ?i din toata semin?ia lui Manase ?i au alergat dupa Madiani?i.
\par 24 Iar Ghedeon a trimis soli în tot muntele lui Efraim sa spuna: "Ie?i?i înaintea Madiani?ilor ?i prinde?i vadul înaintea lor pâna la Betbara ?i Iordan". ?i s-au adunat to?i Efraimi?ii ?i au prins vadul pâna la Betbara ?i Iordan.
\par 25 ?i au prins pe cele doua capetenii ale Madiani?ilor: pe Oreb ?i Zeeb; au ucis pe Oreb la ?ur-Oreb, iar pe Zeeb la Iecheb-Zeeb; ?i au urmarit pe Madiani?i; iar capetele lui Oreb ?i Zeeb le-au adus la Ghedeon, dincolo de Iordan.

\chapter{8}

\par 1 Zis-au Efraimi?ii catre el: "De ce ai facut a?a?i nu ne-ai chemat când ai mers sa te lup?i cu Madiani?ii?" ?i s-au certat stra?nic cu el.
\par 2 Iar Ghedeon le-a raspuns: "Facut-am eu oare ceva la fel cu ceea ce a?i facut voi? Nu e mai fericit oare Efraim ca a cules toata via, decât Abiezer care s-a ales cu câ?iva ciorchini?
\par 3 În mâinile voastre a dat Dumnezeu pe capeteniile Madiani?ilor Oreb ?i Zeeb ?i ce-am putut sa fac eu asemenea cu ce a?i facut voi?" Atunci s-a lini?tit duhul lor cel întarâtat împotriva lui, când le-a spus asemenea cuvinte.
\par 4 Apoi a venit Ghedeon la Iordan ?i a trecut ?i el ?i cei trei sute de oameni care erau cu dânsul ?i care obosisera ?i flamânzisera, urmarind pe du?man.
\par 5 El a zis catre locuitorii din Sucot: "Da?i pâine oamenilor care merg cu mine, caci sunt obosi?i ?i urmarim pe Zebah ?i pe ?almuna, regii Madiani?ilor".
\par 6 Iar capeteniile din Sucot au raspuns: "Dar este oare mâna lui Zebah ?i ?almuna în stapânirea ta, ca sa dam pâine o?tirii tale?"
\par 7 Atunci Ghedeon a zis: "Când va da Domnul pe Zebah ?i pe ?almuna în mâna mea, am sa scarpin trupul vostru cu spinii pustiului ?i cu maracini".
\par 8 Dupa aceea s-a dus el la Penuel ?i a spus la fel locuitorilor lui; dar locuitorii din Penuel i-au raspuns la fel cum raspunsesera ?i cei din Sucot.
\par 9 ?i a zis el ?i locuitorilor din Penuel: "Daca ma voi întoarce biruitor, am sa darâm turnul acesta".
\par 10 Zebah ?i ?almuna erau în Carcor ?i cu ei erau o?tirile lor pâna la cincisprezece mii de oameni, to?i cei ce mai ramasesera din toata o?tirea locuitorilor Rasaritului; cazusera însa o suta douazeci de mii de oameni purtatori de sabie.
\par 11 ?i s-a dus Ghedeon la cei ce traiau în corturi la Rasarit de Nobah ?i de Iogbeha, ?i au lovit tabara tocmai când erau mai fara grija.
\par 12 Atunci Zebah ?i ?almuna au fugit, iar el a alergat dupa dân?ii ?i a prins pe amândoi regii Madiani?ilor, pe Zebah ?i pe ?almuna, ?i a pus toata tabara în învalma?eala.
\par 13 Apoi s-a întors Ghedeon, fiul lui Ioa?, de la razboi de pe colina Heres.
\par 14 ?i a prins un tânar locuitor din Sucot ?i l-a întrebat ?i acesta i-a în?irat în scris pe capeteniile ?i batrânii Sucotului, care erau în numar de ?aptezeci ?i ?apte de oameni.
\par 15 Apoi a venit la locuitorii Sucotului ?i a zis: "Iata Zebah ?i ?almuna, din pricina carora a?i râs de mine ?i mi-a?i zis: Au doara mâna lui Zebah ?i ?almuna e în stapânirea ta, ca sa dam pâine oamenilor tai celor obosi?i?"
\par 16 Dupa aceea a luat spini din pustiu ?i maracini ?i a pedepsit cu ei pe batrânii ceta?ii ?i pe locuitorii din Sucot.
\par 17 ?i turnul din Penuel l-a darâmat, iar pe locuitorii ceta?ii i-a ucis.
\par 18 ?i a zis catre Zebah ?i ?almuna: "Ce fel erau aceia pe care i-a?i ucis voi în Tabor?" Zis-au ei: "A?a, cum e?ti ?i tu; fiecare avea înfa?i?area unui fiu de rege".
\par 19 Iar Ghedeon a zis: "Aceia erau fra?ii mei, fiii mamei mele! Viu este Domnul, de i-a?i fi lasat cu via?a, eu nu v-a? ucide!"
\par 20 Apoi a zis catre Ieter, întâiul sau nascut: "Scoala ?i-i ucide". Dar tânarul nu ?i-a scos sabia, caci s-a temut, pentru ca era înca tânar.
\par 21 Zis-au Zebah ?i ?almuna: "Scoala tu ?i ne ucide, pentru ca dupa cum este omul a?a este ?i puterea lui!" ?i s-a sculat Ghedeon ?i a ucis pe Zebah ?i ?almuna ?i a luat frâiele de la gâtul camilelor lor.
\par 22 Dupa aceea au zis Israeli?ii catre Ghedeon: "Domne?te peste noi tu ?i fiul tau ?i fiul fiului tau, pentru ca ne-ai izbavit din mâinile Madiani?ilor!"
\par 23 Iar Ghedeon le-a zis: "Nici eu nu voi domni peste voi, nici fiul meu nu va domni peste voi, ci Domnul sa domneasca peste voi!
\par 24 Dar am sa va rog ?i eu un lucru, a adaugat Ghedeon, sa-mi dea fiecare din voi câte un cercel din prazile voastre; caci vrajma?ii aveau mul?i cercei de aur, pentru ca erau Ismaeli?i".
\par 25 Ei au zis: "Î?i vom da". ?i au întins o manta ?i au aruncat acolo fiecare câte un cercel din prada sa.
\par 26 ?i greutatea cerceilor de aur pe care i-a cerut el a fost o mie ?apte sute de sicli de aur, afara de verigi, de nasturi ?i de hainele de purpura de pe cei doi regi ai Madiani?ilor ?i afara de lan?urile lor de aur care erau la gâtul camilelor lor.
\par 27 Din acestea a facut Ghedeon un efod ?i l-a pus în cetatea sa, în Ofra; ?i a fost aceasta pricina de pacat pentru tot Israelul ?i cursa pentru Ghedeon ?i pentru toata casa lui.
\par 28 Astfel s-au supus Madiani?ii înaintea fiilor lui Israel ?i nu s-au mai apucat sa-?i ridice capul ?i s-a odihnit ?ara patruzeci de ani, în zilele lui Ghedeon.
\par 29 Apoi s-a dus Ierubaal, fiul lui Ioa?, ?i a trait în casa sa.
\par 30 ?i a avut Ghedeon ?aptezeci de fii care au rasarit din coapsele lui, caci el a avut femei multe.
\par 31 De asemenea i-a nascut un fiu ?i concubina sa care traia în Sichem ?i el i-a pus numele Abimelec.
\par 32 Apoi a murit Ghedeon, fiul lui Ioa?, la batrâne?i adânci, ?i a fost înmormântat în mormântul tatalui sau Ioa?, în Ofra lui Abiezer.
\par 33 Dupa ce a murit Ghedeon, fiii lui Israel au început iara?i a pacatui pe urma baalilor ?i ?i-au a?ezat ca dumnezeu pe Baal-Berit;
\par 34 Nu ?i-au mai adus aminte fiii lui Israel de Domnul Dumnezeul lor, Care îi izbavise din mâinile tuturor vrajma?ilor care îi înconjurau.
\par 35 Casei lui Ierubaal, adica a lui Ghedeon, nu i-au dat nici o cinste pentru toate binefacerile pe care acesta le facuse întregului Israel.

\chapter{9}

\par 1 În vremea aceea Abimelec, fiul lui Ierubaal, s-a dus la Sichem, la fra?ii mamei sale, ?i a grait cu el ?i cu tot neamul tatalui mamei sale ?i a zis:
\par 2 "?opti?i la to?i locuitorii din Sichem: Cum e mai bine pentru voi: sa domneasca peste voi to?i cei ?aptezeci de fii ai lui Ierubaal sau sa domneasca numai unul? ?i aminti?i-va ca eu sunt osul vostru ?i carnea voastra!"
\par 3 ?i au ?optit fra?ii mamei sale din partea lui toate cuvintele acestea locuitorilor din Sichem. ?i s-a înduplecat inima acestora pentru Abimelec, caci î?i ziceau a?a: "E fratele nostru!"
\par 4 ?i i-au dat ?aptezeci de sicli de argint din casa lui Baal-Berit, iar Abimelec ?i-a tocmit cu ei oameni rai ?i fara capatâi care s-au ?i dus cu el.
\par 5 Apoi a venit la casa tatalui sau în Ofra ?i a ucis pe fra?ii sai, pe cei ?aptezeci de fii ai lui Ierubaal, pe o piatra, ramânând numai Iotam, fiul cel mai mic al lui Ierubaal, pentru ca s-a ascuns.
\par 6 Dupa aceea s-au adunat to?i locuitorii Sichemului ?i toata casa lui Milo ?i s-au dus de au pus rege pe Abimelec la stejarul cel de lânga Sichem.
\par 7 Iar daca s-a spus acestea lui Iotam, acesta s-a dus ?i a stat pe vârful muntelui Garizim ?i, ridicându-?i glasul, a strigat ?i a zis: "Asculta?i-ma, locuitori ai Sichemului, ?i Dumnezeu sa va asculte!
\par 8 S-au dus odata copacii sa-?i unga împarat peste ei. ?i au zis catre maslin: Domne?te peste noi!
\par 9 Iar maslinul a zis: Lasa-voi eu oare grasimea mea, cu care se cinste?te Dumnezeu ?i oamenii se mândresc ?i ma voi duce sa umblu prin copaci?
\par 10 Atunci copacii au zis catre smochin: Vino tu ?i domne?te peste noi!
\par 11 Dar ?i smochinul a raspuns  Sa-mi las eu oare dulcea?a mea ?i fructul meu cel bun ?i sa ma duc sa cârmuiesc copacii?
\par 12 Apoi au zis copacii catre vi?a de vie: Vino tu de domne?te paste noi!
\par 13 ?i vi?a de vie a zis catre ei: Cum sa-mi las eu mustul meu care vesele?te pe Dumnezeu ?i pe oameni ?i sa ma duc sa cârmuiesc copacii?
\par 14 În cele din urma au zis to?i copacii catre un spin: Vino tu ?i domne?te peste noi!
\par 15 Iar spinul a zis catre copaci: Daca voi ma pune?i cu adevarat împarat peste voi, atunci veni?i ?i va odihni?i sub umbra mea; iar de nu, atunci va ie?i foc din spini ?i va arde cedrii Libanului.
\par 16 A?adar lua?i seama: Dupa dreptate ?i dupa adevar v-a?i purtat voi, când a?i pus rege pe Abimelec? ?i bine a?i facut ce aii facut cu Ierubaal ?i cu casa lui? ?i v-a?i purtat oare potrivit Cu binefacerile lui?
\par 17 Tatal meu a luptat pentru voi, fara sa-?i cru?e via?a, ?i v-a izbavit din mâna Madiani?ilor;
\par 18 Iar voi v-a?i sculat acum împotriva casei tatalui meu ?i a?i ucis pe cei ?aptezeci de feciori ai tatalui meu pe o piatra ?i a?i pus rege peste locuitorii Sichemului pe Abimelec, fiul unei roabe a lui, pentru ca e fratele vostru.
\par 19 Daca voi v-a?i purtat dupa adevar ?i dupa dreptate cu Ierubaal ?i cu casa lui, atunci sa fie asupra voastra binecuvântare ?i sa va bucura?i de Abimelec ?i sa se bucure ?i el de voi!
\par 20 Daca insa nu, atunci sa iasa foc din Abimelec ?i sa arda pe locuitorii Sichemului ?i toata casa lui Milo; sa iasa foc din locuitorii Sichemului ?i din casa lui Milo ?i sa arda pe Abimalec".
\par 21 Apoi a fugit Iotam ?i s-a facut nevazut ?i s-a dus la Beer ?i a trait acolo, ascunzându-se de fratele sau Abimelec.
\par 22 Iar Abimelec a domnit paste Israel trei ani.
\par 23 Dupa aceea a trimis Dumnezeu un duh rau între Abimelec ?i între locuitorii Sichemului, nemaivoind locuitorii din Sichem sa se supuna lui Abimelec;
\par 24 Ca astfel sa vina razbunarea pentru cei ?aptezeci de fii ai iui Ierubaal ?i sângele lor sa se întoarca asupra lui Abimelec, fratele lor, care-i ucisese, ?i asupra locuitorilor Sichemului care au îmbarbatat mâna lui ca sa-?i ucida fra?ii.
\par 25 ?i au trimis locuitorii Sichemului împotriva lui oameni la pânda pe vârfurile mun?ilor, ca sa prade pe oricine va trece pe lânga ei pe cale. ?i s-a spus aceasta lui Abimelec.
\par 26 Atunci a venit ?i Gaal, fiul lui Ebed, cu fra?ii sai, la Sichem ?i au umblat ei prin Sichem; iar locuitorii Sichemului s-au încrezut în el.
\par 27 Apoi au ie?it ei în ?arina ?i au cules viile, au stors strugurii, au facut praznic ?i s-au dus la casa dumnezeului lor, unde au mâncat ?i au baut ?i au blestemat pe Abimelec.
\par 28 Gaal însa, fiul lui Ebed, zicea: "Cine este Abimelec ?i ce este Sichemul, ca sa-i slujim? Nu este el, oare, fiul lui Ierubaal, ?i capetenia cea mai de seama a Sichemului nu este oare Zebul? Sa sluji?i mai bine urma?ilor lui Hemor, tatal lui Sichem, iar aceluia pentru ce sa-i slujim?
\par 29 De mi-ar da cineva poporul acesta pe mâna mea, eu a? alunga pe Abimelec". Atunci s-a zis lui Abimelec: "Înmul?e?te-?i o?tirea ?i ie?i!
\par 30 Iar Zebul, capetenia ceta?ii, a aflat ce zisese Gaal, fiul lui Ebed, ?i s-a aprins de mânie.
\par 31 Apoi a trimis el cu vicle?ug soli la Abimelec, ca sa-i spuna: "Iata Gaal, fiul lui Ebed, ?i fra?ii lui au venit în Sichem ?i a?â?a cetatea împotriva ta.
\par 32 Scoala dar la noapte, tu ?i poporul care e eu tine, ?i stai de pânda în câmp;
\par 33 Iar diminea?a, la rasaritul soarelui, scoala repede ?i înainteaza spre cetate; ?i când ei ?i poporul ce este cu ei var ie?i la tine, atunci sa faci cu ei ce se va pricepe mâna ta".
\par 34 S-a sculat deci Abimelec noaptea ?i tot poporul ce era cu dânsul ?i au stat da pânda la Sichem patru cete.
\par 35 Iar diminea?a, Gaal, fiul lui Ebed, a ie?it ?i a stat în poarta ceta?ii. Atunci s-a sculat Abimelec ?i poporul ce era cu el în ascunzatoare.
\par 36 Gaal însa, vazând poporul, a zis câtre Zebul: "Iata poporul se coboara de pe vârful mun?ilor". Iar Zebul i-a raspuns: "Umbrele mun?ilor ?i se par oameni".
\par 37 ?i a grait iara?i Gaal ?i a zis: "Iata poporul se coboara de pe înal?ime ?i o ceata vine de la stejarul Meanim".
\par 38 A zis atunci Zebul: "Unde sunt buzele tale care ziceau: "Cine este Abimelec, ca sa-i slujim lui? Acesta este poporul pe care tu l-ai nesocotit. Ie?i acum ?i te lupta cu dânsul!"
\par 39 ?i s-a dus Gaal în fruntea locuitorilor Sichemului ?i s-a luptat cu Abimelec.
\par 40 ?i s-a napustit Abimelec asupra lui ?i el a fugit de dânsul ?i au cazut mul?i uci?i pâna la por?ile ceta?ii.
\par 41 Abimelec însa a ramas în Aruma; iar pe Gaal ?i pe fra?ii lui i-a alungat Zebul, ca sa nu mai locuiasca Sichem.
\par 42 A doua zi a ie?it poporul la câmp ?i au spus despre acestea lui Abimelec.
\par 43 Iar acesta ?i-a luat poporul sau ?i l-a împar?it în trei cete ?i l-a pus la pânda în câmp. ?i vazând ca a ie?it popor din cetate, s-a ridicat asupra acelora ?i i-a ucis.
\par 44 Pe când Abimelec ?i cetele ce erau cu dânsul s-au apropiat ?i s-au oprit la por?ile ceta?ii, celelalte doua cete, tabarând asupra tuturor celor ce erau În câmp, i-au ucis.
\par 45 ?i s-a luptat Abimelec cu cetatea toata ziua aceea, a luat cetatea, a ucis poporul care era în ea ?i a darâmat cetatea ?i a presarat locul ei cu sare.
\par 46 Auzind de acestea, to?i acei ce erau în turnul Sichemului s-au dus în turnul capi?tei lui Baal-Berit.
\par 47 Dar i s-a spus lui Abimelec ca s-au adunat acolo to?i cei ce fusesera în turnul Sichemului.
\par 48 Atunci Abimelec s-a dus în muntele ?almon, el ?i tot poporul ce era eu dânsul; a luat Abimelec cu sine topoare ?i a taiat lemne din padure ?i le-a pus pe umar ?i a zis catre popor: "A?i vazut ce am facut eu? Face?i repede ?i voi ceea ce am facut eu!"
\par 49 ?i a taiat fiecare din popor lemne ?i s-au dus to?i cu Abimelec ?i le-au pus sub turn ?i au aprins cu ele turnul; ?i au murit to?i cei ce erau în turnul Sichemului, aproape o mie de barba?i ?i de femei.
\par 50 Dupa aceea s-a dus Abimelec la Teve? ?i a împresurat Teve?ul ?i l-a luat.
\par 51 ?i era în mijlocul ceta?ii un turn întarit ?i au fugit acolo to?i barba?ii ?i femeile ?i to?i oamenii din cetate; ?i l-au încuiat ?i s-au suit pe acoperi?ul turnului.
\par 52 Abimelec însa a venit la turn ?i l-a împresurat ?i s-a apropiat de u?a turnului ca sa-i dea foc.
\par 53 Atunci o femeie a aruncat o bucata de piatra de râ?ni?a în capul lui Abimelec ?i i-a spart capul.
\par 54 Abimelec a chemat îndata un tânar, care era purtatorul de arme al sau, ?i i-a zis: "Scoate-?i sabia ?i ma ucide, ca sa nu zica despre mine: A fost ucis de o femeie". ?i l-a strapuns tânarul acela ?i a murit.
\par 55 Când au vazut Israeli?ii ca a murit Abimelec, s-a dus fiecare la locul sau.
\par 56 A?a a platit Dumnezeu lui Abimelec, pentru nelegiuirea pe care el o faptuise fa?a de tatal sau, ucigând pe cei ?aptezeci de fra?i ai sai.
\par 57 ?i toate nelegiuirile locuitorilor Sichemului le-a întors Dumnezeu asupra capului lor. ?i a?a i-a ajuns blestemul lui Iotam, fiul lui Ierubaal.

\chapter{10}

\par 1 Dupa Abimelec s-a ridicat ca sa izbaveasca pe Israel Tola, fiul lui Pua, fiul lui Dodo, din semin?ia lui Isahar. Acesta traia în ?amir, pe muntele lui Efraim.
\par 2 ?i a fost el judecatorul lui Israel douazeci ?i trei de ani ?i; murind, a fost îngropat în ?amir.
\par 3 Dupa dânsul s-a sculat Iair din Galaad ?i a fost judecator lui Israel douazeci ?i doi de ani.
\par 4 Acesta a avut treizeci ?i doi de fii, care calareau pe treizeci ?i doi de asini ?i aveau treizeci ?i doua de ceta?i.
\par 5 Murind Iair, a fost îngropat în Camon.
\par 6 Dar fiii lui Israel au facut iara?i rele înaintea ochilor Domnului ?i au slujit baalilor ?i astartelor ?i dumnezeilor Amoreilor, dumnezeilor Sidonului, dumnezeilor Amoni?ilor, dumnezeilor Moabi?ilor ?i dumnezeilor Filistenilor, iar pe Domnul L-au parasit ?i nu L-au slujit.
\par 7 Atunci s-a aprins mânia Domnului asupra lui Israel, ?i l-a dat în mâinile Filistenilor ?i în mâinile Amoni?ilor.
\par 8 Ace?tia au strâmtorat ?i au chinuit pe fiii lui Israel din anul acela optsprezece ani în ?ir, adica pe to?i fiii lui Israel de dincolo de Iordan, din ?ara Amoreilor, care este în Galaad.
\par 9 Iar Amoni?ii au trecut Iordanul, ca sa se razboiasca cu Iuda, cu Veniamin ?i cu casa lui Efraim. A?a ca fiii lui Israel erau foarte strâmtora?i.
\par 10 Atunci au strigat fiii lui Israel catre Domnul ?i au zis: "Gre?it-am înaintea Ta, pentru ca am parasit pe Dumnezeul nostru ?i am slujit baalilor".
\par 11 Domnul însa a zis catre fiii lui Israel: "Nu v-au împilat oare Egiptenii, Amoreii, Amoni?ii ?i Filistenii,
\par 12 Sidonienii, Amaleci?ii ?i Moabi?ii, ?i când a?i strigat catre Mine, nu v-am izbavit Eu oare din mâinile lor?
\par 13 Dar voi M-a?i parasit iara?i ?i v-a?i apucat sa sluji?i la al?i dumnezei. De aceea nu va voi mai izbavi.
\par 14 Merge?i ?i striga?i catre dumnezeii pe care vi i-a?i ales; sa va izbaveasca aceia la vreme de necaz!"
\par 15 Iar fiii lui Israel au zis catre Domnul: "Gre?it-am! Fa cu noi cum vei crede ca e mai bine, numai izbave?te-ne ?i acum".
\par 16 ?i au lepadat de la ei pe dumnezeii cei straini ?i au început sa slujeasca numai Domnului. ?i S-a îndurat Domnul de suferin?ele lui Israel.
\par 17 Amoni?ii însa s-au adunat ?i ?i-au a?ezat tabara în Galaad. S-au adunat de asemenea ?i fiii lui Israel ?i au tabarât la Mi?pa.
\par 18 Atunci poporul ?i capeteniile Galaadului au zis unii catre al?ii: "Cine va începe lupta contra Amoni?ilor acela va fi capetenie peste to?i locuitorii Galaadului".

\chapter{11}

\par 1 Ieftae Galaaditul era un luptator viteaz. Acesta era fiul unei desfrânate care nascuse lui Galaad pe Ieftae.
\par 2 Dar ?i so?ia lui Galaad i-a nascut acestuia fii. Iar daca s-au facut mari, fiii so?iei au izgonit pe Ieftae, zicându-i: "Tu nu e?ti mo?tenitor în casa tatalui nostru, pentru ca e?ti feciorul alte femei".
\par 3 Atunci Ieftae a fugit de fra?ii sai ?i a trait în ?inutul Tob. Acolo s-au adunat împrejurul lui Ieftae oameni fara capatâi ?i umblau cu dânsul.
\par 4 Dupa câtva timp Amoni?ii s-au ridicat cu razboi împotriva lui Israel.
\par 5 Iar în timpul razboiului Amoni?ilor cu Israeli?ii, au venit batrânii Galaadului sa ia pe Ieftae din ?inutul Tob,
\par 6 ?i au zis catre Ieftae: "Vino sa ne fii capetenie ?i te lupta cu Amoni?ii".
\par 7 Ieftae însa a zis catre batrânii Galaadului: "Oare nu m-a?i urât voi ?i m-a?i alungat din casa tatalui meu? La ce a?i venit la mine acum, când sunte?i la necaz?"
\par 8 Zis-au batrânii Galaadului catre Ieftae: "De aceea am venit acum la tine, ca sa mergi cu noi, sa te lup?i cu Amoni?ii ?i sa ne fii capetenie noua, tuturor locuitorilor Galaadului".
\par 9 Iar Ieftae a zis catre batrânii Galaadului: "Daca ma lua?i înapoi, ca sa ma lupt cu Amoni?ii, ?i daca Domnul îmi va da mie izbânda, voi mai ramâne eu, oare, capetenie la voi?"
\par 10 Atunci au raspuns batrânii Galaadului catre Ieftae: "Domnul sa fie martor între noi ca vom face cum vei zice tu!"
\par 11 ?i s-a dus Ieftae cu batrânii Galaadului ?i poporul l-a pus capetenie ?i pova?uitor al sau. ?i a rostit Ieftae toate cuvintele sale înaintea fe?ei Domnului în Mi?pa.
\par 12 Apoi a trimis Ieftae soli la regele Amoni?ilor sa-i spuna: "Ce ai cu mine de ai venit la mine sa te razboie?ti pe pamântul meu?"
\par 13 Iar regele Amoni?ilor a raspuns solilor lui Ieftae: "Israel, când venea din Egipt, a luat pamântul meu de la Arnon pâna la Iaboc ?i Iordan. Întoarce-mi-l dara cu pace ?i ma voi retrage".
\par 14 ?i daca s-au întors solii la Ieftae, Ieftae a trimis a doua oara soli la regele Amoni?ilor,
\par 15 Ca sa-i spuna: "A?a zice Ieftae: Israel n-a luat pamântul Moabi?ilor, nici pamântul Amoni?ilor;
\par 16 Caci, când a venit din Egipt, Israel s-a dus în pustiu catre Marea Ro?ie ?i apoi a venit la Cade?.
\par 17 De acolo a trimis Israel la regele Edomului soli sa-i spuna: "Lasa-ma sa trec prin ?ara ta". Dar regele Edomului n-a voit sa auda. ?i a trimis el ?i la regele Moabului, dar nici acela n-a îngaduit. De aceea Israel a ramas la Cade?.
\par 18 Apoi a plecat în pustiu ?i a ocolit pamântul Edomului ?i pamântul Moabului, ajungând la rasaritul lui. Atunci au tabarât dincolo de Arnon, dar n-au intrat în hotarele Moabului, caci Arnonul este hotarul Moabului.
\par 19 De acolo a trimis Israel soli la Sihon, regele Amoreilor, regele He?bonului ?i a zis Israel catre el: "Da-ne voie sa trecem prin ?ara ta la locul nostru!"
\par 20 Dar Sihon nu s-a învoit sa dea drumul lui Israel prin hotarele sale ?i a adunat Sihon tot poporul sau ?i a tabarât în Iah?a ?i s-a batut cu Israel.
\par 21 ?i a dat Domnul Dumnezeul lui Israel pe Sihon ?i tot poporul lui în mâinile lui Israel ?i acesta i-a ucis. Apoi a luat Israel de mo?tenire toata ?ara Amoreilor, care locuiau în ?ara aceea.
\par 22 ?i atunci au primit ei de mo?tenire toate hotarele Amoreilor de la Arnon pâna la Iaboc ?i de la pustie pâna la Iordan.
\par 23 ?i a?a Domnul Dumnezeul lui Israel a izgonit pe Amorei de la fa?a poporului Sau Israel ?i tu voie?ti acum sa-i iei mo?tenirea lui?
\par 24 Nu stapâne?ti tu oare ceea ce ti-a dat ?ie Chemo?, dumnezeul tau? ?i noi stapânim de asemenea ceea ce ne-a dat de mo?tenire Domnul Dumnezeul nostru.
\par 25 Oare tu e?ti mai bun decât Balac, fiul lui Sefor, regele Moabi?ilor? S-a certat cu el Israel, sau s-a luptat cu el?
\par 26 Israel traie?te acum de mai bine de trei sute de ani în He?bon ?i în ceta?ile care ?in de el ?i în Aroer ?i în toate împrejurimile lui ?i în toate ceta?ile din apropierea Arnonului; de ce nu le-a?i luat voi în vremea aceea?
\par 27 Eu însa nu sunt vinovat fa?a de tine; dar tu-mi faci un rau, venind asupra mea cu razboi. Domnul sa fie judecator între fiii lui Israel ?i Amoni?i!"
\par 28 Dar regele Amoni?ilor n-a ?inut seama de cuvintele lui Ieftae, cu care îi trimisese acesta pe soli la el.
\par 29 Atunci a fost peste Ieftae Duhul Domnului ?i a strabatut Ieftae pamântul Galaadului ?i al lui Manase, apoi a ajuns pâna la Mi?pa Galaadului ?i de la Mi?pa Galaadului a plecat asupra Amoni?ilor.
\par 30 În acel timp a facut Ieftae fagaduin?a Domnului ?i a zis: "De vei da pe Amoni?i în mâinile mele,
\par 31 Când ma voi întoarce biruitor de la Amoni?i, oricine va ie?i din por?ile casei mele în întâmpinarea mea va fi afierosit Domnului ?i-l voi aduce ardere de tot".
\par 32 Apoi a venit Ieftae la Amoni?i sa se bata cu ei ?i i-a dat Domnul în mâinile lui.
\par 33 ?i i-a batut cumplit de la Aroer pâna spre Minit în douazeci de ceta?i ?i pâna la Abel-Cheramim ?i au fost umili?i Amoni?ii în fa?a fiilor lui Israel.
\par 34 Dupa aceea a venit Ieftae în la casa sa ?i iata fiica sa i-a ie?it în întâmpinare cu timpane ?i jocuri; aceasta era singurul lui copil, caci el nu mai avea nici baie?i, nici fete.
\par 35 ?i când a vazut-o el, ?i-a sfâ?iat haina ?i a zis: "Ah, fiica mea! Tu m-ai rapus ?i e?ti dintre cei ce-mi tulbura biruin?a. Eu mi-am deschis gura pentru tine înaintea Domnului ?i nu ma voi putea lepada!"
\par 36 Iar ea a zis catre el: "Tatal meu, daca tu ?i-ai deschis gura pentru mine înaintea Domnului, fa cu mine ceea ce a rostit gura ta, de vreme ce Domnul a savâr?it prin tine razbunarea împotriva Amoni?ilor, vrajma?ii tai!"
\par 37 Apoi a zis iara?i catre tatal sau: "Iarta numai ce sa-mi faci: Lasa-ma doua luni, sa ma duc sa ma sui pe munte ?i sa-mi plâng fecioria cu prietenele mele!"
\par 38 Atunci el a zis: "Du-te!" ?i a lasat-o doua luni. ?i s-a dus cu prietenele sale ?i ?i-a plâns fecioria în mun?i.
\par 39 Apoi dupa trecerea celor doua luni ea s-a întors la tatal sau ?i acesta a facut cu ea cum fagaduise; ?i ea n-a cunoscut barbat.
\par 40 ?i s-a facut obicei în Israel, ca în fiecare an fiicele lui Israel sa mearga sa plânga pe fata lui Ieftae Galaaditeanul patru zile pe an.

\chapter{12}

\par 1 Dupa aceea s-au adunat Efraimi?ii ?i au purces spre ?afon ?i au zis catre Ieftae: "Pentru ce te-ai dus sa te ba?i cu Amoni?ii, iar pe noi nu ne-ai chemat cu tine? Vom arde dar casa ta cu foc, împreuna cu tine".
\par 2 Iar Ieftae a zis: "Eu ?i poporul meu am avut cu Amoni?ii cearta mare ?i eu v-am chemat, dar voi nu m-a?i scapat din mâinile lor.
\par 3 Vazând însa ca nu este nici un izbavitor, mi-am pus via?a în primejdie ?i m-am dus împotriva Amoni?ilor ?i Domnul i-a dat în mâinile mele. De ce dar a?i venit sa va bate?i cu mine?"
\par 4 Atunci a adunat Ieftae to?i oamenii din Galaad ?i s-a batut cu Efraimi?ii ?i au batut locuitorii Galaadului pe Efraimi?i, zicând: "Voi sunte?i ni?te fugari din Efraim, Galaadul însa e între Efraim ?i Manase".
\par 5 ?i au luat Galaaditenii vadul Iordanului de la Efraimi?i ?i când vreunul din Efraimi?i zicea: "Îngaduie-mi sa trec", atunci oamenii din Galaad îi raspundeau: "Nu cumva e?ti Efraimit?" Acela raspundea: "Nu!"
\par 6 Ei însa îi ziceau: "Zi: ?ibbolet"; el însa zicea: "Sibbolet", ca nu putea zice altfel. Atunci ei îl luau ?i-l junghiau acolo la vadul Iordanului. ?i au cazut în vremea aceea din Efraimi?i patruzeci ?i doua de mii.
\par 7 ?i a fost Ieftae judecator în Israel ?ase ani; apoi a murit Ieftae Galaaditeanul ?i a fost îngropat în unul din ora?ele Galaadului.
\par 8 Dupa el a fost judecator în Israel Ib?an din Betleem.
\par 9 Acesta a avut treizeci de feciori ?i treizeci de fete a dat el din casa sa în casatorie, iar treizeci de fete a luat de afara pentru fiii sai ?i a fost judecator în Israel ?apte ani.
\par 10 Apoi a murit Ib?an ?i a fost îngropat în Betleem.
\par 11 Dupa dânsul a fost judecator în Israel Elon Zabuloneanul ?i a judecat pe Israel zece ani.
\par 12 Apoi a murit Elon Zabuloneanul ?i a fost înmormântat la Aialon, în pamântul lui Zabulon.
\par 13 Dupa el a fost judecator în Israel Abdon, fiul lui Hilel Piratoneanul.
\par 14 Acesta a avut patruzeci de fii ?i treizeci de nepo?i care calareau pe ?aptezeci de mânji de asin ?i a judecat pe Israel opt ani.
\par 15 Apoi a murit Abdon, fiul lui Hilel Piratoneanul ?i a fost îngropat în Piraton, în pamântul lui Efraim, pe muntele lui Amalec.

\chapter{13}

\par 1 ?i fiii lui Israel au facut iara?i rele înaintea ochilor Domnului ?i i-a dat Domnul în mâinile Filistenilor pentru patruzeci de ani.
\par 2 Era însa în vremea aceea un om de la ?ara, din semin?ia lui Dan, cu numele Manoe ?i femeia lui era stearpa ?i nu na?tea.
\par 3 Udata însa s-a aratat îngerul Domnului femeii ?i i-a zis: "Iata tu e?ti stearpa ?i nu na?ti; dar vei zamisli ?i vei na?te fiu.
\par 4 Paze?te-te dar, sa nu bei vin, nici sichera ?i nimic necurat sa nu manânci;
\par 5 Ca iata ai sa zamisle?ti ?i al sa na?ti un fiu; ?i nu se va atinge briciul de capul lui, pentru ca pruncul acesta va fi chiar din pântecele mamei sale nazireu al lui Dumnezeu ?i va începe sa izbaveasca pe Israel din mâna Filistenilor".
\par 6 ?i a venit femeia ?i a spus barbatului sau, zicând: "A venit la mine un om al lui Dumnezeu, a carui înfa?i?are era ca înfa?i?area unui înger al lui Dumnezeu, foarte luminos; nici eu nu l-am întrebat de unde este ?i nici el nu mi-a spus numele sau;
\par 7 Dar mi-a zis: Iata ai sa zamisle?ti ?i ai sa na?ti un fiu; a?adar sa nu bei vin ?i sichera ?i sa nu manânci nimic necurat, caci copilul chiar din pântecele mamei ?i pâna la moarte va fi nazireu al lui Dumnezeu".
\par 8 Atunci Manoe s-a rugat Domnului ?i a zis: "Doamne, fa sa vina iara?i pe la noi omul lui Dumnezeu pe care l-ai trimis Tu, ?i sa ne înve?e ce sa facem cu copilul care se va na?te!"
\par 9 ?i a ascultat Dumnezeu glasul lui Manoe ?i a venit îngerul iara?i la femeie, când era la câmp, însa Manoe, barbatul ei, nu era cu dânsa.
\par 10 Dar femeia a alergat îndata ?i a vestit pe barbatul sau, zicându-i: "Iata mi s-a aratat omul cel ce a venit atunci la mine".
\par 11 ?i s-a sculat Manoe ?i s-a dus cu femeia sa ?i a venit la omul acela ?i a zis catre el: "Tu, oare, e?ti omul acela care ai vorbit cu femeia?" Iar îngerul i-a raspuns: "Eu!"
\par 12 ?i a zis Manoe: "A?adar, daca se va împlini cuvântul tau, cum sa ne purtam cu copilul acesta ?i ce sa facem cu el?"
\par 13 Iar îngerul a zis: "Sa se pazeasca el de toate cele ce am spus eu femeii;
\par 14 Sa nu manânce nimic din câte rode?te vi?a de vie; sa nu bea vin, nici sichera ?i sa nu manânce nimic necurat ?i sa pazeasca toate câte i-am poruncit ei".
\par 15 Atunci Manoe a zis: "Îngaduie-ne sa te oprim pâna vom gati un ied".
\par 16 Iar îngerul a zis catre Manoe: "De?i ma vei opri, eu nu voi mânca pâinea ta; dar de voie?ti sa faci ardere de tot Domnului, atunci adu-o". ?i nu ?tia Manoe ca acesta e îngerul Domnului.
\par 17 ?i a zis Manoe catre îngerul Domnului: "Cum î?i este numele? Ca sa te marim, când se va împlini cuvântul tau".
\par 18 Zis-a îngerul catre el: "La ce ma întrebi tu de numele meu? Ca el este minunata".
\par 19 Atunci a luat Manoe un ied ?i prinos de pâine ?i le-a adus Domnului pe o stânca. ?i a facut acela minunea pe care au vazut-o Manoe ?i femeia sa.
\par 20 Când a început a se înal?a flacara de la jertfelnic spre cer, îngerul Domnului s-a ridicat cu flacara de pe jertfelnic. Vazând aceasta, Manoe ?i femeia lui au cazut cu fa?a la pamânt.
\par 21 ?i s-a facut nevazut îngerul Domnului de Manoe ?i de femeia lui. Atunci Manoe a în?eles ca acela fusese îngerul Domnului.
\par 22 ?i a zis Manoe catre femeia sa: "De buna seama avem sa murim, caci am vazut pe Dumnezeu!"
\par 23 Iar femeia lui i-a zis: "Daca Domnul ar voi sa ne omoare, n-ar fi primit din mâinile noastre arderea de tot ?i prinosul de pâine ?i nu ne-ar fi aratat toate acelea ?i nu ne-ar fi descoperit acum aceasta".
\par 24 ?i a nascut femeia un fiu ?i i-au pus numele Samson. ?i a crescut copilul ?i l-a binecuvântat Domnul.
\par 25 ?i a început Duhul Domnului sa lucreze prin el în tabara lui Dan, între ?ora ?i E?taol.

\chapter{14}

\par 1 În vremea aceea s-a dus Samson la Timna ?i a vazut în Timna o femeie din fiicele Filistenilor ?i aceasta i-a placut.
\par 2 ?i s-a dus ?i a spus el tatalui sau ?i mamei sale ?i a zis: "Am vazut în Timna o femeie din fiicele Filistenilor; lua?i-mi-o mie de so?ie!"
\par 3 Iar tatal sau ?i mama sa i-au raspuns: "Au doara nu se gasesc femei printre fiicele fra?ilor tai ?i în tot poporul meu, de te duci sa-?i iei so?ie de la Filistenii cei netaia?i împrejur?" A zis Samson catre tatal sau: "Ia-mi-o pe aceea, pentru ca mi-a placut!"
\par 4 ?i nu ?tiau tatal sau ?i mama sa ca aceasta este de la Domnul ?i ca el cauta prilej sa se razbune pe Filisteni. Caci în vremea aceea Filistenii domneau peste Israel.
\par 5 Deci s-a dus Samson cu tatal sau ?i cu mama sa la Timna; iar când s-au apropiat de viile Timnei, iata un leu tânar venea racnind înaintea lor.
\par 6 Atunci s-a coborât peste el Duhul Domnului ?i el a sfâ?iat leul ca pe un ied; ?i nu avea nimic în mâna. ?i n-a spus tatalui sau ?i mamei sale ce facuse.
\par 7 ?i a venit ?i a vorbit cu femeia ?i aceasta a placut lui Samson.
\par 8 Iar dupa câteva zile s-a dus el iara?i ca sa o ia ?i s-a abatut sa vada trupul leului ?i iata un roi de albine ?i miere în trupul leului.
\par 9 ?i a luat el fagurele în mâna ?i s-a dus ?i a mâncat pe cale; iar daca a venit la tatal sau ?i la mama sa, le-a dat ?i lor de au mâncat; dar nu le-a spus ca a luat fagurele acesta din trupul leului celui mort.
\par 10 Apoi a mers tatal sau la femeie ?i a facut acolo Samson ospa? de ?apte zile, cum au obiceiul sa faca mirii.
\par 11 ?i când l-au vazut cei de acolo, au ales treizeci de nunta?i care sa fie împrejurul lui.
\par 12 Iar Samson a zis catre ei: "Am sa va spun o ghicitoare ?i daca mi-o ve?i ghici în cele ?apte zile ale ospa?ului ?i mi-o ve?i dezlega, va voi da treizeci de cama?i ?i treizeci de rânduri de haine.
\par 13 Iar daca nu ve?i putea s-o ghici?i, atunci sa-mi da?i voi mie treizeci de cama?i ?i treizeci de rânduri de haine". Au zis aceia: "Spune ghicitoarea ta, ca s-o auzim".
\par 14 Atunci le-a zis: "Din cel ce manânca a ie?it mâncare, ?i din cel tare a ie?it dulcea?a". ?i n-au putut sa dezlege ghicitoarea în trei zile.
\par 15 Iar în ziua a ?aptea au zis aceia catre femeia lui Samson: "Ademene?te pe barbatul tau sa dezlege ghicitoarea; altfel te vom arde cu foc pe tine ?i casa tatalui tau; ne-a?i chemat, oare, ca sa ne jefui?i?"
\par 16 ?i a plâns femeia lui Samson înaintea lui, zicând: "Tu ma ura?ti ?i nu ma iube?ti; ai dat o ghicitoare fiilor poporului meu, iar mie nu mi-o dezlegi". ?i a zis el catre ea: "Eu n-am dezlegat-o tatalui meu ?i mamei mele ?i sa ?i-o dezleg ?ie?"
\par 17 ?i a plâns ea înaintea lui ?apte zile, cât a ?inut ospa?ul la ei. În sfâr?it în ziua a ?aptea i-a dezlegat-o caci ea îl ruga staruitor.
\par 18 Iar ea a spus dezlegarea ghicitorii fiilor poporului sau. ?i iata în ziua a ?aptea, înainte de rasaritul soarelui, au zis catre oamenii ceta?ii: "Ce e mai dulce ca mierea ?i ce e mai tare ca leul?" ?i el le-a zis: "De nu a?i fi arat cu juninca mea, ghicitoarea mea n-o mai ghicea?i voi".
\par 19 Atunci s-a coborât peste el Duhul Domnului ?i s-a dus în Ascalon ?i, ucigând acolo treizeci de oameni, a dezbracat de pe ei hainele ?i a dat rândurile de haine celor ce au ghicit ghicitoarea sa. ?i s-a aprins mânia lui ?i s-a dus la casa tatalui sau.
\par 20 Iar femeia lui Samson s-a maritat cu unul din nunta?ii de la nunta sa, care au fost împrejurul lui.

\chapter{15}

\par 1 Peste câteva zile, în timpul seceratului grâului, a venit Samson sa se vada cu femeia sa, aducând cu sine un ied. Iar când a zis: "Ma duc la femeia mea în odaia de dormit", tatal ei nu l-a lasat sa intre.
\par 2 ?i a zis tatal ei: "Eu am socotit ca ai urât-o ?i am maritat-o cu un prieten al tau; iata sora ei mai mica e mai frumoasa decât ea; sa fie aceasta în locul aceleia".
\par 3 Samson însa le-a zis: "De acum eu voi fi drept înaintea Filistenilor, daca ma voi apuca sa le fac rau".
\par 4 Apoi Samson s-a dus ?i a prins trei sute de vulpi, a luat faclii, a legat câte doua vulpi de coada ?i intre ele câte o faclie;
\par 5 Dupa aceea a aprins facliile ?i a dat drumul vulpilor prin grânele Filistenilor ?i a aprins ?i claile ?i grâul nesecerat, viile ?i livezile de maslini.
\par 6 ?i ziceau Filistenii: "Cine oare a facut aceasta?" ?i li s-a spus: "Samson, ginerele Timneanului, caci acesta i-a luat femeia ?i a dat-o dupa un prieten al lui". Atunci Filistenii s-au dus ?i au ars-o cu foc ?i pe ea ?i casa tatalui ei.
\par 7 Dar Samson le-a zis: "Cu toate ca a?i facut aceasta, eu tot am sa ma razbun pe voi ?i numai atunci am sa ma lini?tesc".
\par 8 ?i le-a sfarâmat fluierele picioarelor ?i ?oldurile ?i apoi s-a dus ?i a ?ezut în pe?tera de la stânca Etam.
\par 9 Filistenii însa s-au dus ?i ?i-au a?ezat tabara în Iuda ?i s-au întins pâna la Lehi.
\par 10 Iar locuitorii lui Iuda au zis: "Pentru ce a?i ie?it voi asupra noastra?" ?i ei au zis: "Am venit sa legam pe Samson, ca sa facem cu el cum a facut ?i el cu noi".
\par 11 Atunci s-au dus trei mii de oameni din Iuda la pe?tera de la stânca Etam ?i au zis catre Samson: "Nu ?tii tu, oare, ca Filistenii domnesc peste noi? De ce ne-ai facut tu una ca asta?" El însa a zis: "Cum s-au purtat ei cu mine, a?a m-am purtat ?i eu cu ei".
\par 12 I-au zis lui: "Noi am venit sa te legam, ca sa te dam în mâinile Filistenilor". Atunci Samson le-a zis: "Jura?i-va mie ca nu ma ve?i ucide!"
\par 13 ?i ei au raspuns: "Nu, noi numai te vom lega ?i te vom da în mâinile lor, dar de omorât nu te vom omorî". ?i l-au legat cu doua funii noi ?i l-au dus din pe?tera.
\par 14 Dar când s-a apropiat el de Lehi, Filistenii l-au întâmpinat cu strigate. Atunci s-a coborât peste el Duhul Domnului ?i funiile care erau peste mâinile lui s-au facut ca ni?te câl?i ar?i de foc ?i au cazut legaturile de pe mâinile lui.
\par 15 Iar el gasind o falca sanatoasa de asin, ?i-a întins mâna, a luat-o ?i a ucis cu ea o mie de oameni.
\par 16 Apoi a zis Samson: "Cu o falca de magar o ceata, doua cete am stins, Cu o falca de magar o mie de oameni am ucis".
\par 17 ?i zicând acestea, a aruncat falca din mâini ?i a numit locul acela Ramat-Lehi.
\par 18 Sim?ind însa sete mare, a strigat catre Domnul ?i a zis: "Tu ai facut prin mâna robului Tau aceasta mare izbavire; iar acum eu mor de sete ?i voi cadea în mâinile celor netaia?i împrejur".
\par 19 Atunci a deschis Domnul o crapatura într-o stânca din Lehi ?i a curs din ea apa. ?i a baut Samson ?i ?i-a astâmparat setea ?i duhul lui s-a înviorat. De aceea s-a ?i numit locul acela: "Izvorul celui ce striga", care este în Lehi pâna în ziua de astazi.
\par 20 ?i a fost el judecator în Israel pe vremea Filistenilor douazeci de ani.

\chapter{16}

\par 1 Venind însa odata Samson la Gaza, a vazut acolo o femeie desfrânata ?i a intrat la ea.
\par 2 ?i li s-a spus oamenilor din Gaza: "Samson a venit aici". Atunci ace?tia l-au înconjurat ?i l-au pândit toata noaptea la por?ile ceta?ii ?i s-au ascuns toata noaptea, zicând: "Sa a?teptam pâna se va lumina de ziua ?i sa-l ucidem!"
\par 3 Samson însa a dormit pâna la miezul nop?ii; iar la miezul nop?ii a luat por?ile ceta?ii din amândoi u?orii ?i, ridicându-le împreuna cu zavoarele, le-a pus pe umerii sai ?i le-a dus pe vârful muntelui care este pe drumul spre Hebron ?i le-a lasat acolo.
\par 4 Dupa acestea a iubit el o femeie care traia în valea Sorec ?i pe care o chema Dalila.
\par 5 La aceasta au venit frunta?ii Filistenilor ?i i-au zis: "Amage?te-l ?i afla în ce sta puterea lui cea mare ?i cum l-am putea prinde, ca sa-l legam ?i sa-l supunem; ?i-li vom da pentru aceasta o mie ?i o suta de sicli de argint".
\par 6 ?i a zis Dalila catre Samson: "Spune-mi ?i mie în ce sta puterea ta cea mare ?i cu ce sa te lege ca sa te supuna?"
\par 7 I-a raspuns Samson: "De ma vor lega cu ?apte vine crude ?i înca neuscate, voi ajunge slab ?i voi fi ca ?i ceilal?i oameni".
\par 8 Atunci i-au adus ei frunta?ii Filistenilor ?apte vine crude ?i înca neuscate ?i ea l-a legat cu ele.
\par 9 ?i unii stateau la pânda la ea în odaia de dormit; ?i ea a zis catre Samson: "Samsoane, Filistenii vin asupra ta!" Atunci el a rupt vinele, cum ar fi rupt o a?a de câl?i ar?i de foc. ?i astfel nu s-a aflat de unde vine puterea lui.
\par 10 Dalila a zis însa catre Samson: "Iata tu m-ai amagit ?i mi-ai spus minciuni. Spune-mi dar cu ce sa te lege?
\par 11 Iar el i-a zis: "De ma vor lega cu funii noi, care sa nu mai fi fost întrebuin?ate, atunci eu voi slabi ?i voi fi ca ?i ceilal?i oameni".
\par 12 ?i a luat Dalila funii noi ?i l-a legat, iar cineva pândea. Apoi ea i-a zis: "Samsoane, Filistenii vin asupra ta!" ?i el le-a rupt de pe mâinile sale, ca pe ni?te a?e.
\par 13 Atunci Dalila a zis catre Samson: "Tu ma amage?ti mereu ?i-mi spui minciuni. Spune-mi drept, cu ce sa te lege?" Iar el i-a zis: "De vei împleti ?apte ?uvi?e do par din capul meu ?i le vei prinde cu un cui de sulul de la razboiul de ?esut, atunci eu vai slabi ?i voi fi ca ?i ceilal?i oameni".
\par 14 ?i l-a adormit Dalila pe bra?ele sale; iar daca a adormit el, Dalila a luat ?apte ?uvi?e din capul lui ?i le-a pironit de sulul de la razboi ?i apoi a strigat: "Samsoane, Filistenii vin asupra ta!" Atunci el s-a de?teptat din somn ?i a smucit razboiul împreuna cu ?esatura ?i nu s-a aflat de unde vine puterea lui.
\par 15 Dalila însa i-a zis: "Cum de po?i tu spune: "Te iubesc", când inima ta nu este cu mine? Iata, de trei ori m-ai amagit ?i nu mi-ai spus în ce sta puterea ta cea mare".
\par 16 ?i fiindca ea tot staruia ?i-l necajea cu vorbele sale în fiecare zi, s-a tulburat sufletul lui pâna la moarte.
\par 17 ?i atunci i-a descoperit el toata inima sa ?i i-a zis: "Briciul nu s-a atins de capul meu, caci eu sunt nazireu al lui Dumnezeu din pântecele maicii mele; de m-ar tunde cineva, atunci s-ar departa de la mine puterea mea ?i eu a? slabi ?i a? fi ca ceilal?i oameni".
\par 18 Vazând Dalila ca el i-a descoperit toata inima sa, a trimis de au chemat pe frunta?ii Filistenilor, zicându-le: "Veni?i acum, ca el mi-a descoperit toata inima sa!" ?i au venit la ea frunta?ii Filistenilor ?i au adus argintul cu ei.
\par 19 Apoi Dalila l-a adormit pe genunchii sai ?i a chemat un om ?i i-a poruncit sa tunda cele ?apte ?uvi?e ale capului lui. Atunci el a început a slabi ?i s-a departat de el puterea lui;
\par 20 Iar ea a zis: "Samsoane; Filistenii vin asupra ta!" ?i de?teptându-se el din somnul sau, a zis: "Voi face ca mai înainte ?i ma voi scapa de ei". Dar nu ?tia ca Domnul Se departase de el.
\par 21 Atunci l-au luat frunta?ii Filistenilor ?i i-au scos ochii ?i l-au dus la Gaza ?i l-au legat cu doua lan?uri de arama ?i râ?nea în temni?a.
\par 22 ?i a început sa-i creasca parul pe capul lui, pe unde fusese tuns.
\par 23 Atunci s-au adunat frunta?ii Filistenilor, ca sa aduca jertfa marelui Dagon, dumnezeul lor, ?i sa se veseleasca, zicând: "Dumnezeul nostru a dat pe Samson, vrajma?ul nostru, în mâinile noastre".
\par 24 De asemenea ?i mul?imea, vazându-l, slavea pe dumnezeul sau, zicând: "Dumnezeul nostru a dat în mâinile noastre pe vrajma?ul nostru ?i pe pustiitorul ?arii noastre, care a ucis pe mul?i dintre noi".
\par 25 Iar dupa ce s-a veselit inima lor, au zis: "Aduce?i pe Samson din închisoare, ca sa mai râdem de el". ?i au adus pe Samson din temni?a ?i râdeau de el ?i-l trageau de urechi ?i l-au pus între doi stâlpi.
\par 26 Atunci a zis Samson tânarului care-l ducea de mâna: "Du-ma ca sa pipai stâlpii pe care este întemeiata casa ?i sa ma reazem de ei". ?i tânarul a facut a?a.
\par 27 Casa însa era plina de barba?i ?i de femei, caci erau acolo to?i frunta?ii Filistenilor, iar pe acoperi? se aflau ca la trei mii de oameni, barba?i ?i femei, care se uitau ?i râdeau de Samson.
\par 28 Atunci a strigat Samson catre Domnul ?i a zis: "Doamne Dumnezeule, adu-?i aminte de mine ?i întare?te-ma înca o data, o, Dumnezeule, ca printr-o singura lovitura sa ma razbun pe Filisteni pentru cei doi ochi ai mei !"
\par 29 ?i a mi?cat Samson din loc doi stâlpi din mijloc pe care era sprijinita casa, rezemându-se de ei, de unul cu mâna dreapta ?i de celalalt cu stânga.
\par 30 ?i a zis Samson: "Mori, suflete al meu, cu Filistenii!" Apoi s-a sprijinit cu toata puterea ?i s-a prabu?it casa peste frunta?ii Filistenilor ?i peste tot poporul ce era în ea. ?i cei pe care i-a ucis Samson la moartea sa au fost mai mul?i decât to?i cei pe care-i ucisese în via?a sa.
\par 31 Atunci au venit fra?ii lui ?i toata casa tatalui sau ?i l-au luat ?i l-au dus de l-au îngropat între ?ora ?i E?taol, în mormântul lui Manoe, tata sau. ?i a fost el judecator în Israel douazeci de ani. Iar dupa Samson s-a sculat Emegar, fiul lui Enan, ?i a ucis din Filisteni ?ase sute de oameni, afara de vite. Acesta a izbavit pe Israel.

\chapter{17}

\par 1 În vremea aceea era cineva în Muntele Efraim, cu numele Mica.
\par 2 Acesta a zis catre mama sa: "Cei o mie ?i o suta de sicli de argint care ?i s-au luat ?i pentru care tu ai rostit blestem în fa?a mea, acel argint este la mine, eu l-am luat". A zis mama sa: "Binecuvântat fie fiul meu de Domnul!"
\par 3 ?i a întors acela cei o mie ?i o suta sicli de argint mamei sale. Iar mama lui a zis: "Acest argint eu l-am afierosit de la mine Domnului pentru tine, fiul meu, ca sa fac din el un idol, un chip turnat. A?adar ti-l dau ?ie".
\par 4 El însa a întors argintul mamei sale. Iar mama sa a luat doua sute sicli de argint ?i i-a dat unui turnator ?i acela a facut din ei un idol, un chip turnat, care se ?i afla în casa lui Mica.
\par 5 ?i era la Mica loca?ul lui dumnezeu ?i a facut un efod ?i un terafim ?i a pus el pe unul din fiii sai sa fie preotul lui.
\par 6 În zilele acelea nu era rege în Israel, ci fiecare facea ce i se parea ca este drept.
\par 7 ?i traia pe atunci la Betleemul cel din semin?ia lui Iuda un tânar levit.
\par 8 ?i s-a dus omul acesta din cetatea Betleemului lui Iuda, ca sa traiasca unde se va nimeri, ?i, mergând el pe cale, a ajuns pe Muntele Efraim la casa lui Mica.
\par 9 Mica însa i-a zis: "De unde vii tu?" Iar el a raspuns: "Eu sunt levit din Betleemul lui Iuda ?i ma duc sa traiesc unde voi nimeri".
\par 10 Atunci Mica i-a zis: "Ramâi la mine ?i fii parinte aici la mine ?i preot; eu î?i voi da câte zece sicli de argint pa an ?i hainele ?i hrana trebuitoare".
\par 11 ?i a venit levitul la el ?i s-a învoit levitul sa ramâna la omul acesta; ?i era tânar, ca unul din fiii lui.
\par 12 ?i Mica a sfin?it pe levit ?i tânarul acesta a fost preot la el ?i a trait în casa lui Mica.
\par 13 Apoi a zis Mica: "Acum eu ?tiu ca Domnul îmi va face bine, pentru ca am preot pe un levit".

\chapter{18}

\par 1 În zilele acelea nu era rege în Israel; ?i în timpul acela semin?ia lui Dan î?i cauta mo?ie unde sa se a?eze, caci pâna atunci nu-i cazuse înca parte deplina printre semin?iile lui Israel.
\par 2 ?i au trimis fiii lui Dan din neamul lor cinci oameni, barba?i puternici, din ?ora ?i din E?taol, ca sa cerceteze ?ara ?i s-o cunoasca ?i li s-a zis: "Duce?i-va ?i cunoa?te?i ?ara aceea!" ?i s-au dus aceia în Muntele Efraim, la casa lui Mica ?i au ramas acolo.
\par 3 Pe când se aflau ei la casa lui Mica, au cunoscut glasul tânarului levit ?i intrând la el, l-au întrebat: "Cine te-a adus aici? Ce faci ?i pentru ce stai aici?"
\par 4 Iar el le-a raspuns: "Cutare ?i cutare a facut pentru mine Mica ?i mi-a dat simbrie ?i iata eu îi sunt preot".
\par 5 Aceia însa i-au zis: "Întreaba pe Dumnezeu, ca sa ?tim, de vom izbuti pe calea în care am plecat".
\par 6 Iar preotul le-a zis: "Merge?i cu pace, calea voastra în care merge?i este înaintea Domnului".
\par 7 ?i s-au dus cei cinci barba?i ?i au ajuns la Lai? ?i au vazut ca poporul din el traie?te în pace, dupa obiceiul Sidonienilor, lini?tit ?i fara grija, ?i ca nu era în ?ara aceea cine sa obijduiasca cu ceva sau sa aiba stapânire: de Sidonieni ei traiau departe ?i cu nimeni nu aveau ei nici o treaba.
\par 8 Atunci s-au întors cei cinci oameni la fra?ii lor în ?ora ?i E?taol ?i au zis fra?ii lor catre ei: "Ce ne spune?i?"
\par 9 Iar ei au raspuns: "Sa ne sculam ?i sa mergem asupra lor. Am vazut ?ara ?i este foarte buna. Voi însa sa nu sta?i pe gânduri ?i sa nu zabovi?i a merge ?i a lua în stapânire ?ara aceea.
\par 10 Când ve?i merge, ve?i da de un popor fara grija ?i de o ?ara întinsa; Dumnezeu o va da în mâinile voastre; acolo este un loc, unde nu lipse?te nimic din tot ce da pamântul".
\par 11 ?i au plecat într-acolo din ?ara ?i din E?taol din semin?ia lui Dan ?ase sute de oameni, încin?i cu arme de razboi.
\par 12 Ace?tia s-au dus ?i au tabarât în Chiriat-Iearim în Iuda. De aceea se ?i nume?te locul acela tabara lui Dan pâna în ziua de astazi ?i e în dosul lui Chiriat-Iearim.
\par 13 De acolo s-au îndreptat spre Muntele Efraim ?i au venit la casa lui Mica,
\par 14 Atunci au zis cei cinci barba?i, care fusesera sa iscodeasca ?ara Lai?, catre fra?ii lor: "?ti?i voi oare ca în una din casele acestea este un efod, un terafim, un idol ?i un chip turnat? A?adar, gândi?i-va ce trebuie sa face?i".
\par 15 Apoi s-au abatut într-acolo ?i au intrat la casa levitului celui tânar, în casa lui Mica ?i i-au dat buna ziua.
\par 16 Cei ?ase sute de oameni din fiii lui Dan, încin?i cu arme de razboi, s-au oprit la poarta.
\par 17 Iar cei cinci oameni, care fusesera de iscodisera ?ara, s-au dus ?i au intrat acolo, au luat idolul ?i efodul ?i terafimul ?i chipul cel turnat. Preotul însa statea la poarta cu cei ?ase sute de oameni încin?i cu arme de razboi.
\par 18 Când au intrat ei în casa lui Mica ?i au luat idolul, terafimul, efodul ?i chipul cel turnat, preotul le-a zis: "Ce face?i voi?"
\par 19 Iar ei au zis: "Taci, pune-?i mâna la gura ?i vino cu noi ?i ne fii parinte ?i preot; este mai bine oare de tine sa fii preot în casa unui singur om decât sa fii preot într-o semin?ie sau într-o familie a lui Israel?"
\par 20 Atunci preotul s-a îmbunat ?i a luat efodul, terafimul, idolul ?i chipul cel turnat ?i s-a dus cu mul?imea.
\par 21 Dupa aceea ei s-au întors ?i au plecat, punând copiii, vitele ?i avutul înainte.
\par 22 Iar dupa ce s-au departat de casa lui Mica, Mica ?i locuitorii caselor vecine cu casa lui Mica s-au strâns ?i au alergat dupa fiii lui Dan,
\par 23 ?i au strigat catre fiii lui Dan, care s-au întors ?i au zis catre Mica: "Ce ai de strigi a?a?"
\par 24 A zis Mica: "Voi mi-a?i luat dumnezeul meu, pe care l-am facut eu, ?i pe preotul meu ?i v-a?i dus. Ce-mi mai ramâne? Cum dar zice?i: Ce ai?"
\par 25 Iar fiii lui Dan i-au zis: "Taci, sa nu-?i mai auzim gura! Altfel, suparându-se, unii din noi vor tabarî pe tine, ?i vei pieri ?i tu ?i familia ta".
\par 26 ?i s-au dus fiii lui Dan în drumul lor; iar Mica, vazând ca aceia sunt mai tari decât el, s-a întors ?i s-a dus la casa sa.
\par 27 Iar fiii lui Dan au luat ceea ce facuse Mica ?i pe preotul care era la el ?i s-au dus la Lai? asupra unui popor lini?tit ?i fara grija ?i l-au ucis cu sabia ?i cetatea au ars-o cu foc.
\par 28 ?i n-a avut cine sa-l ajute, caci era departe de Sidon ?i nu avea legaturi cu nimeni. Cetatea aceasta se afla în valea cea din apropiere de Bet-Rehob. ?i au cladit din nou cetatea ?i s-au a?ezat în ea.
\par 29 Apoi au pus numele ceta?ii Dan, dupa numele strabunului lor Dan, fiul lui Israel; mai înainte însa numele ceta?ii era Lai?.
\par 30 ?i au a?ezat fiii lui Dan idolul la ei. Iar Ionatan, fiul lui Gher?om, ?i fiii sai au fost preo?i în semin?ia lui Dan pâna în ziua robirii ?arii.
\par 31 ?i au avut la ei idolul facut de Mica în tot timpul cât cortul lui Dumnezeu a fost la ?ilo.

\chapter{19}

\par 1 În zilele acelea, când nu era rege în Israel, traia un levit pe coasta Muntelui Efraim. Acesta ?i-a luat o concubina din Betleemul Iudei.
\par 2 ?i s-a certat cu el ?i s-a dus de la el înapoi la casa tatalui ei în Betleemul Iudei ?i a stat acolo patru luni.
\par 3 Atunci barbatul ei s-a sculat ?i s-a dus dupa ea, ca sa se împace cu ea ?i s-o aduca acasa. Cu el era o sluga a sa ?i o pereche de asini. ?i l-a dus ea în casa tatalui sau.
\par 4 Socrul sau, tatal acestei tinere femei, vazându-l, l-a întâmpinat cu bucurie ?i l-a oprit ?i el a ramas la el trei zile. Au mâncat ?i au baut ?i au ramas acolo.
\par 5 A patra zi s-au sculat ei de vreme ?i s-au gatit sa plece. Iar tatal tinerei femei a zis catre ginerele sau: "Întare?te-?i inima cu o buca?ica de pâine ?i apoi vei pleca".
\par 6 ?i au ramas ?i au mâncat ?i au baut amândoi împreuna. Apoi tatal tinerei femei a zis catre omul acela: "Ramâi înca ?i noaptea asta, ca sa se veseleasca inima ta".
\par 7 Omul acela însa s-a sculat sa plece, dar socrul sau l-a rugat ?i el a mai ramas o noapte acolo.
\par 8 A cincea zi s-a sculat el de diminea?a ca sa plece. ?i tatal acelei tinere femei iar i-a zis: "Întare?te-?i inima ta cu pâine ?i zabove?te pâna când va fi soarele spre asfin?it". ?i au mâncat ei amândoi ?i au baut.
\par 9 Apoi s-a sculat omul acela ca sa plece el ?i concubina sa ?i sluga sa. Iar socrul sau, tatal tinerei femei, a zis: "Iata s-a plecat ziua spre seara, ramâi, rogu-te, iata ziua se va sfâr?i curând, ramâi aici; sa se veseleasca inima ta! Mâine va ve?i scula de diminea?a ?i ve?i pleca în calea voastra ?i te vei duce la casa ta".
\par 10 Dar omul nu s-a învoit sa ramâna, ci s-a sculat ?i a plecat ?i a venit pâna la Iebus, care acum este Ierusalimul. Cu el erau doi asini încarca?i ?i concubina lui.
\par 11 Dar când s-au apropiat ei de Iebus, ziua se apropia de seara. Atunci a zis sluga catre stapânul sau: "Sa ne abatem în cetatea aceasta a Iebuseilor ?i sa ramânem în ea".
\par 12 Stapânul lui însa i-a zis: "Nu, sa nu intram în cetatea unor oameni de alt neam, care nu sunt din fiii lui Israel, ci sa mergem pâna la Ghibeea".
\par 13 Apoi a zis iar catre sluga sa: "Sa mergem pâna la unul din aceste locuri ?i sa ramânem în Ghibeea sau Rama".
\par 14 ?i au mers ei ?i au ajuns aproape de Ghibeea ?i, când au ajuns la Ghibeea lui Veniamin, a asfin?it soarele.
\par 15 ?i s-au abatut într-acolo, ca sa mearga sa ramâna în Ghibeea. ?i au venit ?i au ramas în uli?a ceta?ii, caci nimeni nu i-a chemat În casa ca sa-i gazduiasca.
\par 16 Iata însa ca venea un batrân de la lucru din câmp, seara; acesta era de neam din Muntele Efraim ?i traia în Ghibeea. Iar locuitorii din cetatea aceasta erau din fiii lui Veniamin.
\par 17 ?i ridicându-?i ochii sai, vazu un trecator pe uli?a ceta?ii; ?i a zis batrânul: "Încotro mergi ?i de unde vii?"
\par 18 A zis acela: "Noi mergem de la Betleemul Iudei la muntele lui Efraim, de unde sunt eu; am fost la Betleemul Iudei ?i acum ma duc la casa Domnului; dar nimeni nu ma cheama în casa;
\par 19 Noi avem ?i paie ?i nutre? pentru asinii no?tri; de asemenea pâine ?i vin pentru mine ?i pentru roaba ta ?i pentru sluga aceasta a robilor tai; n-avem nevoie de nimic".
\par 20 Atunci batrânul a zis: "Fi?i lini?ti?i; toate lipsurile ramân asupra mea, numai sa nu ramâi în uli?a!"
\par 21 Apoi l-a dus în casa sa ?i a dat nutre? asinilor lui, iar ei ?i-au spalat picioarele ?i au mâncat ?i au baut.
\par 22 Dar dupa ce s-a veselit inima lor, iata locuitorii ceta?ii, oameni desfrâna?i, au înconjurat casa, batând în u?a ?i zicând batrânului, stapânul casei: "Scoate pe omul care a intrat în casa ta, ca sa-l cunoa?tem".
\par 23 Atunci stapânul casei a ie?it la ei ?i le-a zis: "Nu, fra?ii mei, sa nu face?i rau omului, de vreme ce a intrat în casa mea, sa nu face?i aceasta ticalo?ie!
\par 24 Iata, eu am o fiica fecioara ?i el are o concubina; vi le voi scoate, ca sa le cunoa?te?i pe ele ?i sa face?i cu ele ce va place; iar cu omul acesta sa nu face?i aceasta nebunie!"
\par 25 Dar ei n-au voit sa-l asculte. Atunci omul a luat pe concubina sa ?i a scos-o în uli?a. Iar ei au cunoscut-o pe ea ?i ?i-au batut joc de ea toata noaptea pâna diminea?a, ?i la ivirea zorilor au parasit-o.
\par 26 ?i în revarsatul zorilor a venit femeia ?i a cazut înaintea u?ii casei omului aceluia, la care era stapânul ei ?i a zacut acolo pâna s-a facut ziua.
\par 27 Stapânul ei însa s-a sculat diminea?a, a deschis u?a casei ?i a ie?it, ca sa plece în drumul sau; ?i iata concubina sa zacea la u?a casei, ?i mâinile ei erau pe prag.
\par 28 Atunci el i-a zis: "Scoala sa mergem!" Dar n-a primit nici un raspuns, pentru ca murise. Atunci el a pus-o pe un asin ?i s-a ridicat ?i a plecat la casa sa.
\par 29 Iar daca a ajuns la casa sa, a luat un cu?it ?i, apucând pe concubina sa, a taiat-o buca?ica eu buca?ica în douasprezece par?i ?i le-a trimis în toate hotarele lui Israel.
\par 30 Tot cel ce vedea acestea zicea: "N-a mai fost, nici nu s-a mai vazut ceva asemenea din zilele ie?irii fiilor lui Israel din ?ara Egiptului, pâna în ziua aceasta". Iar oamenilor trimi?i din partea sa le daduse porunca, zicându-le: "A?a sa spune?i la tot Israelul: A mai fost oare cândva asemenea cu aceasta? Lua?i seama la aceasta, sfatui?i-va ?i hotarâ?i!"

\chapter{20}

\par 1 Atunci au ie?it fiii lui Israel ?i s-a adunat toata ob?tea, ca un singur om, de la Dan pâna la Beer?eba, cu ?ara Galaadului, înaintea Domnului la Mi?pa.
\par 2 Capeteniile întregului popor ?i toate semin?iile lui Israel s-au înfa?i?at înaintea Domnului, la adunarea poporului lui Dumnezeu, ca la patru sute de mii pede?tri, purtatori de sabie.
\par 3 ?i au auzit fiii lui Veniamin ca fiii lui Israel au venit la Mi?pa. Atunci au zis fiii lui Israel: "Spune?i cum s-a facut nelegiuirea aceasta?"
\par 4 Iar levitul, barbatul femeii celei ucise, a raspuns ?i a zis: "Eu cu concubina mea am venit sa ramânem în Ghibeea lui Veniamin.
\par 5 ?i s-au ridicat asupra mea locuitorii din Ghibeea ?i au înconjurat pentru mine casa, noaptea; aveau de gând sa ma ucida ?i au chinuit pe concubina mea, batându-?i joc de ea, a?a încât ea a murit.
\par 6 Atunci eu am luat concubina mea, am taiat-o ?i am trimis-o în toate ?inuturile stapânirii lui Israel, pentru ca ei au facut un lucru nelegiuit ?i de ru?ine în Israel.
\par 7 Iata acum voi, fiii lui Israel, cerceta?i cu to?ii acest lucru ?i hotarâ?i aici".
\par 8 ?i s-a ridicat tot poporul, ca un singur om, ?i a zis: "Nu ne vom duce nici unul la corturile noastre ?i nimeni nu se va întoarce la casa sa,
\par 9 Ci iata ce vom face acum cu Ghibeea: Vom merge asupra ei dupa sor?i;
\par 10 ?i vom lua câte zece oameni la suta din toate semin?iile lui Israel, câte o suta la mie ?i câte o mie la zece mii, ca sa aduca merinde pentru poporul care se va duce asupra Ghibeii lui Veniamin, ca sa o pedepseasca pentru lucrul ru?inos pe care l-a facut ea în Israel".
\par 11 ?i s-au adunat to?i Israeli?ii asupra ceta?ii într-un cuget, ca un singur om.
\par 12 ?i au trimis semin?iile lui Israel în toata semin?ia lui Veniamin sa se spuna: "Ce lucru ru?inos s-a facut la voi?
\par 13 Da?i pe acei oameni ticalo?i care sunt în Ghibeea, ca avem sa-i omorâm ?i sa stârpim raul din Israel!" Dar fiii lui Veniamin n-au voit sa asculte glasul fra?ilor lor, adica al fiilor lui Israel.
\par 14 ?i s-au adunat fiii lui Veniamin de prin ceta?i la Ghibeea, ca sa mearga cu razboi asupra fiilor lui Israel.
\par 15 ?i s-au numarat în ziua aceea fiii lui Veniamin, care se adunasera de prin ceta?i, douazeci ?i ?ase de mii de oameni purtatori de sabie; afara de ace?tia se mai numarau din locuitorii Ghibeii ?apte sute de oameni ale?i.
\par 16 Din tot poporul acesta erau ?apte sute de oameni ale?i, care erau stângaci, ?i to?i ace?tia nimereau drept la ?inta când aruncau pietre cu pra?tia în firul de par.
\par 17 Israeli?ii însa, afara de fiii lui Veniamin, numarau patru sute de mii de oameni purtatori de sabie ?i to?i ace?tia erau destoinici la lupta.
\par 18 ?i s-au sculat ?i s-au dus la casa Domnului ?i au întrebat pe Dumnezeu ?i au zis fiii lui Israel: "Cine din noi va pleca întâi la razboi cu fiii lui Veniamin?" ?i Domnul a zis: "Iuda va pleca întâi!"
\par 19 Apoi s-au sculat fiii lui Israel diminea?a ?i au tabarât lânga Ghibeea.
\par 20 ?i au pornit fiii lui Israel la razboi împotriva lui Veniamin ?i s-au pus fiii lui Israel în rânduiala de razboi aproape de Ghibeea.
\par 21 Iar fiii lui Veniamin au ie?it din Ghibeea ?i au pus în ziua aceea douazeci ?i doua de mii de Israeli?i la pamânt.
\par 22 Dar poporul israelit se îmbarbata ?i se puse din nou în rânduiala de razboi în acela?i loc unde statuse în ziua întâi.
\par 23 ?i s-au dus fiii lui Israel ?i au plâns înaintea Domnului pâna seara ?i au întrebat pe Domnul: "Sa mai mergem oare la lupta cu fiii lui Veniamin, fratele meu?" ?i Domnul a zis: "Merge?i asupra lui!"
\par 24 ?i au înaintat fiii lui Israel asupra fiilor lui Veniamin a doua oara.
\par 25 ?i a ie?it Veniamin asupra lor din Ghibeea a doua zi ?i au mai pus la pamânt din fiii lui Israel înca optsprezece mii de oameni purtatori de sabie.
\par 26 Atunci to?i fiii lui Israel ?i tot poporul au plecat ?i au venit la casa Domnului ?i au postit în ziua aceea pâna seara ?i au adus arderi de tot ?i jertfe de împacare înaintea Domnului.
\par 27 ?i au întrebat fiii lui Israel pe Domnul. Pe atunci chivotul legii Domnului se afla acolo,
\par 28 ?i Finees, fiul lui Eleazar, fiul lui Aaron, sta înaintea lui Dumnezeu, zicând: "Sa mai ies eu oare la lupta cu fiii lui Veniamin, fratele meu, sau nu?" Iar Domnul a zis: "Duce?i-va, ca mâine Eu am sa-l dau în mâinile tale!"
\par 29 ?i au pus fiii lui Israel oameni de paza împrejurul Ghibeii.
\par 30 Apoi s-au dus fiii lui Israel asupra fiilor lui Veniamin a treia zi ?i s-au pus în rânduiala de razboi înaintea Ghibeii, ca ?i mai înainte.
\par 31 Iar fiii lui Veniamin au ie?it asupra poporului ?i s-au departat de cetate, începând, ca ?i mai înainte, a ucide din popor pe caile ce duceau una spre Betleem, iar alta spre Ghibeea, peste câmp; ?i au ucis pâna la treizeci de in?i dintre Israeli?i.
\par 32 Atunci au zis fiii lui Veniamin: "Ace?tia au sa cada înaintea noastra, ca ?i înainte". Iar fiii lui Israel au zis: "Sa fugim de ei ?i sa-i departam de cetate pe cale". ?i au facut a?a.
\par 33 Atunci to?i Israeli?ii s-au sculat de la locurile lor ?i au tabarât la Baal-Tamar. Iar oamenii de paza ai lui Israel au alergat de la locul lor, din partea de apus a Ghibeii.
\par 34 ?i au sosit înaintea Ghibeii zece mii de oameni ale?i din tot Israelul ?i s-a început o lupta crâncena; dar fiii lui Veniamin nu ?tiau ca-i amenin?a primejdia.
\par 35 ?i a lovit Domnul pe Veniamin înaintea Israeli?ilor ?i Israeli?ii au rapus în ziua aceea din fiii lui Veniamin douazeci ?i cinci de mii o suta de oameni purtatori de sabie.
\par 36 Atunci au vazut fiii lui Veniamin ca sunt înfrân?i; caci Israeli?ii nu se retrageau din fa?a fiilor lui Veniamin, decât pentru ca se bizuiau pe oamenii pe care ei îi pusesera de paza împotriva Ghibeii.
\par 37 Cei pu?i la pânda s-au aruncat repede asupra Ghibeii ?i au intrat în ea ?i au trecut toata cetatea prin ascu?i?ul sabiei.
\par 38 Israeli?ii însa se în?elesesera cu oamenii de paza ca sa le fie semn al navalirii fumul ce se va ridica din cetate.
\par 39 Deci când Israeli?ii s-au tras înapoi de la locul de lupta, Veniamin a început sa loveasca ?i a ranit pâna la vreo treizeci de Israeli?i, ?i ziceau: "Iara?i au sa cada înaintea noastra, ca ?i în luptele dinainte!"
\par 40 Atunci a început sa se ridice din cetate un stâlp de fum. ?i uitându-se Veniamin înapoi, iata din toata cetatea se înal?a fum spre cer.
\par 41 În clipa aceasta Israeli?ii se întoarsera, iar Veniamin s-a speriat, caci a vazut ca-l ajunsese primejdia;
\par 42 ?i au fugit ei de Israeli?i pe calea ce ducea spre pustie; dar macelul îi urmarea ?i cei ce ie?eau din cetate erau uci?i pe loc.
\par 43 ?i au împresurat Israeli?ii pe Veniamin ?i l-au urmarit pâna la Menoha ?i i-au macelarit pâna în partea rasariteana a Ghibeii.
\par 44 Atunci au cazut din fiii lui Veniamin optsprezece mii de in?i, to?i barba?i voinici.
\par 45 Iar câ?i au ramas s-au abatut ?i au fugit în pustiu spre stânca lui Rimon ?i au mai ucis Israeli?ii pe drum cinci mii de oameni; alergând dupa ei pâna la Ghideom au mai ucis din ei înca doua mii de oameni.
\par 46 Iar to?i fiii lui Veniamin care au cazut în ziua aceea au fost douazeci ?i cinci de mii, purtatori de sabie, ?i to?i ace?tia erau oameni voinici.
\par 47 ?i au fugit cei ce scapasera în pustiu, la stânca lui Rimon, ca la ?ase sute de oameni ?i au ramas acolo în muntele cel stâncos al lui Rimon patru luni.
\par 48 Iar Israeli?ii s-au întors la fiii lui Veniamin ?i i-au lovit cu sabia în cetate: ?i oameni ?i vite ?i tot ce au întâlnit în toate ceta?ile ?i toate ceta?ile ce-au întâlnit în cale le-au ars cu foc.

\chapter{21}

\par 1 ?i s-au jurat Israeli?ii în Mi?pa, zicând: "Nimeni din noi sa nu-?i dea fetele sale de so?ii dupa fiii lui Veniamin".
\par 2 Apoi a venit poporul la Betel ?i a stat acolo pâna seara înaintea lui Dumnezeu, ?i a ridicat glasul sau ?i a plâns cu jale mare,
\par 3 Zicând: "Doamne, Dumnezeul lui Israel, pentru ce oare s-a petrecut aceasta în Israel, ca iata acum lipse?te din Israel o semin?ie?"
\par 4 Iar a doua zi s-a sculat poporul de diminea?a ?i a facut acolo jertfelnic ?i a adus arderi de tot ?i jertfe de izbavire.
\par 5 Apoi au zis fiii lui Israel: "Cine oare n-a venit la adunarea ce s-a ?inut înaintea Domnului dintre toate semin?iile lui Israel?" Caci blestem mare se rostise asupra acelora care nu aveau sa vina înaintea Domnului în Mi?pa ?i se zisese ca aceia sa fie da?i mor?ii.
\par 6 ?i s-au înduio?at fiii lui Israel fa?a de Veniamin, fratele lor, zicând: "Acum s-a taiat o semin?ie din Israel.
\par 7 Ce vom face pentru a gasi femei celor care au ramas, deoarece ne-am jurat înaintea Domnului sa nu le dam femei din fetele noastre?
\par 8 Atunci s-a vazut ca din Iabe?-Galaad nu venise nimeni înaintea Domnului la adunarea din tabara.
\par 9 ?i s-a cercetat poporul ?i iata nu era acolo nici unul din locuitorii Iabe?ului din Galaad.
\par 10 Atunci a trimis acolo ob?tea douasprezece mii de oameni, barba?i voinici ?i le-a dat porunca, zicând: "Merge?i ?i lovi?i pe locuitorii din Iabe?ul Galaadului cu sabia, ?i femeile, ?i copiii.
\par 11 ?i iata ce sa mai face?i: pe orice barbat ?i orice femeie care a cunoscut barbat, sa-i da?i pieirii, iar fetele lasa?i-le cu via?a". ?i a?a au facut.
\par 12 ?i au gasit ei printre locuitorii din Iabe?ul Galaadului patru sute de fete care nu cunoscusera barbat ?i le-au adus în tabara la ?ilo, care e în pamântul Canaan.
\par 13 Atunci toata ob?tea a trimis sa graiasca fiilor lui Veniamin care erau la stânca lui Rimon ?i sa le vesteasca pace.
\par 14 ?i s-au întors fiii lui Veniamin la Israeli?i ?i Israeli?ii le-au dat so?ii din femeile ramase în via?a din Iabe?ul Galaadului. Dar curând s-a vazut ca acestea nu erau de ajuns.
\par 15 Poporul însa jelea dupa Veniamin, ca Domnul n-a pastrat în întregime semin?iile lui Israel.
\par 16 Au zis drept aceea batrânii ob?tii: "Ce sa facem cu cei rama?i fara femei, caci au fost stârpite femeile în Veniamin?"
\par 17 Apoi au zis: "Pamântul de mo?tenire sa ramâna în întregime fiilor lui Veniamin, ca sa nu piara semin?ia lui din Israel.
\par 18 Dar noi nu le putem da femei din fetele noastre, caci fiii lui Israel s-au jurat, zicând: Blestemat sa fie cel ce va da femei lui Veniamin!"
\par 19 ?i au mai zis: "Iata, în fiecare an se face sarbatoarea Domnului în ?ilo, care este a?ezat la miazanoapte de Betel ?i la rasarit de drumul ce duce de la Betel la Sichem ?i la miazazi de Lebona".
\par 20 Drept aceea au poruncit fiilor lui Veniamin ?i au zis: "Merge?i ?i pândi?i din vii
\par 21 ?i baga?i de seama când vor ie?i fetele din ?ilo sa joace la hora; atunci sa ie?i?i din vii ?i sa va lua?i femei din fetele din ?ilo ?i merge?i în pamântul lui Veniamin.
\par 22 Iar când vor veni parin?ii lor sau fra?ii lor cu plângere la noi, noi le vom zice: "Ierta?i-i pentru noi, caci noi n-am luat în razboi femei pentru fiecare dintre ei ?i nici voi nu le-a?i dat; acum ?i voi sunte?i de vina".
\par 23 ?i fiii lui Veniamin a?a au ?i facut ?i ?i-au luat femei dupa numarul lor din cele ce erau la hora ?i pe care ei le-au rapit ?i s-au dus înapoi în mo?tenirea lor ?i au zidit ceta?i ?i au început sa traiasca în ele.
\par 24 În acela?i timp Israeli?ii s-au împar?it de acolo ?i s-a dus fiecare în semin?ia sa ?i la mo?tenirea lui.
\par 25 În zilele acelea nu era rege în Israel ?i fiecare facea ce i se parea ca este cu dreptate.


\end{document}