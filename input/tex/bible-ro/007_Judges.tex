\begin{document}

\title{Judecători}


\chapter{1}

\par 1 După moartea lui Iosua, au întrebat fiii lui Israel pe Domnul, zicând: "Cine din noi să meargă mai întâi asupra Canaaneilor, ca să se lupte cu ei?"
\par 2 Și a zis Domnul: "Iuda să meargă și Eu voi da țara în mâna lui!"
\par 3 Iuda a zis către Simeon, fratele său: "Intră cu mine în ținutul care mi-a căzut la sorți și să ne luptăm cu Canaaneii și voi intra și eu cu tine în ținutul care ți-a căzut la sorți". și s-a dus Simeon cu el.
\par 4 Atunci s-a dus Iuda cu Simeon și Domnul a dat pe Canaanei și pe Ferezei în mâinile lor și au ucis ei din aceia zece mii de oameni în Bezec.
\par 5 Și s-au întâlnit ei în Bezec cu Adoni-Bezec și s-au bătut cu dânsul și au răpus pe Canaanei și pe Ferezei,
\par 6 Iar Adoni-Bezec a fugit; dar ei au alergat după el și l-au prins și i-au tăiat degetele cele mari de la mâini și de la picioare.
\par 7 Atunci Adoni-Bezec a zis: "Șaptezeci de regi cu degetele cele mari tăiate de la mâinile și de la picioarele lor adunau fărâmituri sub masa mea; cum am făcut eu, așa mi-a plătit și mie Dumnezeu. Și l-au dus în Ierusalim și a murit acolo.
\par 8 Apoi au mers fiii lui Iuda cu război asupra Ierusalimului și l-au luat și l-au trecut prin sabie, iar cetatea au dat-o focului.
\par 9 După aceea s-au dus fiii lui Iuda să se lupte cu Canaaneii care trăiau în munți, în țara de miazăzi și în părțile de jos.
\par 10 Și a mers Iuda asupra Canaaneilor care locuiau în Hebron; iar numele Hebronului era mai înainte Chiriat-Arba; și a bătut per Șeșai, pe Ahiman și pe Talmai, din neamul lui Enac.
\par 11 Iar de acolo s-au dus împotriva locuitorilor Debirului al cărui nume era mai înainte Chiriat-Sefer.
\par 12 Atunci Caleb a zis: "Cine va lovi Chiriat-Seferul și-l va lua, aceluia îi voi da pe Acsa, fiica mea, de femeie".
\par 13 Și l-a luat Otniel, fiul lui Chenaz, fratele mai mic al lui Caleb, și acesta i-a dat de soție pe Acsa, fiica sa.
\par 14 Și când a vrut ea să plece, Otniel a învățat-o să ceară la tatăl ei o țarină și ea s-a coborât de pe asin. Și i-a zis Caleb: "Ce vrei?"
\par 15 Iar Acsa a zis către el: "Dă-mi binecuvântare; tu mi-ai dat pământul de la miazăzi, dă-mi și izvoarele de apă". Și i-a dat Caleb, după dorința ei, izvoarele cele de sus și izvoarele cele de jos.
\par 16 Iar fiii lui Hobab cheneul, socrul lui Moise, au mers din cetatea Palmierilor cu fiii lui Iuda în pustiul lui Iuda din Negheb, care este la miazăzi de Arad și, venind, s-au așezat între Amaleciți.
\par 17 Apoi a mers Iuda cu Simeon, fratele său și a lovit pe Canaaneii care locuiau în Țefat și i-au dat pieirii; de aceea s-a numit cetatea aceea Horma.
\par 18 Și a mai luat Iuda și Gaza cu împrejurimile ei, Ascalonul cu împrejurimile lui și Ecronul cu ținutul lui.
\par 19 Și a fost Domnul cu Iuda și acesta a luat în stăpânire partea muntoasă; dar pe locuitorii din vale nu i-a putut alunga, pentru că ei aveau care de fier.
\par 20 Și au dat lui Caleb Hebronul, cum zisese Moise, și a primit el acolo de moștenire cele trei cetăți ale fiilor lui Enac și a alungat pe cei trei fii ai lui Enac.
\par 21 Iar fiii lui Veniamin n-au alungat pe Iebuseii care locuiau în Ierusalim; și trăiesc Iebusei cu fiii lui Veniamin în Ierusalim până în ziua de astăzi.
\par 22 S-au suit de asemenea și fiii lui Iosif asupra Betelului și Domnul a fost cu ei,
\par 23 Și s-au oprit ei și au cercetat Betelul; iar numele cetății acesteia era mai înainte Luz.
\par 24 Și au văzut cei de strajă un om ieșind din cetate și l-au prins și i-au zis: "Arată-ne intrarea cetății și vom avea milă de tine".
\par 25 Și le-a arătat intrarea cetății și ei au lovit cetatea cu sabia, iar omului aceluia și la tot neamul lui i-au dat drumul.
\par 26 Și s-a dus omul acela în pământul Heteilor și a zidit acolo o cetate și i-a pus numele Luz. Și acesta este numele ei până în ziua de astăzi.
\par 27 Manase de asemenea n-a alungat pe locuitorii Bet-Șeanului, care e Schitopole, și ai cetăților supuse lui, nici pe ai Taanacului și ai cetăților supuse lui, nici pe locuitorii din Dor și ai cetăților supuse lui, nici pe locuitorii Ibleamului și ai cetăților supuse lui, nici pe locuitorii Meghidonului și ai cetăților supuse lui; și au rămas Canaaneii să trăiască în pământul acesta.
\par 28 Când Israel a ajuns puternic, atunci a făcut pe Canaanei birnici, dar de izgonit nu i-a izgonit.
\par 29 Nici Efraim n-a izgonit pe Canaaneii care locuiau în Ghezer, și au trăit Canaaneii în mijlocul lor în Ghezer și le-au plătit bir.
\par 30 Zabulon încă n-a izgonit pe locuitorii Chitronului și pe locuitorii Nahalolului; și au locuit Canaaneii în mijlocul lor și le-au plătit bir.
\par 31 Nici Așer n-a izgonit pe locuitorii din Aco care-i plăteau bir, nici pe locuitorii din Dor, nici pe locuitorii din Sidon și Mahaleb, din Aczib, din Helba, din Afec și din Rehob.
\par 32 Și a locuit Așer între Canaanei, locuitorii rării aceleia, căci nu i-a izgonit.
\par 33 Nici Neftali n-a izgonit pe locuitorii din Bet-Șemeș, nici pe locuitorii din Bet-Anat, și a trăit între Canaanei, locuitorii jării aceleia. Iar locuitorii din Bet-Șemeș și din Bet-Anat erau birnicii lui.
\par 34 Dar Amoreii au împins pe fiii lui Dan în munți și nu i-a lăsat să se coboare în vale.
\par 35 Și au rămas Amoreii să locuiască în Har-Heres, în Aialon și Șaalbim; dar mâna fiilor lui Iosif a răpus pe Amorei și au ajuns aceștia birnicii lor.
\par 36 Iar hotarele Amoreilor se întindeau până dincolo de înălțimea Acrabimului și de Sela.

\chapter{2}

\par 1 Atunci s-a suit un înger al Domnului din Ghilgal către Bochim, către Betel și către casa lui Israel și le-a zis: "Așa grăiește Domnul:
\par 2 Eu v-am scos din Egipt și v-am băgat în țara pentru care M-am jurat părinților voștri să v-o dau și am zis: Nu voi rupe în veac legământul Meu cu voi; voi însă să nu intrați în legătură cu locuitorii țării acesteia; dumnezeilor lor să nu vă închinați, idolii lor să-i sfărâmați și jertfelnicele lor să le dărâmați. Dar voi n-ați ascultat glasul Meu. Pentru ce ați făcut aceasta?
\par 3 De aceea vă zic: Nu Mă voi apuca să strămut pe locuitorii aceștia pe care Eu am voit să-i alung, nu-i voi izgoni de la voi și ei vă vor fi laț, iar dumnezeii lor vor fi pentru voi mreajă".
\par 4 Când a spus îngerul Domnului cuvintele acestea tuturor fiilor lui Israel, atunci poporul a ridicat strigare mare și a plâns.
\par 5 De aceea s-a și numit locul acela Bochim. Și au adus ei acolo jertfă Domnului.
\par 6 După ce a dat Iosua drumul poporului și s-au dus fiii lui Israel fiecare la casa sa și fiecare la moșia sa, ca să ia țara de moștenire,
\par 7 Atunci poporul a slujit Domnului în toate zilele lui Iosua și în toate zilele bătrânilor, a căror viață s-a prelungit după Iosua și care văzuseră toate lucrurile cele mari ale Domnului, pe care le făcuse El cu Israel.
\par 8 Iosua, fiul lui Navi, sluga Domnului, a murit fiind de o sută zece ani,
\par 9 Și l-au îngropat în cuprinsul moștenirii sale, la Timnat-Heres, în muntele lui Efraim, spre miazănoapte de muntele Gaaș;
\par 10 Și după ce tot rândul acela de oameni a trecut la părinții lor și când s-a ridicat în locul lor alt rând de oameni, care nu cunoșteau pe Domnul și lucrurile Sale pe care le făcuse cu Israel,
\par 11 Atunci fiii lui Israel au început a face rele înaintea ochilor Domnului și s-au apucat să slujească baalilor;
\par 12 Au părăsit pe Domnul Dumnezeul părinților lor, Care îi scosese din pământul Egiptului și s-au întors la alți dumnezei, către dumnezeii popoarelor dimprejurul lor și au început să se închine acelora și au mâniat pe Domnul;
\par 13 Au părăsit pe Domnul și au început a se închina lui Baal și Astartelor.
\par 14 De aceea s-a aprins mânia Domnului asupra lui Israel și l-a dat în mâinile jefuitorilor care i-au jefuit; i-a dat în mâinile vrăjmașilor dimprejurul lor și n-au mai putut să se împotrivească vrăjmașilor lor.
\par 15 Ori încotro apucau, mâna Domnului pretutindeni era împotriva lor, ca să facă rău, cum le grăise Domnul și cum li Se jurase. Și erau foarte strâmtorați.
\par 16 Atunci le-a ridicat Domnul judecători care i-au izbăvit din mâinile jefuitorilor lor.
\par 17 Dar nici pe judecători nu-i ascultau ei, ci se purtau desfrânat mergând pe urmele altor dumnezei și se închinau acelora și mâniau pe Domnul; ușor se abăteau de la calea pe care umblaseră părinții lor care se supuseseră poruncilor Domnului. Ei însă nu făceau așa.
\par 18 Când le ridica lor Domnul judecători, atunci Însuși Domnul era cu judecătorul și-i izbăvia pe ei de vrăjmașii lor în toate zilele judecătorului.
\par 19 Dar cum murea judecătorul, ei iarăși făceau și mai rău decât părinții lor, abătându-se la alți dumnezei, slujind acelora și închinându-se lor, nu se lăsau de lucrurile lor și nu se abăteau de la calea lor cea rea.
\par 20 Și se aprindea mânia Domnului asupra lui Israel și zicea: "Pentru că poporul acesta calcă poruncile Mele, pe care Eu le-am așezat cu părinții lor și nu ascultă glasul Meu,
\par 21 De aceea nici Eu nu voi mai izgoni de la ei nici unul din acele popoare pe care le-a lăsat Iosua, fiul lui Navi, în țară, când a murit,
\par 22 Ca să ispitească prin ele pe Israel și să vadă de va ține el calea Domnului și de va umbla pe ea, cum s-au ținut părinții lor, sau nu".
\par 23 Și a lăsat Domnul pe popoarele acestea și nu le-a alungat îndată, nici nu le-a dat în mâinile lui Iosua.

\chapter{3}

\par 1 Iată popoarele acelea pe care le-a lăsat Domnul ca să ispitească prin ele pe Israel și pe toți aceia care nu cunoșteau toate războaiele Canaanului;
\par 2 Pe care le lăsase numai pentru ca generațiile viitoare de oameni ale fiilor lui Israel să știe și să învețe războiul pe care nu-l cunoscuseră mai înainte:
\par 3 Cinci stăpânitori Filisteni, toți Canaaneii, Sidonienii și Heveii care locuiau în munții Libanului de la muntele Baal-Hermon până la intrarea Hamatului.
\par 4 Acestea fuseseră lăsate ca să se încerce prin ele Israeliții și să se afle dacă se supun ei poruncilor Domnului, pe care le-a dat El părinților lor prin Moise.
\par 5 Și au trăit fiii lui Israel între Canaanei, Hetei, Amorei, Ferezei, Hevei, Gherghesei și Iebusei,
\par 6 Și și-au luat femei din fetele acelora și pe fetele lor le-au dat după feciorii acelora și au slujit dumnezeilor lor.
\par 7 Deci au făcut rele înaintea ochilor Domnului fiii lui Israel și au uitat pe Domnul Dumnezeul lor și au slujit baalilor și astartelor.
\par 8 Atunci s-a aprins mânia Domnului asupra lui Israel și i-a dat în mâinile lui Cușan-Rișeataim, regele Mesopotamiei, și fiii lui Israel au robit lui Cușan-Rișeataim opt ani.
\par 9 După aceea au strigat fiii lui Israel către Domnul și a ridicat Domnul un izbăvitor pentru fiii lui Israel, care i-a izbăvit și anume pe Otniel, fiul lui Chenaz, fratele mai mic al lui Caleb.
\par 10 Și a fost Duhul Domnului peste acesta și a fost el judecător lui Israel. Acesta a ieșit la război împotriva lui Cușan-Rișeataim și Domnul a dat în mâinile lui pe Cușan-Rișeataim, regele Mesopotamiei, și a apăsat mâna lui pe Cușan-Rișeataim.
\par 11 După aceea s-a odihnit țara patruzeci de ani și apoi a murit Otniel, fiul lui Chenaz.
\par 12 Apoi fiii lui Israel iarăși s-au apucat să facă rele înaintea ochilor Domnului și a întărit Domnul pe Eglon, regele Moabului, împotriva Israeliților, pentru că ei făceau rele înaintea ochilor Domnului.
\par 13 Și a adunat acela la sine pe toți Moabiții și Amaleciții și a plecat să lovească pe Israel; și au luat cetatea Palmierilor.
\par 14 Iar fiii lui Israel au slujit lui Eglon, regele Moabului, optsprezece ani.
\par 15 Atunci au strigat fiii lui Israel către Domnul și Domnul le-a ridicat ca izbăvitor pe Aod, fiul lui Ghera, din neamul lui Veniamin, care era stângaci. Și au trimis fiii lui Israel prin el daruri lui Eglon, regele Moabului.
\par 16 Aod și-a făcut sabie cu două ascuțișuri, lungă de un cot și a încins-o sub mantaua sa la șoldul drept
\par 17 Și a mers cu daruri la Eglon, regele Moabului. Eglon însă era om foarte gras.
\par 18 După ce a înfățișat Aod toate darurile, a dat drumul oamenilor care aduseseră darurile,
\par 19 Iar el însuși, întorcându-se de la idolii de lângă Ghilgal, a zis regelui: "Cuvânt în taină am a-ți spune, o rege!" Iar el a zis: "Mai încet!" Atunci au ieșit de la dânsul toți cei ce stăteau pe lângă el.
\par 20 Și a intrat Aod la dânsul; căci el ședea într-un foișor răcoros, pe care îl avea acolo la o parte. Și a zis Aod: "Eu, o rege, am către tine un cuvânt al lui Dumnezeu". Atunci Eglon s-a sculat de pe scaun înaintea lui.
\par 21 Și când s-a sculat el, Aod și-a întins mâna sa stângă și a scos sabia de la coapsa sa dreaptă și a împlântat-o în pântecele lui,
\par 22 Așa încât a intrat după ascuțișul sabiei și mânerul și grăsimea a acoperit rana pe unde intrase sabia, căci Aod n-a scos-o din pântecele lui.
\par 23 Apoi Aod a ieșit în tindă, trăgând după sine ușa foișorului și încuind-o.
\par 24 Iar după ce a ieșit el, au venit slugile lui Eglon și, văzând ușa foișorului încuiată, au zis: "Se vede că el își acoperă picioarele În camera de vară".
\par 25 Și au așteptat ei destulă vreme; dar văzând că nu mai deschide nimeni ușa foișorului, au adus o cheie și au deschis și iată stăpânul lor zăcea mort la pământ.
\par 26 Până să se dumirească aceia, Aod a plecat și nimeni nu se mai gândea la el; a trecut pe lângă idoli și a scăpat în Seira.
\par 27 Iar după ce a venit în țara lui Israel, Aod a trâmbițat din trâmbiță pe muntele Efraim și s-au coborât la dânsul fiii lui Israel din muntele Efraim și el mergea înaintea lor.
\par 28 Și ța zis el către dânșii: "Veniți după mine, că a dat Domnul pe vrăjmașii noștri Moabiți în mâinile voastre". Și s-au dus după dânsul și au apucat vadul Iordanului spre Moab și nu au lăsat pe nimeni să treacă.
\par 29 Și au ucis atunci din Moabiți până la zece mii de oameni, toți sănătoși și voinici, încât nimeni n-a scăpat.
\par 30 Așa au fost supuși în ziua aceea Moabiții înaintea lui Israel și s-a liniștit țara lui optzeci de ani. Iar Aod a fost judecătorul lor până la moartea sa.
\par 31 După dânsul a fost judecător Șamgar, fiul lui Anat, care a ucis șase sute de Filisteni cu un băț, cu bold de mânat boii, și acesta a izbăvit de asemenea pe Israel.

\chapter{4}

\par 1 După ce a murit Aod, fiii lui Israel au început iar să facă rele înaintea ochilor Domnului.
\par 2 Și Domnul i-a dat în mâinile lui Iabin, regele Canaanului, care domnea în Hațor. Acesta avea căpetenie peste oștire pe Sisera care locuia în Haroșet-Goim.
\par 3 Și au strigat fiii lui Israel către Domnul; căci Iabin avea nouă sute de care de fier și a apăsat cumplit pe fiii lui Israel douăzeci de ani.
\par 4 În vremea aceea era judecător în Israel Debora-proorocița, soția lui Lapidot.
\par 5 Aceasta locuia sub palmierul Deborei, între Rama și Betel, pe muntele Efraim și veneau acolo la ea fiii lui Israel să se judece.
\par 6 Și a trimis Debora de a chemat pe Barac, fiul lui Abinoam, din Chedeșul Neftalimului și i-a zis: "Domnul Dumnezeul lui Israel îți poruncește: Du-te și te suie pe muntele Tabor și ia cu tine zece mii de oameni din fiii lui Neftali și din fiii lui Zabulon;
\par 7 Iar Eu voi aduce la tine, la pârâul Chișon, pe Sisera, căpetenia oștirilor lui Iabin și carele lui și oastea lui cea multă și-l voi da în mâinile tale".
\par 8 Iar Barac a zis către dânsa: "De mergi tu cu mine, mă voi duce; iar de nu mergi cu mine, eu nu mă voi duce. Căci eu nu știu ziua când are să trimită Domnul pe îngerul Său în ajutorul meu".
\par 9 Atunci ea a zis către el: "De mers voi merge cu tine, dar să știi că nu va mai fi slava ta în calea aceasta în care mergi; ci în mâna unei femei va da Domnul pe Sisera". Și s-a sculat Debora și s-a dus cu Barac la Chedeș.
\par 10 Și a chemat Barac pe Zabulonieni și Neftalimieni la Chedeș și s-au dus după dânsul zece mii de oameni și s-a dus și Debora cu dânșii.
\par 11 Atunci Heber Cheneul s-a despărțit de Chenei, fiii lui Hobab, rudenia lui Moise, și și-a întins cortul său la dumbrava din Țaanaim, aproape de Chedeș.
\par 12 Și i s-a spus lui Sisera că Barac, fiul lui Abinoam, s-a suit pe muntele Taborului.
\par 13 Atunci Sisera a adunat toate carele sale, nouă sute de care de fier, și tot poporul pe care-l avea și a venit din Haroșet-Goim la râul Chișon.
\par 14 Iar Debora a zis către Barac: "Scoală, că aceasta este ziua aceea în care Domnul are să dea pe Sisera în mâinile tale. Însuși Domnul are să meargă înaintea ta". Și s-a coborât Barac din muntele Taborului și după el și cei zece mii de oameni.
\par 15 Atunci Domnul a pus pe fugă pe Sisera și toate carele lui și toată tabăra lui prin sabia lui Barac; și s-a coborât Sisera din carul său și a fugit pe jos.
\par 16 Iar Barac a urmărit carele lui și tabăra lui până la Haroșet-Goim și a căzut toată oștirea lui Sisera de sabie și nimeni n-a rămas.
\par 17 Sisera însă a fugit pe jos în cortul Iaelei, femeia lui Heber Cheneul; căci între Iabin, regele Hațorului, și casa lui Heber Cheneul era pace.
\par 18 Și a ieșit Iaela în întâmpinarea lui Sisera și i-a zis: "Intră, domnul meu intră la mine, nu te teme!" Și el a intrat la ea în cort și ea l-a acoperit cu haina sa.
\par 19 Și a zis Sisera către ea: "Dă-mi putină apă să beau, că mi-e sete! Și ea a dezlegat un burduf cu lapte și l-a adăpat și iar l-a acoperit.
\par 20 Apoi Sisera i-a zis: "Stai la ușa cortului și de va veni cineva să te întrebe și va zice: Nu este aici cineva? Tu să zici: Nu!"
\par 21 Apoi Iaela, femeia lui Heber, a luat un țăruș de la cort și un ciocan în mâna sa și s-a apropiat de el încetișor și i-a înfipt țărușul în tâmpla lui, așa încât l-a pironit la pământ, căci el dormea, fiind obosit; și așa a murit.
\par 22 Și iată Barac alerga în urmărirea lui Sisera. Atunci Iaela a ieșit în întâmpinarea lui și i-a zis: "Intră și-ți voi arăta pe omul pe care tu îl cauți". și el a intrat și iată Sisera zăcea mort cu țărușul în tâmplă.
\par 23 Și a supus Domnul Dumnezeu în ziua aceea pe Iabin, regele Canaanului, în fața fiilor lui Israel.
\par 24 Și s-a întărit mâna fiilor lui Israel din ce în ce mai mult asupra lui Iabin, regele Canaanului, până ce au stârpit pe Iabin, regele Canaaneilor,

\chapter{5}

\par 1 Atunci Debora și Barac, fiul lui Abinoam, au cântat cântarea aceasta:
\par 2 "Când se arată judecători în Israel, poporul merge de bunăvoie la război; Binecuvântați pe Domnul!
\par 3 Ascultați dar, regilor! Căpetenii, luați aminte! Căci cântare voi cânta Domnului. Cânta-voi Domnului Dumnezeului lui Israel:
\par 4 Când ieșeai Tu, Doamne, din Seir, Când treceai Tu prin câmpiile Edomului, Pământul se cutremura și cerurile se topeau, Norii picurau picuri de ploaie,
\par 5 Munții se năruiau înaintea Domnului, Ca și acest Sinai, la vederea Dumnezeului lui Israel.
\par 6 În zilele lui Șamgar, fiul lui Anat, Pe vremea Iaelei, drumurile erau pustii; Călătorii umblau atunci pe poteci ascunse.
\par 7 Satele în Israel erau pustii... pustii... Până m-am sculat eu, Debora, Până m-am ridicat eu, mamă în Israel.
\par 8 Pe atunci Israel își alesese dumnezei noi, De aceea războiul bătea la porți; Dar nu se vedea nici scut, nici lance în mâini, La cei patruzeci de mii din Israel.
\par 9 Inima îmi e la căpeteniile lui Israel, La cei ce plecau din popor de voie la război. Binecuvântați pe Domnul!
\par 10 Aceia călăreau pe asine murge, Sau ședeau în căruțe, sub coviltire de scoarțe scumpe Și mergeau pe drum cântând.
\par 11 Și în rândurile oștirii, ce tabăra la fântâni, Și acolo răsuna lauda Domnului, Lauda căpeteniilor lui Israel. Atunci poporul Domnului ieșea la porți.
\par 12 Deșteaptă-te, Debora, deșteaptă-te! Deșteaptă-te, deșteaptă-te și cântă! Scoală și tu, Barac! Scoală fiul lui Abinoam Și ia în robie pe cei ce te-au robit!
\par 13 Atunci poporul Domnului s-a trezit. Rămășița lui s-a strâns cu cei viteji.
\par 14 Din Efraim au purces căpetenii în vale la Chișon. Fratele tău Veniamin a fost printre oștenii tăi. Din Machir au venit căpetenii Și din Zabulon cârmuitori de oștire.
\par 15 Principii din Israel au fost lângă Debora. Isahar, credincios lui Barac, Se îndrepta în urma lui spre vale. La pâraiele lui Ruben Sunt grele cumpene sufletești!
\par 16 Pentru ce ai rămas tu în mijlocul staulelor? Ca să asculți behăitul turmelor? La pâraiele lui Ruben, Sunt grele cumpene sufletești!
\par 17 Galaadul șade liniștit dincolo de Iordan. Și Dan de ce stă el la corăbiile sale? Așer stă pe malul mării Și se odihnește în limanurile sale.
\par 18 Zabulon este un popor ce înfruntă moartea, Și este gata a-și da viața în luptă. Nu mai puf în ca el e Neftali, Care locuiește podișurile înalte.
\par 19 Atunci au venit regi să se războiască, Războitu-s-au atunci regii Canaanului La Taanac, pe apa Meghidonului, Dar n-au luat pradă, nici argint, nici bani.
\par 20 Stelele de sus s-au luptat atunci, Din mersul lor s-au războit cu Sisera.
\par 21 Pârâul Chișon, pârâu străvechi! Pârâul Chișon i-a măturat. Suflete al meu, calcă-i în picioare!
\par 22 Atunci copitele cailor în ropot loveau pământul, Fugeau vitejii lor să-și frângă gâtul.
\par 23 Blestem cetății Meroz, zice îngerul Domnului! Blestem, blestem celor ce locuiesc în ea! Că n-au venit în ajutorul Domnului, În ajutorul Domnului cu cei viteji.
\par 24 Binecuvântată să fie între femei Iaela, femeia lui Heber Cheneul! Binecuvântată fie între femeile țării.
\par 25 Sisera i-a cerut apă; ea i-a dat lapte; în cupă scumpă i-a dat cel mai bun lapte.
\par 26 Cu stânga a apucat țărușul, Iar cu dreapta ciocan greu de lucrător. Cu ciocanul a zdrobit lui Sisera capul, Cu țărușul i-a străpuns tâmpla.
\par 27 Atunci a căzut îndată la picioarele ei și acolo a rămas. Căzut-a la picioarele ei și nu s-a mai sculat. Unde a căzut, acolo a rămas zdrobit.
\par 28 Pe fereastră printre gratii privește, Se uită mama lui Sisera și strigă: "De ce nu mai vin oare carele lui  Oare de ce zăbovesc ele așa de mult?"
\par 29 Cea mai pricepută din femeile ei zice Și singură răspunde la întrebarea sa:
\par 30 "Se vede că au găsit și împart pradă: O fată sau două de cap de om, Haine pestrițe pradă pentru Sisera, Pradă de haine pestrițe cu aur cusute; Două, trei șaluri vărgate, cusute cu aur, Pentru grumajii viteazului".
\par 31 Așa să piară toți vrăjmașii Tăi, Doamne! Iar cei ce Te iubesc să fie ca soarele Când răsare în toată strălucirea lui". După aceea țara s-a bucurat de pace patruzeci de ani.

\chapter{6}

\par 1 Fiii lui Israel au început iarăși să facă rele înaintea Domnului și Domnul i-a dat în mâinile Madianiților pentru șapte ani.
\par 2 Și mâna Madianiților era grea pentru Israel, și fiii lui Israel și-au făcut, de răul Madianiților, ascunzători în munți și peșteri și stânci greu de pătruns.
\par 3 Când Israel semăna, veneau Madianiții, Amaleciții și locuitorii din pustie la el
\par 4 Și stăteau la ei în corturi, mâncând roadele pământului până la Gaza, și nu lăsau pentru hrana lui Israel nici oaie, nici bou, nici asin.
\par 5 Căci ei veneau cu vitele și cu corturile lor și veneau mulți ca lăcustele; ei și cămilele lor erau fără număr și cutreierau țara lui Israel și o pustiau.
\par 6 Și Israel a sărăcit cumplit din pricina Madianiților și a strigat către Domnul.
\par 7 Și când au strigat fiii lui Israel către Domnul împotriva Madianiților,
\par 8 A trimis Domnul prooroc la fiii lui Israel și le-a zis: "Așa grăiește Domnul Dumnezeul lui Israel: Eu v-am scos din țara Egiptului, Eu v-am scos din casa robiei;
\par 9 Eu v-am scăpat din mâinile Egiptenilor și din mâinile tuturor celor ce vă apăsau, i-am  alungat de la voi și țara lor am dat-o vouă,
\par 10 Și v-am spus: Eu sunt Domnul Dumnezeul vostru; să nu cinstiți pe dumnezeii Amoreilor, în țara cărora trăiți. Dar voi n-ați ascultat glasul Meu".
\par 11 Atunci a venit îngerul Domnului și a șezut în Ofra sub un stejar, care era al lui Ioaș, tatăl lui Abiezer; și fiul său Ghedeon treiera atunci grâul în arie, ca să-l ascundă de Madianiți.
\par 12 Și i s-a arătat îngerul Domnului și i-a zis: "Domnul este cu tine, voinicule!"
\par 13 Iar Ghedeon i-a zis: "Domnul meu, dacă Domnul e cu noi, pentru ce ne-au ajuns pe noi toate necazurile acestea? Și unde sunt oare toate minunile Lui de care ne-au istorisit nouă părinții noștri când ne spuneau: Din Egipt ne-a scos pe noi Domnul. Acum însă ne-a părăsit Domnul și ne-a dat în mâinile Madianiților".
\par 14 Și căutând Domnul spre el, a zis: "Mergi cu această putere a ta și izbăvește pe Israel din mâinile Madianiților. Iată, Eu te trimit!"
\par 15 Atunci Ghedeon a zis: "Doamne, cum să izbăvesc eu pe Israel? Iată neamul meu este cel mai sărac din seminția lui Manase, iar eu sunt cel mai mic în casa tatălui meu". Domnul însă i-a zis:
\par 16 "Eu voi fi cu tine și tu vei bate pe Madianiți, ca pe un singur om".
\par 17 A zis Ghedeon către Dânsul: "De am aflat eu trecere în ochii Tăi, arată-mi un semn, ca să-mi dovedești cele ce-mi vorbești:
\par 18 Să nu Te duci de aici, până nu mă voi întoarce la Tine și-mi voi aduce darul meu și ți-l voi da". Și Domnul a zis: "Voi sta până te vei întoarce".
\par 19 Și s-a dus Ghedeon și a gătit un ied și azime din o efă de făină; carnea a pus-o într-un coș, iar zeama a turnat-o într-o oală și a dus-o la El sub stejar și I-a pus-o înainte.
\par 20 Și a zis către dânsul îngerul Domnului: "Ia carnea și azimile și pune-le pe piatra aceasta și toarnă zeama peste ele". Și a făcut Ghedeon așa.
\par 21 Atunci îngerul Domnului, întinzându-și vârful toiagului ce-l avea în mâna sa, s-a atins de carne și de azime; și a ieșit foc din piatră și a mistuit carnea și azimile; și îngerul Domnului s-a făcut nevăzut de la ochii lui.
\par 22 Și a cunoscut Ghedeon că acesta este îngerul Domnului, și a zis Ghedeon: "Vai de mine, Stăpâne Doamne, că am văzut pe îngerul Domnului față către față!"
\par 23 Zis-a Domnul: "Pace ție. Nu te teme, căci nu vei muri!"
\par 24 Și a făcut acolo Ghedeon un jertfelnic Domnului și l-a numit "Iahve-Șalom". Și se află acesta și astăzi în Ofra lui Abiezer.
\par 25 În noaptea aceea i-a zis Domnul: "Ia un vițel din cireada tatălui tău și un taur de șapte ani și sfărâmă jertfelnicul lui Baal pe care-l are tatăl tău și taie copacul cel sfânt de lângă el;
\par 26 Și zidește un jertfelnic în cinstea Domnului Dumnezeului tău, Care ți S-a arătat pe vârful stâncii acesteia; apoi ia taurul și-l adu ardere de tot pe lemnele copacului pe care ai să-l tai".
\par 27 Atunci Ghedeon a luat zece oameni dintre slugile sale și a făcut cum îi grăise Domnul. Și fiindcă se temea de casnicii tatălui său și de oamenii din cetate să facă acestea ziua, le-a făcut noaptea.
\par 28 Și când s-au sculat dimineața locuitorii cetății, au văzut jertfelnicul lui Baal dărâmat și copacul cel de lângă el tăiat și taurul adus ardere de tot pe jertfelnicul cel nou.
\par 29 Și ziceau unii către alții: "Cine oare a făcut acestea?" Iar după ce au cercetat și au întrebat, au zis: "Ghedeon, fiul lui Ioaș, a făcut acestea!"
\par 30 Atunci au zis locuitorii cetății către Ioaș: "Scoate pe fiul tău, că trebuie să moară, pentru că a dărâmat jertfelnicul lui Baal și a tăiat copacul cel de lângă el".
\par 31 Iar Ioaș a zis celor ce veniseră la dânsul: "Voi oare vreți să treceți de partea lui Baal? Vreri voi oare să-l apărați? Cine va trece de partea lui acela va fi dat morții, chiar în dimineața aceasta; dacă el este dumnezeu, să se apere singur pe sine pentru că i s-a stricat jertfelnicul".
\par 32 Din acea zi au început a numi pe Ghedeon, Ierubaal, pentru că ziceau: "Să se judece singur Baal cu dânsul, pentru că i-a stricat jertfelnicul".
\par 33 În timpul acesta toți Madianiții, Amaleciții și locuitorii Răsăritului s-au adunat împreună, au trecut râul și și-au așezat tabăra în valea Izreel.
\par 34 Atunci a cuprins Duhul Domnului pe Ghedeon și a trâmbițat acesta din trâmbiță și a fost chemată familia lui Abiezer să meargă cu dânsul.
\par 35 Apoi s-au trimis soli prin toată seminția lui Manase și aceasta a răspuns că merge cu dânsul. Și tot așa s-au trimis soli la Așer, la Zabulon și la Neftali și au venit și aceștia în întâmpinarea lor.
\par 36 Atunci a zis Ghedeon către Dumnezeu: "De vrei să izbăvești pe Israel prin mâna mea, cum zici,
\par 37 Apoi iată eu întind aici în arie lâna ce am tuns; și de va fi rouă numai pe lână, iar încolo peste tot locul uscăciune, atunci voi ști că vei izbăvi pe Israel prin mâna mea, cum ai zis".
\par 38 Și s-a făcut așa; și a doua zi s-a sculat Ghedeon dis-de-dimineață și s-a apucat să stoarcă lâna și a stors rouă din lână un vas plin de apă.
\par 39 Apoi iarăși a zis Ghedeon către Domnul: "Să nu Te mânii pe mine, dacă am să mai zic o dată și dacă am să mai fac numai o încercare cu lâna: să fie uscată numai lâna, iar peste tot locul să fie rouă".
\par 40 Și a făcut așa Dumnezeu în noaptea aceea: a fost uscăciune numai pe lână, iar peste tot locul a fost rouă.

\chapter{7}

\par 1 Atunci s-a sculat Ierubaal, adică Ghedeon, și tot poporul care era cu dânsul dis-de-dimineață și au tăbărât la En-Harod, iar tabăra Madianiților era spre miazănoapte de dânsul pe colina More cea din șes.
\par 2 Iar Domnul a zis către Ghedeon: "E prea mult popor cu tine; nu voi putea Eu să dau pe Madian în mâinile lor, ca să nu se mândrească Israel înaintea Mea și să nu zică: Mâna mea m-a izbăvit!
\par 3 De aceea grăiește în auzul poporului și zi: Cine este fricos și se teme, acela să se întoarcă și să se ducă înapoi din Muntele Galaad". Și s-au întors din popor douăzeci și două de mii și au rămas zece mii.
\par 4 Apoi a zis Domnul către Ghedeon: "Tot e prea mult popor; du-l la apă; acolo ți-l voi alege. și de care voi zice să meargă cu tine, acela să meargă cu tine, iar de care îți voi zice că nu trebuie să meargă cu tine, acela să nu meargă".
\par 5 Și a dus el poporul la apă, iar Domnul a zis către Ghedeon: "Cine va limpăi apa cu limba din pumni, cum limpăie câinele, pe acela să-l pui deoparte; de asemenea să pui deoparte și pe toți aceia care-și vor pleca genunchii și vor bea apă".
\par 6 Și a fost numărul celor ce au limpăit cu limba lor din pumni trei sute de oameni; iar tot celălalt popor s-a plecat pe genunchii săi să bea apă.
\par 7 Atunci a zis Domnul către Ghedeon: "Cu cei trei sute care au limpăit am să vă izbăvesc Eu și am să dau pe Madianiți în mâinile voastre, iar tot poporul celălalt să se ducă fiecare la locul său".
\par 8 Și au luat de la popor merindele și trâmbițele; apoi a dat Ghedeon drumul tuturor Israeliților pe la corturi și a oprit la sine pe cei trei sute de oameni, iar tabăra Madianiților era din jos de el, în vale.
\par 9 În noaptea aceea i-a zis Domnul: "Scoală și te coboară la tabără, că Eu o voi da în mâinile tale.
\par 10 Dacă însă te temi să te duci singur, atunci du-te la tabără tu și Pura, sluga ta,
\par 11 Și ai să auzi ce se grăiește și atunci au să se îmbărbăteze mâinile tale și ai să te duci în tabără". Și s-a dus el și Pura, sluga sa, până la cele dintâi străji ale taberei.
\par 12 Iar Madianiții și Amaleciții și toți locuitorii Răsăritului se așezaseră în vale atât de mulți, ca lăcustele; cămilele nu mai aveau număr și erau multe, ca nisipul de pe malurile mării.
\par 13 Ghedeon veni. Și iată unul povestea altuia un vis și zicea: "Am visat parcă o pâine rotundă de orz, ce se rostogolea prin tabăra madianită și, ajungând la un cort, a izbit în el așa de tare, încât el a căzut, s-a răsturnat și s-a desfăcut".
\par 14 Celălalt i-a răspuns: "Aceasta nu este alta decât sabia lui Ghedeon, fiul lui Ioaș israelitul; Dumnezeu a dat în mâna lui pe Madianiți și toată tabăra".
\par 15 Auzind povestirea visului și tălmăcirea lui, Ghedeon s-a închinat Domnului și s-a întors în tabăra israelită, zicând: "Sculați! Domnul a dat tabăra Madianiților în mâinile noastre".
\par 16 Apoi a împărțit pe cei trei sute de oameni în trei cete și le-a dat la toți în mâini trâmbițe și oale goale și în oale făclii.
\par 17 Și le-a zis: "Să vă uitați la mine și să faceți ce voi face eu; iată eu mă duc la tabără și ce voi face eu, să faceți și voi.
\par 18 Când eu și cei cu mine vom trâmbița, să trâmbițați și voi din trâmbițele voastre împrejurul întregii tabere și să strigați: Sabia Domnului și a lui Ghedeon!"
\par 19 Și s-a apropiat de tabără Ghedeon și cu el o sută de oameni, pe la începutul străjii de mijloc a nopții, și au deșteptat străjile și au trâmbițat din trâmbițe și au sfărâmat oalele pe care le aveau în mâini.
\par 20 Și au trâmbițat tustreile cete din trâmbițe și au spart oalele și țineau în mâna stângă făclia, iar în mâna dreaptă trâmbițele și trâmbițau din trâmbițe și strigau: "Sabia Domnului și a lui Ghedeon!"
\par 21 Și stăteau fiecare la locul său împrejurul taberei și au început cei din tabără a alerga în toată tabăra și a striga și au luat-o la fugă.
\par 22 Pe când cei trei sute de oameni sunau din trâmbițe, în toată tabăra a întors Domnul sabia unora asupra altora, și a fugit tabăra către Țerera până la Betșita și până la hotarele lui Abelmehola, aproape de Tabat.
\par 23 Atunci au fost chemați Israeliții din semințiile lui Neftali și Așer și din toată seminția lui Manase și au alergat după Madianiți.
\par 24 Iar Ghedeon a trimis soli în tot muntele lui Efraim să spună: "Ieșiți înaintea Madianiților și prindeți vadul înaintea lor până la Betbara și Iordan". Și s-au adunat toți Efraimiții și au prins vadul până la Betbara și Iordan.
\par 25 Și au prins pe cele două căpetenii ale Madianiților: pe Oreb și Zeeb; au ucis pe Oreb la Țur-Oreb, iar pe Zeeb la Iecheb-Zeeb; și au urmărit pe Madianiți; iar capetele lui Oreb și Zeeb le-au adus la Ghedeon, dincolo de Iordan.

\chapter{8}

\par 1 Zis-au Efraimiții către el: "De ce ai făcut așași nu ne-ai chemat când ai mers să te lupți cu Madianiții?" Și s-au certat strașnic cu el.
\par 2 Iar Ghedeon le-a răspuns: "Făcut-am eu oare ceva la fel cu ceea ce ați făcut voi? Nu e mai fericit oare Efraim că a cules toată via, decât Abiezer care s-a ales cu câțiva ciorchini?
\par 3 În mâinile voastre a dat Dumnezeu pe căpeteniile Madianiților Oreb și Zeeb și ce-am putut să fac eu asemenea cu ce ați făcut voi?" Atunci s-a liniștit duhul lor cel întărâtat împotriva lui, când le-a spus asemenea cuvinte.
\par 4 Apoi a venit Ghedeon la Iordan și a trecut și el și cei trei sute de oameni care erau cu dânsul și care obosiseră și flămânziseră, urmărind pe dușman.
\par 5 El a zis către locuitorii din Sucot: "Dați pâine oamenilor care merg cu mine, căci sunt obosiți și urmărim pe Zebah și pe Țalmuna, regii Madianiților".
\par 6 Iar căpeteniile din Sucot au răspuns: "Dar este oare mâna lui Zebah și Țalmuna în stăpânirea ta, ca să dăm pâine oștirii tale?"
\par 7 Atunci Ghedeon a zis: "Când va da Domnul pe Zebah și pe Țalmuna în mâna mea, am să scarpin trupul vostru cu spinii pustiului și cu mărăcini".
\par 8 După aceea s-a dus el la Penuel și a spus la fel locuitorilor lui; dar locuitorii din Penuel i-au răspuns la fel cum răspunseseră și cei din Sucot.
\par 9 Și a zis el și locuitorilor din Penuel: "Dacă mă voi întoarce biruitor, am să dărâm turnul acesta".
\par 10 Zebah și Țalmuna erau în Carcor și cu ei erau oștirile lor până la cincisprezece mii de oameni, toți cei ce mai rămăseseră din toată oștirea locuitorilor Răsăritului; căzuseră însă o sută douăzeci de mii de oameni purtători de sabie.
\par 11 Și s-a dus Ghedeon la cei ce trăiau în corturi la Răsărit de Nobah și de Iogbeha, și au lovit tabăra tocmai când erau mai fără grijă.
\par 12 Atunci Zebah și Țalmuna au fugit, iar el a alergat după dânșii și a prins pe amândoi regii Madianiților, pe Zebah și pe Țalmuna, și a pus toată tabăra în învălmășeală.
\par 13 Apoi s-a întors Ghedeon, fiul lui Ioaș, de la război de pe colina Heres.
\par 14 Și a prins un tânăr locuitor din Sucot și l-a întrebat și acesta i-a înșirat în scris pe căpeteniile și bătrânii Sucotului, care erau în număr de șaptezeci și șapte de oameni.
\par 15 Apoi a venit la locuitorii Sucotului și a zis: "Iată Zebah și Țalmuna, din pricina cărora ați râs de mine și mi-ați zis: Au doară mâna lui Zebah și Țalmuna e în stăpânirea ta, ca să dăm pâine oamenilor tăi celor obosiți?"
\par 16 După aceea a luat spini din pustiu și mărăcini și a pedepsit cu ei pe bătrânii cetății și pe locuitorii din Sucot.
\par 17 Și turnul din Penuel l-a dărâmat, iar pe locuitorii cetății i-a ucis.
\par 18 Și a zis către Zebah și Țalmuna: "Ce fel erau aceia pe care i-ați ucis voi în Tabor?" Zis-au ei: "Așa, cum ești și tu; fiecare avea înfățișarea unui fiu de rege".
\par 19 Iar Ghedeon a zis: "Aceia erau frații mei, fiii mamei mele! Viu este Domnul, de i-ați fi lăsat cu viață, eu nu v-aș ucide!"
\par 20 Apoi a zis către Ieter, întâiul său născut: "Scoală și-i ucide". Dar tânărul nu și-a scos sabia, căci s-a temut, pentru că era încă tânăr.
\par 21 Zis-au Zebah și Țalmuna: "Scoală tu și ne ucide, pentru că după cum este omul așa este și puterea lui!" Și s-a sculat Ghedeon și a ucis pe Zebah și Țalmuna și a luat frâiele de la gâtul cămilelor lor.
\par 22 După aceea au zis Israeliții către Ghedeon: "Domnește peste noi tu și fiul tău și fiul fiului tău, pentru că ne-ai izbăvit din mâinile Madianiților!"
\par 23 Iar Ghedeon le-a zis: "Nici eu nu voi domni peste voi, nici fiul meu nu va domni peste voi, ci Domnul să domnească peste voi!
\par 24 Dar am să vă rog și eu un lucru, a adăugat Ghedeon, să-mi dea fiecare din voi câte un cercel din prăzile voastre; căci vrăjmașii aveau mulți cercei de aur, pentru că erau Ismaeliți".
\par 25 Ei au zis: "Îți vom da". Și au întins o manta și au aruncat acolo fiecare câte un cercel din prada sa.
\par 26 Și greutatea cerceilor de aur pe care i-a cerut el a fost o mie șapte sute de sicli de aur, afară de verigi, de nasturi și de hainele de purpură de pe cei doi regi ai Madianiților și afară de lanțurile lor de aur care erau la gâtul cămilelor lor.
\par 27 Din acestea a făcut Ghedeon un efod și l-a pus în cetatea sa, în Ofra; și a fost aceasta pricină de păcat pentru tot Israelul și cursă pentru Ghedeon și pentru toată casa lui.
\par 28 Astfel s-au supus Madianiții înaintea fiilor lui Israel și nu s-au mai apucat să-și ridice capul și s-a odihnit țara patruzeci de ani, în zilele lui Ghedeon.
\par 29 Apoi s-a dus Ierubaal, fiul lui Ioaș, și a trăit în casa sa.
\par 30 Și a avut Ghedeon șaptezeci de fii care au răsărit din coapsele lui, căci el a avut femei multe.
\par 31 De asemenea i-a născut un fiu și concubina sa care trăia în Sichem și el i-a pus numele Abimelec.
\par 32 Apoi a murit Ghedeon, fiul lui Ioaș, la bătrâneți adânci, și a fost înmormântat în mormântul tatălui său Ioaș, în Ofra lui Abiezer.
\par 33 După ce a murit Ghedeon, fiii lui Israel au început iarăși a păcătui pe urma baalilor și și-au așezat ca dumnezeu pe Baal-Berit;
\par 34 Nu și-au mai adus aminte fiii lui Israel de Domnul Dumnezeul lor, Care îi izbăvise din mâinile tuturor vrăjmașilor care îi înconjurau.
\par 35 Casei lui Ierubaal, adică a lui Ghedeon, nu i-au dat nici o cinste pentru toate binefacerile pe care acesta le făcuse întregului Israel.

\chapter{9}

\par 1 În vremea aceea Abimelec, fiul lui Ierubaal, s-a dus la Sichem, la frații mamei sale, și a grăit cu el Și cu tot neamul tatălui mamei sale și a zis:
\par 2 "Șoptiți la toți locuitorii din Sichem: Cum e mai bine pentru voi: să domnească peste voi toți cei șaptezeci de fii ai lui Ierubaal sau să domnească numai unul? Și amintiți-vă că eu sunt osul vostru și carnea voastră!"
\par 3 Și au șoptit frații mamei sale din partea lui toate cuvintele acestea locuitorilor din Sichem. Și s-a înduplecat inima acestora pentru Abimelec, căci își ziceau așa: "E fratele nostru!"
\par 4 Și i-au dat șaptezeci de sicli de argint din casa lui Baal-Berit, iar Abimelec și-a tocmit cu ei oameni răi și fără căpătâi care s-au și dus cu el.
\par 5 Apoi a venit la casa tatălui său în Ofra și a ucis pe frații săi, pe cei șaptezeci de fii ai lui Ierubaal, pe o piatră, rămânând numai Iotam, fiul cel mai mic al lui Ierubaal, pentru că s-a ascuns.
\par 6 După aceea s-au adunat toți locuitorii Sichemului și toată casa lui Milo și s-au dus de au pus rege pe Abimelec la stejarul cel de lângă Sichem.
\par 7 Iar dacă s-a spus acestea lui Iotam, acesta s-a dus și a stat pe vârful muntelui Garizim și, ridicându-și glasul, a strigat și a zis: "Ascultați-mă, locuitori ai Sichemului, și Dumnezeu să vă asculte!
\par 8 S-au dus odată copacii să-și ungă împărat peste ei. Și au zis către măslin: Domnește peste noi!
\par 9 Iar măslinul a zis: Lăsa-voi eu oare grăsimea mea, cu care se cinstește Dumnezeu și oamenii se mândresc și mă voi duce să umblu prin copaci?
\par 10 Atunci copacii au zis către smochin: Vino tu și domnește peste noi!
\par 11 Dar și smochinul a răspuns  Să-mi las eu oare dulceața mea și fructul meu cel bun și să mă duc să cârmuiesc copacii?
\par 12 Apoi au zis copacii către vița de vie: Vino tu de domnește paste noi!
\par 13 Și vița de vie a zis către ei: Cum să-mi las eu mustul meu care veselește pe Dumnezeu și pe oameni și să mă duc să cârmuiesc copacii?
\par 14 În cele din urmă au zis toți copacii către un spin: Vino tu și domnește peste noi!
\par 15 Iar spinul a zis către copaci: Dacă voi mă puneți cu adevărat împărat peste voi, atunci veniți și vă odihniți sub umbra mea; iar de nu, atunci va ieși foc din spini și va arde cedrii Libanului.
\par 16 Așadar luați seama: După dreptate și după adevăr v-ați purtat voi, când ați pus rege pe Abimelec? Și bine ați făcut ce aii făcut cu Ierubaal și cu casa lui? Și v-ați purtat oare potrivit Cu binefacerile lui?
\par 17 Tatăl meu a luptat pentru voi, fără să-și cruțe viața, și v-a izbăvit din mâna Madianiților;
\par 18 Iar voi v-ați sculat acum împotriva casei tatălui meu și ați ucis pe cei șaptezeci de feciori ai tatălui meu pe o piatră și ați pus rege peste locuitorii Sichemului pe Abimelec, fiul unei roabe a lui, pentru că e fratele vostru.
\par 19 Dacă voi v-ați purtat după adevăr și după dreptate cu Ierubaal și cu casa lui, atunci să fie asupra voastră binecuvântare și să vă bucurați de Abimelec și să se bucure și el de voi!
\par 20 Dacă insă nu, atunci să iasă foc din Abimelec și să ardă pe locuitorii Sichemului și toată casa lui Milo; să iasă foc din locuitorii Sichemului și din casa lui Milo și să ardă pe Abimalec".
\par 21 Apoi a fugit Iotam și s-a făcut nevăzut și s-a dus la Beer și a trăit acolo, ascunzându-se de fratele său Abimelec.
\par 22 Iar Abimelec a domnit paste Israel trei ani.
\par 23 După aceea a trimis Dumnezeu un duh rău între Abimelec și între locuitorii Sichemului, nemaivoind locuitorii din Sichem să se supună lui Abimelec;
\par 24 Ca astfel să vină răzbunarea pentru cei șaptezeci de fii ai iui Ierubaal și sângele lor să se întoarcă asupra lui Abimelec, fratele lor, care-i ucisese, și asupra locuitorilor Sichemului care au îmbărbătat mâna lui ca să-și ucidă frații.
\par 25 Și au trimis locuitorii Sichemului împotriva lui oameni la pândă pe vârfurile munților, ca să prade pe oricine va trece pe lângă ei pe cale. Și s-a spus aceasta lui Abimelec.
\par 26 Atunci a venit și Gaal, fiul lui Ebed, cu frații săi, la Sichem și au umblat ei prin Sichem; iar locuitorii Sichemului s-au încrezut în el.
\par 27 Apoi au ieșit ei în țarină și au cules viile, au stors strugurii, au făcut praznic și s-au dus la casa dumnezeului lor, unde au mâncat și au băut și au blestemat pe Abimelec.
\par 28 Gaal însă, fiul lui Ebed, zicea: "Cine este Abimelec și ce este Sichemul, ca să-i slujim? Nu este el, oare, fiul lui Ierubaal, și căpetenia cea mai de seamă a Sichemului nu este oare Zebul? Să slujiți mai bine urmașilor lui Hemor, tatăl lui Sichem, iar aceluia pentru ce să-i slujim?
\par 29 De mi-ar da cineva poporul acesta pe mâna mea, eu aș alunga pe Abimelec". Atunci s-a zis lui Abimelec: "Înmulțește-ți oștirea și ieși!
\par 30 Iar Zebul, căpetenia cetății, a aflat ce zisese Gaal, fiul lui Ebed, și s-a aprins de mânie.
\par 31 Apoi a trimis el cu vicleșug soli la Abimelec, ca să-i spună: "Iată Gaal, fiul lui Ebed, și frații lui au venit în Sichem și ațâță cetatea împotriva ta.
\par 32 Scoală dar la noapte, tu și poporul care e eu tine, și stai de pândă în câmp;
\par 33 Iar dimineața, la răsăritul soarelui, scoală repede și înaintează spre cetate; și când ei și poporul ce este cu ei var ieși la tine, atunci să faci cu ei ce se va pricepe mâna ta".
\par 34 S-a sculat deci Abimelec noaptea și tot poporul ce era cu dânsul și au stat da pândă la Sichem patru cete.
\par 35 Iar dimineața, Gaal, fiul lui Ebed, a ieșit și a stat în poarta cetății. Atunci s-a sculat Abimelec și poporul ce era cu el în ascunzătoare.
\par 36 Gaal însă, văzând poporul, a zis câtre Zebul: "Iată poporul se coboară de pe vârful munților". Iar Zebul i-a răspuns: "Umbrele munților ți se par oameni".
\par 37 Și a grăit iarăși Gaal și a zis: "Iată poporul se coboară de pe înălțime și o ceată vine de la stejarul Meanim".
\par 38 A zis atunci Zebul: "Unde sunt buzele tale care ziceau: "Cine este Abimelec, ca să-i slujim lui? Acesta este poporul pe care tu l-ai nesocotit. Ieși acum și te luptă cu dânsul!"
\par 39 Și s-a dus Gaal în fruntea locuitorilor Sichemului și s-a luptat cu Abimelec.
\par 40 Și s-a năpustit Abimelec asupra lui și el a fugit de dânsul și au căzut mulți uciși până la porțile cetății.
\par 41 Abimelec însă a rămas în Aruma; iar pe Gaal și pe frații lui i-a alungat Zebul, ca să nu mai locuiască Sichem.
\par 42 A doua zi a ieșit poporul la câmp și au spus despre acestea lui Abimelec.
\par 43 Iar acesta și-a luat poporul său și l-a împărțit în trei cete și l-a pus la pândă în câmp. Și văzând că a ieșit popor din cetate, s-a ridicat asupra acelora și i-a ucis.
\par 44 Pe când Abimelec și cetele ce erau cu dânsul s-au apropiat și s-au oprit la porțile cetății, celelalte două cete, tăbărând asupra tuturor celor ce erau În câmp, i-au ucis.
\par 45 Și s-a luptat Abimelec cu cetatea toată ziua aceea, a luat cetatea, a ucis poporul care era în ea și a dărâmat cetatea și a presărat locul ei cu sare.
\par 46 Auzind de acestea, toți acei ce erau în turnul Sichemului s-au dus în turnul capiștei lui Baal-Berit.
\par 47 Dar i s-a spus lui Abimelec că s-au adunat acolo toți cei ce fuseseră în turnul Sichemului.
\par 48 Atunci Abimelec s-a dus în muntele Țalmon, el și tot poporul ce era eu dânsul; a luat Abimelec cu sine topoare și a tăiat lemne din pădure și le-a pus pe umăr și a zis către popor: "Ați văzut ce am făcut eu? Faceți repede și voi ceea ce am făcut eu!"
\par 49 Și a tăiat fiecare din popor lemne și s-au dus toți cu Abimelec și le-au pus sub turn și au aprins cu ele turnul; și au murit toți cei ce erau în turnul Sichemului, aproape o mie de bărbați și de femei.
\par 50 După aceea s-a dus Abimelec la Teveț și a împresurat Tevețul și l-a luat.
\par 51 Și era în mijlocul cetății un turn întărit și au fugit acolo toți bărbații și femeile și toți oamenii din cetate; și l-au încuiat și s-au suit pe acoperișul turnului.
\par 52 Abimelec însă a venit la turn și l-a împresurat și s-a apropiat de ușa turnului ca să-i dea foc.
\par 53 Atunci o femeie a aruncat o bucată de piatră de râșniță în capul lui Abimelec și i-a spart capul.
\par 54 Abimelec a chemat îndată un tânăr, care era purtătorul de arme al său, și i-a zis: "Scoate-ți sabia și mă ucide, ca să nu zică despre mine: A fost ucis de o femeie". Și l-a străpuns tânărul acela și a murit.
\par 55 Când au văzut Israeliții că a murit Abimelec, s-a dus fiecare la locul său.
\par 56 Așa a plătit Dumnezeu lui Abimelec, pentru nelegiuirea pe care el o făptuise față de tatăl său, ucigând pe cei șaptezeci de frați ai săi.
\par 57 Și toate nelegiuirile locuitorilor Sichemului le-a întors Dumnezeu asupra capului lor. Și așa i-a ajuns blestemul lui Iotam, fiul lui Ierubaal.

\chapter{10}

\par 1 După Abimelec s-a ridicat ca să izbăvească pe Israel Tola, fiul lui Pua, fiul lui Dodo, din seminția lui Isahar. Acesta trăia în Șamir, pe muntele lui Efraim.
\par 2 Și a fost el judecătorul lui Israel douăzeci și trei de ani și; murind, a fost îngropat în Șamir.
\par 3 După dânsul s-a sculat Iair din Galaad și a fost judecător lui Israel douăzeci și doi de ani.
\par 4 Acesta a avut treizeci și doi de fii, care călăreau pe treizeci și doi de asini și aveau treizeci și două de cetăți.
\par 5 Murind Iair, a fost îngropat în Camon.
\par 6 Dar fiii lui Israel au făcut iarăși rele înaintea ochilor Domnului și au slujit baalilor și astartelor și dumnezeilor Amoreilor, dumnezeilor Sidonului, dumnezeilor Amoniților, dumnezeilor Moabiților și dumnezeilor Filistenilor, iar pe Domnul L-au părăsit și nu L-au slujit.
\par 7 Atunci s-a aprins mânia Domnului asupra lui Israel, și l-a dat în mâinile Filistenilor și în mâinile Amoniților.
\par 8 Aceștia au strâmtorat ți au chinuit pe fiii lui Israel din anul acela optsprezece ani în șir, adică pe toți fiii lui Israel de dincolo de Iordan, din țara Amoreilor, care este în Galaad.
\par 9 Iar Amoniții au trecut Iordanul, ca să se războiască cu Iuda, cu Veniamin și cu casa lui Efraim. Așa că fiii lui Israel erau foarte strâmtorați.
\par 10 Atunci au strigat fiii lui Israel către Domnul și au zis: "Greșit-am înaintea Ta, pentru că am părăsit pe Dumnezeul nostru și am slujit baalilor".
\par 11 Domnul însă a zis către fiii lui Israel: "Nu v-au împilat oare Egiptenii, Amoreii, Amoniții și Filistenii,
\par 12 Sidonienii, Amaleciții și Moabiții, și când ați strigat către Mine, nu v-am izbăvit Eu oare din mâinile lor?
\par 13 Dar voi M-ați părăsit iarăși și v-ați apucat să slujiți la alți dumnezei. De aceea nu vă voi mai izbăvi.
\par 14 Mergeți și strigați către dumnezeii pe care vi i-ați ales; să vă izbăvească aceia la vreme de necaz!"
\par 15 Iar fiii lui Israel au zis către Domnul: "Greșit-am! Fă cu noi cum vei crede că e mai bine, numai izbăvește-ne și acum".
\par 16 Și au lepădat de la ei pe dumnezeii cei străini și au început să slujească numai Domnului. Și S-a îndurat Domnul de suferințele lui Israel.
\par 17 Amoniții însă s-au adunat și și-au așezat tabăra în Galaad. S-au adunat de asemenea și fiii lui Israel și au tăbărât la Mițpa.
\par 18 Atunci poporul și căpeteniile Galaadului au zis unii către alții: "Cine va începe lupta contra Amoniților acela va fi căpetenie peste toți locuitorii Galaadului".

\chapter{11}

\par 1 Ieftae Galaaditul era un luptător viteaz. Acesta era fiul unei desfrânate care născuse lui Galaad pe Ieftae.
\par 2 Dar și soția lui Galaad i-a născut acestuia fii. Iar dacă s-au făcut mari, fiii soției au izgonit pe Ieftae, zicându-i: "Tu nu ești moștenitor în casa tatălui nostru, pentru că ești feciorul alte femei".
\par 3 Atunci Ieftae a fugit de frații săi și a trăit în ținutul Tob. Acolo s-au adunat împrejurul lui Ieftae oameni fără căpătâi și umblau cu dânsul.
\par 4 După câtva timp Amoniții s-au ridicat cu război împotriva lui Israel.
\par 5 Iar în timpul războiului Amoniților cu Israeliții, au venit bătrânii Galaadului să ia pe Ieftae din ținutul Tob,
\par 6 Și au zis către Ieftae: "Vino să ne fii căpetenie și te luptă cu Amoniții".
\par 7 Ieftae însă a zis către bătrânii Galaadului: "Oare nu m-ați urât voi și m-ați alungat din casa tatălui meu? La ce ați venit la mine acum, când sunteți la necaz?"
\par 8 Zis-au bătrânii Galaadului către Ieftae: "De aceea am venit acum la tine, ca să mergi cu noi, să te lupți cu Amoniții și să ne fii căpetenie nouă, tuturor locuitorilor Galaadului".
\par 9 Iar Ieftae a zis către bătrânii Galaadului: "Dacă mă luați înapoi, ca să mă lupt cu Amoniții, și dacă Domnul îmi va da mie izbândă, voi mai rămâne eu, oare, căpetenie la voi?"
\par 10 Atunci au răspuns bătrânii Galaadului către Ieftae: "Domnul să fie martor între noi că vom face cum vei zice tu!"
\par 11 Și s-a dus Ieftae cu bătrânii Galaadului și poporul l-a pus căpetenie și povățuitor al său. Și a rostit Ieftae toate cuvintele sale înaintea feței Domnului în Mițpa.
\par 12 Apoi a trimis Ieftae soli la regele Amoniților să-i spună: "Ce ai cu mine de ai venit la mine să te războiești pe pământul meu?"
\par 13 Iar regele Amoniților a răspuns solilor lui Ieftae: "Israel, când venea din Egipt, a luat pământul meu de la Arnon până la Iaboc și Iordan. Întoarce-mi-l dară cu pace și mă voi retrage".
\par 14 Și dacă s-au întors solii la Ieftae, Ieftae a trimis a doua oară soli la regele Amoniților,
\par 15 Ca să-i spună: "Așa zice Ieftae: Israel n-a luat pământul Moabiților, nici pământul Amoniților;
\par 16 Căci, când a venit din Egipt, Israel s-a dus în pustiu către Marea Roșie și apoi a venit la Cadeș.
\par 17 De acolo a trimis Israel la regele Edomului soli să-i spună: "Lasă-mă să trec prin țara ta". Dar regele Edomului n-a voit să audă. Și a trimis el și la regele Moabului, dar nici acela n-a îngăduit. De aceea Israel a rămas la Cadeș.
\par 18 Apoi a plecat în pustiu și a ocolit pământul Edomului și pământul Moabului, ajungând la răsăritul lui. Atunci au tăbărât dincolo de Arnon, dar n-au intrat în hotarele Moabului, căci Arnonul este hotarul Moabului.
\par 19 De acolo a trimis Israel soli la Sihon, regele Amoreilor, regele Heșbonului și a zis Israel către el: "Dă-ne voie să trecem prin țara ta la locul nostru!"
\par 20 Dar Sihon nu s-a învoit să dea drumul lui Israel prin hotarele sale și a adunat Sihon tot poporul său și a tăbărât în Iahța și s-a bătut cu Israel.
\par 21 Și a dat Domnul Dumnezeul lui Israel pe Sihon și tot poporul lui în mâinile lui Israel și acesta i-a ucis. Apoi a luat Israel de moștenire toată țara Amoreilor, care locuiau în țara aceea.
\par 22 Și atunci au primit ei de moștenire toate hotarele Amoreilor de la Arnon până la Iaboc și de la pustie până la Iordan.
\par 23 Și așa Domnul Dumnezeul lui Israel a izgonit pe Amorei de la fața poporului Său Israel și tu voiești acum să-i iei moștenirea lui?
\par 24 Nu stăpânești tu oare ceea ce ti-a dat ție Chemoș, dumnezeul tău? Și noi stăpânim de asemenea ceea ce ne-a dat de moștenire Domnul Dumnezeul nostru.
\par 25 Oare tu ești mai bun decât Balac, fiul lui Sefor, regele Moabiților? S-a certat cu el Israel, sau s-a luptat cu el?
\par 26 Israel trăiește acum de mai bine de trei sute de ani în Heșbon și în cetățile care țin de el și în Aroer și în toate împrejurimile lui și în toate cetățile din apropierea Arnonului; de ce nu le-ați luat voi în vremea aceea?
\par 27 Eu însă nu sunt vinovat față de tine; dar tu-mi faci un rău, venind asupra mea cu război. Domnul să fie judecător între fiii lui Israel și Amoniți!"
\par 28 Dar regele Amoniților n-a ținut seamă de cuvintele lui Ieftae, cu care îi trimisese acesta pe soli la el.
\par 29 Atunci a fost peste Ieftae Duhul Domnului și a străbătut Ieftae pământul Galaadului și al lui Manase, apoi a ajuns până la Mițpa Galaadului și de la Mițpa Galaadului a plecat asupra Amoniților.
\par 30 În acel timp a făcut Ieftae făgăduință Domnului și a zis: "De vei da pe Amoniți în mâinile mele,
\par 31 Când mă voi întoarce biruitor de la Amoniți, oricine va ieși din porțile casei mele în întâmpinarea mea va fi afierosit Domnului și-l voi aduce ardere de tot".
\par 32 Apoi a venit Ieftae la Amoniți să se bată cu ei și i-a dat Domnul în mâinile lui.
\par 33 Și i-a bătut cumplit de la Aroer până spre Minit în douăzeci de cetăți și până la Abel-Cheramim și au fost umiliți Amoniții în fața fiilor lui Israel.
\par 34 După aceea a venit Ieftae în la casa sa și iată fiica sa i-a ieșit în întâmpinare cu timpane și jocuri; aceasta era singurul lui copil, căci el nu mai avea nici băieți, nici fete.
\par 35 Și când a văzut-o el, și-a sfâșiat haina și a zis: "Ah, fiica mea! Tu m-ai răpus și ești dintre cei ce-mi tulbură biruința. Eu mi-am deschis gura pentru tine înaintea Domnului și nu mă voi putea lepăda!"
\par 36 Iar ea a zis către el: "Tatăl meu, dacă tu ți-ai deschis gura pentru mine înaintea Domnului, fă cu mine ceea ce a rostit gura ta, de vreme ce Domnul a săvârșit prin tine răzbunarea împotriva Amoniților, vrăjmașii tăi!"
\par 37 Apoi a zis iarăși către tatăl său: "Iartă numai ce să-mi faci: Lasă-mă două luni, să mă duc să mă sui pe munte și să-mi plâng fecioria cu prietenele mele!"
\par 38 Atunci el a zis: "Du-te!" Și a lăsat-o două luni. Și s-a dus cu prietenele sale și și-a plâns fecioria în munți.
\par 39 Apoi după trecerea celor două luni ea s-a întors la tatăl său și acesta a făcut cu ea cum făgăduise; și ea n-a cunoscut bărbat.
\par 40 Și s-a făcut obicei în Israel, ca în fiecare an fiicele lui Israel să meargă să plângă pe fata lui Ieftae Galaaditeanul patru zile pe an.

\chapter{12}

\par 1 După aceea s-au adunat Efraimiții și au purces spre Țafon și au zis către Ieftae: "Pentru ce te-ai dus să te bați cu Amoniții, iar pe noi nu ne-ai chemat cu tine? Vom arde dar casa ta cu foc, împreună cu tine".
\par 2 Iar Ieftae a zis: "Eu și poporul meu am avut cu Amoniții ceartă mare și eu v-am chemat, dar voi nu m-ați scăpat din mâinile lor.
\par 3 Văzând însă că nu este nici un izbăvitor, mi-am pus viața în primejdie și m-am dus împotriva Amoniților și Domnul i-a dat în mâinile mele. De ce dar ați venit să vă bateți cu mine?"
\par 4 Atunci a adunat Ieftae toți oamenii din Galaad și s-a bătut cu Efraimiții și au bătut locuitorii Galaadului pe Efraimiți, zicând: "Voi sunteți niște fugari din Efraim, Galaadul însă e între Efraim și Manase".
\par 5 Și au luat Galaaditenii vadul Iordanului de la Efraimiți și când vreunul din Efraimiți zicea: "Îngăduie-mi să trec", atunci oamenii din Galaad îi răspundeau: "Nu cumva ești Efraimit?" Acela răspundea: "Nu!"
\par 6 Ei însă îi ziceau: "Zi: Șibbolet"; el însă zicea: "Sibbolet", că nu putea zice altfel. Atunci ei îl luau și-l junghiau acolo la vadul Iordanului. Și au căzut în vremea aceea din Efraimiți patruzeci și două de mii.
\par 7 Și a fost Ieftae judecător în Israel șase ani; apoi a murit Ieftae Galaaditeanul și a fost îngropat în unul din orașele Galaadului.
\par 8 După el a fost judecător în Israel Ibțan din Betleem.
\par 9 Acesta a avut treizeci de feciori și treizeci de fete a dat el din casa sa în căsătorie, iar treizeci de fete a luat de afară pentru fiii săi și a fost judecător în Israel șapte ani.
\par 10 Apoi a murit Ibțan și a fost îngropat în Betleem.
\par 11 După dânsul a fost judecător în Israel Elon Zabuloneanul și a judecat pe Israel zece ani.
\par 12 Apoi a murit Elon Zabuloneanul și a fost înmormântat la Aialon, în pământul lui Zabulon.
\par 13 După el a fost judecător în Israel Abdon, fiul lui Hilel Piratoneanul.
\par 14 Acesta a avut patruzeci de fii și treizeci de nepoți care călăreau pe șaptezeci de mânji de asin și a judecat pe Israel opt ani.
\par 15 Apoi a murit Abdon, fiul lui Hilel Piratoneanul și a fost îngropat în Piraton, în pământul lui Efraim, pe muntele lui Amalec.

\chapter{13}

\par 1 Și fiii lui Israel au făcut iarăși rele înaintea ochilor Domnului și i-a dat Domnul în mâinile Filistenilor pentru patruzeci de ani.
\par 2 Era însă în vremea aceea un om de la țara, din seminția lui Dan, cu numele Manoe și femeia lui era stearpă și nu năștea.
\par 3 Udată însă s-a arătat îngerul Domnului femeii și i-a zis: "Iată tu ești stearpă și nu naști; dar vei zămisli și vei naște fiu.
\par 4 Păzește-te dar, să nu bei vin, nici sicheră și nimic necurat să nu mănânci;
\par 5 Că iată ai să zămislești și al să naști un fiu; și nu se va atinge briciul de capul lui, pentru că pruncul acesta va fi chiar din pântecele mamei sale nazireu al lui Dumnezeu și va începe să izbăvească pe Israel din mâna Filistenilor".
\par 6 Și a venit femeia și a spus bărbatului său, zicând: "A venit la mine un om al lui Dumnezeu, a cărui înfățișare era ca înfățișarea unui înger al lui Dumnezeu, foarte luminos; nici eu nu l-am întrebat de unde este și nici el nu mi-a spus numele său;
\par 7 Dar mi-a zis: Iată ai să zămislești și ai să naști un fiu; așadar să nu bei vin și sicheră și să nu mănânci nimic necurat, căci copilul chiar din pântecele mamei și până la moarte va fi nazireu al lui Dumnezeu".
\par 8 Atunci Manoe s-a rugat Domnului și a zis: "Doamne, fă să vină iarăși pe la noi omul lui Dumnezeu pe care l-ai trimis Tu, și să ne învețe ce să facem cu copilul care se va naște!"
\par 9 Și a ascultat Dumnezeu glasul lui Manoe și a venit îngerul iarăși la femeie, când era la câmp, însă Manoe, bărbatul ei, nu era cu dânsa.
\par 10 Dar femeia a alergat îndată și a vestit pe bărbatul său, zicându-i: "Iată mi s-a arătat omul cel ce a venit atunci la mine".
\par 11 Și s-a sculat Manoe și s-a dus cu femeia sa și a venit la omul acela și a zis către el: "Tu, oare, ești omul acela care ai vorbit cu femeia?" Iar îngerul i-a răspuns: "Eu!"
\par 12 Și a zis Manoe: "Așadar, dacă se va împlini cuvântul tău, cum să ne purtăm cu copilul acesta și ce să facem cu el?"
\par 13 Iar îngerul a zis: "Să se păzească el de toate cele ce am spus eu femeii;
\par 14 Să nu mănânce nimic din câte rodește vița de vie; să nu bea vin, nici sicheră și să nu mănânce nimic necurat și să păzească toate câte i-am poruncit ei".
\par 15 Atunci Manoe a zis: "Îngăduie-ne să te oprim până vom găti un ied".
\par 16 Iar îngerul a zis către Manoe: "Deși mă vei opri, eu nu voi mânca pâinea ta; dar de voiești să faci ardere de tot Domnului, atunci adu-o". Și nu știa Manoe că acesta e îngerul Domnului.
\par 17 Și a zis Manoe către îngerul Domnului: "Cum îți este numele? Ca să te mărim, când se va împlini cuvântul tău".
\par 18 Zis-a îngerul către el: "La ce mă întrebi tu de numele meu? Că el este minunata".
\par 19 Atunci a luat Manoe un ied și prinos de pâine și le-a adus Domnului pe o stâncă. Și a făcut acela minunea pe care au văzut-o Manoe și femeia sa.
\par 20 Când a început a se înălța flacăra de la jertfelnic spre cer, îngerul Domnului s-a ridicat cu flacăra de pe jertfelnic. Văzând aceasta, Manoe și femeia lui au căzut cu fața la pământ.
\par 21 Și s-a făcut nevăzut îngerul Domnului de Manoe și de femeia lui. Atunci Manoe a înțeles că acela fusese îngerul Domnului.
\par 22 Și a zis Manoe către femeia sa: "De bună seamă avem să murim, căci am văzut pe Dumnezeu!"
\par 23 Iar femeia lui i-a zis: "Dacă Domnul ar voi să ne omoare, n-ar fi primit din mâinile noastre arderea de tot și prinosul de pâine și nu ne-ar fi arătat toate acelea și nu ne-ar fi descoperit acum aceasta".
\par 24 Și a născut femeia un fiu și i-au pus numele Samson. Și a crescut copilul și l-a binecuvântat Domnul.
\par 25 Și a început Duhul Domnului să lucreze prin el în tabăra lui Dan, între Țora și Eștaol.

\chapter{14}

\par 1 În vremea aceea s-a dus Samson la Timna și a văzut în Timna o femeie din fiicele Filistenilor și aceasta i-a plăcut.
\par 2 Și s-a dus și a spus el tatălui său și mamei sale și a zis: "Am văzut în Timna o femeie din fiicele Filistenilor; luați-mi-o mie de soție!"
\par 3 Iar tatăl său și mama sa i-au răspuns: "Au doară nu se găsesc femei printre fiicele fraților tăi și în tot poporul meu, de te duci să-ți iei soție de la Filistenii cei netăiați împrejur?" A zis Samson către tatăl său: "Ia-mi-o pe aceea, pentru că mi-a plăcut!"
\par 4 Și nu știau tatăl său și mama sa că aceasta este de la Domnul și că el caută prilej să se răzbune pe Filisteni. Căci în vremea aceea Filistenii domneau peste Israel.
\par 5 Deci s-a dus Samson cu tatăl său și cu mama sa la Timna; iar când s-au apropiat de viile Timnei, iată un leu tânăr venea răcnind înaintea lor.
\par 6 Atunci s-a coborât peste el Duhul Domnului și el a sfâșiat leul ca pe un ied; și nu avea nimic în mână. și n-a spus tatălui său și mamei sale ce făcuse.
\par 7 Și a venit și a vorbit cu femeia și aceasta a plăcut lui Samson.
\par 8 Iar după câteva zile s-a dus el iarăși ca să o ia și s-a abătut să vadă trupul leului și iată un roi de albine și miere în trupul leului.
\par 9 Și a luat el fagurele în mână și s-a dus și a mâncat pe cale; iar dacă a venit la tatăl său și la mama sa, le-a dat și lor de au mâncat; dar nu le-a spus că a luat fagurele acesta din trupul leului celui mort.
\par 10 Apoi a mers tatăl său la femeie și a făcut acolo Samson ospăț de șapte zile, cum au obiceiul să facă mirii.
\par 11 Și când l-au văzut cei de acolo, au ales treizeci de nuntași care să fie împrejurul lui.
\par 12 Iar Samson a zis către ei: "Am să vă spun o ghicitoare și dacă mi-o veți ghici în cele șapte zile ale ospățului și mi-o veți dezlega, vă voi da treizeci de cămăși și treizeci de rânduri de haine.
\par 13 Iar dacă nu veți putea s-o ghiciți, atunci să-mi dați voi mie treizeci de cămăși și treizeci de rânduri de haine". Au zis aceia: "Spune ghicitoarea ta, ca s-o auzim".
\par 14 Atunci le-a zis: "Din cel ce mănâncă a ieșit mâncare, și din cel tare a ieșit dulceață". Și n-au putut să dezlege ghicitoarea în trei zile.
\par 15 Iar în ziua a șaptea au zis aceia către femeia lui Samson: "Ademenește pe bărbatul tău să dezlege ghicitoarea; altfel te vom arde cu foc pe tine și casa tatălui tău; ne-ați chemat, oare, ca să ne jefuiți?"
\par 16 Și a plâns femeia lui Samson înaintea lui, zicând: "Tu mă urăști și nu mă iubești; ai dat o ghicitoare fiilor poporului meu, iar mie nu mi-o dezlegi". Și a zis el către ea: "Eu n-am dezlegat-o tatălui meu și mamei mele și să ți-o dezleg ție?"
\par 17 Și a plâns ea înaintea lui șapte zile, cât a ținut ospățul la ei. În sfârșit în ziua a șaptea i-a dezlegat-o căci ea îl ruga stăruitor.
\par 18 Iar ea a spus dezlegarea ghicitorii fiilor poporului său. Și iată în ziua a șaptea, înainte de răsăritul soarelui, au zis către oamenii cetății: "Ce e mai dulce ca mierea și ce e mai tare ca leul?" Și el le-a zis: "De nu ați fi arat cu juninca mea, ghicitoarea mea n-o mai ghiceați voi".
\par 19 Atunci s-a coborât peste el Duhul Domnului și s-a dus în Ascalon și, ucigând acolo treizeci de oameni, a dezbrăcat de pe ei hainele și a dat rândurile de haine celor ce au ghicit ghicitoarea sa. Și s-a aprins mânia lui și s-a dus la casa tatălui său.
\par 20 Iar femeia lui Samson s-a măritat cu unul din nuntașii de la nunta sa, care au fost împrejurul lui.

\chapter{15}

\par 1 Peste câteva zile, în timpul seceratului grâului, a venit Samson să se vadă cu femeia sa, aducând cu sine un ied. Iar când a zis: "Mă duc la femeia mea în odaia de dormit", tatăl ei nu l-a lăsat să intre.
\par 2 Și a zis tatăl ei: "Eu am socotit că ai urât-o și am măritat-o cu un prieten al tău; iată sora ei mai mică e mai frumoasă decât ea; să fie aceasta în locul aceleia".
\par 3 Samson însă le-a zis: "De acum eu voi fi drept înaintea Filistenilor, dacă mă voi apuca să le fac rău".
\par 4 Apoi Samson s-a dus și a prins trei sute de vulpi, a luat făclii, a legat câte două vulpi de coadă și intre ele câte o făclie;
\par 5 După aceea a aprins făcliile și a dat drumul vulpilor prin grânele Filistenilor și a aprins și clăile și grâul nesecerat, viile și livezile de măslini.
\par 6 Și ziceau Filistenii: "Cine oare a făcut aceasta?" Și li s-a spus: "Samson, ginerele Timneanului, căci acesta i-a luat femeia și a dat-o după un prieten al lui". Atunci Filistenii s-au dus și au ars-o cu foc și pe ea și casa tatălui ei.
\par 7 Dar Samson le-a zis: "Cu toate că ați făcut aceasta, eu tot am să mă răzbun pe voi și numai atunci am să mă liniștesc".
\par 8 Și le-a sfărâmat fluierele picioarelor și șoldurile și apoi s-a dus și a șezut în peștera de la stânca Etam.
\par 9 Filistenii însă s-au dus și și-au așezat tabăra în Iuda și s-au întins până la Lehi.
\par 10 Iar locuitorii lui Iuda au zis: "Pentru ce ați ieșit voi asupra noastră?" Și ei au zis: "Am venit să legăm pe Samson, ca să facem cu el cum a făcut și el cu noi".
\par 11 Atunci s-au dus trei mii de oameni din Iuda la peștera de la stânca Etam și au zis către Samson: "Nu știi tu, oare, că Filistenii domnesc peste noi? De ce ne-ai făcut tu una ca asta?" El însă a zis: "Cum s-au purtat ei cu mine, așa m-am purtat și eu cu ei".
\par 12 I-au zis lui: "Noi am venit să te legăm, ca să te dăm în mâinile Filistenilor". Atunci Samson le-a zis: "Jurați-vă mie că nu mă veți ucide!"
\par 13 Și ei au răspuns: "Nu, noi numai te vom lega și te vom da în mâinile lor, dar de omorât nu te vom omorî". Și l-au legat cu două funii noi și l-au dus din peșteră.
\par 14 Dar când s-a apropiat el de Lehi, Filistenii l-au întâmpinat cu strigăte. Atunci s-a coborât peste el Duhul Domnului și funiile care erau peste mâinile lui s-au făcut ca niște câlți arși de foc și au căzut legăturile de pe mâinile lui.
\par 15 Iar el găsind o falcă sănătoasă de asin, și-a întins mâna, a luat-o și a ucis cu ea o mie de oameni.
\par 16 Apoi a zis Samson: "Cu o falcă de măgar o ceată, două cete am stins, Cu o falcă de măgar o mie de oameni am ucis".
\par 17 Și zicând acestea, a aruncat falca din mâini și a numit locul acela Ramat-Lehi.
\par 18 Simțind însă sete mare, a strigat către Domnul și a zis: "Tu ai făcut prin mâna robului Tău această mare izbăvire; iar acum eu mor de sete și voi cădea în mâinile celor netăiați împrejur".
\par 19 Atunci a deschis Domnul o crăpătură într-o stâncă din Lehi și a curs din ea apă. Și a băut Samson și și-a astâmpărat setea și duhul lui s-a înviorat. De aceea s-a și numit locul acela: "Izvorul celui ce strigă", care este în Lehi până în ziua de astăzi.
\par 20 Și a fost el judecător în Israel pe vremea Filistenilor douăzeci de ani.

\chapter{16}

\par 1 Venind însă odată Samson la Gaza, a văzut acolo o femeie desfrânată și a intrat la ea.
\par 2 Și li s-a spus oamenilor din Gaza: "Samson a venit aici". Atunci aceștia l-au înconjurat și l-au pândit toată noaptea la porțile cetății și s-au ascuns toată noaptea, zicând: "Să așteptăm până se va lumina de ziuă și să-l ucidem!"
\par 3 Samson însă a dormit până la miezul nopții; iar la miezul nopții a luat porțile cetății din amândoi ușorii și, ridicându-le împreună cu zăvoarele, le-a pus pe umerii săi și le-a dus pe vârful muntelui care este pe drumul spre Hebron și le-a lăsat acolo.
\par 4 După acestea a iubit el o femeie care trăia în valea Sorec și pe care o chema Dalila.
\par 5 La aceasta au venit fruntașii Filistenilor și i-au zis: "Amăgește-l și află în ce stă puterea lui cea mare și cum l-am putea prinde, ca să-l legăm și să-l supunem; și-li vom da pentru aceasta o mie și o sută de sicli de argint".
\par 6 Și a zis Dalila către Samson: "Spune-mi și mie în ce stă puterea ta cea mare și cu ce să te lege ca să te supună?"
\par 7 I-a răspuns Samson: "De mă vor lega cu șapte vine crude și încă neuscate, voi ajunge slab și voi fi ca și ceilalți oameni".
\par 8 Atunci i-au adus ei fruntașii Filistenilor șapte vine crude și încă neuscate și ea l-a legat cu ele.
\par 9 Și unii stăteau la pândă la ea în odaia de dormit; și ea a zis către Samson: "Samsoane, Filistenii vin asupra ta!" Atunci el a rupt vinele, cum ar fi rupt o ață de câlți arși de foc. Și astfel nu s-a aflat de unde vine puterea lui.
\par 10 Dalila a zis însă către Samson: "Iată tu m-ai amăgit și mi-ai spus minciuni. Spune-mi dar cu ce să te lege?
\par 11 Iar el i-a zis: "De mă vor lega cu funii noi, care să nu mai fi fost întrebuințate, atunci eu voi slăbi și voi fi ca și ceilalți oameni".
\par 12 Și a luat Dalila funii noi și l-a legat, iar cineva pândea. Apoi ea i-a zis: "Samsoane, Filistenii vin asupra ta!" Și el le-a rupt de pe mâinile sale, ca pe niște ațe.
\par 13 Atunci Dalila a zis către Samson: "Tu mă amăgești mereu și-mi spui minciuni. Spune-mi drept, cu ce să te lege?" Iar el i-a zis: "De vei împleti șapte șuvițe do păr din capul meu și le vei prinde cu un cui de sulul de la războiul de țesut, atunci eu vai slăbi și voi fi ca și ceilalți oameni".
\par 14 Și l-a adormit Dalila pe brațele sale; iar dacă a adormit el, Dalila a luat șapte șuvițe din capul lui și le-a pironit de sulul de la război și apoi a strigat: "Samsoane, Filistenii vin asupra ta!" Atunci el s-a deșteptat din somn și a smucit războiul împreună cu țesătura și nu s-a aflat de unde vine puterea lui.
\par 15 Dalila însă i-a zis: "Cum de poți tu spune: "Te iubesc", când inima ta nu este cu mine? Iată, de trei ori m-ai amăgit și nu mi-ai spus în ce stă puterea ta cea mare".
\par 16 Și fiindcă ea tot stăruia și-l necăjea cu vorbele sale în fiecare zi, s-a tulburat sufletul lui până la moarte.
\par 17 Și atunci i-a descoperit el toată inima sa și i-a zis: "Briciul nu s-a atins de capul meu, căci eu sunt nazireu al lui Dumnezeu din pântecele maicii mele; de m-ar tunde cineva, atunci s-ar depărta de la mine puterea mea și eu aș slăbi și aș fi ca ceilalți oameni".
\par 18 Văzând Dalila că el i-a descoperit toată inima sa, a trimis de au chemat pe fruntașii Filistenilor, zicându-le: "Veniți acum, că el mi-a descoperit toată inima sa!" și au venit la ea fruntașii Filistenilor și au adus argintul cu ei.
\par 19 Apoi Dalila l-a adormit pe genunchii săi și a chemat un om și i-a poruncit să tundă cele șapte șuvițe ale capului lui. Atunci el a început a slăbi și s-a depărtat de el puterea lui;
\par 20 Iar ea a zis: "Samsoane; Filistenii vin asupra ta!" Și deșteptându-se el din somnul său, a zis: "Voi face ca mai înainte și mă voi scăpa de ei". Dar nu știa că Domnul Se depărtase de el.
\par 21 Atunci l-au luat fruntașii Filistenilor și i-au scos ochii și l-au dus la Gaza și l-au legat cu două lanțuri de aramă și râșnea în temniță.
\par 22 Și a început să-i crească părul pe capul lui, pe unde fusese tuns.
\par 23 Atunci s-au adunat fruntașii Filistenilor, ca să aducă jertfă marelui Dagon, dumnezeul lor, și să se veselească, zicând: "Dumnezeul nostru a dat pe Samson, vrăjmașul nostru, în mâinile noastre".
\par 24 De asemenea și mulțimea, văzându-l, slăvea pe dumnezeul său, zicând: "Dumnezeul nostru a dat în mâinile noastre pe vrăjmașul nostru și pe pustiitorul țării noastre, care a ucis pe mulți dintre noi".
\par 25 Iar după ce s-a veselit inima lor, au zis: "Aduceți pe Samson din închisoare, ca să mai râdem de el". Și au adus pe Samson din temniță și râdeau de el și-l trăgeau de urechi și l-au pus între doi stâlpi.
\par 26 Atunci a zis Samson tânărului care-l ducea de mână: "Du-mă ca să pipăi stâlpii pe care este întemeiată casa și să mă reazem de ei". Și tânărul a făcut așa.
\par 27 Casa însă era plină de bărbați și de femei, căci erau acolo toți fruntașii Filistenilor, iar pe acoperiș se aflau ca la trei mii de oameni, bărbați și femei, care se uitau și râdeau de Samson.
\par 28 Atunci a strigat Samson către Domnul și a zis: "Doamne Dumnezeule, adu-ți aminte de mine și întărește-mă încă o dată, o, Dumnezeule, ca printr-o singură lovitură să mă răzbun pe Filisteni pentru cei doi ochi ai mei !"
\par 29 Și a mișcat Samson din loc doi stâlpi din mijloc pe care era sprijinită casa, rezemându-se de ei, de unul cu mâna dreaptă și de celălalt cu stânga.
\par 30 Și a zis Samson: "Mori, suflete al meu, cu Filistenii!" Apoi s-a sprijinit cu toată puterea și s-a prăbușit casa peste fruntașii Filistenilor și peste tot poporul ce era în ea. Și cei pe care i-a ucis Samson la moartea sa au fost mai mulți decât toți cei pe care-i ucisese în viața sa.
\par 31 Atunci au venit frații lui și toată casa tatălui său și l-au luat și l-au dus de l-au îngropat între Țora și Eștaol, în mormântul lui Manoe, tată său. Și a fost el judecător în Israel douăzeci de ani. Iar după Samson s-a sculat Emegar, fiul lui Enan, și a ucis din Filisteni șase sute de oameni, afară de vite. Acesta a izbăvit pe Israel.

\chapter{17}

\par 1 În vremea aceea era cineva în Muntele Efraim, cu numele Mica.
\par 2 Acesta a zis către mama sa: "Cei o mie și o sută de sicli de argint care ți s-au luat și pentru care tu ai rostit blestem în fața mea, acel argint este la mine, eu l-am luat". A zis mama sa: "Binecuvântat fie fiul meu de Domnul!"
\par 3 Și a întors acela cei o mie și o sută sicli de argint mamei sale. Iar mama lui a zis: "Acest argint eu l-am afierosit de la mine Domnului pentru tine, fiul meu, ca să fac din el un idol, un chip turnat. Așadar ti-l dau ție".
\par 4 El însă a întors argintul mamei sale. Iar mama sa a luat două sute sicli de argint și i-a dat unui turnător și acela a făcut din ei un idol, un chip turnat, care se și afla în casa lui Mica.
\par 5 Și era la Mica locașul lui dumnezeu și a făcut un efod și un terafim și a pus el pe unul din fiii săi să fie preotul lui.
\par 6 În zilele acelea nu era rege în Israel, ci fiecare făcea ce i se părea că este drept.
\par 7 Și trăia pe atunci la Betleemul cel din seminția lui Iuda un tânăr levit.
\par 8 Și s-a dus omul acesta din cetatea Betleemului lui Iuda, ca să trăiască unde se va nimeri, și, mergând el pe cale, a ajuns pe Muntele Efraim la casa lui Mica.
\par 9 Mica însă i-a zis: "De unde vii tu?" Iar el a răspuns: "Eu sunt levit din Betleemul lui Iuda și mă duc să trăiesc unde voi nimeri".
\par 10 Atunci Mica i-a zis: "Rămâi la mine și fii părinte aici la mine și preot; eu îți voi da câte zece sicli de argint pa an și hainele și hrana trebuitoare".
\par 11 Și a venit levitul la el și s-a învoit levitul să rămână la omul acesta; și era tânăr, ca unul din fiii lui.
\par 12 Și Mica a sfințit pe levit și tânărul acesta a fost preot la el și a trăit în casa lui Mica.
\par 13 Apoi a zis Mica: "Acum eu știu că Domnul îmi va face bine, pentru că am preot pe un levit".

\chapter{18}

\par 1 În zilele acelea nu era rege în Israel; și în timpul acela seminția lui Dan își căuta moșie unde să se așeze, căci până atunci nu-i căzuse încă parte deplină printre semințiile lui Israel.
\par 2 Și au trimis fiii lui Dan din neamul lor cinci oameni, bărbați puternici, din Țora și din Eștaol, ca să cerceteze țara și s-o cunoască și li s-a zis: "Duceți-vă și cunoașteți țara aceea!" Și s-au dus aceia în Muntele Efraim, la casa lui Mica și au rămas acolo.
\par 3 Pe când se aflau ei la casa lui Mica, au cunoscut glasul tânărului levit și intrând la el, l-au întrebat: "Cine te-a adus aici? Ce faci și pentru ce stai aici?"
\par 4 Iar el le-a răspuns: "Cutare și cutare a făcut pentru mine Mica și mi-a dat simbrie și iată eu îi sunt preot".
\par 5 Aceia însă i-au zis: "Întreabă pe Dumnezeu, ca să știm, de vom izbuti pe calea în care am plecat".
\par 6 Iar preotul le-a zis: "Mergeți cu pace, calea voastră în care mergeți este înaintea Domnului".
\par 7 Și s-au dus cei cinci bărbați și au ajuns la Laiș și au văzut că poporul din el trăiește în pace, după obiceiul Sidonienilor, liniștit și fără grijă, și că nu era în țara aceea cine să obijduiască cu ceva sau să aibă stăpânire: de Sidonieni ei trăiau departe și cu nimeni nu aveau ei nici o treabă.
\par 8 Atunci s-au întors cei cinci oameni la frații lor în Țora și Eștaol și au zis frații lor către ei: "Ce ne spuneți?"
\par 9 Iar ei au răspuns: "Să ne sculăm și să mergem asupra lor. Am văzut țara și este foarte bună. Voi însă să nu stați pe gânduri și să nu zăboviți a merge și a lua în stăpânire țara aceea.
\par 10 Când veți merge, veți da de un popor fără grijă și de o țară întinsă; Dumnezeu o va da în mâinile voastre; acolo este un loc, unde nu lipsește nimic din tot ce dă pământul".
\par 11 Și au plecat într-acolo din țara și din Eștaol din seminția lui Dan șase sute de oameni, încinși cu arme de război.
\par 12 Aceștia s-au dus și au tăbărât în Chiriat-Iearim în Iuda. De aceea se și numește locul acela tabăra lui Dan până în ziua de astăzi și e în dosul lui Chiriat-Iearim.
\par 13 De acolo s-au îndreptat spre Muntele Efraim și au venit la casa lui Mica,
\par 14 Atunci au zis cei cinci bărbați, care fuseseră să iscodească țara Laiș, către frații lor: "Știți voi oare că în una din casele acestea este un efod, un terafim, un idol și un chip turnat? Așadar, gândiți-vă ce trebuie să faceți".
\par 15 Apoi s-au abătut într-acolo și au intrat la casa levitului celui tânăr, în casa lui Mica și i-au dat bună ziua.
\par 16 Cei șase sute de oameni din fiii lui Dan, încinși cu arme de război, s-au oprit la poartă.
\par 17 Iar cei cinci oameni, care fuseseră de iscodiseră țara, s-au dus și au intrat acolo, au luat idolul și efodul și terafimul și chipul cel turnat. Preotul însă stătea la poartă cu cei șase sute de oameni încinși cu arme de război.
\par 18 Când au intrat ei în casa lui Mica și au luat idolul, terafimul, efodul și chipul cel turnat, preotul le-a zis: "Ce faceți voi?"
\par 19 Iar ei au zis: "Taci, pune-ți mâna la gură și vino cu noi și ne fii părinte și preot; este mai bine oare de tine să fii preot în casa unui singur om decât să fii preot într-o seminție sau într-o familie a lui Israel?"
\par 20 Atunci preotul s-a îmbunat și a luat efodul, terafimul, idolul și chipul cel turnat și s-a dus cu mulțimea.
\par 21 După aceea ei s-au întors și au plecat, punând copiii, vitele și avutul înainte.
\par 22 Iar după ce s-au depărtat de casa lui Mica, Mica și locuitorii caselor vecine cu casa lui Mica s-au strâns și au alergat după fiii lui Dan,
\par 23 Și au strigat către fiii lui Dan, care s-au întors și au zis către Mica: "Ce ai de strigi așa?"
\par 24 A zis Mica: "Voi mi-ați luat dumnezeul meu, pe care l-am făcut eu, și pe preotul meu și v-ați dus. Ce-mi mai rămâne? Cum dar ziceți: Ce ai?"
\par 25 Iar fiii lui Dan i-au zis: "Taci, să nu-ți mai auzim gura! Altfel, supărându-se, unii din noi vor tăbărî pe tine, și vei pieri și tu și familia ta".
\par 26 Și s-au dus fiii lui Dan în drumul lor; iar Mica, văzând că aceia sunt mai tari decât el, s-a întors și s-a dus la casa sa.
\par 27 Iar fiii lui Dan au luat ceea ce făcuse Mica și pe preotul care era la el și s-au dus la Laiș asupra unui popor liniștit și fără grijă și l-au ucis cu sabia și cetatea au ars-o cu foc.
\par 28 Și n-a avut cine să-l ajute, căci era departe de Sidon și nu avea legături cu nimeni. Cetatea aceasta se afla în valea cea din apropiere de Bet-Rehob. Și au clădit din nou cetatea și s-au așezat în ea.
\par 29 Apoi au pus numele cetății Dan, după numele străbunului lor Dan, fiul lui Israel; mai înainte însă numele cetății era Laiș.
\par 30 Și au așezat fiii lui Dan idolul la ei. Iar Ionatan, fiul lui Gherșom, și fiii săi au fost preoți în seminția lui Dan până în ziua robirii țării.
\par 31 Și au avut la ei idolul făcut de Mica în tot timpul cât cortul lui Dumnezeu a fost la Șilo.

\chapter{19}

\par 1 În zilele acelea, când nu era rege în Israel, trăia un levit pe coasta Muntelui Efraim. Acesta și-a luat o concubină din Betleemul Iudei.
\par 2 Și s-a certat cu el și s-a dus de la el înapoi la casa tatălui ei în Betleemul Iudei și a stat acolo patru luni.
\par 3 Atunci bărbatul ei s-a sculat și s-a dus după ea, ca să se împace cu ea și s-o aducă acasă. Cu el era o slugă a sa și o pereche de asini. Și l-a dus ea în casa tatălui său.
\par 4 Socrul său, tatăl acestei tinere femei, văzându-l, l-a întâmpinat cu bucurie și l-a oprit și el a rămas la el trei zile. Au mâncat și au băut și au rămas acolo.
\par 5 A patra zi s-au sculat ei de vreme și s-au gătit să plece. Iar tatăl tinerei femei a zis către ginerele său: "Întărește-ți inima cu o bucățică de pâine și apoi vei pleca".
\par 6 Și au rămas și au mâncat și au băut amândoi împreună. Apoi tatăl tinerei femei a zis către omul acela: "Rămâi încă și noaptea asta, ca să se veselească inima ta".
\par 7 Omul acela însă s-a sculat să plece, dar socrul său l-a rugat și el a mai rămas o noapte acolo.
\par 8 A cincea zi s-a sculat el de dimineață ca să plece. și tatăl acelei tinere femei iar i-a zis: "Întărește-ți inima ta cu pâine și zăbovește până când va fi soarele spre asfințit". Și au mâncat ei amândoi și au băut.
\par 9 Apoi s-a sculat omul acela ca să plece el și concubina sa și sluga sa. Iar socrul său, tatăl tinerei femei, a zis: "Iată s-a plecat ziua spre seară, rămâi, rogu-te, iată ziua se va sfârși curând, rămâi aici; să se veselească inima ta! Mâine vă veți scula de dimineață și veți pleca în calea voastră și te vei duce la casa ta".
\par 10 Dar omul nu s-a învoit să rămână, ci s-a sculat și a plecat și a venit până la Iebus, care acum este Ierusalimul. Cu el erau doi asini încărcați și concubina lui.
\par 11 Dar când s-au apropiat ei de Iebus, ziua se apropia de seară. Atunci a zis sluga către stăpânul său: "Să ne abatem în cetatea aceasta a Iebuseilor și să rămânem în ea".
\par 12 Stăpânul lui însă i-a zis: "Nu, să nu intrăm în cetatea unor oameni de alt neam, care nu sunt din fiii lui Israel, ci să mergem până la Ghibeea".
\par 13 Apoi a zis iar către sluga sa: "Să mergem până la unul din aceste locuri și să rămânem în Ghibeea sau Rama".
\par 14 Și au mers ei și au ajuns aproape de Ghibeea și, când au ajuns la Ghibeea lui Veniamin, a asfințit soarele.
\par 15 și s-au abătut într-acolo, ca să meargă să rămână în Ghibeea. Și au venit și au rămas în ulița cetății, căci nimeni nu i-a chemat În casă ca să-i găzduiască.
\par 16 Iată însă că venea un bătrân de la lucru din câmp, seara; acesta era de neam din Muntele Efraim și trăia în Ghibeea. Iar locuitorii din cetatea aceasta erau din fiii lui Veniamin.
\par 17 Și ridicându-și ochii săi, văzu un trecător pe ulița cetății; și a zis bătrânul: "Încotro mergi și de unde vii?"
\par 18 A zis acela: "Noi mergem de la Betleemul Iudei la muntele lui Efraim, de unde sunt eu; am fost la Betleemul Iudei și acum mă duc la casa Domnului; dar nimeni nu mă cheamă în casă;
\par 19 Noi avem și paie și nutreț pentru asinii noștri; de asemenea pâine și vin pentru mine și pentru roaba ta și pentru sluga aceasta a robilor tăi; n-avem nevoie de nimic".
\par 20 Atunci bătrânul a zis: "Fiți liniștiți; toate lipsurile rămân asupra mea, numai să nu rămâi în uliță!"
\par 21 Apoi l-a dus în casa sa și a dat nutreț asinilor lui, iar ei și-au spălat picioarele și au mâncat și au băut.
\par 22 Dar după ce s-a veselit inima lor, iată locuitorii cetății, oameni desfrânați, au înconjurat casa, bătând în ușă și zicând bătrânului, stăpânul casei: "Scoate pe omul care a intrat în casa ta, ca să-l cunoaștem".
\par 23 Atunci stăpânul casei a ieșit la ei și le-a zis: "Nu, frații mei, să nu faceți rău omului, de vreme ce a intrat în casa mea, să nu faceți această ticăloșie!
\par 24 Iată, eu am o fiică fecioară și el are o concubină; vi le voi scoate, ca să le cunoașteți pe ele și să faceți cu ele ce vă place; iar cu omul acesta să nu faceți această nebunie!"
\par 25 Dar ei n-au voit să-l asculte. Atunci omul a luat pe concubina sa și a scos-o în uliță. Iar ei au cunoscut-o pe ea și și-au bătut joc de ea toată noaptea până dimineața, și la ivirea zorilor au părăsit-o.
\par 26 Și în revărsatul zorilor a venit femeia și a căzut înaintea ușii casei omului aceluia, la care era stăpânul ei și a zăcut acolo până s-a făcut ziuă.
\par 27 Stăpânul ei însă s-a sculat dimineața, a deschis ușa casei și a ieșit, ca să plece în drumul său; și iată concubina sa zăcea la ușa casei, și mâinile ei erau pe prag.
\par 28 Atunci el i-a zis: "Scoală să mergem!" Dar n-a primit nici un răspuns, pentru că murise. Atunci el a pus-o pe un asin și s-a ridicat și a plecat la casa sa.
\par 29 Iar dacă a ajuns la casa sa, a luat un cuțit și, apucând pe concubina sa, a tăiat-o bucățică eu bucățică în douăsprezece părți și le-a trimis în toate hotarele lui Israel.
\par 30 Tot cel ce vedea acestea zicea: "N-a mai fost, nici nu s-a mai văzut ceva asemenea din zilele ieșirii fiilor lui Israel din țara Egiptului, până în ziua aceasta". Iar oamenilor trimiși din partea sa le dăduse poruncă, zicându-le: "Așa să spuneți la tot Israelul: A mai fost oare cândva asemenea cu aceasta? Luați seama la aceasta, sfătuiți-vă și hotărâți!"

\chapter{20}

\par 1 Atunci au ieșit fiii lui Israel și s-a adunat toată obștea, ca un singur om, de la Dan până la Beerșeba, cu țara Galaadului, înaintea Domnului la Mițpa.
\par 2 Căpeteniile întregului popor și toate semințiile lui Israel s-au înfățișat înaintea Domnului, la adunarea poporului lui Dumnezeu, ca la patru sute de mii pedeștri, purtători de sabie.
\par 3 Și au auzit fiii lui Veniamin că fiii lui Israel au venit la Mițpa. Atunci au zis fiii lui Israel: "Spuneți cum s-a făcut nelegiuirea aceasta?"
\par 4 Iar levitul, bărbatul femeii celei ucise, a răspuns și a zis: "Eu cu concubina mea am venit să rămânem în Ghibeea lui Veniamin.
\par 5 Și s-au ridicat asupra mea locuitorii din Ghibeea și au înconjurat pentru mine casa, noaptea; aveau de gând să mă ucidă și au chinuit pe concubina mea, bătându-și joc de ea, așa încât ea a murit.
\par 6 Atunci eu am luat concubina mea, am tăiat-o și am trimis-o în toate ținuturile stăpânirii lui Israel, pentru că ei au făcut un lucru nelegiuit și de rușine în Israel.
\par 7 Iată acum voi, fiii lui Israel, cercetați cu toții acest lucru și hotărâți aici".
\par 8 Și s-a ridicat tot poporul, ca un singur om, și a zis: "Nu ne vom duce nici unul la corturile noastre și nimeni nu se va întoarce la casa sa,
\par 9 Ci iată ce vom face acum cu Ghibeea: Vom merge asupra ei după sorți;
\par 10 Și vom lua câte zece oameni la sută din toate semințiile lui Israel, câte o sută la mie și câte o mie la zece mii, ca să aducă merinde pentru poporul care se va duce asupra Ghibeii lui Veniamin, ca să o pedepsească pentru lucrul rușinos pe care l-a făcut ea în Israel".
\par 11 Și s-au adunat toți Israeliții asupra cetății într-un cuget, ca un singur om.
\par 12 Și au trimis semințiile lui Israel în toată seminția lui Veniamin să se spună: "Ce lucru rușinos s-a făcut la voi?
\par 13 Dați pe acei oameni ticăloși care sunt în Ghibeea, că avem să-i omorâm și să stârpim răul din Israel!" Dar fiii lui Veniamin n-au voit să asculte glasul fraților lor, adică al fiilor lui Israel.
\par 14 Și s-au adunat fiii lui Veniamin de prin cetăți la Ghibeea, ca să meargă cu război asupra fiilor lui Israel.
\par 15 Și s-au numărat în ziua aceea fiii lui Veniamin, care se adunaseră de prin cetăți, douăzeci și șase de mii de oameni purtători de sabie; afară de aceștia se mai numărau din locuitorii Ghibeii șapte sute de oameni aleși.
\par 16 Din tot poporul acesta erau șapte sute de oameni aleși, care erau stângaci, și toți aceștia nimereau drept la țintă când aruncau pietre cu praștia în firul de păr.
\par 17 Israeliții însă, afară de fiii lui Veniamin, numărau patru sute de mii de oameni purtători de sabie și toți aceștia erau destoinici la luptă.
\par 18 Și s-au sculat și s-au dus la casa Domnului și au întrebat pe Dumnezeu și au zis fiii lui Israel: "Cine din noi va pleca întâi la război cu fiii lui Veniamin?" Și Domnul a zis: "Iuda va pleca întâi!"
\par 19 Apoi s-au sculat fiii lui Israel dimineața și au tăbărât lângă Ghibeea.
\par 20 Și au pornit fiii lui Israel la război împotriva lui Veniamin și s-au pus fiii lui Israel în rânduială de război aproape de Ghibeea.
\par 21 Iar fiii lui Veniamin au ieșit din Ghibeea și au pus în ziua aceea douăzeci și două de mii de Israeliți la pământ.
\par 22 Dar poporul israelit se îmbărbătă și se puse din nou în rânduială de război în același loc unde stătuse în ziua întâi.
\par 23 Și s-au dus fiii lui Israel și au plâns înaintea Domnului până seara și au întrebat pe Domnul: "Să mai mergem oare la luptă cu fiii lui Veniamin, fratele meu?" Și Domnul a zis: "Mergeți asupra lui!"
\par 24 Și au înaintat fiii lui Israel asupra fiilor lui Veniamin a doua oară.
\par 25 Și a ieșit Veniamin asupra lor din Ghibeea a doua zi și au mai pus la pământ din fiii lui Israel încă optsprezece mii de oameni purtători de sabie.
\par 26 Atunci toți fiii lui Israel și tot poporul au plecat și au venit la casa Domnului și au postit în ziua aceea până seara și au adus arderi de tot și jertfe de împăcare înaintea Domnului.
\par 27 Și au întrebat fiii lui Israel pe Domnul. Pe atunci chivotul legii Domnului se afla acolo,
\par 28 Și Finees, fiul lui Eleazar, fiul lui Aaron, sta înaintea lui Dumnezeu, zicând: "Să mai ies eu oare la luptă cu fiii lui Veniamin, fratele meu, sau nu?" Iar Domnul a zis: "Duceți-vă, că mâine Eu am să-l dau în mâinile tale!"
\par 29 Și au pus fiii lui Israel oameni de pază împrejurul Ghibeii.
\par 30 Apoi s-au dus fiii lui Israel asupra fiilor lui Veniamin a treia zi și s-au pus în rânduială de război înaintea Ghibeii, ca și mai înainte.
\par 31 Iar fiii lui Veniamin au ieșit asupra poporului și s-au depărtat de cetate, începând, ca și mai înainte, a ucide din popor pe căile ce duceau una spre Betleem, iar alta spre Ghibeea, peste câmp; și au ucis până la treizeci de inși dintre Israeliți.
\par 32 Atunci au zis fiii lui Veniamin: "Aceștia au să cadă înaintea noastră, ca și înainte". Iar fiii lui Israel au zis: "Să fugim de ei și să-i depărtăm de cetate pe cale". Și au făcut așa.
\par 33 Atunci toți Israeliții s-au sculat de la locurile lor și au tăbărât la Baal-Tamar. Iar oamenii de pază ai lui Israel au alergat de la locul lor, din partea de apus a Ghibeii.
\par 34 Și au sosit înaintea Ghibeii zece mii de oameni aleși din tot Israelul și s-a început o luptă crâncenă; dar fiii lui Veniamin nu știau că-i amenință primejdia.
\par 35 Și a lovit Domnul pe Veniamin înaintea Israeliților și Israeliții au răpus în ziua aceea din fiii lui Veniamin douăzeci și cinci de mii o sută de oameni purtători de sabie.
\par 36 Atunci au văzut fiii lui Veniamin că sunt înfrânți; căci Israeliții nu se retrăgeau din fața fiilor lui Veniamin, decât pentru că se bizuiau pe oamenii pe care ei îi puseseră de pază împotriva Ghibeii.
\par 37 Cei puși la pândă s-au aruncat repede asupra Ghibeii și au intrat în ea și au trecut toată cetatea prin ascuțișul sabiei.
\par 38 Israeliții însă se înțeleseseră cu oamenii de pază ca să le fie semn al năvălirii fumul ce se va ridica din cetate.
\par 39 Deci când Israeliții s-au tras înapoi de la locul de luptă, Veniamin a început să lovească și a rănit până la vreo treizeci de Israeliți, și ziceau: "Iarăși au să cadă înaintea noastră, ca și în luptele dinainte!"
\par 40 Atunci a început să se ridice din cetate un stâlp de fum. Și uitându-se Veniamin înapoi, iată din toată cetatea se înălța fum spre cer.
\par 41 În clipa aceasta Israeliții se întoarseră, iar Veniamin s-a speriat, căci a văzut că-l ajunsese primejdia;
\par 42 Și au fugit ei de Israeliți pe calea ce ducea spre pustie; dar măcelul îi urmărea și cei ce ieșeau din cetate erau uciși pe loc.
\par 43 Și au împresurat Israeliții pe Veniamin și l-au urmărit până la Menoha și i-au măcelărit până în partea răsăriteană a Ghibeii.
\par 44 Atunci au căzut din fiii lui Veniamin optsprezece mii de inși, toți bărbați voinici.
\par 45 Iar câți au rămas s-au abătut și au fugit în pustiu spre stânca lui Rimon și au mai ucis Israeliții pe drum cinci mii de oameni; alergând după ei până la Ghideom au mai ucis din ei încă două mii de oameni.
\par 46 Iar toți fiii lui Veniamin care au căzut în ziua aceea au fost douăzeci și cinci de mii, purtători de sabie, și toți aceștia erau oameni voinici.
\par 47 Și au fugit cei ce scăpaseră în pustiu, la stânca lui Rimon, ca la șase sute de oameni și au rămas acolo în muntele cel stâncos al lui Rimon patru luni.
\par 48 Iar Israeliții s-au întors la fiii lui Veniamin și i-au lovit cu sabia în cetate: și oameni și vite și tot ce au întâlnit în toate cetățile și toate cetățile ce-au întâlnit în cale le-au ars cu foc.

\chapter{21}

\par 1 Și s-au jurat Israeliții în Mițpa, zicând: "Nimeni din noi să nu-și dea fetele sale de soții după fiii lui Veniamin".
\par 2 Apoi a venit poporul la Betel și a stat acolo până seara înaintea lui Dumnezeu, și a ridicat glasul său și a plâns cu jale mare,
\par 3 Zicând: "Doamne, Dumnezeul lui Israel, pentru ce oare s-a petrecut aceasta în Israel, că iată acum lipsește din Israel o seminție?"
\par 4 Iar a doua zi s-a sculat poporul de dimineață și a făcut acolo jertfelnic și a adus arderi de tot și jertfe de izbăvire.
\par 5 Apoi au zis fiii lui Israel: "Cine oare n-a venit la adunarea ce s-a ținut înaintea Domnului dintre toate semințiile lui Israel?" Căci blestem mare se rostise asupra acelora care nu aveau să vină înaintea Domnului în Mițpa și se zisese că aceia să fie dați morții.
\par 6 Și s-au înduioșat fiii lui Israel față de Veniamin, fratele lor, zicând: "Acum s-a tăiat o seminție din Israel.
\par 7 Ce vom face pentru a găsi femei celor care au rămas, deoarece ne-am jurat înaintea Domnului să nu le dăm femei din fetele noastre?
\par 8 Atunci s-a văzut că din Iabeș-Galaad nu venise nimeni înaintea Domnului la adunarea din tabără.
\par 9 Și s-a cercetat poporul și iată nu era acolo nici unul din locuitorii Iabeșului din Galaad.
\par 10 Atunci a trimis acolo obștea douăsprezece mii de oameni, bărbați voinici și le-a dat poruncă, zicând: "Mergeți și loviți pe locuitorii din Iabeșul Galaadului cu sabia, și femeile, și copiii.
\par 11 Și iată ce să mai faceți: pe orice bărbat și orice femeie care a cunoscut bărbat, să-i dați pieirii, iar fetele lăsați-le cu viață". Și așa au făcut.
\par 12 Și au găsit ei printre locuitorii din Iabeșul Galaadului patru sute de fete care nu cunoscuseră bărbat și le-au adus în tabără la Șilo, care e în pământul Canaan.
\par 13 Atunci toată obștea a trimis să grăiască fiilor lui Veniamin care erau la stânca lui Rimon și să le vestească pace.
\par 14 Și s-au întors fiii lui Veniamin la Israeliți și Israeliții le-au dat soții din femeile rămase în viață din Iabeșul Galaadului. Dar curând s-a văzut că acestea nu erau de ajuns.
\par 15 Poporul însă jelea după Veniamin, că Domnul n-a păstrat în întregime semințiile lui Israel.
\par 16 Au zis drept aceea bătrânii obștii: "Ce să facem cu cei rămași fără femei, căci au fost stârpite femeile în Veniamin?"
\par 17 Apoi au zis: "Pământul de moștenire să rămână în întregime fiilor lui Veniamin, ca să nu piară seminția lui din Israel.
\par 18 Dar noi nu le putem da femei din fetele noastre, căci fiii lui Israel s-au jurat, zicând: Blestemat să fie cel ce va da femei lui Veniamin!"
\par 19 Și au mai zis: "Iată, în fiecare an se face sărbătoarea Domnului în Șilo, care este așezat la miazănoapte de Betel și la răsărit de drumul ce duce de la Betel la Sichem și la miazăzi de Lebona".
\par 20 Drept aceea au poruncit fiilor lui Veniamin și au zis: "Mergeți și pândiți din vii
\par 21 Și băgați de seamă când vor ieși fetele din Șilo să joace la horă; atunci să ieșiți din vii și să vă luați femei din fetele din Șilo și mergeți în pământul lui Veniamin.
\par 22 Iar când vor veni părinții lor sau frații lor cu plângere la noi, noi le vom zice: "Iertați-i pentru noi, căci noi n-am luat în război femei pentru fiecare dintre ei și nici voi nu le-ați dat; acum și voi sunteți de vină".
\par 23 Și fiii lui Veniamin așa au și făcut și și-au luat femei după numărul lor din cele ce erau la horă și pe care ei le-au răpit și s-au dus înapoi în moștenirea lor și au zidit cetăți și au început să trăiască în ele.
\par 24 În același timp Israeliții s-au împărțit de acolo și s-a dus fiecare în seminția sa și la moștenirea lui.
\par 25 În zilele acelea nu era rege în Israel și fiecare făcea ce i se părea că este cu dreptate.


\end{document}