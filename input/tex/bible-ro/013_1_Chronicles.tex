\begin{document}

\title{1 Chronicles}

1Ch 1:1  Adam, Set, Enos;
1Ch 1:2  Chenan, Mahalaleel, Iared;
1Ch 1:3  Enoh, Matusalem, Lameh;
1Ch 1:4  Noe, Sem, Ham ?i Iafet.
1Ch 1:5  Fiii lui Iafet: Gomer, Magog, Madai, Iavan, Eli?a, Tubal, Me?ec ?i Tiras.
1Ch 1:6  Fiii lui Gomer: A?chenaz, Rifat ?i Togarma.
1Ch 1:7  Fiii lui Iavan: Eli?a, Tar?i?, Chitim ?i Dodanim.
1Ch 1:8  Fiii lui Ham: Cu?, Mi?raim, Put ?i Canaan.
1Ch 1:9  Fiii lui Cu?: Seba, Havila, Savta Rama ?i Sabteca. Fiii lui Rama: ?eba ?i Dedan.
1Ch 1:10  Lui Cu? i s-a mai nascut de asemenea ?i Nimrod. Acesta a început sa fie puternic pe pamânt.
1Ch 1:11  Lui Mi?raim i s-a nascut: Ludim, Anamim, Lehabim, Naftuhim,
1Ch 1:12  Patrusim, Casluhim, din care se trag Filistenii ?i Caftorim.
1Ch 1:13  Lui Canaan i s-au nascut: Sidon, întâiul sau nascut ?i Het,
1Ch 1:14  Iebuseu, Amoreu, Ghergheseu,
1Ch 1:15  Heveu, Archeu, Sineu,
1Ch 1:16  Arvadeu, ?emareu ?i Hamateu.
1Ch 1:17  Fiii lui Sem: Elam, Asur, Arpaxad, Lud ?i Aram. Fiii lui Aram: Ut, Hul, Gheter ?i Me?ec.
1Ch 1:18  Lui Arpaxad i s-a nascut Cainan, lui Cainan i s-a nascut ?elah, lui ?elah i s-a nascut Eber.
1Ch 1:19  Lui Eber i s-au nascut doi fii: numele unuia era Peleg, pentru ca în zilele lui s-a împar?it ?ara; iar numele fratelui sau era Ioctan.
1Ch 1:20  Lui Ioctan i s-au nascut: Almodad, ?elef, Ha?armavet, Iarah,
1Ch 1:21  Hadoram, Uzal, Dicla,
1Ch 1:22  Ebal, Abimael, ?eba,
1Ch 1:23  Ofir, Havila ?i Iobab. To?i ace?tia sunt fiii lui Ioctan.
1Ch 1:24  Iar fiii lui Sim sunt: Arpaxad, Cainan, Selah,
1Ch 1:25  Eber, Peleg, Reu,
1Ch 1:26  Serug, Nahor, Terah
1Ch 1:27  ?i Avram, adica Avraam.
1Ch 1:28  Fiii lui Avraam sunt Isaac ?i Ismael.
1Ch 1:29  Iata spi?a neamului lor: Nebaiot, întâiul nascut al lui Ismael, apoi: Chedar, Adbeel, Mibsam,
1Ch 1:30  Mi?ma, Duma, Ma?a, Hadad, Tema,
1Ch 1:31  Ietur, Nafi? ?i Chedma. Ace?tia sunt fiii lui Ismael.
1Ch 1:32  Fiii Cheturei, ?iitoarea lui Avraam. Ea a nascut pe Zimran, Ioc?an, Medan, Madian, I?bac ?i ?uah. Fiii lui Ioc?an sunt: ?eba ?i Dedan. Fiii lui Dedan sunt: Raguel, Navdeel, A?urim, Letu?im ?i Leumim.
1Ch 1:33  Fiii lui Madian sunt: Efa, Efer, Enoh, Abida ?i Eldaa. To?i ace?tia sunt fiii Cheturei.
1Ch 1:34  Lui Avraam i s-a nascut Isaac. Fiii lui Isaac sunt: Isav ?i Israel.
1Ch 1:35  Fiii lui Isav sunt: Elifaz, Raguel, Ieu?, Ialam ?i Core.
1Ch 1:36  Fiii lui Elifaz sunt: Teman, Omar, ?efi, Gatam, Chenaz; iar Temna, concubina lui Elifaz, i-a nascut pe Amalec.
1Ch 1:37  Fiii lui Raguel sunt: Nahat, Zerah, ?ama ?i Miza.
1Ch 1:38  Fiii lui Seir sunt: Lotan, ?obal, ?ibeon, Ana, Di?on, E?er ?i Di?an.
1Ch 1:39  Fiii lui Lotan sunt: Hori ?i Heman; iar sora lui Lotan se numea Timna.
1Ch 1:40  Fiii lui ?obal sunt: Alvan, Manahat, Ebal, ?efo ?i Onam. Fiii lui ?ibeon sunt: Aia ?i Ana.
1Ch 1:41  Fiii lui Ana sunt: Di?on ?i Olibama; fiii lui Di?on sunt: Hemdan, E?ban, Itran ?i Cheran.
1Ch 1:42  Fiii lui E?er sunt: Bilhan, Zaavan ?i Acan; fiii lui Di?an sunt: U? ?i Aran.
1Ch 1:43  Ace?tia sunt regii care au domnit în pamântul Edom, înainte de a se ridica rege, peste fiii lui Israel, Bela, fiul lui Beor, cetatea caruia se numea Dinhaba.
1Ch 1:44  Murind Bela, dupa el a fost facut rege Iobab, fiul lui Zerah din Bo?ra.
1Ch 1:45  Dupa moartea lui Iobab s-a facut rege Hu?am, în ?ara Temani?ilor.
1Ch 1:46  Murind Hu?am, s-a facut rege dupa el Hadad, fiul lui Bedad, care a lovit pe Madiani?i în câmpia Moabului. Ora?ul lui se numea Avit.
1Ch 1:47  Murind Hadad, s-a facut rege dupa el ?amla, din Masreca;
1Ch 1:48  Murind ?amla, s-a facut rege dupa el ?aul, din Rehobotul cel de lânga râu.
1Ch 1:49  Murind ?aul, s-a facut rege dupa el Baal-Hanan, fiul lui Acbor.
1Ch 1:50  Murind Baal-Hanan, s-a facut rege dupa el Hadad. Numele ceta?ii lui era Pau, iar numele femeii lui era Mehetabeel, fiica lui Matred, fiica lui Mezahab.
1Ch 1:51  Murind Hadad, au urmat capetenii peste Edom: capetenia Timna, capetenia Alia, capetenia Ietet,
1Ch 1:52  Capetenia Oholibama, capetenia Ela, capetenia Pinon,
1Ch 1:53  Capetenia Chenaz, capetenia Teman, capetenia Mib?ar,
1Ch 1:54  Capetenia Magdiel, capetenia Iram. Acestea sunt capeteniile Edomului.
1Ch 2:1  Iata acum fiii lui Israel: Ruben, Simeon, Levi, Iuda, Isahar, Zabulon,
1Ch 2:2  Dan, Iosif, Veniamin, Neftali, Gad ?i A?er.
1Ch 2:3  Fiii lui Iuda sunt: Ir, Onan ?i ?ela. Ace?ti trei i s-au nascut lui din fata unui canaanit anume ?ua. Ir, întâiul nascut al lui Iuda, a fost rau în ochii Domnului ?i l-a omorât.
1Ch 2:4  Tamara, nora lui Iuda, i-a nascut acestuia pe Fares ?i pe Zara. A?a ca, de to?ii, fiii lui Iuda au fost cinci.
1Ch 2:5  Fiii lui Fares sunt He?ron ?i Hamul.
1Ch 2:6  Fiii lui Zara sunt: Zimri, Etan, Heman, Calcol ?i Darda; cinci de to?i.
1Ch 2:7  Fiul lui Carmi este Acar, care a adus nenorocire asupra lui Israel, calcând juramântul.
1Ch 2:8  Fiul lui Etan este Azaria.
1Ch 2:9  Fiii lui He?ron care i s-a nascut sunt: Ierahmeel, Ram ?i Chelubai (Caleb).
1Ch 2:10  Lui Ram însa i s-a nascut Aminadab; lui Aminadab i s-a nascut Naason, capetenia fiilor lui Iuda.
1Ch 2:11  Lui Naason i s-a nascut Salmon, lui Salmon i s-a nascut Booz.
1Ch 2:12  Lui Booz i s-a nascut Obed, lui Obed i s-a nascut Iesei.
1Ch 2:13  Lui Iesei i s-a nascut Eliab, întâiul sau nascut, apoi al doilea, Aminadab, al treilea, ?ama,
1Ch 2:14  Al patrulea, Natanael, al cincilea, Radai,
1Ch 2:15  Al ?aselea, O?em ?i al ?aptelea, David.
1Ch 2:16  Surorile lor au fost ?eruia ?i Abigail. Fiii ?eruiei au fost trei: Abi?ai, Ioab ?i Asael.
1Ch 2:17  Abigail a nascut pe Amasa; iar tatal lui Amasa este Ieter Ismaelitul.
1Ch 2:18  Caleb, fiul lui He?ron, a avut de la Azuba, femeia sa, ?i de la Ieriot urmatorii copii: Ie?er, ?obab ?i Ardon.
1Ch 2:19  Murind însa Azuba, Caleb ?i-a luat de femeie pe Efrata ?i aceasta i-a nascut pe Hur.
1Ch 2:20  Lui Hur i s-a nascut Urie; lui Urie i s-a nascut Be?aleel.
1Ch 2:21  Dupa aceea He?ron a intrat la fata lui Machir, tatal lui Galaad; ?i a luat-o, fiind de ?aizeci de ani ?i ea i-a nascut fiu pe Segub.
1Ch 2:22  Lui Segub i s-a nascut Iair ?i avea el atunci douazeci ?i trei de ceta?i în pamântul Galaadului.
1Ch 2:23  Dar Ghe?urenii ?i Sirienii le-au luat sala?urile lui Iair cu Chenatul ?i ceta?ile care ?ineau de el, în numar de ?aizeci. Toate aceste ceta?i erau ale fiilor lui Machir, tatal lui Galaad.
1Ch 2:24  Dupa ce a murit He?ron, Caleb a intrat la Efrata, femeia lui He?ron, tatal sau, care a nascut pe A?ur, tatal lui Tecoa.
1Ch 2:25  Fiii lui Ierahmeel, întâiul nascut al lui He?ron, sunt: întâiul nascut Ram, dupa el Vuna, Oren, O?em ?i Ahia.
1Ch 2:26  Ierahmeel a mai avut ?i alta femeie, cu numele Atara; aceasta este mama lui Onan.
1Ch 2:27  Fiii lui Ram, întâiul nascut al lui Ierahmeel, sunt: Maa?, Iamin ?i Echer.
1Ch 2:28  Fiii lui Onan au fost: ?amai ?i Iada. Fiii lui ?amai au fost: Nadab ?i Abi?ur.
1Ch 2:29  Numele femeii lui Abi?ur era Abihail ?i aceasta i-a nascut pe Ahban ?i pe Molid.
1Ch 2:30  Fiii lui Nadab au fost: Seled ?i Efraim. Dar Seled a murit fara copii.
1Ch 2:31  Fiul lui Efraim a fost I?ei, iar fiul lui I?ei a fost ?e?an; iar fiul lui ?e?an a fost Ahlai.
1Ch 2:32  Fiii lui Iada, fratele lui ?amai, au fost Ieter ?i Ionatan. Ieter a murit fara copii.
1Ch 2:33  Fiii lui Ionatan au fost: Pelet ?i Zaza. Ace?tia sunt fiii lui Ierahmeel.
1Ch 2:34  ?e?an n-a avut fii, ci numai fiice. ?e?an avea un rob egiptean, cu numele Iarha.
1Ch 2:35  ?e?an a dat pe o fata a sa lui Iarha, robul sau, de femeie ?i ea a nascut pe Atai.
1Ch 2:36  Atai a avut de fiu pe Natan, iar lui Natan i s-a nascut Zabad.
1Ch 2:37  Lui Zabad i s-a nascut Eflal, iar lui Eflal i s-a nascut Obed.
1Ch 2:38  Lui Obed i s-a nascut Iehu, iar lui Iehu i s-a nascut Azaria.
1Ch 2:39  Lui Azaria i s-a nascut Hele?, iar lui Hele? i s-a nascut Eleasa.
1Ch 2:40  Lui Eleasa i s-a nascut Sismai, iar lui Sismai i s-a nascut ?alum.
1Ch 2:41  ?alum a avut de fiu pe Iecamia, iar Iecamia pe Eli?ama.
1Ch 2:42  Fiul lui Caleb, fratele lui Ierahmeel, era Me?a, întâiul sau nascut, tatal lui Zif. Acesta a avut ca fiu pe Mare?a, tatal lui Hebron.
1Ch 2:43  Fiii lui Hebron sunt: Core, Tapuah, Rechem ?i ?ema.
1Ch 2:44  Lui ?ema i s-a nascut Raham tatal lui Iorchean, iar lui Rechem i s-a nascut ?amai.
1Ch 2:45  Fiul lui ?amai a fost Maon, iar Maon este tatal lui Bet-?ur.
1Ch 2:46  ?i Efa, concubina lui Caleb, a nascut pe Haran, Mo?a ?i Gazez; iar Haran a fost tatal lui Gazez.
1Ch 2:47  Fiii lui Iahdai sunt: Reghem, Iotan, Ghe?an, Pelet, Efa ?i ?aaf.
1Ch 2:48  Concubina lui Caleb, Maaca, a nascut pe ?eber ?i pe Tirhana;
1Ch 2:49  Tot ea a nascut pe ?aaf, tatal Madmanei, pe ?eva, tatal Macbenei ?i tatal Ghibeii. Fiica lui Caleb este Acsa.
1Ch 2:50  Ace?tia au fost fiii lui Caleb. Fiul lui Hur, întâiul nascut al Efratei a fost ?obal, tatal lui Chiriat-Iearim;
1Ch 2:51  Salma, tatal lui Betleem; Haref, tatal lui Betgader.
1Ch 2:52  ?obal, tatal Chiriat-Iearimului a avut fii pe Haroe, Ha?i ?i Hamenuhot.
1Ch 2:53  Familiile Chiriat-Iearimului sunt: Itrienii, Putienii, ?umatienii ?i Mi?raenii. Din acestea se trag ?oreenii ?i E?tauleenii.
1Ch 2:54  Fiii lui Salma sunt: Betleem, Netofati?ii, Atrot-Bet-Ioab, jumatate din Manahteni, ?oareni,
1Ch 2:55  Familiile Soferi?ilor, care traiau în Iabe?, Tirati?ii, ?imati?ii, Sucati?ii. Ace?tia sunt Chineenii, care se trag din Hamat, tatal casei lui Recab.
1Ch 3:1  Fiii lui David, care i s-au nascut în Hebron, au fost: întâiul nascut Amnon din Ahinoama Izreeliteanca; al doilea, Daniel, din Abigail Carmeliteanca.
1Ch 3:2  Al treilea, Abesalom, fiul Maacai, fata lui Talmai, regele din Ghe?ur; al patrulea, Adonia, fiul Haghitei;
1Ch 3:3  Al cincilea, ?efatia din Abitala; al ?aselea, Itrean din Egla, femeia sa.
1Ch 3:4  Ace?ti ?ase i s-au nascut în Hebron. În Hebron David a domnit ?apte ani ?i ?ase luni, iar în Ierusalim a domnit treizeci ?i trei de ani.
1Ch 3:5  Iata ?i cei ce i s-au nascut în Ierusalim: ?imea, ?obab, Natan ?i Solomon, patru, din Bat?eba, fiica lui Amiel.
1Ch 3:6  Ibhar, Eli?ama, Elifelet,
1Ch 3:7  Nogah, Nefeg, Iafia,
1Ch 3:8  Eli?ama, Eliada ?i Elifelet; în total noua.
1Ch 3:9  Ace?tia sunt to?i fiii lui David, afara de cei de la ?iitoare. Iar sora lor era Tamara.
1Ch 3:10  Fiul lui Solomon este Roboam; fiul acestuia este Abia, iar al acestuia, Asa, iar al lui Asa este Iosafat.
1Ch 3:11  Fiul acestuia este Ioram, al acestuia este Ahazia ?i al acestuia este Ioa?.
1Ch 3:12  Fiul lui este Amasia, al acestuia este Azaria, iar al acestuia este Ioatam.
1Ch 3:13  Fiul acestuia este Ahaz, al acestuia este Iezechia, iar al acestuia este Manase;
1Ch 3:14  Fiul acestuia este Amon, iar al acestuia este Iosia.
1Ch 3:15  Fiii lui Iosia au fost: întâiul nascut Iohanan, al doilea Ioiachim, al treilea Sedechia ?i al patrulea ?alum.
1Ch 3:16  Fiii lui Ioiachim au fost: Iehonia, fiul lui; Sedechia, fiul lui.
1Ch 3:17  Fiii lui Iehonia, cel dus în robie, au fost: Salatiel,
1Ch 3:18  Malchiram, Pedaia, ?ena?ar, Iecamia, Ho?ama ?i Nedabia.
1Ch 3:19  Iar fiii lui Pedaia au fost: Zorobabel ?i ?imei. Iar fiii lui Zorobabel au fost: Me?ulam ?i Hanania, ?i sora lor ?elomit.
1Ch 3:20  Fiii lui Me?ulam: Ha?uba, Ohel, Berechia, Hasadia ?i Iu?ab-Hesed.
1Ch 3:21  Fiii lui Hanania au fost Pelatia ?i Isaia; fiul acestuia a fost Refaia, al acestuia a fost Arnan, al acestuia a fost Obadia, iar al acestuia ?ecania.
1Ch 3:22  Fiii lui ?ecania au fost ?ase: ?emaia, Hatu?, Igheal, Bariah, Nearia ?i ?afat.
1Ch 3:23  Fiii lui Nearia au fost trei: Elioenai, Iezechia ?i Azricam.
1Ch 3:24  Fiii lui Elioenai au fost ?apte: Hodavia, Elia?ib, Pelaia, Acub, Iohanan, Delaia ?i Anani.
1Ch 4:1  Fiii lui Iuda au fost: Fares, He?ron, Carmi, Hur ?i ?obal.
1Ch 4:2  Reaia, fiul lui ?obal, a avut fiu pe Iahat; lui Iahat i s-a nascut Ahumai ?i Lahad. Din el se trag familiile ?oreenilor.
1Ch 4:3  Fiii lui Etam sunt: Izreel, I?ma ?i Idba?, ?i sora lor cu numele Ha?lelponi.
1Ch 4:4  Panuel, tatal lui Ghedor ?i Ezer, tatal lui Hu?a sunt fiii lui Hur, întâiul nascut din Efrata ?i tatal lui Betleem.
1Ch 4:5  A?hur, tatal lui Tecoa, a avut doua femei: pe Helea ?i Naara.
1Ch 4:6  Naara i-a nascut pe Ahuzam, Hefer, Temni ?i Aha?tari. Ace?tia sunt fiii Naarei.
1Ch 4:7  Iar fiii Helei sunt: ?eret, ?ohar Etna ?i Co?.
1Ch 4:8  Lui Co? i s-au nascut: Anub, ?obeba, Iahe? ?i familiile lui Aharhel, fiul lui Harum.
1Ch 4:9  Iabe? a fost mai însemnat decât fra?ii sai. Mama lui i-a dat numele de Iabe?, zicând: "Cu durere l-am nascut".
1Ch 4:10  ?i a strigat Iabe? catre Dumnezeul lui Israel ?i a zis: "O, de m-ai binecuvânta Tu cu binecuvântare, de ai largi hotarele mele ?i de ar fi mâna Ta cu mine, pazindu-ma de rele, ca sa nu fiu omorât!..." Atunci Dumnezeu i-a trimis ceea ce a dorit el.
1Ch 4:11  ?i lui Chelub, fratele lui ?uha, i s-a nascut Mehir. Acesta e tatal lui E?ton.
1Ch 4:12  Lui E?ton i s-au nascut Bet-Rafa, Paseah ?i Techina, tatal ceta?ii Naha?; ace?tia sunt locuitorii din Recab.
1Ch 4:13  Fiii lui Chenaz sunt Otniel ?i Seraia. Fiii lui Otniel au fost Hatat ?i Meonotai.
1Ch 4:14  Lui Meonotai i s-a nascut Ofra. Lui Seraia i s-a nascut Ioab, stramo?ul lui Ghehara?im, numi?i a?a pentru ca ei erau dulgheri.
1Ch 4:15  Fiii lui Caleb, fiul lui Iefoni, au fost: Ir, Ela ?i Naam. Fiul lui Ela a fost Chenaz.
1Ch 4:16  Fiii lui Iehaleleel au fost: Zif, Zifa, Tiria ?i Asareel.
1Ch 4:17  Fiii lui Ezra sunt: Ieter, Mered, Efer ?i Ialon; iar lui Ieter i s-au nascut Miriam, ?amai ?i I?bah, tatal lui E?temoa.
1Ch 4:18  Femeia acestuia, Iehudia, a nascut pe Iered, tatal lui Ghedor, pe Heber, tatal lui Soco, ?i pe Iecutiel, tatal lui Zanoah. Ace?tia sunt fiii Bitiei, fata lui Faraon, pe care a luat-o Mered.
1Ch 4:19  Fiii femeii acestuia, Hodia, sora lui Naham, tatal Cheilei, sunt: Garmi ?i E?temoa Maacateanul.
1Ch 4:20  Fiii lui Simeon sunt: Amnon, Rina, Benhanan ?i Tilon. Fiii lui I?i sunt Zohet ?i Benzohet.
1Ch 4:21  Fiii lui ?ela, fiul lui Iuda, sunt: Er, tatal lui Leca, Laeda, tatal lui Mare?a, ?i familiile din casa lui A?beia, care lucrau visonul,
1Ch 4:22  Iochim ?i locuitorii din Cozeba; Ioa? ?i Saraf, care au stapânit asupra Moabului ?i Ia?ubi-Lehem. Dar acestea sunt întâmplari mai vechi.
1Ch 4:23  Ace?tia erau olari ?i traiau la gradini ?i la livezi ?i prin ceta?i; ei traiau acolo la rege ca sa-i lucreze lui.
1Ch 4:24  Fiii lui Simeon au fost: Nemuel, Iamin, Iarib, Zerah ?i Saul.
1Ch 4:25  Fiul lui Saul a fost ?alum, fiul acestuia a fost Mibsam, iar al acestuia a fost Mi?ma.
1Ch 4:26  Fiii lui Mi?ma au fost: Hamuel, fiul lui; fiul acestuia a fost Zacur, iar al acestuia a fost ?imei.
1Ch 4:27  ?imei a avut ?aisprezece fii ?i ?ase fete, iar fra?ii lui au avut pu?ini copii ?i tot neamul lor n-a fost a?a de numeros ca neamul fiilor lui Iuda.
1Ch 4:28  Ei traiau în Beer-?eba, Molada ?i Ha?ar-?ual,
1Ch 4:29  În Bilha, E?em, Tolad,
1Ch 4:30  Betuel, Horma, ?iclag,
1Ch 4:31  În Bet-Marcabot, Ha?ar-Susim, Bet-Birei ?i ?aaraim. Iata ceta?ile lor dinainte de domnia lui David cu satele lor.
1Ch 4:32  ?i mai aveau: Etam, Ain, Rimon, Tochen ?i A?an, cinci ceta?i,
1Ch 4:33  Cu toate satele lor, care se aflau împrejurul acestor ceta?i pâna la Baal. Iata locurile lor de locuin?a ?i spi?a neamului lor:
1Ch 4:34  Me?obab, Iamlec ?i Io?a, fiul lui Amasia
1Ch 4:35  Ioil ?i Iehu, fiul lui Io?ibia, fiul lui Seraia, fiul lui Asiel;
1Ch 4:36  Elioenai, Iaacoba, Ie?ohaia, Asaia, Adiel, Ie?imiel, Benaia,
1Ch 4:37  Ziza, fiul lui ?ifei, fiul lui Alon, fiul lui Iedaia, fiul lui ?imri, fiul lui ?emaia.
1Ch 4:38  Ace?ti numi?i mai sus au fost capetenii neamurilor lor, iar casa tatalui lor s-a împar?it în multe ramuri.
1Ch 4:39  Ei s-au întins pâna în partea Gherarei ?i pâna în partea de rasarit a vaii Gai, ca sa gaseasca pa?uni pentru turmele lor;
1Ch 4:40  ?i au gasit pa?uni grase ?i bune ?i pamânt larg, lini?tit ?i lipsit de primejdii, pentru ca înainte de ei au trait acolo numai pu?ini Hami?i.
1Ch 4:41  ?i au venit ace?tia, care sunt scri?i pe nume, în zilele lui Iezechia, regele Iudei, ?i au batut pe nomazi ?i pe cei a?eza?i, care se aflau acolo ?i i-au nimicit pentru totdeauna ?i s-au a?ezat în locul lor, caci acolo se aflau pa?uni pentru turmele lor.
1Ch 4:42  Dar din ei, din fiii lui Simeon, s-au dus catre muntele Seir cinci sute de oameni, în frunte cu Pelatia, Nearia, Refaia ?i Uziel, fiii lui I?ei,
1Ch 4:43  ?i au batut rama?i?a de Amaleci?i, ce se mai gasea acolo, ?i traiesc acolo pâna în ziua de astazi.
1Ch 5:1  Fiii lui Ruben, întâiul nascut al lui Israel, caci el a fost nascut întâi, dar pentru ca a întinat el patul tatalui sau, întâietatea lui a fost data fiilor lui Iosif, fiul lui Israel, ca sa nu se mai înscrie ei (fiii lui Ruben) ca întâi nascu?i;
1Ch 5:2  Caci Iuda era cel mai puternic dintre fra?ii sai ?i pova?uitorul e din el, dar întâietatea a trecut la Iosif.
1Ch 5:3  Fiii lui Ruben, întâiul nascut al lui Israel, au fost: Enoh, Palu, He?ron ?i Carmi.
1Ch 5:4  Fiii lui Ioil au fost: ?emaia, fiul lui, fiul acestuia a fost Gog, iar al acestuia a fost ?imei;
1Ch 5:5  Fiul acestuia a fost Mihea, al acestuia a fost Reaia, iar al acestuia a fost Baal;
1Ch 5:6  Fiul acestuia a fost Beera, pe care l-a dus în robie Tiglatfalasar, regele Asiriei. El era capetenia Rubeni?ilor.
1Ch 5:7  ?i fra?ii lui, dupa familiile lor, dupa spira neamului lor, au fost: cel mai însemnat Ioil, apoi Zaharia,
1Ch 5:8  ?i Bela, fiul lui Azaz, fiul lui ?ema, fiul lui Ioil. El locuia în Aroer pâna la Nebo ?i Baal-Meon.
1Ch 5:9  Iar spre rasarit a locuit el pâna la marginea pustiului, care pleaca de la râul Eufrat, pentru ca turmele lui erau foarte multe în ?inutul Galaadului.
1Ch 5:10  În zilele lui Saul au purtat ei razboi cu Agarenii, care au cazut în mâinile lor, ?i au locuit în corturi, în toata latura de rasarit a Galaadului.
1Ch 5:11  Fiii lui Gad traiau în fa?a lor, în ?ara Vasanului, pâna la Salca.
1Ch 5:12  În Vasan, cel mai de seama era Ioil. ?afam era al doilea, apoi venea Iaenai ?i ?afat.
1Ch 5:13  Fra?ii lor cu familiile lor erau în numar de ?apte: Micael, Me?ulam, ?eba, Iorai, Iacan, Zia ?i Eber.
1Ch 5:14  Iata fiii lui Abihail, fiul lui Huri, fiul lui Iaroah, fiul lui Galaad, fiul lui Micael, fiul lui Ie?i?ai, fiul lui Iahdo, fiul lui Buz.
1Ch 5:15  Ahi, fiul lui Abdiel, fiul lui Guni, era capul neamului sau.
1Ch 5:16  Ei traiau în Galaad, în Vasan ?i în ceta?ile care ?ineau de el în toate împrejurimile ?irionului, pâna la capatul lor.
1Ch 5:17  Ei cu to?ii au fost numara?i în zilele lui Ioatam, regele Iudei, ?i în zilele lui Ieroboam, regele lui Israel.
1Ch 5:18  Urma?ii lui Ruben, ai lui Gad ?i o jumatate din semin?ia lui Manase aveau oameni razboinici, barba?i care purtau scut ?i sabie, care trageau cu arcul ?i deprin?i la lupta, patruzeci ?i patru de mii ?apte sute ?aizeci care ie?eau la razboi.
1Ch 5:19  ?i s-au luptat ei cu Agarenii, cu Ietur, cu Nafis ?i cu Nodab.
1Ch 5:20  Dar li s-a dat ajutor contra acelora ?i au fost da?i în mâna lor Agarenii ?i toate ale lor, pentru ca ei în vremea luptei au strigat catre Dumnezeu ?i El i-a auzit, pentru ca ei nadajduiau în El.
1Ch 5:21  Atunci au luat ei turmele acelora: cincizeci de mii de camile, doua sute cincizeci de mii de oi ?i capre, doua mii de asini ?i o suta de mii de oameni,
1Ch 5:22  Pentru ca mul?i au cazut uci?i, caci lupta aceasta a fost de la Dumnezeu. ?i au locuit ei în locul acelora pâna la ducerea în robie.
1Ch 5:23  Urma?ii jumata?ii din tribul lui Manase au trait în acel pamânt de la Vasan pâna la Baal-Ermon ?i Senir ?i pâna la muntele Hermon, ?i erau mul?i la numar.
1Ch 5:24  Iata capii de familie cei mai însemna?i ai lor: Efer, I?ei, Eliel, Azriel, Ieremia, Hodavia ?i Iahdiel, barba?i puternici, barba?i vesti?i, capetenii ale neamurilor lor.
1Ch 5:25  Dar când ei au gre?it împotriva Dumnezeului parin?ilor lor ?i au început sa se desfrâneze dupa dumnezeii popoarelor pamântului aceluia, pe care le stârpise Dumnezeu de la fa?a lor,
1Ch 5:26  Atunci Dumnezeul lui Israel a întarâtat duhul lui Ful, regele Asiriei, adica al lui Tiglatfalasar, regele Asiriei ?i acesta a stramutat pe Rubeni?i ?i pe Gadi?i ?i jumatate din tribul lui Manase ?i i-a dus în Halach, Habor, Hara ?i la râul Gozan, unde sunt pâna astazi.
1Ch 6:1  Fiii lui Levi sunt: Gher?om, Cahat ?i Merari.
1Ch 6:2  Fiii lui Cahat sunt: Amram, I?har, Hebron ?i Uziel.
1Ch 6:3  Copiii lui Amram sunt: Aaron, Moise ?i Mariam. Fiii lui Aaron sunt: Nadab, Abiud, Eleazar ?i Itamar.
1Ch 6:4  Lui Eleazar i s-a nascut Finees, lui Finees i s-a nascut Abi?ua;
1Ch 6:5  Lui Abi?ua i s-a nascut Buchi, iar lui Buchi i s-a nascut Uzi;
1Ch 6:6  Lui Uzi i s-a nascut Zerahia, iar lui Zerahia i s-a nascut Meraiot.
1Ch 6:7  Lui Meraiot i s-a nascut Amaria, iar lui Amaria i s-a nascut Ahitub;
1Ch 6:8  Lui Ahitub i s-a nascut ?adoc, iar lui ?adoc i s-a nascut Ahimaa?;
1Ch 6:9  Lui Ahimaa? i s-a nascut Azaria, iar lui Azaria i s-a nascut Iohanan;
1Ch 6:10  Lui Iohanan. i s-a nascut Azaria; acesta e acela care a fost preot la templul zidit de Solomon în Ierusalim.
1Ch 6:11  Lui Azaria i s-a nascut Amaria, iar lui Amaria i s-a nascut Ahitub;
1Ch 6:12  Lui Ahitub i s-a nascut ?adoc, iar lui ?adoc i s-a nascut ?alum;
1Ch 6:13  Lui ?alum i s-a nascut Hilchia, iar lui Hilchia i s-a nascut Azaria;
1Ch 6:14  Lui Azaria i s-a nascut Seraia, iar lui Seraia i s-a nascut Iehosadac.
1Ch 6:15  Iehosadac a mers în robie când Domnul a stramutat pe cei din Iuda ?i pe cei din Ierusalim prin mâna lui Nabucodonosor.
1Ch 6:16  Astfel fiii lui Levi au fost: Gher?om, Cahat ?i Merari.
1Ch 6:17  Iata numele fiilor lui Gher?om: Libni ?i ?imei.
1Ch 6:18  Fiii lui Cahat au fost: Amram, I?har, Hebron ?i Uziel.
1Ch 6:19  Fiii lui Merari au fost: Mahli ?i Mu?i. Iata urma?ii lui Levi dupa neamurile lor.
1Ch 6:20  Gher?om a avut pe Libni, fiul lui; pe Iahat, fiul lui, ?i pe Zima, fiul lui;
1Ch 6:21  Pe Ioah, fiul lui; pe Ido, fiul lui; pe Zerah, fiul lui, ?i pe Ieatrai, fiul lui.
1Ch 6:22  Fiii lui Cahat au fost: Aminadab, fiul lui; Core, fiul lui, ?i Asir, fiul lui;
1Ch 6:23  Elcana, fiul lui; Ebiasaf, fiul lui, ?i Asir, fiul lui;
1Ch 6:24  Tahat, fiul lui, Uriel, fiul lui; Uzia, fiul lui, ?i Saul, fiul lui.
1Ch 6:25  Fiii lui Elcana sunt: Amasai ?i Ahimot,
1Ch 6:26  Elcana, fiul lui; ?ofai fiul lui, ?i Nahat, fiul lui,
1Ch 6:27  Eliab, fiul lui; Ieroham, fiul lui; Elcana, fiul lui; Samuel, fiul lui.
1Ch 6:28  Fiii lui Samuel au fost: întâiul nascut Ioil, al doilea, Abia.
1Ch 6:29  Fiii lui Merari au fost: Mahli, Libni, fiul lui; ?imei, fiul lui; Uza, fiul lui;
1Ch 6:30  ?imea, fiul lui; Aghia, fiul lui, ?i Asaia, fiul lui.
1Ch 6:31  Iata cei pe care David i-a pus capetenii peste cântare?i în casa Domnului, în timpul când a a?ezat în ea chivotul legii,
1Ch 6:32  Care au servit de cântare?i înaintea cortului adunarii, pâna când Solomon a zidit templul Domnului în Ierusalim, ?i care fusesera rândui?i la slujba lor dupa rânduiala lor;
1Ch 6:33  Iata pe cei care au fost rândui?i cu fiii lor: din fiii lui Cahat: Heman cântare?ul, fiul lui Ioil, fiul lui Samuel,
1Ch 6:34  Fiul lui Elcana, fiul lui Ieroham, fiul lui Eliel, fiul lui Toah,
1Ch 6:35  Fiul lui ?uf, fiul lui Elcana, fiul lui Mahat, fiul lui Amasai,
1Ch 6:36  Fiul lui Elcana, fiul lui Ioil, fiul lui Azaria, fiul lui ?efania,
1Ch 6:37  Fiul lui Tahat, fiul lui Asir, fiul lui Abiasaf, fiul lui Core,
1Ch 6:38  Fiul lui I?har, fiul lui Cahat, fiul lui Levi, fiul lui Israel.
1Ch 6:39  ?i fratele sau Asaf, care statea în partea dreapta a lui, adica Asaf, fiul lui Berechia, fiul lui ?imea,
1Ch 6:40  Fiul lui Micael, fiul lui Baaseia, fiul lui Malchia,
1Ch 6:41  Fiul lui Etni, fiul lui Zerah, fiul lui Adaia,
1Ch 6:42  Fiul lui Etan, fiul lui Zima, fiul lui ?imei,
1Ch 6:43  Fiul lui Iahat, fiul lui Gher?om, fiul lui Levi.
1Ch 6:44  Iar din fiii lui Merari, fra?ii lor, au fost în partea stânga: Etan, fiul lui Chi?i, fiul lui Abdi, fiul lui Maluc,
1Ch 6:45  Fiul lui Ha?abia, fiul lui Amasia, fiul lui Hilchia,
1Ch 6:46  Fiul lui Am?i, fiul lui Bani, fiul lui ?emer,
1Ch 6:47  Fiul lui Mahli, fiul lui Mu?i, fiul lui Merari, fiul lui Levi.
1Ch 6:48  Fra?ii lor levi?i erau rândui?i la tot felul de slujbe, la casa Domnului.
1Ch 6:49  Iar Aaron ?i fiii lor ardeau pe jertfelnic arderi de tot ?i tamâie pe altarul tamâierii, savâr?ind toate slujbele sfinte în Sfânta Sfintelor ?i pentru ispa?irea lui Israel, în toate, cum poruncise robul lui Dumnezeu Moise.
1Ch 6:50  Iata fiii lui Aaron: Eleazar, fiul lui; Finees, fiul lui; Abi?ua, fiul lui;
1Ch 6:51  Buchi, fiul lui; Uzi, fiul lui; Zerahia, fiul lui;
1Ch 6:52  Meraiot, fiul lui; Amaria, fiul lui; Ahitub, fiul lui;
1Ch 6:53  ?adoc, fiul lui; Ahimaa?, fiul lui.
1Ch 6:54  Iata locuin?ele lor dupa satele lor în hotarele lor: fiilor lui Aaron din familia lui Cahat, dupa cum le-a cazut sor?ul,
1Ch 6:55  Li s-au dat Hebronul, în pamântul lui Iuda ?i împrejurimile lui,
1Ch 6:56  Iar ?arinile acestei ceta?i ?i satele ei s-au dat lui Caleb, fiul lui Iefonie.
1Ch 6:57  Fiilor lui Aaron li s-au dat de asemenea ora?ele de scapare: Hebron ?i Libna ?i împrejurimile lor, Iatir ?i E?temoa ?i ?inuturile lor,
1Ch 6:58  Hilenul (Holonul) ?i pa?unile lui, Debirul ?i pa?unile lui;
1Ch 6:59  A?anul (Ainul) ?i împrejurimile lui, Bet?eme?ul ?i împrejurimile lui;
1Ch 6:60  Iar de la tribul lui Veniamin: Gheba ?i pa?unile ei, Alemetul (Almonul) ?i împrejurimile lui, Anatotul ?i ?inuturile lui; ceta?ile familiilor lor erau de toate treisprezece ceta?i.
1Ch 6:61  Celorlal?i fii ai lui Cahat, din familiile acestui trib, li s-au dat, dupa sor?i, zece ceta?i din hotarele jumata?ii tribului lui Manase.
1Ch 6:62  Fiilor lui Gher?om, dupa familiile lor, li s-au dat treisprezece ceta?i din tribul lui Isahar, din tribul lui A?er, din tribul lui Neftali ?i din tribul lui Manase, în Vasan.
1Ch 6:63  Fiilor lui Merari, dupa familiile lor, li s-au dat prin sor?i douasprezece ceta?i din tribul lui Ruben, din tribul lui Gad ?i din tribul lui Zabulon.
1Ch 6:64  A?a au dat fiii lui Israel Levi?ilor ceta?i cu împrejurimile lor.
1Ch 6:65  Li s-au dat prin sor?i din tribul fiilor lui Iuda, din tribul fiilor lui Simeon ?i din tribul fiilor lui Veniamin acele ceta?i pe care ei le-au numit pe nume.
1Ch 6:66  Iar unora din familiile fiilor lui Cahat li s-au dat ceta?i din tribul lui Efraim.
1Ch 6:67  Li s-au dat ceta?ile de scapare: Sichemul ?i împrejurimile lui, pe muntele Efraim ?i Ghezerul cu împrejurimile lui;
1Ch 6:68  Iocmeamul (Chib?oimul) cu împrejurimile lui ?i Bethoronul cu împrejurimile lui;
1Ch 6:69  Aialonul cu împrejurimile lui ?i Gat-Rimonul cu împrejurimile lui.
1Ch 6:70  Din jumatatea tribului lui Manase li s-au dat: Anerul cu împrejurimile lui, Bileanul cu împrejurimile lui. Acestea sunt locuin?ele pentru ceilal?i fii ai lui Cahat.
1Ch 6:71  Fiilor lui Gher?om din familiile din jumatatea tribului lui Manase li s-au dat Golanul în Vasan cu împrejurimile lui ?i A?tarotul cu împrejurimile lui.
1Ch 6:72  Din tribul lui Isahar li s-au dat Chede?ul (Chi?ionul) cu împrejurimile lui, Dobratul cu împrejurimile lui,
1Ch 6:73  Ramotul cu împrejurimile lui ?i Anemul cu împrejurimile lui.
1Ch 6:74  Din tribul lui A?er li s-au dat: Ma?alul cu împrejurimile lui ?i Abdonul cu împrejurimile lui;
1Ch 6:75  Hucocul (Helcotul) cu ?inutul lui ?i Rehobul cu ?inutul lui;
1Ch 6:76  Din tribul lui Neftali li s-au dat Chede?ul în Galileea, cu ?inutul lui, Hamonul cu ?inutul lui ?i Chiriataimul cu ?inutul lui.
1Ch 6:77  Iar celorlal?i fii ai lui Merari li s-au dat: din tribul lui Zabulon, Rimonul cu ?inutul lui ?i Taborul cu ?inutul lui,
1Ch 6:78  Iar dincolo de Iordan, în fa?a Ierihonului, la rasarit de Iordan, li s-a dat în tribul lui Ruben: Be?erul, în pustiu, cu ?inutul lui, ?i Iah?a cu ?inutul ei,
1Ch 6:79  Chedemotul cu împrejurimile lui ?i Mefaatul cu ?inutul lui.
1Ch 6:80  Din tribul lui Gad li s-au dat: Ramotul în Galaad cu ?inutul lui ?i Mahanaimul cu ?inutul lui;
1Ch 6:81  He?bonul ?i Iazerul cu ?inuturile lor.
1Ch 7:1  Fiii lui Isahar au fost patru: Tola, Pua, Ia?ub ?i ?imron.
1Ch 7:2  Fiii lui Tola au fost: Uzi, Refaia, Ieriil, Iahmai, Ibsam ?i, Samuel; ace?tia sunt cei mai de seama în neamul lui Tola, oameni razboinici în neamul lor; numarul lor în zilele lui David era douazeci ?i doua de mii ?i ?ase sute.
1Ch 7:3  Fiul lui Uzi a fost Izrahia; iar fiii lui Izrahia au fost: Micael, Obadia, Ioil ?i I?ia, de to?i cinci. To?i ace?tia sunt capetenii.
1Ch 7:4  Ei, dupa familiile lor ?i dupa neamurile lor, aveau o?tire de treizeci ?i ?ase de mii de oameni, pentru ca ei au avut multe femei ?i mul?i copii.
1Ch 7:5  Iar fra?ii lor, în toate neamurile lui Isahar, aveau oameni de lupta optzeci ?i ?apte de mii, înscri?i în tabli?ele cele cu spi?a neamului.
1Ch 7:6  Veniamin a avut trei: pe Bela, Becher ?i Iediael (A?bel).
1Ch 7:7  Fiii lui Bela au fost cinci: E?bon, Uzi, Uziel, Ierimot ?i Iri, to?i capetenii de familii, oameni razboinici. În tabli?ele cu spi?a neamului sunt înscri?i douazeci ?i doua de mii treizeci ?i patru.
1Ch 7:8  Fiii lui Becher au fost: Zemira, Ioa?, Eliezer, Elioenai, Omri, Ieremot, Abia, Anatot ?i Alemet; to?i ace?tia sunt fiii lui Becher.
1Ch 7:9  În tabli?ele cu spi?a neamului sunt înscri?i din ace?tia, dupa familiile ?i dupa neamurile lor, oameni razboinici douazeci de mii ?i doua sute.
1Ch 7:10  Fiul lui Iediael (A?bel) a fost Bilhan. Fiii lui Bilhan au fost: Ieu?, Veniamin, Ehud, Chenaana, Zetan, Tar?i? ?i Ahi?ahar.
1Ch 7:11  To?i ace?ti fii ai lui Iediael (A?bel), au fost capi de familie, oameni razboinici; ?aptesprezece mii ?i doua sute erau în stare de a ie?i la razboi.
1Ch 7:12  ?upim ?i Hupim erau fiii lui Ir, iar Hu?im era fiul lui Aher.
1Ch 7:13  Fiii lui Neftali au fost: Iah?iel, Guni, Ie?er ?i ?alum (Silem), copiii Bilhei.
1Ch 7:14  Fiii lui Manase au fost: Asriel, pe care l-a nascut concubina sa arameana; tot aceasta a nascut pe Machir, tatal lui Galaad.
1Ch 7:15  Machir ?i-a luat de femeie pe sora lui Hupim ?i a lui ?upim, al carei nume era Maaca. Numele fiului al doilea a fost Salfaad. Salfaad a avut numai fete.
1Ch 7:16  Maaca, femeia lui Machir, a nascut un fiu ?i i-a pus numele Pere?, iar numele fratelui lui era ?ere?. Fiii acestuia au fost Ulam ?i Rechem.
1Ch 7:17  Fiul lui Ulam a fost Bedan. Ace?tia sunt fiii lui Galaad, fiul lui Machir, fiul lui Manase.
1Ch 7:18  Sora sa, Molechet, a nascut pe I?hod, pe Abiezer ?i pe Mahla.
1Ch 7:19  Fiii lui ?emida au fost: Ahian, ?echem, Lichi ?i Aniam.
1Ch 7:20  Fiii lui Efraim au fost: ?utelah, Bered, fiul lui; Tahat, fiul lui; Eleadah, fiul lui ?i Tahat, fiul lui;
1Ch 7:21  Zabad, fiul lui; ?utelah, fiul lui; Ezer ?i Elead. Pe ace?tia i-au ucis locuitorii din Gat, ba?tina?ii jarii aceleia, pentru ca ei se dusesera sa le apuce turmele lor.
1Ch 7:22  Dupa ei a plâns Efraim, tatal lor, zile multe ?i au venit fra?ii lui sa-l mângâie.
1Ch 7:23  Apoi a intrat el la femeia sa ?i ea a zamislit ?i a nascut un fiu ?i el i-a pus numele Beria, pentru ca nenorocirea se atinsese de casa lui.
1Ch 7:24  ?i a avut el ?i o fata: ?eera. Aceasta a zidit Bethoronul de jos ?i de sus ?i Uzen-?eera.
1Ch 7:25  Refah, fiul sau, ?i Re?ef, fiul sau; Telah, fiul sau, ?i Tahan, fiul sau.
1Ch 7:26  Ladan, fiul sau; Amiud, fiul sau, ?i Eli?ama, fiul sau.
1Ch 7:27  Non, fiul sau; Iosua, fiul sau.
1Ch 7:28  Mo?iile lor ?i locurile lor de locuit au fost: Betelul ?i ceta?ile care ?ineau de el, spre rasarit Naaranul, spre apus Ghezerul ?i ceta?ile care ?ineau de el; Sichemul ?i ceta?ile care ?ineau de el pâna la Gaza ?i ceta?ile ce ?ineau de aceasta.
1Ch 7:29  Iar din partea fiilor lui Manase: Bet-?eanul ?i ceta?ile ce ?ineau de el, Taanacul ?i ceta?ile ce ?ineau de el, Meghidonul ?i ceta?ile ce ?ineau de el, Dorul ?i ceta?ile ce ?ineau de el. în ele locuiau fiii lui Iosif, fiul lui Israel.
1Ch 7:30  Fiii lui A?er erau: Imna, I?va, I?vi ?i Beria ?i sora lor Serah.
1Ch 7:31  Fiii lui Beria au fost: Heber ?i Malchiel. Acesta e tatal lui Birzait.
1Ch 7:32  Heber a avut fii pe Iaflet, ?emer ?i Hotam ?i pe sora lor ?ua.
1Ch 7:33  Fiii lui Iaflet au fost: Pasac, Bimhal ?i A?vat. Ace?tia sunt fiii lui Iaflet.
1Ch 7:34  Fiii lui ?emer au fost: Ahi, Rohga, Huba ?i Aram.
1Ch 7:35  Fiii lui Helem, fratele lui, au fost: ?ofah, Imna, ?ele? ?i Amal.
1Ch 7:36  Fiii lui ?ofah au fost: Suah, Harnefer, ?ual, Beri, Imra,
1Ch 7:37  Be?er, Hod, ?ama, ?il?a, Itran ?i Beera.
1Ch 7:38  Fiii lui Ieter au fost: Iefune, Pispa ?i Ara.
1Ch 7:39  Fiii nascu?i din Ula au fost: Arah, Haniel ?i Ri?ia.
1Ch 7:40  To?i ace?tia sunt fiii lui A?er, capi de familie, oameni ale?i, razboinici, capetenii de mâna întâi. În tabli?ele lor cu spi?a neamului sunt înscri?i în o?tire pentru razboi un numar de douazeci ?i ?ase de mii de oameni.
1Ch 8:1  Lui Veniamin i s-a nascut Bela, întâiul sau nascut; al doilea, A?bel, al treilea, Ahrah,
1Ch 8:2  Al patrulea, Noha ?i al cincilea, Rafa.
1Ch 8:3  Fiii lui Bela au fost: Adar, Ghera, Abiud,
1Ch 8:4  Abi?ua, Naaman, Ahoah,
1Ch 8:5  Ghera, ?efufan ?i Huram.
1Ch 8:6  Iata fiii lui Ehud, care au fost capi de familii, care au trait în Gheba ?i au fost stramuta?i în Manahat:
1Ch 8:7  Naaman, Ahia ?i Ghera, care i-a stramutat, a nascut pe Uza ?i pe Ahiud.
1Ch 8:8  Lui ?aharaim i s-au nascut copii în ?ara Moabi?ilor, dupa ce a dat drumul femeilor sale, Hu?im ?i Baara.
1Ch 8:9  I s-au nascut din Hode?a, femeia sa: Iobab, ?ibia, Me?a ?i Malcam;
1Ch 8:10  Ieu?, ?achia ?i Mirma. Ace?tia sunt fiii lui, capi de familie.
1Ch 8:11  Din Hu?im i s-au nascut Abitub ?i Elpaal.
1Ch 8:12  Fiii lui Elpaal au fost: Eber, Mi?eam ?i ?emed, care au zidit Ono ?i Lodul ?i ceta?ile ce ?ineau de el,
1Ch 8:13  Precum ?i Beria ?i Sema. - Ace?tia au fost capii familiilor locuitorilor Aialonului; ei au alungat pe locuitorii din Gat. -
1Ch 8:14  Ahio, ?a?ac, ?i Iremot;
1Ch 8:15  Zebadia, Arad ?i Eder;
1Ch 8:16  Micael, I?pa ?i Ioha, fiii lui Beria.
1Ch 8:17  Zecadia, Me?ulam, Hizchi ?i Heber,
1Ch 8:18  Ie?mere, Izliah ?i Iobab, fiii lui Elpaal.
1Ch 8:19  Iachim, Zicri ?i Zabdi,
1Ch 8:20  Elienai, ?iltai ?i Eliel,
1Ch 8:21  Adaia, Beraia ?i ?imrat, fiii lui ?imei.
1Ch 8:22  I?pan, Eber ?i Eliel,
1Ch 8:23  Abdon, Zicri ?i Hanan,
1Ch 8:24  Hanania, Elam ?i Antotia,
1Ch 8:25  Ifdia ?i Fanuil, fiii lui ?i?ac.
1Ch 8:26  ?am?erai, ?eharia ?i Atalia,
1Ch 8:27  Iaare?ia, Elia ?i Zicri, fiii lui Ieroham.
1Ch 8:28  Acestea sunt capeteniile familiilor mai de seama ale neamului lor. Ei au locuit în Ierusalim.
1Ch 8:29  În Ghibeon a trait Ieguel, tatal Ghibeoni?ilor. Numele femeii lui a fost Maaca
1Ch 8:30  ?i fiul lui, întâiul nascut, Abdon, dupa el ?ur, Chi?, Baal, Nadab, Ner,
1Ch 8:31  Ghedeor, Ahio, Zecher ?i Miclot;
1Ch 8:32  Lui Miclot i s-a nascut ?imea. ?i au trait ei împreuna cu fra?ii lor în Ierusalim.
1Ch 8:33  Lui Ner i s-a nascut Chi?; lui Chi? i s-a nascut Saul; lui Saul i s-a nascut Ionatan, Melchi?ua, Aminadab ?i E?baal.
1Ch 8:34  Fiul lui Ionatan a fost Meribaal; (Mefibo?et); a avut de fiu pe Mica.
1Ch 8:35  Fiii lui Mica au fost: Piton, Melec, Tarea ?i Ahaz.
1Ch 8:36  Lui Ahaz i s-a nascut Iehoada, lui Iehoada i s-a nascut Alemet, Asmavet ?i Zimri; lui Zimri i s-a nascut Mo?a;
1Ch 8:37  Lui Mo?a i s-a nascut Binea; Rafa, fiul lui; Eleasa, fiul lui; A?el, fiul lui.
1Ch 8:38  A?el a avut ?ase feciori ?i iata numele lor: Azricam, Bocru, Ismael, ?earia, Obadia ?i Hanan. To?i ace?tia sunt fiii lui A?el.
1Ch 8:39  Fiii lui E?ec, fratele sau, au fost: Ulam, întâiul sau nascut; al doilea, Ieu?; al treilea Elifelet.
1Ch 8:40  Fiii lui Ulam au fost oameni razboinici, tragatori din arc, având mul?i copii ?i nepo?i pâna la o suta cincizeci. To?i ace?tia sunt din fiii lui Veniamin.
1Ch 9:1  A?a au fost numara?i dupa neamurile lor to?i Israeli?ii ?i iata sunt înscri?i în cartea regilor lui Israel. Iar Iudeii, pentru faradelegile lor, au fost du?i în Babilon.
1Ch 9:2  Cei dintâi locuitori care au trait în pamânturile lor, prin ora?ele lui Israel, au fost Israeli?ii, preo?ii, levi?ii ?i cei afierosi?i templului.
1Ch 9:3  În Ierusalim au trait unii din fiii lui Iuda, din fiii lui Veniamin ?i din fiii lui Efraim ?i ai lui Manase.
1Ch 9:4  Utai, fiul lui Amihud, fiul lui Omri, fiul lui Imri, fiul lui Bani, din fiii lui Fares, fiul lui Iuda;
1Ch 9:5  Din fiii lui ?iloni: Asaia, întâiul nascut, ?i fiii acestuia.
1Ch 9:6  Din fiii lui Zerah: Ieuel ?i fra?ii lui, ?ase sute nouazeci;
1Ch 9:7  Din fiii lui Veniamin: Salu, fiul lui Me?ulam, fiul lui Hodavia, fiul lui Asenua
1Ch 9:8  ?i Ibneia, fiul lui Ieroham, ?i Ela, fiul lui Uzi, fiul lui Micri, ?i Me?ulam, fiul lui ?efatia, fiul lui Reguel, fiul lui Ibneia
1Ch 9:9  ?i fra?ii lor, dupa neamurile lor: noua sute cincizeci ?i ?ase. Ace?ti barba?i erau capi de familie în neamul lor.
1Ch 9:10  Iar dintre preo?i: Iedaia, Ioiarib, Iachin
1Ch 9:11  ?i Azaria, fiul lui Hilchia, fiul lui Me?ulam, fiul lui ?adoc, fiul lui Meraiot, fiul lui Ahitub, capetenia în casa lui Dumnezeu;
1Ch 9:12  Adaia, fiul lui Ieroham, fiul lui Pa?hur, fiul lui Malchia; ?i Maesai, fiul lui Adiel, fiul lui Iahzera, fiul lui Me?ulam, fiul lui Me?ilemit, fiul lui Imer.
1Ch 9:13  ?i fra?ii lor, capi în familiile lor: o mie ?apte sute ?aizeci, barba?i de isprava la slujbele din casa Domnului.
1Ch 9:14  Iar din levi?i: ?emaia, fiul lui Ha?ub, fiul lui Azricam, fiul lui Ha?abia; ace?tia sunt din fiii lui Merari.
1Ch 9:15  Bacbacar, Here?, Galal ?i Matania, fiul lui Mica, fiul lui Zicri, fiul lui Asaf,
1Ch 9:16  Obadia, fiul lui ?emaia, fiul lui Galal, fiul lui Iedutun; Berechia, fiul lui Asa, fiul lui Elcana, care locuia în satele Netofati?ilor.
1Ch 9:17  Iar din portari: ?alum, Acub, Talmon ?i Ahiman ?i fra?ii lor; ?alum era capetenie.
1Ch 9:18  Ace?ti portari fac straja fiilor levi?ilor pâna astazi la por?ile regale cele de la rasarit.
1Ch 9:19  ?alum, fiul lui Core, fiul lui Ebiasaf, fiul lui Corah, ?i fra?ii lui cei din neamul lui Corah, dupa datoria slujbei lor, aveau paza cortului, iar parin?ii lor pazeau intrarea în tabara Domnului.
1Ch 9:20  Finees, fiul lui Eleazar, fusese înainte capetenie peste ei ?i Domnul era cu el.
1Ch 9:21  Zaharia, fiul lui Me?elemia, era portar la u?a cortului adunarii.
1Ch 9:22  To?i cei ale?i ca portari la praguri erau doua sute doisprezece. Ei erau înscri?i în registre dupa a?ezarile lor. Pe ei îi pusese David ?i Samuel înainte-vazatorul, pentru credincio?ia lor.
1Ch 9:23  ?i ei ?i fiii lor ?ineau straja la u?ile casei Domnului, la casa cortului.
1Ch 9:24  În patru laturi se aflau portari: la rasarit, la apus, la miazanoapte ?i la miazazi.
1Ch 9:25  Iar fra?ii lor traiau în sala?urile lor, venind la ei din timp în timp, pentru ?apte zile.
1Ch 9:26  Aceste patru capetenii de portari levi?i erau de încredere ?i tot ei aveau grija casei Domnului ?i vistieriei ei.
1Ch 9:27  Împrejurul casei lui Dumnezeu ei petreceau ?i noaptea, caci asupra lor era lasata paza ?i trebuia sa deschida în fiecare diminea?a u?ile.
1Ch 9:28  Unii din ei erau pu?i de paza la vasele de slujba, a?a ca ei cu numar le primeau ?i cu numar le dadeau.
1Ch 9:29  Altora din ei le era încredin?ata cealalta zestre ?i toate lucrurile trebuitoare pentru cele sfinte: faina cea mai buna, vinul, untdelemnul ?i tamâia cea mirositoare.
1Ch 9:30  Iar dintre fiii preo?ilor unii pregateau mir cu aromate.
1Ch 9:31  Matatia levitul, care era întâiul nascut al lui ?alum, fiul lui Core, era pus sa aiba grija de coptul aluaturilor în tigai.
1Ch 9:32  Unora din fra?ii lor, din fiii lui Cahat, le era încredin?ata pregatirea pâinilor punerii înainte, ca sa le puna în fiecare zi de odihna.
1Ch 9:33  Iar cântare?ii, cei mai de seama din neamul Levi?ilor, erau liberi de ocupa?ii în camarile templului, pentru ca ziua ?i noaptea erau îndatora?i sa se îndeletniceasca cu cântarea.
1Ch 9:34  Ace?tia erau cei mai de seama între familiile levi?ilor ?i locuiau în Ierusalim.
1Ch 9:35  În Ghibeon locuiau: tatal Ghibeoni?ilor, Ieguel, a carui femeie se numea Maaca;
1Ch 9:36  ?i fiul lui, întâiul nascut, se numea Abdon; dupa el venea ?ur, Chi?, Baal, Ner, Nadab,
1Ch 9:37  Ghedor, Ahio, Zaharia ?i Miclot.
1Ch 9:38  Lui Miclot i s-a nascut ?imeam. ?i ei traiau lânga fra?ii lor, în Ierusalim, împreuna cu fra?ii lor.
1Ch 9:39  Lui Ner i s-a nascut Chi?; lui Chi? i s-a nascut Saul, lui Saul i s-a nascut Ionatan, Melchi?ua, Aminadab ?i E?baal.
1Ch 9:40  Fiul lui Ionatan a fost Meribaal; lui Meribaal i s-a nascut Mica.
1Ch 9:41  Fiii lui Mica au fost: Piton, Melec, Tareia ?i Ahaz.
1Ch 9:42  Lui Ahaz i s-a nascut Iara; lui Iara i s-au nascut Alemet, Asmavet ?i Zimri; lui Zimri i s-a nascut Mo?a.
1Ch 9:43  Lui Mo?a i s-a nascut Binea; Refaia, fiul lui; Eleasa, fiul lui; A?el, fiul lui.
1Ch 9:44  Lui A?el i s-au nascut ?ase fii ?i iata numele lor: Azricam, Bocru, Ismael, ?earia, Obadia ?i Hanan. Ace?tia sunt fiii lui A?el.
1Ch 10:1  În vremea aceea Filistenii s-au ridicat cu razboi asupra lui Israel ?i Israeli?ii au fugit de Filisteni ?i au cazut birui?i pe muntele Ghilboa.
1Ch 10:2  Atunci au alergat Filistenii dupa Saul ?i dupa fiii lui ?i au ucis Filistenii pe Ionatan, pe Aminadab ?i pe Melchi?ua, fiii lui Saul.
1Ch 10:3  Iar lupta împotriva lui Saul s-a înte?it ?i arca?ii au tras asupra lui, a?a ca el a fost ranit de sage?i.
1Ch 10:4  Saul a zis purtatorului sau de arme: "Scoate sabia ta ?i ma strapunge cu ea, ca sa nu vina ace?ti netaia?i împrejur ?i sa-?i bata joc de mine". Dar purtatorul de arme nu s-a hotarât la aceasta, pentru ca se speriase foarte tare. Atunci Saul a luat sabia ?i s-a aruncat în ea.
1Ch 10:5  Vazând purtatorul de arme ca Saul a murit, s-a aruncat ?i el în sabia sa.
1Ch 10:6  A?a a murit Saul cu cei trei fii ai lui ?i toata casa a murit împreuna cu el.
1Ch 10:7  Când au vazut Israeli?ii, care erau în vale, ca fug to?i ?i ca Saul ?i fiii lui au murit, au lasat ceta?ile lor ?i au fugit în toate par?ile, iar Filistenii au venit ?i s-au a?ezat în ele.
1Ch 10:8  A doua zi au venit Filistenii sa ridice pe cei uci?i ?i, gasind pe Saul ?i pe fiii lui cazu?i pe muntele Ghilboa,
1Ch 10:9  L-au dezbracat ?i i-au luat capul ?i armele ?i au trimis prin ?ara Filistenilor, sa se vesteasca aceasta înaintea idolilor lor ?i înaintea poporului.
1Ch 10:10  Armele lui le-au pus în templul zeilor lor, iar capul lui l-au spânzurat în templul lui Dagon.
1Ch 10:11  Atunci auzind tot Iabe?ul Galaadului ce au facut Filistenii cu Saul,
1Ch 10:12  S-au ridicat to?i oamenii de lupta, au luat trupul lui Saul ?i trupurile fiilor lui, le-au dus în Iabe? ?i au îngropat oasele lor sub un stejar în Iabe? ?i au postit ?apte zile.
1Ch 10:13  A?a a murit Saul pentru nelegiuirea sa pe care o facuse el înaintea Domnului, pentru ca n-a pazit cuvântul Domnului ?i pentru ca a întrebat ?i a cercetat o vrajitoare
1Ch 10:14  ?i nu a cercetat pe Domnul. De aceea a ?i fost el omorât ?i domnia a fost data lui David, fiul lui Iesei.
1Ch 11:1  Dupa aceea s-au adunat to?i Israeli?ii la David în Hebron zicând: "Iata noi suntem oasele tale ?i carnea ta.
1Ch 11:2  ?i mai înainte, când Saul era înca rege, tu ai pova?uit pe Israel la razboi ?i l-ai adus teafar înapoi ?i Domnul Dumnezeul tau ?i-a spus: Tu vei pa?te pe poporul Meu Israel ?i tu vei fi pova?uitorul poporului Meu Israel".
1Ch 11:3  ?i au venit toate capeteniile lui Israel la rege în Hebron ?i a încheiat cu ei David legamânt în Hebron înaintea fe?ei Domnului; ?i au uns pe David de rege peste Israel, dupa cuvântul Domnului care fusese prin Samuel.
1Ch 11:4  Apoi s-a dus David ?i tot Israelul la Ierusalim, adica la Iebus. Acolo însa erau Iebuseii, locuitorii ?arii aceleia.
1Ch 11:5  ?i au zis locuitorii Iebusului catre David: "Nu vei intra aici!" Dar David a luat cetatea Sionului. Aceasta este cetatea lui David.
1Ch 11:6  Apoi a zis David: "Cine va lovi cel dintâi pe Iebuseu, acela va fi cap ?i capetenie peste o?tire". ?i s-a sculat înainte de to?i Ioab, fiul ?eruiei, ?i s-a facut capetenie.
1Ch 11:7  David a locuit în cetatea aceea ?i ea s-a ?i numit cetatea lui David.
1Ch 11:8  El a zidit cetatea împrejur, începând de la Milo, iar Ioab a reînnoit celelalte par?i ale ceta?ii.
1Ch 11:9  Dupa aceea a propa?it David ?i s-a ridicat din ce în ce mai mult ?i Domnul Savaot era cu. el.
1Ch 11:10  Iata cei mai de seama dintre puternicii lui David, care s-au luptat tare împreuna cu el, în domnia lui, împreuna cu tot Israelul, ca sa întareasca domnia lui asupra lui Israel, dupa cuvântul Domnului.
1Ch 11:11  ?i iata numarul vitejilor pe care i-a avut David: Ia?obeam (Io?eb-Ba?ebet), fiul lui Hacmoni, cel mai de seama între cei treizeci; el ?i-a ridicat suli?a asupra a trei sute de oameni ?i i-a ucis dintr-o data.
1Ch 11:12  Dupa el vine Eleazar, fiul lui Dodo Ahohitul, unul din cei trei viteji.
1Ch 11:13  Acesta a fost cu David la Pasdamim, unde se adunasera Filistenii pentru razboi. Acolo, parte din câmp era semanat cu orz ?i Israeli?ii au fugit de Filisteni;
1Ch 11:14  Dar ei au stat în mijlocul câmpului, l-au aparat ?i au înfrânt pe Filisteni ?i le-a daruit Domnul biruin?a mare.
1Ch 11:15  Trei din cele treizeci de capetenii s-au coborât pe stânca la David, în pe?tera Adulam, când tabara Filistenilor era a?ezata în valea Refaim.
1Ch 11:16  David atunci era la loc întarit, iar o?tirea de întarire a Filistenilor era atunci în Betleem.
1Ch 11:17  ?i a dorit David ?i a zis: "Cine ma va adapa cu apa din fântâna Betleemului care este la poarta?"
1Ch 11:18  Atunci ace?ti trei au strabatut prin tabara Filistenilor, au scos apa din fântâna Betleemului cea de la poarta ?i au luat-o ?i au dus-o lui David. Dar David n-a voit sa o bea ?i a varsat-o înaintea Domnului,
1Ch 11:19  Zicând: "Sa ma fereasca Dumnezeu sa fac eu aceasta! A? putea sa beau eu sângele celor care s-au dus acolo cu primejduirea vie?ii lor? Caci cu primejduirea vie?ii lor au adus-o!" ?i n-a vrut sa o bea. Iata ce au facut ace?ti trei viteji.
1Ch 11:20  ?i Abi?ai, fratele lui Ioab, era capetenia celor trei; el a ranit deodata cu suli?a trei sute de oameni ?i era vestit între cei trei.
1Ch 11:21  El era mai stralucit decât cei treizeci ?i le era capetenie, dar cu ceilal?i trei nu era deopotriva.
1Ch 11:22  Benaia, fiul lui Iehoiada, barbat viteaz, mare dupa fapte, era din Cab?eel; el a ucis doi fii ai lui Ariel Moabitul ?i s-a coborât într-o groapa ?i a ucis un leu pe o vreme cu zapada.
1Ch 11:23  Tot el a ucis un egiptean, un om cu statura de cinci co?i, care avea în mâna o suli?a, ca un sul de la razboiul de ?esut; el s-a dus la el cu toiagul, i-a smuls suli?a din mâna ?i l-a ucis cu suli?a lui.
1Ch 11:24  Iata ce a facut Benaia, fiul lui Iehoiada, care era în cinste la cei trei viteji.
1Ch 11:25  El era mai vestit decât cei treizeci, dar cu cei trei nu era deopotriva ?i David l-a pus cel mai de aproape împlinitor al poruncilor sale.
1Ch 11:26  Iar dintre o?tenii cei mai de seama erau: Asael, fratele lui Ioab; Elhanan, fiul lui Dodo, din Betleem;
1Ch 11:27  ?amot din Haror; Hele? din Pelon;
1Ch 11:28  Ira, fiul lui Iche? din Tecoa; Abiezer din Anatot;
1Ch 11:29  Sibecai Hu?ateul, Ilai din Ahoh;
1Ch 11:30  Maharai din Netofat, Heled, fiul lui Baana din Netofat,
1Ch 11:31  Itai, fiul lui Ribai, din Ghibeea lui Veniamin; Benaia din Piraton
1Ch 11:32  Hurai din Nahali-Gaa?; Abiel din Araba;
1Ch 11:33  Azmavet din Bahurim; Eliahba din ?aalbon;
1Ch 11:34  Fiii lui Ha?em din Ghizon: Ionatan, fiul lui ?aghi din Harar;
1Ch 11:35  Ahiam, fiul lui Sacar din Harar; Elifelet, fiul lui Uri.
1Ch 11:36  Hefer din Mechera, Ahia din Pelon,
1Ch 11:37  He?ro din Carmel, Naarai, fiul lui Ezbai;
1Ch 11:38  Ioil, fratele lui Natan; Mibhar, fiul lui Hagri;
1Ch 11:39  ?elec Amonitul; Nahrai din Beerot, purtatorul de arme al lui Ioab, fiul ?eruiei,
1Ch 11:40  Ira din Iatir, Gareb din Iatir;
1Ch 11:41  Urie Heteul; Zabad, fiul lui Ahlai;
1Ch 11:42  Adina, fiul lui ?iza Rubenitul, capetenia Rubeni?ilor care avea sub el treizeci de in?i,
1Ch 11:43  Hanan, fiul lui Maaca; Iosafat din Mitni,
1Ch 11:44  Uzia din A?tarot; ?ama ?i Iehiel, fiii lui Hotam din Aroer,
1Ch 11:45  Iediael, fiul lui ?imri ?i Ioha, fratele lui Ti?itul,
1Ch 11:46  Eliel din Mahavim, Ieribai ?i Io?avia, fiii lui Elnaam ?i Itma Moabitul,
1Ch 11:47  Eliel, Obed ?i Iaasiel din Me?oba.
1Ch 12:1  Iata pe cei care au mai mers la David în ?iclag, pe când statea el înca ascuns de Saul, fiul lui Chi?. Ace?tia erau dintre vitejii care ajutasera la lupta.
1Ch 12:2  Ei erau arca?i, aruncau pietre ?i cu dreapta ?i cu stânga ?i din arcuri trageau cu sage?i ?i faceau parte dintre Veniamineni, fra?ii lui Saul, ?i anume:
1Ch 12:3  Capetenia Ahiezer, apoi Ioa?, fiii lui ?emaa din Ghibeea, Ieziel ?i Pelet, fiii lui Azmavet, Beraca ?i Iehu din Anatot;
1Ch 12:4  I?maia Ghibeoneanul, un viteaz dintre cei treizeci ?i capetenie peste treizeci; Ieremia, Iahaziel, Iohanan ?i Iozabad din Ghedera;
1Ch 12:5  Eluzai, Ierimot, Bealia, ?emaria ?i ?efatia Harifianul;
1Ch 12:6  Elcana, I?ia, Azareel, Ioezer ?i Ia?obeam, Corei?i;
1Ch 12:7  Ioela ?i Zebadia, fiii lui Ieroham din Ghedor.
1Ch 12:8  Din Gaditeni au trecut la David, în cetatea din pustie, oameni curajo?i, razboinici ?i înarma?i cu scut ?i suli?a, cu fa?a lor ca fa?a leului ?i iu?i ca ?i caprioarele din mun?i ?i anume:
1Ch 12:9  Capetenia Ezer, al doilea Obadia ?i al treilea Eliab;
1Ch 12:10  Al patrulea Ma?mana, al cincilea Ieremia,
1Ch 12:11  Al ?aselea Atai, al ?aptelea Eliel,
1Ch 12:12  Al optulea Iohanan ?i al noualea Elzabad,
1Ch 12:13  Al zecelea Ieremia ?i al unsprezecelea Macbanai.
1Ch 12:14  Ace?tia sunt din fiii lui Gad ?i erau capetenii în o?tire: cei mai mici, peste sute ?i cei mai mari, peste mii.
1Ch 12:15  Ace?tia au trecut Iordanul în luna întâi, când el iese din matca sa, ?i au alungat pe to?i cei ce locuiau pe vai, spre rasarit ?i apus.
1Ch 12:16  Au mai venit de asemenea ?i dintre fiii lui Veniamin ?i ai lui Iuda în cetate, la David.
1Ch 12:17  Iar David a ie?it în întâmpinarea lor ?i le-a zis: "De a?i venit cu pace, ca sa-mi ajuta?i, atunci sa fie în mine ?i în voi o singura inima; iar de a?i venit ca prin vicle?ug sa ma da?i vrajma?ilor mei, atunci, cum nu este prihana în mâinile mele, va vedea ?i va judeca Dumnezeul parin?ilor no?tri".
1Ch 12:18  Atunci a cuprins Duhul pe Amasai, capetenia celor treizeci ?i a zis: "Suntem cu tine, Davide, ?i pacea sa fie cu tine, fiul lui Iesei! Pace ?ie ?i pace celor ce-?i ajuta, ca î?i ajuta Dumnezeul tau". ?i i-a primit David ?i i-a pus în capul o?tirii.
1Ch 12:19  ?i din tribul lui Manase au trecut unii la David, când mergea el cu Filistenii la razboi contra lui Saul, dar nu l-au ajutat, pentru ca conducatorii Filistenilor, sfatuindu-se, l-au trimis înapoi, zicând: "Pentru primejduirea capului nostru, el va trece la domnul sau Saul".
1Ch 12:20  Dupa ce s-a întors el la ?iclag, au trecut la dânsul din ai lui Manase: Adnah, Iozabad, Iediael, Micael, Iozabad, Elihu ?i ?iltai, capetenii peste mii în Manase.
1Ch 12:21  Ace?tia au ajutat lui David contra navalitorilor, caci to?i ace?tia erau oameni viteji ?i capetenii în o?tire.
1Ch 12:22  Astfel în fiecare zi veneau lui David în ajutor pâna întru atât, încât tabara lui ajunsese mare, ca o tabara a lui Dumnezeu.
1Ch 12:23  Iata acum numarul capeteniilor de o?tire, care au venit la David în Hebron, ca sa-i încredin?eze domnia lui Saul, dupa cuvântul Domnului:
1Ch 12:24  Fii de-ai lui Iuda care purtau scut ?i suli?a erau ?ase mii opt sute, gata de lupta;
1Ch 12:25  Din fiii lui Simeon erau ?apte mii o suta, oameni viteji de o?tire;
1Ch 12:26  Din fiii lui Levi, patru mii ?ase sute;
1Ch 12:27  Iehoiada, capetenie din neamul lui Aaron, ?i cu el trei mii ?apte sute,
1Ch 12:28  ?i ?adoc, un tânar voinic cu neamurile lui douazeci ?i doua de capetenii.
1Ch 12:29  Din fiii lui Veniamin, fra?ii lui Saul, au venit trei mii, dar înca mul?i din ei se ?ineau de casa lui Saul;
1Ch 12:30  Din fiii lui Efraim, douazeci de mii opt sute de oameni viteji, oameni cunoscu?i în neamul lor;
1Ch 12:31  Din jumatate din semin?ia lui Manase, optsprezece mii care au fost chema?i pe nume ca sa mearga sa faca rege pe David;
1Ch 12:32  Dintre fiii lui Isahar au venit oameni în?elep?i care ?tiau ce ?i când trebuie sa faca Israel; ace?tia erau doua sute capetenii ?i to?i fra?ii lor urmau sfatul lor;
1Ch 12:33  Din tribul lui Zabulon au venit oameni gata de lupta ?i înarma?i cu tot felul de arme, în numar de cincizeci de mii, în ordine ?i într-un suflet;
1Ch 12:34  Din tribul lui Neftali, o mie de capetenii ?i cu ei treizeci ?i ?apte de mii cu scuturi ?i cu suli?e;
1Ch 12:35  Din tribul lui Dan au venit douazeci ?i opt de mii ?ase sute oameni gata de lupta;
1Ch 12:36  Din A?er au venit o?teni, gata de lupta, patruzeci de mii;
1Ch 12:37  De peste Iordan, din tribul lui Ruben, al lui Gad ?i din jumatate de trib al lui Manase, au venit o suta douazeci de mii cu tot felul de arme de lupta.
1Ch 12:38  To?i ace?ti o?teni, gata de lupta ?i cu toata inima, au venit la Hebron sa faca rege pe David peste Israel. Dar ?i to?i ceilal?i Israeli?i erau într-un cuget pentru a fi facut rege David.
1Ch 12:39  ?i au ramas acolo la David trei zile ?i au mâncat ?i au baut, pentru ca fra?ii lor pregatisera toate pentru ei.
1Ch 12:40  ?i apoi chiar vecinii lor, pâna chiar ?i Isahar, Zabulon ?i Neftali, adusesera toate de ale mâncarii pe asini, pe camile, pe catâri ?i cu carele de boi: faina, smochine ?i stafide, vin, untdelemn ?i vite mari ?i marunte, mul?ime multa, pentru ca bucurie mare era peste Israel.
1Ch 13:1  Atunci David s-a sfatuit cu capeteniile cele peste mii, cu suta?ii ?i cu toate capeteniile,
1Ch 13:2  ?i a zis David catre toata adunarea Israeli?ilor: "Daca binevoi?i voi ?i daca este voia Domnului Dumnezeului nostru, sa trimitem pretutindeni la ceilal?i fra?i ai no?tri, în toata ?ara lui Israel ?i totodata ?i la preo?i ?i levi?i prin ora?ele ?i prin satele lor, ca sa Se adune la noi;
1Ch 13:3  ?i sa aducem la noi chivotul Dumnezeului nostru, pentru ca în zilele lui Saul ne-am îndreptat spre el".
1Ch 13:4  Atunci toata adunarea a zis: "A?a sa fie", pentru ca lucrul acesta s-a parut drept înaintea a tot poporul.
1Ch 13:5  Astfel a adunat David pe to?i Israeli?ii, de la ?ihorul egiptean pâna la intrarea Hamatului, ca sa stramute chivotul Domnului din Chiriat-Iearim.
1Ch 13:6  Atunci s-a dus David ?i tot Israelul la Chiriat-Iearim, ce este în Iuda, ca sa stramute de acolo chivotul lui Dumnezeu, înaintea caruia se cheama numele Domnului Celui care sade pe heruvimi.
1Ch 13:7  ?i au adus chivotul lui Dumnezeu într-un car nou din casa lui Abinadab; iar carul îl conduceau Uza ?i Ahia.
1Ch 13:8  David însa ?i to?i Israeli?ii jucau înaintea lui Dumnezeu cât puteau cu cântari din gura, din chitara, din psaltirion, din timpane ?i ?imbale ?i din trâmbi?e;
1Ch 13:9  Dar când au ajuns la aria lui Chidon, Uza ?i-a întins mâna, ca sa sprijine chivotul, caci boii erau sa-l rastoarne.
1Ch 13:10  Atunci S-a mâniat Domnul pe Uza ?i l-a lovit, pentru ca ?i-a întins mâna spre chivot; ?i el a murit acolo pe loc înaintea lui Dumnezeu.
1Ch 13:11  ?i s-a întristat David ca Domnul lovise pe Uza ?i a numit locul acela Pere?-Uza ?i a?a se nume?te pâna azi.
1Ch 13:12  ?i s-a temut David de Dumnezeu în ziua aceea ?i a zis: "Cum voi duce la mine chivotul lui Dumnezeu?"
1Ch 13:13  De aceea nu a dus David chivotul la sine, în cetatea lui David, ci l-a întors la casa lui Obed-Edom.
1Ch 13:14  ?i a ramas chivotul Domnului la Obed-Edom, în casa lui, trei luni ?i a binecuvântat Domnul casa lui Obed-Edom ?i toate ale lui.
1Ch 14:1  În vremea aceea a trimis Hiram, regele Tirului, soli la David ?i lemne de cedru ?i pietrari ?i dulgheri, ca sa-i ridice casa.
1Ch 14:2  Când a aflat David ca l-a întarit Domnul rege peste Israel, ca domnia lui a fost înal?ata sus pentru poporul sau Israel,
1Ch 14:3  ?i-a luat înca alte femei din Ierusalim ?i i s-au mai nascut lui David fii ?i fiice.
1Ch 14:4  Iata numele celor ce i s-au nascut în Ierusalim: ?amua, ?obab, Natan ?i Solomon;
1Ch 14:5  Ibhar, Eli?ua ?i Elifelet;
1Ch 14:6  Nogah, Nefeg ?i Iafia;
1Ch 14:7  Eli?ama, Beeliada ?i Elifelet.
1Ch 14:8  Auzind însa Filistenii ca David a fost uns rege peste tot Israelul, s-au ridicat ei to?i sa caute pe David. ?i auzind David de aceasta, a ie?it înaintea lor.
1Ch 14:9  Iar Filistenii au venit ?i s-au a?ezat în valea Refaim.
1Ch 14:10  Atunci a întrebat David pe Dumnezeu, zicând: "Sa merg eu oare contra Filistenilor ?i-i vei da Tu în mâna mea?" ?i Domnul i-a raspuns: "Mergi, ca îi voi da în mâna ta!"
1Ch 14:11  ?i s-au dus ei în Baal-Pera?im ?i acolo i-a lovit David; apoi David a zis: "Dumnezeu a zdrobit pe vrajma?i cu mâna mea, ca pe o surpatura de apa". De aceea au ?i dat vaii aceleia numele de Baal- Pera?im.
1Ch 14:12  Filistenii ?i-au lasat acolo zeii, iar David i-a adunat ?i i-a ars cu foc.
1Ch 14:13  ?i au venit iara?i Filistenii ?i au tabarât în vale.
1Ch 14:14  Iar David a întrebat din nou pe Dumnezeu, iar Dumnezeu i-a zis: "Nu te duce de-a dreptul asupra lor; abate-te de la ei ?i te du asupra lor pe la dumbrava duzilor;
1Ch 14:15  ?i când vei auzi zgomot ca de pa?i prin vârful duzilor, atunci sa intri în lupta, caci a ie?it Dumnezeu înaintea ta, ca sa bata tabara Filistenilor".
1Ch 14:16  ?i a facut David cum îi poruncise Dumnezeu; ?i a lovit tabara Filistenilor de la Ghibeon pâna la Ghezer.
1Ch 14:17  Atunci a rasunat numele lui David prin toate ?arile dimprejur ?i l-a facut Domnul înfrico?ator pentru toate popoarele vecine.
1Ch 15:1  David ?i-a facut apoi case în cetatea lui, a pregatit loc pentru chivotul lui Dumnezeu ?i a facut pentru el un cort.
1Ch 15:2  Atunci David a zis: "Nimeni, afara de levi?i, nu trebuie sa poarte chivotul lui Dumnezeu ?i sa-I slujeasca Lui în veci".
1Ch 15:3  Atunci a adunat David pe to?i Israeli?ii la Ierusalim, ca sa duca chivotul lui Dumnezeu la locul lui, pe care i-l pregatise el.
1Ch 15:4  A chemat deci David pe fiii lui Aaron ?i pe levi?i ?i anume:
1Ch 15:5  Din urma?ii lui Cahat a chemat pe capetenia Uriel ?i pe fra?ii lui, o suta douazeci;
1Ch 15:6  Din urma?ii lui Merari a chemat pe capetenia Asaia ?i pe fra?ii lui, doua sute douazeci de oameni;
1Ch 15:7  Din urma?ii lui Gher?om a chemat pe capetenia Ioil ?i pe fra?ii lui, o suta treizeci de oameni;
1Ch 15:8  Din urma?ii lui Eli?afan a chemat pe capetenia ?emaia ?i pe fra?ii lui, doua sute de oameni;
1Ch 15:9  Din urma?ii lui Hebron a chemat pe capetenia Eliel ?i pe fra?ii lui, optzeci de oameni;
1Ch 15:10  Din urma?ii lui Uziel a chemat pe capetenia Aminadab ?i pe fra?ii lui, o suta douazeci de oameni.
1Ch 15:11  Apoi a chemat David pe preo?ii ?adoc ?i Abiatar ?i pe levi?ii Uriel, Asaia, Ioil, ?emaia, Eliel ?i Aminadab,
1Ch 15:12  ?i le-a zis: "Voi, capeteniile familiilor levite, sfin?i?i-va voi ?i fra?ii vo?tri ?i aduce?i chivotul Domnului Dumnezeului lui Israel la locul pe care l-am pregatit eu pentru el.
1Ch 15:13  Deoarece înainte n-a?i facut voi aceasta, Domnul Dumnezeul nostru ne-a lovit, pentru ca nu L-am cautat cum se cuvine".
1Ch 15:14  Atunci s-au sfin?it preo?ii ?i levi?ii, ca sa aduca chivotul Domnului Dumnezeului lui Israel.
1Ch 15:15  ?i au adus fiii levi?ilor chivotul Domnului, cum poruncise Moise dupa cuvântul Domnului, pe pârghii; pe umeri, iar nu cu caru?a.
1Ch 15:16  Apoi a poruncit David capeteniilor levi?ilor sa puna pe fra?ii lor cântare?i cu instrumente muzicale, cu psaltirioane, ca sa vesteasca cu glas tare de bucurie.
1Ch 15:17  Ace?tia au pus pe levi?ii: Heman, fiul lui Ioil, iar din fra?ii lui, pe Asaf, fiul lui Berechia. Din urma?ii lui Merari, fra?ii lor, au pus pe Etan, fiul lui Cu?aia.
1Ch 15:18  Iar pe fra?ii lor din a doua spi?a: Zaharia, fiul lui Iaaziel, ?emiramot, Iehiel, Uni, Eliab, Benaia, Maaseia, Matitia, Elifelehu, Micneia, Obed-Edom ?i Ieiel, i-au pus portari.
1Ch 15:19  Heman, Asaf ?i Etan cântau puternic din ?imbale de arama;
1Ch 15:20  Zaharia, Iaaziel, ?emiramot, Iehiel, Uni, Eliab, Maaseia ?i Benaia cântau din psaltirioanele cu sunete sub?iri.
1Ch 15:21  Matitia însa, Elifelehu, Micneia, Obed-Edom, Ieiel ?i Azazia faceau începutul cu harpele cu câte opt coarde.
1Ch 15:22  Iar Chenaia, capetenia levi?ilor, era dascal de cântari, pentru ca era iscusit în acestea.
1Ch 15:23  Berechia ?i Elcana erau u?ieri la chivot.
1Ch 15:24  Preo?ii ?ebania, Iosafat, Natanael, Amasai, Zaharia, Benaia ?i Eliezer trâmbi?au din trâmbi?e înaintea chivotului lui Dumnezeu. Obed-Edom ?i Iehia erau u?ieri la chivot.
1Ch 15:25  Astfel s-au dus David cu batrânii lui Israel ?i capeteniile cele peste mii sa aduca chivotul Domnului din casa lui Obed-Edom cu veselie.
1Ch 15:26  ?i dupa ce a ajutat Dumnezeu levi?ilor sa aduca chivotul Domnului, atunci au junghiat pentru jertfe ?apte vi?ei ?i ?apte berbeci.
1Ch 15:27  David era îmbracat cu ve?minte de vison, asemenea erau îmbraca?i ?i to?i levi?ii care aduceau chivotul ?i cântare?ii ?i Chenania, capetenia muzican?ilor ?i cântarelilor. David însa mai avea pe el ?i un efod de in.
1Ch 15:28  A?a tot Israelul a adus chivotul legamântului Domnului cu strigate de bucurie, cu sunete de corn, de trâmbi?e, de ?imbale ?i de harpe.
1Ch 15:29  Când chivotul legamântului Domnului a intrat în cetatea lui David, Micol, fata lui Saul, privea de la fereastra ?i, vazând pe regele David jucând ?i veselindu-se, l-a dispre?uit în inima sa.
1Ch 16:1  Astfel au adus chivotul lui Dumnezeu ?i l-au a?ezat în mijlocul cortului pe care-l facuse David pentru el ?i au înal?at lui Dumnezeu arderi de tot ?i jertfe de pace.
1Ch 16:2  Dupa ce David a ispravit de adus arderile de tot ?i jertfele de pace, a binecuvântat poporul în numele Domnului,
1Ch 16:3  ?i a împar?it tuturor Israeli?ilor, femei ?i barba?i, câte o pâine ?i câte o buca?ica de carne ?i câte o turta de struguri.
1Ch 16:4  Apoi a pus la slujba înaintea chivotului Domnului din levi?i, ca sa preaslaveasca, sa mul?umeasca ?i sa preaînal?e pe Domnul Dumnezeul lui Israel, ?i anume:
1Ch 16:5  Pe Asaf, capetenie; al doilea dupa el a pus pe Zaharia; apoi urmau Uziel, ?emiramot, Iehiel, Matitia, Eliab, Benaia, Obed-Edom ?i Ieiel cu psaltirioane ?i harpe, iar Asaf cânta din ?imbale.
1Ch 16:6  A pus de asemenea pe preo?ii Benaia ?i Oziel sa sune necontenit din trâmbi?e înaintea chivotului legamântului lui Dumnezeu.
1Ch 16:7  În aceasta zi David, pentru întâia oara, a dat, prin Asaf ?i fra?ii lui, urmatorul psalm de lauda Domnului:
1Ch 16:8  "Lauda?i pe Domnul ?i chema?i numele Lui; vesti?i între neamuri lucrurile Lui!
1Ch 16:9  Cânta?i, cânta?i în cinstea Lui! Spune?i toate minunile Lui!
1Ch 16:10  Lauda?i-va cu numele Lui cel sfânt! Sa se bucure inima celor ce-L cauta pe El!
1Ch 16:11  Alerga?i la Domnul ?i la ajutorul Lui; cauta?i pururea fa?a Lui!
1Ch 16:12  Neamul lui Israel, sluga Lui, fiii lui Iacov, ale?ii Lui,
1Ch 16:13  Aduce?i-va aminte de minunile Lui, de semnele Lui ?i de judeca?ile gurii Lui!
1Ch 16:14  Caci El este Domnul Dumnezeul nostru ?i dreptatea Lui este peste tot pamântul.
1Ch 16:15  Aduce?i-va aminte de a?ezamântul Lui, de fagaduin?a data pentru mii de neamuri.
1Ch 16:16  De legamântul facut cu Avraam ?i de juramântul Sau catre Isaac,
1Ch 16:17  Juramânt pus ca o lege pentru Iacov, ?i ca un legamânt ve?nic pentru Israel,
1Ch 16:18  Zicând: ?ie-?i voi da pamântul Canaan, ca partea voastra de mo?tenire.
1Ch 16:19  Ei atunci erau pu?ini la numar ?i neînsemna?i, ?i straini în ?ara aceasta.
1Ch 16:20  ?i treceau de la popor la popor ?i dintr-o împara?ie la alta împara?ie.
1Ch 16:21  Dar El n-a lasat pe nimeni sa-i apese, ?i a pedepsit regi pentru ei, zicând:
1Ch 16:22  Nu va atinge?i de un?ii Mei ?i proorocilor Mei sa nu le face?i rau.
1Ch 16:23  Cânta?i Domnului tot pamântul, binevesti?i din zi în zi izbavirea Lui!
1Ch 16:24  Vestiri pagânilor slava Lui, spune?i la toate popoarele minunile Lui!
1Ch 16:25  Ca mare este Domnul ?i vrednic de lauda ?i mai înfrico?at decât to?i dumnezeii.
1Ch 16:26  Ca to?i dumnezeii pagânilor sunt nimic, iar Domnul a facut cerurile.
1Ch 16:27  Înaintea Lui este stralucire ?i mare?ie, putere ?i bucurie în loca?ul Lui cel sfânt.
1Ch 16:28  Da?i Domnului, neamuri pagâne, da?i Domnului slava ?i cinste!
1Ch 16:29  Da?i Domnului slava pentru numele Lui; aduce?i-va darul, merge?i înaintea fe?ei Lui, închina?i-va Domnului în podoabele sfin?eniei Lui!
1Ch 16:30  Sa tremure înaintea Lui tot pamântul, ca El a întemeiat lumea ?i nu se va clatina.
1Ch 16:31  Sa se bucure cerurile ?i sa praznuiasca pamântul, iar printre popoare sa se spuna: Domnul este Împarat!
1Ch 16:32  Sa se zguduie marea ?i toate cele din ea; câmpia ?i toate cele de pe ea sa se veseleasca!
1Ch 16:33  Sa dan?uiasca împreuna to?i copacii dumbravii înaintea fe?ei Domnului, ca vine sa judece pamântul.
1Ch 16:34  Lauda?i pe Domnul, ca în veac este mila Lui!
1Ch 16:35  Zice?i: Izbave?te-ne pe noi, Dumnezeule, Izbavitorul nostru! Aduna-ne ?i ne izbave?te de prin popoare, ca sa slavim sfânt numele Tau ?i sa ne laudam cu slava Ta!
1Ch 16:36  Binecuvântat fie Domnul Dumnezeul lui Israel din veac în veac!" ?i tot poporul a zis: "Amin! Aliluia!"
1Ch 16:37  ?i a lasat David acolo, înaintea chivotului legamântului Domnului, pe Asaf ?i pe fra?ii lui, ca sa slujeasca ei înaintea chivotului neîncetat, în fiecare zi;
1Ch 16:38  Pe Obed-Edom ?i pe fra?ii lui, ?aizeci ?i opt de oameni; pe Obed-Edom, fiul lui Iedutun ?i pe Hosa, i-a lasat u?ieri.
1Ch 16:39  Iar pe preotul ?adoc ?i pe ceilal?i preo?i, fra?ii sai, i-a pus înaintea loca?ului Domnului, cel de pe înal?imea din Ghibeon,
1Ch 16:40  Ca sa înal?e arderi de tot Domnului pe jertfelnicul arderilor de tot neîncetat, diminea?a ?i seara, pentru toate cele scrise în legea Domnului, pe care El le-a poruncit lui Israel.
1Ch 16:41  ?i cu ei a lasat pe Heman, pe Iedutun ?i pe ceilal?i ale?i, care au fost numi?i pe nume, ca sa slaveasca pe Domnul, ca în veac este mila Lui.
1Ch 16:42  Împreuna cu ei, Heman ?i Iedutun preaslaveau pe Dumnezeu, cântând din trâmbi?e ?i felurite instrumente muzicale; pe fiii lui Iedutun i-a pus la poarta.
1Ch 16:43  Apoi s-a dus tot poporul, fiecare la casa sa. De asemenea s-a întors ?i David, ca sa binecuvânteze casa sa.
1Ch 17:1  Când David locuia în casa sa, a zis el catre Natan proorocul: "Iata eu traiesc în casa de cedru, iar chivotul legamântului Domnului este în cort".
1Ch 17:2  Iar Natan a zis catre David: "Fa tot ce ai la inima, ca Dumnezeu este cu tine!"
1Ch 17:3  Dar în aceea?i noapte a fost cuvântul Domnului catre Natan ?i i-a zis:
1Ch 17:4  "Mergi ?i spune robului Meu David: A?a zice Domnul: Nu tu ai sa-Mi zide?ti Mie casa de locuit,
1Ch 17:5  Caci Eu n-am locuit în casa din ziua în care am scos pe fiii lui Israel ?i pâna astazi, ci am umblat din cort în cort ?i din loca? în loca?.
1Ch 17:6  Oriunde am mers Eu cu tot Israelul, spus-am Eu oare macar un cuvânt cuiva din judecatorii lui Israel, carora le-am poruncit sa pastoreasca pe poporul Meu, pentru ce nu-Mi zide?ti Mie casa de cedru?
1Ch 17:7  ?i acum a?a sa spui robului Meu David: Eu te-am luat de la turma de oi, ca sa fii conducatorul poporului Meu Israel;
1Ch 17:8  ?i am fost cu tine pretutindeni, oriunde ai umblat; am stârpit pe to?i vrajma?ii tai înaintea fe?ei tale ?i am facut numele tau ca numele celor puternici ai pamântului.
1Ch 17:9  Am rânduit loc pentru poporul Meu Israel ?i l-am înradacinat, ?i va trai el în pace la locul sau, ?i nu va mai fi nelini?tit ?i necredincio?ii nu-l vor mai strâmtora ca altadata,
1Ch 17:10  Ca în zilele acelea, când puneam judecatori peste poporul Meu Israel; dar am supus pe to?i vrajma?ii tai ?i ?i-am vestit ca Domnul ?i-a pregatit ?ie casa.
1Ch 17:11  Când se vor împlini zilele tale ?i vei trece la parin?ii tai, atunci Eu voi ridica pe urma?ul tau dupa tine, ?i voi întari domnia ta.
1Ch 17:12  Acela Îmi va zidi Mie casa ?i voi întemeia tronul lui pe veci.
1Ch 17:13  Eu îi voi fi tata ?i el Îmi va fi fiu ?i mila Mea nu o voi lua de la el, cum am luat-o de la cel ce a fost înaintea ta.
1Ch 17:14  Îl voi pune pe acela în casa Mea ?i în împara?ia Mea pe veci ?i tronul lui în veci va fi tare".
1Ch 17:15  Toate cuvintele ?i toata vedenia aceasta le-a spus Natan lui David.
1Ch 17:16  Atunci a venit regele David ?i a stat înaintea fe?ei Domnului ?i a zis: "Cine sunt eu, Doamne Dumnezeule, ?i ce este casa mea, de m-ai înal?at a?a?
1Ch 17:17  Dar ?i aceasta s-a parut înca pu?in în ochii Tai, Dumnezeule, caci iata veste?ti despre casa robului Tau în viitor ?i prive?ti la mine, ca la un om mare, Doamne Dumnezeule!
1Ch 17:18  Ce mai poate adauga David înaintea Ta pentru marirea robului Tau? Tu cuno?ti pe robul Tau.
1Ch 17:19  Doamne, pentru robul Tau, dupa inima Ta, faci toate aceste lucruri mari, ca sa ara?i toata marirea.
1Ch 17:20  Doamne, nu este altul asemenea ?ie, ?i nu este Dumnezeu afara de Tine, dupa câte am auzit noi.
1Ch 17:21  ?i nu este înca alt popor pe pamânt ca poporul Tau Israel, pe care l-a calauzit Dumnezeu, ca sa-l rascumpere Sie?i de popor, sa-?i faca nume mare ?i stralucit, izgonind popoarele de la fa?a poporului Tau, pe care l-ai izbavit din Egipt.
1Ch 17:22  Tu ai facut pe poporul Tau, Israel, poporul Tau pe veci ?i Tu, Doamne, Te-ai facut Dumnezeul lui.
1Ch 17:23  A?adar, Doamne, cuvântul pe care l-ai grait Tu acum despre robul Tau ?i despre casa lui, întare?te-l pe veci, ?i fa cum ai zis Tu.
1Ch 17:24  Sa ramâna ?i sa se preamareasca numele Tau în veci, ca sa se zica ca Domnul Savaot, Dumnezeul lui Israel, este Dumnezeu peste Israel, ?i casa robului Tau David sa fie tare înaintea fe?ei Tale.
1Ch 17:25  Caci Tu, Dumnezeul meu, ai descoperit robului Tau ca-i vei zidi casa, de aceea robul Tau a ?i îndraznit sa se roage înaintea Ta.
1Ch 17:26  ?i acum, Doamne, Tu e?ti adevaratul Dumnezeu ?i Tu ai vestit despre robul Tau astfel de lucruri bune.
1Ch 17:27  Începe dar a binecuvânta casa robului Tau, ca sa fie ea ve?nica înaintea fe?ei Tale. Caci daca Tu, Doamne, o vei binecuvânta, binecuvântata va fi ea în veci".
1Ch 18:1  Dupa aceasta a lovit David pe Filisteni ?i i-a supus ?i a luat cetatea Gat ?i ora?ele ce ?ineau de ea din mâinile Filistenilor.
1Ch 18:2  A lovit el de asemenea ?i pe Moabi?i ?i au ajuns Moabi?ii robii lui David, platindu-i tribut.
1Ch 18:3  Apoi a lovit David pe Hadadezer, regele ?obei, la Hamat, când mergea acela sa-?i întareasca stapânirea la râul Eufrat.
1Ch 18:4  ?i a luat David de la el o mie de care de razboi, ?apte mii de calare?i ?i douazeci de mii de pedestra?i. ?i a stricat David toate carele, neoprind din ele decât numai o suta.
1Ch 18:5  Sirienii din Damasc ar fi venit în ajutor lui Hadadezer, regele ?obei, dar David a batut douazeci ?i doua de mii de Sirieni.
1Ch 18:6  Apoi a a?ezat David o?tire de paza în Siria Damascului ?i au ajuns Sirienii robii lui David, platindu-i tribut. Domnul a ajutat lui David pretutindeni, oriunde a mers el.
1Ch 18:7  Atunci a luat David scuturile de aur, care erau la robii lui Hadadezer, ?i le-a adus în Ierusalim.
1Ch 18:8  Iar din Tibhat ?i din Cun, ceta?ile lui Hadadezer, a luat David foarte multa arama. Din arama aceasta a facut Solomon marea cea de arama, stâlpii ?i vasele cele de arama.
1Ch 18:9  Auzind Tou, regele din Hamat, ca David a batut toata armata lui Hadadezer, regele ?obei,
1Ch 18:10  A trimis pe Hadoram, fiul sau, la regele David, sa-l salute ?i sa-i mul?umeasca, pentru ca s-a razboit cu Hadadezer ?i l-a batut, caci Tou era în razboi cu Hadadezer, ?i a trimis cu el tot felul de vase de aur, de argint ?i de arama.
1Ch 18:11  Regele David a închinat aceste vase Domnului, împreuna cu aurul ?i argintul pe care-l luase el de la toate popoarele: de la Edomi?i, Moabi?i, Amoni?i, Filisteni ?i Amaleci?i.
1Ch 18:12  ?i Abi?ai, fiul ?eruiei, a batut optsprezece mii de Edomi?i în Valea Sarata.
1Ch 18:13  ?i a pus în Edom oaste de paza ?i s-au facut Edomi?ii robii lui David, caci Domnul ajuta lui David oriunde mergea.
1Ch 18:14  ?i a domnit David peste tot Israelul ?i a facut judecata ?i dreptate la tot poporul sau.
1Ch 18:15  Ioab, fiul ?eruiei, era comandantul o?tirii, iar Iosafat, fiul lui Ahilud, era cronicar.
1Ch 18:16  ?adoc, fiul lui Ahitub, ?i Ahimelec, fiul lui Abiatar, au fost preo?i, iar ?ausa (Serais) a fost secretar.
1Ch 18:17  Benaia, fiul lui Iehoiada, era capetenie peste Cheretieni ?i Peletieni, iar fiii lui David erau cei întâi pe lânga rege,
1Ch 19:1  Dupa aceasta a murit Naha?, regele Amoni?ilor, ?i în locul lui s-a facut rege fiul sau.
1Ch 19:2  Atunci David a zis: "Am sa arat mila lui Hanun, fiul lui Naha?, pentru binefacerea ce mi-a aratat tatal sau". ?i a trimis David soli sa-l mângâie pentru pierderea tatalui sau. Au mers deci solii lui David în ?ara Amoni?ilor, ca sa-l mângâie pe Hanun.
1Ch 19:3  Însa capeteniile Amoni?ilor au zis catre Hanun: "Socote?ti tu, oare, ca David din dragoste pentru tatal tau a trimis la tine mângâietori? Nu cumva au venit slugile lui la tine, ca sa iscodeasca ?i sa vada ?ara ?i apoi sa o pustiiasca "
1Ch 19:4  Atunci a prins Hanun pe trimi?ii lui David ?i i-a ras ?i le-a taiat hainele pe jumatate, pâna la coapsa, ?i a?a le-a dat drumul ?i ei au plecat.
1Ch 19:5  Spunându-se lui David de pa?ania oamenilor acelora, a trimis el în întâmpinarea lor, ca erau tare batjocori?i, ?i li s-a zis: "Ramâne?i în Ierihon pâna va vor cre?te barbile ?i atunci va ve?i întoarce acasa".
1Ch 19:6  Dupa ce Amoni?ii au vazut ca au ajuns urâ?i lui David, au trimis Hanun ?i Amoni?ii o mie de talan?i de argint, ca sa ia în solda lor ni?te care de razboi ?i calare?i din Siria Mesopotamiei, Siria Maacai ?i din ?oba.
1Ch 19:7  ?i au luat în solda lor treizeci ?i doua de mii de care ?i pe regele Maacai cu poporul lui, care au venit ?i au tabarât înaintea Medebei. Iar Amoni?ii s-au adunat din ceta?ile lor ?i au ie?it la razboi.
1Ch 19:8  Când David a auzit de acestea, a trimis pe Ioab cu toata o?tirea de viteji.
1Ch 19:9  Atunci au înaintat Amoni?ii ?i s-au a?ezat în linie de bataie la por?ile ceta?ii, iar regii care venisera erau la o parte în câmp.
1Ch 19:10  Ioab, vazând ca are a lupta pe doua laturi, una în fa?a ?i alta în spate, a ales o?teni din to?i cei mai de seama din Israel ?i i-a rânduit contra Sirienilor.
1Ch 19:11  Iar cealalta parte de popor a încredin?at-o lui Abi?ai, fratele sau, ca sa se îndrepte contra Amoni?ilor.
1Ch 19:12  Apoi a zis: "Daca Sirienii vor fi mai tari decât mine, sa-mi vii tu în ajutor, iar daca Amoni?ii vor fi mai tari decât tine, î?i voi veni eu în ajutor.
1Ch 19:13  Fii curajos ?i sa stam cu tarie pentru apararea poporului nostru ?i pentru ceta?ile Dumnezeului nostru, ?i Domnul sa faca ce va binevoi".
1Ch 19:14  ?i a intrat Ioab ?i oamenii ce erau cu el în lupta cu Sirienii, dar ace?tia au fugit de el.
1Ch 19:15  Amoni?ii însa, vazând ca Sirienii fug, au fugit ?i ei de Abi?ai, fratele lui Ioab, ?i s-au dus în cetate. Atunci Ioab a venit la Ierusalim.
1Ch 19:16  Sirienii, vazând ca sunt batu?i de Israeli?i, au trimis soli ?i au scos pe Sirienii care erau dincolo de râu, iar ?ofac, capetenia lui Hadadezer, îi conducea.
1Ch 19:17  Când s-a spus aceasta lui David, el a adunat pe to?i Israeli?ii, a trecut Iordanul ?i, venind asupra acelora, s-a a?ezat în linie de bataie în fa?a lor. ?i a intrat David în lupta cu Sirienii ?i ace?tia s-au luptat cu el.
1Ch 19:18  Dar curând Sirienii au fugit de Israeli?i, iar David, a nimicit Sirienilor ?apte mii de care ?i patruzeci de mii de pedestra?i, ?i pe ?ofac, comandantul o?tirii, I-a ucis.
1Ch 19:19  Când au vazut slugile lui Hadadezer ca sunt birui?i de Israeli?i, au încheiat pace cu David ?i s-au supus. ?i n-au mai voit Sirienii sa mai ajute pe Amoni?i.
1Ch 20:1  Peste un an, pe vremea când regii ies la razboi, a scos Ioab o?tirea ?i a început sa pustiiasca ?ara Amoni?ilor ?i a venit ?i a înconjurat Raba. Însa David a ramas în Ierusalim. Ioab a cucerit Raba ?i a darâmat-o.
1Ch 20:2  ?i a luat David coroana regelui lor de pe capul lui ?i s-a aflat ca are aur în greutate de un talant ?i erau pe ea ?i pietre scumpe; ?i s-a pus coroana aceasta pe capul lui David. ?i au fost scoase din cetatea aceea ?i foarte multe prazi.
1Ch 20:3  Iar poporul care era în ea a fost scos ?i omorât cu fierastraie, cu ciocane de fier ?i cu securi. A?a a facut David cu toate ora?ele Amoni?ilor ?i apoi s-a întors el ?i tot poporul la Ierusalim.
1Ch 20:4  Dupa aceea s-a început razboiul cu Filistenii la Ghezer. Atunci Sibecai Hu?atitul a batut pe Sipai, unul din urma?ii Refaimilor, ?i s-au supus ?i ei.
1Ch 20:5  Apoi iar a fost razboi cu Filistenii. Dar Elhanan, fiul lui Iair, a lovit pe Lahmi, fratele lui Goliat Gateul; coada suli?ei lui era ca a sulului de la razboiul de ?esut.
1Ch 20:6  ?i a mai fost o lupta la Gat. Acolo era un om înalt care avea câte ?ase degete la mâini ?i la picioare, adica de toate douazeci ?i patru. ?i acesta era tot din urma?ii Refaimilor.
1Ch 20:7  El batjocorea pe Israel, dar Ionatan, fiul lui ?ama, fratele lui David, l-a ucis.
1Ch 20:8  Ace?tia erau oameni nascu?i din Refaimi în Gat ?i au cazut de mâna lui David ?i de mâna oamenilor lui.
1Ch 21:1  Atunci s-a sculat Satana împotriva lui Israel ?i a îndemnat pe David sa faca numaratoarea Israeli?ilor.
1Ch 21:2  Deci a zis David catre Ioab ?i catre capeteniile poporului: "Merge?i ?i numara?i pe Israeli?i de la Beer-?eba pâna la Dan ?i-mi aduce?i raspuns ca sa ?tiu numarul lor!"
1Ch 21:3  Ioab însa a zis: "Sa înmul?easca Domnul pe poporul Sau de o suta de ori mai mult decât este el acum! Au doara nu sunt ei to?i, o, rege, domnul meu, robii stapânului meu? Pentru ce dar cere aceasta domnul meu? Oare pentru a se scoate asta ca o vina lui Israel?"
1Ch 21:4  Dar cuvântul regesc biruind pe Ioab, s-a dus acesta de a cutreierat tot Israelul ?i venind la Ierusalim,
1Ch 21:5  A dat Ioab lui David catagrafia înscrierii poporului ?i s-au aflat în tot Israelul un milion ?i o suta de mii de barba?i destoinici de razboi, iar în Iuda, patru sute ?aptezeci de mii în stare de a lua parte la razboi.
1Ch 21:6  Pe levi?i însa ?i pe Veniamineni el nu i-a numarat împreuna cu ei, pentru ca cuvântul regelui nu placuse lui Ioab.
1Ch 21:7  Lucrul acesta n-a fost placut nici înaintea lui Dumnezeu ?i de aceea a lovit El pe Israel.
1Ch 21:8  Atunci a zis David catre Dumnezeu: "Am gre?it mult, facând aceasta; iarta dar vina robului Tau, ca m-am purtat cu totul nepriceput".
1Ch 21:9  Iar Domnul a grait cu Gad, proorocul lui David ?i i-a zis:
1Ch 21:10  "Mergi ?i spune lui David: A?a zice Domnul: Î?i pun înainte trei pedepse; alege-?i una din ele ?i o voi trimite asupra ta".
1Ch 21:11  A venit deci Gad la David ?i i-a zis: "A?a graie?te Domnul, alege:
1Ch 21:12  Sau trei ani de foamete, sau trei luni sa fii tu urmarit de vrajma?ii tai ?i sabia du?manilor sa ajunga pâna la tine, sau trei zile sabia Domnului ?i molima sa fie pe pamânt ?i îngerul Domnului sa pustiiasca în toate hotarele lui Israel. Vezi acum ce trebuie sa raspund Celui ce m-a trimis cu acest cuvânt".
1Ch 21:13  ?i a raspuns David lui Gad: "Sunt într-o mare nelini?te! Sa cad mai bine în mâinile Domnului, caci îndurarile Lui sunt foarte multe, dar sa nu cad în mâinile oamenilor".
1Ch 21:14  Atunci a trimis Domnul molima asupra lui Israel ?i au murit ?aptezeci de mii de Israeli?i.
1Ch 21:15  ?i a trimis Dumnezeu îngerul la Ierusalim ca sa-l piarda. ?i când a început el sa pustiiasca, a vazut Domnul ?i I s-a facut mila pentru aceasta nenorocire ?i a zis catre îngerul pierzator: "Destul! De acum lasa-?i mâinile în jos!" Îngerul Domnului statea atunci deasupra ariei lui Ornan (Aravna) Iebuseul.
1Ch 21:16  Atunci ridicându-?i David ochii sai, a vazut pe îngerul Domnului stând între pamânt ?i cer cu sabia goala în mâna sa, întinsa asupra Ierusalimului; ?i a cazut David ?i batrânii cu fe?ele la pamânt, îmbraca?i în sac.
1Ch 21:17  ?i a zis David catre Dumnezeu: "Oare nu eu am poruncit sa se numere poporul? Eu am gre?it, eu am facut rau; dar aceste oi ce au facut? Doamne Dumnezeul meu, sa vina mâna Ta asupra mea, asupra casei tatalui meu, iar nu asupra poporului Tau, ca sa-l piarda pe el!"
1Ch 21:18  Iar îngerul Domnului a zis lui Gad proorocul sa spuna lui David: "Sa se suie David ?i sa ridice un jertfelnic Domnului în aria lui Ornan Iebuseul".
1Ch 21:19  ?i s-a dus David, dupa cuvântul lui Gad pe care i-l graise în numele Domnului.
1Ch 21:20  Ornan, întorcându-se, a vazut îngerul, ?i cei trei fii ai lui s-au ascuns împreuna cu el; în vremea aceea Ornan treiera.
1Ch 21:21  A venit deci David la Ornan; iar Ornan, privind ?i vazând pe David, a ie?it din arie ?i s-a plecat înaintea lui David cu fa?a pâna la pamânt.
1Ch 21:22  Atunci David a zis catre Ornan: "Da-mi un loc în arie, ca am sa fac pe el jertfelnic Domnului; da-mi-l cu pre?ul cât costa, ca sa înceteze prapadul în popor".
1Ch 21:23  Iar Ornan a zis lui David: "Ia-?i! Faca domnul meu regele ce binevoie?te! Iata eu dau ?i boi pentru ardere de tot ?i uneltele de treierat ca lemne pentru foc ?i grâu pentru prinos. Toate acestea le dau în dar!"
1Ch 21:24  Regele David a zis catre Ornan: "Nu, eu voiesc sa cumpar de la tine cu pre? adevarat, caci nu ma voi apuca sa aduc avutul tau Domnului ?i nu voi aduce ardere de tot vite luate în dar".
1Ch 21:25  ?i a dat David lui Ornan pentru acest loc ?ase sute de sicli de aur.
1Ch 21:26  Apoi a facut David acolo jertfelnic Domnului ?i a înal?at ardere de tot ?i jertfe de împacare ?i a chemat pe Domnul ?i Dumnezeu l-a auzit trimi?ând foc din cer pe altarul arderii de tot.
1Ch 21:27  Atunci a zis Domnul catre înger: "Pune-?i sabia în teaca".
1Ch 21:28  În vremea aceasta, vazând David ca Domnul l-a auzit în aria lui Ornan Iebuseul, a adus acolo jertfa.
1Ch 21:29  Cortul Domnului însa, pe care-l facuse Moise în pustiu ?i altarul arderii de tot se aflau în vremea aceea pe înal?imea de la Ghibeon.
1Ch 21:30  ?i nu s-a putut duce David acolo, ca sa întrebe pe Domnul, pentru ca era îngrozit de sabia îngerului Domnului.
1Ch 22:1  Apoi a zis David: "Iata templul Domnului Dumnezeu ?i iata jertfelnicul arderilor de tot al lui Israel!"
1Ch 22:2  ?i a poruncit David sa aduca pe strainii din pamântul lui Israel ?i i-a pus cioplitori de piatra ca sa ciopleasca pietre pentru zidirea templului Domnului.
1Ch 22:3  Apoi a pregatit David fier foarte mult pentru piroane la ferecarea u?ilor ?i pentru legaturi; arama atât de multa încât nu se mai putea cântari;
1Ch 22:4  Lemne de cedru, nemasurat, pentru ca Sidonienii ?i Tirienii adusesera lui David foarte mult lemn de cedru.
1Ch 22:5  Dupa aceea a zis David: "Solomon, fiul meu, e tânar ?i cu pu?ina putere, iar templul care are a se zidi pentru Domnul trebuie sa fie foarte mare?, spre slava ?i podoaba a toata lumea; deci voi pregati eu toate pentru el". ?i a pregatit David multe pâna la moartea sa.
1Ch 22:6  Chemând apoi pe fiul sau Solomon, i-a poruncit sa ridice templul Domnului Dumnezeului lui Israel.
1Ch 22:7  ?i a zis David lui Solomon: "Fiul meu, eu am avut la inima sa zidesc templu în numele Domnului Dumnezeului meu;
1Ch 22:8  Dar a fost catre mine cuvântul Domnului ?i a zis: Tu ai varsat mult sânge ?i ai purtat razboaie mari; nu se cuvine sa zide?ti tu casa numelui Meu, pentru ca ai varsat mult sânge pe pamânt înaintea fe?ei Mele.
1Ch 22:9  Iata însa ?ie ?i se va na?te un fiu; acela va fi pa?nic ?i Eu îi voi da lini?te din partea tuturor vrajma?ilor dimprejur; de aceea numele lui va fi Solomon. ?i voi da lui Israel în zilele lui pace ?i lini?te.
1Ch 22:10  El va zidi templu numelui Meu ?i el Îmi va fi fiu, iar Eu îi voi fi tata ?i voi întari tronul domniei lui peste Israel pe veci.
1Ch 22:11  ?i acum, fiul meu, sa fie Domnul cu tine, ca sa ai spor ?i sa zide?ti templul Domnului Dumnezeului tau, cum a vorbit El despre tine.
1Ch 22:12  Sa-?i dea ?ie Domnul minte ?i în?elepciune ?i sa te puna peste Israel! Sa paze?ti legile Domnului Dumnezeului tau!
1Ch 22:13  Atunci vei fi cu spor, de te vei sili sa împline?ti a?ezamintele ?i legile pe care le-a dat Domnul lui Moise pentru Israel. Fii tare ?i curajos, nu te teme ?i nu te deznadajdui.
1Ch 22:14  ?i iata eu din saracia mea am pregatit pentru templul Domnului o suta de mii de talan?i de aur ?i un milion de talan?i de argint, iar fier ?i arama fara numar, pentru ca acestea sunt foarte multe; ?i lemn ?i piatra de asemenea am pregatit ?i sa mai adaugi ?i tu la acestea.
1Ch 22:15  Ai mul?ime de lucratori ?i cioplitori de piatra, sculptori, dulgheri ?i tot felul de oameni pricepu?i la tot felul de lucruri.
1Ch 22:16  Aur, argint, arama ?i fier ai cât nu se pot cântari; începe ?i fa! Domnul va fi cu tine".
1Ch 22:17  Apoi a poruncit David tuturor mai-marilor lui Israel sa ajute lui Solomon, fiul sau, zicând:
1Ch 22:18  "Au nu este cu voi Domnul Dumnezeul vostru, Care v-a daruit pace din toate par?ile, pentru ca El a dat în mâinile mele pe locuitorii ?arii ?i s-a supus ?ara înaintea Domnului ?i înaintea poporului Sau?
1Ch 22:19  A?adar îndrepta?i-va inima ?i sufletul vostru, ca sa caute pe Domnul Dumnezeul vostru! Scula?i-va ?i zidi?i loca? sfânt Domnului Dumnezeu, ca sa muta?i chivotul legamântului Domnului ?i vasele sfinte ale lui Dumnezeu în loca?ul zidit numelui Domnului".
1Ch 23:1  Îmbatrânind David ?i saturându-se de via?a, a facut rege peste Israel pe fiul sau Solomon.
1Ch 23:2  A adunat pe toate capeteniile lui Israel ?i pe preo?i ?i pe levi?i.
1Ch 23:3  ?i au fost numara?i levi?ii, de la treizeci de ani în sus ?i s-a aflat numarul lor, numara?i pe cap, treizeci ?i opt de mii de barba?i.
1Ch 23:4  Dintre ei au fost rândui?i pentru lucru în templul Domnului douazeci ?i patru de mii; ?ase mii sa fie scriitori ?i judecatori,
1Ch 23:5  Patru mii sa fie portari ?i patru mii sa laude pe Domnul cu instrumentele muzicale pe care el le facuse pentru aceasta.
1Ch 23:6  ?i i-a împar?it David în cete, care sa faca de rând, dupa fiii lui Levi: Gher?on, Cahat ?i Merari.
1Ch 23:7  Din Gher?oni?i erau Laedan ?i ?imei.
1Ch 23:8  Fiii lui Laedan au fost trei: Iehiel, capetenie, Zetam ?i Ioil.
1Ch 23:9  Fiii lui ?imei au fost trei: ?elomit, Haziel ?i Haran. Ace?tia sunt capeteniile familiilor lui Laedan.
1Ch 23:10  Tot fii ai lui ?imei au mai fost: Iahat, Ziza, Ieu? ?i Beraia. Ace?ti patru sunt tot fii ai lui ?imei.
1Ch 23:11  Iahat a fost capetenie; Ziza era al doilea; Ieu? ?i Beraia au avut pu?ini copii ?i de aceea ei au fost socoti?i la un loc la casa tatalui lor.
1Ch 23:12  Fiii lui Cahat au fost patru: Amram, I?har, Hebron ?i Uziel.
1Ch 23:13  Fiii lui Amram au fost: Aaron ?i Moise. Aaron a fost ales ca sa fie sfin?it pentru Sfânta Sfintelor, el ?i fiii lui pe veci, pentru a savâr?i tamâierea înaintea fe?ei Domnului, ca sa-I slujeasca Lui ?i sa binecuvânteze numele Lui în veci.
1Ch 23:14  Iar Moise, omul lui Dumnezeu ?i fiii lui au fost numara?i la tribul lui Levi.
1Ch 23:15  Fiii lui Moise au fost: Gher?om ?i Eliezer.
1Ch 23:16  Fiul lui Gher?om a fost ?ebuel, capetenia.
1Ch 23:17  Fiul lui Eliezer a fost: Rehabia, capetenia. Eliezer n-a mai avut al?i copii. Rehabia însa a avut foarte mul?i copii.
1Ch 23:18  Fiul lui I?har a fost ?elomit, capetenia.
1Ch 23:19  Fiii lui Hebron au fost: întâiul Ieria, al doilea Amaria, al treilea Iahaziel ?i al patrulea Iecameam.
1Ch 23:20  Fiii lui Uziel au fost: întâiul Mica ?i al doilea I?ia.
1Ch 23:21  Fiii lui Merari au fost: Mahli ?i Mu?i; fiii lui Mahli au fost: Eleazar ?i Chi?.
1Ch 23:22  Eleazar însa a murit ?i n-a avut feciori, ci numai fete ?i le-au luat de so?ii fiii lui Chi?, verii lor.
1Ch 23:23  Fiii lui Mu?i au fost trei: Mahli, Eder ?i Ieremot.
1Ch 23:24  Ace?tia sunt fiii lui Levi, dupa casele parin?ilor lor, adica a capilor de familie, dupa numaratoarea lor pe nume ?i pe capete, care faceau slujba la templul Domnului de la douazeci de ani în sus.
1Ch 23:25  Caci David a zis: "Domnul Dumnezeul lui Israel a dat lini?te poporului Sau ?i l-a a?ezat în Ierusalim pe veci,
1Ch 23:26  ?i levi?ii nu vor mai duce cortul ?i tot felul de lucruri ale lui pentru slujbele din el".
1Ch 23:27  De aceea, dupa cele din urma porunci ale lui David, au fost numara?i levi?ii de la douazeci de ani în sus,
1Ch 23:28  Ca sa fie pe lânga fiii lui Aaron, pentru a sluji la templul Domnului, în curte ?i în camerele din jur, pentru cura?irea a tot ceea ce este sfânt ?i pentru savâr?irea slujbelor în templul lui Dumnezeu,
1Ch 23:29  Pentru a îngriji de pâinile punerii înainte, de faina de grâu pentru prinosul de pâine ?i azime, pentru cele de copt, de fript ?i de toata masura ?i cântarul,
1Ch 23:30  ?i pentru a sta diminea?a ?i seara sa laude ?i sa slaveasca pe Domnul,
1Ch 23:31  ?i la toate arderile de tot aduse Domnului sâmbata, la luna noua ?i la sarbatori, dupa numar, cum este scris pentru ele, sa fie totdeauna înaintea Domnului,
1Ch 23:32  ?i ca sa pazeasca chivotul legii ?i loca?ul sfânt ?i pe fiii lui Aaron, fra?ii lor, la slujbele din templul Domnului.
1Ch 24:1  Iata acum cetele în care au fost împar?i?i fiii lui Aaron: Fiii lui Aaron au fost: Nadab, Abiud, Eleazar ?i Itamar.
1Ch 24:2  Nadab ?i Abiud au murit înainte de tatal lor, iar copii n-au avut ?i de aceea au preo?it numai pe Eleazar ?i Itamar.
1Ch 24:3  David i-a împar?it astfel: Pe ?adoc, unul din fiii lui Eleazar ?i pe Ahimelec, unul din fiii lui Itamar, i-a pus sa faca slujbele cu rândul.
1Ch 24:4  S-au gasit însa între fiii lui Eleazar mai multe capetenii decât între fiii lui Itamar ?i i-a împar?it astfel: Din fiii lui Eleazar ?aisprezece capi de familii, iar din fiii lui Itamar opt.
1Ch 24:5  ?i i-a împar?it prin sor?i, pentru ca cei mai însemna?i în loca?ul sfânt ?i înaintea lui erau dintre fiii lui Eleazar ?i dintre fiii lui Itamar.
1Ch 24:6  ?i i-a înscris ?emaia, fiul lui Natanael, scriitor din levi?i, înaintea fe?ei regelui ?i a capeteniilor, înaintea preo?ilor ?adoc ?i Ahimelec, fiul lui Abiatar ?i înaintea capilor de familie ai preo?ilor ?i levi?ilor, luând prin tragere la sor?i o familie din neamul lui Eleazar ?i apoi una din neamul lui Itamar.
1Ch 24:7  Întâiul sor? a cazut lui Iehoiarib, al doilea lui Iedaia,
1Ch 24:8  Al treilea lui Harim, al patrulea lui Seorim,
1Ch 24:9  Al cincilea lui Malchia, al ?aselea lui Miiamin,
1Ch 24:10  Al ?aptelea lui Haco?, al optulea lui Abia,
1Ch 24:11  Al noualea lui Ie?ua, al zecelea lui ?ecania,
1Ch 24:12  Al unsprezecelea lui Elia?ib, al doisprezecelea lui Iachim,
1Ch 24:13  Al treisprezecelea lui Hupa, al paisprezecelea lui I?baal,
1Ch 24:14  Al cincisprezecelea lui Bilga, al ?aisprezecelea lui Imer,
1Ch 24:15  Al ?aptesprezecelea lui Hezir, al optsprezecelea lui Hapi?e?,
1Ch 24:16  Al nouasprezecelea lui Petahia, al douazecilea lui Iezechiel,
1Ch 24:17  Al douazeci ?i unulea lui Iachin, al douazeci ?i doilea lui Gamul,
1Ch 24:18  Al douazeci ?i treilea lui Delaia ?i al douazeci ?i patrulea lui Maazia.
1Ch 24:19  Aceasta era în?irarea lor la slujba, cum trebuia sa vina în templul Domnului, dupa rânduiala lor data prin Aaron, tatal lor, cum poruncise acestuia Domnul Dumnezeul lui Israel.
1Ch 24:20  Ceilal?i fiii ai lui Levi au fost împar?i?i astfel: Din fiii lui Amram: ?ubael; din fiii lui ?ubael: Iehdia;
1Ch 24:21  Din fiii lui Rehabia, întâiul era I?ia;
1Ch 24:22  Din ai lui I?har, ?elomot; din ai lui ?elomot era Iahat;
1Ch 24:23  Din ai lui Hebron întâiul era Ieria, al doilea, Amaria, al treilea, Iahaziel, al patrulea, Iecameam.
1Ch 24:24  Din ai lui Uziel era Mica; din ai lui Mica era ?amir.
1Ch 24:25  Fratele lui Mica a fost I?ia; din fiii lui I?ia era Zaharia.
1Ch 24:26  Fiii lui Merari au fost: Mahli ?i Mu?i; din fiii lui Iaazia a fost Beno;
1Ch 24:27  Din fiii lui Merari, dupa Iaazia, au fost: Beno, ?oham, Zacur ?i Ibri.
1Ch 24:28  Mahli a avut pe Eleazar, iar acesta n-a avut fii.
1Ch 24:29  Chi? a avut pe Ierahmeel.
1Ch 24:30  Fiii lui Mu?i au fost: Mahli, Eder ?i Ierimot. Ace?tia sunt fiii levi?ilor, dupa familii.
1Ch 24:31  Au aruncat ?i ei sor?i la fel ca ?i fra?ii lor, fiii lui Aaron, înaintea fe?ei regelui David, a lui ?adoc ?i Ahimelec, ?i a capilor familiilor preo?e?ti ?i levite, fara sa se faca deosebire între cei batrâni ?i cei tineri.
1Ch 25:1  Apoi David ?i capeteniile o?tirii au împar?it la slujba pe fiii lui Asaf, ai lui Heman ?i ai lui Iedutun, ca sa prooroceasca acompania?i de harfe, alaute ?i chimvale.
1Ch 25:2  Din fiii lui Asaf au fost rândui?i la slujbele lor ace?tia: Zacur, Iosif, Netania ?i A?arela, fiii lui Asaf, sub conducerea lui Asaf, care cântau dupa porunca regelui.
1Ch 25:3  Din ai lui Iedutun au fost rândui?i fiii lui Iedutun: Ghedalia, ?eri, Isaia, ?imei, Ha?abia ?i Matitia; ei erau ?ase sub conducerea tatalui lor Iedutun, care cânta din chitara spre slava ?i lauda Domnului.
1Ch 25:4  Din ai lui Heman au fost rândui?i fiii lui Heman: Buchia, Matania, Uziel, ?ebuel, Ierimot, Hanania, Hanani, Eliata, Ghidalti, Romamti-Ezer, Io?beca?a, Maloti, Hotir ?i Mahaziot.
1Ch 25:5  To?i ace?tia sunt fiii lui Heman, care era vazatorul regelui, dupa cuvintele lui Dumnezeu, ca sa mareasca slava Lui. ?i i-a dat Dumnezeu lui Heman paisprezece fii ?i trei fete.
1Ch 25:6  To?i ace?tia cântau sub conducerea tatalui lor în templul Domnului din chimvale, psaltirioane ?i harpe, la slujbele din templul Domnului, dupa aratarile lui David sau ale lui Asaf, Iedutun ?i Heman.
1Ch 25:7  Iar numarul lor, cu al fra?ilor lor care înva?asera sa cânte înaintea Domnului, ?i al tuturor care ?tiau acest lucru era doua sute optzeci ?i opt.
1Ch 25:8  Au aruncat ?i ei sor?i pentru rândul la slujba, mic cu mare, dascal ?i ucenic deopotriva.
1Ch 25:9  Întâiul sor? a cazut pentru Iosif, fiul lui Asaf, cu fra?ii ?i fiii lui; ei erau doisprezece; al doilea, lui Ghedalia cu fra?ii ?i fiii lui, ei erau doisprezece;
1Ch 25:10  Al treilea, lui Zacur cu fra?ii lui ?i cu fiii lui; ei erau doisprezece.
1Ch 25:11  Al patrulea, lui I?ri cu fiii ?i cu fra?ii lui; ei erau doisprezece.
1Ch 25:12  Al cincilea, lui Netania cu fiii ?i fra?ii lui; ei erau doisprezece.
1Ch 25:13  Al ?aselea, lui Buchia cu fiii ?i fra?ii lui; ei erau doisprezece.
1Ch 25:14  Al ?aptelea lui Ie?arela cu fiii ?i fra?ii lui; ei erau doisprezece.
1Ch 25:15  Al optulea, lui Isaia cu fiii ?i fra?ii lui; ei erau doisprezece.
1Ch 25:16  Al noualea, lui Matania cu fiii ?i fra?ii lui; ei erau doisprezece.
1Ch 25:17  Al zecelea, lui ?imei cu fiii ?i cu fra?ii lui; ei erau doisprezece.
1Ch 25:18  Al unsprezecelea, lui Azareel cu fiii ?i fra?ii lui; ei erau doisprezece.
1Ch 25:19  Al doisprezecelea, lui Ha?abia cu fiii ?i fra?ii lui; ei erau doisprezece.
1Ch 25:20  Al treisprezecelea, lui ?ebuel cu fiii ?i fra?ii lui; ei erau doisprezece.
1Ch 25:21  Al paisprezecelea, lui Matitia cu fiii ?i fra?ii lui; ei erau doisprezece.
1Ch 25:22  Al cincisprezecelea, lui Ierimot cu fiii ?i fra?ii lui; ei erau doisprezece.
1Ch 25:23  Al ?aisprezecelea, lui Hanania cu fiii ?i fra?ii lui; ei erau doisprezece.
1Ch 25:24  Al ?aptesprezecelea, lui Io?beca?a cu fiii ?i cu fra?ii lui; ei erau doisprezece.
1Ch 25:25  Al optsprezecelea, lui Hanani cu fiii ?i fra?ii lui; ace?tia erau doisprezece.
1Ch 25:26  Al nouasprezecelea, lui Maloti cu fiii ?i fra?ii lui; ace?tia erau doisprezece.
1Ch 25:27  Al douazecilea, lui Eliata cu fiii ?i fra?ii lui; ei erau doisprezece.
1Ch 25:28  Al douazeci ?i unulea, lui Hotir cu fiii ?i fra?ii lui; ei erau doisprezece.
1Ch 25:29  Al douazeci ?i doilea, lui Ghidalti cu fiii ?i fra?ii lui; ei erau doisprezece.
1Ch 25:30  Al douazeci ?i treilea, lui Mahaziot cu fiii ?i fra?ii lui; ace?tia erau doisprezece.
1Ch 25:31  Al douazeci ?i patrulea, lui Romamti-Ezer cu fiii ?i fra?ii lui; ace?tia erau doisprezece.
1Ch 26:1  Iata acum împar?irea portarilor. Din fiii lui Core: Me?elemia, fiul lui Core, unul din fiii lui Asaf.
1Ch 26:2  Fiii lui Me?elemia au fost: întâiul nascut Zaharia, al doilea Iediael, al treilea Zebadia, al patrulea Iatniel,
1Ch 26:3  Al cincilea Elam, al ?aselea Iohanan, al ?aptelea Elihoenai.
1Ch 26:4  Fiii lui Obed-Edom au fost: Întâiul nascut ?emaia, al doilea Iehozabad, al treilea Ioah, al patrulea Sacar, al cincilea Natanael,
1Ch 26:5  Al ?aselea Amiel, al ?aptelea Isahar, al optulea Peultai, pentru ca Dumnezeu l-a binecuvântat.
1Ch 26:6  Fiului sau ?emaia i s-a nascut de asemenea fii, care au fost capetenii în neamul lor, pentru ca au fost oameni puternici.
1Ch 26:7  Fiii lui ?emaia au fost: Otni, Rafael, Obed ?i Elzabad; fra?ii lui, oameni puternici, au fost: Elihu, Semachia ?i Isbacom.
1Ch 26:8  To?i ace?tia sunt dintre fiii lui Obed-Edom; ei, fiii lor ?i fra?ii lor, au fost oameni sârguincio?i ?i la slujba pricepu?i; au fost ?aizeci ?i doi din Obed-Edom.
1Ch 26:9  Me?elemia a avut optsprezece fii ?i fra?i, oameni vrednici.
1Ch 26:10  Hosa, unul din fiii lui Merari, a avut fii pe ?imri, capetenie, de?i n-a fost întâiul nascut, dar tatal sau l-a pus capetenie;
1Ch 26:11  Al doilea Hilchia, al treilea Tebalia, al patrulea Zaharia; to?i fiii ?i fra?ii lui Hosa au fost treisprezece.
1Ch 26:12  A?a a fost împar?irea portarilor, dupa capii de familie, vrednici de slujba, împreuna cu fra?ii lor, ca sa slujeasca la templul Domnului.
1Ch 26:13  ?i au aruncat ?i ei sor?i, mare ?i mic, dupa familiile lor, pentru fiecare poarta.
1Ch 26:14  ?i pentru poarta dinspre rasarit a cazut sor?ul lui ?elemia; ?i lui Zaharia, fiul lui, sfetnic în?elept, i s-a aruncat sor? ?i i-a cazut sor? pentru poarta de miazanoapte.
1Ch 26:15  Lui Obed-Edom i-a cazut poarta dinspre miazazi; iar fiilor lui le-a cazut sor?ul pentru camari.
1Ch 26:16  Lui ?upim ?i Hosa le-a cazut pentru cea dinspre apus, la por?ile ?elechet, unde drumul urca ?i unde sunt straji fa?a în fa?a.
1Ch 26:17  Spre rasarit câte ?ase levi?i, spre miazanoapte câte patru, spre miazazi câte patru, iar la camari câte doi.
1Ch 26:18  Spre apus, în fa?a pridvorului la drum, câte patru, iar la pridvor câte doi.
1Ch 26:19  Acestea sunt cetele de portari din fiii lui Core ?i din fiii lui Merari.
1Ch 26:20  Iar al?ii dintre levi?i, fra?ii lor, pazeau vistieria templului lui Dumnezeu ?i vistieria lucrurilor sfinte,
1Ch 26:21  ?i anume: Fiii lui Laedan, fiul lui Gher?on, Capeteniile familiilor din Laedan Gher?onitul: Iehiel,
1Ch 26:22  ?i fiii lui Iehiel: Zetam ?i Ioil, fratele lui, care pazeau vistieria templului lui Dumnezeu,
1Ch 26:23  Împreuna cu urma?ii lui Amram I?har, Hebron, Uziel;
1Ch 26:24  ?ebuel, fiul lui Gher?on, fiul lui Moise, era strajuitor de capetenie al vistieriilor.
1Ch 26:25  Fratele sau Eleazar avea fiu pe Rehabia; acesta a avut fiu pe Isaia; acesta a avut fiu pe Ioram; acesta a avut fiu pe Zicri, iar acesta a avut fiu pe ?elomit.
1Ch 26:26  ?elomit ?i fra?ii lui privegheau asupra tuturor vistieriilor lucrurilor sfinte care le harazise regele David, capeteniile familiilor,  capeteniile peste mii ?i peste sute ?i capeteniile de o?tire.
1Ch 26:27  Din cuceriri ?i prazi ei afierosisera pentru între?inerea templului Domnului
1Ch 26:28  ?i tot ce afierosise Samuel proorocul ?i Saul, fiul lui Chi?, Abner, fiul lui Ner, ?i Ioab, fiul ?eruiei; toate cele afierosite erau în grija lui ?elomit ?i a fra?ilor lui.
1Ch 26:29  Din neamul lui I?har, Hanania ?i fiii lui erau rândui?i la slujbele din afara ale Israeli?ilor, ca scriitori ?i judecatori.
1Ch 26:30  Din neamul lui Hebron, Ha?abia ?i fiii lui, oameni curajo?i, în numar de o mie ?apte sute, aveau supravegherea asupra lui Israel de cealalta parte de Iordan, spre apus, pentru tot felul de slujbe ale Domnului ?i ale regelui.
1Ch 26:31  În neamul Hebroni?ilor, Ieria era capetenia Hebroni?ilor, în neamul ?i familiile lor. În anul al patruzecilea al domniei lui David ei au fost numara?i ?i s-au gasit între ei barba?i curajo?i în Iazerul Galaadului.
1Ch 26:32  ?i fra?ii lui, oameni vrednici, în numar de doua mii ?apte sute erau capi de familie. Pe ace?tia i-a pus regele David peste triburile lui Ruben ?i Gad ?i peste jumatate din semin?ia lui Manase, pentru toate treburile lui Dumnezeu ?i ale regelui.
1Ch 27:1  Iata fiii lui Israel, dupa numarul lor, capii de familie, capeteniile peste mii, peste sute ?i cârmuitorii care, împar?i?i în cete, slujeau regelui la tot cuvântul, ducându-se ?i venind în fiecare luna, în toate lunile anului. În fiecare ceata erau câte douazeci ?i patru de mii.
1Ch 27:2  Peste ceata întâi, pentru luna întâi, era capetenia Ia?obeam, fiul lui Zabdiel; în ceata lui erau douazeci ?i patru de mii;
1Ch 27:3  El era din fiii lui Fares, mai-mare peste toate capeteniile de o?tire în luna întâi.
1Ch 27:4  Peste ceata din luna a doua era Dodai Ahohitul; în ceata lui se afla ?i capetenia Miclot; ?i în ceata lui erau douazeci ?i patru de mii.
1Ch 27:5  A treia mare capetenie de o?tire, pentru luna a treia, era Benaia, fiul lui Iehoiada preotul; ?i în ceata lui erau douazeci ?i patru de mii.
1Ch 27:6  Acest Benaia era unul dintre cei treizeci de viteji ?i capetenie peste ei; ?i în ceata lui se afla Amizabad, fiul sau.
1Ch 27:7  A patra capetenie, pentru luna a patra, era Asael, fratele lui Ioab, ?i dupa el era Zebadia, fiul sau; ?i în ceata lui erau douazeci ?i patru de mii;
1Ch 27:8  A cincea capetenie, pentru luna a cincea, era ?amhut Izrahitul; ?i în ceata lui erau douazeci ?i patru de mii.
1Ch 27:9  A ?asea capetenie, pentru luna a ?asea, era Ira, fiul lui Iche? Tecoanul; ?i în ceata lui erau douazeci ?i patru de mii.
1Ch 27:10  A ?aptea capetenie, pentru luna a ?aptea, era Hele? Peloninul, din fiii lui Efraim; ?i în ceata lui erau douazeci ?i patru de mii.
1Ch 27:11  A opta capetenie, pentru luna a opta, era Sibecai Hu?atitul, din semin?ia lui Zarah; ?i în ceata lui erau douazeci ?i patru de mii.
1Ch 27:12  A noua capetenie, pentru luna a noua, era Abiezer Anatoteanul, din fiii lui Veniamin; ?i în ceata lui erau douazeci ?i patru de mii.
1Ch 27:13  A zecea capetenie, pentru luna a zecea, era Maherai din Netofat, din familia lui Zara; ?i în ceata lui erau douazeci ?i patru de mii.
1Ch 27:14  A unsprezecea capetenie, pentru luna a unsprezecea, era Benaia din Piraton, din fiii lui Efraim; ?i în ceata lui erau douazeci ?i patru de mii.
1Ch 27:15  A douasprezecea capetenie, pentru luna a douasprezecea, era Heldai din Netofat, din urma?ii lui Otniel; ?i în ceata lui erau douazeci ?i patru de mii.
1Ch 27:16  Iar peste triburile lui Israel capetenii înalte erau: la Rubeni?i, Eliezer, fiul lui Zicri; la Simeon, ?efatia, fiul lui Maaca;
1Ch 27:17  La levi?i era Ha?abia, fiul lui Chemuel; la Aaron era ?adoc;
1Ch 27:18  La Iuda era Elihu, din fra?ii lui David; la Isahar era Omri, fiul lui Micael;
1Ch 27:19  La Zabulon era I?maia, fiul lui Obadia; la Neftali era Ierimot, fiul lui Azriel;
1Ch 27:20  La fiii lui Efraim era Hoseia, fiul lui Azazia; la jumatatea de trib a lui Manase era Ioil, fiul lui Pedaia;
1Ch 27:21  La jumatatea de trib al lui Manase din Galaad, era Ido, fiul lui Zaharia; la Veniamin era Iaasiel, fiul lui Abner;
1Ch 27:22  La Dan era Azareel, fiul lui Ieroham. Iata capeteniile triburilor lui Israel.
1Ch 27:23  David n-a facut numaratoarea acelora, care erau de la douazeci de ani în jos, pentru ca Domnul zisese ca El va înmul?i pe Israel ca stelele cerului.
1Ch 27:24  Ioab, fiul ?eruiei, începuse sa faca numaratoarea, dar nu o sfâr?ise. ?i pentru aceasta a venit mânia lui Dumnezeu asupra lui Israel ?i numaratoarea aceea n-a intrat în cronica regelui David.
1Ch 27:25  Peste vistieriile regale era Azmavet, fiul lui Adiel, iar peste depozitele de provizii de la câmp, de prin ceta?i ?i de prin sate ?i turnuri era Ionatan, fiul lui Uzia.
1Ch 27:26  Peste cei ce se îndeletniceau cu lucrul câmpului, cu agricultura, era Ezri, fiul lui Chelub.
1Ch 27:27  Peste vii era ?imei din Rama, iar peste depozitele de vin din vii era Zabdi, fiul lui ?ifmi.
1Ch 27:28  Peste livezile de maslini ?i de smochini din câmpie era Baal-Hanan din Gheder, iar peste depozitele de untdelemn era Ioa?.
1Ch 27:29  Peste vitele mari care pa?teau în ?aron era ?itrai Ha?aroneanul; iar peste cele din vai, ?afat, fiul lui Adlai.
1Ch 27:30  Peste camile era Obil Ismaelitul; peste asini era Iehdia Meroneanul.
1Ch 27:31  Peste oi ?i capre era Iaziz Agariteanul. To?i ace?tia erau capetenii peste averea regelui David.
1Ch 27:32  Ionatan, unchiul lui David, era sfetnic, om în?elept ?i scriitor; Iehiel, fiul lui Hacmoni, era pe lânga fiii regelui.
1Ch 27:33  Ahitofel era sfetnicul regelui; Hu?ai Architul era prietenul regelui.
1Ch 27:34  Iar dupa Ahitofel a fost Iehoiada, fiul lui Benaia ?i Abiatar, iar Ioab era capetenia o?tirii pe lânga rege.
1Ch 28:1  Apoi a adunat David la Ierusalim pe toate capeteniile lui Israel, pe mai marii triburilor, capeteniile cetelor care slujeau regelui, capeteniile peste mii, peste sute, îngrijitorii mo?iilor ?i turmelor regelui, pe fiii sai cu eunucii, capeteniile o?tirii ?i pe to?i vitejii lui.
1Ch 28:2  ?i sculându-se, regele David a zis: "Asculta?i-ma, fra?ilor ?i poporul meu! Am avut la inima mea gând sa zidesc loca? de odihna pentru chivotul legamântului Domnului ?i a?ternut picioarelor Dumnezeului nostru ?i cele de trebuin?a pentru zidire le-am pregatit.
1Ch 28:3  Dar Dumnezeu mi-a zis: Sa nu zide?ti loca? numelui Meu, pentru ca tu e?ti om razboinic ?i ai varsat sânge.
1Ch 28:4  Cu toate acestea m-a ales Domnul Dumnezeul lui Israel din toata casa tatalui meu, ca sa fiu rege peste Israel în veci, pentru ca pe Iuda l-a  ales El domn, iar din casa lui Iuda a ales casa tatalui meu ?i din casa tatalui meu ?i dintre fiii tatalui meu a binevoit a ma pune pe mine rege peste tot Israelul;
1Ch 28:5  Iar din to?i fiii mei - caci mul?i fii mi-a dat Domnul - a ales pe Solomon, fiul meu, sa ?ada pe tronul regatului Domnului, peste Israel.
1Ch 28:6  ?i mi-a zis: Solomon, fiul tau, va zidi loca?ul Meu ?i cur?ile Mele, pentru ca Mi l-am ales pe el de fiu ?i Eu îi voi fi lui tata.
1Ch 28:7  ?i voi întari domnia lui pe veci, daca va fi tare în împlinirea poruncilor Mele ?i a a?ezamintelor Mele, ca pâna astazi.
1Ch 28:8  ?i acum înaintea ochilor a tot Israelul, a adunarii Domnului ?i în auzul Dumnezeului nostru va graiesc: Pazi?i ?i ?ine?i toate poruncile Domnului Dumnezeului vostru, ca sa stapâni?i tot pamântul cel bun ?i sa-l lasa?i dupa voi mo?tenire copiilor vo?tri pe veci.
1Ch 28:9  ?i tu, Solomon, fiul meu, cunoa?te pe Dumnezeul tatalui tau ?i Îi sluje?te din toata inima ?i din tot sufletul, caci Domnul cerceteaza toate inimile ?i cunoa?te toata mi?carea gândurilor. De Îl vei cauta pe El, Îl vei gasi, iar de Îl vei parasi ?i El te va parasi.
1Ch 28:10  Baga de seama însa, de vreme ce Domnul te-a ales sa zide?ti loca? sfin?eniei Lui, fii tare ?i fa ceea ce a rânduit El".
1Ch 28:11  ?i a dat David lui Solomon, fiul sau, planul pridvorului ?i al caselor lui, al camarilor lui, al odailor celor mari de primire, al odailor celor mai dinauntru de odihna ?i al casei chivotului legii.
1Ch 28:12  A dat de asemenea planul tuturor celor ce avea el în gândul sau: Planul cur?ii templului Domnului, al tuturor camarilor dimprejur, al vistieriilor lucrurilor sfinte
1Ch 28:13  Al încaperilor preo?ilor ?i levi?ilor, al tuturor slujitorilor din templul Domnului ?i al tuturor vaselor sfinte din templul Domnului,
1Ch 28:14  Al lucrurilor de aur, cu aratarea greuta?ii, al tuturor vaselor de slujba, al tuturor lucrurilor de argint, cu aratarea greuta?ii lor ?i al tuturor celorlalte vase de slujba.
1Ch 28:15  Apoi i-a dat aurul pentru sfe?nicele ?i pentru candelele de aur ale lor, cu însemnarea greuta?ii fiecaruia din sfe?nice ?i din candelele lui; de asemenea ?i argintul pentru sfe?nicele de argint, cu însemnarea greuta?ii fiecaruia din sfe?nice ?i din candelele lui, potrivit cu menirea de slujba a fiecaruia din sfe?nice;
1Ch 28:16  ?i aurul pentru mesele pâinilor punerii înainte, cu însemnarea greuta?ii fiecareia din mesele de aur ?i argintul pentru mesele de argint.
1Ch 28:17  I-a dat aurul pentru furculi?ele, castroanele ?i cupele cele de aur curat ?i pentru vasele de aur, cu însemnarea greuta?ii fiecarui vas ?i argintul pentru vasele de argint, cu însemnarea greuta?ii fiecarui vas,
1Ch 28:18  Precum ?i aurul pentru jertfelnicul tamâierii, turnat din aur, cu însemnarea greuta?ii. I-a dat modelul carului divin, al heruvimilor de aur, cu aripile întinse pentru acoperirea chivotului legamântului Domnului.
1Ch 28:19  "Toate acestea sunt în scrisoarea insuflata de la Domnul - a zis David - cum m-a luminat El pentru toate lucrarile zidirii".
1Ch 28:20  Apoi a zis David catre fiul sau Solomon: "Fii tare ?i curajos ?i pa?e?te la lucru, nu te teme, nici te speria, caci Domnul Dumnezeu, Dumnezeul meu, este cu tine. El nu se va departa de tine, nici nu te va parasi, pâna nu vei ispravi toata lucrarea ce se cere la templul Domnului.
1Ch 28:21  Iata ?i cetele de preo?i ?i de levi?i pentru toate slujbele cele de la templul lui Dumnezeu. Ai de asemenea oameni sârguincio?i pentru orice lucru ?i iscusi?i pentru orice lucrare; ?i capeteniile ?i tot poporul sunt gata sa împlineasca toate poruncile tale".
1Ch 29:1  Dupa aceea a zis regele David catre toata adunarea: "Solomon, fiul meu, singurul pe care l-a ales Dumnezeu, este tânar ?i cu pu?ina putere, iar lucrul acesta este mare, fiindca nu este pentru om zidirea aceasta, ci pentru Domnul Dumnezeu.
1Ch 29:2  Din toate puterile am pregatit eu pentru templul lui Dumnezeu mult aur pentru lucrurile cele de aur, argint pentru cele de argint, arama pentru cele de arama, fier pentru cele de fier, lemn pentru cele de lemn, pietre de onix ?i pietre pentru încrustat, pietre frumoase de diferite culori ?i tot felul de pietre scumpe ?i multa marmura.
1Ch 29:3  Mai mult! Din dragoste pentru templul Dumnezeului meu, dau înca tot ce am eu aur ?i argint, templului Dumnezeului meu, afara de ceea ce am pregatit eu pentru templul cel sfânt,
1Ch 29:4  ?i anume: trei mii de talan?i de aur, aur de ofir, ?apte mii de talan?i de argint curat, pentru captu?irea pere?ilor în templu,
1Ch 29:5  Pentru orice lucru de aur, pentru tot lucrul de argint ?i pentru tot lucrul de mâna de me?ter. ?i cine vrea sa vina astazi cu mâinile pline la Domnul?"
1Ch 29:6  Au început atunci sa aduca jertfa capii de familii ?i capeteniile triburilor, capeteniile peste mii ?i peste sute ?i capeteniile cele peste averea regelui.
1Ch 29:7  ?i au dat pentru zidirea templului lui Dumnezeu cinci mii de talan?i ?i zece mii de drahme aur, argint zece mii de talan?i, arama optsprezece mii talan?i ?i fier o suta de mii de talan?i.
1Ch 29:8  ?i cei care aveau pietre scumpe, le-au dat ?i pe acelea în vistieria templului Domnului, prin mâinile lui Iehiel Gher?onitul.
1Ch 29:9  ?i s-a bucurat poporul de râvna lor, pentru ca din toata inima au jertfit Domnului. De asemenea s-a bucurat foarte mult ?i regele David.
1Ch 29:10  Atunci a slavit David pe Domnul înaintea a toata adunarea ?i a zis: "Binecuvântat e?ti Tu, Doamne Dumnezeul lui Israel, Tatal nostru, din veac ?i pâna în veac.
1Ch 29:11  A Ta este, Doamne, mare?ia ?i puterea ?i slava ?i biruin?a ?i stralucirea; toate câte sunt în cer ?i pe pamânt sunt ale Tale; a Ta este, Doamne, împara?ia ?i Tu e?ti mai presus de toate, ca unul ce împara?e?ti peste toate.
1Ch 29:12  Boga?ia ?i slava sunt de la fa?a Ta ?i Tu domne?ti peste toate; în mâna Ta este taria ?i puterea ?i în puterea Ta sta sa mare?ti ?i sa întare?ti toate.
1Ch 29:13  ?i acum dar, Dumnezeul nostru, Te slavim pe Tine ?i laudam preaslavit numele Tau.
1Ch 29:14  Ca cine sunt eu ?i cine este poporul meu, încât sa avem putin?a de a face asemenea jertfe? Dar de la Tine sunt toate ?i cele primite din mâna Ta ?i le-am dat ?ie;
1Ch 29:15  Caci calatori suntem noi înaintea Ta ?i pribegi, ca to?i parin?ii no?tri; ca umbra sunt zilele noastre pe pamânt ?i nimic nu este statornic.
1Ch 29:16  Doamne Dumnezeul nostru, toata aceasta mul?ime de lucruri, pe care am pregatit-o noi pentru a zidi templu sfânt numelui Tau, din mâna Ta le avem ?i ale Tale sunt toate.
1Ch 29:17  ?tiu, Dumnezeul meu, ca ispite?ti inimile ?i iube?ti cura?enia inimii! Eu din inima curata am jertfit toate ?i vad acum ca ?i poporul Tau, care se afla aici, cu bucurie Î?i jertfe?te ?ie.
1Ch 29:18  Doamne Dumnezeul lui Avraam ?i al lui Isaac ?i al lui Iacov, parin?ii no?tri, paze?te acestea în veci, aceasta aplecare a gândurilor inimii poporului Tau ?i îndreapta inimile lor catre Tine.
1Ch 29:19  Iar lui Solomon, fiul meu, daruie?te-i inima dreapta ca sa pazeasca poruncile Tale, descoperirile Tale ?i legile Tale ?i sa împlineasca toate acestea ?i sa înal?e cladirea pentru care am facut pregatire".
1Ch 29:20  Apoi a zis David catre toata adunarea: "Binecuvânta?i pe Domnul Dumnezeul nostru!" ?i toata adunarea a binecuvântat pe Domnul Dumnezeul parin?ilor sai ?i a cazut ?i s-a închinat Domnului ?i regelui.
1Ch 29:21  Apoi au adus Domnului jertfe ?i au înal?at arderi de tot Domnului, a doua zi dupa aceasta: o mie de miei, o mie de berbeci ?i o mie de vitei cu turnarile lor ?i o mul?ime de jertfe de la tot Israelul.
1Ch 29:22  ?i au mâncat ?i au baut înaintea Domnului în ziua aceea cu mare bucurie; iar în alt rând au facut rege pe Solomon, fiul lui David, ?i l-au uns înaintea Domnului ca rege, iar pe ?adoc arhiereu.
1Ch 29:23  ?i s-a urcat Solomon pe tronul Domnului, ca rege, în locul lui David, tatal sau, ?i a avut spor ?i tot Israelul s-a supus lui.
1Ch 29:24  S-au supus lui Solomon de asemenea toate capeteniile ?i puternicii, precum ?i to?i fiii lui David.
1Ch 29:25  Iar Domnul a marit pe Solomon în ochii a tot Israelul ?i i-a daruit domnie slavita, cum nu mai avusese înainte de el nici unul din regii lui Israel.
1Ch 29:26  David, fiul lui Iesei, a domnit peste tot Israelul.
1Ch 29:27  Timpul domniei lui peste Israel a fost patruzeci de ani: în Hebron a domnit el ?apte ani, iar în Ierusalim a domnit treizeci ?i trei.
1Ch 29:28  ?i a murit la adânci batrâne?e, satul de via?a, de boga?ie ?i de slava, iar în locul lui s-a facut rege Solomon, fiul sau.
1Ch 29:29  Faptele lui David, cele dintâi ?i cele de pe urma, sunt scrise în însemnarile lui Samuel vazatorul ?i în însemnarile lui Natan proorocul ?i în însemnarile lui Gad vazatorul,
1Ch 29:30  Precum ?i toata domnia lui ?i barba?ia lui ?i întâmplarile ce s-au petrecut cu el ?i cu Israel ?i cu toate împara?iile pamântului.


\end{document}