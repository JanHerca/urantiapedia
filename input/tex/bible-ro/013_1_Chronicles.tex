\begin{document}

\title{1 Cronici}


\chapter{1}

\par 1 Adam, Set, Enos;
\par 2 Chenan, Mahalaleel, Iared;
\par 3 Enoh, Matusalem, Lameh;
\par 4 Noe, Sem, Ham și Iafet.
\par 5 Fiii lui Iafet: Gomer, Magog, Madai, Iavan, Elișa, Tubal, Meșec și Tiras.
\par 6 Fiii lui Gomer: Așchenaz, Rifat și Togarma.
\par 7 Fiii lui Iavan: Elișa, Tarșiș, Chitim și Dodanim.
\par 8 Fiii lui Ham: Cuș, Mițraim, Put și Canaan.
\par 9 Fiii lui Cuș: Seba, Havila, Savta Rama și Sabteca. Fiii lui Rama: Șeba și Dedan.
\par 10 Lui Cuș i s-a mai născut de asemenea și Nimrod. Acesta a început să fie puternic pe pământ.
\par 11 Lui Mițraim i s-a născut: Ludim, Anamim, Lehabim, Naftuhim,
\par 12 Patrusim, Casluhim, din care se trag Filistenii și Caftorim.
\par 13 Lui Canaan i s-au născut: Sidon, întâiul său născut și Het,
\par 14 Iebuseu, Amoreu, Ghergheseu,
\par 15 Heveu, Archeu, Sineu,
\par 16 Arvadeu, Țemareu și Hamateu.
\par 17 Fiii lui Sem: Elam, Asur, Arpaxad, Lud și Aram. Fiii lui Aram: Ut, Hul, Gheter și Meșec.
\par 18 Lui Arpaxad i s-a născut Cainan, lui Cainan i s-a născut Șelah, lui Șelah i s-a născut Eber.
\par 19 Lui Eber i s-au născut doi fii: numele unuia era Peleg, pentru că în zilele lui s-a împărțit țara; iar numele fratelui său era Ioctan.
\par 20 Lui Ioctan i s-au născut: Almodad, Șelef, Hațarmavet, Iarah,
\par 21 Hadoram, Uzal, Dicla,
\par 22 Ebal, Abimael, Șeba,
\par 23 Ofir, Havila și Iobab. Toți aceștia sunt fiii lui Ioctan.
\par 24 Iar fiii lui Sim sunt: Arpaxad, Cainan, Selah,
\par 25 Eber, Peleg, Reu,
\par 26 Serug, Nahor, Terah
\par 27 Și Avram, adică Avraam.
\par 28 Fiii lui Avraam sunt Isaac și Ismael.
\par 29 Iată spița neamului lor: Nebaiot, întâiul născut al lui Ismael, apoi: Chedar, Adbeel, Mibsam,
\par 30 Mișma, Duma, Mașa, Hadad, Tema,
\par 31 Ietur, Nafiș și Chedma. Aceștia sunt fiii lui Ismael.
\par 32 Fiii Cheturei, țiitoarea lui Avraam. Ea a născut pe Zimran, Iocșan, Medan, Madian, Ișbac și Șuah. Fiii lui Iocșan sunt: Șeba și Dedan. Fiii lui Dedan sunt: Raguel, Navdeel, Așurim, Letușim și Leumim.
\par 33 Fiii lui Madian sunt: Efa, Efer, Enoh, Abida și Eldaa. Toți aceștia sunt fiii Cheturei.
\par 34 Lui Avraam i s-a născut Isaac. Fiii lui Isaac sunt: Isav și Israel.
\par 35 Fiii lui Isav sunt: Elifaz, Raguel, Ieuș, Ialam și Core.
\par 36 Fiii lui Elifaz sunt: Teman, Omar, Țefi, Gatam, Chenaz; iar Temna, concubina lui Elifaz, i-a născut pe Amalec.
\par 37 Fiii lui Raguel sunt: Nahat, Zerah, Șama și Miza.
\par 38 Fiii lui Seir sunt: Lotan, Șobal, Țibeon, Ana, Dișon, Ețer și Dișan.
\par 39 Fiii lui Lotan sunt: Hori și Heman; iar sora lui Lotan se numea Timna.
\par 40 Fiii lui Șobal sunt: Alvan, Manahat, Ebal, Șefo și Onam. Fiii lui Țibeon sunt: Aia și Ana.
\par 41 Fiii lui Ana sunt: Dișon și Olibama; fiii lui Dișon sunt: Hemdan, Eșban, Itran și Cheran.
\par 42 Fiii lui Ețer sunt: Bilhan, Zaavan și Acan; fiii lui Dișan sunt: Uț și Aran.
\par 43 Aceștia sunt regii care au domnit în pământul Edom, înainte de a se ridica rege, peste fiii lui Israel, Bela, fiul lui Beor, cetatea căruia se numea Dinhaba.
\par 44 Murind Bela, după el a fost făcut rege Iobab, fiul lui Zerah din Boțra.
\par 45 După moartea lui Iobab s-a făcut rege Hușam, în țara Temaniților.
\par 46 Murind Hușam, s-a făcut rege după el Hadad, fiul lui Bedad, care a lovit pe Madianiți în câmpia Moabului. Orașul lui se numea Avit.
\par 47 Murind Hadad, s-a făcut rege după el Șamla, din Masreca;
\par 48 Murind Șamla, s-a făcut rege după el Șaul, din Rehobotul cel de lângă râu.
\par 49 Murind Șaul, s-a făcut rege după el Baal-Hanan, fiul lui Acbor.
\par 50 Murind Baal-Hanan, s-a făcut rege după el Hadad. Numele cetății lui era Pau, iar numele femeii lui era Mehetabeel, fiica lui Matred, fiica lui Mezahab.
\par 51 Murind Hadad, au urmat căpetenii peste Edom: căpetenia Timna, căpetenia Alia, căpetenia Ietet,
\par 52 Căpetenia Oholibama, căpetenia Ela, căpetenia Pinon,
\par 53 Căpetenia Chenaz, căpetenia Teman, căpetenia Mibțar,
\par 54 Căpetenia Magdiel, căpetenia Iram. Acestea sunt căpeteniile Edomului.

\chapter{2}

\par 1 Iată acum fiii lui Israel: Ruben, Simeon, Levi, Iuda, Isahar, Zabulon,
\par 2 Dan, Iosif, Veniamin, Neftali, Gad și Așer.
\par 3 Fiii lui Iuda sunt: Ir, Onan și Șela. Acești trei i s-au născut lui din fata unui canaanit anume Șua. Ir, întâiul născut al lui Iuda, a fost rău în ochii Domnului și l-a omorât.
\par 4 Tamara, nora lui Iuda, i-a născut acestuia pe Fares și pe Zara. Așa că, de toții, fiii lui Iuda au fost cinci.
\par 5 Fiii lui Fares sunt Hețron și Hamul.
\par 6 Fiii lui Zara sunt: Zimri, Etan, Heman, Calcol și Darda; cinci de toți.
\par 7 Fiul lui Carmi este Acar, care a adus nenorocire asupra lui Israel, călcând jurământul.
\par 8 Fiul lui Etan este Azaria.
\par 9 Fiii lui Hețron care i s-a născut sunt: Ierahmeel, Ram și Chelubai (Caleb).
\par 10 Lui Ram însă i s-a născut Aminadab; lui Aminadab i s-a născut Naason, căpetenia fiilor lui Iuda.
\par 11 Lui Naason i s-a născut Salmon, lui Salmon i s-a născut Booz.
\par 12 Lui Booz i s-a născut Obed, lui Obed i s-a născut Iesei.
\par 13 Lui Iesei i s-a născut Eliab, întâiul său născut, apoi al doilea, Aminadab, al treilea, Șama,
\par 14 Al patrulea, Natanael, al cincilea, Radai,
\par 15 Al șaselea, Oțem și al șaptelea, David.
\par 16 Surorile lor au fost Țeruia și Abigail. Fiii Țeruiei au fost trei: Abișai, Ioab și Asael.
\par 17 Abigail a născut pe Amasa; iar tatăl lui Amasa este Ieter Ismaelitul.
\par 18 Caleb, fiul lui Hețron, a avut de la Azuba, femeia sa, și de la Ieriot următorii copii: Ieșer, Șobab și Ardon.
\par 19 Murind însă Azuba, Caleb și-a luat de femeie pe Efrata și aceasta i-a născut pe Hur.
\par 20 Lui Hur i s-a născut Urie; lui Urie i s-a născut Bețaleel.
\par 21 După aceea Hețron a intrat la fata lui Machir, tatăl lui Galaad; și a luat-o, fiind de șaizeci de ani și ea i-a născut fiu pe Segub.
\par 22 Lui Segub i s-a născut Iair și avea el atunci douăzeci și trei de cetăți în pământul Galaadului.
\par 23 Dar Gheșurenii și Sirienii le-au luat sălașurile lui Iair cu Chenatul și cetățile care țineau de el, în număr de șaizeci. Toate aceste cetăți erau ale fiilor lui Machir, tatăl lui Galaad.
\par 24 După ce a murit Hețron, Caleb a intrat la Efrata, femeia lui Hețron, tatăl său, care a născut pe Așur, tatăl lui Tecoa.
\par 25 Fiii lui Ierahmeel, întâiul născut al lui Hețron, sunt: întâiul născut Ram, după el Vuna, Oren, Oțem și Ahia.
\par 26 Ierahmeel a mai avut și altă femeie, cu numele Atara; aceasta este mama lui Onan.
\par 27 Fiii lui Ram, întâiul născut al lui Ierahmeel, sunt: Maaț, Iamin și Echer.
\par 28 Fiii lui Onan au fost: Șamai și Iada. Fiii lui Șamai au fost: Nadab și Abișur.
\par 29 Numele femeii lui Abișur era Abihail și aceasta i-a născut pe Ahban și pe Molid.
\par 30 Fiii lui Nadab au fost: Seled și Efraim. Dar Seled a murit fără copii.
\par 31 Fiul lui Efraim a fost Ișei, iar fiul lui Ișei a fost Șeșan; iar fiul lui Șeșan a fost Ahlai.
\par 32 Fiii lui Iada, fratele lui Șamai, au fost Ieter și Ionatan. Ieter a murit fără copii.
\par 33 Fiii lui Ionatan au fost: Pelet și Zaza. Aceștia sunt fiii lui Ierahmeel.
\par 34 Șeșan n-a avut fii, ci numai fiice. Șeșan avea un rob egiptean, cu numele Iarha.
\par 35 Șeșan a dat pe o fată a sa lui Iarha, robul său, de femeie și ea a născut pe Atai.
\par 36 Atai a avut de fiu pe Natan, iar lui Natan i s-a născut Zabad.
\par 37 Lui Zabad i s-a născut Eflal, iar lui Eflal i s-a născut Obed.
\par 38 Lui Obed i s-a născut Iehu, iar lui Iehu i s-a născut Azaria.
\par 39 Lui Azaria i s-a născut Heleț, iar lui Heleț i s-a născut Eleasa.
\par 40 Lui Eleasa i s-a născut Sismai, iar lui Sismai i s-a născut Șalum.
\par 41 Șalum a avut de fiu pe Iecamia, iar Iecamia pe Elișama.
\par 42 Fiul lui Caleb, fratele lui Ierahmeel, era Meșa, întâiul său născut, tatăl lui Zif. Acesta a avut ca fiu pe Mareșa, tatăl lui Hebron.
\par 43 Fiii lui Hebron sunt: Core, Tapuah, Rechem și Șema.
\par 44 Lui Șema i s-a născut Raham tatăl lui Iorchean, iar lui Rechem i s-a născut Șamai.
\par 45 Fiul lui Șamai a fost Maon, iar Maon este tatăl lui Bet-Țur.
\par 46 Și Efa, concubina lui Caleb, a născut pe Haran, Moța și Gazez; iar Haran a fost tatăl lui Gazez.
\par 47 Fiii lui Iahdai sunt: Reghem, Iotan, Gheșan, Pelet, Efa și Șaaf.
\par 48 Concubina lui Caleb, Maaca, a născut pe Șeber și pe Tirhana;
\par 49 Tot ea a născut pe Șaaf, tatăl Madmanei, pe Șeva, tatăl Macbenei și tatăl Ghibeii. Fiica lui Caleb este Acsa.
\par 50 Aceștia au fost fiii lui Caleb. Fiul lui Hur, întâiul născut al Efratei a fost Șobal, tatăl lui Chiriat-Iearim;
\par 51 Salma, tatăl lui Betleem; Haref, tatăl lui Betgader.
\par 52 Șobal, tatăl Chiriat-Iearimului a avut fii pe Haroe, Hați și Hamenuhot.
\par 53 Familiile Chiriat-Iearimului sunt: Itrienii, Putienii, Șumatienii și Mișraenii. Din acestea se trag Țoreenii și Eștauleenii.
\par 54 Fiii lui Salma sunt: Betleem, Netofatiții, Atrot-Bet-Ioab, jumătate din Manahteni, Țoareni,
\par 55 Familiile Soferiților, care trăiau în Iabeț, Tiratiții, Șimatiții, Sucatiții. Aceștia sunt Chineenii, care se trag din Hamat, tatăl casei lui Recab.

\chapter{3}

\par 1 Fiii lui David, care i s-au născut în Hebron, au fost: întâiul născut Amnon din Ahinoama Izreeliteanca; al doilea, Daniel, din Abigail Carmeliteanca.
\par 2 Al treilea, Abesalom, fiul Maacăi, fata lui Talmai, regele din Gheșur; al patrulea, Adonia, fiul Haghitei;
\par 3 Al cincilea, Șefatia din Abitala; al șaselea, Itrean din Egla, femeia sa.
\par 4 Acești șase i s-au născut în Hebron. În Hebron David a domnit șapte ani și șase luni, iar în Ierusalim a domnit treizeci și trei de ani.
\par 5 Iată și cei ce i s-au născut în Ierusalim: Șimea, Șobab, Natan și Solomon, patru, din Batșeba, fiica lui Amiel.
\par 6 Ibhar, Elișama, Elifelet,
\par 7 Nogah, Nefeg, Iafia,
\par 8 Elișama, Eliada și Elifelet; în total nouă.
\par 9 Aceștia sunt toți fiii lui David, afară de cei de la țiitoare. Iar sora lor era Tamara.
\par 10 Fiul lui Solomon este Roboam; fiul acestuia este Abia, iar al acestuia, Asa, iar al lui Asa este Iosafat.
\par 11 Fiul acestuia este Ioram, al acestuia este Ahazia și al acestuia este Ioaș.
\par 12 Fiul lui este Amasia, al acestuia este Azaria, iar al acestuia este Ioatam.
\par 13 Fiul acestuia este Ahaz, al acestuia este Iezechia, iar al acestuia este Manase;
\par 14 Fiul acestuia este Amon, iar al acestuia este Iosia.
\par 15 Fiii lui Iosia au fost: întâiul născut Iohanan, al doilea Ioiachim, al treilea Sedechia și al patrulea Șalum.
\par 16 Fiii lui Ioiachim au fost: Iehonia, fiul lui; Sedechia, fiul lui.
\par 17 Fiii lui Iehonia, cel dus în robie, au fost: Salatiel,
\par 18 Malchiram, Pedaia, Șenațar, Iecamia, Hoșama și Nedabia.
\par 19 Iar fiii lui Pedaia au fost: Zorobabel și Șimei. Iar fiii lui Zorobabel au fost: Meșulam și Hanania, și sora lor Șelomit.
\par 20 Fiii lui Meșulam: Hașuba, Ohel, Berechia, Hasadia și Iușab-Hesed.
\par 21 Fiii lui Hanania au fost Pelatia și Isaia; fiul acestuia a fost Refaia, al acestuia a fost Arnan, al acestuia a fost Obadia, iar al acestuia Șecania.
\par 22 Fiii lui Șecania au fost șase: Șemaia, Hatuș, Igheal, Bariah, Nearia și Șafat.
\par 23 Fiii lui Nearia au fost trei: Elioenai, Iezechia și Azricam.
\par 24 Fiii lui Elioenai au fost șapte: Hodavia, Eliașib, Pelaia, Acub, Iohanan, Delaia și Anani.

\chapter{4}

\par 1 Fiii lui Iuda au fost: Fares, Hețron, Carmi, Hur și Șobal.
\par 2 Reaia, fiul lui Șobal, a avut fiu pe Iahat; lui Iahat i s-a născut Ahumai și Lahad. Din el se trag familiile Țoreenilor.
\par 3 Fiii lui Etam sunt: Izreel, Ișma și Idbaș, și sora lor cu numele Hațlelponi.
\par 4 Panuel, tatăl lui Ghedor și Ezer, tatăl lui Hușa sunt fiii lui Hur, întâiul născut din Efrata și tatăl lui Betleem.
\par 5 Așhur, tatăl lui Tecoa, a avut două femei: pe Helea și Naara.
\par 6 Naara i-a născut pe Ahuzam, Hefer, Temni și Ahaștari. Aceștia sunt fiii Naarei.
\par 7 Iar fiii Helei sunt: Țeret, Țohar Etna și Coț.
\par 8 Lui Coț i s-au născut: Anub, Țobeba, Iaheț și familiile lui Aharhel, fiul lui Harum.
\par 9 Iabeț a fost mai însemnat decât frații săi. Mama lui i-a dat numele de Iabeț, zicând: "Cu durere l-am născut".
\par 10 Și a strigat Iabeț către Dumnezeul lui Israel și a zis: "O, de m-ai binecuvânta Tu cu binecuvântare, de ai lărgi hotarele mele și de ar fi mâna Ta cu mine, păzindu-mă de rele, ca să nu fiu omorât!..." Atunci Dumnezeu i-a trimis ceea ce a dorit el.
\par 11 Și lui Chelub, fratele lui Șuha, i s-a născut Mehir. Acesta e tatăl lui Eșton.
\par 12 Lui Eșton i s-au născut Bet-Rafa, Paseah și Techina, tatăl cetății Nahaș; aceștia sunt locuitorii din Recab.
\par 13 Fiii lui Chenaz sunt Otniel și Seraia. Fiii lui Otniel au fost Hatat și Meonotai.
\par 14 Lui Meonotai i s-a născut Ofra. Lui Seraia i s-a născut Ioab, strămoșul lui Gheharașim, numiți așa pentru că ei erau dulgheri.
\par 15 Fiii lui Caleb, fiul lui Iefoni, au fost: Ir, Ela și Naam. Fiul lui Ela a fost Chenaz.
\par 16 Fiii lui Iehaleleel au fost: Zif, Zifa, Tiria și Asareel.
\par 17 Fiii lui Ezra sunt: Ieter, Mered, Efer și Ialon; iar lui Ieter i s-au născut Miriam, Șamai și Ișbah, tatăl lui Eștemoa.
\par 18 Femeia acestuia, Iehudia, a născut pe Iered, tatăl lui Ghedor, pe Heber, tatăl lui Soco, și pe Iecutiel, tatăl lui Zanoah. Aceștia sunt fiii Bitiei, fata lui Faraon, pe care a luat-o Mered.
\par 19 Fiii femeii acestuia, Hodia, sora lui Naham, tatăl Cheilei, sunt: Garmi și Eștemoa Maacateanul.
\par 20 Fiii lui Simeon sunt: Amnon, Rina, Benhanan și Tilon. Fiii lui Iși sunt Zohet și Benzohet.
\par 21 Fiii lui Șela, fiul lui Iuda, sunt: Er, tatăl lui Leca, Laeda, tatăl lui Mareșa, și familiile din casa lui Așbeia, care lucrau visonul,
\par 22 Iochim și locuitorii din Cozeba; Ioaș și Saraf, care au stăpânit asupra Moabului și Iașubi-Lehem. Dar acestea sunt întâmplări mai vechi.
\par 23 Aceștia erau olari și trăiau la grădini și la livezi și prin cetăți; ei trăiau acolo la rege ca să-i lucreze lui.
\par 24 Fiii lui Simeon au fost: Nemuel, Iamin, Iarib, Zerah și Saul.
\par 25 Fiul lui Saul a fost Șalum, fiul acestuia a fost Mibsam, iar al acestuia a fost Mișma.
\par 26 Fiii lui Mișma au fost: Hamuel, fiul lui; fiul acestuia a fost Zacur, iar al acestuia a fost Șimei.
\par 27 Șimei a avut șaisprezece fii și șase fete, iar frații lui au avut puțini copii și tot neamul lor n-a fost așa de numeros ca neamul fiilor lui Iuda.
\par 28 Ei trăiau în Beer-Șeba, Molada și Hațar-Șual,
\par 29 În Bilha, Ețem, Tolad,
\par 30 Betuel, Horma, Țiclag,
\par 31 În Bet-Marcabot, Hațar-Susim, Bet-Birei și Șaaraim. Iată cetățile lor dinainte de domnia lui David cu satele lor.
\par 32 Și mai aveau: Etam, Ain, Rimon, Tochen și Așan, cinci cetăți,
\par 33 Cu toate satele lor, care se aflau împrejurul acestor cetăți până la Baal. Iată locurile lor de locuință și spița neamului lor:
\par 34 Meșobab, Iamlec și Ioșa, fiul lui Amasia
\par 35 Ioil și Iehu, fiul lui Ioșibia, fiul lui Seraia, fiul lui Asiel;
\par 36 Elioenai, Iaacoba, Ieșohaia, Asaia, Adiel, Ieșimiel, Benaia,
\par 37 Ziza, fiul lui Șifei, fiul lui Alon, fiul lui Iedaia, fiul lui Șimri, fiul lui Șemaia.
\par 38 Acești numiți mai sus au fost căpetenii neamurilor lor, iar casa tatălui lor s-a împărțit în multe ramuri.
\par 39 Ei s-au întins până în partea Gherarei și până în partea de răsărit a văii Gai, ca să găsească pășuni pentru turmele lor;
\par 40 Și au găsit pășuni grase și bune și pământ larg, liniștit și lipsit de primejdii, pentru că înainte de ei au trăit acolo numai puțini Hamiți.
\par 41 și au venit aceștia, care sunt scriși pe nume, în zilele lui Iezechia, regele Iudei, și au bătut pe nomazi și pe cei așezați, care se aflau acolo și i-au nimicit pentru totdeauna și s-au așezat în locul lor, căci acolo se aflau pășuni pentru turmele lor.
\par 42 Dar din ei, din fiii lui Simeon, s-au dus către muntele Seir cinci sute de oameni, în frunte cu Pelatia, Nearia, Refaia și Uziel, fiii lui Ișei,
\par 43 Și au bătut rămășița de Amaleciți, ce se mai găsea acolo, și trăiesc acolo până în ziua de astăzi.

\chapter{5}

\par 1 Fiii lui Ruben, întâiul născut al lui Israel, căci el a fost născut întâi, dar pentru că a întinat el patul tatălui său, întâietatea lui a fost dată fiilor lui Iosif, fiul lui Israel, ca să nu se mai înscrie ei (fiii lui Ruben) ca întâi născuți;
\par 2 Căci Iuda era cel mai puternic dintre frații săi și povățuitorul e din el, dar întâietatea a trecut la Iosif.
\par 3 Fiii lui Ruben, întâiul născut al lui Israel, au fost: Enoh, Palu, Hețron și Carmi.
\par 4 Fiii lui Ioil au fost: Șemaia, fiul lui, fiul acestuia a fost Gog, iar al acestuia a fost Șimei;
\par 5 Fiul acestuia a fost Mihea, al acestuia a fost Reaia, iar al acestuia a fost Baal;
\par 6 Fiul acestuia a fost Beera, pe care l-a dus în robie Tiglatfalasar, regele Asiriei. El era căpetenia Rubeniților.
\par 7 Și frații lui, după familiile lor, după spira neamului lor, au fost: cel mai însemnat Ioil, apoi Zaharia,
\par 8 Și Bela, fiul lui Azaz, fiul lui Șema, fiul lui Ioil. El locuia în Aroer până la Nebo și Baal-Meon.
\par 9 Iar spre răsărit a locuit el până la marginea pustiului, care pleacă de la râul Eufrat, pentru că turmele lui erau foarte multe în ținutul Galaadului.
\par 10 În zilele lui Saul au purtat ei război cu Agarenii, care au căzut în mâinile lor, și au locuit în corturi, în toată latura de răsărit a Galaadului.
\par 11 Fiii lui Gad trăiau în fața lor, în țara Vasanului, până la Salca.
\par 12 În Vasan, cel mai de seamă era Ioil. Șafam era al doilea, apoi venea Iaenai și Șafat.
\par 13 Frații lor cu familiile lor erau în număr de șapte: Micael, Meșulam, Șeba, Iorai, Iacan, Zia și Eber.
\par 14 Iată fiii lui Abihail, fiul lui Huri, fiul lui Iaroah, fiul lui Galaad, fiul lui Micael, fiul lui Ieșișai, fiul lui Iahdo, fiul lui Buz.
\par 15 Ahi, fiul lui Abdiel, fiul lui Guni, era capul neamului său.
\par 16 Ei trăiau în Galaad, în Vasan și în cetățile care țineau de el în toate împrejurimile Șirionului, până la capătul lor.
\par 17 Ei cu toții au fost numărați în zilele lui Ioatam, regele Iudei, și în zilele lui Ieroboam, regele lui Israel.
\par 18 Urmașii lui Ruben, ai lui Gad și o jumătate din seminția lui Manase aveau oameni războinici, bărbați care purtau scut și sabie, care trăgeau cu arcul și deprinși la luptă, patruzeci și patru de mii șapte sute șaizeci care ieșeau la război.
\par 19 Și s-au luptat ei cu Agarenii, cu Ietur, cu Nafis și cu Nodab.
\par 20 Dar li s-a dat ajutor contra acelora și au fost dați în mâna lor Agarenii și toate ale lor, pentru că ei în vremea luptei au strigat către Dumnezeu și El i-a auzit, pentru că ei nădăjduiau în El.
\par 21 Atunci au luat ei turmele acelora: cincizeci de mii de cămile, două sute cincizeci de mii de oi și capre, două mii de asini și o sută de mii de oameni,
\par 22 Pentru că mulți au căzut uciși, căci lupta aceasta a fost de la Dumnezeu. Și au locuit ei în locul acelora până la ducerea în robie.
\par 23 Urmașii jumătății din tribul lui Manase au trăit în acel pământ de la Vasan până la Baal-Ermon și Senir și până la muntele Hermon, și erau mulți la număr.
\par 24 Iată capii de familie cei mai însemnați ai lor: Efer, Ișei, Eliel, Azriel, Ieremia, Hodavia și Iahdiel, bărbați puternici, bărbați vestiți, căpetenii ale neamurilor lor.
\par 25 Dar când ei au greșit împotriva Dumnezeului părinților lor și au început să se desfrâneze după dumnezeii popoarelor pământului aceluia, pe care le stârpise Dumnezeu de la fața lor,
\par 26 Atunci Dumnezeul lui Israel a întărâtat duhul lui Ful, regele Asiriei, adică al lui Tiglatfalasar, regele Asiriei și acesta a strămutat pe Rubeniți și pe Gadiți și jumătate din tribul lui Manase și i-a dus în Halach, Habor, Hara și la râul Gozan, unde sunt până astăzi.

\chapter{6}

\par 1 Fiii lui Levi sunt: Gherșom, Cahat și Merari.
\par 2 Fiii lui Cahat sunt: Amram, Ițhar, Hebron și Uziel.
\par 3 Copiii lui Amram sunt: Aaron, Moise și Mariam. Fiii lui Aaron sunt: Nadab, Abiud, Eleazar și Itamar.
\par 4 Lui Eleazar i s-a născut Finees, lui Finees i s-a născut Abișua;
\par 5 Lui Abișua i s-a născut Buchi, iar lui Buchi i s-a născut Uzi;
\par 6 Lui Uzi i s-a născut Zerahia, iar lui Zerahia i s-a născut Meraiot.
\par 7 Lui Meraiot i s-a născut Amaria, iar lui Amaria i s-a născut Ahitub;
\par 8 Lui Ahitub i s-a născut Țadoc, iar lui Țadoc i s-a născut Ahimaaț;
\par 9 Lui Ahimaaț i s-a născut Azaria, iar lui Azaria i s-a născut Iohanan;
\par 10 Lui Iohanan. i s-a născut Azaria; acesta e acela care a fost preot la templul zidit de Solomon în Ierusalim.
\par 11 Lui Azaria i s-a născut Amaria, iar lui Amaria i s-a născut Ahitub;
\par 12 Lui Ahitub i s-a născut Țadoc, iar lui Țadoc i s-a născut Șalum;
\par 13 Lui Șalum i s-a născut Hilchia, iar lui Hilchia i s-a născut Azaria;
\par 14 Lui Azaria i s-a născut Seraia, iar lui Seraia i s-a născut Iehosadac.
\par 15 Iehosadac a mers în robie când Domnul a strămutat pe cei din Iuda și pe cei din Ierusalim prin mâna lui Nabucodonosor.
\par 16 Astfel fiii lui Levi au fost: Gherșom, Cahat și Merari.
\par 17 Iată numele fiilor lui Gherșom: Libni și Șimei.
\par 18 Fiii lui Cahat au fost: Amram, Ițhar, Hebron și Uziel.
\par 19 Fiii lui Merari au fost: Mahli și Muși. Iată urmașii lui Levi după neamurile lor.
\par 20 Gherșom a avut pe Libni, fiul lui; pe Iahat, fiul lui, și pe Zima, fiul lui;
\par 21 Pe Ioah, fiul lui; pe Ido, fiul lui; pe Zerah, fiul lui, și pe Ieatrai, fiul lui.
\par 22 Fiii lui Cahat au fost: Aminadab, fiul lui; Core, fiul lui, și Asir, fiul lui;
\par 23 Elcana, fiul lui; Ebiasaf, fiul lui, și Asir, fiul lui;
\par 24 Tahat, fiul lui, Uriel, fiul lui; Uzia, fiul lui, și Saul, fiul lui.
\par 25 Fiii lui Elcana sunt: Amasai și Ahimot,
\par 26 Elcana, fiul lui; Țofai fiul lui, și Nahat, fiul lui,
\par 27 Eliab, fiul lui; Ieroham, fiul lui; Elcana, fiul lui; Samuel, fiul lui.
\par 28 Fiii lui Samuel au fost: întâiul născut Ioil, al doilea, Abia.
\par 29 Fiii lui Merari au fost: Mahli, Libni, fiul lui; Șimei, fiul lui; Uza, fiul lui;
\par 30 Șimea, fiul lui; Aghia, fiul lui, și Asaia, fiul lui.
\par 31 Iată cei pe care David i-a pus căpetenii peste cântăreți în casa Domnului, în timpul când a așezat în ea chivotul legii,
\par 32 Care au servit de cântăreți înaintea cortului adunării, până când Solomon a zidit templul Domnului în Ierusalim, și care fuseseră rânduiți la slujba lor după rânduiala lor;
\par 33 Iată pe cei care au fost rânduiți cu fiii lor: din fiii lui Cahat: Heman cântărețul, fiul lui Ioil, fiul lui Samuel,
\par 34 Fiul lui Elcana, fiul lui Ieroham, fiul lui Eliel, fiul lui Toah,
\par 35 Fiul lui Țuf, fiul lui Elcana, fiul lui Mahat, fiul lui Amasai,
\par 36 Fiul lui Elcana, fiul lui Ioil, fiul lui Azaria, fiul lui Țefania,
\par 37 Fiul lui Tahat, fiul lui Asir, fiul lui Abiasaf, fiul lui Core,
\par 38 Fiul lui Ițhar, fiul lui Cahat, fiul lui Levi, fiul lui Israel.
\par 39 Și fratele său Asaf, care stătea în partea dreaptă a lui, adică Asaf, fiul lui Berechia, fiul lui Șimea,
\par 40 Fiul lui Micael, fiul lui Baaseia, fiul lui Malchia,
\par 41 Fiul lui Etni, fiul lui Zerah, fiul lui Adaia,
\par 42 Fiul lui Etan, fiul lui Zima, fiul lui Șimei,
\par 43 Fiul lui Iahat, fiul lui Gherșom, fiul lui Levi.
\par 44 Iar din fiii lui Merari, frații lor, au fost în partea stângă: Etan, fiul lui Chiși, fiul lui Abdi, fiul lui Maluc,
\par 45 Fiul lui Hașabia, fiul lui Amasia, fiul lui Hilchia,
\par 46 Fiul lui Amți, fiul lui Bani, fiul lui Șemer,
\par 47 Fiul lui Mahli, fiul lui Muși, fiul lui Merari, fiul lui Levi.
\par 48 Frații lor leviți erau rânduiți la tot felul de slujbe, la casa Domnului.
\par 49 Iar Aaron și fiii lor ardeau pe jertfelnic arderi de tot și tămâie pe altarul tămâierii, săvârșind toate slujbele sfinte în Sfânta Sfintelor și pentru ispășirea lui Israel, în toate, cum poruncise robul lui Dumnezeu Moise.
\par 50 Iată fiii lui Aaron: Eleazar, fiul lui; Finees, fiul lui; Abișua, fiul lui;
\par 51 Buchi, fiul lui; Uzi, fiul lui; Zerahia, fiul lui;
\par 52 Meraiot, fiul lui; Amaria, fiul lui; Ahitub, fiul lui;
\par 53 Țadoc, fiul lui; Ahimaaț, fiul lui.
\par 54 Iată locuințele lor după satele lor în hotarele lor: fiilor lui Aaron din familia lui Cahat, după cum le-a căzut sorțul,
\par 55 Li s-au dat Hebronul, în pământul lui Iuda și împrejurimile lui,
\par 56 Iar țarinile acestei cetăți și satele ei s-au dat lui Caleb, fiul lui Iefonie.
\par 57 Fiilor lui Aaron li s-au dat de asemenea orașele de scăpare: Hebron și Libna și împrejurimile lor, Iatir și Eștemoa și ținuturile lor,
\par 58 Hilenul (Holonul) și pășunile lui, Debirul și pășunile lui;
\par 59 Așanul (Ainul) și împrejurimile lui, Betșemeșul și împrejurimile lui;
\par 60 Iar de la tribul lui Veniamin: Gheba și pășunile ei, Alemetul (Almonul) și împrejurimile lui, Anatotul și ținuturile lui; cetățile familiilor lor erau de toate treisprezece cetăți.
\par 61 Celorlalți fii ai lui Cahat, din familiile acestui trib, li s-au dat, după sorți, zece cetăți din hotarele jumătății tribului lui Manase.
\par 62 Fiilor lui Gherșom, după familiile lor, li s-au dat treisprezece cetăți din tribul lui Isahar, din tribul lui Așer, din tribul lui Neftali și din tribul lui Manase, în Vasan.
\par 63 Fiilor lui Merari, după familiile lor, li s-au dat prin sorți douăsprezece cetăți din tribul lui Ruben, din tribul lui Gad și din tribul lui Zabulon.
\par 64 Așa au dat fiii lui Israel Leviților cetăți cu împrejurimile lor.
\par 65 Li s-au dat prin sorți din tribul fiilor lui Iuda, din tribul fiilor lui Simeon și din tribul fiilor lui Veniamin acele cetăți pe care ei le-au numit pe nume.
\par 66 Iar unora din familiile fiilor lui Cahat li s-au dat cetăți din tribul lui Efraim.
\par 67 Li s-au dat cetățile de scăpare: Sichemul și împrejurimile lui, pe muntele Efraim și Ghezerul cu împrejurimile lui;
\par 68 Iocmeamul (Chibțoimul) cu împrejurimile lui și Bethoronul cu împrejurimile lui;
\par 69 Aialonul cu împrejurimile lui și Gat-Rimonul cu împrejurimile lui.
\par 70 Din jumătatea tribului lui Manase li s-au dat: Anerul cu împrejurimile lui, Bileanul cu împrejurimile lui. Acestea sunt locuințele pentru ceilalți fii ai lui Cahat.
\par 71 Fiilor lui Gherșom din familiile din jumătatea tribului lui Manase li s-au dat Golanul în Vasan cu împrejurimile lui și Aștarotul cu împrejurimile lui.
\par 72 Din tribul lui Isahar li s-au dat Chedeșul (Chișionul) cu împrejurimile lui, Dobratul cu împrejurimile lui,
\par 73 Ramotul cu împrejurimile lui și Anemul cu împrejurimile lui.
\par 74 Din tribul lui Așer li s-au dat: Mașalul cu împrejurimile lui și Abdonul cu împrejurimile lui;
\par 75 Hucocul (Helcotul) cu ținutul lui și Rehobul cu ținutul lui;
\par 76 Din tribul lui Neftali li s-au dat Chedeșul în Galileea, cu ținutul lui, Hamonul cu ținutul lui și Chiriataimul cu ținutul lui.
\par 77 Iar celorlalți fii ai lui Merari li s-au dat: din tribul lui Zabulon, Rimonul cu ținutul lui și Taborul cu ținutul lui,
\par 78 Iar dincolo de Iordan, în fața Ierihonului, la răsărit de Iordan, li s-a dat în tribul lui Ruben: Bețerul, în pustiu, cu ținutul lui, și Iahța cu ținutul ei,
\par 79 Chedemotul cu împrejurimile lui și Mefaatul cu ținutul lui.
\par 80 Din tribul lui Gad li s-au dat: Ramotul în Galaad cu ținutul lui și Mahanaimul cu ținutul lui;
\par 81 Heșbonul și Iazerul cu ținuturile lor.

\chapter{7}

\par 1 Fiii lui Isahar au fost patru: Tola, Pua, Iașub și Șimron.
\par 2 Fiii lui Tola au fost: Uzi, Refaia, Ieriil, Iahmai, Ibsam și, Samuel; aceștia sunt cei mai de seamă în neamul lui Tola, oameni războinici în neamul lor; numărul lor în zilele lui David era douăzeci și două de mii și șase sute.
\par 3 Fiul lui Uzi a fost Izrahia; iar fiii lui Izrahia au fost: Micael, Obadia, Ioil și Ișia, de toți cinci. Toți aceștia sunt căpetenii.
\par 4 Ei, după familiile lor și după neamurile lor, aveau oștire de treizeci și șase de mii de oameni, pentru că ei au avut multe femei și mulți copii.
\par 5 Iar frații lor, în toate neamurile lui Isahar, aveau oameni de luptă optzeci și șapte de mii, înscriși în tăblițele cele cu spița neamului.
\par 6 Veniamin a avut trei: pe Bela, Becher și Iediael (Așbel).
\par 7 Fiii lui Bela au fost cinci: Ețbon, Uzi, Uziel, Ierimot și Iri, toți căpetenii de familii, oameni războinici. În tăblițele cu spița neamului sunt înscriși douăzeci și două de mii treizeci și patru.
\par 8 Fiii lui Becher au fost: Zemira, Ioaș, Eliezer, Elioenai, Omri, Ieremot, Abia, Anatot și Alemet; toți aceștia sunt fiii lui Becher.
\par 9 În tăblițele cu spița neamului sunt înscriși din aceștia, după familiile și după neamurile lor, oameni războinici douăzeci de mii și două sute.
\par 10 Fiul lui Iediael (Așbel) a fost Bilhan. Fiii lui Bilhan au fost: Ieuș, Veniamin, Ehud, Chenaana, Zetan, Tarșiș și Ahișahar.
\par 11 Toți acești fii ai lui Iediael (Așbel), au fost capi de familie, oameni războinici; șaptesprezece mii și două sute erau în stare de a ieși la război.
\par 12 Șupim și Hupim erau fiii lui Ir, iar Hușim era fiul lui Aher.
\par 13 Fiii lui Neftali au fost: Iahțiel, Guni, Iețer și Șalum (Silem), copiii Bilhei.
\par 14 Fiii lui Manase au fost: Asriel, pe care l-a născut concubina sa arameană; tot aceasta a născut pe Machir, tatăl lui Galaad.
\par 15 Machir și-a luat de femeie pe sora lui Hupim și a lui Șupim, al cărei nume era Maaca. Numele fiului al doilea a fost Salfaad. Salfaad a avut numai fete.
\par 16 Maaca, femeia lui Machir, a născut un fiu și i-a pus numele Pereș, iar numele fratelui lui era Șereș. Fiii acestuia au fost Ulam și Rechem.
\par 17 Fiul lui Ulam a fost Bedan. Aceștia sunt fiii lui Galaad, fiul lui Machir, fiul lui Manase.
\par 18 Sora sa, Molechet, a născut pe Ișhod, pe Abiezer și pe Mahla.
\par 19 Fiii lui Șemida au fost: Ahian, Șechem, Lichi și Aniam.
\par 20 Fiii lui Efraim au fost: Șutelah, Bered, fiul lui; Tahat, fiul lui; Eleadah, fiul lui și Tahat, fiul lui;
\par 21 Zabad, fiul lui; Șutelah, fiul lui; Ezer și Elead. Pe aceștia i-au ucis locuitorii din Gat, băștinașii jării aceleia, pentru că ei se duseseră să le apuce turmele lor.
\par 22 După ei a plâns Efraim, tatăl lor, zile multe și au venit frații lui să-l mângâie.
\par 23 Apoi a intrat el la femeia sa și ea a zămislit și a născut un fiu și el i-a pus numele Beria, pentru că nenorocirea se atinsese de casa lui.
\par 24 Și a avut el și o fată: Șeera. Aceasta a zidit Bethoronul de jos și de sus și Uzen-Șeera.
\par 25 Refah, fiul său, și Reșef, fiul său; Telah, fiul său, și Tahan, fiul său.
\par 26 Ladan, fiul său; Amiud, fiul său, și Elișama, fiul său.
\par 27 Non, fiul său; Iosua, fiul său.
\par 28 Moșiile lor și locurile lor de locuit au fost: Betelul și cetățile care țineau de el, spre răsărit Naaranul, spre apus Ghezerul și cetățile care țineau de el; Sichemul și cetățile care țineau de el până la Gaza și cetățile ce țineau de aceasta.
\par 29 Iar din partea fiilor lui Manase: Bet-Șeanul și cetățile ce țineau de el, Taanacul și cetățile ce țineau de el, Meghidonul și cetățile ce țineau de el, Dorul și cetățile ce țineau de el. în ele locuiau fiii lui Iosif, fiul lui Israel.
\par 30 Fiii lui Așer erau: Imna, Ișva, Ișvi și Beria și sora lor Serah.
\par 31 Fiii lui Beria au fost: Heber și Malchiel. Acesta e tatăl lui Birzait.
\par 32 Heber a avut fii pe Iaflet, Șemer și Hotam și pe sora lor Șua.
\par 33 Fiii lui Iaflet au fost: Pasac, Bimhal și Așvat. Aceștia sunt fiii lui Iaflet.
\par 34 Fiii lui Șemer au fost: Ahi, Rohga, Huba și Aram.
\par 35 Fiii lui Helem, fratele lui, au fost: Țofah, Imna, Șeleș și Amal.
\par 36 Fiii lui Țofah au fost: Suah, Harnefer, Șual, Beri, Imra,
\par 37 Bețer, Hod, Șama, Șilșa, Itran și Beera.
\par 38 Fiii lui Ieter au fost: Iefune, Pispa și Ara.
\par 39 Fiii născuți din Ula au fost: Arah, Haniel și Riția.
\par 40 Toți aceștia sunt fiii lui Așer, capi de familie, oameni aleși, războinici, căpetenii de mâna întâi. În tăblițele lor cu spița neamului sunt înscriși în oștire pentru război un număr de douăzeci și șase de mii de oameni.

\chapter{8}

\par 1 Lui Veniamin i s-a născut Bela, întâiul său născut; al doilea, Așbel, al treilea, Ahrah,
\par 2 Al patrulea, Noha și al cincilea, Rafa.
\par 3 Fiii lui Bela au fost: Adar, Ghera, Abiud,
\par 4 Abișua, Naaman, Ahoah,
\par 5 Ghera, Șefufan și Huram.
\par 6 Iată fiii lui Ehud, care au fost capi de familii, care au trăit în Gheba și au fost strămutați în Manahat:
\par 7 Naaman, Ahia și Ghera, care i-a strămutat, a născut pe Uza și pe Ahiud.
\par 8 Lui Șaharaim i s-au născut copii în țara Moabiților, după ce a dat drumul femeilor sale, Hușim și Baara.
\par 9 I s-au născut din Hodeșa, femeia sa: Iobab, Țibia, Meșa și Malcam;
\par 10 Ieuț, Șachia și Mirma. Aceștia sunt fiii lui, capi de familie.
\par 11 Din Hușim i s-au născut Abitub și Elpaal.
\par 12 Fiii lui Elpaal au fost: Eber, Mișeam și Șemed, care au zidit Ono și Lodul și cetățile ce țineau de el,
\par 13 Precum și Beria și Sema. - Aceștia au fost capii familiilor locuitorilor Aialonului; ei au alungat pe locuitorii din Gat. -
\par 14 Ahio, Șașac, și Iremot;
\par 15 Zebadia, Arad și Eder;
\par 16 Micael, Ișpa și Ioha, fiii lui Beria.
\par 17 Zecadia, Meșulam, Hizchi și Heber,
\par 18 Ieșmere, Izliah și Iobab, fiii lui Elpaal.
\par 19 Iachim, Zicri și Zabdi,
\par 20 Elienai, Țiltai și Eliel,
\par 21 Adaia, Beraia și Șimrat, fiii lui Șimei.
\par 22 Ișpan, Eber și Eliel,
\par 23 Abdon, Zicri și Hanan,
\par 24 Hanania, Elam și Antotia,
\par 25 Ifdia și Fanuil, fiii lui Șișac.
\par 26 Șamșerai, Șeharia și Atalia,
\par 27 Iaareșia, Elia și Zicri, fiii lui Ieroham.
\par 28 Acestea sunt căpeteniile familiilor mai de seamă ale neamului lor. Ei au locuit în Ierusalim.
\par 29 În Ghibeon a trăit Ieguel, tatăl Ghibeoniților. Numele femeii lui a fost Maaca
\par 30 Și fiul lui, întâiul născut, Abdon, după el Țur, Chiș, Baal, Nadab, Ner,
\par 31 Ghedeor, Ahio, Zecher și Miclot;
\par 32 Lui Miclot i s-a născut Șimea. Și au trăit ei împreună cu frații lor în Ierusalim.
\par 33 Lui Ner i s-a născut Chiș; lui Chiș i s-a născut Saul; lui Saul i s-a născut Ionatan, Melchișua, Aminadab și Eșbaal.
\par 34 Fiul lui Ionatan a fost Meribaal; (Mefiboșet); a avut de fiu pe Mica.
\par 35 Fiii lui Mica au fost: Piton, Melec, Tarea și Ahaz.
\par 36 Lui Ahaz i s-a născut Iehoada, lui Iehoada i s-a născut Alemet, Asmavet și Zimri; lui Zimri i s-a născut Moța;
\par 37 Lui Moța i s-a născut Binea; Rafa, fiul lui; Eleasa, fiul lui; Ațel, fiul lui.
\par 38 Ațel a avut șase feciori și iată numele lor: Azricam, Bocru, Ismael, Șearia, Obadia și Hanan. Toți aceștia sunt fiii lui Ațel.
\par 39 Fiii lui Eșec, fratele său, au fost: Ulam, întâiul său născut; al doilea, Ieuș; al treilea Elifelet.
\par 40 Fiii lui Ulam au fost oameni războinici, trăgători din arc, având mulți copii și nepoți până la o sută cincizeci. Toți aceștia sunt din fiii lui Veniamin.

\chapter{9}

\par 1 Așa au fost numărați după neamurile lor toți Israeliții și iată sunt înscriși în cartea regilor lui Israel. Iar Iudeii, pentru fărădelegile lor, au fost duși în Babilon.
\par 2 Cei dintâi locuitori care au trăit în pământurile lor, prin orașele lui Israel, au fost Israeliții, preoții, leviții și cei afierosiți templului.
\par 3 În Ierusalim au trăit unii din fiii lui Iuda, din fiii lui Veniamin și din fiii lui Efraim și ai lui Manase.
\par 4 Utai, fiul lui Amihud, fiul lui Omri, fiul lui Imri, fiul lui Bani, din fiii lui Fares, fiul lui Iuda;
\par 5 Din fiii lui Șiloni: Asaia, întâiul născut, și fiii acestuia.
\par 6 Din fiii lui Zerah: Ieuel și frații lui, șase sute nouăzeci;
\par 7 Din fiii lui Veniamin: Salu, fiul lui Meșulam, fiul lui Hodavia, fiul lui Asenua
\par 8 Și Ibneia, fiul lui Ieroham, și Ela, fiul lui Uzi, fiul lui Micri, și Meșulam, fiul lui Șefatia, fiul lui Reguel, fiul lui Ibneia
\par 9 Și frații lor, după neamurile lor: nouă sute cincizeci și șase. Acești bărbați erau capi de familie în neamul lor.
\par 10 Iar dintre preoți: Iedaia, Ioiarib, Iachin
\par 11 Și Azaria, fiul lui Hilchia, fiul lui Meșulam, fiul lui Țadoc, fiul lui Meraiot, fiul lui Ahitub, căpetenia în casa lui Dumnezeu;
\par 12 Adaia, fiul lui Ieroham, fiul lui Pașhur, fiul lui Malchia; și Maesai, fiul lui Adiel, fiul lui Iahzera, fiul lui Meșulam, fiul lui Meșilemit, fiul lui Imer.
\par 13 Și frații lor, capi în familiile lor: o mie șapte sute șaizeci, bărbați de ispravă la slujbele din casa Domnului.
\par 14 Iar din leviți: Șemaia, fiul lui Hașub, fiul lui Azricam, fiul lui Hașabia; aceștia sunt din fiii lui Merari.
\par 15 Bacbacar, Hereș, Galal și Matania, fiul lui Mica, fiul lui Zicri, fiul lui Asaf,
\par 16 Obadia, fiul lui Șemaia, fiul lui Galal, fiul lui Iedutun; Berechia, fiul lui Asa, fiul lui Elcana, care locuia în satele Netofatiților.
\par 17 Iar din portari: Șalum, Acub, Talmon și Ahiman și frații lor; Șalum era căpetenie.
\par 18 Acești portari fac strajă fiilor leviților până astăzi la porțile regale cele de la răsărit.
\par 19 Șalum, fiul lui Core, fiul lui Ebiasaf, fiul lui Corah, și frații lui cei din neamul lui Corah, după datoria slujbei lor, aveau paza cortului, iar părinții lor păzeau intrarea în tabăra Domnului.
\par 20 Finees, fiul lui Eleazar, fusese înainte căpetenie peste ei și Domnul era cu el.
\par 21 Zaharia, fiul lui Meșelemia, era portar la ușa cortului adunării.
\par 22 Toți cei aleși ca portari la praguri erau două sute doisprezece. Ei erau înscriși în registre după așezările lor. Pe ei îi pusese David și Samuel înainte-văzătorul, pentru credincioșia lor.
\par 23 Și ei și fiii lor țineau strajă la ușile casei Domnului, la casa cortului.
\par 24 În patru laturi se aflau portari: la răsărit, la apus, la miazănoapte și la miazăzi.
\par 25 Iar frații lor trăiau în sălașurile lor, venind la ei din timp în timp, pentru șapte zile.
\par 26 Aceste patru căpetenii de portari leviți erau de încredere și tot ei aveau grija casei Domnului și vistieriei ei.
\par 27 Împrejurul casei lui Dumnezeu ei petreceau și noaptea, căci asupra lor era lăsată paza și trebuia să deschidă în fiecare dimineață ușile.
\par 28 Unii din ei erau puși de pază la vasele de slujbă, așa că ei cu număr le primeau și cu număr le dădeau.
\par 29 Altora din ei le era încredințată cealaltă zestre și toate lucrurile trebuitoare pentru cele sfinte: făina cea mai bună, vinul, untdelemnul și tămâia cea mirositoare.
\par 30 Iar dintre fiii preoților unii pregăteau mir cu aromate.
\par 31 Matatia levitul, care era întâiul născut al lui Șalum, fiul lui Core, era pus să aibă grijă de coptul aluaturilor în tigăi.
\par 32 Unora din frații lor, din fiii lui Cahat, le era încredințată pregătirea pâinilor punerii înainte, ca să le pună în fiecare zi de odihnă.
\par 33 Iar cântăreții, cei mai de seamă din neamul Leviților, erau liberi de ocupații în cămările templului, pentru că ziua și noaptea erau îndatorați să se îndeletnicească cu cântarea.
\par 34 Aceștia erau cei mai de seamă între familiile leviților și locuiau în Ierusalim.
\par 35 În Ghibeon locuiau: tatăl Ghibeoniților, Ieguel, a cărui femeie se numea Maaca;
\par 36 Și fiul lui, întâiul născut, se numea Abdon; după el venea Țur, Chiș, Baal, Ner, Nadab,
\par 37 Ghedor, Ahio, Zaharia și Miclot.
\par 38 Lui Miclot i s-a născut Șimeam. Și ei trăiau lângă frații lor, în Ierusalim, împreună cu frații lor.
\par 39 Lui Ner i s-a născut Chiș; lui Chiș i s-a născut Saul, lui Saul i s-a născut Ionatan, Melchișua, Aminadab și Eșbaal.
\par 40 Fiul lui Ionatan a fost Meribaal; lui Meribaal i s-a născut Mica.
\par 41 Fiii lui Mica au fost: Piton, Melec, Tareia și Ahaz.
\par 42 Lui Ahaz i s-a născut Iara; lui Iara i s-au născut Alemet, Asmavet și Zimri; lui Zimri i s-a născut Moța.
\par 43 Lui Moța i s-a născut Binea; Refaia, fiul lui; Eleasa, fiul lui; Ațel, fiul lui.
\par 44 Lui Ațel i s-au născut șase fii și iată numele lor: Azricam, Bocru, Ismael, Șearia, Obadia și Hanan. Aceștia sunt fiii lui Ațel.

\chapter{10}

\par 1 În vremea aceea Filistenii s-au ridicat cu război asupra lui Israel și Israeliții au fugit de Filisteni și au căzut biruiți pe muntele Ghilboa.
\par 2 Atunci au alergat Filistenii după Saul și după fiii lui și au ucis Filistenii pe Ionatan, pe Aminadab și pe Melchișua, fiii lui Saul.
\par 3 Iar lupta împotriva lui Saul s-a întețit și arcașii au tras asupra lui, așa că el a fost rănit de săgeți.
\par 4 Saul a zis purtătorului său de arme: "Scoate sabia ta și mă străpunge cu ea, ca să nu vină acești netăiați împrejur și să-și bată joc de mine". Dar purtătorul de arme nu s-a hotărât la aceasta, pentru că se speriase foarte tare. Atunci Saul a luat sabia și s-a aruncat în ea.
\par 5 Văzând purtătorul de arme că Saul a murit, s-a aruncat și el în sabia sa.
\par 6 Așa a murit Saul cu cei trei fii ai lui și toată casa a murit împreună cu el.
\par 7 Când au văzut Israeliții, care erau în vale, că fug toți și că Saul și fiii lui au murit, au lăsat cetățile lor și au fugit în toate părțile, iar Filistenii au venit și s-au așezat în ele.
\par 8 A doua zi au venit Filistenii să ridice pe cei uciși și, găsind pe Saul și pe fiii lui căzuți pe muntele Ghilboa,
\par 9 L-au dezbrăcat și i-au luat capul și armele și au trimis prin țara Filistenilor, să se vestească aceasta înaintea idolilor lor și înaintea poporului.
\par 10 Armele lui le-au pus în templul zeilor lor, iar capul lui l-au spânzurat în templul lui Dagon.
\par 11 Atunci auzind tot Iabeșul Galaadului ce au făcut Filistenii cu Saul,
\par 12 S-au ridicat toți oamenii de luptă, au luat trupul lui Saul și trupurile fiilor lui, le-au dus în Iabeș și au îngropat oasele lor sub un stejar în Iabeș și au postit șapte zile.
\par 13 Așa a murit Saul pentru nelegiuirea sa pe care o făcuse el înaintea Domnului, pentru că n-a păzit cuvântul Domnului și pentru că a întrebat și a cercetat o vrăjitoare
\par 14 Și nu a cercetat pe Domnul. De aceea a și fost el omorât și domnia a fost dată lui David, fiul lui Iesei.

\chapter{11}

\par 1 După aceea s-au adunat toți Israeliții la David în Hebron zicând: "Iată noi suntem oasele tale și carnea ta.
\par 2 Și mai înainte, când Saul era încă rege, tu ai povățuit pe Israel la război și l-ai adus teafăr înapoi și Domnul Dumnezeul tău ți-a spus: Tu vei paște pe poporul Meu Israel și tu vei fi povățuitorul poporului Meu Israel".
\par 3 Și au venit toate căpeteniile lui Israel la rege în Hebron și a încheiat cu ei David legământ în Hebron înaintea feței Domnului; și au uns pe David de rege peste Israel, după cuvântul Domnului care fusese prin Samuel.
\par 4 Apoi s-a dus David și tot Israelul la Ierusalim, adică la Iebus. Acolo însă erau Iebuseii, locuitorii țării aceleia.
\par 5 și au zis locuitorii Iebusului către David: "Nu vei intra aici!" Dar David a luat cetatea Sionului. Aceasta este cetatea lui David.
\par 6 Apoi a zis David: "Cine va lovi cel dintâi pe Iebuseu, acela va fi cap și căpetenie peste oștire". Și s-a sculat înainte de toți Ioab, fiul Țeruiei, și s-a făcut căpetenie.
\par 7 David a locuit în cetatea aceea și ea s-a și numit cetatea lui David.
\par 8 El a zidit cetatea împrejur, începând de la Milo, iar Ioab a reînnoit celelalte părți ale cetății.
\par 9 După aceea a propășit David și s-a ridicat din ce în ce mai mult și Domnul Savaot era cu. el.
\par 10 Iată cei mai de seamă dintre puternicii lui David, care s-au luptat tare împreună cu el, în domnia lui, împreună cu tot Israelul, ca să întărească domnia lui asupra lui Israel, după cuvântul Domnului.
\par 11 Și iată numărul vitejilor pe care i-a avut David: Iașobeam (Ioșeb-Bașebet), fiul lui Hacmoni, cel mai de seamă între cei treizeci; el și-a ridicat sulița asupra a trei sute de oameni și i-a ucis dintr-o dată.
\par 12 După el vine Eleazar, fiul lui Dodo Ahohitul, unul din cei trei viteji.
\par 13 Acesta a fost cu David la Pasdamim, unde se adunaseră Filistenii pentru război. Acolo, parte din câmp era semănat cu orz și Israeliții au fugit de Filisteni;
\par 14 Dar ei au stat în mijlocul câmpului, l-au apărat și au înfrânt pe Filisteni și le-a dăruit Domnul biruință mare.
\par 15 Trei din cele treizeci de căpetenii s-au coborât pe stâncă la David, în peștera Adulam, când tabăra Filistenilor era așezată în valea Refaim.
\par 16 David atunci era la loc întărit, iar oștirea de întărire a Filistenilor era atunci în Betleem.
\par 17 Și a dorit David și a zis: "Cine mă va adăpa cu apă din fântâna Betleemului care este la poartă?"
\par 18 Atunci acești trei au străbătut prin tabăra Filistenilor, au scos apă din fântâna Betleemului cea de la poartă și au luat-o și au dus-o lui David. Dar David n-a voit să o bea și a vărsat-o înaintea Domnului,
\par 19 Zicând: "Să mă ferească Dumnezeu să fac eu aceasta! Aș putea să beau eu sângele celor care s-au dus acolo cu primejduirea vieții lor? Căci cu primejduirea vieții lor au adus-o!" Și n-a vrut să o bea. Iată ce au făcut acești trei viteji.
\par 20 Și Abișai, fratele lui Ioab, era căpetenia celor trei; el a rănit deodată cu sulița trei sute de oameni și era vestit între cei trei.
\par 21 El era mai strălucit decât cei treizeci și le era căpetenie, dar cu ceilalți trei nu era deopotrivă.
\par 22 Benaia, fiul lui Iehoiada, bărbat viteaz, mare după fapte, era din Cabțeel; el a ucis doi fii ai lui Ariel Moabitul și s-a coborât într-o groapă și a ucis un leu pe o vreme cu zăpadă.
\par 23 Tot el a ucis un egiptean, un om cu statură de cinci coți, care avea în mână o suliță, ca un sul de la războiul de țesut; el s-a dus la el cu toiagul, i-a smuls sulița din mână și l-a ucis cu sulița lui.
\par 24 Iată ce a făcut Benaia, fiul lui Iehoiada, care era în cinste la cei trei viteji.
\par 25 El era mai vestit decât cei treizeci, dar cu cei trei nu era deopotrivă și David l-a pus cel mai de aproape împlinitor al poruncilor sale.
\par 26 Iar dintre oștenii cei mai de seamă erau: Asael, fratele lui Ioab; Elhanan, fiul lui Dodo, din Betleem;
\par 27 Șamot din Haror; Heleț din Pelon;
\par 28 Ira, fiul lui Icheș din Tecoa; Abiezer din Anatot;
\par 29 Sibecai Hușateul, Ilai din Ahoh;
\par 30 Maharai din Netofat, Heled, fiul lui Baana din Netofat,
\par 31 Itai, fiul lui Ribai, din Ghibeea lui Veniamin; Benaia din Piraton
\par 32 Hurai din Nahali-Gaaș; Abiel din Araba;
\par 33 Azmavet din Bahurim; Eliahba din Șaalbon;
\par 34 Fiii lui Hașem din Ghizon: Ionatan, fiul lui Șaghi din Harar;
\par 35 Ahiam, fiul lui Sacar din Harar; Elifelet, fiul lui Uri.
\par 36 Hefer din Mechera, Ahia din Pelon,
\par 37 Hețro din Carmel, Naarai, fiul lui Ezbai;
\par 38 Ioil, fratele lui Natan; Mibhar, fiul lui Hagri;
\par 39 Țelec Amonitul; Nahrai din Beerot, purtătorul de arme al lui Ioab, fiul Țeruiei,
\par 40 Ira din Iatir, Gareb din Iatir;
\par 41 Urie Heteul; Zabad, fiul lui Ahlai;
\par 42 Adina, fiul lui Șiza Rubenitul, căpetenia Rubeniților care avea sub el treizeci de inși,
\par 43 Hanan, fiul lui Maaca; Iosafat din Mitni,
\par 44 Uzia din Aștarot; Șama și Iehiel, fiii lui Hotam din Aroer,
\par 45 Iediael, fiul lui Șimri și Ioha, fratele lui Tițitul,
\par 46 Eliel din Mahavim, Ieribai și Ioșavia, fiii lui Elnaam și Itma Moabitul,
\par 47 Eliel, Obed și Iaasiel din Mețoba.

\chapter{12}

\par 1 Iată pe cei care au mai mers la David în Țiclag, pe când stătea el încă ascuns de Saul, fiul lui Chiș. Aceștia erau dintre vitejii care ajutaseră la luptă.
\par 2 Ei erau arcași, aruncau pietre și cu dreapta și cu stânga și din arcuri trăgeau cu săgeți și făceau parte dintre Veniamineni, frații lui Saul, și anume:
\par 3 Căpetenia Ahiezer, apoi Ioaș, fiii lui Șemaa din Ghibeea, Ieziel și Pelet, fiii lui Azmavet, Beraca și Iehu din Anatot;
\par 4 Ișmaia Ghibeoneanul, un viteaz dintre cei treizeci și căpetenie peste treizeci; Ieremia, Iahaziel, Iohanan și Iozabad din Ghedera;
\par 5 Eluzai, Ierimot, Bealia, Șemaria și Șefatia Harifianul;
\par 6 Elcana, Ișia, Azareel, Ioezer și Iașobeam, Coreiți;
\par 7 Ioela și Zebadia, fiii lui Ieroham din Ghedor.
\par 8 Din Gaditeni au trecut la David, în cetatea din pustie, oameni curajoși, războinici și înarmați cu scut și suliță, cu fața lor ca fața leului și iuți ca și căprioarele din munți și anume:
\par 9 Căpetenia Ezer, al doilea Obadia și al treilea Eliab;
\par 10 Al patrulea Mașmana, al cincilea Ieremia,
\par 11 Al șaselea Atai, al șaptelea Eliel,
\par 12 Al optulea Iohanan și al nouălea Elzabad,
\par 13 Al zecelea Ieremia și al unsprezecelea Macbanai.
\par 14 Aceștia sunt din fiii lui Gad și erau căpetenii în oștire: cei mai mici, peste sute și cei mai mari, peste mii.
\par 15 Aceștia au trecut Iordanul în luna întâi, când el iese din matca sa, și au alungat pe toți cei ce locuiau pe văi, spre răsărit și apus.
\par 16 Au mai venit de asemenea și dintre fiii lui Veniamin și ai lui Iuda în cetate, la David.
\par 17 Iar David a ieșit în întâmpinarea lor și le-a zis: "De ați venit cu pace, ca să-mi ajutați, atunci să fie în mine și în voi o singură inimă; iar de ați venit ca prin vicleșug să mă dați vrăjmașilor mei, atunci, cum nu este prihană în mâinile mele, va vedea și va judeca Dumnezeul părinților noștri".
\par 18 Atunci a cuprins Duhul pe Amasai, căpetenia celor treizeci și a zis: "Suntem cu tine, Davide, și pacea să fie cu tine, fiul lui Iesei! Pace ție și pace celor ce-ți ajută, că îți ajută Dumnezeul tău". Și i-a primit David și i-a pus în capul oștirii.
\par 19 Și din tribul lui Manase au trecut unii la David, când mergea el cu Filistenii la război contra lui Saul, dar nu l-au ajutat, pentru că conducătorii Filistenilor, sfătuindu-se, l-au trimis înapoi, zicând: "Pentru primejduirea capului nostru, el va trece la domnul său Saul".
\par 20 După ce s-a întors el la Țiclag, au trecut la dânsul din ai lui Manase: Adnah, Iozabad, Iediael, Micael, Iozabad, Elihu și Țiltai, căpetenii peste mii în Manase.
\par 21 Aceștia au ajutat lui David contra năvălitorilor, căci toți aceștia erau oameni viteji și căpetenii în oștire.
\par 22 Astfel în fiecare zi veneau lui David în ajutor până întru atât, încât tabăra lui ajunsese mare, ca o tabără a lui Dumnezeu.
\par 23 Iată acum numărul căpeteniilor de oștire, care au venit la David în Hebron, ca să-i încredințeze domnia lui Saul, după cuvântul Domnului:
\par 24 Fii de-ai lui Iuda care purtau scut și suliță erau șase mii opt sute, gata de luptă;
\par 25 Din fiii lui Simeon erau șapte mii o sută, oameni viteji de oștire;
\par 26 Din fiii lui Levi, patru mii șase sute;
\par 27 Iehoiada, căpetenie din neamul lui Aaron, și cu el trei mii șapte sute,
\par 28 Și Țadoc, un tânăr voinic cu neamurile lui douăzeci și două de căpetenii.
\par 29 Din fiii lui Veniamin, frații lui Saul, au venit trei mii, dar încă mulți din ei se țineau de casa lui Saul;
\par 30 Din fiii lui Efraim, douăzeci de mii opt sute de oameni viteji, oameni cunoscuți în neamul lor;
\par 31 Din jumătate din seminția lui Manase, optsprezece mii care au fost chemați pe nume ca să meargă să facă rege pe David;
\par 32 Dintre fiii lui Isahar au venit oameni înțelepți care știau ce și când trebuie să facă Israel; aceștia erau două sute căpetenii și toți frații lor urmau sfatul lor;
\par 33 Din tribul lui Zabulon au venit oameni gata de luptă și înarmați cu tot felul de arme, în număr de cincizeci de mii, în ordine și într-un suflet;
\par 34 Din tribul lui Neftali, o mie de căpetenii și cu ei treizeci și șapte de mii cu scuturi și cu sulițe;
\par 35 Din tribul lui Dan au venit douăzeci și opt de mii șase sute oameni gata de luptă;
\par 36 Din Așer au venit oșteni, gata de luptă, patruzeci de mii;
\par 37 De peste Iordan, din tribul lui Ruben, al lui Gad și din jumătate de trib al lui Manase, au venit o sută douăzeci de mii cu tot felul de arme de luptă.
\par 38 Toți acești oșteni, gata de luptă și cu toată inima, au venit la Hebron să facă rege pe David peste Israel. Dar și toți ceilalți Israeliți erau într-un cuget pentru a fi făcut rege David.
\par 39 Și au rămas acolo la David trei zile și au mâncat și au băut, pentru că frații lor pregătiseră toate pentru ei.
\par 40 Și apoi chiar vecinii lor, până chiar și Isahar, Zabulon și Neftali, aduseseră toate de ale mâncării pe asini, pe cămile, pe catâri și cu carele de boi: făină, smochine și stafide, vin, untdelemn și vite mari și mărunte, mulțime multă, pentru că bucurie mare era peste Israel.

\chapter{13}

\par 1 Atunci David s-a sfătuit cu căpeteniile cele peste mii, cu sutașii și cu toate căpeteniile,
\par 2 Și a zis David către toată adunarea Israeliților: "Dacă binevoiți voi și dacă este voia Domnului Dumnezeului nostru, să trimitem pretutindeni la ceilalți frați ai noștri, în toată țara lui Israel și totodată și la preoți și leviți prin orașele și prin satele lor, ca sa Se adune la noi;
\par 3 Și să aducem la noi chivotul Dumnezeului nostru, pentru că în zilele lui Saul ne-am îndreptat spre el".
\par 4 Atunci toată adunarea a zis: "Așa să fie", pentru că lucrul acesta s-a părut drept înaintea a tot poporul.
\par 5 Astfel a adunat David pe toți Israeliții, de la Șihorul egiptean până la intrarea Hamatului, ca să strămute chivotul Domnului din Chiriat-Iearim.
\par 6 Atunci s-a dus David și tot Israelul la Chiriat-Iearim, ce este în Iuda, ca să strămute de acolo chivotul lui Dumnezeu, înaintea căruia se cheamă numele Domnului Celui care sade pe heruvimi.
\par 7 Și au adus chivotul lui Dumnezeu într-un car nou din casa lui Abinadab; iar carul îl conduceau Uza și Ahia.
\par 8 David însă și toți Israeliții jucau înaintea lui Dumnezeu cât puteau cu cântări din gură, din chitară, din psaltirion, din timpane și țimbale și din trâmbițe;
\par 9 Dar când au ajuns la aria lui Chidon, Uza și-a întins mâna, ca să sprijine chivotul, căci boii erau să-l răstoarne.
\par 10 Atunci S-a mâniat Domnul pe Uza și l-a lovit, pentru că și-a întins mâna spre chivot; și el a murit acolo pe loc înaintea lui Dumnezeu.
\par 11 Și s-a întristat David că Domnul lovise pe Uza și a numit locul acela Pereț-Uza și așa se numește până azi.
\par 12 Și s-a temut David de Dumnezeu în ziua aceea și a zis: "Cum voi duce la mine chivotul lui Dumnezeu?"
\par 13 De aceea nu a dus David chivotul la sine, în cetatea lui David, ci l-a întors la casa lui Obed-Edom.
\par 14 Și a rămas chivotul Domnului la Obed-Edom, în casa lui, trei luni și a binecuvântat Domnul casa lui Obed-Edom și toate ale lui.

\chapter{14}

\par 1 În vremea aceea a trimis Hiram, regele Tirului, soli la David și lemne de cedru și pietrari și dulgheri, ca să-i ridice casa.
\par 2 Când a aflat David că l-a întărit Domnul rege peste Israel, că domnia lui a fost înălțată sus pentru poporul său Israel,
\par 3 Și-a luat încă alte femei din Ierusalim și i s-au mai născut lui David fii și fiice.
\par 4 Iată numele celor ce i s-au născut în Ierusalim: Șamua, Șobab, Natan și Solomon;
\par 5 Ibhar, Elișua și Elifelet;
\par 6 Nogah, Nefeg și Iafia;
\par 7 Elișama, Beeliada și Elifelet.
\par 8 Auzind însă Filistenii că David a fost uns rege peste tot Israelul, s-au ridicat ei toți să caute pe David. Și auzind David de aceasta, a ieșit înaintea lor.
\par 9 Iar Filistenii au venit și s-au așezat în valea Refaim.
\par 10 Atunci a întrebat David pe Dumnezeu, zicând: "Să merg eu oare contra Filistenilor și-i vei da Tu în mâna mea?" Și Domnul i-a răspuns: "Mergi, că îi voi da în mâna ta!"
\par 11 Și s-au dus ei în Baal-Perațim și acolo i-a lovit David; apoi David a zis: "Dumnezeu a zdrobit pe vrăjmași cu mâna mea, ca pe o surpătură de apă". De aceea au și dat văii aceleia numele de Baal- Perațim.
\par 12 Filistenii și-au lăsat acolo zeii, iar David i-a adunat și i-a ars cu foc.
\par 13 Și au venit iarăși Filistenii și au tăbărât în vale.
\par 14 Iar David a întrebat din nou pe Dumnezeu, iar Dumnezeu i-a zis: "Nu te duce de-a dreptul asupra lor; abate-te de la ei și te du asupra lor pe la dumbrava duzilor;
\par 15 Și când vei auzi zgomot ca de pași prin vârful duzilor, atunci să intri în luptă, căci a ieșit Dumnezeu înaintea ta, ca să bată tabăra Filistenilor".
\par 16 Și a făcut David cum îi poruncise Dumnezeu; și a lovit tabăra Filistenilor de la Ghibeon până la Ghezer.
\par 17 Atunci a răsunat numele lui David prin toate țările dimprejur și l-a făcut Domnul înfricoșător pentru toate popoarele vecine.

\chapter{15}

\par 1 David și-a făcut apoi case în cetatea lui, a pregătit loc pentru chivotul lui Dumnezeu și a făcut pentru el un cort.
\par 2 Atunci David a zis: "Nimeni, afară de leviți, nu trebuie să poarte chivotul lui Dumnezeu și să-I slujească Lui în veci".
\par 3 Atunci a adunat David pe toți Israeliții la Ierusalim, ca să ducă chivotul lui Dumnezeu la locul lui, pe care i-l pregătise el.
\par 4 A chemat deci David pe fiii lui Aaron și pe leviți și anume:
\par 5 Din urmașii lui Cahat a chemat pe căpetenia Uriel și pe frații lui, o sută douăzeci;
\par 6 Din urmașii lui Merari a chemat pe căpetenia Asaia și pe frații lui, două sute douăzeci de oameni;
\par 7 Din urmașii lui Gherșom a chemat pe căpetenia Ioil și pe frații lui, o sută treizeci de oameni;
\par 8 Din urmașii lui Elițafan a chemat pe căpetenia Șemaia și pe frații lui, două sute de oameni;
\par 9 Din urmașii lui Hebron a chemat pe căpetenia Eliel și pe frații lui, optzeci de oameni;
\par 10 Din urmașii lui Uziel a chemat pe căpetenia Aminadab și pe frații lui, o sută douăzeci de oameni.
\par 11 Apoi a chemat David pe preoții Țadoc și Abiatar și pe leviții Uriel, Asaia, Ioil, Șemaia, Eliel și Aminadab,
\par 12 Și le-a zis: "Voi, căpeteniile familiilor levite, sfințiți-vă voi și frații voștri și aduceți chivotul Domnului Dumnezeului lui Israel la locul pe care l-am pregătit eu pentru el.
\par 13 Deoarece înainte n-ați făcut voi aceasta, Domnul Dumnezeul nostru ne-a lovit, pentru că nu L-am căutat cum se cuvine".
\par 14 Atunci s-au sfințit preoții și leviții, ca să aducă chivotul Domnului Dumnezeului lui Israel.
\par 15 Și au adus fiii leviților chivotul Domnului, cum poruncise Moise după cuvântul Domnului, pe pârghii; pe umeri, iar nu cu căruța.
\par 16 Apoi a poruncit David căpeteniilor leviților să pună pe frații lor cântăreți cu instrumente muzicale, cu psaltirioane, ca să vestească cu glas tare de bucurie.
\par 17 Aceștia au pus pe leviții: Heman, fiul lui Ioil, iar din frații lui, pe Asaf, fiul lui Berechia. Din urmașii lui Merari, frații lor, au pus pe Etan, fiul lui Cușaia.
\par 18 Iar pe frații lor din a doua spiță: Zaharia, fiul lui Iaaziel, Șemiramot, Iehiel, Uni, Eliab, Benaia, Maaseia, Matitia, Elifelehu, Micneia, Obed-Edom și Ieiel, i-au pus portari.
\par 19 Heman, Asaf și Etan cântau puternic din țimbale de aramă;
\par 20 Zaharia, Iaaziel, Șemiramot, Iehiel, Uni, Eliab, Maaseia și Benaia cântau din psaltirioanele cu sunete subțiri.
\par 21 Matitia însă, Elifelehu, Micneia, Obed-Edom, Ieiel și Azazia făceau începutul cu harpele cu câte opt coarde.
\par 22 Iar Chenaia, căpetenia leviților, era dascăl de cântări, pentru că era iscusit în acestea.
\par 23 Berechia și Elcana erau ușieri la chivot.
\par 24 Preoții Șebania, Iosafat, Natanael, Amasai, Zaharia, Benaia și Eliezer trâmbițau din trâmbițe înaintea chivotului lui Dumnezeu. Obed-Edom și Iehia erau ușieri la chivot.
\par 25 Astfel s-au dus David cu bătrânii lui Israel și căpeteniile cele peste mii să aducă chivotul Domnului din casa lui Obed-Edom cu veselie.
\par 26 Și după ce a ajutat Dumnezeu leviților să aducă chivotul Domnului, atunci au junghiat pentru jertfe șapte viței și șapte berbeci.
\par 27 David era îmbrăcat cu veșminte de vison, asemenea erau îmbrăcați și toți leviții care aduceau chivotul și cântăreții și Chenania, căpetenia muzicanților și cântărelilor. David însă mai avea pe el și un efod de in.
\par 28 Așa tot Israelul a adus chivotul legământului Domnului cu strigăte de bucurie, cu sunete de corn, de trâmbițe, de țimbale și de harpe.
\par 29 Când chivotul legământului Domnului a intrat în cetatea lui David, Micol, fata lui Saul, privea de la fereastră și, văzând pe regele David jucând și veselindu-se, l-a disprețuit în inima sa.

\chapter{16}

\par 1 Astfel au adus chivotul lui Dumnezeu și l-au așezat în mijlocul cortului pe care-l făcuse David pentru el și au înălțat lui Dumnezeu arderi de tot și jertfe de pace.
\par 2 După ce David a isprăvit de adus arderile de tot și jertfele de pace, a binecuvântat poporul în numele Domnului,
\par 3 Și a împărțit tuturor Israeliților, femei și bărbați, câte o pâine și câte o bucățică de carne și câte o turtă de struguri.
\par 4 Apoi a pus la slujbă înaintea chivotului Domnului din leviți, ca să preaslăvească, să mulțumească și să preaînalțe pe Domnul Dumnezeul lui Israel, și anume:
\par 5 Pe Asaf, căpetenie; al doilea după el a pus pe Zaharia; apoi urmau Uziel, Șemiramot, Iehiel, Matitia, Eliab, Benaia, Obed-Edom și Ieiel cu psaltirioane și harpe, iar Asaf cânta din țimbale.
\par 6 A pus de asemenea pe preoții Benaia și Oziel să sune necontenit din trâmbițe înaintea chivotului legământului lui Dumnezeu.
\par 7 În această zi David, pentru întâia oară, a dat, prin Asaf și frații lui, următorul psalm de laudă Domnului:
\par 8 "Lăudați pe Domnul și chemați numele Lui; vestiți între neamuri lucrurile Lui!
\par 9 Cântați, cântați în cinstea Lui! Spuneți toate minunile Lui!
\par 10 Lăudați-vă cu numele Lui cel sfânt! Să se bucure inima celor ce-L caută pe El!
\par 11 Alergați la Domnul și la ajutorul Lui; căutați pururea fața Lui!
\par 12 Neamul lui Israel, sluga Lui, fiii lui Iacov, aleșii Lui,
\par 13 Aduceți-vă aminte de minunile Lui, de semnele Lui și de judecățile gurii Lui!
\par 14 Căci El este Domnul Dumnezeul nostru și dreptatea Lui este peste tot pământul.
\par 15 Aduceți-vă aminte de așezământul Lui, de făgăduința dată pentru mii de neamuri.
\par 16 De legământul făcut cu Avraam și de jurământul Său către Isaac,
\par 17 Jurământ pus ca o lege pentru Iacov, și ca un legământ veșnic pentru Israel,
\par 18 Zicând: ție-ți voi da pământul Canaan, ca partea voastră de moștenire.
\par 19 Ei atunci erau puțini la număr și neînsemnați, și străini în țara aceasta.
\par 20 Și treceau de la popor la popor și dintr-o împărăție la altă împărăție.
\par 21 Dar El n-a lăsat pe nimeni să-i apese, și a pedepsit regi pentru ei, zicând:
\par 22 Nu vă atingeți de unșii Mei și proorocilor Mei să nu le faceți rău.
\par 23 Cântați Domnului tot pământul, binevestiți din zi în zi izbăvirea Lui!
\par 24 Vestiri păgânilor slava Lui, spuneți la toate popoarele minunile Lui!
\par 25 Că mare este Domnul și vrednic de laudă și mai înfricoșat decât toți dumnezeii.
\par 26 Că toți dumnezeii păgânilor sunt nimic, iar Domnul a făcut cerurile.
\par 27 Înaintea Lui este strălucire și măreție, putere și bucurie în locașul Lui cel sfânt.
\par 28 Dați Domnului, neamuri păgâne, dați Domnului slavă și cinste!
\par 29 Dați Domnului slavă pentru numele Lui; aduceți-vă darul, mergeți înaintea feței Lui, închinați-vă Domnului în podoabele sfințeniei Lui!
\par 30 Să tremure înaintea Lui tot pământul, că El a întemeiat lumea și nu se va clătina.
\par 31 Să se bucure cerurile și să prăznuiască pământul, iar printre popoare să se spună: Domnul este Împărat!
\par 32 Să se zguduie marea și toate cele din ea; câmpia și toate cele de pe ea să se veselească!
\par 33 Să dănțuiască împreună toți copacii dumbrăvii înaintea feței Domnului, că vine să judece pământul.
\par 34 Lăudați pe Domnul, că în veac este mila Lui!
\par 35 Ziceți: Izbăvește-ne pe noi, Dumnezeule, Izbăvitorul nostru! Adună-ne și ne izbăvește de prin popoare, ca să slăvim sfânt numele Tău și să ne lăudăm cu slava Ta!
\par 36 Binecuvântat fie Domnul Dumnezeul lui Israel din veac în veac!" Și tot poporul a zis: "Amin! Aliluia!"
\par 37 Și a lăsat David acolo, înaintea chivotului legământului Domnului, pe Asaf și pe frații lui, ca să slujească ei înaintea chivotului neîncetat, în fiecare zi;
\par 38 Pe Obed-Edom și pe frații lui, șaizeci și opt de oameni; pe Obed-Edom, fiul lui Iedutun și pe Hosa, i-a lăsat ușieri.
\par 39 Iar pe preotul Țadoc și pe ceilalți preoți, frații săi, i-a pus înaintea locașului Domnului, cel de pe înălțimea din Ghibeon,
\par 40 Ca să înalțe arderi de tot Domnului pe jertfelnicul arderilor de tot neîncetat, dimineața și seara, pentru toate cele scrise în legea Domnului, pe care El le-a poruncit lui Israel.
\par 41 Și cu ei a lăsat pe Heman, pe Iedutun și pe ceilalți aleși, care au fost numiți pe nume, ca să slăvească pe Domnul, că în veac este mila Lui.
\par 42 Împreună cu ei, Heman și Iedutun preaslăveau pe Dumnezeu, cântând din trâmbițe și felurite instrumente muzicale; pe fiii lui Iedutun i-a pus la poartă.
\par 43 Apoi s-a dus tot poporul, fiecare la casa sa. De asemenea s-a întors și David, ca să binecuvânteze casa sa.

\chapter{17}

\par 1 Când David locuia în casa sa, a zis el către Natan proorocul: "Iată eu trăiesc în casă de cedru, iar chivotul legământului Domnului este în cort".
\par 2 Iar Natan a zis către David: "Fă tot ce ai la inimă, că Dumnezeu este cu tine!"
\par 3 Dar în aceeași noapte a fost cuvântul Domnului către Natan și i-a zis:
\par 4 "Mergi și spune robului Meu David: Așa zice Domnul: Nu tu ai să-Mi zidești Mie casă de locuit,
\par 5 Căci Eu n-am locuit în casă din ziua în care am scos pe fiii lui Israel și până astăzi, ci am umblat din cort în cort și din locaș în locaș.
\par 6 Oriunde am mers Eu cu tot Israelul, spus-am Eu oare măcar un cuvânt cuiva din judecătorii lui Israel, cărora le-am poruncit să păstorească pe poporul Meu, pentru ce nu-Mi zidești Mie casă de cedru?
\par 7 Și acum așa să spui robului Meu David: Eu te-am luat de la turma de oi, ca să fii conducătorul poporului Meu Israel;
\par 8 Și am fost cu tine pretutindeni, oriunde ai umblat; am stârpit pe toți vrăjmașii tăi înaintea feței tale și am făcut numele tău ca numele celor puternici ai pământului.
\par 9 Am rânduit loc pentru poporul Meu Israel și l-am înrădăcinat, Și va trăi el în pace la locul său, și nu va mai fi neliniștit și necredincioșii nu-l vor mai strâmtora ca altădată,
\par 10 Ca în zilele acelea, când puneam judecători peste poporul Meu Israel; dar am supus pe toți vrăjmașii tăi și ți-am vestit că Domnul ți-a pregătit ție casă.
\par 11 Când se vor împlini zilele tale și vei trece la părinții tăi, atunci Eu voi ridica pe urmașul tău după tine, și voi întări domnia ta.
\par 12 Acela Îmi va zidi Mie casă și voi întemeia tronul lui pe veci.
\par 13 Eu îi voi fi tată și el Îmi va fi fiu și mila Mea nu o voi lua de la el, cum am luat-o de la cel ce a fost înaintea ta.
\par 14 Îl voi pune pe acela în casa Mea și în împărăția Mea pe veci și tronul lui în veci va fi tare".
\par 15 Toate cuvintele și toată vedenia aceasta le-a spus Natan lui David.
\par 16 Atunci a venit regele David și a stat înaintea feței Domnului și a zis: "Cine sunt eu, Doamne Dumnezeule, și ce este casa mea, de m-ai înălțat așa?
\par 17 Dar și aceasta s-a părut încă puțin în ochii Tăi, Dumnezeule, căci iată vestești despre casa robului Tău în viitor și privești la mine, ca la un om mare, Doamne Dumnezeule!
\par 18 Ce mai poate adăuga David înaintea Ta pentru mărirea robului Tău? Tu cunoști pe robul Tău.
\par 19 Doamne, pentru robul Tău, după inima Ta, faci toate aceste lucruri mari, ca să arăți toată mărirea.
\par 20 Doamne, nu este altul asemenea Ție, și nu este Dumnezeu afară de Tine, după câte am auzit noi.
\par 21 Și nu este încă alt popor pe pământ ca poporul Tău Israel, pe care l-a călăuzit Dumnezeu, ca să-l răscumpere Sieși de popor, să-și facă nume mare și strălucit, izgonind popoarele de la fața poporului Tău, pe care l-ai izbăvit din Egipt.
\par 22 Tu ai făcut pe poporul Tău, Israel, poporul Tău pe veci și Tu, Doamne, Te-ai făcut Dumnezeul lui.
\par 23 Așadar, Doamne, cuvântul pe care l-ai grăit Tu acum despre robul Tău și despre casa lui, întărește-l pe veci, și fă cum ai zis Tu.
\par 24 Să rămână și să se preamărească numele Tău în veci, ca să se zică că Domnul Savaot, Dumnezeul lui Israel, este Dumnezeu peste Israel, și casa robului Tău David să fie tare înaintea feței Tale.
\par 25 Căci Tu, Dumnezeul meu, ai descoperit robului Tău că-i vei zidi casă, de aceea robul Tău a și îndrăznit să se roage înaintea Ta.
\par 26 Și acum, Doamne, Tu ești adevăratul Dumnezeu și Tu ai vestit despre robul Tău astfel de lucruri bune.
\par 27 Începe dar a binecuvânta casa robului Tău, ca să fie ea veșnică înaintea feței Tale. Căci dacă Tu, Doamne, o vei binecuvânta, binecuvântată va fi ea în veci".

\chapter{18}

\par 1 După aceasta a lovit David pe Filisteni și i-a supus și a luat cetatea Gat și orașele ce țineau de ea din mâinile Filistenilor.
\par 2 A lovit el de asemenea și pe Moabiți și au ajuns Moabiții robii lui David, plătindu-i tribut.
\par 3 Apoi a lovit David pe Hadadezer, regele Țobei, la Hamat, când mergea acela să-și întărească stăpânirea la râul Eufrat.
\par 4 Și a luat David de la el o mie de care de război, șapte mii de călăreți și douăzeci de mii de pedestrași. Și a stricat David toate carele, neoprind din ele decât numai o sută.
\par 5 Sirienii din Damasc ar fi venit în ajutor lui Hadadezer, regele Țobei, dar David a bătut douăzeci și două de mii de Sirieni.
\par 6 Apoi a așezat David oștire de pază în Siria Damascului și au ajuns Sirienii robii lui David, plătindu-i tribut. Domnul a ajutat lui David pretutindeni, oriunde a mers el.
\par 7 Atunci a luat David scuturile de aur, care erau la robii lui Hadadezer, și le-a adus în Ierusalim.
\par 8 Iar din Tibhat și din Cun, cetățile lui Hadadezer, a luat David foarte multă aramă. Din arama aceasta a făcut Solomon marea cea de aramă, stâlpii și vasele cele de aramă.
\par 9 Auzind Tou, regele din Hamat, că David a bătut toată armata lui Hadadezer, regele Țobei,
\par 10 A trimis pe Hadoram, fiul său, la regele David, să-l salute și să-i mulțumească, pentru că s-a războit cu Hadadezer și l-a bătut, căci Tou era în război cu Hadadezer, și a trimis cu el tot felul de vase de aur, de argint și de aramă.
\par 11 Regele David a închinat aceste vase Domnului, împreună cu aurul și argintul pe care-l luase el de la toate popoarele: de la Edomiți, Moabiți, Amoniți, Filisteni și Amaleciți.
\par 12 Și Abișai, fiul Țeruiei, a bătut optsprezece mii de Edomiți în Valea Sărată.
\par 13 Și a pus în Edom oaste de pază și s-au făcut Edomiții robii lui David, căci Domnul ajuta lui David oriunde mergea.
\par 14 Și a domnit David peste tot Israelul și a făcut judecată și dreptate la tot poporul său.
\par 15 Ioab, fiul Țeruiei, era comandantul oștirii, iar Iosafat, fiul lui Ahilud, era cronicar.
\par 16 Țadoc, fiul lui Ahitub, și Ahimelec, fiul lui Abiatar, au fost preoți, iar Șausa (Serais) a fost secretar.
\par 17 Benaia, fiul lui Iehoiada, era căpetenie peste Cheretieni și Peletieni, iar fiii lui David erau cei întâi pe lângă rege,

\chapter{19}

\par 1 După aceasta a murit Nahaș, regele Amoniților, și în locul lui s-a făcut rege fiul său.
\par 2 Atunci David a zis: "Am să arăt milă lui Hanun, fiul lui Nahaș, pentru binefacerea ce mi-a arătat tatăl său". Și a trimis David soli să-l mângâie pentru pierderea tatălui său. Au mers deci solii lui David în țara Amoniților, ca să-l mângâie pe Hanun.
\par 3 Însă căpeteniile Amoniților au zis către Hanun: "Socotești tu, oare, că David din dragoste pentru tatăl tău a trimis la tine mângâietori? Nu cumva au venit slugile lui la tine, ca să iscodească și să vadă țara și apoi să o pustiiască "
\par 4 Atunci a prins Hanun pe trimișii lui David și i-a ras și le-a tăiat hainele pe jumătate, până la coapsă, și așa le-a dat drumul și ei au plecat.
\par 5 Spunându-se lui David de pățania oamenilor acelora, a trimis el în întâmpinarea lor, că erau tare batjocoriți, și li s-a zis: "Rămâneți în Ierihon până vă vor crește bărbile și atunci vă veți întoarce acasă".
\par 6 După ce Amoniții au văzut că au ajuns urâți lui David, au trimis Hanun și Amoniții o mie de talanți de argint, ca să ia în solda lor niște care de război și călăreți din Siria Mesopotamiei, Siria Maacăi și din Țoba.
\par 7 Și au luat în solda lor treizeci și două de mii de care și pe regele Maacăi cu poporul lui, care au venit și au tăbărât înaintea Medebei. Iar Amoniții s-au adunat din cetățile lor și au ieșit la război.
\par 8 Când David a auzit de acestea, a trimis pe Ioab cu toată oștirea de viteji.
\par 9 Atunci au înaintat Amoniții și s-au așezat în linie de bătaie la porțile cetății, iar regii care veniseră erau la o parte în câmp.
\par 10 Ioab, văzând că are a lupta pe două laturi, una în față și alta în spate, a ales oșteni din toți cei mai de seamă din Israel și i-a rânduit contra Sirienilor.
\par 11 Iar cealaltă parte de popor a încredințat-o lui Abișai, fratele său, ca să se îndrepte contra Amoniților.
\par 12 Apoi a zis: "Dacă Sirienii vor fi mai tari decât mine, să-mi vii tu în ajutor, iar dacă Amoniții vor fi mai tari decât tine, îți voi veni eu în ajutor.
\par 13 Fii curajos și să stăm cu tărie pentru apărarea poporului nostru și pentru cetățile Dumnezeului nostru, și Domnul să facă ce va binevoi".
\par 14 și a intrat Ioab Și oamenii ce erau cu el în luptă cu Sirienii, dar aceștia au fugit de el.
\par 15 Amoniții însă, văzând că Sirienii fug, au fugit și ei de Abișai, fratele lui Ioab, și s-au dus în cetate. Atunci Ioab a venit la Ierusalim.
\par 16 Sirienii, văzând că sunt bătuți de Israeliți, au trimis soli și au scos pe Sirienii care erau dincolo de râu, iar Șofac, căpetenia lui Hadadezer, îi conducea.
\par 17 Când s-a spus aceasta lui David, el a adunat pe toți Israeliții, a trecut Iordanul și, venind asupra acelora, s-a așezat în linie de bătaie în fața lor. Și a intrat David în luptă cu Sirienii și aceștia s-au luptat cu el.
\par 18 Dar curând Sirienii au fugit de Israeliți, iar David, a nimicit Sirienilor șapte mii de care și patruzeci de mii de pedestrași, și pe Șofac, comandantul oștirii, I-a ucis.
\par 19 Când au văzut slugile lui Hadadezer că sunt biruiți de Israeliți, au încheiat pace cu David și s-au supus. Și n-au mai voit Sirienii să mai ajute pe Amoniți.

\chapter{20}

\par 1 Peste un an, pe vremea când regii ies la război, a scos Ioab oștirea și a început să pustiiască țara Amoniților și a venit și a înconjurat Raba. Însă David a rămas în Ierusalim. Ioab a cucerit Raba și a dărâmat-o.
\par 2 Și a luat David coroana regelui lor de pe capul lui și s-a aflat că are aur în greutate de un talant și erau pe ea și pietre scumpe; și s-a pus coroana aceasta pe capul lui David. și au fost scoase din cetatea aceea și foarte multe prăzi.
\par 3 Iar poporul care era în ea a fost scos și omorât cu fierăstraie, cu ciocane de fier și cu securi. Așa a făcut David cu toate orașele Amoniților și apoi s-a întors el și tot poporul la Ierusalim.
\par 4 După aceea s-a început războiul cu Filistenii la Ghezer. Atunci Sibecai Hușatitul a bătut pe Sipai, unul din urmașii Refaimilor, și s-au supus și ei.
\par 5 Apoi iar a fost război cu Filistenii. Dar Elhanan, fiul lui Iair, a lovit pe Lahmi, fratele lui Goliat Gateul; coada suliței lui era ca a sulului de la războiul de țesut.
\par 6 Și a mai fost o luptă la Gat. Acolo era un om înalt care avea câte șase degete la mâini și la picioare, adică de toate douăzeci și patru. Și acesta era tot din urmașii Refaimilor.
\par 7 El batjocorea pe Israel, dar Ionatan, fiul lui Șama, fratele lui David, l-a ucis.
\par 8 Aceștia erau oameni născuți din Refaimi în Gat și au căzut de mâna lui David și de mâna oamenilor lui.

\chapter{21}

\par 1 Atunci s-a sculat Satana împotriva lui Israel și a îndemnat pe David să facă numărătoarea Israeliților.
\par 2 Deci a zis David către Ioab și către căpeteniile poporului: "Mergeți și numărați pe Israeliți de la Beer-Șeba până la Dan și-mi aduceți răspuns ca să știu numărul lor!"
\par 3 Ioab însă a zis: "Să înmulțească Domnul pe poporul Său de o sută de ori mai mult decât este el acum! Au doară nu sunt ei toți, o, rege, domnul meu, robii stăpânului meu? Pentru ce dar cere aceasta domnul meu? Oare pentru a se scoate asta ca o vină lui Israel?"
\par 4 Dar cuvântul regesc biruind pe Ioab, s-a dus acesta de a cutreierat tot Israelul și venind la Ierusalim,
\par 5 A dat Ioab lui David catagrafia înscrierii poporului și s-au aflat în tot Israelul un milion și o sută de mii de bărbați destoinici de război, iar în Iuda, patru sute șaptezeci de mii în stare de a lua parte la război.
\par 6 Pe leviți însă și pe Veniamineni el nu i-a numărat împreună cu ei, pentru că cuvântul regelui nu plăcuse lui Ioab.
\par 7 Lucrul acesta n-a fost plăcut nici înaintea lui Dumnezeu și de aceea a lovit El pe Israel.
\par 8 Atunci a zis David către Dumnezeu: "Am greșit mult, făcând aceasta; iartă dar vina robului Tău, că m-am purtat cu totul nepriceput".
\par 9 Iar Domnul a grăit cu Gad, proorocul lui David și i-a zis:
\par 10 "Mergi și spune lui David: Așa zice Domnul: Îți pun înainte trei pedepse; alege-ți una din ele și o voi trimite asupra ta".
\par 11 A venit deci Gad la David și i-a zis: "Așa grăiește Domnul, alege:
\par 12 Sau trei ani de foamete, sau trei luni să fii tu urmărit de vrăjmașii tăi și sabia dușmanilor să ajungă până la tine, sau trei zile sabia Domnului și molima să fie pe pământ și îngerul Domnului să pustiiască în toate hotarele lui Israel. Vezi acum ce trebuie să răspund Celui ce m-a trimis cu acest cuvânt".
\par 13 Și a răspuns David lui Gad: "Sunt într-o mare neliniște! Să cad mai bine în mâinile Domnului, căci îndurările Lui sunt foarte multe, dar să nu cad în mâinile oamenilor".
\par 14 Atunci a trimis Domnul molimă asupra lui Israel și au murit șaptezeci de mii de Israeliți.
\par 15 Și a trimis Dumnezeu îngerul la Ierusalim ca să-l piardă. Și când a început el să pustiiască, a văzut Domnul și I s-a făcut milă pentru această nenorocire și a zis către îngerul pierzător: "Destul! De acum lasă-ți mâinile în jos!" Îngerul Domnului stătea atunci deasupra ariei lui Ornan (Aravna) Iebuseul.
\par 16 Atunci ridicându-și David ochii săi, a văzut pe îngerul Domnului stând între pământ și cer cu sabia goală în mâna sa, întinsă asupra Ierusalimului; și a căzut David și bătrânii cu fețele la pământ, îmbrăcați în sac.
\par 17 Și a zis David către Dumnezeu: "Oare nu eu am poruncit să se numere poporul? Eu am greșit, eu am făcut rău; dar aceste oi ce au făcut? Doamne Dumnezeul meu, să vină mâna Ta asupra mea, asupra casei tatălui meu, iar nu asupra poporului Tău, ca să-l piardă pe el!"
\par 18 Iar îngerul Domnului a zis lui Gad proorocul să spună lui David: "Să se suie David și să ridice un jertfelnic Domnului în aria lui Ornan Iebuseul".
\par 19 Și s-a dus David, după cuvântul lui Gad pe care i-l grăise în numele Domnului.
\par 20 Ornan, întorcându-se, a văzut îngerul, și cei trei fii ai lui s-au ascuns împreună cu el; în vremea aceea Ornan treiera.
\par 21 A venit deci David la Ornan; iar Ornan, privind și văzând pe David, a ieșit din arie și s-a plecat înaintea lui David cu fața până la pământ.
\par 22 Atunci David a zis către Ornan: "Dă-mi un loc în arie, că am să fac pe el jertfelnic Domnului; dă-mi-l cu prețul cât costă, ca să înceteze prăpădul în popor".
\par 23 Iar Ornan a zis lui David: "Ia-ți! Facă domnul meu regele ce binevoiește! Iată eu dau și boi pentru ardere de tot și uneltele de treierat ca lemne pentru foc și grâu pentru prinos. Toate acestea le dau în dar!"
\par 24 Regele David a zis către Ornan: "Nu, eu voiesc să cumpăr de la tine cu preț adevărat, căci nu mă voi apuca să aduc avutul tău Domnului și nu voi aduce ardere de tot vite luate în dar".
\par 25 Și a dat David lui Ornan pentru acest loc șase sute de sicli de aur.
\par 26 Apoi a făcut David acolo jertfelnic Domnului și a înălțat ardere de tot și jertfe de împăcare și a chemat pe Domnul și Dumnezeu l-a auzit trimițând foc din cer pe altarul arderii de tot.
\par 27 Atunci a zis Domnul către înger: "Pune-ți sabia în teacă".
\par 28 În vremea aceasta, văzând David că Domnul l-a auzit în aria lui Ornan Iebuseul, a adus acolo jertfă.
\par 29 Cortul Domnului însă, pe care-l făcuse Moise în pustiu și altarul arderii de tot se aflau în vremea aceea pe înălțimea de la Ghibeon.
\par 30 Și nu s-a putut duce David acolo, ca să întrebe pe Domnul, pentru că era îngrozit de sabia îngerului Domnului.

\chapter{22}

\par 1 Apoi a zis David: "Iată templul Domnului Dumnezeu și iată jertfelnicul arderilor de tot al lui Israel!"
\par 2 Și a poruncit David să aducă pe străinii din pământul lui Israel și i-a pus cioplitori de piatră ca să cioplească pietre pentru zidirea templului Domnului.
\par 3 Apoi a pregătit David fier foarte mult pentru piroane la ferecarea ușilor și pentru legături; aramă atât de multă încât nu se mai putea cântări;
\par 4 Lemne de cedru, nemăsurat, pentru că Sidonienii și Tirienii aduseseră lui David foarte mult lemn de cedru.
\par 5 După aceea a zis David: "Solomon, fiul meu, e tânăr și cu puțină putere, iar templul care are a se zidi pentru Domnul trebuie să fie foarte măreț, spre slava și podoaba a toată lumea; deci voi pregăti eu toate pentru el". Și a pregătit David multe până la moartea sa.
\par 6 Chemând apoi pe fiul său Solomon, i-a poruncit să ridice templul Domnului Dumnezeului lui Israel.
\par 7 Și a zis David lui Solomon: "Fiul meu, eu am avut la inimă să zidesc templu în numele Domnului Dumnezeului meu;
\par 8 Dar a fost către mine cuvântul Domnului și a zis: Tu ai vărsat mult sânge și ai purtat războaie mari; nu se cuvine să zidești tu casă numelui Meu, pentru că ai vărsat mult sânge pe pământ înaintea feței Mele.
\par 9 Iată însă ție ți se va naște un fiu; acela va fi pașnic și Eu îi voi da liniște din partea tuturor vrăjmașilor dimprejur; de aceea numele lui va fi Solomon. Și voi da lui Israel în zilele lui pace și liniște.
\par 10 El va zidi templu numelui Meu și el Îmi va fi fiu, iar Eu îi voi fi tată și voi întări tronul domniei lui peste Israel pe veci.
\par 11 Și acum, fiul meu, să fie Domnul cu tine, ca să ai spor și să zidești templul Domnului Dumnezeului tău, cum a vorbit El despre tine.
\par 12 Să-ți dea ție Domnul minte și înțelepciune și să te pună peste Israel! Să păzești legile Domnului Dumnezeului tău!
\par 13 Atunci vei fi cu spor, de te vei sili să împlinești așezămintele și legile pe care le-a dat Domnul lui Moise pentru Israel. Fii tare și curajos, nu te teme și nu te deznădăjdui.
\par 14 Și iată eu din sărăcia mea am pregătit pentru templul Domnului o sută de mii de talanți de aur și un milion de talanți de argint, iar fier și aramă fără număr, pentru că acestea sunt foarte multe; și lemn și piatră de asemenea am pregătit și să mai adaugi și tu la acestea.
\par 15 Ai mulțime de lucrători și cioplitori de piatră, sculptori, dulgheri și tot felul de oameni pricepuți la tot felul de lucruri.
\par 16 Aur, argint, aramă și fier ai cât nu se pot cântări; începe și fă! Domnul va fi cu tine".
\par 17 Apoi a poruncit David tuturor mai-marilor lui Israel să ajute lui Solomon, fiul său, zicând:
\par 18 "Au nu este cu voi Domnul Dumnezeul vostru, Care v-a dăruit pace din toate părțile, pentru că El a dat în mâinile mele pe locuitorii țării și s-a supus țara înaintea Domnului și înaintea poporului Său?
\par 19 Așadar îndreptați-vă inima și sufletul vostru, ca să caute pe Domnul Dumnezeul vostru! Sculați-vă și zidiți locaș sfânt Domnului Dumnezeu, ca să mutați chivotul legământului Domnului și vasele sfinte ale lui Dumnezeu în locașul zidit numelui Domnului".

\chapter{23}

\par 1 Îmbătrânind David și săturându-se de viață, a făcut rege peste Israel pe fiul său Solomon.
\par 2 A adunat pe toate căpeteniile lui Israel și pe preoți și pe leviți.
\par 3 Și au fost numărați leviții, de la treizeci de ani în sus și s-a aflat numărul lor, numărați pe cap, treizeci și opt de mii de bărbați.
\par 4 Dintre ei au fost rânduiți pentru lucru în templul Domnului douăzeci și patru de mii; șase mii să fie scriitori și judecători,
\par 5 Patru mii să fie portari și patru mii să laude pe Domnul cu instrumentele muzicale pe care el le făcuse pentru aceasta.
\par 6 Și i-a împărțit David în cete, care să facă de rând, după fiii lui Levi: Gherșon, Cahat și Merari.
\par 7 Din Gherșoniți erau Laedan și Șimei.
\par 8 Fiii lui Laedan au fost trei: Iehiel, căpetenie, Zetam și Ioil.
\par 9 Fiii lui Șimei au fost trei: Șelomit, Haziel și Haran. Aceștia sunt căpeteniile familiilor lui Laedan.
\par 10 Tot fii ai lui Șimei au mai fost: Iahat, Ziza, Ieuș și Beraia. Acești patru sunt tot fii ai lui Șimei.
\par 11 Iahat a fost căpetenie; Ziza era al doilea; Ieuș și Beraia au avut puțini copii și de aceea ei au fost socotiți la un loc la casa tatălui lor.
\par 12 Fiii lui Cahat au fost patru: Amram, Ițhar, Hebron și Uziel.
\par 13 Fiii lui Amram au fost: Aaron și Moise. Aaron a fost ales ca să fie sfințit pentru Sfânta Sfintelor, el și fiii lui pe veci, pentru a săvârși tămâierea înaintea feței Domnului, ca să-I slujească Lui și să binecuvânteze numele Lui în veci.
\par 14 Iar Moise, omul lui Dumnezeu și fiii lui au fost numărați la tribul lui Levi.
\par 15 Fiii lui Moise au fost: Gherșom și Eliezer.
\par 16 Fiul lui Gherșom a fost Șebuel, căpetenia.
\par 17 Fiul lui Eliezer a fost: Rehabia, căpetenia. Eliezer n-a mai avut alți copii. Rehabia însă a avut foarte mulți copii.
\par 18 Fiul lui Ițhar a fost Șelomit, căpetenia.
\par 19 Fiii lui Hebron au fost: întâiul Ieria, al doilea Amaria, al treilea Iahaziel și al patrulea Iecameam.
\par 20 Fiii lui Uziel au fost: întâiul Mica și al doilea Ișia.
\par 21 Fiii lui Merari au fost: Mahli și Muși; fiii lui Mahli au fost: Eleazar și Chiș.
\par 22 Eleazar însă a murit și n-a avut feciori, ci numai fete și le-au luat de soții fiii lui Chiș, verii lor.
\par 23 Fiii lui Muși au fost trei: Mahli, Eder și Ieremot.
\par 24 Aceștia sunt fiii lui Levi, după casele părinților lor, adică a capilor de familie, după numărătoarea lor pe nume și pe capete, care făceau slujbă la templul Domnului de la douăzeci de ani în sus.
\par 25 Căci David a zis: "Domnul Dumnezeul lui Israel a dat liniște poporului Său și l-a așezat în Ierusalim pe veci,
\par 26 Și leviții nu vor mai duce cortul și tot felul de lucruri ale lui pentru slujbele din el".
\par 27 De aceea, după cele din urmă porunci ale lui David, au fost numărați leviții de la douăzeci de ani în sus,
\par 28 Ca să fie pe lângă fiii lui Aaron, pentru a sluji la templul Domnului, în curte și în camerele din jur, pentru curățirea a tot ceea ce este sfânt și pentru săvârșirea slujbelor în templul lui Dumnezeu,
\par 29 Pentru a îngriji de pâinile punerii înainte, de făina de grâu pentru prinosul de pâine și azime, pentru cele de copt, de fript și de toată măsura și cântarul,
\par 30 Și pentru a sta dimineața și seara să laude și să slăvească pe Domnul,
\par 31 Și la toate arderile de tot aduse Domnului sâmbăta, la lună nouă și la sărbători, după număr, cum este scris pentru ele, să fie totdeauna înaintea Domnului,
\par 32 Și ca să păzească chivotul legii și locașul sfânt și pe fiii lui Aaron, frații lor, la slujbele din templul Domnului.

\chapter{24}

\par 1 Iată acum cetele în care au fost împărțiți fiii lui Aaron: Fiii lui Aaron au fost: Nadab, Abiud, Eleazar și Itamar.
\par 2 Nadab și Abiud au murit înainte de tatăl lor, iar copii n-au avut și de aceea au preoțit numai pe Eleazar și Itamar.
\par 3 David i-a împărțit astfel: Pe Țadoc, unul din fiii lui Eleazar și pe Ahimelec, unul din fiii lui Itamar, i-a pus să facă slujbele cu rândul.
\par 4 S-au găsit însă între fiii lui Eleazar mai multe căpetenii decât între fiii lui Itamar și i-a împărțit astfel: Din fiii lui Eleazar șaisprezece capi de familii, iar din fiii lui Itamar opt.
\par 5 Și i-a împărțit prin sorți, pentru că cei mai însemnați în locașul sfânt și înaintea lui erau dintre fiii lui Eleazar și dintre fiii lui Itamar.
\par 6 Și i-a înscris Șemaia, fiul lui Natanael, scriitor din leviți, înaintea feței regelui și a căpeteniilor, înaintea preoților Țadoc și Ahimelec, fiul lui Abiatar și înaintea capilor de familie ai preoților și leviților, luând prin tragere la sorți o familie din neamul lui Eleazar și apoi una din neamul lui Itamar.
\par 7 Întâiul sorț a căzut lui Iehoiarib, al doilea lui Iedaia,
\par 8 Al treilea lui Harim, al patrulea lui Seorim,
\par 9 Al cincilea lui Malchia, al șaselea lui Miiamin,
\par 10 Al șaptelea lui Hacoț, al optulea lui Abia,
\par 11 Al nouălea lui Ieșua, al zecelea lui Șecania,
\par 12 Al unsprezecelea lui Eliașib, al doisprezecelea lui Iachim,
\par 13 Al treisprezecelea lui Hupa, al paisprezecelea lui Ișbaal,
\par 14 Al cincisprezecelea lui Bilga, al șaisprezecelea lui Imer,
\par 15 Al șaptesprezecelea lui Hezir, al optsprezecelea lui Hapițeț,
\par 16 Al nouăsprezecelea lui Petahia, al douăzecilea lui Iezechiel,
\par 17 Al douăzeci și unulea lui Iachin, al douăzeci și doilea lui Gamul,
\par 18 Al douăzeci și treilea lui Delaia și al douăzeci și patrulea lui Maazia.
\par 19 Aceasta era înșirarea lor la slujbă, cum trebuia să vină în templul Domnului, după rânduiala lor dată prin Aaron, tatăl lor, cum poruncise acestuia Domnul Dumnezeul lui Israel.
\par 20 Ceilalți fiii ai lui Levi au fost împărțiți astfel: Din fiii lui Amram: Șubael; din fiii lui Șubael: Iehdia;
\par 21 Din fiii lui Rehabia, întâiul era Ișia;
\par 22 Din ai lui Ițhar, Șelomot; din ai lui Șelomot era Iahat;
\par 23 Din ai lui Hebron întâiul era Ieria, al doilea, Amaria, al treilea, Iahaziel, al patrulea, Iecameam.
\par 24 Din ai lui Uziel era Mica; din ai lui Mica era Șamir.
\par 25 Fratele lui Mica a fost Ișia; din fiii lui Ișia era Zaharia.
\par 26 Fiii lui Merari au fost: Mahli și Muși; din fiii lui Iaazia a fost Beno;
\par 27 Din fiii lui Merari, după Iaazia, au fost: Beno, Șoham, Zacur și Ibri.
\par 28 Mahli a avut pe Eleazar, iar acesta n-a avut fii.
\par 29 Chiș a avut pe Ierahmeel.
\par 30 Fiii lui Muși au fost: Mahli, Eder și Ierimot. Aceștia sunt fiii leviților, după familii.
\par 31 Au aruncat și ei sorți la fel ca și frații lor, fiii lui Aaron, înaintea feței regelui David, a lui Țadoc și Ahimelec, și a capilor familiilor preoțești și levite, fără să se facă deosebire între cei bătrâni și cei tineri.

\chapter{25}

\par 1 Apoi David și căpeteniile oștirii au împărțit la slujbă pe fiii lui Asaf, ai lui Heman și ai lui Iedutun, ca să proorocească acompaniați de harfe, alăute și chimvale.
\par 2 Din fiii lui Asaf au fost rânduiți la slujbele lor aceștia: Zacur, Iosif, Netania și Așarela, fiii lui Asaf, sub conducerea lui Asaf, care cântau după porunca regelui.
\par 3 Din ai lui Iedutun au fost rânduiți fiii lui Iedutun: Ghedalia, Țeri, Isaia, Șimei, Hașabia și Matitia; ei erau șase sub conducerea tatălui lor Iedutun, care cânta din chitară spre slava și lauda Domnului.
\par 4 Din ai lui Heman au fost rânduiți fiii lui Heman: Buchia, Matania, Uziel, Șebuel, Ierimot, Hanania, Hanani, Eliata, Ghidalti, Romamti-Ezer, Ioșbecașa, Maloti, Hotir și Mahaziot.
\par 5 Toți aceștia sunt fiii lui Heman, care era văzătorul regelui, după cuvintele lui Dumnezeu, ca să mărească slava Lui. Și i-a dat Dumnezeu lui Heman paisprezece fii și trei fete.
\par 6 Toți aceștia cântau sub conducerea tatălui lor în templul Domnului din chimvale, psaltirioane și harpe, la slujbele din templul Domnului, după arătările lui David sau ale lui Asaf, Iedutun și Heman.
\par 7 Iar numărul lor, cu al fraților lor care învățaseră să cânte înaintea Domnului, și al tuturor care știau acest lucru era două sute optzeci și opt.
\par 8 Au aruncat și ei sorți pentru rândul la slujbă, mic cu mare, dascăl și ucenic deopotrivă.
\par 9 Întâiul sorț a căzut pentru Iosif, fiul lui Asaf, cu frații și fiii lui; ei erau doisprezece; al doilea, lui Ghedalia cu frații și fiii lui, ei erau doisprezece;
\par 10 Al treilea, lui Zacur cu frații lui și cu fiii lui; ei erau doisprezece.
\par 11 Al patrulea, lui Ițri cu fiii și cu frații lui; ei erau doisprezece.
\par 12 Al cincilea, lui Netania cu fiii și frații lui; ei erau doisprezece.
\par 13 Al șaselea, lui Buchia cu fiii și frații lui; ei erau doisprezece.
\par 14 Al șaptelea lui Ieșarela cu fiii și frații lui; ei erau doisprezece.
\par 15 Al optulea, lui Isaia cu fiii și frații lui; ei erau doisprezece.
\par 16 Al nouălea, lui Matania cu fiii și frații lui; ei erau doisprezece.
\par 17 Al zecelea, lui Șimei cu fiii și cu frații lui; ei erau doisprezece.
\par 18 Al unsprezecelea, lui Azareel cu fiii și frații lui; ei erau doisprezece.
\par 19 Al doisprezecelea, lui Hașabia cu fiii și frații lui; ei erau doisprezece.
\par 20 Al treisprezecelea, lui Șebuel cu fiii și frații lui; ei erau doisprezece.
\par 21 Al paisprezecelea, lui Matitia cu fiii și frații lui; ei erau doisprezece.
\par 22 Al cincisprezecelea, lui Ierimot cu fiii și frații lui; ei erau doisprezece.
\par 23 Al șaisprezecelea, lui Hanania cu fiii și frații lui; ei erau doisprezece.
\par 24 Al șaptesprezecelea, lui Ioșbecașa cu fiii și cu frații lui; ei erau doisprezece.
\par 25 Al optsprezecelea, lui Hanani cu fiii și frații lui; aceștia erau doisprezece.
\par 26 Al nouăsprezecelea, lui Maloti cu fiii și frații lui; aceștia erau doisprezece.
\par 27 Al douăzecilea, lui Eliata cu fiii și frații lui; ei erau doisprezece.
\par 28 Al douăzeci și unulea, lui Hotir cu fiii și frații lui; ei erau doisprezece.
\par 29 Al douăzeci și doilea, lui Ghidalti cu fiii și frații lui; ei erau doisprezece.
\par 30 Al douăzeci și treilea, lui Mahaziot cu fiii și frații lui; aceștia erau doisprezece.
\par 31 Al douăzeci și patrulea, lui Romamti-Ezer cu fiii și frații lui; aceștia erau doisprezece.

\chapter{26}

\par 1 Iată acum împărțirea portarilor. Din fiii lui Core: Meșelemia, fiul lui Core, unul din fiii lui Asaf.
\par 2 Fiii lui Meșelemia au fost: întâiul născut Zaharia, al doilea Iediael, al treilea Zebadia, al patrulea Iatniel,
\par 3 Al cincilea Elam, al șaselea Iohanan, al șaptelea Elihoenai.
\par 4 Fiii lui Obed-Edom au fost: Întâiul născut Șemaia, al doilea Iehozabad, al treilea Ioah, al patrulea Sacar, al cincilea Natanael,
\par 5 Al șaselea Amiel, al șaptelea Isahar, al optulea Peultai, pentru că Dumnezeu l-a binecuvântat.
\par 6 Fiului său Șemaia i s-a născut de asemenea fii, care au fost căpetenii în neamul lor, pentru că au fost oameni puternici.
\par 7 Fiii lui Șemaia au fost: Otni, Rafael, Obed și Elzabad; frații lui, oameni puternici, au fost: Elihu, Semachia și Isbacom.
\par 8 Toți aceștia sunt dintre fiii lui Obed-Edom; ei, fiii lor și frații lor, au fost oameni sârguincioși și la slujbă pricepuți; au fost șaizeci și doi din Obed-Edom.
\par 9 Meșelemia a avut optsprezece fii și frați, oameni vrednici.
\par 10 Hosa, unul din fiii lui Merari, a avut fii pe Șimri, căpetenie, deși n-a fost întâiul născut, dar tatăl său l-a pus căpetenie;
\par 11 Al doilea Hilchia, al treilea Tebalia, al patrulea Zaharia; toți fiii și frații lui Hosa au fost treisprezece.
\par 12 Așa a fost împărțirea portarilor, după capii de familie, vrednici de slujbă, împreună cu frații lor, ca să slujească la templul Domnului.
\par 13 Și au aruncat și ei sorți, mare și mic, după familiile lor, pentru fiecare poartă.
\par 14 Și pentru poarta dinspre răsărit a căzut sorțul lui Șelemia; și lui Zaharia, fiul lui, sfetnic înțelept, i s-a aruncat sorț și i-a căzut sorț pentru poarta de miazănoapte.
\par 15 Lui Obed-Edom i-a căzut poarta dinspre miazăzi; iar fiilor lui le-a căzut sorțul pentru cămări.
\par 16 Lui Șupim și Hosa le-a căzut pentru cea dinspre apus, la porțile Șelechet, unde drumul urcă și unde sunt străji față în față.
\par 17 Spre răsărit câte șase leviți, spre miazănoapte câte patru, spre miazăzi câte patru, iar la cămări câte doi.
\par 18 Spre apus, în fața pridvorului la drum, câte patru, iar la pridvor câte doi.
\par 19 Acestea sunt cetele de portari din fiii lui Core și din fiii lui Merari.
\par 20 Iar alții dintre leviți, frații lor, păzeau vistieria templului lui Dumnezeu și vistieria lucrurilor sfinte,
\par 21 Și anume: Fiii lui Laedan, fiul lui Gherșon, Căpeteniile familiilor din Laedan Gherșonitul: Iehiel,
\par 22 Și fiii lui Iehiel: Zetam și Ioil, fratele lui, care păzeau vistieria templului lui Dumnezeu,
\par 23 Împreună cu urmașii lui Amram Ițhar, Hebron, Uziel;
\par 24 Șebuel, fiul lui Gherșon, fiul lui Moise, era străjuitor de căpetenie al vistieriilor.
\par 25 Fratele său Eleazar avea fiu pe Rehabia; acesta a avut fiu pe Isaia; acesta a avut fiu pe Ioram; acesta a avut fiu pe Zicri, iar acesta a avut fiu pe Șelomit.
\par 26 Șelomit și frații lui privegheau asupra tuturor vistieriilor lucrurilor sfinte care le hărăzise regele David, căpeteniile familiilor, căpeteniile peste mii și peste sute și căpeteniile de oștire.
\par 27 Din cuceriri și prăzi ei afierosiseră pentru întreținerea templului Domnului
\par 28 Și tot ce afierosise Samuel proorocul și Saul, fiul lui Chiș, Abner, fiul lui Ner, și Ioab, fiul Țeruiei; toate cele afierosite erau în grija lui Șelomit și a fraților lui.
\par 29 Din neamul lui Ițhar, Hanania și fiii lui erau rânduiți la slujbele din afară ale Israeliților, ca scriitori și judecători.
\par 30 Din neamul lui Hebron, Hașabia și fiii lui, oameni curajoși, în număr de o mie șapte sute, aveau supravegherea asupra lui Israel de cealaltă parte de Iordan, spre apus, pentru tot felul de slujbe ale Domnului și ale regelui.
\par 31 În neamul Hebroniților, Ieria era căpetenia Hebroniților, în neamul și familiile lor. În anul al patruzecilea al domniei lui David ei au fost numărați și s-au găsit între ei bărbați curajoși în Iazerul Galaadului.
\par 32 Și frații lui, oameni vrednici, în număr de două mii șapte sute erau capi de familie. Pe aceștia i-a pus regele David peste triburile lui Ruben și Gad și peste jumătate din seminția lui Manase, pentru toate treburile lui Dumnezeu și ale regelui.

\chapter{27}

\par 1 Iată fiii lui Israel, după numărul lor, capii de familie, căpeteniile peste mii, peste sute și cârmuitorii care, împărțiți în cete, slujeau regelui la tot cuvântul, ducându-se și venind în fiecare lună, în toate lunile anului. În fiecare ceată erau câte douăzeci și patru de mii.
\par 2 Peste ceata întâi, pentru luna întâi, era căpetenia Iașobeam, fiul lui Zabdiel; în ceata lui erau douăzeci și patru de mii;
\par 3 El era din fiii lui Fares, mai-mare peste toate căpeteniile de oștire în luna întâi.
\par 4 Peste ceata din luna a doua era Dodai Ahohitul; în ceata lui se afla și căpetenia Miclot; și în ceata lui erau douăzeci și patru de mii.
\par 5 A treia mare căpetenie de oștire, pentru luna a treia, era Benaia, fiul lui Iehoiada preotul; și în ceata lui erau douăzeci și patru de mii.
\par 6 Acest Benaia era unul dintre cei treizeci de viteji și căpetenie peste ei; și în ceata lui se afla Amizabad, fiul său.
\par 7 A patra căpetenie, pentru luna a patra, era Asael, fratele lui Ioab, și după el era Zebadia, fiul său; și în ceata lui erau douăzeci și patru de mii;
\par 8 A cincea căpetenie, pentru luna a cincea, era Șamhut Izrahitul; și în ceata lui erau douăzeci și patru de mii.
\par 9 A șasea căpetenie, pentru luna a șasea, era Ira, fiul lui Icheș Tecoanul; și în ceata lui erau douăzeci și patru de mii.
\par 10 A șaptea căpetenie, pentru luna a șaptea, era Heleț Peloninul, din fiii lui Efraim; și în ceata lui erau douăzeci și patru de mii.
\par 11 A opta căpetenie, pentru luna a opta, era Sibecai Hușatitul, din seminția lui Zarah; și în ceata lui erau douăzeci și patru de mii.
\par 12 A noua căpetenie, pentru luna a noua, era Abiezer Anatoteanul, din fiii lui Veniamin; și în ceata lui erau douăzeci și patru de mii.
\par 13 A zecea căpetenie, pentru luna a zecea, era Maherai din Netofat, din familia lui Zara; și în ceata lui erau douăzeci și patru de mii.
\par 14 A unsprezecea căpetenie, pentru luna a unsprezecea, era Benaia din Piraton, din fiii lui Efraim; și în ceata lui erau douăzeci și patru de mii.
\par 15 A douăsprezecea căpetenie, pentru luna a douăsprezecea, era Heldai din Netofat, din urmașii lui Otniel; și în ceata lui erau douăzeci și patru de mii.
\par 16 Iar peste triburile lui Israel căpetenii înalte erau: la Rubeniți, Eliezer, fiul lui Zicri; la Simeon, Șefatia, fiul lui Maaca;
\par 17 La leviți era Hașabia, fiul lui Chemuel; la Aaron era Țadoc;
\par 18 La Iuda era Elihu, din frații lui David; la Isahar era Omri, fiul lui Micael;
\par 19 La Zabulon era Ișmaia, fiul lui Obadia; la Neftali era Ierimot, fiul lui Azriel;
\par 20 La fiii lui Efraim era Hoseia, fiul lui Azazia; la jumătatea de trib a lui Manase era Ioil, fiul lui Pedaia;
\par 21 La jumătatea de trib al lui Manase din Galaad, era Ido, fiul lui Zaharia; la Veniamin era Iaasiel, fiul lui Abner;
\par 22 La Dan era Azareel, fiul lui Ieroham. Iată căpeteniile triburilor lui Israel.
\par 23 David n-a făcut numărătoarea acelora, care erau de la douăzeci de ani în jos, pentru că Domnul zisese că El va înmulți pe Israel ca stelele cerului.
\par 24 Ioab, fiul Țeruiei, începuse să facă numărătoarea, dar nu o sfârșise. Și pentru aceasta a venit mânia lui Dumnezeu asupra lui Israel și numărătoarea aceea n-a intrat în cronica regelui David.
\par 25 Peste vistieriile regale era Azmavet, fiul lui Adiel, iar peste depozitele de provizii de la câmp, de prin cetăți și de prin sate și turnuri era Ionatan, fiul lui Uzia.
\par 26 Peste cei ce se îndeletniceau cu lucrul câmpului, cu agricultura, era Ezri, fiul lui Chelub.
\par 27 Peste vii era Șimei din Rama, iar peste depozitele de vin din vii era Zabdi, fiul lui Șifmi.
\par 28 Peste livezile de măslini și de smochini din câmpie era Baal-Hanan din Gheder, iar peste depozitele de untdelemn era Ioaș.
\par 29 Peste vitele mari care pășteau în Șaron era Șitrai Hașaroneanul; iar peste cele din văi, Șafat, fiul lui Adlai.
\par 30 Peste cămile era Obil Ismaelitul; peste asini era Iehdia Meroneanul.
\par 31 Peste oi și capre era Iaziz Agariteanul. Toți aceștia erau căpetenii peste averea regelui David.
\par 32 Ionatan, unchiul lui David, era sfetnic, om înțelept și scriitor; Iehiel, fiul lui Hacmoni, era pe lângă fiii regelui.
\par 33 Ahitofel era sfetnicul regelui; Hușai Architul era prietenul regelui.
\par 34 Iar după Ahitofel a fost Iehoiada, fiul lui Benaia și Abiatar, iar Ioab era căpetenia oștirii pe lângă rege.

\chapter{28}

\par 1 Apoi a adunat David la Ierusalim pe toate căpeteniile lui Israel, pe mai marii triburilor, căpeteniile cetelor care slujeau regelui, căpeteniile peste mii, peste sute, îngrijitorii moșiilor și turmelor regelui, pe fiii săi cu eunucii, căpeteniile oștirii și pe toți vitejii lui.
\par 2 Și sculându-se, regele David a zis: "Ascultați-mă, fraților și poporul meu! Am avut la inima mea gând să zidesc locaș de odihnă pentru chivotul legământului Domnului și așternut picioarelor Dumnezeului nostru și cele de trebuință pentru zidire le-am pregătit.
\par 3 Dar Dumnezeu mi-a zis: Să nu zidești locaș numelui Meu, pentru că tu ești om războinic și ai vărsat sânge.
\par 4 Cu toate acestea m-a ales Domnul Dumnezeul lui Israel din toată casa tatălui meu, ca să fiu rege peste Israel în veci, pentru că pe Iuda l-a ales El domn, iar din casa lui Iuda a ales casa tatălui meu și din casa tatălui meu și dintre fiii tatălui meu a binevoit a mă pune pe mine rege peste tot Israelul;
\par 5 Iar din toți fiii mei - căci mulți fii mi-a dat Domnul - a ales pe Solomon, fiul meu, să șadă pe tronul regatului Domnului, peste Israel.
\par 6 Și mi-a zis: Solomon, fiul tău, va zidi locașul Meu și curțile Mele, pentru că Mi l-am ales pe el de fiu și Eu îi voi fi lui tată.
\par 7 Și voi întări domnia lui pe veci, dacă va fi tare în împlinirea poruncilor Mele și a așezămintelor Mele, ca până astăzi.
\par 8 Și acum înaintea ochilor a tot Israelul, a adunării Domnului și în auzul Dumnezeului nostru vă grăiesc: Păziți și țineți toate poruncile Domnului Dumnezeului vostru, ca să stăpâniți tot pământul cel bun și să-l lăsați după voi moștenire copiilor voștri pe veci.
\par 9 Și tu, Solomon, fiul meu, cunoaște pe Dumnezeul tatălui tău și Îi slujește din toată inima și din tot sufletul, căci Domnul cercetează toate inimile și cunoaște toată mișcarea gândurilor. De Îl vei căuta pe El, Îl vei găsi, iar de Îl vei părăsi și El te va părăsi.
\par 10 Bagă de seamă însă, de vreme ce Domnul te-a ales să zidești locaș sfințeniei Lui, fii tare și fă ceea ce a rânduit El".
\par 11 Și a dat David lui Solomon, fiul său, planul pridvorului și al caselor lui, al cămărilor lui, al odăilor celor mari de primire, al odăilor celor mai dinăuntru de odihnă și al casei chivotului legii.
\par 12 A dat de asemenea planul tuturor celor ce avea el în gândul său: Planul curții templului Domnului, al tuturor cămărilor dimprejur, al vistieriilor lucrurilor sfinte
\par 13 Al încăperilor preoților și leviților, al tuturor slujitorilor din templul Domnului și al tuturor vaselor sfinte din templul Domnului,
\par 14 Al lucrurilor de aur, cu arătarea greutății, al tuturor vaselor de slujbă, al tuturor lucrurilor de argint, cu arătarea greutății lor și al tuturor celorlalte vase de slujbă.
\par 15 Apoi i-a dat aurul pentru sfeșnicele și pentru candelele de aur ale lor, cu însemnarea greutății fiecăruia din sfeșnice și din candelele lui; de asemenea și argintul pentru sfeșnicele de argint, cu însemnarea greutății fiecăruia din sfeșnice și din candelele lui, potrivit cu menirea de slujbă a fiecăruia din sfeșnice;
\par 16 Și aurul pentru mesele pâinilor punerii înainte, cu însemnarea greutății fiecăreia din mesele de aur și argintul pentru mesele de argint.
\par 17 I-a dat aurul pentru furculițele, castroanele și cupele cele de aur curat și pentru vasele de aur, cu însemnarea greutății fiecărui vas și argintul pentru vasele de argint, cu însemnarea greutății fiecărui vas,
\par 18 Precum și aurul pentru jertfelnicul tămâierii, turnat din aur, cu însemnarea greutății. I-a dat modelul carului divin, al heruvimilor de aur, cu aripile întinse pentru acoperirea chivotului legământului Domnului.
\par 19 "Toate acestea sunt în scrisoarea insuflată de la Domnul - a zis David - cum m-a luminat El pentru toate lucrările zidirii".
\par 20 Apoi a zis David către fiul său Solomon: "Fii tare și curajos și pășește la lucru, nu te teme, nici te speria, căci Domnul Dumnezeu, Dumnezeul meu, este cu tine. El nu se va depărta de tine, nici nu te va părăsi, până nu vei isprăvi toată lucrarea ce se cere la templul Domnului.
\par 21 Iată și cetele de preoți și de leviți pentru toate slujbele cele de la templul lui Dumnezeu. Ai de asemenea oameni sârguincioși pentru orice lucru și iscusiți pentru orice lucrare; și căpeteniile și tot poporul sunt gata să împlinească toate poruncile tale".

\chapter{29}

\par 1 După aceea a zis regele David către toată adunarea: "Solomon, fiul meu, singurul pe care l-a ales Dumnezeu, este tânăr și cu puțină putere, iar lucrul acesta este mare, fiindcă nu este pentru om zidirea aceasta, ci pentru Domnul Dumnezeu.
\par 2 Din toate puterile am pregătit eu pentru templul lui Dumnezeu mult aur pentru lucrurile cele de aur, argint pentru cele de argint, aramă pentru cele de aramă, fier pentru cele de fier, lemn pentru cele de lemn, pietre de onix și pietre pentru încrustat, pietre frumoase de diferite culori și tot felul de pietre scumpe și multă marmură.
\par 3 Mai mult! Din dragoste pentru templul Dumnezeului meu, dau încă tot ce am eu aur și argint, templului Dumnezeului meu, afară de ceea ce am pregătit eu pentru templul cel sfânt,
\par 4 Și anume: trei mii de talanți de aur, aur de ofir, șapte mii de talanți de argint curat, pentru căptușirea pereților în templu,
\par 5 Pentru orice lucru de aur, pentru tot lucrul de argint și pentru tot lucrul de mână de meșter. Și cine vrea să vină astăzi cu mâinile pline la Domnul?"
\par 6 Au început atunci să aducă jertfă capii de familii și căpeteniile triburilor, căpeteniile peste mii și peste sute și căpeteniile cele peste averea regelui.
\par 7 Și au dat pentru zidirea templului lui Dumnezeu cinci mii de talanți și zece mii de drahme aur, argint zece mii de talanți, aramă optsprezece mii talanți și fier o sută de mii de talanți.
\par 8 Și cei care aveau pietre scumpe, le-au dat și pe acelea în vistieria templului Domnului, prin mâinile lui Iehiel Gherșonitul.
\par 9 Și s-a bucurat poporul de râvna lor, pentru că din toată inima au jertfit Domnului. De asemenea s-a bucurat foarte mult și regele David.
\par 10 Atunci a slăvit David pe Domnul înaintea a toată adunarea și a zis: "Binecuvântat ești Tu, Doamne Dumnezeul lui Israel, Tatăl nostru, din veac și până în veac.
\par 11 A Ta este, Doamne, măreția și puterea și slava și biruința și strălucirea; toate câte sunt în cer și pe pământ sunt ale Tale; a Ta este, Doamne, împărăția și Tu ești mai presus de toate, ca unul ce împărățești peste toate.
\par 12 Bogăția și slava sunt de la fața Ta și Tu domnești peste toate; în mâna Ta este tăria și puterea și în puterea Ta stă să mărești și să întărești toate.
\par 13 Și acum dar, Dumnezeul nostru, Te slăvim pe Tine și lăudăm preaslăvit numele Tău.
\par 14 Că cine sunt eu și cine este poporul meu, încât să avem putința de a face asemenea jertfe? Dar de la Tine sunt toate și cele primite din mâna Ta ți le-am dat ție;
\par 15 Căci călători suntem noi înaintea Ta și pribegi, ca toți părinții noștri; ca umbra sunt zilele noastre pe pământ și nimic nu este statornic.
\par 16 Doamne Dumnezeul nostru, toată această mulțime de lucruri, pe care am pregătit-o noi pentru a zidi templu sfânt numelui Tău, din mâna Ta le avem și ale Tale sunt toate.
\par 17 Știu, Dumnezeul meu, că ispitești inimile și iubești curățenia inimii! Eu din inimă curată am jertfit toate și văd acum că și poporul Tău, care se află aici, cu bucurie Îți jertfește ție.
\par 18 Doamne Dumnezeul lui Avraam și al lui Isaac și al lui Iacov, părinții noștri, păzește acestea în veci, această aplecare a gândurilor inimii poporului Tău și îndreaptă inimile lor către Tine.
\par 19 Iar lui Solomon, fiul meu, dăruiește-i inimă dreaptă ca să păzească poruncile Tale, descoperirile Tale și legile Tale și să împlinească toate acestea și să înalțe clădirea pentru care am făcut pregătire".
\par 20 Apoi a zis David către toată adunarea: "Binecuvântați pe Domnul Dumnezeul nostru!" Și toată adunarea a binecuvântat pe Domnul Dumnezeul părinților săi și a căzut și s-a închinat Domnului și regelui.
\par 21 Apoi au adus Domnului jertfe și au înălțat arderi de tot Domnului, a doua zi după aceasta: o mie de miei, o mie de berbeci și o mie de vitei cu turnările lor și o mulțime de jertfe de la tot Israelul.
\par 22 Și au mâncat și au băut înaintea Domnului în ziua aceea cu mare bucurie; iar în alt rând au făcut rege pe Solomon, fiul lui David, și l-au uns înaintea Domnului ca rege, iar pe Țadoc arhiereu.
\par 23 Și s-a urcat Solomon pe tronul Domnului, ca rege, în locul lui David, tatăl său, și a avut spor și tot Israelul s-a supus lui.
\par 24 S-au supus lui Solomon de asemenea toate căpeteniile și puternicii, precum și toți fiii lui David.
\par 25 Iar Domnul a mărit pe Solomon în ochii a tot Israelul și i-a dăruit domnie slăvită, cum nu mai avusese înainte de el nici unul din regii lui Israel.
\par 26 David, fiul lui Iesei, a domnit peste tot Israelul.
\par 27 Timpul domniei lui peste Israel a fost patruzeci de ani: în Hebron a domnit el șapte ani, iar în Ierusalim a domnit treizeci și trei.
\par 28 Și a murit la adânci bătrânețe, sătul de viață, de bogăție și de slavă, iar în locul lui s-a făcut rege Solomon, fiul său.
\par 29 Faptele lui David, cele dintâi și cele de pe urmă, sunt scrise în însemnările lui Samuel văzătorul și în însemnările lui Natan proorocul și în însemnările lui Gad văzătorul,
\par 30 Precum și toată domnia lui și bărbăția lui și întâmplările ce s-au petrecut cu el și cu Israel și cu toate împărățiile pământului.


\end{document}