\begin{document}

\title{Marcu}


\chapter{1}

\par 1 Începutul Evangheliei lui Iisus Hristos, Fiul lui Dumnezeu,
\par 2 Precum este scris în proorocie (la Maleahi) și Isaia: "Iată Eu trimit îngerul Meu înaintea feței Tale, care va pregăti calea Ta.
\par 3 Glasul celui ce strigă în pustie: Gătiți calea Domnului, drepte faceți cărările Lui".
\par 4 Ioan boteza în pustie, propovăduind botezul pocăinței întru iertarea păcatelor.
\par 5 Și ieșeau la el tot ținutul Iudeii și toți cei din Ierusalim și se botezau de către el, în râul Iordan, mărturisindu-și păcatele.
\par 6 Și Ioan era îmbrăcat în haină de păr de cămilă, avea cingătoare de piele împrejurul mijlocului și mânca lăcuste și miere sălbatică.
\par 7 Și propovăduia, zicând: Vine în urma mea Cel ce este mai tare decât mine, Căruia nu sunt vrednic, plecându-mă, să-I dezleg cureaua încălțămintelor.
\par 8 Eu v-am botezat pe voi cu apă, El însă vă va boteza cu Duh Sfânt.
\par 9 Și în zilele acelea, Iisus a venit din Nazaretul Galileii și s-a botezat în Iordan, de către Ioan.
\par 10 Și îndată, ieșind din apă, a văzut cerurile deschise și Duhul ca un porumbel coborându-Se peste El.
\par 11 Și glas s-a făcut din ceruri: Tu ești Fiul Meu cel iubit, întru Tine am binevoit.
\par 12 Și îndată Duhul L-a mânat în pustie.
\par 13 Și a fost în pustie patruzeci de zile, fiind ispitit de satana. Și era împreună cu fiarele și îngerii Îi slujeau.
\par 14 După ce Ioan a fost prins, Iisus a venit în Galileea, propovăduind Evanghelia împărăției lui Dumnezeu.
\par 15 Și zicând: S-a împlinit vremea și s-a apropiat împărăția lui Dumnezeu. Pocăiți-vă și credeți în Evanghelie.
\par 16 Și umblând pe lângă Marea Galileii, a văzut pe Simon și pe Andrei, fratele lui Simon, aruncând mrejele în mare, căci ei erau pescari.
\par 17 Și le-a zis Iisus: Veniți după Mine și vă voi face să fiți pescari de oameni.
\par 18 Și îndată, lăsând mrejele, au mers după El.
\par 19 Și mergând puțin mai înainte, a văzut pe Iacov al lui Zevedeu și pe Ioan, fratele lui. Și ei erau în corabie, dregându-și mrejele.
\par 20 Și i-a chemat pe ei îndată. Iar ei, lăsând pe tatăl lor Zevedeu în corabie, cu lucrătorii, s-au dus după El.
\par 21 Și venind în Capernaum și îndată intrând sâmbăta în sinagogă, îi învăța.
\par 22 Și erau uimiți de învățătura Lui, căci El îi învăța pe ei ca Cel ce are putere, iar nu în felul cărturarilor.
\par 23 Și era în sinagoga lor un om cu duh necurat, care striga tare,
\par 24 Zicând: Ce ai cu noi, Iisuse Nazarinene? Ai venit ca să ne pierzi? Te știm cine ești: Sfântul lui Dumnezeu.
\par 25 Și Iisus l-a certat, zicând: Taci și ieși din el.
\par 26 Și scuturându-l duhul cel necurat și strigând cu glas mare, a ieșit din el.
\par 27 Și s-au spăimântat toți, încât se întrebau între ei, zicând: Ce este aceasta? O învățătură nouă și cu putere; că și duhurilor necurate le poruncește, și I se supun.
\par 28 Și a ieșit vestea despre El îndată pretutindeni în toată împrejurimea Galileii.
\par 29 Și îndată ieșind ei din sinagogă, au venit în casa lui Simon și a lui Andrei, cu Iacov și cu Ioan.
\par 30 Iar soacra lui Simon zăcea, prinsă de friguri, și îndată I-au vorbit despre ea.
\par 31 Și apropiindu-Se a ridicat-o, apucând-o de mână. Și au lăsat-o frigurile și ea le slujea.
\par 32 Iar când s-a făcut seară și soarele apusese, au adus la El pe toți bolnavii și demonizații.
\par 33 Și toată cetatea era adunată la ușă.
\par 34 Și a tămăduit pe mulți care pătimeau de felurite boli și demoni mulți a alungat. Iar pe demoni nu-i lăsa să vorbească, pentru că-L știau că El e Hristos.
\par 35 Și a doua zi, foarte de dimineață, sculându-Se, a ieșit și S-a dus într-un loc pustiu și Se ruga acolo.
\par 36 Și a mers după El Simon și cei ce erau cu el.
\par 37 Și aflându-L, I-au zis: Toți Te caută pe Tine.
\par 38 Și El a zis lor: Să mergem în altă parte, prin cetățile și satele învecinate, ca să propovăduiesc și acolo, căci pentru aceasta am venit.
\par 39 Și venind propovăduia în sinagogile lor, în toată Galileea, alungând pe demoni.
\par 40 Și un lepros a venit la El, rugându-L și îngenunchind și zicând: De voiești, poți să mă curățești.
\par 41 Și făcându-I-se milă, a întins mâna și S-a atins de el și i-a zis: Voiesc, curățește-te.
\par 42 Și îndată s-a îndepărtat lepra de la el și s-a curățit.
\par 43 Și poruncindu-i cu asprime, îndată l-a alungat,
\par 44 Și i-a zis: Vezi, nimănui să nu spui nimic, ci mergi de te arată preotului și adu, pentru curățirea ta, cele ce a rânduit Moise, spre mărturie lor.
\par 45 Iar el, ieșind, a început să propovăduiască multe și să răspândească cuvântul, încât Iisus nu mai putea să intre pe față în cetate, ci stătea afară, în locuri pustii, și veneau la El de pretutindeni.

\chapter{2}

\par 1 Și intrând iarăși în Capernaum, după câteva zile s-a auzit că este în casă.
\par 2 Și îndată s-au adunat mulți, încât nu mai era loc, nici înaintea ușii, și le grăia lor cuvântul.
\par 3 Și au venit la El, aducând un slăbănog, pe care-l purtau patru inși.
\par 4 Și neputând ei, din pricina mulțimii, să se apropie de El, au desfăcut acoperișul casei unde era Iisus și, prin spărtură, au lăsat în jos patul în care zăcea slăbănogul.
\par 5 Și văzând Iisus credința lor, i-a zis slăbănogului: Fiule, iertate îți sunt păcatele tale!
\par 6 Și erau acolo unii dintre cărturari, care ședeau și cugetau în inimile lor:
\par 7 Pentru ce vorbește Acesta astfel? El hulește. Cine poate să ierte păcatele, fără numai unul Dumnezeu?
\par 8 Și îndată cunoscând Iisus, cu duhul Lui, că așa cugetau ei în sine, le-a zis lor: De ce cugetați acestea în inimile voastre?
\par 9 Ce este mai ușor a zice slăbănogului: Iertate îți sunt păcatele, sau a zice: Scoală-te, ia-ți patul tău și umblă?
\par 10 Dar, ca să știți că putere are Fiul Omului a ierta păcatele pe pământ, a zis slăbănogului:
\par 11 Zic ție: Scoală-te, ia-ți patul tău și mergi la casa ta.
\par 12 Și s-a sculat îndată și, luându-și patul, a ieșit înaintea tuturor, încât erau toți uimiți și slăveau pe Dumnezeu, zicând: Asemenea lucruri n-am văzut niciodată.
\par 13 Și iarăși a ieșit la mare și toată mulțimea venea la El și îi învăța.
\par 14 Și trecând, a văzut pe Levi al lui Alfeu, șezând la vamă, și i-a zis: Urmează-Mi! Iar el, sculându-se, I-a urmat.
\par 15 Și când ședea El în casa lui Levi, mulți vameși și păcătoși ședeau la masă cu Iisus și cu ucenicii Lui. Că erau mulți și-I urmau.
\par 16 Iar cărturarii și fariseii, văzându-L că mănâncă împreună cu vameșii și păcătoșii, ziceau către ucenicii Lui: De ce mănâncă și bea Învățătorul vostru cu vameșii și păcătoșii?
\par 17 Dar, auzind, Iisus le-a zis: Nu cei sănătoși au nevoie de doctor, ci cei bolnavi. N-am venit să chem pe cei drepți ci pe păcătoși la pocăință.
\par 18 Ucenicii lui Ioan și ai fariseilor posteau și au venit și I-au zis Lui: De ce ucenicii lui Ioan și ucenicii fariseilor postesc, iar ucenicii Tăi nu postesc?
\par 19 Și Iisus le-a zis: Pot, oare, prietenii mirelui să postească cât timp este mirele cu ei? Câtă vreme au pe mire cu ei, nu pot să postească.
\par 20 Dar vor veni zile, când se va lua mirele de la ei și atunci vor posti în acele zile.
\par 21 Nimeni nu coase la haină veche petic dintr-o bucată de stofă nouă, iar de nu, peticul nou va trage din haina veche și se va face o ruptură și mai rea.
\par 22 Nimeni, iarăși, nu pune vin nou în burdufuri vechi, iar de nu, vinul nou sparge burdufurile și vinul se vară și burdufurile se strică; încât vinul nou trebuie să fie în burdufuri noi.
\par 23 Și pe când mergea El într-o sâmbătă prin semănături, ucenicii Lui, în drumul lor, au început să smulgă spice.
\par 24 Și fariseii Îi ziceau: Vezi, de ce fac sâmbăta ce nu se cuvine?
\par 25 Și Iisus le-a răspuns: Au niciodată n-ați citit ce a făcut David, când a avut nevoie și a flămânzit, el și cei ce erau cu el?
\par 26 Cum a intrat în casa lui Dumnezeu, în zilele lui Abiatar arhiereul, și a mâncat pâinile punerii înainte, pe care nu se cuvenea să le mănânce decât numai preoții, și a dat și celor ce erau cu el?
\par 27 Și le zicea lor: Sâmbăta a fost făcut pentru om, iar nu omul pentru sâmbătă.
\par 28 Astfel că Fiul Omului este domn și al sâmbetei.

\chapter{3}

\par 1 Și iarăși a intrat în sinagogă. Și era acolo un om având mâna uscată.
\par 2 Și Îl pândeau pe Iisus să vadă dacă îl va vindeca sâmbăta, ca să-L învinuiască.
\par 3 Și a zis omului care avea mâna uscată: Ridică-te în mijloc!
\par 4 Și a zis lor: Se cuvine, sâmbăta, a face bine sau a face rău, a mântui un suflet sau a-l pierde? Dar ei tăceau;
\par 5 Și privindu-i pe ei cu mânie și întristându-Se de învârtoșarea inimii lor, a zis omului: Întinde mâna ta! Și a întins-o, și mâna lui s-a făcut sănătoasă.
\par 6 Și ieșind, fariseii au făcut îndată sfat cu irodianii împotriva Lui, ca să-L piardă.
\par 7 Iisus, împreună cu ucenicii Lui, a plecat înspre mare și mulțime multă din Galileea și din Iudeea L-a urmat.
\par 8 Din Ierusalim, din Idumeea, de dincolo de Iordan, dimprejurul Tirului și Sidonului, mulțime mare, care, auzind câte făcea, a venit la El.
\par 9 Și a zis ucenicilor Săi să-I fie pusă la îndemână o corăbioară, ca să nu-L îmbulzească mulțimea;
\par 10 Fiindcă vindecase pe mulți, de aceea năvăleau asupra Lui, ca să se atingă de El toți câți erau bolnavi.
\par 11 Iar duhurile cele necurate, când Îl vedeau, cădeau înaintea Lui și strigau, zicând: Tu ești Fiul lui Dumnezeu.
\par 12 Și El le certa mult ca să nu-L dea pe față.
\par 13 Și S-a suit pe munte și a chemat la Sine pe câți a voit, și au venit la El.
\par 14 Și a rânduit pe cei doisprezece, pe care i-a numit apostoli, ca să fie cu El și să-i trimită să propovăduiască,
\par 15 Și să aibă putere să vindece bolile și să alunge demonii.
\par 16 Deci a rânduit pe cei doisprezece: pe Simon, căruia i-a pus numele Petru;
\par 17 Pe Iacov al lui Zevedeu și pe Ioan, fratele lui Iacov, și le-a pus lor numele Boanerghes, adică fii tunetului.
\par 18 Și pe Andrei, și pe Filip, și pe Bartolomeu, pe Matei, și pe Toma, și pe Iacov al lui Alfeu, și pe Tadeu, și pe Simon Cananeul,
\par 19 Și pe Iuda Iscarioteanul, cel care L-a și vândut.
\par 20 Și a venit în casă, și iarăși mulțimea s-a adunat, încât ei nu puteau nici să mănânce.
\par 21 Și auzind ai Săi, au ieșit ca să-L prindă, că ziceau: Și-a ieșit din fire.
\par 22 Iar cărturarii, care veneau din Ierusalim, ziceau că are pe Beelzebul și că, cu domnul demonilor, alungă demonii.
\par 23 Și chemându-i la Sine, le-a vorbit în pilde: Cum poate satana să alunge pe satana?
\par 24 Dacă o împărăție se va dezbina în sine, acea împărăție nu mai poate dăinui.
\par 25 Și dacă o casă se va dezbina în sine, casa aceea nu va mai putea să se țină.
\par 26 Și dacă satana s-a sculat împotriva sa însuși și s-a dezbinat, nu poate să dăinuiască, ci are sfârșit.
\par 27 Dar nimeni nu poate, intrând în casa celui tare, să-i răpească lucrurile, de nu va lega întâi pe cel tare, și atunci va jefui casa lui.
\par 28 Adevărat grăiesc vouă că toate vor fi iertate fiilor oamenilor, păcatele și hulele câte vor fi hulit;
\par 29 Dar cine va huli împotriva Duhului Sfânt nu are iertare în veac, ci este vinovat de osânda veșnică.
\par 30 Pentru că ziceau: Are duh necurat.
\par 31 Și au venit mama Lui și frații Lui și, stând afară, au trimis la El ca sn-L cheme.
\par 32 Iar mulțimea ședea împrejurul Lui. Și I-au zis unii: Iată mama Ta și frații Tăi și surorile Tale sunt afară. Te caută.
\par 33 Și, răspunzând lor, le-a zis: Cine este mama Mea și frații Mei?
\par 34 Și privind pe cei ce ședeau în jurul Lui, a zis: Iată mama Mea și frații Mei.
\par 35 Că oricine va face voia lui Dumnezeu, acesta este fratele Meu și sora Mea și mama Mea.

\chapter{4}

\par 1 Și iarăși a început Iisus să învețe, lângă mare, și s-a adunat la El mulțime foarte multă, încât El a intrat în corabie și ședea pe mare, iar toată mulțimea era lângă mare, pe uscat.
\par 2 Și-i învăța multe în pilde, și în învățătura Sa le zicea:
\par 3 Ascultați: Iată, ieșit-a semănătorul să semene.
\par 4 Și pe când semăna el, o sămânță a căzut lângă cale și păsările cerului au venit și au mâncat-o.
\par 5 Și alta a căzut pe loc pietros, unde nu avea pământ mult, și îndată a răsărit, pentru că nu avea pământ mult.
\par 6 Și când s-a ridicat soarele, s-a veștejit și, neavând rădăcină, s-a uscat.
\par 7 Altă sămânță a căzut în spini, a crescut, dar spinii au înăbușit-o și rod n-a dat.
\par 8 Și altele au căzut pe pământul cel bun și, înălțându-se și crescând, au dat roade și au adus: una treizeci, alta șaizeci, alta o sută.
\par 9 Și zicea: Cine are urechi de auzit să audă.
\par 10 Iar când a fost singur, cei ce erau lângă El, împreună cu cei doisprezece, Îl întrebau despre pilde.
\par 11 Și le-a răspuns: Vouă vă e dat să cunoașteți taina împărăției lui Dumnezeu, dar pentru cei de afară totul se face în pilde,
\par 12 Ca uitându-se, să privească și să nu vadă, și, auzind, să nu înțeleagă, ca nu cumva să se întoarcă și să fie iertați.
\par 13 Și le-a zis: Nu pricepeți pilda aceasta? Dar cum veți înțelege toate pildele?
\par 14 Semănătorul seamănă cuvântul.
\par 15 Cele de lângă cale sunt aceia în care se seamănă cuvântul, și, când îl aud, îndată vine satana și ia cuvântul cel semănat în inimile lor.
\par 16 Cele semănate pe loc pietros sunt aceia care, când aud cuvântul, îl primesc îndată cu bucurie,
\par 17 Dar n-au rădăcină în ei, ci țin până la un timp; apoi când se întâmplă strâmtorare sau prigoană pentru cuvânt, îndată se smintesc.
\par 18 Și cele semănate între spini sunt cei ce ascultă cuvântul,
\par 19 Dar grijile veacului și înșelăciunea bogăției și poftele după celelalte, pătrunzând în ei, înăbușă cuvântul și îl fac neroditor.
\par 20 Iar cele semănate pe pământul cel bun sunt cei ce aud cuvântul și-l primesc și aduc roade: unul treizeci, altul șaizeci și altul o sută.
\par 21 Și le zicea: Se aduce oare făclia ca să fie pusă sub obroc sau sub pat? Oare nu ca să fie pusă în sfeșnic?
\par 22 Căci nu e nimic ascuns ca să nu se dea pe față; nici n-a fost ceva tăinuit, decât ca să vină la arătare.
\par 23 Cine are urechi de auzit să audă.
\par 24 Și le zicea: Luați seama la ce auziți: Cu ce măsură măsurați, vi se va măsura; iar vouă celor ce ascultați, vi se va da și vă va prisosi.
\par 25 Căci celui ce are i se va da; dar de la cel ce nu are, și ce are i se va lua.
\par 26 Și zicea: Așa este împărăția lui Dumnezeu, ca un om care aruncă sămânța în pământ,
\par 27 Și doarme și se scoală, noaptea și ziua, și sămânța răsare și crește, cum nu știe el.
\par 28 Pământul rodește de la sine: mai întâi pai, apoi spic, după aceea grâu deplin în spic.
\par 29 Iar când rodul se coace, îndată trimite secera, că a sosit secerișul.
\par 30 Și zicea: Cum vom asemăna împărăția lui Dumnezeu, sau în ce pildă o vom închipui?
\par 31 Cu grăuntele de muștar care, când se seamănă în pământ, este mai mic decât toate semințele de pe pământ;
\par 32 Dar, după ce s-a semănat, crește și se face mai mare decât toate legumele și face ramuri mari, încât sub umbra lui pot să sălășluiască păsările cerului.
\par 33 Și cu multe pilde ca acestea le grăia cuvântul după cum puteau să înțeleagă.
\par 34 Iar fără pildă nu le grăia; și ucenicilor Săi le lămurea toate, deosebi.
\par 35 Și în ziua aceea, când s-a înserat, a zis către ei: Să trecem pe țărmul celălalt.
\par 36 Și lăsând ei mulțimea, L-au luat cu ei în corabie, așa cum era, căci erau cu El și alte corăbii.
\par 37 Și s-a pornit o furtună mare de vânt și valurile se prăvăleau peste corabie, încât corabia era aproape să se umple.
\par 38 Iar Iisus era la partea dindărăt a corăbiei, dormind pe căpătâi. L-au deșteptat și I-au zis: Învățătorule, nu-Ți este grijă că pierim?
\par 39 Și El, sculându-Se, a certat vântul și a poruncit mării: Taci! Încetează! Și vântul s-a potolit și s-a făcut liniște mare.
\par 40 Și le-a zis lor: Pentru ce sunteți așa de fricoși? Cum de nu aveți credință?
\par 41 Și s-au înfricoșat cu frică mare și ziceau unul către altul: Cine este oare, Acesta, că și vântul și marea I se supun?

\chapter{5}

\par 1 Și a venit de cealaltă parte a mării în ținutul Gadarenilor.
\par 2 Iar după ce a ieșit din corabie, îndată L-a întâmpinat, din morminte, un om cu duh necurat,
\par 3 Care își avea locuința în morminte, și nimeni nu putea să-l lege nici măcar în lanțuri,
\par 4 Pentru că de multe ori fiind legat în obezi și lanțuri, el rupea lanțurile, și obezile le sfărâma, și nimeni nu putea să-l potolească;
\par 5 Și neîncetat noaptea și ziua era prin morminte și prin munți, strigând și tăindu-se cu pietre.
\par 6 Iar văzându-L de departe pe Iisus, a alergat și s-a închinat Lui.
\par 7 Și strigând cu glas puternic, a zis: Ce ai cu mine, Iisuse, Fiule al lui Dumnezeu Celui Preaînalt? Te jur pe Dumnezeu să nu mă chinuiești.
\par 8 Căci îi zicea: Ieși duh necurat din omul acesta.
\par 9 Și l-a întrebat: Care îți este numele? Și I-a răspuns: Legiune este numele meu, căci suntem mulți.
\par 10 Și Îl rugau mult să nu-i trimită afară din acel ținut.
\par 11 Iar acolo, lângă munte, era o turmă mare de porci, care păștea.
\par 12 Și L-au rugat, zicând: Trimite-ne pe noi în porci, ca să intrăm în ei.
\par 13 Și El le-a dat voie. Atunci, ieșind, duhurile necurate au intrat în porci și turma s-a aruncat de pe țărmul înalt, în mare. Și erau ca la două mii și s-au înecat în mare.
\par 14 Iar cei care-i pășteau au fugit și au vestit în cetate și prin sate. Și au venit oamenii să vadă ce s-a întâmplat.
\par 15 Și s-au dus la Iisus și au văzut pe cel demonizat șezând jos, îmbrăcat și întreg la minte, el care avusese legiune de demoni, și s-au înfricoșat.
\par 16 Iar cei ce au văzut le-au povestit cum a fost cu demonizatul și despre porci.
\par 17 Și ei au început să-L roage să se ducă din hotarele lor.
\par 18 Iar intrând El în corabie, cel ce fusese demonizat Îl ruga ca să-l ia cu El.
\par 19 Iisus însă nu l-a lăsat, ci i-a zis: Mergi în casa ta, la ai tăi, și spune-le câte ți-a făcut ție Domnul și cum te-a miluit.
\par 20 Iar el s-a dus și a început să vestească în Decapole câte i-a făcut Iisus lui; și toți se minunau.
\par 21 Și trecând Iisus cu corabia iarăși de partea cealaltă, s-a adunat la El mulțime multă și era lângă mare.
\par 22 Și a venit unul din mai-marii sinagogilor, anume Iair, și văzându-L pe Iisus, a căzut la picioarele Lui,
\par 23 Și L-a rugat mult, zicând: Fiica mea este pe moarte, ci, venind, pune mâinile tale peste ea, ca să scape și să trăiască.
\par 24 Și a mers cu el. Și mulțime multă îl urma pe Iisus Și Îl îmbulzea.
\par 25 Și era o femeie care avea, de doisprezece ani, curgere de sânge.
\par 26 Și multe îndurase de la mulți doctori, cheltuindu-și toate ale sale, dar nefolosind nimic, ci mai mult mergând înspre mai rău.
\par 27 Auzind ea cele despre Iisus, a venit în mulțime și pe la spate s-a atins de haina Lui.
\par 28 Căci își zicea: De mă voi atinge măcar de haina Lui, mă voi vindeca!
\par 29 Și îndată izvorul sângelui ei a încetat și ea a simțit în trup că s-a vindecat de boală.
\par 30 Și îndată, cunoscând Iisus în Sine puterea ieșită din El, întorcându-Se către mulțime, a întrebat: Cine s-a atins de Mine?
\par 31 Și I-au zis ucenicii Lui: Vezi mulțimea îmbulzindu-Te și zici: Cine s-a atins de Mine?
\par 32 Și Se uita împrejur să vadă pe aceea care făcuse aceasta.
\par 33 Iar femeia, înfricoșându-se și tremurând, știind ce i se făcuse, a venit și a căzut înaintea Lui și I-a mărturisit tot adevărul;
\par 34 Iar El i-a zis: Fiică, credința ta te-a mântuit, mergi în pace și fii sănătoasă de boala ta!
\par 35 Încă vorbind El, au venit unii de la mai-marele sinagogii, zicând: Fiica ta a murit. De ce mai superi pe Învățătorul?
\par 36 Dar Iisus, auzind cuvântul ce s-a grăit, a zis mai-marelui sinagogii: Nu te teme. Crede numai.
\par 37 Și n-a lăsat pe nimeni să meargă cu El, decât numai pe Petru și pe Iacov și pe Ioan, fratele lui Iacov.
\par 38 Și au venit la casa mai-marelui sinagogii și a văzut tulburare și pe cei ce plângeau și se tânguiau mult.
\par 39 Și intrând, le-a zis: De vă tulburați și plângeți? Copila n-a murit, ci doarme.
\par 40 Și-L luau în râs. Iar El, scoțându-i pe toți afară, a luat cu Sine pe tatăl copilei, pe mama ei și pe cei ce îl însoțeau și a intrat unde era copila.
\par 41 Și apucând pe copilă de mână, i-a grăit: Talita kumi, care se tâlcuiește: Fiică, ție zic, scoală-te!
\par 42 Și îndată s-a sculat copila și umbla, căci era de doisprezece ani. Și s-au mirat îndată cu uimire mare.
\par 43 Dar El le-a poruncit, cu stăruință, ca nimeni să nu afle de aceasta. Și le-a zis să-i dea copilei să mănânce.

\chapter{6}

\par 1 Și a ieșit de acolo și a venit în patria Sa, iar ucenicii Lui au mers după El.
\par 2 Și, fiind sâmbătă, a început să învețe în sinagogă. Și mulți, auzindu-L, erau uimiți și ziceau: De unde are El acestea? Și ce este înțelepciunea care I s-a dat Lui? Și cum se fac minuni ca acestea prin mâinile Lui?
\par 3 Au nu este Acesta teslarul, fiul Mariei și fratele lui Iacov și al lui Iosi și al lui Iuda și al lui Simon? Și nu sunt, oare, surorile Lui aici la noi? Și se sminteau întru El.
\par 4 Și le zicea Iisus: Nu este prooroc disprețuit, decât în patria sa și între rudele sale și în casa sa.
\par 5 Și n-a putut acolo să facă nici o minune, decât că, punându-Și mâinile peste puțini bolnavi, i-a vindecat.
\par 6 Și se mira de necredința lor. Și străbătea satele dimprejur învățând.
\par 7 Și a chemat la Sine pe cei doisprezece și a început să-i trimită doi câte doi și le-a dat putere asupra duhurilor necurate.
\par 8 Și le-a poruncit să nu ia nimic cu ei, pe cale, ci numai toiag. Nici pâine, nici traistă, nici bani la cingătoare;
\par 9 Ci să fie încălțați cu sandale și să nu se îmbrace cu două haine.
\par 10 Și le zicea: În orice casă veți intra, acolo să rămâneți până ce veți ieși de acolo.
\par 11 Și dacă într-un loc nu vă vor primi pe voi, nici nu vă vor asculta, ieșind de acolo, scuturați praful de sub picioarele voastre, spre mărturie lor. Adevărat grăiesc vouă: Mai ușor va fi Sodomei și Gomorei, în ziua judecății, decât cetății aceleia.
\par 12 Și ieșind, ei propovăduiau să se pocăiască.
\par 13 Și scoteau mulți demoni și ungeau cu untdelemn pe mulți bolnavi și-i vindecau.
\par 14 Și a auzit regele Irod, căci numele lui Iisus se făcuse cunoscut, și zicea că Ioan Botezătorul s-a sculat din morți și de aceea se fac minuni prin el.
\par 15 Alții însă ziceau că este Ilie și alții că este prooroc, ca unul din prooroci.
\par 16 Iar Irod, auzind zicea: Este Ioan căruia eu am pus să-i taie capul; el s-a sculat din morți.
\par 17 Căci Irod, trimițând, l-a prins pe Ioan și l-a legat, în temniță, din pricina Irodiadei, femeia lui Filip, fratele său, pe care o luase de soție.
\par 18 Căci Ioan îi zicea lui Irod: Nu-ți este îngăduit să ții pe femeia fratelui tău.
\par 19 Iar Irodiada îl ura și voia să-l omoare, dar nu putea,
\par 20 Căci Irod se temea de Ioan, știindu-l bărbat drept și sfânt, și-l ocrotea. Și ascultându-l, multe făcea și cu drag îl asculta.
\par 21 Și fiind o zi cu bun prilej, când Irod, de ziua sa de naștere, a făcut ospăț dregătorilor lui și căpeteniilor oștirii și fruntașilor din Galileea,
\par 22 Și fiica Irodiadei, intrând și jucând, a plăcut lui Irod și celor ce ședeau cu el la masă. Iar regele a zis fetei: Cere de la mine orice vei voi și îți voi da.
\par 23 Și s-a jurat ei: Orice vei cere de la mine îți voi da, până la jumătate din regatul meu.
\par 24 Și ea, ieșind, a zis mamei sale: Ce să cer? Iar Irodiada i-a zis: Capul lui Ioan Botezătorul.
\par 25 Și intrând îndată, cu grabă, la rege, i-a cerut, zicând: Vreau să-mi dai îndată, pe tipsie, capul lui Ioan Botezătorul.
\par 26 Și regele s-a mâhnit adânc, dar pentru jurământ și pentru cei ce ședeau cu el la masă, n-a voit s-o întristeze.
\par 27 Și îndată trimițând regele un paznic, a poruncit a-i aduce capul.
\par 28 Și acela, mergând, i-a tăiat capul în temniță, l-a adus pe tipsie și l-a dat fetei, iar fata l-a dat mamei sale.
\par 29 Și auzind, ucenicii lui au venit, au luat trupul lui Ioan și l-au pus în mormânt.
\par 30 Și s-au adunat apostolii la Iisus și I-au spus Lui toate câte au făcut și au învățat.
\par 31 Și El le-a zis: Veniți voi înșivă de o parte, în loc pustiu, și odihniți-vă puțin. Căci mulți erau care veneau și mulți erau care se duceau și nu mai aveau timp nici să mănânce.
\par 32 Și au plecat cu corabia spre un loc pustiu, de o parte.
\par 33 Și i-au văzut plecând și mulți au înțeles și au alergat acolo pe jos de prin toate cetățile și au sosit înaintea lor.
\par 34 Și ieșind din corabie, Iisus a văzut mulțime mare și I s-a făcut milă de ei, căci erau ca niște oi fără păstor, și a început să-i învețe multe.
\par 35 Dar făcându-se târziu, ucenicii Lui, apropiindu-se, I-au zis: Locul e pustiu și ceasul e târziu;
\par 36 Slobozește-i, ca mergând prin cetățile și prin satele dimprejur, să-și cumpere să mănânce.
\par 37 Răspunzând, El le-a zis: Dați-le voi să mănânce. Și ei I-au zis: Să mergem noi să cumpărăm pâini de două sute de dinari și să le dăm să mănânce?
\par 38 Iar El le-a zis: Câte pâini aveți? Duceți-vă și vedeți. Și după ce au văzut, I-au spus: Cinci pâini și doi pești.
\par 39 Și El le-a poruncit să-i așeze pe toți cete, cete, pe iarbă verde.
\par 40 Și au șezut cete, cete, câte o sută și câte cincizeci.
\par 41 Și luând cele cinci pâini și cei doi pești, privind la cer, a binecuvântat și a frânt pâinile și le-a dat ucenicilor, ca să le pună înainte, asemenea și cei doi pești i-a împărțit tuturor.
\par 42 Și au mâncat toți și s-au săturat.
\par 43 Și au luat douăsprezece coșuri pline cu fărâmituri și cu ce-a rămas din pești.
\par 44 Iar cei ce au mâncat pâinile erau cinci mii de bărbați.
\par 45 Și îndată a silit pe ucenicii Lui să intre în corabie și să meargă înaintea Lui, de cealaltă parte, spre Betsaida, până ce El va slobozi mulțimea.
\par 46 Iar după ce i-a slobozit, S-a dus în munte ca să Se roage.
\par 47 Și făcându-se seară, era corabia în mijlocul mării, iar El singur pe țărm.
\par 48 Și i-a văzut cum se chinuiau vâslind, căci vântul le era împotrivă. Și către a patra strajă a nopții a venit la ei umblând pe mare și voia să treacă pe lângă ei.
\par 49 Iar lor, văzându-L umblând pe mare, li s-a părut că este nălucă și au strigat.
\par 50 Căci toți L-au văzut și s-au tulburat. Dar îndată El a vorbit cu ei și le-a zis: Îndrăzniți! Eu sunt; nu vă temeți!
\par 51 Și s-a suit la ei în corabie și s-a potolit vântul. Și erau peste măsură de uimiți în sinea lor;
\par 52 Căci nu pricepuseră nimic de la minunea pâinilor, deoarece inima lor era învârtoșată.
\par 53 Și trecând marea, au venit în ținutul Ghenizaretului și au tras la țărm.
\par 54 Și ieșind ei din corabie, îndată L-au cunoscut.
\par 55 Și străbăteau tot ținutul acela și au început să-I aducă pe bolnavi pe paturi, acolo unde auzeau că este El.
\par 56 Și oriunde intra în sate sau în cetăți sau în sătulețe, puneau la răspântii pe cei bolnavi, și-L rugau să le îngăduie să se atingă măcar de poala hainei Sale. Și câți se atingeau de El se vindecau.

\chapter{7}

\par 1 Și s-au adunat la El fariseii și unii dintre cărturari, care veniseră din Ierusalim.
\par 2 Și văzând pe unii din ucenicii Lui că mănâncă cu mâinile necurate, adică nespălate, cârteau;
\par 3 Căci fariseii și toți iudeii, dacă nu-și spală mâinile până la cot, nu mănâncă, ținând datina bătrânilor.
\par 4 Și când vin din piață, dacă nu se spală, nu mănâncă; și alte multe sunt pe care au primit să le țină: spălarea paharelor și a urcioarelor și a vaselor de aramă și a paturilor.
\par 5 Și L-au întrebat pe El fariseii și cărturarii: Pentru ce nu umblă ucenicii Tăi după datina bătrânilor, ci mănâncă cu mâinile nespălate?
\par 6 Iar El le-a zis: Bine a proorocit Isaia despre voi, fățarnicilor, precum este scris: "Acest popor Mă cinstește cu buzele, dar inima lui este departe de Mine".
\par 7 Dar în zadar Mă cinstesc, învățând învățături care sunt porunci omenești.
\par 8 Căci lăsând porunca lui Dumnezeu, țineți datina oamenilor: spălarea urcioarelor și a paharelor și altele ca acestea multe, pe care le faceți.
\par 9 Și le zicea lor: Bine, ați lepădat porunca lui Dumnezeu, ca să țineți datina voastră!
\par 10 Căci Moise a zis: "Cinstește pe tatăl tău și pe mama ta", și "cel ce va grăi de rău pe tatăl său, sau pe mama sa, cu moarte să se sfârșească".
\par 11 Voi însă ziceți: Dacă un om va spune tatălui sau mamei: Corban! adică: Cu ce te-aș fi putut ajuta este dăruit lui Dumnezeu,
\par 12 Nu-l mai lăsați să facă nimic pentru tatăl său sau pentru mama sa.
\par 13 Și astfel desființați cuvântul lui Dumnezeu cu datina voastră pe care singuri ați dat-o. Și faceți multe asemănătoare cu acestea.
\par 14 Și chemând iarăși mulțimea la El, le zicea: Ascultați-Mă toți și înțelegeți:
\par 15 Nu este nimic din afară de om care, intrând în el, să poată să-l spurce. Dar cele ce ies din om, acelea sunt care îl spurcă.
\par 16 De are cineva urechi de auzit să audă.
\par 17 Și când a intrat în casă de la mulțime, L-au întrebat ucenicii despre această pildă.
\par 18 Și El le-a zis: Așadar și voi sunteți nepricepuți? Nu înțelegeți, oare, că tot ce intră în om, din afară, nu poate să-l spurce?
\par 19 Că nu intră în inima lui, ci în pântece, și iese afară, pe calea sa, bucatele fiind toate curate.
\par 20 Dar zicea că ceea ce iese din om, aceea spurcă pe om.
\par 21 Căci dinăuntru, din inima omului, ies cugetele cele rele, desfrânările, hoțiile, uciderile,
\par 22 Adulterul, lăcomiile, vicleniile, înșelăciunea, nerușinarea, ochiul pizmaș, hula, trufia, ușurătatea.
\par 23 Toate aceste rele ies dinăuntru și spurcă pe om.
\par 24 Și ridicându-Se de acolo, S-a dus în hotarele Tirului și ale Sidonului și, intrând într-o casă, voia ca nimeni să nu știe, dar n-a putut să rămână tăinuit.
\par 25 Căci îndată auzind despre El o femeie, a cărei fiică avea duh necurat, a venit și a căzut la picioarele Lui.
\par 26 Și femeia era păgână, de neam din Fenicia Siriei. Și Îl ruga să alunge demonii din fiica ei.
\par 27 Dar Iisus i-a vorbit: Lasă întâi să se sature copiii. Căci nu este bine să iei pâinea copiilor și s-o arunci câinilor.
\par 28 Ea însă a răspuns și I-a zis: Da, Doamne, dar și câinii, sub masă, mănâncă din fărâmiturile copiilor.
\par 29 Și Iisus i-a zis: Pentru acest cuvânt, mergi. A ieșit demonul din fiica ta.
\par 30 Iar ea, ducându-se acasă, a găsit pe copilă culcată în pat, iar demonul ieșise.
\par 31 Și, ieșind din părțile Tirului, a venit, prin Sidon, la Marea Galileii, prin mijlocul hotarelor Decapolei.
\par 32 Și I-au adus un surd, care era și gângav, și L-au rugat ca să-Și pună mâna peste el.
\par 33 Și luându-l din mulțime, la o parte, Și-a pus degetele în urechile lui, și scuipând, S-a atins de limba lui.
\par 34 Și privind la cer, a suspinat și a zis lui: Effatta! ceea ce înseamnă: Deschide-te!
\par 35 Și urechile lui s-au deschis, iar legătura limbii lui îndată s-a dezlegat, și vorbea bine.
\par 36 Și le poruncea să nu spună nimănui. Dar, cu cât le poruncea, cu atât mai mult ei Îl vesteau.
\par 37 Și erau uimiți peste măsură, zicând: Toate le-a făcut bine: pe surzi îi face să audă și pe muți să vorbească.

\chapter{8}

\par 1 În zilele acelea, fiind iarăși mulțime multă și neavând ce să mănânce, Iisus, chemând la Sine pe ucenici, le-a zis:
\par 2 Milă Îmi este de mulțime, că sunt trei zile de când așteaptă lângă Mine și n-au ce să mănânce.
\par 3 Și de-i voi slobozi flămânzi la casa lor, se vor istovi pe drum, că unii dintre ei au venit de departe.
\par 4 Și ucenicii Lui I-au răspuns: De unde va putea cineva să-i sature pe aceștia cu pâine, aici în pustie.
\par 5 El însă i-a întrebat: Câte pâini aveți? Răspuns-au Lui: șapte.
\par 6 Și a poruncit mulțimii să șeadă jos pe pământ. Și, luând cele șapte pâini, a mulțumit, a frânt și a dat ucenicilor Săi, ca să le pună înainte. Și ei le-au pus mulțimii înainte.
\par 7 Și aveau și puțini peștișori. Și binecuvântându-i, a zis să-i pună și pe aceștia înaintea lor.
\par 8 Și au mâncat și s-au săturat și au luat șapte coșuri cu rămășițe de fărâmituri.
\par 9 Și ei erau ca la patru mii. Și i-a slobozit.
\par 10 Și îndată intrând în corabie cu ucenicii Săi, a venit în părțile Dalmanutei.
\par 11 Și au ieșit fariseii și se sfădeau cu El, cerând de la El semn din cer, ispitindu-L.
\par 12 Și Iisus, suspinând cu duhul Său, a zis: Pentru ce neamul acesta cere semn? Adevărat grăiesc vouă că nu se va da semn acestui neam.
\par 13 Și lăsându-i, a intrat iarăși în corabie și a trecut de cealaltă parte.
\par 14 Dar ucenicii au uitat să ia pâine și numai o pâine aveau cu ei în corabie.
\par 15 Și El le-a poruncit, zicând: Vedeți, păziți-vă de aluatul fariseilor și de aluatul lui Irod.
\par 16 Și vorbeau între ei, zicând: Aceasta o zice, fiindcă n-avem pâine.
\par 17 Și Iisus, înțelegând, le-a zis: De ce gândiți că n-aveți pâine? Tot nu înțelegeți, nici nu pricepeți? Atât de învârtoșată este inima voastră?
\par 18 Ochi aveți și nu vedeți, urechi aveți și nu auziți și nu vă aduceți aminte.
\par 19 Când am frânt cele cinci pâini, la cei cinci mii de oameni, atunci câte coșuri pline de fărâmituri ați luat? Zis-au Lui: Douăsprezece.
\par 20 Și când cu cele șapte pâini, la cei patru mii de oameni, câte coșuri pline de fărâmituri ați luat? Iar ei au zis: Șapte.
\par 21 Și le zicea: Tot nu pricepeți?
\par 22 Și au venit la Betsaida. Și au adus la El un orb și L-au rugat să se atingă de el.
\par 23 Și luând pe orb de mână, l-a scos afară din sat și, scuipând în ochii lui și punându-Și mâinile peste el, l-a întrebat dacă vede ceva.
\par 24 Și el, ridicându-și ochii, a zis: zăresc oamenii; îi văd ca pe niște copaci umblând.
\par 25 După aceea a pus iarăși mâinile pe ochii lui, și el a văzut bine și s-a îndreptat, căci vedea toate, lămurit.
\par 26 Și l-a trimis la casa sa, zicându-i: Să nu intri în sat, nici să spui cuiva din sat.
\par 27 Și a ieșit Iisus și ucenicii Lui prin satele din preajma Cezareii lui Filip. Și pe drum întreba pe ucenicii Săi, zicându-le: Cine zic oamenii că sunt?
\par 28 Ei au răspuns Lui, zicând: Unii spun că ești Ioan Botezătorul, alții că ești Ilie, iar alții că ești unul din prooroci.
\par 29 Și El i-a întrebat: Dar voi cine ziceți că sunt Eu? Răspunzând, Petru a zis Lui: Tu ești Hristosul.
\par 30 Și El le-a dat poruncă să nu spună nimănui despre El.
\par 31 Și a început să-i învețe că Fiul Omului trebuie să pătimească multe și să fie defăimat de bătrâni, de arhierei și de cărturari și să fie omorât, iar după trei zile să învieze.
\par 32 Și spunea acest cuvânt pe față. Și luându-L Petru de o parte, a început să-L dojenească.
\par 33 Dar El, întorcându-Se și uitându-Se la ucenicii Săi, a certat pe Petru și i-a zis: Mergi, înapoia mea, satano! Căci tu nu cugeți cele ale lui Dumnezeu, ci cele ale oamenilor.
\par 34 Și chemând la Sine mulțimea, împreună cu ucenicii Săi, le-a zis: Oricine voiește să vină după Mine să se lepede de sine, să-și ia crucea și să-Mi urmeze Mie.
\par 35 Căci cine va voi să-și scape sufletul îl va pierde, iar cine va pierde sufletul Său pentru Mine și pentru Evanghelie, acela îl va scăpa.
\par 36 Căci ce-i folosește omului să câștige lumea întreagă, dacă-și pierde sufletul?
\par 37 Sau ce ar putea să dea omul, în schimb, pentru sufletul său?
\par 38 Căci de cel ce se va rușina de Mine și de cuvintele Mele, în neamul acesta desfrânat și păcătos, și Fiul Omului Se va rușina de el, când va veni întru slava Tatălui său cu sfinții îngeri.

\chapter{9}

\par 1 Și le zicea lor: Adevărat grăiesc vouă că sunt unii, din cei ce stau aici, care nu vor gusta moartea, până ce nu vor vedea împărăția lui Dumnezeu, venind întru putere.
\par 2 Și după șase zile a luat Iisus cu Sine pe Petru și pe Iacov și pe Ioan și i-a dus într-un munte înalt, de o parte, pe ei singuri, și S-a schimbat la față înaintea lor.
\par 3 Și veșmintele Lui s-au făcut strălucitoare, albe foarte, ca zăpada, cum nu poate înălbi așa pe pământ înălbitorul.
\par 4 Și li s-a arătat Ilie împreună cu Moise și vorbeau cu Iisus.
\par 5 Și răspunzând Petru, a zis lui Iisus: Învățătorule, bine este ca noi să fim aici; și să facem trei colibe: Ție una și lui Moise una și lui Ilie una.
\par 6 Căci nu știa ce să spună, fiindcă erau înspăimântați.
\par 7 Și s-a făcut un nor care îi umbrea, iar un glas din nor a venit zicând: Acesta este Fiul Meu cel iubit, pe Acesta să-L ascultați.
\par 8 Dar, deodată, privind ei împrejur, n-au mai văzut pe nimeni decât pe Iisus, singur cu ei.
\par 9 Și coborându-se ei din munte, le-a poruncit ca nimănui să nu spună cele ce văzuseră, decât numai când Fiul Omului va învia din morți.
\par 10 Iar ei au ținut cuvântul, întrebându-se între ei: Ce înseamnă a învia din morți?
\par 11 Și L-au întrebat pe El, zicând: Pentru ce zic fariseii și cărturarii că trebuie să vină mai întâi Ilie?
\par 12 Iar El le-a răspuns: Ilie, venind întâi, va așeza iarăși toate. Și cum este scris despre Fiul Omului că va să pătimească multe și să fie defăimat?
\par 13 Dar vă zic vouă că Ilie a și venit și i-au făcut toate câte au voit, precum s-a scris despre el.
\par 14 Și venind la ucenici, a văzut mulțime mare împrejurul lor și pe cărturari sfădindu-se între ei.
\par 15 Și îndată toată mulțimea, văzându-L, s-a spăimântat și, alergând, I se închina.
\par 16 Și Iisus a întrebat pe cărturari: Ce vă sfădiți între voi?
\par 17 Și I-a răspuns Lui unul din mulțime: Învățătorule, am adus la Tine pe fiul meu, care are duh mut.
\par 18 Și oriunde-l apucă, îl aruncă la pământ și face spume la gură și scrâșnește din dinți și înțepenește. Și am zis ucenicilor Tăi să-l alunge, dar ei n-au putut.
\par 19 Iar El, răspunzând lor, a zis: O, neam necredincios, până când voi fi cu voi? Până când vă voi răbda pe voi? Aduceți-l la Mine.
\par 20 Și l-au adus la El. Și văzându-L pe Iisus, duhul îndată a zguduit pe copil, și, căzând la pământ, se zvârcolea spumegând.
\par 21 Și l-a întrebat pe tatăl lui: Câtă vreme este de când i-a venit aceasta? Iar el a răspuns: din pruncie.
\par 22 Și de multe ori l-a aruncat și în foc și în apă ca să-l piardă. Dar de poți ceva, ajută-ne, fiindu-Ți milă de noi.
\par 23 Iar Iisus i-a zis: De poți crede, toate sunt cu putință celui ce crede.
\par 24 Și îndată strigând tatăl copilului, a zis cu lacrimi: Cred, Doamne! Ajută necredinței mele.
\par 25 Iar Iisus, văzând că mulțimea dă năvală, a certat duhul cel necurat, zicându-i: Duh mut și surd, Eu îți poruncesc: Ieși din el și să nu mai intri în el!
\par 26 Și răcnind și zguduindu-l cu putere, duhul a ieșit; iar copilul a rămas ca mort, încât mulți ziceau că a murit.
\par 27 Dar Iisus, apucându-l de mână, l-a ridicat, și el s-a sculat în picioare.
\par 28 Iar după ce a intrat în casă, ucenicii Lui L-au întrebat, de o parte: Pentru ce noi n-am putut să-l izgonim?
\par 29 El le-a zis: Acest neam de demoni cu nimic nu poate ieși, decât numai cu rugăciune și cu post.
\par 30 Și, ieșind ei de acolo, străbăteau Galileea, dar El nu voia să știe cineva.
\par 31 Căci învăța pe ucenicii Săi și le spunea că Fiul Omului se va da în mâinile oamenilor și-L vor ucide, iar după ce-L vor ucide, a treia zi va învia.
\par 32 Ei însă nu înțelegeau cuvântul și se temeau să-L întrebe.
\par 33 Și au venit în Capernaum. Și fiind în casă, i-a întrebat: Ce vorbeați între voi pe drum?
\par 34 Iar ei tăceau, fiindcă pe cale se întrebaseră unii pe alții cine dintre ei este mai mare.
\par 35 Și șezând jos, a chemat pe cei doisprezece și le-a zis: Dacă cineva vrea să fie întâiul, să fie cel din urmă dintre toți și slujitor al tuturor.
\par 36 Și luând un copil, l-a pus în mijlocul lor și, luându-l în brațe, le-a zis:
\par 37 Oricine va primi, în numele Meu, pe unul din acești copii pe Mine Mă primește; și oricine Mă primește, nu pe Mine Mă primește, ci pe Cel ce M-a trimis pe Mine.
\par 38 Și I-a zis Ioan: Învățătorule, am văzut pe cineva scoțând demoni în numele Tău, care nu merge după noi, și l-am oprit, pentru că nu merge după noi.
\par 39 Iar Iisus a zis: Nu-l opriți, căci nu e nimeni care, făcând vreo minune în numele Meu, să poată, degrabă, să Mă vorbească de rău.
\par 40 Căci cine nu este împotriva noastră este pentru noi.
\par 41 Iar oricine vă va da să beți un pahar de apă, în numele Meu, fiindcă sunteți ai lui Hristos, adevărat zic vouă că nu-și va pierde plata sa.
\par 42 Și cine va sminti pe unul din aceștia mici, care cred în Mine, mai bine i-ar fi lui dacă și-ar lega de gât o piatră de moară și să fie aruncat în mare.
\par 43 Și de te smintește mâna ta, tai-o că mai bine îți este să intri ciung în viață, decât, amândouă mâinile având, să te duci în gheena, în focul cel nestins.
\par 44 Unde viermele lor nu moare și focul nu se stinge.
\par 45 Și de te smintește piciorul tău, taie-l, că mai bine îți este ție să intri fără un picior în viață, decât având amândouă picioarele să fii azvârlit în gheena, în focul cel nestins,
\par 46 Unde viermele lor nu moare și focul nu se stinge.
\par 47 Și de te smintește ochiul tău, scoate-l, că mai bine îți este ție cu un singur ochi în împărăția lui Dumnezeu, decât, având amândoi ochii, să fii aruncat în gheena focului.
\par 48 Unde viermele lor nu moare și focul nu se stinge.
\par 49 Căci fiecare (om) va fi sărat cu foc, după cum orice jertfă va fi sărată cu sare.
\par 50 Bună este sarea; dacă însă sarea își pierde puterea, cu ce o veți drege? Aveți sare întru voi și trăiți în pace unii cu alții.

\chapter{10}

\par 1 Și sculându-Se de acolo, a venit în hotarele Iudeii, de cealaltă parte a Iordanului, și mulțimile s-au adunat iarăși la El și iarăși le învăța, după cum obișnuia.
\par 2 Și apropiindu-se fariseii, Îl întrebau, ispitindu-L, dacă este îngăduit unui bărbat să-și lase femeia.
\par 3 Iar El, răspunzând, le-a zis: Ce v-a poruncit vouă Moise?
\par 4 Iar ei au zis: Moise a dat voie să-i scrie carte de despărțire și să o lase.
\par 5 Și răspunzând, Iisus le-a zis: Pentru învârtoșarea inimii voastre, v-a scris porunca aceasta;
\par 6 Dar de la începutul făpturii, bărbat și femeie i-a făcut Dumnezeu.
\par 7 De aceea va lăsa omul pe tatăl său și pe mama sa și se va lipi de femeia sa.
\par 8 Și vor fi amândoi un trup; așa că nu mai sunt doi, ci un trup.
\par 9 Deci ceea ce a împreunat Dumnezeu, omul să nu mai despartă.
\par 10 Dar în casă ucenicii L-au întrebat iarăși despre aceasta.
\par 11 Și El le-a zis: Oricine va lăsa pe femeia sa și va lua alta, săvârșește adulter cu ea.
\par 12 Iar femeia, de-și va lăsa bărbatul ei și se va mărita cu altul, săvârșește adulter.
\par 13 Și aduceau la El copii, ca să-Și pună mâinile peste ei, dar ucenicii certau pe cei ce-i aduceau.
\par 14 Iar Iisus, văzând, S-a mâhnit și le-a zis: Lăsați copiii să vină la Mine și nu-i opriți, căci a unora ca aceștia este împărăția lui Dumnezeu.
\par 15 Adevărat zic vouă: Cine nu va primi împărăția lui Dumnezeu ca un copil nu va intra în ea.
\par 16 Și, luându-i în brațe, i-a binecuvântat, punându-Și mâinile peste ei.
\par 17 Și când ieșea El în drum, alergând la El unul și îngenunchind înaintea Lui, Îl întreba: Învățătorule bun, ce să fac ca să moștenesc viața veșnică?
\par 18 Iar Iisus i-a răspuns: De ce-Mi zici bun? Nimeni nu este bun decât unul Dumnezeu.
\par 19 Știi poruncile: Să nu ucizi, să nu săvârșești adulter, să nu furi, să nu mărturisești strâmb, să nu înșeli pe nimeni, cinstește pe tatăl tău și pe mama ta.
\par 20 Iar el I-a zis: Învățătorule, acestea toate le-am păzit din tinerețile mele.
\par 21 Iar Iisus, privind la el cu dragoste, i-a zis: Un lucru îți mai lipsește: Mergi, vinde tot ce ai, dă săracilor și vei avea comoară în cer; și apoi, luând crucea, vino și urmează Mie.
\par 22 Dar el, întristându-se de cuvântul acesta, a plecat mâhnit, căci avea multe bogății.
\par 23 Și Iisus, uitându-Se în jur, a zis către ucenicii Săi: Cât de greu vor intra bogații în împărăția lui Dumnezeu!
\par 24 Iar ucenicii erau uimiți de cuvintele Lui. Dar Iisus, răspunzând iarăși, le-a zis: Fiilor, cât de greu este celor ce se încred în bogății să intre în împărăția lui Dumnezeu!
\par 25 Mai lesne este cămilei să treacă prin urechile acului, decât bogatului să intre în împărăția lui Dumnezeu.
\par 26 Iar ei, mai mult uimindu-se, ziceau unii către alții: Și cine poate să se mântuiască?
\par 27 Iisus, privind la ei, le-a zis: La oameni lucrul e cu neputință, dar nu la Dumnezeu. Căci la Dumnezeu toate sunt cu putință.
\par 28 Și a început Petru a-I zice: Iată, noi am lăsat toate și Ți-am urmat.
\par 29 Iisus i-a răspuns: Adevărat grăiesc vouă: Nu este nimeni care și-a lăsat casă, sau frați, sau surori, sau mamă, sau tată, sau copii, sau țarine pentru Mine și pentru Evanghelie,
\par 30 Și să nu ia însutit - acum, în vremea aceasta, de prigoniri - case și frați și surori și mame și copii și țarine, iar în veacul ce va să vină: viață veșnică.
\par 31 Și mulți dintre cei dintâi vor fi pe urmă, și din cei de pe urmă întâi.
\par 32 Și erau pe drum, suindu-se la Ierusalim, iar Iisus mergea înaintea lor. Și ei erau uimiți și cei ce mergeau după El se temeau. Și luând la Sine, iarăși, pe cei doisprezece, a început să le spună ce aveau să I se întâmple:
\par 33 Că, iată, ne suim la Ierusalim și Fiul Omului va fi predat arhiereilor și cărturarilor; și-L vor osândi la moarte și-L vor da în mâna păgânilor.
\par 34 Și-L vor batjocori și-L vor scuipa și-L vor biciui și-L vor omorî, dar după trei zile va învia.
\par 35 Și au venit la El Iacov și Ioan, fiii lui Zevedeu, zicându-I: Învățătorule, voim să ne faci ceea ce vom cere de la Tine.
\par 36 Iar El le-a zis: Ce voiți să vă fac?
\par 37 Iar ei I-au zis: Dă-ne nouă să ședem unul de-a dreapta Ta, și altul de-a stânga Ta, întru slava Ta.
\par 38 Dar Iisus le-a răspuns: Nu știți ce cereți! Puteți să beți paharul pe care îl beau Eu sau să vă botezați cu botezul cu care Mă botez Eu?
\par 39 Iar ei I-au zis: Putem. Și Iisus le-a zis: Paharul pe care Eu îl beau îl veți bea, și cu botezul cu care Eu mă botez vă veți boteza.
\par 40 Dar a ședea de-a dreapta Mea, sau de-a stânga Mea, nu este al Meu a da, ci celor pentru care s-a pregătit.
\par 41 Și auzind cei zece, au început a se mânia pe Iacov și pe Ioan.
\par 42 Și Iisus, chemându-i la Sine, le-a zis: Știți că cei ce se socotesc cârmuitori ai neamurilor domnesc peste ele și cei mai mari ai lor le stăpânesc.
\par 43 Dar între voi nu trebuie să fie așa, ci care va vrea să fie mare între voi, să fie slujitor al vostru.
\par 44 Și care va vrea să fie întâi între voi, să fie tuturor slugă.
\par 45 Că și Fiul Omului n-a venit ca să I se slujească, ci ca El să slujească și să-Și dea sufletul răscumpărare pentru mulți.
\par 46 Și au venit în Ierihon. Și ieșind din Ierihon El, ucenicii Lui și mulțime mare, Bartimeu orbul, fiul lui Timeu, ședea jos, pe marginea drumului.
\par 47 Și, auzind că este Iisus Nazarineanul, a început să strige și să zică: Iisuse, Fiul lui David, miluiește-mă!
\par 48 Și mulți îl certau ca să tacă, el însă cu mult mai tare striga: Fiule al lui David, miluiește-mă!
\par 49 Și Iisus, oprindu-Se, a zis: Chemați-l! Și l-au chemat pe orb, zicându-i: Îndrăznește, scoală-te! Te cheamă.
\par 50 Iar orbul, lepădând haina de pe el, a sărit în picioare și a venit la Iisus.
\par 51 Și l-a întrebat Iisus, zicându-i: Ce voiești să-ți fac? Iar orbul I-a răspuns: Învățătorule, să văd iarăși.
\par 52 Iar Iisus i-a zis: Mergi, credința ta te-a mântuit. Și îndată a văzut și urma lui Iisus pe cale.

\chapter{11}

\par 1 Și când s-au apropiat de Ierusalim, la Betfaghe și la Betania, lângă Muntele Măslinilor, a trimis pe doi dintre ucenicii Săi,
\par 2 Și le-a zis: Mergeți în satul care este înaintea voastră și, intrând în el, îndată veți afla un mânz legat, pe care n-a șezut până acum nici un om. Dezlegați-l și aduceți-l.
\par 3 Iar de vă va zice cineva: De ce faceți aceasta? Spuneți că Domnul are trebuință de el și îndată îl va trimite aici.
\par 4 Deci au mers și au găsit mânzul legat la o poartă, afară la răspântie, și l-au dezlegat.
\par 5 Și unii din cei ce stăteau acolo, le-au zis: De ce dezlegați mânzul?
\par 6 Iar ei le-au spus precum le zisese Iisus, și i-au lăsat.
\par 7 Și au adus mânzul la Iisus și și-au pus hainele pe el și Iisus a șezut pe el.
\par 8 Și mulți își așterneau hainele pe cale, iar alții așterneau ramuri, pe care le tăiau de prin grădini.
\par 9 Iar cei ce mergeau înainte și cei ce veneau pe urmă strigau, zicând: Osana! Bine este cuvântat Cel ce vine întru numele Domnului!
\par 10 Binecuvântată este împărăția ce vine a părintelui nostru David! Osana întru cei de sus!
\par 11 Și a intrat Iisus în Ierusalim și în templu și, privind toate în jur și vremea fiind spre seară, a ieșit spre Betania cu cei doisprezece.
\par 12 Și a doua zi, ieșind ei din Betania, El a flămânzit.
\par 13 Și văzând de departe un smochin care avea frunze, a mers acolo, doar va găsi ceva în el; și, ajungând la smochin, n-a găsit nimic decât frunze. Căci nu era timpul smochinelor.
\par 14 Și, vorbind, i-a zis: De acum înainte, rod din tine nimeni în veac să nu mănânce. Și ucenicii Lui ascultau.
\par 15 Și au venit în Ierusalim. Și, intrând în templu, a început să dea afară pe cei ce vindeau și pe cei ce cumpărau în templu, iar mesele schimbătorilor de bani și scaunele vânzătorilor de porumbei le-a răsturnat.
\par 16 Și nu îngăduia să mai treacă nimeni cu vreun vas prin templu.
\par 17 Și-i învăța și le spunea: Nu este, oare, scris: "Casa Mea casă de rugăciune se va chema, pentru toate neamurile"? Voi însă ați făcut din ea peșteră de tâlhari.
\par 18 Și au auzit arhiereii și cărturarii. Și căutau cum să-L piardă. Căci se temeau de El, pentru că toată mulțimea era uimită de învățătura Lui.
\par 19 Iar când s-a făcut seară, au ieșit afară din cetate.
\par 20 Dimineața, trecând pe acolo, au văzut smochinul uscat din rădăcini.
\par 21 Și Petru, aducându-și aminte, I-a zis: Învățătorule, iată smochinul pe care l-ai blestemat s-a uscat.
\par 22 Și răspunzând, Iisus le-a zis: Aveți credință în Dumnezeu.
\par 23 Adevărat zic vouă că oricine va zice acestui munte: Ridică-te și te aruncă în mare, și nu se va îndoi în inima lui, ci va crede că ceea ce spune se va face, fi-va lui orice va zice.
\par 24 De aceea vă zic vouă: Toate câte cereți, rugându-vă, să credeți că le-ați primit și le veți avea.
\par 25 Iar când stați de vă rugați, iertați orice aveți împotriva cuiva, ca și Tatăl vostru Cel din ceruri să vă ierte vouă greșealele voastre.
\par 26 Că de nu iertați voi, nici Tatăl vostru Cel din ceruri nu vă va ierta vouă greșealele voastre.
\par 27 Și au intrat iarăși în Ierusalim. Și pe când se plimba Iisus prin templu, au venit la El arhiereii, cărturarii și bătrânii.
\par 28 Și I-au zis: Cu ce putere faci acestea? Sau cine Ți-a dat Ție puterea aceasta, ca să le faci?
\par 29 Iar Iisus le-a zis: Vă voi întreba și Eu un cuvânt: răspundeți-Mi și vă voi spune și Eu cu ce putere fac acestea:
\par 30 Botezul lui Ioan din cer a fost, sau de la oameni? Răspundeți-Mi!
\par 31 Și ei vorbeau între ei, zicând: De vom zice: Din cer, va zice: Pentru ce, dar, n-ați crezut în el?
\par 32 Iar de vom zice: De la oameni - se temeau de mulțime, căci toți îl socoteau că Ioan era într-adevăr prooroc.
\par 33 Și răspunzând, au zis lui Iisus: Nu știm. Și Iisus le-a zis: Nici Eu nu vă spun vouă cu ce putere fac acestea.

\chapter{12}

\par 1 Și a început să le vorbească în pilde: Un om a sădit o vie, a împrejmuit-o cu gard, a săpat în ea teasc, a clădit turn și a dat-o lucrătorilor, iar el s-a dus departe.
\par 2 Și la vreme, a trimis la lucrători o slugă, ca să ia de la ei din roadele viei.
\par 3 Dar ei, punând mâna pe ea, au bătut-o și i-au dat drumul fără nimic.
\par 4 Și a trimis la ei, iarăși, altă slugă, dar și pe aceea, lovind-o cu pietre, i-au spart capul și au ocărât-o.
\par 5 Și a trimis alta. Dar și pe aceea au ucis-o; și pe multe altele: pe unele bătându-le, iar pe altele ucigându-le.
\par 6 Mai avea și un fiu iubit al său și în cele din urmă l-a trimis la lucrători, zicând: Se vor rușina de fiul meu.
\par 7 Dar acei lucrători au zis între ei: Acesta este moștenitorul; veniți să-l omorâm și moștenirea va fi a noastră.
\par 8 Și prinzându-l l-au omorât și l-au aruncat afară din vie.
\par 9 Ce va face acum stăpânul viei? Va veni și va pierde pe lucrători, iar via o va da altora.
\par 10 Oare nici Scriptura aceasta n-ați citit-o: "Piatra pe care au nesocotit-o ziditorii, aceasta a ajuns să fie în capul unghiului?
\par 11 De la Domnul s-a făcut aceasta și este lucru minunat în ochii noștri".
\par 12 Și căutau să-L prindă, dar se temeau de popor. Căci înțeleseseră că împotriva lor zisese pilda aceasta. Și lăsându-L, s-au dus.
\par 13 Și au trimis la El pe unii din farisei și din irodiani, ca să-L prindă în cuvânt.
\par 14 Iar ei, venind, I-au zis: Învățătorule, știm că spui adevărul și nu-Ți pasă de nimeni, fiindcă nu cauți la fața oamenilor, ci cu adevărat înveți calea lui Dumnezeu. Se cuvine a da dajdie Cezarului sau nu? Să dăm sau să nu dăm?
\par 15 El însă, cunoscând fățărnicia lor, le-a zis: Pentru ce Mă ispitiți? Aduceți-Mi un dinar ca să-l văd.
\par 16 Și I-au adus. Și i-a întrebat Iisus: Al cui e chipul acesta în inscripția de pe el? Iar ei I-au zis: Ale Cezarului.
\par 17 Iar Iisus a zis: Dați Cezarului cele ale Cezarului, iar lui Dumnezeu cele ale lui Dumnezeu. Și se mirau de El.
\par 18 Și au venit la El saducheii care zic că nu este înviere și-L întrebau zicând:
\par 19 Învățătorule, Moise ne-a lăsat scris, că de va muri fratele cuiva și va lăsa femeia fără copil, să ia fratele său pe femeia lui și să ridice urmaș fratelui.
\par 20 Și erau șapte frați. Și cel dintâi și-a luat femeie, dar, murind, n-a lăsat urmaș.
\par 21 Și a luat-o pe ea al doilea, și a murit, nelăsând urmaș. Tot așa și al treilea.
\par 22 Și au luat-o toți șapte și n-au lăsat urmaș. În urma tuturor a murit și femeia.
\par 23 La înviere, când vor învia, a căruia dintre ei va fi femeia? Căci toți șapte au avut-o de soție.
\par 24 Și le-a zis Iisus: Oare nu pentru aceasta rătăciți, neștiind Scripturile, nici puterea lui Dumnezeu?
\par 25 Căci, când vor învia din morți, nici nu se mai însoară, nici nu se mai mărită, ci sunt ca îngerii din ceruri.
\par 26 Iar despre morți că vor învia, n-ați citit, oare, în cartea lui Moise, când i-a vorbit Dumnezeu din rug, zicând: "Eu sunt Dumnezeul lui Avraam și Dumnezeul lui Isaac și Dumnezeul lui Iacov"?
\par 27 Dumnezeu nu este Dumnezeul celor morți, ci a celor vii. Mult rătăciți.
\par 28 Și apropiindu-se unul din cărturari, care îi auzise vorbind între ei și, văzând că bine le-a răspuns, L-a întrebat: Care poruncă este întâia dintre toate?
\par 29 Iisus i-a răspuns că întâia este: "Ascultă Israele, Domnul Dumnezeul nostru este singurul Domn".
\par 30 Și: "Să iubești pe Domnul Dumnezeul tău din toată inima ta, din tot sufletul tău, din tot cugetul tău și din toată puterea ta". Aceasta este cea dintâi poruncă.
\par 31 Iar a doua e aceasta: "Să iubești pe aproapele tău ca pe tine însuți". Mai mare decât acestea nu este altă poruncă.
\par 32 Și I-a zis cărturarul: Bine, Învățătorule. Adevărat ai zis că unul este Dumnezeu și nu este altul afară de El.
\par 33 Și a-L iubi pe El din toată inima, din tot sufletul, din tot cugetul și din toată puterea și a iubi pe aproapele tău ca pe tine însuți este mai mult decât toate arderile de tot și decât toate jertfele.
\par 34 Iar Iisus, văzându-l că a răspuns cu înțelepciune, i-a zis: Nu ești departe de împărăția lui Dumnezeu. Și nimeni nu mai îndrăznea să-L mai întrebe.
\par 35 Și învățând Iisus în templu, grăia zicând: Cum zic cărturarii că Hristos este Fiul lui David?
\par 36 Însuși David a zis întru Duhul Sfânt: "Zis-a Domnul Domnului meu: Șezi de-a dreapta Mea până ce voi pune pe vrăjmașii tăi așternut picioarelor Tale".
\par 37 Deci însuși David Îl numește pe El Domn; de unde dar este fiul lui? Și mulțimea cea multă Îl asculta cu bucurie.
\par 38 Și le zicea în învățătura Sa: Luați seama la cărturari cărora le place să se plimbe în haine lungi și să li se plece lumea în piețe,
\par 39 Și să stea în băncile dintâi în sinagogi și să stea în capul mesei la ospețe,
\par 40 Ei, care secătuiesc casele văduvelor și de ochii lumii se roagă îndelung, își vor lua mai multă osândă.
\par 41 Și șezând în preajma cutiei darurilor, Iisus privea cum mulțimea aruncă bani în cutie. Și mulți bogați aruncau mult.
\par 42 Și venind o văduvă săracă, a aruncat doi bani, adică un codrant.
\par 43 Și chemând la Sine pe ucenicii Săi le-a zis: Adevărat grăiesc vouă că această văduvă săracă a aruncat în cutia darurilor mai mult decât toți ceilalți.
\par 44 Pentru că toți au aruncat din prisosul lor, pe când ea, din sărăcia ei, a aruncat tot ce avea, toată avuția sa.

\chapter{13}

\par 1 Și ieșind din templu, unul dintre ucenicii Săi I-a zis: Învățătorule, privește ce fel de pietre și ce clădiri!
\par 2 Dar Iisus a zis: Vezi aceste mari clădiri? Nu va rămâne piatră peste piatră să nu se risipească.
\par 3 Și șezând pe Muntele Măslinilor, în fața templului, Îl întrebau, de o parte, Petru, Iacov, Ioan și cu Andrei:
\par 4 Spune-ne nouă când vor fi acestea? Și care va fi semnul când va fi să se împlinească toate acestea?
\par 5 Iar Iisus a început să le spună: Vedeți să nu vă înșele cineva.
\par 6 Căci mulți vor veni în numele Meu, zicând că sunt Eu, și vor amăgi pe mulți.
\par 7 Iar când veți auzi de războaie, și de zvonuri de războaie, să nu vă tulburați, căci trebuie să fie, dar încă nu va fi sfârșitul.
\par 8 Și se va ridica neam peste neam și împărăție peste împărăție, vor fi cutremure pe alocuri și foamete și tulburări vor fi. Iar acestea sunt începutul durerilor.
\par 9 Luați seama la voi înșivă. Că vă vor da în adunări și veți fi bătuți în sinagogi și veți sta înaintea conducătorilor și a regilor, pentru Mine, spre mărturie lor.
\par 10 Ci mai întâi Evanghelia trebuie să se propovăduiască la toate neamurile.
\par 11 Iar când vă vor duce ca să vă predea, nu vă îngrijiți dinainte ce veți vorbi, ci să vorbiți ceea ce se va da vouă în ceasul acela. Căci nu voi sunteți cei care veți vorbi, ci Duhul Sfânt.
\par 12 Și va da frate pe frate la moarte și tată pe copil și copiii se vor răzvrăti împotriva părinților și îi vor ucide.
\par 13 Și veți fi urâți de toți pentru numele Meu; iar cel ce va răbda până la urmă, acela se va mântui.
\par 14 Iar când veți vedea urâciunea pustiirii, stând unde nu se cuvine - cine citește să înțeleagă - atunci cei ce vor fi în Iudeea să fugă în munți,
\par 15 Și cel de pe acoperiș să nu se coboare în casă, nici să intre ca să-și ia ceva din casa sa,
\par 16 Și cel ce va fi în țarină să nu se întoarcă îndărăt, ca să-și ia haina.
\par 17 Dar vai celor ce vor avea în pântece și celor ce vor alăpta în zilele acelea!
\par 18 Rugați-vă, dar, ca să nu fie fuga voastră iarna.
\par 19 Căci în zilele acelea va fi necaz cum nu a mai fost până acum, de la începutul făpturii, pe care a zidit-o Dumnezeu, și nici nu va mai fi.
\par 20 Și de nu ar fi scurtat Domnul zilele acelea, n-ar scăpa nici un trup, dar pentru cei aleși, pe care i-a ales, a scurtat acele zile.
\par 21 Și atunci dacă vă va zice cineva: Iată, aci este Hristos, sau iată acolo, să nu credeți.
\par 22 Se vor scula hristoși mincinoși și prooroci mincinoși și vor face semne și minuni, ca să ducă în rătăcire, de se poate, pe cei aleși.
\par 23 Dar voi luați seama. Iată dinainte v-am spus vouă toate.
\par 24 Ci în acele zile, după necazul acela, soarele se va întuneca și luna nu-și va mai da lumina ei.
\par 25 Și stelele vor cădea din cer și puterile care sunt în ceruri se vor clătina.
\par 26 Atunci vor vedea pe Fiul Omului venind pe nori, cu putere multă și cu slavă.
\par 27 Și atunci El va trimite pe îngeri și va aduna pe aleșii Săi din cele patru vânturi, de la marginea pământului până la marginea cerului.
\par 28 Învățați de la smochin pilda: Când mlădița lui se face fragedă și înfrunzește, cunoașteți că vara este aproape.
\par 29 Tot așa și voi, când veți vedea împlinindu-se aceste lucruri, să știți că El este aproape, lângă uși.
\par 30 Adevărat grăiesc vouă că nu va trece neamul acesta până ce nu vor fi toate acestea.
\par 31 Cerul și pământul vor trece, dar cuvintele Mele nu vor trece.
\par 32 Iar despre ziua aceea și despre ceasul acela nimeni nu știe, nici îngerii din cer, nici Fiul, ci numai Tatăl.
\par 33 Luați aminte, privegheați și vă rugați, că nu știți când va fi acea vreme.
\par 34 Este ca un om care a plecat în altă țară și, lăsându-și casa, a dat puterea în mâna slugilor, dând fiecăruia lucrul lui, iar portarului i-a poruncit să vegheze.
\par 35 Vegheați, dar, că nu știți când va veni stăpânul casei: sau seara, sau la miezul nopții, sau la cântatul cocoșilor, sau dimineața.
\par 36 Ca nu cumva venind fără veste, să vă afle pe voi dormind.
\par 37 Iar ceea ce zic vouă, zic tuturor: Privegheați!

\chapter{14}

\par 1 Și după două zile erau Paștile și Azimile. Și arhiereii și cărturarii căutau cum să-l prindă cu vicleșug, ca să-L omoare.
\par 2 Dar ziceau: Nu la sărbătoare, ca să nu fie tulburare în popor.
\par 3 Și fiind El în Betania, în casa lui Simon Leprosul, și șezând la masă, a venit o femeie având un alabastru, cu mir de nard curat, de mare preț, și, spărgând vasul, a vărsat mirul pe capul lui Iisus.
\par 4 Dar erau unii mâhniți între ei, zicând: Pentru ce s-a făcut această risipă de mir?
\par 5 Căci putea să se vândă acest mir cu peste trei sute de dinari, și să se dea săracilor. Și cârteau împotriva ei.
\par 6 Dar Iisus a zis: Lăsați-o. De ce îi faceți supărare? Lucru bun a făcut ea cu Mine.
\par 7 Că pe săraci totdeauna îi aveți cu voi și, oricând voiți, puteți să le faceți bine, dar pe mine nu Mă aveți totdeauna.
\par 8 Ea a făcut ceea ce avea de făcut: mai dinainte a uns trupul Meu, spre înmormântare.
\par 9 Adevărat zic vouă: Oriunde se va propovădui Evanghelia, în toată lumea, se va spune și ce-a făcut aceasta, spre pomenirea ei.
\par 10 Iar Iuda Iscarioteanul, unul din cei doisprezece, s-a dus la arhierei ca să li-L dea pe Iisus.
\par 11 Și, auzind ei, s-au bucurat și au făgăduit să-i dea bani. Și el căuta cum să-L dea lor, la timp potrivit.
\par 12 Iar în ziua cea dintâi a Azimilor, când jertfeau Paștile, ucenicii Lui L-au întrebat: Unde voiești să gătim, ca să mănânci Paștile?
\par 13 Și a trimis doi din ucenicii Lui, zicându-le: Mergeți în cetate și vă va întâmpina un om, ducând un urcior cu apă; mergeți după el.
\par 14 Și unde va intra, spuneți stăpânului casei că Învățătorul zice: Unde este odaia în care să mănânc Paștile împreună cu ucenicii Mei?
\par 15 Iar el vă va arăta un foișor mare așternut gata. Acolo să pregătiți pentru noi.
\par 16 Și au ieșit ucenicii și au venit în cetate și au găsit așa precum le-a spus și au pregătit Paștile.
\par 17 Iar făcându-se seară, a venit cu cei doisprezece.
\par 18 Pe când ședeau la masă și mâncau, Iisus a zis: Adevărat grăiesc vouă că unul dintre voi, care mănâncă împreună cu Mine, Mă va vinde.
\par 19 Ei au început să se întristeze și să-I zică, unul câte unul: Nu cumva sunt eu?
\par 20 Iar El le-a zis: Unul dintre cei doisprezece, care întinge cu Mine în blid.
\par 21 Că Fiul Omului merge precum este scris despre El; dar vai de omul acela prin care este vândut Fiul Omului. Bine era de omul acela dacă nu s-ar fi născut.
\par 22 Și, mâncând ei, a luat Iisus pâine și binecuvântând, a frânt și le-a dat lor și a zis: Luați, mâncați, acesta este Trupul Meu.
\par 23 Și luând paharul, mulțumind, le-a dat și au băut din el toți.
\par 24 Și a zis lor: Acesta este Sângele Meu, al Legii celei noi, care pentru mulți se varsă.
\par 25 Adevărat grăiesc vouă că de acum nu voi mai bea din rodul viței până în ziua aceea când îl voi bea nou în împărăția lui Dumnezeu.
\par 26 Și după ce au cântat cântări de laudă, au ieșit la Muntele Măslinilor.
\par 27 Și le-a zis Iisus: Toți vă veți sminti, că scris este: "Bate-voi păstorul și se vor risipi oile".
\par 28 Dar după învierea Mea, voi merge mai înainte de voi în Galileea.
\par 29 Iar Petru I-a zis: Chiar dacă toți se vor sminti întru Tine, totuși eu nu.
\par 30 Și i-a zis Iisus: Adevărat grăiesc ție: Că tu astăzi, în noaptea aceasta, mai înainte de a cânta de două ori cocoșul, de trei ori te vei lepăda de Mine.
\par 31 El însă spunea mai stăruitor: Și de-ar fi să mor cu Tine, nu Te voi tăgădui. Și tot așa ziceau toți.
\par 32 Și au venit la un loc al cărui nume este Ghetsimani, și acolo a zis către ucenicii Săi: Ședeți aici până ce Mă voi ruga.
\par 33 Și a luat cu El pe Petru și pe Iacov și pe Ioan și a început a Se tulbura și a Se mâhni.
\par 34 Și le-a zis lor: Întristat este sufletul Meu până la moarte. Rămâneți aici și privegeheați.
\par 35 Și mergând puțin mai înainte, a căzut cu fața la pământ și Se ruga, ca, de este cu putință, să treacă de la El ceasul (acesta).
\par 36 Și zicea: Avva Părinte, toate sunt Ție cu putință. Depărtează paharul acesta de la Mine. Dar nu ce voiesc Eu, ci ceea ce voiești Tu.
\par 37 Și a venit și i-a găsit dormind și a zis lui Petru: Simone, dormi? N-ai avut tărie ca să veghezi un ceas?
\par 38 Privegheați și vă rugați, ca să nu intrați în ispită. Căci duhul este osârduitor, dar trupul neputincios.
\par 39 Și iarăși mergând, s-a rugat, același cuvânt zicând.
\par 40 Și iarăși venind, i-a găsit dormind, căci ochii lor erau îngreuiați și nu știau ce să-I răspundă.
\par 41 Și a venit a treia oară și le-a zis: Dormiți de acum și vă odihniți! E gata! A sosit ceasul. Iată Fiul Omului este dat în mâinile păcătoșilor.
\par 42 Sculați-vă să mergem. Iată, cel ce M-a vândut s-a apropiat.
\par 43 Și îndată, încă vorbind El, a venit Iuda Iscarioteanul, unul din cei doisprezece, și cu el mulțime cu săbii și cu ciomege, de la arhierei, de la cărturari și de la bătrâni.
\par 44 Iar vânzătorul le dăduse semn, zicând: Pe care-L voi săruta, Acela este. Prindeți-L și duceți-L cu pază.
\par 45 Și venind îndată și apropiindu-se de El, a zis Lui: Învățătorule! Și L-a sărutat.
\par 46 Iar ei au pus mâna pe El și L-au prins.
\par 47 Unul din cei ce stăteau pe lângă El, scoțând sabia, a lovit pe sluga arhiereului și i-a tăiat urechea.
\par 48 Și răspunzând, Iisus le-a zis: Ca la un tâlhar ați ieșit cu săbii și cu toiege, ca să Mă prindeți.
\par 49 În fiecare zi eram la voi în templu, învățând, și nu M-ați prins. Dar acestea sunt ca să se împlinească Scripturile.
\par 50 Și, lăsându-L, au fugit toți.
\par 51 Iar un tânăr mergea după El, înfășurat într-o pânzătură, pe trupul gol, și au pus mâna pe el.
\par 52 El însă, smulgându-se din pânzătură, a fugit gol.
\par 53 Și au dus pe Iisus la arhiereu și s-au adunat acolo toți arhiereii și bătrânii și cărturarii.
\par 54 Iar Petru, de departe, a mers după El, până a intrat înăuntru în curtea arhiereului și ședea împreună cu slugile, încălzindu-se la foc.
\par 55 Arhiereii și tot sinedriul căutau împotriva lui Iisus mărturie ca să-L dea la moarte, dar nu găseau.
\par 56 Că mulți mărturiseau mincinos împotriva Lui, dar mărturiile nu se potriveau.
\par 57 Și ridicându-se unii, au dat mărturie mincinoasă împotriva Lui, zicând:
\par 58 Noi L-am auzit zicând: Voi dărâma acest templu făcut de mână, și în trei zile altul, nefăcut de mână, voi clădi.
\par 59 Dar nici așa mărturia lor nu era la fel.
\par 60 Și, sculându-se în mijlocul lor, arhiereul L-a întrebat pe Iisus, zicând: Nu răspunzi nimic la tot ce mărturisesc împotriva Ta aceștia?
\par 61 Iar El tăcea și nu răspundea nimic. Iarăși L-a întrebat arhiereul și I-a zis: Ești tu Hristosul, Fiul Celui binecuvântat?
\par 62 Iar Iisus a zis: Eu sunt și veți vedea pe Fiul Omului șezând de-a dreapta Celui Atotputernic și venind pe norii cerului.
\par 63 Iar arhiereul, sfâșiindu-și hainele, a zis: Ce trebuință mai avem de martori?
\par 64 Ați auzit hula. Ce vi se pare vouă? Iar ei toți au judecat că El este vinovat de moarte.
\par 65 Și unii au început să-L scuipe și să-I acopere fața și să-L bată cu pumnii și să-I zică: Proorocește! Și slugile Îl băteau cu palmele.
\par 66 Și Petru fiind jos în curte, a venit una din slujnicele arhiereului,
\par 67 Și văzându-l pe Petru, încălzindu-se, s-a uitat la el și a zis: Și tu erai cu Iisus Nazarineanul.
\par 68 El însă a tăgăduit, zicând: Nici nu știu, nici nu înțeleg ce zici. Și a ieșit afară înaintea curții; și a cântat cocoșul.
\par 69 Iar slujnica, văzându-l, a început iarăși să spună celor de față că acesta este dintre ei.
\par 70 Iar el a tăgăduit iarăși. Și după puțin timp, cei de față ziceau iarăși lui Petru: Cu adevărat ești dintre ei, căci ești și galileian și vorbirea ta se aseamănă.
\par 71 Iar el a început să se blesteme și să se jure: Nu știu pe omul acesta despre care ziceți.
\par 72 Și îndată cocoșul a cântat a doua oară. Și Petru și-a adus aminte de cuvântul pe care i-l spusese Iisus: Înainte de a cânta de două ori cocoșul, de trei ori te vei lepăda de Mine. Și a început să plângă.

\chapter{15}

\par 1 Și îndată dimineața, arhiereii, ținând sfat cu bătrânii, cu cărturarii și cu tot sinedriul și legând pe Iisus, L-au dus și L-au predat lui Pilat.
\par 2 Și L-a întrebat Pilat: Tu ești regele iudeilor? Iar El, răspunzând, i-a zis: Tu zici.
\par 3 Iar arhiereii Îl învinuiau de multe.
\par 4 Iar Pilat L-a întrebat: Nu răspunzi nimic? Iată câte spun împotriva Ta.
\par 5 Dar Iisus nimic n-a mai răspuns, încât Pilat se mira.
\par 6 Iar la sărbătoarea Paștilor, le elibera un întemnițat pe care-l cereau ei.
\par 7 Și era unul cu numele Baraba închis împreună cu niște răzvrătiți, care în răscoală săvârșiseră ucidere.
\par 8 Și mulțimea, venind sus, a început să ceară lui Pilat să le facă precum obișnuia pentru ei.
\par 9 Iar Pilat le-a răspuns, zicând: Voiți să vă eliberez pe regele iudeilor?
\par 10 Fiindcă știa că arhiereii Îl dăduseră în mâna lui din invidie.
\par 11 Dar arhiereii au ațâțat mulțimea ca să le elibereze mai degrabă pe Baraba.
\par 12 Iar Pilat, răspunzând iarăși, le-a zis: Ce voi face deci cu cel despre care ziceți că este regele iudeilor?
\par 13 Ei iarăși au strigat: Răstignește-L!
\par 14 Iar Pilat le-a zis: Dar ce rău a făcut? Iar ei mai mult strigau: Răstignește-L!
\par 15 Și Pilat, vrând să facă pe voia mulțimii, le-a eliberat pe Baraba, iar pe Iisus, biciuindu-L, L-a dat ca să fie răstignit.
\par 16 Iar ostașii L-au dus înăuntrul curții, adică în pretoriu, și au adunat toată cohorta.
\par 17 Și L-au îmbrăcat în purpură și, împletindu-I o cunună de spini, I-au pus-o pe cap.
\par 18 Și au început să se plece în fața Lui, zicând: Bucură-Te regele iudeilor!
\par 19 Și-L băteau peste cap cu o trestie și-L scuipau și, căzând în genunchi, I se închinau.
\par 20 Și după ce L-au batjocorit, L-au dezbrăcat de purpură și L-au îmbrăcat cu hainele Lui. Și L-au dus afară ca să-L răstignească.
\par 21 Și au silit pe un trecător, care venea din țarină, pe Simon Cirineul, tatăl lui Alexandru și al lui Ruf, ca să ducă crucea Lui.
\par 22 Și L-au dus la locul zis Golgota, care se tălmăcește "locul Căpățânii".
\par 23 Și I-au dat să bea vin amestecat cu smirnă, dar El n-a luat.
\par 24 Și L-au răstignit și au împărțit între ei hainele Lui, aruncând sorți pentru ele, care ce să ia.
\par 25 Iar când L-au răstignit, era ceasul al treilea.
\par 26 Și vina Lui era scrisă deasupra: Regele iudeilor.
\par 27 Și împreună cu El au răstignit doi tâlhari: unul de-a dreapta și altul de-a stânga Lui.
\par 28 Și s-a împlinit Scriptura care zice: Cu cei fără de lege a fost socotit.
\par 29 Iar cei ce treceau pe acolo Îl huleau, clătinându-și capetele și zicând: Huu! Cel care dărâmi templul și în trei zile îl zidești.
\par 30 Mântuiește-Te pe Tine Însuți, coborându-Te de pe cruce!
\par 31 De asemenea și arhiereii, batjocorindu-L între ei, împreună cu cărturarii, ziceau: Pe alții a mântuit, dar pe Sine nu poate să Se mântuiască!
\par 32 Hristos, regele lui Israel, să Se coboare de pe cruce, ca să vedem și să credem. Și cei împreună răstigniți cu El Îl ocărau.
\par 33 Iar când a fost ceasul al șaselea, întuneric s-a făcut peste tot pământul până la ceasul al nouălea.
\par 34 Și la al nouălea ceas, a strigat Iisus cu glas mare: Eloi, Eloi, lama sabahtani?, care se tălmăcește: Dumnezeul Meu, Dumnezeul Meu, de ce M-ai părăsit?
\par 35 Iar unii din cei ce stăteau acolo, auzind, ziceau: Iată, îl strigă pe Ilie.
\par 36 Și, alergând, unul a înmuiat un burete în oțet, l-a pus într-o trestie și I-a dat să bea, zicând: Lăsați să vedem dacă vine Ilie ca să-L coboare.
\par 37 Iar Iisus, scoțând un strigăt mare, Și-a dat duhul.
\par 38 Și catapeteasma templului s-a rupt în două, de sus până jos.
\par 39 Iar sutașul care stătea în fața Lui, văzând că astfel Și-a dat duhul, a zis: Cu adevărat omul acesta era Fiul lui Dumnezeu!
\par 40 Și erau și femei care priveau de departe; între ele: Maria Magdalena, Maria, mama lui Iacov cel Mic și a lui Iosi, și Salomeea,
\par 41 Care, pe când era El în Galileea, mergeau după El și Îi slujeau, și multe altele care se suiseră cu El la Ierusalim.
\par 42 Și făcându-se seară, fiindcă era vineri, care este înaintea sâmbetei,
\par 43 Și venind Iosif cel din Arimateea, sfetnic ales, care aștepta și el împărăția lui Dumnezeu, și, îndrăznind, a intrat la Pilat și a cerut trupul lui Iisus.
\par 44 Iar Pilat s-a mirat că a și murit și, chemând pe sutaș, l-a întrebat dacă a murit de mult.
\par 45 Și aflând de la sutaș, a dăruit lui Iosif trupul.
\par 46 Și Iosif, cumpărând giulgiu și coborându-L de pe cruce, L-a înfășurat în giulgiu și L-a pus într-un mormânt care era săpat în stâncă, și a prăvălit o piatră la ușa mormântului.
\par 47 Iar Maria Magdalena și Maria, mama lui Iosi, priveau unde L-au pus.

\chapter{16}

\par 1 Și după ce a trecut ziua sâmbetei, Maria Magdalena, Maria, mama lui Iacov, și Salomeea au cumpărat miresme, ca să vină să-L ungă.
\par 2 Și dis-de-dimineață, în prima zi a săptămânii (Duminică), pe când răsărea soarele, au venit la mormânt.
\par 3 Și ziceau între ele: Cine ne va prăvăli nouă piatra de la ușa mormântului?
\par 4 Dar, ridicându-și ochii, au văzut că piatra fusese răsturnată; căci era foarte mare.
\par 5 Și, intrând în mormânt, au văzut un tânăr șezând în partea dreaptă, îmbrăcat în veșmânt alb, și s-au spăimântat.
\par 6 Iar el le-a zis: Nu vă înspăimântați! Căutați pe Iisus Nazarineanul, Cel răstignit? A înviat! Nu este aici. Iată locul unde L-au pus.
\par 7 Dar mergeți și spuneți ucenicilor Lui și lui Petru că va merge în Galileea, mai înainte de voi; acolo îl veți vedea, după cum v-a spus.
\par 8 Și ieșind, au fugit de la mormânt, că erau cuprinse de frică și de uimire, și nimănui nimic n-au spus, căci se temeau.
\par 9 Și înviind dimineața, în ziua cea dintâi a săptămânii (Duminică) El s-a arătat întâi Mariei Magdalena, din care scosese șapte demoni.
\par 10 Aceea, mergând, a vestit pe cei ce fuseseră cu El și care se tânguiau și plângeau.
\par 11 Și ei, auzind că este viu și că a fost văzut de ea, n-au crezut.
\par 12 După aceea, S-a arătat în alt chip, la doi dintre ei, care mergeau la o țarină.
\par 13 Și aceia, mergând, au vestit celorlalți, dar nici pe ei nu i-au crezut.
\par 14 La urmă, pe când cei unsprezece ședeau la masă, li S-a arătat și I-a mustrat pentru necredința și împietrirea inimii lor, căci n-au crezut pe cei ce-L văzuseră înviat.
\par 15 Și le-a zis: Mergeți în toată lumea și propovăduiți Evanghelia la toată făptura.
\par 16 Cel ce va crede și se va boteza se va mântui; iar cel ce nu va crede se va osândi.
\par 17 Iar celor ce vor crede, le vor urma aceste semne: în numele Meu, demoni vor izgoni, în limbi noi vor grăi,
\par 18 Șerpi vor lua în mână și chiar ceva dătător de moarte de vor bea nu-i va vătăma, peste cei bolnavi își vor pune mâinile și se vor face sănătoși.
\par 19 Deci Domnul Iisus, după ce a vorbit cu ei, S-a înălțat la cer și a șezut de-a dreapta lui Dumnezeu.
\par 20 Iar ei, plecând, au propovăduit pretutindeni și Domnul lucra cu ei și întărea cuvântul, prin semnele care urmau. Amin.


\end{document}