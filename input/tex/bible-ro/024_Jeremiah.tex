\begin{document}

\title{Ieremia}


\chapter{1}

\par 1 Cuvintele lui Ieremia, fiul lui Hilchia, dintre preoții de la Anatot, din pământul lui Veniamin,
\par 2 Către care a fost cuvântul Domnului, în zilele lui Iosia, fiul lui Amon, regele lui Iuda, în anul al treisprezecelea al domniei acestuia,
\par 3 Precum și în zilele lui Ioiachim, fiul lui Iosia, regele lui Iuda, până în luna a cincea din anul al unsprezecelea al lui Sedechia, fiul lui Iosia, regele lui Iuda, adică până la robirea Ierusalimului.
\par 4 Fost-a cuvântul Domnului către mine și mi-a zis:
\par 5 "Înainte de a te fi zămislit în pântece, te-am cunoscut, și înainte de a ieși din pântece, te-am sfințit și te-am rânduit prooroc pentru popoare".
\par 6 Iar eu am răspuns: "O, Doamne, Dumnezeule, eu nu știu să vorbesc, pentru că sunt încă tânăr".
\par 7 Domnul însă mi-a zis: "Să nu zici: Sunt încă tânăr; căci la câți te voi trimite, la toți vei merge și tot ce-ți voi porunci vei spune.
\par 8 Să nu te temi de dânșii, căci Eu sunt cu tine, ca să te izbăvesc", zice Domnul.
\par 9 Și Domnul mi-a întins mâna, mi-a atins gura și mi-a zis: "Iată am pus cuvintele Mele în gura ta!
\par 10 Iată te-am pus în ziua aceasta peste popoare și peste regate, ca să smulgi și să arunci la pământ, să pierzi și să dărâmi, să zidești și să sădești!"
\par 11 Apoi a fost cuvântul Domnului către mine și mi-a zis: "Ieremia, ce vezi tu?" Și eu am răspuns: "Văd un toiag din lemn de migdal".
\par 12 Zisu-mi-a Domnul: "Tu ai văzut bine, că Eu priveghez asupra cuvântului Meu, ca să-l împlinesc!"
\par 13 Și iar a fost cuvântul Domnului către mine și mi-a zis: "Ce vezi tu?" Și eu am răspuns: "Văd un cazan clocotind, care e gata să se verse dinspre miazănoapte".
\par 14 Iar Domnul mi-a zis: De la miazănoapte se va deschide nenorocire asupra tuturor locuitorilor țării acesteia.
\par 15 Că iată voi chema toate popoarele țărilor de la miazănoapte, zice Domnul, și vor veni acelea și își vor pune fiecare din ele scaunul său la poarta de intrare a Ierusalimului, pe lângă toate zidurile lui și în toate cetățile lui Iuda;
\par 16 Și voi rosti asupra lor judecata Mea pentru toate fărădelegile lor, pentru că M-au părăsit, au aprins tămâie înaintea altor dumnezei și s-au închinat la lucrurile mâinilor lor.
\par 17 Dar tu, încinge-ți coapsele tale și scoală de le spune tot ce-ți voi porunci. Nu tremura înaintea lor, fiindcă nu vreau să tremuri înaintea lor.
\par 18 Că iată, Eu te-am făcut astăzi cetate întărită, stâlp de fier și zid de aramă înaintea acestei țări întregi: înaintea regilor lui Iuda, înaintea căpeteniilor ei, înaintea preoților ei și înaintea poporului țării.
\par 19 Ei se vor lupta împotriva ta, dar nu te vor birui că Eu sunt cu tine, ca să te izbăvesc", zice Domnul.

\chapter{2}

\par 1 Fost-a cuvântul Domnului către mine și a zis:
\par 2 "Mergi și strigă la urechile fiicei Ierusalimului și zi: "Așa grăiește Domnul: Mi-am adus aminte de prietenia cea din tinerețea ta, de iubirea de pe când erai mireasă și mi-ai urmat în pustiu, în pământul cel nesemănat.
\par 3 Atunci Israel era sfințenia Domnului și pârga roadelor lui; toți câți mâncau din ea se făceau vinovați și nenorocirea venea asupra lor", zice Domnul.
\par 4 Casa lui Iacov și toate semințiile casei lui Israel, ascultați cuvântul Domnului! Așa zice Domnul:
\par 5 "Ce nedreptate au găsit în Mine părinții voștri, de s-au depărtat de Mine și s-au dus după deșertăciune și au devenit ei înșiși deșertăciune?
\par 6 În loc să zică: Unde este Domnul, Cel ce ne-a scos din pământul Egiptului și ne-a povățuit prin pustiu, prin pământul cel gol și nelocuit, prin pământul cel sec, prin pământul umbrei morții, prin care nimeni nu umblase și unde nu locuia om?
\par 7 Eu v-am dus în pământ roditor, ca să vă hrăniți cu roadele lui și cu bunătățile lui; voi însă ați intrat și ați spurcat pământul Meu și moștenirea Mea ați făcut-o urâciune.
\par 8 Preoții n-au zis: Unde este Domnul? Învățătorii legii nu M-au cunoscut; păstorii au lepădat credința și proorocii au proorocit în numele lui Baal și s-au dus după cei ce nu-i pot ajuta.
\par 9 De aceea la judecată voi grăi împotriva voastră, zice Domnul, și împotriva nepoților voștri voi cere osândă!
\par 10 Să vă duceți în insulele Chitim și să vedeți; trimiteți în Chedar și cercetați cu de-amănuntul și aflați:
\par 11 Fost-a, oare, acolo ceva de felul acesta? Schimbatu-și-a oare vreun popor dumnezeii săi, deși aceia nu sunt dumnezei? Poporul Meu însă și-a schimbat slava cu ceea ce nu-l poate ajuta.
\par 12 Mirați-vă de acestea, ceruri; cutremurați-vă, îngroziți-vă, zice Domnul!
\par 13 Că două rele a făcut poporul Meu: pe Mine, izvorul apei celei vii, M-au părăsit, și și-au săpat fântâni sparte, care nu pot ține apă.
\par 14 Au doară rob sau fiu de rob e Israel? Pentru ce dar s-a făcut el pradă?
\par 15 Mugit-au asupra lui puii de leu, ridicatu-și-au glasul lor și au făcut pustiu țara lui; cetățile lui sunt fără locuitori.
\par 16 Chiar și locuitorii din Nof și cei din Tahpanhes ți-au ras capul, Israele!
\par 17 Oare nu ți-ai pricinuit tu singur aceasta, părăsind pe Domnul Dumnezeul tău, când te povățuia?
\par 18 Și acum pentru ce ai luat drumul Egiptului, ca să bei apă din Nil? Și pentru ce ți-ai luat drumul spre Asiria, ca să bei apă din râul ei?
\par 19 Lepădarea ta de credință te va pedepsi și răutatea ta te va mustra. Înțelege și vezi cât e de rău și de amar de a părăsi pe Domnul Dumnezeul tău și de a nu mai avea nici o teamă de Mine, zice Domnul Dumnezeul puterilor.
\par 20 Că în vechime am sfărâmat jugul tău și am rupt cătușele tale, și tu ai zis: "Nu voi sluji la idoli, și totuși pe tot dealul înalt și sub tot pomul umbros ai făcut desfrânare.
\par 21 Eu te-am sădit ca pe o viță de soi, ca pe cea mai curată sămânță; cum dar Mi te-ai prefăcut în ramură sălbatică de viță străină?
\par 22 Chiar de te-ai spăla cu silitră și chiar dacă te-ai freca cu leșie, tot pătat ești în nedreptățile tale față de Mine, zice Domnul Dumnezeu.
\par 23 Cum poți tu să zici: Nu m-am întinat și n-am umblat după Baal? Privește la purtarea ta din vale și află ce ai făcut tu, cămilă zburdalnică, tu care cutreieri toate drumurile,
\par 24 Asină sălbatică, deprinsă în pustiu, care în aprinderea poftei ei, soarbe aerul! Cine o va putea împiedica să-și împlinească pofta? Toți cei ce o caută nu se vor osteni, că în luna ei o vor găsi. Când i s-a zis:
\par 25 Întoarce-ți piciorul de la calea strâmbă și gâtul nu-l deprinde a înseta, ea a răspuns: Nu, zadarnic! Căci iubesc dumnezei străini și merg după ei.
\par 26 Cum furul când este prins se rușinează, așa se va rușina și casa lui Israel și poporul și regii lui și căpeteniile lui și proorocii lui și preoții lui, căci au zis lemnului:
\par 27 "Tu ești tatăl meu!" Și pietrei i-au grăit: "Tu m-ai născut!", și nu și-au întors spre Mine fața, ci spatele, iar la vreme de nevoie vor zice: "Scoală și ne izbăvește!"
\par 28 Dar unde-ți sunt, Iudo, dumnezeii care ți i-ai făcut? Să se scoale și să te izbăvească la vreme de necaz, dacă pot! Căci câte cetăți ai, atâția sunt și dumnezeii tăi.
\par 29 Pentru ce vă certați cu Mine? Toți v-ați purtat cu necredincioșie și ați păcătuit împotriva Mea, zice Domnul.
\par 30 În zadar am bătut pe copiii voștri, că n-au primit învățătură; pe proorocii voștri i-a mâncat sabia voastră, ca un leu pierzător, și nu v-ați temut".
\par 31 Ascultă, poporule, cuvântul Domnului care zice: "Pustiu am fost Eu oare pentru Israel? Sau țara întunericului am fost? Pentru ce dar poporul Meu zice: Noi înșine ne suntem stăpâni și nu mai venim la Tine?
\par 32 Au doară uită fata podoaba sa și mireasa găteala sa? Poporul Meu însă M-a uitat de zile nenumărate.
\par 33 Fiica Ierusalimului, cât de iscusită îți ești tu în căile tale ca să cauți iubirea! Ba pentru aceasta și la nelegiuiri ți-ai deprins căile tale,
\par 34 Și chiar și în poalele hainei tale se află sângele săracilor nevinovați, pe care nu i-ai prins spărgând zidul, și totuși zici:
\par 35 "De vreme ce sunt nevinovată, mânia Lui de bună seamă se va abate de la mine". Pentru că zici: "N-am greșit", de aceea iată Eu cu tine mă voi judeca.
\par 36 De ce atâta grabă ca să-ți schimbi calea? Vei fi rușinată de Egipt, cum ai fost rușinată de Asiria.
\par 37 Și de acolo vei ieși cu mâinile pe cap, pentru că a lepădat Domnul pe cei în care tu ți-ai pus nădejdea, pe cei cu care tu nu vei avea izbândă".

\chapter{3}

\par 1 Și a mai spus: "Dacă un bărbat își lasă femeia, și ea se duce de la el și se face soție altui bărbat, mai poate ea oare să se întoarcă la el? Prin aceasta nu s-ar întina în adevăr, oare, țara aceea?" Și tu te-ai desfrânat cu mulți iubiți și vrei să te întorci la Mine? zice Domnul.
\par 2 Ridică-ți ochii spre înălțimi și privește: Unde oare nu s-au desfrânat aceia cu tine? Șezut-ai pentru ei lângă cale, ca arabul în pustiu, și ai spurcat țara cu desfrânarea ta Și cu vicleșugul tău.
\par 3 De aceea ploile de toamnă au lipsit și la fel și cele de primăvară, dar Tu ai avut frunte de desfrânată și nu te-ai rușinat.
\par 4 Și acum strigi către Mine: "Tatăl meu, Tu ai fost povățuitorul tinereților mele!
\par 5 Oare pentru totdeauna va fi El mânios și oare veșnic va păstra în Sine mânia?" Iată ce ai zis, dar de făcut faci rele și sporești în acelea.
\par 6 Zisu-mi-a Domnul în zilele lui Iosia: "Văzut-ai ce-a făcut Israel, această fiică necredincioasă? A umblat pe toți munții înalți și pe sub tot copacul umbros și s-a desfrânat pe acolo.
\par 7 După ce a făcut toate acestea, i-am zis: "Întoarce-te la Mine!" Dar nu s-a întors. Și a văzut acestea Iuda, sora sa cea necredincioasă.
\par 8 Și deși am lăsat pe fiica lui Israel cea necredincioasă pentru atâtea fapte de desfrânare și i-am dat carte de despărțire, am văzut că necredincioasa ei soră, Iuda, nu s-a temut, ci s-a dus și ea să se desfrâneze.
\par 9 Și prin nerușinarea desfrânărilor ei a pângărit țara și s-a desfrânat cu pietrele și cu lemnele.
\par 10 Peste toate acestea Iuda, necredincioasa soră a fiicei lui Israel, nu s-a întors la Mine din toată inima sa, ci numai din prefăcătorie", zice Domnul.
\par 11 Apoi iarăși mi-a zis Domnul: "Fiica lui Israel cea necredincioasă s-a dovedit că e mai dreaptă decât Iuda cea lepădată de Dumnezeu.
\par 12 Mergi de vestește cuvintele acestea la miazănoapte și zi: Întoarce-te, necredincioasă fiică a lui Israel, zice Domnul, că nu voi vărsa asupra voastră mânia Mea, pentru că sunt milostiv și nu Mă voi mânia pe veci, zice Domnul.
\par 13 Recunoaște-ți însă vinovăția ta, căci te-ai abătut de la Domnul Dumnezeul tău și te-ai desfrânat cu dumnezei străini sub tot arborele umbros și glasul Meu nu l-ai ascultat, zice Domnul.
\par 14 Întoarceți-vă, voi copii căzuți de la credință, zice Domnul, că M-am unit cu voi și vă voi lua câte unul de cetate și câte doi de seminție și vă voi aduce în Sion.
\par 15 Apoi vă voi da păstori după inima Mea, care vă vor păstori cu știință și pricepere.
\par 16 Când vă veți înmulți și veți ajunge mult roditori pe pământ, în zilele acelea, zice Domnul, nu se va mai zice: "O, chivotul așezământului Domnului". Nimeni nu se va mai gândi la el, nimeni nu-și va mai aduce aminte de el, nimănui nu-i va mai părea rău de el, nimeni nu va mai face altul.
\par 17 În vremea aceea Ierusalimul se va numi tronul Domnului și toate popoarele se vor aduna acolo pentru numele Domnului și nu se vor mai purta după îndărătnicia inimii lor celei rele.
\par 18 În zilele acelea va veni casa lui Iuda la casa lui Israel și se vor duce împreună din țara de la miazănoapte, în țara pe care am dat-o Eu moștenire părinților voștri.
\par 19 Eu Mi-am zis: Cum să te pun pe tine în numărul fiilor și să-ți dau țara cea plăcută, care este moștenirea cea mai frumoasă a mulțimii poporului? Dar iarăși Mi-am zis: Tu Mă vei numi Tată al tău și nu te vei mai depărta de Mine.
\par 20 Însă tocmai cum femeia necredincioasă înșală pe iubitul său, așa și voi, casa lui Israel, v-ați purtat cu înșelăciune față de Mine, zice Domnul.
\par 21 Glas se aude pe înălțimi, s-aude plânsul jalnic al fiilor lui Israel, care se tânguiesc pentru că și-au stricat căile lor și au uitat pe Domnul Dumnezeul lor.
\par 22 Întoarceți-vă, copii răzvrătiți, și Eu voi vindeca neascultarea voastră! Ziceți: Iată venim la Tine, că Tu ești Domnul Dumnezeul nostru.
\par 23 Cu adevărat în deșert ne-am pus nădejdea în dealuri și în mulțimea munților; cu adevărat, în Domnul Dumnezeul nostru este mântuirea lui Israel.
\par 24 Din tinerețea noastră această urâciune a mâncat ostenelile părinților noștri: oile lor, boii lor, fiii lor și fiicele lor.
\par 25 Iar noi zăcem în rușinea noastră și ocara noastră ne acoperă, pentru că am păcătuit înaintea Domnului Dumnezeului nostru, și noi și părinții noștri din tinerețea noastră și până în ziua de astăzi și n-am ascultat glasul Domnului Dumnezeului nostru".

\chapter{4}

\par 1 De vrei să te întorci, Israele, zice Domnul, întoarce-te la Mine și, de vei depărta urâciunile de la fața Mea, nu vei mai rătăci.
\par 2 Dacă tu vei jura: "Viu este Domnul!", în adevăr, în judecată și în dreptate, neamurile se vor binecuvânta și se vor lăuda în El.
\par 3 Căci așa zice Domnul către bărbăția lui Iuda și ai Ierusalimului: "Arați-vă ogoare noi și nu mai semănați prin spini!
\par 4 Bărbați ai lui Iuda și locuitori ai Ierusalimului, tăiați-vă împrejur pentru Domnul și lepădați învârtoșarea inimii voastre, ca nu cumva să izbucnească mânia Mea ca focul și să ardă nestinsă din pricina răutății faptelor voastre.
\par 5 Spuneți acestea în Iuda și le vestiți în Ierusalim! Grăiți și trâmbițați cu trâmbița prin țară! Strigați tare și ziceți:
\par 6 "Adunați-vă și să intrăm în cetatea cea întărită!" Înălțați steagul spre Sion, fugiți și nu vă opriți, că iată aduc de la miazănoapte necaz și nevoie mare!
\par 7 Iată, iese leul din desișul său și pierzătorul popoarelor se apropie; plecat-a din locul său, ca să pustiiască pământul tău; cetățile tale vor fi stricate și fără locuitori.
\par 8 De aceea încingeți-vă cu sac, plângeți și vă tânguiți, că iuțimea mâniei Domnului nu se va abate de la voi.
\par 9 În ziua aceea, zice Domnul, va Încremeni inima regelui și inima căpeteniilor; preoții se vor îngrozi și proorocii se vor mira".
\par 10 Atunci eu am zis: "O, Doamne Dumnezeule, amăgit-ai Tu oare pe poporul acesta și Ierusalimul, când ai zis: Veți avea pace, și iată sabia a ajuns până la suflet?"
\par 11 În vremea aceea se va zice poporului acestuia și Ierusalimului: "Iată vine vânt arzător din munții cei pustii asupra fiicei poporului Meu și vine nu pentru a vântura, nici pentru a curăți grâul;
\par 12 Dar va veni dintr-acolo de la Mine vânt mai puternic decât acesta și voi rosti judecata Mea asupra lor.
\par 13 Iată, se va ridica, cum se ridică norii; căruțele lui vor fi ca furtuna și caii lui mai iuți decât vulturii". Vai de noi, căci vom fi prăpădiți!
\par 14 Ierusalime, spală răul din inima ta, ca să te izbăvești! Până când se vor sălășlui în tine cugete necredincioase?
\par 15 Că iată se aude glas de la Dan și vestea pieirii din munții lui Efraim:
\par 16 "Spuneți popoarelor și vestiți Ierusalimului că din țară depărtată vin împresurători și scot strigăte împotriva cetăților lui Iuda".
\par 17 Ca paznicii câmpului, așa l-au înconjurat pe Israel de jur împrejur, pentru că el s-a răzvrătit împotriva Mea, zice Domnul.
\par 18 Căile tale și faptele tale, Israele, ți-au pricinuit acestea; din pricina necredincioșii tale fi-a venit acest amar, care a străbătut până la inima ta".
\par 19 Inima mea! Inima mea! Mă doare inima până în adânc! Tulburatu-s-a inima mea în mine și nu pot tăcea, că tu, suflete al meu, auzi glasul trâmbiței, auzi strigătul de război.
\par 20 Nenorocire peste nenorocire: tot pământul se pustiește și fără de veste mi s-au stricat corturile și într-o clipeală sălașurile mele.
\par 21 Oare mult îmi este dat să văd steagul și să aud sunetul trâmbiței?
\par 22 Și toate acestea sunt numai pentru că poporul Meu e fără minte și nu Mă cunoaște, sunt copii nepricepuți și n-au înțelegere; sunt pricepuți numai la rele, iar binele nu știu să-l facă.
\par 23 Mă uit peste țară și iată este ruinată și pustie;
\par 24 Caut la ceruri și iată nu este lumină pe ele; privesc la munți și iată că ei tremură și dealurile toate se clatină.
\par 25 Mă uit și iată nu este nici un om și toate păsările cerului au fugit.
\par 26 Mă uit și iată Carmelul este o pustietate și toate cetățile lui sunt arse cu foc de la fața Domnului și au pierit de la fața mâniei Lui.
\par 27 Că așa a zis Domnul: "Toată țara va fi pustiită, dar nu o voi nimici de tot.
\par 28 Va plânge de aceasta pământul și cerurile sus se vor întuneca, pentru că Eu am zis, Eu am hotărât și nu Mă voi căi, nici Mă voi întoarce de la aceasta.
\par 29 De strigătele călăreților și ale arcașilor țara întreagă este pusă pe fugă; toți vor fugi în pădurile cele dese și se vor sui pe stânci; toate cetățile vor fi părăsite și nici un locuitor nu va mai fi în acestea.
\par 30 Și tu, pustiito, ce vei face? Chiar când te-ai îmbrăca în purpură, chiar dacă te-ai găti cu podoabe de aur și ți-ai sulemeni ochii cu dresuri, în zadar te-ai face frumoasă, că iubiții tăi te disprețuiesc și vor numai viața.
\par 31 Când aud glas ca de femeie ce naște, aud geamăt ca al uneia ce naște pentru întâia oară: este glasul fiicei Sionului; ea geme și întinde mâna, zicând: "O, vai de mine, mi se istovește sufletul înaintea ucigașilor!"

\chapter{5}

\par 1 "Cutreierați ulițele Ierusalimului, uitați-vă, cercetați și căutați prin piețele lui: nu cumva veți găsi vreun om, măcar unul, care păzește dreptatea și caută adevărul?
\par 2 Căci Eu aș cruța Ierusalimul. Chiar când ei zic: "Viu este Domnul", ei jură mincinos.
\par 3 O, Doamne, ochii Tăi nu privesc ei oare la adevăr? Tu îi bați și ei nu simt durerea; Tu îi pierzi și ei nu vor să ia învățătură; și-au făcut obrazul mai vârtos ca piatra și nu vor să se întoarcă.
\par 4 Și mi-am zis: "Poate că aceștia sunt niște bieți nenorociți! Sunt niște proști, pentru că nu cunosc calea Domnului, legea Dumnezeului lor.
\par 5 Voi merge deci la cei mari și voi grăi cu aceștia, că ei știu calea Domnului; legea Dumnezeului lor". Dar și aceștia cu toții au sfărâmat jugul și au rupt cătușele.
\par 6 De aceea îi va lovi leul din pădure și lupul din pustiu îi va răpi; leopardul le va fi păzitor cetăților lor; care din ei va ieși va fi sfâșiat, că s-au înmulțit fărădelegile lor și lepădările de credință au sporit.
\par 7 Cum, adică, să te iert, Ierusalime, pentru aceasta? Fiii tăi M-au părăsit și se jură pe dumnezei care n-au ființă. Eu i-am săturat, iar ei au făcut desfrânare, umblând în grup prin casele desfrânatelor.
\par 8 Ei sunt cai îngrășați și fiecare din ei nechează după femeia aproapelui său.
\par 9 E cu putință să nu pedepsesc aceasta, zice Domnul, și Duhul Meu să nu se răzbune asupra unui popor ca acesta?
\par 10 Suiți-vă pe zidurile lui și le dărâmați, dar nu de tot, ci stricați numai crestele lor, pentru că acestea nu sunt ale Domnului;
\par 11 Căci casa lui Israel și casa lui Iuda s-au purtat față de Mine cu multă necredință, zice Domnul.
\par 12 Au tăgăduit pe Domnul și au zis: "El nu este și nenorocirea nu va veni asupra noastră; și nu vom vedea nici sabie, nici foamete!
\par 13 Proorocii sunt vânt și nu este în aceștia cuvântul Domnului. De aceasta și ei să aibă parte!"
\par 14 De aceea, așa zice Domnul Dumnezeul puterilor: "Pentru că voi grăiți asemenea vorbe, iată voi face cuvintele Mele foc în gura ta, iar pe poporul acesta îl voi face lemne ș-l va mistui focul acesta.
\par 15 Casa lui Israel, iată voi aduce asupra voastră un neam de departe, zice Domnul, un popor puternic, un popor vechi, un popor a cărui limbă tu nu o știi și nu vei înțelege ce grăiește el.
\par 16 Tolba lui e ca un mormânt deschis și ai lui toți sunt viteji;
\par 17 și vor mânca aceia secerișul tău și pâinea ta; vor mânca pe fiii tăi și pe fiicele tale; vor mânca oile tale și boii tăi; vor mânca strugurii tăi și smochinele tale și vor trece prin sabie cetățile tale cele întărite în care tu te încrezi.
\par 18 Dar nici în zilele acelea nu vă voi pierde cu totul, zice Domnul.
\par 19 Și de veți zice: Pentru ce ne face Domnul Dumnezeul nostru toate acestea? Atunci să ți se răspundă: Pentru că M-ați părăsit pe Mine și ați slujit la dumnezei străini, în țara voastră, de aceea veți sluji la dumnezei străini într-o țară care nu este a voastră.
\par 20 Spuneți aceasta în casa lui Iacov, vestiți-o în Iuda și ziceți:
\par 21 Ascultați acestea, popor nebun și fără inimă! Ei au ochi și nu văd, urechi au, dar nu aud.
\par 22 Au doar nu vă temeți de Mine, zice Domnul, și nu tremurați înaintea Mea? Eu am pus nisipul hotar împrejurul mării și hotar veșnic, peste care nu se va trece. Deși valurile ei se înfurie, nu pot să-l biruiască și, deși ele se aruncă, nu pot să-l treacă.
\par 23 Dar poporul acesta are inimă îndârjită și răzvrătită:
\par 24 Răzvrătitu-s-au și s-au dus și n-au zis în inima lor: "Să ne temem de Domnul Dumnezeul nostru, Care ne dă la vreme ploaie timpurie și târzie și ne păstrează săptămânile hotărâte ale culesului".
\par 25 Fărădelegile voastre au schimbat aceasta și păcatele voastre au depărtat acest bine de la voi.
\par 26 Că se află necredincioși prin poporul Meu, care pândesc ca prinzătorii de păsări, se ascund la pământ, pun curse și prind pe oameni.
\par 27 Cum sunt cotețele pline de păsări, așa sunt casele lor pline de înșelătorie;
\par 28 Prin aceasta s-au ridicat și s-au îmbogățit ei, s-au făcut grași și cu pielea lucioasă și în rele au trecut orice măsură;
\par 29 Nu fac dreptate nimănui, nici chiar orfanului, nu dau dreptate săracului și huzuresc. E cu putință oare să nu pedepsesc acestea și să nu Mă răzbun asupra unui popor ca acesta? - zice Domnul.
\par 30 Lucruri înspăimântătoare se petrec în țara aceasta:
\par 31 Proorocii profețesc minciuni, preoții învață ca și ei, și poporului Meu îi place aceasta. Dar la urmă ce veți face?"

\chapter{6}

\par 1 Fugiți, fiii lui Veniamin, fugiți din Ierusalim, trâmbițați cu trâmbița în Tecoa și dați semne prin focuri la Bethacherem, că iată se ivește de la miazănoapte o nenorocire și zdrobire mare!
\par 2 Pierde-voi pe fiica Sionului cea frumoasă și gingașă.
\par 3 Păstorii vor veni la ea cu turmele lor, își vor întinde corturile împrejurul ei și va paște fiecare partea sa.
\par 4 Pregătiți război împotriva ei! Sculați-vă și haideți spre miazăzi! Vai! Ziua este spre sfârșit și iată se lasă umbrele serii.
\par 5 Dar sculați-vă și hai să mergem noaptea și să stricăm palatele ei.
\par 6 Căci așa zice Domnul Savaot: "Tăiați copaci și faceți val împotriva Ierusalimului; această cetate trebuie pedepsită, pentru că în ea se află numai nedreptate.
\par 7 Cum aruncă izvorul apă din sine, așa aruncă și ea din sine răutate; în ea se aude împilare și jaf și pururea se văd înaintea feței Mele dureri și răni.
\par 8 Înțelepțește-te, Ierusalime, ca să nu se depărteze sufletul Meu de la tine și ca să nu te fac pustietate și pământ nelocuit".
\par 9 Așa zice Domnul Savaot: "Până la sfârșit se vor culege rămășițele lui Israel, cum se culege via; lucrează cu mâna ta și îți umple panerul, ca și culegătorul de struguri.
\par 10 Cu cine să vorbesc și cui să vestesc, ca să audă? Că iată urechea lor este netăiată împrejur și nu pot să ia aminte; și iată, cuvântul Domnului a ajuns de râs la ei și nu găsesc în el nici o plăcere.
\par 11 De aceea sunt plin de mânia Domnului și n-o mai pot ține în mine; o voi vărsa deci asupra copiilor pe ulițe și asupra adunării tinerilor, că vor fi luați și bărbat și femeie, și cel în vârstă și cel încărcat de zile;
\par 12 și casele lor vor trece la alții; tot așa și țarinele și femeile. Pentru că voi întinde mâna Mea asupra locuitorilor țării acesteia, zice Domnul,
\par 13 Pentru că fiecare din ei, de la mic până la mare, este robit de lăcomie și, de la prooroc până la preot, toți se poartă mincinos.
\par 14 Ei leagă rănile poporului meu cu nepăsare și zic: "Pace! Pace!" Și numai pace nu este!
\par 15 Dar se rușinează ei, oare, când fac urâciuni? Nu, nu se rușinează deloc, nici roșesc. De aceea vor cădea printre cei căzuți și se vor prăbuși în ziua în care îi voi pedepsi, zice Domnul.
\par 16 Așa zice Domnul: "Opriți-vă de la căile voastre! Priviți și întrebați de căile celor de demult; de calea cea bună și mergeți pe dânsa și veți afla odihnă sufletelor voastre.
\par 17 Pus-am păzitori peste voi și am zis: Ascultați sunetul trâmbiței!
\par 18 Iar ei au zis: Nu vom asculta! Așadar, ascultă, poporule, și află, adunare, ce are să se întâmple cu aceștia.
\par 19 Ascultă, pământule: Iată voi aduce asupra acestui popor o nenorocire, rodul cugetelor lor, că n-au ascultat cuvintele Mele și legea Mea au lepădat-o.
\par 20 La ce Îmi este bună tămâia care vine din Șeba și scorțișoara din țară depărtată? Arderile de tot ale voastre nu le voiesc și jertfele voastre Îmi sunt neplăcute".
\par 21 De aceea așa zice Domnul: "Iată pun înaintea poporului acestuia piedici și se vor poticni de ele deodată și părinții și copiii, vecinul și prietenii lui, și vor pieri".
\par 22 Așa zice Domnul: "Iată vine un popor din țara de la miazănoapte, un popor mare se ridică de la marginile pământului și ai săi țin în mână arcul și sulița.
\par 23 Și sunt cruzi și neîndurați; glasul lor mugește ca marea și vin pe cai, gata să lupte ca un singur om, împotriva ta, fiica Sionului".
\par 24 Noi am auzit de ei și ne-au slăbit mâinile de spaimă; ne-au cuprins groază și dureri ca ale femeii ce naște.
\par 25 Să nu ieșiți la câmp, nici la drum să nu plccați, căci sabia dușmanilor și groaza sunt pretutindeni.
\par 26 Fiica poporului Meu, încinge-te cu sac și-ți presară cenușă pe cap; tânguiește-te ca după moartea singurului tău fiu! Plângi amar, că fără de veste va veni pierzătorul asupra voastră!
\par 27 Turn te-am pus în mijlocul poporului Meu și stâlp, ca să afli și să urmărești drumul lor.
\par 28 Aceștia cu toții sunt răzvrătiți, îndârjiți și semănători de clevetiri; sunt aramă și fier, toți sunt niște stricați.
\par 29 Foalele s-au ars, plumbul s-a topit de foc și turnătorul în zadar a topit, că cei răi nu s-au ales.
\par 30 Argint de lepădat se vor numi, că Domnul i-a lepădat".

\chapter{7}

\par 1 Cuvântul ce a fost de la Domnul către Ieremia: "Stai în ușa templului Domnului și rostește acolo cuvântul acesta și zi:
\par 2 Ascultați cuvântul Domnului, toți bărbații lui Iuda, care intrați pe această poartă ca să vă închinați Domnului!
\par 3 Așa zice Domnul Savaot, Dumnezeul lui Israel: Îndreptați-vă căile și faptele voastre și vă voi lăsa să trăiți în locul acesta!
\par 4 Nu vă încredeți în cuvintele mincinoase care zic: "Acesta este templul Domnului, templul Domnului, templul Domnului".
\par 5 Iar dacă vă veți îndrepta cu totul căile și faptele voastre, dacă veți face judecată cu dreptate între om și pârâșul lui,
\par 6 Dacă nu veți strâmtora pe străin, pe orfan și pe văduvă, nu veți vărsa sânge nevinovat în locul acesta și nu veți merge după alți dumnezei spre pieirea voastră,
\par 7 Atunci vă voi lăsa să trăiți în locul acesta și pe pământul acesta, pe care l-am dat părinților voștri din neam în neam.
\par 8 Iată, vă încredeți în cuvinte mincinoase, care nu vă vor aduce folos.
\par 9 Cum? Voi furați, ucideți și faceți adulter; jurați mincinos, tămâiați pe Baal și umblați după alți dumnezei, pe care nu-i cunoașteți,
\par 10 Și apoi veniți să vă înfățișați înaintea Mea în templul Meu, asupra căruia s-a chemat numele Meu, și ziceți: "Suntem izbăviți", ca apoi să faceți iar toate ticăloșiile acelea?
\par 11 Templul acesta, asupra căruia s-a chemat numele Meu, n-a ajuns el oare, în ochii voștri peșteră de tâlhari? Iată, Eu am văzut aceasta, zice Domnul.
\par 12 Mergeri deci la locul Meu din Silo, unde făcusem altădată să locuiască numele Meu, și vedeți ce am făcut Eu cu el, pentru necredința poporului Meu Israel!
\par 13 Și acum, de vreme ce ați făcut toate faptele acestea, zice Domnul, și Eu v-am grăit dis-de-dimineață și n-ați ascultat, v-am chemat și n-ați răspuns,
\par 14 De aceea și cu templul acesta, asupra căruia s-a chemat numele Meu și în care voi vă puneți încrederea, și cu locul pe care vi l-am dat vouă și părinților voștri, voi face tot așa, cum am făcut cu Șilo;
\par 15 Vă voi lepăda de la fața Mea, cum am lepădat pe toți frații voștri, toată seminția lui Efraim.
\par 16 Tu însă nu te ruga pentru acest popor și nu înălța rugăciune și cerere pentru dânșii, nici nu mijloci înaintea Mea, că nu te voi asculta.
\par 17 Nu vezi tu ce fac ei prin cetățile lui Iuda și pe ulițele Ierusalimului?
\par 18 Copiii adună lemne, iar părinții ațâță focul și femeile frământă aluatul ca să facă turte pentru zeița cerului și să săvârșească turnări în cinstea altor dumnezei, ca să Mă rănească pe Mine.
\par 19 Dar oare pe Mine Mă rănesc ei - zice Domnul - și nu pe ei înșiși, spre rușinea lor?"
\par 20 De aceea, așa zice Domnul Dumnezeu: Iată se revarsă mânia Mea asupra locului acestuia, asupra oamenilor și dobitoacelor, asupra copacilor, asupra țarinii și asupra roadelor pământului, și se vor aprinde și nu se vor mai stinge".
\par 21 Așa zice Domnul Savaot, Dumnezeul lui Israel: "Arderile de tot ale voastre adăugați-le la jertfele voastre și mâncați carne;
\par 22 Că părinților voștri nu le-am vorbit și nu le-am dat poruncă în ziua aceea, în care i-am scos din pământul Egiptului, pentru arderea de tot și pentru jertfă;
\par 23 Ci iată porunca pe care ți-am dat-o: Să ascultați glasul Meu, și Eu voi fi Dumnezeul vostru, iar voi Îmi veți fi poporul Meu, și să umblați pe toată calea pe care vă poruncesc Eu, ea să vă fie bine.
\par 24 Dar ei n-au ascultat glasul Meu și nu și-au plecat urechea lor, ci au trăit după pofta și îndărătnicia inimii lor rele și s-au întors cu spatele către Mine, iar nu cu fața.
\par 25 Din ziua când părinții voștri au ieșit din pământul Egiptului și până în ziua aceasta am trimis la voi pe toți robii Mei - proorocii - și i-am trimis în fiecare zi dis-de-dimineață;
\par 26 Dar ei nu M-au ascultat și nu și-au plecat urechea lor, ci și-au învârtoșat cerbicia și s-au purtat mai rău decât părinții lor.
\par 27 Și când le vei vorbi cuvintele acestea, ei nu te vor asculta; și când îi vei chema, nu-ți vor răspunde.
\par 28 Atunci să le zici: Iată un popor care nu ascultă glasul Domnului Dumnezeului său și nu primește învățătură! Credința nu mai este; ea a dispărut din gura lor.
\par 29 Tunde-ți părul tău și-l aruncă, și ridică plângere pe munți, că a lepădat Domnul și a părăsit pe neamul care și-a atras mânia Sa.
\par 30 Că fiii lui Iuda fac rele înaintea ochilor Mei, zice Domnul, și și-au pus urâciunile lor în templul asupra căruia s-a chemat numele Meu ca să-l pângărească;
\par 31 Și au zidit locurile înalte la Tofet, în valea fiilor lui Hinom, ca să-și ardă fiii și fiicele cu foc, ceea ce Eu nu le-am poruncit și ceea ce Mie nu Mi-a trecut prin minte.
\par 32 De aceea, iată vin zile, zice Domnul, când locul acesta nu se va mai chema Tofet și valea fiilor lui Hinom, ci valea uciderii, și în Tofet se vor face înmormântări din pricina lipsei de loc.
\par 33 Și vor fi trupurile poporului acestuia hrană păsărilor cerului și fiarelor pământului, și nu va fi cine să le alunge.
\par 34 Și voi curma în cetățile lui Iuda și pe ulițele Ierusalimului glasul de bucurie și glasul de veselie, glasul mirelui și glasul miresei, pentru că pământul acesta va fi pustiu".

\chapter{8}

\par 1 "În vremea aceea, zice Domnul, oasele regilor lui Iuda și oasele căpeteniilor lui, oasele preoților și proorocilor și oasele locuitorilor Ierusalimului vor fi aruncate din mormintele lor,
\par 2 Și vor fi aruncate înaintea soarelui și a lunii și înaintea întregii oștiri cerești, pe care ei le-au iubit și cărora au slujit și pe urma cărora au umblat, pe care le-au căutat și cărora s-au închinat. Nimeni nu le va aduna, nici le va îngropa, ci vor zăcea ca gunoiul pe pământ.
\par 3 Și toți ceilalți care vor rămâne din acest neam rău vor dori moartea în locul vieții, în toate locurile, pe unde îi voi izgoni, zice Domnul.
\par 4 Să le mai spui de asemenea: Așa zice Domnul: Oare cei ce cad nu se mai scoală, și cei ce rătăcesc drumul nu se mai întorc?
\par 5 De ce dar poporul acesta, Ierusalime, stăruie în rătăcire? De ce ține tare la minciună și nu vrea să se întoarcă?
\par 6 Privit-am și am ascultat, și iată-i nu spun adevărul, nu se căiesc de necredința lor și nimeni nu zice: "Vai, ce am făcut?" Fiecare se întoarce la calea sa, ca un cal ce se aruncă în bătălie.
\par 7 Până și barza își știe vremea sa hotărâtă sub cer; și turturica și rândunica și cocorul iau aminte la timpul când trebuie să vină;
\par 8 Iar poporul Meu nu cunoaște hotărârea Domnului. Cum puteți voi să ziceți: "Suntem înțelepți și avem legea Domnului?" Căci iată, pana cea mincinoasă a cărturarilor a prefăcut-o în minciună.
\par 9 S-au făcut de ocară înțelepții, au turbat și s-au prins în curse; iată, au lepădat cuvântul Domnului și atunci unde este înțelepciunea lor?
\par 10 De aceea, pe femeile lor le voi da altora și ogoarele lor le voi trece altor stăpânitori, pentru că ei cu toții, de la mic până la mare, se dedau la jaf, și de la prooroc până la preot, toți înșală,
\par 11 Și leagă rana fiicei poporului Meu cu nepăsare, zicând: "Pace, pace!" și pace nu este.
\par 12 Se rușinează ei, oare, când fac ticăloșii? Nu se rușinează deloc, nici nu roșesc. De aceea vor cădea printre cei ce cad și se vor prăbuși când îi voi pedepsi, zice Domnul.
\par 13 Îi voi culege cu totul, zice Domnul, și nu va rămâne nici o bobiță pe viță și nici o smochină în smochin, și va cădea și frunza și, ceea ce le-am dat Eu, se va duce de la ei".
\par 14 De ce ședem noi oare? Adunați-vă și haideți în cetățile cele întărite, ca să pierim acolo; că Domnul Dumnezeul nostru ne-a hotărât la pieire și ne dă să bem apă cu fiere, pentru că am greșit înaintea Domnului.
\par 15 Așteptăm pace și iată nu este nici un bine; așteptăm timpul vindecării, și iată groaza.
\par 16 De la Dan se aude ropotul cailor și de nechezatul puternic al armăsarilor se cutremură tot pământul; iată vin și vor pustii pământul și tot ce este pe el, cetatea și locuitorii ei.
\par 17 "Căci iată voi trimite asupra voastră șerpi și scorpii, împotriva cărora nu este descântec, și vă vor mușca, zice Domnul.
\par 18 Când Mă voi mângâia Eu de scârba Mea? Mi s-a amărât inima în Mine.
\par 19 Iată, aud plânsul fiicei poporului Meu din țară depărtată, zicând: "Oare nu mai este Domnul în Sion? Oare acesta nu mai are asupra sa pe regele său?" - "De ce M-au împins ei oare la mânie cu idolii lor cei străini și de nimic?" - zice Domnul.
\par 20 Secerișul a trecut, vara este pe sfârșite și noi tot nu suntem izbăviți:
\par 21 De durerea fiicei poporului meu sunt îndurerat, umblu posomorât și groaza m-a cuprins.
\par 22 Au doară nu mai este balsam în Galaad? Au doară nu mai este acolo doctor? De ce dar nu se vindecă fiica poporului Meu?

\chapter{9}

\par 1 O, cine va da capului meu apă și ochilor mei izvoare de lacrimi, ca să plâng ziua și noaptea pe cei loviți ai fiicei poporului meu?
\par 2 O, de mi-ar da cineva un adăpost de călători în pustiu, aș părăsi pe poporul meu și m-aș duce de la ei, căci eu toții sunt niște desfrânați și ceată de defăimători!
\par 3 "Ca un arc își încordează limbile lor pentru minciună și se întăresc pe pământ prin nedreptate, că trec de la rău la mai rău și pe Mine nu Mă cunosc, zice Domnul.
\par 4 Păziți-vă fiecare de prietenul vostru și nu vă încredeți în nici unul din frații voștri, că fiecare frate pune piedică celuilalt și fiecare prieten împrăștie clevetiri.
\par 5 Fiecare înșală pe prietenul său și nu spune adevărul; și-au deprins limba la minciună și viclenesc până obosesc.
\par 6 Tu trăiești în mijlocul viclenilor și ei din pricina vicleniei nu Mă cunosc pe Mine", zice Domnul.
\par 7 De aceea, așa zice Domnul Savaot: "Îi voi topi și-i voi încerca; oare ce altceva pot să fac Eu cu fiica poporului Meu?
\par 8 Limba lor este săgeată ucigătoare și grăiește viclenii; cu buzele lor grăiesc prietenos către aproapele lor, iar în inimă ei făuresc cătușe.
\par 9 E cu putință, oare, să nu-i pedepsesc pentru aceasta, zice Domnul, și sufletul Meu să nu se răzbune pe un popor ca acesta?
\par 10 Pentru munți voi ridica plângere și bocet și pentru pășunile pustiului mă voi tângui, pentru că vor fi arse, și nimeni nu va mai umbla pe acolo și nu se va mai auzi glasul turmelor; de la păsările cerului și până la animale toate s-au împrăștiat și s-au dus!
\par 11 Și voi face Ierusalimul o movilă de pietre și sălaș șacalilor, iar cetățile lui Iuda le voi face pustietate fără locuitori".
\par 12 Cine este înțeleptul, care să priceapă, și cui a grăit gura Domnului, ca să spună pentru ce a pierit țara și a fost arsă ca un pustiu, încât nimeni să nu mai treacă prin ea?
\par 13 Și Domnul a răspuns: "Pentru că au părăsit Legea Mea, pe care le-am pus-o Eu, și n-au ascultat glasul Meu, nici nu s-au purtat cum le poruncea acel glas,
\par 14 Ci au umblat după îndărătnicia inimii lor și în urma lui Baal, cum i-au învățat părinții lor".
\par 15 De aceea, așa zice Domnul Savaot, Dumnezeul lui Israel: "Iată, îi voi hrăni cu pelin și le voi da să bea apă cu fiere, și-i voi risipi printre popoarele
\par 16 Pe care nu le-au cunoscut nici ei, nici părinții lor, și voi trimite pe urma lor sabie până îi voi pierde".
\par 17 Așa zice Domnul Savaot: "Gândiți-vă și chemați bocitoare ca să bocească; trimiteți la cele mai istețe în asemenea lucru, ca să vină!"
\par 18 Să se grăbească dar a înălța o cântare de jale pentru noi, ca să curgă lacrimi din ochii noștri și din genele noastre să curgă apă.
\par 19 Căci glas de plângere se aude din Sion, zicând: "Cât suntem de prăpădiți și cât de cumplit batjocoriți, că trebuie să părăsim țara, pentru că locuințele noastre au fost dărâmate!
\par 20 Așadar, femei, ascultați cuvântul Domnului și să ia aminte urechea voastră la cuvântul gurii Lui! învățați pe fiicele voastre a boci și una pe alta să se învețe cântece de bocet,
\par 21 Că iată moartea intră pe ferestrele noastre și dă năvală în casele noastre, ca să piardă pe copiii din uliță și pe tinerii din piețe;
\par 22 Și vor cădea trupurile oamenilor ca gunoiul aruncat pe ogor și ca snopii în urma secerătorului, și nu va fi cine să-i strângă".
\par 23 Așa zice Domnul: "Să nu se laude cel înțelept cu înțelepciunea sa, să nu se laude cel puternic cu puterea sa, nici cel bogat să nu se laude cu bogăția sa;
\par 24 Ci de se laudă cineva, să se laude numai cu aceea că pricepe; și Mă cunoaște că Eu sunt Domnul, Cel ce fac milă și judecată și dreptate pe pământ, căci numai aceasta este plăcut înaintea Mea, zice Domnul.
\par 25 Iată, vin zile, zice Domnul, când voi cerceta pe toți cei tăiați și netăiați împrejur:
\par 26 Egiptul și Iuda, Edomul și pe fiii lui Amon, Moabul și pe toți locuitorii pustiului, care-și tund părul împrejurul frunții lor, căci toate aceste popoare sunt netăiate împrejur, iar casa lui Israel toată este cu inima netăiată împrejur".

\chapter{10}

\par 1 Casa lui Israel, ascultați cuvântul ce vi-l grăiește Domnul!
\par 2 Așa zice Domnul: "Nu deprindeți căile neamurilor și nu vă îngroziți de semnele cerului, de care se îngrozesc neamurile;
\par 3 Că datinile neamurilor sunt deșertăciune: aceia taie un lemn din pădure și mâinile meșterului îl lucrează cu toporul;
\par 4 Apoi îl îmbracă cu argint și cu aur, îl întăresc cu cuie și cu ciocanul, ca să nu se clatine. Așa sunt dumnezeii neamurilor.
\par 5 Stau ca niște sperietori într-o grădină cu pepeni și nu grăiesc, sunt purtați pentru că nu pot merge. Nu vă temeți de ei, că nu pot face rău, dar nici bine nu sunt în stare să facă".
\par 6 Nimeni nu este ca Tine, Doamne! Mare ești Tu și mare este puterea numelui Tău!
\par 7 Cine nu se va teme de Tine, Împărate al neamurilor? Numai ție unuia se cuvine aceasta, pentru că printre toți înțelepții neamului și în toate regatele lor nu este nimeni asemenea ție.
\par 8 Dumnezeii neamurilor, toți până la unul, sunt fără minte și fără pricepere; lemne lipsite de orice înțelegere;
\par 9 Argint prefăcut în foi și adus din Tarsis; aur din Ofir; lucruri de meșter și de mână de turnător; haina de pe ei este de iacint și purpură: iarăși lucru de oameni iscusiți.
\par 10 Iar Domnul este adevăratul Dumnezeu, este Dumnezeu viu și împărat veșnic; de mânia Lui tremură pământul și neamurile nu pot suferi urgia Lui.
\par 11 Așadar să ziceți neamurilor: "Dumnezeii, care n-au făcut cerul și pământul, vor pieri de pe pământ și de sub ceruri;
\par 12 Iar Domnul a făcut cerul cu puterea Sa, a întărit lumea cu înțelepciunea Sa și cu priceperea Sa a întins cerurile.
\par 13 La glasul Lui freamătă apele în ceruri și El ridică norii de la marginile pământului, făurește fulgerele în mijlocul ploii și scoate vânturile din vistieriile Sale.
\par 14 Atunci se vede cât este de neștiutor omul, cu toată știința lui, și orice argintar se rușinează de idolul său, căci chipul turnat de el nu este decât minciună; n-are nici o suflare în el.
\par 15 Aceasta este deșertăciune adevărată, rodul rătăcirii, și la vremea pedepsei va pieri.
\par 16 Iar soarta lui Iacov nu este ca a neamurilor, că Dumnezeul lui este făcătorul a toate, iar Israel este toiagul moștenirii Lui; și numele Lui este Domnul Savaot.
\par 17 O, tu, care ești împresurată, strânge-ți de pe pământ avuția, că așa zice Domnul:
\par 18 "Iată de data aceasta voi arunca pe locuitorii țării acesteia și-i voi mâna la lac strâmt, ca să fie prinși".
\par 19 Atunci vei grăi: "Vai mie din pricina rănii mele! Rana mea e dureroasă; însă îmi zic: Aceasta este nenorocirea mea, dar o voi îndura!
\par 20 Cortul îmi este pustiit și toate funiile lui sunt rupte; fiii mei m-au părăsit și nu mai sunt, și n-are cine să-mi întindă cortul și să-mi ridice pânzele.
\par 21 Pentru că păstorii și-au ieșit din minte și n-au căutat pe Domnul și de aceea s-au și purtat ei nebunește și toată turma ei s-a risipit".
\par 22 Iată, vine vuiet și zgomot mare din partea de miazănoapte ca să pustiiască cetățile lui Iuda și să le facă locuință șacalilor.
\par 23 Știu, Doamne, că nu este în voia omului calea lui, nici în voia celui ce merge, putința să-și îndrepte pașii săi.
\par 24 Pedepsește-mă, Doamne, dar după dreptate și nu întru mânia Ta, ca să nu mă micșorezi.
\par 25 Varsă-ți iuțimea Ta asupra neamurilor care nu Te cunosc și asupra popoarelor care nu cheamă numele Tău, că acelea au mâncat pe Iacov, l-au mistuit și l-au stins și locuințele lui le-au pustiit.

\chapter{11}

\par 1 Cuvântul ce a fost de la Domnul către Ieremia:
\par 2 "Ascultă cuvintele legământului acestuia și spune bărbaților lui Iuda și celor ce locuiesc în Ierusalim și le zi:
\par 3 Așa zice Domnul Dumnezeul lui Israel: Blestemat să fie omul care nu ascultă cuvintele acestui legământ,
\par 4 Pe care l-am dat Eu părinților voștri când i-am scos din pământul Egiptului și din cuptorul cel de fier și le-am zis: Ascultați glasul Meu și faceți tot ce v-am poruncit, și veți fi poporul Meu, iar Eu voi fi Dumnezeul vostru,
\par 5 Ca să împlinesc jurământul cu care M-am jurat părinților voștri, că le voi da o țară în care curge lapte și miere, cum este astăzi". Și eu am răspuns: "Așa, Doamne!"
\par 6 Iar Domnul mi-a zis iarăși: "Vestește toate cuvintele acestea în cetățile lui Iuda și pe ulițele Ierusalimului și zi: ascultați cuvintele lgământului acestuia și le împliniți!
\par 7 Căci am înștiințat necontenit pe părinții voștri, din ziua în care i-am scos din țara Egiptului și până astăzi; i-am înștiințat fără încetare și le-am zis: Ascultați glasul Meu!
\par 8 Dar ei n-au ascultat și nu și-au plecat urechea, ci au umblat fiecare după inima sa cea rea și îndărătnică. De aceea am adus asupra lor toate cele spuse în legământul acesta pe care l-am poruncit să-l țină, și nu l-au ținut".
\par 9 Și iarăși mi-a zis Domnul: "Între bărbații lui Iuda și locuitorii Ierusalimului se pune la cale răzvrătire;
\par 10 Ei s-au întors iarăși la fărădelegile strămoșilor lor, care n-au voit să asculte cuvintele Mele și au mers după dumnezei străini, slujind acelora. Casa lui Israel și casa lui Iuda au călcat legământul Meu, pe care l-am încheiat cu părinții lor.
\par 11 De aceea, așa zice Domnul: Iată voi aduce asupra lor nenorociri, de care nu se vor putea izbăvi și, când vor striga către Mine, nu-i voi auzi.
\par 12 Atunci cetățile lui Iuda și locuitorii Ierusalimului vor alerga și vor striga către dumnezeii pe care-i tămâiază, dar aceia nu le vor ajuta în vremea nenorocirii lor.
\par 13 Căci câte cetăți ai, tot atâția dumnezei ai, Iuda, și câte uliți sunt în Ierusalim, tot atâtea sunt și jertfelnicele celui de rușine, jertfelnicele pentru tămâierea lui Baal.
\par 14 Tu însă nu te ruga pentru poporul acesta și nu înălța pentru e! rugăciune și cereri, căci nu voi auzi când vor striga către Mine în vremea nenorocirii lor.
\par 15 Ce cauți, iubitul Meu popor, în templul Meu, când în acesta se săvârșesc atâtea netrebnicii? Jertfele nu-li vor ajuta când, făcând rău, te bucuri.
\par 16 Măslin verde, împodobit cu roade frumoase, te-a numit Domnul. Iar acum în zgomotul cumplitei tulburări a aprins foc împrejurul lui și ramurile lui s-au stricat.
\par 17 Domnul Savaot, Cel ce te-a sădit, a hotărât asupra ta nenorocirea pentru răutatea pe care casa lui Israel și casa lui Iuda și-au pricinuit-o singure, împingându-Mă la mânie prin tămâierea lui Baal".
\par 18 Domnul mi-a descoperit ca să știu și mi-a arătat faptele lor.
\par 19 Eu insă, ca un miel blând, dus la junghiere, nici nu știam că ei urzesc gânduri rele împotriva mea, zicând: "Să-i punem lemn otrăvit în mâncarea lui și să-l smulgem de pe pământul celor vii, pentru ca nici numele să nu i se mai pomenească".
\par 20 Dar Tu, Doamne al puterilor, Judecătorul cel drept, Care cercetezi inimile și rărunchii, dă-mi să văd răzbunarea Ta asupra lor, pentru că ți-am încredințat pricina mea.
\par 21 Pentru aceasta, așa zice Domnul despre oamenii din Anatot care caută sufletul meu și zic: "Nu mai prooroci în numele Domnului, ca să nu mori de mâinile noastre!"
\par 22 De aceea, așa zice Domnul Savaot: "Iată, îi voi pedepsi și tinerii lor vor muri de sabie, iar fiii lor și fiicele lor vor muri de foame;
\par 23 Nu va scăpa niciunul, din ei, căci voi aduce nenorocirea asupra oamenilor din Anatot, în anul când îi voi pedepsi.

\chapter{12}

\par 1 De voi intra la judecată cu Tine, Doamne, dreptatea va fi de partea Ta, și totuși, despre dreptate vreau să grăiesc cu Tine: "Pentru ce calea necredincioșilor este cu izbândă și pentru ce toți călcătorii de lege sunt în fericire?
\par 2 Tu i-ai sădit și ei au prins rădăcini, au crescut și au făcut roade; Tu ești aproape numai de buzele lor, iar de inima lor ești departe.
\par 3 Pe mine însă mă cunoști, Doamne, mă vezi și cercetezi dacă inima mea este cu Tine. Osebește-i dară, ca pe niște oi de junghiat, și pregătește-i pentru ziua junghierii.
\par 4 Până când va jeli țara și iarba de prin toate țarinile se va usca? Dobitoacele și păsările pier pentru necredința locuitorilor ei, că aceștia zic: "Domnul nu vede căile noastre!"
\par 5 "Dacă alergând cu cei ce merg pe jos, ai ostenit, cum te vei lua la întrecere cu caii? Și dacă nu ești în siguranță intr-o țară pașnică, ce vei face în hățișurile Iordanului?
\par 6 Că și frații tăi și casa tatălui tău se poartă necredincios cu tine și strigă tare în urma ta. Și chiar când îți grăiesc bine nu te încrede în ei", zice Domnul.
\par 7 "Părăsit-am casa Mea și moștenirea Mea am lăsat-o; dat-am pe iubita sufletului Meu în mâinile vrăjmașilor ei.
\par 8 Făcutu-s-a moștenirea Mea pentru Mine ca un leu din pădure, ridicându-și glasul împotriva Mea, și de aceea am urât-o.
\par 9 Moștenirea Mea s-a făcut pentru Mine ca o pasăre de pradă pestriță, asupra căreia au năvălit din toate părțile celelalte păsări de pradă. Strângeți-vă și vă duceți toate fiarele câmpului, duceți-vă și o mâncați.
\par 10 Mulțime de păstori au călcat via Mea, călcat-au cu picioarele lor partea Mea; partea Mea cea iubită au făcut-o deșert neroditor.
\par 11 Pustietate au făcut-o, și ea, în pustiirea sa, plânge înaintea Mea. Toată țara e pustiită, pentru că pe nici un om nu-l doare inima de aceasta.
\par 12 Pe toate dealurile din pustiu au venit prădători, că sabia Domnului mistuie totul de la o margine la alta a țării și nici un muritor n-are pace.
\par 13 Semănat-au grâu și au secerat spini! Muncit-au și n-au avut nici un folos! Rușinați-vă dar de asemenea venituri ale voastre, pe care le aveți din pricina mâniei celei aprinse a Domnului!"
\par 14 Așa zice Domnul despre toți vecinii mei cei răi, care năpădesc asupra părții pe care El a dat-o de moștenire poporului Său Israel: "Iată, îi voi smulge din țara lor și casa lui Iuda o voi înlătura din mijlocul lor.
\par 15 Dar după ce îi voi smulge, iarăși ,i voi întoarce și-i voi milui și voi aduce pe fiecare la țarina sa și la ogorul său.
\par 16 Și dacă vor învăța ei căile poporului Meu, ca să jure pe numele Meu, zicând: "Viu este Domnul", cum au învățat pe poporul Meu să jure pe Baal, atunci vor fi așezați în mijlocul poporului Meu.
\par 17 Iar de nu vor asculta, atunci îi voi dezrădăcina și voi pierde cu totul poporul acesta", zice Domnul.

\chapter{13}

\par 1 Așa mi-a grăit Domnul: "Mergi și-ți cumpără un brâu de in și-l încinge peste mijlocul tău, dar în apă să nu-l bagi!"
\par 2 Și am cumpărat brâul, după cuvântul Domnului, și l-am încins peste coapsele mele.
\par 3 Apoi a fost cuvântul Domnului iarăși către mine și mi-a zis:
\par 4 "Ia brâul, pe care l-ai cumpărat și care este peste coapsele tale, și, sculându-te, du-te la Eufrat și-l ascunde acolo în crăpătura unei stânci!"
\par 5 Și m-am dus și l-am ascuns la Eufrat, cum îmi poruncise Domnul.
\par 6 Iar după ce au trecut mai multe zile, mi-a zis Domnul: "Scoală, du-te la Eufrat și ia de acolo brâul, pe care ti-am poruncit să-l ascunzi acolo!"
\par 7 Și m-am dus la Eufrat, am săpat și am luat brâul din locul unde-l ascunsesem; dar iată brâul se stricase și nu mai era bun de nimic.
\par 8 Atunci a fost cuvântul Domnului către mine:
\par 9 "Iată ce zice Domnul: Așa voi sfărâma mândria lui Iuda și trufia cea mare a Ierusalimului.
\par 10 Acest popor rău, care nu vrea să asculte cuvintele Mele, ci trăiește după îndărătnicia inimii lui, merge pe urmele altor dumnezei și se închină și slujește lor, va fi ca brâul acesta care nu este bun de nimic.
\par 11 Că precum e brâul aproape de coapsele omului, așa am apropiat Eu de Mine toată casa lui Israel și toată casa lui Iuda, zice Domnul, ca să-Mi fie poporul Meu, numele Meu, lauda Mea și slava Mea, dar ei n-au ascultat.
\par 12 De aceea, spune-le cuvântul acesta: Așa zice Domnul Dumnezeul lui Israel: Tot urciorul se umple de vin. Și ei toți vor zice: Au doară noi nu știm că tot urciorul se umple de vin?
\par 13 Iar tu să le spui: Așa zice Domnul: Iată, Eu voi umple cu vin până la îmbătare pe toți locuitorii țării acesteia, pe regii care șed pe scaunul lui David, pe preoți, pe prooroci și, pe toți locuitorii Ierusalimului,
\par 14 Și-i voi zdrobi pe unii de alții, pe părinți și pe fii laolaltă, zice Domnul. Nu-i voi cruța și nu-i voi milui, nici Îmi va fi milă ca să-i pierd.
\par 15 Ascultați și luați aminte! Nu fiți mândri, căci Domnul grăiește:
\par 16 Dați slavă Domnului Dumnezeului vostru, până nu vine întunericul și până nu se lovesc picioarele voastre de munții nopții. Voi veți aștepta lumina, dar El o va preface în umbra morții, o va preface în negură adâncă.
\par 17 Iar dacă nu ascultați acestea, atunci sufletul meu va plânge în locuri tainice mândria voastră și va plânge amar și ochii mei vor vărsa lacrimi, pentru că turma Domnului va fi dusă în robie.
\par 18 Spune regelui și reginei: Smeriți-vă și ședeți mai jos, deoarece a căzut de pe capul vostru cununa slavei voastre!
\par 19 Cetățile din Negheb sunt închise și n-are cine le deschide. Iuda tot va fi dus în robie; în întregime va fi dus în robie.
\par 20 Ridicați-vă ochii voștri și priviți pe cei ce vin de la miazănoapte! Unde este turma ce ți s-a dat, turma ta cea frumoasă?
\par 21 Fiica Sionului, ce vei zice tu când te vor cerceta, ca biruitori, cei pe care i-ai obișnuit să-ți fie prieteni apropiați? Nu te vor cuprinde durerile ca pe femeia care naște?
\par 22 Iar de vei zice în inima ta: "Pentru ce au venit peste mine acestea?" Ți se va răspunde: Pentru mulțimea nelegiuirilor tale ți s-au desfăcut poalele și ți s-au dezgolit picioarele tale.
\par 23 Poate oare să-și schimbe etiopianul pielea sa și leopardul petele sale? Așa și voi! Puteți oare să faceți bine, când sunteți deprinși a face rău?
\par 24 De aceea Eu vă voi spulbera ca pleava împrăștiată de vântul pustiului.
\par 25 Fiica Sionului, iată soarta ta, plata răzvrătirii tale, măsurată ție de Mine, zice Domnul, pentru că M-ai uitat și ți-ai pus încrederea în minciună.
\par 26 De aceea ți se vor da poalele peste cap, ca să se dezgolească rușinea ta.
\par 27 Văzut-am desfrânarea ta și strigătele tale de plăcere, netrebniciile tale și urâciunile tale de pe dealuri și din câmp. Vai ție, Ierusalime, tu ești necurat! Dar până când...?"

\chapter{14}

\par 1 Cuvântul Domnului care a fost către Ieremia în vremea secetei.
\par 2 Plânge Iuda și porțile Ierusalimului au căzut și stau înnegrite pe pământ, și strigăt se ridică din Ierusalim.
\par 3 Cei mari trimit pe cei mici după apă; aceștia merg la fântâni, dar nu găsesc apă, și se întorc înapoi cu vasele goale; de aceea se rușinează, roșesc și își acopăr capetele.
\par 4 Ogoarele au crăpat, pentru că n-a fost ploaie pe pământ; de aceea plugarii își acoperă capetele și sunt tulburați.
\par 5 Până și cerboaica naște în câmp și își părăsește puii, pentru că nu este iarbă.
\par 6 Și asinii sălbatici stau în locuri înalte și înghit aer, ca șacalii, și ochii li s-au înfundat, pentru că n-au iarbă.
\par 7 Deși fărădelegile noastre mărturisesc împotriva noastră, dar Tu, Doamne, fă milă cu noi pentru numele Tău! Mare este abaterea noastră și am păcătuit înaintea Ta.
\par 8 Nădejdea lui Israel și Izbăvitorul lui la vreme de strâmtorare, pentru ce ești ca un străin în țara aceasta și ca un trecător care se oprește pentru o noapte?
\par 9 Pentru ce ești Tu ca un om fără vlagă și ca un războinic care nu poate ajuta? Totuși Tu, Doamne, ești în mijlocul nostru și numele Tău este chemat asupra noastră: Nu ne lăsa!
\par 10 Așa zice Domnul către poporul acesta: "Pentru că le place să rătăcească și nu-și cruță picioarele, de aceea Domnul nu mai găsește plăcere în ei; pomenește acum fărădelegile lor și numește păcatele lor".
\par 11 Apoi Domnul mi-a spus: "Tu să nu te rogi pentru poporul acesta spre binele lui.
\par 12 De vor posti, nu voi auzi strigarea lor; de vor aduce arderi de tot și prinoase, nu voi primi, ci cu sabie, cu foamete și cu molimă îi voi pierde".
\par 13 Atunci am zis: "Doamne Dumnezeule! Iată ce le grăiesc proorocii: Nu veți vedea sabie și foamete nu va fi la voi, ci voi da în locul acesta pace necontenită".
\par 14 Iar Domnul mi-a răspuns: "Proorocii proorocesc lucruri mincinoase în numele Meu; Eu nu i-am trimis, nici nu le-am dat poruncă și nici nu le-am grăit; ci ei vă vestesc vedenii mincinoase, proorociri deșarte și închipuiri ale inimii lor".
\par 15 De aceea, așa zice Domnul despre prooroci: "Ei proorocesc în numele Meu, dar Eu nu i-am trimis; ei zic: "Sabie și foamete nu va fi în țara aceasta", dar de sabie și de foamete vor pieri acești prooroci și poporul căruia au proorocit ei.
\par 16 Va fi risipit pe ulițele Ierusalimului de foamete și de sabie, și nu va avea cine să-i îngroape pe ei și pe femeile lor, pe fiii lor și pe fiicele lor, că Eu voi vărsa asupra lor răutatea lor.
\par 17 Și să le mai spui cuvântul acesta: Ochii Mei varsă lacrimi ziua și noaptea și nu se opresc! Căci cu bătaie mare a fost bătută fecioara, fiica poporului Meu, și cu lovitură grea.
\par 18 De ies în câmp, iată numai oameni uciși cu sabia! De intru în cetate, iată numai oameni istoviți de foame! Chiar și proorocul și preotul rătăcesc prin țară fără să știe unde merg".
\par 19 Atunci am zis iarăși: "Lepădat-ai Tu oare cu totul pe Iuda? Au doar Te-a dezgustat cu totul Sionul? Pentru ce ne-ai lovit așa, încât nu mai avem leac? Așteptăm pace, dar iată nu vine nimic bun! Așteptăm vremea vindecării, și iată grozăvie!
\par 20 Mărturisim, Doamne, necredința noastră și fărădelegile părinților noștri, că am păcătuit înaintea Ta.
\par 21 Nu ne lepăda pe noi pentru numele Tău! Nu necinsti tronul slavei Tale! Adu-ți aminte și nu strica legământul Tău cu noi!
\par 22 Sunt oare printre dumnezeii deșerți ai neamurilor dătători de ploaie? Sau poate oare cerul singur să verse apă? Au nu ești Tu, Doamne Dumnezeule, Cel ce dai ploaie? În Tine nădăjduim, că Tu faci toate!"

\chapter{15}

\par 1 Și mi-a zis iarăși Domnul: "Chiar Moise și Samuel de ar sta înaintea Mea, sufletul Meu tot nu s-ar îndupleca spre poporul acesta. Izgonește-i de la fața Mea, ca să se ducă.
\par 2 Iar de-ți vor zice: "Unde să ne ducem?" să le spui: Așa zice Domnul: "Cel rânduit pentru moarte să se ducă la moarte, cel pentru sabie, la sabie, cel pentru foamete, la foamete, și cel pentru robie, în robie!
\par 3 Voi trimite asupra lor patru feluri de pedepse, zice Domnul: sabia, ca să-i taie, câinii, ca să-i sfâșie, păsările cerului și fiarele câmpului, ca să-i piardă.
\par 4 Și-i voi da spre muncă la toate regatele pământului, pentru Manase, fiul lui Iezechia, regele lui Iuda, și pentru cele ce a făcut el în Ierusalim.
\par 5 Oare cui îi va părea rău de tine, Ierusalime, și cine-ți va arăta milă sau cine va veni la tine și te va întreba de sănătate?
\par 6 Pentru că tu M-ai lăsat, zice Domnul, și te-ai întors înapoi, de aceea-Mi voi întinde mâna Mea asupra ta și te voi pierde, că M-am săturat miluindu-te.
\par 7 Cu vânturătoarea îi voi vântura la porțile țării, îi voi lipsi de copii și voi pierde pe poporul Meu, dar tot nu se vor întoarce din căile lor.
\par 8 Văduvele lor sunt mai multe decât nisipul mării. Voi aduce un pustiitor în plină amiază asupra lor, asupra mamelor celor tinere, și va cădea asupra lor, fără de veste, frică și groază.
\par 9 Cea care născuse șapte copii zace în neputință, își dă duhul și-i apune soarele încă ziuă fiind; este rușinată și ocărâtă. Pe cei rămași îi voi da sabiei înaintea ochilor vrăjmașilor lor", zice Domnul.
\par 10 Vai de mine, mamă, că m-ai născut să fiu om de ceartă și de pricină pentru toată țara! Nimănui n-am dat cu dobândă și nici mie nu mi-a dat nimeni cu dobândă, și tot mă blestemă toți.
\par 11 Zis-a Domnul: "Da, te voi întări pentru binele tău; singur voi conduce pe vrăjmașul tău să te roage la vreme de nenorocire și de restriște.
\par 12 Poate cineva să rupă fierul, fierul de la miazănoapte și arama?
\par 13 Averea ta și comorile tale le voi da pradă fără plată, pentru toate păcatele tale, în toate hotarele tale;
\par 14 Și le voi trimite cu vrăjmașii tăi într-o țară pe care tu n-o cunoști, că s-a aprins focul mâniei Mele și va arde peste voi".
\par 15 O, Doamne, Tu știi toate! Adu-ți aminte de mine, cercetează-mă și mă răzbună împotriva prigonitorilor mei! Nu mă pierde după îndelungata Ta răbdare, știind că pentru Tine sufăr ocara.
\par 16 Aflat-am cuvintele Tale și le-am sorbit și cuvântul Tău a fost bucurie și veselie pentru inima mea, că s-a chemat asupra mea numele Tău, Doamne Dumnezeul puterilor.
\par 17 În adunarea celor ce râd n-am șezut, nici m-am veselit, ci am stat singur sub mâna Ta ce apasă asupra mea, că Tu mă umpluseși de mânie.
\par 18 De ce este așa de grea boala mea și de ce rana mea este așa de greu de vindecat, încât nu suferă doctorie? Oare vei fi Tu pentru mine ca un izvor amăgitor și ca o apă înșelătoare?
\par 19 Și la acestea Domnul mi-a răspuns așa: "De te vei întoarce, Eu te voi aduce la starea cea dintâi și vei sta înaintea feței Mele; dacă tu vei deosebi lucrul de preț de cel fără de preț, vei fi ca gura Mea; și nu te vei întoarce la ei, ci ei se vor întoarce la tine;
\par 20 Și te voi face pentru poporul acesta zid tare de aramă. Vor lupta împotriva ta, dar nu te vor birui, pentru că Eu sunt cu tine, ca să te scap și să te izbăvesc, zice Domnul.
\par 21 Și te voi scăpa din mâna celor răi; te voi izbăvi din mina asupritorilor".

\chapter{16}

\par 1 Și a fost cuvântul Domnului către mine:
\par 2 "Să nu-îi iei femeie și să nu ai nici fii, nici fiice în locul acesta,
\par 3 Deoarece așa grăiește Domnul despre fiii și fiicele care se vor naște în locul acesta, despre mamele care îi vor naște și despre părinții care îi vor face pe pământul acesta:
\par 4 Vor muri de moarte grea și nu vor fi nici bociți, nici îngropați, ci vor fi ca gunoiul pe fața pământului; vor fi pierduți prin sabie și foamete și trupurile lor vor fi mâncarea păsărilor cerului și fiarelor pământului".
\par 5 Și iarăși a zis Domnul: "Să nu intri în casa celor ce jelesc și să nu te duci să plângi și să jelești cu ei, căci am luat de la poporul acesta pacea Mea, mila și părerea de rău, zice Domnul.
\par 6 Și vor muri cei mari și cei mici în pământul acesta, și nu vor fi îngropați, și după ei nimeni nu va plânge, nimeni nu-și va face tăieturi, nici se va tunde pentru ei.
\par 7 Nu se va frânge pentru ei pâine de jale ca mângâiere pentru cel mort; și nu li se va da cupa mângâierii ca să bea după tatăl lor și după mama lor.
\par 8 De asemenea să nu intri în casa ospățului, ca să șezi cu ei să mănânci și să bei, că așa zice Domnul Savaot, Dumnezeul lui Israel:
\par 9 Iată, voi curma în locul acesta chiar în zilele voastre și sub ochii voștri glasul bucuriei și glasul de veselie, glasul mirelui și glasul miresei.
\par 10 Când vei spune poporului acestuia toate acestea și când ei îți vor zice: "Pentru ce a rostit Domnul asupra noastră această mare nenorocire? Care este nedreptatea noastră și care este păcatul cu care am păcătuit noi înaintea Domnului Dumnezeului nostru?"
\par 11 Atunci să le spui: Pentru că părinții voștri M-au părăsit, zice Domnul, și s-au dus după alți dumnezei, au slujit acelora și li s-au închinat, iar pe Mine M-au părăsit și legea Mea n-au păzit-o.
\par 12 Dar voi faceți încă și mai rău decât părinții voștri, și trăiți fiecare după inima voastră cea rea și îndărătnică și nu Mă ascultați.
\par 13 De aceea vă voi arunca din țara aceasta într-o rară pe care n-ați cunoscut-o nici voi, nici părinții voștri, și veți sluji acolo ziua și noaptea la alți dumnezei, că Eu nu vă voi arăta îndurare.
\par 14 Pentru că vin zile, zice Domnul, când nu se va mai zice: "Viu este Domnul, Care a scos pe fiii lui Israel din țara Egiptului",
\par 15 Ci, "Viu este Domnul, Care a scos pe fiii lui Israel din țara cea de la miazănoapte și din toate țările în care-i izgonise", că îi voi întoarce în țara pe care le-am dăruit-o părinților lor.
\par 16 Iată, voi trimite mulțime de pescari, zice Domnul, și-i vor pescui; iar apoi voi trimite mulțime de vânători, și-i vor vâna de prin toți munții, de pe toate dealurile și de prin crăpăturile stâncilor.
\par 17 Pentru că ochii Mei sunt asupra tuturor căilor lor, care nu sunt ascunse de la fața Mea, și nedreptatea lor nu se ascunde privirilor Mele.
\par 18 Și le voi răsplăti mai întâi pentru nedreptatea lor și pentru îndoitul lor păcat, pentru că au spurcat țara Mea eu stârvurile grozăviilor lor și cu urâciunile lor au umplut moștenirea Mea".
\par 19 Doamne, puterea mea, tăria mea și scăparea mea la vreme de necaz, la Tine vor veni popoarele de la marginile pământului și vor zice: "Numai minciună au moștenit părinții voștri, idoli deșerți, care nu sunt de nici un folos!"
\par 20 "Poate oare omul să-și facă dumnezei, care, de altfel, nici nu sunt dumnezei?
\par 21 De aceea iată le voi arăta acum, le voi arăta mâna Mea și puterea Mea și vor afla că numele Meu este Domnul!"

\chapter{17}

\par 1 "Păcatul lui Iuda este scris cu condei de fier, cu vârf de diamant este săpat pe lespedea inimii lor și pe coarnele jertfelnicelor lor.
\par 2 Ca de copiii lor își aduc aminte de jertfelnicele lor și de locurile de jertfă de pe sub copacii verzi și de pe vârfurile dealurilor.
\par 3 Ierusalime, muntele Meu, câmpul, avuția și toate comorile tale le voi da spre pradă; voi da și locurile tale înalte din toate hotarele tale, pentru păcatele tale.
\par 4 Și tu prin tine însuți te vei lipsi de moștenirea ta, pe care ți-am dat-o Eu și te voi da în robie vrăjmașilor tăi, în țara pe care tu n-o știi, pentru că ai aprins focul mâniei Mele și în veci va arde".
\par 5 Așa zice Domnul: "Blestemat fie omul care se încrede în om și își face sprijin din trup omenesc și a cărui inimă se depărtează de Domnul.
\par 6 Acela va fi ca ierburile pustiului și nu va vedea când va veni binele, ci va locui în locurile arse ale pustiului, în pământ neroditor și nelocuit.
\par 7 Binecuvântat fie omul care nădăjduiește în Domnul și a cărui nădejde este Domnul,
\par 8 Deoarece acesta va fi ca pomul sădit lângă ape, care-și întinde rădăcinile pe lângă râu și nu știe când vine arșița; frunzele lui sunt verzi, la timp de secetă nu se teme și nu încetează a rodi.
\par 9 Inima omului este mai vicleană decât orice și foarte stricată! Cine o va cunoaște!
\par 10 Eu, Domnul, pătrund inima și încerc rărunchii, ca să răsplătesc fiecăruia după căile lui și după roada faptelor lui.
\par 11 Prepelița clocește ouăle pe care nu le-a ouat; așa este și cel ce câștigă avuție nedreaptă, o lasă la jumătatea zilelor sale și la sfârșitul său se va trezi că este un nebun".
\par 12 Tronul slavei, înălțat de la început, este locul sfințirii noastre.
\par 13 Tu, Doamne, ești nădejdea lui Israel! Toți cei ce Te părăsesc se vor rușina, că Tu ai zis: "Cei ce se depărtează de Mine vor fi scriși pe pulbere, pentru că au părăsit pe Domnul, izvorul apei celei vii".
\par 14 Vindecă-mă, Doamne, și voi fi vindecat; mântuiește-mă și voi fi mântuit, căci Tu ești lauda mea!
\par 15 Iată acestea-mi zic ei: "Unde este cuvântul Domnului? Să vină!"
\par 16 Eu nu te-am îndemnat totuși la mai rău, nici n-am dorit ziua nenorocirii. Tu știi acestea; și ce a ieșit din gura mea e descoperit înaintea feței Tale.
\par 17 Nu fi pricină de groază pentru mine, că Tu ești nădejdea mea în ziua strâmtorării.
\par 18 Să nu mă rușinez eu, ci să se rușineze apăsătorii mei; ei să tremure, dar să nu tremur eu! Adu asupra lor ziua necazului și zdrobește-i cu zdrobire îndoită.
\par 19 Așa mi-a zis Domnul: "Du-te și stai la poarta fiilor poporului, pe care intră regii lui Iuda, și pe care ies ei, și la toate porțile Ierusalimului,
\par 20 Și le spune: Regii lui Iuda și tot Iuda și toți locuitorii Ierusalimului, care intrați pe porțile acestea, ascultați cuvântul acesta:
\par 21 Așa grăiește Domnul: "Păziți-vă sufletele și nu duceți sarcini în ziua de odihnă, nici le băgați pe porțile Ierusalimului;
\par 22 Nu scoateți sarcini din casele voastre în ziua odihnei și nu vă îndeletniciți cu nici un fel de muncă, ci sfințiți ziua odihnei, așa cum am poruncit Eu părinților voștri,
\par 23 Care n-au ascultat și nu și-au plecat urechea, ci și-au învârtoșat neascultarea, ca să nu îndeplinească și să nu ia învățătură.
\par 24 De Mă veți asculta, zice Domnul, și nu veți aduce sarcini pe porțile acestei cetăți în ziua odihnei; ci veți sfinți ziua odihnei și nu vă veți îndeletnici în acea zi cu nici o muncă,
\par 25 Atunci pe porțile acestei cetăți vor intra regi care șed pe scaunul lui David, în alai de care și de cai, ei și căpeteniile oștirii lor, oamenii din Iuda și locuitorii Ierusalimului; și cetatea aceasta în veac va fi locuită.
\par 26 Și vor veni din cetățile lui Iuda, din împrejurimile Ierusalimului și din pământul lui Veniamin, din șes, din munți și de la miazăzi, și vor aduce arderi de tot și jertfe, prinoase și tămâie și jertfe de mulțumire în templul Domnului.
\par 27 Iar de nu Mă veți asculta ca să sfințiți ziua odihnei și să nu duceți sarcini, când intrați pe porțile Ierusalimului în ziua odihnei, voi aprinde foc la porțile lui și va arde palatele Ierusalimului și nu se va stinge".

\chapter{18}

\par 1 Cuvântul care a fost de la Domnul către Ieremia și i-a zis:
\par 2 "Scoală și intră în casa olarului și acolo îți voi vesti cuvintele Mele!"
\par 3 Și am intrat eu în casa olarului și iată acesta lucra cu roata
\par 4 Și vasul pe care-l făcea olarul din lut s-a stricat în mâna lui; dar olarul a făcut dintr-însul alt vas, cum a crezut că-i mai bine să-l facă.
\par 5 Și a fost cuvântul Domnului iarăși către mine și mi-a zis: "Casa lui Israel, oare nu pot să fac și Eu cu voi, ca olarul acesta, zice Domnul?
\par 6 Iată, ce este lutul în mâna olarului, aceea sunteți și voi în mâna Mea, casa lui Israel!
\par 7 Dacă voi zice cândva despre un popor, sau despre un rege, că-l voi dezrădăcina, îl voi sfărâma și-l voi pierde;
\par 8 Și dacă poporul acela, despre care am zis Eu acestea, se va întoarce de la faptele lui cele rele, atunci voi îndepărta răul ce gândeam să-i fac.
\par 9 Sau dacă voi zice despre un popor sau despre un rege că-l voi întocmi și-l voi întări,
\par 10 Și dacă acela va face rele înaintea ochilor Mei și nu va asculta de glasul Meu, atunci voi schimba binele cu care voiam să-l fericesc.
\par 11 Spune deci bărbaților lui Iuda și locuitorilor Ierusalimului: Așa zice Domnul: Iată, Eu vă gătesc rele și uneltiri împotriva voastră. Așadar să se întoarcă fiecare de la calea lui cea rea; îndreptați-vă căile și purtările voastre!
\par 12 Dar ei zic: "Este zadarnic! Noi vom trăi după gândul nostru și ne vom purta fiecare după învârtoșarea inimii noastre celei rele".
\par 13 De aceea, așa zice Domnul: "Întrebați popoarele: Auzit-a oare cineva asemenea lucru? Lucruri peste măsură de urâcioase a făcut fecioara lui Israel.
\par 14 Părăsește oare zăpada Libanului stânca muntelui? Ori seacă apele ce vin de departe și sunt reci și curgătoare?
\par 15 Poporul Meu însă M-a părăsit. Tămâiază idoli, s-a poticnit în căile sale și a părăsit căile cele vechi, ca să umble pe poteci și pe drumuri nebătătorite, ca să-și facă țara grozăvie și batjocură veșnică,
\par 16 Încât tot cel ce va trece prin aceasta să se mire și să clatine din cap.
\par 17 Îi voi spulbera înaintea vrăjmașilor ca vântul cel de la răsărit, și nu fața, ci spatele îl voi întoarce spre ei în ziua necazului lor".
\par 18 Zis-au ei: "Veniți să uneltim împotriva lui Ieremia, că nu va pieri legea din mâna preotului, nici sfatul de la înțelept, nici cuvântul (lui Dumnezeu) de la prooroc. Veniți să-l biruim cu limba și să nu luăm aminte la cuvintele lui!
\par 19 Ia aminte la mine, Doamne, și auzi glasul potrivnicilor mei!
\par 20 Se cuvine oare a răsplăti cu rău pentru bine? Ei însă sapă groapă sufletului meu. Adu-ți aminte că stau înaintea feței Tale, ca să grăiesc bine de ei și ca să abat de la ei mânia Ta.
\par 21 Dă dar pe fiii lor la foamete și pune-i sub sabie! Femeile lor să fie fără copii și văduve, bărbații lor să fie loviți de moarte și tinerii lor să fie uciși cu sabia. în război.
\par 22 Bocete să se audă prin casele lor, când fără de veste vei aduce oștiri asupra lor, că sapă groapă ca să mă prindă și pe ascuns au întins curse pentru picioarele mele.
\par 23 Dar Tu, Doamne, știi tot ce uneltesc ei împotriva mea ca să mă omoare! Nu ierta nedreptatea lor și păcatul lor nu-l șterge dinaintea feței Tale! Doboară-i înaintea Ta și lucrează împotriva lor la vremea mâniei Tale!

\chapter{19}

\par 1 Așa a zis Domnul: "Mergi și cumpără o oală de lut de la olar; ia cu tine pe cei mai bătrâni din popor și din căpeteniile preoților și ieși în valea Ben-Hinom, care este la Poarta Olăriei.
\par 2 Acolo să rostești cuvintele pe care ti le voi spune și să zici:
\par 3 "Regi ai lui Iuda și locuitori ai Ierusalimului, ascultați cuvântul Domnului! Așa grăiește Domnul Savaot, Dumnezeul lui Israel: Iată voi aduce așa strâmtorare asupra lacului acestuia, încât, auzind cineva, să-i țiuie urechile,
\par 4 Pentru că M-au părăsit, au înstrăinat lacul acesta și tămâiază pe el alți dumnezei, pe care nu i-au cunoscut nici ei, nici părinții lor și nici regii lui Iuda; au umplut locul acesta de sângele nevinovaților și au făcut înălțimi pentru Baal, ca să ardă pe fiii lor cu foc.
\par 5 Ei aduc ardere de tot pentru Baal, ceea ce Eu nu le-am poruncit, nici le-am grăit și ceea ce nici prin minte nu Mi-a trecut.
\par 6 De aceea iată vin zile, zice Domnul, când locul acesta nu se va mai chema Tofet sau valea fiilor lui Hinom, ci valea uciderii.
\par 7 Voi pierde cu desăvârșire pe Iuda și Ierusalimul în locul acesta și-i voi răpune cu sabia înaintea vrăjmașilor lor și cu mâna celor ce caută sufletul lor, și voi da trupurile lor spre hrană păsărilor cerului și fiarelor pământului.
\par 8 Voi face cetatea aceasta pustie și batjocură. Tot cel ce va trece prin ea se va mira și va fluiera a pustiu, văzând toate rănile ei.
\par 9 Și-i voi ospăta cu carnea fiilor lor și cu carnea fiicelor lor; și va mânca fiecare carnea aproapelui său, fiind în împresurare și în strâmtorare, când îi vor strâmtora vrăjmașii lor și cei ce vor să le ia viața.
\par 10 Apoi să spargi oala înaintea ochilor bărbaților acelora, care vor merge cu tine, și să le zici:
\par 11 "Iată ce zice Domnul Savaot: Voi sfărâma poporul acesta și cetatea aceasta, așa cum am sfărâmat vasul olarului, care nu mai poate fi făcut la loc, și-i vor îngropa în Tofet din pricina lipsei de loc pentru îngropare.
\par 12 Așa voi face cu locul acesta, zice Domnul, și cu locuitorii lui, și voi face cetatea aceasta ca Tofetul.
\par 13 Și casele Ierusalimului și casele regilor lui Iuda vor fi necurate, ca Tofetul, pentru că pe acoperișul tuturor caselor s-aduce tămâie întregii oștiri cerești și se săvârșesc turnări în cinstea dumnezeilor străini".
\par 14 Și s-a întors Ieremia din Tofet, unde-l trimisese Domnul să proorocească, și a stat în curtea templului Domnului și a zis către tot poporul:
\par 15 "Așa zice Domnul Savaot, Dumnezeul lui Israel: Iată voi aduce asupra cetății acesteia și asupra celorlalte cetăți toate nenorocirile pe care le-am rostit împotriva ei, pentru că și-au învârtoșat inima și nu ascultă cuvintele Mele".

\chapter{20}

\par 1 Auzind Pașhurr, fiul preotului Imer, care era supraveghetor în templul Domnului, că Ieremia a profețit aceste cuvinte,
\par 2 A lovit Pașhurr pe Ieremia proorocul și l-a aruncat în temnița care se afla la poarta cea de sus a lui Veniamin, în templul Domnului.
\par 3 Dar a doua zi Pașhurr a dat drumul lui Ieremia din temniță și Ieremia i-a zis: "Domnul nu te mai numește Pașhurr (Noroc din toate părțile), ci Magor Misabib (Spaimă din toate părțile),
\par 4 Pentru că așa zice Domnul: Iată, te voi face groază și pentru tine însuți și pentru toți prietenii tăi; aceștia vor cădea de sabia vrăjmașilor lor și ochii tăi vor vedea aceasta. Și pe tot Iuda îl voi da în mâinile regelui Babilonului, care-i va duce la Babilon și-i va lovi cu sabia.
\par 5 Și toată bogăția cetății acesteia, toată agonisita ei, toate cele de preț ale ei și toate comorile regilor lui Iuda le voi da în mâinile vrăjmașilor lor, care, jefuindu-le, le vor lua și le vor duce la Babilon.
\par 6 Iar tu, Pașhurr, cu toți cei ce trăiesc în casa ta, vă veți duce în robie și, mergând la Babilon, vei muri acolo și acolo vei fi îngropat și tu și toți prietenii tăi, cărora tu le-ai proorocit mincinos.
\par 7 Doamne, Tu m-ai aprins și iată sunt înflăcărat; Tu ești mai tare decât mine și ai biruit, iar eu în toate zilele sunt batjocorit și fiecare își bate joc de mine;
\par 8 Că de când vorbesc, scoțând strigăte împotriva silniciei și rostind pustiirea, cuvântul Domnului s-a prefăcut în ocară pentru mine și în batjocură zilnică.
\par 9 De aceea mi-am zis: "Nu voi mai pomeni de El și nu voi mai grăi, în numele Lui!" Dar iată era în inima mea ceva, ca un fel de foc aprins, închis în oasele mele, și eu mă sileam să-l înfrânez și n-am putut;
\par 10 Că am auzit ocări de la mulți și amenințări din toate părțile, zicând: "Pârâți-l și-l vom pârî și noi!" Toți cei ce trăiau în pace cu mine, mă pândesc să vadă nu cumva mă voi poticni, și ziceau: "Poate va cădea și-l vom birui și ne vom răzbuna pe el!"
\par 11 Dar Domnul este cu mine, ca un apărător puternic. De aceea prigonitorii mei se vor poticni și nu vor birui; se vor face de rușine, pentru că n-au izbutit; ocara lor va fi veșnică și niciodată nu se va uita.
\par 12 Doamne al puterilor, Cel ce cercetezi cu dreptate și pătrunzi rărunchii și inimile, fă-mă să văd răzbunarea Ta asupra lor, că ție ți-am încredințat pricina mea!
\par 13 Cântați Domnului! Lăudați pe Domnul, căci El izbăvește sufletul celui împilat din mâna făcătorilor de rele.
\par 14 Blestemată fie ziua în care m-am născut, ziua în care m-a născut maică-mea să nu fie binecuvântată.
\par 15 Blestemat fie omul care a adus tatălui meu veste și a zis: "Ți s-a născut fiu", și s-a bucurat mult de aceasta.
\par 16 Întâmple-se omului aceluia ceea ce s-a întâmplat cetăților pe care le-a dărâmat Domnul și nu le-a cruțat! Să audă el dimineața bocet și la amiază tânguire mare,
\par 17 Că nu m-a ucis chiar din pântece, ca mama mea să-mi fi fost mormânt și pântecele ei să fi rămas veșnic însărcinat.
\par 18 Pentru ce am ieșit eu din pântece, ca să văd suferință și durere și ca zilele mele să se sfârșească în rușine?"

\chapter{21}

\par 1 Cuvântul care a fost de la Domnul către Ieremia, când a trimis regele Sedechia la el pe Pașhurr, fiul lui Malchia, și pe Sofonie, fiul preotului Maaseia, ca să-i zică:
\par 2 "Întreabă pentru noi pe Domnul, că Nabucodonosor, regele Babilonului, ne face război. Poate că va face Domnul cu noi ceva în felul minunilor Lui, ca să se depărteze de la noi".
\par 3 Iar Ieremia le-a răspuns: "Așa să spuneți lui Sedechia:
\par 4 Domnul Dumnezeul lui Israel așa zice: "Iată, voi întoarce înapoi armele de război, care sunt în mâinile voastre și cu care vă luptați cu regele Babilonului și cu Caldeii, care vă impresară pe din afară de ziduri, și le voi aduna în mijlocul cetății acesteia;
\par 5 Și voi lupta și Eu Însumi împotriva voastră cu mâna întinsă și cu braț puternic, cu mânie, cu urgie și cu multă furie.
\par 6 Și voi lovi pe cei ce trăiesc în cetatea aceasta de la oameni până la animale și vor muri de ciumă cumplită.
\par 7 Iar după aceea, zice Domnul, pe Sedechia, regele lui Iuda, pe slugile lui, pe popor și pe cei ce au rămas în cetatea aceasta pe urma ciumei, a sabiei și a foametei, îi voi da în mâinile lui Nabucodonosor, regele Babilonului, și în mâinile vrăjmașilor lor și în mâinile celor ce vor să le ia viața; și acela îi va lovi cu ascuțișul sabiei și nu-i va cruța, nici se va îndura să-i miluiască.
\par 8 Iar poporului acestuia spune-i: Așa zice Domnul: Iată, vă pun înainte calea vieții și calea morții:
\par 9 Cine va rămâne în cetatea aceasta, acela va muri de sabie și de foamete și de ciumă; iar cine va ieși și se va preda Caldeilor, care vă împresoară, acela va trăi și va fi luat ca pradă,
\par 10 Că Eu Mi-am întors fața împotriva cetății acesteia, zice Domnul, în rău, nu în bine; și va fi dată în mâinile regelui Babilonului, care o va arde cu foc.
\par 11 Iar casei regelui Sedechia să-i spui: Ascultați cuvântul Domnului!
\par 12 Casa lui David, așa zice Domnul: Faceți judecată dis-de-dimineață și scăpați pe cel asuprit din mâna asupritorului, pentru ca să nu izbucnească mânia Mea ca focul și pentru ca să nu se aprindă din pricina faptelor voastre cele rele, așa încât nimeni să n-o stingă.
\par 13 Cetate a văii și stâncă din câmp, iată sunt împotriva ta, zice Domnul! O, voi, care ziceți: "Cine se va ridica împotriva noastră și cine va intra în sălașul nostru?"
\par 14 Iată, sunt împotriva voastră! Dar Eu vă voi pedepsi după roadele faptelor voastre, zice Domnul, și voi aprinde foc în pădurea voastră și voi mistui totul împrejurul ei".

\chapter{22}

\par 1 Așa a zis Domnul: "Coboară-te în casa regelui lui Iuda și rostește cuvintele acestea și zi:
\par 2 Regele lui Iuda, cel ce șezi pe scaunul lui David, ascultă cuvântul Domnului și tu și slugile tale și poporul tău, care intrați pe porțile acestea!
\par 3 Așa zice Domnul: Faceți judecată și dreptate și scoateți pe cel asuprit din mâna asupritorului, nu asupriți și nu împilați pe străin, pe orfan și pe văduvă și sânge nevinovat să nu vărsați în locul acesta!
\par 4 Căci, de veți împlini cuvântul acesta, vor intra pe porțile casei acesteia regii cei ce șed în locul lui David, pe scaunul lui, și umblă în care și pe cai, vor intra și ei și slugile lor și poporul lor.
\par 5 Iar de nu veți asculta cuvintele acestea, Mă jur pe Mine Însumi, zice Domnul, că această casă va ajunge pustie.
\par 6 De aceea, așa zice Domnul casei regelui lui Iuda: Galaad Îmi ești și vârf de Liban, dar te voi face pustietate și cetăți nelocuite.
\par 7 Și voi găti împotriva ta pierzători, care vor avea fiecare arma sa și var tăia cei mai frumoși cedri ai tăi și-i vor arunca în foc.
\par 8 Multe popoare vor trece prin cetatea aceasta și vor zice unii către alții: "Pentru ce a făcut Domnul așa cu această cetate mare?"
\par 9 Și li se va răspunde: "Pentru că locuitorii ei au părăsit legământul Domnului Dumnezeului lor, s-au închinat la alți dumnezei și au slujit acelora".
\par 10 Nu plângeți după mort și nu-l bociți, ci plângeți amar după cel dus în robie, că acela nu se va mai întoarce și nu își va mai vedea țara sa de naștere;
\par 11 Căci așa zice Domnul despre Șalum, fiul lui Iosia, regele lui Iuda, care a domnit după tatăl său, Iosia, și care a ieșit din locul acesta: "Nu se va mai întoarce acolo,
\par 12 Ci va muri în locul acela unde a fost dus rob și nu va mai vedea pământul acesta.
\par 13 Vai de cel care își zidește casa din nedreptate și își face încăperi din fărădelegi, care silește pe aproapele său să-i lucreze degeaba și nu-i dă plata lui,
\par 14 Și care zice: "Am să-mi fac casă mare și odăi încăpătoare, am să fac ferestre, am să le căptușesc cu cedru și am să le vopsesc cu roșu!
\par 15 Ai ajuns tu, oare, rege ca să te fălești cu palate clădite din lemn de cedru? Tatăl tău n-a mâncat oare și n-a băut? Israel a făcut judecată și dreptate și de aceea i-a fost bine.
\par 16 El a judecat pricina săracului și a nenorocitului și de aceea i-a fost bine. Oare nu aceasta înseamnă a Mă cunoaște pe Mine? - zice Domnul.
\par 17 Inima ta însă și ochii tăi caută numai la lucrul tău și la vărsarea sângelui nevinovat; caută să facă numai împilare și silnicie.
\par 18 De aceea, așa zice Domnul despre Ioiachim, fiul lui Iosia, regele lui Iuda: "Nu-l vor plânge, zicând: "Vai, fratele meu!" sau: "Vai, sora mea!" Nu-l vor plânge, zicând: "Vai, doamne!" sau: "Vai, cinstea ta!"
\par 19 Ci el va fi îngropat ca un asin; îl vor târî și-l vor arunca departe peste porțile Ierusalimului.
\par 20 Urcă-te pe Liban și strigă, înalță-ți glasul de pe Vasan și strigă de pe Abarim, că s-au zdrobit toți prietenii tăi.
\par 21 În vremea propășirii tale Eu ți-am grăit, dar tu ai zis: "Nu ascult!" Așa a fost purtarea ta chiar din tinerețea ta și n-ai ascultat glasul Meu.
\par 22 Pe toți păstorii tăi îi va împrăștia vântul, iar prietenii tăi se vor duce în robie. Atunci vei fi rușinat și batjocorit pentru toate faptele tale cele rele.
\par 23 O, cel ce locuiești pe Liban și-ți faci cuibul în cedri, cum vei geme tu, când te vor ajunge chinurile ca durerile femeii ce naște!
\par 24 Precum e adevărat că Eu trăiesc, a zis Domnul, tot așa este de adevărat că dacă Iehonia, fiul lui Ioiachim, regele lui Iuda, ar fi inel la mâna Mea cea dreaptă, apoi și de acolo l-aș smulge.
\par 25 Și te voi da în mâinile celor ce vor să-ți ia viața și în mâinile celor de care tu te temi, în mâinile lui Nabucodonosor, regele Babilonului, și în mâinile Caldeilor;
\par 26 Și te voi arunca pe tine și pe mama ta, care te-a născut, în .ară străină, unde nu v-ați născut
\par 27 și veți muri acolo; iar în pământul unde va dori sufletul vostru să se întoarcă, nu vă veți mai întoarce".
\par 28 Au doară acest om, Iehonia, este făptură blestemată și lepădată? Sau este vas netrebnic? Pentru ce sunt aruncați, el și neamul lui, și încă într-o țară pe care ei n-o știau?
\par 29 O, țară, țară, țară, ascultă cuvântul Domnului!
\par 30 Așa zice Domnul: "Scrieți pe omul acesta ca lipit de copii, ca om nenorocit în zilele sale, pentru că nimeni din neamul lui nu va mai ședea pe tronul lui David și să domnească peste Iuda!"

\chapter{23}

\par 1 "Vai de părinții care pierd și împrăștie oile turmei Mele, zice Domnul.
\par 2 De aceea, așa zice Domnul Dumnezeul lui Israel către păstorii care pasc pe poporul Meu: Ați risipit oile Mele și le-ați împrăștiat și nu le-ați păzit; de aceea, vă voi pedepsi pentru faptele voastre cele rele, zice Domnul.
\par 3 Și voi aduna rămășițele turmei Mele din toate țările prin care le-am risipit, le voi întoarce la staulele lor și vor naște și se vor înmulți.
\par 4 Voi pune peste acestea păstori, care le vor paște și ele nu se vor mai teme, nici se vor mai speria, nici se vor mai pierde, zice Domnul.
\par 5 Iată vin zile, zice Domnul, când voi ridica lui David Odraslă dreaptă și va ajunge rege și va domni cu înțelepciune; va face judecată și dreptate pe pământ.
\par 6 În zilele Lui, Iuda va fi izbăvit și Israel va trăi în liniște; iată numele cu care-L voi numi: "Domnul-dreptatea-noastră!"
\par 7 De aceea, vor veni zilele, zice Domnul, când nu se va mai zice: "Viu este Domnul, Care a scos pe fiii lui Israel din țara Egiptului",
\par 8 Ci: "Viu este Domnul, Care a scos și a adus neamul casei lui Israel din țara de la miazănoapte și din toate țările în care îi risipise și vor trăi în pământul acesta".
\par 9 Despre prooroci: "Mi se sfâșie inima în mine și toate oasele mi se cutremură; sunt ca un am beat, ca omul biruit de vin, pentru Domnul și pentru cuvintele Lui cele sfinte;
\par 10 Pentru că s-a umplut țara de desfrânați și pentru că plânge țara sub blestem; uscatu-s-au pășunile în stepă; ținta alergăturii lor este fărădelegea și țara lor este nedreptatea.
\par 11 Că profetul și preotul sunt necredincioși; și până și în casa Mea am găsit răutatea lor, zice Domnul.
\par 12 De aceea, calea lor le va fi ca locurile lunecoase în întuneric; vor fi îmbrânciți și vor cădea acolo, căci voi aduce asupra lor nenorocirea în anul când îi voi pedepsi, zice Domnul.
\par 13 Și în proorocii Samariei am văzut nebunie, căci au profețit în numele lui Baal și au dus în rătăcire pe poporul Meu Israel.
\par 14 Dar în proorocii Ierusalimului am văzut grozăvii: aceștia fac desfrânare și umblă cu minciuni, ajută mâinile făcătorilor de rele, ca nimeni să nu se întoarcă de la necredința sa; toți sunt pentru Mine ca Sodoma, și locuitorii lui ca Gomora.
\par 15 De aceea, așa zice Domnul Savaot despre prooroci: Iată, îi voi hrăni cu pelin și le voi da să bea apă cu fiere, că de la profeții Ierusalimului s-a întins necredința în toată țara.
\par 16 Așa zice Domnul Savaot: "Nu ascultați cuvintele proorocilor, care vă profețesc, că vă înșeală, povestindu-vă închipuirile inimii lor, și nimic din cele ale Domnului.
\par 17 Necontenit grăiesc ei celor ce Mă disprețuiesc: "Domnul, a zis că va fi pace peste voi". Și tuturor celor care urmează inima lor învârtoșată le zic: "Nici un rău nu va veni asupra voastră!"
\par 18 Că cine a luat parte la sfatul Domnului, ca să vadă și să audă cuvântul Lui? Sau cine a luat aminte la cuvântul Lui ca să-l vestească?
\par 19 Iată vine furtuna Domnului cu iuțime, vine furtună mare și va cădea peste capetele necredincioșilor.
\par 20 Mânia Domnului nu se va potoli până nu va împlini și va înfăptui planurile inimii Sale, și în zilele ce vin veți pricepe aceasta lămurit.
\par 21 N-am trimis Eu pe proorocii aceștia, ci au alergat ei singuri; Eu nu le-am spus, ci au profețit ei de la ei.
\par 22 Au luat ei parte la sfatul Meu? Atunci să vestească cuvintele Mele poporului Meu, să facă să se întoarcă oamenii de la calea lor rea și de la faptele lor cele rele.
\par 23 Au doară Eu numai de aproape sunt Dumnezeu, zice Domnul, iar de departe nu mai sunt Dumnezeu?
\par 24 Poate oare omul să se ascundă în loc tainic, unde să nu-l văd Eu, zice Domnul? Au nu umplu Eu cerul și pământul, zice Domnul?
\par 25 Am auzit ce zic proorocii care profețesc minciună în numele Meu. Ei zic: "Am visat, am văzut în vis".
\par 26 Până când proorocii aceștia vor spune minciuni și vor vesti înșelătoria inimii lor?
\par 27 Cred ei cu visele lor, pe care și le povestesc unul altuia, să facă uitat, poporului Meu, numele Meu, așa cum au uitat părinții lor numele Meu pentru Baal?
\par 28 Proorocul care a văzut vis, povestească-l ca vis, iar cel ce are cuvântul Meu, acela să spună cuvântul Meu adevărat. Ce legătură poate fi între pleavă și grăuntele de grâu curat, zice Domnul?
\par 29 Cuvântul Meu nu este el, oare, ca un foc, zice Domnul, ca un ciocan care sfărâmă stânca?
\par 30 De aceea iată Eu, zice Domnul, sunt împotriva proorocilor care fură cuvântul Meu unul de la altul.
\par 31 Deoarece Eu, zice Domnul, sunt împotriva proorocilor care vorbesc cu limba lor, dar zic: "El a spus".
\par 32 Tot Eu, zice Domnul, sunt împotriva proorocilor care spun visuri mincinoase, care le povestesc pe acestea și duc pe poporul Meu la rătăcire cu amăgirile lor și cu lingușirile lor, deși Eu nu i-am trimis, nici le-am poruncit; ei nu aduc nici un folos poporului acestuia, zice Domnul.
\par 33 Deci, de te va întreba poporul acesta, sau vreun prooroc, sau vreun preot: "Care este amenințarea Domnului?" Să le spui: "Ce amenințare? Am să vă înlătur", zice Domnul.
\par 34 Iar dacă un prooroc sau preot, sau poporul va zice: "Este amenințarea Domnului", voi pedepsi pe omul acela și casa lui.
\par 35 Așa să ziceți unul către altul și frate către frate: "Ce-a răspuns Domnul?" sau: "Ce-a zis Domnul?"
\par 36 Iar cuvântul acesta: "Amenințare de la Domnul", de-acum să nu-l mai întrebuințați, că amenințare va fi unui astfel de om cuvântul lui, pentru că stricați cuvintele Domnului celui viu, cuvintele Domnului Savaot, cuvintele Dumnezeului nostru.
\par 37 Așa să zici proorocului: "Ce a răspuns Domnul?", sau: "Ce a zis Domnul?"
\par 38 Și de veți mai zice: "Amenințare de la Domnul", apoi așa zice Domnul: Pentru că spuneți acea vorbă: "Amenințare de la Domnul" când Eu am trimis să vi se spună: Nu mai ziceți "Amenințare de la Domnul",
\par 39 De aceea, iată vă voi uita cu totul și vă voi părăsi, și cetatea aceasta, pe care v-am dat-o vouă și părinților voștri, o voi lepăda de la fața Mea,
\par 40 Și voi pune asupra voastră ocară și necinste veșnică care nu se vor uita".

\chapter{24}

\par 1 După ce Nabucodonosor, regele Babilonului, a luat din Ierusalim robi pe Iehonia, fiul lui Ioiachim, regele lui Iuda și pe căpeteniile lui Iuda cu lemnarii și cu fierarii și i-a dus la Babilon, mi-a arătat Domnul vedenia aceasta: Iată erau două coșuri cu smochine, așezate înaintea templului Domnului:
\par 2 Un coș era cu smochine foarte bune, cum sunt smochinele văratice, iar celălalt coș era cu smochine foarte rele, care, de rele ce erau, nici nu se puteau mânca.
\par 3 Atunci mi-a zis Domnul: "Ce vezi tu, Ieremia?" Și eu am zis: "Smochine! Smochinele cele bune sunt foarte bune, iar cele rele sunt foarte rele, încât nu se pot mânca de rele ce sunt".
\par 4 Și a fost iarăși cuvântul Domnului către mine și mi-a zis:
\par 5 "Așa zice Domnul Dumnezeul lui Israel: Ca pe aceste smochine bune, așa voi privi cu bunăvoință pe cei ai lui Iuda, duși în robie, pe care i-am trimis din locul acesta în țara Caldeilor.
\par 6 Și-Mi întorc ochii Mei asupra lor, spre binele lor, și-i voi întoarce în țara aceasta și nu-i voi nimici, ei-i voi zidi; nu-i vai smulge, ci-i voi sădi;
\par 7 Le voi da inimă ca să Mă cunoască pe Mine că Eu sunt Domnul și ei vor fi poporul Meu, iar Eu voi fi Dumnezeul lor, că se vor întoarce la Mine cu toată inima lor".
\par 8 Despre smochinele rele care, de rele ce sunt, nu se pot mânca, zice Domnul: Așa voi face să ajungă Sedechia, regele lui Iuda, căpeteniile lui și restul Ierusalimului, cei care au rămas în țara aceasta și cei care trăiesc în țara Egiptului;
\par 9 Îl voi face pricină de groază, o nenorocire pentru toate regatele pământului, ca să fie de ocară și de pildă, de batjocură și blestem prin toate locurile pe unde-i voi izgoni.
\par 10 Și voi trimite asupra lor sabie, foamete și ciumă, până vor pieri din țara pe care le-o dădusem lor și părinților lor".

\chapter{25}

\par 1 Cuvântul care a fost către Ieremia pentru tot poporul iudeu, în anul al patrulea al lui Ioiachim, fiut lui Iosia, regele lui Iuda sau în anul întâi al lui Nabucodonosor, regele Babilonului
\par 2 Pe care cuvânt l-a rostit proorocul Ieremia către tot poporul iudeu și către toți locuitorii Ierusalimului:
\par 3 "De la anul al treisprezecelea al lui Iosia, fiul lui Amon, regele lui Iuda, și până în ziua aceasta - timp de douăzeci și trei de ani - a fost cuvântul Domnului către mine. Și eu de dimineața până seara v-am vorbit, dar voi n-ați ascultat.
\par 4 Trimis-a Domnul la noi necontenit pe toți robii Săi prooroci și voi n-ați ascultat, nici v-ați plecat urechea ca să-i ascultați.
\par 5 Vi s-a zis: Să se întoarcă fiecare de la calea sa cea rea și de la faptele sale cele urâte, și veri trăi în țara pe care Domnul a dat-o vouă și părinților voștri din veac în veac.
\par 6 Și să nu umblați după alți dumnezei, ca să le slujiți și să vă închinați lor, și să nu Mă mâniați cu faptele mâinilor voastre, și nu vă voi face nici un fel de rău.
\par 7 Voi însă nu M-ați ascultat, zice Domnul, ci M-ați mâniat eu faptele mâinilor voastre, spre răul vostru.
\par 8 De aceea, așa zice Domnul Savaot: Pentru că n-ați ascultat cuvintele Mele,
\par 9 Iată, voi trimite să aducă toate neamurile de la miazănoapte, zice Domnul; și voi trimite la Nabucodonosor, regele Babilonului, robul Meu, ca să le aducă împotriva acestei țări și a locuitorilor ei și împotriva tuturor neamurilor dimprejur pe care le voi pustii, 1e voi înspăimânta, le voi face de râs și de ocară veșnică.
\par 10 Nu vor mai putea avea glas de bucurie și glas de veselie, glasul mirelui și glasul miresei, uruitul pietrelor de moară și lumina sfeșnicului.
\par 11 Toată țara aceasta va fi pustiită și jefuită; popoarele acestea vor sluji regelui Babilonului șaptezeci de ani.
\par 12 Iar când se vor împlini șaptezeci de ani, voi pedepsi pe regele Babilonului și pe poporul acela, zice Domnul, pentru necredința lor, și țara Caldeilor o voi pedepsi și o voi face pustie pentru totdeauna.
\par 13 Voi împlini asupra țării acesteia toate cuvintele Mele pe care le-am rostit împotriva ei; tot ce-i scris în cartea aceasta și ceea ce Ieremia a proorocit împotriva tuturor neamurilor.
\par 14 Pentru că și pe ele le vor robi multe popoare și regi mari și le voi răsplăti după purtarea lor și după faptele mâinilor lor.
\par 15 Că așa mi-a zis Domnul Dumnezeul lui Israel: "Ia din mâna Mea cupa aceasta cu vinul urgiei și adapă cu ea toate popoarele la care te voi trimite;
\par 16 Acelea vor bea și se vor clătina și vor înnebuni la vederea sabiei pe care o voi trimite asupra lor!"
\par 17 Și am luat cupa din mâna Domnului și am dat să bea tuturor neamurilor, la care m-a trimis Domnul.
\par 18 Ierusalimului și cetăților lui Iuda, regilor lui și căpeteniilor lui, spre pustiire și groază, spre batjocură și blestem, precum se și vede astăzi;
\par 19 Lui Faraon, regele Egiptului, slujitorilor lui, căpeteniilor lui și întregului popor al lui;
\par 20 La toată Arabia, tuturor regilor țării Uț, tuturor regilor țării Filistenilor, Ascalonului, Gazei, Ecronului și rămășițelor din Așdod;
\par 21 Edomului, Moabului și fiilor lui Amon;
\par 22 Tuturor regilor Tirului, tuturor regilor Sidonului și regilor insulelor care sunt dincolo de mare:
\par 23 Dedanului și Temei, Buzului și tuturor care-și rad tâmplele;
\par 24 Tuturor regilor Arabiei și tuturor regilor popoarelor amestecate, care locuiesc în pustiu;
\par 25 Tuturor regilor Zimrei, tuturor regilor Elamului și tuturor regilor Mediei;
\par 26 Tuturor regilor de la miazănoapte, de aproape sau de departe, unora și altora, și tuturor regatelor lumii, care se află pe fața pământului; iar regele Șișacului va bea după ei.
\par 27 Și să le zici: "Așa zice Domnul Savaot, Dumnezeul lui Israel: Beți și vă îmbătați, vărsați și cădeți și nu vă ridicați la vederea sabiei pe care o trimit Eu asupra voastră!"
\par 28 Iar de nu vor vrea să ia cupa din mâna ta, ca să bea, să le zici:
\par 29 "Așa grăiește Domnul Savaot: Trebuie să beți numaidecât, pentru că încep să aduc nenorocire asupra cetății acesteia, peste care s-a chemat numele Meu. Să rămâneți voi nepedepsiți? Nu, nu veți rămâne nepedepsiți, căci voi chema sabie asupra tuturor celor ce locuiesc pe pământ", zice Domnul Savaot.
\par 30 De aceea proorocește asupra lor toate cuvintele acestea și le zi: "Domnul va striga cu putere de sus și din locașul sfințeniei Sale Își va ridica glasul Său; va striga cu putere împotriva pășunii Sale; va striga, ca cei ce calcă în teasc, tuturor celor ce locuiesc pe pământ.
\par 31 Zgomotul va ajunge până la marginea pământului, căci Domnul deschide procesul neamurilor; intră la judecată cu tot trupul și pe cei nelegiuiți îi va da sabiei", zice Domnul.
\par 32 Așa zice Domnul Savaot: "Iată nenorocirea se întinde de la neam la neam și vijelie mare se ridică de la marginile pământului.
\par 33 Și în ziua aceea, cei loviți de Domnul vor zăcea de la un capăt la celălalt al pământului și nu vor fi bociți, nici nu vor fi adunați și îngropați, ci vor sta ca gunoiul pe fața pământului:
\par 34 Plângeți, păstori, și suspinați și tăvăliți-vă în cenușă, stăpâni ai turmei, că s-au împlinit zilele, ca să fiii junghiați și împrăștiați.
\par 35 Atunci veți cădea ca berbecii aleși. Nu va fi adăpost pentru păstori, nici scăpare pentru stăpânii turmei.
\par 36 Ascultați strigătul păstorilor și urletul stăpânilor turmei, că Domnul a pustiit pășunea lor.
\par 37 Și câmpiile cele pașnice sunt pustiite de urgia mâniei Domnului.
\par 38 Părăsitu-și-a el locașul său, ca un leu, și țara lor s-a pustiit de urgia pustiitorului și de mânia aprinsă a Domnului".

\chapter{26}

\par 1 La începutul domniei lui Ioiachim, fiul lui Iosia, a fost cuvântul acesta de la Domnul:
\par 2 "Așa zice Domnul: Stai în curtea templului Domnului și grăiește tuturor cetăților lui Iuda, care vin la închinare în templul Domnului, toate cuvintele ce îți voi porunci să le grăiești, să nu lași nici un cuvânt.
\par 3 Poate vor asculta și se vor întoarce de la calea cea rea și atunci Îmi va părea rău de nenorocirea pe care aveam de gând să le-o fac din cauza faptelor lor rele.
\par 4 Și să le spui: Așa zice Domnul: "Dacă nu Mă veți asculta să umblați după legea Mea, pe care v-am dat-o,
\par 5 Și să luați aminte la cuvintele robilor Mei prooroci, pe care-i trimit la voi, pe care-i trimit dis-de-dimineață, și voi nu-i ascultați,
\par 6 Atunci voi face cu templul acesta ceea ce am făcut cu șilo, iar cetatea aceasta o voi da spre blestem tuturor popoarelor pământului".
\par 7 Preoții și proorocii și tot poporul au ascultat pe Ieremia, când a grăit el aceste cuvinte în templul Domnului.
\par 8 Iar după ce a spus Ieremia tot ce-i poruncise Domnul să spună la tot poporul, preoții, proorocii și tot poporul l-au prins și i-au zis: "Tu trebuie să mori!
\par 9 De ce proorocești în numele Domnului și zici: Templul acesta va fi ca Șilo și cetatea aceasta se va pustii și va rămâne fără locuitori?" Și s-a adunat tot poporul împotriva lui Ieremia la templul Domnului.
\par 10 și când au auzit acestea, căpeteniile lui Iuda au venit din casa regelui la templul Domnului și au șezut la intrare în poarta cea nouă a templului.
\par 11 Atunci preoții și proorocii au zis către căpetenii și către tot poporul așa: "Osândă cu moarte se cuvine acestui om, pentru că proorocește împotriva cetății acesteia, cum ați auzit și voi cu urechile voastre!"
\par 12 Iar Ieremia a zis către toate căpeteniile și către tot poporul: "Domnul m-a trimis să proorocesc împotriva templului acestuia și împotriva cetății acesteia cuvintele pe care le-ați auzit.
\par 13 Îndreptați-vă dar căile voastre și faptele voastre și supuneți-vă glasului Domnului Dumnezeului vostru, și Domnului Îi va părea rău de nenorocirea pe care o rostise împotriva voastră.
\par 14 Iar cât despre mine, iată sunt în mâinile voastre, faceți cu mine ce vi se pare bun și drept!
\par 15 Dar să știți bine că, de mă veți omorî, veți aduce asupra voastră, asupra cetății acesteia și asupra locuitorilor ei sânge nevinovat, căci cu adevărat Domnul m-a trimis la voi să vă spun în urechile voastre toate cuvintele acelea".
\par 16 Atunci căpeteniile și tot poporul au zis către preoți și către prooroci: "Omul acesta nu este vrednic de osândă cu moarte, pentru că ne-a grăit în numele Domnului Dumnezeului nostru".
\par 17 Atunci s-au ridicat unii din bătrânii poporului și au zis către toată adunarea poporului:
\par 18 "Miheia din Moreșet a proorocit în zilele lui Iezechia, regele lui Iuda, și a zis către tot poporul iudeu: așa zice Domnul Savaot: "Sionul va fi arat ca un ogor, Ierusalimul va ajunge o movilă de dărâmături și muntele templului acestuia va fi un deal împădurit".
\par 19 Omorâtu-l-a oare pentru aceasta Iezechia, regele lui Iuda și tot Iuda? Au nu s-a temut el de Domnul și nu s-a rugat Domnului, ca să-I pară rău de nenorocirea pe care o rostise împotriva lor? și noi să ne împovărăm sufletele noastre cu o nelegiuire așa de mare!
\par 20 De asemenea a mai proorocit în numele Domnului un oarecare Urie, fiul lui Șemaia, din Chiriat-Iearim, și a proorocit împotriva cetății acesteia și împotriva țării acesteia tocmai cu aceleași cuvinte, ca și Ieremia.
\par 21 Și când au auzit cuvintele lui regele Ioiachim și toți curtenii lui și toate căpeteniile, a căutat regele să-l omoare. Și auzind de aceasta, Urie s-a temut și a fugit și a trecut în Egipt.
\par 22 Dar regele Ioiachim a trimis și în Egipt oameni și anume: Pe Elnatan, fiul lui Acbor, și pe alții împreună cu aceștia.
\par 23 Și au adus pe Urie din Egipt și l-au înfățișat la regele Ioiachim, și acesta l-a ucis cu sabia și a aruncat trupul lui acolo unde erau mormintele oamenilor de rând".
\par 24 Dar Ahicam, fiul lui Șafan, ocrotea pe Ieremia, ca să nu fie dat în mâna poporului spre ucidere.

\chapter{27}

\par 1 La începutul domniei lui Ioiachim, fiul lui Iosia, rețele lui Iuda, a fost de la Domnul către Ieremia acest cuvânt:
\par 2 "Așa mi-a zis Domnul: Fă-ți cătușe și jug și le pune pe grumaz;
\par 3 Trimite la fel regelui iui Iuda și regelui Moabului, regelui fiilor lui Amon, regelui Tirului și regelui Sidonului, prin solii care au venit în Ierusalim la Sedechia, regele Ierusalimului;
\par 4 Și poruncește-le să spună stăpânilor lor: Iată ce zice Domnul Savaot, Dumnezeul lui Israel, așa să spuneți stăpânilor voștri:
\par 5 "Eu am făcut pământul și pe om și viețuitoarele cele de pe fața pământului cu puterea Mea cea mare și cu brațul Meu cel puternic și l-am dat cui am vrut.
\par 6 Și acum voi da toate țările acestea în mâna lui Nabucodonosor, regele Babilonului, robul Meu, ba și fiarele câmpului le voi da lui spre slujbă;
\par 7 Toate popoarele vor sluji lui și fiului lui și fiului fiului lui, până când îi va veni vremea și lui și țării lui; îi vor sluji popoare multe și regi mari.
\par 8 Dacă vreun popor sau vreun regat nu va voi să-i slujească lui Nabucodonosor, regele Babilonului, și nu-și va pleca grumazul său sub jugul regelui Babilonului, pe acel popor îl voi pedepsi cu sabie, cu foamete și cu ciumă, până-l voi stârpi cu mâna lui, zice Domnul.
\par 9 Iar voi să nu ascultați pe proorocii voștri, pe ghicitorii voștri, pe visătorii voștri, pe vrăjitorii voștri și pe-ai voștrii cititori de stele care vă zic: "Nu veți sluji regelui Babilonului",
\par 10 Că aceștia vă prezic minciună, ca să vă depărteze din țara voastră și ca Eu să vă izgonesc, ca să pieriți.
\par 11 Iar pe poporul care-și va pleca grumazul sub jugul regelui Babilonului și i se va supune, îl voi lăsa în pământul său și-l va lucra și va trăi pe el", zice Domnul.
\par 12 Și lui Sedechia, regele lui Iuda, i-am grăit toate cuvintele acestea și am zis: "plecați-vă grumazul sub jugul regelui Babilonului și sluțiți-i lui și poporului. lui și veți trăi.
\par 13 De ce să mori tu și poporul tău de sabie, de foamete și de ciumă, cum a zis Domnul de poporul acela, care nu va sluți regelui Babilonului?
\par 14 Și să nu ascultați cuvintele proorocilor care vă zic:
\par 15 "Nu veți sluji regelui Babilonului, că minciună vă prezic și nici nu i-am trimis Eu, zice Domnul; ei vă profețesc mincinos în numele Meu, ca să vă izgonesc și să pieriți și voi și proorocii voștri, care vă proorocesc".
\par 16 Preoților și poporului întreg le-am grăit acestea: "Așa zice Domnul: Nu ascultați cuvintele proorocilor voștri care vă proorocesc și vă zic: Iată, curând vor fi întoarse de la Babilon vasele templului Domnului, că minciună vă proorocesc.
\par 17 Nu-i ascultați, ci slujiți regelui Babilonului și veți trăi. De ce să duceți cetatea aceasta la pustiire?
\par 18 Iar dacă sunt ei prooroci și dacă au ei cuvântul Domnului, atunci să mijlocească înaintea Domnului Savaot ca vasele rămase în templul Domnului și în casa regelui lui Iuda și în Ierusalim să nu treacă la Babilon.
\par 19 Că așa zice Domnul Savaot de stâlpi, de baia cea de aramă, de postamente și de celelalte lucruri care au rămas în cetatea aceasta,
\par 20 Și pe care Nabucodonosor, regele Babilonului, nu le-a luat când a dus din Ierusalim la Babilon pe Iehonia, fiul lui Ioiachim, regele lui Iuda, și pe toți îi însemnați din Iuda și din Ierusalim;
\par 21 Așa zice Domnul Savaot, Dumnezeul lui Israel, despre vasele care au rămas în templul Domnului și în casa regelui lui Iuda și în Ierusalim:
\par 22 "Vor fi duse și acelea la Babilon și vor rămâne acolo până în ziua când le voi căuta Eu și le voi scoate și le voi întoarce la locul acesta, zice Domnul".

\chapter{28}

\par 1 Tot în anul acela, la începutul domniei lui Sedechia, regele lui Iuda, adică în anul al patrulea, în luna a cincea, Anania, fiul lui Azur, proorocul cel din Gabon, mi-a grăit în templul Domnului, înaintea ochilor preoților și a tot poporul și mi-a zis:
\par 2 "Așa zice Domnul Savaot, Dumnezeul lui Israel: Voi sfărâma jugul regelui Babilonului;
\par 3 Peste doi ani voi întoarce în locul acesta toate vasele templului Domnului, pe care Nabucodonosor, regele Babilonului, le-a luat din acest loc și le-a dus la Babilon.
\par 4 Voi întoarce la locul acesta și pe Iehonia, fiul lui Ioiachim, regele lui Iuda, și pe toți robii iudei, care au mers la Babilon, zice Domnul, căci voi sfărâma jugul regelui Babilonului".
\par 5 Atunci proorocul Ieremia a grăit către Anania proorocul, înaintea ochilor preoților și înaintea ochilor a tot poporul, ce stătea în templul Domnului, și a zis Ieremia proorocul:
\par 6 "Așa să fie și să facă aceasta Domnul! Împlinească Domnul cuvintele tale, pe care le-ai rostit tu pentru întoarcerea din Babilon a vaselor templului Domnului și a tuturor robilor la locul acesta.
\par 7 Dar ascultă cuvântul acesta pe care ți-l spun eu în auzul tău și în auzul a tot poporul:
\par 8 Proorocii care au fost demult înaintea mea și înaintea ta au prevestit multor țări și regate puternice război, strâmtorare și molimă.
\par 9 Când însă vreun prooroc a prevestit pace, atunci numai așa a fost luat el ca prooroc, cu adevărat trimis de Domnul, dacă s-a împlinit cuvântul acelui prooroc".
\par 10 Iar proorocul Anania a luat jugul de pe grumazul lui Ieremia proorocul și l-a sfărâmat;
\par 11 Și a zis Anania înaintea ochilor a tot poporul cuvintele acestea: "Iată ce zice Domnul: Așa voi sfărâma jugul lui Nabucodonosor, regele Babilonului, peste doi ani, luându-l de pe grumazul tuturor popoarelor". Și Ieremia s-a dus în drumul său.
\par 12 După ce proorocul Anania a sfărâmat jugul de pe grumazul proorocului Ieremia, a fost cuvântul Domnului către Ieremia și i-a zis:
\par 13 "Mergi și spune lui Anania: Așa zice Domnul: Tu ai sfărâmat un jug de lemn și voi face în locul lui altui de fier,
\par 14 Că așa zice Domnul Savaot, Dumnezeul lui Israel: Jug de fier voi pune pe grumazul tuturor acestor popoare, ca să muncească ele la Nabucodonosor, regele Babilonului, și ca să-i slujească, ba și fiarele câmpului le voi da lui!"
\par 15 Și a mai zis proorocul Ieremia către Anania proorocul: "Ascultă, Anania: Domnul nu te-a trimis și tu dai poporului acestuia încredere în minciună.
\par 16 De aceea așa zice Domnul: Iată te voi arunca de pe fața pământului; chiar în anul acesta vei muri, pentru că ai grăit împotriva Domnului".
\par 17 Și a murit Anania proorocul chiar în anul acela, în luna a șaptea.

\chapter{29}

\par 1 Iată cuvintele scrisorii pe care Ieremia proorocul a trimis-o din Ierusalim către rămășița bătrânilor din robie, preoților și proorocilor și către tot poporul pe care Nabucodonosor i-a dus din Ierusalim la Babilon,
\par 2 După ce a ieșit din Ierusalim regele Iehonia și regina și eunucii și căpeteniile lui Iuda și ai Ierusalimului și lemnarii și fierarii,
\par 3 Prin Eleasa, fiul lui Șafan, și prin Ghemaria, fiul lui Hilchia, pe care Sedechia, regele lui Iuda, i-a trimis la Babilon către Nabucodonosor, regele Babilonului:
\par 4 "Așa zice Domnul Savaot, Dumnezeul lui Israel, către toți robii pe care i-am strămutat din Ierusalim la Babilon: Zidiți casă și trăiți;
\par 5 Faceți grădini și mâncați roadele lor!
\par 6 Luați femei și nașteți fii și fiice! Fiilor voștri luați-le soții, iar pe fiicele voastre măritați-le, ca să nască fii și fiice, și să nu vă împuținați acolo, ci înmulțiri-vă!
\par 7 Căutați binele țării în care v-am dus robi și rugați-vă Domnului pentru ea, că de propășirea ei atârnă și fericirea voastră!
\par 8 Pentru că așa zice Domnul Savaot, Dumnezeul lui Israel: Să nu vă lăsați amăgiți de proorocii voștri și de ghicitorii voștri care sunt în mijlocul vostru, și să nu ascultați visele voastre, pe care le veți visa,
\par 9 Căci vă proorocesc minciună și Eu nu i-am trimis, zice Domnul!
\par 10 Pentru că Domnul zice: Când vi se vor împlini în Babilon șaptezeci de ani, atunci vă voi cerceta și voi împlini cuvântul Meu cel bun pentru voi, ca să vă întoarceți la locul acesta.
\par 11 Pentru că numai Eu știu gândul ce-l am pentru voi, zice Domnul, gând bun, nu rău, ca să vă dau viitorul și nădejdea.
\par 12 Și veți striga către Mine și veți veni și vă veți ruga Mie, și Eu vă voi auzi;
\par 13 Și Mă veți căuta și Mă veți găsi, dacă Mă veți căuta cu toată inima voastră.
\par 14 Și voi fi găsit de voi, zice Domnul, și vă voi întoarce din robie și vă voi strânge din toate popoarele și din toate locurile, de pe unde v-am izgonit, zice Domnul, și vă voi întoarce din locul de unde v-am dus.
\par 15 Voi ziceți: "Domnul ne-a ridicat prooroci în Babilon".
\par 16 Așa zice Domnul de regele care șade pe tronul lui David și de tot poporul care trăiește în cetatea aceasta și de frații voștri care n-au fost duși cu voi în robie.
\par 17 Așa zice Domnul de aceștia: Iată, voi trimite asupra lor sabie, foamete și ciumă și-i voi face ca smochinele cele stricate care, de rele ce sunt, nu se pot mânca,
\par 18 și-i voi urmări cu sabie, cu foamete și cu ciumă, și-i voi da spre chin în toate regatele pământului, spre blestem și grozăvie, spre râs și batjocură în fața tuturor popoarelor la care-i voi izgoni,
\par 19 Pentru că n-au ascultat cuvintele Mele, zice Domnul, deși le-am trimis dis-de-dimineață pe slujitorii Mei, proorocii; zice Domnul.
\par 20 Iar voi toți cei ce sunteți în robie, pe care v-am trimis din Ierusalim la Babilon, ascultați cuvântul Domnului.
\par 21 Așa zice Domnul Savaot, Dumnezeul lui Israel despre Ahab, fiul lui Colaia și despre Sedechia, fiul lui Maaseia, care vă proorocesc vouă minciună în numele Meu: pe aceștia îi voi da în mâinile lui Nabucodonosor, regele Babilonului, și acela îi va omorî înaintea ochilor voștri.
\par 22 și se va obișnui după ei. printre toți robii lui Iuda din Babilon a se blestema astfel: "Să-i facă Domnul cum a făcut lui Sedechia și lui Ahab", pe care i-a fript regele Babilonului pe jeratic,
\par 23 Pentru că au făcut ticăloșie în Israel: se desfrânau cu femeile aproapelui lor și spuneau minciuni în numele Meu, ceea ce Eu nu le-am poruncit; Eu știu acestea și sunt martor, zice Domnul.
\par 24 Iar lui Șemaia Nehelamitul spune-i:
\par 25 Așa zice Domnul Savaot, Dumnezeul lui Israel: Pentru că ai trimis scrisori în numele tău către tot poporul cel din Ierusalim și către preotul Sofonie, fiul lui Maaseia, și către toți preoții și ai scris:
\par 26 "Domnul te-a pus preot în locul preotului Iehoiada, ca să fii supraveghetor în templul Domnului peste tot omul nebun și peste tot omul ce proorocește, și ca să pui pe unul ca acesta în temniță și în cătușe:
\par 27 Pentru ce dar nu oprești pe Ieremia din Anatot de a mai prooroci acolo la voi?
\par 28 Că acesta și în Babilon a trimis să mi se spună: "Robia va fi lungă: zidiți case și locuiți în ele, sădiți grădini și mâncați roadele lor".
\par 29 Preotul Sofonie a citit această scrisoare în auzul proorocului Ieremia,
\par 30 Și cuvântul Domnului a fost către Ieremia și a zis:
\par 31 "Trimite la toți cei strămutați să li se spună: Așa zice Domnul de Șemaia Nehelamitul: Pentru că Șemaia proorocește la voi, dar Eu nu l-am trimis, și vă dă încredere într-o minciună,
\par 32 De aceea așa zice Domnul: Iată Eu voi pedepsi pe Șemaia Nehelamitul și neamul lui; nu va fi din el om care să locuiască în mijlocul poporului acestuia, iar el nu va vedea binele acela, pe care Eu îl voi face poporului Meu, zice Domnul, pentru că a grăit împotriva Domnului".

\chapter{30}

\par 1 Cuvântul Domnului care a fost către Ieremia:
\par 2 "Așa zice Domnul Dumnezeul lui Israel: Scrie-ți într-o carte toate cuvintele pe care ți le-am grăit Eu,
\par 3 Că iată vin zile, zice Domnul când voi întoarce din robie pe poporul Meu, pe Israel și pe Iuda, zice Domnul, și-i voi aduce iarăși în pământul acela, pe care l-am dat părinților lor, și-l vor stăpâni".
\par 4 Iată cuvintele acelea pe care le-a spus Domnul despre Israel și despre Iuda:
\par 5 Așa a zis Domnul: "Glas de tulburare, glas de groază auzim, ți nu glas de pace.
\par 6 Întrebați și vedeți dacă naște vreun bărbat? De ce văd Eu pe toți bărbații cu mâinile pe șolduri, ca o femeie care naște, și fețele tuturor palide?
\par 7 O, vai! Mare este ziua aceea! Asemenea ei n-a mai fost alta! Acela este timp de mare strâmtorare pentru Iacov, dar el va fi izbăvit din ea.
\par 8 Și în ziua aceea, zice Domnul Savaot, voi sfărâma jugul de pe grumazul lor și lanțurile lor le voi rupe.
\par 9 Și nu vor mai sluji străinilor, ci vor sluji Domnului Dumnezeului lor și lui David, regele lor, pe care îl voi pune iarăși pe tron.
\par 10 Și tu robul Meu Iacov, nu te teme, zice Domnul, nici nu te înspăimânta, Israele, că iată Eu te voi scăpa din țara cea depărtată și neamul tău îl voi aduce din țara robiei tale. Și se va întoarce Iacov și va trăi în pace și în liniște și nimeni nu-l va mai îngrozi.
\par 11 Căci Eu sunt cu tine, zice Domnul, ca să te izbăvesc; voi pierde de fot toate popoarele, printre care te-am risipit pe tine, iar pe tine nu te voi pierde; te voi pedepsi după dreptate și nepedepsit nu te voi lăsa,
\par 12 Că așa zice Domnul: Rana ta e de nevindecat și plaga ta e dureroasă.
\par 13 Nimeni nu se îngrijește de pricina ta, ca să-ți vindece rana, și leac vindecător nu este pentru tine.
\par 14 Toți prietenii tăi te-au uitat și nu te mai caută; că Eu te-am lovit ca pe un vrăjmaș, cu pedeapsă crudă, pentru mulțimea fărădelegilor tale, pentru că se înmulțiseră păcatele tale.
\par 15 De ce te plângi de rana ta? Durerea ta este de nevindecat? După mulțimea fărădelegilor tale ți-am făcut aceasta, pentru că se înmulțiseră păcatele tale.
\par 16 Dar toți cei ce te mănâncă vor fi mâncați și toți vrăjmașii tăi vor merge și ei toți în robie și toți jefuitorii tăi vor fi jefuiți.
\par 17 Și îți voi lega rănile și te voi vindeca, zice Domnul. Că ei te numesc: "Cel izgonit", "Sionul, de care nu mai întreabă nimeni!"
\par 18 Așa zice Domnul: "Iată voi restatornici corturile lui Iacov și sălașurile lui le voi milui; cetatea va fi zidită iar pe dealul său și templul se va zidi ca mai înainte.
\par 19 Se vor înălța din ele mulțumiri și glas de bucurie; îi voi înmulți și nu vor mai fi împuținați, îi voi încununa cu glorie și nu vor mai fi înjosiți.
\par 20 Fiii lui Iacov vor fi ca mai înainte, adunarea lui va sta înaintea Mea și Eu voi pedepsi pe toți asupritorii lui.
\par 21 Căpetenia lui va fi din el și stăpânul lui se va ridica din mijlocul lui; îl voi apropia și se va apropia de Mine: că cine va îndrăzni să se apropie singur de Mine? - zice Domnul.
\par 22 Voi veți fi poporul Meu, și Eu voi fi Dumnezeul vostru.
\par 23 Iată vijelie sălbatică vine de la Domnul, vijelie groaznică, și va cădea pe capul necredincioșilor.
\par 24 Mânia cea aprinsă a Domnului nu se va abate până ce nu se va sfârși și nu se va împlini gândul inimii Lui. În zilele cele de apoi veți pricepe aceasta".

\chapter{31}

\par 1 "În vremea aceea Eu voi fi Dumnezeul tuturor semințiilor lui Israel, iar ele vor fi poporul Meu", zice Domnul.
\par 2 Așa zice Domnul: "Poporul care a scăpat de sabie a aflat milă în pustiu. Israel a ajuns la locul său de odihnă".
\par 3 Atunci mi S-a arătat Domnul din depărtare și mi-a zis: "Cu iubire veșnică te-am iubit și de aceea Mi-am întins spre tine bunăvoința.
\par 4 Din nou te voi pune în rânduială, fecioara lui Israel, din nou te voi înfrumuseța cu timpanele tale și vei ieși la hora celor ce se veselesc;
\par 5 Din nou vei lua viile de pe dealurile Samariei, și vierii care le vor lucra se vor folosi ei singuri de ele.
\par 6 Că va veni vremea când străjile de pe muntele Efraim vor striga: "Sculați-vă să ne suim în Sion, la Domnul Dumnezeul nostru!"
\par 7 Căci așa zice Domnul: "Strigați de bucurie pentru Iacov și strigați înaintea căpeteniei popoarelor, vestiți; lăudați și ziceți: Doamne, izbăvește pe poporul Tău, rămășița lui Israel!
\par 8 Iată, îi voi aduce din țara cea de la miazănoapte și-i voi aduna de la marginile pământului. Orbul și șchiopul, cea însărcinată și cea care naște vor fi împreună cu ei; se va întoarce aici mare mulțime.
\par 9 Cu lacrimi au plecat și îi voi întoarce cu mângâiere, îi voi întoarce la izvoarele apelor pe cale netedă, pe care nu se vor poticni, că Eu sunt părintele lui Israel și Efraim este întâiul Meu născut".
\par 10 Ascultați, popoare, cuvântul Domnului, vestiți țării depărtate și ziceți: Cel ce a împrăștiat pe Israel, Acela îl va aduna și-l va păzi, ca păstorul turma sa, că va răscumpăra Domnul pe Iacov,
\par 11 Și-l va izbăvi din mâna celui ce a fost mai tare decât el.
\par 12 Și vor veni și vor prăznui pe înălțimile Sionului și vor curge bunătățile Domnului: grâu și vin și untdelemn, miei și boi. Sufletul lor va fi ca o grădină bine-udată și ei nu vor mai tânji.
\par 13 Atunci fecioara se va veseli la horă, tineri și bătrâni vor fi fericiți. Voi schimba întristarea lor în veselie și-i voi mângâia după întristarea lor și-i voi bucura.
\par 14 Voi hrăni sufletele preoților cu grăsime și poporul Meu se va sătura din bunătățile Mele", zice Domnul.
\par 15 Așa zice Domnul: "Glas se aude în Rama, bocet și plângere amară. Rahila își plânge copiii și nu vrea să se mângâie de copiii săi, pentru că nu mai sunt".
\par 16 Așa zice Domnul: Înfrânează-ți glasul de la bocet și ochii de la lacrimi, că vei avea plată pentru munca ta, zice Domnul, și ei se vor întoarce din țara dușmană;
\par 17 Este nădejde pentru viitorul tău, zice Domnul, că fiii tăi se vor întoarce în hotarele lor.
\par 18 Aud pe Efraim plângând și zicând: "Tu m-ai pedepsit și sunt pedepsit ca un junc neînvățat. Întoarce-mă și mă voi întoarce, că Tu ești Domnul Dumnezeul meu!
\par 19 După ce m-am întors, m-am căit, și când am luat cunoștință, m-am bătut în piept; am fost rușinat și tulburat am fost, pentru că am purtat ocara tinereții mele".
\par 20 Dar Efraim nu este feciorul Meu scump, un copil atât de alintat? Atunci când vorbesc de el, totdeauna cu dragoste Mi-amintesc de el; pentru el Mi se mișcă inima și voi avea milă de el, zice Domnul.
\par 21 Pune-ți semne pe lingă drum, pune-ți stâlpi, fii atentă la drum, la calea pe care te-ai dus; întoarce-te, fecioara lui Israel, întoarce-te în aceste cetăți ale tale.
\par 22 Până când vei rătăci, fiică neascultătoare? Pentru că Domnul va face pe pământ lucru nou: femeia va căuta pe bărbatul ei.
\par 23 Așa zice Domnul Savaot, Dumnezeul lui Israel: "De acum înainte, când voi întoarce robia lor, vor grăi în pământul lui Iuda și în cetățile lui cuvintele acestea: "Locaș al dreptății și munte sfânt, Domnul să te binecuvânteze!"
\par 24 Și se vor așeza în această țară Iuda și toate cetățile lui, plugarii și cei ce umblă cu turmele.
\par 25 Că voi adăpa sufletul obosit și voi sătura toată inima amărâtă".
\par 26 Atunci m-am deșteptat și am privit, și somnul meu a fost liniștit.
\par 27 "Iată vin zile, zice Domnul, când voi semăna în casa lui Israel și în casa lui Iuda o sămânță de om și sămânță de vite;
\par 28 Precum am privegheat asupra lor ca să-i smulg și să-i zdrobesc, ca să-i risipesc, să-i vatăm și să-i pierd, așa voi priveghea asupra lor, ca să-i zidesc și să-i sădesc, zice Domnul.
\par 29 În zilele acelea nu vor mai zice: "Părinții au mâncat aguridă și copiilor li s-au strepezit dinții".
\par 30 Ci fiecare va muri pentru fărădelegea sa; cine va mânca aguridă, aceluia i se vor strepezi dinții.
\par 31 Iată vin zile, zice Domnul, când voi încheia cu casa lui Israel și cu casa lui Iuda legământ nou.
\par 32 Însă nu ca legământul pe care l-am încheiat cu părinții lor în ziua când i-am luat de mână, ca să-i scot din pământul Egiptului. Acel legământ ei l-au călcat, deși Eu am rămas în legătură cu ei, zice Domnul.
\par 33 Dar iată legământul pe care-l voi încheia cu casa lui Israel, după zilele acela, zice Domnul: Voi pune legea Mea înăuntrul lor și pe inimile lor voi scrie și le voi fi Dumnezeu, iar ei Îmi vor fi popor.
\par 34 Și nu se vor mai învăța unul pe altul și frate pe frate, zicând: "Cunoașteți pe Domnul" că toți de la sine Mă vor cunoaște, de la mic până la mare, zice Domnul, pentru că Eu voi ierta fărădelegile lor și păcatele lor nu le voi mai pomeni".
\par 35 Așa zice Domnul, Cel ce a dat soarele ca să lumineze ziua și a pus legi lunii și stelelor, ca să lumineze noaptea; Cel ce tulbură marea, de-i mugesc valurile, și numele Lui este Domnul Savaot:
\par 36 "Când aceste legi vor înceta să mai aibă putere înaintea Mea, zice Domnul, atunci și seminția lui Israel va înceta pentru totdeauna să mai fie popor înaintea Mea.
\par 37 Așa zice Domnul: Dacă cerul poate fi măsurat sus și temeliile pământului cercetate jos, atunci și Eu voi lepăda toată seminția lui Israel pentru toate câte au făcut ei, zice Domnul.
\par 38 Iată vin zile - zice Domnul când cetatea se va zidi întru slava Domnului de la turnul Hananeel până la Poarta Colțului,
\par 39 Și funia de măsurat pământul va înainta de-a dreptul până la muntele Gareb și se va întoarce spre Goa.
\par 40 Și toată valea trupurilor moarte și a cenușii și tot câmpul până la pârâul Chedron, până la colțul porcilor cailor, spre răsărit, vor fi locuri sfințite pentru Domnul și nu vor mai fi niciodată nici pustiite, nici nimicite".

\chapter{32}

\par 1 Cuvântul care a fost de la Domnul către Ieremia, în anul al zecelea al lui Sedechia, regele lui Iuda. Anul acesta era anul al optsprezecelea al lui Nabucodonosor.
\par 2 Atunci oștirea regelui Babilonului înconjurase Ierusalimul și proorocul Ieremia era închis în curtea temniței, care se afla lângă casa regelui lui Iuda.
\par 3 Acolo îl închisese regele Sedechia. zicând: "De ce proorocești și zici: Așa grăiește Domnul: Iată, voi da cetatea în mâinile regelui Babilonului, și acesta o va lua;
\par 4 Sedechia, regele lui Iuda, nu va scăpa din mâinile Caldeilor, ci va fi dat fără îndoială în mâinile regelui Babilonului și va grăi cu el gură către gură; ochii lui var vedea ochii aceluia,
\par 5 Și acela va duce pe Sedechia la Babilon, unde va și sta el până-l voi cerceta, zice Domnul. Dacă vă veți lupta cu Caldeii, nu veți avea izbândă!"
\par 6 Atunci Ieremia a zis: "Așa a fost cuvântul Domnului către mine:
\par 7 Iată vine Hanameel, fiut lui Șalum, unchiul tău, vine la tine ca să-îi zică: "Cumpără pentru tine țarina mea cea din Anatot, pentru că după dreptul de înrudire se cuvine să o cumperi tu".
\par 8 și a venit la mine în curtea gărzii Hanameel, fiul unchiului meu, după cuvântul Domnului, și mi-a zis: "Cumpără țarina mea cea din Anatot, din pământul lui Veniamin, că al tău este dreptul de moștenire și al tău este dreptul de răscumpărare. Cumpăr-o pentru tine!"
\par 9 Atunci am cunoscut că acesta fusese cuvântul Domnului și am cumpărat de la Hanameel, fiul unchiului meu, țarina cea din Anatot și i-am cântărit șapte sicli de argint și zece arginți;
\par 10 Apoi am scris zapisul și l-am întărit cu pecetea, am chemat la aceasta martori și am cântărit argintul cu cântarul.
\par 11 Am luat atât zapisul de cumpărare cel pecetluit după lege și rânduială, cât și pe cel deschis;
\par 12 Și am dat acest zapis de cumpărare lui Baruh, fiul lui Neria, fiul lui Maasia, în fața lui Hanameel, fiul unchiului meu, și în fața martorilor care iscăliseră acest zapis de cumpărare, și în fața tuturor Iudeilor, care se aflau în curtea gărzii,
\par 13 Și în fața tuturor am poruncit lui Baruh:
\par 14 Așa zice Domnul Savaot, Dumnezeul lui Israel: "Ia zapisele acestea, - zapisul de cumpărare care este pecetluit și zapisul acesta deschis - și le pune într-un vas de lut, ca să stea acolo zile multe.
\par 15 Că așa zice Domnul Savant, Dumnezeul lui Israel: Casele, țarinile și viile vor fi din nou cumpărate în țara aceasta".
\par 16 Dând eu zapisul de cumpărare lui Baruh, fiul lui Neria, m-am rugat Domnului și am zis:
\par 17 "O, Doamne Dumnezeule, Tu ai făcut cerul și pământul cu puterea Ta cea mare și cu braț înalt și pentru Tine nimic nu este cu neputință!
\par 18 Tu arăți milă la mii și pedepsești fărădelegile părinților în sânul copiilor lor după ei; Tu ești Dumnezeu cel mare și puternic, al Cărui nume este Domnul Savaot;
\par 19 Cel mare în sfat și puternic în faptele Tale, ai Cărui ochi sunt deschiși asupra tuturor căilor fiilor oamenilor, ca să dai fiecăruia după căile lui și după roadele faptelor lui;
\par 20 Tu ai săvârșit semne și minuni în țara Egiptului și săvârșești și astăzi în Israel și printre oameni,
\par 21 Și ți-ai făcut nume, ca și astăzi, și ai scos pe poporul Tău Israel din pământul Egiptului prin semne și minuni, cu mână tare și cu braț înalt, în groază mare,
\par 22 Și le-ai dat lor țara aceasta, în care curge lapte și miere, pe care o făgăduiseși cu jurământ părinților lor, că le-o vei da;
\par 23 Și ei au intrat și au pus stăpânire pe ea, dar n-au ascultat glasul Tău ca să se poarte după legea Ta și n-au făcut ceea ce le-ai poruncit Tu să facă. De aceea ai adus asupra lor toate aceste nenorociri.
\par 24 Iată valurile de pământ se întind până la cetate ca să fie luată! Și cetatea prin sabie, foamete și ciumă se dă în mâinile Caldeilor, care luptă împotriva ei; ceea ce ai zis Tu, aceea se și împlinește. Și Tu vezi aceasta.
\par 25 Dar Tu, Doamne Dumnezeule, mi-ai zis: Cumpără-ți o țarină cu argint și cheamă martori, tocmai când cetatea se dă în mâinile Caldeilor".
\par 26 Atunci a fost cuvântul Domnului către Ieremia și a zis:
\par 27 "Iată, Eu sunt Domnul Dumnezeu a tot trupul! Este oare ceva cu neputință la Mine?"
\par 28 De aceea așa zice Domnul: "Iată voi da cetatea aceasta în mâinile Caldeilor și în mâinile lui Nabucodonosor, regele Babilonului, și acesta o va lua.
\par 29 Și vor intra Caldeii, care împresoară cetatea aceasta, vor da foc cetății și o vor arde cu foc pe ea și casele pe ale căror acoperișuri s-au adus tămâieri lui Baal și jertfe cu turnare în cinstea dumnezeilor străini, ca să Mă mânie pe Mine.
\par 30 Că fiii lui Israel și fiii lui Iuda au făcut numai rău înaintea ochilor Mei din tinerețile lor; fiii lui Israel M-au mâniat necontenit cu faptele mâinilor lor, zice Domnul.
\par 31 Cetatea aceasta, chiar din ziua zidirii sale și până astăzi, pare că a fost făcută pentru mânia Mea și pentru urgia Mea, ca s-o lepăd de la fala Mea,
\par 32 Din pricina răului fiilor lui Israel și al fiilor lui Iuda, pe care l-au făcut pentru mânierea Mea, ei și regii lor, căpeteniile lor, preoții lor și proorocii lor, bărbații lui Iuda și locuitorii Ierusalimului.
\par 33 Și Mi-au întors spatele, iar nu fața și, când i-am învățat, dis-de-dimineață i-am învățat și ei n-au voit să primească învățătură;
\par 34 În templul asupra căruia s-a chemat numele Meu au pus urâciunile lor, întinându-l.
\par 35 Au făcut locuri înalte lui Baal în valea fiilor lui Hinom, ca să treacă prin foc pe fiii lor și pe fiicele lor în cinstea lui Moloh, ceea ce Eu nu le-am poruncit, și nici prin minte nu Mi-a trecut că ei vor face această urâciune, ducând în păcat pe Iuda".
\par 36 Și acum, așa vorbește Domnul Dumnezeul lui Israel despre cetatea aceasta, de care ziceți: "Ea se va da în mâinile regelui Babilonului prin sabie, foamete și boala ciumei":
\par 37 "Iată îi voi aduna din toate țările, prin care i-am împrăștiat în mânia Mea și cu iuțimea Mea și în marea Mea întărâtare, și îi voi întoarce la locul acesta și le voi da viață fără de primejdie.
\par 38 Ei vor fi poporul Meu și Eu le voi fi Dumnezeu.
\par 39 Și le voi da o inimă și o cale, ca să se teamă de Mine în toate zilele vieții, spre binele dar și spre binele copiilor lor după ei.
\par 40 Voi încheia cu ei legământ veșnic, după care Eu nu Mă voi mai întoarce de la ei, ci le voi face bine și voi pune frica Mea în inima lor, ca ei să nu se mai abată de la Mine.
\par 41 Mă voi bucura să le fac bine și-i voi sădi tare în pământul acesta din toată inima Mea și din tot sufletul Meu".
\par 42 Că așa zice Domnul: "După cum am adus asupra poporului acestuia acest mare rău, așa voi aduce asupra lor tot binele pe care l-am rostit pentru ei.
\par 43 Și vor cumpăra țarină în țara aceasta, de care ziceți: "Aceasta este pustietate fără oameni și fără vite și este dată în mâna Caldeilor".
\par 44 Vor cumpăra țarini pe argint și le vor scrie în zapis și-l vor pecetlui și vor chema martori; cumpăra-vor țarine în pământul lui Veniamin și în împrejurimile Ierusalimului, prin cetățile lui Iuda și prin cetățile din munți, prin cetățile din câmp, prin cetățile de la miazăzi, că voi întoarce pe robii lor", zice Domnul.

\chapter{33}

\par 1 Fost-a cuvântul Domnului către Ieremia a doua oară, când era el tot închis în curtea gărzii, și a zis:
\par 2 "Așa grăiește Domnul, Cel ce a făcut pământul. Domnul Cel ce l-a zidit Și l-a întărit, al Cărui nume este Domnul:
\par 3 Strigă către Mine, că Eu îți voi răspunde și iți voi arăta lucruri mari și nepătrunse pe care tu nu le știi.
\par 4 Că așa zice Domnul Dumnezeul lui Israel despre casele acestei cetăți și despre casele regilor lui Iuda, care se strică pentru a fi făcute întărituri împotriva împresurării și sabiei
\par 5 Caldeilor veniți să lupte și să le umple cu trupurile oamenilor, pe care-i lovesc cu mânia Mea și cu urgia Mea și pentru ale căror fărădelegi Mi-am ascuns fața de la cetatea aceasta.
\par 6 Le voi lega rănile ca să-i vindec și le voi descoperi belșug de pace și de adevăr;
\par 7 Voi întoarce aici pe robii lui Iuda și pe robii lui Israel și-i voi așeza ca la început;
\par 8 Îi voi curăți de necredința lor, cu care au greșit ei înaintea Mea și le voi ierta toate fărădelegile lor, cu care au păcătuit ei înaintea Mea și au căzut de la Mine.
\par 9 Ierusalimul va fi pentru Mine nume de bucurie, laudă și cinste în fața tuturor popoarelor pământului, care vor auzi de toate bunătățile ce i le voi face și se vor mira, se vor cutremura de toate binefacerile și de toată starea cea bună cu care-l voi învrednici".
\par 10 Așa zice Domnul: "În locul acesta de care voi ziceți că este pustiu și lipsit de oameni și de vite, precum și în cetățile lui Iuda, și pe ulițele Ierusalimului, care sunt pustii, fără oameni, fără locuitori și fără vite,
\par 11 Iar se va auzi glas de bucurie, glas de veselie, glas de mire și glas de mireasă, glasul celor ce zic: "Slăviți pe Domnul Savaot, că bun este Domnul, că în veac este mila Lui" și glasul celor ce aduc jertfă de mulțumire în templul Domnului; căci voi întoarce robii țării acesteia la starea cea de altădată", zice Domnul.
\par 12 Așa zice Domnul Savaot: "În țara aceasta, care este pustie, fără locuitori și fără vite, și în toate cetățile ei iarăși vor fi locașuri de păstori, care-și vor odihni turmele lor.
\par 13 În cetățile cele de munte, în cetățile cele de la șes și în cetățile cele de la miazăzi, în pământul lui Veniamin, în împrejurimile Ierusalimului și în cetățile lui Iuda, iarăși se vor perinda turme pe sub mina celui ce numără, zice Domnul.
\par 14 Iată vin zilele când voi împlini acel cuvânt bun pe care l-am rostit Eu pentru casa lui Israel și pentru casa lui Iuda, zice Domnul.
\par 15 În zilele acelea și în vremea aceea voi ridica lui David Odraslă dreaptă și aceea va face judecată și dreptate pe pământ.
\par 16 În zilele acelea Iuda va fi izbăvit și Ierusalimul va trăi fără primejdie Și Odraslei aceleia I se va pune numele: "Domnul - dreptatea noastră"
\par 17 Că așa zice Domnul: "Nu va lipsi lui David bărbat care să șadă pe scaunul casei lui Israel.
\par 18 Și preoții-leviți nu vor avea lipsă de bărbat care să stea înaintea feței Mele și să aducă în toate zilele arderi de tot, să aprindă tămâie și să săvârșească jertfe".
\par 19 Și a mai fost cuvântul Domnului către Ieremia, zicând:
\par 20 "Așa zice Domnul: De poți strica legământul Meu cel pentru ziuă și legământul Meu cel pentru noapte și să faci ea ziua și noaptea să nu mai vină ta vremea lor, atunci poate se va strica și legământul Meu cel încheiat ou robul Meu David,
\par 21 Ca să nu mai aibă el fiu care să domnească pe scaunul lui; de asemenea și cel încheiat cu leviții-preoți, slujitorii Mei.
\par 22 Precum e nenumărată oștirea cerească și nenumărat nisipul mării, așa voi înmulți neamul lui David, robul Meu, și al leviților celor ce-Mi slujesc Mie".
\par 23 Fost-a iarăși cuvântul Domnului către Ieremia, zicând:
\par 24 Nu vezi tu oare că poporul acesta zice: "Domnul a lepădat cele două seminții pe care le alesese El?" Și ei disprețuiesc poporul Meu, ca și cum acesta n-ar fi popor în ochii lor.
\par 25 Așa zice Domnul: Dacă legământul Meu cel pentru ziuă și pentru noapte și rânduiala cerului și a pământului nu le-am întărit Eu,
\par 26 Atunci și neamul lui Iacov și al lui David, robul Meu, îl voi lepăda și nu voi mai lua stăpânitori din neamul lui pentru seminția lui Avraam și a lui Isaac și a lui Iacov, căci voi aduce înapoi pe prinșii lor de război și-i voi milui".

\chapter{34}

\par 1 Cuvântul care a fost de la Domnul către Ieremia, când Nabucodonosor, regele Babilonului, și toată oștirea lui și toate regatele pământului, supuse sub mâna lui, și toate popoarele au venit cu război împotriva Ierusalimului și împotriva tuturor cetăților lui luda, zicând:
\par 2 "Așa zice Domnul Dumnezeul lui Israel: Mergi și grăiește lui Sedechia, regele lui Iuda, și-i spune: Așa zice Domnul: lată voi da cetatea aceasta în mâinile regelui Babilonului, și acesta o va arde cu foc.
\par 3 Nici tu nu vei scăpa din mina lui, ci vei fi cu adevărat luat și dat în mâinile lui; ochii tăi vor vedea ochii regelui Babilonului și buzele lui vor grăi buzelor tale și vei merge la Babilon.
\par 4 Dar ascultă cuvântul Domnului, Sedechia, rege al lui Iuda! Așa zice Domnul de tine: Nu vei muri de sabie, ci vei muri în pace.
\par 5 Și precum s-a aprins tămâie pentru părinții tăi și pentru ceilalți regi, care au fost înainte de tine, la înmormântarea lor, așa se va aprinde și pentru tine și te vor plânge, zicând: "Vai, doamne!" Că Eu am rostit cuvântul acesta", zice Domnul.
\par 6 Și a spus Ieremia proorocul toate cuvintele acestea lui Sedechia, regele lui Iuda, în Ierusalim.
\par 7 În vremea aceea oștirea regelui Babilonului se războia împotriva Ierusalimului și împotriva tuturor cetăților lui Iuda, care mai rămăseseră, și anume împotriva Lachișului și Azecăi, deoarece din cetățile lui Iuda numai acestea mai rămăseseră cetăți întărite.
\par 8 Cuvântul care a fost de la Domnul către Ieremia, după ce regele Sedechia a încheiat legământ cu tot poporul ce era în Ierusalim,
\par 9 Ca să dea slobozenie, așa încât fiecare să dea drumul robului său și roabei sale, evreu și evreică, și ca nimeni să nu mai țină în robie evreu, frate al său.
\par 10 Și s-au supus toți cei mari și tot poporul, care au încheiat legământul, ca să dea fiecare drumul robului său și roabei sale, ca să nu-i mai țină pe: viitor în robie, și au ascultat și s-au supus.
\par 11 Dar după aceea, răzgândindu-se, au început să ia înapoi pe robii și pe roabele cărora le dăduseră drumul și-i siliră să le fie robi și roabe.
\par 12 Dar a fost cuvântul Domnului către Ieremia, căruia i s-a zis din partea Domnului:
\par 13 "Așa zice Domnul Dumnezeul lui Israel: Eu am încheiat un legământ cu părinții voștri, când i-am scos din pământul Egiptului și din casa robiei, și am zis:
\par 14 La sfârșitul anului al șaptelea să dea drumul fiecare din voi fratelui său , evreu, care ți s-a vândut ție; să-ți lucreze el șase ani, iar după aceea să-i dai drumul să fie slobod. Dar părinții voștri nu m-au ascultat și nu ți-au plecat urechea lor.
\par 15 Acum voi v-ați întors și ați făcut cum e drept în ochii Mei, dând fiecare slobozenie aproapelui său, și ați încheiat înaintea Mea legământ în templul asupra căruia s-a chemat numele Meu.
\par 16 Dar apoi v-ați răzgândit și ați necinstit numele Meu și ați întors fiecare pe robul vostru și pe roaba voastră cărora le dăduserăți drumul să se ducă unde le place, și-i siliți să vă fie robi și roabe.
\par 17 De aceea, așa zice Domnul: Voi nu v-ați supus Mie, ca să dați fiecare drumul aproapelui vostru; pentru aceea iată, Eu, zice Domnul, voi da libertate sabiei asupra voastră, ciumei și foametei și vă voi face "pricină" de groază pentru toate regatele pământului.
\par 18 Pe cei ce au călcat legământul Meu și pe cei ce nu s-au ținut de cuvintele legământului pe care l-am încheiat înaintea feței Mele, despicând în două vițelul și trecând printre cele două bucăți ale lui,
\par 19 Pe cei mari ai lui Iuda și pe cei mari ai Ierusalimului, pe eunuci, pe preoți și pe tot poporul țării, care au trecut printre bucățile vițelului despicat,
\par 20 Îi voi da în mâinile vrăjmașilor și în mâinile celor ce vor să le ia viața, iar trupurile lor vor fi hrană păsărilor cerului și fiarelor pământului.
\par 21 Pe Sedechia, regele lui Iuda, și pe cei mari ai lui îi voi da în mâinile vrăjmașilor lor și în mâinile celor ce vor să le ia viața, în mâinile oștirii regelui Babilonului, care s-a retras de la voi.
\par 22 Iată Eu voi da poruncă, zice Domnul, și-i voi întoarce la cetatea aceasta și vor năvăli asupra ei și o vor lua și o vor arde cu foc, și cetățile lui Iuda le vor face pustiu nelocuit".

\chapter{35}

\par 1 Cuvântul care a fost de la Domnul către Ieremia, în zilele lui Ioiachim, fiul lui Iosia, regele lui Iuda:
\par 2 "Mergi în casa Recabiților și vorbește cu ei și-i adu în templul Domnului, într-o cămară, și dă-le să bea vin".
\par 3 Atunci am luat eu pe Iaazania, fiul lui Ieremia, fiul lui Habaținia, pe frații lui, pe toți fiii lui și toată casa Recabiților
\par 4 și i-am adus în templul Domnului, în cămara fiilor lui Hanan, fiul lui Igdalia, omul lui Dumnezeu, care e lângă cămara dregătorilor, deasupra camerei lui Maaseia, fiul lui Șalum, păzitorul pragului.
\par 5 Apoi am pus înaintea fiilor casei Recabiților cupe pline cu vin și pahare și le-am zis: "Beți vin!"
\par 6 Dar ei au zis: "Noi nu bem vin, pentru că Ionadab, fiul lui Recab, tatăl nostru, ne-a dat poruncă, zicând: Să nu beli vin nici voi, nici fiii voștri în veac!
\par 7 Nici case să nu zidiți, nici semințe să nu semănați, nici vii să nu sădiți, nici să aveți, ci să trăiți în corturi în toate zilele vieții voastre, ca să trăiți vreme îndelungată pe pământul în care sunteți călători.
\par 8 Și noi am ascultat glasul lui Ionadab, fiul lui Recab, tatăl nostru, întru toate câte ne-a poruncit el, ca să nu bem vin în toate zilele noastre nici noi, nici femeile noastre, nici fiii noștri, nici fiicele noastre,
\par 9 Și să nu zidim case ca locuințe pentru noi și să nu avem nici vii; nici țarini, nici semănături,
\par 10 Ci să trăim în corturi. Întru toate ascultăm și facem toate câte ne-a poruncit Ionadab, strămoșul nostru.
\par 11 Când însă a venit Nabucodonosor, regele Babilonului, împotriva acestei țări, noi am zis: Hai să intrăm în Ierusalim dinaintea oștirilor Caldeilor și a oștirilor Sirienilor! Și iată acum trăim în Ierusalim".
\par 12 Atunci a fost cuvântul Domnului către Ieremia și a zis:
\par 13 "Așa zice Domnul Savaot, Dumnezeul lui Israel: Mergi și spune bărbaților lui Iuda și locuitorilor Ierusalimului: Se poate oare să nu luați voi învățătură din aceasta și să nu ascultați cuvântul Meu?" - zice Domnul.
\par 14 Cuvintele lui Ionadab, fiul lui Recab, pe care le-a spus fiilor săi de a nu bea vin, se împlinesc, și ei nu beau până în ziua de astăzi, pentru că se supun celor rânduite de tatăl lor; iar Eu necontenit v-am vorbit, v-am vorbit dis-de-dimineață și voi nu M-ați ascultat.
\par 15 Trimis-am la voi pe toți proorocii, robii Mei, i-am trimis dis-de-dimineață și am zis: Să se întoarcă fiecare de la calea lui cea rea și să vă îndreptați purtările, să nu mergeți după alți dumnezei, ca să le slujiți, și veți trăi în pământul acesta pe care l-am dat vouă și părinților voștri; dar voi nu v-ați plecat urechea și nu M-ați ascultat.
\par 16 Deoarece fiii lui Ionadab, fiul lui Recab, împlinesc porunca pe care le-a dat-o tatăl lor, iar poporul Meu nu Mă ascultă,
\par 17 De aceea așa zice Domnul Dumnezeu Savaot, Dumnezeul lui Israel: Iată Eu voi aduce asupra lui Iuda și a tuturor locuitorilor Ierusalimului tot răul acela pe care l-am rostit asupra lor, pentru că Eu le-am vorbit și ei n-au ascultat, i-am chemat și ei n-au răspuns".
\par 18 Iar casei Recabiților, Ieremia i-a grăit: "Așa zice Domnul Savaot, Dumnezeul lui Israel: Pentru că voi ați ascultat porunca lui Ionadab, tatăl vostru, și păziți toate poruncile lui și în toate faceți cum v-a poruncit el,
\par 19 De aceea, așa zice Domnul Savaot, Dumnezeul lui Israel: Nu va lipsi lui Ionadab, fiul lui Recab, bărbatul care să stea înaintea felei Mele în toate zilele".

\chapter{36}

\par 1 În anul al patrulea al lui Ioiachim, fiul lui Iosia, regele lui Iuda, a fost de la Domnul către Ieremia cuvântul acesta:
\par 2 "Ia-ți un sul de hârtie și scrie pe el toate cuvintele pe care ți le-am grăit Eu despre Israel, despre Iuda și despre toate popoarele, din ziua de când am început a-ți grăi, din zilele lui Iosia și până în ziua de astăzi;
\par 3 Că poate va auzi casa lui Iuda tot răul ce Mi-am pus în gând să i-l fac, ca să se întoarcă fiecare de la calea sa cea rea, pentru ca Eu să le iert nedreptățile lor și păcatele lor".
\par 4 Atunci a chemat Ieremia pe Baruh, fiul lui Neria, și după spusa lui Ieremia a scris Baruh pe hârtia sulului toate cuvintele Domnului pe care le grăise El aceluia.
\par 5 Și a poruncit Ieremia lui Baruh și a zis: "Eu sunt închis și nu mă pot duce în templul Domnului.
\par 6 Deci, du-te tu și cuvintele Domnului din gura mea scrise de tine în sul, citește-le în templul Domnului, în ziua postului, în auzul poporului; citește-le de asemenea și în auzul tuturor acelora din Iuda, care sunt veniți de prin toate cetățile;
\par 7 Poate vor înălța rugă smerită înaintea feței Domnului și se va întoarce fiecare de la calea sa cea rea, căci mare este mânia și supărarea pe care a arătat-o Domnul asupra poporului acestuia".
\par 8 Și a făcut Baruh, fiul lui Neria, tot ce i-a poruncit Ieremia, citind în templul Domnului cuvintele Domnului cele scrise în sul.
\par 9 În anul al cincilea al lui Ioiachim, fiul lui Iosia, regele lui Iuda, în luna a opta, a vestit post înaintea feței Domnului pentru tot poporul din Ierusalim și pentru tot poporul care venise la Ierusalim de prin orașele lui Iuda.
\par 10 și cuvintele lui Ieremia, cele scrise în sul, le-a citit Baruh în templul Domnului, în camera lui Ghemaria, fiul lui Șafan, scriitorul, în curtea de sus, la intrarea de la poarta cea nouă a templului Domnului, în auzul poporului.
\par 11 Iar Miheia, fiul lui Ghemaria, fiul lui Șafan, a auzit toate cuvintele Domnului cele scrise în sul.
\par 12 Și s-a coborât în casa regelui, în camera secretarului regelui și iată acolo stăteau toți dregătorii: Elișama, secretarul regelui, Delaia, fiul lui Șemaia, Elnatan, fiul lui Acbor, Ghemaria, fiul lui Șafan, Sedechia, fiul lui Hanania și toți ceilalți dregători.
\par 13 Și le-a povestit acestora Miheia toate cuvintele ce le auzise, când a citit Baruh cartea în auzul poporului.
\par 14 Atunci dregătorii au trimis la Baruh pe Iehudi, fiul lui Netania, fiul lui Șelemia, fiul lui Cuși, ca să-i spună: "Cartea pe care tu ai citit-o în auzul poporului, să o iei în mâna ta și să vii".
\par 15 Atunci Baruh, fiul lui Neria, a luat cartea în mâna sa și a venit la ei. Iar ei i-au zis: "Șezi și ne citește în auz!" Și a citit Baruh în auzul lor.
\par 16 Iar dacă au ascultat ei toate cuvintele, s-au uitat cu groază unii la alții și au zis către Baruh: "Vom spune numaidecât regelui toate cuvintele acestea!"
\par 17 Și apoi au întrebat pe Baruh: "Spune-ne insă cum ai scris tu din gura lui toate cuvintele acestea?"
\par 18 Și Baruh le-a spus: "El mia rostit cu gura sa toate cuvintele acestea, iar eu le-am scris cu cerneală în sulul acesta".
\par 19 Atunci au zis dregătorii către Baruh: "Du-te și te ascunde și tu și Ieremia, ca nimeni să nu știe unde sunteți!"
\par 20 Apoi ei s-au dus la rege în palat, iar cartea au lăsat-o în odaia lui Elișama, secretarul regelui, și au spus în auzul regelui toate cuvintele acestea.
\par 21 Și a trimis regele pe Iehudi să aducă îndată cartea, și acesta a luat-o din camera lui Elișama, secretarul regelui; și a citit-o Iehudi în auzul regelui și în auzul tuturor dregătorilor care stăteau lângă rege.
\par 22 În vremea aceea, în luna a noua, regele ședea în palatul de iarnă și înaintea lui ardea o tavă cu jeratic.
\par 23 După ce a citit Iehudi trei sau patru coloane, regele a tăiat cu cuțitașul secretarului cartea și a aruncat-o în focul de pe tavă, nimicind-o toată.
\par 24 Și nu s-au temut, nici și-au sfâșiat hainele lor, nici regele, nici toți slujitorii lui, care au auzit toate cuvintele acestea.
\par 25 Deși Elnatan, Delaia și Ghemaria au rugat pe rege să nu ardă cartea, el însă nu i-a ascultat.
\par 26 Și a poruncit regele lui Ierahmeel, fiul regelui, și lui Seraia, fiul lui Azirel, și lui Șelemia, fiul lui Abdeel, să prindă pe scriitorul Baruh și pe Ieremia proorocul. Dar Domnul i-a ascuns.
\par 27 Atunci a fost cuvântul Domnului către Ieremia, după ce regele a ars cartea și cuvintele pe care Baruh le scrisese din gura lui Ieremia, și i-a zis:
\par 28 "Ia-ți alt sul de hârtie și scrie în el toate cuvintele de mai înainte, care au fost în celălalt sul, pe care l-a ars Ioiachim, regele lui Iuda, iar regelui lui Iuda, Ioiachim, să-i spui:
\par 29 Așa zice Domnul: Tu ai ars cartea aceasta, zicând: Pentru ce ai scris în ea: Va veni îndată regele Babilonului și va pustii țara aceasta și va pierde oamenii și dobitoacele din ea?
\par 30 De aceea, așa zice Domnul despre Ioiachim, regele lui Iuda: Nu va mai fi din el urmaș care să șadă pe scaunul lui David; trupul lui va fi aruncat în arșița zilei și în frigul nopții
\par 31 Și-l voi pedepsi pe el și neamul lui și slugile lui pentru nedreptatea lor; voi aduce asupra lor și asupra locuitorilor Ierusalimului și asupra oamenilor lui Iuda tot răul pe care l-am rostit asupra lor și ei n-au ascultat".
\par 32 Și a luat Ieremia alt sul de hârtie și l-a dat lui Baruh scriitorul, fiul lui Neria, și acesta a scris în el, din gura lui Ieremia, toate cuvintele din sulul ce-l aruncase în foc Ioiachim, regele lui Iuda, și a mai adăugat la ele multe cuvinte asemenea acelora.

\chapter{37}

\par 1 În locul lui Iehonia, fiul lui Ioiachim, a fost făcut rege Sedechia, fiul lui Iosia, pe care Nabucodonosor, regele Babilonului, l-a pus rege în țara lui Iuda.
\par 2 Dar nici el, nici slujitorii lui și nici poporul țării n-au ascultat cuvintele Domnului, cele grăite prin proorocul Ieremia.
\par 3 Regele Sedechia a trimis pe Iucal, fiul lui Șelemia, și pe Sofonie preotul, fiul lui Maaseia, la Ieremia proorocul, ca să-i zică: "Roagă-te pentru noi Domnului Dumnezeului nostru!"
\par 4 Căci pe atunci Ieremia intra și umbla încă slobod prin popor, pentru că nu-l aruncaseră încă în închisoare.
\par 5 Oștirea lui Faraon a ieșit din Egipt, iar Caldeii, care împresuraseră Ierusalimul, auzind vestea aceasta, s-au retras de la Ierusalim.
\par 6 Atunci a fost cuvântul Domnului către Ieremia proorocul și a zis:
\par 7 "Așa grăiește Domnul, Dumnezeul lui Israel: Regelui lui Iuda, care v-a trimis la Mine să Mă întrebați, așa să-i spuneți: Iată, oștirea lui Faraon care v-a veni în ajutor se va întoarce în țara sa, în Egipt,
\par 8 Iar Caldeii vor veni din nou și vor lupta împotriva cetății acesteia, o vor lua și o vor arde cu foc.
\par 9 Așa zice Domnul: Nu vă înșelați pe voi înșivă, zicând: "Fără îndoială se vor duce de la noi Caldeii" că nu se vor duce.
\par 10 Chiar dacă ați, sfărâma cu totul oștirea Caldeilor, care luptă împotriva voastră, și ar rămâne la ei numai răniți, apoi și aceia s-ar ridica fiecare din cortul său și ar arde cetatea aceasta cu foc".
\par 11 În vremea când oștirea Caldeilor s-a retras de la Ierusalim din pricina oștirii egiptene,
\par 12 Ieremia a plecat din Ierusalim, ca să se ducă în țara lui. Veniamin, să descurce niște treburi de moștenire cu cei de acolo.
\par 13 Dar când se afla el la poarta lui Veniamin, căpetenia gărzii, care era acolo și care se numea Ireia, fiul lui Șelemia, fiul lui Hanania, a prins pe Ieremia proorocul și i-a zis: "Tu vrei să fugi la Caldei!"
\par 14 "Aceasta este minciună; eu nu vreau să fug la Caldei", a zis Ieremia. Dar fără să-l asculte, Ireia l-a arestat pe Ieremia și l-a dus la dregători.
\par 15 Și s-au mâniat dregătorii pe Ieremia, l-au bătut și l-au închis în temniță, în casa lui Ionatan scriitorul, pentru că o făcuseră temniță.
\par 16 Și după ce a intrat Ieremia în temniță, în beci, și a stat acolo Ieremia zile multe,
\par 17 Regele Sedechia a trimis și i-a adus. Și l-a întrebat regele în casa sa în taină și i-a zis: "N-ai oare cuvânt de la Domnul?" Iar Ieremia a zis: "Ba am!" Și a adăugat: "Tu vei fi dat în mâinile regelui Babilonului!"
\par 18 Și a mai zis Ieremia către regele Sedechia: "Cu ce am greșit eu înaintea ta, înaintea slujitorilor tăi și înaintea poporului acestuia, de m-ați băgat în temniță?
\par 19 Și unde sunt proorocii voștri care v-au proorocit și au zis: "Regele Babilonului nu va veni împotriva voastră și împotriva pământului acestuia?"
\par 20 Și acum ascultă, domnul meu rege, să aibă trecere cererea mea înaintea ta și să nu mă mai întorc în casa lui Ionatan scriitorul, ca să nu mor acolo".
\par 21 Și a dat regele Sedechia poruncă să închidă pe Ieremia în curtea temniței și i s-a dat câte o bucată de pâine pe zi din ulița pitarilor, până s-a isprăvit pâinea în cetate. Și așa a rămas Ieremia în curtea temniței.

\chapter{38}

\par 1 Și Șefatia, fiul lui Matan, și Ghedalia, fiul lui Pashor, Iucal, fiul lui Șelemia și Pașhurr, fiul lui Malchia, au auzit cuvintele pe care le-a rostit Ieremia către tot poporul, zicând:
\par 2 "Așa zice Domnul: Cine va rămâne în această cetate va muri de sabie, de foame și de boala ciumei; iar cel ce se va duce la Caldei va trăi, va avea ca pradă viața lui și va rămâne viu. "
\par 3 Așa zice Domnul: Cetatea aceasta va fi dată fără îndoială în mâinile oștirilor regelui Babilonului, și o vor lua".
\par 4 Atunci căpeteniile au zis către rege: "Omul acesta să fie dat morții, pentru că el descurajează pe luptătorii care au rămas în cetatea aceasta și pe tot poporul, spunându-le asemenea cuvinte. Căci omul acesta nu dorește propășirea poporului său, ci nenorocirea".
\par 5 Iar regele Sedechia a zis: "Iată e în mâinile voastre, căci regele nu poate să facă nimic împotriva voastră".
\par 6 Atunci au luat pe Ieremia și l-au aruncat în groapa lui Malchia, fiul regelui, care era în curtea temniții, coborându-l în ea cu funii. în groapa aceea nu era apă, ci numai noroi și s-a afundat Ieremia în noroi.
\par 7 Ebed-Melec, Etiopianul, unul din eunucii care se aflau în casa regelui, auzind că Ieremia a fost aruncat în groapă - regele atunci ședea la poarta lui Veniamin -
\par 8 A ieșit din palatul regal și a zis către rege:
\par 9 "Domnul meu rege, rău au făcut oamenii aceștia care s-au purtat așa cu Ieremia proorocul, pe care l-au aruncat în groapă. El va muri acolo de foame, pentru că nu mai este pâine în cetate".
\par 10 Atunci regele a dat poruncă lui Ebed-Melec, Etiopianul, zicând: "Ia de aici treizeci de oameni și scoate pe Ieremia proorocul din groapă, până nu moare".
\par 11 Și a luat Ebed-Melec cu sine oamenii și a intrat în palatul regal, în veșmântărie, și a luat de acolo bucăți de haine vechi și rupturi și le-a dat drumul cu frânghia în groapă la Ieremia.
\par 12 Și a zis Ebed-Melec, Etiopianul, către Ieremia: "Pune aceste cârpe și rupturi vechi, ce ți s-au aruncat, la subsuorile tale, sub frânghie"; și a făcut Ieremia așa.
\par 13 Și au tras pe Ieremia din groapă; și a rămas Ieremia în curtea temniții.
\par 14 Atunci regele Sedechia a trimis și a chemat pe Ieremia proorocul la sine, la ușa a treia a templului Domnului; și a zis regele către Ieremia: "Am să te întreb ceva, dar să nu ascunzi nimic de mine".
\par 15 Iar Ieremia a zis către Sedechia: "Dacă-ți voi descoperi, nu mă vei da oare morții? Și de-ți voi da sfat, îl vei asculta tu oare?"
\par 16 Zis-a regele Sedechia către Ieremia: "Viu este Domnul, Care ne-a dat această viață, nu te voi da la moarte, nici în mâinile acestor oameni, care vor să-ți ia viața, nu te voi da".
\par 17 Atunci a zis Ieremia către Sedechia: "Așa zice Domnul Dumnezeul Savaot, Dumnezeul lui Israel: Dacă tu vei ieși la căpeteniile regelui Babilonului, vei scăpa cu viață și cetatea aceasta nu va fi arsă cu foc și vei trăi și tu și casa ta;
\par 18 Iar de nu vei ieși la căpeteniile regelui Babilonului, cetatea aceasta va fi dată în mâinile Caldeilor, care o vor arde cu foc, și nici tu nu vei scăpa din mâinile lor".
\par 19 Iar regele Sedechia a zis către Ieremia: "Mă tem de Iudeii care au trecut la Caldei, ca nu cumva să nu fiu dat de Caldei pe mâna lor și ca nu cumva aceia să-și bată, joc de mine".
\par 20 Zis-a Ieremia: "Nu te vor da! Ascultă glasul Domnului în cele ce-ți grăiesc eu și bine-ți va fi și sufletul tău va fi viu.
\par 21 Iar dacă tu nu vei vrea să ieși, atunci iată cuvântul pe care mi l-a descoperit Domnul:
\par 22 Toate femeile care au rămas în casa regelui lui Iuda var fi duse la căpeteniile regelui Babilonului și acelea vor zice: Te-au amăgit și te-au înșelat bunii tăi prieteni; piciorul tău s-a afundat în noroi și aceștia au fugit de tine;
\par 23 Și toate femeile tale și copiii tăi vor fi dați Caldeilor și nici tu nu vei scăpa din mâinile lor, căci vei fi prins de mâna regelui Babilonului și această cetate va fi arsă".
\par 24 Zis-e Sedechia către Ieremia: "Nimeni nu trebuie să știe cuvintele acestea și atunci tu nu vei muri.
\par 25 Iar de vor auzi căpeteniile că eu am grăit cu tine și de vor veni la tine și-ți vor zice: "Spune-ne ce ai vorbit cu regele? Nu ascunde de noi și nu te vom da la moarte! Și ce ți-e vorbit regele?"
\par 26 Atunci să le spui: Am prezentat înaintea regelui cererea mea, ca să nu mă trimită înapoi în casa lui Ionatan și să mor acolo".
\par 27 Atunci au venit toți dregătorii la Ieremia și l-au întrebat, iar el le-a răspuns întocmai cum îi poruncise regele, și aceia, tăcând, l-au lăsat, pentru că n-au aflat nimic.
\par 28 Și a rămas Ieremia în curtea temniții până în ziua în care a fost luat Ierusalimul, căci Ierusalimul a fost luat.

\chapter{39}

\par 1 În anul al nouălea al lui Sedechia, regele lui Iuda, în luna a zecea, Nabucodonosor, regele Babilonului, a venit cu toată oștirea sa la Ierusalim și l-a împresurat.
\par 2 Iar în anul al unsprezecelea al lui Sedechia, în luna a patra, în ziua a noua a lunii acesteia, cetatea a fost luată.
\par 3 Și au intrat în ea și s-au așezat la porțile de la mijloc toate căpeteniile regelui Babilonului: Nergal-Șarețer, Samgar-Nebu, Sarsehim, căpetenia eunucilor, Nergal-Sarețer, căpetenia vrăjitorilor, și toate celelalte căpetenii ale regelui Babilonului.
\par 4 Când Sedechia, regele lui Iuda, și toți oamenii de oaste i-au văzut, au fugit și au ieșit noaptea din cetate prin grădina regelui, pe poarta dintre cele două ziduri, și au apucat pe calea șesului.
\par 5 Dar oștirea Caldeilor a alergat după ei și a ajuns pe Sedechia în șesul Ierihonului, l-au prins și l-au dus la Nabucodonosor, regele Babilonului, la Ribla, în țara Hamat, unde acesta a rostit judecată asupra lui.
\par 6 Atunci regele Babilonului a junghiat pe fiii lui Sedechia în Ribla, înaintea ochilor acestuia, și pe toți dregătorii lui Iuda i-a junghiat regele Babilonului.
\par 7 Iar lui Sedechia i-a scos ochii și l-a pus în cătușe, ca să-l ducă la Babilon.
\par 8 Casa regelui și casele poporului le-au ars Caldeii cu foc, iar zidurile Ierusalimului le-au dărâmat.
\par 9 Restul poporului, care rămăsese în cetate, pe fugarii, care trecuseră la el și pe celălalt popor ce mai rămăsese, Nebuzaradan, căpetenia gărzii regelui, i-a dus robi în Babilon.
\par 10 Pe cei săraci din popor, care nu aveau nimic, Nebuzaradan, căpetenia gărzii, i-a lăsat în pământul lui Iuda și tot atunci le-a dat viile și țarinile.
\par 11 Iar cu privire la Ieremia, Nabucodonosor, regele Babilonului, a dat lui Nebuzaradan, căpetenia gărzii, porunca aceasta:
\par 12 "Ia-l și să ai purtare de grijă pentru el; să nu-i faci nici un rău, ci să te porți cu el așa cum îți va zice el!"
\par 13 Nebuzaradan, căpetenia gărzii, Nebuzaradan, căpetenia eunucilor, Nergal-Șarețer, căpetenia vrăjitorilor, și toate căpeteniile regelui Babilonului
\par 14 Au trimis și au luat pe Ieremia din curtea temniții și l-au încredințat lui Godolia, fiul lui Ahicam, fiul lui Șafan, ca să-l ducă acasă. Așa a rămas el să locuiască în mijlocul poporului.
\par 15 Atunci a fost cuvântul Domnului către Ieremia, pe când era el încă ținut în curtea temniții, și i-a zis:
\par 16 "Mergi și spune lui Ebed-Melec, Etiopianul: Așa zice Domnul Savaot, Dumnezeul lui Israel: Iată, Eu voi împlini cuvintele Mele despre cetatea aceasta spre răul, iar nu spre binele ei, și se vor împlini ele în ziua aceea înaintea ochilor tăi.
\par 17 Dar pe tine te voi izbăvi, zice Domnul, și nu vei fi dat în mâinile oamenilor de care te temi.
\par 18 Te voi izbăvi și nu vei cădea în sabie; sufletul tău va rămâne la tine ca și când ai fi dobândit o pradă, pentru că tu ti-ai pus nădejdea în Mine", zice Domnul.

\chapter{40}

\par 1 Cuvântul care a fost de la Domnul către Ieremia, după ce Nebuzaradan, căpetenia gărzii, i-a dat drumul din Rama, unde fusese găsit ferecat în lanțuri în mijlocul celorlalți robi ai Ierusalimului și ai lui Iuda, care au fost strămutați la Babilon.
\par 2 Atunci a luat căpetenia gărzii pe Ieremia și i-a zis: "Domnul Dumnezeul tău a rostit această nenorocire asupra locului acestuia,
\par 3 Și acum a adus-o peste el și a făcut ceea ce zisese, pentru că ați păcătuit înaintea Domnului și n-ați ascultat glasul Lui, de aceea v-a și ajuns pe voi aceasta.
\par 4 Deci, iată eu îți dau drumul din lanțurile ce le ai la mâini! De vrei să mergi cu mine la Babilon, vino, și eu voi avea grijă de tine, iar de nu ai plăcere să mergi cu mine la Babilon, rămâi. Iată, toată țara e înaintea ta, unde vrei Și unde-ți place, acolo du-te!"
\par 5 Înainte de a ieși el, Nebuzaradan i-a zis: "Du-te la Godolia, fiul lui Ahicam, fiul lui Șafan, pe care regele Babilonului l-a pus căpetenie peste cetatea Ierusalimului, și rămâi cu dânsul în mijlocul poporului; sau du-te unde-ți place ție să te duci!" Și căpetenia gărzii i-a dat merinde și daruri și l-a slobozit.
\par 6 Deci a venit Ieremia la Godolia, fiul lui Ahicam, la Mițpa, și a trăit cu el în mijlocul poporului care rămăsese în țară.
\par 7 Auzind toate căpeteniile oștirilor, care erau în câmp cu oamenii lor, că regele Babilonului î pus pe Godolia, fiul lui Ahicam, căpetenie peste țară și i-a încredințat lui bărbații, femeile și copiii și pe aceia din săracii țării, care n-au fost strămutați la Babilon,
\par 8 Au venit la Godolia, în Mițpa, Ismael, fiul lui Netania, Iohanan și Ionatan, fiii lui Carea, Seraia, fiul lui Tanhumet, și fiii lui Efai Netofitul și Iezania, fiul lui Maacat, ei și oamenii lor.
\par 9 Iar Godolia, fiul lui Ahicam, fiul lui Șafan, li s-a jurat lor și oamenilor lor, zicând: "Nu vă temeți a sluji Caldeilor; rămâneți în țară, slujiți regelui Babilonului și vă va fi bine.
\par 10 Iar eu voi rămâne în Mițpa, ca să mijlocesc înaintea Caldeilor, care vor veni la noi. Voi însă strângeți vinul și fructele verii și untdelemnul și le puneți în vasele voastre și trăiți în cetățile voastre, în care vă aflați".
\par 11 De asemenea și toți Iudeii care se aflau în Moab, la Amoniți, în Idumeea și prin toate țările, au auzit că regele Babilonului a lăsat o parte din Iudei și a pus peste ei pe Godolia, fiul lui Ahicam, fiul lui Șafan.
\par 12 Și s-au întors toți acești Iudei de prin toate locurile pe unde fuseseră împrăștiați și au venit în pământul lui Iuda, la Godolia, în Mițpa, și au adunat vin și fructe multe foarte.
\par 13 Iar Iohanan, fiul lui Carea, și toate căpeteniile oștirii, care erau în câmp, au venit la Godolia, în Mițpa, și au zis către dânsul:
\par 14 "Știi tu oare că Baalis, regele Amoniților, a trimis pe Ismael, fiul lui Netania, ca să te ucidă?" Dar Godolia, fiul lui Ahicam, nu i-a crezut.
\par 15 Atunci Iohanan, fiul lui Carea, a spus lui Godolia în taină, la Mițpa: "Dă-mi voie să mă duc să ucid pe Ismael, fiul lui Netania, și nimeni nu va afla. De ce să-l lași să te ucidă el pe tine și să se risipească toți cei din Iuda, care s-au adunat la tine și să piară rămășița lui Iuda?"
\par 16 Însă Godolia, fiul lui Ahicam, a zis către Iohanan, fiul lui Carea: "Să nu faci aceasta, căci tu grăiești neadevăr despre Ismael!"

\chapter{41}

\par 1 Prin luna a șaptea, Ismael, fiul lui Netania, fiul lui Elișama, din neam regesc, a venit cu dregătorii regelui și cu zece oameni la Godolia, fiul lui Ahicam, în Mițpa, și au mâncat acolo pâine împreună.
\par 2 Apoi s-a sculat Ismael, fiul lui Netania, și cei zece oameni care erau cu el, și au lovit cu sabia pe Godolia, fiul lui Ahicam, fiul lui Șafan, și au ucis pe acela pe care regele Babilonului îl pusese căpetenie peste țară.
\par 3 De asemenea a ucis Ismael și pe toți Iudeii care erau cu Godolia, în Mițpa, precum și pe Caldeii oșteni care se aflau acolo.
\par 4 Iar a doua zi după uciderea lui Godolia, pe când nu știa nimeni de aceasta,
\par 5 Au venit din Sihem, din Șilo și din Samaria optzeci de oameni cu bărbile rase și cu hainele sfâșiate și cu tăieturi pe trup, aducând daruri și tămâie în mâinile lor pentru jertfă în templul Domnului.
\par 6 Iar Ismael, fiul lui Netania, din Mițpa, a ieșit întru întâmpinarea lor; și mergând el plângând, s-a întâlnit cu ei și le-a zis: "Duceți-vă la Godolia, fiul lui Ahicam".
\par 7 Și după ce au ajuns ei în mijlocul cetății, Ismael, fiul lui Netania, i-a ucis și i-a aruncat într-o groapă, ajutat de oamenii care erau cu el.
\par 8 Dar s-au găsit printre aceștia zece oameni care au zis lui Ismael: "Nu ne ucide, căci noi avem agoniseli de grâu, de orz, de untdelemn și de miere, ascunse la câmp". Și el s-a oprit și nu i-a omorât pe aceia împreună cu ceilalți frați ai lor.
\par 9 Groapa însă în care a aruncat Ismael toate trupurile oamenilor, pe care el i-a ucis pentru Godolia, era chiar aceea pe care o făcuse regele Asa, temându-se de Baeșa, regele lui Israel. Pe aceasta a umplut-o Ismael, fiul lui Netania, cu cei uciși.
\par 10 Apoi Ismael a luat c, prizonieri tot restul de popor, ce era în Mițpa, pe fiicele regelui și tot poporul ce mai rămăsese în Mițpa, pe care Nebuzaradan, căpetenia gărzii, îl încredințase lui Godolia, fiul lui Ahicam; și luându-i pe aceștia, Ismael, fiul lui Netania, s-a îndreptat spre fiii lui Amon.
\par 11 Dar când Iohanan, fiul lui Carea, și toți căpitanii oștirii, care erau cu el, au auzit de nelegiuirile ce săvârșise Ismael, fiul lui Netania, au luat toți oamenii lor și s-au dus să se bată cu Ismael, fiul lui Netania,
\par 12 Și I-au ajuns la iazul cel mare din Gabaon.
\par 13 Când tot poporul ce era cu Ismael a văzut pe Iohanan, fiul lui Carea, și pe toate căpeteniile oștirii care erau cu dânsul, s-au bucurat.
\par 14 Și s-a întors tot poporul pe care Ismael îl ducea în robie din Mițpa și, plecând, s-a dus la Iohanan, fiul lui Carea,
\par 15 Iar Ismael, fiul lui Netania, a fugit de Iohanan cu opt oameni și s-a dus la fiii lui Amon.
\par 16 Atunci Iohanan, fiul lui Carea, și toate căpeteniile oștirii, care erau cu el, au luat din Mițpa tot poporul ce rămăsese și pe care el îl scăpase de Ismael, fiul lui Netania, după ce acesta ucisese pe Godolia, fiul lui Ahicam - bărbații, oștenii, femeile, copiii și eunucii pe care îi scosese din Gabaon,
\par 17 Și s-a dus și s-a oprit în satul Chimham, lângă Betleem,
\par 18 Ca apoi să plece în Egipt de frica Caldeilor, căci se temea de ei, pentru că Ismael, fiul lui Netania, ucisese pe Godolia, fiul lui Ahicam, pe care regele Babilonului îl pusese căpetenie peste țară.

\chapter{42}

\par 1 Atunci toate căpeteniile oștirii și Iohanan, fiul lui Carea, și Azaria, fiul lui Hoșaia, și tot poporul, de la mic până la mare, au venit și au zis către Ieremia proorocul:
\par 2 "Să aibă primire rugămintea noastră înaintea ta! Roagă-te Domnului Dumnezeului tău pentru noi, pentru toți care au rămas, că din mulți, puțini au rămas, cum ne vezi cu ochii tăi;
\par 3 Roagă-te ca Domnul Dumnezeul tău să ne arate calea pe care să mergem și ce să facem".
\par 4 Zis-a către ei Ieremia proorocul: "Ascult și iată mă voi ruga Domnului Dumnezeului vostru, după cuvântul vostru, și tot ce vă va răspunde Domnul vă voi spune și nu voi ascunde de voi nici un cuvânt".
\par 5 Iar ei au zis către Ieremia: "Domnul să fie martor împotriva noastră, credincios și adevărat, dacă nu vom face întocmai după cuvântul pe care Domnul Dumnezeul tău ți-l va fi trimis pentru noi.
\par 6 Fie bine, fie rău, noi vom asculta de glasul Domnului Dumnezeului nostru, către Care te trimitem, ca să fim fericiți ascultând de glasul Domnului Dumnezeului nostru".
\par 7 După trecerea a zece zile, a fost cuvântul Domnului către Ieremia.
\par 8 Și a chemat acesta la el pe Iohanan, fiul lui Carea, și pe toate căpeteniile oștirii, care erau cu el, și tot poporul, de la mic până la mare, și le-a zis:
\par 9 "Așa zice Domnul Dumnezeul lui Israel către Care m-ați trimis, ca să aduc înaintea Lui rugămintea voastră:
\par 10 De veți rămâne în țara aceasta, Eu vă voi zidi și nu vă voi mai dărâma, vă voi sădi și nu vă voi mai smulge, căci Îmi pare rău de răul pe care vi l-am făcut.
\par 11 Nu vă temeți de regele Babilonului, de care vă cutremurați; nu vă temeți de el, zice Domnul, căci Eu sunt cu voi, ca să vă scap și să vă izbăvesc din mâna lui.
\par 12 și vă voi arăta milă și el se va milostivi spre voi și vă va întoarce în pământul vostru.
\par 13 Iar de veți zice: "Nu vrem să trăim în țara aceasta" și de nu veți asculta glasul Domnului Dumnezeului vostru și veți zice:
\par 14 "Ba noi ne ducem în țara Egiptului, unde război nu vom vedea și glas de trâmbiță nu vom auzi și foame nu vom duce, și acolo vom trăi",
\par 15 Apoi ascultați cuvântul Domnului, voi, care ați mai rămas din Iuda. Așa zice Domnul Savaot, Dumnezeul lui Israel: "Dacă voi vă întoarceți cu hotărâre nestrămutată fața voastră, ca să vă duceți în Egipt și vă veți duce să trăiți acolo,
\par 16 Sabia de care vă temeți vă va ajunge acolo, în țara Egiptului, și foamea de care vă îngroziți vă va însoți pașii voștri acolo în Egipt și acolo veți muri.
\par 17 Toți cei ce își întorc fața lor, ca să se ducă în Egipt și să trăiască acolo; vor muri de sabie, de foame și de molimă și nici unul din ei nu va rămâne și nu va scăpa de nenorocirea aceea pe care o voi aduce asupra lor.
\par 18 Căci așa zice Domnul Savaot, Dumnezeul lui Israel: Cum s-a vărsat mânia Mea și iuțimea Mea asupra locuitorilor Ierusalimului, așa se va vărsa iuțimea Mea și asupra voastră, când vă veți duce în Egipt și veți fi blestem și grozăvie, ocară și râs, și nu veți mai vedea locul acesta.
\par 19 Vouă, cei rămași ai lui Iuda, vă zice Domnul: Nu vă duceți în Egipt. Să știți bine că Eu astăzi v-am spus lucrul acesta în chip solemn.
\par 20 Vă faceți rău vouă înșivă, trimițându-mă la Domnul Dumnezeul vostru, zicând: "Roagă-te pentru noi Domnului Dumnezeului nostru, și tot ce va zice Domnul Dumnezeul nostru să ne spui și noi vom face".
\par 21 Așadar eu v-am spus acum, dar voi n-ați ascultat glasul Domnului Dumnezeului vostru și toate cu câte m-a trimis El la voi.
\par 22 Deci, să știți că veți muri de sabie, de foamete și de ciumă în locul acela, unde voiți să vă duceți ca să trăiți".

\chapter{43}

\par 1 După ce Ieremia a spus întregului popor toate cuvintele Domnului Dumnezeului lor, toate cuvintele acelea cu care Domnul Dumnezeul lor îl trimisese la ei,
\par 2 Atunci Azaria, fiul lui Hoșaia, Iohanan, fiul lui Carea, și toți oamenii cei îngâmfați au zis către Ieremia: "Neadevăr spui, nu te-a trimis Domnul Dumnezeul nostru să ne spui: "Nu vă duceți în Egipt ca să trăiți acolo!"
\par 3 Ci Baruh, fiul lui Neria, te ațâță împotriva noastră, ca să ne dea în mâinile Caldeilor, să ne omoare, sau să ne ducă robi la Babilon".
\par 4 Apoi Iohanan, fiul lui Carea, toate căpeteniile oștirii și tot poporul n-au ascultat glasul Domnului, ca să rămână în pământul lui Iuda.
\par 5 Și Iohanan, fiul lui Carea, și toate căpeteniile oștirii au luat pe toți cei care rămăseseră din Iuda și care se întorseseră din toate neamurile pe unde fuseseră izgoniți, ca să trăiască în pământul lui Iuda,
\par 6 Bărbați, femei și copii, fetele regelui și pe toți aceia pe care Nebuzaradan, căpetenia gărzii, îi lăsase cu Godolia, fiul lui Ahicam, fiul lui Șafan, pe Ieremia proorocul și pe Baruh, fiul lui Neria,
\par 7 Și s-au dus în țara Egiptului, că n-au ascultat glasul Domnului și au mers până la Tahpanhes.
\par 8 Iar la Tahpanhes a fost cuvântul Domnului către Ieremia, zicând:
\par 9 "Ia în mâinile tale niște pietre mari și le ascunde în lut frământat, la intrarea casei lui Faraon, în Tahpanhes, înaintea ochilor Iudeilor și să le spui:
\par 10 "Așa grăiește Domnul Savaot, Dumnezeul lui Israel: Iată, Eu voi trimite și voi lua pe Nabucodonosor. regele Babilonului, robul Meu, și voi așeza tronul lui pe aceste pietre. ascunse de Mine, și își va întinde el pe ele cortul său cel minunat.
\par 11 El va veni și va lovi țara Egiptului; cel rânduit spre moarte va fi dat morții, cel rânduit pentru robie va merge în robie și cel rânduit spre sabie va cădea de sabie.
\par 12 Voi aprinde foc în capiștile dumnezeilor Egiptului; Nabucodonosor le va arde pe acelea, pe idoli îi va distruge cu totul și se va îmbrăca cu țara Egiptului, cum se îmbracă păstorul cu haina sa; și va ieși de acolo nesupărat;
\par 13 El va sfărâma stâlpii templului Soarelui de la On și capiștile dumnezeilor Egiptului le va arde cu foc".

\chapter{44}

\par 1 Cuvântul ce a fost către Ieremia pentru toți Iudeii care trăiau în țara Egiptului și care se așezaseră în Migdol, în Tahpanhes, în Nof și în țara Patros:
\par 2 "Așa zice Domnul Savaot, Dumnezeul lui Israel: Ați văzut toată nenorocirea pe care am adus-o Eu asupra Ierusalimului și asupra tuturor cetăților lui Iuda; iată acelea sunt acum pustii și nimeni nu mai locuiește în ele,
\par 3 Pentru necredința lor, pe care au săvârșit-o ele, mâniindu-Mă și mergând să tămâieze și să slujească altor dumnezei, pe care nu i-au cunoscut nici ei, nici voi, nici. părinții voștri.
\par 4 Trimis-am la voi necontenit pe toți slujitorii Mei, proorocii; i-am trimis dis-de-dimineață, ca să vă spună: Nu faceți acest lucru urâcios, pe care Eu îl urăsc.
\par 5 Dar ei n-au ascultat și nu și-au plecat urechea lor ca să se întoarcă de la necredință și să nu tămâieze altor dumnezei.
\par 6 De aceea s-a revărsat urgia Mea și mânia Mea și s-a aprins în cetățile lui Iuda și pe ulițele Ierusalimului, și s-au prefăcut acelea în ruine și în pustiu, cum vedeți astăzi.
\par 7 Iar acum așa zice Domnul Dumnezeu Savaot, Dumnezeul lui Israel: Pentru ce faceți voi acest rău mare sufletelor voastre, pierzând din cuprinsul lui Iuda, bărbați, Femei, copii Și prunci ca să nu lăsați după voi rămășiță,
\par 8 Mâniindu-Mă prin lucrul mâinilor voastre, prin tămâierea altor dumnezei în pământul Egiptului, unde ați venit să trăiți, ca să vă pierdeți pe voi înșivă și să ajungeți blestem și defăimare înaintea tuturor neamurilor pământului?
\par 9 Au doar ați uitat fărădelegile părinților voștri, fărădelegile regilor lui Iuda și cele ale femeilor lor, fărădelegile voastre și cele ale femeilor voastre, săvârșite în țara lui Iuda și pe ulițele Ierusalimului?
\par 10 Ei nu s-au smerit nici astăzi și nu se tem și nu se poartă după legea Mea și după poruncile Mele, pe care vi le-am dat vouă și părinților voștri.
\par 11 De aceea, așa zice Domnul Savaot, Dumnezeul lui Israel: Iată Eu voi întoarce fața Mea împotriva voastră și spre nimicirea întregului Iuda.
\par 12 Și voi lua pe Iudeii care au rămas și care și-au întors fața lor și s-au dus în rara Egiptului, ca să trăiască acolo; toți aceia vor fi nimiciți, vor cădea în țara Egiptului; de foamete și de sabie vor fi nimiciți; vor muri de sabie și de foame de la mare până la mic și vor ajunge de blestem, de ocară, de groază și de râs.
\par 13 Voi pedepsi pe cei ce locuiesc în țara Egiptului, precum am pedepsit Ierusalimul, cu sabie, cu foamete și cu ciumă,
\par 14 Și nimeni nu va scăpa și nu va rămâne din rămășița Iudeilor care au venit în țara Egiptului, ca să trăiască acolo și apoi să se întoarcă în țara lui Iuda, unde doresc ei din tot sufletul să se întoarcă, ca să trăiască acolo. Nimeni nu se va întoarce, decât numai aceia care vor fugi de aici".
\par 15 Atunci au răspuns lui Ieremia toți bărbații care știau că femeile lor tămâiază pe alți dumnezei, și toate femeile care se aflau acolo în număr mare, și tot poporul, care locuia în pământul Egiptului, în Patros, și au zis:
\par 16 "Cuvântul pe care tu l-ai grăit în numele Domnului, nu voim să-l ascultăm de la tine;
\par 17 Dar vom face tot ce am făgăduit cu gura noastră, vom tămâia pe zeița cerului și vom săvârși pentru ea jertfe cu turnare, cum am mai făcut și noi și părinții noștri, regii noștri și mai-marii noștri în cetățile lui Iuda și pe ulițele Ierusalimului, că atunci eram sătui și fericiți și n-am văzut nenorociri.
\par 18 Iar de când am încetat de a mai tămâia pe zeița cerului și a-i săvârși jertfe cu turnare, suferim toate lipsurile și pierim de sabie și de foame".
\par 19 Și femeile au adăugat: "Când tămâiam noi zeița cerului și-i săvârșeam jertfe cu turnare, oare fără știrea bărbaților noștri îi făceam noi turte cu chipul ei și-i săvârșeam jertfe cu turnare?"
\par 20 Atunci Ieremia a zis către tot poporul, către bărbați și către femei, și către oamenii care îi răspundeau, astfel:
\par 21 "Oare nu această tămâiere, pe care o săvârșeați în cetățile lui Iuda și pe ulițele Ierusalimului și voi și părinții voștri, regii voștri și mai-marii voștri și ai poporului țării, o amintește Domnul? Oare nu ea I s-a suit la inimă?
\par 22 Domnul n-a mai putut suferi faptele voastre cele rele și urâciunile pe care le făceați, și de aceea s-a Și prefăcut țara voastră în pustietate, lucru de spaimă și de blestem și fără locuitori, cum vedeți acum.
\par 23 Deoarece voi, săvârșind acea tămâiere, ați greșit înaintea Domnului și n-ați ascultat glasul Domnului, nu v-ați purtat după legea Lui, după poruncile Lui și după rânduiala Lui, de aceea v-a ajuns nenorocirea aceasta, cum vedeți acum".
\par 24 Și a mai zis Ieremia către tot poporul și către toate femeile: "Ascultați cuvântul Domnului, voi, toți Iudeii, cei ce sunteți în țara Egiptului:
\par 25 Așa zice Domnul Savaot, Dumnezeul lui Israel: Voi și femeile voastre, ceea ce ați grăit cu gurile voastre aceea ați făcut și cu mâinile, și mai și ziceați: Vom îndeplini întocmai făgăduințele pe care le-am făcut, ca să tămâiem pe zeița cerului și să-i săvârșim jertfe cu turnare". Voi vă țineți bine de fărădelegile voastre și îndepliniți întocmai acele făgăduințe.
\par 26 De aceea ascultați cuvântul Domnului: Iată Eu M-am jurat pe numele Meu cel mare, zice Domnul, că nu va mai fi rostit numele Meu de gura vreunui iudeu în toată țara Egiptului. Nimeni nu va zice: Viu este Domnul Dumnezeu!
\par 27 Iată Eu voi veghea asupra lor spre pieire, iar nu spre bine; și toți oamenii din Iuda, care sunt în țara Egiptului vor pieri de sabie și de foame, până se vor stinge de tot.
\par 28 Numai un mic număr, care vor fugi de sabie, se va întoarce din țara Egiptului în țara lui Iuda, și vor ști toți cei rămași din Iuda, care au venit în țara Egiptului ca să trăiască acolo, al cui cuvânt se va împlini: al Meu sau al lor.
\par 29 Iată semnul ce vi-l dau, zice Domnul, că vă voi pedepsi în locul acesta, ca să știți că se va împlini cuvântul Meu cel pentru voi spre pieirea voastră.
\par 30 Asa zice Domnul: Iată Eu voi da pe Faraonul Hofra, regele Egiptului, în mâinile vrăjmașilor lui, cum am dat pe Sedechia, regele lui Iuda, în mâinile lui Nabucodonosor, regele Babilonului, vrăjmașul lui, care voia să-i ia viața".

\chapter{45}

\par 1 Cuvântul pe care l-a spus proorocul Ieremia lui Baruh, fiul lui Neria, când a scris el cuvintele acestea din gura lui Ieremia în carte, în anul al patrulea al domniei lui Ioiachim, fiul lui Iosia, regele lui Iuda:
\par 2 "Asa zice Domnul Dumnezeul lui Israel către tine, Baruh:
\par 3 Tu zici: "Vai de mine, că Domnul a adăugat durere la boala mea; am slăbit suspinând și nu-mi găsesc liniștea!"
\par 4 Spune-i, zice Domnul: Iată, ceea ce am zidit, voi dărâma, și ce am sădit, voi smulge, - adică toată țara aceasta.
\par 5 Iar tu ceri pentru tine lucru mare. Nu cere, deoarece Eu voi aduce nenorocire asupra a tot trupul, zice Domnul, iar ție, drept pradă, îți voi lăsa viața ta, în toate locurile, oriunde vei merge".

\chapter{46}

\par 1 Cuvântul Domnului care a fost către Ieremia proorocul, privitor la neamuri:
\par 2 Asupra Egiptului, împotriva oștirii Faraonului Neco, regele Egiptului, care se afla în Carchemiș, lângă fluviul Eufratului Și pe care a zdrobit-o Nabucodonosor, regele Babilonului, în anul al patrulea al lui Ioiachim, fiul lui Iosia, regele lui Iuda:
\par 3 "Gătiți scuturile și sulițele și pășiți la luptă!
\par 4 Călăreți, înșeuați caii și încălecați! Puneți-vă coifurile, ascuțiți sulițele și îmbrăcați-vă în zale!
\par 5 Dar ce văd Eu? Aceia s-au înspăimântat și s-au întors înapoi, cei puternici ai lor au fost zdrobiți și fug fără a se uita înapoi. Pretutindeni e groază, zice Domnul.
\par 6 Cel iute de picior nu va scăpa cu fuga, nici se va mântui cel puternic; la miazănoapte, pe râul Eufratului, se vor poticni și vor cădea.
\par 7 Cine este acela, care se înalță ca Nilul și își învăluie apele ca un fluviu?
\par 8 Egiptul se înalță ca Nilul și ca un fluviu își învăluie apele sale și zice: Ridica-mă-voi și voi acoperi pământul, pierde-voi cetățile și locuitorii lor.
\par 9 Încălecați pe cai și vă aruncați în căruțe și porniți, puternici Etiopieni și Libieni înarmați cu scut, și voi Lidieni, ținând arcurile și încordându-le,
\par 10 Căci ziua aceasta e zi de răzbunare la Domnul Dumnezeu, ca să Se răzbune pe vrăjmașii Săi, și sabia va mânca, se va sătura și se va îmbăta de sângele lor; și aceasta va fi jertfă Domnului Dumnezeului Savaot în țara cea de la miazănoapte, la râul Eufratului.
\par 11 Fecioară, fiica Egiptului, mergi în Galaad și ia-ți balsam; în zadar vei spori leacurile tale, căci nu mai este vindecare pentru tine.
\par 12 Auzit-au popoarele de rușinea ta și bocetul tău a umplut pământul, căci cel puternic s-a lovit cu cel puternic și au căzut amândoi împreună".
\par 13 Cuvântul pe care l-a grăit Domnul proorocului Ieremia despre năvălirea lui Nabucodonosor, regele Babilonului, ca să lovească pământul Egiptului:
\par 14 "Vestiți în Egipt și dați știre în Migdol; spuneți în Nof și în Tahpanhes, și ziceți: Stai și te gătește, căci sabia nimicește în jurul tău!
\par 15 Ce? Apis a fugit? Cel puternic al tău a căzut? Da, fiindcă Domnul l-a lovit.
\par 16 El înmulțește pe cei care se clatină și cad unii peste alții, fiecare dintre ei zicând: Să ne sculăm și să ne întoarcem la poporul nostru, în pământul patriei noastre, departe de sabia ucigătoare.
\par 17 Iar lui Faraon, regele Egiptului, dați-i nume: "Zarvă care scapă clipa hotărâtoare!"
\par 18 Viu sunt Eu, zice Împăratul, al Cărui nume este Domnul Savaot, pe cât e de adevărat că Taborul este în rândul munților și Carmelul lângă mare, tot atât e de adevărat că el va veni.
\par 19 Fecioară, locuitoarea Egiptului, gătește-ți cele trebuitoare pentru bejenie, căci Noful va fi pustiit, dărăpănat și fără locuitori.
\par 20 Egiptul este o junincă preafrumoasă peste care s-a năpustit un tăune de la miazănoapte.
\par 21 Și simbriașii lui sunt în mijlocul lui ca niște tauri îngrășați; aceștia s-au întors înapoi, au fugit toți și n-au ținut piept, pentru că a venit asupra lor ziua pieirii lor, timpul pedepsirii lor.
\par 22 Glasul lor se aude ca șuierul șarpelui, vin cu oștire și tăbărăsc asupra lui cu topoarele, ca tăietorii de lemne;
\par 23 Taie pădurea lui, zice Domnul, căci sunt nenumărați, sunt mai mulți decât lăcustele și nu mai au număr.
\par 24 Fiica Egiptului batjocorită e dată pe mâna poporului de la miazănoapte.
\par 25 Domnul Savaot, Dumnezeul lui Israel, zice: "Iată, Eu voi pedepsi pe Amon, care se află în No, pe Faraon și Egiptul, pe dumnezeii și pe regii lui, pe Faraon și pe cei ce-și pun nădejdea în el.
\par 26 Și-i voi da în mâinile celor ce caută viața lor, în mâinile lui Nabucodonosor, regele Babilonului, și în mâinile slujitorilor lui. Dar după aceea Egiptul va fi iarăși locuit, ca în zilele de altădată, zice Domnul.
\par 27 Iar tu, Iacove, robul Meu, nu te teme și nu te înspăimânta Israele! Căci iată, Eu te voi elibera din țară depărtată și neamul tău îl voi aduce din țara robiei lui; Iacov se va întoarce și va trăi liniștit și pașnic și nimeni nu-l va înspăimânta.
\par 28 Iacove, robul Meu, nu te teme, zice Domnul, căci Eu sunt cu tine și voi pierde toate neamurile printre care te-am împrăștiat, iar pe tine nu te voi pierde, ci numai te voi pedepsi cu măsură, căci nepedepsit nu te voi lăsa".

\chapter{47}

\par 1 Cuvântul Domnului care a fost către proorocul Ieremia pentru Filisteni, înainte de a lovi Faraon Gaza.
\par 2 Așa zice Domnul: "Iată se ridică ape de la miazănoapte și cresc ca un fluviu ieșit din matcă, îneacă țara și tot ce cuprinde ea, cetățile și locuitorii lor. Atunci oamenii vor striga și vor plânge toți locuitorii țării,
\par 3 La ropotul zgomotos al copitelor puternicilor lui cai, la vuietul carelor lui și la huruitul roților lui. Părinții nu se vor mai uita la copiii lor, pentru că mâinile lor vor fi încremenite
\par 4 De groaza zilei aceleia, care vine să piardă pe toți Filistenii și să răpească Tirului și Sidonului orice nădejde de ajutor, căci va pustii Domnul pe Filisteni, rămășița insulei Caftor.
\par 5 Gaza a pleșuvit și va pieri Ascalonul, cu rămășița văilor lui". Până când îți vei face tăieturi de jale în piele?
\par 6 Până când vei tăia, sabia Domnului? Până când nu te vei liniști? Întoarce-te în teaca ta, oprește-te și te liniștește!
\par 7 Dar cum să te liniștești, când ți-a dat Domnul poruncă împotriva Ascalonului și împotriva țărmului mării? Într-acolo te-a îndreptat El".

\chapter{48}

\par 1 Asupra Moabului, așa zice Domnul Savaot, Dumnezeul lui Israel: "Vai de Nebo, că e pustiit! Chiriataimul este acoperit de rușine și luat. Cetatea este acoperită de rușine și înrobită!
\par 2 Nu mai este slava Moabului! În Heșbon se urzesc rele împotriva lui, zicând: "Să mergem și să-l ștergem dintre neamuri!" Și tu, Madmen, vei pieri! Sabia vine în urma ta!
\par 3 Strigăte se aud din Horonaim, căci e pustiire și dărâmare grozavă.
\par 4 Moab e zdrobit și copiii lui au ridicat bocet.
\par 5 Pe suișul de la Luhit se ridică plânsete peste plânsete și pe povârnișul din Horonaim se aud strigăte de nenorocire:
\par 6 Fugiți! Scăpați-vă viața și fiți asemenea asinului sălbatic din pustiu.
\par 7 Și fiindcă te-ai încrezut în lucrările și în vistieriile tale vei fi luat și tu. Chemoșul va merge în robie cu preoții și cu mai-marii săi, toți împreună.
\par 8 Va veni pustiitorul asupra fiecărei cetăți și cetatea nu Va rămâne nedărâmată; va pieri valea și șesul și se va pustii, precum a zis Domnul.
\par 9 Dați un mormânt lui Moab, căci el va fi cu totul pustiit. Cetățile lui vor fi prefăcute în pustiu și nu vor mai avea locuitori.
\par 10 Blestemat să fie tot cel ce face lucrurile Domnului cu nebăgare de seamă și blestemat fie tot cel ce oprește sabia lui de la sânge!
\par 11 Moab din tinerețea lui a fost liniștit, se odihnea pe drojdia sa și n-a fost trecut din vas în vas, nici în robie n-a fost. De aceea a rămas în el gustul său și mirosul său nu s-a schimbat.
\par 12 De aceea iată vin zile, zice Domnul, când voi trimite la el pritocitori, care îl vor pritoci și vor strica vasele și urcioarele lui le vor sparge.
\par 13 Și va fi rușinat Moab pentru Chemoș, după cum casa lui Israel a fost rușinată pentru Betel, nădejdea ei.
\par 14 Cum puteți voi să ziceți: "Noi suntem oameni viteji și tari pentru război?"
\par 15 Moab este pustiit, cetățile lui ard și tinerii lui aleși s-au dus la junghiere, zice Împăratul al Cărui nume este Domnul Savaot.
\par 16 Aproape este pieirea Moabului și nenorocirea lui vine în grabă mare.
\par 17 jeliți-l toți vecinii lui și toți cei ce cunoașteți numele lui ziceți: Cum s-a sfărâmat toiagul puterii, sceptrul slavei!
\par 18 Fecioară, locuitoarea Dibonului, coboară-te din înălțime și șezi în pământ ars de soare, căci pustiitorul Moabului vine la tine și va dărâma întăriturile tale.
\par 19 Stai la drum și privește, locuitoarea Aroerului, și întreabă pe cel care fuge și pe cel scăpat: Ce s-a întâmplat?
\par 20 Rușinat este Moabul, căci este biruit. Plângeți și strigați, dați de veste în Amon că Moabul este pustiit.
\par 21 Judecata a venit împotriva țării din șes, împotriva Holonului și a Iahței, împotriva Mefaatului și a Dibonului,
\par 22 Împotriva lui Nebo și a Bet-Diblataimului.
\par 23 Împotriva Chiriataimului și a Bet-Gamului, împotriva Bet-Meonului și a Cheriotului,
\par 24 Împotriva Bosrei și împotriva tuturor cetăților țării Moabului, de aproape și de departe.
\par 25 S-a tăiat cornul Moabului și brațul lui este zdrobit, zice Domnul.
\par 26 Îmbătați-l, căci s-a ridicat împotriva Domnului. Tăvălească-se Moabul în vărsătura sa și de râs să fie.
\par 27 N-a fost Israel de râsul tău? Oare a fost el prins printre tâlhari, de clătinai din cap ori de câte ori vorbeai cu el?
\par 28 Părăsiți cetățile și trăiți pe stânci, locuitori ai Moabului, și veți fi ca porumbeii care-și fac cuiburile pe la intrarea peșterilor.
\par 29 Auzit-am de mândria Moabului, de mândria lui cea nemăsurată, de trufia lui și de îngâmfarea lui, de înfumurarea lui și de semeția inimii lui.
\par 30 Eu cunosc îndrăzneala lui, zice Domnul, lăudăroșeniile lui, vorbele goale și faptele deșarte.
\par 31 De aceea voi plânge pe Moab și voi striga pentru Moabul întreg; voi suspina după oamenii din Chir-Heres.
\par 32 Și te voi plânge pe tine, vie din Sibma, cu mai mult plânset decât Iazerul; ramurile tale s-au întins peste mare, ajuns-au până la Iazer, pustiitorul a năvălit asupra roadelor tale celor văratice și asupra strugurilor copți.
\par 33 Bucuria și veselia au fost luate din Carmel și din rara Moabului; voi seca vinul din teascuri și nimeni nu va mai călca teascul cu strigăte de veselie; va fi strigăt de război și nu strigăt de bucurie.
\par 34 De la Heșbon până la Eleale și Iahaț, de la Țoar până la Horonaim și până la Eglat-Șelișia vor ridica glas de tânguire, căci apele Nimrimului vor seca.
\par 35 Voi stârpi din Moab, zice Domnul, pe cei ce aduc jertfe pe locuri înalte și pe cei ce tămâiază pe dumnezeii lui.
\par 36 De aceea inima Mea geme pentru Moab, ca un fluier; geme inima Mea ca un fluier pentru oamenii din Chir-Heres, căci au pierit bogățiile adunate de ei;
\par 37 Fiecare își are capul ras, fiecare își are barba tunsă; toți au tăieturi pe mâini și peste coapse sac.
\par 38 Pe toate acoperișurile Moabului și în piețele lui nu este decât o jale, căci am sfărâmat Moabul, ca pe un vas de aruncat, zice Domnul.
\par 39 Cum a fost zdrobit!, vor zice plângând. Cum s-a acoperit Moabul de rușine, întorcând spatele! Și va fi Moabul de râs și de groază pentru toți cei ce-l înconjoară;
\par 40 Căci așa zice Domnul: Iată, vrăjmașul ca un vultur va zbura și-și va întinde aripile sale deasupra Moabului.
\par 41 Orașele vor fi luate și cetățile cucerite; inima vitejilor Moabiți va fi în ziua aceea ca inima unei femei chinuită de durerile nașterii.
\par 42 Moabul va fi șters din numărul popoarelor, pentru că s-a ridicat împotriva Domnului.
\par 43 Groază, groapă și laț sunt pentru tine, locuitorule al Moabului, a zis Domnul.
\par 44 Cel ce va scăpa de groază va cădea în groapă; și cel ce va scăpa de groapă va cădea în laț, căci Eu voi aduce asupra lui, asupra Moabului, anul pedepsei lui, zice Domnul.
\par 45 Fugarii obosiți s-au oprit la umbra Heșbonului, dar a ieșit foc din Heșbon, și din inima Sihonului flăcări, și va mistui tâmplele lui Moab și creștetul capului fiilor răzvrătirii.
\par 46 Vai ție, Moab! Pierit-a poporul din Chemoș, că fiii tăi sunt luați în robie și fiicele tale sunt robite.
\par 47 Dar în zilele cele de apoi voi întoarce pe Moab din robie", zice Domnul. Până aici e judecata lui Moab.

\chapter{49}

\par 1 Asupra fiilor lui Amon așa grăiește Domnul: "Au doară Israel n-are fii? Au doară el n-are moștenitor? Pentru ce, dar, Milcom a pus stăpânire pe Gad și poporul lui trăiește în cetățile acestuia?
\par 2 De aceea, iată vin zile, zice Domnul, când în Raba, cetatea fiilor lui Amon, se va auzi strigăt de război, și aceasta va fi prefăcută în movilă de dărâmături; cetățile ei vor fi arse cu foc și Israel va stăpâni pe cei ce-l stăpâneau, zice Domnul.
\par 3 Plângi Heșbon, că s-a pustiit Ai! Strigați, voi fiice din Raba, încingeți-vă cu sac, plângeți și rătăciți prin livezi, căci Milcom va merge în robie cu preoții și cu mai-marii săi, toți împreună.
\par 4 "Fiică nepăsătoare, tu te lauzi cu Valea ta, tu te încrezi în comorile tale, zicând: "Cine va îndrăzni să vină împotriva mea?"
\par 5 Iată Eu voi aduce asupra ta groază din toate părțile, zice Domnul Dumnezeul Savaot și veți fi fugăriți care încotro și nimeni nu va aduna pe fugari.
\par 6 Dar după aceea voi întoarce din robie pe fiii lui Amon", zice Domnul.
\par 7 Iar asupra Edomului așa grăiește Domnul Savaot: "Au doară nu mai este înțelepciune în Teman? Au doară lipsește sfatul la cei înțelepți?
\par 8 Au doară a secat înțelepciunea lor? Fugiți, întorcând spatele! Ascundeți-vă în peșteri, locuitori din Dedan, căci voi aduce nenorocire peste Isav, la vremea pedepsirii lui.
\par 9 Dacă ar fi venit la tine culegătorii de struguri, de bună seamă ar fi lăsat puține bobițe neculese. De ar fi venit hoții noaptea, ar fi răpit cât le-ar fi fost de trebuință.
\par 10 Eu însă voi jefui pe Isav până la piele, voi descoperi ascunzătorile lui, și el nu se va putea ascunde. Stârpit va fi neamul lui, frații lui și vecinii lui, și el nu va mai fi.
\par 11 Lăsați orfanii, că Eu voi ține viața lor, și văduvele tale să creadă în Mine;
\par 12 Căci așa grăiește Domnul: Iată, celor ce nu le-a fost dat să bea paharul, îl vor bea negreșit. Au doară numai tu vei rămâne nepedepsit? Nu, nu vei rămâne nepedepsit, ci vei bea negreșit paharul.
\par 13 Căci Mă jur pe Mine Însumi, zice Domnul, că Boțra va fi o groază și o rușine, un deșert și un blestem, și cetățile ei vor ajunge ruine veșnice.
\par 14 Auzit-am veste de la Domnul și sol s-a trimis la popoare să spună: "Adunați-vă și mergeți împotriva ei și ridicați-vă pentru luptă".
\par 15 Iată Eu te voi face mic între popoare.
\par 16 Starea ta groaznică și semeția inimii tale te-au adeverit pe tine, cel ce locuiești în crăpăturile stâncilor și care te-ai așezat pe vârfurile dealurilor! Deși ți-ai împletit cuibul tău sus, ca un vultur, și de acolo te voi arunca, zice Domnul.
\par 17 Edom va fi o groază; toți cei ce vor trece pe lângă el vor fi uimiți, vor fluiera, văzând toate rănile lui.
\par 18 Cum au fost aruncate Sodoma, Gomora și cetățile vecine cu ele, zice Domnul, tot așa și acolo nu va trăi nici om, nici fiu de om nu se va opri în el.
\par 19 Iată-l, se înalță ca un leu de la obârșiile Iordanului spre sălașurile întărite; dar Eu îi voi sili să plece cu grăbire din Edom și cine va fi ales, pe acela îl voi pune peste acesta. Căci cine este asemenea Mie și ce păstor se va împotrivi Mie?
\par 20 Ascultați dar hotărârea Domnului pe care a luat-o El asupra lui Edom și planurile pe care El le-a gândit împotriva locuitorilor din Teman: "Da, chiar oile cele mai plăpânde vor fi târâte; la vederea lor pășunea lor se va înfiora de spaimă.
\par 21 La vuietul căderii lor se va cutremura pământul și răsunetul strigătului lor se va auzi până la Marea Roșie.
\par 22 Iată-l se înalță, ca un vultur, și zboară și își întinde aripile sale deasupra Boțrei, și inima războinicilor din Edom va fi în ziua aceea ca inima unei femei ce naște".
\par 23 Asupra Damascului: "Hamatul și Arpadul sunt tulburate, căci au primit o veste rea. Inima lor se topește de frică; este o mare în furtună care nu se poate potoli.
\par 24 Temutu-s-a Damascul și a luat-o la fugă; de frică l-au cuprins dureri și chinuri, ca pe femeia ce naște.
\par 25 Cum n-a scăpat cetatea slavei, cetatea bucuriei mele!
\par 26 Deci vor cădea tinerii lui pe ulițele lui și toți vor pieri în ziua aceea, zice Domnul Savaot.
\par 27 Voi aprinde foc în zidurile Damascului și el va mistui palatele lui Benhadad".
\par 28 Asupra Chedarului și asupra regatelor Hațorului, pe care le-a lovit Nabucodonosor, regele Babilonului, așa grăiește Domnul: "Sculați-vă și ieșiți înaintea lui Chedar și pustiiți pe fiii Răsăritului!
\par 29 Corturile lor și turmele lor să fie luate, țesăturile lor și toate lucrurile lor și cămilele lor să fie luate. Și să li se strige: Groază din toate părțile!
\par 30 Fugiți, duceți-vă repede, ascundeți-vă în prăpăstii, locuitori ai Hațorului, zice Domnul, căci Nabucodonosor, regele Babilonului, a luat hotărâre asupra voastră, și a urzit un plan împotriva voastră.
\par 31 Sculați-vă și pășiți împotriva poporului celui pașnic, care trăiește fără grijă, zice Domnul; acela n-are nici uși, nici zăvoare și locuiește singur.
\par 32 Cămilele lor vor fi date prăzii și mulțimea turmelor lor răpirii și-i voi împrăștia în toate vânturile pe aceștia care-și tund părul de pe tâmple și din toate părțile lui voi aduce asupra lor pieirea, zice Domnul.
\par 33 Hațorul va fi sălașul șacalilor și pustietate veșnică; nu va trăi acolo om, nici fiu de om nu se va opri acolo".
\par 34 Cuvântul Domnului care a fost către Ieremia proorocul împotriva Elamului, la începutul domniei lui Sedechia, regele lui Iuda:
\par 35 "Așa zice Domnul Savaot: Iată voi sfărâma arcul Elamului, puterea lui de căpetenie,
\par 36 Și voi aduce asupra Elamului patru vânturi din cele patru margini ale cerului și-l voi împrăștia în toate vânturile acestea și nu va fi neam la care să nu ajungă Elamiți izgoniți;
\par 37 Voi lovi pe Elamiți cu frică înaintea vrăjmașilor lor și înaintea celor care vor să le ia viața; asupra lor voi aduce nenorociri, mânia Mea aprinsă, zice Domnul, și în urma lor voi trimite sabie, până îi voi stârpi.
\par 38 Voi pune tronul Meu în Elam și voi stârpi de acolo pe rege și pe dregători, zice Domnul.
\par 39 Dar în zilele cele de apoi, voi întoarce pe Elam din robie", zice Domnul.

\chapter{50}

\par 1 Cuvântul pe care l-a rostit Domnul despre Babilon și despre țara Caldeilor prin Ieremia proorocul:
\par 2 "Vestiți și faceți cunoscut între popoare, ridicați steag, spuneți și nu tăinuiți, ci ziceți: Babilonul e luat, Bel e rușinat, Merodah e zdrobit, chipurile lui cele cioplite sunt batjocorite și idolii lui sfărâmați.
\par 3 Căci de la miazănoapte s-a ridicat asupra lui poporul care va preface pământul în pustiu și nimeni nu va locui acolo, nici om, nici dobitoc; toți se vor ridica și se vor duce.
\par 4 În zilele acelea și în vremea aceea, zice Domnul, vor veni fiii lui Israel, și împreună cu ei vor merge și fiii lui Iuda și, plângând, voi căuta pe Domnul Dumnezeul lor;
\par 5 Vor întreba de calea Sionului și, întorcându-și fețele spre el, vor zice: Veniți și vă uniți cu Domnul prin legământ veșnic, care nu se va uita.
\par 6 Poporul Meu a fost ca oile cele pierdute; păstorii lor le-au abătut din cale și le-au împrăștiat prin munți; rătăcit-au ele din deal în munte și și-au uitat staulul lor.
\par 7 Toți cei ce le găseau le mâncau și dușmanii lor ziceau: Nu suntem noi de vină, pentru că ele au păcătuit împotriva Domnului, Locașul neprihănirii, împotriva Domnului, Nădejdea părinților lor.
\par 8 Fugiți din Babilon și plecați din țara Caldeilor și veți fi ca berbecii înaintea turmelor de oi.
\par 9 Căci iată, voi ridica și voi aduce împotriva Babilonului o mulțime de neamuri mari din țara de la miazănoapte; acelea se vor înșira împotriva lui și el va fi luat. Săgețile lor sunt ca ale arcașului iscusit, nu cad în zadar.
\par 10 Caldeea va fi pradă lor și pustiitorii ei se vor sătura, zice Domnul.
\par 11 "Căci voi v-ați bucurat și v-ați veselit când ați răpit moștenirea Mea; zburdați ca vițeii pe pajiște, nechezați ca armăsarii!
\par 12 Mama voastră va fi în rușine mare și aceea care v-a născut va roși. Iată, că ea va fi cea mai de, pe urmă dintre neamuri: un pustiu, un pământ uscat și fără apă.
\par 13 De mânia Domnului țara lor va ajunge nelocuită și toată va fi pustiu. Tot cel ce va trece prin Babilon se va mira și va fluiera, văzând toate rănile lui.
\par 14 Așezați-vă în rânduială de bătaie împrejurul Babilonului. Cei ce încordați arcul, trageți în el, nu cruțați săgețile, căci el a păcătuit împotriva Domnului.
\par 15 Ridicați strigăt de război împotriva lui din toate părțile; el întinde mâna; căzut-au întăriturile lui și zidurile lui s-au prăbușit, căci aceasta este răsplata Domnului! Răzbunați-vă pe el!
\par 16 Cum s-a purtat el, așa purtați-vă și voi! Stârpiți din Babilon pe cel ce seamănă și pe cel ce lucrează cu secera în vremea secerișului! De frica sabiei ucigătoare, să se întoarcă fiecare la poporul său și fiecare să fugă în țara sa.
\par 17 Pe Israel, turma cea risipită leii l-au prigonit; mai înainte l-a mâncat regele Asiriei, iar acum în urmă Nabucodonosor, regele Babilonului, i-a zdrobit oasele.
\par 18 De aceea, așa zice Domnul Savaot, Dumnezeul lui Israel: Iată, Eu voi pedepsi pe regele Babilonului și țara lui, cum am pedepsit și pe regele Asiriei,
\par 19 și voi întoarce pe Israel la pășunea lui și va paște el pe Carmel și în Vasan; sufletul lui se va sătura în muntele lui Efraim și în Galaad.
\par 20 În zilele acelea și în vremea aceea se va căuta nedreptatea lui Israel, zice Domnul, și nu se va afla, se vor căuta și păcatele lui Iuda și nu se vor găsi, căci voi ierta pe aceia pe care îi voi lăsa în viață.
\par 21 Ridică-te împotriva țării Merataim, împotriva țării Răzvrătirii și împotriva locuitorilor din Pecod, țara pedepsirii! Pustiește și nimicește, zice Domnul, și fă tot ce ți-am poruncit!
\par 22 Strigăte de război se aud în toată țara și prăpădul este mare!
\par 23 Cum s-a sfărâmat și s-a zdrobit ciocanul lumii întregi! Cum a ajuns Babilonul de plâns pe pământ!
\par 24 Întins-am curse pentru tine și te-ai prins, Babilonule, fără să te aștepți. Găsit ai fost și prins, pentru că te-ai ridicat împotriva Domnului.
\par 25 Domnul Și-a deschis vistieria Sa și a luat din ea vasele mâniei Sale, pentru că Domnul Dumnezeul Savaot are de lucru în țara Caldeilor.
\par 26 Alergați împotriva ei din toate părțile, deschideți jitnițele ei, călcați-o ca pe niște snopi, nimiciți-o de tot, ca să nu mai rămână nimic din ea!
\par 27 Ucideți toți boii ei! Să meargă la junghiere! Vai lor, căci a venit ziua lor și vremea pedepsirii lor!
\par 28 Se aude glasul celor ce fug și al celor care scapă din țara Babilonului, ca să vestească în Sion răzbunarea Domnului Dumnezeului nostru, răzbunarea cea pentru templul Său!
\par 29 Chemați împotriva Babilonului săgetători! Toți cei ce încordați arcul, așezați-vă tabăra împrejurul lui, ca nimeni să nu scape din el! Răsplătiți-i după faptele lui! Cum s-a purtat el, așa să vă purtați și voi cu el, căci el s-a ridicat împotriva Domnului, împotriva Sfântului lui Israel.
\par 30 De aceea vor cădea tinerii lui pe ulițele lui și toți oștenii lui vor fi pierduți în ziua aceea, zice Domnul.
\par 31 Iată, Eu sunt împotriva ta, "trufașule", zice Domnul Savaot, căci a venit ziua ta și timpul pedepsirii tale.
\par 32 Și se va împiedica "trufașule" și va cădea și nimeni nu-l va ridica. Voi aprinde foc în cetățile lui și acesta va mistui toate împrejurul lui.
\par 33 Așa zice Domnul Savaot: Apăsați sunt fiii lui Israel, ca și fiii lui Iuda, și toți cei ce i-au robit îi țin tare și nu vor să le dea drumul.
\par 34 Dar Răscumpărătorul lor este puternic și numele Lui este Domnul Savaot. Acesta va apăra pricina lor, ca să liniștească țara și să facă pe locuitorii Babilonului să tremure.
\par 35 Sabie împotriva Caldeilor, zice Domnul, și împotriva locuitorilor Babilonului și a căpeteniilor lui și a înțelepților lui.
\par 36 Sabie împotriva proorocilor minciunii, ca ei să-și piardă mintea! Sabie împotriva oștenilor lui: să se teamă!
\par 37 Sabie împotriva cailor și a carelor lui și împotriva a toată mulțimea de oameni din el: să fie ca niște femei! Sabie împotriva comorilor lui: să fie jefuite!
\par 38 Uscăciune peste apele lui: să sece! Fiindcă aceasta este o țară de idoli și au înnebunit cu idolii lor.
\par 39 Acolo se vor așeza fiarele pustiului cu șacalii și vor trăi în ea struții; în veci nu va mai fi locuită și din neam în neam nu vor locui acolo oameni.
\par 40 Cum au fost aruncate de Domnul Sodoma și Gomora și cetățile vecine cu ele, zice Domnul, așa și aici nici un om nu va trăi, nici fiu de om nu va poposi acolo.
\par 41 Iată vine de la miazănoapte un popor mare și regi mulți se ridică de la marginile pământului.
\par 42 Și țin în mână arc și suliță; aceștia sunt cruzi și nemilostivi și glasul lor e zgomotos ca marea, și vin pe cai, ca unii care sunt gata să lupte cu tine, fiica Babilonului!
\par 43 Auzit-a regele Babilonului veste despre ei și au început să-i tremure mâinile; l-a cuprins întristarea și dureri ca ale femeii ce naște.
\par 44 Iată ca un leu se ridică din tufișurile Iordanului spre pășunea totdeauna verde! Dar Eu îl voi face să plece cu grăbire de acolo și voi așeza acolo pe cel pe care l-am ales. Căci cine este asemenea Mie? Și cine-Mi va cere socoteală? și ce păstor se va împotrivi Mie?
\par 45 Ascultați dar hotărârea Domnului pe care a luat-o El împotriva Babilonului și planurile pe care le-a făcut El împotriva țării Caldeilor! Da, vor fi târâți ca oile cele mărunte; la vederea lor pășunea lor se va înfiora de spaimă.
\par 46 La vuietul luării Babilonului tremură pământul și un strigăt se întinde printre neamuri".

\chapter{51}

\par 1 Așa zice Domnul: "Iată, voi ridica împotriva Babilonului și a locuitorilor țării Caldeii un duh nimicitor.
\par 2 Și voi trimite la Babilon vânturători, care-l vor vântura și vor pustii țara lui, căci în ziua nenorocirii vor năvăli asupra lui din toate părțile.
\par 3 Arcașul să-și încordeze arcul împotriva arcașului și împotriva celui ce se laudă cu platoșele sale, și să nu cruțe pe tinerii lui! Nimiciți toată oștirea lui!
\par 4 Să cadă răniți de moarte în țara Caldeilor și străpunși pe străzile Babilonului.
\par 5 Căci Iuda și Israel n-au rămas văduvi de Dumnezeul lor, de Domnul Savaot, și țara Caldeilor e plină de păcate înaintea Sfântului lui Israel.
\par 6 Fugiți din Babilon, și fiecare să-și scape viața, ca să nu pieriți pentru fărădelegile lui, căci acesta este timpul răzbunării pentru Domnul, căci El îi va da răsplată.
\par 7 Babilonul a fost în mâna Domnului cupă de aur, care a îmbătat tot pământul; băut-au popoarele din vinul ei și au înnebunit.
\par 8 Căzut-a fără de veste Babilonul, și s-a zdrobit. Plângeți-l și aduceți balsam pentru rănile lui, că poate se va vindeca!
\par 9 Am voit să vindecăm Babilonul, dar nu s-a vindecat! Lăsați-l și haideți să mergem fiecare în țara noastră, pentru că osânda lui s-a ridicat până la nori și a ajuns până la cer!
\par 10 Domnul a scos la lumină dreptatea noastră! Veniți să vestim în Sion fapta Domnului Dumnezeului nostru!
\par 11 Ascuțiți săgețile și vă umpleți tolbele! Că Domnul a trezit duhul regilor Mediei, pentru că vrea să nimicească Babilonul. Aceasta este răzbunarea Domnului, răzbunarea pentru templul Său.
\par 12 Ridicați steagul împotriva zidurilor Babilonului, întăriți paza, puneți străjeri, întindeți curse, căci Domnul a făcut un plan și a împlinit ceea ce a rostit împotriva locuitorilor Babilonului.
\par 13 O, tu, cel ce locuiești lângă apele cele mari și ești plin de comori, venit-a sfârșitul tău și măsura lăcomiei tale ți s-a umplut!
\par 14 Domnul Savaot S-a jurat pe Sine Însuși și a zis: "Adevărul grăiesc, că te voi umple de oameni, ca de lăcuste, și ei vor ridica strigăt de biruință".
\par 15 El a făcut pământul cu puterea Sa, a întemeiat lumea cu înțelepciunea Sa și cu mintea Sa a întins cerurile.
\par 16 La glasul Lui freamătă apele în ceruri și El ridică nori de la marginile pământului; făurește fulgerele în mijlocul ploii și scoate vânturile din vistieriile Sale.
\par 17 Tot omul rătăcește în știința sa și orice argintar se rușinează de idolul său, căci chipurile turnate de el nu sunt decât minciună; n-au nici o suflare în ele.
\par 18 Aceasta este deșertăciune adevărată, faptă rătăcită și în vremea pedepsirii lor vor pieri.
\par 19 Dar soarta lui Iacov nu e ca a lor, pentru că Dumnezeul lui este făcătorul a toate, și Israel este toiagul moștenirii Lui, și numele Lui este Domnul Savaot.
\par 20 "Ciocan Îmi ești tu și armă de război. Cu tine am lovit popoare și cu tine am stricat regate.
\par 21 Cu tine am lovit cal și călăreț și cu tine am sfărâmat carul și pe conducătorul lui.
\par 22 Cu tine am lovit bărbat și femeie, cu tine am bătut tânăr și bătrân și cu tine am lovit fecior și fecioară.
\par 23 Cu tine am lovit păstor și turmă, cu tine am bătut pe plugar și boii lui și cu tine am lovit pe mai-marii ținuturilor și pe căpeteniile cetăților.
\par 24 Și voi răsplăti Babilonului și tuturor locuitorilor Caldeii pentru toate relele ce au făcut ei Sionului sub ochii voștri", zice Domnul.
\par 25 "Iată, Eu sunt împotriva ta, munte al nimicirii, care ai pustiit tot pământul; voi întinde împotriva ta mâna Mea, te voi arunca jos de pe stânci și te voi face munte dogorit de soare, zice Domnul!
\par 26 Nimeni nu va lua din tine pietre de pus în capul unghiului, nici pietre de temelie, ci veșnic vei fi pustiu", zice Domnul.
\par 27 "Ridicați steag pe pământ, trâmbițați cu trâmbița printre neamuri, pregătiți neamurile împotriva lui, chemați împotriva lui regatele: Ararat, Mini și Așchenaz, puneți căpetenie împotriva lui și aduceți cai ca lăcusta cea grozavă.
\par 28 Înarmați împotriva lui neamurile și pe regii Mediei, căpeteniile provinciilor ei și pe toate căpeteniile cetăților ei și toată țara de sub stăpânirea lor.
\par 29 Pământul se cutremură și se zbuciumă, căci se împlinește împotriva Babilonului planul Domnului de a face pământul Babilonului pustietate fără locuitori.
\par 30 Cei puternici ai Babilonului au încetat lupta, șed în întăriturile lor; secătuitu-s-a puterea lor și au ajuns ca femeile; locuințele lor sunt arse și zăvoarele sfărâmate.
\par 31 Sol după sol, vestitor după vestitor aleargă să dea de știre regelui Babilonului că cetatea sa e cuprinsă din toate părțile,
\par 32 Că vadurile sunt luate, bălțile cu stuf arse și ostașii loviți de spaimă".
\par 33 Că așa zice Domnul Savaot, Dumnezeul lui Israel: "Fiica Babilonului e asemenea unei arii în timpul treieratului: încă puțin și vine vremea secerișului".
\par 34 "Mâncatu-m-a și m-a ros Nabucodonosor, regele Babilonului; făcutu-m-a vas gol; înghițitu-m-a ca un dragon; umplutu-și-a pântecele cu bunătățile mele și m-a aruncat.
\par 35 Ocara mea și carnea mea cea sfâșiată să fie asupra Babilonului", să zică ceea ce locuiește în Sion - "și sângele meu să fie asupra locuitorilor Caldeii", să zică Ierusalimul.
\par 36 De aceea, așa zice Domnul: "Iată Eu iau apărarea pricinii tale și te voi răzbuna și voi seca marea lui Și canalele lui le voi usca.
\par 37 Babilonul va fi o movilă de dărâmături, adăpost pentru șacali, groază și batjocură fără locuitori.
\par 38 Ei toți vor mugi ca niște lei și vor mârâi ca niște pui de leu.
\par 39 În vremea aprinderii lor le voi face ospăț și-i voi adăpa, ca să se veselească și să doarmă somnul de veci și să nu se mai trezească, zice Domnul.
\par 40 Îi voi coborî ca pe niște miei la junghiere, ca pe niște berbeci și țapi.
\par 41 Cum a fost luat Șisac! Cum a fost cucerit acela a cărui slavă umplea tot pământul! Cum a ajuns Babilonul spaimă printre neamuri!
\par 42 S-a ridicat marea împotriva Babilonului și acesta e acoperit de mulțimea valurilor.
\par 43 Cetățile lui au ajuns pustii, pământul lui sec, ținut unde nu locuiește nici un om și pe unde nu trece fiu de om.
\par 44 Voi pedepsi pe Bel în Babilon și voi smulge din gura lui cele înghițite de el; popoarele nu se vor mai îngrămădi spre el de acum înainte. Dar zidurile Babilonului au și căzut.
\par 45 Ieși din mijlocul lor, poporul Meu, și fiecare să-și scape viața de flacăra mâniei Domnului.
\par 46 Să nu slăbească inima voastră și să nu vă temeți de zvonul care se va auzi în țară; căci anul acesta va veni un zvon, iar în anul următor altul și pe pământ va fi silnicie; tiran peste tiran se va scula.
\par 47 De aceea iată vin zile când voi pedepsi pe idolii Babilonului și toată țara lui va fi rușinată, toți cei loviți ai lui vor cădea în mijlocul lui.
\par 48 Atunci cerul și pământul și tot ce cuprind ele vor striga de bucurie împotriva Babilonului, căci vin asupra lui pustiitorii de la miazănoapte, zice Domnul.
\par 49 Cum Babilonul a dobândit și a zdrobit pe Israeliți, așa vor fi doborâți și zdrobiți în Babilon locuitorii pământului lui.
\par 50 Cei scăpați de sabie, plecați, nu vă opriți, aduceți-vă aminte din depărtare de Domnul și să-și găsească Ierusalimul loc în inima voastră.
\par 51 Ne rușinam când auzeam ocara, și necinstea acoperea obrazul nostru, când străinii au venit în locul cel sfânt al templului Domnului;
\par 52 Dar în schimb iată vin zile, zice Domnul, când voi pedepsi idolii lui și tot pământul lui va fi plin de gemetele celor ce sunt uciși.
\par 53 De s-ar ridica Babilonul până la cer și de și-ar întări întru înălțime cetatea sa, tot vor veni din porunca Mea pustiitorii, zice Domnul.
\par 54 Ascultați țipătul care se ridică din Babilon și uriașa trosnitură din Caldeea!
\par 55 Că Domnul va pustii Babilonul și va pune capăt glasului celui mândru al lui. Suna-vor valurile lor, ca apele cele mari, răsuna-va glasul lor.
\par 56 Căci pustiitorul va veni asupra lui, asupra Babilonului, și apărătorii lui vor fi luați și arcurile lor vor fi sfărâmate, că Domnul Dumnezeul răsplătirilor va da fiecăruia plata cuvenită.
\par 57 Voi îmbăta pe mai-marii lui și pe înțelepții iui, pe căpeteniile ținuturilor lui și pe căpeteniile cetăților lui și pe ostașii lui și vor dormi somnul de veci și nu se vor mai trezi, zice Domnul, al Cărui nume e Domnul Savaot".
\par 58 Așa zice Domnul Savaot: "Zidurile cele groase ale Babilonului le voi dărâma până la temelie și porțile lui cele înalte vor fi arse cu foc. Așadar în deșert s-au trudit popoarele și neamurile au muncit pentru foc".
\par 59 Cuvântul pe care proorocul Ieremia l-a încredințat lui Seraia, fiului Neria, fiul lui Maaseia, când acesta a plecat la Babilon cu Sedechia, regele lui Iuda, în anul al patrulea al domniei lui; Seraia era mare cămăraș.
\par 60 Atunci a scris Ieremia într-o carte toate nenorocirile care trebuia să vină asupra Babilonului.
\par 61 Și a zis Ieremia către Seraia: "Când vei ajunge la Babilon, caută să citești toate cuvintele acestea și să zici:
\par 62 "Doamne, Tu ai grăit de locul acesta că-l vei pierde așa încât să nu rămână într-însul nici om, nici animal, ci să fie pustietate veșnică".
\par 63 Și după ce vei isprăvi de citit cartea aceasta, leagă o piatră de ea și arunc-o în mijlocul Eufratului și zi:
\par 64 "Așa se va cufunda Babilonul și nu se va mai ridica din acea nenorocire pe care o voi aduce asupra lui și se va istovi". Până aici este vorbirea lui Ieremia.

\chapter{52}

\par 1 Sedechia era de douăzeci și unu de ani când a început să domnească și a domnit în Ierusalim unsprezece ani. Numele mamei sale era Hamutal, fata lui Ieremia din Libna.
\par 2 El a făcut rele în ochii Domnului, precum făcuse și Ioiachim.
\par 3 De aceea a venit mânia Domnului asupra Ierusalimului și a lui Iuda până într-atât încât i-a lepădat de la fața Sa, și Sedechia a fost dat jos de regele Babilonului.
\par 4 Era în al nouălea an al domniei lui, în luna a zecea, în ziua a zecea a lunii acesteia, când a venit Nabucodonosor, regele Babilonului, cu toată oștirea sa împotriva Ierusalimului, l-a înconjurat și a făcut împrejurul lui valuri de pământ.
\par 5 Cetatea a stat împresurată până în anul al unsprezecelea al regelui Sedechia.
\par 6 Iar în luna a patra, în ziua a noua a lunii acesteia, s-a întărit foametea în cetate, și poporul țării nu mai avea pâine.
\par 7 Atunci s-a făcut o spărtură în cetate, pe unde au ieșit toți oștenii și au fugit din cetate noaptea pe porțile ce se aflau între cele două ziduri de lângă grădina regelui; iar Caldeii erau împrejurul cetății.
\par 8 Și a alergat oștirea Caldeilor după rege și a ajuns pe Sedechia în șesurile Ierihonului. Atunci toată oștirea lui a fugit de la el.
\par 9 Deci au luat pe rege și l-au dus la regele Babilonului în Ribla, în ținutul Hamat, unde acesta l-a judecat.
\par 10 Regele Babilonului a junghiat pe fiii lui Sedechia înaintea ochilor acestuia; a junghiat de asemenea în Ribla și pe toți cei mari din Iuda.
\par 11 Iar lui Sedechia i-a scos ochii și a poruncit să-l încătușeze cu cătușe de aramă; și l-a dus regele Babilonului la Babilon și l-a pus în casa cea de pază, unde l-a ținut până în ziua morții lui.
\par 12 în luna a cincea, în ziua a zecea a lunii acesteia, în anul al nouăsprezecelea al regelui Nabucodonosor, regele Babilonului, a venit Nebuzaradan, căpetenia gărzii, care stătea înaintea regelui Babilonului, la Ierusalim și a ars templul Domnului,
\par 13 Casa regelui și toate casele din Ierusalim; toate casele cele mari le-a ars cu foc;
\par 14 Iar oștirea Caldeilor, care era cu căpetenia gărzii, a dărâmat toate zidurile dimprejurul Ierusalimului.
\par 15 Nebuzaradan, căpetenia gărzii, a strămutat pe cei săraci din popor și tot poporul care rămăsese în cetate și pe cei care se predaseră regelui Babilonului și toată rămășița de popor.
\par 16 Și numai puțini din poporul sărac al țării au fost lăsați de Nebuzaradan, căpetenia gărzii, ca lucrători pentru vii și ogoare.
\par 17 Caldeii au sfărâmat stâlpii cei de aramă, care se aflau în templul Domnului, postamentele și marea de aramă care se afla în templul Domnului, și toată arama lor au dus-o la Babilon.
\par 18 Au luat lighenele, lopețile, cuțitele și castroanele, lingurile și toate vasele de aramă, care erau întrebuințate la slujbele dumnezeiești;
\par 19 Și căpetenia gărzii a mai luat vasele și cleștele, cazanele și candelele, cățuile și cupele - tot ce era de aur și ce era de argint.
\par 20 De asemenea au luat cei doi stâlpi, marea și cei doisprezece boi de aramă, care serveau de postament și pe care regele Solomon îi făcuse pentru templul Domnului. În toate acestea era atâta aramă, cât nu se putea cântări.
\par 21 Fiecare stâlp din aceștia era de optsprezece coți în înălțime și o sfoară de doisprezece coți îl putea cuprinde împrejur, iar grosimea pereților lor era de patru degete, căci înăuntru nu erau plini.
\par 22 Coroana unui stâlp era de aramă și înălțimea ei era de cinci coți. Și pereții ei și rodiile dimprejur erau toate de aramă. Asemenea coroană cu rodii era și la celălalt stâlp.
\par 23 De jur împrejur erau nouăzeci și șase de rodii; tot așa și la celălalt stâlp împrejurul coroanei lui erau rodii.
\par 24 Căpetenia gărzii a luat și pe Seraia arhiereul și pe Sofonie, preotul al doilea, și pe trei străjeri ai pragurilor.
\par 25 Din cetate a luat un eunuc, care era căpetenie peste oștiri, și șapte oameni, care stăteau înaintea regelui și care se aflau în cetate; au mai luat pe secretarul căpeteniei oștirii, care înscrisese la oștire pe poporul țării, precum și șaizeci de oameni din poporul țării, care s-au găsit în cetate.
\par 26 Pe aceștia i-a luat Nebuzaradan, căpetenia gărzii, și i-a dus la Nabucodonosor, regele Babilonului, în Ribla.
\par 27 Și i a lovit pe ei regele Babilonului și i-a omorât în Ribla cea din țara Hamat. Așa a fost strămutat Iuda din țara sa.
\par 28 Iată acum poporul pe care l-a strămutat Nabucodonosor: în luna a șaptea, trei mii douăzeci și trei de oameni;
\par 29 În al optsprezecelea an al lui Nabucodonosor au fost strămutați din Ierusalim opt sute treizeci și două de suflete;
\par 30 În anul al douăzeci și treilea al lui Nabucodonosor, Nebuzaradan, căpetenia gărzii, a strămutat din Iudei șapte sute patruzeci și cinci de suflete: în total patru mii șase sute de suflete.
\par 31 În anul al treizeci și șaptelea după strămutarea lui Ioiachim, regele lui Iuda, în luna a douăsprezecea, în douăzeci și cinci ale lunii, Evil-Merodac, regele Babilonului, în anul întâi al domniei lui, s-a îndurat de Ioiachim, regele lui Iuda, și l-a scos din închisoare;
\par 32 A vorbit cu el prietenește și a pus scaunul lui mai sus decât al altor regi care erau la el în Babilon.
\par 33 A schimbat hainele lui de închisoare și Ioiachim a mâncat întotdeauna la masa regelui în toate zilele lui.
\par 34 Hrana lui i s-a dat de la rege zilnic, până la moartea sa, în toate zilele vieții sale.


\end{document}