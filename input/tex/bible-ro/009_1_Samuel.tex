\begin{document}

\title{1 Samuel}

1Sa 1:1  Era în vremea aceea un om la Ramataim-?ofim, în Muntele Efraim, cu numele Elcana, fiul lui Ieroham, fiul lui Elihu, fiul lui Tohu, fiul lui ?uf Efraimitul.
1Sa 1:2  Omul acela avea doua femei: numele uneia era Ana ?i numele celeilalte era Penina. Penina avea copii, iar Ana nu avea copii.
1Sa 1:3  Omul acela se ducea în fiecare an din cetatea sa, la ?ilo, sa se închine ?i sa aduca jertfa Domnului Savaot; acolo însa erau preo?i ai Domnului cei doi fii ai lui Eli: Ofni ?i Finees.
1Sa 1:4  În ziua când Elcana aducea jertfa dadea parte Peninei, femeii sale ?i tuturor fiilor ?i fiicelor ei;
1Sa 1:5  Iar Anei îi dadea parte îndoita, de?i aceasta nu avea copii, pentru ca el iubea pe Ana mai mult decât pe Penina, caci Domnul închisese pântecele ei.
1Sa 1:6  Potrivnica ei însa o amara grozav, a?â?ând-o ca sa cârteasca din pricina ca nu i-a dat Domnul prunci.
1Sa 1:7  A?a se întâmpla în fiecare an, când mergea ea la casa Domnului: aceea o amara, iar aceasta plângea ?i se tânguia ?i nu mânca.
1Sa 1:8  Dar iata o data Elcana, barbatul sau, i-a zis: "Ana!" ?i ea a raspuns: "Iata-ma!" A zis Elcana: "Ce plângi ?i de ce nu manânci ?i pentru ce e întristata inima ta? Nu sunt eu oare pentru tine mai bun decât zece copii?"
1Sa 1:9  Atunci Ana, dupa ce au mâncat ?i au baut ei în ?ilo, s-a sculat ?i a stat înaintea Domnului. Iar preotul Eli ?edea atunci pe scaun la u?a cortului Domnului.
1Sa 1:10  Ea însa s-a rugat Domnului cu sufletul întristat ?i a plâns amarnic,
1Sa 1:11  ?i a dat fagaduin?a, zicând: "Atotputernice Doamne, Dumnezeule Savaot, de vei cauta la întristarea roabei Tale ?i-?i vei aduce aminte de mine ?i de nu vei uita pe roaba Ta, ci vei da roabei Tale un copil de parte barbateasca, îl voi da ?ie, ?i nu va bea el nici vin, nici sichera, nici brici nu se va atinge de capul lui".
1Sa 1:12  Dar pe când se ruga ea a?a îndelung înaintea Domnului, Eli privea la gura ei;
1Sa 1:13  ?i fiindca Ana vorbea în inima sa, iar buzele ei numai se mi?cau, dar glasul nu i se auzea, Eli socotea ca ea e beata.
1Sa 1:14  De aceea i-a ?i zis Eli: "Pâna când ai sa stai aici beata? Treze?te-te ?i te du de la fa?a Domnului!"
1Sa 1:15  Raspunzând însa Ana a zis: "Nu, domnul meu! Eu sunt o femeie cu inima întristata; nici vin, nici sichera n-am baut, ci îmi dezvalui sufletul meu înaintea Domnului.
1Sa 1:16  Sa nu soco?i pe roaba ta femeie netrebnica, caci din durerea mea cea mare ?i din întristarea mea am vorbit pâna acum".
1Sa 1:17  Atunci Eli i-a raspuns ?i i-a zis: "Mergi în pace ?i Dumnezeul lui Israel sa-?i plineasca cererea pe care I-ai facut-o!"
1Sa 1:18  Iar ea i-a zis: "Sa afle roaba ta trecere înaintea ochilor tai!" Apoi s-a dus ea în calea sa ?i a mâncat ?i fa?a nu-i mai era trista ca mai înainte.
1Sa 1:19  Iar diminea?a s-au sculat ei ?i s-au închinat înaintea Domnului ?i, întorcându-se, au venit la casa lor în Rama. Dupa aceea a cunoscut Elcana pe Ana, femeia sa, ?i ?i-a adus aminte Domnul de ea.
1Sa 1:20  Dupa câtva timp a zamislit Ana ?i a nascut un fiu ?i i-a pus numele Samuel, caci î?i zicea ea: "De la Domnul Dumnezeul Savaot l-am cerut!"
1Sa 1:21  ?i s-a dus Elcana cu toata familia lui la ?ilo sa aduca jertfa Domnului, dupa fagaduin?ele sale, ?i toate zeciuielile de la pamântul sau.
1Sa 1:22  Ana însa nu s-a dus cu el, spunând barbatului sau: "Când pruncul va fi în?arcat de la sân ?i va cre?te, atunci am sa-l duc ?i va fi înfa?i?at el înaintea Domnului ?i va ramâne acolo pe totdeauna".
1Sa 1:23  Iar Elcana, barbatul ei, a zis catre ea: "Fa cum ?i se pare ca este bine; ramâi pâna-l vei în?arca; dar sa întareasca Domnul cuvântul ce a ie?it din gura ta". ?i a ramas femeia aceasta ?i a alaptat pruncul pâna l-a în?arcat.
1Sa 1:24  Iar dupa ce l-a în?arcat, s-a dus cu el la ?ilo, luând trei vi?ei, câteva pâini, o efa de faina ?i un burduf de vin, ?i a venit la casa Domnului în ?ilo ?i a venit ?i copilul împreuna cu ei, dar copilul era înca prunc.
1Sa 1:25  ?i l-au adus înaintea fe?ei Domnului, iar tatal sau a adus jertfa rânduita în asemenea zile ?i au junghiat un vi?el. Ana a adus pe prunc la Eli,
1Sa 1:26  Zicând: "O, domnul meu!, viu sa fie sufletul tau, domnul meu! Eu sunt acea femeie care am stat aici înaintea ta ?i m-am rugat Domnului.
1Sa 1:27  Pentru acest copil m-am rugat eu ?i Domnul mi-a plinit cererea ce am cerut de la Dânsul.
1Sa 1:28  ?i acum eu îl dau Domnului pentru toate zilele vie?ii lui, sa slujeasca Domnului". Apoi s-au închinat acolo Domnului.
1Sa 2:1  S-a rugat deci Ana ?i a zis: "Bucuratu-s-a inima mea întru Domnul, înal?ata a fost fruntea mea de Domnul Dumnezeul meu ?i gura mea s-a deschis larg asupra vrajma?ilor mei, caci m-am bucurat de izbavirea Ta.
1Sa 2:2  Nimeni nu este sfânt ca Domnul, caci nu e altul afara de Tine; ?i nimeni nu e puternic ca Dumnezeul nostru.
1Sa 2:3  Nu va lauda?i ?i cuvinte trufa?e sa nu iasa din gura voastra, caci Domnul este Dumnezeul a toata cuno?tin?a ?i lucrurile la Dânsul sunt cântarite.
1Sa 2:4  Arcul celor puternici s-a frânt, iar cei slabi s-au încins cu putere.
1Sa 2:5  Cei satui vor munci pentru pâine, iar cei flamânzi nu vor mai avea foame. Cea stearpa va na?te de ?apte ori, iar cea cu copii mul?i va fi neputincioasa.
1Sa 2:6  Domnul omoara ?i învie; El coboara la locuin?a mor?ilor ?i iara?i scoate.
1Sa 2:7  Domnul sarace?te pe om ?i tot El îl îmboga?e?te; El smere?te ?i El înal?a. El ridica pe cel sarac din pulbere ?i din gunoi pe cel lipsit, punându-i în rând cu cei puternici ?i dându-le scaunul maririi, caci ale Domnului sunt temeliile pamântului ?i El întemeiaza lumea pe ele.
1Sa 2:8  Pa?ii sfin?ilor Sai El îi paze?te, iar nelegiui?ii vor pieri întru întuneric caci omul nu prin putere este tare.
1Sa 2:9  Domnul va zdrobi pe cei ce se împotrivesc Lui; va tuna din cer asupra lor. Sfânt este Domnul; sa nu se laude cel în?elept cu în?elepciunea sa ?i cel puternic sa nu se laude cu puterea sa, nici cel bogat sa nu se faleasca cu boga?ia sa; ci cel ce voie?te sa se laude, cu aceea sa se laude ca ?tie ?i cunoa?te pe Domnul ?i face judecata ?i dreptate în mijlocul pamântului.
1Sa 2:10  Domnul din înal?imea cerului va tuna peste vrajma?ii Sai, El va judeca marginile pamântului, drept fiind, El va da tarie regilor no?tri ?i fruntea Unsului Sau o va înal?a".
1Sa 2:11  ?i au lasat pe Samuel acolo, înaintea Domnului. Apoi Elcana s-a dus la casa sa în Rama, iar copilul a ramas sa slujeasca Domnului sub pova?a preotului Eli.
1Sa 2:12  Însa fiii lui Eli erau oameni netrebnici. Ei nu ?tiau de Domnul,
1Sa 2:13  Nici de datoria preo?easca fa?a de popor. Când aducea cineva jertfa, feciorul preotului venea în timpul când se fierbea carnea cu o furculi?a în mâna,
1Sa 2:14  O vâra în caldare, sau în blid, sau în tava, sau în oala, ?i ce prindea furculi?a, aceea lua feciorul preotului. A?a faceau ei cu to?i Israeli?ii care veneau acolo în ?ilo.
1Sa 2:15  Chiar ?i înainte de a arde grasimea venea feciorul preotului ?i zicea catre cel ce aducea jertfa: "Da carne de friptura pentru preot, caci el n-are sa ia de la tine carne fiarta, ci da-i-o cruda".
1Sa 2:16  ?i daca cineva zicea: "Lasa sa se arda mai întâi grasimea, cum se cuvine, ?i apoi î?i vei lua cât î?i va pofti sufletul", atunci el raspundea: "Ba nu, da-mi chiar acum, iar de nu, voi lua cu de-a sila".
1Sa 2:17  ?i pacatul acestor tineri era foarte mare înaintea Domnului, caci ei departau lumea de a mai aduce jertfa Domnului.
1Sa 2:18  Iar copilul Samuel slujea înaintea Domnului, îmbracat cu efod de in.
1Sa 2:19  Meilul de deasupra i-l facea mama sa ?i i-l aducea în fiecare an, când venea cu barbatul ei sa aduca jertfa rânduita.
1Sa 2:20  ?i a binecuvântat Eli pe Elcana ?i pe femeia lui ?i a zis: "Sa-?i dea ?ie Domnul copii de la femeia aceasta în locul celui afierosit, pe care tu l-ai daruit Domnului!" ?i s-au dus ei la casa lor.
1Sa 2:21  Dupa aceea a cercetat Domnul pe Ana ?i ea a zamislit ?i a mai nascut trei baie?i ?i doua fete, iar copilul Samuel cre?tea înaintea Domnului.
1Sa 2:22  Eli însa era tare batrân ?i auzea de toata purtarea fiilor sai fa?a de întreg Israelul ?i ca se culcau cu femeile celor ce se adunau la u?a cortului adunarii.
1Sa 2:23  ?i le-a zis: "Pentru ce face?i voi asemenea lucruri, caci aud vorbe rele despre voi de la tot poporul Domnului?
1Sa 2:24  Nu, copiii mei, nu este buna vestea ce o aud eu despre voi; nu mai face?i a?a, caci nu este buna vestea care o aud eu; voi razvrati?i poporul Domnului.
1Sa 2:25  Ca de va gre?i omul fa?a de alt om, atunci se vor ruga pentru el lui Dumnezeu; iar de va pacatui cineva fa?a de Domnul, atunci cine va mijloci pentru el?" Dar ei nu ascultau spusele tatalui lor, caci Domnul hotarâse sa-i dea mor?ii.
1Sa 2:26  Iar copilul Samuel cre?tea mereu cu vârsta ?i era iubit ?i de Dumnezeu ?i de oameni.
1Sa 2:27  În vremea aceea a venit la Eli un om al lui Dumnezeu ?i i-a zis: "A?a graie?te Domnul: Nu M-am aratat Eu oare casei tatalui tau când erau ei înca în Egipt, în casa lui Faraon?
1Sa 2:28  ?i nu l-am ales Eu oare din toate semin?iile lui Israel sa-Mi fie preot, ca sa aprinda tamâie ?i sa poarte efod înaintea Mea? ?i nu am dat Eu oare casei tatalui tau din toate jertfele ce se frig pe foc de fiii lui Israel?
1Sa 2:29  Pentru ce dar voi calca?i în picioare jertfele Mele ?i prinoasele Mele de pâine, pe care le-am poruncit Eu pentru loca?ul Meu, ?i pentru ce tu ?ii mai mult la fiii tai decât la Mine, îngra?ându-i pe ei cu toata pârga poporului Meu Israel?
1Sa 2:30  De aceea a?a zice Domnul Dumnezeul lui Israel: Eu am zis odinioara: Casa ta ?i casa tatalui tau va umbla nestramutat înaintea fe?ei Mele în veac; dar acum Domnul zice: Sa nu mai fie a?a, caci Eu preamaresc pe cei ce Ma preaslavesc pe Mine, iar cei ce Ma necinstesc vor fi ru?ina?i.
1Sa 2:31  Iata vin zile când Eu voi taia bra?ul tau ?i bra?ul casei tatalui tau, încât sa nu mai fie batrân în casa ta niciodata.
1Sa 2:32  ?i tu vei vedea casa Mea ca un du?man pentru tine, de?i Domnul va milui pe Israel ?i nu va mai fi în casa ta batrân în toate zilele.
1Sa 2:33  Nu voi departa pe to?i ai tai de la jertfelnicul Meu, ca sa chinuiesc ochii tai ?i sa apas sufletul tau; dar to?i urma?ii casei tale vor muri la mijlocul anilor lor.
1Sa 2:34  ?i iata un semn pentru tine, care se va petrece cu cei doi fii ai tai, Ofni ?i Finees: amândoi vor muri în aceea?i zi.
1Sa 2:35  ?i-Mi voi pune un preot credincios. Acela se va purta dupa inima Mea ?i dupa sufletul Meu; ?i casa lui o voi întari ?i va umbla el înaintea Unsului Meu în toate zilele;
1Sa 2:36  ?i tot cel ramas din casa ta va veni sa se închine lui pentru o ghera de argint ?i pentru o buca?ica de pâine, ?i va zice: "Numara-ma la vreuna din slujbele levitice, ca sa pot mânca o bucata de pâine!"
1Sa 3:1  ?i pruncul Samuel slujea Domnului sub pova?a preotului Eli. În zilele acelea cuvântul Domnului era rar ?i nici vedeniile nu erau dese.
1Sa 3:2  ?i iata în vremea aceea, când Eli statea culcat la locul sau ?i ochii lui începusera a se închide ?i nu mai putea sa vada;
1Sa 3:3  Când sfe?nicul Domnului nu se stinsese înca ?i Samuel era culcat în cortul Domnului, unde era chivotul lui Dumnezeu,
1Sa 3:4  A strigat Domnul catre Samuel: "Samuele, Samuele!" Iar el a raspuns: " Iata-ma! "
1Sa 3:5  ?i a alergat la Eli ?i a zis: "Iata-ma! La ce m-ai chemat?" Acela însa a raspuns: "Nu te-am chemat. Du-te ?i te culca!" ?i s-a dus Samuel ?i s-a culcat.
1Sa 3:6  Iar Domnul a strigat a doua oara pe Samuel: "Samuele, Samuele!" ?i acesta s-a sculat ?i a venit iar la Eli ?i a zis: "Iata-ma! De ce m-ai chemat?" ?i acela i-a zis: "Nu te-am chemat, fiul meu! Du-te înapoi ?i te culca!"
1Sa 3:7  Samuel nu cuno?tea atunci glasul Domnului ?i cuvântul Domnului nu i se descoperise înca.
1Sa 3:8  Dar Domnul a strigat pe Samuel ?i a treia oara. ?i s-a sculat acesta ?i a venit la Eli ?i a zis: "Iata-ma! La ce m-ai chemat?" Atunci a în?eles Eli ca Domnul cheama pe copil.
1Sa 3:9  ?i a zis Eli catre Samuel: "Du-te înapoi ?i te culca ?i când Cel ce te cheama te va mai chema, tu sa zici: "Vorbe?te, Doamne, ca robul Tau asculta!" ?i s-a dus Samuel ?i s-a culcat la locul sau.
1Sa 3:10  ?i a venit Domnul ?i a stat ?i a strigat ca întâia ?i ca a doua oara: "Samuele, Samuele!" Iar Samuel a zis: "Vorbe?te, Doamne, ca robul Tau asculta!"
1Sa 3:11  A zis Domnul catre Samuel: "Iata, am sa fac în Israel un lucru, încât cine va auzi de el, aceluia îi vor ?iui amândoua urechile.
1Sa 3:12  În ziua aceea voi face cu Eli toate câte am spus despre casa lui; toate le voi începe ?i le voi sfâr?i.
1Sa 3:13  Eu i-am spus ca am sa pedepsesc casa lui pe veci pentru vina pe care el a ?tiut-o, ?i anume ca fiii fac nelegiuiri, dar nu i-a înfrânat.
1Sa 3:14  De aceea ma jur casei lui Eli ca vina casei lui Eli nu se va ?terge, nici prin jertfe, nici prin prinoase de pâine în veci".
1Sa 3:15  Dupa aceea a adormit Samuel pâna diminea?a, s-a sculat de noapte ?i a deschis u?ile casei Domnului. Dar Samuel s-a temut sa spuna lui Eli vedenia aceasta.
1Sa 3:16  Eli însa a chemat pe Samuel ?i a zis: "Fiul meu Samuel! " Iar acesta a raspuns: "Iata-ma!"
1Sa 3:17  A zis Eli: "Ce ?i s-a spus? Sa nu ascunzi de mine! Dumnezeu sa se poarte cu tine cu toata asprimea, daca tu vei ascunde ceva de mine din toate câte ?i s-au spus!"
1Sa 3:18  ?i i-a spus Samuel tot ?i n-a ascuns nimic de el. Atunci Eli a zis: "El este Domnul; faca dar ce va binevoi!"
1Sa 3:19  ?i a crescut Samuel ?i Domnul a fost cu el ?i n-a ramas neîmplinit nici unul din cuvintele Lui.
1Sa 3:20  Atunci a aflat tot Israelul, de la Dan pâna la Beer-?eba, ca Samuel s-a învrednicit sa fie prooroc al Domnului.
1Sa 3:21  ?i a urmat Domnul a Se arata în ?ilo dupa ce Se aratase lui Samuel acolo prin cuvântul Sau. ?i s-au încredin?at to?i în tot Israelul, de la o margine pâna la cealalta margine a ?arii, ca Samuel este proorocul Domnului. Iar Eli a ajuns foarte batrân, ?i feciorii lui staruiau pe calea lor înaintea Domnului.
1Sa 4:1  În vremea aceea s-au sculat Filistenii sa se razboiasca cu Israeli?ii ?i a fost cuvântul lui Samuel catre tot Israelul. ?i au pa?it Israeli?ii împotriva Filistenilor cu razboi ?i au tabarât la Eben-Ezer, iar Filistenii au tabarât la Afec.
1Sa 4:2  Apoi Filistenii s-au a?ezat în rânduiala de razboi în fa?a Israeli?ilor ?i, dându-se batalia, au fost batu?i Israeli?ii de catre Filisteni, care au ucis pe câmpul de lupta ca la patru mii de oameni.
1Sa 4:3  Dupa aceea au venit oamenii în, tabara ?i au spus batrânilor lui Israel: "Pentru ce oare ne-a lovit pe noi Domnul înaintea Filistenilor? Sa luam cu noi din ?ilo chivotul legii Domnului, ca sa mearga în mijlocul nostru ?i sa ne izbaveasca din mâinile vrajma?ilor no?tri!"
1Sa 4:4  ?i a trimis poporul la ?ilo, de au adus de acolo chivotul legii Domnului Savaot, Cel ce ?ade pe heruvimi; iar pe lânga chivotul legii Domnului erau ?i cei doi fii ai lui Eli: Ofni ?i Finees.
1Sa 4:5  Iar daca a sosit chivotul legii Domnului în tabara, tot Israelul a ridicat strigat a?a de mare, încât s-a cutremurat pamântul.
1Sa 4:6  ?i auzind Filistenii rasunetul strigatelor, au zis: "Ce înseamna aceste strigate puternice în tabara Evreilor?" ?i au aflat ca a sosit în tabara chivotul legii Domnului.
1Sa 4:7  Atunci s-au înspaimântat Filistenii, caci ziceau: "Dumnezeul lor a venit la ei în tabara". Apoi au zis: "Vai de noi! Caci n-a mai fost asemenea lucru nici ieri, nici alaltaieri!
1Sa 4:8  Vai de noi! Cine ne va scapa din mâinile acestui Dumnezeu puternic? Acesta este acel Dumnezeu Care a batut pe Egipteni cu tot felul de pedepse în pustiu.
1Sa 4:9  Întari?i-va ?i fi?i curajo?i, Filistenilor, ca sa nu cade?i în robie la Evrei, cum sunt ei în robie la noi! Fi?i curajo?i ?i va lupta?i cu ei!"
1Sa 4:10  ?i s-au luptat Filistenii cu Israeli?ii ?i au fost înfrân?i ace?tia, ?i a fugit fiecare în cortul sau; batalia aceasta a fost foarte mare, cazând dintre Israeli?i treizeci de mii de pedestra?i.
1Sa 4:11  ?i a fost luat chivotul legii Domnului, iar cei doi fii ai lui Eli, Ofni ?i Finees, au murit.
1Sa 4:12  Atunci a alergat un veniaminean de la locul bataliei ?i a venit la ?ilo în aceea?i zi, având hainele de pe el rupte ?i pulbere pe cap.
1Sa 4:13  Iar când a venit acela, Eli ?edea pe scaun lânga drum la poarta ?i privea, caci i se batea inima pentru chivotul lui Dumnezeu. ?i dupa ce a sosit omul acela ?i a spus în cetate, atunci s-a tânguit stra?nic toata cetatea.
1Sa 4:14  Auzind Eli rasunetele tânguirilor, a întrebat: "Pentru ce este acest bocet?" Dar a sosit îndata omul acela ?i i-a spus toate lui Eli.
1Sa 4:15  Eli însa era atunci de nouazeci ?i opt de ani; ochii i se întunecasera ?i nu mai putea sa vada.
1Sa 4:16  ?i a zis omul acela catre Eli: "Eu vin din tabara. Chiar astazi am fugit de pe câmpul de lupta". Iar Eli a zis: "Ce s-a întâmplat, fiul meu?" Vestitorul însa a raspuns ?i a zis: "Israeli?ii au fugit din fa?a Filistenilor ?i s-a facut în popor ucidere mare; amândoi fiii tai, Ofni ?i Finees, au murit ?i chivotul lui Dumnezeu a fost luat".
1Sa 4:17  Când a pomenit el de chivotul Domnului, Eli a cazut de pe scaun pe spate la poarta, ?i-a rupt spinarea ?i a murit, caci era batrân ?i greoi.
1Sa 4:18  El a fost judecator în Israel patruzeci de ani.
1Sa 4:19  Iar nora lui, femeia lui Finees, era însarcinata ?i aproape de na?tere. Când a auzit vestea despre luarea chivotului Domnului ?i despre moartea socrului sau ?i a barbatului sau, a cazut în genunchi ?i a nascut, caci o apucasera durerile ei.
1Sa 4:20  ?i pe când murea ea, femeile care stateau împrejur îi zisera: "Nu te teme, ca ai nascut baiat". Ea însa nu a raspuns ?i nu a dat semn de luare aminte.
1Sa 4:21  ?i au pus copilului numele: Icabod, zicând: "S-a dus slava din Israel, din pricina pierderii chivotului Domnului ?i a mor?ii socrului ?i a barbatului ei.
1Sa 4:22  ?i a zis ea: "S-a dus slava din Israel, caci s-a luat chivotul Domnului!
1Sa 5:1  Atunci au luat Filistenii chivotul Domnului ?i l-au dus din Eben-Ezer la A?dod.
1Sa 5:2  Apoi au ridicat Filistenii chivotul Domnului ?i l-au vârât în capi?tea lui Dagon ?i l-au pus lânga Dagon.
1Sa 5:3  Iar a doua zi s-au sculat A?dodenii dis-de-diminea?a ?i iata Dagon zacea cu fa?a la pamânt înaintea chivotului Domnului. ?i au luat ei pe Dagon ?i l-au pus iar la locul lui.
1Sa 5:4  ?i s-au sculat ei dis-de-diminea?a în ziua urmatoare, ?i iata Dagon zacea cu fa?a la pamânt înaintea chivotului Domnului; dar capul lui Dagon ?i amândoua picioarele ?i mâinile lui zaceau taiate pe prag, fiecare deosebi, ?i ramasese numai trunchiul lui.
1Sa 5:5  De aceea preo?ii lui Dagon ?i to?i câ?i vin în capi?tea lui Dagon din A?dod nu calca pe pragul lui Dagon pâna în ziua de astazi, ci pa?esc peste el.
1Sa 5:6  ?i a apasat mâna Domnului asupra A?dodenilor ?i i-a lovit ?i i-a pedepsit cu bube usturatoare pe cei din A?dod ?i din împrejurimile lui, iar înauntrul ?arii s-au înmul?it ?oarecii ?i s-a la?it în cetate deznadejde mare.
1Sa 5:7  Vazând aceasta, A?dodenii au zis: "Sa nu mai ramâna chivotul Dumnezeului lui Israel la noi, ca e grea mâna Lui ?i pentru noi ?i pentru Dagon, dumnezeul nostru!"
1Sa 5:8  Apoi au trimis ?i au adunat la ei pe to?i mai-marii Filistenilor ?i le-au zis: "Ce sa facem cu chivotul Dumnezeului lui Israel?" Iar Gateenii au zis: "Sa treaca dar chivotul Dumnezeului lui Israel la noi în Gat". ?i au trimis la Gat chivotul Dumnezeului lui Israel.
1Sa 5:9  Iar dupa ce l-au trimis, a fost mâna Domnului asupra ceta?ii aceleia cu stra?nicie mare ?i a batut Domnul pe locuitorii ceta?ii de la mic pâna la mare ?i s-au ivit pe ei buboaie.
1Sa 5:10  De aceea au trimis chivotul Domnului la Ecron. Iar când a sosit chivotul Domnului în Ecron au strigat Ecronenii ?i au zis: "Chivotul Dumnezeului lui Israel a venit la noi ca sa omoare ?i pe poporul nostru".
1Sa 5:11  Atunci au trimis ?i au adunat toate capeteniile Filistenilor ?i au zis: "Trimite?i de aici chivotul Dumnezeului lui Israel; lasa?i-l sa se întoarca la locul sau, ca sa nu ne ucida pe noi ?i pe poporul nostru". Caci groaza de moarte era în tot ora?ul ?i mâna Domnului apasa foarte tare asupra lor, de cum venise acolo chivotul Dumnezeului lui Israel.
1Sa 5:12  ?i aceia care nu murisera fusesera lovi?i cu buboaie, a?a încât plânsetele ceta?ii se înal?au pâna la cer.
1Sa 6:1  ?i a stat chivotul Domnului în ?ara Filistenilor ?apte luni ?i s-a umplut ?ara aceea de ?oareci.
1Sa 6:2  Atunci au adunat Filistenii pe preo?i, pe ghicitori ?i pe descântatori ?i au zis: "Ce sa facem oare cu chivotul Domnului? Înva?a?i-ne cum sa-l trimitem la locul lui?"
1Sa 6:3  Iar aceia au zis: "Daca voi?i sa trimite?i chivotul legii Domnului Dumnezeului lui Israel, sa nu-l trimite?i fara nimic, ci aduce?i-I jertfa pentru pacat, ?i atunci va ve?i vindeca ?i ve?i afla pentru ce nu s-a îndepartat de la voi mâna Lui".
1Sa 6:4  Apoi au mai zis: "Ce fel de jertfa pentru pacat trebuie sa-I aducem?" Iar aceia au zis: "Dupa numarul capeteniilor Filistenilor, cinci buboaie de aur ?i cinci ?oareci de aur, caci pedeapsa este una ?i asupra voastra ?i asupra capeteniilor voastre.
1Sa 6:5  A?adar, face?i ni?te chipuri cioplite de buboaie de ale voastre ?i ni?te chipuri de ?oareci de ai vo?tri care pustiesc pamântul ?i da?i slava Dumnezeului lui Israel; poate ca Î?i va ridica mâna de deasupra voastra, de deasupra dumnezeilor vo?tri ?i de deasupra pamântului vostru.
1Sa 6:6  De ce sa va învârto?a?i inimile voastre, cum ?i-au învârto?at inimile Egiptenii ?i Faraon? Iata când Domnul ?i-a aratat puterea Sa asupra lor, atunci ei i-au lasat ?i aceia au plecat.
1Sa 6:7  Lua?i dar ?i face?i un car nou ?i lua?i doua vaci care au fatat întâia oara, care n-au mai purtat jug, ?i înjuga?i vacile la car, iar vi?eii lor duce?i-i acasa.
1Sa 6:8  Apoi lua?i chivotul Domnului ?i-l pune?i în car, iar lucrurile cele de aur care I se aduc jertfa pentru pacat, sa le pune?i într-o ladi?a alaturi, ?i da?i-i drumul sa se duca.
1Sa 6:9  Sa va uita?i însa: Daca el va pleca spre hotarele sale, spre Bet?eme?, atunci acest mare rau ni l-a facut El; daca nu va porni într-acolo, atunci vom ?ti ca nu ne-a lovit mâna Lui, ci aceasta ne-a venit din întâmplare".
1Sa 6:10  ?i au facut ei a?a: au luat doua vaci care au fatat întâia oara ?i le-au înjugat la car, iar vi?eii i-au oprit acasa;
1Sa 6:11  Apoi au pus chivotul Domnului în car, iar ladi?a cu ?oarecii cei de aur ?i cu chipurile cioplite în chip de buboaie au pus-o alaturi de chivot.
1Sa 6:12  ?i au pornit vacile de-a dreptul pe drumul spre Bet?eme?; ?i au ?inut calea mereu înainte, mugind, dar neoprindu-se ?i neabatându-se nici la dreapta nici la stânga; iar capeteniile Filistenilor au mers în urma lor pâna la hotarele Bet?eme?ului.
1Sa 6:13  Tocmai atunci locuitorii Bet?eme?ului secerau grâul în vale; ?i ridicându-?i ochii, au vazut chivotul Domnului ?i s-au bucurat când l-au vazut.
1Sa 6:14  Carul însa a venit în ?arina lui Iosua din Bet?eme? ?i s-a oprit acolo. ?i se afla acolo o piatra mare; au despicat lemnele carului, iar vacile au fost aduse ardere de tot Domnului.
1Sa 6:15  Apoi levi?ii au ridicat chivotul Domnului ?i ladi?a cea de lânga el în care se aflau lucrurile cele de aur ?i le-au pus pe piatra cea mare; iar locuitorii din Bet?eme? au adus în ziua aceea arderi de tot ?i au junghiat jertfe Domnului.
1Sa 6:16  ?i cele cinci capetenii ale Filistenilor, dupa ce au vazut aceasta, s-au întors în aceea?i zi la Ecron.
1Sa 6:17  Iar buboaiele cele de aur pe care le-au adus Filistenii jertfa Domnului pentru pacat erau: unul pentru A?dod, unul pentru Gaza, unul pentru Ascalon, unul pentru Gat, unul pentru Ecron.
1Sa 6:18  Iar ?oarecii de aur erau dupa numarul tuturor ceta?ilor filistene ale celor cinci capetenii, de la ceta?ile întarite pâna la satele deschise, pâna la piatra cea mare pe care s-a pus chivotul Domnului ?i care se afla pâna în ziua de astazi în ?arina lui Iosua din Bet?eme?.
1Sa 6:19  Dar dintre oamenii din Bet?eme? nu s-au bucurat fiii lui Iehonia ca au vazut chivotul Domnului. ?i a lovit Domnul pe locuitorii din Bet?eme?, pentru ca ei s-au uitat la chivotul Domnului, ?i a ucis din popor cincizeci de mii ?aptezeci de oameni. Atunci a plâns poporul, pentru ca lovise Domnul poporul cu pedeapsa mare.
1Sa 6:20  Au zis locuitorii din Bet?eme?: "Cine poate sa stea înaintea Domnului, a Acestui Dumnezeu sfânt? ?i la cine se va duce El de la noi?"
1Sa 6:21  Deci au trimis soli la locuitorii din Chiriat-Iearim sa le spuna: "Filistenii au întors chivotul Domnului; veni?i ?i-l lua?i la voi!"
1Sa 7:1  Atunci au venit locuitorii din Chiriat-Iearim ?i au luat chivotul Domnului ?i l-au adus în casa lui Aminadab, pe deal, iar pe Eleazar, fiul lui, l-au sfin?it ca sa pazeasca chivotul Domnului.
1Sa 7:2  Din ziua aceea, de când a ramas chivotul în Chiriat-Iearim, a trecut multa vreme, ca la douazeci de ani. ?i s-a întors toata casa lui Israel la Domnul.
1Sa 7:3  Iar Samuel a zis catre toata casa lui Israel: "De va întoarce?i cu toata inima voastra la Domnul, atunci departa?i din mijlocul vostru dumnezeii straini ?i astartele ?i va lipi?i inimile voastre de Domnul ?i-I sluji?i numai Lui, ?i El va va izbavi din mâinile Filistenilor!"
1Sa 7:4  ?i au departat fiii lui Israel baalii ?i astartele ?i au început sa slujeasca numai Domnului.
1Sa 7:5  Apoi a zis iara?i Samuel: "Aduna?i pe to?i Israeli?ii la Mi?pa ?i eu ma voi ruga pentru voi Domnului".
1Sa 7:6  ?i s-au adunat la Mi?pa ?i au scos apa, ?i au turnat înaintea Domnului ?i au postit în ziua aceea, zicând: "Gre?it-am înaintea Domnului!" ?i a judecat Samuel pe fiii lui Israel în Mi?pa.
1Sa 7:7  Când însa au auzit Filistenii ca s-au adunat fiii lui Israel în Mi?pa, s-au dus capeteniile Filistenilor asupra lui Israel. Auzind Israeli?ii de aceasta, s-au temut de Filisteni.
1Sa 7:8  Atunci au zis fiii lui Israel catre Samuel: "Nu înceta a striga pentru noi catre Domnul Dumnezeul nostru, ca sa ne izbaveasca din mâinile Filistenilor". Iar Samuel a zis: "Sa nu fie cu mine una ca aceasta, ca sa ma departez de Domnul Dumnezeul meu ?i sa nu strig pentru voi întru rugaciune!
1Sa 7:9  Apoi a luat Samuel un miel de lapte ?i l-a adus împreuna cu tot poporul ardere de tot Domnului ?i a strigat Samuel catre Domnul pentru Israel ?i l-a auzit pe el Domnul.
1Sa 7:10  Când înal?a Samuel ardere de tot, au venit Filistenii sa se bata cu Israel. Dar Domnul a tunat în ziua aceea cu tunet mare asupra Filistenilor ?i a adus groaza asupra lor, a?a ca au fost înfrân?i în fa?a lui Israel.
1Sa 7:11  Iar Israeli?ii au ie?it din Mi?pa ?i au urmarit pe Filisteni ?i i-au batut pâna sub Bet-Car.
1Sa 7:12  Atunci a luat Samuel o piatra ?i a pus-o între Mi?pa ?i ?en ?i a numit-o Eben-Ezer, zicând: "Pâna la locul acesta ne-a ajutat noua Domnul!"
1Sa 7:13  A?a au fost înfrân?i Filistenii ?i nu s-au mai apucat sa mai umble peste hotarele lui Israel. Mâna Domnului a fost asupra Filistenilor în toate zilele lui Samuel.
1Sa 7:14  ?i au fost întoarse lui Israel toate ceta?ile pe care le luasera Filistenii de la Israel, de la Ecron pâna la Gat, ?i ?inuturile acestora le-a liberat Israel din mâinile Filistenilor ?i a fost pace între Israel ?i Amorei.
1Sa 7:15  Astfel a fost Samuel judecator în Israel în toate zilele vie?ii sale.
1Sa 7:16  El mergea din an în an ?i cerceta Betelul ?i Ghilgalul ?i Mi?pa ?i judeca pe Israel în toate locurile acestea.
1Sa 7:17  Apoi se întorcea la Rama, caci acolo era casa lui ?i acolo judeca el pe Israel. ?i a ridicat acolo jertfelnic Domnului.
1Sa 8:1  Iar daca a îmbatrânit Samuel, a pus pe fiii sai judecatori peste Israel.
1Sa 8:2  Numele fiului sau celui mai mare era Ioil, iar numele fiului sau al doilea era Abia. Ace?tia erau judecatori în Beer-?eba.
1Sa 8:3  Dar fiii lui nu umblau pe caile sale, ci se abateau la lacomie, luau daruri ?i judecau strâmb.
1Sa 8:4  Atunci s-au adunat to?i batrânii lui Israel, au venit la Samuel, în Rama,
1Sa 8:5  ?i au zis catre el: "Tu ai îmbatrânit, iar fiii tai nu-?i urmeaza caile. De aceea pune peste noi un rege, ca sa ne judece acela, ca ?i la celelalte popoare!"
1Sa 8:6  Cuvântul acesta însa n-a placut lui Samuel când i-au zis: "Da-ne rege, ca sa ne judece!" ?i s-a rugat Samuel Domnului.
1Sa 8:7  ?i a zis Domnul catre Samuel: "Asculta glasul poporului în toate câte î?i graie?te; caci nu pe tine te-au lepadat, ci M-au lepadat pe Mine, ca sa nu mai domnesc Eu peste ei.
1Sa 8:8  Cum s-au purtat ei cu Mine din ziua aceea, când i-am scos din Egipt, pâna astazi, parasindu-Ma ?i slujind la dumnezei straini, a?a se poarta ?i cu tine.
1Sa 8:9  Asculta deci glasul lor, dar sa le spui ?i sa le ara?i drepturile regelui, care va domni peste ei".
1Sa 8:10  ?i a spus Samuel toate cuvintele Domnului poporului care îi cerea rege
1Sa 8:11  ?i a zis: "Iata care vor fi drepturile regelui care va domni peste voi: pe fiii vo?tri îi va lua ?i-i va pune la carele sale ?i va face din ei calare?ii sai ?i vor fugi pe lânga carele lui.
1Sa 8:12  Va pune din ei capetenii peste mii, capetenii peste sute, capetenii peste cincizeci; sa lucreze ?arinile sale, sa-i secere pâinea sa, sa-i faca arme de razboi ?i unelte la carele lui.
1Sa 8:13  Fetele voastre le va lua, ca sa faca miresme, sa gateasca mâncare ?i sa coaca pâine.
1Sa 8:14  ?arinile, viile ?i gradinile de maslini cele mai bune ale voastre le va lua ?i le va da slugilor sale.
1Sa 8:15  Din semanaturile voastre ?i din viile voastre va lua zeciuiala ?i va da oamenilor sai ?i slugilor sale.
1Sa 8:16  Din robii vo?tri, din roabele voastre, din cei mai buni feciori ai vo?tri ?i din asinii vo?tri va lua ?i-i va întrebuin?a la treburile sale.
1Sa 8:17  Din oile voastre va lua a zecea parte ?i chiar voi ve?i fi robii lui.
1Sa 8:18  Ve?i suspina atunci sub regele vostru, pe care vi l-a?i ales, ?i atunci nu va va raspunde Domnul".
1Sa 8:19  Poporul însa nu s-a învoit sa asculte pe Samuel, ci a zis: "Nu, lasa sa fie rege peste noi,
1Sa 8:20  ?i vom fi ?i noi ca celelalte popoare, ne va judeca regele nostru, va merge înainte ?i va purta razboaiele noastre".
1Sa 8:21  A ascultat deci Samuel toate cuvintele poporului ?i le-a spus Domnului;
1Sa 8:22  Iar Domnul a zis catre Samuel: "Asculta glasul lor ?i pune-le rege!" Atunci a zis Samuel catre Israeli?i: "Duce?i-va fiecare în cetatea sa!"
1Sa 9:1  În vremea aceea era unul din fiii lui Veniamin, cu numele Chi?, fiul lui Abiel, fiul lui ?eror, fiul lui Becorat, fiul lui Afia, fiul unui veniaminean, om de isprava.
1Sa 9:2  Acesta avea un fiu, cu numele Saul, tânar ?i frumos, încât nu mai era nimeni în Israel mai frumos ca el; acesta era de la umeri în sus mai înalt decât tot poporul.
1Sa 9:3  O data s-au ratacit asinele lui Chi?, tatal lui Saul, ?i a zis Chi? catre Saul, fiul sau: "Ia cu tine pe unul din arga?i ?i, sculându-te, du-te de cauta asinele!"
1Sa 9:4  ?i a suit acesta muntele lui Efraim ?i a strabatut ?inutul ?ali?a, dar nu le-a gasit. Apoi a strabatut ?inutul ?aalim, ?i nici acolo nu le-a gasit. Apoi a strabatut ?i pamântul lui Veniamin ?i tot nu le-a gasit.
1Sa 9:5  Iar când au ajuns în ?inutul ?uf, a zis Saul catre sluga sa care era cu el: "Haidem înapoi, ca nu cumva tatal meu, lasând asinele, sa înceapa a fi nelini?tit de noi".
1Sa 9:6  Sluga însa i-a zis: "Iata în cetatea aceasta este un om al lui Dumnezeu, om cinstit de to?i ?i tot ce spune el se pline?te. Sa mergem dar acolo; poate ne va arata ?i noua calea pe care sa apucam".
1Sa 9:7  A zis Saul catre sluga sa: "Haidem sa mergem, dar ce sa ducem noi omului aceluia? Caci pâinea s-a ispravit din traistele noastre ?i daruri nu avem ca sa ducem omului lui Dumnezeu. Ce avem noi?"
1Sa 9:8  Sluga a raspuns iara?i ?i a zis: "Iata am în mâna un sfert de siclu de argint; îl voi da omului lui Dumnezeu ?i el ne va arata calea".
1Sa 9:9  Înainte vreme în Israel, când mergea cineva sa întrebe pe Dumnezeu, zicea a?a: "Hai la vazatorul!" Caci acela care astazi se nume?te prooroc, înainte se numea vazator.
1Sa 9:10  ?i a zis Saul catre sluga sa: "Bine zici tu; hai sa mergem!" ?i s-au dus în cetate, unde era omul lui Dumnezeu.
1Sa 9:11  Dar când se suiau ei la deal spre cetate, i-au întâmpinat ni?te fete care ie?isera sa aduca apa ?i le-au zis acestora: "Aici este vazatorul?"
1Sa 9:12  Iar acelea au raspuns ?i au zis: "Aici; iata-l înaintea ta; dar grabe?te ca el astazi a venit în cetate, pentru ca astazi poporul axe jertfe pe deal.
1Sa 9:13  Când ve?i ajunge în cetate, îl ve?i gasi pâna nu se duce pe acel deal la prânz; caci poporul nu începe sa manânce pâna nu vine el; pentru ca el le binecuvânteaza jertfa ?i dupa aceea începe sa manânce. Duce?i-va dar, ca-l ve?i apuca înca acasa".
1Sa 9:14  ?i s-au dus ei în cetate. Dar când au sosit în mijlocul ceta?ii, atunci iata ?i Samuel le ie?i înainte, ca sa se duca pe deal.
1Sa 9:15  Domnul însa cu o zi înainte de sosirea lui Saul, îi descoperise lui Samuel ?i-i zisese:
1Sa 9:16  "Mâine pe vremea asta voi trimite la tine pe un om din ?inutul lui Veniamin ?i tu îl vei unge cârmuitor al poporului Meu Israel; acela va izbavi pe poporul Meu din mâna Filistenilor, caci am cautat spre poporul Meu, deoarece strigatul lui a ajuns pâna la Mine".
1Sa 9:17  Deci, când a vazut Samuel pe Saul, atunci Domnul i-a zis: "Iata omul despre care ?i-am vorbit Eu. Acesta va cârmui pe poporul Meu!"
1Sa 9:18  ?i s-a apropiat Saul de Samuel la poarta ?i l-a întrebat: "Spune-mi unde este casa vazatorului?
1Sa 9:19  Iar Samuel a raspuns lui Saul ?i a zis: "Eu sunt vazatorul; mergi înaintea mea pe deal, ca ave?i sa prânzi?i cu mine astazi, iar diminea?a te voi lasa sa pleci; ?i tot ce ai pe inima, î?i voi spune.
1Sa 9:20  Iar de asinele care ?i s-au ratacit acum trei zile nu purta grija, caci s-au gasit. ?i cui sunt oare pastrate cele mai scumpe lucruri în Israel, daca nu ?ie ?i la toata casa tatalui tau?"
1Sa 9:21  Atunci a raspuns Saul ?i a zis: "Nu sunt eu oare fiul lui Veniamin, una din cele mai mici din semin?iile lui Israel? ?i familia mea oare nu este cea mai mica din toate familiile semin?iei lui Veniamin? De ce dar îmi vorbe?ti tu mie acestea?"
1Sa 9:22  ?i a luat Samuel pe Saul ?i pe sluga lui ?i i-a dus în casa ?i le-a dat locul cel dintâi între oaspe?i, care erau ca la treizeci de oameni.
1Sa 9:23  Dupa aceea a zis Samuel catre bucatar: "Da-mi partea pe care ?i-am dat-o ?i de care li-am zis: Pastreaz-o la tine!"
1Sa 9:24  ?i a luat bucatarul spata ?i cele ce erau cu ea ?i le-a pus înaintea lui Saul. Apoi a zis Samuel: "Iata aceasta este pastrata pentru tine; pune-?i-o dinainte ?i manânca, fiindca pentru vremea aceasta s-au pastrat acestea pentru tine, când am adunat poporul!" ?i a prânzit Saul cu Samuel în ziua aceea.
1Sa 9:25  Apoi s-au coborât de pe deal în cetate ?i a stat de vorba Samuel cu Saul în foi?orul de sus al casei, unde i s-a a?ternut lui Saul ?i a dormit.
1Sa 9:26  Iar diminea?a s-au sculat ei a?a: la ivirea zorilor a strigat Samuel pe Saul din foi?or ?i a zis: "Scoala ?i te voi petrece! ?i s-a sculat Saul ?i au ie?it amândoi din casa, el ?i Samuel.
1Sa 9:27  Iar daca au ajuns ei la marginea ceta?ii, a zis Samuel catre Saul: "Spune-i slugii sa treaca înaintea noastra"; ?i a plecat acela înainte. Apoi a zis: "Tu însa opre?te-te acum, ca am sa-?i descopar ceea ce a zis Dumnezeu".
1Sa 10:1  Atunci, luând Samuel vasul cel cu untdelemn, a turnat pe capul lui Saul ?i l-a sarutat, zicând: "Iata Domnul te unge pe tine cârmuitor al mo?tenirii Sale; vei domni peste poporul Domnului ?i-l vei izbavi din mâna vrajma?ilor celor dimprejurul lor. Iata care-?i va fi semnul ca Domnul te-a uns rege peste mo?tenirea Sa:
1Sa 10:2  Când vei pleca acum de la mine, vei întâlni doi oameni aproape de mormântul Rahilei, în hotarele lui Veniamin, la ?el?ah ?i aceia î?i vor spune: S-au gasit asinele dupa care ai umblat ?i le-ai cautat ?i iata tatal tau, uitând de asine, este nelini?tit pentru voi, zicând: "Ce este cu fiul meu?"
1Sa 10:3  Mergând apoi de acolo mai departe ?i ajungând la dumbrava Tabor, te vor întâmpina acolo trei oameni, care merg la Dumnezeu în Betel: unul duce trei iezi, altul duce trei pâini, iar al treilea duce un burduf cu vin.
1Sa 10:4  Aceia, dupa ce te vor saluta, î?i vor da doua pâini ?i tu le vei lua din mâinile lor.
1Sa 10:5  Dupa aceea vei ajunge la Ghibeea Elohim, unde se afla garda de paza a Filistenilor; acolo sunt capeteniile filistene. ?i când vei intra în cetate, vei întâlni o ceata de prooroci coborându-se de pe înal?ime, iar înaintea lor se cânta din psaltire ?i din timpan ?i din fluier ?i din harpa, iar ei proorocesc.
1Sa 10:6  Atunci va veni peste tine Duhul Domnului ?i vei prooroci ?i tu cu ei ?i te vei face alt om.
1Sa 10:7  Dupa ce se vor adeveri cu tine aceste semne, atunci sa faci ce vei putea, caci Dumnezeu este cu tine.
1Sa 10:8  Dar sa te duci înainte de mine în Ghilgal, unde am sa vin ?i eu la tine, ca sa aducem arderi de tot ?i jertfe de împacare. Sa a?tep?i ?apte zile, pâna voi veni la tine ?i atunci am sa-?i spun ce ai sa faci".
1Sa 10:9  Îndata ce ?i-a întors spatele Saul, ca sa plece de la Samuel, îi dadu Dumnezeu alta inima ?i s-au împlinit cu el toate acele semne în aceea?i zi.
1Sa 10:10  Iar daca au ajuns ei la Ghibeea, iata i-a întâmpinat o ceata de prooroci ?i a venit peste el Duhul lui Dumnezeu ?i a proorocit ?i el în mijlocul lor.
1Sa 10:11  To?i cei ce-l cuno?teau de mai înainte, vazând ca prooroce?te cu proorocii, vorbeau prin popor unul catre altul: "Ce s-a întâmplat cu fiul lui Chi?? Au doara ?i Saul este printre prooroci?"
1Sa 10:12  Iar unul din cei ce erau acolo a raspuns ?i a zis: "Dar tatal acelora cine este oare?" De atunci a ramas zicatoarea: "Au doara ?i Saul este printre prooroci?"
1Sa 10:13  Apoi a încetat el sa prooroceasca ?i s-a suit pe un deal.
1Sa 10:14  Atunci a zis unchiul lui Saul catre acesta ?i catre sluga lui: "Unde a?i fost voi?" ?i el a zis: "Am cautat asinele, dar, vazând ca nu le gasim, ne-am abatut pe la Samuel".
1Sa 10:15  ?i a zis unchiul lui Saul: "Spune-mi ce v-a spus Samuel?"
1Sa 10:16  Iar Saul a zis catre unchiul sau: "Ne-a spus ca asinele s-au gasit". Iar ceea ce îi spusese Samuel de domnie, nu i-a descoperit.
1Sa 10:17  Atunci a adunat Samuel poporul la Domnul, în Mi?pa,
1Sa 10:18  ?i a zis catre fiii lui Israel: "A?a zice Domnul Dumnezeul lui Israel: "Eu am scos pe Israel din Egipt ?i v-am izbavit din mâna Egiptenilor ?i din mâna tuturor împara?iilor care va apasau.
1Sa 10:19  Dar voi acum a?i lepadat pe Domnul Dumnezeul vostru, Care va scapa din toate necazurile ?i nevoile voastre ?i a?i zis catre El: "Pune rege peste noi!" Înfa?i?a?i-va dar înaintea Domnului, dupa semin?iile voastre ?i dupa familiile voastre!"
1Sa 10:20  ?i a poruncit Samuel tuturor semin?iilor lui Israel sa se apropie ?i a fost aratata semin?ia lui Veniamin.
1Sa 10:21  Apoi a poruncit sa se apropie familiile din semin?ia lui Veniamin ?i a ie?it la sor?i familia lui Matri; dupa aceea au fost adu?i barba?ii din familia lui Matri ?i a ie?it la sor?i Saul, fiul lui Chi? ?i l-au cautat, dar nu l-au gasit.
1Sa 10:22  ?i au întrebat iara?i pe Domnul: "Va veni el oare aici?" Iar Domnul a zis: "Iata-l, se ascunde printre lucruri".
1Sa 10:23  Atunci au alergat ?i l-au luat de acolo ?i el a stat în mijlocul poporului ?i poporul îi venea numai pâna la umeri.
1Sa 10:24  ?i a zis Samuel catre tot poporul: "Vede?i pe cine a ales Domnul? Asemenea lui nu este în tot poporul". Atunci tot poporul a strigat ?i a zis: "Sa traiasca regele!"
1Sa 10:25  ?i a în?irat Samuel poporului drepturile regelui, le-a scris în carte ?i le-a pus înaintea Domnului. Dupa aceea a dat drumul la tot poporul sa mearga fiecare la casa sa.
1Sa 10:26  De asemenea s-a dus ?i Saul la casa sa, în Ghibeea, ?i s-au dus cu el ?i vitejii a caror inima o atinsese Dumnezeu.
1Sa 10:27  Iar oamenii netrebnici ziceau: "Acesta oare ne va izbavi pe noi?" ?i-l dispre?uiau ?i nu i-au adus daruri. Dar el s-a facut ca nu-i aude.
1Sa 11:1  S-a întâmplat însa, dupa vreo luna, sa vina Naha? Amoniteanul ?i sa împresoare cetatea Iabe? din Galaad. Atunci to?i locuitorii din Iabe? au zis catre Naha?: "Încheie legamânt cu noi ?i noi î?i vom sluji ?ie".
1Sa 11:2  Dar Naha? Amoniteanul a zis catre ei: "Eu voi încheia cu voi legamânt ca sa se scoata fiecaruia din voi ochiul drept ?i prin aceasta sa se arunce necinste asupra întregului Israel".
1Sa 11:3  Atunci batrânii din Iabe? au zis catre el: "Da-ne vreme de ?apte zile, ca sa trimitem împuternici?i în toate hotarele lui Israel ?i de nu ne va ajuta nimeni, vom ie?i la tine".
1Sa 11:4  Au mers deci tinerii la Ghibeea lui Saul ?i au spus vorbele acestea în auzul poporului ?i tot poporul a ridicat glas ?i a plâns.
1Sa 11:5  Dar iata ca a venit Saul de la câmp, în urma boilor ?i a zis: "Ce are poporul de plânge?" ?i i s-au spus vorbele locuitorilor din Iabe?.
1Sa 11:6  Atunci s-a coborât Duhul lui Dumnezeu asupra lui Saul, când a auzit el cuvintele acestea ?i s-a aprins stra?nic mânia lui.
1Sa 11:7  ?i a luat el o pereche de boi, i-a junghiat, i-a taiat buca?i ?i a trimis în toate hotarele lui Israel prin împuternici?ii aceia, spunând ca a?a se va face cu boii aceluia care nu va merge dupa Saul ?i Samuel. Atunci a cazut frica Domnului peste popor ?i au ie?it to?i ca un singur om.
1Sa 11:8  ?i i-a cercetat Saul în Bezec ?i s-au gasit din fiii lui Israel trei sute de mii, iar din Iuda treizeci de mii de oameni.
1Sa 11:9  Atunci s-a spus solilor veni?i din Iabe?: "A?a sa spune?i locuitorilor din Iabe?ul Galaadului: ;Mâine, când va începe soarele sa încalzeasca, ajutorul va fi la voi". Au venit deci trimi?ii ?i au spus locuitorilor Iabe?ului ?i ace?tia s-au bucurat.
1Sa 11:10  Apoi au zis locuitorii Iabe?ului catre Naha?: "Mâine vom ie?i la voi ?i ve?i face cu noi cum va va place".
1Sa 11:11  A doua zi Saul împar?i poporul în trei tabere. ?i acestea au patruns dis-de-diminea?a în tabara Amoni?ilor ?i i-a macelarit pâna la sosirea ar?i?ei zilei; iar câ?i au ramas s-au risipit de n-au ramas doi la un loc.
1Sa 11:12  Atunci poporul a zis catre Samuel: "Cine a zis: Saul oare are sa domneasca peste noi? Da?i-ne pe ace?ti nelegiui?i ?i-i vom omorî!"
1Sa 11:13  Saul însa a zis: "Astazi nu trebuie sa fie ucis nimeni, caci astazi Domnul a savâr?it izbavirea în Israel".
1Sa 11:14  Iar Samuel a zis catre popor: "Sa mergem la Ghilgal ?i sa începem acolo noua domnie!"
1Sa 11:15  ?i s-a dus tot poporul la Ghilgal ?i au pus acolo pe Saul rege înaintea Domnului în Ghilgal ?i au adus acolo jertfe de împacare înaintea Domnului. ?i s-a veselit foarte Saul acolo ?i to?i Israeli?ii.
1Sa 12:1  A zis Samuel catre tot poporul: "Iata eu am ascultat glasul vostru în toate câte mi-a?i grait ?i am pus rege peste voi.
1Sa 12:2  Iata regele umbla înaintea voastra, iar eu am îmbatrânit ?i am încarun?it; fiii mei sunt cu voi ?i eu am umblat înaintea voastra din tinere?ile mele pâna acum.
1Sa 12:3  Iata-ma, marturisi?i asupra mea înaintea Domnului ?i a unsului Lui, de am luat cuiva boul, de am luat cuiva asinul, de am asuprit pe cineva ?i de am apasat pe cineva; de am luat de la cineva mita ?i am închis ochii la judecata lui, va voi despagubi".
1Sa 12:4  ?i au raspuns to?i: "Tu nu ne-ai nedrepta?it, nici nu ne-ai asuprit, nici nu ai luat nimic de la nimeni".
1Sa 12:5  Atunci el a zis: "Martor ne este Domnul ?i martor este unsul Lui în ziua aceasta, ca voi n-a?i gasit nimic asupra mea!" Iar ei au zis: "Martor!"
1Sa 12:6  Apoi a zis Samuel catre popor: "Martor este Domnul, Cel ce a pus pe Moise ?i pe Aaron ?i Care a scos pe parin?ii vo?tri din ?ara Egiptului.
1Sa 12:7  Acum însa veni?i, ca eu am sa ma judec cu voi înaintea Domnului pentru toate binefacerile pe care le-a facut El voua ?i parin?ilor vo?tri.
1Sa 12:8  Când a venit Iacov în Egipt ?i parin?ii vo?tri au strigat catre Domnul, atunci Domnul a trimis pe Moise ?i pe Aaron ?i au scos ei pe parin?ii vo?tri din Egipt ?i i-au stramutat în locul acesta.
1Sa 12:9  Dar ei au uitat pe Domnul Dumnezeul lor ?i El i-a dat în mâinile lui Sisera, capetenia o?tirilor Ha?orului, în mâinile Filistenilor ?i în mâinile regelui Moabului, care s-au razboit împotriva lor.
1Sa 12:10  Însa când au strigat ei catre Domnul ?i au zis: "Am pacatuit, parasind pe Domnul ?i apucându-ne sa slujim baalilor ?i astartelor; acum însa izbave?te-ne din mâinile vrajma?ilor, ?i-?i vom sluji ?ie",
1Sa 12:11  Atunci a trimis Domnul pe Ierubaal, pe Barac, pe Ieftae ?i pe Samuel ?i v-a izbavit din mâinile vrajma?ilor vo?tri celor din jurul vostru ?i a?i trait în pace.
1Sa 12:12  Iar când a?i vazut ca Naha?, regele Amoni?ilor, vine împotriva voastra, a?i zis catre mine: "Nu, ci sa domneasca peste noi un rege!", de?i peste voi împara?ea Domnul Dumnezeul vostru.
1Sa 12:13  A?adar iata regele pe care l-a?i cerut; iata Domnul a pus peste voi rege.
1Sa 12:14  De va ve?i teme de Domnul, de-I ve?i sluji Lui ?i de ve?i asculta glasul Lui, de nu va ve?i împotrivi poruncilor Domnului ?i de ve?i umbla ?i voi ?i regele care domne?te peste voi în urma Domnului Dumnezeului vostru, atunci mâna Domnului nu va fi împotriva voastra.
1Sa 12:15  Iar de nu ve?i asculta glasul Domnului, ci va ve?i împotrivi poruncilor Lui, atunci mina Domnului va fi împotriva voastra, cum a fost împotriva parin?ilor vo?tri.
1Sa 12:16  Acum scula?i-va ?i privi?i la lucrul cel mare pe care-l va face Domnul înaintea ochilor vo?tri:
1Sa 12:17  Nu e acum oare seceri?ul grâului? Dar eu voi striga catre Domnul ?i El va trimite trasnet ?i ploaie ?i ve?i afla ?i ve?i vedea cât de mare este pacatul pe care l-a?i facut voi înaintea ochilor Domnului, când a?i cerut rege".
1Sa 12:18  ?i a strigat Samuel catre Domnul ?i a trimis Domnul tunete ?i ploaie în ziua aceea; ?i teama de Domnul ?i de Samuel  cuprins tot poporul.
1Sa 12:19  ?i a zis tot poporul catre Samuel: "Roaga-te pentru robii tai înaintea Domnului Dumnezeului tau, ca sa nu murim; caci la toate celelalte pacate ale noastre am mai adaugat un pacat, când am cerut rege".
1Sa 12:20  Iar Samuel a raspuns poporului: "Nu va teme?i. Pacatul acesta este facut de voi, dar voi sa nu va departa?i de Domnul, ci sa-I sluji?i Lui cu toata inima.
1Sa 12:21  Sa nu apuca?i dupa dumnezeii cei de nimic, care nu aduc folos, nici nu izbavesc, pentru ca sunt nimic.
1Sa 12:22  Domnul însa nu va lasa pe poporul Sau pentru numele Sau cel mare, caci Domnul a binevoit sa va aleaga pe voi ca popor al Sau.
1Sa 12:23  ?i eu de asemenea nu-mi voi îngadui sa fac înaintea Domnului pacatul de a înceta sa ma rog pentru voi ?i va voi pova?ui pe cai bune ?i drepte. Decât numai sa va teme?i de Domnul ?i sa-I sluji?i Lui cu adevarat, din toata inima voastra,
1Sa 12:24  Caci vede?i ce lucruri minunate a facut El cu voi.
1Sa 12:25  Iar de ve?i face rau, atunci ve?i pieri ?i voi ?i regele vostru".
1Sa 13:1  Se împlinise un an de când fusese facut Saul rege ?i acum domnea în al doilea an peste Israel, când ?i-a ales el trei mii de Israeli?i:
1Sa 13:2  Doua mii erau cu Saul la Micma? ?i pe Muntele Betelului ?i o mie era cu Ionatan în Ghibeea lui Veniamin; iar celalalt popor era lasat de el pe la casele lor.
1Sa 13:3  ?i a sfarâmat Ionatan tabara de paza a Filistenilor, care era în Gheba. ?i au auzit de aceasta Filistenii, iar Saul a sunat din trâmbi?a în toata ?ara, strigând: "Sa auda Evreii!"
1Sa 13:4  ?i dupa ce a auzit tot Israelul ca Saul a sfarâmat tabara de paza a Filistenilor ?i ca Filistenii au urât pe Israeli?i, s-a adunat poporul la Saul în Ghilgal.
1Sa 13:5  Filistenii s-au adunat ?i ei, ca sa se razboiasca împotriva Israeli?ilor: treizeci de mii de care, ?ase mii de calare?i ?i popor mult ca nisipul de pe malul marii; ?i au venit ?i ?i-au pus tabara în Micma?, în partea de rasarit a Bet-Avenului.
1Sa 13:6  Iar Israeli?ii, vazându-se în primejdie, pentru ca poporul era strâmtorat, s-au ascuns prin pe?teri, prin stufi?uri, printre stânci, prin turnuri ?i prin ?an?uri;
1Sa 13:7  Ba unii din Israeli?i au trecut peste Iordan în pamântul lui Gad ?i al Galaadului. Saul însa se afla înca tot în Galaad ?i tot poporul care era cu el era cuprins de frica.
1Sa 13:8  Acolo a a?teptat el opt zile, pâna la vremea hotarâta de Samuel; dar Samuel nu mai venea la Ghilgal. Acum poporul începuse sa fuga de la el.
1Sa 13:9  De aceea a zis Saul: "Aduce?i-mi cele pentru jertfa de cura?ire". ?i a adus ardere de tot.
1Sa 13:10  Dar nu apucase el bine sa aduca ardere de tot, ?i iata veni ?i Samuel. Saul ie?i înaintea lui ca sa-l întâmpine ?i sa-i ureze de bine.
1Sa 13:11  Samuel însa i-a zis: "Ce ai facut?" ?i i-a raspuns Saul: "Am vazut ca poporul se împra?tie ?i fuge de la mine ?i tu nu ai venit la vremea hotarâta, iar Filistenii s-au adunat la Micmas.
1Sa 13:12  Atunci am socotit ca au sa navaleasca Filistenii asupra mea în Ghilgal ?i eu n-am întrebat înca pe Domnul; de aceea m-am hotarât sa aduc ardere de tot".
1Sa 13:13  Iar Samuel i-a zis: "Rau ai facut, ca nu ai împlinit porunca Domnului Dumnezeului tau care ?i s-a dat, caci acum ar fi întarit Domnul domnia ta peste Israel de-a pururi.
1Sa 13:14  Acum însa nu va dura domnia ta; Domnul Î?i va gasi un barbat dupa inima Sa ?i-i va porunci Domnul sa fie conducatorul poporului Sau, deoarece tu nu ai împlinit ceea ce ?i s-a poruncit de la Domnul".
1Sa 13:15  Apoi s-a sculat Samuel ?i s-a dus din Ghilgal în Ghibeea lui Veniamin; iar oamenii care au ramas s-au dus cu Saul în întâmpinarea taberei vrajma?ului, care a navalit asupra lor când mergeau ei de la Ghilgal la Ghibeea lui Veniamin. ?i a numarat Saul oamenii care erau cu el ?i s-au gasit pâna la ?ase sute de barba?i.
1Sa 13:16  Apoi Saul cu fiul sau Ionatan ?i cu oamenii care erau cu dân?ii au ?ezut în Ghibeea ?i au plâns; iar Filistenii stateau în tabara la Micma?.
1Sa 13:17  ?i au ie?it din tabara Filistenilor trei cete sa pustiiasca ?ara: una a plecat pe drumul spre Ofra din ?inutul ?ual;
1Sa 13:18  A doua ceata a plecat pe drumul Bethoronului, iar a treia s-a îndreptat pe calea ce duce spre hotar, în fala vaii ?eboim, spre pustie.
1Sa 13:19  Fierar nu era în toata Iara lui Israel, caci Filistenii se temeau ca nu cumva Israeli?ii sa-?i faca sabii ?i suli?e.
1Sa 13:20  Trebuia deci sa se duca to?i Israeli?ii la Filisteni ca sa-?i ascuta fiarele plugurilor ?i sapelor lor, topoarele ?i securile lor,
1Sa 13:21  Când se facea vreo ?tirbitura la ascu?i?ul fiarelor de plug, al sapelor, al topoarelor ?i securilor lor, sau când trebuia sa îndrepte vreun corn de furca.
1Sa 13:22  De aceea în timpul razboiului de la Micma?, tot poporul care era cu Saul ?i Ionatan nu avea nici sabii, nici suli?e; numai Saul ?i Ionatan, fiul sau, aveau.
1Sa 13:23  ?i a venit ceata întâi de Filisteni ?i s-a a?ezat la trecatoarea Micma?.
1Sa 14:1  Într-o zi Ionatan, fiul lui Saul, a zis catre tânarul care purta armele sale: "Hai sa trecem la ceata Filistenilor, care e dincolo", iar tatalui sau nu i-a spus de aceasta.
1Sa 14:2  Saul însa se afla la marginea Ghibeii, sub rodiul cel din Migron; cu el era o ceata ca de ?ase sute de oameni.
1Sa 14:3  ?i Ahia, fiul lui Ahituv, fratele lui Icabod, fiul lui Finees, feciorul lui Eli, preotul Domnului din ?ilo, purta efodul. Poporul însa nu ?tia ca Ionatan plecase.
1Sa 14:4  Strâmtoarea prin care cauta Ionatan sa se strecoare spre ceata Filistenilor trecea printre ni?te stânci ascu?ite; numele uneia era Bo?e? ?i numele alteia era Sene;
1Sa 14:5  O stânca se ridica la miazanoapte, spre Micma?; cealalta la miazazi, spre Ghibeea.
1Sa 14:6  Atunci a zis Ionatan catre tânarul care purta armele sale: "Hai sa trecem la ace?ti netaia?i împrejur; poate ne va ajuta Domnul, caci pentru Domnul nu e greu sa izbaveasca ?i prin pu?ini, ca ?i prin mul?i".
1Sa 14:7  Cel ce purta armele a raspuns: "Fa tot ce-?i spune inima; mergi unde vrei ?i iata eu sunt cu tine".
1Sa 14:8  Iar Ionatan a zis: "Bine, atunci hai la acei oameni ?i sa ne aratam catre ei.
1Sa 14:9  Daca ei ne vor zice: Sta?i pâna vom veni la voi, vom ramâne pe loc ?i nu ne vom sui la ei;
1Sa 14:10  Iar de ne vor zice: Sui?i-va la noi, atunci ne vom sui, caci Domnul i-a dat în mâinile noastre ?i acesta va fi semnul pentru noi".
1Sa 14:11  Când au aparut ei amândoi în vazul cetei Filistenilor, atunci Filistenii au zis: "Iata Evreii ies din pe?terile în care s-au ascuns".
1Sa 14:12  Apoi oamenii din ceata Filistenilor au strigat catre Ionatan ?i catre cel ce-i purta armele sale ?i a zis: "Sui?i-va la noi, ca avem sa va spunem ceva". Atunci Ionatan a zis catre purtatorul lui de arme: "Vino dupa mine, ca Domnul i-a dat în mâinile lui Israel".
1Sa 14:13  ?i a început Ionatan sa se urce, ca?arându-se cu mâinile ?i cu picioarele, ?i cel ce-i purta armele se ca?ara dupa el. ?i au cazut Filistenii înaintea lui, iar cel ce purta armele în urma lui le da cele din urma lovituri.
1Sa 14:14  ?i au cazut în acest atac dintâi, savâr?it de Ionatan ?i de purtatorul lui de arme ca la douazeci de oameni, pe o bucata de loc cât o jumatate de pogon, atât cât pot sa are doi boi într-o zi.
1Sa 14:15  Atunci s-a facut învalma?eala în tabara din câmp ?i în tot poporul; rândurile dinainte ce pustiau ?ara s-au umplut de frica ?i nu voiau sa se lupte; toata ?ara s-a cutremurat ?i i-a cuprins frica mare de la Domnul.
1Sa 14:16  Atunci strajile lui Saul din Ghibeea lui Veniamin au vazut ca mul?imea se împra?tie ?i fuge încolo ?i încoace.
1Sa 14:17  Saul a zis catre oamenii care erau cu el: "Cauta?i ?i vede?i care din ai no?tri a plecat". ?i au cautat ?i iata nu erau Ionatan ?i purtatorul lui de arme.
1Sa 14:18  A zis Saul catre Ahia: "Adu chivotul lui Dumnezeu", caci chivotul lui Dumnezeu în vremea aceea era cu fiii lui Israel.
1Sa 14:19  ?i pe când graia înca Saul cu preotul, tulburarea din tabara Filistenilor se la?ea ?i cre?tea din ce în ce mai mult. Atunci Saul a zis catre preot: "Încruci?eaza-?i mâinile!"
1Sa 14:20  ?i a strigat Saul ?i tot poporul care era cu el ?i au venit la locul luptei ?i iata acolo sabia fiecaruia era ridicata asupra aproapelui sau ?i tulburarea era foarte mare.
1Sa 14:21  Atunci ?i Evreii, care mai dinainte erau la Filisteni ?i care umblau pretutindeni în tabara lor, s-au unit cu Israeli?ii cei ce erau cu Saul ?i Ionatan;
1Sa 14:22  ?i to?i Israeli?ii care se ascunsesera în muntele lui Efraim, auzind ca Filistenii au fugit, s-au unit de asemenea cu ai lor la lupta.
1Sa 14:23  ?i a izbavit Domnul în ziua aceea pe Israel; lupta se întinsese însa pâna la Bet-Aven. ?i tot poporul care era cu Saul era ca la zece mii de barba?i; ?i se da razboi în toate ceta?ile din muntele lui Efraim.
1Sa 14:24  În ziua aceea s-au obosit oamenii din Israel. Iar Saul a pus poporul sa jure, zicând: "Blestemat tot cel ce va mânca pâna seara, pâna când eu îmi voi razbuna pe vrajma?ii mei". De aceea nimeni din popor n-a gustat hrana,
1Sa 14:25  Ci s-a dus tot poporul în padure, ?i acolo intr-o poiana era miere.
1Sa 14:26  ?i a intrat poporul în padure ?i a zis: "Iata curge miere". Dar nimeni nu ?i-a dus mâna spre gura sa, caci poporul se temea de blestem.
1Sa 14:27  Ionatan însa nu auzise de juramântul pe care-l pusese tatal sau pe popor; ?i, întinzând vârful toiagului ce-l avea în mâna, l-a muiat într-un fagure de miere ?i, întorcându-l cu mâna spre gura sa, i s-au luminat ochii.
1Sa 14:28  Atunci unul din popor i-a spus: "Tatal tau a pus juramânt asupra poporului, zicând: Blestemat sa fie cel ce va gusta astazi hrana, ?i de aceea poporul e istovit".
1Sa 14:29  Dar Ionatan a zis: "Tatal meu a tulburat ?ara. Iata mie mi s-au luminat ochii când am gustat pu?in din aceasta miere.
1Sa 14:30  De ar fi mâncat astazi poporul din prazile ce s-au gasit la vrajma?ii lor, oare n-ar fi fost mai mare înfrângerea Filistenilor?"
1Sa 14:31  În ziua aceea au batut pe Filisteni de la Micma? pâna la Aialon, dar poporul se istovise peste masura.
1Sa 14:32  ?i s-a aruncat poporul asupra prazilor ?i a luat oi ?i boi ?i vitei ?i au junghiat pe pamânt ?i au mâncat oamenii carne cu sânge.
1Sa 14:33  ?i i s-a vestit lui Saul, zicându-i-se: "Iata poporul a gre?it înaintea Domnului, mâncând carne cu sânge". Atunci Saul a zis: "Voi a?i gre?it. Pravali?i acum aici spre mine o piatra mare".
1Sa 14:34  Apoi a zis Saul: "Trece?i prin popor ?i zice?i-i: "Sa-?i aduca fiecare la mine boul sau ?i oaia ?i sa înjunghia?i aici ?i sa mânca?i, ?i sa nu gre?i?i înaintea Domnului mâncând carne cu sânge". ?i to?i din popor ?i-au adus cu mâna sa fiecare boul sau, noaptea, ?i l-au junghiat acolo.
1Sa 14:35  ?i a zidit Saul jertfelnic Domnului; acesta a fost cel dintâi jertfelnic facut de el Domnului.
1Sa 14:36  Atunci a zis Saul: "Hai dupa Filisteni noaptea aceasta ?i sa-i pradam pâna diminea?a ?i sa nu lasam din ei nici un om". Iar ei au zis: "Fa tot ce este bine în ochii tai". Preotul însa a zis: "Sa ne apropiem aici de Dumnezeu!"
1Sa 14:37  ?i a întrebat Saul pe Dumnezeu: "Sa merg eu oare dupa Filisteni? Îi vei da, oare, pe ei în mâinile lui Israel?" Dar El nu i-a raspuns în ziua aceea.
1Sa 14:38  Atunci Saul a zis: "Sa vina aici toate capeteniile poporului ?i voi cauta sa aflu asupra cui este pacatul acum.
1Sa 14:39  Ca viu este Domnul, Cel ce a izbavit pe Israel, ca de va fi chiar asupra lui Ionatan, fiul meu, apoi ?i el va muri. Dar nimeni din tot poporul nu i-a raspuns.
1Sa 14:40  ?i a zis Saul catre to?i Israeli?ii: "Sta?i voi de o parte iar eu ?i Ionatan, fiul meu, vom sta de alta parte". ?i a raspuns poporul lui Saul: "Fa ce este bine în ochii tai!" Apoi Saul a zis: "Doamne, Dumnezeul lui Israel, pentru ce n-ai raspuns Tu acum robului Tau? Daca vina este asupra mea sau asupra fiului meu Ionatan, Domnul Dumnezeul lui Israel, fa sa iasa Urim, daca vina este asupra poporului Tau Israel, fa sa iasa Tumim.
1Sa 14:41  ?i sor?ul a cazut pe Saul ?i pe Ionatan, iar poporul a ie?it drept.
1Sa 14:42  Atunci Saul a zis: "Arunca?i sor?i asupra mea ?i asupra lui Ionatan, fiul meu, ?i pe cine-l va arata Domnul, acela sa moara". Poporul însa a zis: "Sa nu fie a?a!" Dar Saul a staruit în hotarârea sa, ?i au aruncat sor?i asupra sa ?i a lui Ionatan, fiul lui, ?i a cazut sor?ul pe Ionatan.
1Sa 14:43  Atunci Saul a zis catre Ionatan: "Spune-mi ce ai facut?" Iar Ionatan i-a raspuns ?i a zis: "Doar am gustat pu?ina miere cu vârful toiagului pe care îl aveam în mâna ?i iata trebuie sa mor".
1Sa 14:44  Iar Saul a zis: "A?a ?i a?a sa-mi faca mie Dumnezeu ?i înca ?i mai mult sa-mi faca, daca nu vei muri astazi, Ionatane!"
1Sa 14:45  Dar poporul a zis catre Saul: "Sa moara oare Ionatan care a adus o izbavire a?a de minunata poporului! Sa nu fie aceasta! Viu este Domnul, nici un par din capul lui nu va cadea, pentru ca el cu Dumnezeu a lucrat astazi!" ?i a izbavit poporul pe Ionatan ?i el n-a murit.
1Sa 14:46  ?i s-a întors Saul de la urmarirea Filistenilor, iar Filistenii s-au dus la locul lor.
1Sa 14:47  Astfel ?i-a întarit Saul domnia sa peste Israel ?i s-a luptat cu to?i vrajma?ii de primprejur: cu Moab ?i Amoni?ii, cu Edom, cu regii din ?oba ?i cu Filistenii ?i pretutindeni, împotriva oricui a mers, a avut izbânda.
1Sa 14:48  ?i ?i-a rânduit oaste, a batut pe Amalec ?i a izbavit pe Israel din mâinile jefuitorilor sai.
1Sa 14:49  Fiii lui Saul erau: Ionatan, Iesui ?i Melchi?ua; iar numele celor doua fiice ale sale erau: Merob, numele celei mai mari, ?i Micol, numele celei mai mici.
1Sa 14:50  Iar numele femeii lui Saul era Ahinoam, fiica lui Ahimaa?, iar numele capeteniei o?tirii lui era Abner, fiul lui Ner, unchiul lui Saul.
1Sa 14:51  Chi? era tatal lui Saul, Ner era tatal lui Abner, fiul lui Abiel.
1Sa 14:52  În tot timpul domniei lui Saul s-au dus razboaie crâncene cu Filistenii. Când Saul vedea vreun om voinic ?i razboinic, îl lua la el.
1Sa 15:1  În vremea aceea a zis Samuel catre Saul: "Domnul m-a trimis sa te ung rege peste poporul Lui, peste Israel; acum asculta glasul Domnului.
1Sa 15:2  A?a zice Domnul Savaot: Adusu-Mi-am aminte de cele ce a facut Amalec lui Israel, cum i s-a împotrivit în cale, când venea din Egipt.
1Sa 15:3  Mergi acum ?i bate pe Amalec ?i pe Ierim ?i nimice?te toate ale lui. Sa nu iei pentru tine nimic de la ei, ci nimice?te ?i da blestemului toate câte are. Sa nu-i cru?i, ci sa dai mor?ii de la barbat pâna la femeie, de la tânar pâna la pruncul de sân, de la bou pâna la oaie, de la camila pâna la asin".
1Sa 15:4  Atunci a adunat Saul poporul ?i l-a numarat în Telaim ?i s-au aflat doua sute de mii Israeli?i pede?tri ?i zece mii din semin?ia lui Iuda.
1Sa 15:5  ?i a mers Saul pâna la cetatea lui Amalec ?i a pus oameni la pânda în vale.
1Sa 15:6  Apoi Saul a zis catre Chenei: "Merge?i de va despar?i?i ?i ie?i?i din mijlocul lui Amalec, ca sa nu va pierd împreuna cu el, caci a?i aratat bunavoin?a catre to?i Israeli?ii, când veneau ei din Egipt". ?i Cheneii s-au despar?it de Amalec.
1Sa 15:7  Atunci a lovit Saul pe Amalec ?i l-a batut de la Havila pâna la ?ur, care este în fa?a Egiptului; iar pe Agag, regele lui Amalec, l-a prins " iu ?i pe popor l-a ucis tot cu sabia ?i a ucis ?i pe Ierim.
1Sa 15:8  Dar Saul ?i poporul au cru?at pe Agag, pe cele mai bune din oi ?i din vitele cornute, mieii îngra?a?i ?i tot ce era bun ?i n-a vrut sa le piarda;
1Sa 15:9  Iar toate lucrurile neînsemnate ?i rele le-au pierdut.
1Sa 15:10  Atunci a fost cuvântul Domnului catre Samuel astfel: "Îmi pare rau ca am pus pe Saul rege, caci el s-a abatut de la Mine ?i cuvântul Meu nu l-a împlinit".
1Sa 15:11  ?i s-a întristat Samuel ?i a strigat catre Domnul toata noaptea.
1Sa 15:12  Iar a doua zi dis-de-diminea?a, sculându-se, a ie?it în întâmpinarea lui Saul. ?i i s-a spus lui Samuel ca Saul a fost pe Carmel ?i ?i-a ridicat acolo semn de aducere aminte, iar de acolo s-a întors ?i s-a coborât la Ghilgal.
1Sa 15:13  Iar dupa ce a ajuns Samuel la Saul, Saul i-a spus: "Iata am împlinit cuvântul tau!"
1Sa 15:14  Samuel a zis: "Dar ce este acest behait de oi ce-mi ajunge la urechi ?i acel muget de boi pe care-l aud?"
1Sa 15:15  Iar Saul a raspuns: "Le-am adus de la Amalec, de vreme ce poporul a cru?at pe cele mai bune din oi ?i din vitele mari, ca sa fie aduse jertfa Domnului Dumnezeului tau. Iar pe celelalte le-a nimicit".
1Sa 15:16  Samuel a zis catre Saul: "Îngaduie-mi sa-?i spun ce mi-a spus Domnul asta-noapte". Iar Saul a zis: "Spune!"
1Sa 15:17  ?i a zis Samuel: "Când erai tu mic în ochii tai, n-ai ajuns tu oare capetenia semin?iilor lui Israel ?i Domnul te-a uns rege peste Israel?
1Sa 15:18  Apoi te-a trimis Domnul la drum, zicând: Mergi ?i da junghierii pe Amaleci?ii cei necredincio?i ?i lupta împotriva lor pâna îi vei stârpi.
1Sa 15:19  Pentru ce n-ai ascultat glasul Domnului, ?i te-ai aruncat asupra prazii ?i ai facut rau în ochii Domnului?"
1Sa 15:20  Iar Saul a zis catre Samuel: "Eu am ascultat glasul Domnului ?i am plecat la drum încotro m-a trimis Domnul ?i am adus pe Agag, regele amalecit, iar pe Amalec l-am pierdut;
1Sa 15:21  Poporul însa a luat din prazi, din oi ?i din vite, a luat cele mai bune din cele afierosite ca sa le aduca jertfa Domnului Dumnezeului tau în Ghilgal".
1Sa 15:22  A raspuns Samuel: "Au doara arderile de tot ?i jertfele sunt tot a?a de placute Domnului, ca ?i ascultarea glasului Domnului? Ascultarea este mai buna decât jertfa ?i supunerea mai buna decât grasimea berbecilor.
1Sa 15:23  Caci nesupunerea este un pacat la fel cu vrajitoria ?i împotrivirea este la fel cu închinarea la idoli. Pentru ca ai lepadat cuvântul Domnului, ?i El te-a lepadat, ca sa nu mai fii rege peste Israel".
1Sa 15:24  Atunci Saul a zis catre Samuel: "Am pacatuit, calcând porunca Domnului ?i cuvântul tau; dar m-am temut de popor ?i am ascultat glasul lui.
1Sa 15:25  Ridica dar pacatul de pe mine ?i întoarce-te cu mine, ca sa ma închin Domnului Dumnezeului tau".
1Sa 15:26  Iar Samuel a raspuns lui Saul: "Nu ma voi întoarce cu tine, pentru ca ai lepadat cuvântul Domnului ?i Domnul te-a lepadat pe tine, ca sa nu mai fii rege peste Israel".
1Sa 15:27  Apoi Samuel s-a întors sa plece. Dar Saul s-a apropiat de poala hainei lui ?i a rupt-o.
1Sa 15:28  Atunci Samuel a zis: "Astazi a rupt Domnul regatul lui Israel de la tine ?i l-a dat altuia care este mai bun decât tine,
1Sa 15:29  ?i nu va spune neadevar Cel ce este taria lui Israel ?i nu Se va cai, caci El nu este om ca sa Se caiasca".
1Sa 15:30  Zis-a Saul: "Am gre?it; dar da-mi acum cinste înaintea batrânilor poporului meu ?i înaintea lui Israel ?i te întoarce cu mine ?i eu ma voi închina Domnului Dumnezeului tau".
1Sa 15:31  ?i s-a întors Samuel dupa Saul ?i s-a închinat Saul Domnului.
1Sa 15:32  Apoi Samuel a zis: "Adu la mine pe Agag, regele amalecit"; ?i s-a apropiat de el Agag, tremurând, ?i a zis: "De buna seama amaraciunea mor?ii a trecut".
1Sa 15:33  Samuel însa i-a raspuns: "Precum sabia ta a lipsit pe mame de copiii lor, a?a ?i mama ta sa fie între femei lipsita de fiu". ?i a taiat Samuel pe Agag înaintea Domnului în Ghilgal.
1Sa 15:34  Apoi s-a dus Samuel la Rama, iar Saul s-a dus la casa sa în Ghibeea lui Saul.
1Sa 15:35  ?i nu s-a mai vazut Samuel cu Saul pâna în ziua mor?ii sale. Dar s-a întristat Samuel pentru Saul, ca se caise Domnul pentru ca-l facuse rege peste Israel.
1Sa 16:1  Domnul a zis catre Samuel: "Pâna când te vei tângui tu pentru Saul, pe care l-am lepadat, ca sa nu mai fie rege peste Israel? Umple cornul tau cu mir ?i du-te, ca te trimit la Iesei Betleemitul, caci dintre fiii lui Mi-am ales rege".
1Sa 16:2  Samuel a zis: "Cum sa ma duc? Va auzi Saul ?i ma va ucide". Iar Domnul a zis: "Ia cu tine o junca din cireada ?i zi: Am venit sa aduc jertfa Domnului.
1Sa 16:3  ?i cheama pe Iesei ?i pe fiii lui la jertfa, ?i Eu î?i voi arata ce sa faci ?i-Mi vei unge pe acela pe care î?i voi spune Eu".
1Sa 16:4  ?i a facut Samuel a?a, cum i-a spus Domnul. ?i când a sosit el la Betleem, batrânii poporului, tremurând, i-au ie?it în întâmpinare ?i au zis: "Cu pace este venirea ta, vazatorule?"
1Sa 16:5  Iar el a raspuns: "Cu pace. Am venit sa aduc jertfa Domnului; sfin?i?i-va ?i veni?i cu mine sa aducem jertfa!" ?i a sfin?it pe Iesei ?i pe fiii lui ?i i-a chemat la jertfa.
1Sa 16:6  Iar dupa ce au venit ei, vazând el pe Eliab, a zis: "De buna seama, acesta este înaintea Domnului unsul Lui".
1Sa 16:7  Dar Domnul a zis catre Samuel: "Nu te uita la înfa?i?area lui ?i la înal?imea staturii lui; Eu nu Ma uit ca omul; caci omul se uita la fa?a, iar Domnul se uita la inima".
1Sa 16:8  Apoi a chemat Iesei pe Aminadab ?i l-a dus la Samuel, iar Samuel a zis: "Nici pe acesta nu l-a ales Domnul".
1Sa 16:9  Dupa aceea a adus Iesei pe ?ama, ?i Samuel a zis: "Nici pe acesta nu l-a ales Domnul".
1Sa 16:10  ?i a?a a adus Iesei pe ?apte din fiii sai, dar Samuel a zis catre Iesei: "Pe nici unul din ace?tia nu l-a ales Domnul!"
1Sa 16:11  Dupa aceea a zis Samuel catre Iesei: "Oare to?i fiii tai sunt aici?" Iar Iesei a raspuns: "Mai am unul mai mic. Acela pa?te oile". A zis Samuel: "Trimite sa-l aduca, pentru ca nu vom ?edea sa prânzim pâna nu vine acela".
1Sa 16:12  ?i a trimis Iesei ?i l-au adus. Acela era balan, cu ochi frumo?i ?i placut la fa?a. Atunci Domnul a zis: "Scoala de-l unge, caci acesta este!"
1Sa 16:13  ?i a luat Samuel cornul cu mir ?i l-a uns în mijlocul fra?ilor lui, ?i a odihnit Duhul Domnului asupra lui David din ziua aceea ?i dupa aceea. Iar Samuel s-a sculat ?i a plecat la Rama.
1Sa 16:14  Atunci s-a departat de la Saul Duhul Domnului ?i-l tulbura un duh rau, trimis de Domnul.
1Sa 16:15  ?i au zis slugile lui Saul: "Iata un duh rau trimis de Domnul te tulbura.
1Sa 16:16  Sa porunceasca dar domnul nostru slugilor sale care sunt înaintea ta ?i sa caute un om iscusit lâ cântarea din harpa, ?i când va veni asupra ta duhul cel rau trimis de la Dumnezeu, atunci acela, cântând cu mâna sa, te va lini?ti".
1Sa 16:17  ?i a raspuns Saul slugilor sale: "Cauta?i-mi un om care cânta bine ?i mi-l aduce?i".
1Sa 16:18  Atunci unul din slujitorii lui a zis: "Iata eu am vazut la Iesei Betleemitul un fiu care ?tie sa cânte, om voinic ?i razboinic, priceput la vorba ?i barbat chipe? ?i Domnul este cu el".
1Sa 16:19  A trimis deci Saul vestitori la Iesei ?i i-a spus: "Trimite la mine pe David, fiul tau cel de la turma".
1Sa 16:20  ?i a luat Iesei un asin încarcat cu pâine ?i un burduf cu vin ?i un ied ?i le-a trimis cu David, fiul sau, la Saul.
1Sa 16:21  ?i a venit David la Saul, s-a înfa?i?at înaintea lui ?i a placut acestuia foarte mult ?i l-a facut purtatorul sau de arme.
1Sa 16:22  Dupa aceea a trimis Saul sa i se spuna lui Iesei: "Lasa pe David sa slujeasca la mine, ca a aflat el bunavoin?a în ochii mei! "
1Sa 16:23  Iar când duhul cel trimis de Dumnezeu era peste Saul, David, luând harpa, cânta ?i lui Saul îi era mai u?or ?i mai bine ?i duhul cel rau se departa de el.
1Sa 17:1  În vremea aceea Filistenii ?i-au strâns o?tile pentru razboi ?i au tabarât la Soco cel din Iuda ?i ?i-au a?ezat tabara între Soco ?i Azeca, la Efes-Damim.
1Sa 17:2  Iar Saul cu Israeli?ii s-au adunat ?i ?i-au a?ezat tabara la Valea Stejarului ?i s-au pregatit de lupta cu Filistenii.
1Sa 17:3  Filistenii stateau pe munte de o parte ?i Israeli?ii stateau pe munte de alta parte, iar la mijloc era valea.
1Sa 17:4  Atunci a ie?it din tabara Filistenilor un luptator cu numele Goliat, din Gat. Acesta era la statura de ?ase co?i ?i o palma.
1Sa 17:5  Pe cap avea coif de arama ?i era îmbracat cu plato?a în solzi; greutatea plato?ei lui cântarea cinci mii de sicli de arama;
1Sa 17:6  În picioare avea cizme cu tureci de arama ?i la umar purta un scut de arama.
1Sa 17:7  Coada suli?ei lui era ca sulul de la razboaiele de ?esut, iar fierul suli?ei era de ?ase sute sicli de fier, ?i înaintea lui mergea purtatorul lui de arme.
1Sa 17:8  ?i a început acesta sa strige catre cetele lui Israel ?i sa le zica: "De ce a?i ie?it voi sa va razboi?i? Nu sunt eu oare filistean, iar voi robii lui Saul? Alege?i dintre voi un om sa se coboare la mine!
1Sa 17:9  De se va putea acela lupta cu mine ?i ma va ucide, atunci noi sa fim robii vo?tri; iar de-l voi birui eu ?i-l voi ucide, atunci voi sa firi robii no?tri ?i sa ne sluji?i noua".
1Sa 17:10  ?i a mai zis filisteanul: "Astazi voi ru?ina tabara lui Israel. Da?i-ne un om ?i ne vom lupta în doi".
1Sa 17:11  ?i a auzit Saul ?i to?i Israeli?ii cuvintele acestea ale filisteanului ?i s-au speriat ?i s-au temut foarte tare.
1Sa 17:12  David era feciorul unui efraimit din Betleemul lui Iuda, anume Iesei, care avea opt feciori. Acest om în zilele lui Saul ajunsese la batrâne?e ?i era cel mai batrân între ceilal?i oameni.
1Sa 17:13  Cei trei feciori mai mari ai lui Iesei plecasera cu Saul la razboi. Numele acestor feciori mai mari ai lui, care se dusera la razboi, erau: cel mai mare Eliab, al doilea dupa el Aminadab ?i al treilea ?ama.
1Sa 17:14  David însa era cel mai mic. Când cei trei mai mari plecasera cu Saul,
1Sa 17:15  David se întorsese de la Saul, ca sa pasca oile tatalui sau în Betleem.
1Sa 17:16  Filisteanul acela însa ie?ea diminea?a ?i seara ?i s-a aratat patruzeci de zile.
1Sa 17:17  Atunci a zis Iesei catre David, fiul sau: "Ia pentru fra?ii tai o efa de graun?e uscate ?i aceste zece pâini ?i du-le cât mai degraba în tabara la fra?ii tai;
1Sa 17:18  Iar ace?ti zece ca?i du-i capeteniei celei peste mia lor; cerceteaza de sanatatea lor ?i afla ce nevoi au".
1Sa 17:19  Atunci Saul ?i ei ?i to?i Israeli?ii se aflau în Valea Stejarului ?i se pregateau de lupta cu Filistenii.
1Sa 17:20  S-a sculat deci David dis-de-diminea?a ?i, încredin?ând oile unui pastor, a luat sacul ?i a plecat, cum îi zisese Iesei, ?i a ajuns în tabara când o?tirea era a?ezata în linie de bataie ?i se gatea cu strigate de razboi.
1Sa 17:21  ?i ?i-au a?ezat Israeli?ii ?i Filistenii rândurile unii în fa?a altora.
1Sa 17:22  Iar David, lasându-?i lucrurile unei straji din tabara, a alergat între rânduri ?i, ajungând, a întrebat pe fra?ii sai de sanatate.
1Sa 17:23  ?i iata, pe când vorbea el cu ei, luptatorul cu numele de Goliat, filistean din Gat, a ie?it din rândurile Filistenilor ?i a spus aceste cuvinte, ?i David le-a auzit.
1Sa 17:24  To?i Israeli?ii, vazând pe omul acela, fugeau de el, temându-se foarte tare;
1Sa 17:25  ?i ziceau Israeli?ii: "Vede?i pe omul acesta care a ie?it înainte? Iese ca sa înfrunte pe Israel. De l-ar ucide cineva, regele ar rasplati pe acela cu mari boga?ii ?i ar da pe fiica sa dupa el, iar casa tatalui aceluia ar ajunge libera în Israel".
1Sa 17:26  David a zis catre oamenii care stateau cu el: "Ce se va face aceluia care va ucide pe acest filistean ?i va ?terge ocara de pe Israel? Caci cine este acest filistean netaiat împrejur, de batjocore?te a?a o?tirea Dumnezeului celui viu?"
1Sa 17:27  ?i i-a spus mul?imea acelea?i cuvinte, zicând: "Iata ce se va face omului aceluia care-l va ucide".
1Sa 17:28  ?i auzind Eliab, fratele cel mai mare al lui David, ce vorbea acesta cu oamenii, s-a mâniat Eliab pe David ?i a zis: "Pentru ce ai venit aici ?i cu cine ai lasat acele pu?ine oi în pustiu? Eu cunosc mândria ta ?i inima ta cea rea. Ai venit sa prive?ti la lupta".
1Sa 17:29  Iar David a zis: "Dar ce am facut eu? Au nu sunt acestea numai ni?te vorbe?"
1Sa 17:30  ?i s-a întors de la el catre altul ?i a spus acelea?i vorbe, iar mul?imea i-a raspuns ca ?i mai înainte.
1Sa 17:31  Auzindu-se cuvintele pe care le graise David, s-au spus lui Saul ?i acesta l-a chemat.
1Sa 17:32  Atunci David a zis catre Saul: "Sa nu se împu?ineze nimeni cu duhul din pricina lui; robul tau se va duce ?i se va bate cu acest filistean!"
1Sa 17:33  A zis Saul catre David: "Tu nu vei putea sa mergi împotriva acestui filistean, ca sa te ba?i cu el, caci e?ti înca un copilandru, iar acesta este osta? din tinere?ile lui".
1Sa 17:34  David însa a zis catre Saul: "Robul tau a pascut oile tatalui sau ?i când se întâmpla sa vina leul sau ursul sa ia vreo oaie din turma,
1Sa 17:35  Atunci eu alergam dupa el ?i i-o luam din gura lui; iar daca el se arunca asupra mea, eu îl apucam de coama ?i-l loveam pâna-l ucideam.
1Sa 17:36  ?i ur?i ?i lei a ucis robul tau; ?i cu acest filistean netaiat împrejur se va întâmpla acela?i lucru ca ?i cu aceia, pentru ca hule?te a?a o?tirea Dumnezeului celui viu. Sa ma duc dar ?i sa-l lovesc, ca sa spal ru?inea lui Israel? Caci cine e oare acest filistean?"
1Sa 17:37  Apoi a mai zis David: "Domnul, Cel ce m-a scapat de la lei ?i ur?i, ma va scapa ?i din mina acestui filistean!" Atunci Saul a zis lui David: "Du-te ?i Domnul sa fie cu tine".
1Sa 17:38  ?i a îmbracat Saul pe David cu hainele sale, a pus pe capul lui coif de arama ?i l-a îmbracat cu zale.
1Sa 17:39  ?i s-a încins David cu sabia lui peste haine ?i a început sa umble, caci nu era deprins cu astfel de armura; apoi a zis David catre Saul: "Nu pot sa umblu cu acestea, ca nu sunt deprins". ?i s-a dezbracat David de toate acestea,
1Sa 17:40  ?i ?i-a luat toiagul în mâna, a ales cinci pietricele lucii din pârâu ?i le-a pus în traista sa de pastor; ?i cu traista ?i cu pra?tia în mâna a ie?it înaintea filisteanului.
1Sa 17:41  Atunci a ie?it ?i filisteanul, înaintând ?i apropiindu-se de David; iar purtatorul lui de arme mergea înainte.
1Sa 17:42  Deci cautând filisteanul ?i vazând pe David, a privit cu dispre? la el, caci acesta era tânar, balan ?i frumos la fa?a.
1Sa 17:43  A zis filisteanul catre David: "Ce vii asupra mea cu toiag ?i cu pietre? Au doara eu sunt câine?" Iar David a raspuns: "Nu, ci mai rau decât un câine". ?i a blestemat filisteanul pe David în numele dumnezeilor sai.
1Sa 17:44  Apoi a zis filisteanul catre David: "Apropie-te de mine ?i voi da trupul tau pasarilor cerului ?i fiarelor câmpului!"
1Sa 17:45  Iar David a raspuns filisteanului: "Tu vii asupra mea cu sabie ?i cu lance ?i cu scut; eu însa vin asupra ta în numele Domnului Savaot, Dumnezeul o?tirilor lui Israel pe Care tu L-ai hulit.
1Sa 17:46  Acum însa te va da Domnul în mâna mea ?i eu te voi ucide ?i-?i voi taia capul, iar trupul tau ?i trupurile o?tirii filistene le voi da pasarilor cerului ?i fiarelor câmpului, ?i va afla tot pamântul ca în Israel este Dumnezeu;
1Sa 17:47  ?i toata adunarea aceasta va cunoa?te ca nu cu sabia ?i cu suli?a izbave?te Domnul, caci acest razboi este al Domnului ?i El va va da în mâinile noastre".
1Sa 17:48  Iar dupa ce s-a ridicat filisteanul ?i a început a veni ?i a se apropia în întâmpinarea lui David, David a alergat cu grabire spre rândurile o?tirii în întâmpinarea filisteanului.
1Sa 17:49  ?i î?i vârî David mâna în traista, lua de acolo o pietricica, o repezi cu pra?tia ?i lovi pe filistean în frunte, a?a încât piatra se înfipse în fruntea lui ?i el cazu cu fa?a la pamânt.
1Sa 17:50  A?a a biruit David pe filistean, cu pra?tia ?i cu piatra, lovind pe filistean ?i ucigându-l; sabie nu se afla în mâna lui David.
1Sa 17:51  Atunci David a alergat ?i, calcând pe filistean, lua sabia lui ?i, sco?ând-o din teaca, îl lovi cu ea ?i-i taie capul; Filistenii, vazând ca uria?ul lor a murit, au fugit.
1Sa 17:52  Deci s-au sculat barba?ii lui Israel ?i ai lui Iuda ?i cu strigate au gonit pe Filisteni pâna la gura vaii ?i pâna la por?ile Ecronului. ?i au cazut uci?i Filistenii pe calea ?aaraim, pâna la Gat ?i Ecron.
1Sa 17:53  Dupa aceea s-au întors fiii lui Israel din urmarirea Filistenilor ?i au pradat tabara lor.
1Sa 17:54  Iar David a luat capul filisteanului ?i l-a dus la Ierusalim, ?i armele lui le-a pus în cortul sau.
1Sa 17:55  Când a vazut Saul pe David ie?ind împotriva filisteanului, a zis catre Abner, capetenia o?tirilor: "Abner, al cui este tânarul acesta?" Iar Abner, a raspuns: "Rege, viu fie sufletul tau, nu ?tiu!"
1Sa 17:56  "Întreaba dar, a zis regele, al cui fiu este tânarul acesta?"
1Sa 17:57  Iar când se întorcea David, dupa uciderea filisteanului, Abner l-a luat ?i l-a dus la Saul ?i capul filisteanului era în mâna lui.
1Sa 17:58  Atunci Saul l-a întrebat: "Tinere, al cui fiu e?ti tu?" ?i David a raspuns: "Fiul robului tau Iesei din Betleem".
1Sa 18:1  Dupa ce a ispravit David de vorbit cu Saul, sufletul lui Ionatan s-a lipit de sufletul lui ?i l-a iubit Ionatan, ca pe sufletul sau.
1Sa 18:2  Iar Saul l-a luat în ziua aceea ?i nu l-a lasat sa se mai întoarca la casa tatalui lui.
1Sa 18:3  ?i a încheiat Ionatan legatura cu David, pentru ca îl iubea ca pe sufletul sau.
1Sa 18:4  ?i ?i-a dezbracat Ionatan haina sa cea de deasupra, pe care o avea pe el, ?i a dat-o lui David; de asemenea ?i celelalte haine ale sale, sabia sa, arcul sau ?i brâul sau.
1Sa 18:5  David însa lucra cu pricepere peste tot, oriunde-l trimetea Saul; ?i Saul l-a facut capetenie peste o?teni; iar aceasta a placut la tot poporul ?i slujitorilor lui Saul.
1Sa 18:6  Dar când se întorceau ei, dupa izbânda lui David asupra filisteanului, femeile din toate ceta?ile lui Israel ie?eau în întâmpinarea regelui Saul cu cântari ?i jocuri, cu timpane de sarbatoare ?i cu chimvale;
1Sa 18:7  ?i jucând, femeile strigau ?i ziceau: "Saul a biruit mii, iar David zeci de mii!"
1Sa 18:8  De aceea s-a mâniat Saul foarte tare, neplacându-i cuvintele acestea ?i a zis: "Lui David i s-au dat zeci de mii, iar mie numai mii; acum numai domnia îi mai lipse?te".
1Sa 18:9  ?i din ziua aceasta în tot timpul urmator, s-a uitat la David banuitor.
1Sa 18:10  Iar a doua zi s-a întâmplat de a cazut duhul cel rau de la Dumnezeu asupra lui Saul ?i acesta se îndracea în casa sa, iar David cânta cu mâna sa pe strune, ca ?i în alte zile; Saul avea în mâna o lance.
1Sa 18:11  ?i a aruncat Saul lancea, cugetând: "Voi pironi pe David de perete!" Dar David s-a ferit de doua ori de Saul.
1Sa 18:12  ?i a început a se teme Saul de David, pentru ca Domnul era cu el, iar de Saul se departase.
1Sa 18:13  De aceea Saul l-a îndepartat de la sine ?i l-a pus capetenie peste o mie; ?i se ducea ?i se întorcea el în fruntea poporului.
1Sa 18:14  David în toate treburile sale se purta cu chibzuin?a ?i Domnul era cu el.
1Sa 18:15  Saul vedea ca este foarte chibzuit ?i se temea de el.
1Sa 18:16  Iar Israelul tot ?i Iuda iubea pe David, pentru ca el se ducea ?i se întorcea în fruntea lor.
1Sa 18:17  Deci a zis Saul catre David: "Iata fata mea cea mai mare, Merob, î?i voi da-o de so?ie, numai sa-mi fii viteaz ?i sa duci razboaiele Domnului". Caci Saul socotea: "Lasa, sa nu fie mâna mea asupra lui, ci sa fie asupra lui mâna Filistenilor".
1Sa 18:18  David însa a zis catre Saul: "Cine sunt eu ?i ce este via?a mea ?i neamul tatalui meu în Israel, ca sa fiu ginerele regelui?"
1Sa 18:19  Iar când a venit vremea sa dea pe Merob, fiica lui Saul, dupa David, ea a fost maritata cu Adriel din Mehola.
1Sa 18:20  Pe David însa îl iubea alta fata a lui Saul, Micol; ?i când i s-a spus despre aceasta lui Saul, aceasta i-a placut;
1Sa 18:21  Caci Saul cugeta: "Am s-o dau dupa el ?i ea are sa-i fie cursa ?i mâna Filistenilor are sa fie asupra lui". ?i a zis Saul catre David: "A doua oara te înrude?ti acum cu mine".
1Sa 18:22  Atunci a poruncit Saul slujitorilor sai: "Spune?i lui David: Iata regele este binevoitor catre tine ?i to?i slujitorii lui te iubesc; fii dar ginerele meu!"
1Sa 18:23  ?i au vorbit slujitorii lui Saul în urechile lui David toate vorbele acestea. Iar David a zis: "Oare u?or lucru vi se pare voua a fi ginerele regelui? Eu sunt un sarac ?i un neînsemnat".
1Sa 18:24  ?i au în?tiin?at pe rege slugile sale ?i au zis: "Iata ce a spus David".
1Sa 18:25  Iar Saul a zis: "A?a sa-i spune?i lui David: Regele nu voie?te zestre decât numai o suta de prepu?uri filistene, ca razbunare împotriva vrajma?ilor regelui". Caci Saul avea în gând sa piarda pe David prin mâna Filistenilor.
1Sa 18:26  ?i slugile lui Saul au spus lui David cuvintele acestea ?i i-a placut lui David sa se faca ginerele regelui.
1Sa 18:27  Dar nu apucase înca sa vina ziua sorocita, când David se scula ?i merse el însu?i ?i oamenii lui ?i ucise 200 de Filisteni; ?i aduse David prepu?urile lor ?i le înfa?i?a regelui numar deplin, ca sa se poata face ginerele regelui. ?i a dat Saul dupa el pe Micol, fiica sa, de femeie.
1Sa 18:28  Dar vazând ?i aflând Saul ca Domnul este cu David ?i tot Israelul îl iube?te ?i ca ?i fiica sa Micol îl iube?te,
1Sa 18:29  Începu înca ?i mai mult sa se teama de David ?i s-a facut vrajma?ul lui pe via?a ?i pe moarte.
1Sa 18:30  De aceea, când au ie?it la razboi capeteniile Filistenilor, David, chiar de la ie?irea lor, lucra mai în?elep?e?te decât toate slugile lui Saul ?i numele lui a ajuns foarte vestit.
1Sa 19:1  Atunci a zis Saul catre Ionatan, fiul sau ?i catre toate slugile sale, sa ucida pe David. Ionatan fiul lui Saul însa iubea foarte mult pe David.
1Sa 19:2  ?i a vestit Ionatan pe David, zicând: "Tatal meu Saul cauta sa te omoare. Deci sa te paze?ti mine; ascunde-te ?i stai la loc tainic;
1Sa 19:3  Iar eu voi ie?i ?i voi fi lânga tatal meu în câmp unde vei fi tu ?i voi vorbi tatalui meu de tine ?i ce voi vedea i?i voi spune".
1Sa 19:4  Ionatan a vorbit de bine lui Saul, tatal sau, pentru David ?i i-a zis: "Sa nu gre?easca regele împotriva robului tau David, caci el cu nimic n-a gre?it împotriva ta ?i faptele lui sunt foarte folositoare pentru tine.
1Sa 19:5  El ?i-a pus via?a în primejdie, ca sa loveasca pe Filisteni ?i Domnul a facut izbavire mare la tot Israelul. Tu ai vazut aceasta ?i te-ai bucurat. De ce dar vrei tu sa pacatuie?ti împotriva unui sânge nevinovat ?i sa ucizi pe David fara nici o pricina?"
1Sa 19:6  ?i a ascultat Saul glasul lui Ionatan ?i s-a jurat Saul: "Viu este Domnul! David nu va muri!"
1Sa 19:7  Atunci a chemat Ionatan pe David ?i i-a spus Ionatan toate cuvintele acestea; ?i a adus Ionatan pe David la Saul ?i a slujit el ca ?i mai înainte.
1Sa 19:8  Dar a început iara?i razboiul ?i a ie?it David ?i s-a luptat cu Filistenii ?i le-a pricinuit înfrângere mare ?i au fugit ei de el.
1Sa 19:9  Iar duhul cel rau de la Dumnezeu a cazut asupra lui Saul ?i acesta ?edea în casa sa ?i suli?a lui era în mâna lui; David însa cânta din harfa.
1Sa 19:10  Atunci Saul a vrut sa pironeasca cu suli?a pe David de perete, însa David s-a ferit de Saul ?i suli?a s-a înfipt în perete; apoi David a fugit în noaptea aceea ?i a scapat.
1Sa 19:11  ?i a trimis Saul slujitorii acasa la David, ca sa-l pândeasca ?i sa-l omoare pâna diminea?a. Micol însa so?ia lui David, i-a zis: "Daca tu nu-?i scapi sufletul în aceasta noapte, diminea?a vei fi ucis".
1Sa 19:12  ?i a dat Micol drumul lui David pe fereastra; iar David, ie?ind, a fugit ?i a scapat.
1Sa 19:13  Dupa aceea Micol a luat un idol ?i l-a pus în pat, a pus o piele de capra pe capul lui ?i l-a învelit cu o haina.
1Sa 19:14  Când Saul a trimis slujitorii ca sa aduca pe David, Micol a zis: "E bolnav!"
1Sa 19:15  Saul însa a trimis din nou slujitorii, ca sa vada bine pe David, zicând: "Aduce?i-l la mine cu patul, ca sa-l omor!"
1Sa 19:16  ?i au mers slujitorii la casa lui David, dar iata în pat era un idol ?i pe capul lui o piele de capra.
1Sa 19:17  Atunci Saul a zis catre Micol: "De ce m-ai amagit tu a?a ?i ai lasat pe vrajma?ul meu sa fuga?" Iar Micol a raspuns lui Saul: "Pentru ca el mi-a zis: Da-mi drumul, caci de nu, te ucid!"
1Sa 19:18  A?a a scapat David ?i a fugit ?i s-a dus la Samuel în Rama ?i i-a povestit toate cele ce-i facuse  Saul. Apoi a mers el cu Samuel ?i s-a oprit la Naiotul cel din Rama.
1Sa 19:19  ?i s-a spus lui Saul: "Iata David este la Naiotul Ramei!"
1Sa 19:20  ?i a trimis Saul slujitori sa prinda pe David; dar când au vazut ace?tia ceata proorocilor proorocind ?i pe Samuel pova?uindu-i, S-a pogorât Duhul lui Dumnezeu peste slujitorii lui Saul ?i au început ?i ei a prooroci.
1Sa 19:21  Spunându-se acestea lui Saul, el a trimis al?i slujitori, dar ?i ace?tia au început a prooroci. Apoi Saul a trimis al treilea rând de slujitori ?i începura ?i ace?tia sa prooroceasca.
1Sa 19:22  Mâniindu-se, în sfâr?it, Saul a plecat însu?i la Rama ?i a mers pâna la izvorul cel mare din Soco. Acolo a întrebat ?i a zis: "Unde sunt Samuel ?i David?" ?i i s-a spus: "Iata aici în Naiotul Ramei".
1Sa 19:23  ?i a plecat el acolo, la Naiotul Ramei. Dar pe cale S-a pogorât peste el Duhul lui Dumnezeu, iar el a mers proorocind pâna a ajuns la Naiotul Ramei.
1Sa 19:24  Acolo s-a dezbracat de haine ?i a proorocit înaintea lui Samuel ?i toata ziua aceea ?i toata noaptea a ?ezut dezbracat. De aceea se zice: "Au doara ?i Saul este printre prooroci?"
1Sa 20:1  Atunci David a fugit din Naiotul Ramei ?i venind a zis catre Ionatan: "Ce-am facut eu oare? Care este strâmbatatea mea ?i cu ce am gre?it înaintea tatalui tau, de-mi cauta sufletul meu?"
1Sa 20:2  Iar Ionatan i-a raspuns: "Nu, nu vei muri. Iata tatal meu nu face nici un lucru mare sau mic, fara sa-l descopere urechilor mele. Pentru ce dar ar ascunde tatal meu de mine lucrul acesta? Aceasta nu se poate".
1Sa 20:3  David însa s-a jurat ?i a zis: "Tatal tau ?tie bine ca eu am dobândit trecere la tine, ?i de aceea î?i zice: "Nu trebuie sa ?tie de aceasta Ionatan, ca sa nu se amarasca. Dar viu este Domnul ?i viu este sufletul tau; între mine ?i moarte n-a fost decât un pas".
1Sa 20:4  Atunci Ionatan a zis catre David: "Tot ce dore?te sufletul tau voi face pentru tine".
1Sa 20:5  ?i David a zis catre Ionatan: "Iata mâine este luna noua ?i eu trebuie sa stau cu regele la masa; dar lasa-ma sa ma ascund în câmp pâna poimâine seara.
1Sa 20:6  De va întreba tatal tau de mine, tu sa spui: "David s-a cerut de la mine sa se duca în cetatea sa Betleem, pentru ca acolo se face jertfa anuala pentru tot neamul sau".
1Sa 20:7  Daca el la aceasta va raspunde: "Bine", atunci este semn de pace pentru robul tau, iar daca se va mânia, atunci sa ?tii ca el a pus la cale lucru rau.
1Sa 20:8  Tu însa sa faci mila cu robul tau, caci ai primit pe robul tau sa faca legamântul Domnului cu tine, ?i, de este vreo vina asupra mea, atunci ucide-ma tu; de ce sa ma mai duci la tatal tau?"
1Sa 20:9  Ionatan însa a zis: "În nici un chip nu se va întâmpla aceasta cu tine; caci de voi afla ca în adevar tatal meu a hotarât sa-?i faca vreun lucru rau, nu te voi vesti eu oare despre aceasta?"
1Sa 20:10  ?i a zis David catre Ionatan: "Cine ma va vesti, daca tatal tau î?i va raspunde aspru?"
1Sa 20:11  A zis Ionatan catre David: "Hai sa ie?im la câmp". ?i au ie?it amândoi la câmp.
1Sa 20:12  Acolo Ionatan a zis catre David: "Viu este Domnul Dumnezeul lui Israel, mâine pe vremea aceasta sau poimâine, voi cauta sa aflu de la tatal meu, ?i daca el este binevoitor lui David ?i eu nu voi trimite la tine ?i nu voi descoperi aceasta urechilor tale,
1Sa 20:13  Atunci a?a ?i a?a sa faca Domnul cu Ionatan ?i înca ?i mai mult sa faca. Daca însa tatal meu planuie?te sa-?i faca rau, aceasta voi descoperi-o urechilor tale ?i-?i voi da drumul sa mergi în pace ?i sa fie Domnul cu tine, cum a fost cu tatal meu!
1Sa 20:14  Dar ?i tu, de voi mai fi în via?a, sa-mi ara?i mila Domnului.
1Sa 20:15  Iar de voi muri, sa nu-?i aba?i mila de la casa mea în veci, chiar ?i când Domnul ar pierde de pe fa?a pamântului pe to?i vrajma?ii lui David".
1Sa 20:16  A?a a încheiat Ionatan legamânt cu casa lui David ?i a zis: "Sa pedepseasca Domnul pe vrajma?ii lui David!"
1Sa 20:17  ?i iara?i s-a jurat Ionatan lui David pe iubirea sa cea catre el, caci îl iubea ca pe sufletul sau.
1Sa 20:18  ?i i-a zis Ionatan: "Mâine este luna noua ?i se va întreba despre tine, caci locul tau va fi gol.
1Sa 20:19  De aceea poimâine pleaca ?i grabe?te spre locul acela unde te-ai ascuns ?i înainte ?i ?ezi lânga piatra Ezel;
1Sa 20:20  Iar eu voi slobozi într-acolo trei sage?i, ca ?i cum a? trage la ?inta.
1Sa 20:21  Apoi voi trimite un baiat ?i-i voi zice: "Du-te de cauta sage?ile". ?i de voi zice baiatului: "Iata sage?ile sunt dincoace de tine, ia-le!", atunci sa vii la mine, ca este pace pentru tine, ?i viu este Domnul, nimic nu ?i se va întâmpla.
1Sa 20:22  Daca însa voi zice baiatului a?a: "Iata sage?ile sunt dincolo de tine", atunci sa pleci, caci Domnul te libereaza.
1Sa 20:23  Iar la cele ce am grait eu cu tine, este martor Domnul între mine ?i tine în veci! "
1Sa 20:24  ?i s-a ascuns David în câmp ?i, venind luna noua, a ie?it regele la masa.
1Sa 20:25  Regele a stat la locul sau, ca de obicei, pe scaunul de la perete; Ionatan s-a sculat ?i Abner a stat lânga Saul; iar locul lui David a ramas gol.
1Sa 20:26  În ziua aceea Saul nu a zis nimic, caci socotea ca aceasta este o întâmplare, ca David nu este curat, nu s-a cura?it.
1Sa 20:27  Dar a venit ?i ziua a doua dupa luna noua ?i locul lui David a ramas gol. Atunci a zis Saul catre fiul sau Ionatan: "Pentru ce fiul lui Iesei n-a venit la masa nici ieri, nici astazi?"
1Sa 20:28  Ionatan însa a raspuns lui Saul: "David s-a cerut la mine sa mearga la Betleem".
1Sa 20:29  ?i a zis: Da-mi voie sa ma duc, ca în cetatea noastra este jertfa pentru neamul nostru ?i m-a poftit fratele meu. Deci de am aflat bunavoin?a în ochii tai, ma duc sa ma vad cu fra?ii mei. De aceea n-a venit el la masa regelui".
1Sa 20:30  Atunci regele s-a mâniat stra?nic pe Ionatan ?i i-a zis: "Fiu netrebnic ?i neascultator! Oare nu ?tiu eu ca te-ai împrietenit cu fiul lui Iesei, spre ru?inea ta ?i spre batjocura mamei tale?
1Sa 20:31  Caci atâta vreme cât fiul lui Iesei va fi viu pe pamânt, nu e?ti scutit de primejdie, nici tu, nici regatul tau. Trimite dar acum ?i adu-mi-l mie, ca este hotarât la moarte!"
1Sa 20:32  A raspuns Ionatan lui Saul, tatal sau, ?i i-a zis: "De ce sa-l omori? Ce-a facut el?"
1Sa 20:33  Atunci Saul a repezit suli?a în el ca sa-l loveasca. ?i a în?eles Ionatan ca tatal sau este hotarât sa ucida pe David.
1Sa 20:34  Deci s-a sculat Ionatan de la masa, prins de mânie mare, ?i n-a mâncat a doua zi dupa luna noua, pentru ca era trist dupa David ?i pentru ca-l ocarâse tatal sau.
1Sa 20:35  A doua zi diminea?a a ie?it Ionatan la câmp, la vremea sorocita lui David, ?i un baiat mic a ie?it cu el.
1Sa 20:36  ?i a zis el baiatului: "Fugi ?i cauta sage?ile pe care am sa le slobod eu!" ?i a alergat baiatul, iar el a slobozit sage?ile, a?a încât au cazut dincolo de baiat.
1Sa 20:37  ?i a alergat baiatul spre locul unde aruncase Ionatan sage?ile. Iar Ionatan a strigat în urma lui ?i a zis: "Vezi ca sage?ile sunt înaintea ta".
1Sa 20:38  Apoi iar a strigat Ionatan dupa baiat: "Umbla mai repede ?i nu te opri!" Baiatul a adunat sage?ile lui Ionatan ?i a venit la stapânul sau.
1Sa 20:39  Baiatul însa nu ?tia nimic din toate acestea; numai Ionatan ?i David ?tiau de ce este vorba.
1Sa 20:40  ?i a dat Ionatan arma baiatului, care era cu el, ?i i-a zis: "Du-te ?i o du în cetate".
1Sa 20:41  Dupa ce s-a dus baiatul, David s-a ridicat din partea de miazazi a stâncii ?i s-a închinat de trei ori; apoi s-au sarutat ei unul pe altul ?i au plâns amândoi, împreuna, iar David a plâns mai tare.
1Sa 20:42  ?i a zis Ionatan catre David: "Mergi cu pace! Iar cele pentru care ne-am jurat noi amândoi pe numele Domnului zicând: "Domnul sa fie între mine ?i tine ?i între copiii mei ?i copiii tai, aceea sa fie pe veci". ?i s-a sculat David ?i s-a dus, iar Ionatan s-a întors în cetate.
1Sa 21:1  Dupa aceea a mers David în Nobe, la preotul Ahimelec ?i s-a mixat Ahimelec la întâlnirea cu David ?i i-a zis: "De ce e?ti singur ?i nu este nimeni cu tine?"
1Sa 21:2  Iar David a raspuns preotului Ahimelec: "Regele mi-a încredin?at o taina ?i mi-a zis: Sa nu ?tie nimeni pentru ce te-am trimis ?i ce însarcinare ?i-am dat. De aceea mi-am lasat oamenii într-un loc anumit.
1Sa 21:3  Da-mi dar ce ai la îndemâna, vreo cinci pâini, sau ce se va gasi!"
1Sa 21:4  Preotul însa a raspuns lui David ?i i-a zis: "Pâine obi?nuita n-am la îndemâna, dar este pâine sfin?ita; daca oamenii tai s-au înfrânat de la femei, pot sa manânce".
1Sa 21:5  Iar David a raspuns preotului ?i i-a zis: "Femei n-am avut cu noi nici ieri, nici alaltaieri, de când am plecat, ?i vasele (trupurile) oamenilor sunt curate; de?i calatoria nu este dupa orânduiala religioasa, pâinea va ramâne curata în vasele (trupurile) lor".
1Sa 21:6  ?i i-a dat preotul pâinea sfin?ita, caci nu avea alta pâine, afara de pâinile punerii înainte, care fusesera luate de la fala Domnului, ca sa se puna în locul lor pâini proaspete.
1Sa 21:7  În ziua aceea se afla acolo înaintea Domnului unul din slujitorii lui Saul, cu numele Doeg, idumeu, capetenia pastorilor lui Saul.
1Sa 21:8  ?i a zis David catre Ahimelec: "N-ai cumva la îndemâna vreo suli?a sau vreo sabie? Caci eu nu mi-am luat nici sabia, nici alta arma, deoarece porunca regelui a fost grabnica".
1Sa 21:9  Preotul însa a raspuns: "Iata sabia lui Goliat filisteanul pe care l-ai ucis în Valea Stejarului; ea este învelita într-o haina, dupa efod; de vrei, ia-o; alta afara de aceea n-am aici". David a raspuns: "Ca ea nu mai este alta, da-mi-o!" ?i i-a dat-o.
1Sa 21:10  Apoi David s-a sculat ?i a fugit în aceea?i zi de la fa?a lui Saul ?i a mers ?i s-a dus la Achi?, regele din Gat.
1Sa 21:11  Iar slugile lui Achi? au zis acestuia: "Oare nu este acesta David, regele ?arii aceleia, ?i nu lui oare i se cânta în cor ?i se zicea: "Saul a biruit mii, iar David zeci de mii?"
1Sa 21:12  David a pus cuvintele acestea la inima ?i s-a temut tare de Achi?, regele din Gat,
1Sa 21:13  ?i s-a prefacut nebun înaintea ochilor lui, facând nazdravanii ?i scriind pe u?i; mergea în mâini ?i lasa sa-i curga balele pe barba.
1Sa 21:14  Atunci a zis Achi? robilor sai: "Nu vede?i ca este un om nebun? La ce l-a?i adus la mine?
1Sa 21:15  N-am eu destui nebuni? De ce l-a?i adus ?i pe acesta sa se schimonoseasca înaintea mea? Nu cumva vre?i sa intre în casa la mine?"
1Sa 22:1  ?i a plecat David de acolo ?i a fugit în pe?tera Adulam. Auzind aceasta, fra?ii lui ?i toata casa tatalui sau au venit la el.
1Sa 22:2  ?i s-au adunat la el to?i prigoni?ii, to?i datornicii ?i to?i cei cu sufletul amarât ?i s-a facut el capetenie peste ei; ?i erau cu el ca la patru sute de oameni.
1Sa 22:3  De acolo David s-a dus la Mi?pa Moabului ?i a zis catre regele Moabului: "Lasa pe tatal meu ?i pe mama mea sa stea la voi, pâna voi afla ce are sa faca Dumnezeu cu mine".
1Sa 22:4  ?i i-a adus la regele Moabului ?i au trait ei tot timpul la el, cât David a ramas în cetatea aceea.
1Sa 22:5  Dar proorocul Gad a zis lui David: "Nu mai ramâne în cetatea aceasta, ci pleaca ?i mergi în pamântul lui Iuda". ?i a plecat David ?i a venit în padurea Heret.
1Sa 22:6  Dar a auzit Saul ca s-a ivit David ?i oamenii cei ce erau cu el. Saul ?edea atunci în Ghibeea, pe deal, sub un stejar, cu suli?a în mâna ?i toate slugile sale stateau împrejurul lui.
1Sa 22:7  Zis-a Saul catre slugile cele dimprejurul lui: "Asculta?i, fiii lui Veniamin. Oare tuturor va va da fiul lui Iesei ?arini ?i vii, ?i va va pune pe to?i suta?i ?i capetenii peste mii,
1Sa 22:8  De v-a?i sfatuit cu to?ii în contra mea ?i nimeni nu mi-a descoperit, când fiul meu a intrat în prietenie cu fiul lui Iesei, ?i nimeni din voi n-a avut mila de mine ?i nu mi-a descoperit ca fiul meu a a?â?at împotriva mea pe robul meu sa-mi urzeasca intrigi, cum se vede acum?"
1Sa 22:9  Atunci a raspuns Doeg idumeul, care statea cu slugile lui Saul, ?i a zis: "Eu am vazut cum a venit fiul lui Iesei în Nobe, la Ahimelec, fiul lui Ahituv,
1Sa 22:10  ?i acela a întrebat pentru el pe Domnul ?i i-a dat merinde; ba i-a dat ?i sabia lui Goliat filisteanul.
1Sa 22:11  Atunci a trimis regele sa cheme pe Ahimelec, fiul lui Ahituv preotul, ?i toata casa tatalui lui, preo?ii din Nobe. ?i au venit ei cu to?ii la rege.
1Sa 22:12  Iar Saul le-a zis: "Asculta, fiul lui Ahituv!" ?i acela raspunse: "Da, domnul meu!"
1Sa 22:13  ?i a zis Saul catre el: "Pentru ce v-aii unit voi împotriva mea, tu ?i fiul lui Iesei, ca i-ai dat pâini ?i sabie ?i ai întrebat pentru el pe Dumnezeu, ca sa se razvrateasca împotriva mea ?i sa ma pândeasca, cum se vede acum?"
1Sa 22:14  A raspuns Ahimelec regelui ?i a zis: "Cine din to?i robii tai este credincios ca David? ?i apoi el este ?i ginerele regelui, îndeplinitorul poruncilor tale, ?i cu cinste în casa ta.
1Sa 22:15  ?i apoi oare de astazi am început eu sa întreb pe Dumnezeu pentru el? Nu, nu învinui de asta, o, rege, pe robul tau ?i toata casa tatalui meu, caci în toata pricina aceasta nu cunoa?te robul tau nici un lucru mare sau mic".
1Sa 22:16  Atunci regele a zis: "Tu, Ahimelec, trebuie sa mori, tu ?i toata casa tatalui tau".
1Sa 22:17  Apoi regele a zis catre paznicii lui, care stateau împrejurul sau: "Merge?i ?i omorâ?i pe preo?ii Domnului, caci ?i mâna lor este cu David; au ?tiut ca el a fugit ?i nu mi-au descoperit". Paznicii regelui însa n-au voit sa-?i ridice mâna, ca sa ucida pe preo?ii Domnului.
1Sa 22:18  Atunci regele a zis lui Doeg: "Mergi tu ?i ucide pe preo?i". ?i s-a dus Doeg idumeul ?i a navalit asupra preo?ilor ?i a ucis în ziua aceea optzeci ?i cinci de barba?i care purtau efod de in,
1Sa 22:19  Iar cetatea preo?easca Nobe a trecut-o prin ascu?i?ul sabiei: ?i barba?i ?i femei ?i tineri ?i copii ?i boi ?i asini ?i oi, tot a trecut prin ascu?i?ul sabiei.
1Sa 22:20  A scapat numai un singur fiu al lui Ahimelec, fiul lui Ahituv, anume Abiatar, ?i a fugit la David.
1Sa 22:21  ?i a spus Abiatar lui David ca Saul a ucis pe preo?ii Domnului.
1Sa 22:22  Atunci David a zis lui Abiatar: "Am ?tiut eu din ziua aceea ca, fiind acolo, Doeg idumeul va da de ?tire negre?it lui Saul ?i eu sunt vinovat pentru toate sufletele casei tatalui tau.
1Sa 22:23  Ramâi la mine ?i nu te teme, caci cine va cauta sufletul meu are sa caute ?i sufletul tau; tu vei fi aici, la mine, în paza!"
1Sa 23:1  Atunci i s-a vestit lui David ?i i s-a spus: "Iata Filistenii au navalit în Cheila ?i prada ariile".
1Sa 23:2  ?i a întrebat David pe Domnul, zicând: "Sa merg oare sa lovesc pe ace?ti Filisteni?" Iar Domnul a raspuns lui David: "Mergi, love?te pe Filisteni ?i izbave?te Cheila!"
1Sa 23:3  Dar cei ce erau cu David i-au zis: "Iata noi ne temem aici în Iuda. Cum dar sa mergem în Cheila contra taberelor filistene? Vrei sa cadem prada Filistenilor?"
1Sa 23:4  Atunci David a întrebat din nou pe Domnul ?i Domnul i-a raspuns ?i i-a zis: "Scoala ?i mergi la Cheila, caci Eu am sa dau pe Filisteni în mâinile tale".
1Sa 23:5  ?i s-a dus David cu oamenii sai la Cheila, de s-a luptat cu Filistenii, le-a luat vitele, le-a pricinuit înfrângere mare ?i a salvat David pe locuitorii din Cheila.
1Sa 23:6  Când Abiatar, fiul lui Ahimelec, a fugit la David ?i apoi s-a dus cu el la Cheila, a adus cu sine ?i efodul.
1Sa 23:7  Atunci s-a spus lui Saul ca David a mers la Cheila; iar Saul a zis: "Dumnezeu l-a dat în mâinile mele, caci a intrat în cetate ?i s-a închis cu porii ?i cu zavoare".
1Sa 23:8  ?i a chemat Saul tot poporul la razboi, ca sa mearga la Cheila sa împresoare pe David ?i pe oamenii lui.
1Sa 23:9  Când însa David a aflat ca Saul i-a pus gând rau, a zis preotului Abiatar: "Adu efodul Domnului!"
1Sa 23:10  Apoi David a adaugat: "Doamne, Dumnezeul lui Israel, robul Tau a aflat ca Saul vrea sa vina la Cheila sa darâme cetatea din pricina mea.
1Sa 23:11  Ma vor da locuitorii din Cheila pe mâinile lui ?i va veni Saul aici, cum a auzit robul Tau? Doamne Dumnezeul lui Israel, descopera aceasta robului Tau". Iar Domnul a zis: "Va veni!"
1Sa 23:12  ?i a zis David: "Ma vor da locuitorii din Cheila pe mine ?i oamenii mei în mâinile lui Saul?" ?i a zis Domnul: "Te vor da!"
1Sa 23:13  Atunci s-a ridicat David ?i oamenii lui ca la ?ase sute de in?i, au ie?it din Cheila ?i s-au dus unde au putut. Lui Saul însa i s-a spus ca David a fugit din Cheila ?i atunci el ?i-a schimbat planul.
1Sa 23:14  Iar David a petrecut prin pustiu în locuri nestrabatute ?i apoi pe un munte din pustiul Zif. Saul îl cauta în toate zilele, dar Dumnezeu nu l-a dat în mâinile lui.
1Sa 23:15  David vazuse ca Saul a ie?it sa caute sufletul lui, dar el se afla într-o padure din pustiul Zif.
1Sa 23:16  Atunci s-a sculat Ionatan, fiul lui Saul, a venit la David în padure ?i l-a întarit cu nadejdea în Dumnezeu,
1Sa 23:17  Zicându-i: "Nu te teme, caci nu te va gasi mâna tatalui meu Saul ?i tu vei împara?i peste Israel, iar eu voi fi al doilea dupa tine; Saul, tatal meu, ?tie aceasta".
1Sa 23:18  ?i au încheiat ei între ei legamânt înaintea fe?ei Domnului. Apoi Ionatan s-a întors la casa sa, iar David a ramas în padure.
1Sa 23:19  Atunci au venit Zifeii la Saul în Ghibeea ?i au zis: "Iata David sta ascuns la noi prin locuri nestrabatute, în padure, pe muntele Hachila, care vine la dreapta Ie?imonului.
1Sa 23:20  A?adar, o, rege, mergi dupa dorin?a sufletului tau, iar treaba noastra va fi sa-l dam în mâinile regelui".
1Sa 23:21  Saul însa le-a zis: "Binecuvânta?i sa fi?i voi la Domnul, ca a?i avut mila de mine.
1Sa 23:22  Merge?i ?i va mai încredin?a?i înca; cerceta?i ?i vede?i locul lui, pe unde îi calca piciorul ?i cine l-a vazut acolo, caci mie mi se spune ca este foarte ?iret.
1Sa 23:23  Cerceta?i ?i afla?i toate ascunzi?urile în care se dose?te; apoi întoarce?i-va la mine cu ?tiri amanun?ite ?i eu voi merge cu voi, de este în acea ?ara; îl voi cauta în toate miile lui Iuda".
1Sa 23:24  S-au sculat deci aceia ?i s-au dus la Zif, înainte de Saul. David însa ?i oamenii lui erau în pustia Maon, în ?es, la dreapta Ie?imonului.
1Sa 23:25  ?i a plecat Saul cu oamenii sai sa-l caute, dar David a fost vestit de aceasta ?i a trecut spre stânca, ramânând în pustia Maon. De aceasta a auzit ?i Saul ?i a alergat dupa David în pustia Maon:
1Sa 23:26  Saul mergea pe o coasta a muntelui, iar David cu oamenii sai se afla pe cealalta coasta a muntelui. Când David grabea sa se departeze de Saul, iar Saul cu oamenii lui se sileau sa împresoare pe David ?i pe oamenii lui, ca sa-i prinda,
1Sa 23:27  Atunci a venit la Saul un crainic ?i a zis: "Grabe?te ?i vino, ca Filistenii au intrat în ?ara".
1Sa 23:28  Atunci s-a întors Saul din urmarirea lui David ?i s-a dus în întâmpinarea Filistenilor, din care pricina s-a ?i numit locul acela: Sela-Hamahlecot (Stânca împar?irii).
1Sa 23:29  David însa, plecând de acolo, petrecea prin locurile neprimejdioase ale de?ertului Enghedi.
1Sa 24:1  Iar dupa ce s-a întors Saul de la Filisteni, i s-a spus, zicându-i-se: "Iata David este în pustiul Enghedi".
1Sa 24:2  Atunci a luat Saul trei mii de barba?i ale?i din tot Israelul ?i s-a dus sa caute pe David ?i oamenii lui pe stânci, unde locuiesc caprioarele;
1Sa 24:3  ?i a mers pâna la o stâna de oi, care era lânga drum; acolo era o pe?tera ?i Saul a intrat în ea pentru nevoile sale; David însa ?i oamenii lui ?edeau în fundul pe?terii.
1Sa 24:4  Atunci au zis catre David oamenii lui: "Aceasta este ziua de care îi-a vorbit Domnul, zicând: "Iata Eu voi da pe vrajma?ul tau în mâinile tale ?i vei face cu el ce vei vrea".
1Sa 24:5  David s-a sculat ?i a taiat înceti?or poala hainei de deasupra a lui Saul.
1Sa 24:6  Apoi a zis catre oamenii sai: "Sa ma fereasca Dumnezeu sa fac aceasta domnului meu, unsul Domnului, ?i sa-mi ridic mâna mea asupra lui, caci este unsul Domnului".
1Sa 24:7  ?i a?a a oprit David pe oamenii sai cu aceste cuvinte ?i nu i-a lasat sa se ridice asupra lui Saul. Iar Saul s-a sculat ?i a ie?it din pe?tera la drum.
1Sa 24:8  Apoi s-a sculat ?i David ?i, ie?ind din pe?tera, a strigat dupa Saul ?i a zis: "Domnul meu, rege!" Saul s-a uitat înapoi, iar David s-a aruncat cu fa?a la pamânt ?i i s-a închinat.
1Sa 24:9  Apoi a zis David catre Saul: "De ce ascul?i de vorbele oamenilor care zic: Iata David unelte?te rele împotriva ta?
1Sa 24:10  Iata, astazi vad ochii tai ca Domnul te-a dat acum în mâinile mele, aici în pe?tera, ?i mie mi s-a zis sa te ucid; eu însa te-am cru?at ?i am zis: Nu voi ridica mâna asupra domnului meu, pentru ca este unsul Domnului.
1Sa 24:11  Prive?te, parintele meu, poala hainei tale în mâinile mele; ?i-am taiat poala hainei, dar de ucis nu te-am ucis. Afla dar ?i te încredin?eaza ca nu este rau în mâna mea, nici vicle?ug ?i n-am gre?it cu nimic împotriva ta; tu însa cau?i sufletul meu ca sa-l iei.
1Sa 24:12  Sa judece dar Domnul între mine ?i între tine ?i sa ma razbune împotriva ta; dar mâna mea nu va fi asupra ta. Din nelegiui?i, nelegiui?i ies, dar mâna mea nu va fi asupra ta.
1Sa 24:13  Raul de la cel rau vine, zice vechea zicala. De aceea eu nu voi pune mâna pe tine.
1Sa 24:14  Asupra cui a ie?it regele lui Israel? Dupa cine alergi tu? Dupa un câine mort, dupa un purice.
1Sa 24:15  Domnul sa fie judecator ?i sa ne judece pe amândoi. El va cerceta, va descurca pricina mea ?i ma va izbavi din mâinile tale!"
1Sa 24:16  Dupa ce David a ispravit de vorbit cuvintele acestea catre Saul, Saul a zis: "Al tau e oare glasul acesta, fiul meu David? ?i ridicându-?i glasul, a plâns.
1Sa 24:17  ?i a zis catre David: "Tu e?ti mai drept decât mine, caci mi-ai rasplatit cu bine, iar eu te-am rasplatit cu rau;
1Sa 24:18  Tu astazi ai dovedit aceasta, purtându-te cu mine milostiv; când Domnul m-a dat în mâinile tale, tu nu m-ai omorât.
1Sa 24:19  Cine oare, prinzând pe vrajma?ul sau, i-ar da drumul sa mearga cu bine? Domnul sa-?i rasplateasca cu bine pentru ceea cea ai facut tu astazi cu mine!
1Sa 24:20  De acum ?tiu ca fara îndoiala vei domni ?i regatul lui Israel va fi tare în mâna ta.
1Sa 24:21  A?adar, jura-mi pe Domnul ca nu vei stârpi pe urma?ii mei ?i nu vei ?terge numele meu din casa tatalui meu".
1Sa 24:22  ?i s-a jurat David lui Saul. Apoi Saul s-a dus la casa sa, iar David ?i oamenii sai s-au suit în ni?te locuri întarite.
1Sa 25:1  În vremea aceea a murit Samuel ?i s-a adunat tot Israelul de l-au plâns ?i l-au îngropat în casa lui, în Rama. Iar David s-a sculat ?i s-a coborât în pustiul Maonului.
1Sa 25:2  În Maon era un om foarte bogat, care-?i avea turmele în Carmel; acesta avea trei mii de oi ?i o mie de capre ?i venise în Carmel la tunsul oilor sale. Numele omului aceluia era Nabal, iar numele femeii lui era Abigail.
1Sa 25:3  Femeia aceasta era foarte de?teapta ?i frumoasa la chip, iar el era om aspru ?i rau la narav; se tragea din neamul lui Caleb.
1Sa 25:4  Auzind David în pustiu ca Nabal î?i tunde oile în Carmel,
1Sa 25:5  A trimis zece tineri carora David le-a zis: "Sui?i-va în Carmel ?i merge?i la Nabal sa-i spune?i multa sanatate din partea mea!
1Sa 25:6  ?i-i zice?i a?a: "Sa trai?i! Pace ?ie! Pace casei tale! Pace la toate ale tale!
1Sa 25:7  Am auzit acum ca la tine se tund oile; iata pastorii tai  au fost cu noi ?i noi nu le-am facut nici un rau; nimic din ale lor nu s-a pierdut în tot timpul ?ederii lor în Carmel,
1Sa 25:8  Întreaba slugile ?i î?i vor spune. Sa afle dar baie?ii ace?tia bunavoin?a înaintea ochilor tai, ca la zi buna am venit. Da robilor tai ?i fiului tau David ce se îndura mâna ta"
1Sa 25:9  S-au dus deci oamenii lui David ?i au spus lui Nabal, în numele lui David, toate cuvintele acestea. Apoi au tacut. Nabal însa a sarit ?i a raspuns trimi?ilor lui David ?i a zis:
1Sa 25:10  "Cine este David ?i cine este fiul lui Iesei? Acum sunt o mul?ime de robi care fug de la stapânii lor.
1Sa 25:11  Nu cumva voi lua pâinile mele ?i apa mea ?i vinul meu ?i carnea pregatita pentru cei ce tund oile mele ?i sa le dau la ni?te oameni pe care nu-i ?tiu de unde sunt?"
1Sa 25:12  ?i s-au dus înapoi pe calea lor oamenii lui David ?i, întorcându-se, au venit ?i i-au spus toate vorbele acestea.
1Sa 25:13  Atunci David a zis oamenilor sai: "Încinge?i-va fiecare sabia!" ?i ?i-a încins fiecare sabia ?i a încins însu?i David sabia sa; ?i au plecat dupa David ca la patru sute de oameni, iar doua sute au ramas în tabara.
1Sa 25:14  Iar Abigail, femeia lui Nabal, fusese vestita de unul din slujitori, care-i zisese: "Iata David a trimis din pustie soli sa salute pe stapânul nostru, dar el s-a purtat cu el ca un netrebnic.
1Sa 25:15  Oamenii ace?tia sunt însa foarte buni cu noi, nu ne-au facut rau ?i nimic din ale noastre nu s-a pierdut în timpul cât am umblat cu ei când eram la câmp.
1Sa 25:16  Ei au fost pentru noi ca un zid de aparare ?i ziua ?i noaptea în timpul cât am pascut oile aproape de ei.
1Sa 25:17  A?adar gânde?te-te ?i vezi ce este de facut, caci de buna seama primejdia amenin?a pe stapânul nostru ?i toata casa lui; el însa este om rau ?i nu putem grai cu el".
1Sa 25:18  Atunci Abigail a luat repede doua sute de pâini, doua burdufuri de vin, cinci oi gatite, cinci masuri de graun?e prajite, o suta de legaturi de stafide ?i doua sute de legaturi de smochine ?i, încarcându-le pe asini,
1Sa 25:19  A zis slugilor sale: "Pleca?i înaintea mea, caci iata eu vin dupa voi". Iar barbatului sau, Nabal, nu i-a spus nimic.
1Sa 25:20  Când însa ea, ?ezând pe un asin, se cobora pe drumul ?erpuitor al muntelui, iata David ?i oamenii sai veneau în întâmpinarea ei ?i ea s-a întâlnit cu el.
1Sa 25:21  Atunci David a zis: "În zadar am aparat eu în pustiu toata averea acestui om, încât nimic nu s-î pierdut din cele ce erau ale lui ?i iata, el îmi plate?te cu rau pentru bine.
1Sa 25:22  A?a ?i a?a sa faca Dumnezeu cu robul Sau David ?i înca ?i mai mult sa faca, daca pâna mâine în revarsatul zorilor voi mai lasa pe cineva de parte barbateasca din tot ce are Nabal".
1Sa 25:23  Când Abigail a vazut pe David, s-a coborât repede de pe asina ?i a cazut înaintea lui David cu fa?a sa ?i i s-a închinat pâna la pamânt.
1Sa 25:24  Apoi a cazut la picioarele lui ?i a zis: "Asupra mea este pacatul, domnul meu! Îngaduie roabei tale sa vorbeasca urechilor tale ?i asculta cuvintele roabei tale.
1Sa 25:25  Sa nu întoarca domnul meu luarea aminte asupra acestui om rau, asupra lui Nabal; caci cum îi e numele, ?i nebunia se ?ine de el. Iar eu, roaba ta, n-am vazut pe tinerii domnului meu, pe care i-ai trimis.
1Sa 25:26  Dar acum, domnul meu, viu este Domnul ?i viu este sufletul tau, nu-?i va îngadui Domnul sa mergi la varsare de sânge ?i va înfrâna mâna ta de la razbunare.
1Sa 25:27  Vrajma?ii tai, cei ce planuiesc rele împotriva domnului meu, sa ajunga ca Nabal.
1Sa 25:28  Iarta vinova?ia roabei tale! Domnul negre?it va ridica casa tare domnului meu, caci lupta Domnului lupta domnul meu ?i rau nu se afla în tine în toata via?a ta.
1Sa 25:29  De se va scula vreun om sa te urmareasca ?i sa caute sufletul tau, atunci sufletul domnului meu va fi legat în manunchiul celor vii de lânga Domnul Dumnezeul tau, iar sufletul vrajma?ilor tai îl va arunca el ca dintr-o pra?tie.
1Sa 25:30  Iar când va face Domnul domnului meu tot binele ce l-a grait pentru tine ?i te va pune pova?uitor peste Israel,
1Sa 25:31  Atunci nu va fi pentru inima domnului meu amaraciune ?i nelini?te faptul ca n-a varsat în zadar sânge ?i s-a ferit de razbunare. Iar Domnul va face bine stapânului meu ?i-?i vei aduce aminte de roaba ta ?i-i vei arata mila".
1Sa 25:32  Atunci David a zis catre Abigail: "Binecuvântat fie Domnul Dumnezeul lui Israel, Care te-a trimis acum în întâmpinarea mea!
1Sa 25:33  Binecuvântata fie mintea ta ?i binecuvântata sa fii ?i tu, ca nu m-ai lasat acum sa merg la varsare de sânge ?i sa ma razbun:
1Sa 25:34  Dar viu este Domnul Dumnezeul lui Israel, Care m-a oprit de-a?i face ?ie rau; daca nu te-ai fi grabit sa vii întru întâmpinarea mea, apoi pâna mâine în revarsatul zorilor n-a? fi lasat lui Nabal pe nimeni de parte barbateasca".
1Sa 25:35  ?i a primit David din mâinile ei cele ce-i adusese ?i i-a zis: "Mergi sanatoasa la casa ta! Iata am ascultat glasul tau ?i ji-am cinstit fa?a".
1Sa 25:36  Dupa aceea a venit Abigail la Nabal ?i iata acesta avea ospa? în casa sa, ospa? împaratesc, ?i inima lui Nabal era vesela, caci bause de se îmbatase. De aceea nu i-a spus ea nici un cuvânt, nici mare, nici mic, pâna diminea?a.
1Sa 25:37  Iar diminea?a, când Nabal s-a trezit, femeia sa i-a spus toate cele ce se întâmplasera. Atunci a încremenit inima lui ?i el a ramas ca de piatra.
1Sa 25:38  Iar dupa vreo zece zile a lovit Domnul pe Nabal ?i acesta a murit.
1Sa 25:39  Auzind David ca Nabal a murit, a zis: "Binecuvântat fie Domnul, Care a rasplatit înjosirea ce mi-a pricinuit-o Nabal ?i a ferit pe robul Sau de la rau. Domnul a întors raul lui Nabal în capul lui". ?i a trimis David sa spuna Abigailei ca el o ia de so?ie.
1Sa 25:40  Deci au venit slujitorii lui David la Abigail în Carmel ?i i-au zis a?a: "David ne-a trimis la tine, ca sa te luam sa-i fii femeie".
1Sa 25:41  Atunci ea s-a sculat ?i s-a închinat cu fala pâna la pamânt ?i a zis: "Iata roaba ta este gata sa fie slujnica, ca sa spele picioarele slugilor domnului meu".
1Sa 25:42  ?i s-a gatit repede Abigail, s-a suit pe asin ?i cinci slujnice au înso?it-o; ?i s-a dus ea dupa trimi?ii lui David ?i a ajuns femeia lui.
1Sa 25:43  ?i a mai luat David ?i pe Ahinoam din Izreel ?i au fost amândoua femeile lui.
1Sa 25:44  Iar Saul a dat pe fiica sa Micol, femeia lui David, lui Paltiel, fiul lui Lai?, din Galim.
1Sa 26:1  Atunci au venit Zifeii de la miazazi la Saul în Ghibeea ?i au zis: "Iata David sta ascuns la noi, pe muntele Hachila, care vine în fa?a Ie?imonului".
1Sa 26:2  S-a sculat deci Saul ?i s-a coborât în pustiul Zif, cu trei mii de barba?i, israeli?i ale?i, ca sa caute pe David prin pustiul Zif.
1Sa 26:3  Saul ?i-a a?ezat tabara pe muntele Hachila, care vine în fa?a Ie?imonului, lânga drum; iar David se afla în pustiu ?i vedea ca Saul merge dupa el în pustiu.
1Sa 26:4  Deci a trimis David iscoade ?i a aflat ca Saul venise cu adevarat din Cheila.
1Sa 26:5  ?i sculându-se David pe ascuns, s-a dus la locul unde se a?ezase Saul cu tabara ?i a vazut David locul unde dormea Saul ?i Abner, fiul lui Ner, capetenia o?tirii lui. Saul însa dormea în cort, iar o?tenii erau a?eza?i împrejurul lui.
1Sa 26:6  Apoi întorcându-se, David a grait cu Ahimelec Heteul ?i cu Abi?ai, fiul lui ?eruia, fratele lui Ioab, ?i a zis: "Cine merge cu mine la Saul în tabara?" Raspuns-a Abi?ai: "Eu merg cu tine".
1Sa 26:7  ?i au venit David ?i Abi?ai la oamenii lui Saul, noaptea. Saul era culcat ?i dormea în cort; suli?a lui era înfipta în pamânt la capatâiul lui; iar Abner ?i o?tenii dormeau împrejurul lui.
1Sa 26:8  Atunci Abi?ai a zis lui David: "Acum Dumnezeu a dat pe vrajma?ul tau în mâinile tale; îngaduie-mi deci sa-l pironesc de pamânt cu suli?a dintr-o singura lovitura".
1Sa 26:9  David însa a zis catre Abi?ai: "Sa nu-l ucizi, caci cine-?i va ridica mâna asupra unsului Domnului ?i va ramâne nepedepsit?"
1Sa 26:10  Apoi David a zis: "Viu este Domnul! El dar sa-l loveasca; sau va veni ziua lui ?i va muri, sau va merge la razboi ?i va pieri; iar mie sa nu-mi îngaduie Domnul sa-mi ridic mâna asupra unsului Domnului!
1Sa 26:11  Dar ia suli?a lui, care este la capatâiul lui, ?i vasul cu apa ?i sa mergem într-ale noastre".
1Sa 26:12  ?i a luat David suli?a ?i vasul cu apa de la capatâiul lui Saul ?i s-au dus ei într-ale lor ?i nimeni nu i-a vazut, nici nu i-a sim?it; caci nimeni nu s-a de?teptat, ca Domnul trimisese peste ei somn adânc ?i dormeau to?i.
1Sa 26:13  Iar dupa ce David a trecut dincolo, a stat pe vârful muntelui, fiind acum departare mare între ei.
1Sa 26:14  A strigat David catre o?teni ?i catre Abner, fiul lui Ner ?i a zis: "Ei! Abner!" Iar Abner a zis: "Cine e?ti ?i de ce strigi sa-l tulburi pe rege?"
1Sa 26:15  A raspuns David lui Abner: "Au nu e?ti tu barbat ?i cine este asemenea ?ie în Israel? De ce dar nu paze?ti pe regele, stapânul tau? Caci a venit cineva din popor ca sa ucida pe regele, stapânul tau.
1Sa 26:16  Nu faci bine ce faci! Viu este Domnul, voi sunte?i vrednici de moarte, pentru ca nu pazi?i pe domnul vostru, unsul Domnului. Prive?te, unde este suli?a regelui ?i vasul cu apa care era la capul lui!"
1Sa 26:17  Saul însa a cunoscut glasul lui David ?i a zis: "Al tau este glasul acesta, fiul meu?" Iar David a raspuns: "Al meu, domnul meu, rege!"
1Sa 26:18  Apoi a adaugat: "Pentru ce urmare?te domnul meu pe robul sau? Ce-am facut eu? Sau ce rau este în mâna mea?
1Sa 26:19  Sa asculte dar regele, stapânul meu, cuvintele robului sau: Daca Domnul te-a îndemnat împotriva mea, sa fie aceasta din partea ta jertfa cu buna mireasma; iar daca fiii oamenilor te-au pus la cale, blestema?i sa fie ei înaintea Domnului, caci ei m-au izgonit acum, ca sa nu mai fac parte din mo?tenirea Domnului, zicând: Mergi de sluje?te la dumnezei straini.
1Sa 26:20  Sa nu se verse însa sângele meu pe pamânt înaintea Domnului; caci regele lui Israel a ie?it sa caute un purice, cum se cauta prepeli?ele pe dealuri".
1Sa 26:21  Iar Saul a zis: "Am gre?it! Întoarce-te, fiul meu David, ca nu-?i voi mai face rau, pentru ca sufletul meu a fost acum scump în ochii tai; nebune?te m-am purtat ?i am gre?it foarte mult".
1Sa 26:22  David însa a raspuns ?i a zis: "Iata suli?a regelui; sa vina unul din oameni ?i sa o ia,
1Sa 26:23  ?i sa-i dea fiecaruia Domnul dupa dreptatea lui ?i dupa credin?a, deoarece Domnul te-a dat pe tine astazi în mâinile mele, dar cu n-am vrut sa-mi ridic mâna mea asupra unsului Domnului;
1Sa 26:24  ?i precum a fost acum via?a ta pre?ioasa în ochii mei, a?a sa se pre?uiasca via?a mea în ochii Domnului ?i sa ma izbaveasca El de toata strâmtorarea ".
1Sa 26:25  Iar Saul a zis catre David: "Binecuvântat sa fii tu, fiul meu David; ?i sa lucrezi lucrul tau cu spor ?i sa-l isprave?ti cu bine". ?i s-a dus David în drumul sau, iar Saul s-a întors la locul lui.
1Sa 27:1  Atunci ?i-a zis David în inima sa: "Voi cadea cândva în mâinile lui Saul ?i nu-mi ramâne nimic mai bun decât sa fug în ?ara Filistenilor; astfel Saul va înceta de a ma mai cauta prin toate meleagurile lui Israel, iar eu voi scapa din mâna lui".
1Sa 27:2  S-a sculat deci David ?i a plecat el însu?i ?i cei ?ase sute de barba?i care erau cu el, la Achi?, fiul lui Maoc, regele Gatului.
1Sa 27:3  ?i a trait David la Achi? în Gat, el ?i oamenii lui, fiecare cu familia sa, David ?i amândoua femeile sale, Ahinoam izreeliteanca ?i Abigail carmeliteanca, fosta femeie a lui Nabal carmelitul.
1Sa 27:4  ?i i s-a spus lui Saul ca David a fugit în Gat ?i nu s-a mai apucat sa-l caute.
1Sa 27:5  David însa a zis catre Achi?: "De am aflat trecere înaintea ta, atunci sa mi se dea loc în una din ceta?ile ?arii ?i voi locui acolo; la ce sa locuiasca robul tau în cetatea regelui împreuna cu tine?"
1Sa 27:6  Atunci i-a dat Achi? ?iclagul ?i de aceea ?iclagul a ramas al regilor Iudei pâna astazi.
1Sa 27:7  Tot timpul cât a trait David în ?ara Filistenilor a fost un an ?i patru luni.
1Sa 27:8  În vremea aceea a ie?it David cu oamenii sai ?i au navalit asupra Ghe?urenilor, Ghirzenilor ?i Amaleci?ilor, care locuiau demult aceasta ?ara pâna la ?ur ?i pâna la ?ara Egiptului.
1Sa 27:9  ?i a pustiit David ?ara aceea ?i n-a lasat în via?a nici barbat, nici femeie; iar oile ?i boii, asinii, camilele ?i lucrurile le-a luat ?i, întorcându-se, a venit la Achi?.
1Sa 27:10  Iar Achi? a zis catre David: "Asupra cui ai navalit acum?" Raspuns-a David: "Asupra laturii de miazazi de Ierahmeel ?i asupra laturii de miazazi a Cheneilor".
1Sa 27:11  ?i nu a lasat David în via?a nici barbat, nici femeie, nici n-a adus în Gat, zicând: "Ace?tia ar putea sa ne pârasca ?i sa zica: A?a a fost David ?i astfel a fost purtarea lui în timpul ?ederii lui în ?ara Filistenilor".
1Sa 27:12  ?i s-a încrezut Achi? în David, zicând: "Acesta a ajuns sa fie urât poporului sau Israel ?i va fi pe veci sluga mea".
1Sa 28:1  În vremea aceea ?i-au adunat Filistenii o?tirea pentru razboi, ca sa se bata cu Israel. Deci a zis Achi? catre David: "?tiut sa-?i fie ca ai sa mergi cu mine la razboi ?i tu ?i oamenii tai".
1Sa 28:2  Iar David a raspuns lui Achi?: "Acum ai sa afli ce are sa faca robul tau". ?i a zis Achi? lui David: "De aceea te ?i fac eu paznicul capului meu pentru totdeauna".
1Sa 28:3  Murind Samuel, l-a plâns tot Israelul ?i l-au îngropat în Rama, cetatea lui. Saul însa izgonise pe cei ce chemau mor?ii ?i pe ghicitori din ?ara.
1Sa 28:4  S-au adunat deci Filistenii ?i s-au dus de ?i-au a?ezat tabara la ?unem; ?i-a adunat ?i Saul tot poporul lui Israel ?i ?i-a a?ezat tabara pe Ghilboa.
1Sa 28:5  Vazând însa Saul tabara Filistenilor, s-a spaimântat ?i s-a tulburat tare inima lui.
1Sa 28:6  ?i a întrebat Saul pe Domnul, dar Domnul nu i-a raspuns nici în vis, nici prin Urim, nici prin prooroci.
1Sa 28:7  Atunci Saul a zis slugilor sale: "Cauta?i-mi o femeie vrajitoare, ca sa merg la ea s-o întreb". Iar slugile i-au raspuns: "Este aici în Endor o femeie vrajitoare".
1Sa 28:8  Apoi ?i-a dezbracat Saul hainele sale ?i a îmbracat altele ?i s-a dus el însu?i cu doi oameni ?i au venit la femeie noaptea; ?i i-a zis Saul: "Rogu-te, ghice?te-mi chemând un mort ?i scoate-mi pe cine î?i voi spune eu!"
1Sa 28:9  Dar femeia i-a raspuns: "Tu ?tii ce a facut Saul, cum a alungat el din ?ara pe cei ce cheama mor?ii ?i pe ghicitori. Pentru ce dar întinzi tu cursa sufletului meu spre pieirea mea?"
1Sa 28:10  ?i s-a jurat Saul pe Domnul, zicând: "Viu este Domnul, nu vei suferi nici un necaz pentru fapta aceasta".
1Sa 28:11  Atunci femeia a întrebat: "Pe cine sa-?i scot?" Raspuns-a el: "Pe Samuel sa mi-l scoli!"
1Sa 28:12  Când a vazut femeia pe Samuel, a racnit tare. Apoi întorcându-se femeia catre Saul, a zis: "Pentru ce m-ai amagit? Tu e?ti Saul".
1Sa 28:13  I-a zis regele: "Nu te teme. Spune-mi ce vezi?" ?i raspunzând, femeia a zis: "Vad parca un dumnezeu, ie?ind din pamânt".
1Sa 28:14  "Ce înfa?i?are are?" a întrebat-o regele. Ea a raspuns: "Iese din pamânt un barbat foarte batrân, îmbracat cu o haina lunga". Atunci a cunoscut Saul ca acela este Samuel ?i a cazut cu fa?a la pamânt ?i s-a închinat.
1Sa 28:15  A zis Samuel catre Saul: "Pentru ce ma tulburi, ca sa ies?" Iar Saul a raspuns: "Îmi este tare greu; Filistenii se lupta împotriva mea, iar Dumnezeu S-a departat de mine ?i nu-mi mai raspunde nici prin prooroci, nici în vis, nici în vedenie; de aceea te-am chemat, ca sa ma înve?i ce sa fac".
1Sa 28:16  A zis Samuel: "La ce ma mai întrebi pe mine, daca Domnul S-a departat de tine ?i S-a facut vrajma?ul tau?
1Sa 28:17  Domnul face ceea ce a grait prin mine: Va lua Domnul domnia din mâinile tale ?i o va da lui David, aproapele tau.
1Sa 28:18  Deoarece tu n-ai ascultat glasul Domnului ?i n-ai împlinit iu?imea mâniei Lui asupra lui Amalec, de aceea Domnul face aceasta cu tine acum.
1Sa 28:19  ?i va da Domnul pe Israel împreuna cu tine în mâinile Filistenilor; mâine tu ?i fiii tai ve?i fi cu mine ?i tabara lui Israel o va da Domnul în mâinile Filistenilor".
1Sa 28:20  Atunci Saul a cazut deodata cu tot trupul sau la pamânt, caci se spaimântase grozav de cuvintele lui Samuel; afara de aceasta ?i puterile îl parasisera, caci nu mâncase pâine toata ziua aceea ?i toata noaptea.
1Sa 28:21  ?i s-a apropiat femeia aceea de Saul ?i vazând ca este tare înspaimântat, a zis: "Iata roaba ta a ascultat glasul tau ?i ?i-a pus via?a în primejdie ?i a împlinit porunca ce i-ai dat.
1Sa 28:22  Rogu-te dar acum, asculta ?i tu glasul roabei tale; î?i voi aduce o buca?ica de pâine; manânca, sa prinzi putere, ca sa pleci la drum".
1Sa 28:23  Dar el n-a voit, ci a zis: "Nu voi mânca!" ?i au început slugile lui sa-l îndemne, precum ?i femeia; ?i el a ascultat glasul lor ?i s-a sculat de la pamânt ?i a ?ezut pe pat.
1Sa 28:24  Iar femeia avea la casa ei ?i un vi?el îngra?at ?i s-a grabit sa-l taie; apoi, luând faina, a framântat ?i a copt azime.
1Sa 28:25  ?i a pus înaintea lui Saul ?i slugilor lui ?i ei au mâncat; apoi s-au sculat ?i au plecat în aceea?i noapte.
1Sa 29:1  Atunci ?i-au adunat Filistenii toate cetele la Afec, iar Israeli?ii ?i-au a?ezat tabara la fântâna cea din Izreel.
1Sa 29:2  Capeteniile Filistenilor mergeau cu sutele ?i cu miile lor; iar David ?i cu oamenii lui mergeau în urma cu Achi?.
1Sa 29:3  Capeteniile Filistenilor însa au zis: "Ce este cu Evreii ace?tia?" Achi? a raspuns capeteniilor Filistenilor: "Nu ?ti?i oare ca acesta este David, robul lui Saul, regele lui Israel? El este la mine de mai bine de un an ?i n-am gasit nimic rau la el de când a venit ?i pâna acum".
1Sa 29:4  ?i s-au mâniat pe el capeteniile Filistenilor: "Da drumul omului acestuia sa se duca sa ?ada la locul lui pe care i l-ai hotarât tu ?i sa nu mai mearga cu noi la razboi pentru ca sa nu se faca în razboi vrajma?ul nostru. Cu ce poate el sa dobândeasca mila domnului sau decât cu capetele acestor oameni?
1Sa 29:5  Nu este oare el acel David, caruia i se cânta la hora: Saul a biruit mii, iar David zeci de mii?"
1Sa 29:6  Atunci a chemat Achi? pe David ?i i-a zis: "Viu este Domnul! Tu e?ti om cinstit ?i ochilor mei le-ar fi placut ca tu sa intri ?i sa ie?i cu mine în tabere; caci eu n-am vazut rau la tine de când ai venit la mine ?i pâna în ziua aceasta; dar în ochii capeteniilor tu nu e?ti bun.
1Sa 29:7  Întoarce-te dar acum ?i mergi sanatos, ca sa nu ara?i pe capeteniile Filistenilor".
1Sa 29:8  David însa a zis catre Achi?: "Ce-am facut eu oare ?i ce-ai gasit tu la robul tau de când am venit înaintea fe?ei ?ale ?i pâna în ziua aceasta? Pentru ce sa nu merg ?i sa ma lupt cu vrajma?ii domnului meu, regele?"
1Sa 29:9  Raspuns-a Achi? lui David: "Fii încredin?at ca în ochii mei tu e?ti bun, ca un înger al lui Dumnezeu; dar capeteniile Filistenilor au zis: El sa nu mearga cu noi la razboi.
1Sa 29:10  A?adar scoala-te diminea?a, tu ?i robii stapânului tau, care au venit cu tine, ?i duce?i-va la locul pe care vi l-am rânduit eu ?i sa nu ai gând râu în inima ta. Scula?i-va dar diminea?a ?i, când se va lumina, pleca?i".
1Sa 29:11  ?i s-a sculat David, ?i oamenii lui, diminea?a ca sa plece ?i sa se întoarca în pamântul Filistenilor, iar Filistenii s-au dus la razboi în Israel.
1Sa 30:1  A treia zi dupa ce David ?i oamenii lui au plecat la ?iclag, Amaleci?ii au navalit din miazazi asupra ?iclagului ?i, luându-l, l-au ars cu foc.
1Sa 30:2  Iar femeile ?i pe to?i câ?i erau în el de la mic pâna la mare nu i-au ucis, ci i-au luat în robie ?i s-au dus în drumul lor.
1Sa 30:3  Când au ajuns David ?i oamenii sai Ia cetate, iata aceasta era arsa cu foc, iar femeile lor ?i fiii ?i fiicele erau du?i în robie.
1Sa 30:4  Atunci David ?i poporul ce era cu el au ridicat bocet ?i au plâns pâna când li s-au istovit puterile de plâns.
1Sa 30:5  ?i au fost duse în robie ?i amândoua femeile lui David, Ahinoam izreeliteanca ?i Abigail carmeliteanca, fosta femeie a lui Nabal.
1Sa 30:6  David a fost tare tulburat, deoarece poporul voise sa-l ucida cu pietre, caci poporul tot era amarât la suflet, fiecare pentru fiii sai ?i pentru fiicele sale.
1Sa 30:7  Dar David s-a întarit cu nadejdea în Domnul Dumnezeul sau ?i a zis catre preotul Abiatar, fiul lui Ahimelec: "Adu-mi efodul". ?i a adus Abiatar efodul la David.
1Sa 30:8  ?i a întrebat David pe Domnul zicând: "Sa urmaresc eu oare aceasta ceata? O voi ajunge oare?" ?i i s-a raspuns: "Urmare?te-o, o vei ajunge ?i-i vei lua prazile".
1Sa 30:9  Atunci s-a dus David el însu?i ?i cei ?ase sute de barba?i care erau cu el ?i, sosind la pârâul Besor, s-au oprit acolo cei osteni?i.
1Sa 30:10  David însa cu patru sute de oameni au urmarit înainte, iar doua sute de oameni s-au oprit, pentru ca n-au mai fost în stare sa treaca prin Besor.
1Sa 30:11  Atunci au gasit în câmp pe un egiptean ?i l-au adus la David, i-au dat pâine sa manânce ?i l-au adapat cu apa.
1Sa 30:12  I-au mai dat înca o jumatate de legatura de smochine, doua legaturi de stafide ?i a mâncat acela ?i s-a întarit, caci nu mâncase pâine ?i nu bause apa de trei zile ?i trei nop?i.
1Sa 30:13  Apoi David i-a zis: "Al cui ?i de unde e?ti tu?" Iar acela a zis: "Eu sunt fiu de egiptean, robul unui amalecit ?i m-a lepadat stapânul meu, pentru ca ma îmbolnavisem de vreo trei zile.
1Sa 30:14  Noi am navalit în latura de miazazi a Cheretienilor, în ?ara lui Iuda ?i în latura de miazazi a lui Caleb, iar ?iclagul l-am ars cu foc".
1Sa 30:15  I-a zis David: "Po?i sa ma duci pâna la aceasta ceata?" ?i el a zis: "Jura-mi pe Dumnezeu ca nu ma vei omorî ?i nu ma vei da în mâinile stapânului meu ?i eu te voi duce la aceasta ceata".
1Sa 30:16  David i s-a jurat ?i el l-a dus. ?i iata Amaleci?ii se risipisera prin toata ?ara aceea, mâncau, beau ?i jucau de bucurie pentru marea prada care o luasera ei din ?ara Filistenilor ?i din pamântul lui Iuda.
1Sa 30:17  Atunci a navalit asupra lor David ?i i-a macelarit din zori ?i pâna a doua zi seara ?i nimeni din ei n-a scapat, afara de patru sute de tineri care s-au suit pe camile ?i au fugit.
1Sa 30:18  Apoi a luat David toate câte rapisera Amaleci?ii ?i a luat ?i pe cele doua femei ale sale.
1Sa 30:19  ?i n-a pierit din al lor nimic, nici mare, nici mic, nici din fii, nici din fiice, nici din prazi; nici din tot ce luasera Amaleci?ii de la ei: toate le-a adus înapoi David.
1Sa 30:20  ?i a luat David toate vitele mari ?i mici ?i cei ce mânau turma aceasta ziceau: "Aceasta este prada lui David".
1Sa 30:21  Dupa aceea a venit David la cele doua sute de oameni care nu fusesera în stare sa-i urmeze ?i pe care îi lasase la pârâul Besor; ?i au ie?it ace?tia în întâmpinarea lui David ?i în întâmpinarea oamenilor care erau cu el. Iar David, apropiindu-se de acei oameni, i-a întrebat de sanatate.
1Sa 30:22  Atunci cei rai ?i netrebnici dintre oamenii cei ce mersesera cu David au început sa zica: "Pentru ca ei n-au mers cu noi, nu le vom da din prazile ce am luat, ci sa-?i ia fiecare numai femeia sa ?i copiii sai ?i sa plece".
1Sa 30:23  David însa a zis: "Sa nu face?i a?a cu fra?ii mei, acum, dupa ce Domnul ne-a dat noua acestea ?i ne-a pazit ?i a dat în mâinile noastre taberele celor ce venisera asupra noastra.
1Sa 30:24  ?i apoi cine va va asculta pe voi, în aceasta treaba? Ace?tia nu sunt mai rai decât noi. Ce parte au cei ce au mers la razboi, aceea?i parte trebuie sa se împarta ?i celorlal?i".
1Sa 30:25  A?a a fost totdeauna, din timpul acela ?i pâna acum, caci a pus aceasta ca lege ?i ca dreptar pâna în ziua de astazi.
1Sa 30:26  Apoi a venit David în ?iclag ?i a trimis din prazi batrânilor lui Iuda, prietenii sai:
1Sa 30:27  Celor din Betel, din Rama de miazazi ?i din Iatir;
1Sa 30:28  Celor din Aroer, din Amada, din Sifmot, din E?temoa ?i din Gat;
1Sa 30:29  Celor din Chimat, din Safec, din Timat, din Racal, din ceta?ile Ierahmeeli?ilor ?i din ceta?ile cheneene;
1Sa 30:30  Celor din Horma, din Cora?an ?i din Atac;
1Sa 30:31  Celor din Hebron ?i din toate locurile pe unde fusese David ?i oamenii lui, zicând: "Iata va trimit dar din prazile luate de la vrajma?ii Domnului".
1Sa 31:1  Filistenii însa s-au batut cu Israeli?ii ?i au fugit ace?tia de Filisteni ?i au cazut uci?i pe muntele Ghelboa.
1Sa 31:2  ?i au ajuns Filistenii pe Saul ?i pe fiii lui ?i au ucis pe Ionatan, pe Aminadab ?i pe Melchi?ua, fiii lui Saul.
1Sa 31:3  Lupta contra lui Saul ajunsese cumplita ?i arca?ii îl lovira pe acesta, ranindu-l greu.
1Sa 31:4  Atunci a zis Saul purtatorului sau de arme: Trage-?i sabia ?i ma strapunge cu ea, ca sa nu vina ace?ti netaia?i împrejur sa ma ucida ?i sa-?i bata joc de mine". Purtatorul de arme însa n-a voit, caci se temea cumplit. Atunci Saul ?i-a luat sabia ?i s-a aruncat în ea.
1Sa 31:5  Vazând purtatorul de arme ca Saul a murit, s-a aruncat ?i el în sabia sa ?i a murit cu el.
1Sa 31:6  A?a a murit în ziua aceea Saul ?i cei trei fii ai lui ?i purtatorul de arme al sau, precum ?i to?i oamenii lui.
1Sa 31:7  Israeli?ii, care locuiau peste vale ?i peste Iordan, vazând ca osta?ii israeli?i au fugit ?i ca Saul ?i fiii lui au murit, ?i-au parasit ceta?ile ?i au fugit, iar Filistenii au venit ?i s-au a?ezat în ele.
1Sa 31:8  A doua zi Filistenii au venit sa jefuiasca pe cei uci?i ?i au gasit pe Saul ?i pe cei trei fii ai lui cazu?i pe muntele Ghelboa.
1Sa 31:9  Ei i-au taiat capul, au luat armele de pe el ?i au trimis în toata ?ara Filistenilor ca sa se vesteasca despre aceasta în capi?tile idolilor lor ?i poporului.
1Sa 31:10  Armele lui le-au pus în capi?tea Astartei, iar trupul lui l-au spânzurat pe zidurile ceta?ii Bet-San.
1Sa 31:11  Auzind locuitorii Iabe?ului din Galaad cele ce facusera Filistenii cu Saul,
1Sa 31:12  S-au ridicat to?i oamenii puternici, au mers toata noaptea ?i au luat trupul lui Saul ?i trupurile fiilor lui de pe zidurile ceta?ii Bet-San ?i le-au adus în Iabe? ?i le-au ars acolo;
1Sa 31:13  Iar oasele lor le-au luat ?i le-au îngropat sub un stejar în Iabe?. Apoi au postit ?apte zile.


\end{document}