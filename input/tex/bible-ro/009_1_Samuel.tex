\begin{document}

\title{1 Samuel}


\chapter{1}

\par 1 Era în vremea aceea un om la Ramataim-Țofim, în Muntele Efraim, cu numele Elcana, fiul lui Ieroham, fiul lui Elihu, fiul lui Tohu, fiul lui Țuf Efraimitul.
\par 2 Omul acela avea două femei: numele uneia era Ana și numele celeilalte era Penina. Penina avea copii, iar Ana nu avea copii.
\par 3 Omul acela se ducea în fiecare an din cetatea sa, la Șilo, să se închine și să aducă jertfă Domnului Savaot; acolo însă erau preoți ai Domnului cei doi fii ai lui Eli: Ofni și Finees.
\par 4 În ziua când Elcana aducea jertfă dădea parte Peninei, femeii sale și tuturor fiilor și fiicelor ei;
\par 5 Iar Anei îi dădea parte îndoită, deși aceasta nu avea copii, pentru că el iubea pe Ana mai mult decât pe Penina, căci Domnul închisese pântecele ei.
\par 6 Potrivnica ei însă o amăra grozav, ațâțând-o ca să cârtească din pricină că nu i-a dat Domnul prunci.
\par 7 Așa se întâmpla în fiecare an, când mergea ea la casa Domnului: aceea o amăra, iar aceasta plângea și se tânguia și nu mânca.
\par 8 Dar iată o dată Elcana, bărbatul său, i-a zis: "Ana!" Și ea a răspuns: "Iată-mă!" A zis Elcana: "Ce plângi și de ce nu mănânci și pentru ce e întristată inima ta? Nu sunt eu oare pentru tine mai bun decât zece copii?"
\par 9 Atunci Ana, după ce au mâncat și au băut ei în Șilo, s-a sculat și a stat înaintea Domnului. Iar preotul Eli ședea atunci pe scaun la ușa cortului Domnului.
\par 10 Ea însă s-a rugat Domnului cu sufletul întristat și a plâns amarnic,
\par 11 Și a dat făgăduință, zicând: "Atotputernice Doamne, Dumnezeule Savaot, de vei căuta la întristarea roabei Tale și-ți vei aduce aminte de mine și de nu vei uita pe roaba Ta, ci vei da roabei Tale un copil de parte bărbătească, îl voi da ție, și nu va bea el nici vin, nici sicheră, nici brici nu se va atinge de capul lui".
\par 12 Dar pe când se ruga ea așa îndelung înaintea Domnului, Eli privea la gura ei;
\par 13 Și fiindcă Ana vorbea în inima sa, iar buzele ei numai se mișcau, dar glasul nu i se auzea, Eli socotea că ea e beată.
\par 14 De aceea i-a și zis Eli: "Până când ai să stai aici beată? Trezește-te și te du de la fața Domnului!"
\par 15 Răspunzând însă Ana a zis: "Nu, domnul meu! Eu sunt o femeie cu inima întristată; nici vin, nici sicheră n-am băut, ci îmi dezvălui sufletul meu înaintea Domnului.
\par 16 Să nu socoți pe roaba ta femeie netrebnică, căci din durerea mea cea mare și din întristarea mea am vorbit până acum".
\par 17 Atunci Eli i-a răspuns și i-a zis: "Mergi în pace și Dumnezeul lui Israel să-ți plinească cererea pe care I-ai făcut-o!"
\par 18 Iar ea i-a zis: "Să afle roaba ta trecere înaintea ochilor tăi!" Apoi s-a dus ea în calea sa și a mâncat și fața nu-i mai era tristă ca mai înainte.
\par 19 Iar dimineața s-au sculat ei și s-au închinat înaintea Domnului și, întorcându-se, au venit la casa lor în Rama. După aceea a cunoscut Elcana pe Ana, femeia sa, și și-a adus aminte Domnul de ea.
\par 20 După câtva timp a zămislit Ana și a născut un fiu și i-a pus numele Samuel, căci își zicea ea: "De la Domnul Dumnezeul Savaot l-am cerut!"
\par 21 Și s-a dus Elcana cu toată familia lui la Șilo să aducă jertfă Domnului, după făgăduințele sale, și toate zeciuielile de la pământul său.
\par 22 Ana însă nu s-a dus cu el, spunând bărbatului său: "Când pruncul va fi înțărcat de la sân și va crește, atunci am să-l duc și va fi înfățișat el înaintea Domnului și va rămâne acolo pe totdeauna".
\par 23 Iar Elcana, bărbatul ei, a zis către ea: "Fă cum ți se pare că este bine; rămâi până-l vei înțărca; dar să întărească Domnul cuvântul ce a ieșit din gura ta". Și a rămas femeia aceasta și a alăptat pruncul până l-a înțărcat.
\par 24 Iar după ce l-a înțărcat, s-a dus cu el la Șilo, luând trei viței, câteva pâini, o efă de făină și un burduf de vin, și a venit la casa Domnului în Șilo și a venit și copilul împreună cu ei, dar copilul era încă prunc.
\par 25 Și l-au adus înaintea feței Domnului, iar tatăl său a adus jertfa rânduită în asemenea zile și au junghiat un vițel. Ana a adus pe prunc la Eli,
\par 26 Zicând: "O, domnul meu!, viu să fie sufletul tău, domnul meu! Eu sunt acea femeie care am stat aici înaintea ta și m-am rugat Domnului.
\par 27 Pentru acest copil m-am rugat eu și Domnul mi-a plinit cererea ce am cerut de la Dânsul.
\par 28 Și acum eu îl dau Domnului pentru toate zilele vieții lui, să slujească Domnului". Apoi s-au închinat acolo Domnului.

\chapter{2}

\par 1 S-a rugat deci Ana și a zis: "Bucuratu-s-a inima mea întru Domnul, înălțată a fost fruntea mea de Domnul Dumnezeul meu și gura mea s-a deschis larg asupra vrăjmașilor mei, căci m-am bucurat de izbăvirea Ta.
\par 2 Nimeni nu este sfânt ca Domnul, căci nu e altul afară de Tine; și nimeni nu e puternic ca Dumnezeul nostru.
\par 3 Nu vă lăudați și cuvinte trufașe să nu iasă din gura voastră, căci Domnul este Dumnezeul a toată cunoștința și lucrurile la Dânsul sunt cântărite.
\par 4 Arcul celor puternici s-a frânt, iar cei slabi s-au încins cu putere.
\par 5 Cei sătui vor munci pentru pâine, iar cei flămânzi nu vor mai avea foame. Cea stearpă va naște de șapte ori, iar cea cu copii mulți va fi neputincioasă.
\par 6 Domnul omoară și învie; El coboară la locuința morților și iarăși scoate.
\par 7 Domnul sărăcește pe om și tot El îl îmbogățește; El smerește și El înalță. El ridică pe cel sărac din pulbere și din gunoi pe cel lipsit, punându-i în rând cu cei puternici și dându-le scaunul măririi, căci ale Domnului sunt temeliile pământului și El întemeiază lumea pe ele.
\par 8 Pașii sfinților Săi El îi păzește, iar nelegiuiții vor pieri întru întuneric căci omul nu prin putere este tare.
\par 9 Domnul va zdrobi pe cei ce se împotrivesc Lui; va tuna din cer asupra lor. Sfânt este Domnul; să nu se laude cel înțelept cu înțelepciunea sa și cel puternic să nu se laude cu puterea sa, nici cel bogat să nu se fălească cu bogăția sa; ci cel ce voiește să se laude, cu aceea să se laude că știe și cunoaște pe Domnul și face judecată și dreptate în mijlocul pământului.
\par 10 Domnul din înălțimea cerului va tuna peste vrăjmașii Săi, El va judeca marginile pământului, drept fiind, El va da tărie regilor noștri și fruntea Unsului Său o va înălța".
\par 11 Și au lăsat pe Samuel acolo, înaintea Domnului. Apoi Elcana s-a dus la casa sa în Rama, iar copilul a rămas să slujească Domnului sub povața preotului Eli.
\par 12 Însă fiii lui Eli erau oameni netrebnici. Ei nu știau de Domnul,
\par 13 Nici de datoria preoțească față de popor. Când aducea cineva jertfă, feciorul preotului venea în timpul când se fierbea carnea cu o furculiță în mână,
\par 14 O vâra în căldare, sau în blid, sau în tavă, sau în oală, și ce prindea furculița, aceea lua feciorul preotului. Așa făceau ei cu toți Israeliții care veneau acolo în Șilo.
\par 15 Chiar și înainte de a arde grăsimea venea feciorul preotului și zicea către cel ce aducea jertfă: "Dă carne de friptură pentru preot, căci el n-are să ia de la tine carne fiartă, ci dă-i-o crudă".
\par 16 Și dacă cineva zicea: "Lasă să se ardă mai întâi grăsimea, cum se cuvine, și apoi îți vei lua cât îți va pofti sufletul", atunci el răspundea: "Ba nu, dă-mi chiar acum, iar de nu, voi lua cu de-a sila".
\par 17 și păcatul acestor tineri era foarte mare înaintea Domnului, căci ei depărtau lumea de a mai aduce jertfă Domnului.
\par 18 Iar copilul Samuel slujea înaintea Domnului, îmbrăcat cu efod de in.
\par 19 Meilul de deasupra i-l făcea mama sa și i-l aducea în fiecare an, când venea cu bărbatul ei să aducă jertfa rânduită.
\par 20 Și a binecuvântat Eli pe Elcana și pe femeia lui și a zis: "Să-ți dea ție Domnul copii de la femeia aceasta în locul celui afierosit, pe care tu l-ai dăruit Domnului!" Și s-au dus ei la casa lor.
\par 21 După aceea a cercetat Domnul pe Ana și ea a zămislit și a mai născut trei băieți și două fete, iar copilul Samuel creștea înaintea Domnului.
\par 22 Eli însă era tare bătrân și auzea de toată purtarea fiilor săi față de întreg Israelul și că se culcau cu femeile celor ce se adunau la ușa cortului adunării.
\par 23 Și le-a zis: "Pentru ce faceți voi asemenea lucruri, căci aud vorbe rele despre voi de la tot poporul Domnului?
\par 24 Nu, copiii mei, nu este bună vestea ce o aud eu despre voi; nu mai faceți așa, căci nu este bună vestea care o aud eu; voi răzvrătiți poporul Domnului.
\par 25 Că de va greși omul față de alt om, atunci se vor ruga pentru el lui Dumnezeu; iar de va păcătui cineva față de Domnul, atunci cine va mijloci pentru el?" Dar ei nu ascultau spusele tatălui lor, căci Domnul hotărâse să-i dea morții.
\par 26 Iar copilul Samuel creștea mereu cu vârsta și era iubit și de Dumnezeu și de oameni.
\par 27 În vremea aceea a venit la Eli un om al lui Dumnezeu și i-a zis: "Așa grăiește Domnul: Nu M-am arătat Eu oare casei tatălui tău când erau ei încă în Egipt, în casa lui Faraon?
\par 28 Și nu l-am ales Eu oare din toate semințiile lui Israel să-Mi fie preot, ca să aprindă tămâie și să poarte efod înaintea Mea? Și nu am dat Eu oare casei tatălui tău din toate jertfele ce se frig pe foc de fiii lui Israel?
\par 29 Pentru ce dar voi călcați în picioare jertfele Mele și prinoasele Mele de pâine, pe care le-am poruncit Eu pentru locașul Meu, și pentru ce tu ții mai mult la fiii tăi decât la Mine, îngrășându-i pe ei cu toată pârga poporului Meu Israel?
\par 30 De aceea așa zice Domnul Dumnezeul lui Israel: Eu am zis odinioară: Casa ta și casa tatălui tău va umbla nestrămutat înaintea feței Mele în veac; dar acum Domnul zice: Să nu mai fie așa, căci Eu preamăresc pe cei ce Mă preaslăvesc pe Mine, iar cei ce Mă necinstesc vor fi rușinați.
\par 31 Iată vin zile când Eu voi tăia brațul tău și brațul casei tatălui tău, încât să nu mai fie bătrân în casa ta niciodată.
\par 32 Și tu vei vedea casa Mea ca un dușman pentru tine, deși Domnul va milui pe Israel și nu va mai fi în casa ta bătrân în toate zilele.
\par 33 Nu voi depărta pe toți ai tăi de la jertfelnicul Meu, ca să chinuiesc ochii tăi și să apăs sufletul tău; dar toți urmașii casei tale vor muri la mijlocul anilor lor.
\par 34 Și iată un semn pentru tine, care se va petrece cu cei doi fii ai tăi, Ofni și Finees: amândoi vor muri în aceeași zi.
\par 35 Și-Mi voi pune un preot credincios. Acela se va purta după inima Mea și după sufletul Meu; și casa lui o voi întări și va umbla el înaintea Unsului Meu în toate zilele;
\par 36 Și tot cel rămas din casa ta va veni să se închine lui pentru o gheră de argint și pentru o bucățică de pâine, și va zice: "Numără-mă la vreuna din slujbele levitice, ca să pot mânca o bucată de pâine!"

\chapter{3}

\par 1 Și pruncul Samuel slujea Domnului sub povața preotului Eli. În zilele acelea cuvântul Domnului era rar și nici vedeniile nu erau dese.
\par 2 Și iată în vremea aceea, când Eli stătea culcat la locul său și ochii lui începuseră a se închide și nu mai putea să vadă;
\par 3 Când sfeșnicul Domnului nu se stinsese încă și Samuel era culcat în cortul Domnului, unde era chivotul lui Dumnezeu,
\par 4 A strigat Domnul către Samuel: "Samuele, Samuele!" Iar el a răspuns: " Iată-mă! "
\par 5 Și a alergat la Eli și a zis: "Iată-mă! La ce m-ai chemat?" Acela însă a răspuns: "Nu te-am chemat. Du-te și te culcă!" Și s-a dus Samuel și s-a culcat.
\par 6 Iar Domnul a strigat a doua oară pe Samuel: "Samuele, Samuele!" și acesta s-a sculat și a venit iar la Eli și a zis: "Iată-mă! De ce m-ai chemat?" și acela i-a zis: "Nu te-am chemat, fiul meu! Du-te înapoi și te culcă!"
\par 7 Samuel nu cunoștea atunci glasul Domnului și cuvântul Domnului nu i se descoperise încă.
\par 8 Dar Domnul a strigat pe Samuel și a treia oară. Și s-a sculat acesta și a venit la Eli și a zis: "Iată-mă! La ce m-ai chemat?" Atunci a înțeles Eli că Domnul cheamă pe copil.
\par 9 Și a zis Eli către Samuel: "Du-te înapoi și te culcă și când Cel ce te cheamă te va mai chema, tu să zici: "Vorbește, Doamne, că robul Tău ascultă!" Și s-a dus Samuel și s-a culcat la locul său.
\par 10 Și a venit Domnul și a stat și a strigat ca întâia și ca a doua oară: "Samuele, Samuele!" Iar Samuel a zis: "Vorbește, Doamne, că robul Tău ascultă!"
\par 11 A zis Domnul către Samuel: "Iată, am să fac în Israel un lucru, încât cine va auzi de el, aceluia îi vor țiui amândouă urechile.
\par 12 În ziua aceea voi face cu Eli toate câte am spus despre casa lui; toate le voi începe și le voi sfârși.
\par 13 Eu i-am spus că am să pedepsesc casa lui pe veci pentru vina pe care el a știut-o, și anume că fiii fac nelegiuiri, dar nu i-a înfrânat.
\par 14 De aceea mă jur casei lui Eli că vina casei lui Eli nu se va șterge, nici prin jertfe, nici prin prinoase de pâine în veci".
\par 15 După aceea a adormit Samuel până dimineața, s-a sculat de noapte și a deschis ușile casei Domnului. Dar Samuel s-a temut să spună lui Eli vedenia aceasta.
\par 16 Eli însă a chemat pe Samuel și a zis: "Fiul meu Samuel! " Iar acesta a răspuns: "Iată-mă!"
\par 17 A zis Eli: "Ce ți s-a spus? Să nu ascunzi de mine! Dumnezeu să se poarte cu tine cu toată asprimea, dacă tu vei ascunde ceva de mine din toate câte ți s-au spus!"
\par 18 Și i-a spus Samuel tot și n-a ascuns nimic de el. Atunci Eli a zis: "El este Domnul; facă dar ce va binevoi!"
\par 19 Și a crescut Samuel și Domnul a fost cu el și n-a rămas neîmplinit nici unul din cuvintele Lui.
\par 20 Atunci a aflat tot Israelul, de la Dan până la Beer-Șeba, că Samuel s-a învrednicit să fie prooroc al Domnului.
\par 21 Și a urmat Domnul a Se arăta în Șilo după ce Se arătase lui Samuel acolo prin cuvântul Său. Și s-au încredințat toți în tot Israelul, de la o margine până la cealaltă margine a țării, că Samuel este proorocul Domnului. Iar Eli a ajuns foarte bătrân, și feciorii lui stăruiau pe calea lor înaintea Domnului.

\chapter{4}

\par 1 În vremea aceea s-au sculat Filistenii să se războiască cu Israeliții și a fost cuvântul lui Samuel către tot Israelul. Și au pășit Israeliții împotriva Filistenilor cu război și au tăbărât la Eben-Ezer, iar Filistenii au tăbărât la Afec.
\par 2 Apoi Filistenii s-au așezat în rânduială de război în fața Israeliților și, dându-se bătălia, au fost bătuți Israeliții de către Filisteni, care au ucis pe câmpul de luptă ca la patru mii de oameni.
\par 3 După aceea au venit oamenii în, tabără și au spus bătrânilor lui Israel: "Pentru ce oare ne-a lovit pe noi Domnul înaintea Filistenilor? Să luăm cu noi din Șilo chivotul legii Domnului, ca să meargă în mijlocul nostru și să ne izbăvească din mâinile vrăjmașilor noștri!"
\par 4 Și a trimis poporul la Șilo, de au adus de acolo chivotul legii Domnului Savaot, Cel ce șade pe heruvimi; iar pe lângă chivotul legii Domnului erau și cei doi fii ai lui Eli: Ofni și Finees.
\par 5 Iar dacă a sosit chivotul legii Domnului în tabără, tot Israelul a ridicat strigăt așa de mare, încât s-a cutremurat pământul.
\par 6 Și auzind Filistenii răsunetul strigătelor, au zis: "Ce înseamnă aceste strigăte puternice în tabăra Evreilor?" Și au aflat că a sosit în tabără chivotul legii Domnului.
\par 7 Atunci s-au înspăimântat Filistenii, căci ziceau: "Dumnezeul lor a venit la ei în tabără". Apoi au zis: "Vai de noi! Căci n-a mai fost asemenea lucru nici ieri, nici alaltăieri!
\par 8 Vai de noi! Cine ne va scăpa din mâinile acestui Dumnezeu puternic? Acesta este acel Dumnezeu Care a bătut pe Egipteni cu tot felul de pedepse în pustiu.
\par 9 Întăriți-vă și fiți curajoși, Filistenilor, ca să nu cădeți în robie la Evrei, cum sunt ei în robie la noi! Fiți curajoși și vă luptați cu ei!"
\par 10 Și s-au luptat Filistenii cu Israeliții și au fost înfrânți aceștia, și a fugit fiecare în cortul său; bătălia aceasta a fost foarte mare, căzând dintre Israeliți treizeci de mii de pedestrași.
\par 11 Și a fost luat chivotul legii Domnului, iar cei doi fii ai lui Eli, Ofni și Finees, au murit.
\par 12 Atunci a alergat un veniaminean de la locul bătăliei și a venit la Șilo în aceeași zi, având hainele de pe el rupte și pulbere pe cap.
\par 13 Iar când a venit acela, Eli ședea pe scaun lângă drum la poartă și privea, căci i se bătea inima pentru chivotul lui Dumnezeu. Și după ce a sosit omul acela și a spus în cetate, atunci s-a tânguit strașnic toată cetatea.
\par 14 Auzind Eli răsunetele tânguirilor, a întrebat: "Pentru ce este acest bocet?" Dar a sosit îndată omul acela și i-a spus toate lui Eli.
\par 15 Eli însă era atunci de nouăzeci și opt de ani; ochii i se întunecaseră și nu mai putea să vadă.
\par 16 Și a zis omul acela către Eli: "Eu vin din tabără. Chiar astăzi am fugit de pe câmpul de luptă". Iar Eli a zis: "Ce s-a întâmplat, fiul meu?" Vestitorul însă a răspuns și a zis: "Israeliții au fugit din fața Filistenilor și s-a făcut în popor ucidere mare; amândoi fiii tăi, Ofni și Finees, au murit și chivotul lui Dumnezeu a fost luat".
\par 17 Când a pomenit el de chivotul Domnului, Eli a căzut de pe scaun pe spate la poartă, și-a rupt spinarea și a murit, căci era bătrân și greoi.
\par 18 El a fost judecător în Israel patruzeci de ani.
\par 19 Iar nora lui, femeia lui Finees, era însărcinată și aproape de naștere. Când a auzit vestea despre luarea chivotului Domnului și despre moartea socrului său și a bărbatului său, a căzut în genunchi și a născut, căci o apucaseră durerile ei.
\par 20 Și pe când murea ea, femeile care stăteau împrejur îi ziseră: "Nu te teme, că ai născut băiat". Ea însă nu a răspuns și nu a dat semn de luare aminte.
\par 21 Și au pus copilului numele: Icabod, zicând: "S-a dus slava din Israel, din pricina pierderii chivotului Domnului și a morții socrului și a bărbatului ei.
\par 22 Și a zis ea: "S-a dus slava din Israel, căci s-a luat chivotul Domnului!

\chapter{5}

\par 1 Atunci au luat Filistenii chivotul Domnului și l-au dus din Eben-Ezer la Așdod.
\par 2 Apoi au ridicat Filistenii chivotul Domnului și l-au vârât în capiștea lui Dagon și l-au pus lângă Dagon.
\par 3 Iar a doua zi s-au sculat Așdodenii dis-de-dimineață și iată Dagon zăcea cu fața la pământ înaintea chivotului Domnului. Și au luat ei pe Dagon și l-au pus iar la locul lui.
\par 4 Și s-au sculat ei dis-de-dimineață în ziua următoare, și iată Dagon zăcea cu fața la pământ înaintea chivotului Domnului; dar capul lui Dagon și amândouă picioarele și mâinile lui zăceau tăiate pe prag, fiecare deosebi, și rămăsese numai trunchiul lui.
\par 5 De aceea preoții lui Dagon și toți câți vin în capiștea lui Dagon din Așdod nu calcă pe pragul lui Dagon până în ziua de astăzi, ci pășesc peste el.
\par 6 Și a apăsat mâna Domnului asupra Așdodenilor și i-a lovit și i-a pedepsit cu bube usturătoare pe cei din Așdod și din împrejurimile lui, iar înăuntrul țării s-au înmulțit șoarecii și s-a lățit în cetate deznădejde mare.
\par 7 Văzând aceasta, Așdodenii au zis: "Să nu mai rămână chivotul Dumnezeului lui Israel la noi, că e grea mâna Lui și pentru noi și pentru Dagon, dumnezeul nostru!"
\par 8 Apoi au trimis și au adunat la ei pe toți mai-marii Filistenilor și le-au zis: "Ce să facem cu chivotul Dumnezeului lui Israel?" Iar Gateenii au zis: "Să treacă dar chivotul Dumnezeului lui Israel la noi în Gat". Și au trimis la Gat chivotul Dumnezeului lui Israel.
\par 9 Iar după ce l-au trimis, a fost mâna Domnului asupra cetății aceleia cu strășnicie mare și a bătut Domnul pe locuitorii cetății de la mic până la mare și s-au ivit pe ei buboaie.
\par 10 De aceea au trimis chivotul Domnului la Ecron. Iar când a sosit chivotul Domnului în Ecron au strigat Ecronenii și au zis: "Chivotul Dumnezeului lui Israel a venit la noi ca să omoare și pe poporul nostru".
\par 11 Atunci au trimis și au adunat toate căpeteniile Filistenilor și au zis: "Trimiteți de aici chivotul Dumnezeului lui Israel; lăsați-l să se întoarcă la locul său, ca să nu ne ucidă pe noi și pe poporul nostru". Căci groază de moarte era în tot orașul și mâna Domnului apăsa foarte tare asupra lor, de cum venise acolo chivotul Dumnezeului lui Israel.
\par 12 Și aceia care nu muriseră fuseseră loviți cu buboaie, așa încât plânsetele cetății se înălțau până la cer.

\chapter{6}

\par 1 Și a stat chivotul Domnului în țara Filistenilor șapte luni și s-a umplut țara aceea de șoareci.
\par 2 Atunci au adunat Filistenii pe preoți, pe ghicitori și pe descântători și au zis: "Ce să facem oare cu chivotul Domnului? Învățați-ne cum să-l trimitem la locul lui?"
\par 3 Iar aceia au zis: "Dacă voiți să trimiteți chivotul legii Domnului Dumnezeului lui Israel, să nu-l trimiteți fără nimic, ci aduceți-I jertfă pentru păcat, și atunci vă veți vindeca și veți afla pentru ce nu s-a îndepărtat de la voi mâna Lui".
\par 4 Apoi au mai zis: "Ce fel de jertfă pentru păcat trebuie să-I aducem?" Iar aceia au zis: "După numărul căpeteniilor Filistenilor, cinci buboaie de aur și cinci șoareci de aur, căci pedeapsa este una și asupra voastră și asupra căpeteniilor voastre.
\par 5 Așadar, faceți niște chipuri cioplite de buboaie de ale voastre și niște chipuri de șoareci de ai voștri care pustiesc pământul și dați slavă Dumnezeului lui Israel; poate că Își va ridica mâna de deasupra voastră, de deasupra dumnezeilor voștri și de deasupra pământului vostru.
\par 6 De ce să vă învârtoșați inimile voastre, cum și-au învârtoșat inimile Egiptenii și Faraon? Iată când Domnul Și-a arătat puterea Sa asupra lor, atunci ei i-au lăsat și aceia au plecat.
\par 7 Luați dar și faceți un car nou și luați două vaci care au fătat întâia oară, care n-au mai purtat jug, și înjugați vacile la car, iar vițeii lor duceți-i acasă.
\par 8 Apoi luați chivotul Domnului și-l puneți în car, iar lucrurile cele de aur care I se aduc jertfă pentru păcat, să le puneți într-o lădiță alături, și dați-i drumul să se ducă.
\par 9 Să vă uitați însă: Dacă el va pleca spre hotarele sale, spre Betșemeș, atunci acest mare rău ni l-a făcut El; dacă nu va porni într-acolo, atunci vom ști că nu ne-a lovit mâna Lui, ci aceasta ne-a venit din întâmplare".
\par 10 Și au făcut ei așa: au luat două vaci care au fătat întâia oară și le-au înjugat la car, iar vițeii i-au oprit acasă;
\par 11 Apoi au pus chivotul Domnului în car, iar lădița cu șoarecii cei de aur și cu chipurile cioplite în chip de buboaie au pus-o alături de chivot.
\par 12 Și au pornit vacile de-a dreptul pe drumul spre Betșemeș; și au ținut calea mereu înainte, mugind, dar neoprindu-se și neabătându-se nici la dreapta nici la stânga; iar căpeteniile Filistenilor au mers în urma lor până la hotarele Betșemeșului.
\par 13 Tocmai atunci locuitorii Betșemeșului secerau grâul în vale; și ridicându-și ochii, au văzut chivotul Domnului și s-au bucurat când l-au văzut.
\par 14 Carul însă a venit în țarina lui Iosua din Betșemeș și s-a oprit acolo. și se afla acolo o piatră mare; au despicat lemnele carului, iar vacile au fost aduse ardere de tot Domnului.
\par 15 Apoi leviții au ridicat chivotul Domnului și lădița cea de lângă el în care se aflau lucrurile cele de aur și le-au pus pe piatra cea mare; iar locuitorii din Betșemeș au adus în ziua aceea arderi de tot și au junghiat jertfe Domnului.
\par 16 Și cele cinci căpetenii ale Filistenilor, după ce au văzut aceasta, s-au întors în aceeași zi la Ecron.
\par 17 Iar buboaiele cele de aur pe care le-au adus Filistenii jertfă Domnului pentru păcat erau: unul pentru Așdod, unul pentru Gaza, unul pentru Ascalon, unul pentru Gat, unul pentru Ecron.
\par 18 Iar șoarecii de aur erau după numărul tuturor cetăților filistene ale celor cinci căpetenii, de la cetățile întărite până la satele deschise, până la piatra cea mare pe care s-a pus chivotul Domnului și care se află până în ziua de astăzi în țarina lui Iosua din Betșemeș.
\par 19 Dar dintre oamenii din Betșemeș nu s-au bucurat fiii lui Iehonia că au văzut chivotul Domnului. și a lovit Domnul pe locuitorii din Betșemeș, pentru că ei s-au uitat la chivotul Domnului, și a ucis din popor cincizeci de mii șaptezeci de oameni. Atunci a plâns poporul, pentru că lovise Domnul poporul cu pedeapsă mare.
\par 20 Au zis locuitorii din Betșemeș: "Cine poate să stea înaintea Domnului, a Acestui Dumnezeu sfânt? Și la cine se va duce El de la noi?"
\par 21 Deci au trimis soli la locuitorii din Chiriat-Iearim să le spună: "Filistenii au întors chivotul Domnului; veniți și-l luați la voi!"

\chapter{7}

\par 1 Atunci au venit locuitorii din Chiriat-Iearim și au luat chivotul Domnului și l-au adus în casa lui Aminadab, pe deal, iar pe Eleazar, fiul lui, l-au sfințit ca să păzească chivotul Domnului.
\par 2 Din ziua aceea, de când a rămas chivotul în Chiriat-Iearim, a trecut multă vreme, ca la douăzeci de ani. Și s-a întors toată casa lui Israel la Domnul.
\par 3 Iar Samuel a zis către toată casa lui Israel: "De vă întoarceți cu toată inima voastră la Domnul, atunci depărtați din mijlocul vostru dumnezeii străini și astartele și vă lipiți inimile voastre de Domnul și-I slujiți numai Lui, și El vă va izbăvi din mâinile Filistenilor!"
\par 4 Și au depărtat fiii lui Israel baalii și astartele și au început să slujească numai Domnului.
\par 5 Apoi a zis iarăși Samuel: "Adunați pe toți Israeliții la Mițpa și eu mă voi ruga pentru voi Domnului".
\par 6 Și s-au adunat la Mițpa și au scos apă, și au turnat înaintea Domnului și au postit în ziua aceea, zicând: "Greșit-am înaintea Domnului!" Și a judecat Samuel pe fiii lui Israel în Mițpa.
\par 7 Când însă au auzit Filistenii că s-au adunat fiii lui Israel în Mițpa, s-au dus căpeteniile Filistenilor asupra lui Israel. Auzind Israeliții de aceasta, s-au temut de Filisteni.
\par 8 Atunci au zis fiii lui Israel către Samuel: "Nu înceta a striga pentru noi către Domnul Dumnezeul nostru, ca să ne izbăvească din mâinile Filistenilor". Iar Samuel a zis: "Să nu fie cu mine una ca aceasta, ca să mă depărtez de Domnul Dumnezeul meu și să nu strig pentru voi întru rugăciune!
\par 9 Apoi a luat Samuel un miel de lapte și l-a adus împreună cu tot poporul ardere de tot Domnului și a strigat Samuel către Domnul pentru Israel și l-a auzit pe el Domnul.
\par 10 Când înălța Samuel ardere de tot, au venit Filistenii să se bată cu Israel. Dar Domnul a tunat în ziua aceea cu tunet mare asupra Filistenilor și a adus groază asupra lor, așa că au fost înfrânți în fața lui Israel.
\par 11 Iar Israeliții au ieșit din Mițpa și au urmărit pe Filisteni și i-au bătut până sub Bet-Car.
\par 12 Atunci a luat Samuel o piatră și a pus-o între Mițpa și Șen și a numit-o Eben-Ezer, zicând: "Până la locul acesta ne-a ajutat nouă Domnul!"
\par 13 Așa au fost înfrânți Filistenii și nu s-au mai apucat să mai umble peste hotarele lui Israel. Mâna Domnului a fost asupra Filistenilor în toate zilele lui Samuel.
\par 14 Și au fost întoarse lui Israel toate cetățile pe care le luaseră Filistenii de la Israel, de la Ecron până la Gat, și ținuturile acestora le-a liberat Israel din mâinile Filistenilor și a fost pace între Israel și Amorei.
\par 15 Astfel a fost Samuel judecător în Israel în toate zilele vieții sale.
\par 16 El mergea din an în an și cerceta Betelul și Ghilgalul și Mițpa și judeca pe Israel în toate locurile acestea.
\par 17 Apoi se întorcea la Rama, căci acolo era casa lui și acolo judeca el pe Israel. Și a ridicat acolo jertfelnic Domnului.

\chapter{8}

\par 1 Iar dacă a îmbătrânit Samuel, a pus pe fiii săi judecători peste Israel.
\par 2 Numele fiului său celui mai mare era Ioil, iar numele fiului său al doilea era Abia. Aceștia erau judecători în Beer-Șeba.
\par 3 Dar fiii lui nu umblau pe căile sale, ci se abăteau la lăcomie, luau daruri și judecau strâmb.
\par 4 Atunci s-au adunat toți bătrânii lui Israel, au venit la Samuel, în Rama,
\par 5 Și au zis către el: "Tu ai îmbătrânit, iar fiii tăi nu-ți urmează căile. De aceea pune peste noi un rege, ca să ne judece acela, ca și la celelalte popoare!"
\par 6 Cuvântul acesta însă n-a plăcut lui Samuel când i-au zis: "Dă-ne rege, ca să ne judece!" și s-a rugat Samuel Domnului.
\par 7 Și a zis Domnul către Samuel: "Ascultă glasul poporului în toate câte îți grăiește; căci nu pe tine te-au lepădat, ci M-au lepădat pe Mine, ca să nu mai domnesc Eu peste ei.
\par 8 Cum s-au purtat ei cu Mine din ziua aceea, când i-am scos din Egipt, până astăzi, părăsindu-Mă și slujind la dumnezei străini, așa se poartă și cu tine.
\par 9 Ascultă deci glasul lor, dar să le spui și să le arăți drepturile regelui, care va domni peste ei".
\par 10 Și a spus Samuel toate cuvintele Domnului poporului care îi cerea rege
\par 11 Și a zis: "Iată care vor fi drepturile regelui care va domni peste voi: pe fiii voștri îi va lua și-i va pune la carele sale și va face din ei călăreții săi și vor fugi pe lângă carele lui.
\par 12 Va pune din ei căpetenii peste mii, căpetenii peste sute, căpetenii peste cincizeci; să lucreze țarinile sale, să-i secere pâinea sa, să-i facă arme de război și unelte la carele lui.
\par 13 Fetele voastre le va lua, ca să facă miresme, să gătească mâncare și să coacă pâine.
\par 14 țarinile, viile și grădinile de măslini cele mai bune ale voastre le va lua și le va da slugilor sale.
\par 15 Din semănăturile voastre și din viile voastre va lua zeciuială și va da oamenilor săi și slugilor sale.
\par 16 Din robii voștri, din roabele voastre, din cei mai buni feciori ai voștri și din asinii voștri va lua și-i va întrebuința la treburile sale.
\par 17 Din oile voastre va lua a zecea parte și chiar voi veți fi robii lui.
\par 18 Veți suspina atunci sub regele vostru, pe care vi l-ați ales, și atunci nu vă va răspunde Domnul".
\par 19 Poporul însă nu s-a învoit să asculte pe Samuel, ci a zis: "Nu, lasă să fie rege peste noi,
\par 20 Și vom fi și noi ca celelalte popoare, ne va judeca regele nostru, va merge înainte și va purta războaiele noastre".
\par 21 A ascultat deci Samuel toate cuvintele poporului și le-a spus Domnului;
\par 22 Iar Domnul a zis către Samuel: "Ascultă glasul lor și pune-le rege!" Atunci a zis Samuel către Israeliți: "Duceți-vă fiecare în cetatea sa!"

\chapter{9}

\par 1 În vremea aceea era unul din fiii lui Veniamin, cu numele Chiș, fiul lui Abiel, fiul lui Țeror, fiul lui Becorat, fiul lui Afia, fiul unui veniaminean, om de ispravă.
\par 2 Acesta avea un fiu, cu numele Saul, tânăr și frumos, încât nu mai era nimeni în Israel mai frumos ca el; acesta era de la umeri în sus mai înalt decât tot poporul.
\par 3 O dată s-au rătăcit asinele lui Chiș, tatăl lui Saul, și a zis Chiș către Saul, fiul său: "Ia cu tine pe unul din argați și, sculându-te, du-te de caută asinele!"
\par 4 Și a suit acesta muntele lui Efraim și a străbătut ținutul Șalișa, dar nu le-a găsit. Apoi a străbătut ținutul Șaalim, și nici acolo nu le-a găsit. Apoi a străbătut și pământul lui Veniamin și tot nu le-a găsit.
\par 5 Iar când au ajuns în ținutul Țuf, a zis Saul către sluga sa care era cu el: "Haidem înapoi, ca nu cumva tatăl meu, lăsând asinele, să înceapă a fi neliniștit de noi".
\par 6 Sluga însă i-a zis: "Iată în cetatea aceasta este un om al lui Dumnezeu, om cinstit de toți și tot ce spune el se plinește. Să mergem dar acolo; poate ne va arăta și nouă calea pe care să apucăm".
\par 7 A zis Saul către sluga sa: "Haidem să mergem, dar ce să ducem noi omului aceluia? Căci pâinea s-a isprăvit din traistele noastre și daruri nu avem ca să ducem omului lui Dumnezeu. Ce avem noi?"
\par 8 Sluga a răspuns iarăși și a zis: "Iată am în mână un sfert de siclu de argint; îl voi da omului lui Dumnezeu și el ne va arăta calea".
\par 9 Înainte vreme în Israel, când mergea cineva să întrebe pe Dumnezeu, zicea așa: "Hai la văzătorul!" Căci acela care astăzi se numește prooroc, înainte se numea văzător.
\par 10 Și a zis Saul către sluga sa: "Bine zici tu; hai să mergem!" Și s-au dus în cetate, unde era omul lui Dumnezeu.
\par 11 Dar când se suiau ei la deal spre cetate, i-au întâmpinat niște fete care ieșiseră să aducă apă și le-au zis acestora: "Aici este văzătorul?"
\par 12 Iar acelea au răspuns și au zis: "Aici; iată-l înaintea ta; dar grăbește că el astăzi a venit în cetate, pentru că astăzi poporul axe jertfe pe deal.
\par 13 Când veți ajunge în cetate, îl veți găsi până nu se duce pe acel deal la prânz; căci poporul nu începe să mănânce până nu vine el; pentru că el le binecuvântează jertfa și după aceea începe să mănânce. Duceți-vă dar, că-l veți apuca încă acasă".
\par 14 Și s-au dus ei în cetate. Dar când au sosit în mijlocul cetății, atunci iată și Samuel le ieși înainte, ca să se ducă pe deal.
\par 15 Domnul însă cu o zi înainte de sosirea lui Saul, îi descoperise lui Samuel și-i zisese:
\par 16 "Mâine pe vremea asta voi trimite la tine pe un om din ținutul lui Veniamin și tu îl vei unge cârmuitor al poporului Meu Israel; acela va izbăvi pe poporul Meu din mâna Filistenilor, căci am căutat spre poporul Meu, deoarece strigătul lui a ajuns până la Mine".
\par 17 Deci, când a văzut Samuel pe Saul, atunci Domnul i-a zis: "Iată omul despre care ți-am vorbit Eu. Acesta va cârmui pe poporul Meu!"
\par 18 Și s-a apropiat Saul de Samuel la poartă și l-a întrebat: "Spune-mi unde este casa văzătorului?
\par 19 Iar Samuel a răspuns lui Saul și a zis: "Eu sunt văzătorul; mergi înaintea mea pe deal, că aveți să prânziți cu mine astăzi, iar dimineață te voi lăsa să pleci; și tot ce ai pe inimă, îți voi spune.
\par 20 Iar de asinele care ți s-au rătăcit acum trei zile nu purta grijă, căci s-au găsit. Și cui sunt oare păstrate cele mai scumpe lucruri în Israel, dacă nu ție și la toată casa tatălui tău?"
\par 21 Atunci a răspuns Saul și a zis: "Nu sunt eu oare fiul lui Veniamin, una din cele mai mici din semințiile lui Israel? Și familia mea oare nu este cea mai mică din toate familiile seminției lui Veniamin? De ce dar îmi vorbești tu mie acestea?"
\par 22 Și a luat Samuel pe Saul și pe sluga lui și i-a dus în casă și le-a dat locul cel dintâi între oaspeți, care erau ca la treizeci de oameni.
\par 23 După aceea a zis Samuel către bucătar: "Dă-mi partea pe care ți-am dat-o și de care li-am zis: Păstreaz-o la tine!"
\par 24 Și a luat bucătarul spata și cele ce erau cu ea și le-a pus înaintea lui Saul. Apoi a zis Samuel: "Iată aceasta este păstrată pentru tine; pune-ți-o dinainte și mănâncă, fiindcă pentru vremea aceasta s-au păstrat acestea pentru tine, când am adunat poporul!" Și a prânzit Saul cu Samuel în ziua aceea.
\par 25 Apoi s-au coborât de pe deal în cetate și a stat de vorbă Samuel cu Saul în foișorul de sus al casei, unde i s-a așternut lui Saul și a dormit.
\par 26 Iar dimineața s-au sculat ei așa: la ivirea zorilor a strigat Samuel pe Saul din foișor și a zis: "Scoală și te voi petrece! Și s-a sculat Saul și au ieșit amândoi din casă, el și Samuel.
\par 27 Iar dacă au ajuns ei la marginea cetății, a zis Samuel către Saul: "Spune-i slugii să treacă înaintea noastră"; și a plecat acela înainte. Apoi a zis: "Tu însă oprește-te acum, că am să-ți descopăr ceea ce a zis Dumnezeu".

\chapter{10}

\par 1 Atunci, luând Samuel vasul cel cu untdelemn, a turnat pe capul lui Saul și l-a sărutat, zicând: "Iată Domnul te unge pe tine cârmuitor al moștenirii Sale; vei domni peste poporul Domnului și-l vei izbăvi din mâna vrăjmașilor celor dimprejurul lor. Iată care-ți va fi semnul că Domnul te-a uns rege peste moștenirea Sa:
\par 2 Când vei pleca acum de la mine, vei întâlni doi oameni aproape de mormântul Rahilei, în hotarele lui Veniamin, la Țelțah și aceia îți vor spune: S-au găsit asinele după care ai umblat și le-ai căutat și iată tatăl tău, uitând de asine, este neliniștit pentru voi, zicând: "Ce este cu fiul meu?"
\par 3 Mergând apoi de acolo mai departe și ajungând la dumbrava Tabor, te vor întâmpina acolo trei oameni, care merg la Dumnezeu în Betel: unul duce trei iezi, altul duce trei pâini, iar al treilea duce un burduf cu vin.
\par 4 Aceia, după ce te vor saluta, îți vor da două pâini și tu le vei lua din mâinile lor.
\par 5 După aceea vei ajunge la Ghibeea Elohim, unde se află garda de pază a Filistenilor; acolo sunt căpeteniile filistene. și când vei intra în cetate, vei întâlni o ceată de prooroci coborându-se de pe înălțime, iar înaintea lor se cântă din psaltire și din timpan și din fluier și din harpă, iar ei proorocesc.
\par 6 Atunci va veni peste tine Duhul Domnului și vei prooroci și tu cu ei și te vei face alt om.
\par 7 După ce se vor adeveri cu tine aceste semne, atunci să faci ce vei putea, căci Dumnezeu este cu tine.
\par 8 Dar să te duci înainte de mine în Ghilgal, unde am să vin și eu la tine, ca să aducem arderi de tot și jertfe de împăcare. Să aștepți șapte zile, până voi veni la tine și atunci am să-ți spun ce ai să faci".
\par 9 Îndată ce și-a întors spatele Saul, ca să plece de la Samuel, îi dădu Dumnezeu altă inimă și s-au împlinit cu el toate acele semne în aceeași zi.
\par 10 Iar dacă au ajuns ei la Ghibeea, iată i-a întâmpinat o ceată de prooroci și a venit peste el Duhul lui Dumnezeu și a proorocit și el în mijlocul lor.
\par 11 Toți cei ce-l cunoșteau de mai înainte, văzând că proorocește cu proorocii, vorbeau prin popor unul către altul: "Ce s-a întâmplat cu fiul lui Chiș? Au doară și Saul este printre prooroci?"
\par 12 Iar unul din cei ce erau acolo a răspuns și a zis: "Dar tatăl acelora cine este oare?" De atunci a rămas zicătoarea: "Au doară și Saul este printre prooroci?"
\par 13 Apoi a încetat el să proorocească și s-a suit pe un deal.
\par 14 Atunci a zis unchiul lui Saul către acesta și către sluga lui: "Unde ați fost voi?" și el a zis: "Am căutat asinele, dar, văzând că nu le găsim, ne-am abătut pe la Samuel".
\par 15 Și a zis unchiul lui Saul: "Spune-mi ce v-a spus Samuel?"
\par 16 Iar Saul a zis către unchiul său: "Ne-a spus că asinele s-au găsit". Iar ceea ce îi spusese Samuel de domnie, nu i-a descoperit.
\par 17 Atunci a adunat Samuel poporul la Domnul, în Mițpa,
\par 18 Și a zis către fiii lui Israel: "Așa zice Domnul Dumnezeul lui Israel: "Eu am scos pe Israel din Egipt și v-am izbăvit din mâna Egiptenilor și din mâna tuturor împărățiilor care vă apăsau.
\par 19 Dar voi acum ați lepădat pe Domnul Dumnezeul vostru, Care vă scapă din toate necazurile și nevoile voastre și ați zis către El: "Pune rege peste noi!" Înfățișați-vă dar înaintea Domnului, după semințiile voastre și după familiile voastre!"
\par 20 Și a poruncit Samuel tuturor semințiilor lui Israel să se apropie și a fost arătată seminția lui Veniamin.
\par 21 Apoi a poruncit să se apropie familiile din seminția lui Veniamin și a ieșit la sorți familia lui Matri; după aceea au fost aduși bărbații din familia lui Matri și a ieșit la sorți Saul, fiul lui Chiș și l-au căutat, dar nu l-au găsit.
\par 22 Și au întrebat iarăși pe Domnul: "Va veni el oare aici?" Iar Domnul a zis: "Iată-l, se ascunde printre lucruri".
\par 23 Atunci au alergat și l-au luat de acolo și el a stat în mijlocul poporului și poporul îi venea numai până la umeri.
\par 24 Și a zis Samuel către tot poporul: "Vedeți pe cine a ales Domnul? Asemenea lui nu este în tot poporul". Atunci tot poporul a strigat și a zis: "Să trăiască regele!"
\par 25 Și a înșirat Samuel poporului drepturile regelui, le-a scris în carte și le-a pus înaintea Domnului. După aceea a dat drumul la tot poporul să meargă fiecare la casa sa.
\par 26 De asemenea s-a dus și Saul la casa sa, în Ghibeea, și s-au dus cu el și vitejii a căror inimă o atinsese Dumnezeu.
\par 27 Iar oamenii netrebnici ziceau: "Acesta oare ne va izbăvi pe noi?" Și-l disprețuiau și nu i-au adus daruri. Dar el s-a făcut că nu-i aude.

\chapter{11}

\par 1 S-a întâmplat însă, după vreo lună, să vină Nahaș Amoniteanul și să împresoare cetatea Iabeș din Galaad. Atunci toți locuitorii din Iabeș au zis către Nahaș: "Încheie legământ cu noi și noi îți vom sluji ție".
\par 2 Dar Nahaș Amoniteanul a zis către ei: "Eu voi încheia cu voi legământ ca să se scoată fiecăruia din voi ochiul drept și prin aceasta să se arunce necinste asupra întregului Israel".
\par 3 Atunci bătrânii din Iabeș au zis către el: "Dă-ne vreme de șapte zile, ca să trimitem împuterniciți în toate hotarele lui Israel și de nu ne va ajuta nimeni, vom ieși la tine".
\par 4 Au mers deci tinerii la Ghibeea lui Saul și au spus vorbele acestea în auzul poporului și tot poporul a ridicat glas și a plâns.
\par 5 Dar iată că a venit Saul de la câmp, în urma boilor și a zis: "Ce are poporul de plânge?" Și i s-au spus vorbele locuitorilor din Iabeș.
\par 6 Atunci s-a coborât Duhul lui Dumnezeu asupra lui Saul, când a auzit el cuvintele acestea și s-a aprins strașnic mânia lui.
\par 7 Și a luat el o pereche de boi, i-a junghiat, i-a tăiat bucăți și a trimis în toate hotarele lui Israel prin împuterniciții aceia, spunând că așa se va face cu boii aceluia care nu va merge după Saul și Samuel. Atunci a căzut frica Domnului peste popor și au ieșit toți ca un singur om.
\par 8 și i-a cercetat Saul în Bezec și s-au găsit din fiii lui Israel trei sute de mii, iar din Iuda treizeci de mii de oameni.
\par 9 Atunci s-a spus solilor veniți din Iabeș: "Așa să spuneți locuitorilor din Iabeșul Galaadului: ;Mâine, când va începe soarele să încălzească, ajutorul va fi la voi". Au venit deci trimișii și au spus locuitorilor Iabeșului și aceștia s-au bucurat.
\par 10 Apoi au zis locuitorii Iabeșului către Nahaș: "Mâine vom ieși la voi și veți face cu noi cum vă va place".
\par 11 A doua zi Saul împărți poporul în trei tabere. Și acestea au pătruns dis-de-dimineață în tabăra Amoniților și i-a măcelărit până la sosirea arșiței zilei; iar câți au rămas s-au risipit de n-au rămas doi la un loc.
\par 12 Atunci poporul a zis către Samuel: "Cine a zis: Saul oare are să domnească peste noi? Dați-ne pe acești nelegiuiți și-i vom omorî!"
\par 13 Saul însă a zis: "Astăzi nu trebuie să fie ucis nimeni, căci astăzi Domnul a săvârșit izbăvirea în Israel".
\par 14 Iar Samuel a zis către popor: "Să mergem la Ghilgal și să începem acolo noua domnie!"
\par 15 Și s-a dus tot poporul la Ghilgal și au pus acolo pe Saul rege înaintea Domnului în Ghilgal și au adus acolo jertfe de împăcare înaintea Domnului. Și s-a veselit foarte Saul acolo și toți Israeliții.

\chapter{12}

\par 1 A zis Samuel către tot poporul: "Iată eu am ascultat glasul vostru în toate câte mi-ați grăit și am pus rege peste voi.
\par 2 Iată regele umblă înaintea voastră, iar eu am îmbătrânit și am încărunțit; fiii mei sunt cu voi și eu am umblat înaintea voastră din tinerețile mele până acum.
\par 3 Iată-mă, mărturisiți asupra mea înaintea Domnului și a unsului Lui, de am luat cuiva boul, de am luat cuiva asinul, de am asuprit pe cineva și de am apăsat pe cineva; de am luat de la cineva mită și am închis ochii la judecata lui, vă voi despăgubi".
\par 4 Și au răspuns toți: "Tu nu ne-ai nedreptățit, nici nu ne-ai asuprit, nici nu ai luat nimic de la nimeni".
\par 5 Atunci el a zis: "Martor ne este Domnul și martor este unsul Lui în ziua aceasta, că voi n-ați găsit nimic asupra mea!" Iar ei au zis: "Martor!"
\par 6 Apoi a zis Samuel către popor: "Martor este Domnul, Cel ce a pus pe Moise și pe Aaron și Care a scos pe părinții voștri din țara Egiptului.
\par 7 Acum însă veniți, că eu am să mă judec cu voi înaintea Domnului pentru toate binefacerile pe care le-a făcut El vouă și părinților voștri.
\par 8 Când a venit Iacov în Egipt și părinții voștri au strigat către Domnul, atunci Domnul a trimis pe Moise și pe Aaron și au scos ei pe părinții voștri din Egipt și i-au strămutat în locul acesta.
\par 9 Dar ei au uitat pe Domnul Dumnezeul lor și El i-a dat în mâinile lui Sisera, căpetenia oștirilor Hațorului, în mâinile Filistenilor și în mâinile regelui Moabului, care s-au războit împotriva lor.
\par 10 Însă când au strigat ei către Domnul și au zis: "Am păcătuit, părăsind pe Domnul și apucându-ne să slujim baalilor și astartelor; acum însă izbăvește-ne din mâinile vrăjmașilor, și-ți vom sluji ție",
\par 11 Atunci a trimis Domnul pe Ierubaal, pe Barac, pe Ieftae și pe Samuel și v-a izbăvit din mâinile vrăjmașilor voștri celor din jurul vostru și ați trăit în pace.
\par 12 Iar când ați văzut că Nahaș, regele Amoniților, vine împotriva voastră, ați zis către mine: "Nu, ci să domnească peste noi un rege!", deși peste voi împărățea Domnul Dumnezeul vostru.
\par 13 Așadar iată regele pe care l-ați cerut; iată Domnul a pus peste voi rege.
\par 14 De vă veți teme de Domnul, de-I veți sluji Lui și de veți asculta glasul Lui, de nu vă veți împotrivi poruncilor Domnului și de veți umbla și voi și regele care domnește peste voi în urma Domnului Dumnezeului vostru, atunci mâna Domnului nu va fi împotriva voastră.
\par 15 Iar de nu veți asculta glasul Domnului, ci vă veți împotrivi poruncilor Lui, atunci mina Domnului va fi împotriva voastră, cum a fost împotriva părinților voștri.
\par 16 Acum sculați-vă și priviți la lucrul cel mare pe care-l va face Domnul înaintea ochilor voștri:
\par 17 Nu e acum oare secerișul grâului? Dar eu voi striga către Domnul și El va trimite trăsnet și ploaie și veți afla și veți vedea cât de mare este păcatul pe care l-ați făcut voi înaintea ochilor Domnului, când ați cerut rege".
\par 18 Și a strigat Samuel către Domnul și a trimis Domnul tunete și ploaie în ziua aceea; și teama de Domnul și de Samuel cuprins tot poporul.
\par 19 Și a zis tot poporul către Samuel: "Roagă-te pentru robii tăi înaintea Domnului Dumnezeului tău, ca să nu murim; căci la toate celelalte păcate ale noastre am mai adăugat un păcat, când am cerut rege".
\par 20 Iar Samuel a răspuns poporului: "Nu vă temeți. Păcatul acesta este făcut de voi, dar voi să nu vă depărtați de Domnul, ci să-I slujiți Lui cu toată inima.
\par 21 Să nu apucați după dumnezeii cei de nimic, care nu aduc folos, nici nu izbăvesc, pentru că sunt nimic.
\par 22 Domnul însă nu va lăsa pe poporul Său pentru numele Său cel mare, căci Domnul a binevoit să vă aleagă pe voi ca popor al Său.
\par 23 Și eu de asemenea nu-mi voi îngădui să fac înaintea Domnului păcatul de a înceta să mă rog pentru voi și vă voi povățui pe căi bune și drepte. Decât numai să vă temeți de Domnul și să-I slujiți Lui cu adevărat, din toată inima voastră,
\par 24 Căci vedeți ce lucruri minunate a făcut El cu voi.
\par 25 Iar de veți face rău, atunci veți pieri și voi și regele vostru".

\chapter{13}

\par 1 Se împlinise un an de când fusese făcut Saul rege și acum domnea în al doilea an peste Israel, când și-a ales el trei mii de Israeliți:
\par 2 Două mii erau cu Saul la Micmaș și pe Muntele Betelului și o mie era cu Ionatan în Ghibeea lui Veniamin; iar celălalt popor era lăsat de el pe la casele lor.
\par 3 Și a sfărâmat Ionatan tabăra de pază a Filistenilor, care era în Gheba. Și au auzit de aceasta Filistenii, iar Saul a sunat din trâmbiță în toată țara, strigând: "Să audă Evreii!"
\par 4 Și după ce a auzit tot Israelul că Saul a sfărâmat tabăra de pază a Filistenilor și că Filistenii au urât pe Israeliți, s-a adunat poporul la Saul în Ghilgal.
\par 5 Filistenii s-au adunat și ei, ca să se războiască împotriva Israeliților: treizeci de mii de care, șase mii de călăreți și popor mult ca nisipul de pe malul mării; și au venit și și-au pus tabăra în Micmaș, în partea de răsărit a Bet-Avenului.
\par 6 Iar Israeliții, văzându-se în primejdie, pentru că poporul era strâmtorat, s-au ascuns prin peșteri, prin stufișuri, printre stânci, prin turnuri și prin șanțuri;
\par 7 Ba unii din Israeliți au trecut peste Iordan în pământul lui Gad și al Galaadului. Saul însă se afla încă tot în Galaad și tot poporul care era cu el era cuprins de frică.
\par 8 Acolo a așteptat el opt zile, până la vremea hotărâtă de Samuel; dar Samuel nu mai venea la Ghilgal. Acum poporul începuse să fugă de la el.
\par 9 De aceea a zis Saul: "Aduceți-mi cele pentru jertfa de curățire". Și a adus ardere de tot.
\par 10 Dar nu apucase el bine să aducă ardere de tot, și iată veni și Samuel. Saul ieși înaintea lui ca să-l întâmpine și să-i ureze de bine.
\par 11 Samuel însă i-a zis: "Ce ai făcut?" Și i-a răspuns Saul: "Am văzut că poporul se împrăștie și fuge de la mine și tu nu ai venit la vremea hotărâtă, iar Filistenii s-au adunat la Micmas.
\par 12 Atunci am socotit că au să năvălească Filistenii asupra mea în Ghilgal și eu n-am întrebat încă pe Domnul; de aceea m-am hotărât să aduc ardere de tot".
\par 13 Iar Samuel i-a zis: "Rău ai făcut, că nu ai împlinit porunca Domnului Dumnezeului tău care ți s-a dat, căci acum ar fi întărit Domnul domnia ta peste Israel de-a pururi.
\par 14 Acum însă nu va dura domnia ta; Domnul Își va găsi un bărbat după inima Sa și-i va porunci Domnul să fie conducătorul poporului Său, deoarece tu nu ai împlinit ceea ce ți s-a poruncit de la Domnul".
\par 15 Apoi s-a sculat Samuel și s-a dus din Ghilgal în Ghibeea lui Veniamin; iar oamenii care au rămas s-au dus cu Saul în întâmpinarea taberei vrăjmașului, care a năvălit asupra lor când mergeau ei de la Ghilgal la Ghibeea lui Veniamin. Și a numărat Saul oamenii care erau cu el și s-au găsit până la șase sute de bărbați.
\par 16 Apoi Saul cu fiul său Ionatan și cu oamenii care erau cu dânșii au șezut în Ghibeea și au plâns; iar Filistenii stăteau în tabără la Micmaș.
\par 17 Și au ieșit din tabăra Filistenilor trei cete să pustiiască țara: una a plecat pe drumul spre Ofra din ținutul Șual;
\par 18 A doua ceată a plecat pe drumul Bethoronului, iar a treia s-a îndreptat pe calea ce duce spre hotar, în fala văii Țeboim, spre pustie.
\par 19 Fierar nu era în toată Iara lui Israel, căci Filistenii se temeau ca nu cumva Israeliții să-și facă săbii și sulițe.
\par 20 Trebuia deci să se ducă toți Israeliții la Filisteni ca să-și ascută fiarele plugurilor și sapelor lor, topoarele și securile lor,
\par 21 Când se făcea vreo știrbitură la ascuțișul fiarelor de plug, al sapelor, al topoarelor și securilor lor, sau când trebuia să îndrepte vreun corn de furcă.
\par 22 De aceea în timpul războiului de la Micmaș, tot poporul care era cu Saul și Ionatan nu avea nici săbii, nici sulițe; numai Saul și Ionatan, fiul său, aveau.
\par 23 Și a venit ceata întâi de Filisteni și s-a așezat la trecătoarea Micmaș.

\chapter{14}

\par 1 Într-o zi Ionatan, fiul lui Saul, a zis către tânărul care purta armele sale: "Hai să trecem la ceata Filistenilor, care e dincolo", iar tatălui său nu i-a spus de aceasta.
\par 2 Saul însă se afla la marginea Ghibeii, sub rodiul cel din Migron; cu el era o ceată ca de șase sute de oameni.
\par 3 Și Ahia, fiul lui Ahituv, fratele lui Icabod, fiul lui Finees, feciorul lui Eli, preotul Domnului din Șilo, purta efodul. Poporul însă nu știa că Ionatan plecase.
\par 4 Strâmtoarea prin care căuta Ionatan să se strecoare spre ceata Filistenilor trecea printre niște stânci ascuțite; numele uneia era Boțeț și numele alteia era Sene;
\par 5 O stâncă se ridica la miazănoapte, spre Micmaș; cealaltă la miazăzi, spre Ghibeea.
\par 6 Atunci a zis Ionatan către tânărul care purta armele sale: "Hai să trecem la acești netăiați împrejur; poate ne va ajuta Domnul, căci pentru Domnul nu e greu să izbăvească și prin puțini, ca și prin mulți".
\par 7 Cel ce purta armele a răspuns: "Fă tot ce-ți spune inima; mergi unde vrei și iată eu sunt cu tine".
\par 8 Iar Ionatan a zis: "Bine, atunci hai la acei oameni și să ne arătăm către ei.
\par 9 Dacă ei ne vor zice: Stați până vom veni la voi, vom rămâne pe loc și nu ne vom sui la ei;
\par 10 Iar de ne vor zice: Suiți-vă la noi, atunci ne vom sui, căci Domnul i-a dat în mâinile noastre și acesta va fi semnul pentru noi".
\par 11 Când au apărut ei amândoi în văzul cetei Filistenilor, atunci Filistenii au zis: "Iată Evreii ies din peșterile în care s-au ascuns".
\par 12 Apoi oamenii din ceata Filistenilor au strigat către Ionatan și către cel ce-i purta armele sale și a zis: "Suiți-vă la noi, că avem să vă spunem ceva". Atunci Ionatan a zis către purtătorul lui de arme: "Vino după mine, că Domnul i-a dat în mâinile lui Israel".
\par 13 Și a început Ionatan să se urce, cățărându-se cu mâinile și cu picioarele, și cel ce-i purta armele se cățăra după el. Și au căzut Filistenii înaintea lui, iar cel ce purta armele în urma lui le da cele din urmă lovituri.
\par 14 Și au căzut în acest atac dintâi, săvârșit de Ionatan și de purtătorul lui de arme ca la douăzeci de oameni, pe o bucată de loc cât o jumătate de pogon, atât cât pot să are doi boi într-o zi.
\par 15 Atunci s-a făcut învălmășeală în tabăra din câmp și în tot poporul; rândurile dinainte ce pustiau țara s-au umplut de frică și nu voiau să se lupte; toată țara s-a cutremurat și i-a cuprins frică mare de la Domnul.
\par 16 Atunci străjile lui Saul din Ghibeea lui Veniamin au văzut că mulțimea se împrăștie și fuge încolo și încoace.
\par 17 Saul a zis către oamenii care erau cu el: "Căutați și vedeți care din ai noștri a plecat". Și au căutat și iată nu erau Ionatan și purtătorul lui de arme.
\par 18 A zis Saul către Ahia: "Adu chivotul lui Dumnezeu", căci chivotul lui Dumnezeu în vremea aceea era cu fiii lui Israel.
\par 19 Și pe când grăia încă Saul cu preotul, tulburarea din tabăra Filistenilor se lățea și creștea din ce în ce mai mult. Atunci Saul a zis către preot: "Încrucișează-ți mâinile!"
\par 20 Și a strigat Saul și tot poporul care era cu el și au venit la locul luptei și iată acolo sabia fiecăruia era ridicată asupra aproapelui său și tulburarea era foarte mare.
\par 21 Atunci și Evreii, care mai dinainte erau la Filisteni și care umblau pretutindeni în tabăra lor, s-au unit cu Israeliții cei ce erau cu Saul și Ionatan;
\par 22 Și toți Israeliții care se ascunseseră în muntele lui Efraim, auzind că Filistenii au fugit, s-au unit de asemenea cu ai lor la luptă.
\par 23 Și a izbăvit Domnul în ziua aceea pe Israel; lupta se întinsese însă până la Bet-Aven. Și tot poporul care era cu Saul era ca la zece mii de bărbați; și se da război în toate cetățile din muntele lui Efraim.
\par 24 În ziua aceea s-au obosit oamenii din Israel. Iar Saul a pus poporul să jure, zicând: "Blestemat tot cel ce va mânca până seara, până când eu îmi voi răzbuna pe vrăjmașii mei". De aceea nimeni din popor n-a gustat hrană,
\par 25 Ci s-a dus tot poporul în pădure, și acolo intr-o poiană era miere.
\par 26 Și a intrat poporul în pădure și a zis: "Iată curge miere". Dar nimeni nu și-a dus mâna spre gura sa, căci poporul se temea de blestem.
\par 27 Ionatan însă nu auzise de jurământul pe care-l pusese tatăl său pe popor; și, întinzând vârful toiagului ce-l avea în mână, l-a muiat într-un fagure de miere și, întorcându-l cu mâna spre gura sa, i s-au luminat ochii.
\par 28 Atunci unul din popor i-a spus: "Tatăl tău a pus jurământ asupra poporului, zicând: Blestemat să fie cel ce va gusta astăzi hrană, și de aceea poporul e istovit".
\par 29 Dar Ionatan a zis: "Tatăl meu a tulburat țara. Iată mie mi s-au luminat ochii când am gustat puțin din această miere.
\par 30 De ar fi mâncat astăzi poporul din prăzile ce s-au găsit la vrăjmașii lor, oare n-ar fi fost mai mare înfrângerea Filistenilor?"
\par 31 În ziua aceea au bătut pe Filisteni de la Micmaș până la Aialon, dar poporul se istovise peste măsură.
\par 32 Și s-a aruncat poporul asupra prăzilor și a luat oi și boi și vitei și au junghiat pe pământ și au mâncat oamenii carne cu sânge.
\par 33 Și i s-a vestit lui Saul, zicându-i-se: "Iată poporul a greșit înaintea Domnului, mâncând carne cu sânge". Atunci Saul a zis: "Voi ați greșit. Prăvăliți acum aici spre mine o piatră mare".
\par 34 Apoi a zis Saul: "Treceți prin popor și ziceți-i: "Să-și aducă fiecare la mine boul său și oaia și să înjunghiați aici și să mâncați, și să nu greșiți înaintea Domnului mâncând carne cu sânge". Și toți din popor și-au adus cu mâna sa fiecare boul său, noaptea, și l-au junghiat acolo.
\par 35 Și a zidit Saul jertfelnic Domnului; acesta a fost cel dintâi jertfelnic făcut de el Domnului.
\par 36 Atunci a zis Saul: "Hai după Filisteni noaptea aceasta și să-i prădăm până dimineața și să nu lăsăm din ei nici un om". Iar ei au zis: "Fă tot ce este bine în ochii tăi". Preotul însă a zis: "Să ne apropiem aici de Dumnezeu!"
\par 37 Și a întrebat Saul pe Dumnezeu: "Să merg eu oare după Filisteni? Îi vei da, oare, pe ei în mâinile lui Israel?" Dar El nu i-a răspuns în ziua aceea.
\par 38 Atunci Saul a zis: "Să vină aici toate căpeteniile poporului și voi căuta să aflu asupra cui este păcatul acum.
\par 39 Că viu este Domnul, Cel ce a izbăvit pe Israel, că de va fi chiar asupra lui Ionatan, fiul meu, apoi și el va muri. Dar nimeni din tot poporul nu i-a răspuns.
\par 40 Și a zis Saul către toți Israeliții: "Stați voi de o parte iar eu și Ionatan, fiul meu, vom sta de altă parte". Și a răspuns poporul lui Saul: "Fă ce este bine în ochii tăi!" Apoi Saul a zis: "Doamne, Dumnezeul lui Israel, pentru ce n-ai răspuns Tu acum robului Tău? Dacă vina este asupra mea sau asupra fiului meu Ionatan, Domnul Dumnezeul lui Israel, fă să iasă Urim, dacă vina este asupra poporului Tău Israel, fă să iasă Tumim.
\par 41 Și sorțul a căzut pe Saul și pe Ionatan, iar poporul a ieșit drept.
\par 42 Atunci Saul a zis: "Aruncați sorți asupra mea și asupra lui Ionatan, fiul meu, și pe cine-l va arăta Domnul, acela să moară". Poporul însă a zis: "Să nu fie așa!" Dar Saul a stăruit în hotărârea sa, și au aruncat sorți asupra sa și a lui Ionatan, fiul lui, și a căzut sorțul pe Ionatan.
\par 43 Atunci Saul a zis către Ionatan: "Spune-mi ce ai făcut?" Iar Ionatan i-a răspuns și a zis: "Doar am gustat puțină miere cu vârful toiagului pe care îl aveam în mână și iată trebuie să mor".
\par 44 Iar Saul a zis: "Așa și așa să-mi facă mie Dumnezeu și încă și mai mult să-mi facă, dacă nu vei muri astăzi, Ionatane!"
\par 45 Dar poporul a zis către Saul: "Să moară oare Ionatan care a adus o izbăvire așa de minunată poporului! Să nu fie aceasta! Viu este Domnul, nici un păr din capul lui nu va cădea, pentru că el cu Dumnezeu a lucrat astăzi!" Și a izbăvit poporul pe Ionatan și el n-a murit.
\par 46 Și s-a întors Saul de la urmărirea Filistenilor, iar Filistenii s-au dus la locul lor.
\par 47 Astfel și-a întărit Saul domnia sa peste Israel și s-a luptat cu toți vrăjmașii de primprejur: cu Moab și Amoniții, cu Edom, cu regii din Țoba și cu Filistenii și pretutindeni, împotriva oricui a mers, a avut izbândă.
\par 48 Și și-a rânduit oaste, a bătut pe Amalec și a izbăvit pe Israel din mâinile jefuitorilor săi.
\par 49 Fiii lui Saul erau: Ionatan, Iesui și Melchișua; iar numele celor două fiice ale sale erau: Merob, numele celei mai mari, și Micol, numele celei mai mici.
\par 50 Iar numele femeii lui Saul era Ahinoam, fiica lui Ahimaaț, iar numele căpeteniei oștirii lui era Abner, fiul lui Ner, unchiul lui Saul.
\par 51 Chiș era tatăl lui Saul, Ner era tatăl lui Abner, fiul lui Abiel.
\par 52 În tot timpul domniei lui Saul s-au dus războaie crâncene cu Filistenii. Când Saul vedea vreun om voinic și războinic, îl lua la el.

\chapter{15}

\par 1 În vremea aceea a zis Samuel către Saul: "Domnul m-a trimis să te ung rege peste poporul Lui, peste Israel; acum ascultă glasul Domnului.
\par 2 Așa zice Domnul Savaot: Adusu-Mi-am aminte de cele ce a făcut Amalec lui Israel, cum i s-a împotrivit în cale, când venea din Egipt.
\par 3 Mergi acum și bate pe Amalec și pe Ierim și nimicește toate ale lui. Să nu iei pentru tine nimic de la ei, ci nimicește și dă blestemului toate câte are. Să nu-i cruți, ci să dai morții de la bărbat până la femeie, de la tânăr până la pruncul de sân, de la bou până la oaie, de la cămilă până la asin".
\par 4 Atunci a adunat Saul poporul și l-a numărat în Telaim și s-au aflat două sute de mii Israeliți pedeștri și zece mii din seminția lui Iuda.
\par 5 Și a mers Saul până la cetatea lui Amalec și a pus oameni la pândă în vale.
\par 6 Apoi Saul a zis către Chenei: "Mergeți de vă despărțiți și ieșiți din mijlocul lui Amalec, ca să nu vă pierd împreună cu el, căci ați arătat bunăvoință către toți Israeliții, când veneau ei din Egipt". Și Cheneii s-au despărțit de Amalec.
\par 7 Atunci a lovit Saul pe Amalec și l-a bătut de la Havila până la Șur, care este în fața Egiptului; iar pe Agag, regele lui Amalec, l-a prins " iu și pe popor l-a ucis tot cu sabia și a ucis și pe Ierim.
\par 8 Dar Saul și poporul au cruțat pe Agag, pe cele mai bune din oi și din vitele cornute, mieii îngrășați și tot ce era bun și n-a vrut să le piardă;
\par 9 Iar toate lucrurile neînsemnate și rele le-au pierdut.
\par 10 Atunci a fost cuvântul Domnului către Samuel astfel: "Îmi pare rău că am pus pe Saul rege, căci el s-a abătut de la Mine și cuvântul Meu nu l-a împlinit".
\par 11 Și s-a întristat Samuel și a strigat către Domnul toată noaptea.
\par 12 Iar a doua zi dis-de-dimineață, sculându-se, a ieșit în întâmpinarea lui Saul. și i s-a spus lui Samuel că Saul a fost pe Carmel și și-a ridicat acolo semn de aducere aminte, iar de acolo s-a întors și s-a coborât la Ghilgal.
\par 13 Iar după ce a ajuns Samuel la Saul, Saul i-a spus: "Iată am împlinit cuvântul tău!"
\par 14 Samuel a zis: "Dar ce este acest behăit de oi ce-mi ajunge la urechi și acel muget de boi pe care-l aud?"
\par 15 Iar Saul a răspuns: "Le-am adus de la Amalec, de vreme ce poporul a cruțat pe cele mai bune din oi și din vitele mari, ca să fie aduse jertfă Domnului Dumnezeului tău. Iar pe celelalte le-a nimicit".
\par 16 Samuel a zis către Saul: "Îngăduie-mi să-ți spun ce mi-a spus Domnul astă-noapte". Iar Saul a zis: "Spune!"
\par 17 Și a zis Samuel: "Când erai tu mic în ochii tăi, n-ai ajuns tu oare căpetenia semințiilor lui Israel și Domnul te-a uns rege peste Israel?
\par 18 Apoi te-a trimis Domnul la drum, zicând: Mergi și dă junghierii pe Amaleciții cei necredincioși și luptă împotriva lor până îi vei stârpi.
\par 19 Pentru ce n-ai ascultat glasul Domnului, și te-ai aruncat asupra prăzii și ai făcut rău în ochii Domnului?"
\par 20 Iar Saul a zis către Samuel: "Eu am ascultat glasul Domnului și am plecat la drum încotro m-a trimis Domnul și am adus pe Agag, regele amalecit, iar pe Amalec l-am pierdut;
\par 21 Poporul însă a luat din prăzi, din oi și din vite, a luat cele mai bune din cele afierosite ca să le aducă jertfă Domnului Dumnezeului tău în Ghilgal".
\par 22 A răspuns Samuel: "Au doară arderile de tot și jertfele sunt tot așa de plăcute Domnului, ca și ascultarea glasului Domnului? Ascultarea este mai bună decât jertfa și supunerea mai bună decât grăsimea berbecilor.
\par 23 Căci nesupunerea este un păcat la fel cu vrăjitoria și împotrivirea este la fel cu închinarea la idoli. Pentru că ai lepădat cuvântul Domnului, și El te-a lepădat, ca să nu mai fii rege peste Israel".
\par 24 Atunci Saul a zis către Samuel: "Am păcătuit, călcând porunca Domnului și cuvântul tău; dar m-am temut de popor și am ascultat glasul lui.
\par 25 Ridică dar păcatul de pe mine și întoarce-te cu mine, ca să mă închin Domnului Dumnezeului tău".
\par 26 Iar Samuel a răspuns lui Saul: "Nu mă voi întoarce cu tine, pentru că ai lepădat cuvântul Domnului și Domnul te-a lepădat pe tine, ca să nu mai fii rege peste Israel".
\par 27 Apoi Samuel s-a întors să plece. Dar Saul s-a apropiat de poala hainei lui și a rupt-o.
\par 28 Atunci Samuel a zis: "Astăzi a rupt Domnul regatul lui Israel de la tine și l-a dat altuia care este mai bun decât tine,
\par 29 Și nu va spune neadevăr Cel ce este tăria lui Israel și nu Se va căi, căci El nu este om ca să Se căiască".
\par 30 Zis-a Saul: "Am greșit; dar dă-mi acum cinste înaintea bătrânilor poporului meu și înaintea lui Israel și te întoarce cu mine și eu mă voi închina Domnului Dumnezeului tău".
\par 31 Și s-a întors Samuel după Saul și s-a închinat Saul Domnului.
\par 32 Apoi Samuel a zis: "Adu la mine pe Agag, regele amalecit"; și s-a apropiat de el Agag, tremurând, și a zis: "De bună seamă amărăciunea morții a trecut".
\par 33 Samuel însă i-a răspuns: "Precum sabia ta a lipsit pe mame de copiii lor, așa și mama ta să fie între femei lipsită de fiu". Și a tăiat Samuel pe Agag înaintea Domnului în Ghilgal.
\par 34 Apoi s-a dus Samuel la Rama, iar Saul s-a dus la casa sa în Ghibeea lui Saul.
\par 35 Și nu s-a mai văzut Samuel cu Saul până în ziua morții sale. Dar s-a întristat Samuel pentru Saul, că se căise Domnul pentru că-l făcuse rege peste Israel.

\chapter{16}

\par 1 Domnul a zis către Samuel: "Până când te vei tângui tu pentru Saul, pe care l-am lepădat, ca să nu mai fie rege peste Israel? Umple cornul tău cu mir și du-te, că te trimit la Iesei Betleemitul, căci dintre fiii lui Mi-am ales rege".
\par 2 Samuel a zis: "Cum să mă duc? Va auzi Saul și mă va ucide". Iar Domnul a zis: "Ia cu tine o juncă din cireadă și zi: Am venit să aduc jertfă Domnului.
\par 3 Și cheamă pe Iesei și pe fiii lui la jertfă, și Eu îți voi arăta ce să faci și-Mi vei unge pe acela pe care îți voi spune Eu".
\par 4 Și a făcut Samuel așa, cum i-a spus Domnul. Și când a sosit el la Betleem, bătrânii poporului, tremurând, i-au ieșit în întâmpinare și au zis: "Cu pace este venirea ta, văzătorule?"
\par 5 Iar el a răspuns: "Cu pace. Am venit să aduc jertfă Domnului; sfințiți-vă și veniți cu mine să aducem jertfă!" Și a sfințit pe Iesei și pe fiii lui și i-a chemat la jertfă.
\par 6 Iar după ce au venit ei, văzând el pe Eliab, a zis: "De bună seamă, acesta este înaintea Domnului unsul Lui".
\par 7 Dar Domnul a zis către Samuel: "Nu te uita la înfățișarea lui și la înălțimea staturii lui; Eu nu Mă uit ca omul; căci omul se uită la față, iar Domnul se uită la inimă".
\par 8 Apoi a chemat Iesei pe Aminadab și l-a dus la Samuel, iar Samuel a zis: "Nici pe acesta nu l-a ales Domnul".
\par 9 După aceea a adus Iesei pe Șama, și Samuel a zis: "Nici pe acesta nu l-a ales Domnul".
\par 10 Și așa a adus Iesei pe șapte din fiii săi, dar Samuel a zis către Iesei: "Pe nici unul din aceștia nu l-a ales Domnul!"
\par 11 După aceea a zis Samuel către Iesei: "Oare toți fiii tăi sunt aici?" Iar Iesei a răspuns: "Mai am unul mai mic. Acela paște oile". A zis Samuel: "Trimite să-l aducă, pentru că nu vom ședea să prânzim până nu vine acela".
\par 12 Și a trimis Iesei și l-au adus. Acela era bălan, cu ochi frumoși și plăcut la față. Atunci Domnul a zis: "Scoală de-l unge, căci acesta este!"
\par 13 Și a luat Samuel cornul cu mir și l-a uns în mijlocul fraților lui, și a odihnit Duhul Domnului asupra lui David din ziua aceea și după aceea. Iar Samuel s-a sculat și a plecat la Rama.
\par 14 Atunci s-a depărtat de la Saul Duhul Domnului și-l tulbura un duh rău, trimis de Domnul.
\par 15 Și au zis slugile lui Saul: "Iată un duh rău trimis de Domnul te tulbură.
\par 16 Să poruncească dar domnul nostru slugilor sale care sunt înaintea ta și să caute un om iscusit lâ cântarea din harpă, și când va veni asupra ta duhul cel rău trimis de la Dumnezeu, atunci acela, cântând cu mâna sa, te va liniști".
\par 17 Și a răspuns Saul slugilor sale: "Căutați-mi un om care cântă bine și mi-l aduceți".
\par 18 Atunci unul din slujitorii lui a zis: "Iată eu am văzut la Iesei Betleemitul un fiu care știe să cânte, om voinic și războinic, priceput la vorbă și bărbat chipeș și Domnul este cu el".
\par 19 A trimis deci Saul vestitori la Iesei și i-a spus: "Trimite la mine pe David, fiul tău cel de la turmă".
\par 20 Și a luat Iesei un asin încărcat cu pâine și un burduf cu vin și un ied și le-a trimis cu David, fiul său, la Saul.
\par 21 Și a venit David la Saul, s-a înfățișat înaintea lui și a plăcut acestuia foarte mult și l-a făcut purtătorul său de arme.
\par 22 După aceea a trimis Saul să i se spună lui Iesei: "Lasă pe David să slujească la mine, că a aflat el bunăvoință în ochii mei! "
\par 23 Iar când duhul cel trimis de Dumnezeu era peste Saul, David, luând harpa, cânta și lui Saul îi era mai ușor și mai bine și duhul cel rău se depărta de el.

\chapter{17}

\par 1 În vremea aceea Filistenii și-au strâns oștile pentru război și au tăbărât la Soco cel din Iuda și și-au așezat tabăra între Soco și Azeca, la Efes-Damim.
\par 2 Iar Saul cu Israeliții s-au adunat și și-au așezat tabăra la Valea Stejarului și s-au pregătit de luptă cu Filistenii.
\par 3 Filistenii stăteau pe munte de o parte și Israeliții stăteau pe munte de altă parte, iar la mijloc era valea.
\par 4 Atunci a ieșit din tabăra Filistenilor un luptător cu numele Goliat, din Gat. Acesta era la statură de șase coți și o palmă.
\par 5 Pe cap avea coif de aramă și era îmbrăcat cu platoșă în solzi; greutatea platoșei lui cântărea cinci mii de sicli de aramă;
\par 6 În picioare avea cizme cu tureci de aramă și la umăr purta un scut de aramă.
\par 7 Coada suliței lui era ca sulul de la războaiele de țesut, iar fierul suliței era de șase sute sicli de fier, și înaintea lui mergea purtătorul lui de arme.
\par 8 Și a început acesta să strige către cetele lui Israel și să le zică: "De ce ați ieșit voi să vă războiți? Nu sunt eu oare filistean, iar voi robii lui Saul? Alegeți dintre voi un om să se coboare la mine!
\par 9 De se va putea acela lupta cu mine și mă va ucide, atunci noi să fim robii voștri; iar de-l voi birui eu și-l voi ucide, atunci voi să firi robii noștri și să ne slujiți nouă".
\par 10 Și a mai zis filisteanul: "Astăzi voi rușina tabăra lui Israel. Dați-ne un om și ne vom lupta în doi".
\par 11 Și a auzit Saul și toți Israeliții cuvintele acestea ale filisteanului și s-au speriat și s-au temut foarte tare.
\par 12 David era feciorul unui efraimit din Betleemul lui Iuda, anume Iesei, care avea opt feciori. Acest om în zilele lui Saul ajunsese la bătrânețe și era cel mai bătrân între ceilalți oameni.
\par 13 Cei trei feciori mai mari ai lui Iesei plecaseră cu Saul la război. Numele acestor feciori mai mari ai lui, care se duseră la război, erau: cel mai mare Eliab, al doilea după el Aminadab și al treilea Șama.
\par 14 David însă era cel mai mic. Când cei trei mai mari plecaseră cu Saul,
\par 15 David se întorsese de la Saul, ca să pască oile tatălui său în Betleem.
\par 16 Filisteanul acela însă ieșea dimineața și seara și s-a arătat patruzeci de zile.
\par 17 Atunci a zis Iesei către David, fiul său: "Ia pentru frații tăi o efă de grăunțe uscate și aceste zece pâini și du-le cât mai degrabă în tabără la frații tăi;
\par 18 Iar acești zece cași du-i căpeteniei celei peste mia lor; cercetează de sănătatea lor și află ce nevoi au".
\par 19 Atunci Saul și ei și toți Israeliții se aflau în Valea Stejarului și se pregăteau de luptă cu Filistenii.
\par 20 S-a sculat deci David dis-de-dimineață și, încredințând oile unui păstor, a luat sacul și a plecat, cum îi zisese Iesei, și a ajuns în tabără când oștirea era așezată în linie de bătaie și se gătea cu strigăte de război.
\par 21 Și și-au așezat Israeliții și Filistenii rândurile unii în fața altora.
\par 22 Iar David, lăsându-și lucrurile unei străji din tabără, a alergat între rânduri și, ajungând, a întrebat pe frații săi de sănătate.
\par 23 Și iată, pe când vorbea el cu ei, luptătorul cu numele de Goliat, filistean din Gat, a ieșit din rândurile Filistenilor și a spus aceste cuvinte, și David le-a auzit.
\par 24 Toți Israeliții, văzând pe omul acela, fugeau de el, temându-se foarte tare;
\par 25 Și ziceau Israeliții: "Vedeți pe omul acesta care a ieșit înainte? Iese ca să înfrunte pe Israel. De l-ar ucide cineva, regele ar răsplăti pe acela cu mari bogății și ar da pe fiica sa după el, iar casa tatălui aceluia ar ajunge liberă în Israel".
\par 26 David a zis către oamenii care stăteau cu el: "Ce se va face aceluia care va ucide pe acest filistean și va șterge ocara de pe Israel? Căci cine este acest filistean netăiat împrejur, de batjocorește așa oștirea Dumnezeului celui viu?"
\par 27 Și i-a spus mulțimea aceleași cuvinte, zicând: "Iată ce se va face omului aceluia care-l va ucide".
\par 28 Și auzind Eliab, fratele cel mai mare al lui David, ce vorbea acesta cu oamenii, s-a mâniat Eliab pe David și a zis: "Pentru ce ai venit aici și cu cine ai lăsat acele puține oi în pustiu? Eu cunosc mândria ta și inima ta cea rea. Ai venit să privești la luptă".
\par 29 Iar David a zis: "Dar ce am făcut eu? Au nu sunt acestea numai niște vorbe?"
\par 30 Și s-a întors de la el către altul și a spus aceleași vorbe, iar mulțimea i-a răspuns ca și mai înainte.
\par 31 Auzindu-se cuvintele pe care le grăise David, s-au spus lui Saul și acesta l-a chemat.
\par 32 Atunci David a zis către Saul: "Să nu se împuțineze nimeni cu duhul din pricina lui; robul tău se va duce și se va bate cu acest filistean!"
\par 33 A zis Saul către David: "Tu nu vei putea să mergi împotriva acestui filistean, ca să te bați cu el, căci ești încă un copilandru, iar acesta este ostaș din tinerețile lui".
\par 34 David însă a zis către Saul: "Robul tău a păscut oile tatălui său și când se întâmpla să vină leul sau ursul să ia vreo oaie din turmă,
\par 35 Atunci eu alergam după el și i-o luam din gura lui; iar dacă el se arunca asupra mea, eu îl apucam de coamă și-l loveam până-l ucideam.
\par 36 Și urși și lei a ucis robul tău; și cu acest filistean netăiat împrejur se va întâmpla același lucru ca și cu aceia, pentru că hulește așa oștirea Dumnezeului celui viu. Să mă duc dar și să-l lovesc, ca să spăl rușinea lui Israel? Căci cine e oare acest filistean?"
\par 37 Apoi a mai zis David: "Domnul, Cel ce m-a scăpat de la lei și urși, mă va scăpa și din mina acestui filistean!" Atunci Saul a zis lui David: "Du-te și Domnul să fie cu tine".
\par 38 Și a îmbrăcat Saul pe David cu hainele sale, a pus pe capul lui coif de aramă și l-a îmbrăcat cu zale.
\par 39 Și s-a încins David cu sabia lui peste haine și a început să umble, căci nu era deprins cu astfel de armură; apoi a zis David către Saul: "Nu pot să umblu cu acestea, că nu sunt deprins". Și s-a dezbrăcat David de toate acestea,
\par 40 Și și-a luat toiagul în mână, a ales cinci pietricele lucii din pârâu și le-a pus în traista sa de păstor; și cu traista și cu praștia în mână a ieșit înaintea filisteanului.
\par 41 Atunci a ieșit și filisteanul, înaintând și apropiindu-se de David; iar purtătorul lui de arme mergea înainte.
\par 42 Deci căutând filisteanul și văzând pe David, a privit cu dispreț la el, căci acesta era tânăr, bălan și frumos la față.
\par 43 A zis filisteanul către David: "Ce vii asupra mea cu toiag și cu pietre? Au doară eu sunt câine?" Iar David a răspuns: "Nu, ci mai rău decât un câine". Și a blestemat filisteanul pe David în numele dumnezeilor săi.
\par 44 Apoi a zis filisteanul către David: "Apropie-te de mine și voi da trupul tău păsărilor cerului și fiarelor câmpului!"
\par 45 Iar David a răspuns filisteanului: "Tu vii asupra mea cu sabie și cu lance și cu scut; eu însă vin asupra ta în numele Domnului Savaot, Dumnezeul oștirilor lui Israel pe Care tu L-ai hulit.
\par 46 Acum însă te va da Domnul în mâna mea și eu te voi ucide și-ți voi tăia capul, iar trupul tău și trupurile oștirii filistene le voi da păsărilor cerului și fiarelor câmpului, și va afla tot pământul că în Israel este Dumnezeu;
\par 47 Și toată adunarea aceasta va cunoaște că nu cu sabia și cu sulița izbăvește Domnul, căci acest război este al Domnului și El vă va da în mâinile noastre".
\par 48 Iar după ce s-a ridicat filisteanul și a început a veni și a se apropia în întâmpinarea lui David, David a alergat cu grăbire spre rândurile oștirii în întâmpinarea filisteanului.
\par 49 Și își vârî David mâna în traistă, luă de acolo o pietricică, o repezi cu praștia și lovi pe filistean în frunte, așa încât piatra se înfipse în fruntea lui și el căzu cu fața la pământ.
\par 50 Așa a biruit David pe filistean, cu praștia și cu piatra, lovind pe filistean și ucigându-l; sabie nu se afla în mâna lui David.
\par 51 Atunci David a alergat și, călcând pe filistean, luă sabia lui și, scoțând-o din teacă, îl lovi cu ea și-i tăie capul; Filistenii, văzând că uriașul lor a murit, au fugit.
\par 52 Deci s-au sculat bărbații lui Israel și ai lui Iuda și cu strigăte au gonit pe Filisteni până la gura văii și până la porțile Ecronului. Și au căzut uciși Filistenii pe calea Șaaraim, până la Gat și Ecron.
\par 53 După aceea s-au întors fiii lui Israel din urmărirea Filistenilor și au prădat tabăra lor.
\par 54 Iar David a luat capul filisteanului și l-a dus la Ierusalim, și armele lui le-a pus în cortul său.
\par 55 Când a văzut Saul pe David ieșind împotriva filisteanului, a zis către Abner, căpetenia oștirilor: "Abner, al cui este tânărul acesta?" Iar Abner, a răspuns: "Rege, viu fie sufletul tău, nu știu!"
\par 56 "Întreabă dar, a zis regele, al cui fiu este tânărul acesta?"
\par 57 Iar când se întorcea David, după uciderea filisteanului, Abner l-a luat și l-a dus la Saul și capul filisteanului era în mâna lui.
\par 58 Atunci Saul l-a întrebat: "Tinere, al cui fiu ești tu?" Și David a răspuns: "Fiul robului tău Iesei din Betleem".

\chapter{18}

\par 1 După ce a isprăvit David de vorbit cu Saul, sufletul lui Ionatan s-a lipit de sufletul lui și l-a iubit Ionatan, ca pe sufletul său.
\par 2 Iar Saul l-a luat în ziua aceea și nu l-a lăsat să se mai întoarcă la casa tatălui lui.
\par 3 Și a încheiat Ionatan legătură cu David, pentru că îl iubea ca pe sufletul său.
\par 4 Și și-a dezbrăcat Ionatan haina sa cea de deasupra, pe care o avea pe el, și a dat-o lui David; de asemenea și celelalte haine ale sale, sabia sa, arcul său și brâul său.
\par 5 David însă lucra cu pricepere peste tot, oriunde-l trimetea Saul; și Saul l-a făcut căpetenie peste oșteni; iar aceasta a plăcut la tot poporul și slujitorilor lui Saul.
\par 6 Dar când se întorceau ei, după izbânda lui David asupra filisteanului, femeile din toate cetățile lui Israel ieșeau în întâmpinarea regelui Saul cu cântări și jocuri, cu timpane de sărbătoare și cu chimvale;
\par 7 Și jucând, femeile strigau și ziceau: "Saul a biruit mii, iar David zeci de mii!"
\par 8 De aceea s-a mâniat Saul foarte tare, neplăcându-i cuvintele acestea și a zis: "Lui David i s-au dat zeci de mii, iar mie numai mii; acum numai domnia îi mai lipsește".
\par 9 Și din ziua aceasta în tot timpul următor, s-a uitat la David bănuitor.
\par 10 Iar a doua zi s-a întâmplat de a căzut duhul cel rău de la Dumnezeu asupra lui Saul și acesta se îndrăcea în casa sa, iar David cânta cu mâna sa pe strune, ca și în alte zile; Saul avea în mână o lance.
\par 11 Și a aruncat Saul lancea, cugetând: "Voi pironi pe David de perete!" Dar David s-a ferit de două ori de Saul.
\par 12 Și a început a se teme Saul de David, pentru că Domnul era cu el, iar de Saul se depărtase.
\par 13 De aceea Saul l-a îndepărtat de la sine și l-a pus căpetenie peste o mie; și se ducea și se întorcea el în fruntea poporului.
\par 14 David în toate treburile sale se purta cu chibzuință și Domnul era cu el.
\par 15 Saul vedea că este foarte chibzuit și se temea de el.
\par 16 Iar Israelul tot și Iuda iubea pe David, pentru că el se ducea și se întorcea în fruntea lor.
\par 17 Deci a zis Saul către David: "Iată fata mea cea mai mare, Merob, îți voi da-o de soție, numai să-mi fii viteaz și să duci războaiele Domnului". Căci Saul socotea: "Lasă, să nu fie mâna mea asupra lui, ci să fie asupra lui mâna Filistenilor".
\par 18 David însă a zis către Saul: "Cine sunt eu și ce este viața mea și neamul tatălui meu în Israel, ca să fiu ginerele regelui?"
\par 19 Iar când a venit vremea să dea pe Merob, fiica lui Saul, după David, ea a fost măritată cu Adriel din Mehola.
\par 20 Pe David însă îl iubea altă fată a lui Saul, Micol; și când i s-a spus despre aceasta lui Saul, aceasta i-a plăcut;
\par 21 Căci Saul cugeta: "Am s-o dau după el și ea are să-i fie cursă și mâna Filistenilor are să fie asupra lui". Și a zis Saul către David: "A doua oară te înrudești acum cu mine".
\par 22 Atunci a poruncit Saul slujitorilor săi: "Spuneți lui David: Iată regele este binevoitor către tine și toți slujitorii lui te iubesc; fii dar ginerele meu!"
\par 23 Și au vorbit slujitorii lui Saul în urechile lui David toate vorbele acestea. Iar David a zis: "Oare ușor lucru vi se pare vouă a fi ginerele regelui? Eu sunt un sărac și un neînsemnat".
\par 24 Și au înștiințat pe rege slugile sale și au zis: "Iată ce a spus David".
\par 25 Iar Saul a zis: "Așa să-i spuneți lui David: Regele nu voiește zestre decât numai o sută de prepuțuri filistene, ca răzbunare împotriva vrăjmașilor regelui". Căci Saul avea în gând să piardă pe David prin mâna Filistenilor.
\par 26 Și slugile lui Saul au spus lui David cuvintele acestea și i-a plăcut lui David să se facă ginerele regelui.
\par 27 Dar nu apucase încă să vină ziua sorocită, când David se sculă și merse el însuși și oamenii lui și ucise 200 de Filisteni; și aduse David prepuțurile lor și le înfățișă regelui număr deplin, ca să se poată face ginerele regelui. Și a dat Saul după el pe Micol, fiica sa, de femeie.
\par 28 Dar văzând și aflând Saul că Domnul este cu David și tot Israelul îl iubește și că și fiica sa Micol îl iubește,
\par 29 Începu încă și mai mult să se teamă de David și s-a făcut vrăjmașul lui pe viață și pe moarte.
\par 30 De aceea, când au ieșit la război căpeteniile Filistenilor, David, chiar de la ieșirea lor, lucra mai înțelepțește decât toate slugile lui Saul și numele lui a ajuns foarte vestit.

\chapter{19}

\par 1 Atunci a zis Saul către Ionatan, fiul său și către toate slugile sale, să ucidă pe David. Ionatan fiul lui Saul însă iubea foarte mult pe David.
\par 2 Și a vestit Ionatan pe David, zicând: "Tatăl meu Saul caută să te omoare. Deci să te păzești mine; ascunde-te și stai la loc tainic;
\par 3 Iar eu voi ieși și voi fi lângă tatăl meu în câmp unde vei fi tu și voi vorbi tatălui meu de tine și ce voi vedea iți voi spune".
\par 4 Ionatan a vorbit de bine lui Saul, tatăl său, pentru David și i-a zis: "Să nu greșească regele împotriva robului tău David, căci el cu nimic n-a greșit împotriva ta și faptele lui sunt foarte folositoare pentru tine.
\par 5 El și-a pus viața în primejdie, ca să lovească pe Filisteni și Domnul a făcut izbăvire mare la tot Israelul. Tu ai văzut aceasta și te-ai bucurat. De ce dar vrei tu să păcătuiești împotriva unui sânge nevinovat și să ucizi pe David fără nici o pricină?"
\par 6 Și a ascultat Saul glasul lui Ionatan și s-a jurat Saul: "Viu este Domnul! David nu va muri!"
\par 7 Atunci a chemat Ionatan pe David și i-a spus Ionatan toate cuvintele acestea; și a adus Ionatan pe David la Saul și a slujit el ca și mai înainte.
\par 8 Dar a început iarăși războiul și a ieșit David și s-a luptat cu Filistenii și le-a pricinuit înfrângere mare și au fugit ei de el.
\par 9 Iar duhul cel rău de la Dumnezeu a căzut asupra lui Saul și acesta ședea în casa sa și sulița lui era în mâna lui; David însă cânta din harfă.
\par 10 Atunci Saul a vrut să pironească cu sulița pe David de perete, însă David s-a ferit de Saul și sulița s-a înfipt în perete; apoi David a fugit în noaptea aceea și a scăpat.
\par 11 Și a trimis Saul slujitorii acasă la David, ca să-l pândească și să-l omoare până dimineața. Micol însă soția lui David, i-a zis: "Dacă tu nu-ți scapi sufletul în această noapte, dimineață vei fi ucis".
\par 12 Și a dat Micol drumul lui David pe fereastră; iar David, ieșind, a fugit și a scăpat.
\par 13 După aceea Micol a luat un idol și l-a pus în pat, a pus o piele de capră pe capul lui și l-a învelit cu o haină.
\par 14 Când Saul a trimis slujitorii ca să aducă pe David, Micol a zis: "E bolnav!"
\par 15 Saul însă a trimis din nou slujitorii, ca să vadă bine pe David, zicând: "Aduceți-l la mine cu patul, ca să-l omor!"
\par 16 Și au mers slujitorii la casa lui David, dar iată în pat era un idol și pe capul lui o piele de capră.
\par 17 Atunci Saul a zis către Micol: "De ce m-ai amăgit tu așa și ai lăsat pe vrăjmașul meu să fugă?" Iar Micol a răspuns lui Saul: "Pentru că el mi-a zis: Dă-mi drumul, căci de nu, te ucid!"
\par 18 Așa a scăpat David și a fugit și s-a dus la Samuel în Rama și i-a povestit toate cele ce-i făcuse Saul. Apoi a mers el cu Samuel și s-a oprit la Naiotul cel din Rama.
\par 19 Și s-a spus lui Saul: "Iată David este la Naiotul Ramei!"
\par 20 Și a trimis Saul slujitori să prindă pe David; dar când au văzut aceștia ceata proorocilor proorocind și pe Samuel povățuindu-i, S-a pogorât Duhul lui Dumnezeu peste slujitorii lui Saul și au început și ei a prooroci.
\par 21 Spunându-se acestea lui Saul, el a trimis alți slujitori, dar și aceștia au început a prooroci. Apoi Saul a trimis al treilea rând de slujitori și începură și aceștia să proorocească.
\par 22 Mâniindu-se, în sfârșit, Saul a plecat însuși la Rama și a mers până la izvorul cel mare din Soco. Acolo a întrebat și a zis: "Unde sunt Samuel și David?" Și i s-a spus: "Iată aici în Naiotul Ramei".
\par 23 Și a plecat el acolo, la Naiotul Ramei. Dar pe cale S-a pogorât peste el Duhul lui Dumnezeu, iar el a mers proorocind până a ajuns la Naiotul Ramei.
\par 24 Acolo s-a dezbrăcat de haine și a proorocit înaintea lui Samuel și toată ziua aceea și toată noaptea a șezut dezbrăcat. De aceea se zice: "Au doară și Saul este printre prooroci?"

\chapter{20}

\par 1 Atunci David a fugit din Naiotul Ramei și venind a zis către Ionatan: "Ce-am făcut eu oare? Care este strâmbătatea mea și cu ce am greșit înaintea tatălui tău, de-mi caută sufletul meu?"
\par 2 Iar Ionatan i-a răspuns: "Nu, nu vei muri. Iată tatăl meu nu face nici un lucru mare sau mic, fără să-l descopere urechilor mele. Pentru ce dar ar ascunde tatăl meu de mine lucrul acesta? Aceasta nu se poate".
\par 3 David însă s-a jurat și a zis: "Tatăl tău știe bine că eu am dobândit trecere la tine, și de aceea își zice: "Nu trebuie să știe de aceasta Ionatan, ca să nu se amărască. Dar viu este Domnul și viu este sufletul tău; între mine și moarte n-a fost decât un pas".
\par 4 Atunci Ionatan a zis către David: "Tot ce dorește sufletul tău voi face pentru tine".
\par 5 Și David a zis către Ionatan: "Iată mâine este lună nouă și eu trebuie să stau cu regele la masă; dar lasă-mă să mă ascund în câmp până poimâine seară.
\par 6 De va întreba tatăl tău de mine, tu să spui: "David s-a cerut de la mine să se ducă în cetatea sa Betleem, pentru că acolo se face jertfă anuală pentru tot neamul său".
\par 7 Dacă el la aceasta va răspunde: "Bine", atunci este semn de pace pentru robul tău, iar dacă se va mânia, atunci să știi că el a pus la cale lucru rău.
\par 8 Tu însă să faci milă cu robul tău, căci ai primit pe robul tău să facă legământul Domnului cu tine, și, de este vreo vină asupra mea, atunci ucide-mă tu; de ce să mă mai duci la tatăl tău?"
\par 9 Ionatan însă a zis: "În nici un chip nu se va întâmpla aceasta cu tine; căci de voi afla că în adevăr tatăl meu a hotărât să-ți facă vreun lucru rău, nu te voi vesti eu oare despre aceasta?"
\par 10 Și a zis David către Ionatan: "Cine mă va vesti, dacă tatăl tău îți va răspunde aspru?"
\par 11 A zis Ionatan către David: "Hai să ieșim la câmp". Și au ieșit amândoi la câmp.
\par 12 Acolo Ionatan a zis către David: "Viu este Domnul Dumnezeul lui Israel, mâine pe vremea aceasta sau poimâine, voi căuta să aflu de la tatăl meu, și dacă el este binevoitor lui David și eu nu voi trimite la tine și nu voi descoperi aceasta urechilor tale,
\par 13 Atunci așa și așa să facă Domnul cu Ionatan și încă și mai mult să facă. Dacă însă tatăl meu plănuiește să-ți facă rău, aceasta voi descoperi-o urechilor tale și-ți voi da drumul să mergi în pace și să fie Domnul cu tine, cum a fost cu tatăl meu!
\par 14 Dar și tu, de voi mai fi în viață, să-mi arăți mila Domnului.
\par 15 Iar de voi muri, să nu-ți abați mila de la casa mea în veci, chiar și când Domnul ar pierde de pe fața pământului pe toți vrăjmașii lui David".
\par 16 Așa a încheiat Ionatan legământ cu casa lui David și a zis: "Să pedepsească Domnul pe vrăjmașii lui David!"
\par 17 Și iarăși s-a jurat Ionatan lui David pe iubirea sa cea către el, căci îl iubea ca pe sufletul său.
\par 18 Și i-a zis Ionatan: "Mâine este lună nouă și se va întreba despre tine, căci locul tău va fi gol.
\par 19 De aceea poimâine pleacă și grăbește spre locul acela unde te-ai ascuns și înainte și șezi lângă piatra Ezel;
\par 20 Iar eu voi slobozi într-acolo trei săgeți, ca și cum aș trage la țintă.
\par 21 Apoi voi trimite un băiat și-i voi zice: "Du-te de caută săgețile". Și de voi zice băiatului: "Iată săgețile sunt dincoace de tine, ia-le!", atunci să vii la mine, că este pace pentru tine, și viu este Domnul, nimic nu ți se va întâmpla.
\par 22 Dacă însă voi zice băiatului așa: "Iată săgețile sunt dincolo de tine", atunci să pleci, căci Domnul te liberează.
\par 23 Iar la cele ce am grăit eu cu tine, este martor Domnul între mine și tine în veci! "
\par 24 Și s-a ascuns David în câmp și, venind lună nouă, a ieșit regele la masă.
\par 25 Regele a stat la locul său, ca de obicei, pe scaunul de la perete; Ionatan s-a sculat și Abner a stat lângă Saul; iar locul lui David a rămas gol.
\par 26 În ziua aceea Saul nu a zis nimic, căci socotea că aceasta este o întâmplare, că David nu este curat, nu s-a curățit.
\par 27 Dar a venit și ziua a doua după lună nouă și locul lui David a rămas gol. Atunci a zis Saul către fiul său Ionatan: "Pentru ce fiul lui Iesei n-a venit la masă nici ieri, nici astăzi?"
\par 28 Ionatan însă a răspuns lui Saul: "David s-a cerut la mine să meargă la Betleem".
\par 29 Și a zis: Dă-mi voie să mă duc, că în cetatea noastră este jertfă pentru neamul nostru și m-a poftit fratele meu. Deci de am aflat bunăvoință în ochii tăi, mă duc să mă văd cu frații mei. De aceea n-a venit el la masa regelui".
\par 30 Atunci regele s-a mâniat strașnic pe Ionatan și i-a zis: "Fiu netrebnic și neascultător! Oare nu știu eu că te-ai împrietenit cu fiul lui Iesei, spre rușinea ta și spre batjocura mamei tale?
\par 31 Căci atâta vreme cât fiul lui Iesei va fi viu pe pământ, nu ești scutit de primejdie, nici tu, nici regatul tău. Trimite dar acum și adu-mi-l mie, că este hotărât la moarte!"
\par 32 A răspuns Ionatan lui Saul, tatăl său, și i-a zis: "De ce să-l omori? Ce-a făcut el?"
\par 33 Atunci Saul a repezit sulița în el ca să-l lovească. Și a înțeles Ionatan că tatăl său este hotărât să ucidă pe David.
\par 34 Deci s-a sculat Ionatan de la masă, prins de mânie mare, și n-a mâncat a doua zi după lună nouă, pentru că era trist după David și pentru că-l ocărâse tatăl său.
\par 35 A doua zi dimineața a ieșit Ionatan la câmp, la vremea sorocită lui David, și un băiat mic a ieșit cu el.
\par 36 Și a zis el băiatului: "Fugi și caută săgețile pe care am să le slobod eu!" Și a alergat băiatul, iar el a slobozit săgețile, așa încât au căzut dincolo de băiat.
\par 37 Și a alergat băiatul spre locul unde aruncase Ionatan săgețile. Iar Ionatan a strigat în urma lui și a zis: "Vezi că săgețile sunt înaintea ta".
\par 38 Apoi iar a strigat Ionatan după băiat: "Umblă mai repede și nu te opri!" Băiatul a adunat săgețile lui Ionatan și a venit la stăpânul său.
\par 39 Băiatul însă nu știa nimic din toate acestea; numai Ionatan și David știau de ce este vorba.
\par 40 Și a dat Ionatan arma băiatului, care era cu el, și i-a zis: "Du-te și o du în cetate".
\par 41 După ce s-a dus băiatul, David s-a ridicat din partea de miazăzi a stâncii și s-a închinat de trei ori; apoi s-au sărutat ei unul pe altul și au plâns amândoi, împreună, iar David a plâns mai tare.
\par 42 Și a zis Ionatan către David: "Mergi cu pace! Iar cele pentru care ne-am jurat noi amândoi pe numele Domnului zicând: "Domnul să fie între mine și tine și între copiii mei și copiii tăi, aceea să fie pe veci". Și s-a sculat David și s-a dus, iar Ionatan s-a întors în cetate.

\chapter{21}

\par 1 După aceea a mers David în Nobe, la preotul Ahimelec și s-a mixat Ahimelec la întâlnirea cu David și i-a zis: "De ce ești singur și nu este nimeni cu tine?"
\par 2 Iar David a răspuns preotului Ahimelec: "Regele mi-a încredințat o taină și mi-a zis: Să nu știe nimeni pentru ce te-am trimis și ce însărcinare ți-am dat. De aceea mi-am lăsat oamenii într-un loc anumit.
\par 3 Dă-mi dar ce ai la îndemână, vreo cinci pâini, sau ce se va găsi!"
\par 4 Preotul însă a răspuns lui David și i-a zis: "Pâine obișnuită n-am la îndemână, dar este pâine sfințită; dacă oamenii tăi s-au înfrânat de la femei, pot să mănânce".
\par 5 Iar David a răspuns preotului și i-a zis: "Femei n-am avut cu noi nici ieri, nici alaltăieri, de când am plecat, și vasele (trupurile) oamenilor sunt curate; deși călătoria nu este după orânduială religioasă, pâinea va rămâne curată în vasele (trupurile) lor".
\par 6 Și i-a dat preotul pâinea sfințită, căci nu avea altă pâine, afară de pâinile punerii înainte, care fuseseră luate de la fala Domnului, ca să se pună în locul lor pâini proaspete.
\par 7 În ziua aceea se afla acolo înaintea Domnului unul din slujitorii lui Saul, cu numele Doeg, idumeu, căpetenia păstorilor lui Saul.
\par 8 Și a zis David către Ahimelec: "N-ai cumva la îndemână vreo suliță sau vreo sabie? Căci eu nu mi-am luat nici sabia, nici altă armă, deoarece porunca regelui a fost grabnică".
\par 9 Preotul însă a răspuns: "Iată sabia lui Goliat filisteanul pe care l-ai ucis în Valea Stejarului; ea este învelită într-o haină, după efod; de vrei, ia-o; alta afară de aceea n-am aici". David a răspuns: "Ca ea nu mai este alta, dă-mi-o!" și i-a dat-o.
\par 10 Apoi David s-a sculat și a fugit în aceeași zi de la fața lui Saul și a mers și s-a dus la Achiș, regele din Gat.
\par 11 Iar slugile lui Achiș au zis acestuia: "Oare nu este acesta David, regele țării aceleia, și nu lui oare i se cânta în cor și se zicea: "Saul a biruit mii, iar David zeci de mii?"
\par 12 David a pus cuvintele acestea la inimă și s-a temut tare de Achiș, regele din Gat,
\par 13 Și s-a prefăcut nebun înaintea ochilor lui, făcând năzdrăvănii și scriind pe uși; mergea în mâini și lăsa să-i curgă balele pe barbă.
\par 14 Atunci a zis Achiș robilor săi: "Nu vedeți că este un om nebun? La ce l-ați adus la mine?
\par 15 N-am eu destui nebuni? De ce l-ați adus și pe acesta să se schimonosească înaintea mea? Nu cumva vreți să intre în casă la mine?"

\chapter{22}

\par 1 Și a plecat David de acolo și a fugit în peștera Adulam. Auzind aceasta, frații lui și toată casa tatălui său au venit la el.
\par 2 Și s-au adunat la el toți prigoniții, toți datornicii și toți cei cu sufletul amărât și s-a făcut el căpetenie peste ei; și erau cu el ca la patru sute de oameni.
\par 3 De acolo David s-a dus la Mițpa Moabului și a zis către regele Moabului: "Lasă pe tatăl meu și pe mama mea să stea la voi, până voi afla ce are să facă Dumnezeu cu mine".
\par 4 Și i-a adus la regele Moabului și au trăit ei tot timpul la el, cât David a rămas în cetatea aceea.
\par 5 Dar proorocul Gad a zis lui David: "Nu mai rămâne în cetatea aceasta, ci pleacă și mergi în pământul lui Iuda". Și a plecat David și a venit în pădurea Heret.
\par 6 Dar a auzit Saul că s-a ivit David și oamenii cei ce erau cu el. Saul ședea atunci în Ghibeea, pe deal, sub un stejar, cu sulița în mână și toate slugile sale stăteau împrejurul lui.
\par 7 Zis-a Saul către slugile cele dimprejurul lui: "Ascultați, fiii lui Veniamin. Oare tuturor vă va da fiul lui Iesei țarini și vii, și vă va pune pe toți sutași și căpetenii peste mii,
\par 8 De v-ați sfătuit cu toții în contra mea și nimeni nu mi-a descoperit, când fiul meu a intrat în prietenie cu fiul lui Iesei, și nimeni din voi n-a avut milă de mine și nu mi-a descoperit că fiul meu a ațâțat împotriva mea pe robul meu să-mi urzească intrigi, cum se vede acum?"
\par 9 Atunci a răspuns Doeg idumeul, care stătea cu slugile lui Saul, și a zis: "Eu am văzut cum a venit fiul lui Iesei în Nobe, la Ahimelec, fiul lui Ahituv,
\par 10 Și acela a întrebat pentru el pe Domnul și i-a dat merinde; ba i-a dat și sabia lui Goliat filisteanul.
\par 11 Atunci a trimis regele să cheme pe Ahimelec, fiul lui Ahituv preotul, și toată casa tatălui lui, preoții din Nobe. Și au venit ei cu toții la rege.
\par 12 Iar Saul le-a zis: "Ascultă, fiul lui Ahituv!" Și acela răspunse: "Da, domnul meu!"
\par 13 Și a zis Saul către el: "Pentru ce v-aii unit voi împotriva mea, tu și fiul lui Iesei, că i-ai dat pâini și sabie și ai întrebat pentru el pe Dumnezeu, ca să se răzvrătească împotriva mea și să mă pândească, cum se vede acum?"
\par 14 A răspuns Ahimelec regelui și a zis: "Cine din toți robii tăi este credincios ca David? Și apoi el este și ginerele regelui, îndeplinitorul poruncilor tale, și cu cinste în casa ta.
\par 15 Și apoi oare de astăzi am început eu să întreb pe Dumnezeu pentru el? Nu, nu învinui de asta, o, rege, pe robul tău și toată casa tatălui meu, căci în toată pricina aceasta nu cunoaște robul tău nici un lucru mare sau mic".
\par 16 Atunci regele a zis: "Tu, Ahimelec, trebuie să mori, tu și toată casa tatălui tău".
\par 17 Apoi regele a zis către paznicii lui, care stăteau împrejurul său: "Mergeți și omorâți pe preoții Domnului, căci și mâna lor este cu David; au știut că el a fugit și nu mi-au descoperit". Paznicii regelui însă n-au voit să-și ridice mâna, ca să ucidă pe preoții Domnului.
\par 18 Atunci regele a zis lui Doeg: "Mergi tu și ucide pe preoți". Și s-a dus Doeg idumeul și a năvălit asupra preoților și a ucis în ziua aceea optzeci și cinci de bărbați care purtau efod de in,
\par 19 Iar cetatea preoțească Nobe a trecut-o prin ascuțișul sabiei: și bărbați și femei și tineri și copii și boi și asini și oi, tot a trecut prin ascuțișul sabiei.
\par 20 A scăpat numai un singur fiu al lui Ahimelec, fiul lui Ahituv, anume Abiatar, și a fugit la David.
\par 21 Și a spus Abiatar lui David că Saul a ucis pe preoții Domnului.
\par 22 Atunci David a zis lui Abiatar: "Am știut eu din ziua aceea că, fiind acolo, Doeg idumeul va da de știre negreșit lui Saul și eu sunt vinovat pentru toate sufletele casei tatălui tău.
\par 23 Rămâi la mine și nu te teme, căci cine va căuta sufletul meu are să caute și sufletul tău; tu vei fi aici, la mine, în pază!"

\chapter{23}

\par 1 Atunci i s-a vestit lui David și i s-a spus: "Iată Filistenii au năvălit în Cheila și pradă ariile".
\par 2 Și a întrebat David pe Domnul, zicând: "Să merg oare să lovesc pe acești Filisteni?" Iar Domnul a răspuns lui David: "Mergi, lovește pe Filisteni și izbăvește Cheila!"
\par 3 Dar cei ce erau cu David i-au zis: "Iată noi ne temem aici în Iuda. Cum dar să mergem în Cheila contra taberelor filistene? Vrei să cădem pradă Filistenilor?"
\par 4 Atunci David a întrebat din nou pe Domnul și Domnul i-a răspuns și i-a zis: "Scoală și mergi la Cheila, căci Eu am să dau pe Filisteni în mâinile tale".
\par 5 Și s-a dus David cu oamenii săi la Cheila, de s-a luptat cu Filistenii, le-a luat vitele, le-a pricinuit înfrângere mare și a salvat David pe locuitorii din Cheila.
\par 6 Când Abiatar, fiul lui Ahimelec, a fugit la David și apoi s-a dus cu el la Cheila, a adus cu sine și efodul.
\par 7 Atunci s-a spus lui Saul că David a mers la Cheila; iar Saul a zis: "Dumnezeu l-a dat în mâinile mele, căci a intrat în cetate și s-a închis cu porii și cu zăvoare".
\par 8 Și a chemat Saul tot poporul la război, ca să meargă la Cheila să împresoare pe David și pe oamenii lui.
\par 9 Când însă David a aflat că Saul i-a pus gând rău, a zis preotului Abiatar: "Adu efodul Domnului!"
\par 10 Apoi David a adăugat: "Doamne, Dumnezeul lui Israel, robul Tău a aflat că Saul vrea să vină la Cheila să dărâme cetatea din pricina mea.
\par 11 Mă vor da locuitorii din Cheila pe mâinile lui și va veni Saul aici, cum a auzit robul Tău? Doamne Dumnezeul lui Israel, descoperă aceasta robului Tău". Iar Domnul a zis: "Va veni!"
\par 12 Și a zis David: "Mă vor da locuitorii din Cheila pe mine și oamenii mei în mâinile lui Saul?" Și a zis Domnul: "Te vor da!"
\par 13 Atunci s-a ridicat David și oamenii lui ca la șase sute de inși, au ieșit din Cheila și s-au dus unde au putut. Lui Saul însă i s-a spus că David a fugit din Cheila și atunci el și-a schimbat planul.
\par 14 Iar David a petrecut prin pustiu în locuri nestrăbătute și apoi pe un munte din pustiul Zif. Saul îl căuta în toate zilele, dar Dumnezeu nu l-a dat în mâinile lui.
\par 15 David văzuse că Saul a ieșit să caute sufletul lui, dar el se afla într-o pădure din pustiul Zif.
\par 16 Atunci s-a sculat Ionatan, fiul lui Saul, a venit la David în pădure și l-a întărit cu nădejdea în Dumnezeu,
\par 17 Zicându-i: "Nu te teme, căci nu te va găsi mâna tatălui meu Saul și tu vei împărăți peste Israel, iar eu voi fi al doilea după tine; Saul, tatăl meu, știe aceasta".
\par 18 Și au încheiat ei între ei legământ înaintea feței Domnului. Apoi Ionatan s-a întors la casa sa, iar David a rămas în pădure.
\par 19 Atunci au venit Zifeii la Saul în Ghibeea și au zis: "Iată David stă ascuns la noi prin locuri nestrăbătute, în pădure, pe muntele Hachila, care vine la dreapta Ieșimonului.
\par 20 Așadar, o, rege, mergi după dorința sufletului tău, iar treaba noastră va fi să-l dăm în mâinile regelui".
\par 21 Saul însă le-a zis: "Binecuvântați să fiți voi la Domnul, că ați avut milă de mine.
\par 22 Mergeți și vă mai încredințați încă; cercetați și vedeți locul lui, pe unde îi calcă piciorul și cine l-a văzut acolo, căci mie mi se spune că este foarte șiret.
\par 23 Cercetați și aflați toate ascunzișurile în care se dosește; apoi întoarceți-vă la mine cu știri amănunțite și eu voi merge cu voi, de este în acea țară; îl voi căuta în toate miile lui Iuda".
\par 24 S-au sculat deci aceia și s-au dus la Zif, înainte de Saul. David însă și oamenii lui erau în pustia Maon, în șes, la dreapta Ieșimonului.
\par 25 Și a plecat Saul cu oamenii săi să-l caute, dar David a fost vestit de aceasta și a trecut spre stâncă, rămânând în pustia Maon. De aceasta a auzit și Saul și a alergat după David în pustia Maon:
\par 26 Saul mergea pe o coastă a muntelui, iar David cu oamenii săi se afla pe cealaltă coastă a muntelui. Când David grăbea să se depărteze de Saul, iar Saul cu oamenii lui se sileau să împresoare pe David și pe oamenii lui, ca să-i prindă,
\par 27 Atunci a venit la Saul un crainic și a zis: "Grăbește și vino, că Filistenii au intrat în țară".
\par 28 Atunci s-a întors Saul din urmărirea lui David și s-a dus în întâmpinarea Filistenilor, din care pricină s-a și numit locul acela: Sela-Hamahlecot (Stânca împărțirii).
\par 29 David însă, plecând de acolo, petrecea prin locurile neprimejdioase ale deșertului Enghedi.

\chapter{24}

\par 1 Iar după ce s-a întors Saul de la Filisteni, i s-a spus, zicându-i-se: "Iată David este în pustiul Enghedi".
\par 2 Atunci a luat Saul trei mii de bărbați aleși din tot Israelul și s-a dus să caute pe David și oamenii lui pe stânci, unde locuiesc căprioarele;
\par 3 Și a mers până la o stână de oi, care era lângă drum; acolo era o peșteră și Saul a intrat în ea pentru nevoile sale; David însă și oamenii lui ședeau în fundul peșterii.
\par 4 Atunci au zis către David oamenii lui: "Aceasta este ziua de care îi-a vorbit Domnul, zicând: "Iată Eu voi da pe vrăjmașul tău în mâinile tale și vei face cu el ce vei vrea".
\par 5 David s-a sculat și a tăiat încetișor poala hainei de deasupra a lui Saul.
\par 6 Apoi a zis către oamenii săi: "Să mă ferească Dumnezeu să fac aceasta domnului meu, unsul Domnului, și să-mi ridic mâna mea asupra lui, căci este unsul Domnului".
\par 7 Și așa a oprit David pe oamenii săi cu aceste cuvinte și nu i-a lăsat să se ridice asupra lui Saul. Iar Saul s-a sculat și a ieșit din peșteră la drum.
\par 8 Apoi s-a sculat și David și, ieșind din peșteră, a strigat după Saul și a zis: "Domnul meu, rege!" Saul s-a uitat înapoi, iar David s-a aruncat cu fața la pământ și i s-a închinat.
\par 9 Apoi a zis David către Saul: "De ce asculți de vorbele oamenilor care zic: Iată David uneltește rele împotriva ta?
\par 10 Iată, astăzi văd ochii tăi că Domnul te-a dat acum în mâinile mele, aici în peșteră, și mie mi s-a zis să te ucid; eu însă te-am cruțat și am zis: Nu voi ridica mâna asupra domnului meu, pentru că este unsul Domnului.
\par 11 Privește, părintele meu, poala hainei tale în mâinile mele; ți-am tăiat poala hainei, dar de ucis nu te-am ucis. Află dar și te încredințează că nu este rău în mâna mea, nici vicleșug și n-am greșit cu nimic împotriva ta; tu însă cauți sufletul meu ca să-l iei.
\par 12 Să judece dar Domnul între mine și între tine și să mă răzbune împotriva ta; dar mâna mea nu va fi asupra ta. Din nelegiuiți, nelegiuiți ies, dar mâna mea nu va fi asupra ta.
\par 13 Răul de la cel rău vine, zice vechea zicală. De aceea eu nu voi pune mâna pe tine.
\par 14 Asupra cui a ieșit regele lui Israel? După cine alergi tu? După un câine mort, după un purice.
\par 15 Domnul să fie judecător și să ne judece pe amândoi. El va cerceta, va descurca pricina mea și mă va izbăvi din mâinile tale!"
\par 16 După ce David a isprăvit de vorbit cuvintele acestea către Saul, Saul a zis: "Al tău e oare glasul acesta, fiul meu David? Și ridicându-și glasul, a plâns.
\par 17 Și a zis către David: "Tu ești mai drept decât mine, căci mi-ai răsplătit cu bine, iar eu te-am răsplătit cu rău;
\par 18 Tu astăzi ai dovedit aceasta, purtându-te cu mine milostiv; când Domnul m-a dat în mâinile tale, tu nu m-ai omorât.
\par 19 Cine oare, prinzând pe vrăjmașul său, i-ar da drumul să meargă cu bine? Domnul să-ți răsplătească cu bine pentru ceea cea ai făcut tu astăzi cu mine!
\par 20 De acum știu că fără îndoială vei domni și regatul lui Israel va fi tare în mâna ta.
\par 21 Așadar, jură-mi pe Domnul că nu vei stârpi pe urmașii mei și nu vei șterge numele meu din casa tatălui meu".
\par 22 Și s-a jurat David lui Saul. Apoi Saul s-a dus la casa sa, iar David și oamenii săi s-au suit în niște locuri întărite.

\chapter{25}

\par 1 În vremea aceea a murit Samuel și s-a adunat tot Israelul de l-au plâns și l-au îngropat în casa lui, în Rama. Iar David s-a sculat și s-a coborât în pustiul Maonului.
\par 2 În Maon era un om foarte bogat, care-și avea turmele în Carmel; acesta avea trei mii de oi și o mie de capre și venise în Carmel la tunsul oilor sale. Numele omului aceluia era Nabal, iar numele femeii lui era Abigail.
\par 3 Femeia aceasta era foarte deșteaptă și frumoasă la chip, iar el era om aspru și rău la nărav; se trăgea din neamul lui Caleb.
\par 4 Auzind David în pustiu că Nabal își tunde oile în Carmel,
\par 5 A trimis zece tineri cărora David le-a zis: "Suiți-vă în Carmel și mergeți la Nabal să-i spuneți multă sănătate din partea mea!
\par 6 Și-i ziceți așa: "Să trăiți! Pace ție! Pace casei tale! Pace la toate ale tale!
\par 7 Am auzit acum că la tine se tund oile; iată păstorii tăi au fost cu noi și noi nu le-am făcut nici un rău; nimic din ale lor nu s-a pierdut în tot timpul șederii lor în Carmel,
\par 8 Întreabă slugile și îți vor spune. Să afle dar băieții aceștia bunăvoință înaintea ochilor tăi, că la zi bună am venit. Dă robilor tăi și fiului tău David ce se îndură mâna ta"
\par 9 S-au dus deci oamenii lui David și au spus lui Nabal, în numele lui David, toate cuvintele acestea. Apoi au tăcut. Nabal însă a sărit și a răspuns trimișilor lui David și a zis:
\par 10 "Cine este David și cine este fiul lui Iesei? Acum sunt o mulțime de robi care fug de la stăpânii lor.
\par 11 Nu cumva voi lua pâinile mele și apa mea și vinul meu și carnea pregătită pentru cei ce tund oile mele și să le dau la niște oameni pe care nu-i știu de unde sunt?"
\par 12 Și s-au dus înapoi pe calea lor oamenii lui David și, întorcându-se, au venit și i-au spus toate vorbele acestea.
\par 13 Atunci David a zis oamenilor săi: "Încingeți-vă fiecare sabia!" Și și-a încins fiecare sabia și a încins însuși David sabia sa; și au plecat după David ca la patru sute de oameni, iar două sute au rămas în tabără.
\par 14 Iar Abigail, femeia lui Nabal, fusese vestită de unul din slujitori, care-i zisese: "Iată David a trimis din pustie soli să salute pe stăpânul nostru, dar el s-a purtat cu el ca un netrebnic.
\par 15 Oamenii aceștia sunt însă foarte buni cu noi, nu ne-au făcut rău și nimic din ale noastre nu s-a pierdut în timpul cât am umblat cu ei când eram la câmp.
\par 16 Ei au fost pentru noi ca un zid de apărare și ziua și noaptea în timpul cât am păscut oile aproape de ei.
\par 17 Așadar gândește-te și vezi ce este de făcut, căci de bună seamă primejdia amenință pe stăpânul nostru și toată casa lui; el însă este om rău și nu putem grăi cu el".
\par 18 Atunci Abigail a luat repede două sute de pâini, două burdufuri de vin, cinci oi gătite, cinci măsuri de grăunțe prăjite, o sută de legături de stafide și două sute de legături de smochine și, încărcându-le pe asini,
\par 19 A zis slugilor sale: "Plecați înaintea mea, căci iată eu vin după voi". Iar bărbatului său, Nabal, nu i-a spus nimic.
\par 20 Când însă ea, șezând pe un asin, se cobora pe drumul șerpuitor al muntelui, iată David și oamenii săi veneau în întâmpinarea ei și ea s-a întâlnit cu el.
\par 21 Atunci David a zis: "În zadar am apărat eu în pustiu toată averea acestui om, încât nimic nu s-î pierdut din cele ce erau ale lui și iată, el îmi plătește cu rău pentru bine.
\par 22 Așa și așa să facă Dumnezeu cu robul Său David și încă și mai mult să facă, dacă până mâine în revărsatul zorilor voi mai lăsa pe cineva de parte bărbătească din tot ce are Nabal".
\par 23 Când Abigail a văzut pe David, s-a coborât repede de pe asină și a căzut înaintea lui David cu fața sa și i s-a închinat până la pământ.
\par 24 Apoi a căzut la picioarele lui și a zis: "Asupra mea este păcatul, domnul meu! Îngăduie roabei tale să vorbească urechilor tale și ascultă cuvintele roabei tale.
\par 25 Să nu întoarcă domnul meu luarea aminte asupra acestui om rău, asupra lui Nabal; căci cum îi e numele, și nebunia se ține de el. Iar eu, roaba ta, n-am văzut pe tinerii domnului meu, pe care i-ai trimis.
\par 26 Dar acum, domnul meu, viu este Domnul și viu este sufletul tău, nu-ți va îngădui Domnul să mergi la vărsare de sânge și va înfrâna mâna ta de la răzbunare.
\par 27 Vrăjmașii tăi, cei ce plănuiesc rele împotriva domnului meu, să ajungă ca Nabal.
\par 28 Iartă vinovăția roabei tale! Domnul negreșit va ridica casă tare domnului meu, căci lupta Domnului luptă domnul meu și rău nu se află în tine în toată viața ta.
\par 29 De se va scula vreun om să te urmărească și să caute sufletul tău, atunci sufletul domnului meu va fi legat în mănunchiul celor vii de lângă Domnul Dumnezeul tău, iar sufletul vrăjmașilor tăi îl va arunca el ca dintr-o praștie.
\par 30 Iar când va face Domnul domnului meu tot binele ce l-a grăit pentru tine și te va pune povățuitor peste Israel,
\par 31 Atunci nu va fi pentru inima domnului meu amărăciune și neliniște faptul că n-a vărsat în zadar sânge și s-a ferit de răzbunare. Iar Domnul va face bine stăpânului meu și-ți vei aduce aminte de roaba ta și-i vei arăta milă".
\par 32 Atunci David a zis către Abigail: "Binecuvântat fie Domnul Dumnezeul lui Israel, Care te-a trimis acum în întâmpinarea mea!
\par 33 Binecuvântată fie mintea ta și binecuvântată să fii și tu, că nu m-ai lăsat acum să merg la vărsare de sânge și să mă răzbun:
\par 34 Dar viu este Domnul Dumnezeul lui Israel, Care m-a oprit de-ați face ție rău; dacă nu te-ai fi grăbit să vii întru întâmpinarea mea, apoi până mâine în revărsatul zorilor n-aș fi lăsat lui Nabal pe nimeni de parte bărbătească".
\par 35 Și a primit David din mâinile ei cele ce-i adusese și i-a zis: "Mergi sănătoasă la casa ta! Iată am ascultat glasul tău și ji-am cinstit fața".
\par 36 După aceea a venit Abigail la Nabal și iată acesta avea ospăț în casa sa, ospăț împărătesc, și inima lui Nabal era veselă, căci băuse de se îmbătase. De aceea nu i-a spus ea nici un cuvânt, nici mare, nici mic, până dimineața.
\par 37 Iar dimineața, când Nabal s-a trezit, femeia sa i-a spus toate cele ce se întâmplaseră. Atunci a încremenit inima lui și el a rămas ca de piatră.
\par 38 Iar după vreo zece zile a lovit Domnul pe Nabal și acesta a murit.
\par 39 Auzind David că Nabal a murit, a zis: "Binecuvântat fie Domnul, Care a răsplătit înjosirea ce mi-a pricinuit-o Nabal și a ferit pe robul Său de la rău. Domnul a întors răul lui Nabal în capul lui". Și a trimis David să spună Abigailei că el o ia de soție.
\par 40 Deci au venit slujitorii lui David la Abigail în Carmel și i-au zis așa: "David ne-a trimis la tine, ca să te luăm să-i fii femeie".
\par 41 Atunci ea s-a sculat și s-a închinat cu fala până la pământ și a zis: "Iată roaba ta este gata să fie slujnică, ca să spele picioarele slugilor domnului meu".
\par 42 și s-a gătit repede Abigail, s-a suit pe asin și cinci slujnice au însoțit-o; și s-a dus ea după trimișii lui David și a ajuns femeia lui.
\par 43 Și a mai luat David și pe Ahinoam din Izreel și au fost amândouă femeile lui.
\par 44 Iar Saul a dat pe fiica sa Micol, femeia lui David, lui Paltiel, fiul lui Laiș, din Galim.

\chapter{26}

\par 1 Atunci au venit Zifeii de la miazăzi la Saul în Ghibeea și au zis: "Iată David stă ascuns la noi, pe muntele Hachila, care vine în fața Ieșimonului".
\par 2 S-a sculat deci Saul și s-a coborât în pustiul Zif, cu trei mii de bărbați, israeliți aleși, ca să caute pe David prin pustiul Zif.
\par 3 Saul și-a așezat tabăra pe muntele Hachila, care vine în fața Ieșimonului, lângă drum; iar David se afla în pustiu și vedea că Saul merge după el în pustiu.
\par 4 Deci a trimis David iscoade și a aflat că Saul venise cu adevărat din Cheila.
\par 5 Și sculându-se David pe ascuns, s-a dus la locul unde se așezase Saul cu tabăra și a văzut David locul unde dormea Saul și Abner, fiul lui Ner, căpetenia oștirii lui. Saul însă dormea în cort, iar oștenii erau așezați împrejurul lui.
\par 6 Apoi întorcându-se, David a grăit cu Ahimelec Heteul și cu Abișai, fiul lui Țeruia, fratele lui Ioab, și a zis: "Cine merge cu mine la Saul în tabără?" Răspuns-a Abișai: "Eu merg cu tine".
\par 7 Și au venit David și Abișai la oamenii lui Saul, noaptea. Saul era culcat și dormea în cort; sulița lui era înfiptă în pământ la căpătâiul lui; iar Abner și oștenii dormeau împrejurul lui.
\par 8 Atunci Abișai a zis lui David: "Acum Dumnezeu a dat pe vrăjmașul tău în mâinile tale; îngăduie-mi deci să-l pironesc de pământ cu sulița dintr-o singură lovitură".
\par 9 David însă a zis către Abișai: "Să nu-l ucizi, căci cine-și va ridica mâna asupra unsului Domnului și va rămâne nepedepsit?"
\par 10 Apoi David a zis: "Viu este Domnul! El dar să-l lovească; sau va veni ziua lui și va muri, sau va merge la război și va pieri; iar mie să nu-mi îngăduie Domnul să-mi ridic mâna asupra unsului Domnului!
\par 11 Dar ia sulița lui, care este la căpătâiul lui, și vasul cu apă și să mergem într-ale noastre".
\par 12 Și a luat David sulița și vasul cu apă de la căpătâiul lui Saul și s-au dus ei într-ale lor și nimeni nu i-a văzut, nici nu i-a simțit; căci nimeni nu s-a deșteptat, că Domnul trimisese peste ei somn adânc și dormeau toți.
\par 13 Iar după ce David a trecut dincolo, a stat pe vârful muntelui, fiind acum depărtare mare între ei.
\par 14 A strigat David către oșteni și către Abner, fiul lui Ner și a zis: "Ei! Abner!" Iar Abner a zis: "Cine ești și de ce strigi să-l tulburi pe rege?"
\par 15 A răspuns David lui Abner: "Au nu ești tu bărbat și cine este asemenea ție în Israel? De ce dar nu păzești pe regele, stăpânul tău? Căci a venit cineva din popor ca să ucidă pe regele, stăpânul tău.
\par 16 Nu faci bine ce faci! Viu este Domnul, voi sunteți vrednici de moarte, pentru că nu păziți pe domnul vostru, unsul Domnului. Privește, unde este sulița regelui și vasul cu apă care era la capul lui!"
\par 17 Saul însă a cunoscut glasul lui David și a zis: "Al tău este glasul acesta, fiul meu?" Iar David a răspuns: "Al meu, domnul meu, rege!"
\par 18 Apoi a adăugat: "Pentru ce urmărește domnul meu pe robul său? Ce-am făcut eu? Sau ce rău este în mâna mea?
\par 19 Să asculte dar regele, stăpânul meu, cuvintele robului său: Dacă Domnul te-a îndemnat împotriva mea, să fie aceasta din partea ta jertfă cu bună mireasmă; iar dacă fiii oamenilor te-au pus la cale, blestemați să fie ei înaintea Domnului, căci ei m-au izgonit acum, ca să nu mai fac parte din moștenirea Domnului, zicând: Mergi de slujește la dumnezei străini.
\par 20 Să nu se verse însă sângele meu pe pământ înaintea Domnului; căci regele lui Israel a ieșit să caute un purice, cum se caută prepelițele pe dealuri".
\par 21 Iar Saul a zis: "Am greșit! Întoarce-te, fiul meu David, că nu-ți voi mai face rău, pentru că sufletul meu a fost acum scump în ochii tăi; nebunește m-am purtat și am greșit foarte mult".
\par 22 David însă a răspuns și a zis: "Iată sulița regelui; să vină unul din oameni și să o ia,
\par 23 Și să-i dea fiecăruia Domnul după dreptatea lui și după credință, deoarece Domnul te-a dat pe tine astăzi în mâinile mele, dar cu n-am vrut să-mi ridic mâna mea asupra unsului Domnului;
\par 24 Și precum a fost acum viața ta prețioasă în ochii mei, așa să se prețuiască viața mea în ochii Domnului și să mă izbăvească El de toată strâmtorarea ".
\par 25 Iar Saul a zis către David: "Binecuvântat să fii tu, fiul meu David; și să lucrezi lucrul tău cu spor și să-l isprăvești cu bine". Și s-a dus David în drumul său, iar Saul s-a întors la locul lui.

\chapter{27}

\par 1 Atunci și-a zis David în inima sa: "Voi cădea cândva în mâinile lui Saul și nu-mi rămâne nimic mai bun decât să fug în țara Filistenilor; astfel Saul va înceta de a mă mai căuta prin toate meleagurile lui Israel, iar eu voi scăpa din mâna lui".
\par 2 S-a sculat deci David și a plecat el însuși și cei șase sute de bărbați care erau cu el, la Achiș, fiul lui Maoc, regele Gatului.
\par 3 Și a trăit David la Achiș în Gat, el și oamenii lui, fiecare cu familia sa, David și amândouă femeile sale, Ahinoam izreeliteanca și Abigail carmeliteanca, fosta femeie a lui Nabal carmelitul.
\par 4 Și i s-a spus lui Saul că David a fugit în Gat și nu s-a mai apucat să-l caute.
\par 5 David însă a zis către Achiș: "De am aflat trecere înaintea ta, atunci să mi se dea loc în una din cetățile țării și voi locui acolo; la ce să locuiască robul tău în cetatea regelui împreună cu tine?"
\par 6 Atunci i-a dat Achiș Țiclagul și de aceea Țiclagul a rămas al regilor Iudei până astăzi.
\par 7 Tot timpul cât a trăit David în țara Filistenilor a fost un an și patru luni.
\par 8 În vremea aceea a ieșit David cu oamenii săi și au năvălit asupra Gheșurenilor, Ghirzenilor și Amaleciților, care locuiau demult această țară până la Șur și până la țara Egiptului.
\par 9 Și a pustiit David țara aceea și n-a lăsat în viață nici bărbat, nici femeie; iar oile și boii, asinii, cămilele și lucrurile le-a luat și, întorcându-se, a venit la Achiș.
\par 10 Iar Achiș a zis către David: "Asupra cui ai năvălit acum?" Răspuns-a David: "Asupra laturii de miazăzi de Ierahmeel și asupra laturii de miazăzi a Cheneilor".
\par 11 Și nu a lăsat David în viață nici bărbat, nici femeie, nici n-a adus în Gat, zicând: "Aceștia ar putea să ne pârască și să zică: Așa a fost David și astfel a fost purtarea lui în timpul șederii lui în țara Filistenilor".
\par 12 Și s-a încrezut Achiș în David, zicând: "Acesta a ajuns să fie urât poporului său Israel și va fi pe veci sluga mea".

\chapter{28}

\par 1 În vremea aceea și-au adunat Filistenii oștirea pentru război, ca să se bată cu Israel. Deci a zis Achiș către David: "Știut să-ți fie că ai să mergi cu mine la război și tu și oamenii tăi".
\par 2 Iar David a răspuns lui Achiș: "Acum ai să afli ce are să facă robul tău". Și a zis Achiș lui David: "De aceea te și fac eu paznicul capului meu pentru totdeauna".
\par 3 Murind Samuel, l-a plâns tot Israelul și l-au îngropat în Rama, cetatea lui. Saul însă izgonise pe cei ce chemau morții și pe ghicitori din țară.
\par 4 S-au adunat deci Filistenii și s-au dus de și-au așezat tabăra la Șunem; și-a adunat și Saul tot poporul lui Israel și și-a așezat tabăra pe Ghilboa.
\par 5 Văzând însă Saul tabăra Filistenilor, s-a spăimântat și s-a tulburat tare inima lui.
\par 6 Și a întrebat Saul pe Domnul, dar Domnul nu i-a răspuns nici în vis, nici prin Urim, nici prin prooroci.
\par 7 Atunci Saul a zis slugilor sale: "Căutați-mi o femeie vrăjitoare, ca să merg la ea s-o întreb". Iar slugile i-au răspuns: "Este aici în Endor o femeie vrăjitoare".
\par 8 Apoi și-a dezbrăcat Saul hainele sale și a îmbrăcat altele și s-a dus el însuși cu doi oameni și au venit la femeie noaptea; și i-a zis Saul: "Rogu-te, ghicește-mi chemând un mort și scoate-mi pe cine îți voi spune eu!"
\par 9 Dar femeia i-a răspuns: "Tu știi ce a făcut Saul, cum a alungat el din țară pe cei ce cheamă morții și pe ghicitori. Pentru ce dar întinzi tu cursă sufletului meu spre pieirea mea?"
\par 10 Și s-a jurat Saul pe Domnul, zicând: "Viu este Domnul, nu vei suferi nici un necaz pentru fapta aceasta".
\par 11 Atunci femeia a întrebat: "Pe cine să-ți scot?" Răspuns-a el: "Pe Samuel să mi-l scoli!"
\par 12 Când a văzut femeia pe Samuel, a răcnit tare. Apoi întorcându-se femeia către Saul, a zis: "Pentru ce m-ai amăgit? Tu ești Saul".
\par 13 I-a zis regele: "Nu te teme. Spune-mi ce vezi?" Și răspunzând, femeia a zis: "Văd parcă un dumnezeu, ieșind din pământ".
\par 14 "Ce înfățișare are?" a întrebat-o regele. Ea a răspuns: "Iese din pământ un bărbat foarte bătrân, îmbrăcat cu o haină lungă". Atunci a cunoscut Saul că acela este Samuel și a căzut cu fața la pământ și s-a închinat.
\par 15 A zis Samuel către Saul: "Pentru ce mă tulburi, ca să ies?" Iar Saul a răspuns: "Îmi este tare greu; Filistenii se luptă împotriva mea, iar Dumnezeu S-a depărtat de mine și nu-mi mai răspunde nici prin prooroci, nici în vis, nici în vedenie; de aceea te-am chemat, ca să mă înveți ce să fac".
\par 16 A zis Samuel: "La ce mă mai întrebi pe mine, dacă Domnul S-a depărtat de tine și S-a făcut vrăjmașul tău?
\par 17 Domnul face ceea ce a grăit prin mine: Va lua Domnul domnia din mâinile tale și o va da lui David, aproapele tău.
\par 18 Deoarece tu n-ai ascultat glasul Domnului și n-ai împlinit iuțimea mâniei Lui asupra lui Amalec, de aceea Domnul face aceasta cu tine acum.
\par 19 Și va da Domnul pe Israel împreună cu tine în mâinile Filistenilor; mâine tu și fiii tăi veți fi cu mine și tabăra lui Israel o va da Domnul în mâinile Filistenilor".
\par 20 Atunci Saul a căzut deodată cu tot trupul său la pământ, căci se spăimântase grozav de cuvintele lui Samuel; afară de aceasta și puterile îl părăsiseră, căci nu mâncase pâine toată ziua aceea și toată noaptea.
\par 21 Și s-a apropiat femeia aceea de Saul și văzând că este tare înspăimântat, a zis: "Iată roaba ta a ascultat glasul tău și și-a pus viața în primejdie și a împlinit porunca ce i-ai dat.
\par 22 Rogu-te dar acum, ascultă și tu glasul roabei tale; îți voi aduce o bucățică de pâine; mănâncă, să prinzi putere, ca să pleci la drum".
\par 23 Dar el n-a voit, ci a zis: "Nu voi mânca!" Și au început slugile lui să-l îndemne, precum și femeia; și el a ascultat glasul lor și s-a sculat de la pământ și a șezut pe pat.
\par 24 Iar femeia avea la casa ei și un vițel îngrășat și s-a grăbit să-l taie; apoi, luând făină, a frământat și a copt azime.
\par 25 Și a pus înaintea lui Saul și slugilor lui și ei au mâncat; apoi s-au sculat și au plecat în aceeași noapte.

\chapter{29}

\par 1 Atunci și-au adunat Filistenii toate cetele la Afec, iar Israeliții și-au așezat tabăra la fântâna cea din Izreel.
\par 2 Căpeteniile Filistenilor mergeau cu sutele și cu miile lor; iar David și cu oamenii lui mergeau în urmă cu Achiș.
\par 3 Căpeteniile Filistenilor însă au zis: "Ce este cu Evreii aceștia?" Achiș a răspuns căpeteniilor Filistenilor: "Nu știți oare că acesta este David, robul lui Saul, regele lui Israel? El este la mine de mai bine de un an și n-am găsit nimic rău la el de când a venit și până acum".
\par 4 Și s-au mâniat pe el căpeteniile Filistenilor: "Dă drumul omului acestuia să se ducă să șadă la locul lui pe care i l-ai hotărât tu și să nu mai meargă cu noi la război pentru ca să nu se facă în război vrăjmașul nostru. Cu ce poate el să dobândească mila domnului său decât cu capetele acestor oameni?
\par 5 Nu este oare el acel David, căruia i se cânta la horă: Saul a biruit mii, iar David zeci de mii?"
\par 6 Atunci a chemat Achiș pe David și i-a zis: "Viu este Domnul! Tu ești om cinstit și ochilor mei le-ar fi plăcut ca tu să intri și să ieși cu mine în tabere; căci eu n-am văzut rău la tine de când ai venit la mine și până în ziua aceasta; dar în ochii căpeteniilor tu nu ești bun.
\par 7 Întoarce-te dar acum și mergi sănătos, ca să nu arăți pe căpeteniile Filistenilor".
\par 8 David însă a zis către Achiș: "Ce-am făcut eu oare și ce-ai găsit tu la robul tău de când am venit înaintea feței țale și până în ziua aceasta? Pentru ce să nu merg și să mă lupt cu vrăjmașii domnului meu, regele?"
\par 9 Răspuns-a Achiș lui David: "Fii încredințat că în ochii mei tu ești bun, ca un înger al lui Dumnezeu; dar căpeteniile Filistenilor au zis: El să nu meargă cu noi la război.
\par 10 Așadar scoală-te dimineață, tu și robii stăpânului tău, care au venit cu tine, și duceți-vă la locul pe care vi l-am rânduit eu și să nu ai gând râu în inima ta. Sculați-vă dar dimineață și, când se va lumina, plecați".
\par 11 Și s-a sculat David, și oamenii lui, dimineața ca să plece și să se întoarcă în pământul Filistenilor, iar Filistenii s-au dus la război în Israel.

\chapter{30}

\par 1 A treia zi după ce David și oamenii lui au plecat la Țiclag, Amaleciții au năvălit din miazăzi asupra Țiclagului și, luându-l, l-au ars cu foc.
\par 2 Iar femeile și pe toți câți erau în el de la mic până la mare nu i-au ucis, ci i-au luat în robie și s-au dus în drumul lor.
\par 3 Când au ajuns David și oamenii săi Ia cetate, iată aceasta era arsă cu foc, iar femeile lor și fiii și fiicele erau duși în robie.
\par 4 Atunci David și poporul ce era cu el au ridicat bocet și au plâns până când li s-au istovit puterile de plâns.
\par 5 Și au fost duse în robie și amândouă femeile lui David, Ahinoam izreeliteanca și Abigail carmeliteanca, fosta femeie a lui Nabal.
\par 6 David a fost tare tulburat, deoarece poporul voise să-l ucidă cu pietre, căci poporul tot era amărât la suflet, fiecare pentru fiii săi și pentru fiicele sale.
\par 7 Dar David s-a întărit cu nădejdea în Domnul Dumnezeul său și a zis către preotul Abiatar, fiul lui Ahimelec: "Adu-mi efodul". Și a adus Abiatar efodul la David.
\par 8 Și a întrebat David pe Domnul zicând: "Să urmăresc eu oare această ceată? O voi ajunge oare?" și i s-a răspuns: "Urmărește-o, o vei ajunge și-i vei lua prăzile".
\par 9 Atunci s-a dus David el însuși și cei șase sute de bărbați care erau cu el și, sosind la pârâul Besor, s-au oprit acolo cei osteniți.
\par 10 David însă cu patru sute de oameni au urmărit înainte, iar două sute de oameni s-au oprit, pentru că n-au mai fost în stare să treacă prin Besor.
\par 11 Atunci au găsit în câmp pe un egiptean și l-au adus la David, i-au dat pâine să mănânce și l-au adăpat cu apă.
\par 12 I-au mai dat încă o jumătate de legătură de smochine, două legături de stafide și a mâncat acela și s-a întărit, căci nu mâncase pâine și nu băuse apă de trei zile și trei nopți.
\par 13 Apoi David i-a zis: "Al cui și de unde ești tu?" Iar acela a zis: "Eu sunt fiu de egiptean, robul unui amalecit și m-a lepădat stăpânul meu, pentru că mă îmbolnăvisem de vreo trei zile.
\par 14 Noi am năvălit în latura de miazăzi a Cheretienilor, în țara lui Iuda și în latura de miazăzi a lui Caleb, iar Țiclagul l-am ars cu foc".
\par 15 I-a zis David: "Poți să mă duci până la această ceată?" și el a zis: "Jură-mi pe Dumnezeu că nu mă vei omorî și nu mă vei da în mâinile stăpânului meu și eu te voi duce la această ceată".
\par 16 David i s-a jurat și el l-a dus. Și iată Amaleciții se risipiseră prin toată țara aceea, mâncau, beau și jucau de bucurie pentru marea pradă care o luaseră ei din țara Filistenilor și din pământul lui Iuda.
\par 17 Atunci a năvălit asupra lor David și i-a măcelărit din zori și până a doua zi seara și nimeni din ei n-a scăpat, afară de patru sute de tineri care s-au suit pe cămile și au fugit.
\par 18 Apoi a luat David toate câte răpiseră Amaleciții și a luat și pe cele două femei ale sale.
\par 19 Și n-a pierit din al lor nimic, nici mare, nici mic, nici din fii, nici din fiice, nici din prăzi; nici din tot ce luaseră Amaleciții de la ei: toate le-a adus înapoi David.
\par 20 Și a luat David toate vitele mari și mici și cei ce mânau turma aceasta ziceau: "Aceasta este prada lui David".
\par 21 După aceea a venit David la cele două sute de oameni care nu fuseseră în stare să-i urmeze și pe care îi lăsase la pârâul Besor; și au ieșit aceștia în întâmpinarea lui David și în întâmpinarea oamenilor care erau cu el. Iar David, apropiindu-se de acei oameni, i-a întrebat de sănătate.
\par 22 Atunci cei răi și netrebnici dintre oamenii cei ce merseseră cu David au început să zică: "Pentru că ei n-au mers cu noi, nu le vom da din prăzile ce am luat, ci să-și ia fiecare numai femeia sa și copiii săi și să plece".
\par 23 David însă a zis: "Să nu faceți așa cu frații mei, acum, după ce Domnul ne-a dat nouă acestea și ne-a păzit și a dat în mâinile noastre taberele celor ce veniseră asupra noastră.
\par 24 Și apoi cine vă va asculta pe voi, în această treabă? Aceștia nu sunt mai răi decât noi. Ce parte au cei ce au mers la război, aceeași parte trebuie să se împartă și celorlalți".
\par 25 Așa a fost totdeauna, din timpul acela și până acum, căci a pus aceasta ca lege și ca dreptar până în ziua de astăzi.
\par 26 Apoi a venit David în Țiclag și a trimis din prăzi bătrânilor lui Iuda, prietenii săi:
\par 27 Celor din Betel, din Rama de miazăzi și din Iatir;
\par 28 Celor din Aroer, din Amada, din Sifmot, din Eștemoa și din Gat;
\par 29 Celor din Chimat, din Safec, din Timat, din Racal, din cetățile Ierahmeeliților și din cetățile cheneene;
\par 30 Celor din Horma, din Corașan și din Atac;
\par 31 Celor din Hebron și din toate locurile pe unde fusese David și oamenii lui, zicând: "Iată vă trimit dar din prăzile luate de la vrăjmașii Domnului".

\chapter{31}

\par 1 Filistenii însă s-au bătut cu Israeliții și au fugit aceștia de Filisteni și au căzut uciși pe muntele Ghelboa.
\par 2 Și au ajuns Filistenii pe Saul și pe fiii lui și au ucis pe Ionatan, pe Aminadab și pe Melchișua, fiii lui Saul.
\par 3 Lupta contra lui Saul ajunsese cumplită și arcașii îl loviră pe acesta, rănindu-l greu.
\par 4 Atunci a zis Saul purtătorului său de arme: Trage-ți sabia și mă străpunge cu ea, ca să nu vină acești netăiați împrejur să mă ucidă și să-și bată joc de mine". Purtătorul de arme însă n-a voit, căci se temea cumplit. Atunci Saul și-a luat sabia și s-a aruncat în ea.
\par 5 Văzând purtătorul de arme că Saul a murit, s-a aruncat și el în sabia sa și a murit cu el.
\par 6 Așa a murit în ziua aceea Saul și cei trei fii ai lui și purtătorul de arme al său, precum și toți oamenii lui.
\par 7 Israeliții, care locuiau peste vale și peste Iordan, văzând că ostașii israeliți au fugit și că Saul și fiii lui au murit, și-au părăsit cetățile și au fugit, iar Filistenii au venit și s-au așezat în ele.
\par 8 A doua zi Filistenii au venit să jefuiască pe cei uciși și au găsit pe Saul și pe cei trei fii ai lui căzuți pe muntele Ghelboa.
\par 9 Ei i-au tăiat capul, au luat armele de pe el și au trimis în toată țara Filistenilor ca să se vestească despre aceasta în capiștile idolilor lor și poporului.
\par 10 Armele lui le-au pus în capiștea Astartei, iar trupul lui l-au spânzurat pe zidurile cetății Bet-San.
\par 11 Auzind locuitorii Iabeșului din Galaad cele ce făcuseră Filistenii cu Saul,
\par 12 S-au ridicat toți oamenii puternici, au mers toată noaptea și au luat trupul lui Saul și trupurile fiilor lui de pe zidurile cetății Bet-San și le-au adus în Iabeș și le-au ars acolo;
\par 13 Iar oasele lor le-au luat și le-au îngropat sub un stejar în Iabeș. Apoi au postit șapte zile.


\end{document}