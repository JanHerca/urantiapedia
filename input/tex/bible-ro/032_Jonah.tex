\begin{document}

\title{Iona}


\chapter{1}

\par 1 Și a fost cuvântul Domnului către Iona, fiul lui Amitai, zicând:
\par 2 "Scoală-te și du-te în cetatea cea mare a Ninivei și propovăduiește acolo, căci fărădelegile lor au ajuns până în fața Mea!"
\par 3 Și s-a sculat Iona să fugă la Tarsis, departe de Domnul. Și s-a coborât la Iope, unde a găsit o corabie, care mergea la Tarsis, și, plătind prețul călătoriei, s-a coborât în ea ca să meargă la Tarsis împreună cu toți cei de acolo, el fugind din fața Domnului.
\par 4 Dar Domnul a ridicat un vânt năpraznic pe mare și o furtună puternică s-a stârnit, încât corabia era gata să se sfărâme.
\par 5 Corăbierii s-au înfricoșat și au strigat fiecare către dumnezeul său și au aruncat în mare încărcătura corăbiei ca să se ușureze. Dar Iona se coborâse în fundul corăbiei, se culcase și adormise.
\par 6 Atunci s-a apropiat de el cârmaciul corăbiei, și i-a zis: "Pentru ce dormi? Scoală-te și strigă către Dumnezeul tău, poate El Își va aduce aminte de noi, ca să nu pierim!"
\par 7 Și au zis unul către altul: "Haidem să aruncăm sorți, ca să știm din pricina cui a venit peste noi nenorocirea aceasta!" Și au aruncat sorți, și sorțul a căzut pe Iona.
\par 8 Și l-au întrebat pe el: "Spune-ne nouă din pricina cui s-a abătut nenorocirea aceasta asupra noastră? Care este meșteșugul tău, de unde și din ce țară vii și din ce popor ești?"
\par 9 Atunci el le-a răspuns: "Sunt evreu și Domnului Dumnezeului cerului mă închin - Cel care a făcut marea și uscatul".
\par 10 Și toți oamenii s-au temut cu frică mare și i-au zis lui: "Pentru ce ai săvârșit una ca aceasta?" Căci ei știau că el fuge din fața lui Dumnezeu, fiindcă el le spusese.
\par 11 Și i-au zis lui: "Ce să-ți facem ca să se potolească marea?" Căci marea se ridica din ce în ce mai mult.
\par 12 Atunci el a răspuns: "Luați-mă și mă aruncați în mare și ea se va potoli, căci știu bine că din pricina mea s-a pornit peste voi această vijelie".
\par 13 Și marinarii vâsleau ca să ajungă la țărm, dar în zadar, căci marea se ridica din ce în ce mai mult împotriva lor.
\par 14 Atunci au strigat către Domnul și au zis: "O, Doamne, de-am putea să nu pierim din pricina vieții acestui om și să nu ne împovărezi pe noi cu un sânge nevinovat! Că Tu, Doamne, precum ai voit ai făcut!"
\par 15 Și îl ridicară pe Iona și îl aruncară în mare și s-a potolit urgia ei.
\par 16 Și oamenii s-au temut cu teamă mare de Domnul și au adus jertfă lui Dumnezeu și I-au făcut Lui făgăduințe.
\par 17 Și Dumnezeu a dat poruncă unui pește mare să înghită pe Iona. Și a stat Iona în pântecele peștelui trei zile și trei nopți.

\chapter{2}

\par 1 Atunci s-a rugat Iona din pântecele peștelui către Domnul Dumnezeul lui, zicând:
\par 2 "Strigat-am către Domnul în strâmtorarea mea, și El m-a auzit; din pântecele locuinței morților către El am strigat, și El a luat aminte la glasul meu!
\par 3 Tu m-ai aruncat în adânc, în sânul mării și undele m-au înconjurat; toate talazurile și valurile Tale au trecut peste mine.
\par 4 Și gândeam: Aruncat sunt dinaintea ochilor Tăi! Dar voi vedea din nou templul cel sfânt al Tău!
\par 5 Apele m-au învăluit pe de-a întregul, adâncul m-a împresurat, iarba mării s-a încolăcit în jurul capului meu;
\par 6 Mă coborâsem până la temeliile munților, zăvoarele pământului erau trase asupra mea pentru totdeauna, dar Tu ai scos din stricăciune viața mea, Doamne Dumnezeul meu!
\par 7 Când se sfârșea în mine duhul meu, de Domnul mi-am adus aminte, și la Tine a ajuns rugăciunea mea, în templul Tău cel sfânt!
\par 8 Cei ce slujesc idolilor deșerți disprețuiesc harul Tău;
\par 9 Dar eu Îți voi aduce Ție jertfe cu glas de laudă și toate făgăduințele mele le voi împlini, căci mântuirea vine de la Domnul!"
\par 10 Și Domnul a dat poruncă peștelui și peștele a aruncat pe Iona la țărm!

\chapter{3}

\par 1 Și a fost cuvântul Domnului către Iona, pentru a doua oară, zicând:
\par 2 "Scoală și pornește către cetatea cea mare a Ninivei și vestește-le ceea ce îți voi spune!"
\par 3 Și s-a sculat Iona și a mers în Ninive, după cuvântul Domnului. Și Ninive era cetate mare înaintea lui Dumnezeu; îți trebuia trei zile ca s-o străbați.
\par 4 Și a pătruns Iona în cetate, zicând: "Patruzeci de zile mai sunt, și Ninive va fi distrusă!"
\par 5 Atunci Ninivitenii au crezut în Dumnezeu, au ținut post și s-au îmbrăcat cu sac, de la cei mai mari și până la cei mai mici.
\par 6 Și a ajuns vestea până la regele Ninivei. Acesta s-a sculat de pe tronul său, și-a lepădat veșmântul lui cel scump, s-a acoperit cu sac și s-a culcat în cenușă.
\par 7 Apoi, din porunca regelui și a dregătorilor săi, s-au strigat și s-au zis acestea: Oamenii și animalele, vitele mari și mici să nu mănânce nimic, să nu pască și nici să bea apă;
\par 8 Iar oamenii să se îmbrace cu sac și către Dumnezeu să strige din toată puterea și fiecare să se întoarcă de pe calea lui cea rea și de la nedreptatea pe care o săvârșesc mâinile lui;
\par 9 Poate că Dumnezeu Se va întoarce și Se va milostivi și va ține în loc iuțimea mâniei Lui ca să nu pierim!"
\par 10 Atunci Dumnezeu a văzut faptele lor cele de pocăință, că s-au întors din căile lor cele rele. Și i-a părut rău Domnului de prezicerile de rău pe care li le făcuse și nu le-a împlinit.

\chapter{4}

\par 1 Și Iona a fost cuprins de mare supărare și s-a aprins de mânie.
\par 2 Și a rugat pe Domnul, zicând: "O, Doamne, iată tocmai ceea ce cugetam eu când eram în țara mea! Pentru aceasta eu am încercat să fug în Tarsis, că știam că Tu ești Dumnezeu îndurat și milostiv, îndelung-răbdător și mult-milosârd și Îți pare rău de fărădelegi.
\par 3 Și acum, Doamne, ia-mi sufletul meu, căci este mai bine să mor decât să fiu viu!"
\par 4 Și a zis Domnul: "Faci tu oare bine că ți-ai aprins mânia?"
\par 5 Atunci Iona a ieșit din cetate și s-a așezat la răsăritul ei, și-a făcut o colibă și a stat sub ea la umbră, ca să vadă ce se va întâmpla cu cetatea.
\par 6 Și Domnul Dumnezeu a făcut să crească un vrej care s-a ridicat deasupra capului lui Iona, ca să-i țină umbră și să-i mai potolească mânia. Și s-a bucurat Iona cu bucurie mare pentru vrej.
\par 7 Dar Dumnezeu, a doua zi, la revărsatul zorilor, a poruncit unui vierme să reteze vrejul. Iar el s-a uscat.
\par 8 Și la răsăritul soarelui a pornit Dumnezeu un vânt arzător de la răsărit și soarele a dogorit capul lui Iona, încât el se prăpădea de căldură. Și și-a rugat moartea zicând: "Mai bine este să mor decât să trăiesc!"
\par 9 Și a grăit Domnul către Iona: "Ai tu dreptate să te mânii pentru vrej?" Și el a răspuns: "Da, am dreptate să fiu supărat de moarte!"
\par 10 Și a zis Domnul: "Tu ți-ai făcut necaz pentru acest vrej pentru care nu te-ai trudit și nici nu l-ai crescut, care și-a luat ființă într-o noapte și într-alta a pierit!
\par 11 Dar Mie cum să nu-Mi fie milă de cetatea cea mare a Ninivei cu mai mult de o sută douăzeci de mii de oameni, care nu știu să deosebească dreapta de stânga lor, și cu un mare număr de dobitoace?"


\end{document}