\begin{document}

\title{Galateni}

\chapter{1}

\par 1 Pavel, apostol nu de la oameni, nici prin vreun om, ci prin Iisus Hristos și prin Dumnezeu-Tatăl, Care L-a înviat pe El din morți.
\par 2 Și toți frații care sunt împreună cu mine - Bisericilor Galatiei:
\par 3 Har vouă și pace de la Dumnezeu-Tatăl și de la Domnul nostru Iisus Hristos,
\par 4 Cel ce S-a dat pe Sine pentru păcatele noastre, ca să ne scoată pe noi din acest veac rău de acum, după voia lui Dumnezeu și a Tatălui nostru,
\par 5 Căruia fie slava în vecii vecilor. Amin!
\par 6 Mă mir că așa degrabă treceți de la cel ce v-a chemat pe voi, prin harul lui Hristos, la altă Evanghelie,
\par 7 Care nu este alta, decât că sunt unii care vă tulbură și voiesc să schimbe Evanghelia lui Hristos.
\par 8 Dar chiar dacă noi sau un înger din cer v-ar vesti altă Evanghelie decât aceea pe care v-am vestit-o - să fie anatema!
\par 9 Precum v-am spus mai înainte, și acum vă spun iarăși: Dacă vă propovăduiește cineva altceva decât ați primit - să fie anatema!
\par 10 Căci acum caut bunăvoința oamenilor sau pe a lui Dumnezeu? Sau caut să plac oamenilor? Dacă aș plăcea însă oamenilor, n-aș fi rob al lui Hristos.
\par 11 Dar vă fac cunoscut, fraților, că Evanghelia cea binevestită de mine nu este după om;
\par 12 Pentru că nici eu n-am primit-o de la om, nici n-am învățat-o, ci prin descoperirea lui Iisus Hristos.
\par 13 Căci ați auzit despre purtarea mea de altădată întru iudaism, că prigoneam peste măsură Biserica lui Dumnezeu și o pustiiam.
\par 14 Și spoream în iudaism mai mult decât mulți dintre cei care erau de vârsta mea în neamul meu, fiind mult râvnitor al datinilor mele părintești.
\par 15 Dar când a binevoit Dumnezeu Care m-a ales din pântecele mamei mele și m-a chemat prin harul Său,
\par 16 Să descopere pe Fiul Său întru mine, pentru ca să-L binevestesc la neamuri, îndată nu am primit sfat de la trup și de la sânge,
\par 17 Nici nu m-am suit la Ierusalim, la Apostolii cei dinainte de mine, ci m-am dus în Arabia și m-am întors iarăși la Damasc.
\par 18 Apoi, după trei ani, m-am suit la Ierusalim, ca să-l cunosc pe Chefa și am rămas la el cincisprezece zile.
\par 19 Iar pe altul din apostoli n-am văzut decât numai pe Iacov, fratele Domnului.
\par 20 Dar cele ce vă scriu, iată (spun) înaintea lui Dumnezeu, că nu vă mint.
\par 21 După aceea am venit în ținuturile Siriei și ale Ciliciei.
\par 22 Și după față eram necunoscut Bisericilor lui Hristos celor din Iudeea.
\par 23 Ci numai auziseră că cel ce ne prigonea pe noi, odinioară, acum binevestește credința pe care altădată o nimicea;
\par 24 Și slăveau pe Dumnezeu în mine.

\chapter{2}

\par 1 Apoi, după paisprezece ani, m-am suit iarăși la Ierusalim cu Barnaba, luând cu mine și pe Tit.
\par 2 M-am suit, potrivit unei descoperiri, și le-am arătat Evanghelia pe care o propovăduiesc la neamuri, îndeosebi celor mai de seamă, ca nu cumva să alerg sau să fi alergat în zadar.
\par 3 Dar nici Tit, care era cu mine și care era elin, n-a fost silit să se taie împrejur,
\par 4 Din cauza fraților mincinoși, care veniseră, furișându-se, să iscodească libertatea noastră, pe care o avem în Hristos Iisus, ca să ne robească,
\par 5 Cărora nici măcar un ceas nu ne-am plecat cu supunere, pentru ca adevărul Evangheliei să rămână neclintit la voi.
\par 6 Iar de cei ce sunt mai de seamă - oricine ar fi fost ei cândva, nu mă privește; Dumnezeu nu caută la fața omului, - cei mai de seamă n-au adăugat nimic la Evanghelia mea,
\par 7 Ci dimpotrivă, văzând că mie mi-a fost încredințată Evanghelia pentru cei netăiați împrejur, după cum lui Petru, Evanghelia pentru cei tăiați împrejur,
\par 8 Căci Cel ce a lucrat prin Petru în apostolia tăierii împrejur a lucrat și prin mine la neamuri;
\par 9 Și cunoscând harul ce mi-a fost dat mie, Iacov și Chefa și Ioan, cei socotiți a fi stâlpi, mi-au dat mie și lui Barnaba dreapta spre unire cu ei, pentru ca noi să binevestim la neamuri, iar ei la cei tăiați împrejur,
\par 10 Numai să ne aducem aminte de săraci, ceea ce tocmai m-am și silit să fac.
\par 11 Iar când Chefa a venit în Antiohia, pe față i-am stat împotrivă, căci era vrednic de înfruntare.
\par 12 Căci înainte de a veni unii de la Iacov, el mânca cu cei dintre neamuri; dar când au venit ei, se ferea și se osebea, temându-se de cei din tăierea împrejur.
\par 13 Și, împreună cu el, s-au fățărnicit și ceilalți iudei, încât și Barnaba a fost atras în fățărnicia lor.
\par 14 Dar când am văzut că ei nu calcă drept, după adevărul Evangheliei, am zis lui Chefa, înaintea tuturor: Dacă tu, care ești iudeu, trăiești ca păgânii și nu ca iudeii, de ce silești pe păgâni să trăiască ca iudeii?
\par 15 Noi suntem din fire iudei, iar nu păcătoși dintre neamuri.
\par 16 Știind însă că omul nu se îndreptează din faptele Legii, ci prin credința în Hristos Iisus, am crezut și noi în Hristos Iisus, ca să ne îndreptă din credința în Hristos, iar nu din faptele Legii, căci din faptele Legii, nimeni nu se va îndrepta.
\par 17 Dacă însă, căutând să ne îndreptăm în Hristos, ne-am aflat și noi înșine păcătoși, este, oare, Hristos slujitor al păcatului? Nicidecum!
\par 18 Căci dacă zidesc iarăși ceea ce am dărâmat, mă arăt pe mine însumi călcător (de poruncă).
\par 19 Căci, eu, prin Lege, am murit față de Lege, ca să trăiesc lui Dumnezeu.
\par 20 M-am răstignit împreună cu Hristos; și nu eu mai trăiesc, ci Hristos trăiește în mine. Și viața de acum, în trup, o trăiesc în credința în Fiul lui Dumnezeu, Care m-a iubit și S-a dat pe Sine însuși pentru mine.
\par 21 Nu lepăd harul lui Dumnezeu; căci dacă dreptatea vine prin Lege, atunci Hristos a murit în zadar.

\chapter{3}

\par 1 O, galateni fără de minte, cine v-a ademenit pe voi, să nu vă încredeți adevărului, - pe voi, în ochii cărora a fost zugrăvit Iisus Hristos răstignit?
\par 2 Numai aceasta voiesc să aflu de la voi: Din faptele Legii primit-ați voi Duhul, sau din ascultarea credinței?
\par 3 Atât de fără de minte sunteți? După ce ați început în Duh, sfârșiți acum în trup?
\par 4 Ați pătimit atâtea în zadar? - dacă a fost în zadar, cu adevărat.
\par 5 Deci Cel care vă dă vouă Duhul și săvârșește minuni la voi, le face, oare, din faptele Legii, sau din ascultarea credinței?
\par 6 Precum și Avraam a crezut în Dumnezeu și i s-a socotit lui ca dreptate.
\par 7 Să știți, deci, că cei ce sunt din credință, aceștia sunt fii ai lui Avraam.
\par 8 Iar Scriptura, văzând dinainte că Dumnezeu îndreptează neamurile din credință, dinainte a binevestit lui Avraam: "Că se vor binecuvânta în tine toate neamurile".
\par 9 Deci cei ce sunt din credință se binecuvintează împreună cu credinciosul Avraam.
\par 10 Căci toți câți sunt din faptele Legii sub blestem sunt, că scris este: "Blestemat este oricine nu stăruie întru toate cele scrise în cartea Legii, ca să le facă".
\par 11 Iar acum că, prin Lege, nu se îndreptează nimeni înaintea lui Dumnezeu este lucru lămurit, deoarece "dreptul din credință va fi viu".
\par 12 Legea însă nu este din credință, dar cel care va face acestea, va fi viu prin ele.
\par 13 Hristos ne-a răscumpărat din blestemul Legii, făcându-Se pentru noi blestem; pentru că scris este: "Blestemat este tot cel spânzurat pe lemn".
\par 14 Ca, prin Hristos Iisus, să vină la neamuri binecuvântarea lui Avraam, ca să primim, prin credință, făgăduința Duhului.
\par 15 Fraților, ca un om grăiesc; că și testamentul întărit al unui om nimeni nu-l strică, sau îi mai adaugă ceva.
\par 16 Făgăduințele au fost rostite lui Avraam și urmașului său. Nu zice: "și urmașilor", - ca de mai mulți, - ci ca de unul singur: "și Urmașului tău", Care este Hristos.
\par 17 Aceasta zic dar: Un testament întărit dinainte de Dumnezeu în Hristos nu desființează Legea, care a venit după patru sute treizeci de ani, ca să desființeze făgăduința.
\par 18 Căci dacă moștenirea este din Lege, nu mai este din făgăduință, dar Dumnezeu i-a dăruit lui Avraam moștenirea prin făgăduință.
\par 19 Deci ce este Legea? Ea a fost adăugată pentru călcările de lege, până când era să vină Urmașul, Căruia I s-a dat făgăduința, și a fost rânduită prin îngeri, în mâna unui Mijlocitor.
\par 20 Mijlocitorul însă nu este al unuia singur, iar Dumnezeu este unul.
\par 21 Este deci Legea împotriva făgăduințelor lui Dumnezeu? Nicidecum! Căci dacă s-ar fi dat Lege, care să poată da viață, cu adevărat dreptatea ar veni din Lege.
\par 22 Dar Scriptura a închis toate sub păcat, pentru ca făgăduința să se dea din credința în Iisus Hristos celor ce cred.
\par 23 Iar înainte de venirea credinței, noi eram păziți sub Lege, fiind închiși pentru credința care avea să se descopere.
\par 24 Astfel că Legea ne-a fost călăuză spre Hristos, pentru ca să ne îndreptăm din credință.
\par 25 Iar dacă a venit credința, nu mai suntem sub călăuză.
\par 26 Căci toți sunteți fii ai lui Dumnezeu prin credința în Hristos Iisus.
\par 27 Căci, câți în Hristos v-ați botezat, în Hristos v-ați îmbrăcat.
\par 28 Nu mai este iudeu, nici elin; nu mai este nici rob, nici liber; nu mai este parte bărbătească și parte femeiască, pentru că voi toți una sunteți în Hristos Iisus.
\par 29 Iar dacă voi sunteți ai lui Hristos, sunteți deci urmașii lui Avraam, moștenitori după făgăduință.

\chapter{4}

\par 1 Zic însă: Câtă vreme moștenitorul este copil, nu se deosebește cu nimic de rob, deși este stăpân peste toate;
\par 2 Ci este sub epitropi și iconomi, până la vremea rânduită de tatăl său.
\par 3 Tot așa și noi, când eram copii, eram robi înțelesurilor celor slabe ale lumii;
\par 4 Iar când a venit plinirea vremii, Dumnezeu, a trimis pe Fiul Său, născut din femeie, născut sub Lege,
\par 5 Ca pe cei de sub Lege să-i răscumpere, ca să dobândim înfierea.
\par 6 Și pentru că sunteți fii, a trimis Dumnezeu pe Duhul Fiului Său în inimile noastre, care strigă: Avva, Părinte!
\par 7 Astfel dar, nu mai ești rob, ci fiu; iar de ești fiu, ești și moștenitor al lui Dumnezeu, prin Iisus Hristos.
\par 8 Dar atunci necunoscând pe Dumnezeu, slujeați celor ce din fire nu sunt dumnezei;
\par 9 Acum însă, după ce ați cunoscut pe Dumnezeu, sau mai degrabă după ce ați fost cunoscuți de Dumnezeu, cum vă întoarceți iarăși la înțelesurile cele slabe și sărace, cărora iarăși voiți să le slujiți ca înainte?
\par 10 Țineți zile și luni și timpuri și ani?
\par 11 Mă tem de voi, să nu mă fi ostenit la voi, în zadar.
\par 12 Fiți, vă rog, fraților, precum sunt eu, că și eu am fost precum sunteți voi. Nu mi-ați făcut nici un rău;
\par 13 Dar știți că din cauza unei slăbiciuni a trupului, am binevestit vouă mai întâi,
\par 14 Și voi nu ați disprețuit încercarea mea, ce era în trupul meu, nici nu v-ați scârbit, ci m-ați primit ca pe un înger al lui Dumnezeu, ca pe Hristos Iisus.
\par 15 Unde este deci fericirea voastră? Căci vă mărturisesc că, de ar fi fost cu putință, v-ați fi scos ochii voștri și mi i-ați fi dat mie.
\par 16 Am ajuns deci vrăjmașul vostru spunându-vă adevărul?
\par 17 Aceia vă râvnesc, dar nu cu gând bun; ci vor să vă despartă (de mine), ca să-i iubiți pe ei.
\par 18 Dar e bine să râvniți totdeauna binele, și nu numai atunci când eu sunt de față la voi.
\par 19 O, copiii mei, pentru care sufăr iarăși durerile nașterii, până ce Hristos va lua chip în voi!
\par 20 Aș vrea acum să mă găsesc la voi și glasul să mi-l schimb, căci nu știu ce să cred despre voi!
\par 21 Spuneți-mi voi, care vreți să fiți sub Lege, nu auziți Legea?
\par 22 Căci scris este că Avraam a avut doi fii: unul din femeia roabă și altul din femeia liberă.
\par 23 Dar cel din roabă s-a născut după trup, iar cel din cea liberă s-a născut prin făgăduință.
\par 24 Unele ca acestea au altă însemnare, căci acestea (femei) sunt două testamente: Unul de la Muntele Sinai, care naște spre robie și care este Agar;
\par 25 Căci Agar este Muntele Sinai, în Arabia, și răspunde Ierusalimului de acum, care zace în robie cu copiii lui;
\par 26 Iar cea liberă este Ierusalimul cel de sus, care este mama noastră.
\par 27 Căci scris este: "Veselește-te, tu, cea stearpă, care nu naști! Izbucnește de bucurie și strigă, tu care nu ai durerile nașterii, căci mulți sunt copiii celei părăsite, mai mulți decât ai celei care are bărbat".
\par 28 Iar noi, fraților, suntem după Isaac, fii ai făgăduinței.
\par 29 Ci precum atunci cel ce se născuse după trup prigonea pe cel ce se născuse după Duh, tot așa și acum.
\par 30 Dar ce zice Scriptura? "Izgonește pe roabă și fiul ei, căci nu va moșteni fiul roabei, împreună cu fiul celei libere".
\par 31 Deci, fraților, nu suntem copii ai roabei, ci copii ai celei libere.

\chapter{5}

\par 1 Stați deci tari în libertatea cu care Hristos ne-a făcut liberi și nu vă prindeți iarăși în jugul robiei.
\par 2 Iată eu, Pavel, vă spun vouă: Că de vă veți tăia împrejur, Hristos nu vă va folosi la nimic.
\par 3 Și mărturisesc, iarăși, oricărui om ce se taie împrejur, că el este dator să împlinească toată Legea.
\par 4 Cei ce voiți să vă îndreptați prin Lege v-ați îndepărtat de Hristos, ați căzut din har;
\par 5 Căci noi așteptăm în Duh nădejdea dreptății din credință.
\par 6 Căci în Hristos Iisus, nici tăierea împrejur nu poate ceva, nici netăierea împrejur, ci credința care este lucrătoare prin iubire.
\par 7 Voi alergați bine. Cine v-a oprit ca să nu vă supuneți adevărului?
\par 8 Înduplecarea aceasta nu este de la cel care vă cheamă.
\par 9 Puțin aluat dospește toată frământătura.
\par 10 Eu am încredere în voi, întru Domnul, că nimic altceva nu veți cugeta; dar cel ce vă tulbură pe voi își va purta osânda, oricine ar fi el.
\par 11 Dar eu, fraților, dacă propovăduiesc încă tăierea împrejur, pentru ce mai sunt prigonit? Deci, sminteala crucii a încetat!
\par 12 O, de s-ar tăia de tot cei ce vă răzvrătesc pe voi!
\par 13 Căci voi, fraților, ați fost chemați la libertate; numai să nu folosiți libertatea ca prilej de a sluji trupului, ci slujiți unul altuia prin iubire.
\par 14 Căci toată Legea se cuprinde într-un singur cuvânt, în acesta: Iubește pe aproapele tău ca pe tine însuți.
\par 15 Iar dacă vă mușcați unul pe altul și vă mâncați, vedeți să nu vă nimiciți voi între voi.
\par 16 Zic dar: În Duhul să umblați și să nu împliniți pofta trupului.
\par 17 Căci trupul poftește împotriva duhului, iar duhul împotriva trupului; căci acestea se împotrivesc unul altuia, ca să nu faceți cele ce ați voi.
\par 18 Iar de vă purtați în Duhul nu sunteți sub Lege.
\par 19 Iar faptele trupului sunt cunoscute, și ele sunt: adulter, desfrânare, necurăție, destrăbălare,
\par 20 Închinare la idoli, fermecătorie, vrajbe, certuri, zavistii, mânii, gâlcevi, dezbinări, eresuri,
\par 21 Pizmuiri, ucideri, beții, chefuri și cele asemenea acestora, pe care vi le spun dinainte, precum dinainte v-am și spus, că cei ce fac unele ca acestea nu vor moșteni împărăția lui Dumnezeu.
\par 22 Iar roada Duhului este dragostea, bucuria, pacea, îndelungă-răbdarea, bunătatea, facerea de bine, credința,
\par 23 Blândețea, înfrânarea, curăția; împotriva unora ca acestea nu este lege.
\par 24 Iar cei ce sunt ai lui Hristos Iisus și-au răstignit trupul împreună cu patimile și cu poftele.
\par 25 Dacă trăim în Duhul, în Duhul să și umblăm.
\par 26 Să nu fim iubitori de mărire deșartă, supărându-ne unii pe alții și pizmuindu-ne unii pe alții.

\chapter{6}

\par 1 Fraților, chiar de va cădea un om în vreo greșeală, voi cei duhovnicești îndreptați-l, pe unul ca acesta cu duhul blândeții, luând seama la tine însuți, ca să nu cazi și tu în ispită.
\par 2 Purtați-vă sarcinile unii altora și așa veți împlini legea lui Hristos.
\par 3 Căci de se socotește cineva că este ceva, deși nu este nimic, se înșeală pe sine însuși.
\par 4 Iar fapta lui însuși să și-o cerceteze fiecare și atunci va avea laudă, dar numai față de sine însuși și nu față de altul.
\par 5 Căci fiecare își va purta sarcina sa.
\par 6 Cel care primește cuvântul învățăturii să facă parte învățătorului său din toate bunurile.
\par 7 Nu vă amăgiți: Dumnezeu nu Se lasă batjocorit; căci ce va semăna omul, aceea va și secera.
\par 8 Cel ce seamănă în trupul său însuși, din trup va secera stricăciune; iar cel ce seamănă în Duhul, din Duh va secera viață veșnică.
\par 9 Să nu încetăm de a face binele, căci vom secera la timpul său, dacă nu ne vom lenevi.
\par 10 Deci, dar, până când avem vreme, să facem binele către toți, dar mai ales către cei de o credință cu noi.
\par 11 Vedeți cu ce fel de litere v-am scris eu, cu mâna mea.
\par 12 Câți vor să placă în trup, aceia vă silesc să vă tăiați împrejur, numai ca să nu fie prigoniți pentru crucea lui Hristos.
\par 13 Căci nici ei singuri, cei ce se taie împrejur, nu păzesc Legea, ci voiesc să vă tăiați voi împrejur, ca să se laude ei în trupul vostru.
\par 14 Iar mie, să nu-mi fie a mă lăuda, decât numai în crucea Domnului nostru Iisus Hristos, prin care lumea este răstignită pentru mine, și eu pentru lume!
\par 15 Că în Hristos Iisus nici tăierea împrejur nu este ceva, nici netăierea împrejur, ci făptura cea nouă.
\par 16 Și câți vor umbla după dreptarul acesta, - pace și milă asupra lor și asupra Israelului lui Dumnezeu!
\par 17 De acum înainte, nimeni să nu-mi mai facă supărare, căci eu port în trupul meu, semnele Domnului Iisus.
\par 18 Harul Domnului nostru Iisus Hristos să fie cu duhul vostru, fraților! Amin.


\end{document}