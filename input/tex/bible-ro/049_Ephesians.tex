\begin{document}

\title{Efeseni}


\chapter{1}

\par 1 Pavel, apostol al lui Iisus Hristos prin voința lui Dumnezeu, sfinților care sunt în Efes și credincioșilor întru Hristos Iisus:
\par 2 Har vouă și pace de la Dumnezeu, Tatăl nostru, și de la Domnul Iisus Hristos!
\par 3 Binecuvântat fie Dumnezeu și Tatăl Domnului nostru Iisus Hristos, Cel ce, întru Hristos, ne-a binecuvântat pe noi, în ceruri, cu toată binecuvântarea duhovnicească;
\par 4 Precum întru El ne-a și ales, înainte de întemeierea lumii, ca să fim sfinți și fără de prihană înaintea Lui,
\par 5 Mai înainte rânduindu-ne, în a Sa iubire, spre înfierea întru El, prin Iisus Hristos, după buna socotință a voii Sale,
\par 6 Spre lauda slavei harului Său, cu care ne-a dăruit pe noi prin Fiul Său cel iubit;
\par 7 Întru El avem răscumpărarea prin sângele Lui și iertarea păcatelor, după bogăția harului Lui,
\par 8 Pe care l-a făcut să prisosească în noi, în toată înțelepciunea și priceperea;
\par 9 Făcându-ne cunoscută taina voii Sale, după buna Lui socotință, astfel cum hotărâse în Sine mai înainte,
\par 10 Spre iconomia plinirii vremilor, ca toate să fie iarăși unite în Hristos, cele din ceruri și cele de pe pământ - toate întru El,
\par 11 Întru Care și moștenire am primit, rânduiți fiind mai înainte - după rânduiala Celui ce toate le lucrează, potrivit sfatului voii Sale, -
\par 12 Ca să fim spre lauda slavei Sale, noi cei ce mai înainte am nădăjduit întru Hristos.
\par 13 Întru Care și voi, auzind cuvântul adevărului, Evanghelia mântuirii voastre, crezând în El, ați fost pecetluiți cu Sfântul Duh al făgăduinței,
\par 14 Care este arvuna moștenirii noastre, spre răscumpărarea celor dobândiți de El și spre lauda slavei Sale.
\par 15 Drept aceea, și eu auzind de credința voastră în Domnul Iisus și de dragostea cea către toți sfinții,
\par 16 Nu încetez a mulțumi pentru voi, pomenindu-vă în rugăciunile mele,
\par 17 Ca Dumnezeul Domnului nostru Iisus Hristos, Tatăl slavei, să vă dea vouă duhul înțelepciunii și al descoperirii, spre deplina Lui cunoaștere,
\par 18 Și să vă lumineze ochii inimii, ca să pricepeți care este nădejdea la care v-a chemat, care este bogăția slavei moștenirii Lui, în cei sfinți,
\par 19 Și cât de covârșitoare este mărimea puterii Lui față de noi, după lucrarea puterii tăriei Lui, pentru noi cei ce credem.
\par 20 Pe aceasta, Dumnezeu a lucrat-o în Hristos, sculându-L din morți și așezându-L de-a dreapta Sa, în ceruri,
\par 21 Mai presus decât toată începătoria și stăpânia și puterea și domnia și decât tot numele ce se numește, nu numai în veacul acesta, ci și în cel viitor.
\par 22 Și toate le-a supus sub picioarele Lui și, mai presus de toate, L-a dat pe El cap Bisericii,
\par 23 Care este trupul Lui, plinirea Celui ce plinește toate întru toți.

\chapter{2}

\par 1 Iar pe voi v-a făcut vii, cei ce erați morți prin greșealele și prin păcatele voastre,
\par 2 În care ați umblat mai înainte, potrivit veacului lumii acesteia, potrivit stăpânitorului puterii văzduhului, a duhului care lucrează acum în fiii neascultării,
\par 3 Întru care și noi toți am petrecut odinioară, în poftele trupului nostru, făcând voile trupului și ale simțurilor și, din fire, eram fiii mâniei ca și ceilalți.
\par 4 Dar Dumnezeu, bogat fiind în milă, pentru multa Sa iubire cu care ne-a iubit,
\par 5 Pe noi cei ce eram morți prin greșealele noastre, ne-a făcut vii împreună cu Hristos - prin har sunteți mântuiți! -
\par 6 Și împreună cu El ne-a sculat și împreună ne-a așezat întru ceruri, în Hristos Iisus,
\par 7 Ca să arate în veacurile viitoare covârșitoarea bogăție a harului Său, prin bunătatea ce a avut către noi întru Hristos Iisus.
\par 8 Căci în har sunteți mântuiți, prin credință, și aceasta nu e de la voi: este darul lui Dumnezeu;
\par 9 Nu din fapte, ca să nu se laude nimeni.
\par 10 Pentru că a Lui făptură suntem, zidiți în Hristos Iisus spre fapte bune, pe care Dumnezeu le-a gătit mai înainte, ca să umblăm întru ele.
\par 11 De aceea, aduceți-vă aminte că, odinioară, voi, păgânii cu trupul, numiți netăiere-împrejur de către cei numiți tăiere-împrejur, făcută de mână în trup,
\par 12 Erați, în vremea aceea, în afară de Hristos, înstrăinați de cetățenia lui Israel, lipsiți de nădejde și fără de Dumnezeu, în lume.
\par 13 Acum însă, fiind în Hristos Iisus, voi care altădată erați departe, v-ați apropiat prin sângele lui Hristos,
\par 14 Căci El este pacea noastră, El care a făcut din cele două - una, surpând peretele din mijloc al despărțiturii,
\par 15 Desființând vrăjmășia în trupul Său, legea poruncilor și învățăturile ei, ca, întru Sine, pe cei doi să-i zidească într-un singur om nou și să întemeieze pacea,
\par 16 Și să-i împace cu Dumnezeu pe amândoi, uniți într-un trup, prin cruce, omorând prin ea vrăjmășia.
\par 17 Și, venind, a binevestit pace, vouă celor de departe și pace celor de aproape;
\par 18 Că prin El avem și unii și alții apropierea către Tatăl, într-un Duh.
\par 19 Deci, dar, nu mai sunteți străini și locuitori vremelnici, ci sunteți împreună cetățeni cu sfinții și casnici ai lui Dumnezeu,
\par 20 Zidiți fiind pe temelia apostolilor și a proorocilor, piatra cea din capul unghiului fiind însuși Iisus Hristos.
\par 21 Întru El, orice zidire bine alcătuită crește ca să ajungă un locaș sfânt în Domnul,
\par 22 În Care voi împreună sunteți zidiți, spre a fi locaș al lui Dumnezeu în Duh.

\chapter{3}

\par 1 Pentru aceasta, eu Pavel, întemnițatul lui Iisus Hristos pentru voi, neamurile,
\par 2 Dacă în adevăr ați auzit de iconomia harului lui Dumnezeu care mi-a fost dat mie pentru voi,
\par 3 Că prin descoperire mi s-a dat în cunoștință această taină, precum v-am scris înainte pe scurt.
\par 4 De unde, citind, puteți să cunoașteți înțelegerea mea în taina lui Hristos,
\par 5 Care, în alte veacuri, nu s-a făcut cunoscută fiilor oamenilor, cum s-a descoperit acum sfinților Săi apostoli și prooroci, prin Duhul:
\par 6 Anume că neamurile sunt împreună moștenitoare (cu iudeii) și mădulare ale aceluiași trup și împreună-părtași ai făgăduinței, în Hristos Iisus, prin Evanghelie,
\par 7 Al cărei slujitor m-am făcut după darul harului lui Dumnezeu, ce mi-a fost dat mie, prin lucrarea puterii Sale;
\par 8 Mie, celui mai mic decât toți sfinții, mi-a fost dat harul acesta, ca să binevestesc neamurilor bogăția lui Hristos, de nepătruns,
\par 9 Și să descopăr tuturor care este iconomia tainei celei din veci ascunse în Dumnezeu, Ziditorul a toate, prin Iisus Hristos,
\par 10 Pentru ca înțelepciunea lui Dumnezeu cea de multe feluri să se facă cunoscută acum, prin Biserică, începătoriilor și stăpâniilor, în ceruri,
\par 11 După sfatul cel din veci, pe care El l-a împlinit în Hristos Iisus, Domnul nostru,
\par 12 Întru Care avem, prin credința în El, îndrăzneală și apropiere de Dumnezeu, cu deplină încredere.
\par 13 De aceea, vă rog să nu vă pierdeți cumpătul, din pricina necazurilor mele pentru voi; ele sunt slava voastră.
\par 14 Pentru aceasta, îmi plec genunchii înaintea Tatălui Domnului nostru Iisus Hristos,
\par 15 Din Care își trage numele orice neam în cer și pe pământ,
\par 16 Să vă dăruiască, după bogăția slavei Sale, ca să fiți puternic întăriți, prin Duhul Său, în omul dinăuntru,
\par 17 Și Hristos să Se sălășluiască, prin credință, în inimile voastre, înrădăcinați și întemeiați fiind în iubire,
\par 18 Ca să puteți înțelege împreună cu toți sfinții care este lărgimea și lungimea și înălțimea și adâncimea,
\par 19 Și să cunoașteți iubirea lui Hristos, cea mai presus de cunoștință, ca să vă umpleți de toată plinătatea lui Dumnezeu.
\par 20 Iar Celui ce poate să facă, prin puterea cea lucrătoare în noi, cu mult mai presus decât toate câte cerem sau pricepem noi,
\par 21 Lui fie slava în Biserică și întru Hristos Iisus în toate neamurile veacului veacurilor. Amin!

\chapter{4}

\par 1 De aceea, vă îndemn, eu cel întemnițat pentru Domnul, să umblați cu vrednicie, după chemarea cu care ați fost chemați,
\par 2 Cu toată smerenia și blândețea, cu îndelungă-răbdare, îngăduindu-vă unii pe alții în iubire,
\par 3 Silindu-vă să păziți unitatea Duhului, întru legătura păcii.
\par 4 Este un trup și un Duh, precum și chemați ați fost la o singură nădejde a chemării voastre;
\par 5 Este un Domn, o credință, un botez,
\par 6 Un Dumnezeu și Tatăl tuturor, Care este peste toate și prin toate și întru toți.
\par 7 Iar fiecăruia dintre noi, i s-a dat harul după măsura darului lui Hristos.
\par 8 Pentru aceea zice: "Suindu-Se la înălțime, a robit robime și a dat daruri oamenilor".
\par 9 Iar aceea că "S-a suit" - ce înseamnă decât că S-a pogorât în părțile cele mai de jos ale pământului?
\par 10 Cel ce S-a pogorât, Acela este Care S-a suit mai presus de toate cerurile, ca pe toate să le umple.
\par 11 Și el a dat pe unii apostoli, pe alții prooroci, pe alții evangheliști, pe alții păstori și învățători,
\par 12 Spre desăvârșirea sfinților, la lucrul slujirii, la zidirea trupului lui Hristos,
\par 13 Până vom ajunge toți la unitatea credinței și a cunoașterii Fiului lui Dumnezeu, la starea bărbatului desăvârșit, la măsura vârstei deplinătății lui Hristos.
\par 14 Ca să nu mai fim copii duși de valuri, purtați încoace și încolo de orice vânt al învățăturii, prin înșelăciunea oamenilor, prin vicleșugul lor, spre uneltirea rătăcirii,
\par 15 Ci ținând adevărul, în iubire, să creștem întru toate pentru El, Care este capul - Hristos.
\par 16 Din El, tot trupul bine alcătuit și bine încheiat, prin toate legăturile care îi dau tărie, își săvârșește creșterea, potrivit lucrării măsurate fiecăruia din mădulare, și se zidește întru dragoste.
\par 17 Așadar, aceasta zic și mărturisesc în Domnul, ca voi să nu mai umblați de acum cum umblă neamurile, în deșertăciunea minții lor,
\par 18 Întunecați fiind la cuget, înstrăinați fiind de viața lui Dumnezeu, din pricina necunoștinței care este în ei, din pricina împietririi inimii lor;
\par 19 Aceștia petrec în nesimțire și s-au dat pe sine desfrânării, săvârșind cu nesaț toate faptele necurăției.
\par 20 Voi însă n-ați învățat așa pe Hristos,
\par 21 Dacă, într-adevăr, L-ați ascultat și ați fost învățați întru El, așa cum este adevărul întru Iisus;
\par 22 Să vă dezbrăcați de viețuirea voastră de mai înainte, de omul cel vechi, care se strică prin poftele amăgitoare,
\par 23 Și să vă înnoiți în duhul minții voastre,
\par 24 Și să vă îmbrăcați în omul cel nou, cel după Dumnezeu, zidit întru dreptate și în sfințenia adevărului.
\par 25 Pentru aceea, lepădând minciuna, grăiți adevărul fiecare cu aproapele său, căci unul altuia suntem mădulare.
\par 26 Mâniați-vă și nu greșiți; soarele să nu apună peste mânia voastră.
\par 27 Nici nu dați loc diavolului.
\par 28 Cel ce fură să nu mai fure, ci mai vârtos să se ostenească lucrând cu mâinile sale, lucrul cel bun, ca să aibă să dea și celui ce are nevoie.
\par 29 Din gura voastră să nu iasă nici un cuvânt rău, ci numai ce este bun, spre zidirea cea de trebuință, ca să dea har celor ce ascultă.
\par 30 Să nu întristați Duhul cel Sfânt al lui Dumnezeu, întru Care ați fost pecetluiți pentru ziua răscumpărării.
\par 31 Orice amărăciune și supărare și mânie și izbucnire și defăimare să piară de la voi, împreună cu orice răutate.
\par 32 Ci fiți buni între voi și milostivi, iertând unul altuia, precum și Dumnezeu v-a iertat vouă, în Hristos.

\chapter{5}

\par 1 Fiți dar următori ai lui Dumnezeu, ca niște fii iubiți,
\par 2 Și umblați întru iubire, precum și Hristos ne-a iubit pe noi și S-a dat pe Sine pentru noi, prinos și jertfă lui Dumnezeu, întru miros cu bună mireasmă.
\par 3 Iar desfrâu și orice necurăție și lăcomie de avere nici să se pomenească între voi, cum se cuvine sfinților;
\par 4 Nici vorbe de rușine, nici vorbe nebunești, nici glume care nu se cuvin, ci mai degrabă mulțumire.
\par 5 Căci aceasta s-o știți bine, că nici un desfrânat, sau necurat, sau lacom de avere, care este un închinător la idoli, nu are moștenire în împărăția lui Hristos și a lui Dumnezeu.
\par 6 Nimeni să nu vă amăgească cu cuvinte deșarte, căci pentru acestea vine mânia lui Dumnezeu peste fiii neascultării.
\par 7 Deci să nu vă faceți părtași cu ei.
\par 8 Altădată erați întuneric, iar acum sunteți lumină întru Domnul; umblați ca fii ai luminii!
\par 9 Pentru că roada luminii e în orice bunătate, dreptate și adevăr.
\par 10 Încercând ce este bineplăcut Domnului.
\par 11 Și nu fiți părtași la faptele cele fără roadă ale întunericului, ci mai degrabă, osândiți-le pe față.
\par 12 Căci cele ce se fac întru ascuns de ei, rușine este a le și grăi.
\par 13 Iar tot ce este pe față, se descoperă prin lumină,
\par 14 Căci tot ceea ce este descoperit, lumină este. Pentru aceea zice: "Deșteaptă-te cel ce dormi și te scoală din morți și te va lumina Hristos".
\par 15 Deci luați seama cu grijă, cum umblați, nu ca niște neînțelepți, ci ca cei înțelepți,
\par 16 Răscumpărând vremea, căci zilele rele sunt.
\par 17 Drept aceea, nu fiți fără de minte, ci înțelegeți care este voia Domnului.
\par 18 Și nu vă îmbătați de vin, în care este pierzare, ci vă umpleți de Duhul.
\par 19 Vorbiți între voi în psalmi și în laude și în cântări duhovnicești, lăudând și cântând Domnului, în inimile voastre,
\par 20 Mulțumind totdeauna pentru toate întru numele Domnului nostru Iisus Hristos, lui Dumnezeu (și) Tatăl.
\par 21 Supuneți-vă unul altuia, întru frica lui Hristos.
\par 22 Femeile să se supună bărbaților lor ca Domnului,
\par 23 Pentru că bărbatul este cap femeii, precum și Hristos este cap Bisericii, trupul Său, al cărui mântuitor și este.
\par 24 Ci precum Biserica se supune lui Hristos, așa și femeile bărbaților lor, întru totul.
\par 25 Bărbaților, iubiți pe femeile voastre, după cum și Hristos a iubit Biserica, și S-a dat pe Sine pentru ea,
\par 26 Ca s-o sfințească, curățind-o cu baia apei prin cuvânt,
\par 27 Și ca s-o înfățișeze Sieși, Biserică slăvită, neavând pată sau zbârcitură, ori altceva de acest fel, ci ca să fie sfântă și fără de prihană.
\par 28 Așadar, bărbații sunt datori să-și iubească femeile ca pe înseși trupurile lor. Cel ce-și iubește femeia pe sine se iubește.
\par 29 Căci nimeni vreodată nu și-a urât trupul său, ci fiecare îl hrănește și îl încălzește, precum și Hristos Biserica,
\par 30 Pentru că suntem mădulare ale trupului Lui, din carnea Lui și din oasele Lui.
\par 31 De aceea, va lăsa omul pe tatăl său și pe mama sa și se va alipi de femeia sa și vor fi amândoi un trup.
\par 32 Taina aceasta mare este; iar eu zic în Hristos și în Biserică.
\par 33 Astfel și voi, fiecare așa să-și iubească femeia ca pe sine însuși; iar femeia să se teamă de bărbat.

\chapter{6}

\par 1 Copii, ascultați pe părinții voștri în Domnul că aceasta este cu dreptate.
\par 2 "Cinstește pe tatăl tău și pe mama ta, care este porunca cea dintâi cu făgăduința:
\par 3 Ca să-ți fie ție bine și să trăiești ani mulți pe pământ".
\par 4 Și voi, părinților, nu întărâtați la mânie pe copiii voștri, ci creșteți-i întru învățătura și certarea Domnului.
\par 5 Slugilor, ascultați de stăpânii voștri cei după trup, cu frică și cu cutremur, întru curăția inimii voastre, ca și de Hristos,
\par 6 Nu slujind numai când sunt cu ochii pe voi, ca cei ce caută să placă oamenilor, ci ca slugile lui Hristos, făcând din suflet voia lui Dumnezeu,
\par 7 Slujind cu bunăvoință, ca și Domnului și nu ca oamenilor,
\par 8 Știind fiecare, fie rob, fie de sine stăpân, că faptele bune pe care le va face, pe acelea le va lua ca plată de la Domnul.
\par 9 Iar voi, stăpânilor, faceți tot așa față de ei, lăsând la o parte amenințarea, știind că Domnul lor și al vostru este în ceruri și că la El nu încape părtinire.
\par 10 În sfârșit, fraților, întăriți-vă în Domnul și întru puterea tăriei Lui.
\par 11 Îmbrăcați-vă cu toate armele lui Dumnezeu, ca să puteți sta împotriva uneltirilor diavolului.
\par 12 Căci lupta noastră nu este împotriva trupului și a sângelui, ci împotriva începătoriilor, împotriva stăpâniilor, împotriva stăpânitorilor întunericului acestui veac, împotriva duhurilor răutății, care sunt în văzduh.
\par 13 Pentru aceea, luați toate armele lui Dumnezeu, ca să puteți împotrivă în ziua cea rea, și, toate biruindu-le, să rămâneți în picioare.
\par 14 Stați deci tari, având mijlocul vostru încins cu adevărul și îmbrăcându-vă cu platoșa dreptății,
\par 15 Și încălțați picioarele voastre, gata fiind pentru Evanghelia păcii.
\par 16 În toate luați pavăza credinței, cu care veți putea să stingeți toate săgețile cele arzătoare ale vicleanului.
\par 17 Luați și coiful mântuirii și sabia Duhului, care este cuvântul lui Dumnezeu.
\par 18 Faceți în toată vremea, în Duhul, tot felul de rugăciuni și de cereri, și întru aceasta priveghind cu toată stăruința și rugăciunea pentru toți sfinții.
\par 19 Rugați-vă și pentru mine, ca să mi se dea mie cuvânt, când voi deschide gura mea, să fac cunoscută cu îndrăzneală taina Evangheliei,
\par 20 Pe care o binevestesc, în lanțuri, ca să vorbesc despre Evanghelie, fără sfială, precum și trebuie să vorbesc.
\par 21 Iar ca să știți și voi cele despre mine și ce fac, Tihic, iubitul frate și credinciosul slujitor întru Domnul, vi le va aduce la cunoștință pe toate;
\par 22 L-am trimis pe el la voi, pentru aceasta, ca să aflați cele despre noi și să mângâie inimile voastre.
\par 23 Pace fraților și dragoste, cu credință de la Dumnezeu-Tatăl și de la Domnul Iisus Hristos.
\par 24 Harul fie cu toți care iubesc pe Domnul nostru Iisus Hristos întru curăție.


\end{document}