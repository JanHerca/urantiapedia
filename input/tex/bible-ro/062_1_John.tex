\begin{document}

\title{1 Ioan}


\chapter{1}

\par 1 Ce era de la început, ce am auzit, ce am vazut cu ochii no?tri, ce am privit ?i mâinile noastre au pipait despre Cuvântul vie?ii,
\par 2 - ?i Via?a s-a aratat ?i am vazut-o ?i marturisim ?i va vestim Via?a de veci, care era la Tatal ?i s-a aratat noua -
\par 3 Ce am vazut ?i am auzit, va vestim ?i voua, ca ?i voi sa ave?i împarta?ire cu noi. Iar împarta?irea noastra este cu Tatal ?i cu Fiul Sau, Iisus Hristos.
\par 4 ?i acestea noi vi le scriem, ca bucuria noastra sa fie deplina.
\par 5 ?i aceasta este vestirea pe care am auzit-o de la El ?i v-o vestim: ca Dumnezeu este lumina ?i nici un întuneric nu este întru El.
\par 6 Daca zicem ca avem împarta?ire cu El ?i umblam în întuneric, min?im ?i nu savâr?im adevarul.
\par 7 Iar daca umblam întru lumina, precum El este în lumina, atunci avem împarta?ire unul cu altul ?i sângele lui Iisus, Fiul Lui, ne cura?e?te pe noi de orice pacat.
\par 8 Daca zicem ca pacat nu avem, ne amagim pe noi în?ine ?i adevarul nu este întru noi.
\par 9 Daca marturisim pacatele noastre, El este credincios ?i drept, ca sa ne ierte pacatele ?i sa ne cura?easca pe noi de toata nedreptatea.
\par 10 Daca zicem ca n-am pacatuit, Îl facem mincinos ?i cuvântul Lui nu este întru noi.

\chapter{2}

\par 1 Copiii mei, acestea vi le scriu, ca sa nu pacatui?i, ?i daca va pacatui cineva, avem mijlocitor catre Tatal, pe Iisus Hristos cel drept.
\par 2 El este jertfa de ispa?ire pentru pacatele noastre, dar nu numai pentru pacatele noastre, ci ?i pentru ale lumii întregi.
\par 3 ?i întru aceasta ?tim ca L-am cunoscut, daca pazim poruncile Lui.
\par 4 Cel ce zice: L-am cunoscut, dar poruncile Lui nu le paze?te, mincinos este ?i întru el adevarul nu se afla.
\par 5 Iar cine paze?te cuvântul Lui, întru acela, cu adevarat, dragostea lui Dumnezeu este desavâr?ita. Prin aceasta, cunoa?tem ca suntem întru El.
\par 6 Cine zice ca petrece întru El dator este, precum Acela a umblat, ?i el a?a sa umble.
\par 7 Iubi?ilor, nu va scriu porunca noua, ci o porunca veche pe care o avea?i de la început; porunca cea veche este cuvântul pe care l-a?i auzit.
\par 8 Iara?i, va scriu porunca noua, ceea ce adevarat întru El ?i întru voi, pentru ca întunericul se duce ?i lumina cea adevarata începe sa rasara.
\par 9 Cine zice ca este în lumina ?i pe fratele sau îl ura?te, acela este în întuneric pâna acum.
\par 10 Cine iube?te pe fratele sau ramâne în lumina ?i sminteala nu este în el.
\par 11 Iar cel ce ura?te pe fratele sau este în întuneric ?i umbla în întuneric ?i nu ?tie încotro se duce, pentru ca întunericul a orbit ochii lui.
\par 12 Va scriu voua, copiilor, fiindca iertate v-au fost pacatele pentru numele Lui.
\par 13 Va scriu voua, parin?ilor, pentru ca a?i cunoscut pe Cel ce este de la început. Va scriu voua, tinerilor, fiindca a?i biruit pe cel viclean. V-am scris, copiilor, pentru ca a?i cunoscut pe Tatal.
\par 14 V-am scris, parin?ilor, fiindca a?i cunoscut pe Cel ce este de la început. Scris-am voua, tinerilor, caci sunte?i tari ?i cuvântul lui Dumnezeu ramâne în voi ?i a?i biruit pe cel viclean.
\par 15 Nu iubi?i lumea, nici cele ce sunt în lume. Daca cineva iube?te lumea, iubirea Tatalui nu este întru el;
\par 16 Pentru ca tot ce este în lume, adica pofta trupului ?i pofta ochilor ?i trufia vie?ii, nu sunt de la Tatal, ci sunt din lume.
\par 17 ?i lumea trece ?i pofta ei, dar cel ce face voia lui Dumnezeu ramâne în veac.
\par 18 Copii, este ceasul de pe urma, ?i precum a?i auzit ca vine antihrist, iar acum mul?i antihri?ti s-au aratat; de aici cunoa?tem noi ca este ceasul de pe urma.
\par 19 Dintre noi au ie?it, dar nu erau de-ai no?tri, caci de-ar fi fost de-ai no?tri, ar fi ramas cu noi; ci ca sa se arate ca nu sunt to?i de-ai no?tri, de aceea au ie?it.
\par 20 Iar voi, ungere ave?i de la Cel Sfânt ?i ?ti?i toate.
\par 21 V-am scris voua, nu pentru ca nu ?ti?i adevarul, ci pentru ca îl ?ti?i ?i ?ti?i ca nici o minciuna nu vine din adevar.
\par 22 Cine este mincinosul, daca nu cel ce tagaduie?te ca Iisus este Hristosul? Acesta este antihristul, cel care tagaduie?te pe Tatal ?i pe Fiul.
\par 23 Oricine tagaduie?te pe Fiul nu are nici pe Tatal; cine marturise?te pe Fiul are ?i pe Tatal.
\par 24 Deci, ceea ce a?i auzit de la început, în voi sa ramâna; de va ramâne în voi ceea ce a?i auzit de la început, ve?i ramâne ?i voi în Fiul ?i în Tatal.
\par 25 ?i aceasta este fagaduin?a pe care El ne-a fagaduit-o: Via?a ve?nica.
\par 26 Acestea v-am scris voua despre cei ce va amagesc.
\par 27 Cât despre voi, ungerea pe care a?i luat-o de la El ramâne întru voi ?i n-ave?i trebuin?a ca sa va înve?e cineva, ci precum ungerea Lui va înva?a despre toate, ?i adevarat este ?i nu este minciuna, ramâne?i întru El, a?a cum v-a înva?at.
\par 28 ?i acum, copii, ramâne?i întru El, ca sa avem îndrazneala când Se va arata ?i sa nu ne ru?inam de El, la venirea Lui.
\par 29 Daca ?ti?i ca este drept, cunoa?te?i ca oricine face dreptate este nascut din El.

\chapter{3}

\par 1 Vede?i ce fel de iubire ne-a daruit noua Tatal, ca sa ne numim fii ai lui Dumnezeu, ?i suntem. Pentru aceea lumea nu ne cunoa?te, fiindca nu L-a cunoscut nici pe El.
\par 2 Iubi?ilor, acum suntem fii ai lui Dumnezeu ?i ce vom fi nu s-a aratat pâna acum. ?tim ca daca El Se va arata, noi vom fi asemenea Lui, fiindca Îl vom vedea cum este.
\par 3 ?i oricine ?i-a pus în El nadejdea, acesta se cura?e?te pe sine, a?a cum Acela curat este.
\par 4 Oricine faptuie?te pacatul savâr?e?te ?i nelegiuirea, ?i pacatul este nelegiuirea.
\par 5 ?i voi ?ti?i ca El S-a aratat ca sa ridice pacatele ?i pacat întru El nu este.
\par 6 Oricine ramâne întru El nu pacatuie?te; oricine pacatuie?te nu L-a vazut nici nu L-a cunoscut.
\par 7 Copii, nimeni sa nu va amageasca. Cel ce savâr?e?te dreptatea este drept, precum Acela drept este.
\par 8 Cine savâr?e?te pacatul este de la diavolul, pentru ca de la început diavolul pacatuie?te. Pentru aceasta S-a aratat Fiul lui Dumnezeu, ca sa strice lucrurile diavolului.
\par 9 Oricine este nascut din Dumnezeu nu savâr?e?te pacat, pentru ca samân?a lui Dumnezeu ramâne în acesta; ?i nu poate sa pacatuiasca, fiindca este nascut din Dumnezeu.
\par 10 Prin aceasta cunoa?tem pe fiii lui Dumnezeu ?i pe fiii diavolului; oricine nu face dreptate nu este din Dumnezeu, nici cel ce nu iube?te pe fratele sau.
\par 11 Pentru ca aceasta este vestea pe care a?i auzit-o de la început, ca sa ne iubim unul pe altul,
\par 12 Nu precum Cain, care era de la cel viclean ?i a ucis pe fratele sau. ?i pentru care pricina l-a ucis? Fiindca faptele lui erau rele, iar ale fratelui sau erau drepte.
\par 13 Nu va mira?i, fra?ilor, daca lumea va ura?te.
\par 14 Noi ?tim ca am trecut din moarte la via?a, pentru ca iubim pe fra?i; cine nu iube?te pe fratele sau ramâne în moarte.
\par 15 Oricine ura?te pe fratele sau este uciga? de oameni ?i ?ti?i ca orice uciga? de oameni  nu are via?a ve?nica, dainuitoare în El.
\par 16 În aceasta am cunoscut iubirea: ca El ?i-a pus sufletul Sau pentru noi, ?i noi datori suntem sa ne punem sufletele pentru fra?i.
\par 17 Iar cine are boga?ia lumii acesteia ?i se uita la fratele sau care este în nevoie ?i î?i închide inima fa?a de el, cum ramâne în acela dragostea lui Dumnezeu?
\par 18 Fiii mei, sa nu iubim cu vorba, numai din gura, ci cu fapta ?i cu adevarul.
\par 19 În aceasta vom cunoa?te ca suntem din adevar ?i în fa?a lui Dumnezeu vom afla odihna inimii noastre,
\par 20 Fiindca, daca ne osânde?te inima noastra, Dumnezeu este mai mare decât inima noastra ?i ?tie toate.
\par 21 Iubi?ilor, daca inima noastra nu ne osânde?te, avem îndraznire catre Dumnezeu.
\par 22 ?i orice cerem, primim de la El, pentru ca pazim poruncile Lui ?i cele placute înaintea Lui facem.
\par 23 ?i aceasta este porunca Lui, ca sa credem întru numele lui Iisus Hristos, Fiul Sau, ?i sa ne iubim unul pe altul, precum ne-a dat porunca.
\par 24 Cel ce paze?te poruncile Lui ramâne în Dumnezeu ?i Dumnezeu în el; ?i prin aceasta cunoa?tem ca El ramâne în noi, din Duhul pe care ni L-a dat.

\chapter{4}

\par 1 Iubi?ilor, nu da?i crezare oricarui duh, ci cerca?i duhurile daca sunt de la Dumnezeu, fiindca mul?i prooroci mincino?i au ie?it în lume.
\par 2 În aceasta sa cunoa?te?i duhul lui Dumnezeu: orice duh care marturise?te ca Iisus Hristos a venit în trup, este de la Dumnezeu.
\par 3 ?i orice duh, care nu marturise?te pe Iisus Hristos, nu este de la Dumnezeu, ci este duhul lui antihrist, despre care a?i auzit ca vine ?i acum este chiar în lume.
\par 4 Voi, copii, sunte?i din Dumnezeu ?i i-a?i biruit pe acei prooroci, caci mai mare este Cel ce e în voi, decât cel ce este în lume.
\par 5 Aceia sunt din lume, de aceea graiesc ca din lume ?i lumea îi asculta.
\par 6 Noi suntem din Dumnezeu; cine cunoa?te pe Dumnezeu asculta de noi; cine nu este din Dumnezeu nu asculta de noi. Din aceasta cunoa?tem Duhul adevarului ?i duhul ratacirii.
\par 7 Iubi?ilor, sa ne iubim unul pe altul, pentru ca dragostea este de la Dumnezeu ?i oricine iube?te este nascut din Dumnezeu ?i cunoa?te pe Dumnezeu.
\par 8 Cel ce nu iube?te n-a cunoscut pe Dumnezeu, pentru ca Dumnezeu este iubire.
\par 9 Întru aceasta s-a aratat dragostea lui Dumnezeu catre noi, ca pe Fiul Sau cel Unul Nascut L-a trimis Dumnezeu în lume, ca prin El via?a sa avem.
\par 10 În aceasta este dragostea, nu fiindca noi am iubit pe Dumnezeu, ci fiindca El ne-a iubit pe noi ?i a trimis pe Fiul Sau jertfa de ispa?ire pentru pacatele noastre.
\par 11 Iubi?ilor, daca Dumnezeu astfel ne-a iubit pe noi, ?i noi datori suntem sa ne iubim unul pe altul.
\par 12 Pe Dumnezeu nimeni nu L-a vazut vreodata, dar de ne iubim unul pe altul, Dumnezeu ramâne întru noi ?i dragostea Lui în noi este desavâr?ita.
\par 13 Din aceasta cunoa?tem ca ramânem în El ?i El întru noi, fiindca ne-a dat din Duhul Sau.
\par 14 ?i noi am vazut ?i marturisim ca Tatal a trimis pe Fiul, Mântuitor al lumii.
\par 15 Cine marturise?te ca Iisus este fiul lui Dumnezeu, Dumnezeu ramâne întru el ?i el în Dumnezeu.
\par 16 ?i noi am cunoscut ?i am crezut iubirea, pe care Dumnezeu o are catre noi. Dumnezeu este iubire ?i cel ce ramâne în iubire ramâne în Dumnezeu ?i Dumnezeu ramâne întru el.
\par 17 Întru aceasta a fost desavâr?ita iubirea Lui fa?a de noi, ca sa avem îndraznire în ziua judeca?ii, fiindca precum este Acela, a?a suntem ?i noi, în lumea aceasta.
\par 18 În iubire nu este frica, ci iubirea desavâr?ita alunga frica, pentru ca frica are cu sine pedeapsa, iar cel ce se teme nu este desavâr?it în iubire.
\par 19 Noi iubim pe Dumnezeu, fiindca El ne-a iubit cel dintâi.
\par 20 Daca zice cineva: iubesc pe Dumnezeu, iar pe fratele sau îl ura?te, mincinos este! Pentru ca cel ce nu iube?te pe fratele sau, pe care l-a vazut, pe Dumnezeu, pe Care nu L-a vazut, nu poate sa-L iubeasca.
\par 21 ?i aceasta porunca avem de la El: cine iube?te pe Dumnezeu sa iubeasca ?i pe fratele sau.

\chapter{5}

\par 1 Oricine crede ca Iisus este Hristos, este nascut din Dumnezeu, ?i oricine iube?te pe Cel care a nascut iube?te ?i pe Cel ce S-a nascut din El.
\par 2 Întru aceasta cunoa?tem ca iubim pe fiii lui Dumnezeu, daca iubim pe Dumnezeu ?i împlinim poruncile Lui.
\par 3 Caci dragostea de Dumnezeu aceasta este: Sa pazim poruncile Lui; ?i poruncile Lui nu sunt grele.
\par 4 Pentru ca oricine este nascut din Dumnezeu biruie?te lumea, ?i aceasta este biruin?a care a biruit lumea: credin?a noastra.
\par 5 Cine este cel ce biruie?te lumea daca nu cel ce crede ca Iisus este Fiul lui Dumnezeu?
\par 6 Acesta este Cel care a venit prin apa ?i prin sânge: Iisus Hristos; nu numai prin apa, ci prin apa ?i prin sânge; ?i Duhul este Cel ce marturise?te, ca Duhul este adevarul.
\par 7 Caci trei sunt care marturisesc în cer: Tatal, Cuvântul ?i Sfântul Duh, ?i Ace?ti trei Una sunt.
\par 8 ?i trei sunt care marturisesc pe pamânt: Duhul ?i apa ?i sângele, ?i ace?ti trei marturisesc la fel.
\par 9 Daca primim marturia oamenilor, marturia lui Dumnezeu este mai mare, ca aceasta este marturia lui Dumnezeu: ca a marturisit pentru Fiul Sau.
\par 10 Cine crede în Fiul lui Dumnezeu are aceasta marturie în el însu?i. Cine nu crede în Dumnezeu, L-a facut mincinos, pentru ca n-a crezut în marturia pe care a marturisit-o Dumnezeu pentru Fiul Sau.
\par 11 ?i aceasta este marturia, ca Dumnezeu ne-a dat via?a ve?nica ?i aceasta via?a este în Fiul Sau.
\par 12 Cel ce are pe Fiul are via?a; cel ce nu are pe Fiul lui Dumnezeu nu are via?a.
\par 13 Acestea am scris voua, care crede?i în numele Fiului lui Dumnezeu, ca sa ?ti?i ca ave?i via?a ve?nica.
\par 14 ?i aceasta este încrederea pe care o avem catre El, ca, daca cerem ceva dupa voin?a Lui, El ne asculta.
\par 15 ?i daca ?tim ca El ne asculta ceea ce Îi cerem, ?tim ca dobândim cererile pe care I le-am cerut.
\par 16 Daca vede cineva pe fratele sau pacatuind - pacat nu de moarte - sa se roage, ?i Dumnezeu va da via?a acelui frate, anume celor ce nu pacatuiesc de moarte. Este ?i pacat de moarte; nu zic sa se roage pentru acela.
\par 17 Orice nedreptate este pacat, dar este ?i pacat care nu e de moarte.
\par 18 ?tim ca oricine e nascut din Dumnezeu nu pacatuie?te; ci cel ce s-a nascut din Dumnezeu se paze?te pe sine, ?i cel rau nu se atinge de el.
\par 19 ?tim ca suntem din Dumnezeu ?i lumea întreaga zace sub puterea celui rau.
\par 20 ?tim iara?i ca Fiul lui Dumnezeu a venit ?i ne-a dat noua pricepere, ca sa cunoa?tem pe Dumnezeul cel adevarat; ?i noi suntem în Dumnezeul cel adevarat, adica întru Fiul Sau Iisus Hristos. Acesta este adevaratul Dumnezeu ?i via?a de veci.
\par 21 Fiilor, pazi?i-va de idoli.


\end{document}