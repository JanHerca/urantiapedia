\begin{document}

\title{1 Ioan}


\chapter{1}

\par 1 Ce era de la început, ce am auzit, ce am văzut cu ochii noștri, ce am privit și mâinile noastre au pipăit despre Cuvântul vieții,
\par 2 - Și Viața s-a arătat și am văzut-o și mărturisim și vă vestim Viața de veci, care era la Tatăl și s-a arătat nouă -
\par 3 Ce am văzut și am auzit, vă vestim și vouă, ca și voi să aveți împărtășire cu noi. Iar împărtășirea noastră este cu Tatăl și cu Fiul Său, Iisus Hristos.
\par 4 Și acestea noi vi le scriem, ca bucuria noastră să fie deplină.
\par 5 Și aceasta este vestirea pe care am auzit-o de la El și v-o vestim: că Dumnezeu este lumină și nici un întuneric nu este întru El.
\par 6 Dacă zicem că avem împărtășire cu El și umblăm în întuneric, mințim și nu săvârșim adevărul.
\par 7 Iar dacă umblăm întru lumină, precum El este în lumină, atunci avem împărtășire unul cu altul și sângele lui Iisus, Fiul Lui, ne curățește pe noi de orice păcat.
\par 8 Dacă zicem că păcat nu avem, ne amăgim pe noi înșine și adevărul nu este întru noi.
\par 9 Dacă mărturisim păcatele noastre, El este credincios și drept, ca să ne ierte păcatele și să ne curățească pe noi de toată nedreptatea.
\par 10 Dacă zicem că n-am păcătuit, Îl facem mincinos și cuvântul Lui nu este întru noi.

\chapter{2}

\par 1 Copiii mei, acestea vi le scriu, ca să nu păcătuiți, și dacă va păcătui cineva, avem mijlocitor către Tatăl, pe Iisus Hristos cel drept.
\par 2 El este jertfa de ispășire pentru păcatele noastre, dar nu numai pentru păcatele noastre, ci și pentru ale lumii întregi.
\par 3 Și întru aceasta știm că L-am cunoscut, dacă păzim poruncile Lui.
\par 4 Cel ce zice: L-am cunoscut, dar poruncile Lui nu le păzește, mincinos este și întru el adevărul nu se află.
\par 5 Iar cine păzește cuvântul Lui, întru acela, cu adevărat, dragostea lui Dumnezeu este desăvârșită. Prin aceasta, cunoaștem că suntem întru El.
\par 6 Cine zice că petrece întru El dator este, precum Acela a umblat, și el așa să umble.
\par 7 Iubiților, nu vă scriu poruncă nouă, ci o poruncă veche pe care o aveați de la început; porunca cea veche este cuvântul pe care l-ați auzit.
\par 8 Iarăși, vă scriu poruncă nouă, ceea ce adevărat întru El și întru voi, pentru că întunericul se duce și lumina cea adevărată începe să răsară.
\par 9 Cine zice că este în lumină și pe fratele său îl urăște, acela este în întuneric până acum.
\par 10 Cine iubește pe fratele său rămâne în lumină și sminteală nu este în el.
\par 11 Iar cel ce urăște pe fratele său este în întuneric și umblă în întuneric și nu știe încotro se duce, pentru că întunericul a orbit ochii lui.
\par 12 Vă scriu vouă, copiilor, fiindcă iertate v-au fost păcatele pentru numele Lui.
\par 13 Vă scriu vouă, părinților, pentru că ați cunoscut pe Cel ce este de la început. Vă scriu vouă, tinerilor, fiindcă ați biruit pe cel viclean. V-am scris, copiilor, pentru că ați cunoscut pe Tatăl.
\par 14 V-am scris, părinților, fiindcă ați cunoscut pe Cel ce este de la început. Scris-am vouă, tinerilor, căci sunteți tari și cuvântul lui Dumnezeu rămâne în voi și ați biruit pe cel viclean.
\par 15 Nu iubiți lumea, nici cele ce sunt în lume. Dacă cineva iubește lumea, iubirea Tatălui nu este întru el;
\par 16 Pentru că tot ce este în lume, adică pofta trupului și pofta ochilor și trufia vieții, nu sunt de la Tatăl, ci sunt din lume.
\par 17 Și lumea trece și pofta ei, dar cel ce face voia lui Dumnezeu rămâne în veac.
\par 18 Copii, este ceasul de pe urmă, și precum ați auzit că vine antihrist, iar acum mulți antihriști s-au arătat; de aici cunoaștem noi că este ceasul de pe urmă.
\par 19 Dintre noi au ieșit, dar nu erau de-ai noștri, căci de-ar fi fost de-ai noștri, ar fi rămas cu noi; ci ca să se arate că nu sunt toți de-ai noștri, de aceea au ieșit.
\par 20 Iar voi, ungere aveți de la Cel Sfânt și știți toate.
\par 21 V-am scris vouă, nu pentru că nu știți adevărul, ci pentru că îl știți și știți că nici o minciună nu vine din adevăr.
\par 22 Cine este mincinosul, dacă nu cel ce tăgăduiește că Iisus este Hristosul? Acesta este antihristul, cel care tăgăduiește pe Tatăl și pe Fiul.
\par 23 Oricine tăgăduiește pe Fiul nu are nici pe Tatăl; cine mărturisește pe Fiul are și pe Tatăl.
\par 24 Deci, ceea ce ați auzit de la început, în voi să rămână; de va rămâne în voi ceea ce ați auzit de la început, veți rămâne și voi în Fiul și în Tatăl.
\par 25 Și aceasta este făgăduința pe care El ne-a făgăduit-o: Viața veșnică.
\par 26 Acestea v-am scris vouă despre cei ce vă amăgesc.
\par 27 Cât despre voi, ungerea pe care ați luat-o de la El rămâne întru voi și n-aveți trebuință ca să vă învețe cineva, ci precum ungerea Lui vă învață despre toate, și adevărat este și nu este minciună, rămâneți întru El, așa cum v-a învățat.
\par 28 Și acum, copii, rămâneți întru El, ca să avem îndrăzneală când Se va arăta și să nu ne rușinăm de El, la venirea Lui.
\par 29 Dacă știți că este drept, cunoașteți că oricine face dreptate este născut din El.

\chapter{3}

\par 1 Vedeți ce fel de iubire ne-a dăruit nouă Tatăl, ca să ne numim fii ai lui Dumnezeu, și suntem. Pentru aceea lumea nu ne cunoaște, fiindcă nu L-a cunoscut nici pe El.
\par 2 Iubiților, acum suntem fii ai lui Dumnezeu și ce vom fi nu s-a arătat până acum. Știm că dacă El Se va arăta, noi vom fi asemenea Lui, fiindcă Îl vom vedea cum este.
\par 3 Și oricine și-a pus în El nădejdea, acesta se curățește pe sine, așa cum Acela curat este.
\par 4 Oricine făptuiește păcatul săvârșește și nelegiuirea, și păcatul este nelegiuirea.
\par 5 Și voi știți că El S-a arătat ca să ridice păcatele și păcat întru El nu este.
\par 6 Oricine rămâne întru El nu păcătuiește; oricine păcătuiește nu L-a văzut nici nu L-a cunoscut.
\par 7 Copii, nimeni să nu vă amăgească. Cel ce săvârșește dreptatea este drept, precum Acela drept este.
\par 8 Cine săvârșește păcatul este de la diavolul, pentru că de la început diavolul păcătuiește. Pentru aceasta S-a arătat Fiul lui Dumnezeu, ca să strice lucrurile diavolului.
\par 9 Oricine este născut din Dumnezeu nu săvârșește păcat, pentru că sămânța lui Dumnezeu rămâne în acesta; și nu poate să păcătuiască, fiindcă este născut din Dumnezeu.
\par 10 Prin aceasta cunoaștem pe fiii lui Dumnezeu și pe fiii diavolului; oricine nu face dreptate nu este din Dumnezeu, nici cel ce nu iubește pe fratele său.
\par 11 Pentru că aceasta este vestea pe care ați auzit-o de la început, ca să ne iubim unul pe altul,
\par 12 Nu precum Cain, care era de la cel viclean și a ucis pe fratele său. Și pentru care pricină l-a ucis? Fiindcă faptele lui erau rele, iar ale fratelui său erau drepte.
\par 13 Nu vă mirați, fraților, dacă lumea vă urăște.
\par 14 Noi știm că am trecut din moarte la viață, pentru că iubim pe frați; cine nu iubește pe fratele său rămâne în moarte.
\par 15 Oricine urăște pe fratele său este ucigaș de oameni și știți că orice ucigaș de oameni nu are viață veșnică, dăinuitoare în El.
\par 16 În aceasta am cunoscut iubirea: că El Și-a pus sufletul Său pentru noi, și noi datori suntem să ne punem sufletele pentru frați.
\par 17 Iar cine are bogăția lumii acesteia și se uită la fratele său care este în nevoie și își închide inima față de el, cum rămâne în acela dragostea lui Dumnezeu?
\par 18 Fiii mei, să nu iubim cu vorba, numai din gură, ci cu fapta și cu adevărul.
\par 19 În aceasta vom cunoaște că suntem din adevăr și în fața lui Dumnezeu vom afla odihnă inimii noastre,
\par 20 Fiindcă, dacă ne osândește inima noastră, Dumnezeu este mai mare decât inima noastră și știe toate.
\par 21 Iubiților, dacă inima noastră nu ne osândește, avem îndrăznire către Dumnezeu.
\par 22 Și orice cerem, primim de la El, pentru că păzim poruncile Lui și cele plăcute înaintea Lui facem.
\par 23 Și aceasta este porunca Lui, ca să credem întru numele lui Iisus Hristos, Fiul Său, și să ne iubim unul pe altul, precum ne-a dat poruncă.
\par 24 Cel ce păzește poruncile Lui rămâne în Dumnezeu și Dumnezeu în el; și prin aceasta cunoaștem că El rămâne în noi, din Duhul pe care ni L-a dat.

\chapter{4}

\par 1 Iubiților, nu dați crezare oricărui duh, ci cercați duhurile dacă sunt de la Dumnezeu, fiindcă mulți prooroci mincinoși au ieșit în lume.
\par 2 În aceasta să cunoașteți duhul lui Dumnezeu: orice duh care mărturisește că Iisus Hristos a venit în trup, este de la Dumnezeu.
\par 3 Și orice duh, care nu mărturisește pe Iisus Hristos, nu este de la Dumnezeu, ci este duhul lui antihrist, despre care ați auzit că vine și acum este chiar în lume.
\par 4 Voi, copii, sunteți din Dumnezeu și i-ați biruit pe acei prooroci, căci mai mare este Cel ce e în voi, decât cel ce este în lume.
\par 5 Aceia sunt din lume, de aceea grăiesc ca din lume și lumea îi ascultă.
\par 6 Noi suntem din Dumnezeu; cine cunoaște pe Dumnezeu ascultă de noi; cine nu este din Dumnezeu nu ascultă de noi. Din aceasta cunoaștem Duhul adevărului și duhul rătăcirii.
\par 7 Iubiților, să ne iubim unul pe altul, pentru că dragostea este de la Dumnezeu și oricine iubește este născut din Dumnezeu și cunoaște pe Dumnezeu.
\par 8 Cel ce nu iubește n-a cunoscut pe Dumnezeu, pentru că Dumnezeu este iubire.
\par 9 Întru aceasta s-a arătat dragostea lui Dumnezeu către noi, că pe Fiul Său cel Unul Născut L-a trimis Dumnezeu în lume, ca prin El viață să avem.
\par 10 În aceasta este dragostea, nu fiindcă noi am iubit pe Dumnezeu, ci fiindcă El ne-a iubit pe noi și a trimis pe Fiul Său jertfă de ispășire pentru păcatele noastre.
\par 11 Iubiților, dacă Dumnezeu astfel ne-a iubit pe noi, și noi datori suntem să ne iubim unul pe altul.
\par 12 Pe Dumnezeu nimeni nu L-a văzut vreodată, dar de ne iubim unul pe altul, Dumnezeu rămâne întru noi și dragostea Lui în noi este desăvârșită.
\par 13 Din aceasta cunoaștem că rămânem în El și El întru noi, fiindcă ne-a dat din Duhul Său.
\par 14 Și noi am văzut și mărturisim că Tatăl a trimis pe Fiul, Mântuitor al lumii.
\par 15 Cine mărturisește că Iisus este fiul lui Dumnezeu, Dumnezeu rămâne întru el și el în Dumnezeu.
\par 16 Și noi am cunoscut și am crezut iubirea, pe care Dumnezeu o are către noi. Dumnezeu este iubire și cel ce rămâne în iubire rămâne în Dumnezeu și Dumnezeu rămâne întru el.
\par 17 Întru aceasta a fost desăvârșită iubirea Lui față de noi, ca să avem îndrăznire în ziua judecății, fiindcă precum este Acela, așa suntem și noi, în lumea aceasta.
\par 18 În iubire nu este frică, ci iubirea desăvârșită alungă frica, pentru că frica are cu sine pedeapsa, iar cel ce se teme nu este desăvârșit în iubire.
\par 19 Noi iubim pe Dumnezeu, fiindcă El ne-a iubit cel dintâi.
\par 20 Dacă zice cineva: iubesc pe Dumnezeu, iar pe fratele său îl urăște, mincinos este! Pentru că cel ce nu iubește pe fratele său, pe care l-a văzut, pe Dumnezeu, pe Care nu L-a văzut, nu poate să-L iubească.
\par 21 Și această poruncă avem de la El: cine iubește pe Dumnezeu să iubească și pe fratele său.

\chapter{5}

\par 1 Oricine crede că Iisus este Hristos, este născut din Dumnezeu, și oricine iubește pe Cel care a născut iubește și pe Cel ce S-a născut din El.
\par 2 Întru aceasta cunoaștem că iubim pe fiii lui Dumnezeu, dacă iubim pe Dumnezeu și împlinim poruncile Lui.
\par 3 Căci dragostea de Dumnezeu aceasta este: Să păzim poruncile Lui; și poruncile Lui nu sunt grele.
\par 4 Pentru că oricine este născut din Dumnezeu biruiește lumea, și aceasta este biruința care a biruit lumea: credința noastră.
\par 5 Cine este cel ce biruiește lumea dacă nu cel ce crede că Iisus este Fiul lui Dumnezeu?
\par 6 Acesta este Cel care a venit prin apă și prin sânge: Iisus Hristos; nu numai prin apă, ci prin apă și prin sânge; și Duhul este Cel ce mărturisește, că Duhul este adevărul.
\par 7 Căci trei sunt care mărturisesc în cer: Tatăl, Cuvântul și Sfântul Duh, și Acești trei Una sunt.
\par 8 Și trei sunt care mărturisesc pe pământ: Duhul și apa și sângele, și acești trei mărturisesc la fel.
\par 9 Dacă primim mărturia oamenilor, mărturia lui Dumnezeu este mai mare, că aceasta este mărturia lui Dumnezeu: că a mărturisit pentru Fiul Său.
\par 10 Cine crede în Fiul lui Dumnezeu are această mărturie în el însuși. Cine nu crede în Dumnezeu, L-a făcut mincinos, pentru că n-a crezut în mărturia pe care a mărturisit-o Dumnezeu pentru Fiul Său.
\par 11 Și aceasta este mărturia, că Dumnezeu ne-a dat viață veșnică și această viață este în Fiul Său.
\par 12 Cel ce are pe Fiul are viața; cel ce nu are pe Fiul lui Dumnezeu nu are viața.
\par 13 Acestea am scris vouă, care credeți în numele Fiului lui Dumnezeu, ca să știți că aveți viață veșnică.
\par 14 Și aceasta este încrederea pe care o avem către El, că, dacă cerem ceva după voința Lui, El ne ascultă.
\par 15 Și dacă știm că El ne ascultă ceea ce Îi cerem, știm că dobândim cererile pe care I le-am cerut.
\par 16 Dacă vede cineva pe fratele său păcătuind - păcat nu de moarte - să se roage, și Dumnezeu va da viață acelui frate, anume celor ce nu păcătuiesc de moarte. Este și păcat de moarte; nu zic să se roage pentru acela.
\par 17 Orice nedreptate este păcat, dar este și păcat care nu e de moarte.
\par 18 Știm că oricine e născut din Dumnezeu nu păcătuiește; ci cel ce s-a născut din Dumnezeu se păzește pe sine, și cel rău nu se atinge de el.
\par 19 Știm că suntem din Dumnezeu și lumea întreagă zace sub puterea celui rău.
\par 20 Știm iarăși că Fiul lui Dumnezeu a venit și ne-a dat nouă pricepere, ca să cunoaștem pe Dumnezeul cel adevărat; și noi suntem în Dumnezeul cel adevărat, adică întru Fiul Său Iisus Hristos. Acesta este adevăratul Dumnezeu și viața de veci.
\par 21 Fiilor, păziți-vă de idoli.


\end{document}