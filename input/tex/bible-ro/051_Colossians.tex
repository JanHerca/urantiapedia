\begin{document}

\title{Coloseni}


\chapter{1}

\par 1 Pavel, apostol al lui Hristos Iisus prin voința lui Dumnezeu, și Timotei, fratele nostru,
\par 2 Sfinților și credincioșilor frați întru Hristos, care sunt în Colose: Har vouă și pace de la Dumnezeu, Tatăl nostru, și de la Domnul nostru Iisus Hristos.
\par 3 Mulțumim lui Dumnezeu și Tatăl Domnului nostru Iisus Hristos, rugându-ne totdeauna pentru voi,
\par 4 Căci am auzit despre credința voastră în Hristos Iisus și despre dragostea ce aveți către toți sfinții,
\par 5 Pentru nădejdea cea gătită vouă în ceruri, de care ați auzit mai înainte în cuvântul adevărului Evangheliei,
\par 6 Care, ajungând la voi, precum și în toată lumea, aduce roadă și sporește întocmai ca la voi, din ziua în care ați auzit și ați cunoscut, întru adevăr, harul lui Dumnezeu.
\par 7 Precum ați primit învățătură de la Epafras, iubitul, împreună-slujitor cu noi, care este pentru voi credincios slujitor al lui Hristos,
\par 8 Și care ne-a arătat iubirea voastră cea întru Duhul.
\par 9 De aceea și noi, din ziua în care am auzit aceasta, nu încetăm să ne rugăm pentru voi și să cerem să vă umpleți de cunoașterea voii Lui, întru toată înțelepciunea și priceperea duhovnicească,
\par 10 Ca voi să umblați cu vrednicie întru Domnul, plăcuți Lui în toate, aducând roadă în orice lucru bun și sporind în cunoașterea lui Dumnezeu.
\par 11 Și întăriți fiind cu toată puterea, după puterea slavei Lui, spre toată stăruința și îndelunga-răbdare,
\par 12 Mulțumind cu bucurie Tatălui celui ce ne-a învrednicit pe noi să luăm parte la moștenirea sfinților, întru lumină.
\par 13 El ne-a scos de sub puterea întunericului și ne-a strămutat în împărăția Fiului iubirii Sale,
\par 14 Întru Care avem răscumpărarea prin sângele Lui, adică iertarea păcatelor;
\par 15 Acesta este chipul lui Dumnezeu celui nevăzut, mai întâi născut decât toată făptura.
\par 16 Pentru că întru El au fost făcute toate, cele din ceruri și cele de pe pământ, cele văzute, și cele nevăzute, fie tronuri, fie domnii, fie începătorii, fie stăpânii. Toate s-au făcut prin El și pentru El.
\par 17 El este mai înainte decât toate și toate prin El sunt așezate.
\par 18 Și El este capul trupului, al Bisericii; El este începutul, întâiul născut din morți, ca să fie El cel dintâi întru toate.
\par 19 Căci în El a binevoit (Dumnezeu) să sălășluiască toată plinirea.
\par 20 Și printr-Însul toate cu Sine să le împace, fie cele de pe pământ, fie cele din ceruri, făcând pace prin El, prin sângele crucii Sale.
\par 21 Dar pe voi, care oarecând erați înstrăinați și vrăjmași cu mintea voastră către lucrurile rele, de acum v-a împăcat,
\par 22 Prin moartea (Fiului Său) în trupul cărnii Lui, ca să vă pună înaintea Sa sfinți, fără de prihană și nevinovați,
\par 23 Dacă, într-adevăr, rămâneți întemeiați în credință, întăriți și neclintiți de la nădejdea Evangheliei pe care ați auzit-o, care a fost propovăduită la toată făptura de sub cer și al cărei slujitor m-am făcut eu, Pavel.
\par 24 Acum mă bucur de suferințele mele pentru voi și împlinesc, în trupul meu, lipsurile necazurilor lui Hristos, pentru trupul Lui, adică Biserica,
\par 25 Al cărei slujitor m-am făcut, potrivit iconomiei lui Dumnezeu, ce mi-a fost dată mie pentru voi, ca să aduc la îndeplinire cuvântul lui Dumnezeu,
\par 26 Taina cea din veci ascunsă neamurilor, iar acum descoperită sfinților Săi,
\par 27 Cărora a voit Dumnezeu să le arate care este bogăția slavei acestei taine între neamuri, adică Hristos cel dintru voi, nădejdea slavei.
\par 28 Pe El noi Îl vestim, sfătuind pe orice om și învățând pe orice om, întru toată înțelepciunea, ca să înfățișăm pe tot omul, desăvârșit, în Hristos Iisus.
\par 29 Spre aceasta mă și ostenesc și mă lupt, potrivit lucrării Lui, care se săvârșește în mine cu putere.

\chapter{2}

\par 1 Căci voiesc ca voi să știți cât de mare luptă am pentru voi și pentru cei din Laodiceea și pentru toți câți n-au văzut fața mea în trup,
\par 2 Ca să se mângâie inimile lor, și ca ei, strâns uniți în iubire, să aibă belșugul deplinei înțelegeri pentru cunoașterea tainei lui Dumnezeu-Tatăl și a lui Hristos,
\par 3 Întru care sunt ascunse toate vistieriile înțelepciunii și ale cunoștinței.
\par 4 Vă spun aceasta, ca nimeni să nu vă înșele prin cuvinte amăgitoare.
\par 5 Căci deși cu trupul sunt departe, cu duhul însă sunt împreună cu voi, bucurându-mă și văzând buna voastră rânduială și tăria credinței voastre în Hristos.
\par 6 Deci, precum ați primit pe Hristos Iisus, Domnul, așa să umblați întru El.
\par 7 Înrădăcinați și zidiți fiind într-Însul, întăriți în credință, după cum ați fost învățați, și prisosind în ea cu mulțumire.
\par 8 Luați aminte să nu vă fure mințile cineva cu filozofia și cu deșarta înșelăciune din predania omenească, după înțelesurile cele slabe ale lumii și nu după Hristos.
\par 9 Căci întru El locuiește, trupește, toată plinătatea Dumnezeirii,
\par 10 Și sunteți deplini întru El, Care este cap a toată domnia și stăpânirea.
\par 11 În El ați și fost tăiați împrejur, cu tăiere împrejur nefăcută de mână, prin dezbrăcarea de trupul cărnii, întru tăierea împrejur a lui Hristos.
\par 12 Îngropați fiind împreună cu El prin botez, cu El ați și înviat prin credința în lucrarea lui Dumnezeu, Cel ce L-a înviat pe El din morți.
\par 13 Iar pe voi care erați morți, în fărădelegile și în netăierea împrejur a trupului vostru, v-a făcut vii, împreună cu Sine, iertându-ne toate greșealele;
\par 14 Ștergând zapisul ce era asupra noastră, care ne era potrivnic cu rânduielile lui, și l-a luat din mijloc, pironindu-l pe cruce.
\par 15 Dezbrăcând (de putere) începătoriile și stăpâniile, le-a dat de ocară în văzul tuturor, biruind asupra lor prin cruce.
\par 16 Nimeni deci să nu vă judece pentru mâncare sau băutură, sau cu privire la vreo sărbătoare, sau lună nouă, sau la sâmbete,
\par 17 Care sunt umbră celor viitoare iar trupul (este) al lui Hristos.
\par 18 Nimeni să nu vă smulgă biruința printr-o prefăcută smerenie și printr-o fățarnică închinare la îngeri, încercând să pătrundă în cele ce n-a văzut, și îngâmfându-se zadarnic cu închipuirea lui trupească,
\par 19 În loc să se țină strâns de capul de la care trupul tot, - prin încheieturi și legături, îndestulându-se și întocmindu-se S, sporește în creșterea lui Dumnezeu.
\par 20 Dacă deci ați murit împreună cu Hristos pentru înțelesurile cele slabe ale lumii, pentru ce atunci, ca și cum ați viețui în lume, răbdați porunci ca acestea:
\par 21 Nu lua, nu gusta, nu te atinge!
\par 22 - Toate lucruri menite să piară prin întrebuințare - potrivit unor rânduieli și învățături omenești?
\par 23 Unele ca acestea au oarecare înfățișare de înțelepciune, în păruta lor cucernicie, în smerenie și în necruțarea trupului, dar n-au nici un preț și sunt numai pentru sațiul trupului.

\chapter{3}

\par 1 Așadar, dacă ați înviat împreună cu Hristos, căutați cele de sus, unde se află Hristos, șezând de-a dreapta lui Dumnezeu;
\par 2 Cugetați cele de sus, nu cele de pe pământ;
\par 3 Căci voi ați murit și viața voastră este ascunsă cu Hristos întru Dumnezeu.
\par 4 Iar când Hristos, Care este viața voastră, Se va arăta, atunci și voi, împreună cu El, vă veți arăta întru slavă.
\par 5 Drept aceea, omorâți mădularele voastre, cele pământești: desfrânarea, necurăția, patima, pofta rea și lăcomia, care este închinare la idoli,
\par 6 Pentru care vine mânia lui Dumnezeu peste fiii neascultării,
\par 7 În care păcate ați umblat și voi odinioară, pe când trăiați în ele.
\par 8 Acum deci vă lepădați și voi de toate acestea: mânia, iuțimea, răutatea, defăimarea, cuvântul de rușine din gura voastră.
\par 9 Nu vă mințiți unul pe altul, fiindcă v-ați dezbrăcat de omul cel vechi, dimpreună cu faptele lui,
\par 10 Și v-ați îmbrăcat cu cel nou, care se înnoiește, spre deplină cunoștință, după chipul Celui ce l-a zidit,
\par 11 Unde nu mai este elin și iudeu, tăiere împrejur și netăiere împrejur, barbar, scit, rob ori liber, ci toate și întru toți Hristos.
\par 12 Îmbrăcați-vă, dar, ca aleși ai lui Dumnezeu, sfinți și prea iubiți, cu milostivirile îndurării, cu bunătate, cu smerenie, cu blândețe, cu îndelungă-răbdare,
\par 13 Îngăduindu-vă unii pe alții și iertând unii altora, dacă are cineva vreo plângere împotriva cuiva; după cum și Hristos v-a iertat vouă, așa să iertați și voi.
\par 14 Iar peste toate acestea, îmbrăcați-vă întru dragoste, care este legătura desăvârșirii.
\par 15 Și pacea lui Hristos, întru care ați fost chemați, ca să fiți un singur trup, să stăpânească în inimile voastre; și fiți mulțumitori.
\par 16 Cuvântul lui Hristos să locuiască întru voi cu bogăție. Învățați-vă și povățuiți-vă între voi, cu toată înțelepciunea. Cântați în inimile voastre lui Dumnezeu, mulțumindu-I, în psalmi, în laude și în cântări duhovnicești.
\par 17 Orice ați face, cu cuvântul sau cu lucrul, toate să le faceți în numele Domnului Iisus și prin El să mulțumiți lui Dumnezeu-Tatăl.
\par 18 Femeilor, supuneți-vă bărbaților voștri, precum se cuvine, în Domnul.
\par 19 Bărbaților, iubiți pe femeile voastre și nu fiți aspri cu ele.
\par 20 Copiilor, ascultați pe părinții voștri întru toate, căci aceasta este bine-plăcut Domnului.
\par 21 Părinților, nu ațâțați la mânie pe copiii voștri, ca să nu se deznădăjduiască.
\par 22 Slugilor, ascultați întru toate pe stăpânii voștri cei trupești, nu slujind numai când sunt cu ochii pe voi, ca cei ce caută să placă oamenilor, ci în curăția inimii, temându-vă de Domnul.
\par 23 Orice ați face, lucrați din toată inima, ca pentru Domnul și nu ca pentru oameni,
\par 24 Bine știind că de la Domnul veți primi răsplata moștenirii; căci Domnului Hristos slujiți.
\par 25 Iar cel ce face nedreptate își va lua plata nedreptății, întrucât la Dumnezeu nu este părtinire.

\chapter{4}

\par 1 Stăpânilor, dați slugilor voastre ce este drept și potrivit, știind că și voi aveți Stăpân în ceruri.
\par 2 Stăruiți în rugăciune, priveghind în ea cu mulțumire,
\par 3 Rugându-vă totodată și pentru noi, ca Dumnezeu să ne deschidă ușa cuvântului, spre a vesti taina lui Hristos, pentru care mă și găsesc în lanțuri,
\par 4 Ca să o arăt așa cum se cuvine să grăiesc.
\par 5 Umblați cu înțelepciune față de cei ce sunt afară (de Biserică), prețuind vremea.
\par 6 Vorba voastră să fie totdeauna plăcută, dreasă cu sare, ca să știți cum trebuie să răspundeți fiecăruia.
\par 7 Toate câte mă privesc pe mine le va face cunoscute Tihic, iubitul frate, credincios slujitor și împreună-rob cu mine în Domnul.
\par 8 L-am trimis pe el la voi tocmai pentru aceasta, ca să știți cum ne aflăm și ca să mângâie inimile voastre,
\par 9 Împreună cu Onisim, credinciosul și iubitul frate, care este dintre voi; ei vă vor aduce la cunoștință toate cele de aici.
\par 10 Vă îmbrățișează Aristarh, cel întemnițat împreună cu mine, și Marcu, vărul lui Barnaba - în privința căruia ați primit porunci; de va veni la voi primiți-l -,
\par 11 Asemenea și Iisus, cel ce se numește Iustus, care sunt din tăierea împrejur; numai aceștia au lucrat împreună cu mine pentru împărăția lui Dumnezeu. Ei au fost cei ce mi-au adus mângâiere.
\par 12 Vă îmbrățișează Epafras, care este dintre voi, rob al lui Iisus Hristos, pururea luptând pentru voi în rugăciunile sale, ca să stați desăvârșiți și plini de tot ce este voința lui Dumnezeu.
\par 13 Căci martor îi sunt că are multă râvnă pentru voi și pentru cei din Laodiceea și din Ierapole.
\par 14 Vă îmbrățișează Luca, doctorul cel iubit, și Dima.
\par 15 Îmbrățișați pe frații din Laodiceea și pe Nimfas și pe Biserica din casa lui.
\par 16 Și după ce scrisoarea aceasta se va citi de către voi, faceți să se citească și în Biserica laodiceenilor, iar pe cea din Laodiceea să o citiți și voi.
\par 17 Și spuneți lui Arhip: Vezi de slujba pe care ai primit-o întru Domnul, ca să o îndeplinești.
\par 18 Salutarea cu mâna mea, a lui Pavel. Aduceți-vă aminte de lanțurile mele. Harul fie cu voi! Amin.


\end{document}