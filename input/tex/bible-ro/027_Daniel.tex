\begin{document}

\title{Daniel}

Dan 1:1  În anul al treilea al domniei lui Ioiachim, regele lui Iuda, a pornit Nabucodonosor, regele Babilonului, împotriva Ierusalimului ?i 1-a împresurat.
Dan 1:2  ?i a dat Domnul în mâna lui pe Ioiachim, regele lui Iuda, ?i o parte din vasele templului lui Dumnezeu, pe care le-a dus în ?ara ?inear, in templul dumnezeului lui ?i a a?ezat aceste vase în vistieria acestui dumnezeu.
Dan 1:3  ?i a zis regele catre A?penaz, cel mai mare peste famenii sai, sa aduca dintre fiii lui Israel, din neam regesc ?i din cei din vi?a aleasa,
Dan 1:4  Patru tineri fara nici un cusur trupesc, frumo?i la chip ?i iscusi?i în toata în?elepciunea, cunoscatori a toata ?tiin?a, cu adânca putere de patrundere ?i plini de râvna, ca sa slujeasca in palatul regelui, ca sa-i înve?e pe ei scrisul ?i limba Caldeilor.
Dan 1:5  ?i sa-i înve?e pe ei vreme de trei ani, dupa trecerea carora sa intre în slujba regelui. Pentru aceasta, regele le-a rânduit sa li se dea în fiecare zi bucate din bucatele lui ?i vin din vinul lui.
Dan 1:6  Printre ei se aflau din fiii lui Iuda: Daniel, Anania, Misael ?i Azaria.
Dan 1:7  ?i capetenia famenilor le-a dat nume ?i a chemat pe Daniel, Belt?a?ar, pe Anania, ?adrac, pe Misael, Me?ac ?i pe Azaria, Abed-Nego.
Dan 1:8  ?i Daniel ?i-a pus in gând sa nu se spurce cu bucatele ?i vinul regelui ?i a cerut voie capeteniei famenilor sa nu se pângareasca.
Dan 1:9  ?i Dumnezeu i-a dat sa afle înaintea capeteniei famenilor bunavoin?a ?l îndurare.
Dan 1:10  ?i a zis capetenia famenilor catre Daniel: "Ma tem numai de domnul meu, regele, care v-a rânduit mîncarurile ?i bauturile; pentru ce sa vada el mai trase chipurile voastre decât ale tinerilor celor de o vârsta cu voi? Voi pune?i în primejdie capul meu înaintea regelui.
Dan 1:11  ?i a grait Daniel catre supraveghetorul pe care capetenia famenilor îl pusese peste Daniel, Anania, Misael ?i Azaria:
Dan 1:12  "Încearca o data cu servii tai timp de zece zile ?i da-ne sa mâncam bucate din zarzavaturi ?i sa bem apa!
Dan 1:13  Apoi sa te ui?i la chipurile noastre ?i la chipul tinerilor care manânca din bucatele regelui ?i cum ?i se va parea, a?a fa cu supu?ii tai!"
Dan 1:14  Atunci el le-a ascultat aceasta rugaminte ?i a încercat timp de zece zile.
Dan 1:15  ?i dupa un rastimp de zece zile, ei aratau mai frumo?i ?i mai gra?i la trup decât to?i tinerii care mâncau din bucatele regelui.
Dan 1:16  Atunci supraveghetorul le-a înlocuit mâncarea lor ?i vinul - bautura dor - ?i le-a dat mâncaruri de legume.
Dan 1:17  ?i acestor patru tineri le-a dat Dumnezeu ?tiin?a ?i pricepere în oricare scriere, precum ?i în?elepciune, încât Daniel putea sa tâlcuiasca vedeniile ?i visele.
Dan 1:18  ?i dupa acel rastimp hotarât de rege ca sa-i aduca, i-a înfa?i?at capetenia famenilor înaintea lui Nabucodonosor.
Dan 1:19  ?i a grait regele cu ei, dar n-a aflat pe nimeni printre ei to?i ca pe Daniel, Anania, Misael ?i Azaria. ?i ei au început sa slujeasca pe rege.
Dan 1:20  ?i în toate întrebarile, asupra carora regele le cerea dezlegarea, când era vorba de în?elepciune ?i de pricepere, îi gasea pe ei de zece ori mai iste?i decât magii ?i prezicatorii din tot regatul lui.
Dan 1:21  ?i a ramas Daniel acolo pâna în anul întâi al domniei lui Cirus.
Dan 2:1  ?i în anul al doilea al domniei lui Nabucodonosor, a visat Nabucodonosor vise, iar duhul lui s-a tulburat ?i somnul nu-l mai prindea.
Dan 2:2  ?i a poruncit regele sa cheme pe vrajitori, pe prezicatori, pe magi ?i pe caldei, ca sa-i spuna regelui visele lui; ?i ei au intrat ?i s-au înfa?i?at înaintea regelui.
Dan 2:3  ?i le-a zis lor regele: "Am visat un vis; duhul îmi este tulburat ?i vreau sa ?tiu visul".
Dan 2:4  Atunci caldeii au grait catre rege în grai arameian: "O, rege, sa traie?ti în veac! Spune servilor tai visul ?i noi î?i vom descoperi tâlcuirea".
Dan 2:5  Raspuns-a regele ?i a zis catre caldei: "Sa ?ti?i ca hotarârea am luat-o! Daca nu-mi ve?i face cunoscut visul ?i tâlcuirea lui, ve?i fi taia?i în buca?i, iar casele voastre prefacute în gramezi de ruine.
Dan 2:6  Dar daca îmi face?i cunoscut visul ?i tâlcuirea lui ve?i primi de la mine daruri bogate ?i cinstire multa; deci arata?i-mi visul ?i tâlcuirea lui!"
Dan 2:7  Ei au raspuns pentru a doua oara ?i au zis: "0, rege, spune servilor tai visul, iar noi î?i vom face cunoscuta tâlcuirea lui!"
Dan 2:8  Raspuns-a ?i a zis regele: "Fara îndoiala, eu ?tiu ca voi cauta?i sa câ?tiga?i vreme, fiindca vede?i ca eu hotarârea am luat-o;
Dan 2:9  Ca daca nu-mi face?i cunoscut visul, este ca ave?i de gând sa va sfatui?i unul cu altul ?i sa spune?i înaintea mea vorbe mincinoase ?i în?elatoare, pâna când vremurile se vor schimba. De aceea spune?i-mi acum visul ?i eu voi ?ti daca voi pute?i sa-mi descoperi?i ?i tâlcuirea lui!"
Dan 2:10  Raspuns-au caldeii în fa?a regelui ?i au zis: "Nu se afla om pe pamânt care sa poata face cunoscut ceea ce cere regele, fiindca nici un rege, oricât de mare ?i de puternic ar fi, nu ar cere a?a ceva de la nici un tâlcuitor de semne, vrajitor, sau caldeu.
Dan 2:11  ?i lucrul pe care îl cere regele este greu ?i nimeni altul nu poate sa-l descopere înaintea lui, decât zeii al caror loca? nu este printre cei muritori".
Dan 2:12  Din aceasta pricina regele s-a mâniat ?i, în marea lui furie, a poruncit sa fie uci?i to?i în?elep?ii din Babilon.
Dan 2:13  Când a ie?it porunca, sa fie omorâ?i în?elep?ii, trebuia sa fie omorât ?i Daniel ?i cei dimpreuna cu el.
Dan 2:14  Atunci Daniel ?i-a îndreptat cuvântul cu în?elepciune ?i iscusin?a catre Arioh, mai-marele garzii regelui, care ie?ise sa omoare pe în?elep?ii din Babilon,
Dan 2:15  ?i a început sa graiasca catre Arioh, împuternicitul regelui: "De ce a dat regele o porunca a?a de aspra?" Dupa ce Arioh i-a lamurit lui Daniel rostul poruncii,
Dan 2:16  Daniel a plecat ?i a rugat pe rege sa-i lase vreme ca sa-i descopere tâlcul.
Dan 2:17  Dupa aceasta Daniel s-a dus în casa lui ?i a dat de ?tire lui Anania, Misael ?i Azaria, prietenii lui, care este pricina,
Dan 2:18  Cerându-le sa roage fierbinte milostivirea lui Dumnezeu din ceruri pentru aceasta taina, ca sa nu lase sa piara Daniel ?i prietenii lui împreuna cu ceilal?i în?elep?i ai Babilonului.
Dan 2:19  Atunci i s-a descoperit lui Daniel taina aceasta într-o vedenie de noapte. ?i a preaslavit Daniel pe Dumnezeul cerului.
Dan 2:20  ?i a început Daniel a grai: "Sa fie numele lui Dumnezeu binecuvântat din veac ?i pâna în veac, ca a Lui este în?elepciunea ?i puterea.
Dan 2:21  ?i El este Cel care schimba timpurile ?i ceasurile, Cel care da jos de pe tron pe regi ?i Cel care îi pune; El da în?elepciune celor în?elep?i ?i ?tiin?a celor pricepu?i.
Dan 2:22  El descopera cele mai adânci ?i cele mai ascunse lucruri, ?tie ce se petrece în întuneric ?i lumina sala?luie?te cu El.
Dan 2:23  Pe Tine, Dumnezeule al parin?ilor mei, Te preaslavesc ?i Î?i mul?umesc ?ie, ca mi-ai dat mie în?elepciune ?i pricepere ?i m-ai facut sa cunosc acum ceea ce noi ?i-am cerut rugându-Te, caci Tu ne-ai descoperit taina regelui".
Dan 2:24  Apoi Daniel s-a dus la Arioh, pe care regele îl însarcinase sa omoare pe în?elep?ii Babilonului ?i i-a grait a?a: "Nu da mor?ii pe în?elep?ii Babilonului! Du-ma înaintea regelui ?i eu îi voi descoperi regelui tâlcuirea".
Dan 2:25  Atunci Arioh a dus grabnic înauntru pe Daniel în fa?a regelui ?i i-a vorbit astfel: "Am gasit un iudeu dintre cei adu?i în robie care poate sa tâlcuiasca visul".
Dan 2:26  Raspuns-a regele ?i a zis catre Daniel, care se cheama Belt?a?ar: "Oare e?ti tu în stare sa-mi spui visul pe care l-am avut precum ?i tâlcuirea lui?"
Dan 2:27  Daniel a raspuns înaintea regelui zicând: "Taina pe care vrea sa o afle regele nu pot s-o faca cunoscuta lui nici în?elep?ii, nici prezicatorii, nici vrajitorii, nici cititorii în stele.
Dan 2:28  Dar este un Dumnezeu în ceruri, Care descopera tainele ?i Care a facut cunoscut regelui Nabucodonosor ce se va întâmpla în vremurile ce vor veni. Iata care este visul ?i vedenia pe care le-ai avut când erai culcat în patul tau:
Dan 2:29  ?ie, rege, î?i treceau gânduri prin minte, când erai în patul tau, pentru ceea ce se va întâmpla mai pe urma ?i Cel ce ?i-a descoperit taina ?i-a dat sa ?tii ce va fi.
Dan 2:30  ?i mie, nu prin în?elepciunea care ar fi în mine mai mult decât la to?i cei vii, mi s-a descoperit taina aceasta, ci ca sa se faca ?tiut regelui în?elesul ei ca sa cuno?ti gândurile inimii tale.
Dan 2:31  O, rege! Tu priveai ?i iata un chip - acest chip era peste masura de mare ?i stralucirea lui neobi?nuita statea înaintea ta ?i înfa?i?area lui era grozava.
Dan 2:32  Acest chip avea capul de aur curat, pieptul ?i bra?ele de argint, pântecele ?i coapsele de arama,
Dan 2:33  Pulpele de fier, iar picioarele o parte de fier ?i o parte de lut.
Dan 2:34  Tu priveai ?i iata o piatra desprinsa, nu de mâna, a lovit chipul peste picioarele de fier ?i de lut ?i le-a sfarâmat.
Dan 2:35  Atunci au fost sfarâmate în acela?i timp fierul, lutul, arama, argintul ?i aurul ?i au ajuns ca pleava de pe arie vara ?i vântul le-a luat cu sine fara ca sa se gaseasca locul lor; iar piatra care a lovit chipul a crescut munte mare ?i a umplut tot pamântul.
Dan 2:36  Iata visul, iar tâlcuirea lui o vom spune înaintea regelui.
Dan 2:37  Tu, rege al regilor, caruia Dumnezeul cerului i-a dat regatul, puterea, taria ?i marirea,
Dan 2:38  ?i în mâinile caruia a dat pe fiii oamenilor în orice ?inut ar locui, precum ?i fiarele câmpului ?i pasarile cerului ?i i-a dat stapânire peste toate, tu e?ti capul de aur.
Dan 2:39  ?i dupa tine se va ridica un alt regat mai mic decât al tau, apoi un al treilea regat de arama, care va stapâni peste tot pamântul.
Dan 2:40  ?i un al patrulea regat va fi tare ca fierul ?i, dupa cum fierul sfarâma ?i zdrobe?te totul, a?a ?i el va sfarâma ?i va preface totul în pulbere, ca fierul care face totul buca?i.
Dan 2:41  Iar picioarele pe care le-ai vazut ?i degetele, unele de lut de olar ?i altele de fier, înseamna ca va fi un regat împar?it ?i va fi tare ea fierul, dupa cum tu ai vazut fier amestecat cu lut.
Dan 2:42  ?i degetele picioarelor, unele de fier ?i altele de lut, înseamna ca regatul va fi parte tare, parte ?ubred.
Dan 2:43  ?i dupa cum ai vazut fierul amestecat cu lutul, a?a se vor amesteca prin înrudiri, dar nu vor avea legatura temeinica între ele, dupa cum fierul nu se poate amesteca la un loc cu lutul.
Dan 2:44  Iar în vremea acestor regi, Dumnezeul cerului va ridica un regat ve?nic care nu va fi nimicit niciodata ?i care nu va fi trecut la alt popor; El va sfarâma ?i va nimici toate aceste regate ?i singur El va ramâne în veci.
Dan 2:45  Dupa cum tu ai vazut ca o piatra a fost desprinsa din munte, nu de mâna, ?i a zdrobit fierul, arama, lutul, argintul ?i aurul, Marele Dumnezeu a dat de ?tire regelui ceea ce va fi în viitor; visul este adevarat ?i tâlcuirea lui neîndoielnica".
Dan 2:46  Atunci împaratul Nabucodonosor a cazut cu fa?a la pamânt ?i s-a închinat înaintea lui Daniel ?i a dat porunca sa-i aduca jertfe ?i tamâieri.
Dan 2:47  Raspuns-a regele catre Daniel ?i a zis: "Cu adevarat ca Dumnezeul vostru este Dumnezeul dumnezeilor ?i Stapânul regilor, descoperitorul tainelor, caci tu ai putut sa descoperi aceasta taina".
Dan 2:48  Atunci regele a ridicat la mare vrednicie pe Daniel ?i i-a dat daruri ?i numeroase lucruri de mare pre? ?i 1-a pus guvernator peste tot ?inutul Babilonului ?i capetenie peste to?i în?elep?ii Babilonului.
Dan 2:49  La rugamintea lui Daniel, regele a însarcinat pe ?adrac, Me?ac ?i Abed-Nego cu ocârmuirea ?inutului Babilonului, iar Daniel a ramas la curtea regelui.
Dan 3:1  Regele Nabucodonosor a facut un chip de aur înalt de ?aizeci de co?i, lat de ?ase co?i ?i 1-a a?ezat în câmpia Dura (Deir) din ?inutul Babilonului.
Dan 3:2  ?i regele Nabucodonosor a trimis sa adune pe satrapi, pe mai-marii dregatori, pe cârmuitori, pe conducatorii o?tirilor, pe vistiernici, pe cunoscatorii de legi, pe judecatori ?i pe to?i ceilal?i dregatori ai ?inuturilor, ca sa vina la sfin?irea chipului pe care îl ridicase regele Nabucodonosor.
Dan 3:3  Atunci s-au adunat satrapii, dregatorii cei mari, cârmuitorii, conducatorii o?tirilor, vistiernicii, legiuitorii, judecatorii ?i to?i ceilal?i dregatori ai ?inuturilor la sfin?irea chipului pe care îl ridicase Nabucodonosor ?i au stat înaintea chipului ridicat de Nabucodonosor.
Dan 3:4  ?i îndata un crainic a strigat cu glas tare: "Iata ce vi se porunce?te voua, popoarelor, neamurilor ?i limbilor:
Dan 3:5  De îndata ce ve?i auzi glasul trâmbi?ei, flautului, chitarei, harpei, psalterionului, cimpoiului ?i al tuturor instrumentelor muzicale, ve?i cadea la pamânt ?i va ve?i închina chipului de aur pe care 1-a ridicat regele Nabucodonosor;
Dan 3:6  Iar cine nu va cadea la pamânt ?i nu se va închina, chiar în acea clipa va fi aruncat în mijlocul unui cuptor cu foc arzator!"
Dan 3:7  De aceea, când toate popoarele au auzit glasul trâmbi?ei, al flautului, al chitarei, al harpei, al psalterionului ?i al tuturor instrumentelor muzicale, toate popoarele, neamurile ?i limbile au cazut la pamânt ?i s-au închinat chipului de aur pe care îl ridicase regele Nabucodonosor.
Dan 3:8  În acela?i timp s-au apropiat câ?iva barba?i caldei, care au pârât pe iudei.
Dan 3:9  Ei au început sa spuna regelui Nabucodonosor: "O, rege, sa traie?ti în veac!
Dan 3:10  Tu porunca ai dat, ca oricine va auzi glasul trâmbi?ei, al flautului, al chitarei, al harpei, al psalterionului, al cimpoiului ?i al altor instrumente muzicale, sa cada la pamânt ?i sa se închine chipului de aur.
Dan 3:11  Iar cine nu va cadea la pamânt, nici se va închina, sa fie aruncat în mijlocul unui cuptor cu foc arzator.
Dan 3:12  Dar sunt ni?te iudei, pe care i-ai pus cârmuitori peste ?inutul Babilonului: ?adrac, Me?ac ?i Abed-Nego. Ace?ti barba?i, nici ca au luat în seama porunca ta, o, rege; dumnezeului tau nu-i slujesc ?i chipului de aur pe care tu l-ai înal?at nu-i aduc închinare!"
Dan 3:13  Atunci regele Nabucodonosor, plin de mânie ?i de zbucium, a poruncit sa i se aduca înainte ?adrac, Me?ac ?i Abed-Nego. Îndata au adus pe ace?ti barba?i înaintea regelui.
Dan 3:14  Nabucodonosor le-a zis: "Este, oare, adevarat, ?adrac, Me?ac ?i Abed-Nego, ca voi nu sluji?i dumnezeului meu ?i chipului de aur pe care eu l-am a?ezat ?i nu-i cade?i la pamânt cu rugaciuni?
Dan 3:15  Acum, fi?i gata ?i atunci când ve?i auzi glasul trâmbi?ei, al flautului, al chitarei, al harpei, al psalterionului, al cimpoiului ?i al altor instrumente muzicale, sa cade?i la pamânt ?i sa va închina?i chipului pe care eu l-am facut; iar daca nu vre?i sa va închina?i, într-o clipa ve?i fi arunca?i în mijlocul unui cuptor cu foc arzator. ?i care dumnezeu va va scapa din mâna mea?"
Dan 3:16  Raspuns-au ?adrac, Me?ac ?i Abed-Nego ?i au zis regelui: O, Nabucodonosor, noi n-avem nevoie ca la aceasta sa-?i dam un raspuns!
Dan 3:17  Daca, într-adevar, Dumnezeul nostru Caruia Îi slujim poate sa ne scape, El ne va scapa din cuptorul cel cu foc arzator ?i din mâna ta, o, rege!
Dan 3:18  ?i chiar daca nu ne va scapa, ?tiut sa fie de tine, o, rege, ca noi nu vom sluji dumnezeilor tai ?i înaintea chipului de aur pe care tu l-ai a?ezat nu vom cadea la pamânt!"
Dan 3:19  Atunci Nabucodonosor s-a umplut de mânie ?i ?i-a schimbat înfa?i?area fe?ei sale fa?a de ?adrac, Me?ac ?i Abed-Nego. ?i începând iara?i a grai, a poruncit sa încalzeasca cuptorul de ?apte ori mai mult decât era de obicei.
Dan 3:20  ?i a poruncit celor mai puternici oameni din o?tirea lui sa lege pe ?adrac, Me?ac ?i Abed-Nego ?i sa-i arunce în cuptorul cel cu foc arzator.
Dan 3:21  Atunci ace?ti oameni, îmbraca?i cum erau, cu mantie, încal?aminte, palarie ?i cu toata îmbracamintea lor, au fost lega?i ?i arunca?i în mijlocul cuptorului cu foc arzator.
Dan 3:22  Fiindca porunca regelui era grabnica ?i cuptorul foarte înfierbântat, acei oameni care au aruncat pe ?adrac, Me?ac ?i Abed-Nego au fost mistui?i de vapaia focului.
Dan 3:23  ?i ace?ti trei barba?i, ?adrac, Me?ac ?i Abed-Nego au cazut lega?i în mijlocul cuptorului cu foc arzator.
Dan 3:24  Atunci regele Nabucodonosor a fost cuprins de spaima ?i s-a sculat în graba. El a început a grai ?i a zis catre sfetnicii sai: "Oare, n-am aruncat noi trei barba?i lega?i în mijlocul cuptorului cu foc arzator?" Raspuns-au ?i i-au zis: "Cu adevarat, a?a este, o, rege!"
Dan 3:25  ?i începând din nou a grai, a zis: "Iata, eu vad patru barba?i dezlega?i, umblând prin mijlocul cuptorului, nevatama?i, iar chipul celui de al patrulea, ca fa?a unuia dintre fiii zeilor".
Dan 3:26  Atunci s-a apropiat Nabucodonosor de gura cuptorului cu foc arzator ?i, începând a grai, a zis: "?adrac, Me?ac ?i Abed-Nego, slujitorii mei, ie?i?i afara ?i veni?i la mine!" Atunci ?adrac, Me?ac ?i Abed-Nego au ie?it dinauntrul cuptorului.
Dan 3:27  ?i adunându-se satrapii, dregatorii cei mai mari, cârmuitorii ?i sfetnicii regelui, au vazut ca focul nu pricinuise nici o vatamare trupului acestor oameni, ca nici perii capului nu se pârlisera ?i ca hainele lor erau neschimbate ?i ca nici macar nu miroseau a foc.
Dan 3:28  Raspuns-a Nabucodonosor ?i a zis: "Binecuvântat sa fie Dumnezeul lui ?adrac, Me?ac ?i Abed-Nego, Care a trimis pe îngerul Sau ?i a izbavit pe servii Sai, care î?i pusesera nadejdea în El ?i care au calcat porunca regelui ?i ?i-au dat trupurile lor ca sa nu slujeasca ?i sa nu se închine altor dumnezei decât Dumnezeului lor.
Dan 3:29  ?i poruncesc: Popoare, neamuri ?i limbi, to?i aceia care ar vorbi de rau pe Dumnezeul lui ?adrac, Me?ac ?i Abed-Nego, sa fie taia?i în buca?i ?i casele lor sa fie nimicite, fiindca nu este un alt dumnezeu care sa-i poata izbavi într-acest chip".
Dan 3:30  Dupa aceasta a întarit regele în slujbele lor pe ?adrac, Me?ac ?i Abed-Nego, peste ?inutul Babilonului.
Dan 4:1  Regele Nabucodonosor a dat hrisov catre toate popoarele, neamurile ?i limbile care locuiesc pe tot pamântul: "Pacea voastra sa sporeasca!
Dan 4:2  Placutu-mi-a sa vestesc minunile ?i faptele cele peste fire, pe care le-a facut mie Dumnezeul cel Preaînalt.
Dan 4:3  Cât de mari sunt minunile Lui ?i cât de puternice sunt faptele cele peste fire! Împara?ia Lui este împara?ie ve?nica ?i stapânirea Lui ?ine din neam în neam!
Dan 4:4  Eu, Nabucodonosor, stam fara de grija în casa mea ?i bucuros de via?a în patul meu.
Dan 4:5  Am visat un vis care m-a înspaimântat; ?i gândurile mele când stam culcat în patul meu ?i vedeniile pe care le-am avut m-au framântat adânc.
Dan 4:6  ?i am poruncit sa mi se aduca în fa?a mea to?i în?elep?ii din Babilon, care sa-mi tâlcuiasca visul.
Dan 4:7  Atunci au sosit tâlcuitorii de semne, prezicatorii, caldeii ?i cititorii în stele ?i le-am spus visul, dar ei nu mi-au dat tâlcuirea lui.
Dan 4:8  Iar în cele din urma s-a înfa?i?at înaintea mea Daniel, al carui nume este Belt?a?ar, dupa numele dumnezeului meu ?i care are în al Duhul Dumnezeului celui Sfânt ?i i-am spus visul:
Dan 4:9  "Belt?a?ar, tu, mai-marele tâlcuitorilor de semne, tu, cel în care ?tiu ca locuie?te Duhul lui Dumnezeu celui Sfânt ?i ca nici o taina nu-?i este grea, ia aminte la visul pe care l-am visat ?i spune-mi tâlcuirea lui!
Dan 4:10  Vedenia pe care am avut-o, când eram culcat în patul meu, a fost: "Ma uitam ?i iata un copac în mijlocul pamântului, înalt foarte.
Dan 4:11  Copacul cre?tea ?i era puternic ?i vârful lui ajungea pâna la cer ?i se putea vedea pâna la capatul pamântului.
Dan 4:12  Frunzi?ul lui era frumos ?i roadele lui multe, ?i hrana pentru to?i se afla în el. Sub el cautau umbra fiarele câmpului, iar în ramurile lui î?i faceau cuiburi pasarile cerului ?i din el se hraneau toate vie?uitoarele.
Dan 4:13  Priveam în vedenia pe care am avut-o, când eram în patul meu ?i iata un înger, un sfânt, se cobora din ceruri;
Dan 4:14  El a strigat cu glas tare ?i a poruncit a?a: Doborâ?i copacul ?i taia?i-i crengile, scutura?i frunzele lui ?i împra?tia?i roadele lui, ca animalele sa fuga de sub el ?i pasarile din frunzi?ul lui!
Dan 4:15  Iar butucul ?i radacinile sa ramâna în pamânt în legaturi de fier ?i de arama, în iarba câmpului! Din roua cerului sa fie udat ?i cu dobitoacele câmpului sa împarta iarba pamântului.
Dan 4:16  Inima lui sa nu mai fie inima de om, ci o inima de dobitoc sa-i fie data ?i ?apte ani sa treaca peste el!
Dan 4:17  Aceasta hotarâre se sprijina pe porunca îngerilor, iar porunca sfin?ilor este ca sa cunoasca cei vii ca Cel Preaînalt stapâne?te peste împara?ia oamenilor, pe care o da cui vrea ?i poate sa ridice peste ea pe cel mai de jos dintre oameni.
Dan 4:18  Acesta este visul pe care l-am visat eu, regele Nabucodonosor, iar tu, Belt?a?ar, spune tâlcuirea lui, caci to?i în?elep?ii regatului meu nu pot sa-mi faca cunoscuta tâlcuirea. Tu însa e?ti în stare, fiindca ai în tine Duhul Dumnezeului celui Sfânt".
Dan 4:19  Atunci Daniel, al carui nume este Belt?a?ar, a ramas înmarmurit pentru o clipa ?i gândurile lui s-au tulburat. Regele a prins din nou a grai ?i a zis: "Belt?a?ar, visul ?i tâlcuirea lui sa nu te înfrico?eze!" Raspuns-a Belt?a?ar ?i a zis: "O, stapâne, visul sa fie pentru cei ce te urasc pe tine, iar tâlcuirea lui pentru vrajma?ii tai!
Dan 4:20  Copacul pe care tu l-ai vazut, mare ?i puternic, care cu vârful ajungea pâna la cer ?i se vedea pâna la capatul pamântului,
Dan 4:21  Cu frunzi? frumos, cu rod mult ?i din care se hraneau to?i, sub care se adaposteau fiarele câmpului, iar în ramurile lui faceau cuiburi pasarile cerului,
Dan 4:22  Acela e?ti tu, o, rege, tu, care te-ai marit ?i te-ai facut puternic, ai crescut ?i ai ajuns pâna la ceruri, ai stapânirea ta pâna la marginea pamântului.
Dan 4:23  Iar ca a vazut regele un înger, un sfânt, coborându-se din cer ?i zicând: Doborâ?i copacul ?i nimici?i-l, dar butucul ?i radacinile lui lasa?i-le în pamânt ?i în legaturi de fier ?i de arama, în iarba pamântului, ?i de roua cerului sa fie udat ?i cu animalele câmpului sa fie parta? pâna ce vor trece peste el ?apte ani,
Dan 4:24  Aceasta înseamna, o, rege, ca hotarârea Celui Preaînalt se va împlini peste stapânul meu regele,
Dan 4:25  Ca tu vei fi alungat dintre oameni ?i vei locui împreuna cu animalele câmpului ?i vei mânca iarba ?i din roua cerului vei fi udat ?i vor trece peste tine ?apte ani, pâna ce tu vei cunoa?te ca Cel Preaînalt are stapânirea peste împara?ia oamenilor ?i o da cui voie?te.
Dan 4:26  ?i daca a poruncit sa lase butucul ?i radacinile copacului, înseamna ca regatul tau va fi ocrotit pentru tine îndata ce tu vei recunoa?te ca Cerul are stapânirea.
Dan 4:27  De aceea, o, rege, placut sa-?i fie sfatul meu înaintea ta: Rascumpara pacatele tale prin fapte de dreptate ?i nedrepta?ile tale prin mila catre cei saraci, daca vrei ca bunastarea în care te afli sa dainuiasca".
Dan 4:28  Totul s-a împlinit cu regele Nabucodonosor.
Dan 4:29  Dupa douasprezece luni, când regele Nabucodonosor se plimba în palatul regal din Babilon,
Dan 4:30  A prins a grai zicând: "Oare nu este acesta Babilonul cel mare pe care l-am cladit eu întru taria puterii mele ?i spre cinstea stralucirii mele, ca re?edin?a regala?"
Dan 4:31  Pe când cuvântul era înca în gura regelui, un glas s-a coborât din cer: "?ie, rege Nabucodonosor, ?i se spune: Regatul s-a luat de la tine.
Dan 4:32  ?i dintre oameni vei fi izgonit, vei locui cu animalele câmpului ?i vei pa?te iarba ?i vor trece ?apte ani peste tine, pâna ce vei recunoa?te ca Cel Preaînalt are putere peste împara?ia oamenilor ?i ca o da cui voie?te!"
Dan 4:33  Îndata s-a împlinit cuvântul asupra lui Nabucodonosor, caci a fost alungat dintre oameni ?i a mâncat iarba ca animalele ?i trupul lui era udat de roua pâna când parul i-a crescut ca penele vulturilor ?i unghiile ca ghiarele pasarilor.
Dan 4:34  "?i dupa trecerea acestui timp, eu Nabucodonosor, am ridicat ochii mei la cer ?i mintea mi-a venit din nou ?i am binecuvântat pe Cel Preaînalt ?i Celui ve?nic viu l-am adus lauda ?i preamarire, ca puterea Lui este putere ve?nica, iar împara?ia Lui din neam în neam.
Dan 4:35  To?i locuitorii pamântului sînt socoti?i ca o nimica ?i El face ce voie?te cu o?tirea cereasca ?i cu locuitorii pamântului ?i nimeni nu poate sa-L împiedice la lucrul Lui ?i sa-I zica: "Ce faci Tu?"
Dan 4:36  În acela?i timp mi-a venit mintea la loc ?i, spre gloria regatului meu, mi-a venit iara?i mare?ia ?i stralucirea ?i sfetnicii mei ?i dregatorii cei mari m-au chemat ?i regatul mi-a fost dat în stapânire, iar puterea mea a crescut ?i mai mult.
Dan 4:37  Acum, eu, Nabucodonosor, laud, înal? ?i preamaresc pe Împaratul cerului; toate faptele Lui sunt adevarate ?i caile Lui drepte, iar pe cei ce umbla mândri poate sa-i smereasca!"
Dan 5:1  Regele Bel?a?ar a facut un mare ospa? pentru o mie din dregatorii sai ?i în fa?a celor o mie a baut vin.
Dan 5:2  Bel?a?ar când era în toiul ospa?ului, la bautul vinului, a poruncit sa aduca vasele de aur ?i de argint pe care Nabucodonosor, tatal sau, le luase din templul din Ierusalim, ca regele sa bea vin din ele, împreuna cu dregatorii sai, femeile sale ?i concubinele sale.
Dan 5:3  Atunci au fost aduse vasele de aur ?i de argint care fusesera luate din templul lui Dumnezeu din Ierusalim ?i au baut din ele regele ?i dregatorii sai, femeile sale ?i concubinele sale.
Dan 5:4  Ei au baut vin ?i au preamarit pe dumnezeii de aur, de argint, de arama, de fier, de lemn ?i de piatra.
Dan 5:5  În aceea?i clipa au ie?it degetele unei mâini de om, care au scris în fa?a sfe?nicului celui mare pe tencuiala peretelui palatului regal, ?i regele a vazut vârful degetelor mâinii care scria.
Dan 5:6  Atunci fa?a regelui s-a îngalbenit ?i gândurile lui s-au tulburat; încheieturile coapselor sale au slabit, iar genunchii i se izbeau unul de altul neîncetat.
Dan 5:7  Regele a început sa strige din toate puterile sa i se aduca prezicatorii, caldeii ?i tâlcuitorii de semne. Atunci el a prins a grai ?i a zis tuturor în?elep?ilor din Babilon: "Oricine va citi scrisul acesta ?i îmi va arata tâlcuirea lui va fi îmbracat în ve?mânt de purpura, lan? de aur i se va pune împrejurul gâtului lui ?i va cârmui ca al treilea în regatul meu!"
Dan 5:8  Atunci au venit to?i în?elep?ii regelui, dar nu au putut citi scrisul, nici sa-i faca cunoscut în?elesul lui.
Dan 5:9  Regele Bel?a?ar s-a înspaimântat foarte, fa?a lui s-a îngalbenit, iar dregatorii lui au ramas înmarmuri?i.
Dan 5:10  Regina, auzind strigatul regelui ?i al dregatorilor, s-a dus in camara de ospa?. Ea a început a grai ?i a zis: "O, rege, sa traie?ti în veac! Gândurile tale sa nu te înspaimânte, ?i chipul fe?ei tale sa nu se schimbe!
Dan 5:11  În regatul tau se afla un om care are în el Duhul lui Dumnezeu celui Sfânt ?i în vremea domniei tatalui tau a fost descoperita în el lumina, pricepere ?i în?elepciune, ca în?elepciunea dumnezeiasca, iar regele Nabucodonosor, tatal tau, l-a pus mai-marele tâlcuitorilor ele semne, al caldeilor, al cititorilor în stele ?i al în?elep?ilor.
Dan 5:12  Din pricina ca s-a descoperit în Daniel un duh înalt, o ?tiin?a ?i o pricepere de a tâlcui visele, de a dezlega lucrurile greu de în?eles ?i de a descoperi tainele, pentru aceasta regele i-a dat numele de Belt?a?ar. Deci cheama pe Daniel, ?i el î?i va arata tâlcuirea".
Dan 5:13  Daniel a fost adus înaintea regelui. Regele a zis atunci lui Daniel: "Tu e?ti Daniel, cel dintre robii iudei pe care i-a adus tatal meu din Iuda?
Dan 5:14  Am auzit ca în tine este Duhul lui Dumnezeu ?i ca în tine se afla lumina, pricepere ?i în?elepciune fara seaman.
Dan 5:15  Acum au fost adu?i la mine în?elep?ii ?i prezicatorii ca sa-mi citeasca scrisul acesta ?i sa-mi faca cunoscuta tâlcuirea lui, dar nu au fost în stare sa-mi spuna tâlcuirea acestor cuvinte.
Dan 5:16  ?i eu am auzit despre tine ca tu po?i sa tâlcuie?ti visele ?i sa dezlegi cele tainice. Acum, daca tu e?ti în stare sa cite?ti scrisul ?i sa-mi faci cunoscuta tâlcuirea lui, vei fi îmbracat în ve?mânt de purpura ?i lan? de aur vei avea împrejurul gâtului tau ?i vei cârmui ca al treilea în regatul meu".
Dan 5:17  Atunci Daniel a început sa vorbeasca ?i a grait regelui: "Darurile tale po?i sa le pastrezi pentru tine, iar lucrurile de pre? da-le altora; caci eu voi citi regelui scrisul ?i îi voi face cunoscuta tâlcuirea lui.
Dan 5:18  O, rege! Dumnezeu cel Preaînalt a dat lui Nabucodonosor, tatal tau, regatul, marirea, cinstea ?i stralucirea.
Dan 5:19  Iar din pricina puterii pe care El i-o daduse, toate popoarele, neamurile ?i limbile erau înfrico?ate ?i tremurau înaintea lui; el omora pe cine voia ?i lasa în via?a pe cine voia, înal?a pe cine voia ?i cobora pe cine voia.
Dan 5:20  ?i pentru ca inima lui se trufise ?i duhul lui se împietrise pâna la mândrie, a fost coborât de pe scaunul regatului sau ?i vrednicia lui i-a fost luata;
Dan 5:21  ?i a fost izgonit din neamul omenesc, iar inima i s-a facut asemenea dobitoacelor, ?i a locuit cu asinii salbatici, mâncând iarba ca boii ?i ?i-a udat trupul din roua cerului pâna a recunoscut ca Dumnezeu cel Preaînalt are putere peste împara?ia oamenilor ?i a?aza peste ea pe cine vrea.
Dan 5:22  ?i tu, fiul sau, Bel?a?ar, tu nu e?ti smerit cu inima, macar ca tu ?tii toate acestea.
Dan 5:23  ?i te-ai ridicat împotriva Stapânului cerului ?i ai adus vasele templului Sau înaintea ta, ?i ai baut vin din ele, tu ?i dregatorii tai, femeile tale ?i concubinele tale, ?i ai preamarit dumnezei de argint, de aur, de arama, de fier, de lemn ?i de piatra, dumnezei care nu vad, nici nu aud ?i nici nu cunosc nimic, iar pe Dumnezeul în mâna Caruia este suflarea ta ?i toate caile tale, nu L-ai cinstit.
Dan 5:24  Atunci a trimis El vârful mâinii care a scris aceste cuvinte.
Dan 5:25  Iata inscrip?ia care a fost scrisa: Mene, mene, techel ufarsin.
Dan 5:26  Aceasta este tâlcuirea cuvântului mene: Dumnezeu a numarat zilele regatului tau ?i i-a pus capat.
Dan 5:27  Techel: l-a cântarit în cântar ?i l-a gasit u?or.
Dan 5:28  Peres: a împar?it regatul tau ?i 1-a dat Mezilor ?i Per?ilor".
Dan 5:29  Atunci a poruncit Bel?a?ar ?i au îmbracat pe Daniel în ve?mânt de purpura ?i i-au pus lan? de aur la gâtul lui ?i au dat de veste ca el va cârmui ca al treilea în împara?ie.
Dan 5:30  Chiar în noaptea aceea a fost omorât Bel?a?ar, împaratul Caldeilor.
Dan 5:31  ?i Darius Medul a ajuns rege când era în vârsta de aproape 62 de ani.
Dan 6:1  ?i i-a placut lui Darius sa puna peste regatul lui 120 de satrapi, care sa poarte de grija în tot regatul,
Dan 6:2  Iar în fruntea lor, trei dregatori dintre care unul era Daniel, ?i ace?ti satrapi trebuia sa le dea lor socoteala, astfel ca regele sa nu fie pagubit.
Dan 6:3  Însa Daniel era mai presus decât to?i dregatorii ?i satrapii, fiindca în el era un duh înalt ?i regele î?i pusese în gând sa-l puna mai mare peste tot regatul.
Dan 6:4  Atunci dregatorii ?i satrapii s-au trudit sa gaseasca lui Daniel vreo pricina din partea cârmuirii regatului, dar n-au putut sa-i afle nici o pricina sau lucru rau, caci el era credincios ?i nici o trecere cu vederea sau gre?eala nu i s-a putut pune în seama.
Dan 6:5  Atunci oamenii ace?tia au zis: "Daca nu-i gasim lui Daniel nici o pricina, cu toate acestea îi vom afla lui una, în legea Dumnezeului lui".
Dan 6:6  Dregatorii ?i satrapii ace?tia s-au dus atunci în graba la rege ?i a?a i-au grait: "O, rege Darius, sa traie?ti în veac!
Dan 6:7  To?i marii dregatori ai regatului, marii cârmuitori, satrapii, sfetnicii ?i guvernatorii s-au sfatuit laolalta ca regele sa dea o porunca ?i sa se rânduiasca oprirea ca oricine s-ar ruga vreme de treizeci de zile altui dumnezeu ?i om în afara de tine, rege, sa fie aruncat în groapa cu lei.
Dan 6:8  Acum, o, rege, fa cunoscuta oprirea ?i da porunca scrisa, care, potrivit legii Mezilor ?i Per?ilor, nu se mai poate schimba".
Dan 6:9  A?adar, regele Darius a dat oprire scrisa ?i porunca.
Dan 6:10  Îndata ce Daniel a aflat ca o porunca a fost data, a intrat în casa sa, care avea în camara de sus fereastra deschisa înspre Ierusalim ?i în fiecare zi îngenunchea de trei ori, s-a rugat ?i a laudat pe Dumnezeu, cum facea ?i mai înainte.
Dan 6:11  Atunci barba?ii aceia au venit în numar mare ?i au aflat pe Daniel rugându-se ?i cerând mila lui Dumnezeu.
Dan 6:12  Apoi s-au apropiat ?i au grait înaintea regelui privitor la porunca regala: "Oare n-ai poruncit tu ca oricine s-ar ruga timp de treizeci de zile la oricare alt dumnezeu sau om, în afara de tine, rege, sa fie aruncat într-o groapa cu lei?" Raspuns-a regele ?i a zis: "Lucrul ramâne hotarât ?i, dupa legile Mezilor ?i Per?ilor, nu se poate schimba".
Dan 6:13  Atunci au raspuns ei regelui ?i au zis: "Daniel, cel dintre robii iudei, nu a luat în seama porunca ta, rege, nici nu s-a îngrijit de oprirea ta, ci de trei ori pe zi î?i face rugaciunea".
Dan 6:14  Când a auzit regele acestea, s-a tulburat foarte ?i ?i-a îndreptat gândul spre Daniel, cum ar putea sa-l scape, ?i pâna la apusul soarelui s-a straduit ca sa-l scoata.
Dan 6:15  În urma, oamenii aceia au intrat în graba la rege ?i i-au zis: "?tiut sa-?i fie, o, rege, ca, dupa legea Mezilor ?i a Per?ilor, orice porunca sau oprire data de rege nu se mai poate schimba".
Dan 6:16  Atunci regele a dat porunca sa aduca pe Daniel ?i 1-a aruncat în groapa cu lei. Dupa acestea regele a prins a grai ?i a zis lui Daniel: "Dumnezeul tau pe Care tu Îl cinste?ti fara încetare, Acela te va scapa!"
Dan 6:17  Apoi s-a adus o piatra care a fost pusa peste gura gropii, iar regele a pecetluit-o cu inelul sau ?i cu inelul dregatorilor sai, a?a ca nimic sa nu se schimbe cu privire la Daniel.
Dan 6:18  Pe urma, împaratul s-a dus în palatul sau ?i a petrecut noaptea în post ?i nu au adus lânga el concubine, iar somnul nu l-a mai prins.
Dan 6:19  Apoi regele s-a sculat dis-de-diminea?a, în revarsat de zori ?i a venit în graba la groapa cu lei.
Dan 6:20  ?i când s-a apropiat de groapa, a strigat pe Daniel cu glas tare. Atunci regele a prins a grai ?i a zis lui Daniel: "Daniel, slujitorul Dumnezeului celui viu, Dumnezeul tau, Caruia te închini neîncetat, oare a putut sa te scape de lei?"
Dan 6:21  Apoi Daniel a vorbit cu regele: "O, rege, în veci sa traie?ti!
Dan 6:22  Dumnezeu a trimis pe îngerul Sau ?i a astupat gura leilor, ?i ei nu mi-au facut nici un rau, pentru ca am fost gasit nevinovat înaintea Lui, precum ?i în fa?a ta, rege, n-am facut nici un rau!"
Dan 6:23  Regele s-a bucurat foarte ?i a poruncit sa scoata pe Daniel din groapa ?i Daniel a fost scos din groapa ?i nici o rana nu i-a fost gasita, caci nadajduise în Dumnezeul lui.
Dan 6:24  Atunci a poruncit regele sa aduca pe barba?ii aceia care defaimasera pe Daniel ?i au fost arunca?i în groapa cu lei, ei, fiii lor ?i femeile lor, ?i nici nu au ajuns bine în fundul gropii, ca leii s-au ?i napustit asupra lor ?i le-au sfarâmat toate oasele.
Dan 6:25  Regele Darius a scris la toate popoarele, neamurile ?i limbile care locuiesc peste tot pamântul: "Pacea voastra sa sporeasca!
Dan 6:26  Porunca iese de la mine ca în tot cuprinsul regatului meu sa se teama ?i sa tremure lumea înaintea Dumnezeului lui Daniel, ca El este Dumnezeul cel viu, Care ramâne în veci ?i împara?ia lui nu se va nimici, iar stapânirea Lui nu va avea sfâr?it.
Dan 6:27  El poate sa scape ?i sa libereze, face semne ?i minuni în cer ?i pe pamânt; El a scapat pe Daniel din ghearele leilor".
Dan 6:28  ?i Daniel se afla într-o stare fericita în regatul lui Darius ?i în regatul lui Cirus, regele Per?ilor.
Dan 7:1  În anul dintâi al lui Bel?a?ar, regele Babilonului, Daniel a visat un vis, ?i vedeniile pe care le-a avut când era culcat în patul lui îl înfrico?ara. Atunci el a scris visul ?i a povestit ceea ce era mai de seama dintre fapte.
Dan 7:2  Daniel a început a grai, zicând: "Vazut-am în vedenia mea din timpul nop?ii cum cele patru vânturi ale cerului au sfredelit marea cea necuprinsa.
Dan 7:3  ?i patru fiare uria?e au ie?it din mare, una mai deosebita decât alta.
Dan 7:4  Cea dintâi semana cu un leu ?i avea aripi de vultur. M-am uitat la ea pâna ce aripile i-au fost smulse ?i a fost ridicata de pe pamânt ?i pusa pe picioare ca un om ?i i s-a dat inima de an.
Dan 7:5  ?i iata o a doua fiara, cu înfa?i?are de urs, stând într-o râna, cu trei coaste în gura, între din?i, ?i a?a i s-a poruncit: "Scoala-te! Manânca multa carne!"
Dan 7:6  Apoi m-am uitat din nou ?i iata o alta fiara, asemenea unui leopard, având pe spate patru aripi de pasare; ?i fiara avea patru capete, ?i i s-a dat putere.
Dan 7:7  În urma am privit în vedeniile mele de noapte ?i iata o a patra fiara înspaimântatoare ?i înfrico?atoare ?i nespus de puternica. Ea avea din?i mari de fier ?i gheare de arama; mânca ?i sfarâma, iar rama?i?a o calca in picioare. Ea se deosebea de toate celelalte fiare de mai înainte ?i avea zece coarne.
Dan 7:8  M-am uitat cu luare aminte la coarne, ?i iata un alt corn mic cre?tea între ele, ?i trei din coarnele cele dintâi au fost smulse de el. ?i iata ca acest corn avea ochi ca ochii de om ?i gura care graia lucruri mari.
Dan 7:9  Am privit pâna când au fost a?ezate scaune, ?i S-a a?ezat Cel vechi de zile; îmbracamintea Lui era alba ca zapada, iar parul capului Sau curat ca lâna; tronul Sau, flacari de foc; ro?ile lui, foc arzator.
Dan 7:10  Un râu de foc se varsa ?i ie?ea din el; mii de mii Îi slujeau ?i miriade de miriade stateau înaintea Lui! Judecatorul S-a a?ezat ?i car?ile au fost deschise.
Dan 7:11  Eu ma uitam mereu, din pricina multelor vorbe pe care cornul cel mare le graia. Am privit pâna când fiara a fost omorâta ?i trupul ei nimicit ?i dat focului.
Dan 7:12  Dar ?i celorlalte fiare li s-a luat stapânirea, ?i lungimea vie?ii lor a fost hotarâta pâna la o vreme ?i un anumit timp.
Dan 7:13  Am privit în vedenia de noapte, ?i iata pe norii cerului venea cineva ca Fiul Omului ?i El a înaintat pâna la Cel vechi de zile, ?i a fost dus în fa?a Lui.
Dan 7:14  ?i Lui I s-a dat stapânirea, slava ?i împara?ia, ?i toate popoarele, neamurile ?i limbile Îi slujeau Lui. Stapânirea Lui este ve?nica, stapânire care nu va trece, iar împara?ia Lui nu va fi nimicita niciodata.
Dan 7:15  Pentru aceasta, eu, Daniel, am fost tulburat cu duhul meu ?i vedeniile pe care le-am avut ma înspaimântau.
Dan 7:16  M-am apropiat atunci de unul din cei de fa?a ?i l-am rugat sa-mi spuna adevarul privitor la toate acestea. ?i el mi-a vorbit ?i mi-a facut cunoscut în?elesul acestor lucruri.
Dan 7:17  "Aceste fiare, patru la numar, înseamna ca patru regi se vor ridica pe pamânt,
Dan 7:18  ?i sfin?ii Celui Preaînalt vor primi regatul ?i îl vor ?ine în stapânire în veci ?i în vecii vecilor.
Dan 7:19  Dupa aceasta l-am rugat sa-mi spuna adevarul despre fiara a patra, care se deosebea de toate celelalte ?i care era afara din cale de înspaimântatoare, cu din?i de fier ?i cu gheare de arama ?i care mânca, sfarâma, iar ceea ce ramânea calca în picioare;
Dan 7:20  ?i despre cele zece coarne care erau pe capul sau ?i despre celalalt care cre?tea ?i înaintea caruia au cazut cele trei ?i avea ochi ?i gura care graia lucruri mari ?i care era mult mai mare decât celelalte.
Dan 7:21  M-am uitat, ?i cornul acela purta razboi cu cei sfin?i ?i i-a biruit,
Dan 7:22  Pâna ce a venit Cel vechi de zile ?i a facut dreptate sfin?ilor Celui Preaînalt, pâna ce s-a împlinit vremea ?i împara?ia a ajuns sub stapânirea sfin?ilor.
Dan 7:23  El a raspuns astfel: "Fiara a patra înseamna ca un al patrulea rege va fi pe pamânt, care se va deosebi de toate celelalte regate, care va mânca tot pamântul, îl va calca în picioare ?i îl va zdrobi.
Dan 7:24  ?i cele zece coarne înseamna ca din acest regat se vor ridica zece regi, ?i un altul se va scula dupa ei; el se va deosebi de cei dinaintea lui ?i va doborî la pamânt trei regi.
Dan 7:25  ?i va grai cuvinte de defaimare împotriva Celui Preaînalt ?i va asupri pe sfin?ii Celui Preaînalt, ?i î?i va pune în gând sa schimbe sarbatorile ?i legea, ?i ei vor fi da?i în mâna lui o vreme ?i vremuri ?i jumatate de vreme.
Dan 7:26  ?i judecata se va face ?i i se va lua stapânirea, ca sa-l nimiceasca ?i sa-l prabu?easca pentru totdeauna.
Dan 7:27  Iar regatul ?i stapânirea ?i marirea regilor de sub ceruri se vor da poporului sfin?ilor Celui Preaînalt; împara?ia Lui este împara?ie ve?nica ?i toate stapânirile Îi vor sluji Lui ?i pe El Îl vor asculta".
Dan 7:28  Iata sfâr?itul vorbirii mele cu el. Pe mine, Daniel, gândurile mele m-au înfrico?at foarte ?i fa?a mi s-a schimbat ?i am pastrat cuvântul în inima mea.
Dan 8:1  În anul al treilea al domniei regelui Bel?a?ar mi s-a aratat mie, lui Daniel, o vedenie, afara de cele ce mi se aratasera la început.
Dan 8:2  ?i m-am uitat în vedenie ?i, când priveam, parca eram în capitala Suza, care este în ?ara Elamului ?i, stând cu privirea a?intita, eram pe fluviul Ulai.
Dan 8:3  ?i am ridicat ochii mei ?i m-am uitat ?i iata un berbec cu doua coarne stând în picioare în fa?a fluviului ?i coarnele lui erau lungi, iar unul mai lung decât celalalt ?i cel mai lung cre?tea cel din urma.
Dan 8:4  Am vazut berbecul lovind cu coarnele la apus, la miazanoapte ?i la miazazi ?i nici o fiara nu-i putea sta împotriva ?i nimeni nu scapa de asuprirea lui. El facea ce voia ?i cre?tea.
Dan 8:5  ?i m-am uitat cu luare aminte ?i iata un ?ap venea de la apus pe deasupra fe?ei pamântului, fara ca sa-l atinga. ?i ?apul avea un corn între ochi, care corn se putea zari.
Dan 8:6  ?i a venit pâna la berbecul cel cu doua coarne pe care l-am vazut stând în fa?a fluviului ?i s-a napustit spre el cu toata taria puterii lui.
Dan 8:7  ?i l-am vazut cum s-a apropiat de berbec, s-a întarâtat împotriva lui ?i a lovit berbecul ?i i-a sfarâmat cele doua coarne, iar berbecul nu mai avea putere sa i se împotriveasca; ?i l-a aruncat la pamânt ?i 1-a calcat în picioare ?i nimeni n-a scapat pe berbec.
Dan 8:8  ?i ?apul a crescut foarte ?i când a ajuns puternic, cornul cel mare s-a sfarâmat ?i am zarit patru coarne crescând în locul lor, spre cele patru vânturi ale cerului.
Dan 8:9  ?i din unul dintre ele a ie?it un corn mic, care a crescut afara din cale, catre miazazi, catre rasarit ?i catre ?ara stralucirii.
Dan 8:10  ?i el s-a înal?at pâna catre o?tirea cereasca ?i a doborât la pamânt din o?tire ?i din stele ?i le-a calcat în picioare.
Dan 8:11  ?i a mai crescut pâna la mai-marele o?tirii ?i i-a luat jertfa de fiecare zi ?i i-a rasturnat locul templului sau.
Dan 8:12  ?i peste jertfa de fiecare zi el a pus nelegiuirea ?i a aruncat adevarul la pamânt, ?i el a facut ?i a izbutit.
Dan 8:13  ?i am auzit un sfânt care graia, ?i un alt sfânt a zis catre cel ce graia: "Pâna când va dura vedenia ?i jertfa de fiecare zi nu se va mai aduce ?i un pacat al pustiirii va fi pus în loc ?i templul ?i o?tirea vor fi calcate în picioare?"
Dan 8:14  Atunci el i-a raspuns: "Pâna la doua mii trei sute de seri ?i de dimine?i; dupa aceasta templul î?i va avea din nou rostul lui".
Dan 8:15  ?i când eu, Daniel, am vazut vedenia ?i m-am straduit sa o în?eleg, iata ca atunci a stat cineva înaintea mea cu chip de om.
Dan 8:16  Atunci am auzit un glas de om deasupra fluviului Ulai, glas care striga ?i spunea: "Gavriile, tâlcuie?te celui de acolo vedenia!"
Dan 8:17  ?i el a venit unde eram eu ?i, când se apropia, m-am înspaimântat ?i am cazut cu fa?a la pamânt. ?i el mi-a grait: "Ia aminte, fiul omului, caci vedenia este pentru a arata sfâr?itul veacurilor!"
Dan 8:18  ?i când vorbea cu mine, stam înmarmurit cu fa?a la pamânt; atunci el s-a atins de mine ?i m-a ridicat în picioare in locul în care ma aflam,
Dan 8:19  ?i mi-a spus: "Iata, î?i voi face cunoscut ceea ce se va întâmpla în vremea din urma a mâniei lui Dumnezeu; ca pentru vremea cea din urma este vedenia.
Dan 8:20  Berbecul cu doua coarne, pe care tu îl vezi, înseamna regii Mediei ?i ai Persiei.
Dan 8:21  Iar ?apul este regele Greciei ?i cornul cel mare, care este între ochi, este regele cel dintâi.
Dan 8:22  ?i daca el a fost sfarâmat, patru s-au ridicat în locul lui; înseamna ca patru regate se vor ridica din neamul lui, dar fara sa aiba puterea lui.
Dan 8:23  La sfâr?itul stapânirii lor, la vremea covâr?irii pacatelor lor, se va ridica un rege cu chip seme? ?i iste? în lucrurile ascunse.
Dan 8:24  ?i stapânirea lui va cre?te în putere - dar nu prin puterea lui însu?i - ?i va face pustiiri uria?e ?i în orice lucru va izbuti ?i va prabu?i pe cei tari ?i pe poporul sfin?ilor.
Dan 8:25  Din pricina iste?imii lui, va izbuti în?elaciunea în mâna lui ?i se va seme?i în inima sa ?i în plina vreme de pace va doborî pe mul?i. ?i se va ridica împotriva Regelui regilor, dar va fi aruncat la pamânt nu de mâna omeneasca.
Dan 8:26  Iar vedenia despre seri ?i despre dimine?i, care a fost spusa, este adevarata; tu însa pecetluie?te vedenia, ca se va întâmpla dupa multe zile".
Dan 8:27  ?i eu, Daniel, am fast istovit de puteri ?i am cazut bolnav câteva zile. Dar m-am sculat ?i am avut grija de lucrurile regelui. Am ramas uimit de cele ce am vazut ?i n-am putut sa le în?eleg.
Dan 9:1  În anul întâi al lui Darius, fiul lui Aha?vero? (Artaxerxe), din neamul Mezilor, care a domnit peste regatul Caldeilor,
Dan 9:2  În anul întâi al domniei lui - eu, Daniel, am citit în car?i numarul de ?aptezeci de ani, pentru care a fost cuvântul Domnului catre proorocul Ieremia, ani care trebuia sa se împlineasca de la darâmarea Ierusalimului.
Dan 9:3  ?i mi-am îndreptat fa?a catre Domnul Dumnezeu, staruind în rugaciune ?i în rugi fierbin?i, cu post, sac ?i cenu?a.
Dan 9:4  ?i m-am rugat Domnului Dumnezeu ?i m-am marturisit ?i am zis: "O, Doamne, Dumnezeule cel mare ?i minunat, Care paze?ti legamântul ?i îndrumarea pentru cei ce Te iubesc pe Tine ?i iau aminte la poruncile Tale!
Dan 9:5  Pacatuit-am, faradelege am facut, ca ?i cei nelegiui?i ne-am purtat, rasculatu-ne-am ?i ne-am departat de la poruncile ?i de la legile Tale.
Dan 9:6  ?i nu am ascultat de slujitorii Tai prooroci, care ne-au grait în numele Tau: catre regii no?tri, catre mai-marii no?tri, parin?ilor no?tri ?i la tot poporul ?arii.
Dan 9:7  A Ta este, Doamne, dreptatea, iar a noastra ru?inarea fe?elor noastre, precum se arata astazi oamenilor din Iuda ?i locuitorilor din Ierusalim ?i la tot Israelul, cei de aproape ?i cei de departe, în toate ?arile în care Tu i-ai izgonit din pricina faradelegilor ce le-au savâr?it împotriva Ta.
Dan 9:8  Doamne Dumnezeule, a noastra este ru?inarea fe?elor, a regilor no?tri, a mai-marilor no?tri ?i a parin?ilor no?tri, caci noi am pacatuit ?ie;
Dan 9:9  A Domnului Dumnezeului nostru este milostivirea ?i îndurarea. Razvratitu-ne-am împotriva Lui.
Dan 9:10  ?i nu am ascultat de glasul Domnului Dumnezeului nostru ca sa umblam în legea Lui, pe care ne-a dat-o noua prin mâna slujitorilor Sai profe?i.
Dan 9:11  ?i tot Israelul a calcat legea Ta ?i s-a departat, ca sa nu mai auda glasul Tau. Varsatu-s-a peste noi blestemul ?i juramântul scris în legea lui Moise, slujitorul lui Dumnezeu, caci am pacatuit împotriva Ta.
Dan 9:12  ?i a adeverit cuvintele Sale pe care le-a grait catre noi ?i catre judecatorii no?tri, care au cârmuit peste noi, ca a voit sa abata peste noi stra?nic prapad, ce nu s-a mai întâmplat niciodata sub cer, asemenea celui din Ierusalim.
Dan 9:13  Precum este scris în legea lui Moise, toata aceasta nenorocire s-a napustit asupra noastra, dar n-am îmbunat fa?a Domnului Dumnezeului nostru, întorcându-ne de la nelegiuirile noastre ?i luând aminte la adevarul Sau.
Dan 9:14  Gândit-a îndelung Domnul asupra nenorocirii pe care a abatut-o peste noi, ca drept este Domnul Dumnezeul nostru în toate faptele pe care le-a facut, dar noi n-am ascultat de glasul Lui.
Dan 9:15  ?i acum, Doamne Dumnezeul nostru, Tu Care ai scos pe poporul Tau din ?ara Egiptului cu mâna tare ?i Te-ai facut vestit pâna în ziua de astazi, pacatuit-am, faradelege am facut.
Dan 9:16  O, Doamne! Întoarca-se, dupa milostivirile Tale, toata mânia ?i toata vapaia urgiei Tale de la cetatea Ierusalimului, de la muntele cel sfânt al Tau! Ca, pentru pacatele noastre ?i pentru faradelegile parin?ilor no?tri, Ierusalimul ?i poporul Tau au ajuns de ocara pentru to?i vecinii no?tri.
Dan 9:17  Acum asculta, Dumnezeul nostru, rugaciunea slujitorului Tau ?i ruga fierbinte ?i lumineaza fa?a Ta spre templul Tau pustiit, pentru numele Tau, Doamne!
Dan 9:18  Pleaca, Dumnezeul meu, urechea Ta ?i auzi, deschide ochii Tai ?i vezi mâhnirea noastra adânca ?i cetatea asupra careia se cheama numele Tau. Ca nu pentru faptele noastre drepte aducem înaintea Ta rugaciunile noastre cele fierbin?i, ci pentru milele Tale cele mari.
Dan 9:19  O, Doamne, asculta! O, Doamne, iarta! O, Doamne, ia aminte ?i lucreaza! Nu întârzia pentru numele Tau, Dumnezeul meu; ca numele Tau îl poarta cetatea ?i poporul Tau!"
Dan 9:20  ?i în vreme ce graiam ?i ma rugam ?i marturiseam pacatul meu ?i pacatul poporului meu Israel ?i cadeam cu ruga mea fierbinte înaintea Domnului Dumnezeului meu, pentru sfânt muntele Dumnezeului meu,
Dan 9:21  ?i pe când vorbeam în rugaciunea mea, iata un om, Gavriil, pe care l-am vazut în vedenia mea cea de la început, în zbor grabit, s-a apropiat de mine pe la vremea jertfei de seara.
Dan 9:22  ?i a venit ?i mi-a grait zicând: "Daniele, chiar acum am sosit ca sa-?i deschid mintea.
Dan 9:23  Când tu ai început sa te rogi, porunca mi-a fost data ?i eu am venit ca sa-?i vestesc, caci tu e?ti un om iubit de Dumnezeu. Ia aminte la cuvânt ?i în?elege vedenia!
Dan 9:24  ?aptezeci de saptamâni sunt hotarâte pentru poporul tau ?i pentru cetatea ta cea sfânta pâna ce faradelegea va trece peste margini ?i se va pecetlui pacatul ?i se va ispa?i nelegiuirea, pâna ce dreptatea cea ve?nica va veni, vedenia ?i proorocia se vor pecetlui ?i se va unge Sfântul Sfin?ilor.
Dan 9:25  Sa ?tii ?i sa în?elegi ca de la ie?irea poruncii pentru zidirea din nou a Ierusalimului ?i pâna la Cel-Uns - Cel-Vestit - sunt ?apte saptamâni ?i ?aizeci ?i doua de saptamâni; ?i din nou vor fi zidite pie?ele ?i zidul din afara, în vremuri de strâmtorare.
Dan 9:26  Iar dupa cele ?aizeci ?i doua de saptamâni, Cel-Uns va pieri fara sa se gaseasca vreo vina în El, iar poporul unui domn va veni ?i va darâma cetatea ?i templul. ?i sfâr?itul ceta?ii va veni prin potopul mâniei lui Dumnezeu ?i pâna la capat va fi razboi - prapadul cel hotarât.
Dan 9:27  ?i El va încheia un legamânt cu mul?i într-o saptamâna, iar la mijlocul saptamânii va înceta jertfa ?i prinosul ?i în templu va fi urâciunea pustiirii, pâna când pedeapsa nimicirii cea hotarâta se va varsa peste locul pustiirii".
Dan 10:1  În anul al treilea al lui Cirus, regele Per?ilor, i s-a descoperit un cuvânt lui Daniel - care se chema Belt?a?ar - ?i adevarat este cuvântul ?i el veste?te razboi mare. El a patruns cuvântul ?i a în?eles vedenia.
Dan 10:2  "În vremea aceea, eu, Daniel, am petrecut trei saptamâni de zile în jale.
Dan 10:3  Pâine buna n-am mâncat, carne ?i vin n-am pus în gura mea ?i cu miresme nu m-am uns, pâna ce nu s-au împlinit trei saptamâni de zile.
Dan 10:4  Dar în ziua a douazeci ?i patra a lunii întâi, eu, Daniel, ma aflam pe malul fluviului celui mare, adica Tigrul,
Dan 10:5  ?i mi-am ridicat ochii mei ?i iata un om îmbracat în ve?minte de in, iar coapsele lui încinse cu aur curat ?i de pre?;
Dan 10:6  Trupul lui ca ?i crisolitul ?i fa?a lui ca fulgerul, iar ochii lui ca flacarile de foc, bra?ele ?i picioarele lui straluceau ca arama lustruita ?i sunetul cuvintelor lui ca vuietul unei mul?imi.
Dan 10:7  ?i am vazut numai eu, Daniel, o vedenie, iar oamenii care erau cu mine nu au vazut vedenia; ci o mare spaima a cazut peste ei ?i au fugit sa se ascunda.
Dan 10:8  Atunci eu am ramas singur ?i am vazut aceasta mare vedenie ?i n-a ramas în mine putere, fa?a mea ?i-a schimbat înfa?i?area, stricându-se, ?i nu mai aveam vlaga.
Dan 10:9  ?i am auzit glasul cuvintelor lui ?i, la glasul cuvintelor lui, eu am cazut înmarmurit cu fa?a la pamânt.
Dan 10:10  ?i iata ca o mâna s-a atins de mine ?i m-a ridicat în genunchi ?i pe palmele mâinilor mele.
Dan 10:11  ?i a grait catre mine: "Daniele, om placut al lui Dumnezeu, ia aminte la cuvintele pe care ?i le spun ?ie ?i stai drept, ca acum sunt trimis catre tine". ?i pe când îmi graia cuvântul acesta, m-am sculat tremurând.
Dan 10:12  ?i a zis catre mine: "Nu te teme, Daniele, ca din ziua cea dintâi, de când ?i-ai sârguit inima ta ca sa în?elegi ?i sa te smere?ti înaintea Dumnezeului tau, cuvintele tale au fost auzite ?i eu am sosit din pricina cuvintelor tale.
Dan 10:13  ?i îngerul pazitor al Persiei mi-a stat împotriva douazeci ?i una de zile, dar iata ca Mihail, cel dintâi dintre îngerii pazitori, a venit în ajutorul meu ?i eu l-am lasat acolo la îngerul pazitor al regelui Per?ilor
Dan 10:14  ?i am venit ca sa-?i fac cunoscut ce se va întâmpla poporului tau la sfâr?itul zilelor; ca mai este o vedenie pentru zilele cele din urma".
Dan 10:15  ?i pe când graia cu mine astfel de cuvinte, mi-am întors privirea spre pamânt ?i am ramas ca un mut.
Dan 10:16  ?i iata! Acela care avea înfa?i?area fiului omului s-a atins de buzele mele; atunci am deschis gura mea ?i am grait ?i am zis catre cel ce sta înaintea mea: "O, Stapânul meu! Din pricina acestei vedenii m-au cuprins zvârcoliri de durere ?i am ramas fara putere.
Dan 10:17  ?i cum poate un slujitor atât de mic al Domnului sau sa graiasca cu un Stapân atât de mare ca Tine! ?i de spaima îmi piere toata puterea ?i suflarea mi se opre?te".
Dan 10:18  Atunci s-a atins iara?i de mine acela care avea înfa?i?area Fiului Omului ?i mi-a dat tarie.
Dan 10:19  ?i mi-a zis: "Nu te teme, om placut al lui Dumnezeu! Pace ?ie! Fii tare ?i curajos!" ?i pe când graia cu mine, m-am sim?it întarit ?i am zis: "Spune, Stapâne, caci Tu m-ai întarit!"
Dan 10:20  Atunci El a zis: "?tii tu, oare, pentru ce am venit la tine? Acum Ma voi întoarce sa fac razboi cu îngerul pazitor al Persiei ?i, când Eu Ma voi duce, iata ca îngerul pazitor al Greciei va veni.
Dan 10:21  Î?i voi vesti ?ie ceea ce este scris în cartea adevarului. Nimeni nu poate sa Ma ajute mai bine la aceasta decât Mihail, îngerul vostru pazitor".
Dan 11:1  ?i eu în anul dintâi al lui Darius Medul stam lânga el ca sa-l ajut ?i sa-l întaresc.
Dan 11:2  ?i acum î?i fac cunoscut adevarul: Iata ca se vor scula înca trei regi în Persia, iar al patrulea va stapâni boga?ii mai mari decât to?i ?i, când va fi puternic prin boga?iile sale, va ridica pe to?i împotriva regatului Greciei.
Dan 11:3  ?i va ie?i la iveala un rege viteaz ?i va stapâni peste un regat puternic ?i va face numai ceea ce i se va parea bun.
Dan 11:4  Iar când va fi în culmea puterii sale, regatul lui se va prabu?i ?i se va împar?i dupa cele patru vânturi ale carului, fara ca sa ramâna urma?ilor lui ?i nici sa aiba putere întocmai ca mai înainte, ca regatul lui va fi sfâ?iat ?i se va împar?i la al?ii decât la aceia din neamul lui.
Dan 11:5  ?i regele de la miazazi va ajunge puternic, dar unul din capeteniile lui va fi mai puternic decât el ?i va domni, iar stapânirea lui va fi un regat puternic.
Dan 11:6  ?i dupa trecere de ani se vor uni ?i fiica regelui de la miazazi va veni catre regele de la miazanoapte, ca sa statorniceasca pacea. Dar ea nu va pastra taria bra?ului sau ?i nu va dainui nici el, nici bra?ul lui; ?i va fi data mor?ii, ea ?i înso?itorii ? ei ?i copiii ei ?i aliatul ei în acele vremuri.
Dan 11:7  Iar unul dintre odraslale din radacinile ei se va ridica ?i va porni împotriva o?tirii ?i va intra în cetatea cea întarita a regelui de la miazanoapte ?i va face cu ei ce va voi ?i va fi biruitor.
Dan 11:8  Chiar ?i dumnezeii lor, împreuna cu chipurile lor turnate, cu vasele lor de pre?, aur ?i argint, vor fi duse în robia Egiptului ?i el va fi mai puternic decât regele de la miazanoapte, ani de-a rândul.
Dan 11:9  El va navali în regatul regelui de la miazanoapte, apoi se va întoarce în ?ara sa.
Dan 11:10  ?i feciorul lui va pregati razboiul ?i va strânge o mare mul?ime de o?ti de lupta ?i va da navala peste el ?i îl va potopi ?i îl va cotropi ?i se va întoarce ?i va patrunde pâna la cetatea lui cea întarita.
Dan 11:11  Atunci regele de la miazazi va fi amarât foarte ?i va ie?i ?i va face razboi cu el - cu regele de la miazanoapte, care va ridica o mare o?tire; dar o?tirea va cadea în mâna regelui de la miazazi;
Dan 11:12  ?i o?tirea va fi nimicita; inima regelui se va îngâmfa; zeci de mii va doborî la pamânt, dar nu va fi mai puternic.
Dan 11:13  ?i înca o data regele de la miazanoapte va ridica o?tiri mai puternice decât cele dintâi ?i, dupa un rastimp de câ?iva ani, va da navala peste el, cu o mare o?tire ?i cu numeroasa calarime.
Dan 11:14  ?i în vremea aceea mul?i se vor scula împotriva regelui de la miazazi ?i oameni silnici din poporul tau se vor ridica, a?a ca sa se împlineasca vedenia, dar se vor poticni.
Dan 11:15  Iar regele de la miazanoapte va veni ?i va ridica întarituri ?i va cuprinde o cetate întarita, iar o?tirea de ajutor a regelui de la miazanoapte nu va putea sa ?ina piept ?i trupele sale vor fugi ?i nu va fi nici un chip de stat împotriva.
Dan 11:16  Cel care va porni împotriva lui va face dupa placul sau ?i nimeni nu i se va împotrivi, ?i se va opri în ?ara stralucirii ?i totul va fi în mâna lui.
Dan 11:17  ?i î?i va îndrepta privirea sa ia în stapânire întregul lui regat ?i va face o învoiala cu el ?i îi va da de so?ie pe o fiica a sa, ca sa aduca pieirea ?arii, dar aceasta nu se va întâmpla ?i nu va duce la izbânda.
Dan 11:18  ?i î?i va întoarce privirea spre insule ?i va cuprinde multe din ele, iar o capetenie va pune capat ocarii lui, fara putin?a de raspuns.
Dan 11:19  ?i î?i va întoarce fa?a spre întariturile ?arii sale, dar se va poticni, va cadea ?i va pieri.
Dan 11:20  ?i în locul lui va veni altul, care va trimite un strângator de dari în locul care este gloria regatului (Ierusalimul), dar în câteva zile va fi doborât nu prin mânie, nici prin razboi.
Dan 11:21  ?i în locul lui va veni un batjocoritor, care nu avusese nici un drept la vrednicia regala, ?i el va veni pa?nic ?i prin uneltiri se va face stapân pe regat.
Dan 11:22  ?i o?tirile de ajutor vor da înapoi înaintea lui ?i vor fi sfarâmate, de asemenea ?i o capetenie a legamântului.
Dan 11:23  ?i dupa împrietenirea cu el se va servi de vicle?ug ?i va porni ?i va birui cu pu?in popor.
Dan 11:24  ?i pe nea?teptate va veni în cele mai bogate ?inuturi ale ?arii ?i va face ceea ce n-au facut parin?ii lui ?i parin?ii parin?ilor lui; el le va împar?i cu risipa, prada ?i jaf ?i boga?ii; ?i va urzi planuri împotriva ceta?ilor întarite, numai pentru o vreme.
Dan 11:25  ?i î?i va îndrepta puterea ?i inima împotriva regelui de la miazazi, eu o?tire mare, iar el se va prinde în lupta cu oaste mare ?i puternica, dar nu va putea sa i se împotriveasca, ca se vor urzi uneltiri împotriva lui.
Dan 11:26  Cei ce manânca la masa cu el, îl vor prabu?i, iar oastea lui se va sfarâma ?i mul?i vor cadea lovi?i de moarte.
Dan 11:27  ?i cei doi regi vor pune la cale viclenii în inima lor ?i la masa î?i vor spune lucruri mincinoase, dar fara nici o izbânda, ca înca n-a venit sfâr?itul rânduit de Dumnezeu.
Dan 11:28  ?i se va duce cu mari averi în ?ara lui ?i inima lui va fi împotriva Legamântului cel sfânt; a?a va lucra ?i se va întoarce în ?ara lui.
Dan 11:29  La vremea hotarâta va navali din nou la miazazi ?i aceasta din urma batalie nu va fi ca batalia dintâi.
Dan 11:30  Corabii din Chitim vor veni împotriva lui; ?i el va pierde curajul, se va întoarce ?i se va întarâta împotriva legamântului sfânt ?i va lucra ?i se va învoi iara?i cu cei ce au parasit legamântul sfânt.
Dan 11:31  ?i o?ti trimise de el vor sta ?i vor pângari loca?ul sfânt ?i cetatea, iar jertfa de fiecare zi o var da de o parte ?i vor pune în loc urâciunea pustiirii.
Dan 11:32  ?i pe cei ce au savâr?it faradelegi împotriva legamântului îi va în?ela prin lingu?iri, iar poporul care cunoa?te pe Dumnezeul sau va ramâne statornic ?i îl va urma.
Dan 11:33  Cei mai în?elep?i vor înva?a pe cei mul?i, dar ei vor cadea un timp de sabie ?i foc, de temni?a ?i pustiire.
Dan 11:34  ?i în vremea caderii lor, vor primi pu?in ajutor ?i mul?i se vor uni cu ei, dar din fa?arnicie.
Dan 11:35  ?i printre în?elep?i, mul?i vor cadea ca sa se lamureasca, sa se cura?easca ?i sa se albeasca pâna la sfâr?itul vremii, ca mai este înca pâna la vremea rânduita.
Dan 11:36  ?i regele va face dupa placul sau ?i se va ridica ?i se va trufi împotriva oricarui dumnezeu ?i împotriva Dumnezeului dumnezeilor va spune lucruri nemaiauzite ?i va propa?i pâna ce sfâr?itul mâniei va veni, ca ceea ce este hotarât se va întâmpla.
Dan 11:37  ?i nu va lua aminte la dumnezeii parin?ilor lui ?i nici la dumnezeul placut femeilor ?i nu va baga în seama pe nici un alt dumnezeu caci el se va ridica deasupra tuturor.
Dan 11:38  Iar în locul lui va cinsti pe dumnezeul ceta?ilor, pe un dumnezeu pe care nu l-au cunoscut parin?ii lui; aceluia i se va închina cu aur, cu argint, cu pietre scumpe ?i lucruri de pre?.
Dan 11:39  El va lua ca aparatori ai ceta?ilor întarite pe poporul unui dumnezeu strain; pe cei care îl vor recunoa?te, el îi va cinsti mult, îi va pune stapâni peste mul?ime ?i le va împar?i pamânturi ca rasplata.
Dan 11:40  La sfâr?itul vremii se va razboi cu el regele cel de la miazazi ?i se va napusti împotriva lui regele cel de la miazanoapte, cu care de razboi, cu calare?i ?i cu multe corabii. El va veni în ?arile pe care le va cotropi ?i le va strabate.
Dan 11:41  ?i va veni ?i în ?ara stralucirii ?i zeci de mii se var prabu?i; ?i iata care vor scapa din mâna lui: Edomul, Moabul ?i restul fiilor lui Amon.
Dan 11:42  ?i va întinde mâna sa peste ?ari ?i ?ara Egiptului nu va scapa.
Dan 11:43  ?i va ajunge stapân peste comorile de aur ?i de argint ?i peste toate lucrurile pre?ioase ale Egiptului, iar Libienii ?i Etiopienii vor merge dupa el.
Dan 11:44  Dar zvonuri de la rasarit ?i de la miazanoapte vor veni sa-l înspaimânte ?i el va ie?i cu furie grozava ca sa prapadeasca ?i sa nimiceasca pe mul?i.
Dan 11:45  ?i el va înfige corturile palatului sau între mare ?i muntele cel sfânt ?i stralucit, apoi va veni sfâr?itul lui ?i nimeni nu-i va veni în ajutor!
Dan 12:1  ?i în vremea aceea se va scula Mihail, marele voievod care ocrote?te pe fiii poporului tau, ?i va fi vreme de strâmtorare cum n-a mai fost de când sunt popoarele ?i pâna în vremea de acum. Dar în vremea aceea, poporul tau va fi mântuit ?i anume oricine va fi gasit scris în carte.
Dan 12:2  ?i mul?i dintre cei care dorm în ?arâna pamântului se vor scula, unii la via?a ve?nica, iar al?ii spre ocara ?i ru?ine ve?nica.
Dan 12:3  ?i cei în?elep?i vor lumina ca stralucirea cerului ?i cei care vor fi îndrumat pe mul?i pe calea drepta?ii vor fi ca stelele în vecii vecilor.
Dan 12:4  Iar tu, Daniele, ?ine ascunse cuvintele ?i pecetluie?te cartea pâna la sfâr?itul vremii. Mul?i vor cerceta-o cu de-amanuntul ?i va cre?te ?tiin?a.
Dan 12:5  ?i eu, Daniel, m-am uitat, ?i iata al?i doi barba?i stând în picioare, unul pe un mal al fluviului, iar altul pe celalalt mal al fluviului.
Dan 12:6  ?i unul a zis celui ce era îmbracat în ve?minte de in ?i statea deasupra apelor fluviului: "Pe când se var sfâr?i aceste fapte minunate?"
Dan 12:7  ?i am ascultat pe barbatul cel îmbracat în ve?minte de in care statea deasupra apelor fluviului ?i el ?i-a ridicat dreapta ?i stânga lui catre ceruri ?i a jurat pe Cel ce este viu în veci: "Va mai ?ine o vreme; vremuri ?i jumatate de vreme, iar când se va ispravi de sfarâmat puterea poporului celui sfânt, atunci vor lua sfâr?it toate acestea".
Dan 12:8  ?i eu am auzit, dar nu am în?eles ?i am zis: "Stapâne, care va fi sfâr?itul acestora?"
Dan 12:9  ?i mi-a raspuns: "Du-te, Daniele, ca sunt închise ?i pecetluite cuvintele acestea pâna la sfâr?it!
Dan 12:10  Mul?i vor fi cura?i?i, albi?i ?i lamuri?i, iar cei nelegiui?i se vor purta ca cei nelegiui?i; to?i cei fara de lege nu var pricepe, ci numai cei în?elep?i vor în?elege.
Dan 12:11  ?i din vremea când va înceta jertfa cea de-a pururi ?i va începe urâciunea pustiirii vor fi o mie doua sute nouazeci de zile.
Dan 12:12  Fericit va fi cel ce va a?tepta ?i va ajunge la o mie trei sute treizeci ?i cinci de zile.
Dan 12:13  ?i tu mergi spre sfâr?itul tau ?i te odihne?te ?i te vei scula, ca sa prime?ti mo?tenirea ta în vremea cea de apoi!".


\end{document}