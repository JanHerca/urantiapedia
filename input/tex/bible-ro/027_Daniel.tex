\begin{document}

\title{Daniel}


\chapter{1}

\par 1 În anul al treilea al domniei lui Ioiachim, regele lui Iuda, a pornit Nabucodonosor, regele Babilonului, împotriva Ierusalimului și 1-a împresurat.
\par 2 Și a dat Domnul în mâna lui pe Ioiachim, regele lui Iuda, și o parte din vasele templului lui Dumnezeu, pe care le-a dus în țara Șinear, in templul dumnezeului lui și a așezat aceste vase în vistieria acestui dumnezeu.
\par 3 Și a zis regele către Așpenaz, cel mai mare peste famenii săi, să aducă dintre fiii lui Israel, din neam regesc și din cei din viță aleasă,
\par 4 Patru tineri fără nici un cusur trupesc, frumoși la chip și iscusiți în toată înțelepciunea, cunoscători a toată știința, cu adâncă putere de pătrundere și plini de râvnă, ca să slujească in palatul regelui, ca să-i învețe pe ei scrisul și limba Caldeilor.
\par 5 Și să-i învețe pe ei vreme de trei ani, după trecerea cărora să intre în slujba regelui. Pentru aceasta, regele le-a rânduit să li se dea în fiecare zi bucate din bucatele lui și vin din vinul lui.
\par 6 Printre ei se aflau din fiii lui Iuda: Daniel, Anania, Misael și Azaria.
\par 7 Și căpetenia famenilor le-a dat nume și a chemat pe Daniel, Beltșațar, pe Anania, Șadrac, pe Misael, Meșac și pe Azaria, Abed-Nego.
\par 8 și Daniel și-a pus in gând să nu se spurce cu bucatele și vinul regelui și a cerut voie căpeteniei famenilor să nu se pângărească.
\par 9 Și Dumnezeu i-a dat să afle înaintea căpeteniei famenilor bunăvoință șl îndurare.
\par 10 Și a zis căpetenia famenilor către Daniel: "Mă tem numai de domnul meu, regele, care v-a rânduit mîncărurile și băuturile; pentru ce să vadă el mai trase chipurile voastre decât ale tinerilor celor de o vârstă cu voi? Voi puneți în primejdie capul meu înaintea regelui.
\par 11 Și a grăit Daniel către supraveghetorul pe care căpetenia famenilor îl pusese peste Daniel, Anania, Misael și Azaria:
\par 12 "Încearcă o dată cu servii tăi timp de zece zile și dă-ne să mâncăm bucate din zarzavaturi și să bem apă!
\par 13 Apoi să te uiți la chipurile noastre și la chipul tinerilor care mănâncă din bucatele regelui și cum ți se va părea, așa fă cu supușii tăi!"
\par 14 Atunci el le-a ascultat această rugăminte și a încercat timp de zece zile.
\par 15 Și după un răstimp de zece zile, ei arătau mai frumoși și mai grași la trup decât toți tinerii care mâncau din bucatele regelui.
\par 16 Atunci supraveghetorul le-a înlocuit mâncarea lor și vinul - băutura dor - și le-a dat mâncăruri de legume.
\par 17 Și acestor patru tineri le-a dat Dumnezeu știință și pricepere în oricare scriere, precum și înțelepciune, încât Daniel putea să tâlcuiască vedeniile și visele.
\par 18 Și după acel răstimp hotărât de rege ca să-i aducă, i-a înfățișat căpetenia famenilor înaintea lui Nabucodonosor.
\par 19 Și a grăit regele cu ei, dar n-a aflat pe nimeni printre ei toți ca pe Daniel, Anania, Misael și Azaria. Și ei au început să slujească pe rege.
\par 20 Și în toate întrebările, asupra cărora regele le cerea dezlegarea, când era vorba de înțelepciune și de pricepere, îi găsea pe ei de zece ori mai isteți decât magii și prezicătorii din tot regatul lui.
\par 21 Și a rămas Daniel acolo până în anul întâi al domniei lui Cirus.

\chapter{2}

\par 1 Și în anul al doilea al domniei lui Nabucodonosor, a visat Nabucodonosor vise, iar duhul lui s-a tulburat și somnul nu-l mai prindea.
\par 2 Și a poruncit regele să cheme pe vrăjitori, pe prezicători, pe magi și pe caldei, ca să-i spună regelui visele lui; și ei au intrat și s-au înfățișat înaintea regelui.
\par 3 Și le-a zis lor regele: "Am visat un vis; duhul îmi este tulburat și vreau să știu visul".
\par 4 Atunci caldeii au grăit către rege în grai arameian: "O, rege, să trăiești în veac! Spune servilor tăi visul și noi îți vom descoperi tâlcuirea".
\par 5 Răspuns-a regele și a zis către caldei: "Să știți că hotărârea am luat-o! Dacă nu-mi veți face cunoscut visul și tâlcuirea lui, veți fi tăiați în bucăți, iar casele voastre prefăcute în grămezi de ruine.
\par 6 Dar dacă îmi faceți cunoscut visul și tâlcuirea lui veți primi de la mine daruri bogate și cinstire multă; deci arătați-mi visul și tâlcuirea lui!"
\par 7 Ei au răspuns pentru a doua oară și au zis: "0, rege, spune servilor tăi visul, iar noi îți vom face cunoscută tâlcuirea lui!"
\par 8 Răspuns-a și a zis regele: "Fără îndoială, eu știu că voi căutați să câștigați vreme, fiindcă vedeți că eu hotărârea am luat-o;
\par 9 Că dacă nu-mi faceți cunoscut visul, este că aveți de gând să vă sfătuiți unul cu altul și să spuneți înaintea mea vorbe mincinoase și înșelătoare, până când vremurile se vor schimba. De aceea spuneți-mi acum visul și eu voi ști dacă voi puteți să-mi descoperiți și tâlcuirea lui!"
\par 10 Răspuns-au caldeii în fața regelui și au zis: "Nu se află om pe pământ care să poată face cunoscut ceea ce cere regele, fiindcă nici un rege, oricât de mare și de puternic ar fi, nu ar cere așa ceva de la nici un tâlcuitor de semne, vrăjitor, sau caldeu.
\par 11 Și lucrul pe care îl cere regele este greu și nimeni altul nu poate să-l descopere înaintea lui, decât zeii al căror locaș nu este printre cei muritori".
\par 12 Din această pricină regele s-a mâniat și, în marea lui furie, a poruncit să fie uciși toți înțelepții din Babilon.
\par 13 Când a ieșit porunca, să fie omorâți înțelepții, trebuia să fie omorât și Daniel și cei dimpreună cu el.
\par 14 Atunci Daniel și-a îndreptat cuvântul cu înțelepciune și iscusință către Arioh, mai-marele gărzii regelui, care ieșise să omoare pe înțelepții din Babilon,
\par 15 Și a început să grăiască către Arioh, împuternicitul regelui: "De ce a dat regele o poruncă așa de aspră?" După ce Arioh i-a lămurit lui Daniel rostul poruncii,
\par 16 Daniel a plecat și a rugat pe rege să-i lase vreme ca să-i descopere tâlcul.
\par 17 După aceasta Daniel s-a dus în casa lui și a dat de știre lui Anania, Misael și Azaria, prietenii lui, care este pricina,
\par 18 Cerându-le să roage fierbinte milostivirea lui Dumnezeu din ceruri pentru această taină, ca să nu lase să piară Daniel și prietenii lui împreună cu ceilalți înțelepți ai Babilonului.
\par 19 Atunci i s-a descoperit lui Daniel taina aceasta într-o vedenie de noapte. Și a preaslăvit Daniel pe Dumnezeul cerului.
\par 20 Și a început Daniel a grăi: "Să fie numele lui Dumnezeu binecuvântat din veac și până în veac, că a Lui este înțelepciunea și puterea.
\par 21 Și El este Cel care schimbă timpurile și ceasurile, Cel care dă jos de pe tron pe regi și Cel care îi pune; El dă înțelepciune celor înțelepți și știință celor pricepuți.
\par 22 El descoperă cele mai adânci și cele mai ascunse lucruri, știe ce se petrece în întuneric și lumina sălășluiește cu El.
\par 23 Pe Tine, Dumnezeule al părinților mei, Te preaslăvesc și Îți mulțumesc ție, că mi-ai dat mie înțelepciune și pricepere și m-ai făcut să cunosc acum ceea ce noi Ți-am cerut rugându-Te, căci Tu ne-ai descoperit taina regelui".
\par 24 Apoi Daniel s-a dus la Arioh, pe care regele îl însărcinase să omoare pe înțelepții Babilonului și i-a grăit așa: "Nu da morții pe înțelepții Babilonului! Du-mă înaintea regelui și eu îi voi descoperi regelui tâlcuirea".
\par 25 Atunci Arioh a dus grabnic înăuntru pe Daniel în fața regelui și i-a vorbit astfel: "Am găsit un iudeu dintre cei aduși în robie care poate să tâlcuiască visul".
\par 26 Răspuns-a regele și a zis către Daniel, care se cheamă Beltșațar: "Oare ești tu în stare să-mi spui visul pe care l-am avut precum și tâlcuirea lui?"
\par 27 Daniel a răspuns înaintea regelui zicând: "Taina pe care vrea să o afle regele nu pot s-o facă cunoscută lui nici înțelepții, nici prezicătorii, nici vrăjitorii, nici cititorii în stele.
\par 28 Dar este un Dumnezeu în ceruri, Care descoperă tainele și Care a făcut cunoscut regelui Nabucodonosor ce se va întâmpla în vremurile ce vor veni. Iată care este visul și vedenia pe care le-ai avut când erai culcat în patul tău:
\par 29 Ție, rege, îți treceau gânduri prin minte, când erai în patul tău, pentru ceea ce se va întâmpla mai pe urmă și Cel ce ți-a descoperit taina ți-a dat să știi ce va fi.
\par 30 Și mie, nu prin înțelepciunea care ar fi în mine mai mult decât la toți cei vii, mi s-a descoperit taina aceasta, ci ca să se facă știut regelui înțelesul ei ca să cunoști gândurile inimii tale.
\par 31 O, rege! Tu priveai și iată un chip - acest chip era peste măsură de mare și strălucirea lui neobișnuită stătea înaintea ta și înfățișarea lui era grozavă.
\par 32 Acest chip avea capul de aur curat, pieptul și brațele de argint, pântecele și coapsele de aramă,
\par 33 Pulpele de fier, iar picioarele o parte de fier și o parte de lut.
\par 34 Tu priveai și iată o piatră desprinsă, nu de mână, a lovit chipul peste picioarele de fier și de lut și le-a sfărâmat.
\par 35 Atunci au fost sfărâmate în același timp fierul, lutul, arama, argintul și aurul și au ajuns ca pleava de pe arie vara și vântul le-a luat cu sine fără ca să se găsească locul lor; iar piatra care a lovit chipul a crescut munte mare și a umplut tot pământul.
\par 36 Iată visul, iar tâlcuirea lui o vom spune înaintea regelui.
\par 37 Tu, rege al regilor, căruia Dumnezeul cerului i-a dat regatul, puterea, tăria și mărirea,
\par 38 Și în mâinile căruia a dat pe fiii oamenilor în orice ținut ar locui, precum și fiarele câmpului și păsările cerului și i-a dat stăpânire peste toate, tu ești capul de aur.
\par 39 Și după tine se va ridica un alt regat mai mic decât al tău, apoi un al treilea regat de aramă, care va stăpâni peste tot pământul.
\par 40 Și un al patrulea regat va fi tare ca fierul și, după cum fierul sfărâmă și zdrobește totul, așa și el va sfărâma și va preface totul în pulbere, ca fierul care face totul bucăți.
\par 41 Iar picioarele pe care le-ai văzut și degetele, unele de lut de olar și altele de fier, înseamnă că va fi un regat împărțit și va fi tare ea fierul, după cum tu ai văzut fier amestecat cu lut.
\par 42 Și degetele picioarelor, unele de fier și altele de lut, înseamnă că regatul va fi parte tare, parte șubred.
\par 43 Și după cum ai văzut fierul amestecat cu lutul, așa se vor amesteca prin înrudiri, dar nu vor avea legătură temeinică între ele, după cum fierul nu se poate amesteca la un loc cu lutul.
\par 44 Iar în vremea acestor regi, Dumnezeul cerului va ridica un regat veșnic care nu va fi nimicit niciodată și care nu va fi trecut la alt popor; El va sfărâma și va nimici toate aceste regate și singur El va rămâne în veci.
\par 45 După cum tu ai văzut că o piatră a fost desprinsă din munte, nu de mână, și a zdrobit fierul, arama, lutul, argintul și aurul, Marele Dumnezeu a dat de știre regelui ceea ce va fi în viitor; visul este adevărat și tâlcuirea lui neîndoielnică".
\par 46 Atunci împăratul Nabucodonosor a căzut cu fața la pământ și s-a închinat înaintea lui Daniel și a dat poruncă să-i aducă jertfe și tămâieri.
\par 47 Răspuns-a regele către Daniel și a zis: "Cu adevărat că Dumnezeul vostru este Dumnezeul dumnezeilor și Stăpânul regilor, descoperitorul tainelor, căci tu ai putut să descoperi această taină".
\par 48 Atunci regele a ridicat la mare vrednicie pe Daniel și i-a dat daruri și numeroase lucruri de mare preț și 1-a pus guvernator peste tot ținutul Babilonului și căpetenie peste toți înțelepții Babilonului.
\par 49 La rugămintea lui Daniel, regele a însărcinat pe Șadrac, Meșac și Abed-Nego cu ocârmuirea ținutului Babilonului, iar Daniel a rămas la curtea regelui.

\chapter{3}

\par 1 Regele Nabucodonosor a făcut un chip de aur înalt de șaizeci de coți, lat de șase coți și 1-a așezat în câmpia Dura (Deir) din ținutul Babilonului.
\par 2 Și regele Nabucodonosor a trimis să adune pe satrapi, pe mai-marii dregători, pe cârmuitori, pe conducătorii oștirilor, pe vistiernici, pe cunoscătorii de legi, pe judecători și pe toți ceilalți dregători ai ținuturilor, ca să vină la sfințirea chipului pe care îl ridicase regele Nabucodonosor.
\par 3 Atunci s-au adunat satrapii, dregătorii cei mari, cârmuitorii, conducătorii oștirilor, vistiernicii, legiuitorii, judecătorii și toți ceilalți dregători ai ținuturilor la sfințirea chipului pe care îl ridicase Nabucodonosor și au stat înaintea chipului ridicat de Nabucodonosor.
\par 4 Și îndată un crainic a strigat cu glas tare: "Iată ce vi se poruncește vouă, popoarelor, neamurilor și limbilor:
\par 5 De îndată ce veți auzi glasul trâmbiței, flautului, chitarei, harpei, psalterionului, cimpoiului și al tuturor instrumentelor muzicale, veți cădea la pământ și vă veți închina chipului de aur pe care 1-a ridicat regele Nabucodonosor;
\par 6 Iar cine nu va cădea la pământ și nu se va închina, chiar în acea clipă va fi aruncat în mijlocul unui cuptor cu foc arzător!"
\par 7 De aceea, când toate popoarele au auzit glasul trâmbiței, al flautului, al chitarei, al harpei, al psalterionului și al tuturor instrumentelor muzicale, toate popoarele, neamurile și limbile au căzut la pământ și s-au închinat chipului de aur pe care îl ridicase regele Nabucodonosor.
\par 8 În același timp s-au apropiat câțiva bărbați caldei, care au pârât pe iudei.
\par 9 Ei au început să spună regelui Nabucodonosor: "O, rege, să trăiești în veac!
\par 10 Tu poruncă ai dat, ca oricine va auzi glasul trâmbiței, al flautului, al chitarei, al harpei, al psalterionului, al cimpoiului și al altor instrumente muzicale, să cadă la pământ și să se închine chipului de aur.
\par 11 Iar cine nu va cădea la pământ, nici se va închina, să fie aruncat în mijlocul unui cuptor cu foc arzător.
\par 12 Dar sunt niște iudei, pe care i-ai pus cârmuitori peste ținutul Babilonului: Șadrac, Meșac și Abed-Nego. Acești bărbați, nici că au luat în seamă porunca ta, o, rege; dumnezeului tău nu-i slujesc și chipului de aur pe care tu l-ai înălțat nu-i aduc închinare!"
\par 13 Atunci regele Nabucodonosor, plin de mânie și de zbucium, a poruncit să i se aducă înainte Șadrac, Meșac și Abed-Nego. Îndată au adus pe acești bărbați înaintea regelui.
\par 14 Nabucodonosor le-a zis: "Este, oare, adevărat, Șadrac, Meșac și Abed-Nego, că voi nu slujiți dumnezeului meu și chipului de aur pe care eu l-am așezat și nu-i cădeți la pământ cu rugăciuni?
\par 15 Acum, fiți gata și atunci când veți auzi glasul trâmbiței, al flautului, al chitarei, al harpei, al psalterionului, al cimpoiului și al altor instrumente muzicale, să cădeți la pământ și să vă închinați chipului pe care eu l-am făcut; iar dacă nu vreți să vă închinați, într-o clipă veți fi aruncați în mijlocul unui cuptor cu foc arzător. Și care dumnezeu vă va scăpa din mâna mea?"
\par 16 Răspuns-au Șadrac, Meșac și Abed-Nego și au zis regelui: O, Nabucodonosor, noi n-avem nevoie ca la aceasta să-ți dăm un răspuns!
\par 17 Dacă, într-adevăr, Dumnezeul nostru Căruia Îi slujim poate să ne scape, El ne va scăpa din cuptorul cel cu foc arzător și din mâna ta, o, rege!
\par 18 Și chiar dacă nu ne va scăpa, știut să fie de tine, o, rege, că noi nu vom sluji dumnezeilor tăi și înaintea chipului de aur pe care tu l-ai așezat nu vom cădea la pământ!"
\par 19 Atunci Nabucodonosor s-a umplut de mânie și și-a schimbat înfățișarea feței sale față de Șadrac, Meșac și Abed-Nego. Și începând iarăși a grăi, a poruncit să încălzească cuptorul de șapte ori mai mult decât era de obicei.
\par 20 Și a poruncit celor mai puternici oameni din oștirea lui să lege pe Șadrac, Meșac și Abed-Nego și să-i arunce în cuptorul cel cu foc arzător.
\par 21 Atunci acești oameni, îmbrăcați cum erau, cu mantie, încălțăminte, pălărie și cu toată îmbrăcămintea lor, au fost legați și aruncați în mijlocul cuptorului cu foc arzător.
\par 22 Fiindcă porunca regelui era grabnică și cuptorul foarte înfierbântat, acei oameni care au aruncat pe Șadrac, Meșac și Abed-Nego au fost mistuiți de văpaia focului.
\par 23 Și acești trei bărbați, Șadrac, Meșac și Abed-Nego au căzut legați în mijlocul cuptorului cu foc arzător.
\par 24 Atunci regele Nabucodonosor a fost cuprins de spaimă și s-a sculat în grabă. El a început a grăi și a zis către sfetnicii săi: "Oare, n-am aruncat noi trei bărbați legați în mijlocul cuptorului cu foc arzător?" Răspuns-au și i-au zis: "Cu adevărat, așa este, o, rege!"
\par 25 Și începând din nou a grăi, a zis: "Iată, eu văd patru bărbați dezlegați, umblând prin mijlocul cuptorului, nevătămați, iar chipul celui de al patrulea, ca fața unuia dintre fiii zeilor".
\par 26 Atunci s-a apropiat Nabucodonosor de gura cuptorului cu foc arzător și, începând a grăi, a zis: "Șadrac, Meșac și Abed-Nego, slujitorii mei, ieșiți afară și veniți la mine!" Atunci Șadrac, Meșac și Abed-Nego au ieșit dinăuntrul cuptorului.
\par 27 Și adunându-se satrapii, dregătorii cei mai mari, cârmuitorii și sfetnicii regelui, au văzut că focul nu pricinuise nici o vătămare trupului acestor oameni, că nici perii capului nu se pârliseră și că hainele lor erau neschimbate și că nici măcar nu miroseau a foc.
\par 28 Răspuns-a Nabucodonosor și a zis: "Binecuvântat să fie Dumnezeul lui Șadrac, Meșac și Abed-Nego, Care a trimis pe îngerul Său și a izbăvit pe servii Săi, care își puseseră nădejdea în El și care au călcat porunca regelui și și-au dat trupurile lor ca să nu slujească și să nu se închine altor dumnezei decât Dumnezeului lor.
\par 29 Și poruncesc: Popoare, neamuri și limbi, toți aceia care ar vorbi de rău pe Dumnezeul lui Șadrac, Meșac și Abed-Nego, să fie tăiați în bucăți și casele lor să fie nimicite, fiindcă nu este un alt dumnezeu care să-i poată izbăvi într-acest chip".
\par 30 După aceasta a întărit regele în slujbele lor pe Șadrac, Meșac și Abed-Nego, peste ținutul Babilonului.

\chapter{4}

\par 1 Regele Nabucodonosor a dat hrisov către toate popoarele, neamurile și limbile care locuiesc pe tot pământul: "Pacea voastră să sporească!
\par 2 Plăcutu-mi-a să vestesc minunile și faptele cele peste fire, pe care le-a făcut mie Dumnezeul cel Preaînalt.
\par 3 Cât de mari sunt minunile Lui și cât de puternice sunt faptele cele peste fire! Împărăția Lui este împărăție veșnică și stăpânirea Lui ține din neam în neam!
\par 4 Eu, Nabucodonosor, stam fără de grijă în casa mea și bucuros de viață în patul meu.
\par 5 Am visat un vis care m-a înspăimântat; și gândurile mele când stam culcat în patul meu și vedeniile pe care le-am avut m-au frământat adânc.
\par 6 Și am poruncit să mi se aducă în fața mea toți înțelepții din Babilon, care să-mi tâlcuiască visul.
\par 7 Atunci au sosit tâlcuitorii de semne, prezicătorii, caldeii și cititorii în stele și le-am spus visul, dar ei nu mi-au dat tâlcuirea lui.
\par 8 Iar în cele din urmă s-a înfățișat înaintea mea Daniel, al cărui nume este Beltșațar, după numele dumnezeului meu și care are în al Duhul Dumnezeului celui Sfânt și i-am spus visul:
\par 9 "Beltșațar, tu, mai-marele tâlcuitorilor de semne, tu, cel în care știu că locuiește Duhul lui Dumnezeu celui Sfânt și că nici o taină nu-ți este grea, ia aminte la visul pe care l-am visat și spune-mi tâlcuirea lui!
\par 10 Vedenia pe care am avut-o, când eram culcat în patul meu, a fost: "Mă uitam și iată un copac în mijlocul pământului, înalt foarte.
\par 11 Copacul creștea și era puternic și vârful lui ajungea până la cer și se putea vedea până la capătul pământului.
\par 12 Frunzișul lui era frumos și roadele lui multe, și hrană pentru toți se afla în el. Sub el căutau umbră fiarele câmpului, iar în ramurile lui își făceau cuiburi păsările cerului și din el se hrăneau toate viețuitoarele.
\par 13 Priveam în vedenia pe care am avut-o, când eram în patul meu și iată un înger, un sfânt, se cobora din ceruri;
\par 14 El a strigat cu glas tare și a poruncit așa: Doborâți copacul și tăiați-i crengile, scuturați frunzele lui și împrăștiați roadele lui, ca animalele să fugă de sub el și păsările din frunzișul lui!
\par 15 Iar butucul și rădăcinile să rămână în pământ în legături de fier și de aramă, în iarba câmpului! Din roua cerului să fie udat și cu dobitoacele câmpului să împartă iarba pământului.
\par 16 Inima lui să nu mai fie inimă de om, ci o inimă de dobitoc să-i fie dată și șapte ani să treacă peste el!
\par 17 Această hotărâre se sprijină pe porunca îngerilor, iar porunca sfinților este ca să cunoască cei vii că Cel Preaînalt stăpânește peste împărăția oamenilor, pe care o dă cui vrea și poate să ridice peste ea pe cel mai de jos dintre oameni.
\par 18 Acesta este visul pe care l-am visat eu, regele Nabucodonosor, iar tu, Beltșațar, spune tâlcuirea lui, căci toți înțelepții regatului meu nu pot să-mi facă cunoscută tâlcuirea. Tu însă ești în stare, fiindcă ai în tine Duhul Dumnezeului celui Sfânt".
\par 19 Atunci Daniel, al cărui nume este Beltșațar, a rămas înmărmurit pentru o clipă și gândurile lui s-au tulburat. Regele a prins din nou a grăi și a zis: "Beltșațar, visul și tâlcuirea lui să nu te înfricoșeze!" Răspuns-a Beltșațar și a zis: "O, stăpâne, visul să fie pentru cei ce te urăsc pe tine, iar tâlcuirea lui pentru vrăjmașii tăi!
\par 20 Copacul pe care tu l-ai văzut, mare și puternic, care cu vârful ajungea până la cer și se vedea până la capătul pământului,
\par 21 Cu frunziș frumos, cu rod mult și din care se hrăneau toți, sub care se adăposteau fiarele câmpului, iar în ramurile lui făceau cuiburi păsările cerului,
\par 22 Acela ești tu, o, rege, tu, care te-ai mărit și te-ai făcut puternic, ai crescut și ai ajuns până la ceruri, ai stăpânirea ta până la marginea pământului.
\par 23 Iar că a văzut regele un înger, un sfânt, coborându-se din cer și zicând: Doborâți copacul și nimiciți-l, dar butucul și rădăcinile lui lăsați-le în pământ și în legături de fier și de aramă, în iarba pământului, și de roua cerului să fie udat și cu animalele câmpului să fie părtaș până ce vor trece peste el șapte ani,
\par 24 Aceasta înseamnă, o, rege, că hotărârea Celui Preaînalt se va împlini peste stăpânul meu regele,
\par 25 Că tu vei fi alungat dintre oameni și vei locui împreună cu animalele câmpului și vei mânca iarbă și din roua cerului vei fi udat și vor trece peste tine șapte ani, până ce tu vei cunoaște că Cel Preaînalt are stăpânirea peste împărăția oamenilor și o dă cui voiește.
\par 26 Și dacă a poruncit să lase butucul și rădăcinile copacului, înseamnă că regatul tău va fi ocrotit pentru tine îndată ce tu vei recunoaște că Cerul are stăpânirea.
\par 27 De aceea, o, rege, plăcut să-ți fie sfatul meu înaintea ta: Răscumpără păcatele tale prin fapte de dreptate și nedreptățile tale prin milă către cei săraci, dacă vrei ca bunăstarea în care te afli să dăinuiască".
\par 28 Totul s-a împlinit cu regele Nabucodonosor.
\par 29 După douăsprezece luni, când regele Nabucodonosor se plimba în palatul regal din Babilon,
\par 30 A prins a grăi zicând: "Oare nu este acesta Babilonul cel mare pe care l-am clădit eu întru tăria puterii mele și spre cinstea strălucirii mele, ca reședință regală?"
\par 31 Pe când cuvântul era încă în gura regelui, un glas s-a coborât din cer: "Ție, rege Nabucodonosor, ți se spune: Regatul s-a luat de la tine.
\par 32 Și dintre oameni vei fi izgonit, vei locui cu animalele câmpului și vei paște iarbă și vor trece șapte ani peste tine, până ce vei recunoaște că Cel Preaînalt are putere peste împărăția oamenilor și că o dă cui voiește!"
\par 33 Îndată s-a împlinit cuvântul asupra lui Nabucodonosor, căci a fost alungat dintre oameni și a mâncat iarbă ca animalele și trupul lui era udat de rouă până când părul i-a crescut ca penele vulturilor și unghiile ca ghiarele păsărilor.
\par 34 "Și după trecerea acestui timp, eu Nabucodonosor, am ridicat ochii mei la cer și mintea mi-a venit din nou și am binecuvântat pe Cel Preaînalt și Celui veșnic viu l-am adus laudă și preamărire, că puterea Lui este putere veșnică, iar împărăția Lui din neam în neam.
\par 35 Toți locuitorii pământului sînt socotiți ca o nimica și El face ce voiește cu oștirea cerească și cu locuitorii pământului și nimeni nu poate să-L împiedice la lucrul Lui și să-I zică: "Ce faci Tu?"
\par 36 În același timp mi-a venit mintea la loc și, spre gloria regatului meu, mi-a venit iarăși măreția și strălucirea și sfetnicii mei și dregătorii cei mari m-au chemat și regatul mi-a fost dat în stăpânire, iar puterea mea a crescut și mai mult.
\par 37 Acum, eu, Nabucodonosor, laud, înalț și preamăresc pe Împăratul cerului; toate faptele Lui sunt adevărate și căile Lui drepte, iar pe cei ce umblă mândri poate să-i smerească!"

\chapter{5}

\par 1 Regele Belșațar a făcut un mare ospăț pentru o mie din dregătorii săi și în fața celor o mie a băut vin.
\par 2 Belșațar când era în toiul ospățului, la băutul vinului, a poruncit să aducă vasele de aur și de argint pe care Nabucodonosor, tatăl său, le luase din templul din Ierusalim, ca regele să bea vin din ele, împreună cu dregătorii săi, femeile sale și concubinele sale.
\par 3 Atunci au fost aduse vasele de aur și de argint care fuseseră luate din templul lui Dumnezeu din Ierusalim și au băut din ele regele și dregătorii săi, femeile sale și concubinele sale.
\par 4 Ei au băut vin și au preamărit pe dumnezeii de aur, de argint, de aramă, de fier, de lemn și de piatră.
\par 5 În aceeași clipă au ieșit degetele unei mâini de om, care au scris în fața sfeșnicului celui mare pe tencuiala peretelui palatului regal, și regele a văzut vârful degetelor mâinii care scria.
\par 6 Atunci fața regelui s-a îngălbenit și gândurile lui s-au tulburat; încheieturile coapselor sale au slăbit, iar genunchii i se izbeau unul de altul neîncetat.
\par 7 Regele a început să strige din toate puterile să i se aducă prezicătorii, caldeii și tâlcuitorii de semne. Atunci el a prins a grăi și a zis tuturor înțelepților din Babilon: "Oricine va citi scrisul acesta și îmi va arăta tâlcuirea lui va fi îmbrăcat în veșmânt de purpură, lanț de aur i se va pune împrejurul gâtului lui și va cârmui ca al treilea în regatul meu!"
\par 8 Atunci au venit toți înțelepții regelui, dar nu au putut citi scrisul, nici să-i facă cunoscut înțelesul lui.
\par 9 Regele Belșațar s-a înspăimântat foarte, fața lui s-a îngălbenit, iar dregătorii lui au rămas înmărmuriți.
\par 10 Regina, auzind strigătul regelui și al dregătorilor, s-a dus in cămara de ospăț. Ea a început a grăi și a zis: "O, rege, să trăiești în veac! Gândurile tale să nu te înspăimânte, și chipul feței tale să nu se schimbe!
\par 11 În regatul tău se află un om care are în el Duhul lui Dumnezeu celui Sfânt și în vremea domniei tatălui tău a fost descoperită în el lumină, pricepere și înțelepciune, ca înțelepciunea dumnezeiască, iar regele Nabucodonosor, tatăl tău, l-a pus mai-marele tâlcuitorilor ele semne, al caldeilor, al cititorilor în stele și al înțelepților.
\par 12 Din pricină că s-a descoperit în Daniel un duh înalt, o știință și o pricepere de a tâlcui visele, de a dezlega lucrurile greu de înțeles și de a descoperi tainele, pentru aceasta regele i-a dat numele de Beltșațar. Deci cheamă pe Daniel, și el îți va arăta tâlcuirea".
\par 13 Daniel a fost adus înaintea regelui. Regele a zis atunci lui Daniel: "Tu ești Daniel, cel dintre robii iudei pe care i-a adus tatăl meu din Iuda?
\par 14 Am auzit că în tine este Duhul lui Dumnezeu și că în tine se află lumină, pricepere și înțelepciune fără seamăn.
\par 15 Acum au fost aduși la mine înțelepții și prezicătorii ca să-mi citească scrisul acesta și să-mi facă cunoscută tâlcuirea lui, dar nu au fost în stare să-mi spună tâlcuirea acestor cuvinte.
\par 16 Și eu am auzit despre tine că tu poți să tâlcuiești visele și să dezlegi cele tainice. Acum, dacă tu ești în stare să citești scrisul și să-mi faci cunoscută tâlcuirea lui, vei fi îmbrăcat în veșmânt de purpură și lanț de aur vei avea împrejurul gâtului tău și vei cârmui ca al treilea în regatul meu".
\par 17 Atunci Daniel a început să vorbească și a grăit regelui: "Darurile tale poți să le păstrezi pentru tine, iar lucrurile de preț dă-le altora; căci eu voi citi regelui scrisul și îi voi face cunoscută tâlcuirea lui.
\par 18 O, rege! Dumnezeu cel Preaînalt a dat lui Nabucodonosor, tatăl tău, regatul, mărirea, cinstea și strălucirea.
\par 19 Iar din pricina puterii pe care El i-o dăduse, toate popoarele, neamurile și limbile erau înfricoșate și tremurau înaintea lui; el omora pe cine voia și lăsa în viață pe cine voia, înălța pe cine voia și cobora pe cine voia.
\par 20 Și pentru că inima lui se trufise și duhul lui se împietrise până la mândrie, a fost coborât de pe scaunul regatului său și vrednicia lui i-a fost luată;
\par 21 Și a fost izgonit din neamul omenesc, iar inima i s-a făcut asemenea dobitoacelor, și a locuit cu asinii sălbatici, mâncând iarbă ca boii și și-a udat trupul din roua cerului până a recunoscut că Dumnezeu cel Preaînalt are putere peste împărăția oamenilor și așază peste ea pe cine vrea.
\par 22 Și tu, fiul său, Belșațar, tu nu ești smerit cu inima, măcar că tu știi toate acestea.
\par 23 Și te-ai ridicat împotriva Stăpânului cerului și ai adus vasele templului Său înaintea ta, și ai băut vin din ele, tu și dregătorii tăi, femeile tale și concubinele tale, și ai preamărit dumnezei de argint, de aur, de aramă, de fier, de lemn și de piatră, dumnezei care nu văd, nici nu aud și nici nu cunosc nimic, iar pe Dumnezeul în mâna Căruia este suflarea ta și toate căile tale, nu L-ai cinstit.
\par 24 Atunci a trimis El vârful mâinii care a scris aceste cuvinte.
\par 25 Iată inscripția care a fost scrisă: Mene, mene, techel ufarsin.
\par 26 Aceasta este tâlcuirea cuvântului mene: Dumnezeu a numărat zilele regatului tău și i-a pus capăt.
\par 27 Techel: l-a cântărit în cântar și l-a găsit ușor.
\par 28 Peres: a împărțit regatul tău și 1-a dat Mezilor și Perșilor".
\par 29 Atunci a poruncit Belșațar și au îmbrăcat pe Daniel în veșmânt de purpură și i-au pus lanț de aur la gâtul lui și au dat de veste că el va cârmui ca al treilea în împărăție.
\par 30 Chiar în noaptea aceea a fost omorât Belșațar, împăratul Caldeilor.
\par 31 Și Darius Medul a ajuns rege când era în vârstă de aproape 62 de ani.

\chapter{6}

\par 1 Și i-a plăcut lui Darius să pună peste regatul lui 120 de satrapi, care să poarte de grijă în tot regatul,
\par 2 Iar în fruntea lor, trei dregători dintre care unul era Daniel, și acești satrapi trebuia să le dea lor socoteală, astfel ca regele să nu fie păgubit.
\par 3 Însă Daniel era mai presus decât toți dregătorii și satrapii, fiindcă în el era un duh înalt și regele își pusese în gând să-l pună mai mare peste tot regatul.
\par 4 Atunci dregătorii și satrapii s-au trudit să găsească lui Daniel vreo pricină din partea cârmuirii regatului, dar n-au putut să-i afle nici o pricină sau lucru rău, căci el era credincios și nici o trecere cu vederea sau greșeală nu i s-a putut pune în seamă.
\par 5 Atunci oamenii aceștia au zis: "Dacă nu-i găsim lui Daniel nici o pricină, cu toate acestea îi vom afla lui una, în legea Dumnezeului lui".
\par 6 Dregătorii și satrapii aceștia s-au dus atunci în grabă la rege și așa i-au grăit: "O, rege Darius, să trăiești în veac!
\par 7 Toți marii dregători ai regatului, marii cârmuitori, satrapii, sfetnicii și guvernatorii s-au sfătuit laolaltă ca regele să dea o poruncă și să se rânduiască oprirea ca oricine s-ar ruga vreme de treizeci de zile altui dumnezeu și om în afară de tine, rege, să fie aruncat în groapa cu lei.
\par 8 Acum, o, rege, fă cunoscută oprirea și dă poruncă scrisă, care, potrivit legii Mezilor și Perșilor, nu se mai poate schimba".
\par 9 Așadar, regele Darius a dat oprire scrisă și poruncă.
\par 10 Îndată ce Daniel a aflat că o poruncă a fost dată, a intrat în casa sa, care avea în cămara de sus fereastră deschisă înspre Ierusalim și în fiecare zi îngenunchea de trei ori, s-a rugat și a lăudat pe Dumnezeu, cum făcea și mai înainte.
\par 11 Atunci bărbații aceia au venit în număr mare și au aflat pe Daniel rugându-se și cerând mila lui Dumnezeu.
\par 12 Apoi s-au apropiat și au grăit înaintea regelui privitor la porunca regală: "Oare n-ai poruncit tu ca oricine s-ar ruga timp de treizeci de zile la oricare alt dumnezeu sau om, în afară de tine, rege, să fie aruncat într-o groapă cu lei?" Răspuns-a regele și a zis: "Lucrul rămâne hotărât și, după legile Mezilor și Perșilor, nu se poate schimba".
\par 13 Atunci au răspuns ei regelui și au zis: "Daniel, cel dintre robii iudei, nu a luat în seamă porunca ta, rege, nici nu s-a îngrijit de oprirea ta, ci de trei ori pe zi își face rugăciunea".
\par 14 Când a auzit regele acestea, s-a tulburat foarte și și-a îndreptat gândul spre Daniel, cum ar putea să-l scape, și până la apusul soarelui s-a străduit ca să-l scoată.
\par 15 În urmă, oamenii aceia au intrat în grabă la rege și i-au zis: "Știut să-ți fie, o, rege, că, după legea Mezilor și a Perșilor, orice poruncă sau oprire dată de rege nu se mai poate schimba".
\par 16 Atunci regele a dat poruncă să aducă pe Daniel și 1-a aruncat în groapa cu lei. După acestea regele a prins a grăi și a zis lui Daniel: "Dumnezeul tău pe Care tu Îl cinstești fără încetare, Acela te va scăpa!"
\par 17 Apoi s-a adus o piatră care a fost pusă peste gura gropii, iar regele a pecetluit-o cu inelul său și cu inelul dregătorilor săi, așa ca nimic să nu se schimbe cu privire la Daniel.
\par 18 Pe urmă, împăratul s-a dus în palatul său și a petrecut noaptea în post și nu au adus lângă el concubine, iar somnul nu l-a mai prins.
\par 19 Apoi regele s-a sculat dis-de-dimineață, în revărsat de zori și a venit în grabă la groapa cu lei.
\par 20 Și când s-a apropiat de groapă, a strigat pe Daniel cu glas tare. Atunci regele a prins a grăi și a zis lui Daniel: "Daniel, slujitorul Dumnezeului celui viu, Dumnezeul tău, Căruia te închini neîncetat, oare a putut să te scape de lei?"
\par 21 Apoi Daniel a vorbit cu regele: "O, rege, în veci să trăiești!
\par 22 Dumnezeu a trimis pe îngerul Său și a astupat gura leilor, și ei nu mi-au făcut nici un rău, pentru că am fost găsit nevinovat înaintea Lui, precum și în fața ta, rege, n-am făcut nici un rău!"
\par 23 Regele s-a bucurat foarte și a poruncit să scoată pe Daniel din groapă și Daniel a fost scos din groapă și nici o rană nu i-a fost găsită, căci nădăjduise în Dumnezeul lui.
\par 24 Atunci a poruncit regele să aducă pe bărbații aceia care defăimaseră pe Daniel și au fost aruncați în groapa cu lei, ei, fiii lor și femeile lor, și nici nu au ajuns bine în fundul gropii, că leii s-au și năpustit asupra lor și le-au sfărâmat toate oasele.
\par 25 Regele Darius a scris la toate popoarele, neamurile și limbile care locuiesc peste tot pământul: "Pacea voastră să sporească!
\par 26 Poruncă iese de la mine ca în tot cuprinsul regatului meu să se teamă și să tremure lumea înaintea Dumnezeului lui Daniel, că El este Dumnezeul cel viu, Care rămâne în veci și împărăția lui nu se va nimici, iar stăpânirea Lui nu va avea sfârșit.
\par 27 El poate să scape și să libereze, face semne și minuni în cer și pe pământ; El a scăpat pe Daniel din ghearele leilor".
\par 28 Și Daniel se afla într-o stare fericită în regatul lui Darius și în regatul lui Cirus, regele Perșilor.

\chapter{7}

\par 1 În anul dintâi al lui Belșațar, regele Babilonului, Daniel a visat un vis, și vedeniile pe care le-a avut când era culcat în patul lui îl înfricoșară. Atunci el a scris visul și a povestit ceea ce era mai de seamă dintre fapte.
\par 2 Daniel a început a grăi, zicând: "Văzut-am în vedenia mea din timpul nopții cum cele patru vânturi ale cerului au sfredelit marea cea necuprinsă.
\par 3 Și patru fiare uriașe au ieșit din mare, una mai deosebită decât alta.
\par 4 Cea dintâi semăna cu un leu și avea aripi de vultur. M-am uitat la ea până ce aripile i-au fost smulse și a fost ridicată de pe pământ și pusă pe picioare ca un om și i s-a dat inimă de an.
\par 5 Și iată o a doua fiară, cu înfățișare de urs, stând într-o rână, cu trei coaste în gură, între dinți, și așa i s-a poruncit: "Scoală-te! Mănâncă multă carne!"
\par 6 Apoi m-am uitat din nou și iată o altă fiară, asemenea unui leopard, având pe spate patru aripi de pasăre; și fiara avea patru capete, și i s-a dat putere.
\par 7 În urmă am privit în vedeniile mele de noapte și iată o a patra fiară înspăimântătoare și înfricoșătoare și nespus de puternică. Ea avea dinți mari de fier și gheare de aramă; mânca și sfărâma, iar rămășița o călca in picioare. Ea se deosebea de toate celelalte fiare de mai înainte și avea zece coarne.
\par 8 M-am uitat cu luare aminte la coarne, și iată un alt corn mic creștea între ele, și trei din coarnele cele dintâi au fost smulse de el. Și iată că acest corn avea ochi ca ochii de om și gură care grăia lucruri mari.
\par 9 Am privit până când au fost așezate scaune, și S-a așezat Cel vechi de zile; îmbrăcămintea Lui era albă ca zăpada, iar părul capului Său curat ca lâna; tronul Său, flăcări de foc; roțile lui, foc arzător.
\par 10 Un râu de foc se vărsa și ieșea din el; mii de mii Îi slujeau și miriade de miriade stăteau înaintea Lui! Judecătorul S-a așezat și cărțile au fost deschise.
\par 11 Eu mă uitam mereu, din pricina multelor vorbe pe care cornul cel mare le grăia. Am privit până când fiara a fost omorâtă și trupul ei nimicit și dat focului.
\par 12 Dar și celorlalte fiare li s-a luat stăpânirea, și lungimea vieții lor a fost hotărâtă până la o vreme și un anumit timp.
\par 13 Am privit în vedenia de noapte, și iată pe norii cerului venea cineva ca Fiul Omului și El a înaintat până la Cel vechi de zile, și a fost dus în fața Lui.
\par 14 Și Lui I s-a dat stăpânirea, slava și împărăția, și toate popoarele, neamurile și limbile Îi slujeau Lui. Stăpânirea Lui este veșnică, stăpânire care nu va trece, iar împărăția Lui nu va fi nimicită niciodată.
\par 15 Pentru aceasta, eu, Daniel, am fost tulburat cu duhul meu și vedeniile pe care le-am avut mă înspăimântau.
\par 16 M-am apropiat atunci de unul din cei de față și l-am rugat să-mi spună adevărul privitor la toate acestea. Și el mi-a vorbit și mi-a făcut cunoscut înțelesul acestor lucruri.
\par 17 "Aceste fiare, patru la număr, înseamnă că patru regi se vor ridica pe pământ,
\par 18 Și sfinții Celui Preaînalt vor primi regatul și îl vor ține în stăpânire în veci și în vecii vecilor.
\par 19 După aceasta l-am rugat să-mi spună adevărul despre fiara a patra, care se deosebea de toate celelalte și care era afară din cale de înspăimântătoare, cu dinți de fier și cu gheare de aramă și care mânca, sfărâma, iar ceea ce rămânea călca în picioare;
\par 20 Și despre cele zece coarne care erau pe capul său și despre celălalt care creștea și înaintea căruia au căzut cele trei și avea ochi și gură care grăia lucruri mari și care era mult mai mare decât celelalte.
\par 21 M-am uitat, și cornul acela purta război cu cei sfinți și i-a biruit,
\par 22 Până ce a venit Cel vechi de zile și a făcut dreptate sfinților Celui Preaînalt, până ce s-a împlinit vremea și împărăția a ajuns sub stăpânirea sfinților.
\par 23 El a răspuns astfel: "Fiara a patra înseamnă că un al patrulea rege va fi pe pământ, care se va deosebi de toate celelalte regate, care va mânca tot pământul, îl va călca în picioare și îl va zdrobi.
\par 24 Și cele zece coarne înseamnă că din acest regat se vor ridica zece regi, și un altul se va scula după ei; el se va deosebi de cei dinaintea lui și va doborî la pământ trei regi.
\par 25 Și va grăi cuvinte de defăimare împotriva Celui Preaînalt și va asupri pe sfinții Celui Preaînalt, și își va pune în gând să schimbe sărbătorile și legea, și ei vor fi dați în mâna lui o vreme și vremuri și jumătate de vreme.
\par 26 Și judecata se va face și i se va lua stăpânirea, ca să-l nimicească și să-l prăbușească pentru totdeauna.
\par 27 Iar regatul și stăpânirea și mărirea regilor de sub ceruri se vor da poporului sfinților Celui Preaînalt; împărăția Lui este împărăție veșnică și toate stăpânirile Îi vor sluji Lui și pe El Îl vor asculta".
\par 28 Iată sfârșitul vorbirii mele cu el. Pe mine, Daniel, gândurile mele m-au înfricoșat foarte și fața mi s-a schimbat și am păstrat cuvântul în inima mea.

\chapter{8}

\par 1 În anul al treilea al domniei regelui Belșațar mi s-a arătat mie, lui Daniel, o vedenie, afară de cele ce mi se arătaseră la început.
\par 2 Și m-am uitat în vedenie și, când priveam, parcă eram în capitala Suza, care este în țara Elamului și, stând cu privirea ațintită, eram pe fluviul Ulai.
\par 3 Și am ridicat ochii mei și m-am uitat și iată un berbec cu două coarne stând în picioare în fața fluviului și coarnele lui erau lungi, iar unul mai lung decât celălalt și cel mai lung creștea cel din urmă.
\par 4 Am văzut berbecul lovind cu coarnele la apus, la miazănoapte și la miazăzi și nici o fiară nu-i putea sta împotrivă și nimeni nu scăpa de asuprirea lui. El făcea ce voia și creștea.
\par 5 Și m-am uitat cu luare aminte și iată un țap venea de la apus pe deasupra feței pământului, fără ca să-l atingă. Și țapul avea un corn între ochi, care corn se putea zări.
\par 6 și a venit până la berbecul cel cu două coarne pe care l-am văzut stând în fața fluviului și s-a năpustit spre el cu toată tăria puterii lui.
\par 7 Și l-am văzut cum s-a apropiat de berbec, s-a întărâtat împotriva lui și a lovit berbecul și i-a sfărâmat cele două coarne, iar berbecul nu mai avea putere să i se împotrivească; și l-a aruncat la pământ și 1-a călcat în picioare și nimeni n-a scăpat pe berbec.
\par 8 Și țapul a crescut foarte și când a ajuns puternic, cornul cel mare s-a sfărâmat și am zărit patru coarne crescând în locul lor, spre cele patru vânturi ale cerului.
\par 9 Și din unul dintre ele a ieșit un corn mic, care a crescut afară din cale, către miazăzi, către răsărit și către țara strălucirii.
\par 10 Și el s-a înălțat până către oștirea cerească și a doborât la pământ din oștire și din stele și le-a călcat în picioare.
\par 11 Și a mai crescut până la mai-marele oștirii și i-a luat jertfa de fiecare zi și i-a răsturnat locul templului său.
\par 12 Și peste jertfa de fiecare zi el a pus nelegiuirea și a aruncat adevărul la pământ, și el a făcut și a izbutit.
\par 13 Și am auzit un sfânt care grăia, și un alt sfânt a zis către cel ce grăia: "Până când va dura vedenia și jertfa de fiecare zi nu se va mai aduce și un păcat al pustiirii va fi pus în loc și templul și oștirea vor fi călcate în picioare?"
\par 14 Atunci el i-a răspuns: "Până la două mii trei sute de seri și de dimineți; după aceasta templul își va avea din nou rostul lui".
\par 15 Și când eu, Daniel, am văzut vedenia și m-am străduit să o înțeleg, iată că atunci a stat cineva înaintea mea cu chip de om.
\par 16 Atunci am auzit un glas de om deasupra fluviului Ulai, glas care striga și spunea: "Gavriile, tâlcuiește celui de acolo vedenia!"
\par 17 Și el a venit unde eram eu și, când se apropia, m-am înspăimântat și am căzut cu fața la pământ. Și el mi-a grăit: "Ia aminte, fiul omului, căci vedenia este pentru a arăta sfârșitul veacurilor!"
\par 18 Și când vorbea cu mine, stam înmărmurit cu fața la pământ; atunci el s-a atins de mine și m-a ridicat în picioare in locul în care mă aflam,
\par 19 Și mi-a spus: "Iată, îți voi face cunoscut ceea ce se va întâmpla în vremea din urmă a mâniei lui Dumnezeu; că pentru vremea cea din urmă este vedenia.
\par 20 Berbecul cu două coarne, pe care tu îl vezi, înseamnă regii Mediei și ai Persiei.
\par 21 Iar țapul este regele Greciei și cornul cel mare, care este între ochi, este regele cel dintâi.
\par 22 Și dacă el a fost sfărâmat, patru s-au ridicat în locul lui; înseamnă că patru regate se vor ridica din neamul lui, dar fără să aibă puterea lui.
\par 23 La sfârșitul stăpânirii lor, la vremea covârșirii păcatelor lor, se va ridica un rege cu chip semeț și isteț în lucrurile ascunse.
\par 24 Și stăpânirea lui va crește în putere - dar nu prin puterea lui însuși - și va face pustiiri uriașe și în orice lucru va izbuti și va prăbuși pe cei tari și pe poporul sfinților.
\par 25 Din pricina istețimii lui, va izbuti înșelăciunea în mâna lui și se va semeți în inima sa și în plină vreme de pace va doborî pe mulți. Și se va ridica împotriva Regelui regilor, dar va fi aruncat la pământ nu de mână omenească.
\par 26 Iar vedenia despre seri și despre dimineți, care a fost spusă, este adevărată; tu însă pecetluiește vedenia, că se va întâmpla după multe zile".
\par 27 Și eu, Daniel, am fast istovit de puteri și am căzut bolnav câteva zile. Dar m-am sculat și am avut grijă de lucrurile regelui. Am rămas uimit de cele ce am văzut și n-am putut să le înțeleg.

\chapter{9}

\par 1 În anul întâi al lui Darius, fiul lui Ahașveroș (Artaxerxe), din neamul Mezilor, care a domnit peste regatul Caldeilor,
\par 2 În anul întâi al domniei lui - eu, Daniel, am citit în cărți numărul de șaptezeci de ani, pentru care a fost cuvântul Domnului către proorocul Ieremia, ani care trebuia să se împlinească de la dărâmarea Ierusalimului.
\par 3 Și mi-am îndreptat fața către Domnul Dumnezeu, stăruind în rugăciune și în rugi fierbinți, cu post, sac și cenușă.
\par 4 Și m-am rugat Domnului Dumnezeu și m-am mărturisit și am zis: "O, Doamne, Dumnezeule cel mare și minunat, Care păzești legământul și îndrumarea pentru cei ce Te iubesc pe Tine și iau aminte la poruncile Tale!
\par 5 Păcătuit-am, fărădelege am făcut, ca și cei nelegiuiți ne-am purtat, răsculatu-ne-am și ne-am depărtat de la poruncile și de la legile Tale.
\par 6 Și nu am ascultat de slujitorii Tăi prooroci, care ne-au grăit în numele Tău: către regii noștri, către mai-marii noștri, părinților noștri și la tot poporul țării.
\par 7 A Ta este, Doamne, dreptatea, iar a noastră rușinarea fețelor noastre, precum se arată astăzi oamenilor din Iuda și locuitorilor din Ierusalim și la tot Israelul, cei de aproape și cei de departe, în toate țările în care Tu i-ai izgonit din pricina fărădelegilor ce le-au săvârșit împotriva Ta.
\par 8 Doamne Dumnezeule, a noastră este rușinarea fețelor, a regilor noștri, a mai-marilor noștri și a părinților noștri, căci noi am păcătuit ție;
\par 9 A Domnului Dumnezeului nostru este milostivirea și îndurarea. Răzvrătitu-ne-am împotriva Lui.
\par 10 Și nu am ascultat de glasul Domnului Dumnezeului nostru ca să umblăm în legea Lui, pe care ne-a dat-o nouă prin mâna slujitorilor Săi profeți.
\par 11 Și tot Israelul a călcat legea Ta și s-a depărtat, ca să nu mai audă glasul Tău. Vărsatu-s-a peste noi blestemul și jurământul scris în legea lui Moise, slujitorul lui Dumnezeu, căci am păcătuit împotriva Ta.
\par 12 Și a adeverit cuvintele Sale pe care le-a grăit către noi și către judecătorii noștri, care au cârmuit peste noi, că a voit să abată peste noi strașnic prăpăd, ce nu s-a mai întâmplat niciodată sub cer, asemenea celui din Ierusalim.
\par 13 Precum este scris în legea lui Moise, toată această nenorocire s-a năpustit asupra noastră, dar n-am îmbunat fața Domnului Dumnezeului nostru, întorcându-ne de la nelegiuirile noastre și luând aminte la adevărul Său.
\par 14 Gândit-a îndelung Domnul asupra nenorocirii pe care a abătut-o peste noi, că drept este Domnul Dumnezeul nostru în toate faptele pe care le-a făcut, dar noi n-am ascultat de glasul Lui.
\par 15 Și acum, Doamne Dumnezeul nostru, Tu Care ai scos pe poporul Tău din țara Egiptului cu mână tare și Te-ai făcut vestit până în ziua de astăzi, păcătuit-am, fărădelege am făcut.
\par 16 O, Doamne! Întoarcă-se, după milostivirile Tale, toată mânia și toată văpaia urgiei Tale de la cetatea Ierusalimului, de la muntele cel sfânt al Tău! Că, pentru păcatele noastre și pentru fărădelegile părinților noștri, Ierusalimul și poporul Tău au ajuns de ocară pentru toți vecinii noștri.
\par 17 Acum ascultă, Dumnezeul nostru, rugăciunea slujitorului Tău și ruga fierbinte și luminează fața Ta spre templul Tău pustiit, pentru numele Tău, Doamne!
\par 18 Pleacă, Dumnezeul meu, urechea Ta și auzi, deschide ochii Tăi și vezi mâhnirea noastră adâncă și cetatea asupra căreia se cheamă numele Tău. Că nu pentru faptele noastre drepte aducem înaintea Ta rugăciunile noastre cele fierbinți, ci pentru milele Tale cele mari.
\par 19 O, Doamne, ascultă! O, Doamne, iartă! O, Doamne, ia aminte și lucrează! Nu întârzia pentru numele Tău, Dumnezeul meu; că numele Tău îl poartă cetatea și poporul Tău!"
\par 20 Și în vreme ce grăiam și mă rugam și mărturiseam păcatul meu și păcatul poporului meu Israel și cădeam cu ruga mea fierbinte înaintea Domnului Dumnezeului meu, pentru sfânt muntele Dumnezeului meu,
\par 21 Și pe când vorbeam în rugăciunea mea, iată un om, Gavriil, pe care l-am văzut în vedenia mea cea de la început, în zbor grăbit, s-a apropiat de mine pe la vremea jertfei de seară.
\par 22 Și a venit și mi-a grăit zicând: "Daniele, chiar acum am sosit ca să-ți deschid mintea.
\par 23 Când tu ai început să te rogi, poruncă mi-a fost dată și eu am venit ca să-ți vestesc, căci tu ești un om iubit de Dumnezeu. Ia aminte la cuvânt și înțelege vedenia!
\par 24 Șaptezeci de săptămâni sunt hotărâte pentru poporul tău și pentru cetatea ta cea sfântă până ce fărădelegea va trece peste margini și se va pecetlui păcatul și se va ispăși nelegiuirea, până ce dreptatea cea veșnică va veni, vedenia și proorocia se vor pecetlui și se va unge Sfântul Sfinților.
\par 25 Să știi și să înțelegi că de la ieșirea poruncii pentru zidirea din nou a Ierusalimului și până la Cel-Uns - Cel-Vestit - sunt șapte săptămâni și șaizeci și două de săptămâni; și din nou vor fi zidite piețele și zidul din afară, în vremuri de strâmtorare.
\par 26 Iar după cele șaizeci și două de săptămâni, Cel-Uns va pieri fără să se găsească vreo vină în El, iar poporul unui domn va veni și va dărâma cetatea și templul. Și sfârșitul cetății va veni prin potopul mâniei lui Dumnezeu și până la capăt va fi război - prăpădul cel hotărât.
\par 27 Și El va încheia un legământ cu mulți într-o săptămână, iar la mijlocul săptămânii va înceta jertfa și prinosul și în templu va fi urâciunea pustiirii, până când pedeapsa nimicirii cea hotărâtă se va vărsa peste locul pustiirii".

\chapter{10}

\par 1 În anul al treilea al lui Cirus, regele Perșilor, i s-a descoperit un cuvânt lui Daniel - care se chema Beltșațar - și adevărat este cuvântul și el vestește război mare. El a pătruns cuvântul și a înțeles vedenia.
\par 2 "În vremea aceea, eu, Daniel, am petrecut trei săptămâni de zile în jale.
\par 3 Pâine bună n-am mâncat, carne și vin n-am pus în gura mea și cu miresme nu m-am uns, până ce nu s-au împlinit trei săptămâni de zile.
\par 4 Dar în ziua a douăzeci și patra a lunii întâi, eu, Daniel, mă aflam pe malul fluviului celui mare, adică Tigrul,
\par 5 Și mi-am ridicat ochii mei și iată un om îmbrăcat în veșminte de in, iar coapsele lui încinse cu aur curat și de preț;
\par 6 Trupul lui ca și crisolitul și fața lui ca fulgerul, iar ochii lui ca flăcările de foc, brațele și picioarele lui străluceau ca arama lustruită și sunetul cuvintelor lui ca vuietul unei mulțimi.
\par 7 Și am văzut numai eu, Daniel, o vedenie, iar oamenii care erau cu mine nu au văzut vedenia; ci o mare spaimă a căzut peste ei și au fugit să se ascundă.
\par 8 Atunci eu am rămas singur și am văzut această mare vedenie și n-a rămas în mine putere, fața mea și-a schimbat înfățișarea, stricându-se, și nu mai aveam vlagă.
\par 9 Și am auzit glasul cuvintelor lui și, la glasul cuvintelor lui, eu am căzut înmărmurit cu fața la pământ.
\par 10 Și iată că o mână s-a atins de mine și m-a ridicat în genunchi și pe palmele mâinilor mele.
\par 11 Și a grăit către mine: "Daniele, om plăcut al lui Dumnezeu, ia aminte la cuvintele pe care ți le spun ție și stai drept, că acum sunt trimis către tine". Și pe când îmi grăia cuvântul acesta, m-am sculat tremurând.
\par 12 Și a zis către mine: "Nu te teme, Daniele, că din ziua cea dintâi, de când ți-ai sârguit inima ta ca să înțelegi și să te smerești înaintea Dumnezeului tău, cuvintele tale au fost auzite și eu am sosit din pricina cuvintelor tale.
\par 13 Și îngerul păzitor al Persiei mi-a stat împotrivă douăzeci și una de zile, dar iată că Mihail, cel dintâi dintre îngerii păzitori, a venit în ajutorul meu și eu l-am lăsat acolo la îngerul păzitor al regelui Perșilor
\par 14 Și am venit ca să-ți fac cunoscut ce se va întâmpla poporului tău la sfârșitul zilelor; că mai este o vedenie pentru zilele cele din urmă".
\par 15 Și pe când grăia cu mine astfel de cuvinte, mi-am întors privirea spre pământ și am rămas ca un mut.
\par 16 Și iată! Acela care avea înfățișarea fiului omului s-a atins de buzele mele; atunci am deschis gura mea și am grăit și am zis către cel ce sta înaintea mea: "O, Stăpânul meu! Din pricina acestei vedenii m-au cuprins zvârcoliri de durere și am rămas fără putere.
\par 17 Și cum poate un slujitor atât de mic al Domnului său să grăiască cu un Stăpân atât de mare ca Tine! Și de spaimă îmi piere toată puterea și suflarea mi se oprește".
\par 18 Atunci s-a atins iarăși de mine acela care avea înfățișarea Fiului Omului și mi-a dat tărie.
\par 19 Și mi-a zis: "Nu te teme, om plăcut al lui Dumnezeu! Pace ție! Fii tare și curajos!" Și pe când grăia cu mine, m-am simțit întărit și am zis: "Spune, Stăpâne, căci Tu m-ai întărit!"
\par 20 Atunci El a zis: "Știi tu, oare, pentru ce am venit la tine? Acum Mă voi întoarce să fac război cu îngerul păzitor al Persiei și, când Eu Mă voi duce, iată că îngerul păzitor al Greciei va veni.
\par 21 Îți voi vesti ție ceea ce este scris în cartea adevărului. Nimeni nu poate să Mă ajute mai bine la aceasta decât Mihail, îngerul vostru păzitor".

\chapter{11}

\par 1 Și eu în anul dintâi al lui Darius Medul stam lângă el ca să-l ajut și să-l întăresc.
\par 2 Și acum îți fac cunoscut adevărul: Iată că se vor scula încă trei regi în Persia, iar al patrulea va stăpâni bogății mai mari decât toți și, când va fi puternic prin bogățiile sale, va ridica pe toți împotriva regatului Greciei.
\par 3 Și va ieși la iveală un rege viteaz și va stăpâni peste un regat puternic și va face numai ceea ce i se va părea bun.
\par 4 Iar când va fi în culmea puterii sale, regatul lui se va prăbuși și se va împărți după cele patru vânturi ale carului, fără ca să rămână urmașilor lui și nici să aibă putere întocmai ca mai înainte, că regatul lui va fi sfâșiat și se va împărți la alții decât la aceia din neamul lui.
\par 5 Și regele de la miazăzi va ajunge puternic, dar unul din căpeteniile lui va fi mai puternic decât el și va domni, iar stăpânirea lui va fi un regat puternic.
\par 6 Și după trecere de ani se vor uni și fiica regelui de la miazăzi va veni către regele de la miazănoapte, ca să statornicească pacea. Dar ea nu va păstra tăria brațului său și nu va dăinui nici el, nici brațul lui; și va fi dată morții, ea și însoțitorii ț ei și copiii ei Și aliatul ei în acele vremuri.
\par 7 Iar unul dintre odraslale din rădăcinile ei se va ridica și va porni împotriva oștirii și va intra în cetatea cea întărită a regelui de la miazănoapte și va face cu ei ce va voi și va fi biruitor.
\par 8 Chiar și dumnezeii lor, împreună cu chipurile lor turnate, cu vasele lor de preț, aur și argint, vor fi duse în robia Egiptului și el va fi mai puternic decât regele de la miazănoapte, ani de-a rândul.
\par 9 El va năvăli în regatul regelui de la miazănoapte, apoi se va întoarce în țara sa.
\par 10 Și feciorul lui va pregăti războiul și va strânge o mare mulțime de oști de luptă și va da năvală peste el și îl va potopi și îl va cotropi și se va întoarce și va pătrunde până la cetatea lui cea întărită.
\par 11 Atunci regele de la miazăzi va fi amărât foarte și va ieși și va face război cu el - cu regele de la miazănoapte, care va ridica o mare oștire; dar oștirea va cădea în mâna regelui de la miazăzi;
\par 12 Și oștirea va fi nimicită; inima regelui se va îngâmfa; zeci de mii va doborî la pământ, dar nu va fi mai puternic.
\par 13 Și încă o dată regele de la miazănoapte va ridica oștiri mai puternice decât cele dintâi și, după un răstimp de câțiva ani, va da năvală peste el, cu o mare oștire și cu numeroasă călărime.
\par 14 Și în vremea aceea mulți se vor scula împotriva regelui de la miazăzi și oameni silnici din poporul tău se vor ridica, așa ca să se împlinească vedenia, dar se vor poticni.
\par 15 Iar regele de la miazănoapte va veni și va ridica întărituri și va cuprinde o cetate întărită, iar oștirea de ajutor a regelui de la miazănoapte nu va putea să țină piept și trupele sale vor fugi și nu va fi nici un chip de stat împotrivă.
\par 16 Cel care va porni împotriva lui va face după placul său și nimeni nu i se va împotrivi, și se va opri în țara strălucirii și totul va fi în mâna lui.
\par 17 Și își va îndrepta privirea să ia în stăpânire întregul lui regat și va face o învoială cu el și îi va da de soție pe o fiică a sa, ca să aducă pieirea țării, dar aceasta nu se va întâmpla și nu va duce la izbândă.
\par 18 Și își va întoarce privirea spre insule și va cuprinde multe din ele, iar o căpetenie va pune capăt ocării lui, fără putință de răspuns.
\par 19 Și își va întoarce fața spre întăriturile țării sale, dar se va poticni, va cădea și va pieri.
\par 20 Și în locul lui va veni altul, care va trimite un strângător de dări în locul care este gloria regatului (Ierusalimul), dar în câteva zile va fi doborât nu prin mânie, nici prin război.
\par 21 Și în locul lui va veni un batjocoritor, care nu avusese nici un drept la vrednicia regală, și el va veni pașnic și prin uneltiri se va face stăpân pe regat.
\par 22 Și oștirile de ajutor vor da înapoi înaintea lui și vor fi sfărâmate, de asemenea și o căpetenie a legământului.
\par 23 Și după împrietenirea cu el se va servi de vicleșug și va porni și va birui cu puțin popor.
\par 24 Și pe neașteptate va veni în cele mai bogate ținuturi ale țării și va face ceea ce n-au făcut părinții lui și părinții părinților lui; el le va împărți cu risipă, pradă și jaf și bogății; și va urzi planuri împotriva cetăților întărite, numai pentru o vreme.
\par 25 Și își va îndrepta puterea și inima împotriva regelui de la miazăzi, eu oștire mare, iar el se va prinde în luptă cu oaste mare și puternică, dar nu va putea să i se împotrivească, că se vor urzi uneltiri împotriva lui.
\par 26 Cei ce mănâncă la masă cu el, îl vor prăbuși, iar oastea lui se va sfărâma și mulți vor cădea loviți de moarte.
\par 27 Și cei doi regi vor pune la cale viclenii în inima lor și la masă își vor spune lucruri mincinoase, dar fără nici o izbândă, că încă n-a venit sfârșitul rânduit de Dumnezeu.
\par 28 Și se va duce cu mari averi în țara lui și inima lui va fi împotriva Legământului cel sfânt; așa va lucra și se va întoarce în țara lui.
\par 29 La vremea hotărâtă va năvăli din nou la miazăzi și această din urmă bătălie nu va fi ca bătălia dintâi.
\par 30 Corăbii din Chitim vor veni împotriva lui; și el va pierde curajul, se va întoarce și se va întărâta împotriva legământului sfânt și va lucra și se va învoi iarăși cu cei ce au părăsit legământul sfânt.
\par 31 Și oști trimise de el vor sta și vor pângări locașul sfânt și cetatea, iar jertfa de fiecare zi o var da de o parte și vor pune în loc urâciunea pustiirii.
\par 32 Și pe cei ce au săvârșit fărădelegi împotriva legământului îi va înșela prin lingușiri, iar poporul care cunoaște pe Dumnezeul său va rămâne statornic și îl va urma.
\par 33 Cei mai înțelepți vor învăța pe cei mulți, dar ei vor cădea un timp de sabie și foc, de temniță și pustiire.
\par 34 Și în vremea căderii lor, vor primi puțin ajutor și mulți se vor uni cu ei, dar din fățărnicie.
\par 35 Și printre înțelepți, mulți vor cădea ca să se lămurească, să se curățească și să se albească până la sfârșitul vremii, că mai este încă până la vremea rânduită.
\par 36 Și regele va face după placul său și se va ridica și se va trufi împotriva oricărui dumnezeu și împotriva Dumnezeului dumnezeilor va spune lucruri nemaiauzite și va propăși până ce sfârșitul mâniei va veni, că ceea ce este hotărât se va întâmpla.
\par 37 Și nu va lua aminte la dumnezeii părinților lui și nici la dumnezeul plăcut femeilor și nu va băga în seamă pe nici un alt dumnezeu căci el se va ridica deasupra tuturor.
\par 38 Iar în locul lui va cinsti pe dumnezeul cetăților, pe un dumnezeu pe care nu l-au cunoscut părinții lui; aceluia i se va închina cu aur, cu argint, cu pietre scumpe și lucruri de preț.
\par 39 El va lua ca apărători ai cetăților întărite pe poporul unui dumnezeu străin; pe cei care îl vor recunoaște, el îi va cinsti mult, îi va pune stăpâni peste mulțime și le va împărți pământuri ca răsplată.
\par 40 La sfârșitul vremii se va război cu el regele cel de la miazăzi și se va năpusti împotriva lui regele cel de la miazănoapte, cu care de război, cu călăreți și cu multe corăbii. El va veni în țările pe care le va cotropi și le va străbate.
\par 41 Și va veni și în țara strălucirii și zeci de mii se var prăbuși; și iată care vor scăpa din mâna lui: Edomul, Moabul și restul fiilor lui Amon.
\par 42 Și va întinde mâna sa peste țări și țara Egiptului nu va scăpa.
\par 43 Și va ajunge stăpân peste comorile de aur și de argint și peste toate lucrurile prețioase ale Egiptului, iar Libienii și Etiopienii vor merge după el.
\par 44 Dar zvonuri de la răsărit și de la miazănoapte vor veni să-l înspăimânte și el va ieși cu furie grozavă ca să prăpădească și să nimicească pe mulți.
\par 45 Și el va înfige corturile palatului său între mare și muntele cel sfânt și strălucit, apoi va veni sfârșitul lui și nimeni nu-i va veni în ajutor!

\chapter{12}

\par 1 Și în vremea aceea se va scula Mihail, marele voievod care ocrotește pe fiii poporului tău, și va fi vreme de strâmtorare cum n-a mai fost de când sunt popoarele și până în vremea de acum. Dar în vremea aceea, poporul tău va fi mântuit și anume oricine va fi găsit scris în carte.
\par 2 Și mulți dintre cei care dorm în țărâna pământului se vor scula, unii la viață veșnică, iar alții spre ocară și rușine veșnică.
\par 3 Și cei înțelepți vor lumina ca strălucirea cerului și cei care vor fi îndrumat pe mulți pe calea dreptății vor fi ca stelele în vecii vecilor.
\par 4 Iar tu, Daniele, ține ascunse cuvintele și pecetluiește cartea până la sfârșitul vremii. Mulți vor cerceta-o cu de-amănuntul și va crește știința.
\par 5 Și eu, Daniel, m-am uitat, și iată alți doi bărbați stând în picioare, unul pe un mal al fluviului, iar altul pe celălalt mal al fluviului.
\par 6 Și unul a zis celui ce era îmbrăcat în veșminte de in și stătea deasupra apelor fluviului: "Pe când se var sfârși aceste fapte minunate?"
\par 7 Și am ascultat pe bărbatul cel îmbrăcat în veșminte de in care stătea deasupra apelor fluviului și el și-a ridicat dreapta și stânga lui către ceruri și a jurat pe Cel ce este viu în veci: "Va mai ține o vreme; vremuri și jumătate de vreme, iar când se va isprăvi de sfărâmat puterea poporului celui sfânt, atunci vor lua sfârșit toate acestea".
\par 8 Și eu am auzit, dar nu am înțeles și am zis: "Stăpâne, care va fi sfârșitul acestora?"
\par 9 Și mi-a răspuns: "Du-te, Daniele, că sunt închise și pecetluite cuvintele acestea până la sfârșit!
\par 10 Mulți vor fi curățiți, albiți și lămuriți, iar cei nelegiuiți se vor purta ca cei nelegiuiți; toți cei fără de lege nu var pricepe, ci numai cei înțelepți vor înțelege.
\par 11 Și din vremea când va înceta jertfa cea de-a pururi și va începe urâciunea pustiirii vor fi o mie două sute nouăzeci de zile.
\par 12 Fericit va fi cel ce va aștepta și va ajunge la o mie trei sute treizeci și cinci de zile.
\par 13 Și tu mergi spre sfârșitul tău și te odihnește și te vei scula, ca să primești moștenirea ta în vremea cea de apoi!".


\end{document}