\begin{document}

\title{2 Petru}


\chapter{1}

\par 1 Simon Petru, slujitor și apostol al lui Iisus Hristos, celor ce prin dreptatea Dumnezeului nostru și a Mântuitorului Iisus Hristos au dobândit o credință de același preț cu a noastră:
\par 2 Har vouă și pacea să se înmulțească, întru cunoștința lui Dumnezeu și a lui Iisus, Domnul nostru.
\par 3 Dumnezeiasca Lui putere ne-a dăruit toate cele ce sunt spre viață și spre bună cucernicie, făcându-ne să cunoaștem pe Cel ce ne-a chemat prin slava Sa și prin puterea Sa,
\par 4 Prin care El ne-a hărăzit mari și prețioase făgăduințe, ca prin ele să vă faceți părtași dumnezeieștii firi, scăpând de stricăciunea poftei celei din lume.
\par 5 Pentru aceasta, puneți și din partea voastră toată sârguința și adăugați la credința voastră: fapta bună, iar la fapta bună: cunoștința,
\par 6 La cunoștință: înfrânarea; la înfrânare: răbdarea; la răbdare: evlavia;
\par 7 La evlavie: iubirea frățească, iar la iubirea frățească: dragostea.
\par 8 Căci dacă aceste lucruri sunt în voi și tot sporesc, ele nu vă vor lăsa nici trândavi, nici fără roade în cunoașterea Domnului nostru Iisus Hristos.
\par 9 Iar cel ce nu are acestea este slab văzător și orb și a uitat de curățirea păcatelor lui de demult.
\par 10 Pentru aceea, fraților, siliți-vă cu atât mai vârtos să faceți temeinică chemarea și alegerea voastră, căci, făcând acestea, nu veți greși niciodată.
\par 11 Că așa vi se va da cu bogăție intrarea în veșnica împărăție a Domnului nostru și Mântuitorului Iisus Hristos.
\par 12 Drept aceea, vă voi aminti pururea de acestea, cu toate că le știți și sunteți întăriți întru adevărul în care stați.
\par 13 Socotesc, dar, că este drept, câtă vreme sunt în acest cort, să vă țin treji, prin aducerea aminte,
\par 14 Fiindcă știu că degrabă voi lepăda cortul acesta, precum mi-a arătat Domnul nostru Iisus Hristos.
\par 15 Dar mă voi sili ca, și după plecarea mea, să vă amintiți necontenit de acestea,
\par 16 Pentru că noi v-am adus la cunoștință puterea Domnului nostru Iisus Hristos și venirea Lui, nu luându-ne după basme meșteșugite, ci fiindcă am văzut slava Lui cu ochii noștri.
\par 17 Căci El a primit de la Dumnezeu-Tatăl cinste și slavă atunci când, din înălțimea slavei, un glas ca acesta a venit către El: "Acesta este Fiul Meu cel iubit, întru Care am binevoit".
\par 18 Și acest glas noi l-am auzit, pogorându-se din cer, pe când eram cu Domnul în muntele cel sfânt.
\par 19 Și avem cuvântul proorocesc mai întărit, la care bine faceți luând aminte, ca la o făclie ce strălucește în loc întunecos, până când va străluci ziua și Luceafărul va răsări în inimile voastre,
\par 20 Aceasta știind mai dinainte că nici o proorocie a Scripturii nu se tâlcuiește după socotința fiecăruia;
\par 21 Pentru că niciodată proorocia nu s-a făcut din voia omului, ci oamenii cei sfinți ai lui Dumnezeu au grăit, purtați fiind de Duhul Sfânt.

\chapter{2}

\par 1 Dar au fost în popor și prooroci mincinoși, după cum și între voi vor fi învățători mincinoși, care vor strecura eresuri pierzătoare și, tăgăduind chiar pe Stăpânul Care i-a răscumpărat, își vor aduce lor grabnică pieire;
\par 2 Și mulți se vor lua după învățăturile lor rătăcite și, din pricina lor, calea adevărului va fi hulită;
\par 3 Și din poftă de avere și cu cuvinte amăgitoare, ei vă vor momi pe voi. Dar osânda lor, de mult pregătită, nu zăbovește și pierzarea lor nu dormitează.
\par 4 Căci dacă Dumnezeu n-a cruțat pe îngerii care au păcătuit, ci, legându-i cu legăturile întunericului în iad, i-a dat să fie păziți spre judecată,
\par 5 Și n-a cruțat lumea veche, ci a păstrat numai pe Noe, ca al optulea propovăduitor al dreptății, când a adus potopul peste cei fără de credință,
\par 6 Și cetățile Sodomei și Gomorei, osândindu-le la nimicire, le-a prefăcut în cenușă, dându-le ca o pildă nelegiuiților din viitor;
\par 7 Iar pe dreptul Lot, chinuit de petrecerea în desfrânare a celor nelegiuiți, l-a izbăvit,
\par 8 Pentru că dreptul acesta, locuind între ei, prin ce vedea și auzea, zi de zi, chinuia sufletul său cel drept, din pricina faptelor lor nelegiuite.
\par 9 Domnul poate să scape din ispite pe cei credincioși, iar pe cei nedrepți să-i păstreze, ca să fie pedepsiți în ziua judecății,
\par 10 Și mai vârtos pe cei ce umblă după îmboldirile cărnii, în pofte spurcate și disprețuiesc domnia cerească. Îndrăzneți, îngâmfați, ei nu se cutremură să hulească măririle (din cer),
\par 11 Pe când îngerii, deși sunt mai mari în tărie și în putere, nu aduc în fața Domnului judecată defăimătoare împotriva lor.
\par 12 Aceștia însă, ca niște dobitoace fără minte, din fire făcute să fie prinse și nimicite, hulind cele ce nu cunosc vor pieri în stricăciunea lor;
\par 13 Ei înșiși fiind nedrepți își vor lua plata nedreptății, socotind o plăcere desfătarea de fiecare zi; ei sunt pete și ocară, făcându-și plăcere, în rătăcirile lor, să ospăteze cu voi la mesele voastre;
\par 14 Având ochii plini de pofta desfrânării și fiind nesățioși de păcat, ei amăgesc sufletele cele nestatornice; inima lor e deprinsă la lăcomie și sunt fiii blestemului.
\par 15 Părăsind calea cea dreaptă, au rătăcit și au apucat calea lui Balaam, fiul lui Bosor, care a iubit plata nedreptății,
\par 16 Dar a primit mustrare pentru călcarea lui de lege; căci dobitocul fără grai, pe care era călare, grăind cu glas omenesc, a oprit nebunia proorocului.
\par 17 Aceștia sunt izvoare fără de apă și nori purtați fără de furtună, cărora li se păstrează, în veac, întunericul cel de nepătruns,
\par 18 Căci rostind vorbe trufașe și deșarte, ei momesc întru poftele trupului, cu desfrânări, pe cei care de abia au scăpat de cei ce viețuiesc în rătăcire.
\par 19 Ei le făgăduiesc libertate, fiind ei înșiși robii stricăciunii, fiindcă ceea ce te biruiește, aceea te și stăpânește.
\par 20 Căci dacă, după ce au scăpat de întinăciunile lumii, prin cunoașterea Domnului și Mântuitorului nostru Iisus Hristos, iarăși se încurcă în acestea, ei sunt învinși; li s-au făcut cele de pe urmă mai rele decât cele dintâi.
\par 21 Căci mai bine era pentru ei să nu fi cunoscut calea dreptății, decât, după ce au cunoscut-o, să se întoarcă de la porunca sfântă, dată lor.
\par 22 Cu ei s-a întâmplat adevărul din zicală: Câinele se întoarce la vărsătura lui și porcul scăldat la noroiul mocirlei lui.

\chapter{3}

\par 1 Iubiților, aceasta este acum a doua epistolă pe care v-o scriu. În ele caut să trezesc, în amintirea voastră, dreapta voastră judecată,
\par 2 Ca să vă aduceți aminte de cuvintele cele mai înainte grăite de sfinții prooroci și de porunca Domnului și Mântuitorului, dată prin apostolii voștri.
\par 3 Întâi, trebuie să știți că, în zilele cele de apoi, vor veni, cu batjocură, batjocoritori care vor umbla după poftele lor,
\par 4 Și vor zice: Unde este făgăduința venirii Lui? Că de când au adormit părinții, toate așa rămân, ca de la începutul făpturii.
\par 5 Căci ei în chip voit uită aceasta, că cerurile erau de demult și că pământul s-a închegat, la cuvântul Domnului, din apă și prin apă,
\par 6 Și prin apă lumea de atunci a pierit înecată,
\par 7 Iar cerurile de acum și pământul sunt ținute prin același cuvânt și păstrate pentru focul din ziua judecății și a pieirii oamenilor necredincioși.
\par 8 Și aceasta una să nu vă rămână ascunsă, iubiților, că o singură zi, înaintea Domnului, este ca o mie de ani și o mie de ani ca o zi.
\par 9 Domnul nu întârzie cu făgăduința Sa, după cum socotesc unii că e întârziere, ci îndelung rabdă pentru voi, nevrând să piară cineva, ci toți să vină la pocăință.
\par 10 Iar ziua Domnului va veni ca un fur, când cerurile vor pieri cu vuiet mare, stihiile, arzând, se vor desface, și pământul și lucrurile de pe el se vor mistui.
\par 11 Deci dacă toate acestea se vor desființa, cât de mult vi se cuvine vouă să umblați întru viață sfântă și în cucernicie,
\par 12 Așteptând și grăbind venirea zilei Domnului, din pricina căreia cerurile, luând foc, se vor nimici, iar stihiile, aprinse, se vor topi!
\par 13 Dar noi așteptăm, potrivit făgăduințelor Lui, ceruri noi și pământ nou, în care locuiește dreptatea.
\par 14 Pentru aceea, iubiților, așteptând acestea, sârguiți-vă să fiți aflați de El în pace, fără prihană și fără vină.
\par 15 Și îndelunga-răbdare a Domnului nostru socotiți-o drept mântuire, precum v-a scris și iubitul nostru frate Pavel după înțelepciunea dată lui,
\par 16 Cum vorbește despre acestea, în toate epistolele sale, în care sunt unele lucruri cu anevoie de înțeles, pe care cei neștiutori și neîntăriți le răstălmăcesc, ca și pe celelalte Scripturi, spre a lor pierzare.
\par 17 Deci voi, iubiților, cunoscând acestea de mai înainte, păziți-vă, ca nu cumva, lăsându-vă târâți de rătăcirea celor fără de lege, să cădeți din întărirea voastră,
\par 18 Ci creșteți în har și în cunoașterea Domnului nostru și Mântuitorului Iisus Hristos. A Lui este slava, acum și în ziua veacului! Amin.


\end{document}