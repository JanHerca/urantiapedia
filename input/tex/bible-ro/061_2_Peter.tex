\begin{document}

\title{2 Petru}


\chapter{1}

\par 1 Simon Petru, slujitor ?i apostol al lui Iisus Hristos, celor ce prin dreptatea Dumnezeului nostru ?i a Mântuitorului Iisus Hristos au dobândit o credin?a de acela?i pre? cu a noastra:
\par 2 Har voua ?i pacea sa se înmul?easca, întru cuno?tin?a lui Dumnezeu ?i a lui Iisus, Domnul nostru.
\par 3 Dumnezeiasca Lui putere ne-a daruit toate cele ce sunt spre via?a ?i spre buna cucernicie, facându-ne sa cunoa?tem pe Cel ce ne-a chemat prin slava Sa ?i prin puterea Sa,
\par 4 Prin care El ne-a harazit mari ?i pre?ioase fagaduin?e, ca prin ele sa va face?i parta?i dumnezeie?tii firi, scapând de stricaciunea poftei celei din lume.
\par 5 Pentru aceasta, pune?i ?i din partea voastra toata sârguin?a ?i adauga?i la credin?a voastra: fapta buna, iar la fapta buna: cuno?tin?a,
\par 6 La cuno?tin?a: înfrânarea; la înfrânare: rabdarea; la rabdare: evlavia;
\par 7 La evlavie: iubirea fra?easca, iar la iubirea fra?easca: dragostea.
\par 8 Caci daca aceste lucruri sunt în voi ?i tot sporesc, ele nu va vor lasa nici trândavi, nici fara roade în cunoa?terea Domnului nostru Iisus Hristos.
\par 9 Iar cel ce nu are acestea este slab vazator ?i orb ?i a uitat de cura?irea pacatelor lui de demult.
\par 10 Pentru aceea, fra?ilor, sili?i-va cu atât mai vârtos sa face?i temeinica chemarea ?i alegerea voastra, caci, facând acestea, nu ve?i gre?i niciodata.
\par 11 Ca a?a vi se va da cu boga?ie intrarea în ve?nica împara?ie a Domnului nostru ?i Mântuitorului Iisus Hristos.
\par 12 Drept aceea, va voi aminti pururea de acestea, cu toate ca le ?ti?i ?i sunte?i întari?i întru adevarul în care sta?i.
\par 13 Socotesc, dar, ca este drept, câta vreme sunt în acest cort, sa va ?in treji, prin aducerea aminte,
\par 14 Fiindca ?tiu ca degraba voi lepada cortul acesta, precum mi-a aratat Domnul nostru Iisus Hristos.
\par 15 Dar ma voi sili ca, ?i dupa plecarea mea, sa va aminti?i necontenit de acestea,
\par 16 Pentru ca noi v-am adus la cuno?tin?a puterea Domnului nostru Iisus Hristos ?i venirea Lui, nu luându-ne dupa basme me?te?ugite, ci fiindca am vazut slava Lui cu ochii no?tri.
\par 17 Caci El a primit de la Dumnezeu-Tatal cinste ?i slava atunci când, din înal?imea slavei, un glas ca acesta a venit catre El: "Acesta este Fiul Meu cel iubit, întru Care am binevoit".
\par 18 ?i acest glas noi l-am auzit, pogorându-se din cer, pe când eram cu Domnul în muntele cel sfânt.
\par 19 ?i avem cuvântul proorocesc mai întarit, la care bine face?i luând aminte, ca la o faclie ce straluce?te în loc întunecos, pâna când va straluci ziua ?i Luceafarul va rasari în inimile voastre,
\par 20 Aceasta ?tiind mai dinainte ca nici o proorocie a Scripturii nu se tâlcuie?te dupa socotin?a fiecaruia;
\par 21 Pentru ca niciodata proorocia nu s-a facut din voia omului, ci oamenii cei sfin?i ai lui Dumnezeu au grait, purta?i fiind de Duhul Sfânt.

\chapter{2}

\par 1 Dar au fost în popor ?i prooroci mincino?i, dupa cum ?i între voi vor fi înva?atori mincino?i, care vor strecura eresuri pierzatoare ?i, tagaduind chiar pe Stapânul Care i-a rascumparat, î?i vor aduce lor grabnica pieire;
\par 2 ?i mul?i se vor lua dupa înva?aturile lor ratacite ?i, din pricina lor, calea adevarului va fi hulita;
\par 3 ?i din pofta de avere ?i cu cuvinte amagitoare, ei va vor momi pe voi. Dar osânda lor, de mult pregatita, nu zabove?te ?i pierzarea lor nu dormiteaza.
\par 4 Caci daca Dumnezeu n-a cru?at pe îngerii care au pacatuit, ci, legându-i cu legaturile întunericului în iad, i-a dat sa fie pazi?i spre judecata,
\par 5 ?i n-a cru?at lumea veche, ci a pastrat numai pe Noe, ca al optulea propovaduitor al drepta?ii, când a adus potopul peste cei fara de credin?a,
\par 6 ?i ceta?ile Sodomei ?i Gomorei, osândindu-le la nimicire, le-a prefacut în cenu?a, dându-le ca o pilda nelegiui?ilor din viitor;
\par 7 Iar pe dreptul Lot, chinuit de petrecerea în desfrânare a celor nelegiui?i, l-a izbavit,
\par 8 Pentru ca dreptul acesta, locuind între ei, prin ce vedea ?i auzea, zi de zi, chinuia sufletul sau cel drept, din pricina faptelor lor nelegiuite.
\par 9 Domnul poate sa scape din ispite pe cei credincio?i, iar pe cei nedrep?i sa-i pastreze, ca sa fie pedepsi?i în ziua judeca?ii,
\par 10 ?i mai vârtos pe cei ce umbla dupa îmboldirile carnii, în pofte spurcate ?i dispre?uiesc domnia cereasca. Îndrazne?i, îngâmfa?i, ei nu se cutremura sa huleasca maririle (din cer),
\par 11 Pe când îngerii, de?i sunt mai mari în tarie ?i în putere, nu aduc în fa?a Domnului judecata defaimatoare împotriva lor.
\par 12 Ace?tia însa, ca ni?te dobitoace fara minte, din fire facute sa fie prinse ?i nimicite, hulind cele ce nu cunosc vor pieri în stricaciunea lor;
\par 13 Ei în?i?i fiind nedrep?i î?i vor lua plata nedrepta?ii, socotind o placere desfatarea de fiecare zi; ei sunt pete ?i ocara, facându-?i placere, în ratacirile lor, sa ospateze cu voi la mesele voastre;
\par 14 Având ochii plini de pofta desfrânarii ?i fiind nesa?io?i de pacat, ei amagesc sufletele cele nestatornice; inima lor e deprinsa la lacomie ?i sunt fiii blestemului.
\par 15 Parasind calea cea dreapta, au ratacit ?i au apucat calea lui Balaam, fiul lui Bosor, care a iubit plata nedrepta?ii,
\par 16 Dar a primit mustrare pentru calcarea lui de lege; caci dobitocul fara grai, pe care era calare, graind cu glas omenesc, a oprit nebunia proorocului.
\par 17 Ace?tia sunt izvoare fara de apa ?i nori purta?i fara de furtuna, carora li se pastreaza, în veac, întunericul cel de nepatruns,
\par 18 Caci rostind vorbe trufa?e ?i de?arte, ei momesc întru poftele trupului, cu desfrânari, pe cei care de abia au scapat de cei ce vie?uiesc în ratacire.
\par 19 Ei le fagaduiesc libertate, fiind ei în?i?i robii stricaciunii, fiindca ceea ce te biruie?te, aceea te ?i stapâne?te.
\par 20 Caci daca, dupa ce au scapat de întinaciunile lumii, prin cunoa?terea Domnului ?i Mântuitorului nostru Iisus Hristos, iara?i se încurca în acestea, ei sunt învin?i; li s-au facut cele de pe urma mai rele decât cele dintâi.
\par 21 Caci mai bine era pentru ei sa nu fi cunoscut calea drepta?ii, decât, dupa ce au cunoscut-o, sa se întoarca de la porunca sfânta, data lor.
\par 22 Cu ei s-a întâmplat adevarul din zicala: Câinele se întoarce la varsatura lui ?i porcul scaldat la noroiul mocirlei lui.

\chapter{3}

\par 1 Iubi?ilor, aceasta este acum a doua epistola pe care v-o scriu. În ele caut sa trezesc, în amintirea voastra, dreapta voastra judecata,
\par 2 Ca sa va aduce?i aminte de cuvintele cele mai înainte graite de sfin?ii prooroci ?i de porunca Domnului ?i Mântuitorului, data prin apostolii vo?tri.
\par 3 Întâi, trebuie sa ?ti?i ca, în zilele cele de apoi, vor veni, cu batjocura, batjocoritori care vor umbla dupa poftele lor,
\par 4 ?i vor zice: Unde este fagaduin?a venirii Lui? Ca de când au adormit parin?ii, toate a?a ramân, ca de la începutul fapturii.
\par 5 Caci ei în chip voit uita aceasta, ca cerurile erau de demult ?i ca pamântul s-a închegat, la cuvântul Domnului, din apa ?i prin apa,
\par 6 ?i prin apa lumea de atunci a pierit înecata,
\par 7 Iar cerurile de acum ?i pamântul sunt ?inute prin acela?i cuvânt ?i pastrate pentru focul din ziua judeca?ii ?i a pieirii oamenilor necredincio?i.
\par 8 ?i aceasta una sa nu va ramâna ascunsa, iubi?ilor, ca o singura zi, înaintea Domnului, este ca o mie de ani ?i o mie de ani ca o zi.
\par 9 Domnul nu întârzie cu fagaduin?a Sa, dupa cum socotesc unii ca e întârziere, ci îndelung rabda pentru voi, nevrând sa piara cineva, ci to?i sa vina la pocain?a.
\par 10 Iar ziua Domnului va veni ca un fur, când cerurile vor pieri cu vuiet mare, stihiile, arzând, se vor desface, ?i pamântul ?i lucrurile de pe el se vor mistui.
\par 11 Deci daca toate acestea se vor desfiin?a, cât de mult vi se cuvine voua sa umbla?i întru via?a sfânta ?i în cucernicie,
\par 12 A?teptând ?i grabind venirea zilei Domnului, din pricina careia cerurile, luând foc, se vor nimici, iar stihiile, aprinse, se vor topi!
\par 13 Dar noi a?teptam, potrivit fagaduin?elor Lui, ceruri noi ?i pamânt nou, în care locuie?te dreptatea.
\par 14 Pentru aceea, iubi?ilor, a?teptând acestea, sârgui?i-va sa fi?i afla?i de El în pace, fara prihana ?i fara vina.
\par 15 ?i îndelunga-rabdare a Domnului nostru socoti?i-o drept mântuire, precum v-a scris ?i iubitul nostru frate Pavel dupa în?elepciunea data lui,
\par 16 Cum vorbe?te despre acestea, în toate epistolele sale, în care sunt unele lucruri cu anevoie de în?eles, pe care cei ne?tiutori ?i neîntari?i le rastalmacesc, ca ?i pe celelalte Scripturi, spre a lor pierzare.
\par 17 Deci voi, iubi?ilor, cunoscând acestea de mai înainte, pazi?i-va, ca nu cumva, lasându-va târâ?i de ratacirea celor fara de lege, sa cade?i din întarirea voastra,
\par 18 Ci cre?te?i în har ?i în cunoa?terea Domnului nostru ?i Mântuitorului Iisus Hristos. A Lui este slava, acum ?i în ziua veacului! Amin.


\end{document}