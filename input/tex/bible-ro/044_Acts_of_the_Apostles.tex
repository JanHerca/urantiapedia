\begin{document}

\title{Faptele Apostolilor}


\chapter{1}

\par 1 Cuvântul cel dintâi l-am făcut o, Teofile, despre toate cele ce a început Iisus a face și a învăța,
\par 2 Până în ziua în care S-a înălțat la cer, poruncind prin Duhul Sfânt apostolilor pe care i-a ales,
\par 3 Cărora S-a și înfățișat pe Sine viu după patima Sa prin multe semne doveditoare, arătându-li-Se timp de patruzeci de zile și vorbind cele despre împărăția lui Dumnezeu.
\par 4 Și cu ei petrecând, le-a poruncit să nu se depărteze de Ierusalim, ci să aștepte făgăduința Tatălui, pe care (a zis El) ați auzit-o de la Mine:
\par 5 Că Ioan a botezat cu apă, iar voi veți fi botezați cu Duhul Sfânt, nu mult după aceste zile.
\par 6 Iar ei, adunându-se, Îl întrebau, zicând: Doamne, oare, în acest timp vei așeza Tu, la loc, împărăția lui Israel?
\par 7 El a zis către ei: Nu este al vostru a ști anii sau vremile pe care Tatăl le-a pus în stăpânirea Sa,
\par 8 Ci veți lua putere, venind Duhul Sfânt peste voi, și Îmi veți fi Mie martori în Ierusalim și în toată Iudeea și în Samaria și până la marginea pământului.
\par 9 Și acestea zicând, pe când ei priveau, S-a înălțat și un nor L-a luat de la ochii lor.
\par 10 Și privind ei, pe când El mergea la cer, iată doi bărbați au stat lângă ei, îmbrăcați în haine albe,
\par 11 Care au și zis: Bărbați galileieni, de ce stați privind la cer? Acest Iisus care S-a înălțat de la voi la cer, astfel va și veni, precum L-ați văzut mergând la cer.
\par 12 Atunci ei s-au întors la Ierusalim de la muntele ce se cheamă al Măslinilor, care este aproape de Ierusalim, cale de o sâmbătă.
\par 13 Și când au intrat, s-au suit în încăperea de sus, unde se adunau de obicei: Petru și Ioan și Iacov și Andrei, Filip și Toma, Bartolomeu și Matei, Iacov al lui Alfeu și Simon Zelotul și Iuda al lui Iacov.
\par 14 Toți aceștia, într-un cuget, stăruiau în rugăciune împreună cu femeile și cu Maria, mama lui Iisus și cu frații Lui.
\par 15 Și în zilele acelea, sculându-se Petru în mijlocul fraților (iar numărul lor era ca la o sută douăzeci), a zis:
\par 16 Bărbați frați, trebuia să se împlinească Scriptura aceasta pe care Duhul Sfânt, prin gura lui David, a spus-o dinainte despre Iuda, care s-a făcut călăuză celor ce L-au prins pe Iisus.
\par 17 Căci era numărat cu noi și luase sorțul acestei slujiri.
\par 18 Deci acesta a dobândit o țarină din plata nedreptății și, căzând cu capul înainte, a crăpat pe la mijloc și i s-au vărsat toate măruntaiele.
\par 19 Și s-a făcut cunoscută aceasta tuturor celor ce locuiesc în Ierusalim, încât țarina aceasta s-a numit în limba lor Hacheldamah, adică Țarina Sângelui.
\par 20 Căci este scris în Cartea Psalmilor: "Facă-se casa lui pustie și să nu aibă cine să locuiască în ea! Și slujirea lui s-o ia altul".
\par 21 Deci trebuie ca unul din acești bărbați, care s-au adunat cu noi în timpul cât a petrecut între noi Domnul Iisus,
\par 22 Începând de la botezul lui Ioan, până în ziua în care S-a înălțat de la noi, să fie împreună cu noi martor al învierii Lui.
\par 23 Și au pus înainte pe doi: pe Iosif, numit Barsaba, zis și Iustus, și pe Matia.
\par 24 Și, rugându-se, au zis: Tu, Doamne, Care cunoști inimile tuturor, arată pe care din aceștia doi l-ai ales,
\par 25 Ca să ia locul acestei slujiri și al apostoliei din care Iuda a căzut, ca să meargă în locul lui.
\par 26 Și au tras la sorți, și sorțul a căzut pe Matia, și s-a socotit împreună cu cei unsprezece apostoli.

\chapter{2}

\par 1 Și când a sosit ziua Cincizecimii, erau toți împreună în același loc.
\par 2 Și din cer, fără de veste, s-a făcut un vuiet, ca de suflare de vânt ce vine repede, și a umplut toată casa unde ședeau ei.
\par 3 Și li s-au arătat, împărțite, limbi ca de foc și au șezut pe fiecare dintre ei.
\par 4 Și s-au umplut toți de Duhul Sfânt și au început să vorbească în alte limbi, precum le dădea lor Duhul a grăi.
\par 5 Și erau în Ierusalim locuitori iudei, bărbați cucernici, din toate neamurile care sunt sub cer.
\par 6 Și iscându-se vuietul acela, s-a adunat mulțimea și s-a tulburat, căci fiecare îi auzea pe ei vorbind în limba sa.
\par 7 Și erau uimiți toți și se minunau zicând: Iată, nu sunt aceștia care vorbesc toți galileieni?
\par 8 Și cum auzim noi fiecare limba noastră, în care ne-am născut?
\par 9 Parți și mezi și elamiți și cei ce locuiesc în Mesopotamia, în Iudeea și în Capadocia, în Pont și în Asia,
\par 10 În Frigia și în Pamfilia, în Egipt și în părțile Libiei cea de lângă Cirene, și romani în treacăt, iudei și prozeliți,
\par 11 Cretani și arabi, îi auzim pe ei vorbind în limbile noastre despre faptele minunate ale lui Dumnezeu!
\par 12 Și toți erau uimiți și nu se dumireau, zicând unul către altul: Ce va să fie aceasta?
\par 13 Iar alții batjocorindu-i, ziceau că sunt plini de must.
\par 14 Și stând Petru cu cei unsprezece, a ridicat glasul și le-a vorbit: Bărbați iudei, și toți care locuiți în Ierusalim, aceasta să vă fie cunoscută și luați în urechi cuvintele mele;
\par 15 Că aceștia nu sunt beți, cum vi se pare vouă, căci este al treilea ceas din zi;
\par 16 Ci aceasta este ce s-a spus prin proorocul Ioil:
\par 17 "Iar în zilele din urmă, zice Domnul, voi turna din Duhul Meu peste tot trupul și fiii voștri și fiicele voastre vor prooroci și cei mai tineri ai voștri vor vedea vedenii și bătrânii voștri vise vor visa.
\par 18 Încă și peste slugile Mele și peste slujnicele Mele voi turna în acele zile, din Duhul Meu și vor prooroci.
\par 19 Și minuni voi face sus în cer și jos pe pământ semne: sânge, foc și fumegare de fum.
\par 20 Soarele se va schimba în întuneric și luna în sânge, înainte de a veni ziua Domnului, cea mare și strălucită.
\par 21 Și tot cel ce va chema numele Domnului se va mântui".
\par 22 Bărbați israeliți, ascultați cuvintele acestea: Pe Iisus Nazarineanul, bărbat adeverit între voi de Dumnezeu, prin puteri, prin minuni și prin semne pe care le-a făcut prin El Dumnezeu în mijlocul vostru, precum și voi știți,
\par 23 Pe Acesta, fiind dat, după sfatul cel rânduit și după știința cea dinainte a lui Dumnezeu, voi L-ați luat și, pironindu-L, prin mâinile celor fără de lege, L-ați omorât,
\par 24 Pe Care Dumnezeu L-a înviat, dezlegând durerile morții, întrucât nu era cu putință ca El să fie ținut de ea.
\par 25 Căci David zice despre El: "Totdeauna am văzut pe Domnul înaintea mea, căci El este de-a dreapta mea, ca să nu mă clatin.
\par 26 De aceea s-a bucurat inima mea și s-a veselit limba mea; chiar și trupul meu se va odihni întru nădejde.
\par 27 Căci nu vei lăsa sufletul meu în iad, nici nu vei da pe cel sfânt al Tău să vadă stricăciune.
\par 28 Făcutu-mi-ai cunoscute căile vieții; cu înfățișarea Ta mă vei umple de bucurie".
\par 29 Bărbați frați, cuvine-se a vorbi cu îndrăznire către voi despre strămoșul David, că a murit și s-a îngropat, iar mormântul lui este la noi, până în ziua aceasta.
\par 30 Deci el, fiind prooroc și știind că Dumnezeu i S-a jurat cu jurământ să așeze pe tronu-i din rodul coapselor lui,
\par 31 Mai înainte văzând, a vorbit despre învierea lui Hristos: că n-a fost lăsat în iad sufletul Lui și nici trupul Lui n-a văzut putreziciunea.
\par 32 Dumnezeu a înviat pe Acest Iisus, Căruia noi toți suntem martori.
\par 33 Deci, înălțându-Se prin dreapta lui Dumnezeu și primind de la Tatăl făgăduința Duhului Sfânt, L-a revărsat pe Acesta, cum vedeți și auziți voi.
\par 34 Căci David nu s-a suit la ceruri, dar el a zis: "Zis-a Domnul Domnului meu: Șezi de-a dreapta Mea,
\par 35 Până ce voi pune pe vrăjmașii Tăi așternut picioarelor Tale".
\par 36 Cu siguranță să știe deci toată casa lui Israel că Dumnezeu, pe Acest Iisus pe Care voi L-ați răstignit, L-a făcut Domn și Hristos.
\par 37 Ei auzind acestea, au fost pătrunși la inimă și au zis către Petru și ceilalți apostoli: Bărbați frați, ce să facem?
\par 38 Iar Petru a zis către ei: Pocăiți-vă și să se boteze fiecare dintre voi în numele lui Iisus Hristos, spre iertarea păcatelor voastre, și veți primi darul Duhului Sfânt.
\par 39 Căci vouă este dată făgăduința și copiilor voștri și tuturor celor de departe, pe oricâți îi va chema Domnul Dumnezeul nostru.
\par 40 Și cu alte mai multe vorbe mărturisea și-i îndemna, zicând: Mântuiți-vă de acest neam viclean.
\par 41 Deci cei ce au primit cuvântul lui s-au botezat și în ziua aceea s-au adăugat ca la trei mii de suflete.
\par 42 Și stăruiau în învățătura apostolilor și în împărtășire, în frângerea pâinii și în rugăciuni.
\par 43 Și tot sufletul era cuprins de teamă, căci multe minuni și semne se făceau în Ierusalim prin apostoli, și mare frică îi stăpânea pe toți.
\par 44 Iar toți cei ce credeau erau laolaltă și aveau toate de obște.
\par 45 Și își vindeau bunurile și averile și le împărțeau tuturor, după cum avea nevoie fiecare.
\par 46 Și în fiecare zi, stăruiau într-un cuget în templu și, frângând pâinea în casă, luau împreună hrana întru bucurie și întru curăția inimii.
\par 47 Lăudând pe Dumnezeu și având har la tot poporul. Iar Domnul adăuga zilnic Bisericii pe cei ce se mântuiau.

\chapter{3}

\par 1 Iar Petru și Ioan se suiau la templu pentru rugăciunea din ceasul al nouălea.
\par 2 Și era un bărbat olog din pântecele mamei sale, pe care-l aduceau și-l puneau în fiecare zi la poarta templului, zisă Poarta Frumoasă, ca să ceară milostenie de la cei ce intrau în templu,
\par 3 Care, văzând că Petru și Ioan vor să intre în templu, le-a cerut milostenie.
\par 4 Iar Petru, căutând spre el, împreună cu Ioan, a zis: Privește noi;
\par 5 Iar el se uita la ei cu luare-aminte, așteptând să primească ceva de la ei.
\par 6 Iar Petru a zis: Argint și aur nu am; dar ce am, aceea îți dau. În numele lui Iisus Hristos Nazarineanul, scoală-te și umblă!
\par 7 Și apucându-l de mâna dreaptă, l-a ridicat și îndată gleznele și tălpile picioarelor lui s-au întărit.
\par 8 Și sărind, a stat în picioare și umbla, și a intrat cu ei în templu, umblând și sărind și lăudând pe Dumnezeu.
\par 9 Și tot poporul l-a văzut umblând și lăudând pe Dumnezeu.
\par 10 Și îl cunoșteau că el era cel care ședea pentru milostenie, la Poarta Frumoasă a templului, și s-au umplut de uimire și de mirare pentru ceea ce i s-a întâmplat.
\par 11 Și ținându-se el de Petru și de Ioan, tot poporul, uimit, alerga la ei, în pridvorul numit al lui Solomon.
\par 12 Iar Petru, văzând aceasta, a răspuns către popor: Bărbați israeliți, de ce vă mirați de acest lucru, sau de ce stați cu ochii ațintiți la noi, ca și cum cu a noastră putere sau cucernicie l-am fi făcut pe acesta să umble?
\par 13 Dumnezeul lui Avraam și al lui Isaac și al lui Iacov, Dumnezeul părinților noștri a slăvit pe Fiul Său Iisus, pe Care voi L-ați predat și L-ați tăgăduit în fața lui Pilat, care găsise cu cale să-L elibereze.
\par 14 Dar voi v-ați lepădat de Cel sfânt și drept și ați cerut să vă dăruiască un bărbat ucigaș.
\par 15 Iar pe Începătorul vieții L-ați omorât, pe Care însă Dumnezeu L-a înviat din morți și ai Cărui martori suntem noi.
\par 16 Și prin credința în numele Lui, pe acesta pe care îl vedeți și îl cunoașteți, l-a întărit numele lui Iisus și credința cea întru El i-a dat lui întregirea aceasta a trupului, înaintea voastră, a tuturor.
\par 17 Și acum, fraților, știu că din neștiință ați făcut rău ca și mai-marii voștri.
\par 18 Dar Dumnezeu a împlinit astfel cele ce vestise dinainte prin gura tuturor proorocilor, că Hristosul Său va pătimi.
\par 19 Deci pocăiți-vă și vă întoarceți, ca să se șteargă păcatele voastre,
\par 20 Ca să vină de la fața Domnului vremuri de ușurare și ca să vă trimită pe Cel mai dinainte vestit vouă, pe Iisus Hristos,
\par 21 Pe Care trebuie să-L primească Cerul până la vremile stabilirii din nou a tuturor celor despre care a vorbit Dumnezeu prin gura sfinților Săi prooroci din veac.
\par 22 Moise a zis către părinți: "Domnul Dumnezeu va ridica vouă dintre frații voștri Prooroc ca mine. Pe El să-L ascultați în toate câte vă va spune.
\par 23 Și tot sufletul care nu va asculta de Proorocul Acela, va fi nimicit din popor".
\par 24 Iar toți proorocii de la Samuel și cei câți le-au urmat au vorbit și au vestit zilele acestea.
\par 25 Voi sunteți fiii proorocilor și ai legământului pe care l-a încheiat Dumnezeu cu părinții noștri, grăind către Avraam: "Și întru seminția ta se vor binecuvânta toate neamurile pământului".
\par 26 Dumnezeu, înviind pe Fiul Său, L-a trimis întâi la voi, să vă binecuvânteze, ca fiecare să se întoarcă de la răutățile sale.

\chapter{4}

\par 1 Dar pe când vorbeau ei către popor, au venit peste ei preoții, căpetenia gărzii templului și saducheii,
\par 2 Mâniindu-se că ei învață poporul și vestesc întru Iisus învierea din morți.
\par 3 Și punând mâna pe ei, i-au pus sub pază, până a doua zi, căci acum era seară.
\par 4 Totuși mulți din cei ce auziseră cuvântul au crezut și numărul bărbaților credincioși s-a făcut ca la cinci mii.
\par 5 Și a doua zi s-au adunat căpeteniile lor și bătrânii și cărturarii din Ierusalim,
\par 6 Și Anna arhiereul și Caiafa și Ioan și Alexandru și câți erau din neamul arhieresc,
\par 7 Și punându-i în mijloc, îi întrebau: Cu ce putere sau în al cui nume ați făcut voi aceasta?
\par 8 Atunci Petru, plin fiind de Duhul Sfânt, le-a vorbit: Căpetenii ale poporului și bătrâni ai lui Israel,
\par 9 Fiindcă noi suntem astăzi cercetați pentru facere de bine unui om bolnav, prin cine a fost el vindecat,
\par 10 Cunoscut să vă fie vouă tuturor, și la tot poporul Israel, că în numele lui Iisus Hristos Nazarineanul, pe Care voi L-ați răstignit, dar pe Care Dumnezeu L-a înviat din morți, întru Acela stă acesta sănătos înaintea voastră!
\par 11 Acesta este piatra cea neluată în seamă de către voi, zidarii, care a ajuns în capul unghiului;
\par 12 Și întru nimeni altul nu este mântuirea, căci nu este sub cer nici un alt nume, dat între oameni, în care trebuie să ne mântuim noi.
\par 13 Și văzând ei îndrăzneala lui Petru și a lui Ioan și știind că sunt oameni fără carte și simpli, se mirau, și îi cunoșteau că fuseseră împreună cu Iisus;
\par 14 Și văzând pe omul cel tămăduit, stând cu ei, n-aveau nimic de zis împotrivă,
\par 15 Dar poruncindu-le să iasă afară din sinedriu, vorbeau între ei,
\par 16 Zicând: ce vom face acestor oameni? Căci este învederat tuturor celor ce locuiesc în Ierusalim că prin ei s-a făcut o minune cunoscută și nu putem să tăgăduim.
\par 17 Dar ca aceasta să nu se răspândească mai mult în popor, să le poruncim cu amenințare să nu mai vorbească, în numele acesta, nici unui om.
\par 18 Și chemându-i, le-au poruncit ca nicidecum să nu mai grăiască, nici să mai învețe în numele lui Iisus.
\par 19 Iar Petru și Ioan, răspunzând, au zis către ei: Judecați dacă este drept înaintea lui Dumnezeu să ascultăm de voi mai mult decât de Dumnezeu.
\par 20 Căci noi nu putem să nu vorbim cele ce am văzut și am auzit.
\par 21 Dar ei, amenințându-i din nou, le-au dat drumul, negăsind nici un chip cum să-i pedepsească, din cauza poporului, fiindcă toți slăveau pe Dumnezeu, pentru ceea ce se făcuse.
\par 22 Căci omul cu care se făcuse această minune a vindecării avea mai mult ca patruzeci de ani.
\par 23 Fiind sloboziți, au venit la ai lor și le-au spus câte le-au vorbit lor arhiereii și bătrânii.
\par 24 Iar ei, auzind, într-un cuget au ridicat glasul către Dumnezeu și au zis: Stăpâne, Dumnezeule, Tu, Care ai făcut cerul și pământul și marea și toate cele ce sunt în ele,
\par 25 Care prin Duhul Sfânt și prin gura părintelui nostru David, slujitorul Tău, ai zis: Pentru ce s-au întărâtat neamurile și popoarele au cugetat cele deșarte?
\par 26 Ridicatu-s-au regii pământului și căpeteniile s-au adunat laolaltă împotriva Domnului și împotriva Unsului Lui,
\par 27 Căci asupra Sfântului Tău Fiu Iisus, pe care Tu L-ai uns, s-au adunat laolaltă, cu adevărat, în cetatea aceasta, și Irod și Pontius Pilat cu păgânii și cu popoarele lui Israel,
\par 28 Ca să facă toate câte mâna Ta și sfatul Tău mai dinainte au rânduit să fie.
\par 29 Și acum, Doamne, caută spre amenințările lor și dă robilor Tăi să grăiască cuvântul Tău cu toată îndrăzneala,
\par 30 Întinzând dreapta Ta spre vindecare și săvârșind semne și minuni, prin numele Sfântului Tău Fiu Iisus.
\par 31 Și pe când se rugau astfel, s-a cutremurat locul în care erau adunați, și s-au umplut toți de Duhul Sfânt și grăiau cu îndrăzneală cuvântul lui Dumnezeu.
\par 32 Iar inima și sufletul mulțimii celor ce au crezut erau una și nici unul nu zicea că este al său ceva din averea sa, ci toate le erau de obște.
\par 33 Și cu mare putere apostolii mărturiseau despre învierea Domnului Iisus Hristos și mare har era peste ei toți.
\par 34 Și nimeni nu era între ei lipsit, fiindcă toți câți aveau țarini sau case le vindeau și aduceau prețul celor vândute,
\par 35 Și-l puneau la picioarele apostolilor. Și se împărțea fiecăruia după cum avea cineva trebuință.
\par 36 Iar Iosif, cel numit de apostoli Barnaba (care se tâlcuiește fiul mângâierii), un levit, născut în Cipru,
\par 37 Având țarină și vânzând-o, a adus banii și i-a pus la picioarele apostolilor.

\chapter{5}

\par 1 Iar un om, anume Anania, cu Safira, femeia lui, și-a vândut țarina.
\par 2 Și a dosit din preț, știind și femeia lui, și aducând o parte, a pus-o la picioarele apostolilor.
\par 3 Iar Petru a zis: Anania, de ce a umplut satana inima ta, ca să minți tu Duhului Sfânt și să dosești din prețul țarinei?
\par 4 Oare, păstrând-o, nu-ți rămânea ție, și vândută nu era în stăpânirea ta? Pentru ce ai pus în inima ta lucrul acesta? N-ai mințit oamenilor, ci lui Dumnezeu.
\par 5 Iar Anania, auzind aceste cuvinte, a căzut și a murit. Și frică mare i-a cuprins pe toți care au auzit.
\par 6 Și sculându-se cei mai tineri, l-au înfășurat și, scoțându-l afară, l-au îngropat.
\par 7 După un răstimp, ca de trei ceasuri, a intrat și femeia lui, neștiind ce se întâmplase.
\par 8 Iar Petru a zis către ea: Spune-mi dacă ați vândut țarina cu atât? Iar ea a zis: Da, cu atât.
\par 9 Iar Petru a zis către ea: De ce v-ați învoit voi să ispitiți Duhul Domnului? Iată picioarele celor ce au îngropat pe bărbatul tău sunt la ușă și te vor scoate afară și pe tine.
\par 10 Și ea a căzut îndată la picioarele lui Petru și a murit. Și intrând tinerii, au găsit-o moartă și, scoțând-o afară, au îngropat-o lângă bărbatul ei.
\par 11 Și frică mare a cuprins toată Biserica și pe toți care au auzit acestea.
\par 12 Iar prin mâinile apostolilor se făceau semne și minuni multe în popor, și erau toți, într-un cuget, în pridvorul lui Solomon.
\par 13 Și nimeni dintre ceilalți nu cuteza să se alipească de ei, dar poporul îi lăuda.
\par 14 Și din ce în ce mai mult se adăugau cei ce credeau în Domnul, mulțime de bărbați și de femei,
\par 15 Încât scoteau pe cei bolnavi în ulițe și-i puneau pe paturi și pe tărgi, ca venind Petru, măcar umbra lui să umbrească pe vreunul dintre ei.
\par 16 Și se aduna și mulțimea din cetățile dimprejurul Ierusalimului, aducând bolnavi și bântuiți de duhuri necurate, și toți se vindecau.
\par 17 Și sculându-se arhiereul și toți cei împreună cu el - cei din eresul saducheilor - s-au umplut de pizmă.
\par 18 Și au pus mâna pe apostoli și i-au băgat în temnița obștească.
\par 19 Iar un înger al Domnului, în timpul nopții, a deschis ușile temniței și, scoțându-i, le-a zis:
\par 20 Mergeți și, stând, grăiți poporului în templu toate cuvintele vieții acesteia.
\par 21 Și, auzind, au intrat de dimineață în templu și învățau. Dar venind arhiereul și cei împreună cu el, au adunat sinedriul și tot sfatul bătrânilor fiilor lui Israel și au trimis la temniță să-i aducă pe apostoli.
\par 22 Dar, ducându-se, slugile nu i-au găsit în temniță și, întorcându-se, au vestit,
\par 23 Zicând: Temnița am găsit-o încuiată în toată siguranța și pe paznici stând la uși, dar când am descuiat, înăuntru n-am găsit pe nimeni.
\par 24 Când au auzit aceste cuvinte, căpetenia pazei templului și arhiereii erau nedumeriți cu privire la ei, ce-ar putea să fie aceasta.
\par 25 Dar venind cineva, le-a dat de veste: Iată, bărbații pe care i-ați pus în temniță sunt în templu, stând și învățând poporul.
\par 26 Atunci, ducându-se, căpetenia pazei templului împreună cu slujitorii i-au adus dar nu cu sila, că se temeau de popor să nu-i omoare cu pietre.
\par 27 Și, aducându-i, i-au pus în fața sinedriului. Iar arhiereul i-a întrebat,
\par 28 Zicând: Au nu v-am poruncit vouă cu poruncă să nu mai învățați în numele acesta? Și iată ați umplut Ierusalimul cu învățătura voastră și voiți să aduceți asupra noastră sângele Acestui Om!
\par 29 Iar Petru și apostolii, răspunzând, au zis: Trebuie să ascultăm pe Dumnezeu mai mult decât pe oameni.
\par 30 Dumnezeul părinților noștri a înviat pe Iisus, pe Care voi L-ați omorât, spânzurându-L pe lemn.
\par 31 Pe Acesta, Dumnezeu, prin dreapta Sa, L-a înălțat Stăpânitor și Mântuitor, ca să dea lui Israel pocăință și iertarea păcatelor.
\par 32 Și suntem martori ai acestor cuvinte noi și Duhul Sfânt, pe Care Dumnezeu L-a dat celor ce Îl ascultă.
\par 33 Iar ei, auzind, se mâniau foarte și se sfătuiau să-i omoare.
\par 34 Și ridicându-se în sinedriu un fariseu, anume Gamaliel, învățător de Lege, cinstit de tot poporul, a poruncit să-i scoată pe oameni afară puțin,
\par 35 Și a zis către ei: Bărbați israeliți, luați aminte la voi, ce aveți să faceți cu acești oameni.
\par 36 Că înainte de zilele acestea s-a ridicat Teudas, zicând că el este cineva, căruia i s-au alăturat un număr de bărbați ca la patru sute, care a fost ucis și toți câți l-au ascultat au fost risipiți și nimiciți.
\par 37 După aceasta s-a ridicat Iuda Galileianul, în vremea numărătorii, și a atras popor mult după el; și acela a pierit și toți câți au ascultat de el au fost împrăștiați.
\par 38 Și acum zic vouă: Feriți-vă de oamenii aceștia și lăsați-i, căci dacă această hotărâre sau lucrul acesta este de la oameni, se va nimici;
\par 39 Iar dacă este de la Dumnezeu, nu veți putea să-i nimiciți, ca nu cumva să vă aflați și luptători împotriva lui Dumnezeu.
\par 40 Și l-au ascultat pe el; și chemând pe apostoli și bătându-i, le-au poruncit să nu mai vorbească în numele lui Iisus, și le-au dat drumul.
\par 41 Iar ei au plecat din fața sinedriului, bucurându-se că s-au învrednicit, pentru numele Lui, să sufere ocară.
\par 42 Și toată ziua, în templu și prin case, nu încetau să învețe și să binevestească pe Hristos Iisus.

\chapter{6}

\par 1 În zilele acelea, înmulțindu-se ucenicii, eleniștii (iudei) murmurau împotriva evreilor, pentru că văduvele lor erau trecute cu vederea la slujirea cea de fiecare zi.
\par 2 Și chemând cei doisprezece mulțimea ucenicilor, au zis: Nu este drept ca noi, lăsând de-o parte cuvântul lui Dumnezeu, să slujim la mese.
\par 3 Drept aceea, fraților, căutați șapte bărbați dintre voi, cu nume bun, plini de Duh Sfânt și de înțelepciune, pe care noi să-i rânduim la această slujbă.
\par 4 Iar noi vom stărui în rugăciune și în slujirea cuvântului.
\par 5 Și a plăcut cuvântul înaintea întregii mulțimi, și au ales pe Ștefan, bărbat plin de credință și de Duh Sfânt, și pe Filip, și pe Prohor, și pe Nicanor, și pe Timon, și pe Parmena, și pe Nicolae, prozelit din Antiohia,
\par 6 Pe care i-au pus înaintea apostolilor, și ei, rugându-se și-au pus mâinile peste ei.
\par 7 Și cuvântul lui Dumnezeu creștea, și se înmulțea foarte numărul ucenicilor în Ierusalim, încă și mulțime de preoți se supuneau credinței.
\par 8 Iar Ștefan, plin de har și de putere, făcea minuni și semne mari în popor.
\par 9 Și s-au ridicat unii din sinagoga ce se zicea a libertinilor și a cirenenilor și a alexandrinilor și a celor din Cilicia și din Asia, sfădindu-se cu Ștefan.
\par 10 Și nu puteau să stea împotriva înțelepciunii și a Duhului cu care el vorbea.
\par 11 Atunci au pus pe niște bărbați să zică: L-am auzit spunând cuvinte de hulă împotriva lui Moise și a lui Dumnezeu.
\par 12 Și au întărâtat poporul și pe bătrâni și pe cărturari și, năvălind asupră-i, l-au răpit și l-au dus în sinedriu.
\par 13 Și au pus martori mincinoși, care ziceau: Acest om nu încetează a vorbi cuvinte de hulă împotriva acestui loc sfânt și a Legii.
\par 14 Că l-au auzit zicând că Acest Iisus Nazarineanul va strica locul acesta și va schimba datinile pe care ni le-a lăsat nouă Moise.
\par 15 Și ațintindu-și ochii asupra lui, toți cei ce ședeau în sinedriu au văzut fața lui ca o față de înger.

\chapter{7}

\par 1 Și a zis arhiereul: Adevărate sunt acestea?
\par 2 Iar el a zis: Bărbați frați și părinți, ascultați! Dumnezeul slavei S-a arătat părintelui nostru Avraam, când era în Mesopotamia, mai înainte de a locui în Haran,
\par 3 Și a zis către el: Ieși din pământul tău și din rudenia ta și vino în pământul pe care ți-l voi arăta.
\par 4 Atunci, ieșind din pământul caldeilor, a locuit în Haran. Iar de acolo, după moartea tatălui său, l-a strămutat în această țară, în care locuiți voi acum,
\par 5 Și nu i-a dat moștenire în ea nici o palmă de pământ, ci i-a făgăduit că i-o va da lui spre stăpânire și urmașilor lui după el, neavând el copil.
\par 6 Și Dumnezeu a vorbit astfel: "Urmașii lui vor fi străini în pământ străin, și acolo îi vor robi și-i vor asupri patru sute de ani.
\par 7 Și pe poporul la care vor fi robi, Eu îl voi judeca - a zis Dumnezeu - iar după acestea vor ieși și-Mi vor sluji Mie în locul acesta".
\par 8 Și i-a dat legământul tăierii împrejur; și așa a născut pe Isaac și l-a tăiat împrejur a opta zi; și Isaac a născut pe Iacov și Iacov pe cei doisprezece patriarhi.
\par 9 Și patriarhii, pizmuind pe Iosif, l-au vândut în Egipt; dar Dumnezeu era cu el,
\par 10 Și l-a scos din toate necazurile lui și i-a dat har și înțelepciune înaintea lui Faraon, regele Egiptului, iar acesta l-a pus mai mare peste Egipt și peste toată casa lui.
\par 11 Și a venit foamete peste tot Egiptul și peste Canaan, și strâmtorare mare, și părinții noștri nu mai găseau hrană.
\par 12 Și Iacov, auzind că este grâu în Egipt, a trimis pe părinții noștri întâia oară.
\par 13 Iar a doua oară Iosif s-a făcut cunoscut fraților săi și Faraon a aflat neamul lui Iosif.
\par 14 Și Iosif, trimițând, a chemat pe Iacov, tatăl său, și toată rudenia sa, cu șaptezeci și cinci de suflete.
\par 15 Și Iacov s-a coborât în Egipt; și a murit și el și părinții noștri;
\par 16 Și au fost strămutați la Sichem și au fost puși în mormântul pe care Avraam l-a cumpărat cu preț de argint, de la fiii lui Emor, în Sichem;
\par 17 Dar cum se apropia vremea făgăduinței pentru care s-a jurat Dumnezeu lui Avraam, a crescut poporul și s-a înmulțit în Egipt,
\par 18 Până când s-a ridicat peste Egipt alt rege, care nu știa de Iosif.
\par 19 Acesta, purtându-se ca un viclean cu neamul nostru, a asuprit pe părinții noștri să-și lepede pruncii lor, ca să nu mai trăiască.
\par 20 În vremea aceea s-a născut Moise și era plăcut lui Dumnezeu. Și trei luni a fost hrănit în casa tatălui său.
\par 21 Și fiind lepădat, l-a luat fiica lui Faraon și l-a crescut ca pe un fiu al ei.
\par 22 Și a fost învățat Moise în toată înțelepciunea egiptenilor și era puternic în cuvintele și în faptele lui.
\par 23 Iar când a împlinit patruzeci de ani, și-a pus în gând să cerceteze pe frații săi, fiii lui Israel.
\par 24 Și văzând pe unul dintre ei că suferă strâmbătate, l-a apărat și, omorând pe egiptean, a răzbunat pe cel asuprit.
\par 25 Și el credea că frații săi vor pricepe că Dumnezeu, prin mâna lui, le dăruiește izbăvire, dar ei n-au înțeles.
\par 26 Și a doua zi s-a arătat unora care se băteau și i-a îndemnat la pace, zicând: Bărbaților, sunteți frați. De ce vă faceți rău unul altuia?
\par 27 Dar cel ce asuprea pe aproapele l-a îmbrâncit, zicând: Cine te-a pus pe tine domn și judecător peste noi?
\par 28 Nu cumva vrei să mă omori, cum ai omorât ieri pe egiptean?
\par 29 La acest cuvânt, Moise a fugit și a trăit ca străin în țara Madian, unde a născut doi fii.
\par 30 Și după ce s-au împlinit patruzeci de ani, îngerul Domnului i s-a arătat în pustiul Muntelui Sinai, în flacăra focului unui rug.
\par 31 Iar Moise, văzând, s-a minunat de vedenie, dar când s-a apropiat ca să ia seama mai bine, a fost glasul Domnului către el:
\par 32 "Eu sunt Dumnezeul părinților tăi, Dumnezeul lui Avraam și Dumnezeul lui Isaac și Dumnezeul lui Iacov". Și Moise, tremurând, nu îndrăznea să privească;
\par 33 Iar Domnul i-a zis: "Dezleagă încălțămintea picioarelor tale, căci locul pe care stai este pământ sfânt.
\par 34 Privind, am văzut asuprirea poporului Meu în Egipt și suspinul lor l-am auzit și M-am pogorât ca să-i scot. Și acum vino, să te trimit în Egipt".
\par 35 Pe Moise acesta de care s-au lepădat, zicând: Cine te-a pus pe tine domn și judecător?, pe acesta l-a trimis Dumnezeu domn și răscumpărător, prin mâna îngerului care i se arătase lui în rug.
\par 36 Acesta i-a scos afară, făcând minuni și semne în țara Egiptului și în Marea Roșie și în pustie, timp de patruzeci de ani.
\par 37 Acesta este Moise cel ce a zis fiilor lui Israel: "Prooroc ca mine vă va ridica Dumnezeu din frații voștri; pe EL să-L ascultați".
\par 38 Acesta este cel ce a fost la adunarea în pustie, cu îngerul care i-a vorbit pe Muntele Sinai și cu părinții noștri, primind cuvinte de viață ca să ni le dea nouă.
\par 39 De acesta n-au voit să asculte părinții noștri, ci l-au lepădat și inimile lor s-au întors către Egipt,
\par 40 Zicând lui Aaron: "Fă-ne dumnezei care să meargă înaintea noastră; căci acestui Moise, care ne-a scos din țara Egiptului, nu știm ce i s-a întâmplat".
\par 41 Și au făcut, în zilele acelea, un vițel și au adus idolului jertfă și se veseleau de lucrurile mâinilor lor.
\par 42 Și S-a întors Dumnezeu și i-a dat pe ei să slujească oștirii cerului, precum este scris în cartea proorocilor: "Adus-ați voi Mie, casă a lui Israel, timp de patruzeci de ani, în pustie, junghieri și jertfe?
\par 43 Și ați purtat cortul lui Moloh și steaua dumnezeului vostru Remfan, chipurile pe care le-ați făcut, ca să vă închinați la ele! De aceea vă voi strămuta dincolo de Babilon".
\par 44 Părinții noștri aveau în pustie cortul mărturiei, precum orânduise Cel ce a vorbit cu Moise, ca să-l facă după chipul pe care îl văzuse;
\par 45 Și pe acesta primindu-l, părinții noștri l-au adus cu Iosua în țara stăpânită de neamuri, pe care Dumnezeu le-a izgonit din fața părinților noștri, până în zilele lui David,
\par 46 Care a aflat har înaintea lui Dumnezeu și a cerut să găsească un locaș pentru Dumnezeul lui Iacov.
\par 47 Iar Solomon I-a zidit Lui casă,
\par 48 Dar Cel Preaînalt nu locuiește în temple făcute de mâini, precum zice proorocul:
\par 49 "Cerul este tronul Meu și pământul așternut picioarelor Mele. Ce casă Îmi veți zidi Mie? - zice Domnul - sau care este locul odihnei Mele?
\par 50 Nu mâna Mea a făcut toate acestea?"
\par 51 Voi cei tari în cerbice și netăiați împrejur la inimă și la urechi, voi pururea stați împotriva Duhului Sfânt, precum Și părinții voștri, așa și voi!
\par 52 Pe care dintre prooroci nu l-au prigonit părinții voștri? Și au ucis pe cei ce au vestit mai dinainte sosirea Celui Drept, ai Cărui vânzători și ucigași v-ați făcut voi acum,
\par 53 Voi, care ați primit Legea întru rânduieli de îngeri și n-ați păzit-o!
\par 54 Iar ei, auzind acestea, fremătau de furie în inimile lor și scrâșneau din dinți împotriva lui.
\par 55 Iar Ștefan, fiind plin de Duh Sfânt și privind la cer, a văzut slava lui Dumnezeu și pe Iisus stând de-a dreapta lui Dumnezeu.
\par 56 Și a zis: Iată, văd cerurile deschise și pe Fiul Omului stând de-a dreapta lui Dumnezeu!
\par 57 Iar ei, strigând cu glas mare, și-au astupat urechile și au năvălit asupra lui.
\par 58 Și scoțându-l afară din cetate, îl băteau cu pietre. Iar martorii și-au pus hainele la picioarele unui tânăr, numit Saul.
\par 59 Și îl băteau cu pietre pe Ștefan, care se ruga și zicea: Doamne, Iisuse, primește duhul meu!
\par 60 Și, îngenunchind, a strigat cu glas mare: Doamne, nu le socoti lor păcatul acesta! Și zicând acestea, a murit.

\chapter{8}

\par 1 Și Saul s-a făcut părtaș la uciderea lui. Și s-a făcut în ziua aceea prigoană mare împotriva Bisericii din Ierusalim. Și toți, afară de apostoli, s-au împrăștiat prin ținuturile Iudeii și ale Samariei.
\par 2 Iar bărbați cucernici au îngropat pe Ștefan și au făcut plângere mare pentru el.
\par 3 Și Saul pustiia Biserica, intrând prin case și, târând pe bărbați și pe femei, îi preda la temniță.
\par 4 Iar cei ce se împrăștiaseră străbăteau țara, binevestind cuvântul.
\par 5 Iar Filip, coborându-se într-o cetate a Samariei, le propovăduia pe Hristos.
\par 6 Și mulțimile luau aminte într-un cuget la cele spuse de către Filip, ascultându-l și văzând semnele pe care le săvârșea.
\par 7 Căci din mulți care aveau duhuri necurate, strigând cu glas mare, ele ieșeau și mulți slăbănogi și șchiopi s-au vindecat.
\par 8 Și s-a făcut mare bucurie în cetatea aceea.
\par 9 Dar era mai dinainte în cetate un bărbat, anume Simon, vrăjind și uimind neamul Samariei, zicând că el este cineva mare,
\par 10 La care luau aminte toți, de la mic până la mare, zicând: Acesta este puterea lui Dumnezeu, numită cea mare.
\par 11 Și luau aminte la el, fiindcă de multă vreme, cu vrăjile lui, îi uimise.
\par 12 Iar când au crezut lui Filip, care le propovăduia despre împărăția lui Dumnezeu și despre numele lui Iisus Hristos, bărbați și femei se botezau.
\par 13 Iar Simon a crezut și el și, botezându-se, era mereu cu Filip. Și văzând semnele și minunile mari ce se făceau, era uimit.
\par 14 Iar apostolii din Ierusalim, auzind că Samaria a primit cuvântul lui Dumnezeu, au trimis la ei pe Petru și pe Ioan,
\par 15 Care, coborând, s-au rugat pentru ei, ca să primească Duhul Sfânt.
\par 16 Căci nu Se pogorâse încă peste nici unul dintre ei, ci erau numai botezați în numele Domnului Iisus.
\par 17 Atunci își puneau mâinile peste ei, și ei luau Duhul Sfânt.
\par 18 Și Simon văzând că prin punerea mâinilor apostolilor se dă Duhul Sfânt, le-a adus bani,
\par 19 Zicând: Dați-mi și mie puterea aceasta, ca acela pe care voi pune mâinile să primească Duhul Sfânt.
\par 20 Iar Petru a zis către el: Banii tăi să fie cu tine spre pierzare! Căci ai socotit că darul lui Dumnezeu se agonisește cu bani.
\par 21 Tu n-ai parte, nici moștenire, la chemarea aceasta, pentru că inima ta nu este dreaptă înaintea lui Dumnezeu.
\par 22 Pocăiește-te deci de această răutate a ta și te roagă lui Dumnezeu, doară ți se va ierta cugetul inimii tale,
\par 23 Căci întru amărăciunea fierii și întru legătura nedreptății te văd că ești.
\par 24 Și răspunzând, Simon a zis: Rugați-vă voi la Domnul, pentru mine, ca să nu vină asupra mea nimic din cele ce ați zis.
\par 25 Iar ei, mărturisind și grăind cuvântul Domnului, s-au întors la Ierusalim și în multe sate ale samarinenilor binevesteau.
\par 26 Și un înger al Domnului a grăit către Filip, zicând: Ridică-te și mergi spre miazăzi, pe calea care coboară de la Ierusalim la Gaza; aceasta este pustie.
\par 27 Ridicându-se, a mers. Și iată un bărbat din Etiopia, famen, mare dregător al Candachiei, regina Etiopiei, care era peste toată vistieria ei și care venise la Ierusalim ca să se închine,
\par 28 Se întorcea acasă; și șezând în carul său, citea pe proorocul Isaia.
\par 29 Iar Duhul i-a zis lui Filip: Apropie-te și te alipește de carul acesta.
\par 30 Și alergând Filip l-a auzit citind pe proorocul Isaia, și i-a zis: Înțelegi, oare, ce citești?
\par 31 Iar el a zis: Cum aș putea să înțeleg, dacă nu mă va călăuzi cineva? Și a rugat pe Filip să se urce și să șadă cu el.
\par 32 Iar locul din Scriptură pe care-l citea era acesta: "Ca un miel care se aduce spre junghiere și ca o oaie fără de glas înaintea celui ce-o tunde, așa nu și-a deschis gura sa.
\par 33 Întru smerenia Lui, judecata Lui s-a ridicat și neamul Lui cine-l va spune? Că se ridică de pe pământ viața Lui".
\par 34 Iar famenul, răspunzând, a zis lui Filip: Rogu-te, despre cine zice proorocul acesta, despre sine ori despre altcineva?
\par 35 Iar Filip, deschizând gura sa și începând de la scriptura aceasta, i-a binevestit pe Iisus.
\par 36 Și, pe când mergeau pe cale, au ajuns la o apă; iar famenul a zis: Iată apă. Ce mă împiedică să fiu botezat?
\par 37 Filip a zis: Dacă crezi din toată inima, este cu putință. Și el, răspunzând, a zis: Cred că Iisus Hristos este Fiul lui Dumnezeu.
\par 38 Și a poruncit să stea carul; și s-au coborât amândoi în apă, și Filip și famenul, și l-a botezat.
\par 39 Iar când au ieșit din apă, Duhul Domnului a răpit pe Filip, și famenul nu l-a mai văzut. Și el s-a dus în calea sa, bucurându-se.
\par 40 Iar Filip s-a aflat în Azot și, mergând, binevestea prin toate cetățile, până ce a sosit în Cezareea.

\chapter{9}

\par 1 Iar Saul, suflând încă amenințare și ucidere împotriva ucenicilor Domnului, a mers la arhiereu,
\par 2 Și a cerut de la el scrisori către sinagogile din Damasc ca, dacă va afla acolo pe vreunii, atât bărbați, cât și femei, că merg pe calea aceasta, să-i aducă legați la Ierusalim.
\par 3 Dar pe când călătorea el și se apropia de Damasc, o lumină din cer, ca de fulger, l-a învăluit deodată.
\par 4 Și, căzând la pământ, a auzit un glas, zicându-i: Saule, Saule, de ce Mă prigonești?
\par 5 Iar el a zis: Cine ești, Doamne? Și Domnul a zis: Eu sunt Iisus, pe Care tu Îl prigonești. Greu îți este să izbești cu piciorul în țepușă.
\par 6 Și el, tremurând și înspăimântat fiind, a zis: Doamne, ce voiești să fac? Iar Domnul i-a zis: Ridică-te, intră în cetate și ți se va spune ce trebuie să faci.
\par 7 Iar bărbații, care erau cu el pe cale, stăteau înmărmuriți, auzind glasul, dar nevăzând pe nimeni.
\par 8 Și s-a ridicat Saul de la pământ, dar, deși avea ochii deschiși, nu vedea nimic. Și luându-l de mână, l-au dus în Damasc.
\par 9 Și trei zile a fost fără vedere; și n-a mâncat, nici n-a băut.
\par 10 Și era în Damasc un ucenic, anume Anania, și Domnul i-a zis în vedenie: Anania! Iar el a zis: Iată-mă, Doamne;
\par 11 Și Domnul a zis către el: Sculându-te, mergi pe ulița care se cheamă Ulița Dreaptă și caută în casa lui Iuda, pe un om din Tars, cu numele Saul; că, iată, se roagă.
\par 12 Și a văzut în vedenie pe un bărbat, anume Anania, intrând la el și punându-și mâinile peste el, ca să vadă iarăși.
\par 13 Și a răspuns Anania: Doamne, despre bărbatul acesta am auzit de la mulți câte rele a făcut sfinților Tăi în Ierusalim.
\par 14 Și aici are putere de la arhierei să lege pe toți care cheamă numele Tău.
\par 15 Și a zis Domnul către el: Mergi, fiindcă acesta Îmi este vas ales, ca să poarte numele Meu înaintea neamurilor și a regilor și a fiilor lui Israel;
\par 16 Căci Eu îi voi arăta câte trebuie să pătimească el pentru numele Meu.
\par 17 Și a mers Anania și a intrat în casă și, punându-și mâinile pe el, a zis: Frate Saul, Domnul Iisus, Cel ce ți S-a arătat pe calea pe care tu veneai, m-a trimis ca să vezi iarăși și să te umpli de Duh Sfânt.
\par 18 Și îndată au căzut de pe ochii lui ca niște solzi; și a văzut iarăși și, sculându-se, a fost botezat.
\par 19 Și luând mâncare, s-a întărit. Și a stat cu ucenicii din Damasc câteva zile.
\par 20 Apoi propovăduia în sinagogi pe Iisus, că Acesta este Fiul lui Dumnezeu.
\par 21 Și se mirau toți care îl auzeau și ziceau: Nu este, oare, acesta cel care prigonea în Ierusalim pe cei ce cheamă acest nume și a venit aici pentru aceea ca să-i ducă pe ei legați la arhierei?
\par 22 Și Saul se întărea mai mult și tulbura pe iudeii care locuiau în Damasc, dovedind că Acesta este Hristos.
\par 23 Și după ce au trecut destule zile, iudeii s-au sfătuit să-l omoare.
\par 24 Și s-a făcut cunoscut lui Saul vicleșugul lor. Și ei păzeau porțile și ziua și noaptea, ca să-l ucidă.
\par 25 Și luându-l ucenicii lui noaptea, l-au coborât peste zid, lăsându-l jos într-un coș.
\par 26 Și venind la Ierusalim, Saul încerca să se alipească de ucenici; și toți se temeau de el, necrezând că este ucenic.
\par 27 Iar Barnaba, luându-l pe el, l-a dus la apostoli și le-a istorisit cum a văzut pe cale pe Domnul și că El i-a vorbit lui și cum a propovăduit la Damasc, cu îndrăzneală în numele lui Iisus.
\par 28 Și era cu ei intrând și ieșind în Ierusalim și propovăduia cu îndrăzneală în numele Domnului.
\par 29 Și vorbea și se sfădea cu eleniștii, iar ei căutau să-l ucidă.
\par 30 Dar frații, aflând aceasta, l-au dus pe Saul la Cezareea și de acolo l-au trimis la Tars.
\par 31 Deci Biserica, în toată Iudeea și Galileea și Samaria, avea pace, zidindu-se și umblând în frica de Domnul, și sporea prin mângâierea Duhului Sfânt.
\par 32 Și trecând Petru pe la toți, a coborât și la sfinții care locuiau în Lida.
\par 33 Și acolo a găsit pe un om, anume Enea, care de opt ani zăcea în pat, fiindcă era paralitic.
\par 34 Și Petru i-a zis: Enea, te vindecă Iisus Hristos. Ridică-te și strânge-ți patul. Și îndată s-a ridicat.
\par 35 Și l-au văzut toți cei ce locuiau în Lida și în Saron, care s-au și întors la Domnul.
\par 36 Iar în Iope era o uceniță, cu numele Tavita, care, tâlcuindu-se, se zice Căprioară. Aceasta era plină de fapte bune și de milosteniile pe care le făcea.
\par 37 Și în zilele acelea ea s-a îmbolnăvit și a murit. Și, scăldând-o, au pus-o în camera de sus.
\par 38 Și fiind aproape Lida de Iope, ucenicii, auzind că Petru este în Lida, au trimis pe doi bărbați la el, rugându-l: Nu pregeta să vii până la noi.
\par 39 Și Petru, sculându-se, a venit cu ei. Când a sosit, l-au dus în camera de sus și l-au înconjurat toate văduvele, plângând și arătând cămășile și hainele câte le făcea Căprioara, pe când era cu ele.
\par 40 Și Petru, scoțând afară pe toți, a îngenunchiat și s-a rugat și, întorcându-se către trup, a zis: Tavita scoală-te! Iar ea și-a deschis ochii și, văzând pe Petru, a șezut.
\par 41 Și dându-i mâna, Petru a ridicat-o și, chemând pe sfinți și pe văduve, le-a dat-o vie.
\par 42 Și s-a făcut cunoscută aceasta în întreaga Iope și mulți au crezut în Domnul.
\par 43 Și el a rămas în Iope multe zile, la un oarecare Simon, tăbăcar.

\chapter{10}

\par 1 Iar în Cezareea era un bărbat cu numele Corneliu, sutaș, din cohorta ce se chema Italica,
\par 2 Cucernic și temător de Dumnezeu, cu toată casa lui și care făcea multe milostenii poporului și se ruga lui Dumnezeu totdeauna.
\par 3 Și a văzut în vedenie, lămurit, cam pe la ceasul al nouălea din zi, un înger al lui Dumnezeu, intrând la el și zicându-i: Corneliu!
\par 4 Iar Corneliu, căutând spre el și înfricoșându-se, a zis: Ce este, Doamne? Și îngerul i-a zis: Rugăciunile tale și milosteniile tale s-au suit, spre pomenire, înaintea lui Dumnezeu.
\par 5 Și acum, trimite bărbați la Iope și cheamă să vină un oarecare Simon, care se numește și Petru.
\par 6 El este găzduit la un om oarecare Simon, tăbăcar, a cărui casă este lângă mare. Acesta îți va spune ce să faci.
\par 7 Și după ce s-a dus îngerul care vorbea cu el, Corneliu a chemat două din slugile sale de casă și pe un ostaș cucernic din cei ce îi erau mai apropiați.
\par 8 Și după ce le-a istorisit toate, i-a trimis la Iope.
\par 9 Iar a doua zi, pe când ei mergeau pe drum și se apropiau de cetate, Petru s-a suit pe acoperiș, să se roage pe la ceasul al șaselea.
\par 10 Și i s-a făcut foame și voia să mănânce, dar, pe când ei îi pregăteau să mănânce, a căzut în extaz.
\par 11 Și a văzut cerul deschis și coborându-se ceva ca o față mare de pânză, legată în patru colțuri, lăsându-se pe pământ.
\par 12 În ea erau toate dobitoacele cu patru picioare și târâtoarele pământului și păsările cerului.
\par 13 Și glas a fost către el: Sculându-te, Petre, junghie și mănâncă.
\par 14 Iar Petru a zis: Nicidecum, Doamne, căci niciodată n-am mâncat nimic spurcat și necurat.
\par 15 Și iarăși, a doua oară, a fost glas către el: Cele ce Dumnezeu a curățit, tu să nu le numești spurcate.
\par 16 Și aceasta s-a făcut de trei ori și îndată acel ceva s-a ridicat la cer.
\par 17 Pe când Petru nu se dumirea întru sine ce ar putea să fie vedenia pe care o văzuse, iată bărbații cei trimiși de Corneliu, întrebând de casa lui Simon, s-au oprit la poartă,
\par 18 Și, strigând, întrebau dacă Simon, numit Petru, este găzduit acolo.
\par 19 Și tot gândindu-se Petru la vedenie, Duhul i-a zis: Iată, trei bărbați te caută;
\par 20 Ci sculându-te, coboară-te și mergi împreună cu ei, de nimic îndoindu-te, fiindcă Eu i-am trimis.
\par 21 Și Petru, coborându-se la bărbații trimiși la el de Corneliu, le-a zis: Iată, eu sunt acela pe care îl căutați. Care este pricina pentru care ați venit?
\par 22 Iar ei au zis: Corneliu sutașul, om drept și temător de Dumnezeu și mărturisit de tot neamul iudeilor, a fost înștiințat de către un sfânt înger să trimită să te cheme acasă la el, ca să audă cuvinte de la tine.
\par 23 Deci, chemându-i înăuntru, i-a găzduit. Iar a doua zi, sculându-se, a plecat împreună cu ei; iar câțiva din frații cei din Iope l-au însoțit.
\par 24 Și în ziua următoare au intrat în Cezareea. Iar Corneliu îi aștepta și chemase acasă la el rudeniile sale și prietenii cei mai de aproape.
\par 25 Și când a fost să intre Petru, Corneliu, întâmpinându-l, i s-a închinat, căzând la picioarele lui.
\par 26 Iar Petru l-a ridicat, zicându-i: Scoală-te. Și eu sunt om.
\par 27 Și, vorbind cu el, a intrat și a găsit pe mulți adunați.
\par 28 Și a zis către ei: Voi știți că nu se cuvine unui bărbat iudeu să se unească sau să se apropie de cel de alt neam, dar mie Dumnezeu mi-a arătat să nu numesc pe nici un om spurcat sau necurat.
\par 29 De aceea, chemat fiind să vin, am venit fără împotrivire. Deci vă întreb: Pentru care cuvânt ați trimis după mine?
\par 30 Corneliu a zis: Acum patru zile eram postind până la ceasul acesta și mă rugam în casa mea, în ceasul al nouălea, și iată un bărbat în haină strălucitoare a stat în fața mea.
\par 31 Și el a zis: Corneliu, a fost ascultată rugăciunea ta și milosteniile tale au fost pomenite înaintea lui Dumnezeu.
\par 32 Trimite, deci, la Iope și cheamă pe Simon, cel ce se numește Petru; el este găzduit în casa lui Simon, tăbăcarul, lângă mare.
\par 33 Deci îndată am trimis la tine; și tu ai făcut bine că ai venit. Și acum noi toți suntem de față înaintea lui Dumnezeu, ca să ascultăm toate cele poruncite ție de Domnul.
\par 34 Iar Petru, deschizându-și gura, a zis: Cu adevărat înțeleg că Dumnezeu nu este părtinitor.
\par 35 Ci, în orice neam, cel ce se teme de El și face dreptate este primit de El.
\par 36 Și El a trimis fiilor lui Israel cuvântul, binevestind pacea prin Iisus Hristos: Acesta este Domn peste toate.
\par 37 Voi știți cuvântul care a fost în toată Iudeea, începând din Galileea, după botezul pe care l-a propovăduit Ioan.
\par 38 (Adică despre) Iisus din Nazaret, cum a uns Dumnezeu cu Duhul Sfânt și cu putere pe Acesta care a umblat făcând bine și vindecând pe toți cei asupriți de diavolul, pentru că Dumnezeu era cu El.
\par 39 Și noi suntem martori pentru toate cele ce a făcut El în țara iudeilor și în Ierusalim; pe Acesta L-au omorât, spânzurându-L pe lemn.
\par 40 Dar Dumnezeu L-a înviat a treia zi și I-a dat să Se arate,
\par 41 Nu la tot poporul, ci nouă martorilor, dinainte rânduiți de Dumnezeu, care am mâncat și am băut cu El, după învierea Lui din morți.
\par 42 Și ne-a poruncit să propovăduim poporului și să mărturisim că El este Cel rânduit de Dumnezeu să fie judecător al celor vii și al celor morți.
\par 43 Despre Acesta mărturisesc toți proorocii, că tot cel ce crede în El va primi iertarea păcatelor, prin numele Lui.
\par 44 Încă pe când Petru vorbea aceste cuvinte, Duhul Sfânt a căzut peste toți cei care ascultau cuvântul.
\par 45 Iar credincioșii tăiați împrejur, care veniseră cu Petru, au rămas uimiți pentru că darul Duhului Sfânt s-a revărsat și peste neamuri.
\par 46 Căci îi auzeau pe ei vorbind în limbi și slăvind pe Dumnezeu. Atunci a răspuns Petru:
\par 47 Poate, oare, cineva să oprească apa, ca să nu fie botezați aceștia care au primit Duhul Sfânt ca și noi?
\par 48 Și a poruncit ca aceștia să fie botezați în numele lui Iisus Hristos. Atunci l-au rugat pe Petru să rămână la ei câteva zile.

\chapter{11}

\par 1 Apostolii și frații, care erau prin Iudeea, au auzit că și păgânii au primit cuvântul lui Dumnezeu.
\par 2 Și când Petru s-a suit în Ierusalim, credincioșii tăiați împrejur se împotriveau,
\par 3 Zicându-i: Ai intrat la oameni netăiați împrejur și ai mâncat cu ei.
\par 4 Și începând, Petru le-a înfățișat pe rând, zicând:
\par 5 Eu eram în cetatea Iope și mă rugam; și am văzut, în extaz, o vedenie: ceva coborându-se ca o față mare de pânză, legată în patru colțuri, lăsându-se în jos din cer, și a venit până la mine.
\par 6 Privind spre aceasta, cu luare aminte, am văzut dobitoacele cele cu patru picioare ale pământului și fiarele și târâtoarele și păsările cerului.
\par 7 Și am auzit un glas, care-mi zicea: Sculându-te, Petre, junghie și mănâncă.
\par 8 Și am zis: Nicidecum, Doamne, căci nimic spurcat sau necurat n-a intrat vreodată în gura mea.
\par 9 Și glasul mi-a grăit a doua oară din cer: Cele ce Dumnezeu a curățit, tu să nu le numești spurcate.
\par 10 Și aceasta s-a făcut de trei ori și au fost luate iarăși toate în cer.
\par 11 Și iată, îndată, trei bărbați, trimiși de la Cezareea către mine, s-au oprit la casa în care eram.
\par 12 Iar Duhul mi-a zis să merg cu ei, de nimic îndoindu-mă. Și au mers cu mine și acești șase frați și am intrat în casa bărbatului;
\par 13 Și el ne-a povestit cum a văzut îngerul stând în casa lui și zicând: Trimite la Iope și cheamă pe Simon, cel numit și Petru.
\par 14 Care va grăi către tine cuvinte, prin care te vei mântui tu și toată casa ta.
\par 15 Și când am început eu să vorbesc, Duhul Sfânt a căzut peste ei, ca și peste noi la început.
\par 16 Și mi-am adus aminte de cuvântul Domnului, când zicea: Ioan a botezat cu apă; voi însă vă veți boteza cu Duh Sfânt.
\par 17 Deci, dacă Dumnezeu a dat lor același dar ca și nouă, acelora care au crezut în Domnul Iisus Hristos, cine eram eu ca să-L pot opri pe Dumnezeu?
\par 18 Auzind acestea, au tăcut și au slăvit pe Dumnezeu, zicând: Așadar și păgânilor le-a dat Dumnezeu pocăința spre viață;
\par 19 Deci cei ce se risipiseră din cauza tulburării făcute pentru Ștefan, au trecut până în Fenicia și în Cipru, și în Antiohia, nimănui grăind cuvântul, decât numai iudeilor.
\par 20 Și erau unii dintre ei, bărbați ciprieni și cireneni care, venind în Antiohia, vorbeau și către elini, binevestind pe Domnul Iisus.
\par 21 Și mâna Domnului era cu ei și era mare numărul celor care au crezut și s-au întors la Domnul.
\par 22 Și vorba despre ei a ajuns la urechile Bisericii din Ierusalim, și au trimis pe Barnaba până la Antiohia.
\par 23 Acesta, sosind și văzând harul lui Dumnezeu, s-a bucurat și îndemna pe toți să rămână în Domnul, cu inimă statornică.
\par 24 Căci era bărbat bun și plin de Duh Sfânt și de credință. Și s-a adăugat Domnului mulțime multă.
\par 25 Și a plecat Barnaba la Tars, ca să caute pe Saul
\par 26 Și aflându-l, l-a adus la Antiohia. Și au stat acolo un an întreg, adunându-se în biserică și învățând mult popor. Și în Antiohia, întâia oară, ucenicii s-au numit creștini.
\par 27 În acele zile s-au coborât, de la Ierusalim în Antiohia, prooroci.
\par 28 Și sculându-se unul dintre ei, cu numele Agav, a arătat prin Duhul, că va fi în toată lumea foamete mare, care a și fost în zilele lui Claudiu.
\par 29 Iar ucenicii au hotărât ca fiecare dintre ei, după putere, să trimită spre ajutorare fraților care locuiau în Iudeea;
\par 30 Ceea ce au și făcut, trimițând preoților prin mâna lui Barnaba și a lui Saul.

\chapter{12}

\par 1 Și în vremea aceea, regele Irod (Agripa) a pus mâna pe unii din Biserică, ca să-i piardă.
\par 2 Și a ucis cu sabia pe Iacov, fratele lui Ioan.
\par 3 Și văzând că este pe placul iudeilor, a mai luat și pe Petru, (și erau zilele Azimelor)
\par 4 Pe care și prinzându-l l-a băgat în temniță, dându-l la patru străji de câte patru ostași, ca să-l păzească, vrând să-l scoată la popor după Paști.
\par 5 Deci Petru era păzit în temniță și se făcea necontenit rugăciune către Dumnezeu pentru el, de către Biserică.
\par 6 Dar când Irod era să-l scoată afară, în noaptea aceea, Petru dormea între doi ostași, legat cu două lanțuri, iar înaintea ușii paznicii păzeau temnița.
\par 7 Și iată un înger al Domnului a venit deodată, iar în cameră a strălucit lumină. Și lovind pe Petru în coastă, îngerul l-a deșteptat, zicând: Scoală-te degrabă! Și lanțurile i-au căzut de la mâini.
\par 8 Și a zis îngerul către el: Încinge-te și încalță-te cu sandalele. Și el a făcut așa. Și i-a zis lui: Pune haina pe tine și vino după mine.
\par 9 Și, ieșind, mergea după înger, dar nu știa că ceea ce s-a făcut prin înger este adevărat, ci i se părea că vede vedenie.
\par 10 Și trecând de straja întâi și de a doua, au ajuns la poarta cea de fier care duce în cetate, și poarta s-a deschis singură. Și ieșind, au trecut o uliță și îndată îngerul s-a depărtat de la el.
\par 11 Și Petru, venindu-și în sine, a zis: Acum știu cu adevărat că Domnul a trimis pe îngerul Său și m-a scos din mâna lui Irod și din toate câte aștepta poporul iudeilor.
\par 12 Și chibzuind, a venit la casa Mariei, mama lui Ioan, cel numit Marcu, unde erau adunați mulți și se rugau.
\par 13 Și bătând Petru la ușa de la poartă, o slujnică cu numele Rodi, s-a dus să asculte.
\par 14 Și recunoscând glasul lui Petru, de bucurie nu a deschis ușa, ci, alergând înăuntru, a spus că Petru stă înaintea porții.
\par 15 Iar ei au zis către ea: Ai înnebunit. Dar ea stăruia că este așa. Iar ei ziceau: Este îngerul lui.
\par 16 Dar Petru bătea mereu în poartă. Și deschizându-i, l-au văzut și au rămas uimiți.
\par 17 Și făcându-le semn cu mâna să tacă, le-a istorisit cum l-a scos Domnul pe el din temniță. Și a zis: Vestiți acestea lui Iacov și fraților. Și ieșind, s-a dus în alt loc.
\par 18 Și făcându-se ziuă, mare a fost tulburarea între ostași: Ce s-a făcut, oare, cu Petru?
\par 19 Iar Irod cerându-l și negăsindu-l, după ce au fost cercetați paznicii, a poruncit să fie uciși. Și el, coborând din Iudeea la Cezareea, a rămas acolo.
\par 20 Și Irod era mânios pe locuitorii din Tir și din Sidon. Dar ei, înțelegându-se între ei, au venit la el și câștigând pe Vlast, care era cămărașul regelui, cereau pace, pentru că țara lor se hrănea din cea a regelui.
\par 21 Și într-o zi rânduită, Irod, îmbrăcându-se în veșminte regești și șezând la tribună, vorbea către ei;
\par 22 Iar poporul striga: Acesta e glas dumnezeiesc, nu omenesc!
\par 23 Și îndată îngerul Domnului l-a lovit, pentru că nu a dat slavă lui Dumnezeu. Și mâncându-l viermii, a murit.
\par 24 Iar cuvântul lui Dumnezeu creștea și se înmulțea.
\par 25 Iar Barnaba și Saul, după ce au îndeplinit slujba lor, s-au întors de la Ierusalim la Antiohia, luând cu ei pe Ioan, cel numit Marcu.

\chapter{13}

\par 1 Și erau în Biserica din Antiohia prooroci și învățători: Barnaba și Simeon, ce se numea Niger, Luciu Cirineul, Manain, cel ce fusese crescut împreună cu Irod tetrarhul, și Saul.
\par 2 Și pe când slujeau Domnului și posteau, Duhul Sfânt a zis: Osebiți-mi pe Barnaba și pe Saul, pentru lucrul la care i-am chemat.
\par 3 Atunci, postind și rugându-se, și-au pus mâinile peste ei și i-au lăsat să plece.
\par 4 Deci, ei, mânați de Duhul Sfânt, au coborât la Seleucia și de acolo au plecat cu corabia la Cipru.
\par 5 Și ajungând în Salamina, au vestit cuvântul lui Dumnezeu în sinagogile iudeilor. Și aveau și pe Ioan slujitor.
\par 6 Și străbătând toată insula până la Pafos, au găsit pe un oarecare bărbat iudeu, vrăjitor, prooroc mincinos, al cărui nume era Bariisus,
\par 7 Care era în preajma proconsulului Sergius Paulus, bărbat înțelept. Acesta chemând la sine pe Barnaba și pe Saul, dorea să audă cuvântul lui Dumnezeu,
\par 8 Dar le stătea împotrivă Elimas vrăjitorul - căci așa se tâlcuiește numele lui - căutând să întoarcă pe proconsul de la credință.
\par 9 Iar Saul - care se numește și Pavel - plin fiind de Duh Sfânt, a privit țintă la el,
\par 10 Și a zis: O, tu cel plin de toată viclenia și de toată înșelăciunea, fiule al diavolului, vrăjmașule a toată dreptatea, nu vei înceta de a strâmba căile Domnului cele drepte?
\par 11 Și acum, iată mâna Domnului este asupra ta și vei fi orb, nevăzând soarele până la o vreme. Și îndată a căzut peste el pâclă și întuneric și, dibuind împrejur, căuta cine să-l ducă de mână.
\par 12 Atunci proconsulul, văzând ce s-a făcut, a crezut, mirându-se foarte de învățătura Domnului.
\par 13 Și plecând cu corabia de la Pafos, Pavel și cei împreună cu el au venit la Perga Pamfiliei. Iar Ioan, despărțindu-se de ei, s-a întors la Ierusalim.
\par 14 Iar ei, trecând de la Perga, au ajuns la Antiohia Pisidiei și, intrând în sinagogă, într-o zi de sâmbătă, au șezut.
\par 15 Și după citirea Legii și a Proorocilor, mai-marii sinagogii au trimis la ei, zicându-le: Bărbați frați, dacă aveți vreun cuvânt de mângâiere către popor, vorbiți.
\par 16 Și, ridicându-se Pavel și făcându-le semn cu mâna, a zis: Bărbați israeliți și cei temători de Dumnezeu, ascultați:
\par 17 Dumnezeul acestui popor al lui Israel a ales pe părinții noștri și pe popor l-a înălțat, când era străin în pământul Egiptului, și cu braț înalt i-a scos de acolo,
\par 18 Și vreme de patruzeci de ani i-a hrănit în pustie.
\par 19 Și nimicind șapte neamuri în țara Canaanului, pământul acela l-a dat lor spre moștenire.
\par 20 Și după acestea, ca la patru sute cincizeci de ani, le-a dat judecători, până la Samuel proorocul.
\par 21 Și de acolo au cerut rege și Dumnezeu le-a dat, timp de patruzeci de ani, pe Saul, fiul lui Chiș, bărbat din seminția lui Veniamin.
\par 22 Și înlăturându-l, le-a ridicat rege pe David, pentru care a zis, mărturisind: "Aflat-am pe David al lui Iesei, bărbat după inima Mea, care va face toate voile Mele".
\par 23 Din urmașii acestuia, Dumnezeu, după făgăduință, i-a adus lui Israel un Mântuitor, pe Iisus,
\par 24 După ce Ioan a propovăduit, înaintea venirii Lui, botezul pocăinței, la tot poporul lui Israel.
\par 25 Iar dacă și-a împlinit Ioan calea sa, zicea: Nu sunt eu ce socotiți voi că sunt. Dar, iată, vine după mine Cel căruia nu sunt vrednic să-I dezleg încălțămintea picioarelor.
\par 26 Bărbați frați, fii din neamul lui Avraam și cei dintre voi temători de Dumnezeu, vouă vi s-a trimis cuvântul acestei mântuiri.
\par 27 Căci cei ce locuiesc în Ierusalim și căpeteniile lor, necunoscându-L și osândindu-L, au împlinit glasurile proorocilor care se citesc în fiecare sâmbătă.
\par 28 Și, neaflând în El nici o vină de moarte, au cerut de la Pilat ca să-L omoare.
\par 29 Iar când au săvârșit toate cele scrise despre El, coborându-L de pe cruce, L-au pus în mormânt.
\par 30 Dar Dumnezeu L-a înviat din morți.
\par 31 El S-a arătat mai multe zile celor ce împreună cu El s-au suit din Galileea la Ierusalim și care sunt acum martorii Lui către popor.
\par 32 Și noi vă binevestim făgăduința făcută părinților.
\par 33 Că pe aceasta Dumnezeu a împlinit-o cu noi, copiii lor, înviindu-L pe Iisus, precum este scris și în Psalmul al doilea: "Fiul Meu ești Tu; Eu astăzi Te-am născut".
\par 34 Și cum că L-a înviat din morți, ca să nu se mai întoarcă la stricăciune, a spus-o altfel: "Vă voi da vouă cele sfinte și vrednice de credință ale lui David".
\par 35 De aceea și în alt loc zice: "Nu vei da pe Sfântul Tău să vadă stricăciune".
\par 36 Pentru că David, slujind în timpul său voii lui Dumnezeu, a adormit și s-a adăugat la părinții lui și a văzut stricăciune.
\par 37 Dar Acela pe Care Dumnezeu L-a înviat n-a văzut stricăciune.
\par 38 Cunoscut deci să vă fie vouă, bărbați frați, că prin Acesta vi se vestește iertarea păcatelor și că, de toate câte n-ați putut să vă îndreptați în Legea lui Moise,
\par 39 Întru Acesta tot cel ce crede se îndreptează.
\par 40 Deci luați aminte să nu vină peste voi ceea ce s-a zis în prooroci:
\par 41 "Vedeți, îngâmfaților, mirați-vă și pieriți, că Eu lucrez un lucru, în zilele voastre, un lucru pe care nu-l veți crede, dacă vă va spune cineva".
\par 42 Și ieșind ei din sinagoga iudeilor, îi rugau neamurile ca sâmbăta viitoare să li se grăiască cuvintele acestea.
\par 43 După ce s-a terminat adunarea, mulți dintre iudei și dintre prozeliții cucernici au mers după Pavel și după Barnaba, care, vorbind către ei, îi îndemnau să stăruie în harul lui Dumnezeu.
\par 44 Iar în sâmbăta următoare, mai toată cetatea s-a adunat ca să audă cuvântul lui Dumnezeu.
\par 45 Dar iudeii, văzând mulțimile, s-au umplut de pizmă și vorbeau împotriva celor spuse de Pavel, hulind.
\par 46 Iar Pavel și Barnaba, îndrăznind, au zis: Vouă se cădea să vi se grăiască, mai întâi, cuvântul lui Dumnezeu; dar de vreme ce îl lepădați și vă judecați pe voi nevrednici de viața veșnică, iată ne întoarcem către neamuri.
\par 47 Căci așa ne-a poruncit nouă Domnul: "Te-am pus spre lumină neamurilor, ca să fii Tu spre mântuire până la marginea pământului".
\par 48 Iar neamurile pământului, auzind, se bucurau și slăveau cuvântul lui Dumnezeu și câți erau rânduiți spre viață veșnică au crezut;
\par 49 Iar cuvântul Domnului se răspândea prin tot ținutul.
\par 50 Dar iudeii au întărâtat pe femeile cucernice și de cinste și pe cei de frunte ai cetății, și au ridicat prigoane împotriva lui Pavel și a lui Barnaba, și i-au scos din hotarele lor.
\par 51 Iar ei, scuturând asupra lor praful de pe picioare, au venit la Iconiu;
\par 52 Și ucenicii se umpleau de bucurie și de Duh Sfânt.

\chapter{14}

\par 1 Și în Iconiu au intrat ei, ca de obicei, în sinagoga iudeilor și astfel au vorbit, încât o mare mulțime de iudei și de elini au crezut.
\par 2 Dar iudeii care n-au crezut au răsculat și au înrăit sufletele păgânilor împotriva fraților.
\par 3 Deci multă vreme au stat acolo, grăind cu îndrăzneală întru Domnul, Care da mărturie pentru cuvântul harului Său, făcând semne și minuni prin mâinile lor.
\par 4 Și mulțimea din cetate s-a dezbinat și unii țineau cu iudeii, iar alții țineau cu apostolii.
\par 5 Și când păgânii și iudeii, împreună cu căpeteniile lor, au dat năvală ca să-i ocărască și să-i ucidă cu pietre,
\par 6 Înțelegând, au fugit în cetățile Licaoniei: la Listra și Derbe și în ținutul dimprejur.
\par 7 Și acolo propovăduiau Evanghelia.
\par 8 Și ședea jos în Listra un om neputincios de picioare, fiind olog, din pântecele maicii sale și care nu umblase niciodată.
\par 9 Acesta asculta la Pavel când vorbea. Iar Pavel, căutând spre el și văzând că are credință ca să se mântuiască,
\par 10 A zis cu glas puternic: Scoală-te drept, pe picioarele tale. Și el a sărit și umbla.
\par 11 Iar mulțimile, văzând ceea ce făcuse Pavel, au ridicat glasul lor în limba licaonă, zicând: Zeii, asemănându-se oamenilor, s-au coborât la noi.
\par 12 Și numeau pe Barnaba Zeus, iar pe Pavel Hermes, fiindcă el era purtătorul cuvântului.
\par 13 Iar preotul lui Zeus, care era înaintea cetății, aducând la porți tauri și cununi, voia să le aducă jertfă împreună cu mulțimile.
\par 14 Și auzind Apostolii Pavel și Barnaba, și-au rupt veșmintele, au sărit în mulțime, strigând,
\par 15 Și zicând: Bărbaților, de ce faceți acestea? Doar și noi suntem oameni, asemenea pătimitori ca voi, binevestind să vă întoarceți de la aceste deșertăciuni către Dumnezeu cel viu, Care a făcut cerul și pământul, marea și toate cele ce sunt în ele,
\par 16 Și Care, în veacurile trecute, a lăsat ca toate neamurile să meargă în căile lor,
\par 17 Deși El nu S-a lăsat pe Sine nemărturisit, făcându-vă bine, dându-vă din cer ploi și timpuri roditoare, umplând de hrană și de bucurie inimile voastre.
\par 18 Și acestea zicând, abia au potolit mulțimile, ca să nu le aducă jertfă.
\par 19 Iar de la Antiohia și de la Iconiu au venit iudei, care au atras mulțimile de partea lor, și, bătând pe Pavel cu pietre, l-au târât afară din cetate, gândind că a murit.
\par 20 Dar înconjurându-l ucenicii, el s-a sculat și a intrat în cetate. Și a doua zi a ieșit cu Barnaba către Derbe.
\par 21 Și binevestind cetății aceleia și făcând ucenici mulți, s-au înapoiat la Listra, la Iconiu și la Antiohia,
\par 22 Întărind sufletele ucenicilor, îndemnându-i să stăruie în credință și (arătându-le) că prin multe suferințe trebuie să intrăm în împărăția lui Dumnezeu.
\par 23 Și hirotonindu-le preoți în fiecare biserică, rugându-se cu postiri, i-au încredințat pe ei Domnului în Care crezuseră.
\par 24 Și străbătând Pisidia, au venit în Pamfilia.
\par 25 Și după ce au grăit cuvântul Domnului în Perga, au coborât la Atalia.
\par 26 Și de acolo au mers cu corabia spre Antiohia, de unde fuseseră încredințați harului lui Dumnezeu, spre lucrul pe care l-au împlinit.
\par 27 Și venind și adunând Biserica, au vestit câte a făcut Dumnezeu cu ei și că a deschis păgânilor ușa credinței.
\par 28 Și au petrecut acolo cu ucenicii nu puțină vreme.

\chapter{15}

\par 1 Și unii, coborându-se din Iudeea, învățau pe frați că: Dacă nu vă tăiați împrejur, după rânduiala lui Moise, nu puteți să vă mântuiți.
\par 2 Și făcându-se pentru ei împotrivire și discuție nu puțină cu Pavel și Barnaba, au rânduit ca Pavel și Barnaba și alți câțiva dintre ei să se suie la apostolii și la preoții din Ierusalim pentru această întrebare.
\par 3 Deci ei, trimiși fiind de Biserică, au trecut prin Fenicia și prin Samaria, istorisind despre convertirea neamurilor și făceau tuturor fraților mare bucurie.
\par 4 Și sosind ei la Ierusalim, au fost primiți de Biserică și de apostoli și de preoți și au vestit câte a făcut Dumnezeu cu ei.
\par 5 Dar unii din eresul fariseilor, care trecuseră la credință, s-au ridicat zicând că trebuie să-i taie împrejur și să le poruncească a păzi Legea lui Moise.
\par 6 Și apostolii și preoții s-au adunat ca să cerceteze despre acest cuvânt.
\par 7 Și făcându-se multă vorbire, s-a sculat Petru și le-a zis: Bărbați frați, voi știți că, din primele zile, Dumnezeu m-a ales între voi, ca prin gura mea neamurile să audă cuvântul Evangheliei și să creadă.
\par 8 Și Dumnezeu, Cel ce cunoaște inimile, le-a mărturisit, dându-le Duhul Sfânt, ca și nouă.
\par 9 Și nimic n-a deosebit între noi și ei, curățind inimile lor prin credință.
\par 10 Acum deci, de ce ispitiți pe Dumnezeu și vreți să puneți pe grumazul ucenicilor un jug pe care nici părinții noștri, nici noi n-am putut să-l purtăm?
\par 11 Ci prin harul Domnului nostru Iisus Hristos, credem că ne vom mântui în același chip ca și aceia.
\par 12 Și a tăcut toată mulțimea și asculta pe Barnaba și pe Pavel, care istoriseau câte semne și minuni a făcut Dumnezeu prin ei între neamuri.
\par 13 Și după ce au tăcut ei, a răspuns Iacob, zicând: Bărbați frați, ascultați-mă!
\par 14 Simon a istorisit cum de la început a avut grijă Dumnezeu să ia dintre neamuri un popor pentru numele Său.
\par 15 Și cu aceasta se potrivesc cuvintele proorocilor, precum este scris:
\par 16 "După acestea Mă voi întoarce și voi ridica iarăși cortul cel căzut al lui David și cele distruse ale lui iarăși le voi zidi și-l voi îndrepta,
\par 17 Ca să-L caute pe Domnul ceilalți oameni și toate neamurile peste care s-a chemat numele Meu asupra lor, zice Domnul, Cel ce a făcut acestea".
\par 18 Lui Dumnezeu Îi sunt cunoscute din veac lucrurile Lui.
\par 19 De aceea eu socotesc să nu tulburăm pe cei ce, dintre neamuri, se întorc la Dumnezeu,
\par 20 Ci să le scriem să se ferească de întinările idolilor și de desfrâu și de (animale) sugrumate și de sânge.
\par 21 Căci Moise are din timpuri vechi prin toate cetățile propovăduitorii săi, fiind citit în sinagogi în fiecare sâmbătă.
\par 22 Atunci apostolii și preoții, cu toată Biserica, au hotărât să aleagă bărbați dintre ei și să-i trimită la Antiohia, cu Pavel și cu Barnaba: pe Iuda cel numit Barsaba, și pe Sila, bărbați cu vază între frați.
\par 23 Scriind prin mâinile lor acestea: Apostolii și preoții și frații, fraților dintre neamuri, care sunt în Antiohia și în Siria și în Cilicia, salutare!
\par 24 Deoarece am auzit că unii dintre noi, fără să fi avut porunca noastră, venind, v-au tulburat cu vorbele lor și au răvășit sufletele voastre, zicând că trebuie să vă tăiați împrejur și să păziți legea,
\par 25 Noi am hotărât, adunați într-un gând, ca să trimitem la voi bărbați aleși, împreună cu iubiții noștri Barnaba și Pavel,
\par 26 Oameni care și-au pus sufletele lor pentru numele Domnului nostru Iisus Hristos.
\par 27 Drept aceea, am trimis pe Iuda și pe Sila, care vă vor vesti și ei, cu cuvântul, aceleași lucruri.
\par 28 Pentru că, părutu-s-a Duhului Sfânt și nouă, să nu vi se pună nici o greutate în plus în afară de cele ce sunt necesare:
\par 29 Să vă feriți de cele jertfite idolilor și de sânge și de (animale) sugrumate și de desfrâu, de care păzindu-vă, bine veți face. Fiți sănătoși!
\par 30 Deci cei trimiși au coborât la Antiohia și, adunând mulțimea, au predat scrisoarea.
\par 31 Și citind-o s-au bucurat pentru mângâiere.
\par 32 Iar Iuda și cu Sila, fiind și ei prooroci, au mângâiat prin multe cuvântări pe frați și i-au întărit.
\par 33 Și petrecând un timp, au fost trimiși cu pace de către frați la apostoli.
\par 34 Iar Sila s-a hotărât să rămână acolo, și Iuda a plecat singur la Ierusalim.
\par 35 Iar Pavel și Barnaba petreceau în Antiohia, învățând și binevestind, împreună cu mulți alții, cuvântul Domnului.
\par 36 Și după câteva zile, Pavel a zis către Barnaba: Întorcându-ne, să cercetăm cum se află frații noștri în toate cetățile în care am vestit cuvântul Domnului.
\par 37 Barnaba voia să ia împreună cu ei și pe Ioan cel numit Marcu;
\par 38 Dar Pavel cerea să nu-l ia pe acesta cu ei, fiindcă se despărțise de ei din Pamfilia și nu venise alături de ei la lucrul la care fuseseră trimiși.
\par 39 Deci s-a iscat neînțelegere între ei, încât s-au despărțit unul de altul, și Barnaba, luând pe Marcu, a plecat cu corabia în Cipru;
\par 40 Iar Pavel, alegând pe Sila, a plecat, fiind încredințat de către frați harului Domnului.
\par 41 Și străbătea Siria și Cilicia, întărind Bisericile.

\chapter{16}

\par 1 Și a sosit la Derbe și la Listra. Și iată era acolo un ucenic cu numele Timotei, fiul unei femei iudee credincioase, și al unui tată elin,
\par 2 Care avea bune mărturii de la frații din Listra și din Iconiu.
\par 3 Pavel a voit ca acesta să vină împreună cu el și, luându-l, l-a tăiat împrejur, din pricina iudeilor care erau în acele locuri; căci toți știau că tatăl lui era elin.
\par 4 Și când treceau prin cetăți, învățau să păzească învățăturile rânduite de apostolii și de preoții din Ierusalim.
\par 5 Deci Bisericile se întăreau în credință și sporeau cu numărul în fiecare zi.
\par 6 Și ei au străbătut Frigia și ținutul Galatiei, opriți fiind de Duhul Sfânt ca să propovăduiască cuvântul în Asia.
\par 7 Venind la hotarele Misiei, încercau să meargă în Bitinia, dar Duhul lui Iisus nu i-a lăsat.
\par 8 Și trecând dincolo de Misia, au coborât la Troa.
\par 9 Și noaptea i s-a arătat lui Pavel o vedenie: Un bărbat macedonean sta rugându-l și zicând: Treci în Macedonia și ne ajută.
\par 10 Când a văzut el această vedenie, am căutat să plecăm îndată în Macedonia, înțelegând că Dumnezeu ne cheamă să le vestim Evanghelia.
\par 11 Pornind cu corabia de la Troa, am mers drept la Samotracia, iar a doua zi la Neapoli,
\par 12 Și de acolo la Filipi, care este cea dintâi cetate a acestei părți a Macedoniei și colonie romană. Iar în această cetate am rămas câteva zile.
\par 13 Și în ziua sâmbetei am ieșit în afara porții, lângă râu, unde credeam că este loc de rugăciune și, șezând, vorbeam femeilor care se adunaseră.
\par 14 Și o femeie, cu numele Lidia, vânzătoare de porfiră, din cetatea Tiatirelor, temătoare de Dumnezeu, asculta. Acesteia Dumnezeu i-a deschis inima ca să ia aminte la cele grăite de Pavel.
\par 15 Iar după ce s-a botezat și ea și casa ei, ne-a rugat, zicând: De m-ați socotit că sunt credincioasă Domnului, intrând în casa mea, rămâneți. Și ne-a făcut să rămânem.
\par 16 Dar odată, pe când ne duceam la rugăciune, ne-a întâmpinat o slujnică, care avea duh pitonicesc și care aducea mult câștig stăpânilor ei, ghicind.
\par 17 Aceasta, ținându-se după Pavel și după noi, striga, zicând: Acești oameni sunt robi ai Dumnezeului celui Preaînalt, care vă vestesc vouă calea mântuirii.
\par 18 Și aceasta o făcea timp de multe zile. Iar Pavel mâniindu-se și întorcându-se, a zis duhului: În numele lui Iisus Hristos îți poruncesc să ieși din ea. Și în acel ceas a ieșit.
\par 19 Și stăpânii ei, văzând că s-a dus nădejdea câștigului lor, au pus mâna pe Pavel și pe Sila și i-au în piață înaintea dregătorilor.
\par 20 Și ducându-i la judecători, au zis: Acești oameni, care sunt iudei, tulbură cetatea noastră.
\par 21 Și vestesc obiceiuri care nouă nu ne este îngăduit să le primim, nici să le facem, fiindcă suntem romani.
\par 22 Și s-a sculat și mulțimea împotriva lor. Și judecătorii, rupându-le hainele, au poruncit să-i bată cu vergi.
\par 23 Și, după ce le-au dat multe lovituri, i-au aruncat în temniță, poruncind temnicerului să-i păzească cu grijă.
\par 24 Acesta, primind o asemenea poruncă, i-a băgat în fundul temniței și le-a strâns picioarele în butuci;
\par 25 Iar la miezul nopții, Pavel și Sila, rugându-se, lăudau pe Dumnezeu în cântări, iar cei ce erau în temniță îi ascultau.
\par 26 Și deodată s-a făcut cutremur mare, încât s-au zguduit temeliile temniței și îndată s-au deschis toate ușile și legăturile tuturor s-au dezlegat.
\par 27 Și deșteptându-se temnicerul și văzând deschise ușile temniței, scoțând sabia, voia să se omoare, socotind că cei închiși au fugit.
\par 28 Iar Pavel a strigat cu glas mare, zicând: Să nu-ți faci nici un rău, că toți suntem aici.
\par 29 Iar el, cerând lumină, s-a repezit înăuntru și, tremurând de spaimă, a căzut înaintea lui Pavel și a lui Sila;
\par 30 Și scoțându-i afară (după ce pe ceilalți i-a zăvorât la loc), le-a zis: Domnilor, ce trebuie să fac ca să mă mântuiesc?
\par 31 Iar ei au zis: Crede în Domnul Iisus și te vei mântui tu și casa ta.
\par 32 Și i-au grăit lui cuvântul lui Dumnezeu și tuturor celor din casa lui.
\par 33 Și el, luându-i la sine, în acel ceas al nopții, a spălat rănile lor și s-a botezat el și toți ai lui îndată.
\par 34 Și ducându-i în casă, a pus masa și s-a veselit cu toată casa, crezând în Dumnezeu.
\par 35 Și făcându-se ziuă, judecătorii au trimis pe purtătorii de vergi, zicând: Dă drumul oamenilor acelora.
\par 36 Iar temnicerul a spus cuvintele acestea către Pavel: Că au trimis judecătorii să fiți lăsați liberi. Acum deci ieșiți și mergeți în pace.
\par 37 Dar Pavel a zis către ei: După ce, fără judecată, ne-au bătut în fața lumii, pe noi care suntem cetățeni romani și ne-au băgat în temniță, acum ne scot afară pe ascuns? Nu așa! Ci să vină ei înșiși să ne scoată afară.
\par 38 Și purtătorii de vergi au spus judecătorilor aceste cuvinte. Și auzind că sunt cetățeni romani, judecătorii s-au temut.
\par 39 Și venind, se rugau de ei și, scoțându-i afară, îi rugau să plece din cetate.
\par 40 Iar ei, ieșind din închisoare, s-au dus în casa Lidiei; și văzând pe frați, i-au mângâiat și au plecat.

\chapter{17}

\par 1 Și după ce au trecut prin Amfipoli și prin Apolonia, au venit la Tesalonic, unde era o sinagogă a iudeilor.
\par 2 Și după obiceiul său, Pavel a intrat la ei și în trei sâmbete le-a grăit din Scripturi,
\par 3 Deschizându-le și arătându-le că Hristos trebuia să pătimească și să învieze din morți, și că Acesta, pe Care vi-L vestesc eu, este Hristosul, Iisus.
\par 4 Și unii dintre ei au crezut și au trecut de partea lui Pavel și a lui Sila, și mare mulțime de elini închinători la Dumnezeu și dintre femeile de frunte nu puține.
\par 5 Iar iudeii, umplându-se de invidie și luând cu ei pe câțiva oameni de rând, răi, adunând gloată întărâtau cetatea și, ducându-se la casa lui Iason, căutau să-i scoată afară, înaintea poporului.
\par 6 Dar, negăsindu-i, târau pe Iason și pe câțiva frați la mai-marii cetății, strigând că cei ce au tulburat toată lumea au venit și aici;
\par 7 Pe aceștia i-a găzduit Iason; și toți aceștia lucrează împotriva poruncilor Cezarului, zicând că este un alt împărat: Iisus.
\par 8 Și au tulburat mulțimea și pe mai-marii cetății, care auzeau acestea.
\par 9 Și luând chezășie de la Iason și de la ceilalți, le-au dat drumul.
\par 10 Iar frații au trimis îndată, noaptea, la Bereea, pe Pavel și pe Sila care, ajungând acolo, au intrat în sinagoga iudeilor.
\par 11 Și aceștia erau mai buni la suflet decât cei din Tesalonic; ei au primit cuvântul cu toată osârdia, în toate zilele, cercetând Scripturile, dacă ele sunt așa.
\par 12 Au crezut mulți dintre ei și dintre femeile de cinste ale elinilor, și dintre bărbați nu puțini.
\par 13 Și când au aflat iudeii din Tesalonic că și în Bereea s-a vestit de către Pavel cuvântul lui Dumnezeu, au venit și acolo, întărâtând și tulburând mulțimile.
\par 14 Și atunci îndată frații au trimis pe Pavel, ca să meargă spre mare; iar Sila și cu Timotei au rămas acolo în Bereea.
\par 15 Iar cei ce însoțeau pe Pavel l-au dus până la Atena; și luând ei porunci către Sila și Timotei, ca să vină la el cât mai curând, au plecat.
\par 16 Iar în Atena, pe când Pavel îi aștepta, duhul lui se îndârjea în el, văzând că cetatea este plină de idoli.
\par 17 Deci discuta în sinagogă cu iudeii și cu cei credincioși, și în piață, în fiecare zi, cu cei ce erau de față.
\par 18 Iar unii dintre filozofii epicurei și stoici discutau cu el, și unii ziceau: Ce voiește, oare, să ne spună acest semănător de cuvinte? Iar alții ziceau: Se pare că este vestitor de dumnezei străini, fiindcă binevestește pe Iisus și Învierea.
\par 19 Și luându-l cu ei, l-au dus în Areopag, zicând: Putem să cunoaștem și noi ce este această învățătură nouă, grăită de tine?
\par 20 Căci tu aduci la auzul nostru lucruri străine. Voim deci să știm ce vor să fie acestea.
\par 21 Toți atenienii și străinii, care locuiau acolo, nu-și petreceau timpul decât spunând sau auzind ceva nou.
\par 22 Și Pavel, stând în mijlocul Areopagului, a zis: Bărbați atenieni, în toate vă văd că sunteți foarte evlavioși.
\par 23 Căci străbătând cetatea voastră și privind locurile voastre de închinare, am aflat și un altar pe care era scris: "Dumnezeului necunoscut". Deci pe Cel pe Care voi, necunoscându-L, Îl cinstiți, pe Acesta Îl vestesc eu vouă.
\par 24 Dumnezeu, Care a făcut lumea și toate cele ce sunt în ea, Acesta fiind Domnul cerului și al pământului, nu locuiește în temple făcute de mâini,
\par 25 Nici nu este slujit de mâini omenești, ca și cum ar avea nevoie de ceva, El dând tuturor viață și suflare și toate.
\par 26 Și a făcut dintr-un sânge tot neamul omenesc, ca să locuiască peste toată fața pământului, așezând vremile cele de mai înainte rânduite și hotarele locuirii lor,
\par 27 Ca ei să caute pe Dumnezeu, doar L-ar pipăi și L-ar găsi, deși nu e departe de fiecare dintre noi.
\par 28 Căci în El trăim și ne mișcăm și suntem, precum au zis și unii dintre poeții voștri: căci ai Lui neam și suntem.
\par 29 Fiind deci neamul lui Dumnezeu, nu trebuie să socotim că dumnezeirea este asemenea aurului sau argintului sau pietrei cioplite de meșteșugul și de iscusința omului.
\par 30 Dar Dumnezeu, trecând cu vederea veacurile neștiinței, vestește acum oamenilor ca toți de pretutindeni să se pocăiască,
\par 31 Pentru că a hotărât o zi în care va să judece lumea întru dreptate, prin Bărbatul pe care L-a rânduit, dăruind tuturor încredințare, prin Învierea Lui din morți.
\par 32 Și auzind despre învierea morților, unii l-au luat în râs, iar alții i-au zis: Te vom asculta despre aceasta și altădată.
\par 33 Astfel Pavel a ieșit din mijlocul lor.
\par 34 Iar unii bărbați, alipindu-se de el, au crezut, între care și Dionisie Areopagitul și o femeie cu numele Damaris, și alții împreună cu ei.

\chapter{18}

\par 1 După acestea Pavel, plecând din Atena, a venit la Corint.
\par 2 Și găsind pe un iudeu, cu numele Acvila, de neam din Pont, venit de curând din Italia, și pe Priscila, femeia lui, pentru că poruncise Claudiu ca toți iudeii să plece din Roma, a venit la ei.
\par 3 Și pentru că erau de aceeași meserie, a rămas la ei și lucrau, căci erau făcători de corturi.
\par 4 Și vorbea în sinagogă în fiecare sâmbătă și aducea la credință iudei și elini.
\par 5 Iar când Sila și Timotei au venit din Macedonia, Pavel era prins cu totul de cuvânt, mărturisind iudeilor că Iisus este Hristosul.
\par 6 Și stând ei împotrivă și hulind, el, scuturându-și hainele, a zis către ei: Sângele vostru asupra capului vostru! Eu sunt curat. De acum înainte mă voi duce la neamuri.
\par 7 Și mutându-se de acolo, a venit în casa unuia, cu numele Titus Iustus, cinstitor al lui Dumnezeu, a cărui casă era alături de sinagogă.
\par 8 Dar Crispus, mai-marele sinagogii, a crezut în Domnul, împreună cu toată casa sa; și mulți dintre corinteni, auzind, credeau și se botezau.
\par 9 Și Domnul a zis lui Pavel, noaptea în vedenie: Nu te teme, ci vorbește și nu tăcea,
\par 10 Pentru că Eu sunt cu tine și nimeni nu va pune mâna pe tine, ca să-ți facă rău. Căci am mult popor în cetatea aceasta.
\par 11 Și a stat în Corint un an și șase luni, învățând între ei cuvântul lui Dumnezeu.
\par 12 Dar pe când Galion era proconsulul Ahaiei, iudeii s-au ridicat toți într-un cuget împotriva lui Pavel și l-au adus la tribunal,
\par 13 Zicând că acesta caută să convingă pe oameni să se închine lui Dumnezeu, împotriva legii.
\par 14 Și când Pavel era gata să deschidă gura, Galion a zis către iudei: Dacă ar fi vreo nedreptate sau vreo faptă vicleană, o, iudeilor, v-aș asculta precum se cuvine;
\par 15 Dar dacă sunt la voi nedumeriri despre învățătură și despre nume și despre legea voastră, vedeți-vă voi înșivă de ele. Judecător pentru acestea eu nu voiesc să fiu.
\par 16 Și i-a izgonit de la tribunal.
\par 17 Și punând mâna toți pe Sostene, mai-marele sinagogii, îl băteau înaintea tribunalului. Dar Galion nu lua în seamă nimic din acestea;
\par 18 Iar Pavel, după ce a stat încă multe zile în Corint, și-a luat rămas bun de la frați și a plecat cu corabia în Siria, împreună cu Priscila și cu Acvila, care și-a tuns capul la Chenhrea, căci făcuse o făgăduință.
\par 19 Și au sosit la Efes și pe aceia i-a lăsat acolo, iar el, intrând în sinagogă, discuta cu iudeii.
\par 20 Și rugându-l să rămână la ei mai multă vreme, n-a voit,
\par 21 Ci, despărțindu-se de ei, a zis: Trebuie, negreșit, ca sărbătoarea care vine s-o fac la Ierusalim, dar, cu voia Domnului, mă voi întoarce iarăși la voi. Și a plecat de la Efes, cu corabia.
\par 22 Și coborându-se la Cezareea, s-a suit (la Ierusalim) și, îmbrățișând Biserica, s-a coborât la Antiohia.
\par 23 Și stând acolo câtva timp, a plecat, străbătând pe rând ținutul Galatiei și Frigia, întărind pe toți ucenicii.
\par 24 Iar un iudeu, cu numele Apollo, alexandrin de neam, bărbat iscusit la cuvânt, puternic fiind în Scripturi, a sosit la Efes.
\par 25 Acesta era învățat în calea Domnului și, arzând cu duhul, grăia și învăța drept cele despre Iisus, cunoscând numai botezul lui Ioan.
\par 26 Și el a început să vorbească, fără sfială, în sinagogă. Auzindu-l, Priscila și Acvila l-au luat cu ei și i-au arătat mai cu de-amănuntul calea lui Dumnezeu.
\par 27 Și voind el să treacă în Ahaia, l-au îndemnat frații și au scris ucenicilor să-l primească. Și sosind (în Corint), a ajutat mult cu harul celor ce crezuseră;
\par 28 Căci cu tărie și în fața tuturor, el înfrunta pe iudei, dovedind din Scripturi că Iisus este Hristos.

\chapter{19}

\par 1 Și pe când Apollo era în Corint, Pavel, după ce a străbătut părțile de sus, a venit în Efes. Și găsind câțiva ucenici,
\par 2 A zis către ei: Primit-ați voi Duhul Sfânt când ați crezut? Iar ei au zis către el: Dar nici n-am auzit dacă este Duh Sfânt.
\par 3 Și el a zis: Deci în ce v-ați botezat? Ei au zis: În botezul lui Ioan.
\par 4 Iar Pavel a zis: Ioan a botezat cu botezul pocăinței, spunând poporului să creadă în Cel ce avea să vină după el, adică în Iisus Hristos.
\par 5 Și auzind ei, s-au botezat în numele Domnului Iisus.
\par 6 Și punându-și Pavel mâinile peste ei, Duhul Sfânt a venit asupra lor și vorbeau în limbi și prooroceau.
\par 7 Și erau toți ca la doisprezece bărbați.
\par 8 Și el, intrând în sinagogă, a vorbit cu îndrăzneală timp de trei luni, vorbind cu ei și căutând să-i încredințeze de împărăția lui Dumnezeu.
\par 9 Dar fiindcă unii erau învârtoșați și nu credeau, bârfind calea Domnului înaintea mulțimii, Pavel, plecând de la ei, a osebit pe ucenici, învățând în fiecare zi în școala unuia Tiranus.
\par 10 Și acesta a ținut vreme de doi ani, încât toți, cei ce locuiau în Asia, și iudei și elini, au auzit cuvântul Domnului.
\par 11 Și Dumnezeu făcea, prin mâinile lui Pavel, minuni nemaiîntâlnite.
\par 12 Încât și peste cei ce erau bolnavi se puneau ștergare sau șorțuri purtate de Pavel, și bolile se depărtau de ei, iar duhurile cele rele ieșeau din ei.
\par 13 Și au încercat unii dintre iudeii care cutreierau lumea, scoțând demoni, să cheme peste cei ce aveau duhuri rele, numele Domnului Iisus, zicând: Vă jur pe Iisus, pe Care-l propovăduiește Pavel!
\par 14 Iar cei care făceau aceasta erau cei șapte fii ai unuia Scheva, arhiereu iudeu.
\par 15 Și răspunzând, duhul cel rău le-a zis: Pe Iisus Îl cunosc și îl știu și pe Pavel, dar voi cine sunteți?
\par 16 Și sărind asupra lor omul în care era duhul cel rău și biruindu-i, s-a întărâtat asupra lor, încât ei au fugit goi și răniți din casa aceea.
\par 17 Și acest lucru s-a făcut cunoscut tuturor iudeilor și elinilor care locuiau în Efes, și frică a căzut peste toți aceștia și se slăvea numele Domnului Iisus.
\par 18 Și mulți dintre cei ce crezuseră veneau ca să se mărturisească și să spună faptele lor.
\par 19 Iar mulți dintre cei ce făcuseră vrăjitorie, aducând cărțile, le ardeau în fața tuturor. Și au socotit prețul lor și au găsit cincizeci de mii de arginți.
\par 20 Astfel creștea cu putere cuvântul Domnului și se întărea.
\par 21 Și după ce s-au săvârșit acestea, Pavel și-a pus în gând să treacă prin Macedonia și prin Ahaia și să se ducă la Ierusalim, zicând că: După ce voi fi acolo, trebuie să văd și Roma.
\par 22 Și trimițând în Macedonia pe doi dintre cei care îl slujeau, pe Timotei și pe Erast, el a rămas o vreme în Asia.
\par 23 Și în vremea aceea s-a făcut mare tulburare pentru calea Domnului.
\par 24 Căci un argintar, cu numele Dimitrie, care făcea temple de argint Artemisei și da meșterilor săi foarte mare câștig,
\par 25 I-a adunat pe aceștia și pe cei care lucrau unele ca acestea, și le-a zis: Bărbaților, știți că din această îndeletnicire este câștigul vostru.
\par 26 Și voi vedeți și auziți că nu numai în Efes, ci aproape în toată Asia, Pavel acesta, convingând, a întors multă mulțime, zicând că nu sunt dumnezei cei făcuți de mâini.
\par 27 Din aceasta nu numai că meseria noastră e în primejdie să ajungă fără trecere, dar și templul marii zeițe Artemisa e în primejdie să nu mai aibă nici un preț, iar cu vremea, mărirea ei - căreia i se închină toată Asia și toată lumea - să fie doborâtă.
\par 28 Și auzind ei și umplându-se de mânie, strigau zicând: Mare este Artemisa efesenilor!
\par 29 Și s-a umplut toată cetatea de tulburare și au pornit într-un cuget la teatru, răpind împreună pe macedonenii Gaius și Aristarh, însoțitorii lui Pavel.
\par 30 Iar Pavel, voind să intre în mijlocul poporului, ucenicii nu l-au lăsat.
\par 31 Încă și unii dintre dregătorii Asiei, care îi erau prieteni, trimițând la el, îl rugau să nu se ducă la teatru.
\par 32 Deci unii strigau una, alții strigau alta, căci adunarea era învălmășită, iar cei mai mulți nu știau pentru ce s-au adunat acolo.
\par 33 Iar unii din mulțime l-au smuls pe Alexandru, pe care l-au împins înainte iudeii. Iar el, făcând semn cu mâna, voia să se apere înaintea poporului.
\par 34 Și cunoscând ei că este iudeu, toți într-un glas au strigat aproape două ceasuri: Mare este Artemisa efesenilor!
\par 35 Iar secretarul, potolind mulțimea, a zis: Bărbați efeseni, cine este, între oameni, care să nu știe că cetatea efesenilor este păzitoarea templului Artemisei celei mari și a statuii ei, căzută din cer?
\par 36 Deci, acestea fiind mai presus de orice îndoială, trebuie să vă liniștiți și să nu faceți nimic cu ușurință.
\par 37 Căci ați adus pe bărbații aceștia, care nu sunt nici furi de cele sfinte, nici nu hulesc pe zeița voastră.
\par 38 Deci dacă Dimitrie și meșterii cei împreună cu el au vreo plângere împotriva cuiva, au judecători și proconsuli care să judece, și să se cheme în judecată unii pe alții.
\par 39 Iar dacă urmăriți altceva, se va hotărî în adunarea cea legiuită,
\par 40 Căci noi suntem în primejdie să fim învinuiți de răscoală pentru ziua de azi, fiindcă nu avem nici o pricină pentru care am putea da seama de tulburarea aceasta.
\par 41 Zicând acestea, a slobozit adunarea.

\chapter{20}

\par 1 Iar după ce a încetat tulburarea, Pavel, chemând pe ucenici și dându-le îndemnuri, după ce și-a luat rămas bun, a ieșit să meargă în Macedonia.
\par 2 Și străbătând acele părți și dând ucenicilor multe sfaturi și îndemnuri, a sosit în Grecia.
\par 3 Și a stat acolo trei luni. Dar când era să plece pe apă în Siria, iudeii au uneltit împotriva vieții lui, iar el s-a hotărât să se întoarcă prin Macedonia.
\par 4 Și mergeau împreună cu el, până în Asia, Sosipatru al lui Piru din Bereea, Aristarh și Secundus din Tesalonic și Gaius din Derbe și Timotei, iar din Asia: Tihic și Trofim.
\par 5 Aceștia, plecând înainte, ne-au așteptat în Troa.
\par 6 Iar noi, după zilele Azimelor, am pornit cu corabia de la Filipi și în cinci zile am sosit la ei în Troa, unde am rămas șapte zile.
\par 7 În ziua întâi a săptămânii (Duminică) adunându-ne noi să frângem pâinea, Pavel, care avea de gând să plece a doua zi, a început să le vorbească și a prelungit cuvântul lui până la miezul nopții.
\par 8 Iar în camera de sus, unde erau adunați, erau multe lumini aprinse.
\par 9 Dar un tânăr cu numele Eutihie, șezând pe fereastră, pe când Pavel ținea lungul său cuvânt, a adormit adânc și, doborât de somn, a căzut jos de la catul al treilea, și l-au ridicat mort.
\par 10 Iar Pavel, coborându-se, s-a plecat peste el și, luându-l în brațe, a zis: Nu vă tulburați, căci sufletul lui este în el.
\par 11 Și suindu-se și frângând pâinea și mâncând, a vorbit cu ei mult până în zori, și atunci a plecat.
\par 12 Iar pe tânăr l-au adus viu și foarte mult s-au mângâiat.
\par 13 Iar noi, venind la corabie, am plutit spre Asson, ca să luăm de acolo pe Pavel, căci astfel rânduise el, voind să meargă pe jos.
\par 14 După ce s-a întâlnit cu noi la Asson, luându-l cu noi, am venit la Mitilene.
\par 15 Și de acolo, mergând cu corabia, am sosit a doua zi în fața insulei Hios. Iar în ziua următoare, am ajuns în Samos și, după ce am poposit la Troghilion, a doua zi am venit la Milet.
\par 16 Căci Pavel hotărâse să treacă pe apă pe lângă Efes, ca să nu i se întârzie în Asia, pentru că se grăbea să fie, dacă i-ar fi cu putință, la Ierusalim, de ziua Cincizecimii.
\par 17 Și trimițând din Milet la Efes, a chemat la sine pe preoții Bisericii.
\par 18 Și când ei au venit la el, le-a zis: Voi știți cum m-am purtat cu voi, în toată vremea, din ziua cea dintâi, când am venit în Asia,
\par 19 Slujind Domnului cu toată smerenia și cu multe lacrimi și încercări care mi s-au întâmplat prin uneltirile iudeilor.
\par 20 Și cum n-am ascuns nimic din cele folositoare, ca să nu vi le vestesc și să nu vă învăț, fie înaintea poporului, fie prin case,
\par 21 Mărturisind și iudeilor și elinilor întoarcerea la Dumnezeu prin pocăință și credința în Domnul nostru Iisus Hristos.
\par 22 Iar acum iată că fiind eu mânat de Duhul, merg la Ierusalim, neștiind cele ce mi se vor întâmpla acolo,
\par 23 Decât numai că Duhul Sfânt mărturisește prin cetăți, spunându-mi că mă așteaptă lanțuri și necazuri.
\par 24 Dar nimic nu iau în seamă și nu pun nici un preț pe sufletul meu, numai să împlinesc calea mea și slujba mea pe care am luat-o de la Domnul Iisus, de a mărturisi Evanghelia harului lui Dumnezeu.
\par 25 Și acum, iată, eu știu că voi toți, printre care am petrecut propovăduind împărăția lui Dumnezeu, nu veți mai vedea fața mea.
\par 26 Pentru aceea vă mărturisesc în ziua de astăzi că sunt curat de sângele tuturor.
\par 27 Căci nu m-am ferit să vă vestesc toată voia lui Dumnezeu.
\par 28 Drept aceea, luați aminte de voi înșivă și de toată turma, întru care Duhul Sfânt v-a pus pe voi episcopi, ca să păstrați Biserica lui Dumnezeu, pe care a câștigat-o cu însuși sângele Său.
\par 29 Căci eu știu aceasta, că după plecarea mea vor intra, între voi, lupi îngrozitori, care nu vor cruța turma.
\par 30 Și dintre voi înșivă se vor ridica bărbați, grăind învățături răstălmăcite, ca să tragă pe ucenici după ei.
\par 31 Drept aceea, privegheați, aducându-vă aminte că, timp de trei ani, n-am încetat noaptea și ziua să vă îndemn, cu lacrimi, pe fiecare dintre voi.
\par 32 Și acum vă încredințez lui Dumnezeu și cuvântului harului Său, cel ce poate să vă zidească și să vă dea moștenire între toți cei sfințiți.
\par 33 Argint, sau aur, sau haină, n-am poftit de la nimeni;
\par 34 Voi înșivă știți că mâinile acestea au lucrat pentru trebuințele mele și ale celor ce erau cu mine.
\par 35 Toate vi le-am arătat, căci ostenindu-vă astfel, trebuie să ajutați pe cei slabi și să vă aduceți aminte de cuvintele Domnului Iisus, căci El a zis: Mai fericit este a da decât a lua.
\par 36 Și după ce a spus acestea, plecându-și genunchii, s-a rugat împreună cu toți aceștia.
\par 37 Și mare jale i-a cuprins pe toți și, căzând pe grumazul lui Pavel, îl sărutau,
\par 38 Cuprinși de jale mai ales pentru cuvântul pe care îl spusese, că n-au să mai vadă fața lui. Și îl petrecură la corabie.

\chapter{21}

\par 1 Și după ce ne-am despărțit de ei, am plecat pe apă și, mergând drept, am venit la Cos și a doua zi la Rodos, iar de acolo la Patara.
\par 2 Și găsind o corabie, care mergea în Fenicia, ne-am urcat în ea și am plecat.
\par 3 Și zărind Ciprul și lăsându-l la stânga, am plutit spre Siria și ne-am coborât în Tir, căci acolo corabia avea să descarce povara.
\par 4 Și găsind pe ucenici, am rămas acolo șapte zile. Aceștia spuneau lui Pavel, prin duhul, să nu se suie la Ierusalim.
\par 5 Și când am împlinit zilele, ieșind, am plecat, petrecându-ne toți, împreună cu femei și cu copii, până afară din cetate și, plecând genunchii pe țărm, ne-am rugat.
\par 6 Și ne-am îmbrățișat unii pe alții și ne-am urcat în corabie, iar aceia s-au întors la ale lor.
\par 7 Iar noi, sfârșind călătoria noastră pe apă, de la Tir am venit la Ptolemaida și, îmbrățișând pe frați, am rămas la ei o zi.
\par 8 Iar a doua zi, ieșind, am venit la Cezareea. Și intrând în casa lui Filip binevestitorul, care era dintre cei șapte (diaconi), am rămas la el.
\par 9 Și acesta avea patru fiice, fecioare, care prooroceau.
\par 10 Și rămânând noi acolo mai multe zile, a coborât din Iudeea un prooroc cu numele Agav;
\par 11 Și, venind el la noi, a luat brâul lui Pavel și legându-și picioarele și mâinile a zis: Acestea zice Duhul Sfânt: Pe bărbatul al căruia este acest brâu, așa îl vor lega iudeii la Ierusalim și-l vor da în mâinile neamurilor.
\par 12 Și când am auzit acestea, îl rugam și noi și localnicii să nu se suie la Ierusalim.
\par 13 Atunci a răspuns Pavel: Ce faceți de plângeți și-mi sfâșiați inima? Căci eu sunt gata nu numai să fiu legat, ci să și mor în Ierusalim pentru numele Domnului Iisus.
\par 14 Și neînduplecându-se el, ne-am liniștit, zicând: Facă-se voia Domnului.
\par 15 Iar după zilele acestea, pregătindu-ne, ne-am suit la Ierusalim.
\par 16 Și au venit împreună cu noi și dintre ucenicii din Cezareea, ducându-ne la un oarecare Mnason din Cipru, vechi ucenic, la care am fost găzduiți.
\par 17 Și sosind la Ierusalim, frații ne-au primit cu bucurie.
\par 18 Iar a doua zi Pavel a mers cu noi la Iacov și au venit acolo toți preoții.
\par 19 Și îmbrățișându-i le povestea cu de-amănuntul cele ce a făcut Dumnezeu între neamuri, prin slujirea lui.
\par 20 Iar ei, auzind, slăveau pe Dumnezeu, și i-au zis: Vezi frate, câte mii de iudei au crezut și toți sunt plini de râvnă pentru lege.
\par 21 Și ei au auzit despre tine că înveți pe toți iudeii, care trăiesc printre neamuri, să se lepede de Moise, spunându-le să nu-și taie împrejur copiii, nici să umble după datini.
\par 22 Ce este deci? Fără îndoială, trebuie să se adune mulțime, căci vor auzi că ai venit.
\par 23 Fă, deci, ceea ce îți spunem. Noi avem patru bărbați, care au asupra lor o făgăduință;
\par 24 Pe aceștia luându-i, curățește-te împreună cu ei și cheltuiește pentru ei ca să-și radă capul, și vor cunoaște toți că nimic nu este (adevărat) din cele ce au auzit despre tine, dar că tu însuți umbli după Lege și o păzești.
\par 25 Cât despre păgânii care au crezut, noi le-am trimis o scrisoare, hotărându-le să se ferească de ceea ce este jertfit idolilor și de sânge și de (animal) sugrumat și de desfrâu.
\par 26 Atunci Pavel, luând cu el pe acei bărbați, curățindu-se împreună cu ei a doua zi, a intrat în templu, vestind împlinirea zilelor curățirii, până când a fost adusă ofranda pentru fiecare din ei.
\par 27 Și când era să se împlinească cele șapte zile, iudeii din Asia, văzându-l în templu, au întărâtat toată mulțimea și au pus mâna pe el,
\par 28 Strigând: Bărbați israeliți, ajutați! Acesta este omul care învață pe toți pretutindeni, împotriva poporului și a Legii și a locului acestuia; încă și elini a adus în templu și a spurcat acest loc sfânt.
\par 29 Căci ei văzuseră mai înainte cu el împreună în cetate pe Trofim din Efes, pe care socoteau că Pavel l-a adus în templu.
\par 30 Și s-a mișcat toată cetatea și poporul a alergat din toate părțile și, punând mâna pe Pavel, îl trăgeau afară din templu și îndată au închis porțile.
\par 31 Dar când căutau ei ca să-l omoare, a ajuns veste la comandantul cohortei, că tot Ierusalimul s-a tulburat.
\par 32 Acela, luând îndată ostași și sutași, a alergat la ei; iar ei, văzând pe comandant și pe ostași, au încetat de a mai bate pe Pavel.
\par 33 Apropiindu-se atunci comandantul, a pus mâna pe el și a poruncit să fie legat cu două lanțuri și întrebat cine este și ce a făcut.
\par 34 Iar unii strigau în mulțime una, alții altceva și neputând să înțeleagă adevărul, din cauza tulburării, a poruncit să fie dus în fortăreață.
\par 35 Când a ajuns la trepte, a trebuit, de furia mulțimii, să fie purtat de ostași.
\par 36 Căci mergea după el mulțime de popor, strigând: Omoară-l!
\par 37 Și vrând să-l ducă în fortăreață, Pavel a zis comandantului: Îmi este îngăduit să vorbesc ceva cu tine? Iar el a zis: Știi grecește?
\par 38 Nu ești tu, oare, egipteanul care, înainte de zilele acestea, te-ai răsculat și ai scos în pustie pe cei patru mii de bărbați răzvrătiți?
\par 39 Și a zis Pavel: Eu sunt iudeu din Tarsul Ciliciei, cetățean al unei cetăți care nu este neînsemnată. Te rog dă-mi voie să vorbesc către popor.
\par 40 Și dându-i-se voie, Pavel, stând în picioare pe trepte, a făcut poporului semn cu mâna. Și făcându-se mare tăcere, a vorbit în limba evreiască, zicând:

\chapter{22}

\par 1 Bărbați frați și părinți, ascultați acum, apărarea mea față de voi!
\par 2 Și auzind că le vorbea în limba evreiască, au făcut mai multă liniște, și el le-a zis:
\par 3 Eu sunt bărbat iudeu, născut în Tarsul Ciliciei și crescut în cetatea aceasta, învățând la picioarele lui Gamaliel în chip amănunțit Legea părintească, plin fiind de râvnă pentru Dumnezeu, precum și voi toți sunteți astăzi.
\par 4 Eu am prigonit până la moarte această cale, legând și dând la închisoare și bărbați și femei,
\par 5 Precum mărturisește pentru mine și arhiereul și tot sfatul bătrânilor, de la care primind și scrisori către frați, mergeam la Damasc, ca să-i aduc legați la Ierusalim și pe cei ce erau acolo, spre a fi pedepsiți.
\par 6 Dar pe când mergeam eu și mă apropiam de Damasc, pe la amiază, deodată o lumină puternică din cer m-a învăluit ca un fulger.
\par 7 Și am căzut la pământ și am auzit un glas, zicându-mi: Saule, Saule, de ce Mă prigonești?
\par 8 Iar eu am răspuns: Cine ești, Doamne? Zis-a către mine: Eu sunt Iisus Nazarineanul, pe Care tu Îl prigonești.
\par 9 Iar cei ce erau cu mine au văzut lumina și s-au înfricoșat, dar glasul Celui care îmi vorbea ei nu l-au auzit.
\par 10 Și am zis: Ce să fac, Doamne? Iar Domnul a zis către mine: Ridică-te și mergi în Damasc și acolo ți se va spune despre toate cele ce ți s-au rânduit să faci.
\par 11 Și pentru că nu mai vedeam, din cauza strălucirii acelei lumini, am venit în Damasc, fiind dus de mână de către cei ce erau împreună cu mine.
\par 12 Iar un oarecare Anania, bărbat evlavios, după Lege, mărturisit de toți iudeii care locuiau în Damasc,
\par 13 Venind la mine și stând alături, mi-a zis: Frate Saule, vezi iarăși! Și eu în ceasul acela l-am văzut.
\par 14 Iar el a zis: Dumnezeul părinților noștri te-a ales de mai înainte pe tine ca să cunoști voia Lui și să vezi pe Cel Drept și să auzi glas din gura Lui;
\par 15 Că martor vei fi Lui, în fața tuturor oamenilor, despre cele ce ai văzut și auzit.
\par 16 Și acum de ce zăbovești? Sculându-te, botează-te și spală-ți păcatele, chemând numele Lui.
\par 17 Și s-a întâmplat, când m-am întors la Ierusalim și mă rugam în templu, să fiu în extaz,
\par 18 Și să-L văd zicându-mi: Grăbește-te, și ieși degrabă din Ierusalim, pentru că nu vor primi mărturia ta despre Mine.
\par 19 Și eu am zis: Doamne, ei știu că eu duceam la închisoare și băteam, prin sinagogi, pe cei care credeau în Tine;
\par 20 Și când se vărsa sângele lui Ștefan, mucenicul Tău, eram și eu de față și încuviințam uciderea lui și păzeam hainele celor care îl ucideau.
\par 21 Și a zis către mine: Mergi, că Eu te voi trimite departe, la neamuri.
\par 22 Și l-au ascultat până la acest cuvânt, și au ridicat glasul lor, zicând: Ia-l de pe pământ pe unul ca acesta! Căci nu se cuvine ca el să mai trăiască.
\par 23 Și strigând ei și aruncând hainele și azvârlind pulbere în aer,
\par 24 Comandantul a poruncit să-l ducă în fortăreață, spunând să-l ia la cercetare, cu biciul, ca să cunoască pentru care pricină strigau așa împotriva lui.
\par 25 Și când l-au întins ca să-l biciuiască, Pavel a zis către sutașul care era de față: Oare vă este îngăduit să biciuiți un cetățean roman și nejudecat?
\par 26 Și auzind sutașul s-a dus la comandant să-i vestească, zicând: Ce ai de gând să faci? Că omul acesta este (cetățean) roman.
\par 27 Și venind la el, comandantul i-a zis: Spune-mi, ești tu (cetățean) roman? Iar el a zis: Da!
\par 28 Și a răspuns comandantul: Eu am dobândit această cetățenie cu multă cheltuială. Iar Pavel a zis: Eu însă m-am și născut.
\par 29 Deci cei ce erau gata să-l ia la cercetare s-au depărtat îndată de la el, iar comandantul s-a temut, aflând că el este (cetățean) roman și că a fost legat.
\par 30 Și a doua zi, voind să cunoască adevărul, pentru care era pârât de iudei, l-a dezlegat și a poruncit să se adune arhiereii și tot sinedriul și, aducând pe Pavel, l-a pus înaintea lor.

\chapter{23}

\par 1 Și Pavel, fixând sinedriul cu privirea, a zis: Bărbați frați, eu cu bun cuget am viețuit înaintea lui Dumnezeu până în ziua aceasta.
\par 2 Arhiereul Anania a poruncit celor ce ședeau lângă el să-l bată peste gură.
\par 3 Atunci Pavel a zis către el: Te va bate Dumnezeu, perete văruit! Și tu șezi să mă judeci pe mine după Lege, și, călcând Legea, poruncești să mă bată?
\par 4 Iar cei ce stăteau lângă el au zis: Pe arhiereul lui Dumnezeu îl faci tu de ocară?
\par 5 Iar Pavel a zis: Fraților, nu știam că este arhiereu; căci este scris: "Pe mai-marele poporului tău să nu-l vorbești de rău".
\par 6 Dar Pavel, știind că o parte erau saduchei și cealaltă farisei, a strigat în sinedriu: Bărbați frați! Eu sunt fariseu, fiu de farisei. Pentru nădejdea și învierea morților sunt eu judecat!
\par 7 Și grăind el aceasta, între farisei și saduchei s-a iscat neînțelegere și mulțimea s-a dezbinat;
\par 8 Căci saducheii zic că nu este înviere, nici înger, nici duh, iar fariseii mărturisesc și una și alta.
\par 9 Și s-a făcut mare strigare, și, ridicându-se unii cărturari din partea fariseilor, se certau zicând: Nici un rău nu găsim în acest om; iar dacă i-a vorbit lui un duh sau înger, să nu ne împotrivim lui Dumnezeu.
\par 10 Deci făcându-se mare neînțelegere și temându-se comandantul ca Pavel să nu fie sfâșiat de ei, a poruncit ostașilor să se coboare și să-l smulgă din mijlocul lor și să-l ducă în fortăreață.
\par 11 Iar în noaptea următoare, arătându-i-Se, Domnul i-a zis: Îndrăznește, Pavele! Căci precum ai mărturisit cele despre Mine la Ierusalim, așa trebuie să mărturisești și la Roma.
\par 12 Iar când s-a făcut ziuă, iudeii, făcând sfat împotrivă-i, s-au legat cu blestem zicând că nu vor mânca, nici nu vor bea până ce nu vor ucide pe Pavel.
\par 13 Și cei ce făcuseră între ei acest jurământ erau mai mulți de patruzeci,
\par 14 Care, ducându-se la arhierei și la bătrâni, au zis: Ne-am legat pe noi înșine cu blestem să nu gustăm nimic până ce nu vom ucide pe Pavel.
\par 15 Acum deci voi, împreună cu sinedriul, faceți cunoscut comandantului să-l coboare mâine la voi, ca având să cerceteze mai cu de-amănuntul cele despre el; iar noi, înainte de a se apropia el, suntem gata să-l ucidem.
\par 16 Dar fiul surorii lui Pavel, auzind despre această uneltire, ducându-se și intrând în fortăreață, i-a vestit lui Pavel.
\par 17 Și chemând Pavel pe unul din sutași, i-a zis: Du pe tânărul acesta la comandant, căci are să-i vestească ceva.
\par 18 Iar el, luându-l, l-a dus la comandant și a zis: Pavel cel legat, chemându-mă, m-a rugat să aduc pe acest tânăr la tine, având să-ți spună ceva.
\par 19 Comandantul, luându-l de mână, s-a retras cu el la o parte și îl întreba: Ce ai să-mi vestești?
\par 20 Iar el a zis că iudeii s-au înțeles să te roage, ca mâine să-l cobori pe Pavel la sinedriu, ca având să cerceteze mai cu de-amănuntul despre el;
\par 21 Dar tu să nu te încrezi în ei, căci dintre ei îl pândesc mai mulți de patruzeci de bărbați, care s-au legat cu blestem să nu mănânce, nici să bea până ce nu-l vor ucide; și acum ei sunt gata, așteptând aprobarea ta.
\par 22 Deci comandantul a dat drumul tânărului, poruncindu-i: Nimănui să nu spui că mi-ai făcut cunoscut acestea.
\par 23 Și chemând la sine pe doi dintre sutași, le-a zis: Pregătiți de la ceasul al treilea din noapte două sute de ostași, șaptezeci de călăreți și două sute de sulițași, ca să meargă până la Cezareea.
\par 24 Și să fie animale (de călărie), ca punând pe Pavel să-l ducă teafăr la Felix procuratorul.
\par 25 Scriind o scrisoare, având acest cuprins:
\par 26 Claudius Lysias, prea puternicului procurator, Felix, salutare!
\par 27 Pe acest bărbat, prins de iudei și având să fie ucis de ei, mergând eu cu oaste l-am scos, aflând că este (cetățean) roman.
\par 28 Și vrând să știu pricina pentru care îl pârau, l-am coborât la sinedriul lor.
\par 29 Și am aflat că este pârât pentru întrebări din legea lor, dar fără să aibă vreo vină vrednică de moarte sau de lanțuri.
\par 30 Și vestindu-mi-se că va să fie o cursă împotriva acestui bărbat din partea iudeilor, îndată l-am trimis la tine, poruncind și pârâșilor să spună înaintea ta cele ce au asupra lui. Fii sănătos!
\par 31 Deci ostașii, luând pe Pavel, precum li se poruncise, l-au adus noaptea la Antipatrida.
\par 32 Iar a doua zi, lăsând pe călăreți să meargă cu el, s-au întors la fortăreață.
\par 33 Și ei, intrând în Cezareea și dând procuratorului scrisoarea, i-au înfățișat și pe Pavel.
\par 34 Și citind procuratorul și întrebând din ce provincie este el și aflând că este din Cilicia,
\par 35 A zis: Te voi asculta când vor veni și pârâșii tăi. Și a poruncit să fie păzit în pretoriul lui Irod.

\chapter{24}

\par 1 Iar după cinci zile s-a coborât arhiereul Anania cu câțiva bătrâni și cu un oarecare retor Tertul, care s-au înfățișat procuratorului împotriva lui Pavel.
\par 2 Iar după ce l-au chemat pe Pavel, Tertul a început să-l învinuiască, zicând: Prin tine dobândim multă pace și îndreptările făcute acestui neam, prin purtarea ta de grijă,
\par 3 Totdeauna și pretutindeni le primim, prea puternice Felix, cu toată mulțumirea.
\par 4 Dar, ca să nu te ostenesc mai mult, te rog să ne asculți, pe scurt, cu bunăvoința ta.
\par 5 Căci am aflat pe omul acesta ca o ciumă și urzitor de răzvrătiri printre toți iudeii din lume, fiind căpetenia eresului nazarinenilor,
\par 6 Care a încercat să pângărească și templul și pe care l-am prins și am voit să-l judecăm după legea noastră.
\par 7 Dar venind Lysias comandantul l-a scos cu de-a sila din mâinile noastre,
\par 8 Poruncind pârâșilor lui să vină la tine. De la el vei putea, cercetând tu însuți, să cunoști toate învinuirile aduse de noi.
\par 9 Iar iudeii împreună susțineau, zicând că acestea așa sunt.
\par 10 Și, procuratorul făcându-i semn să vorbească, Pavel a răspuns: Fiindcă știu că de mulți ani ești judecător acestui neam, bucuros vorbesc pentru apărarea mea.
\par 11 Tu poți să afli că nu sunt mai mult decât douăsprezece zile de când m-am suit la Ierusalim ca să mă închin.
\par 12 Și nici în templu nu m-au găsit discutând cu cineva sau făcând tulburare în mulțime, nici în sinagogi, nici în cetate,
\par 13 Nici nu pot să-ți dovedească cele ce spun acum împotriva mea.
\par 14 Și-ți mărturisesc aceasta, că așa mă închin Dumnezeului părinților mei, după învățătura pe care ei o numesc eres, și cred toate cele scrise în Lege și în Prooroci,
\par 15 Având nădejde în Dumnezeu, pe care și aceștia înșiși o așteaptă, că va să fie învierea morților: și a drepților și a nedrepților.
\par 16 Și întru aceasta mă străduiesc și eu ca să am totdeauna înaintea lui Dumnezeu și a oamenilor un cuget neîntinat.
\par 17 După mulți ani, am venit ca să aduc neamului meu milostenii și prinoase,
\par 18 Când niște iudei din Asia m-au găsit, curățit, în templu, dar nu cu mulțime, nici cu gâlceavă.
\par 19 Aceia trebuia să fie de față înaintea ta și să mă învinuiască, dacă aveau ceva împotriva mea;
\par 20 Sau chiar aceștia să spună ce nedreptate mi-au găsit când am stat înaintea sinedriului,
\par 21 Decât numai pentru acest singur cuvânt pe care l-am strigat stând între ei, că pentru învierea morților sunt eu astăzi judecat între voi.
\par 22 Și Felix, auzind acestea, i-a amânat, cunoscând destul de bine cele privitoare la învățătura (creștină), zicând: Când se va coborî comandantul Lysias, voi hotărî asupra acelora ale voastre.
\par 23 Și a poruncit sutașului să țină pe Pavel sub pază, dar să-i lase tihnă și să nu oprească pe nimeni dintre ai lui, ca să vină să-i slujească.
\par 24 Iar după câteva zile, Felix, venind cu Drusila, femeia lui, care era din neamul iudeilor, a trimis să cheme pe Pavel și l-a ascultat despre credința în Hristos Iisus.
\par 25 Și vorbind el despre dreptate și despre înfrânare și despre judecata viitoare, Felix s-a înfricoșat și a răspuns: Acum mergi, și când voi găsi timp potrivit te voi mai chema.
\par 26 În același timp el nădăjduia că i se vor da bani de către Pavel; de aceea, și mai des trimițând să-l cheme, vorbea cu el.
\par 27 Dar când s-au împlinit doi ani, în locul lui Felix a urmat Porcius Festus. Și voind să le fie iudeilor pe plac, Felix a lăsat pe Pavel legat.

\chapter{25}

\par 1 Deci Festus, trecând în ținutul său, după trei zile s-a suit de la Cezareea la Ierusalim.
\par 2 Și arhiereii și fruntașii iudeilor i s-au înfățișat cu învinuiri împotriva lui Pavel și îl rugau,
\par 3 Cerându-i ca o favoare asupra lui, să fie trimis la Ierusalim, pregătind cursă ca să-l ucidă pe drum.
\par 4 Dar Festus a răspuns că Pavel e păzit Cezareea și că el însuși avea să plece în curând.
\par 5 Deci a zis el: Cei dintre voi care pot, să se coboare cu mine, și dacă este ceva rău în acest bărbat, să-l învinovățească.
\par 6 Și rămânând la ei nu mai mult de opt sau zece zile, s-a coborât în Cezareea, iar a doua zi, șezând la judecată, a poruncit să fie adus Pavel.
\par 7 Și venind el, iudeii coborâți din Ierusalim l-au înconjurat, aducând împotriva lui multe și grele învinuiri, pe care nu puteau să le dovedească.
\par 8 Iar Pavel se apăra: N-am greșit cu nimic nici față de legea iudeilor, nici față de templu, nici față de Cezarul.
\par 9 Iar Festus, voind să facă plăcere iudeilor, răspunzând lui Pavel, a zis: Vrei să mergi la Ierusalim și acolo să fi judecat înaintea mea pentru acestea?
\par 10 Dar Pavel a zis: Stau la judecata Cezarului, unde trebuie să fiu judecat. Iudeilor nu le-am făcut nici un rău, precum mai bine știi și tu.
\par 11 Dar dacă fac nedreptate și am săvârșit ceva vrednic de moarte, nu mă feresc de moarte; dacă însă nu este nimic din cele de care ei mă învinuiesc - nimeni nu poate să mă dăruiască lor. Cer să fiu judecat de Cezarul.
\par 12 Atunci Festus, vorbind cu sfatul său, a răspuns: Ai cerut să fii judecat de Cezarul, la Cezarul te vei duce.
\par 13 Și după ce au trecut câteva zile, regele Agripa și Berenice au sosit la Cezareea, ca să salute pe Festus.
\par 14 Și rămânând acolo mai multe zile, Festus a vorbit regelui despre Pavel, zicând: Este aici un bărbat, lăsat legat de Felix,
\par 15 În privința căruia, când am fost în Ierusalim, mi s-au înfățișat arhiereii și bătrânii iudeilor, cerând osândirea lui.
\par 16 Eu le-am răspuns că romanii n-au obiceiul să dea pe vreun om la pierzare, înainte ca cel învinuit să aibă de față pe pârâșii lui și să aibă putința să se apere pentru vina sa.
\par 17 Adunându-se deci ei aici și nefăcând eu nici o amânare, a doua zi am stat la judecată și am poruncit să fie adus bărbatul.
\par 18 Dar pârâșii care s-au ridicat împotriva lui nu i-au adus nici o învinuire dintre cele rele, pe care le bănuiam eu,
\par 19 Ci aveau cu el niște neînțelegeri cu privire la religia lor și la un oarecare Iisus mort, de Care Pavel zice că trăiește.
\par 20 Și nedumerindu-mă cu privire la cercetarea acestor lucruri, l-am întrebat dacă voiește să meargă la Ierusalim și să fie judecat acolo pentru acestea.
\par 21 Dar Pavel, cerând să fie reținut pentru judecata Cezarului, am poruncit să fie ținut până ce îl voi trimite la Cezarul.
\par 22 Iar Agripa a zis către Festus: Aș vrea să aud și eu pe acest om. Iar el a zis: Mâine îl vei auzi.
\par 23 Deci a doua zi, Agripa și Berenice venind cu mare alai și intrând în sala de judecată împreună cu tribunii și cu bărbații cei mai de frunte ai cetății, Festus a dat poruncă să fie adus Pavel.
\par 24 Și a zis Festus: Rege Agripa, și voi toți bărbații care sunteți cu noi de față, vedeți pe acela pentru care toată mulțimea iudeilor a venit la mine, și în Ierusalim și aici, strigând că el nu trebuie să mai trăiască.
\par 25 Iar eu am înțeles că n-a făcut nimic vrednic de moarte; iar el însuși cerând să fie judecat de Cezarul, am hotărât să-l trimit.
\par 26 Dar ceva sigur să scriu stăpânului despre el, nu am. De aceea l-am adus înaintea voastră și mai ales înaintea ta, rege Agripa, ca, după ce va fi cercetat, să am ce să scriu,
\par 27 Căci mi se pare nepotrivit să-l trimit legat, fără să arăt învinuirile ce i se aduc.

\chapter{26}

\par 1 Agripa a zis către Pavel: Îți este îngăduit să vorbești pentru tine. Atunci Pavel, întinzând mâna, se apăra:
\par 2 Mă socotesc fericit, o, rege Agripa, că astăzi, înaintea ta, pot să mă apăr de toate câte mă învinuiesc iudeii;
\par 3 Mai ales, pentru că tu cunoști toate obiceiurile și neînțelegerile iudeilor. De aceea te rog să mă asculți cu îngăduință.
\par 4 Viețuirea mea din tinerețe, cum a fost ea de la început în poporul meu și în Ierusalim, o știu toți iudeii.
\par 5 Dacă vor să dea mărturie, ei știu despre mine, de mult, că am trăit ca fariseu, în tagma cea mai riguroasă a religiei noastre.
\par 6 Și acum stau la judecată pentru nădejdea făgăduinței făcute de Dumnezeu către părinții noștri,
\par 7 Și la care cele douăsprezece seminții ale noastre, slujind lui Dumnezeu fără încetare, zi și noapte, nădăjduiesc să ajungă. Pentru nădejdea aceasta, o, rege Agripa, sunt pârât de iudei.
\par 8 De ce se socotește la voi lucru de necrezut că Dumnezeu înviază pe cei morți?
\par 9 Eu unul am socotit, în sinea mea, că față de numele lui Iisus Nazarineanul trebuia să fac multe împotrivă;
\par 10 Ceea ce am și făcut în Ierusalim, și pe mulți dintre sfinți i-am închis în temnițe cu puterea pe care o luasem de la arhierei. Iar când erau dați la moarte, mi-am dat și eu încuviințarea.
\par 11 Și îi pedepseam adesea prin toate sinagogile și-i sileam să hulească și, mult înfuriindu-mă împotriva lor, îi urmăream până și prin cetățile de din afară;
\par 12 Și în felul acesta, mergând și la Damasc, cu putere și cu însărcinare de la arhierei,
\par 13 Am văzut, o, rege, la amiază, în calea mea, o lumină din cer, mai puternică decât strălucirea soarelui, strălucind împrejurul meu și a celor ce mergeau împreună cu mine.
\par 14 Și noi toți căzând la pământ, eu am auzit un glas care-mi zicea în limba evreiască: Saule, Saule, de ce Mă prigonești? Greu îți este să lovești în țepușă cu piciorul.
\par 15 Iar eu am zis: Cine ești Doamne? Iar Domnul a zis: Eu sunt Iisus, pe Care tu Îl prigonești.
\par 16 Dar, scoală-te și stai pe picioarele tale. Căci spre aceasta M-am arătat ție: ca să te rânduiesc slujitor și martor, și al celor ce ai văzut, și al celor întru care Mă voi arăta ție.
\par 17 Alegându-te pe tine din popor și din neamurile la care te trimit,
\par 18 Să le deschizi ochii, ca să se întoarcă de la întuneric la lumină și de la stăpânirea lui satana la Dumnezeu, ca să ia iertarea păcatelor și parte cu cei ce s-au sfințit, prin credința în Mine.
\par 19 Drept aceea, rege Agripa, n-am fost neascultător cereștii arătări;
\par 20 Ci mai întâi celor din Damasc și din Ierusalim, și din toată țara Iudeii, și neamurilor le-am vestit să se pocăiască și să se întoarcă la Dumnezeu, făcând lucruri vrednice de pocăință.
\par 21 Pentru acestea, iudeii, prinzându-mă în templu, încercau să mă ucidă.
\par 22 Dobândind deci ajutorul de la Dumnezeu, am stat până în ziua aceasta, mărturisind la mic și la mare, fără să spun nimic decât ceea ce și proorocii și Moise au spus că va să fie:
\par 23 Că Hristos avea să pătimească și să fie cel dintâi înviat din morți și să vestească lumină și poporului și neamurilor.
\par 24 Și acestea grăind el, întru apărarea sa, i-a zis Festus cu glas mare: Pavele, ești nebun! Învățătura ta cea multă te duce la nebunie.
\par 25 Iar Pavel a zis: Nu sunt nebun, prea puternice Festus, ci grăiesc cuvintele adevărului și ale înțelepciunii.
\par 26 Regele știe despre acestea, și în fața lui vorbesc fără sfială, fiind încredințat că nimic nu i-a rămas ascuns, pentru că aceasta nu s-a întâmplat, într-un ungher.
\par 27 Crezi tu, rege Agripa, în prooroci? Știu că crezi.
\par 28 Iar Agripa a zis către Pavel: Cu puțin de nu mă îndupleci să mă fac și eu creștin!
\par 29 Iar Pavel a zis: Ori cu puțin, ori cu mult, eu m-aș ruga lui Dumnezeu ca nu numai tu, ci și toți care mă ascultă astăzi să fie așa cum sunt și eu, afară de aceste lanțuri.
\par 30 Și s-a ridicat și regele și guvernatorul și Berenice și cei care ședeau împreună cu ei,
\par 31 Și plecând, vorbeau unii cu alții zicând: Omul acesta n-a făcut nimic vrednic de moarte sau de lanțuri.
\par 32 Iar Agripa a zis lui Festus: Acest om putea să fie lăsat liber, dacă n-ar fi cerut să fie judecat de Cezarul.

\chapter{27}

\par 1 Iar după ce s-a hotărât să plecăm pe apă în Italia, au dat în primire pe Pavel și pe alți câțiva legați unui sutaș cu numele Iuliu, din cohorta Augusta.
\par 2 Și întorcându-se pe o corabie de la Adramit, care avea să treacă prin locurile de pe coasta Asiei, am plecat; și era cu noi Aristarh, macedonean din Tesalonic.
\par 3 Și a doua zi am ajuns la Sidon. Iuliu, purtându-se față de Pavel cu omenie, i-a dat voie să se ducă la prieteni ca să primească purtarea lor de grijă.
\par 4 Și plecând de acolo, am plutit pe lângă Cipru, pentru că vânturile erau împotrivă.
\par 5 Și străbătând marea Ciliciei și a Pamfiliei, am sosit la Mira Liciei.
\par 6 Și găsind sutașul acolo o corabie din Alexandria plutind spre Italia, ne-a suit în ea.
\par 7 Și multe zile plutind cu încetineală, abia am ajuns în dreptul Cnidului și, fiindcă vântul nu ne slăbea am plutit pe sub Creta, pe lângă Salmone.
\par 8 Și abia trecând noi pe lângă ea, am ajuns într-un loc numit Limanuri Bune, de care era aproape orașul Lasea.
\par 9 Și trecând multă vreme și plutirea fiind periculoasă, fiindcă trecuse și postul (sărbătorii Ispășirii, care se ținea la evrei toamna), Pavel îi îndemna,
\par 10 Zicându-le: Bărbaților, văd că plutirea va să fie cu necaz și cu multă pagubă, nu numai pentru încărcătură și pentru corabie, ci și pentru sufletele noastre.
\par 11 Iar sutașul se încredea mai mult în cârmaci și în stăpânul corabiei decât în cele spuse de Pavel.
\par 12 Și limanul nefiind bun de iernat, cei mai mulți dintre ei au dat sfatul să plecăm de acolo și, dacă s-ar putea, să ajungem și să iernăm la Fenix, un port al Cretei, deschis spre vântul de miazăzi-apus și spre vântul de miazănoapte-apus.
\par 13 Și suflând ușor un vânt de miazăzi și crezând că sunt în stare să-și împlinească gândul, ridicând ancora, pluteau cât mai aproape de Creta.
\par 14 Și nu după multă vreme s-a pornit asupra ei un vânt puternic, numit Euroclidon (dinspre miazănoapte-răsărit).
\par 15 Și smulgând corabia, iar ea neputând să meargă împotriva vântului, ne-am lăsat duși în voia lui.
\par 16 Și trecând pe lângă o insulă mică, numită Clauda, cu greu am putut să fim stăpâni pe corabie.
\par 17 Și după ce au ridicat-o, au folosit unelte ajutătoare, încingând corabia pe dedesubt. Și temându-se să nu cadă în Sirta, au lăsat pânzele jos și erau duși așa.
\par 18 Și fiind tare loviți de furtună, în ziua următoare au aruncat încărcătura.
\par 19 Și a treia zi, cu mâinile lor, au aruncat uneltele corăbiei.
\par 20 Și nearătându-se nici soarele, nici stelele, timp de mai multe zile, și amenințând furtună mare, ni se luase orice nădejde de scăpare.
\par 21 Și fiindcă nu mâncaseră de mult, Pavel, stând în mijlocul lor, le-a zis: Trebuia, o, bărbaților, ca ascultându-mă pe mine, să nu fi plecat din Creta; și n-ați fi îndurat nici primejdia aceasta, nici paguba aceasta.
\par 22 Dar acum vă îndemn să aveți voie bună, căci nici un suflet dintre voi nu va pieri, ci numai corabia.
\par 23 Căci mi-a apărut în noaptea aceasta un înger al Dumnezeului, al Căruia eu sunt și Căruia mă închin,
\par 24 Zicând: Nu te teme, Pavele. Tu trebuie să stai înaintea Cezarului; și iată, Dumnezeu ți-a dăruit pe toți cei ce sunt în corabie cu tine.
\par 25 De aceea, bărbaților, aveți curaj, căci am încredere în Dumnezeu, că așa va fi după cum mi s-a spus.
\par 26 Și trebuie să ajungem pe o insulă.
\par 27 Și când a fost a paisprezecea noapte de când eram purtați încoace și încolo pe Adriatica, pe la miezul nopții corăbierii au presimțit că se apropie de un țărm.
\par 28 Și aruncând măsurătoarea în jos au găsit douăzeci de stânjeni și, trecând puțin mai departe și măsurând iarăși, au găsit cincisprezece stânjeni.
\par 29 Și temându-se ca nu cumva să nimerim pe locuri stâncoase, au aruncat patru ancore de la partea din urmă a corăbiei, și doreau să se facă ziuă.
\par 30 Dar corăbierii căutau să fugă din corabie și au coborât luntrea în mare, sub motiv că vor să întindă și ancorele de la partea dinainte.
\par 31 Pavel a spus sutașului și ostașilor: Dacă aceștia nu rămân în corabie, voi nu puteți să scăpați.
\par 32 Atunci ostașii au tăiat funiile luntrei și au lăsat-o să cadă.
\par 33 Iar, până să se facă ziuă, Pavel îi ruga pe toți să mănânce, zicându-le: Paisprezece zile sunt azi de când n-ați mâncat, așteptând și nimic gustând.
\par 34 De aceea, vă rog să mâncați, căci aceasta este spre scăparea voastră. Că nici unuia din voi un fir de păr din cap nu-i va pieri.
\par 35 Și zicând acestea și luând pâine, a mulțumit lui Dumnezeu înaintea tuturor și, frângând, a început să mănânce.
\par 36 Și devenind toți voioși, au luat și ei și au mâncat.
\par 37 Și eram în corabie, de toți, două sute șaptezeci și șase de suflete.
\par 38 Și săturându-se de bucate, au ușurat corabia, aruncând grâul în mare.
\par 39 Și când s-a făcut ziuă, ei n-au cunoscut pământul, dar au zărit un sân de mare, având țărm nisipos, în care voiau, dacă ar putea, să scoată corabia.
\par 40 Și desfăcând ancorele, le-au lăsat în mare, slăbind totodată funiile cârmelor și, ridicând pânza din frunte în bătaia vântului, se îndreptau spre țărm.
\par 41 Și căzând pe un dâmb de nisip au înțepenit corabia și partea dinainte, înfigându-se, stătea neclintită, iar partea dinapoi se sfărâma de puterea valurilor.
\par 42 Iar ostașii au făcut sfat să omoare pe cei legați, ca să nu scape vreunul, înotând.
\par 43 Dar sutașul, voind să ferească pe Pavel, i-a împiedicat de la gândul lor și a poruncit ca aceia care pot să înoate, aruncându-se cei dintâi, să iasă la uscat;
\par 44 Iar ceilalți, care pe scânduri, care pe câte ceva de la corabie. Și așa au ajuns cu toții să scape, la uscat.

\chapter{28}

\par 1 Și după ce am scăpat, am aflat că insula se numește Malta.
\par 2 Iar locuitorii ei ne arătau o deosebită omenie, căci, aprinzând foc, ne-au luat pe toți la ei din pricina ploii care era și a frigului.
\par 3 Și strângând Pavel grămadă de găteje și punându-le în foc, o viperă a ieșit de căldură și s-a prins de mâna lui.
\par 4 Și când locuitorii au văzut vipera atârnând de mâna lui, ziceau unii către alții: Desigur că ucigaș este omul acesta, pe care dreptatea nu l-a lăsat să trăiască, deși a scăpat din mare.
\par 5 Deci el, scuturând vipera în foc, n-a pătimit nici un rău.
\par 6 Iar ei așteptau ca el să se umfle, sau să cadă deodată mort. Dar așteptând ei mult și văzând că nu i se întâmplă nimic rău, și-au schimbat gândul și ziceau că el este un zeu.
\par 7 Și împrejurul acelui loc erau țarinile căpeteniei insulei, Publius, care, primindu-ne, ne-a găzduit prietenos trei zile.
\par 8 Și s-a întâmplat că tatăl lui Publius zăcea în pat, cuprins de friguri și de urdinare cu sânge, la care intrând Pavel și rugându-se, și-a pus mâinile peste el și l-a vindecat.
\par 9 Și întâmplându-se aceasta, veneau la el și ceilalți din insulă care aveau boli și se vindecau;
\par 10 Și aceștia ne-au cinstit mult și, când am plecat, ne-au pus la îndemână toate cele de trebuință.
\par 11 După trei luni am pornit cu o corabie din Alexandria, care iernase în insulă și care avea pe ea însemnul Dioscurilor.
\par 12 Și ajungând la Siracuza, am rămas acolo trei zile.
\par 13 De unde, înconjurând, am sosit la Regium. Și după o zi, suflând vânt de miazăzi, am ajuns la Puteoli în cealaltă zi.
\par 14 Găsind acolo frați, am fost rugați să rămânem la ei șapte zile. Și așa am venit la Roma.
\par 15 Și de acolo, auzind frații cele despre noi, au venit întru întâmpinarea noastră până la Forul lui Apius și la Trei Taverne, pe care, văzându-i, Pavel a mulțumit lui Dumnezeu și s-a îmbărbătat.
\par 16 Iar când am intrat în Roma, sutașul a predat pe cei legați comandantului taberei, iar lui Pavel i s-a îngăduit să locuiască aparte cu ostașul care îl păzea.
\par 17 Și după trei zile Pavel a chemat la el pe cei care erau fruntașii iudeilor. Și, adunându-se, zicea către ei: Bărbați frați, deși eu n-am făcut nimic rău împotriva poporului (nostru) sau a datinilor părintești, am fost predat de la Ierusalim, în mâinile romanilor.
\par 18 Aceștia, după ce m-au cercetat, voiau să-mi dea drumul, fiindcă nu era în mine nici o vină vrednică de moarte.
\par 19 Dar iudeii, împotrivindu-mi-se, am fost nevoit să cer să fiu judecat de Cezarul, dar nu că aș avea de adus vreo pâră neamului meu.
\par 20 Deci pentru această cauză v-am chemat să vă văd și să vorbesc cu voi. Căci pentru nădejdea lui Israel mă aflu eu în acest lanț.
\par 21 Iar ei au zis către el: Noi n-am primit din Iudeea nici scrisori despre tine, nici nu a venit cineva dintre frați, ca să ne vestească sau să ne vorbească ceva rău despre tine.
\par 22 Dar dorim să auzim de la tine cele ce gândești; căci despre eresul acesta ne este cunoscut; că pretutindeni i se stă împotrivă.
\par 23 Deci, rânduindu-i o zi, au venit la el, la gazdă, mai mulți. Și de dimineața până seara, el le vorbea, dând mărturie despre împărăția lui Dumnezeu, căutând să-i încredințeze despre Iisus din Legea lui Moise și din prooroci.
\par 24 Și unii credeau celor spuse, iar alții nu credeau.
\par 25 Și neînțelegându-se unii cu alții, au plecat, zicând Pavel un cuvânt că: Bine a vorbit Duhul Sfânt prin Isaia proorocul, către părinții noștri,
\par 26 Când a zis: "Mergi la poporul acesta și zi: Cu auzul veți auzi și nu veți înțelege și uitându-vă veți privi, dar nu veți vedea.
\par 27 Căci inima acestui popor s-a învârtoșat și cu urechile greu au auzit și ochii lor i-au închis. Ca nu cumva să vadă cu ochii și să audă cu urechile și cu inima să înțeleagă și să se întoarcă și Eu să-l vindec".
\par 28 Deci cunoscut să vă fie vouă că această mântuire a lui Dumnezeu s-a trimis păgânilor, și ei vor asculta.
\par 29 Și după ce a zis el acestea, iudeii au plecat având între ei mare neînțelegere.
\par 30 Iar Pavel a rămas doi ani întregi în casa luată de el cu chirie, și primea pe toți care veneau la el,
\par 31 Propovăduind împărăția lui Dumnezeu și învățând cele despre Domnul Iisus Hristos, cu toată îndrăzneala și fără nici o piedică.


\end{document}