\begin{document}

\title{1 Corinteni}


\chapter{1}

\par 1 Pavel, chemat apostol al lui Hristos, prin voia lui Dumnezeu, ?i fratele Sostene,
\par 2 Bisericii lui Dumnezeu care este în Corint, celor sfin?i?i în Iisus Hristos, celor numi?i sfin?i, împreuna cu to?i cei ce cheama numele Domnului nostru Iisus Hristos în tot locul, ?i al lor ?i al nostru:
\par 3 Har voua ?i pace de la Dumnezeu, Tatal nostru, ?i de la Domnul nostru Iisus Hristos.
\par 4 Mul?umesc totdeauna Dumnezeului meu pentru voi, pentru harul lui Dumnezeu, dat voua în Hristos Iisus.
\par 5 Caci întru El v-a?i îmboga?it deplin întru toate, în tot cuvântul ?i în toata cuno?tin?a;
\par 6 Astfel marturia lui Hristos s-a întarit în voi,
\par 7 Încât voi nu sunte?i lipsi?i de nici un dar, a?teptând aratarea Domnului nostru Iisus Hristos,
\par 8 Care va va ?i întari pâna la sfâr?it, ca sa fi?i nevinova?i în ziua Domnului nostru Iisus Hristos.
\par 9 Credincios este Dumnezeu, prin Care a?i fost chema?i la împarta?irea cu Fiul Sau Iisus Hristos, Domnul nostru.
\par 10 Va îndemn, fra?ilor, pentru numele Domnului nostru Iisus Hristos, ca to?i sa vorbi?i la fel ?i sa nu fie dezbinari între voi; ci sa fi?i cu totul uni?i în acela?i cuget ?i în aceea?i în?elegere.
\par 11 Caci, fra?ii mei, despre voi, prin cei din casa lui Hloe mi-a venit ?tire ca la voi sunt certuri;
\par 12 ?i spun aceasta, ca fiecare dintre voi zice: Eu sunt al lui Pavel, iar eu sunt al lui Apollo, iar eu sunt al lui Chefa, iar eu sunt al lui Hristos!
\par 13 Oare s-a împar?it Hristos? Nu cumva s-a rastignit Pavel pentru voi? Sau fost-a?i boteza?i în numele lui Pavel?
\par 14 Mul?umesc lui Dumnezeu ca pe nici unul din voi n-am botezat, decât pe Crispus ?i pe Gaius,
\par 15 Ca sa nu zica cineva ca a?i fost boteza?i în numele meu.
\par 16 Am botezat ?i casa lui ?tefana; afara de ace?tia nu ?tiu sa mai fi botezat pe altcineva.
\par 17 Caci Hristos nu m-a trimis ca sa botez, ci sa binevestesc, dar nu cu în?elepciunea cuvântului, ca sa nu ramâna zadarnica crucea lui Hristos.
\par 18 Caci cuvântul Crucii, pentru cei ce pier, este nebunie; iar pentru noi, cei ce ne mântuim, este puterea lui Dumnezeu.
\par 19 Caci scris este: "Pierde-voi în?elepciunea în?elep?ilor ?i ?tiin?a celor înva?a?i voi nimici-o".
\par 20 Unde este în?eleptul? Unde e carturarul? Unde e cercetatorul acestui veac? Au n-a dovedit Dumnezeu nebuna în?elepciunea lumii acesteia?
\par 21 Caci de vreme ce întru în?elepciunea lui Dumnezeu lumea n-a cunoscut prin în?elepciune pe Dumnezeu, a binevoit Dumnezeu sa mântuiasca pe cei ce cred prin nebunia propovaduirii.
\par 22 Fiindca ?i iudeii cer semne, iar elinii cauta în?elepciune,
\par 23 Însa noi propovaduim pe Hristos cel rastignit: pentru iudei, sminteala; pentru neamuri, nebunie.
\par 24 Dar pentru cei chema?i, ?i iudei ?i elini: pe Hristos, puterea lui Dumnezeu ?i în?elepciunea lui Dumnezeu.
\par 25 Pentru ca fapta lui Dumnezeu, socotita de catre oameni nebunie, este mai în?eleapta decât în?elepciunea lor ?i ceea ce se pare ca slabiciune a lui Dumnezeu, mai puternica decât taria oamenilor.
\par 26 Caci, privi?i chemarea voastra, fra?ilor, ca nu mul?i sunt în?elep?i dupa trup, nu mul?i sunt puternici, nu mul?i sunt de bun neam;
\par 27 Ci Dumnezeu ?i-a ales pe cele nebune ale lumii, ca sa ru?ineze pe cei în?elep?i; Dumnezeu ?i-a ales pe cele slabe ale lumii, ca sa le ru?ineze pe cele tari;
\par 28 Dumnezeu ?i-a ales pe cele de neam jos ale lumii, pe cele nebagate în seama, pe cele ce nu sunt, ca sa nimiceasca pe cele ce sunt,
\par 29 Ca nici un trup sa nu se laude înaintea lui Dumnezeu.
\par 30 Din El, dar, sunte?i voi în Hristos Iisus, Care pentru noi S-a facut în?elepciune de la Dumnezeu ?i dreptate ?i sfin?ire ?i rascumparare,
\par 31 Pentru ca, dupa cum este scris: "Cel ce se lauda în Domnul sa se laude".

\chapter{2}

\par 1 ?i eu, fra?ilor, când am venit la voi ?i v-am vestit taina lui Dumnezeu, n-am venit ca iscusit cuvântator sau ca în?elept.
\par 2 Caci am judecat sa nu ?tiu între voi altceva, decât pe Iisus Hristos, ?i pe Acesta rastignit.
\par 3 ?i eu întru slabiciune ?i cu frica ?i cu cutremur mare am fost la voi.
\par 4 Iar cuvântul meu ?i propovaduirea mea nu stateau în cuvinte de înduplecare ale în?elepciunii omene?ti, ci în adeverirea Duhului ?i a puterii,
\par 5 Pentru ca credin?a voastra sa nu fie în în?elepciunea oamenilor, ci în puterea lui Dumnezeu.
\par 6 ?i în?elepciunea o propovaduim la cei desavâr?i?i, dar nu în?elepciunea acestui veac, nici a stapânitorilor acestui veac, care sunt pieritori,
\par 7 Ci propovaduim în?elepciunea de taina a lui Dumnezeu, ascunsa, pe care Dumnezeu a rânduit-o mai înainte de veci, spre slava noastra,
\par 8 Pe care nici unul dintre stapânitorii acestui veac n-a cunoscut-o, caci, daca ar fi cunoscut-o, n-ar fi rastignit pe Domnul slavei;
\par 9 Ci precum este scris: "Cele ce ochiul n-a vazut ?i urechea n-a auzit, ?i la inima omului nu s-au suit, pe acestea le-a gatit Dumnezeu celor ce-L iubesc pe El".
\par 10 Iar noua ni le-a descoperit Dumnezeu prin Duhul Sau, fiindca Duhul toate le cerceteaza, chiar ?i adâncurile lui Dumnezeu.
\par 11 Caci cine dintre oameni ?tie ale omului, decât duhul omului, care este în el? A?a ?i cele ale lui Dumnezeu, nimeni nu le-a cunoscut, decât Duhul lui Dumnezeu.
\par 12 Iar noi n-am primit duhul lumii, ci Duhul cel de la Dumnezeu, ca sa cunoa?tem cele daruite noua de Dumnezeu;
\par 13 Pe care le ?i graim, dar nu în cuvinte înva?ate din în?elepciunea omeneasca, ci în cuvinte înva?ate de la Duhul Sfânt, lamurind lucruri duhovnice?ti oamenilor duhovnice?ti.
\par 14 Omul firesc nu prime?te cele ale Duhului lui Dumnezeu, caci pentru el sunt nebunie ?i nu poate sa le în?eleaga, fiindca ele se judeca duhovnice?te.
\par 15 Dar omul duhovnicesc toate le judeca, pe el însa nu-l judeca nimeni;
\par 16 Caci "Cine a cunoscut gândul Domnului, ca sa-L înve?e pe El?" Noi însa avem gândul lui Hristos.

\chapter{3}

\par 1 ?i eu, fra?ilor, n-am putut sa va vorbesc ca unor oameni duhovnice?ti, ci ca unora trupe?ti, ca unor prunci în Hristos.
\par 2 Cu lapte v-am hranit, nu cu bucate, caci înca nu putea?i mânca ?i înca nici acum nu pute?i,
\par 3 Fiindca sunte?i tot trupe?ti. Câta vreme este între voi pizma ?i cearta ?i dezbinari, nu sunte?i, oare, trupe?ti ?i nu dupa firea omeneasca umbla?i?
\par 4 Caci, când zice unul: Eu sunt al lui Pavel, iar altul: Eu sunt al lui Apollo, au nu sunte?i oameni trupe?ti?
\par 5 Dar ce este Apollo? ?i ce este Pavel? Slujitori prin care a?i crezut voi ?i dupa cum i-a dat Domnul fiecaruia.
\par 6 Eu am sadit, Apollo a udat, dar Dumnezeu a facut sa creasca.
\par 7 Astfel nici cel ce sade?te nu e ceva, nici cel ce uda, ci numai Dumnezeu care face sa creasca.
\par 8 Cel care sade?te ?i cel care uda sunt una ?i fiecare î?i va lua plata dupa osteneala sa.
\par 9 Caci noi împreuna-lucratori cu Dumnezeu suntem; voi sunte?i ogorul lui Dumnezeu, zidirea lui Dumnezeu.
\par 10 Dupa harul lui Dumnezeu, cel dat mie, eu, ca un în?elept me?ter, am pus temelia; iar altul zide?te. Dar fiecare sa ia seama cum zide?te;
\par 11 Caci nimeni nu poate pune alta temelie, decât cea pusa, care este Iisus Hristos.
\par 12 Iar de zide?te cineva pe aceasta temelie: aur, argint, sau pietre scumpe, lemne, fân, trestie.
\par 13 Lucrul fiecaruia se va face cunoscut; îl va vadi ziua (Domnului). Pentru ca în foc se descopera, ?i focul însu?i va lamuri ce fel este lucrul fiecaruia.
\par 14 Daca lucrul cuiva, pe care l-a zidit, va ramâne, va lua plata.
\par 15 Daca lucrul cuiva se va arde, el va fi pagubit; el însa se va mântui, dar a?a ca prin foc.
\par 16 Nu ?ti?i, oare, ca voi sunte?i templu al lui Dumnezeu ?i ca Duhul lui Dumnezeu locuie?te în voi?
\par 17 De va strica cineva templul lui Dumnezeu, îl va strica Dumnezeu pe el, pentru ca sfânt este templul lui Dumnezeu, care sunte?i voi.
\par 18 Nimeni sa nu se amageasca. Daca i se pare cuiva, între voi, ca este în?elept în veacul acesta, sa se faca nebun, ca sa fie în?elept.
\par 19 Caci în?elepciunea lumii acesteia este nebunie înaintea lui Dumnezeu, pentru ca scris este: "El prinde pe cei în?elep?i în viclenia lor".
\par 20 ?i iara?i: "Domnul cunoa?te gândurile în?elep?ilor, ca sunt de?arte".
\par 21 A?a ca nimeni sa nu se laude cu oameni. Caci toate sunt ale voastre:
\par 22 Fie Pavel, fie Apollo, fie Chefa, fie lumea, fie via?a, fie moartea, fie cele de fa?a, fie cele viitoare, toate sunt ale voastre.
\par 23 Iar voi sunte?i ai lui Hristos, iar Hristos al lui Dumnezeu.

\chapter{4}

\par 1 A?a sa ne socoteasca pe noi fiecare om: ca slujitori ai lui Hristos ?i ca iconomi ai tainelor lui Dumnezeu.
\par 2 Iar, la iconomi, mai ales, se cere ca fiecare sa fie aflat credincios.
\par 3 Dar mie prea pu?in îmi este ca sunt judecat de voi sau de vreo omeneasca judecata de toata ziua; fiindca nici eu nu ma judec pe mine însumi.
\par 4 Caci nu ma ?tiu vinovat cu nimic, dar nu întru aceasta m-am îndreptat. Cel care ma judeca pe mine este Domnul.
\par 5 De aceea, nu judeca?i ceva înainte de vreme, pâna ce nu va veni Domnul, Care va lumina cele ascunse ale întunericului ?i va vadi sfaturile inimilor. ?i atunci fiecare va avea de la Dumnezeu lauda.
\par 6 ?i acestea, fra?ilor, le-am zis ca despre mine ?i despre Apollo, dar ele sunt pentru voi, ca sa înva?a?i din pilda noastra, sa nu trece?i peste ce e scris, ca sa nu va fali?i unul cu altul împotriva celuilalt.
\par 7 Caci cine te deosebe?te pe tine? ?i ce ai, pe care sa nu-l fi primit? Iar daca l-ai primit, de ce te fale?ti, ca ?i cum nu l-ai fi primit?
\par 8 Iata, sunte?i satui; iata, v-a?i îmboga?it; fara de noi a?i domnit, ?i, macar nu a?i domnit, ca ?i noi sa domnim împreuna cu voi.
\par 9 Caci mi se pare ca Dumnezeu, pe noi, apostolii, ne-a aratat ca pe cei din urma oameni, ca pe ni?te osândi?i la moarte, fiindca ne-am facut priveli?te lumii ?i îngerilor ?i oamenilor.
\par 10 Noi suntem nebuni pentru Hristos; voi însa în?elep?i întru Hristos. Noi suntem slabi; voi însa sunte?i tari. Voi sunte?i întru slava, iar noi suntem întru necinste!
\par 11 Pâna în ceasul de acum flamânzim ?i însetam; suntem goi ?i suntem palmui?i ?i pribegim,
\par 12 ?i ne ostenim, lucrând cu mâinile noastre. Ocarâ?i fiind, binecuvântam. Prigoni?i fiind, rabdam.
\par 13 Huli?i fiind, ne rugam. Am ajuns ca gunoiul lumii, ca maturatura tuturor, pâna astazi.
\par 14 Nu ca sa va ru?inez va scriu acestea, ci ca sa va dojenesc, ca pe ni?te copii ai mei iubi?i.
\par 15 Caci de a?i avea zeci de mii de înva?atori în Hristos, totu?i nu ave?i mul?i parin?i. Caci eu v-am nascut prin Evanghelie în Iisus Hristos.
\par 16 Deci, va rog, sa-mi fi?i mie urmatori, precum ?i eu lui Hristos.
\par 17 Pentru aceasta am trimis la voi pe Timotei, care este fiul meu iubit ?i credincios în Domnul. El va va aduce aminte caile mele cele în Hristos Iisus, cum înva? eu pretutindeni în toata Biserica.
\par 18 ?i unii, crezând ca n-am sa mai vin la voi, s-au seme?it.
\par 19 Dar eu voi veni la voi degraba - daca Domnul va voi - ?i voi cunoa?te nu cuvântul celor ce s-au seme?it, ci puterea lor.
\par 20 Caci împara?ia lui Dumnezeu nu sta în cuvânt, ci în putere.
\par 21 Ce voi?i? Sa vin la voi cu toiagul sau sa vin cu dragoste ?i cu duhul blânde?ii?

\chapter{5}

\par 1 Îndeob?te se aude ca la voi e desfrânare, ?i o astfel de desfrânare cum nici între neamuri nu se pomene?te, ca unul sa traiasca cu femeia tatalui sau.
\par 2 Iar voi v-a?i seme?it, în loc mai degraba sa va fi întristat, ca sa fie scos din mijlocul vostru cel ce a savâr?it aceasta fapta.
\par 3 Ci eu, de?i departe cu trupul, însa de fa?a cu duhul, am ?i judecat, ca ?i cum a? fi de fa?a, pe cel ce a facut una ca aceasta:
\par 4 În numele Domnului nostru Iisus Hristos, adunându-va voi ?i duhul meu, cu puterea Domnului nostru Iisus Hristos,
\par 5 Sa da?i pe unul ca acesta satanei, spre pieirea trupului, ca duhul sa se mântuiasca în ziua Domnului Iisus.
\par 6 Seme?ia voastra nu e buna. Oare nu ?ti?i ca pu?in aluat dospe?te toata framântatura?
\par 7 Cura?i?i aluatul cel vechi, ca sa fi?i framântatura noua, precum ?i sunte?i fara aluat; caci Pa?tile nostru Hristos S-a jertfit pentru noi.
\par 8 De aceea sa praznuim nu cu aluatul cel vechi, nici cu aluatul rauta?ii ?i al vicle?ugului, ci cu azimele cura?iei ?i ale adevarului.
\par 9 V-am scris în epistola sa nu va amesteca?i cu desfrâna?ii;
\par 10 Dar nu am spus, desigur, despre desfrâna?ii acestei lumi, sau despre lacomi, sau despre rapitori, sau despre închinatorii la idoli, caci altfel ar trebui sa ie?i?i afara din lume.
\par 11 Dar eu v-am scris acum sa nu va amesteca?i cu vreunul, care, numindu-se frate, va fi desfrânat, sau lacom, sau închinator la idoli, sau ocarâtor, sau be?iv, sau rapitor. Cu unul ca acesta nici sa nu ?ede?i la masa.
\par 12 Caci ce am eu ca sa judec ?i pe cei din afara? Însa pe cei dinauntru, oare, nu-i judeca?i voi?
\par 13 Iar pe cei din afara îi va judeca Dumnezeu. Scoate?i afara dintre voi pe cel rau.

\chapter{6}

\par 1 Îndrazne?te, oare, cineva dintre voi, având vreo pâra împotriva altuia, sa se judece înaintea celor nedrep?i ?i nu înaintea celor sfin?i?
\par 2 Au nu ?ti?i ca sfin?ii vor judeca lumea? ?i daca lumea este judecata de voi, oare sunte?i voi nevrednici sa judeca?i lucruri atât de mici?
\par 3 Nu ?ti?i, oare, ca noi vom judeca pe îngeri? Cu cât mai mult cele lume?ti?
\par 4 Deci daca ave?i judeca?i lume?ti, pune?i pe cei nebaga?i în seama din Biserica, ca sa va judece.
\par 5 O spun spre ru?inea voastra. Nu este, oare, între voi nici un om în?elept, care sa poata judeca între frate ?i frate?
\par 6 Ci frate cu frate se judeca, ?i aceasta înaintea necredincio?ilor?
\par 7 Negre?it, ?i aceasta este o scadere pentru voi, ca ave?i judeca?i unii cu al?ii. Pentru ce nu suferi?i mai bine strâmbatatea? Pentru ce nu rabda?i mai bine paguba?
\par 8 Ci voi în?iva face?i strâmbatate ?i aduce?i paguba, ?i aceasta, fra?ilor!
\par 9 Nu ?ti?i, oare, ca nedrep?ii nu vor mo?teni împara?ia lui Dumnezeu? Nu va amagi?i: Nici desfrâna?ii, nici închinatorii la idoli, nici adulterii, nici malahienii, nici sodomi?ii,
\par 10 Nici furii, nici lacomii, nici be?ivii, nici batjocoritorii, nici rapitorii nu vor mo?teni împara?ia lui Dumnezeu.
\par 11 ?i a?a era?i unii dintre voi. Dar v-a?i spalat, dar v-a?i sfin?it, dar v-a?i îndreptat în numele Domnului Iisus Hristos ?i în Duhul Dumnezeului nostru.
\par 12 Toate îmi sunt îngaduite, dar nu toate îmi sunt de folos. Toate îmi sunt îngaduite, dar nu ma voi lasa biruit de ceva.
\par 13 Bucatele sunt pentru pântece ?i pântecele pentru bucate ?i Dumnezeu va nimici ?i pe unul ?i pe celelalte. Trupul însa nu e pentru desfrânare, ci pentru Domnul, ?i Domnul este pentru trup.
\par 14 Iar Dumnezeu, Care a înviat pe Domnul, ne va învia ?i pe noi prin puterea Sa.
\par 15 Au nu ?ti?i ca trupurile voastre sunt madularele lui Hristos? Luând deci madularele lui Hristos le voi face madularele unei desfrânate? Nicidecum!
\par 16 Sau nu ?ti?i ca cel ce se alipe?te de desfrânate este un singur trup cu ea? "Caci vor fi - zice Scriptura - cei doi un singur trup".
\par 17 Iar cel ce se alipe?te de Domnul este un duh cu El.
\par 18 Fugi?i de desfrânare! Orice pacat pe care-l va savâr?i omul este în afara de trup. Cine se deda însa desfrânarii pacatuie?te în însu?i trupul sau.
\par 19 Sau nu ?ti?i ca trupul vostru este templu al Duhului Sfânt care este în voi, pe care-L ave?i de la Dumnezeu ?i ca voi nu sunte?i ai vo?tri?
\par 20 Caci a?i fost cumpara?i cu pre?! Slavi?i, dar, pe Dumnezeu în trupul vostru ?i în duhul vostru, care sunt ale lui Dumnezeu.

\chapter{7}

\par 1 Cât despre cele ce mi-a?i scris, bine este pentru om sa nu se atinga de femeie.
\par 2 Dar din cauza desfrânarii, fiecare sa-?i aiba femeia sa ?i fiecare femeie sa-?i aiba barbatul sau.
\par 3 Barbatul sa-i dea femeii iubirea datorata, asemenea ?i femeia barbatului.
\par 4 Femeia nu este stapâna pe trupul sau, ci barbatul; asemenea nici barbatul nu este stapân pe trupul sau, ci femeia.
\par 5 Sa nu va lipsi?i unul de altul, decât cu buna învoiala pentru un timp, ca sa va îndeletnici?i cu postul ?i cu rugaciunea, ?i iara?i sa fi?i împreuna, ca sa nu va ispiteasca satana, din pricina neînfrânarii voastre.
\par 6 ?i aceasta o spun ca un sfat, nu ca o porunca.
\par 7 Eu voiesc ca to?i oamenii sa fie cum sunt eu însumi. Dar fiecare are de la Dumnezeu darul lui: unul a?a, altul într-alt fel.
\par 8 Celor ce sunt necasatori?i ?i vaduvelor le spun: Bine este pentru ei sa ramâna ca ?i mine.
\par 9 Daca însa nu pot sa se înfrâneze, sa se casatoreasca. Fiindca mai bine este sa se casatoreasca, decât sa arda.
\par 10 Iar celor ce sunt casatori?i, le poruncesc, nu eu, ci Domnul: Femeia sa nu se desparta de barbat!
\par 11 Iar daca s-a despar?it, sa ramâna nemaritata, sau sa se împace cu barbatul sau; tot a?a barbatul sa nu-?i lase femeia.
\par 12 Celorlal?i le graiesc eu, nu Domnul: Daca un frate are o femeie necredincioasa, ?i ea voie?te sa vie?uiasca cu el, sa nu o lase.
\par 13 ?i o femeie, daca are barbat necredincios, ?i el binevoie?te sa locuiasca cu ea, sa nu-?i lase barbatul.
\par 14 Caci barbatul necredincios se sfin?e?te prin femeia credincioasa ?i femeia necredincioasa se sfin?e?te prin barbatul credincios. Altminterea, copiii vo?tri ar fi necura?i, dar acum ei sunt sfin?i.
\par 15 Daca însa cel necredincios se desparte, sa se desparta. În astfel de împrejurare, fratele sau sora nu sunt lega?i; caci Dumnezeu ne-a chemat spre pace.
\par 16 Caci, ce ?tii tu, femeie, daca î?i vei mântui barbatul? Sau ce ?tii tu, barbate, daca î?i vei mântui femeia?
\par 17 Numai ca, a?a cum a dat Domnul fiecaruia, a?a cum l-a chemat Dumnezeu pe fiecare, astfel sa umble. ?i a?a rânduiesc în toate Bisericile.
\par 18 A fost cineva chemat, fiind taiat împrejur? Sa nu se ascunda. A fost cineva chemat în netaiere împrejur? Sa nu se taie împrejur.
\par 19 Taierea împrejur nu este nimic; ?i netaierea împrejur nu este nimic, ci paza poruncilor lui Dumnezeu.
\par 20 Fiecare, în chemarea în care a fost chemat, în aceasta sa ramâna.
\par 21 Ai fost chemat fiind rob? Fii fara grija. Iar de po?i sa fii liber, mai mult folose?te-te!
\par 22 Caci robul, care a fost chemat în Domnul, este un liberat al Domnului. Tot a?a cel chemat liber este rob al lui Hristos.
\par 23 Cu pre? a?i fost cumpara?i. Nu va face?i robi oamenilor.
\par 24 Fiecare, fra?ilor, în starea în care a fost chemat, în aceea sa ramâna înaintea lui Dumnezeu.
\par 25 Cât despre feciorie, n-am porunca de la Domnul. Va dau însa sfatul meu, ca unul care am fost miluit de Domnul sa fiu vrednic de crezare.
\par 26 Socotesc deci ca aceasta este bine pentru nevoia ceasului de fa?a: Bine este pentru om sa fie a?a.
\par 27 Te-ai legat de femeie? Nu cauta dezlegare. Te-ai dezlegat de femeie? Nu cauta femeie.
\par 28 Daca însa te vei însura, n-ai gre?it. Ci fecioara, de se va marita, n-a gre?it. Numai ca unii ca ace?tia vor avea suferin?a în trupul lor. Eu însa va cru? pe voi.
\par 29 ?i aceasta v-o spun, fra?ilor: Ca vremea s-a scurtat de acum, a?a încât ?i cei ce au femei sa fie ca ?i cum n-ar avea.
\par 30 ?i cei ce plâng sa fie ca ?i cum n-ar plânge; ?i cei ce se bucura, ca ?i cum nu s-ar bucura; ?i cei ce cumpara, ca ?i cum n-ar stapâni;
\par 31 ?i cei ce se folosesc de lumea aceasta, ca ?i cum nu s-ar folosi deplin de ea. Caci chipul acestei lumi trece.
\par 32 Dar eu vreau ca voi sa fi?i fara de  grija. Cel necasatorit se îngrije?te de cele ale Domnului, cum sa placa Domnului.
\par 33 Cel ce s-a casatorit se îngrije?te de cele ale lumii, cum sa placa femeii.
\par 34 ?i este împar?ire: ?i femeia nemaritata ?i fecioara poarta de grija de cele ale Domnului, ca sa fie sfânta ?i cu trupul ?i cu duhul. Iar cea care s-a maritat poarta de grija de cele ale lumii, cum sa placa barbatului.
\par 35 ?i aceasta o spun chiar în folosul vostru, nu ca sa va întind o cursa, ci spre bunul chip ?i alipirea de Domnul, fara clintire.
\par 36 Iar de socote?te cineva ca i se va face vreo necinste pentru fecioara sa, daca trece de floarea vârstei, ?i ca trebuie sa faca a?a, faca ce voie?te. Nu pacatuie?te; casatoreasca-se.
\par 37 Dar cel ce sta neclintit în inima sa ?i nu este silit, ci are stapânire peste voin?a sa ?i a hotarât aceasta în inima sa, ca sa-?i ?ina fecioara, bine va face.
\par 38 A?a ca, cel ce î?i marita fecioara bine face; dar cel ce n-o marita ?i mai bine face.
\par 39 Femeia este legata prin lege atâta vreme cât traie?te barbatul ei. Iar daca barbatul ei va muri, este libera sa se marite cu cine vrea, numai întru Domnul.
\par 40 Dar mai fericita este daca ramâne a?a, dupa parerea mea. ?i socot ca ?i eu am Duhul lui Dumnezeu.

\chapter{8}

\par 1 Cât despre cele jertfite idolilor, ?tim ca to?i avem cuno?tin?a. Cuno?tin?a însa seme?e?te, iar iubirea zide?te.
\par 2 Iar daca i se pare cuiva ca cunoa?te ceva, înca n-a cunoscut cum trebuie sa cunoasca.
\par 3 Dar daca iube?te cineva pe Dumnezeu, acela este cunoscut de El.
\par 4 Iar despre mâncarea celor jertfite idolilor, ?tim ca idolul nu este nimic în lume ?i ca nu este alt Dumnezeu decât Unul singur.
\par 5 Caci de?i sunt a?a-zi?i dumnezei, fie în cer, fie pe pamânt, - precum ?i sunt dumnezei mul?i ?i domni mul?i, S
\par 6 Totu?i, pentru noi, este un singur Dumnezeu, Tatal, din Care sunt toate ?i noi întru El; ?i un singur Domn, Iisus Hristos, prin Care sunt toate ?i noi prin El.
\par 7 Dar nu to?i au cuno?tin?a. Caci unii, din obi?nuin?a de pâna acum cu idolul, manânca din carnuri jertfite idolilor, ?i con?tiin?a lor fiind slaba, se întineaza.
\par 8 Dar nu mâncarea ne va pune înaintea lui Dumnezeu. Ca nici daca vom mânca, nu ne prisose?te, nici daca nu vom mânca, nu ne lipse?te.
\par 9 Dar vede?i ca nu cumva aceasta libertate a voastra sa ajunga poticnire pentru cei slabi.
\par 10 Caci daca cineva te-ar vedea pe tine, cel ce ai cuno?tin?a, ?ezând la masa în templul idolilor, oare con?tiin?a lui, slab fiind el, nu se va întari sa manânce din cele jertfite idolilor?
\par 11 ?i va pieri prin cuno?tin?a ta cel slab, fratele tau, pentru care a murit Hristos!
\par 12 ?i a?a, pacatuind împotriva fra?ilor ?i lovind con?tiin?a lor slaba, pacatui?i fa?a de Hristos.
\par 13 De aceea, daca o mâncare sminte?te pe fratele meu, nu voi mânca în veac carne, ca sa nu aduc sminteala fratelui meu.

\chapter{9}

\par 1 Oare nu sunt eu liber? Nu sunt eu apostol? N-am vazut eu pe Iisus Domnul nostru? Nu sunte?i voi lucrul meu întru Domnul?
\par 2 Daca altora nu le sunt apostol, voua, negre?it, va sunt. Caci voi sunte?i pecetea apostoliei mele în Domnul.
\par 3 Apararea mea catre cei ce ma judeca aceasta este.
\par 4 N-avem, oare, dreptul, sa mâncam ?i sa bem?
\par 5 N-avem, oare, dreptul sa purtam cu noi o femeie sora, ca ?i ceilal?i apostoli, ca ?i fra?ii Domnului, ca ?i Chefa?
\par 6 Sau numai eu ?i Barnaba nu avem dreptul de a nu lucra?
\par 7 Cine sluje?te vreodata, în oaste, cu solda lui? Cine sade?te vie ?i nu manânca din roada ei? Sau cine pa?te o turma ?i nu manânca din laptele turmei?
\par 8 Nu în felul oamenilor spun eu acestea. Nu spune, oare, ?i legea acestea?
\par 9 Caci în Legea lui Moise este scris: "Sa nu legi gura boului care treiera". Oare de boi se îngrije?te Dumnezeu?
\par 10 Sau în adevar pentru noi zice? Caci pentru noi s-a scris: "Cel ce ara trebuie sa are cu nadejde, ?i cel ce treiera, cu nadejdea ca va avea parte de roade".
\par 11 Daca noi am semanat la voi cele duhovnice?ti, este, oare, mare lucru daca noi vom secera cele pamânte?ti ale voastre?
\par 12 Daca al?ii se bucura de acest drept asupra voastra, oare, nu cu atât mai mult noi? Dar nu ne-am folosit de dreptul acesta, ci toate le rabdam, ca sa nu punem piedica Evangheliei lui Hristos.
\par 13 Au nu ?ti?i ca cei ce savâr?esc cele sfinte manânca de la templu ?i cei ce slujesc altarului au parte de la altar?
\par 14 Tot a?a a poruncit ?i Domnul celor ce propovaduiesc Evanghelia, ca sa traiasca din Evanghelie.
\par 15 Dar eu nu m-am folosit de nimic din acestea ?i nu am scris acestea, ca sa se faca cu mine a?a. Caci mai bine este pentru mine sa mor, decât sa-mi zadarniceasca cineva lauda.
\par 16 Caci daca vestesc Evanghelia, nu-mi este lauda, pentru ca sta asupra mea datoria. Caci, vai mie daca nu voi binevesti!
\par 17 Caci daca fac aceasta de buna voie, am plata; dar daca o fac fara voie, am numai o slujire încredin?ata.
\par 18 Care este, deci, plata mea? Ca, binevestind, pun fara plata Evanghelia lui Hristos înaintea oamenilor, fara sa ma folosesc de dreptul meu din Evanghelie.
\par 19 Caci, de?i sunt liber fa?a de to?i, m-am facut rob tuturor, ca sa dobândesc pe cei mai mul?i;
\par 20 Cu iudeii am fost ca un iudeu, ca sa dobândesc pe iudei; cu cei de sub lege, ca unul de sub lege, de?i eu nu sunt sub lege, ca sa dobândesc pe cei de sub lege;
\par 21 Cu cei ce n-au Legea, m-am facut ca unul fara lege, de?i nu sunt fara Legea lui Dumnezeu, ci având Legea lui Hristos, ca sa dobândesc pe cei ce n-au Legea;
\par 22 Cu cei slabi m-am facut slab, ca pe cei slabi sa-i dobândesc; tuturor toate m-am facut, ca, în orice chip, sa mântuiesc pe unii.
\par 23 Dar toate le fac pentru Evanghelie, ca sa fiu parta? la ea.
\par 24 Nu ?ti?i voi ca acei care alearga în stadion, to?i alearga, dar numai unul ia premiul? Alerga?i a?a ca sa-l lua?i.
\par 25 ?i oricine se lupta se înfrâneaza de la toate. ?i aceia, ca sa ia o cununa stricacioasa, iar noi, nestricacioasa.
\par 26 Eu, deci, a?a alerg, nu ca la întâmplare. A?a ma lupt, nu ca lovind în aer,
\par 27 Ci îmi chinuiesc trupul meu ?i îl supun robiei; ca nu cumva, altora propovaduind, eu însumi sa ma fac netrebnic.

\chapter{10}

\par 1 Caci nu voiesc, fra?ilor, ca voi sa nu ?ti?i ca parin?ii no?tri au fost to?i sub nor ?i ca to?i au trecut prin mare.
\par 2 ?i to?i, întru Moise, au fost boteza?i în nor ?i în mare.
\par 3 ?i to?i au mâncat aceea?i mâncare duhovniceasca;
\par 4 ?i to?i, aceea?i bautura duhovniceasca au baut, pentru ca beau din piatra duhovniceasca ce îi urma. Iar piatra era Hristos.
\par 5 Dar cei mai mul?i dintre ei nu au placut lui Dumnezeu, caci au cazut în pustie.
\par 6 ?i acestea s-au facut pilde pentru noi, ca sa nu poftim la cele rele, cum au poftit aceia;
\par 7 Nici închinatori la idoli sa nu va face?i, ca unii dintre ei, precum este scris: "A ?ezut poporul sa manânce ?i sa bea ?i s-au sculat la joc";
\par 8 Nici sa ne desfrânam cum s-au desfrânat unii dintre ei, ?i au cazut, într-o zi, douazeci ?i trei de mii;
\par 9 Nici sa ispitim pe Domnul, precum L-au ispitit unii dintre ei ?i au pierit de ?erpi;
\par 10 Nici sa cârti?i, precum au cârtit unii dintre ei ?i au fost nimici?i de catre pierzatorul.
\par 11 ?i toate acestea li s-au întâmplat acelora, ca preînchipuiri ale viitorului, ?i au fost scrise spre pova?uirea noastra, la care au ajuns sfâr?iturile veacurilor.
\par 12 De aceea, cel caruia i se pare ca sta neclintit sa ia seama sa nu cada.
\par 13 Nu v-a cuprins ispita care sa fi fost peste puterea omeneasca. Dar credincios este Dumnezeu; El nu va îngadui ca sa fi?i ispiti?i mai mult decât pute?i, ci odata cu ispita va aduce ?i scaparea din ea, ca sa pute?i rabda.
\par 14 De aceea, iubi?ii mei, fugi?i de închinarea la idoli.
\par 15 Ca unor în?elep?i va vorbesc. Judeca?i voi ce va spun.
\par 16 Paharul binecuvântarii, pe care-l binecuvântam, nu este, oare, împarta?irea cu sângele lui Hristos? Pâinea pe care o frângem nu este, oare, împarta?irea cu trupul lui Hristos?
\par 17 Ca o pâine, un trup, suntem cei mul?i; caci to?i ne împarta?im dintr-o pâine.
\par 18 Privi?i pe Israel dupa trup: Cei care manânca jertfele nu sunt ei, oare, parta?i altarului?
\par 19 Deci ce spun eu? Ca ce s-a jertfit pentru idol e ceva? Sau idolul este ceva?
\par 20 Ci (zic) ca cele ce jertfesc neamurile, jertfesc demonilor ?i nu lui Dumnezeu. ?i nu voiesc ca voi sa fi?i parta?i ai demonilor.
\par 21 Nu pute?i sa be?i paharul Domnului ?i paharul demonilor; nu pute?i sa va împarta?i?i din masa Domnului ?i din masa demonilor.
\par 22 Oare vrem sa mâniem pe Domnul? Nu cumva suntem mai tari decât El?
\par 23 Toate îmi sunt îngaduite, dar nu toate îmi folosesc. Toate îmi sunt îngaduite, dar nu toate zidesc.
\par 24 Nimeni sa nu caute pe ale sale, ci fiecare pe ale aproapelui.
\par 25 Mânca?i tot ce se vinde în macelarie, fara sa întreba?i nimic pentru cugetul vostru.
\par 26 Caci "al Domnului este pamântul ?i plinirea lui".
\par 27 Daca cineva dintre necredincio?i va cheama pe voi la masa ?i voi?i sa va duce?i, mânca?i orice va este pus înainte, fara sa întreba?i nimic pentru con?tiin?a.
\par 28 Dar de va va spune cineva: Aceasta este din jertfa idolilor, sa nu mânca?i pentru cel care v-a spus ?i pentru con?tiin?a.
\par 29 Iar con?tiin?a, zic, nu a ta însu?i, ci a altuia. Caci de ce libertatea mea sa fie judecata de o alta con?tiin?a?
\par 30 Daca eu sunt parta? harului, de ce sa fiu hulit pentru ceea ce aduc mul?umire?
\par 31 De aceea, ori de mânca?i, ori de be?i, ori altceva de face?i, toate spre slava lui Dumnezeu sa le face?i.
\par 32 Nu fi?i piatra de poticnire nici iudeilor, nici elinilor, nici Bisericii lui Dumnezeu,
\par 33 Precum ?i eu plac tuturor în toate, necautând folosul meu, ci pe al celor mul?i, ca sa se mântuiasca.

\chapter{11}

\par 1 Fi?i urmatori ai mei, precum ?i eu sunt al lui Hristos.
\par 2 Fra?ilor, va laud ca în toate va aduce?i aminte de mine ?i ?ine?i predaniile cum vi le-am dat.
\par 3 Dar voiesc ca voi sa ?ti?i ca Hristos este capul oricarui barbat, iar capul femeii este barbatul, iar capul lui Hristos: Dumnezeu.
\par 4 Orice barbat care se roaga sau prooroce?te, având capul acoperit, necinste?te capul sau.
\par 5 Iar orice femeie care se roaga sau prooroce?te, cu capul neacoperit, î?i necinste?te capul; caci tot una este ca ?i cum ar fi rasa.
\par 6 Caci daca o femeie nu-?i pune val pe cap, atunci sa se ?i tunda. Iar daca este lucru de ru?ine pentru femeie ca sa se tunda ori sa se rada, sa-?i puna val.
\par 7 Caci barbatul nu trebuie sa-?i acopere capul, fiind chip ?i slava a lui Dumnezeu, iar femeia este slava barbatului.
\par 8 Pentru ca nu barbatul este din femeie, ci femeia din barbat.
\par 9 ?i pentru ca n-a fost zidit barbatul pentru femeie, ci femeia pentru barbat.
\par 10 De aceea ?i femeia este datoare sa aiba (semn de) supunere asupra capului ei, pentru îngeri.
\par 11 Totu?i, nici femeia fara barbat, nici barbatul fara femeie, în Domnul.
\par 12 Caci precum femeia este din barbat, a?a ?i barbatul este prin femeie ?i toate sunt de la Dumnezeu.
\par 13 Judeca?i în voi în?iva: Este, oare, cuviincios ca o femeie sa se roage lui Dumnezeu cu capul descoperit?
\par 14 Nu va înva?a oare însa?i firea ca necinste este pentru un barbat sa-?i lase parul lung?
\par 15 ?i ca pentru o femeie, daca î?i lasa parul lung, este cinste? Caci parul i-a fost dat ca acoperamânt.
\par 16 Iar daca se pare cuiva ca aici poate sa ne gaseasca pricina, un astfel de obicei (ca femeile sa se roage cu capul descoperit) noi nu avem, nici Bisericile lui Dumnezeu.
\par 17 ?i aceasta poruncindu-va, nu va laud, fiindca voi va aduna?i nu spre mai bine, ci spre mai rau.
\par 18 Caci mai întâi aud ca atunci când va aduna?i în biserica, între voi sunt dezbinari, ?i în parte cred.
\par 19 Caci trebuie sa fie între voi ?i eresuri, ca sa se învedereze între voi cei încerca?i.
\par 20 Când va aduna?i deci laolalta, nu se poate mânca Cina Domnului;
\par 21 Caci, ?ezând la masa, fiecare se grabe?te sa ia mâncarea sa, încât unuia îi este foame, iar altul se îmbata.
\par 22 N-ave?i, oare, case ca sa mânca?i ?i sa be?i? Sau dispre?ui?i Biserica lui Dumnezeu ?i ru?ina?i pe cei ce nu au? Ce sa va zic? Sa va laud? În aceasta nu va laud.
\par 23 Caci eu de la Domnul am primit ceea ce v-am dat ?i voua: Ca Domnul Iisus, în noaptea în care a fost vândut, a luat pâine,
\par 24 ?i, mul?umind, a frânt ?i a zis: Lua?i, mânca?i; acesta este trupul Meu care se frânge pentru voi. Aceasta sa face?i spre pomenirea Mea.
\par 25 Asemenea ?i paharul dupa Cina, zicând: Acest pahar este Legea cea noua întru sângele Meu. Aceasta sa face?i ori de câte ori ve?i bea, spre pomenirea Mea.
\par 26 Caci de câte ori ve?i mânca aceasta pâine ?i ve?i bea acest pahar, moartea Domnului vesti?i pâna când va veni.
\par 27 Astfel, oricine va mânca pâinea aceasta sau va bea paharul Domnului cu nevrednicie, va fi vinovat fa?a de trupul ?i sângele Domnului.
\par 28 Sa se cerceteze însa omul pe sine ?i a?a sa manânce din pâine ?i sa bea din pahar.
\par 29 Caci cel ce manânca ?i bea cu nevrednicie, osânda î?i manânca ?i bea, nesocotind trupul Domnului.
\par 30 De aceea, mul?i dintre voi sunt neputincio?i ?i bolnavi ?i mul?i au murit.
\par 31 Caci de ne-am fi judecat noi în?ine, nu am mai fi judeca?i.
\par 32 Dar, fiind judeca?i de Domnul, suntem pedepsi?i, ca sa nu fim osândi?i împreuna cu lumea.
\par 33 De aceea, fra?ii mei, când va aduna?i ca sa mânca?i, a?tepta?i-va unii pe al?ii.
\par 34 Iar daca îi este cuiva foame, sa manânce acasa, ca sa nu va aduna?i spre osânda. Celelalte însa le voi rândui când voi veni.

\chapter{12}

\par 1 Iar cât prive?te darurile duhovnice?ti nu vreau, fra?ilor, sa fi?i în necuno?tin?a.
\par 2 ?ti?i ca, pe când era?i pagâni, va ducea?i la idolii cei mu?i, ca ?i cum era?i mâna?i.
\par 3 De aceea, va fac cunoscut ca precum nimeni, graind în Duhul lui Dumnezeu, nu zice: Anatema fie Iisus! - tot a?a nimeni nu poate sa zica: Domn este Iisus, - decât în Duhul Sfânt.
\par 4 Darurile sunt felurite, dar acela?i Duh.
\par 5 ?i felurite slujiri sunt, dar acela?i Domn.
\par 6 ?i lucrarile sunt felurite, dar este acela?i Dumnezeu, care lucreaza toate în to?i.
\par 7 ?i fiecaruia se da aratarea Duhului spre folos.
\par 8 Ca unuia i se da prin Duhul Sfânt cuvânt de în?elepciune, iar altuia, dupa acela?i Duh, cuvântul cuno?tin?ei.
\par 9 ?i unuia i se da întru acela?i Duh credin?a, iar altuia, darurile vindecarilor, întru acela?i Duh;
\par 10 Unuia faceri de minuni, iar altuia proorocie; unuia deosebirea duhurilor, iar altuia feluri de limbi ?i altuia talmacirea limbilor.
\par 11 ?i toate acestea le lucreaza unul ?i acela?i Duh, împar?ind fiecaruia deosebi, dupa cum voie?te.
\par 12 Caci precum trupul unul este, ?i are madulare multe, iar toate madularele trupului, multe fiind, sunt un trup, a?a ?i Hristos.
\par 13 Pentru ca într-un Duh ne-am botezat noi to?i, ca sa fim un singur trup, fie iudei, fie elini, fie robi, fie liberi, ?i to?i la un Duh ne-am adapat.
\par 14 Caci ?i trupul nu este un madular, ci multe.
\par 15 Daca piciorul ar zice: Fiindca nu sunt mâna nu sunt din trup, pentru aceasta nu este el din trup?
\par 16 ?i urechea daca ar zice: Fiindca nu sunt ochi, nu fac parte din trup, - pentru aceasta nu este ea din trup?
\par 17 Daca tot trupul ar fi ochi, unde ar fi auzul? ?i daca ar fi tot auz, unde ar fi mirosul?
\par 18 Dar acum Dumnezeu a pus madularele, pe fiecare din ele, în trup, cum a voit.
\par 19 Daca toate ar fi un singur madular, unde ar fi trupul?
\par 20 Dar acum sunt multe madulare, însa un singur trup.
\par 21 ?i nu poate ochiul sa zica mâinii: N-am trebuin?a de tine; sau, iara?i capul sa zica picioarelor: N-am trebuin?a de voi.
\par 22 Ci cu mult mai mult madularele trupului, care par a fi mai slabe, sunt mai trebuincioase.
\par 23 ?i pe cele ale trupului care ni se par ca sunt mai de necinste, pe acelea cu mai multa evlavie le îmbracam; ?i cele necuviincioase ale noastre au mai multa cuviin?a.
\par 24 Iar cele cuviincioase ale noastre n-au nevoie de acoperamânt. Dar Dumnezeu a întocmit astfel trupul, dând mai multa cinste celui caruia îi lipse?te,
\par 25 Ca sa nu fie dezbinare în trup, ci madularele sa se îngrijeasca deopotriva unele de altele.
\par 26 ?i daca un madular sufera, toate madularele sufera împreuna; ?i daca un madular este cinstit, toate madularele se bucura împreuna.
\par 27 Iar voi sunte?i trupul lui Hristos ?i madulare (fiecare) în parte.
\par 28 ?i pe unii i-a pus Dumnezeu, în Biserica: întâi apostoli, al doilea prooroci, al treilea înva?atori; apoi pe cei ce au darul de a face minuni; apoi darurile vindecarilor, ajutorarile, cârmuirile, felurile limbilor.
\par 29 Oare to?i sunt apostoli? Oare to?i sunt prooroci? Oare to?i înva?atori? Oare to?i au putere sa savâr?easca minuni?
\par 30 Oare to?i au darurile vindecarilor? Oare to?i vorbesc în limbi? Oare to?i pot sa talmaceasca?
\par 31 Râvni?i însa la darurile cele mai bune. ?i va arat înca o cale care le întrece pe toate:

\chapter{13}

\par 1 De a? grai în limbile oamenilor ?i ale îngerilor, iar dragoste nu am, facutu-m-am arama sunatoare ?i chimval rasunator.
\par 2 ?i de a? avea darul proorociei ?i tainele toate le-a? cunoa?te ?i orice ?tiin?a, ?i de a? avea atâta credin?a încât sa mut ?i mun?ii, iar dragoste nu am, nimic nu sunt.
\par 3 ?i de a? împar?i toata avu?ia mea ?i de a? da trupul meu ca sa fie ars, iar dragoste nu am, nimic nu-mi folose?te.
\par 4 Dragostea îndelung rabda; dragostea este binevoitoare, dragostea nu pizmuie?te, nu se lauda, nu se trufe?te.
\par 5 Dragostea nu se poarta cu necuviin?a, nu cauta ale sale, nu se aprinde de mânie, nu gânde?te raul.
\par 6 Nu se bucura de nedreptate, ci se bucura de adevar.
\par 7 Toate le sufera, toate le crede, toate le nadajduie?te, toate le rabda.
\par 8 Dragostea nu cade niciodata. Cât despre proorocii - se vor desfiin?a; darul limbilor va înceta; ?tiin?a se va sfâr?i;
\par 9 Pentru ca în parte cunoa?tem ?i în parte proorocim.
\par 10 Dar când va veni ceea ce e desavâr?it, atunci ceea ce este în parte se va desfiin?a.
\par 11 Când eram copil, vorbeam ca un copil, sim?eam ca un copil; judecam ca un copil; dar când m-am facut barbat, am lepadat cele ale copilului.
\par 12 Caci vedem acum ca prin oglinda, în ghicitura, iar atunci, fa?a catre fa?a; acum cunosc în parte, dar atunci voi cunoa?te pe deplin, precum am fost cunoscut ?i eu.
\par 13 ?i acum ramân acestea trei: credin?a, nadejdea ?i dragostea. Iar mai mare dintre acestea este dragostea.

\chapter{14}

\par 1 Cauta?i dragostea. Râvni?i însa cele duhovnice?ti, dar mai ales ca sa prooroci?i.
\par 2 Pentru ca cel ce vorbe?te într-o limba straina nu vorbe?te oamenilor, ci lui Dumnezeu; ?i nimeni nu-l în?elege, fiindca el, în duh, graie?te taine.
\par 3 Cel ce prooroce?te vorbe?te oamenilor, spre zidire, îndemn ?i mângâiere.
\par 4 Cel ce graie?te într-o limba straina pe sine singur se zide?te, iar cel ce prooroce?te zide?te Biserica.
\par 5 Voiesc ca voi to?i sa grai?i în limbi; dar mai cu seama sa prooroci?i. Cel ce prooroce?te e mai mare decât cel ce graie?te în limbi, afara numai daca talmace?te, ca Biserica sa ia întarire.
\par 6 Iar acum, fra?ilor, daca a? veni la voi, graind în limbi, de ce folos v-a? fi, daca nu v-a? vorbi - sau în descoperire, sau în cuno?tin?a, sau în proorocie, sau în înva?atura?
\par 7 Ca precum cele neînsufle?ite, care dau sunet, fie fluier, fie chitara, de nu vor da sunete deosebite, cum se va cunoa?te ce este din fluier, sau ce este din chitara?
\par 8 ?i daca trâmbi?a va da sunet nelamurit, cine se va pregati de razboi?
\par 9 A?a ?i voi: Daca prin limba nu ve?i da cuvânt lesne de în?eles, cum se va cunoa?te ce a?i grait? Ve?i fi ni?te oameni care vorbesc în vânt.
\par 10 Sunt a?a de multe feluri de limbi în lume, dar nici una din ele nu este fara în?elesul ei.
\par 11 Deci daca nu voi ?ti în?elesul cuvintelor, voi fi barbar pentru cel care vorbe?te, ?i cel care vorbe?te barbar pentru mine.
\par 12 A?a ?i voi, de vreme ce sunte?i râvnitori dupa cele duhovnice?ti, cauta?i sa prisosi?i în ele, spre zidirea Bisericii.
\par 13 De aceea, cel ce graie?te într-o limba straina sa se roage ca sa ?i talmaceasca.
\par 14 Caci, daca ma rog într-o limba straina, duhul meu se roaga, dar mintea mea este neroditoare.
\par 15 Atunci ce voi face? Ma voi ruga cu duhul, dar ma voi ruga ?i cu mintea; voi cânta cu duhul, dar voi cânta ?i cu mintea.
\par 16 Fiindca daca vei binecuvânta cu duhul, cum va raspunde omul simplu "Amin" la mul?umirea ta, de vreme ce el nu ?tie ce zici?
\par 17 Caci tu, într-adevar, mul?ume?ti bine, dar celalalt nu se zide?te.
\par 18 Mul?umesc Dumnezeului meu, ca vorbesc în limbi mai mult decât voi to?i;
\par 19 Dar în Biserica vreau sa graiesc cinci cuvinte cu mintea mea, ca sa înva? ?i pe al?ii, decât zeci de mii de cuvinte într-o limba straina.
\par 20 Fra?ilor, nu fi?i copii la minte. Fi?i copii când e vorba de rautate. La minte însa, fi?i desavâr?i?i.
\par 21 În Lege este scris: "Voi grai acestui popor în alte limbi ?i prin buzele altora, ?i nici a?a nu vor asculta de Mine, zice Domnul".
\par 22 A?a ca vorbirea în limbi este semn nu pentru cei credincio?i ci pentru cei necredincio?i; iar proorocia nu pentru cei necredincio?i, ci pentru cei ce cred.
\par 23 Deci, daca s-ar aduna Biserica toata laolalta ?i to?i ar vorbi în limbi ?i ar intra ne?tiutori sau necredincio?i, nu vor zice, oare, ca sunte?i nebuni?
\par 24 Iar daca to?i ar prooroci ?i ar intra vreun necredincios sau vreun ne?tiutor, el este dovedit de to?i, el este judecat de to?i;
\par 25 Cele ascunse ale inimii lui se învedereaza, ?i astfel, cazând cu fa?a la pamânt, se va închina lui Dumnezeu, marturisind ca Dumnezeu este într-adevar printre voi.
\par 26 Ce este deci, fra?ilor? Când va aduna?i împreuna, fiecare din voi are psalm, are înva?atura, are descoperire, are limba, are talmacire: toate spre zidire sa se faca.
\par 27 Daca graie?te cineva într-o limba straina, sa fie câte doi, sau cel mult trei ?i pe rând sa graiasca ?i unul sa talmaceasca.
\par 28 Iar daca nu e talmacitor, sa taca în biserica ?i sa-?i graiasca numai lui ?i lui Dumnezeu.
\par 29 Iar proorocii sa vorbeasca doi sau trei, iar ceilal?i sa judece.
\par 30 Iar daca se va descoperi ceva altuia care ?ade, sa taca cei dintâi.
\par 31 Caci pute?i sa prooroci?i to?i câte unul, ca to?i sa înve?e ?i to?i sa se mângâie.
\par 32 ?i duhurile proorocilor se supun proorocilor.
\par 33 Pentru ca Dumnezeu nu este al neorânduielii, ci al pacii.
\par 34 Ca în toate Bisericile sfin?ilor, femeile voastre sa taca în biserica, caci lor nu le este îngaduit sa vorbeasca, ci sa se supuna, precum zice ?i Legea.
\par 35 Iar daca voiesc sa înve?e ceva, sa întrebe acasa pe barba?ii lor, caci este ru?inos ca femeile sa vorbeasca în biserica.
\par 36 Oare de la voi a ie?it cuvântul lui Dumnezeu sau a ajuns numai la voi?
\par 37 Daca i se pare cuiva ca este prooroc sau om duhovnicesc, sa cunoasca ca cele ce va scriu sunt porunci ale Domnului.
\par 38 Iar daca cineva nu vrea sa ?tie, sa nu ?tie.
\par 39 A?a ca, fra?ii mei, râvni?i a prooroci ?i nu opri?i sa se graiasca în limbi.
\par 40 Dar toate sa se faca cu cuviin?a ?i dupa rânduiala.

\chapter{15}

\par 1 Va aduc aminte, fra?ilor, Evanghelia pe care v-am binevestit-o, pe care a?i ?i primit-o, întru care ?i sta?i,
\par 2 Prin care ?i sunte?i mântui?i; cu ce cuvânt v-am binevestit-o - daca o ?ine?i cu tarie, afara numai daca n-a?i crezut în zadar S
\par 3 Caci v-am dat, întâi de toate, ceea ce ?i eu am primit, ca Hristos a murit pentru pacatele noastre dupa Scripturi;
\par 4 ?i ca a fost îngropat ?i ca a înviat a treia zi, dupa Scripturi;
\par 5 ?i ca S-a aratat lui Chefa, apoi celor doisprezece;
\par 6 În urma S-a aratat deodata la peste cinci sute de fra?i, dintre care cei mai mul?i traiesc pâna astazi, iar unii au ?i adormit;
\par 7 Dupa aceea S-a aratat lui Iacov, apoi tuturor apostolilor;
\par 8 Iar la urma tuturor, ca unui nascut înainte de vreme, mi S-a aratat ?i mie.
\par 9 Caci eu sunt cel mai mic dintre apostoli, care nu sunt vrednic sa ma numesc apostol, pentru ca am prigonit Biserica lui Dumnezeu.
\par 10 Dar prin harul lui Dumnezeu sunt ceea ce sunt; ?i harul Lui care este în mine n-a fost în zadar, ci m-am ostenit mai mult decât ei to?i. Dar nu eu, ci harul lui Dumnezeu care este cu mine.
\par 11 Deci ori eu, ori aceia, a?a propovaduim ?i voi a?a a?i crezut.
\par 12 Iar daca se propovaduie?te ca Hristos a înviat din mor?i, cum zic unii dintre voi ca nu este înviere a mor?ilor?
\par 13 Daca nu este înviere a mor?ilor, nici Hristos n-a înviat.
\par 14 ?i daca Hristos n-a înviat, zadarnica este atunci propovaduirea noastra, zadarnica este ?i credin?a voastra.
\par 15 Ne aflam înca ?i martori mincino?i ai lui Dumnezeu, pentru ca am marturisit împotriva lui Dumnezeu ca a înviat pe Hristos, pe Care nu L-a înviat, daca deci mor?ii nu înviaza.
\par 16 Caci daca mor?ii nu înviaza, nici Hristos n-a înviat.
\par 17 Iar daca Hristos n-a înviat, zadarnica este credin?a voastra, sunte?i înca în pacatele voastre;
\par 18 ?i atunci ?i cei ce au adormit în Hristos au pierit.
\par 19 Iar daca nadajduim în Hristos numai în via?a aceasta, suntem mai de plâns decât to?i oamenii.
\par 20 Dar acum Hristos a înviat din mor?i, fiind începatura (a învierii) celor adormi?i.
\par 21 Ca de vreme ce printr-un om a venit moartea, tot printr-un om ?i învierea mor?ilor.
\par 22 Caci, precum în Adam to?i mor, a?a ?i în Hristos to?i vor învia.
\par 23 Dar fiecare în rândul cetei sale: Hristos începatura, apoi cei ai lui Hristos, la venirea Lui,
\par 24 Dupa aceea, sfâr?itul, când Domnul va preda împara?ia lui Dumnezeu ?i Tatalui, când va desfiin?a orice domnie ?i orice stapânire ?i orice putere.
\par 25 Caci El trebuie sa împara?easca pâna ce va pune pe to?i vrajma?ii Sai sub picioarele Sale.
\par 26 Vrajma?ul cel din urma, care va fi nimicit, este moartea.
\par 27 "Caci toate le-a supus sub picioarele Lui". Dar când zice: "Ca toate I-au fost supuse Lui" - învederat este ca afara de Cel care I-a supus Lui toate.
\par 28 Iar când toate vor fi supuse Lui, atunci ?i Fiul însu?i Se va supune Celui ce I-a supus Lui toate, ca Dumnezeu sa fie toate în to?i.
\par 29 Fiindca ce vor face cei care se boteaza pentru mor?i? Daca mor?ii nu înviaza nicidecum, pentru ce se mai boteaza pentru ei?
\par 30 De ce mai suntem ?i noi în primejdie în tot ceasul?
\par 31 Mor în fiecare zi! V-o spun, fra?ilor, pe lauda pe care o am pentru voi, în Hristos Iisus, Domnul nostru.
\par 32 Daca m-am luptat, ca om, cu fiarele în Efes, care îmi este folosul? Daca mor?ii nu înviaza, sa bem ?i sa mâncam, caci mâine vom muri!
\par 33 Nu va lasa?i în?ela?i. Tovara?iile rele strica obiceiurile bune.
\par 34 Trezi?i-va cum se cuvine ?i nu pacatui?i. Caci unii nu au cuno?tin?a de Dumnezeu; o spun spre ru?inea voastra.
\par 35 Dar va zice cineva: Cum înviaza mor?ii? ?i cu ce trup au sa vina?
\par 36 Nebun ce e?ti! Tu ce semeni nu da via?a, daca nu va fi murit.
\par 37 ?i ceea ce semeni nu este trupul ce va sa fie, ci graunte gol, poate de grâu, sau de altceva din celelalte;
\par 38 Iar Dumnezeu îi da un trup, precum a voit, ?i fiecarei semin?e un trup al sau.
\par 39 Nu toate trupurile sunt acela?i trup, ci unul este trupul oamenilor ?i altul este trupul dobitoacelor ?i altul este trupul pasarilor ?i altul este trupul pe?tilor.
\par 40 Sunt ?i trupuri cere?ti ?i trupuri pamânte?ti; dar alta este slava celor cere?ti ?i alta a celor pamânte?ti.
\par 41 Alta este stralucirea soarelui ?i alta stralucirea lunii ?i alta stralucirea stelelor. Caci stea de stea se deosebe?te în stralucire.
\par 42 A?a este ?i învierea mor?ilor: Se seamana (trupul) întru stricaciune, înviaza întru nestricaciune;
\par 43 Se seamana întru necinste, înviaza întru slava, se seamana întru slabiciune, înviaza întru putere;
\par 44 Se seamana trup firesc, înviaza trup duhovnicesc. Daca este trup firesc, este ?i trup duhovnicesc.
\par 45 Precum ?i este scris: "Facutu-s-a omul cel dintâi, Adam, cu suflet viu; iar Adam cel de pe urma cu duh datator de via?a";
\par 46 Dar nu este întâi cel duhovnicesc, ci cel firesc, apoi cel duhovnicesc.
\par 47 Omul cel dintâi este din pamânt, pamântesc; omul cel de-al doilea este din cer.
\par 48 Cum este cel pamântesc, a?a sunt ?i cei pamânte?ti; ?i cum este cel ceresc, a?a sunt ?i cei cere?ti.
\par 49 ?i dupa cum am purtat chipul celui pamântesc, sa purtam ?i chipul celui ceresc.
\par 50 Aceasta însa zic, fra?ilor: Carnea ?i sângele nu pot sa mo?teneasca împara?ia lui Dumnezeu, nici stricaciunea nu mo?tene?te nestricaciunea.
\par 51 Iata, taina va spun voua: Nu to?i vom muri, dar to?i ne vom schimba,
\par 52 Deodata, într-o clipeala de ochi la trâmbi?a cea de apoi. Caci trâmbi?a va suna ?i mor?ii vor învia nestricacio?i, iar noi ne vom schimba.
\par 53 Caci trebuie ca acest trup stricacios sa se îmbrace în nestricaciune ?i acest (trup) muritor sa se îmbrace în nemurire.
\par 54 Iar când acest (trup) stricacios se va îmbraca în nestricaciune ?i acest (trup) muritor se va îmbraca în nemurire, atunci va fi cuvântul care este scris: "Moartea a fost înghi?ita de biruin?a.
\par 55 Unde î?i este, moarte, biruin?a ta? Unde î?i este, moarte, boldul tau?".
\par 56 ?i boldul mor?ii este pacatul, iar puterea pacatului este legea.
\par 57 Dar sa dam mul?umire lui Dumnezeu, Care ne-a dat biruin?a prin Domnul nostru Iisus Hristos!
\par 58 Drept aceea, fra?ii mei iubi?i, fi?i tari, neclinti?i, sporind totdeauna în lucrul Domnului, ?tiind ca osteneala voastra nu este zadarnica în Domnul.

\chapter{16}

\par 1 Cât despre strângerea de ajutoare pentru sfin?i, precum am rânduit pentru Bisericile Galatiei, a?a sa face?i ?i voi.
\par 2 În ziua întâi a saptamânii (Duminica), fiecare dintre voi sa-?i puna deoparte, strângând cât poate, ca sa nu se faca strângerea abia atunci când voi veni.
\par 3 Iar când voi veni, pe cei pe care îi ve?i socoti, pe aceia îi voi trimite cu scrisori sa duca darul vostru la Ierusalim.
\par 4 ?i de se va cuveni sa merg ?i eu, vor merge împreuna cu mine.
\par 5 Ci voi veni la voi, când voi trece prin Macedonia, caci prin Macedonia trec.
\par 6 La voi ma voi opri, poate, sau voi ?i ierna, ca sa ma petrece?i în calatoria ce voi face.
\par 7 Caci nu vreau sa va vad acum numai în treacat, ci nadajduiesc sa ramân la voi câtava vreme, daca va îngadui Domnul.
\par 8 Voi ramâne însa în Efes, pâna la praznicul Cincizecimii.
\par 9 Caci mi s-a deschis u?a mare spre lucru mult, dar sunt mul?i potrivnici.
\par 10 Iar de va veni Timotei, vede?i sa fie fara teama la voi, caci lucreaza ca ?i mine lucrul Domnului.
\par 11 Nimeni deci sa nu-l dispre?uiasca; ci sa-l petrece?i cu pace, ca sa vina la mine; ca îl a?tept cu fra?ii.
\par 12 Cât despre fratele Apollo, l-am rugat mult sa vina la voi cu fra?ii; totu?i nu i-a fost voia sa vina acum. Ci va veni când va gasi prilej.
\par 13 Priveghea?i, sta?i tari în credin?a, îmbarbata?i-va, întari?i-va.
\par 14 Toate ale voastre cu dragoste sa se faca.
\par 15 Va îndemn însa, fra?ilor, - ?ti?i casa lui ?tefanas, ca este pârga Ahaei ?i ca spre slujirea sfin?ilor s-au rânduit pe ei în?i?i S
\par 16 Ca ?i voi sa va supune?i unora ca ace?tia ?i oricui lucreaza ?i se ostene?te împreuna cu ei.
\par 17 Ma bucur de venirea lui ?tefanas, a lui Fortunat ?i a lui Ahaic, pentru ca ace?tia au împlinit lipsa voastra.
\par 18 ?i au lini?tit duhul meu ?i al vostru. Cunoa?te?i bine deci pe unii ca ace?tia.
\par 19 Va îmbra?i?eaza Bisericile Asiei. Va îmbra?i?eaza mult, în Domnul, Acvila ?i Priscila, împreuna cu Biserica din casa lor.
\par 20 Va îmbra?i?eaza fra?ii to?i. Îmbra?i?a?i-va unii pe al?ii cu sarutare sfânta.
\par 21 Salutarea cu mâna mea, Pavel.
\par 22 Cel ce nu iube?te pe Domnul sa fie anatema! Maran atha! (Domnul vine).
\par 23 Harul Domnului Iisus Hristos cu voi.
\par 24 Dragostea mea cu voi to?i, în Hristos Iisus! Amin.


\end{document}