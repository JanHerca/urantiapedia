\begin{document}

\title{1 Corinteni}


\chapter{1}

\par 1 Pavel, chemat apostol al lui Hristos, prin voia lui Dumnezeu, și fratele Sostene,
\par 2 Bisericii lui Dumnezeu care este în Corint, celor sfințiți în Iisus Hristos, celor numiți sfinți, împreună cu toți cei ce cheamă numele Domnului nostru Iisus Hristos în tot locul, și al lor și al nostru:
\par 3 Har vouă și pace de la Dumnezeu, Tatăl nostru, și de la Domnul nostru Iisus Hristos.
\par 4 Mulțumesc totdeauna Dumnezeului meu pentru voi, pentru harul lui Dumnezeu, dat vouă în Hristos Iisus.
\par 5 Căci întru El v-ați îmbogățit deplin întru toate, în tot cuvântul și în toată cunoștința;
\par 6 Astfel mărturia lui Hristos s-a întărit în voi,
\par 7 Încât voi nu sunteți lipsiți de nici un dar, așteptând arătarea Domnului nostru Iisus Hristos,
\par 8 Care vă va și întări până la sfârșit, ca să fiți nevinovați în ziua Domnului nostru Iisus Hristos.
\par 9 Credincios este Dumnezeu, prin Care ați fost chemați la împărtășirea cu Fiul Său Iisus Hristos, Domnul nostru.
\par 10 Vă îndemn, fraților, pentru numele Domnului nostru Iisus Hristos, ca toți să vorbiți la fel și să nu fie dezbinări între voi; ci să fiți cu totul uniți în același cuget și în aceeași înțelegere.
\par 11 Căci, frații mei, despre voi, prin cei din casa lui Hloe mi-a venit știre că la voi sunt certuri;
\par 12 Și spun aceasta, că fiecare dintre voi zice: Eu sunt al lui Pavel, iar eu sunt al lui Apollo, iar eu sunt al lui Chefa, iar eu sunt al lui Hristos!
\par 13 Oare s-a împărțit Hristos? Nu cumva s-a răstignit Pavel pentru voi? Sau fost-ați botezați în numele lui Pavel?
\par 14 Mulțumesc lui Dumnezeu că pe nici unul din voi n-am botezat, decât pe Crispus și pe Gaius,
\par 15 Ca să nu zică cineva că ați fost botezați în numele meu.
\par 16 Am botezat și casa lui Ștefana; afară de aceștia nu știu să mai fi botezat pe altcineva.
\par 17 Căci Hristos nu m-a trimis ca să botez, ci să binevestesc, dar nu cu înțelepciunea cuvântului, ca să nu rămână zadarnică crucea lui Hristos.
\par 18 Căci cuvântul Crucii, pentru cei ce pier, este nebunie; iar pentru noi, cei ce ne mântuim, este puterea lui Dumnezeu.
\par 19 Căci scris este: "Pierde-voi înțelepciunea înțelepților și știința celor învățați voi nimici-o".
\par 20 Unde este înțeleptul? Unde e cărturarul? Unde e cercetătorul acestui veac? Au n-a dovedit Dumnezeu nebună înțelepciunea lumii acesteia?
\par 21 Căci de vreme ce întru înțelepciunea lui Dumnezeu lumea n-a cunoscut prin înțelepciune pe Dumnezeu, a binevoit Dumnezeu să mântuiască pe cei ce cred prin nebunia propovăduirii.
\par 22 Fiindcă și iudeii cer semne, iar elinii caută înțelepciune,
\par 23 Însă noi propovăduim pe Hristos cel răstignit: pentru iudei, sminteală; pentru neamuri, nebunie.
\par 24 Dar pentru cei chemați, și iudei și elini: pe Hristos, puterea lui Dumnezeu și înțelepciunea lui Dumnezeu.
\par 25 Pentru că fapta lui Dumnezeu, socotită de către oameni nebunie, este mai înțeleaptă decât înțelepciunea lor și ceea ce se pare ca slăbiciune a lui Dumnezeu, mai puternică decât tăria oamenilor.
\par 26 Căci, priviți chemarea voastră, fraților, că nu mulți sunt înțelepți după trup, nu mulți sunt puternici, nu mulți sunt de bun neam;
\par 27 Ci Dumnezeu Și-a ales pe cele nebune ale lumii, ca să rușineze pe cei înțelepți; Dumnezeu Și-a ales pe cele slabe ale lumii, ca să le rușineze pe cele tari;
\par 28 Dumnezeu Și-a ales pe cele de neam jos ale lumii, pe cele nebăgate în seamă, pe cele ce nu sunt, ca să nimicească pe cele ce sunt,
\par 29 Ca nici un trup să nu se laude înaintea lui Dumnezeu.
\par 30 Din El, dar, sunteți voi în Hristos Iisus, Care pentru noi S-a făcut înțelepciune de la Dumnezeu și dreptate și sfințire și răscumpărare,
\par 31 Pentru ca, după cum este scris: "Cel ce se laudă în Domnul să se laude".

\chapter{2}

\par 1 Și eu, fraților, când am venit la voi și v-am vestit taina lui Dumnezeu, n-am venit ca iscusit cuvântător sau ca înțelept.
\par 2 Căci am judecat să nu știu între voi altceva, decât pe Iisus Hristos, și pe Acesta răstignit.
\par 3 Și eu întru slăbiciune și cu frică și cu cutremur mare am fost la voi.
\par 4 Iar cuvântul meu și propovăduirea mea nu stăteau în cuvinte de înduplecare ale înțelepciunii omenești, ci în adeverirea Duhului și a puterii,
\par 5 Pentru ca credința voastră să nu fie în înțelepciunea oamenilor, ci în puterea lui Dumnezeu.
\par 6 Și înțelepciunea o propovăduim la cei desăvârșiți, dar nu înțelepciunea acestui veac, nici a stăpânitorilor acestui veac, care sunt pieritori,
\par 7 Ci propovăduim înțelepciunea de taină a lui Dumnezeu, ascunsă, pe care Dumnezeu a rânduit-o mai înainte de veci, spre slava noastră,
\par 8 Pe care nici unul dintre stăpânitorii acestui veac n-a cunoscut-o, căci, dacă ar fi cunoscut-o, n-ar fi răstignit pe Domnul slavei;
\par 9 Ci precum este scris: "Cele ce ochiul n-a văzut și urechea n-a auzit, și la inima omului nu s-au suit, pe acestea le-a gătit Dumnezeu celor ce-L iubesc pe El".
\par 10 Iar nouă ni le-a descoperit Dumnezeu prin Duhul Său, fiindcă Duhul toate le cercetează, chiar și adâncurile lui Dumnezeu.
\par 11 Căci cine dintre oameni știe ale omului, decât duhul omului, care este în el? Așa și cele ale lui Dumnezeu, nimeni nu le-a cunoscut, decât Duhul lui Dumnezeu.
\par 12 Iar noi n-am primit duhul lumii, ci Duhul cel de la Dumnezeu, ca să cunoaștem cele dăruite nouă de Dumnezeu;
\par 13 Pe care le și grăim, dar nu în cuvinte învățate din înțelepciunea omenească, ci în cuvinte învățate de la Duhul Sfânt, lămurind lucruri duhovnicești oamenilor duhovnicești.
\par 14 Omul firesc nu primește cele ale Duhului lui Dumnezeu, căci pentru el sunt nebunie și nu poate să le înțeleagă, fiindcă ele se judecă duhovnicește.
\par 15 Dar omul duhovnicesc toate le judecă, pe el însă nu-l judecă nimeni;
\par 16 Căci "Cine a cunoscut gândul Domnului, ca să-L învețe pe El?" Noi însă avem gândul lui Hristos.

\chapter{3}

\par 1 Și eu, fraților, n-am putut să vă vorbesc ca unor oameni duhovnicești, ci ca unora trupești, ca unor prunci în Hristos.
\par 2 Cu lapte v-am hrănit, nu cu bucate, căci încă nu puteați mânca și încă nici acum nu puteți,
\par 3 Fiindcă sunteți tot trupești. Câtă vreme este între voi pizmă și ceartă și dezbinări, nu sunteți, oare, trupești și nu după firea omenească umblați?
\par 4 Căci, când zice unul: Eu sunt al lui Pavel, iar altul: Eu sunt al lui Apollo, au nu sunteți oameni trupești?
\par 5 Dar ce este Apollo? Și ce este Pavel? Slujitori prin care ați crezut voi și după cum i-a dat Domnul fiecăruia.
\par 6 Eu am sădit, Apollo a udat, dar Dumnezeu a făcut să crească.
\par 7 Astfel nici cel ce sădește nu e ceva, nici cel ce udă, ci numai Dumnezeu care face să crească.
\par 8 Cel care sădește și cel care udă sunt una și fiecare își va lua plata după osteneala sa.
\par 9 Căci noi împreună-lucrători cu Dumnezeu suntem; voi sunteți ogorul lui Dumnezeu, zidirea lui Dumnezeu.
\par 10 După harul lui Dumnezeu, cel dat mie, eu, ca un înțelept meșter, am pus temelia; iar altul zidește. Dar fiecare să ia seama cum zidește;
\par 11 Căci nimeni nu poate pune altă temelie, decât cea pusă, care este Iisus Hristos.
\par 12 Iar de zidește cineva pe această temelie: aur, argint, sau pietre scumpe, lemne, fân, trestie.
\par 13 Lucrul fiecăruia se va face cunoscut; îl va vădi ziua (Domnului). Pentru că în foc se descoperă, și focul însuși va lămuri ce fel este lucrul fiecăruia.
\par 14 Dacă lucrul cuiva, pe care l-a zidit, va rămâne, va lua plată.
\par 15 Dacă lucrul cuiva se va arde, el va fi păgubit; el însă se va mântui, dar așa ca prin foc.
\par 16 Nu știți, oare, că voi sunteți templu al lui Dumnezeu și că Duhul lui Dumnezeu locuiește în voi?
\par 17 De va strica cineva templul lui Dumnezeu, îl va strica Dumnezeu pe el, pentru că sfânt este templul lui Dumnezeu, care sunteți voi.
\par 18 Nimeni să nu se amăgească. Dacă i se pare cuiva, între voi, că este înțelept în veacul acesta, să se facă nebun, ca să fie înțelept.
\par 19 Căci înțelepciunea lumii acesteia este nebunie înaintea lui Dumnezeu, pentru că scris este: "El prinde pe cei înțelepți în viclenia lor".
\par 20 Și iarăși: "Domnul cunoaște gândurile înțelepților, că sunt deșarte".
\par 21 Așa că nimeni să nu se laude cu oameni. Căci toate sunt ale voastre:
\par 22 Fie Pavel, fie Apollo, fie Chefa, fie lumea, fie viața, fie moartea, fie cele de față, fie cele viitoare, toate sunt ale voastre.
\par 23 Iar voi sunteți ai lui Hristos, iar Hristos al lui Dumnezeu.

\chapter{4}

\par 1 Așa să ne socotească pe noi fiecare om: ca slujitori ai lui Hristos și ca iconomi ai tainelor lui Dumnezeu.
\par 2 Iar, la iconomi, mai ales, se cere ca fiecare să fie aflat credincios.
\par 3 Dar mie prea puțin îmi este că sunt judecat de voi sau de vreo omenească judecată de toată ziua; fiindcă nici eu nu mă judec pe mine însumi.
\par 4 Căci nu mă știu vinovat cu nimic, dar nu întru aceasta m-am îndreptat. Cel care mă judecă pe mine este Domnul.
\par 5 De aceea, nu judecați ceva înainte de vreme, până ce nu va veni Domnul, Care va lumina cele ascunse ale întunericului și va vădi sfaturile inimilor. Și atunci fiecare va avea de la Dumnezeu lauda.
\par 6 Și acestea, fraților, le-am zis ca despre mine și despre Apollo, dar ele sunt pentru voi, ca să învățați din pilda noastră, să nu treceți peste ce e scris, ca să nu vă făliți unul cu altul împotriva celuilalt.
\par 7 Căci cine te deosebește pe tine? Și ce ai, pe care să nu-l fi primit? Iar dacă l-ai primit, de ce te fălești, ca și cum nu l-ai fi primit?
\par 8 Iată, sunteți sătui; iată, v-ați îmbogățit; fără de noi ați domnit, și, măcar nu ați domnit, ca și noi să domnim împreună cu voi.
\par 9 Căci mi se pare că Dumnezeu, pe noi, apostolii, ne-a arătat ca pe cei din urmă oameni, ca pe niște osândiți la moarte, fiindcă ne-am făcut priveliște lumii și îngerilor și oamenilor.
\par 10 Noi suntem nebuni pentru Hristos; voi însă înțelepți întru Hristos. Noi suntem slabi; voi însă sunteți tari. Voi sunteți întru slavă, iar noi suntem întru necinste!
\par 11 Până în ceasul de acum flămânzim și însetăm; suntem goi și suntem pălmuiți și pribegim,
\par 12 Și ne ostenim, lucrând cu mâinile noastre. Ocărâți fiind, binecuvântăm. Prigoniți fiind, răbdăm.
\par 13 Huliți fiind, ne rugăm. Am ajuns ca gunoiul lumii, ca măturătura tuturor, până astăzi.
\par 14 Nu ca să vă rușinez vă scriu acestea, ci ca să vă dojenesc, ca pe niște copii ai mei iubiți.
\par 15 Căci de ați avea zeci de mii de învățători în Hristos, totuși nu aveți mulți părinți. Căci eu v-am născut prin Evanghelie în Iisus Hristos.
\par 16 Deci, vă rog, să-mi fiți mie următori, precum și eu lui Hristos.
\par 17 Pentru aceasta am trimis la voi pe Timotei, care este fiul meu iubit și credincios în Domnul. El vă va aduce aminte căile mele cele în Hristos Iisus, cum învăț eu pretutindeni în toată Biserica.
\par 18 Și unii, crezând că n-am să mai vin la voi, s-au semețit.
\par 19 Dar eu voi veni la voi degrabă - dacă Domnul va voi - și voi cunoaște nu cuvântul celor ce s-au semețit, ci puterea lor.
\par 20 Căci împărăția lui Dumnezeu nu stă în cuvânt, ci în putere.
\par 21 Ce voiți? Să vin la voi cu toiagul sau să vin cu dragoste și cu duhul blândeții?

\chapter{5}

\par 1 Îndeobște se aude că la voi e desfrânare, și o astfel de desfrânare cum nici între neamuri nu se pomenește, ca unul să trăiască cu femeia tatălui său.
\par 2 Iar voi v-ați semețit, în loc mai degrabă să vă fi întristat, ca să fie scos din mijlocul vostru cel ce a săvârșit această faptă.
\par 3 Ci eu, deși departe cu trupul, însă de față cu duhul, am și judecat, ca și cum aș fi de față, pe cel ce a făcut una ca aceasta:
\par 4 În numele Domnului nostru Iisus Hristos, adunându-vă voi și duhul meu, cu puterea Domnului nostru Iisus Hristos,
\par 5 Să dați pe unul ca acesta satanei, spre pieirea trupului, ca duhul să se mântuiască în ziua Domnului Iisus.
\par 6 Semeția voastră nu e bună. Oare nu știți că puțin aluat dospește toată frământătura?
\par 7 Curățiți aluatul cel vechi, ca să fiți frământătură nouă, precum și sunteți fără aluat; căci Paștile nostru Hristos S-a jertfit pentru noi.
\par 8 De aceea să prăznuim nu cu aluatul cel vechi, nici cu aluatul răutății și al vicleșugului, ci cu azimele curăției și ale adevărului.
\par 9 V-am scris în epistolă să nu vă amestecați cu desfrânații;
\par 10 Dar nu am spus, desigur, despre desfrânații acestei lumi, sau despre lacomi, sau despre răpitori, sau despre închinătorii la idoli, căci altfel ar trebui să ieșiți afară din lume.
\par 11 Dar eu v-am scris acum să nu vă amestecați cu vreunul, care, numindu-se frate, va fi desfrânat, sau lacom, sau închinător la idoli, sau ocărâtor, sau bețiv, sau răpitor. Cu unul ca acesta nici să nu ședeți la masă.
\par 12 Căci ce am eu ca să judec și pe cei din afară? Însă pe cei dinăuntru, oare, nu-i judecați voi?
\par 13 Iar pe cei din afară îi va judeca Dumnezeu. Scoateți afară dintre voi pe cel rău.

\chapter{6}

\par 1 Îndrăznește, oare, cineva dintre voi, având vreo pâră împotriva altuia, să se judece înaintea celor nedrepți și nu înaintea celor sfinți?
\par 2 Au nu știți că sfinții vor judeca lumea? Și dacă lumea este judecată de voi, oare sunteți voi nevrednici să judecați lucruri atât de mici?
\par 3 Nu știți, oare, că noi vom judeca pe îngeri? Cu cât mai mult cele lumești?
\par 4 Deci dacă aveți judecăți lumești, puneți pe cei nebăgați în seamă din Biserică, ca să vă judece.
\par 5 O spun spre rușinea voastră. Nu este, oare, între voi nici un om înțelept, care să poată judeca între frate și frate?
\par 6 Ci frate cu frate se judecă, și aceasta înaintea necredincioșilor?
\par 7 Negreșit, și aceasta este o scădere pentru voi, că aveți judecăți unii cu alții. Pentru ce nu suferiți mai bine strâmbătatea? Pentru ce nu răbdați mai bine paguba?
\par 8 Ci voi înșivă faceți strâmbătate și aduceți pagubă, și aceasta, fraților!
\par 9 Nu știți, oare, că nedrepții nu vor moșteni împărăția lui Dumnezeu? Nu vă amăgiți: Nici desfrânații, nici închinătorii la idoli, nici adulterii, nici malahienii, nici sodomiții,
\par 10 Nici furii, nici lacomii, nici bețivii, nici batjocoritorii, nici răpitorii nu vor moșteni împărăția lui Dumnezeu.
\par 11 Și așa erați unii dintre voi. Dar v-ați spălat, dar v-ați sfințit, dar v-ați îndreptat în numele Domnului Iisus Hristos și în Duhul Dumnezeului nostru.
\par 12 Toate îmi sunt îngăduite, dar nu toate îmi sunt de folos. Toate îmi sunt îngăduite, dar nu mă voi lăsa biruit de ceva.
\par 13 Bucatele sunt pentru pântece și pântecele pentru bucate și Dumnezeu va nimici și pe unul și pe celelalte. Trupul însă nu e pentru desfrânare, ci pentru Domnul, și Domnul este pentru trup.
\par 14 Iar Dumnezeu, Care a înviat pe Domnul, ne va învia și pe noi prin puterea Sa.
\par 15 Au nu știți că trupurile voastre sunt mădularele lui Hristos? Luând deci mădularele lui Hristos le voi face mădularele unei desfrânate? Nicidecum!
\par 16 Sau nu știți că cel ce se alipește de desfrânate este un singur trup cu ea? "Căci vor fi - zice Scriptura - cei doi un singur trup".
\par 17 Iar cel ce se alipește de Domnul este un duh cu El.
\par 18 Fugiți de desfrânare! Orice păcat pe care-l va săvârși omul este în afară de trup. Cine se dedă însă desfrânării păcătuiește în însuși trupul său.
\par 19 Sau nu știți că trupul vostru este templu al Duhului Sfânt care este în voi, pe care-L aveți de la Dumnezeu și că voi nu sunteți ai voștri?
\par 20 Căci ați fost cumpărați cu preț! Slăviți, dar, pe Dumnezeu în trupul vostru și în duhul vostru, care sunt ale lui Dumnezeu.

\chapter{7}

\par 1 Cât despre cele ce mi-ați scris, bine este pentru om să nu se atingă de femeie.
\par 2 Dar din cauza desfrânării, fiecare să-și aibă femeia sa și fiecare femeie să-și aibă bărbatul său.
\par 3 Bărbatul să-i dea femeii iubirea datorată, asemenea și femeia bărbatului.
\par 4 Femeia nu este stăpână pe trupul său, ci bărbatul; asemenea nici bărbatul nu este stăpân pe trupul său, ci femeia.
\par 5 Să nu vă lipsiți unul de altul, decât cu bună învoială pentru un timp, ca să vă îndeletniciți cu postul și cu rugăciunea, și iarăși să fiți împreună, ca să nu vă ispitească satana, din pricina neînfrânării voastre.
\par 6 Și aceasta o spun ca un sfat, nu ca o poruncă.
\par 7 Eu voiesc ca toți oamenii să fie cum sunt eu însumi. Dar fiecare are de la Dumnezeu darul lui: unul așa, altul într-alt fel.
\par 8 Celor ce sunt necăsătoriți și văduvelor le spun: Bine este pentru ei să rămână ca și mine.
\par 9 Dacă însă nu pot să se înfrâneze, să se căsătorească. Fiindcă mai bine este să se căsătorească, decât să ardă.
\par 10 Iar celor ce sunt căsătoriți, le poruncesc, nu eu, ci Domnul: Femeia să nu se despartă de bărbat!
\par 11 Iar dacă s-a despărțit, să rămână nemăritată, sau să se împace cu bărbatul său; tot așa bărbatul să nu-și lase femeia.
\par 12 Celorlalți le grăiesc eu, nu Domnul: Dacă un frate are o femeie necredincioasă, și ea voiește să viețuiască cu el, să nu o lase.
\par 13 Și o femeie, dacă are bărbat necredincios, și el binevoiește să locuiască cu ea, să nu-și lase bărbatul.
\par 14 Căci bărbatul necredincios se sfințește prin femeia credincioasă și femeia necredincioasă se sfințește prin bărbatul credincios. Altminterea, copiii voștri ar fi necurați, dar acum ei sunt sfinți.
\par 15 Dacă însă cel necredincios se desparte, să se despartă. În astfel de împrejurare, fratele sau sora nu sunt legați; căci Dumnezeu ne-a chemat spre pace.
\par 16 Căci, ce știi tu, femeie, dacă îți vei mântui bărbatul? Sau ce știi tu, bărbate, dacă îți vei mântui femeia?
\par 17 Numai că, așa cum a dat Domnul fiecăruia, așa cum l-a chemat Dumnezeu pe fiecare, astfel să umble. Și așa rânduiesc în toate Bisericile.
\par 18 A fost cineva chemat, fiind tăiat împrejur? Să nu se ascundă. A fost cineva chemat în netăiere împrejur? Să nu se taie împrejur.
\par 19 Tăierea împrejur nu este nimic; și netăierea împrejur nu este nimic, ci paza poruncilor lui Dumnezeu.
\par 20 Fiecare, în chemarea în care a fost chemat, în aceasta să rămână.
\par 21 Ai fost chemat fiind rob? Fii fără grijă. Iar de poți să fii liber, mai mult folosește-te!
\par 22 Căci robul, care a fost chemat în Domnul, este un liberat al Domnului. Tot așa cel chemat liber este rob al lui Hristos.
\par 23 Cu preț ați fost cumpărați. Nu vă faceți robi oamenilor.
\par 24 Fiecare, fraților, în starea în care a fost chemat, în aceea să rămână înaintea lui Dumnezeu.
\par 25 Cât despre feciorie, n-am poruncă de la Domnul. Vă dau însă sfatul meu, ca unul care am fost miluit de Domnul să fiu vrednic de crezare.
\par 26 Socotesc deci că aceasta este bine pentru nevoia ceasului de față: Bine este pentru om să fie așa.
\par 27 Te-ai legat de femeie? Nu căuta dezlegare. Te-ai dezlegat de femeie? Nu căuta femeie.
\par 28 Dacă însă te vei însura, n-ai greșit. Ci fecioara, de se va mărita, n-a greșit. Numai că unii ca aceștia vor avea suferință în trupul lor. Eu însă vă cruț pe voi.
\par 29 Și aceasta v-o spun, fraților: Că vremea s-a scurtat de acum, așa încât și cei ce au femei să fie ca și cum n-ar avea.
\par 30 Și cei ce plâng să fie ca și cum n-ar plânge; și cei ce se bucură, ca și cum nu s-ar bucura; și cei ce cumpără, ca și cum n-ar stăpâni;
\par 31 Și cei ce se folosesc de lumea aceasta, ca și cum nu s-ar folosi deplin de ea. Căci chipul acestei lumi trece.
\par 32 Dar eu vreau ca voi să fiți fără de grijă. Cel necăsătorit se îngrijește de cele ale Domnului, cum să placă Domnului.
\par 33 Cel ce s-a căsătorit se îngrijește de cele ale lumii, cum să placă femeii.
\par 34 Și este împărțire: și femeia nemăritată și fecioara poartă de grijă de cele ale Domnului, ca să fie sfântă și cu trupul și cu duhul. Iar cea care s-a măritat poartă de grijă de cele ale lumii, cum să placă bărbatului.
\par 35 Și aceasta o spun chiar în folosul vostru, nu ca să vă întind o cursă, ci spre bunul chip și alipirea de Domnul, fără clintire.
\par 36 Iar de socotește cineva că i se va face vreo necinste pentru fecioara sa, dacă trece de floarea vârstei, și că trebuie să facă așa, facă ce voiește. Nu păcătuiește; căsătorească-se.
\par 37 Dar cel ce stă neclintit în inima sa și nu este silit, ci are stăpânire peste voința sa și a hotărât aceasta în inima sa, ca să-și țină fecioara, bine va face.
\par 38 Așa că, cel ce își mărită fecioara bine face; dar cel ce n-o mărită și mai bine face.
\par 39 Femeia este legată prin lege atâta vreme cât trăiește bărbatul ei. Iar dacă bărbatul ei va muri, este liberă să se mărite cu cine vrea, numai întru Domnul.
\par 40 Dar mai fericită este dacă rămâne așa, după părerea mea. Și socot că și eu am Duhul lui Dumnezeu.

\chapter{8}

\par 1 Cât despre cele jertfite idolilor, știm că toți avem cunoștință. Cunoștința însă semețește, iar iubirea zidește.
\par 2 Iar dacă i se pare cuiva că cunoaște ceva, încă n-a cunoscut cum trebuie să cunoască.
\par 3 Dar dacă iubește cineva pe Dumnezeu, acela este cunoscut de El.
\par 4 Iar despre mâncarea celor jertfite idolilor, știm că idolul nu este nimic în lume și că nu este alt Dumnezeu decât Unul singur.
\par 5 Căci deși sunt așa-ziși dumnezei, fie în cer, fie pe pământ, - precum și sunt dumnezei mulți și domni mulți, S
\par 6 Totuși, pentru noi, este un singur Dumnezeu, Tatăl, din Care sunt toate și noi întru El; și un singur Domn, Iisus Hristos, prin Care sunt toate și noi prin El.
\par 7 Dar nu toți au cunoștința. Căci unii, din obișnuința de până acum cu idolul, mănâncă din cărnuri jertfite idolilor, și conștiința lor fiind slabă, se întinează.
\par 8 Dar nu mâncarea ne va pune înaintea lui Dumnezeu. Că nici dacă vom mânca, nu ne prisosește, nici dacă nu vom mânca, nu ne lipsește.
\par 9 Dar vedeți ca nu cumva această libertate a voastră să ajungă poticnire pentru cei slabi.
\par 10 Căci dacă cineva te-ar vedea pe tine, cel ce ai cunoștință, șezând la masă în templul idolilor, oare conștiința lui, slab fiind el, nu se va întări să mănânce din cele jertfite idolilor?
\par 11 Și va pieri prin cunoștința ta cel slab, fratele tău, pentru care a murit Hristos!
\par 12 Și așa, păcătuind împotriva fraților și lovind conștiința lor slabă, păcătuiți față de Hristos.
\par 13 De aceea, dacă o mâncare smintește pe fratele meu, nu voi mânca în veac carne, ca să nu aduc sminteală fratelui meu.

\chapter{9}

\par 1 Oare nu sunt eu liber? Nu sunt eu apostol? N-am văzut eu pe Iisus Domnul nostru? Nu sunteți voi lucrul meu întru Domnul?
\par 2 Dacă altora nu le sunt apostol, vouă, negreșit, vă sunt. Căci voi sunteți pecetea apostoliei mele în Domnul.
\par 3 Apărarea mea către cei ce mă judecă aceasta este.
\par 4 N-avem, oare, dreptul, să mâncăm și să bem?
\par 5 N-avem, oare, dreptul să purtăm cu noi o femeie soră, ca și ceilalți apostoli, ca și frații Domnului, ca și Chefa?
\par 6 Sau numai eu și Barnaba nu avem dreptul de a nu lucra?
\par 7 Cine slujește vreodată, în oaste, cu solda lui? Cine sădește vie și nu mănâncă din roada ei? Sau cine paște o turmă și nu mănâncă din laptele turmei?
\par 8 Nu în felul oamenilor spun eu acestea. Nu spune, oare, și legea acestea?
\par 9 Căci în Legea lui Moise este scris: "Să nu legi gura boului care treieră". Oare de boi se îngrijește Dumnezeu?
\par 10 Sau în adevăr pentru noi zice? Căci pentru noi s-a scris: "Cel ce ară trebuie să are cu nădejde, și cel ce treieră, cu nădejdea că va avea parte de roade".
\par 11 Dacă noi am semănat la voi cele duhovnicești, este, oare, mare lucru dacă noi vom secera cele pământești ale voastre?
\par 12 Dacă alții se bucură de acest drept asupra voastră, oare, nu cu atât mai mult noi? Dar nu ne-am folosit de dreptul acesta, ci toate le răbdăm, ca să nu punem piedică Evangheliei lui Hristos.
\par 13 Au nu știți că cei ce săvârșesc cele sfinte mănâncă de la templu și cei ce slujesc altarului au parte de la altar?
\par 14 Tot așa a poruncit și Domnul celor ce propovăduiesc Evanghelia, ca să trăiască din Evanghelie.
\par 15 Dar eu nu m-am folosit de nimic din acestea și nu am scris acestea, ca să se facă cu mine așa. Căci mai bine este pentru mine să mor, decât să-mi zădărnicească cineva lauda.
\par 16 Căci dacă vestesc Evanghelia, nu-mi este laudă, pentru că stă asupra mea datoria. Căci, vai mie dacă nu voi binevesti!
\par 17 Căci dacă fac aceasta de bună voie, am plată; dar dacă o fac fără voie, am numai o slujire încredințată.
\par 18 Care este, deci, plata mea? Că, binevestind, pun fără plată Evanghelia lui Hristos înaintea oamenilor, fără să mă folosesc de dreptul meu din Evanghelie.
\par 19 Căci, deși sunt liber față de toți, m-am făcut rob tuturor, ca să dobândesc pe cei mai mulți;
\par 20 Cu iudeii am fost ca un iudeu, ca să dobândesc pe iudei; cu cei de sub lege, ca unul de sub lege, deși eu nu sunt sub lege, ca să dobândesc pe cei de sub lege;
\par 21 Cu cei ce n-au Legea, m-am făcut ca unul fără lege, deși nu sunt fără Legea lui Dumnezeu, ci având Legea lui Hristos, ca să dobândesc pe cei ce n-au Legea;
\par 22 Cu cei slabi m-am făcut slab, ca pe cei slabi să-i dobândesc; tuturor toate m-am făcut, ca, în orice chip, să mântuiesc pe unii.
\par 23 Dar toate le fac pentru Evanghelie, ca să fiu părtaș la ea.
\par 24 Nu știți voi că acei care aleargă în stadion, toți aleargă, dar numai unul ia premiul? Alergați așa ca să-l luați.
\par 25 Și oricine se luptă se înfrânează de la toate. Și aceia, ca să ia o cunună stricăcioasă, iar noi, nestricăcioasă.
\par 26 Eu, deci, așa alerg, nu ca la întâmplare. Așa mă lupt, nu ca lovind în aer,
\par 27 Ci îmi chinuiesc trupul meu și îl supun robiei; ca nu cumva, altora propovăduind, eu însumi să mă fac netrebnic.

\chapter{10}

\par 1 Căci nu voiesc, fraților, ca voi să nu știți că părinții noștri au fost toți sub nor și că toți au trecut prin mare.
\par 2 Și toți, întru Moise, au fost botezați în nor și în mare.
\par 3 Și toți au mâncat aceeași mâncare duhovnicească;
\par 4 Și toți, aceeași băutură duhovnicească au băut, pentru că beau din piatra duhovnicească ce îi urma. Iar piatra era Hristos.
\par 5 Dar cei mai mulți dintre ei nu au plăcut lui Dumnezeu, căci au căzut în pustie.
\par 6 Și acestea s-au făcut pilde pentru noi, ca să nu poftim la cele rele, cum au poftit aceia;
\par 7 Nici închinători la idoli să nu vă faceți, ca unii dintre ei, precum este scris: "A șezut poporul să mănânce și să bea și s-au sculat la joc";
\par 8 Nici să ne desfrânăm cum s-au desfrânat unii dintre ei, și au căzut, într-o zi, douăzeci și trei de mii;
\par 9 Nici să ispitim pe Domnul, precum L-au ispitit unii dintre ei și au pierit de șerpi;
\par 10 Nici să cârtiți, precum au cârtit unii dintre ei și au fost nimiciți de către pierzătorul.
\par 11 Și toate acestea li s-au întâmplat acelora, ca preînchipuiri ale viitorului, și au fost scrise spre povățuirea noastră, la care au ajuns sfârșiturile veacurilor.
\par 12 De aceea, cel căruia i se pare că stă neclintit să ia seama să nu cadă.
\par 13 Nu v-a cuprins ispită care să fi fost peste puterea omenească. Dar credincios este Dumnezeu; El nu va îngădui ca să fiți ispitiți mai mult decât puteți, ci odată cu ispita va aduce și scăparea din ea, ca să puteți răbda.
\par 14 De aceea, iubiții mei, fugiți de închinarea la idoli.
\par 15 Ca unor înțelepți vă vorbesc. Judecați voi ce vă spun.
\par 16 Paharul binecuvântării, pe care-l binecuvântăm, nu este, oare, împărtășirea cu sângele lui Hristos? Pâinea pe care o frângem nu este, oare, împărtășirea cu trupul lui Hristos?
\par 17 Că o pâine, un trup, suntem cei mulți; căci toți ne împărtășim dintr-o pâine.
\par 18 Priviți pe Israel după trup: Cei care mănâncă jertfele nu sunt ei, oare, părtași altarului?
\par 19 Deci ce spun eu? Că ce s-a jertfit pentru idol e ceva? Sau idolul este ceva?
\par 20 Ci (zic) că cele ce jertfesc neamurile, jertfesc demonilor și nu lui Dumnezeu. Și nu voiesc ca voi să fiți părtași ai demonilor.
\par 21 Nu puteți să beți paharul Domnului și paharul demonilor; nu puteți să vă împărtășiți din masa Domnului și din masa demonilor.
\par 22 Oare vrem să mâniem pe Domnul? Nu cumva suntem mai tari decât El?
\par 23 Toate îmi sunt îngăduite, dar nu toate îmi folosesc. Toate îmi sunt îngăduite, dar nu toate zidesc.
\par 24 Nimeni să nu caute pe ale sale, ci fiecare pe ale aproapelui.
\par 25 Mâncați tot ce se vinde în măcelărie, fără să întrebați nimic pentru cugetul vostru.
\par 26 Căci "al Domnului este pământul și plinirea lui".
\par 27 Dacă cineva dintre necredincioși vă cheamă pe voi la masă și voiți să vă duceți, mâncați orice vă este pus înainte, fără să întrebați nimic pentru conștiință.
\par 28 Dar de vă va spune cineva: Aceasta este din jertfa idolilor, să nu mâncați pentru cel care v-a spus și pentru conștiință.
\par 29 Iar conștiința, zic, nu a ta însuți, ci a altuia. Căci de ce libertatea mea să fie judecată de o altă conștiință?
\par 30 Dacă eu sunt părtaș harului, de ce să fiu hulit pentru ceea ce aduc mulțumire?
\par 31 De aceea, ori de mâncați, ori de beți, ori altceva de faceți, toate spre slava lui Dumnezeu să le faceți.
\par 32 Nu fiți piatră de poticnire nici iudeilor, nici elinilor, nici Bisericii lui Dumnezeu,
\par 33 Precum și eu plac tuturor în toate, necăutând folosul meu, ci pe al celor mulți, ca să se mântuiască.

\chapter{11}

\par 1 Fiți următori ai mei, precum și eu sunt al lui Hristos.
\par 2 Fraților, vă laud că în toate vă aduceți aminte de mine și țineți predaniile cum vi le-am dat.
\par 3 Dar voiesc ca voi să știți că Hristos este capul oricărui bărbat, iar capul femeii este bărbatul, iar capul lui Hristos: Dumnezeu.
\par 4 Orice bărbat care se roagă sau proorocește, având capul acoperit, necinstește capul său.
\par 5 Iar orice femeie care se roagă sau proorocește, cu capul neacoperit, își necinstește capul; căci tot una este ca și cum ar fi rasă.
\par 6 Căci dacă o femeie nu-și pune văl pe cap, atunci să se și tundă. Iar dacă este lucru de rușine pentru femeie ca să se tundă ori să se radă, să-și pună văl.
\par 7 Căci bărbatul nu trebuie să-și acopere capul, fiind chip și slavă a lui Dumnezeu, iar femeia este slava bărbatului.
\par 8 Pentru că nu bărbatul este din femeie, ci femeia din bărbat.
\par 9 Și pentru că n-a fost zidit bărbatul pentru femeie, ci femeia pentru bărbat.
\par 10 De aceea și femeia este datoare să aibă (semn de) supunere asupra capului ei, pentru îngeri.
\par 11 Totuși, nici femeia fără bărbat, nici bărbatul fără femeie, în Domnul.
\par 12 Căci precum femeia este din bărbat, așa și bărbatul este prin femeie și toate sunt de la Dumnezeu.
\par 13 Judecați în voi înșivă: Este, oare, cuviincios ca o femeie să se roage lui Dumnezeu cu capul descoperit?
\par 14 Nu vă învață oare însăși firea că necinste este pentru un bărbat să-și lase părul lung?
\par 15 Și că pentru o femeie, dacă își lasă părul lung, este cinste? Căci părul i-a fost dat ca acoperământ.
\par 16 Iar dacă se pare cuiva că aici poate să ne găsească pricină, un astfel de obicei (ca femeile să se roage cu capul descoperit) noi nu avem, nici Bisericile lui Dumnezeu.
\par 17 Și aceasta poruncindu-vă, nu vă laud, fiindcă voi vă adunați nu spre mai bine, ci spre mai rău.
\par 18 Căci mai întâi aud că atunci când vă adunați în biserică, între voi sunt dezbinări, și în parte cred.
\par 19 Căci trebuie să fie între voi și eresuri, ca să se învedereze între voi cei încercați.
\par 20 Când vă adunați deci laolaltă, nu se poate mânca Cina Domnului;
\par 21 Căci, șezând la masă, fiecare se grăbește să ia mâncarea sa, încât unuia îi este foame, iar altul se îmbată.
\par 22 N-aveți, oare, case ca să mâncați și să beți? Sau disprețuiți Biserica lui Dumnezeu și rușinați pe cei ce nu au? Ce să vă zic? Să vă laud? În aceasta nu vă laud.
\par 23 Căci eu de la Domnul am primit ceea ce v-am dat și vouă: Că Domnul Iisus, în noaptea în care a fost vândut, a luat pâine,
\par 24 Și, mulțumind, a frânt și a zis: Luați, mâncați; acesta este trupul Meu care se frânge pentru voi. Aceasta să faceți spre pomenirea Mea.
\par 25 Asemenea și paharul după Cină, zicând: Acest pahar este Legea cea nouă întru sângele Meu. Aceasta să faceți ori de câte ori veți bea, spre pomenirea Mea.
\par 26 Căci de câte ori veți mânca această pâine și veți bea acest pahar, moartea Domnului vestiți până când va veni.
\par 27 Astfel, oricine va mânca pâinea aceasta sau va bea paharul Domnului cu nevrednicie, va fi vinovat față de trupul și sângele Domnului.
\par 28 Să se cerceteze însă omul pe sine și așa să mănânce din pâine și să bea din pahar.
\par 29 Căci cel ce mănâncă și bea cu nevrednicie, osândă își mănâncă și bea, nesocotind trupul Domnului.
\par 30 De aceea, mulți dintre voi sunt neputincioși și bolnavi și mulți au murit.
\par 31 Căci de ne-am fi judecat noi înșine, nu am mai fi judecați.
\par 32 Dar, fiind judecați de Domnul, suntem pedepsiți, ca să nu fim osândiți împreună cu lumea.
\par 33 De aceea, frații mei, când vă adunați ca să mâncați, așteptați-vă unii pe alții.
\par 34 Iar dacă îi este cuiva foame, să mănânce acasă, ca să nu vă adunați spre osândă. Celelalte însă le voi rândui când voi veni.

\chapter{12}

\par 1 Iar cât privește darurile duhovnicești nu vreau, fraților, să fiți în necunoștință.
\par 2 Știți că, pe când erați păgâni, vă duceați la idolii cei muți, ca și cum erați mânați.
\par 3 De aceea, vă fac cunoscut că precum nimeni, grăind în Duhul lui Dumnezeu, nu zice: Anatema fie Iisus! - tot așa nimeni nu poate să zică: Domn este Iisus, - decât în Duhul Sfânt.
\par 4 Darurile sunt felurite, dar același Duh.
\par 5 Și felurite slujiri sunt, dar același Domn.
\par 6 Și lucrările sunt felurite, dar este același Dumnezeu, care lucrează toate în toți.
\par 7 Și fiecăruia se dă arătarea Duhului spre folos.
\par 8 Că unuia i se dă prin Duhul Sfânt cuvânt de înțelepciune, iar altuia, după același Duh, cuvântul cunoștinței.
\par 9 Și unuia i se dă întru același Duh credință, iar altuia, darurile vindecărilor, întru același Duh;
\par 10 Unuia faceri de minuni, iar altuia proorocie; unuia deosebirea duhurilor, iar altuia feluri de limbi și altuia tălmăcirea limbilor.
\par 11 Și toate acestea le lucrează unul și același Duh, împărțind fiecăruia deosebi, după cum voiește.
\par 12 Căci precum trupul unul este, și are mădulare multe, iar toate mădularele trupului, multe fiind, sunt un trup, așa și Hristos.
\par 13 Pentru că într-un Duh ne-am botezat noi toți, ca să fim un singur trup, fie iudei, fie elini, fie robi, fie liberi, și toți la un Duh ne-am adăpat.
\par 14 Căci și trupul nu este un mădular, ci multe.
\par 15 Dacă piciorul ar zice: Fiindcă nu sunt mână nu sunt din trup, pentru aceasta nu este el din trup?
\par 16 Și urechea dacă ar zice: Fiindcă nu sunt ochi, nu fac parte din trup, - pentru aceasta nu este ea din trup?
\par 17 Dacă tot trupul ar fi ochi, unde ar fi auzul? Și dacă ar fi tot auz, unde ar fi mirosul?
\par 18 Dar acum Dumnezeu a pus mădularele, pe fiecare din ele, în trup, cum a voit.
\par 19 Dacă toate ar fi un singur mădular, unde ar fi trupul?
\par 20 Dar acum sunt multe mădulare, însă un singur trup.
\par 21 Și nu poate ochiul să zică mâinii: N-am trebuință de tine; sau, iarăși capul să zică picioarelor: N-am trebuință de voi.
\par 22 Ci cu mult mai mult mădularele trupului, care par a fi mai slabe, sunt mai trebuincioase.
\par 23 Și pe cele ale trupului care ni se par că sunt mai de necinste, pe acelea cu mai multă evlavie le îmbrăcăm; și cele necuviincioase ale noastre au mai multă cuviință.
\par 24 Iar cele cuviincioase ale noastre n-au nevoie de acoperământ. Dar Dumnezeu a întocmit astfel trupul, dând mai multă cinste celui căruia îi lipsește,
\par 25 Ca să nu fie dezbinare în trup, ci mădularele să se îngrijească deopotrivă unele de altele.
\par 26 Și dacă un mădular suferă, toate mădularele suferă împreună; și dacă un mădular este cinstit, toate mădularele se bucură împreună.
\par 27 Iar voi sunteți trupul lui Hristos și mădulare (fiecare) în parte.
\par 28 Și pe unii i-a pus Dumnezeu, în Biserică: întâi apostoli, al doilea prooroci, al treilea învățători; apoi pe cei ce au darul de a face minuni; apoi darurile vindecărilor, ajutorările, cârmuirile, felurile limbilor.
\par 29 Oare toți sunt apostoli? Oare toți sunt prooroci? Oare toți învățători? Oare toți au putere să săvârșească minuni?
\par 30 Oare toți au darurile vindecărilor? Oare toți vorbesc în limbi? Oare toți pot să tălmăcească?
\par 31 Râvniți însă la darurile cele mai bune. Și vă arăt încă o cale care le întrece pe toate:

\chapter{13}

\par 1 De aș grăi în limbile oamenilor și ale îngerilor, iar dragoste nu am, făcutu-m-am aramă sunătoare și chimval răsunător.
\par 2 Și de aș avea darul proorociei și tainele toate le-aș cunoaște și orice știință, și de aș avea atâta credință încât să mut și munții, iar dragoste nu am, nimic nu sunt.
\par 3 Și de aș împărți toată avuția mea și de aș da trupul meu ca să fie ars, iar dragoste nu am, nimic nu-mi folosește.
\par 4 Dragostea îndelung rabdă; dragostea este binevoitoare, dragostea nu pizmuiește, nu se laudă, nu se trufește.
\par 5 Dragostea nu se poartă cu necuviință, nu caută ale sale, nu se aprinde de mânie, nu gândește răul.
\par 6 Nu se bucură de nedreptate, ci se bucură de adevăr.
\par 7 Toate le suferă, toate le crede, toate le nădăjduiește, toate le rabdă.
\par 8 Dragostea nu cade niciodată. Cât despre proorocii - se vor desființa; darul limbilor va înceta; știința se va sfârși;
\par 9 Pentru că în parte cunoaștem și în parte proorocim.
\par 10 Dar când va veni ceea ce e desăvârșit, atunci ceea ce este în parte se va desființa.
\par 11 Când eram copil, vorbeam ca un copil, simțeam ca un copil; judecam ca un copil; dar când m-am făcut bărbat, am lepădat cele ale copilului.
\par 12 Căci vedem acum ca prin oglindă, în ghicitură, iar atunci, față către față; acum cunosc în parte, dar atunci voi cunoaște pe deplin, precum am fost cunoscut și eu.
\par 13 Și acum rămân acestea trei: credința, nădejdea și dragostea. Iar mai mare dintre acestea este dragostea.

\chapter{14}

\par 1 Căutați dragostea. Râvniți însă cele duhovnicești, dar mai ales ca să proorociți.
\par 2 Pentru că cel ce vorbește într-o limbă străină nu vorbește oamenilor, ci lui Dumnezeu; și nimeni nu-l înțelege, fiindcă el, în duh, grăiește taine.
\par 3 Cel ce proorocește vorbește oamenilor, spre zidire, îndemn și mângâiere.
\par 4 Cel ce grăiește într-o limbă străină pe sine singur se zidește, iar cel ce proorocește zidește Biserica.
\par 5 Voiesc ca voi toți să grăiți în limbi; dar mai cu seamă să proorociți. Cel ce proorocește e mai mare decât cel ce grăiește în limbi, afară numai dacă tălmăcește, ca Biserica să ia întărire.
\par 6 Iar acum, fraților, dacă aș veni la voi, grăind în limbi, de ce folos v-aș fi, dacă nu v-aș vorbi - sau în descoperire, sau în cunoștință, sau în proorocie, sau în învățătură?
\par 7 Că precum cele neînsuflețite, care dau sunet, fie fluier, fie chitară, de nu vor da sunete deosebite, cum se va cunoaște ce este din fluier, sau ce este din chitară?
\par 8 Și dacă trâmbița va da sunet nelămurit, cine se va pregăti de război?
\par 9 Așa și voi: Dacă prin limbă nu veți da cuvânt lesne de înțeles, cum se va cunoaște ce ați grăit? Veți fi niște oameni care vorbesc în vânt.
\par 10 Sunt așa de multe feluri de limbi în lume, dar nici una din ele nu este fără înțelesul ei.
\par 11 Deci dacă nu voi ști înțelesul cuvintelor, voi fi barbar pentru cel care vorbește, și cel care vorbește barbar pentru mine.
\par 12 Așa și voi, de vreme ce sunteți râvnitori după cele duhovnicești, căutați să prisosiți în ele, spre zidirea Bisericii.
\par 13 De aceea, cel ce grăiește într-o limbă străină să se roage ca să și tălmăcească.
\par 14 Căci, dacă mă rog într-o limbă străină, duhul meu se roagă, dar mintea mea este neroditoare.
\par 15 Atunci ce voi face? Mă voi ruga cu duhul, dar mă voi ruga și cu mintea; voi cânta cu duhul, dar voi cânta și cu mintea.
\par 16 Fiindcă dacă vei binecuvânta cu duhul, cum va răspunde omul simplu "Amin" la mulțumirea ta, de vreme ce el nu știe ce zici?
\par 17 Căci tu, într-adevăr, mulțumești bine, dar celălalt nu se zidește.
\par 18 Mulțumesc Dumnezeului meu, că vorbesc în limbi mai mult decât voi toți;
\par 19 Dar în Biserică vreau să grăiesc cinci cuvinte cu mintea mea, ca să învăț și pe alții, decât zeci de mii de cuvinte într-o limbă străină.
\par 20 Fraților, nu fiți copii la minte. Fiți copii când e vorba de răutate. La minte însă, fiți desăvârșiți.
\par 21 În Lege este scris: "Voi grăi acestui popor în alte limbi și prin buzele altora, și nici așa nu vor asculta de Mine, zice Domnul".
\par 22 Așa că vorbirea în limbi este semn nu pentru cei credincioși ci pentru cei necredincioși; iar proorocia nu pentru cei necredincioși, ci pentru cei ce cred.
\par 23 Deci, dacă s-ar aduna Biserica toată laolaltă și toți ar vorbi în limbi și ar intra neștiutori sau necredincioși, nu vor zice, oare, că sunteți nebuni?
\par 24 Iar dacă toți ar prooroci și ar intra vreun necredincios sau vreun neștiutor, el este dovedit de toți, el este judecat de toți;
\par 25 Cele ascunse ale inimii lui se învederează, și astfel, căzând cu fața la pământ, se va închina lui Dumnezeu, mărturisind că Dumnezeu este într-adevăr printre voi.
\par 26 Ce este deci, fraților? Când vă adunați împreună, fiecare din voi are psalm, are învățătură, are descoperire, are limbă, are tălmăcire: toate spre zidire să se facă.
\par 27 Dacă grăiește cineva într-o limbă străină, să fie câte doi, sau cel mult trei și pe rând să grăiască și unul să tălmăcească.
\par 28 Iar dacă nu e tălmăcitor, să tacă în biserică și să-și grăiască numai lui și lui Dumnezeu.
\par 29 Iar proorocii să vorbească doi sau trei, iar ceilalți să judece.
\par 30 Iar dacă se va descoperi ceva altuia care șade, să tacă cei dintâi.
\par 31 Căci puteți să proorociți toți câte unul, ca toți să învețe și toți să se mângâie.
\par 32 Și duhurile proorocilor se supun proorocilor.
\par 33 Pentru că Dumnezeu nu este al neorânduielii, ci al păcii.
\par 34 Ca în toate Bisericile sfinților, femeile voastre să tacă în biserică, căci lor nu le este îngăduit să vorbească, ci să se supună, precum zice și Legea.
\par 35 Iar dacă voiesc să învețe ceva, să întrebe acasă pe bărbații lor, căci este rușinos ca femeile să vorbească în biserică.
\par 36 Oare de la voi a ieșit cuvântul lui Dumnezeu sau a ajuns numai la voi?
\par 37 Dacă i se pare cuiva că este prooroc sau om duhovnicesc, să cunoască că cele ce vă scriu sunt porunci ale Domnului.
\par 38 Iar dacă cineva nu vrea să știe, să nu știe.
\par 39 Așa că, frații mei, râvniți a prooroci și nu opriți să se grăiască în limbi.
\par 40 Dar toate să se facă cu cuviință și după rânduială.

\chapter{15}

\par 1 Vă aduc aminte, fraților, Evanghelia pe care v-am binevestit-o, pe care ați și primit-o, întru care și stați,
\par 2 Prin care și sunteți mântuiți; cu ce cuvânt v-am binevestit-o - dacă o țineți cu tărie, afară numai dacă n-ați crezut în zadar S
\par 3 Căci v-am dat, întâi de toate, ceea ce și eu am primit, că Hristos a murit pentru păcatele noastre după Scripturi;
\par 4 Și că a fost îngropat și că a înviat a treia zi, după Scripturi;
\par 5 Și că S-a arătat lui Chefa, apoi celor doisprezece;
\par 6 În urmă S-a arătat deodată la peste cinci sute de frați, dintre care cei mai mulți trăiesc până astăzi, iar unii au și adormit;
\par 7 După aceea S-a arătat lui Iacov, apoi tuturor apostolilor;
\par 8 Iar la urma tuturor, ca unui născut înainte de vreme, mi S-a arătat și mie.
\par 9 Căci eu sunt cel mai mic dintre apostoli, care nu sunt vrednic să mă numesc apostol, pentru că am prigonit Biserica lui Dumnezeu.
\par 10 Dar prin harul lui Dumnezeu sunt ceea ce sunt; și harul Lui care este în mine n-a fost în zadar, ci m-am ostenit mai mult decât ei toți. Dar nu eu, ci harul lui Dumnezeu care este cu mine.
\par 11 Deci ori eu, ori aceia, așa propovăduim și voi așa ați crezut.
\par 12 Iar dacă se propovăduiește că Hristos a înviat din morți, cum zic unii dintre voi că nu este înviere a morților?
\par 13 Dacă nu este înviere a morților, nici Hristos n-a înviat.
\par 14 Și dacă Hristos n-a înviat, zadarnică este atunci propovăduirea noastră, zadarnică este și credința voastră.
\par 15 Ne aflăm încă și martori mincinoși ai lui Dumnezeu, pentru că am mărturisit împotriva lui Dumnezeu că a înviat pe Hristos, pe Care nu L-a înviat, dacă deci morții nu înviază.
\par 16 Căci dacă morții nu înviază, nici Hristos n-a înviat.
\par 17 Iar dacă Hristos n-a înviat, zadarnică este credința voastră, sunteți încă în păcatele voastre;
\par 18 Și atunci și cei ce au adormit în Hristos au pierit.
\par 19 Iar dacă nădăjduim în Hristos numai în viața aceasta, suntem mai de plâns decât toți oamenii.
\par 20 Dar acum Hristos a înviat din morți, fiind începătură (a învierii) celor adormiți.
\par 21 Că de vreme ce printr-un om a venit moartea, tot printr-un om și învierea morților.
\par 22 Căci, precum în Adam toți mor, așa și în Hristos toți vor învia.
\par 23 Dar fiecare în rândul cetei sale: Hristos începătură, apoi cei ai lui Hristos, la venirea Lui,
\par 24 După aceea, sfârșitul, când Domnul va preda împărăția lui Dumnezeu și Tatălui, când va desființa orice domnie și orice stăpânire și orice putere.
\par 25 Căci El trebuie să împărățească până ce va pune pe toți vrăjmașii Săi sub picioarele Sale.
\par 26 Vrăjmașul cel din urmă, care va fi nimicit, este moartea.
\par 27 "Căci toate le-a supus sub picioarele Lui". Dar când zice: "Că toate I-au fost supuse Lui" - învederat este că afară de Cel care I-a supus Lui toate.
\par 28 Iar când toate vor fi supuse Lui, atunci și Fiul însuși Se va supune Celui ce I-a supus Lui toate, ca Dumnezeu să fie toate în toți.
\par 29 Fiindcă ce vor face cei care se botează pentru morți? Dacă morții nu înviază nicidecum, pentru ce se mai botează pentru ei?
\par 30 De ce mai suntem și noi în primejdie în tot ceasul?
\par 31 Mor în fiecare zi! V-o spun, fraților, pe lauda pe care o am pentru voi, în Hristos Iisus, Domnul nostru.
\par 32 Dacă m-am luptat, ca om, cu fiarele în Efes, care îmi este folosul? Dacă morții nu înviază, să bem și să mâncăm, căci mâine vom muri!
\par 33 Nu vă lăsați înșelați. Tovărășiile rele strică obiceiurile bune.
\par 34 Treziți-vă cum se cuvine și nu păcătuiți. Căci unii nu au cunoștință de Dumnezeu; o spun spre rușinea voastră.
\par 35 Dar va zice cineva: Cum înviază morții? Și cu ce trup au să vină?
\par 36 Nebun ce ești! Tu ce semeni nu dă viață, dacă nu va fi murit.
\par 37 Și ceea ce semeni nu este trupul ce va să fie, ci grăunte gol, poate de grâu, sau de altceva din celelalte;
\par 38 Iar Dumnezeu îi dă un trup, precum a voit, și fiecărei semințe un trup al său.
\par 39 Nu toate trupurile sunt același trup, ci unul este trupul oamenilor și altul este trupul dobitoacelor și altul este trupul păsărilor și altul este trupul peștilor.
\par 40 Sunt și trupuri cerești și trupuri pământești; dar alta este slava celor cerești și alta a celor pământești.
\par 41 Alta este strălucirea soarelui și alta strălucirea lunii și alta strălucirea stelelor. Căci stea de stea se deosebește în strălucire.
\par 42 Așa este și învierea morților: Se seamănă (trupul) întru stricăciune, înviază întru nestricăciune;
\par 43 Se seamănă întru necinste, înviază întru slavă, se seamănă întru slăbiciune, înviază întru putere;
\par 44 Se seamănă trup firesc, înviază trup duhovnicesc. Dacă este trup firesc, este și trup duhovnicesc.
\par 45 Precum și este scris: "Făcutu-s-a omul cel dintâi, Adam, cu suflet viu; iar Adam cel de pe urmă cu duh dătător de viață";
\par 46 Dar nu este întâi cel duhovnicesc, ci cel firesc, apoi cel duhovnicesc.
\par 47 Omul cel dintâi este din pământ, pământesc; omul cel de-al doilea este din cer.
\par 48 Cum este cel pământesc, așa sunt și cei pământești; și cum este cel ceresc, așa sunt și cei cerești.
\par 49 Și după cum am purtat chipul celui pământesc, să purtăm și chipul celui ceresc.
\par 50 Aceasta însă zic, fraților: Carnea și sângele nu pot să moștenească împărăția lui Dumnezeu, nici stricăciunea nu moștenește nestricăciunea.
\par 51 Iată, taină vă spun vouă: Nu toți vom muri, dar toți ne vom schimba,
\par 52 Deodată, într-o clipeală de ochi la trâmbița cea de apoi. Căci trâmbița va suna și morții vor învia nestricăcioși, iar noi ne vom schimba.
\par 53 Căci trebuie ca acest trup stricăcios să se îmbrace în nestricăciune și acest (trup) muritor să se îmbrace în nemurire.
\par 54 Iar când acest (trup) stricăcios se va îmbrăca în nestricăciune și acest (trup) muritor se va îmbrăca în nemurire, atunci va fi cuvântul care este scris: "Moartea a fost înghițită de biruință.
\par 55 Unde îți este, moarte, biruința ta? Unde îți este, moarte, boldul tău?".
\par 56 Și boldul morții este păcatul, iar puterea păcatului este legea.
\par 57 Dar să dăm mulțumire lui Dumnezeu, Care ne-a dat biruința prin Domnul nostru Iisus Hristos!
\par 58 Drept aceea, frații mei iubiți, fiți tari, neclintiți, sporind totdeauna în lucrul Domnului, știind că osteneala voastră nu este zadarnică în Domnul.

\chapter{16}

\par 1 Cât despre strângerea de ajutoare pentru sfinți, precum am rânduit pentru Bisericile Galatiei, așa să faceți și voi.
\par 2 În ziua întâi a săptămânii (Duminică), fiecare dintre voi să-și pună deoparte, strângând cât poate, ca să nu se facă strângerea abia atunci când voi veni.
\par 3 Iar când voi veni, pe cei pe care îi veți socoti, pe aceia îi voi trimite cu scrisori să ducă darul vostru la Ierusalim.
\par 4 Și de se va cuveni să merg și eu, vor merge împreună cu mine.
\par 5 Ci voi veni la voi, când voi trece prin Macedonia, căci prin Macedonia trec.
\par 6 La voi mă voi opri, poate, sau voi și ierna, ca să mă petreceți în călătoria ce voi face.
\par 7 Căci nu vreau să vă văd acum numai în treacăt, ci nădăjduiesc să rămân la voi câtăva vreme, dacă va îngădui Domnul.
\par 8 Voi rămâne însă în Efes, până la praznicul Cincizecimii.
\par 9 Căci mi s-a deschis ușă mare spre lucru mult, dar sunt mulți potrivnici.
\par 10 Iar de va veni Timotei, vedeți să fie fără teamă la voi, căci lucrează ca și mine lucrul Domnului.
\par 11 Nimeni deci să nu-l disprețuiască; ci să-l petreceți cu pace, ca să vină la mine; că îl aștept cu frații.
\par 12 Cât despre fratele Apollo, l-am rugat mult să vină la voi cu frații; totuși nu i-a fost voia să vină acum. Ci va veni când va găsi prilej.
\par 13 Privegheați, stați tari în credință, îmbărbătați-vă, întăriți-vă.
\par 14 Toate ale voastre cu dragoste să se facă.
\par 15 Vă îndemn însă, fraților, - știți casa lui Ștefanas, că este pârga Ahaei și că spre slujirea sfinților s-au rânduit pe ei înșiși S
\par 16 Ca și voi să vă supuneți unora ca aceștia și oricui lucrează și se ostenește împreună cu ei.
\par 17 Mă bucur de venirea lui Ștefanas, a lui Fortunat și a lui Ahaic, pentru că aceștia au împlinit lipsa voastră.
\par 18 Și au liniștit duhul meu și al vostru. Cunoașteți bine deci pe unii ca aceștia.
\par 19 Vă îmbrățișează Bisericile Asiei. Vă îmbrățișează mult, în Domnul, Acvila și Priscila, împreună cu Biserica din casa lor.
\par 20 Vă îmbrățișează frații toți. Îmbrățișați-vă unii pe alții cu sărutare sfântă.
\par 21 Salutarea cu mâna mea, Pavel.
\par 22 Cel ce nu iubește pe Domnul să fie anatema! Maran atha! (Domnul vine).
\par 23 Harul Domnului Iisus Hristos cu voi.
\par 24 Dragostea mea cu voi toți, în Hristos Iisus! Amin.


\end{document}