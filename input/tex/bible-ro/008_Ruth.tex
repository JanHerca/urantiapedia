\begin{document}

\title{Ruth}


\chapter{1}

\par 1 În zilele acelea, când cârmuiau în Israel judecatorii, s-a întâmplat foamete pe pamânt. Atunci un om din Betleemul lui Iuda s-a dus cu femeia sa ?i cu cei doi feciori ai sai sa locuiasca în ?esul Moabi?ilor.
\par 2 Numele omului aceluia era Elimelec; pe femeia sa o chema Noemina, iar numele celor doi feciori ai lor erau Mahlon ?i Chilion. Ace?tia erau Efrateni din Betleemul lui Iuda ?i, venind în ?esul Moabi?ilor, au ramas acolo.
\par 3 Dupa un timp Elimelec, barbatul Noeminei, a murit ea a ramas cu cei doi feciori ai sai.
\par 4 Ace?tia ?i-au luat so?ii dintre moabitence: numele uneia era Orfa, iar numele celeilalte era Rut. Ace?tia au trait acolo ca la zece ani.
\par 5 Dupa aceea au murit amândoi feciorii ei, Mahlon ?i Chilion ?i a ramas femeia aceea ?i fara barbatul sau ?i fara cei doi feciori ai sai.
\par 6 Atunci s-a hotarât ea cu nurorile sale sa se întoarca din ?esul Moabi?ilor, caci auzise ea în ?esul Moabi?ilor ca Dumnezeu a cercetat pe poporul Sau ?i i-a dat pâine.
\par 7 Deci a plecat ea din locul acela în care traia, împreuna cu cele doua nurori ale sale. Dar mergând ele pe cale, pentru a se întoarce în pamântul lui Iuda,
\par 8 Noemina a zis catre cele doua nurori ale sale: "Întoarce?i-va ?i va duce?i fiecare la casa mamei voastre; ?i sa faca Domnul mila cu voi, cum a?i facut ?i voi cu cei mor?i ?i cu mine!
\par 9 Domnul sa va ajute, ca sa va gasi?i adapost fiecare în casa barbatului sau!" Apoi le-a sarutat; iar ele, începând a se tângui ?i a plânge,
\par 10 Au zis: "Nu, ci ne vom întoarce împreuna la poporul tau!"
\par 11 Noemina însa a zis: "Întoarce?i-va, fiicele mele, de ce sa merge?i voi cu mine? Au doara mai am eu feciori în pântecele meu care sa va poata fi barba?i?
\par 12 Întoarce?i-va, fiicele mele, întoarceri-va, caci eu sunt prea batrâna ca sa ma mai marit. ?i chiar de v-a? spune ca tot mai am nadejde ?i chiar daca la noapte a? avea barbat ?i apoi a? na?te fii,
\par 13 A?i putea voi oare a?tepta pâna vor cre?te? Pute?i voi oare sa întârzia?i sa nu va marita?i? Nu, fiicele mele; mie îmi pare foarte rau de voi, caci mâna Domnului m-a apasat".
\par 14 Atunci ele din nou ?i-au ridicat glasul ?i au început a plânge. Apoi Orfa ?i-a luat ramas bun de la soacra sa ?i s-a întors la poporul sau, iar Rut a ramas cu ea.
\par 15 ?i a zis Noemina catre Rut: "Iata cumnata ta s-a întors la poporul sau ?i la dumnezeii sai. Întoarce-te ?i tu dupa cumnata ta!"
\par 16 Iar Rut a zis: "Nu ma sili sa te parasesc ?i sa ma duc de la tine; caci unde te vei duce tu, acolo voi merge ?i eu ?i unde vei trai tu, voi trai ?i eu; poporul tau va fi poporul meu ?i Dumnezeul tau va fi Dumnezeul meu;
\par 17 Unde vei muri tu, voi muri ?i eu ?i voi fi îngropata acolo. Orice-mi va face Domnul, numai moartea ma va despar?i de tine!"
\par 18 Vazând Noemina ca este a?a de hotarâta sa mearga cu ea, a încetat de a o mai îndemna sa se întoarca.
\par 19 ?i au plecat amândoua ?i au venit la Betleem. Iar daca au sosit aici, s-a zvonit de ele în toata cetatea ?i se zicea: "Oare aceasta este Noemina?"
\par 20 Iar ea zicea: "Nu ma mai numi?i Noemina, ci numi?i-ma Mara, pentru ca amaraciune mare mi-a trimis Atot?iitorul.
\par 21 Îndestulata am ie?it eu de aici, iar Domnul m-a întors cu mâinile goale. La ce sa ma mai numi?i Noemina, când Domnul m-a facut sa sufar ?i Atot?iitorul mi-a trimis necaz?"
\par 22 A?a s-a întors Noemina cu nora sa Rut moabiteanca, venind din ?esul Moabi?ilor ?i au intrat în Betleem pe la începutul seceri?ului orzului.

\chapter{2}

\par 1 Noemina avea ruda dupa barbatul sau pe un om foarte bogat, din neamul lui Elimelec, al carui nume era Booz.
\par 2 ?i a zis Rut moabiteanca Noeminei: "Ma duc în ?arina sa adun spice pe urma aceluia la care voi afla trecere". ?i aceasta a zis catre ea: "Du-te, fiica mea!"
\par 3 ?i plecând ea, s-a dus în ?arina sa adune spice de pe urma seceratorilor. ?i s-a întâmplat ca acea parte de ?arina era a lui Booz, din neamul lui Elimelec.
\par 4 ?i iata a venit Booz de la Betleem ?i a zis catre seceratori: "Domnul sa fie cu voi!" Iar ace?tia i-au raspuns: "Domnul sa te binecuvânteze!"
\par 5 Apoi a zis Booz catre sluga sa, care era pusa peste seceratori: Cine este aceasta femeie tânara?"
\par 6 Iar sluga care era pusa peste seceratori a raspuns ?i a zis: "Aceasta femeie tânara este moabiteanca aceea care a venit cu Noemina din ?ara Moabi?ilor.
\par 7 Ea m-a rugat: "Voi culege ?i voi aduna spice printre snopi pe urma seceratorilor. ?i se afla aici de azi diminea?a ?i acasa ?ade foarte pu?in".
\par 8 Atunci Booz a zis catre Rut: "Asculta, fiica mea, sa nu te duci sa strângi în alta ?arina ?i sa nu te departezi de aici, ci ramâi aici cu slujnicele mele;
\par 9 Sa ai înaintea ochilor tai ?arina unde secera ele ?i sa mergi dupa ele. Iata am poruncit slugilor mele sa nu te atinga. Când vei vrea sa bei, mergi ?i bea de unde beau slugile mele".
\par 10 ?i a cazut ea cu fala la pamânt ?i s-a închinat pâna la pamânt ?i a zis catre el: "Cu ce am dobândit eu mila înaintea ta de ma prime?ti, cu toate ca sunt straina?"
\par 11 Raspuns-a Booz ?i i-a zis: "Mie mi s-au spus toate cele ce ai facut tu cu soacra ta, dupa moartea barbatului tau, ca ?i-ai lasat pe tatal tau ?i pe mama ta ?i ?ara ta de na?tere ?i ai venit la poporul pe care nu l-ai cunoscut nici ieri, nici alaltaieri.
\par 12 Sa-?i plateasca Domnul pentru aceasta fapta a ta ?i sa ai plata deplina de la Domnul Dumnezeul lui Israel, la care ai venit, ca sa te adaposte?ti sub aripile Lui!"
\par 13 Iar ea a zis: "Domnul meu, fie sa am mila înaintea ochilor tai! Tu m-ai mângâiat ?i ai vorbit dupa inima roabei tale, de?i nu sunt macar ca una din slujnicele tale!"
\par 14 Atunci Booz a zis catre ea: "E vremea prânzului; vino de manânca pâine ?i-?i moaie bucatura în o?et". ?i a ?ezut lânga seceratori, iar el i-a dat pâine ?i ea a mâncat ?i sa saturat ?i i-a mai ?i ramas.
\par 15 Apoi s-a sculat ?i s-a apucat de strâns. Iar Booz a dat porunca slugilor sale, zicând: "Lasa?i-o sa adune ?i printre snopi ?i sa nu o stânjeni?i!
\par 16 Ba ?i din snopi sa arunca?i ?i sa lasa?i pentru ea; lasa?i-o sa adune ?i sa manânce; sa n-o ocarâ?i".
\par 17 ?i a?a a adunat ea în ?arina pâna seara ?i a batut cele adunate ?i i-a ie?it aproape o efa de orz.
\par 18 ?i luând aceasta, s-a dus în cetate ?i soacra sa a vazut ce adunase. Apoi a scos Rut din sin ?i i-a dat ceea ce-i ramasese dupa ce se saturase.
\par 19 ?i a zis soacra sa catre ea: "Unde ai adunat tu astazi ?i unde ai lucrat? Binecuvântat sa fie cel ce te-a primit! ?i Rut a spus soacrei sale la cine a lucrat ?i a zis: "Pe omul acela, la care am lucrat astazi, îl cheama Booz".
\par 20 ?i a zis Noemina catre nora sa: "Binecuvântat este el de Domnul, Ca re n-a lipsit de mila Sa nici pe cei vii, nici pe cei mor?i!" Apoi Noemina a adaugat: "Omul acela e aproape de noi, e una din rudeniile noastre".
\par 21 ?i a zis Rut moabiteanca soacrei sale: "El chiar mi-a zis: "Ramâi cu slujnicele mele pâna când vor ispravi seceri?ul meu".
\par 22 A zis Noemina catre nora sa Rut: "Este bine, fiica mea, ca ai sa umbli cu slujnicele lui ?i nu vei fi stânjenita, ca în alta ?arina".
\par 23 ?i a?a a ramas ea cu slujnicele lui Booz ?i a adunat spice pâna când s-a ispravit seceri?ul orzului ?i seceri?ul griului. Traia însa la soacra sa.

\chapter{3}

\par 1 Dupa aceea a zis catre ea Noemina, soacra sa: "Fiica mea, n-ar fi bine oare sa-?i cau?i un adapost, ca sa-?i fie bine?
\par 2 Iata Booz, cu ale carui slujnice ai fost, îmi este ruda ?i iata el în noaptea aceasta treiera orzul la arie.
\par 3 Spala-te ?i te unge, îmbraca-?i hainele tale cele bune ?i du-t?. la arie, dar nu te arata lui pâna nu va fi ispravit de mâncat ?i de baut.
\par 4 Iar dupa ce se va culca sa doarma, afla locul unde este culcat ?i fa-?i loc la picioarele lui ?i te culca, ?i el î?i va spune ce sa faci".
\par 5 Atunci Rut a zis: "Voi face tot ce mi-ai grait".
\par 6 Ducându-se deci la arie, a facut toate cum îi poruncise soacra sa.
\par 7 Iar Booz a mâncat, a baut, s-a veselit inima lui ?i s-a dus de s-a culcat lânga un stog. Iar ea a venit înceti?or, ?i-a facut loc la picioarele lui ?i s-a culcat acolo.
\par 8 Pe la miezul nop?ii însa a tresarit el ?i s-a ridicat; ?i iata la picioarele lui o femeie culcata.
\par 9 ?i a zis Booz catre ea: "Cine e?ti tu?" Iar ea a zis: "Eu sunt Rut, roaba ta. Întinde-?i aripa ta peste roaba ta, ca îmi e?ti ruda!"
\par 10 Zis-a Booz: "Binecuvântata e?ti tu de Domnul Dumnezeu, fiica mea! Aceasta de pe urma fapta buna a ta este înca ?i mai frumoasa decât celelalte, caci nu te-ai dus sa cau?i oameni tineri, saraci sau boga?i.
\par 11 Deci, fiica mea, nu te teme, î?i voi face tot ce ai zis, caci în toate par?ile poporului meu se ?tie ca e?ti femeie vrednica.
\par 12 Adevarat e ca î?i sunt ruda, dar mai ai o ruda înca ?i mai aproape decât mine.
\par 13 Ramâi noaptea aceasta aici ?i mâine, de va vrea acela sa te rascumpere, bine, sa te rascumpere; iar de nu va vrea sa te rascumpere el, te voi rascumpara eu; viu este Domnul! Dormi aici pâna mâine!"
\par 14 ?i a dormit ea la picioarele lui pâna diminea?a. Apoi s-a sculat înainte de a se fi putut ea cunoa?te unul pe altul. ?i a zis Booz: "Sa nu se ?tie ca a venit femeia la arie!"
\par 15 Iar catre ea a zis: "Dezbraca-?i haina ta cea de deasupra ?i ?ine-o". ?i ea a ?inut-o, iar el i-a masurat ?ase masuri de orz ?i i le-a pus pe umar ?i s-a dus în cetate.
\par 16 Atunci a venit Rut la soacra sa ?i i-a zis: "Ce e, fiica mea?" ?i ea i-a povestit tot ce i-a facut omul acela ?i a zis:
\par 17 "Aceste ?ase masuri de orz mi le-a dat el, zicându-mi: "Sa nu te duci la soacra ta cu mâinile goale!"
\par 18 Iar soacra a zis: "Ai rabdare fiica mea, pâna vei afla cum se va ispravi lucrul acesta; caci omul acela nu se va lini?ti pâna nu va ispravi chiar astazi lucrul acesta".

\chapter{4}

\par 1 În ziua aceea a ie?it Booz la poarta ceta?ii ?i a ?ezut acolo. ?i iata trecea pe acolo ruda de care graise Booz; ?i Booz i-a zis: "Vino încoace ?i ?ezi aici". ?i acela s-a dus ?i a ?ezut.
\par 2 ?i a luat Booz zece oameni dintre batrânii ceta?ii ?i a zis: "?ede?i aici!" ?i ei au ?ezut.
\par 3 Apoi a zis Booz catre ruda sa: "Noemina, întorcându-se din ?esul Moabi?ilor, vinde partea de ?arina, cuvenita fratelui nostru Elimelec; ?i eu m-am hotarât sa fac cunoscut auzului tau ?i sa-?i spun: Cumpar-o în fa?a celor ce ?ed aici ?i în fa?a batrânilor poporului meu.
\par 4 De vrei s-o cumperi, cumpar-o, iar de nu vrei s-o cumperi, spune-mi, ca sa ?tiu ?i eu. Caci afara de tine n-are cine s-o cumpere, iar dupa tine vin eu". ?i acela a zis: "O cumpar!"
\par 5 Raspuns-a Booz: "De cumperi ?arina de la Noemina, atunci trebuie sa cumperi ?i pe Rut moabiteanca, femeia celui mort, ?i trebuie sa o iei de so?ie, ca sa pastrezi numele celui mort în mo?tenirea lui".
\par 6 Iar ruda aceea a zis: "Nu pot sa o iau, ca sa nu-mi stric mo?tenirea mea; ia-o tu, caci eu nu pot sa o iau!"
\par 7 Înainte, la facerea unei cumparaturi sau a unui schimb, pentru întarirea lucrului, era în Israel obiceiul acesta: unul î?i descal?a sandaua sa ?i o da celuilalt, care primea dreptul de rudenie mai apropiata ?i aceasta era marturie în Israel.
\par 8 ?i a zis ruda aceea catre Booz: "Cumpar-o pentru tine!", ?i ?i-a descal?at sandaua sa ?i a dat-o acestuia.
\par 9 Iar Booz a zis catre batrâni ?i catre tot poporul: "Voi sunte?i martori astazi, ca eu am cumparat de la Noemina toate ale lui Elimelec ?i toate ale lui Chilion ?i toate ale lui Mahlon.
\par 10 De asemenea ?i pe Rut moabiteanca, femeia lui Mahlon, o iau de so?ie, ca sa pastrez numele celui mort în mo?tenirea lui ?i ca sa nu piara numele celui mort dintre fra?ii lui ?i din poarta locuin?ei lui; voi astazi sunte?i martori la aceasta".
\par 11 ?i tot poporul care era la poarta ?i batrânii au zis: "Suntem martori! Sa faca Domnul pe femeia care intra în casa ta ca pe Rahila ?i ca pe Lia, care amândoua au ridicat casa lui Israel. Sa câ?tigi avere în Efrata ?i numele tau sa fie marit în Betleem.
\par 12 Iar casa ta sa fie cum a fost casa lui Fares, pe care l-a nascut Tamara lui Iuda, ?i sa se slaveasca prin samân?a ce ?i-o va da Domnul din aceasta femeie tânara".
\par 13 ?i a luat Booz pe Rut ?i ea s-a facut so?ia lui. ?i intrând el la ea, Domnul i-a dat ei sarcina ?i a nascut un fiu.
\par 14 ?i ziceau femeile catre Noemina: "Binecuvântat este Domnul, ca nu te-a lasat fara mo?tenitor! Slavit sa fie numele lui Israel!
\par 15 Acesta î?i va fi bucurie ?i hranitor la batrâne?ile tale, caci l-a nascut nora ta, care este mai buna pentru tine decât ?apte fii".
\par 16 ?i a luat Noemina pe copilul acesta ?i l-a purtat în bra?ele sale ?i i-a fost doica.
\par 17 Iar vecinele i-au pus nume ?i au zis: "Noeminei i s-a nascut fiu ?i i s-a pus numele Obed". Acesta este parintele lui Iesei, tatal lui David.
\par 18 Iata acum spi?a neamului lui Fares: lui Fares i s-a nascut Esron:
\par 19 Lui Esron i s-a nascut Aram; lui Aram i s-a nascut Aminadab;
\par 20 Lui Aminadab i s-a nascut Naason; lui Naason i s-a nascut Salmon;
\par 21 Lui Salmon. i s-a nascut Booz; lui Booz i s-a nascut Obed; lui Obed i s-a nascut Iesei;
\par 22 Lui Iesei i s-a nascut David.


\end{document}