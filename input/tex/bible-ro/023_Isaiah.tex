\begin{document}

\title{Isaia}


\chapter{1}

\par 1 Vedenia lui Isaia, fiul lui Amos, pe care a văzut-o despre Iuda și Ierusalim, în vremea lui Ozia, Iotam, Ahaz și Iezechia, regii lui Iuda.
\par 2 Ascultă, cerule, și ia aminte, pământule, că Domnul grăiește: Hrănit-am feciori și i-am crescut, dar ei s-au răzvrătit împotriva Mea.
\par 3 Boul își cunoaște stăpânul și asinul ieslea domnului său, dar Israel nu Mă cunoaște; poporul Meu nu Mă pricepe.
\par 4 Vai ție neam păcătos, popor împovărat de nedreptate, soi rău, fii ai pieirii! Ei au părăsit pe Domnul, tăgăduit-au pe Sfântul lui Israel, întorsu-I-au spatele.
\par 5 Pe unde să mai fiți loviți voi, cei ce mereu vă răzvrătiți? Tot capul vă este numai răni și toată inima slăbănogită.
\par 6 Din creștet până în tălpile picioarelor nu-i nici un loc sănătos; totul este numai plăgi, vânătăi și răni pline de puroi, necurățate, nemuiate cu untdelemn și nelegate.
\par 7 Țara voastră este pustiită, cetățile voastre arse cu foc, țarinile voastre le mănâncă străinii înaintea ochilor voștri, totul este pustiit, ca la nimicirea Sodomei.
\par 8 Sionul ajuns-a ca o colibă într-o vie, ca o covercă într-o bostănărie, ca o cetate împresurată!
\par 9 Dacă Domnul Savaot nu ne-ar fi lăsat o rămășiră, am fi ajuns ca Sodoma și ne-am fi asemănat cu Gomora.
\par 10 Ascultați cuvântul Domnului, voi conducători ai Sodomei, luați aminte la învățătura Domnului, voi popor al Gomorei!
\par 11 Ce-mi folosește mulțimea jertfelor voastre?, zice Domnul. M-am săturat de arderile de tot cu berbeci și de grăsimea vițeilor grași și nu mai vreau sânge de tauri, de miei și de țapi!
\par 12 Când veneați să le aduceți, cine vi le ceruse? Nu mai călcați în curtea templului Meu!
\par 13 Nu mai aduceți daruri zadarnice! Tămâierile Îmi sunt dezgustătoare; lunile noi, zilele de odihnă și adunările de la sărbători nu le mai pot suferi. Însăși prăznuirea voastră e nelegiuire!
\par 14 Urăsc lunile noi și sărbătorile voastre sunt pentru Mine o povară. Ajunge!
\par 15 Când ridicați mâinile voastre către Mine, Eu Îmi întorc ochii aiurea, și când înmulțiți rugăciunile voastre, nu le ascult. Mâinile voastre sunt pline de sânge; spălați-vă, curățiți-vă!
\par 16 Nu mai faceți rău înaintea ochilor Mei. Încetați odată!
\par 17 Învățați să faceți binele, căutați dreptatea, ajutați pe cel apăsat, faceți dreptate orfanului, apărați pe văduvă!
\par 18 Veniți să ne judecăm, zice Domnul. De vor fi păcatele voastre cum e cârmâzul, ca zăpada le voi albi, și de vor fi ca purpura, ca lâna albă le voi face.
\par 19 De veți vrea și de Mă veți asculta, bunătățile pământului veți mânca.
\par 20 Iar de nu veți vrea și nu Mă veți asculta, atunci sabia vă va mânca, căci gura Domnului grăiește.
\par 21 Cum a ajuns ca o desfrânată cetatea cea credincioasă și plină de dreptate? Dreptatea locuia în ea, iar acum este plină de ucigași.
\par 22 Argintul tău s-a prefăcut în zgură și vinul tău este amestecat cu apă;
\par 23 Mai-marii tăi sunt răzvrătiți și părtași cu hoții; toți iubesc darurile și umblă după răsplată. Ei nu judecă orfanul, iar pricina văduvei nu ajunge până la ei.
\par 24 Pentru aceasta zice Domnul, Dumnezeul Savaot, puternicul lui Israel: Răzbuna-Mă-voi împotriva asupritorilor Mei și Mă voi întărâta cu răzbunare asupra vrăjmașilor Mei!
\par 25 Voi întoarce mâna Mea împotriva ta și te voi curăți de toată zgura ta, ca în cuptor.
\par 26 Voi întoarce judecătorii tăi să judece ca la început și sfetnicii tăi ca odinioară. După aceasta te vei putea numi iarăși cetate dreaptă, oraș credincios.
\par 27 Sionul va fi răscumpărat prin judecată și locuitorii săi care se vor întoarce la credință, prin dreptate.
\par 28 Domnul va zdrobi pe cei răzvrătiți, iar cei ce au părăsit pe Domnul vor fi nimiciți.
\par 29 Ei vor fi rușinați pentru dumbrăvile sfinte pe care le-au îndrăgit și se vor roși la față din pricina grădinilor pe care le-au ales.
\par 30 Vor fi ca un stejar ale cărui frunze cad și ca o grădină fără nici un strop de apă.
\par 31 Cel puternic va fi ca puzderiile de câlți și faptele lui ca o leasă de mărăcini. Și aceia și aceasta vor arde laolaltă și nimeni nu va putea să-i stingă.

\chapter{2}

\par 1 Vedenia pe care a văzut-o Isaia, fiul lui Amos, pentru Iuda și Ierusalim.
\par 2 Fi-va în vremurile cele de pe urmă, că muntele templului Domnului va fi întărit peste vârfurile munților și se va ridica pe deasupra dealurilor. Și toate popoarele vor curge într-acolo.
\par 3 Multe popoare vor veni și vor zice: "Veniți să ne suim în muntele Domnului, în casa Dumnezeului lui Iacov, ca El să ne învețe căile Sale și să mergem pe cărările Sale". Căci din Sion va ieși legea și cuvântul lui Dumnezeu din Ierusalim.
\par 4 El va judeca neamurile și la popoare fără de număr va da legile Sale. Preface-vor săbiile în fiare de pluguri și lăncile lor în cosoare. Nici un neam nu va mai ridica sabia împotriva altuia și nu vor mai învăța războiul.
\par 5 Voi, cei din casa lui Iacov, veniți să umblăm în lumina Domnului!
\par 6 Tu ai lepădat neamul Tău, casa lui Iacov. Ea este plină de vrăjitori și de magi ca Filistenii și ea face legământ cu cei de alt neam.
\par 7 Pământul ei este plin de aur și de argint, de comori fără număr, pământul ei este plin de cai și de căruțe fără sfârșit.
\par 8 Pământul ei este plin de idoli și locuitorii se închină la lucrul mâinilor lor, înaintea celor făcuți de degetele lor.
\par 9 Și omul va fi umilit și muritorul va fi pogorât și Tu nu-i vei ierta!
\par 10 Intrați în crăpăturile stâncilor și ascundeți-vă în pulbere, din pricina fricii de Dumnezeu, de strălucirea slavei Lui.
\par 11 Ochii celui mândru vor fi smeriți, mândria celor de rând va fi pogorâtă și numai Domnul în ziua aceea va fi ridicat în slăvi,
\par 12 Că Domnul Savaot va avea ziua Lui, se va ridica împotriva a tot ceea ce este mândru și semeț și-l va pogorî.
\par 13 Împotriva tuturor cedrilor Libanului și stejarilor celor înalți ai Vasanului.
\par 14 Împotriva tuturor munților înalți și colinelor celor mândre.
\par 15 Împotriva tuturor turnurilor ridicate sus și zidurilor întărite.
\par 16 Împotriva tuturor corăbiilor Tarsisului și lucrurilor de preț.
\par 17 Mândria omului va fi pogorâtă și semeția celor muritori va fi smerită; în ziua aceea numai Domnul va fi înalt;
\par 18 Și toți idolii vor pieri.
\par 19 Iar oamenii vor intra în scorburile stâncilor, în prăpăstiile și în crăpăturile pământului, de frica Domnului și de strălucirea slavei Lui, când va veni El ca să lovească pământul.
\par 20 În ziua aceea idolii de argint și de aur, pe care omul i-a făcut ca să li se închine, vor fi părăsiți ca să fie sălaș șobolanilor și liliecilor.
\par 21 Iar el va intra în crăpăturile stâncilor și în prăpăstiile munților, de frica Domnului și de strălucirea slavei Lui, când va veni El ca să lovească pământul.
\par 22 Nu mai nădăjduiți în omul cel muritor, în nările căruia nu este decât o suflare! Oare, ce putere are el?

\chapter{3}

\par 1 Iată, Domnul Dumnezeu Savaot va lua din Ierusalim și din Iuda orice sprijin și orice ajutor, orice hrană de pâine și orice sprijin de apă,
\par 2 Pe viteaz și pe omul de luptă, pe judecător și pe prooroc, pe prezicător și pe bătrân;
\par 3 Pe căpetenia peste cincizeci, pe sfătuitor, pe înțelept, pe fermecător și pe ghicitor.
\par 4 Voi pune băieți căpetenii peste ei și copiii vor domni peste aceia.
\par 5 În popor se vor strâmtora unul pe altul și fiecare va împila pe aproapele său; cel tânăr se va purta obraznic cu cel bătrân și cel de neam rău cu cel de neam bun.
\par 6 Atunci va alerga omul la fratele său, în casa tatălui său, și-i va zice: Tu mai ai o haină, fii căpetenie peste noi și să fie aceste dărâmături sub mâna ta.
\par 7 Iar acela cu jurământ se va lepăda, zicând: "Nu pot vindeca rănile poporului! N-am în casa mea nici pâine, nici haină, nu mă faceți conducător peste popor!"
\par 8 Ierusalimul va ajunge dărâmătură și Iuda este gata să cadă, căci limba lor și gândurile lor sunt împotriva Domnului și disprețuiesc privirea slavei Lui.
\par 9 Înfățișarea lor mărturisește împotriva lor, căci ei își vădesc păcatele lor ca Sodoma, în loc să fie ascundă. Vai de ei! Căci ei și-au făcut loruși rău...
\par 10 Fericit este omul drept, că el va mânca din rodul lucrurilor lui.
\par 11 Vai de cel rău, că răutatea este a lui și va fi judecat după faptele lui.
\par 12 Poporul meu este asuprit de niște copii, și femeile domnesc peste el. Poporul meu! Cei care te conduc te rătăcesc și te abat de la calea pe care tu mergi.
\par 13 Domnul Se ridică la judecată și stă ca să judece pe poporul Său.
\par 14 Domnul intră la judecată cu bătrânii și cârmuitorii poporului Său și zice: "Voi, voi ați pustiit via Mea și prada luată de la cei sărmani se află în casele voastre.
\par 15 Pentru ce ați zdrobit pe poporul Meu și ați sfărâmat fața celor sărmani?" zice Domnul Dumnezeu Savaot.
\par 16 Și mai zice Domnul: "Pentru că fiicele Sionului sunt atât de mândre și umblă cu capul pe sus și cu priviri obraznice, cu pași domoli, cu zăngănit de inele la picioarele lor,
\par 17 Domnul va pleșuvi creștetul capului fiicelor Sionului, Domnul va descoperi goliciunea lor".
\par 18 În ziua aceea va lua Domnul toate podoabele: inele, sori, lunițe,
\par 19 Cercei, brățări, văluri,
\par 20 Cununi, lănțișoare, cingătoare, miresme, talismane,
\par 21 Inele, verigi de nas,
\par 22 Veșminte de sărbătoare, mantii, șaluri, pungi,
\par 23 Oglinzi, pânzeturi subțiri, turbane și tunici.
\par 24 Atunci va fi în loc de miresme, putreziciune, și în loc de cingători, frânghie, în loc de cârlionți făcuți cu fierul, pleșuvie, în loc de veșmânt prețios, zdrențe, și în loc de frumusețe: pecete de robie.
\par 25 Locuitorii Sionului vor cădea de sabie și vitejii lui în războaie.
\par 26 Porțile fiicei Sionului vor scârțâi și se vor jeli și, jefuită, ea va ședea despuiată pe pământ.

\chapter{4}

\par 1 În ziua aceea șapte femei se vor certa pentru un singur om, zicând: "Noi vom mânca pâinea noastră și vom purta veșmintele noastre. Nu cerem altceva decât să purtăm numele tău. Ridică ocara noastră!"
\par 2 În ziua aceea se va arăta mlădița Domnului în podoabă și în slavă și roadele pământului în mărire și în cinste pentru aceia din Israel care vor fi scăpat.
\par 3 Rămășița Sionului și cei ce vor fi scăpat cu viață din Ierusalim se vor chema sfinți și oricine va fi înscris să trăiască în Ierusalim.
\par 4 Când Domnul va fi spălat necurățenia fiicelor Sionului și va fi șters fărădelegile din mijlocul lui prin duhul dreptății și al nimicirii,
\par 5 Domnul va veni pe Muntele Sionului și în adunările Sale ca un nor și ca un fum ziua, iar noaptea ca un foc strălucitor și ca o văpaie. Că peste tot locul slava Domnului va fi acoperământ:
\par 6 Fi-va în timpul zilei cort, care să-l apere de căldură și să-l adăpostească de vreme rea și de ploaie.

\chapter{5}

\par 1 Vreau să cânt pentru prietenul meu cântecul lui de dragoste pentru via lui. Prietenul meu avea o vie pe o coastă mănoasă.
\par 2 El a săpat-o, a curățit-o de pietre și a sădit-o cu viță de bun soi. Ridicat-a în mijlocul ei un turn, săpat-a și un teasc. Și avea nădejde că va face struguri, dar ea a făcut aguridă.
\par 3 Și acum voi, locuitori ai Ierusalimului și bărbați ai lui Iuda, fiți judecători intre mine și via mea.
\par 4 Ce se putea face pentru via mea și n-am făcut eu? Pentru ce atunci când nădăjduiam să-mi rodească struguri, mi-a rodit aguridă?
\par 5 Acum vă voi face să știți cum mă voi purta cu via mea: Strica-voi gardul ei și ea va fi pustiită, dărâma-voi zidul ei și va fi călcată în picioare.
\par 6 Și o voi pustii! Nu va mai fi tăiată, nici săpată și o vor năpădi spinii și bălăriile. De asemenea și norilor le voi da poruncă să nu-și mai verse ploaia peste ea.
\par 7 Dar via Domnului Savaot este casa lui Israel, iar oamenii din Iuda sunt sădirea Sa dragă. El nădăjduia ca acesta să fie un popor fără păcate, dar iată-l plin de sânge. Nădăjduit-a să-I rodească dreptate, dar iată: răzvrătire.
\par 8 Vai vouă care clădiți casă lângă casă și grămădiți țarini lângă țarini până nu mai rămâne nici un loc, ca să fiți numai voi stăpânitori în țară!
\par 9 Urechile mele au auzit de asemenea acest jurământ al Domnului Savaot: "Jur că aceste case multe, mari și frumoase, vor fi pustii și nimeni nu va mai locui în ele.
\par 10 Zece pogoane de vie vor rodi un bat, și un homer de sămânță, numai o efă".
\par 11 Vai de cei ce dis-de-dimineață aleargă după băuturi îmbătătoare; vai de cei ce până târziu seara se înfierbântă cu vin!
\par 12 Cei care doresc, la ospețele lor, chitară, harpă, tobă, flaut și vin ei nu iau în seamă faptele Domnului și nu văd lucrurile mâinilor Sale.
\par 13 Pentru aceasta poporul meu va fi dus în robie fără să bage de seamă, mai-marii săi vor fi doborâți de foame, iar gloata se va usca de sete!
\par 14 De aceea și iadul și-a mărit de două ori lăcomia lui, căscat-a gura sa peste măsură; acolo se vor coborî mărirea Sionului și gloatele sale, chiotele de veselie...
\par 15 Omul cel muritor va fi smerit și umilit și ochii celor mândri vor fi pogorâți.
\par 16 Dar Domnul Savaot este mare prin judecata Sa și Dumnezeul cel sfânt este sfânt prin dreptatea Sa.
\par 17 Oile vor paște în voie, iar străinii se vor hrăni în locurile mănoase, lăsate de cei bogați.
\par 18 Vai de cei ce își atrag pedeapsa ca și cu niște frânghii și plata păcatului ca și cu niște ștreanguri,
\par 19 Căci ei zic: "Grăbească Domnul să-Și facă lucrul Său curând, ca să vedem și să se plinească planul Sfântului lui Israel, ca să-l cunoaștem".
\par 20 Vai de cei ce zic răului bine și binelui rău; care numesc lumina întuneric și întunericul lumină; care socotesc amarul dulce și dulcele amar!
\par 21 Vai de cei care sunt înțelepți în ochii lor și pricepuți după gândurile lor!
\par 22 Vai de cei viteji la băut vin și meșteri la făcut băuturi îmbătătoare!
\par 23 Vai de cei ce dau dreptate celui nelegiuit pentru mită și lipsesc de dreptate pe cel drept!
\par 24 Pentru aceasta, după cum paiele sunt mistuite de foc și iarba uscată de flăcări, așa rădăcina lor va fi topită ca pleava și floarea lor va fi spulberată precum este cenușa, căci au călcat legea Domnului Savaot și au nesocotit cuvântul Sfântului lui Israel!
\par 25 De aceea, mânia Domnului s-a aprins împotriva poporului Său! El întinde mâna Sa spre el, îl lovește și munții se clatină. Cadavrele lor sunt ca gunoiul pe cale. Cu toate acestea mânia Lui nu se domolește și mâna Lui stă mereu întinsă
\par 26 Și va ridica steagul pentru un popor de departe și îl va chema de la capătul pământului. Iată-l că se zorește și vine.
\par 27 Nimeni din ai lui nu va obosi, nici va boli, nu va dormita, nici va adormi; nimeni nu-și va descinge brâul și nici cureaua încălțămintei lui nu se va rupe.
\par 28 Săgețile lor sunt ascuțite și arcurile lor gata să tragă. Copitele cailor sunt ca și cremenea cea tare, roțile căruțelor sunt ca o furtună.
\par 29 Strigătul, strigăt de leu, răcnesc ca puii de leu, mugesc și apucă prada, și nimeni nu roate s-o scape.
\par 30 În vremea aceea fi-va împotriva lui un vuiet ca vuietul mării. Toți vor arunca privirea spre pământ și iată: întuneric și strâmtorare; lumina se va întuneca întocmai ca o noapte, fără să se mai ivească zorile!

\chapter{6}

\par 1 În anul morții regelui Ozia, am văzut pe Domnul stând pe un scaun înalt și măreț și poalele hainelor Lui umpleau templul.
\par 2 Serafimi stăteau înaintea Lui, fiecare având câte șase aripi: cu două își acopereau fețele, cu două picioarele, iar cu două zburau
\par 3 Și strigau unul către altul, zicând: "Sfânt, sfânt, sfânt este Domnul Savaot, plin este tot pământul de slava Lui!"
\par 4 Din pricina acestor strigăte, porțile se zguduiau din țâțânele lor, iar templul s-a umplut de fum.
\par 5 Și am zis: "Vai mie, că sunt pierdut! Sunt om cu buze spurcate și locuiesc în mijlocul unui popor cu buze necurate. Și pe Domnul Savaot L-am văzut cu ochii mei!"
\par 6 Atunci unul dintre serafimi a zburat spre mine, având în mâna sa un cărbune, pe care îl luase cu cleștele de pe jertfelnic.
\par 7 Și l-a apropiat de gura mea și a zis: "Iată s-a atins de buzele tale și va șterge toate păcatele tale, și fărădelegile tale le va curăți".
\par 8 Și am auzit glasul Domnului care zicea: "Pe cine îl voi trimite și cine va merge pentru Noi?" Și am răspuns: "Iată-mă, trimite-mă pe mine!"
\par 9 Și El a zis: "Du-te și spune poporului acestuia: Cu auzul veți auzi și nu veți înțelege și, uitându-vă, vă veți uita, dar nu veți vedea.
\par 10 Că s-a învârtoșat inima poporului acestuia și cu urechile sale greu a auzit și ochii săi i-a închis, ca nu cumva să vadă cu ochii și cu urechile să audă și cu inima să înțeleagă și să se întoarcă la Mine și să-l vindec".
\par 11 Și am zis: "Până când, Doamne!" Atunci El mi-a răspuns: "Până când cetățile vor fi pustiite și vor rămâne fără locuitori, și casele fără oameni și pământul pustiu;
\par 12 Până când Domnul va izgoni pe oameni și pustiirea va fi mare în mijlocul acestei țări.
\par 13 Și dacă va rămâne încă unul din zece, și acela va fi hărăzit focului, ca și terebintul și stejarul, ale căror trunchiuri sunt trântite la pământ. Din butucul rămas va lăstări o mlădiță sfântă".

\chapter{7}

\par 1 Și a fost în zilele lui Ahaz, fiul lui Iotam, fiul lui Ozia, regele lui Iuda, că s-a suit Rețin, regele Siriei, împreună cu Pecah, fiul lui Remalia, regele lui Israel, ca să cuprindă Ierusalimul. Și n-a izbutit ca să-l cuprindă.
\par 2 Atunci a venit cineva să dea de știre casei lui David, zicând: "Armata Sirienilor a tăbărât în Efraim". Și inima regelui și a poporului tremura de spaimă în ziua aceea, precum tremură copacii pădurii din pricina vântului.
\par 3 Și a grăit Domnul către Isaia, zicând: "Ieși întru întâmpinarea lui Ahaz, tu și Șear-Iașub, fiul tău, la capătul canalului lacului celui de sus, pe drumul țarinii nălbitorului,
\par 4 Și îi vei zice: "Ia aminte, fii liniștit și nu te teme și inima ta să nu se slăbească din pricina acestor doi tăciuni care fumegă: de iuțimea mâniei lui Rețin și a lui Aram și a fiului lui Remalia
\par 5 De vreme ce Aram a hotărât pustiirea ta, împreună cu Efraim și cu fiul Remaliei, zicând:
\par 6 "Să ne suim în Iuda, să-l speriem, să ne facem stăpâni pe el și să punem rege peste el pe feciorul lui Tabeel".
\par 7 Așa zice Domnul Dumnezeu: "Aceasta nu va fi, nici nu se va împlini!
\par 8 Căci căpetenia Aramului este Damascul și mai mare peste Damasc este Rețin. - Mai sunt încă șaizeci și cinci de ani și Efraim va pieri din rândul popoarelor.
\par 9 Și capitala lui Efraim este Samaria și mai mare peste Samaria este feciorul lui Remalia. Dacă nu credeți, veți fi zdrobiți!"
\par 10 Și Isaia mai grăi către Ahaz:
\par 11 "Cere un semn de la Domnul Dumnezeul tău, în adâncurile iadului sau în înălțimile cele de sus".
\par 12 Și a spus Ahaz: "Nu voi cere și nu voi ispiti pe Domnul!"
\par 13 Și a zis Isaia: "Ascultați voi cei din casa lui David! Nu vă ajunge să obosiți pe oameni, de veniți să obosiți și pe Dumnezeul meu?
\par 14 Pentru aceasta Domnul meu vă va da un semn: Iată, Fecioara va lua în pântece și va naște fiu și vor chema numele lui Emanuel.
\par 15 El se va hrăni cu lapte și cu miere până în vremea când va ști să arunce răul și să aleagă binele.
\par 16 Că înainte ca fiul acesta să știe să dea la o parte răul și să aleagă binele, pământul de care îți este teamă, din pricina celor doi regi, va fi pustiit.
\par 17 Dar Domnul va aduce peste tine, peste poporul tău și peste casa tatălui tău, vremuri care n-au mai venit de când Efraim s-a desfăcut de Iuda; va aduce pe regele Asiriei.
\par 18 Și va fi că în ziua aceea Domnul va chema muștele care se află la capătul Nilului - fluviul Egiptului - și albinele din pământul Asiriei;
\par 19 Și vor veni și se vor așeza cu toate în văile cele prăpăstioase și în crăpăturile stâncilor și în toate tufișurile și în toate ținuturile nelocuite.
\par 20 În vremea aceea, va rade Domnul cu un brici, luat de împrumut de dincolo de Eufrat, pe regele Asiriei, capul, părul de pe trup și îi va smulge și barba.
\par 21 În vremea aceea, cine va hrăni o vacă și două oi
\par 22 Va avea belșug de unt din pricina mulțimii laptelui și cei ce vor fi rămas în țară se vor hrăni cu smântână și cu miere.
\par 23 În ziua aceea, unde era un loc de o mie de butuci pe preț de o mie de sicli, va fi plin de spini și de bălării.
\par 24 Acolo oamenii vor intra înarmați cu arcuri și cu săgeți, căci toată țara va fi plină de spini și de bălării.
\par 25 Și în toți munții care erau curățați cu săpăliga, tu nu te vei duce, de frica spinilor și a bălăriilor. Acolo se va da drumul boilor și oilor, ca să calce pământul.

\chapter{8}

\par 1 Și a zis Domnul către mine: "Ia o carte mare și scrie deasupra ei cu slove omenești: "Maher-Șalal-Haș-Baz" (grabnic-pradă-apropiat-jaf).
\par 2 Adu-Mi martori credincioși pe Urie preotul și pe Zaharia, fiul lui Ieberechia".
\par 3 Atunci m-am apropiat de proorociță și a luat în pântece și a născut un fiu. Și a zis Domnul către mine: "Pune-i numele Maher-Șalal-Haș-Baz".
\par 4 Căci înainte ca băiatul să zică: "tată și mamă!", toată bogăția Damascului și prada Samariei vor fi duse înaintea regelui Asiriei".
\par 5 Și mi-a mai grăit Domnul astfel:
\par 6 "Fiindcă poporul acesta a nesocotit apele Siloamului, care curg lin, și a tremurat înaintea lui Rețin, feciorul Remaliei,
\par 7 Iată acum că Domnul va aduce peste ei apele cele mari și furioase ale Eufratului: pe regele Asiriei și toată strălucirea lui. Ele vor trece peste toate zăgazurile și vor da afară peste malurile lui.
\par 8 Și se va revărsa în Iuda, îl va îneca și îl va umple de apă, va ajunge până la gât și cu revărsările lui întinse va acoperi toată țara.
\par 9 Cu noi este Dumnezeu, înțelegeți neamuri și vă plecați. Auziți până la marginile pământului, cei puternici plecați-vă. De vă veți întări, iarăși veți fi biruiți.
\par 10 Și orice sfat veți sfătui, îl va risipi Domnul, și cuvântul pe care îl veți grăi nu va rămâne întru voi, căci cu noi este Dumnezeu!"
\par 11 Așa îmi zicea mie Domnul, ținând peste mine mâna Sa cea tare și însuflându-mi să nu umblu pe căile acestui popor. Apoi mi-a zis:
\par 12 "Nu numiți uneltire tot ceea ce poporul acesta socotește uneltire, și nu vă temeți, nici nu vă înfricoșați de ceea ce se tem ei.
\par 13 Numai pe Domnul Savaot socotiți-L sfânt, de El să vă temeți și să vă înfricoșați.
\par 14 El va fi pentru voi piatră de încercare și stâncă de poticnire pentru cele două case ale lui Israel, cursă și laț pentru cei ce locuiesc în Ierusalim.
\par 15 Și mulți se vor poticni, vor cădea și se vor sfărâma, vor fi prinși în cursă și vor fi duși în robie!"
\par 16 Voi strânge laolaltă această mărturie și voi sigila această învățătură pentru ucenicii mei.
\par 17 Voi aștepta deci pe Domnul, Care își ascunde fața Sa de la casa lui Iacov și voi nădăjdui întru El.
\par 18 Iată eu și pruncii pe care mi i-a dat Dumnezeu spre semne și minuni în Israel, din partea Domnului Savaot, Care locuiește în Muntele Sionului.
\par 19 Și când vă vor zice: "Întrebați pe cei ce cheamă morții și ghicitorii care șoptesc și bolborosesc", să le răspundeți: "Nu se cuvine oare poporului să alerge la Dumnezeul său? Să întrebe oare pe morți pentru soarta celor vii?"
\par 20 Întrebați legea și descoperirea! De nu vă vor vorbi asemenea cuvântului acesta, atunci nu-i lumină în ei.
\par 21 Vor rătăci pe pământ flămânzi și cumplit apăsați, și în vremea foametei își vor arăta colții și vor huli pe regele lor și pe Dumnezeul lor.
\par 22 Apoi își vor întoarce privirea spre pământ și iată că acolo va fi strâmtorare, întuneric și scârbă și nevoie! Dar noaptea va fi alungată!

\chapter{9}

\par 1 Căci nu va mai fi întuneric pentru țara care era în nevoie. În vremurile de dedemult el a supus pământul Zabulonului și ținutul lui Neftali; în vremurile cele de pe urmă el va acoperi de slavă calea mării, celălalt țărm al Iordanului, Galileea neamurilor.
\par 2 Poporul care locuia întru întuneric va vedea lumină mare și voi cei ce locuiați în latura umbrei morții lumină va străluci peste voi.
\par 3 Tu vei înmulți poporul și vei spori bucuria lui. El se va veseli înaintea Ta, cum se bucură oamenii în timpul secerișului și se veselesc la împărțirea prăzilor.
\par 4 Căci jugul ce-l apasă, și toiagul ce-l lovește, și nuiaua ce-l asuprește, Tu le vei sfărâma, ca în zilele lui Madian.
\par 5 Încălțămintea cea zgomotoasă de om războinic și haina cea stropită de sânge vor fi aruncate în foc și mistuite în flăcări!
\par 6 Căci Prunc s-a născut nouă, un Fiu s-a dat nouă, a Cărui stăpânire e pe umărul Lui și se cheamă numele Lui: Înger de mare sfat, Sfetnic minunat, Dumnezeu tare, biruitor, Domn al păcii, Părinte al veacului ce va să fie.
\par 7 Și mare va fi stăpânirea Lui și pacea Lui nu va avea hotar. Va împărăți pe tronul și peste împărăția lui David, ca s-o întărească și s-o întemeieze prin judecată și prin dreptate, de acum și până-n veac. Râvna Domnului Savaot va face aceasta.
\par 8 Cuvânt va trimite Domnul peste Iacob, și el se va pogorî peste Israel.
\par 9 Ca să știe tot poporul, Efraim și locuitorii Samariei, care întru mândria lor și întru semeția inimii lor zic:
\par 10 "Cărămizile au căzut, să zidim cu piatră cioplită; smochinii au fost tăiați, să punem cedri în locul lor!"
\par 11 Ridica-va Domnul împotriva lui pe vrăjmașii lui Rețin și pe dușmanii lui îi va înarma:
\par 12 Pe Sirienii de la răsărit și pe Filistenii de la asfințit; și vor mânca aceștia pe Israel cu toată gura. Cu toate acestea mânia Lui nu se va potoli și mâna Lui tot întinsă va fi;
\par 13 Dar poporul nu se va întoarce la Cel care îl lovise și nu va căuta pe Domnul Savaot.
\par 14 Și Domnul va tăia din Israel, într-o singură zi, capul și coada, ramura de finic și trestia.
\par 15 Bătrânii și căpeteniile sunt capul; proorocul și învățătorul mincinos sunt coada.
\par 16 Căpeteniile acestui popor îl duc în rătăcire și cei conduși de ei vor pieri.
\par 17 Pentru aceasta, Domnul nu se bucură de cei tineri și de orfanii lui și de văduve nu-i este milă, fiindcă toți sunt nelegiuiți și răi și gura lor grăiește vorbe nesocotite. Pentru toate acestea, mânia Lui nu se va potoli și mâna Lui mereu întinsă va fi.
\par 18 Că fărădelegea arde ca focul, care mistuie spinii și bălăriile uscate; el arde tot mărăcinișul pădurii, iar fumul se înalță în rotocoale.
\par 19 Din pricina iuțimii mâniei Domnului Savaot, pământul va fi ca un jeratic, iar poporul va ajunge pradă focului. Nimeni nu va cruța pe vecinul său.
\par 20 Jefui-vor la dreapta și vor rămâne flămânzi; la stânga vor mânca și nu se vor sătura; fiecare va mânca din carnea aproapelui său:
\par 21 Manase pe Efraim, Efraim pe Manase, și amândoi sunt împotriva lui Iuda. Pe lângă toate acestea, mânia Lui nu se va potoli și brațul Lui mereu întins va fi.

\chapter{10}

\par 1 Vai de cei ce fac legi nedrepte și de cei ce scriu hotărâri silnice
\par 2 Ca să îndepărteze pe cei slabi de la judecată și să răpească dreptatea sărmanilor poporului Meu, ca să facă din văduve prada lor și să jefuiască pe cei orfani!
\par 3 Dar ce veți face voi în ziua pedepsirii și cum vă veți feri de furtuna ce vine de departe? Către cine veți fugi ca să fiți ajutați și unde veți lăsa comorile voastre?
\par 4 Fără mine vor merge cu frunțile plecate printre robi și vor cădea printre cei uciși și totuși mânia Lui nu se va potoli și mâna Lui mereu întinsă va fi.
\par 5 O, Asirie, varga mâniei Mele și toiagul urgiei Mele!
\par 6 Împotriva unui neam fără de lege o voi trimite și împotriva unui popor al urgiei Mele o voi îndrepta, ca să-l prade și să-l jefuiască și să-l calce ca pe tina ulițelor.
\par 7 Dar ea n-are aceeași judecată și inima ei nu simte la fel; să prade este în inima ei și să nimicească neamuri fără număr!
\par 8 Căci ea zice: "Oare mai-marii mei nu sunt ei laolaltă regi?
\par 9 Calno oare nu este ca și Carchemișul? Și Hamatul ca Arpadul și Samaria ca Damascul?"
\par 10 Cum a cuprins mâna Mea împărățiile idolilor, unde idolii erau mai numeroși decât în Ierusalim și în Samaria;
\par 11 Precum am făcut Samariei și idolilor ei, tot așa voi face și Ierusalimului și chipurilor lui!
\par 12 Și când Domnul va sfârși tot lucrul Lui în muntele Sionului și în Ierusalim, atunci va pedepsi pe regele Asiriei pentru graiul cel mândru din inima lui și pentru semeția privirilor lui,
\par 13 Că a zis: "Prin puterea mâinii mele am făcut aceasta și prin înțelepciunea mea; căci sunt priceput! Trecut-am peste granițele popoarelor, jefuit-am comorile lor și ca un atotputernic am dat jos de pe tron pe conducători.
\par 14 Mâna mea a apucat ca pe un cuib bogățiile popoarelor și, precum iei ouă părăsite, tot așa am cuprins eu tot pământul. și nimeni n-a scuturat aripile, n-a deschis ciocul și nici n-a scos vreun țipăt!
\par 15 Oare securea este ea măreață față de cel ce o ridică sau ferăstrăul se înalță împotriva celui ce-l mânuiește? Ca și cum varga ar da avânt celui care o ridică și toiagul ar însufleți brațul care îl duce!
\par 16 De aceea Domnul Dumnezeu Savaot va trimite prăpădul în această voinică oștire asiriană și sănătatea lor o vor mistui frigurile ca un pârjol.
\par 17 Și lumina lui Israel se va face foc și Sfântul său o flacără și va arde și va mistui spinii și bălăriile uscate, într-o singură zi!
\par 18 Și strălucirea pădurii lui și a livezii lui va fi nimicită de sus și până jos.
\par 19 Copacii rămași din pădurea lui vor fi așa de puțini la număr, încât și un copil va putea să-i numere.
\par 20 În vremea aceea rămășița lui Iuda și cei scăpați din casa lui Iacov nu se vor mai sprijini pe cel ce i-a lovit, ci se vor sprijini, cu credință, pe Dumnezeu, Sfântul lui Israel.
\par 21 O rămășiță din Iacov se va întoarce la Dumnezeul cel puternic.
\par 22 Chiar dacă poporul tău, Israele, țar fi ca nisipul mării, numai o rămășiță se va întoarce. Nimicirea este hotărâtă de dreptatea cea nemăsurată.
\par 23 Această poruncă de nimicire, Domnul Dumnezeu Savaot o va împlini în tot cuprinsul țării.
\par 24 Pentru aceasta, așa zice Domnul Dumnezeu Savaot: "Poporul Meu, care locuiește în Sion, nu te teme de Asiria, care te lovește cu toiagul pe care îl ridică asupra ta, ca altădată Egiptul.
\par 25 Dar, peste puțină vreme, urgia va înceta și mânia Mea îi va nimici".
\par 26 Domnul Savaot ridica-va asupra lor un bici, ca atunci când a bătut pe Madian la stânca Oreb și Își va întinde toiagul Său spre mare și-l va ridica precum l-a ridicat asupra Egiptenilor.
\par 27 În vremea aceea va ridica povara de pe umerii tăi și jugul de pe grumajii tăi.
\par 28 Vine din latura Rimonului și ajunge la Aiat, trece la Migron, la Micmas lasă poverile sale de drum.
\par 29 Ei au trecut pasul și noaptea au rămas la Gheba. Rama este înspăimântată, Ghibeea lui Saul a luat-o la fugă.
\par 30 Urlă fiică a lui Galim, ia aminte Laișa, răspunde-i tu, Anatot.
\par 31 Madmena se împrăștie, locuitorii din Ghebim au luat-o la fugă.
\par 32 O zi va sta la Nob, amenință cu mâna muntele Sionului și colina Ierusalimului!
\par 33 Iată că Domnul Dumnezeu Savaot frânge crengile dintr-o lovitură năprasnică: vârfurile sunt tăiate și crengile de sus date jos.
\par 34 Desișul pădurii cade sub lovituri de unelte de fier, cedrii Libanului se prăbușesc la pământ.

\chapter{11}

\par 1 O Mlădiță va ieși din tulpina lui Iesei și un Lăstar din rădăcinile lui va da.
\par 2 Și Se va odihni peste El Duhul lui Dumnezeu, duhul înțelepciunii și al înțelegerii, duhul sfatului și al tăriei, duhul cunoștinței și al bunei-credințe.
\par 3 Și-L va umple pe El duhul temerii de Dumnezeu. Și va judeca nu după înfățișarea cea din afară și nici nu va da hotărârea Sa după cele ce se zvonesc,
\par 4 Ci va judeca pe cei săraci întru dreptate și după lege va mustra pe sărmanii din țară. Pe cel aprig îl va bate cu toiagul gurii Lui și cu suflarea buzelor Lui va omorî pe cel fără de lege.
\par 5 Dreptatea va fi ca o cingătoare pentru rărunchii Lui și credincioșia ca un brâu pentru coapsele Lui.
\par 6 Atunci lupul va locui laolaltă cu mielul și leopardul se va culca lângă căprioară; și vițelul și puiul de leu vor mânca împreună și un copil îi va paște.
\par 7 Juninca se va duce la păscut împreună cu ursoaica și puii lor vor sălășlui la un loc, iar leul ca și boul va mânca paie;
\par 8 Pruncul de țâță se va juca lângă culcușul viperei și în vizuina șarpelui otrăvitor copilul abia înțărcat își va întinde mâna.
\par 9 Nu va fi nici o nenorocire și nici un prăpăd în tot muntele Meu cel sfânt! Că tot pământul este plin de cunoștința și de temerea de Dumnezeu, precum marea este umplută de ape!
\par 10 Și în vremea aceea, Mlădița cea din rădăcina lui Iesei, va fi ca un steag pentru popoare; pe Ea o vor căuta neamurile și sălașul Ei va fi plin de slavă.
\par 11 în ziua aceea, Domnul va ridica din nou mâna Sa ca să răscumpere rămășița poporului Său dintre robii din Asiria și din Egipt, din Patros, din Etiopia, din Elam, din Babilon, din Hamat și din insulele mării.
\par 12 El va ridica steag pentru neamuri și va aduna pe cei risipiți ai lui Israel și va strânge la un loc pe cei împrăștiați ai lui Iuda din cele patru colțuri ale pământului.
\par 13 Atunci pizma lui Efraim va înceta și dușmanii lui Iuda vor fi zdrobiți. Efraim nu va mai pizmui pe Iuda și Iuda nu va mai fi vrăjmașul lui Efraim.
\par 14 Ci se vor avânta în latura Filistenilor la apus și vor jefui împreună pe feciorii răsăritului; asupra Edomului și Moabului își vor întinde mâna lor, și copiii lui Amon vor asculta de ei.
\par 15 Domnul va seca limba de mare a Egiptului și mâna Lui va amenința groaznic Eufratul, și-l va împărți în șapte râuri și se va putea trece cu piciorul.
\par 16 Atunci se va croi un drum pentru rămășița din poporul Său, pentru cei scăpați din robia Asiriei, precum s-a întâmplat altădată cu Israel, în ziua când el a ieșit din Egipt.

\chapter{12}

\par 1 Și tu vei zice în ziua aceea: Lăuda-Te-voi, Doamne, că deși pornit împotriva mea, mânia Ta s-a întors de la mine și m-ai miluit.
\par 2 Iată Dumnezeul cel tare al mântuirii mele; nădăjdui-voi întru El și nu mă voi înfricoșa, că izvorul puterii mele și cântarea mea de laudă este Domnul Dumnezeu și izbăvirea mea.
\par 3 Veți scoate apa cu veselie din izvoarele mântuirii
\par 4 Și veți zice în ziua aceea: "Lăudați pe Domnul, chemați numele Lui, vestiți printre neamuri lucrările Lui, dați de știre că înalt este numele Lui!
\par 5 Cântați în strune pe Domnul, căci El a făcut fapte strălucite! Să știe aceasta tot pământul!
\par 6 Săltați și vă veseliți locuitori ai Sionului, căci mare este în mijlocul vostru Sfântul lui Israel!"

\chapter{13}

\par 1 Proorocia despre Babilon pe care a văzut-o Isaia, fiul lui Amos.
\par 2 Pe un munte pleșuv înălțați steag, strigați către ei, faceți semn cu mâna, ca să intre pe poarta asupritorilor.
\par 3 "Eu am poruncit sfintei Mele oștiri, zice Domnul, chemat-am pe vitejii mâniei Mele, pe cei ce se veselesc de slava Mea".
\par 4 Ascultați acest zgomot surd în munți, vuiet al unui neam fără de număr; auziți această zarvă de împărății, de neamuri adunate; Domnul Savaot cercetează oștirea gata de luptă.
\par 5 Ele vin dintr-un ținut depărtat, de la capătul cerului; vine Domnul și uneltele mâniei Lui, ca să nimicească tot pământul.
\par 6 Strigați, că aproape este ziua Domnului, ea vine ca o pustiire de la Cel Atotputernic.
\par 7 Drept aceea, toate brațele vor fi neputincioase și inima omului se va topi de frică.
\par 8 Vor fi cuprinși de spaimă, vor vedea năluci și durerile îi vor cuprinde; zvârcolise-vor în dureri ca femeia gata să nască. Se vor privi unul pe altul cu groază, iar fețele lor vor fi roșii ca flacăra.
\par 9 Iată ziua Domnului, ea vine aprigă, mânioasă și întărâtată la mânie ca să pustiiască pământul și să stârpească pe păcătoși de pe el.
\par 10 Luceferii de pe cer și grămezile de stele nu-și vor mai da lumina lor; soarele se va întuneca în răsăritul lui și luna nu va mai străluci.
\par 11 Atunci voi pedepsi lumea pentru fărădelegile ei și pe cei nelegiuiți pentru păcatele lor. Voi smeri mândria celor îngâmfați și obrăznicia celor cruzi o voi arunca la pământ.
\par 12 Voi face ca oamenii să fie mai rari decât aurul cel mai scump, mai căutați decât aurul de Ofir.
\par 13 Pentru aceasta voi prăbuși cerurile; și pământul se va clătina din locul lui, din pricina mâniei Domnului Savaot, în ziua iuțimii mâniei Lui.
\par 14 Atunci, ca o gazelă sperioasă și ca o turmă pe care nimeni nu poate s-o adune, fiecare se va întoarce la poporul său și fiecare va fugi în pământul său.
\par 15 Oricine va fi aflat va fi străpuns și oricare va fi prins va cădea de sabie.
\par 16 Copiii lor vor fi zdrobiți înaintea ochilor lor, casele lor vor fi jefuite și femeile lor necinstite.
\par 17 Iată că Eu ridic asupra lor pe Mezi, care nu pun preț pe argint și care nu se lăcomesc pentru aur.
\par 18 Arcurile oamenilor de luptă vor doborî pe cei tineri. De roada pântecelui nu se vor milostivi, și pentru copii ochii lor nu vor simți nici o milă.
\par 19 Atunci Babilonul, podoaba împărățiilor, cununa mândriei Caldeilor, fi-va ca Sodoma și ca Gomora, pe care Dumnezeu le-a nimicit.
\par 20 Nu va mai fi locuit în veci și din neam în neam. Arabii nu vor mai înfige acolo corturi și nici ciobanii nu-și vor mai face târle în latura aceea.
\par 21 Ci numai animale sălbatice se vor sălășlui într-însul, și bufnițele vor locui prin case, struții își vor face cuiburi acolo și oameni cu chip de țap vor juca în acel loc.
\par 22 Șacalii vor urla în palatele lor și lupii în casele lor de petrecere. Vremea este aproape să sosească și zilele ei nu vor zăbovi!

\chapter{14}

\par 1 Dar Domnul Se va milostivi de Iacov și va alege încă o dată pe Israel și îl va statornici în pământul lui. Cei străini se vor alătura lor și se vor uni cu casa lui Iacov.
\par 2 Pe popoare le va lua și le va duce la ei, iar casa lui Israel le va avea în pământul Domnului ca robi și roabe. Ei vor duce în robie pe cei care i-au dus în robie și vor stăpâni peste apăsătorii lor.
\par 3 Iar în ziua în care Domnul te va odihni de osteneli, de chinuri și de amarnica ta robie în care ai fost,
\par 4 Tu vei cânta cântecul acesta de ocară împotriva împăratului Babilonului și vei zice: "Cum s-a sfârșit cu asupritorul și cum a încetat chinul nostru!
\par 5 Domnul a zdrobit toiagul celor fără de lege, sceptrul răilor apăsători!
\par 6 Iată pe cel care lovea popoarele fără încetare cu mânia lui și care în întărâtarea lui punea neamurile sub stăpânirea lui, supunându-le fără cruțare!
\par 7 Tot pământul este în pace și se odihnește; toți izbucnesc în cântece de veselie.
\par 8 Până și chiparoșii împreună cu cedrii cei din Liban se bucură de căderea ta: "De când tu te-ai prăbușit, nimeni nu se mai suie la noi ca să ne doboare!"
\par 9 Șeolul (iadul) se mișcă în adâncurile sale, ca să iasă întru întâmpinarea ta. Pentru tine el deșteaptă umbrele, pe toți stăpânitorii pământului; el ridică de pe jilțurile lor pe toți împărații pământului.
\par 10 Toți iau cuvântul și îți zic: "Și tu ești slab ca noi și te asemeni nouă".
\par 11 În iad s-a pogorât mărirea ta în cântecul harfelor tale. Sub tine se vor așterne viermii și viermii vor fi acoperământul tău.
\par 12 Cum ai căzut tu din ceruri, stea strălucitoare, fecior al dimineții! Cum ai fost aruncat la pământ, tu, biruitor de neamuri!
\par 13 Tu care ziceai în cugetul tău: "Ridica-mă-voi în ceruri și mai presus de stelele Dumnezeului celui puternic voi așeza jilțul meu! În muntele cel sfânt voi pune sălașul meu, în fundurile laturei celei de miazănoapte.
\par 14 Sui-mă-voi deasupra norilor și asemenea cu Cel Preaînalt voi fi".
\par 15 Și acum, tu te pogori în iad, în cele mai de jos ale adâncului!
\par 16 Cei ce te văd își întorc privirea în spre tine și se uită cu luare aminte zicând: "Oare acesta este omul de care tremura pământul și împărățiile se cutremurau?"
\par 17 Oare acesta este cel ce prefăcea lumea în pustiu și cetățile le dobora și nu da drumul robilor săi?"
\par 18 Toți împărații popoarelor se odihnesc cu cinste, fiecare în locașul său.
\par 19 Și numai tu ești azvârlit departe de mormântul tău, ca o ramură fără de preț, ca rămășițele celor care au fost uciși cu lovituri de sabie, zvârliți pe pietre de mormânt, ca un hoit călcat în picioare.
\par 20 Tu nu te vei pogorî în mormânt, căci tu ai pustiit pământul tău și pe poporul tău l-ai ucis! Niciodată nu se va mai vorbi despre neamul celor răi!
\par 21 Pregătiți măcelul feciorilor, din pricina fărădelegilor părinților lor, ca nu cumva să se ridice și să cucerească pământul și să umple de ruine fața a tot pământul.
\par 22 "Eu Mă voi scula împotriva lor, zice Domnul Savaot, și voi nimici numele Babilonului și pe cei care au mai rămas: și mugurii și mlădițele, zice Domnul.
\par 23 Acolo va stăpâni ariciul și va fi mlaștină și îl voi mătura cu mătura nimicirii", zice Domnul Savaot.
\par 24 Juratu-S-a Domnul Savaot și a zis: "Cum am hotărât, așa va fi, precum M-am sfătuit, așa se va întâmpla!
\par 25 Sfărâma-voi Asiria în pământul Meu și o voi călca în picioare pe munții Mei. Și robii vor fi liberați de jugul lor și umerii de povara lor".
\par 26 Iată hotărârea pentru tot pământul, iată mâna întinsă peste toate neamurile!
\par 27 Dacă Domnul Savaot a hotărât, cine îl va putea împiedica? Și dacă mâna Lui stă întinsă, cine o va întoarce la loc?
\par 28 În anul morții lui Ahaz, fost-a această proorocie:
\par 29 "Nu te veseli, toată țara Filistenilor, fiindcă a fost zdrobit toiagul care te lovea. Căci din rădăcina șarpelui va ieși o viperă și din urmașii lui un șarpe zburător.
\par 30 Cei sărmani vor paște pe pășunile Mele, iar cei săraci vor fi fără de grijă. Voi face să moară de foame neamul tău, iar pe cei ce vor rămâne din tine îi voi ucide.
\par 31 Tu, poartă, urlă! Și tu, cetate, țipă! Cutremură-te tu, țară a Filistenilor, toată! Că din partea de miazănoapte vine un fum și șirurile vrăjmașilor sunt strânse".
\par 32 Și ce se va răspunde în ziua aceea celor trimiși dintre popoare? Că "Domnul a întemeiat Sionul, limanul celor îndurerați din poporul Lui".

\chapter{15}

\par 1 Prins fără de veste în vreme de noapte Ar-Moabul a fost pustiit. Luat fără de veste noaptea, Chir-Moabul a fost nimicit.
\par 2 Poporul se urcă la templul de la Dibon, la locurile înalte, ca să plângă pe Nebo și la Medeba, Moabul se tânguiește. Toate capetele sunt rase, toate bărbile sunt tăiate.
\par 3 Pe ulițele lui toți ies îmbrăcați în sac; pe acoperișuri, în piețe, toți se jelesc și izbucnesc în plâns.
\par 4 Heșbonul și Eleale bocesc, iar glasul lor până la Iahaț se aude. Chiar și războinicii Moabului se vaită și sufletul le este cuprins de groază.
\par 5 Din adâncul inimii, Moabul strigă; fugarii lui sosesc până la Țoar, până la Eglat-Șelișia. Coasta Luhitului toți o urcă plângând; pe drumul de la Horonaim scot strigăte de deznădejde,
\par 6 Că apele de la Nimrim au secat, iarba s-a uscat, iarba verde nu mai este, verdeața a pierit.
\par 7 De aceea ei își fac provizii și duc bunurile lor dincolo de pârâul Sălciilor.
\par 8 Țipetele au făcut înconjurul Moabului, vaietele sale au ajuns până la Eglaim, jeluirea lui până la Beer-Elim, că apele Dimonului sunt pline de sânge!
\par 9 Asupra Dimonului voi trimite iarăși nenorociri; pentru cei ce au scăpat din Moab cât și pentru cei rămași în țară voi trimite lei.

\chapter{16}

\par 1 "Trimiteți miei stăpânitorului țării, trimiteți-i din Petra, prin pustiu, la muntele fiicei Sionului".
\par 2 Ca o pasăre care fuge sperioasă din cuibul ei, ca un cuib risipit, așa sunt fiicele Moabului la vadurile Arnonului.
\par 3 "Dă un sfat, dă o hotărâre, întinde umbra ta, ca noaptea, în miezul zilei, ascunde pe cei duși în robie, nu descoperi pe cei fugari!
\par 4 Adăpostește la tine pe toți robii Moabului, să le fii acoperitor în fața pustiitorului, până când năvala va fi trecut, prăpădul va lua sfârșit și vrăjmașul va lăsa țara în pace.
\par 5 Jilțul lui se va întări prin milostivire și pe el va ședea de-a pururi în cortul lui David un judecător apărător al pricinei drepte și râvnitor dreptății.
\par 6 Am auzit de semeția Moabului, că foarte mândru este; am auzit de obrăznicia, de mândria, de trufia și de graiurile lui deșarte".
\par 7 Pentru aceasta, Moabiții se tânguiesc pentru Moab, ton împreună se bocesc! Ei suspină pentru turtele de struguri de la Chir-Hareset, înmărmuriți.
\par 8 Câmpiile Heșbonului au sărăcit, asemenea și via de la Sibma; stăpânitorul popoarelor a distrus cele mai bune vițe ale ei, care se întinseseră până la Iazer și acoperiseră pustiul; lăstarii lor se întinseseră și trecuseră marea.
\par 9 Pentru aceasta plâng împreună cu Iazerul pentru via din Sibma. Vă ud cu lacrimile mele pe voi, Heșbon și Eleale, că nu se mai aud acolo strigătele vesele din timpul secerișului și al culesului viilor.
\par 10 Nici bucurie, nici veselie prin grădini, iar prin vii nici cântece, nici chiote! Nimeni nu mai dă vinul la teasc, strigătul călcătorului a încetat.
\par 11 Pentru aceasta lăuntrul meu se zbuciumă pentru Moab ca o harfă și inima mea pentru Chir-Hares.
\par 12 Iată că Moabul este văzut urcând obosit pe locurile înalte, intră în templul său să se roage, dar nu dobândește nimic.
\par 13 Aceasta este proorocia pe care a grăit-o Domnul odinioară pentru Moab.
\par 14 Iar acum Domnul a zis așa: "Peste trei ani, socotiți ca anii unui simbriaș, mărirea Moabului se va micșora cu vuiet mare și va rămâne mică și slabă, fără nici o putere".

\chapter{17}

\par 1 Proorocie împotriva Damascului: "Damascul este scos din numărul cetăților și a rămas o grămadă de ruine.
\par 2 Cetățile Aroerului sunt pustiite pentru vecie; ele sunt bune de păscut turmele, care se culcă acolo și nimeni nu le gonește.
\par 3 Nici cetate întărită pentru Efraim și nici împărăție la Damasc. Tot așa va fi cu rămășița Siriei și cu mărirea ei, precum a fost cu fiii lui Israel, zice Domnul Savaot.
\par 4 Și va fi în ziua aceea că mărirea lui Iacob se va împuțina și acest trup gras se va usca.
\par 5 Va fi atunci ca pe urma secerătorului ce seceră holda, când mâna lui adună spice și cum e când oamenii adună spice în valea Refaim;
\par 6 Vor rămâne pe urmă câteva roade, ca la scuturatul măslinului, două-trei măsline pe vârf, patru-cinci pe ramuri", zice Domnul Dumnezeul lui Israel.
\par 7 În ziua aceea, omul își va întoarce privirea către Ziditorul său și ochii lui către Sfântul lui Israel se vor întoarce.
\par 8 Și nu va mai privi către jertfelnice, lucrurile mâinilor lui, și nu se va mai uita la făptura degetelor lui, la Astartele și statuile ridicate soarelui.
\par 9 În vremea aceea, cetățile sale întărite vor fi părăsite ca ale Amoreilor și Heveilor, lăsate înaintea fiilor lui Israel și vor rămâne pustii.
\par 10 Căci tu ai uitat pe Dumnezeul izbăvirii tale și de Stânca scăpării tale nu ți-ai adus aminte. Iată pentru ce tu întemeiezi grădini lui Adonis și acolo sădești vie pentru un dumnezeu străin.
\par 11 În ziua când o sădești, tu vezi că se ridică și a doua zi are flori; dar de culesul roadelor nu te bucuri în ziua nenorocirii și durerea este fără leac.
\par 12 Ah! Această zarvă de popoare este ca vuietul de ape multe, acest zgomot de neamuri este ca zgomotul de ape mari;
\par 13 El le amenință și ele fug departe, gonite ca pleava pe care vânturătorii o vântură în vânt și ca vârtejul de pulbere în vreme de furtună.
\par 14 În vremea serii, atunci e ceasul spaimei, iar mai înainte de a se face ziuă, ei nu mai sunt. Iată partea, partea jefuitorilor noștri și soarta celor ce ne-au prădat pe noi.

\chapter{18}

\par 1 Vai ție, țară în care se aude zăngănit de arme și care ești dincolo de fluviile Etiopiei!
\par 2 Tu, care trimiți soli pe Nil în bărci de papură pe întinsele ape. Mergeți voi, soli iuți, către un neam de statură înaltă și cu pielea lucie, departe către un popor de temut, popor plin de putere și viteaz, a cărui țară este străbătută de fluvii.
\par 3 Voi, toți locuitori ai lumii și care stăpâniți pământul! Când veți vedea înălțându-se steagul deasupra munților, priviți! Și când va suna trâmbița, ascultați!
\par 4 Că așa zice Domnul către mine: "Privesc liniștit din locașul Meu, întocmai ca adierea fierbinte a verii la lumina soarelui, ca norul de rouă în zăduful secerișului.
\par 5 Căci înainte de cules, după ce florile s-au scuturat și mugurii s-au prefăcut în ciorchini copți, vițele vor fi tăiate cu cosoarele, ramurile vor fi luate, smulse vor fi.
\par 6 Toate vor fi lăsate vulturilor de munte și fiarelor pământului; vulturii vor petrece acolo vara, iar fiarele câmpului iarna.
\par 7 În vremea aceea, se vor aduce daruri de la neamul de statură înaltă și cu pielea lucie, de la poporul de temut cel de departe, de la poporul cel plin de putere și viteaz, a cărui țară este străbătută de fluvii, către locul numelui Domnului Savaot, muntele Sionului".

\chapter{19}

\par 1 Iată Domnul vine pe nor ușor și ajunge în Egipt. Idolii Egiptului tremură înaintea feței Lui și inima Egiptenilor se topește în ei.
\par 2 Voi întărâta pe Egipteni unii împotriva altora și se vor război frate cu frate și prieten cu prieten, cetate cu cetate, împărăție cu împărăție.
\par 3 Egiptul își va pierde mintea și voi încurca istețimea lui și vor merge ei să întrebe pe idoli și pe vrăjitori, pe fermecători și pe ghicitori.
\par 4 Și voi da Egiptul în mâna unui stăpânitor crud și un împărat puternic îl va stăpâni, zice Domnul Dumnezeu Savaot.
\par 5 Apele mării se vor sfârși și fluviul va seca și se va usca de tot.
\par 6 Canalele se vor preface în ape stătătoare. Râurile Egiptului vor scădea și se vor usca, papura și trestia se vor veșteji.
\par 7 Lunca Nilului și toată verdeața de pe malurile lui se vor usca, vor cădea și nu vor mai fi!
\par 8 Pescarii vor suspina și se vor tângui; toți cei care aruncă undița în Nil, cei care aruncă năvodul pe fața apelor, vor fi deznădăjduiți.
\par 9 Cei care lucrează inul vor fi nedumeriți și pieptănătoarele și țesătorii vor fi în mare încurcătură.
\par 10 Țesătorii vor fi tulburați și toți lucrătorii, în întristare mare.
\par 11 Mai-marii Țoanului au ajuns nebuni, sfătuitorii cei înțelepți ai lui Faraon dau sfaturi fără de minte! Cum îndrăzniți voi să ziceți lui Faraon: "Eu sunt ucenicul celor înțelepți, al regilor de altădată?"
\par 12 Unde sunt oare înțelepții tăi? Să te vestească și să-ți dea de știre ceea ce a hotărât Domnul Savaot împotriva Egiptului.
\par 13 Mai-marii Țoanului au ajuns nebuni, mai-marii Nofului și-au pierdut mintea și căpeteniile semințiilor duc Egiptul pe căi greșite.
\par 14 Domnul a aruncat peste ei un duh de zăpăceală; în orice faptă a lor ei rătăcesc Egiptul și nu-și dau seama, cum nu-și dă seama bețivul când varsă.
\par 15 Și nu va fi nici un lucru în Egipt cu rostul lui: nici cap, nici coadă, nici început, nici sfârșit.
\par 16 În ziua aceea, Egiptenii vor fi ca femeile fricoase și tremurătoare, din pricina amenințării mâinii Domnului Savaot pe care o va ridica peste ei.
\par 17 Atunci pământul lui Iuda va fi pentru Egipt înfricoșare mare; ori de câte ori i se va aminti numele; Egiptul va tremura, din pricina hotărârii luate împotriva lui de Domnul Savaot.
\par 18 În vremea aceea, vor fi numai cinci cetăți în pământul Egiptului care vor grăi limba Canaanului și vor jura în numele Domnului Savaot; una se va numi "Cetatea Soarelui".
\par 19 În ziua aceea, va fi un jertfelnic în mijlocul pământului Egiptului și un stâlp de pomenire la hotarul lui, pentru Domnul.
\par 20 Acesta va fi un semn și o mărturie pentru Domnul Savaot în pământul Egiptului. Când vor striga către Domnul în strâmtorările lor, atunci El le va trimite un mântuitor și un răzbunător oare-i va mântui.
\par 21 Domnul se va face știut în Egipt și Egiptenii vor cunoaște pe Domnul în ziua aceea. Și vor aduce arderi de tot și prinoase și vor face făgăduințe Domnului și le vor împlini.
\par 22 Și Domnul va bate Egiptul, îl va lovi și apoi îl va vindeca. Și ei se vor întoarce la Domnul și El se va îndupleca și îi va tămădui.
\par 23 În vremea aceea, va fi un drum din Egipt în Asiria și Asiria va merge în Egipt și Egiptul în Asiria și Egiptenii și Asirienii vor sluji pe Domnul.
\par 24 În ziua aceea, Israel va fi al treilea în legământul cu Egiptul și cu Asiria, ca o binecuvântare în mijlocul pământului,
\par 25 Binecuvântare a Domnului Savaot, Care zice: "Binecuvântat să fie poporul Meu, Egiptul și Asiria, lucrul mâinilor Mele și Israel, moștenirea Mea!"

\chapter{20}

\par 1 În anul în care Tartan a venit la Așdod, trimis de Sargon, regele Asiriei, și a împresurat Așdodul și l-a cuprins,
\par 2 În vremea aceea a grăit Domnul prin gura lui Isaia, fiul lui Amos, zicând: "Du-te și dezbracă sacul de pe coapsele tale și descalță încălțămintele tale". Și a făcut așa și mergea gol și desculț.
\par 3 Și a zis Domnul: "Precum a umblat robul Meu Isaia gol și desculț vreme de trei ani, ca semn și prevestire pentru Egipt și pentru Etiopia,
\par 4 Astfel va aduce regele Asiriei robi din Egipt și surghiuniți din Etiopia, tineri și bătrâni, goi și desculți și cu spatele descoperit, spre rușinea Egiptului.
\par 5 Și cei care se bizuiau pe Etiopia și erau mândri cu Egiptul vor fi cuprinși de teamă și de rușine.
\par 6 Locuitorii acestui ținut vor zice în ziua aceea: "Iată pe cine ne bizuim, către care vrem să fugim să căutăm ajutor și scăpare dinaintea regelui Asiriei! Și acum cum vom scăpa?"

\chapter{21}

\par 1 Ca furtuna care vine de la miazănoapte, aceasta vine din pustiu, dintr-un ținut înfricoșător.
\par 2 O vedenie grozavă mi s-a descoperit: jefuitorul jefuiește și pustiitorul pustiește. Avântă-te, Elame! împresurați pe Mezi, n-aveți nici o milă!
\par 3 De aceea inima mea s-a umplut de neliniște, dureri m-apucă, ca durerile unei femei care este gata să nască. înspăimântat cum sunt, nu mai aud; tulburat, nici că mai văd;
\par 4 Duhul meu rătăcește, frica dă năvală pește mine. Noaptea care atât îmi plăcea mă umple de groază!
\par 5 Masa este pusă, așternuturile întinse, toți mănâncă și beau. Voi, căpetenii, sculați-vă, prindeți scutul!
\par 6 Că așa zice Domnul către mine: "Du-te și pune pe cineva de strajă, care să-Mi dea de știre despre ceea ce va vedea!
\par 7 Dacă va vedea călăreți, doi câte doi pe cai, călăreți pe asini, pe cămile, să se uite cu băgare de seamă, cu mare băgare de seamă".
\par 8 Și el a strigat ca un leu: "Stau de strajă, Doamne, neîncetat toată ziua și la locul meu de veghe în fiecare noapte.
\par 9 Și iată că sosește călărime, călăreți doi câte doi". Și el a vorbit și a zis: "A căzut, a căzut Babilonul și toate chipurile cioplite ale idolilor lui stau sfărâmate la pământ!"
\par 10 O, poporul meu, fecior al ariei mele, bătut cum se bate grâul, ceea ce am auzit de la Domnul Savaot, Dumnezeul lui Israel, ti le dau de știre!
\par 11 Proorocie despre Edom. Cineva strigă din Seir către mine: "Străjerule, cât a trecut din noapte? Străjerule, cât mai este până trece noaptea?
\par 12 Și străjerul răspunde: "Dimineața se apropie, dar este încă noapte. De voiți, întrebați, întoarceți-vă și veniți iarăși".
\par 13 Proorocie despre Arabia. Într-o pădure de stepă petreceți noaptea, voi, caravane din Dedan!
\par 14 Aduceți apă celor însetați, voi, locuitori ai ținutului Tema, întâmpinați cu pâine pe cei fugari,
\par 15 Că ei au fugit dinaintea sabiei, din fața sabiei scoase din teacă, de arcul întins și de grozăviile războiului!
\par 16 Că iată ce mi-a spus Domnul: "Încă un an, ca anii unui simbriaș, și toată strălucirea lui Chedar se duce.
\par 17 Vitejii arcași ai fiilor lui Chedar se vor împuțina; că Domnul Dumnezeul lui Israel a grăit".

\chapter{22}

\par 1 Proorocia despre valea vedeniei. Ce ai tu că tot poporul tău s-a urcat pe acoperișuri,
\par 2 Tu, cetate zgomotoasă, cetate plină de zarvă și de chiote de veselie? Răniții tăi nu sunt răniți de sabie și n-au murit în luptă.
\par 3 Mai-marii tăi au fugit laolaltă, au fost luați robi nu cu puterea arcului; toți vitejii tăi de luptă prinși au fost cu toții, când ei fugeau departe.
\par 4 De aceea vă zic: "Depărtați-vă de mine și lăsați-mă să plâng amar, nu vă îmbulziți să mă mângâiați pentru nenorocirea fiicei poporului meu.
\par 5 Că este o zi de tulburare, de zdrobire, de uluire de la Domnul Dumnezeu Savaot în valea vedeniei, prăbușire de zid și țipetele celor ce fug înspre munți!
\par 6 Elamul a luat tolba de săgeți, Aramul a încălecat pe cal și Chirul a scos pavăza!
\par 7 Văile tale mărețe sunt pline de care și călăreți, tăbărâți la porțile tale;
\par 8 Vălul va fi ridicat de pe Iuda! Și voi veți privi în ziua aceea grămezile de arme din casa cea din pădure.
\par 9 Spărturile zidurilor cetății lui David sunt fără număr, voi le vedeți. Adunați apele din iazul cel mai de jos,
\par 10 Numărați casele cele din Ierusalim, dărâmați-le ca să întăriți zidul.
\par 11 Un iaz mai mare faceți între cele două ziduri, ca să strângeți apa din iazul cel mai de demult. Dar voi nu luați aminte la Cel care a făcut toate acestea, la Cel care le-a pregătit de demult. Voi nu-L vedeți!
\par 12 Și în ziua aceea ne va îndemna Domnul Dumnezeu Savaot să plângem, să suspinăm, să ne radem capul și să ne încingem cu sac.
\par 13 Iată bucuria și veselia, boi tăiați și oi junghiate; toți mănâncă din carne și beau vin: "Să mâncăm și să bem, că mâine vom muri!"
\par 14 Domnul Savaot a descoperit urechilor mele: Acest păcat nu vă va fi iertat nici până la moarte, zice Domnul Dumnezeu Savaot.
\par 15 Împotriva lui Șebna, mai-marele palatului, iată ce spune Domnul Dumnezeu Savaot: "Du-te la acest dregător,
\par 16 Care își sapă mormânt pe un loc înalt, care își pregătește locaș în stâncă și zi-i: "Ce ai tu și cine ești tu de-ți sapi aici mormânt?
\par 17 Iată că Domnul te azvârle, dintr-o singură aruncătură, te strânge cu o singură strângere.
\par 18 El te înfășură și te rostogolește ca pe un ghem pe un câmp întins. Acolo tu vei muri; acolo vor merge carele tale mărețe, tu, rușinea palatului stăpânului tău.
\par 19 El îți va lua slujba ta și te va lipsi de dregătoria ta.
\par 20 Și în ziua aceea voi chema pe sluga mea, pe Eliachim, feciorul lui Hilchia,
\par 21 Și îl voi îmbrăca cu veșmintele tale, îl voi încinge cu brâul tău și-i voi da în mână dregătoria ta. El va fi tată pentru cei ce locuiesc în Ierusalim și pentru casa lui Iuda.
\par 22 Și îi voi pune pe umeri cheile casei lui David și dacă el va deschide, nimeni nu va închide, și dacă el va închide, nimeni nu va deschide.
\par 23 Și îl voi înfige ca pe un cui într-un loc de nădejde și va fi scaun de cinste pentru casa tatălui său.
\par 24 Pe el se va rezema toată slava casei tatălui său, fii și nepoți; toate vasele cele mai mici de la căni și până la marile lighene.
\par 25 În ziua aceea, zice Domnul Savaot, cuiul înfipt într-un loc tare se va slăbi; se va smulge și va cădea și povara atârnată de el va fi nimicită, că așa a grăit Domnul!"

\chapter{23}

\par 1 Tânguiți-vă voi, corăbii ale Tarsisului, căci limanul vostru a fost nimicit. Venitu-le-a această știre din țara Chitim.
\par 2 Amuțiți voi, locuitori ai coastei pe care o umpleau neguțătorii din Sidon care străbăteau marea!
\par 3 Veniturile lui erau grâul Nilului, secerișul din valea lui, adus pe ape mari; el era târgul neamurilor.
\par 4 Rușinează-te, Sidonule, că marea îți zice: "Tu n-ai avut dureri de mamă, tu n-ai născut și nici n-ai crescut băieți și nici n-ai ridicat fete".
\par 5 Când Egiptul va prinde de veste, va tremura la auzul nenorocirilor Tirului.
\par 6 Treceți în Tarsis, bociți-vă, voi, locuitori de pe țărmuri!
\par 7 Aceasta este, oare, cetatea voastră de petrecere, a cărei obârșie se urcă în vremuri vechi și care își călăuzea pașii spre sălașuri depărtate?
\par 8 Cine a poruncit acest lucru împotriva Tirului cel încercat, ai cărui neguțători erau prinți și ai cărui vânzători erau cei mari ai pământului?
\par 9 Domnul Savaot a hotărât aceasta, ca să veștejească mândria a tot ce strălucește, să smerească pe toți cei mari ai pământului.
\par 10 Treci și du-te în pământul tău, tu fiică a Tarsisului, căci portul tău nu mai este.
\par 11 El a întins mâna spre mare, a doborât regatele. Domnul a hotărât împotriva lui Canaan ruina întărituri lor lui.
\par 12 El a zis: "Nu tresălta de bucurie, tu, fecioară necinstită a Sidonului! Scoală-te și du-te la Chitim, dar nici acolo nu vei avea odihnă!"
\par 13 Iată țara Caldeilor! Acest popor nu sunt Asirienii; El a dat-o pradă fiarelor de câmp. Ei și-au înălțat turnuri, au dărâmat palate, făcut-au totul o ruină.
\par 14 Bociți-vă voi, corăbii ale Tarsisului, căci portul vostru a fost dărâmat.
\par 15 Și va fi în ziua aceea că Tirul va fi uitat șaptezeci de ani, ca în zilele unui singur rege, și la sfârșitul celor șaptezeci de ani Tirul va fi așa cum se află în cântecul desfrânatei:
\par 16 "Ia chitara, dă ocol cetății, tu, desfrânată! Cântă cât mai bine, reia cântările ca lumea să-și aducă aminte de tine!"
\par 17 Și după cei șaptezeci de ani, Domnul va cerceta iarăși cetatea Tirului și ea va reîncepe să primească prețul desfrâului ei. Ea se va desfrâna pentru toate regatele lumii de pe fața pământului.
\par 18 Dar tot câștigul, toate foloasele ei vor fi afierosite Domnului și nu vor fi adunate, nici puse la păstrare; ci câștigul va fi pentru cei ce locuiesc înaintea Domnului, ca să aibă hrană din belșug și haine strălucite.

\chapter{24}

\par 1 Iată Domnul pustiește pământul și îl preface în deșert, răstoarnă fața lui și împrăștie pe locuitori.
\par 2 Și preotului i se întâmplă ca și poporului, stăpânului ca și robului, slugii ca și stăpânei sale; vânzătorului ca și cumpărătorului, celui care dă cu împrumut ca și celui care se împrumută, datornicului ca și cel căruia îi este dator.
\par 3 Pământul va fi pustiit, el va fi jefuit, că Domnul a grăit cuvântul acesta.
\par 4 Pământul este în chin și sleit, lumea tânjește și se istovește, cerul împreună cu pământul vor pieri.
\par 5 Pământul este pângărit sub locuitorii lui, căci ei au călcat legea, au înfrânt orânduiala și legământul stricatu-l-au pe veci!
\par 6 Pentru aceasta, blestemul mistuie pământul și locuitorii îndură pedeapsa lor; drept aceea cei ce locuiesc pe pământ sunt mistuiți, iar oamenii rămași sunt puțini la număr!
\par 7 Via tânjește, vițele sale sunt firave, cei cu inima veselă suspină.
\par 8 Glasul cel plin de veselie al lirei a încetat, chiotele zgomotoase nu mai sunt, încetat-a glasul harpei.
\par 9 La cântec nu se mai bea, și amar este vinul pentru băutor.
\par 10 Cetatea pustiită este în ruină, intrarea fiecărei case este închisă.
\par 11 Pe uliță lumea strigă: "Nici un strop de vin!" Nu mai este bucurie, veselia este izgonită de pe pământ.
\par 12 În cetate au rămas numai dărâmături, porți sfărâmate și stricate.
\par 13 Așa se va întâmpla în mijlocul acestui ținut, înăuntrul popoarelor, ca și când se scutură măslinii și ca pe urma culesului viei.
\par 14 Aceia înalță glasul și cântă, preaslăvesc mărirea Domnului la apus.
\par 15 Pentru aceasta, în insule se preaslăvește Domnul, în insulele mării numele Domnului Dumnezeului lui Israel.
\par 16 De la marginile pământului auzim cântând: "Slavă celui drept!" Și eu am zis: "Vai de cei fără de lege, care lucrează, depărtându-se de lege!"
\par 17 Groază, laț și groapă pentru voi, locuitori ai pământului!
\par 18 Cel care va fugi de groază va cădea în groapă, cel care va scăpa din mijlocul gropii se va prinde în laț! Zăgazurile cele de sus se vor deschide și temeliile pământului se vor clătina.
\par 19 Pământul se sfărâmă, pământul sare în bucăți, se clatină pământul.
\par 20 Pământul se mișcă încoace și încolo ca un om bețiv, se dă în sus și în jos ca un scrânciob; păcatele apasă asupra lui, ca să nu se mai scoale!
\par 21 Și în ziua aceea Domnul va cerceta cu asprime, acolo sus, oștirea cea de sus și pe pământ pe regii pământului.
\par 22 Și ca robii vor fi închiși într-o închisoare sub pământ și după multe zile vor fi cercetați.
\par 23 Luna va fi roșie, iar soarele va pierde din lumina lui, căci Domnul Savaot va fi rege și glava Lui va străluci înaintea bătrânilor în muntele Sionului și în Ierusalim!

\chapter{25}

\par 1 Doamne Dumnezeul meu, pe Tine Te voi înălța, lăuda-voi numele Tău, că Tu ai făcut lucruri minunate; planurile Tale de mult întocmite sunt adevărate și statornice.
\par 2 Că Tu ai făcut din cetate o grămadă de pietre și din cetatea cea întărită o dărâmătură. Cetatea celor fără de lege nu mai este cetate, zidită nu va mai fi în veci.
\par 3 Pentru aceasta un popor tare Te va preaslăvi, cetatea puternicelor neamuri de Tine se va teme.
\par 4 Fost-ai scăpare pentru cel sărman, adăpost pentru cel ce era în strâmtorare, liman în vremi vijelioase, umbră în vreme de căldură. Căci suflarea celor apăsători este ca furtuna de iarnă
\par 5 Și ca arșița soarelui într-un pământ uscat. Ai potolit zarva celor nelegiuiți. Precum se potolește căldura la umbra unui nor, așa se va domoli cântecul de biruință al stăpânitorilor silnici.
\par 6 Și Domnul Savaot va pregăti în muntele acesta pentru toate popoarele un ospăț de cărnuri grase, un ospăț cu vinuri bune, cărnuri grase cu măduvă, vinuri bune, limpezite!
\par 7 Și în muntele acesta El va da la o parte vălul care învăluie toate popoarele și perdeaua care acoperă toate neamurile.
\par 8 El va înlătura moartea pe vecie! Și Domnul Dumnezeu va șterge lacrimile de pe toate fețele și rușinea poporului Său o va îndepărta de pe pământ, căci Domnul a grăit!
\par 9 Și se va zice în ziua aceea: Iată Dumnezeul nostru în Care nădăjduiam ca să fim mântuiți. Iată Domnul, în Care am nădăjduit, să ne bucurăm și să ne veselim de mântuirea Lui,
\par 10 Că mâna Domnului se va odihni pe acest munte. Moabul însă va fi călcat în picioare pe locul lui, ca niște paie în groapa cu gunoi.
\par 11 Și va întinde mâinile sale, precum înotătorul le întinde ca să înoate. Dar Domnul va zdrobi mândria lui și silințele mâinilor lui.
\par 12 Întăriturile lui mărețe și înalte le va nimici, le va răsturna și la pământ le va culca, în țărână.

\chapter{26}

\par 1 În ziua aceea se va cânta cântarea aceasta în pământul lui Iuda: "Avem o cetate întărită. Domnul ne vine într-ajutor cu ziduri și întărituri.
\par 2 Deschideți porțile, ca să intre un neam drept care păzește credincioșia!
\par 3 Nădejde neclintită, Tu ne vei păstra pacea noastră, că întru Tine ne punem nădejdea.
\par 4 Încredeți-vă în Domnul pururea, căci Domnul Dumnezeu este stânca veacurilor.
\par 5 Că El a coborât pe cei ce locuiau pe înălțime, cetatea cea mândră El a supus-o până la pământ, a culcat-o în pulbere.
\par 6 Ea este călcată în picioare, în picioarele săracilor, sub pașii obijduiților!
\par 7 Calea celui drept este dreaptă; Tu netezești drumul drept al celui drept.
\par 8 Pe calea judecăților Tale, Doamne, noi Te așteptăm; numele Tău și amintirea Ta erau nădejdea sufletului nostru.
\par 9 Sufletul meu Te-a dorit în vreme de noapte, duhul meu năzuiește spre Tine; căci când îndreptările Tale vor fi pe pământ, cei ce locuiesc lumea vor învăța ce este dreptatea.
\par 10 Dacă de cel fără de lege ne este milă, el nu mai învață ce este dreptatea și în pământul celor sfinți va săvârși strâmbătatea. Să nu mai fie pe pământ cei fără de lege și să nu mai vadă slava Celui Preaînalt.
\par 11 Doamne, mâna Ta era ridicată, dar ei n-au văzut-o! Vor vedea râvna Ta pentru poporul Tău și se vor rușina. Și focul hărăzit vrăjmașilor Tăi îi va mânca!
\par 12 Doamne, revarsă pacea peste noi, căci toate lucrurile noastre, pentru noi le-ai făcut!
\par 13 Doamne, Dumnezeul nostru, am avut peste noi și alți stăpâni afară de Tine, dar noi ne vom aduce aminte numai de numele Tău!
\par 14 Morții nu vor mai trăi și umbrele nu vor învia, fiindcă Tu le-ai pedepsit și le-ai nimicit și ai șters până și numele lor.
\par 15 Înmulțește poporul, Doamne, înmulțește poporul și arată-Te mare, lărgește din nou toate hotarele țării!
\par 16 Doamne, pe Tine Te-au căutat ei în vreme de restriște, către Tine am strigat în scârba noastră, când Tu ne pedepseai.
\par 17 Ca femeia însărcinată și gata să nască prunc, care se zvârcolește și strigă în durerea ei, așa am fost noi, Doamne, cu toții în fața Ta!
\par 18 Zămislit-am, dureri de facere am avut și am născut vânt! Mântuire țării noi n-am dat și în lume nu s-au născut locuitorii ei!
\par 19 Morții Tăi vor trăi și trupurile lor vor învia! Deșteptați-vă, cântați de bucurie, voi cei ce sălășluiți în pulbere! Căci roua Ta este rouă de lumină și din sânul pământului umbrele vor învia.
\par 20 Du-te, poporul meu, intră în cămările tale și închide ușa după tine; ascunde-te puține clipe, până când mânia va fi trecut!
\par 21 Că iată Domnul va ieși din locașul Său, ca să pedepsească fărădelegile locuitorilor pământului. Pământul va arăta sângele pe care l-a supt și nu va mai ascunde pe ucigașii lui".

\chapter{27}

\par 1 În ziua aceea Domnul se va năpusti cu sabia Sa grea, mare și puternică, asupra leviatanului, a șarpelui care fuge, asupra leviatanului, a șarpelui încolăcit, și va omorî balaurul cel din Nil.
\par 2 Și se va zice în ziua aceea: "Vie cu vin bun, cântă!
\par 3 Eu, Domnul, sunt străjerul ei, în fiecare clipă Eu o ud, ca frunzele ei să nu cadă. Zi și noapte Eu o păzesc;
\par 4 Nu sunt mâniat de fel pe ea. Dar dacă voi găsi mărăcini și spini, voi porni război împotriva lor și-i voi arde pe toți.
\par 5 Sau mai bine să caute ocrotirea Mea și cu Mine să facă pace, și cu Mine să fie în pace!...".
\par 6 Dar într-o zi Iacov va prinde rădăcini, Israel va înflori, va rodi și cu roadele sale lumea o va umple.
\par 7 L-a lovit oare Domnul cum l-au lovit cei ce l-au lovit, sau i-a omorât El cum au făcut ucigașii lui?
\par 8 Cu izgonire, cu robie pedepsitu-i-a și i-a măturat cu suflarea Lui năprasnică de vânt de răsărit.
\par 9 Așa a fost ispășită fărădelegea lui Iacov, și acesta este rodul iertării păcatului său. El a sfărâmat în bucăți toate pietrele jertfelnicului, ca niște pietre de var; dumbrăvile Astartei și stâlpii soarelui nu se vor mai ridica.
\par 10 Cetatea cea întărită a rămas singură, un loc părăsit și neumblat ca un pustiu. Acolo paște vițelul, în ea își are sălașul și îi mănâncă mlădițele.
\par 11 Când crengile se usucă, se rup și cad, femeile vin și le dau foc. Acesta este un popor fără de minte și nici Ziditorul lui nu Se milostivește de el și nici Făcătorul lui nu are milă de el.
\par 12 Și în ziua aceea, Domnul va aduna roade de la Eufrat și până la râul Egiptului; și voi veți fi culeși unul câte unul, feciori ai lui Israel!
\par 13 În vremea aceea, trâmbița cea mare va trâmbița și cei care se pierduseră în pământul Asiriei și cei ce se risipiseră în țara Egiptului vor veni și se vor închina Domnului, în muntele cel sfânt, în Ierusalim.

\chapter{28}

\par 1 Vai de cununa mândriei bețivilor din Efraim; vai de floarea veștedă din podoaba lor, care stă pe culmea de deasupra văii celei mănoase a celor beți de vin!
\par 2 Iată un om tare și puternic vine de la Domnul: ca un potop de grindină, ca o vijelie nimicitoare, ca o năvală de apă potopitoare o va răsturna la pământ.
\par 3 Și va fi călcată în picioare cununa cea îngâmfată a bețivilor din Efraim;
\par 4 Iar floarea cea veștejită din strălucita sa găteală care strălucește pe culmea de deasupra văii celei mănoase, va fi ca o smochină timpurie și înainte de vreme; cine o vede o ia și o mănâncă!
\par 5 În ziua aceea Domnul Savaot va fi o cunună strălucitoare și o strălucită găteală pentru cei ce au mai rămas din popor,
\par 6 Duh de dreptate pentru cei ce stau la judecată cu dreptate și tărie pentru cei ce se luptă la porți.
\par 7 Dar și aceștia se clatină de vin și rătăcesc drumul din pricina băuturilor tari; preotul și proorocul se poticnesc de băutură, sunt biruiți de vin, au amețeli din pricina băuturilor tari, în vedenii se înșală, în hotărâri șovăiesc.
\par 8 Toate mesele sunt pline de vărsături, nici un loc curat nu mai este.
\par 9 Dar totuși zic: "Pe cine vrea acesta să învețe cu vedenia? Și pe cine vrea el cu propovăduirea să înțelepțească? Au doar pe cei înțărcați sau pe cei abia depărtați de la sânul mamei lor?
\par 10 Căci țav lațav, țav lațav, cav lacav, cav lacav, zeher șam, zeher șam, (poruncă peste poruncă, poruncă peste poruncă, regulă peste regulă, regulă peste regulă, când pe aici, când pe acolo!)
\par 11 De aceea într-o limbă străină și stâlcită se va grăi poporului acestuia,
\par 12 Căruia i se spunea: "Iată odihna, să se odihnească cel care este obosit; iată ușurarea, dar el n-a vrut să asculte".
\par 13 Și cuvântul Domnului va fi pentru ei: țav lațav, țav lațav, cav lacav, cav lacav, zeher șam, zeher șam, (poruncă peste poruncă, poruncă peste poruncă, regulă peste regulă, regulă peste regulă, când pe aici, când pe acolo) ca să meargă și să cadă peste cap, să se sfărâme și în cursă să fie prinși!
\par 14 Pentru aceasta, ascultați cuvântul Domnului, voi, oameni de râs, îndrumători ai poporului celui din Ierusalim!
\par 15 Voi ziceți: Noi am făcut legământ cu moartea și cu iadul (șeolul) învoială; urgia va trece fără să ne atingă, căci ne-am făcut din minciună un adăpost și din viclenie un liman!
\par 16 Pentru aceasta așa zice Dumnezeu: "Pus-am în Sion o piatră, o piatră de încercare, piatra din capul unghiului, de mare preț, bine pusă în temelie; cel care se va bizui pe ea, nu se va clătina!
\par 17 Și voi face judecata dreptar și dreptatea cumpănă. Și grindina va lua la vale adăpostul minciunii și potop de ape va peste locul ei de scăpare!
\par 18 Și legământul vostru cu moartea va fi stricat și înțelegerea voastră cu iadul (șeolul) va fi desfăcută. Când urgia va trece, vă va zdrobi,
\par 19 Ori de câte ori va trece, vă va apuca! Căci ea va trece în fiecare dimineață, ziua și noaptea, și nu va fi decât groază pentru a pricepe descoperirea!
\par 20 Patul acesta va fi scurt și nu te vei putea întinde, iar așternutul lui prea scurt, ca să te învelești".
\par 21 Că Domnul se va ridica precum altădată în muntele Perațim și se va întărâta ca în valea Ghibeonului ca să săvârșească fapta Lui, fapta Lui ciudată, să împlinească lucrul Lui, lucrul Lui minunat.
\par 22 Deci nu vă mai bateți joc, ca legăturile voastre să nu se strângă, că am auzit de la Domnul Dumnezeu Savaot că nimicirea este hotărâtă să fie pentru toată țara!
\par 23 Luați aminte și ascultați; fiți cu luare aminte și ascultați graiul meu!
\par 24 Oare în fiecare zi plugarul ară, seamănă, desfundă pământul și îl grăpează?
\par 25 Nu vine el apoi, după ce i-a netezit fața, să arunce în brazde chimenul, să pună grâul, orzul și alacul pe margini?
\par 26 Dumnezeul lui îl învață și dă aceste rânduieli.
\par 27 Meiul nu este călcat sub copita cailor și tăvălugul nu trece peste chimen; ci meiul cu un băl este bătut și chimenul cu o nuia.
\par 28 Grâul este treierat, dar nu sfărâmat. Peste el trece un tăvălug purtat de cai și îl scutură din spice.
\par 29 Și aceasta vine de la Domnul Savaot. Minunat este sfatul Lui și mare purtarea Lui de grijă!

\chapter{29}

\par 1 Vai ție, Ariele, Ariele, cetate în care a trăit David! Treacă an de an, șirul de praznice să se sfârșească!
\par 2 Apoi voi împresura Arielul și el va plânge și va geme! Cetatea va fi ca un Ariel pentru Mine.
\par 3 Ca David voi tăbărî asupra ta, te voi înconjura cu valuri și voi ridica întărituri împotriva ta.
\par 4 Vei fi doborât la pământ și de acolo se va auzi glasul tău; graiul tău din țărână se va auzi; glasul tău va fi ca al unei năluci ce iese din pământ și din praf spusele tale ca un murmur vor părea.
\par 5 Mulțimea vrăjmașilor tăi va fi ca pulberea măruntă, ceata asupritorilor ca pleava care zboară. Dar aceasta se va petrece într-o clipă.
\par 6 Domnul Savaot te va cerceta cu tunet, cutremur și zgomot mare, uragan și vijelie și flăcări de foc mistuitor!
\par 7 Și ca un vis, ca o vedenie de noapte va fi mulțimea de popoare luptătoare împotriva lui Ariel, care se vor război cu el, cu cetatea lui și de jur împrejur o vor strânge.
\par 8 După cum cel flămând visează că mănâncă și se trezește tot cu stomacul gol, și după cum cel însetat visează că bea și se trezește istovit și tot însetat, tot așa se va întâmpla cu mulțimea de popoare care vor merge împotriva muntelui Sion!
\par 9 Stați încremeniți și înmărmuriți, fiți orbi și orbi rămâneți! Îmbătați-vă, dar nu de vin; clătinați-vă, dar nu de băutură!
\par 10 Că Domnul a turnat peste voi un duh de toropeală. El a închis ochii voștri, profeților, și capetele voastre, văzătorilor, le-a acoperit cu văl.
\par 11 Drept aceea orice descoperire este pentru voi ca graiurile dintr-o carte pecetluită. Dacă le dai cuiva care știe carte și-i zici: "Citește!" el iți răspunde: "Nu pot, căci ea este pecetluită!"
\par 12 Și dacă o dai cuiva care nu știe carte și-i zici: "Citește!", el îți va răspunde: "Nu știu carte!"
\par 13 Și a zis Domnul: "De aceea poporul acesta se apropie de Mine cu gura și cu buzele Mă cinstește, dar cu inima este departe, căci închinarea înaintea Mea nu este decât o rânduială omenească învățată de la oameni.
\par 14 De aceea voi face pentru poporul acesta minuni fără seamăn. Înțelepciunea celor înțelepți se va pierde și istețimea celor isteți va pieri.
\par 15 Vai de cei ce ascund lui Dumnezeu taina planurilor lor, ca faptele lor să se facă la întuneric! Vai de cei care zic: "Cine ne vede? Cine ne știe?"
\par 16 Ce stricăciune! Oare olarul poate fi socotit drept lut? Lucrul poate oare zice despre lucrător: "Nu m-a făcut el!" Vasul zice oare despre olar: "El nu pricepe?"
\par 17 Încă puțină vreme și Libanul se va schimba în grădină, și grădina va fi socotită pădure.
\par 18 În vremea aceea, cei surzi vor auzi cuvintele cărții și ochii celor orbi vor vedea fără umbră și fără întuneric.
\par 19 Cei smeriți se vor bucura întru Domnul și cei săraci se vor veseli de Sfântul lui Israel.
\par 20 Că apăsătorul nu va mai fi, cel batjocoritor va pieri, distruși vor fi cei ce pândeau să facă rău,
\par 21 Cei care găseau vină oricui, pentru un cuvânt în fața lumii întind cursă judecătorului și pentru nimic răpesc dreptul celui cinstit.
\par 22 Pentru aceasta, Domnul, Care a răscumpărat pe Avraam așa zice către casa lui Iacov: "De aici încolo, nu se va mai rușina Iacov și fața lui nu se va mai îngălbeni.
\par 23 Și atunci când vor vedea lucrul mâinilor Mele în mijlocul lor, sfinți-vor numele Meu, vor chema sfânt pe Sfântul lui Iacov și se vor teme de Dumnezeul lui Israel.
\par 24 Cei rătăciți cu duhul vor căpăta înțelepciune și cei cârtitori învățătură".

\chapter{30}

\par 1 Vai de feciorii răzvrătiți, zice Domnul, vai de cei ce fac planuri fără Mine, care fac legăminte ce nu sunt în Duhul Meu, ca să grămădească păcate peste păcate.
\par 2 Ei iau calea Egiptului, fără să fi întrebat gura Mea, să cerșească de la Faraon ajutor și la umbra Egiptului să se adăpostească.
\par 3 Pentru aceasta sprijinul lui Faraon va fi pentru voi rușine și râs adăpostul la umbra lui.
\par 4 Deși căpeteniile lui sunt la Țoan și până la Hanes ajung trimișii lui,
\par 5 Toți sunt neliniștiți de acest popor, care nu le va fi de folos, care nu le va da nici un ajutor, ci numai nedumerire și ocară.
\par 6 Proorocie despre fiarele de la miazăzi: Printr-o țară de strâmtorare și îngrijorare, cu lei și leoaice mugitoare, năpârci și șerpi zburători, ei duc pe măgari avuțiile lor și pe cămile comorile lor, către un popor care nu le folosește la nimic.
\par 7 Căci ajutorul Egiptului este deșertăciune și nimic, pentru aceea l-am numit Rahab cel adormit.
\par 8 Acum, du-te" scrie acestea pe o tablă și trece-le într-o carte, ca să fie pentru mai târziu mărturie veșnică.
\par 9 Pentru că ei sunt un popor de răzvrătiți, feciori mincinoși, care nu voiesc să asculte de legea Domnului,
\par 10 Care zic proorocilor: "Voi nu vedeți!" Și văzătorilor: "Nu ne proorociți pedepse, ci spuneți-ne lucruri măgulitoare, proorociți-ne închipuiri amăgitoare!
\par 11 Dați-vă la o parte din cale, nu ne împiedicați în drum, luați din fața noastră pe Sfântul lui Israel!"
\par 12 Pentru aceasta zice Sfântul lui Israel: "Fiindcă voi ați disprețuit cuvântul acesta și v-ați încrezut în nedreptate și minciună și ați nădăjduit numai în ele,
\par 13 Iată cum va fi păcatul vostru: ca o spărtură într-un zid înalt, care dintr-o dată și pe neașteptate se prăbușește;
\par 14 Ca un vas de lut, care este așa de spart și zdrobit fără de milă, încât între cioburile lui nu se află măcar unul cu care să iei foc din vatră sau să scoți apă din fântână".
\par 15 Că așa zice Domnul Dumnezeu, Sfântul lui Israel: "Dacă vă întoarceți și sunteți în bună pace, vă veți izbăvi; liniștea și nădejdea sunt vârtutea voastră". Dar voi n-ați vrut să ascultați,
\par 16 Ci ați zis: "Nu! Noi vom fugi călări pe cai!" Așa, fugiți! "Vom încăleca pe cai iuți ca vântul!" Ei bine, veți fi urmăriți și mai repede!
\par 17 O mie vor fugi de amenințarea unuia și când vă vor amenința cinci, toți veți fugi, până când veți rămâne ca un stâlp pe vârful muntelui și ca un steag pe vârf de deal.
\par 18 Pentru aceasta Domnul așteaptă să Se milostivească spre voi, de aceea El Se ridică să aibă milă de voi. Că Domnul este Dumnezeu al dreptății; fericiți sunt cei care nădăjduiesc în El!
\par 19 Popor din Sion, care locuiești în Ierusalim, nu plânge! El se va milostivi la glasul strigătului tău și te va auzi degrabă!
\par 20 Când Domnul îți va fi dat ție pâinea îngrijorării și apa strâmtorării, și cei ce te învață nu se vor mai ascunde, ci ochii tăi vor vedea pe dascălii tăi
\par 21 Și urechile tale vor auzi cuvântul celor ce te călăuzesc pe tine, zicând: "Iată calea, mergeți pe ea!", fie că ați merge la dreapta sau la stânga,
\par 22 Atunci argintul care acoperă idolii îl veți găsi spurcat și aurul care împodobește chipurile turnate, ca necurat îl veți arunca, zicând: "Afară de aici!"
\par 23 Și El îți va da ploaie pentru semănătura ta pe care vei fi semănat-a pe pământ și pâinea pe care o va rodi pământul va fi gustoasă și hrănitoare. Turmele tale vor paște în ziua aceea pe pajiști întinse.
\par 24 Și boii și asinii care lucrează pământul, vor mânca nutreț dat cu sare, cu lopata și cu banița vânturat.
\par 25 Atunci pe orice munte înalt și pe orice deal mare, vor fi râulețe și pâraie de apă, în ziua măcelului groaznic, când turnurile vor cădea.
\par 26 Și luna va străluci ca soarele, iar soarele va străluci de șapte ori mai mult, ca lumina a șapte zile, în ziua când Domnul va lega rana poporului Său și va tămădui vânătăile de pe trupul lui.
\par 27 Iată numele Domnului Care vine de departe, mânie înfocată și nor greu; buzele Sale sunt pline de urgie și limba Lui e foc mistuitor!
\par 28 Duhul Lui ca un șuvoi revărsat care ajunge până la gât, ca să cearnă pe neamuri cu sita nimicirii.
\par 29 Voi veți cânta atunci, ca în noaptea cea de praznic, cu bucurie în inimi, în sunetul de flaut, ca să mergeți în muntele Domnului, vârtutea lui Israel.
\par 30 Și Domnul va face să răsune glasul Său măreț și va prăvăli brațul Său în aprinderea mâniei Sale, în mijlocul unui foc mistuitor, al vijeliei și al potopului de ape și grindină.
\par 31 La glasul Domnului va tremura Asiria; cu toiagul Său o va lovi.
\par 32 La fiecare lovitură pe care Domnul i-o va da cu toiagul cel de mustrare, sunete de tobă, de harpă și de joc vor izbucni. în cântece și Domnul va lupta împotriva ei cu mina ridicată.
\par 33 Un jertfelnic de multă vreme este pregătit, hotărât pentru Moloh. Pus-a un rug mare și larg, paiele și lemnele sunt din belșug. Suflarea Domnului îl va aprinde ca un șuvoi de pucioasă.

\chapter{31}

\par 1 Vai de cei ce se coboară în Egipt după ajutor și se bizuie pe caii lor și își pun nădejdea în mulțimea carelor și în puterea călăreților, dar nu-și ațintesc privirea către Sfântul lui Israel și nu caută pe Domnul.
\par 2 Dar El este înțelept, El face să vină nenorocirea și nu Își ia înapoi cuvintele. El Se ridică împotriva casei celor fără de lege și împotriva ajutorului celor care săvârșesc nedreptatea.
\par 3 Egipteanul este om, nu Dumnezeu, caii lui sunt carne și nu duh. Când Domnul Își va întinde mâna Lui, ocrotitorul se va împiedica și ocrotitul va cădea, iar amândoi împreună vor pieri.
\par 4 Că iată ce mi-a grăit Domnul: "Precum leul și puiul de leu răcnesc asupra prăzii și împotriva lor se adună toată ceata de păstori și nu se înfioară de strigătele lor, nici nu se tulbură de mulțimea lor, tot astfel Domnul Savaot Se va pogorî să se războiască pe muntele Sionului și pe colina lui. Și dușmanii se vor risipi toți,
\par 5 Ca păsările care zboară. Așa Domnul Savaot va ocroti Ierusalimul, îl va acoperi, îl va mântui, îl va cruța, îl va libera".
\par 6 Întoarceți-vă către Acela de Care adâncul vă desparte, copii ai lui Israel!
\par 7 În vremea aceea fiecare din voi veți da la o parte idolii de argint și cei de aur pe care i-aii făcut cu mâinile voastre cele păcătoase.
\par 8 Și Asiria va cădea în sabie care nu este omenească, va fi nimicită nu de sabia unui muritor. Ea o va lua la fugă în fața sabiei, iar tinerii vor fi duși în robie!
\par 9 De frică întăritura ei va fi nimicită, iar căpeteniile vor fugi din jurul steagului, zice Domnul, a Cărui văpaie este în Sion și cuptorul în Ierusalim!

\chapter{32}

\par 1 Iată că un rege va stăpâni prin dreptate și căpeteniile vor cârmui cu dreptate.
\par 2 Fiecare va fi ca un adăpost împotriva vântului, ca un liman împotriva vijeliei, ca pâraiele de apă într-un pământ uscat și ca umbra unei stânci înalte într-un ținut însetat.
\par 3 Ochii celor care văd nu vor fi închiși și urechile celor care aud vor lua aminte.
\par 4 Inima celor ușuratici va judeca sănătos și limba celor gângavi va grăi iute și deslușit.
\par 5 Nebunului nu i se va mai zice că e de neam bun și celui viclean că e mare la suflet.
\par 6 Că nebunul grăiește nebunii și inima lui gândește răul ca să săvârșească nelegiuiri, să rostească cuvinte mincinoase împotriva Domnului, să lase nemâncat pe cel flămând și celor însetați să nu le dea să bea.
\par 7 Uneltele celui mișel sunt ticăloase, el plăsmuiește uneltiri ca să piardă pe cei smeriți prin cuvinte mincinoase, pe cel sărac care-și caută dreptate.
\par 8 Omul de viță bună sfătuiește cele cuviincioase și stăruiește în cuviința lui.
\par 9 Femei fără de grijă, sculați-vă și ascultați glasul meu! Fecioare încrezătoare, luați aminte la graiul meu!
\par 10 Într-un an și câteva zile veți tremura, voi încrezătoarelor, culesul va fi trecut și strânsul nu se va mai face!
\par 11 Tremurați, nepăsătoarelor, înfiorați-vă, încrezătoarelor, scoateți îmbrăcămintea, dezbrăcați-vă, încingeți-vă peste mijloc cu haine de jale.
\par 12 Bateți-vă în piept și plângeți pentru țarinele cele frumoase, și rodnicia viilor.
\par 13 Pe pământul poporului meu vor crește spini și ciulini, ba și în toate casele de petrecere ale veselei cetăți.
\par 14 Palatul va fi pustiu, cetatea cea zgomotoasă, părăsită, colina și turnul de strajă, pustiite, prefăcute pe vecie în vizuini, loc de zburdare pentru asini și pășune pentru turme,
\par 15 Până când se va turna din Duhul cel de sus și pustiul va fi ca o grădină cu pomi și grădina socotită ca o pădure.
\par 16 Atunci judecata va locui în deșert și dreptatea va sălășlui în grădina cea cu pomi.
\par 17 Pacea va fi lucrul dreptății, roada dreptății va fi liniștea și nădejdea în veci de veci.
\par 18 Atunci poporul meu va locui într-un loc de pace, în sălașuri de nădejde și în adăposturi fără grijă.
\par 19 Pădurea va cădea de grindină, iar cetatea va fi supusă.
\par 20 Fericiți sunteți voi, care semănați și nu legați nici boul, nici asinul!

\chapter{33}

\par 1 Vai ție, pustiitorule, care n-ai fost pustiit și ție, jefuitorule, care n-ai fost încă jefuit. Când vei sfârși de pustiit, vei fi pustiit, când vei fi jefuit din destul, vei fi jefuit și tu.
\par 2 Doamne, miluiește-ne, că întru Tine am nădăjduit, fii ajutorul nostru în fiecare dimineață și izbăvirea noastră în vremuri de strâmtorare!
\par 3 La glasul tunetului Tău neamurile vor fugi; când Te ridici Tu, popoarele se vor risipi.
\par 4 Și vor aduna prada voastră, cum adună lăcustele; arunca-se-vor asupra ei, cum se aruncă lăcustele.
\par 5 Domnul este mare, El locuiește în înălțime; Sionul este plin de judecată și de dreptate.
\par 6 Ocrotirea Domnului în aceste vremuri va fi pentru Sion comoară de fericire; înțelepciunea, știința și temerea de Dumnezeu sunt avuția lui.
\par 7 Iată că locuitorii din Ariel strigă pe ulițe, solii pentru pace plâng amar.
\par 8 Drumurile sunt pustii, nici un trecător pe cale. El strică legământul, nesocotește cetățile, nu mai ține seamă de nimeni.
\par 9 Țara plânge și tânjește, Libanul este tulburat și ofilit. Șaronul a ajuns ca un pustiu, Basanul și Carmelul își scutură frunzișul lor.
\par 10 "Acum Mă voi scula, zice Domnul, acum Mă voi ridica, acum Mă voi înălța!"
\par 11 Zămislit-ați fin și ați născut paie, suflarea voastră e foc care vă va mistui,
\par 12 Popoarele vor fi prefăcute în cenușă ca spinii tăiați și mistuiți de foc!
\par 13 Voi cei de departe, auziți ce am făcut, și voi cei de aproape, cunoașteți puterea Mea!
\par 14 Păcătoșii vor tremura în Sion și pe cei fără de lege fierul îi va cuprinde: "Care din noi poate să îndure focul mistuitor, care din noi poate să stea pe jarul cel de veci?"
\par 15 Omul cel drept în calea sa și cel ce grăiește cuvinte de cinste, care dă la o parte câștigul cel nedrept, cel ce mâinile înapoi le trage și mită nu primește, care-și astupă urechile când aude fărădelegi și își pune văl pe ochi ca să nu mai vadă răul,
\par 16 Acela va locui pe înălțimi, Și stâncile cele tari vor fi cetatea lui; pâine i se va da și apa nu-i va lipsi.
\par 17 Ochii tăi vor privi pe rege în toată frumusețea lui și o țară îndepărtată vor vedea.
\par 18 Inima ta își va aduce aminte de aceste vremuri de groază, zicând: "Unde este scriitorul, unde este vistiernicul, unde este străjuitorul cel din turnuri?
\par 19 Atunci nu vei mai vedea pe poporul acesta îndrăzneț, acest neam cu vorbe încâlcite, pe care nu-l înțelegem, care bâlbâie o limbă care nu se poate pricepe.
\par 20 Privește Sionul, cetatea sărbătorilor noastre; ochii tăi să vadă Ierusalimul, loc de liniște, cort bine înfipt, ai cărui țăruși nu se pot scoate, ale cărui frânghii nu se pot desface.
\par 21 Domnul este pentru noi aici în toată slava Sa; El ține loc pentru noi de fluvii, de largi canaluri, pe care nici o barcă cu vâsle nu trece, pe care nici o corabie mare nu merge.
\par 22 Domnul este Judecătorul nostru, Domnul este Căpetenia noastră, Domnul este Împăratul nostru, El ne va izbăvi!
\par 23 Frânghiile tale sunt dezlegate, ele nu mai sprijină catargul, nici nu mai întind pânzele. Atunci se va împărți o mare pradă și șchiopii vor avea parte de ea.
\par 24 Nimeni dintre locuitorii Sionului nu va zice: "Sunt bolnav!" Poporul care-l locuiește va dobândi iertarea păcatelor.

\chapter{34}

\par 1 Apropiați-vă, voi neamuri, și ascultați, și voi popoare, luați aminte; să asculte pământul și cei ce-l locuiesc, lumea cu toate făpturile ei.
\par 2 Că Domnul este mâniat asupra popoarelor, cu urgie împotriva oștirii lor. El le nimicește și le dă la junghiere;
\par 3 Morții lor vor fi aruncați pe câmp, cadavrele lor greu vor mirosi și prin munți vor șerpui pârâiașe din sângele lor.
\par 4 Toată oștirea cerului se va topi, cerurile se vor strânge ca un sul de hârtie și toată oștirea lor va cădea cum cad frunzele de viță și cele de smochin.
\par 5 Că s-a îmbătat de mânie în ceruri sabia Domnului și iată că asupra lui Edom coboară, asupra poporului hărăzit pedepsei.
\par 6 Sabia Domnului este plină de sânge, acoperită de grăsime, de sânge de miei și de țapi, de grăsimea rărunchilor de berbeci. Că Domnul face jertfă la Boțra și mare junghiere în țara lui Edom.
\par 7 Bivolii cad împreună cu ei, și boii cu taurii. Și pământul se îmbată de sângele lor și pulberea de grăsime este plină.
\par 8 Căci aceasta este ziua de răzbunare a Domnului, an de răsplătire pentru pricina Sionului!
\par 9 Râurile în păcură se vor preface și pulberea în pucioasă. Pământul lui va fi pucioasă arzătoare,
\par 10 Zi și noapte. Niciodată nu se va mai stinge în veci de veci și din neam în neam se va înălța văpaia și fumul lui. Pe veci el va rămâne pustiu și nimeni pe acolo nu va trece.
\par 11 Pelicanul și ariciul vor fi stăpânii lui, bufnița și corbul, locuitorii lui. Și pe deasupra, Domnul va întinde peste el frânghia nimicirii și cumpăna pustiirii. Și oameni cu chip de țap într-însul vor locui și de viță bună socotiți vor fi.
\par 12 Nu se va pomeni acolo de nici un regat și toți prinții lui vor fi nimiciți.
\par 13 În palatele lui vor crește spini, iar în turnurile dărâmate mărăcini și urgie. Acolo va fi sălașul șacalilor și adăpostul struților.
\par 14 Câini și pisici sălbatice se vor pripăși pe acolo și făpturi omenești cu chip de țap se vor strânge (fără număr). Acolo vor zăbovi năluci ce umblă noaptea și în acele locuri își vor găsi odihna.
\par 15 Acolo își va face șarpele cuibul, și va depune ouă în el, va cloci și va scoate pui. Acolo se vor strânge vulturii de pradă și în acele locuri se vor găsi cu toții.
\par 16 Cercetați cartea Domnului și citiți, că nimic din acestea nu lipsește. Căci gura Domnului a poruncit și suflarea Lui le-a adunat.
\par 17 El singur a aruncat sorții și mâna Lui le-a împărțit pământul cu funia. Pentru totdeauna ei le vor stăpâni și în el vor locui din neam în neam.

\chapter{35}

\par 1 Veselește-te pustiu însetat, să se bucure pustiul; ca și crinul să înflorească.
\par 2 Și va înflori și se va bucura pustiul Iordanului și mărirea Libanului se va da lui și cinstea Carmelului; și poporul meu va vedea slava Domnului, strălucirea Dumnezeului nostru.
\par 3 Întăriți-vă voi, mâini slabe și prindeți putere genunchi slăbănogi.
\par 4 Ziceți celor slabi la inimă și la cuget: "Întăriți-vă și nu vă temeți. Iată Dumnezeul nostru! Cu judecată răsplătește și va răsplăti; El va veni și ne va mântui".
\par 5 Atunci se vor deschide ochii celor orbi și urechile celor surzi vor auzi.
\par 6 Atunci va sări șchiopul ca cerbul și limpede va fi limba gângavilor; că izvoare de apă vor curge în pustiu și pâraie în pământ însetat.
\par 7 Pământul cel fără de apă se va preface în bălți și ținutul cel însetat va fi izvor de apă. Acolo va fi veselia păsărilor, iarbă, trestie și bălți.
\par 8 Acolo va fi cale curată și cale sfântă se va chema și nu va trece pe acolo nimeni necurat și nici nu va fi acolo cale întinată. Chiar și cei fără de minte vor merge pe dânsa și nu se vor rătăci.
\par 9 Și nu va fi acolo leu, nici fiare cumplite nu se vor sui pe ea și nici nu se vor afla acolo; ci vor merge pe dânsa cei mântuiți și cei răscumpărați de Domnul se vor întoarce.
\par 10 Și vor veni în Sion în chiote de bucurie și veselia cea veșnică va încununa capul lor. Lauda și bucuria și veselia îi vor ajunge pe aceștia și vor fugi durerea, întristarea și suspinarea.

\chapter{36}

\par 1 În anul al paisprezecelea al domniei lui Iezechia, Sanherib, regele Asiriei, a pornit cu război împotriva cetăților celor întărite ale lui Iuda și le-a cuprins.
\par 2 Și regele Asiriei a trimis pe Rabșache cu mare oștire din Lachiș la Ierusalim către regele Iezechia. Rabșache a tăbărât lângă canalul de apă al iazului de sus, pe drumul către țarina nălbitorului.
\par 3 Atunci a ieșit întru întâmpinarea lui Eliachim, feciorul lui Hilchia, căpetenia casei regelui, și Șebna scriitorul și Ioah cronicarul, feciorul lui Asaf.
\par 4 Și Rabșache a zis către ei: "Spuneți lui Iezechia: Așa zice regele cel mare, regele Asiriei: De unde vine încrederea aceasta pe care te bizui?
\par 5 Crezi tu că vorbele goale slujesc drept sfat și tărie în luptă? În cine ți-ai pus nădejdea, de te-ai răzvrătit împotriva mea?
\par 6 Ah, știu! ți-ai pus nădejdea în Egipt; ai luat ca ocrotitor această trestie frântă, care sparge și intră în mâna oricui se sprijină de ea. Așa este Faraon pentru toți cei ce se încred în el!
\par 7 Și dacă voi îmi ziceți: "În Domnul, Dumnezeul nostru, ne-am pus nădejdea noastră", oare nu este Acesta Dumnezeul pentru Care Iezechia a oprit închinarea pe dealurile înalte și altarele Lui le-a nimicit, zicând către Iuda și Ierusalim: "Voi vă veți închina numai înaintea acestui jertfelnic?"
\par 8 Și acum fă acest legământ cu stăpânul meu, regele Asiriei, și eu îți voi da ție două mii de cai, numai să ai tot atâți călăreți care să-i încalece.
\par 9 Cum ai putea tu să nu iei în seamă pe unul din cei mai mici slujitori ai stăpânului meu? Dar tu te duci în Egipt pentru cai și pentru călăreți.
\par 10 Și crezi tu că fără voia Domnului m-am suit eu în această țară ca să o pustiesc? Domnul mi-a spus: "Suie-te în ținutul acesta și-l pustiește!"
\par 11 Atunci Eliachim, Șebna și Ioah răspunseră lui Rabșache: "Grăiește robilor tăi în graiul arameian, că noi îl înțelegem, și nu ne grăi în limba iudaică în auzul poporului care este pe ziduri!"
\par 12 Și a zis Rabșache: "Către stăpânul tău și către tine m-a trimis stăpânul meu ca să grăiesc cuvintele acestea? Oare nu către oamenii care stau pe ziduri și curând vor fi siliți să își mănânce cu voi excrementele și să își bea urina?"
\par 13 Și a stat Rabșache și a strigat cu glas mare în limba iudaică și a zis: "Ascultați cuvintele marelui rege, regele Asiriei!
\par 14 Că iată ce vă spune regele: "Iezechia să nu vă înșele pe voi, căci el nu vă va putea scăpa;
\par 15 Și Iezechia să nu vă facă să nădăjduiți în Domnul, zicând: "Domnul ne va izbăvi și nu va da cetatea aceasta în mâna regelui Asiriei!"
\par 16 Nu ascultați pe Iezechia, că iată ce zice regele Asiriei: "Faceți pace cu mine și fiți supușii mei, și fiecare va mânca din via și din smochinul său și va bea apă din puțul său,
\par 17 Până ce voi veni ca să vă duc într-o țară ca a voastră, țară de grâu și de must, de pâine și de vii.
\par 18 Și Iezechia să nu înșele credința voastră, zicând: "Domnul ne va scăpa!" Oare dumnezeii neamurilor au scăpat fiecare țara lui din mâna regelui Asiriei?
\par 19 Unde sunt dumnezeii Hamatului și Arpadului și ai Samariei? Au scăpat ei oare Samaria din mâinile mele?
\par 20 Care din toți dumnezeii țărilor acestora au scăpat țara lor din mina mea, ca Domnul Dumnezeul vostru să elibereze Ierusalimul din mina mea?"
\par 21 Și ei au tăcut și nimic nu i-au răspuns, pentru că era porunca regelui care spunea: "Să nu-i răspundeți!"
\par 22 Atunci Eliachim, feciorul lui Hilchia, mai-marele peste casa regelui, și Șebna scriitorul și Ioah cronicarul, feciorul lui Asaf, au venit la Iezechia și, rupându-și hainele, i-au făcut cunoscut cuvintele lui Rabșache.

\chapter{37}

\par 1 Și când a auzit regele Iezechia cuvintele acestea, și-a rupt veșmintele, s-a îmbrăcat în sac și a intrat în templul Domnului.
\par 2 Și a trimis pe Eliachim, cel de peste casa sa, și pe Șebna scriitorul și pe cei mai bătrâni dintre preoți, îmbrăcați în sac, către proorocul Isaia, fiul lui Amos.
\par 3 Și au zis către dânsul: "Ziua de astăzi este zi de strâmtorare, de pedeapsă și de rușine; căci pruncii sunt gata a ieși din pântecele maicii lor, dar ele nu au putere să-i nască!
\par 4 Poate Domnul Dumnezeul tău a auzit cuvintele lui Rabșache, pe care le-a trimis regele Asiriei, stăpânul său, ca să facă de ocară pe Dumnezeul cel viu, și Domnul Dumnezeul tău poate îl va pedepsi pentru cuvintele pe care le-a auzit. Înalță dar o rugăciune pentru rămășița care se mai află!"
\par 5 Și au intrat robii regelui Iezechia la proorocul Isaia.
\par 6 Și le-a zis Isaia: "Așa veți răspunde stăpânului vostru: Așa grăiește Domnul Dumnezeu: Nu te teme de cuvintele pângăritoare pe care le-ai auzit din partea slujitorilor regelui Asiriei.
\par 7 Iată, voi pune în el un astfel de duh, că va primi o veste și se va întoarce în țara lui și acolo va cădea în ascuțișul sabiei".
\par 8 Și s-a întors Rabșache și a aflat pe regele Asiriei tăbărât la Libna, căci i se spusese că a plecat din Lachiș.
\par 9 Atunci (regele Asiriei) a aflat că Tirhaca, regele Etiopiei, pornise împotriva lui și iarăși a trimis soli către Iezechia, zicând:
\par 10 "Așa veți zice lui Iezechia, regele lui Iuda: Să nu te încrezi în Dumnezeul tău, și să nu te amăgești, zicând: Ierusalimul nu va fi dat în mâinile regelui Asiriei.
\par 11 Tu ai aflat ceea ce au făcut regii Asiriei tuturor țărilor, cum le-au nimicit și numai tu ai scăpat!
\par 12 Oare dumnezeii lor au izbăvit popoarele pe care le-au distrus părinții mei: Gozanul, Haranul, Rețeful și pe fiii lui Eden din Telasar?
\par 13 Unde este regele Hamatului, al Arpadului, cel al cetății Sefarvaim, al Henei și al Ivei?"
\par 14 Atunci Iezechia a luat scrisoarea din mina trimișilor și a citit-o. Apoi el a intrat în templul Domnului și a întins-o desfăcută înaintea Domnului.
\par 15 Și s-a rugat Iezechia către Domnul, zicând:
\par 16 "Doamne Savaot, Dumnezeul lui Israel, Care stai pe heruvimi, numai Tu singur ești Dumnezeu al tuturor regatelor de pe pământ. Tu ai făcut cerul și pământul.
\par 17 Pleacă, Doamne, urechea Ta și deschide, Doamne, ochii Tăi și vezi și ia aminte la cuvintele lui Sanherib, pe care le-a trimis ca să facă de batjocură pe Dumnezeul cel viu.
\par 18 Cu adevărat, Doamne, regii Asiriei au nimicit toate neamurile și țările lor;
\par 19 Și pe dumnezeii lor i-au ars cu foc, că ei nu sunt dumnezei, ci lucruri de mâini omenești: lemn și piatră; pentru aceea ei i-au nimicit.
\par 20 Și acum, Doamne, Dumnezeul nostru, izbăvește-ne din mâna lui ca să știe toate împărățiile pământului că Tu singur ești Domnul nostru!"
\par 21 Și a trimis Isaia, fiul lui Amos, către Iezechia, zicând: "Așa zice Domnul Dumnezeul lui Israel, către Care te-ai rugat cu privire la Sanherib, regele Asiriei;
\par 22 Iată hotărârea pe care a rostit-o împotriva lui: "Te disprețuiește și își bate joc de tine fecioara, fiica Sionului; în spatele tău clatină din cap fiica Ierusalimului!
\par 23 Pe cine ai pângărit și îi făcut de râs și împotriva cui ai ridicat glasul și sus ai înălțat ochii tăi? împotriva Sfântului lui Israel!
\par 24 Prin mina servilor tăi ai hulit pe Domnul meu și ai zis: Cu carele mele multe voi urca pe vârfurile munților, pe cele mai înalte piscuri ale Libanului! Voi tăia cedrii cei falnici și cei mai de seamă dintre chiparoși și voi ajunge pe cele mai înalte culmi cu păduri dese.
\par 25 Că eu sunt cel ce am săpat fântâni și am băut apă și am secat sub pașii mei toate pâraiele Egiptului!
\par 26 Oare nu auzi tu? Din vremi străvechi am pregătit aceasta; din veac le-am hotărât și acum le aduc la îndeplinire! Tu voiai să prefaci în ruină cetățile cele întărite.
\par 27 Cei ce locuiau în ele erau fără putere, înspăimântați și uluiți. Ca iarba câmpului erau ei, ca frageda verdeață, ca iarba de pe acoperișuri înainte ca paiul ei să fi fost crescut.
\par 28 Știu când te scoli și când te culci, toate faptele tale Îmi sunt cunoscute.
\par 29 Întărâtarea ta împotriva Mea, trufia ta au ajuns până la urechile Mele. De aceea voi pune belciug în nările tale și frâul Meu buzelor tale și te voi întoarce pe calea pe care ai venit!
\par 30 Și pentru tine, acesta va fi semnul: anul acesta mâncați din pâinea ce crește pe ogoare, în anul al doilea, din ceea ce crește de la sine, iar în al treilea an semănați, secerați, sădiți vii și mâncați din roadele lor.
\par 31 Și rămășița care va fi scăpat din casa lui Iuda își va înfige rădăcini în jos și va face roade în sus.
\par 32 Că din Ierusalim va ieși o rămășită și din muntele Sionului cei scăpați cu viață. Râvna Domnului Savaot va face aceasta.
\par 33 Pentru aceasta, așa zice Domnul către regele Asiriei: Nu va intra în această cetate și nu va arunca nici o săgeată. Nu va porni împotriva ei cu scut și nu o va înconjura cu valuri.
\par 34 Pe calea pe care a venit se va întoarce și nu va intra în cetatea aceasta, zice Domnul.
\par 35 Apăra-voi cetatea aceasta și o voi scăpa pentru Mine și pentru David, sluga Mea!"
\par 36 Și a ieșit îngerul Domnului și a bătut în tabăra Asiriei o sută și cincizeci de mii; iar dimineața, la sculare, toți erau morți.
\par 37 Atunci Sanherib, regele Asiriei, a ridicat tabăra și a plecat și s-a oprit la Ninive.
\par 38 Și pe când el se închina în templul lui Nisroc, dumnezeul său, Adramelec și Șareser, feciorii lui, l-au lovit cu sabia și au fugit în ținutul Ararat. Iar în locul lui, a domnit fiul său Asarhadon.

\chapter{38}

\par 1 În vremea aceea Iezechia s-a îmbolnăvit de moarte. Și a intrat la el Isaia, fiul lui Amos, și i-a zis: "Așa grăiește Domnul: Pune rânduială în casa ta, că nu vei mai trăi, ci vei muri".
\par 2 Atunci s-a întors Iezechia cu fața la perete și s-a rugat Domnului:
\par 3 "O, Doamne! Adu-ți aminte că am umblat înaintea Ta întru credincioșie și cu inimă curată, săvârșind ceea ce este plăcut înaintea ochilor Tăi!" Și a izbucnit Iezechia în hohote de plâns.
\par 4 Și a fost cuvântul Domnului către Isaia, zicând:
\par 5 "Du-te și spune lui Iezechia: Așa grăiește Domnul Dumnezeul lui David, tatăl tău: Ascultat-am rugăciunea ta, văzut-am lacrimile tale, iată voi adăuga la viața ta încă cincisprezece ani
\par 6 Și din mâna regelui Asiriei te voi izbăvi pe tine și cetatea aceasta și o voi ocroti".
\par 7 Și iată semnul care ți se va da fie de la Domnul că El Își va împlini cuvântul Său:
\par 8 "Iată voi întoarce umbra cu atâtea linii pe care soarele le-a străbătut pe ceasornicul lui Ahaz, să zic cu zece linii". Și soarele s-a dat înapoi cu zece linii pe care el le străbătuse.
\par 9 Rugăciunea lui Iezechia, regele lui Iuda, când a căzut bolnav și s-a tămăduit de boala lui:
\par 10 "Atunci eu am zis: Mă duc la amiaza zilelor mele, la porțile locuinței morților voi fi ținut pentru restul anilor mei.
\par 11 Nu voi mai vedea pe Domnul în pământul celor vii; și nu voi mai privi pe nimeni dintre locuitorii lumii.
\par 12 Casa mea este smulsă și dusă departe de mine, ca o colibă de ciobani. Îmi simt firul vieții tăiat ca de un țesător care m-ar rupe din țesătura lui. De dimineață până seara, Tu ai sfârșit cu mine.
\par 13 Strig până dimineața. Ca un leu (boala) îmi sfărâmă oasele mele! De dimineață până seara, Tu ai sfârșit cu mine.
\par 14 țip cumplit ca o rândunică, gem ca o porumbiță. Ochii mei slăbesc, uitându-se în sus. Doamne, sunt în mare cumpănă, nu mă lăsa!
\par 15 Ce să mai grăiesc! El mi-a dat de știre și a făcut! Sfârși-voi firul vieții mele, aducându-mi aminte de amărăciunea sufletului meu!
\par 16 Doamne, prin îndurarea Ta se bucură omul de viață, prin ea mai am și eu suflare; Tu mă tămăduiești și-mi dai iarăși viață!
\par 17 Iată că boala mea se schimbă în sănătate. Tu ai păzit viața mea de adâncul mistuitor! Tu ai aruncat înapoia Ta toate păcatele mele!
\par 18 Că locuința morților nu Te va lăuda și moartea nu Te va preaslăvi; cei ce se coboară în mormânt nu mai nădăjduiesc în credincioșia Ta.
\par 19 Cel viu, cel viu Te laudă, ca mine astăzi; părinții învață pe copiii lor credincioșia Ta.
\par 20 Domnul să ne mântuiască și vom cânta din harpă în toate zilele vieții noastre înaintea templului Domnului!"
\par 21 Și Isaia a adus o turtă de smochine și a pus-o deasupra bubei și Iezechia s-a vindecat.
\par 22 Și Iezechia a întrebat: "După care semn voi ști că voi intra în templul Domnului?"

\chapter{39}

\par 1 În vremea aceea, Merodac-Baladan, fiul lui Baladan, regele Babilonului, a trimis scrisori și un dar lui Iezechia, auzind că a fost bolnav și s-a făcut sănătos.
\par 2 Și s-a bucurat pentru ele Iezechia și a arătat solilor vistieria, argintul, aurul, miresmele și untdelemnul cel bun și toată strânsura lui de arme și tot ceea ce se afla în cămările lui. și n-a rămas nimic în casa lui și în tot cuprinsul stăpânirii lui pe care Iezechia să nu-l fi arătat.
\par 3 Atunci a zis proorocul Isaia către regele Iezechia: "Ce au zis oamenii aceștia și de unde au venit ei la tine?" Și a răspuns Iezechia: "Au venit dintr-o țară depărtată, din Babilon!"
\par 4 Și a mai întrebat: "Ce au văzut în casa ta?" Și a zis Iezechia: "Au văzut toate câte sunt în casa mea; și n-a rămas nimic în vistieriile mele pe care să nu-l fi arătat".
\par 5 Și a zis Isaia către Iezechia: "Ascultă ceea ce grăiește Domnul Savaot!
\par 6 Iată vin zile, când tot ceea ce au agonisit părinții tăi până astăzi va fi dus în Babilon și nu va rămâne nimic, așa zice Domnul.
\par 7 Și din feciorii care vor ieși din tine și îi vei naște, vor lua. Și vor fi eunuci la curtea regelui din Babilon".
\par 8 Și a zis Iezechia către Isaia: "Bun este cuvântul Domnului pe care l-ai grăit!" Căci, se gândea el: "Va fi pace și liniște în timpul vieții mele!"

\chapter{40}

\par 1 "Mângâiați, mângâiați pe poporul Meu", zice Dumnezeul vostru.
\par 2 "Dați curaj Ierusalimului și strigați-i că munca de rob a luat sfârșit, fărădelegea sa a fost ispășită și că a luat pedeapsă îndoită din mâna Domnului pentru păcatele sale".
\par 3 Un glas strigă: "În pustiu gătiți calea Domnului, drepte faceți în loc neumblat cărările Dumnezeului nostru.
\par 4 Toată valea să se umple și tot muntele și dealul să se plece; și să fie cele strâmbe, drepte și cele colțuroase, căi netede.
\par 5 Și se va arăta slava Domnului și tot trupul o va vedea căci gura Domnului a grăit".
\par 6 Un glas zice: "Strigă!" Și eu zic: "Ce să strig?" Tot trupul este ca iarba și toată mărirea lui, ca floarea câmpului!
\par 7 Se usucă iarba, floarea se veștejește, că Duhul Domnului a trecut pe deasupra. Poporul este ca iarba.
\par 8 Iarba se usucă și floarea se veștejește, dar cuvântul Dumnezeului nostru rămâne în veac!
\par 9 Suie-te pe munte înalt, cel ce binevestești Ierusalimului, ridică glasul tău cu putere, cel ce binevestești Ierusalimului, înalță glasul și nu te teme, zi cetăților lui Iuda: "Iată Dumnezeul vostru!"
\par 10 Că Domnul Dumnezeu vine cu putere și brațul Lui supune tot. Iată că prețul biruinței Lui este cu El și rodul izbânzii merge înaintea Lui.
\par 11 El va paște turma Sa ca un Păstor și cu brațul Său o va aduna. Pe miei îi va purta la sânul Său și de cele ce alăptează va avea grijă.
\par 12 Cine a măsurat apele cu pumnul și cine a măsurat pământul cu cotul? Cine a pus pulberea pământului în baniță și cine a cântărit munții și văile cu cântarul?
\par 13 Cine a căutat în adânc Duhul Domnului și cine L-a sfătuit pe El?
\par 14 De la cine a luat El sfat ca să judece bine și să învețe căile dreptății, să învețe știința și calea înțelepciunii să o cunoască?
\par 15 Iată, neamurile sunt ca o picătură de apă pe marginea unei găleți, ca un fir de pulbere într-un cântar. Iată insulele care cântăresc cât un fir de praf.
\par 16 Libanul nu ajunge pentru aprinderea focului și dobitoacele pentru arderi de tot.
\par 17 Toate popoarele sunt ca o nimica înaintea Lui; ele prețuiesc înaintea Lui cât o suflare.
\par 18 Cu cine veți asemăna voi pe Cel Preaputernic și unde veți găsi altul asemenea Lui?
\par 19 Chipul cel turnat este turnat de un făurar, argintarul îl îmbracă cu aur și-l înfrumusețează cu lănțișoare.
\par 20 Săracul, care nu poate oferi mult, alege un lemn care nu putrezește; își caută un meșter iscusit ca să facă un idol care să nu se clatine.
\par 21 Nu știți voi, oare, n-ați auzit, nu vi s-a spus oare de la început, n-ați înțeles voi ce vă învață întemeierea lumii?
\par 22 El stă în scaun deasupra cercului pământului; pe locuitori îi vede ca pe lăcuste; El întinde cerul ca un văl ușor și îl desface ca un cort de locuit.
\par 23 El preface în nimic pe căpetenii; pe judecătorii pământului îi nimicește.
\par 24 Abia sunt sădiți, abia sunt semănați, abia a prins rădăcini tulpina lor în pământ; El suflă peste ele și le usucă și vijelia le spulberă ca pe pleavă.
\par 25 Cu cine Mă asemănați voi ca să-i fiu asemenea?, zice Sfântul.
\par 26 Ridicați ochii în sus și priviți: Cine le-a zidit pe toate acestea? - Cel ce scoate oștirea lor cu număr și pe toate pe nume le cheamă! Celui Atotputernic și cu mare vârtute nici una nu-I scapă!
\par 27 Pentru ce zici tu, Iacove, pentru ce grăiești, Israele: "Calea mea este ascunsă Domnului, dreptul meu este trecut cu vederea de Dumnezeul meu?"
\par 28 Nu știi tu, sau n-ai aflat tu că Domnul este Dumnezeu veșnic, Care a făcut marginile pământului, Care nu obosește și nici nu Își sleiește puterea? Că înțelepciunea Lui este nemărginit de adâncă?
\par 29 El dă tărie celui obosit și celui slab îi dă putere mare.
\par 30 Cei tineri se obosesc, își risipesc puterile și vitejii luptători vor putea să se clatine;
\par 31 Dar cei ce nădăjduiesc întru Domnul vor înnoi puterea lor, le vor crește aripi ca ale vulturului; vor alerga și nu-și vor slei puterea, vor merge și nu se vor obosi.

\chapter{41}

\par 1 Tăceți înaintea Mea, ostroave, și ascultați-Mă; popoarele să-și împrospăteze puterea, să vină mai lângă Mine și să grăiască; apoi să intrăm la judecată!
\par 2 Cine a ridicat din Răsărit pe acela pe care biruința îl întâmpină pas cu pas? Cine i-a dat în stăpânire neamuri și i-a supus regi? Cu sabia lui în pulbere îi preface, și cu arcul îi risipește ca pe pleava cea măruntă.
\par 3 El îi urmărește și trece în pace pe căi pe unde n-au mai călcat picioarele lui.
\par 4 Cine a făcut aceasta și cine a pus-o la cale? Cel ce dintru început cheamă neamurile; Eu, Domnul Care sunt cel dintâi și voi fi cu cei din urmă.
\par 5 Ostroavele Îl văd și sunt cuprinse de spaimă, marginile pământului tremură, se apropie, vin, intră la judecată!
\par 6 Fiecare se ajută unul pe altul și-și zic: "Curaj!"
\par 7 Turnătorul îmbărbătează pe argintar și cel ce bate aurul, pe cel care bate pe nicovală, zicând: "Îmbinarea este bună". Și țintuiește idolul în cuie ca să nu se clatine.
\par 8 Dar tu, Israele, sluga Mea, Iacove, pe care te-am ales, sămânța lui Avraam, iubitul Meu!...
\par 9 Pe tine care te-am smuls din cele mai depărtate margini ale pământului și te-am chemat din cele mai depărtate colțuri, și ți-am zis: Tu ești robul Meu, pe tine te-am ales și nu te-am lepădat;
\par 10 Nu te teme, că Eu sunt cu tine, nu privi cu îngrijorare, că Eu sunt Dumnezeul tău. Eu îți dau tărie și te ocrotesc și dreapta Mea cea tare te va sprijini.
\par 11 Iată că se vor rușina și de ocară se vor face toți cei ce sunt aprinși împotriva ta; toți vor fi nimiciți și vor pieri cei ce se fac vrăjmași ai tăi!
\par 12 Căuta-vei și nu vei găsi pe cei ce te urăsc pe tine și ca o nimica vor fi cei ce vor să se lupte cu tine.
\par 13 Că Eu sunt Dumnezeul tău, Eu întăresc dreapta ta și îți zic ție: "Nu te teme, căci Eu sunt ajutorul tău!"
\par 14 Nu-îi fie frică, vierme al lui Iacov, viermișor al lui Israel, Eu sunt ajutorul tău, zice Domnul, Mântuitorul tău și Sfântul lui Israel.
\par 15 Iată voi face din tine o grapă cu dinți, ascuțită și nouă. Vei merge peste munți și-i vei preface în pulbere și văile în pleavă măruntă.
\par 16 Tu le vei vântura, vântul le va lua și vijelia le va risipi. Iar tu te vei bucura întru Domnul și întru Sfântul lui Israel te vei preamări!
\par 17 Cei săraci și lipsiți caută apă, dar nu o găsesc; limba lor este uscată de sete, Eu, Domnul lor, îi voi auzi; Eu, Dumnezeul lui Israel, nu-i voi părăsi!
\par 18 Pe dealuri înalte voi da drumul la râuri și la izvoare în mijlocul văilor, pustiul îl voi preface în iaz și pământul uscat în pâraie de apă!
\par 19 Sădi-voi în pustiu: cedri, salcâmi, mirți și măslini și în lacuri neumblate: chiparoși, platani și ienuperi laolaltă,
\par 20 Ca să vadă și să-și dea seama, să cerceteze și să priceapă cu toții că mâna Domnului a făcut acestea și că Sfântul lui Israel le-a zidit!
\par 21 Veniți și vă apărați pricina voastră, zice Domnul; apropiați-vă cu dovezile voastre, zice regele lui Iacov.
\par 22 Să se apropie și să ne spună mai dinainte ceea ce va fi! Vremea cea străveche, așa cum ne-au dat de știre, cu de-amănuntul o vom cerceta și viitorul pe care-l proorocesc vom vedea ce este.
\par 23 Vesti]i cele ce vor fi în vremile mai de pe urmă, ca să știm că sunteți dumnezei! Haidem! Bine sau rău, face]i ceva ca să ne putem încerca puterea!
\par 24 Dar iată că lucrarea voastră este nimic și nimic sânte]i și voi, urâciune este a vă alege!
\par 25 De la miazănoapte l-am chemat ca să vină, de la răsărit l-am chemat pe nume. El a călcat în picioare pe satrapi ca pe noroi, cum calcă olarul lutul.
\par 26 Cine l-a descoperit odinioară ca să-l știm și cu mult înainte ca să zicem: "Este adevărat?" Dar nimeni n-a descoperit nimic, nimeni n-a vestit nimic și nimeni n-a auzit cuvintele voastre.
\par 27 Eu Cel dintâi am zis Sionului: "Iată-i, iată-i!" și Ierusalimului am adus veste nouă.
\par 28 Privesc și nu este nimeni; printre ei nu se află nici un profet. Eu îi întreb: "De unde vine el?" Dar ei nu răspund nimic!
\par 29 Drept aceea, toți sunt nimic, lucrările lor deșertăciune, idolii lor sunt vânare de vânt!

\chapter{42}

\par 1 Iată Sluga Mea pe Care o sprijin, Alesul Meu, întru Care binevoiește sufletul Meu. Pus-am peste El Duhul Meu și El va propovădui popoarelor legea Mea.
\par 2 Nu va striga, nici nu va grăi tare, și în piețe nu se va auzi glasul Lui.
\par 3 Trestia frântă nu o va zdrobi și feștila ce fumegă nu o va stinge. El va propovădui legea Mea cu credincioșie;
\par 4 El nu va fi nici obosit, nici sleit de puteri, până ce nu va fi așezat legea pe pământ; căci învățătura Lui toate ținuturile o așteaptă.
\par 5 Așa grăiește Domnul cel Atotputernic, Care a făcut cerurile și le-a întins, Care a întărit pământul și cele de pe el, Care a dat suflare poporului de pe el și duh celor ce umblă pe întinsul lui:
\par 6 "Eu, Domnul, Te-am chemat întru dreptatea Mea și Te-am luat de mână și Te-am ocrotit și Te-am dat ca legământ al poporului Meu, spre luminarea neamurilor;
\par 7 Ca să deschizi ochii celor orbi, să scoți din temniță pe cei robiți și din adâncul închisorii pe cei ce locuiesc întru întuneric.
\par 8 Eu sunt Domnul și acesta este numele Meu. Nu voi da nimănui slava Mea și nici chipurilor cioplite cinstirea Mea".
\par 9 Cele proorocite altădată s-au împlinit și altele mai noi vă vestesc; înainte ca să ia ființă vi le dau de știre.
\par 10 Cântați Domnului cântare nouă, cântați în strune laudele Lui până la marginile pământului! Marea să se zbuciume cu tot ce este în ea, ostroavele și locuitorii lor!
\par 11 Pustiul și cetățile lui să înalțe glas, și satele în care are sălaș Chedar! Locuitorii din Sela să chiuie de veselie; și din vârfurile munților să strige de bucurie!
\par 12 Să preaslăvească pe Domnul și lauda Lui s-o vestească în depărtatele ostroave.
\par 13 Domnul iese ca un viteaz, ca un războinic Își aprinde râvna Lui; strigă puternic, un strigăt de război. Împotriva vrăjmașilor Lui El luptă ca un viteaz!
\par 14 Tăcut-am multă vreme, stat-am liniștit și mi-am stăpânit tăcerea; acum, ca o femeie care naște, voi suspina, voi striga și voi răsufla!
\par 15 Munții îi voi pustii, la fel și dealurile, toată verdea]a lor o voi usca; pâraiele le voi preface în văi uscate și bălțile fără apă le voi lăsa.
\par 16 Îndrepta-voi pe cei orbi pe drumuri pe care nu le cunosc, pe poteci neștiute îi voi povățui; întunericul îl voi preface înaintea lor în lumină și povârnișurile în câmpii întinse. Acestea sunt făgăduințele Mele pe care le voi împlini și cu vederea nu le voi trece.
\par 17 Să dea înapoi și să se rușineze cei ce își pun nădejdea în idoli, cei ce zic chipurilor turnate: "Voi sunteți dumnezeii noștri!
\par 18 Surzilor, auziți; orbilor, priviți, vedeți!
\par 19 Cine este orb, fără numai sluga Mea? Cine este surd ca trimisul Meu? Cine este orb ca cel de un neam cu Mine și surd ca Slujitorul Domnului?
\par 20 Tu multe ai văzut fără să te uiți cu luare-aminte, urechile ti-au fost deschise, dar n-ai auzit.
\par 21 Binevoit-a Domnul întru dreptatea Lui ca legea Lui s-o facă mare și măreață.
\par 22 Dar poporul Lui este jefuit și pustiit; închiși toți în peșteri și ascunși în temnițe. Prădați au fost și nimeni nu i-a scăpat, jefuiți și nimeni n-a zis: "Dați înapoi!"
\par 23 Cine dintre voi va pleca urechea la acestea, va fi cu luare-aminte și va asculta la cele ce vor să fie?
\par 24 Cine a dat pe Iacov jafului și pe Israel jefuitorilor? Oare nu Domnul, împotriva Căruia noi am păcătuit, ale Cărui căi n-am voit să le urmăm și a Cărui lege n-am ascultat-o?
\par 25 El a vărsat asupra lor iuțimea mâniei Lui și furiile războiului. Văpaia i-a cuprins și n-au priceput; arși au fost și n-au luat seama.

\chapter{43}

\par 1 Și acum așa zice Domnul, Ziditorul tău, Iacove, și Creatorul tău, Israele: "Nu te teme, căci Eu te-am răscumpărat și te-am chemat pe nume, al Meu ești!
\par 2 Dacă tu vei trece prin ape, Eu sunt cu tine și în valuri tu nu vei fi înecat. Dacă vei trece prin foc, nu vei fi ars și flăcările nu te vor mistui.
\par 3 Că Eu sunt Domnul Dumnezeul tău, Sfântul lui Israel, Mântuitorul. Eu dau Egiptul preț de răscumpărare pentru tine, Etiopia și Saba în locul tău;
\par 4 Fiindcă tu ești de preț în ochii Mei și de cinste și te iubesc; voi da neamurile în locul tău și popoarele în locul sufletului tău.
\par 5 Nu te teme, că Eu sunt cu tine! De la răsărit voi aduce seminția ta și de la apus te voi strânge pe tine.
\par 6 Voi zice către miazănoapte: "Dă-Mi-i" și către miazăzi: "Nu-i opri!" Aduceți pe fiii Mei din ținuturi depărtate și pe fiicele Mele de la marginile pământului;
\par 7 Pe toți acei care poartă numele Meu și pentru slava Mea i-am creat, i-am zidit și i-am pregătit!
\par 8 Să vină poporul cel orb care are ochi și cel surd care are urechi!
\par 9 Neamurile toate laolaltă să se adune și să se strângă popoarele! Care dintre ele ne-au dat de știre aceasta și ne-au făcut proorocii? Să-și aducă martorii și să dovedească, să audă toți și să zică: "Adevărat!"
\par 10 Voi sunteți martorii Mei, zice Domnul, și Sluga pe care am ales-o, ca să știți, să credeți și să pricepeți că Eu sunt: înainte de Mine n-a fost Dumnezeu și nici după Mine nu va mai fi!
\par 11 Eu, Eu sunt Domnul și nu este izbăvitor afară de Mine!
\par 12 Eu sunt Cel ce am vestit, Cel ce am izbăvit și Cel ce am prezis și nu sunt străin la voi. Voi sunteți martorii Mei, zice Domnul.
\par 13 Eu sunt Dumnezeu din veșnicie și de aici încolo Eu sunt! Nimeni nu poate să iasă de sub puterea Mea și ceea ce fac Eu, cine poate strica?"
\par 14 Așa zice Domnul, Izbăvitorul vostru, Sfântul lui Israel: "Pentru voi trimit prăpăd la Babilon, ca să-i pun pe toți pe fugă, pe acești Caldei așa de mândri pe corăbiile lor.
\par 15 Eu sunt Domnul, Sfântul vostru, Ziditorul lui Israel, Împăratul vostru!"
\par 16 Așa zice Domnul, Cel ce croiește drum pe mare și cărare pe întinsele ape;
\par 17 Cel care scoate carele de război și caii, oștirea ți căpeteniile, ca să se culce la pământ și să nu se mai scoale și să se stingă ca o feștilă de opaiț;
\par 18 Nu vă mai amintiți de întâmplările trecute și nu mai luați în seamă lucrurile de altădată".
\par 19 Iată că Eu fac un lucru nou, el dă muguri; nu-l vedeți voi oare? Croi-voi în deșert o cale, în loc uscat izvoare de apă.
\par 20 Pe Tine Te vor preaslăvi fiarele câmpului, șacalii și struții, că Tu ai izvorât apă în pustiu, șuvoaie de apă în pământ neumblat, ca să adăpi pe poporul Meu cel ales;
\par 21 Poporul pe care l-am făcut pentru Mine, ca să Mă preaslăvească întru laude.
\par 22 Dar tu nu M-ai chemat, Iacove, și tu nu te-ai ostenit pentru Mine, Israele!
\par 23 Tu nu Mi-ai junghiat nici măcar o oaie ca ardere de tot și cu jertfă sângeroasă tu nu M-ai preaslăvit. Eu nu te-am supărat cerând prinoase și nu te-am împovărat cu jertfe de tămâie.
\par 24 Tu n-ai cumpărat pe bani miresme pentru Mine și de grăsimea jertfelor tale tu nu M-ai săturat, ci M-ai copleșit cu păcatele tale și cu fărădelegile tale tu M-ai chinuit.
\par 25 Eu, Eu sunt Acel Care șterge păcatele tale și nu tși mai aduce aminte de fărădelegile tale.
\par 26 Adu-Mi aminte ca să judecăm împreună, fă tu însuți socoteala ca să te dezvinovățești:
\par 27 Tatăl tău dintâi a păcătuit și urmașii tăi și-au bătut joc de Mine;
\par 28 Căpeteniile tale au pângărit altarul Meu. Pentru aceasta am dat pe Iacov pierzării și pe Israel spre bătaie de joc!"

\chapter{44}

\par 1 Și acum ascultă Iacove, sluga Mea și Israele, pe care te-am ales!
\par 2 Așa zice Domnul, Făcătorul și Ziditorul tău din pântecele maicii tale, și Ocrotitorul tău: "Nu te teme, sluga Mea Iacov și tu Israele, pe care te-am ales.
\par 3 Că Eu voi vărsa apă peste pământul însetat și pâraie de apă în ținut uscat. Vărsa-voi din Duhul Meu peste odrasla ta și binecuvântarea Mea peste mlădițele tale.
\par 4 Și vor odrăsli ca iarba pe malul pâraielor și ca pajiștile de-a lungul apelor curgătoare!"
\par 5 Unul va zice: "Eu sunt al Domnului!" Altul se va numi cu numele lui Iacov. Unul va scrie cu mâna lui: "Sunt al Domnului" și va vrea să-și dea numele de Israel!
\par 6 Așa zise Domnul, Regele lui Israel și Izbăvitorul său, Domnul Savaot: "Eu sunt Cel dintâi și Cel de pe urmă și nu este alt dumnezeu afară de Mine!
\par 7 Cine este ca Mine să vină lângă Mine, să grăiască, să proorocească și să se măsoare cu Mine! Cine a vestit de la început viitorul? Ceea ce se va întâmpla, cine poate să le prevestească?
\par 8 Nu vă temeți, nici nu vă spăimântați! N-am arătat Eu odinioară și n-am vestit, când v-am luat pe voi de martori? Este oare un alt dumnezeu afară de Mine? Este un alt adăpost ca Mine?
\par 9 Toți făcătorii de idoli nu sunt nimic și cele mai alese lucrări ale lor nu slujesc la nimic. Martorii lor nu văd nimic și, spre rușinea lor, nici nu înțeleg nimic.
\par 10 Cine a făcut un dumnezeu și a turnat un idol fără să tragă un folos din aceasta?
\par 11 Iată toți cinstitorii lor se rușinează; meșteșugarii nu sunt decât oameni! Să se adune toți și să se apropie! Ei tremură laolaltă și simt mare rușine!
\par 12 Fierarul ascute o daltă și dă chip lucrului său cu cărbuni aprinși. Alcătuiește idolul cu lovituri de ciocan și-i dă chip cu puterea brațului său. Lui îi este foame și, sleit de puteri, el rabdă de sete și este tare obosit.
\par 13 Lemnarul întinde sfoara, face un semn cu plumbul. El lucrează cu sculele lui și măsoară cu compasul. El face lucrul lui după chipul unui om, după frumoasa înfățișare a unui pământean, ca să fie așezat într-o casă.
\par 14 El și-a tăiat un cedru, sau a luat chiparos sau stejar, pe care și-i alesese dintre copacii pădurii, sau a plantat un cedru pe care ploaia l-a făcut să crească.
\par 15 Omul se slujește de ei pentru aprins focul și îi ia să se încălzească. El îi arde ca să coacă pâinea, ba mai mult, tot din el face și un dumnezeu și se închină la el, face un idol pe care îl cinstește.
\par 16 El a ars jumătate din lemne, a fript pe jeratic carne pe care o mănâncă și se satură. Se mai încălzește și zice: "Mi-e cald! Simt văpaia lui!"
\par 17 Și cu ce a rămas, el face un dumnezeu, un idol pe care îl cinstește și căruia i se închină și căruia se roagă zicând: "Izbăvește-mă, că tu ești dumnezeul meu!"
\par 18 Ei nu-și dau seama și nici nu pricep că ochii lor sunt închiși și nu pot să vadă și inima lor este împietrită și nu pot să înțeleagă.
\par 19 Cu toate acestea el nu-și face socoteală în inima sa, că este simplu și fără pricepere, și nu zice: "Jumătate l-am pus pe foc și am copt pâine, pe cărbuni am fript carne și am mâncat-o, iar cu cealaltă jumătate care a mai rămas, voi face un idol urâcios și mă voi închina la un trunchi de copac".
\par 20 El se hrănește cu năluci, inima lui înșelată l-a dus la rătăcire. El nu-și mântuiește sufletul său și nu zice: "Oare ce am eu în mina mea nu este o momeală?"
\par 21 Adu-ți aminte despre aceasta, Iacove, Israele, că tu ești sluga Mea! Te-am făcut să-Mi fii Mie slugă, Israele, Eu nu te voi uita!
\par 22 Risipit-am păcatele tale ca pe un nor și fărădelegile tale ca pe o negură. Întoarce-te către Mine, că Eu te-am mântuit!
\par 23 Cântați voi ceruri, că Domnul a făcut aceasta; răsunați adâncuri ale pământului; munților, tresăltați de bucurie, voi toți copacii pădurii, cântați, că a răscumpărat Domnul pe Iacov și în Israel Și-a dat pe față slava Sa!
\par 24 Așa grăiește Domnul, Izbăvitorul tău și Care te-a zidit din sânul maicii tale: "Eu sunt Domnul, Care a zidit lumea; singur am făcut cerurile, Eu am întărit pământul, și cine Mi-a fost într-ajutor?
\par 25 Eu zădărnicesc semnele mincinoșilor și pe ghicitori îi fac să fie nebuni, rușinez pe cei înțelepți și înțelepciunea lor o prefac în nebunie.
\par 26 Eu sunt Domnul Care întărește cuvântul slugilor Mele și împlinește sfatul trimișilor Mei. Eu am zis Ierusalimului: "Va fi locuit" și cetăților lui Iuda: "Zidite vor fi". Și Eu le voi ridica din dărâmături!
\par 27 Și adâncului i-am zis: "Seacă!" Iată, îți voi lăsa râurile fără apă.
\par 28 Și am zis despre Cirus: "El este păstorul Meu, el va împlini toate voile Mele". Și despre Ierusalim am zis: "Să fie rezidit și templul să fie ridicat din temelii!"

\chapter{45}

\par 1 Așa zice Domnul unsului Său Cirus, pe care îl ține de mâna lui cea dreaptă, ca să doboare neamurile înaintea lui și ca să dezlege cingătorile regilor, să deschidă porțile înaintea lui și ca ele să nu mai fie închise:
\par 2 "Eu voi merge înaintea ta și drumurile cele muntoase le voi netezi, voi zdrobi porțile cele de aramă și zăvoarele cele de fier le voi sfărâma.
\par 3 Și îți voi da ție vistierii ascunse, bogății îngropate în pământ, ca să știi că Eu sunt Domnul Cel Care te-a chemat pe nume, Eu sunt Dumnezeul lui Israel.
\par 4 Pentru sluga Mea Iacov și pentru Israel, alesul Meu, te-am chemat pe nume și un nume de cinste i-am dat fără ca tu să Mă știi.
\par 5 Eu sunt Domnul și nimeni altul! Afară de Mine nu este Dumnezeu. Eu te-am încins fără ca tu să Mă cunoști.
\par 6 Ca să se știe de la răsărit și până la apus că nu este nimic afară de Mine! Eu sunt Domnul și nimeni altul!
\par 7 Eu întocmesc lumina și dau chip întunericului, Cel ce sălășluiește pacea și restriștei îi lasă cale: Eu sunt Domnul Care fac toate acestea.
\par 8 Picurați rouă de sus, voi ceruri, și norii să reverse în ploaie dreptatea! Pământul să se deschidă și să odrăslească mântuirea și dreptatea să dea mlădițe laolaltă: Eu, Domnul, am zidit toate acestea!
\par 9 Vai de cel ce se ceartă cu Ziditorul său, ciob printre hârburile de pământ! Oare lutul zice olarului: "Ce faci tu?" Și lucrul către meșter: "Tu nu ești iscusit!"
\par 10 Vai de cel ce zice către părinte: "Pentru ce dai naștere?" și femeii: "Pentru ce ai copii?"
\par 11 Așa zice Domnul, Sfântul lui Israel și Ziditorul său: "Îndrăzniți voi oare să Mă întrebați despre cele viitoare și să dați poruncă lucrului mâinilor Mele?
\par 12 Eu am făcut pământul și omul de pe el Eu l-am zidit. Eu cu mâinile am întins cerurile și la toată oștirea lor Eu îi dau poruncă.
\par 13 Eu l-am ridicat întru dreptatea Mea și toate căile lui le voi netezi. El va zidi cetatea Mea și va libera pe robii Mei, fără răscumpărare și fără daruri", zice Domnul Savaot.
\par 14 Așa zice Domnul: "Bogățiile Egiptului și câștigurile Etiopiei și ale Sabeenilor celor înalți la stat vor trece la tine și ai tăi vor fi; în lanțuri iți vor sluji ție și vor cădea înaintea ta și rugându-se ție vor zice: "Numai tu ai un Dumnezeu tare, și nu este alt dumnezeu afară de El.
\par 15 Cu adevărat Tu ești Dumnezeu ascuns, Dumnezeul lui Israel Cel izbăvitor!
\par 16 Cei care se aprindeau împotriva Ta vor fi rușinați și umiliți, făcătorii de idoli se vor face de râs.
\par 17 Israel va fi izbăvit de Domnul cu mântuire veșnică. Voi nu veți fi rușinați, nici umiliți în vecii vecilor!"
\par 18 Că așa zice Domnul, Care a făcut cerurile, Dumnezeu, Care a întocmit pământul, l-a făcut și l-a întărit; și nu în deșert l-a făcut, ci ca să fie locuit: "Eu sunt Domnul și nu este altul!"
\par 19 N-am grăit acestea într-ascuns, undeva în vreun colț întunecos al pământului; și n-am zis fără rost neamului lui Iacov: "Căutați-Mă!" Eu sunt Domnul Cel ce grăiește drept și spune adevărul!
\par 20 Adunați-vă, veniți, apropiați-vă laolaltă, cei rămași cu viață dintre neamuri! Nu își dau seama de nimic cei ce duc după ei un idol de lemn și se închină unui dumnezeu care nu poate izbăvi!
\par 21 Grăiți, apropiați-vă și sfătuiți-vă unul cu altul! Cine a vestit aceasta, cine altădată a dat de știre? Oare nu Eu Domnul? Nu este alt dumnezeu afară de Mine. Dumnezeu drept și izbăvitor nu este altul decât Mine!
\par 22 Întoarceți-vă către Mine și veți fi mântuiți, voi cei ce locuiți toate ținuturile cele mai îndepărtate ale pământului! Că Eu sunt Dumnezeu tare și nu este altul!
\par 23 Am jurat pe Mine Însumi! Din gura Mea iese dreptatea și nu-Mi întorc cuvântul; înaintea Mea tot genunchiul se va pleca; pe Mine jura-va toată limba
\par 24 Și va zice: "Numai în Domnul este dreptatea și virtutea! Către Dânsul vor veni și înfruntați vor fi cei ce sunt întărâtați împotriva Lui.
\par 25 Întru Domnul se vor îndrepta și va fi preaslăvită toată seminția lui Israel!"

\chapter{46}

\par 1 Bel se prăbușește, Nebo este răsturnat, chipurile lor așezate pe vite și pe dobitoace. Chipurile, pe care voi acum le purtați, sunt încărcate și au ajuns o povară pentru vitele trudite.
\par 2 Idolii cad, se prăbușesc laolaltă, nu pot să izbăvească pe cei care îi poartă; ei înșiși sunt duși în robie.
\par 3 "Ascultați voi, cei din casa lui Iacov și toți cei care ați mai rămas din casa lui Israel, pe care v-am purtat din sânul maicii voastre, de care am avut grijă de la nașterea voastră.
\par 4 Până la bătrânețea voastră Eu sunt Același, până la adâncile voastre căruntele Eu vă voi ocroti. Precum am făcut în trecut, Mă leg înaintea voastră că vă voi ocroti și vă voi izbăvi și în viitor.
\par 5 Cu cine Mă veți pune alături și Mă veți face egal, cu cine Mă veți asemăna, ca să fim deopotrivă?
\par 6 Ei scot aurul din pungile lor și argintul în cântar îl cântăresc; plătesc un argintar ca să le facă un chip de dumnezeu, apoi se închină lui și îl cinstesc.
\par 7 Îl poartă pe umeri, îl duc, îl pun jos și el stă fără să se clintească din locul său. El nu răspunde celui care strigă către el și din primejdii nu-l scapă.
\par 8 Amintiți-vă de aceasta și învățați-vă minte, păcătoșilor!
\par 9 Aduceți-vă aminte de vremurile străvechi, de la obârșia lor, că Eu sunt Dumnezeu și nu este un altul. Eu sunt Dumnezeu și nu este nimeni asemenea Mie!
\par 10 De la început Eu vestesc sfârșitul și mai dinainte ceea ce are să se întâmple. Și zic: Planul Meu va dăinui și toată voia Mea o voi face!
\par 11 De la răsărit chem o pasăre de pradă, dintr-un ținut depărtat un om care să împlinească planul Meu. Am vorbit, voi împlini; am hotărât, voi înfăptui!
\par 12 Ascultați-Mă voi, oameni cu inima împietrită, voi cei care stați departe de mântuirea voastră!
\par 13 Apropia-voi mântuirea Mea, căci ea nu este departe și izbăvirea Mea nu va zăbovi. Atunci voi pune mântuirea Mea în Sion și slava Mea pentru Israel!"

\chapter{47}

\par 1 "Coboară-te și șezi în țărână, fecioară, fiica Babilonului, stai pe pământ fără tron, fiică a Caldeilor, că nu te va mai numi nimeni pe tine gingașă și plăcută!
\par 2 Învârtește la râșniță și macină făină, dă-ți la o parte vălul tău, ridică-și veșmântul tău, rămâi cu picioarele goale și treci râurile!
\par 3 Goliciunea ta să se descopere, să se vadă rușinea ta. Mă voi răzbuna și nu voi cruța pe nimeni",
\par 4 Zice Izbăvitorul nostru; Domnul Savaot este numele Lui, Sfântul lui Israel!
\par 5 "Stai tăcută și mai la întuneric, fiică a Caldeilor, nimeni nu te va mai chema pe tine stăpâna regatelor".
\par 6 Mâniat am fost pe poporul Meu, pângărit-am moștenirea Mea și am dat-o în mâna ta. Dar tu n-ai avut milă și asupra bătrânului ai apăsat cu jug greu.
\par 7 Și tu îți închipuiai: "Fi-voi pe veci stăpână!", dar niciodată n-ai cugetat și de sfârșit nu ti-ai adus aminte!
\par 8 Și acum ascultă, tu cea în plăceri crescută, care stăpâneai fără de grijă și ziceai în inima ta: "Nimeni alta nu este ca mine! Nu voi rămâne văduvă și nu voi ști ce este lipsa de copii!"
\par 9 Și aceste două într-o clipă, în aceeași zi, vor da peste tine: lipsa de copii și văduvia; și te vor copleși cu toată mulțimea și puterea fermecătoriilor și vrăjitoriilor tale!
\par 10 Întru fărădelegile tale tu nădăjduiai și ziceai: "Nimeni nu mă vede!" Înțelepciunea ta și știința ta te-au amăgit astfel, că ziceai în inima ta: "Eu și nimeni alta nu este ca mine!"
\par 11 Drept aceea va veni peste tine o nenorocire pe care tu nu vei ști să o înlături cu frumusețea ta și te va copleși nenorocirea pe care tu nu o vei putea ocoli și pe neașteptate va da peste tine pieirea, fără să fi avut vreme s-o prevestești!
\par 12 Păstrează pentru tine fermecătoriile tale și vrăjitoriile cu care te-ai trudit din tinerețe, poate îți vor sluji, poate vei insufla temere!
\par 13 Tu te-ai obosit întrebând pe atâția sfătuitori! Să iasă la iveală și să te izbăvească acei care măsoară cerul și iscodesc stelele; care în fiecare lună nouă spun ceea ce se va întâmpla.
\par 14 Iată-i ca pleava pe care o mistuie focul, așa vor ajunge ei și de puterea flăcărilor viața lor nu vor putea s-o scape căci nu va fi jeratic la care să se încălzească, nici vatră ca să stea dinaintea ei.
\par 15 Așa se va întâmpla cu aceia pe care te-ai ostenit să-i întrebi și cu care ai făcut negoț din vremea tinereții tale. Fiecare va pleca la ale sale și nimeni nu va fi să te scape".

\chapter{48}

\par 1 Ascultați aceasta, voi, cei din casa lui Iacov, care purtați numele lui Israel, voi cei ieșiți din sămânța lui Iuda, care vă jurați pe numele Domnului și vă lăudați cu Dumnezeul lui Israel, dar nu întru credincioșie și dreptate.
\par 2 Căci voi purtați numele cetății celei sfinte și vă bizuiți pe Dumnezeul lui Israel, al Cărui nume este Domnul Savaot.
\par 3 "Vestit-am din vremuri străvechi cele ce aveau să se întâmple; din gura Mea au ieșit și Eu le-am dat de știre; pe dată le-am făcut și ce s-au întâmplat,
\par 4 Fiindcă Eu știu că tu ești tare la cerbice ca un drug de fier și fruntea îți este de aramă.
\par 5 ți-am prezis acestea înainte ca să se întâmple și auzite ți le-am făcut ca să nu zici: "Idolul meu le-a făcut, chipul cel cioplit și turnat le-a hotărât!"
\par 6 Tu ai auzit; privește acum toate acestea! De ce nu mărturisești? De aci înainte îți voi împărtăși lucruri noi, ascunse, pe care nu le știai.
\par 7 Ele sunt zidite acum și nu de atunci; înainte de ziua aceasta tu n-ai auzit nimic despre ele ca să nu zici: "Iată, eu le știam!"
\par 8 Nu, tu n-ai auzit și nici n-ai știut, atunci urechea ta nu era deschisă; că Eu știu că tu ești necredincios și că din pântecele maicii tale ai fost numit răzvrătit.
\par 9 Pentru numele Meu, Îmi opresc mânia și pentru slava Mea o potolesc, ca să nu te nimicesc.
\par 10 Iată că te-am lămurit în foc și n-am aflat argint, te-am încercat în cuptorul nenorocirii.
\par 11 Pentru Mine, și numai pentru Mine o fac; oare cum voi îngădui ca numele Meu să fie pângărit? Nimănui nu voi da slava Mea!
\par 12 Ascultă, Iacove, și tu Israele, pe care te-am chemat. Eu sunt Cel dintâi și Cel de pe urmă.
\par 13 Mâna Mea a întemeiat pământul și dreapta Mea a desfășurat cerurile. Eu le chem și iată ele stau de față.
\par 14 Adunați-vă toți și ascultați! Care din voi a prezis aceste lucruri? Cel pe care Domnul îl iubește va împlini voia Lui împotriva Babilonului și împotriva seminției Caldeilor.
\par 15 Eu, Eu am grăit și l-am chemat, l-am adus și l-am făcut să propășească în calea lui.
\par 16 Apropiați-vă de Mine și ascultați acestea: De la început Eu n-am grăit întru ascuns, de când se întâmplă aceste lucruri Eu sunt de față". Și acum, Domnul Dumnezeu mă trimite cu Duhul Său!
\par 17 Așa grăiește Domnul, Izbăvitorul tău, Sfântul lui Israel: "Eu sunt Domnul Dumnezeul tău Care te învață spre folosul tău și te duce pe calea pe care trebuie să mergi.
\par 18 Dacă ai fi luat aminte la poruncile Mele, fericirea ta ar fi fost asemenea unui râu și dreptatea ta ca valurile mării.
\par 19 Și va fi seminția ta ca nisipul mării și odraslele pântecelui tău ca pulberea pământului. Nimic nu va nimici, nici nu va șterge numele tău înaintea Mea!
\par 20 Ieșiți din Babilon, fugiți din Caldeea cu cântece de veselie! Vestiți, faceți cunoscută știrea, duceți-o până la marginile pământului! Ziceți: Domnul răscumpără pe sluga Sa Iacov.
\par 21 Și nu vor suferi de sete în pustiul unde El îi duce; El le izvorăște apă din stâncă. El despică stânca și apa țâșnește!
\par 22 Nu este pace, zice Domnul, pentru cei fără de lege!"

\chapter{49}

\par 1 Ascultați, ostroave, luați aminte, popoare depărtate! Domnul M-a chemat de la nașterea Mea, din pântecele maicii Mele Mi-a spus pe nume.
\par 2 Făcut-a din gura Mea sabie ascuțită; ascunsu-M-a la umbra mâinii Sale. Făcut-a din Mine săgeată ascuțită și în tolba Sa de o parte M-a pus,
\par 3 Și Mi-a zis Mie: "Tu ești sluga Mea, Israel, întru care Eu Mă voi preaslăvi!"
\par 4 Dar Eu Îmi spuneam: "În deșert M-am trudit, în zadar și pentru nimic Mi-am prăpădit puterea Mea!" Partea ce Mi se cuvine Mie este la Domnul și răsplata Mea la Dumnezeul Meu.
\par 5 Și acum Domnul Cel Care M-a zidit din pântecele maicii Mele ca să-l slujesc Lui și să întorc pe Iacov către El și să strâng la un loc pe Israel - căci așa am fost Eu cinstit în ochii Domnului și Dumnezeul Meu fost-a puterea Mea, -
\par 6 Mi-a zis: "Puțin lucru este să fii sluga Mea ca să aduci la loc semințiile lui Iacov și să întorci pe cei ce-au scăpat dintre ai lui Israel. Te voi face Lumina popoarelor ca să duci mântuirea Mea până la marginile pământului!"
\par 7 Așa grăiește Răscumpărătorul și Sfântul lui, Israel către Cel disprețuit și către urâciunea neamurilor, Sluga tiranilor: "Regi Te vor vedea și se vor ridica, căpetenii se vor închina pentru Domnul cel credincios și pentru Sfântul lui Israel, Cel Care Te-a ales!"
\par 8 Așa grăiește Domnul: "În vremea milostivirii Te voi asculta și în vremea mântuirii Te voi ajuta. Te-am făcut și Te-am hotărât Legământ al poporului, ca să așezi rânduială în țară și să dai fiecăruia moștenirile nimicite!"
\par 9 Ca să zici celor robiți: "Ieșiți!" și celor care sunt în întuneric: "Veniți la lumină!" Ei vor paște oriunde pe calea lor și pe toate povârnișurile va fi pășunea lor.
\par 10 Nu le va fi nici foame, nici sete, soarele și vântul cel arzător nu-i va atinge, că Cel Care se va milostivi de ei va fi Povățuitorul lor și îi va îndrepta către izvoare de apă.
\par 11 Voi preface toți munții Mei în drumuri și cărările Mele vor fi bine gătite.
\par 12 Iată că unii vin din ținuturi depărtate, de la miazănoapte, de la apus, iar alții din țara Sinim.
\par 13 Săltați, ceruri, de bucurie și tu, pământule, bucură-te; munților, chiotiți de veselie, că Domnul a mângâiat pe poporul Său și de cei în necaz ai Lui S-a milostivit.
\par 14 Sionul zicea: "Domnul m-a părăsit și Stăpânul meu m-a uitat!"
\par 15 Oare femeia uită pe pruncul ei și de rodul pântecelui ei n-are ea milă? Chiar când ea îl va uita, Eu nu te voi uita pe tine.
\par 16 Iată, te-am însemnat în palmele Mele; zidurile tale sunt totdeauna înaintea ochilor Mei!
\par 17 Cei ce te vor ridica din ruini aleargă către tine și cei ce te-au pustiit fug departe de tine.
\par 18 Ridică ochii tăi de jur împrejur și vezi: toți se adună, toți vin la tine. Viu sunt Eu, zice Domnul, tu te vei îmbrăca întru ei ca într-un veșmânt de podoabă Și te vei încinge cu ei ca o mireasă.
\par 19 Căci locurile tale pustii, ruinele tale și țara ta pustiită vor fi prea strâmte pentru locuitorii tăi, iar pustiitorii tăi vor fi departe.
\par 20 Și vor mai grăi la urechile tale feciorii tăi de care tu erai lipsită: "ținutul este prea strâmt pentru mine, fă-mi loc să stau și eu!"
\par 21 Atunci tu vei zice în inima ta: "Cine mi i-a născut pe aceștia? Pierdusem copiii mei și eram stearpă, dusă în robie și gonită; dar pe aceștia cine i-a născut? Iată că rămăsesem singură! Dar aceștia de unde vin?"
\par 22 Așa zice Domnul Dumnezeu: "Iată voi ridica mâna Mea către neamuri și către popoare voi înălța steagul meu. Ele vor aduce pe feciorii tăi pe brațe și pe fiicele tale pe umeri le vor purta.
\par 23 Regi te vor crește și prințese te vor alăpta. Cu fața la pământ se vor închina înaintea ta și vor linge pulberea de pe picioarele tale. Atunci tu vei ști că Eu sunt Domnul, Care nu rușinează pe cei ce își pun nădejdea în El!
\par 24 Oare poate să i se ia celui viteaz prada și celui puternic să i se smulgă din mână cei robiri?
\par 25 Da! zice Domnul: "Chiar robii unui viteaz i se vor lua, prada unui războinic îi va scăpa; Eu Mă voi război cu potrivnicii tăi și pe fiii tăi Eu îi voi scăpa!
\par 26 Și pe asupritorii tăi îi voi face să-și mănânce carnea lor și să se îmbete de sângele lor ca de vin. Atunci toată făptura va ști că Eu sunt Domnul, Mântuitorul tău și Răscumpărătorul tău, viteazul lui Iacov!"

\chapter{50}

\par 1 Așa zice Domnul: Unde este cartea de despărțire cu care am alungat pe mama voastră? Sau care este datornicul Meu, căruia Eu v-am vândut? Pentru că numai pentru fărădelegile voastre ați fost vânduți și pentru păcatele voastre am alungat pe mama voastră.
\par 2 Pentru ce când veneam nu găseam pe nimeni și când strigam nimeni nu răspundea? Oare mâna Mea este prea scurtă, ca să răscumpere, sau nu am destulă putere, ca să izbăvesc? Prin certarea Mea sec marea și râurile le prefac în pustiu; peștii din ele mor, că nu mai este apă și se sfârșesc de sete.
\par 3 Eu îmbrac cerul cu zăbranic și îl acopăr cu un veșmânt de jale.
\par 4 Domnul Dumnezeu Mi-a dat Mie limbă de ucenic, ca să știu să grăiesc celor deznădăjduiți. În fiecare dimineață El deșteaptă, trezește urechea Mea, ca să ascult ca un ucenic.
\par 5 Domnul Dumnezeu Mi-a deschis urechea, dar Eu nu M-am împotrivit și nici nu M-am dat înapoi.
\par 6 Spatele l-am dat spre bătăi și obrajii mei spre pălmuiri, și fața Mea nu am ferit-o de rușinea scuipărilor.
\par 7 Și Domnul Dumnezeu Mi-a venit în ajutor și n-am fost făcut de ocară. De aceea am și întărit fața Mea ca o cremene, căci știam că nu voi fi făcut de ocară.
\par 8 Apărătorul Meu este aproape. Cine se judecă cu Mine? Să ne măsurăm împreună! Cine este potrivnicul Meu? Să se apropie!
\par 9 Iată, Domnul Dumnezeu Îmi este întru ajutor; cine Mă va osândi? Iată, ca un veșmânt vechi toți se vor prăpădi și molia îi va mânca!
\par 10 Cine din voi se teme de Domnul să asculte glasul Slugii Sale! Cel care umblă în întuneric și fără lumină să nădăjduiască întru numele Domnului și să se bizuie pe Dumnezeul lui!
\par 11 Voi toți, care aprindeți focul și pregătiți săgeți arzătoare, aruncați-vă în focul săgeților voastre pe care l-ați aprins! Din mâna Mea vi se întâmplă una ca aceasta; pe patul durerii veți fi culcați!

\chapter{51}

\par 1 Ascultați-Mă pe Mine, voi care umblați după dreptate, voi care căutați pe Domnul! Priviți la stânca din care ați fost tăiați și către cariera de piatră din care ați fost scoși.
\par 2 Priviți pe Avraam, tatăl vostru, și la Sarra cea care în dureri v-a născut. Că pe el singur l-am chemat, l-am binecuvântat și l-am înmulțit.
\par 3 Iar Domnul va mângâia Sionul, și dărâmăturilor lui le va da nădejde. El va preface pustiul lui în rai și pământul lui neroditor în grădina Domnului; bucurie și veselie va fi acolo, mulțumiri și cântări de laudă!
\par 4 Ia aminte la Mine, poporul Meu, și voi, neamuri, fiți cu urechea la Mine, că de la Mine va veni învățătura și legea Mea va fi lumină popoarelor.
\par 5 Dreptatea Mea este aproape, vine mântuirea Mea și brațul Meu va face dreptate popoarelor, întru Mine vor nădăjdui ținuturile cele depărtate, că de la brațul Meu așteaptă scăparea.
\par 6 Ridicați la ceruri ochii voștri și priviți jos pământul; cerurile vor trece ca un fum și pământul ca o haină se va învechi; locuitorii vor muri ca muștele, mântuirea Mea va dăinui în veac și în veac și dreptatea Mea nu va avea sfârșit.
\par 7 Ascultați-Mă pe Mine, voi, cunoscători ai dreptății, popor care ești cu legea Mea în inimă! Nu te teme de ocara oamenilor și de batjocura lor să nu te înfricoșezi. La fel ca pe un veșmânt îi va mânca molia și ca pe lână viermii îi vor mistui.
\par 8 Dar dreptatea Mea va rămâne în veac și mântuirea Mea din neam în neam.
\par 9 Ridică-te, scoală-te, îmbracă-te cu tărie, braț al Domnului! Înalță-te ca odinioară, ca în veacurile trecute! N-ai zdrobit Tu pe Rahab și n-ai spintecat Tu balaurul?
\par 10 Nu ești Tu, oare, Cel ce ai secat marea și apele adâncului celui fără fund, Cel ce adâncimile mării le-ai prefăcut în cărare largă pentru cei răscumpărați ai Tăi?
\par 11 Și astfel cei mântuiți ai Domnului se vor întoarce și vor veni în Sion, în cântări de biruință și o bucurie veșnică va încununa capul lor. Bucuria și veselia vor veni peste ei, iar durerea, întristarea și suspinarea se vor depărta de la ei.
\par 12 Eu, Eu sunt Cel ce dă nădejde! Cine ești tu, ca să te temi de un muritor și de un om de rând care trece ca iarba?
\par 13 Și să dai uitării pe Domnul, Ziditorul tău, Care a întins cerurile și a întemeiat pământul? Să te înfricoșezi mereu, în fiecare zi, de urgia asupritorului care umblă să te piardă? Unde este oare urgia asupritorului?
\par 14 Curând cel ferecat în cătușe va fi dezlegat și nu va muri în temniță și de pâine nu va duce lipsă.
\par 15 Eu sunt Domnul Dumnezeul Care stârnesc marea și face să mugească valurile ei; Domnul Savaot este numele Lui.
\par 16 Pune-voi cuvintele Mele în, gura ta și la umbra mâinii Mele te voi acoperi, ca să întind cerurile, să întemeiez pământul și să zic Sionului: "Tu ești poporul Meu!"
\par 17 Trezește-te, trezește-te, scoală-te, Ierusalime, tu care ai băut din mâna Domnului paharul urgiei Lui; potirul amețelii l-ai băut și l-ai sorbit.
\par 18 Nici unul din toți copiii pe care i-a născut nu este care să-l fi călăuzit. Nimeni nu l-a ținut de mână din toți feciorii pe care i-a crescut!
\par 19 Aceste două nenorociri te-au lovit: Cine te va plânge? Pustiirea și dărâmarea, foametea și sabia. Cine te va mângâia?
\par 20 Feciorii tăi zac fără vlagă în colțurile ulițelor, ca o antilopă prinsă în cursă, beți de urgia Domnului, de certarea Dumnezeului tău.
\par 21 Drept aceea, ia aminte, sărmană cetate, amețită, dar nu de vin!
\par 22 Așa grăiește Stăpânul, Domnul Dumnezeul tău, Care Se luptă pentru poporul Său: "Iată Eu iau din mâna ta paharul amețelii, cupa mâniei Mele, și tu nu o vei mai bea!
\par 23 Și o voi pune în mâna asupritorilor tăi, în mâna celor ce te-au supus, care îți ziceau: Pleacă-te la pământ ca să trecem peste tine! Și tu făceai spatele tău ca un pământ și ca o cale pentru trecători!"

\chapter{52}

\par 1 Trezește-te, trezește-te, îmbracă-te cu puterea ta, Sioane, înveșmântează-te în haine de sărbătoare, Ierusalime, cetate sfântă! Că nu va mai intra în tine cel netăiat împrejur și cel necurat!
\par 2 Scutură-te de pulbere, scoală-te, Ierusalime robit, dezleagă funiile de pe grumazul tău, robită fiică a Sionului?
\par 3 Căci iată ce spune Domnul: "Fără preț ați fost vânduți și fără argint veți fi răscumpărați".
\par 4 Că așa zice Domnul Dumnezeu: "Poporul Meu a coborât odinioară în Egipt ca să aibă sălaș, apoi Asiria l-a împilat fără cuvânt.
\par 5 Și acum ce să fac Eu, zice Domnul, când poporul Meu a fost luat pe nedrept? Stăpânitorii lui strigă în semn de biruință, zice Domnul, iar numele Meu, mereu, cât ține ziua, este defăimat.
\par 6 Drept aceea poporul va cunoaște numele Meu, el va înțelege în ziua aceea că Eu sunt Cel Care grăiește: Iată-Mă!"
\par 7 Cât de frumoase sunt pe munți picioarele trimisului care vestește pacea, a solului de veste bună, care dă de știre mântuirea, care zice Sionului: Dumnezeul tău este împărat!
\par 8 Toți străjerii tăi ridică glas și laolaltă strigă de bucurie, că ei văd cu ochii când Domnul Se întoarce în Sion.
\par 9 Izbucniți în chiote de veselie, dărâmături ale Ierusalimului, că Domnul mângâie pe poporul Său, răscumpărat-a Ierusalimul.
\par 10 Descoperit-a Domnul brațul Său cel sfânt în ochii tuturor popoarelor și toate marginile cele îndepărtate ale pământului vor vedea mântuirea Dumnezeului nostru, zicând:
\par 11 "Plecați, plecați, ieșiți de acolo! Nu vă atingeți de lucru spurcat! Ieșiți, curățiți-vă, voi cei care purtați vasele Domnului!
\par 12 Dar nu veți ieși îngrămădindu-vă și nu veți pleca fugind, că înaintea voastră merge Domnul și în urma voastră tot El, Dumnezeul lui Israel!"
\par 13 Iată că Sluga Mea va propăși, Se va sui, mare Se va face și Se va înălța pe culmile slavei!
\par 14 Precum mulți s-au spăimântat de El - așa de schimonosită li era înfățișarea Lui, și chipul Lui atât de fără asemănare cu oamenii, -
\par 15 Tot așa va fi pricină de uimire pentru multe popoare; înaintea Lui regii vor închide gura, că acum văd ceea ce nu li s-a spus, și înțeleg ceea ce n-au auzit.

\chapter{53}

\par 1 Cine va crede ceea ce noi am auzit și brațul Domnului cui se va descoperi?
\par 2 Crescut-a înaintea Lui ca o odraslă, și ca o rădăcină în pământ uscat; nu avea nici chip, nici frumusețe, ca să ne uităm la El, și nici o înfățișare, ca să ne fie drag.
\par 3 Disprețuit era și cel din urmă dintre oameni; om al durerilor și cunoscător al suferinței, unul înaintea căruia să-ți acoperi fața; disprețuit și nebăgat în seamă.
\par 4 Dar El a luat asupră-Și durerile noastre și cu suferințele noastre S-a împovărat. Și noi Îl socoteam pedepsit, bătut și chinuit de Dumnezeu,
\par 5 Dar El fusese străpuns pentru păcatele noastre și zdrobit pentru fărădelegile noastre. El a fost pedepsit pentru mântuirea noastră și prin rănile Lui noi toți ne-am vindecat.
\par 6 Toți umblam rătăciți ca niște oi, fiecare pe calea noastră, și Domnul a făcut să cadă asupra Lui fărădelegile noastre ale tuturor.
\par 7 Chinuit a fost, dar S-a supus și nu și-a deschis gura Sa; ca un miel spre junghiere s-a adus și ca o oaie fără de glas înaintea celor ce o tund, așa nu Și-a deschis gura Sa.
\par 8 Întru smerenia Lui judecata Lui s-a ridicat și neamul Lui cine îl va spune? Că s-a luat de pe pământ viața Lui! Pentru fărădelegile poporului Meu a fost adus spre moarte.
\par 9 Mormântul Lui a fost pus lângă cei fără de lege și cu cei făcători de rele, după moartea Lui, cu toate că nu săvârșise nici o nedreptate și nici înșelăciune nu fusese în gura Lui.
\par 10 Dar a fost voia Domnului să-L zdrobească prin suferință. Și fiindcă Și-a dat viața ca jertfă pentru păcat, va vedea pe urmașii Săi, își va lungi viața și lucrul Domnului în mâna Lui va propăși.
\par 11 Scăpat de chinurile sufletului Său, va vedea rodul ostenelilor Sale și de mulțumire Se va sătura. Prin suferințele Lui, Dreptul, Sluga Mea, va îndrepta pe mulți, și fărădelegile lor le va lua asupra Sa.
\par 12 Pentru aceasta Îi voi da partea Sa printre cei mari și cu cei puternici va împărți prada, ca răsplată că Și-a dat sufletul Său spre moarte și cu cei făcători de rele a fost numărat. Că El a purtat fărădelegile multora și pentru cei păcătoși Și-a dat viața.

\chapter{54}

\par 1 Veselește-te, cea stearpă, care nu nășteai, dă glas și strigă tu care nu te-ai zvârcolit în dureri de naștere, căci mai mulți sunt fiii celei părăsite, decât ai celei cu bărbat, zice Domnul.
\par 2 Lărgește locul cortului tău și acoperământul sălașului tău întinde-l, nu cruța nimic! Lungește funiile și întărește țărușii!
\par 3 Căci tu te vei lăți la dreapta și la stânga și seminția ta va cuceri neamurile și cetățile cele pustiite le va umple de oameni.
\par 4 Nu te înfricoșa, căci nu vei rămâne de ocară; nu te rușina, căci nu vei avea de ce să te rușinezi; că tu vei uita rușinea tinereții tale și de ocara văduviei tale nu-ți vei mai aduce aminte.
\par 5 Căci bărbatul tău este Făcătorul tău, și numele Lui: Domnul Savaot și Răscumpărătorul tău este Sfântul lui Israel: "Dumnezeul a tot pământul" se cheamă!
\par 6 Ca pe o femeie părăsită și cu inima întristată te cheamă Domnul; ca pe soția din tinerețe care a fost alungată; zice Dumnezeul tău.
\par 7 O clipă te-am părăsit, dar cu mari îndurări te iau lângă Mine.
\par 8 Într-o izbucnire de mânie, pentru o clipă Mi-am întors fața de la tine, dar în îndurarea Mea cea veșnică Mă voi milostivi de tine, zice Răscumpărătorul tău, Domnul.
\par 9 Și va fi ca în vremea lui Noe, când M-am jurat că apele potopului nu se vor mai răspândi pe pământ; tot așa Mă jur acum să nu Mă mai mânii împotriva ta și să nu te mai cert.
\par 10 Munții pot să se mute din loc și colinele să se clatine, dar milostivirea Mea nu se va depărta de la tine și legământul Meu de pace nu se va zdruncina, zice Domnul, Care are milă de tine.
\par 11 Sărmană, lovită de vijelie și fără mângâiere! Iată, zidurile tale le voi împodobi cu pietre scumpe și voi pune temelia ta pe safire.
\par 12 Și-ți voi face crestele zidurilor de rubin și porțile tale de cristal, iar împrejmuirea de pietre nestemate.
\par 13 Toți copiii tăi vor fi ucenici ai Domnului și se vor bucura de mare fericire.
\par 14 Și vei fi întemeiată pe dreptate: depărtează silnicia, căci nu ai de ce să te mai temi; leapădă și groaza, căci nu se va mai apropia de tine.
\par 15 Dacă cineva va mai da năvală, nu este pornită de la Mine, și cine se hărțuiește cu tine va cădea în lupta împotriva ta!
\par 16 Iată Eu am făcut pe meșterul, care suflă în focul de cărbuni și făurește arma cu meșteșugul lui, dar Eu am lăsat și pe cel ce trebuie s-o nimicească.
\par 17 Orice armă făurită împotriva ta nu va izbuti și orice limbă oare se ridică la judecată cu tine osândită va fi. Aceasta este moștenirea slugilor Domnului și dreptatea care vine de la Mine, zice Domnul.

\chapter{55}

\par 1 Cei ce sunteți însetați, mergeți la apă, și cei care nu aveți argint, mergeți de cumpărați și mâncați, mergeți și cumpărați fără de argint și fără preț vin și grăsime.
\par 2 Pentru ce cheltuiți argintul vostru pentru un lucru care nu hrănește și câștigul muncii voastre pentru ceva care nu vă satură? Ascultați-Mă pe Mine și veți mânca cele bune și întru bunătăți se va desfăta sufletul vostru.
\par 3 Luați aminte cu urechile voastre și mergeți pe căile Mele. Ascultați-Mă pe Mine și viu va fi sufletul vostru. Voi face cu voi legământ veșnic, dându-vă îndurările Mele cele făgăduite lui David.
\par 4 Iată l-am făcut mărturie popoarelor, căpetenie și stăpânitor peste neamuri.
\par 5 Iată, tu vei chema popoare pe care nu le știi și popoare care pe tine nu te-au cunoscut vor alerga la tine, pentru Domnul Dumnezeul tău și pentru Sfântul lui Israel, căci El te preamărește.
\par 6 Căutați pe Domnul cât Îl puteți găsi, strigați către Dânsul cât El este aproape de voi.
\par 7 Cel rău să lase calea lui și omul cel nelegiuit vicleniile lui și să se întoarcă spre Domnul, căci El Se va milostivi de dânsul, și către Dumnezeul nostru cel mult iertător.
\par 8 Căci gândurile Mele nu sunt ca gândurile voastre și căile Mele ca ale voastre, zice Domnul.
\par 9 Și cât de departe sunt cerurile de la pământ, așa de departe sunt căile Mele de căile voastre și cugetele Mele de cugetele voastre.
\par 10 Precum se coboară ploaia și zăpada din cer și nu se mai întoarce până nu adapă pământul și-l face de răsare și rodește și dă sămânță semănătorului și pâine spre mâncare,
\par 11 Asa va fi cuvântul Meu care iese din gura Mea; el nu se întoarce către Mine fără să dea rod, ci el face voia Mea și își îndeplinește rostul lui.
\par 12 Și voi cu veselie veți ieși și în pace veți fi călăuziți; munții și colinele vor izbucni în strigăte de veselie înaintea voastră și toți copacii câmpului vor bate din palme!
\par 13 În locul spinilor va crește Chiparosul și în locul urzicii va crește mirtul. A Domnului fi-va slava, spre veșnică și nepieritoare pomenire.

\chapter{56}

\par 1 Așa zice Domnul: "Păziți dreptatea și faceți lucruri drepte, că în curând va veni mântuirea Mea și dreptatea Mea se va descoperi.
\par 2 Fericit este omul care săvârșește acestea și care ține la ele: Păzește ziua de odihnă ca să nu fie pângărită și își ferește mâna lui ca să nu făptuiască nici un rău.
\par 3 și să nu zică cel de alt neam, care s-a alipit de Domnul: "Domnul mă va despărți de poporul Său!" Și famenul să nu zică: "Iată eu sunt un copac uscat!"
\par 4 Pentru că așa zice Domnul către fameni: Celor care păzesc zilele Mele de odihnă și aleg ceea ce Îmi este plăcut Mie și stăruie în legământul Meu,
\par 5 Le voi da în casa Mea și înăuntrul zidurilor Mele un nume și un loc mai de preț decât fii și fiice; le voi da un nume veșnic și nepieritor.
\par 6 Și pentru străinii alipiți de Domnul ca să slujească și să iubească numele Domnului și să fie slujitorii Săi, toți câți păzesc ziua de odihnă ca să nu fie pângărită și stăruie în legământul Meu,
\par 7 Pe aceștia îi voi aduce în muntele cel sfânt al Meu și îi voi bucura în locașul Meu de rugăciune. Arderile lor de tot și jertfele lor vor fi primite pe altarul Meu; căci templul Meu, locaș de rugăciune se va chema pentru toate popoarele!"
\par 8 Acestea sunt zisele Domnului, Care adună pe cei risipiți ai lui Israel: "La cei adunați voi mai aduna și alții!"
\par 9 Fiare ale câmpului, veniți, mâncați și voi, toate animalele pădurii!
\par 10 Străjerii Mei sunt orbi cu toții, ei nu înțeleg nimic. Toți sunt câini muți care nu pot să latre. Ei visează, stau tolăniți și le place să doarmă.
\par 11 Aceștia sunt câini hrăpăreți care nu se mai satură; sunt păstorii care nu pricep nimic. Toți umblă în căile lor și se silesc pentru câștigul lor.
\par 12 "Veniți, zic ei, voi aduce vin, bea-vom băuturi îmbătătoare! Și mâine va fi, ca astăzi, mare zi de veselie".

\chapter{57}

\par 1 Dreptul piere și nimeni nu ia aminte; se duc oamenii cinstiți șt nimănui nu-i pasă că din pricina răutății a pierit cel drept.
\par 2 El intră în pace în groapă. Cel care umblă pe calea cea dreaptă se odihnește în sălașurile sale.
\par 3 Dar voi, feciori de vrăjitoare, neam de stricați și desfrânați, apropiați-vă.
\par 4 De cine vă bateți joc? La cine vă strâmbați și scoateți limba? Nu sunteți voi copii păcătoși, neam de mincinoși?
\par 5 Ardeți de poftă pe lângă idolii de sub orice copac verde și jertfiți pe fii în albia râurilor și în peșteri.
\par 6 Pietrele cele lucioase ale râurilor sunt partea ta! Iată, iată sorțul tău! Lor le aduci jertfă cu turnare și prinoase! Pot Eu să fiu mulțumit de aceasta?
\par 7 Pe un munte înalt și ridicat îți așezi patul tău și acolo te urci ca să aduci jertfa ta!
\par 8 După ușă, în dosul ușorilor, ai pus amintirea ta; și departe de Mine tu desfaci patul tău, te urci și îl mai lărgești, faci legământ cu ei, îți place să te culci cu ei...
\par 9 Tu alergi după Melec cu untdelemn și cu miresme multe; tu trimiți solii tăi departe, și te cobori până la locuința morților.
\par 10 Călătoria ta cea lungă te obosește, dar nu zici: "Nu-mi mai trebuie!" Tu găsești insă puteri noi, pentru aceasta tu nu te dai bătut!
\par 11 De cine îți era frică? De cine te temeai ca să Mă mânii pe Mine, să nu-ți mai aduci aminte și nici să nu-ți mai pese? Fiindcă n-am deschis gura și am închis ochii, tu nu te-ai temut de Mine!
\par 12 Eu îți voi face știută dreptatea ta, căci lucrurile tale nu slujesc la nimic.
\par 13 Când tu vei striga, să te izbăvească idolii tăi! Pe toți îi va duce vântul Și o suflare îi va face nevăzuți! Dar cel care își pune nădejdea în Mine va moșteni pământul și va stăpâni în muntele cel sfânt.
\par 14 Și li se va zice: Gătiți, gătiți, faceți drum, dați la o parte orice piedică din calea poporului Meu.
\par 15 Că așa zice Domnul, a Cărui locuință este veșnică și al Cărui nume este sfânt: Sălășluiesc într-un loc înalt și sfânt și sunt cu cei smeriți și înfrânți, ca să înviorez pe cei cu duhul umilit și să îmbărbătez pe cei cu inima frântă.
\par 16 Căci nu vreau să cert totdeauna și să stărui în mânie, căci înaintea Mea ar cădea în nesimțire duhul și sufletele pe care le-am creat.
\par 17 Pentru fărădelegea sa, M-am întărâtat o clipă și, stând ascuns, l-am lovit întru mânia Mea. Și el, răzvrătit, mergea pe calea inimii sale!
\par 18 Am văzut căile sale și îl voi vindeca, îl voi povățui, îl voi odihni și îl voi mângâia.
\par 19 Și cei care îl jeleau vor izbucni în cântări de mulțumire. Pace, pace celor de aproape și celor de departe, zice Domnul, și Eu îl voi tămădui.
\par 20 Cei fără de lege sunt ca marea cea înviforată, care nu se poate astâmpăra și valurile ei scormonesc tină și nămol.
\par 21 Cei fără de lege n-au pace, zice Domnul.

\chapter{58}

\par 1 Strigă din toate puterile și nu te opri, dă drumul glasului să sune ca o trâmbiță, vestește poporului Meu păcatele sale și casei lui Iacov fărădelegile sale.
\par 2 În fiecare zi Mă caută, pentru că ei voiesc să știe căile Mele ca un popor care făptuiește dreptatea și de la legea Dumnezeului său nu se abate. Ei Mă întreabă despre legile dreptății și doresc să se apropie de Dumnezeu.
\par 3 Pentru ce să postim, dacă Tu nu vezi? La ce să ne smerim sufletul nostru, dacă Tu nu iei aminte? Da, în zi de post, voi vă vedeți de treburile voastre și asupriți pe toți lucrătorii voștri.
\par 4 Voi postiți ca să vă certați și să vă sfădiți și să bateți furioși cu pumnul; nu postiți cum se cuvine zilei aceleia, ca glasul vostru să se audă sus.
\par 5 Este oare acesta un post care Îmi place, o zi în care omul își smerește sufletul său? Să-și plece capul ca o trestie, să se culce pe sac și în cenușă, oare acesta se cheamă post, zi plăcută Domnului?
\par 6 Nu știți voi postul care Îmi place? - zice Domnul. Rupeți lanțurile nedreptății, dezlegați legăturile jugului, dați drumul celor asupriți și sfărâmați jugul lor.
\par 7 Împarte pâinea ta cu cel flămând, adăpostește în casă pe cel sărman, pe cel gol îmbracă-l și nu te ascunde de cel de un neam cu tine.
\par 8 Atunci lumina ta va răsări ca zorile și tămăduirea ta se va grăbi. Dreptatea ta va merge înaintea ta, iar în urma ta slava lui Dumnezeu.
\par 9 Atunci vei striga și Domnul te va auzi; la strigătul tău El va zice: Iată-mă! Dacă tu îndepărtezi din mijlocul tău asuprirea, amenințarea cu mâna și cuvântul de cârtire,
\par 10 Dacă dai pâinea ta celui flămând și tu saturi sufletul amărât, lumina ta va răsări în întuneric și bezna ta va fi ca miezul zilei.
\par 11 Domnul te va călăuzi necontenit și în pustiu va sătura sufletul tău. El va da tărie oaselor tale și vei fi ca o grădină adăpată, ca un izvor de apă vie, care nu seacă niciodată.
\par 12 Pe vechile tale ruine se vor înălța clădiri noi, vei ridica din nou temeliile străbune și vei fi numit dregător de spărturi și înnoitor de drumuri, ca țara să poată fi locuită.
\par 13 Dacă îți vei opri piciorul tău în ziua de odihnă și nu-ți vei mai vedea de treburile tale în ziua Mea cea sfântă, ci vei socoti ziua de odihnă ca desfătare și vrednică de cinste, ca sfințită de Domnul, și vei cinsti-o, fără să mai umbli, fără să te mai îndeletnicești cu treburile tale și fără să mai vorbești deșertăciuni,
\par 14 Atunci vei afla desfătarea ta în Domnul. Eu te voi purta în car de biruință pe culmile cele mai înalte ale țării și te voi bucura de moștenirea tatălui tău Iacov, căci gura Domnului a grăit acestea.

\chapter{59}

\par 1 Iată, mâna Domnului nu este prea scurtă ca să nu poată să izbăvească, și urechea Lui prea tare ca să nu audă.
\par 2 Ci nelegiuirile voastre au pus despărțire intre voi și Dumnezeul vostru și păcatele voastre L-au făcut să-Și ascundă fața ca să nu vă audă.
\par 3 Pentru că mâinile voastre sunt întinate cu sânge și degetele voastre cu nelegiuiri; buzele voastre grăiesc cuvinte mincinoase și limba voastră, strâmbătate.
\par 4 Nimeni nu cheamă în sprijinul său dreptatea și cu cinste nici un judecător nu hotărăște; ci toți își pun nădejdea în lucruri deșarte și în vorbe fără rost: zămislesc silnicia și nasc păcatul.
\par 5 Clocesc ouă de șarpe și urzesc pânză de păianjen: cine mănâncă din ouăle lor moare, iar din cele sparte ies năpârci.
\par 6 Din pânza lor veșminte nu se pot face și cu lucrul făcut de mâna lor nu se acoperă, căci lucrul lor este lucru rău; în mâinile lor sunt numai fapte silnice.
\par 7 Picioarele lor aleargă spre rău, grabnice să verse sânge nevinovat; cugetele lor sunt cugete viclene; în calea lor sălășluiesc pustiirea și prăpădul.
\par 8 Nu cunosc drumul păcii și pe urmele lor nu este nici o dreptate; cărările lor sunt întortocheate și cine pornește pe ele nu știe de pace.
\par 9 Pentru aceasta, judecata este departe de noi și dreptatea nu ne ajunge. Noi așteptăm lumina, dar iată întunericul; așteptăm revărsatul zorilor, dar umblăm în beznă.
\par 10 Umblăm bâjbâind, ca orbii pe lângă zid; ca și cei fără ochi bâjbâim mereu, ne poticnim în miezul zilei ca și pe înserate; între oamenii în putere suntem ca niște morți.
\par 11 Mormăim toți ca urșii, ne văităm ca porumbița, așteptăm judecata, dar ea nu vine; mântuirea, dar ea este departe de noi.
\par 12 Că păcatele noastre s-au înmulțit înaintea Ta și nelegiuirile sunt mărturie împotriva noastră; fărădelegile noastre sunt de față și faptele noastre nelegiuite le știm:
\par 13 Necredința și tăgada Domnului, căderea de la credința în Dumnezeu, grăirea minciunii și răzvrătirea, născocirea și cugetarea la lucruri viclene.
\par 14 Și lăsată la o parte este judecata, iar dreptatea stă departe; adevărul se poticnește în piețe și fapta cinstită nu mai are loc.
\par 15 Adevărul nu mai este și cel ce se dă la o parte din calea răutății este sfărâmat. Și Dumnezeul nostru a văzut și S-a mâniat că nu mai este dreptate.
\par 16 Și a văzut că nu este nici un om și S-a mirat că nimeni nu mijlocește. Atunci brațul Lui I-a venit în ajutor, și dreptatea Sa a fost sprijinul Său.
\par 17 S-a îmbrăcat cu dreptatea ca și cu o platoșă și a pus pe capul Său coiful izbăvirii; S-a îmbrăcat cu răzbunarea ca și cu o haină și S-a înfășurat în râvna Sa ca și într-o mantie.
\par 18 După faptă și răsplată: urgie împotriva vrăjmașilor și răsplată după faptă împotrivitorilor Lui; ținuturilor celor de departe, răsplata cuvenită.
\par 19 Cei de la apus se vor teme de numele Domnului și cei de la răsărit, de slava Lui; că va veni ca un șuvoi îngust pe care Duhul Domnului îl mână.
\par 20 "Pentru Sion El va veni ca un Mântuitor, pentru cei din Iacov care se vor căi de păcatele lor", zice Domnul.
\par 21 Iată, acesta este legământul Meu cu ei, zice Domnul: "Duhul Meu, Care odihnește peste tine, și cuvintele Mele pe care le-am pus în gura ta, să nu se depărteze din gura ta, nici din gura urmașilor tăi și nici din gura urmașilor urmașilor tăi, zice Domnul, de acum și până în veac!"

\chapter{60}

\par 1 Luminează-te, luminează-te, Ierusalime, că vine lumina ta, și slava Domnului peste tine a răsărit!
\par 2 Căci iată întunericul acoperă pământul, și bezna, popoarele; iar peste tine răsare Domnul, și slava Lui strălucește peste tine.
\par 3 Și vor umbla regi întru lumina ta și neamuri întru strălucirea ta.
\par 4 Ridică împrejur ochii tăi și vezi, că toți se adună și se îndreaptă către tine. Fiii tăi vin de departe și fiicele tale sunt aduse pe umeri.
\par 5 Atunci vei vedea, vei străluci și va bate tare inima ta și se va lărgi, căci către tine se va îndrepta bogăția mării și avuțiile popoarelor către tine vor curge.
\par 6 Caravane de cămile te vor acoperi, și dromadere din Madian și Efa. Toate sosesc din Șeba, încărcate cu aur și cu tămâie, cântând laudele Domnului.
\par 7 Toate turmele Chedarului la tine se vor aduna, berbecii din Nebaiot te vor sluji pe tine și ca o jertfă bineplăcută se vor urca pe jertfelnicul Meu, și templul rugăciunii Mele se va slăvi.
\par 8 Cine zboară ca norii și ca porumbița spre sălașul ei?
\par 9 Căci pentru Mine se adună corăbiile, în frunte cu cele din Tarsis, ca să aducă de departe pe feciorii tăi; aurul și argintul lor pentru numele Dumnezeului tău și pentru Sfântul lui Israel, Care te preamărește.
\par 10 Feciorii de neam străin zidi-vor zidurile tale și regii lor în slujba ta vor fi, că întru mânia Mea te-am lovit și în îndurarea Mea M-am milostivit de tine.
\par 11 Porțile tale mereu vor fi în lături, zi și noapte vor rămâne deschise, ca să se care la tine bogățiile neamurilor, iar regii lor în fruntea lor vor fi.
\par 12 Căci neamul și regatul care nu vor sluji ție vor pieri și neamurile acelea vor fi nimicite.
\par 13 Mărirea Libanului, chiparosul, ulmul și merișorul la tine vor veni, cu toții laolaltă, ca să împodobească locașul cel sfânt al Meu, și Eu voi slăvi locul unde se odihnesc picioarele Mele.
\par 14 Și feciorii asupritorilor tăi smeriți la tine vor veni și se vor închina la picioarele tale toți cei ce te-au urât și pe tine te vor numi: cetatea Domnului, Sionul Sfântului lui Israel.
\par 15 Din părăsită și defăimată ce erai pe veci, voi face din tine mândria veacurilor, bucurie din neam în neam.
\par 16 Tu vei suge laptele neamurilor și vei mânca bunătățile regilor. Și vei ști că Eu, Domnul, sunt Mântuitorul tău, că Cel puternic al lui Iacov este Răscumpărătorul tău.
\par 17 În loc de aramă îți voi aduce aur, în loc de fier, argint, în loc de lemn, aramă și în loc de pietre, fier. Și voi pune judecător al tău pacea și stăpânitor peste tine dreptatea.
\par 18 Și nu se va mai auzi de silnicie în țara ta, de pustiire și de ruină în hotarele tale. Zidurile tale le vei numi mântuire și porțile tale laudă.
\par 19 Nu vei mai avea soarele ca lumină în timpul zilei și strălucirea lunii nu te va mai lumina; ci Domnul va fi pentru tine o lumină veșnică și Dumnezeul tău va fi slava ta.
\par 20 Soarele tău nu va mai asfinți și luna nu va mai descrește; că Domnul va fi pentru tine lumină veșnică și zilele întristării tale se vor sfârși.
\par 21 În poporul tău vor fi numai drepți și vor stăpâni țara pentru totdeauna; vlăstar pe care l-am sădit Eu, lucrul mâinilor Mele făcut spre slava Mea.
\par 22 Cel mai mic va fi cât o mie, cel mai neînsemnat va fi cât un neam puternic: Eu, Domnul, am hotărât acestea și la vreme voi fi împlinitorul lor.

\chapter{61}

\par 1 Duhul Domnului este peste Mine, că Domnul M-a uns să binevestesc săracilor, M-a trimis să vindec pe cei cu inima zdrobită, să propovăduiesc celor robiți slobozire și celor prinși în război libertate;
\par 2 Să dau de știre un an de milostivire al Domnului și o zi de răzbunare a Dumnezeului nostru;
\par 3 Să mângâi pe cei întristați; celor ce jelesc Sionul, să le pun pe cap cunună în loc de cenușă, untdelemn de bucurie în loc de veșminte de doliu, slavă în loc de deznădejde. Ei vor fi numiți: stejari ai dreptății, sad al Domnului spre slăvirea Lui.
\par 4 Ei vor zidi pe vechile ruine, vor ridica dărâmăturile de altădată, vor reface cetățile distruse, pustiite din neam în neam.
\par 5 Oameni de neam străin vor veni și vă vor paște turmele, feciori din alt neam vor fi plugarii și vierii voștri.
\par 6 Și voi, voi veți fi numiți preoți ai Domnului, slujitori ai Dumnezeului nostru. Bunătățile popoarelor, voi le veți mânca și cu averile lor voi vă veți mândri.
\par 7 Fiindcă ocara lor era îndoită, batjocură și scuipări erau partea lor, pentru aceasta îndoit în pământul lor vor moșteni și de slava cea de-a pururi ei se vor bucura!
\par 8 Că Eu sunt Domnul, Care iubesc dreptatea ți urăsc răpirile nedrepte. Eu le voi da cu credincioșie plata lor și legământ veșnic cu ei voi încheia.
\par 9 Cu nume mare va fi neamul lor între neamuri și urmașii lor printre popoare. Toți cei ce îi vor vedea vor da mărturie că ei sunt un neam binecuvântat de Domnul.
\par 10 Bucura-Mă-voi întru Domnul, sălta-va de veselie sufletul Meu întru Dumnezeul Meu, că M-a îmbrăcat cu haina mântuirii, cu veșmântul veseliei M-a acoperit. Ca unui mire Mi-a pus Mie cunună și ca pe o mireasă M-a împodobit cu podoabă.
\par 11 Ca pământul care răsare ierburi, și ca o grădină în care sămânța încolțește, așa Domnul Dumnezeu va face dreptatea să răsară, și înaintea tuturor neamurilor preaslăvirea Sa.

\chapter{62}

\par 1 Pentru Sion nu voi tăcea și pentru Ierusalim nu voi avea odihnă până ce dreptatea lui nu va ieși ca lumina și mântuirea lui nu va arde ca o flacără.
\par 2 Atunci neamurile vor vedea dreptatea ta și toți regii slava ta și te vor chema pe tine cu nume nou, pe care îl va rosti gura Domnului.
\par 3 Și tu vei fi ca o cunună de mărire în mâna Domnului și ca o diademă regală în mâna Dumnezeului tău.
\par 4 Și nu ți se va mai zice ție: "Alungată", și țării tale: "Pustiită", ci tu te vei chema: "Întru tine am binevoit" și țara ta: "Cea cu bărbat", că Domnul a binevoit întru tine și pământul tău va avea un soț.
\par 5 Și în ce chip se însorește flăcăul cu fecioara, Cel ce te-a zidit Se va însoți cu tine, și în ce chip mirele se veselește de mireasă, așa Se va veseli de tine Dumnezeul tău!
\par 6 Pe zidurile tale, Ierusalime, Eu pun străjeri, care nici zi, nici noapte nu vor tăcea! Voi, care aduceți aminte Domnului de făgăduințele Lui, să n-aveți odihnă!
\par 7 Și să nu-I dați răgaz până ce nu va așeza din nou Ierusalimul, ca să facă din el lauda pământului.
\par 8 Juratu-S-a Domnul pe dreapta Lui și pe brațul Lui cel tare: "Nu voi mai da de aici înainte grâul tău spre hrană vrăjmașilor tăi, și cei de neam străin, nu vor bea mustul tău, rodul muncii tale.
\par 9 Ci numai cei ce îl vor fi adunat îl vor mânca și vor lăuda pe Domnul, și cei care vor fi făcut culesul vor bea vinul în curțile templului Meu cel sfânt!"
\par 10 Intrați, întrați pe porți! Gătiți cale poporului, gătiți, gătiți-i drum, curățiți-l de pietre, înălțați un steag peste neamuri!
\par 11 Iată, Domnul vestește acestea până la marginile pământului: "Ziceți fiicei Sionului: Mântuitorul tău vine! El vine cu plata, și răsplătirile merg înaintea Lui!"
\par 12 Și ei se vor chema: "Popor sfânt, răscumpărați ai Domnului și ție și se va zice: "Cea căutată", "Cetatea nepărăsită!"

\chapter{63}

\par 1 Cine este Cel ce vine împurpurat, cu veșmintele Sale mai roșii decât ale celui ce culege la vie, cu podoabă în îmbrăcămintea Lui și mândru de belșugul puterii Lui? "Eu sunt Acela al Cărui cuvânt este dreptatea și puternic este să răscumpere!"
\par 2 Pentru ce ai îmbrăcămintea roșie și veșmântul Tău este roșu ca al celui care calcă în teasc?
\par 3 "Singur am călcat în teasc și dintre popoare nimeni nu era cu Mine; și i-am călcat în mânia Mea, i-am strivit în urgia Mea, încât sângele lor a râșnit pe veșmântul Meu, și Mi-am pătat toate hainele Mele.
\par 4 Căci o zi de răzbunare era sortită în inima Mea și anul răscumpărării sosise.
\par 5 Priveam în jur: nici un ajutor! Mă cuprindea mirarea: nici un sprijin! Atunci brațul Meu M-a ajutat și urgia Mea sprijin Mi-a fost.
\par 6 În mânia Mea am călcat în picioare popoare și le-am zdrobit în urgia Mea Și Sângele lor l-am împrăștiat pe pământ".
\par 7 Voi pomeni îndurările Domnului, faptele Lui minunate, după tot ce a făcut Domnul pentru noi și pentru marea bunătate pe care El ne-a mărturisit-o în milostivirea Sa și după mulțimea milelor Sale.
\par 8 Și a zis: "Cu adevărat ei sunt poporul Meu, fii care nu vor fi necredincioși!"
\par 9 Și El le-a fost izbăvitor în toate strâmtorările lor. Și n-a fost un trimis și nici un înger, ci fața Lui i-a mântuit. Întru iubirea Lui și întru îndurarea Lui El i-a răscumpărat, i-a ridicat și i-a purtat în toată vremea de demult.
\par 10 Dar ei s-au răzvrătit și au amărât Duhul Lui cel sfânt, din țare pricină El S-a făcut împotrivitorul lor și s-a războit împotriva lor.
\par 11 Atunci ei și-au adus aminte de vremurile trecute, de sluga Sa Moise: Unde este Cel ce a scos din mare pe păstorul și turma Sa? Unde este Cel ce a pus în mijlocul ei Duhul Său cel sfânt?
\par 12 Cel Care a călăuzit dreapta lui Moise cu brațul Său slăvit? Cel Care a despicat apele înaintea lor ca să-Și facă un nume veșnic?
\par 13 Care i-a călăuzit prin adâncurile mării, ca pe un cal în pustiu și ei nu s-au poticnit?
\par 14 Ca dobitoacele care coboară la șes, așa Duhul Domnului îi aducea la odihnă. Astfel ai povățuit Tu pe poporul Tău, ca să-ți faci un nume slăvit.
\par 15 Privește din ceruri și vezi, din locașul Tău cel sfânt și strălucit: Unde este râvna și puterea Ta nesfârșită, zbuciumul lăuntrului Tău și milostivirile Tale?
\par 16 Pentru mine, acestea au încetat! Dar Tu ești Părintele nostru! Avraam nu știe nimic, Israel nu ne cunoaște. Tu, Doamne, ești Tatăl nostru, Mântuitorul rostru: acesta este numele Tău de totdeauna.
\par 17 Pentru ce, Doamne, ne-ai lăsat să rătăcim departe de căile Tale și ne-ai învârtoșat inimile noastre ca să nu ne temem de Tine? Întoarce-Te pentru robii Tăi, pentru semințiile moștenirii Tale.
\par 18 Pentru ce au pășit cei nelegiuiți în templul Tău și vrăjmașii noștri au călcat în picioare altarul Tău?
\par 19 Am ajuns ca unii peste care Tu de multă vreme nu mai stăpânești și care nu mai sunt chemați cu numele Tău.

\chapter{64}

\par 1 Dacă ai rupe cerurile și Te-ai pogorî, munții s-ar cutremura!
\par 2 Ca un foc care arde vreascurile, ca o vâlvătaie care fierbe apa în clocot, fă pe vrăjmașii Tăi să cunoască numele Tău și să tremure înaintea Ta neamurile, văzându-Te
\par 3 Făcând minuni neașteptate,
\par 4 Despre care niciodată nu s-a auzit grăind. Nici urechea n-a auzit, nici ochiul n-a văzut un dumnezeu, afară de Tine, care ar săvârși unele ca acestea pentru cei ce nădăjduiesc în el.
\par 5 Tu Te duci întru întâmpinarea celor ce săvârșesc faptele dreptății și își aduc aminte de căile Tale. Iată, Tu Te-ai pornit cu mânie și noi eram vinovați prin necredința și fărădelegea, noastră!
\par 6 Toți am ajuns ca necurații și toate faptele dreptății noastre ca un veșmânt întinat. Noi toți am căzut ca frunzele uscate și fărădelegile noastre ne luau ca vântul.
\par 7 Nimeni nu chema numele Tău și nici unul nu se deștepta ca să se întărească întru Tine. Căci Tu ai ascuns fața Ta de la noi și ne-ai lăsat în voia fărădelegilor noastre.
\par 8 Și acum, Doamne, Tu ești Tatăl nostru, noi suntem lutul și Tu olarul, toți lucrul mâinilor Tale suntem!
\par 9 O, Doamne! Nu Te mânia pe noi foarte și nu-ți aduce aminte la nesfârșit de fărădelegea noastră! Privește, căci noi toți suntem poporul Tău!
\par 10 Cetățile Tale sfinte au ajuns pustii, Sionul este ca un deșert, Ierusalimul ca un loc pustiit!
\par 11 Templul nostru sfânt și mărit în care Te-au preaslăvit părinții noștri a ajuns pradă focului și toate cele scumpe nouă, dărâmături!
\par 12 Poți Tu oare să Te stăpânești, să taci, Doamne, și să ne întristezi atât de mult?

\chapter{65}

\par 1 Căutat am fost de cei ce nu întrebau de Mine, găsit am fost de cei ce nu Mă căutau. Și am zis: "Iată-Mă, iată-Mă aici, la un neam care nu chema numele Meu!
\par 2 Tins-am mâinile Mele în toată vremea către un popor răzvrătit, care mergea pe căi silnice, după cugetele sale,
\par 3 Oameni care întărâtau fără încetare fața Mea jertfind în grădini și pe lespezile acoperișului ardeau miresme;
\par 4 Stăteau în morminte și mâncau în crăpături de stâncă, mâncau carne de porc, ale căror vase erau pline de mâncăruri spurcate
\par 5 Și care ziceau: "Dă-te înapoi, nu te apropia de mine, căci eu sunt sfânt față de tine!" - Aceștia sunt ca un fum care se urcă în nările Mele, o văpaie care arde fără sfârșit.
\par 6 Iată este scris înaintea Mea: "Nu voi tăcea până ce nu voi răsplăti
\par 7 Fărădelegile voastre și fărădelegile părinților voștri laolaltă, - zice Domnul - ale celor care au adus jertfă de tămâie pe munți și pe dealuri și au râs de Mine. Eu îi voi răsplăti după faptele lor și cu măsură plină.
\par 8 Așa zice Domnul: "Ca atunci când găsești must într-un strugure și zici: "Nu-l rupe, că în acesta se află o binecuvântare", tot astfel voi face și cu slujitorii Mei; Mă voi feri să stric tot!
\par 9 Și voi face să răsară din Iacov o odraslă și din Iuda un moștenitor peste munții Mei; și cei aleși ai Mei li vor stăpâni și slujitorii Mei vor locui acolo.
\par 10 Și Șaronul va ajunge pășune pentru turme și Acorul, imaș pentru vite, pentru poporul Meu care M-a căutat pe Mine.
\par 11 Și voi, cei ce ați părăsit pe Domnul, care ați uitat de muntele Meu cel sfânt, care întindeți masă pentru dumnezeul Gad și umpleți o cupă pentru Meni;
\par 12 Pe toți vă voi da în ascuțișul sabiei și junghierii vă veți pleca, pentru că am strigat către voi și nu Mi-ați răspuns, am grăit și nu M-ați auzit, ci ați făcut cele rele în ochii Mei și ceea ce nu am binevoit ați ales".
\par 13 Pentru aceasta, așa zice Domnul Dumnezeu: "Iată, slugile Mele vor mânca și vouă vă va fi foame, vor bea și voi veți fi însetați, se vor bucura, iar voi veți fi înfruntați!
\par 14 Iată slugile Mele vor sălta de veselie, iar voi veți striga de întristată ce vă va fi inima, și de frânt ce vă va fi sufletul veți urla!"
\par 15 Și veți lăsa numele vostru aleșilor Mei spre blestem: "Domnul Dumnezeu să te ucidă!... Dar slujitorii Mei vor fi numiți cu alt nume.
\par 16 Cine se va binecuvânta pe pământ se va binecuvânta de Dumnezeul adevărului, și cel ce se va jura pe pământ se va jura pe Dumnezeul adevărului; că nenorocirile din vremurile de demult au fost uitate și ei stau departe de ochii Mei.
\par 17 Pentru că Eu voi face ceruri noi și pământ nou. Nimeni nu-și va mai aduce aminte de vremurile trecute și nimănui nu-i vor mai veni în minte,
\par 18 Ci se vor bucura și se vor veseli de ceea ce Eu voi fi făcut, căci iată întemeiez Ierusalimul pentru bucurie și poporul lui pentru desfătare.
\par 19 Și Mă voi bucura de Ierusalim și Mă voi veseli de poporul Meu și nu se va mai auzi în acesta nici plâns, nici țipăt.
\par 20 Să nu mai fie acolo copii care mor în floarea vârstei și nici bătrâni care nu ajung la capătul vieții lor! Așa că cine va muri la o sută de ani va fi tânăr și cine nu o va ajunge va fi blestemat.
\par 21 Și ei vor zidi case și vor locui și vor sădi vii și din rodul lor vor mânca.
\par 22 Dar nu vor clădi ca altul să locuiască, nici nu vor sădi ca altul să mănânce. Într-adevăr vârsta poporului Meu va fi ca vârsta copacilor, și cei aleși ai Mei se vor bucura de osteneala mâinilor lor.
\par 23 Nu se vor trudi în zadar și nu vor naște feciori pentru moarte fără de vreme, că ei vor fi un neam binecuvântat de Domnul și împreună cu ei și odraslele lor.
\par 24 Și înainte de a Mă chema pe Mine, Eu le voi răspunde, și grăind ei încă, Eu îi voi fi ascultat.
\par 25 Lupul va paște la un loc cu mielul, leul va mânca paie ca boul și șarpele cu țărână se va hrăni. Nimic rău și vătămător nu va fi în muntele Meu cel sfânt", zice Domnul.

\chapter{66}

\par 1 Așa zice Domnul: "Cerul este scaunul Meu și pământul așternut picioarelor Mele! Ce fel de casă Îmi veți zidi voi și ce loc de odihnă pentru Mine?"
\par 2 "Toate acestea mâna Mea le-a făcut și sunt ale Mele, zice Domnul. Spre unii ca aceștia Îmi îndrept privirea Mea: spre cei smeriți, cu duhul umilit și care tremură la cuvântul Meu!
\par 3 Cel ce junghie un bou și în același timp omoară un om, cel ce jertfește o oaie și în același timp rupe gâtul unui câine, cel ce aduce prinos și în același timp aduce sânge de porc, cel ce aduce jertfă de tămâie și în același timp se închină la idoli, - toți aceștia și-au ales căi nelegiuite și în urâciunile lor trăiește sufletul lor.
\par 4 Pentru aceasta și Eu voi alege pentru ei soarta cea rea și cele ce îi înfricoșează le voi aduce peste ei; că am strigat și nu Mi-au răspuns, am grăit și nu M-au auzit, au făcut fărădelegi înaintea ochilor Mei și ceea ce Eu nu binevoiesc întru Mine, au ales.
\par 5 Luați aminte la cuvântul Domnului, voi care tremurați de El! Iată ce grăiesc frații voștri care vă urăsc și vă prigonesc pentru numele Meu: "Să-Și arate Domnul slava Sa și noi să o vedem din bucuria voastră!" Dar ei se vor face de ocară.
\par 6 Un glas! Un vuiet din cetate! Un glas din templu! Este glasul Domnului! El răsplătește vrăjmașilor Săi după faptele lor.
\par 7 Înainte de a se zvârcoli în dureri de naștere, ea a născut; înainte de a simți chinul, ea a născut un fiu.
\par 8 Cine a auzit sau cine a văzut unele ca acestea? Oare o țară se naște într-o singură zi și un popor dintr-odată? Abia au apucat-o durerile nașterii și fiica Sionului a și născut fii!
\par 9 Oare, Eu voi deschide pântecele fără să-l laț să nască? Zice Domnul. Sau Eu, Cel ce fac să nască, îl voi închide?
\par 10 Bucură-te, Ierusalime, și voi, cei care îl iubiți, săltați de veselie. Fiii în culmea veseliei, voi cei care îl plângeați!
\par 11 Astfel ca voi să fiii alăptați și să vă săturați la pieptul mângâierilor sale, să sorbiți și să vă desfătați la sânul slavei sale!
\par 12 Acestea zice Domnul: "Vărsa-voi pacea peste el ca un râu și slava popoarelor ca un șuvoi ieșit din albia lui. Pruncii lui vor fi purtați în brațe și dezmierdați pe genunchi.
\par 13 După cum mama își mângâie pe fiul ei și Eu vă voi mângâia pe voi, și voi veți fi mângâiați în Ierusalim.
\par 14 Când veți vedea, inima voastră va tresări de bucurie și oasele voastre vor odrăsli ca iarba. Și mâna Domnului se va arăta slujitorilor Săi, iar urgia, peste vrăjmașii Săi.
\par 15 Căci Domnul vine în văpaie și carele Lui sunt ca o vijelie, ca să dezlănțuie cu fierbințeală mânia Lui și certarea Lui cu văpăi de foc.
\par 16 Domnul va judeca cu foc și cu sabie pe tot omul și mulți vor fi cei ce vor cădea de bătaia Domnului!
\par 17 Faptele și gândurile celor ce se sfințesc și se curățesc pentru închinăciunile din grădini, într-un loc ascuns, în mijlocul unei adunări de ucenici, ale celor care mănâncă din carnea de porc, mâncăruri scârnave și șoareci, vor fi zădărnicite, zice Domnul.
\par 18 Dar Eu vin ca să strâng la un loc popoarele și toate limbile. Ele vor veni și vor vedea slava Mea,
\par 19 Și le voi da un semn. Și pe cei scăpați cu viață îi voi trimite către popoarele din Tarsis, Put, Lud, Meșec, Roș, Tubal, Iavan, către ținuturile cele mai depărtate care n-au auzit despre Mine și nu au văzut slava Mea. Și la aceste neamuri vor vesti slava Mea.
\par 20 Și din toate neamurile vor aduce pe frații voștri prinos Domnului: pe cai, în căruțe, pe paturi, pe catâri și pe cămile, până la muntele cel sfânt al Meu, la Ierusalim, zice Domnul, precum fiii lui Israel aduc prinoase în vase curate pentru templul Domnului.
\par 21 Și din ei voi lua preoți și leviți, zice Domnul.
\par 22 Într-adevăr, precum cerul cel nou și pământul cel nou pe care le voi face, zice Domnul, vor rămânea înaintea Mea, așa va dăinui totdeauna seminția voastră și numele vostru.
\par 23 Și din lună nouă în lună nouă și din zi de odihnă în zi de odihnă vor veni toți și se vor închina înaintea Mea, zice Domnul.
\par 24 Și când vor ieși, vor vedea trupurile moarte ale celor care s-au răzvrătit împotriva Mea, că viermele lor nu va muri și focul lor nu se va stinge. Și ei vor fi o sperietoare pentru toți.




\end{document}