\begin{document}

\title{2 Corinteni}


\chapter{1}

\par 1 Pavel, apostol al lui Hristos Iisus, prin voința lui Dumnezeu, și Timotei, fratele: Bisericii lui Dumnezeu celei din Corint, împreună cu toți sfinții care sunt în toată Ahaia:
\par 2 Har vouă și pace de la Dumnezeu Tatăl nostru și de la Domnul Iisus Hristos.
\par 3 Binecuvântat este Dumnezeu și Tatăl Domnului nostru Iisus Hristos, Părintele îndurărilor și Dumnezeul a toată mângâierea,
\par 4 Cel ce ne mângâie pe noi în tot necazul nostru, ca să putem să mângâiem și noi pe cei care se află în tot necazul, prin mângâierea cu care noi înșine suntem mângâiați de Dumnezeu.
\par 5 Că precum prisosesc pătimirile lui Hristos întru noi, așa prisosește prin Hristos și mângâierea noastră.
\par 6 Deci fie că suntem strâmtorați, este pentru a voastră mângâiere și mântuire, fie că suntem mângâiați, este pentru a voastră mângâiere, care vă dă putere să îndurați cu răbdare aceleași suferințe pe care le suferim și noi.
\par 7 Și nădejdea noastră este tare pentru voi, știind că precum sunteți părtași suferințelor, așa și mângâierii.
\par 8 Căci nu voim, fraților, ca voi să nu știți de necazul nostru, care ni s-a făcut în Asia, că peste măsură, peste puteri, am fost îngreuiați, încât nu mai nădăjduiam să mai scăpăm cu viață.
\par 9 Ci noi, în noi înșine, ne-am socotit ca osândiți la moarte, ca să nu ne punem încrederea în noi, ci în Dumnezeu, Cel ce înviază pe cei morți,
\par 10 Care ne-a izbăvit pe noi dintr-o moarte ca aceasta și ne izbăvește și în Care nădăjduim că încă ne va mai izbăvi,
\par 11 Ajutându-ne și voi cu rugăciunea pentru noi, așa încât darul acesta făcut nouă, prin rugăciunea multora, să ne fie prilej de mulțumire adusă de către mulți, pentru noi.
\par 12 Căci lauda noastră aceasta este: mărturia conștiinței noastre că am umblat în lume, și mai ales la voi, în sfințenie și în curăție dumnezeiască, nu în înțelepciune trupească, ci în harul lui Dumnezeu.
\par 13 Căci nu vă scriem vouă altele decât cele ce citiți și înțelegeți. Și am nădejde că până în sfârșit veți înțelege;
\par 14 După cum ne-ați și înțeles în parte, - că noi suntem lauda voastră, precum și voi lauda noastră, în ziua Domnului nostru Iisus.
\par 15 Cu această încredințare voiam să vin întâi la voi, ca să aveți bucurie a doua oară,
\par 16 Și să trec pe la voi în Macedonia și din Macedonia iarăși să vin la voi și să fiu petrecut de voi în Iudeea.
\par 17 Deci, aceasta voind, m-am purtat, oare, cu ușurință? Sau cele ce hotărăsc, le hotărăsc trupește ca la mine da, da să fie și nu, nu?
\par 18 Credincios este Dumnezeu, că n-a fost cuvântul nostru către voi da și nu.
\par 19 Fiul lui Dumnezeu, Iisus Hristos, Cel propovăduit vouă - prin noi, prin mine, prin Silvan și prin Timotei - nu a fost da și nu, ci da a fost în El.
\par 20 Căci toate făgăduințele lui Dumnezeu, în El, sunt da; și prin El, amin, spre slava lui Dumnezeu prin noi.
\par 21 Iar Cel ce ne întărește pe noi împreună cu voi, în Hristos, și ne-a uns pe noi este Dumnezeu,
\par 22 Care ne-a și pecetluit pe noi și a dat arvuna Duhului, în inimile noastre.
\par 23 Și eu chem pe Dumnezeu mărturie asupra sufletului meu, că din cruțare pentru voi n-am venit încă la Corint.
\par 24 Nu că doar avem stăpânire peste credința voastră, dar suntem împreună-lucrători ai bucuriei voastre; căci stați tari în credință.

\chapter{2}

\par 1 Și am judecat în mine aceasta, să nu vin iarăși la voi cu întristare.
\par 2 Căci dacă eu vă întristez, cine este cel care să mă înveselească, dacă nu cel întristat de mine?
\par 3 Și v-am scris vouă aceasta, ca nu cumva la venirea mea să am întristare de la aceia care trebuie să mă bucure, fiind încredințat despre voi toți că bucuria mea este și a voastră a tuturor.
\par 4 Căci din multă supărare și cu inima strânsă de durere, v-am scris cu multe lacrimi, nu ca să vă întristați, ci ca să cunoașteți dragostea pe care o am cu prisosință către voi.
\par 5 Și dacă m-a întristat cineva, nu pe mine m-a întristat, ci în parte - ca să nu spun mai mult - pe voi toți.
\par 6 Destul este pentru un astfel de om pedeapsa aceasta dată de către cei mai mulți.
\par 7 Așa încât voi, dimpotrivă, mai bine să-l iertați și să-l mângâiați, ca să nu fie copleșit de prea multă întristare unul ca acesta.
\par 8 De aceea vă îndemn să întăriți în el dragostea.
\par 9 Căci pentru aceasta v-am și scris, ca să cunosc încercarea voastră, dacă sunteți ascultători în toate.
\par 10 Iar cui îi iertați ceva, îi iert și eu; pentru că și eu, dacă am iertat ceva, am iertat pentru voi, în fața lui Hristos,
\par 11 Ca să nu ne lăsăm covârșiți de satana, căci gândurile lui nu ne sunt necunoscute.
\par 12 Și venind eu la Troa, pentru Evanghelia lui Hristos, și ușa fiindu-mi deschisă în Domnul,
\par 13 N-am avut odihnă în duhul meu, fiindcă n-am găsit pe Tit, fratele meu, ci despărțindu-mă de ei, am plecat în Macedonia.
\par 14 Mulțumire fie adusă deci lui Dumnezeu, Celui ce ne face pururea biruitori în Hristos și descoperă prin noi, în tot locul, mireasma cunoștinței Sale!
\par 15 Pentru că suntem lui Dumnezeu bună mireasmă a lui Hristos între cei ce se mântuiesc și între cei ce pier;
\par 16 Unora, adică, mireasmă a morții spre moarte, iar altora mireasmă a vieții spre viață. Și pentru acestea, cine e destoinic?
\par 17 Căci nu suntem ca cei mulți, care strică cuvântul lui Dumnezeu, ci grăim ca din curăția inimii, ca de la Dumnezeu înaintea lui Dumnezeu, în Hristos.

\chapter{3}

\par 1 Au doară începem iarăși să spunem cine suntem? Sau nu cumva avem nevoie - cum au unii - de scrisori de laudă către voi sau de la voi?
\par 2 Scrisoarea noastră sunteți voi, scrisă în inimile noastre, cunoscută și citită de toți oamenii,
\par 3 Arătându-vă că sunteți scrisoare a lui Hristos, slujită de noi, scrisă nu cu cerneală, ci cu Duhul Dumnezeului celui viu, nu pe table de piatră, ci pe tablele de carne ale inimii.
\par 4 Și o astfel de încredere avem în Hristos față de Dumnezeu;
\par 5 Nu că de la noi înșine suntem destoinici să cugetăm ceva ca de la noi înșine, ci destoinicia noastră este de la Dumnezeu,
\par 6 Cel ce ne-a învrednicit să fim slujitori ai Noului Testament, nu ai literei, ci ai duhului; pentru că litera ucide, iar duhul face viu.
\par 7 Iar dacă slujirea cea spre moarte, săpată în litere, pe piatră, s-a făcut întru slavă, încât fiii lui Israel nu puteau să-și ațintească ochii la fața lui Moise, din pricina slavei celei trecătoare a feței lui,
\par 8 Cum să nu fie mai mult întru slavă slujirea Duhului?
\par 9 Căci de a avut parte de slavă slujirea care aduce osânda, cu mult mai mult prisosește în slavă slujirea dreptății.
\par 10 Și nici măcar nu este slăvit ceea ce era slăvit în această privință, față de slava cea covârșitoare.
\par 11 Căci dacă ce este trecător s-a săvârșit prin slavă, cu atât mai mult ce e netrecător va fi în slavă.
\par 12 Având deci o astfel de nădejde, noi lucrăm cu multă îndrăzneală,
\par 13 Și nu ca Moise, care își punea un văl pe fața sa, ca fiii lui Israel să nu privească sfârșitul a ceea ce era trecător.
\par 14 Dar mințile lor s-au învârtoșat, căci până în ziua de azi, la citirea Vechiului Testament, rămâne același văl, neridicându-se, căci el se desființează prin Hristos;
\par 15 Ci până astăzi, când se citește Moise, stă un văl pe inima lor;
\par 16 Iar când se vor întoarce către Domnul, vălul se va ridica.
\par 17 Domnul este Duh, și unde este Duhul Domnului, acolo este libertate.
\par 18 Iar noi toți, privind ca în oglindă, cu fața descoperită, slava Domnului, ne prefacem în același chip din slavă în slavă, ca de la Duhul Domnului.

\chapter{4}

\par 1 De aceea, având această slujire, după cum am fost miluiți, nu ne pierdem nădejdea,
\par 2 Ci ne-am lepădat de cele ascunse ale rușinii, neumblând în vicleșug, nici stricând cuvântul lui Dumnezeu, ci făcându-ne cunoscuți prin arătarea adevărului față de orice conștiință omenească înaintea lui Dumnezeu.
\par 3 Iar dacă Evanghelia noastră este încă acoperită, este pentru cei pierduți,
\par 4 În care Dumnezeul veacului acestuia a orbit mințile necredincioșilor, ca să nu le lumineze lumina Evangheliei slavei lui Hristos, Care este chipul lui Dumnezeu.
\par 5 Căci nu ne propovăduim pe noi înșine, ci pe Hristos Iisus, Domnul, iar noi înșine suntem slugile voastre, pentru Iisus.
\par 6 Fiindcă Dumnezeu, Care a zis: "Strălucească, din întuneric, lumina" - El a strălucit în inimile noastre, ca să strălucească cunoștința slavei lui Dumnezeu, pe fața lui Hristos.
\par 7 Și avem comoara aceasta în vase de lut, ca să se învedereze că puterea covârșitoare este a lui Dumnezeu și nu de la noi,
\par 8 În toate pătimind necaz, dar nefiind striviți; lipsiți fiind, dar nu deznădăjduiți;
\par 9 Prigoniți fiind, dar nu părăsiți; doborâți, dar nu nimiciți;
\par 10 Purtând totdeauna în trup omorârea lui Iisus, pentru ca și viața lui Iisus să se arate în trupul nostru.
\par 11 Căci pururea noi cei vii suntem dați spre moarte pentru Iisus, ca și viața lui Iisus să se arate în trupul nostru cel muritor.
\par 12 Astfel că în noi lucrează moartea, iar în voi viața.
\par 13 Dar având același duh al credinței, - după cum este scris: "Crezut-am, pentru aceea am și grăit", - și noi credem: pentru aceea și grăim,
\par 14 Știind că Cel ce a înviat pe Domnul Iisus ne va învia și pe noi cu Iisus și ne va înfățișa împreună cu voi.
\par 15 Căci toate sunt pentru voi, pentru ca, înmulțindu-se harul să prisosească prin mai mulți mulțumirea, spre slava lui Dumnezeu.
\par 16 De aceea nu ne pierdem curajul și, chiar dacă omul nostru cel din afară se trece, cel dinăuntru însă se înnoiește din zi în zi.
\par 17 Căci necazul nostru de acum, ușor și trecător, ne aduce nouă, mai presus de orice măsură, slavă veșnică covârșitoare,
\par 18 Neprivind noi la cele ce se văd, ci la cele ce nu se văd, fiindcă cele ce se văd sunt trecătoare, iar cele ce nu se văd sunt veșnice.

\chapter{5}

\par 1 Căci știm că, dacă acest cort, locuința noastră pământească, se va strica, avem zidire de la Dumnezeu, casă nefăcută de mână, veșnică, în ceruri.
\par 2 Căci de aceea și suspinăm, în acest trup, dorind să ne îmbrăcăm cu locuința noastră cea din cer,
\par 3 Dacă totuși vom fi găsiți îmbrăcați, iar nu goi.
\par 4 Că noi, cei ce suntem în cortul acesta, suspinăm îngreuiați, de vreme ce dorim să nu ne scoatem haina noastră, ci să ne îmbrăcăm cu cealaltă pe deasupra, ca ceea ce este muritor să fie înghițit de viață.
\par 5 Iar Cel ce ne-a făcut spre aceasta este Dumnezeu, Care ne-a dat nouă arvuna Duhului.
\par 6 Îndrăznind deci totdeauna și știind că, petrecând în trup, suntem departe de Domnul,
\par 7 Căci umblăm prin credință, nu prin vedere,
\par 8 Avem încredere și voim mai bine să plecăm din trup și să petrecem la Domnul.
\par 9 De aceea ne și străduim ca, fie că petrecem în trup, fie că plecăm din el, să fim bineplăcuți Lui.
\par 10 Pentru că noi toți trebuie să ne înfățișăm înaintea scaunului de judecată al lui Hristos, ca să ia fiecare după cele ce a făcut prin trup, ori bine, ori rău.
\par 11 Cunoscând deci frica de Domnul, căutăm să înduplecăm pe oameni, dar lui Dumnezeu Îi suntem binecunoscuți și nădăjduiesc că suntem binecunoscuți și în cugetele voastre.
\par 12 Căci nu vă spunem iarăși cine suntem, ci vă dăm prilej de laudă pentru noi, ca să aveți ce să spuneți acelora care se laudă cu fața și nu cu inima.
\par 13 Căci, dacă ne-am ieșit din fire, este pentru Dumnezeu, iar dacă suntem cu mintea întreagă, este pentru voi.
\par 14 Căci dragostea lui Hristos ne stăpânește pe noi care socotim aceasta, că dacă unul a murit pentru toți, au murit deci toți.
\par 15 Și a murit pentru toți, ca cei ce viază să nu mai loruși, ci Aceluia care, pentru ei, a murit și a înviat.
\par 16 De aceea, noi nu mai știm de acum pe nimeni după trup; chiar dacă am cunoscut pe Hristos după trup, acum nu-L mai cunoaștem.
\par 17 Deci, dacă este cineva în Hristos, este făptură nouă; cele vechi au trecut, iată toate s-au făcut noi.
\par 18 Și toate sunt de la Dumnezeu, Care ne-a împăcat cu Sine prin Hristos și Care ne-a dat nouă slujirea împăcării.
\par 19 Pentru că Dumnezeu era în Hristos, împăcând lumea cu Sine însuși, nesocotindu-le greșelile lor și punând în noi cuvântul împăcării.
\par 20 În numele lui Hristos, așadar, ne înfățișăm ca mijlocitori, ca și cum Însuși Dumnezeu v-ar îndemna prin noi. Vă rugăm, în numele lui Hristos, împăcați-vă cu Dumnezeu!
\par 21 Căci pe El, Care n-a cunoscut păcatul, L-a făcut pentru noi păcat, ca să dobândim, întru El, dreptatea lui Dumnezeu.

\chapter{6}

\par 1 Fiind, dar, împreună-lucrători cu Hristos, vă îndemnăm să nu primiți în zadar harul lui Dumnezeu.
\par 2 Căci zice: "La vreme potrivită te-am ascultat și în ziua mântuirii te-am ajutat"; iată acum vreme potrivită, iată acum ziua mântuirii,
\par 3 Nedând nici o sminteală întru nimic, ca să nu fie slujirea noastră defăimată,
\par 4 Ci în toate înfățișându-ne pe noi înșine ca slujitori ai lui Dumnezeu, în multă răbdare, în necazuri, în nevoi, în strâmtorări,
\par 5 În bătăi, în temniță, în tulburări, în osteneli, în privegheri, în posturi;
\par 6 În curăție, în cunoștință, în îndelungă-răbdare, în bunătate, în Duhul Sfânt, în dragoste nefățarnică;
\par 7 În cuvântul adevărului, în puterea lui Dumnezeu, prin armele dreptății, cele de-a dreapta și cele de-a stânga,
\par 8 Prin slavă și necinste, prin defăimare și laudă; ca niște amăgitori, deși iubitori de adevăr,
\par 9 Ca niște necunoscuți, deși bine cunoscuți, ca fiind pe pragul morții, deși iată că trăim, ca niște pedepsiți, dar nu uciși;
\par 10 Ca niște întristați, dar pururea bucurându-ne; ca niște săraci, dar pe mulți îmbogățind; ca unii care n-au nimic, dar toate le stăpânesc.
\par 11 O, corintenilor, gura noastră s-a deschis către voi, inima noastră s-a lărgit.
\par 12 În inima noastră nu sunteți la strâmtorare; dar strâmtorare este pentru noi, în inimile voastre.
\par 13 Plătiți-mi și voi aceeași plată, vă vorbesc ca unor copii ai mei - lărgiți și voi inimile voastre!
\par 14 Nu vă înjugați la jug străin cu cei necredincioși, căci ce însoțire are dreptatea cu fărădelegea? Sau ce împărtășire are lumina cu întunericul?
\par 15 Și ce învoire este între Hristos și Veliar sau ce parte are un credincios cu un necredincios?
\par 16 Sau ce înțelegere este între templul lui Dumnezeu și idoli? Căci noi suntem templu al Dumnezeului celui viu, precum Dumnezeu a zis că: "Voi locui în ei și voi umbla și voi fi Dumnezeul lor și ei vor fi poporul Meu".
\par 17 De aceea: "Ieșiți din mijlocul lor și vă osebiți, zice Domnul, și de ce este necurat să nu vă atingeți și Eu vă voi primi pe voi.
\par 18 Și voi fi vouă tată, și veți fi Mie fii și fiice", zice Domnul Atotțiitorul.

\chapter{7}

\par 1 Având deci aceste făgăduințe, iubiților, să ne curățim pe noi de toată întinarea trupului și a duhului, desăvârșind sfințenia în frica lui Dumnezeu.
\par 2 Faceți-ne loc în inimile voastre! N-am nedreptățit pe nimeni; n-am vătămat pe nimeni, n-am înșelat pe nimeni.
\par 3 Nu o spun spre osândirea voastră, căci v-am spus înainte că sunteți în inimile noastre, ca împreună să murim și împreună să trăim.
\par 4 Multă îmi este încrederea în voi! Multă îmi este lauda pentru voi! Umplutu-m-am de mângâiere! Cu tot necazul nostru, sunt covârșit de bucurie!
\par 5 Căci, după ce am sosit în Macedonia, trupul nostru n-a avut nici o odihnă, necăjiți fiind în tot felul: din afară lupte, dinăuntru temeri.
\par 6 Dar Dumnezeu, Cel ce mângâie pe cei smeriți, ne-a mângâiat pe noi cu venirea lui Tit.
\par 7 Și nu numai cu venirea lui, ci și cu mângâierea cu care el a fost mângâiat la voi, vestindu-ne nouă dorința voastră, plânsul vostru, râvna voastră pentru mine, ca eu mai mult să mă bucur.
\par 8 Că, chiar dacă v-am întristat prin scrisoare, nu-mi pare rău, deși îmi părea rău; căci văd că scrisoarea aceea, fie și numai pentru un timp, v-a întristat.
\par 9 Acum mă bucur, nu pentru că v-ați întristat, ci pentru că v-ați întristat spre pocăință. Căci v-ați întristat după Dumnezeu, ca să nu fiți întru nimic păgubiți de către noi.
\par 10 Căci întristarea cea după Dumnezeu aduce pocăință spre mântuire, fără părere de rău; iar întristarea lumii aduce moarte.
\par 11 Că iată, însăși aceasta, că v-ați întristat după Dumnezeu, câtă sârguință v-a adus, ba încă și dezvinovățire și mâhnire și teamă și dorință și râvnă și ispășire! Întru totul ați dovedit că voi înșivă sunteți curați în acest lucru.
\par 12 Deci, deși v-am scris, aceasta n-a fost din cauza celui ce a nedreptățit, nici din cauza celui ce a fost nedreptățit, ci ca să ne învedereze la voi sârguința voastră pentru noi, înaintea lui Dumnezeu.
\par 13 De aceea, ne-am mângâiat; dar pe lângă mângâierea noastră, ne-am bucurat peste măsură mai ales de bucuria lui Tit, căci duhul lui s-a liniștit din partea voastră a tuturor.
\par 14 Căci dacă m-am lăudat înaintea lui cu ceva pentru voi, n-am fost dat de rușine; ci precum toate vi le-am grăit întru adevăr, așa și lauda noastră pentru Tit s-a făcut adevăr.
\par 15 Și inima lui este și mai mult la voi, aducându-și aminte de ascultarea voastră a tuturor, cum l-ați primit cu frică și cu cutremur.
\par 16 Mă bucur că în toate pot să mă încred în voi.

\chapter{8}

\par 1 Și vă fac cunoscut, fraților, harul lui Dumnezeu cel dăruit în Bisericile Macedoniei;
\par 2 Că în multa lor încercare de necaz, prisosul bucuriei lor și sărăcia lor cea adâncă au sporit în bogăția dărniciei lor,
\par 3 Căci mărturisesc că de voia lor au dat, după putere și peste putere,
\par 4 Cu multă rugăminte cerând har de a lua și ei parte la ajutorarea sfinților.
\par 5 Și au făcut nu după cum au nădăjduit, ci s-au dat pe ei înșiși întâi Domnului și apoi nouă, prin voia lui Dumnezeu,
\par 6 Încât am rugat pe Tit ca, precum a început dinainte, așa să și desăvârșească, la voi, și darul acesta.
\par 7 Ci precum întru toate prisosiți: în credință, în cuvânt, în cunoștință, în orice sârguință, în iubirea voastră către noi, așa și în acest dar să prisosiți.
\par 8 Nu cu poruncă o spun, ci încercând și curăția dragostei voastre, prin sârguința altora.
\par 9 Căci cunoașteți harul Domnului nostru Iisus Hristos, că El, bogat fiind, pentru voi a sărăcit, ca voi cu sărăcia Lui să vă îmbogățiți.
\par 10 Și sfat vă dau în aceasta: că aceasta vă este de folos vouă, care încă de anul trecut ați început nu numai să faceți, ci să și voiți.
\par 11 Duceți dar acum până la capăt fapta, ca precum ați fost gata să voiți, tot așa să și îndepliniți din ce aveți.
\par 12 Căci dacă este bunăvoință, bine primit este darul, după cât are cineva, nu după cât nu are.
\par 13 Nu doar ca să fie altora ușurare, iar vouă necaz, ci ca să fie potrivire:
\par 14 Prisosința voastră să împlinească lipsa acelora, pentru ca și prisosința lor să împlinească lipsa voastră, spre a fi potrivire,
\par 15 Precum este scris: "Celui cu mult nu i-a prisosit, și celui cu puțin nu i-a lipsit".
\par 16 Mulțumire fie adusă lui Dumnezeu, Care a dat aceeași râvnă pentru voi în inima lui Tit.
\par 17 Căci, pe de o parte, a primit îndemnul nostru, iar, pe de altă parte, făcându-se și mai sârguitor, de bună voie a plecat către voi.
\par 18 Și am trimis împreună cu el și pe fratele a cărui laudă, întru Evanghelie, este în toate Bisericile;
\par 19 Dar nu numai atât, ci este și ales de către Biserici ca împreună-călător cu noi la darul acesta, slujit de noi, spre slava Domnului însuși și spre osârdia noastră.
\par 20 Prin aceasta ne ferim ca să nu ne defăimeze cineva, în această îmbelșugată strângere de daruri, de care ne îngrijim noi.
\par 21 Pentru că ne îngrijim de cele bune nu numai înaintea Domnului, ci și înaintea oamenilor.
\par 22 Și l-am trimis împreună cu ei și pe fratele nostru, pe care l-am încercat în multe, de multe ori, ca fiind sârguitor, iar acum este și mai sârguitor, prin multa încredere în voi.
\par 23 Astfel, dacă e vorba de Tit, el este însoțitorul meu și împreună-lucrător la voi; dacă e vorba despre frații noștri, ei sunt apostoli ai Bisericilor, slavă a lui Hristos.
\par 24 Arătați deci către ei, în fața Bisericilor, dovada dragostei voastre, ca și a laudei noastre pentru voi.

\chapter{9}

\par 1 Despre strângerea de ajutoare pentru sfinți îmi este de prisos să vă scriu.
\par 2 Că știu bunăvoința voastră, cu care, pentru voi, mă laud către macedoneni; că Ahaia s-a pregătit din anul trecut, și râvna voastră a însuflețit pe cei mai mulți.
\par 3 Am trimis dar pe frați, ca lauda noastră pentru voi, în privința aceasta, să nu fie zadarnică, ci să fiți gata, precum ziceam,
\par 4 Ca nu cumva, când macedonenii vor veni împreună cu mine și vă vor găsi nepregătiți, să fim rușinați noi, ca să nu zicem voi, în această laudă încrezătoare.
\par 5 Am socotit deci că este nevoie să îndemn pe frați să vină întâi la voi și să pregătească darul vostru cel dinainte făgăduit, ca el să fie gata, așa ca un dar, nu ca o faptă de zgârcenie.
\par 6 Aceasta însă zic: Cel ce seamănă cu zgârcenie, cu zgârcenie va și secera, iar cel ce seamănă cu dărnicie, cu dărnicie va și secera.
\par 7 Fiecare să dea cum socotește cu inima sa, nu cu părere de rău, sau de silă, căci Dumnezeu iubește pe cel care dă cu voie bună.
\par 8 Și Dumnezeu poate să înmulțească tot harul la voi, ca, având totdeauna toată îndestularea în toate, să prisosiți spre tot lucrul bun,
\par 9 Precum este scris: "Împărțit-a, dat-a săracilor; dreptatea Lui rămâne în veac".
\par 10 Iar Cel ce dă sămânță semănătorului și pâine spre mâncare, vă va da și va înmulți sămânța voastră și va face să crească roadele dreptății voastre,
\par 11 Ca întru toate să vă îmbogățiți, spre toată dărnicia care aduce prin noi mulțumire lui Dumnezeu.
\par 12 Pentru că slujirea acestui dar nu numai că împlinește lipsurile sfinților, ci prisosește prin multe mulțumiri în fața lui Dumnezeu;
\par 13 Slăvind ei pe Dumnezeu, prin adeverirea acestei ajutorări, pentru supunerea mărturisirii voastre Evangheliei lui Hristos și pentru dărnicia împărtășirii către ei și către toți,
\par 14 Se roagă pentru voi, și vă iubesc pentru harul lui Dumnezeu cel ce prisosește la voi.
\par 15 Iar lui Dumnezeu mulțumire pentru darul Său cel negrăit.

\chapter{10}

\par 1 Însumi eu, Pavel, vă îndemn prin blândețea și îngăduința lui Hristos - eu care de față sunt smerit între voi, dar, în lipsă, îndrăznesc față de voi S
\par 2 Vă rog, dar, să nu mă siliți, când voi fi de față, să îndrăznesc cu încrederea cu care gândesc că voi îndrăzni împotriva unora care ne socotesc că umblăm după trup.
\par 3 Pentru că, deși umblăm în trup, nu ne luptăm trupește.
\par 4 Căci armele luptei noastre nu sunt trupești, ci puternice înaintea lui Dumnezeu, spre dărâmarea întăriturilor. Noi surpăm iscodirile minții,
\par 5 Și toată trufia care se ridică împotriva cunoașterii lui Dumnezeu și tot gândul îl robim, spre ascultarea lui Hristos,
\par 6 Și gata suntem să pedepsim toată neascultarea, atunci când supunerea voastră va fi deplină.
\par 7 Judecați lucrurile așa cum se arată: dacă cineva are încredere în sine că este al lui Hristos, să gândească iarăși de la sine aceasta, că precum este el al lui Hristos, tot așa suntem și noi.
\par 8 Și chiar de mă voi lăuda, ceva mai mult, cu puterea noastră, pe care ne-a dat-o Domnul spre zidirea, iar nu spre dărâmarea voastră, nu mă voi rușina,
\par 9 Ca să nu par că v-aș înfricoșa prin scrisori.
\par 10 Că scrisorile lui, zic ei, sunt grele și tari, dar înfățișarea trupului este slabă și cuvântul lui este disprețuit.
\par 11 Cel ce vorbește astfel să-și dea seama că așa cum suntem cu cuvântul prin scrisori, când nu suntem de față, tot așa suntem și cu fapta, când suntem de față.
\par 12 Căci nu îndrăznim să ne numărăm sau să ne asemănăm cu unii care se laudă singuri; dar aceia, măsurându-se și asemănându-se pe ei cu ei înșiși, nu au pricepere.
\par 13 Iar noi nu ne vom lăuda fără măsură, ci după măsura dreptarului cu care ne-a măsurat nouă Dumnezeu, ca să ajungem și până la voi.
\par 14 Căci nu ne întindem peste măsură, ca și cum n-am fi ajuns la voi, căci am și ajuns cu Evanghelia lui Hristos până la voi.
\par 15 Nu ne lăudăm peste măsură cu ostenelile altora, ci avem nădejde că, tot crescând credința voastră, ne vom mări în voi cu prisosință, după măsura noastră,
\par 16 Ca să propovăduim Evanghelia și în ținuturile de dincolo de voi, dar fără să ne lăudăm cu măsură străină, în cele de-a gata.
\par 17 Iar cel ce se laudă, în Domnul să se laude.
\par 18 Pentru că nu cel ce se laudă singur este dovedit bun, ci acela pe care Domnul îl laudă.

\chapter{11}

\par 1 O, de mi-ați îngădui puțină neînțelepție! Dar îmi și îngăduiți,
\par 2 Căci vă râvnesc pe voi cu râvna lui Dumnezeu, pentru că v-am logodit unui singur bărbat, ca să vă înfățișez lui Hristos fecioară neprihănită.
\par 3 Dar mă tem ca nu cumva, precum șarpele a amăgit pe Eva în viclenia lui, tot așa să se abată și gândurile voastre de la curăția și nevinovăția cea în Hristos.
\par 4 Căci dacă cel ce vine vă propovăduiește un alt Iisus, pe care nu l-am propovăduit noi, sau luați un alt duh, pe care nu l-ați luat, sau altă evanghelie pe care nu ați primit-o, - voi l-ați îngădui foarte bine.
\par 5 Dar eu socotesc că nu sunt cu nimic mai prejos decât cei mai de frunte dintre apostoli.
\par 6 Și chiar dacă sunt neiscusit în cuvânt, nu însă în cunoștință, ci v-am dovedit-o în totul față de voi toți.
\par 7 Sau am făcut păcat că v-am propovăduit în dar Evanghelia lui Dumnezeu, smerindu-mă pe mine însumi, pentru ca voi să vă înălțați?
\par 8 Alte Biserici am prădat, luând plată ca să vă slujesc pe voi.
\par 9 Și de față fiind la voi și în lipsuri aflându-mă, n-am făcut supărare nimănui. Căci în cele ce mi-au lipsit, m-au îndestulat frații veniți din Macedonia. Și în toate m-am păzit și mă voi păzi, să nu vă fiu povară.
\par 10 Este în mine adevărul lui Hristos, că lauda aceasta nu-mi va fi îngrădită în ținuturile Ahaei.
\par 11 Pentru ce? Pentru că nu vă iubesc? Dumnezeu știe!
\par 12 Dar ceea ce fac, voi face și în viitor, ca să tai pricina celor ce poftesc pricină, pentru a se afla ca și noi în ceea ce se laudă.
\par 13 Pentru că unii ca aceștia sunt apostoli mincinoși, lucrători vicleni, care iau chip de apostoli ai lui Hristos.
\par 14 Nu este de mirare, deoarece însuși satana se preface în înger al luminii.
\par 15 Nu este deci lucru mare dacă și slujitorii lui iau chip de slujitori ai dreptății, al căror sfârșit va fi după faptele lor.
\par 16 Iarăși zic: Să nu mă socotească cineva că sunt fără minte, iar de nu primiți-mă măcar ca pe un fără-de-minte, ca să mă laud și eu puțin.
\par 17 Ceea ce grăiesc, nu după Domnul grăiesc, ci ca în neînțelepție, în această stare de laudă.
\par 18 Deoarece mulți se laudă după trup, mă voi lăuda și eu.
\par 19 Pentru că voi, înțelepți fiind, îngăduiți bucuros pe cei neînțelepți.
\par 20 Căci de vă robește cineva, de vă mănâncă cineva, de vă ia ce e al vostru, de vă privește cineva cu mândrie, de vă lovește cineva peste obraz, - răbdați.
\par 21 Spre necinste o spun, că noi ne-am arătat slabi. Dar în orice ar cuteza cineva - întru neînțelepție zic, - cutez și eu!
\par 22 Sunt ei evrei? Sunt și eu. Sunt ei israeliți? Israelit sunt și eu. Sunt ei sămânța lui Avraam? Sunt și eu.
\par 23 Sunt ei slujitori ai lui Hristos? Nebunește spun: eu nu mai mult ca ei! În osteneli mai mult, în închisori mai mult, în bătăi peste măsură, la moarte adeseori.
\par 24 De la iudei, de cinci ori am luat patruzeci de lovituri de bici fără una.
\par 25 De trei ori am fost bătut cu vergi; o dată am fost bătut cu pietre; de trei ori s-a sfărâmat corabia cu mine; o noapte și o zi am petrecut în largul mării.
\par 26 În călătorii adeseori, în primejdii de râuri, în primejdii de la tâlhari, în primejdii de la neamul meu, în primejdii de la păgâni; în primejdii în cetăți, în primejdii în pustie, în primejdii pe mare, în primejdii între frații cei mincinoși;
\par 27 În osteneală și în trudă, în privegheri adeseori, în foame și în sete, în posturi de multe ori, în frig și în lipsă de haine.
\par 28 Pe lângă cele din afară, ceea ce mă împresoară în toate zilele este grija de toate Bisericile.
\par 29 Cine este slab și eu să nu fiu slab? Cine se smintește și eu să nu ard?
\par 30 Dacă trebuie să mă laud, mă voi lăuda cu cele ale slăbiciunii mele!
\par 31 Dumnezeu și Tatăl Domnului nostru Iisus, Cel ce este binecuvântat în veci, știe că nu mint!
\par 32 În Damasc, dregătorul regelui Areta păzea cetatea Damascului, ca să mă prindă,
\par 33 Și printr-o fereastră am fost lăsat în jos, peste zid, într-un coș, și am scăpat din mâinile lui.

\chapter{12}

\par 1 Dacă trebuie să mă laud, nu-mi este de folos, dar voi veni totuși la vedenii și la descoperiri de la Domnul.
\par 2 Cunosc un om în Hristos, care acum paisprezece ani - fie în trup, nu știu; fie în afară de trup, nu știu, Dumnezeu știe - a fost răpit unul ca acesta până la al treilea cer.
\par 3 Și-l știu pe un astfel de om - fie în trup, fie în afară de trup, nu știu, Dumnezeu știe S
\par 4 Că a fost răpit în rai și a auzit cuvinte de nespus, pe care nu se cuvine omului să le grăiască.
\par 5 Pentru unul ca acesta mă voi lăuda; iar pentru mine însumi nu mă voi lăuda decât numai în slăbiciunile mele.
\par 6 Căci chiar dacă aș vrea să mă laud, nu voi fi fără minte, căci voi spune adevărul; dar mă feresc de aceasta, ca să nu mă socotească nimeni mai presus decât ceea ce vede sau aude de la mine.
\par 7 Și pentru ca să nu mă trufesc cu măreția descoperirilor, datu-mi-s-a mie un ghimpe în trup, un înger al satanei, să mă bată peste obraz, ca să nu mă trufesc.
\par 8 Pentru aceasta de trei ori am rugat pe Domnul ca să-l îndepărteze de la mine;
\par 9 Și mi-a zis: Îți este de ajuns harul Meu, căci puterea Mea se desăvârșește în slăbiciune. Deci, foarte bucuros, mă voi lăuda mai ales întru slăbiciunile mele, ca să locuiască în mine puterea lui Hristos.
\par 10 De aceea mă bucur în slăbiciuni, în defăimări, în nevoi, în prigoniri, în strâmtorări pentru Hristos, căci, când sunt slab, atunci sunt tare.
\par 11 M-am făcut ca unul fără minte, lăudându-mă. Voi m-ați silit! Căci se cuvenea să vorbiți voi de bine despre mine, pentru că nu sunt cu nimic mai prejos decât cei mai de frunte dintre apostoli, deși nu sunt nimic.
\par 12 Dovezile mele de apostol s-au arătat la voi în toată răbdarea, prin semne, prin minuni și prin puteri.
\par 13 Căci cu ce sunteți voi mai prejos decât celelalte Biserici, decât numai că eu nu v-am fost povară? Dăruiți-mi mie această nedreptate.
\par 14 Iată, a treia oară sunt gata să vin la voi și nu vă voi fi povară, căci nu caut ale voastre, ci pe voi. Pentru că nu copiii sunt datori să agonisească pentru părinți, ci părinții pentru copii.
\par 15 Deci eu foarte bucuros voi cheltui și mă voi cheltui pentru sufletele voastre, deși, iubindu-vă mai mult, eu sunt iubit mai puțin.
\par 16 Dar fie! Eu nu v-am împovărat. Ci, fiind isteț, v-am prins cu înșelăciune.
\par 17 Am tras eu folos de la voi, prin vreunul din aceia pe care i-am trimis?
\par 18 L-am rugat pe Tit și am trimis, împreună cu el, pe fratele. V-a asuprit Tit cu ceva? N-am umblat noi în același duh? N-am călcat noi pe aceleași urme?
\par 19 De mult vi se pare că ne apărăm față de voi. Dar noi grăim în Hristos, înaintea lui Dumnezeu. Și toate acestea, iubiții mei, pentru zidirea voastră.
\par 20 Căci mă tem ca nu cumva venind, să nu vă găsesc pe voi așa precum voiesc, iar eu să fiu găsit de voi așa precum nu voiți; mă tem adică de certuri, de pizmă, de mânii, de întărâtări, de clevetiri, de murmure, de îngâmfări, de tulburări;
\par 21 Mă tem ca nu cumva, venind iarăși, să mă umilească Dumnezeul meu la voi și să plâng pe mulți care au păcătuit înainte și nu s-au pocăit de necurăția și de desfrânarea și de necumpătarea pe care le-au făcut.

\chapter{13}

\par 1 A treia oară vin la voi. În gura a doi sau trei martori va sta tot cuvântul.
\par 2 Am spus dinainte și spun iarăși dinainte, ca atunci când am fost de față a doua oară, și acum, nefiind de față, scriu celor ce au păcătuit înainte și tuturor celorlalți că, de voi veni iarăși, nu voi cruța,
\par 3 De vreme ce voi căutați dovadă că Hristos grăiește întru mine, Care nu este slab față de voi, ci puternic în voi.
\par 4 Căci, deși a fost răstignit din slăbiciune, din puterea lui Dumnezeu este însă viu. Și noi suntem slabi întru El, dar vom fi împreună cu El, din puterea lui Dumnezeu față de voi.
\par 5 Cercetați-vă pe voi înșivă dacă sunteți în credință; încercați-vă pe voi înșivă. Sau nu vă cunoașteți voi singuri bine că Hristos Iisus este întru voi? Afară numai dacă nu sunteți netrebnici.
\par 6 Nădăjduiesc însă că veți cunoaște că noi nu suntem netrebnici.
\par 7 Și ne rugăm lui Dumnezeu ca să nu săvârșiți voi nici un rău, nu ca să ne arătăm noi încercați, ci pentru ca voi să faceți binele, iar noi să fim ca niște netrebnici.
\par 8 Căci împotriva adevărului n-avem nici o putere; avem pentru adevăr.
\par 9 Căci ne bucurăm când noi suntem slabi, iar voi sunteți tari. Aceasta și cerem în rugăciunea noastră: desăvârșirea voastră.
\par 10 Pentru aceea vă scriu acestea, nefiind de față, ca atunci, când voi fi de față, să nu cutez cu asprime, după puterea pe care mi-a dat-o Domnul spre zidire, iar nu spre dărâmare.
\par 11 Deci, fraților, bucurați-vă! Desăvârșiți-vă, mângâiați-vă, fiți uniți în cuget, trăiți în pace și Dumnezeul dragostei și al păcii va fi cu voi.
\par 12 Îmbrățișați-vă unii pe alții cu sărutare sfântă.
\par 13 Sfinții toți vă îmbrățișează.
\par 14 Harul Domnului nostru Iisus Hristos și dragostea lui Dumnezeu și împărtășirea Sfântului Duh să fie cu voi cu toți!


\end{document}