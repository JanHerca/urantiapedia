\begin{document}

\title{Eclesiastul}


\chapter{1}

\par 1 Cuvintele Ecclesiastului, fiul lui David, rege în Ierusalim.
\par 2 Deșertăciunea deșertăciunilor, zice Ecclesiastul, deșertăciunea deșertăciunilor, toate sânt deșertăciuni!
\par 3 Ce folos are omul din toată truda lui cu care se trudește sub soare?
\par 4 Un neam trece și altul vine, dar pământul rămâne totdeauna!
\par 5 Soarele răsare, soarele apune și zorește către locul lui ca să răsară iarăși.
\par 6 Vântul suflă către miazăzi, vântul se întoarce către miazănoapte și, făcând roate-roate, el trece neîncetat prin cercurile sale.
\par 7 Toate fluviile curg în mare, dar marea nu se umple, căci ele se întorc din nou la locul din care au plecat.
\par 8 Toate lucrurile se zbuciumă mai mult decât poate omul să o spună: ochiul nu se satură de câte vede și urechea nu se umple de câte aude.
\par 9 Ceea ce a mai fost, aceea va mai fi, și ceea ce s-a întâmplat se va mai petrece, căci nu este nimic nou sub soare.
\par 10 Dacă este vreun lucru despre care să se spună: "Iată ceva nou!" aceasta a fost în vremurile străvechi, de dinaintea noastră.
\par 11 Nu ne aducem aminte despre cei ce au fost înainte, și tot așa despre cei ce vor veni pe urmă; nici o pomenire nu va fi la urmașii lor.
\par 12 Eu  Ecclesiastul  am fost regele lui Israel în Ierusalim.
\par 13 Și m-am sârguit în inima mea să cercetez și să iau aminte cu înțelepciune la tot ceea ce se petrece sub cer. Acesta este un chin cumplit pe care Dumnezeu l-a dat fiilor oamenilor, ca să se chinuiască întru el.
\par 14 M-am uitat cu luare aminte la toate lucrările care se fac sub soare și iată: totul este deșertăciune și vânare de vânt.
\par 15 Ceea ce este strâmb nu se poate îndrepta și ceea ce lipsește nu se poate număra.
\par 16 Grăit-am în inima mea: Cu adevărat am adunat și am strâns înțelepciune - mai mult decât toți cei care au fost înaintea mea în Ierusalim căci inima mea a avut cu belșug înțelepciune și știind.
\par 17 Și mi-am silit inima ca să pătrund înțelepciunea și știința, nebunia și prostia, dar am înțeles că și aceasta este vânare de vânt,
\par 18 Că unde este multă înțelepciune este și multă amărăciune, și cel ce își înmulțește știința își sporește suferința.

\chapter{2}

\par 1 Grăiam inimii mele: "Vino să te ispitesc cu veselia și să te fac să guști plăcerea!" Și iată că și aceasta este deșertăciune.
\par 2 "E nebunie!" am zis despre râs. Și despre veselie: "La ce poate să folosească!"
\par 3 Am cugetat apoi în inima mea, să desfătez trupul meu cu vin - pe când cugetul meu umbla după înțelepciune și cerceta nebunia - până ce voi vedea ceea ce este bun pentru fiii oamenilor să facă sub cer în vremea vieții lor.
\par 4 Am început lucrări mari: am zidit case, am sădit vii,
\par 5 Am făcut grădini și parcuri și am sădit în ele tot felul de pomi roditori;
\par 6 Mi-am făcut iazuri, ca să pot uda din ele o dumbravă unde creșteau copacii;
\par 7 Am cumpărat robi și roabe și am avut feciori născuți în casă, asemenea și turme de vite, și oi fără de număr, mai mult decât toți cei care au fost înaintea mea în Ierusalim.
\par 8 Am strâns aur și argint și număr mare de regi și de satrapi; am adus cântăreți și cântărețe și desfătarea fiilor anului mi-am agonisit: o prințesă și alte prințese.
\par 9 Am fost mare și am întrecut pe toți cei ce au trăit înaintea mea în Ierusalim și înțelepciunea a rămas cu mine.
\par 10 Și tot ceea ce doreau ochii mei nu am dat la o parte și n-am oprit inima mea de la nici o veselie, căci inima mea s-a bucurat de toată osteneala mea, și aceasta mi-a fost partea din toată munca mea.
\par 11 Apoi m-am uitat cu luare aminte la toate lucrurile pe care le-au făcut mâinile mele și la truda cu care m-am trudit ca să le săvârșesc și iată, totul este deșertăciune și vânare de vânt și fără nici un folos sub soare.
\par 12 Și mi-am întors privirea să văd înțelepciunea, nebunia și prostia. Căci ce poate să facă un om de rând peste ceea ce a făcut un rege?
\par 13 Atunci m-am încredințat că înțelepciunea are întâietate asupra nebuniei tot atât cât are lumina asupra întunericului.
\par 14 Înțeleptul are ochii în cap, iar nebunul merge întru întuneric. Dar am cunoscut și eu că aceeași soartă vor avea toți.
\par 15 Deci am zis în inima mea: "Aceeași soartă ca și cel nebun avea-voi și eu; atunci la ce îmi folosește înțelepciunea?" Și am zis în mintea mea că și aceasta este deșertăciune.
\par 16 Căci pomenirea celui înțelept ca și a celui nebun nu este veșnică, fiindcă în zilele ce vor veni amândoi vor fi uitați; atunci înțeleptul moare ca și nebunul.
\par 17 Drept aceea am urât viața, căci rele sunt cele ce se fac sub soare; și totul este deșertăciune și vânare de vânt.
\par 18 Și am urât toată munca pe care am muncit-o sub soare, fiindcă voi lăsa-o omului care va veni după mine.
\par 19 Și cine știe dacă el va fi înțelept sau nebun! Și el va face ce va găsi cu cale din tot lucrul cu care m-am trudit și m-am chibzuit sub soare! Și aceasta este deșertăciune!
\par 20 Și am început să mă las deznădejdii pentru toată munca cea de sub soare,
\par 21 Căci un om care a pus în lucrul lui înțelepciune și știință și a avut izbândă, îl împarte cu cel care n-a lucrat. Și aceasta aste deșertăciune și un rău nespus de mare.
\par 22 Oare, ce-i rămâne omului din toată munca lui și din grija inimii lui cu care s-a trudit sub soare?
\par 23 Toate zilele lui nu sunt decât suferință și îndeletnicirea lui nu-i decât necaz; nici chiar noaptea n-are odihnă inima lui. Și aceasta este deșertăciune!
\par 24 Nimic nu este mai bun pentru om decât să mănânce și să bea și să-și desfăteze sufletul cu mulțumirea din munca sa. Și am văzut că și aceasta vine numai din mâna lui Dumnezeu.
\par 25 Cine poate oare să mănânce și să bea fără să mulțumească Lui?
\par 26 Omului care este bun înaintea lui Dumnezeu, Dumnezeu îi dă înțelepciune, știință și bucurie, iar păcătosului îi dă sarcina să adune și să strângă pentru a da celui ce este bun în fața lui Dumnezeu. Și aceasta este deșertăciune și vinare de vânt!

\chapter{3}

\par 1 Pentru orice lucru este o clipă prielnică și vreme pentru orice îndeletnicire de sub cer.
\par 2 Vreme este să te naști și vreme să mori; vreme este să sădești și vreme să smulgi ceea ce ai sădit.
\par 3 Vreme este să rănești și vreme să tămăduiești; vreme este să dărâmi și vreme să zidești.
\par 4 Vreme este să plângi și vreme să râzi; vreme este să jelești și vreme să dănțuiești.
\par 5 Vreme este să arunci pietre și vreme să le strângi; vreme este să îmbrățișezi și vreme este să fugi de îmbrățișare.
\par 6 Vreme este să agonisești și vreme să prăpădești; vreme este să păstrezi și vreme să arunci.
\par 7 Vreme este să rupi și vreme să coși; vreme este să taci și vreme să grăiești.
\par 8 Vreme este să iubești și vreme să urăști. Este vreme de război și vreme de pace.
\par 9 Care este folosul celui ce lucrează întru osteneala pe care o ia asupră-și?
\par 10 Am văzut zbuciumul pe care l-a dat Dumnezeu fiilor oamenilor, ca să se zbuciume.
\par 11 Toate le-a făcut Dumnezeu frumoase și la timpul lor; El a pus în inima lor și veșnicia, dar fără ca omul să poată înțelege lucrarea pe care o face Dumnezeu, de la început până la sfârșit.
\par 12 Atunci mi-am dat cu socoteala că nu este fericire decât să te bucuri și să trăiești bine în timpul vieții tale.
\par 13 Drept aceea dacă un om mănâncă și bea și trăiește bine de pe urma muncii lui, acesta este un dar de la Dumnezeu.
\par 14 Atunci mi-am dat seama că tot ceea ce a făcut Dumnezeu va ține în veac de veac și nimic nu se poate adăuga, nici nu se poate micșora și că Dumnezeu lucrează așa ca să ne temem de fața Lui.
\par 15 Ceea ce este a mai fost și ceea ce va mai fi a fost în alte vremuri; și Dumnezeu cheamă iarăși aceea ce a lăsat să treacă.
\par 16 Dar am mai văzut sub soare că în locul dreptății este fărădelegea și în locul celui cucernic, cel nelegiuit.
\par 17 Și am gândit în inima mea: "Dumnezeu va judeca pe cel drept ca și pe cel nelegiuit", căci este vreme pentru orice punere la cale și pentru orice faptă.
\par 18 și am zis iar în inima mea despre fiii oamenilor: "Dumnezeu a orânduit să-i încerce, ca ei să-și dea seama că nu sânt decât dobitoace".
\par 19 Căci soarta omului și soarta dobitocului este aceeași: precum moare unul, moare și celălalt și toți au un singur duh de viață, iar omul nu are nimic mai mult decât dobitocul. Și totul este deșertăciune!
\par 20 Amândoi merg în același loc: amândoi au ieșit din pulbere și amândoi în pulbere se întorc.
\par 21 Cine știe dacă duhul omului se urcă în sus și duhul dobitocului se coboară în jos către pământ?
\par 22 Și mi-am dat seama că nimic nu este mai de preț pentru om decât să se bucure de lucrurile sale, că aceasta este partea lui, fiindcă cine îi va da lui putere să mai vadă ceea ce se va întâmpla în urma lui?

\chapter{4}

\par 1 Și iarăși am luat aminte la toate silniciile care se săvârșesc sub soare. Și iată lacrimile celor apăsați și nimeni nu era care să-i mângâie, iar în mâna celor silnici toată asuprirea și nici un mângâietor nu se găsea!
\par 2 Și am fericit pe cei ce au murit în vremi străvechi mai mult decât pe cei vii care sânt acum în viață.
\par 3 Iar mai fericit și decât unii și decât alții este cel ce n-a venit pe lume, cel care n-a văzut faptele cele rele care se săvârșesc sub soare.
\par 4 Și am văzut că toată strădania și toată izbânda omului la lucru nu este decât pizma unuia față de altul. Și aceasta este deșertăciune și vânare de vânt!
\par 5 Nebunul stă cu mâinile în sân și își mănâncă singur timpul zicând:
\par 6 "Mai de preț este un pumn plin de odihnă decât doi pumni plini de trudă și de vânare de vânt".
\par 7 Și iarăși am văzut o nepotrivire sub soare:
\par 8 Este câte un om care este stingher și care nu are nici copil, nici frate și totuși lucrul nu-l mai sfârșește și ochii lui nu se mai satură de bogăție. Dar vine o vreme când zice: "Pentru cine m-am trudit și am lipsit sufletul meu de traiul cel bun?" Și aceasta este deșertăciune și rea îndeletnicire.
\par 9 Mai fericiți sânt doi laolaltă decât unul, fiindcă au răsplată bună pentru munca lor;
\par 10 Căci dacă unul cade, îl scoală tovarășul lui. Dar vai de cel singur care cade și nu este cel de-al doilea ca să-l ridice!
\par 11 Asemenea când doi se culcă se încălzesc, iar unul cum s-ar putea încălzi?
\par 12 Și dacă unul este luat fără de veste, cel de-al doilea sare pentru el; căci sfoara pusă în trei nu se rupe degrabă.
\par 13 Mai de preț este un copil sărman și înțelept decât un rege bătrân și nebun, care nu mai este în stare să asculte de sfaturi;
\par 14 Căci el poate să iasă din închisoare ca să domnească, deși s-a născut sărac în țara celuilalt.
\par 15 Văzut-am pe toți cei vii care merg sub soare îmbulzindu-se lângă tânărul care va sta în locul regelui ca moștenitor.
\par 16 Și nu se mai sfârșea poporul în fruntea căruia era; totuși urmașii lui nu se vor bucura de el. Și aceasta este deșertăciune și vânare de vânt.

\chapter{5}

\par 1 Ia seama la picioarele tale când te duci în templul Domnului. Dacă te apropii să asculți este mai bine, decât să aduci jertfa nebunilor, căci ei nu știu decât să facă rău.
\par 2 Nu te grăbi să deschizi gura ta și inima ta să nu se pripească să scoată o vorbă înaintea lui Dumnezeu, că Dumnezeu este în ceruri, iar tu pe pământ; pentru aceasta să fie cuvintele tale puține.
\par 3 Visurile vin din multele griji, iar glasul celui nebun din mulțimea de vorbe.
\par 4 Dacă ai făcut un jurământ lui Dumnezeu, nu pierde din vedere să-l împlinești, că nebunii nu au nici o trecere; tu însă împlinește ce ai făgăduit.
\par 5 Mai bine să nu făgăduiești decât să nu împlinești ce ai făgăduit.
\par 6 Nu îngădui gurii tale să tragă spre păcat trupul tău și înaintea trimisului lui Dumnezeu nu spune: "A fost o rătăcire!" Pentru ce să Se mânie Dumnezeu de cuvântul tău și să nimicească lucrul mâinilor tale?
\par 7 Căci din mulțimea grijilor se nasc visele și deșertăciunile din prea multe cuvinte. De aceea, teme-te de Dumnezeu!
\par 8 Dacă vezi asuprirea celui sărac și obijduirea dreptului și a dreptății în cetate, nu te mira de lucrul acesta, căci peste cel mare este unul mai mare, iar Cel Atotputernic veghează peste toți.
\par 9 Totuși este un folos pentru țară și anume: un rege care să poarte grijă muncii pământului.
\par 10 Cine iubește banii nu se va sătura de bani, iar cel ce iubește bogăția nu va avea parte de rodul ei. Și aceasta este deșertăciune!
\par 11 Când se înmulțesc averile, sporesc și cei ce le mănâncă și ce folos are stăpânul lor că numai le vede?
\par 12 Dulce este somnul lucrătorului, fie că mănâncă mult, fie că mănâncă puțin, dar belșugul bogatului nu-i dă răgaz să doarmă.
\par 13 Este un rău cumplit pe care l-am văzut sub soare: bogății puse la o parte de stăpânul lor pentru a lui nenorocire.
\par 14 Și dacă bogăția se pierde dintr-o întâmplare nenorocită și el are un copil, acestuia nu-i rămâne nimic în mină.
\par 15 Precum a ieșit din pântecele maicii sale, gol se va duce, așa cum a venit, și pentru munca lui el nu va primi nimic, ca să poată lua în mâna lui.
\par 16 Și aceasta este o întâmplare nenorocită, ca să se ducă așa cum a venit; și ce folos că i-a fost munca în vânt?
\par 17 Mai mult încă, toată viața lui este întuneric și supărare, necaz peste fire și boală și durere!
\par 18 Cu adevărat iată ceea ce am văzut că este bine și frumos: să mănânce și să bea și să trăiască omul bine din tot lucrul cu care se trudește sub soare în vremea vieții dăruite lui de Dumnezeu, căci aceasta este partea lui.
\par 19 Și ori de câte ori Dumnezeu dă omului bogății și bunuri și îi îngăduie să mănânce și să-și ia partea lui și să se bucure de munca lui, acesta este un dar de la Dumnezeu;
\par 20 Căci el nu se gândește prea mult la zilele vieții lui, fiindcă Dumnezeu îl sine prins cu bucuria inimii lui.

\chapter{6}

\par 1 Este un rău pe care l-am văzut sub soare și care apasă greu asupra omului;
\par 2 Omului căruia Dumnezeu i-a dat averi și bunuri, iar sufletului lui nu-i lipsește nimic din ceea ce ar putea să dorească, Dumnezeu nu-i îngăduie însă să se bucure de ele, ci un străin le va mânca. Iată o deșertăciune și un rău nespus de mare!
\par 3 Dacă un om ar fi să aibă o sută de fii și să trăiască mulți ani și numeroase să fie zilele anilor săi, dacă nu s-a săturat sufletul lui de bine și el nu are loc de îngropare zic: "Chiar și fătul lepădat e mai fericit decât el!"
\par 4 A venit în zadar și se duce în întuneric și în întuneric numele lui va fi învăluit;
\par 5 Nici n-a văzut, nici n-a cunoscut soarele, și fătul lepădat a avut mai multă odihnă decât omul acesta.
\par 6 Și dacă ar fi trăit de două ori câte o mie de ani și nu s-a bucurat de fericire, oare nu toți se duc în același loc?
\par 7 Toată munca omului este pentru gura lui și cu toate acestea pofta lui nu e astâmpărată.
\par 8 Căci ce are înțeleptul mai mult decât nebunul? Ce folos are săracul care știe să se poarte înaintea celor vii?
\par 9 Mai bine să te uiți cu ochii decât să pribegești cu dorința. Și aceasta este deșertăciune și vinare de vânt!
\par 10 La tot ce își ia ființă i s-a hotărât numele de mai înainte; se știe ce va fi omul; el nu poate să intre în pricină cu cel ce este mai tare decât el.
\par 11 Cu cit se spun mai multe cuvinte, cu atât este mai multă deșertăciune. Ce folos trage omul?
\par 12 Căci cine știe ce este de folos pentru om în viață, în vremea zilelor sale de nimicnicie pe care le trece asemenea unei umbre? Și cine va spune mai dinainte omului ce va fi după el sub soare?

\chapter{7}

\par 1 Mai de preț este un nume bun decât untdelemnul cel binemirositor și ziua morții decât ziua nașterii.
\par 2 Mai bine este să mergi în casă de plâns, decât să te duci în casă de ospăț, căci acolo se vede sfârșitul omului și cine îl vede pune la inimă.
\par 3 Mai bun este necazul decât râsul, căci întristarea feței este bună pentru inimă.
\par 4 Inima celor înțelepți este în casa cea cu tânguire, iar inima celor nebuni în casa veseliei.
\par 5 Mai degrabă să auzi certarea unui înțelept, decât să asculți cântecul celor nebuni;
\par 6 Că precum este pârâitul spinilor sub căldare, tot așa este și râsul celui nebun. Și aceasta este deșertăciune!...
\par 7 Căci asuprirea poate să facă nebun pe un înțelept, și mita strică inima.
\par 8 Mai bun este sfârșitul unui lucru decât începutul lui; mai de preț este un duh răbdător decât un duh semeț.
\par 9 Nu te grăbi să te întărâți întru duhul tău, pentru că mânia  sălășluiește în sânul celor nebuni.
\par 10 Nu spune niciodată: "Cum se face că zilele cele de altădată au fost mai bune decât acestea?" Căci nu din înțelepciune întrebi una ca aceasta.
\par 11 Înțelepciunea este tot atât ție bună ca și o moștenire de folos celor ce văd soarele,
\par 12 Că așa este ocrotirea înțelepciunii și a banului; dar folosul științei este că înțelepciunea ține cu viață pe stăpânul ei.
\par 13 Socotește cu mintea faptele lui Dumnezeu; cine poate să îndrepte ceea ce El a strâmbat?
\par 14 În zi de fericire fii bucuros, iar în zi de nenorocire gândește-te că Dumnezeu a făcut și pe una și pe cealaltă, așa ca omul să nu descopere nimic din cele viitoare.
\par 15 Aceste două lucruri le-am văzut eu în zilele nimicniciei mele: este câte un drept, care piere întru dreptatea lui, și este câte un nelegiuit care trăiește mereu în răutatea lui.
\par 16 Nu fii drept peste măsură și nu te arăta prea înțelept! Pentru ce vrei să te nimicești?
\par 17 Nu fii nelegiuit până la sfârșit și nu fii nici nebun; de ce să mori înainte de timpul tău?
\par 18 Este bine să te ții de una și de cealaltă să nu te desfaci, că cine se teme de Dumnezeu scapă din toate.
\par 19 Înțelepciunea dă înțeleptului mai multă putere decât au zece nerozi într-o cetate.
\par 20 Căci nu este om drept pe pământ care să facă binele și să nu păcătuiască.
\par 21 Nu lua aminte la toate vorbele pe care cineva le spune, ca nu cumva să auzi că sluga ta te grăiește de rău;
\par 22 Căci inima ta singură știe de câte ori și tu ai defăimat pe alții.
\par 23 Toate acestea le-am încercat prin înțelepciune și am zis: "Vreau să fiu înțelept!" dar înțelepciunea a rămas departe de mine.
\par 24 Ceea ce a fost este departe, și adânc, adânc! Cine poate acum să-i dea de înțeles?
\par 25 Și eu m-am silit și inima mea a cercetat și a urmărit știința și înțelepciunea și buna chibzuială și mi-am dat seama că răutatea este o nebunie, iar prostia este zănatică răutate.
\par 26 Și am găsit femeia mai amară decât moartea, pentru că ea este o cursă, inima ei este un laț și mâinile ei sânt cătușe. Cel ce este bun înaintea lui Dumnezeu scapă, iar păcătosul este prins.
\par 27 Iată ce am aflat, zice Ecclesiastul, și încă și altele ca să pot descoperi înțelesul,
\par 28 Pe care sufletul meu l-a căutat și nu l-a aflat. Am găsit însă un om la o mie, dar n-am găsit nici o femeie din toate câte sânt.
\par 29 Dar iată numai ce am găsit: Dumnezeu a făcut pe om drept, iar oamenii născocesc multe vicleșuguri.

\chapter{8}

\par 1 Cine este ca înțeleptul și cine poate să știe ca el tâlcuirea lucrurilor? Înțelepciunea unui om îi luminează fața și asprimea feței lui i se schimbă.
\par 2 Ascultă de porunca regelui din pricina jurământului făcut lui Dumnezeu.
\par 3 Nu te grăbi să te depărtezi de fața lui. Nu stărui în lucrul cel rău, că ceea ce voiește face.
\par 4 Cuvântul regelui este hotărâtor și cine poate să-i spună: "Ce faci?"
\par 5 Celui ce ascultă porunca nu i se va întâmpla nimic rău, că inima unui om înțelept va cunoaște timpul și judecata.
\par 6 Pentru orice lucru este un timp și o judecată; că mare este nenorocirea care apasă asupra omului,
\par 7 De vreme ce el nu poate să știe mai dinainte ceea ce se va întâmpla. Oare, cine îi va da de știre ceea ce va fi cu el?
\par 8 Omul nu este stăpân pe duhul său de viață, ca să-l poată opri; la fel nu este stăpân pe ziua morții și în această luptă nu încape amânare. Nelegiuirea nu va scăpa pe cel care o săvârșește.
\par 9 Am văzut toate aceste lucruri și mi-am sârguit inima mea spre tot lucrul care se face sub soare într-o vreme când omul stăpânește pe altul spre nenorocirea lui.
\par 10 Și am văzut păcătoși în mare cinste purtați la locul de odihnă, pe când cei ce lucraseră drept au fost izgoniți de la locul cel sfânt și au fost uitați în cetate. Și aceasta este deșertăciune!
\par 11 Din pricină că hotărârea pentru pedepsirea răutății nu este îndeplinită de îndată, pentru aceasta se umple de răutate inima oamenilor ca să facă rău.
\par 12 Dar deși păcătosul face rău de o sută de ori și își lungește zilele, eu știu că fericirea este a acelora care se tem de Dumnezeu și se sfiesc în fața Lui.
\par 13 Fericirea nu va fi pentru cel fără de lege, care, asemenea umbrei, nu-și va lungi viața, fiindcă el nu se teme de Domnul.
\par 14 Este încă o nepotrivire care se petrece pe pământ, adică: sânt drepți cărora li se răsplătește ca după faptele celor nelegiuiți și sânt păcătoși cărora li se răsplătește ca după faptele celor drepți. Am zis că și aceasta este încă o deșertăciune!
\par 15 Și am ridicat în slăvi veselia, căci nu este nimic mai bun pentru om sub soare, decât să mănânce, să bea și să se veselească. Numai de l-ar întovărăși la lucrul lui în tot timpul vieții pe care Dumnezeu i-o dă sub soare!
\par 16 Când mi-am îndreptat inima ca să cunosc înțelepciunea și să pătrund care este menirea omului pe pământ, căci nici zi, nici noapte ochii lui nu văd somnul,
\par 17 Atunci mi-am dat seama, privind lucrarea lui Dumnezeu, că omul nu poate să înțeleagă toate câte se fac sub soare, dar se ostenește căutându-le, fără să le dea de rost; iar dacă înțeleptul crede că le cunoaște, el nu poate să le pătrundă.

\chapter{9}

\par 1 Cu adevărat toate acestea le-am pus la inimă și inima mea le-a văzut: că cei drepți și cei înțelepți împreună cu toate faptele lor sânt în mina lui Dumnezeu. Nici măcar iubirea, nici ura nu o cunoaște omul; ci totul este deșertăciune înaintea oamenilor,
\par 2 Căci toți au aceeași soartă: cel drept ca și cel păcătos, cel bun ca și cel rău, cel curat ea și cel necurat, cel ce aduce jertfă ca și cel ce nu aduce, cel bun ca și cel rău, cel ce jură ca și cel ce cinstește jurământul.
\par 3 Este un mare rău în tot ceea ce se face sub soare, căci toți au aceeași soartă și pe lângă aceasta inima oamenilor este plină de răutate și nebunia în inima lor dăinuiește toată viața lor și se duc în același loc cu cei morți;
\par 4 Oare cine va rămâne în viață? Pentru toți cei vii este o nădejde, căci un câine viu este mai de preț decât un câine mort.
\par 5 Cei vii știu că vor muri, dar cei morți nu știu nimic și parte de răsplată nu mai au, căci numele lor a fost uitat.
\par 6 Și dragostea lor, ura lor și pizma lor a pierit de mult și nu se vor mai bucura niciodată de ceea ce se face sub soare.
\par 7 Du-te și mănâncă cu bucurie pâinea ta și bea cu inimă bună vinul tău, pentru că Dumnezeu este îndurător pentru faptele tale.
\par 8 Toată vremea veșmintele tale să fie albe și untdelemnul să nu lipsească de pe capul tău!
\par 9 Bucură-te de viață cu femeia pe care o iubești în toate zilele vieții tale celei deșarte, pe care ți-a hărăzit-o Dumnezeu sub soare; căci aceasta este partea ta în viață și în mijlocul trudei cu care te ostenești sub soare.
\par 10 Tot ceea ce mâna ta prinde să săvârșească, fă cu hotărâre, căci în locuința morților în care te vei duce nu se află nici faptă, nici punere la cale, nici știință, nici înțelepciune.
\par 11 Și iarăși am văzut sub soare că izbânda în alergare nu este a celor iuți și biruința a celor viteji, și pâinea a celor înțelepți, nici bogăția a celor pricepuți, nici faima pentru cei învățați, căci timpul și întâmplarea întâmpină pe toți.
\par 12 Că omul nu știe nici măcar vremea lui: întocmai ea și peștii care sânt prinși în vicleanul năvod, întocmai ca și păsările în laț, așa sânt prinși fără de veste oamenii în vremea de restriște, când vine dintr-odată peste ei.
\par 13 Și am mai văzut sub soare acest fapt de înțelepciune care mi s-a părut într-adevăr mare:
\par 14 A fost odată o cetate mică, locuită de oameni puțini și împotriva ei a pornit un rege vestit, care a împresurat-o și a ridicat de jur împrejur întărituri puternice.
\par 15 Într-însa se afla un sărac înțelept care a scăpat cetatea prin înțelepciunea lui și nimeni nu-și mai aduce aminte de acest om sărman.
\par 16 Și am zis: "Mai bună este înțelepciunea decât puterea; dar înțelepciunea celui sărac este urgisită, și cuvintele lui nu sânt luate în seamă".
\par 17 Vorbele celui înțelept spuse domol sunt mai ascultate decât strigătul unui stăpân între nebuni.
\par 18 Înțelepciunea este mai de preț decât armele de luptă; dar o singură greșeală nimicește mult bine.

\chapter{10}

\par 1 O muscă moartă strică amestecul de untdelemn al celui ce pregătește miresme; puțină nebunie strică prețul la multă înțelepciune.
\par 2 Inima celui înțelept este la dreapta lui, iar inima celui nebun la stânga.
\par 3 La fel celui nebun când merge pe drum îi lipsește dreapta pricepere, iar toată lumea zice: "E nebun!"
\par 4 Dacă mânia stăpânitorului se ridică împotriva ta, nu te clinti din locul tău. Căci firea domoală înlătură mari neajunsuri.
\par 5 Am mai luat seama la încă un rău de sub soare, ca o greșeală care pornește de la stăpânitor:
\par 6 Nebunul este ridicat la dregătorii înalte, iar cei vrednici stau în locuri de jos.
\par 7 Am văzut robi călări pe cai și căpetenii mergând ca robii pe jos.
\par 8 Cel ce sapă o groapă poate să cadă în ea și cel ce dărâmă un zid poate fi mușcat de șarpe.
\par 9 Cel ce sfărâmă pietre se poate răni cu ele, iar cel ce despică lemne se primejduiește.
\par 10 Dacă o unealtă s-a tocit și nu a fost ascuțită, trebuie să îndoim puterile, dar înțelepciunea are ca parte izbânda.
\par 11 Dacă șarpele mușcă înainte de a fi descântat, atunci descântătorul nu are nici un folos.
\par 12 Graiurile gurii celui înțelept sânt har, iar buzele celui nebun îl doboară.
\par 13 Începutul cuvintelor gurii lui este prostia, iar sfârșitul graiului lui nebunie curată.
\par 14 Nebunul sporește vorbele. Omul nu știe ce va fi după el, căci cine îi va spune ce se va întâmpla după el?
\par 15 Munca obosește pe cel nebun; cine nu știe drumul nu poate să se ducă în cetate.
\par 16 Vai de tine, țară, care ai un copil rege și căpeteniile tale mănâncă dis-de-dimineață!
\par 17 Fericită ești tu, țară, care ai rege un fecior de neam mare și căpeteniile tale mănâncă la vreme, ca toți oamenii, și nu se dau la băutură.
\par 18 Din pricina lenei, grinzile casei se lasă, iar când stai cu mâinile în sin, apa picură în casă.
\par 19 Ospețele se fac pentru a gusta plăcerea; vinul înveselește viața și banii răspund la toate.
\par 20 Chiar în gândul tău nu blestema pe rege și în cămara unde dormi nu defăima pe cel puternic; căci păsările cerului pot să ducă un cuvânt și neamul celor înaripate să dea vorba ta pe față.

\chapter{11}

\par 1 Aruncă pâinea ta pe apă, căci o vei afla după multe zile.
\par 2 Împarte o bucată în șapte și chiar în opt, căci nu știi ce nenorocire se poate întâmpla în țară.
\par 3 Când norii se umplu de ploaie, ei se deșartă pe pământ. Și dacă un capac cade la miazăzi sau la miazănoapte, unde a căzut, acolo rămâne.
\par 4 Cel ce păzește vântul nu seamănă, și cel ce se uită la nori nu seceră.
\par 5 După cum nu știi care este calea vântului, cum se întocmesc oasele în pântecele maicii, tot așa nu cunoști lucrarea lui Dumnezeu, care face toate.
\par 6 Dis-de-dimineață seamănă sămânța și până seara nu odihni mâna ta, căci nu știi care va izbuti, aceasta sau aceea, sau dacă amândouă sânt deopotrivă de bune.
\par 7 Și lumina este dulce și plăcut este ochilor să privească soarele.
\par 8 Chiar dacă ar trăi mulți ani, omul să se bucure de toate și să-și aducă aminte de zilele cele din întuneric, căci multe vor fi. Tot ce se întâmplă este deșertăciune.
\par 9 Bucură-te, omule, cât ești tânăr și inima ta să fie veselă în zilele tinereții tale și mergi în căile inimii tale și după ce-ți arată ochii tăi, dar să știi că, pentru toate acestea, Dumnezeu te va aduce la judecata Sa.
\par 10 Alungă necazul din inima ta și depărtează suferințele de trupul tău, căci copilăria și tinerețea sânt deșertăciune.

\chapter{12}

\par 1 Adu-ți aminte de Ziditorul tău în zilele tinereții tale, înainte ca să vină zilele de restriște și să se apropie anii despre care vei zice: "N-am nici o plăcere de ei ..."
\par 2 Înainte ca să se întunece soarele și lumina și luna și stelele și ca norii să mai vină după ploaie;
\par 3 Atunci este vremea când străjerii casei tremură și oamenii cei tari se încovoaie la pământ și cele ce macină nu mai lucrează, căci sânt puține la număr și privitoarele de la ferestre se întunecă;
\par 4 Și se închid porțile care dau spre uliță și se domolește huruitul morii și te scoli la ciripitul de dimineață al păsării și se potolesc toate cântărețele;
\par 5 Și te temi să mai urci colina și spaimele pândesc în cale și capul se face alb ca floarea de migdal și lăcusta sprintenă se face grea și toți mugurii s-au deschis, fiindcă omul merge la locașul său de veci și bocitoarele dau târcoale pe uliță;
\par 6 Mai înainte ca să se rupă funia de argint și să se spargă vasul de aur și să se strice ulciorul la izvor și să se sfărâme roata fântânii,
\par 7 Și ca pulberea să se întoarcă în pământ cum a fost, iar sufletul să se întoarcă la Dumnezeu, Care l-a dat.
\par 8 Deșertăciunea deșertăciunilor, a zis Ecclesiastul, toate sânt deșertăciuni!
\par 9 Și pe lângă că Ecclesiastul a fost un înțelept, el a mai învățat pe popor cunoștința și a cercetat și a privit cu luare aminte și a cules multe pilde.
\par 10 Ecclesiastul s-a străduit să găsească sfaturi folositoare și îndrumări adevărate și să le scrie întocmai.
\par 11 Cuvintele celor înțelepți sânt ca boldurile de îmboldit dobitoacele și ca niște cuie înfipte și ascuțite și sânt date de un Păstor.
\par 12 Și peste toate acestea, fiul meu, să fii cu luare aminte: scrisul de cărți este fără sfârșit, iar învățătura multă este oboseală pentru trup.
\par 13 Iată pe scurt; tot ceea ce ai auzit aceasta este: Teme-te de Dumnezeu și păzește poruncile Lui! Acesta este lucru cuvenit fiecărui om.
\par 14 Căci Dumnezeu va judeca toate faptele ascunse, fie bune, fie rele.


\end{document}