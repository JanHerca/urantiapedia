\begin{document}

\title{Ioan}


\chapter{1}

\par 1 La început era Cuvântul și Cuvântul era la Dumnezeu și Dumnezeu era Cuvântul.
\par 2 Acesta era întru început la Dumnezeu.
\par 3 Toate prin El s-au făcut; și fără El nimic nu s-a făcut din ce s-a făcut.
\par 4 Întru El era viață și viața era lumina oamenilor.
\par 5 Și lumina luminează în întuneric și întunericul nu a cuprins-o.
\par 6 Fost-a om trimis de la Dumnezeu, numele lui era Ioan.
\par 7 Acesta a venit spre mărturie, ca să mărturisească despre Lumină, ca toți să creadă prin el.
\par 8 Nu era el Lumina ci ca să mărturisească despre Lumină.
\par 9 Cuvântul era Lumina cea adevărată care luminează pe tot omul, care vine în lume.
\par 10 În lume era și lumea prin El s-a făcut, dar lumea nu L-a cunoscut.
\par 11 Întru ale Sale a venit, dar ai Săi nu L-au primit.
\par 12 Și celor câți L-au primit, care cred în numele Lui, le-a dat putere ca să se facă fii ai lui Dumnezeu,
\par 13 Care nu din sânge, nici din poftă trupească, nici din poftă bărbătească, ci de la Dumnezeu s-au născut.
\par 14 Și Cuvântul S-a făcut trup și S-a sălășluit între noi și am văzut slava Lui, slavă ca a Unuia-Născut din Tatăl, plin de har și de adevăr.
\par 15 Ioan mărturisea despre El și striga, zicând: Acesta era despre Care am zis: Cel care vine după mine a fost înaintea mea, pentru că mai înainte de mine era.
\par 16 Și din plinătatea Lui noi toți am luat, și har peste har.
\par 17 Pentru că Legea prin Moise s-a dat, iar harul și adevărul au venit prin Iisus Hristos.
\par 18 Pe Dumnezeu nimeni nu L-a văzut vreodată; Fiul cel Unul-Născut, Care este în sânul Tatălui, Acela L-a făcut cunoscut.
\par 19 Și aceasta este mărturia lui Ioan, când au trimis la El iudeii din Ierusalim, preoți și leviți, ca să-l întrebe: Cine ești tu?
\par 20 Și el a mărturisit și n-a tăgăduit; și a mărturisit: Nu sunt eu Hristosul.
\par 21 Și ei l-au întrebat: Dar cine ești? Ești Ilie? Zis-a el: Nu sunt. Ești tu Proorocul? Și a răspuns: Nu.
\par 22 Deci i-au zis: Cine ești? Ca să dăm un răspuns celor ce ne-au trimis. Ce spui tu despre tine însuți?
\par 23 El a zis: Eu sunt glasul celui ce strigă în pustie: "Îndreptați calea Domnului", precum a zis Isaia proorocul.
\par 24 Și trimișii erau dintre farisei.
\par 25 Și l-au întrebat și i-au zis: De ce botezi deci, dacă tu nu ești Hristosul, nici Ilie, nici Proorocul?
\par 26 Ioan le-a răspuns, zicând: Eu botez cu apă; dar în mijlocul vostru Se află Acela pe Care voi nu-L știți,
\par 27 Cel care vine după mine, Care înainte de mine a fost și Căruia eu nu sunt vrednic să-I dezleg cureaua încălțămintei.
\par 28 Acestea se petreceau în Betabara, dincolo de Iordan, unde boteza Ioan.
\par 29 A doua zi a văzut Ioan pe Iisus venind către el și a zis: Iată Mielul lui Dumnezeu, Cel ce ridică păcatul lumii.
\par 30 Acesta este despre Care eu am zis: După mine vine un bărbat, Care a fost înainte de mine, fiindcă mai înainte de mine era,
\par 31 Și eu nu-L știam; dar ca să fie arătat lui Israel, de aceea am venit eu, botezând cu apă.
\par 32 Și a mărturisit Ioan zicând: Am văzut Duhul coborându-Se, din cer, ca un porumbel și a rămas peste El.
\par 33 Și eu nu-L cunoșteam pe El, dar Cel ce m-a trimis să botez cu apă, Acela mi-a zis: Peste Care vei vedea Duhul coborându-Se și rămânând peste El, Acela este Cel ce botează cu Duh Sfânt.
\par 34 Și eu am văzut și am mărturisit că Acesta este Fiul lui Dumnezeu.
\par 35 A doua zi iarăși stătea Ioan și doi dintre ucenicii lui.
\par 36 Și privind pe Iisus, Care trecea, a zis: Iată Mielul lui Dumnezeu!
\par 37 Și cei doi ucenici l-au auzit când a spus aceasta și au mers după Iisus.
\par 38 Iar Iisus, întorcându-Se și văzându-i că merg după El, le-a zis: Ce căutați? Iar ei I-au zis: Rabi (care se tâlcuiește: Învățătorule), unde locuiești?
\par 39 El le-a zis: Veniți și veți vedea. Au mers deci și au văzut unde locuia; și au rămas la El în ziua aceea. Era ca la ceasul al zecelea.
\par 40 Unul dintre cei doi care auziseră de la Ioan și veniseră după Iisus era Andrei, fratele lui Simon Petru.
\par 41 Acesta a găsit întâi pe Simon, fratele său, și i-a zis: am găsit pe Mesia (care se tâlcuiește: Hristos).
\par 42 Și l-a adus la Iisus. Iisus, privind la el, i-a zis: Tu ești Simon, fiul lui Iona; tu te vei numi Chifa (ce se tâlcuiește: Petru).
\par 43 A doua zi voia să plece în Galileea și a găsit pe Filip. Și i-a zis Iisus: Urmează-Mi.
\par 44 Iar Filip era din Betsaida, din cetatea lui Andrei și a lui Petru.
\par 45 Filip a găsit pe Natanael și i-a zis: Am aflat pe Acela despre Care au scris Moise în Lege și proorocii, pe Iisus, fiul lui Iosif din Nazaret.
\par 46 Și i-a zis Natanael: Din Nazaret poate fi ceva bun? Filip i-a zis: Vino și vezi.
\par 47 Iisus a văzut pe Natanael venind către El și a zis despre el: Iată, cu adevărat, israelit în care nu este vicleșug.
\par 48 Natanael I-a zis: De unde mă cunoști? A răspuns Iisus și i-a zis: Mai înainte de a te chema Filip, te-am văzut când erai sub smochin.
\par 49 Răspunsu-I-a Natanael: Rabi, Tu ești Fiul lui Dumnezeu, Tu ești regele lui Israel.
\par 50 Răspuns-a Iisus și i-a zis: Pentru că ți-am spus că te-am văzut sub smochin, crezi? Mai mari decât acestea vei vedea.
\par 51 Și i-a zis: Adevărat, adevărat zic vouă, de acum veți vedea cerul deschizându-se și pe îngerii lui Dumnezeu suindu-se și coborându-se peste Fiul Omului.

\chapter{2}

\par 1 Și a treia zi s-a făcut nuntă în Cana Galileii și era și mama lui Iisus acolo.
\par 2 Și a fost chemat și Iisus și ucenicii Săi la nuntă.
\par 3 Și sfârșindu-se vinul, a zis mama lui Iisus către El: Nu mai au vin.
\par 4 A zis ei Iisus: Ce ne privește pe mine și pe tine, femeie? Încă n-a venit ceasul Meu.
\par 5 Mama Lui a zis celor ce slujeau: Faceți orice vă va spune.
\par 6 Și erau acolo șase vase de piatră, puse pentru curățirea iudeilor, care luau câte două sau trei vedre.
\par 7 Zis-a lor Iisus: Umpleți vasele cu apă. Și le-au umplut până sus.
\par 8 Și le-a zis: Scoateți acum și aduceți nunului. Iar ei i-au dus.
\par 9 Și când nunul a gustat apa care se făcuse vin și nu știa de unde este, ci numai slujitorii care scoseseră apa știau, a chemat nunul pe mire,
\par 10 Și i-a zis: Orice om pune întâi vinul cel bun și, când se amețesc, pune pe cel mai slab. Dar tu ai ținut vinul cel bun până acum.
\par 11 Acest început al minunilor l-a făcut Iisus în Cana Galileii și Și-a arătat slava Sa; și ucenicii Săi au crezut în El.
\par 12 După aceasta S-a coborât în Capernaum, El și mama Sa și frații și ucenicii Săi, și acolo n-a rămas decât puține zile.
\par 13 Și erau aproape Paștile iudeilor, și Iisus S-a urcat la Ierusalim.
\par 14 Și a găsit șezând în templu pe cei ce vindeau boi și oi și porumbei și pe schimbătorii de bani.
\par 15 Și, făcându-Și un bici din ștreanguri, i-a scos pe toți afară din templu, și oile și boii, și schimbătorilor le-a vărsat banii și le-a răsturnat mesele.
\par 16 Și celor ce vindeau porumbei le-a zis: Luați acestea de aici. Nu faceți casa Tatălui Meu casă de negustorie.
\par 17 Și și-au adus aminte ucenicii Lui că este scris: "Râvna casei Tale mă mistuie".
\par 18 Au răspuns deci iudeii și I-au zis: Ce semn ne arăți că faci acestea?
\par 19 Iisus a răspuns și le-a zis: Dărâmați templul acesta și în trei zile îl voi ridica.
\par 20 Și au zis deci iudeii: În patruzeci și șase de ani s-a zidit templul acesta! Și Tu îl vei ridica în trei zile?
\par 21 Iar El vorbea despre templul trupului Său.
\par 22 Deci, când S-a sculat din morți, ucenicii Lui și-au adus aminte că aceasta o spusese și au crezut Scripturii și cuvântului pe care Îl spusese Iisus.
\par 23 Și când era în Ierusalim, la sărbătoarea Paștilor, mulți au crezut în numele Lui, văzând minunile pe care le făcea.
\par 24 Iar Iisus însuși nu Se încredea în ei, pentru că îi cunoștea pe toți.
\par 25 Și pentru că nu avea nevoie să-I mărturisească cineva despre om, căci El însuși cunoștea ce era în om.

\chapter{3}

\par 1 Și era un om dintre farisei, care se numea Nicodim și care era fruntaș al iudeilor.
\par 2 Acesta a venit noaptea la Iisus și I-a zis: Rabi, știm că de la Dumnezeu ai venit învățător; că nimeni nu poate face aceste minuni, pe care le faci Tu, dacă nu este Dumnezeu cu el.
\par 3 Răspuns-a Iisus și i-a zis: Adevărat, adevărat zic ție: De nu se va naște cineva de sus, nu va putea să vadă împărăția lui Dumnezeu.
\par 4 Iar Nicodim a zis către El: Cum poate omul să se nască, fiind bătrân? Oare, poate să intre a doua oară în pântecele mamei sale și să se nască?
\par 5 Iisus a răspuns: Adevărat, adevărat zic ție: De nu se va naște cineva din apă și din Duh, nu va putea să intre în împărăția lui Dumnezeu.
\par 6 Ce este născut din trup, trup este; și ce este născut din Duh, duh este.
\par 7 Nu te mira că ți-am zis: Trebuie să vă nașteți de sus.
\par 8 Vântul suflă unde voiește și tu auzi glasul lui, dar nu știi de unde vine, nici încotro se duce. Astfel este cu oricine e născut din Duhul.
\par 9 A răspuns Nicodim și i-a zis: Cum pot să fie acestea?
\par 10 Iisus a răspuns și i-a zis: Tu ești învățătorul lui Israel și nu cunoști acestea?
\par 11 Adevărat, adevărat zic ție, că noi ceea ce știm vorbim și ce am văzut mărturisim, dar mărturia noastră nu o primiți.
\par 12 Dacă v-am spus cele pământești și nu credeți, cum veți crede cele cerești?
\par 13 Și nimeni nu s-a suit în cer, decât Cel ce S-a coborât din cer, Fiul Omului, Care este în cer.
\par 14 Și după cum Moise a înălțat șarpele în pustie, așa trebuie să se înalțe Fiul Omului,
\par 15 Ca tot cel ce crede în El să nu piară, ci să aibă viață veșnică.
\par 16 Căci Dumnezeu așa a iubit lumea, încât pe Fiul Său Cel Unul-Născut L-a dat ca oricine crede în El să nu piară, ci să aibă viață veșnică.
\par 17 Căci n-a trimis Dumnezeu pe Fiul Său în lume ca să judece lumea, ci ca să se mântuiască, prin El, lumea.
\par 18 Cel ce crede în El nu este judecat, iar cel ce nu crede a și fost judecat, fiindcă nu a crezut în numele Celui Unuia-Născut, Fiul lui Dumnezeu.
\par 19 Iar aceasta este judecata, că Lumina a venit în lume și oamenii au iubit întunericul mai mult decât Lumina. Căci faptele lor erau rele.
\par 20 Că oricine face rele urăște Lumina și nu vine la Lumină, pentru ca faptele lui să nu se vădească.
\par 21 Dar cel care lucrează adevărul vine la Lumină, ca să se arate faptele lui, că în Dumnezeu sunt săvârșite.
\par 22 După acestea a venit Iisus și ucenicii Lui în pământul Iudeii și stătea acolo și boteza.
\par 23 Și boteza și Ioan în Enom, aproape de Salim, că erau acolo ape multe și veneau și se botezau.
\par 24 Căci Ioan nu fusese încă aruncat în închisoare.
\par 25 Și s-a iscat o neînțelegere între ucenicii lui Ioan și un iudeu, asupra curățirii.
\par 26 Și au venit la Ioan și i-au zis: Rabi, Acela care era cu tine, dincolo de Iordan, și despre Care tu ai mărturisit, iată El botează și toți se duc la El.
\par 27 Ioan a răspuns și a zis: Nu poate un om să ia nimic, dacă nu i s-a dat lui din cer.
\par 28 Voi înșivă îmi sunteți martori că am zis: Nu sunt eu Hristosul, ci sunt trimis înaintea Lui.
\par 29 Cel ce are mireasă este mire, iar prietenul mirelui, care stă și ascultă pe mire, se bucură cu bucurie de glasul lui. Deci această bucurie a mea s-a împlinit.
\par 30 Acela trebuie să crească, iar eu să mă micșorez.
\par 31 Cel ce vine de sus este deasupra tuturor; cel ce este de pe pământ pământesc este și de pe pământ grăiește. Cel ce vine din cer este deasupra tuturor.
\par 32 Și ce a văzut și a auzit, aceea mărturisește, dar mărturia Lui nu o primește nimeni.
\par 33 Cel ce a primit mărturia Lui a pecetluit că Dumnezeu este adevărat.
\par 34 Căci cel pe care l-a trimis Dumnezeu vorbește cuvintele lui Dumnezeu, pentru că Dumnezeu nu dă Duhul cu măsură.
\par 35 Tatăl iubește pe Fiul și toate le-a dat în mâna Lui.
\par 36 Cel ce crede în Fiul are viață veșnică, iar cel ce nu ascultă de Fiul nu va vedea viața, ci mânia lui Dumnezeu rămâne peste el.

\chapter{4}

\par 1 Deci când a cunoscut Iisus că fariseii au auzit că El face și botează mai mulți ucenici ca Ioan,
\par 2 Deși Iisus nu boteza El, ci ucenicii Lui,
\par 3 A lăsat Iudeea și S-a dus iarăși în Galileea.
\par 4 Și trebuia să treacă prin Samaria.
\par 5 Deci a venit la o cetate a Samariei, numită Sihar, aproape de locul pe care Iacov l-a dat lui Iosif, fiul său;
\par 6 Și era acolo fântâna lui Iacov. Iar Iisus, fiind ostenit de călătorie, S-a așezat lângă fântână și era ca la al șaselea ceas.
\par 7 Atunci a venit o femeie din Samaria să scoată apă. Iisus i-a zis: Dă-Mi să beau.
\par 8 Căci ucenicii Lui se duseseră în cetate, ca să cumpere merinde.
\par 9 Femeia samarineancă I-a zis: Cum Tu, care ești iudeu, ceri să bei de la mine, care sunt femeie samarineancă? Pentru că iudeii nu au amestec cu samarinenii.
\par 10 Iisus a răspuns și i-a zis: Dacă ai fi știut darul lui Dumnezeu și Cine este Cel ce-ți zice: Dă-Mi să beau, tu ai fi cerut de la El, și ți-ar fi dat apă vie.
\par 11 Femeia I-a zis: Doamne, nici găleată nu ai, și fântâna e adâncă; de unde, dar, ai apa cea vie?
\par 12 Nu cumva ești Tu mai mare decât părintele nostru Iacov, care ne-a dat această fântână și el însuși a băut din ea și fiii lui și turmele lui?
\par 13 Iisus a răspuns și i-a zis: Oricine bea din apa aceasta va înseta iarăși;
\par 14 Dar cel ce va bea din apa pe care i-o voi da Eu nu va mai înseta în veac, căci apa pe care i-o voi da Eu se va face în el izvor de apă curgătoare spre viață veșnică.
\par 15 Femeia a zis către El: Doamne, dă-mi această apă ca să nu mai însetez, nici să mai vin aici să scot.
\par 16 Iisus i-a zis: Mergi și cheamă pe bărbatul tău și vino aici.
\par 17 Femeia a răspuns și a zis: N-am bărbat. Iisus i-a zis: Bine ai zis că nu ai bărbat.
\par 18 Căci cinci bărbați ai avut și cel pe care îl ai acum nu-ți este bărbat. Aceasta adevărat ai spus.
\par 19 Femeia I-a zis: Doamne, văd că Tu ești prooroc.
\par 20 Părinții noștri s-au închinat pe acest munte, iar voi ziceți că în Ierusalim este locul unde trebuie să ne închinăm.
\par 21 Și Iisus i-a zis: Femeie, crede-Mă că vine ceasul când nici pe muntele acesta, nici în Ierusalim nu vă veți închina Tatălui.
\par 22 Voi vă închinați căruia nu știți; noi ne închinăm Căruia știm, pentru că mântuirea din iudei este.
\par 23 Dar vine ceasul și acum este, când adevărații închinători se vor închina Tatălui în duh și în adevăr, că și Tatăl astfel de închinători își dorește.
\par 24 Duh este Dumnezeu și cei ce I se închină trebuie să i se închine în duh și în adevăr.
\par 25 I-a zis femeia: Știm că va veni Mesia care se cheamă Hristos; când va veni, Acela ne va vesti nouă toate.
\par 26 Iisus i-a zis: Eu sunt, Cel ce vorbesc cu tine.
\par 27 Dar atunci au sosit ucenicii Lui. Și se mirau că vorbea cu o femeie. Însă nimeni n-a zis: Ce o întrebi, sau: Ce vorbești cu ea?
\par 28 Iar femeia și-a lăsat găleata și s-a dus în cetate și a zis oamenilor:
\par 29 Veniți de vedeți un om care mi-a spus toate câte am făcut. Nu cumva aceasta este Hristosul?
\par 30 Și au ieșit din cetate și veneau către El.
\par 31 Între timp, ucenicii Lui Îl rugau, zicând: Învățătorule, mănâncă.
\par 32 Iar El le-a zis: Eu am de mâncat o mâncare pe care voi nu o știți.
\par 33 Ziceau deci ucenicii între ei: Nu cumva I-a adus cineva să mănânce?
\par 34 Iisus le-a zis: Mâncarea Mea este să fac voia Celui ce M-a trimis pe Mine și să săvârșesc lucrul Lui.
\par 35 Nu ziceți voi că mai sunt patru luni și vine secerișul? Iată zic vouă: Ridicați ochii voștri și priviți holdele că sunt albe pentru seceriș.
\par 36 Iar cel ce seceră primește plată și adună roade spre viața veșnică, ca să se bucure împreună și cel ce seamănă și cel ce seceră.
\par 37 Căci în aceasta se adeverește cuvântul: Că unul este semănătorul și altul secerătorul.
\par 38 Eu v-am trimis să secerați ceea ce voi n-ați muncit; alții au muncit și voi ați intrat în munca lor.
\par 39 Și mulți samarineni din cetatea aceea au crezut în El, pentru cuvântul femeii care mărturisea: Mi-a spus toate câte am făcut.
\par 40 Deci, după ce au venit la El, samarinenii Îl rugau să rămână la ei. Și a rămas acolo două zile.
\par 41 Și cu mult mai mulți au crezut pentru cuvântul Lui,
\par 42 Iar femeii i-au zis: Credem nu numai pentru cuvântul tău, căci noi înșine am auzit și știm că Acesta este cu adevărat Hristosul, Mântuitorul lumii.
\par 43 Și după cele două zile, a plecat de acolo în Galileea.
\par 44 Căci Iisus însuși a mărturisit că un prooroc nu e cinstit în țara lui.
\par 45 Deci, când a venit în Galileea, L-au primit galileenii, cei ce văzuseră toate câte făcuse El în Ierusalim, la sărbătoare, căci și ei veniseră la sărbătoare.
\par 46 Deci iarăși a mers în Cana Galileii, unde prefăcuse apa în vin. Și era un slujitor regesc, al cărui fiu era bolnav în Capernaum.
\par 47 Acesta, auzind că Iisus a venit din Iudeea în Galileea, s-a dus la El și Îl ruga să Se coboare și să vindece pe fiul lui, că era gata să moară.
\par 48 Deci Iisus i-a zis: Dacă nu veți vedea semne și minuni, nu veți crede.
\par 49 Slujitorul regesc a zis către El: Doamne, coboară-Te înainte de a muri copilul meu.
\par 50 Iisus i-a zis: Mergi, copilul tău trăiește. Și omul a crezut cuvântului pe care i l-a spus Iisus și a plecat.
\par 51 Iar pe când cobora, slugile lui, l-au întâmpinat spunându-i că fiul lui trăiește.
\par 52 Și cerea, deci, să afle de la ele ceasul în care i-a fost mai bine. Deci i-au spus că ieri, în ceasul al șaptelea, l-au lăsat frigurile.
\par 53 Așadar tatăl a cunoscut că în ceasul acela a fost în care Iisus i-a zis: Fiul tău trăiește. Și a crezut el și toată casa lui.
\par 54 Aceasta este a doua minune pe care a făcut-o iarăși Iisus, venind din Iudeea în Galileea.

\chapter{5}

\par 1 După acestea era o sărbătoare a iudeilor și Iisus S-a suit la Ierusalim.
\par 2 Iar în Ierusalim, lângă Poarta Oilor, era o scăldătoare, care pe evreiește se numește Vitezda, având cinci pridvoare.
\par 3 În acestea zăceau mulțime de bolnavi, orbi, șchiopi, uscați, așteptând mișcarea apei.
\par 4 Căci un înger al Domnului se cobora la vreme în scăldătoare și tulbura apa și cine intra întâi, după tulburarea apei, se făcea sănătos, de orice boală era ținut.
\par 5 Și era acolo un om, care era bolnav de treizeci și opt de ani.
\par 6 Iisus, văzându-l pe acesta zăcând și știind că este așa încă de multă vreme, i-a zis: Voiești să te faci sănătos?
\par 7 Bolnavul I-a răspuns: Doamne, nu am om, care să mă arunce în scăldătoare, când se tulbură apa; că, până când vin eu, altul se coboară înaintea mea.
\par 8 Iisus i-a zis: Scoală-te, ia-ți patul tău și umblă.
\par 9 Și îndată omul s-a făcut sănătos, și-a luat patul și umbla. Dar în ziua aceea era sâmbătă.
\par 10 Deci ziceau iudeii către cel vindecat: Este zi de sâmbătă și nu-ți este îngăduit să-ți iei patul.
\par 11 El le-a răspuns: Cel ce m-a făcut sănătos, Acela mi-a zis: Ia-ți patul și umblă.
\par 12 Ei l-au întrebat: Cine este omul care ți-a zis: Ia-ți patul tău și umblă?
\par 13 Iar cel vindecat nu știa cine este, căci Iisus se dăduse la o parte din mulțimea care era în acel loc.
\par 14 După aceasta Iisus l-a aflat în templu și i-a zis: Iată că te-ai făcut sănătos. De acum să nu mai păcătuiești, ca să nu-ți fie ceva mai rău.
\par 15 Atunci omul a plecat și a spus iudeilor că Iisus este Cel ce l-a făcut sănătos.
\par 16 Pentru aceasta iudeii prigoneau pe Iisus și căutau să-L omoare, că făcea aceasta sâmbăta.
\par 17 Dar Iisus le-a răspuns: Tatăl Meu până acum lucrează; și Eu lucrez.
\par 18 Deci pentru aceasta căutau mai mult iudeii să-L omoare, nu numai pentru că dezlega sâmbăta, ci și pentru că zicea că Dumnezeu este Tatăl Său, făcându-Se pe Sine deopotrivă cu Dumnezeu.
\par 19 A răspuns Iisus și le-a zis: Adevărat, adevărat zic vouă: Fiul nu poate să facă nimic de la Sine, dacă nu va vedea pe Tatăl făcând; căci cele ce face Acela, acestea le face și Fiul întocmai.
\par 20 Că Tatăl iubește pe Fiul și-I arată toate câte face El și lucruri mai mari decât acestea va arăta Lui, ca voi să vă mirați.
\par 21 Căci, după cum Tatăl scoală pe cei morți și le dă viață, tot așa și Fiul dă viață celor ce voiește.
\par 22 Tatăl nu judecă pe nimeni, ci toată judecata a dat-o Fiului.
\par 23 Ca toți să cinstească pe Fiul cum cinstesc pe Tatăl. Cine nu cinstește pe Fiul nu cinstește pe Tatăl care L-a trimis.
\par 24 Adevărat, adevărat zic vouă: Cel ce ascultă cuvântul Meu și crede în Cel ce M-a trimis are viață veșnică și la judecată nu va veni, ci s-a mutat de la moarte la viață.
\par 25 Adevărat, adevărat zic vouă, că vine ceasul și acum este, când morții vor auzi glasul Fiului lui Dumnezeu și cei ce vor auzi vor învia.
\par 26 Căci precum Tatăl are viață în Sine, așa I-a dat și Fiului să aibă viață în Sine;
\par 27 Și I-a dat putere să facă judecată, pentru că este Fiul Omului.
\par 28 Nu vă mirați de aceasta; căci vine ceasul când toți cei din morminte vor auzi glasul Lui,
\par 29 Și vor ieși, cei ce au făcut cele bune spre învierea vieții și cei ce au făcut cele rele spre învierea osândirii.
\par 30 Eu nu pot să fac de la Mine nimic; precum aud, judec; dar judecata Mea este dreaptă, pentru că nu caut la voia Mea, ci voia Celui care M-a trimis.
\par 31 Dacă mărturisesc Eu despre mine însumi, mărturia Mea nu este adevărată.
\par 32 Altul mărturisește despre Mine; și știu că adevărată este mărturia pe care o mărturisește despre Mine.
\par 33 Voi ați trimis la Ioan, și el a mărturisit adevărul.
\par 34 Dar Eu nu de la om iau mărturia, ci spun aceasta ca să vă mântuiți.
\par 35 Acela (Ioan) era făclia care arde și luminează, și voi ați voit să vă veseliți o clipă în lumina lui.
\par 36 Iar Eu am mărturie mai mare decât a lui Ioan; căci lucrurile pe care Mi le-a dat Tatăl ca să le săvârșesc, lucrurile acestea pe care le fac Eu, mărturisesc despre Mine că Tatăl M-a trimis.
\par 37 Și Tatăl care M-a trimis, Acela a mărturisit despre Mine. Nici glasul Lui nu l-ați văzut vreodată, nici fața Lui nu ați văzut-o;
\par 38 Și cuvântul Lui nu sălășluiește în voi, pentru că voi nu credeți în Cel pe care l-a trimis Acela.
\par 39 Cercetați Scripturile, că socotiți că în ele aveți viață veșnică. Și acelea sunt care mărturisesc despre Mine.
\par 40 Și nu voiți să veniți la Mine, ca să aveți viață!
\par 41 Slavă de la oameni nu primesc;
\par 42 Dar v-am cunoscut că n-aveți în voi dragostea lui Dumnezeu.
\par 43 Eu am venit în numele Tatălui Meu și voi nu Mă primiți; dacă va veni altul în numele său, pe acela îl veți primi.
\par 44 Cum puteți voi să credeți, când primiți slavă unii de la alții și slava care vine de la unicul Dumnezeu nu o căutați?
\par 45 Să nu socotiți că Eu vă voi învinui la Tatăl; cel ce vă învinuiește este Moise, în care voi ați nădăjduit.
\par 46 Că dacă ați fi crezut lui Moise, ați fi crezut și Mie, căci despre Mine a scris acela.
\par 47 Iar dacă celor scrise de el nu credeți, cum veți crede în cuvintele Mele?

\chapter{6}

\par 1 După acestea, Iisus S-a dus dincolo de marea Galileii, în părțile Tiberiadei.
\par 2 Și a mers după El mulțime multă, pentru că vedeau minunile pe care le făcea cu cei bolnavi.
\par 3 Și S-a suit Iisus în munte și a șezut acolo cu ucenicii Săi.
\par 4 Și era aproape Paștile, sărbătoarea iudeilor.
\par 5 Deci ridicându-Și Iisus ochii și văzând că mulțime multă vine către El, a zis către Filip: De unde vom cumpăra pâine, ca să mănânce aceștia?
\par 6 Iar aceasta o zicea ca să-l încerce, că El știa ce avea să facă.
\par 7 Și Filip i-a răspuns: Pâini de două sute de dinari nu le vor ajunge, ca să ia fiecare câte puțin.
\par 8 Și a zis Lui unul din ucenici, Andrei, fratele lui Simon Petru:
\par 9 Este aici un băiat care are cinci pâini de orz și doi pești. Dar ce sunt acestea la atâția?
\par 10 Și a zis Iisus: Faceți pe oameni să se așeze. Și era iarbă multă în acel loc. Deci au șezut bărbații în număr ca la cinci mii.
\par 11 Și Iisus a luat pâinile și, mulțumind, a dat ucenicilor, iar ucenicii celor ce ședeau; asemenea și din pești, cât au voit.
\par 12 Iar după ce s-au săturat, a zis ucenicilor Săi: adunați fărâmiturile ce au rămas, ca să nu se piardă ceva.
\par 13 Deci au adunat și au umplut douăsprezece coșuri de fărâmituri, care au rămas de la cei ce au mâncat din cele cinci pâini de orz.
\par 14 Iar oamenii văzând minunea pe care a făcut-o, ziceau: Acesta este într-adevăr Proorocul, Care va să vină în lume.
\par 15 Cunoscând deci Iisus că au să vină și să-L ia cu sila, ca să-L facă rege, S-a dus iarăși în munte, El singur.
\par 16 Și când s-a făcut seră, ucenicii Lui s-au coborât la mare.
\par 17 Și intrând în corabie, mergeau spre Capernaum, dincolo de mare. Și s-a făcut întuneric și Iisus încă nu venise la ei,
\par 18 Și suflând vânt mare, marea se întărâta.
\par 19 După ce au vâslit deci ca la douăzeci și cinci sau treizeci de stadii, au văzut pe Iisus umblând pe apă și apropiindu-Se de corabie, ei s-au înfricoșat.
\par 20 Iar El le-a zis: Eu sunt; nu vă temeți!
\par 21 Deci voiau să-L ia în corabie, și îndată corabia a sosit la țărmul la care mergeau.
\par 22 A doua zi, mulțimea, care sta de cealaltă parte a mării, a văzut că nu era acolo decât numai o corabie mai mică și că Iisus nu intrase în corabie împreună cu ucenicii Săi, ci plecaseră numai ucenicii Lui.
\par 23 Și alte corăbii mai mic au venit din Tiberiada în apropiere de locul unde ei mâncaseră pâinea, după ce Domnul mulțumise.
\par 24 Deci, când a văzut mulțimea că Iisus nu este acolo, nici ucenicii Lui, au intrat și ei în corăbiile cele mici și au venit în Capernaum, căutându-L pe Iisus.
\par 25 Și găsindu-L dincolo de mare, I-au zis: Învățătorule, când ai venit aici?
\par 26 Iisus le-a răspuns și a zis: Adevărat, adevărat zic vouă: Mă căutați nu pentru că ați văzut minuni, ci pentru că ați mâncat din pâini și v-ați săturat.
\par 27 Lucrați nu pentru mâncarea cea pieritoare, ci pentru mâncarea ce rămâne spre viața veșnică și pe care o va da vouă Fiul Omului, căci pe El L-a pecetluit Dumnezeu-Tatăl.
\par 28 Deci au zis către El: Ce să facem, ca să săvârșim lucrările lui Dumnezeu?
\par 29 Iisus a răspuns și le-a zis: Aceasta este lucrarea lui Dumnezeu, ca să credeți în Acela pe Care El L-a trimis.
\par 30 Deci I-au zis: Dar ce minune faci Tu, ca să vedem și să credem în Tine? Ce lucrezi?
\par 31 Părinții noștri au mâncat mană în pustie, precum este scris: "Pâine din cer le-a dat lor să mănânce".
\par 32 Deci Iisus le-a zis: Adevărat, adevărat zic vouă: Nu Moise v-a dat pâinea cea din cer; ci Tatăl Meu vă dă din cer pâinea cea adevărată.
\par 33 Căci pâinea lui Dumnezeu este cea care se coboară din cer și care dă viață lumii.
\par 34 Deci au zis către El: Doamne, dă-ne totdeauna pâinea aceasta.
\par 35 Și Iisus le-a zis: Eu sunt pâinea vieții; cel ce vine la Mine nu va flămânzi și cel ce va crede în Mine nu va înseta niciodată.
\par 36 Dar am spus vouă că M-ați și văzut și nu credeți.
\par 37 Tot ce-Mi dă Tatăl, va veni la Mine; și pe cel ce vine la Mine nu-l voi scoate afară;
\par 38 Pentru că M-am coborât din cer, nu ca să fac voia mea, ci voia Celui ce M-a trimis pe Mine.
\par 39 Și aceasta este voia Celui ce M-a trimis, ca din toți pe care Mi i-a dat Mie să nu pierd nici unul, ci să-i înviez pe ei în ziua cea de apoi.
\par 40 Că aceasta este voia Tatălui Meu, ca oricine vede pe Fiul și crede în El să aibă viață veșnică și Eu îl voi învia în ziua cea de apoi.
\par 41 Deci iudeii murmurau împotriva Lui, fiindcă zisese: Eu sunt pâinea ce s-a coborât din cer.
\par 42 Și ziceau: Au nu este Acesta Iisus, fiul lui Iosif, și nu știm noi pe tatăl Său și pe mama Sa? Cum spune El acum: M-am coborât din cer?
\par 43 Iisus a răspuns și le-a zis: Nu murmurați între voi.
\par 44 Nimeni nu poate să vină la Mine, dacă nu-l va trage Tatăl, Care M-a trimis, și Eu îl voi învia în ziua de apoi.
\par 45 Scris este în prooroci: "Și vor fi toți învățați de Dumnezeu". Deci oricine a auzit și a învățat de la Tatăl la Mine vine.
\par 46 Nu doar că pe Tatăl l-a văzut cineva, decât numai Cel ce este la Dumnezeu; Acesta L-a văzut pe Tatăl.
\par 47 Adevărat, adevărat zic vouă: Cel ce crede în Mine are viață veșnică.
\par 48 Eu sunt pâinea vieții.
\par 49 Părinții voștri au mâncat mană în pustie și au murit.
\par 50 Pâinea care se coboară din cer este aceea din care, dacă mănâncă cineva, nu moare.
\par 51 Eu sunt pâinea cea vie, care s-a pogorât din cer. Cine mănâncă din pâinea aceasta viu va fi în veci. Iar pâinea pe care Eu o voi da pentru viața lumii este trupul Meu.
\par 52 Deci iudeii se certau între ei, zicând: Cum poate Acesta să ne dea trupul Lui să-l mâncăm?
\par 53 Și le-a zis Iisus: Adevărat, adevărat zic vouă, dacă nu veți mânca trupul Fiului Omului și nu veți bea sângele Lui, nu veți avea viață în voi.
\par 54 Cel ce mănâncă trupul Meu și bea sângele Meu are viață veșnică, și Eu îl voi învia în ziua cea de apoi.
\par 55 Trupul este adevărată mâncare și sângele Meu, adevărată băutură.
\par 56 Cel ce mănâncă trupul Meu și bea sângele Meu rămâne întru Mine și Eu întru el.
\par 57 Precum M-a trimis pe Mine Tatăl cel viu și Eu viez pentru Tatăl, și cel ce Mă mănâncă pe Mine va trăi prin Mine.
\par 58 Aceasta este pâinea care s-a pogorât din cer, nu precum au mâncat părinții voștri mana și au murit. Cel ce mănâncă această pâine va trăi în veac.
\par 59 Acestea le-a zis pe când învăța în sinagoga din Capernaum.
\par 60 Deci mulți din ucenicii Lui, auzind, au zis: Greu este cuvântul acesta! Cine poate să-l asculte?
\par 61 Iar Iisus, știind în Sine că ucenicii Lui murmură împotriva Lui, le-a zis: Vă smintește aceasta?
\par 62 Dacă veți vedea pe Fiul Omului, suindu-Se acolo unde era mai înainte?
\par 63 Duhul este cel ce dă viață; trupul nu folosește la nimic. Cuvintele pe care vi le-am spus sunt duh și sunt viață.
\par 64 Dar sunt unii dintre voi care nu cred. Căci Iisus știa de la început cine sunt cei ce nu cred și cine este cel care Îl va vinde.
\par 65 Și zicea: De aceea am spus vouă că nimeni nu poate să vină la Mine, dacă nu-i este dat de la Tatăl.
\par 66 Și de atunci mulți dintre ucenicii Săi s-au dus înapoi și nu mai umblau cu El.
\par 67 Deci a zis Iisus celor doisprezece: Nu vreți și voi să vă duceți?
\par 68 Simon Petru I-a răspuns: Doamne, la cine ne vom duce? Tu ai cuvintele vieții celei veșnice.
\par 69 Și noi am crezut și am cunoscut că Tu ești Hristosul, Fiul Dumnezeului Celui viu.
\par 70 Le-a răspuns Iisus: Oare, nu v-am ales Eu pe voi, cei doisprezece? Și unul dintre voi este diavol!
\par 71 Iar El zicea de Iuda al lui Simon Iscarioteanul, căci acesta, unul din cei doisprezece fiind, avea să-L vândă.

\chapter{7}

\par 1 Și după aceea mergea Iisus prin Galileea, căci nu voia să meargă prin Iudeea, deoarece iudeii căutau să-L ucidă.
\par 2 Și era aproape sărbătoarea iudaică a corturilor.
\par 3 Au zis deci către El frații Lui: Pleacă de aici și du-Te în Iudeea, pentru ca și ucenicii Tăi să vadă lucrurile pe care Tu le faci.
\par 4 Căci nimeni nu lucrează ceva în ascuns, ci caută să se facă cunoscut. Dacă faci acestea, arată-Te pe Tine lumii.
\par 5 Pentru că nici frații Lui nu credeau în El.
\par 6 Deci le-a zis Iisus: Vremea Mea încă n-a sosit; dar vremea voastră totdeauna este gata.
\par 7 Pe voi lumea nu poate să vă urască, dar pe Mine Mă urăște, pentru că Eu mărturisesc despre ea că lucrurile ei sunt rele.
\par 8 Voi duceți-vă la sărbătoare; Eu nu merg la sărbătoarea aceasta, căci vremea Mea nu s-a împlinit încă.
\par 9 Acestea spunându-le, a rămas în Galileea.
\par 10 Dar după ce frații Săi s-au dus la sărbătoare, atunci S-a suit și El, dar nu pe față, ci pe ascuns.
\par 11 În timpul sărbătorii iudeii Îl căutau și ziceau: Unde este Acela?
\par 12 Și cârtire multă era despre El în mulțime; unii ziceau: Este bun; iar alții ziceau: Nu, ci amăgește mulțimea.
\par 13 Totuși, de frica iudeilor, nimeni nu vorbea despre El pe față.
\par 14 Iar la jumătatea praznicului Iisus S-a suit în templu și învăța.
\par 15 Și iudeii se mirau zicând: Cum știe Acesta carte fără să fi învățat?
\par 16 Deci le-a răspuns Iisus și a zis: Învățătura Mea nu este a Mea, ci a Celui ce M-a trimis.
\par 17 De vrea cineva să facă voia Lui, va cunoaște despre învățătura aceasta dacă este de la Dumnezeu sau dacă Eu vorbesc de la Mine Însumi.
\par 18 Cel care vorbește de la sine își caută slava sa; iar cel care caută slava celui ce l-a trimis pe el, acela este adevărat și nedreptate nu este în el.
\par 19 Oare nu Moise v-a dat Legea? Și nimeni dintre voi nu ține Legea. De ce căutați să Mă ucideți?
\par 20 Și mulțimea a răspuns: Ai demon. Cine caută să te ucidă?
\par 21 Iisus a răspuns și le-a zis: Un lucru am făcut și toți vă mirați.
\par 22 De aceea Moise v-a dat tăierea împrejur, nu că este de la Moise, ci de la părinți, și sâmbăta tăiați împrejur pe om.
\par 23 Dacă omul primește tăierea împrejur sâmbăta, ca să nu se strice Legea lui Moise, vă mâniați pe Mine că am făcut sâmbăta un om întreg sănătos?
\par 24 Nu judecați după înfățișare, ci judecați judecată dreaptă.
\par 25 Deci ziceau unii dintre ierusalimiteni: Nu este, oare, Acesta pe care-L căutau să-L ucidă?
\par 26 Și iată că vorbește pe față și ei nu-I zic nimic. Nu cumva căpeteniile au cunoscut cu adevărat că Acesta e Hristos?
\par 27 Dar pe Acesta Îl știm de unde este. Însă Hristosul, când va veni, nimeni nu știe de unde este.
\par 28 Deci a strigat Iisus în templu, învățând și zicând: Și pe Mine Mă știți și știți de unde sunt; și Eu n-am venit de la Mine, dar adevărat este Cel ce M-a trimis pe Mine și pe Care voi nu-L știți.
\par 29 Eu Îl știu pe El, căci de la El sunt și El M-a trimis pe Mine.
\par 30 Deci căutau să-L prindă, dar nimeni n-a pus mâna pe El, pentru că nu venise încă ceasul Lui.
\par 31 Dar mulți din mulțime au crezut în El și ziceau: Hristosul când va veni va face El minuni mai multe decât a făcut Acesta?
\par 32 Au auzit fariseii mulțimea murmurând acestea despre El și au trimis arhiereii și fariseii slujitori ca să-L prindă.
\par 33 Dar Iisus le-a zis: Puțin timp mai sunt cu voi și Mă duc la Cel ce M-a trimis.
\par 34 Mă veți căuta și nu Mă veți găsi; și unde sunt Eu, voi nu puteți să veniți.
\par 35 Deci au zis iudeii, între ei: Unde are să Se ducă Acesta, ca noi să nu-L găsim? Nu cumva va merge la cei împrăștiați printre elini și va învăța pe elini?
\par 36 Ce înseamnă acest cuvânt pe care l-a spus: Mă veți căuta și nu Mă veți găsi și unde sunt Eu, voi nu puteți să veniți?
\par 37 Iar în ziua cea din urmă - ziua cea mare a sărbătorii - Iisus a stat între ei și a strigat, zicând: Dacă însetează cineva, să vină la Mine și să bea.
\par 38 Cel ce crede în Mine, precum a zis Scriptura: râuri de apă vie vor curge din pântecele lui.
\par 39 Iar aceasta a zis-o despre Duhul pe Care aveau să-L primească acei ce cred în El. Căci încă nu era (dat) Duhul, pentru că Iisus încă nu fusese preaslăvit.
\par 40 Deci din mulțime, auzind cuvintele acestea, ziceau: Cu adevărat, Acesta este Proorocul.
\par 41 Iar alții ziceau: Acesta este Hristosul. Iar alții ziceau: Nu cumva din Galileea va să vină Hristos?
\par 42 N-a zis, oare, Scriptura că Hristos va să vină din sămânța lui David și din Betleem, cetatea lui David?
\par 43 Și s-a făcut dezbinare în mulțime pentru El.
\par 44 Și unii dintre ei voiau să-L prindă, dar nimeni n-a pus mâinile pe El.
\par 45 Deci slugile au venit la arhierei și farisei, și le-au zis aceia: De ce nu L-ați adus?
\par 46 Slugile au răspuns: Niciodată n-a vorbit un om așa cum vorbește Acest Om.
\par 47 Și le-au răspuns deci fariseii: Nu cumva ați fost și voi amăgiți?
\par 48 Nu cumva a crezut în El cineva dintre căpetenii sau dintre farisei?
\par 49 Dar mulțimea aceasta, care nu cunoaște Legea, este blestemată!
\par 50 A zis către ei Nicodim, cel ce venise mai înainte la El, noaptea, fiind unul dintre ei:
\par 51 Nu cumva Legea noastră judecă pe om, dacă nu-l ascultă mai întâi și nu știe ce a făcut?
\par 52 Ei au răspuns și i-au zis: Nu cumva și tu ești din Galileea? Cercetează și vezi că din Galileea nu s-a ridicat prooroc.
\par 53 Și s-a dus fiecare la casa sa.

\chapter{8}

\par 1 Iar Iisus S-a dus la Muntele Măslinilor.
\par 2 Dar dimineața iarăși a venit în templu, și tot poporul venea la El; și El, șezând, îi învăța.
\par 3 Și au adus la El fariseii și cărturarii pe o femeie, prinsă în adulter și, așezând-o în mijloc,
\par 4 Au zis Lui: Învățătorule, această femeie a fost prinsă asupra faptului de adulter;
\par 5 Iar Moise ne-a poruncit în Lege ca pe unele ca acestea să le ucidem cu pietre. Dar Tu ce zici?
\par 6 Și aceasta ziceau, ispitindu-L, ca să aibă de ce să-L învinuiască. Iar Iisus, plecându-Se în jos, scria cu degetul pe pământ.
\par 7 Și stăruind să-L întrebe, El S-a ridicat și le-a zis: Cel fără de păcat dintre voi să arunce cel dintâi piatra asupra ei.
\par 8 Iarăși plecându-Se, scria pe pământ.
\par 9 Iar ei auzind aceasta și mustrați fiind de cuget, ieșeau unul câte unul, începând de la cei mai bătrâni și până la cel din urmă, și a rămas Iisus singur și femeia, stând în mijloc.
\par 10 Și ridicându-Se Iisus și nevăzând pe nimeni decât pe femeie, i-a zis: Femeie, unde sunt pârâșii tăi? Nu te-a osândit nici unul?
\par 11 Iar ea a zis: Nici unul, Doamne. Și Iisus i-a zis: Nu te osândesc nici Eu. Mergi; de acum să nu mai păcătuiești.
\par 12 Deci iarăși le-a vorbit Iisus zicând: Eu sunt Lumina lumii; cel ce Îmi urmează Mie nu va umbla în întuneric, ci va avea lumina vieții.
\par 13 De aceea fariseii I-au zis: Tu mărturisești despre Tine Însuți; mărturia Ta nu este adevărată.
\par 14 A răspuns Iisus și le-a zis: Chiar dacă Eu mărturisesc despre Mine Însumi, mărturia Mea este adevărată, fiindcă știu de unde am venit și unde Mă duc. Voi nu știți de unde vin, nici unde Mă duc.
\par 15 Voi judecați după trup; Eu nu judec pe nimeni.
\par 16 Și chiar dacă Eu judec, judecata Mea este adevărată, pentru că nu sunt singur, ci Eu și Cel ce M-a trimis pe Mine.
\par 17 Și în Legea voastră este scris că mărturia a doi oameni este adevărată.
\par 18 Eu sunt Cel ce mărturisesc despre Mine Însumi și mărturisește despre Mine Tatăl, Cel ce M-a trimis.
\par 19 Îi ziceau deci: Unde este Tatăl Tău? Răspuns-a Iisus: Nu mă știți nici pe Mine nici pe Tatăl Meu; dacă M-ați ști pe Mine, ați ști și pe Tatăl Meu.
\par 20 Cuvintele acestea le-a grăit Iisus în vistierie, pe când învăța în templu; și nimeni nu L-a prins, că încă nu venise ceasul Lui.
\par 21 Și iarăși le-a zis: Eu Mă duc și Mă veți căuta și veți muri în păcatul vostru. Unde Mă duc Eu, voi nu puteți veni.
\par 22 Deci ziceau iudeii: Nu cumva Își va ridica singur viața? Că zice: Unde Mă duc Eu, voi nu puteți veni.
\par 23 Și El le zicea: Voi sunteți din cele de jos; Eu sunt din cele de sus. Voi sunteți din lumea aceasta; Eu nu sunt din lumea aceasta.
\par 24 V-am spus deci vouă că veți muri în păcatele voastre. Căci dacă nu credeți că Eu sunt, veți muri în păcatele voastre.
\par 25 Deci Îi ziceau ei: Cine ești Tu? Și a zis lor Iisus: Ceea ce v-am spus de la început.
\par 26 Multe am de spus despre voi și de judecat. Dar Cel ce M-a trimis pe Mine adevărat este, și cele ce am auzit de la El, Eu acestea le grăiesc în lume.
\par 27 Și ei n-au înțeles că le vorbea despre Tatăl.
\par 28 Deci le-a zis Iisus: Când veți înălța pe Fiul Omului, atunci veți cunoaște că Eu sunt și că de la Mine însumi nu fac nimic, ci precum M-a învățat Tatăl, așa vorbesc.
\par 29 Și Cel ce M-a trimis este cu Mine; nu M-a lăsat singur, fiindcă Eu fac pururea cele plăcute Lui.
\par 30 Spunând El acestea, mulți au crezut în El.
\par 31 Deci zicea Iisus către iudeii care crezuseră în El: Dacă veți rămâne în cuvântul Meu, sunteți cu adevărat ucenici ai Mei;
\par 32 Și veți cunoaște adevărul, iar adevărul vă va face liberi.
\par 33 Ei însă I-au răspuns: Noi suntem sămânța lui Avraam și nimănui niciodată n-am fost robi. Cum zici Tu că: Veți fi liberi?
\par 34 Iisus le-a răspuns: Adevărat, adevărat vă spun: Oricine săvârșește păcatul este rob al păcatului.
\par 35 Iar robul nu rămâne în casă în veac; Fiul însă rămâne în veac.
\par 36 Deci, dacă Fiul vă va face liberi, liberi veți fi într-adevăr.
\par 37 Știu vă sunteți sămânța lui Avraam, dar căutați să Mă omorâți, pentru că cuvâtul Meu nu încape în voi.
\par 38 Eu vorbesc ceea ce am văzut la Tatăl Meu, iar voi faceți ceea ce ați auzit de la tatăl vostru.
\par 39 Ei au răspuns și I-au zis: Tatăl nostru este Avraam. Iisus le-a zis: Dacă ați fi fiii lui Avraam, ați face faptele lui Avraam.
\par 40 Dar voi acum căutați să Mă ucideți pe Mine, Omul care v-am spus adevărul pe care l-am auzit de la Dumnezeu. Avraam n-a făcut aceasta.
\par 41 Voi faceți faptele tatălui vostru. Zis-au Lui: Noi nu ne-am născut din desfrânare. Un tată avem: pe Dumnezeu.
\par 42 Le-a zis Iisus: Dacă Dumnezeu are fi Tatăl vostru, M-ați iubi pe Mine, căci de la Dumnezeu am ieșit și am venit. Pentru că n-am venit de la Mine însumi, ci El M-a trimis.
\par 43 De ce nu înțelegeți vorbirea Mea? Fiindcă nu puteți să dați ascultare cuvântului Meu.
\par 44 Voi sunteți din tatăl vostru diavolul și vreți să faceți poftele tatălui vostru. El, de la început, a fost ucigător de oameni și nu a stat întru adevăr, pentru că nu este adevăr întru el. Când grăiește minciuna, grăiește dintru ale sale, căci este mincinos și tatăl minciunii.
\par 45 Dar pe Mine, fiindcă spun adevărul, nu Mă credeți.
\par 46 Cine dintre voi Mă vădește de păcat? Dacă spun adevărul, de ce voi nu Mă credeți?
\par 47 Cel care este de la Dumnezeu ascultă cuvintele lui Dumnezeu; de aceea voi nu ascultați pentru că nu sunteți de la Dumnezeu.
\par 48 Au răspuns iudeii și I-au zis: Oare, nu zicem noi bine că Tu ești samarinean și ai demon?
\par 49 A răspuns Iisus: Eu nu am demon, ci cinstesc pe Tatăl Meu, și voi nu Mă cinstiți pe Mine.
\par 50 Dar Eu nu caut slava Mea. Este cine să o caute și să judece.
\par 51 Adevărat, adevărat zic vouă: Dacă cineva va păzi cuvântul Meu, nu va vedea moartea în veac.
\par 52 Iudeii I-au zis: Acum am cunoscut că ai demon. Avraam a murit, de asemenea și proorocii; și Tu zici: Dacă cineva va păzi cuvântul Meu, nu va gusta moartea în veac.
\par 53 Nu cumva ești Tu mai mare decât tatăl nostru Avraam, care a murit? Și au murit și proorocii. Cine te faci Tu a fi?
\par 54 Iisus a răspuns: Dacă Mă slăvesc Eu pe Mine Însumi, slava Mea nimic nu este. Tatăl Meu este Cel care Mă slăvește, despre Care ziceți voi că este Dumnezeul vostru.
\par 55 Și nu L-ați cunoscut, dar Eu Îl știu; și dacă aș zice că nu-L știu, aș fi mincinos asemenea vouă. Ci Îl știu și păzesc cuvântul Lui.
\par 56 Avraam, părintele vostru, a fost bucuros să vadă ziua Mea și a văzut-o și s-a bucurat.
\par 57 Deci au zis iudeii către El: Încă nu ai cincizeci de ani și l-ai văzut pe Avraam?
\par 58 Iisus le-a zis: Adevărat, adevărat zic vouă: Eu sunt mai înainte de a fi fost Avraam.
\par 59 Deci au luat pietre ca să arunce asupra Lui. Dar Iisus S-a ferit și a ieșit din templu și, trecând prin mijlocul lor, S-a dus.

\chapter{9}

\par 1 Și trecând Iisus, a văzut un om orb din naștere.
\par 2 Și ucenicii Lui L-au întrebat, zicând: Învățătorule, cine a păcătuit; acesta sau părinții lui, de s-a născut orb?
\par 3 Iisus a răspuns: Nici el n-a păcătuit, nici părinții lui, ci ca să se arate în el lucrările lui Dumnezeu.
\par 4 Trebuie să fac, până este ziuă, lucrările Celui ce M-a trimis pe Mine; că vine noaptea, când nimeni nu poate să lucreze.
\par 5 Atât cât sunt în lume, Lumină a lumii sunt.
\par 6 Acestea zicând, a scuipat jos și a făcut tină din scuipat, și a uns cu tină ochii orbului.
\par 7 Și i-a zis: Mergi de te spală în scăldătoarea Siloamului (care se tâlcuiește: trimis). Deci s-a dus și s-a spălat și a venit văzând.
\par 8 Iar vecinii și cei ce-l văzuseră mai înainte că era orb ziceau: Nu este acesta cel ce ședea și cerșea?
\par 9 Unii ziceau: El este. Alții ziceau: Nu este el, ci seamănă cu el. Dar acela zicea: Eu sunt.
\par 10 Deci îi ziceau: Cum ți s-au deschis ochii?
\par 11 Acela a răspuns: Omul care se numește Iisus a făcut tină și a uns ochii mei; și mi-a zis: Mergi la scăldătoarea Siloamului și te spală. Deci, ducându-mă și spălându-mă, am văzut.
\par 12 Zis-au lui: Unde este Acela? Și el a zis: Nu știu.
\par 13 L-au dus la farisei pe cel ce fusese oarecând orb.
\par 14 Și era sâmbătă în ziua în care Iisus a făcut tină și i-a deschis ochii.
\par 15 Deci iarăși îl întrebau și fariseii cum a văzut. Iar el le-a zis: Tină a pus pe ochii mei, și m-am spălat și văd.
\par 16 Deci ziceau unii dintre farisei: Acest om nu este de la Dumnezeu, fiindcă nu ține sâmbăta. Iar alții ziceau: Cum poate un om păcătos să facă asemenea minuni? Și era dezbinare între ei.
\par 17 Au zis deci orbului iarăși: Dar tu ce zici despre El, că ți-a deschis ochii? Iar el a zis că prooroc este.
\par 18 Dar iudeii n-au crezut despre el că era orb și a văzut, până ce n-au chemat pe părinții celui ce vedea.
\par 19 Și i-au întrebat, zicând: Acesta este fiul vostru, despre care ziceți că s-a născut orb? Deci cum vede el acum?
\par 20 Au răspuns deci părinții lui și au zis: Știm că acesta este fiul nostru și că s-a născut orb.
\par 21 Dar cum vede el acum, noi nu știm; sau cine i-a deschis ochii lui, noi nu știm. Întrebați-l pe el; este în vârstă; va vorbi singur despre sine.
\par 22 Acestea le-au spus părinții lui, pentru că se temeau de iudei. Căci iudeii puseseră acum la cale că, dacă cineva va mărturisi că El este Hristos, să fie dat afară din sinagogă.
\par 23 De aceea au zis părinții lui: Este în vârstă; întrebați-l pe el.
\par 24 Deci au chemat a doua oară pe omul care fusese orb și i-au zis: Dă slavă lui Dumnezeu. Noi știm că Omul Acesta e păcătos.
\par 25 A răspuns deci acela: Dacă este păcătos, nu știu. Un lucru știu: că fiind orb, acum văd.
\par 26 Deci i-au zis: Ce ți-a făcut? Cum ți-a deschis ochii?
\par 27 Le-a răspuns: V-am spus acum și n-ați auzit? De ce voiți să auziți iarăși? Nu cumva voiți și voi să vă faceți ucenici ai Lui?
\par 28 Și l-au ocărât și i-au zis: Tu ești ucenic al Aceluia, iar noi suntem ucenici ai lui Moise.
\par 29 Noi știm că Dumnezeu a vorbit lui Moise, iar pe Acesta nu-L știm de unde este.
\par 30 A răspuns omul și le-a zis: Tocmai în aceasta stă minunea: că voi nu știți de unde este și El mi-a deschis ochii.
\par 31 Și noi știm că Dumnezeu nu-i ascultă pe păcătoși; dar de este cineva cinstitor de Dumnezeu și face voia Lui, pe acesta îl ascultă.
\par 32 Din veac nu s-a auzit să fi deschis cineva ochii unui orb din naștere.
\par 33 De n-ar fi Acesta de la Dumnezeu n-ar putea să facă nimic.
\par 34 Au răspuns și i-au zis: În păcate te-ai născut tot, și tu ne înveți pe noi? Și l-au dat afară.
\par 35 Și a auzit Iisus că l-au dat afară. Și, găsindu-l, i-a zis: Crezi tu în Fiul lui Dumnezeu?
\par 36 El a răspuns și a zis: Dar cine este, Doamne, ca să cred în El?
\par 37 Și a zis Iisus: L-ai și văzut! Și Cel ce vorbește cu tine Acela este.
\par 38 Iar el a zis: Cred, Doamne. Și s-a închinat Lui.
\par 39 Și a zis: Spre judecată am venit în lumea aceasta, ca cei care nu văd să vadă, iar cei care văd să fie orbi.
\par 40 Și au auzit acestea unii dintre fariseii, care erau cu El, și I-au zis: Oare și noi suntem orbi?
\par 41 Iisus le-a zis: Dacă ați fi orbi n-ați avea păcat. Dar acum ziceți: Noi vedem. De aceea păcatul rămâne asupra voastră.

\chapter{10}

\par 1 Adevărat, adevărat zic vouă: Cel ce nu intră pe ușă, în staulul oilor, ci sare pe aiurea, acela este fur și tâlhar.
\par 2 Iar cel ce intră prin ușă este păstorul oilor.
\par 3 Acestuia portarul îi deschide și oile ascultă de glasul lui, și oile sale le cheamă pe nume și le mână afară.
\par 4 Și când le scoate afară pe toate ale sale, merge înaintea lor, și oile merg după el, căci cunosc glasul lui.
\par 5 Iar după un străin, ele nu vor merge, ci vor fugi de el, pentru că nu cunosc glasul lui.
\par 6 Această pildă le-a spus-o Iisus, dar ei n-au înțeles ce înseamnă cuvintele Lui.
\par 7 A zis deci iarăși Iisus: Adevărat, adevărat zic vouă: Eu sunt ușa oilor.
\par 8 Toți câți au venit mai înainte de Mine sunt furi și tâlhari, dar oile nu i-au ascultat.
\par 9 Eu sunt ușa: de va intra cineva prin Mine, se va mântui; și va intra și va ieși și pășune va afla.
\par 10 Furul nu vine decât ca să fure și să junghie și să piardă. Eu am venit ca viață să aibă și din belșug să aibă.
\par 11 Eu sunt păstorul cel bun. Păstorul cel bun își pune sufletul pentru oile sale.
\par 12 Iar cel plătit și cel care nu este păstor, și ale cărui oi nu sunt ale lui, vede lupul venind și lasă oile și fuge; și lupul le răpește și le risipește.
\par 13 Dar cel plătit fuge, pentru că este plătit și nu are grijă de oi.
\par 14 Eu sunt păstorul cel bun și cunosc pe ale Mele și ale Mele Mă cunosc pe Mine.
\par 15 Precum Mă cunoaște Tatăl și Eu cunosc pe Tatăl. Și sufletul Îmi pun pentru oi.
\par 16 Am și alte oi, care sunt din staulul acesta. Și pe acelea trebuie să le aduc, și vor auzi glasul Meu și va fi o turmă și un păstor.
\par 17 Pentru aceasta Mă iubește Tatăl, fiindcă Eu Îmi pun sufletul, ca iarăși să-l iau.
\par 18 Nimeni nu-l ia de la Mine, ci Eu de la Mine Însumi îl pun. Putere am Eu ca să-l pun și putere am iarăși ca să-l iau. Această poruncă am primit-o de la Tatăl Meu.
\par 19 Iarăși s-a făcut dezbinare între iudei, pentru cuvintele acestea.
\par 20 Și mulți dintre ei ziceau: Are demon și este nebun. De ce să-L ascultați?
\par 21 Alții ziceau: Cuvintele acestea nu sunt ale unui demonizat. Cum poate un demon să deschidă ochii orbilor?
\par 22 Și era atunci la Ierusalim sărbătoarea înnoirii templului și era iarnă.
\par 23 Iar Iisus umbla prin templu, în pridvorul lui Solomon.
\par 24 Deci L-au împresurat iudeii și Îi ziceau: Până când ne scoți sufletul? Dacă Tu ești Hristosul, spune-o nouă pe față.
\par 25 Iisus le-a răspuns: V-am spus și nu credeți. Lucrările pe care le fac în numele Tatălui Meu, acestea mărturisesc despre Mine.
\par 26 Dar voi nu credeți, pentru că nu sunteți dintre oile Mele.
\par 27 Oile Mele ascultă de glasul Meu și Eu le cunosc pe ele, și ele vin după Mine.
\par 28 Și Eu le dau viață veșnică și nu vor pieri în veac, și din mâna Mea nimeni nu le va răpi.
\par 29 Tatăl Meu, Care Mi le-a dat, este mai mare decât toți, și nimeni nu poate să le răpească din mâna Tatălui Meu.
\par 30 Iar Eu și Tatăl Meu una suntem.
\par 31 Iarăși au luat pietre iudeii ca să arunce asupra Lui.
\par 32 Iisus le-a răspuns: Multe lucruri bune v-am arătat vouă de la Tatăl Meu. Pentru care din ele, aruncați cu pietre asupra Mea?
\par 33 I-au răspuns iudeii: Nu pentru lucru bun aruncăm cu pietre asupra Ta, ci pentru hulă și pentru că Tu, om fiind, Te faci pe Tine Dumnezeu.
\par 34 Iisus le-a răspuns: Nu e scris în Legea voastră că "Eu am zis: dumnezei sunteți?"
\par 35 Dacă i-a numit dumnezei pe aceia către care a fost cuvântul lui Dumnezeu - și Scriptura nu poate să fie desființată -
\par 36 Despre Cel pe care Tatăl L-a sfințit și L-a trimis în lume, voi ziceți: Tu hulești, căci am spus: Fiul lui Dumnezeu sunt?
\par 37 Dacă nu fac lucrările Tatălui Meu, să nu credeți în Mine.
\par 38 Iar dacă le fac, chiar dacă nu credeți în Mine, credeți în aceste lucrări, ca să știți și să cunoașteți că Tatăl este în Mine și Eu în Tatăl.
\par 39 Căutau deci iarăși să-L prindă și Iisus a scăpat din mâna lor.
\par 40 Și a plecat iarăși dincolo de Iordan, în locul unde Ioan boteza la început, și a rămas acolo.
\par 41 Și mulți au venit la El și ziceau: Ioan n-a făcut nici o minune, dar toate câte Ioan a zis despre Acesta erau adevărate.
\par 42 Și mulți au crezut în El acolo.

\chapter{11}

\par 1 Și era bolnav un oarecare Lazăr din Betania, satul Mariei și al Martei, sora ei.
\par 2 Iar Maria era aceea care a uns cu mir pe Domnul și I-a șters picioarele cu părul capului ei, al cărei frate Lazăr era bolnav.
\par 3 Deci au trimis surorile la El, zicând: Doamne, iată, cel pe care îl iubești este bolnav.
\par 4 Iar Iisus, auzind, a zis: Această boală nu este spre moarte, ci pentru slava lui Dumnezeu, ca, prin ea, Fiul lui Dumnezeu să Se slăvească.
\par 5 Și iubea Iisus pe Marta și pe sora ei și pe Lazăr.
\par 6 Când a auzit, deci, că este bolnav, atunci a rămas două zile în locul în care era.
\par 7 Apoi, după aceea, a zis ucenicilor: Să mergem iarăși în Iudeea.
\par 8 Ucenicii I-au zis: Învățătorule, acum căutau iudeii să Te ucidă cu pietre, și iarăși Te duci acolo?
\par 9 A răspuns Iisus: Nu sunt oare douăsprezece ceasuri într-o zi? Dacă umblă cineva ziua, nu se împiedică, pentru că el vede lumina acestei lumi;
\par 10 Iar dacă umblă cineva noaptea se împiedică, pentru că lumina nu este în el.
\par 11 A zis acestea, și după aceea le-a spus: Lazăr, prietenul nostru, a adormit; Mă duc să-l trezesc.
\par 12 Deci I-au zis ucenicii: Doamne, dacă a adormit, se va face bine.
\par 13 Iar Iisus vorbise despre moartea lui, iar ei credeau că vorbește despre somn ca odihnă.
\par 14 Deci atunci Iisus le-a spus lor pe față: Lazăr a murit.
\par 15 Și Mă bucur pentru voi, ca să credeți că n-am fost acolo. Dar să mergem la el.
\par 16 Deci a zis Toma, care se numește Geamănul, celorlalți ucenici: Să mergem și noi și să murim cu El.
\par 17 Deci, venind, Iisus l-a găsit pus de patru zile în mormânt.
\par 18 Iar Betania era aproape de Ierusalim, ca la cincisprezece stadii.
\par 19 Și mulți dintre iudei veniseră la Marta și Maria ca să le mângâie pentru fratele lor.
\par 20 Deci Marta, când a auzit că vine Iisus, a ieșit în întâmpinarea Lui, iar Maria ședea în casă.
\par 21 Și a zis către Iisus: Doamne, dacă ai fi fost aici, fratele meu n-ar fi murit.
\par 22 Dar și acum știu că oricâte vei cere de la Dumnezeu, Dumnezeu îți va da.
\par 23 Iisus i-a zis: Fratele tău va învia.
\par 24 Marta i-a zis: Știu că va învia la înviere, în ziua cea de apoi.
\par 25 Și Iisus i-a zis: Eu sunt învierea și viața; cel ce crede în Mine, chiar dacă va muri, va trăi.
\par 26 Și oricine trăiește și crede în Mine nu va muri în veac. Crezi tu aceasta?
\par 27 Zis-a Lui: Da, Doamne. Eu am crezut că Tu ești Hristosul, Fiul lui Dumnezeu, Care a venit în lume.
\par 28 Și zicând aceasta, s-a dus și a chemat pe Maria, sora ei, zicându-i în taină: Învățătorul este aici și te cheamă.
\par 29 Când a auzit aceea, s-a sculat degrabă și a venit la El.
\par 30 Și Iisus nu venise încă în sat, ci era în locul unde Îl întâmpinase Marta.
\par 31 Iar iudeii care erau cu ea în casă și o mângâiau, văzând pe Maria că s-a sculat degrabă și a ieșit afară, au mers după ea socotind că a plecat la mormânt, ca să plângă acolo.
\par 32 Deci Maria, când a venit unde era Iisus, văzându-L, a căzut la picioarele Lui, zicându-I: Doamne, dacă ai fi fost aici, fratele meu n-ar fi murit.
\par 33 Deci Iisus, când a văzut-o plângând și pe iudeii care veniseră cu ea plângând și ei, a suspinat cu duhul și S-a tulburat întru Sine.
\par 34 Și a zis: Unde l-ați pus? Zis-au Lui: Doamne, vino și vezi.
\par 35 Și a lăcrimat Iisus.
\par 36 Deci ziceau iudeii: Iată cât de mult îl iubea.
\par 37 Iar unii dintre ei ziceau: Nu putea, oare, Acesta care a deschis ochii orbului să facă așa ca și acesta să nu moară?
\par 38 Deci suspinând iarăși Iisus întru Sine, a mers la mormânt. Și era o peșteră și o piatră era așezată pe ea.
\par 39 Iisus a zis: Ridicați piatra. Marta, sora celui răposat, I-a zis: Doamne, deja miroase, că este a patra zi.
\par 40 Iisus i-a zis: Nu ți-am spus că dacă vei crede, vei vedea slava lui Dumnezeu?
\par 41 Au ridicat deci piatra, iar Iisus Și-a ridicat ochii în sus și a zis: Părinte, Îți mulțumesc că M-ai ascultat.
\par 42 Eu știam că întotdeauna Mă asculți, dar pentru mulțimea care stă împrejur am zis, ca să creadă că Tu M-ai trimis.
\par 43 Și zicând acestea, a strigat cu glas mare: Lazăre, vino afară!
\par 44 Și a ieșit mortul, fiind legat la picioare și la mâini cu fâșii de pânză și fața lui era înfășurată cu mahramă. Iisus le-a zis: Dezlegați-l și lăsați-l să meargă.
\par 45 Deci mulți dintre iudeii care veniseră la Maria și văzuseră ce a făcut Iisus au crezut în El.
\par 46 Iar unii dintre ei s-au dus la farisei și le-au spus cele ce făcuse Iisus.
\par 47 Deci arhiereii și fariseii au adunat sinedriul și ziceau: Ce facem, pentru că Omul Acesta face multe minuni?
\par 48 Dacă-L lăsăm așa toți vor crede în El, și vor veni romanii și ne vor lua țara și neamul.
\par 49 Iar Caiafa, unul dintre ei, care în anul acela era arhiereu le-a zis: Voi nu știți nimic;
\par 50 Nici nu gândiți că ne este mai de folos să moară un om pentru popor, decât să piară tot neamul.
\par 51 Dar aceasta n-a zis-o de la sine, ci, fiind arhiereu al anului aceluia, a proorocit că Iisus avea să moară pentru neam,
\par 52 Și nu numai pentru neam, ci și ca să adune laolaltă pe fiii lui Dumnezeu cei împrăștiați.
\par 53 Deci, din ziua aceea, s-au hotărât ca să-L ucidă.
\par 54 De aceea Iisus nu mai umbla pe față printre iudei, ci a plecat de acolo într-un ținut aproape de pustie, într-o cetate numită Efraim, și acolo a rămas cu ucenicii Săi.
\par 55 Și era aproape Paștile iudeilor și mulți din țară s-au suit la Ierusalim, mai înainte de Paști, ca să se curățească.
\par 56 Deci căutau pe Iisus și, pe când stăteau în templu, ziceau între ei: Ce vi se pare? Oare nu va veni la sărbătoare?
\par 57 Iar arhiereii și fariseii dăduseră porunci, că dacă va ști cineva unde este, să dea de veste, ca să-L prindă.

\chapter{12}

\par 1 Deci, cu șase zile înainte de Paști, Iisus a venit în Betania, unde era Lazăr, pe care îl înviase din morți.
\par 2 Și I-au făcut acolo cină și Marta slujea. Iar Lazăr era unul dintre cei ce ședeau cu El la masă.
\par 3 Deci Maria, luând o litră cu mir de nard curat, de mare preț, a uns picioarele lui Iisus și le-a șters cu părul capului ei, iar casa s-a umplut de mirosul mirului.
\par 4 Iar Iuda Iscarioteanul, unul dintre ucenicii Lui, care avea să-L vândă, a zis:
\par 5 Pentru ce nu s-a vândut mirul acesta cu trei sute de dinari și să-i fi dat săracilor?
\par 6 Dar el a zis aceasta, nu pentru că îi era grijă de săraci, ci pentru că era fur și, având punga, lua din ce se punea în ea.
\par 7 A zis deci Iisus: Las-o, că pentru ziua îngropării Mele l-a păstrat.
\par 8 Că pe săraci totdeauna îi aveți cu voi, dar pe Mine nu Mă aveți totdeauna.
\par 9 Deci mulțime mare de iudei au aflat că este acolo și au venit nu numai pentru Iisus, ci să vadă și pe Lazăr pe care-l înviase din morți.
\par 10 Și s-au sfătuit arhiereii ca și pe Lazăr să-l omoare.
\par 11 Căci, din cauza lui mulți dintre iudei mergeau și credeau în Iisus.
\par 12 A doua zi, mulțime multă, care venise la sărbătoare, auzind că Iisus vine în Ierusalim,
\par 13 Au luat ramuri de finic și au ieșit întru întâmpinarea Lui și strigau: Osana! Binecuvântat este Cel ce vine întru numele Domnului, Împăratul lui Israel!
\par 14 Și Iisus, găsind un asin tânăr, a șezut pe el, precum este scris:
\par 15 "Nu te teme, fiica Sionului! Iată Împăratul tău vine, șezând pe mânzul asinei".
\par 16 Acestea nu le-au înțeles ucenicii Lui la început, dar când S-a preaslăvit Iisus, atunci și-au adus aminte că acestea I le-au făcut Lui.
\par 17 Deci da mărturie mulțimea care era cu El, când l-a strigat pe Lazăr din mormânt și l-a înviat din morți.
\par 18 De aceea L-a și întâmpinat mulțimea, pentru că auzise că El a făcut minunea aceasta.
\par 19 Deci fariseii ziceau între ei: Vedeți că nimic nu folosiți! Iată, lumea s-a dus după El.
\par 20 Și erau niște elini din cei ce se suiseră să se închine la sărbătoare.
\par 21 Deci aceștia au venit la Filip, cel ce era din Betsaida Galileii, și l-au rugat zicând: Doamne, voim să vedem pe Iisus.
\par 22 Filip a venit și i-a spus la Andrei, și Andrei și Filip au venit și I-au spus lui Iisus.
\par 23 Iar Iisus le-a răspuns, zicând: A venit ceasul ca să fie preaslăvit Fiul Omului.
\par 24 Adevărat, adevărat zic vouă că dacă grăuntele de grâu, când cade în pământ, nu va muri, rămâne singur; iar dacă va muri, aduce multă roadă.
\par 25 Cel ce își iubește sufletul îl va pierde; iar cel ce își urăște sufletul în lumea aceasta îl va păstra pentru viața veșnică.
\par 26 Dacă-Mi slujește cineva, să-Mi urmeze, și unde sunt Eu, acolo va fi și slujitorul Meu. Dacă-Mi slujește cineva, Tatăl Meu îl va cinsti.
\par 27 Acum sufletul Meu e tulburat, și ce voi zice? Părinte, izbăvește-Mă, de ceasul acesta. Dar pentru aceasta am venit în ceasul acesta.
\par 28 Părinte, preaslăvește-Ți numele! Atunci a venit glas din cer: Și L-am preaslăvit și iarăși Îl voi preaslăvi.
\par 29 Iar mulțimea care sta și auzea zicea: A fost tunet! Alții ziceau: Înger I-a vorbit!
\par 30 Iisus a răspuns și a zis: Nu pentru Mine s-a făcut glasul acesta, ci pentru voi.
\par 31 Acum este judecata acestei lumi; acum stăpânitorul lumii acesteia va fi aruncat afară.
\par 32 Iar Eu, când Mă voi înălța de pe pământ, îi voi trage pe toți la Mine.
\par 33 Iar aceasta zicea, arătând cu ce moarte avea să moară.
\par 34 I-a răspuns deci mulțimea: Noi am auzit din Lege că Hristosul rămâne în veac; și cum zici Tu că Fiul Omului trebuie să fie înălțat? Cine este acesta, Fiul Omului?
\par 35 Deci le-a zis Iisus: Încă puțină vreme Lumina este cu voi. Umblați cât aveți Lumina ca să nu vă prindă întunericul. Căci cel ce umblă în întuneric nu știe unde merge.
\par 36 Cât aveți Lumina, credeți în Lumină, ca să fiți fii ai Luminii. Acestea le-a vorbit iisus și, plecând, S-a ascuns de ei.
\par 37 Și, deși a făcut atâtea minuni înaintea lor, ei tot nu credeau în El,
\par 38 Ca să se împlinească cuvântul proorocului Isaia, pe care l-a zis: "Doamne, cine a crezut în ceea ce a auzit de la noi? Și brațul Domnului cui s-a descoperit?"
\par 39 De aceea nu puteau să creadă, că iarăși a zis Isaia:
\par 40 "Au orbit ochii lor și a împietrit inima lor, ca să nu vadă cu ochii și să nu înțeleagă cu inima și ca nu cumva să se întoarcă și Eu să-i vindec"
\par 41 Acestea a zis Isaia, când a văzut slava Lui și a grăit despre El.
\par 42 Totuși și dintre căpetenii mulți au crezut în El, dar nu mărturiseau din pricina fariseilor, ca să nu fie izgoniți din sinagogă;
\par 43 Căci au iubit slava oamenilor mai mult decât slava lui Dumnezeu.
\par 44 Iar Iisus a strigat și a zis: Cel ce crede în Mine nu crede în Mine, ci în Cel ce M-a trimis pe Mine.
\par 45 Și cel ce Mă vede pe Mine vede pe Cel ce M-a trimis pe Mine.
\par 46 Eu, Lumină am venit în lume, ca tot cel ce crede în Mine să nu rămână întuneric.
\par 47 Și dacă aude cineva cuvintele Mele și nu le păzește, nu Eu îl judec; căci n-am venit ca să judec lumea ci ca să mântuiesc lumea.
\par 48 Cine Mă nesocotește pe Mine și nu primește cuvintele Mele are judecător ca să-l judece: cuvântul pe care l-am spus acela îl va judeca în ziua cea de apoi.
\par 49 Pentru că Eu n-am vorbit de la Mine, ci Tatăl care M-a trimis, Acesta Mi-a dat poruncă ce să spun și ce să vorbesc.
\par 50 Și știu că porunca Lui este viața veșnică. Deci cele ce vorbesc Eu, precum Mi-a spus Mie Tatăl, așa vorbesc.

\chapter{13}

\par 1 Iar înainte de sărbătoarea Paștilor, știind Iisus că a sosit ceasul Lui, ca să treacă din lumea aceasta la Tatăl, iubind pe ai Săi cei din lume, până la sfârșit i-a iubit.
\par 2 Și făcându-se Cină, și diavolul punând în inima lui Iuda fiul lui Simon Iscarioteanul, ca să-l vândă,
\par 3 Iisus, știind că Tatăl I-a dat Lui toate în mâini și că de la Dumnezeu a ieșit și la Dumnezeu merge,
\par 4 S-a sculat de la Cină, S-a dezbrăcat de haine și, luând un ștergar, S-a încins cu el.
\par 5 După aceea a turnat apă în vasul de spălat și a început să spele picioarele ucenicilor și să le șteargă cu ștergarul cu care era încins.
\par 6 A venit deci la Simon Petru. Acesta I-a zis: Doamne, oare Tu să-mi speli mie picioarele?
\par 7 A răspuns Iisus și i-a zis: Ceea ce fac Eu, tu nu știi acum, dar vei înțelege după aceasta.
\par 8 Petru I-a zis: Nu-mi vei spăla picioarele în veac. Iisus i-a răspuns: Dacă nu te voi spăla, nu ai parte de Mine.
\par 9 Zis-a Simon Petru Lui: Doamne, spală-mi nu numai picioarele mele, ci și mâinile și capul.
\par 10 Iisus i-a zis: Cel ce a făcut baie n-are nevoie să-i fie spălate decât picioarele, căci este curat tot. Și voi sunteți curați, însă nu toți.
\par 11 Că știa pe cel ce avea să-L vândă; de aceea a zis: Nu toți sunteți curați.
\par 12 După ce le-a spălat picioarele și Și-a luat hainele, S-a așezat iar la masă și le-a zis: Înțelegeți ce v-am făcut Eu?
\par 13 Voi Mă numiți pe Mine: Învățătorul și Domnul, și bine ziceți, căci sunt.
\par 14 Deci dacă Eu, Domnul și Învățătorul, v-am spălat vouă picioarele, și voi sunteți datori să ca să spălați picioarele unii altora;
\par 15 Că v-am dat vouă pildă, ca, precum v-am făcut Eu vouă, să faceți și voi.
\par 16 Adevărat, zic vouă: Nu este sluga mai mare decât stăpânul său, nici solul mai mare decât cel ce l-a trimis pe el.
\par 17 Când știți acestea, fericiți sunteți dacă le veți face.
\par 18 Nu zic despre voi toți; căci Eu știu pe cei pe care i-am ales. Ci ca să se împlinească Scriptura: "Cel ce mănâncă pâinea cu Mine a ridicat călcâiul împotriva Mea".
\par 19 De acum vă spun vouă, înainte de a fi aceasta, ca să credeți, când se va îndeplini, că Eu sunt.
\par 20 Adevărat, adevărat zic vouă: Cel care primește pe cel pe care-l voi trimite Eu, pe Mine Mă primește; iar cine Mă primește pe Mine primește pe Cel ce M-a trimis pe Mine.
\par 21 Iisus, zicând acestea, S-a tulburat cu duhul și a mărturisit și a zis: Adevărat, adevărat zic vouă că unul dintre voi Mă va vinde.
\par 22 Deci ucenicii se uitau unii la alții, nedumerindu-se despre cine vorbește.
\par 23 Iar la masă era rezemat la pieptul lui Iisus unul dintre ucenicii Lui, pe care-l iubea Iisus.
\par 24 Deci Simon Petru i-a făcut semn acestuia și i-a zis: Întreabă cine este despre care vorbește.
\par 25 Și căzând acela astfel la pieptul lui Iisus, I-a zis: Doamne, cine este?
\par 26 Iisus i-a răspuns: Acela este, căruia Eu, întingând bucățica de pâine, i-o voi da. Și întingând bucățica, a luat-o și a dat-o lui Iuda, fiul lui Simon Iscarioteanul.
\par 27 Și după îmbucătură a intrat satana în el. Iar Iisus i-a zis: Ceea ce faci, fă mai curând.
\par 28 Dar nimeni din cei care ședeau la masă n-a înțeles pentru ce i-a zis aceasta.
\par 29 Căci unii socoteau, deoarece Iuda avea punga, că lui îi zice Iisus: Cumpără cele de care avem de trebuință la sărbătoare, sau să dea ceva săracilor.
\par 30 Deci după ce a luat acela bucățica de pâine, a ieșit numaidecât. Și era noapte.
\par 31 Și când a ieșit el, Iisus a zis: Acum a fost preaslăvit Fiul Omului și Dumnezeu a fost preaslăvit întru El.
\par 32 Iar dacă Dumnezeu a fost preaslăvit întru El, și Dumnezeu Îl va preaslăvi întru El și îndată Îl va preaslăvi.
\par 33 Fiilor, încă puțin timp sunt cu voi. Voi Mă veți căuta, dar, după cum am spus iudeilor - că unde Mă duc Eu, voi nu puteți veni - vă spun vouă acum.
\par 34 Poruncă nouă dau vouă: Să vă iubiți unul pe altul. Precum Eu v-am iubit pe voi, așa și voi să vă iubiți unul pe altul.
\par 35 Întru aceasta vor cunoaște toți că sunteți ucenicii Mei, dacă veți avea dragoste unii față de alții.
\par 36 Doamne, L-a întrebat Simon-Petru, unde Te duci? Răspuns-a Iisus: Unde Mă duc Eu, tu nu poți să urmezi Mie acum, dar mai târziu Îmi vei urma.
\par 37 Zis-a Petru Lui: Doamne, de ce nu pot să urmez Ție acum? Sufletul meu îl voi da pentru Tine.
\par 38 Iisus i-a răspuns: Vei pune sufletul tău pentru Mine? Adevărat, adevărat zic ție că nu va cânta cocoșul, până ce nu te vei lepăda de Mine de trei ori!

\chapter{14}

\par 1 Să nu se tulbure inima voastră; credeți în Dumnezeu, credeți și în Mine.
\par 2 În casa Tatălui Meu multe locașuri sunt. Iar de nu, v-aș fi spus. Mă duc să vă gătesc loc.
\par 3 Și dacă Mă voi duce și vă voi găti loc, iarăși voi veni și vă voi lua la Mine, ca să fiți și voi unde sunt Eu.
\par 4 Și unde Mă duc Eu, voi știți și știți și calea.
\par 5 Toma i-a zis: Doamne, nu știm unde Te duci; și cum putem ști calea?
\par 6 Iisus i-a zis: Eu sunt Calea, Adevărul și Viața. Nimeni nu vine la Tatăl Meu decât prin Mine.
\par 7 Dacă M-ați fi cunoscut pe Mine, și pe Tatăl Meu L-ați fi cunoscut; dar de acum Îl cunoașteți pe El și L-ați și văzut.
\par 8 Filip I-a zis: Doamne, arată-ne nouă pe Tatăl și ne este de ajuns.
\par 9 Iisus i-a zis: De atâta vreme sunt cu voi și nu M-ai cunoscut, Filipe? Cel ce M-a văzut pe Mine a văzut pe Tatăl. Cum zici tu: Arată-ne pe Tatăl?
\par 10 Nu crezi tu că Eu sunt întru Tatăl și Tatăl este întru Mine? Cuvintele pe care vi le spun nu le vorbesc de la Mine, ci Tatăl - Care rămâne întru Mine - face lucrările Lui.
\par 11 Credeți Mie că Eu sunt întru Tatăl și Tatăl întru Mine, iar de nu, credeți-Mă pentru lucrările acestea.
\par 12 Adevărat, adevărat zic vouă: cel ce crede în Mine va face și el lucrările pe care le fac Eu și mai mari decât acestea va face, pentru că Eu Mă duc la Tatăl.
\par 13 Și orice veți cere întru numele Meu, aceea voi face, ca să fie slăvit Tatăl întru Fiul.
\par 14 Dacă veți cere ceva în numele Meu, Eu voi face.
\par 15 De Mă iubiți, păziți poruncile Mele.
\par 16 Și Eu voi ruga pe Tatăl și alt Mângâietor vă va da vouă ca să fie cu voi în veac,
\par 17 Duhul Adevărului, pe Care lumea nu poate să-L primească, pentru că nu-L vede, nici nu-L cunoaște; voi Îl cunoașteți, că rămâne la voi și în voi va fi!
\par 18 Nu vă voi lăsa orfani: voi veni la voi.
\par 19 Încă puțin timp și lumea nu Mă va mai vedea; voi însă Mă veți vedea, pentru că Eu sunt viu și voi veți fi vii.
\par 20 În ziua aceea veți cunoaște că Eu sunt întru Tatăl Meu și voi în Mine și Eu în voi.
\par 21 Cel ce are poruncile Mele și le păzește, acela este care Mă iubește; iar cel ce Mă iubește pe Mine va fi iubit de Tatăl Meu și-l voi iubi și Eu și Mă voi arăta lui.
\par 22 I-a zis Iuda, nu Iscarioteanul: Doamne, ce este că ai să Te arăți nouă, și nu lumii?
\par 23 Iisus a răspuns și i-a zis: Dacă Mă iubește cineva, va păzi cuvântul Meu, și Tatăl Meu îl va iubi, și vom veni la el și vom face locaș la el.
\par 24 Cel ce nu Mă iubește nu păzește cuvintele Mele. Dar cuvântul pe care îl auziți nu este al Meu, ci al Tatălui care M-a trimis.
\par 25 Acestea vi le-am spus, fiind cu voi;
\par 26 Dar Mângâietorul, Duhul Sfânt, pe Care-L va trimite Tatăl, în numele Meu, Acela vă va învăța toate și vă va aduce aminte despre toate cele ce v-am spus Eu.
\par 27 Pace vă las vouă, pacea Mea o dau vouă, nu precum dă lumea vă dau Eu. Să nu se tulbure inima voastră, nici să se înfricoșeze.
\par 28 Ați auzit că v-am spus: Mă duc și voi veni la voi. De M-ați iubi v-ați bucura că Mă duc la Tatăl, pentru că Tatăl este mai mare decât Mine.
\par 29 Și acum v-am spus acestea înainte de a se întâmpla, ca să credeți când se vor întâmpla.
\par 30 Nu voi mai vorbi multe cu voi, căci vine stăpânitorul acestei lumi și el nu are nimic în Mine;
\par 31 Dar ca să cunoască lumea că Eu iubesc pe Tatăl și precum Tatăl Mi-a poruncit așa fac. Sculați-vă, să mergem de aici.

\chapter{15}

\par 1 Eu sunt vița cea adevărată și Tatăl Meu este lucrătorul.
\par 2 Orice mlădiță care nu aduce roadă întru Mine, El o taie; și orice mlădiță care aduce roadă, El o curățește, ca mai multă roadă să aducă.
\par 3 Acum voi sunteți curați, pentru cuvântul pe care vi l-am spus.
\par 4 Rămâneți în Mine și Eu în voi. Precum mlădița nu poate să aducă roadă de la sine, dacă nu rămâne în viță, tot așa nici voi, dacă nu rămâneți în Mine.
\par 5 Eu sunt vița, voi sunteți mlădițele. Cel ce rămâne întru Mine și Eu în el, acela aduce roadă multă, căci fără Mine nu puteți face nimic.
\par 6 Dacă cineva nu rămâne în Mine se aruncă afară ca mlădița și se usucă; și le adună și le aruncă în foc și ard.
\par 7 Dacă rămâneți întru Mine și cuvintele Mele rămân în voi, cereți ceea ce voiți și se va da vouă.
\par 8 Întru aceasta a fost slăvit Tatăl Meu, ca să aduceți roadă multă și să vă faceți ucenici ai Mei.
\par 9 Precum M-a iubit pe Mine Tatăl, așa v-am iubit și Eu pe voi; rămâneți întru iubirea Mea.
\par 10 Dacă păziți poruncile Mele, veți rămâne întru iubirea Mea după cum și Eu am păzit poruncile Tatălui Meu și rămân întru iubirea Lui.
\par 11 Acestea vi le-am spus, ca bucuria Mea să fie în voi și ca bucuria voastră să fie deplină.
\par 12 Aceasta este porunca Mea: să vă iubiți unul pe altul, precum v-am iubit Eu.
\par 13 Mai mare dragoste decât aceasta nimeni nu are, ca sufletul lui să și-l pună pentru prietenii săi.
\par 14 Voi sunteți prietenii Mei, dacă faceți ceea ce vă poruncesc.
\par 15 De acum nu vă mai zic slugi, că sluga nu știe ce face stăpânul său, ci v-am numit pe voi prieteni, pentru că toate câte am auzit de la Tatăl Meu vi le-am făcut cunoscute.
\par 16 Nu voi M-ați ales pe Mine, ci Eu v-am ales pe voi și v-am rânduit să mergeți și roadă să aduceți, și roada voastră să rămână, ca Tatăl să vă dea orice-I veți cere în numele Meu.
\par 17 Aceasta vă poruncesc: să vă iubiți unul pe altul.
\par 18 Dacă vă urăște pe voi lumea, să știți că pe Mine mai înainte decât pe voi M-a urât.
\par 19 Dacă ați fi din lume, lumea ar iubi ce este al său; dar pentru că nu sunteți din lume, ci Eu v-am ales pe voi din lume, de aceea lumea vă urăște.
\par 20 Aduceți-vă aminte de cuvântul pe care vi l-am spus: Nu este sluga mai mare decât stăpânul său. Dacă M-au prigonit pe Mine, și pe voi vă vor prigoni; dacă au păzit cuvântul Meu, și pe al vostru îl vor păzi.
\par 21 Iar toate acestea le vor face vouă din cauza numelui Meu, fiindcă ei nu cunosc pe Cel ce M-a trimis.
\par 22 De n-aș fi venit și nu le-aș fi vorbit, păcat nu ar avea; dar acum n-au cuvânt de dezvinovățire pentru păcatul lor.
\par 23 Cel ce Mă urăște pe Mine, urăște și pe Tatăl Meu.
\par 24 De nu aș fi făcut între ei lucruri pe care nimeni altul nu le-a făcut păcat nu ar avea; dar acum M-au și văzut și M-au urât și pe Mine și pe Tatăl Meu.
\par 25 Dar (aceasta), ca să se împlinească cuvântul cel scris în Legea lor: "M-au urât pe nedrept".
\par 26 Iar când va veni Mângâietorul, pe Care Eu Îl voi trimite vouă de la Tatăl, Duhul Adevărului, Care de la Tatăl purcede, Acela va mărturisi despre Mine.
\par 27 Și voi mărturisiți, pentru că de la început sunteți cu Mine.

\chapter{16}

\par 1 Acestea vi le-am spus, ca să nu vă smintiți.
\par 2 Vă vor scoate pe voi din sinagogi; dar vine ceasul când tot cel ce vă va ucide să creadă că aduce închinare lui Dumnezeu.
\par 3 Și acestea le vor face, pentru că n-au cunoscut nici pe Tatăl, nici pe Mine.
\par 4 Iar acestea vi le-am spus, ca să vă aduceți aminte de ele, când va veni ceasul lor, că Eu vi le-am spus. Și acestea nu vi le-am spus de la început, fiindcă eram cu voi.
\par 5 Dar acum Mă duc la Cel ce M-a trimis și nimeni dintre voi nu întreabă: Unde Te duci?
\par 6 Ci, fiindcă v-am spus acestea, întristarea a umplut inima voastră.
\par 7 Dar Eu vă spun adevărul: Vă este de folos ca să mă duc Eu. Căci dacă nu Mă voi duce, Mângâietorul nu va veni la voi, iar dacă Mă voi duce, Îl voi trimite la voi.
\par 8 Și El, venind, va vădi lumea de păcat și de dreptate și de judecată.
\par 9 De păcat, pentru că ei nu cred în Mine;
\par 10 De dreptate, pentru că Mă duc la Tatăl Meu și nu Mă veți mai vedea;
\par 11 Și de judecată, pentru că stăpânitorul acestei lumi a fost judecat.
\par 12 Încă multe am a vă spune, dar acum nu puteți să le purtați.
\par 13 Iar când va veni Acela, Duhul Adevărului, vă va călăuzi la tot adevărul; căci nu va vorbi de la Sine, ci toate câte va auzi va vorbi și cele viitoare vă va vesti.
\par 14 Acela Mă va slăvi, pentru că din al Meu va lua și vă va vesti.
\par 15 Toate câte are Tatăl ale Mele sunt; de aceea am zis că din al Meu ia și vă vestește vouă.
\par 16 Puțin și nu Mă veți mai vedea, și iarăși puțin și Mă veți vedea, pentru că Eu Mă duc la Tatăl.
\par 17 Deci unii dintre ucenicii Lui ziceau între ei: Ce este aceasta ce ne spune: Puțin și nu Mă veți mai vedea, și iarăși puțin și Mă veți vedea, și că Mă duc la Tatăl?
\par 18 Deci ziceau: Ce este aceasta ce zice: Puțin? Nu știm ce zice.
\par 19 Și a cunoscut Iisus că voiau să-L întrebe și le-a zis: Despre aceasta vă întrebați între voi, că am zis: Puțin și nu Mă veți mai vedea și iarăși puțin și Mă veți vedea?
\par 20 Adevărat, adevărat zic vouă că voi veți plânge și vă veți tângui, iar lumea se va bucura. Voi vă veți întrista, dar întristarea voastră se va preface în bucurie.
\par 21 Femeia, când e să nască, se întristează, fiindcă a sosit ceasul ei; dar după ce a născut copilul, nu-și mai aduce aminte de durere, pentru bucuria că s-a născut om în lume.
\par 22 Deci și voi acum sunteți triști, dar iarăși vă voi vedea și se va bucura inima voastră și bucuria voastră nimeni nu o va lua de la voi.
\par 23 Și în ziua aceea nu Mă veți întreba nimic. Adevărat, adevărat zic vouă: Orice veți cere de la Tatăl în numele Meu El vă va da.
\par 24 Până acum n-ați cerut nimic în numele Meu; cereți și veți primi, ca bucuria voastră să fie deplină.
\par 25 Acestea vi le-am spus în pilde, dar vine ceasul când nu vă voi mai vorbi în pilde, ci pe față vă voi vesti despre Tatăl.
\par 26 În ziua aceea veți cere în numele Meu; și nu vă zic că voi ruga pe Tatăl pentru voi,
\par 27 Căci Însuși Tatăl vă iubește pe voi, fiindcă voi M-ați iubit pe Mine și ați crezut că de la Dumnezeu am ieșit.
\par 28 Ieșit-am de la Tatăl și am venit în lume; iarăși las lumea și Mă duc la Tatăl.
\par 29 Au zis ucenicii Săi: Iată acum vorbești pe față și nu spui nici o pildă.
\par 30 Acum știm că Tu știi toate și nu ai nevoie ca să Te întrebe cineva. De aceea credem că ai ieșit de la Dumnezeu.
\par 31 Iisus le-a răspuns: Acum credeți?
\par 32 Iată vine ceasul, și a și venit, ca să vă risipiți fiecare la ale sale și pe Mine să Mă lăsați singur. Dar nu sunt singur, pentru că Tatăl este cu Mine.
\par 33 Acestea vi le-am grăit, ca întru Mine pace să aveți. În lume necazuri veți avea; dar îndrăzniți. Eu am biruit lumea.

\chapter{17}

\par 1 Acestea a vorbit Iisus și, ridicând ochii Săi la cer, a zis: Părinte, a venit ceasul! Preaslăvește pe Fiul Tău, ca și Fiul să Te preaslăvească.
\par 2 Precum I-ai dat stăpânire peste tot trupul, ca să dea viață veșnică tuturor acelora pe care Tu i-ai dat Lui.
\par 3 Și aceasta este viața veșnică: Să Te cunoască pe Tine, singurul Dumnezeu adevărat, și pe Iisus Hristos pe Care L-ai trimis.
\par 4 Eu Te-am preaslăvit pe Tine pe pământ; lucrul pe care Mi l-ai dat să-l fac, l-am săvârșit.
\par 5 Și acum, preaslăvește-Mă Tu, Părinte, la Tine Însuți, cu slava pe care am avut-o la Tine, mai înainte de a fi lumea.
\par 6 Arătat-am numele Tău oamenilor pe care Mi i-ai dat Mie din lume. Ai Tăi erau și Mie Mi i-ai dat și cuvântul Tău l-au păzit.
\par 7 Acum au cunoscut că toate câte Mi-ai dat sunt de la Tine;
\par 8 Pentru că cuvintele pe care Mi le-ai dat le-am dat lor, iar ei le-au primit și au cunoscut cu adevărat că de la Tine am ieșit, și au crezut că Tu M-ai trimis.
\par 9 Eu pentru aceștia Mă rog; nu pentru lume Mă rog, ci pentru cei pe care Mi i-ai dat, că ai Tăi sunt.
\par 10 Și toate ale Mele sunt ale Tale, și ale Tale sunt ale Mele și M-am preaslăvit întru ei.
\par 11 Și Eu nu mai sunt în lume, iar ei în lume sunt și Eu vin la Tine. Părinte Sfinte, păzește-i în numele Tău, în care Mi i-ai dat, ca să fie una precum suntem și Noi.
\par 12 Când eram cu ei în lume, Eu îi păzeam în numele Tău, pe cei ce Mi i-ai dat; și i-am păzit și n-a pierit nici unul dintre ei, decât fiul pierzării, ca să se împlinească Scriptura.
\par 13 Iar acum, vin la Tine și acestea le grăiesc în lume, ca să fie deplină bucuria Mea în ei.
\par 14 Eu le-am dat cuvântul Tău, și lumea i-a urât, pentru că nu sunt din lume, precum Eu nu sunt din lume.
\par 15 Nu Mă rog ca să-i iei din lume, ci ca să-i păzești pe ei de cel viclean.
\par 16 Ei nu sunt din lume, precum nici Eu nu sunt din lume.
\par 17 Sfințește-i pe ei întru adevărul Tău; cuvântul Tău este adevărul.
\par 18 Precum M-ai trimis pe Mine în lume, și Eu i-am trimis pe ei în lume.
\par 19 Pentru ei Eu Mă sfințesc pe Mine Însumi, ca și ei să fie sfințiți întru adevăr.
\par 20 Dar nu numai pentru aceștia Mă rog, ci și pentru cei ce vor crede în Mine, prin cuvântul lor,
\par 21 Ca toți să fie una, după cum Tu, Părinte, întru Mine și Eu întru Tine, așa și aceștia în Noi să fie una, ca lumea să creadă că Tu M-ai trimis.
\par 22 Și slava pe care Tu Mi-ai dat-o, le-am dat-o lor, ca să fie una, precum Noi una suntem:
\par 23 Eu întru ei și Tu întru Mine, ca ei să fie desăvârșiți întru unime, și să cunoască lumea că Tu M-ai trimis și că i-ai iubit pe ei, precum M-ai iubit pe Mine.
\par 24 Părinte, voiesc ca, unde sunt Eu, să fie împreună cu Mine și aceia pr care Mi i-ai dat, ca să vadă slava mea pe care Mi-ai dat-o, pentru că Tu M-ai iubit pe Mine mai înainte de întemeierea lumii.
\par 25 Părinte drepte, lumea pe Tine nu te-a cunoscut, dar Eu Te-am cunoscut, și aceștia au cunoscut că Tu M-ai trimis.
\par 26 Și le-am făcut cunoscut numele Tău și-l voi face cunoscut, ca iubirea cu care M-ai iubit Tu să fie în ei și Eu în ei.

\chapter{18}

\par 1 Zicând acestea, Iisus a ieșit cu ucenicii Lui dincolo de pârâul Cedrilor, unde era o grădină, în care a intrat El și ucenicii Săi.
\par 2 Iar Iuda vânzătorul cunoștea acest loc, pentru că adesea Iisus și ucenicii Săi se adunau acolo.
\par 3 Deci Iuda, luând oaste și slujitori, de la arhierei și farisei, a venit acolo cu felinare și cu făclii și cu arme.
\par 4 Iar Iisus, știind toate cele ce erau să vină asupra Lui, a ieșit și le-a zis: Pe cine căutați?
\par 5 Răspuns-au Lui: Pe Iisus Nazarineanul. El le-a zis: Eu sunt. Iar Iuda vânzătorul era și el cu ei.
\par 6 Atunci când le-a spus: Eu sunt, ei s-au dat înapoi și au căzut la pământ.
\par 7 Și iarăși i-a întrebat: Pe cine căutați? Iar ei au zis: Pe Iisus Nazarineanul.
\par 8 Răspuns-a Iisus: V-am spus că Eu sunt. Deci, dacă Mă căutați pe Mine, lăsați pe aceștia să se ducă;
\par 9 Ca să se împlinească cuvântul pe care l-a spus: Dintre cei pe care Mi i-ai dat, n-am pierdut pe nici unul.
\par 10 Dar Simon-Petru, având sabie, a scos-o și a lovit pe sluga arhiereului și i-a tăiat urechea dreaptă; iar numele slugii era Malhus.
\par 11 Deci a zis Iisus lui Petru: Pune sabia în teacă. Nu voi bea, oare, paharul pe care Mi l-a dat Tatăl?
\par 12 Deci ostașii și comandantul și slujitorii iudeilor au prins pe Iisus și L-au legat.
\par 13 Și L-au dus întâi la Anna, căci era socrul lui Caiafa, care era arhiereu al anului aceluia.
\par 14 Și Caiafa era cel ce sfătuise pe iudei că este de folos să moară un om pentru popor.
\par 15 Și Simon-Petru și un alt ucenic mergeau după Iisus. Iar ucenicul acela era cunoscut arhiereului și a intrat împreună cu Iisus în curtea arhiereului;
\par 16 Iar Petru a stat la poartă, afară. Deci a ieșit celălalt ucenic, care era cunoscut arhiereului, și a vorbit cu portăreasa și a băgat pe Petru înăuntru.
\par 17 Deci slujnica portăreasă i-a zis lui Petru: Nu cumva ești și tu dintre ucenicii Omului acestuia? Acela a zis: Nu sunt.
\par 18 Iar slugile și slujitorii făcuseră foc, și stăteau și se încălzeau, că era frig, și era cu ei și Petru, stând și încălzindu-se.
\par 19 Deci arhiereul L-a întrebat pe Iisus despre ucenicii Lui și despre învățătura Lui.
\par 20 Iisus i-a răspuns: Eu am vorbit pe față lumii; Eu am învățat întotdeauna în sinagogă și în templu, unde se adună toți iudeii și nimic nu am vorbit în ascuns.
\par 21 De ce Mă întrebi pe Mine? Întreabă pe cei ce au auzit ce le-am vorbit. Iată aceștia știu ce am spus Eu.
\par 22 Și zicând El acestea, unul din slujitorii, care era de față, I-a dat lui Iisus o palmă, zicând: Așa răspunzi Tu arhiereului?
\par 23 Iisus i-a răspuns: Dacă am vorbit rău, dovedește ce este rău, iar dacă am vorbit bine, de ce Mă bați?
\par 24 Deci Anna L-a trimis legat la Caiafa arhiereul.
\par 25 Iar Simon-Petru stătea și se încălzea. Deci i-au zis: Nu cumva ești și tu dintre ucenicii Lui? El s-a lepădat și a zis: Nu sunt.
\par 26 Una din slugile arhiereului, care era rudă cu cel căruia Petru îi tăiase urechea, a zis: Nu te-am văzut eu pe tine, în grădină, cu El?
\par 27 Și iarăși s-a lepădat Petru și îndată a cântat cocoșul.
\par 28 Deci L-au adus pe Iisus de la Caiafa la pretoriu; și era dimineață. Și ei n-au intrat în pretoriu, ca să nu se spurce, ci să mănânce Paștile.
\par 29 Deci Pilat a ieșit la ei, afară, și le-a zis: Ce învinuire aduceți Omului Acestuia?
\par 30 Ei au răspuns și i-au zis: Dacă Acesta n-ar fi răufăcător, nu ți L-am fi dat ție.
\par 31 Deci le-a zis Pilat: Luați-L voi și judecați-L după legea voastră. Iudeii însă i-au răspuns: Nouă nu ne este îngăduit să omorâm pe nimeni;
\par 32 Ca să se împlinească cuvântul lui Iisus, pe care îl spusese, însemnând cu ce moarte avea să moară.
\par 33 Deci Pilat a intrat iarăși în pretoriu și a chemat pe Iisus și I-a zis: Tu ești regele iudeilor?
\par 34 Răspuns-a Iisus: De la tine însuți zici aceasta, sau alții ți-au spus-o despre Mine?
\par 35 Pilat a răspuns: Nu cumva sunt iudeu eu? Poporul Tău și arhiereii Te-au predat mie. Ce ai făcut?
\par 36 Iisus a răspuns: Împărăția Mea nu este din lumea aceasta. Dacă împărăția Mea ar fi din lumea aceasta, slujitorii Mei s-ar fi luptat ca să nu fiu predat iudeilor. Dar acum împărăția Mea nu este de aici.
\par 37 Deci i-a zis Pilat: Așadar ești Tu împărat? Răspuns-a Iisus: Tu zici că Eu sunt împărat. Eu spre aceasta M-am născut și pentru aceasta am venit în lume, ca să dau mărturie pentru adevăr; oricine este din adevăr ascultă glasul Meu.
\par 38 Pilat I-a zis: Ce este adevărul? Și zicând aceasta, a ieșit iarăși la iudei și le-a zis: Eu nu găsesc în El nici o vină;
\par 39 Dar este la voi obiceiul ca la Paști să vă eliberez pe unul. Voiți deci să vă eliberez pe regele iudeilor?
\par 40 Deci au strigat iarăși, zicând: Nu pe Acesta, ci pe Baraba. Iar Baraba era tâlhar.

\chapter{19}

\par 1 Deci atunci Pilat a luat pe Iisus și L-a biciuit.
\par 2 Și ostașii, împletind cunună din spini, I-au pus-o pe cap și L-au îmbrăcat cu o mantie purpurie.
\par 3 Și veneau către El și ziceau: Bucură-te, regele iudeilor! Și-I dădeau palme.
\par 4 Și Pilat a ieșit iarăși afară și le-a zis: Iată vi-L aduc pe El afară, ca să știți că nu găsesc în El nici o vină.
\par 5 Deci a ieșit Iisus afară, purtând cununa de spini și mantia purpurie. Și le-a zis Pilat: Iată Omul!
\par 6 Când L-au văzut deci arhiereii și slujitorii au strigat, zicând: Răstignește-L! Răstignește-L! Zis-a lor Pilat: Luați-L voi și răstigniți-L, căci eu nu-I găsesc nici o vină.
\par 7 Iudeii i-au răspuns: Noi avem lege și după legea noastră El trebuie să moară, că S-a făcut pe Sine Fiu al lui Dumnezeu.
\par 8 Deci, când a auzit Pilat acest cuvânt, mai mult s-a temut.
\par 9 Și a intrat iarăși în pretoriu și I-a zis lui Iisus: De unde ești Tu? Iar Iisus nu i-a dat nici un răspuns.
\par 10 Deci Pilat i-a zis: Mie nu-mi vorbești? Nu știi că am putere să Te eliberez și putere am să Te răstignesc?
\par 11 Iisus a răspuns: N-ai avea nici o putere asupra Mea, dacă nu ți-ar fi fost dat ție de sus. De aceea cel ce M-a predat ție mai mare păcat are.
\par 12 Pentru aceasta, Pilat căuta să-L elibereze; iar iudeii strigau zicând: Dacă Îl eliberezi pe Acesta, nu ești prieten al Cezarului. Oricine se face pe sine împărat este împotriva Cezarului.
\par 13 Deci Pilat, auzind cuvintele acestea, L-a dus afară pe Iisus și a șezut pe scaunul de judecată, în locul numit pardosit cu pietre, iar evreiește Gabbata.
\par 14 Și era Vinerea Paștilor, cam la al șaselea ceas, și a zis Pilat iudeilor: Iată Împăratul vostru.
\par 15 Deci au strigat aceia: Ia-L! Ia-L! Răstignește-L! Pilat le-a zis: Să răstignesc pe Împăratul vostru? Arhiereii au răspuns: Nu avem împărat decât pe Cezarul.
\par 16 Atunci L-a predat lor ca să fie răstignit. Și ei au luat pe Iisus și L-au dus ca să fie răstignit.
\par 17 Și ducându-Și crucea, a ieșit la locul ce se cheamă al Căpățânii, care evreiește se zice Golgota,
\par 18 Unde L-au răstignit, și împreună cu El pe alți doi, de o parte și de alta, iar în mijloc pe Iisus.
\par 19 Iar Pilat a scris și titlu și l-a pus deasupra Crucii. Și era scris: Iisus Nazarineanul, Împăratul iudeilor!
\par 20 Deci mulți dintre iudei au citit acest titlu, căci locul unde a fost răstignit Iisus era aproape de cetate. Și era scris: evreiește, latinește și grecește.
\par 21 Deci arhiereii iudeilor au zis lui Pilat: Nu scrie: Împăratul iudeilor, ci că Acela a zis: Eu sunt Împăratul iudeilor.
\par 22 Pilat a răspuns: Ce am scris, am scris.
\par 23 După ce au răstignit pe Iisus, ostașii au luat hainele Lui și le-au făcut patru părți, fiecărui ostaș câte o parte, și cămașa. Dar cămașa era fără cusătură, de sus țesută în întregime.
\par 24 Deci au zis unii către alții: Să n-o sfâșiem, ci să aruncăm sorții pentru ea, a cui să fie; ca să se împlinească Scriptura care zice: "Împărțit-au hainele Mele loruși, și pentru cămașa Mea au aruncat sorții". Așadar ostașii acestea au făcut.
\par 25 Și stăteau, lângă crucea lui Iisus, mama Lui și sora mamei Lui, Maria lui Cleopa, și Maria Magdalena.
\par 26 Deci Iisus, văzând pe mama Sa și pe ucenicul pe care Îl iubea stând alături, a zis mamei Sale: Femeie, iată fiul tău!
\par 27 Apoi a zis ucenicului: Iată mama ta! Și din ceasul acela ucenicul a luat-o la sine.
\par 28 După aceea, știind Iisus că toate s-au săvârșit acum, ca să se împlinească Scriptura, a zis: Mi-e sete.
\par 29 Și era acolo un vas plin cu oțet; iar cei care Îl loviseră, punând în vârful unei trestii de isop un burete înmuiat în oțet, l-au dus la gura Lui.
\par 30 Deci după ce a luat oțetul, Iisus a zis: Săvârșitu-s-a. Și plecându-Și capul, Și-a dat duhul.
\par 31 Deci iudeii, fiindcă era vineri, ca să nu rămână trupurile sâmbăta pe cruce, căci era mare ziua sâmbetei aceleia, au rugat pe Pilat să le zdrobească fluierele picioarelor și să-i ridice.
\par 32 Deci au venit ostașii și au zdrobit fluierele celui dintâi și ale celuilalt, care era răstignit împreună cu el.
\par 33 Dar venind la Iisus, dacă au văzut că deja murise, nu I-au zdrobit fluierele.
\par 34 Ci unul din ostași cu sulița a împuns coasta Lui și îndată a ieșit sânge și apă.
\par 35 Și cel ce a văzut a mărturisit și mărturia lui e adevărată; și acela știe că spune adevărul, ca și voi să credeți.
\par 36 Căci s-au făcut acestea, ca să se împlinească Scriptura: "Nu I se va zdrobi nici un os".
\par 37 Și iarăși altă Scriptură zice: "Vor privi la Acela pe care L-au împuns".
\par 38 După acestea Iosif din Arimateea, fiind ucenic al lui Iisus, dar într-ascuns, de frica iudeilor, a rugat pe Pilat ca să ridice trupul lui Iisus. Și Pilat i-a dat voie. Deci a venit și a ridicat trupul Lui.
\par 39 Și a venit și Nicodim, cel care venise la El mai înainte noaptea, aducând ca la o sută de litre de amestec de smirnă și aloe.
\par 40 Au luat deci trupul lui Iisus și l-au înfășurat în giulgiu cu miresme, precum este obiceiul de înmormântare la iudei.
\par 41 Iar în locul unde a fost răstignit era o grădină, și în grădină un mormânt nou, în care nu mai fusese nimeni îngropat.
\par 42 Deci, din pricina vinerii iudeilor, acolo L-au pus pe Iisus, pentru că mormântul era aproape.

\chapter{20}

\par 1 Iar în ziua întâia a săptămânii (duminica), Maria Magdalena a venit la mormânt dis-de-dimineață, fiind încă întuneric, și a văzut piatra ridicată de pe mormânt.
\par 2 Deci a alergat și a venit la Simon-Petru și la celălalt ucenic pe care-l iubea Iisus, și le-a zis: Au luat pe Domnul din mormânt și noi nu știm unde L-au pus.
\par 3 Deci a ieșit Petru și celălalt ucenic și veneau la mormânt.
\par 4 Și cei doi alergau împreună, dar celălalt ucenic, alergând înainte, mai repede decât Petru, a sosit cel dintâi la mormânt.
\par 5 Și, aplecându-se, a văzut giulgiurile puse jos, dar n-a intrat.
\par 6 A sosit și Simon-Petru, urmând după el, și a intrat în mormânt și a văzut giulgiurile puse jos,
\par 7 Iar mahrama, care fusese pe capul Lui, nu era pusă împreună cu giulgiurile, ci înfășurată, la o parte, într-un loc.
\par 8 Atunci a intrat și celălalt ucenic care sosise întâi la mormânt, și a văzut și a crezut.
\par 9 Căci încă nu știau Scriptura, că Iisus trebuia să învieze din morți.
\par 10 Și s-au dus ucenicii iarăși la ai lor.
\par 11 Iar Maria stătea afară lângă mormânt plângând. Și pe când plângea, s-a aplecat spre mormânt.
\par 12 Și a văzut doi îngeri în veșminte albe șezând, unul către cap și altul către picioare, unde zăcuse trupul lui Iisus.
\par 13 Și aceia i-au zis: Femeie, de ce plângi? Pe cine cauți? Ea le-a zis: Că au luat pe Domnul meu și nu știu unde L-au pus.
\par 14 Zicând acestea, ea s-a întors cu fața și a văzut pe Iisus stând, dar nu știa că este Iisus.
\par 15 Zis-a ei Iisus: Femeie, de ce plângi? Pe cine cauți? Ea, crezând că este grădinarul, I-a zis: Doamne, dacă Tu L-ai luat, spune-mi unde L-ai pus și eu Îl voi ridica.
\par 16 Iisus i-a zis: Maria! Întorcându-se, aceea I-a zis evreiește: Rabuni! (adică, Învățătorule)
\par 17 Iisus i-a zis: Nu te atinge de Mine, căci încă nu M-am suit la Tatăl Meu. Mergi la frații Mei și le spune: Mă sui la Tatăl Meu și Tatăl vostru și la Dumnezeul Meu și Dumnezeul vostru.
\par 18 Și a venit Maria Magdalena vestind ucenicilor că a văzut pe Domnul și acestea i-a zis ei.
\par 19 Și fiind seară, în ziua aceea, întâia a săptămânii (duminica), și ușile fiind încuiate, unde erau adunați ucenicii de frica iudeilor, a venit Iisus și a stat în mijloc și le-a zis: Pace vouă!
\par 20 Și zicând acestea, le-a arătat mâinile și coasta Sa. Deci s-au bucurat ucenicii, văzând pe Domnul.
\par 21 Și Iisus le-a zis iarăși: Pace vouă! Precum M-a trimis pe Mine Tatăl, vă trimit și Eu pe voi.
\par 22 Și zicând acestea, a suflat asupra lor și le-a zis: Luați Duh Sfânt;
\par 23 Cărora veți ierta păcatele, le vor fi iertate și cărora le veți ține, vor fi ținute.
\par 24 Iar Toma, unul din cei doisprezece, cel numit Geamănul, nu era cu ei când a venit Iisus.
\par 25 Deci au zis lui ceilalți ucenici: Am văzut pe Domnul! Dar el le-a zis: Dacă nu voi vedea, în mâinile Lui, semnul cuielor, și dacă nu voi pune degetul meu în semnul cuielor, și dacă nu voi pune mâna mea în coasta Lui, nu voi crede.
\par 26 Și după opt zile, ucenicii Lui erau iarăși înăuntru, și Toma, împreună cu ei. Și a venit Iisus, ușile fiind încuiate, și a stat în mijloc și a zis: Pace vouă!
\par 27 Apoi a zis lui Toma: Adu degetul tău încoace și vezi mâinile Mele și adu mâna ta și o pune în coasta Mea și nu fi necredincios ci credincios.
\par 28 A răspuns Toma și I-a zis: Domnul meu și Dumnezeul meu!
\par 29 Iisus I-a zis: Pentru că M-ai văzut ai crezut. Fericiți cei ce n-au văzut și au crezut!
\par 30 Deci și alte multe minuni a făcut Iisus înaintea ucenicilor Săi, care nu sunt scrise în cartea aceasta.
\par 31 Iar acestea s-au scris, ca să credeți că Iisus este Hristosul, Fiul lui Dumnezeu, și, crezând, să aveți viață în numele Lui.

\chapter{21}

\par 1 După acestea, Iisus S-a arătat iarăși ucenicilor la Marea Tiberiadei, și S-a arătat așa:
\par 2 Erau împreună Simon-Petru și Toma, cel numit Geamănul, și Natanael, cel din Cana Galileii, și fiii lui Zevedeu și alți doi din ucenicii Lui.
\par 3 Simon-Petru le-a zis: Mă duc să pescuiesc. Și i-au zis ei: Mergem și noi cu tine. Și au ieșit și s-au suit în corabie, și în noaptea aceea n-au prins nimic.
\par 4 Iar făcându-se dimineață, Iisus a stat la țărm; dar ucenicii n-au știut că este Iisus.
\par 5 Deci le-a zis Iisus: Fiilor, nu cumva aveți ceva de mâncare? Ei I-au răspuns: Nu.
\par 6 Iar El le-a zis: Aruncați mreaja în partea dreaptă a corăbiei și veți afla. Deci au aruncat-o și nu mai puteau s-o tragă de mulțimea peștilor.
\par 7 Și a zis lui Petru ucenicul acela pe care-l iubea Iisus: Domnul este! Deci Simon-Petru, auzind că este Domnul, și-a încins haina, căci era dezbrăcat, și s-a aruncat în apă.
\par 8 Și ceilalți ucenici au venit cu corabia, căci nu erau departe de țărm, ci la două sute de coți, trăgând mreaja cu pești.
\par 9 Deci, când au ieșit la țărm, au văzut jar pus jos și pește pus deasupra, și pâine.
\par 10 Iisus le-a zis: Aduceți din peștele pe care l-ați prins acum.
\par 11 Simon-Petru s-a suit în corabie și a tras mreaja la țărm, plină de pești mari: o sută cincizeci și trei, și, deși erau atâția, nu s-a rupt mreaja.
\par 12 Iisus le-a zis: Veniți de prânziți. Și nici unul din ucenici nu îndrăznea să-L întrebe: Cine ești Tu?, știind că este Domnul.
\par 13 Deci a venit Iisus și a luat pâinea și le-a dat lor, și de asemenea și peștele.
\par 14 Aceasta este, acum, a treia oară când Iisus S-a arătat ucenicilor, după ce S-a sculat din morți.
\par 15 Deci după ce au prânzit, a zis Iisus lui Simon-Petru: Simone, fiul lui Iona, Mă iubești tu mai mult decât aceștia? El I-a răspuns: Da, Doamne, Tu știi că Te iubesc. Zis-a lui: Paște mielușeii Mei.
\par 16 Iisus i-a zis iarăși, a doua oară: Simone, fiul lui Iona, Mă iubești? El I-a zis: Da, Doamne, Tu știi că Te iubesc. Zis-a Iisus lui: Păstorește oile Mele.
\par 17 Iisus i-a zis a treia oară: Simone, fiul lui Iona, Mă iubești? Petru s-a întristat, că i-a zis a treia oară: Mă iubești? și I-a zis: Doamne, Tu știi toate. Tu știi că Te iubesc. Iisus i-a zis: Paște oile Mele.
\par 18 Adevărat, adevărat zic ție: Dacă erai mai tânăr, te încingeai singur și umblai unde voiai; dar când vei îmbătrâni, vei întinde mâinile tale și altul te va încinge și te va duce unde nu voiești.
\par 19 Iar aceasta a zis-o, însemnând cu ce fel de moarte va preaslăvi pe Dumnezeu. Și spunând aceasta, i-a zis: Urmează Mie.
\par 20 Dar întorcându-se, Petru a văzut venind după el pe ucenicul pe care-l iubea Iisus, acela care la Cină s-a rezemat de pieptul Lui și I-a zis: Doamne, cine este cel ce Te va vinde?
\par 21 Pe acesta deci, văzându-l, Petru a zis lui Iisus: Doamne, dar cu acesta ce se va întâmpla?
\par 22 Zis-a Iisus lui: Dacă voiesc ca acesta să rămână până voi veni, ce ai tu? Tu urmează Mie.
\par 23 De aceea a ieșit cuvântul acesta între frați, că ucenicul acela nu va muri; dar Iisus nu i-a spus că nu va muri ci: dacă voiesc ca acesta să rămână până voi veni, ce ai tu?
\par 24 Acesta este ucenicul care mărturisește despre acestea și care a scris acestea, și știm că mărturia lui e adevărată.
\par 25 Dar sunt și alte multe lucruri pe care le-a făcut Iisus și care, dacă s-ar fi scris cu de-amănuntul, cred că lumea aceasta n-ar cuprinde cărțile ce s-ar fi scris. Amin.


\end{document}