\begin{document}

\title{2 Cronici}


\chapter{1}

\par 1 În vremea aceea Solomon, fiul lui David, se întărise în domnie și Domnul Dumnezeul lui era cu el, înălțându-l foarte sus.
\par 2 Solomon a poruncit să se adune tot Israelul: căpeteniile peste mii, căpeteniile peste sute, judecătorii și toți cei ce cârmuiau în Israel până la capii de familie.
\par 3 Apoi s-a dus Solomon și toată adunarea pe înălțimea cea din Ghibeon, căci acolo era cortul cel dumnezeiesc al adunării pe care-l făcuse Moise, robul lui Dumnezeu, în pustiu.
\par 4 Chivotul Domnului îl adusese David din Chiriat-Iearim la locul pe care-l pregătise pentru el David, în Ierusalim, unde-i făcuse un cort nou.
\par 5 Iar jertfelnicul cel de aramă, pe care-l făcuse Bețaleel, fiul lui Uri, fiul lui Hur, rămăsese acolo, înaintea cortului celui vechi al Domnului. Pe acesta îl cerceta Solomon cu adunarea.
\par 6 Și aici, înaintea feței Domnului, pe jertfelnicul cel de aramă, care era înaintea cortului adunării, a adus Solomon o mie de arderi de tot.
\par 7 În noaptea aceea S-a arătat Dumnezeu lui Solomon și i-a zis: "Cere ce dorești să-ți dau?"
\par 8 Iar Solomon a zis către Dumnezeu: "Tu ai făcut cu David, tatăl meu, milă mare și m-ai pus pe mine rege în locul lui.
\par 9 Să se împlinească dar, Doamne Dumnezeule, cuvântul Tău cel către David, tatăl meu! De vreme ce m-ai pus pe mine peste un popor mult la număr, ca pulberea pământului,
\par 10 Apoi dă-mi mie acum înțelepciune și știință, ca să pricep cum să cârmuiesc pe poporul acesta, căci cine poate să cârmuiască pe acest mare popor al Tău?"
\par 11 Atunci Dumnezeu a zis către Solomon: "Pentru că tu ai avut la inimă asemenea lucru și n-ai cerut bogăție, averi și slavă, nici sufletele neprietenilor tăi; n-ai cerut de asemenea nici zile multe, ci ai cerut înțelepciune și știință, ca să judeci pe poporul Meu peste care te-am pus rege,
\par 12 De aceea ți se dă înțelepciune și știință, iar bogăție, averi și slavă îți voi da atâta, câtă n-au mai avut regii cei dinainte de tine și nici după tine nu vor mai avea".
\par 13 Apoi a venit Solomon de pe înălțimea cea din Ghibeon, de la cortul adunării, la Ierusalim și a domnit peste Israel.
\par 14 După aceea Solomon a adunat care de război și călăreți; și avea o mie patru sute de care de război și douăsprezece mii de călăreți. Pe toți aceștia i-a așezat în cetățile în care ținea carele de război, precum și pe lângă rege în Ierusalim.
\par 15 Și a făcut regele ca argintul și aurul să fie în Ierusalim la preț ca pietrele de rând, iar cedrii, prin mulțimea lor, ajunseseră la preț ca smochinii cei sălbatici care cresc mulțime prin locurile joase.
\par 16 Caii i se aduceau lui Solomon din Egipt și din Cheve (Coa); negustorii regelui îi cumpărau cu bani, din Cheve (Coa).
\par 17 Un car de război se cumpăra și se aducea din Egipt cu șase sute sicli de argint, iar calul cu o sută cincizeci. Tot astfel aduceau ei și tuturor Heteilor și regilor Siriei.

\chapter{2}

\par 1 Apoi Solomon hotărî să înalțe templu numelui Domnului și casă domnească pentru sine.
\par 2 În acest scop Solomon a tocmit șaptezeci de mii de oameni ca să care poveri și optzeci de mii de tăietori de piatră din munte, iar peste ei, trei mii șase sute de supraveghetori.
\par 3 Apoi a trimis Solomon la Hiram, regele Tirului, să i se spună: "Cum ai făcut cu David, tatăl meu, și i-ai trimis cedrii pentru clădirea casei lui de locuit, așa să faci și cu mine.
\par 4 Iată și eu, fiul lui, zidesc templu numelui Domnului Dumnezeului meu, ca să îl închin Lui, pentru ca să se ardă înaintea Lui tămâie mirositoare, să I se înfățișeze pururea pâinile punerii înainte și să I se aducă arderi de tot dimineața și seara, în ziua odihnei, la lună nouă și la sărbătorile Domnului Dumnezeului nostru, ceea ce s-a poruncit pentru totdeauna lui Israel.
\par 5 Templul pe care voiesc să-l fac este mare, pentru că mare este Dumnezeul nostru și mai presus de toți dumnezeii.
\par 6 Și cine poate să-I zidească Lui templu, când cerul și cerurile cerurilor nu-L încap? Și cine sunt eu, ca să-I pot zidi templu? Fără numai doar pentru a se tămâia înaintea Lui.
\par 7 Trimite-mi dar un om care să știe să facă lucruri de aur, de argint, de aramă și de fier, de tort purpuriu, stacojiu și albastru și care să mai știe a săpa și a face toate acestea împreună cu meșterii care sunt aici la mine în Iuda și Ierusalim și pe care i-a pregătit David, tatăl meu.
\par 8 Să-mi trimiți lemn de cedru, de chiparos și de molift din Liban, căci știu că robii tăi pricep să taie lemn în Liban, și robii mei vor merge cu ai tăi,
\par 9 Ca să-mi pregătească lemn mult, deoarece locașul pe care voiesc să-l fac este mare și minunat.
\par 10 Iată, robilor tăi, tăietorilor care taie lemnul, le dau hrană: douăzeci de mii de core de grâu, douăzeci de mii de core de orz, douăzeci de mii de baturi de vin și douăzeci de mii de baturi de untdelemn".
\par 11 Hiram, regele Tirului, a răspuns la aceasta, prin o scrisoare pe care a trimis-o lui Solomon, următoarele: "Din iubire pentru poporul Său, Domnul te-a pus pe tine rege peste el".
\par 12 Apoi Hiram mai zicea: "Binecuvântat fie Domnul Dumnezeul lui Israel, Cel ce a făcut cerul și pământul și a dat regelui David fiu înțelept, cu minte și pricepere, care are de gând să înalțe templu Domnului și casă regală pentru sine.
\par 13 Așadar îți trimit un om înțelept și înzestrat cu știință, și anume pe meșterul Hiram-Abi,
\par 14 Fiul unei femei dintre fiicele lui Dan, iar tatăl său e tirian. Acela știe să facă lucruri de aur și de argint, de aramă și de fier, de piatră și de lemn, de tort purpuriu, stacojiu și albastru, de in și de purpură; știe să facă tot felul de sculpturi și să îndeplinească tot ce i se va porunci, împreună cu meșterii tăi, cu meșterii domnului meu David, tatăl tău.
\par 15 Iar grâul, orzul, untdelemnul și vinul de care vorbești tu, domnul meu, trimite-l robilor tăi.
\par 16 Noi însă vom tăia lemn din Liban cât îți va trebui și-l vom duce cu plutele pe mare la Iafa, iar de acolo îl vei duce tu la Ierusalim".
\par 17 Apoi Solomon a numărat pe toți străinii care se aflau atunci în pământul lui Israel, afară de numărătoarea pe care o făcuse David, tatăl său, și s-au găsit o sută cincizeci și trei de mii șase sute inși.
\par 18 Din aceștia a făcut șaptezeci de mii cărători cu spatele și optzeci de mii tăietori de piatră în munte, iar trei mii șase sute, supraveghetori, ca să îndemne pe oameni la lucru.

\chapter{3}

\par 1 Solomon a început să înalțe templul Domnului în Ierusalim, pe muntele Moria, unde Se arătase Domnul lui David, tatăl său, și pe locul pe care-l pregătise David, în aria lui Ornan Iebuseul.
\par 2 Zidirea s-a început în ziua a doua a lunii a doua din anul al patrulea al domniei lui Solomon.
\par 3 Iată temelia pusă de Solomon la zidurile templului Domnului: lungimea lui era de șaizeci de coji, iar lățimea de douăzeci de coți.
\par 4 Pridvorul, care era înaintea templului, avea lungimea de douăzeci de coți, cât lățimea templului, iar înălțimea de o sută douăzeci de coți și pe dinăuntru l-a căptușit cu aur curat.
\par 5 Casa cea mare (Sfânta) însă a căptușit-o cu lemn de chiparos și a îmbrăcat-o cu aur de cel mai bun, iar pe el a săpat finici și lănțișoare.
\par 6 Apoi a împodobit casa cu pietre scumpe. Aurul era din Parvaim (Ofir).
\par 7 Pereții casei, tavanul, ușorii, ferestrele și ușile le-a acoperit cu aur și pe pereți a sculptat chipuri de heruvimi.
\par 8 Sfânta Sfintelor a făcut-o lungă de douăzeci de coți, cât lărgimea casei, și largă de douăzeci de coți, și a îmbrăcat-o cu șase sute de talanți din cel mai bun aur.
\par 9 Cuiele de aur erau în greutate de cincizeci de sicli fiecare cui. Încăperile de sus încă le-a îmbrăcat cu aur.
\par 10 Apoi a făcut în Sfânta Sfintelor doi heruvimi săpați în lemn și i-a ferecat cu aur.
\par 11 Aripile heruvimilor aveau în lungime douăzeci de coți: o aripă de cinci coți atingea peretele templului, iar cealaltă aripă de cinci coți atingea aripa celuilalt heruvim.
\par 12 Asemenea și aripa de cinci coți a celuilalt heruvim atingea peretele templului, iar cealaltă aripă de cinci coți atingea aripa heruvimului întâi.
\par 13 Aripile acestor heruvimi întinse erau de douăzeci de coți; heruvimii stăteau în picioare, cu fețele spre Sfânta.
\par 14 După aceea au făcut o perdea de torturi purpurii, stacojii și albastre și de vison, iar pe ea a închipuit heruvimi.
\par 15 Înaintea templului au făcut doi Stâlpi, înalți de treizeci și cinci de coți, iar coroanele de deasupra, de cinci coți fiecare.
\par 16 Au mai făcut lănțișoare, ca cele din Sfânta Sfintelor, și le-a pus pe capetele stâlpilor. Apoi au făcut o sută de rodii și le-au înșirat pe lănțișoare.
\par 17 În sfârșit au așezat stâlpii înaintea templului, unul în dreapta și altul în stânga și le-a pus nume: celui din dreapta Iachin, iar celui din stânga Booz.

\chapter{4}

\par 1 După aceea a făcut jertfelnic de aramă, lung de douăzeci de coli, lat de douăzeci de coți, înalt de zece coți.
\par 2 A făcut așa-numita mare de aramă turnată rotund, care avea de la o margine până la cealaltă zece coți, iar înălțimea ei era de cinci coți; o sfoară de treizeci de coți putea să o cuprindă împrejur.
\par 3 Sub ea de jur-împrejur se aflau chipuri de boi turnate. Boii aceștia, turnați împreună cu marea, erau așezați sub ea de jur-împrejur pe două rânduri, la zece coți depărtare unul de altul.
\par 4 Marea stătea pe doisprezece boi: trei îndreptați spre miazănoapte, trei îndreptați spre apus, trei îndreptați spre miazăzi și trei îndreptați spre răsărit, iar marea deasupra, pe ei; spatele boilor însă erau înăuntru, sub mare.
\par 5 Ea era groasă în pereți de un lat de palmă; marginile ei erau ca marginile unei cupe, semănând cu o floare deschisă de crin; în ea încăpeau până la trei mii de baturi.
\par 6 A mai făcut zece ligheane și le-a așezat cinci în partea dreaptă și cinci în partea stângă, ca să se spele în ele cele pregătite pentru arderile de tot; marea însă era pentru preoți, ca să se spele în ea.
\par 7 De asemenea a făcut zece sfeșnice de aur, cum trebuia să fie, și le-a pus în templul Domnului, cinci în partea dreaptă și cinci în partea stângă.
\par 8 A făcut încă zece mese și le-a pus în templul Domnului, cinci în partea dreaptă și cinci în partea stângă; și a făcut o sută de vase de aur.
\par 9 A făcut apoi curtea preoților și curtea cea mare; la curți a făcut porți și porțile le-a ferecat cu aramă.
\par 10 Marea a pus-o în partea dreaptă, spre miazăzi-răsărit.
\par 11 După aceea Hiram a făcut ligheane și lopeți, castroane, cădelnițe și toate vasele pentru jertfe. Și a sfârșit Hiram lucrările pe care le-a făcut lui Solomon pentru templul lui Dumnezeu:
\par 12 Cei doi stâlpi cu cele două coroane de pe vârfurile stâlpilor, și două rețele pentru acoperit cele două coroane, care erau pe capetele stâlpilor;
\par 13 Patru sute de rodii a pus pe rețele, câte două rânduri de rodii pentru fiecare rețea, care acoperea cele două coroane de pe vârful stâlpilor.
\par 14 A făcut postamente și pe postamente ligheane de spălat;
\par 15 O mare și sub ea doisprezece boi;
\par 16 Ligheane, lopeți, furculițe și toate celelalte lucruri le-a făcut meșterul Hiram regelui Solomon, pentru templul Domnului, din aramă lustruită.
\par 17 Acestea le-a turnat regele în preajma Iordanului, într-un pământ cleios, între Sucot și Țereda (Țartan).
\par 18 Toate lucrurile acestea le-a făcut Solomon în așa de mare număr, încât nu se mai știa greutatea aramei.
\par 19 De asemenea a făcut Solomon toate lucrurile pentru templul Domnului, jertfelnicul cel de aur și mesele pentru pâinile punerii înainte;
\par 20 Sfeșnicele și candelele lor de aur curat, ca să fie aprinse după rânduială înaintea Sfintei Sfintelor.
\par 21 Mucările, candelele și tăvițele de aur, din cel mai curat aur.
\par 22 Cuțitele, potirele, cățuiele și ceștile tot din cel mai bun aur: ușile templului Domnului, ușile dinăuntru de la Sfânta Sfintelor și ușa de la Sfânta, tot de aur.

\chapter{5}

\par 1 Astfel s-a sfârșit tot lucrul pe care l-a făcut Solomon pentru templul Domnului. Și a adus Solomon cele hărăzite de David, tatăl lui: argint și aur și toate lucrurile și le-a dat în vistieria templului Domnului.
\par 2 Atunci a adunat Solomon pe bătrânii lui Israel și pe toți capii semințiilor și căpeteniile familiilor fiilor lui Israel la Ierusalim, ca să strămute chivotul legământului Domnului din cetatea lui David, adică din Sion.
\par 3 Și s-au adunat la rege toți Israeliții la sărbătoare, în luna a șaptea.
\par 4 După ce au venit toate căpeteniile lui Israel, leviții au luat chivotul.
\par 5 Și au dus chivotul și cortul adunării și toate lucrurile sfinte care erau în cort, le-au dus preoții și leviții.
\par 6 Iar regele Solomon și toată obștea lui Israel, care se adunase la el înaintea chivotului, au adus jertfe de oi și boi atâtea, cât nu se puteau număra din pricina mulțimii.
\par 7 Preoții au dus chivotul legământului Domnului la locul lui, înăuntrul templului în Sfânta Sfintelor, sub aripile heruvimilor.
\par 8 Heruvimii își întindeau aripile peste locul chivotului și acopereau ei chivotul și pârghiile lui de sus.
\par 9 Apoi pârghiile au fost împinse așa, încât capetele pârghiilor chivotului se vedeau din Sfânta, în fața Sfintei Sfintelor, iar pârghiile nu se puteau vedea, și acolo sunt ele până astăzi.
\par 10 În chivot nu era nimic, fără numai cele două table pe care le pusese Moise în Horeb, când încheiase Domnul legământul cu fiii lui Israel, după ieșirea din Egipt.
\par 11 După ce au ieșit preoții din Sfânta, căci toți preoții care se aflau acolo se sfințiseră, fără să se țină seamă de rând,
\par 12 și când toți leviții care erau cântăreți, Asaf, Heman, Iedutun, fiii lor și frații lor - îmbrăcați în vison, și cu chimbale, chitare și harfe au stat în partea de răsărit a jertfelnicului și împreună cu ei au stat o sută douăzeci de preoți care trâmbițau din trâmbițe
\par 13 Și îndată ce aceia care sunau din trâmbițe și cei care cântau, uniți într-un singur glas ca să slăvească și să laude pe Domnul, au făcut să răsune trâmbițele, chitarele și celelalte instrumente muzicale, și au slăvit pe Domnul, zicând: "Căci El este bun, că în veac este mila Lui!", atunci templul Domnului s-a umplut de norul slavei Lui,
\par 14 Încât preoții nu puteau sta la slujbă din pricina norului, pentru că slava Domnului umpluse templul Domnului.

\chapter{6}

\par 1 Atunci Solomon a zis: "Domnul a spus că El binevoiește să locuiască în negură,
\par 2 Iar eu am zidit templul, ca să locuiești Tu, Cel Sfânt, și locaș unde să petreci Tu în veci".
\par 3 Apoi și-a întors regele fața sa și a binecuvântat toată adunarea lui Israel, căci toată adunarea Israeliților sta înainte.
\par 4 Și a zis regele: "Binecuvântat este Domnul Dumnezeul lui Israel, Care a grăit cu gura Sa către David, tatăl meu, și a împlinit cu mâinile Sale ceea ce spusese, zicând:
\par 5 Din ziua când am scos pe poporul Meu din țara Egiptului, nu Mi-am ales cetate nici într-una din semințiile lui Israel, ca să-Mi zidesc templu în care să petreacă numele Meu, nici nu Mi-am ales om care să fie cârmuitor poporului Meu Israel.
\par 6 Acum însă am ales Ierusalimul, ca să petreacă numele Meu acolo, și am ales pe David, ca să păstorească peste poporul Meu Israel.
\par 7 Iar lui David, tatăl meu, îi intrase la inimă să înalțe templu numelui Domnului Dumnezeului lui Israel.
\par 8 Însă Domnul a zis lui David, tatăl meu: ți-a intrat la inimă să zidești templu numelui Meu; bine e că ți-a intrat acest lucru la inimă;
\par 9 Dar nu vei zidi tu templul, ci fiul tău care va ieși din coapsele tale, acela va zidi templu numelui Meu.
\par 10 Și a împlinit Domnul cuvântul Său, care l-a grăit; căci eu am urmat în locul lui David, tatăl meu, și am șezut pe tronul lui Israel, cum zisese Domnul, și am zidit templu numelui Domnului Dumnezeului lui Israel.
\par 11 Și am pus acolo chivotul în care se află legământul Domnului, cel încheiat cu fiii lui Israel".
\par 12 Apoi stând Solomon la jertfelnicul Domnului, înaintea adunării Israeliților și-a ridicat mâinile sale,
\par 13 Căci Solomon își făcuse un amvon de aramă, lung de cinci coți, lat de cinci coți și înalt de trei coți, și-l pusese în mijlocul curții. Pe acest amvon a stat el și și-a plecat genunchii înaintea întregii adunări a Israeliților. El a ridicat mâinile sale la cer
\par 14 Și a zis: "Doamne Dumnezeul lui Israel, nu este Dumnezeu asemenea ție, nici în cer și nici pe pământ. Tu păzești legământul și mila cu robii Tăi, care umblă cu toată inima lor înaintea Ta;
\par 15 Tu ai împlinit robului Tău David, tatăl meu, ce i-ai grăit; că ceea ce ai grăit cu gura Ta, aceea ai împlinit în ziua aceasta cu mâna Ta!
\par 16 Și acum, Doamne Dumnezeul lui Israel, împlinește cele ce ai grăit către robul Tău David, tatăl meu, când ai zis: Nu va lipsi bărbat din tine care să șadă înaintea feței Mele pe tronul lui Israel, dacă fiii tăi își vor păzi calea lor purtându-se după legea Mea, așa cum te-ai purtat tu înaintea Mea.
\par 17 Deci acum, Doamne Dumnezeul lui Israel, fă să se adeverească cuvântul Tău, către robul Tău David.
\par 18 Adevărat să fie că Dumnezeu va locui cu oamenii pe pământ? Dacă cerul și cerurile cerurilor nu Te încap, cu cât mai puțin Te va încăpea templul acesta pe care Ți l-am zidit eu?
\par 19 Dar caută la rugăciunea robului Tău și la cererea lui, Doamne Dumnezeul meu! Ascultă strigarea și ruga cu care robul Tău se roagă înaintea Ta:
\par 20 Să fie ochii Tăi deschiși ziua și noaptea spre templul acesta și spre locul unde ai făgăduit să-ți pui numele Tău, ca să asculți rugăciunea cu care robul Tău se va ruga în locul acesta.
\par 21 Să iei aminte la cererile robului Tău și ale poporului Tău Israel, cu care se vor ruga ei în locul acesta; să auzi din locul șederii Tale, din ceruri, să asculți și să miluiești.
\par 22 Când va greși cineva împotriva aproapelui său și i se va cere jurământ ca să jure, jurământul se va face înaintea jertfelnicului Tău în templul acesta.
\par 23 Atunci să asculți din cer și să faci judecată robilor Tăi: să osândești pe cel vinovat, făcând să i se întoarcă asupra capului lui fapta sa și să izbăvești pe cel drept, dându-i după dreptatea lui.
\par 24 Când poporul Tău Israel va fi bătut de dușman, pentru că a păcătuit înaintea Ta, dar apoi se va întoarce către Tine, va preaslăvi numele Tău și va cere și se va ruga înaintea Ta, în templul acesta,
\par 25 Atunci să asculți din cer și să ierți păcatul poporului Tău Israel și să-l întorci în pământul pe care l-ai dat lor și părinților lor.
\par 26 Când se va încuia cerul și nu va fi ploaie, pentru că au păcătuit ei înaintea Ta, și-ți vor aduce rugi în locul acesta, vor mărturisi numele Tău și se vor întoarce de la păcatul lor pentru că i-ai smerit,
\par 27 Atunci să asculți din cer și să ierți păcatul robilor Tăi, al poporului Tău Israel, arătându-le calea cea bună pe care să meargă, și să trimiți ploaie pământului Tău, pe care l-ai dat poporului Tău de moștenire.
\par 28 De va fi foamete pe pământ, de va fi boală molipsitoare, de va fi vânt dogorâtor sau pălitură, lăcustă sau omidă, dușmanii de-i vor strâmtora în țara lor, stau în cetățile lor, de va fi orice necaz, orice boală,
\par 29 Atunci orice rugăciune și orice cerere care se va face de orice om sau de tot poporul Tău Israel, când ei își vor simți fiecare necazul său și durerea și își vor întinde mâinile lor spre templul acesta,
\par 30 Tu să asculți din cer, din locul șederii Tale, și să ierți; să dai fiecăruia după căile lui, căci Tu cunoști inima lui și singur știi inima fiilor oamenilor,
\par 31 Pentru ca să se teamă de Tine și să umble în căile Tale în toate zilele, cât vor trăi pe pământul pe care l-ai dat părinților noștri.
\par 32 Chiar și străinul, care nu este din poporul Tău Israel, când va purcede din pământ depărtat pentru numele Tău cel mare, pentru mâna Ta cea puternică și pentru brațul Tău cel înalt și va veni și se va ruga în templul acesta,
\par 33 Tu să asculți din cer, din locul sălășluirii Tale, și să-i împlinești tot lucrul pentru care străinul va striga către Tine, ca să știe toate popoarele pământului, de numele Tău și să se teamă de Tine, cum se teme poporul Tău Israel, și să știe că în numele Tău este închinat templul pe care l-am zidit eu.
\par 34 Când poporul Tău va pleca cu război împotriva dușmanilor săi, pe drumul pe care-l vei trimite Tu și se va ruga Ție, întorcându-se spre cetatea aceasta, care Ți-ai ales-o, și spre templul acesta pe care l-am zidit eu numelui Tău,
\par 35 Atunci să asculți din cer rugăciunea lor și cererea lor și să le faci dreptate.
\par 36 Când vor păcătui ei înaintea Ta - căci nu este om care să nu păcătuiască - și Tu Te vei supăra pe ei și-i vei da dușmanilor lor și cei ce i-au luat robi îi vor duce în pământ depărtat sau apropiat,
\par 37 Și când ei, în pământul în care vor fi robiți, își vor veni în sine și se vor întoarce și ți se vor ruga în pământul robiei lor, zicând: am păcătuit, am făcut fărădelege, vinovați suntem;
\par 38 Dacă se vor întoarce către Tine cu toată inima lor și cu tot sufletul lor, în pământul robiei lor, unde ei se vor afla duși robi și se vor ruga, întorcându-se spre pământul lor, pe care Tu l-ai dat părinților lor și spre "etatea care ți-ai ales-o și spre templul pe care l-am zidit eu numelui Tău,
\par 39 Atunci să asculți din cer, din locul șederii Tale, rugăciunea lor și cererea lor, și să le faci dreptate și să ierți pe poporul Tău de ceea ce a păcătuit înaintea Ta.
\par 40 Dumnezeul meu, să-Ți fie ochii Tăi deschiși și urechile Tale cu luare aminte la rugăciunea care Ți se va face în locul acesta.
\par 41 Și acum, Doamne Dumnezeule, scoală-Te și vino la locul de odihnă al Tău, Tu și chivotul puterii Tale. Preoții Tăi, Doamne Dumnezeule, se vor îmbrăca întru mântuire și cuvioșii Tăi se vor desfăta de bunătăți.
\par 42 Doamne Dumnezeule, să nu-Ți întorci fața Ta nici de la unsul Tău, ci adu-Ți aminte de îndurările cele către David, robul Tău".

\chapter{7}

\par 1 Când a sfârșit Solomon rugăciunea, s-a pogorât foc din cer și a mistuit arderea de tot și jertfele, și slava Domnului a umplut templul.
\par 2 Atunci n-au putut preoții să intre în templul Domnului din pricina slavei lui Dumnezeu care umpluse templul.
\par 3 Și toți fiii lui Israel, văzând cum s-a coborât focul și slava Domnului peste templu, au căzut cu fața la pământ pe pardoseală, s-au închinat și au slăvit pe Domnul: "Că este bun, că în veac este mila Lui!"
\par 4 Apoi regele și tot poporul au început să aducă jertfe înaintea feței Domnului.
\par 5 Regele Solomon a adus jertfă douăzeci și două de mii de boi și o sută douăzeci de mii de oi; așa au sfințit templul lui Dumnezeu regele și tot poporul.
\par 6 Preoții stăteau la slujbele lor și leviții cu instrumentele de cântare ale Domnului, pe care le făcuse regele David, ca să laude pe Domnul: "Că în veac este mila Lui"; căci David cu acestea Îl slăvea, iar preoții trâmbițau din trâmbițe înaintea lui, iar Israelul tot stătea de față.
\par 7 Apoi a mai sfințit Solomon și mijlocul curții care era înaintea templului Domnului; căci a adus acolo arderile de tot și grăsimea jertfelor de împăcare, fiindcă jertfelnicul cel de aramă pe care îl făcuse Solomon nu putea cuprinde arderile de tot și prinoasele de pâine și de grăsime.
\par 8 În vremea aceea a făcut Solomon sărbătoare de șapte zile și împreună cu el a prăznuit tot Israelul, adunare foarte mare, venită de la intrarea Hamatului, până la râul Egiptului.
\par 9 Iar în ziua a opta au sărbătorit încheierea sărbătorii, căci sfințirea jertfelnicului a ținut șapte zile, iar sărbătoarea alte șapte zile.
\par 10 Și în ziua a douăzeci și treia a lunii a șaptea, regele a dat drumul la corturile sale poporului, care se bucura și se veselea de binele ce dăduse Domnul lui David, lui Solomon și poporului Său Israel.
\par 11 Astfel a isprăvit Solomon templul Domnului și casa regelui; tot ce plănuise Solomon în inima sa să facă pentru templul Domnului și pentru casa sa, le-a isprăvit după dorință.
\par 12 Atunci S-a arătat Domnul lui Solomon, noaptea, și i-a zis: "Am auzit ruga ta și Mi-am ales locul acesta să fie templu pentru aducerea de jertfe.
\par 13 De voi încuia cerul și nu va fi ploaie, de voi porunci lăcustei să mănânce țara, sau voi trimite vreo boală molipsitoare asupra poporului Meu
\par 14 Și se va smeri poporul Meu, care se numește cu numele Meu, și se vor ruga și vor căuta fața Mea, și se vor întoarce de la căile lor cele rele, atunci îi voi auzi din cer, le voi ierta păcatele lor și le voi tămădui țara.
\par 15 Acum ochii Mei vor fi deschiși și urechile Mele vor fi cu luare-aminte la rugăciunea ce se va face în locul acesta.
\par 16 Căci am ales acum și am sfințit templul acesta, pentru ca să fie numele Meu acolo în veci; și ochii Mei și inima Mea să fie acolo în toate zilele.
\par 17 Dacă tu vei umbla înaintea feței Mele, cum a umblat David, tatăl tău, și vei face toate câte ți-am poruncit, și vei păzi rânduielile și legile Mele,
\par 18 Atunci îți voi întări tronul regatului tău, după cum am făgăduit lui David, tatăl tău, când am zis: Nu va lipsi din tine bărbat care să domnească peste Israel.
\par 19 Iar dacă vă veți abate și veți părăsi rânduielile Mele și poruncile Mele, care vi le-am dat, și vă veți duce și veți începe a sluji la alți dumnezei și vă veți închina lor,
\par 20 Atunci voi stârpi pe Israel de pe fața pământului Meu pe care l-am dat lor, și templul acesta pe care l-am sfințit numelui Meu, îl voi lepăda de la fața Mea, iar pe Israel îl vai da de pildă și de ocară la toate popoarele;
\par 21 Iar de templul acesta înalt va rămâne uimit tot cel ce va trece pe lângă el, și va zice: Pentru ce a făcut așa Domnul cu pământul acesta și cu templul acesta?
\par 22 Și vor zice: Pentru că au părăsit pe Domnul Dumnezeul părinților lor, Care i-a scos din pământul Egiptului, și pentru că s-au lipit de alți dumnezei și s-au închinat și au slujit lor, pentru aceea a adus El asupra lor toate relele acestea".

\chapter{8}

\par 1 După sfârșirea celor douăzeci de ani, în care Solomon a zidit templul Domnului și casa sa,
\par 2 A zidit Solomon cetățile pe care i le dăruise Hiram și a așezat în ele pe fiii lui Israel.
\par 3 Apoi a plecat Solomon împotriva Hamat-Țobei și a luat-o
\par 4 Și a zidit el Tadmorul, în pustiu, și toate cetățile cele pentru provizii, pe care le întemeiase în Hamat.
\par 5 El a mai zidit de asemenea Bet-Horonul de Sus și Bet-Horonul de Jos, întărindu-le cu ziduri de jur-împrejur, cu porți și cu zăvoare;
\par 6 Baalatul și toate cetățile pentru provizii, pe care le avea Solomon, și toate cetățile pentru carele de război, cetățile pentru călăreți și tot ce a dorit Solomon să zidească în Ierusalim, în Liban, în tot pământul stăpânirii lui.
\par 7 Tot poporul care a rămas din Hetei, Amorei, Ferezei, Hevei, Iebusei, care nu erau dintre fiii lui Israel;
\par 8 Pe copiii lor care au rămas în țară după ei și pe care fiii lui Israel nu i-au stârpit, Solomon i-a făcut oameni de corvoadă până în ziua de astăzi.
\par 9 Iar pe fiii lui Israel, Solomon nu i-a făcut oameni de corvoadă pentru lucrările lui; pe ei însă îi avea pentru ostași căpetenii peste gărzi și pentru căpetenii peste carele și călăreții lui.
\par 10 Regele Solomon avea două sute cincizeci de conducători mari care supravegheau oamenii la lucru.
\par 11 Iar pe fiica lui Faraon, Solomon a mutat-o din cetatea lui David în casa pe care o zidise pentru ea; căci zicea el: Femeia mea nu trebuie să locuiască în casa lui David, regele lui Israel, pentru că acea casă este sfântă de când a intrat în ea chivotul Domnului.
\par 12 Atunci a început Solomon să aducă arderi de tot Domnului pe jertfelnicul pe care-l făcuse pentru Domnul înaintea pridvorului,
\par 13 Pentru ca să se aducă pe el arderi de tot, după rânduială și după porunca lui Moise, în fiecare zi, în toate zilele de odihnă, la lunile noi și la cele trei sărbători de peste an: la sărbătoarea azimelor, la sărbătoarea săptămânilor și la sărbătoarea corturilor.
\par 14 Și a așezat el, după rânduiala lui David, tatăl său, preoții să-și facă slujba lor, cu rândul; și tot cu rândul a rânduit să fie de strajă leviții, să cânte cântările de laudă și să slujească pe preoți, după rânduiala de fiecare zi; pe ușieri, după cetele lor, i-a rânduit la fiecare ușă, pentru că așa poruncise David, omul lui Dumnezeu.
\par 15 Și nu s-a făcut întru nimic nici o abatere de la poruncile regelui, cele pentru preoți și pentru leviți, și nici de la cele pentru vistierie.
\par 16 Așa s-a săvârșit toată lucrarea lui Solomon din ziua punerii temeliei la templul Domnului până la terminarea lui.
\par 17 Atunci s-a dus Solomon la Ețion-Gheber și la Elot, pe țărmul mării, care este în țara lui Edom.
\par 18 Iar Hiram i-a trimis cu slugile sale corăbii și robi, cunoscători ai mărilor, care s-au dus cu slugile lui Solomon la Ofir și au luat de acolo patru sute cincizeci de talanți de aur și i-au adus regelui Solomon.

\chapter{9}

\par 1 Auzind regina din Saba de faima lui Solomon și voind să-l încerce cu întrebări grele, a venit la Ierusalim cu foarte multă bogăție, cu cămile încărcate cu aromate, cu aur mult și pietre scumpe; și a venit la Solomon și i-a grăit lui toate câte avea în sufletul ei.
\par 2 Și i-a dezlegat Solomon toate întrebările ei și nu s-a găsit nimic necunoscut pentru Solomon, încât să nu-i dezlege el.
\par 3 Văzând regina din Saba înțelepciunea lui Solomon și casa pe care o zidise el,
\par 4 Mâncările de la masa lui, locurile de ședere ale robilor lui, rânduiala slujitorilor lui și îmbrăcămintea lor, paharnicii lui, îmbrăcămintea lor și arderile de tot pe care le aducea în templu, a rămas uimită.
\par 5 Și a zis regelui: "Cele ce am auzit eu în țara mea despre lucrurile tale și despre înțelepciunea ta, sunt adevărate;
\par 6 Dar eu n-am crezut vorbele ce mi se spuneau, până ce n-am venit și n-am văzut cu ochii mei. Și iată nici pe jumătate nu mi s-a grăit de mărirea înțelepciunii tale; tu întreci cu mult faima auzită de mine, despre tine.
\par 7 Fericiți sunt oamenii tăi și fericite sunt aceste slugi ale tale, care-ți stau totdeauna înainte și-ți ascultă înțelepciunea!
\par 8 Binecuvântat să fie Domnul Dumnezeul tău Care a binevoit să te pună pe tronul Său, pentru ca să fii rege în numele Domnului Dumnezeului tău. Din dragostea pe care Dumnezeul tău o are către Israel, ca să-l întărească în veci, te-a făcut rege peste el, ca să faci judecată și dreptate".
\par 9 Și a dăruit ea regelui o sută douăzeci de talanți de aur și mulțime mare de aromate și de pietre scumpe; asemenea aromate, ca cele dăruite de regina din Saba regelui Solomon, nu se mai văzuseră.
\par 10 În vremea aceea slugile lui Hiram și slugile lui Solomon, care-i aduceau aur de la Ofir, îi aduseseră și lemn roșu și pietre scumpe.
\par 11 Din acest lemn roșu făcuse regele scările de la templul Domnului și de la casa domnească, precum și chitare și harpe pentru cântăreți. Astfel de lemn nu se mai văzuse niciodată înainte în țara lui Iuda.
\par 12 Iar regele Solomon a dat reginei din Saba tot ce ea a dorit și a cerut, afară de darul pe care i l-a dat pentru lucrurile pe care ea le adusese regelui. Și așa s-a întors înapoi în țara sa, ea și slugile sale.
\par 13 Greutatea aurului care i se adusese lui Solomon într-un an era de șase sute șaizeci și șase de talanți de aur.
\par 14 Afară de acestea, mai aduceau lui Solomon aur și argint solii de la diferite popoare și negustorii, precum și toți regii Arabiei și conducătorii de provincii.
\par 15 Regele Solomon a făcut două sute de scuturi mari de aur ciocănit în care au intrat câte șase sute de sicli de aur de fiecare scut ciocănit.
\par 16 Și trei sute de scuturi mici, tot de aur ciocănit, în care au intrat câte trei sute de sicli de aur de fiecare scut. Pe acestea le-a pus regele în Casa Pădurii din Liban.
\par 17 Apoi a făcut regele un tron mare de os de fildeș și l-a îmbrăcat peste tot cu aur curat,
\par 18 Iar la tron a făcut șase trepte de suit, un scăunel de aur pentru picioare, prins de tron, rezemători de o parte și de alta a locului de ședere și doi lei care stăteau lângă rezemători
\par 19 Și încă doisprezece lei care stăteau acolo pe cele șase trepte de o parte și de alta. Așa tron nu se mai găsea în nici un regat.
\par 20 Toate vasele de băut ale regelui Solomon erau de aur și toate vasele din Casa Pădurii din Liban erau de aur ales. În zilele lui Solomon argintul se socotea ca nimic,
\par 21 Căci corăbiile regelui umblau la Tarsis eu slugile lui Hiram și la trei ani o dată se întorceau aducând aur și argint, fildeș, maimuțe și păuni.
\par 22 Astfel Solomon a întrecut pe toți regii pământului în bogăție și înțelepciune.
\par 23 Și toți regii țărilor căutau să vadă pe Solomon, ca să-i asculte înțelepciunea pe care i-o pusese Dumnezeu în inima lui.
\par 24 Și fiecare din ei îi aducea vase de argint, de aur, veșminte, arme, aromate, cai și catâri, în fiecare an.
\par 25 Solomon avea patru mii de iesle pentru caii de pe la carele lui și douăsprezece mii de călăreți, așezați în cetățile unde avea carele și pe lângă rege în Ierusalim.
\par 26 El domnea peste toți regii, de la râul Eufrat până la țara Filistenilor și până la hotarul Egiptului.
\par 27 Și a făcut regele să fie aurul și argintul prețuit în Ierusalim ca pietrele de pe drum, iar cedrii, din pricina mulțimii lor, i-a făcut să fie prețuiți ca smochinii cei sălbatici de prin locuri joase.
\par 28 Cai pentru Solomon i se aduceau din Egipt și din toate țările.
\par 29 Celelalte fapte ale lui Solomon, de la cele dintâi până la cele din urmă, sunt scrise în cartea lui Natan proorocul, în proorocia lui Ahia Șilonitul și în vedeniile lui Ido văzătorul despre Ieroboam, fiul lui Nabat.
\par 30 Solomon a domnit în Ierusalim peste tot Israelul patruzeci de ani.
\par 31 Apoi a răposat Solomon cu părinții săi și l-au îngropat în cetatea lui David, tatăl său. Iar în locul lui s-a făcut rege Roboam, fiul său.

\chapter{10}

\par 1 Atunci s-a dus Roboam la Sichem, pentru că la Sichem se adunaseră toți Israeliții, ca să-l facă rege.
\par 2 Când a auzit de aceasta Ierobnam, fiul lui Nabat, care se afla în Egipt, unde fugise de regele Solomon, s-a întors Ieroboam din Egipt.
\par 3 Iar Israeliții au trimis și l-au chemat. Venind deci Ieroboam și tot Israelul, au grăit lui Roboam așa:
\par 4 "Tatăl tău a pus jug greu pe noi. Tu însă ușurează-ne de munca silnică a tatălui tău și de jugul cel greu, care d-a pus el pe noi și îți vom sluji".
\par 5 Iar Roboam le-a zis: "Veniți peste trei zile la mine!" Și s-a împrăștiat poporul.
\par 6 Atunci s-a sfătuit regele Roboam eu bătrânii care fuseseră sfetnicii lui Solomon, tatăl lui, cât a trăit el și le-a zis: "Cum mă sfătuiți să răspund poporului acestuia?"
\par 7 Iar ei i-au zis: "De vei fi bun acum cu poporul acesta, de le vei face pe plac și le vei vorbi cu blândețe, atunci ei îți vor fi robi în toate zilele".
\par 8 Dar el n-a ținut seama de sfatul ce i-au dat bătrânii, ci a început să se sfătuiască cu oamenii cei tineri care crescuseră cu el și-i avea ca sfetnici pe lângă sine
\par 9 Și le-a zis: "Ce mă sfătuiți să răspund poporului acestuia, care mi-a grăit așa: Ușurează-ne jugul pe care tatăl tău l-a pus pe noi?"
\par 10 Oamenii cei tineri, care crescuseră cu el, i-au răspuns și i-au zis: "Poporului care ți-a vorbit: Tatăl tău a pus jug greu peste noi, iar tu ușurează-ni-l, spune-le așa: Degetul meu cel mic e mai gros decât mijlocul tatălui meu.
\par 11 Tatăl meu a pus jug greu pe voi, eu însă voi mări jugul vostru; tatăl meu v-a pedepsit cu biciul, eu însă vă voi bate cu scorpioane".
\par 12 Și a venit Ieroboam împreună eu tot poporul la Roboam, în ziua a treia, după cum le poruncise regele, când zisese: "Veniți la mine peste trei zile".
\par 13 Atunci regele le-a răspuns cu asprime, căci n-a ținut seamă regele Roboam de sfatul bătrânilor, ci le-a grăit după sfatul oamenilor tineri așa:
\par 14 "Tatăl meu pus jug greu peste voi, eu însă îl voi mări; tatăl meu v-a pedepsit cu biciul; eu însă vă voi bate cu scorpioane".
\par 15 Și n-a ascultat regele de popor, pentru că așa fusese rânduit de la Dumnezeu, ca să-și împlinească Domnul cuvântul Său, pe care-l grăise prin Ahia Șilonitul lui Ieroboam, fiul lui Nabat.
\par 16 Când tot Israelul a văzut că regele nu-l ascultă, atunci poporul a răspuns regelui, zicând: "Ce parte mai avem noi cu David? Nu mai avem nimic cu fiul lui Iesei. La corturi, Israele! Davide, vezi de casa ta!" Și s-au împrăștiat toți Israeliții pe la corturile lor.
\par 17 Iar Roboam a rămas rege numai peste fiii lui Israel care locuiau în cetățile Iudei.
\par 18 Atunci regele Roboam a trimis împotriva lor pe Adoniram, care era căpetenie peste strânsul dărilor; dar fiii lui Israel l-au bătut cu pietre și el a murit. Iar regele Roboam s-a grăbit să se suie în carul său, ca să fugă la Ierusalim.
\par 19 Așa s-a despărțit Israel de casa lui David, până în ziua de astăzi.

\chapter{11}

\par 1 Atunci a venit Roboam la Ierusalim și a strâns din casa lui Iuda și a lui Veniamin o sută șaptezeci de mii de ostași aleși, ca să lupte în Israel și să întoarcă regatul iarăși sub stăpânirea lui Roboam.
\par 2 Și a fost cuvântul Domnului spre Șemaia, omul lui Dumnezeu și i-a zis:
\par 3 "Spune lui Roboam, fiul lui Solomon, regele Iudei și la tot Israelul din neamul lui Iuda și al lui Veniamin:
\par 4 Așa zice Domnul: Să nu mergeți și să nu faceți război cu frații voștri. Întoarceți-vă fiecare la casa voastră, căci de Mine s-a făcut acest lucru!" și au ascultat ei de cuvintele Domnului și s-au întors înapoi din drumul lor împotriva lui Ieroboam.
\par 5 Roboam ședea în Ierusalim și a împrejmuit cetățile lui Iuda cu ziduri.
\par 6 El a întărit Betleemul, Etamul și Tecoa;
\par 7 Bet-Țurul, Soco și Adulamul,
\par 8 Gatul, Mareșa și Ziful,
\par 9 Adoraimul, Lachișul și Azeca,
\par 10 Țora, Aialonul și Hebronul, care se aflau în neamul lui Iuda și al lui Veniamin.
\par 11 Cetățile acestea le-a întărit el cu ziduri și a așezat în ele căpetenii și magazii pentru ținut pâine, untdelemn și vin.
\par 12 Fiecărei cetăți i-a dat scuturi și sulițe și le-a întărit cu foarte mare tărie. Și așa a rămas cu el Iuda și Veniamin.
\par 13 Apoi s-au adunat la el din toate părțile preoții și leviții, care erau în tot pământul lui Israel,
\par 14 Căci leviții și-au părăsit așezările și locurile lor, stăpânite de ei, și au venit în Iuda și la Ierusalim, din pricină că Ieroboam și fiii lui îi îndepărtaseră din dregătoria preoției Domnului
\par 15 Și pentru că Ieroboam își pusese preoți pentru înălțimi și pentru țapii și vițeii pe care îi făcuse el.
\par 16 Iar după ei au venit la Ierusalim din toate semințiile lui Israel și aceia care își aveau inima îndreptată să caute pe Domnul Dumnezeul lui Israel, pentru ca să aducă jertfă Domnului Dumnezeului părinților lor.
\par 17 Și așa au întărit aceștia regatul lui Iuda și l-au sprijinit pe Roboam, fiul lui Solomon, timp de trei ani., pentru că trei ani a umblat el pe căile lui David și ale lui Solomon.
\par 18 Roboam și-a luat de femeie pe Mahalat, fiica lui Ierimot, fiul lui David și al Abihailei, fiica lui Eliab, fiul lui Iesei.
\par 19 Aceasta i-a născut fii pe Ieuș, pe Șemaria și pe Zaham.
\par 20 După ea a mai luat pe Maaca, fiica lui Abesalom, care i-a născut pe Abia, pe Atai, pe Ziza și pe Șelomit.
\par 21 Roboam însă iubea pe Maaca, fiica lui Abesalom, mai mult decât pe toate femeile și decât pe toate concubinele sale, căci el a avut optsprezece femei și șaizeci de concubine și a născut cu ele douăzeci și opt de băieți și șaizeci de fete.
\par 22 Și a pus Roboam pe Abia, fiul Maacăi, conducător peste frații lui, pentru că pe el voia să-l facă rege.
\par 23 Și a lucrat înțelepțește, căci și-a împărțit toți feciorii săi în toate cetățile întărite din tot pământul lui Iuda și al lui Veniamin și le-a dat întreținere mare și le-a căutat mulțime de femei.

\chapter{12}

\par 1 După ce s-a întărit regatul lui Roboam și a ajuns destul de puternic, Roboam a părăsit legea Domnului și dimpreună cu el și tot Israelul.
\par 2 Dar pentru că s-au abătut ei de la Domnul, de aceea în anul al cincilea al domniei lui Roboam, Șișac, regele Egiptului, a plecat cu război împotriva Ierusalimului,
\par 3 Cu o mie două sute de care de război și cu șaizeci de mii de călăreți; iar poporul, care venise cu el din Egipt: Libieni, Suchieni și Etiopieni, era foarte numeros.
\par 4 Aceștia au luat cetățile întărite din Iuda și au venit la Ierusalim.
\par 5 Atunci Șemaia proorocul a venit la Roboam și la căpeteniile lui Iuda, care se strânseseră la Ierusalim din pricina lui Șișac și le-a zis: "Așa zice Domnul: Pentru că M-ați părăsit, de aceea vă las pe mâinile lui Șișac".
\par 6 Iar căpeteniile lui Israel și regele s-au smerit și au zis: "Drept este Domnul!"
\par 7 Când a văzut Domnul că ei s-au smerit, atunci a fost cuvântul Domnului din nou către Șemaia și a zis: "S-au smerit; nu-i voi mai stârpi și în curând le voi da și izbăvire. Mânia Mea nu se va mai vărsa asupra Ierusalimului prin mâna lui Șișac.
\par 8 Dar ei tot vor fi slugile lui ca să cunoască ce înseamnă să-Mi slujească Mie și ce înseamnă să slujească regatelor pământului".
\par 9 Atunci a venit Șișac, regele Egiptului, la Ierusalim și a luat vistieriile templului Domnului și vistieriile casei regelui; tot ce a găsit, a luat; a luat și scuturile cele de aur, pe care le făcuse Solomon.
\par 10 Iar regele Roboam a făcut în locul lor scuturi de aramă și le-a dat în mâinile căpeteniilor de peste gărzi, care păzeau intrarea la casa regelui.
\par 11 Și numai când mergea regele la templul Domnului, numai atunci venea garda și le purta și pe urmă le aducea iarăși în casa de gardă.
\par 12 După ce s-a smerit Roboam, s-a îndepărtat mânia Domnului de la el și nu l-a nimicit cu totul. Afară de aceasta și în Iuda se găsea câte ceva bun.
\par 13 Așa s-a întărit iar regele Roboam în Ierusalim și a domnit. Roboam era de patruzeci și unu de ani când s-a făcut rege și a domnit șaptesprezece ani în Ierusalim, în cetatea pe care o alesese Domnul dintre toate semințiile lui Israel, ca să-și pună numele în ea. Pe mama lui Roboam o chema Naama Amonita.
\par 14 Dar a făcut el rele, pentru că nu și-a îndreptat inima sa ca să caute pe Domnul.
\par 15 Faptele lui Roboam, cele dintâi și cele de pe urmă, sunt scrise în amintirile lui Șemaia proorocul și ale lui Ido văzătorul în spițele neamurilor. Roboam a purtat războaie cu Ieroboam în toate zilele vieții lor.
\par 16 Apoi a răposat Roboam cu părinții săi și a fost îngropat în cetatea lui David. Iar în locul lui s-a făcut rege Abia, fiul său.

\chapter{13}

\par 1 Abia a început să domnească peste Iuda în anul al optsprezecelea al domniei lui Ieroboam.
\par 2 Și a domnit trei ani în Ierusalim. Pe mama lui o chema Maaca și era fiica lui Uriel din Ghibea și a fost război și între Abia și Ieroboam.
\par 3 Abia a început războiul cu o armată numai de oameni viteji, adică de patru sute de mii de oameni aleși. Iar Ieroboam a ieșit împotriva lui la luptă cu opt sute de mii de oameni, tot numai oameni aleși și viteji.
\par 4 Abia cu armata sa s-a așezat pe vârfui muntelui Țemaraim, unul dintre munții lui Efraim și a zis: "Ascultați-mă, Ieroboame și voi toți Israeliții.
\par 5 Nu știți voi, oare, că Domnul Dumnezeul lui Israel a dat lui David domnia peste Israel în veci, lui și fiilor lui, prin legământ veșnic?
\par 6 Dar s-a sculat Ieroboam, fiul lui Nabat, care era rob la Solomon, fiul lui David, și s-a răzvrătit împotriva stăpânului său.
\par 7 Atunci s-au strâns împrejurul lui oameni netrebnici, oameni înrăiți și s-au îndârjit împotriva lui Roboam, fiul lui Solomon; iar Roboam era tânăr și cu inima fricoasă și n-a putut să li se împotrivească.
\par 8 Voi și acum ziceți că puteți să vă împotriviți regatului Domnului, care se află în mina fiilor lui David, pentru că sunteți o mulțime mare de popor și aveți viței de aur, pe care vi i-a făcut Ieroboam, ca dumnezei.
\par 9 N-ați izgonit voi oare pe preoții Domnului, pe fiii lui Aaron și pe leviți, și nu v-ați făcut voi oare singuri preoți după pilda popoarelor din celelalte țări? La voi, oricine vine să se curățească cu un vițel și cu șapte berbeci, se face preot celui ce nu este dumnezeu.
\par 10 Iar la noi Domnul este Dumnezeul nostru; noi nu L-am părăsit și slujbele cele către Domnul le fac preoții, care sunt dintre fiii lui Aaron și leviții, care sunt la slujba de strajă.
\par 11 Ei fac să se ridice în fiecare dimineață și în fiecare seară fum de arderi de tot către Domnul, cum și fum de tămâieri bine mirositoare; și așează în rânduri pâinile punerii înainte pe masa cea curată și aprind policandrele cele de aur și candelele lui, ca să ardă în fiecare seară, pentru că noi păzim legea Domnului Dumnezeului nostru, iar voi L-ați părăsit.
\par 12 Și iată, noi avem în fruntea noastră pe Dumnezeu și pe preoții Lui și trâmbițele cele răsunătoare, ca să răsune cu vuiet mare împotriva voastră. Fii ai lui Israel, să nu vă luptați cu Domnul Dumnezeul părinților voștri, că nu veți avea nici un spor".
\par 13 Dar Ieroboam a trimis o ceată mare de oameni care stau la pândă, să învăluiască pe Iuda și să le cadă în spate, așa că ei se aflau în fața lui Iuda, iar cei de la pândă în spatele lui.
\par 14 Dar Iudeii au prins de veste aceasta și au început lupta și în față și în spate și au strigat către Domnul, iar preoții au trâmbițat din trâmbițe.
\par 15 Atunci au ridicat Iudeii strigăt mare. Dar când Iudeii au ridicat strigătul, Dumnezeu a lovit pe Ieroboam și pe toți Israeliții în fața lui Abia și a lui Iuda.
\par 16 Și au fugit fiii lui Israel din fața celor din Iuda și Dumnezeu i-a dat în mâinile lor.
\par 17 Iar Abia dimpreună cu poporul lui le-a dat o lovitură puternică și au căzut morți din Israel cinci sute de mii de oameni aleși.
\par 18 Atunci s-au smerit fiii lui Israel, iar fiii lui Iuda s-au făcut puternici, pentru că au nădăjduit în Domnul Dumnezeul părinților lor.
\par 19 Abia l-a fugărit din urmă pe Ieroboam și i-a luat cetățile Betelul cu satele lui, Ieșana cu satele ei și Efronul cu satele lui.
\par 20 După aceea Ieroboam n-a mai avut putere să se împotrivească în toate zilele lui Abia. Și l-a lovit Domnul și a murit.
\par 21 Iar Abia s-a întărit. El și-a luat paisprezece femei și a născut cu ele douăzeci și doi de băieți și șaisprezece fete.
\par 22 Celelalte fapte ale lui Abia, purtările sale și vorbele sale sunt scrise în amintirile lui Ido proorocul.

\chapter{14}

\par 1 Apoi a răposat Abia cu părinții săi și l-au îngropat în cetatea lui David, iar în locul lui s-a făcut rege Asa, fiul său. În zilele sale țara a avut pace zece ani.
\par 2 Asa a făcut cele bune și plăcute înaintea Domnului Dumnezeului său,
\par 3 Căci el a îndepărtat jertfelnicele dumnezeilor străini și înălțimile, a sfărâmat stâlpii idolești și a tăiat Așerele.
\par 4 El a poruncit Iudeilor să caute pe Domnul Dumnezeul părinților lor și să împlinească legea și poruncile Lui;
\par 5 A îndepărtat din toate cetățile lui Iuda înălțimile și chipurile turnate ale soarelui. Regatul sub domnia lui a avut pace.
\par 6 El a zidit cetăți tari în Iuda; căci țara a fost în pace și n-a avut războaie cu nimeni în acei ani, fiindcă Domnul i-a dat odihnă.
\par 7 Tot el a zis Iudeilor: "Să zidiți cetățile acestea și să le îngrădiți cu ziduri, cu turnuri, cu porți și cu zăvoare; țara este încă a noastră, pentru că am căutat pe Domnul Dumnezeul nostru; L-am căutat și El ne-a dăruit odihnă din toate părțile". Și au început să le zidească și au izbutit.
\par 8 Asa avea în oștirea lui trei sute de mii de oameni aleși din seminția lui Iuda, înarmați cu scuturi și cu lănci; iar din seminția lui Veniamin, două sute optzeci de mii de oameni aleși, înarmați cu scuturi și erau și trăgători din arcuri.
\par 9 Atunci s-a sculat împotriva sa Zerah Etiopianul cu o armată de un milion și cu trei sute de care și a venit până la Mareșa.
\par 10 Aici i-a ieșit Asa înainte și s-a așezat cu armata sa în linie de bătaie, în valea lui Țefat, lângă Mareșa.
\par 11 Apoi a strigat Asa către Domnul Dumnezeul său și a zis: "Doamne, la Tine este puterea, ca să ajuți și celui tare și celui ce nu este tare; ajută-ne dar nouă, Doamne Dumnezeul nostru, căci noi în Tine nădăjduim și în numele Tău am pornit împotriva acestei mulțimi, care este atât de mare; Doamne, Tu ești Dumnezeul nostru, să n-aibă omul putere împotriva Ta".
\par 12 Atunci a lovit Domnul pe Etiopieni înaintea feței lui Asa și înaintea feței lui Iuda și au fugit Etiopienii.
\par 13 Iar Asa i-a fugărit din urmă împreună cu poporul care era cu el, până la Gherar, și au căzut atâția Etiopieni, de se credea că n-a mai rămas nimeni din ei cu suflet, pentru că aceia cădeau zdrobiți înaintea Domnului și înaintea oștirii Lui. Și au luat de la ei o mare mulțime de pradă.
\par 14 și au sfărâmat toate cetățile dimprejurul Gherarei, pentru că groaza Domnului intrase în ele. Și au prădat Israeliții toate cetățile și au adus din ele foarte multă pradă.
\par 15 De asemenea au jefuit și sălașele și au luat o mulțime de turme cu vite mărunte și cămile și s-au întors cu ele la Ierusalim.

\chapter{15}

\par 1 Atunci s-a pogorât Duhul lui Dumnezeu asupra lui Azaria, fiul lui Oded,
\par 2 Și i-a ieșit înaintea lui Asa și i-a zis: "Ascultați-mă, Asa și tot Iuda și Veniamin: Domnul este cu voi când și voi sunteți cu El. De-L veți căuta, va fi găsit de voi, iar de Îl veți părăsi, și El vă va părăsi.
\par 3 Multe zile a fost Israel fără Dumnezeul Cel adevărat și fără de preot-învățător și fără de lege;
\par 4 Dar când el, în strâmtorarea lui, s-a întors la Domnul Dumnezeul lui Israel și L-a căutat, El S-a lăsat găsit de ei.
\par 5 În acele vremuri n-a avut pace, nici cel ce ieșea, nici cel ce intra, căci mari tulburări au fost între toți locuitorii țărilor;
\par 6 Popor cu popor se lupta și cetate cu cetate, pentru că Dumnezeu îi aducea la tulburări prin tot felul de necazuri.
\par 7 Dar voi întăriți-vă și să nu vă obosească mâinile, pentru că veți avea plată pentru faptele voastre!"
\par 8 Când a auzit Asa cuvintele acestea și proorocia lui Azaria, fiul lui Oded, proorocul, s-a înfiripat în putere și a azvârlit idolii păgânești din toată țara lui Iuda și a lui Veniamin și din cetățile pe care le luase din muntele lui Efraim, și a înnoit jertfelnicul Domnului care era dinaintea pridvorului.
\par 9 Apoi a adunat pe tot Iuda și pe tot Veniaminul și pe străinii din Efraim, din Manase și din Simeon, care locuiau cu ei, căci mulți din Israel au trecut la el, când au văzut că Domnul Dumnezeu este cu el.
\par 10 Aceștia s-au adunat la Ierusalim în luna a treia din anul al cincisprezecelea al domniei lui Asa.
\par 11 Și au adus în ziua aceea jertfă Domnului din prada care o adunaseră: din vitele cele mari, șapte sute de jertfe, și din vitele mărunte, șapte mii.
\par 12 După aceea au făcut legământ între ei, ca să caute pe Domnul Dumnezeul părinților lor, cu toată inima lor și cu tot sufletul lor;
\par 13 Iar tot cel ce nu va căuta pe Domnul Dumnezeul lui Israel să moară, ori mic de va fi, ori mare, ori bărbat de va fi, ori femeie.
\par 14 Și s-au legat ei cu jurământ către Domnul cu glas tare și cu strigăte mari și în sunetul trâmbițelor și al buciumelor.
\par 15 Și le-a părut bine la toți Iudeii de acest jurământ, pentru că cu toată inima lor s-au jurat și cu toată stăruința L-au căutat, și El S-a dat lor să fie găsit. Și le-a dat lor Domnul odihnă din toate părțile.
\par 16 Pe Maaca, mama sa, regele Asa a lipsit-o de vrednicia de regină, pentru că ea făcuse un chip turnat pentru Astarte. Asa a răsturnat chipul cel turnat, pe care-l făcuse ea, l-a sfărâmat în bucăți și l-a ars în valea Chedronului.
\par 17 Deși înălțimile din Israel n-au fost de tot depărtate, ci au mai durat încă în Israel, totuși Asa și-a avut inima sa întreagă la Domnul în toate zilele vieții sale.
\par 18 Și a adus el în templul lui Dumnezeu lucrurile care le afierosise tatăl său și lucrurile care le afierosise el: adică aurul, argintul și vasele.
\par 19 Și război n-a mai fost până în anul al treizeci și cincilea al domniei lui Asa.

\chapter{16}

\par 1 Iar în anul al treizeci și șaselea al domniei lui Asa, Baeșa, regele lui Israel, a plecat împotriva lui Iuda și a început să zidească Rama, pentru ca să nu poată nimeni nici să plece de la Asa, regele lui Iuda, nici să vină la el.
\par 2 Atunci a scos Asa argintul și aurul din vistieriile templului Domnului și ale casei regelui și l-a trimis la Damasc lui Benhadad, regele Siriei, zicând:
\par 3 "Legământ să fie între mine și între tine, cum a fost între tatăl meu și între tatăl tău. Iată eu îți trimit argint și aur, du-te și rupe legământul tău, care-l ai cu Baeșa, regele lui Israel, ca să plece de la mine".
\par 4 Și a ascultat Benhadad de regele Asa și a trimis pe mai-marii oștirii, ce-i avea, împotriva cetăților lui Israel; aceștia au pustiit Ainul, Danul, Abel-Maimul (Abel-Bet) și toate hambarele cu provizii din cetățile lui Neftali.
\par 5 Când a auzit Baeșa de aceasta, a încetat să mai zidească Rama, oprind lucrarea sa.
\par 6 Iar Asa regele a strâns pe toți cei ai lui Iuda și a cărat cu ei din Rama pietrele și lemnele pe care Baeșa le folosea la întărire, și a zidit cu ele Gheba și Mițpa.
\par 7 În timpul acela a venit Hanani văzătorul la Asa, regele lui Iuda, și i-a zis: "Fiindcă ai nădăjduit în regele Siriei și n-ai nădăjduit în Domnul Dumnezeul tău, pentru aceea ți-a scăpat oastea regelui Siriei din mâna ta.
\par 8 Oare Etiopienii și Libienii n-au avut ei oaste mare și care și călăreți, cărora nu li se mai știa numărul? Dar pentru că tu ți-ai pus încrederea în Domnul, El ți i-a dat în mâna ta;
\par 9 Căci ochii Domnului văd oriunde, peste tot pământul, pentru ca să sprijine pe cei care sunt cu inima lor întreagă la El. Neînțelepțește ai lucrat dar acum. Pentru aceasta, de acum înainte tu vei avea războaie".
\par 10 Și s-a mâniat Asa pe văzător, și l-a pus la închisoare; fiindcă din pricina aceasta prinsese mare ciudă pe el; și a mai apăsat în acel timp Asa și pe unii din popor.
\par 11 Iar faptele lui Asa de la cele dintâi până la cele de pe urmă sunt scrise în cartea regilor lui Iuda și Israel.
\par 12 în anul al treizeci și nouălea al domniei sale s-a îmbolnăvit Asa de picioare și boala lui s-a întins până la părțile cele mai de sus ale corpului; dar el nici la boala sa n-a căutat pe Domnul, ci pe doctori.
\par 13 Și a răposat Asa cu părinții săi și a murit în anul al patruzeci și unulea al domniei lui.
\par 14 Și l-au îngropat în mormântul pe care și-l săpase el în cetatea lui David; și i-au pus pe un pat pe care-l umpluse cu aromate și cu tot felul de miresme; la înmormântarea lui i s-au ars foarte multe aromate.

\chapter{17}

\par 1 În locul lui Asa, s-a făcut rege Iosafat, fiul său. Și s-a întărit și el împotriva Israeliților;
\par 2 Căci a așezat oștire în toate cetățile întărite ale Iudei și în cetățile lui Efraim pe care le stăpânise Asa, tatăl său.
\par 3 Și a fost Domnul cu Iosafat, pentru că acesta a umblat în căile cele dintâi ale lui David, tatăl său, și n-a căutat la Baali,
\par 4 Ci a căutat pe Dumnezeul tatălui său și a lucrat după poruncile Lui, iar nu după faptele Israeliților.
\par 5 Pentru aceasta i-a întărit Domnul domnia în mâna lui și tot Iuda îi aducea daruri lui Iosafat, și a avut el bogăție și mărire multă.
\par 6 Inima lui s-a întărit din ce în ce în căile Domnului. Afară de aceasta, el a desființat din Iuda și înălțimile cu jertfelnice și Așerele.
\par 7 Iar în anul al treilea al domniei sale, a trimis pe cinci din căpeteniile sale și anume: pe Benhail, Obadia, Zaharia, Natanael și Miheia, ca să învețe poporul prin cetățile lui Iuda;
\par 8 Împreună cu ei a trimis și din leviți: pe Șemaia, Netania, Zebadia, Asael, Șemiramot, Ionatan, Adonie, Tobie, Tob-Adonie; iar din preoți, pe Elișama și pe Ioram.
\par 9 Aceștia au învățat în Iuda, având cu ei cartea legii Domnului; și au cutreierat toate cetățile lui Iuda și au învățat poporul.
\par 10 În vremea aceasta, în toate regatele țărilor, care erau împrejurul Iudei, intrase frica Domnului și nu îndrăznea nimeni să se lupte cu Iosafat.
\par 11 Și i se aduceau lui Iosafat daruri și argint ca bir, până și din partea Filistenilor; asemenea și Arabii îi aduceau vite mărunte; șapte mii șapte sute de berbeci și șapte mii șapte sute de țapi.
\par 12 Astfel a ajuns Iosafat la cea mai înaltă mărire și a zidit în Iuda cetățui și cetăți pentru provizii.
\par 13 El a avut multe provizii în cetățile lui Iuda și oameni de război viteji în Ierusalim.
\par 14 Iată cartea numărătorii lor, după familiile cele după tată ale lor: Din Iuda erau căpetenii peste mii: Adna, căpetenie, care avea sub el trei sute de mii de ostași aleși;
\par 15 După el Iohanan, căpetenie, care avea sub el două sute optzeci de mii de oameni.
\par 16 După acesta, Amasia, fiul lui Zicri, care se făgăduise pe sine Domnului, avea cu el două sute de mii de ostași viteji.
\par 17 Din Veniamin: viteazul ostaș Eliada, care avea sub el două sute de mii de ostași, înarmați cu arcuri și cu scuturi;
\par 18 După el Iehozabad, care avea sub el o sută optzeci de mii de ostași înarmați.
\par 19 Aceștia au slujit pe rege, afară de cei pe care regele îi împărțise prin cetățile întărite în tot ținutul lui Iuda.

\chapter{18}

\par 1 Iosafat a avut multă bogăție și mărire și s-a înrudit cu Ahab.
\par 2 După câțiva ani s-a dus la Ahab în Samaria, iar acesta a junghiat, pentru el și pentru poporul ce era cu el, un număr foarte mare de vite mărunte și mari și l-a câștigat să meargă împotriva Ramotului din Galaad.
\par 3 Căci Ahab, regele lui Israel, a zis lui Iosafat, regele lui Iuda: "Vrei să mergi cu mine la război împotriva Ramotului din Galaad?" Acela însă i-a răspuns: "Cum ești tu, așa sunt și eu; și cum este poporul tău, așa este și poporul meu: merg cu tine la război!"
\par 4 și a zis Iosafat către regele lui Israel: "Întreabă astăzi, ce zice Domnul?"
\par 5 Atunci a adunat regele lui Israel ca la patru sute de prooroci și le-a zis: "Să mergem noi oare la război împotriva Ramotului din Galaad ori să ne lăsăm?" Și ei au zis: "Du-te, că Dumnezeu îl va da în mâinile regelui".
\par 6 Zis-a Iosafat: "Nu se găsește pe aici vreun prooroc al Domnului, ca să-l întrebăm și pe el?"
\par 7 Iar regele lui Israel a răspuns lui Iosafat: "Mai este un om prin care am putea întreba pe Domnul, dar nu-l iubesc, pentru că nu proorocește de bine pentru mine, ci totdeauna îmi proorocește numai de rău; acesta este Miheia, fiul lui Imla". Iosafat însă a zis: "Nu grăi așa, rege!"
\par 8 După aceea a chemat regele lui Israel pe un famen și i-a zis: "Du-te repede după Miheia, fiul lui Imla".
\par 9 Atunci regele lui Israel și Iosafat, regele lui Iuda, ședeau fiecare pe tronul său îmbrăcați în haine regești, în locul din fața porții Samariei, și toți proorocii prooroceau înaintea lor.
\par 10 Sedechia, fiul lui Chenaana își făcuse și coarne de fier și zicea: :Așa zice Domnul: Cu acestea îi vei împunge pe Sirieni până îi vei nimici".
\par 11 Și toți proorocii prooroceau același lucru, zicând: "Du-te împotriva Ramotului din Galaad, și vei izbuti, că Domnul îl va da în mâinile regelui".
\par 12 Trimisul care se dusese să cheme pe Miheia i-a grăit, zicând: "Iată proorocii proorocesc toți într-un glas de bine regelui; să fie și cuvântul tău asemenea cu al fiecăruia din ei; proorocește și tu de bine".
\par 13 Miheia însă a zis: "Viu este Domnul, ce-mi va spune Dumnezeul meu, aceea voi prooroci".
\par 14 După aceea a venit el la rege, iar regele i-a zis: "Miheia, să mergem noi oare la război împotriva Ramotului din Galaad sau să ne lăsăm?" Iar acela a zis: "Duceți-vă, că veți izbuti, căci aceia vor cădea în mâinile voastre".
\par 15 Și i-a zis regele: "De câte ori să te jur eu oare în numele Domnului, ca să grăiești numai adevărul?"
\par 16 Atunci Miheia a zis: "Am văzut pe toți fiii lui Israel risipiți prin munți ca oile care n-au păstor, iar Domnul mi-a zis: Aceștia neavând domn, să se întoarcă fiecare la casa lui cu pace!"
\par 17 Zis-a regele lui Israel către Iosafat: "Nu ți-am spus eu oare, că el nu-mi proorocește de bine, ci numai de rău?"
\par 18 Iar Miheia a zis: "Ascultați așadar cuvântul Domnului: Am văzut pe Domnul șezând pe tronul Său și toată oștirea cerească stătea de-a dreapta și de-a stânga Lui.
\par 19 Atunci Domnul a zis: "Cine din voi ar putea să ducă în ispită pe Ahab, regele lui Israel, ca să se ducă și să cadă în Ramotul din Galaad?" Și a răspuns unul una și altul alta.
\par 20 Atunci a ieșit un duh și a stat înaintea feței Domnului și a zis: "Eu îl voi duce în ispită". Și a zis Domnul: "în ce chip?"
\par 21 Acela a zis: "Mă duc și mă fac duh al minciunii în gurile tuturor proorocilor lui". Iar Domnul a zis: "Tu îl vei duce în ispită și ai să izbutești. Du-te și fă așa!
\par 22 Și acum iată a slobozit Domnul pe duhul minciunii ca să intre în gurile acestor prooroci ai tăi. Dar Domnul n-a grăit de bine pentru tine".
\par 23 Și s-a apropiat Sedechia, fiul lui Chenaana și a lovit pe Miheia peste obraz, zicând: "Pe ce drum anume s-a retras Duhul Domnului de la mine, ca să grăiască prin tine?"
\par 24 Iar Miheia a zis: "Iată ai să vezi aceasta în ziua când vei fugi din cămară în cămară, ca să te ascunzi".
\par 25 Atunci a zis regele lui Israel: "Luați-l pe Miheia, duceți-l la Amon, căpetenia orașului, și la Ioaș, fiul regelui,
\par 26 Și le spuneți: Așa grăiește regele: Puneți pe acesta la închisoare și să-l hrăniți ca în vreme de lipsă, cu puțină pâine și cu puțină apă, până când mă voi întoarce cu pace".
\par 27 Dar Miheia a zis: "Dacă te vei întoarce sănătos, despre aceasta Domnul n-a grăit prin mine". Apoi a zis: "Ascultați acestea toți!"
\par 28 După aceea a plecat regele lui Israel și Iosafat, regele lui Iuda, împotriva Ramotului din Galaad.
\par 29 Zis-a regele lui Israel către Iosafat: "Eu îmi schimb hainele și intru în luptă; iar tu îmbracă-te în hainele tale cele de rege". Și și-a schimbat hainele regele lui Israel și a intrat în luptă.
\par 30 Regele Siriei însă poruncise căpeteniilor carelor de război, zicând: "Să nu vă luptați nici cu mic, nici cu mare, ci numai cu regele lui Israel!"
\par 31 Când căpeteniile carelor de război au văzut pe Iosafat, ei și-au zis: "Acesta este regele lui Israel" și l-au împresurat ca să se lupte cu el. Însă Iosafat a strigat, iar Dumnezeu i-a dat ajutor și i-a îndepărtat Domnul de la el.
\par 32 Și, când căpeteniile carelor de război au văzut că acesta nu este regele lui Israel, s-au dat înapoi de la el.
\par 33 Dar un om, din întâmplare, și-a întins arcul său atunci și a rănit pe regele lui Israel prin încheietura armăturii, iar regele a zis celui ce-i mâna carul: "Întoarce-te și scoate-mă din luptă, căci sunt rănit".
\par 34 În acea zi, lupta a fost crâncenă și regele lui Israel a stat în car în fața Sirienilor până seara, iar la asfințitul soarelui a murit.

\chapter{19}

\par 1 După aceea Iosafat, regele lui Iuda, s-a întors în pace la casa lui în Ierusalim.
\par 2 I-a ieșit însă înainte Iehu, fiul lui Hanani văzătorul, și a zis regelui Iosafat: "Fiindcă ai ajutat pe un nelegiuit și ai legat prietenie cu cei pe care-i urăște Domnul, de aceea mânia Domnului va fi asupra ta.
\par 3 Dar ai și fapte bune, căci ai stricat chipurile cele cioplite din pământul lui Iuda și ți-ai îndreptat inima ta ca să caute pe Dumnezeu".
\par 4 Atunci Iosafat, s-a așezat cu locuința în Ierusalim și a început el să umble iarăși prin poporul său de la Beer-Șeba, până la muntele lui Efraim și l-a întors la Domnul Dumnezeul părinților săi.
\par 5 Și a așezat judecători în țară, prin toate cetățile întărite și în fiecare cetate.
\par 6 Și a zis el judecătorilor: "Luați seama la ce veți face; să nu faceți judecată omenească, ci judecata Domnului, că la rostirea judecății El este cu voi.
\par 7 De aceea să fiți cu frica Domnului în voi, să lucrați cu pază, căci la Domnul nu-i nici nedreptate, nici părtinire și nici daruri".
\par 8 Iosafat a pus și în Ierusalim pe unii din leviți, din preoți și din capii de familie, ca să facă judecata Domnului, să judece pricinile dintre Israeliții ce locuiau în Ierusalim.
\par 9 Și iată ce porunci a dat el acestora: "Așa să lucrați, în frica Domnului, cu credință și cu inimă curată!
\par 10 La orice fel de pricini, care vor veni înaintea voastră de la frații voștri care locuiesc în cetățile lor, fie pentru vărsare de sânge, sau pentru lege și poruncă, fie pentru rânduieli și ceremonii, să-i învățați să nu facă rău, ca să nu se găsească vinovați înaintea Domnului și să nu vină mânia Lui asupra voastră și asupra fraților voștri; lucrați așa și nu veți greși
\par 11 Și iată Amaria arhiereul este pus peste voi în toate lucrurile Domnului; iar Zebadia, fiul lui Israel, căpetenia casei lui Iuda, este pus pentru toate lucrurile regelui, iar ca dregători înaintea voastră aveți pe leviți. Fiți tari și lucrați astfel, și Domnul va fi cu cel bun".

\chapter{20}

\par 1 După aceasta au venit Moabiții și dimpreună cu ei au venit și Amon asupra lui Iosafat cu război.
\par 2 Atunci au venit unii și au spus lui Iosafat, zicând: "Vine o mare mulțime de popor împotriva ta de dincolo de mare, din Siria; și iată-i se află la Hațațon-Tamar, adică la Enghedi".
\par 3 La vestea aceasta s-a înfricoșat Iosafat și și-a îndreptat fața sa, ca să caute pe Domnul făcând cunoscut tuturor din tot Iuda ca să postească.
\par 4 Deci s-a adunat Iuda ca să ceară ajutor de la Domnul și au venit din toate cetățile lui Iuda ca să roage pe Domnul.
\par 5 Iar Iosafat a stat în mijlocul adunării lui Iuda și a Ierusalimului, la templul Domnului dinaintea curții noi
\par 6 Și a zis: "Doamne Dumnezeul părinților noștri! Nu ești Tu oare Dumnezeu în cer sus, domnind peste toate regatele popoarelor și n-ai Tu oare în mâna Ta tăria și puterea, încât nimeni nu este să-ți stea ție împotrivă?
\par 7 Dumnezeul nostru, oare nu Tu ai izgonit pe locuitorii acestei țări din fața poporului Tău Israel și ai dat-o pe veci urmașilor lui Avraam, pe care îl iubeai?
\par 8 Și ei s-au așezat în ea și ți-au zidit templu sfânt numelui Tău, zicând:
\par 9 De va veni vreo nenorocire peste noi, ori sabia care ne pedepsește, sau vreo boală molipsitoare, sau foamete, și noi vom sta înaintea templului acesta și înaintea feței Tale - căci numele Tău este în templul acesta - și vom striga în strâmtorarea noastră, către Tine, Tu să ne asculți și să ne izbăvești.
\par 10 Și acum, iată Amoniții, Moabiții și locuitorii muntelui Seir, prin țara cărora Tu nu le-ai îngăduit Israeliților să treacă, când veneau ei din pământul Egiptului, ci au trecut pe lângă ei și nu i-au nimicit,
\par 11 Iată ei vor să ne răsplătească, fiindcă au venit să ne izgonească din stăpânirea Ta care ne-ai dat-o de moștenire.
\par 12 Dumnezeul nostru, judecă-i Tu, căci noi n-avem nici o putere împotriva acestei mari mulțimi care a venit asupra noastră și nu știm ce să facem; dar ochii noștri ne sunt la Tine".
\par 13 Și stăteau toți Iudeii înaintea feței Domnului și copiii cei mici ai lor, femeile lor și feciorii lor.
\par 14 Atunci S-a pogorât Duhul Domnului în mijlocul adunării peste Iahaziel, fiul Zahariei, fiul lui Benaia, fiul lui Ieiel, fiul lui Matania, levit dintre fiii lui Asaf
\par 15 Și acesta a zis: "Ascultați toți ai lui Iuda și locuitorii Ierusalimului, și tu rege Iosafat! Așa zice Domnul către voi: Să nu vă temeți, nici să vă spăimântați de mulțimea aceasta mare, căci războiul nu este al vostru, ci al lui Dumnezeu.
\par 16 Mâine să plecați împotriva lor; iată ei urcă coasta țiț și aveți să-i găsiți în capătul văii din fața pustiei Ieruel.
\par 17 Nu voi aveți să vă luptați de astă dată; dar înșirați-vă, stați și priviți izbăvirea Domnului, pe care o va trimite El vouă. Iuda și Ierusalime! Să nu vă temeți, nici să vă spăimântați! Să le ieșiți mâine înainte și Domnul va fi cu voi!"
\par 18 Atunci s-a plecat Iosafat cu fața la pământ, de asemenea și toți Iudeii și locuitorii Ierusalimului au căzut înaintea Domnului, ca să se închine Domnului.
\par 19 Apoi. s-au sculat leviții dintre fiii lui Cahat și dintre fiii lui Core, ca să laude pe Domnul Dumnezeul lui Israel cu glas foarte tare.
\par 20 Și s-au sculat ei de dimineață tare și au ieșit să se ducă în pustiul Tecoa; pe când ieșeau ei, a stat Iosafat și a zis: "Ascultați-mă Iudeilor, voi, locuitori ai Ierusalimului! Aveți încredere în Domnul Dumnezeul vostru și fiți tari! Aveți încredere în proorocii Lui și veți izbuti cu bine!"
\par 21 Apoi s-a sfătuit el cu poporul și a rânduit cântăreți care să cânte Domnului, ca, ieșind aceștia în podoabe sfinte înaintea celor înarmați, să preamărească și să zică: "Slăviți pe Domnul, că în veac este mila Lui!"
\par 22 Dar când au început ei să scoată strigăte de laudă și să preamărească, Domnul a stârnit neînțelegeri între Amoniți și Moabiți și între locuitorii muntelui Seir, care veniseră în Iuda, și au fost bătuți;
\par 23 Căci s-au sculat Amoniții și Moabiții împotriva locuitorilor muntelui Seir, bătându-i și stârpindu-i; iar când au sfârșit cu locuitorii din Seir, atunci au început să se ucidă și ei unul pe altul.
\par 24 Și când cei din Iuda au ajuns pe dealul de unde se vedea pustiul și s-au uitat la acea mulțime de lume, iată toți erau stârvuri căzute pe pământ și nici unul nu era teafăr.
\par 25 Atunci a venit Iosafat și poporul lui ca să adune prada; și au găsit la ei multe bogății, haine și lucruri scumpe și au adunat atâta cât nu puteau să ducă. Trei zile au adunat, atât de multă pradă au aflat.
\par 26 Iar în ziua a patra s-au adunat la Emec Beraca, fiindcă acolo binecuvântaseră ei pe Domnul. Pentru aceasta se și cheamă locul acela Valea Binecuvântării până în ziua de azi.
\par 27 Apoi toți bărbații din Iuda și din Ierusalim și cu Iosafat în fruntea lor s-au întors foarte voioși la Ierusalim, pentru că Domnul le dăduse biruință asupra vrăjmașilor lor.
\par 28 Și au venit în Ierusalim la templul Domnului, cu chitare, cu harfe și cu trâmbițe.
\par 29 Și frica lui Dumnezeu a cuprins pe toate regatele pământului, când au auzit că Însuși Domnul a luptat împotriva vrăjmașilor lui Israel.
\par 30 După aceea domnia lui Iosafat a fost în liniște, pentru că Dumnezeu i-a dat odihnă din toate părțile.
\par 31 Așa a domnit Iosafat peste Iuda. Când s-a făcut rege, el era de treizeci și cinci de ani și douăzeci și cinci de ani a domnit în Ierusalim. Pe mama lui o chema Azuba, fiica lui Șilhi.
\par 32 El a ținut drumul lui Asa, tatăl său și nu s-a dat în lături de loc, făcând tot fapte ce erau plăcute în ochii Domnului.
\par 33 Numai înălțimile n-au fost de tot îndepărtate și nici poporul nu se îndreptase încă cu tărie către Dumnezeul părinților lor.
\par 34 Celelalte fapte ale lui Iosafat, de la cele dintâi până la cele din urmă, se află scrise în istoria lui Iehu, fiul lui Hanani, și sunt trecute și în cartea regilor lui Israel.
\par 35 Apoi Iosafat, regele lui Iuda, a intrat în legătură cu Ohozia, regele lui Israel, care petrecea în fărădelegi,
\par 36 Și s-a întovărășit cu el, ca să facă corăbii, spre a fi trimise la Tarsis; și corăbiile le-au făcut ei la Ețion-Gheber.
\par 37 Atunci a proorocit Eliezer, fiul lui Dodava, din Mareșa, o proorocie asupra lui Iosafat, zicând: "Fiindcă tu ai intrat în legătură cu Ohozia, de aceea Domnul ți-a sfărâmat lucrurile tale". Și s-au spart corăbiile și n-au mai putut să mai meargă la Tarsis.

\chapter{21}

\par 1 Iosafat a răposat cu părinții săi și a fost înmormântat la un loc cu părinții lui, în cetatea lui David. Iar în locul lui s-a făcut rege Ioram, fiul său.
\par 2 El avea ca frați pe fiii lui Iosafat, adică pe Azaria, pe Iehiel, pe Zaharia, pe Azaria, pe Micael și pe Șefatia. Toți erau fiii lui Iosafat, regele Iudei.
\par 3 Acestora tatăl lor le făcuse daruri mari de argint, de aur și de lucruri prețioase, dimpreună cu cetățile întărite, care erau în Iuda, iar domnia a dat-o lui Ioram, pentru că el era fiul lui cel întâi-născut.
\par 4 Însă Ioram, dacă a luat domnia tatălui său și s-a întărit, a ucis pe toți frații lui cu sabia, de asemenea și pe unii dintre căpeteniile lui Israel.
\par 5 Ioram era de treizeci și doi de ani când s-a făcut rege și a domnit opt ani în Ierusalim.
\par 6 Dar el a umblat pe drumul regilor lui Israel, după cum făcuse și casa lui Ahab, pentru că el avea de femeie chiar pe o fiică a lui Ahab; și a făcut lucruri care nu erau plăcute în ochii Domnului.
\par 7 Totuși Domnul, pentru legământul pe care-l făcuse cu David, n-a vrut să piardă de tot casa lui David, căci îi făgăduise să-i dea lui și fiilor lui un sfeșnic luminos în toate timpurile.
\par 8 În zilele lui s-a rupt Edomul de sub stăpânirea lui Iuda și și-au pus pentru ei rege osebit.
\par 9 Atunci a plecat Ioram împotriva lor cu mai-marii oștirii lui și cu toate carele de război, care le avea; și, sculându-se noaptea, a bătut pe toți Edomiții care-l împresuraseră și pe toate căpeteniile carelor acelora; și poporul s-a întors repede la taberele lor.
\par 10 Edomul a rămas tot rupt de sub stăpânirea lui Iuda până în ziua de astăzi. Tot în acel timp s-a mai rupt de sub stăpânirea lui Iuda și Libna, pentru că el a părăsit pe Domnul Dumnezeul părinților săi.
\par 11 De asemenea a făcut el înălțimile de pe munții lui Iuda și a dus pe locuitorii Ierusalimului la desfrâu și pe Iuda la rătăcire.
\par 12 Atunci i-a venit o scrisoare de la Ilie proorocul, în care se spunea: "Așa zice Domnul Dumnezeul lui David, strămoșul tău: Fiindcă tu n-ai umblat pe căile lui Iosafat, tatăl tău, și pe căile lui Asa, regele lui Iuda,
\par 13 Ci ai umblat pe drumul regilor lui Israel și ai dus pe Iuda și pe locuitorii Ierusalimului la desfrâu, după cum a dus și casa lui Ahab pe Israel la desfrâu, ba încă ai ucis și pe frații tăi, fiii tatălui tău, care erau mai buni decât tine,
\par 14 Pentru aceasta, iată Domnul va lovi cu lovitură mare pe poporul tău, pe fiii tăi și pe femeile tale și toată averea ta.
\par 15 Iar pe tine însuți te va lovi cu boală grea, cu boala măruntaielor tale, până ce vor ieși din tine zi cu zi din pricina bolii".
\par 16 Apoi a întărâtat Domnul împotriva lui Ioram duhul Filistenilor și al Arabilor, care se învecinează cu Etiopienii;
\par 17 Și au plecat aceștia împotriva Iudei, au năvălit în toată țara și au luat toată averea care se afla în casa regelui; de asemenea și pe fiii lui și pe femeile lui și nu i-a mai rămas lui nici un fecior, afară de Ohozia, cel mai mic dintre fiii lui.
\par 18 Iar după toate acestea l-a lovit Domnul la măruntaiele lui și cu o boală fără vindecare.
\par 19 Așa s-a trecut zi cu zi, una după alta; iar la sfârșitul anului al doilea, din pricina bolii lui, au ieșit din el toate măruntaiele și a murit în cele mai grozave chinuri. La înmormântarea lui poporul nu i-a mai ars aromate, cum făcuse pentru părinții lui.
\par 20 El a fost de treizeci și doi de ani când s-a făcut rege și opt ani a domnit în Ierusalim; și a murit neplâns și l-au îngropat în cetatea lui David; însă nu în gropnițele regilor.

\chapter{22}

\par 1 În locul lui locuitorii Ierusalimului au făcut rege pe Ohozia, fiul cel mai mic al lui, fiindcă pe toți fiii cei mai mari îi ucisese mulțimea care venise cu Arabii la tabără. Astfel s-a făcut rege Ohozia, fiul lui Ioram, regele Iudei.
\par 2 Ohozia era de douăzeci și doi de ani când s-a făcut rege și a domnit în Ierusalim un an. Pe mama lui o chema Atalia și era fiica lui Omri.
\par 3 El de asemenea a mers pe drumul casei lui Ahab, pentru că a avut pe mama lui sfătuitoare la fapte nelegiuite.
\par 4 De aceea a făcut el lucruri care nu erau plăcute ochilor Domnului, ca și casa lui Ahab; pentru că pe casa aceasta a avut-o el de sfătuitor după moartea tatălui său, spre pieirea lui.
\par 5 După sfatul lor, a plecat cu Ioram, fiul lui Ahab, regele lui Israel, la război împotriva lui Hazael, regele Siriei, în Ramotul din Galaad. Atunci arcașii sirieni au rănit pe Ioram.
\par 6 Acesta, ca să se lecuiască de rănile ce căpătase la Ramot, când s-a luptat cu Hazael, regele Siriei, s-a întors înapoi la Izreel. Și Ohozia, fiul lui Ioram, regele Iudei, s-a dus și el la Izreel, ca să vadă pe Ioram, fiul lui Ahab, pentru că acela era bolnav.
\par 7 Cu voia lui Dumnezeu și spre pieirea lui a venit Ohozia la Ioram, căci, după sosirea lui aici, a plecat cu Ioram împotriva lui Iehu, fiul lui Nimși, pe care-l unsese Domnul pentru stârpirea casei lui Ahab.
\par 8 Astfel pe când Iehu aducea la îndeplinire judecata care era asupra casei lui Ahab, a găsit și pe căpeteniile lui Iuda și pe fiii fraților lui Ohozia, care-i slujeau, și i-a ucis.
\par 9 Apoi Iehu a poruncit să caute și pe Ohozia și l-au prins pe când el se ascundea în Samaria, l-au adus la Iehu și l-au ucis și l-au înmormântat, căci ziceau: "Este fiul lui Iosafat care a căutat pe Domnul din toată inima sa". și n-a rămas în casa lui Ohozia nimeni care să poată domni,
\par 10 Căci Atalia, mama lui Ohozia, văzând că a murit fiul ei, s-a sculat și a stârpit toată sămânța de rege din casa lui Iuda.
\par 11 Dar Ioșeba, fiica regelui Ioram, a pus mâna pe Ioaș, fiul lui Ohozia" l-a scos din mijlocul fiilor regelui, care aveau să fie uciși și l-a ascuns pe el și pe doica lui în odaia de dormit. Și astfel Ioșeba, fiica regelui Ioram, femeia lui Iehoiada preotul și sora lui Ohozia, a ascuns pe Ioaș din fața Ataliei, și nu l-a omorât.
\par 12 Și l-au ținut ei ascuns în templul lui Dumnezeu șase ani; iar peste țară a domnit Atalia.

\chapter{23}

\par 1 Iar în anul al șaptelea s-a întărit Iehoiada și a luat de tovarăș cu el pe căpeteniile de peste sute, adică pe Azaria, fiul lui Ieroham, pe Ismael, fiul lui Iohanan, pe Azaria, fiul lui Obed, pe Maaseia, fiul lui Adaia și pe Elișafat, fiul lui Zicri.
\par 2 Și au umblat de jur-împrejur prin Iuda și au adunat pe leviții din toate cetățile lui Iuda și pe capii de familii după tată ai lui Israel și au venit la Ierusalim.
\par 3 Aici a făcut toată adunarea legământ cu regele în templul lui Dumnezeu. Și a zis Iehoiada către ei: "Iată fiul regelui trebuie să fie de acum rege, așa a grăit Domnul pentru fiii lui David.
\par 4 Iată ce trebuie să faceți voi: Trei părți dintre voi, care intră cu rândul de săptămână, atât dintre preoți, cât și dintre leviți, să fie ușieri pe la uși;
\par 5 Trei părți la casa regelui și trei părți la poarta Iesod; iar poporul să stea tot în curțile templului Domnului.
\par 6 Nimeni să nu intre în templul Domnului, afară de preoții și leviții care fac slujbă. Ei pot să intre, pentru că sunt sfințiți; iar tot poporul să stea de strajă pentru Domnul.
\par 7 Leviții să înconjoare pe rege din toate părțile, fiecare cu arma lui în mâna sa; și cine va intra în templu să fie ucis. Și să fiți pe lângă rege, când va intra și când va ieși".
\par 8 Și au făcut leviții și tot Iuda ce le-a poruncit Iehoiada preotul. Atunci și-a luat fiecare oamenii săi, de aceia care intrau eu rândul de săptămână sau pe acei care își făcuseră săptămâna, pentru că Iehoiada preotul nu dăduse drumul rândurilor care erau să fie schimbate.
\par 9 Și a împărțit Iehoiada preotul căpeteniilor de peste sute sulițile și scuturile cele mari și cele mici ale lui David, care erau în templul lui Dumnezeu.
\par 10 Apoi a așezat tot poporul, având fiecare în mână arma sa, de la partea dreaptă a templului până la partea stângă a templului, înconjurând jertfelnicul și templul, ca să facă cerc în jurul regelui.
\par 11 După aceea l-au adus pe rege și i-au pus coroana și podoabele, i-au dat să țină în mână legea și l-au făcut rege; iar ungerea a săvârșit-o Iehoiada preotul și fiii lui, zicând: "Să trăiască regele!"
\par 12 Atunci a auzit Atalia strigătele poporului care alerga și cânta laude pentru rege și a venit și ea la popor în templul Domnului să privească;
\par 13 Și iată regele ședea pe tronul lui la intrare și pe lângă rege, căpeteniile, trâmbițașii și tot poporul țării se veseleau și sunau din trâmbițe; iar cântăreții, care erau cu instrumente de cântare și cei care cântau din gură, cântau cântările de laudă. Atunci Atalia și-a rupt hainele sale și a început să strige: "Vânzare! Vânzare!"
\par 14 Iar Iehoiada preotul a chemat pe căpeteniile cele peste sute, care erau cu comanda armatei și le-a zis: "Scoateți-o afară din templu și cine va urma după ea să fie ucis cu sabia!" Pentru că preotul zisese: "Să nu o ucideți în templul Domnului!"
\par 15 Și i-au făcut loc să iasă. Iar când a ajuns ea la intrarea casei regelui, cea dinspre poarta cailor, au omorât-o acolo.
\par 16 Și a făcut Iehoiada legământ cu tot poporul și cu regele, ca să fie ei popor al Domnului.
\par 17 Apoi s-a dus tot poporul la capiștea lui Baal și au dărâmat-o și au sfărâmat jertfelnicele lui și chipurile lui și a omorât pe Matan, preotul lui Baal, dinaintea jertfelnicelor.
\par 18 După aceea a încredințat Iehoiada dregătoriile templului Domnului în mâinile preoților și ale leviților; a împărțit pe preoți și pe leviți în cete, care trebuiau să facă slujbă cu rândul, cum așezase David pentru templul Domnului, ca să aducă arderi de tot Domnului, cum este scris în legea lui Moise, cu bucurie și cu cântare după cum rânduise David.
\par 19 A așezat de asemenea portari la porțile templului Domnului, ca să nu poată intra cel ce este necurat prin vreo faptă.
\par 20 Apoi a luat pe căpeteniile cele peste sute, pe viteji, pe căpeteniile poporului și pe tot poporul țării și l-au petrecut pe rege de la templul Domnului până la casa domnească, intrând pe poarta cea înaltă, în casa regelui; și l-au așezat pe tronul regatului.
\par 21 Și s-a veselit tot poporul țării și s-a liniștit cetatea. Iar pe Atalia au omorât-o cu sabia.

\chapter{24}

\par 1 Ioaș era de șapte ani când a fost făcut rege și a domnit patruzeci de ani în Ierusalim. Pe mama lui o chema Țibia și era din Beer-Șeba.
\par 2 Ioaș a făcut lucruri plăcute înaintea ochilor Domnului, în toate zilele lui Iehoiada preotul.
\par 3 Iehoiada i-a luat două femei și a avut cu ele băieți și fete.
\par 4 După aceasta și-a pus în gând Ioaș să înnoiască templul Domnului.
\par 5 De aceea a adunat pe preoți și pe leviți și le-a zis: "Duceți-vă prin cetățile lui Iuda și strângeți de la toți Israeliții argint în fiecare an pentru repararea templului Dumnezeului vostru și grăbiți-vă cu lucrul acesta". Însă leviții nu s-au grăbit.
\par 6 Atunci a chemat regele pe Iehoiada, capul lor și i-a zis: "Pentru ce nu ceri de la leviți să adune din Iuda și din Ierusalim darea hotărâtă de Moise, robul Domnului, și de către adunarea Israeliților pentru cortul adunării?
\par 7 Căci Atalia cea nelegiuită și fiii ei au despuiat templul lui Dumnezeu și toate lucrurile care erau sfințite pentru templul Domnului și le-au întrebuințat pentru baali".
\par 8 Apoi a poruncit regele de s-a făcut o ladă și a pus-o afară la ușa templului Domnului.
\par 9 Și s-a dat de veste la tot poporul din Iuda și din Ierusalim ca să aducă Domnului darea pe care o pusese Moise, robul lui Dumnezeu, asupra Israeliților în pustiu.
\par 10 De aceasta le-a părut bine la toate căpeteniile și la tot poporul și au adus și au aruncat argint în ladă până s-a umplut ea.
\par 11 Și când la timpul său se aducea lada de către leviți la dregătorii regelui și când vedeau că era mult argint în ea, atunci veneau scriitorul regelui și dregătorul arhiereului și deșertau lada, o ridicau și o puneau iar la locul ei. Așa făceau în toate zilele și s-a strâns argint mult.
\par 12 Iar regele și Iehoiada preotul l-au dat meșterilor de lucrări ai templului Domnului și aceștia și-au tocmit cioplitori de piatră și dulgheri pentru înnoire, asemenea și lucrători în fier și aramă ca să repare templul Domnului.
\par 13 Și au luat meșterii și au săvârșit lucrările cu mâinile lor și au adus templul lui Dumnezeu la starea lui de mai înainte și l-au întărit.
\par 14 Iar după ce au sfârșit toate de lucru, au adus argintul rămas înaintea regelui și a lui Iehoiada. Și au făcut din acest argint vase pentru templul Domnului, vase pentru slujbe și pentru arderi de tot, cupe și celelalte vase de aur și de argint. Și au adus necontenit arderi de tot în templul Domnului în toate zilele lui Iehoiada.
\par 15 Apoi a îmbătrânit Iehoiada și a murit sătul de zile, căci era de o sută treizeci de ani când a murit.
\par 16 Și l-au înmormântat în cetatea lui David, la un loc cu regii, pentru binele care îl făcuse în Israel, pentru Dumnezeu și pentru templul Lui.
\par 17 Iar după moartea lui Iehoiada, au venit căpeteniile lui Iuda și s-au închinat regelui; și de atunci regele a început să asculte de ei.
\par 18 Și au părăsit templul Domnului Dumnezeului părinților lor și au început să slujească Așerelor și idolilor. Din pricina acestui păcat, s-a coborât mânia Domnului peste Iuda și peste Ierusalim.
\par 19 Și a trimis la ei prooroci, pentru a-i face să se întoarcă la Domnul și i-au sfătuit, dar ei nu i-au ascultat.
\par 20 Atunci Duhul lui Dumnezeu a cuprins pe Zaharia, fiul lui Iehoiada preotul, care s-a suit pe amvon înaintea poporului și le-a zis: "Așa zice Domnul: Pentru ce călcați poruncile Domnului? Nu veți propăși, fiindcă voi ați părăsit pe Domnul și El vă va părăsi pe voi".
\par 21 Dar s-au vorbit cu toții împotriva lui și l-au ucis cu pietre, din porunca regelui Ioaș, în curtea templului Domnului.
\par 22 Regele Ioaș nu și-a adus aminte de binefacerea care i-a făcut-o Iehoiada, tatăl lui Zaharia, ci a ucis pe fiul lui și acesta, când murea, a zis: "Domnul să vadă și să facă dreptate!"
\par 23 Iar după un an, a venit împotriva lui armata siriană, a intrat în Iuda și în Ierusalim și a stârpit din popor pe toate căpeteniile poporului și toată prada luată de la ei a trimis-o regelui, la Damasc.
\par 24 Deși oștirea Sirienilor care venise era alcătuită dintr-un mic număr de oameni, Domnul însă a dat în mâna lor o oaste foarte mare, pentru că ei părăsiseră pe Domnul Dumnezeul părinților lor, iar Sirienii aduceau la îndeplinire judecata care era asupra lui Ioaș.
\par 25 Iar după ce s-au retras Sirienii de la el, rămânând el greu bolnav, slujitorii lui au uneltit împotriva lui, din cauza sângelui fiului lui Iehoiada preotul; și l-au ucis în patul lui și așa a murit. Și l-au înmormântat în cetatea lui David, însă nu l-au îngropat în mormântul regilor.
\par 26 Iar acei care au uneltit împotriva lui au fost: Zabad, fiul Șimeatei Amonita și Iehozabad, fiul Șimritei Moabita.
\par 27 Amănuntele despre fiii lui și despre mulțimea proorociilor împotriva lui, precum și despre înnoirea templului lui Dumnezeu, se găsesc scrise în cartea regilor. Iar în locul său a fost făcut rege Amasia, fiul său.

\chapter{25}

\par 1 Amasia a început să domnească când era în vârstă de douăzeci și cinci de ani și a domnit douăzeci și nouă de ani în Ierusalim. Pe mama lui o chema Iehoadan și era de loc din Ierusalim.
\par 2 El a făcut lucruri plăcute înaintea Domnului, însă nu din toată inima.
\par 3 Când domnia s-a întărit în mâna lui, el a ucis pe robii săi care uciseseră pe rege, tatăl său.
\par 4 Însă pe copiii lor nu i-a ucis, fiindcă a făcut după cum este scris în lege, în cartea lui Moise, unde a poruncit Domnul, zicând: "Părinții nu trebuie să fie uciși pentru feciori, nici feciorii nu trebuie să fie uciși pentru părinți, ci fiecare să fie ucis pentru păcatul lui".
\par 5 Apoi a adunat Amasia pe cei din Iuda și i-a așezat după neamurile cele după tată sub comanda căpeteniilor peste mii și a căpeteniilor peste sute, pe toți ai, lui Iuda și ai lui Veniamin; și i-a socotit de la vârsta de douăzeci de ani în sus și a găsit trei sute de mii de oameni voinici care erau în stare să meargă la război și să poarte suliță și scut.
\par 6 Apoi a mai tocmit cu plată din Israeliți încă o sută de mii de ostași viteji cu o sută de talanți de argint.
\par 7 Dar un om al lui Dumnezeu a venit la el și i-a zis: "O, rege, oștirea care o ai din Israel să nu meargă cu tine, pentru că Domnul nu ține cu Israeliții, nici cu toți fiii lui Efraim;
\par 8 Ci du-te singur și luptă-te cu bărbăție în război. Altfel Dumnezeu te va face să cazi în fața dușmanului; căci Dumnezeu are putere și să te sprijine și să te facă să cazi".
\par 9 Iar Amasia a zis către omul lui Dumnezeu: "Ce să fac cu cei o sută de talanți dăruiți oștirii lui Israel?" și a zis omul lui Dumnezeu: "Domnul poate să-ți dea mai mult decât atâta".
\par 10 Atunci Amasia a ales oștirea care îi venise din țara lui Efraim, ca să se întoarcă la locul său. Dar ei s-au aprins de mânie asupra lui Iuda și s-au întors la locul lor foarte mânioși.
\par 11 Iar Amasia a prins curaj și și-a luat poporul său și a plecat la bătălie cu el în Valea Sării și a omorât zece mii de oameni dintre fiii lui Seir;
\par 12 Și alte zece mii de oameni vii i-au luat ca robi fiii lui Iuda și i-au dus pe vârful unei stânci și i-au aruncat de pe vârful stâncii. și toți s-au zdrobit cu desăvârșire.
\par 13 Iar oștirea pe care Amasia o trimisese înapoi, ca să nu meargă cu el la război, a năvălit asupra cetăților lui Iuda din Samaria până la Bet-Horon și au ucis în ele trei mii de oameni, luând foarte multă pradă.
\par 14 Iar Amasia după ce a lovit pe Edomiți, a adus, la întoarcerea lui acasă, pe zeii fiilor lui Seir și i-a așezat să-i aibă ca dumnezei pentru el și s-a închinat lor și i-a tămâiat.
\par 15 Atunci Domnul S-a făcut foc de mânie asupra lui Amasia și a trimis pe un prooroc și acela i-a zis: "De ce ai căutat pe dumnezeii poporului acestuia, care n-au fost în stare să-și scape poporul din mâna ta?"
\par 16 Când el i-a grăit aceste vorbe, regele i-a răspuns: "Te-a pus cineva să fii sfătuitor regelui? Stăpânește-te, ca nu cumva să fii ucis". Și s-a oprit proorocul și a zis: "Știu că Dumnezeu a hotărât să te piardă, pentru că ai făcut acest lucru și nu ai ascultat sfatul meu!"
\par 17 Atunci s-a sfătuit Amasia, regele lui Iuda, cu ai săi și a trimis la Ioaș, fiul lui Ioahaz, fiul lui Iehu, regele lui Israel, ca să-i spună: "Vino să ne vedem unul cu altul!"
\par 18 Iar Ioaș, regele lui Israel, a trimis la Amasia, regele lui Iuda, ca să-i spună: "Spinul, care este în Liban, a trimis să-i spună cedrului, care este tot în Liban: Dă pe fiica ta fiului meu de femeie. Însă fiarele sălbatice, care sunt în Liban, au trecut pe lângă acest spin și l-au călcat.
\par 19 Tu zici: Iată eu am bătut pe Edomiți și s-a înălțat inima ta de mărire deșartă. Șezi mai bine acasă și te astâmpără; de ce să te apuci de un lucru primejdios, prin care să cazi și tu și Iuda cu tine?"
\par 20 Dar Amasia nici n-a vrut să audă, fiindcă aceasta a fost de la Dumnezeu să cadă el în mâna lui Ioaș, pentru că a căutat pe dumnezeii Edomiților.
\par 21 Și a venit Ioaș, regele lui Israel, să dea ochi unul cu altul, el și cu Amasia, regele lui Iuda, la Bet-Șemeș, care este în Iuda.
\par 22 Și Iuda a fost bătut de Israel și fiecare a fugit la cortul său.
\par 23 Și pe Amasia, regele lui Iuda, fiul lui Ioaș, fiul lui Ioahaz, l-a prins Ioaș, regele lui Israel, la Bet-Șemeș, și l-a dus la, Ierusalim și a dărâmat zidul Ierusalimului pe o întindere de patru sute de coți, de la porțile lui Efraim până la porțile din colțul cetății;
\par 24 Apoi a luat tot aurul și argintul și toate vasele care erau în templul lui Dumnezeu, sub îngrijirea lui Obed-Edom și vistieriile casei regelui, precum și oameni ca ostatici și s-a întors cu ei în Samaria.
\par 25 Amasia, fiul lui Ioaș, regele lui Iuda, a mai trăit după moartea lui Ioaș, fiul lui Ioahaz, cincisprezece ani.
\par 26 Celelalte fapte ale lui Amasia, cele dintâi și cele de pe urmă, se află scrise în cartea regilor lui Iuda și ai lui Israel.
\par 27 De pe timpul când Amasia s-a abătut de la Domnul, s-a urzit în Ierusalim o uneltire împotriva lui și a fugit la Lachiș, dar l-au omorât acolo.
\par 28 Și l-au adus pe cai și l-au înmormântat cu părinții lui în cetatea lui Iuda.

\chapter{26}

\par 1 Atunci tot poporul lui Iuda a luat pe Ozia, care era de șaisprezece ani și l-a făcut rege în locul lui Amasia, tatăl său.
\par 2 Acesta a zidit Elatul și l-a întors la Iuda, după ce a răposat regele cu părinții lui.
\par 3 Ozia era de șaisprezece ani când a fost făcut rege și a domnit cincizeci și doi de ani în Ierusalim. Pe mama lui o chema Iecolia, de loc din Ierusalim.
\par 4 Acesta a făcut lucruri care erau plăcute înaintea ochilor Domnului, întocmai așa cum făcuse și Amasia, tatăl său;
\par 5 Căci el a alergat la Dumnezeu în zilele lui Zaharia, care îl învăța frica lui Dumnezeu. și în timpul cât a alergat el la Domnul și Dumnezeu l-a ajutat de i-a mers bine.
\par 6 Într-o vreme a plecat el de s-a bătut cu Filistenii și a dărâmat zidurile cetății Gat și zidurile cetății Iabne și zidurile cetății Așdod; și a zidit cetăți în ținutul Așdodului și între Filisteni.
\par 7 Iar Dumnezeu i-a ajutat să se bată cu Filistenii și cu Arabii care locuiesc la Gur-Baal și cu Meuniții.
\par 8 Așa că chiar Amoniții îi dădeau daruri lui Ozia încât s-a dus vestea de numele lui până la hotarele Egiptului, căci ajunsese foarte puternic.
\par 9 Apoi a zidit Ozia turnuri în Ierusalim, deasupra porților din colț și deasupra porților din vale, cum și la colțul zidului și l-a întărit.
\par 10 A zidit de asemenea turnuri și în pustiu și a săpat multe fântâni, pentru că avea multe vite și pe șes și pe vale; avea lucrători de pământ și vieri în munte și pe Carmel, căci el iubea lucrarea pământului.
\par 11 A mai avut Ozia și oștire care mergea la bătălie în cete, după cum era rânduită în catagrafia alcătuită de Ieiel, scriitorul, și de Maasia, judecătorul, care se aflau sub conducerea lui Hanania, unul din căpeteniile de frunte ale regelui.
\par 12 Numărul total al capilor de familii ale războinicilor viteji era de două mii șase sute;
\par 13 Și sub conducerea lor era o putere ostășească de trei sute șapte mii cinci sute de ostași, care puteau să meargă la luptă cu bărbăție ostășească, pentru a ajuta pe rege împotriva dușmanului.
\par 14 Pentru ei a pregătit Ozia toată oastea, scuturi, sulițe, coifuri și platoșe, arcuri și pietre de praștie.
\par 15 Apoi a făcut în Ierusalim mașini, construite cu meșteșug, ca să le pună în turnuri și pe la colțurile zidurilor și să arunce cu ele săgeți și pietre mari. Și s-a dus vestea de numele lui foarte departe, pentru că în chip minunat a fost ajutat și s-a făcut puternic.
\par 16 Și când a ajuns puternic, el s-a mândrit în inima lui spre pieirea lui; și a săvârșit o nelegiuire înaintea Domnului Dumnezeului său, căci a intrat în templul Domnului, ca să tămâieze pe jertfelnicul tămâierii.
\par 17 Dar după el a intrat și Azaria preotul, împreună cu optzeci de preoți ai Domnului, oameni aleși,
\par 18 Și s-au împotrivit regelui Ozia și i-au zis: "Nu ți-e dat ție, Ozia, să tămâiezi înaintea Domnului, ci preoților, fiilor lui Aaron, care sunt sfințiți să tămâieze; ieși din locașul sfânt, căci ai făcut o fărădelege și nu-ți va fi de loc spre cinste acest lucru înaintea Domnului Dumnezeu".
\par 19 Ozia însă s-a supărat pe ei; și cum ținea în mâna lui cădelnița, ca să tămâieze, deodată s-a ivit lepra pe fruntea lui în fața preoților, în templul Domnului, dinaintea altarului tămâierii.
\par 20 Iar dacă s-au uitat la el cu luare aminte Azaria arhiereul și toți preoții, iată el avea pe fruntea lui lepră. Și l-au silit să iasă de acolo; dar și el însuși s-a grăbit să iasă, căci îl lovise Domnul.
\par 21 Și a fost lepros regele Ozia până în ceasul morții lui; după aceea el a trăit ascuns într-o casă osebit, fiind oprit de a mai intra în templul Domnului. Iar îngrijirea peste casa regelui și cârmuirea poporului țării a ținut-o Iotam, fiul lui.
\par 22 Celelalte fapte ale lui Ozia, cele dintâi și cele de pe urmă, le-a scris Isaia proorocul, fiul lui Amos.
\par 23 Și a răposat Ozia cu părinții lui și l-au înmormântat la un loc cu părinții lui în câmpul cu mormintele regilor, că ziceau: El a fost lepros. Și în locul lui a fost făcut rege Iotam, fiul său.

\chapter{27}

\par 1 Iotam era de douăzeci și cinci de ani când a fost făcut rege și a domnit șaisprezece ani în Ierusalim. Pe mama lui o chema Ierușa, fiica lui Sadoc.
\par 2 Acesta a făcut lucruri plăcute înaintea ochilor Domnului, întocmai cum a făcut și Ozia, tatăl său, numai că n-a intrat, ca el, în locașul sfânt al Domnului. Și poporul continua încă să se strice.
\par 3 El a zidit porțile cele de sus ale templului Domnului Și a făcut multe adaosuri la zidul Ofel.
\par 4 La fel a zidit și cetăți pe muntele lui Iuda, iar în păduri, cetățui și turnuri.
\par 5 Iotam a avut război cu regele Amoniților și i-a biruit. În acel an Amoniții i-au dat o sută de talanți de argint, zece mii de core de grâu și zece mii core de orz. Această dare i-au dat-o Amoniții și în al doilea și în al treilea an.
\par 6 Și s-a făcut Iotam așa de puternic, pentru că și-a îndreptat căile sale numai înaintea feței Domnului Dumnezeului său.
\par 7 Celelalte fapte ale lui Iotam, toate războaiele sale și purtarea sa se găsesc scrise în cartea regilor lui Israel și a regilor lui Iuda.
\par 8 El era de douăzeci și cinci de ani când a fost făcut rege și a domnit în Ierusalim șaisprezece ani.
\par 9 Apoi a răposat Iotam cu părinții săi, și l-au înmormântat în cetatea lui David. Iar în locul lui a fost făcut rege Ahaz, fiul său.

\chapter{28}

\par 1 Ahaz era de douăzeci de ani când a fost făcut rege și a domnit șaisprezece ani în Ierusalim; însă el n-a făcut lucruri plăcute înaintea ochilor Domnului, precum făcuse David, strămoșul lui,
\par 2 Ci a mers pe urmele regilor lui Israel și a făcut chiar chipuri turnate de baali.
\par 3 A săvârșit tămâieri în valea fiilor lui Hinom și a trecut pe fiii săi prin foc făcând urâciunile popoarelor pe care le alungase Domnul dinaintea fiilor lui Israel.
\par 4 A adus jertfe și tămâieri pe înălțimi, pe dealuri și pe sub orice copac înfrunzit.
\par 5 De aceea l-a dat Domnul Dumnezeu în mâna regelui Sirienilor, care l-au lovit și i-au luat o mulțime de robi și i-au dus în Damasc. De asemenea a mai fost dat și în mâna regelui lui Israel, care și acela a pricinuit mari pierderi regatului său;
\par 6 Căci Pecah, fiul lui Remalia, regele lui Israel, a ucis într-o singură zi o sută douăzeci de mii în Iuda, toți numai oameni de război, pentru că aceia părăsiseră pe Domnul Dumnezeul părinților lor.
\par 7 Iar Zicri, un viteaz din Efraim, a ucis pe Maaseia, fiul regelui, pe Azricam, căpetenia curții, și pe Elcana, care era al doilea după rege.
\par 8 Și au luat fiii lui Israel de la frații lor din Iuda două sute de mii de femei, băieți și fete, ca robi; de asemenea au luat de la ei și multă pradă și au dus prada în Samaria.
\par 9 Acolo însă se afla un prooroc al Domnului, care se numea Oded. Acesta a ieșit înaintea oștirii care venea la Samaria și le-a zis: "Iată Domnul Dumnezeul părinților voștri, fiind mâniat pe Iuda, i-a dat în mâinile voastre, și voi i-ați ucis cu o sălbăticie care a ajuns până la cer.
\par 10 Și acum vă gândiți să vă faceți robi și roabe pe fiii lui Iuda și ai Ierusalimului.
\par 11 Așadar, ascultați-mă și duceți înapoi pe robii care i-ați luat de la frații voștri; căci altfel flacăra mâniei Domnului va fi peste voi".
\par 12 Atunci s-au sculat unii din căpeteniile fiilor lui Efraim: Azaria, fiul lui Iohanan, Berechia, fiul lui Meșilemot, Iezechia, fiul lui Șalum și Amasa, fiul lui Hadlai, împotriva celor care se întorceau din război
\par 13 Și le-au zis: "Să nu aduceți pe robi aici, pentru că am fi vinovați înaintea Domnului. Vreți să mai adăugați la păcatele noastre și la vinovățiile noastre? Căci și așa este mare vinovăția noastră și văpaia mâniei Domnului este peste Israel".
\par 14 Atunci ostașii au luat pe robi și prăzile înaintea căpeteniilor oștirii și a întregii adunări.
\par 15 Și s-au sculat bărbații amintiți mai sus, au luat pe robi și din pradă au îmbrăcat pe toți cei goi, le-au dat haine și încălțăminte, i-au hrănit, i-au adăpat și i-au uns cu untdelemn, și pe toți care erau slabi i-au pus pe asini și i-au dus la Ierihon, cetatea finicilor, la frații lor; iar ei s-au întors apoi la Samaria.
\par 16 În acea vreme a trimis regele Ahaz la regele Asirienilor, ca să-l ajute,
\par 17 Că Edomiții iar au venit și au ucis pe mulți din Iuda și au luat robi;
\par 18 Asemenea și Filistenii năvăliseră în cetățile din ținuturile de la șes și de la miazăzi ale lui Iuda și luaseră Bet-Șemeșul, Aialonul, Ghederotul și Soco cu satele care țineau de el, Timna cu satele care țineau de ea, și Ghimzo cu satele lui și se așezaseră acolo.
\par 19 Așa smerise Domnul pe Iuda din pricina lui Ahaz, regele lui Iuda, pentru că dusese el pe Iuda la destrăbălare și păcătuise greu înaintea Domnului.
\par 20 Tiglatfalasar, regele Asiriei, a venit într-adevăr la el; însă în loc să-l ajute, i-a făcut greutăți,
\par 21 Pentru că Ahaz luase vistieriile din templul Domnului și din casa regelui și de la căpeteniile poporului și le dăduse regelui Asiriei, dar nu spre bine.
\par 22 Căci el și atunci, când se afla în strâmtorare, n-a încetat a săvârși fărădelege înaintea Domnului. Așa era regele Ahaz.
\par 23 El a adus jertfe dumnezeilor din Damasc, crezând că ei l-au bătut, și zicea: "Fiindcă dumnezeii regilor Sirienilor le-au ajutat lor, le voi aduce jertfă, și ei îmi vor ajuta și mie". Însă ei au fost pricina căderii lui și pricina căderii a tot Israelul.
\par 24 Ahaz a strâns vasele templului lui Dumnezeu, le-a sfărâmat și a încuiat ușile templului Domnului; și și-a făcut jertfelnice pe la toate colțurile în Ierusalim;
\par 25 Și prin toate cetățile lui Iuda a făcut locuri înalte ca să tămâieze la alți dumnezei. Prin aceasta a mâniat pe Domnul Dumnezeul părinților săi.
\par 26 Celelalte fapte ale lui și toate căile lui, cele dintâi și cele de pe urmă, se găsesc scrise în cartea regilor lui Iuda și ai lui Israel.
\par 27 Și a răposat Ahaz cu părinții săi și l-au înmormântat în cetate, în Ierusalim; dar nu l-au pus în gropnițele regilor lui Israel. În locul lui s-a făcut rege Iezechia, fiul său.

\chapter{29}

\par 1 Iezechia a fost făcut rege când era de douăzeci și cinci de ani și a domnit douăzeci și nouă de ani în Ierusalim. Pe mama lui o chema Abia și era fiica lui Zaharia.
\par 2 Acesta a făcut lucruri plăcute înaintea ochilor Domnului, după cum făcuse și David, strămoșul lui.
\par 3 În anul întâi al domniei sale, în luna întâi, a descuiat el ușile templului Domnului și le-a înnoit.
\par 4 Apoi a poruncit să vină preoții și leviții; pe aceștia i-a adunat în locul deschis dinspre răsărit,
\par 5 Și le-a zis: "Ascultați-mă, leviți! Să vă curățiți acum înșivă prin jertfe și să sfințiți templul Domnului Dumnezeului părinților voștri și să aruncați necurățenia din locul sfânt afară.
\par 6 Căci părinții voștri au săvârșit fărădelegi și au făcut lucruri care nu erau plăcute ochilor Domnului Dumnezeului nostru, pe Care L-au părăsit, și-au abătut fețele de la locașul Domnului și s-au întors cu spatele la el;
\par 7 Ba și ușile pridvorului le-au încuiat, au stins candelele și n-au mai tămâiat tămâie și nici n-au mai adus arderi de tot în locașul sfânt al Domnului lui Israel.
\par 8 De aceea s-a coborât mânia Domnului peste Iuda și peste Ierusalim și i-a dat El pustiirii, batjocurii și rușinii, cum vedeți și voi singuri cu ochii voștri.
\par 9 Iată părinții noștri au căzut de ascuțișul sabiei, iar fiii noștri, fetele noastre și femeile noastre se găsesc din pricina aceasta în robie până astăzi, într-o țară care nu este a lor.
\par 10 Acum însă mâ-am pus în gând să fac legământ cu Domnul Dumnezeul lui Israel, ca să-Și întoarcă iuțimea mâniei Sale de la noi.
\par 11 Fiii mei, să nu pregetați nicidecum, căci pe voi v-a ales Domnul, ca să-I stați înaintea feței Lui, să-I slujiți și să-I fiți slujitori și aprinzători de tămâie".
\par 12 Atunci s-au sculat leviții: Mahat, fiul lui Amasia și Ioil, fiul lui Azaria, dintre fiii lui Cahat; Chiș, fiul lui Abdi și Azaria, fiul lui Iehaleleel, din seminția lui Merari; Ioah, fiul lui Zima și Eden, fiul lui Ioah, din neamul lui Gherșon;
\par 13 Șimri și Ieiel din fiii lui Elițafan; Zaharia și Matania, din fiii lui Asaf;
\par 14 Iehiel și Șimei, din fiii lui Eman; Șemaia și Uziel din fiii lui Iedutun.
\par 15 Aceștia au adunat pe frații lor și s-au curățit prin aduceri de jertfe; apoi s-au dus, după porunca regelui, să curețe și templul Domnului, după cuvintele Domnului.
\par 16 Și au intrat preoții înăuntrul templului Domnului pentru curățire și au scos afară în curtea templului tot ce au găsit necurat în locașul sfânt al Domnului; iar leviții au luat acestea ca să le scoată afară la pârâul Chedron.
\par 17 Apoi au început a face sfințirea în ziua întâi a lunii întâi și în ziua a opta a aceleiași luni au intrat în pridvorul templului Domnului. Templul Domnului l-au sfințit opt zile; în ziua a șaisprezecea din luna întâi l-au terminat.
\par 18 După aceea s-au dus la regele Iezechia acasă și i-au spus: "Noi am curățit templul Domnului și jertfelnicul cel pentru arderi de tot, cu toate vasele lui; masa pentru pâinile punerii înainte, cu toate vasele ei;
\par 19 Și toate vasele pe care le aruncase regele Ahaz în timpul domniei lui, în nelegiuirea sa, noi le-am pregătit și le-am sfințit; și iată ele sunt înaintea jertfelnicului Domnului".
\par 20 Atunci s-a sculat regele Iezechia foarte de dimineață, a adunat căpeteniile cetății și s-a dus la templul Domnului.
\par 21 Și au adus șapte viței, șapte berbeci, șapte miei și șapte țapi, ca jertfă pentru păcat: pentru regat, pentru locașul de sfințire și pentru Iuda. A poruncit apoi preoților, fiilor lui Aaron, ca să-i aducă arderi de tot pe jertfelnicul Domnului.
\par 22 Au junghiat vitei și au luat preoții sângele și au stropit cu el jertfelnicul; au junghiat berbecii și au stropit cu sângele lor jertfelnicul; au junghiat mieii și au stropit cu sângele lor de asemenea jertfelnicul.
\par 23 Apoi au adus țapii cei pentru păcat înaintea regelui și a adunării; aceștia și-au pus mâinile peste ei.
\par 24 Și i-au junghiat preoții și au curățit cu sângele lor jertfelnicul pentru ștergerea păcatelor întregului Israel, căci regele poruncise să se aducă arderi de tot și jertfă pentru păcat, pentru sine și pentru tot Israelul.
\par 25 Apoi a așezat în templul Domnului leviți cu chimvale, cu harpe și cu chitare, după rânduiala lui David și a lui Gad, văzătorul regelui, și a lui Natan proorocul, fiindcă de către Domnul se așezase această rânduială prin proorocii Lui.
\par 26 Și stăteau leviții cu instrumentele de cântare ale lui David și preoții cu trâmbițele.
\par 27 Și a poruncit Iezechia să se aducă arderi de tot pe jertfelnic. Și în timpul când a început arderea de tot, a început și cântarea pentru Domnul cu trâmbițele și cu instrumentele de cântare ale lui David, regele lui Israel.
\par 28 Și toată adunarea a făcut rugăciuni, iar cântăreții au cântat din fluiere, până s-a terminat arderea de tot.
\par 29 Iar după terminarea arderii de tot, regele și toți care se aflau cu el s-au plecat și s-au închinat.
\par 30 Iezechia regele și căpeteniile au zis către leviți, să slăvească pe Domnul cu cuvintele lui David și ale lui Asaf văzătorul, și ei L-au slăvit cu mare bucurie și s-au plecat la pământ și s-au închinat.
\par 31 Apoi a început Iezechia a grăi și a zis: "Acum v-ați sfințit pe voi Domnului; apropiați-vă și aduceți jertfe și prinoase de mulțumire în templul Domnului". Și a adus toată adunarea jertfe și prinoase de laudă și de împăcare, și cel pe care-l lăsa inima, arderi de tot.
\par 32 Numărul vitelor pentru arderile de tot, aduse de cei ce se adunaseră a fost: șaptezeci de boi, o sută de berbeci, două sute de miei. Toate acestea au fost jertfite ca arderi de tot Domnului.
\par 33 Pentru celelalte jertfe sfinte au fost șase sute de vite mari și trei mii de vite mărunte.
\par 34 Însă preoții au fost puțini și nu puteau să jupoaie toate arderile de tot; de aceea le-au ajutat și leviții, frații lor, până ce au terminat lucrul și până ce s-au sfințit preoții; căci leviții au fost mai silitori la sfințire decât preoții.
\par 35 Afară de aceasta a mai fost o mulțime de arderi de tot cu grăsimi de jertfe de împăcare și cu turnări de vin la arderi de tot. Așa s-a așezat la loc slujba în templul Domnului.
\par 36 Și s-a bucurat Iezechia dimpreună cu tot poporul, că Dumnezeu a potrivit așa pe popor, încât s-a făcut acest lucru, cum nici nu se așteptase.

\chapter{30}

\par 1 Apoi a trimis Iezechia prin toată țara lui Israel și a lui Iuda și a scris scrisori lui Efraim și lui Manase, ca să vină la templul Domnului la Ierusalim pentru sărbătorirea Paștilor Domnului Dumnezeului lui Israel.
\par 2 Căci la sfatul pe care-l ținuse regele în Ierusalim, împreună cu căpeteniile, chibzuiseră ca să sărbătorească Paștile în luna a doua,
\par 3 Din pricină că nu le putuse sărbători la timpul lor, pentru că nu erau nici preoți sfințiți în număr de ajuns și nici poporul nu se strânsese la Ierusalim.
\par 4 Acest lucru a plăcut regelui și întregii adunări.
\par 5 Și au hotărât să se dea de veste la tot Israelul, de la Beer-Șeba până la Dan, ca să vină la Ierusalim pentru sărbătorirea Paștilor Domnului Dumnezeului lui Israel, căci demult nu se mai sărbătorise așa cum este scris.
\par 6 Și s-au dus trimișii cu scrisorile, care erau făcute de rege și de căpeteniile lui, prin toată țara lui Israel și a lui Iuda și, după porunca regelui, le spuneau: "Fii ai lui Israel, întoarceți-vă la Domnul Dumnezeul lui Avraam și al lui Isaac și al lui Israel, și Se va întoarce și El la cei rămași dintre voi, care ați mai scăpat din mâna regilor Asiriei.
\par 7 Și nu mai fiți așa ca părinții voștri și ca frații voștri, care au săvârșit fărădelegi înaintea Domnului Dumnezeului părinților lor, și El i-a dat pustiirii, precum vedeți.
\par 8 Nici să nu vă mai țineți de acum tari la cerbice, ca părinții voștri, ci supuneți-vă Domnului și venin la locașul de sfințire al Lui, pe care El l-a sfințit pe veci; și slujiți Domnului Dumnezeului vostru și El Își va abate flacăra mâniei Sale de la voi.
\par 9 Când vă veți întoarce la Domnul atunci frații voștri și fiii voștri au să găsească și ei milă la acei care i-au luat pe ei robi și se vor întoarce în țara aceasta; căci bun și îndurat este Domnul Dumnezeul vostru și nu-și va întoarce fața de la voi, dacă vă veți întoarce la El".
\par 10 Și au mers trimișii din cetate în cetate prin ținutul lui Efraim și al lui Manase și până la acela al Zabulonului; aceia râdeau, bătându-și joc de ei.
\par 11 Totuși unii din seminția lui Așer, a lui Manase și a lui Zabulon s-au smerit și au venit la Ierusalim.
\par 12 Și a fost mâna Domnului peste Iuda, Care le-a dăruit o singură inimă, ca să împlinească porunca regelui și a căpeteniilor, după cuvântul Domnului.
\par 13 Și s-a adunat la Ierusalim mulțime de popor pentru sărbătorirea praznicului azimelor, în luna a doua și adunarea a fost foarte mare.
\par 14 Și s-au sculat și au răsturnat jertfelnicele care erau în Ierusalim și toate jertfelnicele pe care se săvârșeau tămâierile pentru idoli, le-au sfărâmat și le-au aruncat în pârâul Chedron.
\par 15 Apoi au junghiat mielul Paștilor în ziua a paisprezecea a lunii a doua. Preoții și leviții, rușinându-se, s-au sfințit și au adus arderi de tot în templul Domnului.
\par 16 Apoi au stat la locul lor, după rânduiala pe care o aveau prin legea lui Moise, omul lui Dumnezeu. Preoții stropeau cu sângele pe care-l luau din mâna leviților.
\par 17 Fiindcă în adunare erau mulți din cei care nu erau curați, de aceea, pentru cei necurățiți, leviții junghiau mielul Paștilor
\par 18 Și mulți din popor, din seminția lui Efraim, a lui Manase, a lui Isahar și a lui Zabulon, care nu se sfințiseră, au mâncat Paștile împotriva Scripturii.
\par 19 Dar Iezechia s-a rugat pentru ei, zicând: "Domnul cel bun să ierte pe tot cel ce și-a îndreptat inima să caute pe Domnul Dumnezeul părinților săi, deși ei n-au curățirea cerută pentru cele sfinte!"
\par 20 Și a ascultat Domnul pe Iezechia și a iertat poporul.
\par 21 Și au sărbătorit fiii lui Israel, care s-au aflat în Ierusalim, sărbătoarea azimilor șapte zile cu mare veselie. În fiecare zi leviții și preoții lăudau pe Domnul cu instrumente de cântare făcute spre preaslăvirea Domnului.
\par 22 Apoi a grăit Iezechia către toți leviții care au bună pricepere la serviciul Domnului, după inima lor. Și au mâncat azimile de sărbătoare șapte zile, aducând jertfe de pace și slăvind pe Domnul Dumnezeul părinților lor.
\par 23 Și a hotărât toată adunarea ca să mai sărbătorească alte șapte zile, și le-au petrecut pe acestea cu veselie,
\par 24 Pentru că Iezechia, regele lui Iuda, dăduse celor ce se adunaseră o mie de viței și zece mii de vite mărunte; căpeteniile dăduseră de asemenea celor ce se adunaseră o mie de viței și zece mii de vite mărunte, și mulți preoți se sfințiseră.
\par 25 Și s-a veselit toată obștea lui Iuda și preoții și leviții, toată adunarea și străinii care veniseră din țara lui Israel și locuiau în Iuda.
\par 26 Și a fost veselie mare în Ierusalim, pentru că din zilele lui Solomon, fiul lui David, regele lui Israel, nu se mai făcuse nici o veselie ca aceasta în Ierusalim.
\par 27 Apoi s-au sculat preoții și leviții și au binecuvântat poporul; și glasul lor a fost auzit, căci rugăciunea lor s-a suit până la locașul cel sfânt al Domnului, Care este în ceruri.

\chapter{31}

\par 1 După ce s-au terminat toate acestea, s-au dus toți Israeliții care se găseau acolo în cetățile lui Iuda și au sfărâmat idolii, au tăiat Așerele și au stricat locurile înalte și jertfelnicele din Iuda și din tot pământul lui Veniamin, al lui Efraim și al lui Manase, până la margini. Apoi s-au întors toți fiii lui Israel, fiecare la moșia sa, în cetățile lor.
\par 2 Iar Iezechia a așezat cetele de preoți și de leviți, după împărțirea lor, pe fiecare la slujba sa, pe care o avea de preot sau de levit, pentru ca să facă slujbe, laude și doxologii, la timpul arderilor de tot și al jertfelor de împăcare, la porțile curții templului Domnului.
\par 3 De asemenea a hotărât regele o parte din averea sa pentru arderile de tot, adică pentru arderile de tot de dimineața și de seara; pentru arderile de tot din ziua odihnei, de la lunile noi și de la sărbători, după cum este scris în legea Domnului.
\par 4 Apoi a poruncit el și poporului din Ierusalim să dea preoților și leviților întreținerea hotărâtă, pentru ca ei să fie mai cu tragere de inimă la împlinirea legii Domnului.
\par 5 Când s-a adus la cunoștință tuturor această poruncă, atunci fiii lui Israel au adus prinoase de pâine, de vin, de untdelemn, de miere și din toate roadele țarinii, din belșug; au adus de asemenea din belșug și zeciuieli din toate.
\par 6 Dar și Israeliții și cei din Iuda, care locuiau prin cetățile lui Iuda, au adus asemenea zeciuieli din vitele mari și din vitele mărunte, cum și zeciuieli din jertfele pe care le făgăduiseră ei Domnului Dumnezeului lor, și le-au făcut grămezi.
\par 7 Grămezile au început să le facă în luna a treia, iar în luna a șaptea le-au sfârșit.
\par 8 Atunci a venit Iezechia și cu căpeteniile lui și au văzut grămezile și au binecuvântat pe Domnul și poporul lui Israel.
\par 9 Apoi a întrebat Iezechia pe preoți și pe leviți despre aceste grămezi: "Pentru ce grămezile stau așa?"
\par 10 Și a răspuns Azaria arhiereul, care era din casa lui Țadoc, și a zis: "De când a început a se aduce prinoase în templul Domnului, de atunci am mâncat și noi de ne-am săturat și a mai și rămas din belșug, pentru că Domnul a binecuvântat pe poporul Său. Și această grămadă mare este din cele ce au rămas".
\par 11 Atunci Iezechia a poruncit să se pregătească în templul Domnului cămări; și dacă le-au pregătit,
\par 12 Au cărat acolo prinoasele de pârgă, zeciuielile și darurile, cu toată credincioșia. Și a pus ca ispravnic peste ele pe Conania levitul și pe fratele său Șimei, ca al doilea.
\par 13 Iar pe Iehiel, Azazia, Nahat, Asael, Ierimot, Iozabad, Eliel, Ismachia, Mahat și pe Benaia, i-a pus ca supraveghetori sub mâna lui Conania și a lui Șimei, fratele său, după cum rânduiseră regele Iezechia și Azaria, căpetenia templului lui Dumnezeu.
\par 14 Core, fiul lui Imna levitul, care era portar în partea de răsărit, era supraveghetor peste prinoasele care se aduceau de bunăvoie lui Dumnezeu; ca să împartă prinoasele de pârgă aduse Domnului și lucrurile mai de seamă care se sfințiseră.
\par 15 Sub mâna lui se aflau Eden, Miniamin, Iosua, Șemaia, Amaria și Șecania, în cetățile preoților, ca să împartă cu credincioșie fraților lor părțile ce li se cuveneau, precum celui mare, așa și celui mic,
\par 16 În afara celor scriși în catagrafii, tuturor celor de parte bărbătească de la trei ani în sus, tuturor celor care intrau în templul Domnului la slujbă, pentru trebuințele zilnice, după dregătoriile lor și după cete,
\par 17 Precum și preoților înscriși în catagrafie, după familiile lor, și leviților de la douăzeci de ani în sus, după dregătoriile lor și după cetele lor,
\par 18 Celor înscriși în catagrafie și la toți nevârstnicii lor, cu femeile lor, cu băieții lor și cu fetele lor, adică la toată obștea de neam preoțesc; căci ei se arătaseră cu toată credincioșia la sfânta slujbă.
\par 19 Pentru fiii lui Aaron, care erau preoți în satele dimprejurul cetăților lor, erau puși anume bărbați, la fiecare cetate ca să le împartă tuturor celor de parte bărbătească dintre preoți părțile ce li se cuveneau; precum și la toți leviții înscriși în catagrafie.
\par 20 Așa a făcut Iezechia în tot Iuda. El a făcut lucruri care erau bune, drepte și adevărate în fața Domnului Dumnezeului său.
\par 21 Și tot lucrul pe care l-a început pentru slujba din templul lui Dumnezeu, pentru păzirea legii și a poruncilor, fiind cu gândul la Dumnezeul său, el l-a făcut cu toată tragerea sa de inimă și a avut spor la ei.

\chapter{32}

\par 1 După atâtea lucruri și atâta credincioșie, a venit Sanherib, regele Asirienilor, și a intrat în Iuda și a împresurat cetățile întărite și și-a făcut planul ca să le răpească pentru el.
\par 2 Când Iezechia a văzut că a venit Sanherib cu gândul ca să lupte împotriva Ierusalimului,
\par 3 A hotărât împreună cu sfetnicii și cu vitejii săi ca să astupe izvoarele de apă care erau afară din cetate și aceștia l-au ajutat.
\par 4 Atunci s-a adunat o mulțime de popor și a astupat toate izvoarele și pârâul care curgea prin mijlocul țării, zicând: Să nu vină regele Asiriei și găsind apă multă, să se întărească.
\par 5 Și s-a întărit Iezechia și a făcut la loc tot zidul care se stricase și a ridicat turnuri, a zidit pe din afară un alt zid, a înălțat zidul cetății lui David și a pregătit o mulțime de arme și de scuturi.
\par 6 A pus căpetenii ostășești peste popor și i-a adunat pe lângă el în locul larg de la poarta văii și le-a grăit pe inima lor zicând:
\par 7 "Întăriți-vă și îmbărbătați-vă, să nu vă temeți, nici să vă înfricoșați de fața regelui și de fața întregului neam care este cu el, căci cu noi sunt mai mulți decât cu el.
\par 8 Cu el sunt brațe de carne, iar cu noi este Domnul Dumnezeul nostru Care ne ajută și Care luptă în bătăliile noastre". Și s-a încurajat poporul de cuvintele lui Iezechia, regele lui Iuda.
\par 9 După aceasta Sanherib, regele Asiriei, care era cu toată oștirea lui în fața Lachișului, a trimis pe niște slujbași ai săi la Ierusalim, la Iezechia, regele lui Iuda, și la toți cei din Iuda care erau în Ierusalim, ca să le spună:
\par 10 "Așa zice Sanherib, regele Asiriei: Pe ce vă bizuiți de ședeți închiși în cetatea Ierusalimului?
\par 11 Nu vedeți că Iezechia vă amăgește, ca să vă dea morții prin foame și prin sete, zicând: Domnul Dumnezeul nostru ne va scăpa din mâna regelui Asiriei?
\par 12 Nu vedeți că acest Iezechia a desființat locurile înalte ale lui și jertfelnicele lui și a spus lui Iuda și Ierusalimului: Să vă închinați înaintea unui singur jertfelnic și numai pe el să tămâiați?
\par 13 Oare nu știți ce am făcut eu și părinții mei tuturor popoarelor țărilor? Putut-au oare dumnezeii popoarelor acestor țări să le scape țara lor din mâna mea?
\par 14 Care din toți dumnezeii popoarelor, pe care le-au pierdut părinții mei, a putut să-și scape poporul din mâna mea? Cum dar Dumnezeul vostru vă va putea scăpa din mâna mea?
\par 15 Și acum păziți-vă, ca să nu vă mai amăgească Iezechia, nici să nu vă mai înduplece astfel! Să nu-l credeți! Căci dacă n-a fost în stare nici un dumnezeu de al nici unui popor și regat să-și scape poporul său din mâna mea și din mâna părinților mei, atunci nici Dumnezeul vostru nu vă va scăpa din mâna mea".
\par 16 Încă și altele multe au vorbit robii lui împotriva Domnului Dumnezeu și împotriva lui Iezechia, robul Lui.
\par 17 Ba scrisese el și scrisori prin care hulea pe Domnul Dumnezeul lui Israel și în care grăia împotriva Lui astfel de cuvinte: "Precum dumnezeii popoarelor pământului n-au scăpat pe popoarele lor din mâna mea, așa nici Dumnezeul lui Iezechia nu va scăpa pe poporul Său din mâna mea".
\par 18 Și strigau cu glas tare în limba Iudeilor către poporul din Ierusalim, care era pe zid, ca să-i îngrozească și să-i sperie și să le ia cetatea.
\par 19 Ei vorbeau despre Dumnezeul Ierusalimului, ca despre dumnezeii popoarelor pământului, care sunt lucruri de mâini omenești.
\par 20 Atunci s-a rugat regele Iezechia și Isaia proorocul, fiul lui Amos, și au strigat cu glas mare la cer.
\par 21 Și a trimis Domnul pe un înger care a pierdut pe tot viteazul și războinicul și căpetenia și generalul din tabăra regelui Asiriei, încât acesta s-a întors cu rușine în țara sa; și când a intrat în casa dumnezeului său, l-au ucis cu sabia acolo fiii lui.
\par 22 Așa a scăpat Domnul pe Iezechia și pe locuitorii Ierusalimului din mâna lui Sanherib, regele Asiriei, și din mâna tuturor celorlalți și i-a apărat din toate părțile.
\par 23 Atunci mulți au adus daruri Domnului în Ierusalim și lucruri scumpe lui Iezechia, regele Iudei, care după aceasta a câștigat în ochii tuturor popoarelor mărire mare.
\par 24 În zilele acelea s-a îmbolnăvit Iezechia de moarte și s-a rugat Domnului și Domnul l-a auzit și i-a dat semn.
\par 25 Însă Iezechia n-a fost recunoscător pentru binefacerea care i s-a făcut, căci s-a semețit în inima lui. și a căzut mânia lui Dumnezeu peste el și peste Iuda și peste Ierusalim.
\par 26 Dar îndată ce Iezechia s-a smerit pentru mândria inimii lui și cu el împreună și locuitorii Ierusalimului, mânia Domnului nu s-a mai coborât asupra lor în zilele lui Iezechia.
\par 27 Iezechia a avut bogăție și mărire foarte mare; și și-a făcut vistierii de păstrat argint, aur, pietre scumpe, aromate, scuturi și tot felul de vase prețioase.
\par 28 A făcut de asemenea și hambare pentru roade: grâu, vin și untdelemn; așezări și iesle pentru tot felul de vite și staule pentru turme.
\par 29 Și-a zidit și cetăți și a avut o mulțime de vite mari și de vite mărunte, pentru că Dumnezeu îi dăduse lui foarte multă avere.
\par 30 Tot acest Iezechia a astupat gura de sus a apelor Ghihonului și le-a făcut să curgă în jos prin partea de apus a cetății lui David. Și la tot lucrul lui, Iezechia a lucrat cu spor.
\par 31 Când trimișii regelui Babilonului au venit la el să-l întrebe pentru semnul care se săvârșise în țară, atunci l-a părăsit Dumnezeu, ca să-l încerce și să cunoască tot ce avea el în inima lui,
\par 32 Celelalte fapte ale lui Iezechia și milosteniile lui sunt scrise în vedenia lui Isaia-proorocul, fiul lui Amos, și în cartea regilor lui Iuda și Israel.
\par 33 Apoi a răposat Iezechia cu părinții lui și l-au îngropat în rândul de sus al mormintelor fiilor lui David și tot Iuda și locuitorii din Ierusalim i-au făcut mare cinste la moartea lui. Iar în locul lui a fost făcut rege Manase, fiul său.

\chapter{33}

\par 1 Manase era de doisprezece ani când s-a făcut rege și a domnit cincizeci și cinci de ani în Ierusalim.
\par 2 Acesta a făcut lucruri neplăcute înaintea ochilor Domnului, umblând după urâciunile popoarelor izgonite de Domnul din fața fiilor lui Israel.
\par 3 Căci a făcut din nou locurile înalte pe care le sfărâmase Iezechia, tatăl său, și a așezat jertfelnice pentru baali, a făcut Așere, s-a închinat la toată oștirea cerească și i-a slujit;
\par 4 A făcut jertfelnice și în templul Domnului, despre care Domnul zisese: "În Ierusalim va fi numele Meu în veci";
\par 5 A zidit jertfelnice pentru toată oștirea cerească în amândouă curțile templului Domnului.
\par 6 Tot el a trecut prin foc pe fiii săi în valea Ben-Hinom și a făcut vrăjitorie, farmece și magie; a adus oameni care chemau duhurile morților și fermecători; și a înmulțit relele împotriva Domnului, mâniindu-L.
\par 7 Apoi a făcut un idol cioplit și l-a așezat în templul lui Dumnezeu, deși Dumnezeu grăise către David și către Solomon, fiul lui: "În templul acesta și în Ierusalimul pe care l-am ales dintre toate semințiile lui Israel, îmi voi pune numele Meu în veac;
\par 8 Mai mult, nu voi îngădui ca piciorul lui Israel să pășească afară din pământul acesta, pe care l-am dat părinților lor, numai dacă ei vor fi stăruitori în a face tot ce le-am spus, după toată legea, rânduielile și poruncile date prin Moise".
\par 9 Însă Manase a dus pe Iuda și pe locuitorii Ierusalimului la atâta rătăcire, încât ei au săvârșit mai rău decât acele popoare pe care Domnul le stârpise din fața fiilor lui Israel.
\par 10 De aceea a grăit Domnul lui Manase și poporului său, dar ei n-au ascultat.
\par 11 De aceea a adus Domnul peste ei pe căpeteniile armatei regelui Asiriei, care l-au prins pe Manase cu arcanul și l-au legat cu cătușe de fier și l-au dus la Babilon.
\par 12 Și în strâmtorarea sa, el a căutat fața Domnului Dumnezeului său și s-a smerit foarte înaintea Dumnezeului părinților săi.
\par 13 Iar dacă s-a rugat, Dumnezeu l-a auzit și i-a ascultat rugăciunea lui și l-a adus înapoi la Ierusalim, în regatul său. Și a cunoscut Manase că Domnul este Dumnezeul cel adevărat.
\par 14 După aceea a zidit el zidul cel din afară al cetății lui David, în partea de apus a Ghihonului, pe vale și până la poarta peștilor, a înconjurat Ofelul, l-a făcut înalt și a așezat căpetenii de oaste prin toate cetățile întărite ale lui Iuda.
\par 15 Apoi a doborât pe dumnezeii cei străini și idolul cel din templul Domnului și toate capiștele pe care le zidise pe muntele templului Domnului și în Ierusalim le-a aruncat afară din cetate.
\par 16 A făcut la loc jertfelnicul Domnului și a adus pe el jertfe de împăcare și de laudă iar lui Iuda i-a spus să slujească Domnului Dumnezeului lui Israel.
\par 17 Poporul mai aducea jertfe pe locurile înalte, dar numai pentru Domnul Dumnezeul său.
\par 18 Celelalte fapte ale lui Manase și rugăciunea lui către Dumnezeul său și cuvintele văzătorilor, care i s-au grăit în numele Domnului Dumnezeului lui Israel, se găsesc scrise în istoria regilor lui Israel.
\par 19 Și rugăciunea lui și cum l-a ascultat pe el Dumnezeu și toate păcatele lui și fărădelegile lui și locurile în care el a zidit locuri înalte și a așezat chipurile Astartei și idolii, înainte de a se smeri, se găsesc scrise în istoria lui Hozai.
\par 20 Apoi a răposat Manase cu părinții săi și l-au îngropat în grădina casei lui, iar în locul lui s-a făcut rege Amon, fiul său.
\par 21 Amon era de douăzeci și doi de ani când s-a făcut rege și a domnit doi ani în Ierusalim.
\par 22 Și a făcut el lucruri neplăcute înaintea ochilor Domnului, precum făcuse și Manase, tatăl său; căci a adus jertfe la toate chipurile cioplite, pe care le făcuse Manase, tatăl său, și le-a slujit.
\par 23 Dar el nu s-a smerit înaintea feței Domnului, cum s-a smerit Manase, tatăl său; dimpotrivă, Amon și-a înmulțit păcatele sale.
\par 24 Și au uneltit slugile lui împotriva lui și l-au omorât în casa sa.
\par 25 Însă poporul țării a omorât pe toți care uneltiseră împotriva regelui Amon. Și în locul lui poporul țării a făcut rege pe Iosia, fiul său.

\chapter{34}

\par 1 Iosia era de opt ani când s-a făcut rege și a domnit treizeci și unu de ani în Ierusalim.
\par 2 Acesta a făcut lucruri plăcute înaintea ochilor Domnului și a umblat pe căile lui David, strămoșul său și nu s-a abătut nici la dreapta nici la stânga;
\par 3 Căci în anul al optulea al domniei lui, fiind încă tânăr, a început să caute pe Dumnezeul lui David, strămoșul său, iar în anul al doisprezecelea a început să curețe Iuda și Ierusalimul de locurile înalte, de Așere, de idolii ciopliți și de idolii turnați.
\par 4 A sfărâmat în fața lui jertfelnicele baalilor și idolii de pe ele; a tăiat Așerele și a prefăcut în praf idolii ciopliți și idolii turnați și praful l-a risipit pe mormintele celor ce le aduseseră jertfe;
\par 5 A ars oasele preoților pe jertfelnicele lor și a curățit Iuda și Ierusalimul.
\par 6 De asemenea și cetățile lui Manase, ale lui Efraim, ale lui Simeon, ba și pe ale seminției lui Neftali și locurile pustiite dimprejurul lor.
\par 7 Sfărâmând jertfelnicele și stricând Așerele, făcând chipurile cioplite praf și sfărâmând toți idolii din tot pământul lui Israel s-a întors la Ierusalim.
\par 8 În anul al optsprezecelea al domniei sale, după curățirea țării și a templului lui Dumnezeu, el a trimis pe Șafan, fiul lui Ațalia, pe Maaseia, căpetenia cetății și pe Ioab, fiul lui Ioahaz, cronicarul, ca să înnoiască templul Domnului Dumnezeului său.
\par 9 Aceștia au venit la Hilchia arhiereul și i-au dat argintul care se adusese în templul lui Dumnezeu și pe care leviții cei ce stăteau de pază la uși îl adunaseră din mâinile semințiilor lui Manase și ale lui Efraim și ale tuturor celorlalți Israeliți, de la toți cei din Iuda și Veniamin și de la locuitorii Ierusalimului
\par 10 Și l-au dat în mâinile meșterilor de lucrări care erau tocmiți la templul Domnului, ca să-l împartă lucrătorilor, care lucrau la templul Domnului, la repararea și consolidarea lui.
\par 11 Și ei l-au împărțit la dulgheri și la zidari, ca să cumpere pietre cioplite, bârne pentru legături și pentru pus la acoperișul clădirilor pe care le stricaseră regii lui Iuda.
\par 12 Oamenii aceștia au lucrat cinstit la lucrare și au avut ca supraveghetori peste ei pe Iahat și pe Obadia care erau leviți dintre fiii lui Merari; pe Zaharia și pe Meșulam, care erau dintre fiii lui Cahat, precum și pe toți leviții care știau să cânte cu instrumente muzicale.
\par 13 Tot aceștia erau puși și peste salahori și supravegheau pe toți lucrătorii la fiecare lucru; și dintre leviți erau scriitori, judecători și portari.
\par 14 Când au luat ei argintul care se adusese în templul Domnului, atunci Hilchia preotul a găsit cartea legii Domnului care fusese dată prin mâinile lui Moise.
\par 15 și a început Hilchia să grăiască și a zis către Șafan scriitorul: "Am găsit cartea legii în templul Domnului!" Și Hilchia a dat acea carte lui Șafan.
\par 16 Iar Șafan s-a dus cu cartea la rege și i-a dus regelui o dată cu cartea și știrea: "Tot ce ai încredințat robilor tăi se va face!
\par 17 Apoi au dus argintul care s-a găsit în templul Domnului și l-au dat în mâinile supraveghetorilor și în mâinile meșterilor de lucrări.
\par 18 După aceea Șafan, scriitorul regelui, a mai făcut cunoscut, zicând: "Hilchia preotul mi-a dat o carte". Și Șafan a citit-o înaintea regelui.
\par 19 Când a auzit regele cuvintele legii și-a rupt hainele sale.
\par 20 Și a dat poruncă regele lui Hilchia, lui Ahicam, fiul lui Șafan, lui Abdon, fiul lui Miheia, lui Șafan scriitorul, lui Asaia, slujitorul regelui, zicând:
\par 21 "Duceți-vă de întrebați pe Domnul despre mine și despre cei ce au rămas în Israel și despre acei din Iuda, pentru cuvintele cărții acesteia care s-a găsit, pentru că mare este mânia Domnului care s-a aprins peste noi, din pricină că părinții noștri n-au păzit cuvintele Domnului, ca să facă după cum este scris în cartea aceasta".
\par 22 Atunci s-a dus Hilchia și cei ce erau din partea regelui la Hulda, proorocița, femeia lui Șalum, veșmântarul, fiul lui Tochat, fiul lui Hasra. Ea locuia în despărțământul al doilea în Ierusalim și au vorbit cu ea de aceasta.
\par 23 Iar ea le-a spus: "Așa zice Domnul Dumnezeul lui Israel: Spuneți acelui om care v-a trimis la mine:
\par 24 Așa zice Domnul: Iată voi aduce nenorociri peste locul acesta și peste locuitorii lui toate blestemele, care sunt scrise în cartea aceasta pe care aii citit-o în fața regelui lui Iuda,
\par 25 Din pricină că M-au părăsit și au tămâiat pe alți dumnezei, ca să Mă supere cu toate faptele mâinilor lor. Mânia Mea se va aprinde peste acest loc și nu se va potoli.
\par 26 Iar către regele lui Iuda care v-a trimis să întrebați pe Domnul, așa să-i spuneți: Așa zice Domnul Dumnezeul lui Israel pentru cuvintele pe care tu le-ai auzit:
\par 27 Fiindcă inima ta s-a înmuiat și tu te-ai smerit înaintea lui Dumnezeu, auzind cuvintele Lui pentru locul acesta și pentru locuitorii lui și tu te-ai smerit înaintea Mea și ți-ai rupt hainele tale și ai plâns înaintea Mea și Eu te-am auzit, zice Domnul:
\par 28 Iată Eu te voi așeza la un loc cu părinții tăi și vei fi pus în mormântul tău cu pace; ochii tăi nu vor vedea nenorocirile pe care le voi aduce peste locul acesta și peste locuitorii lui". Și i-au adus regelui răspunsul acesta.
\par 29 Și a trimis regele și a adunat pe toți bătrânii lui Iuda și ai Ierusalimului,
\par 30 Și s-a dus regele la templul Domnului și, împreună cu el, tot luda și locuitorii Ierusalimului, preoții, leviții și tot poporul de la mic până la mare; și el le-a citit în auzul lor toate cuvintele cărții legământului care s-a găsit în templul Domnului.
\par 31 Și a stat regele la locul lui și a făcut legământ în fața Domnului, ca să urmeze Domnului și să păzească poruncile Lui, descoperirile și rânduielile Lui, cu toată inima și cu tot sufletul lor, ca să împlinească cuvintele legământului, care sunt scrise în cartea aceasta.
\par 32 Apoi a poruncit regele ca acest legământ să aibă tărie pentru toți cei ce se află în Ierusalim și în toată țara lui Veniamin; și au început locuitorii Ierusalimului să se poarte după legământul Domnului Dumnezeului părinților lor
\par 33 Și a alungat Iosia toate urâciunile, din toate ținuturile, cărora se închinau fiii lui Israel; și a poruncit tuturor celor care se aflau în pământul lui Israel să slujească Domnului Dumnezeului lor. Și în toate zilele vieții lui nu s-au abătut ei de la Domnul Dumnezeul părinților lor.

\chapter{35}

\par 1 În vremea aceea a sărbătorit Iosia Paștile Domnului în Ierusalim și a junghiat mielul Paștilor în ziua a paisprezecea a lunii întâi.
\par 2 A așezat pe preoți la locul lor și i-a obligat să slujească în templul Domnului;
\par 3 Iar leviților, care învățau pe toți Israeliții și care erau sfințiți pentru Domnul, le-a zis: "Puneți chivotul cel sfânt în templul pe care l-a zidit Solomon, fiul lui David, regele lui Israel; nu mai aveți nevoie să-l mai purtați pe umeri, ci slujiți acum Domnului Dumnezeului vostru și poporului Său Israel
\par 4 Și vă rânduiți după neamurile voastre părintești și după cetele voastre, cum a scris David, regele lui Israel și cum a scris Solomon, fiul lui;
\par 5 Așezați-vă în locașul sfânt după cetele voastre între fiii poporului, frații voștri, și după cetele în care sunt împărțiți leviții, după neamurile lor.
\par 6 Junghiați mielul Paștilor și vă sfințiți și-l pregătiți pentru frații voștri, ca să facă ei după cuvântul Domnului care s-a dat prin Moise".
\par 7 După aceea a dat Iosia ca dar fiilor poporului, tuturor care se aflau acolo, tot pentru jertfa Paștilor, din vite mărunte un număr de treizeci de mii de miei și de țapi tineri și trei mii de boi. Acestea erau din averea regelui.
\par 8 Căpeteniile lui au făcut de bună voie un dar poporului, preoților și leviților: Hilchia, Zaharia și Iehiel, căpeteniile templului lui Dumnezeu, au dat preoților pentru jertfa Paștilor, două mii șase sute de oi, miei și țapi, și trei sute de boi;
\par 9 Conania, Șemaia și Natanael frații lui, și Hașabia, Ieiel și Iozabad, căpeteniile leviților, au dăruit leviților pentru jertfa Paștilor cinci mii de oi și cinci sute de boi.
\par 10 Așa s-a înjghebat slujba: preoții s-au așezat la locul lor și leviții după cetele lor și după porunca regelui;
\par 11 Și au junghiat mielul Paștilor și au stropit preoții cu sânge, luându-l din mâinile leviților; iar leviții jupuiau pielea de pe animalele de jertfă.
\par 12 Apoi au rânduit cele gătite pentru arderile de tot, ca să le împartă poporului, cum era acesta împărțit în cete după familii, ca ei să le aducă Domnului, cum este scris în cartea lui Moise. Tot așa au făcut ei și cu boii.
\par 13 Și au fript mielul Paștilor pe foc, după rânduială; și jertfele cele sfinte le-au fiert în căldări, în oale și în tigăi și le-au împărțit fără multă trudă la tot poporul,
\par 14 Iar pentru ei și pentru preoți au gătit după aceasta, căci preoții, fiii lui Aaron, au fost ocupați cu aducerea de jertfe și de grăsimi până noaptea; și de aceea leviții au gătit și pentru ei și pentru preoți, fiii lui Aaron.
\par 15 Și cântăreții, fiii lui Asaf, au stat la locurile lor, după rânduiala lui David, a lui Asaf, a lui Heman și a lui Iedutun, văzătorii regelui, asemenea și portarii au stat la fiecare poartă; și ei nici nu aveau nevoie să lipsească de la slujba lor, fiindcă leviții, frații lor, găteau pentru ei.
\par 16 Așa s-a gătit în acea zi toată slujba care a fost pentru Domnul, ca să se sărbătorească Paștile și să se aducă arderile de tot pe jertfelnicul Domnului, după porunca regelui Iosia.
\par 17 Fiii lui Israel, care se aflau acolo, au prăznuit în vremea aceea Paștile și sărbătoarea azimelor, timp de șapte zile.
\par 18 Paști ca acestea însă nu mai fuseseră sărbătorite în Israel din zilele lui Samuel proorocul; nici unul dintre regii lui Israel n-a sărbătorit Paștile așa cum a sărbătorit Iosia cu preoții și leviții, cu tot Iuda și Israelul care se aflau acolo, și cu locuitorii Ierusalimului.
\par 19 Paștile acestea au fost sărbătorite în anul al optsprezecelea al domniei lui Iosia.
\par 20 După toate acestea, care le-a făcut Iosia în templul lui Dumnezeu, și cum regele Iosia a ars cu foc pe cei ce grăiesc din pântece, pe magi, capiștile, idolii și Așerele ce erau în Ierusalim și în Iuda, ca să se păzească cuvintele Domnului, scrise în cartea pe care o găsise Hilchia preotul în templul Domnului, n-a fost ca el nici unul dintre regii dinainte de el, ca să se fi întors la Domnul cu toată inima lui și cu tot sufletul lui și cu toată tăria lui, după toată legea lui Moise, și nici după el nu s-a ridicat ca el. Totuși Domnul nu și-a potolit de tot iuțimea cea mare a mâniei Lui, iuțime cu care Domnul Se mâniase pe Iuda, pentru toate fărădelegile pe care le săvârșise Manase. Și a zis Domnul: "Și pe Iuda îl voi lepăda de la fața Mea, cum am lepădat casa lui Israel; și voi lepăda cetatea Ierusalimului, pe care am ales-o, și templul despre care am zis: Numele Meu va fi acolo". Și a venit Neco, regele Egiptului, cu război asupra Carchemișului, pe Eufrat; și Iosia i-a ieșit înainte.
\par 21 În vremea aceea a trimis Neco soli la el, ca să-i spună: "Ce treabă am eu cu tine, rege al lui Israel? Nu împotriva ta vin eu acum, ci merg acolo unde am război. Și Dumnezeu mi-a poruncit să grăbesc; să nu te împotrivești lui Dumnezeu, Care este cu mine, ca să nu te piardă".
\par 22 Dar Iosia nu s-a dat îndărăt dinaintea lui, ci s-a pregătit să se lupte cu el; și n-a ascultat de cuvintele lui Neco, care erau din gura lui Dumnezeu, ci a ieșit la luptă în valea Meghido.
\par 23 Atunci arcașii au tras asupra regelui Iosia; și a zis regele slugilor sale: "Duceri-mă de aici, pentru că sunt greu rănit".
\par 24 Și l-au luat slugile lui din car și l-au așezat în alt car care-l avea el și l-au dus la Ierusalim. Și a murit și a fost înmormântat cu părinții săi. Și tot Iuda și Ierusalimul l-au jelit pe Iosia.
\par 25 Asemenea și Ieremia l-a jelit pe Iosia într-o cântare de jale. De Iosia au vorbit și toți cântăreții și toate cântărețele în cântările lor de jelire, care s-au păstrat până azi și se întrebuințează în Israel. Ele se găsesc scrise în cartea cântărilor de jelire.
\par 26 Celelalte lucrări ale lui Iosia și faptele cele bune ale lui, pe care le-a săvârșit după cele scrise în legea Domnului,
\par 27 Cum și faptele lui, cele dintâi și cele de pe urmă, se găsesc scrise în cartea regilor lui Israel și ai lui Iuda.

\chapter{36}

\par 1 Atunci a luat poporul țării pe Ioahaz, fiul lui Iosia, l-au uns și l-au făcut rege în locul tatălui său, în Ierusalim.
\par 2 Ioahaz era de douăzeci și trei de ani când s-a făcut rege și a domnit trei luni în Ierusalim. Pe mama lui o chema Hamutal și era fiica lui Ieremia din Libna. Acesta a făcut lucruri rele înaintea Domnului, întocmai cum făcuseră părinții lui. Dar faraonul Neco l-a luat legat la Ribla, în ținutul Hamatului, ca să nu mai domnească în Ierusalim.
\par 3 După ce l-a dat jos de pe tronul din Ierusalim, regele Egiptului l-a dus în Egipt și a pus pe țară o dajdie de o sută talanți de argint și un talant de aur.
\par 4 Iar peste Iuda și Ierusalim, regele Egiptului a pus rege pe Eliachim, fratele lui Ioahaz, căruia i-a schimbat numele în Ioiachim; iar pe Ioahaz, fratele lui, l-a luat Neco și l-a dus în Egipt și a murit acolo. Ioiachim i-a dat lui Faraon argintul și aurul cerut. De atunci a început țara să plătească bir după cuvântul lui Faraon și fiecare, după puterea ce avea, cerea argint și aur de la poporul țării pentru bir, care era trimis Faraonului Neco.
\par 5 Ioiachim era de douăzeci și cinci de ani când s-a făcut rege și a domnit unsprezece ani în Ierusalim. Pe mama lui o chema Zebuda și era fiica lui Pedaia din Ruma. Acesta a făcut lucruri neplăcute înaintea ochilor Domnului Dumnezeului său, cum făcuseră și părinții lui. În zilele lui a venit Nabucodonosor, regele Babilonului, asupra țării și Ioiachim i-a fost supus trei ani și apoi s-a lepădat de el. Atunci a trimis Domnul împotriva lor pe Caldei, pe tâlharii Siriei, pe tâlharii Moabiților, pe fiii lui Amon și pe cei ai Samariei și s-au retras pentru acest cuvânt, pentru cuvântul Domnului, pe care l-a grăit prin gura robilor Săi prooroci. Însă mânia Domnului tot a mai dăinuit asupra lui Iuda, ca să-l lepede de la fața Sa, pentru toate păcatele lui Manase, pe care le făcuse acesta și pentru sângele cel nevinovat pe care l-a vărsat Ioiachim, umplând Ierusalimul cu sânge nevinovat. însă Domnul tot n-a vrut să-l stârpească.
\par 6 Împotriva lui s-a ridicat Nabucodonosor, regele Babilonului, și l-a legat în cătușe de fier, ca să-l ducă la Babilon.
\par 7 Nabucodonosor a dus la Babilon și o parte din vasele templului Domnului și le-a pus în capiștea sa în Babilon.
\par 8 Celelalte fapte ale lui Ioiachim și urâciunile lui, pe care le-a făcut el și care s-au mai găsit asupra lui, sunt scrise în cartea regilor lui Israel și ai lui Iuda. Și a răposat Ioiachim cu părinții lui și a fost îngropat în Ganozai la un loc cu părinții săi și în locul lui s-a făcut rege Iehonia, fiul său.
\par 9 Iehonia era de optsprezece ani când s-a făcut rege și a domnit trei luni și zece zile în Ierusalim. Acesta a făcut lucruri neplăcute înaintea ochilor Domnului.
\par 10 După trecerea unui an, a trimis regele Nabucodonosor și a poruncit să-l aducă la Babilon împreună cu vasele cele prețioase din templul Domnului și peste Iuda și Ierusalim a pus rege pe Sedechia, fratele său.
\par 11 Sedechia era de douăzeci și unu de ani când s-a făcut rege și a domnit unsprezece ani în Ierusalim;
\par 12 Și el a făcut lucruri neplăcute înaintea ochilor Domnului Dumnezeului său. El nu s-a smerit înaintea lui Ieremia proorocul, care îi proorocea cuvintele din gura Domnului.
\par 13 El s-a răzvrătit împotriva regelui Nabucodonosor, care-l pusese să jure pe numele Domnului și s-a făcut tare de cerbice și și-a învârtoșat inima sa până într-atâta, că nu s-a mai întors la Domnul Dumnezeul lui Israel.
\par 14 La fel au păcătuit mult și toate căpeteniile preoților și ale poporului, făcând toate urâciunile păgânilor și au spurcat templul Domnului pe care îl sfințise el în Ierusalim.
\par 15 Atunci a trimis la ei Domnul Dumnezeul părinților lor pe trimișii săi foarte de dimineață, pentru că i-a fost milă de popor și de locașul Său.
\par 16 Dar ei și-au bătut joc de trimișii cei de la Dumnezeu și n-au ținut seamă de cuvintele Lui; au batjocorit pe proorocii Lui, până ce mânia Domnului s-a coborât peste poporul Lui, încât acesta n-a mai avut scăpare.
\par 17 Căci El a adus asupra lor pe regele Caldeilor și acela a omorât pe tinerii lor cu sabia în locașul cel sfânt al lor și n-a cruțat nici pe Sedechia, nici pe băieți, nici pe fete, nici pe bătrâni, nici pe cei încărunțiți; pe toți Dumnezeu i-a dat în mâna lui.
\par 18 Toate vasele din templul lui Dumnezeu, cele mari și cele mici, vistieriile templului și vistieriile regelui și ale căpeteniilor lui, toate le-a adus el în Babilon.
\par 19 Apoi a dat foc templului lui Dumnezeu, a dărâmat zidul Ierusalimului, toate cămările lui le-a ars cu foc și toate palatele cele mari le-a nimicit.
\par 20 Iar pe cei care au scăpat de sabie i-a strămutat în Babilon; și au fost ei ca robi ai lui și ai fiilor lui, până în timpul domniei regelui Persiei,
\par 21 Pentru ca să se împlinească cuvântul Domnului cel zis prin gura lui Ieremia, până ce țara va termina de ținut zilele sale de odihnă; căci în toate zilele pustiirii ea s-a odihnit până la împlinirea celor șaptezeci de ani.
\par 22 Iar în anul dintâi al lui Cirus, regele Persiei, pentru împlinirea cuvintelor Domnului rostite prin Ieremia, a trezit Domnul duhul lui Cirus, regele Persiei, și a poruncit acesta să se facă cunoscut tuturor din tot regatul său, prin cuvânt și prin scris și să le spună:
\par 23 "Așa zice Cirus, regele Perșilor: Toate regatele pământului, Domnul Dumnezeul cerului mi le-a dat mie și mi-a poruncit să-I zidesc templul în Ierusalimul cel din Iuda. Cine este între voi din tot poporul Lui? Domnul Dumnezeul lui să fie cu el și să se ducă acolo".


\end{document}