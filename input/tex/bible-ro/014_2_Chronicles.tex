\begin{document}

\title{2 Cronici}


\chapter{1}

\par 1 În vremea aceea Solomon, fiul lui David, se întarise în domnie ?i Domnul Dumnezeul lui era cu el, înal?ându-l foarte sus.
\par 2 Solomon a poruncit sa se adune tot Israelul: capeteniile peste mii, capeteniile peste sute, judecatorii ?i to?i cei ce cârmuiau în Israel pâna la capii de familie.
\par 3 Apoi s-a dus Solomon ?i toata adunarea pe înal?imea cea din Ghibeon, caci acolo era cortul cel dumnezeiesc al adunarii pe care-l facuse Moise, robul lui Dumnezeu, în pustiu.
\par 4 Chivotul Domnului îl adusese David din Chiriat-Iearim la locul pe care-l pregatise pentru el David, în Ierusalim, unde-i facuse un cort nou.
\par 5 Iar jertfelnicul cel de arama, pe care-l facuse Be?aleel, fiul lui Uri, fiul lui Hur, ramasese acolo, înaintea cortului celui vechi al Domnului. Pe acesta îl cerceta Solomon cu adunarea.
\par 6 ?i aici, înaintea fe?ei Domnului, pe jertfelnicul cel de arama, care era înaintea cortului adunarii, a adus Solomon o mie de arderi de tot.
\par 7 În noaptea aceea S-a aratat Dumnezeu lui Solomon ?i i-a zis: "Cere ce dore?ti sa-?i dau?"
\par 8 Iar Solomon a zis catre Dumnezeu: "Tu ai facut cu David, tatal meu, mila mare ?i m-ai pus pe mine rege în locul lui.
\par 9 Sa se împlineasca dar, Doamne Dumnezeule, cuvântul Tau cel catre David, tatal meu! De vreme ce m-ai pus pe mine peste un popor mult la numar, ca pulberea pamântului,
\par 10 Apoi da-mi mie acum în?elepciune ?i ?tiin?a, ca sa pricep cum sa cârmuiesc pe poporul acesta, caci cine poate sa cârmuiasca pe acest mare popor al Tau?"
\par 11 Atunci Dumnezeu a zis catre Solomon: "Pentru ca tu ai avut la inima asemenea lucru ?i n-ai cerut boga?ie, averi ?i slava, nici sufletele neprietenilor tai; n-ai cerut de asemenea nici zile multe, ci ai cerut în?elepciune ?i ?tiin?a, ca sa judeci pe poporul Meu peste care te-am pus rege,
\par 12 De aceea ?i se da în?elepciune ?i ?tiin?a, iar boga?ie, averi ?i slava î?i voi da atâta, câta n-au mai avut regii cei dinainte de tine ?i nici dupa tine nu vor mai avea".
\par 13 Apoi a venit Solomon de pe înal?imea cea din Ghibeon, de la cortul adunarii, la Ierusalim ?i a domnit peste Israel.
\par 14 Dupa aceea Solomon a adunat care de razboi ?i calare?i; ?i avea o mie patru sute de care de razboi ?i douasprezece mii de calare?i. Pe to?i ace?tia i-a a?ezat în ceta?ile în care ?inea carele de razboi, precum ?i pe lânga rege în Ierusalim.
\par 15 ?i a facut regele ca argintul ?i aurul sa fie în Ierusalim la pre? ca pietrele de rând, iar cedrii, prin mul?imea lor, ajunsesera la pre? ca smochinii cei salbatici care cresc mul?ime prin locurile joase.
\par 16 Caii i se aduceau lui Solomon din Egipt ?i din Cheve (Coa); negustorii regelui îi cumparau cu bani, din Cheve (Coa).
\par 17 Un car de razboi se cumpara ?i se aducea din Egipt cu ?ase sute sicli de argint, iar calul cu o suta cincizeci. Tot astfel aduceau ei ?i tuturor Heteilor ?i regilor Siriei.

\chapter{2}

\par 1 Apoi Solomon hotarî sa înal?e templu numelui Domnului ?i casa domneasca pentru sine.
\par 2 În acest scop Solomon a tocmit ?aptezeci de mii de oameni ca sa care poveri ?i optzeci de mii de taietori de piatra din munte, iar peste ei, trei mii ?ase sute de supraveghetori.
\par 3 Apoi a trimis Solomon la Hiram, regele Tirului, sa i se spuna: "Cum ai facut cu David, tatal meu, ?i i-ai trimis cedrii pentru cladirea casei lui de locuit, a?a sa faci ?i cu mine.
\par 4 Iata ?i eu, fiul lui, zidesc templu numelui Domnului Dumnezeului meu, ca sa îl închin Lui, pentru ca sa se arda înaintea Lui tamâie mirositoare, sa I se înfa?i?eze pururea pâinile punerii înainte ?i sa I se aduca arderi de tot diminea?a ?i seara, în ziua odihnei, la luna noua ?i la sarbatorile Domnului Dumnezeului nostru, ceea ce s-a poruncit pentru totdeauna lui Israel.
\par 5 Templul pe care voiesc sa-l fac este mare, pentru ca mare este Dumnezeul nostru ?i mai presus de to?i dumnezeii.
\par 6 ?i cine poate sa-I zideasca Lui templu, când cerul ?i cerurile cerurilor nu-L încap? ?i cine sunt eu, ca sa-I pot zidi templu? Fara numai doar pentru a se tamâia înaintea Lui.
\par 7 Trimite-mi dar un om care sa ?tie sa faca lucruri de aur, de argint, de arama ?i de fier, de tort purpuriu, stacojiu ?i albastru ?i care sa mai ?tie a sapa ?i a face toate acestea împreuna cu me?terii care sunt aici la mine în Iuda ?i Ierusalim ?i pe care i-a pregatit David, tatal meu.
\par 8 Sa-mi trimi?i lemn de cedru, de chiparos ?i de molift din Liban, caci ?tiu ca robii tai pricep sa taie lemn în Liban, ?i robii mei vor merge cu ai tai,
\par 9 Ca sa-mi pregateasca lemn mult, deoarece loca?ul pe care voiesc sa-l fac este mare ?i minunat.
\par 10 Iata, robilor tai, taietorilor care taie lemnul, le dau hrana: douazeci de mii de core de grâu, douazeci de mii de core de orz, douazeci de mii de baturi de vin ?i douazeci de mii de baturi de untdelemn".
\par 11 Hiram, regele Tirului, a raspuns la aceasta, prin o scrisoare pe care a trimis-o lui Solomon, urmatoarele: "Din iubire pentru poporul Sau, Domnul te-a pus pe tine rege peste el".
\par 12 Apoi Hiram mai zicea: "Binecuvântat fie Domnul Dumnezeul lui Israel, Cel ce a facut cerul ?i pamântul ?i a dat regelui David fiu în?elept, cu minte ?i pricepere, care are de gând sa înal?e templu Domnului ?i casa regala pentru sine.
\par 13 A?adar î?i trimit un om în?elept ?i înzestrat cu ?tiin?a, ?i anume pe me?terul Hiram-Abi,
\par 14 Fiul unei femei dintre fiicele lui Dan, iar tatal sau e tirian. Acela ?tie sa faca lucruri de aur ?i de argint, de arama ?i de fier, de piatra ?i de lemn, de tort purpuriu, stacojiu ?i albastru, de in ?i de purpura; ?tie sa faca tot felul de sculpturi ?i sa îndeplineasca tot ce i se va porunci, împreuna cu me?terii tai, cu me?terii domnului meu David, tatal tau.
\par 15 Iar grâul, orzul, untdelemnul ?i vinul de care vorbe?ti tu, domnul meu, trimite-l robilor tai.
\par 16 Noi însa vom taia lemn din Liban cât î?i va trebui ?i-l vom duce cu plutele pe mare la Iafa, iar de acolo îl vei duce tu la Ierusalim".
\par 17 Apoi Solomon a numarat pe to?i strainii care se aflau atunci în pamântul lui Israel, afara de numaratoarea pe care o facuse David, tatal sau, ?i s-au gasit o suta cincizeci ?i trei de mii ?ase sute in?i.
\par 18 Din ace?tia a facut ?aptezeci de mii caratori cu spatele ?i optzeci de mii taietori de piatra în munte, iar trei mii ?ase sute, supraveghetori, ca sa îndemne pe oameni la lucru.

\chapter{3}

\par 1 Solomon a început sa înal?e templul Domnului în Ierusalim, pe muntele Moria, unde Se aratase Domnul lui David, tatal sau, ?i pe locul pe care-l pregatise David, în aria lui Ornan Iebuseul.
\par 2 Zidirea s-a început în ziua a doua a lunii a doua din anul al patrulea al domniei lui Solomon.
\par 3 Iata temelia pusa de Solomon la zidurile templului Domnului: lungimea lui era de ?aizeci de coji, iar la?imea de douazeci de co?i.
\par 4 Pridvorul, care era înaintea templului, avea lungimea de douazeci de co?i, cât la?imea templului, iar înal?imea de o suta douazeci de co?i ?i pe dinauntru l-a captu?it cu aur curat.
\par 5 Casa cea mare (Sfânta) însa a captu?it-o cu lemn de chiparos ?i a îmbracat-o cu aur de cel mai bun, iar pe el a sapat finici ?i lan?i?oare.
\par 6 Apoi a împodobit casa cu pietre scumpe. Aurul era din Parvaim (Ofir).
\par 7 Pere?ii casei, tavanul, u?orii, ferestrele ?i u?ile le-a acoperit cu aur ?i pe pere?i a sculptat chipuri de heruvimi.
\par 8 Sfânta Sfintelor a facut-o lunga de douazeci de co?i, cât largimea casei, ?i larga de douazeci de co?i, ?i a îmbracat-o cu ?ase sute de talan?i din cel mai bun aur.
\par 9 Cuiele de aur erau în greutate de cincizeci de sicli fiecare cui. Încaperile de sus înca le-a îmbracat cu aur.
\par 10 Apoi a facut în Sfânta Sfintelor doi heruvimi sapa?i în lemn ?i i-a ferecat cu aur.
\par 11 Aripile heruvimilor aveau în lungime douazeci de co?i: o aripa de cinci co?i atingea peretele templului, iar cealalta aripa de cinci co?i atingea aripa celuilalt heruvim.
\par 12 Asemenea ?i aripa de cinci co?i a celuilalt heruvim atingea peretele templului, iar cealalta aripa de cinci co?i atingea aripa heruvimului întâi.
\par 13 Aripile acestor heruvimi întinse erau de douazeci de co?i; heruvimii stateau în picioare, cu fe?ele spre Sfânta.
\par 14 Dupa aceea au facut o perdea de torturi purpurii, stacojii ?i albastre ?i de vison, iar pe ea a închipuit heruvimi.
\par 15 Înaintea templului au facut doi Stâlpi, înal?i de treizeci ?i cinci de co?i, iar coroanele de deasupra, de cinci co?i fiecare.
\par 16 Au mai facut lan?i?oare, ca cele din Sfânta Sfintelor, ?i le-a pus pe capetele stâlpilor. Apoi au facut o suta de rodii ?i le-au în?irat pe lan?i?oare.
\par 17 În sfâr?it au a?ezat stâlpii înaintea templului, unul în dreapta ?i altul în stânga ?i le-a pus nume: celui din dreapta Iachin, iar celui din stânga Booz.

\chapter{4}

\par 1 Dupa aceea a facut jertfelnic de arama, lung de douazeci de coli, lat de douazeci de co?i, înalt de zece co?i.
\par 2 A facut a?a-numita mare de arama turnata rotund, care avea de la o margine pâna la cealalta zece co?i, iar înal?imea ei era de cinci co?i; o sfoara de treizeci de co?i putea sa o cuprinda împrejur.
\par 3 Sub ea de jur-împrejur se aflau chipuri de boi turnate. Boii ace?tia, turna?i împreuna cu marea, erau a?eza?i sub ea de jur-împrejur pe doua rânduri, la zece co?i departare unul de altul.
\par 4 Marea statea pe doisprezece boi: trei îndrepta?i spre miazanoapte, trei îndrepta?i spre apus, trei îndrepta?i spre miazazi ?i trei îndrepta?i spre rasarit, iar marea deasupra, pe ei; spatele boilor însa erau înauntru, sub mare.
\par 5 Ea era groasa în pere?i de un lat de palma; marginile ei erau ca marginile unei cupe, semanând cu o floare deschisa de crin; în ea încapeau pâna la trei mii de baturi.
\par 6 A mai facut zece ligheane ?i le-a a?ezat cinci în partea dreapta ?i cinci în partea stânga, ca sa se spele în ele cele pregatite pentru arderile de tot; marea însa era pentru preo?i, ca sa se spele în ea.
\par 7 De asemenea a facut zece sfe?nice de aur, cum trebuia sa fie, ?i le-a pus în templul Domnului, cinci în partea dreapta ?i cinci în partea stânga.
\par 8 A facut înca zece mese ?i le-a pus în templul Domnului, cinci în partea dreapta ?i cinci în partea stânga; ?i a facut o suta de vase de aur.
\par 9 A facut apoi curtea preo?ilor ?i curtea cea mare; la cur?i a facut por?i ?i por?ile le-a ferecat cu arama.
\par 10 Marea a pus-o în partea dreapta, spre miazazi-rasarit.
\par 11 Dupa aceea Hiram a facut ligheane ?i lope?i, castroane, cadelni?e ?i toate vasele pentru jertfe. ?i a sfâr?it Hiram lucrarile pe care le-a facut lui Solomon pentru templul lui Dumnezeu:
\par 12 Cei doi stâlpi cu cele doua coroane de pe vârfurile stâlpilor, ?i doua re?ele pentru acoperit cele doua coroane, care erau pe capetele stâlpilor;
\par 13 Patru sute de rodii a pus pe re?ele, câte doua rânduri de rodii pentru fiecare re?ea, care acoperea cele doua coroane de pe vârful stâlpilor.
\par 14 A facut postamente ?i pe postamente ligheane de spalat;
\par 15 O mare ?i sub ea doisprezece boi;
\par 16 Ligheane, lope?i, furculi?e ?i toate celelalte lucruri le-a facut me?terul Hiram regelui Solomon, pentru templul Domnului, din arama lustruita.
\par 17 Acestea le-a turnat regele în preajma Iordanului, într-un pamânt cleios, între Sucot ?i ?ereda (?artan).
\par 18 Toate lucrurile acestea le-a facut Solomon în a?a de mare numar, încât nu se mai ?tia greutatea aramei.
\par 19 De asemenea a facut Solomon toate lucrurile pentru templul Domnului, jertfelnicul cel de aur ?i mesele pentru pâinile punerii înainte;
\par 20 Sfe?nicele ?i candelele lor de aur curat, ca sa fie aprinse dupa rânduiala înaintea Sfintei Sfintelor.
\par 21 Mucarile, candelele ?i tavi?ele de aur, din cel mai curat aur.
\par 22 Cu?itele, potirele, ca?uiele ?i ce?tile tot din cel mai bun aur: u?ile templului Domnului, u?ile dinauntru de la Sfânta Sfintelor ?i u?a de la Sfânta, tot de aur.

\chapter{5}

\par 1 Astfel s-a sfâr?it tot lucrul pe care l-a facut Solomon pentru templul Domnului. ?i a adus Solomon cele harazite de David, tatal lui: argint ?i aur ?i toate lucrurile ?i le-a dat în vistieria templului Domnului.
\par 2 Atunci a adunat Solomon pe batrânii lui Israel ?i pe to?i capii semin?iilor ?i capeteniile familiilor fiilor lui Israel la Ierusalim, ca sa stramute chivotul legamântului Domnului din cetatea lui David, adica din Sion.
\par 3 ?i s-au adunat la rege to?i Israeli?ii la sarbatoare, în luna a ?aptea.
\par 4 Dupa ce au venit toate capeteniile lui Israel, levi?ii au luat chivotul.
\par 5 ?i au dus chivotul ?i cortul adunarii ?i toate lucrurile sfinte care erau în cort, le-au dus preo?ii ?i levi?ii.
\par 6 Iar regele Solomon ?i toata ob?tea lui Israel, care se adunase la el înaintea chivotului, au adus jertfe de oi ?i boi atâtea, cât nu se puteau numara din pricina mul?imii.
\par 7 Preo?ii au dus chivotul legamântului Domnului la locul lui, înauntrul templului în Sfânta Sfintelor, sub aripile heruvimilor.
\par 8 Heruvimii î?i întindeau aripile peste locul chivotului ?i acopereau ei chivotul ?i pârghiile lui de sus.
\par 9 Apoi pârghiile au fost împinse a?a, încât capetele pârghiilor chivotului se vedeau din Sfânta, în fa?a Sfintei Sfintelor, iar pârghiile nu se puteau vedea, ?i acolo sunt ele pâna astazi.
\par 10 În chivot nu era nimic, fara numai cele doua table pe care le pusese Moise în Horeb, când încheiase Domnul legamântul cu fiii lui Israel, dupa ie?irea din Egipt.
\par 11 Dupa ce au ie?it preo?ii din Sfânta, caci to?i preo?ii care se aflau acolo se sfin?isera, fara sa se ?ina seama de rând,
\par 12 ?i când to?i levi?ii care erau cântare?i, Asaf, Heman, Iedutun, fiii lor ?i fra?ii lor - îmbraca?i în vison, ?i cu chimbale, chitare ?i harfe au stat în partea de rasarit a jertfelnicului ?i împreuna cu ei au stat o suta douazeci de preo?i care trâmbi?au din trâmbi?e
\par 13 ?i îndata ce aceia care sunau din trâmbi?e ?i cei care cântau, uni?i într-un singur glas ca sa slaveasca ?i sa laude pe Domnul, au facut sa rasune trâmbi?ele, chitarele ?i celelalte instrumente muzicale, ?i au slavit pe Domnul, zicând: "Caci El este bun, ca în veac este mila Lui!", atunci templul Domnului s-a umplut de norul slavei Lui,
\par 14 Încât preo?ii nu puteau sta la slujba din pricina norului, pentru ca slava Domnului umpluse templul Domnului.

\chapter{6}

\par 1 Atunci Solomon a zis: "Domnul a spus ca El binevoie?te sa locuiasca în negura,
\par 2 Iar eu am zidit templul, ca sa locuie?ti Tu, Cel Sfânt, ?i loca? unde sa petreci Tu în veci".
\par 3 Apoi ?i-a întors regele fa?a sa ?i a binecuvântat toata adunarea lui Israel, caci toata adunarea Israeli?ilor sta înainte.
\par 4 ?i a zis regele: "Binecuvântat este Domnul Dumnezeul lui Israel, Care a grait cu gura Sa catre David, tatal meu, ?i a împlinit cu mâinile Sale ceea ce spusese, zicând:
\par 5 Din ziua când am scos pe poporul Meu din ?ara Egiptului, nu Mi-am ales cetate nici într-una din semin?iile lui Israel, ca sa-Mi zidesc templu în care sa petreaca numele Meu, nici nu Mi-am ales om care sa fie cârmuitor poporului Meu Israel.
\par 6 Acum însa am ales Ierusalimul, ca sa petreaca numele Meu acolo, ?i am ales pe David, ca sa pastoreasca peste poporul Meu Israel.
\par 7 Iar lui David, tatal meu, îi intrase la inima sa înal?e templu numelui Domnului Dumnezeului lui Israel.
\par 8 Însa Domnul a zis lui David, tatal meu: ?i-a intrat la inima sa zide?ti templu numelui Meu; bine e ca ?i-a intrat acest lucru la inima;
\par 9 Dar nu vei zidi tu templul, ci fiul tau care va ie?i din coapsele tale, acela va zidi templu numelui Meu.
\par 10 ?i a împlinit Domnul cuvântul Sau, care l-a grait; caci eu am urmat în locul lui David, tatal meu, ?i am ?ezut pe tronul lui Israel, cum zisese Domnul, ?i am zidit templu numelui Domnului Dumnezeului lui Israel.
\par 11 ?i am pus acolo chivotul în care se afla legamântul Domnului, cel încheiat cu fiii lui Israel".
\par 12 Apoi stând Solomon la jertfelnicul Domnului, înaintea adunarii Israeli?ilor ?i-a ridicat mâinile sale,
\par 13 Caci Solomon î?i facuse un amvon de arama, lung de cinci co?i, lat de cinci co?i ?i înalt de trei co?i, ?i-l pusese în mijlocul cur?ii. Pe acest amvon a stat el ?i ?i-a plecat genunchii înaintea întregii adunari a Israeli?ilor. El a ridicat mâinile sale la cer
\par 14 ?i a zis: "Doamne Dumnezeul lui Israel, nu este Dumnezeu asemenea ?ie, nici în cer ?i nici pe pamânt. Tu paze?ti legamântul ?i mila cu robii Tai, care umbla cu toata inima lor înaintea Ta;
\par 15 Tu ai împlinit robului Tau David, tatal meu, ce i-ai grait; ca ceea ce ai grait cu gura Ta, aceea ai împlinit în ziua aceasta cu mâna Ta!
\par 16 ?i acum, Doamne Dumnezeul lui Israel, împline?te cele ce ai grait catre robul Tau David, tatal meu, când ai zis: Nu va lipsi barbat din tine care sa ?ada înaintea fe?ei Mele pe tronul lui Israel, daca fiii tai î?i vor pazi calea lor purtându-se dupa legea Mea, a?a cum te-ai purtat tu înaintea Mea.
\par 17 Deci acum, Doamne Dumnezeul lui Israel, fa sa se adevereasca cuvântul Tau, catre robul Tau David.
\par 18 Adevarat sa fie ca Dumnezeu va locui cu oamenii pe pamânt? Daca cerul ?i cerurile cerurilor nu Te încap, cu cât mai pu?in Te va încapea templul acesta pe care ?i l-am zidit eu?
\par 19 Dar cauta la rugaciunea robului Tau ?i la cererea lui, Doamne Dumnezeul meu! Asculta strigarea ?i ruga cu care robul Tau se roaga înaintea Ta:
\par 20 Sa fie ochii Tai deschi?i ziua ?i noaptea spre templul acesta ?i spre locul unde ai fagaduit sa-?i pui numele Tau, ca sa ascul?i rugaciunea cu care robul Tau se va ruga în locul acesta.
\par 21 Sa iei aminte la cererile robului Tau ?i ale poporului Tau Israel, cu care se vor ruga ei în locul acesta; sa auzi din locul ?ederii Tale, din ceruri, sa ascul?i ?i sa miluie?ti.
\par 22 Când va gre?i cineva împotriva aproapelui sau ?i i se va cere juramânt ca sa jure, juramântul se va face înaintea jertfelnicului Tau în templul acesta.
\par 23 Atunci sa ascul?i din cer ?i sa faci judecata robilor Tai: sa osânde?ti pe cel vinovat, facând sa i se întoarca asupra capului lui fapta sa ?i sa izbave?ti pe cel drept, dându-i dupa dreptatea lui.
\par 24 Când poporul Tau Israel va fi batut de du?man, pentru ca a pacatuit înaintea Ta, dar apoi se va întoarce catre Tine, va preaslavi numele Tau ?i va cere ?i se va ruga înaintea Ta, în templul acesta,
\par 25 Atunci sa ascul?i din cer ?i sa ier?i pacatul poporului Tau Israel ?i sa-l întorci în pamântul pe care l-ai dat lor ?i parin?ilor lor.
\par 26 Când se va încuia cerul ?i nu va fi ploaie, pentru ca au pacatuit ei înaintea Ta, ?i-?i vor aduce rugi în locul acesta, vor marturisi numele Tau ?i se vor întoarce de la pacatul lor pentru ca i-ai smerit,
\par 27 Atunci sa ascul?i din cer ?i sa ier?i pacatul robilor Tai, al poporului Tau Israel, aratându-le calea cea buna pe care sa mearga, ?i sa trimi?i ploaie pamântului Tau, pe care l-ai dat poporului Tau de mo?tenire.
\par 28 De va fi foamete pe pamânt, de va fi boala molipsitoare, de va fi vânt dogorâtor sau palitura, lacusta sau omida, du?manii de-i vor strâmtora în ?ara lor, stau în ceta?ile lor, de va fi orice necaz, orice boala,
\par 29 Atunci orice rugaciune ?i orice cerere care se va face de orice om sau de tot poporul Tau Israel, când ei î?i vor sim?i fiecare necazul sau ?i durerea ?i î?i vor întinde mâinile lor spre templul acesta,
\par 30 Tu sa ascul?i din cer, din locul ?ederii Tale, ?i sa ier?i; sa dai fiecaruia dupa caile lui, caci Tu cuno?ti inima lui ?i singur ?tii inima fiilor oamenilor,
\par 31 Pentru ca sa se teama de Tine ?i sa umble în caile Tale în toate zilele, cât vor trai pe pamântul pe care l-ai dat parin?ilor no?tri.
\par 32 Chiar ?i strainul, care nu este din poporul Tau Israel, când va purcede din pamânt departat pentru numele Tau cel mare, pentru mâna Ta cea puternica ?i pentru bra?ul Tau cel înalt ?i va veni ?i se va ruga în templul acesta,
\par 33 Tu sa ascul?i din cer, din locul sala?luirii Tale, ?i sa-i împline?ti tot lucrul pentru care strainul va striga catre Tine, ca sa ?tie toate popoarele pamântului, de numele Tau ?i sa se teama de Tine, cum se teme poporul Tau Israel, ?i sa ?tie ca în numele Tau este închinat templul pe care l-am zidit eu.
\par 34 Când poporul Tau va pleca cu razboi împotriva du?manilor sai, pe drumul pe care-l vei trimite Tu ?i se va ruga ?ie, întorcându-se spre cetatea aceasta, care ?i-ai ales-o, ?i spre templul acesta pe care l-am zidit eu numelui Tau,
\par 35 Atunci sa ascul?i din cer rugaciunea lor ?i cererea lor ?i sa le faci dreptate.
\par 36 Când vor pacatui ei înaintea Ta - caci nu este om care sa nu pacatuiasca - ?i Tu Te vei supara pe ei ?i-i vei da du?manilor lor ?i cei ce i-au luat robi îi vor duce în pamânt departat sau apropiat,
\par 37 ?i când ei, în pamântul în care vor fi robi?i, î?i vor veni în sine ?i se vor întoarce ?i ?i se vor ruga în pamântul robiei lor, zicând: am pacatuit, am facut faradelege, vinova?i suntem;
\par 38 Daca se vor întoarce catre Tine cu toata inima lor ?i cu tot sufletul lor, în pamântul robiei lor, unde ei se vor afla du?i robi ?i se vor ruga, întorcându-se spre pamântul lor, pe care Tu l-ai dat parin?ilor lor ?i spre "etatea care ?i-ai ales-o ?i spre templul pe care l-am zidit eu numelui Tau,
\par 39 Atunci sa ascul?i din cer, din locul ?ederii Tale, rugaciunea lor ?i cererea lor, ?i sa le faci dreptate ?i sa ier?i pe poporul Tau de ceea ce a pacatuit înaintea Ta.
\par 40 Dumnezeul meu, sa-?i fie ochii Tai deschi?i ?i urechile Tale cu luare aminte la rugaciunea care ?i se va face în locul acesta.
\par 41 ?i acum, Doamne Dumnezeule, scoala-Te ?i vino la locul de odihna al Tau, Tu ?i chivotul puterii Tale. Preo?ii Tai, Doamne Dumnezeule, se vor îmbraca întru mântuire ?i cuvio?ii Tai se vor desfata de bunata?i.
\par 42 Doamne Dumnezeule, sa nu-?i întorci fa?a Ta nici de la unsul Tau, ci adu-?i aminte de îndurarile cele catre David, robul Tau".

\chapter{7}

\par 1 Când a sfâr?it Solomon rugaciunea, s-a pogorât foc din cer ?i a mistuit arderea de tot ?i jertfele, ?i slava Domnului a umplut templul.
\par 2 Atunci n-au putut preo?ii sa intre în templul Domnului din pricina slavei lui Dumnezeu care umpluse templul.
\par 3 ?i to?i fiii lui Israel, vazând cum s-a coborât focul ?i slava Domnului peste templu, au cazut cu fa?a la pamânt pe pardoseala, s-au închinat ?i au slavit pe Domnul: "Ca este bun, ca în veac este mila Lui!"
\par 4 Apoi regele ?i tot poporul au început sa aduca jertfe înaintea fe?ei Domnului.
\par 5 Regele Solomon a adus jertfa douazeci ?i doua de mii de boi ?i o suta douazeci de mii de oi; a?a au sfin?it templul lui Dumnezeu regele ?i tot poporul.
\par 6 Preo?ii stateau la slujbele lor ?i levi?ii cu instrumentele de cântare ale Domnului, pe care le facuse regele David, ca sa laude pe Domnul: "Ca în veac este mila Lui"; caci David cu acestea Îl slavea, iar preo?ii trâmbi?au din trâmbi?e înaintea lui, iar Israelul tot statea de fa?a.
\par 7 Apoi a mai sfin?it Solomon ?i mijlocul cur?ii care era înaintea templului Domnului; caci a adus acolo arderile de tot ?i grasimea jertfelor de împacare, fiindca jertfelnicul cel de arama pe care îl facuse Solomon nu putea cuprinde arderile de tot ?i prinoasele de pâine ?i de grasime.
\par 8 În vremea aceea a facut Solomon sarbatoare de ?apte zile ?i împreuna cu el a praznuit tot Israelul, adunare foarte mare, venita de la intrarea Hamatului, pâna la râul Egiptului.
\par 9 Iar în ziua a opta au sarbatorit încheierea sarbatorii, caci sfin?irea jertfelnicului a ?inut ?apte zile, iar sarbatoarea alte ?apte zile.
\par 10 ?i în ziua a douazeci ?i treia a lunii a ?aptea, regele a dat drumul la corturile sale poporului, care se bucura ?i se veselea de binele ce daduse Domnul lui David, lui Solomon ?i poporului Sau Israel.
\par 11 Astfel a ispravit Solomon templul Domnului ?i casa regelui; tot ce planuise Solomon în inima sa sa faca pentru templul Domnului ?i pentru casa sa, le-a ispravit dupa dorin?a.
\par 12 Atunci S-a aratat Domnul lui Solomon, noaptea, ?i i-a zis: "Am auzit ruga ta ?i Mi-am ales locul acesta sa fie templu pentru aducerea de jertfe.
\par 13 De voi încuia cerul ?i nu va fi ploaie, de voi porunci lacustei sa manânce ?ara, sau voi trimite vreo boala molipsitoare asupra poporului Meu
\par 14 ?i se va smeri poporul Meu, care se nume?te cu numele Meu, ?i se vor ruga ?i vor cauta fa?a Mea, ?i se vor întoarce de la caile lor cele rele, atunci îi voi auzi din cer, le voi ierta pacatele lor ?i le voi tamadui ?ara.
\par 15 Acum ochii Mei vor fi deschi?i ?i urechile Mele vor fi cu luare-aminte la rugaciunea ce se va face în locul acesta.
\par 16 Caci am ales acum ?i am sfin?it templul acesta, pentru ca sa fie numele Meu acolo în veci; ?i ochii Mei ?i inima Mea sa fie acolo în toate zilele.
\par 17 Daca tu vei umbla înaintea fe?ei Mele, cum a umblat David, tatal tau, ?i vei face toate câte ?i-am poruncit, ?i vei pazi rânduielile ?i legile Mele,
\par 18 Atunci î?i voi întari tronul regatului tau, dupa cum am fagaduit lui David, tatal tau, când am zis: Nu va lipsi din tine barbat care sa domneasca peste Israel.
\par 19 Iar daca va ve?i abate ?i ve?i parasi rânduielile Mele ?i poruncile Mele, care vi le-am dat, ?i va ve?i duce ?i ve?i începe a sluji la al?i dumnezei ?i va ve?i închina lor,
\par 20 Atunci voi stârpi pe Israel de pe fa?a pamântului Meu pe care l-am dat lor, ?i templul acesta pe care l-am sfin?it numelui Meu, îl voi lepada de la fa?a Mea, iar pe Israel îl vai da de pilda ?i de ocara la toate popoarele;
\par 21 Iar de templul acesta înalt va ramâne uimit tot cel ce va trece pe lânga el, ?i va zice: Pentru ce a facut a?a Domnul cu pamântul acesta ?i cu templul acesta?
\par 22 ?i vor zice: Pentru ca au parasit pe Domnul Dumnezeul parin?ilor lor, Care i-a scos din pamântul Egiptului, ?i pentru ca s-au lipit de al?i dumnezei ?i s-au închinat ?i au slujit lor, pentru aceea a adus El asupra lor toate relele acestea".

\chapter{8}

\par 1 Dupa sfâr?irea celor douazeci de ani, în care Solomon a zidit templul Domnului ?i casa sa,
\par 2 A zidit Solomon ceta?ile pe care i le daruise Hiram ?i a a?ezat în ele pe fiii lui Israel.
\par 3 Apoi a plecat Solomon împotriva Hamat-?obei ?i a luat-o
\par 4 ?i a zidit el Tadmorul, în pustiu, ?i toate ceta?ile cele pentru provizii, pe care le întemeiase în Hamat.
\par 5 El a mai zidit de asemenea Bet-Horonul de Sus ?i Bet-Horonul de Jos, întarindu-le cu ziduri de jur-împrejur, cu por?i ?i cu zavoare;
\par 6 Baalatul ?i toate ceta?ile pentru provizii, pe care le avea Solomon, ?i toate ceta?ile pentru carele de razboi, ceta?ile pentru calare?i ?i tot ce a dorit Solomon sa zideasca în Ierusalim, în Liban, în tot pamântul stapânirii lui.
\par 7 Tot poporul care a ramas din Hetei, Amorei, Ferezei, Hevei, Iebusei, care nu erau dintre fiii lui Israel;
\par 8 Pe copiii lor care au ramas în ?ara dupa ei ?i pe care fiii lui Israel nu i-au stârpit, Solomon i-a facut oameni de corvoada pâna în ziua de astazi.
\par 9 Iar pe fiii lui Israel, Solomon nu i-a facut oameni de corvoada pentru lucrarile lui; pe ei însa îi avea pentru osta?i capetenii peste garzi ?i pentru capetenii peste carele ?i calare?ii lui.
\par 10 Regele Solomon avea doua sute cincizeci de conducatori mari care supravegheau oamenii la lucru.
\par 11 Iar pe fiica lui Faraon, Solomon a mutat-o din cetatea lui David în casa pe care o zidise pentru ea; caci zicea el: Femeia mea nu trebuie sa locuiasca în casa lui David, regele lui Israel, pentru ca acea casa este sfânta de când a intrat în ea chivotul Domnului.
\par 12 Atunci a început Solomon sa aduca arderi de tot Domnului pe jertfelnicul pe care-l facuse pentru Domnul înaintea pridvorului,
\par 13 Pentru ca sa se aduca pe el arderi de tot, dupa rânduiala ?i dupa porunca lui Moise, în fiecare zi, în toate zilele de odihna, la lunile noi ?i la cele trei sarbatori de peste an: la sarbatoarea azimelor, la sarbatoarea saptamânilor ?i la sarbatoarea corturilor.
\par 14 ?i a a?ezat el, dupa rânduiala lui David, tatal sau, preo?ii sa-?i faca slujba lor, cu rândul; ?i tot cu rândul a rânduit sa fie de straja levi?ii, sa cânte cântarile de lauda ?i sa slujeasca pe preo?i, dupa rânduiala de fiecare zi; pe u?ieri, dupa cetele lor, i-a rânduit la fiecare u?a, pentru ca a?a poruncise David, omul lui Dumnezeu.
\par 15 ?i nu s-a facut întru nimic nici o abatere de la poruncile regelui, cele pentru preo?i ?i pentru levi?i, ?i nici de la cele pentru vistierie.
\par 16 A?a s-a savâr?it toata lucrarea lui Solomon din ziua punerii temeliei la templul Domnului pâna la terminarea lui.
\par 17 Atunci s-a dus Solomon la E?ion-Gheber ?i la Elot, pe ?armul marii, care este în ?ara lui Edom.
\par 18 Iar Hiram i-a trimis cu slugile sale corabii ?i robi, cunoscatori ai marilor, care s-au dus cu slugile lui Solomon la Ofir ?i au luat de acolo patru sute cincizeci de talan?i de aur ?i i-au adus regelui Solomon.

\chapter{9}

\par 1 Auzind regina din Saba de faima lui Solomon ?i voind sa-l încerce cu întrebari grele, a venit la Ierusalim cu foarte multa boga?ie, cu camile încarcate cu aromate, cu aur mult ?i pietre scumpe; ?i a venit la Solomon ?i i-a grait lui toate câte avea în sufletul ei.
\par 2 ?i i-a dezlegat Solomon toate întrebarile ei ?i nu s-a gasit nimic necunoscut pentru Solomon, încât sa nu-i dezlege el.
\par 3 Vazând regina din Saba în?elepciunea lui Solomon ?i casa pe care o zidise el,
\par 4 Mâncarile de la masa lui, locurile de ?edere ale robilor lui, rânduiala slujitorilor lui ?i îmbracamintea lor, paharnicii lui, îmbracamintea lor ?i arderile de tot pe care le aducea în templu, a ramas uimita.
\par 5 ?i a zis regelui: "Cele ce am auzit eu în ?ara mea despre lucrurile tale ?i despre în?elepciunea ta, sunt adevarate;
\par 6 Dar eu n-am crezut vorbele ce mi se spuneau, pâna ce n-am venit ?i n-am vazut cu ochii mei. ?i iata nici pe jumatate nu mi s-a grait de marirea în?elepciunii tale; tu întreci cu mult faima auzita de mine, despre tine.
\par 7 Ferici?i sunt oamenii tai ?i fericite sunt aceste slugi ale tale, care-?i stau totdeauna înainte ?i-?i asculta în?elepciunea!
\par 8 Binecuvântat sa fie Domnul Dumnezeul tau Care a binevoit sa te puna pe tronul Sau, pentru ca sa fii rege în numele Domnului Dumnezeului tau. Din dragostea pe care Dumnezeul tau o are catre Israel, ca sa-l întareasca în veci, te-a facut rege peste el, ca sa faci judecata ?i dreptate".
\par 9 ?i a daruit ea regelui o suta douazeci de talan?i de aur ?i mul?ime mare de aromate ?i de pietre scumpe; asemenea aromate, ca cele daruite de regina din Saba regelui Solomon, nu se mai vazusera.
\par 10 În vremea aceea slugile lui Hiram ?i slugile lui Solomon, care-i aduceau aur de la Ofir, îi adusesera ?i lemn ro?u ?i pietre scumpe.
\par 11 Din acest lemn ro?u facuse regele scarile de la templul Domnului ?i de la casa domneasca, precum ?i chitare ?i harpe pentru cântare?i. Astfel de lemn nu se mai vazuse niciodata înainte în ?ara lui Iuda.
\par 12 Iar regele Solomon a dat reginei din Saba tot ce ea a dorit ?i a cerut, afara de darul pe care i l-a dat pentru lucrurile pe care ea le adusese regelui. ?i a?a s-a întors înapoi în ?ara sa, ea ?i slugile sale.
\par 13 Greutatea aurului care i se adusese lui Solomon într-un an era de ?ase sute ?aizeci ?i ?ase de talan?i de aur.
\par 14 Afara de acestea, mai aduceau lui Solomon aur ?i argint solii de la diferite popoare ?i negustorii, precum ?i to?i regii Arabiei ?i conducatorii de provincii.
\par 15 Regele Solomon a facut doua sute de scuturi mari de aur ciocanit în care au intrat câte ?ase sute de sicli de aur de fiecare scut ciocanit.
\par 16 ?i trei sute de scuturi mici, tot de aur ciocanit, în care au intrat câte trei sute de sicli de aur de fiecare scut. Pe acestea le-a pus regele în Casa Padurii din Liban.
\par 17 Apoi a facut regele un tron mare de os de filde? ?i l-a îmbracat peste tot cu aur curat,
\par 18 Iar la tron a facut ?ase trepte de suit, un scaunel de aur pentru picioare, prins de tron, rezematori de o parte ?i de alta a locului de ?edere ?i doi lei care stateau lânga rezematori
\par 19 ?i înca doisprezece lei care stateau acolo pe cele ?ase trepte de o parte ?i de alta. A?a tron nu se mai gasea în nici un regat.
\par 20 Toate vasele de baut ale regelui Solomon erau de aur ?i toate vasele din Casa Padurii din Liban erau de aur ales. În zilele lui Solomon argintul se socotea ca nimic,
\par 21 Caci corabiile regelui umblau la Tarsis eu slugile lui Hiram ?i la trei ani o data se întorceau aducând aur ?i argint, filde?, maimu?e ?i pauni.
\par 22 Astfel Solomon a întrecut pe to?i regii pamântului în boga?ie ?i în?elepciune.
\par 23 ?i to?i regii ?arilor cautau sa vada pe Solomon, ca sa-i asculte în?elepciunea pe care i-o pusese Dumnezeu în inima lui.
\par 24 ?i fiecare din ei îi aducea vase de argint, de aur, ve?minte, arme, aromate, cai ?i catâri, în fiecare an.
\par 25 Solomon avea patru mii de iesle pentru caii de pe la carele lui ?i douasprezece mii de calare?i, a?eza?i în ceta?ile unde avea carele ?i pe lânga rege în Ierusalim.
\par 26 El domnea peste to?i regii, de la râul Eufrat pâna la ?ara Filistenilor ?i pâna la hotarul Egiptului.
\par 27 ?i a facut regele sa fie aurul ?i argintul pre?uit în Ierusalim ca pietrele de pe drum, iar cedrii, din pricina mul?imii lor, i-a facut sa fie pre?ui?i ca smochinii cei salbatici de prin locuri joase.
\par 28 Cai pentru Solomon i se aduceau din Egipt ?i din toate ?arile.
\par 29 Celelalte fapte ale lui Solomon, de la cele dintâi pâna la cele din urma, sunt scrise în cartea lui Natan proorocul, în proorocia lui Ahia ?ilonitul ?i în vedeniile lui Ido vazatorul despre Ieroboam, fiul lui Nabat.
\par 30 Solomon a domnit în Ierusalim peste tot Israelul patruzeci de ani.
\par 31 Apoi a raposat Solomon cu parin?ii sai ?i l-au îngropat în cetatea lui David, tatal sau. Iar în locul lui s-a facut rege Roboam, fiul sau.

\chapter{10}

\par 1 Atunci s-a dus Roboam la Sichem, pentru ca la Sichem se adunasera to?i Israeli?ii, ca sa-l faca rege.
\par 2 Când a auzit de aceasta Ierobnam, fiul lui Nabat, care se afla în Egipt, unde fugise de regele Solomon, s-a întors Ieroboam din Egipt.
\par 3 Iar Israeli?ii au trimis ?i l-au chemat. Venind deci Ieroboam ?i tot Israelul, au grait lui Roboam a?a:
\par 4 "Tatal tau a pus jug greu pe noi. Tu însa u?ureaza-ne de munca silnica a tatalui tau ?i de jugul cel greu, care d-a pus el pe noi ?i î?i vom sluji".
\par 5 Iar Roboam le-a zis: "Veni?i peste trei zile la mine!" ?i s-a împra?tiat poporul.
\par 6 Atunci s-a sfatuit regele Roboam eu batrânii care fusesera sfetnicii lui Solomon, tatal lui, cât a trait el ?i le-a zis: "Cum ma sfatui?i sa raspund poporului acestuia?"
\par 7 Iar ei i-au zis: "De vei fi bun acum cu poporul acesta, de le vei face pe plac ?i le vei vorbi cu blânde?e, atunci ei î?i vor fi robi în toate zilele".
\par 8 Dar el n-a ?inut seama de sfatul ce i-au dat batrânii, ci a început sa se sfatuiasca cu oamenii cei tineri care crescusera cu el ?i-i avea ca sfetnici pe lânga sine
\par 9 ?i le-a zis: "Ce ma sfatui?i sa raspund poporului acestuia, care mi-a grait a?a: U?ureaza-ne jugul pe care tatal tau l-a pus pe noi?"
\par 10 Oamenii cei tineri, care crescusera cu el, i-au raspuns ?i i-au zis: "Poporului care ?i-a vorbit: Tatal tau a pus jug greu peste noi, iar tu u?ureaza-ni-l, spune-le a?a: Degetul meu cel mic e mai gros decât mijlocul tatalui meu.
\par 11 Tatal meu a pus jug greu pe voi, eu însa voi mari jugul vostru; tatal meu v-a pedepsit cu biciul, eu însa va voi bate cu scorpioane".
\par 12 ?i a venit Ieroboam împreuna eu tot poporul la Roboam, în ziua a treia, dupa cum le poruncise regele, când zisese: "Veni?i la mine peste trei zile".
\par 13 Atunci regele le-a raspuns cu asprime, caci n-a ?inut seama regele Roboam de sfatul batrânilor, ci le-a grait dupa sfatul oamenilor tineri a?a:
\par 14 "Tatal meu  pus jug greu peste voi, eu însa îl voi mari; tatal meu v-a pedepsit cu biciul; eu însa va voi bate cu scorpioane".
\par 15 ?i n-a ascultat regele de popor, pentru ca a?a fusese rânduit de la Dumnezeu, ca sa-?i împlineasca Domnul cuvântul Sau, pe care-l graise prin Ahia ?ilonitul lui Ieroboam, fiul lui Nabat.
\par 16 Când tot Israelul a vazut ca regele nu-l asculta, atunci poporul a raspuns regelui, zicând: "Ce parte mai avem noi cu David? Nu mai avem nimic cu fiul lui Iesei. La corturi, Israele! Davide, vezi de casa ta!" ?i s-au împra?tiat to?i Israeli?ii pe la corturile lor.
\par 17 Iar Roboam a ramas rege numai peste fiii lui Israel care locuiau în ceta?ile Iudei.
\par 18 Atunci regele Roboam a trimis împotriva lor pe Adoniram, care era capetenie peste strânsul darilor; dar fiii lui Israel l-au batut cu pietre ?i el a murit. Iar regele Roboam s-a grabit sa se suie în carul sau, ca sa fuga la Ierusalim.
\par 19 A?a s-a despar?it Israel de casa lui David, pâna în ziua de astazi.

\chapter{11}

\par 1 Atunci a venit Roboam la Ierusalim ?i a strâns din casa lui Iuda ?i a lui Veniamin o suta ?aptezeci de mii de osta?i ale?i, ca sa lupte în Israel ?i sa întoarca regatul iara?i sub stapânirea lui Roboam.
\par 2 ?i a fost cuvântul Domnului spre ?emaia, omul lui Dumnezeu ?i i-a zis:
\par 3 "Spune lui Roboam, fiul lui Solomon, regele Iudei ?i la tot Israelul din neamul lui Iuda ?i al lui Veniamin:
\par 4 A?a zice Domnul: Sa nu merge?i ?i sa nu face?i razboi cu fra?ii vo?tri. Întoarce?i-va fiecare la casa voastra, caci de Mine s-a facut acest lucru!" ?i au ascultat ei de cuvintele Domnului ?i s-au întors înapoi din drumul lor împotriva lui Ieroboam.
\par 5 Roboam ?edea în Ierusalim ?i a împrejmuit ceta?ile lui Iuda cu ziduri.
\par 6 El a întarit Betleemul, Etamul ?i Tecoa;
\par 7 Bet-?urul, Soco ?i Adulamul,
\par 8 Gatul, Mare?a ?i Ziful,
\par 9 Adoraimul, Lachi?ul ?i Azeca,
\par 10 ?ora, Aialonul ?i Hebronul, care se aflau în neamul lui Iuda ?i al lui Veniamin.
\par 11 Ceta?ile acestea le-a întarit el cu ziduri ?i a a?ezat în ele capetenii ?i magazii pentru ?inut pâine, untdelemn ?i vin.
\par 12 Fiecarei ceta?i i-a dat scuturi ?i suli?e ?i le-a întarit cu foarte mare tarie. ?i a?a a ramas cu el Iuda ?i Veniamin.
\par 13 Apoi s-au adunat la el din toate par?ile preo?ii ?i levi?ii, care erau în tot pamântul lui Israel,
\par 14 Caci levi?ii ?i-au parasit a?ezarile ?i locurile lor, stapânite de ei, ?i au venit în Iuda ?i la Ierusalim, din pricina ca Ieroboam ?i fiii lui îi îndepartasera din dregatoria preo?iei Domnului
\par 15 ?i pentru ca Ieroboam î?i pusese preo?i pentru înal?imi ?i pentru ?apii ?i vi?eii pe care îi facuse el.
\par 16 Iar dupa ei au venit la Ierusalim din toate semin?iile lui Israel ?i aceia care î?i aveau inima îndreptata sa caute pe Domnul Dumnezeul lui Israel, pentru ca sa aduca jertfa Domnului Dumnezeului parin?ilor lor.
\par 17 ?i a?a au întarit ace?tia regatul lui Iuda ?i l-au sprijinit pe Roboam, fiul lui Solomon, timp de trei ani., pentru ca trei ani a umblat el pe caile lui David ?i ale lui Solomon.
\par 18 Roboam ?i-a luat de femeie pe Mahalat, fiica lui Ierimot, fiul lui David ?i al Abihailei, fiica lui Eliab, fiul lui Iesei.
\par 19 Aceasta i-a nascut fii pe Ieu?, pe ?emaria ?i pe Zaham.
\par 20 Dupa ea a mai luat pe Maaca, fiica lui Abesalom, care i-a nascut pe Abia, pe Atai, pe Ziza ?i pe ?elomit.
\par 21 Roboam însa iubea pe Maaca, fiica lui Abesalom, mai mult decât pe toate femeile ?i decât pe toate concubinele sale, caci el a avut optsprezece femei ?i ?aizeci de concubine ?i a nascut cu ele douazeci ?i opt de baie?i ?i ?aizeci de fete.
\par 22 ?i a pus Roboam pe Abia, fiul Maacai, conducator peste fra?ii lui, pentru ca pe el voia sa-l faca rege.
\par 23 ?i a lucrat în?elep?e?te, caci ?i-a împar?it to?i feciorii sai în toate ceta?ile întarite din tot pamântul lui Iuda ?i al lui Veniamin ?i le-a dat între?inere mare ?i le-a cautat mul?ime de femei.

\chapter{12}

\par 1 Dupa ce s-a întarit regatul lui Roboam ?i a ajuns destul de puternic, Roboam a parasit legea Domnului ?i dimpreuna cu el ?i tot Israelul.
\par 2 Dar pentru ca s-au abatut ei de la Domnul, de aceea în anul al cincilea al domniei lui Roboam, ?i?ac, regele Egiptului, a plecat cu razboi împotriva Ierusalimului,
\par 3 Cu o mie doua sute de care de razboi ?i cu ?aizeci de mii de calare?i; iar poporul, care venise cu el din Egipt: Libieni, Suchieni ?i Etiopieni, era foarte numeros.
\par 4 Ace?tia au luat ceta?ile întarite din Iuda ?i au venit la Ierusalim.
\par 5 Atunci ?emaia proorocul a venit la Roboam ?i la capeteniile lui Iuda, care se strânsesera la Ierusalim din pricina lui ?i?ac ?i le-a zis: "A?a zice Domnul: Pentru ca M-a?i parasit, de aceea va las pe mâinile lui ?i?ac".
\par 6 Iar capeteniile lui Israel ?i regele s-au smerit ?i au zis: "Drept este Domnul!"
\par 7 Când a vazut Domnul ca ei s-au smerit, atunci a fost cuvântul Domnului din nou catre ?emaia ?i a zis: "S-au smerit; nu-i voi mai stârpi ?i în curând le voi da ?i izbavire. Mânia Mea nu se va mai varsa asupra Ierusalimului prin mâna lui ?i?ac.
\par 8 Dar ei tot vor fi slugile lui ca sa cunoasca ce înseamna sa-Mi slujeasca Mie ?i ce înseamna sa slujeasca regatelor pamântului".
\par 9 Atunci a venit ?i?ac, regele Egiptului, la Ierusalim ?i a luat vistieriile templului Domnului ?i vistieriile casei regelui; tot ce a gasit, a luat; a luat ?i scuturile cele de aur, pe care le facuse Solomon.
\par 10 Iar regele Roboam a facut în locul lor scuturi de arama ?i le-a dat în mâinile capeteniilor de peste garzi, care pazeau intrarea la casa regelui.
\par 11 ?i numai când mergea regele la templul Domnului, numai atunci venea garda ?i le purta ?i pe urma le aducea iara?i în casa de garda.
\par 12 Dupa ce s-a smerit Roboam, s-a îndepartat mânia Domnului de la el ?i nu l-a nimicit cu totul. Afara de aceasta ?i în Iuda se gasea câte ceva bun.
\par 13 A?a s-a întarit iar regele Roboam în Ierusalim ?i a domnit. Roboam era de patruzeci ?i unu de ani când s-a facut rege ?i a domnit ?aptesprezece ani în Ierusalim, în cetatea pe care o alesese Domnul dintre toate semin?iile lui Israel, ca sa-?i puna numele în ea. Pe mama lui Roboam o chema Naama Amonita.
\par 14 Dar a facut el rele, pentru ca nu ?i-a îndreptat inima sa ca sa caute pe Domnul.
\par 15 Faptele lui Roboam, cele dintâi ?i cele de pe urma, sunt scrise în amintirile lui ?emaia proorocul ?i ale lui Ido vazatorul în spi?ele neamurilor. Roboam a purtat razboaie cu Ieroboam în toate zilele vie?ii lor.
\par 16 Apoi a raposat Roboam cu parin?ii sai ?i a fost îngropat în cetatea lui David. Iar în locul lui s-a facut rege Abia, fiul sau.

\chapter{13}

\par 1 Abia a început sa domneasca peste Iuda în anul al optsprezecelea al domniei lui Ieroboam.
\par 2 ?i a domnit trei ani în Ierusalim. Pe mama lui o chema Maaca ?i era fiica lui Uriel din Ghibea ?i a fost razboi ?i între Abia ?i Ieroboam.
\par 3 Abia a început razboiul cu o armata numai de oameni viteji, adica de patru sute de mii de oameni ale?i. Iar Ieroboam a ie?it împotriva lui la lupta cu opt sute de mii de oameni, tot numai oameni ale?i ?i viteji.
\par 4 Abia cu armata sa s-a a?ezat pe vârfui muntelui ?emaraim, unul dintre mun?ii lui Efraim ?i a zis: "Asculta?i-ma, Ieroboame ?i voi to?i Israeli?ii.
\par 5 Nu ?ti?i voi, oare, ca Domnul Dumnezeul lui Israel a dat lui David domnia peste Israel în veci, lui ?i fiilor lui, prin legamânt ve?nic?
\par 6 Dar s-a sculat Ieroboam, fiul lui Nabat, care era rob la Solomon, fiul lui David, ?i s-a razvratit împotriva stapânului sau.
\par 7 Atunci s-au strâns împrejurul lui oameni netrebnici, oameni înrai?i ?i s-au îndârjit împotriva lui Roboam, fiul lui Solomon; iar Roboam era tânar ?i cu inima fricoasa ?i n-a putut sa li se împotriveasca.
\par 8 Voi ?i acum zice?i ca pute?i sa va împotrivi?i regatului Domnului, care se afla în mina fiilor lui David, pentru ca sunte?i o mul?ime mare de popor ?i ave?i vi?ei de aur, pe care vi i-a facut Ieroboam, ca dumnezei.
\par 9 N-a?i izgonit voi oare pe preo?ii Domnului, pe fiii lui Aaron ?i pe levi?i, ?i nu v-a?i facut voi oare singuri preo?i dupa pilda popoarelor din celelalte ?ari? La voi, oricine vine sa se cura?easca cu un vi?el ?i cu ?apte berbeci, se face preot celui ce nu este dumnezeu.
\par 10 Iar la noi Domnul este Dumnezeul nostru; noi nu L-am parasit ?i slujbele cele catre Domnul le fac preo?ii, care sunt dintre fiii lui Aaron ?i levi?ii, care sunt la slujba de straja.
\par 11 Ei fac sa se ridice în fiecare diminea?a ?i în fiecare seara fum de arderi de tot catre Domnul, cum ?i fum de tamâieri bine mirositoare; ?i a?eaza în rânduri pâinile punerii înainte pe masa cea curata ?i aprind policandrele cele de aur ?i candelele lui, ca sa arda în fiecare seara, pentru ca noi pazim legea Domnului Dumnezeului nostru, iar voi L-a?i parasit.
\par 12 ?i iata, noi avem în fruntea noastra pe Dumnezeu ?i pe preo?ii Lui ?i trâmbi?ele cele rasunatoare, ca sa rasune cu vuiet mare împotriva voastra. Fii ai lui Israel, sa nu va lupta?i cu Domnul Dumnezeul parin?ilor vo?tri, ca nu ve?i avea nici un spor".
\par 13 Dar Ieroboam a trimis o ceata mare de oameni care stau la pânda, sa învaluiasca pe Iuda ?i sa le cada în spate, a?a ca ei se aflau în fa?a lui Iuda, iar cei de la pânda în spatele lui.
\par 14 Dar Iudeii au prins de veste aceasta ?i au început lupta ?i în fa?a ?i în spate ?i au strigat catre Domnul, iar preo?ii au trâmbi?at din trâmbi?e.
\par 15 Atunci au ridicat Iudeii strigat mare. Dar când Iudeii au ridicat strigatul, Dumnezeu a lovit pe Ieroboam ?i pe to?i Israeli?ii în fa?a lui Abia ?i a lui Iuda.
\par 16 ?i au fugit fiii lui Israel din fa?a celor din Iuda ?i Dumnezeu i-a dat în mâinile lor.
\par 17 Iar Abia dimpreuna cu poporul lui le-a dat o lovitura puternica ?i au cazut mor?i din Israel cinci sute de mii de oameni ale?i.
\par 18 Atunci s-au smerit fiii lui Israel, iar fiii lui Iuda s-au facut puternici, pentru ca au nadajduit în Domnul Dumnezeul parin?ilor lor.
\par 19 Abia l-a fugarit din urma pe Ieroboam ?i i-a luat ceta?ile Betelul cu satele lui, Ie?ana cu satele ei ?i Efronul cu satele lui.
\par 20 Dupa aceea Ieroboam n-a mai avut putere sa se împotriveasca în toate zilele lui Abia. ?i l-a lovit Domnul ?i a murit.
\par 21 Iar Abia s-a întarit. El ?i-a luat paisprezece femei ?i a nascut cu ele douazeci ?i doi de baie?i ?i ?aisprezece fete.
\par 22 Celelalte fapte ale lui Abia, purtarile sale ?i vorbele sale sunt scrise în amintirile lui Ido proorocul.

\chapter{14}

\par 1 Apoi a raposat Abia cu parin?ii sai ?i l-au îngropat în cetatea lui David, iar în locul lui s-a facut rege Asa, fiul sau. În zilele sale ?ara a avut pace zece ani.
\par 2 Asa a facut cele bune ?i placute înaintea Domnului Dumnezeului sau,
\par 3 Caci el a îndepartat jertfelnicele dumnezeilor straini ?i înal?imile, a sfarâmat stâlpii idole?ti ?i a taiat A?erele.
\par 4 El a poruncit Iudeilor sa caute pe Domnul Dumnezeul parin?ilor lor ?i sa împlineasca legea ?i poruncile Lui;
\par 5 A îndepartat din toate ceta?ile lui Iuda înal?imile ?i chipurile turnate ale soarelui. Regatul sub domnia lui a avut pace.
\par 6 El a zidit ceta?i tari în Iuda; caci ?ara a fost în pace ?i n-a avut razboaie cu nimeni în acei ani, fiindca Domnul i-a dat odihna.
\par 7 Tot el a zis Iudeilor: "Sa zidi?i ceta?ile acestea ?i sa le îngradi?i cu ziduri, cu turnuri, cu por?i ?i cu zavoare; ?ara este înca a noastra, pentru ca am cautat pe Domnul Dumnezeul nostru; L-am cautat ?i El ne-a daruit odihna din toate par?ile". ?i au început sa le zideasca ?i au izbutit.
\par 8 Asa avea în o?tirea lui trei sute de mii de oameni ale?i din semin?ia lui Iuda, înarma?i cu scuturi ?i cu lanci; iar din semin?ia lui Veniamin, doua sute optzeci de mii de oameni ale?i, înarma?i cu scuturi ?i erau ?i tragatori din arcuri.
\par 9 Atunci s-a sculat împotriva sa Zerah Etiopianul cu o armata de un milion ?i cu trei sute de care ?i a venit pâna la Mare?a.
\par 10 Aici i-a ie?it Asa înainte ?i s-a a?ezat cu armata sa în linie de bataie, în valea lui ?efat, lânga Mare?a.
\par 11 Apoi a strigat Asa catre Domnul Dumnezeul sau ?i a zis: "Doamne, la Tine este puterea, ca sa aju?i ?i celui tare ?i celui ce nu este tare; ajuta-ne dar noua, Doamne Dumnezeul nostru, caci noi în Tine nadajduim ?i în numele Tau am pornit împotriva acestei mul?imi, care este atât de mare; Doamne, Tu e?ti Dumnezeul nostru, sa n-aiba omul putere împotriva Ta".
\par 12 Atunci a lovit Domnul pe Etiopieni înaintea fe?ei lui Asa ?i înaintea fe?ei lui Iuda ?i au fugit Etiopienii.
\par 13 Iar Asa i-a fugarit din urma împreuna cu poporul care era cu el, pâna la Gherar, ?i au cazut atâ?ia Etiopieni, de se credea ca n-a mai ramas nimeni din ei cu suflet, pentru ca aceia cadeau zdrobi?i înaintea Domnului ?i înaintea o?tirii Lui. ?i au luat de la ei o mare mul?ime de prada.
\par 14 ?i au sfarâmat toate ceta?ile dimprejurul Gherarei, pentru ca groaza Domnului intrase în ele. ?i au pradat Israeli?ii toate ceta?ile ?i au adus din ele foarte multa prada.
\par 15 De asemenea au jefuit ?i sala?ele ?i au luat o mul?ime de turme cu vite marunte ?i camile ?i s-au întors cu ele la Ierusalim.

\chapter{15}

\par 1 Atunci s-a pogorât Duhul lui Dumnezeu asupra lui Azaria, fiul lui Oded,
\par 2 ?i i-a ie?it înaintea lui Asa ?i i-a zis: "Asculta?i-ma, Asa ?i tot Iuda ?i Veniamin: Domnul este cu voi când ?i voi sunte?i cu El. De-L ve?i cauta, va fi gasit de voi, iar de Îl ve?i parasi, ?i El va va parasi.
\par 3 Multe zile a fost Israel fara Dumnezeul Cel adevarat ?i fara de preot-înva?ator ?i fara de lege;
\par 4 Dar când el, în strâmtorarea lui, s-a întors la Domnul Dumnezeul lui Israel ?i L-a cautat, El S-a lasat gasit de ei.
\par 5 În acele vremuri n-a avut pace, nici cel ce ie?ea, nici cel ce intra, caci mari tulburari au fost între to?i locuitorii ?arilor;
\par 6 Popor cu popor se lupta ?i cetate cu cetate, pentru ca Dumnezeu îi aducea la tulburari prin tot felul de necazuri.
\par 7 Dar voi întari?i-va ?i sa nu va oboseasca mâinile, pentru ca ve?i avea plata pentru faptele voastre!"
\par 8 Când a auzit Asa cuvintele acestea ?i proorocia lui Azaria, fiul lui Oded, proorocul, s-a înfiripat în putere ?i a azvârlit idolii pagâne?ti din toata ?ara lui Iuda ?i a lui Veniamin ?i din ceta?ile pe care le luase din muntele lui Efraim, ?i a înnoit jertfelnicul Domnului care era dinaintea pridvorului.
\par 9 Apoi a adunat pe tot Iuda ?i pe tot Veniaminul ?i pe strainii din Efraim, din Manase ?i din Simeon, care locuiau cu ei, caci mul?i din Israel au trecut la el, când au vazut ca Domnul Dumnezeu este cu el.
\par 10 Ace?tia s-au adunat la Ierusalim în luna a treia din anul al cincisprezecelea al domniei lui Asa.
\par 11 ?i au adus în ziua aceea jertfa Domnului din prada care o adunasera: din vitele cele mari, ?apte sute de jertfe, ?i din vitele marunte, ?apte mii.
\par 12 Dupa aceea au facut legamânt între ei, ca sa caute pe Domnul Dumnezeul parin?ilor lor, cu toata inima lor ?i cu tot sufletul lor;
\par 13 Iar tot cel ce nu va cauta pe Domnul Dumnezeul lui Israel sa moara, ori mic de va fi, ori mare, ori barbat de va fi, ori femeie.
\par 14 ?i s-au legat ei cu juramânt catre Domnul cu glas tare ?i cu strigate mari ?i în sunetul trâmbi?elor ?i al buciumelor.
\par 15 ?i le-a parut bine la to?i Iudeii de acest juramânt, pentru ca cu toata inima lor s-au jurat ?i cu toata staruin?a L-au cautat, ?i El S-a dat lor sa fie gasit. ?i le-a dat lor Domnul odihna din toate par?ile.
\par 16 Pe Maaca, mama sa, regele Asa a lipsit-o de vrednicia de regina, pentru ca ea facuse un chip turnat pentru Astarte. Asa a rasturnat chipul cel turnat, pe care-l facuse ea, l-a sfarâmat în buca?i ?i l-a ars în valea Chedronului.
\par 17 De?i înal?imile din Israel n-au fost de tot departate, ci au mai durat înca în Israel, totu?i Asa ?i-a avut inima sa întreaga la Domnul în toate zilele vie?ii sale.
\par 18 ?i a adus el în templul lui Dumnezeu lucrurile care le afierosise tatal sau ?i lucrurile care le afierosise el: adica aurul, argintul ?i vasele.
\par 19 ?i razboi n-a mai fost pâna în anul al treizeci ?i cincilea al domniei lui Asa.

\chapter{16}

\par 1 Iar în anul al treizeci ?i ?aselea al domniei lui Asa, Bae?a, regele lui Israel, a plecat împotriva lui Iuda ?i a început sa zideasca Rama, pentru ca sa nu poata nimeni nici sa plece de la Asa, regele lui Iuda, nici sa vina la el.
\par 2 Atunci a scos Asa argintul ?i aurul din vistieriile templului Domnului ?i ale casei regelui ?i l-a trimis la Damasc lui Benhadad, regele Siriei, zicând:
\par 3 "Legamânt sa fie între mine ?i între tine, cum a fost între tatal meu ?i între tatal tau. Iata eu î?i trimit argint ?i aur, du-te ?i rupe legamântul tau, care-l ai cu Bae?a, regele lui Israel, ca sa plece de la mine".
\par 4 ?i a ascultat Benhadad de regele Asa ?i a trimis pe mai-marii o?tirii, ce-i avea, împotriva ceta?ilor lui Israel; ace?tia au pustiit Ainul, Danul, Abel-Maimul (Abel-Bet) ?i toate hambarele cu provizii din ceta?ile lui Neftali.
\par 5 Când a auzit Bae?a de aceasta, a încetat sa mai zideasca Rama, oprind lucrarea sa.
\par 6 Iar Asa regele a strâns pe to?i cei ai lui Iuda ?i a carat cu ei din Rama pietrele ?i lemnele pe care Bae?a le folosea la întarire, ?i a zidit cu ele Gheba ?i Mi?pa.
\par 7 În timpul acela a venit Hanani vazatorul la Asa, regele lui Iuda, ?i i-a zis: "Fiindca ai nadajduit în regele Siriei ?i n-ai nadajduit în Domnul Dumnezeul tau, pentru aceea ?i-a scapat oastea regelui Siriei din mâna ta.
\par 8 Oare Etiopienii ?i Libienii n-au avut ei oaste mare ?i care ?i calare?i, carora nu li se mai ?tia numarul? Dar pentru ca tu ?i-ai pus încrederea în Domnul, El ?i i-a dat în mâna ta;
\par 9 Caci ochii Domnului vad oriunde, peste tot pamântul, pentru ca sa sprijine pe cei care sunt cu inima lor întreaga la El. Neîn?elep?e?te ai lucrat dar acum. Pentru aceasta, de acum înainte tu vei avea razboaie".
\par 10 ?i s-a mâniat Asa pe vazator, ?i l-a pus la închisoare; fiindca din pricina aceasta prinsese mare ciuda pe el; ?i a mai apasat în acel timp Asa ?i pe unii din popor.
\par 11 Iar faptele lui Asa de la cele dintâi pâna la cele de pe urma sunt scrise în cartea regilor lui Iuda ?i Israel.
\par 12 în anul al treizeci ?i noualea al domniei sale s-a îmbolnavit Asa de picioare ?i boala lui s-a întins pâna la par?ile cele mai de sus ale corpului; dar el nici la boala sa n-a cautat pe Domnul, ci pe doctori.
\par 13 ?i a raposat Asa cu parin?ii sai ?i a murit în anul al patruzeci ?i unulea al domniei lui.
\par 14 ?i l-au îngropat în mormântul pe care ?i-l sapase el în cetatea lui David; ?i i-au pus pe un pat pe care-l umpluse cu aromate ?i cu tot felul de miresme; la înmormântarea lui i s-au ars foarte multe aromate.

\chapter{17}

\par 1 În locul lui Asa, s-a facut rege Iosafat, fiul sau. ?i s-a întarit ?i el împotriva Israeli?ilor;
\par 2 Caci a a?ezat o?tire în toate ceta?ile întarite ale Iudei ?i în ceta?ile lui Efraim pe care le stapânise Asa, tatal sau.
\par 3 ?i a fost Domnul cu Iosafat, pentru ca acesta a umblat în caile cele dintâi ale lui David, tatal sau, ?i n-a cautat la Baali,
\par 4 Ci a cautat pe Dumnezeul tatalui sau ?i a lucrat dupa poruncile Lui, iar nu dupa faptele Israeli?ilor.
\par 5 Pentru aceasta i-a întarit Domnul domnia în mâna lui ?i tot Iuda îi aducea daruri lui Iosafat, ?i a avut el boga?ie ?i marire multa.
\par 6 Inima lui s-a întarit din ce în ce în caile Domnului. Afara de aceasta, el a desfiin?at din Iuda ?i înal?imile cu jertfelnice ?i A?erele.
\par 7 Iar în anul al treilea al domniei sale, a trimis pe cinci din capeteniile sale ?i anume: pe Benhail, Obadia, Zaharia, Natanael ?i Miheia, ca sa înve?e poporul prin ceta?ile lui Iuda;
\par 8 Împreuna cu ei a trimis ?i din levi?i: pe ?emaia, Netania, Zebadia, Asael, ?emiramot, Ionatan, Adonie, Tobie, Tob-Adonie; iar din preo?i, pe Eli?ama ?i pe Ioram.
\par 9 Ace?tia au înva?at în Iuda, având cu ei cartea legii Domnului; ?i au cutreierat toate ceta?ile lui Iuda ?i au înva?at poporul.
\par 10 În vremea aceasta, în toate regatele ?arilor, care erau împrejurul Iudei, intrase frica Domnului ?i nu îndraznea nimeni sa se lupte cu Iosafat.
\par 11 ?i i se aduceau lui Iosafat daruri ?i argint ca bir, pâna ?i din partea Filistenilor; asemenea ?i Arabii îi aduceau vite marunte; ?apte mii ?apte sute de berbeci ?i ?apte mii ?apte sute de ?api.
\par 12 Astfel a ajuns Iosafat la cea mai înalta marire ?i a zidit în Iuda ceta?ui ?i ceta?i pentru provizii.
\par 13 El a avut multe provizii în ceta?ile lui Iuda ?i oameni de razboi viteji în Ierusalim.
\par 14 Iata cartea numaratorii lor, dupa familiile cele dupa tata ale lor: Din Iuda erau capetenii peste mii: Adna, capetenie, care avea sub el trei sute de mii de osta?i ale?i;
\par 15 Dupa el Iohanan, capetenie, care avea sub el doua sute optzeci de mii de oameni.
\par 16 Dupa acesta, Amasia, fiul lui Zicri, care se fagaduise pe sine Domnului, avea cu el doua sute de mii de osta?i viteji.
\par 17 Din Veniamin: viteazul osta? Eliada, care avea sub el doua sute de mii de osta?i, înarma?i cu arcuri ?i cu scuturi;
\par 18 Dupa el Iehozabad, care avea sub el o suta optzeci de mii de osta?i înarma?i.
\par 19 Ace?tia au slujit pe rege, afara de cei pe care regele îi împar?ise prin ceta?ile întarite în tot ?inutul lui Iuda.

\chapter{18}

\par 1 Iosafat a avut multa boga?ie ?i marire ?i s-a înrudit cu Ahab.
\par 2 Dupa câ?iva ani s-a dus la Ahab în Samaria, iar acesta a junghiat, pentru el ?i pentru poporul ce era cu el, un numar foarte mare de vite marunte ?i mari ?i l-a câ?tigat sa mearga împotriva Ramotului din Galaad.
\par 3 Caci Ahab, regele lui Israel, a zis lui Iosafat, regele lui Iuda: "Vrei sa mergi cu mine la razboi împotriva Ramotului din Galaad?" Acela însa i-a raspuns: "Cum e?ti tu, a?a sunt ?i eu; ?i cum este poporul tau, a?a este ?i poporul meu: merg cu tine la razboi!"
\par 4 ?i a zis Iosafat catre regele lui Israel: "Întreaba astazi, ce zice Domnul?"
\par 5 Atunci a adunat regele lui Israel ca la patru sute de prooroci ?i le-a zis: "Sa mergem noi oare la razboi împotriva Ramotului din Galaad ori sa ne lasam?" ?i ei au zis: "Du-te, ca Dumnezeu îl va da în mâinile regelui".
\par 6 Zis-a Iosafat: "Nu se gase?te pe aici vreun prooroc al Domnului, ca sa-l întrebam ?i pe el?"
\par 7 Iar regele lui Israel a raspuns lui Iosafat: "Mai este un om prin care am putea întreba pe Domnul, dar nu-l iubesc, pentru ca nu prooroce?te de bine pentru mine, ci totdeauna îmi prooroce?te numai de rau; acesta este Miheia, fiul lui Imla". Iosafat însa a zis: "Nu grai a?a, rege!"
\par 8 Dupa aceea a chemat regele lui Israel pe un famen ?i i-a zis: "Du-te repede dupa Miheia, fiul lui Imla".
\par 9 Atunci regele lui Israel ?i Iosafat, regele lui Iuda, ?edeau fiecare pe tronul sau îmbraca?i în haine rege?ti, în locul din fa?a por?ii Samariei, ?i to?i proorocii prooroceau înaintea lor.
\par 10 Sedechia, fiul lui Chenaana î?i facuse ?i coarne de fier ?i zicea: :A?a zice Domnul: Cu acestea îi vei împunge pe Sirieni pâna îi vei nimici".
\par 11 ?i to?i proorocii prooroceau acela?i lucru, zicând: "Du-te împotriva Ramotului din Galaad, ?i vei izbuti, ca Domnul îl va da în mâinile regelui".
\par 12 Trimisul care se dusese sa cheme pe Miheia i-a grait, zicând: "Iata proorocii proorocesc to?i într-un glas de bine regelui; sa fie ?i cuvântul tau asemenea cu al fiecaruia din ei; prooroce?te ?i tu de bine".
\par 13 Miheia însa a zis: "Viu este Domnul, ce-mi va spune Dumnezeul meu, aceea voi prooroci".
\par 14 Dupa aceea a venit el la rege, iar regele i-a zis: "Miheia, sa mergem noi oare la razboi împotriva Ramotului din Galaad sau sa ne lasam?" Iar acela a zis: "Duce?i-va, ca ve?i izbuti, caci aceia vor cadea în mâinile voastre".
\par 15 ?i i-a zis regele: "De câte ori sa te jur eu oare în numele Domnului, ca sa graie?ti numai adevarul?"
\par 16 Atunci Miheia a zis: "Am vazut pe to?i fiii lui Israel risipi?i prin mun?i ca oile care n-au pastor, iar Domnul mi-a zis: Ace?tia neavând domn, sa se întoarca fiecare la casa lui cu pace!"
\par 17 Zis-a regele lui Israel catre Iosafat: "Nu ?i-am spus eu oare, ca el nu-mi prooroce?te de bine, ci numai de rau?"
\par 18 Iar Miheia a zis: "Asculta?i a?adar cuvântul Domnului: Am vazut pe Domnul ?ezând pe tronul Sau ?i toata o?tirea cereasca statea de-a dreapta ?i de-a stânga Lui.
\par 19 Atunci Domnul a zis: "Cine din voi ar putea sa duca în ispita pe Ahab, regele lui Israel, ca sa se duca ?i sa cada în Ramotul din Galaad?" ?i a raspuns unul una ?i altul alta.
\par 20 Atunci a ie?it un duh ?i a stat înaintea fe?ei Domnului ?i a zis: "Eu îl voi duce în ispita". ?i a zis Domnul: "în ce chip?"
\par 21 Acela a zis: "Ma duc ?i ma fac duh al minciunii în gurile tuturor proorocilor lui". Iar Domnul a zis: "Tu îl vei duce în ispita ?i ai sa izbute?ti. Du-te ?i fa a?a!
\par 22 ?i acum iata a slobozit Domnul pe duhul minciunii ca sa intre în gurile acestor prooroci ai tai. Dar Domnul n-a grait de bine pentru tine".
\par 23 ?i s-a apropiat Sedechia, fiul lui Chenaana ?i a lovit pe Miheia peste obraz, zicând: "Pe ce drum anume s-a retras Duhul Domnului de la mine, ca sa graiasca prin tine?"
\par 24 Iar Miheia a zis: "Iata ai sa vezi aceasta în ziua când vei fugi din camara în camara, ca sa te ascunzi".
\par 25 Atunci a zis regele lui Israel: "Lua?i-l pe Miheia, duce?i-l la Amon, capetenia ora?ului, ?i la Ioa?, fiul regelui,
\par 26 ?i le spune?i: A?a graie?te regele: Pune?i pe acesta la închisoare ?i sa-l hrani?i ca în vreme de lipsa, cu pu?ina pâine ?i cu pu?ina apa, pâna când ma voi întoarce cu pace".
\par 27 Dar Miheia a zis: "Daca te vei întoarce sanatos, despre aceasta Domnul n-a grait prin mine". Apoi a zis: "Asculta?i acestea to?i!"
\par 28 Dupa aceea a plecat regele lui Israel ?i Iosafat, regele lui Iuda, împotriva Ramotului din Galaad.
\par 29 Zis-a regele lui Israel catre Iosafat: "Eu îmi schimb hainele ?i intru în lupta; iar tu îmbraca-te în hainele tale cele de rege". ?i ?i-a schimbat hainele regele lui Israel ?i a intrat în lupta.
\par 30 Regele Siriei însa poruncise capeteniilor carelor de razboi, zicând: "Sa nu va lupta?i nici cu mic, nici cu mare, ci numai cu regele lui Israel!"
\par 31 Când capeteniile carelor de razboi au vazut pe Iosafat, ei ?i-au zis: "Acesta este regele lui Israel" ?i l-au împresurat ca sa se lupte cu el. Însa Iosafat a strigat, iar Dumnezeu i-a dat ajutor ?i i-a îndepartat Domnul de la el.
\par 32 ?i, când capeteniile carelor de razboi au vazut ca acesta nu este regele lui Israel, s-au dat înapoi de la el.
\par 33 Dar un om, din întâmplare, ?i-a întins arcul sau atunci ?i a ranit pe regele lui Israel prin încheietura armaturii, iar regele a zis celui ce-i mâna carul: "Întoarce-te ?i scoate-ma din lupta, caci sunt ranit".
\par 34 În acea zi, lupta a fost crâncena ?i regele lui Israel a stat în car în fa?a Sirienilor pâna seara, iar la asfin?itul soarelui a murit.

\chapter{19}

\par 1 Dupa aceea Iosafat, regele lui Iuda, s-a întors în pace la casa lui în Ierusalim.
\par 2 I-a ie?it însa înainte Iehu, fiul lui Hanani vazatorul, ?i a zis regelui Iosafat: "Fiindca ai ajutat pe un nelegiuit ?i ai legat prietenie cu cei pe care-i ura?te Domnul, de aceea mânia Domnului va fi asupra ta.
\par 3 Dar ai ?i fapte bune, caci ai stricat chipurile cele cioplite din pamântul lui Iuda ?i ?i-ai îndreptat inima ta ca sa caute pe Dumnezeu".
\par 4 Atunci Iosafat, s-a a?ezat cu locuin?a în Ierusalim ?i a început el sa umble iara?i prin poporul sau de la Beer-?eba, pâna la muntele lui Efraim ?i l-a întors la Domnul Dumnezeul parin?ilor sai.
\par 5 ?i a a?ezat judecatori în ?ara, prin toate ceta?ile întarite ?i în fiecare cetate.
\par 6 ?i a zis el judecatorilor: "Lua?i seama la ce ve?i face; sa nu face?i judecata omeneasca, ci judecata Domnului, ca la rostirea judeca?ii El este cu voi.
\par 7 De aceea sa fi?i cu frica Domnului în voi, sa lucra?i cu paza, caci la Domnul nu-i nici nedreptate, nici partinire ?i nici daruri".
\par 8 Iosafat a pus ?i în Ierusalim pe unii din levi?i, din preo?i ?i din capii de familie, ca sa faca judecata Domnului, sa judece pricinile dintre Israeli?ii ce locuiau în Ierusalim.
\par 9 ?i iata ce porunci a dat el acestora: "A?a sa lucra?i, în frica Domnului, cu credin?a ?i cu inima curata!
\par 10 La orice fel de pricini, care vor veni înaintea voastra de la fra?ii vo?tri care locuiesc în ceta?ile lor, fie pentru varsare de sânge, sau pentru lege ?i porunca, fie pentru rânduieli ?i ceremonii, sa-i înva?a?i sa nu faca rau, ca sa nu se gaseasca vinova?i înaintea Domnului ?i sa nu vina mânia Lui asupra voastra ?i asupra fra?ilor vo?tri; lucra?i a?a ?i nu ve?i gre?i
\par 11 ?i iata Amaria arhiereul este pus peste voi în toate lucrurile Domnului; iar Zebadia, fiul lui Israel, capetenia casei lui Iuda, este pus pentru toate lucrurile regelui, iar ca dregatori înaintea voastra ave?i pe levi?i. Fi?i tari ?i lucra?i astfel, ?i Domnul va fi cu cel bun".

\chapter{20}

\par 1 Dupa aceasta au venit Moabi?ii ?i dimpreuna cu ei au venit ?i Amon asupra lui Iosafat cu razboi.
\par 2 Atunci au venit unii ?i au spus lui Iosafat, zicând: "Vine o mare mul?ime de popor împotriva ta de dincolo de mare, din Siria; ?i iata-i se afla la Ha?a?on-Tamar, adica la Enghedi".
\par 3 La vestea aceasta s-a înfrico?at Iosafat ?i ?i-a îndreptat fa?a sa, ca sa caute pe Domnul facând cunoscut tuturor din tot Iuda ca sa posteasca.
\par 4 Deci s-a adunat Iuda ca sa ceara ajutor de la Domnul ?i au venit din toate ceta?ile lui Iuda ca sa roage pe Domnul.
\par 5 Iar Iosafat a stat în mijlocul adunarii lui Iuda ?i a Ierusalimului, la templul Domnului dinaintea cur?ii noi
\par 6 ?i a zis: "Doamne Dumnezeul parin?ilor no?tri! Nu e?ti Tu oare Dumnezeu în cer sus, domnind peste toate regatele popoarelor ?i n-ai Tu oare în mâna Ta taria ?i puterea, încât nimeni nu este sa-?i stea ?ie împotriva?
\par 7 Dumnezeul nostru, oare nu Tu ai izgonit pe locuitorii acestei ?ari din fa?a poporului Tau Israel ?i ai dat-o pe veci urma?ilor lui Avraam, pe care îl iubeai?
\par 8 ?i ei s-au a?ezat în ea ?i ?i-au zidit templu sfânt numelui Tau, zicând:
\par 9 De va veni vreo nenorocire peste noi, ori sabia care ne pedepse?te, sau vreo boala molipsitoare, sau foamete, ?i noi vom sta înaintea templului acesta ?i înaintea fe?ei Tale - caci numele Tau este în templul acesta - ?i vom striga în strâmtorarea noastra, catre Tine, Tu sa ne ascul?i ?i sa ne izbave?ti.
\par 10 ?i acum, iata Amoni?ii, Moabi?ii ?i locuitorii muntelui Seir, prin ?ara carora Tu nu le-ai îngaduit Israeli?ilor sa treaca, când veneau ei din pamântul Egiptului, ci au trecut pe lânga ei ?i nu i-au nimicit,
\par 11 Iata ei vor sa ne rasplateasca, fiindca au venit sa ne izgoneasca din stapânirea Ta care ne-ai dat-o de mo?tenire.
\par 12 Dumnezeul nostru, judeca-i Tu, caci noi n-avem nici o putere împotriva acestei mari mul?imi care a venit asupra noastra ?i nu ?tim ce sa facem; dar ochii no?tri ne sunt la Tine".
\par 13 ?i stateau to?i Iudeii înaintea fe?ei Domnului ?i copiii cei mici ai lor, femeile lor ?i feciorii lor.
\par 14 Atunci S-a pogorât Duhul Domnului în mijlocul adunarii peste Iahaziel, fiul Zahariei, fiul lui Benaia, fiul lui Ieiel, fiul lui Matania, levit dintre fiii lui Asaf
\par 15 ?i acesta a zis: "Asculta?i to?i ai lui Iuda ?i locuitorii Ierusalimului, ?i tu rege Iosafat! A?a zice Domnul catre voi: Sa nu va teme?i, nici sa va spaimânta?i de mul?imea aceasta mare, caci razboiul nu este al vostru, ci al lui Dumnezeu.
\par 16 Mâine sa pleca?i împotriva lor; iata ei urca coasta ?i? ?i ave?i sa-i gasi?i în capatul vaii din fa?a pustiei Ieruel.
\par 17 Nu voi ave?i sa va lupta?i de asta data; dar în?ira?i-va, sta?i ?i privi?i izbavirea Domnului, pe care o va trimite El voua. Iuda ?i Ierusalime! Sa nu va teme?i, nici sa va spaimânta?i! Sa le ie?i?i mâine înainte ?i Domnul va fi cu voi!"
\par 18 Atunci s-a plecat Iosafat cu fa?a la pamânt, de asemenea ?i to?i Iudeii ?i locuitorii Ierusalimului au cazut înaintea Domnului, ca sa se închine Domnului.
\par 19 Apoi. s-au sculat levi?ii dintre fiii lui Cahat ?i dintre fiii lui Core, ca sa laude pe Domnul Dumnezeul lui Israel cu glas foarte tare.
\par 20 ?i s-au sculat ei de diminea?a tare ?i au ie?it sa se duca în pustiul Tecoa; pe când ie?eau ei, a stat Iosafat ?i a zis: "Asculta?i-ma Iudeilor, voi, locuitori ai Ierusalimului! Ave?i încredere în Domnul Dumnezeul vostru ?i fi?i tari! Ave?i încredere în proorocii Lui ?i ve?i izbuti cu bine!"
\par 21 Apoi s-a sfatuit el cu poporul ?i a rânduit cântare?i care sa cânte Domnului, ca, ie?ind ace?tia în podoabe sfinte înaintea celor înarma?i, sa preamareasca ?i sa zica: "Slavi?i pe Domnul, ca în veac este mila Lui!"
\par 22 Dar când au început ei sa scoata strigate de lauda ?i sa preamareasca, Domnul a stârnit neîn?elegeri între Amoni?i ?i Moabi?i ?i între locuitorii muntelui Seir, care venisera în Iuda, ?i au fost batu?i;
\par 23 Caci s-au sculat Amoni?ii ?i Moabi?ii împotriva locuitorilor muntelui Seir, batându-i ?i stârpindu-i; iar când au sfâr?it cu locuitorii din Seir, atunci au început sa se ucida ?i ei unul pe altul.
\par 24 ?i când cei din Iuda au ajuns pe dealul de unde se vedea pustiul ?i s-au uitat la acea mul?ime de lume, iata to?i erau stârvuri cazute pe pamânt ?i nici unul nu era teafar.
\par 25 Atunci a venit Iosafat ?i poporul lui ca sa adune prada; ?i au gasit la ei multe boga?ii, haine ?i lucruri scumpe ?i au adunat atâta cât nu puteau sa duca. Trei zile au adunat, atât de multa prada au aflat.
\par 26 Iar în ziua a patra s-au adunat la Emec Beraca, fiindca acolo binecuvântasera ei pe Domnul. Pentru aceasta se ?i cheama locul acela Valea Binecuvântarii pâna în ziua de azi.
\par 27 Apoi to?i barba?ii din Iuda ?i din Ierusalim ?i cu Iosafat în fruntea lor s-au întors foarte voio?i la Ierusalim, pentru ca Domnul le daduse biruin?a asupra vrajma?ilor lor.
\par 28 ?i au venit în Ierusalim la templul Domnului, cu chitare, cu harfe ?i cu trâmbi?e.
\par 29 ?i frica lui Dumnezeu a cuprins pe toate regatele pamântului, când au auzit ca Însu?i Domnul a luptat împotriva vrajma?ilor lui Israel.
\par 30 Dupa aceea domnia lui Iosafat a fost în lini?te, pentru ca Dumnezeu i-a dat odihna din toate par?ile.
\par 31 A?a a domnit Iosafat peste Iuda. Când s-a facut rege, el era de treizeci ?i cinci de ani ?i douazeci ?i cinci de ani a domnit în Ierusalim. Pe mama lui o chema Azuba, fiica lui ?ilhi.
\par 32 El a ?inut drumul lui Asa, tatal sau ?i nu s-a dat în laturi de loc, facând tot fapte ce erau placute în ochii Domnului.
\par 33 Numai înal?imile n-au fost de tot îndepartate ?i nici poporul nu se îndreptase înca cu tarie catre Dumnezeul parin?ilor lor.
\par 34 Celelalte fapte ale lui Iosafat, de la cele dintâi pâna la cele din urma, se afla scrise în istoria lui Iehu, fiul lui Hanani, ?i sunt trecute ?i în cartea regilor lui Israel.
\par 35 Apoi Iosafat, regele lui Iuda, a intrat în legatura cu Ohozia, regele lui Israel, care petrecea în faradelegi,
\par 36 ?i s-a întovara?it cu el, ca sa faca corabii, spre a fi trimise la Tarsis; ?i corabiile le-au facut ei la E?ion-Gheber.
\par 37 Atunci a proorocit Eliezer, fiul lui Dodava, din Mare?a, o proorocie asupra lui Iosafat, zicând: "Fiindca tu ai intrat în legatura cu Ohozia, de aceea Domnul ?i-a sfarâmat lucrurile tale". ?i s-au spart corabiile ?i n-au mai putut sa mai mearga la Tarsis.

\chapter{21}

\par 1 Iosafat a raposat cu parin?ii sai ?i a fost înmormântat la un loc cu parin?ii lui, în cetatea lui David. Iar în locul lui s-a facut rege Ioram, fiul sau.
\par 2 El avea ca fra?i pe fiii lui Iosafat, adica pe Azaria, pe Iehiel, pe Zaharia, pe Azaria, pe Micael ?i pe ?efatia. To?i erau fiii lui Iosafat, regele Iudei.
\par 3 Acestora tatal lor le facuse daruri mari de argint, de aur ?i de lucruri pre?ioase, dimpreuna cu ceta?ile întarite, care erau în Iuda, iar domnia a dat-o lui Ioram, pentru ca el era fiul lui cel întâi-nascut.
\par 4 Însa Ioram, daca a luat domnia tatalui sau ?i s-a întarit, a ucis pe to?i fra?ii lui cu sabia, de asemenea ?i pe unii dintre capeteniile lui Israel.
\par 5 Ioram era de treizeci ?i doi de ani când s-a facut rege ?i a domnit opt ani în Ierusalim.
\par 6 Dar el a umblat pe drumul regilor lui Israel, dupa cum facuse ?i casa lui Ahab, pentru ca el avea de femeie chiar pe o fiica a lui Ahab; ?i a facut lucruri care nu erau placute în ochii Domnului.
\par 7 Totu?i Domnul, pentru legamântul pe care-l facuse cu David, n-a vrut sa piarda de tot casa lui David, caci îi fagaduise sa-i dea lui ?i fiilor lui un sfe?nic luminos în toate timpurile.
\par 8 În zilele lui s-a rupt Edomul de sub stapânirea lui Iuda ?i ?i-au pus pentru ei rege osebit.
\par 9 Atunci a plecat Ioram împotriva lor cu mai-marii o?tirii lui ?i cu toate carele de razboi, care le avea; ?i, sculându-se noaptea, a batut pe to?i Edomi?ii care-l împresurasera ?i pe toate capeteniile carelor acelora; ?i poporul s-a întors repede la taberele lor.
\par 10 Edomul a ramas tot rupt de sub stapânirea lui Iuda pâna în ziua de astazi. Tot în acel timp s-a mai rupt de sub stapânirea lui Iuda ?i Libna, pentru ca el a parasit pe Domnul Dumnezeul parin?ilor sai.
\par 11 De asemenea a facut el înal?imile de pe mun?ii lui Iuda ?i a dus pe locuitorii Ierusalimului la desfrâu ?i pe Iuda la ratacire.
\par 12 Atunci i-a venit o scrisoare de la Ilie proorocul, în care se spunea: "A?a zice Domnul Dumnezeul lui David, stramo?ul tau: Fiindca tu n-ai umblat pe caile lui Iosafat, tatal tau, ?i pe caile lui Asa, regele lui Iuda,
\par 13 Ci ai umblat pe drumul regilor lui Israel ?i ai dus pe Iuda ?i pe locuitorii Ierusalimului la desfrâu, dupa cum a dus ?i casa lui Ahab pe Israel la desfrâu, ba înca ai ucis ?i pe fra?ii tai, fiii tatalui tau, care erau mai buni decât tine,
\par 14 Pentru aceasta, iata Domnul va lovi cu lovitura mare pe poporul tau, pe fiii tai ?i pe femeile tale ?i toata averea ta.
\par 15 Iar pe tine însu?i te va lovi cu boala grea, cu boala maruntaielor tale, pâna ce vor ie?i din tine zi cu zi din pricina bolii".
\par 16 Apoi a întarâtat Domnul împotriva lui Ioram duhul Filistenilor ?i al Arabilor, care se învecineaza cu Etiopienii;
\par 17 ?i au plecat ace?tia împotriva Iudei, au navalit în toata ?ara ?i au luat toata averea care se afla în casa regelui; de asemenea ?i pe fiii lui ?i pe femeile lui ?i nu i-a mai ramas lui nici un fecior, afara de Ohozia, cel mai mic dintre fiii lui.
\par 18 Iar dupa toate acestea l-a lovit Domnul la maruntaiele lui ?i cu o boala fara vindecare.
\par 19 A?a s-a trecut zi cu zi, una dupa alta; iar la sfâr?itul anului al doilea, din pricina bolii lui, au ie?it din el toate maruntaiele ?i a murit în cele mai grozave chinuri. La înmormântarea lui poporul nu i-a mai ars aromate, cum facuse pentru parin?ii lui.
\par 20 El a fost de treizeci ?i doi de ani când s-a facut rege ?i opt ani a domnit în Ierusalim; ?i a murit neplâns ?i l-au îngropat în cetatea lui David; însa nu în gropni?ele regilor.

\chapter{22}

\par 1 În locul lui locuitorii Ierusalimului au facut rege pe Ohozia, fiul cel mai mic al lui, fiindca pe to?i fiii cei mai mari îi ucisese mul?imea care venise cu Arabii la tabara. Astfel s-a facut rege Ohozia, fiul lui Ioram, regele Iudei.
\par 2 Ohozia era de douazeci ?i doi de ani când s-a facut rege ?i a domnit în Ierusalim un an. Pe mama lui o chema Atalia ?i era fiica lui Omri.
\par 3 El de asemenea a mers pe drumul casei lui Ahab, pentru ca a avut pe mama lui sfatuitoare la fapte nelegiuite.
\par 4 De aceea a facut el lucruri care nu erau placute ochilor Domnului, ca ?i casa lui Ahab; pentru ca pe casa aceasta a avut-o el de sfatuitor dupa moartea tatalui sau, spre pieirea lui.
\par 5 Dupa sfatul lor, a plecat cu Ioram, fiul lui Ahab, regele lui Israel, la razboi împotriva lui Hazael, regele Siriei, în Ramotul din Galaad. Atunci arca?ii sirieni au ranit pe Ioram.
\par 6 Acesta, ca sa se lecuiasca de ranile ce capatase la Ramot, când s-a luptat cu Hazael, regele Siriei, s-a întors înapoi la Izreel. ?i Ohozia, fiul lui Ioram, regele Iudei, s-a dus ?i el la Izreel, ca sa vada pe Ioram, fiul lui Ahab, pentru ca acela era bolnav.
\par 7 Cu voia lui Dumnezeu ?i spre pieirea lui a venit Ohozia la Ioram, caci, dupa sosirea lui aici, a plecat cu Ioram împotriva lui Iehu, fiul lui Nim?i, pe care-l unsese Domnul pentru stârpirea casei lui Ahab.
\par 8 Astfel pe când Iehu aducea la îndeplinire judecata care era asupra casei lui Ahab, a gasit ?i pe capeteniile lui Iuda ?i pe fiii fra?ilor lui Ohozia, care-i slujeau, ?i i-a ucis.
\par 9 Apoi Iehu a poruncit sa caute ?i pe Ohozia ?i l-au prins pe când el se ascundea în Samaria, l-au adus la Iehu ?i l-au ucis ?i l-au înmormântat, caci ziceau: "Este fiul lui Iosafat care a cautat pe Domnul din toata inima sa". ?i n-a ramas în casa lui Ohozia nimeni care sa poata domni,
\par 10 Caci Atalia, mama lui Ohozia, vazând ca a murit fiul ei, s-a sculat ?i a stârpit toata samân?a de rege din casa lui Iuda.
\par 11 Dar Io?eba, fiica regelui Ioram, a pus mâna pe Ioa?, fiul lui Ohozia" l-a scos din mijlocul fiilor regelui, care aveau sa fie uci?i ?i l-a ascuns pe el ?i pe doica lui în odaia de dormit. ?i astfel Io?eba, fiica regelui Ioram, femeia lui Iehoiada preotul ?i sora lui Ohozia, a ascuns pe Ioa? din fa?a Ataliei, ?i nu l-a omorât.
\par 12 ?i l-au ?inut ei ascuns în templul lui Dumnezeu ?ase ani; iar peste ?ara a domnit Atalia.

\chapter{23}

\par 1 Iar în anul al ?aptelea s-a întarit Iehoiada ?i a luat de tovara? cu el pe capeteniile de peste sute, adica pe Azaria, fiul lui Ieroham, pe Ismael, fiul lui Iohanan, pe Azaria, fiul lui Obed, pe Maaseia, fiul lui Adaia ?i pe Eli?afat, fiul lui Zicri.
\par 2 ?i au umblat de jur-împrejur prin Iuda ?i au adunat pe levi?ii din toate ceta?ile lui Iuda ?i pe capii de familii dupa tata ai lui Israel ?i au venit la Ierusalim.
\par 3 Aici a facut toata adunarea legamânt cu regele în templul lui Dumnezeu. ?i a zis Iehoiada catre ei: "Iata fiul regelui trebuie sa fie de acum rege, a?a a grait Domnul pentru fiii lui David.
\par 4 Iata ce trebuie sa face?i voi: Trei par?i dintre voi, care intra cu rândul de saptamâna, atât dintre preo?i, cât ?i dintre levi?i, sa fie u?ieri pe la u?i;
\par 5 Trei par?i la casa regelui ?i trei par?i la poarta Iesod; iar poporul sa stea tot în cur?ile templului Domnului.
\par 6 Nimeni sa nu intre în templul Domnului, afara de preo?ii ?i levi?ii care fac slujba. Ei pot sa intre, pentru ca sunt sfin?i?i; iar tot poporul sa stea de straja pentru Domnul.
\par 7 Levi?ii sa înconjoare pe rege din toate par?ile, fiecare cu arma lui în mâna sa; ?i cine va intra în templu sa fie ucis. ?i sa fi?i pe lânga rege, când va intra ?i când va ie?i".
\par 8 ?i au facut levi?ii ?i tot Iuda ce le-a poruncit Iehoiada preotul. Atunci ?i-a luat fiecare oamenii sai, de aceia care intrau eu rândul de saptamâna sau pe acei care î?i facusera saptamâna, pentru ca Iehoiada preotul nu daduse drumul rândurilor care erau sa fie schimbate.
\par 9 ?i a împar?it Iehoiada preotul capeteniilor de peste sute suli?ile ?i scuturile cele mari ?i cele mici ale lui David, care erau în templul lui Dumnezeu.
\par 10 Apoi a a?ezat tot poporul, având fiecare în mâna arma sa, de la partea dreapta a templului pâna la partea stânga a templului, înconjurând jertfelnicul ?i templul, ca sa faca cerc în jurul regelui.
\par 11 Dupa aceea l-au adus pe rege ?i i-au pus coroana ?i podoabele, i-au dat sa ?ina în mâna legea ?i l-au facut rege; iar ungerea a savâr?it-o Iehoiada preotul ?i fiii lui, zicând: "Sa traiasca regele!"
\par 12 Atunci a auzit Atalia strigatele poporului care alerga ?i cânta laude pentru rege ?i a venit ?i ea la popor în templul Domnului sa priveasca;
\par 13 ?i iata regele ?edea pe tronul lui la intrare ?i pe lânga rege, capeteniile, trâmbi?a?ii ?i tot poporul ?arii se veseleau ?i sunau din trâmbi?e; iar cântare?ii, care erau cu instrumente de cântare ?i cei care cântau din gura, cântau cântarile de lauda. Atunci Atalia ?i-a rupt hainele sale ?i a început sa strige: "Vânzare! Vânzare!"
\par 14 Iar Iehoiada preotul a chemat pe capeteniile cele peste sute, care erau cu comanda armatei ?i le-a zis: "Scoate?i-o afara din templu ?i cine va urma dupa ea sa fie ucis cu sabia!" Pentru ca preotul zisese: "Sa nu o ucide?i în templul Domnului!"
\par 15 ?i i-au facut loc sa iasa. Iar când a ajuns ea la intrarea casei regelui, cea dinspre poarta cailor, au omorât-o acolo.
\par 16 ?i a facut Iehoiada legamânt cu tot poporul ?i cu regele, ca sa fie ei popor al Domnului.
\par 17 Apoi s-a dus tot poporul la capi?tea lui Baal ?i au darâmat-o ?i au sfarâmat jertfelnicele lui ?i chipurile lui ?i a omorât pe Matan, preotul lui Baal, dinaintea jertfelnicelor.
\par 18 Dupa aceea a încredin?at Iehoiada dregatoriile templului Domnului în mâinile preo?ilor ?i ale levi?ilor; a împar?it pe preo?i ?i pe levi?i în cete, care trebuiau sa faca slujba cu rândul, cum a?ezase David pentru templul Domnului, ca sa aduca arderi de tot Domnului, cum este scris în legea lui Moise, cu bucurie ?i cu cântare dupa cum rânduise David.
\par 19 A a?ezat de asemenea portari la por?ile templului Domnului, ca sa nu poata intra cel ce este necurat prin vreo fapta.
\par 20 Apoi a luat pe capeteniile cele peste sute, pe viteji, pe capeteniile poporului ?i pe tot poporul ?arii ?i l-au petrecut pe rege de la templul Domnului pâna la casa domneasca, intrând pe poarta cea înalta, în casa regelui; ?i l-au a?ezat pe tronul regatului.
\par 21 ?i s-a veselit tot poporul ?arii ?i s-a lini?tit cetatea. Iar pe Atalia au omorât-o cu sabia.

\chapter{24}

\par 1 Ioa? era de ?apte ani când a fost facut rege ?i a domnit patruzeci de ani în Ierusalim. Pe mama lui o chema ?ibia ?i era din Beer-?eba.
\par 2 Ioa? a facut lucruri placute înaintea ochilor Domnului, în toate zilele lui Iehoiada preotul.
\par 3 Iehoiada i-a luat doua femei ?i a avut cu ele baie?i ?i fete.
\par 4 Dupa aceasta ?i-a pus în gând Ioa? sa înnoiasca templul Domnului.
\par 5 De aceea a adunat pe preo?i ?i pe levi?i ?i le-a zis: "Duce?i-va prin ceta?ile lui Iuda ?i strânge?i de la to?i Israeli?ii argint în fiecare an pentru repararea templului Dumnezeului vostru ?i grabi?i-va cu lucrul acesta". Însa levi?ii nu s-au grabit.
\par 6 Atunci a chemat regele pe Iehoiada, capul lor ?i i-a zis: "Pentru ce nu ceri de la levi?i sa adune din Iuda ?i din Ierusalim darea hotarâta de Moise, robul Domnului, ?i de catre adunarea Israeli?ilor pentru cortul adunarii?
\par 7 Caci Atalia cea nelegiuita ?i fiii ei au despuiat templul lui Dumnezeu ?i toate lucrurile care erau sfin?ite pentru templul Domnului ?i le-au întrebuin?at pentru baali".
\par 8 Apoi a poruncit regele de s-a facut o lada ?i a pus-o afara la u?a templului Domnului.
\par 9 ?i s-a dat de veste la tot poporul din Iuda ?i din Ierusalim ca sa aduca Domnului darea pe care o pusese Moise, robul lui Dumnezeu, asupra Israeli?ilor în pustiu.
\par 10 De aceasta le-a parut bine la toate capeteniile ?i la tot poporul ?i au adus ?i au aruncat argint în lada pâna s-a umplut ea.
\par 11 ?i când la timpul sau se aducea lada de catre levi?i la dregatorii regelui ?i când vedeau ca era mult argint în ea, atunci veneau scriitorul regelui ?i dregatorul arhiereului ?i de?ertau lada, o ridicau ?i o puneau iar la locul ei. A?a faceau în toate zilele ?i s-a strâns argint mult.
\par 12 Iar regele ?i Iehoiada preotul l-au dat me?terilor de lucrari ai templului Domnului ?i ace?tia ?i-au tocmit cioplitori de piatra ?i dulgheri pentru înnoire, asemenea ?i lucratori în fier ?i arama ca sa repare templul Domnului.
\par 13 ?i au luat me?terii ?i au savâr?it lucrarile cu mâinile lor ?i au adus templul lui Dumnezeu la starea lui de mai înainte ?i l-au întarit.
\par 14 Iar dupa ce au sfâr?it toate de lucru, au adus argintul ramas înaintea regelui ?i a lui Iehoiada. ?i au facut din acest argint vase pentru templul Domnului, vase pentru slujbe ?i pentru arderi de tot, cupe ?i celelalte vase de aur ?i de argint. ?i au adus necontenit arderi de tot în templul Domnului în toate zilele lui Iehoiada.
\par 15 Apoi a îmbatrânit Iehoiada ?i a murit satul de zile, caci era de o suta treizeci de ani când a murit.
\par 16 ?i l-au înmormântat în cetatea lui David, la un loc cu regii, pentru binele care îl facuse în Israel, pentru Dumnezeu ?i pentru templul Lui.
\par 17 Iar dupa moartea lui Iehoiada, au venit capeteniile lui Iuda ?i s-au închinat regelui; ?i de atunci regele a început sa asculte de ei.
\par 18 ?i au parasit templul Domnului Dumnezeului parin?ilor lor ?i au început sa slujeasca A?erelor ?i idolilor. Din pricina acestui pacat, s-a coborât mânia Domnului peste Iuda ?i peste Ierusalim.
\par 19 ?i a trimis la ei prooroci, pentru a-i face sa se întoarca la Domnul ?i i-au sfatuit, dar ei nu i-au ascultat.
\par 20 Atunci Duhul lui Dumnezeu a cuprins pe Zaharia, fiul lui Iehoiada preotul, care s-a suit pe amvon înaintea poporului ?i le-a zis: "A?a zice Domnul: Pentru ce calca?i poruncile Domnului? Nu ve?i propa?i, fiindca voi a?i parasit pe Domnul ?i El va va parasi pe voi".
\par 21 Dar s-au vorbit cu to?ii împotriva lui ?i l-au ucis cu pietre, din porunca regelui Ioa?, în curtea templului Domnului.
\par 22 Regele Ioa? nu ?i-a adus aminte de binefacerea care i-a facut-o Iehoiada, tatal lui Zaharia, ci a ucis pe fiul lui ?i acesta, când murea, a zis: "Domnul sa vada ?i sa faca dreptate!"
\par 23 Iar dupa un an, a venit împotriva lui armata siriana, a intrat în Iuda ?i în Ierusalim ?i a stârpit din popor pe toate capeteniile poporului ?i toata prada luata de la ei a trimis-o regelui, la Damasc.
\par 24 De?i o?tirea Sirienilor care venise era alcatuita dintr-un mic numar de oameni, Domnul însa a dat în mâna lor o oaste foarte mare, pentru ca ei parasisera pe Domnul Dumnezeul parin?ilor lor, iar Sirienii aduceau la îndeplinire judecata care era asupra lui Ioa?.
\par 25 Iar dupa ce s-au retras Sirienii de la el, ramânând el greu bolnav, slujitorii lui au uneltit împotriva lui, din cauza sângelui fiului lui Iehoiada preotul; ?i l-au ucis în patul lui ?i a?a a murit. ?i l-au înmormântat în cetatea lui David, însa nu l-au îngropat în mormântul regilor.
\par 26 Iar acei care au uneltit împotriva lui au fost: Zabad, fiul ?imeatei Amonita ?i Iehozabad, fiul ?imritei Moabita.
\par 27 Amanuntele despre fiii lui ?i despre mul?imea proorociilor împotriva lui, precum ?i despre înnoirea templului lui Dumnezeu, se gasesc scrise în cartea regilor. Iar în locul sau a fost facut rege Amasia, fiul sau.

\chapter{25}

\par 1 Amasia a început sa domneasca când era în vârsta de douazeci ?i cinci de ani ?i a domnit douazeci ?i noua de ani în Ierusalim. Pe mama lui o chema Iehoadan ?i era de loc din Ierusalim.
\par 2 El a facut lucruri placute înaintea Domnului, însa nu din toata inima.
\par 3 Când domnia s-a întarit în mâna lui, el a ucis pe robii sai care ucisesera pe rege, tatal sau.
\par 4 Însa pe copiii lor nu i-a ucis, fiindca a facut dupa cum este scris în lege, în cartea lui Moise, unde a poruncit Domnul, zicând: "Parin?ii nu trebuie sa fie uci?i pentru feciori, nici feciorii nu trebuie sa fie uci?i pentru parin?i, ci fiecare sa fie ucis pentru pacatul lui".
\par 5 Apoi a adunat Amasia pe cei din Iuda ?i i-a a?ezat dupa neamurile cele dupa tata sub comanda capeteniilor peste mii ?i a capeteniilor peste sute, pe to?i ai, lui Iuda ?i ai lui Veniamin; ?i i-a socotit de la vârsta de douazeci de ani în sus ?i a gasit trei sute de mii de oameni voinici care erau în stare sa mearga la razboi ?i sa poarte suli?a ?i scut.
\par 6 Apoi a mai tocmit cu plata din Israeli?i înca o suta de mii de osta?i viteji cu o suta de talan?i de argint.
\par 7 Dar un om al lui Dumnezeu a venit la el ?i i-a zis: "O, rege, o?tirea care o ai din Israel sa nu mearga cu tine, pentru ca Domnul nu ?ine cu Israeli?ii, nici cu to?i fiii lui Efraim;
\par 8 Ci du-te singur ?i lupta-te cu barba?ie în razboi. Altfel Dumnezeu te va face sa cazi în fa?a du?manului; caci Dumnezeu are putere ?i sa te sprijine ?i sa te faca sa cazi".
\par 9 Iar Amasia a zis catre omul lui Dumnezeu: "Ce sa fac cu cei o suta de talan?i darui?i o?tirii lui Israel?" ?i a zis omul lui Dumnezeu: "Domnul poate sa-?i dea mai mult decât atâta".
\par 10 Atunci Amasia a ales o?tirea care îi venise din ?ara lui Efraim, ca sa se întoarca la locul sau. Dar ei s-au aprins de mânie asupra lui Iuda ?i s-au întors la locul lor foarte mânio?i.
\par 11 Iar Amasia a prins curaj ?i ?i-a luat poporul sau ?i a plecat la batalie cu el în Valea Sarii ?i a omorât zece mii de oameni dintre fiii lui Seir;
\par 12 ?i alte zece mii de oameni vii i-au luat ca robi fiii lui Iuda ?i i-au dus pe vârful unei stânci ?i i-au aruncat de pe vârful stâncii. ?i to?i s-au zdrobit cu desavâr?ire.
\par 13 Iar o?tirea pe care Amasia o trimisese înapoi, ca sa nu mearga cu el la razboi, a navalit asupra ceta?ilor lui Iuda din Samaria pâna la Bet-Horon ?i au ucis în ele trei mii de oameni, luând foarte multa prada.
\par 14 Iar Amasia dupa ce a lovit pe Edomi?i, a adus, la întoarcerea lui acasa, pe zeii fiilor lui Seir ?i i-a a?ezat sa-i aiba ca dumnezei pentru el ?i s-a închinat lor ?i i-a tamâiat.
\par 15 Atunci Domnul S-a facut foc de mânie asupra lui Amasia ?i a trimis pe un prooroc ?i acela i-a zis: "De ce ai cautat pe dumnezeii poporului acestuia, care n-au fost în stare sa-?i scape poporul din mâna ta?"
\par 16 Când el i-a grait aceste vorbe, regele i-a raspuns: "Te-a pus cineva sa fii sfatuitor regelui? Stapâne?te-te, ca nu cumva sa fii ucis". ?i s-a oprit proorocul ?i a zis: "?tiu ca Dumnezeu a hotarât sa te piarda, pentru ca ai facut acest lucru ?i nu ai ascultat sfatul meu!"
\par 17 Atunci s-a sfatuit Amasia, regele lui Iuda, cu ai sai ?i a trimis la Ioa?, fiul lui Ioahaz, fiul lui Iehu, regele lui Israel, ca sa-i spuna: "Vino sa ne vedem unul cu altul!"
\par 18 Iar Ioa?, regele lui Israel, a trimis la Amasia, regele lui Iuda, ca sa-i spuna: "Spinul, care este în Liban, a trimis sa-i spuna cedrului, care este tot în Liban: Da pe fiica ta fiului meu de femeie. Însa fiarele salbatice, care sunt în Liban, au trecut pe lânga acest spin ?i l-au calcat.
\par 19 Tu zici: Iata eu am batut pe Edomi?i ?i s-a înal?at inima ta de marire de?arta. ?ezi mai bine acasa ?i te astâmpara; de ce sa te apuci de un lucru primejdios, prin care sa cazi ?i tu ?i Iuda cu tine?"
\par 20 Dar Amasia nici n-a vrut sa auda, fiindca aceasta a fost de  la Dumnezeu sa cada el în mâna lui Ioa?, pentru ca a cautat pe dumnezeii Edomi?ilor.
\par 21 ?i a venit Ioa?, regele lui Israel, sa dea ochi unul cu altul, el ?i cu Amasia, regele lui Iuda, la Bet-?eme?, care este în Iuda.
\par 22 ?i Iuda a fost batut de Israel ?i fiecare a fugit la cortul sau.
\par 23 ?i pe Amasia, regele lui Iuda, fiul lui Ioa?, fiul lui Ioahaz, l-a prins Ioa?, regele lui Israel, la Bet-?eme?, ?i l-a dus la, Ierusalim ?i a darâmat zidul Ierusalimului pe o întindere de patru sute de co?i, de la por?ile lui Efraim pâna la por?ile din col?ul ceta?ii;
\par 24 Apoi a luat tot aurul ?i argintul ?i toate vasele care erau în templul lui Dumnezeu, sub îngrijirea lui Obed-Edom ?i vistieriile casei regelui, precum ?i oameni ca ostatici ?i s-a întors cu ei în Samaria.
\par 25 Amasia, fiul lui Ioa?, regele lui Iuda, a mai trait dupa moartea lui Ioa?, fiul lui Ioahaz, cincisprezece ani.
\par 26 Celelalte fapte ale lui Amasia, cele dintâi ?i cele de pe urma, se afla scrise în cartea regilor lui Iuda ?i ai lui Israel.
\par 27 De pe timpul când Amasia s-a abatut de la Domnul, s-a urzit în Ierusalim o uneltire împotriva lui ?i a fugit la Lachi?, dar l-au omorât acolo.
\par 28 ?i l-au adus pe cai ?i l-au înmormântat cu parin?ii lui în cetatea lui Iuda.

\chapter{26}

\par 1 Atunci tot poporul lui Iuda a luat pe Ozia, care era de ?aisprezece ani ?i l-a facut rege în locul lui Amasia, tatal sau.
\par 2 Acesta a zidit Elatul ?i l-a întors la Iuda, dupa ce a raposat regele cu parin?ii lui.
\par 3 Ozia era de ?aisprezece ani când a fost facut rege ?i a domnit cincizeci ?i doi de ani în Ierusalim. Pe mama lui o chema Iecolia, de loc din Ierusalim.
\par 4 Acesta a facut lucruri care erau placute înaintea ochilor Domnului, întocmai a?a cum facuse ?i Amasia, tatal sau;
\par 5 Caci el a alergat la Dumnezeu în zilele lui Zaharia, care îl înva?a frica lui Dumnezeu. ?i în timpul cât a alergat el la Domnul ?i Dumnezeu l-a ajutat de i-a mers bine.
\par 6 Într-o vreme a plecat el de s-a batut cu Filistenii ?i a darâmat zidurile ceta?ii Gat ?i zidurile ceta?ii Iabne ?i zidurile ceta?ii A?dod; ?i a zidit ceta?i în ?inutul A?dodului ?i între Filisteni.
\par 7 Iar Dumnezeu i-a ajutat sa se bata cu Filistenii ?i cu Arabii care locuiesc la Gur-Baal ?i cu Meuni?ii.
\par 8 A?a ca chiar Amoni?ii îi dadeau daruri lui Ozia încât s-a dus vestea de numele lui pâna la hotarele Egiptului, caci ajunsese foarte puternic.
\par 9 Apoi a zidit Ozia turnuri în Ierusalim, deasupra por?ilor din col? ?i deasupra por?ilor din vale, cum ?i la col?ul zidului ?i l-a întarit.
\par 10 A zidit de asemenea turnuri ?i în pustiu ?i a sapat multe fântâni, pentru ca avea multe vite ?i pe ?es ?i pe vale; avea lucratori de pamânt ?i vieri în munte ?i pe Carmel, caci el iubea lucrarea pamântului.
\par 11 A mai avut Ozia ?i o?tire care mergea la batalie în cete, dupa cum era rânduita în catagrafia alcatuita de Ieiel, scriitorul, ?i de Maasia, judecatorul, care se aflau sub conducerea lui Hanania, unul din capeteniile de frunte ale regelui.
\par 12 Numarul total al capilor de familii ale razboinicilor viteji era de doua mii ?ase sute;
\par 13 ?i sub conducerea lor era o putere osta?easca de trei sute ?apte mii cinci sute de osta?i, care puteau sa mearga la lupta cu barba?ie osta?easca, pentru a ajuta pe rege împotriva du?manului.
\par 14 Pentru ei a pregatit Ozia toata oastea, scuturi, suli?e, coifuri ?i plato?e, arcuri ?i pietre de pra?tie.
\par 15 Apoi a facut în Ierusalim ma?ini, construite cu me?te?ug, ca sa le puna în turnuri ?i pe la col?urile zidurilor ?i sa arunce cu ele sage?i ?i pietre mari. ?i s-a dus vestea de numele lui foarte departe, pentru ca în chip minunat a fost ajutat ?i s-a facut puternic.
\par 16 ?i când a ajuns puternic, el s-a mândrit în inima lui spre pieirea lui; ?i a savâr?it o nelegiuire înaintea Domnului Dumnezeului sau, caci a intrat în templul Domnului, ca sa tamâieze pe jertfelnicul tamâierii.
\par 17 Dar dupa el a intrat ?i Azaria preotul, împreuna cu optzeci de preo?i ai Domnului, oameni ale?i,
\par 18 ?i s-au împotrivit regelui Ozia ?i i-au zis: "Nu ?i-e dat ?ie, Ozia, sa tamâiezi înaintea Domnului, ci preo?ilor, fiilor lui Aaron, care sunt sfin?i?i sa tamâieze; ie?i din loca?ul sfânt, caci ai facut o faradelege ?i nu-?i va fi de loc spre cinste acest lucru înaintea Domnului Dumnezeu".
\par 19 Ozia însa s-a suparat pe ei; ?i cum ?inea în mâna lui cadelni?a, ca sa tamâieze, deodata s-a ivit lepra pe fruntea lui în fa?a preo?ilor, în templul Domnului, dinaintea altarului tamâierii.
\par 20 Iar daca s-au uitat la el cu luare aminte Azaria arhiereul ?i to?i preo?ii, iata el avea pe fruntea lui lepra. ?i l-au silit sa iasa de acolo; dar ?i el însu?i s-a grabit sa iasa, caci îl lovise Domnul.
\par 21 ?i a fost lepros regele Ozia pâna în ceasul mor?ii lui; dupa aceea el a trait ascuns într-o casa osebit, fiind oprit de a mai intra în templul Domnului. Iar îngrijirea peste casa regelui ?i cârmuirea poporului ?arii a ?inut-o Iotam, fiul lui.
\par 22 Celelalte fapte ale lui Ozia, cele dintâi ?i cele de pe urma, le-a scris Isaia proorocul, fiul lui Amos.
\par 23 ?i a raposat Ozia cu parin?ii lui ?i l-au înmormântat la un loc cu parin?ii lui în câmpul cu mormintele regilor, ca ziceau: El a fost lepros. ?i în locul lui a fost facut rege Iotam, fiul sau.

\chapter{27}

\par 1 Iotam era de douazeci ?i cinci de ani când a fost facut rege ?i a domnit ?aisprezece ani în Ierusalim. Pe mama lui o chema Ieru?a, fiica lui Sadoc.
\par 2 Acesta a facut lucruri placute înaintea ochilor Domnului, întocmai cum a facut ?i Ozia, tatal sau, numai ca n-a intrat, ca el, în loca?ul sfânt al Domnului. ?i poporul continua înca sa se strice.
\par 3 El a zidit por?ile cele de sus ale templului Domnului ?i a facut multe adaosuri la zidul Ofel.
\par 4 La fel a zidit ?i ceta?i pe muntele lui Iuda, iar în paduri, ceta?ui ?i turnuri.
\par 5 Iotam a avut razboi cu regele Amoni?ilor ?i i-a biruit. În acel an Amoni?ii i-au dat o suta de talan?i de argint, zece mii de core de grâu ?i zece mii core de orz. Aceasta dare i-au dat-o Amoni?ii ?i în al doilea ?i în al treilea an.
\par 6 ?i s-a facut Iotam a?a de puternic, pentru ca ?i-a îndreptat caile sale numai înaintea fe?ei Domnului Dumnezeului sau.
\par 7 Celelalte fapte ale lui Iotam, toate razboaiele sale ?i purtarea sa se gasesc scrise în cartea regilor lui Israel ?i a regilor lui Iuda.
\par 8 El era de douazeci ?i cinci de ani când a fost facut rege ?i a domnit în Ierusalim ?aisprezece ani.
\par 9 Apoi a raposat Iotam cu parin?ii sai, ?i l-au înmormântat în cetatea lui David. Iar în locul lui a fost facut rege Ahaz, fiul sau.

\chapter{28}

\par 1 Ahaz era de douazeci de ani când a fost facut rege ?i a domnit ?aisprezece ani în Ierusalim; însa el n-a facut lucruri placute înaintea ochilor Domnului, precum facuse David, stramo?ul lui,
\par 2 Ci a mers pe urmele regilor lui Israel ?i a facut chiar chipuri turnate de baali.
\par 3 A savâr?it tamâieri în valea fiilor lui Hinom ?i a trecut pe fiii sai prin foc facând urâciunile popoarelor pe care le alungase Domnul dinaintea fiilor lui Israel.
\par 4 A adus jertfe ?i tamâieri pe înal?imi, pe dealuri ?i pe sub orice copac înfrunzit.
\par 5 De aceea l-a dat Domnul Dumnezeu în mâna regelui Sirienilor, care l-au lovit ?i i-au luat o mul?ime de robi ?i i-au dus în Damasc. De asemenea a mai fost dat ?i în mâna regelui lui Israel, care ?i acela a pricinuit mari pierderi regatului sau;
\par 6 Caci Pecah, fiul lui Remalia, regele lui Israel, a ucis într-o singura zi o suta douazeci de mii în Iuda, to?i numai oameni de razboi, pentru ca aceia parasisera pe Domnul Dumnezeul parin?ilor lor.
\par 7 Iar Zicri, un viteaz din Efraim, a ucis pe Maaseia, fiul regelui, pe Azricam, capetenia cur?ii, ?i pe Elcana, care era al doilea dupa rege.
\par 8 ?i au luat fiii lui Israel de la fra?ii lor din Iuda doua sute de mii de femei, baie?i ?i fete, ca robi; de asemenea au luat de la ei ?i multa prada ?i au dus prada în Samaria.
\par 9 Acolo însa se afla un prooroc al Domnului, care se numea Oded. Acesta a ie?it înaintea o?tirii care venea la Samaria ?i le-a zis: "Iata Domnul Dumnezeul parin?ilor vo?tri, fiind mâniat pe Iuda, i-a dat în mâinile voastre, ?i voi i-a?i ucis cu o salbaticie care a ajuns pâna la cer.
\par 10 ?i acum va gândi?i sa va face?i robi ?i roabe pe fiii lui Iuda ?i ai Ierusalimului.
\par 11 A?adar, asculta?i-ma ?i duce?i înapoi pe robii care i-a?i luat de la fra?ii vo?tri; caci altfel flacara mâniei Domnului va fi peste voi".
\par 12 Atunci s-au sculat unii din capeteniile fiilor lui Efraim: Azaria, fiul lui Iohanan, Berechia, fiul lui Me?ilemot, Iezechia, fiul lui ?alum ?i Amasa, fiul lui Hadlai, împotriva celor care se întorceau din razboi
\par 13 ?i le-au zis: "Sa nu aduce?i pe robi aici, pentru ca am fi vinova?i înaintea Domnului. Vre?i sa mai adauga?i la pacatele noastre ?i la vinova?iile noastre? Caci ?i a?a este mare vinova?ia noastra ?i vapaia mâniei Domnului este peste Israel".
\par 14 Atunci osta?ii au luat pe robi ?i prazile înaintea capeteniilor o?tirii ?i a întregii adunari.
\par 15 ?i s-au sculat barba?ii aminti?i mai sus, au luat pe robi ?i din prada au îmbracat pe to?i cei goi, le-au dat haine ?i încal?aminte, i-au hranit, i-au adapat ?i i-au uns cu untdelemn, ?i pe to?i care erau slabi i-au pus pe asini ?i i-au dus la Ierihon, cetatea finicilor, la fra?ii lor; iar ei s-au întors apoi la Samaria.
\par 16 În acea vreme a trimis regele Ahaz la regele Asirienilor, ca sa-l ajute,
\par 17 Ca Edomi?ii iar au venit ?i au ucis pe mul?i din Iuda ?i au luat robi;
\par 18 Asemenea ?i Filistenii navalisera în ceta?ile din ?inuturile de la ?es ?i de la miazazi ale lui Iuda ?i luasera Bet-?eme?ul, Aialonul, Ghederotul ?i Soco cu satele care ?ineau de el, Timna cu satele care ?ineau de ea, ?i Ghimzo cu satele lui ?i se a?ezasera acolo.
\par 19 A?a smerise Domnul pe Iuda din pricina lui Ahaz, regele lui Iuda, pentru ca dusese el pe Iuda la destrabalare ?i pacatuise greu înaintea Domnului.
\par 20 Tiglatfalasar, regele Asiriei, a venit într-adevar la el; însa în loc sa-l ajute, i-a facut greuta?i,
\par 21 Pentru ca Ahaz luase vistieriile din templul Domnului ?i din casa regelui ?i de la capeteniile poporului ?i le daduse regelui Asiriei, dar nu spre bine.
\par 22 Caci el ?i atunci, când se afla în strâmtorare, n-a încetat a savâr?i faradelege înaintea Domnului. A?a era regele Ahaz.
\par 23 El a adus jertfe dumnezeilor din Damasc, crezând ca ei l-au batut, ?i zicea: "Fiindca dumnezeii regilor Sirienilor le-au ajutat lor, le voi aduce jertfa, ?i ei îmi vor ajuta ?i mie". Însa ei au fost pricina caderii lui ?i pricina caderii a tot Israelul.
\par 24 Ahaz a strâns vasele templului lui Dumnezeu, le-a sfarâmat ?i a încuiat u?ile templului Domnului; ?i ?i-a facut jertfelnice pe la toate col?urile în Ierusalim;
\par 25 ?i prin toate ceta?ile lui Iuda a facut locuri înalte ca sa tamâieze la al?i dumnezei. Prin aceasta a mâniat pe Domnul Dumnezeul parin?ilor sai.
\par 26 Celelalte fapte ale lui ?i toate caile lui, cele dintâi ?i cele de pe urma, se gasesc scrise în cartea regilor lui Iuda ?i ai lui Israel.
\par 27 ?i a raposat Ahaz cu parin?ii sai ?i l-au înmormântat în cetate, în Ierusalim; dar nu l-au pus în gropni?ele regilor lui Israel. În locul lui s-a facut rege Iezechia, fiul sau.

\chapter{29}

\par 1 Iezechia a fost facut rege când era de douazeci ?i cinci de ani ?i a domnit douazeci ?i noua de ani în Ierusalim. Pe mama lui o chema Abia ?i era fiica lui Zaharia.
\par 2 Acesta a facut lucruri placute înaintea ochilor Domnului, dupa cum facuse ?i David, stramo?ul lui.
\par 3 În anul întâi al domniei sale, în luna întâi, a descuiat el u?ile templului Domnului ?i le-a înnoit.
\par 4 Apoi a poruncit sa vina preo?ii ?i levi?ii; pe ace?tia i-a adunat în locul deschis dinspre rasarit,
\par 5 ?i le-a zis: "Asculta?i-ma, levi?i! Sa va cura?i?i acum în?iva prin jertfe ?i sa sfin?i?i templul Domnului Dumnezeului parin?ilor vo?tri ?i sa arunca?i necura?enia din locul sfânt afara.
\par 6 Caci parin?ii vo?tri au savâr?it faradelegi ?i au facut lucruri care nu erau placute ochilor Domnului Dumnezeului nostru, pe Care L-au parasit, ?i-au abatut fe?ele de la loca?ul Domnului ?i s-au întors cu spatele la el;
\par 7 Ba ?i u?ile pridvorului le-au încuiat, au stins candelele ?i n-au mai tamâiat tamâie ?i nici n-au mai adus arderi de tot în loca?ul sfânt al Domnului lui Israel.
\par 8 De aceea s-a coborât mânia Domnului peste Iuda ?i peste Ierusalim ?i i-a dat El pustiirii, batjocurii ?i ru?inii, cum vede?i ?i voi singuri cu ochii vo?tri.
\par 9 Iata parin?ii no?tri au cazut de ascu?i?ul sabiei, iar fiii no?tri, fetele noastre ?i femeile noastre se gasesc din pricina aceasta în robie pâna astazi, într-o ?ara care nu este a lor.
\par 10 Acum însa mâ-am pus în gând sa fac legamânt cu Domnul Dumnezeul lui Israel, ca sa-?i întoarca iu?imea mâniei Sale de la noi.
\par 11 Fiii mei, sa nu pregeta?i nicidecum, caci pe voi v-a ales Domnul, ca sa-I sta?i înaintea fe?ei Lui, sa-I sluji?i ?i sa-I fi?i slujitori ?i aprinzatori de tamâie".
\par 12 Atunci s-au sculat levi?ii: Mahat, fiul lui Amasia ?i Ioil, fiul lui Azaria, dintre fiii lui Cahat; Chi?, fiul lui Abdi ?i Azaria, fiul lui Iehaleleel, din semin?ia lui Merari; Ioah, fiul lui Zima ?i Eden, fiul lui Ioah, din neamul lui Gher?on;
\par 13 ?imri ?i Ieiel din fiii lui Eli?afan; Zaharia ?i Matania, din fiii lui Asaf;
\par 14 Iehiel ?i ?imei, din fiii lui Eman; ?emaia ?i Uziel din fiii lui Iedutun.
\par 15 Ace?tia au adunat pe fra?ii lor ?i s-au cura?it prin aduceri de jertfe; apoi s-au dus, dupa porunca regelui, sa cure?e ?i templul Domnului, dupa cuvintele Domnului.
\par 16 ?i au intrat preo?ii înauntrul templului Domnului pentru cura?ire ?i au scos afara în curtea templului tot ce au gasit necurat în loca?ul sfânt al Domnului; iar levi?ii au luat acestea ca sa le scoata afara la pârâul Chedron.
\par 17 Apoi au început a face sfin?irea în ziua întâi a lunii întâi ?i în ziua a opta a aceleia?i luni au intrat în pridvorul templului Domnului. Templul Domnului l-au sfin?it opt zile; în ziua a ?aisprezecea din luna întâi l-au terminat.
\par 18 Dupa aceea s-au dus la regele Iezechia acasa ?i i-au spus: "Noi am cura?it templul Domnului ?i jertfelnicul cel pentru arderi de tot, cu toate vasele lui; masa pentru pâinile punerii înainte, cu toate vasele ei;
\par 19 ?i toate vasele pe care le aruncase regele Ahaz în timpul domniei lui, în nelegiuirea sa, noi le-am pregatit ?i le-am sfin?it; ?i iata ele sunt înaintea jertfelnicului Domnului".
\par 20 Atunci s-a sculat regele Iezechia foarte de diminea?a, a adunat capeteniile ceta?ii ?i s-a dus la templul Domnului.
\par 21 ?i au adus ?apte vi?ei, ?apte berbeci, ?apte miei ?i ?apte ?api, ca jertfa pentru pacat: pentru regat, pentru loca?ul de sfin?ire ?i pentru Iuda. A poruncit apoi preo?ilor, fiilor lui Aaron, ca sa-i aduca arderi de tot pe jertfelnicul Domnului.
\par 22 Au junghiat vitei ?i au luat preo?ii sângele ?i au stropit cu el jertfelnicul; au junghiat berbecii ?i au stropit cu sângele lor jertfelnicul; au junghiat mieii ?i au stropit cu sângele lor de asemenea jertfelnicul.
\par 23 Apoi au adus ?apii cei pentru pacat înaintea regelui ?i a adunarii; ace?tia ?i-au pus mâinile peste ei.
\par 24 ?i i-au junghiat preo?ii ?i au cura?it cu sângele lor jertfelnicul pentru ?tergerea pacatelor întregului Israel, caci regele poruncise sa se aduca arderi de tot ?i jertfa pentru pacat, pentru sine ?i pentru tot Israelul.
\par 25 Apoi a a?ezat în templul Domnului levi?i cu chimvale, cu harpe ?i cu chitare, dupa rânduiala lui David ?i a lui Gad, vazatorul regelui, ?i a lui Natan proorocul, fiindca de catre Domnul se a?ezase aceasta rânduiala prin proorocii Lui.
\par 26 ?i stateau levi?ii cu instrumentele de cântare ale lui David ?i preo?ii cu trâmbi?ele.
\par 27 ?i a poruncit Iezechia sa se aduca arderi de tot pe jertfelnic. ?i în timpul când a început arderea de tot, a început ?i cântarea pentru Domnul cu trâmbi?ele ?i cu instrumentele de cântare ale lui David, regele lui Israel.
\par 28 ?i toata adunarea a facut rugaciuni, iar cântare?ii au cântat din fluiere, pâna s-a terminat arderea de tot.
\par 29 Iar dupa terminarea arderii de tot, regele ?i to?i care se aflau cu el s-au plecat ?i s-au închinat.
\par 30 Iezechia regele ?i capeteniile au zis catre levi?i, sa slaveasca pe Domnul cu cuvintele lui David ?i ale lui Asaf vazatorul, ?i ei L-au slavit cu mare bucurie ?i s-au plecat la pamânt ?i s-au închinat.
\par 31 Apoi a început Iezechia a grai ?i a zis: "Acum v-a?i sfin?it pe voi Domnului; apropia?i-va ?i aduce?i jertfe ?i prinoase de mul?umire în templul Domnului". ?i a adus toata adunarea jertfe ?i prinoase de lauda ?i de împacare, ?i cel pe care-l lasa inima, arderi de tot.
\par 32 Numarul vitelor pentru arderile de tot, aduse de cei ce se adunasera a fost: ?aptezeci de boi, o suta de berbeci, doua sute de miei. Toate acestea au fost jertfite ca arderi de tot Domnului.
\par 33 Pentru celelalte jertfe sfinte au fost ?ase sute de vite mari ?i trei mii de vite marunte.
\par 34 Însa preo?ii au fost pu?ini ?i nu puteau sa jupoaie toate arderile de tot; de aceea le-au ajutat ?i levi?ii, fra?ii lor, pâna ce au terminat lucrul ?i pâna ce s-au sfin?it preo?ii; caci levi?ii au fost mai silitori la sfin?ire decât preo?ii.
\par 35 Afara de aceasta a mai fost o mul?ime de arderi de tot cu grasimi de jertfe de împacare ?i cu turnari de vin la arderi de tot. A?a s-a a?ezat la loc slujba în templul Domnului.
\par 36 ?i s-a bucurat Iezechia dimpreuna cu tot poporul, ca Dumnezeu a potrivit a?a pe popor, încât s-a facut acest lucru, cum nici nu se a?teptase.

\chapter{30}

\par 1 Apoi a trimis Iezechia prin toata ?ara lui Israel ?i a lui Iuda ?i a scris scrisori lui Efraim ?i lui Manase, ca sa vina la templul Domnului la Ierusalim pentru sarbatorirea Pa?tilor Domnului Dumnezeului lui Israel.
\par 2 Caci la sfatul pe care-l ?inuse regele în Ierusalim, împreuna cu capeteniile, chibzuisera ca sa sarbatoreasca Pa?tile în luna a doua,
\par 3 Din pricina ca nu le putuse sarbatori la timpul lor, pentru ca nu erau nici preo?i sfin?i?i în numar de ajuns ?i nici poporul nu se strânsese la Ierusalim.
\par 4 Acest lucru a placut regelui ?i întregii adunari.
\par 5 ?i au hotarât sa se dea de veste la tot Israelul, de la Beer-?eba pâna la Dan, ca sa vina la Ierusalim pentru sarbatorirea Pa?tilor Domnului Dumnezeului lui Israel, caci demult nu se mai sarbatorise a?a cum este scris.
\par 6 ?i s-au dus trimi?ii cu scrisorile, care erau facute de rege ?i de capeteniile lui, prin toata ?ara lui Israel ?i a lui Iuda ?i, dupa porunca regelui, le spuneau: "Fii ai lui Israel, întoarce?i-va la Domnul Dumnezeul lui Avraam ?i al lui Isaac ?i al lui Israel, ?i Se va întoarce ?i El la cei rama?i dintre voi, care a?i mai scapat din mâna regilor Asiriei.
\par 7 ?i nu mai fi?i a?a ca parin?ii vo?tri ?i ca fra?ii vo?tri, care au savâr?it faradelegi înaintea Domnului Dumnezeului parin?ilor lor, ?i El i-a dat pustiirii, precum vede?i.
\par 8 Nici sa nu va mai ?ine?i de acum tari la cerbice, ca parin?ii vo?tri, ci supune?i-va Domnului ?i venin la loca?ul de sfin?ire al Lui, pe care El l-a sfin?it pe veci; ?i sluji?i Domnului Dumnezeului vostru ?i El Î?i va abate flacara mâniei Sale de la voi.
\par 9 Când va ve?i întoarce la Domnul atunci fra?ii vo?tri ?i fiii vo?tri au sa gaseasca ?i ei mila la acei care i-au luat pe ei robi ?i se vor întoarce în ?ara aceasta; caci bun ?i îndurat este Domnul Dumnezeul vostru ?i nu-?i va întoarce fa?a de la voi, daca va ve?i întoarce la El".
\par 10 ?i au mers trimi?ii din cetate în cetate prin ?inutul lui Efraim ?i al lui Manase ?i pâna la acela al Zabulonului; aceia râdeau, batându-?i joc de ei.
\par 11 Totu?i unii din semin?ia lui A?er, a lui Manase ?i a lui Zabulon s-au smerit ?i au venit la Ierusalim.
\par 12 ?i a fost mâna Domnului peste Iuda, Care le-a daruit o singura inima, ca sa împlineasca porunca regelui ?i a capeteniilor, dupa cuvântul Domnului.
\par 13 ?i s-a adunat la Ierusalim mul?ime de popor pentru sarbatorirea praznicului azimelor, în luna a doua ?i adunarea a fost foarte mare.
\par 14 ?i s-au sculat ?i au rasturnat jertfelnicele care erau în Ierusalim ?i toate jertfelnicele pe care se savâr?eau tamâierile pentru idoli, le-au sfarâmat ?i le-au aruncat în pârâul Chedron.
\par 15 Apoi au junghiat mielul Pa?tilor în ziua a paisprezecea a lunii a doua. Preo?ii ?i levi?ii, ru?inându-se, s-au sfin?it ?i au adus arderi de tot în templul Domnului.
\par 16 Apoi au stat la locul lor, dupa rânduiala pe care o aveau prin legea lui Moise, omul lui Dumnezeu. Preo?ii stropeau cu sângele pe care-l luau din mâna levi?ilor.
\par 17 Fiindca în adunare erau mul?i din cei care nu erau cura?i, de aceea, pentru cei necura?i?i, levi?ii junghiau mielul Pa?tilor
\par 18 ?i mul?i din popor, din semin?ia lui Efraim, a lui Manase, a lui Isahar ?i a lui Zabulon, care nu se sfin?isera, au mâncat Pa?tile împotriva Scripturii.
\par 19 Dar Iezechia s-a rugat pentru ei, zicând: "Domnul cel bun sa ierte pe tot cel ce ?i-a îndreptat inima sa caute pe Domnul Dumnezeul parin?ilor sai, de?i ei n-au cura?irea ceruta pentru cele sfinte!"
\par 20 ?i a ascultat Domnul pe Iezechia ?i a iertat poporul.
\par 21 ?i au sarbatorit fiii lui Israel, care s-au aflat în Ierusalim, sarbatoarea azimilor ?apte zile cu mare veselie. În fiecare zi levi?ii ?i preo?ii laudau pe Domnul cu instrumente de cântare facute spre preaslavirea Domnului.
\par 22 Apoi a grait Iezechia catre to?i levi?ii care au buna pricepere la serviciul Domnului, dupa inima lor. ?i au mâncat azimile de sarbatoare ?apte zile, aducând jertfe de pace ?i slavind pe Domnul Dumnezeul parin?ilor lor.
\par 23 ?i a hotarât toata adunarea ca sa mai sarbatoreasca alte ?apte zile, ?i le-au petrecut pe acestea cu veselie,
\par 24 Pentru ca Iezechia, regele lui Iuda, daduse celor ce se adunasera o mie de vi?ei ?i zece mii de vite marunte; capeteniile dadusera de asemenea celor ce se adunasera o mie de vi?ei ?i zece mii de vite marunte, ?i mul?i preo?i se sfin?isera.
\par 25 ?i s-a veselit toata ob?tea lui Iuda ?i preo?ii ?i levi?ii, toata adunarea ?i strainii care venisera din ?ara lui Israel ?i locuiau în Iuda.
\par 26 ?i a fost veselie mare în Ierusalim, pentru ca din zilele lui Solomon, fiul lui David, regele lui Israel, nu se mai facuse nici o veselie ca aceasta în Ierusalim.
\par 27 Apoi s-au sculat preo?ii ?i levi?ii ?i au binecuvântat poporul; ?i glasul lor a fost auzit, caci rugaciunea lor s-a suit pâna la loca?ul cel sfânt al Domnului, Care este în ceruri.

\chapter{31}

\par 1 Dupa ce s-au terminat toate acestea, s-au dus to?i Israeli?ii care se gaseau acolo în ceta?ile lui Iuda ?i au sfarâmat idolii, au taiat A?erele ?i au stricat locurile înalte ?i jertfelnicele din Iuda ?i din tot pamântul lui Veniamin, al lui Efraim ?i al lui Manase, pâna la margini. Apoi s-au întors to?i fiii lui Israel, fiecare la mo?ia sa, în ceta?ile lor.
\par 2 Iar Iezechia a a?ezat cetele de preo?i ?i de levi?i, dupa împar?irea lor, pe fiecare la slujba sa, pe care o avea de preot sau de levit, pentru ca sa faca slujbe, laude ?i doxologii, la timpul arderilor de tot ?i al jertfelor de împacare, la por?ile cur?ii templului Domnului.
\par 3 De asemenea a hotarât regele o parte din averea sa pentru arderile de tot, adica pentru arderile de tot de diminea?a ?i de seara; pentru arderile de tot din ziua odihnei, de la lunile noi ?i de la sarbatori, dupa cum este scris în legea Domnului.
\par 4 Apoi a poruncit el ?i poporului din Ierusalim sa dea preo?ilor ?i levi?ilor între?inerea hotarâta, pentru ca ei sa fie mai cu tragere de inima la împlinirea legii Domnului.
\par 5 Când s-a adus la cuno?tin?a tuturor aceasta porunca, atunci fiii lui Israel au adus prinoase de pâine, de vin, de untdelemn, de miere ?i din toate roadele ?arinii, din bel?ug; au adus de asemenea din bel?ug ?i zeciuieli din toate.
\par 6 Dar ?i Israeli?ii ?i cei din Iuda, care locuiau prin ceta?ile lui Iuda, au adus asemenea zeciuieli din vitele mari ?i din vitele marunte, cum ?i zeciuieli din jertfele pe care le fagaduisera ei Domnului Dumnezeului lor, ?i le-au facut gramezi.
\par 7 Gramezile au început sa le faca în luna a treia, iar în luna a ?aptea le-au sfâr?it.
\par 8 Atunci a venit Iezechia ?i cu capeteniile lui ?i au vazut gramezile ?i au binecuvântat pe Domnul ?i poporul lui Israel.
\par 9 Apoi a întrebat Iezechia pe preo?i ?i pe levi?i despre aceste gramezi: "Pentru ce gramezile stau a?a?"
\par 10 ?i a raspuns Azaria arhiereul, care era din casa lui ?adoc, ?i a zis: "De când a început a se aduce prinoase în templul Domnului, de atunci am mâncat ?i noi de ne-am saturat ?i a mai ?i ramas din bel?ug, pentru ca Domnul a binecuvântat pe poporul Sau. ?i aceasta gramada mare este din cele ce au ramas".
\par 11 Atunci Iezechia a poruncit sa se pregateasca în templul Domnului camari; ?i daca le-au pregatit,
\par 12 Au carat acolo prinoasele de pârga, zeciuielile ?i darurile, cu toata credincio?ia. ?i a pus ca ispravnic peste ele pe Conania levitul ?i pe fratele sau ?imei, ca al doilea.
\par 13 Iar pe Iehiel, Azazia, Nahat, Asael, Ierimot, Iozabad, Eliel, Ismachia, Mahat ?i pe Benaia, i-a pus ca supraveghetori sub mâna lui Conania ?i a lui ?imei, fratele sau, dupa cum rânduisera regele Iezechia ?i Azaria, capetenia templului lui Dumnezeu.
\par 14 Core, fiul lui Imna levitul, care era portar în partea de rasarit, era supraveghetor peste prinoasele care se aduceau de bunavoie lui Dumnezeu; ca sa împarta prinoasele de pârga aduse Domnului ?i lucrurile mai de seama care se sfin?isera.
\par 15 Sub mâna lui se aflau Eden, Miniamin, Iosua, ?emaia, Amaria ?i ?ecania, în ceta?ile preo?ilor, ca sa împarta cu credincio?ie fra?ilor lor par?ile ce li se cuveneau, precum celui mare, a?a ?i celui mic,
\par 16 În afara celor scri?i în catagrafii, tuturor celor de parte barbateasca de la trei ani în sus, tuturor celor care intrau în templul Domnului la slujba, pentru trebuin?ele zilnice, dupa dregatoriile lor ?i dupa cete,
\par 17 Precum ?i preo?ilor înscri?i în catagrafie, dupa familiile lor, ?i levi?ilor de la douazeci de ani în sus, dupa dregatoriile lor ?i dupa cetele lor,
\par 18 Celor înscri?i în catagrafie ?i la to?i nevârstnicii lor, cu femeile lor, cu baie?ii lor ?i cu fetele lor, adica la toata ob?tea de neam preo?esc; caci ei se aratasera cu toata credincio?ia la sfânta slujba.
\par 19 Pentru fiii lui Aaron, care erau preo?i în satele dimprejurul ceta?ilor lor, erau pu?i anume barba?i, la fiecare cetate ca sa le împarta tuturor celor de parte barbateasca dintre preo?i par?ile ce li se cuveneau; precum ?i la to?i levi?ii înscri?i în catagrafie.
\par 20 A?a a facut Iezechia în tot Iuda. El a facut lucruri care erau bune, drepte ?i adevarate în fa?a Domnului Dumnezeului sau.
\par 21 ?i tot lucrul pe care l-a început pentru slujba din templul lui Dumnezeu, pentru pazirea legii ?i a poruncilor, fiind cu gândul la Dumnezeul sau, el l-a facut cu toata tragerea sa de inima ?i a avut spor la ei.

\chapter{32}

\par 1 Dupa atâtea lucruri ?i atâta credincio?ie, a venit Sanherib, regele Asirienilor, ?i a intrat în Iuda ?i a împresurat ceta?ile întarite ?i ?i-a facut planul ca sa le rapeasca pentru el.
\par 2 Când Iezechia a vazut ca a venit Sanherib cu gândul ca sa lupte împotriva Ierusalimului,
\par 3 A hotarât împreuna cu sfetnicii ?i cu vitejii sai ca sa astupe izvoarele de apa care erau afara din cetate ?i ace?tia l-au ajutat.
\par 4 Atunci s-a adunat o mul?ime de popor ?i a astupat toate izvoarele ?i pârâul care curgea prin mijlocul ?arii, zicând: Sa nu vina regele Asiriei ?i gasind apa multa, sa se întareasca.
\par 5 ?i s-a întarit Iezechia ?i a facut la loc tot zidul care se stricase ?i a ridicat turnuri, a zidit pe din afara un alt zid, a înal?at zidul ceta?ii lui David ?i a pregatit o mul?ime de arme ?i de scuturi.
\par 6 A pus capetenii osta?e?ti peste popor ?i i-a adunat pe lânga el în locul larg de la poarta vaii ?i le-a grait pe inima lor zicând:
\par 7 "Întari?i-va ?i îmbarbata?i-va, sa nu va teme?i, nici sa va înfrico?a?i de fa?a regelui ?i de fa?a întregului neam care este cu el, caci cu noi sunt mai mul?i decât cu el.
\par 8 Cu el sunt bra?e de carne, iar cu noi este Domnul Dumnezeul nostru Care ne ajuta ?i Care lupta în bataliile noastre". ?i s-a încurajat poporul de cuvintele lui Iezechia, regele lui Iuda.
\par 9 Dupa aceasta Sanherib, regele Asiriei, care era cu toata o?tirea lui în fa?a Lachi?ului, a trimis pe ni?te slujba?i ai sai la Ierusalim, la Iezechia, regele lui Iuda, ?i la to?i cei din Iuda care erau în Ierusalim, ca sa le spuna:
\par 10 "A?a zice Sanherib, regele Asiriei: Pe ce va bizui?i de ?ede?i închi?i în cetatea Ierusalimului?
\par 11 Nu vede?i ca Iezechia va amage?te, ca sa va dea mor?ii prin foame ?i prin sete, zicând: Domnul Dumnezeul nostru ne va scapa din mâna regelui Asiriei?
\par 12 Nu vede?i ca acest Iezechia a desfiin?at locurile înalte ale lui ?i jertfelnicele lui ?i a spus lui Iuda ?i Ierusalimului: Sa va închina?i înaintea unui singur jertfelnic ?i numai pe el sa tamâia?i?
\par 13 Oare nu ?ti?i ce am facut eu ?i parin?ii mei tuturor popoarelor ?arilor? Putut-au oare dumnezeii popoarelor acestor ?ari sa le scape ?ara lor din mâna mea?
\par 14 Care din to?i dumnezeii popoarelor, pe care le-au pierdut parin?ii mei, a putut sa-?i scape poporul din mâna mea? Cum dar Dumnezeul vostru va va putea scapa din mâna mea?
\par 15 ?i acum pazi?i-va, ca sa nu va mai amageasca Iezechia, nici sa nu va mai înduplece astfel! Sa nu-l crede?i! Caci daca n-a fost în stare nici un dumnezeu de al nici unui popor ?i regat sa-?i scape poporul sau din mâna mea ?i din mâna parin?ilor mei, atunci nici Dumnezeul vostru nu va va scapa din mâna mea".
\par 16 Înca ?i altele multe au vorbit robii lui împotriva Domnului Dumnezeu ?i împotriva lui Iezechia, robul Lui.
\par 17 Ba scrisese el ?i scrisori prin care hulea pe Domnul Dumnezeul lui Israel ?i în care graia împotriva Lui astfel de cuvinte: "Precum dumnezeii popoarelor pamântului n-au scapat pe popoarele lor din mâna mea, a?a nici Dumnezeul lui Iezechia nu va scapa pe poporul Sau din mâna mea".
\par 18 ?i strigau cu glas tare în limba Iudeilor catre poporul din Ierusalim, care era pe zid, ca sa-i îngrozeasca ?i sa-i sperie ?i sa le ia cetatea.
\par 19 Ei vorbeau despre Dumnezeul Ierusalimului, ca despre dumnezeii popoarelor pamântului, care sunt lucruri de mâini omene?ti.
\par 20 Atunci s-a rugat regele Iezechia ?i Isaia proorocul, fiul lui Amos, ?i au strigat cu glas mare la cer.
\par 21 ?i a trimis Domnul pe un înger care a pierdut pe tot viteazul ?i razboinicul ?i capetenia ?i generalul din tabara regelui Asiriei, încât acesta s-a întors cu ru?ine în ?ara sa; ?i când a intrat în casa dumnezeului sau, l-au ucis cu sabia acolo fiii lui.
\par 22 A?a a scapat Domnul pe Iezechia ?i pe locuitorii Ierusalimului din mâna lui Sanherib, regele Asiriei, ?i din mâna tuturor celorlal?i ?i i-a aparat din toate par?ile.
\par 23 Atunci mul?i au adus daruri Domnului în Ierusalim ?i lucruri scumpe lui Iezechia, regele Iudei, care dupa aceasta a câ?tigat în ochii tuturor popoarelor marire mare.
\par 24 În zilele acelea s-a îmbolnavit Iezechia de moarte ?i s-a rugat Domnului ?i Domnul l-a auzit ?i i-a dat semn.
\par 25 Însa Iezechia n-a fost recunoscator pentru binefacerea care i s-a facut, caci s-a seme?it în inima lui. ?i a cazut mânia lui Dumnezeu peste el ?i peste Iuda ?i peste Ierusalim.
\par 26 Dar îndata ce Iezechia s-a smerit pentru mândria inimii lui ?i cu el împreuna ?i locuitorii Ierusalimului, mânia Domnului nu s-a mai coborât asupra lor în zilele lui Iezechia.
\par 27 Iezechia a avut boga?ie ?i marire foarte mare; ?i ?i-a facut vistierii de pastrat argint, aur, pietre scumpe, aromate, scuturi ?i tot felul de vase pre?ioase.
\par 28 A facut de asemenea ?i hambare pentru roade: grâu, vin ?i untdelemn; a?ezari ?i iesle pentru tot felul de vite ?i staule pentru turme.
\par 29 ?i-a zidit ?i ceta?i ?i a avut o mul?ime de vite mari ?i de vite marunte, pentru ca Dumnezeu îi daduse lui foarte multa avere.
\par 30 Tot acest Iezechia a astupat gura de sus a apelor Ghihonului ?i le-a facut sa curga în jos prin partea de apus a ceta?ii lui David. ?i la tot lucrul lui, Iezechia a lucrat cu spor.
\par 31 Când trimi?ii regelui Babilonului au venit la el sa-l întrebe pentru semnul care se savâr?ise în ?ara, atunci l-a parasit Dumnezeu, ca sa-l încerce ?i sa cunoasca tot ce avea el în inima lui,
\par 32 Celelalte fapte ale lui Iezechia ?i milosteniile lui sunt scrise în vedenia lui Isaia-proorocul, fiul lui Amos, ?i în cartea regilor lui Iuda ?i Israel.
\par 33 Apoi a raposat Iezechia cu parin?ii lui ?i l-au îngropat în rândul de sus al mormintelor fiilor lui David ?i tot Iuda ?i locuitorii din Ierusalim i-au facut mare cinste la moartea lui. Iar în locul lui a fost facut rege Manase, fiul sau.

\chapter{33}

\par 1 Manase era de doisprezece ani când s-a facut rege ?i a domnit cincizeci ?i cinci de ani în Ierusalim.
\par 2 Acesta a facut lucruri neplacute înaintea ochilor Domnului, umblând dupa urâciunile popoarelor izgonite de Domnul din fa?a fiilor lui Israel.
\par 3 Caci a facut din nou locurile înalte pe care le sfarâmase Iezechia, tatal sau, ?i a a?ezat jertfelnice pentru baali, a facut A?ere, s-a închinat la toata o?tirea cereasca ?i i-a slujit;
\par 4 A facut jertfelnice ?i în templul Domnului, despre care Domnul zisese: "În Ierusalim va fi numele Meu în veci";
\par 5 A zidit jertfelnice pentru toata o?tirea cereasca în amândoua cur?ile templului Domnului.
\par 6 Tot el a trecut prin foc pe fiii sai în valea Ben-Hinom ?i a facut vrajitorie, farmece ?i magie; a adus oameni care chemau duhurile mor?ilor ?i fermecatori; ?i a înmul?it relele împotriva Domnului, mâniindu-L.
\par 7 Apoi a facut un idol cioplit ?i l-a a?ezat în templul lui Dumnezeu, de?i Dumnezeu graise catre David ?i catre Solomon, fiul lui: "În templul acesta ?i în Ierusalimul pe care l-am ales dintre toate semin?iile lui Israel, îmi voi pune numele Meu în veac;
\par 8 Mai mult, nu voi îngadui ca piciorul lui Israel sa pa?easca afara din pamântul acesta, pe care l-am dat parin?ilor lor, numai daca ei vor fi staruitori în a face tot ce le-am spus, dupa toata legea, rânduielile ?i poruncile date prin Moise".
\par 9 Însa Manase a dus pe Iuda ?i pe locuitorii Ierusalimului la atâta ratacire, încât ei au savâr?it mai rau decât acele popoare pe care Domnul le stârpise din fa?a fiilor lui Israel.
\par 10 De aceea a grait Domnul lui Manase ?i poporului sau, dar ei n-au ascultat.
\par 11 De aceea a adus Domnul peste ei pe capeteniile armatei regelui Asiriei, care l-au prins pe Manase cu arcanul ?i l-au legat cu catu?e de fier ?i l-au dus la Babilon.
\par 12 ?i în strâmtorarea sa, el a cautat fa?a Domnului Dumnezeului sau ?i s-a smerit foarte înaintea Dumnezeului parin?ilor sai.
\par 13 Iar daca s-a rugat, Dumnezeu l-a auzit ?i i-a ascultat rugaciunea lui ?i l-a adus înapoi la Ierusalim, în regatul sau. ?i a cunoscut Manase ca Domnul este Dumnezeul cel adevarat.
\par 14 Dupa aceea a zidit el zidul cel din afara al ceta?ii lui David, în partea de apus a Ghihonului, pe vale ?i pâna la poarta pe?tilor, a înconjurat Ofelul, l-a facut înalt ?i a a?ezat capetenii de oaste prin toate ceta?ile întarite ale lui Iuda.
\par 15 Apoi a doborât pe dumnezeii cei straini ?i idolul cel din templul Domnului ?i toate capi?tele pe care le zidise pe muntele templului Domnului ?i în Ierusalim le-a aruncat afara din cetate.
\par 16 A facut la loc jertfelnicul Domnului ?i a adus pe el jertfe de împacare ?i de lauda iar lui Iuda i-a spus sa slujeasca Domnului Dumnezeului lui Israel.
\par 17 Poporul mai aducea jertfe pe locurile înalte, dar numai pentru Domnul Dumnezeul sau.
\par 18 Celelalte fapte ale lui Manase ?i rugaciunea lui catre Dumnezeul sau ?i cuvintele vazatorilor, care i s-au grait în numele Domnului Dumnezeului lui Israel, se gasesc scrise în istoria regilor lui Israel.
\par 19 ?i rugaciunea lui ?i cum l-a ascultat pe el Dumnezeu ?i toate pacatele lui ?i faradelegile lui ?i locurile în care el a zidit locuri înalte ?i a a?ezat chipurile Astartei ?i idolii, înainte de a se smeri, se gasesc scrise în istoria lui Hozai.
\par 20 Apoi a raposat Manase cu parin?ii sai ?i l-au îngropat în gradina casei lui, iar în locul lui s-a facut rege Amon, fiul sau.
\par 21 Amon era de douazeci ?i doi de ani când s-a facut rege ?i a domnit doi ani în Ierusalim.
\par 22 ?i a facut el lucruri neplacute înaintea ochilor Domnului, precum facuse ?i Manase, tatal sau; caci a adus jertfe la toate chipurile cioplite, pe care le facuse Manase, tatal sau, ?i le-a slujit.
\par 23 Dar el nu s-a smerit înaintea fe?ei Domnului, cum s-a smerit Manase, tatal sau; dimpotriva, Amon ?i-a înmul?it pacatele sale.
\par 24 ?i au uneltit slugile lui împotriva lui ?i l-au omorât în casa sa.
\par 25 Însa poporul ?arii a omorât pe to?i care uneltisera împotriva regelui Amon. ?i în locul lui poporul ?arii a facut rege pe Iosia, fiul sau.

\chapter{34}

\par 1 Iosia era de opt ani când s-a facut rege ?i a domnit treizeci ?i unu de ani în Ierusalim.
\par 2 Acesta a facut lucruri placute înaintea ochilor Domnului ?i a umblat pe caile lui David, stramo?ul sau ?i nu s-a abatut nici la dreapta nici la stânga;
\par 3 Caci în anul al optulea al domniei lui, fiind înca tânar, a început sa caute pe Dumnezeul lui David, stramo?ul sau, iar în anul al doisprezecelea a început sa cure?e Iuda ?i Ierusalimul de locurile înalte, de A?ere, de idolii ciopli?i ?i de idolii turna?i.
\par 4 A sfarâmat în fa?a lui jertfelnicele baalilor ?i idolii de pe ele; a taiat A?erele ?i a prefacut în praf idolii ciopli?i ?i idolii turna?i ?i praful l-a risipit pe mormintele celor ce le adusesera jertfe;
\par 5 A ars oasele preo?ilor pe jertfelnicele lor ?i a cura?it Iuda ?i Ierusalimul.
\par 6 De asemenea ?i ceta?ile lui Manase, ale lui Efraim, ale lui Simeon, ba ?i pe ale semin?iei lui Neftali ?i locurile pustiite dimprejurul lor.
\par 7 Sfarâmând jertfelnicele ?i stricând A?erele, facând chipurile cioplite praf ?i sfarâmând to?i idolii din tot pamântul lui Israel s-a întors la Ierusalim.
\par 8 În anul al optsprezecelea al domniei sale, dupa cura?irea ?arii ?i a templului lui Dumnezeu, el a trimis pe ?afan, fiul lui A?alia, pe Maaseia, capetenia ceta?ii ?i pe Ioab, fiul lui Ioahaz, cronicarul, ca sa înnoiasca templul Domnului Dumnezeului sau.
\par 9 Ace?tia au venit la Hilchia arhiereul ?i i-au dat argintul care se adusese în templul lui Dumnezeu ?i pe care levi?ii cei ce stateau de paza la u?i îl adunasera din mâinile semin?iilor lui Manase ?i ale lui Efraim ?i ale tuturor celorlal?i Israeli?i, de la to?i cei din Iuda ?i Veniamin ?i de la locuitorii Ierusalimului
\par 10 ?i l-au dat în mâinile me?terilor de lucrari care erau tocmi?i la templul Domnului, ca sa-l împarta lucratorilor, care lucrau la templul Domnului, la repararea ?i consolidarea lui.
\par 11 ?i ei l-au împar?it la dulgheri ?i la zidari, ca sa cumpere pietre cioplite, bârne pentru legaturi ?i pentru pus la acoperi?ul cladirilor pe care le stricasera regii lui Iuda.
\par 12 Oamenii ace?tia au lucrat cinstit la lucrare ?i au avut ca supraveghetori peste ei pe Iahat ?i pe Obadia care erau levi?i dintre fiii lui Merari; pe Zaharia ?i pe Me?ulam, care erau dintre fiii lui Cahat, precum ?i pe to?i levi?ii care ?tiau sa cânte cu instrumente muzicale.
\par 13 Tot ace?tia erau pu?i ?i peste salahori ?i supravegheau pe to?i lucratorii la fiecare lucru; ?i dintre levi?i erau scriitori, judecatori ?i portari.
\par 14 Când au luat ei argintul care se adusese în templul Domnului, atunci Hilchia preotul a gasit cartea legii Domnului care fusese data prin mâinile lui Moise.
\par 15 ?i a început Hilchia sa graiasca ?i a zis catre ?afan scriitorul: "Am gasit cartea legii în templul Domnului!" ?i Hilchia a dat acea carte lui ?afan.
\par 16 Iar ?afan s-a dus cu cartea la rege ?i i-a dus regelui o data cu cartea ?i ?tirea: "Tot ce ai încredin?at robilor tai se va face!
\par 17 Apoi au dus argintul care s-a gasit în templul Domnului ?i l-au dat în mâinile supraveghetorilor ?i în mâinile me?terilor de lucrari.
\par 18 Dupa aceea ?afan, scriitorul regelui, a mai facut cunoscut, zicând: "Hilchia preotul mi-a dat o carte". ?i ?afan a citit-o înaintea regelui.
\par 19 Când a auzit regele cuvintele legii ?i-a rupt hainele sale.
\par 20 ?i a dat porunca regele lui Hilchia, lui Ahicam, fiul lui ?afan, lui Abdon, fiul lui Miheia, lui ?afan scriitorul, lui Asaia, slujitorul regelui, zicând:
\par 21 "Duce?i-va de întreba?i pe Domnul despre mine ?i despre cei ce au ramas în Israel ?i despre acei din Iuda, pentru cuvintele car?ii acesteia care s-a gasit, pentru ca mare este mânia Domnului care s-a aprins peste noi, din pricina ca parin?ii no?tri n-au pazit cuvintele Domnului, ca sa faca dupa cum este scris în cartea aceasta".
\par 22 Atunci s-a dus Hilchia ?i cei ce erau din partea regelui la Hulda, prooroci?a, femeia lui ?alum, ve?mântarul, fiul lui Tochat, fiul lui Hasra. Ea locuia în despar?amântul al doilea în Ierusalim ?i au vorbit cu ea de aceasta.
\par 23 Iar ea le-a spus: "A?a zice Domnul Dumnezeul lui Israel: Spune?i acelui om care v-a trimis la mine:
\par 24 A?a zice Domnul: Iata voi aduce nenorociri peste locul acesta ?i peste locuitorii lui toate blestemele, care sunt scrise în cartea aceasta pe care aii citit-o în fa?a regelui lui Iuda,
\par 25 Din pricina ca M-au parasit ?i au tamâiat pe al?i dumnezei, ca sa Ma supere cu toate faptele mâinilor lor. Mânia Mea se va aprinde peste acest loc ?i nu se va potoli.
\par 26 Iar catre regele lui Iuda care v-a trimis sa întreba?i pe Domnul, a?a sa-i spune?i: A?a zice Domnul Dumnezeul lui Israel pentru cuvintele pe care tu le-ai auzit:
\par 27 Fiindca inima ta s-a înmuiat ?i tu te-ai smerit înaintea lui Dumnezeu, auzind cuvintele Lui pentru locul acesta ?i pentru locuitorii lui ?i tu te-ai smerit înaintea Mea ?i ?i-ai rupt hainele tale ?i ai plâns înaintea Mea ?i Eu te-am auzit, zice Domnul:
\par 28 Iata Eu te voi a?eza la un loc cu parin?ii tai ?i vei fi pus în mormântul tau cu pace; ochii tai nu vor vedea nenorocirile pe care le voi aduce peste locul acesta ?i peste locuitorii lui". ?i i-au adus regelui raspunsul acesta.
\par 29 ?i a trimis regele ?i a adunat pe to?i batrânii lui Iuda ?i ai Ierusalimului,
\par 30 ?i s-a dus regele la templul Domnului ?i, împreuna cu el, tot luda ?i locuitorii Ierusalimului, preo?ii, levi?ii ?i tot poporul de la mic pâna la mare; ?i el le-a citit în auzul lor toate cuvintele car?ii legamântului care s-a gasit în templul Domnului.
\par 31 ?i a stat regele la locul lui ?i a facut legamânt în fa?a Domnului, ca sa urmeze Domnului ?i sa pazeasca poruncile Lui, descoperirile ?i rânduielile Lui, cu toata inima ?i cu tot sufletul lor, ca sa împlineasca cuvintele legamântului, care sunt scrise în cartea aceasta.
\par 32 Apoi a poruncit regele ca acest legamânt sa aiba tarie pentru to?i cei ce se afla în Ierusalim ?i în toata ?ara lui Veniamin; ?i au început locuitorii Ierusalimului sa se poarte dupa legamântul Domnului Dumnezeului parin?ilor lor
\par 33 ?i a alungat Iosia toate urâciunile, din toate ?inuturile, carora se închinau fiii lui Israel; ?i a poruncit tuturor celor care se aflau în pamântul lui Israel sa slujeasca Domnului Dumnezeului lor. ?i în toate zilele vie?ii lui nu s-au abatut ei de la Domnul Dumnezeul parin?ilor lor.

\chapter{35}

\par 1 În vremea aceea a sarbatorit Iosia Pa?tile Domnului în Ierusalim ?i a junghiat mielul Pa?tilor în ziua a paisprezecea a lunii întâi.
\par 2 A a?ezat pe preo?i la locul lor ?i i-a obligat sa slujeasca în templul Domnului;
\par 3 Iar levi?ilor, care înva?au pe to?i Israeli?ii ?i care erau sfin?i?i pentru Domnul, le-a zis: "Pune?i chivotul cel sfânt în templul pe care l-a zidit Solomon, fiul lui David, regele lui Israel; nu mai ave?i nevoie sa-l mai purta?i pe umeri, ci sluji?i acum Domnului Dumnezeului vostru ?i poporului Sau Israel
\par 4 ?i va rândui?i dupa neamurile voastre parinte?ti ?i dupa cetele voastre, cum a scris David, regele lui Israel ?i cum a scris Solomon, fiul lui;
\par 5 A?eza?i-va în loca?ul sfânt dupa cetele voastre între fiii poporului, fra?ii vo?tri, ?i dupa cetele în care sunt împar?i?i levi?ii, dupa neamurile lor.
\par 6 Junghia?i mielul Pa?tilor ?i va sfin?i?i ?i-l pregati?i pentru fra?ii vo?tri, ca sa faca ei dupa cuvântul Domnului care s-a dat prin Moise".
\par 7 Dupa aceea a dat Iosia ca dar fiilor poporului, tuturor care se aflau acolo, tot pentru jertfa Pa?tilor, din vite marunte un numar de treizeci de mii de miei ?i de ?api tineri ?i trei mii de boi. Acestea erau din averea regelui.
\par 8 Capeteniile lui au facut de buna voie un dar poporului, preo?ilor ?i levi?ilor: Hilchia, Zaharia ?i Iehiel, capeteniile templului lui Dumnezeu, au dat preo?ilor pentru jertfa Pa?tilor, doua mii ?ase sute de oi, miei ?i ?api, ?i trei sute de boi;
\par 9 Conania, ?emaia ?i Natanael fra?ii lui, ?i Ha?abia, Ieiel ?i Iozabad, capeteniile levi?ilor, au daruit levi?ilor pentru jertfa Pa?tilor cinci mii de oi ?i cinci sute de boi.
\par 10 A?a s-a înjghebat slujba: preo?ii s-au a?ezat la locul lor ?i levi?ii dupa cetele lor ?i dupa porunca regelui;
\par 11 ?i au junghiat mielul Pa?tilor ?i au stropit preo?ii cu sânge, luându-l din mâinile levi?ilor; iar levi?ii jupuiau pielea de pe animalele de jertfa.
\par 12 Apoi au rânduit cele gatite pentru arderile de tot, ca sa le împarta poporului, cum era acesta împar?it în cete dupa familii, ca ei sa le aduca Domnului, cum este scris în cartea lui Moise. Tot a?a au facut ei ?i cu boii.
\par 13 ?i au fript mielul Pa?tilor pe foc, dupa rânduiala; ?i jertfele cele sfinte le-au fiert în caldari, în oale ?i în tigai ?i le-au împar?it fara multa truda la tot poporul,
\par 14 Iar pentru ei ?i pentru preo?i au gatit dupa aceasta, caci preo?ii, fiii lui Aaron, au fost ocupa?i cu aducerea de jertfe ?i de grasimi pâna noaptea; ?i de aceea levi?ii au gatit ?i pentru ei ?i pentru preo?i, fiii lui Aaron.
\par 15 ?i cântare?ii, fiii lui Asaf, au stat la locurile lor, dupa rânduiala lui David, a lui Asaf, a lui Heman ?i a lui Iedutun, vazatorii regelui, asemenea ?i portarii au stat la fiecare poarta; ?i ei nici nu aveau nevoie sa lipseasca de la slujba lor, fiindca levi?ii, fra?ii lor, gateau pentru ei.
\par 16 A?a s-a gatit în acea zi toata slujba care a fost pentru Domnul, ca sa se sarbatoreasca Pa?tile ?i sa se aduca arderile de tot pe jertfelnicul Domnului, dupa porunca regelui Iosia.
\par 17 Fiii lui Israel, care se aflau acolo, au praznuit în vremea aceea Pa?tile ?i sarbatoarea azimelor, timp de ?apte zile.
\par 18 Pa?ti ca acestea însa nu mai fusesera sarbatorite în Israel din zilele lui Samuel proorocul; nici unul dintre regii lui Israel n-a sarbatorit Pa?tile a?a cum a sarbatorit Iosia cu preo?ii ?i levi?ii, cu tot Iuda ?i Israelul care se aflau acolo, ?i cu locuitorii Ierusalimului.
\par 19 Pa?tile acestea au fost sarbatorite în anul al optsprezecelea al domniei lui Iosia.
\par 20 Dupa toate acestea, care le-a facut Iosia în templul lui Dumnezeu, ?i cum regele Iosia a ars cu foc pe cei ce graiesc din pântece, pe magi, capi?tile, idolii ?i A?erele ce erau în Ierusalim ?i în Iuda, ca sa se pazeasca cuvintele Domnului, scrise în cartea pe care o gasise Hilchia preotul în templul Domnului, n-a fost ca el nici unul dintre regii dinainte de el, ca sa se fi întors la Domnul cu toata inima lui ?i cu tot sufletul lui ?i cu toata taria lui, dupa toata legea lui Moise, ?i nici dupa el nu s-a ridicat ca el. Totu?i Domnul nu ?i-a potolit de tot iu?imea cea mare a mâniei Lui, iu?ime cu care Domnul Se mâniase pe Iuda, pentru toate faradelegile pe care le savâr?ise Manase. ?i a zis Domnul: "?i pe Iuda îl voi lepada de la fa?a Mea, cum am lepadat casa lui Israel; ?i voi lepada cetatea Ierusalimului, pe care am ales-o, ?i templul despre care am zis: Numele Meu va fi acolo". ?i a venit Neco, regele Egiptului, cu razboi asupra Carchemi?ului, pe Eufrat; ?i Iosia i-a ie?it înainte.
\par 21 În vremea aceea a trimis Neco soli la el, ca sa-i spuna: "Ce treaba am eu cu tine, rege al lui Israel? Nu împotriva ta vin eu acum, ci merg acolo unde am razboi. ?i Dumnezeu mi-a poruncit sa grabesc; sa nu te împotrive?ti lui Dumnezeu, Care este cu mine, ca sa nu te piarda".
\par 22 Dar Iosia nu s-a dat îndarat dinaintea lui, ci s-a pregatit sa se lupte cu el; ?i n-a ascultat de cuvintele lui Neco, care erau din gura lui Dumnezeu, ci a ie?it la lupta în valea Meghido.
\par 23 Atunci arca?ii au tras asupra regelui Iosia; ?i a zis regele slugilor sale: "Duceri-ma de aici, pentru ca sunt greu ranit".
\par 24 ?i l-au luat slugile lui din car ?i l-au a?ezat în alt car care-l avea el ?i l-au dus la Ierusalim. ?i a murit ?i a fost înmormântat cu parin?ii sai. ?i tot Iuda ?i Ierusalimul l-au jelit pe Iosia.
\par 25 Asemenea ?i Ieremia l-a jelit pe Iosia într-o cântare de jale. De Iosia au vorbit ?i to?i cântare?ii ?i toate cântare?ele în cântarile lor de jelire, care s-au pastrat pâna azi ?i se întrebuin?eaza în Israel. Ele se gasesc scrise în cartea cântarilor de jelire.
\par 26 Celelalte lucrari ale lui Iosia ?i faptele cele bune ale lui, pe care le-a savâr?it dupa cele scrise în legea Domnului,
\par 27 Cum ?i faptele lui, cele dintâi ?i cele de pe urma, se gasesc scrise în cartea regilor lui Israel ?i ai lui Iuda.

\chapter{36}

\par 1 Atunci a luat poporul ?arii pe Ioahaz, fiul lui Iosia, l-au uns ?i l-au facut rege în locul tatalui sau, în Ierusalim.
\par 2 Ioahaz era de douazeci ?i trei de ani când s-a facut rege ?i a domnit trei luni în Ierusalim. Pe mama lui o chema Hamutal ?i era fiica lui Ieremia din Libna. Acesta a facut lucruri rele înaintea Domnului, întocmai cum facusera parin?ii lui. Dar faraonul Neco l-a luat legat la Ribla, în ?inutul Hamatului, ca sa nu mai domneasca în Ierusalim.
\par 3 Dupa ce l-a dat jos de pe tronul din Ierusalim, regele Egiptului l-a dus în Egipt ?i a pus pe ?ara o dajdie de o suta talan?i de argint ?i un talant de aur.
\par 4 Iar peste Iuda ?i Ierusalim, regele Egiptului a pus rege pe Eliachim, fratele lui Ioahaz, caruia i-a schimbat numele în Ioiachim; iar pe Ioahaz, fratele lui, l-a luat Neco ?i l-a dus în Egipt ?i a murit acolo. Ioiachim i-a dat lui Faraon argintul ?i aurul cerut. De atunci a început ?ara sa plateasca bir dupa cuvântul lui Faraon ?i fiecare, dupa puterea ce avea, cerea argint ?i aur de la poporul ?arii pentru bir, care era trimis Faraonului Neco.
\par 5 Ioiachim era de douazeci ?i cinci de ani când s-a facut rege ?i a domnit unsprezece ani în Ierusalim. Pe mama lui o chema Zebuda ?i era fiica lui Pedaia din Ruma. Acesta a facut lucruri neplacute înaintea ochilor Domnului Dumnezeului sau, cum facusera ?i parin?ii lui. În zilele lui a venit Nabucodonosor, regele Babilonului, asupra ?arii ?i Ioiachim i-a fost supus trei ani ?i apoi s-a lepadat de el. Atunci a trimis Domnul împotriva lor pe Caldei, pe tâlharii Siriei, pe tâlharii Moabi?ilor, pe fiii lui Amon ?i pe cei ai Samariei ?i s-au retras pentru acest cuvânt, pentru cuvântul Domnului, pe care l-a grait prin gura robilor Sai prooroci. Însa mânia Domnului tot a mai dainuit asupra lui Iuda, ca sa-l lepede de la fa?a Sa, pentru toate pacatele lui Manase, pe care le facuse acesta ?i pentru sângele cel nevinovat pe care l-a varsat Ioiachim, umplând Ierusalimul cu sânge nevinovat. însa Domnul tot n-a vrut sa-l stârpeasca.
\par 6 Împotriva lui s-a ridicat Nabucodonosor, regele Babilonului, ?i l-a legat în catu?e de fier, ca sa-l duca la Babilon.
\par 7 Nabucodonosor a dus la Babilon ?i o parte din vasele templului Domnului ?i le-a pus în capi?tea sa în Babilon.
\par 8 Celelalte fapte ale lui Ioiachim ?i urâciunile lui, pe care le-a facut el ?i care s-au mai gasit asupra lui, sunt scrise în cartea regilor lui Israel ?i ai lui Iuda. ?i a raposat Ioiachim cu parin?ii lui ?i a fost îngropat în Ganozai la un loc cu parin?ii sai ?i în locul lui s-a facut rege Iehonia, fiul sau.
\par 9 Iehonia era de optsprezece ani când s-a facut rege ?i a domnit trei luni ?i zece zile în Ierusalim. Acesta a facut lucruri neplacute înaintea ochilor Domnului.
\par 10 Dupa trecerea unui an, a trimis regele Nabucodonosor ?i a poruncit sa-l aduca la Babilon împreuna cu vasele cele pre?ioase din templul Domnului ?i peste Iuda ?i Ierusalim a pus rege pe Sedechia, fratele sau.
\par 11 Sedechia era de douazeci ?i unu de ani când s-a facut rege ?i a domnit unsprezece ani în Ierusalim;
\par 12 ?i el a facut lucruri neplacute înaintea ochilor Domnului Dumnezeului sau. El nu s-a smerit înaintea lui Ieremia proorocul, care îi proorocea cuvintele din gura Domnului.
\par 13 El s-a razvratit împotriva regelui Nabucodonosor, care-l pusese sa jure pe numele Domnului ?i s-a facut tare de cerbice ?i ?i-a învârto?at inima sa pâna într-atâta, ca nu s-a mai întors la Domnul Dumnezeul lui Israel.
\par 14 La fel au pacatuit mult ?i toate capeteniile preo?ilor ?i ale poporului, facând toate urâciunile pagânilor ?i au spurcat templul Domnului pe care îl sfin?ise el în Ierusalim.
\par 15 Atunci a trimis la ei Domnul Dumnezeul parin?ilor lor pe trimi?ii sai foarte de diminea?a, pentru ca i-a fost mila de popor ?i de loca?ul Sau.
\par 16 Dar ei ?i-au batut joc de trimi?ii cei de la Dumnezeu ?i n-au ?inut seama de cuvintele Lui; au batjocorit pe proorocii Lui, pâna ce mânia Domnului s-a coborât peste poporul Lui, încât acesta n-a mai avut scapare.
\par 17 Caci El a adus asupra lor pe regele Caldeilor ?i acela a omorât pe tinerii lor cu sabia în loca?ul cel sfânt al lor ?i n-a cru?at nici pe Sedechia, nici pe baie?i, nici pe fete, nici pe batrâni, nici pe cei încarun?i?i; pe to?i Dumnezeu i-a dat în mâna lui.
\par 18 Toate vasele din templul lui Dumnezeu, cele mari ?i cele mici, vistieriile templului ?i vistieriile regelui ?i ale capeteniilor lui, toate le-a adus el în Babilon.
\par 19 Apoi a dat foc templului lui Dumnezeu, a darâmat zidul Ierusalimului, toate camarile lui le-a ars cu foc ?i toate palatele cele mari le-a nimicit.
\par 20 Iar pe cei care au scapat de sabie i-a stramutat în Babilon; ?i au fost ei ca robi ai lui ?i ai fiilor lui, pâna în timpul domniei regelui Persiei,
\par 21 Pentru ca sa se împlineasca cuvântul Domnului cel zis prin gura lui Ieremia, pâna ce ?ara va termina de ?inut zilele sale de odihna; caci în toate zilele pustiirii ea s-a odihnit pâna la împlinirea celor ?aptezeci de ani.
\par 22 Iar în anul dintâi al lui Cirus, regele Persiei, pentru împlinirea cuvintelor Domnului rostite prin Ieremia, a trezit Domnul duhul lui Cirus, regele Persiei, ?i a poruncit acesta sa se faca cunoscut tuturor din tot regatul sau, prin cuvânt ?i prin scris ?i sa le spuna:
\par 23 "A?a zice Cirus, regele Per?ilor: Toate regatele pamântului, Domnul Dumnezeul cerului mi le-a dat mie ?i mi-a poruncit sa-I zidesc templul în Ierusalimul cel din Iuda. Cine este între voi din tot poporul Lui? Domnul Dumnezeul lui sa fie cu el ?i sa se duca acolo".


\end{document}