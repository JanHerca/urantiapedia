\begin{document}

\title{1 Regi}


\chapter{1}

\par 1 Când regele David a ajuns la adânci bătrânețe, îl acopereau cu haine, însă nu putea să se încălzească.
\par 2 Atunci slugile lui i-au zis: "Să se caute pentru domnul nostru rege o fată tânără, care să stea înaintea regelui, să-l îngrijească și să se culce cu el, ca să se încălzească domnul nostru regele".
\par 3 Și i-au căutat în toate ținuturile lui Israel o fată frumoasă și au găsit pe Abișag Sunamiteanca și au adus-o la rege.
\par 4 Fata era foarte frumoasă, îngrijea pe rege și-l servea; însă ragele n-a cunoscut-o.
\par 5 Atunci Adonia, fiul Haghitei, s-a fălit și a zis: "Eu am să fiu rege". Și Și-a gătit care de război, călăreți și cincizeci de oameni care să-i alerge înainte.
\par 6 Căci tatăl său niciodată nu-l oprise și nu-i zisese: "Pentru ce faci aceasta?" El însă mai era și foarte frumos la chip și născut după Abesalom.
\par 7 Și s-a înțeles cu Ioab, fiul Țeruiei, și cu Abiatar preotul; și aceștia ajutau pe Adonia.
\par 8 Iar preotul Țadoc și Benaia, fiul lui Iehoiada, proorocul Natan, Șimei, Rei și vitejii lui David n-au fost de partea lui Adonia.
\par 9 Și junghiind Adonia ai, boi și viței grași la piatra Zohelet, cea de la En-Roghel, a chemat pe tați frații săi, fiii regelui, dimpreună cu toți bărbații lui Iuda, care erau în slujbă la rege.
\par 10 Iar pe proorocul Natan, pe Benaia, pe viteji și pe Solomon, fratele lui, nu i-a chemat.
\par 11 Atunci Natan a grăit către Batșeba, mama lui Solomon, zicând: "Auzit-ai tu că Adonia, fiul Haghitei, s-a făcut rege, iar domnul nostru David nu știe de aceasta?
\par 12 Acum, iată, iți dau un sfat: Să-ți scapi viața ta și a fiului tău Solomon.
\par 13 Du-te și intră la regele David șl-i spune: "Nu tu, oare, domnul meu rage, te-ai jurat către roaba ta, zicând: Solomon, fiul tău, va fi rege după mine: și va ședea pe tronul meu? Pentru ce dar Adonia s-a făcut rege?"
\par 14 Și iată, când tu încă vei vorbi acolo cu regele, voi intra și eu în urma ta, voi adeveri și voi întregi cuvintele tale",
\par 15 Deci a intrat Batșeba la rege în odaia unde odihnea. Și iată regele era foarte bătrân și Abișag Sunamiteanca îngrijea de el.
\par 16 Și s-a plecat Batșeba și s-a închinat regelui și el a întrebat-o: "Ce voiești?"
\par 17 Iar ea i-a spus: "Domnul meu rege, tu te-ai jurat pe Domnul Dumnezeul tău câtre roaba ta, zicând: Solomon, fiul tău, va fi rege după mine și va ședea pe tronul meu.
\par 18 Dar acum iată că Adonia s-a făcut rege și tu, domnul meu rege, nu știi nimic da aceasta.
\par 19 Acela a înjunghiat o mulțime de boi, viței grași și oi; și a poftit pa toți fiii regelui, pe preotul Abiatar și pe Ioab, mai-marele oștirii; iar pe Solomon, robul tău, nu l-a chemat.
\par 20 Însă tu ești rege, domnul meu. Ochii tuturor Israeliților la tine privesc, ca să le arăți cine va ședea rege pe tronul domnului meu, după el.
\par 21 Altfel, când domnul meu, regele, va răposa cu părinții săi, voi fi învinuită eu și fiul meu Solomon".
\par 22 Și iată, când încă vorbea cu regele, a venit și proorocul Natan.
\par 23 Și i s-a spus regelui, zicând: "Iată Natan proorocul!" Și a intrat Natan la rege și i s-a închinat regelui cu fața până la pământ.
\par 24 Și a zis Natan: "Domnul meu rege, ai spus tu oare: Adonia va fi rege și va ședea pe tronul meu după mine?
\par 25 Pentru că el chiar astăzi a plecat Și a junghiat o mulțime de boi, viței grași și oi; și a poftit pe toți fiii regelui, pe mai-marii oștirii și pe Abiatar preotul; și, iată, ei mănâncă și beau dinaintea lui și zic: "Trăiască regele Adonia!"
\par 26 Iar pe mine, robul tău, pe preotul  Țadoc, pe Benaia, fiul lui Iehoiada, și pe Solomon, robul tău, nu ne-a chemat.
\par 27 Și dacă cu voia ta, domnul meu rege, s-a făcut lucrul acesta, atunci pentru ce tu n-ai descoperit robului tău cine va ședea rege pe tronul domnului meu, după el?"
\par 28 Dar regele David a răspuns și a zis: "Chemați pe Batșeba la mine!" și a intrat ea la rege și a stat înaintea lui.
\par 29 Atunci regele s-a jurat și a zis: "Viu este Domnul, Care mi-a scăpat sufletul meu de la orice necaz!
\par 30 Precum m-am jurat pe Domnul Dumnezeul lui Israel către tine, zicând: "Solomon, fiul tău, va fi rege după mine, și va ședea pe tronul meu în locul meu, așa voi face chiar astăzi!"
\par 31 Și s-a plecat Batșeba cu fața până la pământ și i s-a închinat regelui și i-a zis: "Să trăiască domnul meu, regele David, În veci!"
\par 32 Apoi a zis regele David: "Chemați la mine pe preotul Țadoc, pe proorocul Natan și pe Benaia, fiul lui Iehoiada!" Și ei au intrat la rege.
\par 33 Și regele le-a spus: "Să luați pe slugile domnului vostru cu voi și să puneți pe Solomon, fiul meu, călare pe catârul meu și să-l duceți până la Ghihon,
\par 34 Și acolo să-l ungă preotul Țadoc și proorocul Natan de rege peste Israel și să sunați din trâmbiță și să ziceți: "Să trăiască regele Solomon!"
\par 35 Apoi să-l petreceți înapoi, ca să vină și să șadă pe tronul meu; căci el va fi rege în locul meu și lui i-am poruncit să fie conducătorul lui Israel și al lui Iuda".
\par 36 Și a răspuns Benaia, fiul lui Iehoiada regelui și i-a zis: "Amin! Așa să întărească Domnul Dumnezeul domnului meu, regele, cuvântul acesta!
\par 37 Și cum Domnul Dumnezeu a fost cu domnul meu, regele, așa să fie el și cu Solomon și să-i preamărească tronul lui mai mult decât tronul domnului meu, regele David!"
\par 38 Și așa au plecat preotul Țadoc, proorocul Natan, Benaia, fiul lui Iehoiada, Cheretienii și Peletienii și au pus pe Solomon călare pe catârul regelui David și l-au dus până la Ghihon.
\par 39 Și a luat preotul Țadoc cornul cu untdelemn sfințit din cort și a uns pe Solomon. Și s-a sunat din trâmbițe și tot poporul a strigat: "Trăiască regele Solomon! "
\par 40 Apoi tot poporul l-a petrecut pe Solomon și i-a cântat din fluiere și bucurie mare era pe popor, încât și pământul se zguduia de strigătele lui.
\par 41 Atunci a auzit Adonia de aceasta și toți cei chemați ai lui, tocmai când isprăviseră de mâncat; iar Ioab, auzind sunetul de trâmbițe, a zis: "Ce este acest zgomot de care răsună cetatea?"
\par 42 Și pe când el încă vorbea, iată că a venit Ionatan, fiul preotului Abiatar. Și a zis Adonia: "Intră, căci tu ești om cinstit și aduci veste bună!"
\par 43 Și a răspuns Ionatan lui Adonia și i-a zis: "Ba chiar veste rea; căci domnul nostru, regele David, a făcut rege pe Solomon.
\par 44 Și a trimis regele cu el pe Țadoc și pe proorocul Natan, pe Benaia, fiul lui Iehoiada, și pe Cheretieni și Peletieni și l-au pus călare pe catârul regelui.
\par 45 Și l-au uns preotul Țadoc și proorocul Natan rege la Ghihon și de acolo s-au întors cu bucurie și au pus în mișcare cetatea. Iată zgomotul pe care-l auziți".
\par 46 Și a șezut Solomon pe tronul regatului.
\par 47 Și slugile regelui au venit să-l binecuvânteze pe domnul nostru regele David, zicând: "Să mărească Dumnezeul tău numele lui Solomon mai mult decât numele tău și să-i aducă tronul lui la mărire mai multă decât tronul tău!" Și s-a închinat regele în patul lui;
\par 48 Și a zis regele așa: "Binecuvântat să fie Domnul Dumnezeul lui Israel, Care a făcut să fie astăzi din sămânța mea un urmaș pe tronul meu și să văd cu ochii mei aceasta!"
\par 49 Atunci toți cei chemați, care erau cu Adonia, s-au spăimântat și, sculându-se, s-au dus fiecare în drumul său.
\par 50 Iar Adonia, temându-se de Solomon, s-a sculat și s-a dus și s-a apucat cu mâinile de coarnele jertfelnicului.
\par 51 Și i s-a spus lui Solomon, zicând: "Iată, Adonia s-a temut de regele Solomon; căci iată, că se ține cu mâinile de coarnele jertfelnicului, zicând: Să-mi făgăduiască astăzi cu jurământ regele Solomon că nu va omorî pe robul său cu sabia".
\par 52 Și a zis Solomon: "Dacă el va fi om cinstit, nici un păr din capul lui nu va cădea pe pământ; iar dacă va fi om viclean, va muri".
\par 53 Și a trimis regele Solomon și l-a adus de la jertfelnic cu de-a sila; și a venit și s-a închinat regelui Solomon. Și Solomon i-a zis: "Du-te la casa ta!"

\chapter{2}

\par 1 Apropiindu-se vremea lui David ca să moară, a lăsat el fiului său Solomon acest legământ:
\par 2 "Iată, eu mă duc pe drumul pe care toți pământenii se duc; fii tare și să fii bărbat.
\par 3 Să păzești legământul Domnului Dumnezeului tău, umblând în căile Lui și păzind legile Lui, poruncile Lui, hotărârile Lui și așezămintele Lui, precum sunt scrise în legea lui Moise, pentru ca să-ți fie bine-cunoscut tot ce vei face, oriunde și ori încotro te vei întoarce;
\par 4 Și ca să Își țină și Domnul cuvântul Său care l-a grăit către mine, zicând: "Dacă fiii tăi își vor păzi drumul lor, ca să se poarte cu credincioșie înaintea Mea, din toată inima și din tot sufletul lor, atunci nu va conteni să fie din tine bărbat pe tronul lui Israel.
\par 5 Dar și tu știi ce mi-a făcut Ioab, fiul Țeruiei, cum s-a purtat cu cele două căpetenii ale oștirii lui Israel, cu Abner, fiul lui Ner, și cu Amasa, fiul lui Ieter, pe care i-a omorât; și cum a vărsat în timp de pace sânge ca în timp de război, mânjind cu sângele cel vărsat ca la război încingătoarea de la coapsele sale și încălțămintea din picioarele sale.
\par 6 Să faci dar cu el după înțelepciunea ta, ca să nu se coboare căruntețea lui cu pace în locuința morților.
\par 7 Iar fiilor lui Barzilai Galaaditul arată-le milă, ca să fie cu cei care se hrănesc la masa ta, căci ei au venit la mine când am fugit de Abesalom, fratele tău.
\par 8 Iată că ai la tine și pe Șimei, fiul lui Ghera veniamineanul din Bahurim, care m-a blestemat cu greu blestem, când mergeam la Mahanaim; dar fiindcă mi-a ieșit în cale la Iordan, m-am jurat pe Domnul către el, zicând: Nu te voi mai ucide cu sabia.
\par 9 Acum însă să nu-l lași nepedepsit, căci ești bărbat înțelept, și știi ce să faci cu el, ca să cobori căruntețea lui cu sânge în locuința morților".
\par 10 Și a răposat David cu părinții săi și a fost înmormântat în cetatea lui David.
\par 11 Timpul domniei lui David peste Israel a fost de patruzeci de ani: în Hebron șapte ani și în Ierusalim treizeci și trei de ani.
\par 12 Și s-a așezat Solomon pe tronul lui David, tatăl său, și domnia lui s fost foarte strălucită.
\par 13 Atunci a venit Adonia, fiul Haghitei, la Batșeba, mama lui Solomon, și i s-a închinat. Și ea i-a zis: "Cu pace îți este venirea?" Iar el a răspuns: "Cu pace!"
\par 14 Apoi a zis: "Am să-ți spun o vorbă!" Și ea a zis: "Spune;!"
\par 15 Și a zis el: "Tu știi că domnia era a mea și că tot Israelul privea la mine ca la viitorul lor rege; dar domnia a trecut de la mine și i s-a dat fratelui meu, pentru că de la Domnul i-a fost lui aceasta.
\par 16 Acum te rog un lucru: să nu mă nesocotești!" Și ea i-a zis: "Grăiește!"
\par 17 Iar el a zis: "Te rog vorbește cu regele Solomon, căci el va ține seamă de vorba ta, ca să-mi dea pe Abișag Sunamiteanca de femeie".
\par 18 Și a zis Batșeba: "Bine, voi vorbi cu regele pentru tine".
\par 19 Deci, a intrat Batșeba la regele Solomon, ca să-i vorbească pentru Adonia. Regele s-a sculat înaintea ei, i s-a închinat și s-a așezat pe tronul lui. Și s-a pus și pentru mama regelui un tron; și ea a stat de-a dreapta lui
\par 20 Și a zis: "Am să-ți fac o mică rugăminte, să nu mi-o treci cu vederea!" A zis regele: "Cere, mama mea!"
\par 21 Ea a zis: "Să dai pe Abișag Sunamiteanca lui Adonia, fratele tău, de femeie!"
\par 22 Atunci a răspuns regele Solomon și a zis mamei sale: "Dar de ce ceri tu pe Abișag Sunamiteanca pentru Adonia? Cere atunci pentru el și domnia, căci el este fratele meu cel mai mare și cu el este preotul Abiatar și tot cu el este prieten și Ioab, fiul Țeruiei, mai-marele oștirii!"
\par 23 Apoi s-a jurat regele Solomon pe Domnul, zicând: "Așa rău să-mi dea mie Dumnezeu și încă și altele să mă ajungă, dacă n-a grăit Adonia cuvântul acesta împotriva vieții mele.
\par 24 Dar acum, viu este Domnul, Care m-a socotit pe mine vrednic de cinste și m-a așezat pe tronul lui David, tatăl meu, și mi-a zidit casă precum a grăit el; Adonia chiar astăzi va muri".
\par 25 Și a trimis regele Solomon pe Benaia, fiul lui Iehoiada, care fără de milă l-a lovit pe acela și a murit Adonia în ziua aceea.
\par 26 Iar lui Abiatar preotul, regele i-a zis: "Să pleci la Anatot, la moșia ta; căci și tu ești vrednic de moarte; dar acum nu te voi omorî, căci ai purtat chivotul Domnului Dumnezeu înaintea lui David, tatăl meu și ai suferit și tu ce a suferit tatăl meu".
\par 27 Și l-a îndepărtat Solomon pe Abiatar de la preoția Domnului, ca să se împlinească cuvântul Domnului care l-a grăit pentru casa lui Eli în Șilo.
\par 28 Vestea aceasta a ajuns până la Ioab, fiul Țeruiei, fiindcă și Ioab se dăduse de partea lui Adonia și nu de partea lui Solomon și a fugit Ioab la cortul Domnului și s-a apucat cu mâinile de coarnele jertfelnicului.
\par 29 Și i s-a spus regelui Solomon, zicând: "Iată Ioab a fugit la cortul Domnului și iată s-a apucat cu mâinile de coarnele jertfelnicului. Și a trimis Solomon la Ioab să-i zică: "Ce ți-am făcut de ai fugit la jertfelnic?" Și a răspuns Ioab: "M-am temut de fața ta și am fugit la Domnul". Și a trimis Solomon pe Benaia, fiul lui Iehoiada, zicând: "Mergi, omoară-l și-l îngroapă!"
\par 30 Și s-a dus Benaia, fiul lui Iehoiada, la Ioab, la cortul Domnului, și i-a zis: "Așa zice regele: Să ieși!" Și a zis Ioab: "Nu ies, pentru că vreau să mor aici! " și s-a întors Benaia, fiul lui Iehoiada și a spus de aceasta regelui, zicând: "Așa a zis Ioab și așa mi-a răspuns!"
\par 31 Iar regele i-a zis: "Du-te și fă-i așa cum a zis; omoară-l și-l îngroapă. Și ia de pe mine și de pe casa tatălui meu sângele nevinovat vărsat de Ioab.
\par 32 Să-i întoarcă Domnul pe capul lui sângele nedreptății lui, pentru că a ucis pe doi bărbați nevinovați și mai buni decât el, omorându-i cu sabia, fără de știrea tatălui meu, David: pe Abner, fiul lui Ner, mai-marele oștirii lui Israel, și pe Amasa, fiul lui Ieter, mai marele oștirii lui Iuda.
\par 33 Pe capul lui și pe capul seminției lui în veci să se întoarcă sângele lor; iar David și seminția lui și casa lui și tronul lui să aibă în veci pace de la Domnul!
\par 34 Și s-a dus Benaia, fiul lui Iehoiada și a lovit pe Ioab și l-a omorât; și a fost înmormântat la casa lui, în pustiu.
\par 35 Iar regele Solomon a pus în locul lui peste oștire pe Benaia, fiul lui Iehoiada. Cârmuirea regatului era însă la Ierusalim; iar mai mare peste preoți, în locul lui Abiatar, regele a pus pe Țadoc preotul. Solomon, fiul lui David, a domnit peste Israel și Iuda, în Ierusalim. Și a dat Domnul lui Solomon înțelepciune și pricepere foarte mare și cunoștințe multe, ca nisipul de pe țărmul mării. Și Solomon a avut înțelepciune mai presus de înțelepciunea tuturor fiilor Răsăritului și de înțelepciunea tuturor înțelepților Egiptenilor. El a luat pentru sine pe fiica lui Faraon și a adus-o în cetatea lui David, până ce a terminat de zidit casa sa și, mai întâi, templul Domnului și zidul dimprejurul Ierusalimului; în șapte ani a făcut acestea și le-a isprăvit. Solomon a avut șaptezeci de mii de oameni salahori și optzeci de mii de tăietori de piatră în munte. El a făcut marea și postamentele și spălătoriile cele mari, stâlpii, fântâna cea din curte și marea de aramă; și a zidit el cetățuia și întăriturile ei și a împărțit cetatea lui David. Atunci fiica lui Faraon o trecut din cetatea lui David în casa sa pe care i-o zidise el; tot atunci a zidit Solomon zidul dimprejurul cetății. Și aducea Solomon de trei ori pe an arderi de tot și jertfe de pace pe jertfelnicul pe care-l făcuse Domnului; și săvârșea la el și tămâieri înaintea Domnului. Și a isprăvit zidirea templului. La lucrările lui Solomon erau trei mii șapte sute de ispravnici mari care conduceau poporul ce făcea lucrul. Și a zidit el Asurul, Magdinul, Gazerul, Bet-Horonul de Sus și Valatul. Dar aceste cetăți le-a zidit el după ce a făcut templul Domnului și zidul dimprejurul Ierusalimului. Încă din timpul vieții sale, David poruncise lui Solomon, zicând: "Iată, ai pe Șimei, fiul lui Ghera, fiul lui Ieminie, din Bahurim; el m-a blestemat cu greu blestem, în ziua când m-am dus la Mahanaim; dar el mi-a ieșit în cale la Iordan și eu m-am jurat pe Domnul față de el, zicând: Nu te voi mai ucide eu sabia! Tu însă să nu-l lași nepedepsit; căci ești bărbat înțelept și știi ce să faci cu el, ca să-i cobori căruntețea lui cu sânge în casa morților".
\par 36 Și trimițând regele, a chemat pe Șimei și i-a zis: "Fă-ți casă în Ierusalim și trăiește aici și de aici să nu mai ieși.
\par 37 Dar să știi că în ziua în care vai ieși și vei trece pârâul Chedron, numaidecât cu moarte vei muri. Sângele tău va fi asupra capului tău!"
\par 38 Și a zis Șimei regelui: "Bine, cum a poruncit domnul meu regele, așa va face robul tău". Și a trăit Șimei în Ierusalim multă vreme.
\par 39 Iar peste trei ani s-a întâmplat că au fugit doi robi de la Șimei la Achiș, fiul lui Maaca, regele Gatului. Și i s-a spus lui Șimei: "Iată, robii tăi sunt la Gat".
\par 40 Și sculându-se Șimei și punând șaua pe asinul său, a plecat la Gat, la Achiș, ca să-și caute robii săi. Și s-a întors Șimei și a adus robii săi de la Gat.
\par 41 Dar, spunându-i-se lui Solomon că Șimei a mers din Ierusalim până la Gat și că s-a întors, a trimis regele și a chemat pe Șimei și i-a zis:
\par 42 "Nu m-am jurat eu, oare, pe Domnul către tine și nu ți-am spus eu, oare, înainte, zicând: Să știi că în ziua în care vei ieși din Ierusalim și te vei duce undeva, numaidecât vai muri? Și tu mi-ai spus: Bine!
\par 43 De ce n-ai păzit legea pe care ți-am dat-o înaintea Domnului cu jurământ?
\par 44 Apoi a mai zis regele către Șimei: "Tu știi și știe și inima ta tot râul care l-ai făcut tatălui meu David; să se întoarcă asupra capului tău răutatea ta!
\par 45 Iar regele Solomon să fie binecuvântat și tronul lui David să fie neclintit înaintea Domnului în veci!"
\par 46 Și a poruncit regele lui Benaia, fiul lui Iehoiada; și el s-a dus și a lovit pe Șimei și acela a murit.

\chapter{3}

\par 1 După ce regatul s-a întărit în mâinile lui Solomon, Solomon s-a înrudit cu Faraon, regele Egiptului, căci a luat pentru el pe fiica lui Faraon și a adus-o în cetatea lui David, până o terminat de zidit casa sa, templul Domnului și zidul dimprejurul Ierusalimului.
\par 2 Poporul tot mai aducea jertfe pe înălțimi, căci nu era încă zidit templul numelui Domnului până în acel timp.
\par 3 Și Solomon, care iubea pe Domnul, purtându-se după legea tatălui său, a adus și el jertfe și tămâieri pe înălțimi.
\par 4 Deci s-a sculat și el și s-a dus la Ghibeon, ca să aducă jertfe acolo, căci acolo era cea mai mare înălțime. O mie de jertfe pentru ardere de tot a adus Solomon pe acel jertfelnic.
\par 5 La Ghibeon însă S-a arătat Domnul lui Solomon noaptea în vis și a zis: "Cere ce vrei să-ți dau!"
\par 6 Și a zis Solomon: "Tu ai făcut cu robul Tău David, tatăl meu, mare milă; și, pentru că el s-a purtat cu vrednicie și cu dreptate și inimă curată înaintea Ta, nu ți-ai luat această milă mare de la el; și i-ai dăruit fiu pe tronul lui, precum și este aceasta astăzi.
\par 7 Și acum Tu, Doamne Dumnezeul meu, ai pus pe robul Tău rege în locul lui David, tatăl meu; însă eu sunt foarte tânăr și nu știu să conduc.
\par 8 Și robul Tău este în mijlocul poporului Tău pe care l-ai ales, popor nesfârșit de mare, care din pricina mulțimii lui nu se poate nici socoti, nici număra.
\par 9 Dăruiește-i dar robului Tău minte pricepută, ca să asculte și să judece poporul Tău și să deosebească ce este bine și ca este rău; căci cine poate să povățuiască pe acest popor al Tău, care este nesfârșit de mare?"
\par 10 Și i-a plăcut Domnului că Solomon a cerut aceasta.
\par 11 Și a zis Dumnezeu: "Deoarece tu ai cerut aceasta și n-ai cerut viață lungă; n-ai cerut bogăție, n-ai cerut sufletele dușmanilor tăi, ci ai cerut înțelepciune, ca să știi să judeci,
\par 12 Iată Eu voi face după cuvântul tău; iată, Eu îți dau minte înțeleaptă și pricepută, cum nici unul n-a fost ca tine înaintea ta și cum nici nu se va mai ridica după tine.
\par 13 Ba îți voi da și ceea ce tu n-ai cerut: bogăție și slavă, așa încât nici unul dintre regi nu va fi asemenea ție, în toate zilele tale.
\par 14 Și dacă vei umbla pe drumul Meu, ca să păzești legile Mele și poruncile Mele, cum a umblat tatăl tău David, îți voi înmulți și zilele tale".
\par 15 Și s-a trezit Solomon din somn Și iată, acesta fusese vis. Apoi s-a sculat și a venit la Ierusalim și a stat înaintea jertfelnicului celui de dinaintea chivotului cu legea Domnului, care era în Sion, a adus arderi de tot, a săvârșit jertfe de împăcare și a făcut mare ospăț pentru toate slugile sale.
\par 16 Atunci au venit două femei desfrânate la rege și au stat înaintea lui.
\par 17 Și a zis una din femei: "Rogu-mă, domnul meu, noi trăim într-o casă; și eu am născut la ea, în casa aceea.
\par 18 A treia zi după ce am născut eu, a născut și această femeie și eram împreună și nu era nimeni străin cu noi în casă, afară de noi amândouă.
\par 19 Însă noaptea a muțit fiul acestei femei, căci a adormit peste el.
\par 20 Și s-a sculat ea pe la miezul nopții și mi-a luat pe fiul meu de lângă mine, când eu, roaba ta, dormeam și l-a pus la pieptul ci; iar pe fiul ei cel mort l-a pus la pieptul meu.
\par 21 Dimineața când m-am sculat ca să-mi alăptez fiul, iată, el era mort; iar când m-am uitat la el mai bine dimineața, acesta nu era fiut meu, pe care-l născusem".
\par 22 Iar cealaltă femeie a zis: "Ba nu, fiul meu e viu, iar fiul tău e mort!" Iar aceasta îi zicea: "Ba nu, fiul tău este mort și al meu e viu!" și vorbeau ele așa înaintea regelui.
\par 23 Atunci regele a zis: "Aceasta zice: Fiul meu este cel viu, iar fiul tău este cel mort; iar aceea zice; Ba nu, fiul tău este cel mort și fiul meu aste cel viu".
\par 24 Apoi a zis Solomon: "Dați-mi o sabie"; și i s-a adus regelui o sabie.
\par 25 Și a zis regele: "Tăiați copilul cel viu în două și dați o jumătate din el uneia și o jumătate din el celeilalte!"
\par 26 Și a răspuns femeia al cărui fiu era viu regelui, - căci i se rupea inima de milă pentru fiul ei: "Rogu-mă, domnul meu, dați-i ei acest prunc viu și nu-l omorâți!" Iar cealaltă a zis: "Ca să nu fie nici al meu, nici al ei, tăiați-l!"
\par 27 Și regele a zis: "Dați-i acesteia copilul cel viu, că aceasta este mama lui!"
\par 28 Și a auzit tot Israelul do judecata aceasta pe care a făcut-o regele. Și au început să se teamă de rege, căci vedeau că înțelepciunea lui Dumnezeu este în el, ca să facă judecată și dreptate.

\chapter{4}

\par 1 Și a fost regele Solomon rege peste tot Israelul.
\par 2 Iată acum căpeteniile pe care le avea el la curtea sa: Azaria, fiul lui Țadoc, preotul;
\par 3 Elihoref și Ahia, fiii lui Șișa, scriitori; Ioasaf, fiul lui Ahilud, cronicar;
\par 4 Benaia, fiul lui Iehoiada, căpetenie peste oștire; Țadoc și Abiatar, preoți;
\par 5 Azaria, fiul lui Natan, căpetenie peste ispravnici, iar Zabud, fiul lui Natan, preot și prieten al regelui;
\par 6 Ahișar era căpetenie peste casa regelui; Eliav, fiul lui Saf, era peste moșii și Adoniram, fiul lui Abda, era peste dări.
\par 7 Și mai avea Solomon doisprezece ispravnici peste tot Israelul, care aduceau alimente pentru rege și casa lui; fiecare trebuia să aducă alimente pe o lună în an.
\par 8 Iată numele lor: Ben-Hur, peste muntele lui Efraim, singur;
\par 9 Ben-Decher, peste Macaț, peste Șaalebim, peste Bet-Șemeș, peste Elon și peste Bet-Hanan;
\par 10 Ben-Hased, peste Arubot; și tot sub el mai era și Soco, cum și tot pământul Hefer;
\par 11 Ben-Abinadab, peste tot Nafat-Dor; Tafat, fiica lui Solomon, era femeia lui;
\par 12 Baana, fiul lui Ahilud, peste Tanac, peste Meghidon și peste tot pământul Bet-Șeanului, care este aproape de Țartan, mai jos de Izreel, de la Bet-Șean până la Abel-Mehol și chiar până dincolo de Iocmeam;
\par 13 Ben-Gheber, peste Ramot-Galaad; sub el mai erau și satele lui Iair, fiul lui Manase, care sunt în Galaad; tot sub el mai era și ținutul Argob, care este în Vasan, șaizeci de cetăți mari cu ziduri și zăvoare de aramă;
\par 14 Ahinadab, fiul lui Ido, peste Mahanaim;
\par 15 Ahimaaț, care a avut de femeie pe Basemat, fiica lui Solomon, era peste pământul Neftali;
\par 16 Baana, fiul lui Hușai, peste Așer și Bealot;
\par 17 Iosafat, fiul lui Paruah, peste Isahar;
\par 18 Șimei, fiul lui Ela, peste Veniamin;
\par 19 Gheber, fiul lui Urie, peste Galaad, peste țara lui Sihon, regele Amoreilor, și a lui Og, regele Vasanului. El era singur ispravnic peste aceste pământuri.
\par 20 Iuda și Israel, care erau nesfârșit de mulți la număr, ca nisipul de pe țărmul mării, mâncau, beau și se veseleau.
\par 21 Solomon domnea peste toate regatele de la râul Eufrat până la pământul Filistenilor și până în hotarul Egiptului. Acestea îi aduceau daruri și au slujit lui Solomon în toate zilele vieții lui.
\par 22 Hrana lui Solomon pe fiecare zi era: treizeci de core făină de grâu și șaizeci core de alte preparate de făină;
\par 23 Zece boi îngrășați, douăzeci boi din cei care pășteau iarbă și o sută de oi, afară de vânatul de cerbi, căprioare, ciute și de păsările îngrășate;
\par 24 Căci domnea peste tot pământul de dincoace de Eufrat, de la Tifsah până la Gaza, și peste toți regii de dincoace de Eufrat, și era în pace cu toate țările de primprejur.
\par 25 Astfel a trăit Iuda și Israelul în liniște, fiecare sub vița sa de vie și sub smochinul său, de la Dan până la Beer-Șeba, în toate zilele lui Solomon.
\par 26 Solomon avea patruzeci de mii de iesle pentru caii de la carele lui și douăsprezece mii de călăreți.
\par 27 Și acei ispravnici aduceau regelui Solomon tot ce trebuia pentru masa regelui, fiecare în luna lui, și nu lăsa să ducă lipsă de nimic.
\par 28 Și orz și paie pentru cai și pentru celelalte vite aducea fiecare, când îi era rândul lui, la locul unde se afla regele.
\par 29 Și a dat Dumnezeu lui Solomon înțelepciune și pricepere foarte mare și cunoștințe multe, ca nisipul de pe țărmul mării.
\par 30 Și era înțelepciunea lui Solomon mai presus de înțelepciunea tuturor fiilor Răsăritului și mai presus de toată înțelepciunea Egiptenilor.
\par 31 El era mai înțelept decât toți oamenii; mai înțelept mult și decât Etan Ezrahiteanul, decât Heman și decât Calcol și Darda, feciorii lui Mahol; și numele lui era în slavă la toate popoarele de primprejur.
\par 32 Solomon a spus trei mii de pilde; și cântările lui au fost o mie și cinci.
\par 33 El a vorbit despre copaci, de la cedrii cei din Liban până la isopul de pe ziduri; a vorbit și despre animale, despre păsări, despre târâtoare și despre pești.
\par 34 Și veneau de la toate popoarele, ca să asculte înțelepciunea lui Solomon, și de la toți regii pământului care auzeau de înțelepciunea lui.

\chapter{5}

\par 1 Atunci a trimis Hiram, regele Tirului, pe slugile sale la Solomon, când a auzit că l-au uns rege în locul tatălui său. Căci Hiram fusese prieten cu David toată viața.
\par 2 ți a trimis și Solomon la Hiram ca să-i spună:
\par 3 "Tu știi că David, tatăl meu, n-a putut să înalțe casă numelui Domnului Dumnezeului său, din pricina războaielor cu popoarele dimprejur, până ce Domnul nu le-a supus sub talpa picioarelor lui.
\par 4 Acum însă Domnul Dumnezeul meu mi-a dăruit odihnă din toate părțile; n-am nici potrivnic, nici alte primejdii.
\par 5 Și iată eu mă gândesc să zidesc templu numelui Domnului Dumnezeului meu, după cum a grăit Domnul către tatăl meu David, zicând: Fiul tău pe care Eu îl voi pune în locul tău pe tron, acela va zidi templu numelui Meu.
\par 6 Așadar poruncește să taie pentru mine cedri din Liban; și iată robii mei vor fi împreună cu robii tăi; și eu îți voi da plată pentru robii tăi cât vei hotărî tu, că tu cunoști că la noi nu sunt oameni care să știe a tăia lemnele așa ca Sidonienii".
\par 7 Când a auzit Hiram cuvintele lui Solomon, s-a bucurat foarte și a zis: "Binecuvântat fie astăzi Domnul, Care a dat lui David fecior înțelept pentru povățuirea acestui popor nesfârșit de mare!
\par 8 Și a trimis Hiram la Solomon să-i spună: "Am auzit pentru ce ai trimis la mine și îți îndeplinesc toată dorința ta pentru lemnul de cedru și lemnul de chiparos.
\par 9 Robii mei le vor scoate din Liban la mare și eu cu plutele le voi duce pe mare la locul care ni-l vei hotărî; și acolo le voi descărca și tu le vei lua; însă și tu să plinești dorința mea: să aduci pâine pentru casa mea!"
\par 10 A dat deci Hiram lui Solomon lemn de cedru și lemn de chiparos, toate tocmai după dorința lui.
\par 11 Iar Solomon a dat lui Hiram douăzeci de mii de core de grâu pentru hrana casei iui și douăzeci de core de untdelemn de măsline curat. Atât îi da Solomon lui Hiram pe fiecare an.
\par 12 Domnul i-a dat înțelepciune lui Solomon după cum i-a făgăduit. Și a fost pace între Hiram și Solomon și amândoi între ei au făcut legământ.
\par 13 Și a pus regele Solomon o corvoadă peste tot Israelul și corvoada era de treizeci de mii de oameni.
\par 14 Și-i trimetea la Liban, câte zece mii pe lună, cu schimbul: o lună erau la Liban, iar două luni la casa lor. Iar Adoniram era căpetenie mai mare peste ei.
\par 15 Și mai avea Solomon încă șaptezeci de mii de salahori și optzeci de mii de oameni tăietori de piatră în munte,
\par 16 Afară de cele trei mii și trei sute de căpetenii, care erau puse de Solomon, să supravegheze poporul care făcea lucrul.
\par 17 Și a poruncit regele să pregătească pietre mari, pietre cu ciubuce pentru temelia templului și pietre cioplite
\par 18 Și le-au lucrat lucrătorii lui Solomon, lucrătorii lui Hiram și lucrătorii din Biblos. Și așa s-a pregătit lemnul și piatra pentru ridicarea templului, timp de trei ani.

\chapter{6}

\par 1 Iar în anul patru sute optzeci, după ieșirea fiilor lui Israel din Egipt, în al patrulea an al domniei lui Solomon peste Israel, în luna Zif, care este a doua lună a anului, a început el să zidească templul Domnului.
\par 2 Templul, pe care l-a zidit regele Solomon Domnului era lung de șaizeci de coți, lat de douăzeci și înalt de treizeci.
\par 3 Pridvorul de dinaintea templului era lung de douăzeci de coți, răspunzând cu lățimea templului, și lat de zece coți înaintea templului.
\par 4 Și a făcut el la acele odăi ferestre cu zăbrele, largi înăuntru și strâmte în afară.
\par 5 Și a mai făcut o clădire lângă zidul templului, cu trei caturi în jurul pereților templului, în jurul Sfintei Sfintelor.
\par 6 Catul de jos al clădirii era lat de cinci coți; cel din mijloc lat de șase coți, iar cel de al treilea, lat de șapte coți; căci împrejurul templului erau făcute prichiciuri de zid, ca zidirea să nu fie lipită de pereții templului.
\par 7 Când era zidit templul, la zidirea lui au întrebuințat pietre cioplite, lucrate mai dinainte. Așa că nici ciocan, nici topor, nici orice altă unealtă de fier nu s-au auzit la zidirea lui.
\par 8 Intrarea la catul de jos al clădirii era pe partea dreaptă a templului. Pe scări în spirală se suiau la catul din mijloc, și de la catul din mijloc, la catul al treilea.
\par 9 Și a zidit el templul și l-a terminat și a pardosit templul cu scânduri de cedru.
\par 10 Odăilor dimprejurul întregului templu le-a dat înălțime de câte cinci coți la fiecare cat și erau legate de templu prin grinzi de cedru.
\par 11 Atunci a fost cuvântul Domnului către Solomon și i-a zis:
\par 12 "Iată, tu-Mi zidești casă; dacă te vei purta după legile Mele, și vei urma după hotărârile Mele, și vei păzi toate poruncile Mele, lucrând după ele, atunci Îmi voi împlini și Eu cu tine cuvântul Meu pe care l-am grăit către David, tatăl tău:
\par 13 Voi locui în mijlocul fiilor lui Israel și nu voi părăsi pe poporul Meu Israel".
\par 14 Și a zidit Solomon templul și l-a terminat.
\par 15 Și a îmbrăcat pereții templului pe dinăuntru cu scânduri de cedru; de la pardoseala templului până la tavan pe dinăuntru l-a îmbrăcat peste tot cu lemn de cedru; iar pardoseala templului a făcut-o din scânduri de chiparos.
\par 16 Și a făcut în partea din fund a templului o despărțitură de douăzeci de coți lungime și a îmbrăcat pereții și tavanul casei acestei despărțituri cu scânduri de cedru; și așa a făcut despărțitura pentru Sfânta Sfintelor.
\par 17 De patruzeci de coți era despărțitura întâi a templului.
\par 18 Pe scândurile de cedru dinăuntru templului erau făcute sculpturi în formă de castraveți și flori de trandafiri îmbobociți; totul era acoperit cu cedru și piatra nu se vedea.
\par 19 Iar despărțitura din fundul templului el a pregătit-o, ca să pună acolo chivotul cu legea Domnului.
\par 20 Și despărțitura aceasta era lungă de douăzeci de coți, lată de douăzeci de coți și înaltă de douăzeci de coți; și a îmbrăcat-o cu aur curat; asemenea a îmbrăcat și jertfelnicul cel de cedru.
\par 21 Și a îmbrăcat Solomon templul și pe dinăuntru cu aur curat; și a întins lanțuri de aur pe dinaintea catapetesmei și a îmbrăcat-o cu aur.
\par 22 Tot templul l-a îmbrăcat cu aur, tot templul până la capăt, și tot jertfelnicul care este dinaintea altarului l-a îmbrăcat cu aur.
\par 23 Și a făcut în Sfânta Sfintelor doi heruvimi de lemn de măslin, înalți de zece coți.
\par 24 O aripă a heruvimului era de cinci coți și cealaltă aripă a heruvimului era tot de cinci coți. Zece coți erau de la un vârf al aripilor lui până la vârful celeilalte aripi.
\par 25 Tot de zece coți era și celălalt heruvim; amândoi heruvimii aveau aceeași măsură și aceeași înfățișare.
\par 26 Înălțimea unui heruvim era de zece coți; la fel și celălalt heruvim.
\par 27 Și a așezat el heruvimii la mijloc în partea de la fund a templului. Aripile heruvimilor erau însă întinse; și atingea aripa unuia un perete și aripa celuilalt heruvim atingea pe celălalt perete. Iar celelalte aripi ale lor se atingeau în mijlocul templului aripă de aripă.
\par 28 Și a îmbrăcat el heruvimii cu aur.
\par 29 Pe toți pereții templului de jur împrejur, pe dinăuntru și pe dinafară, a făcut chipuri săpate de heruvimi, de copaci, de finici și de flori îmbobocite.
\par 30 Și a îmbrăcat cu aur pardoseala în templu, în partea din fund și în partea din față.
\par 31 Pentru intrat în Sfânta Sfintelor, a făcut uși de lemn de măslin care se deschid în două părți, cu ușori în cinci muchii.
\par 32 Pe cele două jumătăți ale ușilor de lemn de măslin, el a făcut heruvimi săpați, finici și flori îmbobocite; și le-a îmbrăcat în aur și heruvimii și finicii.
\par 33 La intrarea în templu a făcut ușori din lemn de măslin în patru muchii
\par 34 Și două uși din lemn de chiparos, fiecare cu câte două canate. Amândouă jumătățile unei uși se învârteau într-o parte și în alta și amândouă jumătățile celeilalte uși de asemenea se; învârteau într-o parte și în alta.
\par 35 Și a săpat pe ele heruvimi, finici și flori îmbobocite, și le-a îmbrăcat cu aur peste săpătură.
\par 36 Și a făcut de asemenea curtea cea dinăuntru din trei rânduri de pietre cioplite și dintr-un rând de grinzi de cedru.
\par 37 În anul al patrulea, în luna Zif, care este a doua lună a anului, a  pus el temelia casei Domnului.
\par 38 Iar în anul al unsprezecelea, în luna Bul, care este luna a opta, a terminat el templul, cu toate părțile lor și după toate rânduielile lor; așa că l-a zidit în șapte ani.

\chapter{7}

\par 1 Iar Solomon a zidit și a terminat casa sa în treisprezece ani.
\par 2 Casa aceasta a făcut-o de lemn din Liban, lungă de o sută de coți, largă de cincizeci de coți și înaltă de treizeci de coți, pe patru șiruri de stâlpi de cedru și pe stâlpi erau puse grinzi de cedru.
\par 3 Iar peste grinzi deasupra era întinsă podeaua da lemn de cedru, care se rezema pe patruzeci și cinci de stâlpi, câte cincisprezece în șir.
\par 4 Și erau trei rânduri și fiecare rând avea ferestre așezate unele în dreptul altora, așa că răspundea fereastră cu fereastră în toate cele trei rânduri.
\par 5 Toate ușile și ușorii de uși erau din grinzi pătrate, și ferestrele la cele trei rânduri față în față unele de altele.
\par 6 Și a mai făcut un pridvor pe stâlpi, lung da cincizeci de coți, lat de treizeci de coți; și înaintea lui un pridvor mai mic, au stâlpi și trepte în față.
\par 7 Și a făcut de asemenea pridvor cu tron, de pe care el judeca, numit pridvorul judecății; pe acesta l-a făcut și l-a îmbrăcat cu cedru, da la pardoseală până la tavan.
\par 8 La casa sa de locuit, era făcută la fel altă curte și tot cu porți. Asemenea și casa fiicei lui Faraon care Solomon o luase de femeie, a făcut el un astfel de pridvor.
\par 9 Toate aceste clădiri au fost făcute din pietre alese și frumos lucrate, cioplite după măsură, retezate cu fierăstrăul înăuntru și în afară, de la temelie până la streașină, și din afară până la curtea cea mare.
\par 10 La temelie au fost puse de asemenea pietre alese, pietre mari, pietre de zece coți și opt coți.
\par 11 Și deasupra, pietre cioplite, lucrate după măsură și lemn de cedru.
\par 12 Curtea cea mare avea împrejur o îngrăditură din trei rânduri de pietre cioplite și un rând de grinzi de cedru. Tot astfel era îngrădită și curtea dinlăuntru a templului Domnului, precum și pridvorul templului.
\par 13 Și a trimis regele Solomon și a luat pe meșterul Hiram din Tir.
\par 14 Acesta era fiul unei văduve din seminția lui Neftali. Tatăl lui, un tirian, era arămar; era și Hiram plin de pricepere, cu meșteșug și cu știința de a face orice lucru din aramă. Și a venit la regele Solomon și a făcut tot felul de lucruri.
\par 15 A turnat pentru pridvor doi stâlpi de aramă, fiecare stâlp de optsprezece coți înălțime, și rotundul fiecărui stâlp ara cât putea să-l cuprindă o sfoară de doisprezece coți. Grosimea era da patru degete, iar pe dinăuntru era gol.
\par 16 Și pentru pus în capetele stâlpilor, a făcut două coroane, turnate din aramă; înălțimea unei coroane era de cinci coți și înălțimea celeilalte coroane de cinci coți;
\par 17 Și pentru acoperit coroanele care erau în capetele stâlpilor, a făcut el două rețele, lucrate împletit, șnururi în formă de lanțuri: șapte la o coroană și șapte la cealaltă coroană.
\par 18 Și, când a făcut stâlpii, a făcut și două șiruri de rodii de aramă, atârnate împrejur pe marginea rețelelor, ca să împodobească coroana ce era pe vârful stâlpului; la fel a făcut și celeilalte coroane de la celălalt stâlp.
\par 19 Coroanele din capetele stâlpilor de la pridvor erau făcute în forma cupei florii de crin, de patru coți la gură;
\par 20 Și la coroanele de la amândoi stâlpii erau sus, la încheietura lor, în dreptul marginii rețelei, rodii de aramă; și șirurile de rodii dimprejur la fiecare coroană erau de câte două sute de rodii.
\par 21 Și a așezat stâlpii la pridvorul templului, punând un stâlp în partea din dreapta și dându-i numele Iachin; și pe celălalt stâlp în partea stângă, dându-i numele Booz.
\par 22 Și pe capul stâlpilor a pus coroanele, făcute în forma cupei florii de crin. Așa s-a sfârșit lucrul stâlpilor.
\par 23 A mai făcut o mare, turnată din aramă, de zece coți de la o margine a ei până la cealaltă margine, rotundă de jur împrejur; înaltă de cinci coți și groasă cât o cuprindea o sfoară de treizeci de coți.
\par 24 Pe la gură de jur împrejur avea sculpturi în formă de colocinți, câte zece la un cot, care împrejmuiau marea din toate părțile. Chipurile colocinților așezați în două șiruri erau turnate odată cu marea dintr-o singură bucată.
\par 25 Aceasta era așezată pe doisprezece boi de aramă, din care: trei priveau spre miazănoapte, trei spre apus, trei spre miazăzi și trei spre răsărit. Marea ședea pe ei și toată partea dinapoi a trupului lor era înăuntru.
\par 26 Grosimea pereților ei era de un lat de mână; și marginile ei, făcute ca marginile potirului, semănau cu floarea de crin îmbobocit. Și încăpeau în ea două mii de baturi (vedre).
\par 27 A mai făcut apoi zece postamente de aramă. Lungimea fiecărui postament era de patru coți, lățimea de patru coți și înălțimea de trei coli.
\par 28 Înfățișarea postamentelor era așa: erau lucrate în tăblii, și tăbliile erau încheiate la unghiuri;
\par 29 Și pe tăbliile acestea, care erau încheiate la unghiuri, erau săpați lei și boi și heruvimi; asemenea și pe încheieturi. Iar deasupra și dedesubtul leilor și boilor erau închipuite ghirlande de flori.
\par 30 Fiecare postament avea câte patru roți de aramă și osii de aramă. La cele patru colțuri ale lor erau niște console, în chipul unor numere; jos sub cupa spălătoarei și pe lângă fiecare ghirlandă de flori era o policioară turnată.
\par 31 Postamentul în partea de deasupra avea înăuntru o adâncitură de pus ligheanul, adâncă de un cot; gura ei era rotundă ca baza stâlpilor, de un cot și jumătate în diametru; și împrejurul gurii erau podoabe săpate; iar tăbliile ei de pe laturi erau pătrate și nu rotunde.
\par 32 Cele patru roți erau sub. tăblii; și osiile roților erau fixate în postamente; înălțimea fiecărei roți era de un cot și jumătate.
\par 33 Forma roților era aceeași, ca forma roților de trăsură. Osiile lor, obezile lor, spițele lor și butucii lor, toate erau turnate.
\par 34 Cele patru console de la cele patru colțuri ale fiecărui postament erau tot turnate; policioarele erau ieșite din postament.
\par 35 Partea de deasupra a postamentului se termina prin o cunună înaltă de o jumătate de cot și făcută așa, ca să se poată pune spălătoarea deasupra; aceasta cu consolele și policioarele ei erau turnate din o bucată.
\par 36 Și a săpat pe fețele consolelor postamentului și pe policioarele dintre ele heruvimi, lei și finici, pe unde a găsit loc; și împrejur a atârnat ghirlande de flori.
\par 37 Așa a făcut el zece postamente: toate aveau aceeași turnătură, aceeași măsură și aceeași înfățișare.
\par 38 Și a mai făcut zece lighene de aramă: în fiecare lighean încăpea câte patruzeci de baturi; fiecare lighean era de patru coli și fiecare lighean sta pe unul din cele zece postamente.
\par 39 Și a așezat postamentele cinci în partea dreaptă a templului și cinci în partea stângă a templului, iar marea a așezat-o în partea dreaptă a templului în partea de răsărit-miazăzi.
\par 40 Și a mai făcut Hiram căldări, lopeți și cupe. Și așa a isprăvit Hiram toate lucrările date de Solomon să le facă pentru templul Domnului:
\par 41 Cei doi stâlpi cu cele două capiteluri rotunde așezate în capetele stâlpilor; cele două împletituri care acopereau capetele rotunde ale capitelurilor pe vârful stâlpilor;
\par 42 Cele patru sute de rodii de aramă de la cele două rețele; două șiruri de rodii la fiecare rețea, pentru acoperirea celor două globuri ale coroanelor care erau pe stâlpi;
\par 43 Cele zece postamente și cele zece spălători de pe postamente;
\par 44 O mare și cei doisprezece boi de sub mare;
\par 45 Căldările, lopețile și cupele. Toate lucrurile care le-a făcut Hiram regelui Solomon pentru templul Domnului erau de aramă șlefuită.
\par 46 Regele a pus să le toarne într-un pământ clisos din împrejurimile Iordanului, între Sucot și Țartan.
\par 47 Și a pus Solomon toate aceste lucruri la locul lor. Din pricina mulțimii lor peste măsură, greutatea aramei nu se mai știa.
\par 48 Și a mai făcut Solomon toate lucrurile care erau în templul Domnului: jertfelnicul cel de aur și masa cea de aur a pâinilor punerii înainte;
\par 49 Sfeșnice de aur curat: cinci în partea dreaptă și cinci în partea stingă, înaintea Sfintei Sfintelor, cu florile, candelele și mucările lor, toate de aur;
\par 50 Lighene, cuțite, cupe, linguri și cădelnițe tot de aur curat; și țâțânile ușilor celor din fundul templului de la Sfânta Sfintelor și ale ușilor celor de la templu erau toate de aur.
\par 51 Așa s-a sfârșit toată lucrarea pe care a făcut-o regele Solomon la templul Domnului. și a adus Solomon și pe cele afierosite de David, tatăl lui: argint și aur și lucruri, și le-a dat în vistieria templului Domnului.

\chapter{8}

\par 1 Atunci a adunat Solomon la el în Ierusalim pe bătrânii lui Israel, pe căpeteniile semințiilor și pe toți capii de familii ai fiilor lui Israel, ca să aducă chivotul cu legea Domnului din cetatea lui David, adică din Sion.
\par 2 Și s-au adunat la regele Solomon toți Israeliții în zilele sărbătorilor din luna Etanim, care este a șaptea lună.
\par 3 Iar după ce au venit toți bătrânii lui Israel, au ridicat preoții chivotul
\par 4 Și au adus chivotul Domnului și cortul adunării și toate lucrurile sfinte care au fost în cort; acestea le-au adus preoții și leviții.
\par 5 Iar regele Solomon împreună cu toată obștea fiilor lui Israel, care se adunaseră la el, mergea înaintea chivotului, aducând jertfe vite mărunte și mari, care nu se puteau socoti și nici număra din pricina mulțimii lor.
\par 6 Și au băgat preoții chivotul cu legea Domnului la locul lui, în Sfânta Sfintelor din templu, sub aripile heruvimilor;
\par 7 Căci heruvimii își aveau aripile întinse peste locul chivotului și heruvimii acopereau de sus chivotul și pârghiile lui:
\par 8 Pârghiile însă se împinseseră așa, încât capetele lor se vedeau din locașul sfânt, din fața Sfintei Sfintelor, iar de afară nu se zăreau; și acolo se află ele până în ziua de azi.
\par 9 În chivot nu era nimic, decât cele două table de piatră pe care Moise le pusese acolo în Horeb, când Domnul a făcut legământ cu fiii lui Israel, după ieșirea lor din pământul Egiptului.
\par 10 Când preoții au ieșit din locașul sfânt, un nor a umplut templul Domnului.
\par 11 Și n-au putut preoții să stea la slujbă, din pricina norului, căci slava Domnului umpluse templul Domnului.
\par 12 Atunci Solomon a zis: "Domnul a spus că binevoiește să locuiască în norul cel întunecos.
\par 13 Eu ți-am zidit templul pentru locuință, în care Tu să petreci în veci".
\par 14 Apoi s-a întors regele cu fața spre mulțime și a binecuvântat toată adunarea Israeliților, căci toată adunarea Israeliților sta de față,
\par 15 și a zis: "Binecuvântat fie Domnul Dumnezeul lui Israel, Care a grăit cu gura Sa către David, tatăl meu, ceea ce astăzi a împlinit cu mâna Sa!
\par 16 El a zis: Din ziua în care am scos pe poporul Meu Israel din Egipt, nu Mi-am ales cetate în nici una din semințiile lui Israel, unde să fie zidită casa în care să petreacă numele Meu; dar apoi am ales Ierusalimul pentru petrecerea numelui Meu în el și am ales pe David ca să fie peste poporul Meu, Israel.
\par 17 Lui David, tatăl meu, îi intrase la inimă să zidească casă numelui Domnului Dumnezeului lui Israel;
\par 18 Însă Domnul a zis câtre David, tatăl meu: ți-ai pus în gând să zidești casă numelui Meu; este bine că ți-ai  pus aceasta la inimă.
\par 19 Însă nu tu v ai zidi templul, ci fiul tău care va ieți din coapsele tale, acela va zidi casă numelui Meu.
\par 20 Și a Împlinit Domnul cuvântul Său care l-a grăit. Eu am urmat în locul tatălui meu. David, și am șezut pe tronul lui Israel, precum Domnul a zis, și am zidit templu numelui Domnului Dumnezeului lui Israel;
\par 21 Și am pregătit acolo loc pentru chivotul în care se află legământul Domnului, făcut cu părinții noștri, când i-a scos din pământul Egiptului".
\par 22 Apoi a stat Solomon înaintea jertfelnicului Domnului, în fața întregii adunări a lui Israel, și și-a ridicat mâinile la cer și a zis:
\par 23 "Doamne Dumnezeul lui Israel! Nu este Dumnezeu asemenea ție, nici în cer sus, nici pe pământ jos; Tu păzești legământul și ai milă de robii Tăi care umblă cu toată inima lor înaintea Ta;
\par 24 Tu ai împlinit ce ai grăit către robul Tău David, tatăl meu; căci ce ai grăit cu gura Ta, aceea astăzi ai împlinit cu mâna Ta.
\par 25 Și asum, Doamne Dumnezeul lui Israel, să împlinești ceea ce ai grăit cu robul Tău David, tatăl meu, zicând: "Nu-ți va lipsi niciodată înaintea Mea un urmaș, care să șadă pe tronul lui Israel, dacă fiii tăi își vor păzi drumul lor, purtându-se așa cum te-ai purtat tu înaintea Mea!"
\par 26 Și acum, Doamne Dumnezeul lui Israel, fă să se adeverească cuvântul Tău care l-ai grăit cu robul Tău David, tatăl meu!
\par 27 Oare adevărat să fie că Domnul va locui cu oamenii pe pământ? Cerul și cerul cerurilor nu Te încap, cu atât mai puțin acest templu pe care l-am zidit numelui Tău;
\par 28 Însă caută la rugăciunea robului Tău și la cererea lui, Doamne Dumnezeul meu! Ascultă strigarea și rugăciunea lui cu care se roagă astăzi;
\par 29 Să-ți fie ochii Tăi deschiși ziua și noaptea la templul acesta, la acest loc, pentru care Tu ai zis: "Numele Meu va fi acolo"; să asculți strigarea și rugăciunea cu care robul Tău se va ruga în locul acesta.
\par 30 Să asculți rugăciunea robului Tău și a poporului Tău, Israel, când ei se vor ruga în locul acesta; să asculți din locul șederii Tale cel din ceruri, să asculți și să miluiești.
\par 31 Când cineva va greși împotriva aproapelui său și i se va cere jurământ ca să jure și pentru jurământ ei vor veni înaintea jertfelnicului Tău la templul acesta,
\par 32 Atunci Tu să asculți din cer și să faci judecată robilor Tăi, să osândești pe cel vinovat, întorcându-i în capul lui fapta lui, și să scapi pe cel drept, dându-i după dreptatea lui!
\par 33 Când poporul Tău Israel va fi bătut de dușman, pentru că a păcătuit înaintea Ta, și ei se vor întoarce la Tine și se vor mărturisi numelui Tău aducând rugăciuni și cereri în acest templu,
\par 34 Atunci Tu să asculți din cer, să ierți păcatul poporului Tău Israel și să-l întorci în pământul pe care l-ai dat părinților lor!
\par 35 Când se va încuia cerul și nu va fi ploaie, pentru că ei au păcătuit înaintea Ta, și Îți vor aduce rugăciuni în locul acesta și vor mărturisi numele Tău și se vor întoarce de la păcatul lor, căci Tu i-ai smerit,
\par 36 Atunci Tu să asculți din cer și să ierți păcatul robilor Tăi și al poporului Tău Israel, arătându-le calea cea bună pe care să meargă, și să trimiți ploaie pământului Tău pe care l-ai dat poporului Tău de moștenire!
\par 37 De va fi foamete pe pământ, de va fi ciumă și boală molipsitoare, de va fi vânt dogoritor, uscăciune, lăcustă, omidă, dușmanul de îl va strâmtora în porțile cetății lui, de va fi orice necaz sau orice boală,
\par 38 Orice rugăciune, orice cerere care se va face de orice om din tot poporul lui Israel, când ei își vor cunoaște mustrarea cugetului lor și își vor întinde mâinile lor la templul acesta,
\par 39 Tu să asculți din cer, din locul șederii Tale, și să miluiești; să faci și să dai fiecăruia după căile sale, după cum Tu cunoști inima lui; căci Tu singur știi inima tuturor fiilor oamenilor;
\par 40 Pentru ca să se teamă de Tine toate zilele, cât vor trăi pe pământul pe care l-ai dat părinților noștri!
\par 41 Chiar străinul, care nu este din poporul Tău Israel, de va veni pentru numele Tău din pământ depărtat,
\par 42 Căci se va auzi de numele Tău cel mare și de mâna Ta cea puternică și de brațul Tău cel întins, și el va veni și se va ruga la templul acesta,
\par 43 Să-l ascuți din cer, din locul șederii Tale, și să faci tot ceea ce străinul Îți va cere ție, pentru ca să știe toate popoarele pământului de numele Tău, să se teamă de Tine, cum se teme poporul Tău Israel, și să știe că numele Tău este chemat peste templul acesta pe care eu l-am zidit!
\par 44 Când poporul Tău va porni cu război împotriva dușmanului său, pe drumul pe care-l  vei trimite, și se v ruga Domnului, întorcându-se spre cetatea care ți-ai ales-o și spre templul pe care l-am zidit numelui Tău,
\par 45 Atunci ascultă din cer rugăciunea lor și să faci ceea ce le este cu dreptate!
\par 46 Când ei vor păcătui înaintea Ta, căci nu este om care să nu păcătuiască, și Tu Te vei mânia pe ei și îi vei da dușmanilor lor, și cei care i-au robit îi vor duce în pământul dușmanului, departe sau aproape,
\par 47 Și când ei, în pământul în care se vor găsi în robie, își vor veni în sine și se vor întoarce și ți se vor ruga în pământul celor ce i-au robit, zicând: "Am păcătuit, fărădelege am făcut, vinovați suntem",
\par 48 Și se vor întoarce către Tine cu toată inima lor și cu tot sufletul lor, în pământul dușmanilor care i-au robit, și se vor ruga către Tine, întorcându-se spre pământul care l-ai dat părinților lor, spre cetatea care ți-ai ales-o și spre templul pe care l-am zidit numelui Tău,
\par 49 Atunci să asculți din cer, din locul șederii Tale, rugăciunea și cererea lor, făcându-le ceea ce este cu dreptate.
\par 50 Să ierți poporului Tău ce a păcătuit înaintea Ta și toate nelegiuirile lui care le-a făcut înaintea Ta și să trezești mila către ei în cei ce i-au robit, pentru ca să fie miloși cu ei;
\par 51 Căci ei sunt poporul Tău și moștenirea Ta, pe care l-ai scos din Egipt, din cuptorul cel de fier!
\par 52 Să-ți fie urechile Tale și ochii Tăi deschiși la rugăciunea robului Tău și la rugăciunea poporului Tău Israel, pentru ca să-i auzi totdeauna, când ei vor striga către Tine,
\par 53 Că Tu ți r-ai ales spre moștenire dintre toata popoarele pământului, precum ai grăit prin Moise, robul Tău, când ai scos pe părinții noștri din Egipt, Stăpâne Doamne!"
\par 54 Când Solomon a sfârșit toată această rugăciune și cerere către Domnul, s-a sculat dinaintea jertfelnicului Domnului, unde stătuse îngenunchiat cu mâinile întinse spre cer,
\par 55 Și stând în picioare, a binecuvântat toată adunarea Israeliților, zicând:
\par 56 "Binecuvântat fie Domnul Dumnezeu, Care a dat odihnă poporului Său Israel, precum a grăit! Nu a rămas neîmplinit nici un cuvânt din toate bunele Lui cuvinte care le-a grăit prin robul Său Moise.
\par 57 Să fie cu noi Domnul Dumnezeul nostru cum a fost El cu părinții noștri și să nu ne lase, părăsindu-ne.
\par 58 Plecând spre El inima noastră, să umblăm pe toate căile Lui și să păzim poruncile, rânduielile și legile Lui, pe care le-a poruncit părinților noștri;
\par 59 Și să fie cuvintele acestea cu care m-am rugat astăzi, înaintea Domnului, aproape de Domnul Dumnezeul nostru, ziua și noaptea, pentru ca să facă dreptate robului și poporului Său Israel, din zi în zi,
\par 60 Pentru ca să cunoască toate popoarele că Domnul este Dumnezeu și nu este altul afară de El!
\par 61 Să fie inima voastră întreagă la Domnul Dumnezeul nostru, ca să petreceți după rânduielile Lui și să păziți poruncile Lui, ca acum!"
\par 62 Și regele împreună cu toți Israeliții au adus jertfă Domnului.
\par 63 Pentru jertfa de împăcare pe care a adus-o el Domnului, Solomon a adus douăzeci și două de mii de vite mari și o sută douăzeci de mii de vite mărunte. Așa a sfințit regele și toți fiii lui Israel templul Domnului.
\par 64 Tot în acea zi regele a mai sfințit și mijlocul curții care era înaintea templului Domnului, săvârșind acolo arderea de tot, darul de pâine și grăsimea jertfelor de împăcare, pentru că jertfelnicul de aramă care se afla înaintea Domnului era mic pentru a încăpea arderea de tot, darul de pâine și grăsimea jertfelor de împăcare.
\par 65 Și a sărbătorit Solomon și sărbătoarea (Corturilor) în același timp împreună cu tot Israelul, strângându-se adunare mare de la intrarea Hamatului și până la râul Egiptului, pentru a fi înaintea Domnului Dumnezeului nostru timp de șapte zile și alte șapte zile, adică paisprezece zile.
\par 66 în ziua a opta Solomon a dat drumul poporului. Și au binecuvântat toți pe rege și s-au întors la corturile lor, bucurându-se și veselindu-se cu inima pentru tot binele ce l-a făcut Domnul robului Său David și poporului Său Israel.

\chapter{9}

\par 1 După ce Solomon a sfârșit de zidit templul Domnului și casa regelui și tot ce Solomon a dorit să facă,
\par 2 S-a arătat Domnul lui Solomon a doua oară, la Ghibeon,
\par 3 Și i-a zis Domnul: "Am auzit rugăciunea ta și cererea ta cu care te-ai rugat către Mine și ți-am îndeplinit toate după cererea ta; am sfințit templul pe care l-ai zidit, ca să petreacă numele Meu acolo în veci și vor fi ochii și inima Mea acolo în toate zilele.
\par 4 Dacă tu te vei purta înaintea feței Mele, cum s-a purtat tatăl tău David, cu inimă curată și cu dreptate, împlinind tot ce Eu ți-am poruncit, și vei păzi rânduielile și legile Mele,
\par 5 Atunci voi întări tronul regatului tău peste Israel în veci, precum i-am grăit lui David, tatăl tău, zicând: Nu vei fi lipsit niciodată de un urmaș pe tronul lui Israel.
\par 6 Iar dacă voi și fiii voștri vă veți depărta de la Mine și nu veți păzi poruncile Mele și rânduielile Mele pe care Eu vi le-am dat și vă veți duce și veți sluji și vă veți închina la alți dumnezei,
\par 7 Atunci Eu voi stârpi pe Israel de pe fața pământului pe care i l-am dat, iar templul pe care l-am sfințit în numele Meu îl voi lepăda de la fața Mea și Israel va fi de pomină și de râs între toate popoarele.
\par 8 Și de acest templu înalt, oricine va trece pe lângă el se va îngrozi și va fluiera și va zice: Pentru ce Domnul a făcut așa cu acest pământ și cu acest templu?
\par 9 Și se va zice: Pentru că au părăsit pe Domnul Dumnezeul lor, Care a scos pe părinții lor din pământul Egiptului, din casa robiei și au primit în schimb alți dumnezei și s-au închinat lor și au slujit lor; pentru aceasta a adus Domnul peste ei toată această nenorocire".
\par 10 După trecerea celor douăzeci de ani în care Solomon a zidit templul Domnului și casa regelui,
\par 11 Pentru care Hiram, regele Tirului, a dat lui Solomon lemn de cedru, lemn de chiparos și aur după cerința lui, regele Solomon a dat lui Hiram douăzeci de cetăți din pământul Galileii.
\par 12 Și a plecat Hiram din Tir și s-a dus în Galileea ca să vadă cetățile dăruite de Solomon și nu i-au plăcut.
\par 13 Și a zis: "Ce sunt, fratele meu, cetățile acestea care mi le-ai dat?" Și le-a numit pământul Cabul, cum se numesc ele până în ziua de astăzi.
\par 14 Și Hiram trimisese regelui Solomon o sută de talanți de aur.
\par 15 Iată hotărârea pentru darea pe care a pus-o regele Solomon, ca să zidească templul Domnului, casa lui, Milo, zidul Ierusalimului, Hațorul, Meghido și Ghezerul;
\par 16 Căci Faraon, regele Egiptului, venise și luase Ghezerul și-l arsese cu foc și pe Canaaneii care locuiau în cetate îi ucisese, și-l dăduse de zestre fiicei sale, femeii lui Solomon.
\par 17 Și a zidit Solomon cetățile Ghezer, Bet-Horonul de Jos,
\par 18 Baalat și Tadmorul din pustiu,
\par 19 Și toate cetățile grânare, care le-a avut Solomon și cetățile pentru carele de război, cetățile pentru călăreți și tot ce a vrut Solomon să zidească în Ierusalim, în Liban și în pământul stăpânirii sale.
\par 20 Pe tot poporul care a rămas de la Amorei, Hetei, Ferezei, Canaanei, Hevei, Iebusei și Gherghesei, care nu erau dintre fiii lui Israel,
\par 21 Și pe fiii acestora, rămași în țară după ei și pe care fiii lui Israel nu au putut să-i stăpânească, Solomon i-a făcut lucrători de corvoadă până în ziua de azi,
\par 22 Iar pe fiii lui Israel, Solomon nu i-a făcut lucrători; pe ei îi avea însă pentru oștirea lui, pentru slujitorii lui, pentru căpitanii lui, pentru căpeteniile lui și pentru conducători la carele lui și la călăreții lui.
\par 23 Iar ispravnicii cei de frunte, de peste lucrările lui Solomon, cei ce supravegheau poporul care făcea lucrul, erau cinci sute cincizeci.
\par 24 Atunci fiica lui Faraon a trecut din cetatea lui David în casa zidită de Solomon pentru ea. Apoi a zidit el Milo.
\par 25 Solomon aducea de trei ori pe an arderi de tot și jertfe de împăcare pe jertfelnicul pe care-l zidise Domnului, săvârșind tămâiere înaintea Domnului. Și a terminat el și zidirea casei lui.
\par 26 Regele Solomon a mai făcut și corăbii la Ețion-Gheber, care este lângă Elot, pe malul Mării Roșii, în pământul lui Edom.
\par 27 Și a trimis Hiram dintre supușii săi corăbieri, cunoscători ai mării, ca să ducă corăbiile cu supușii lui Solomon.
\par 28 Și s-au dus la Ofir și au luat de acolo patru sute douăzeci de talanți de aur și i-au dus regelui Solomon.

\chapter{10}

\par 1 Regina din Saba însă, auzind de slava lui Solomon cea în numele Domnului, a venit să-i încerce înțelepciunea cu cuvinte greu de înțeles.
\par 2 Venind ea la Ierusalim cu foarte mare bogăție, cu cămile încărcate cu aromate, cu foarte mult aur și pietre scumpe, a mers la Solomon și s-a sfătuit cu el pentru tot ce avea ea pe inimă.
\par 3 Și i-a dezlegat Solomon toate vorbele ei și n-a fost vorbă adâncă pe care să n-o cunoască regele și să nu i-o dezlege.
\par 4 Văzând deci regina din Saba toată înțelepciunea lui Solomon, casa care a zidit-o el,
\par 5 Bucatele de la masa lui, locuința robilor lui, rânduiala slugilor lui, îmbrăcămintea lor, paharnicii lui și arderile de tot ale lui care le aducea în templul Domnului, nu a putut să se mai stăpânească
\par 6 Și a zis regelui: "Adevărat este ce am auzit eu în țara mea de lucrurile tale și de înțelepciunea ta;
\par 7 Însă eu nu credeam vorbele, până n-am venit și n-am văzut cu ochii mei și iată, nici pe jumătate nu mi se spusese; tu ai înțelepciune și bogăție mult mai mare decât am auzit eu.
\par 8 Ferice de oamenii tăi și de aceste slugi ale tale, care totdeauna îți stau înainte și ascultă înțelepciunea ta!
\par 9 Binecuvântat să fie Domnul Dumnezeul tău Care a binevoit să te pună pe tronul lui Israel! Domnul, din dragostea cea veșnică a Lui către Israel, te-a pus rege să faci judecată și dreptate".
\par 10 Și a dăruit ea regelui o sută douăzeci de talanți de aur și o mulțime de aromate și de pietre scumpe; niciodată însă nu i s-a adus atât de multe aromate, câte a dăruit regina din Saba regelui Solomon.
\par 11 Iar corăbiile lui Hiram, care aduceau aur de la Ofir, au adus foarte mult lemn roșu șl pietre scumpe.
\par 12 Și a făcut regele din acest lemn roșu balustrade pentru templul Domnului și pentru casa regelui și de asemenea chitare și harpe pentru cântăreți. Niciodată nu i s-a adus lui atâta lemn roșu, nici nu s-a mai văzut până în ziua de azi.
\par 13 Iar regele Solomon a dat reginei din Saba tot ce a dorit și a cerut, pe lângă ce i-a dăruit regele Solomon cu mâna lui. și s-a întors ea înapoi la țara ei, ea și toate slugile ei.
\par 14 Greutatea aurului care i se aducea pe fiecare an lui Solomon era de șase sute șaizeci de talanți de aur,
\par 15 Afară de ce primea el de la aducătorii de mărfuri și de la negustori, de la toți regii arabi și de la căpeteniile ținuturilor.
\par 16 Și a făcut regele Solomon două sute de scuturi de aur ciocănit, câte șase sute de sicli pentru fiecare scut,
\par 17 Și trei sute de scuturi mai mici tot din aur ciocănit; câte trei mine de aur intra în fiecare scut; și le-a pus regele în casa numită Pădurea Libanului.
\par 18 Și a mai făcut regele un tron mare de os de fildeș, ferecându-l cu aur curat.
\par 19 Tronul avea șase trepte, iar vârful spetezei tronului era rotund; și avea de o parte și de alta a locului de ședere rezemători, pe care stăteau doi lei.
\par 20 Și mai erau încă doisprezece lei care stăteau de o parte și de alta a tronului, pe cele șase trepte. Asemenea tron nu mai era în nici un regat.
\par 21 Toate vasele de băut ale regelui Solomon erau de aur; lighenele lui tot de aur și toate vasele din casa Pădurea Libanului erau de aur curat; de argint nu era nimic făcut; argintul nu valora în zilele lui Solomon,
\par 22 Căci regele avea pe mare corăbii care mergeau la Tarsis cu corăbiile lui Hiram și la trei ani o dată veneau corăbiile din Tarsis și îi aduceau aur, argint, fildeș, maimuțe și păuni.
\par 23 Regele Solomon a întrecut pe toți regii pământului în bogăție și în înțelepciune.
\par 24 Și toți regii de pe pământ căutau să vadă pe Solomon, ca să-i asculte înțelepciunea pe care i-o pusese Dumnezeu în inima lui.
\par 25 Și-i aduceau fiecare de la ei, ca dar în fiecare an, vase de aur și argint, haine, arme, aromate, cai și catâri.
\par 26 Și și-a adunat Solomon care și călăreți; și avea el o mie patru sute de care și douăsprezece mii de călăreți; și i-a așezat în cetățile unde ținea carele și pe lângă rege, în Ierusalim. Și e! era domn peste toți regii de la râul Eufrat până la pământul Filistenilor și până în hotarul Egiptului.
\par 27 El a făcut ca argintul să fie tot așa de prețuit la Ierusalim ca pietrele, iar cedrii, prin mulțimea lor i-a făcut să fie prețuiți ca și smochinii cei sălbatici care cresc prin locuri joase.
\par 28 Iar caii pentru regele Solomon se aduceau din Egipt și din Coa. Negustorii regelui luau cai din Coa (Cheve) cu bani.
\par 29 Un car din Egipt se cumpăra și se aducea cu șase sute sicli de argint, iar un cal, cu o sută cincizeci de sicli. Tot astfel aduceau ei toate acestea și pentru regii Heteilor și regii Siriei.

\chapter{11}

\par 1 Regele Solomon, în afară de fata lui Faraon, a iubit și alte multe femei străine: moabite, amonite, idumeiene, sidoniene, hetite și amoriene,
\par 2 Adică din acele popoare pentru care Domnul zisese fiilor lui Israel: "Să nu vă duceri la ele, nici ele să nu vină la voi, ca să nu vă întoarcă inima voastră spre dumnezeii lor". De acestea s-a lipit Solomon cu dragoste.
\par 3 Și a avut el șapte sute de femei și trei sute de concubine; și femeile i-au smintit inima lui.
\par 4 La timpul bătrâneții lui Solomon, femeile lui i-au întors inima spre alți dumnezei și inima lui nu i-a mai fost deloc întreagă la Domnul Dumnezeul său, ca inima lui David, tatăl său.
\par 5 Și a început Solomon să slujească Astartei, zeița Sidonienilor, și lui Moloh, idolul Amoniților.
\par 6 Astfel a făptuit Solomon lucruri neplăcute înaintea ochilor Domnului și nu a urmat cu stăruință după Domnul, cum urmase David, tatăl lui.
\par 7 Atunci a zidit Solomon capiștea lui Chemoș, idolul Moabiților, pe muntele din fața Ierusalimului, și capiștea lui Moloh, idolul Amoniților.
\par 8 Tot așa a făcut el pentru toate femeile sale străine, care tămâiau și aduceau jertfe dumnezeilor lor.
\par 9 Atunci S-a mâniat Domnul pe Solomon, pentru că și-a abătut inima lui de la Domnul Dumnezeul lui Israel, Care de două ori i Se arătase,
\par 10 Și-i poruncise chiar anume pentru aceasta, ca el să nu umble după alți dumnezei, ci să păzească și să facă cele ce i-a poruncit Domnul Dumnezeu; dar el n-a împlinit cele ce i-a poruncit Domnul Dumnezeu.
\par 11 Și a zis Domnul lui Solomon: "Pentru că s-au făcut aceste lucruri cu tine și pentru că tu nu ai ținut legământul și rânduielile Mele pe care ți le-am poruncit, voi rupe regatul tău din mâna ta și-l voi da slujitorului tău;
\par 12 Însă nu voi face aceasta în zilele tale, pentru David, tatăl tău, ci din mâna fiului tău îl voi rupe.
\par 13 Și nu tot regatul îl voi rupe: o seminție o voi da fiului tău, pentru David, robul Meu, și pentru Ierusalimul pe care l-am ales".
\par 14 Și a ridicat Domnul un vrăjmaș împotriva lui Solomon, pe Hadad Idumeul, viță de rege din Edom.
\par 15 Căci pe când David era în Idumeea și Ioab, mai-marele oștirii lui, venise ca să îngroape pe cei uciși și a ucis pe toți cei de parte bărbătească din Idumeea,
\par 16 Pentru că șase luni a stat acolo Ioab și toți Israeliții, stârpind pe toți cei de parte bărbătească din Idumeea,
\par 17 Atunci acest Hadad a fugit în Egipt și împreună cu el și vreo câțiva idumei, care fuseseră în slujba tatălui lui. Hadad era pe atunci copil mic.
\par 18 Plecând din Madian, ei au venit la Paran și au luat cu ei oameni din Paran și au venit în Egipt la Faraon, regele Egiptului.
\par 19 Și s-a dus Hadad la Faraon și acesta i-a dat casă, i-a purtat grijă de mâncare și i-a dat și pământ. În sfârșit, Hadad a găsit la Faraon milă foarte mare, căci el i-a dat femeie pe sora femeii lui, pe sora reginei Tafnes.
\par 20 Și i-a născut sora Tafnesei fiu lui Hadad pe Ghenubat, pe care Tafnes însuși l-a crescut în casa lui Faraon. Și a trăit Ghenubat în casa lui Faraon la un loc cu fiii lui Faraon.
\par 21 Când Hadad a auzit că David a răposat cu părinții lui și că Ioab, mai-marele oștirii a murit, atunci a zis lui Faraon: "Dă-mi drumul, că vreau să plec în țara mea!"
\par 22 Și i-a zis Faraon lui Hadad: "Ce lipsește ție aici la mine, de vrei să pleci? Și el a răspuns: "Nimic; însă așa te rog, să-mi dai drumul!" Și s-a întors Hadad în țara lui.
\par 23 Apoi a mai ridicat Dumnezeu împotriva lui Solomon încă un vrăjmaș, pe Rezon, fiul lui Eliada, care fugise la Hadad-Ezer, regele din Țoba.
\par 24 Acesta, după ce David a zdrobit pe Hadad-Ezer, adunându-și oameni împrejurul său, s-a făcut căpetenia unei bande de hoți și s-a dus la Damasc și a locuit acolo și s-a făcut stăpân pe Damasc.
\par 25 El a dușmănit pe Israel în toate zilele lui Solomon și, afară de răul ce-l pricinuia Hadad acestuia din urmă, Rezon totdeauna a adus vătămare lui Israel și s-a făcut rege în Siria.
\par 26 Asemenea și Ieroboam, fiul lui Nabat, Efremiteanul din Țereda, pe a cărui mamă văduvă o chema Țerua, și care era roaba lui Solomon, a ridicat mâna împotriva regelui.
\par 27 Și iată împrejurarea în care el și-a ridicat mâna împotriva regelui: Pe când Solomon zidea Milo și repara stricăciunile de la cetatea lui David, tatăl său,
\par 28 Se afla acolo și Ieroboam, om tare în putere. Solomon, băgând de seamă că acest om tânăr știe să facă treabă, l-a pus dregător peste lucrătorii de corvoadă din casa lui Iosif.
\par 29 În acel timp i s-a întâmplat lui Ieroboam să iasă din Ierusalim și l-a întâlnit în drum proorocul Ahia din Șilo, care avea pe el o haină nouă. În câmp erau numai ei amândoi.
\par 30 Și a luat Ahia de pe el haina cea nouă, a rupt-o în douăsprezece bucăți
\par 31 Și a zis lui Ieroboam: "Ia-ți pentru tine zece bucăți căci așa grăiește Domnul Dumnezeul lui Israel: Iată, Eu rup regatul din mâna lui Solomon și-ți dau ție zece seminții,
\par 32 Iar două seminții îi vor rămâne lui, pentru robul Meu David și pentru cetatea Ierusalimului, pe care am ales-o dintre toate semințiile lui Israel.
\par 33 Și aceasta o fac, pentru că ei M-au părăsit și au început să se închine Astartei, zeița Sidonienilor, lui Chemoș, dumnezeul Moabiților, și lui Moloh, dumnezeul fiilor Amoniților, și n-au umblat în căile Mele, ca să facă cele plăcute înaintea ochilor Mei și să păzească rânduielile Mele și poruncile Mele, cum a făcut David, tatăl lui Solomon.
\par 34 Tot regatul nu din mâna lui îl iau, căci îl las pe el să fie stăpân în toate zilele vieții lui, pentru David, robul Meu, pe care l-am ales și care a păzit poruncile și rânduielile Mele;
\par 35 Ci din mâna fiului lui voi lua regatul și-ți voi da din el ție zece seminții;
\par 36 Iar fiului lui îi voi da două seminții, pentru ca să fie și pentru David, robul Meu, în toate zilele, o lumină înaintea feței Mele, în cetatea Ierusalimului pe care am ales-o pentru petrecerea numelui Meu acolo.
\par 37 Și pe tine te voi alege și vei domni peste tot ce-ți dorește sufletul tău și vei fi rege peste Israel.
\par 38 Și dacă tu vei păzi tot ce-ți poruncesc și vei umbla în căile Mele și vei face cele plăcute înaintea ochilor Mei ca să păzești rânduielile Mele și poruncile Mele, cum a făcut David, robul Meu, atunci Eu voi fi cu tine și-ți voi zidi casă tare, cum am zidit lui David și îți voi da Israelul;
\par 39 Și voi umili neamul lui David pentru aceasta, însă nu pentru totdeauna".
\par 40 De aceea Solomon a căutat să omoare pe Ieroboam; dar Ieroboam s-a sculat și a fugit în Egipt, la Șișac, regele Egiptului, și a trăit în Egipt până la moartea lui Solomon.
\par 41 Iar celelalte fapte ale lui Solomon de la sfârșit și tot ce el a mai făcut și toată înțelepciunea lui sunt scrise în cartea faptelor lui Solomon.
\par 42 Timpul domniei lui Solomon în Ierusalim peste tot Israelul a fost de patruzeci de ani.
\par 43 Și a adormit cu părinții lui și a fost înmormântat Solomon la un loc cu părinții lui, în cetatea lui David, tatăl său; iar în locul lui în Ierusalim s-a făcut rege Roboam, fiul său.

\chapter{12}

\par 1 Atunci s-a dus Roboam la Sichem, căci la Sichem veniseră toți Israeliții, ca să-l facă rege.
\par 2 Și a auzit și Ieroboam, fiul lui Nabat, de aceasta, pe când se afla încă în Egipt, unde fugise de regele Solomon, și a venit Ieroboam din Egipt, căci au trimis unii după el și l-au chemat.
\par 3 Atunci Ieroboam și toată adunarea Israeliților au venit de i-au grăit regelui Roboam și i-au zis:
\par 4 "Tatăl tău a pus jug greu pe noi; însă ușurează-ne munca cea grea a tatălui tău și jugul cel greu care l-a pus el pe noi, și îți vom sluji!"
\par 5 Și a zis el către ei: "Duceți-vă și să veniți poimâine la mine!" Și a plecat poporul.
\par 6 Atunci regele Roboam a întrebat pe bătrânii care fuseseră sfetnici pe lângă Solomon, tatăl lui, pe când trăia el, și le-a zis: "Cum mă sfătuiți să răspund poporului la cererea aceasta?"
\par 7 Grăit-au lui aceia și au zis: "Dacă tu vei fi slugă astăzi poporului acestuia și-i vei sluji, dacă le vei face gustul lor și le vei vorbi cu blândețe, atunci ei îți vor fi robi în toate zilele!
\par 8 Însă el n-a ținut seamă de sfatul bătrânilor pe care i l-au dat, și a făcut sfat cu cei tineri, crescuți odată cu el, și care erau sfetnicii lui și le-a zis:
\par 9 "Ce mă sfătuiți să răspund poporului care mi-a zis: Ușurează jugul pe care l-a pus tatăl tău pe noi?"
\par 10 Și i-au grăit oamenii cei tineri, care crescuseră odată cu el și erau acum sfetnicii lui, și i-au zis: "Așa să spui poporului acestuia care ți-a grăit și ți-a zis: Tatăl tău a pus jug greu pe noi; tu însă ușurează-l de pe noi; așa spune-le: Degetul meu cel mic este mai gros decât mijlocul tatălui meu.
\par 11 Deci dacă tatăl meu v-a împovărat cu jug greu, eu și mai greu voi face jugul vostru; tatăl meu v-a pedepsit cu bice, iar eu vă voi pedepsi cu scorpioane!
\par 12 Atunci a venit Ieroboam și tot poporul la Roboam a treia zi, după cum poruncise regele, când a zis: "Să veniți la mine poimâine!"
\par 13 Și a răspuns regele poporului cu asprime și n-a ținut seamă de sfatul care i-l dăduseră bătrânii,
\par 14 Ci a grăit către el după sfatul celor tineri și a zis: "Tatăl meu a pus jug greu peste voi; eu însă și mai greu voi face jugul vostru; tatăl meu v-a pedepsit cu bice, eu însă vă voi pedepsi cu scorpioane!"
\par 15 Și n-a ascultat regele de popor, căci așa fusese rânduit de Domnul ca să se împlinească cuvântul Lui pe care l-a grăit Domnul prin Ahia din Șilo către Ieroboam, fiul lui Nabat.
\par 16 Iar când toți Israeliții au văzut că regele n-a vrut să-i asculte, a răspuns și poporul regelui și i-a zis: "Ce parte avem noi cu David? Nici o moștenire nu avem noi cu fiul lui Iesei. Pleacă, dar, la corturile tale, Israele! Și tu, Davide, cunoaște-ți acum casa ta! Și s-a împrăștiat Israel la corturile lui.
\par 17 Acum Roboam domnea numai peste fiii lui Israel care locuiau în cetățile lui Iuda.
\par 18 Și a trimis regele Roboam pe Adoniram, căpetenia cea peste dări, în contra lor; însă toți Israeliții au aruncat cu pietre asupra lui și el a murit; iar regele Roboam s-a suit repede într-un car, ca să fugă la Ierusalim.
\par 19 Și astfel s-a rupt Israelul de casa lui David până în ziua de astăzi.
\par 20 Când ton Israeliții au auzit că Ieroboam s-a întors din Egipt, au trimis și l-au chemat la adunare Și l-au făcut rege peste toți Israeliții. Cu casa lui David n-a rămas nimeni, decât seminția lui Iuda.
\par 21 Roboam, venind la Ierusalim, a adunat din toată casa lui Iuda și din seminția lui Veniamin o sută și optzeci de mii de ostași, pregătiți pentru război, ca să se lupte cu casa lui Israel și să întoarcă regatul iarăși sub stăpânirea lui Roboam, fiul lui Solomon.
\par 22 Și a fost cuvântul Domnului către Șemaia, omul lui Dumnezeu, și i-a zis:
\par 23 "Spune-i lui Roboam, fiul lui Solomon, regele lui Iuda, și la toată casa lui Iuda și a lui Veniamin și la celălalt popor:
\par 24 Așa zice Domnul: Să nu mergeți, nici să nu începeți război cu frații voștri, fiii lui Israel. Să vă întoarceți fiecare la casele voastre, căci de la Mine a fost aceasta! Și au ascultat ei de cuvântul Domnului și s-au întors înapoi după cuvântul Domnului. Și a răposat regele Solomon cu părinții lui și a fost înmormântat la un loc cu părinții lui în cetatea lui David, tatăl său; iar în locul lui în Ierusalim s-a făcut rege Roboam, fiul său. Roboam era de patruzeci și unu de ani când s-a făcut rege și șaptesprezece ani a domnit el în Ierusalim. Pe mama lui o chema Naama și a fost fiica lui Ana, fiul lui Nahaș, regele fiilor Amoniților. Și a făcut fapte ce nu erau plăcute înaintea ochilor Domnului și nici n-a umblat pe căile lui David, strămoșul său. Solomon însă avusese un rob, un om din muntele lui Efraim, pe care-l chema Ieroboam, iar pe mama lui o chema Țerua, femeie desfrânată. Și l-a pus Solomon între dregătorii cei peste lucrătorii de corvoadă ai casei lui Iosif. Și a zidit lui Solomon cetatea Tirța, din muntele lui Efraim. Și avea sub el trei sute de care cu cai. El a zidit Milo cu lucrătorii de corvoadă ai casei lui Efraim; tot el a reparat și stricăciunile de la cetatea lui David și a fost conducător peste regat. Și a căutat Solomon să-l omoare. și s-a temut Ieroboam și a fugit la; Șișac, regele Egiptului. și a fost la Șișac până a murit Solomon. și a auzit Ieroboam, pe când se afla încă în Egipt, că a murit Solomon. A șoptit el la urechile lui Șișac, regele Egiptului, zicând: "Dă-mi drumul, ca să mă duc în țara mea". Ți i-a zis Șișac: Cere ce dorești și îți voi da. Și Șișac i-a dat lui Ieroboam de femeie pe Ano, sora cea mai mare a Tafnesei, femeia lui. Aceasta era cea mai mare între fiicele regelui, și i-a născut lui Ieroboam pe Abia, fiul lui. Și a zis Ieroboam lui Șișac: "De bună seamă, dă-mi drumul, că am să mă duc". Și a plecat Ieroboam din Egipt și a venit în pământul Tirța, din muntele Efraim. Și s-a adunat acolo toată seminția lui Efraim. Și a zidit Ieroboam acolo tabere. Și i s-a îmbolnăvit copilul lui de o boală foarte grea. Și s-a dus Ieroboam să întrebe pentru copil. Și a zis către Ano, femeia lui: "Scoală-te și du-te de întreabă pe un om al lui Dumnezeu pentru copil dacă el are să scape cu viață din boala lui". Și era un om în Șilo, pe care-l chema Ahia. Acesta era un bărbat ca de șaizeci de ani și cuvântul Domnului era cu el. Și a zis Ieroboam către femeia lui: "Scoală-te și ia în mâna ta pentru omul lui Dumnezeu pâini și turte pentru copiii lui și struguri și un ulcior cu miere". și s-a sculat femeia și a luat în mâna ei pâini și două turte, struguri și un ulcior cu miere pentru Ahia. Și omul acesta era bătrân și nu putea să vadă, căci ochii lui i se uscaseră de bătrânețe. Și s-a sculat și a plecat din Tirța. Și, pe când venea ea în cetate la Ahia Șilonitul, a zis Ahia către fiul lui: "Ieși pentru întâmpinarea Anoei, femeia lui Ieroboam, și-i spune: Intră și nu sta, căci așa zice Domnul: Eu sunt trimis la tine ca să-ți vestesc lucruri grozave! și a intrat Ano la omul lui Dumnezeu. Și i-a zis Ahia: Pentru ce mi-ai adus pâinile, strugurii și ulciorul cu miere? Așa zice Domnul: "Iată, tu vei pleca de la mine, și când vei intra în Tirța, în cetate, îți vor ieși locuitorii în întâmpinare și îți vor spune: Copilul a murit. Căci așa zice Domnul: Iată, Eu voi pierde din ai lui Ieroboam pe tot sufletul de parte bărbătească. Cine va muri din ai lui Ieroboam în cetate, pe acela îl vor linge câinii; iar cine va muri în câmp, pe acela păsările cerului îl vor mânca și copilul va fi mult jelit. Vai, Doamne, căci numai în el din casa lui Ieroboam s-a găsit ceva bun înaintea Domnului". Și a plecat înapoi femeia când a auzit aceasta. și când a intrat în Tirța, copilul a murit. Și i-a ieșit în întâmpinare țipătul cel de jale. și s-a dus Ieroboam la Sichem, cetatea cea din muntele lui Efraim. Și s-au adunat acolo semințiile lui Israel. Și a mers acolo și Roboam, fiul lui Solomon. și cuvântul Domnului a fost către Șemaia Enlamitul, zicând: "Ia-ți o haină nouă, care n-a fost spălată cu apă, și o rupe în douăsprezece bucăți și i le dă lui Ieroboam și-i spune: "Așa zice Domnul: Ia-ți pentru tine zece bucăți, ca să te încingi". Și a luat Ieroboam, și a zis Șemaia: "Așa zice Domnul: Să fii peste cele zece seminții ale lui Israel!" și a zis poporul către Roboam, fiul lui Solomon: "Tatăl tău a pus jug greu pe noi și și-a îngreuiat bucatele mesei lui; tu însă acum ușurează-l de pe noi și-ți vom sluji ție!" Și a zis Roboam către popor: "Până în trei zile eu vă voi da răspunsul". Și a zis Roboam: "Aduceți-mi pe cei bătrâni, ca să mă sfătuiesc cu ei pentru cuvântul care voi răspunde poporului în ziua a treia". Și a grăit Roboam la urechile lor pentru cuvântul care i-a trimis poporul la el. Și au zis bătrânii poporului pentru acel cuvânt, care i-a grăit poporul. și a lepădat Roboam sfatul lor, pentru că nu se potrivea cu părerea lui. Și a trimis și a adunat pe cei ce crescuseră împreună cu el și le-a grăit: "Acestea și acestea a trimis poporul la mine, ca să-mi spună". și au zis cei ce crescuseră împreună cu el: "Așa vei grăi poporului: Degetul meu cel mic este mai gros decât mijlocul tatălui meu. Tatăl meu v-a biciuit cu biciul, iar eu vă voi bate cu scorpioane". Și i-a plăcut lui Roboam acest cuvânt și a răspuns el poporului în felul cum l-au sfătuit cei ce crescuseră și copilăriseră împreună cu el. Și a zis tot poporul, fiecare cu vecinul său până într-un om, și au strigat toți cu glas mare și au zis: "Nici o parte n-avem noi cu David, nici o moștenire cu fiul lui Iesei! Fiecare la corturile sale, Israele! Căci omul acesta n-a fost nici căpetenie și nici conducător peste regat". Și s-a împrăștiat tot poporul de la Sichem și s-a dus fiecare la cortul lui. Iar Roboam, scăpând cu fuga, s-a suit în una din căruțele lui și a venit la Ierusalim. Și l-au urmat numai seminția lui Iuda și seminția lui Veniamin. Iar când a fost începutul anului, Roboam a adunat pe toți bărbații lui Iuda și ai lui Veniamin și a plecat ca să se bată cu Ieroboam la Sichem. Și a fost cuvântul Domnului către Șemaia, omul lui Dumnezeu, zicând: "Spune-i lui Roboam, regele lui Iuda, și la toată casa lui Iuda și a lui Veniamin și la celălalt popor, zicând: Așa zice Domnul, să nu mergeți, nici să începeți război cu frații voștri, fiii lui Israel, ci să vă întoarceți fiecare la corturile voastre, că de la Mine a fost aceasta!" Și au ascultat ei de cuvântul Domnului și s-au oprit de a mai începe războiul, după cuvântul Domnului.
\par 25 Și a zidit Ieroboam Sichemul, pe muntele lui Efraim, și s-a așezat cu locuința lui în el; și a plecat de acolo și a zidit Penuelul.
\par 26 Și a zis Ieroboam în sine: "Regatul poate să treacă iar la casa lui David.
\par 27 Dacă poporul acesta va mai merge la Ierusalim pentru aducere de jertfă în templul Domnului, atunci inima poporului acestuia are să se întoarcă la domnul lor Roboam, regele lui Iuda; pe mine au să mă ucidă, iar ei au să se întoarcă iarăși sub stăpânirea lui Roboam, regele lui Iuda".
\par 28 Și sfătuindu-se, regele a făcut doi viței de aur și a zis poporului: "Nu trebuie să vă mai duceți la Ierusalim; iată Israele dumnezeii tăi, care te-au scos din pământul Egiptului!"
\par 29 Și a pus unul în Betel, iar pe celălalt în Dan.
\par 30 Însă fapta aceasta a dus la păcat, căci poporul a început să meargă pentru a se închina unuia din ei, până la Dan, și a părăsit templul Domnului.
\par 31 Și a zidit el și capiști pe înălțimi și a făcut preoți, luați din popor, care nu erau fii ai lui Levi.
\par 32 Și a așezat Ieroboam și o sărbătoare în luna a opta, în ziua a cincisprezecea a lunii, asemenea cu sărbătoarea care era în Iuda, și a adus jertfe pe jertfelnic. Tot așa a făcut el și la Betel, ca să aducă jertfe vițeilor pe care-i făcuse. Și a așezat la Betel preoți pentru înălțimile făcute de el.
\par 33 Și a adus jertfe pe jertfelnicul pe care l-a făcut la Betel în ziua a cincisprezecea din luna a opta, lună pe care și-o alesese el după placul lui pentru sărbătorire. Și a făcut sărbătoare pentru fiii lui Israel și s-a apropiat de jertfelnic, ca să săvârșească tămâiere.

\chapter{13}

\par 1 Iată însă un om al lui Dumnezeu a venit, după cuvântul Domnului, din Iuda la Betel, în timpul când Ieroboam se afla la jertfelnic, ca să tămâieze.
\par 2 Și a grăit cuvântul Domnului înaintea jertfelnicului și a zis: "Jertfelnice, jertfelnice, așa zice Domnul: Iată că i se va naște casei lui David un fiu, numele lui, Iosia, și va jertfi pe tine pe preoții înălțimilor care tămâiază acum înaintea ta și va arde pe tine oase de oameni!"
\par 3 Și a arătat în acea zi și un semn zicând: "Iată semnul după care se va cunoaște că Domnul a grăit aceasta: Jertfelnicul acesta se va despica, și cenușa care este pe el se va împrăștia!"
\par 4 Când regele Ieroboam a auzit cuvântul omului lui Dumnezeu pe care l-a grăit în gura mare în fala jertfelnicului de la Betel, și și-a întins mâna lui de la jertfelnic, zicând: "Puneți mâna pe el!", i-a înlemnit mâna care o întinsese asupra lui și nu putea să o întoarcă înapoi.
\par 5 Și jertfelnicul s-a despicat și cenușa de pe jertfelnic s-a împrăștiat, după semnul care l-a dat omul lui Dumnezeu prin cuvântul Domnului.
\par 6 Atunci a zis regele Ieroboam către omul lui Dumnezeu: "Îmblânzește fața Domnului Dumnezeului tău și roagă-te pentru mine, ca să mi se poată întoarce mâna mea la mine". Și a îmblânzit omul lui Dumnezeu fala Domnului și mâna regelui s-a întors la el și s-a făcut ca și înainte.
\par 7 Și a grăit regele către omul lui Dumnezeu: "Vino la mine acasă și prânzește cu mine și-ți voi da un dar!"
\par 8 Însă omul lui Dumnezeu a zis regelui: "Măcar să-mi dai tu și jumătate din casa ta, eu nu voi merge la tine, nici pâine nu voi mânca și nici apă nu voi bea în acest loc.
\par 9 Căci așa mi s-a poruncit prin cuvântul Domnului: Să nu mănânci acolo pâine, apă să nu bei, nici să te întorci pe drumul pe care te-ai dus!"
\par 10 Și a plecat el pe alt drum și nu s-a întors înapoi pe drumul pe care venise la Betel.
\par 11 În Betel trăia atunci un prooroc bătrân; și au venit fiii săi și i-au istorisit tot ce a făcut omul lui Dumnezeu în acea zi în Betel; de asemenea au istorisit ei tatălui lor și cuvintele care le-a grăit el regelui.
\par 12 Și i-a întrebat tatăl lor: "Pe ce drum a apucat el?" Și fiii lui i-au arătat drumul pe care a apucat omul lui Dumnezeu, care venise din Iuda.
\par 13 Și a zis el fiilor săi: "Puneți-mi șaua pe asin". Și i-a pus șaua pe asin și a încălecat pe el.
\par 14 Și a plecat după omul lui Dumnezeu și l-a găsit șezând sub un stejar și i-a zis: "Tu ești omul lui Dumnezeu venit din Iuda?" El a zis: "Da, eu sunt!"
\par 15 Și i-a zis lui: "Hai cu mine acasă, ca să mănânci pâine!
\par 16 Acela a zis: "Nu pot să mă întorc, nici să merg la tine, că eu pâine nu voi mânca și nici apă nu voi bea în acest loc la tine,
\par 17 Căci prin cuvântul Domnului mi s-a spus: "Să nu mănânci pâine, să nu bei acolo apă și nici să te întorci pe drumul pe care te-ai dus!"
\par 18 Și i-a zis el: "Și eu sunt prooroc ca tine; și îngerul mi-a grăit prin cuvântul Domnului și a zis: Întoarce-l la tine acasă, ca să mănânce pâine și să bea apă!"
\par 19 Și l-a înșelat, căci s-a întors cu el și a mâncat pâine și a băut apă în casa lui.
\par 20 Pe când ei încă ședeau la masă, cuvântul Domnului a fost către proorocul cel întors din drum și a grăit omului lui Dumnezeu venit din Iuda
\par 21 Și a zis: "Așa zice Domnul: Fiindcă tu nu te-ai supus cuvintelor Domnului și nici n-ai păzit porunca dată de Domnul Dumnezeul tău,
\par 22 Ci te-ai întors, ai mâncat pâine și ai băut apă în locul de care El ți-a spus: Să nu mănânci pâine, nici să bei apă, trupul tău nu va fi îngropat în mormântul părinților tăi!",
\par 23 Iar după ce el a mâncat pâine și a băut apă, proorocul bătrân a pus șaua pe asinul său pentru proorocul care fusese întors.
\par 24 Și a plecat acela și l-a întâlnit un leu în drum și l-a omorât. Și zăcea trupul lui, aruncat în drum; iar asinul și leul stăteau lângă el.
\par 25 Și iată niște drumeți, care au trecut pe alături, i-au văzut trupul aruncat în drum și pe leu stând lângă trup. Aceștia au venit și au vestit cetății în care trăia proorocul cel bătrân.
\par 26 Proorocul care l-a întors de pe drum, auzind acestea, a zis: "Acesta este omul lui Dumnezeu, acela care nu s-a supus cuvintelor Domnului; Domnul l-a dat leului care l-a sfâșiat și l-a omorât, după cuvântul Domnului, pe care l-a grăit pentru el".
\par 27 Și a zis fiilor săi: "Puneți șaua pe asin!" Și i-au pus ei șaua pe asin.
\par 28 Și a plecat și a găsit trupul lui, aruncat în drum, iar asinul și leul stăteau lângă trup; leul nu mâncase trupul omului lui Dumnezeu și nici pe asin nu-l sfâșiase.
\par 29 Și a ridicat proorocul trupul omului lui Dumnezeu și l-a pus pe asin și l-a adus înapoi. Și a venit proorocul cel bătrân în cetatea sa, ca să-l plângă și să-l înmormânteze.
\par 30 Și i-a pus trupul lui într-un mormânt al său și l-a plâns, zicând: "Vai, vai, fratele meu!"
\par 31 Iar după ce l-a înmormântat, a zis el fiilor săi: "Când eu voi muri, să mă înmormântați și pe mine tot în mormântul în care este înmormântat omul lui Dumnezeu; să puneți oasele mele lângă oasele lui;
\par 32 Căci cu adevărat se va împlini cuvântul, care l-a grăit el din porunca Domnului pentru jertfelnicul din Betel și pentru toate capiștele înălțimilor, care sunt în cetățile Samariei".
\par 33 Dar Ieroboam nu s-a întors din calea lui cea rea nici după întâmplarea aceasta, ci a venit iarăși la calea dinainte și a făcut din poporul de rând preoți pentru înălțimi; pe cine vrea, pe acela își punea mâna lui și-l făcea pe acela preot de înălțimi.
\par 34 Aceasta a dus casa lui Ieroboam la păcat și la pieire și la stârpirea ei de pe fața pământului.

\chapter{14}

\par 1 În acel timp s-a îmbolnăvit Abia, fiul lui Ieroboam.
\par 2 Și a zis Ieroboam femeii sale: "Scoală-te și te schimbă de haine, ca să nu te cunoască nimeni că tu ești femeia lui Ieroboam, și du-te la Șilo. Acolo este proorocul Ahia; el mi-a spus că voi fi rege peste acest popor.
\par 3 Ia cu tine pentru omul lui Dumnezeu: zece pâini, turte și un ulcior cu miere și du-te la el, că el îți va spune ce se va întâmpla cu copilul acesta".
\par 4 Femeia lui Ieroboam așa a și făcut; s-a sculat și s-a dus la Șilo și a tras la casa lui Ahia. Acum însă Ahia nu putea să vadă, căci ochii i se întunecaseră de bătrânețe.
\par 5 Și a zis Domnul lui Ahia: "Iată, vine femeia lui Ieroboam, ca să te întrebe de fiul ei, că este bolnav; așa și așa îi vei spune; ea vine schimbată de haine".
\par 6 Ahia auzind zgomotul pașilor ei, când a intrat pe ușă, a zis: "Intră femeie a lui Ieroboam. Ce gând ai avut de n-ai schimbat hainele? Eu sunt trimis să-ți vestesc lucruri grozave.
\par 7 Du-te și spune-i lui Ieroboam: Așa zice Domnul Dumnezeul lui Israel: Te-am ridicat din mijlocul poporului de jos și te-am pus conducător peste poporul Meu Israel.
\par 8 Și am rupt regatul de la casa lui David și Zi l-am dat ție; iar tu nu te-ai purtat așa cum s-a purtat robul Meu David, care a păzit poruncile Mele și a umblat după Mine cu toată inima lui, făcând numai lucrurile care sunt plăcute înaintea ochilor Mei.
\par 9 Tu însă ai întrecut în răutate pe toți care au fost înaintea ta, căci te-ai dus și ți-ai făcut alți dumnezei și chipuri turnate, ca să Mă întărâți la mânie, iar pe Mine M-ai aruncat înapoia ta.
\par 10 Pentru aceasta voi aduce necazuri peste casa lui Ieroboam și voi stârpi din casa lui Ieroboam pe toți cei de parte bărbătească, rob sau liber în Israel, și voi curăța casa lui Ieroboam cum se mătură gunoiul, de rămâne curat.
\par 11 Cine va muri dintre cei care sunt ai lui Ieroboam în cetate, pe acela câinii îl vor mânca; cine va muri în câmp, pe acela păsările cerului îl vor ciuguli, pentru că așa a grăit Domnul.
\par 12 Scoală-te dar, și du-te la casa ta; și îndată ce piciorul tău va călca în cetate, copilul va muri.
\par 13 Și-l vor plânge ton Israeliții și-l vor înmormânta; căci el singur dintre acei care sunt ai lui Ieroboam va fi îngropat în mormânt, fiindcă numai în el, din casa lui Ieroboam, s-a găsit ceva bun înaintea Domnului Dumnezeului lui Israel.
\par 14 Și își va așeza Domnul un rege peste Israel care va nimici casa lui Ieroboam în acea zi. Dar când? Chiar acum începe necazul.
\par 15 Și va lovi Domnul pe Israel și va fi el ca trestia, care se clatină în apă, și va azvârli pe Israeliți din acest pământ bun pe care l-a dat părinților lor și-i va spulbera dincolo de Eufrat, pentru că ei și-au făcut dumbrăvi închinate Astartei, întărâtând la mânie pe Domnul.
\par 16 Domnul va vinde pe Israel pentru păcatele lui Ieroboam pe care el însuși le-a făcut și cu care a dus la păcat pe Israel".
\par 17 Atunci s-a sculat femeia lui Ieroboam și a plecat și a venit la Tirța și când a pășit ea cu piciorul peste pragul casei, copilul a murit.
\par 18 Și l-au înmormântat și l-au jelit toți Israeliții, după cuvântul Domnului, încredințat robului Său Ahia proorocul.
\par 19 Celelalte fapte ale lui Ieroboam, cum a purtat războaie și cum a domnit, sunt scrise în cronica regilor lui Israel.
\par 20 Timpul domniei lui Ieroboam a fost de douăzeci și doi de ani; și a adormit cu părinții lui, iar în locul lui s-a făcut rege Nadab, fiul lui.
\par 21 Iar în Iuda domnea Roboam, fiul lui Solomon. Roboam, când s-a făcut rege, era ca de patruzeci și unu de ani; și a domnit șaptesprezece ani în Ierusalim, în cetatea pe care a ales-o Domnul din toate semințiile lui Israel, ca să petreacă numele Lui acolo. Pe mama lui o chema Naama Amonita.
\par 22 Și a făcut și Iuda lucruri foarte rele înaintea ochilor Domnului și L-au mâniat cu păcatele lor, pe care le-au făcut mai mult decât părinții lor;
\par 23 Căci și aceștia și-au făcut înălțimi, idoli și Astarte, pe orice deal înalt și sub orice copac umbros.
\par 24 Și erau de asemenea și sodomiți în această țară, și făceau toate ticăloșiile păgânilor, pe care Domnul îi gonise din fața fiilor lui Israel.
\par 25 Iar în anul al cincilea al domniei lui Roboam, Șișac, regele Egiptului, a pornit război asupra Ierusalimului,
\par 26 Și a luat vistieriile templului Domnului și vistieriile casei regelui și scuturile cele de aur pe care le luase David din mâna robilor lui Hadad Ezer, regele din Țoba, și le adusese în Ierusalim; toate le-a luat și a prădat toate scuturile de aur pe care le făcuse Solomon.
\par 27 Și a făcut regele Roboam în locul lor scuturi de aramă și le-a dat în mâinile căpitanilor de gărzi, care păzeau intrarea casei regelui.
\par 28 Numai când regele mergea la templul Domnului, numai atunci garda le purta; și pe urmă le ducea iarăși în casa de gardă.
\par 29 Celelalte fapte ale lui Roboam și tot ce el a făcut sunt scrise în cronica regilor lui Iuda.
\par 30 Între Roboam și Ieroboam a fost război în toate zilele vieții lor.
\par 31 Și a adormit cu părinții lui și a fost înmormântat Roboam la un loc cu părinții lui în cetatea lui David. Pe mama lui o chema Naama Amonita. Iar în locul lui s-a făcut rege Abia, fiul său.

\chapter{15}

\par 1 Abia s-a făcut rege peste Iuda, în anul al optsprezecelea al domniei lui Ieroboam, fiul lui Nabat.
\par 2 Și a domnit trei ani în Ierusalim. Pe mama lui o chema Maaca și era fiica lui Abesalom.
\par 3 Acesta a umblat în toate păcatele tatălui său, pe care tatăl său le făcuse mai înainte, și inima lui nu i-a fost întreagă la Domnul Dumnezeul lui, cum a fost inima lui David, strămoșul său.
\par 4 Dar din dragostea către David, Domnul Dumnezeul lui i-a dat o lumină în Ierusalim, căci a înălțat după el pe fiul său și a întărit Ierusalimul,
\par 5 Pentru că David a făcut tot ce a fost plăcut înaintea ochilor Domnului și nu s-a abătut în toate zilele vieții lui de la poruncile pe care El i le-a dat, afară de fapta cu Urie Heteul.
\par 6 Și a fost război între Roboam și Ieroboam în toate zilele vieții lor.
\par 7 Celelalte fapte ale lui Abia și tot ce a făcut el sunt scrise în cartea cronicilor regilor lui Iuda. Și a fost război între Abia și Ieroboam.
\par 8 Apoi a trecut Abia la părinții lui și l-au înmormântat în cetatea lui David. și s-a făcut rege Asa, fiul său.
\par 9 Asa a început să domnească peste Iuda în al douăzecilea an al domniei lui Ieroboam, regele lui Israel.
\par 10 Și a domnit patruzeci și unu de ani în Ierusalim. Pe mama lui o chema Ana, din neamul lui Abesalom.
\par 11 Asa a făcut lucruri plăcute înaintea ochilor Domnului, ca David, strămoșul lui;
\par 12 Căci a izgonit pe desfrânați din țară și a îndepărtat toți idolii pe care îi făcuseră părinții lui;
\par 13 Ba a lipsit chiar pe mama lui, Ana, de numele de regină, pentru că ea făcuse un chip turnat Astartei. Așa a sfărâmat Asa chipul cel turnat al Astartei arzându-l pe prundul pârâului Chedron.
\par 14 Dar înălțimile nu le-a stricat. Însă inima lui Asa i-a fost întreagă la Domnul în toate zilele vieții sale.
\par 15 Și a adus el în templul Domnului lucrurile afierosite de tatăl lui și lucrurile afierosite de el: argint, aur și vase sfinte.
\par 16 Război a fost între Asa și Baeșa, regele lui Israel, în toate zilele lor.
\par 17 Căci Baeșa, regele lui Israel, a venit în contra Iudei și a început să zidească Rama, pentru ca nimeni să nu iasă sau să intre la Asa, regele Iudei.
\par 18 Atunci Asa a luat tot argintul și aurul care se mai afla în vistieriile templului Domnului și în vistieriile casei regelui și le-a dat în mina slugilor lui; și i-a trimis regele Asa ca să se ducă cu ele la Benhadad, fiul lui Tabrimon, fiul lui Hezion, regele Siriei, care trăia în Damasc, și a zis:
\par 19 "Legământ să fie între mine și tine, cum a fost între tatăl meu și tatăl tău; iată, eu îți trimit ca dar argint și aur; și tu să rupi legământul tău care îl ai cu Baeșa, regele lui Israel, ca să se retragă acela de la mine!"
\par 20 Și a ascultat Benhadad de regele Asa și a trimis pe mai-marii oștirii lui asupra cetăților lui Israel, și a bătut Ainul, Danul, Abel, Bet-Maaca și tot Chineretul, dimpreună cu tot pământul lui Neftali.
\par 21 Iar Baeșa, cum a auzit de aceasta, a încetat să mai zidească Rama și s-a întors la Tirța.
\par 22 Atunci regele Asa a chemat pe toți Iudeii, fără să scutească pe nimeni, și a cărat cu ei pietrăria și lemnăria din Rama, pe care Baeșa le întrebuințase la zidire; și a zidit regele Asa cu ele Ghibeea lui Veniamin și Mițpa.
\par 23 Toate celelalte fapte ale lui Asa, și toate ostenelile lui, și tot ce el a făcut și cetățile pe care le-a zidit sunt scrise în cronica regilor lui Iuda, afară de faptul că la bătrânețea lui s fost bolnav de picioare.
\par 24 Apoi a adormit cu părinții lui și a fost înmormântat Asa la un loc cu părinții lui în cetatea lui David, strămoșul său. Iar în locul lui s-a făcut rege Iosafat, fiul său.
\par 25 Nadab însă, fiul lui Ieroboam a început să domnească peste Israeliți în anul al doilea al lui Asa, regele Iudei, și a domnit peste Israel doi ani.
\par 26 Acesta a săvârșit fapte neplăcute înaintea ochilor Domnului, căci a umblat pe drumul tatălui său, cum și în păcatele lui cu care a făcut pe Israel să păcătuiască.
\par 27 Împotriva lui a uneltit Baeșa, fiul lui Ahia, din casa lui Isahar; Baeșa l-a omorât la Ghibeton, cetatea Filistenilor, pe când Nadab și toți Israeliții împresuraseră Ghibetonul.
\par 28 Baeșa l-a omorât în anul al treilea al lui Asa, regele Iudei, și s-a făcut rege în locul lui.
\par 29 Acesta, cum s-a făcut rege, a stârpit toată casa lui Ieroboam și n-a lăsat nici un suflet din neamul lui Ieroboam, până ce nu l-a nimicit, după cuvântul Domnului, pe care-l grăise prin robul Său Ahia din Șilo,
\par 30 Din pricina păcatelor pe care Ieroboam le făcuse și cu care a făcut pe Israel să păcătuiască și pentru fărădelegea cu care el a mâniat pe Domnul Dumnezeul lui Israel.
\par 31 Celelalte fapte ale lui Nadab și tot ce el a făcut sunt scrise în cronica regilor lui Israel.
\par 32 Între Asa și Baeșa, regele lui Israel, a fost război în toate zilele vieții lor.
\par 33 Baeșa, fiul lui Ahia, a început să domnească peste toți Israeliții la Tirța, în anul al treilea al lui Asa, regele lui Iuda, și a domnit douăzeci și patru de ani.
\par 34 Acesta a săvârșit fapte rele înaintea ochilor Domnului căci a umblat pe drumul lui Ieroboam și în păcatele lui, cu care acesta a dus pe Israel în păcat.

\chapter{16}

\par 1 Atunci a fost cuvântul Domnului către Iehu, fiul lui Hanani, pentru Baeșa, zicând:
\par 2 "Pentru că Eu te-am scuturat de țărână și te-am făcut domn peste poporul Meu Israel, iar tu ai umblat pe drumul lui Ieroboam și ai dus pe poporul Meu Israelit la păcat, ca să Mă mânii cu păcatele lor,
\par 3 Iată, Eu voi dărâma de tot casa lui Baeșa și casa urmașilor lui și voi face cu casa ta ce am făcut cu casa lui Ieroboam, fiul lui Nabat;
\par 4 Cine va muri din neamul lui Baeșa în cetate, pe acela câinii îl vor mânca, iar cine va muri în câmp, pe acela păsările cerului îl vor mânca".
\par 5 Celelalte fapte ale lui Baeșa, tot ce a făcut el și războaiele lui, sunt scrise în cronica regilor lui Israel.
\par 6 Și a adormit cu părinții săi și a fost înmormântat Baeșa în Tiria. În locul lui s-a făcut rege Ela, fiul său.
\par 7 Dar despre Baeșa, despre casa lui și despre tot răul ce l-a făcut înaintea ochilor Domnului, mâniindu-L cu faptele mâinilor sale și urmând casei lui Ieroboam, pentru care acesta a și fost stârpit, fusese cuvântul Domnului prin Iehu, fiul lui Hanani.
\par 8 Ela, fiul lui Baeșa, a început să domnească peste Israel în Tirța, în anul al douăzeci și șaselea al lui Asa, regele Iudei, și a domnit doi ani.
\par 9 Împotriva lui a uneltit robul său Zimri, care era mai mare peste jumătate din carele de război.
\par 10 Când a băut acesta de s-a îmbătat la Tirța, în casa lui Arța, mai marele curții din Tirța, atunci a intrat Zimri, l-a lovit și l-a omorât, în anul al douăzeci și șaptelea al lui Asa, regele Iudei, și s-a făcut rege în locul lui.
\par 11 Acesta, cum s-a făcut rege și s-a așezat pe tronul lui, a stârpit toată casa lui Baeșa, nelăsând nici picior de bărbat, nici din rudele lui, nici din prietenii lui.
\par 12 Și a stârpit Zimri toată casa lui Baeșa, după cuvântul Domnului grăit pentru Baeșa, prin Iehu proorocul,
\par 13 Pentru toate fărădelegile lui Baeșa și pentru cele ale lui Ela, fiul său, pe care le-a săvârșit, făcând pe Israel să păcătuiască, și mâniind cu idolii lor pe Domnul Dumnezeul lui Israel.
\par 14 Celelalte fapte ale lui Ela, tot ce a făcut el sunt scrise în cronica regilor lui Israel.
\par 15 Zimri s-a făcut rege în anul al douăzeci și șaptelea al lui Asa, regele Iudei, și a domnit șapte zile în Tiria, când poporul se găsea în tabere la Ghibeton, cetatea Filistenilor.
\par 16 Când poporul care se găsea în tabere a auzit că Zimri a uneltit și a omorât pe rege, toți Israeliții au ales și ei rege pe Omri, mai-marele oștirii lui Israel, tot în aceeași zi, în tabără.
\par 17 Și a plecat Omri cu toți Israeliții de la Ghibeton, împresurând Tiria.
\par 18 Când a auzit Zimri că cetatea este luată, s-a dus în odaia din fund a casei domnești și a dat foc casei domnești, în care era, și a pierit
\par 19 În păcatele sale, căci săvârșise fapte rele înaintea ochilor Domnului, umblând pe calea lui Ieroboam și în fărădelegile lui, pe care acela le făptuise și cu care a dus pe Israel la păcat.
\par 20 Celelalte fapte ale lui Zimri și uneltirile lui pe care le-a urzit sunt scrise în cronica regilor lui Israel.
\par 21 Atunci s-a împărțit poporul lui Israel în două: jumătate de popor era pentru Tibni, fiul lui Ghinat, ca să-l facă rege și jumătate pentru Omri.
\par 22 Și cei care urmau pe Omri au biruit pe cei care urmau pe Tibni, fiul lui Ghinat. Și a murit Tibni și Ioram, fratele lui, tot în acel timp; în locul lui Tibni s-a făcut rege Omri.
\par 23 Omri a început să domnească peste Israel în anul treizeci și unu al lui Asa, regele Iudei; și a domnit doisprezece ani. Șase ani a stat în Tirța.
\par 24 Și a cumpărat Omri muntele Samariei de la Șemer, cu doi talanți de argint și a zidit pe acest munte o cetate și a numit cetatea pe care o zidise Samaria, după numele lui Șemer, stăpânul muntelui Samariei.
\par 25 Omri a săvârșit fapte rele înaintea ochilor Domnului și a fost mai nelegiuit decât toți cei dinaintea lui.
\par 26 El a umblat în totul pe căile lui Ieroboam, fiul lui Nabat și întru fărădelegile eu care acesta a dus pe Israel la păcat, mâniind pe Domnul Dumnezeul lui Israel cu idolii lor.
\par 27 Celelalte fapte ale lui Omri, pe care le-a făcut și vitejia în războaie sunt scrise în cronica regilor lui Israel.
\par 28 Și a adormit Omri cu părinții lui și a fost înmormântat în Samaria. În locul lui s-a făcut rege Ahab, fiul său.
\par 29 Ahab, fiul lui Omri, a început să domnească peste Israel în anul al treizeci și optulea al lui Asa, regele lui Iuda; și a domnit Ahab, fiul lui Omri, peste Israel în Samaria douăzeci și doi de ani.
\par 30 Și a săvârșit Ahab, fiul lui Omri, fapte rele înaintea ochilor Domnului, mai mult decât toți cei ce au fost înaintea lui.
\par 31 Căci nu i-a fost de ajuns să cadă numai în păcatele lui Ieroboam, fiul lui Nabat; ci dacă și-a luat de femeie pe Izabela, fiica lui Etbaal, regele Sidonului, a început să slujească lui Baal și să i se închine.
\par 32 Și a ridicat pentru Baal un jertfelnic în templul lui Baal, pe care îl zidise în Samaria.
\par 33 A făcut Ahab și o Așeră (stâlp făcut din lemn, sfințit în cinstea zeiței Astarte), încât Ahab, mai mult decât toți regii lui Israel, care au fost înaintea lui, a săvârșit fărădelegi, prin care a mâniat pe Domnul Dumnezeul lui Israel și și-a pierdut suflatul său.
\par 34 În zilele lui, Hiel din Betel a zidit Ierihonul; temelia a pus-o pe mormântul lui Abiram, întâiul născut al lui, iar porțile le-a pus pe mormântul lui Segub, feciorul cel mai mic al lui, după cuvântul Domnului pe care-l grăise prin Iosua, fiul lui Navi.

\chapter{17}

\par 1 Atunci Ilie Tesviteanul, prooroc din Tesba Galaadului, a zis către Ahab: "Viu este Domnul Dumnezeul lui Israel, înaintea Căruia slujesc au; în acești ani nu va fi nici rouă, nici ploaie decât numai când voi zice eu!"
\par 2 Și a zis Domnul către Ilie:
\par 3 "Du-te de aici, îndreaptă-te spre răsărit și te ascunde la pârâul Cherit, care este în fața Iordanului.
\par 4 Apă vei bea din acel pârâu, iar mâncare am poruncit corbilor să-ți aducă acolo!"
\par 5 Și a plecat Ilie și a făcut după cuvântul Domnului; s-a dus și a șezut la pârâul Cherit, care este în fața Iordanului.
\par 6 Corbii îi aduceau pâine și carne dimineața, pâine și carne seara; iar apă bea din pârâu.
\par 7 După o vreme pârâul a secat, nemaifiind ploaie pe pământ.
\par 8 Atunci a fost cuvântul Domnului către Ilie, zicând:
\par 9 "Scoală și du-te la Sarepta Sidonului și șezi acolo, căci iată am poruncit unei femei văduve să te hrănească!"
\par 10 Și s-a sculat el și s-a dus la Sarepta. Și când a ajuns la porțile cetății, iată o femeie văduvă aduna vreascuri și a chemat-o Ilie și i-a zis: "Adu-mi puțină apă ca să beau! "
\par 11 Și s-a dus ca să-i aducă, dar Ilie a strigat-o și i-a zis: "Adu-mi și o bucată de pâine să mănânc!"
\par 12 Ea însă a zis: "Viu este Domnul Dumnezeul tău, n-am nici o fărâmitură de pâine, ci numai o mână de făină într-un vas și puțin untdelemn într-un urcior. Și iată, am adunat câteva vreascuri și mă duc să o gătesc pentru mine și pentru fiul meu și apoi să mâncăm și să murim!"
\par 13 Atunci i-a zis Ilie: "Nu te teme, ci du-te și fă cum ai zis; dar fă mai întâi de acolo o turtă pentru mine și adu-mi-o, iar pentru tine și pentru fiul tău vei face mai pe urmă.
\par 14 Căci așa zice Domnul Dumnezeul lui Israel: Făina din vas nu va scădea și untdelemnul din urcior nu se va împuțina până în ziua când va da Domnul ploaie pe pământ!"
\par 15 Și s-a dus ea și a făcut așa, cum i-a zis Ilie; și s-a hrănit ea și el și casa ei o bucată de vreme.
\par 16 Căci făina din vas n-a scăzut și untdelemnul din urcior nu s-a împuținat, după cuvântul Domnului, grăit prin Ilie.
\par 17 După aceasta s-a îmbolnăvit copilul femeii, stăpâna casei, și boala lui a fost atât de grea, că n-a mai rămas suflare într-însul.
\par 18 Și a zis ea către Ilie: "Ce ai avut cu mine, omul lui Dumnezeu? Ai venit la mine ca să-mi pomenești păcatele mele și să-mi omori fiul?"
\par 19 Iar Ilie a zis: "Dă-mi pe fiul tău!" Și l-a luat din brațele ei și l-a suit în foișor unde ședea el și l-a pus pe patul său.
\par 20 Apoi a strigat Ilie către Domnul și a zis: "Doamne Dumnezeul meu, oare și văduvei la care locuiesc îi faci rău, omorând pe fiul ei?"
\par 21 Și suflând de trei ori peste copil, a strigat către Domnul și a zis: "Doamne Dumnezeul meu, să se întoarcă sufletul acestui copil în el!"
\par 22 Și a ascultat Domnul glasul lui Ilie; și s-a întors sufletul copilului acestuia în el și a înviat.
\par 23 Și a luat Ilie copilul și s-a coborât cu el din foișor în casă și l-a dat mamei sale și a zis Ilie: "Iată copilul tău este viu!"
\par 24 Atunci a zis femeia către Ilie: "Acum cunosc și eu că tu ești omul lui Dumnezeu și cu adevărat cuvântul lui Dumnezeu este în gura ta!"

\chapter{18}

\par 1 După trecere de mai multe zile, a fost cuvântul Domnului către Ilie în anul al treilea, zicând: "Du-te și te arată lui Ahab și Eu voi da ploaie pe pământ!"
\par 2 Și a plecat Ilie să se arate lui Ahab, în Samaria fiind foamete mare.
\par 3 Atunci a chemat Ahab pe Obadia, mai-marele curții și Obadia era un om foarte temător de Dumnezeu;
\par 4 Căci când Izabela a ucis pe proorocii Domnului, Obadia a luat o sută de prooroci și i-a ascuns: cincizeci într-o peșteră și cincizeci în alta și i-a hrănit cu pâine și cu apă.
\par 5 Și a zis Ahab către Obadia: "Du-te prin țară la toate izvoarele de apă și pe la toate pâraiele, poate să găsim undeva iarbă, ca să ne hrănim caii, catârii și să nu prăpădim vitele".
\par 6 Atunci și-au împărțit între ei țara ca s-o cutreiere: Ahab a apucat singur pe un drum, iar Obadia a apucat singur pe altul.
\par 7 Când Obadia mergea pe drum, iată i-a ieșit înainte Ilie. Obadia l-a cunoscut și a căzut cu fața la pământ, zicând: "Oare tu ești domnul meu, Ilie?"
\par 8 Iar acela i-a zis: "Eu sunt; du-te și spune domnului tău: Ilie este aici!"
\par 9 Obadia însă a zis: "Cu ce am greșit eu de dai pe robul tău pe mâna lui Ahab, ca să mă omoare?
\par 10 Viu este Domnul Dumnezeul tău! Nu este popor și împărăție la care să nu fi trimis domnul meu ca să te caute. Și când i s-a spus că nu ești acolo, a pus pe acea împărăție și pe acel popor să jure că nu te-a găsit pe tine.
\par 11 Iar acum zici: Du-te și spune domnului tău: Ilie este aici!
\par 12 Gând voi pleca de la tine, Duhul Domnului are să te ducă cine știe unde și dacă mă voi duce și voi spune lui Ahab de tine și apoi el nu te va găsi, mă va ucide pe mine; iar robul tău este temător de Dumnezeu din tinerețile mele.
\par 13 Oare nu ți s-a spus, domnul meu, ce am făcut eu când Izabela a ucis pe proorocii Domnului, în ce chip am ascuns o sută de oameni, prooroci ai Domnului: cincizeci de inși într-o peșteră și cincizeci de inși în alta și acolo i-am hrănit cu pâine și cu apă?
\par 14 Tu zici: "Du-te, spune domnului tău: "Ilie este aici". "Vrei să mă ucidă?"
\par 15 Și a zis Ilie: "Viu este Domnul Savaot Căruia Îi slujesc. Astăzi mă voi arăta lui Ahab!"
\par 16 Și a plecat Obadia în întâmpinarea lui Ahab și i-a spus de aceasta. Și a venit Ahab în întâmpinarea lui Ilie.
\par 17 Când a văzut Ahab pe Ilie, i-a zis: "Tu ești oare cel ce aduci nenorociri peste Israel?"
\par 18 Iar Ilie a zis: "Nu eu sunt cel ce aduce nenorocire peste Israel; ci tu și casa tatălui tău, pentru că ați părăsit poruncile Domnului și mergeți după baali.
\par 19 Trimite dar acum și adună la mine în muntele Carmel tot Israelul, dimpreună cu cei patru sute cincizeci de prooroci ai lui Baal și cu cei patru sute de prooroci ai Așerei, care mănâncă la masa Izabelei".
\par 20 Și a trimis Ahab pe toți fiii lui Israel și a luat pe toți proorocii în muntele Carmel.
\par 21 Atunci s-a apropiat Ilie de tot poporul și a zis: "Până când veți șchiopăta de amândouă picioarele? Dacă Domnul este Dumnezeu, urmați Lui! Și dacă este Baal, urmați aceluia". Poporul însă n-a răspuns nimic.
\par 22 Și a zis Ilie către popor: "Prooroc al Domnului am rămas numai eu singur, iar prooroci ai lui Baal sunt patru sute cincizeci de oameni și prooroci ai Așerei patru sute.
\par 23 Dați-ne doi viței; el să-și aleagă unul, să-l taie bucăți și să-l pună pe lemne, dar foc să nu facă, iar eu voi tăia bucăți pe celălalt vițel și-l voi pune pe lemne și foc nu voi face.
\par 24 Apoi voi să chemați numele dumnezeului vostru, iar eu voi chema numele Domnului Dumnezeului meu. Și Dumnezeul Care va răspunde cu foc, Acela este Dumnezeu". Și a răspuns tot poporul: "Bine ai grăit!"
\par 25 Și a zis Ilie proorocilor lui Baal: "Să vă alegeți un vițel și să-l pregătiți voi înainte, căci sunteți mai mulți și să chemați numele dumnezeului vostru, dar foc să nu faceți".
\par 26 Și au luat ei vițelul care li s-a dat și l-au pregătit și au chemat numele lui Baal de dimineață până la amiază, zicând: "Baale, auzi-ne!" Dar n-a fost nici glas, nici răspuns. Și săreau împrejurul jertfelnicului pe care-l făcuseră.
\par 27 Iar pe la amiază, Ilie a început să râdă de ei și zicea: "Strigați mai tare, căci doar este dumnezeu! Poate stă de vorbă cu cineva, sau se îndeletnicește cu ceva, sau este în călătorie, sau poate doarme; strigați tare să se trezească!"
\par 28 Și ei strigau cu glas mai tare și se înțepau după obiceiul lor cu săbii și cu lănci, până ce curgea sânge.
\par 29 Trecuse acum de amiază și ei s-au zbuciumat mereu până la timpul jertfei. Dar n-a fost nici glas, nici răspuns, nici auzire. Atunci a zis Ilie Tesviteanul către proorocii lui Baal: "Dați-vă acum la o parte, ca să-mi săvârșesc și eu jertfa mea!" și s-au dat la o parte.
\par 30 Atunci a zis Ilie către popor: "Apropiați-vă de mine!" Și s-a apropiat tot poporul de el. Și a făcut jertfelnicul Domnului ce fusese dărâmat;
\par 31 A luat Ilie douăsprezece pietre, după numărul semințiilor fiilor lui Iacov, către care a zis Domnul: "Israel va fi numele tău!"
\par 32 Și a zidit din pietrele acelea jertfelnicul în numele Domnului, făcând împrejurul jertfelnicului șanț în care încăpeau două măsuri de sămânță,
\par 33 A așezat lemnele pe jertfelnic, a tăiat vițelul bucăți și le-a pus pe el.
\par 34 Și a zis: "Umpleți patru cofe cu apă și le turnați peste jertfa arderii de tot și peste lemne!" Și au făcut așa. Apoi a zis: "Faceți aceasta a doua oară!" Și au făcut la fel a doua oară. Și a zis: "Faceți așa și a treia oară!"
\par 35 Și umbla apa împrejurul jertfelnicului și șanțul se umpluse de apă.
\par 36 Iar la vremea jertfei de seară, s-a apropiat Ilie proorocul și a strigat la cer și a zis: "Doamne Dumnezeul lui Avraam, al lui Isaac și al lui Israel! Auzi-mă Doamne, auzi-mă acum cu foc, ca să cunoască astăzi poporul acesta că Tu singur ești Dumnezeu în Israel și că eu sunt slujitorul Tău.
\par 37 Auzi-mă, Doamne, auzi-mă, ca să cunoască poporul acesta că Tu Doamne ești Dumnezeu și că Tu le întorci inima la Tine!"
\par 38 și s-a pogorât foc de la Domnul și a mistuit arderea de tot și lemnele și pietrele și țărâna și a mistuit și toată apa care era în șanț.
\par 39 Și tot poporul, când a văzut aceasta, a căzut cu fața la pământ și a zis  "Domnul este Dumnezeu, Domnul este Dumnezeu!"
\par 40 Iar Ilie le-a zis: "Prindeți pe proorocii lui Baal, ca să nu scape nici unul din ei!" Și i-au prins, și s-a dus Ilie la pârâul Chișonului și i-a junghiat acolo.
\par 41 Apoi a zis Ilie către Ahab: "Du-te de mănâncă și bea, căci se aude vuiet de ploaie!"
\par 42 Și a plocat Ahab să mănânce și să bea; iar Ilie s-a suit în vârful Carmelului și s-a aplecat ta pământ până a atins genunchii au fața sa.
\par 43 Și a zis ucenicului său: "Du-te și te uită spre mare!" Și s-a dus el și s-a uitat și a zis: "Nu văd nimic!" El i-a zis: "Du-te și fă aceasta de șapte ori".
\par 44 Iar a șaptea oară a zis: "Iată se ridică din mare un nor cât o palmă". Iar Ilie a zis: "Du-te și spune lui Ahab: Înhamă și fugi, ca să nu te apuce ploaia!"
\par 45 Și pe când grăbea el, cerul s-a întunecat de nori și s-a pornit vijelie și ploaie mare. Iar Ahab, suindu-se în căruță, s-a dus la Izreel.
\par 46 Iar mâna Domnului a fost peste Ilie, care, încingându-și mijlocul, a alergat înaintea lui Ahab, până la Izreel.

\chapter{19}

\par 1 Și a spus Ahab Izabelei. tot ce a făcut Ilie și cum a ucis pe toți proorocii cu sabia.
\par 2 Atunci a trimis Izabela un vestitor la Ilie, ca să-i spună: "Dacă tu ești Ilie și eu Izabela, așa și așa să-mi facă dumnezeii, ba încă și mai mult, dacă mâine pe vremea aceasta nu voi face cu viața ta la fel cum ai făcut și tu cu fiecare din ei!"
\par 3 Când a auzit Ilie aceasta, s-a sculat și a plecat, să-și scape viața sa, și a venit la Beer-Șeba, care este în Iuda, și și-a lăsat ucenicul acolo,
\par 4 Iar el s-a dus mai departe în pustiu, cale de o zi, și s-a așezat sub un ienupăr și își ruga moartea, zicând: "Îmi ajunge acum, Doamne! Ia-mi sufletul că nu sunt eu mai bun decât părinții mei!"
\par 5 Și s-a culcat și a adormit acolo sub ienupăr. Și iată un înger l-a atins și i-a zis: "Scoală de mănâncă și bea!"
\par 6 Și a căutat Ilie și iată, la căpătâiul lui, o turtă coaptă în vatră și un urcior cu apă. Și a mâncat și a băut și a adormit iar.
\par 7 Dar iată îngerul Domnului s-a întors a doua oară, s-a atins de el și a zis: "Scoată de mănâncă și bea, că lungă-ți este calea!"
\par 8 Și s-a sculat Ilie și a mâncat și a bătut și întărindu-se cu acea mâncare, a mers patruzeci de zile și patruzeci  de nopți, până la Horeb, muntele lui Dumnezeu.
\par 9 Și a intrat acolo într-o peșteră și a rămas acolo. Și iată cuvântul Domnului a fost către el și i-a zis: "Ce faci aici, Ilie?"
\par 10 Iar Ilie a zis: "Cu râvnă am râvnit pentru Domnul Dumnezeul Savaot, căci fiii lui Israel au părăsit legământul Tău, au dărâmat jertfelnicele Tale și pe proorocii Tăi i-au ucis cu sabia, rămânând numai eu singur, dar caută să ia și sufletul meu!"
\par 11 A zis Domnul: "Ieși și stai pe munte înaintea feței Domnului! Că iată Domnul va trece; și înaintea Lui va fi vijelie năprasnică ce va despica munții și va sfărâma stâncile, dar Domnul nu va fi în vijelie. După vijelie va fi cutremur, dar Domnul nu va fi în cutremur;
\par 12 După cutremur va fi foc, dar nici în foc nu va fi Domnul. Iar după foc va fi adiere de vânt lin și acolo va fi Domnul".
\par 13 Auzind aceasta, Ilie și-a acoperit fața cu mantia lui și a ieșit și a stat la gura peșterii. Și a fost către el un glas care i-a zis: "Ce faci aici, Ilie?"
\par 14 Iar el a zis: "Cu râvnă am râvnit pentru Domnul Dumnezeul Savaot, că au părăsit fiii lui Israel legământul Tău, au dărâmat jertfelnicele Tale și pe proorocii Tăi i-au ucis cu sabia; numai eu singur am rămas, dar caută să ia și sufletul meu!"
\par 15 Și a zis Domnul: "Mergi și întoarce-te pe calea ta prin pustiu la Damasc și, când vei ajunge acolo, să ungi rege peste Siria pe Hazael;
\par 16 Pe Iehu, fiul lui Nimși, să-l ungi rege peste Israel; iar pe Elisei, fiul lui Șafat din Abel-Mehola, să-l ungi prooroc în locul tău!
\par 17 Cine va fugi de sabia lui Hazael, pe acela să-l omoare Iehu, iar cine va scăpa de sabia lui Iehu, pe acela să-l omoare Elisei.
\par 18 Eu însă mi-am oprit dintre Israeliți șapte mii de bărbați; genunchii tuturor acestora nu s-au plecat înaintea lui Baal și buzele tuturor acestora nu l-au sărutat!"
\par 19 Atunci a plecat Ilie de acolo și a găsit pe Elisei, fiul lui Șafat, arând; acesta avea douăsprezece perechi de boi la pluguri și la perechea a douăsprezecea era el însuși. Și Ilie a trecut pe lângă el aruncându-i mantia.
\par 20 Atunci a lăsat Elisei boii și a alergat după Ilie, zicând: "Lasă-mă să merg să sărut pe tatăl și pe mama mea și voi veni după tine!" Iar el i-a zis: "Du-te și vino înapoi, că ce-am făcut e făcut!"
\par 21 Plecând de la el, a luat o pereche de boi pe care i-a junghiat și, făcând foc cu plugul boilor, a fript carnea lor și au împărțit-o la oameni și au mâncat-o. Iar el s-a sculat și s-a dus după Ilie și a început să-i slujească.

\chapter{20}

\par 1 Benhadad, regele Siriei, și-a adunat oștirea sa; și s-au însoțit cu el treizeci și doi de regi, cu cai și care de război. Și s-a dus și a împresurat Samaria și s-a războit împotriva ei.
\par 2 Atunci a trimis soli la Ahab, regele lui Israel, în cetate și a zis:
\par 3 "Așa zice Benhadad: Argintul tău și aurul tău este al meu, femeile tale și fiii tăi cei frumoși sunt ai mei!"
\par 4 Iar regele lui Israel a răspuns și a zis: "După cuvântul tău să fie, domnul meu rege. Eu și toate ale mele ale tale să fie!"
\par 5 Și s-au întors solii și au zis: "Așa zice Benhadad: Am trimis la tine ca să-ți spună: Argintul și aurul tău, femeile și fiii tăi să mi le dai!
\par 6 De aceea mâine, pe vremea aceasta, voi trimite robii mei la tine, iar ei îți vor scotoci casa ta și casele slugilor tale și pe tot ce este mai scump în ochii tăi vor pune mâna și vor lua!"
\par 7 Atunci a chemat regele lui Israel pe toți bătrânii țării și a zis: "Luați seama și vedeți că îmi caută mereu pricină; când a trimis la mine pentru femeile mele și pentru fiii mei și pentru argintul meu și pentru aurul meu, eu nu l-am respins!"
\par 8 Și au zis toți bătrânii și tot poporul: "Să nu-l asculți, nici să nu te învoiești!"
\par 9 Și a zis el solilor lui Benhadad: "Spuneți domnului meu, regele: Toate lucrurile pentru care ai trimis la robul tău ia început sunt gata să le fac; dar acest lucru nu pot să-l fac!" Și au plecat solii și au dus răspunsul.
\par 10 Și a trimis Benhadad la el, ca să-i spună: "Asta și asta să-mi facă mie dumnezeii și așa să mă pedepsească, dacă țărâna Samariei are să ajungă să umple pumnii tuturor oamenilor care au să vină cu mine!"
\par 11 și a răspuns regele lui Israel și a zis: "Spuneți să nu se laude cel ce se încinge ca cel ce se descinge".
\par 12 Când a primit acest răspuns, Benhadad era la ospăț în corturi, cu ceilalți regi. Și a zis robilor săi: "Împresurați cetatea!" Și au împresurat-o.
\par 13 Dar iată un prooroc s-a apropiat de Ahab, regele lui Israel, și a zis: "Așa zice Domnul: Vezi toată mulțimea aceasta mare? Iată, Eu ți-o dau astăzi în mână, ca să cunoști că Eu sunt Domnul.
\par 14 Iar Ahab a zis: "Prin cine?" Răspuns-a acela: "Așa zice Domnul: Prin slugile căpeteniilor de peste ținuturi!" Și a zis Ahab: "Cine va începe lupta?" Iar acela a zis: "Tu!"
\par 15 Atunci a numărat Ahab slugile căpeteniilor de peste ținuturi, și s-au găsit două sute treizeci și două; după ei a numărat și tot poporul și toți fiii lui Israel au fost șapte mii.
\par 16 Și au ieșit la război pe la amiază. Iar Benhadad băuse până se îmbătase în corturi cu cei treizeci și doi de regi, care-i dăduseră ajutor.
\par 17 Și au ieșit la război mai întâi slugile căpeteniilor de peste ținuturi. Și a trimis Benhadad pe unii din ai lui, și aceștia i-au făcut cunoscut că oamenii din Samaria au ieșit la război.
\par 18 Iar el le-a zis: "Dacă ei au venit pentru pace, să-i prindeți de vii; iar dacă au venit pentru război, tot de vii să-i prindeți!"
\par 19 Au ieșit, așadar, la război slugile căpeteniilor de peste ținuturi din cetate și oștirea oare era după ei și au lovit fiecare pe potrivnicul său;
\par 20 și au fugit Sirienii, iar Israeliții îi fugăreau din urmă; și Benhadad a scăpat pe un cal cu călăreții săi.
\par 21 Atunci, ieșind regele israelitean, a luat caii și carele și a făcut măcel mare în rândul Sirienilor.
\par 22 Și s-a apropiat proorocul de regele israelit și i-a zis: "Du-te de te întărește, ca să știi să bagi de seamă ce ai de făcut, pentru că după un an regele Siriei are să se scoale din nou împotriva ta!"
\par 23 Zis-au slugile regelui sirian către el: "Dumnezeul lor este Dumnezeul munților, iar nu Dumnezeul văilor, pentru aceasta au fost ei mai tari decât noi; dar de ne vom bate cu ei în vale, vom fi noi mai tari decât ei.
\par 24 Așadar, iată ce să faci: Să scoți pe fiecare rege de la locul lui, și să-l înlocuiești cu căpeteniile din ținuturi.
\par 25 Și să strângi atâta oștire, câtă oștire ai pierdut, și cai atâția, câți ai avut, și care atâtea, câte ai avut, și să ne batem cu ei și în vale; și atunci de bună seamă vom fi mai tari decât ei". Și a ascultat el cuvântul lor și a făcut așa.
\par 26 După trecerea anului, Benhadad a numărat pe Sirieni și a mers la Afec, ca să se bată cu Israel.
\par 27 Fiii lui Israel însă erau și ei cu toate gata de luptă și au ieșit în întâmpinarea lor. Fiii lui Israel și-au așezat tabăra înaintea lor, ca două turme mici de capre, iar Sirienii umpleau pământul.
\par 28 Atunci s-a apropiat omul lui Dumnezeu și a zis regelui lui Israel: "Așa zice Domnul: Pentru că Sirienii vorbesc: Domnul este Dumnezeul munților, iar nu Dumnezeul văilor, iată Eu îți voi da toată această mulțime în mâna ta, ca să cunoști că Eu sunt Domnul!"
\par 29 Și au stat taberele una în fața celeilalte șapte zile; iar în ziua a șaptea s-a început bătălia și fiii lui Israel au doborât într-o singură zi o sută de mii de pedestrași sirieni.
\par 30 Ceilalți au fugit la Afec, în cetate; acolo a căzut zidul peste douăzeci și șapte de mii de oameni, din cei rămași. Iar Benhadad a fugit în cetate și a intrat la curtea domnească într-o cămară dosnică.
\par 31 Slugile sale însă i-au zis: "Am auzit că regii casei lui Israel sunt regi milostivi; dă-ne voie să punem sac peste coapsele noastre și peste capetele noastre funii, să mergem la regele israelit; poate ne va cruța viața".
\par 32 Și s-au încins ei peste coapsele lor cu sac și au venit la regele lui Israel și i-au zis: "Robul tău Benhadad zice: Cruță-mi viața!" Acela însă a zis: "De mai trăiește încă, e fratele meu!"
\par 33 Și oamenii aceștia au luat aceasta ca semn bun și repede i-au apucat vorba din gură și au zis: "Fratele tău Benhadad trăiește". Iar el a zis: "Duceți-vă de-l aduceți". Și a venit Benhadad la el, și acesta l-a luat cu el în car.
\par 34 Și i-a zis Benhadad: "Cetățile care le-a luat tatăl meu de la tatăl tău, eu ți le întorc; și îți deschid târg de vânzare în Damasc, cum și-a deschis tatăl meu în Samaria". Și a zis Ahab: "Cu acest legământ și eu îți voi da drumul!" Și încheind cu el legământ, i-a dat drumul.
\par 35 Atunci un om din fiii proorocilor a zis către un prieten al său, după cuvântul Domnului: "Bate-mă!" Dar acesta nu s-a învoit să-l bată.
\par 36 Atunci i-a zis: "Pentru că nu asculți de glasul Domnului, iată, când vei pleca de lângă mine, te va ucide un leu". Și plecând omul acela de la el, l-a întâlnit un leu și l-a ucis pe dânsul.
\par 37 Iar cel din fiii proorocilor a găsit un alt om și i-a zis: "Lovește-mă!" Și acest om l-a lovit și l-a rănit.
\par 38 Și s-a dus proorocul în calea regelui, acoperindu-și ochii cu părul capului său.
\par 39 Iar când a trecut regele pe lângă el, a strigat după rege și a zis: "Robul tău a fost la bătălie, și iată un. om care stătuse la o parte mi-a adus un alt om și mi-a zis: Păzește pe omul acesta; de îl vei scăpa, vei răspunde cu sufletul tău pentru sufletul lui sau va trebui să-l plătiți cu un talant de argint.
\par 40 Când robul tău se îndeletnicea cu unele treburi, acela a scăpat". și i-a zis regele lui Israel: "Osânda ta singur ai spus-o".
\par 41 Atunci el și-a dat la o parte părul de peste ochii lui și l-a cunoscut regele că este unul din prooroci.
\par 42 Și i-a zis: "Așa zice Domnul: Pentru că ai dat drumul din mâinile tale unui om blestemat de Mine, sufletul tău va fi în locul sufletului lui și poporul tău în locul poporului lui!"
\par 43 Și s-a dus regele lui Israel acasă, trist și aprins de mânie și a venit în Samaria.

\chapter{21}

\par 1 Iar după aceasta iată ce s-a mai întâmplat: Nabot Izreeliteanul avea în Izreel o vie lângă curtea lui Ahab, regele Samariei.
\par 2 Și a grăit Ahab cu Nabot și a zis: "Să-mi dai mie via ta, ca să-mi fac din ea o grădină de verdețuri, căci e aproape de casa mea; iar în locul ei îți voi da o vie mai bună decât aceasta, sau dacă îți convine mai bine, îți voi da argint cât prețuiește ea".
\par 3 Nabot însă a zis către Ahab: "Să mă ferească Dumnezeu să-ți dau eu moștenirea părinților mei!"
\par 4 Și a venit Ahab acasă trist și mânios pentru cuvântul care i-l spusese Nabot Izreeliteanul, zicând: Nu-îi dau moștenirea părinților mei. Și cu duhul tulburat, s-a culcat în patul său, s-a întors cu fața la perete și n-a mâncat.
\par 5 Atunci a intrat Izabela, femeia lui la el și i-a zis: "De ce este întristat duhul tău și nu mănânci?"
\par 6 Iar el a zis: "Când am vorbit cu Nabot Izreeliteanul și i-am zis: Dă-mi via ta pe bani, sau dacă vrei, să-îi dau altă vie în locul ei, el a zis: Nu-ți dau via mea, că este moștenirea părinților mei".
\par 7 A zis Izabela, femeia lui: "Ce cârmuire ar mai fi în Israel, dacă tu ai face tot așa? Scoală, mănâncă și fii cu voie bună, că via lui Nabot Izreeliteanul ți-o dau eu!"
\par 8 Apoi ea a scris scrisori în numele lui Ahab, le-a pecetluit cu inelul lui și a trimis aceste scrisori la bătrânii și la fruntașii din cetatea lui Nabot, care locuiau cu el acolo.
\par 9 În scrisori însă ea scria așa: "Vestiți tuturor să postească post și puneți pe Nabot înaintea poporului.
\par 10 Și aduceți doi oameni netrebnici să mărturisească împotriva lui și să spună: Ai hulit pe Dumnezeu și pe rege; și apoi să-l scoateți afară și să-l ucideți cu pietre și așa să moară".
\par 11 Și au făcut bărbații cetății lui Nabot, bătrânii și fruntașii care locuiau cu el în cetate, așa cum le poruncise Izabela și cum era scris în scrisorile trimise de ea lor:
\par 12 Au vestit post tuturor și au pus pe Nabot înaintea poporului.
\par 13 Atunci au ieșit doi oameni ticăloși. și au stat împotriva lui Nabot și au mărturisit acești oameni răi în fața poporului și au zis: "Nabot a hulit pe Dumnezeu și pe rege". Iar ei l-au scos afară din cetate și l-au bătut cu pietre și a murit.
\par 14 Apoi au trimis la Izabela să-i spună: "Nabot a fost omorât cu pietre".
\par 15 Auzind Izabela că Nabot a fost bătut eu pietre și a murit, a zis către Ahab: "Scoală și ia în stăpânire via lui Nabot Izreeliteanul, care n-a vrut să ți-o vândă cu bani; că Nabot nu mai trăiește, ci a murit!"
\par 16 Când a auzit Ahab că Nabot Izreeliteanul a fost ucis, și-a rupt hainele sale, s-a îmbrăcat cu sac, și apoi s-a sculat Ahab ca să se ducă la via lui Nabot Izreeliteanul și să o ia în stăpânire.
\par 17 Atunci a fost cuvântul Domnului către Ilie Tesviteanul:
\par 18 "Scoală și ieși în întâmpinarea lui Ahab, regele lui Israel, care este în Samaria, că iată acum e la via lui Nabot, unde s-a dus ca să o ia în stăpânire,
\par 19 Și spune-i: Așa grăiește Domnul: Ai ucis și vrei încă să intri în moștenire? Și să-i mai spui: Așa zice Domnul: În locul unde au lins câinii sângele lui Nabot, acolo vor linge câinii și sângele tău!",
\par 20 Și a zis Ahab către Ilie: "M-ai aflat, dușmanule, și aici!" Iar el a zis: "Te-am aflat, căci te-ai încumetat să săvârșești fapte nelegiuite înaintea ochilor Domnului și să-L minți.
\par 21 Așa zice Domnul: "Iată voi aduce peste tine necazuri și te voi mătura și voi stârpi din ai lui Ahab pe cei de parte bărbătească, fie rob, fie slobod, în Israel;
\par 22 Și voi face cu casa ța cum am făcut cu casa lui Ieroboam, fiul lui Nabat, și cu casa lui Baeșa, fiul lui Ahia, pentru fărădelegea cu care M-ai mâniat și ai dus pe Israel în păcat".
\par 23 Asemenea și pentru Izabela a grăit Domnul: "Câinii vor mânca pe Izabela pe zidul Izreelului.
\par 24 Cine va muri din ai lui Ahab în cetate, pe acela câinii îl vor mânca, iar cine va muri în câmp, pe acela păsările cerului îl vor mânca;
\par 25 Căci n-a fost încă nimeni ca Ahab, care să se încumete a săvârși fapte urâte înaintea ochilor Domnului, la care l-a împins Izabela soția sa.
\par 26 S-a purtat rău ca un ticălos, urmând după idoli, cum au făcut Amoreii pe care Domnul i-a izgonit de la fața fiilor lui Israel".
\par 27 Când a auzit Ahab toate cuvintele acestea, a început să plângă, și-a rupt hainele sale, s-a îmbrăcat pesta trupul său cu sac, a postit și a dormit în sac și a umblat trist.
\par 28 Atunci a fost cuvântul Domnului către Ilie Tesviteanul, pentru Ahab, și a zis Domnul:
\par 29 "Vezi cum s-a smerit Ahab înaintea Mea? Fiindcă s-a smerit înaintea Mea, de aceea nu voi aduce necazurile în zilele lui, ci în zilele fiului lui voi aduce necazurile peste casa lui!"

\chapter{22}

\par 1 Au trecut trei ani fără război între Siria și Israel.
\par 2 În al treilea an s-a dus Iosafat, regele Iudei, la regele lui Israel;
\par 3 Și a zis regele lui Israel către slugile sale: "Știți voi, oare, că Ramotul Galaadului este al nostru, și noi tăcem de atâta vreme și nu-l scoatem din mâna regelui Siriei?"
\par 4 Apoi a zis el lui Iosafat: "Vei merge și tu cu mine la război împotriva Ramot-Galaadului?" Iar Iosafat a zis către regele lui Israel: "Cum ești tu, așa sunt și eu; cum este poporul tău, așa este și poporul meu; cum sunt caii tăi, așa sunt și caii mei!"
\par 5 Și a mai zis Iosafat, regele lui Iuda, către regele lui Israel: "Întreabă dar, astăzi, ce zice Domnul?"
\par 6 Și a adunat regele lui Israel ca la patru sute de prooroci și le-a zis: "Să merg eu, oare, cu război împotriva Ramot-Galaadului sau nu?" Și ei au zis: "Să mergi, că Domnul îl va da în mâinile regelui!"
\par 7 Și a zis Iosafat: "Nu mai este oare aici vreun prooroc al Domnului, ca să întrebăm pe Domnul prin el?"
\par 8 Și a zis regele lui Israel către Iosafat: "Mai este un om prin care se poate întreba Domnul, însă eu nu-l iubesc, căci nu proorocește de bine pentru mine, ci numai de rău; acesta e Miheia, fiul lui Imla". Și a zis Iosafat: "Nu vorbi așa, rege!"
\par 9 Și a chemat regele lui Israel pe un famen și a zis: "Du-te repede după Miheia, fiul lui Imla!"
\par 10 Apoi regele lui Israel și Iosafat, regale Iudei, s-au așezat fiecare în tronul său, îmbrăcați în haine domnești, pe locul dinaintea porții Samariei, și toți proorocii prooroceau înaintea lor.
\par 11 Iar Sedechia, fiul lui Chenaana, și-a făcut niște coarne de fier și a zis: "Așa zice Domnul: cu acestea vei împunge pe Sirieni până ce vor muri".
\par 12 Și toți proorocii au proorocit la fel, zicând: "Să te duci împotriva Ramot-Galaadului, că vei izbuti. Domnul îl va da în mina regelui".
\par 13 Iar trimisul, care s-a dus să cheme pe Miheia, i-a grăit acestuia, zicând: "Iată toți proorocii proorocesc într-un glas de bine regelui; să fie dar și cuvântul tău asemenea cu cuvântul fiecăruia din ei".
\par 14 Iar Miheia a zis: "Viu este Domnul! Ce-mi va spune Domnul, aceea voi grăi!"
\par 15 Apoi a venit el a rege și regale i-a zis: "Miheia, să mai mergem noi oare cu război împotriva Ramot-Galaadului sau nu?" Și i-a zis acela: "Du-te, că vei izbuti. Domnul îl va da în mâna regelui!"
\par 16 Și i-a zis regele: "Iar și iar te jur, ca să nu-mi grăiești nimic, decât ce este adevărat, în numele Domnului".
\par 17 Și a zis el: "Iată, văd pe toți Israeliții împrăștiați prin munți, ca oile ce n-au păstor. Și a zis Domnul: Ei n-au domn, să se întoarcă fiecare cu pace la casa  sa".
\par 18 Atunci regele lui Israel a zis câtre Iosafat, regele Iudei: "Nu ți-am spus eu oare că el nu proorocește de bine pentru mine, ci numai de rău?"
\par 19 Miheia însă a zis: "Nu este așa. Nu eu grăiesc; ascultă cuvântul Domnului. Nu este așa. Am văzut pe Domnul stând pe tronul Său, și toată oștirea cerească sta lângă El, la dreapta, și la stânga Lui".
\par 20 Și a zis Domnul: "Cine ar îndupleca pe Ahab să meargă în Ramot-Galaad și să piară acolo?" Unul spunea una, și altul alta.
\par 21 Atunci a ieșit un duh și a stat înaintea feței Domnului și a zis: "Eu îl voi ademeni". Domnul a zis: "Cum?"
\par 22 Iar acela a zis: "Mă duc și mă fac duh mincinos în gura tuturor proorocilor lui". Domnul a zis: "Tu îl vei ademeni și vei face aceasta; du-te și fă cum ai zis!"
\par 23 Și iată cum a îngăduit Domnul duhului celui mincinos să fie în gura tuturor acestor prooroci ai tăi; însă Domnul n-a grăit bine de tine!"
\par 24 Atunci s-a apropiat Sedechia, fiul lui Chenaana, și, lovind pe Miheia peste obraz, a zis: "Cum? Au doară s-a depărtat Duhul Domnului de la mine, ca să grăiască prin tine?"
\par 25 Iar Miheia a zis: "Iată, ai să vezi aceasta în ziua când vei fugi din cămară în cămară, ca să te ascunzi".
\par 26 A zis regele lui Israel: "Luați pe Miheia și-l duceți la Amon, căpetenia cetății, și la Ioaș, fiul regelui,
\par 27 Și spuneți: Așa zice regele: Aruncați-l în temniță și-l hrăniți numai cu puțină pâine și cu puțină apă, până ce mă voi întoarce biruitor".
\par 28 Iar Miheia a zis: "Că ai să te întorci biruitor, aceasta n-a grăit-o Domnul prin mine". Apoi a zis: "Ascultați, toate popoarele!"
\par 29 După aceea a purces regele lui Israel și Iosafat, regele Iudei, împreună asupra Ramot-Galaadului.
\par 30 Și a zis regele lui Israel către Iosafat: "Eu îmi schimb hainele și intru în luptă, iar tu îmbracă-ți hainele de rege!" Și și-a schimbat hainele regele lui Israel și a intrat în luptă.
\par 31 Regale sirian însă a poruncit celor treizeci și două de căpetenii ale carelor de război și a zis: "Să nu vă luptați nici cu mic, nici cu mare, ci numai eu regele lui Israel".
\par 32 Căpeteniile carelor, văzând pe Iosafat, au crezut că acesta este cu adevărat regele lui Israel și s-au repezit asupra lui, ca să se lupte cu el. Iosafat însă a strigat.
\par 33 Atunci căpeteniile carelor, văzând că nu este acesta regele lui Israel, s-au abătut de la el.
\par 34 Iar un om și-a întins arcul și a lovit din întâmplare pe regele lui Israel într-o încheietură a platoșei, și acesta a zis cărăușului său: "Întoarce îndărăt și mă scoate din oaste, că sunt rănit".
\par 35 Și s-a pornit luptă mare în acea zi și regele a stat În carul lui în fala Sirienilor toată ziua, iar seara a murit; și a curs sânge din rana sa pe podul carului.
\par 36 Și la asfințitul soarelui s-a dat de veste la toată tabăra, zicând: "Să meargă fiecare la cetatea lui, fiecare la ținutul lui!"
\par 37 Și murind regele, a fost dus și l-au înmormântat în Samaria.
\par 38 Și au spălat carul lui în iazul Samariei; și câinii i-au lins sângele lui Ahab, iar desfrânatele s-au scăldat în spălătura acelui sânge, după cuvântul pe care l-a grăit Domnul.
\par 39 Celelalte fapte ale lui Ahab, tot ce a făcut el și casa cea de fildeș pe care a făcut-o el și toate cetățile pe care el le-a zidit sunt scrise în cronica regilor lui Israel.
\par 40 Adormind cu părinții săi, în locul lui Ahab s-a făcut rege Ohozia, fiul său.
\par 41 Iosafat, fiul lui Asa, a început să domnească peste Iuda în anul al patrulea al lui Ahab, regele lui Israel.
\par 42 Când s-a făcut rege, Iosafat era ca da treizeci și cinci de ani și douăzeci și cinci de ani a domnit el în Ierusalim. Pe mama lui o chema Azuba și era fiica lui Șilhi.
\par 43 El a umblat întru totul pe căile lui Asa, tatăl său și nu s-a abătut de la ele, săvârșind fapte plăcute înaintea ochilor Domnului. Numai înălțimile nu le-a desființat, căci poporul încă săvârșea jertfe și tămâieri pe înălțimi.
\par 44 Iosafat a făcut pace cu regele lui Israel.
\par 45 Celelalte fapte ale lui Iosafat și războaiele pe care le-a purtat sunt scrise în cronica regilor lui Iuda.
\par 46 Și rămășița desfrânaților care mai rămăsese din zilele tatălui său Asa, a stârpit-o din țară.
\par 47 În Idumeea pe atunci nu era rege, ci numai un locțiitor de rege.
\par 48 Regele Iosafat a făcut și corăbii ca cele de Tarsis, ca să se ducă să aducă aur de la Ofir; dar n-au putut să ajungă, sfărâmându-se la Ețion Gheber.
\par 49 Atunci a zis Ohozia, fiul lui Ahab, către Iosafat: "Să se ducă și slugile mele cu slugile tale cu corăbiile". Însă Iosafat n-a vrut.
\par 50 Și a adormit cu părinții săi în cetatea lui David, strămoșul său, și a fost îngropat Iosafat cu ei, în cetatea lui David și în locul lui s-a făcut rege Ioram, fiul său.
\par 51 Ohozia, fiul lui Ahab, s-a făcut rege peste Israel în Samaria în anul al șaptesprezecelea al lui Iosafat, regele Iudei; și a domnit peste Israel în Samaria doi ani.
\par 52 El a săvârșit fapte urâte înaintea ochilor Domnului și a umblat pe căile tatălui său și pe căile mamei sale și pe căile lui Ieroboam, fiul lui Nabat, care a dus pe Israel la păcat;
\par 53 Căci a slujit lui Baal și i s-a închinat lui și a mâniat pe Domnul Dumnezeul lui Israel, cum făcuse și tatăl său.


\end{document}