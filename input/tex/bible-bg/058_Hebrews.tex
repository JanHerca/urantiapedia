\begin{document}

\title{Hebrews}


\chapter{1}

\par 1 Бог,Който при разни частични съобщения, и по много начини, е говорил в старо време на бащите ни чрез пророците,
\par 2 в края на тия дни говори нам чрез Сина, Когото постави наследник на всичко, чрез Когото направи световете,
\par 3 Който, бидейки сияние на Неговата слава, и отпечатък на Неговото същество и държейки всичко чрез Своето могъщо слово, след като извърши [чрез Себе Си] очищение на греховете, седна отдясно на Величието на високо,
\par 4 и стана толкова по-горен от ангелите, колкото името, което е наследил, е по-горно от тяхното.
\par 5 Защото, кому от ангелите е рекъл Бог някога: - "Ти си Мой Син, Аз днес те родих", и пак: "Аз ще Му бъда Отец, и той ще ми бъде Син"?
\par 6 А когато въвежда Първородния във вселената, казва: - "И поклонете се Нему, всички Божии ангели".
\par 7 И за ангелите казва: - "Който прави ангелите Си силни като ветровете, И служителите Си като огнен пламък";
\par 8 А за Сина казва: - "Твоят престол, о Боже е до вечни векове; И скиптърът на Твоето царство е скиптър на правотата.
\par 9 Възлюбил си правда, и намразил си беззаконие; За това, Боже, Твоят Бог Те е помазал с миро на радост повече от Твоите събратя".
\par 10 И пак: "В началото Ти, Господи си основал земята, И делото на твоите ръце е небето;
\par 11 Те ще изчезнат, а Ти пребъдваш; Да! те всички ще овехтяват като дреха,
\par 12 И като одежда ще ги свиеш, И те ще бъдат изменени; Но Ти си същият, И Твоите години няма да се свършат.
\par 13 А кому от ангелите е рекъл някога: - "Седи отдясно Ми Докле положа враговете ти за твоето подножие"?
\par 14 Не са ли те всички служебни духове, изпращани да слугуват на ония, които ще наследят спасение?

\chapter{2}

\par 1 Затова ние сме длъжни да внимаваме повече на туй, което сме чули, да не би да го изгубим някога.
\par 2 Защото словото изговорено чрез ангели стана твърдо и всяко престъпление и непокорство получи справедлива отплата,
\par 3 то как ще избегнем ние, ако пренебрегваме едно толкова велико спасение, което отначало прогласено он Господа, се потвърди между нас от тия, които бяха чули:
\par 4 като му свидетелствуваше и Бог чрез знамения и чудеса, чрез разни, велики дела и чрез раздаване Светия Дух по волята Си?
\par 5 Защото не на ангели подчини бъдещия свят, за който говорим;
\par 6 но някой е засвидетелствувал нейде, като е казал: "Що е човек, та да го помниш, Или човешки син, та да го посещаваш?
\par 7 Ти си го направил само малко по-долен от ангелите, Със слава и чест си го увенчал, И поставил си го над делата на ръцете Си;
\par 8 Всичко си подчинил под нозете му". И като му е подчинил всичко, не е оставил нищо неподчинено нему, обаче сега не виждаме още да му е всичко подчинено.
\par 9 Но виждаме Исуса, Който е бил направен малко по-долен от ангелите, че е увенчан със слава и чест поради претърпяната смърт, за да вкуси смърт с Божията благодат, за всеки човек.
\par 10 Защото беше уместно, щото Онзи, заради Когото е всичко, и чрез Когото е всичко, като привежда много синове в слава, да усъвършенствува чрез страдания начинателя на тяхното спасение.
\par 11 Понеже и Оня, Който освещава, и ония, които се освещават, всички са от Едного; за която причина Той не се срамува да ги нарича братя.
\par 12 казвайки: "Ще възвестявам името Ти на братята Си; Ще Те хваля всред събранието";
\par 13 и пак: - "Аз на Него ще уповавам" и пак: - "Ето Аз и децата, които Ми е дал Бог".
\par 14 И тъй, понеже децата са същества от общата плът и кръв то и Той, подобно на тях, взе участие в същото, за да унищожи чрез смъртта този, който има властта, сиреч, дявола,
\par 15 и да избави всички ония, които, поради страха от смъртта, през целия си живот са били подчинени на робство.
\par 16 (Защото, наистина, Той не помогна на ангелите, но помогна на Авраамовото потомство).
\par 17 Затова трябваше да се оприличи във всичко на братята Си, за да бъде милостив и верен първосвещеник в отношение към Бога, за да извърши умилостивение за греховете на людете.
\par 18 Понеже в това дето и сам Той пострада като изкушен, може и на изкушаваните да помага.

\chapter{3}

\par 1 Затова, свети братя, участници на небесното звание, размислете за апостола и първосвещеника, Когото ние изповядваме, Исуса;
\par 2 Който бе верен на Този, Който Го постави, както беше и Моисей в целия негов дом.
\par 3 Понеже Той се удостои със слава толкоз по-голяма от Моисеевата, колкото е по-голяма почитта що има оня, който е строил къщата, от почитта що има къщата.
\par 4 Защото всяка къща се строи от някого; а Тоя, Който е устроил всичко, е Бог.
\par 5 И Моисей беше верен в целия Божий дом, като слуга, за да засвидетелствува за онова, което щеше да се говори по-после;
\par 6 а Христос, като Син, беше верен над Неговия дом. Неговият дом сме ние, ако удържим до край дръзновението и похвалата на надеждата си.
\par 7 Затова, както казва Светият Дух: - "Днес, ако чуете гласа му,
\par 8 Не закоравявайте сърдцата си както в преогорчението, Както в деня на изкушението в пустинята,
\par 9 Гдето бащите ви Ме изкусиха, изпитаха Ме. И видяха делата Ми четиридесет години.
\par 10 Затова, възнегодувах против това поколение. И рекох: Всякога се заблуждавате със сърдцата си, Те не са познали Моите пътища;
\par 11 Така се заклех в гнева Си: Те няма да влязат в Моята почивка".
\par 12 Внимавайте братя, да не би да има в някого от вас нечестиво, невярващо сърдце, което да отстъпи от живия Бог;
\par 13 но увещавайте се един друг всеки ден, докле още е "днес", да не би някой от вас да се закорави чрез измамата на греха.
\par 14 Защото ние участвуваме в Христа, ако удържим твърдо до край първоначалната си увереност:
\par 15 докато се казва: - "Днес, ако чуете гласа Му, Не закоравявайте сърдцата си както в преогорчението".
\par 16 Защото кои, като чуха, Го преогорчиха? Не всички ли ония, които излязоха от Египет чрез Моисея?
\par 17 А против кои негодува четиридесет години? Не против ония ли, които съгрешиха, и чиито трупове паднаха в пустинята?
\par 18 На кои още се закле, че няма да влязат в Неговата почивка? Не на ония ли, които бяха непокорни?
\par 19 И тъй виждаме, че поради неверието си те не можаха да влязат.

\chapter{4}

\par 1 И тъй, понеже ни остава обещание да влезем в Неговата почивка, нека се боим да не би да се открие, че някой от вас не е достигнал до нея.
\par 2 Защото на нас се донесе едно благовестие, както и на тях; но словото, което те чуха, не ги ползува, понеже не се съедини чрез вяра в ония, които го чуха.
\par 3 Затова ние, повярвалите, влизаме в тая почивка; както рече Бог: - "Така се заклех в гнева Си: Те няма да влязат в Моята почивка"; ако и да са били свършени делата Му още при основаването на света.
\par 4 Защото нейде си е говорил за самия ден така: "И почина си Бог на седмия ден от всичките Си дела";
\par 5 а пък на това място: "Няма да влязат в Моята почивка".
\par 6 И тъй, понеже остава да влязат някои в нея, а ония, на които по-напред се благовести, не са влезли поради неверието ( Или: Непокорството ) си,
\par 7 затова Той пак определя един ден, "днес", като казва толкоз време по-после чрез Давида, както вече рекохме: "Днес, ако чуете Неговия глас, Не закоравявайте сърдцата си".
\par 8 Защото, ако Исус Навиев беше им дал почивка, Бог не би говорил след това за друг ден.
\par 9 Следователно, за Божиите люде остава една съботна почивка. ( Виж. ст. 4. )
\par 10 Защото оня, който е влязъл в Неговата почивка, той си е починал от своите дела, както и Бог от Своите Си.
\par 11 Затова нека се постараем да влезем в тая почивка, за да не падне някой в това, да дава същия пример на неверие.
\par 12 Защото Божието слово е живо, деятелно, по-остро от всеки меч остър и от двете страни, пронизва до разделяне душата и духа, ставите и мозъка, и издирва помислите и намеренията на сърдцето.
\par 13 И няма създание, което да не е явно пред Бога; но всичко е голо и разкрито пред очите на Този, на Когото има да отговаряме.
\par 14 И тъй, като имаме велик Първосвещеник Исуса, Божия Син, Който е преминал до най-високите небеса, нека държим това, което сме изповядали.
\par 15 Защото нямаме такъв първосвещеник, Който да не може да състрадава с нас в нашите немощи, а имаме Един, Който е бил във всичко изкушен като нас, но пак без грях.
\par 16 Затова, нека пристъпваме с дръзновение към престола на благодатта, за да придобием милост, и да намерим благодат, която да помага благовременно.

\chapter{5}

\par 1 Защото всеки първосвещеник, като е взет измежду човеците, се поставя да принася дарове и жертви на Бога за греховете на човеците,
\par 2 такъв първосвещеник, който може да състрадава с невежите и заблудилите, защото и сам той е обиколен с немощ,
\par 3 и за това е длъжен да принася жертва за греховете, както за людете, така и за себе си.
\par 4 А както никой не взема на себе си тая почит, освен когато бъде призван от Бога, както бе Аарон,
\par 5 така и Христос не присвои на Себе Си славата да стане първосвещеник, а Му я даде Оня, Който Му е казал: "Ти си Мой Син. Аз днес Те родих";
\par 6 както и на друго място казва: "Ти си свещеник до века. Според чина Мелхиседеков".
\par 7 Тоя Христос в дните и в плътта Си, като принесе в силен вик и със сълзи молитви и молби на Този, Който можеше да го избави от смърт, и като биде послушан поради благоговението Си,
\par 8 ако и да беше Син, пак се научи на послушание от това, което пострада,
\par 9 и като се усъвършенствува, стана причина за вечно спасение за всички, които Му са послушни,
\par 10 наречен от Бога първосвещеник според Мелхиседековия чин.
\par 11 Върху това имаме да кажем много неща и мъчни за поясняване, защото сте станали тъпи в слушане.
\par 12 Понеже докато вие трябваше до сега, според изтеклото време, и учители да станете, имате нужда да ви учи някой изново най-елементарните начала на Божиите словеса, и достигнахте да имате нужда от мляко, а не от твърда храна.
\par 13 Защото всеки, който се храни с мляко, е неопитен в учението за правдата, понеже е младенец;
\par 14 а твърдата храна е за пълнолетните, които чрез упражнение са обучили чувствата си да разпознават доброто и злото.

\chapter{6}

\par 1 Поради това, нека оставим първоначалното учение за Христа и нека се стремим към съвършенство, без да полагаме изново за основа покаяние от мъртви дела, вяра в Бога,
\par 2 учение за кръщения, за ръкополагане, за възкресяване на мъртви и за вечен съд.
\par 3 И това ще строим, ако Бог позволи.
\par 4 Защото за тия, които еднаж са били просветени, и са вкусили от небесния дар, и са станали причастници на Светия Дух
\par 5 и са вкусили, колко е добро Божието слово, още са вкусили и от великите дела, които въвеждат бъдещия век,
\par 6 а са отпаднали, невъзможно е да се обновят пак и доведат до покаяние, докато разпъват втори път в себе си Божия Син и Го опозоряват.
\par 7 Защото земята, която се е поила от дъжда, що пада често на нея, и която ражда трева полезна на тия, за които се и обработва, получава благословение от Бога;
\par 8 но ако ражда тръни и репеи, отхвърля се; тя скоро ще се прокълне, и сетнината й е да се изгори.
\par 9 Обаче, ако и да говорим така, надяваме се от вас, възлюбени, за нещо по-добро, нещо, което води към спасението.
\par 10 Защото Бог не е неправеден, та да забрави това, което извършихте и любовта, която показахте към Неговото име, като послужихте и още служите на светиите.
\par 11 И желаем всеки от вас да показва същото усърдие за пълна увереност в надеждата до край;
\par 12 да не бъдете лениви, но да подражавате ония, които чрез вяра и устояване наследяват обещаните благословения.
\par 13 Защото, когато Бог даваше обещание на Авраама, понеже нямаше никого по-голям, в когото да се закълне, закле се в Себе Си, казвайки:
\par 14 "Наистина ще те благословя премного и ще те умножа и преумножа".
\par 15 И така, Авраам, като устоя, получи обещаното.
\par 16 Защото както, човеците се кълнат в някого по-голям от тях, и клетвата, дадена в потвърждение на думата, туря край на всеки спор между тях,
\par 17 така и Бог, като искаше да покаже по-пълно на наследниците на обещанието, че намерението Му е неизменимо, си послужи с клетва,
\par 18 така щото чрез две неизменими неща, в които не е възможно за Бога да лъже, да имаме голямо насърдчение ние, които сме прибягнали да се държим за поставената пред нас надежда;
\par 19 която имаме за душата като здрава и непоколебима котва, която прониква в това, което е отвътре завесата;
\par 20 гдето Исус като предтеча влезе за нас, и стана първосвещеник до века според Мелхиседековия чин.

\chapter{7}

\par 1 Защото тоя Мелхиседек, салимски цар, свещеник на Всевишния Бог, Който срещна Авраама, когато се връщаше от поражението на царете и го благослови,
\par 2 комуто Авраам отдели и десетък от всичката плячка - тоя, който е първо, по значението на името му, цар на правда, а после и салимски цар, цар на мир,
\par 3 - без баща, без майка, без родословие, без да има или начало на дни, или край на живот, но уприличен на Божия Син, остава завинаги свещеник.
\par 4 А помислете, колко велик беше тоя човек, комуто патриарх Авраам даде и десетък от най-добрата плячка.
\par 5 Защото, докато ония от Левиевите потомци, които приемат свещенството, имат заповед по закона да вземат десетък от людете, сиреч, от братята си, ако и тия да са произлезли от чреслата на Авраама,
\par 6 той, обаче, който не е произлязъл от техния род, взе десетък от Авраама и благослови този, комуто бяха дадени обещанията.
\par 7 А безспорно по-долният се благославя от по-горния.
\par 8 И в един случай смъртните човеци вземат десетък, а в другия - тоя, за когото се свидетелствува, че живее.
\par 9 И, тъй да кажа, сам Левий, който взема десетък, даде десетък чрез Авраама;
\par 10 защото беше още в чреслата на баща си, когато Мелхиседек срещна Авраама.
\par 11 Прочее, ако би имало съвършенство чрез левитското свещенство (защото под него людете получиха закона), каква нужда е имало вече да се издигне друг свещеник, според Мелхиседековия чин, и да се не счита според Аароновия чин?
\par 12 Защото, ако се промени свещенството, по необходимост става промяна и на закона.
\par 13 Понеже тоя, за когото се казва това, принадлежи на друго племе, от което никой не е служил на олтара.
\par 14 Защото е известно, че нашият Господ произлезе от Юдовото племе, относно което племе Моисей не каза нищо за свещеници.
\par 15 Това, що казваме, става още по-явно, тъй като по подобие на Мелхиседека се издига друг свещеник,
\par 16 Който се установи не по закон, изразен в плътска заповед, но по силата на един безконечен живот;
\par 17 защото за Него свидетелствува: "Ти си свещеник до века Според чина Мелхиседеков";
\par 18 защото по тоя начин се унищожава по-предишната заповед, поради нейната слабост и безполезност,
\par 19 (понеже законът не е усъвършенствувал нищо), и се въвежда една по-добра надежда, чрез която се приближаваме при Бога.
\par 20 И колкото е важно това, че Той не е станал свещеник без заклеване,
\par 21 (защото те ставаха свещеници без заклеване, а Той със заклеване от страна на Този, Който Му казва: "Господ се закле (и не ще се разкае), като каза: Ти си свещеник до века,
\par 22 толкоз на по-добър завет Исус стана поръчител.
\par 23 При това, поставените свещеници са били мнозина, защото смъртта им пречеше да продължават в чина си,
\par 24 но Той, понеже пребъдва вечно, има свещенство, което не преминава на другиго.
\par 25 Затова и може съвършено да спасява тия, които дохождат при Бога чрез Него, понеже всякога живее да ходатайствува за тях.
\par 26 Защото такъв първосвещеник ни трябваше: свет, невинен, непорочен, отделен от грешните и възвисен по-горе от небесата;
\par 27 Който няма нужда всеки ден, като ония първосвещеници, да принася жертви първо за своите грехове, после за греховете на людете; защото стори това еднаж завинаги, като принесе Себе Си.
\par 28 Защото законът поставя за първосвещеници немощни човеци; а думите на клетвата, която беше подир закона, поставят Сина, Който е усъвършенствуван за винаги.

\chapter{8}

\par 1 А от това, което казваме, ето що е смисълът: Ние имаме такъв първосвещеник, Който седна отдясно на престола на Величието в небесата,
\par 2 служител на величието и на истинската скиния, която Господ е поставил, а не човек.
\par 3 Защото всеки първосвещеник се поставя да принася и дарове и жертви; затова, нуждно е и Тоя първосвещеник да има нещо да принася.
\par 4 А ако беше на земята, Той не щеше нито да бъде свещеник, защото има такива, които принасят даровете според закона;
\par 5 (които служат на онова, което е само образ и сянка на небесните неща, както бе заповядано на Моисея, когато щеше да направи скинията; защото: "Внимавай", му каза Бог, "да направиш всичко по образеца, който ти бе показан на планината"),
\par 6 но на дело Христос е получил служение толкоз по-превъзходно, колкото и завета, на който Той е ходатай, е по-превъзходен, като узаконен върху по-превъзходни обещания.
\par 7 Защото, ако оня първи завет е бил без недостатък, Бог не би търсил място за втори.
\par 8 А напротив, когато порицава израилтяните, казва: "Ето, идат дни, казва Господ, Когато ще сключа с Израилевия дом и с Юдовия дом нов завет;
\par 9 Не такъв завет, какъвто направих с бащите им В деня, когато ги хванах за ръка, за да ги изведа из Египетската земя; Защото те не устояха в завета Ми, И Аз ги оставих, казва Господ.
\par 10 Защото, ето заветът, който ще направя с Израилевия дом. След ония дни, казва Господ: Ще положа законите Си в ума им И ще ги напиша в сърдцата им; Аз ще бъда техен Бог, И те ще бъдат мои люде;
\par 11 И няма вече да учат всеки съгражданина си И всеки брата си, като му казват: Познай Господа; Защото всички ще Ме познават, От малък до голям между тях.
\par 12 Защото ще покажа милост към неправдите им И греховете им (и беззаконията им) няма да помня вече".
\par 13 А като каза "нов завет", Той обявява първия за остарял. А онова, което овехтява и остарява, е близу до изчезване.

\chapter{9}

\par 1 А даже при първия завет имаше постановления за богослужение, имаше и земно светилище.
\par 2 Защото беше приготвена скиния, в първата част на която бяха светилникът, трапезата и присъствените хлябове; която част се казва светото място;
\par 3 а зад втората завеса беше оная част от скинията, която се казваше пресветото място,
\par 4 гдето бяха златната кандилница и ковчегът на завета, отвсякъде обкован със злато, в който бяха златната стомна, съдържаща манната, Аароновият жезъл, който процъфтя, и плочите на завета;
\par 5 и над него бяха херувимите на Божията слава, които осеняваха умилистивилището; за които не е сега време да говорим подробно.
\par 6 И когато тия неща бяха така приготвени, в първата част на скинията свещениците влизаха постоянно да извършват богослужението;
\par 7 А във втората еднаж в годината влизаше само първосвещеникът, и то не без кръв, която принасяше за себе си и за греховете на людете, сторени от незнание.
\par 8 С това Светият Дух показваше, че пътят за в светилището не е бил открит, докато е стояла още първата част на скинията,
\par 9 която е образ на сегашното време, съгласно с което се принасят дарове и жертви, които не могат да направят поклонника, колкото за съвестта му, съвършено чист,
\par 10 понеже се състоят само в ядене, пиене и разни омивания, - плътски постановления, наложени до едно време на преобразувание.
\par 11 А понеже Христос дойде като първосвещеник на бъдещите добрини, Той влезе през по-голямата и по-съвършена скиния, не с ръка направена, сиреч, не от настоящето творение,
\par 12 еднаж завинаги в светилището, и то не с кръв от козли и от телци, но със Собствената Си кръв, и придоби за нас вечно изкупление.
\par 13 Защото, ако кръвта от козли и от юнци и пепелта от юница, с които се поръсваха осквернените, освещава за очистването на тялото,
\par 14 то колко повече кръвта на Христа, Който чрез вечния Дух принесе Себе Си без недостатък на Бога, ще очисти съвестта ви от мъртвите дела, за да служите на живия Бог!
\par 15 Той е посредник на нов завет по тая причина, щото призваните да получават обещаното вечно наследство чрез смъртта, станала за изкупване престъпленията, извършени при първия завет.
\par 16 Защото гдето има завещание, трябва, за изпълнението му, да се докаже и смъртта на завещателя.
\par 17 Защото завещанието влиза в сила, само гдето се е случила смърт, понеже никога няма сила, докле е жив завещателя.
\par 18 Затова нито първият завет бе утвърден без кръв;
\par 19 Защото, след като Моисей изговори всяка заповед от закона пред всичките люде, взе кръвта на телците и на козлите, с вода и с червена вълна и исоп, та поръси и самата книга и всичките люде, и казваше:
\par 20 "Това е кръвта, проляна при завета, който Бог е заръчал спрямо вас".
\par 21 При това, Той по същия начин поръси с кръвта и скинията и всичките служебни съдове.
\par 22 И почти мога да кажа, че по закона всичко с кръв се очистя; и без проливането на кръв няма прощение.
\par 23 И тъй, необходимо беше образите на небесните неща да се очистват с тия жертви, а самите небесни - с жертви по-добри от тях.
\par 24 Защото Христос влезе не в ръкотворено светилище, образ на истинското, но в самите небеса, да се яви вече пред Божието лице за нас;
\par 25 и не за да принася Себе Си много пъти, както първосвещеникът влиза в светилището всяка година с чужда кръв,
\par 26 (иначе Той трябва да е страдал много пъти от създанието на света); а на дело в края на вековете се яви еднаж да отмахне греха, като принесе Себе Си в жертва.
\par 27 И тъй като е определено на човеците еднаж да умрат, а след това настава съд,
\par 28 така и Христос, като биде принесен еднаж, за да понесе греховете на мнозина, ще се яви втори път, без да има работа с грях, за спасението на ония, които Го очакват.

\chapter{10}

\par 1 Защото законът, като съдържа в себе си само сянка на бъдещите добрини, а не самата същност на нещата, то свещениците, които непрестанно принасят всяка година същите жертви, никога не могат с тях да направят съвършени в чистота ония, които пристъпват да жертвуват.
\par 2 Другояче те биха престанали да ги принасят; защото жертвоприносителите, еднаж очистени, не биха имали вече никакво изобличение на съвестта за грехове.
\par 3 Но в тия жертви всяка година става спомен за греховете.
\par 4 Защото не е възможно кръв от юнци и от козли да отмахне грехове.
\par 5 Затова Христос, като влиза в света, казва: - "Жертва и принос не си поискал, Но приготвил си Ми тяло;
\par 6 Всеизгаряния и приноси за грях не Ти са угодни.
\par 7 Тогава рекох: Ето, дойдох, (В свитъка на книгата е писано за Мене), Да изпълня Твоята воля, о Боже".
\par 8 Като казва първо: " Жертви и приноси и всеизгаряния и приноси за грях не си поискал, нито Ти са угодни", (които впрочем се принасят според закона),
\par 9 после казва: " Ето, дойдох да изпълня волята Ти". Ще каже: Той отмахва първото, за да постанови второто.
\par 10 С тая воля ние сме осветени чрез принасянето на Исус Христовото тяло еднаж за винаги.
\par 11 И всеки свещеник, като стои та служи всеки ден, принася много пъти същите жертви, които никога не могат да отмахнат грехове;
\par 12 но Той, като принесе една жертва за греховете, седна за винаги отдясно на Бога.
\par 13 та оттогава нататък чака, докле се положат враговете Му за Негово подножие.
\par 14 Защото с един принос Той е усъвършенствувал за винаги ония, които се освещават.
\par 15 А и Светия Дух ни свидетелствува за това; защото след като е казал: -
\par 16 "Ето заветът, който ще направя с тях След ония дни, казва Господ: Ще положа законите Си в сърдцата им, И ще ги напиша в умовете им", прибавя:
\par 17 "И греховете им и беззаконията им няма да помня вече".
\par 18 А гдето има прощение за тия неща, там вече няма принос за грях.
\par 19 И тъй, братя, като имаме чрез кръвта на Исуса дръзновение да влезем в светилището,
\par 20 през новия и живия път, който Той е открил за нас през завесата, сиреч, плътта Си,
\par 21 и като имаме велик Свещеник над Божия дом,
\par 22 нека пристъпваме с искрено сърдце в пълна вяра, със сърдца очистени от лукава съвест ( Или: Поръсена от нечиста съвест ) и с тяло измито в чиста вода;
\par 23 нека държим непоколебимо надеждата, която изповядваме, защото е верен Оня, Който се е обещал;
\par 24 и нека се грижим един за друг, тъй щото да се поощряваме към любов и добри дела,
\par 25 като не преставаме да се събираме заедно, както някои имат обичай да престават, а да увещаваме един друг, и толкова повече, колкото виждате, че денят наближава.
\par 26 Защото, ако съгрешаваме самоволно, след като сме познали истината, не остава вече жертва за грехове,
\par 27 но едно страшно очакване на съд и едно огнено негодуване, което ще изпояде противниците.
\par 28 Някой, който е престъпил Моисеевия закон, умира безпощадно при думата на двама или трима свидетели;
\par 29 тогава колко по-тежко наказание, мислите, ще заслужи оня, който е потъпкал Божия Син, и е счел за просто нещо проляната при завета кръв, с която е осветен, и е оскърбил Духа на благодатта?
\par 30 Защото познаваме Този, Който е рекъл: "На Мене принадлежи възмездието, Аз ще сторя въздаяние"; и пак: "Господ ще съди людете Си".
\par 31 Страшно е да падне човек в ръцете на живия Бог.
\par 32 Припомняйте си още за първите дни, когато, откак се просветихте, претърпяхте голяма борба от страдания,
\par 33 кога опозорявани с хули и оскръбления, кога пък като съучастници с тия, които страдаха така.
\par 34 Защото вие не само състрадавахте с ония, които бяха в окови, но и радостно посрещахте разграбването на имота си, като знаете, че вие си имате по-добър и траен имот.
\par 35 И тъй, не напущайте дръзновението си, за което имате голяма награда.
\par 36 Защото ви е нуждно търпение, та, като извършите Божията воля, да получите обещаното.
\par 37 "Защото още твърде малко време И ще дойде Тоя, Който има да дойде, и не ще се забави.
\par 38 А който е праведен пред Мене ( Гръцки: Моя праведник ), ще живее чрез вяра; Но ако се дръпне назад, няма да благоволи в него душата Ми".
\par 39 Ние, обаче, не сме от ония, които се дърпат назад, та се погубват, а от тия, които вярват та се спасява душата им.

\chapter{11}

\par 1 А вярата е даване твърда увереност в ония неща, за които се надяваме, - убеждения за неща, които не се виждат.
\par 2 Защото поради нея за старовременните добре се свидетелствуваше.
\par 3 С вяра разбираме, че световете са били създадени с Божието слово, тъй щото видимото не стана от видими неща.
\par 4 С вяра Авел принесе Богу жертва по-добра от Каиновата, чрез която за него се засвидетелствува, че е праведен, понеже Бог свидетелствува за даровете му; и чрез тая вяра той и след смъртта си още говори.
\par 5 С вяра Енох бе преселен, за да не види смърт, и не се намираше, защото Бог го пресели; понеже преди неговото преселване беше засвидетелствувано за него, че е бил угоден на Бога.
\par 6 А без вяра не е възможно да се угоди Богу, защото който дохожда при Бога трябва да вярва, че има Бог, и че Той възнаграждава тия, които го търсят.
\par 7 С вяра Ное, предупреден от Бога за неща, които още не се виждаха, подбуден от страхопочитание, направи ковчег за спасение на дома си; чрез която вяра той осъди света и стана наследник на правдата, която е чрез вяра.
\par 8 С вяра Авраам послуша, когато бе повикан да излезе и да отиде на едно място, което щеше да получи в наследство, и излезе без да знае къде отива.
\par 9 С вяра се засели в обещаната земя като в чужда, и живееше в шатри, както и Исаак и Яков, наследниците заедно с него на същото обещание.
\par 10 Защото очакваше града, който има вечни основи, на който архитект и строител е Бог.
\par 11 С вяра и сама Сара доби сила да зачене в преминала възраст, понеже счете за верен Този, Който се бе обещал.
\par 12 Затова само от един човек, и той замъртвял, се народи множество, колкото небесните звезди и като крайморския пясък, който не може да се изброи.
\par 13 Всички тия умряха във вяра, тъй като не бяха получили изпълнението на обещанията; но ги видяха и поздравиха отдалеч, като изповядаха, че са чужденци и пришелци на земята.
\par 14 А ония, които говорят така, явно показват, че търсят свое отечество;
\par 15 и ако наистина, така говорейки, са имали в ума си онова отечество, от което бяха излезли, намерили биха случай да се върнат.
\par 16 Но на дело желаят едно по-добро отечество, сиреч, небесното; затова Бог не се срамува от тях да се нарече техен Бог, защото им е приготвил град.
\par 17 С вяра Авраам, когато го изпитваше Бог, принесе Исаака жертва, да! оня, който беше получил обещанията принасяше единородния си син, -
\par 18 оня, комуто беше казано: " По Исаака ще се наименува твоето потомство".
\par 19 като разсъди, че Бог може да възкресява и от мъртвите, - отгдето по един начин на възкресение го и получи назад.
\par 20 С вяра Исаак благослови, Якова и Исава даже за бъдещите неща.
\par 21 С вяра Яков, на умиране, благослови всекиго от Йосифовите синове и поклони се Богу подпирайки се върху края на тоягата си.
\par 22 С вяра Йосиф, на умиране, спомена за излизането на израилтяните и даде поръчка за костите си.
\par 23 С вяра, когато се роди Моисей, родителите му го криха три месеца, защото видяха, че бе красиво дете; и не се боеха от царската заповед,
\par 24 С вяра Моисей, като стана на възраст, се отказа да се нарича син на фараоновата дъщеря
\par 25 и предпочете да страда с Божиите люде, а не да се наслаждава за кратко време на греха,
\par 26 като разсъди, че укорът за Христа е по-голямо богатство от египетските съкровища; защото гледаше на бъдещата награда.
\par 27 С вяра напустна Египет, без да се бои от царския гняв; защото издържа, като един, който вижда Невидимия.
\par 28 С вяра установи пасхата и поръсването с кръвта, за да се не допре до тях този, които погубваше първородните.
\par 29 С вяра израилтяните минаха през Червеното море като по сухо, на което като се опитаха и египтяните, издавиха се.
\par 30 Чрез вяра ерихонските стени паднаха след седмодневно обикаляне около тях.
\par 31 С вяра Раав блудницата не погина заедно с непокорните, като прие съгледателите с мир.
\par 32 И какво повече да кажа? Защото не ще ми стигне време да приказвам за Гедеона, Варака, Самсона и Иефтае, за Давида още и Самуила и пророците;
\par 33 които с вяра побеждаваха царства, раздаваха правда, получаваха обещания, затуляха устата на лъвове,
\par 34 угасваха силата на огъня, избягваха острото на ножа, оздравяха от болести, ставаха силни във война, обръщаха в бяг чужди войски.
\par 35 Жени приемаха мъртвите си възкресени; а други бяха мъчени, защото, за да получат по-добро възкресение, те не приемаха да бъдат избавени.
\par 36 Други пък изпитваха присмехи и бичувания, а още и окови и тъмници;
\par 37 с камъни биваха убити, с трион претрити, с мъки мъчени; умираха заклани с нож, скитаха се в овчи и кози кожи и търпяха лишение, бедствия и страдания;
\par 38 те, за които светът не беше достоен, се скитаха по пустините и планините, по пещерите и рововете на земята,
\par 39 Но всички тия, ако и да бяха засвидетелствувани чрез вярата им, пак не получиха изпълнението на обещанието,
\par 40 да не би да постигнат в съвършенство без нас; защото за нас Бог промисли нещо по-добро.

\chapter{12}

\par 1 Следователно и ние, като сме обиколени от такъв голям облак свидетели, нека отхвърлим всяка тегота и греха, който лесно ни сплита, и с търпение нека тичаме на предлежащето пред нас поприще,
\par 2 като гледаме на Исуса начинателя и усъвършителя на вярата ни, Който, заради предстоящата Нему радост, издържа кръст, като презря срама и седна отдясно на Божия престол.
\par 3 Защото размислете за Този, Който издържа от грешните такова противоречие против Себе Си, та да ви не дотегва и да не ставате малодушни,
\par 4 Не сте се още съпротивили до кръв в борбата си против греха.
\par 5 И сте забравили увещанието, което ви съветва като синове: "Сине мой, не презирай наказанието на Господа, Нито да ослабваш, когато те изобличава Той;
\par 6 Защото Господ наказва този, когото люби, И бие всеки син, когото приема",
\par 7 Ако търпите наказание, Бог се обхожда с вас като със синове; защото кой е тоя син, когото баща му не наказва?
\par 8 Но ако сте без наказание, на което всички са били определени да участвуват, тогава сте незаконно родени, а не синове.
\par 9 Освен това, имали сме бащи по плът, които са ни наказвали, и сме ги почитали; не щем ли повече да се покоряваме на Отца на духовете ни и да живеем?
\par 10 Защото те за малко време са ни наказвали, според както им е било угодно, а Той - за наша полза, за да съучаствуваме в Неговата светост.
\par 11 Никое наказание не се вижда на времето да е за радост, а е тежко; но после принася правда като мирен плод за тия, които са се обучавали чрез него.
\par 12 Затова "укрепете немощните ръце и отслабналите колена",
\par 13 и направете за нозете си прави пътища, за да не се изкълчи куцото, но напротив, да изцелее.
\par 14 Търсете мир с всички и онова освещение, без което никой няма да види Господа.
\par 15 И внимавайте, да не би някой да не достигне до Божията благодат; да не би да поникне някой горчив корен, та да ви смущава, и мнозинството да се зарази от него;
\par 16 да не би някой да е блудник или нечестив, както Исав, който за едно ястие продаде първородството си;
\par 17 понеже знаете, че даже когато искаше по-после да наследи благословението, той бе отхвърлен, при все че го потърси със сълзи, защото не намери място за промяна на ума у баща си.
\par 18 Защото вие не сте пристъпили до планина осезаема и пламнала в огън, нито до тъмнина и буря,
\par 19 нито до тръбен звук и глас на думи; такъв, щото ония, които го чуха, се примолиха да им се не говори вече дума;
\par 20 (защото не можаха да изтърпят онова, що им се заповядваше: "Даже животно, ако се допре до планината, ще се убие с камъни";
\par 21 и толкоз страшна беше гледката, щото Моисей рече: "Много съм уплашен и разтреперан");
\par 22 но пристъпихме до хълма Сион, до града на живия Бог, небесния Ерусалим, и при десетки хиляди тържествуващи ангели,
\par 23 при събора на първородните, които са записани на небесата, при Бога, Съдията на всички, при духовете на усъвършенствуваните праведници,
\par 24 при Исуса, Посредника на новия завет, и при поръсената кръв, която говори по-добри неща от Авеловата.
\par 25 Внимавайте, да не презрете този, Който говори; защото,ако ония не избегнаха наказанието, като презряха този, който ги предупреждаваше на земята, то колко повече не щем избегна ние, ако се отървем от Този, Който предупреждава от небесата!
\par 26 Чийто глас разтърси тогава земята; а сега Той се обеща, казвайки: "Още веднъж Аз ще разтърся не само земята, но и небето".
\par 27 А това "още еднаж" означава премахването на ония неща, които се клатят, като направени неща, за да останат тия, които не се клатят.
\par 28 Затова, понеже приемаме царство, което не се клати, нека бъдем благодарни, и така да служим благоугодно Богу с благоговение и страхопочитание;
\par 29 защото нашият Бог е огън, който пояжда.

\chapter{13}

\par 1 Постоянствувайте в братолюбието.
\par 2 Не забравяйте гостолюбието; понеже чрез него някои, без да знаят, са приели на гости ангели.
\par 3 Помнете затворените, като че сте с тях заедно затворени, - страдащите, като сте и сами вие в тяло.
\par 4 Женитбата нека бъде на почит у всички, и леглото неосквернено; защото Бог ще съди блудниците и прелюбодейците.
\par 5 Не се впримчвайте в сребролюбието; задоволявайте се с това, що имате, защото сам Бог е рекъл: "Никак няма да те оставя и никак няма да те забравя";
\par 6 така щото дръзновено казваме: "Господ ми е помощник; няма да се убоя; Какво ще ми стори човек?"
\par 7 Помнете ония, които са ви били наставници, които са ви говорили Божието слово; и като се взирате в сетнината на техния начин на живеене, подражавайте вярата им.
\par 8 Исус Христос е същият вчера днес и до века.
\par 9 Не се завличайте от разни и странни учения; защото е добре сърдцето да се укрепява с благодат, а не с наредби за ястия, от които не са се ползували ония, които не са се водили по ( Гръцки: Са ходили в ) тях.
\par 10 Ние имаме олтар, от който нямат право да ядат служащите в скинията.
\par 11 Защото се изгарят вън от стана телата на животните, чиято кръв, първосвещенникът внася в светилището като жертва за греховете.
\par 12 Затова и Исус, за да освети людете чрез собствената си кръв, пострада вън от градската порта.
\par 13 Прочее, нека излизаме и ние към Него вън от стана, понасяйки позор за Него.
\par 14 Защото тук нямаме постоянен град, но търсим бъдещия.
\par 15 Прочее, чрез Него нека принасяме на Бога непрестанно хвалебна жертва, сиреч, плод от устни, които изповядват Неговото име.
\par 16 А не забравяйте да правите благодеяния и да споделяте с другите благата си; защото такива жертви са угодни на Бога,
\par 17 бъдете послушни на вашите наставници и покорявайте им се, (защото те бдят за душите ви, като отговорни за тях), за да изпълнят това бдение с радост, а не с въздишане; защото да бдят с въздишане не би било полезно за вас.
\par 18 Молете се за нас, защото сме уверени, че имаме чиста съвест и искаме да се обхождаме във всичко честно.
\par 19 А особено ви се моля да правите това, за да ви бъда по-скоро повърнат.
\par 20 А Бог на мира, Който чрез кръвта на единия вечен завет е въздигнал от мъртвите великия Пастир на овцете, нашия Господ Исус,
\par 21 дано ви усъвършенствува във всяко добро нещо, за да вършите Неговата воля, като действува във вас това, което е угодно пред Него чрез Исуса Христа, Комуто да бъде слава във вечни векове. Амин.
\par 22 Но моля ви се , братя, да ви не бъде тежко това увещателно слово, защото накъсо ви писах.
\par 23 Знайте, че нашият брат Тимотей е пуснат, с когото ако дойде скоро, ще ви видя.
\par 24 Поздравете всичките ваши наставници и всичките светии. Поздравяват ви тия, които са от Италия.
\par 25 Благодат да бъде с всички вас. Амин.

\end{document}