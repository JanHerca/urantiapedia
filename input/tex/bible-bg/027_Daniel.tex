\begin{document}

\title{Данаил}


\chapter{1}

\par 1 В третата година от царуването на Юдовия цар Иоаким дойде вавилонския цар Навуходоносор в Eрусалим и го обсади.
\par 2 И Господ предаде в ръката му Юдовия цар Иоаким и част от съдовете в Божия дом; и той ги занесе в земята Сенаар, в капището на своя бог, и съдовете внесе в съкровищницата на своя бог.
\par 3 И царят заповяда на началника на скопците си Асфеназ да доведе някой от израиляните, сиреч, от царския род и от благородните,
\par 4 юноши без никакъв недостатък, красиви на глед, които проумяваха всякаква мъдрост, вещи във всякакво знание, такива, които владееха науките и бяха достойни да стоят в царския палат, та да ги учението и езика на халдейците.
\par 5 И царят им определи за всеки ден дял от царските изрядни ястия и от виното, което той пиеше, с които да ги хранят три години, за да стоят пред царя след изтичането на това време.
\par 6 А между тях бяха, от юдейците, Даниил, Анания, Мисаил, и Азария,
\par 7 които началникът на скопците преименува, като нарече Даниила Валтасасар, и Анания Седрах, Мисаила Мисах, и Азария, Авденаго.
\par 8 Но Даниил реши в сърцето си да се не оскверни от изрядните ястия на царя, нито от виното, което той пиеше; затова, помоли началника на скопците да се не оскверни.
\par 9 А Бог направи щото Даниил да придобие благоволение и милост пред началника на скопците.
\par 10 И началникът на скопците рече на Даниила: Аз се боя от господаря си царя, който определи ястието ви и питието ви, да не би да види, че лицата ви са по-малко поправени от лицата на юношите вашите връстници, и така да турите главата ми в опасност пред царя.
\par 11 Тогава Даниил рече на надзирателя, когото началникът на скопците беше поставил над Даниила, Анания, Мисаила, и Азария.
\par 12 Опитай, моля, слугите си десет дни, през което време да ни се даде зеленчук да ядем и вода да пием;
\par 13 после нека се прегледат пред тебе нашите лица и лицата на юношите, които ядат от изрядните ястия на царя; и според каквото видиш постъпи със слугите си.
\par 14 И той ги послуша в това нещо, и ги опита десет дена.
\par 15 И на края на десетте дена техните лица изглеждаха по-красиви и по-пълни, отколкото лицата на всичките юноши, които ядяха от изрядните ястия на царя.
\par 16 И тъй, надзирателят отнемаше от тях изрядното ястие и виното, което трябваше да пият, и им даваше зеленчук.
\par 17 А на тия четири юноша Бог даде знание и разум във  всяко учение и мъдрост; и Даниил можеше да проумява всички видения и сънувания.
\par 18 И на края на дните, когато царят бе заповядал да ги въведат, началникът на скопците ги въведе пред Навуходоносора.
\par 19 И като разговаря с тях царят, между всички тях не се намери подобен на Даниила, Анания, Мисаила, и Азария; за това, те стояха пред царя.
\par 20 И във всяко дело, което изискваше мъдрост и проумяване, за което царят ги попита, намери ги десет пъти по-добри от всичките врачове и вражари, които бяха в цялото му царство.
\par 21 И Даниил остана до първата година на цар Кира.

\chapter{2}

\par 1 И във втората година от царуването на Навуходоносора, Навуходоносор видя сънища, от които духът му се смути и сънят побягна от него.
\par 2 Тогава царят заповяда да повикат врачовете и вражарите, омаятелите и халдейците, за да явят на царя сънищата му. И тъй, те влязоха и застанаха пред царя.
\par 3 И царят им рече: Видях сън; и духът ми се смущава, за да разбера съня.
\par 4 Тогава халдейците говориха на царя на сирийски, казвайки: Царю, да си жив до веки! кажи съня на слугите си, и ние ще явим значението му.
\par 5 В отговор царят рече на халдейците: Указът излезе от мене; ако не ми явите съня и значението му, ще бъдете разсечени, и къщите ви ще се обърнат на бунища;
\par 6 но ако явите съня и значението му, ще получите от мене подаръци, награди, и голяма чест. Явете ми, прочее, съня и значението му.
\par 7 Те отговаряйки втори път рекоха: Нека каже царят съня на слугите си, и ние ще явим значението му.
\par 8 В отговор царят рече: Зная добре, че вие искате да печелите време, понеже виждате, че указът излезе от мене.
\par 9 Обаче, ако не ми явите съня, има само това решение за вас; защото сте се наговорили да говорите лъжливи и празни думи пред мене додето се измени решението (Еврейски: времето) ми. Кажете ми прочее съня, и аз ще узная, че можете да ми явите и значението му.
\par 10 Халдейците отговаряйки пред царя, рекоха: Няма човек на света, който да може да яви тая царева работа; защото няма цар, господар или управител, който да е изискал такова нещо от врач или вражар, или халдеец.
\par 11 Това нещо, което царят изисква е мъчно; и няма друг, който би могъл да го яви пред царя, освен боговете, чието жилище не е между човеците. (Еврейски: не е с плът)
\par 12 За това, царят се разгневи и много се разяри, и заповяда да погубят всичките вавилонски мъдреци.
\par 13 И тъй, като излезе указът да се умъртвят мъдреците, потърсиха Даниила и другарите му, за да ги убият.
\par 14 Тогава Даниил отговори с благоразумие и мъдрост на началника на царските телохранители, Ариох, който беше излязъл да убие вавилонските мъдреци.
\par 15 Отговаряйки, той рече на царския началник Ариох: Защо е тоя царски указ тъй прибързан? Тогава Ариох яви работата на Даниила.
\par 16 И Даниил влезе и помоли царя да му даде време, за да яви на царя значението на съня.
\par 17 Тогава Даниил отиде в къщата си и яви това нещо на другарите си Анания, Мисаила, и Азария,
\par 18 за да просят милост от небесния Бог досежно тая тайна, тъй щото да не погинат Даниил и другарите му с другите вавилонски мъдреци.
\par 19 Тогава се откри тайната на Даниила в нощно видение. Тогава Даниил, като благослови небесния Бог, проговори.
\par 20 Даниил рече: -  Да бъде благословено името Божие От века и до века; Защото мъдростта и силата са негови.
\par 21 Той изменява времената и годините; Сваля царе, и поставя царе; Той е, който дава мъдрост на мъдрите И знание на разумните.
\par 22 Той открива дълбоките и скрити неща; Той познава онова, което е в тъмнината; И светлината обитава с Него.
\par 23 На тебе, Боже на бащите ми, благодаря, И тебе славословя, Който си ми дал мъдрост и сила Като си ми открил онова, Което попросих от тебе; Защото си ни открил царевата работа.
\par 24 И тъй, Даниил влезе при Ариоха, когото царят бе назначил да погуби Вавилонските мъдреци, и като влезе рече му така: Недей погубва вавилонските мъдреци. Въведи ме пред царя, и аз ще явя на царя значението на съня.
\par 25 Тогава Ариох побърза да въведе Даниила пред царя, и му каза така: Намерих човек от юдейските пленници, който ще яви на царя значението.
\par 26 Царят проговаряйки рече на Даниила, чието име бе Валтасасар: Можеш ли да ми откриеш съня, който видях, и значението му?
\par 27 В отговор Даниил рече на царя: Тайната, която царят изисква, не могат да явят на царя ни мъдреци, ни вражари, ни врачове, ни астролози;
\par 28 Но има Бог на небесата, който открива тайни; и той явява на цар Навуходоносора онова, що има да стане в послешните дни. Ето сънят ти и това, което си видял в главата си на леглото си;
\par 29 Царю, размишленията ти дойдоха в ума ти на леглото ти за онова, което има да стане по-после; и оня, който открива тайни, ти е явил онова, що има да стане.
\par 30 Но колкото за мене, тая тайна не ми се откри чрез някоя мъдрост, която имам аз повече от всичките живи, но за да се открие на царя значението на съня, и за да разбереш размишленията на сърцето си.
\par 31 Ти, царю, си видял, и ето голям образ. Тоя образ, който е бил велик, и чийто блясък е бил превъзходен, е стоял пред тебе; и изгледът му е бил страшен.
\par 32 Главата на тоя образ е била от чисто злато, гърдите му и мишците му от сребро, корема му и бедрата му от мед,
\par 33 краката му от желязо, нозете му отчасти от желязо, а отчасти от кал.
\par 34 Ти си гледал додето се е отсякъл камък, не с ръце, който е ударил образа в нозете му, които са били от желязо и кал, и ги е строшил.
\par 35 Тогава желязото, калта, медта, среброто, и златото са се строшили изведнъж, и са станали като прах по гумното лете; вятърът ги е отнесъл, и за тях не се е намерило никакво място. А камъкът, който ударил образа, е станал голяма планина и е изпълнил целия свят.
\par 36 Това е сънят; и ще кажем пред царя значението му.
\par 37 Царю, ти си цар на царете, на когото небесният Бог даде царство и сила, могъщество и слава;
\par 38 и където и да живеят човеците, горските зверове, и небесните птици, Той ги е дал в твоята ръка, и те е поставил господар над всички тях. Ти си оная златна глава.
\par 39 И подир тебе ще се издигне друго царство по-долно от твоето, и друго трето царство от мед, което ще обладае целия свят.
\par 40 Ще се издигне и четвърто царство яко като желязо, понеже желязото строшава и сдробява всичко; и то ще строшава и стрива както желязото, което строшава всичко.
\par 41 А понеже си видял нозете и пръстите отчасти от грънчарска кал и отчасти от желязо, това ще бъде едно разделено царство; но в него ще има нещо от силата на желязото, понеже си видял желязото смесено с глинена кал.
\par 42 И както пръстите на нозете са били отчасти от желязо и отчасти от кал, така и царството ще бъде отчасти яко и отчасти крехко.
\par 43 И както си видял желязото смесено с глинената кал, така те ще се размесят с потомците на други родове човеци; но няма да се слеят един с друг, както желязото не се смесва с калта.
\par 44 И в дните на ония царе небесният Бог ще издигне царство, което до века няма да се разруши, и владичеството над което няма да премине към други люде; но то ще строши и довърши всички тия царства, а само то ще пребъдва до века.
\par 45 Както си видял, че камък се е отсякъл от планината, не с ръце, и че е разтрил желязото, медта, калта, среброто, и златото, великият Бог открива на царя онова, което има да стане по-после. Сънят е истинен и тълкуванието му вярно.
\par 46 Тогава цар Навуходоносор падна на лице та се поклони на Даниила, и заповяда да му принесат принос и кадения.
\par 47 Царят, отговаряйки на Даниила, рече: Наистина вашият Бог е Бог на боговете и Господ на царете, и откривател на тайни, тъй като ти можа да откриеш тая тайна.
\par 48 Тогава царят възвеличи Даниила, даде му много и големи подаръци, и го постави управител над цялата Вавилонска област и началник на управителите над всичките вавилонски мъдреци.
\par 49 И Даниил измоли от царя, и той постави Седраха, Мисаха, и Авденаго над работите на вавилонската област; а Даниил беше в царския дворец.

\chapter{3}

\par 1 Цар Навуходоносор направи златен образ, шестдесет лакти висок и шест лакти широк, и го постави на полето Дура, във вавилонската област.
\par 2 Тогава цар Навуходоносор прати да съберат сатрапите, наместниците, областните управители, съдиите, съкровищниците, съветниците, законоведците, и всичките началници на областите да дойдат на посвещението на образа, който цар Навуходоносор бе поставил.
\par 3 Тогава сатрапите, наместниците, областните управители, съдиите, съкровищниците, съветниците, законоведците и всичките началници на областите се събраха на посвещението на образа, който цар Навуходоносор бе поставил; и застанаха пред образа, който Навуходоносор бе поставил.
\par 4 Тогава глашатай викаше със силен глас: Вам се заповядва, племена, народи, и езици,
\par 5 щото когато чуете звука на тръбата, на свирката, на арфата, на китарата, на псалтира, на гайдата, и на всякакъв вид музика, да паднете та да се поклоните на златния образ, който Навуходоносор е поставил;
\par 6 а който не падне да се поклони, в същия час ще бъде хвърлен всред пламенната огнена пещ.
\par 7 За това, когато всичките племена чуха звука на тръбата, на свирката, на арфата, на китарата, на псалтира, и на всякакъв вид музика, всичките племена, народи, и езици падаха и се кланяха на златния образ, който цар Навуходоносор бе поставил.
\par 8 Тогава някои халдейци се приближиха при царя, та наклеветиха юдеите,
\par 9 като проговориха казвайки на цар Навуходоносора: Царю, да си жив до века!
\par 10 Ти царю, си издал указ, щото всеки човек, който чуе звука на тръбата, на свирката, на арфата, на китарата, на псалтира, на гайдата, и на всякакъв вид музика, да падне и да се поклони на златния образ,
\par 11 а който не падне и не се поклони да бъде хвърлен всред пламенната огнена пещ.
\par 12 Има някои юдеи, които ти си поставил над работите на вавилонската област, Седрах, Мисах, и Авденаго, които човеци, царю, не те зачетоха; на боговете ти не служат, и на златния образ, който си поставил, не се кланят.
\par 13 Тогава Навуходоносор с гняв и ярост заповяда да докарат Седраха, Мисаха и Авденаго. И докараха тия човеци пред царя.
\par 14 Навуходоносор проговаряйки, рече им: Седрахе, Мисахе, и Авденаго, нарочно ли не служите на моя бог, и не се кланяте на златния образ, който поставих?
\par 15 Сега, като чуете звука на тръбата, на свирката, на арфата, на китарата, на псалтира, на гайдата, и на всякакъв вид музика, ако сте готови да паднете и се поклоните на образа, който съм направил, добре; но ако не се поклоните, в същия час ще бъдете хвърлени всред пламенната огнена пещ; и кой е оня бог, който ще ви отърве от ръцете ми?
\par 16 Седрах, Мисах, и Авденаго рекоха в отговор на царя: Навуходоносоре, нам  не ни трябва да ти отговаряме за това нещо.
\par 17 Ако е така нашият Бог, Комуто ние служим, може да ни отърве от пламенната огнена пещ и от твоите ръце, царю, ще ни избави;
\par 18 но ако не, пак да знаеш, царю, че на боговете ти няма да служим, и на златния образ, които си поставил, няма да се кланяме.
\par 19 Тогава Навуходоносор се изпълни с ярост, и изгледът на лицето му се измени против Седраха, Мисаха и Авденаго, та проговаряйки, заповяда да нажежат пещта седем пъти повече, отколкото обикновено се нажежаваше.
\par 20 И на някои силни мъже от войската си заповяда да вържат Седраха, Мисаха и Авденаго, и да ги хвърлят в пламенната огнена пещ.
\par 21 Тогава тия мъже бидоха вързани с шалварите си, хитоните си, мантиите си, и другите си дрехи, и бяха хвърлени всред пламенната огнена пещ.
\par 22 А понеже царската заповед бе настойчива, и пещта се нажежи премного, огненият пламък уби ония мъже, които вдигнаха Седраха, Мисаха и Авденаго.
\par 23 А тия трима мъже, Седрах, Мисах и Авденаго, паднаха вързани всред пламенната огнена пещ.
\par 24 Тогава цар Навуходоносор, ужасен, стана бърже, и като продума, рече на съветниците си: Не хвърлихме ли всред огъня трима мъже вързани? Те отговаряйки, рекоха на царя: Вярно е, царю.
\par 25 В отговор той рече: Ето, виждам четирима мъже развързани, които ходят всред огъня, без да имат някаква повреда; и по изгледа си четвъртият прилича на син на боговете.
\par 26 Тогава Навуходоносор се приближи до устието на пламенната огнена пещ, и проговаряйки рече: Седрахе, Мисахе и Авденаго, слуги на всевишния Бог, излезте и дойдете тук. Тогава Седрах, Мисах и Авденаго излязоха изсред огъня.
\par 27 И като се събраха сатрапите, наместниците, областните управители, и царските съветници, видяха, че огънят не бе имал сила върху телата на тия мъже, косъм от главата им не бе изгорял, и шалварите им не бяха се изменили, нито даже миризма от огън не бе преминала на тях.
\par 28 Навуходоносор продумайки, рече: Благословен да бъде Бог Седрахов, Мисахов и Авденагов, който изпрати ангела си и избави слугите си, които, като уповаха на него, не послушаха думата на царя, но предадоха телата си, за да не служат, нито да се поклонят на друг бог, освен на своя си Бог.
\par 29 За това, издавам указ, щото всеки човек, от които и да било люде, народ и език, който би казал зло против Бога на Седраха, Мисаха и Авденаго, да се разсече, и къщата му да се обърне на бунище; защото друг бог няма, който може да избави така.
\par 30 Тогава царят повиши Седраха, Мисаха и Авденаго във вавилонската област.

\chapter{4}

\par 1 Цар Навуходоносор към всичките племена, народи и езици, които живеят по цял свят - Мир да ви се умножи!
\par 2 Видя ми се за добре да оповестя знаменията и чудесата, които ми направи всевишният Бог.
\par 3 Колко са велики Неговите знамения! И колко могъщи са чудесата му! Неговото царство е вечно царство, и Неговото владичество из род в род.
\par 4 Аз Навуходоносор, като бях спокоен у дома си и благополучен в палата си,
\par 5 видях сън, който ме уплаши; и размишленията ми на леглото ми и виденията на главата ми ме смутиха.
\par 6 За това, издадох указ да се въведат пред мене всичките вавилонски мъдреци, за да ми явят значението на съня.
\par 7 Тогава влязоха врачовете, вражарите, халдейците, и астролозите, и аз разказах съня си пред тях; но не можаха да ми явят значението му.
\par 8 А най- после дойде пред мене Даниил, чието име бе Валтасасар по името на моя бог, и в когото е духът на светите богове; и аз разказах съня пред него, като рекох:
\par 9 Валтасасаре, началниче на врачовете, понеже узнах, че духът на светите богове е в тебе, и че никаква тайна не е мъчна за тебе, обясни ми виденията на съня, който видях, и кажи ми значението му.
\par 10 Ето какви бяха виденията на главата ми и на леглото ми: Гледах, и ето дърво всред света, на което височината бе голяма.
\par 11 Това дърво стана голямо и яко, височината му стигаше до небето, и то се виждаше до краищата на целия свят.
\par 12 Листата му бяха хубави, плодът му изобилен, и в него имаше храна за всички; под сянката му почиваха полските животни, и на клоновете му обитаваха небесните птици, и от него се хранеше всяка твар.
\par 13 Видях във виденията на главата си на леглото си, и ето, един свет страж слезе от небето,
\par 14 и извика със силен глас, казвайки така: Отсечете дървото и изсечете клоновете му; отърсете листата му и разпръснете плода му; нека бягат животните изпод него, и птиците от клоновете му;
\par 15 обаче оставете в земята пъна с корените му всред полската трева, и то с железен и меден обръч наоколо, и нека се мокри с небесна роса, и участта му нека бъде с животните в тревата на земята;
\par 16 нека се измени човешкото му сърце, и нека му се даде животинско сърце; и така нека минат над него седем времена.
\par 17 Тая присъда е по заповед от стражите, и делото на думата на светите, за да знаят живите, че Всевишният владее над царството на човеците, дава го комуто ще, и поставя над него най-нищожният измежду човеците.
\par 18 Тоя сън видях аз цар Навуходоносор, и ти, Валтасасаре, кажи значението му; защото ни един от мъдреците на царството ми не може да ми яви значението; а ти можеш, защото духът на светите богове е в тебе.
\par 19 Тогава Даниил, чието име бе Валтасасар, остана смаян за малко, и размишленията му го смущаваха. Царят продумайки, рече: Валтасасаре, да те не смущава сънят или значението му. Валтасасар рече в отговор: Господарю мой, сънят нека бъде върху ония, които те мразят, и това, което означава, върху неприятелите ти!
\par 20 Дървото, което си видял, че станало голямо и яко, чиято височина стигала до небето, и което се виждало от целия свят,
\par 21 чиито листа били хубави и плодът му изобилен, дори достатъчна храна за всички, под което живеели полските животни, и по клоновете на което се подсланяли небесните птици,
\par 22 това дърво си ти, царю, който си станал голям и як; защото величието ти нарасна и стигна до небето, и владичеството ти до края на света.
\par 23 А дето царят е видял един свет страж да слиза от небето и да казва : Отсечете дървото и го съборете, но оставете в земята, в полската трева, пъна с корените му, и то със железен и меден обръч наоколо, и нека се мокри от небесната роса, и нека бъде участта му с полските животни, докато така минат над него седем времена, -
\par 24 ето значението му, царю: Решението на Всевишния, което постигна господаря ми царя, е
\par 25 да бъдеш изгонен измежду човеците, жилището ти да бъде с полските животни, да ядеш трева като говедата, и да те мокри небесната роса, и да минат над тебе седем времена, догдето познаеш, че Всевишният владее в царството на човеците, и го дава комуто ще.
\par 26 А дето се заповядало да оставят пъна с корените на дървото, значи, че царството ти ще бъде обезпечено щом признаеш, че небесата владеят.
\par 27 Затова, царю, нека ти бъде угоден моят съвет да напуснеш греховете си чрез вършене на правда, и беззаконията си чрез правене благодеяния на бедните, негли се продължи благоденствието ти.(Или: се прости престъплението ти)
\par 28 Всичко това постигна цар Навуходоносора.
\par 29 В края на дванадесет месеца, като ходеше по царския палат у Вавилон,
\par 30 царят проговори, казвайки: Не е ли велик тоя Вавилон, който аз съградих с мощната си сила за царското жилище и за славата на величието си!
\par 31 Думата бе още в устата на царя, и глас дойде от небесата, който рече: На тебе се известява, царю Навуходоносоре, че царството премина от тебе;
\par 32 и ще бъдеш изгонен измежду човеците, между полските животни ще бъде жилището ти, и ще те хранят с трева като говедата; и седем времена ще минат над тебе, додето признаеш, че Всевишният владее над царството на човеците, и го дава комуто ще.
\par 33 В същия час това нещо се изпълни над Навуходоносора; той бе изгонен измежду човеците, ядеше трева като говедата, и тялото му се мокреше от небесната роса, додето космите му пораснаха като пера на орли, и ноктите му като на птици.
\par 34 А в края на тия дни аз Навуходоносор повдигнах очите си към небесата; и разумът ми се възвърна като благослових Всевишния и похвалих и прославих Оногова, който живее до века, понеже владичеството Му е вечно владичество, и царството Му из род в род;
\par 35 пред Него всичките земни жители се считат като нищо; по волята Си Той действа между небесната войска и между земните жители; и никой не може да спре ръката Му, или да Му каже - Що правиш Ти?
\par 36 И в същото време, когато разумът ми се възвърна, възвърнаха ми се, за славата на царството ми, и величието и светлостта ми; защото съветниците ми и големците ми ме търсеха, закрепих се на царството си, и ми се притури превъзходно величие.
\par 37 Сега аз Навуходоносор хваля, превъзнасям, и славя небесния Цар; защото всичко, що върши е с вярност, и пътищата му са справедливи; а Той може да смири ония, които ходят горделиво.

\chapter{5}

\par 1 Цар Валтасар направи голям пир на хиляда от големците си, и пиеше вино пред хилядата.
\par 2 Когато пиеше виното, Валтасар заповяда да донесат златните и сребърните съдове, които дядо му Навуходоносор беше задигнал от храма, който бе в Ерусалим, за да пият с тях царят и големците му, жените му и наложниците му.
\par 3 Тогава донесоха златните съдове, които бяха задигнати от храма на Божия дом, който бе в Ерусалим; и с тях пиеха царят и големците му, жените му и наложниците му.
\par 4 Пиеха вино, и хвалеха златните, сребърните, медните, железните, дървените и каменните богове.
\par 5 В същия час се появиха пръсти на човешка ръка, които пишеха, срещу светилника, по мазилката на стената на царския палат; и царят видя тая част от ръката, която пишеше.
\par 6 Тогава изгледът на лицето на царя се измени и мислите му го смущаваха, така щото ставите на кръста му се разхлабиха, и колената му се удряха едно о друго.
\par 7 Царят извика с голям глас да въведат вражарите, халдейците, и астролозите, на които вавилонски мъдреци царят проговаряйки, рече: Който прочете това писание и ми яви значението му ще бъде облечен в багреница, златна огърлица ще се окачи около шията му, и ще бъде един от тримата, които ще владеят царството.
\par 8 Влязоха прочее всичките царски мъдреци; но не можаха да прочетат написаното, нито да явят на царя значението му.
\par 9 Тогава цар Валтасар се смути много, изгледът на лицето му се измени, и големците му се смаяха.
\par 10 От думите на царя и на големците му овдовялата царица влезе в къщата на пируването; и царицата проговаряйки, рече: Царю, да си жив до века! Да те  не смущават мислите ти, нито да се изменява лицето ти.
\par 11 Има човек в царството ти, в когото е духът на светите богове; и в дните на дядо ти, в него се намериха светлина, разум, и мъдрост като мъдростта на боговете; и дядо ти Навуходоносор, -дядо ти, казвам, царю, - го постави началник на врачовете, на вражарите, на халдейците, и на астролозите,
\par 12 защото превъзходен дух и знание, разум за тълкуване сънища, изясняване на гатанки, и разрешаване на недоумения се намираха в тоя Даниил, когото царят преименува Валтасасар. Сега нека се повика Даниил, и той ще покаже значението на написаното.
\par 13 Тогава Даниил биде въведен пред царя. Царят проговаряйки, рече на Даниила: Ти ли си оня Даниил, който си от пленените юдейци, които дядо ми царят доведе от Юдея?
\par 14 Чух за тебе, че духът на боговете бил в тебе, и че светлина, разум, и превъзходна мъдрост се намирали в тебе.
\par 15 И сега бяха въведени пред мене мъдреците и вражарите, за да прочетат това писание и ми явят значението му; но не можаха да покажат значението на това нещо.
\par 16 Но аз чух за тебе, че си могъл да тълкуваш, и да разрешаваш недоумения; прочее, ако можеш да прочетеш написаното и да ми явиш значението му, ще бъдеш облечен в багреница, златна огърлица ще се окачи около шията ти, и ти ще бъдеш един от тримата, които ще владеят царството.
\par 17 Тогава в отговор Даниил рече пред царя: Подаръците ти нека останат за тебе, и дай на другиго наградите си; все пак аз ще прочета на царя написаното, и ще му явя значението му.
\par 18 Царю, Всевишният Бог даде на дядо ти Навуходоносора царство и величие, сила и чест;
\par 19 и поради величието, което му даде, всичките племена, народи, и езици трепереха и се бояха пред него; когото искаше убиваше, и когото искаше опазваше жив, когото искаше възвишаваше, и когото искаше унижаваше.
\par 20 Но когато се надигна сърцето му, и духът му закоравя та да постъпва гордо, той биде свален от царския си престол, славата му се отне от него,
\par 21 и биде изгонен измежду човеците; сърцето му стана като на животните, и жилището му бе между живите осли; хранеха го с трева като говедата, и тялото му се мокреше от небесната роса, додето призна, че Всевишният Бог владее над царството на човеците, и когото иска поставя над него.
\par 22 А ти, негов внук, Валтасаре, не си смирил сърцето си, ако и да знаеше всичко това,
\par 23 но си се надигнал против небесния Господ; и донесоха пред тебе съдовете на дома му, с които пиехте вино ти и големците ти, жените ти и наложниците ти; и ти славослови сребърните, златните, медните, железните, дървените и каменните богове, които не виждат, нито чуват, нито разбират; а Бога, в Чиято ръка е дишането ти, и в Чиято власт са всичките твои пътища, не си възвеличил.
\par 24 Затова, от Него е била изпратена частта от ръката, и това писание се написа.
\par 25 И ето писанието, което се написа: М'не, М'не, Т'кел, у Фарсин.
\par 26 Ето и значението на това нещо: М'не, Бог е преброил дните на твоето царство, и го е свършил;
\par 27 Т'кел, Претеглен си на везните, и си бил намерен недостатъчен;
\par 28 П'рес, Раздели се царството ти, и се даде на мидяните и персите.
\par 29 Тогава Валтасар заповяда, та облякоха Даниила в багреницата, окачиха златната огърлица около шията му, и прогласиха за него да бъде един от тримата, които да владеят царството.
\par 30 В същата нощ царят на халдейците Валтасар биде убит.
\par 31 И мидянинът Дарий, който бе около на шестдесет и две години, взе царството.

\chapter{6}

\par 1 Видя се добре на Дария да постави над царството сто и двадесет сатрапи, които да бъдат из цялото царство,
\par 2 и над тях трима председатели, един от които бе Даниил, за да дават тия сатрапи сметка на тях, и тъй царят да не губи.
\par 3 В това време тоя Даниил се отличаваше от другите председатели и сатрапи, защото имаше в него превъзходен дух; и царят намисли да го постави над цялото царство.
\par 4 Тогава председателите и сатрапите се стараеха да намерят причина против Даниила относно делата на царството; но не можаха да намерят никаква причина или вина, защото той бе верен, и в него не се намери никаква погрешка или вина.
\par 5 И тъй, тия човеци си рекоха: Няма да намерим никаква причина против тоя Даниил, освен ако намерим нещо относно закона на неговия Бог.
\par 6 Прочее, тия председатели и сатрапи се събраха при царя и му рекоха така: Царю Дарие, да си жив до века!
\par 7 Всичките председатели на царството, наместниците и сатрапите, съветниците и управителите като се съветваха, решиха да поискат от царя да издаде указ и да обяви строга забрана, че който, до тридесет дена, би отправил някаква просба до кой да било бог или човек, освен до тебе, царю, той да се хвърли в рова на лъвовете.
\par 8 Сега, царю, утвърди забраната и подпиши писмената й форма, за да не се измени, според закона на мидяните и персите, който не се изменява.
\par 9 За това, цар Дарий подписа писмената забрана.
\par 10 А Даниил, щом се научи, че била подписана писмената забрана, влезе у дома си, и, като държеше прозорците на стаята си отворени към Ерусалим, падаше на колената си три пъти на ден, молещ се и благодарящ пред своя Бог, както правеше по-напред.
\par 11 Тогава ония човеци се събраха демонстративно и намериха, че Даниил отправяше просба и се молеше пред своя Бог.
\par 12 За това, приближиха се и говориха пред царя за царската забрана, като рекоха: Не подписа ли ти забрана, че всеки човек, който до тридесет дена би отправил просба до кой да било бог или човек, освен до тебе, царю, ще се хвърли в рова на лъвовете? Царят в отговор рече: Това е вярно, според закона на мидяните и персите, който не се изменява.
\par 13 Тогава отговаряйки, те рекоха пред церя: Оня Даниил, който е от пленените юдейци, не зачита ни тебе, царю, нито подписаната от тебе забрана, но принася молбата си три пъти на ден.
\par 14 Тогава царят, като чу тия думи, наскърби се много, и тури присърце да отърве Даниила; и трудеше се до захождането на слънцето да го избави.
\par 15 Тогава ония човеци се събраха при царя и му рекоха: Знай, царю, че е закон на мидяните и на персите какво никаква забрана или повеление, което царят постави, да не се изменява.
\par 16 Тогава царят заповяда та докараха Даниила и го хвърлиха в рова на лъвовете. А царят проговаряйки, рече на Даниила: Твоят Бог, Комуто ти служиш непрестанно, Той ще те отърве.
\par 17 После, като донесоха камък и го поставиха на устието на рова, царят го запечати със своя си печат и с печата на големците си, за да се не измени никакво намерение относно Даниила.
\par 18 Тогава царят отиде в палата си и пренощува гладен, нито остави да донесат пред него музикални инструменти; и сънят побягна от него.
\par 19 И на утринта царят стана много рано и побърза да отиде при рова на лъвовете.
\par 20 И като се приближи при рова извика с плачевен глас на Даниила: Данииле, служителю на живия Бог, твоят Бог, Комуто ти служиш непрестанно, можа ли да те отърве от лъвовете?
\par 21 Тогава Даниил рече на царя: Царю, да си жив до века!
\par 22 Моят Бог прати ангела Си да затули устата на лъвовете, та не ме повредиха, защото се намерих невинен пред Него; още и пред тебе, царю, не съм сторил никакво прегрешение.
\par 23 Тогава царят се зарадва много, и заповяда да извадят Даниила из рова. И когато Даниил бе изваден из рова, никаква повреда не се намери на него, защото бе уповал на своя Бог.
\par 24 Тогава, по заповед на царя, докараха ония човеци, които бяха наклеветили Даниила, и хвърлиха тях, чадата им, и жените им в рова на лъвовете; и преди да стигнат до дъното на рова лъвовете им надвиха и счупиха всичките им кости.
\par 25 Тогава цар Дарий писа до всички племена, народи, и езици, които живеят по целия свят: Мир да се умножи на вас!
\par 26 Издавам указ, щото в цялата държава, над която царувам, да треперят човеците и да се боят пред Данииловия Бог; защото той е живият Бог, който е утвърден до века, и Неговото царство е царство, което няма да се разруши, и властта Му ще трае до край. /
\par 27 Той избавя и отървава, и върши знамения и чудеса на небесата и на земята; Той е, който отърва Даниил от силата на лъвовете.
\par 28 И тъй, тоя Даниил благоденстваше в царуването на Дария и в царуването на персиеца Кир.

\chapter{7}

\par 1 В първата година на вавилонския цар Валтасар Даниил видя сън и видения на главата си върху леглото си. Тогава написа сънят и разказа същността на работите.
\par 2 Даниил проговаряйки, рече: Видях в нощното си видение, и ето, четирите небесни ветрища избухнаха върху голямото море.
\par 3 И четири големи зверове възлязоха из морето, различни един от друг.
\par 4 Първият бе като лъв и имаше орлови крила; а, като го гледах, крилата му се изскубаха, и той се издигна от земята и биде заставен да се изправи на две нозе като човек, и даде му се човешко сърце.
\par 5 След това, ето друг звяр, втори, приличен на мечка, който се повдигна от едната страна, и имаше три ребра в устата си между зъбите си; и му думаха така: Стани, изяж много месо.
\par 6 Подир това, като погледнах, ето друг звяр, приличен на леопард, който имаше на гърба си четири птичи крила: тоя звяр имаше и четири глави; и даде му се власт.
\par 7 Подир това, като погледнах в нощните видения, ето четвъртият звяр, страшен и ужасен и твърде як; той имаше големи железни зъби, с които пояждаше и сломяваше, като стъпкваше останалото с нозете си; той се различаваше от всичките зверове, които бяха преди него; и имаше десет рога.
\par 8 Като разглеждах роговете, ето, между тях излезе друг рог, малък, пред който три от първите рогове се изкорениха; и ето, в тоя рог имаше очи като човешки очи, и уста, които говореха надменно.
\par 9 Гледах, додето се положиха престоли, и Древният по Дни седна, чието облекло беше бяло като сняг, и космите на главата му като чиста вълна, престолът му огнени пламъци, и колелата му пламенен огън.
\par 10 Огнена река излизаше и течеше изпред него; милион служители му слугуваха, и  милиарди по милиарди стояха пред него; съдилището се откри, и книгите се отвориха.
\par 11 Тогава погледнах по причина на гласа на надменните думи, които рогът изговаряше; гледах додето звярът биде убит, и тялото му погубено и предадено да се изгори с огън.
\par 12 А колкото за другите зверове, тяхното владичество биде отнето; животът им обаче се продължи до време и година.
\par 13 Гледах в нощните видения, и ето, един като човешки син идеше с небесните облаци и стигна до Древния по Дни; и доведоха го пред Него.
\par 14 И Нему се даде владичество, слава и царство, за да му слугуват всичките племена, народи и езици. Неговото владичество е вечно владичество, което няма да премине, и царството Му е царство, което няма да се разруши.
\par 15 Колкото за мене Даниила, духът ми се наскърби дълбоко в тялото ми, и виденията на главата ми ме смутиха.
\par 16 Приближих се до едного от предстоящите и го попитах що е истинското значение на всичко това. И той ми разправи и ми даде да разбера значението на тия неща.
\par 17 Тия четири големи зверове, каза той, са четирима царе, които ще се издигнат от земята.
\par 18 Но светиите на Всевишния ще приемат царството, и ще владеят царството до века и до вечни векове.
\par 19 Тогава поисках да узная истината за четвъртия звяр, който се различаваше от всичките други и бе твърде страшен, чиито зъби бяха железни и ноктите му медни, който пояждаше и строшаваше, а останалото стъпкваше с нозете си, -
\par 20 и за десетте рога, които бяха на главата му, и за другия рог, който излезе, и пред който паднаха три, тоест, за оня рог, който имаше очи и уста, които говореха надменно, и който наглед бе по-як от другарите си.
\par 21 Гледах същия рог като воюваше със светиите и превъзмогваше против тях,
\par 22 докато дойде Древният по Дни, и се извърши съд за светиите на Всевишния, и настана времето, когато светиите завладяха царството.
\par 23 Той каза така: Четвъртият звяр ще бъде четвърто царство на света, което ще се различава от всичките царства, и ще погълне целия свят, и ще го стъпче и разтроши.
\par 24 А за десетте рога, те са десет царе, които ще се издигнат от това царство; и след тях ще се издигне друг, който ще се различава от първите, и ще покори трима царе.
\par 25 Той ще говори думи против Всевишния, ще изтощава светиите на Всевишния, и ще замисли да промени времена и закони; и те ще бъдат предаде в ръката му до време и времена и половина време.
\par 26 Но когато съдилището ще заседава, ще му отнемат владичеството, за да го изтребят и погубят до край.
\par 27 А царството и владичеството и величието на царствата, които са под цялото небе, ще се дадат на людете, които са светиите на Всевишния, Чието царство е вечно царство, и на Когото всичките владичества ще служат и се покоряват.
\par 28 Тук е краят на това нещо. Колкото за мене, Даниила, размишленията ми ме смущават много, и изгледът на лицето ми се измени; но запазих това нещо в сърцето си.

\chapter{8}

\par 1 В третата година от царуването на цар Валтасара видение ми се яви, на мене Даниила, подир онова, което бе ми се явило по-напред.
\par 2 Видях във видението, (а когато видях, бях в замъка Суса, който е в областта Елам; и когато видях във видението бях при потока Улай), -
\par 3 прочее повдигнах очите си и видях, и, ето, стоеше пред реката един овен, който имаше два рога; роговете бяха високи, но единият по-висок от другия; и по-високият беше израснал по-после.
\par 4 Видях, овенът че бодеше към запад, към север, и към юг; и никой звяр не можеше да устои пред него, и нямаше кой да избави от силата му; но постъпваше по волята си, и се носеше горделиво.
\par 5 А като размишлявах, ето, козел идеше от запад по лицето на целия свят, без да се допира до земята; и козелът имаше бележит рог между очите си.
\par 6 Той дойде до овена, който имаше двата рога, който видях да стои пред потока, и завтече се върху него с устремителната си сила.
\par 7 И видях, че се приближи при овена и се разсвирепи против него, и като удари овена счупи двата му рога; и нямаше сила в овена да устои пред него; но го хвърли на земята и го стъпка; и нямаше кой да отърве овена от силата му.
\par 8 За това, козелът се възвеличи много; а когато заякна, големият му рог се счупи, вместо който излязоха четири бележити рога към четирите небесни ветрища.
\par 9 И из единия от тях излезе един малък рог, който наголемя твърде много към юг, към изток и към славната земя.
\par 10 Той се възвеличи дори до небесната войска, и хвърли на земята част от множеството и от звездите и ги стъпка.
\par 11 Да! Възвеличи се дори до началника на множеството, и отне от него всегдашната жертва; и мястото на светилището му се съсипа.
\par 12 И множеството биде предадено нему заедно с всегдашната жертва, поради престъплението им; и той тръшна на земята истината, стори по волята си, и успя.
\par 13 Тогава чух един свет да говори; и друг свет рече на тогоз, който говореше: Докога се простира видението за всегдашната жертва, и за престъплението, което докарва запустение, когато светилището и множеството ще бъдат потъпквани?
\par 14 И каза ми: До две хиляди и триста денонощия; тогава светилището ще се очисти.
\par 15 И когато аз, аз Даниил, видях видението, поисках да го разбера. И, ето, застана пред мене нещо като човешки образ;
\par 16 и чух човешки глас, който извика изсред бреговете на Улай и рече: Гаврииле, направи тоя човек да разбере видението.
\par 17 И така, той се приближи дето стоях; и когато дойде аз се уплаших и паднах на лицето си; а той ми рече: Разбери, сине човешки; защото видението се отнася до последните времена.
\par 18 И като ми говореше, аз паднах в несвяст с лицето си на земята; но той се допря до мене и ме изправи.
\par 19 И рече: Ето, аз ще те науча що има да се случи в последните гневни времена; защото видението се отнася до определеното последно време.
\par 20 Двата рога на овена, които си видял, са царете на Мидия и на Персия.
\par 21 Буйният козел е гръцкият цар; и големият рог между очите му е първият цар.
\par 22 А дето се строшил той и излезли четири вместо него, значи, че четири царе ще се издигнат от тоя народ, но не със сила като неговата.
\par 23 И в послешните времена на царуването им, когато беззаконниците стигнат до върха на беззаконието си, ще се издигне цар с жестоко лице и вещ в лукавщини.
\par 24 И силата му ще бъде голяма, но не като силата на другия; и ще погубва чудесно, ще успява и върши по волята си, и ще поквари силните и светите люде.
\par 25 Чрез коварството си ще направи да успява измамата в ръката му, ще се надигне в сърцето си, и ще погуби мнозина в спокойствието им; ще въстане и против Началника на началниците; но ще бъде смазан, не с ръка.
\par 26 А казаното видение за денонощията е вярно; все пак обаче запечатай видението, защото се отнася до далечни дни.
\par 27 Тогава аз Даниил премрях, и боледувах няколко дни; после станах и вършех царевите работи. А чудех се за видението, защото никой не го разбираше.

\chapter{9}

\par 1 В първата година на Дария Асуировия син, от рода на мидяните, който се постави цар над Халдейската държава, -
\par 2 в първата година от царуването му аз Даниил разбрах от свещените книги числото на годините, за които дойде Господното слово към пророк Еремия, че запустението на Ерусалим ще трае седемдесет години.
\par 3 Тогава обърнах лицето си към Господа Бога, за да отправя към него молитва и молби с пост, вретище и пепел.
\par 4 Когато се помолих на Господа моя Бог и се изповядах, рекох: О, Господи, велики и страшни Боже, Който пазиш завета и милостта Си към ония, които Те любят и пазят Твоите заповеди!
\par 5 Съгрешихме, постъпихме извратено, вършихме нечестие, бунтувахме се, и се отклонихме от Твоите заповеди и от Твоите съдби;
\par 6 и не послушахме слугите Ти пророците, които говориха в Твое име на царете ни и на началниците ни, на бащите ни и на всичките люде на земята ни.
\par 7 На Тебе, Господи, прилича правда, а на нас срам на лицето, както е и днес, на юдовите мъже, на ерусалимските жители, и на целия Израил, както и на ония, които са ближни, така и на ония, които са далечни по всички страни дето си ги изгонил подир престъплението, което извършиха против Тебе.
\par 8 Господи, срам на лицето подобава на нас, на царете ни, на началниците ни, и на бащите ни, защото Ти съгрешихме.
\par 9 На Господа нашия Бог принадлежат милост и прощение, защото се възбунтувахме против Него,
\par 10 и не послушахме гласа на Господа Нашия Бог да ходим по законите, които положи пред нас чрез слугите си пророците.
\par 11 Да! Целият Израил престъпи закона Ти, като се отклони та не послуша гласа Ти; за която причина се изля върху нас проклетията и клетвата, написана в закона на Божия слуга Мойсей; защото му съгрешихме.
\par 12 Той потвърди думите, които изговори против нас, и против нашите съдии, които са ни съдили, като докара върху нас голямо зло; защото никъде под небето не е ставало това, което стана на Ерусалим.
\par 13 Всичкото това зло, както е написано в Мойсеевия закон, дойде върху нас; но пак не се молихме пред Господа нашия Бог, за да се върнем от беззаконията си и да постъпваме разумно според истината Ти.
\par 14 Затова, Господ е бдял за това зло, и вече го е докарал върху нас; защото Господ нашият Бог е справедлив във всичките дела, които върши; и ние не послушахме гласа Му.
\par 15 И сега, Господи Боже наш, който си извел людете си из египетската земя с мощна ръка, и си спечелил за себе си име, каквото имаш днес, съгрешихме и вършихме нечестие.
\par 16 Господи, според всичката Твоя правда, моля се, нека гневът и яростта Ти се отвърнат от твоя град Ерусалим, светия Ти хълм; защото поради нашите грехове и поради беззаконията на бащите ни Ерусалим и твоите люде станахме за укор на всички, които са около нас.
\par 17 Сега прочее, послушай, Боже наш, молитвата на слугата ти и молбите му, и, заради Господа, осияй с лицето си върху запустялото си светилище.
\par 18 Боже мой, приклони ухото си и послушай; отвори очите си и виж опустошенията ни, и града, който се нарича с Твоето име; защото ние не принасяме прошенията си пред Тебе заради нашата правда, но заради многото Твои щедроти.
\par 19 Господи, послушай; Господи, прости; Господи, дай внимание и подействай; да не закъснееш, заради Себе Си, Боже мой; защото с Твоето име се наричат града Ти и людете Ти.
\par 20 И докато още говорех, и се молех, и изповядвах своя грях и греха на людете си Израиля, и принасях молбата си пред Господа моя Бог за светия хълм на моя Бог,
\par 21 дори като още говорех в молитвата, мъжът Гавриил, когото бях видял във видението по-напред, като летеше бързо се приближи до мене около часа на вечерната жертва.
\par 22 И вразуми ме като говори с мене, казвайки: Данииле, сега излязох да те направя способен да разбереш.
\par 23 Когато ти почна да се молиш заповедта излезе; и аз дойдох да ти кажа това, защото си възлюбен; за това, размисли за работата и разбери видението.
\par 24 Седемдесет седмици са определени за людете ти и за светия ти град за въздържането на престъплението, за довършване на греховете, и за правене умилостивение за беззаконието,  и да се въведе вечна правда, да се запечата видението и пророчеството, и да се помаже Пресветия.
\par 25 Знай, прочее, и разбери, че от излизането на заповедта да се съгради изново Ерусалим до княза Месия ще бъдат седем седмици; и за шестдесет и две седмици ще се съгради изново, с улици и окоп, макар в размирни времена.
\par 26 И подир шестдесет и две седмици Месия ще бъде посечен, и не ще има кои да му принадлежат; и людете на княза, който ще дойде, ще погубят града и светилището; и краят му ще го постигне чрез потоп; и до края на войната има определени опустошения.
\par 27 и той ще потвърди завет с мнозина за една седмица; а в половината на седмицата ще направи да престанат жертвата и приносът; и един, който запустява, ще дойде яздещ на крилото на мерзостите; и гняв ще се излее върху запустителя до определеното време.

\chapter{10}

\par 1 В третата година на персийския цар Кир едно нещо се откри на Даниила,който се нарече Валтасасар; това нещо бе истинно, и означаваше големи бедствия; и той разбра нещото и проумя видението,
\par 2 (В онова време аз Даниил жалеех цели три седмици;
\par 3 вкусен хляб не ядях, месо и вино не влизаше в устата ми, и ни веднъж не помазах себе си, додето не се навършиха цели три седмици.)
\par 4 И на двадесет и четвъртия ден от първия месец, като бях при брега на голямата река, която е Тигър,
\par 5 като повдигнах очите си видях, и ето един човек облечен в ленени дрехи, чийто кръст бе опасан с чисто уфазско злато.
\par 6 Тялото му бе като хрисолит, лицето му като изгледа на светкавица, очите му като огнени светила, мишците и нозете му бяха наглед като лъскава мед, и гласът на думите му като глас на много народ.
\par 7 Само аз Даниил видях видението; а мъжете, които бяха с мене, не видяха видението; но голям трепет ги нападна, та побягнаха да се скрият.
\par 8 И тъй, аз останах сам да видя това голямо видение, от което не остана сила в мене, защото енергията ми се обърна в тление, та останах безсилен.
\par 9 Чух обаче гласа на думите му; и като слушах гласа на думите му аз паднах на лицето си в несвяст, с лицето си към земята.
\par 10 И, ето, ръка се допря до мене, която ме тури разклатен на коленете ми и на дланите на ръцете ми.
\par 11 И рече ми: Данииле, мъжу възлюбени, разбери думите, които ти говоря, и стой прав, защото при тебе съм изпратен сега, и когато ми изговори тая дума аз  се изправих разтреперан.
\par 12 Тогава ми рече: Не бой се, Данииле; защото от първия ден откак ти приклони сърцето си да разбираш, и да смириш себе си пред своя Бог, димите ти се послушаха; и аз дойдох поради думите ти.
\par 13 Обаче князът на персийското царство ми противостоеше двадесет и един дена; но, ето, Михаил, един от главните князе, дойде да ми помогне; аз прочее останах непотребен вече там при персийските царе,
\par 14 и сега дойдох да те направя да разбереш що има да стане с людете ти в послешните дни; защото видението се отнася до далечни дни.
\par 15 И като ми говореше тия думи, насочих лицето си към земята и останах ням.
\par 16 И, ето, един подобен на човешки син се допря до устните ми. Тогава отворих устата си та говорих, като рекох на оногоз, който стоеше пред мене: Господарю мой, от видението болките ми се върнаха, и не остана сила в мене.
\par 17 Защото как може слугата на тоя мой господар да говори с тоя мой господар? Понеже веднага не остана никаква сила в мене, па и дишане не остана в мене.
\par 18 Тогава пак се допря до мене нещо като човешки образ и ме подкрепи.
\par 19 и рече: Не бой се, мъжо възлюбени; мир на тебе! крепи се! да! крепи се! И като ми говореше аз се подкрепих, и рекох: Нека говори господарят ми, защото си ме подкрепил.
\par 20 Тогава каза: Знаеш ли защо съм дошъл при тебе? А сега ще се върна да воювам против княза на Персия; и когато изляза, ето, князът на Гърция ще дойде.
\par 21 Все пак, обаче, ще ти известя значението на това, което е написано в едно истинско писание, при все че няма кой да ми помага против тия князе, освен вашия княз Михаил.

\chapter{11}

\par 1 А в първата година на мидянина Дария аз Гавриил стоях да го укрепя и уякча.
\par 2 И сега ще ти явя истината. Ето, още трима царе ще се издигнат в Персия; и четвъртият ще бъде много по-богат от всички тях; и когато се засили чрез богатството си ще повдигне всичко против гръцкото царство.
\par 3 И ще се издигне един мощен цар, който ще царува с голяма власт, и ще действа според волята си.
\par 4 А щом се издигне той, царството му ще се съсипе и ще се раздели към четирите небесни ветрища, но не на наследниците му, нито ще владеят над толкова, над колкото той е владял; защото царството му ще се изкорени и раздели на други освен тях.
\par 5 И южният цар ще се уякчи; но един от началниците му ще стане по-силен от него и ще владее; владичеството му ще бъде голямо владичество.
\par 6 И подир няколко години ще се сдружат; и дъщерята на южния цар ще дойде при северния цар, за да направи спогодба; но тя няма да задържа силата на мишцата си; също и той няма да стои, нито мишцата му; но тя ще бъде предадена, както и ония, които я водеха, и родителят и, и оня, който я крепеше в ония времена.
\par 7 Но вместо него ще се издигне един отрасъл от корените й; и като дойде против войската ще влезе в крепостите на северния цар, ще действа против тях, и ще преодолее;
\par 8 тоже и боговете им ще докара пленници в Египет, с леяните им идоли, и с отбраните им сребърни и златни съдове; и той ще се въздържа няколко години да не на пада северния цар.
\par 9 А оня ще влезе в царството на южния цар, но ще се върне в земята си.
\par 10 И синовете му ще воюват, и ще съберат множество от големи войски, които ще дойдат с устрем, ще нахлуят, и ще заминат; а завръщайки се ще воюват дори до крепостта му.
\par 11 И южният цар ще се разсвирепее, и като излезе ще се бие с него - със северния цар, който ще опълчи едно голямо множество; и множеството ще се предаде в неговата ръка.
\par 12 И като закара множеството, сърцето му ще се надигне; и при все че повали десетки хиляди, пак няма да преодолее.
\par 13 Защото северният цар, завръщайки се, ще опълчи множество по-голямо от първото, и в края на определените години ще дойде с устрем, с голяма войска, и с много имот.
\par 14 И в ония времена мнозина ще въстанат против южния цар; ще се повдигнат и насилниците от твоите люде, за да потвърдят видението; но ще паднат.
\par 15 И тъй, северният цар ще дойде, ще издигне могила, и ще превземе укрепените градове; и нито мишците на южния цар, нито отбраните му люде не ще могат да му противостоят, нито ще има сила да противостои.
\par 16 Но оня, който иде против него, ще действа според волята си, и не ще има кой да му противостои; и ще застане в славната земя, и в ръцете му ще бъде разрушителна сила.
\par 17 И ще насочи лицето си да дойде със силата на цялото си царство, и ще му предложи справедливи условия, и ще действа според тях; а ще му даде най-отбраната дъщеря между жените, за да  го разврати, но това не ще успее, нито ще го ползва.
\par 18 После ще обърне лицето си към островите и ще завладее много от тях; но един военачалник ще направи да престане нанесения от него укор; дори, при това, ще възвърне укора му върху самия него.
\par 19 Тогава ще обърне лицето си към крепостите на своята земя; но ще се препъне и падне и няма да се намери.
\par 20 Тогава, вместо него, ще се издигне един, който ще изпрати бирник по най-славната част на царството; но в малко време ще загине, и то не чрез гняв, нито чрез бой.
\par 21 И вместо него ще се издигне един нищожен човек, комуто не ще отдадат царска почет; но той ще дойде във време, когато са спокойни, и ще завладее царството чрез ласкателство.
\par 22 И със силата на потопа ще бъдат пометени и строшени пред него, да! Още и сам съюзеният с него военачалник.
\par 23 И след като сключи с него съюз ще постъпва измамливо; защото ще възлезе и ще преодолее само с малко люде.
\par 24 Във време, когато са спокойни ще дойде в най-плодородните места на областта, и ще извърши това, което не са извършили бащите му или прадедите му; ще разпредели между тях грабеж, користи и имот; даже ще измисли хитростите си против крепостите, но само за време.
\par 25 И ще повдигне силата си и мъжеството си против южния цар с голяма войска; и южният цар ще се бие с него в битка с голяма и много силна войска; но не ще може да устои, защото ще измислят хитрости против него.
\par 26 Да! ония, които ядат от изрядните му ястия ще го погубят; и от войската му макар да е многочислена като потоп, мнозина ще паднат убити.
\par 27 А сърцата на двамата тия царе ще бъдат предадени на зло, и ще говорят лъжи на същата трапеза; но това не ще успее, понеже, при все това, краят ще бъде, в определеното време.
\par 28 Тогава ще се върне в земята си с много имот; и сърцето му ще бъде против светия завет; и като действа по волята си, ще се върне в земята си.
\par 29 На определеното време той ще се върне и дойде към юг; но последният път не ще бъде като първия;
\par 30 защото Китимски (Китим: т.е., остров Кипър) кораби ще дойдат против него; по която причина той, огорчен, наново ще се разяри против светия завет, и ще действа по волята си; даже, като се завърне, ще се споразумее с ония, които са оставили светия завет.
\par 31 И от него ще се повдигнат сили, които ще омърсят светилището, да! крепостта, ще премахнат всегдашната жертва, и ще издигнат мерзостта, която докарва запустение.
\par 32 И ще изврати с ласкателства ония, които беззаконстват против завета; но людете, които познават своя Бог, ще се укрепят и ще вършат подвизи.
\par 33 И разумните между людете ще научат мнозина; при все това, ще падат от меч и от пламък, чрез пленение и чрез разграбване, много дни.
\par 34 А когато паднат, ще им се достави малко помощ; обаче мнозина ще се присъединят към тях чрез ласкателства.
\par 35 Дори някои от разумните ще паднат, за да бъдат опитани, та да се очистят и да се избелят, до края на времето; защото и то ще стане в определеното време.
\par 36 И царят ще действа според волята си, ще се надигне и ще се възвеличи над всякакъв бог, и ще говори чудесно надменно против Бога на боговете; и ще благоденства, додето се изчерпи негодуванието; защото определеното ще се изпълни.
\par 37 И няма да зачита боговете на бащите си, нито обичната на жените богиня, нито ще зачита никакъв бог; защото ще направи себе си на по-велик от всички тях.
\par 38 А вместо него ще почита бога на крепостите; да! със злато и сребро, със скъпоценни камъни и с желателни вещи, ще почита един бог, когото бащите му не са познавали.
\par 39 С помощта на един чужд бог той ще постъпи с най-силните крепости; ще умножи слава на ония, които го признаят, и ще ги постави владетели над мнозина; и ще раздели земята между тях срещу заплата.
\par 40 И в края на времето, южният цар ще се сблъска с него; и северният цар ще дойде против него като вихрушка, с колесници и конници, и с много кораби; и като влезе в страните ще нахлуе и замине.
\par 41 Ще влезе и в славната земя, и много страни ще се разорят; но следните ще се избавят от ръката му: Едом, Моав, и главният град на Амонците.
\par 42 Ще простре ръката си върху страните; и Египетската земя няма да избегне.
\par 43 Защото ще завладее съкровищата на златото, на среброто, и на всичките скъпоценности на Египет; И либийците и етиопяните ще бъдат заставени да следват по стъпките му.
\par 44 Но известия от изток и от север ще го смутят; затова, ще излезе с голяма ярост да погуби мнозина, и да ги обрече на изтребление.
\par 45 И ще постави шатрите на палата си между моретата, върху славния свет хълм; но при все това, ще постигне края си, и не ще има кой да му помага.

\chapter{12}

\par 1 И в онова време великият княз Михаил, който се застъпва за твоите люде, ще се повдигне; и ще настане време на страдание, каквото никога не е бивало, откакто народ съществува до онова време; и в онова време твоите люде ще се отърват, - всеки, който се намери записан в книгата.
\par 2 И множеството от спящите в пръстта на земята ще се събудят, едни за вечен живот, а едни за срам и вечно презрение.
\par 3 Разумните ще сияят със светлостта на простора, и ония, които обръщат мнозина в правда като звездите до вечни векове.
\par 4 А ти, Данииле, затвори думите и запечатай книгата до края на времето, когато мнозина ще я изследват, и знанието за нея ще се умножава.
\par 5 Тогава, като погледнах аз Даниил, ето, стояха двама други, един на брега отсам реката, и един на брега оттам реката.
\par 6 И единият рече на облечения в ленени дрехи човек, който бе над водата на реката: Докога ще се чака за края на тия чудеса?
\par 7 И чух облечения в ленени дрехи човек, който бе над водата на реката, когато издигна десницата си и левицата си към небето и се закле в Оня, който живее до века, че това ще бъде след време, времена, и половина време, и че всичко това ще се изпълни, когато ще са свършили да смажат силата на светите люде.
\par 8 И аз чух, но не разбрах. Тогава рекох: Господарю, мой, каква ще бъде сетнината на това?
\par 9 А той рече: Иди си, Данииле; защото думите са затворени и запечатани до края на времето.
\par 10 Мнозина ще се чистят и избелят и ще бъдат опитани; а нечестивите ще вършат нечестие, и никой от нечестивите не ще разбере; но разумните ще разберат.
\par 11 И от времето, когато се премахне всегдашната жертва, и се постави мерзостта, която докарва запустение; ще има хиляда и двеста и деветдесет дена.
\par 12 Блажен, който изтърпи и стигне до хиляда и триста и тридесет и петте дена.
\par 13 Но ти си иди, додето настане краят; и ще се успокоиш, и в края на дните ще застанеш в дела си.

\end{document}