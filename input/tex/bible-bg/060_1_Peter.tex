\begin{document}

\title{1 Петрово}


\chapter{1}

\par 1 Петър, апостол Исус Христов, до избраните пришелци, пръснати по Понт, Галатия, Кападокия, Азия, Витиния,
\par 2 избрани по предузнанието на Бога Отца, чрез освещението на Духа, за да сте послушни и да бъдете поръсени с кръвта на Исус Христова; благодат и мир да ви се умножи.
\par 3 Благословен да бъде Бог и Отец на нашия Господ Исус Христос, Който според голямата Си милост ни възроди за жива надежда чрез въскресението на Исуса Христа от мъртвите.
\par 4 за наследство нетленно, неоскверняемо, и което не повяхва, запазено на небесата за вас,
\par 5 които с божията сила сте вардени чрез вяра за спасение, готово да се открие в последно време.
\par 6 В което блаженство се радвате, ако и за малко време да скърбите сега, (ако е потребно), в разни изпитни,
\par 7 С цел: изпитването на вашата вяра, като е по-скъпоценно от златото, което гине, но пак се изпитва чрез огън, да излезе за хвала и слава и почест, когато се яви Исус Христос;
\par 8 Когото любите без да сте Го видели, в Когото вярвате, без сега да го виждате, радвате се с неизказана и преславна радост,
\par 9 като получавате следствието на вярата си, спасението душите си.
\par 10 За това спасение претърсваха и изследваха пророците, които пророкуваха за благодатта, която бе назначена за вас;
\par 11 като издирваха, кое или какво време посочваше Христовият Дух, който беше в тях, когато предизвестяваше Христовите страдания, и след тях славите.
\par 12 И откри им се, че не за себе си, а за вас, служеха те в това, което сега ви извести чрез ония, които ви проповядваха благовестието чрез Светия Дух изпратен от небесата, в което и самите ангели желаят да надникнат.
\par 13 Затова препашете се през чреслата на вашите помисли, бъдете въздържани и имайте пълна надежда за благодатта, която ще ви се даде, когато се яви Исус Христос.
\par 14 Като послушни чада, не се съобразявайте с първите страсти, които имахте по време на незнанието си;
\par 15 но, както е свят Тоя, Който ви е призовал, така бивайте свети и вие с цялото си държание;
\par 16 защото е писано: Бъдете свети, понеже Аз съм свят.
\par 17 И ако призовавате като Отец Този, Който без лицеприятие съди, според делото на всекиго, то прекарайте със страх времето на вашето престояване,
\par 18 като знаете, че не са тленни къща - сребро или злато - сте изкупени от суетния живот, предаден вам от бащите ви,
\par 19 но със скъпоценната кръв на Христа, като агнец без недостатък и пречист,
\par 20 Който наистина беше предопределен преди създанието на света, но се яви в окончанието на времената за вас,
\par 21 които чрез Него повярвахте в Бога, Който го възкреси от мъртвите и Му е дал слава, така щото вярата и надеждата ви да бъдат в Бога.
\par 22 Понеже сте очистили душите си, като сте се покорили на истината, която докарва до нелицемерно братолюбие, обичайте се един друг горещо, от сърце,
\par 23 тъй като се възродихте, не от тленно семе, а от не тленно чрез Божието слово, което живее и трае [до века].
\par 24 Защото "Всяка твар е като трева, и всичката - слава като цвят от трева; Тревата изсъхва, и цветът - окапва,
\par 25 Но Словото божие трае до века"; и това е словото, което ви е благовестено.

\chapter{2}

\par 1 И тъй, като отхвърлите всяка злоба, всяка лукавщина, лицемерие, завист и всяко одумване,
\par 2 пожелавайте като новородени младенци чистото духовно мляко, за да пораснете чрез него към спасение,
\par 3 ако сте опитали, че Господ е благ;
\par 4 При Когото идвайки като при жив камък, от човеците отхвърлен, а от Бога избран и скъпоценен,
\par 5 и вие, като живи камъни се съграждате в духовен дом, за да станете свето свещенство, да принасяте духовни жертви, благоприятни на Бога чрез Исуса Христа.
\par 6 Защото е писано в писанието: - "Ето, полагам в Сион крайъгълен камък, избран, скъпоценен; И който вярва в Него, не ще се посрами"
\par 7 За вас, прочее, които вярвате, Той е скъпоценност, а на тия които не вярват, "Камъкът, който отхвърлиха зидарите, Той стана глава на ъгъла",
\par 8 и, "Камъкът, о който да се спъват, и канара, в която да се съблазняват"; защото се спъват о словото и са непокорни, - за което бяха определени.
\par 9 Вие, обаче, сте избран род, царско свещенство, свят народ, люде, които Бог придоби, за да възвестява превъзходствата на Този, Който ви призова от тъмнината в Своята чудесна светлина.
\par 10 вие, които някога си не бяхте народ, а сега сте Божи народ, не бяхте придобили милост, а сега сте придобили.
\par 11 Възлюбени, умоляваме ви, като пришелци и чужденци на света, да се въздържате от плътски страсти, които воюват против душата;
\par 12 да живеете благоприлично между езичниците, тъй щото относно това, за което ви одумват като злодейци, да прославят Бога в времето, когато ще ги посети, понеже виждат добрите ви дела.
\par 13 Покорявайте се заради Господа на всяка човешка власт, било на царя, като върховен владетел,
\par 14 било на управителите, като пратеници от него, за да наказват злодейците и за похвала на добротворците.
\par 15 Защото това е Божията воля, като правите добро, да затуляте устата на невежите и глупави човеци;
\par 16 като свободни, обаче, не употребяващи свободата за покривало на злото, но като Божии слуги.
\par 17 Почитайте всички; обичайте братството; от Бога се бойте, царя почитайте.
\par 18 Слуги, покорявайте се на господарите си с пълен страх, не само на добрите и кротките, но и на опърничавите;
\par 19 защото това е благоугодно, ако някой от съзнанието за Бога претърпява оскърбления, като страда несправедливо.
\par 20 Защото, каква похвала, ако понасяте търпеливо, когато ви бият за престъпленията ви? Но когато вършите добро и страдате, ако понасяте търпеливо, това е угодно пред Бога.
\par 21 Защото и на това сте призовани; понеже и Христос пострада за вас, и ви остави пример да последвате по Неговите стъпки;
\par 22 Който грях не е сторил, нито се е намерила лукавщина в устата Му;
\par 23 Който бидейки охулван, хула не отвръщаше; като страдаше, не заплашваше; но предаваше делото Си на Този, Който съди справедливо;
\par 24 Който сам понесе в тялото Си нашите грехове на дървото, тъй щото, като сме умрели за греховете, да живеем за правдата; с Чиято рана вие оздравяхте.
\par 25 Защото като овца блуждаехте, но сега се върнахте при Пастиря и Епископа на душите ви.

\chapter{3}

\par 1 Подобно и вие, жени, покорявайте се на мъжете си, така щото, даже ако някои от тях не се покоряват на словото, да се придобият без словото, чрез обходата на жените си,
\par 2 като видят, че вие се обхождате със страх и чистота.
\par 3 Вашето украшение да не е вънкашно, сиреч, плетене косата, кичене със злато, или обличане със скъпи дрехи,
\par 4 но скришом в сърцето живот (Гръцки: Човек.), с нетленното украшение на кротък и тих дух, което е скъпоценно пред Бога.
\par 5 Защото така някога и светите жени, които се надяваха на Бога, украсяваха себе си, като се покоряваха на мъжете си.
\par 6 както Сара се покоряваше на Авраама и го наричаше господар. И вие сте нейни дъщери, ако правите добро и не се боите от никакво заплашване.
\par 7 Също и вие мъжете, живейте благоразумно с жените си, като с по-слаб съсЪд, и отдавайте почит на тях като на сънаследници на дадения чрез благодат живот, за да не става препятствие на молитвите ви.
\par 8 А най-после, бъдете всички единомислени, съчувствителни, братолюбиви, милостиви, смиреномъдри.
\par 9 Не въздавайте зло за зло или хула за хула, а напротив благославяйте; понеже на това бяхте призовани, за да наследите благословение.
\par 10 Защото, "Който желае да обича живота И да види добри дни, Нека пази езика си от зло И устните си от лъжливо говорене;
\par 11 Нека се отклонява от зло, и да върши добро; Нека търси мир и да се стреми към него.
\par 12 Защото очите на Господа са върху праведните, И ушите Му към тяхната молитва; Но лицето на Господа е против ония, които вършат зло".
\par 13 И кой ще ви стори зло, ако сте ревностни за доброто?
\par 14 Но даже, ако пострадате за правдата, блажени сте; а "от тяхното застрашаване не се бойте, нито се смущавайте".
\par 15 Но почитайте със сърцата си Христа като Господ, като бъдете винаги готови да отговаряте, (но с кротост и страхопочитание), на всекиго, който ви пита за вашата надежда.
\par 16 Имайте чиста съвест, така щото във всичко, в което ви одумват да се посрамят ония, които клеветят добрата ви в Христа обхода.
\par 17 Защото е по-добре да страдате, като вършите добро, ако е такава Божията воля, а не като вършите зло.
\par 18 Защото и Христос един път пострада за греховете, Праведните за неправедните, за да ни приведе при Бога, бидейки умъртвен по плът, а оживотворен по дух;
\par 19 с който отиде да проповядва на духовете в тъмницата,
\par 20 които едно време бяха непокорни, когато Божието дълготърпение ги чакаше в Ноевите дни, докато се правеше ковчега, в който малцина, тоест, осем души, се избавиха чрез вода.
\par 21 Която в образа на кръщението и сега ви спасява, (не измиването на плътската нечистота, но позива към Бога на чиста съвест), чрез възкресението на Исуса Христа;
\par 22 Който е отдясно на Бога, като се е възнесъл на небето, и Комуто се покориха ангели, власти и сили.

\chapter{4}

\par 1 И тъй понеже Христос пострада по плът, въоръжете се и вие със същата мисъл, защото пострадалият по плът се е оставил от греха,
\par 2 за да живеете през останалото в тялото време, не вече по човешки страсти, а по Божията воля.
\par 3 Защото доволно е миналото време, когато сте живеели така, както желаят да живеят езичниците, като сте прекарвали в нечистоти, в страсти, във винопийства, в пирования, в опивания, и в омразните идолослужения.
\par 4 Относно това те се и чудят и ви хулят за гдето не тичате с тях в същата крайност на разврата;
\par 5 но те ще отговарят пред Онзи, Който скоро ще (Гръцки: Е готов да.) съди живите и мъртвите,
\par 6 понеже затова се проповядва благовестието и на мъртвите, тъй щото, като бъдат съдени по човешки в плът, да живеят по Бога в дух.
\par 7 А краят на всичко е наближил; и тъй, живейте разумно и трезвено, за да се предавате на молитва.
\par 8 Преди всичко имайте усърдна любов помежду си, защото любовта покрива множество грехове
\par 9 Бъдете гостолюбиви едни към други без роптание.
\par 10 Според дарбата, която всеки е приел, служат с нея един на друг като добри настойници на многозначната Божия благодат.
\par 11 Ако говори някой, нека говори като такъв, който прогласява Божии словеса; ако служи някой, нека служи като такъв, който действува със силата, която му дава Бог; за да се слави във всичко Бог чрез Исуса Христа, Комуто е славата и господството до вечни векове. Амин.
\par 12 Възлюбени, не се чудете на огнената изпитня, която дохожда върху вас, за да ви опита, като че ви се случва нещо чудно;
\par 13 но радвайте се за гдето с това вие имате обещание в страданията на Христа, за да се зарадвате премного и когато се яви Неговата слава.
\par 14 Блажени сте, ако ви опозоряват за Христовото име; защото Духът на славата и на Бога почива на вас [откъм тях се хули, а откъм вас се прославя].
\par 15 Никой от вас да не страда като убиец, или крадец, или злодеец, или като такъв, който се бърка в чужди работи.
\par 16 Но, ако страда някой като християнин, да не се срамува, а нека слави Бога с това име.
\par 17 Защото дойде времето да се започне съдът от Божието домочадие; и ако почне първо от нас, каква ще бъде сетнината на тия, които не се покоряват на Божието благовестие?
\par 18 И ако праведният едва се спасява, то нечистият и грешният где ще се явят?
\par 19 Затова и тия, които страдат по Божията воля, нека предават душите си на верния Създател, като вършат добро.

\chapter{5}

\par 1 Прочее, презвитерите, които са между вас, увещавам аз който тоже съм презвитер и свидетел на Христовите страдания и участник на славата, която има да се яви:
\par 2 Пазете Божието стадо, което е между вас; надзиравайте го, не с принуждение, а драговолно, като за Бога; нито за гнусна печалба, но с усърдие;
\par 3 нито като че господарувате над паството, което ви се поверява, а като показвате пример на стадото.
\par 4 И когато се яви Пастиреначалникът, ще получите венеца на славата, който не повяхва.
\par 5 Така и вие, по-млади, покорявайте се на по-старите, да! всички един на друг. Облечете смирението; защото Бог се противи на горделивите, а на смирените дава благодат.
\par 6 И тъй, смирете се под мощната ръка на Бога, за да ви възвиси своевременно;
\par 7 и всяка ваша грижа възложете на Него, защото Той се грижи за вас.
\par 8 Бъдете трезвени, будни. Противникът ви, дяволът, като рикаещ лъв обикаля, търсейки кого да погълне.
\par 9 Съпротивете се нему, стоейки твърдо във вярата, като знаете, че същите страдания се понасят от братята ви в света.
\par 10 А Бог на всяка благодат, Който ви е призовал в Своята вечна слава чрез Христа [Исуса], ще ви усъвършенствува, утвърди, укрепи [и направи непоколебими], след като пострадате малко.
\par 11 Нему да бъде господството до вечни векове. Амин.
\par 12 Чрез Сила, верния брат, както го мисля, писах ви накъсо, да ви увещавам и заявявам, че това е истинската Божия благодат. Стойте твърдо в нея.
\par 13 Поздравява ви с избраната с вас църква във Вавилон, и син мой Марко.
\par 14 Поздравете се един друг с любезна целувка. Мир на всички вас, които сте в Христа [Исуса].

\end{document}