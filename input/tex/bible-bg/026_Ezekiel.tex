\begin{document}

\title{Езекил}


\chapter{1}

\par 1 В тридесетата година, четвъртия месец, на петия ден от месеца, като бях между пленниците при реката Ховар, отвориха се небесата, и аз видях Божии видения.
\par 2 На петия ден от месеца в тая година, която бе петата от пленяването на цар Иоахина,
\par 3 Господното слово дойде нарочно към свещеника Езекиила, син на Вузия, в Халдейската земя, при реката Ховар; и там Господната ръка биде върху него.
\par 4 Видях, и, ето, вихрушка идеше от север, голям облак и пламнал огън, а около него сияние; и отсред него се виждаше нещо на глед като светъл метал, от средата на огъня.
\par 5 Отсред него се виждаше и подобие на четири живи същества. И това беше изгледът им; те имаха човешко подобие.
\par 6 Всяко от тях имаше четири лица, всяко имаше и четири крила.
\par 7 Нозете им бяха прави нозе; и стъпалото на нозете им беше подобно на стъпало на телешка нога; и изпущаха искри като повърхността на лъскава мед.
\par 8 И имаха човешки ръце под крилата си на четирите си страни; и на четирите лицата и крилата бяха така:
\par 9 крилата им се съединяваха едно с друго; не се обръщаха като вървяха; всяко вървеше направо пред себе си.
\par 10 А колкото за изгледа на лицата им, той беше като човешко лице; и четирите имаха лъвово лице от дясната страна; и четирите имаха волско лице от лявата; и четирите имаха орлово лице.
\par 11 и лицата им и крилата им бяха обърнати нагоре; две крила на всяко се съединяваха едно с друго, и двете покриваха телата им.
\par 12 И вървяха всяко направо пред себе си; гдето се носеше духа, там вървяха; като вървяха не се обръщаха.
\par 13 А колкото за подобието на живите същества, изгледът им бе като запалени огнени въглища, като изгледа на факли, които се движеха нагоре надолу между живите същества; огънят беше светъл, и светкавица изскачаше из огъня.
\par 14 И живите същества блещукаха наглед като светкавица.
\par 15 А като гледах живите същества, ето по едно колело на земята при живите същества, за всяко от четирите им лица,
\par 16 Изгледът на колелата и направата им бе като цвят на хрисолит; и четирите имаха еднакво подобие; а изгледът им и направата им бяха като че ли на колело в колело.
\par 17 Когато вървяха, вървяха към четирите си страни; не се обръщаха като вървяха.
\par 18 А колелата им бяха високи и страшни; и колелата около тия четири бяха пълни с очи.
\par 19 И когато вървяха живите същества, и колелата вървяха край тях; и когато се издигаха живите същества от земята, и колелата се издигаха.
\par 20 Гдето имаше да иде духът, там вървяха и те; там гдето духът имаше да иде, и колелата се издигаха край тях; защото духът на всяко от живите същества беше и в колелата му.
\par 21 Когато вървяха ония, вървяха и тия; и когато стояха ония, стояха и тия; а когато ония се издигаха от земята, то и колелата се издигаха край тях; защото духът на всяко от живите същества беше и в колелата му.
\par 22 А над главите на живите същества имаше подобие на един простор, на глед като цвят на страшен кристал, разпрострян над главите им.
\par 23 А под простора крилата им бяха разпрострени, едно срещу друго; всяко същество имаше две, които покриваха телата им отсам; и всяко имаше две, които ги покриваха оттам.
\par 24 И когато вървяха, чувах фученето на крилата им като бучене на големи води, като глас на Всесилния, шум на метеж като шум на войска. Когато се спираха, спущаха крилата си.
\par 25 И глас се издаде от горе из простора, който бе над главите им; и като се спряха спуснаха крилата си.
\par 26 И над простора, който бе върху главите им, се виждаше подобие на престол, наглед като камък сапфир; и върху подобието на престола имаше подобие на глед като човек, седящ на него на високо.
\par 27 И видях нещо наглед като светъл метал, като изглед на огън в него от всяка страна; от това, което се виждаше, че е кръстът му, и нагоре, и от това, което се виждаше, че е кръстът му, и надолу видях като изглед на огън, обиколен със сияние.
\par 28 Какъвто е изгледът на дъгата в облака в дъждовен ден, такъв бе изгледът на обикалящото сияние. Това бе изгледът на подобието на Господната слава. И когато го видях, паднах на лицето си, и чух глас на едного, който говореше.

\chapter{2}

\par 1 И ми рече: Сине човешки, изправи се на нозете си, и ще ти говоря.
\par 2 И като ми говори, Духът влезе в мене и ме постави на нозете ми; и чух онзи, който ми говореше.
\par 3 И рече ми: Сине човешки, аз те изпращам при израилтяните, при бунтовни люде, които въстанаха против Мене; те и бащите им са престъпвали Моите думи дори до тоя днешен ден.
\par 4 Тия чада, при които те изпращам, са безочливи и коравосърдечни; а ти да им думаш: Така казва Господ Иеова.
\par 5 послушали или непослушали, (защото е бунтовен дом), пак те ще познаят, че е имало пророк всред тях.
\par 6 И ти, сине човешки, да се не плашиш от тях, и от думите им да се не убоиш, ако и да има тръни и бодили около тебе, и да живееш между скорпии; да се не убоиш от думите им, нито да се ужасяваш от погледите им, макар че са бунтовен дом.
\par 7 И да им говориш думите Ми, или послушали или непослушали; защото са твърде бунтовни.
\par 8 Ти обаче, сине човешки, слушай това, което ти говоря; не ставай и ти бунтовник като тоя бунтовен дом; отвори устата си и изяж това, което ти давам.
\par 9 И като погледнах, ето ръка простряна към мене, и ето в нея свитък книга.
\par 10 И той го разви пред мене. Писаното бе отвътре и отвън, в което бяха написани плачове, ридание и горко.

\chapter{3}

\par 1 И рече ми: Сине човешки, изяж това, което намираш; изяш тоя свитък, и иди, говори на Израилевия дом.
\par 2 И тъй, отворих устата си; и той ме накара да ям свитъка.
\par 3 И рече ми: Сине човешки, нека смила коремът ти тоя свитък, който ти давам, и нека се наситят червата ти от него. Тогава го изядох; и беше в устата ми сладък като мед.
\par 4 И рече ми: Сине човешки, иди, отправи се към Израилевия дом, и говори им с Моите думи.
\par 5 Защото не си изпратен при люде с непознат език и мъчен говор, но при Израилевия дом, -
\par 6 не при много племена с непознат език и мъчен говор, чиито думи не разбираш. Навярно при такива ако те пращах, те щяха да те послушат.
\par 7 Но Израилевият дом няма да те послуша, защото не щат да послушат и Мене; понеже целият Израилев дом е с кораво чело и жестоко сърце.
\par 8 Ето, направих лицето ти твърдо против техните лица, и челото ти твърдо против техните чела.
\par 9 Като адамант по-твърд от кремък направих челото ти; да се не боиш от тях, нито да се ужасиш от погледа им, макар да са бунтовен дом.
\par 10 Рече ми още: Сине човешки, приеми в сърцето си и послушай с ушите си всичките думи, които ще ти говоря;
\par 11 и иди, отправи се към пленниците, към сънародниците си, и като им говориш кажи: Така говори Господ Иеова, - или послушали, или непослушали.
\par 12 Тогава Духът ме подигна; и чух зад себе си глас на голямо спускане, който казваше: Благословена Господната слава от мястото Му!
\par 13 И чух фученето на крилата на живите същества, които крила се допираха едно до друго, и тропането на колелата край тях, - шум на голямо спускане.
\par 14 И тъй, Духът ме издигна и отнесе ме; и аз отидох с огорчение, в разпалването на духа си; Но Господната ръка бе мощна върху мене.
\par 15 Тогава дойдох в Телавив при пленниците, които живееха при реката Ховар, и настаних се там, гдето те бяха настанени; и там останах смаян между тях седем дни.
\par 16 А след седемте дни Господното слово дойде към мене и рече:
\par 17 Сине човешки, поставих те страж за Израилевия дом; слушай, прочее, словото от устата Ми, и предупреди ги от Моя страна.
\par 18 Когато кажа на беззаконника: Непременно ще умреш; а ти не го предупредиш, и не говориш, за да предпазиш беззаконника от беззаконния му път, та да спасиш живота му, оня беззаконник ще умре в беззаконието си; но от твоята ръка ще изискам кръвта му.
\par 19 Обаче, ако предупредиш беззаконника, но той не се върне от беззаконието си и от беззаконния си път, той ще умре в беззаконието си; а ти си избавил душата си.
\par 20 Пак, ако се върне праведникът от правдата си и извърши беззаконие като Аз поставя препънка пред него, той ще умре; понеже ти не си го предупредил, той ще умре в греха си, и правдата, която е вършил, няма да се помни; но от твоята ръка ще изискам кръвта му.
\par 21 Но ако предупредиш праведника, за да не съгреши, и праведникът не съгреши, той непременно ще живее, защото се е свестил; и ти си избавил душата си.
\par 22 И ръката на Господа биде там върху мене; и Той ми рече: Стани, излез на полето, и там ще ти говоря.
\par 23 Тогава станах, та излязох на полето; и, ето, Господната слава стоеше там, като славата, която видях при реката Ховар; и паднах на лицето си.
\par 24 Тогава Духът влезе в мене, изправи ме на нозете ми, и като ми говори, рече: Иди, затвори се в къщата си.
\par 25 И тебе, сине човешки, ето, ще турят на тебе връзки и ще те вържат с тях, тъй щото няма да излезеш между людете.
\par 26 И ще направя езикът ти да залепне за небцето ти, та ще бъдеш ням, и не ще бъдеш за тях изобличител; защото са бунтовен дом.
\par 27 Но, когато ти говоря, ще отворя устата ти; и ти им кажи - Така казва Господ Иеова: Който слуша, нека слуша; и който не слуша, нека не слуша; защото са бунтовен дом.

\chapter{4}

\par 1 И ти, сине, човешки, вземи си тухла, тури я пред себе си, и начертай на нея града Ерусалим.
\par 2 И постави обсада против него, съгради укрепления против него, и издигни могили против него; разположи още стан против него, и постави стеноломи против него от всяка страна.
\par 3 Вземи си и желязна плоча, тури я като желязна стена между тебе и града, и насочи лицето си против него, и той ще бъде обсаден; и ти постави обсада против него. Това ще бъде знамение за Израилевия дом.
\par 4 Тогава легни на лявата си страна, и положи на нея беззаконието на Израилевия дом; колкото дни лежиш на нея ще носиш беззаконието им.
\par 5 Защото Аз определих годините на беззаконието им да ти бъдат съответствено число дни - триста и деветдесет дни; така ще носиш беззаконието на Израилевия дом.
\par 6 И като навършиш тия, тогава легни на дясната си страна, и носи беззаконието на Юдовия дом четиридесет дни; по един ден ти определих за всяка година.
\par 7 И насочи лицето си към обсадата на Ерусалим, с гола мишца, и пророкувай против него.
\par 8 ето, ще туря на тебе връзки, и няма да се обърнеш от едната си страна на другата, догдето не навършиш дните, през които ще го обсаждаш.
\par 9 Вземи си и пшеница и ечемик, боб и леща, просо и бяло жито, и като ги туриш в един съд, направи си от тях хлябове; и колкото дни лежиш на страната си, триста и деветдесет дни, яж от тях.
\par 10 И храната, която ядеш, да бъде с теглилка, двадесет сикли на ден; от време на време да ядеш от тях.
\par 11 Също и вода с мярка да пиеш, по една шеста от ин на ден; от време на време да пиеш.
\par 12 Да ги ядеш като ечемичени пити, и да ги печеш с човешки нечистотии пред очите им.
\par 13 И Господ рече: Така ще ядат израилтяните хляба си омърсен между народите, дето ще ги изпъдя.
\par 14 Тогава аз рекох: Ах! Господи Иеова, ето, душата ми не се е омърсила; понеже от младостта си до сега не съм ял мърша или разкъсано от звяр, нито е влязло някога в устата ми мръсно месо.
\par 15 Тогава ми рече: Виж, давам ти говежди нечистотии вместо човешки нечистотии; с него опечи хляба си.
\par 16 Рече ми още: Сине човешки, ето, Аз ще строша подпорката от хляба в Ерусалим; те ще ядат хляб с теглилка и икономично, и смаяни ще пият вода с мярка.
\par 17 Това ще направя, за да се лишат от хляб и вода, и да се гледат един други смаяни, и да се изнурят в беззаконието си.

\chapter{5}

\par 1 И ти, сине човешки, вземи си остър нож, то ест, вземи си бръснарски бръснач, и си обръсни главата и брадата; после вземи си везни та претегли и раздели космите.
\par 2 Една трета част изгори в огън всред града, когато се изпълнят дните на обсадата; и една трета част вземи и я насечи с ножа от всяка страна; и една трета част разпръсни по въздуха, и Аз ще изтегля нож след тях.
\par 3 А от тия вземи няколко и вържи ги в полите си.
\par 4 После и от тия вземи малко, хвърли ги всред огъня, и изгори ги в огъня; от тук ще излезе огън по целия Израилев дом.
\par 5 Така казва Господ Иеова: Това е Ерусалим. Аз го поставих всред народите; И разни страни го обикалят.
\par 6 Но той се разбунтува против съдбите Ми като извърши беззаконие повече от народите, И против повеленията Ми повече от страните, които са около него; Защото отхвърлиха съдбите Ми, И не ходиха в повеленията Ми.
\par 7 Затова, така казва Господ Иеова; Понеже вие сте по-немирни от народите, които са около вас, Като не ходихте в повеленията Ми, И не пазихте съдбите Ми, Нито сторихте даже според постановленията На народите, които са около вас,
\par 8 Затова, така казва Господ Иеова: Ето, и Аз съм против тебе, И ще извърша съдби всред тебе пред очите на народите.
\par 9 Поради всичките твои мерзости Ще ти направя онова, което никога не съм направил, Нито ще направя някога подобно нему.
\par 10 Затова, бащи ще изядат чадата си всред тебе, И чада ще изядат бащите си; И Аз ще извърша съдби в тебе, А всички останали от тебе ще разпръсна към всичките ветрища.
\par 11 Затова, заклевам се в живота Си, казва Господ Иеова, Понеже ти оскверни светилището Ми С всичките си мерзости И с всичките си гнусоти, Затова Аз непременно ще те отрежа; Окото Ми няма да пощади, И Аз няма да покажа милост.
\par 12 Една трета част от тебе ще измрат от мор, И ще се довършат всред тебе от глад; И една трета част ще паднат около тебе от меч; А една трета част ще разпръсна по всичките ветрища, И ще изтегля нож след тях.
\par 13 Така ще се изчерпи гнева Ми, И Аз ще удовлетворя яростта Си върху тях, И ще се задоволя; И те ще познаят, че Аз Господ говорих в ревността си, Когато изчерпя яростта Си против тях.
\par 14 При това, пред всекиго, който минава, Ще те направя пустиня и предмет на укор Между народите, които са около тебе.
\par 15 И тъй, това ще бъде за укор и присмех, За поука и удивление, На народите, които са около тебе, Когато извърша в тебе съдби с гняв, С ярост, и с яростни изобличения; Аз Господ говорих това.
\par 16 Когато изпратя върху тях злите стрели на глада, Изтребителите, които ще изпратя, за да ви изтребят, Ще усиля още и гладът у вас, И ще строша подпорката ви от хляба.
\par 17 Ще пратя върху вас и глад и люти зверове, Които ще те обезчадят; Мор и кръв ще преминат през тебе; И ще нанеса меч върху тебе; Аз Господ говорих това.

\chapter{6}

\par 1 И Господното слово дойде към мене и казваше:
\par 2 Сине човешки, насочи лицето си към Израилевите планини, и пророкувай на тях, казвайки:
\par 3 Планини Израилеви, слушайте словото на Господа Иеова. Така казва Господ Иеова към планините и към хълмовете, Към урвите и към долините: Ето, Аз, Аз ще нанеса върху вас меч, И ще разоря високите ви места.
\par 4 Жертвениците ви ще се изоставят, И кумирите ви на слънцето ще се строшат; И ще тръшна пред идолите ви избитите ви мъже.
\par 5 И ще постеля труповете на израилтяните Пред идолите им, И ще разпръсна костите ви Около жертвениците ви.
\par 6 Навсякъде, гдето живеете, градовете ще бъдат запустени, И високите места ще се изоставят, За да запустеят и да се развалят жертвениците ви, Идолите ви да се строшат и да изчезнат, Кумирите ви на слънцето да се съсекат, И изделията ви да се унищожат.
\par 7 Тоже и избитите ще паднат всред вас; И ще разберете, че Аз съм Господ.
\par 8 Ще оставя обаче остатък, Като ще имате между народите някои избягнали от ножа, Когато бъдете разпръснати по разните страни.
\par 9 И колкото от вас избегнат Ще Ме помнят между народите, Гдето ще бъдат заведени пленници, Как съм бил съкрушен поради блудното им сърце, Което се отклони от Мене, И поради очите им, които блудствуват след идолите им; И ще се отвращават от себе си Поради злините, които са вършили във всичките си мерзости.
\par 10 И ще познаят, че Аз Господ Не съм казал напразно, Че щях да им сторя това зло.
\par 11 Така казва Господ Иеова: Плесни с ръката си, И тропни с ногата си, и речи: Горко поради всичките лоши мерзости на Израилевия дом! Защото ще паднат от нож, от глад и от мор.
\par 12 Далечният ще умре от мор, И ближният ще падне от нож, А който остане и бъде обсаден ще умре от глад; Така ще изчерпя яростта Си над тях.
\par 13 И ще познаете, че Аз съм Господ, Когато убитите им лежат между идолите им Около жертвениците им, На всеки висок хълм, По високите върхове на планините, Под всяко зелено дърво, И под всеки гъстолист дъб, На мястото гдето принасяха благоухание На всичките си идоли.
\par 14 И Аз ще простра ръката Си върху тях, И ще направя земята пуста, - По-пуста от пустинята към Дивлат, - Във всичките места гдето живеят; И те ще познаят, че Аз съм Господ.

\chapter{7}

\par 1 При това, Господното слово дойде към мене и казваше:
\par 2 А ти, сине човешки, слушай. Така казва Господ Иеова към Израилевата земя: Край! краят дойде На четирите краища на страната.
\par 3 Сега е дошъл краят върху тебе; Защото ще изпратя гнева Си върху тебе, Ще те съдя според постъпките ти, И ще възвърна върху тебе всичките ти мерзости.
\par 4 Окото ми не ще те пощади, Нито ще ти покажа милост; Но ще възвърна върху тебе постъпките ти, И въздаянията ти за твоите мерзости ще бъдат всред тебе; И ще познаете, че Аз съм Господ.
\par 5 Така казва Господ Бог Иеова: Зло, едно небивало зло, ето, иде;
\par 6 Краят дойде, краят дойде, Бди против тебе; ето, настъпи.
\par 7 Присъдата ти дойде върху тебе, който живееш в тая земя; Времето дойде, денят наближи, Ден на смущение по планините, а не на възклицание.
\par 8 Сега скоро ще излея върху тебе яростта Си, И ще изчерпя гнева Си върху тебе; Ще те съдя според постъпките ти, И ще възвърна върху тебе всичките ти мерзости,
\par 9 Окото Ми няма да пощади, Нито ще покажа милост; Ще ти въздам според постъпките ти, И въздаянията за твоите мерзости ще бъдат всред тебе; И вие ще познаете, че Аз, Който ви поразявам, съм Господ.
\par 10 Ето, денят, ето, иде! Твоята присъда се яви! Жезълът се разцъфтя, гордостта поникна!
\par 11 Насилието порасна в тояга за беззаконието им; Нищо не ще остане от тях, Нито от това множество, Нито от имота им; И няма да остане великолепие между тях.
\par 12 Времето дойде, денят наближи; Който купува, да се не радва; И който продава, да не жали; Защото има гняв върху цялото това множество.
\par 13 Защото продавачът няма да притежава наново продаденото от него, Ако и да е останал жив; Понеже видението относно цялото това множество няма да се отмени, Нито ще утвърди някой себе си Чрез беззаконния си живот.
\par 14 Затръбиха и приготвиха всичко, Но никой не отива на боя; Защото Моят гняв е върху цялото това множество.
\par 15 Ножът е вън, а морът и гладът вътре; Който е на нивата ще умре от нож; А който е в града, глад и мор ще го погълнат.
\par 16 А ония от тях, които избягнат, като се избавят Ще бъдат по планините като гълъбите по долините, Всичките плачещи, всеки за беззаконието си.
\par 17 Всички ръце ще ослабват, И всички колена ще станат като вода.
\par 18 Ще се препашат с вретище, И ужас ще ги покрие; Срам ще има по лицата на всичките, И плешивост по главите на всички тях.
\par 19 Среброто си ще хвърлят по улиците, И златото им ще им бъде като нечисто нещо; Среброто и златото им не ще могат да ги избавят В деня на гнева Господен; Те няма да наситят душите си, нито да напълнят червата си; Защото о тях се спънаха и паднаха в беззаконието си.
\par 20 Господ постави славното Си украшение, за да се гордеят с него; Но те направиха в него образите на своите мерзости, На омразните си идоли; Затова, Аз го обръщам в нещо нечисто за тях.
\par 21 Ще го предам в ръцете на чужденци за плячка, И на нечестивите на земята за корист; И те ще го омърсят;
\par 22 И като отвърна лицето Си от тях, Те ще омърсят светилището Ми, И грабители ще влязат в него и ще го омърсят.
\par 23 Направи верига; Защото земята е пълна с кървави престъпления, И градът е пълен с насилие.
\par 24 Затова, ще докарам най-злите от народите, И те ще се завладеят къщите им; Ще направя да престане и гордостта на силните; И светите им места ще се омърсят.
\par 25 Погибел иде; Ще потърсят мир, но не ще има.
\par 26 Бедствие след бедствие ще дохожда, И слух след слух ще пристига; Тогава ще поискат видение от пророк; Но поуката ще се изгуби от свещеника, И съветът от старейшините.
\par 27 Царят ще жалее, Първенецът ще се облече в смайване, И ръцете на людете на страната ще ослабват. Според постъпките им ще им направя, И според заслужването им ще ги съдя; И ще познаят, че Аз съм Господ.

\chapter{8}

\par 1 А в шестата година, в шестия месец, на петия ден от месеца, като седях в къщата си, и Юдовите старейшини седяха пред мене, ръката на Господа Иеова слетя там върху мене.
\par 2 Погледнах, и ето подобие на глед като огън; от това, което се виждаше, че е кръстът му и надолу, огън; и от кръста му нагоре, наглед като сияние, като че ли беше светъл метал.
\par 3 И Той простря подобие на ръка и ме хвана за космите на главата ми; и Духът ме издигна между земята и небето, и пренесе ме чрез Божии видения в Ерусалим, до входа на северната порта на вътрешния двор, гдето бе поставен възбудителят на ревността идол, който предизвиква ревнование.
\par 4 ето, славата на Израилевия Бог бе там, както във видението, което видях на полето.
\par 5 Тогава ми рече: Сине човешки, подигни сега очите си към север. И тъй, повдигнах очите си към север, и, ето, на север от вратата, която води за олтара, при входа стоеше тоя възбудител на ревнивост идол.
\par 6 И рече ми: Сине човешки, видиш ли що правят тия? - големите мерзости, които Израилевият дом върши тук, та да се отдалеча от светилището Си? Но ще видиш още пак големи мерзости.
\par 7 И тъй, заведе ме до вратата на двора; и, като погледнах, ето една дупка в стената.
\par 8 Тогава ми рече: Сине човешки, копай сега в стената. И като копах в стената, ето един вход.
\par 9 И рече ми: Влез та виж нечестивите мерзости, които тия вършат тука.
\par 10 Влязох прочее и погледнах; и ето всякакви подобия на гадове и гнусни животни, и всичките идоли на Израилевия дом, изобразени на стената от край до край.
\par 11 И пред тях стоеха седемдесет мъже от старейшините на Израилевия дом, всред които стоеше Яазания Сафановия син, всеки с кадилницата си в ръката си; и гъст облак темян се издигаше.
\par 12 Тогава ми рече: Сине човешки, видя ли що правят в тъмнината старейшините на Израилевия дом, всички в скришните стаи за изображения? защото си казват: Господ не ни вижда; Господ е напуснал земята.
\par 13 Рече ми още: Ще видиш още пак големи мерзости, които вършат.
\par 14 Тогава ме заведе до входа на северната порта на Господния дом; и, ето, там седяха жени та оплакваха Тамуза.
\par 15 И рече ми: Видя ли, сине човешки? Ще видиш още пак и по-големи мерзости от тия.
\par 16 И въведе ме във вътрешния двор на Господния дом; и, ето, във входа на Господния храм, между предхрамието и олтара, около двадесет и пет мъже, с гърбовете си към Господния храм, и с лицата си към изток, които се кланяха на слънцето към изток.
\par 17 Тогава ми рече: Видя ли, сине човешки? Малко ли е за Юдовия дом дето вършат мерзости, каквито тия вършат тука? Защото като напълниха земята с насилия, пак предизвикват гнева Ми; и, ето, турят клончето до ноздрите си.
\par 18 Затова и аз ще действувам с ярост; окото Ми няма да пощади, нито ще покажа милост; и макар да извикат в ушите Ми с висок глас, няма да ги послушам.

\chapter{9}

\par 1 Тогава Той извика в ушите ми с висок глас и рече: Нека се приближат ония, на които е заръчано за града, всеки с изтребителното си оръжие в ръка.
\par 2 И, ето, шест мъже идеха по пътя от горната порта, която гледа към север, всеки с разрушително оръжие в ръка, и всред тях един човек облечен в ленено, с писарска мастилница на кръста му; и като влязоха застанаха при медния олтар.
\par 3 И славата на Израилевия Бог се издигна от херувимите, над които бе, та застана над прага на дома; и Той извика към облечения в ленено мъж, който имаше на кръста си писарска мастилница.
\par 4 И Господ му рече: Мини през града, през Ерусалим, и тури белег върху челата на мъжете, които въздишат и плачат поради всичките мерзости, които стават всред него.
\par 5 А на другите рече, като слушах аз: Минете подир него през града, та поразете; окото ви да не пощади, нито да покаже милост;
\par 6 старци, юноши и девици, младенци и жени избийте съвсем; но не се приближавайте при никого от ония, върху които е белегът; и започнете от светилището Ми. Прочее те започнаха от старейшините, които бяха пред дома.
\par 7 Тогава им рече: Осквернете дома, и напълнете дворовете с убити; излезте. И излязоха та поразяваха в града.
\par 8 А те като поразяваха и аз останах, паднах на лицето си, извиках, и рекох: Горко, Господи Иеова! ще изтребиш ли всичките останали от Израиля като изливаш гнева си върху Ерусалим?
\par 9 И Той ми рече: Беззаконието на Израилевия и на Юдовия дом стана твърде голямо; земята е пълна с кръв, и градът пълен с извращение на съд; защото си думат: Господ е напуснал земята, и: Господ не вижда.
\par 10 Затова, и от Моя страна, окото Ми няма да пощади, и Аз няма да покажа милост, а ще възвърна постъпките им върху главите им.
\par 11 И ето, облечения в ленено мъж, който имаше на кръста си мастилницата, даде отчет за работата, като каза: Направих както Ти ми заповяда.

\chapter{10}

\par 1 После погледнах, и, ето, в простора, който бе върху главите на херувимите, се яви над тях като камък сапфир, наглед като подобие на престол.
\par 2 И той проговори на облечения в ленено мъж, казвайки: Влез между търкалящите колела под херувимите, напълни ръцете си с огнени въглища изсред херувимите, и пръсни ги върху града. И той влезе пред очите ми.
\par 3 А херувимите стоеха отдясно на дома, когато влизаше мъжът: и облакът изпълни вътрешния двор.
\par 4 И Господната слава се издигна от херувимите, та застана над прага на дома; и домът се изпълни от сиянието на Господната слава.
\par 5 И фученето на крилата на херувимите се чуваше дори до външния двор, като глас на Всесилния Бог, когато Той говори.
\par 6 И когато заповяда на облечения в ленено мъж, казвайки: Вземи огън измежду търкалящите колела, изсред херувимите, тогава той влезе та застана при едно колело.
\par 7 И един херувим, като простря ръка изсред херувимите към огъня, който бе всред херувимите, взе от него, и тури в ръцете на облечения в ленено, който като го взе излезе.
\par 8 А подобието на човешка ръка в херувимите се виждаше под крилата им.
\par 9 И като погледнах, ето четири колела при херувимите, едно колело при един херувим, и едно колело при друг херувим; и изгледът на колелата бе като цвят на хрисолит.
\par 10 А колкото за изгледа им, и четирите имаха еднакво подобие, като че ли на колело в колело.
\par 11 Когато вървяха, вървяха към четирите си страни; не се обръщаха като вървяха, но на където се управяше първия, следваха го и другите без да се обръщат като вървяха.
\par 12 А цялото им тяло, гърбовете им, ръцете им, крилата им, и колелата, то ест, колелата на четирите живи същества, бяха пълни с очи от всяка страна.
\par 13 А колкото за колелата, те се наричаха, като слушах аз, търкалящи колела.
\par 14 Всяко от живите същества имаше четири лица; първото лице бе херувимско лице; второто лице, човешко лице; третото, лъвово лице; а четвъртото, орлово лице.
\par 15 И херувимите се издигнаха. Това е живото същество, което видях при реката Ховар.
\par 16 И когато вървяха херувимите, и колелата вървяха край тях; и когато херувимите подигаха крилата си, за да се издигнат от земята, то и колелата не се отклоняваха от тях.
\par 17 Когато стоеха ония, стоеха и тия; а когато се издигаха ония, издигаха се и тия заедно с тях; защото духът на всяко от живите същества беше и в тях.
\par 18 Тогава Господната слава излезе изотгоре на прага на дома и застана над херувимите.
\par 19 Когато излязоха, херувимите подигнаха крилата си, та се издигнах от земята, като гледах аз, и колелата край тях; и застанаха във входа на източната порта на Господния дом; и славата на Израилевия Бог бе отгоре им.
\par 20 Това е живото същество, което видях под Израилевия Бог, при реката Ховар; и познах, че бяха херувими.
\par 21 Всеки имаше четири лица, и всеки четири крила; и подобие на човешки ръце се виждаше под крилата им.
\par 22 А колкото за подобието на лицата им, те бяха същите лица, които видях при реката Ховар, - изгледът им и сами те; вървяха всяко направо пред себе си.

\chapter{11}

\par 1 При това духът ме дигна та ме отнесе на източната порта на Господния дом, която гледа към изток; и ето във входа на портата двадесет и пет мъже, между които видях Яазания Азуровия син и Фелатия Венаиевия син, първенци между людете.
\par 2 И Господ ми реч: Сине човешки, тия са мъжете, които измислюват неправда, и които дават нечестив съвет в тоя град, като казват:
\par 3 Времето не е близо да съградим къщи; тоя град е котел, а ние месо.
\par 4 Затова, пророкувай против тях, пророкувай, сине човешки.
\par 5 И Господният Дух слетя върху мене и рече ми: Говори. Така казва Господ: Това е що сте рекли, доме Израилев; Защото Аз зная размишленията на духа ви.
\par 6 Убихте мнозина в тоя град, И напълнихте улиците му със заклани.
\par 7 Затова, така казва Господ Иеова: Избитите от вас, които постлахте всред него, Те са месо, а тоя град е котел; Обаче Аз ще ви извадя изсред него.
\par 8 От ножа се боехте; И нож ще докарам върху вас, казва Господ Иова.
\par 9 Ще ви извадя изсред града, И ще ви предам в ръцете на чужденци, И ще извърша съдби всред вас.
\par 10 От меч ще паднете; В Израилевите предели ще ви съдя; И ще познаете, че Аз съм Господ.
\par 11 Тоя град не ще ви бъде котел, Нито ще бъдете вие месо всред него; В Израилевите предели ще ви съдя;
\par 12 И ще познаете, че Аз съм Господ; Понеже не ходихте в повеленията Ми, Нито извършихте съдбите Ми, Но постъпвахте според постановленията на околните вам народи.
\par 13 А като пророкувах, Фелатия Венаиевия син умря. Тогава паднах на лицето си, и като извиках с висок глас, рекох: Горко, Господи Иеова! ще довършиш ли Ти останалите от Израиля?
\par 14 И Господното слово дойде към мене и рече:
\par 15 Сине човешки, братята ти, да! твоите братя, сродните ти мъже, и целият Израилев дом, те всички са ония, на които Ерусалимските жители рекоха: Отдалечете се от Господа; нам се даде тая земя за владение.
\par 16 Затова речи: Така казва Господ Иеова: Понеже ги преместих далеч между народите, И понеже ги разпръснах по разни страни, И съм бил на тях за малко време светилище В страните гдето са отишли,
\par 17 Затова речи: Така казва Господ Иеова: Ще ви събера от народите, Ще ви прибера от страните гдето сте разпръснати, И ще ви дам Израилевата земя.
\par 18 И като дойдат там, Ще махнат от нея всичките й гнусотии И всичките й мерзости.
\par 19 Аз ще им дам едно сърце, И ще вложа вътре във вас нов дух; И като отнема каменното сърце от плътта им, Ще им дам крехко сърце,
\par 20 За да ходят в повеленията Ми, И да пазят наредбите Ми и да ги вършат. И те ще бъдат Мои люде, и Аз ще бъда техен Бог.
\par 21 А колкото за ония, чието сърце постъпва По гнусните им и мерзостни желания, Ще възвърна постъпките им върху главите им, Казва Господ Иеова.
\par 22 Тогава херувимите подигнаха крилата си, и колелата се издигнаха край тях; и славата на Израилевия Бог бе отгоре им.
\par 23 И Господната слава се издигна изсред града та застана на хълма, който е на изток от града.
\par 24 И Духът като ме издигна отнесе ме чрез видение, с Божия Дух, в Халдейската земя, при пленниците. Тогава видението, което бях видял, си отиде от мене.
\par 25 После, казах на пленниците всичките думи, които Господ беше ми показал.

\chapter{12}

\par 1 Пак Господното слово дойде към мене и рече:
\par 2 Сине човешки, ти живееш всред дом на бунтовници, които имат очи, за да виждат, но не виждат, които имат уши, за да чуват, но не чуят; защото са бунтовен дом.
\par 3 Затова, ти сине човешки, приготви си вещите, потребни за преселване, и пресели се денем пред очите им; пресели се от мястото си на друго място пред очите им; негли дадат внимание, ако и да са бунтовен дом.
\par 4 Изнеси вещите си пред очите им, като вещи за преселване; а привечер излез и сам ти пред очите им, като ония, които отиват на преселване.
\par 5 Прокопай си стената пред очите им, и изнеси вещите си през нея.
\par 6 Пред очите им вдигни ги на рамена, и изнеси ги на мръкване. Покрий лицето си та да не видиш земята; понеже те поставих за знамение на Израилевия дом.
\par 7 И сторих каквото ми бе заповядано; изнесох вещите си денем като вещи за преселване, и привечер си прокопах стената с ръка; и на мръкване ги изнесох пред очите им, като ги вдигнах на рамена.
\par 8 А на утринта Господното слово дойде към мене и рече:
\par 9 Сине човешки, Израилевият дом, бунтовният дом, не ти ли рече: Що правиш ти?
\par 10 Кажи им: Така казва Господ Иеова: Това пророчество наложено на тебе се отнася до княза, който е в Ерусалим, и до всички от Израилевия дом, между които са и те.
\par 11 Речи: Аз съм знамението за вас. Както сторих аз, така ще стане с тях; В преселение, да! в плен ще отидат.
\par 12 Князът, който е между тях, ще дигне товар на рамена. На мръкване, и ще излезе; Ще прокопаят стената, за да изнасят през нея вещите му; Ще покрие лицето си, за да не види земята с очите си.
\par 13 Аз обаче ще простра мрежата си върху него, И ще се хване в примката Ми; Ще го закарам във Вавилон в Халдейската земя; Но няма да я види, при все че там ще умре.
\par 14 И ще разсея към всичките ветрища Всички, които са около него да му помагат, И всичките му полкове; И ще изтегля нож след тях.
\par 15 И ще познаят, че Аз съм Господ, Когато ги разпръсна между народите И ги разсея по разни страни.
\par 16 Обаче неколцина от тях ще оставя, Оцелели от меча, от глада и от мора, За да изявят всичките мерзости Между народите, гдето отиват; И ще познаят, че Аз съм Господ.
\par 17 При това, Господното слово дойде към мене и рече:
\par 18 Сине човешки, яж хляба си с трепет, И пий водата си с треперене и икономично.
\par 19 И кажи на людете на тая земя: Така казва Господ Иеова За жителите на Ерусалим в Израилевата земя: Ще ядат хляба си икономично, И ще пият водата си със смайване, За да запустее земята му, оголена от пълнотата си От беззаконието на всички, които живеят в нея.
\par 20 Населените градове ще запустеят, И земята ще се разори; И ще познаете, че Аз съм Господ.
\par 21 И Господното слово дойде към мене и рече:
\par 22 Сине човешки, каква е тая поговорка, що имате в Израилевата земя, която казва: Дните минават, а никое видение не се сбъдва?
\par 23 Затова кажи им: Така казва Господ Иеова: Ще направя тая поговорка да престане, И няма вече да се употребява за поговорка в Израиля; Но кажи им: Дните наближават, Тоже и изпълнението на всяко видение.
\par 24 Защото не ще има вече никакво лъжливо видение. Нито ласкателно предсказване всред Израилевия дом.
\par 25 Защото Аз съм Господ Аз ще говоря, И словото, което ще говоря ще се изпълни; Няма да се отлага вече; Защото във вашите дни, бунтовни доме, като изговоря слово Ще го и изпълня, казва Господ Иеова.
\par 26 Пак дойде към мене Господното слово и рече:
\par 27 Сине човешки, ето, ония, които са от Израилевия дом, казват: Видението, което той вижда, ще се изпълни след много дни, И той пророкува за далечни времена.
\par 28 Затова, кажи им: Така казва Господ Иеова: Ни една от думите Ми не ще се отлага вече; Но словото, което ще говоря, ще се изпълни, Казва Господ Иеова.

\chapter{13}

\par 1 И Господното слово дойде към мене и рече:
\par 2 Сине човешки, пророкувай против Израилевите пророци, които пророкуват, и кажи на ония, които пророкуват от своето си сърце: Слушайте словото Господно.
\par 3 Така казва Господ Иеова: Горко на глупавите пророци, Които се водят по своя си дух, Без да са видели някое видение!
\par 4 Израилю, твоите пророци са Като лисици в развалините.
\par 5 Вие не се изкачихте в проломите, Нито издигнахте ограда за Израилевия дом, За да устои в боя в деня Господен.
\par 6 Ония видяха суети и лъжливи предсказания, Които казват: Господ говори, Когато Господ не ги е пратил, И които направиха човеците да се надяват, Че думата им щяла да се изпълни.
\par 7 Не видяхте ли суетни видения, И не говорихте ли лъжливи предсказания, Когато казвате: Господ рече, При все че не съм говорил?
\par 8 Затова, така казва Господ Иеова; Понеже говорихте суети и видяхте лъжи, Затова, ето, Аз съм против вас, Казва Господ Иеова.
\par 9 И ръката Ми ще бъде против пророците, Които гледат суети и предсказват лъжи; Те не ще бъдат в съвета на людете Ми, Нито ще бъдат записани в списъка на Израилевия дом, Нито ще влязат в Израилевата земя; И ще познаете, че Аз съм Господ Иеова.
\par 10 Понеже, да! понеже измамиха людете Ми Като казваха: Мир! а пък няма мир; И когато някои градяха една слабичка стена, Ето, те я мажеха с кал,
\par 11 Кажи на ония, които я мажат с кал, че ще падне, Понеже ще вали пороен дъжд, И ти, голяма, каменна градушка, ще паднеш върху нея, И бурен вятър ще я съсипе.
\par 12 Ето, когато падне стената, няма ли да ви рекат: Где е калта, с която я мазахте?
\par 13 Затова, така казва Господ Иеова: Непременно ще я съсипя в яростта Си с бурен вятър, В гнева Ми ще вали пороен дъжд, И в яростта Ми ще падне голяма градушка от камъни, за да я разруша.
\par 14 Така ще съборя стената, Която измазахте с кал, И така ще я съсипя на земята, Щото ще се открият основите й; Тя ще падне, и вие ще загинете всред нея; И ще познаете, че Аз съм Господ.
\par 15 Така ще изчерпя яростта Си върху стената И върху ония, които са я измазали с кал, И ще ви река: Няма стената, Нито ония, които я измазаха, именно,
\par 16 Израилевите пророци, които пророкуват за Ерусалим, И които виждат видения на мир за него, Когато няма мир, казва Господ Иеова.
\par 17 И ти, сине човешки, Насочи лицето си и против дъщерите на людете си, Които пророкуват от своето си сърце; И пророкувай против тях, като речеш:
\par 18 Така казва Господ Иеова: Горко на жените, които пришиват възглавнички на всеки лакът, И правят покривала за главите на лица от всякакъв ръст, за да ловят души! Ще ловите ли душите на людете Ми, И ще пазите ли души живи за себе си?
\par 19 Като лъжете людете Ми, които слушат лъжи, Ще Ме оскверните ли между людете Ми За шепи ечемик и късчета хляб, Та да убивате души, които не трябваше да умрат, И да опазите живи души, които не трябваше да живеят?
\par 20 Затова, така казва Господ Иеова: Ето, Аз съм против възглавничките ви, С които ловите души като птици; Ще ги отдера от мишниците ви, И ще оставя да избягат душите, Душите, които вие ловите като птици.
\par 21 Ще раздера и покривалата ви, И ще избавя людете Си от ръката ви, Та няма вече да бъдат в ръката ви за лов; И ще познаете, че Аз съм Господ.
\par 22 Защото с лъжите оскърбихте сърцето на праведния, Когото Аз не оскърбих; И укрепихте ръцете на злодееца За да се не върне от нечестивия си път, Та да се спаси живота му.
\par 23 Затова, няма вече да виждате суети, Нито ще предсказвате вече предсказания; И ще избавя людете Си от ръката ви! И ще познаете, че Аз съм Господ.

\chapter{14}

\par 1 Тогава дойдоха при мене някои от Израилевите старейшини та седнаха пред мене.
\par 2 И Господното слово дойде към мене и рече:
\par 3 Сине човешки, тия мъже прегърнаха идолите си в сърцата си, и туриха беззаконието си като препънка пред лицето си; бива ли да се допитват тия до Мене?
\par 4 Затова, говори им като им речеш: Така казва Господ Иеова: На всекиго от Израилевия дом, който прегърне идолите си в сърцето си, и тури беззаконието си като препънка пред лицето си, и дойде при пророка, Аз Господ ще му отговоря както подобава според многото му идоли,
\par 5 за да хвана Израилевия дом за самото им сърце, понеже те всички станаха чужди за Мене чрез идолите си.
\par 6 Затова, кажи на Израилевия дом: Така казва Господ Иеова: Покайте се, отвърнете се от идолите си, да! отвърнете лицата си от всичките си мерзости.
\par 7 Защото на всекиго от Израилевия дом, или от чужденците, които пришелствуват в Израиля, който стане чужд за Мене като прегърне идолите си в сърцето си, и тури беззаконието си като препънка пред лицето си, и дойде при пророка, за да се допита до Мене чрез него, Аз Господ ще му отговоря непосредствено от Себе Си;
\par 8 защото ще насоча лицето Си против такъв човек, ще го направя за учудване, за знамение и поговорка, и ще го отсека отсред людете Си; и ще познаете, че Аз съм Господ.
\par 9 И ако би се подмамил пророкът да проговори дума, Аз Господ подмамих този пророк; и ще простра ръката Си върху него, та ще го изтребя отсред людете Си Израиля.
\par 10 И ще понесат наказанието на беззаконието си, (наказанието на пророка ще бъде като наказанието на онзи, който би се допитвал чрез него),
\par 11 за да не блуждае вече Израилевият дом от Мене, нито да се оскверняват вече с всичките си престъпления, но да бъдат Мои люде, и Аз да бъде техен Бог, казва Господ Иеова.
\par 12 И Господното слово дойде към мене и рече:
\par 13 Сине човешки, когато някоя земя Ми съгреши с коварство, и Аз простра ръката Си върху нея, и строша подпорката й от хляба, и пратя глад върху нея, и отсека от нея и човек и животно,
\par 14 та ако тия трима мъже, Ное, Даниил и Иов, бяха всред нея, те щяха да избавят само своите си души чрез правдата си, казва Господ Иеова.
\par 15 Ако бих докарал против земята люти зверове, и те я обезлюдяха, та да запустее, тъй че да не може някой да мине през нея от зверовете,
\par 16 пак, и тримата тия мъже ако бяха всред нея, заклевам се в живота Си, казва Господ Иеова, те не щяха да избавят нито синове нито дъщери, само сами те щяха да се избавят, а земята щеше да запустее.
\par 17 Или ако изнесях меч върху оная земя, и речех: Мечо, мини през земята, та да отсека от нея и човек и животно,
\par 18 то и тримата тия мъже ако бяха всред нея, заклевам се в живота Си, казва Господ Иеова, те не щяха да избавят ни синове ни дъщери, но само те щяха да се избавят.
\par 19 Или ако нанесях мор върху оная земя, и излеех яростта Си върху нея с кръв, та да отсека от нея и човек и животно,
\par 20 пак, заклевам се в живота Си, казва Господ Иеова, ако бяха всред нея Ное, Даниил и Иов, те не щяха да избавят нито син нито дъщеря, но щяха да избавят само своите си души чрез правдата си;
\par 21 прочее (така казва Господ Иеова), колко повече когато изпратя четирите Си тежки съдби върху Ерусалим, меча, глада, лютите зверове, и мора, та да отсека от него и човек и животно!
\par 22 Но, ето, ще останат в него избавилите се, синове и дъщери, които ще се изнесат; ето, те ще излязат при вас, и ще видите постъпките им и делата им; и ще се утешите относно злото, което нанесох върху Ерусалим, да! относно всичко що нанесох върху него.
\par 23 Те ще ви утешат, когато видите постъпките им и делата им; и ще познаете, че Аз не съм сторил без причина всичко това що съм сторил в него, казва Господ Иеова.

\chapter{15}

\par 1 И Господното слово дойде към мене и рече:
\par 2 Сине човешки, какво повече е лозовото дърво от кое да е друго дърво, Или лозовата пръчка, която е такава, каквито са и горските дървета?
\par 3 Ще се вземе ли дърво от него, за да се направи някаква работа? Или ще вземат ли от него колче, за да окачат на него някой съд?
\par 4 Ето, то се хвърля в огъня за гориво; Когато огънят е изял и двата му края Средата му е опърлена. Ще бъде ли полезно за някоя работа?
\par 5 Ето, когато бе цяло не ставаше за работа; Колко по-малко ще стане за работа, Когато го е изял огънят, и е опърлено!
\par 6 Затова, така казва Господ Иеова: Както предадохте на огъня за гориво Лозовото дърво между горските дървета, Така ще предам тия, които живеят в Ерусалим.
\par 7 Ще насоча лицето Си против тях; Те ще излязат из огъня, Но огънят ще ги пояде; И когато насоча лицето Си против тях Ще познаете, че Аз съм Господ.
\par 8 И ще предам земята на запустение, Защото станаха престъпници, казва Господ Иеова.

\chapter{16}

\par 1 Пак дойде към мене Господното слово и рече:
\par 2 Сине човешки, направи да познае Ерусалим мерзостите си, като речеш:
\par 3 Така казва Господ Иеова на ерусалимската дъщеря: Произходът ти и рождението ти е от Ханаанската земя; Баща ти бе амореец, а майка ти хетейка.
\par 4 А при раждането ти, В деня, когато си се родила, пъпът ти не е бил отрязан, Нито си била къпана във вода, за да се очистиш, Със сол не си била насолявана, нито в пелени повивана.
\par 5 Око не те е пощадило Та да ти направи някое от тия неща и да те пожали; Но в деня, когато си се родила, Понеже са се погнусили от тебе, Била си изхвърлена по лицето на полето.
\par 6 И когато заминах край тебе и те видях, че се валяше в кръвта си, Рекох ти като беше в кръвта си: Да си жива; Да! Рекох ти като беше в кръвта си: Да си жива.
\par 7 Направих те да нарастеш извънредно, като полската трева; И ти порасна и стана голяма, И достигна превъзходна красота; Гърдите ти се образуваха, и космите ти растяха; Но ти беше гола и непокрита.
\par 8 И когато заминах край тебе и те погледнах, Ето, възрастта ти беше любовна възраст; И тъй, като прострях полата Си върху тебе, та покрих голотата ти, Клех ти се, и стъпих в завет с тебе, Казва Господ Иеова; и ти стана Моя.
\par 9 Тогава те окъпах с вода; Да! измих кръвта ти от тебе, И помазах те с миро.
\par 10 Облякох те още във везани дрехи, И те обух с чехли от язовски кожи, Опасах те с висон, И те покрих с копринена покривалка.
\par 11 Украсих те още с накити, Турих гривни на ръцете ти И огърлица около шията ти,
\par 12 Турих и колелце на ноздрите ти, Обеци на ушите ти, И славен венец на главата ти.
\par 13 Така ти се украси със злато и сребро; И дрехите ти бяха от висон, коприна, и везано; Тя ядеше чисто брашно, мед, и масло; И стана превъзходно красива, И напреднала си дори до царско положение.
\par 14 Ти още се прочу между народите по твоята хубост, Защото тя стана съвършена чрез Моето величие, С което те облякох, казва Господ Иеова.
\par 15 Но поради известността си Ти си уповала на хубостта си и си блудствувала, И към всеки, който минаваше, си изливала блудството си; и то е станало негово.
\par 16 И взела си от дрехите си, И като си украсила издигнатите си места с пъстри шарки Си блудствувала на тях. Такова нещо, не е било, нито ще бъде.
\par 17 Тоже си взела лъскавите накити От Моето злато и от Моето с ребро, което Аз бях ти дал, И като си си направила от тях мъжки образи Си блудствувала с тях;
\par 18 Взела си и везаните си дрехи та си ги покрила, И турила си пред тях Моето масло и Моя темян;
\par 19 Също и Моя хляб, който ти дадох, - Чистото брашно, маслото и меда, с които те хранех, - Дори тия си положила пред тях за благоухание. Така стана, казва Господ Иеова.
\par 20 При това, си взела синовете и дъщерите, Които си Ми родила, И тях си им пожертвувала да бъдат изядени. Малко ли бяха тия твои блудства
\par 21 Та си заклала и чадата Ми, и си ги предала Да бъдат преведени през огъня за тях?
\par 22 И във всичките си мерзости и блудства Не си помнила дните на младостта си, Когато ти беше гола и непокрита и се валяше в кръвта си.
\par 23 И над всички тия си злини, (Горко, горко ти! казва Господ Иеова),
\par 24 Съградила си за себе си и блудилище, И на всяка улица си направила за себе си издигнато място,
\par 25 На всеки кръстопът си съградила издигнато място, Направила си хубостта си гнусна, И като си разтворила краката си на всеки, който минаваше, Блудствувала си твърде много.
\par 26 Блудствувала си и с Египтяните, Едрите си съседи, И като си блудствувала, твърде много си Ме разгневила.
\par 27 Ето, прочее, прострях ръката Си върху тебе, Намалих определената ти храна, и те предадох на волята на ония, Които те мразят, - на филистимските дъщери, Които се срамуват от твоите позорни постъпки.
\par 28 Блудствувала си и с асирийците, Защо не си се наситила; Да! блудствувала си с тях, Но пак не си се наситила.
\par 29 При това, си блудствувала твърде много. Дори до оная търговска земя, до Халдейската; И пак нито така си се наситила.
\par 30 Колко е слабо сърцето ти, казва Господ Иеова, Като вършиш всички тия работи, Дело на безочлива блудница!
\par 31 Защото си съградила блудилището си на всеки кръстопът, И си направила издигнатото си място във всяка улица, А пък не си станала като блудница, защото си презряла заплата,
\par 32 Но си била като жена прелюбодейца, Която приема чужди мъже вместо своя си.
\par 33 На всичките блудници дават заплата; Но ти даваш заплатите си и на всичките си любовници, И ги подкупваш, за да дохождат при тебе отвред И да блудствуват с тебе.
\par 34 Тъй че в блудствата ти стана с тебе противното на онова, що става с другите жени, Защото никой не те следва, за да блудствува с тебе; И понеже ти даваш заплата, а на тебе заплата не дават, Пак, поради това, става с тебе противното.
\par 35 Затова, чуй, блуднице, словото Господно.
\par 36 Така казва Господ Иеова: Понеже си изливала парите си, И голотата ти се е открила в блудствата ти с любовниците ти, И поради всичките твои мерзостни идоли, И поради кръвта на чадата ти, които си им принесла, -
\par 37 Поради тия неща, ето, Аз ще събера всичките твои любовници, С които си се наслаждавала, И всички, които си възлюбила, Заедно с всички, които си намразила, - Ще ги събера против тебе отвред, И ще им открия голотата ти, За да видят всичката ти голота.
\par 38 И ще те съдя Според както се съдят такива жени, Които прелюбодействуват и проливат кръв; И ще нанеса върху тебе кръвно наказание с ярост и ревнование.
\par 39 Тоже ще те предам в ръката им; И те ще съборят блудилището ти, Ще разорят издигнатите ти места, Ще те съблекат още от дрехите ти, Ще отнемат лъскавите ти накити, И ще те оставят гола и непокрита.
\par 40 И ще доведат против тебе много народ, Които ще те убият с камъни И ще те прободат със сабите си.
\par 41 Ще изгорят с огън къщите ти, И ще извършат съдби над тебе пред очите на много жени; И Аз ще те направя да се оставиш от блудството, И няма да даваш вече заплата.
\par 42 Така ще се уталожи яростта Ми против тебе, И ревнованието ми спрямо тебе ще престане; И за ще утихна, и няма вече да се гневя.
\par 43 Понеже не си си спомнила за дните на младостта си, Но си Ме дразнила във всичко това, Затуй, ето, и Аз ще възвърна върху главата ти постъпките ти, Казва Господ Иеова; И няма да извършиш тоя разврат Над всичките си мръсни дела.
\par 44 Ето, всеки, който си служи с поговорки, Ще употреби тая поговорка против тебе, Като рече: Каквато майката, такава и дъщерята.
\par 45 Ти си дъщеря на майка си, На оная, която отметна мъжа си и чадата си; Ти си сестра на сестрите си, Които отметнаха мъжете си и чадата си; Майка ви бе хетейка, а баща ви амореец.
\par 46 По-голямата ти сестра е Самария, Тя и дъщерите й, които живеят от ляво ти; А по-малката ти сестра е Содом и дъщерите й, Които живеят отдясно ти.
\par 47 А ти не си ходила само по техните пътища, Нито си направила само по техните мерзости, Но, като да беше това много малко, Си надминала разврата им във всичките им пътища.
\par 48 В живота Си се заклевам, казва Господ Иеова, Ни сестра ти Содом направи, ни дъщерите й, Както направи ти и твоите дъщери.
\par 49 Ето що беше беззаконието на сестра ти Содом, Нейното и на дъщерите й: Гордост, пресищане с храна, и безгрижно спокойствие; А сиромахът и немощният не подкрепяше;
\par 50 Те се и възгордяха, и вършеха мерзости пред мене; Затуй, като видях това, махнах ги.
\par 51 Също и Самария не е извършила нито половината от твоите грехове; Но ти си извършила много повече мерзости от тях, И оправда сестрите си във всичките мерзости, които си извършила.
\par 52 Ти, която съдеше сестрите си, понеси сега срама си: Поради съгрешенията, които си извършила, по-гнусни от техните, Те са по-праведни от тебе; Затова, засрами се и ти, и понеси срама си. Понеже си оправдала сестрите си.
\par 53 И Аз, като възвърна техните пленници, Пленниците на Содом и на дъщерите й, И пленниците на Самария и на дъщерите й, Ще възвърна тогава и твоите пленници заедно с тях,
\par 54 За да понасяш безчестието си, И да се срамуваш за всичко, що си извършила, Като си станала утеха за тях.
\par 55 Когато сестрите ти Содом и дъщерите й се върнат в предишното си състояние, И Самария и дъщерите й се върнат в предишното си състояние, Тогава ще се върнеш и ти и дъщерите ти в предишното си състояние.
\par 56 Защото в дните на гордостта ти Името на сестрата си Содом не се чуваше от устата ти,
\par 57 Преди да се открие твоето нечестие, Както се откри във времето, когато те укоряваха Сирийските дъщери, И всичките наоколо тях, и Филистимските дъщери, Които те презираха отвред.
\par 58 Ти трябва да понесеш наказанието за разврата си И мерзостите си, казва Господ.
\par 59 Защото така казва Господ Иеова: Аз ще постъпя с тебе както постъпи ти, Която си презряла клетвата и нарушила завета.
\par 60 Обаче ще помня завета Си, Сключен с тебе в дните на младостта ти, И ще ти утвърдя вечен завет.
\par 61 Тогава ще си спомниш за пътищата си, и ще се засрамиш, Когато приемеш сестрите си, по-големите си и по-малките си; И ще ти ги дам за дъщери, Но не по завета Ми с тебе.
\par 62 А Моя завет с тебе ще утвърдя; И ще познаеш, че аз съм Господ,
\par 63 За да си спомниш, и да се засрамиш, И да не отвориш вече устата си от срам, Когато ти простя за всичко, що си сторила, Казва Господ Иеова.

\chapter{17}

\par 1 И Господното слово дойде към мене и рече:
\par 2 Сине човешки, предложи гатанка, и кажи притча за Израилевия дом, като речеш:
\par 3 Така казва Господ Иеова: Един голям орел, с големи крила и дълги пера, С гъста, пъстрошарена перушина, Дойде в Ливан, та откъсна най-високото клонче на кедъра.
\par 4 Той откърши върховете на клончетата му Та ги занесе в търговска страна, Тури ги в търговски град.
\par 5 Взе и от семето на земята Та го пося на плодородна почва; Постави го при много води, посади го като върба.
\par 6 И то поникна и стана разстлана лоза, ниска на ръст, Чиито клончета се обръщаха към него, И корените й бяха под него. Така стана лоза, Пусна пръчки, и покара отрасли.
\par 7 Имаше и друг голям орел с големи крила и гъста перушина; И, ето, от лехите гдето беше посадена Тая лоза разпростря корените си към него, И простря клончетата си към него, за да я напои.
\par 8 Тя бе посадена на добра почва, при много води, За да пусне пръчки и да принесе плод, Та да стане добра лоза.
\par 9 Речи: Така казва Господ Иеова: Ще благоуспее ли? Не ще ли изскубе той корените й, И отсече плода й, та да изсъхне, Да изсъхнат всичките й пресни листа, Даже без да има голяма сила или много люде, Които да я изскубят из корен?
\par 10 Ето, дори ако и посадена, ще благоуспее ли? Не ще ли изсъхне съвсем Щом я досегне източният вятър? Ще изсъхне в лехите, гдето бе израснала.
\par 11 При това Господното слово дойде към мене и рече:
\par 12 Кажи сега на бунтовния дом: Не проумявате ли що значи това? Поясни им - Ето, вавилонският цар дойде в Ерусалим Та взе царя му и първенците му, И ги заведе със себе си във Вавилон;
\par 13 Взе и от царския род та свърза договор с него, Като го направи да се закълне; И отведе силните на страната,
\par 14 За да се унижи царството, И да се не издигне, Но да се утвърди като пази договора му.
\par 15 Обаче той въстана против него, и прати свои посланици в Египет, За да му дадат коне и много люде. Ще благоуспее ли? ще се избави ли оня, който прави това? Или като нарушава договора ще избегне ли?
\par 16 Заклевам се в живота Си, казва Господ Иеова, В мястото гдето живее царят, който го е поставил цар, Чиято клетва той е презрял и чийто договор е нарушил, Непременно там ще умре при него всред Вавилон.
\par 17 И Фараон със силната си войска и с голямото си множество Няма да направи за него нищо във войната, При все че издига могили и гради укрепления, за да погуби много души.
\par 18 Защото той презря клетвата и наруши договора; И, ето, след като даде ръка си, Пак стори всичко това. Той няма да се отърве.
\par 19 Затова, така казва Господ Иеова: Заклевам се в живота Си, наистина ще възвърна върху главата му Клетвата Ми, която е презрял, И договора Ми, който е нарушил.
\par 20 Ще разпростра мрежата Си върху него, Та ще се хване в примката Ми; Ще го закарам във Вавилон, И там ще се съдя с него за престъплението, Което извърши против Мене.
\par 21 И всичките му бежанци с всичките му полкове Ще паднат от нож, А останалите ще се разпръснат към всичките ветрища; И ще познаете, че Аз Господ изговорих това.
\par 22 Така казва Господ Иеова: Ще взема и от върха на високия кедър и ще го посадя; Ще откърша от върховете на младите му клончета едно крехко клонче, И ще го посадя на планина висока и отлична;
\par 23 На високата Израилева планина ще го посадя; И то ще изкара клончета, ще дава плод, И ще стане великолепен кедър; Под него ще обитават всички птици от всякакъв вид, Под сянката на клоните му ще живеят.
\par 24 И всичките дървета на полето ще познаят, Че Аз Господ сниших високото дърво, възвисих ниското дърво, Изсуших зеленото дърво, и направих сухото дърво да се раззеленее. Аз Господ изговорих това, и ще го извърша.

\chapter{18}

\par 1 Пак дойде към мене Господното слово и рече:
\par 2 Що искате да кажете вие, които употребявате тая поговорка относно Израилевата земя, като казвате: Бащите ядоха кисело грозде, а на чадата оскоминяха зъбите?
\par 3 Заклевам се в живота Си, казва Господ Иеова, няма вече да има повод да употребите тая поговорка в Израил.
\par 4 Ето, всичките души са Мои; както душата на бащата, така и душата на сина е Моя; душата, която е съгрешила, тя ще умре.
\par 5 Но ако е някой праведен и постъпва законно и право,
\par 6 ако не яде по планините, нито подига очите си към идолите на Израилевия дом, не осквернява жената на ближния си, нито се приближава до жена, кога е в нечистотата си,
\par 7 ако не насилва човека, но връща на длъжника залога му, не граби с насилие, но дава хляба си на гладния и покрива с дрехи голия,
\par 8 ако не дава с лихва и не взема придобивка, оттегля ръката си от неправда, върши правосъдие между човека и човека,
\par 9 ходи в повеленията Ми, и пази съдбите Ми, за да постъпва вярно; такъв човек е праведен, непременно той ще живее, казва Господ Иеова.
\par 10 Ако той роди син разбойник, който пролива кръв и върши коя да било от тия работи,
\par 11 и който, освен че не изпълнява ни една от тия длъжности, но и яде по планините и осквернява жената на ближния си,
\par 12 насилва сиромаха и немощния, граби с насилие, не връща залога, подига очите си към идолите и прави мерзости,
\par 13 дава с лихва и взема придобивка; такъв човек ще живее ли? Няма да живее. Като е сторил всички тия мерзости, непременно ще умре; кръвта му ще бъде върху него.
\par 14 Но ако тоя роди син, който като гледа всичките грехове, които баща му е сторил, бои се и не върши такива работи,
\par 15 не яде по планините, нито подига очите си към идолите на Израилевия дом, не осквернява жената на ближния си,
\par 16 не насилва човека нито задържа залог, не граби с насилие, но дава хляба си на гладния и покрива с дреха голия,
\par 17 не угнетява сиромаха, не взема лихва и придобивка, извършва съдбите Ми, и ходи в повеленията Ми; такъв човек няма да умре за беззаконието на баща си; непременно ще живее.
\par 18 Баща му, понеже жестоко е угнетявал, грабил от брата си с насилие, и правил между людете си това, което не е добро, ето, той ще умре за беззаконието си.
\par 19 Но вие казвате: Защо не понася синът наказанието на бащиното си беззаконие? Когато синът е постъпвал законно и право, пазил всичките Ми повеления, и ги е извършвал, той непременно ще живее.
\par 20 Душата, която греши, тя ще умре; синът няма да понесе наказанието на бащиното беззаконие, нито ще понесе бащата наказанието на синовото беззаконие; правдата на праведния ще бъде за него, и беззаконието на беззаконника ще бъде за него.
\par 21 Но ако беззаконникът се обърне от всичките грехове, които е сторил, пази всичките Ми повеления, и постъпва законно и право, непременно той ще живее, няма да умре.
\par 22 Никое от престъпленията, които е извършил, няма да се помни против него; чрез правдата, която е сторил, ще живее.
\par 23 Благоволя ли Аз в смъртта на нечестивия? казва Господ Иеова, а не по-добре да се обърне от пътя си и да живее?
\par 24 Когато обаче праведният се отвърне от правдата си и стори неправда, като извърши всичките мерзости, които нечестивият върши, тогава ще живее ли? Ни едно от праведните дела, които е сторил, няма да се помни; за престъплението, което е извършил, и за греха, който е сторил, за тях ще умре.
\par 25 Но вие казвате: Господният път не е прав. Слушайте сега, доме Израилев: Моят ли път не е прав? Не са ли криви вашите пътища?
\par 26 Когато се отвърне праведният от правдата си и извърши неправдата, ще умре за нея; поради неправдата, която е извършил, той ще умре.
\par 27 А пък когато нечестивият се обърне от нечестието, което е извършил и постъпи законно и право, той ще опази жива душата си.
\par 28 Понеже се е смилил и се е обърнал от всичките престъпления, които е сторил, непременно ще живее, няма да умре.
\par 29 Но Израилевият дом казва: Господният път не е прав. Доме Израилев, Моите ли пътища не са прави? Не са ли криви вашите пътища?
\par 30 Затова, доме Израилев, Аз ще ви съдя, всеки според постъпките му, казва Господ Иеова. Покайте се, и обърнете се от всичките си престъпления, та да ви не погуби беззаконието.
\par 31 Отхвърлете от себе си всичките престъпления, с които беззаконствувахте, и направете си ново сърце и нов дух; защо да умрете, доме Израилев?
\par 32 Понеже Аз не благоволя в смъртта на оня, който умира, казва Господ Иеова; затова, върнете се и живейте.

\chapter{19}

\par 1 При това, дигни плач за Израилевите първенци, и речи:
\par 2 Що беше майка ти? Лъвица. Лежеше между лъвовете, Хранеше лъвчетата си всред млади лъвове.
\par 3 Тя изхрани едно от лъвчетата си, Което, като стана млад лъв, И се научи да граби лов, ядеше човеци.
\par 4 И народите чуха за него; Хванат биде в ямата им. И с куки го закараха в Египетската земя.
\par 5 А тя, като видя, че надеждата й изчезна и се изгуби, Взе още едно от лъвчетата си Та направи и него млад лъв.
\par 6 И като ходеше между лъвовете Стана млад лъв, Който, когато се научи да граби лов, ядеше човеци.
\par 7 Той познаваше палатите им, И запустяваше градовете им; И от гласа на рева му Ужасяваше се земята и това, което я пълнеше.
\par 8 Тогава се опълчиха против него народите от околните области И простряха върху него мрежите си; Той се хвана в ямата им.
\par 9 И с куки го туриха в решетка Та го закараха при вавилонския цар; Туриха го в крепост, За да се не чуе вече гласът му По Израилевите планини.
\par 10 Майка ти с жизненост като твоята, Беше като лоза посадена при вода; Стана плодоносна и клонеста От многото води.
\par 11 Израснаха по нея яки жезли За скиптри на владетелите, И ръстът им стигна на високо всред гъстите клончета, И те се отличаваха с височината си И с многото си отрасли.
\par 12 Но тя биде изтръгната с ярост, Хвърлена бе на земята, И източен вятър изсуши плода й; Яките й жезли се счупиха, изсъхнаха; Огън ги изяде.
\par 13 А сега тя е посадена в пустиня, В суха и безводна земя.
\par 14 И огън излезе из един жезъл от клончетата й Та изяде плода й, Тъй щото няма вече в нея жезъл достатъчно як За владетелски скиптър. Това е плач, и ще служи за плач.

\chapter{20}

\par 1 А в седмата година, в петия месец, на десетия ден от месеца, някои от Израилевите старейшини дойдоха да се допитат до Господа, и седнаха пред мене.
\par 2 И Господното слово дойде към мене и рече:
\par 3 Сине човешки, говори на Израилевите старейшини и кажи им: Така казва Господ Иеова: Да се допитате до Мене ли сте дошли? Заклевам се в живота Си, казва Господ Иеова, не приемам да се допитате до мене.
\par 4 Ще се застъпиш ли за тях, сине човешки? ще се застъпиш ли за тях? Покажи им мерзостите на техните бащи, като им речеш:
\par 5 Така казва Господ Иеова: В деня, когато избрах Израиля, и се заклех към рода на Якововия дом, и им се открих в Египетската земя, и им се заклех като рекох: Аз съм Господ вашият Бог, -
\par 6 в оня ден заклех им се, че ще ги изведа из Египетската земя и ще ги доведа в земя, която бях промислил за тях, земя дето тече мляко и мед, с която се хвалят всички страни.
\par 7 И рекох им: Отхвърлете всеки мерзостите, които е поставил пред очите си, и не се осквернявайте с египетските идоли; Аз съм Господ вашият Бог.
\par 8 Те обаче въстанаха против Мене и не искаха да Ме послушат; не отхвърлиха всеки мерзостите, които бе поставил пред очите си, нито оставиха египетските идоли. Тогава рекох да излея яростта Си върху тях, за да изчерпя гнева Си против тях всред Египетската земя.
\par 9 Но подействувах заради името Си, да се не оскверни то пред народите, между които бяха, и пред които им се открих като ги изведох из Египетската земя,
\par 10 Затова като ги изведох из Египетската земя, заведох ги в пустинята.
\par 11 И дадох им повеленията Си и започнах ги със съдбите Си, които като извършва човек ще живее чрез тях.
\par 12 Дадох им и съботите Си да бъдат знак между Мене и тях, за да познаят, че Аз Господ ги освещавам.
\par 13 Но Израилевият дом въстана против Мене в пустинята; не ходиха в повеленията Ми, но отхвърлиха съдбите Ми, които като извършва човек ще живее чрез тях, и твърде много оскверниха съботите Ми. Тогава рекох да излея яростта Си върху тях в пустинята, за да ги изтребя.
\par 14 Но подействувах заради името Си, да се не оскверни то пред народите, пред които бях ги извел.
\par 15 При това, Аз им се заклех в пустинята, че не ще ги заведа в земята, която им бях дал, земя, гдето тече мляко и мед, с която се хвалят всичките страни,
\par 16 защото отхвърлиха съдбите Ми, не ходиха в повеленията Ми, и оскверниха съботите Ми;. понеже сърцата им отиваха след идолите им.
\par 17 Обаче, окото Ми ги пощади, та не ги изтребих, нито ги довърших в пустинята.
\par 18 И рекох на чадата им в пустинята: Не ходете по повеленията на бащите си, нито се осквернявайте с идолите им.
\par 19 Аз съм Господ вашият Бог; в Моите повеления ходете, и Моите съдби пазете и извършвайте ги;
\par 20 освещавайте още и съботите Ми, и нека бъдат знак между Мене и вас, за да познаете, че Аз Господ съм ваш Бог.
\par 21 Обаче, чадата въстанаха против Мене; не ходиха в повеленията ми, нито опазиха съдбите ми да ги вършат, които ако извършва човек, ще живее чрез тях, и оскверниха съботите Ми. Тогава рекох да излея яростта Си върху тях, за да изчерпя гнева Си против тях в пустинята.
\par 22 Но оттеглих ръката Си, и подействувах заради името Си, да се не оскверни то пред народите, пред които ги бях извел.
\par 23 При това, Аз им се заклех в пустинята, че ще ги разпръсна между народите, и ще ги разсея по разни страни,
\par 24 защото не извършиха съдбите Ми, но отхвърлиха повеленията Ми, оскверниха съботите Ми, и очите им бяха към идолите на бащите им.
\par 25 Затова, и Аз им дадох повеления, които не бяха добри, и съдби, чрез които не щяха да живеят,
\par 26 и оставих ги да се осквернят в приносите си, дето превеждаха през огън всяко първородно, за да ги запустя, та да познаят, че Аз съм Господ.
\par 27 Затова, сине човешки, говори на Израилевия дом, като им речеш: Така казва Господ Иеова: Още и в това Ме охулиха бащите ви, дето вършиха престъпления против Мене.
\par 28 Защото, като ги доведох в земята, която бях се заклел да им дам, тогава разгледаха всеки висок хълм и всяко сенчесто дърво, и там принасяха жертвите си, там представяха оскърбителните си приноси, там полагаха благоуханията си, и там изливаха възлиянията си.
\par 29 Тогава им рекох: Що е това високо място, на което отивате, та да се нарича Вама и до днес?
\par 30 Затова, кажи на Израилевия дом - Така казва Господ Иеова: Като се осквернявате по примера на бащите си, и блудствувате с мерзостите им,
\par 31 и до днес се осквернявате с всичките си идоли, като принасяте даровете си и превеждате синовете си през огъня, то до Мене ли ще се допитате, доме Израилев? Заклевам се в живота Си, казва Господ Иеова, не приемам да се допитате до Мене.
\par 32 И онова, което размишлявате, никак няма да се сбъдне, дето казвате: Ще станем като народите, като племената на другите страни; ще служим на дървета и на камъни.
\par 33 Заклевам се в живота Си, казва Господ Иеова, непременно с мощна ръка, с издигната мишца, и с излеяна ярост ще царувам над вас.
\par 34 Като ви изведа измежду племената, и ви събера от страните, в които сте разпръснати с мощна ръка, с издигната мишца, и с излеяна ярост,
\par 35 ще ви заведа в пустинята на заточение между племената, и там ще се съдя с вас лице с лице.
\par 36 Както се съдих с бащите ви в пустинята на Египетската земя, така ще се съдя с вас, казва Господ Иеова.
\par 37 Ще ви прекарам под жезъла и ще ви заведа във връзките на завета.
\par 38 И ще изчистя отсред вас бунтовниците и ония, които вършат престъпления против Мене; ще ги извадя из земята, гдето пришелствуват; но те няма да влязат в Израилевата земя; и ще познаете, че Аз съм Господ.
\par 39 А вие, доме Израилев, така казва Господ Иеова: Ако не искате да слушате Мене, идете, служете и за напред всеки на идолите си; но не осквернявайте вече светото Ми име с даровете си и с идолите си.
\par 40 Защото на Моя свят хълм, на Израилевия висок хълм, казва Господ Иеова, там целият Израилев дом, всичките в страната, ще Ми служат; там ще ги приема, и там ще поискам приносите ви и първаците на даровете ви, както и всичките ви свети неща.
\par 41 Ще ви приема като благоухание, когато ви изведа изсред племената и ви събера от страните, в които бяхте пръснати; и ще се осветя във вас пред народите.
\par 42 И ще познаете, че Аз съм Господ, когато ви доведа в Израилевата земя, в страната, която се заклех да дам на бащите ви.
\par 43 Там ще си спомните за постъпките си, и за всичките си дела, в които се осквернихте; и ще се погнусите сами от себе си поради всичките злини, които сторихте.
\par 44 И ще познаете, че Аз съм Господ, когато така постъпя с вас заради името Си, а не според злите ви постъпки нито според развратните ви дела, доме Израилев, казва Господ Иеова.
\par 45 И Господното слово дойде към мене и рече:
\par 46 Сине човешки, насочи лицето си към юг и направи да капне словото ти към юг и пророкувай против леса на южното поле.
\par 47 Кажи на южния лес: Слушай Господното слово. Така казва Господ Иеова: Ето, Аз ще запаля огън в тебе, който ще изяде всяка зелено дърво в тебе, и всяко сухо дърво; пламтящият пламък не ще угасне, а от него ще изгори цялата повърхност от юг до север.
\par 48 Всяка твар ще види, че Аз Господ го запалих; няма да угасне.
\par 49 Тогава рекох: Горко, Господи Иеова! Те казват за мене: Този не говори ли притчи?

\chapter{21}

\par 1 И Господното слово дойде към мене и рече:
\par 2 Сине човешки, насочи лицето си към Ерусалим, и направи да капне словото ти към светите места, и пророкувай против Израилевата земя;
\par 3 И кажи на Израилевата земя: Така казва Господ: Ето, Аз съм против тебе, и като изтегля ножа Си из ножницата ще отсека от тебе и праведния и нечестивия.
\par 4 Прочее, понеже ще отсека от тебе и праведния и нечестивия, затова ножът Ми ще излезе из ножницата си против всяка твар, от юг до север;
\par 5 и всяка твар ще познае, че Аз Господ изтеглих ножа Си из ножницата му; няма да се върне вече.
\par 6 Затова ти, сине човешки, въздъхни; със съкрушен кръст и с огорчение въздъхни пред тях.
\par 7 И когато ти рекат: Защо въздишаш? кажи: Поради известието, че иде; всяко сърце ще се стопи, всичките ръце ще ослабват, всеки дух ще примре, и всичките колена ще станат като вода; ето, иде, и ще се сбъдне, казва Господ Иеова.
\par 8 И Господното слово дойде към мене и рече:
\par 9 Сине човешки, пророкувай, казвайки: Така казва Господ: Речи: Меч! меч се остри, още се излъсква;
\par 10 остри се, за да извърши голямо клане; излъсква се, за да лъщи. Можем ли, прочее, да се веселим? Това е жезълът на сина ми, който презира всяко дърво.
\par 11 Даде се да се излъска да се държи в ръка; тоя меч е наострен и излъскан, за да се даде в ръката на погубителя.
\par 12 Извикай и излелекай, сине човешки, защото той е върху людете Ми, върху всичките Израилеви първенци; ужас нападна людете Ми поради меча; затова, удари по бедрото си.
\par 13 Защото има изпитание; и какво ако и презиращия жезъл не би съществувал вече, казва Господ Иеова?
\par 14 Ти, прочее, сине човешки, пророкувай, и изпляскай с ръце; и нека мечът удвои, нека мечът утрои, числото на ранените; той е мечът на ранените големци, който ще ви обсади отвред.
\par 15 Нанесох ужаса на меча против всичките им порти, за да се стопи всяко сърце, и за да се спъват по-често. Уви! приготви се, за да блести, наточи се, за да коли.
\par 16 Стегни се мечо, нападни надясно; насочи се, нападни наляво; където и да се обърне лицето ти.
\par 17 И Аз ще плесна с ръце и ще удовлетворя яростта Си. Аз Господ изговорих това.
\par 18 Господното слово дойде пак към мене и рече:
\par 19 При това, сине човешки, ти си определи два пътя, за да замине мечът на Вавилонския цар; и двата ще излизат от същата земя; и направи показалец, направи го при началото на пътя за града.
\par 20 Определи път, за да мине мечът в Рава на амонците и в укрепения Ерусалим в Юда.
\par 21 Защото Вавилонският цар се спря при раздвояването на пътя, гдето започват двата пътя, за да почародествува; разтърси стрелите, допита се до терафимите, прегледа черния дроб.
\par 22 В десницата му е жребието за Ерусалим, за да се поставят стеноломи, да се даде заповед за клане, да се издигне глас с възклицание, да се поставят стеноломи срещу портите, да се издигнат могили, да се съградят крепости.
\par 23 Но това ще бъде като суетно чародеяние пред очите на ония, които им се заклеха; обаче той ще им напомни беззаконието, за да се хванат.
\par 24 Затова, така казва Господ Иеова: Понеже направихте да се помни беззаконието ви с откриването на престъпленията ви, тъй щото да се явят греховете ви във всичките ви дела, - понеже направихте себе си да бъдете спомнени, ще бъдете хванати в ръце.
\par 25 А ти, смъртоносно ранени, скверни княже Израилев, чийто ден е настъпил, когато беззаконието е стигнало до края си,
\par 26 така казва Господ Иеова: Снеми митрата и свали короната; тя няма вече да бъде такава; възвиси смирения, а смири възвисения.
\par 27 Аз ще я катурна, катурна, катурна, та и това няма да трае, докле дойде оня, комуто принадлежи; и нему ще я дам.
\par 28 И ти сине човешки, пророкувай като речеш: Така казва Господ Иеова за амонците и за тяхното укоряване. Кажи: Меч, меч е изтеглен, излъскан, за да коли, и блестящ, за да изтреби,
\par 29 докато те виждат за тебе суетно видение, и докато чародествуват на тебе лъжа, от което ще бъдеш прострян на врата на смъртно ранените нечестивци, чиито ден е настъпил, когато беззаконието е стигнало до края си.
\par 30 Върни го в ножницата му. На мястото гдето си бил създаден, в родната ти земя, ще те съдя.
\par 31 Ще излея негодуванието Си върху тебе, с огнения Си гняв ще духна върху тебе, и ще те предам в ръцете на скотски мъже, които са вещи да погубят.
\par 32 Ще станеш гориво за огън; кръвта ти ще бъде всред земята ти; не ще има вече спомен за тебе; защото Аз Господ изговорих това.

\chapter{22}

\par 1 При това, Господното слово дойде към мене и рече:
\par 2 А ти, сине човешки, ще се застъпиш ли, ще се застъпиш ли за кръвопролитния град? Тогава направи го да познае всичките си мерзости.
\par 3 Речи, прочее: Така казва Господ Иеова: О, граде, който проливаш кръв всред себе си, та да дойде времето ти, и правиш кумири за своето осъждение, та да се оскверняваш!
\par 4 Ти стана виновен за кръвта, която си пролял, и си се осквернил с кумирите, които си направил; направил си да наближава краят на дните ти, и дошъл си дори до края на на годините си; затова, направих те за укор на народите, и за поругание на всичките страни.
\par 5 Ближните и далечните от тебе ще ти се поругаят, ти който си прочут по мерзост и изобилваш с безмирие.
\par 6 Ето, Израилевите първенци са били в тебе, за да проливат кръв, всеки според силата си.
\par 7 В тебе са презирали баща и майка; всред тебе са постъпили насилствено към чужденеца; в тебе са угнетявали сираче и вдовица.
\par 8 Презирал си светите Ми вещи, и си осквернявал съботите Ми.
\par 9 В тебе е имало мъже клеветници, за да проливат кръв; в тебе е имало ония, които са яли по планините; всред тебе са вършили разврат.
\par 10 В тебе са откривали бащината си голота; в тебе са обезчестявали жена, отлъчена поради нечистотата й.
\par 11 Един е извършил гнусота с жената на ближния си; друг е осквернил нечестиво снаха си; а друг в тебе е обезчестил сестра си, дъщеря на баща си.
\par 12 В тебе са вземали подкупи, за да проливат кръв; ти си вземал лихва и придобивка; и с насилие си се обогатявал от ближните си; а Мене си забравил, казва Господ Иеова.
\par 13 Ето, затова плеснах с ръце поради безчестната печалба, която си събрал, и поради кръвта, която бе всред тебе.
\par 14 Ще издържи ли сърцето ти, или ще имат ли сила ръцете ти, в дните, когато Аз ще се разправя с тебе? Аз Господ изговорих това, и ще го извърша.
\par 15 Ще те разпръсна между народите и разсея по страните, та ще изчистя от тебе като с огън нечистотата ти.
\par 16 И ще бъдеш омърсен в себе си пред народите; И ще познаеш, че Аз съм Господ.
\par 17 И Господното слово дойде към мене и рече:
\par 18 Сине човешки, Израилевият дом ми стана шлак; те всички са мед и калай, желязо и олово всред пещта; те са шлак от сребро.
\par 19 Затова, така казва Господ Иеова: Понеже вие всички станахте шлак, ето, затова ще ви събира всред Ерусалим.
\par 20 Както събират всред пещта среброто и медта, желязото, оловото, и калая, за да раздухат огъня върху тях та да ги стопят, така в гнева Си и в яростта Си ще ви събера, и ще ви туря там и ви разтопя.
\par 21 Да! Ще ви събера, и с огнения Си гняв ще духна върху вас, та ще се разтопите всред него.
\par 22 Както се топи среброто в пещта, така ще се разтопите вие всред него, и ще познаете, че Аз Господ излях яростта Си върху вас.
\par 23 И Господното слово дойде към мене и рече:
\par 24 Сине човешки, кажи й: Ти си земя, която не се е очистила, и върху която не е валяло дъжд в деня на негодуванието.
\par 25 Всред нея има заговор от пророците й; те поглъщат души като лъв, който реве и граби лова; вземат съкровища и скъпоценни вещи; умножиха числото на вдовиците всред нея.
\par 26 Свещениците й престъпваха закона Ми и оскверняваха светите Ми вещи; не правеха разлика между свето и скверно, нито показваха на хората различието между нечисто и чисто; и криеха очите си от съботите Ми; и Аз съм осквернен всред тях.
\par 27 Първенците всред нея са като вълци, които грабят лов, за да проливат кръв, за да погубват души, за да се обогатяват несправедливо.
\par 28 И пророците й я мажеха с кал, като виждаха за тях суетни видения и пророкуваха лъжи, думайки: Така казва Господ Иеова, когато Господ не бе говорил.
\par 29 Людете на тая земя прибягваха до притеснение и грабеха насилствено, да! угнетяваха сиромаха и немощния, и притесняваха чужденеца неправедно.
\par 30 И като потърсих между тях мъж, който би застанал в пролома пред Мене заради страната, та да я не разоря, не намерих.
\par 31 Затова, излях негодуванието Си върху тях, довърших ги с огнения Си гняв, и възвърнах върху главите им техните постъпки, казва Господ Иеова.

\chapter{23}

\par 1 Господното слово пак дойде към мене и рече:
\par 2 Сине човешки, имаше две жени, дъщери на една майка.
\par 3 Те блудствуваха в младостта си; там бяха налягани гърдите им, и там бяха стискани девствените им съсци.
\par 4 Имената им бяха Оола на по-голямата, и Оолива на сестра й; и те станаха Мои, и родиха синове и дъщери. А колкото за имената им, Оола е Самария, а Оолива Ерусалим.
\par 5 Но когато беше Моя Оола блудствува, и залудя за любовниците си, съседите си асирийците,
\par 6 синьо облечени управители и началници, всички привлекателни младежи, конници възсядали на коне.
\par 7 И тя извърши блудствата си с тях, с всичките най-отбрани между асирийците; и за които да било залудяваше, с всички техни идоли тя се оскверни.
\par 8 И тя не остави блудството си научено от Египет; защото лежеха с нея в младостта й, изтискаха девствените й съсци, и изливаха върху нея блудството си.
\par 9 Затова, предадох я в ръцете на любовниците й, в ръцете на асирийците, за които залудяваше.
\par 10 Те откриваха голотата й; откараха синовете и дъщерите й, а нея убиха с меч; и тя стана за посмешище между жените, защото извършиха съдби над нея.
\par 11 А когато сестра й Оолива видя това, в полудата си разврати се повече от нея, и в блудствата си надмина блудствата на сестра си.
\par 12 Залудя за съседите си асирийците светло облечени управители и началници, конници, възсядали на коне, всички привлекателни младежи.
\par 13 И видях, че тя се оскверни; и двете тръгнаха в един път.
\par 14 Тя даже притури на блудствата си; защото щом видя мъже изрисувани на стената, изрисувани с киновар образи на халдейци,
\par 15 опасани с пояси около кръста си, носещи шарени гъжви на главите си, и всички с княжески изглед, прилични на вавилоняните от Халдейската земя, родината им, -
\par 16 щом ги видя, тя залудя за тях, и изпрати посланици при тях в Халдея.
\par 17 И вавилоняните дойдоха при нея в любовното легло и я оскверниха с блудството си; тя се оскверни с тях; и душата й се отврати от тях.
\par 18 Така тя още откриваше блудствата си, откриваше голотата си; тогава душата Ми се отврати от нея, както беше се отвратила душата Ми от сестра й.
\par 19 При все това, тя блудствуваше още повече, като си спомняше за дните на младостта си, когато блудствуваше в Египетската земя.
\par 20 И залудя за любовниците си измежду тях, чиято плът е като плътта на осли, и семеизливането им като семеизливане на коне.
\par 21 Така ти си ламтяла за невъздържаността на младостта си, когато гърдите ти се налягаха от Египтяните заради младите ти съсци.
\par 22 Затова, Ооливо, така казва Господ Иеова: Ето, Аз ще подигна любовниците ти против тебе, от които се е отвратила душата ти, и ще ги докарам против тебе от всякъде;
\par 23 вавилоняните и всичките халдейци, Фекод и Сое и Кое, и с тях всичките асирийци, - всички привлекателни младежи, управители и началници, пълководци и именити, всички възсядали на коне.
\par 24 Те ще дойдат против тебе с оръжия и с колесници, с коли и с много племена, и ще се опълчат против тебе от всяка страна с щитове, щитчета и шлемове; и Аз ще им поверя съдба, и те ще съдят според своите съдби.
\par 25 И като насоча против тебе ревнивостта Си, те ще постъпят с тебе яростно, ще отрежат носа ти и ушите ти; и останалите от тебе ще паднат от нож; те ще хванат синовете ти и дъщерите ти; и останалите от тях между тебе ще бъдат поядени от огън.
\par 26 Ще съблекат и дрехите ти, и ще отнемат лъскавите ти накити.
\par 27 Така ще направя да престане твоят разврат, и блудството ти, научено от Египетската земя; ти няма вече да подигнеш очи към тях, нито ще си наумиш вече за Египет;
\par 28 Защото така казва Господ Иеова: Ето, ще те предам в ръката на ония, които мразиш, в ръката на ония, от които се е отвратила душата ти.
\par 29 Те ще постъпят с омраза към небе, и като отнемат всичките ти трудове, ще те оставят гола и непокрита; и голотата на блудствата ти ще се открие, както разврата ти тъй и блудствата ти.
\par 30 Това ще ти се направи понеже си блудствувала всред езичниците, и понеже си се осквернила с идолите им.
\par 31 Така си ходила в пътя на сестра си; затова ще дам в ръката ти нейната чаша.
\par 32 Така казва Господ Иеова: Ще изпиеш дълбоката и широка чаша на сестра си; ще бъдеш за присмех и поругание повече, отколкото можеш да понесеш.
\par 33 Ще се изпълниш с пиянство и скръб, с чашата на смайването и на запустението, с чашата на сестра си Самария.
\par 34 Ще я изпиеш и изцедиш, ще гризеш черепките й, и ще разкъсаш гърдите си; защото Аз го изрекох, казва Господ Иеова;
\par 35 Затова, така казва Господ Иеова: Понеже си Ме забравила, и си Ме отхвърлила зад гърба си; затова понеси и ти възмездието на разврата си и блудствата си.
\par 36 При това, Господ ми рече: Сине човешки, ще се застъпиш ли за Оола и Оолива? Тогава изяви им мерзостите им, -
\par 37 че прелюбодействуваха, че има кръв в ръцете им, да! че прелюбодействуваха с идолите си, и че, за да бъдат изядени от тях, превеждаха през огън чадата, които Ми родиха;
\par 38 още че това Ми сториха, - в същия ден оскверниха светилището Ми и омърсиха съботите Ми;
\par 39 защото когато бяха заклали чадата си на идолите си, тогава в същия ден влизаха в светилището Ми, та го омърсиха; и, ето, така струваха всред дома Ми.
\par 40 И още пратихте за мъже да дойдат от далеч, до които като се проводи пратеник, ето, дойдоха; и за тях си се окъпала, вапсала си очите си, и украсила си се с накити,
\par 41 седнала си на великолепна постелка, с трапеза приготвена пред нея, и на нея си положила Моя темян и Моето масло.
\par 42 И чуваха се гласове на едно множество, което живееше безгрижно при нея; и заедно с мъжете от простолюдието се въвеждаха пияници из пустинята; и туриха гривни на ръцете на тия две жени, и красиви венци на главите им.
\par 43 Тогава рекох за престарялата в прелюбодействува: Сега ли ще блудствуват с нея, дори с нея!
\par 44 И те влизаха при нея както влизат при блудница; така влизаха при тия невъздържани жени, при Оола и при Оолива.
\par 45 Затова, справедливи мъже ще ги съдят както съдят прелюбодейците и както съдят жени, които проливат кръв; защото те са прелюбодейци, и има кръв в ръцете им.
\par 46 Защото така казва Господ Иеова: Ще доведа множество против тях, и ще ги предам да бъдат тласкани и разграбени.
\par 47 Множеството ще ги убие с камъни, и ще ги съсече със сабите си; ще избият синовете им и дъщерите, и ще изгорят къщите им с огън.
\par 48 Така ще направя да престане разврата от земята, за да се научат всичките жени да не вършат разврат като вашия.
\par 49 И ще въздадат върху вас развратните ви дела; и ще понасяте възмездието на греховете, извършени с идолите си; и ще познаете, че Аз съм Господ Иеова.

\chapter{24}

\par 1 А в деветата година, десетия месец, на десетия ден от месеца, Господното слово пак дойде към мене и рече:
\par 2 Сине човешки, запиши си името на тоя ден, на тоя същия ден; защото в тоя същия ден вавилонският цар се доближи до Ерусалим.
\par 3 И произнеси притча към бунтовния дом, като им речеш: Така казва Господ Иеова: Тури котела, тури, налей тоже вода в него.
\par 4 Събери в него късовете за варене, всеки добър къс, бедро и рамо; напълни го с отбрани кости.
\par 5 Вземи отбраните на стадото, натрупай и кости под него; направи го да ври добре, и костите в него да се сварят.
\par 6 Защото така казва Господ Иеова: Горко на кръвопролитния град, на котела, чиято ръжда е на него, и чиято ръжда не се е очистила от него! Извади от него късовете му, без да се хвърли жребие за тях.
\par 7 Защото кръвта му е всред него; тя я изложи на гол камък; не я изля на земята, та да се покрие с пръст.
\par 8 Аз изложих на гол камък кръвта, която изля, за да се не покрие, за да направя да избухне ярост, та да извърши въздаяние.
\par 9 Затова, така казва Господ Иеова: Горко на кръвопролитния град! защото и Аз ще направя по-голяма огнената грамада.
\par 10 Натрупай дървата, сгорещете огъня, свари добре месото, сгъсти варивото, и нека изгорят костите.
\par 11 Тогава тури котела празен на въглищата, за да се нажежи медта му и да изгори, и да се стопи в него нечистотата му, за да се изгори ръждата му.
\par 12 Уморил се е от трудовете си, но пак многото му ръжда не се очиства из него; ръждата му даже в огъня не се очиства.
\par 13 Понеже Аз те чистех, а ти не се очисти, затова, поради гнусния ти разврат няма вече да се очистиш от нечистотата си, докато не уталожа върху тебе яростта Си.
\par 14 Аз Господ го изговорих; това ще се сбъдне, и ще го извърша; няма да отстъпя, няма да пощадя, и няма да се разкая; според постъпките ти и според делата ти ще те съдят, казва Господ Иеова.
\par 15 При това Господното слово дойде към мене и рече:
\par 16 Сине човешки, ето, Аз с един удар ще отнема от тебе желанието на очите ти; а ти да не жалееш или плачеш, нито да потекат сълзите ти.
\par 17 Въздишай, но не с глас, да не жалееш за мъртви; завий гъжвата на главата си, и обуй обущата на нозете си; да не покриеш устните си, нито да ядеш хляба на жалеещи човеци.
\par 18 И тъй, говорих на людете заранта, а вечерта жена ми умря; и сутринта сторих както ми бе заповядано.
\par 19 Тогава людете ми рекоха: Не ще ли ни обясниш що значи за нас това, което правиш?
\par 20 Тогава им казах: Господното слово дойде към мене и рече:
\par 21 Говори на Израилевия дом: Така казва Господ Иеова: Ето, ще оскверня светилището Си, силата ви с която се гордеете, желателното на очите ви, и това, за което душите ви милеят; и синовете ви и дъщерите ви, които сте оставили, ще паднат от меч.
\par 22 И вие ще направите както направих аз; няма да покриете устните си, и хляб на жалеещи човеци няма да ядете;
\par 23 гъжвите ви ще бъдат на главите ви, и обущата на нозете ви; няма да жалеете, нито да плачете; но ще се стопите в беззаконията си, и ще охкате един към друг.
\par 24 Така Езекиил ще ви бъде знамение; всичко що направи той ще направите и вие; когато настане това, тогава ще познаете, че Аз съм Господ Иеова.
\par 25 А колкото за тебе, сине човешки, в оня ден, когато им отнема силата им, славата им, на която се радват, желанието на очите им, и милите на душите им, синовете им и дъщерите им, -
\par 26 в оня ден не ще ли дойде при тебе оня, който избягва, за да извести това в ушите ти?
\par 27 В оня ден устата ти ще се отворят към онзи, който избягва, ще говориш и няма да бъдеш вече ням; така ще им бъдеш знамение; и те ще познаят, че Аз съм Господ.

\chapter{25}

\par 1 И Господното слово дойде към мене и рече:
\par 2 Сине човешки, насочи лицето си към амонците и пророкувай за тях.
\par 3 Кажи на амонците: Слушайте словото на Господа Иеова: Така казва Господ Иеова: Понеже си рекъл: О хохо! против светилището Ми, когато биде осквернено, и против Израилевата земя, когато запустя, и против Юдовия дом, когато отидоха в плен,
\par 4 затова, ето, ще те предам на жителите на изток да те владеят; и те ще поставят селенията си в тебе, и ще разпрострат шатрите си в тебе; ще ядат плодовете ти, и ще пият млякото ти.
\par 5 И ще направя Рава обор на камили, и земята на амонците място гдето да лежат стада; и ще познаете, че Аз съм Господ.
\par 6 Защото така казва Господ Иеова: Понеже си изплескал с ръце, и тропнал с нога, и зарадвал си се против Израилевата земя с всичкото презрение на сърцето си,
\par 7 затова, ето, ще простра ръката Си върху тебе, ще те предам на народите за грабеж, ще те отсека от племената, и ще те направя да погинеш от земите; ще те изтребя; и ще познаеш, че Аз съм Господ.
\par 8 Така казва Господ Иеова: Понеже Моав и Сиир думат: Ето, Юдовият дом стана като всичките народи,
\par 9 затова, ето, ще отворя Моавовата граница откъм градовете, откъм неговите погранични градове, Вет-есимот, Ваалмеон, и Кириатаим, славата на оная земя, -
\par 10 ще я отворя за жителите на изток, за да отидат против амонците, и ще им ги предам да ги владеят, за да се не помнят амонците между народите.
\par 11 И ще извърша съдби върху Моава; и ще познаят, че Аз съм Господ.
\par 12 Така казва Господ Иеова: Понеже Едом се отнесе отмъстително към Юдовия дом, и престъпи тежко като си отмъсти против тях,
\par 13 затова, така казва Господ Иеова: Ще простра ръката Си върху Едом, ще отсека от него и човек и животно, и ще го запустя из Теман, та ще паднат от меч до Дедан.
\par 14 И ще наложа въздаянието Си върху Едом чрез ръката на людете си Израиля; и те ще постъпят с Едом според гнева Ми и според яростта Ми; и ще познаят въздаянието Ми, казва Господ Иеова.
\par 15 Така казва Господ Иеова: Понеже Филистимците се отнесоха отмъстително, и си отмъстиха с душевно презрение, за да погубят с непрекъсната омраза,
\par 16 затова, така казва Господ Иеова: Ето, аз ще простра ръката Си върху филистимците, ще изсека херетците, и ще погубя останалите край приморието;
\par 17 И ще извърша върху тях голямо въздаяние с яростни изобличения; и когато извърша въздаянието Си върху тях, ще познаят, че Аз съм Господ.

\chapter{26}

\par 1 А в единадесетата година, на първия ден от месеца, Господното слово дойде към мене и рече:
\par 2 Сине човешки, понеже Тир рече против Ерусалим: О хохо! строши се оня, който беше порта на племената! обърна се към Мене! ще се напълня аз като запустя той!
\par 3 Затова, каза казва Господ Иеова: Ето, Аз съм против тебе, Тире, и ще подигна против тебе много народи, както морето подига вълните си.
\par 4 Те ще сринат стените на Тир и ще съборят кулите му; и Аз ще изстържа пръстта от него, и ще го направя гола скала.
\par 5 Ще бъде място за простиране мрежи всред морето; защото Аз го изрекох, казва Господ Иеова; и ще стане корист на народите,
\par 6 И селата му, които са в полето ще бъдат изтребени с нож; и ще познаят, че Аз съм Господ.
\par 7 Защото така казва Господ Иеова: Ето, ще доведа от север против Тир вавилонския цар Навуходоносор, цар на царе, с коне, с колесници, и с конници, с множество, и с много люде.
\par 8 Той ще изтреби с нож селата ти в полето; а против тебе ще издигне укрепления, ще направи могили против тебе, и ще се опълчи против тебе с щитове;
\par 9 Ще постави бойните си оръдия против стените ти, и със секирите си ще разори кулите ти.
\par 10 Понеже конете му са много, прахът им ще те покрие; стените ти ще се потресат от шума на конниците, на колите, и на колесниците, когато влезе той в портите ти, както влизат в проломен град.
\par 11 С копитата на конете ще стъпче всичките ти улици; ще изсече с нож людете ти; и яките ти стълбове ще бъдат повалени на земята.
\par 12 Те ще разграбят богатството ти, и ще оберат имота ти, ще съборят стените ти, ще разорят красивите ти къщи, и ще хвърлят всред водата камъните ти, дърветата ти, и пръстта ти.
\par 13 Ще направя да престане шумът на песните ти; и звукът на китарите ти няма вече да се чува.
\par 14 И ще те направя гола скала, та ще бъдеш място за простиране мрежи; няма да се съградиш вече; защото Аз Господ го изрекох, казва Господ Иеова.
\par 15 Така казва Господ Иеова на Тир: Не ще ли се потресат островите от шума на падането ти, когато ранените охкат, когато става клането всред тебе?
\par 16 Тогава всичките морски големци ще слязат от престолите си, ще отметнат мантиите си, и ще съблекат везаните си дрехи; ще се облекат с трепет; на земята ще насядат, ще треперят всяка минута, и ще се удивляват за тебе.
\par 17 И като дигнат плач за тебе, ще ти рекат: Как биде съсипан ти, който се населяваше от мореплаватели, прочути граде, който бе силен в морето, ти и жителите ти, които причиняваха трепет на всички, които се намираха по него!
\par 18 Сега островите ще треперят в деня на падането ти; да! островите, които са в морето, ще се смутят, когато се изгубиш.
\par 19 Защото така казва Господ Иеова: Когато те направя пуст град, като ненаселените градове, когато докарам върху тебе бездната, та те покрият големи води,
\par 20 когато те смъкна с ония, които слизат в ямата при древните люде, и те туря в най-дълбоките места на света, в места пусти от века, с ония, които слизат в ямата, за да не бъдеш вече обитаем, тогава ще поставя слава в земята на живите.
\par 21 Ще направя да треперят народите от участта ти и няма да съществуваш; и ако те потърсят, ти няма вече да бъдеш намерен до века, казва Господ Иеова.

\chapter{27}

\par 1 Пак Господното слово дойде към мене и рече:
\par 2 И ти, сине човешки, дигни плач за Тир, и кажи на Тир:
\par 3 Ти, който седиш при входа на морето, който търгуваш с народите в много острови, така казва Господ Иеова: Тире, ти си рекъл: Аз съм съвършен по хубост.
\par 4 Пределите ти са всред моретата; ония, които те съградиха, направиха съвършена хубостта ти.
\par 5 Направиха всичките дъски на корабите ти от санирски елхи; взеха кедри от Ливан, за да ти направят мачти.
\par 6 Направиха веслата ти от Васански дъбове; направиха седалищата ти от слонова кост и букс от китайските острови.
\par 7 От висон с везана работа от Египет беше платното ти, за да ти служи за знаме; синьо и мораво от островите на Елиса, бе покрова ти.
\par 8 Жителите на Сидон и на Арвад бяха твоите веслари; твоите мъдри, които бяха в тебе, Тире, бяха кормчиите ти.
\par 9 Старейшините на Гевал и мъдрите му насмоляваха в тебе корабите ти; всичките морски кораби и моряците им бяха в тебе за да вършат търговията ти.
\par 10 Персийци, лидийци и ливийци бяха твоите войници, твоите военни мъже; в тебе окачваха щитове и шлемове; те ти придаваха великолепие.
\par 11 Арвадските мъже с войската ти бяха изоколо на стените ти, адгамадците на кулите ти; окачваха щитовете си изоколо на стените ти; те направиха съвършена хубостта ти.
\par 12 Тарсис търгуваше с тебе с изобилие от всякакво богатство; със сребро, желязо, калай и олово търгуваха за твоите стоки.
\par 13 Яван, Тувал, и Мосох търгуваха с тебе; за твоите стоки разменяха човешки души и медни съдове.
\par 14 Ония от дома Тогарма търгуваха за стоките ти с коне, военни коне, и мъски.
\par 15 Деданските мъже търгуваха с тебе, търговията на много острови бе в ръката ти: докарваха ти в размяна слонова кост и ебен.
\par 16 Сирия търгуваше с тебе поради многото ти изделия; даваше за стоките ти антракс и мораво, везано и висон, корали и рубини.
\par 17 Юда и Израилевата земя търгуваха с тебе; даваха за стоките ти менитско жито, сухари и мед, масло и балсама.
\par 18 Дамаск търгуваше с тебе, поради многото ти изделия, с изобилие от всяко богатство, с хелвонско вино и с бяла вълна.
\par 19 Ведан и Яван даваха прежда за стоките ти. Изработено желязо, касия и благоуханна тръстика бяха между стоките ти.
\par 20 Дедан търгуваше с тебе със скъпи платове за колесници.
\par 21 Аравия и всичките кидарски първенци бяха търговци в тебе, и търгуваха с тебе с агнета, овце и козли.
\par 22 Търговците на Шева и на Рама търгуваха с тебе, и даваха за стоките ти всякакво изрядно благоухание, всякакви скъпоценни камъни, и злато.
\par 23 Харан, Хане и Еден, търговците на Шева, Асур и Хилмад, търгуваха с тебе.
\par 24 Тия търгуваха с тебе с отбрани стоки, с бали от синя и везана изработка, и с ковчези богати облекла, вързани с въжа и направени от кедър. Тия бяха между търговците ти.
\par 25 Тарсийските кораби обикаляха с търговията ти; и ти стана пълен и бе твърде славен всред моретата.
\par 26 Твоите веслари те заведоха в големи води; но източният вятър те разби всред моретата.
\par 27 Богатството ти и стоките ти, търговията ти и моряците ти, кормчиите ти и които насмоляваха корабите ти, разменителите на стоките ти и всичките военни мъже, които са в тебе, с всичкото множество, което е всред тебе, ще паднат всред моретата в деня на погибелта ти.
\par 28 Предместията ще се потресат от гласа на писъка на твоите кормчии.
\par 29 И всичките веслари, моряците, и всичките морски кормчии ще слязат из корабите си, ще застанат на земята,
\par 30 и ще извикат с глас над тебе, ще писнат горко, ще посипят пръст на главите си, и ще се валят в пепелта;
\par 31 ще оплешивеят съвсем поради тебе, ще се опашат с вретище, и ще плачат за тебе с душевна горест, с горчиво ридание.
\par 32 И в риданието си ще дигнат плач за тебе, и като плачат за тебе ще рекат: Кой е бил като Тир, който загина всред морето?
\par 33 Когато стоките ти излизаха из моретата, ти насищаше много племена; с голямото си богатство и търговия ти обогатяваше земните царе.
\par 34 Сега като си разбит в моретата в дълбочината на водите, търговията ти и всичкото ти множество паднаха всред тебе.
\par 35 Всичките жители на островите се удивиха за тебе, и царете им ужасно се уплашиха; лицата им побледняха от страх.
\par 36 Търговците между племената подсвирнаха поради тебе; ужас си станал, и не ще те има до века.

\chapter{28}

\par 1 Пак Господното слово дойде към мене и рече:
\par 2 Сине човешки, речи на тирския княз: Така казва Господ Иеова: Понеже се е надигнало сърцето ти, и ти си рекъл: Аз съм Бог, седя на Божието седалище всред моретата; а пък ти си човек, а не бог, ако и да си поставил сърцето си като че е Божие сърце;
\par 3 (ето, ти си по-мъдър от Даниила; никаква тайна не се укрива от тебе;
\par 4 с мъдростта и разума си ти си придобил богатство за себе си, и придобил си злато и сребро в съкровищниците си;
\par 5 с голямата си мъдрост ти си придобил богатството си чрез търговията си, и сърцето ти се е надигнало поради богатството ти);
\par 6 затова, така казва Господ Иеова: Понеже си поставил сърцето си като че е Божие сърце,
\par 7 затова, ето, ще докарам против тебе чужденци, страшните между народите; и ще изтеглят сабите си против красивите произведения на твоята мъдрост, и ще помрачат блясъка ти;
\par 8 ще те смъкнат в ямата; и ще умреш, както умират убитите, всред моретата.
\par 9 Тогава ще кажеш ли пред онзи, който те убива: Аз съм бог? Но ти си човек, а не бог, в ръката на онзи, който те убива.
\par 10 Ще умреш от ръката на чужденците както умират необрязаните; защото Аз го изрекох, казва Господ Иеова.
\par 11 При това, Господното слово дойде към мене и рече:
\par 12 Сине човешки, дигни плач за тирския цар, и кажи му: Така казва Господ Иеова: Ти си печат на съвършенство, пълен си с мъдрост и съвършен по хубост.
\par 13 Ти бе в Божията градина, в Едем; ти бе обсипан с всякакви скъпоценни камъни; със сард, топаз, диамант, хрисолит, оникс, яспис, сапфир, антракт, смарагд и със злато; направата на тъпанчетата и на свирките ти е била приготвена за тебе в деня, когато си бил създаден.
\par 14 Ти беше херувим, помазан, за да засеняваш; и Аз те поставих така, щото беше на Божия свят хълм; ти ходеше всред огнени камъни.
\par 15 Ти бе съвършен в постъпките си от деня, когато бе създаден, догдето се намери беззаконие в тебе.
\par 16 От много голямата ти търговия напълниха всичко всред тебе с насилие, и ти съгреши; затова те отхвърлих като скверен от Божия хълм, и те изтребих отсред огнените камъни, херувиме засеняващи!
\par 17 Сърцето ти се надигна поради хубостта ти; ти разврати мъдростта си поради блясъка си; Аз те хвърлих на земята, изложих те пред царете, за да те гледат.
\par 18 Ти омърси светилищата си чрез многото си беззакония, чрез неправедната си търговия; затова, извадих огън изсред тебе, който те изяде, и те обърнах на пепел по земята пред очите на всички, които те гледат.
\par 19 Всички, които те познаваха между племената, се удивиха на тебе; ужас си станал, и не ще те има до века.
\par 20 И Господното слово дойде към мене и рече:
\par 21 Сине човешки, насочи лицето си към Сидон, и пророкувай против него, като речеш:
\par 22 Така казва Господ Иеова: Ето, Аз съм против тебе, Сидоне, и ще си придобия слава всред тебе; и когато извърша съдби в него и се осветя чрез него, тогава ще познаят, че Аз съм Господ.
\par 23 Защото ще изпратя в него мор, и кръв в улиците му; и убитите ще паднат всред него от нож, дошъл в него отвред; и ще познаят, че Аз съм Господ.
\par 24 И между всички, които са около тях, не ще остане за Израилевия дом бодлива къпина или мъчителен трън, каквито им напакостяваха; и ще познаят, че Аз съм Господ Иеова.
\par 25 Така казва Господ Иеова: Когато събера Израилевия дом от племената, между които са разпръснати, и се осветя чрез тях пред очите на народите, тогава ще живеят в своята земя, която дадох на слугата Си Якова.
\par 26 Ще живеят в нея безопасно, да! ще построят къщи и ще насадят лозя; и ще живеят безопасно, когато извърша съдби върху всичките около тях, които са им напакостили; и ще познаят, че Аз съм Господ техният Бог.

\chapter{29}

\par 1 В десетата година, в десетия месец, на дванадесетия ден от месеца, Господното слово дойде към мене и рече:
\par 2 Сине човешки, насочи лицето си против египетския цар Фараона, и пророкувай против него и против целия Египет.
\par 3 Говори и речи: Така казва Господ Иеова: Ето, Аз съм против тебе, Фараоне египетски царю, великото чудовище, което лежиш всред реките си, което си рекло: Реката ми е моя; аз я направих за себе си.
\par 4 Аз ще туря въдици в челюстите ти, и рибите в реките ти ще прилепя за люспите ти; и ще те извлека изсред реките ти с всичките риби в реките ти, прилепнали при люспите ти.
\par 5 И ще те изхвърля в пустинята, тебе и всичките риби в реките ти; ще паднеш на отвореното поле; не ще се събереш нито прибереш; дадох те на земните зверове и на небесните птици за храна.
\par 6 И всичките египетски жители ще познаят, че Аз съм Господ; защото бяха жезъл от тръстика за Израилевия дом.
\par 7 Когато те взеха в ръка, ти се счупи, и промуши рамото на всички тях; и когато се облегнаха на тебе, ти се сломи и разклати кръста на всички тях.
\par 8 Затова, така казва Господ Иеова: Ето, ще докарам меч върху тебе, и ще отсека от тебе и човек и животно.
\par 9 И Египетската земя ще бъде опустошена и пуста; и ще познаят, че Аз съм Господ, защото рече: Реката е моя, и аз я направих.
\par 10 Затова, ето, Аз съм против тебе и против реката ти; и ще обърна Египетската земя на пустота, разорение и опустошение, от Мигдол до Сиина и до границата на Етиопия.
\par 11 Човешка нога не ще да мине през нея, нито нога на животно ще мине през нея, нито ще бъде населена за четиридесет години.
\par 12 И ще обърна Египетската земя на пустиня между опустошените страни, и градовете й ще бъдат пусти за четиридесет години между разорените градове; и ще разсея Египтяните между народите и ще ги разпръсна по страните.
\par 13 Но така казва Господ Иеова: На края на четиридесетте години ще събера Египтяните от племената, между които бяха разпръснати;
\par 14 и ще доведа Египетските пленници, ще ги върна в земята Патрос, в родната земя, и ще бъдат там унижено царство.
\par 15 Ще бъде най-униженото царство от царствата, и няма да се издигне вече над народите; защото ще ги намаля, за да не владеят вече над народите.
\par 16 На Египет не ще уповае вече Израилевият дом, щото да му напомня беззаконието, когато погледнат към тях за помощ; и те ще познаят, че Аз съм Господ Иеова.
\par 17 И в двадесет и седмата година, в първия месец, на първия ден от месеца, Господното слово дойде към мене и рече:
\par 18 Сине човешки, вавилонският цар Навуходоносор накара войската си да извърши една толкова голяма работа против Тир, щото всяка глава стана плешива и всяко рамо ожулено; но пак не получи заплатата за Тир, ни той, нито войската му, за работата, която извърши против него.
\par 19 Затова, така казва Господ Иеова: Ето, Аз давам Египетската земя на Вавилонския цар Навуходоносора; той ще откара многото й население, ще я обере, и ще вземе користи от нея; и това ще бъде заплатата на войската му.
\par 20 Дадох му Египетската земя срещу труда, който положи; понеже се трудиха за Мене, казва Господ Иеова.
\par 21 В оня ден ще направя да израсте рог на Израилевия дом, и ще ти дам да отвориш уста всред тях; и ще познаят, че Аз съм Господ.

\chapter{30}

\par 1 Пак Господното слово дойде към мене и рече:
\par 2 Сине човешки, пророкувай и речи: Така казва Господ Иеова: Лелекайте - Олеле за деня!
\par 3 Защото е близо денят, дори е близо денят на Господа, облачен ден; ще бъде времето на езичниците.
\par 4 Нож ще дойде върху Египет, и голямо измъчване ще има в Етиопия, когато убитите паднат в Египет, и когато откарат многото му население и сринат основите му.
\par 5 Етиопяни, левийци, лидийци и всичките разноплеменни народи, Хув, и жителите на съюзните земи ще паднат заедно с тях от нож.
\par 6 Така казва Господ: Ще паднат и ония, които подпират Египет, и гордата му сила ще се сниши; от Мигдол до Сиина ще паднат в него от нож, казва Господ Иеова.
\par 7 И те ще запустеят между опустошените земи, и градовете му ще бъдат между разорените градове.
\par 8 И ще познаят, че Аз съм Господ, когато туря огън в Египет, и се смажат всички, които му помагат.
\par 9 В оня ден, вестители ще излязат от Мене с кораби, за да стреснат безгрижните етиопяни; и голямо измъчване ще ги нападне както в деня на Египет; защото, ето, иде.
\par 10 Така казва Господ Иеова: При това, Аз ще погубя гяламото египетско население чрез ръката на Вавилонския цар Навуходоносора.
\par 11 Той и людете му с него, страшните между народите, ще бъдат доведени, за да разорят земята; и ще изтеглят мечовете си против Египет, и ще изпълнят земята с убити.
\par 12 Аз ще пресуша реките, ще продам земята в ръцете на зли човеци; и ще запустя земята и всичко, що има в нея чрез ръката на чужденци; Аз Господ го изрекох.
\par 13 Така казва Господ Иеова: Ще погубя и кумирите, и ще махна от Мемфис нищожните идоли; не ще има вече княз от египетската земя и ще туря страх в Египетската земя.
\par 14 Аз ще запустя Патрос, ще запаля огън в Танис, и ще извърша съдби в Но.
\par 15 Ще излея яростта Си върху Египетската крепост Син, и ще изсека голямото население на Но.
\par 16 И ще запаля огън в Египет; Син ще бъде в голямо измъчване. Но ще се съкруши, и Мемфис ще има противници всред бял ден.
\par 17 Младежите на Илиупол и на Пивесет ще паднат от нож; а останалите ще отидат в плен.
\par 18 И в Тафнес денят ще се помрачи, когато строша там хомотите на Египет, и гордата му сила ще престане в него; а него, облак ще го покрие, и дъщерите му ще отидат в плен.
\par 19 Така ще извърша съдби над Египет; и ще познаят, че Аз съм Господ.
\par 20 А в единадесетата година, в първия месец, на седмия ден от месеца, Господното слово дойде към мене и рече:
\par 21 Сине човешки, строших мишцата на египетския цар Фараон; и, ето, тя не е била превързана за церене, или за да я обвият в превръзки, за да й се даде сила да държи нож.
\par 22 Затова, така казва Господ Иеова: Ето, Аз съм против Египетския цар Фараона, и ще строша двете му мишци - и здравата и оная, която вече биде строшена; и ще направя ножът да падне от ръката му.
\par 23 И ще разсея египтяните между народите, и ще ги разпръсна по страните.
\par 24 Но ще укрепя мишците на Вавилонския цар, и ще туря меча Си в ръката му; а мишците на Фараона ще строша, и той ще охка пред него, както охка смъртно ранен човек.
\par 25 Мишците обаче на Вавилонския цар ще засиля; а Фараоновите мишци ще отпаднат; и ще познаят, че Аз съм Господ, когато туря меча Си в ръката на вавилонския цар, и той я простря върху Египетската земя.
\par 26 И ще разсея египтяните между народите и ще ги разпръсна по страните; и те ще познаят, че Аз съм Господ.

\chapter{31}

\par 1 И в единадесетата година, в третия месец, на първия ден от месеца, Господното слово дойде към мене и рече:
\par 2 Сине човешки, кажи на египетския цар Фараона и на множеството му: На кого си се уподобил ти във величието си?
\par 3 Ето, асириецът бе кедър в Ливан, с хубави клонове, с дебела сянка, и с висок ръст; и върхът му бе всред гъсти клончета.
\par 4 Водите го хранеха, бездната го отрастваше с реките си, които течеха около посаждението му; и изпращаше каналите си по всичките дървета на полето.
\par 5 Затова, ръстът му се издигна над всичките дървета на полето, клоновете му се умножиха, и като растеше клончетата му се разпростряха поради изобилните води.
\par 6 Всичките небесни птици правеха гнезда в клончетата му; и всичките полски животни раждаха под клоновете му; а под сянката му живееха всичките големи народи.
\par 7 Така бе красив по големината си и по дължината на клоновете си; защото корените му бяха при много води.
\par 8 Кедрите в Божията градина не можеха да го скрият; елхите не се сравняваха с клоновете му, и яровите не приличаха на клончетата му; никакво дърво в Божията градина не се сравняваше с него по красотата му.
\par 9 Направих го красив с многото му клонове; тъй щото всичките едемски дървета, които бяха в Божията градина, му завиждаха.
\par 10 Затова, така казва Господ Иеова: Понеже ти си се издигнал високо, и понеже си дигнал върха си между гъстите клончета, и сърцето му се надигна поради височината му,
\par 11 затова, ще го предам в ръката на силния от народите, който непременно ще се разправи с него; изпъдих го поради нечестието му.
\par 12 Чужденци, страшните между народите, отсякоха го и го оставиха; клончетата му паднаха по планините и по всичките долини, и клоновете му се строшиха по всичките потоци на земята; и всичките народи на света слязоха от сянката му и го оставиха.
\par 13 На трупа му ще си починат всичките небесни птици, и върху клоновете му ще бъдат всичките животни от полето,
\par 14 за да не се възвиси във височината си никое от дърветата край водите, нито да издигне върха си между гъстите клончета, и за да се не надигат поради височината си, техните великани, да! всички, които се поят с вода; защото те всички са предадени на смърт, подобно на всички други човешки синове, с ония, които слизат в ямата.
\par 15 Така казва Господ Иеова: В деня, когато той слезе в преизподнята причиних жалеене; покрих бездната за него, и направих да престанат реките й, така щото големите води се спряха; и направих да жалее за него Ливан, и всичките дървета на полето повяхнаха за него.
\par 16 Направих народите да потреперят при шума на падането му, когато го свалих в преизподнята с ония, които слизат в ямата; и всичките едемски дървета, отбраните и добрите ливански дървета, всичките пиещи води, се утешиха в най-дълбоките места на света.
\par 17 И те и ония, които бяха негова мишца, които живееха под сянката му всред народите, слязоха в преизподнята подобно на него, при убитите от нож.
\par 18 На кого си се уподобил така по слава и по величие между едемските дървета? При все това, ще бъдеш свален, както всичките други едемски дървета, в най-дълбоките места на света; ще лежиш всред необрязаните, с убитите от нож. Тъй ще стане с Фараона и цялото му множество, казва Господ Иеова.

\chapter{32}

\par 1 И в дванадесетата година, в дванадесетия месец, на първия ден от месеца, Господното слово дойде към мене и рече:
\par 2 Сине човешки, дигни плач за египетския цар Фараон, и кажи му: Уподобил си се на млад лъв между народите, но си като чудовище в моретата; и устремил си се в реките си, и мътиш водите с нозете си, и тъпчеш реките им.
\par 3 Така казва Господ Иеова: Затова, ще простра мрежата Си върху тебе със събрание от много племена, които ще те извлекат в мрежата Ми.
\par 4 И ще те оставя на земята, ще те отхвърля на отвореното поле, ще направя да кацнат на тебе всичките небесни птици, и ще наситя с тебе зверовете на целия свят.
\par 5 Ще туря месата ти на планините, ще напълня долините с купове от твоите убити,
\par 6 и ще напоя с кръвта ти земята гдето плаваш, дори до планините; и реките ще се напълнят с тебе.
\par 7 И когато те угася, ще покрия небето и ще помрача звездите му; ще покрия слънцето с облак, и луната няма да свети със светлината си.
\par 8 Ще помрача над тебе всичките небесни светила, и ще туря тъмнина на твоята земя, казва Господ Иеова.
\par 9 Ще досаждам сърцето на много племена, когато докарам между народите разорените останали от тебе, в страни, които ти не си познал.
\par 10 Да! ще направя да се удивят поради тебе много племена; и царете им ще се ужасят много поради тебе, когато размахат меча Си пред тях, и ще треперят всяка минута, всеки за живота си, в деня на падането ти.
\par 11 Защото така казва Господ Иеова: Мечът на вавилонския цар ще дойде върху тебе.
\par 12 С ножовете на силните ще сваля множеството ти; те всички са страшните между народите; те ще разорят гордостта на Египет, и цялото му множество ще погине.
\par 13 И ще изтребя всичките му животни от при много води; няма вече да ги размъти човешка нога, и копито на животно няма да ги размъти.
\par 14 Тогава ще избистря водите им, и ще направя реките им да текат като масло, казва Господ Иеова.
\par 15 Когато направя Египетската земя разорена и пуста, земя лишена от това, което я изпълваше, когато поразя всички, които живеят в нея, тогава ще познаят, че Аз към Господ.
\par 16 Това е плачът, с който ще се оплакват; дъщерите на народите ще се оплакват с него. За Египет и за цялото му множество ще оплакват с него, казва Господ Иеова.
\par 17 Пак в дванадесетата година, в дванадесетия месец, на петнадесетия ден от месеца, Господното слово дойде към мене и рече:
\par 18 Сине човешки, заридай за множеството на Египет, и яви свалянето им, да! него и дъщерите на знаменитите народи, в най-дълбоките места на света, с ония, които слизат в ямата.
\par 19 От кого си по-хубав? Слез и лежи с необрязаните.
\par 20 Ще паднат всред убитите от нож; ножът се приготви; отвлечете него и всичките му множества.
\par 21 Най-мощните между силните ще му говорят отсред преизподнята, заедно с ония, които му помагаха; слязоха, лежат необрязани, убити от нож.
\par 22 Там е Асурс и цялата му дружина; гробовете му са около него; всички убити паднали от нож,
\par 23 чиито гробове са поставени в дълбочината на ямата, и дружината му около гроба му; всички убити, паднали от нож, тия, които причиняваха ужас в земята на живите.
\par 24 Там е Елам и цялото му множество около гроба му; всички убити, паднали от нож, слезли необрязани в най-дълбоките места на света; те причиняваха трепет в земята на живите, но понесоха срама си както всички други, които слизат в ямата.
\par 25 Всред убитите поставиха легло за него с цялото му множество; гробовете му около него; те всички са необрязани, убити от нож, защото бяха причинявали ужас в земята на живите; но понесоха срама си както всички други, които слизат в ямата; той е поставен всред убитите.
\par 26 Там е Мосох, Тувал, и цялото му множество; гробовете му са около него; всички са необрязани, убити от нож; защото причиняваха ужас в земята на живите.
\par 27 Те няма да лежат със силните паднали измежду необрязаните, които слязоха в преизподнята с бойните си оръжия, и туриха ножовете си под главите си; но техните беззакония ще са върху костите им, защото са причинявали ужас на силните в земята на живите.
\par 28 Но и ти ще бъдеш сломен всред необрязаните, и ще лежиш с убитите от нож.
\par 29 Там е Едом, царете му, и всичките му първенци, които, отсечени всред силата си, са положени между убитите от нож; те ще лежат с необрязаните и с ония, които слизат в ямата.
\par 30 Там са всичките северни князе, и всичките сидонци, които слязоха с убитите; въпреки ужаса, който причиняваха от силата си, те се посрамиха; и лежат необрязани с убитите от нож, и понасят срама си както всички други, които слизат в ямата.
\par 31 Фараон ще ги види, и ще се утеши за цялото си мнозинство, - Фараон и цялата му войска, убити от нож, казва Господ Иеова.
\par 32 Защото Аз нанесох трепет от Мене върху земята на живите; и той ще бъде положен всред необрязаните, с убитите от нож, - Фараон и цялото му множество, казва Господ Иеова.

\chapter{33}

\par 1 И Господното слово дойде към мене и рече:
\par 2 Сине човешки, говори на людете си и кажи им: Когато нанеса меча върху някоя земя, и людете на оная земя вземат някой човек изпомежду си и си го поставят за страж,
\par 3 и той, като види че мечът иде върху земята, затръби и предупреди людете,
\par 4 тогава, ако мечът дойде и постигне някого, който чуе гласа на тръбата, а не се пази, кръвта му ще бъде на главата му.
\par 5 Той е чул гласа на тръбата, а не се е свестил; кръвта му ще бъде върху него; когато, ако беше се свестил, той би избавил живота си:
\par 6 Но ако види стражът, че мечът иде и не затръби, и людете не се свестят, и мечът дойде и постигне някого от тях, той наистина биде постигнат поради беззаконието си; но кръвта му ще изискам от ръката на стража.
\par 7 Така е и с тебе, сине човешки; Аз те поставих страж на Израилевия дом; чуй, прочее, словото из устата Ми, и предупреди ги от Моя страна.
\par 8 Когато казвам на беззаконника: Беззаконнико, непременно ще умреш, а ти не проговориш, за да предупредиш беззаконника да се върне от пътя си, оня беззаконник ще умре за беззаконието си, обаче от твоята ръка ще изискам кръвта му.
\par 9 Но ако предупредиш беззаконника да се върне от пътя си, а не се върне от пътя си, той ще умре за беззаконието си, а ти си избавил душата си.
\par 10 Затова, сине човешки, речи на Израилевия дом: Вие така говорихте, казвайки: Престъпленията ни и греховете ни са върху нас, и ние тлеем поради тях; как тогава ще живеем?
\par 11 Речи им: Заклевам се в живота Си, казва Господ Иеова, не благоволя в смъртта на нечестивия, но да се върне нечестивият от пътя си и да живее. Върнете се, върнете се от лошите си пътища; защо да умрете, доме Израилев?
\par 12 Затова, сине човешки, кажи на людете си: Правдата на праведния няма да го избави в деня, когато престъпи; и нечестивият няма да падне поради нечестието си, също както праведният не ще може да живее поради правдата си в деня, когато съгреши.
\par 13 Когато река на праведния, че непременно ще живее, а той като уповае на правдата си, извърши неправда, то ни едно от неговите праведни дела няма да се спомни; а поради неправдата, която е извършил, той ще умре.
\par 14 И когато кажа на нечестивия: Непременно ще умреш; а той се върне от греха си и постъпи законно и праведно; -
\par 15 ако нечестивият повърне залог, върне грабнатото, ходи в повеленията на живота, и не върши неправда, непременно ще живее; няма да умре;
\par 16 ни един от греховете, които е извършил няма да се помни против него; той е постъпил законно и праведно; непременно ще живее.
\par 17 Но твоите люде казват: Господният път не е прав. Обаче техният път не е прав.
\par 18 Когато праведният се върне от правдата си и извърши неправда, то поради нея ще умре.
\par 19 А когато беззаконникът се върне от беззаконието си и постъпи законно и праведно, той ще живее поради това.
\par 20 Вие обаче казвате: Господният път не е прав. Доме Израилев, ще ви съдя всекиго според постъпките му.
\par 21 В дванадесетата година от плена ни, в десетия месец, на петия ден от месеца, дойде при мене един бежанец от Ерусалим и каза: Градът се превзе.
\par 22 А вечерта, преди да дойде бежанецът, Господната ръка биде върху мене и отваряше устата ми, докле дойде той при мене заранта; и тъй, устата ми се отвориха, и не бях вече ням.
\par 23 И Господното слово дойде към мене и рече:
\par 24 Сине човешки, тия, които живеят в ония опустошени места в Израилевата земя, говорят, казвайки: Авраам бе само един, но пак наследи земята; а ние сме мнозина; нам се даде земята в наследство.
\par 25 Затова, кажи им: Така казва Господ Иеова: Вие ядете месо с кръвта му, подигате очи към идолите си и проливате кръв; и ще владеете ли земята?
\par 26 Вие се облягате на меча си, вършите мерзости, и осквернявате всеки жената на ближния си; и ще владеете ли земята?
\par 27 Кажи им това: Така казва Господ Иеова: Заклевам се в живота Си, ония, които са в опустошените места, непременно ще паднат от нож; и който е на отворено поле ще го предам на зверовете да го изядат; а които са в крепостите и в пещерите ще измрат от мор.
\par 28 И ще обърна земята на пустота и да бъде за удивление, и горделивата й сила ще престане; и Израилевите планини ще запустеят, та да няма кой да минава.
\par 29 Тогава ще познаят, че Аз съм Господ, който обърна земята на пустота и да бъде за удивление, поради всичките мерзости, които сториха.
\par 30 А колкото за тебе, сине човешки, твоите люде приказват за тебе при стените и вратите на къщите, и като говорят един на друг, всеки на брата си, казва: Дойдете сега та чуйте що е словото, което излиза от Господа.
\par 31 Те дохождат при тебе както дохождат людете, та седят пред тебе като Мои люде, и слушат твоите думи, но не ги изпълняват; защото с устата си показват много любов, но сърцето им отива след печалбите им.
\par 32 ето, ти им си като любима песен от човек, който има сладък глас и свири добре; защото слушат думите ти, а не ги изпълняват.
\par 33 А когато настане това, (и ето, иде), тогава ще познаят, че е имало пророк между тях.

\chapter{34}

\par 1 И Господното слово дойде към мене и рече:
\par 2 Сине човешки, пророкувай против Израилевите пастири, пророкувай и речи им: Така казва Господ Иеова на пастирите: Горко на Израилевите пастири, които пасат себе си! Не трябва ли пастирите да пасат стадата?
\par 3 Вие ядете тлъстината, обличате се с вълната, и колите угоените, но не пасете стадата.
\par 4 Не подкрепихте немощната, нито изцелихте болната, не превързахте ранената, не докарахте заблудилата се, нито потърсихте изгубената; но с насилие и строгост властвувахте над тях.
\par 5 Те се разпръснаха понеже нямаше пастир, и, като се разпръснаха, станаха храна на всичките полски зверове.
\par 6 Моите овце се скитаха по високите планини и по всеки висок хълм, дори овците Ми бяха разпръснати по целия свят; и нямаше кой да ги потърси или подири.
\par 7 Затова, слушайте, пастири, Господното слово.
\par 8 Заклевам се в живота Си, казва Господ Иеова, понеже стадото Ми стана корист, и овцете Ми станаха храна на всичките полски зверове, защото нямаше пастир, и пастирите Ми не потърсиха овцете Ми, но пасяха себе си, и не пасяха овцете,
\par 9 затова, слушайте, пастири, Господното слово:
\par 10 Така казва Господ Иеова: Ето, Аз съм против пастирите; и ще изискам овцете Си от ръцете им, и ще ги направя да не пасат вече овцете; пастирите няма вече да пасат себе си, защото ще избавя овцете Си от устата им, та да не им бъдат за храна.
\par 11 Защото така казва Господ Иеова: Ето, Аз, сам Аз, ще потърся овцете Си и ще ги подиря.
\par 12 Както овчарят дири стадото си в деня, когато се намира между разпръснатите Си овце, така и Аз ще подиря овцете си, и ще ги избавя от всичките места, гдето бяха разпръснати в облачния и мрачен ден.
\par 13 Ще ги изведа из племената, и ще ги събера от страните, ще ги доведа в земята им, и ще ги паса на Израилевите хълмове близо до потоците и по всичките населяеми места в тяхната земя.
\par 14 Ще ги паса в добро пасбище, и кошарата им ще бъде на високите Израилеви хълмове; там ще почиват в добра кошара и ще пасат на тлъсто пасбище върху Израилевите хълмове.
\par 15 Сам Аз ще паса овцете Си, и Аз ще ги успокоя, казва Господ Иеова.
\par 16 Ще потърся изгубената, и ще докарам пропъдената, ще превържа ранената, и ще подкрепя немощната; но угоените и яките ще погубя; с правосъдие ще ги паса.
\par 17 А колкото за вас, паство Мое, така говори Господ Иеова: Ето, Аз ще съдя между овца и овца, между овца и овни и козли.
\par 18 Малко ли ви е дето пасете доброто пасбище, та тъпчете с нозете си останалата част от пасбището си? и дето пиете бистра вода, та мътите с нозете си останалата?
\par 19 А овцете Ми пасат утъпканото от вашите нозе, и пият вода, размътена с вашите нозе.
\par 20 Затова, така им казва Господ Иеова: Ето, Аз, сам Аз, ще съдя между угоена овца и мършава овца.
\par 21 Понеже тласкате със страната си и с рамената си, и бодете с рогата си всичките болни, догде ги разпръснете далеч,
\par 22 затова, Аз ще избавя овцете Си, та не ще бъдат вече за корист; и ще съдя между овца и овца.
\par 23 И ще поставя над тях един пастир, слугата Си Давида, който ще ги пасе; той ще ги пасе, и той ще им бъде пастир.
\par 24 Аз Господ ще им бъда Бог, и слугата Ми Давид княз между тях; Аз Господ изговорих това.
\par 25 И като направя с тях завет на мир, ще премахна лютите зверове от земята, така щото те ще живеят безопасно в пустинята, и ще спят в горите.
\par 26 Ще направя тях и местата около хълма Си за благословение; и ще изпращам дъжда на времето му; дъжд на благословение ще бъде.
\par 27 Полските дървета ще дават плода си, и земята ще даде произведението си; и те ще бъдат в безопасност в земята си, и ще познаят, че Аз съм Господ, когато строша оковите на хомота им, и ги освободя от ръката на ония, които са ги поробили.
\par 28 Те не ще бъдат вече корист на народите, и земните зверове не ще ги изпояждат; но ще живеят в безопасност, и не ще има кой да ги плаши.
\par 29 И ще им направя прочуто по плодородие садение; и те няма вече да гинат от глад в земята, нито ще носят вече хулене на народите.
\par 30 И ще познаят, че Аз Господ техният Бог съм с тях, и че те, Израилевия дом, са мои люде, казва Господ Иеова.
\par 31 И вие човеци, сте Мои овце, овцете на пасбището Ми, и Аз съм ваш Бог, казва Господ Иеова.

\chapter{35}

\par 1 При това, Господното слово дойде към мене и рече:
\par 2 Сине човешки, насочи лицето си против хълма Сиир, и пророкувай против него, като му речеш:
\par 3 Така казва Господ Иеова: Ето, хълме Сиир, Аз съм против тебе; ще простра ръката Си против тебе, и ще те обърна в опустошение и ще бъдеш за учудване.
\par 4 Ще разоря градовете ти, и ти ще запустееш; и ще познаеш, че Аз съм Господ.
\par 5 Понеже си хранил непрекъсната омраза, и си предал израилтяните в силата на меча във времето на бедствието им, когато беззаконието им е стигнало до края си;
\par 6 затова, заклевам се в живота Си, казва Господ Иеова, ще те предам на кръв, и кръв ще те преследва; понеже не си намразил кръвта, то кръв ще те преследва.
\par 7 Така ще направя хълма Сиир да бъде за учудване и да запустее; и ще изтребя от него и оня, който минава, и оня, който се връща.
\par 8 Ще напълня планините му с убитите му; по хълмовете ти, по долините ти, и по всичките ти реки ще паднат убитите от нож.
\par 9 Ще обърна на вечно опустошение, и градовете ти няма да се населят; и ще познаете, че Аз съм Господ.
\par 10 Понеже си рекъл: Тия два народа и тия две страни ще бъдат мои, и ние ще ги владеем, ако и да е бил Господ там,
\par 11 затова, заклевам се в живота Си, казва Господ Иеова, ще постъпя според гнева и според завистта, която си показал поради омразата си към тях; и ще им се бъда познат, когато те съдя.
\par 12 И ти ще познаеш, че Аз Господ чух всичките хули, които си произнесъл против Израилевите планини, като си рекъл: Те запустяха; нам са дадени за храна.
\par 13 И с устата си говорихте високомерно против Мене, и казахте много думи против Мене; аз чух.
\par 14 Така казва Господ Иеова: Когато се весели целият свят, Аз ще те направя пуст.
\par 15 Както си се развеселил за дето запустя наследството на Израилевия дом, така направя на тебе; ще запустееш, хълме Сиир, и целият Едом, да! целият; и ще познаят, че Аз съм Господ.

\chapter{36}

\par 1 И ти, сине човешки, пророкувай към Израилевите планини, като речеш: Израилеви планини, слушайте Господното слово.
\par 2 Така казва Господ Иеова: Понеже неприятелят рече против вас: О хохо! и - Древните височини станаха наше владение,
\par 3 затова, като пророкуваш речи: Така казва Господ Иеова: Понеже ви запустиха и погълнаха отвред, за да станете владение на другите народи, и станахте предмет на говоренето на устни, и на зли отзиви на людете,
\par 4 затова, Израилеви планини, слушайте словото на Господа Иеова. Така казва Господ Иеова на планините и на хълмовете, на потоците и на долините, на запустелите пустоти и на напуснатите градове, които станаха корист и са за присмех на другите народи около тях, -
\par 5 затова, казва Господ Иеова: Непременно в пламенната Си ревнивост говорих против другите народи, и против целия Едом, които с всесърдечна радост и с душевно презрение направиха Моята земя свое владение, за да я изхвърлят и разграбят;
\par 6 затова, пророкувай за Израилевата земя, и кажи на планините и на хълмовете, на потоците и на долините: Така казва Господ Иеова: Ето, Аз говорих в ревността Си и в яростта Си, понеже понесох укор от народите;
\par 7 затова, така казва Господ Иеова: Аз се заклех, че народите, които са около вас, непременно ще носят срама си.
\par 8 А вие, Израилеви планини, ще изкарате клоновете си; и ще давате плода си на людете Ми Израиля; защото скоро ще дойдат.
\par 9 Защото, ето, Аз съм за вас, и ще се обърна към вас, и вие ще бъдете обработвани и засявани.
\par 10 И ще заселя върху вас много човеци, и целият Израилев дом, да! целия; и градовете ще се населят, и запустелите места ще се съградят.
\par 11 И ще заселя върху вас много човеци и животни, които ще нараснат и се наплодят; и ще ви населя както бяхте по-напред, и ще ви сторя по-голямо добро отколкото в началото; и ще познаете, че Аз съм Господ.
\par 12 Да! Ще направя да ходят по вас човеци, людете Ми Израил; те ще ви владеят, и вие ще им бъдете наследство, и за напред няма да ги обезчадите.
\par 13 Така казва Господ Иеова: Понеже ви думат: Ти си земя, която поглъщаш човеци и обезчадваш людете си,
\par 14 затова, няма вече да поглъщаш човеци, нито да обезчадваш вече народа си, казва Господ Иеова;
\par 15 и няма вече да ти допусна да понесеш укор от народите, нито да носиш поругание от племената, и няма вече да направиш народа си да се препъва, казва Господ Иеова.
\par 16 При това, Господното слово дойде към мене и рече:
\par 17 Сине човешки, когато Израилевият дом живееха в земята си, те я оскверниха с постъпките си и с делата си; техните постъпки бяха пред мене отвратителни като нечистотата на отлъчена жена.
\par 18 Затова, излях яростта Си върху тях поради кръвта, която бяха излели на земята, и поради идолите, с които я бяха омърсили;
\par 19 разсях ги между народите, и те бидоха разпръснати по страните; според постъпките им и според делата им ги съдих.
\par 20 И когато влязоха между народите гдето отидоха, омърсиха Моето свето име, тъй щото се говореше за тях: Тия са людете на Иеова, и из земята Му излязоха!
\par 21 Смилих се обаче за светото Си име, което Израилевият дом бяха омърсили между народите, при които отидоха.
\par 22 Затова, речи на Израилевия дом: Така казва Господ Иеова: Аз не правя това заради вас, доме Израилев, но заради Моето свето име, което омърсихте между народите, при които отидохте.
\par 23 Аз ще осветя великото Си име, което е било омърсено между народите, което вие омърсихте между тях; и народите ще познаят, че Аз съм Господ, казва Иеова, когато се осветя у вас пред очите им.
\par 24 Защото ще ви взема изсред народите, и ще ви събера от всичките страни, и ще ви доведа в земята ви.
\par 25 Тогава ще поръся върху вас чиста вода, и ще се очистите; от всичките ви нечистоти и от всичките ви идоли ще ви очистя.
\par 26 Ще ви дам и ново сърце, и нов дух ще вложа вътре във вас, и, като отнема каменното сърце от плътта ви, ще ви дам меко сърце.
\par 27 И ще вложа Духа Си вътре във вас, и ще ви направя да ходите в повеленията Ми, да пазите съдбите Ми, и да ги извършвате.
\par 28 Ще живеете в земята, която дадох на бащите ви; и вие ще ми бъдете люде, и Аз ще бъда ваш Бог.
\par 29 Ще ви спася от всичките ви нечистоти; и като призова живото, ще го умножа, и не ще вече да ви докарам глад.
\par 30 Ще умножа плода на дървото и рожбите на полето, за да не ви се присмиват вече народите, за гдето гладувате.
\par 31 Тогава, като си спомните нечестивите си постъпки и недобрите си дела, ще се отвратите сами от себе си пред очите си поради беззаконията си и поради мерзостите си.
\par 32 Не заради вас правя Аз това, казва Господ Иеова, нека ви бъде известно. Засрамете се и се смутете поради постъпките си, доме Израилев!
\par 33 Така казва Господ Иеова: В деня, когато ви очистя от всичките ви беззакония, ще направя и да се населят градовете, и запустелите места ще се съградят.
\par 34 Опустошената земя ще се обработи, макар че е била пуста пред очите на всекиго, който минаваше.
\par 35 И ще казват: Тая земя, която бе запустяла, стана като Едемската градина, и запустелите, опустошените, и разорените градове се укрепиха и населиха.
\par 36 Тогава народите, останали около вас ще познаят, че Аз Господ съградих разореното и насадих запустялото. Аз Господ изговорих това, и ще го извърша.
\par 37 Така казва Господ Иеова: При това, Израилевият дом ще Ме потърси, за да им го сторя. Ще ги умножа с човеци като стадо;
\par 38 като стадото за жертва, като стадото, което пълни Ерусалим във време на определените празници, така човешки стада ще пълнят запустелите градове; и ще познаят, че Аз съм Господ.

\chapter{37}

\par 1 Господната ръка биде върху мене, та ме изведе чрез Господния Дух и ме постави всред поле, което бе пълно с кости.
\par 2 И преведе ме край тях наоколо; и, ето, имаше твърде много по отвореното поле; и, ето, бяха твърде сухи.
\par 3 И рече ми: Сине човешки, могат ли да оживеят тия кости? И отговорих: Господи Иеова, Ти знаеш.
\par 4 Пак ми рече: Пророкувай над тия кости и рече им: Сухи кости, слушайте Господното слово.
\par 5 Така казва Господ Иеова на тия кости: Ето, ще направя да влезе във вас дух та ще оживеете;
\par 6 ще туря и жили върху вас, ще ви облека с меса и ще ви покрия с кожа, и като туря дух у вас, ще оживеете; и ще познаете, че Аз съм Господ.
\par 7 И тъй, пророкувах както ми бе заповядано; и като пророкувах, започна да гърми, и ето трус, и костите се събираха, кост с костта си.
\par 8 И като погледнах, ето, жили и меса израснаха по тях, и кожа ги покри отгоре; дух, обаче, нямаше в тях.
\par 9 Тогава ми рече: Пророкувай за духа; пророкувай, сине човешки, и речи на духа: Така казва Господ Иеова: Дойди, духо, от четирите ветрища и духни върху тия убити, за да оживеят.
\par 10 Пророкувах, прочее, както ми заповяда; и духът влезе в тях, и те оживяха и изправиха се на нозете си, една твърде голяма войска.
\par 11 Тогава ми рече: Сине човешки, тия кости са целият Израилев дом. Ето, те казват: Костите ни изсъхнаха, и надеждата ни се изгуби; ние сме загинали.
\par 12 Затова, пророкувай и речи им: Така казва Господ Иеова: Ето, люде Мои, Аз ще отворя гробовете ви, и като ви изведа из гробовете ви, ще ви заведа в Израилевата земя.
\par 13 И ще познаете, люде Мои, че Аз съм Господ, когато отворя гробовете ви и ви изведа из гробовете ви.
\par 14 И като туря духа Си във вас, ще оживеете; и ще ви поставя във вашата си земя; и ще познаете, че Аз Господ изговорих това и го извърших, казва Господ.
\par 15 Пак дойде Господното слово към мене и рече:
\par 16 Ти, сине човешки, вземи си един жезъл та напиши на него: За Юда и за другарите му израилтяните; тогава вземи друг жезъл, та напиши на него: За Иосифа, жезъла на Ефрема и другарите му целия Израилев дом.
\par 17 И съедини си ги един с друг в един жезъл, за да станат един в ръката ти.
\par 18 И когато людете ти продумат и ти рекат: Няма ли да ни обясниш що искаш да кажеш с това?
\par 19 Речи им: Така казва Господ Иеова: Ето, Аз ще взема Иосифовия жезъл, който е в ръката на Ефрема и на другарите му Израилевите племена, та ще ги туря към него, към Юдовия жезъл, и ще ги направя един жезъл; те ще бъдат един в ръката Ми.
\par 20 И жезлите, на които си написал, нека бъдат в ръката ти пред очите им.
\par 21 И речи им: Така казва Господ Иеова: Ето, Аз ще взема израилтяните изсред народите гдето са отишли, и като ги събера от всякъде, ще ги доведа в земята им;
\par 22 и ще ги направя един народ в земята, върху Израилевите планини; един цар ще царува над всички тях; и не ще бъдат вече два народа, нито ще бъдат занапред разделени в две царства.
\par 23 Те няма вече да се оскверняват с идолите си, нито с мерзостите си, нито с кое да било от престъпленията си; но ще ги избавя от всичките отстъпления, с които са съгрешили, и ще ги очистя; така те ще бъдат Мои люде, и Аз ще бъда техен Бог.
\par 24 И слугата Ми Давид ще бъде цар над тях; над всички тях ще има един пастир; и те ще ходят в съдбите Ми, и ще пазят повеленията Ми и ще ги извършват.
\par 25 Тоже ще живеят в земята, която дадох на слугата Си Якова, гдето живееха бащите ви; в нея ще живеят те, чадата им, и внуците им до века; и слугата Ми Давид ще им бъде княз до века.
\par 26 При това, ще направя с тях завет на мир, който ще бъде вечен завет с тях; и като ги настаня, ще ги умножа, и ще положа светилището Си всред тях до века.
\par 27 И скинията Ми ще бъде всред тях; и Аз ще бъда техен Бог, и те ще бъдат Мои люде.
\par 28 Тогава народите ще познаят, че Аз Господ освещавам Израиля, когато светилището Ми бъде всред тях до века.

\chapter{38}

\par 1 И Господното слово дойде към мене и рече:
\par 2 Сине човешки, насочи лицето си към Гога, в земята на Магога, княза на Рос, Мосох и Тувал, и пророкувай против него, като речеш:
\par 3 Така казва Господ Иеова: Ето, Аз съм против тебе, Гоге, княже на Рос, Мосох и Тувал.
\par 4 Ще те обърна, ще туря кука на челюстите ти, и ще те извадя, с цялата ти войска, коне и конници, всички напълно въоръжени, едно голямо пълчище с щитове и щитчета, всички употребяващи ножове, -
\par 5 персийци, етиопяни, и ливийци с тях, всички с щитове и шлемове, -
\par 6 Гомера и всичките му пълчища, дома на Тогарма, от най-далечните страни на север, и всичките му пълчища, и много народи с тебе.
\par 7 Приготви се, да! приготви себе си, ти и цялото пълчище, което се е събрало при тебе, и стани им военачалник.
\par 8 След много дни ще бъдеш наказан; в послешните години ще дойдеш в земята, която е била отървана от ножа, и е била събрана от много племена, върху Израилевите планини, които са били непрекъснато пусти; но Израил биде пренесен изсред племената, и те всички ще живеят в нея в безопасност.
\par 9 И ти като възлезеш, ще дойдеш като вихрушка; ще бъдеш като облак, за да покриеш земята, ти и всичките твои пълчища, и много племена с тебе.
\par 10 Така казва Господ Иеова: В оня ден ще дойдат мисли в ума ти, и ще скроиш лоши намерения, като речеш:
\par 11 Ще вляза в земята на неукрепени села! ще отида при ония, които са в спокойствие и живеят безгрижно, всички които живеят в неукрепени места, без лостове и порти,
\par 12 за да ги обереш и да вземеш корист, за да обърнеш ръката си против запустелите места сега населени, и против събраните изсред народите люде, които са придобили добитък и имот, и живеят в най-отбраната страна на света.
\par 13 Шева, Дедан, и търговците на Тарсис с всичките му млади лъвове ще ти рекат: Да обереш ли си дошъл? Да вземеш корист ли си събрал множеството си? Да грабнеш сребро и злато ли, да вземеш добитък и имот ли, да направиш голям обир ли?
\par 14 Затова, сине човешки, пророкувай като речеш на Гога: Така казва Господ Иеова: В оня ден, когато людете Ми Израил ще живеят в безопасност, ти няма ли да го знаеш?
\par 15 Ще дойдеш от мястото си, от най-далечните страни на север, ти и много племена с тебе, всички яздещи на коне, голямо множество и силна войска;
\par 16 и ще възлезеш против людете Ми Израиля като облак, покриващ земята. Това ще бъде в послешните дни; и Аз ще те доведа против земята, за да Ме познаят народите, когато се осветя у тебе, Гоге, пред очите им.
\par 17 Така казва Господ Иеова: Ти ли си оня, за когото говорих в древно време чрез слугите Си Израилевите пророци, които в ония дни пророкуваха през много години, че щях да те доведа против тях?
\par 18 Но в оня ден, в деня когато Бог дойде против Израилевата земя, яростта Ми ще възлезе в ноздрите Ми, казва Господ Иеова.
\par 19 Защото в ревнивостта Си и в пламенния Си гняв казах: Непременно в оня ден ще има голям трепет в Израилевата земя,
\par 20 така щото морските риби и небесните птици, полските зверове и всички гадини, които пълзят по света, и всичките човеци, които са по лицето на света, ще се разтреперят от присъствието Ми; и планините ще се сринат, стръмните височини ще паднат, и всяка стена ще се събори до земята.
\par 21 И ще призова нож против него по всичките Си планини, казва Господ Иеова; ножът на всеки човек ще бъде против брата му.
\par 22 Аз ще се съдя с него чрез мор, и чрез кръв; и ще наваля върху него, върху пълчищата му, и върху многото племена, които са с него, пороен дъжд и градушка от големи камъни, огън и сяра,
\par 23 Аз ще възвелича и осветя, и ще стана познат пред очите на много народи; и те ще познаят, че Аз съм Господ.

\chapter{39}

\par 1 И ти, сине човешки, пророкувай против Гога като речеш: Така казва Господ Иеова: Ето, Аз съм против тебе, Гоге, княже на Рос, Мосох, и Тувал.
\par 2 Ще те обърна и примамя, и, като те възведа от най-далечните страни на север, ще те доведа върху Израилевите планини;
\par 3 и ще избия лъка ти от лявата ти ръка, и ще направя да паднат стрелите от дясната ти ръка.
\par 4 Ще паднеш върху Израилевите планини, ти, и всичките ти пълчища, и племената, които са с тебе; и ще те предам да бъдеш изяден от хищни птици от всякакъв вид и от полските зверове.
\par 5 Ще паднеш на отвореното поле; защото Аз го изговорих, казва Господ Иеова.
\par 6 И ще изпратя огън върху Магога и върху ония, които живеят безгрижно в островите; и те ще познаят, че Аз съм Господ.
\par 7 Ще направя светото Си име познато всред людете Си Израиля, и не ще оставя да се омърси вече светото Ми име; и народите ще познаят, че Аз съм Господ, Светият в Израил.
\par 8 Ето, това иде, и ще стане, казва Господ Иеова; тоя е денят, за който говорих.
\par 9 И ония, които живеят в Израилевите градове, като излизат, ще кладат огън с оръжията, които ще им служат за гориво - щитовете и щитчетата, лъковете и стрелите, сулиците и копията; ще кладат огън с тях седем години;
\par 10 така щото няма да носят дърва от полето, нито ще отсекат от гората, защото ще кладат огън с оръжията; и ще оберат ония, които са ги обрали, и ще оголят ония, които са ги оголили, казва Господ Иеова.
\par 11 В оня ден ще дам на Гога място за погребване в Израил, долината на ония, които преминават отпред морето; и то ще затваря пътя на преминаващите; и там ще заровят Гога и цялото му множество; и ще нарекат мястото Долината на Амон-гога.
\par 12 Седем месеца ще ги рови Израилевият дом, за да очистят земята;
\par 13 да! всичките люде на Израилевата земя ще ги ровят; и бележит ще им бъде денят, в който аз ще се прославя, казва Господ Иеова.
\par 14 И ще определят мъже, които, като обикалят непрестанно земята, ще заравят с помощта на преминаващите останалите по лицето на земята, за да я очистят; докле се свършат седем месеца ще дирят падналите.
\par 15 И определените предирвачи, като обикалят земята, щом някой от тях види човешка кост, ще изправи знак при нея, докле погребвачите я заровят в долината на Амон-гога.
\par 16 А името и на града ще бъде Амона. Така ще очистят земята.
\par 17 А ти, сине човешки, така казва Господ Иеова: Говори на всичките видове птици и не всеки полски звяр, като кажеш: Съберете се та дойдете; натрупайте се от всякъде на жертвата, която жертвувам за вас, голяма жертва върху Израилевите планини, за да ядете месо и да пиете кръв.
\par 18 Ще ядете месата на юнаците, и ще пиете кръвта на земните князе, и на овни, на агнета, на козли, и на телци, всички васански угоени.
\par 19 Ще ядете тлъстина до насита, и ще пиете кръв до опиване, от жертвата, която пожертвувах за вас;
\par 20 И ще се наситите на трапезата Ми с коне и ездачи, с юнаци и с всякакви военни мъже, казва Господ Иеова.
\par 21 И Аз ще поставя славата Си между народите; и всичките народи ще видят съдбата, която извърших, и ръката Ми, която положих върху тях.
\par 22 Така Израилевият дом ще познае, че Аз съм Господ техният Бог от днес нататък.
\par 23 И народите ще познаят, че Израилевият дом бе пленен поради беззаконието си. Понеже станаха непокорни на Мене, затова скрих лицето Си от тях и ги предадох в ръката на неприятелите им; и те всички паднаха от нож.
\par 24 Сторих им според нечистотата им и според престъпленията им, и скрих лицето Си от тях.
\par 25 Затова, така казва Господ Иеова: Сега ще върна Якова от плен, ще се смиля за целия Израилев дом, и ще бъда ревнив за светото Си име.
\par 26 И те ще носят срама си и всичките престъпления, чрез които станаха непокорни на Мене, когато живеят безопасно в земята си, без да има кой да ги плаши.
\par 27 Когато ги доведа пак от племената, и ги събера от страните на неприятелите им, тогава ще се осветя у тях пред очите на много народи;
\par 28 и те ще познаят, че Аз съм Господ техният Бог, понеже ги направих да бъдат закарани в плен между народите, а после ги събрах в земята им; и няма да оставя вече там никого от тях.
\par 29 И няма вече да скрия лицето Си от тях; защото излях Духа си върху Израилевия дом, казва Господ Иеова.

\chapter{40}

\par 1 В двадесет и петата година от плена ни, в началото на годината, на десетия ден от месеца, в четиринадесетата година подир превземането на града, в същия ден Господната ръка биде върху мене и ме заведе там, -
\par 2 чрез Божии видения ме заведе в Израилевата земя, та ме постави върху една твърде висока планина, на която имаше към юг нещо като здание подобно на град.
\par 3 И като ме заведе там, ето, човек, чийто изглед бе като изглед на мед, и който държеше в ръката си ленена връв и мярка от тръстика, стоеше в портата.
\par 4 И човекът ми рече: Сине човешки, погледни с очите си, чуй с ушите си, и приложи сърцето си върху всичко, което ще ти покажа; защото ти биде въведен тука с цел да ти покажа това. Всичко, що видиш, изяви го на Израилевия дом.
\par 5 И ето извън дома една стена околовръст, и в ръката на човека мярка от тръстика дълга шест лакти, всеки лакът дълъг лакът и длан; и като измери широчината на зданието, тя беше една тръстика, и височината една тръстика.
\par 6 Тогава дойде при портата, която гледаше към изток, и се изкачи по стъпалата й; и като измери прага на портата, широчината му беше една тръстика, и широчината на другия праг една тръстика.
\par 7 И всяка стражарска стая бе една тръстика дълга и една тръстика широка; и разстоянието между стаите бе пет лакти; а прагът на портата, при предверието на портата към дома, бе една тръстика.
\par 8 Измери тоже предверието на портата към дома, една тръстика.
\par 9 Тогава като измери предверието на портата, то беше осем лакти, а стълбовете им два лакътя; и предверието на портата беше откъм дома.
\par 10 И стражарските стаи на източната порта бяха три отсам и три оттам, и трите на една мярка; и стълбовете имаха една мярка отсам и оттам.
\par 11 И като измери широчината на входа на портата, тя беше десет лакти, а дължината на портата тринадесет лакти.
\par 12 А пред стаите имаше един лакът разстояние отсам и един лакът разстояние оттам; и стаите бяха шест лакти отсам и шест лакти оттам.
\par 13 После измери портата от покрива на едната стражарска стая до покрива на другата, и широчината бе двадесет и пет лакти, врата срещу врата.
\par 14 И намери стълбовете шестдесет лакти; и предверието стигаше до стълбовете околовръст портата.
\par 15 И от лицето на портата, при входа, до лицето на предверието на вътрешната порта, имаше петдесет лакти.
\par 16 И на стражарските стаи имаше затворени прозорци, също и на стълбовете им извътре портата околовръст, така и на сводовете; а имаше прозорци извътре околовръст; а върху всеки стълб имаше палми.
\par 17 Тогава ме заведе във външния двор; и ето стаи и под направени около двора; имаше тридесет стаи върху пода.
\par 18 И подът, който бе от страните на портата, съответстващ на дължината на портите, беше долният под.
\par 19 Тогава измери широчината от лицето на долната порта до лицето на вътрешния двор извън, и тя беше сто лакти и откъм изток и откъм север.
\par 20 И измери дължината и широчината на портата на външния двор, която гледаше към север.
\par 21 И стражарските стаи бяха три отсам и три оттам; а стълбовете й и сводовете й бяха според мярката на първата порта; дължината й беше петдесет лакти, и широчината й двадесет и пет лакти.
\par 22 И прозорците й, и сводовете й, и палмите й бяха според мярката на портата, която гледа към изток; и изкачваха се към нея по седем стъпала; и сводовете й бяха пред нея.
\par 23 И портата на вътрешния двор бе срещу портата, която бе към север и към изток; и като измери от порта до порта, намери сто лакти.
\par 24 И заведе ме към юг, и ето порта, която гледаше към юг; и като измери стълбовете й и сводовете й, те имаха същите мерки.
\par 25 И имаше прозорци на нея и на сводовете й околовръст, подобни на ония прозорци; дължината им беше петдесет лакти, и широчината двадесет и пет лакти.
\par 26 И изкачваха се към нея по седем стъпала; и сводовете й бяха пред нея; и на стълбовете й имаше палми, една отсам и една оттам.
\par 27 И във вътрешния двор имаше порта към юг; и като измери от порта до порта към юг, намери сто лакти.
\par 28 Тогава ме заведе във вътрешния двор през южната порта; и като измери южната порта, тя имаше същите мерки,
\par 29 и стражарските й стаи, и стълбовете й, и сводовете й имаха същите мерки; и на нея и на сводовете й имаше прозорци околовръст; дължината й беше петдесет лакти, а широчината й двадесет и пет лакти.
\par 30 И сводовете околовръст бяха двадесет и пет лакти дълги, и пет лакти широки.
\par 31 И сводовете й бяха към външния двор; и имаше палми по стълбовете й; и изкачваха се към нея по осем стъпала.
\par 32 И заведе ме във вътрешния двор към изток; и като измери портата тя имаше същите мерки;
\par 33 и стражарските й стаи, стълбовете й, и сводовете й имаха същите мерки; и на нея и на сводовете й имаше прозорци околовръст; а тя беше петдесет лакти дълга и двадесет и пет лакти широка.
\par 34 И сводовете й бяха към външния двор; и по стълбовете й имаше палми отсам и оттам; и изкачваха се към нея по осем стъпала.
\par 35 И заведе ме при северната порта; и като я измери, тя имаше същите мерки,
\par 36 както и стражарските й стаи, стълбовете й, и сводовете й; и на нея имаше прозорци околовръст; а тя беше петдесет лакти дълга и двадесет и пет лакти широка.
\par 37 И стълбовете й бяха към външния двор; и по стълбовете й имаше палми отсам и оттам; и изкачваха се към нея по осем стъпала.
\par 38 И имаше една стая, чиято врата беше при стълбовете на портата, гдето миеха всеизгарянето.
\par 39 И в предверието на портата имаше две, трапези отсам и две трапези оттам, на които да колят всеизгарянето, приноса за грях, и приноса за престъпление.
\par 40 И на външната страна, при стъпалата към входа на северната порта, имаше две трапези; и на другата страна, която принадлежеше към предверието на портата, имаше две трапези.
\par 41 Четири трапези имаше отсам и четири трапези оттам при страните на портата; всичко осем трапези, на които колеха жертвите.
\par 42 И четирите трапези за всеизгарянето бяха от дялан камък, един лакът и половина дълги, един лакът и половина широки, и един лакът високи; и на тях туряха оръдията, с които колеха всеизгарянето и жертвата.
\par 43 И извътре имаше полица, една длан широка, прикована околовръст; а на трапезите туряха месото на приносите.
\par 44 И извън вътрешната порта бяха стаите на певците, във вътрешния двор, който бе на страните на северната порта; и лицата им бяха към юг; а една от тях, на страната на източната порта, гледаше към север.
\par 45 И рече ми: Тая стая, която гледа към юг, е за свещениците, които се грижат за дома;
\par 46 а стаята, която гледа към север, е за свещениците, които прислужват около олтара; те са потомци на Садока, които измежду Левиевите потомци се приближават при Господа да Му служат.
\par 47 И като измери двора, дължината му беше сто лакти, и широчината му сто лакти; беше квадратен, и олтарът беше пред дома.
\par 48 Тогава ме заведе в предхрамието на дома; и като измери всеки стълб на предхрамието, те бяха пет лакти отсам и пет лакти оттам; а широчината на портата беше три лакти отсам и три лакти оттам.
\par 49 Дължината на предхрамието бе двадесет лакти, и широчината му единадесет лакти; и изкачваха се към него по десет стъпала; и при стълбовете имаше други стълбове, един отсам и един оттам.

\chapter{41}

\par 1 После ме заведе в храма; и като измери стълбовете, те имаха шест лакти широчина отсам и шест лакти широчина оттам, според широчината на скинията.
\par 2 И широчината на входа бе десет лакти; и страните на входа бяха пет лакти отсам и пет лакти оттам; и като измери дължината на храма, беше четиридесет лакти, а широчината му двадесет лакти.
\par 3 Тогава влезе по-навътре; и като измери всеки стълб на входа, имаха два лакътя, и входът шест лакти, и широчината на входа седем лакти.
\par 4 И като измери дължината му, беше двадесет лакти, и широчината двадесет лакти, според широчината на храма. И рече ми: Това е пресветото място.
\par 5 Тогава измери стената на дома; тя имаше шест лакти; а широчината на страничните стаи, които бяха около дома на всяка страна, беше четири лакти.
\par 6 И страничните стаи бяха на три етажа, стая върху стая, и тридесет на ред; и влизаха в стената, която принадлежеше на дома за страничните стаи околовръст, за да се държат здраво, без да се държат за стената на дома.
\par 7 И страничните стаи се разширяваха; и имаше вита стълба, която водеше в страничните стаи; защото витата стълба на дома водеше нагоре околовръст на дома; затова, домът ставаше по-широк нагоре; и така се изкачваха от долния етаж до най-горния през средния.
\par 8 И видях, че домът беше висок от всяка страна; основите на страничните стаи бяха една цяла тръстика от шест големи лакти.
\par 9 Външната стена на страничните стаи беше пет лакти широка; и оставеното празно място бе за страничните стаи, които принадлежаха на дома.
\par 10 И между стаите имаше двадесет лакти разстояние около дома на всяка страна.
\par 11 И вратите на страничните стаи бяха към оставеното място, една врата към север и една врата към юг; и широчината на оставеното място бе пет лакти околовръст.
\par 12 А зданието, което бе пред отделеното място към западната страна, бе седемдесет лакти широко; и стената на зданието бе пет лакти дебела околовръст, а дължината му деветдесет лакти.
\par 13 И тъй, като измери дома, беше сто лакти дълъг; и отделеното място, зданието, и стените му сто лакти дълги.
\par 14 Тоже и широчината на лицето на дома и на отделеното място към изток беше сто лакти.
\par 15 И като измери дължината на зданието, което беше в лицето на отделеното място зад него, и галериите му отсам и оттам, те имаха сто лакти; измери и вътрешния храм, предверията на двора,
\par 16 праговете, затворените прозорци, и галериите наоколо в трите им етажа, срещу прага, облечени с дърво околовръст от земята до прозорците, (а прозорците бяха покрити),
\par 17 до над вратата, до вътрешния дом, и извън, и през цялата стена околовръст извътре и извън; всичко бе според мерките.
\par 18 И то бе изработено с херувими и с палми, така щото имаше палма между херувим и херувим. Всеки херувим имаше две лица,
\par 19 тъй щото имаше човешко лице към палмата отсам, а лице на млад лъв към палмата оттам; така бе изработено по целия дом околовръст.
\par 20 От пода до над вратата бяха изработени херувими и палми; такава беше стената на храма.
\par 21 Колкото за храма, стълбовете му бяха квадратни, а колкото за лицето на светилището, изгледът му беше като изгледа на храма.
\par 22 Олтарът бе дървен, три лакти висок, и два лакътя дълъг; и ъглите му, подножието му, и страните му бяха дървени. И той ми рече: Това е трапезата, която стои пред Господа.
\par 23 А храмът и светилището имаха две врати.
\par 24 И вратите имаха по две крила, две движещи се крила, две крила за едната врата, и две крила за другата.
\par 25 И по тях, по вратите на храма, бяха изработени херувими и палми, както бяха работени по стените; и имаше дебели дъски по лицето на предверието извън.
\par 26 И имаше затворени прозорци и палми отсам и оттам от страните на предверието, и по страничните стаи на дома, и по дебелите дъски.

\chapter{42}

\par 1 Тогава ме изведе във външния двор при пътя към север: и заведе ме в стаята, която бе срещу отделеното място, и срещу зданието към север.
\par 2 В лицето му, което беше сто лакти дълго, имаше северната врата; а широчината му бе петдесет лакти.
\par 3 Срещу двадесетте лакти, които принадлежаха на вътрешния двор, и срещу пода, който принадлежеше на външния двор, имаше галерия срещу галерия на третия етаж.
\par 4 И пред стаите имаше коридор десет лакти широк навътре - път сто лакти дълъг; и вратите на стаите бяха към север.
\par 5 А най-горните стаи бяха по-тесни, понеже галериите отнемаха от тях повече, отколкото от долните и средните етажи на зданието.
\par 6 Защото те бяха на третия етаж, и нямаха стълбове като стълбовете на дворовете; затова, най-горният етаж се стесняваше повече нежели най-долният и средният етаж, почвайки от земята.
\par 7 И на външната стена, която бе от страните на стаите, към външния двор, пред стаите, дължината бе петдесет лакти.
\par 8 Защото дължината на стаите, които бяха във външния двор беше петдесет лакти; и, ето, пред храма имаше сто лакти.
\par 9 А под тия стаи беше входът от изток като се отива към тях от външния двор.
\par 10 Стаите бяха в дебелината на дворовата стена към изток, пред определеното място, и пред зданието.
\par 11 И коридорът пред тях беше на изглед като коридора пред стаите, които бяха към север; имаха еднаква дължина и широчина; и всичките им изходи бяха според техните кроежи и според техните врати.
\par 12 И като имаше врати на стаите, които бяха на юг, така имаше и врата гдето почваше коридорът, коридорът право срещу стената към изток като се влиза в тях.
\par 13 Тогава ми рече: Северните стаи и южните стаи, които са пред отделеното място, са светите стаи, гдето свещениците, които се приближават при Господа, ще ядат пресветите неща; и хлебния принос, приноса за грях, и приноса за престъпление; защото мястото е свето.
\par 14 Когато свещениците влизат в храма, да не излизат от светото място във външния двор, но там да слагат дрехите, с които служат, защото са свети; а когато обличат други дрехи, тогава да се приближават при онова, което принадлежи на людете.
\par 15 А като свърши измерванията на дома отвътре, изведе ме по портата, която гледа към изток, и измери дома околовръст.
\par 16 Като измери източната страна с тръстикова мярка, тя беше петстотин тръстики, измерена наоколо с тръстикова мярка.
\par 17 Като измери северната страна, тя беше петстотин тръстики, измерена наоколо с тръстиковата мярка.
\par 18 Измери южната страна с тръстиковата мярка; тя беше петстотин тръстики.
\par 19 После се обърна към западната страна и я измери с тръстиковата мярка; тя беше петстотин тръстики.
\par 20 Измери го от четирите страни. Имаше и стена околовръст, дълга петстотин тръстики и широка петстотин тръстики, за да отделя светото от несветото място.

\chapter{43}

\par 1 После ме заведе при портата, портата, която гледа към изток;
\par 2 и ето, славата на Израилевия Бог идеше от източния път; гласът Му беше като глас на много води; и светът сияеше от славата Му.
\par 3 И видението, което видях, бе като видението, което видях, когато дойдох да пророкувам, че градът щял да се разруши; виденията бяха като видението, което видях при реката Ховар; и паднах на лицето си.
\par 4 И Господната слава влезе в дома през пътя на портата, която гледа към изток.
\par 5 И Духът ме дигна та ме заведе във вътрешния двор; и, ето, домът бе пълен с Господната слава.
\par 6 И чух едного да ми говори из дома; и човек стоеше при мене и ми рече:
\par 7 Сине човешки, това е мястото на престола Ми, и мястото на стъпалата на нозете Ми, гдето ще обитавам всред израилтяните до века; и Израилевият дом няма вече да омърси светото Ми име, ни те, ни царете им, с блудствата си или с труповете на идолитеси на високите си места.
\par 8 Като поставиха своя праг при Моя праг и стълбовете на своите врати при стълбовете на Моите врати, така щото нямаше друго освен стената между Мене и тях, те мърсяха светото Ми име с мерзостите, които вършеха; затова, изтребих ги в гнева Си.
\par 9 Сега нека отдалечат от Мене блудствата си и труповете на идолите си; и Аз ще обитавам всред тях до века.
\par 10 Ти, сине човешки, покажи тоя дом на Израилевия дом, за да се засрамят поради престъпленията си; и нека измерят плана му.
\par 11 И ако се засрамят за всичко, което са сторили, покажи им чертежа на дома и образа му, изходите му и входовете му, всичките му разпореждания и всичките му наредби, [всичките му разпореждания] и всичките му закони, и опиши го пред тях, за да пазят всичките му разпореждания и всичките му наредби, и да ги извършват.
\par 12 Ето законът на дома: Целият предел на върха на планината околовръст ще бъде пресвет. Ето, това е законът на дома.
\par 13 И ето мерките на олтара в лакти, като се смята лакът един лакът и длан: Дълбочината му да бъде един лакът, и широчината му един лакът, и первазът му около краищата му една педя; това ще бъде основата на олтара.
\par 14 А от дъното му, което е на земята, до долната полица да бъде два лакътя, и широчината един лакът; и от по-малката полица до по-голямата полица да бъде четири лакти, широчината един лакът.
\par 15 И горната част на олтара да бъде четири лакти висока; а от огнището на олтара нагоре да има четири рога.
\par 16 И огнището на олтара да бъде дванадесет лакти на длъж и дванадесет на шир; да образува квадрат с четирите си страни.
\par 17 И полицата да бъде четиринадесет лакти на длъж и четиринадесет на шир по четирите си страни; и первазът около него да бъде половин лакът; и дъното му един лакът наоколо, а стъпалата му да гледат към изток.
\par 18 И рече ми: Сине човешки, така казва Господ Иеова: Тия са наредбите на олтара, в деня, когато го направят, за да принасят върху него всеизгаряния и да ръсят върху него кръв.
\par 19 И на Левитските свещеници, които са от Садоковото потомство, които се приближават при Мене да Ми служат, казва Господ Иеова, да дадеш юнец в приноса за грях.
\par 20 И като вземеш от кръвта му, да я туриш на четирите рога на олтара, на четирите ъгли на полицата, и на перваза наоколо; така ще го очистиш и ще направиш умилостивение за него.
\par 21 После да вземеш и юнеца, който е в принос за грях, и нека го изгорят в определеното място на дома, вън от светилището.
\par 22 А на втория ден да принесеш козел без недостатък в принос за грях; и така ще очистят олтара както го очистиха с юнеца.
\par 23 Като свършиш чистенето му, да принесеш юнец без недостатък, и овен от стадото без недостатък.
\par 24 И като ги принесеш пред Господа, свещениците нека хвърлят сол на тях, и нека ги принесат всеизгаряне Господу.
\par 25 Седем дни да принасяш всеки ден козел в принос за грях; нека принасят тоже юнец и овен от стадото, които нямат недостатък.
\par 26 Седем дни нека правят умилостивение за олтара и го чистят; така ще го осветят.
\par 27 И като се свършат тия дни, от осмия ден нататък нека принасят свещениците всеизгарянията ви на олтара, и примирителните ви приноси; и Аз ще ви приема, казва Господ Иеова.

\chapter{44}

\par 1 Тогава Той ме върна по пътя на външната порта на светилището, която гледа към изток; и тя бе затворена.
\par 2 И Господ ми рече: Тая порта ще бъде затворена, няма да се отвори, и никой човек да не влезе през нея; защото Господ Израилевият Бог е влязъл през нея; затова, тя ще бъде затворена.
\par 3 А князът, който като княз ще седне в нея, за да яде хляб пред Господа, той ще влезе през пътя на предверието на тая порта, и през същия път ще излезе.
\par 4 Тогава ме заведе по пътя на северната порта срещу дома; и като погледнах, ето, Господният дом бе пълен с Господната слава; и паднах на лицето си.
\par 5 И Господ ми рече: Сине човешки, внимавай в сърцето си, погледни с очите си, и чуй с ушите си всичко, което ти казвам за всичките наредби на Господния дом, и за всичките му закони; и забележи добре входа на дома, с всичките изходи и светилището.
\par 6 И кажи на бунтовниците, сиреч, на Израилевия дом: Така казва Господ Иеова: Доме Израилев, нека ви са доволно всичките мерзости, които извършихте,
\par 7 дето въведохте инородци, с необрязано сърце и необрязана плът, да бъдат в светилището Ми та да го мърсят, дори в Моя дом, и дето, когато принасяте хляба Ми, тлъстината и кръвта, престъпихте завета Ми, в прибавка на всичките ви други мерзости.
\par 8 И не сте изпълнили службата на светите Ми неща, но сте поставили за себе си стражари над службата в светилището ми.
\par 9 Така казва Господ Иеова: От всичките инородци, които са между израилтяните, никой инородец с необрязано сърце, и необрязана плът да не влиза в светилището Ми.
\par 10 Но и левитите, които се отдалечиха от мене, когато заблуждаваше Израил, който заблуди от Мене и отиде след идолите си, те ще носят беззаконието си.
\par 11 Но пак, те ще бъдат служители в светилището Ми, да надзирават върху портите на дома и да служат в дома; те ще колят всеизгарянията и жертвите на людете, и те ще стоят пред тях, за да им слугуват.
\par 12 Понеже им слугуваха пред идолите им, и станаха спънка за увличане Израилевия дом в беззаконие, затова Аз дигнах ръката Си против тях, казва Господ Иеова; и те ще носят беззаконието си.
\par 13 И няма да се приближават при мене, за да ми свещенодействуват, нито да се приближават при нищо от светите ми неща, нито при пресветите; но ще носят срама си и мерзостите, които извършиха.
\par 14 Обаче, ще ги поставя стражари над службата на дома за всичката му прислуга, и за всичко, което ще се върши в него.
\par 15 А левитските свещеници, Садоковите потомци, които извършваха службата на светилището Ми, когато израилтяните заблуждаваха от Мене, те нека се приближават при Мене, за да Ми служат, и нека стоят пред Мене да Ми принасят тлъстината и кръвта, казва Господ Иеова.
\par 16 Те нека влизат в светилището Ми, и те нека се приближават при трапезата, за да Ми служат, и те нека извършват службата Ми.
\par 17 И когато влизат в портите на вътрешния двор нека обличат ленени дрехи; да няма нищо вълнено на тях, докато служат в портите на вътрешния двор и в дома.
\par 18 Нека имат ленени гъжви на главите си, нека имат и ленени гащи на кръста си; да не опасват нищо, което причинява пот.
\par 19 А когато излизат във външния двор, във външния двор към людете, нека събличат дрехите, с които са служили, и като ги слагат в светите стаи нека обличат други дрехи, за да не освещават людете с одеждите си.
\par 20 И да не бръснат главите си, нито да оставят космите си да растат, но само да стрижат главите си.
\par 21 И никой свещеник да не пие вино, когато влиза във вътрешния двор.
\par 22 И да не си вземе за жена вдовица или напусната; но да вземат девица от рода на Израилевия дом, или вдовица овдовяла от свещеник.
\par 23 И нека учат людете Ми да различават между свето и несвето, и нека ги упътват да разпознават нечисто от чисто.
\par 24 И в препирните те нека стоят да съдят; според Моите съдби нека ги съдят; и нека пазят законите Ми и повеленията Ми във всичките Ми определени празници; и нека освещават съботите Ми.
\par 25 Да не се допират при мъртъв човек та да се осквернят; обаче за баща или за майка, за син или за дъщеря, за брат или за неженена сестра, за тях бива да се оскверняват.
\par 26 А след като се очисти оскверненият и му минат седем дни,
\par 27 тогава в деня, когато влиза в светилището, във вътрешния двор, за да служи в светилището, нека принася приноса си за грях, казва Господ Иеова.
\par 28 А колкото за наследството им, Аз съм наследството им; да не им давате притежание в Израиля, защото Аз съм притежанието им.
\par 29 Нека ядат хлебния принос, приноса за грях и приноса за престъпление; и всяко обречено нещо в Израиля да бъде тяхно.
\par 30 И първенците от всичките първи рожби от всичко, и всеки принос от всичко, от всичките видове на вашите приноси, да бъдат на свещениците; тоже да давате на свещеника първака от тестото си, за да почива благословение на домовете ви.
\par 31 Свещениците да не ядат никаква мърша или разкъсано от звяр, било птица или животно.

\chapter{45}

\par 1 При това, когато делите земята с жребие за наследство, отделете свет дял от земята за принос на Господа; дължината му да бъде дължина от двадесет и пет хиляди тръстики, а широчината му десет хиляди; целият му предел да бъде свет околовръст.
\par 2 От тоя дял да се определи за светилището място петстотин тръстики дълго и петстотин широко, четвъртито наоколо, и петдесет лакти за предместието около него.
\par 3 И тъй, според тая мярка да отмериш място двадесет и пет хиляди дълго и десет хиляди широко; и в него да бъде светилището, пресветото място.
\par 4 Това да бъде свят дял от Израилевата земя за свещениците, служителите в светилището, които се приближават да служат Господу; нека им бъде място за къщи, и свето място за светилището.
\par 5 И Левитите да имат за себе си, като служителите на дома, едно място двадесет и пет хиляди дълго и десет хиляди широко за свое притежание, за градове, в които да живеят.
\par 6 И за притежание на града дайте едно място пет хиляди широко и двадесет и пет хиляди дълго, покрай принесения свети дял; това да бъде за целия Израилев дом.
\par 7 А за княза да има дял от двете страни на светия дял и на градското притежание, пред светия дял и пред градското притежание, от западната страна към запад, и от източната страна към изток; и дължината от западната му граница до източната му граница да бъде според дължината на всеки един от дяловете.
\par 8 Тоя дял от земята да бъде негово притежание в Израил; и князете Ми да не угнетяват вече людете Ми, а да дадат останалото от земята на Израилевия дом според племената им.
\par 9 Така казва Господ Иеова: Да ви е доволно, князе Израилеви; отдалечете насилието и грабителството, и извършвайте съдба и правда; престанете да изпъждате людете Ми от притежанията им, казва Господ Иеова.
\par 10 Имайте прави везни, права ефа, и прав ват.
\par 11 Ефата и ватът нека имат същия обем, тъй щото ватът да побира една десета от кора, и ефата една десета от кора; обемът им да се определя от кора.
\par 12 И сикълът нека бъде двадесет гери; двадесет сикла, двадесет и пет сикла, и петнадесет сикла нека бъде мнаса ви.
\par 13 Ето приносът, който ще принасяте: Шестата част на ефа от един кор жито; така и от ечемика да давате шестата част на ефа от един кор.
\par 14 А относно правилото за дървеното масло, от един ват масло принасяйте десетата част на ват от един кор, който е един хомер от десет вати; защото десет вати са един хомер.
\par 15 И от добре напоените Израилеви пасбища принасяйте и по едно агне от всяко стадо от двеста, които да бъдат за хлебен принос, за всеизгаряне, и за примирителен принос, за да се прави умилостивение за тях, казва Господ Иеова.
\par 16 Всичките люде от Израилевата земя нека дават тоя принос на княза в Израиля.
\par 17 А на княза ще принадлежи да дава всеизгарянията, хлебните приноси, и възлиянията в празниците, в новолунията и в съботите, във време на всичките определени празници на Израилевия дом; тоя нека приготвя приноса за грях, хлебния принос, всеизгарянето и примирителните приноси, за да прави умилостивение за Израилевия дом.
\par 18 Така казва Господ Иеова: В първия месец, на първия ден от месеца, да вземеш юнец без недостатък, и с него да очистиш светилището.
\par 19 Свещеникът нека взема от кръвта на приноса за грях, и нека я тури върху стълбовете на вратите на дома, върху четирите ъгъла на полицата на олтара, и върху стълбовете на портата на вътрешния двор.
\par 20 Така да правиш и на седмия ден от месеца за всекиго, който съгрешава от незнание, и за простия; така ще правите умилостивение за дома.
\par 21 В първия месец, на четиринадесетия ден от месеца да ви бъде пасхата, седемдневен празник, безквасен хляб да се яде.
\par 22 И в същия ден нека приготви князът за себе си и за всичките люде на Израилевата земя юнец в принос за грях.
\par 23 И през седемте дни на празника нека приготвя всеизгаряне Господу, по седем юнеца и по седем овена без недостатък всеки ден през седемте дни, тоже и козел всеки ден в принос за грях.
\par 24 И за хлебен принос нека приготвя по една ефа за всеки юнец, и по една ефа за всеки овен, и по един ин дървено масло за всяка ефа.
\par 25 В седмия месец, на петнадесетия ден от месеца, през седемте дни на празника, нека приготвя същото според определеното за приноса за грях, за всеизгарянето, и за хлебния принос, и за дървеното масло.

\chapter{46}

\par 1 Така казва Господ Иеова: Портата на вътрешния двор, която гледа към изток, нека бъде затворена през шестте делнични дни; а в съботния ден да се отваря, и в деня на новолунието да се отваря.
\par 2 И князът нека влезе по пътя на предверието на външната порта, и нека застане при стълба на портата; а свещениците нека принасят всеизгарянето му и примирителните му приноси; и той да се поклони при прага на портата. Тогава да излезе; обаче да се не затваря портата до вечерта.
\par 3 А людете на Израилевата земя да се кланят във входа на същата порта пред Господа в съботите и в новолунията.
\par 4 А всеизгарянето, което князът ще принася Господу в съботен ден да бъде шест агнета без недостатък и един овен без недостатък.
\par 5 И хлебният принос за овена да бъде една ефа, а хлебният принос за агнетата колкото му дава ръка, и един ин дървено масло за една ефа.
\par 6 И в деня на новолунието приносът му да бъде юнец без недостатък, шест агнета, и един овен, които да бъдат без недостатък.
\par 7 И нека принася хлебен принос, една ефа за юнеца, и една ефа за овена, а за агнетата, колкото му стига ръка, и по един ин дървено масло за всяка ефа.
\par 8 И когато князът влиза, нека влиза по пътя на предверието на портата, и нека излиза по същия път.
\par 9 Но когато людете на тая земя дохождат пред Господа във време на определените празници, тогава оня, който влиза по пътя на северната порта, за да се поклони, нека излиза по пътя на южната порта, а който влиза по пътя на южната порта, нека излиза по пътя на северната порта; да се не връща по пътя на портата, по който е влязъл, но да излиза като върви право напред.
\par 10 И князът да влиза всред тях, когато влизат; и когато излизат, да излизат заедно.
\par 11 И на тържествата и на празниците хлебният принос да бъде една ефа за юнеца, и една ефа за овена; а за агнетата, колкото му дава ръка, и един ин дървено масло за всяка ефа.
\par 12 А когато князът принася доброволно всеизгаряне или доброволни примирителни приноси Господу, тогава да му отварят портата, която гледа към изток, и нека принася всеизгарянето си и примирителните си приноси както прави в съботен ден; тогава нека излиза; и подир излизането му нека затварят портата.
\par 13 И да принасяш всеки ден във всеизгаряне Господу едногодишно агне без недостатък; всяка сутрин да го принасяш.
\par 14 И заедно с него да принасяш всяка сутрин за хлебен принос шестата част от една ефа, и третата част от един ин дървено масло, за да го смесиш с брашното; това да бъде хлебен принос Господу за винаги чрез вечна наредба.
\par 15 Така нека принасят агнето, хлебния принос, и дървеното масло всяка сутрин, за всегдашно всеизгаряне.
\par 16 Така казва Господ Иеова: Ако князът даде подарък на някого от синовете си, това ще му бъде наследство, ще бъде притежание на неговите синове, ще им бъде притежание по наследство.
\par 17 Но ако даде подарък от наследството си на някого от слугите си, тогава ще бъде негов само до годината на освобождението, подир която да се възвръща на княза; а наследството му ще бъде за синовете му.
\par 18 При това, князът да не взема от наследството на людете като ги извежда с насилие от притежанието им; ако даде наследството на синовете си, то от своето притежание да им даде, за да се не разпръсват людете Ми всеки от притежанието си.
\par 19 После ме заведе през входа, който бе от страните на портата, в светите свещенически стаи, които гледат към север; и, ето, там имаше място отзад към запад.
\par 20 И рече ми: Това е мястото гдето свещениците ще варят приноса за престъпление и приноса за грях, и гдето ще пекат хлебния принос, за да не ги изнасят във външния двор и да осветят людете.
\par 21 Тогава като ме изведе във външния двор, преведе ме около четирите ъгъла на двора; и, ето, във всеки ъгъл на двора имаше двор.
\par 22 В четирите ъгъла на двора имаше заградени дворове, четиридесет лакти на длъж и тридесет на шир; тия четири двора в ъглите имаха еднаква мярка.
\par 23 И имаше ред здания наоколо в тях, около четирите двора; имаше готварници устроени наоколо под редовете.
\par 24 Тогава ми каза: Тия са готварниците, дето служителите на дома ще варят жертвите на людете.

\chapter{47}

\par 1 Тогава ме върна при вратата на дома; и, ето, вода извираше изпод прага на дома към изток; защото лицето на дома бе към изток; и водата слизаше изпод дясната страна на дома, при южната страна на олтара.
\par 2 После ме изведе по пътя на северната порта, и ме преведе наоколо по външния път към външната порта, по пътя на портата, която гледа към изток; и, ето, вода течеше от дясната страна.
\par 3 И човекът, който държеше мярката в ръката си, като излезе към изток премери хиляда лакти, и ме преведе през водата; водата бе до глезени.
\par 4 Пак премери хиляда, и ме преведе през водата; водата бе до колене. Пак премери хиляда, и ме преведе; водата бе до кръста.
\par 5 После премери хиляда, и беше станала река, през която не можах да премина; защото водата беше се издигнала и бе станала вода за плуване, река непроходима.
\par 6 И рече ми: Видя ли, сине човешки? Тогава като ме заведе върна ме към брега на реката.
\par 7 А когато се върнах, ето, при брега на реката твърде много дървета и от двете й страни.
\par 8 Тогава ми рече: Тая вода изтича откъм източната страна, слиза към полето, и се влива в морето; и когато се излее в морето, водата му ще се изцери.
\par 9 И всяко одушевено, с което морето изобилва, ще живее на всичките места дето би отишла тая пълна река; и там ще има твърде голямо множество риба по причина, че тая вода е дошла там и че водите на морето са се изцерили; понеже дето отиде реката, всичко ще живее.
\par 10 И рибари ще стоят край нея от Енгади до Енеглаим, там ще простират мрежите си; рибите им ще бъдат твърде много по видовете си, като рибите на голямото море.
\par 11 Но тинестите и блатистите му места няма да се изцерят; ще бъдат предадени на сол.
\par 12 А край реката, по бреговете й от двете й страни, ще растат всякакви видове дървета за храна, чиито листа няма да вехнат, нито плодът им да оскъдее; всеки месец ще раждат нов плод, по причини че водата, която ги пои, изтича из светилището; и плодът им ще бъде за храна, а листът им за изцеление.
\par 13 Така казва Господ Иеова: Ето пределите, по които ще наследите земята, като я разделите според дванадесетте Израилеви племена: Иосиф ще има два дяла.
\par 14 Всеки както брат му ще наследите тая земя, за която се заклех, че ще я дам на бащите ви; да! тая земя ще ви се даде в наследство.
\par 15 И ето границата на земята: На северната страна ще бъде от голямото море, по пътя за Етлон, до прохода на Седад;
\par 16 после, Емат, Вирота, Сибраим, (който е между предела на Дамаск и предела на Емат), и Асаратихон, (който е при пределите на Ауран);
\par 17 а границата от морето ще бъде Асеренан (при предела на Дамаск), а към север северната граница е Емат. Това е северната страна.
\par 18 А източната, между Ауран, Дамаск и Галаад оттатък и Израилевата земя отсам, ще бъде Иордан, като я премерите от северната граница до източното море. Това е източната страна.
\par 19 А към юг южната страна ще бъде от Тамар до водата на Мерива Кадис, и през Египетския поток до голямото море. Това, към юг, е южната страна.
\par 20 А западната страна ще бъде голямото море от южната граница до прохода на Емат. Това е западната страна.
\par 21 Така да разделите тая земя помежду си според Израилевите племена.
\par 22 Да я разделите в наследство за вас и за чужденците, които пришелствуват между вас, които родят чада всред вас; тия нека ви бъдат като туземци между израилтяните; нека имат наследство с вас между Израилевите племена.
\par 23 В което племе пришелствува чужденецът, там му дайте наследство, казва Господ Иеова.

\chapter{48}

\par 1 А ето племената начиная от северния край, покрай пътя за Етлон, до прохода на Емат, и Асаренан (при предела на Дамаск, северно край Емат). И племената ще имат страните си към изток и към запад. Дан ще има един дял.
\par 2 А до Дановата граница, от източната страна до западната страна, Асир ще има един дял.
\par 3 А до Асировата граница, от източната страна до западната страна, Нефталим ще има един дял.
\par 4 А до Нефталимовата граница, от източната страна до западната страна, Манасия ще има един дял.
\par 5 А до Манасиевата граница, от източната страна до западната страна, Ефрем ще има един дял.
\par 6 А до Ефремовата граница, от източната страна до западната страна, Рувим ще има един дял.
\par 7 А до Рувимовата граница, от източната страна до западната страна, Юда ще има един дял.
\par 8 И до Юдовата граница, от източната страна до западната страна, да бъде приносът, който ще посветите, двадесет и пет хиляди тръстики широк, а дълъг колкото всеки от другите дялове от източната страна до западната страна; и светилището да бъде всред него.
\par 9 Приносът, който ще посветите Господу, да бъде двадесет и пет хиляди тръстики на длъж, и десет хиляди на шир.
\par 10 И тоя свет принос да бъде за тях, за свещениците, дълъг към север двадесет и пет хиляди тръстики, широк към запад десет хиляди, широк и към изток десет хиляди, а дълъг към юг двадесет и пет хиляди; И Господното светилище ще бъде всред него.
\par 11 Това да бъде за осветените свещеници от Садоковите потомци, които пазиха заръчването Ми и не се заблудиха, когато израилтяните се заблудиха, както се заблудиха левитите.
\par 12 И тоя посветен принос от приноса на земята ще им бъде пресвет, до предела на левитите.
\par 13 И съответстващ на границата на свещениците, левитите ще имат дял двадесет и пет хиляди тръстики дълъг и десет хиляди широк; цялата дължина да бъде двадесет и пет хиляди, и широчината десет хиляди.
\par 14 От него да не продават, нито да го дават в размяна, нито да предадат на други земните първаци; защото делът им е свет Господу.
\par 15 А петте хиляди тръстики, които остават в широчината срещу двадесет и петте хиляди, да бъдат за обща употреба, за града, за живеене и за пасбища; а градът да бъде всред него.
\par 16 И ето мерките му: северната страна четири хиляди и петстотин тръстики, южната страна четири хиляди и петстотин, източната страна четири хиляди и петстотин и западната страна четири хиляди и петстотин.
\par 17 И пасбищата на града да бъдат на север двеста и петдесет, на юг двеста и петдесет, на изток двеста и петдесет, и на запад двеста и петдесет.
\par 18 И остатъкът от дължината, съответствуващ на светия принос, ще бъде десет хиляди тръстики на изток, и десет хиляди на запад; ще съответствува на светия принос; и рожбите му ще бъдат за храна на ония, които слугуват в града.
\par 19 Ония от всичките Израилеви племена, които слугуват в града, да го обработват.
\par 20 Цялото посветено място да бъде двадесет и пет хиляди тръстики на двадесет и пет хиляди; четвъртит ще направите светия принос, включително градското притежание.
\par 21 И остатъкът да бъде за княза, от двете страни на светия принос и на градското притежание, пред посветеното място от двадесет и петте хиляди тръстики към източната граница, а на запад пред двадесет и петте хиляди към западната граница, съответстващ на всеки от дяловете; това да бъде за княза. А светият принос със светилището на дома ще бъде всред него.
\par 22 И така, мястото от притежанието на Левитите и от притежанието на града, като е по средата между границата на Юда и границата на Вениамина, ще бъде на княза; на княза ще бъде.
\par 23 А колкото за останалите племена: от източната страна до западната страна Вениамин ще има един дял.
\par 24 А до Вениаминовата граница, от източната страна до западната страна, Симеон ще има един дял.
\par 25 А до Симеоновата граница, от източната страна до западната страна, Исахар ще има един дял.
\par 26 А до Исахаровата граница, от източната страна до западната страна, Завулон ще има един дял.
\par 27 А до Завулоновата граница, от източната страна до западната страна, Гад ще има един дял.
\par 28 А до Гадовата граница, на южната страна към юг, границата да бъде от Тамар до водата на Мерива Кадис, и през Египетския поток до голямото море.
\par 29 Тая е земята, която ще разделите с жребие на Израилевите племена в наследсвто, и тия са дяловете на всяко от тях, казва Господ Иеова.
\par 30 А ето предградията на града: на северната страна четири хиляди и петстотин тръстики по мярка;
\par 31 а градските порти да се наричат с имената на Израилевите племена; три от портите да гледат към север - Рувимовата порта една, Юдовата порта една, и Левиевата порта една;
\par 32 и на източната страна четири хиляди и петстотин тръстики, и три порти - Иосифовата порта една, Вениаминовата порта една, и Дановата порта една;
\par 33 и на южната страна четири хиляди и петстотин тръстики по мярка, и три порти - Симеоновата порта една, Исахаровата порта една и Завулоновата порта една;
\par 34 а на западната страна четири хиляди и петстотин тръстики, и три порти - Гадовата порта една, Асировата порта една и Нефталимовата порта една.
\par 35 Окръжността да бъде осемнадесет хиляди тръстики; и от оня ден името на града ще бъде Иеовашама.

\end{document}