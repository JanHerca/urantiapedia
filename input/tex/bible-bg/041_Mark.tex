\begin{document}

\title{Mark}


\chapter{1}

\par 1 Началото на благовестието на Исуса Христа Божий Син;
\par 2 както е писано в книгата на пророк Исаия: - "Ето, Аз изпращам пред лицето Ти вестителя Си, който ще устрои Твоя път;
\par 3 Глас на един, който вика в пустинята: Пригответе пътя за Господа. Прави направете пътеките за Него."
\par 4 Йоан дойде, който кръщаваше в пустинята, и проповядваше кръщение на покаяние за опрощаване греховете.
\par 5 И излизаше при него цялата Юдейска страна и всичките ерусалимяни, и кръщаваха се от него в реката Йордан, като изповядваха греховете си.
\par 6 А Йоан носеше облекло от камилска козина и кожен пояс около кръста си, и хранеше се с акриди и див мед.
\par 7 И като проповядваше, казваше: Подир мене иде Онзи, Който е по-силен от мене, Комуто не съм достоен да се наведа и развържа ремъка на обущата Му.
\par 8 Аз ви кръщавам с вода; а Той ще ви кръсти със Святия Дух.
\par 9 През тия дни дойде Исус от Назарет Галилейски и се кръсти от Йоана в Йордан.
\par 10 И като излезе веднага от водата, видя, че се разтварят небесата, и че Духът като гълъб слизаше на Него.
\par 11 И дойде глас от небесата: Ти си Моят възлюблен Син; в Тебе е Моето благоволение.
\par 12 И веднага Духът Го закара в пустинята.
\par 13 И беше в пустинята четиридесет дни изкушаван от Сатана, и беше със зверовете; а ангелите Му служеха.
\par 14 А след като Йоан биде предаден на властта, Исус дойде в Галилея и проповядваше Божието благовестие, казвайки:
\par 15 Времето се изпълни, и Божието царство наближи; покайте се и повярвайте в благовестието.
\par 16 А когато минаваше край Галилейското езеро(Гръцки: Море, така и навсякъде в това евангелие.), видя Симона и брата на Симона, Андрея, че хвърляха мрежи в езерото, понеже бяха рибари.
\par 17 И рече им Исус: Вървете след Мене, и Аз, ще ви направя да станете ловци на човеци.
\par 18 И те веднага оставиха мрежите и отидоха след Него.
\par 19 И като отмина малко, видя Якова Заведеев и брата му Йоана, които също бяха в ладията си и кърпеха мрежите.
\par 20 И веднага ги повика; и те оставиха баща си Заведея в ладията с надничарите, и отидоха след Него.
\par 21 И дохождат в Капернаум; и незабавно в съботата Исус влезе в синагогата и поучаваше.
\par 22 И те се чудеха на учението Му; защото ги поучаваше като един, който има власт, а не като книжниците.
\par 23 И скоро след това се намираше в синагогата им човек хванат от нечист дух, който извика, думайки:
\par 24 [Остави ни!] какво имаш Ти с нас, Исусе назарянине? Нима си дошъл да ни погубиш? Познавам Те Кой си, Светий Божий.
\par 25 Но Исус го смъмра, казвайки: Млъкни, и излез из него.
\par 26 Тогава нечистият дух, като го сгърчи, изкрещя със силен глас и излезе из него.
\par 27 И всички се смаяха, тъй щото разискваха помежду си, думайки: Що е това? Едно ново учение! С власт заповядва и на нечистите духове, и те Му се покоряват!
\par 28 И слухът за Него се разнесе скоро навред по цялата галилейска околност.
\par 29 И щом излязоха от синагогата, дойдоха с Якова и Йоана в къщата на Симона н Андрея.
\par 30 А Симоновата тъща лежеше болна от треска; и веднага Му казват за нея.
\par 31 И Той се приближи, и като я хвана за ръка, дигна я; и [на часа] треската я остави, и тя им прислужваше.
\par 32 И като се свечери, когато залезе слънцето, доведоха при Него всичките болни и хванати от бяс.
\par 33 И целият град се събра пред вратата.
\par 34 И той изцели мнозина, които страдаха от разни болести, и изгони много бесове; и не позволяваше на бесовете да говорят, понеже Го познаваха.
\par 35 И сутринта, когато беше още тъмно, стана та излезе, и отиде в уединено място, и там се молеше.
\par 36 А Симон и ония, които бяха с Него, изтичаха подире Му.
\par 37 И като Го намериха, казват Му: Всички те търсят.
\par 38 А Той им казва: Да идем другаде, в близките градчета и там да проповядвам; защото за това съм излязъл.
\par 39 И влизаше в синагогите им по цяла Галилея, като проповядваше и изгонваше бесовете.
\par 40 И дохожда при него при един прокажен и Му се моли, коленичил при Него, казвайки: Ако искаш, можеш да ме очистиш.
\par 41 А Той се смили, простря ръка и се допря до него, и му каза: Искам, бъди очистен.
\par 42 И веднага проказата го остави, и той се очисти.
\par 43 И на часа го отпрати, и му заръча строго, като му каза:
\par 44 Внимавай да не кажеш никому нищо; но за свидетелство на тях, иди и се покажи на свещеника и принеси за очистването си това, което е заповядал Мойсей.
\par 45 А той, като излезе, почна да разгласява много и да разнася станалото, така щото Исус не можеше вече да влезе явно в някой град, но се намираше вън в уединени места; и дохождаха при Него отвсякъде.

\chapter{2}

\par 1 След известно време, Той пак влезе в Капернаум; и разчу се че бил в къщата.
\par 2 И мнозина се събраха, така щото и около вратата не можаха да се поберат; и Той им говореше словото.
\par 3 Дохождат и донасят при Него един паралитик. Четирима го носеха.
\par 4 И като не можаха да се приближат до Него, поради народа, разкриха покрива на къщата, гдето беше, пробиха го и пуснаха постелката, на която лежеше паралитикът.
\par 5 А Исус, като видя вярата им, каза на паралитика: Синко, прощават ти се греховете.
\par 6 А имаше там някои от книжниците, които седяха и размишляваха в сърцата си:
\par 7 Тоя защо говори така? Той богохулствува. Кой може да прощава грехове, освен един Бог?
\par 8 Исус, като разбра веднага с духа Си, че така размишляват в себе си, рече им: Защо размишлявате това в сърцата си?
\par 9 Кое е по-лесно, да река на паралитика: Прощават ти се греховете, или да река: Стани, вдигни постелката си и ходи?
\par 10 Но за да познаете, че Човешкият Син има власт на земята да прощава грехове, (казва на паралитика):
\par 11 Тебе казвам: Стани, вдигни си постелката и иди у дома си.
\par 12 И той стана, веднага вдигна постелката и излезе пред всичките; така щото всички се зачудиха и славеха Бога, и думаха: Никога не сме виждали такова нещо.
\par 13 И пак излезе край езерото; и цялото множество идеше при Него, и Той ги поучаваше.
\par 14 Като минаваше видя Левия Алфеев седящ в бирничеството, и каза му: Върви след Мене. И той стана и отиде след Него.
\par 15 И когато Исус седеше на трапезата в къщата му, заедно с Него и учениците Му насядаха и много бирници и грешници; защото бяха мнозина и отиваха подир Него.
\par 16 Тогава книжниците, които бяха от фарисеите, като видяха, че Той яде с грешниците и бирниците, казаха на учениците Му: Защо яде и пие с бирниците и грешниците?
\par 17 А Исус, като чу това, каза им: Здравите нямат нужда от лекар, а болните; не съм дошъл да призова праведните, но грешниците (на покаяние).
\par 18 А Йоановите ученици и фарисеите постеха; и дохождат и казват Му: Защо постят Йоановите и фарисейските ученици, а Твоите не постят?
\par 19 А Исус им рече: Могат ли сватбарите да постят, докато е с тях младоженецът? Дотогава, докогато младоженецът е с тях, не могат да постят.
\par 20 Ще дойдат, обаче, дни, когато младоженецът ще се отнеме от тях, и тогава, през ония дни, ще постят.
\par 21 Никой не пришива кръпка от невалян плат на вехта дреха; а инак, това, което трябваше да запълни дрехата, отдира от нея, новото от вехтото, и съдраното става по-лошо.
\par 22 И никой не налива ново вино в стари мехове; инак, виното ще спука меховете; и се изхабяват и виното и меховете; но наливат ново вино в нови мехове.
\par 23 И в една събота, когато минаваше през посевите, учениците Му, вървейки из пътя, почнаха да късат класове.
\par 24 И фарисеите Му рекоха: Виж! Защо вършат в събота нещо, което не е позволено?
\par 25 А Той им рече: Не сте ли чели що стори Давид, когато беше в нужда, и огладня и мъжете, които бяха с него?
\par 26 Как влезе в Божия дом, когато Авиатар беше първосвещеник, и изяде присъствените хлябове, които не е позволено никому да яде, освен на свещениците, като даде и на ония, които бяха с него?
\par 27 И каза им: Съботата е направена за човека, а не човек за съботата;
\par 28 така щото Човешкият Син е господар и на съботата.

\chapter{3}

\par 1 И влезе пак в синагогата; и там имаше човек с изсъхнала ръка.
\par 2 И наблюдаваха Го, дали ще го изцели в съботен ден, за да Го обвинят.
\par 3 Той каза на човека с изсъхналата ръка: Изправи се насред.
\par 4 Тогава на тях казва: Позволено ли е да се прави добро в съботен ден, или да се прави зло? Да се спаси ли живот, или да се погуби? А те мълчаха.
\par 5 А като ги изгледа с гняв, наскърбен поради закоравяването на сърцата им, каза на човека: Простри си ръката. Той я простря; и ръката му оздравя.
\par 6 И фарисеите, като излязоха, веднага се наговориха с Иродианите против Него, как да Го погубят.
\par 7 Тогава Исус се оттегли с учениците Си към езерото, и голямо множество от Галилея отиде изподир.
\par 8 И от Юдея, от Ерусалим, от Идумея, отвъд Йордан, и от местата около Тир и Сидон, едно голямо множество дойде при Него, като чуха колко много чудеса правел.
\par 9 И поръча на учениците Си да Му услужат с една ладия, поради народа, за да Го не притискат.
\par 10 Защото беше изцелил мнозина, така че онези, които страдаха от язви, натискаха Го, за да се допрат до Него.
\par 11 И нечистите духове, когато Го виждаха, падаха пред Него и викаха, казвайки: Ти си Божий Син.
\par 12 Но Той строго им заръчваше да Го не изявяват.
\par 13 След това се възкачи на хълма и повика при Себе Си ония, които си искаше; и те отидоха при Него.
\par 14 И определи дванадесет души, за да бъдат с Него, и за да ги изпраща да проповядват,
\par 15 и да имат власт да изгонват бесове.
\par 16 Определи: Симона, на когото даде и името Петър;
\par 17 и Якова Заведеев и Якововия брат Йоан, на които даде и името Воанергес, сиреч, синове на гръма,
\par 18 и Андрея и Филипа, Вартоломея и Матея, Тома и Якова Алфеев, Тадея и Симона Зилот,
\par 19 и Юда Искариотски, който Го и предаде.
\par 20 И дохожда в една къща; и пак се събира народ, така щото те не можаха нито хляб да ядат.
\par 21 А своите Му, като чуха това, излязоха за да Го хванат; защото казваха, че не бил на Себе Си.
\par 22 И книжниците, които бяха слезли от Ерусалим, казваха, че Той има Веелзевул, и че изгонва бесовете чрез началника на бесовете.
\par 23 Но Той, като ги повика, казваше им с притчи: Как може Сатана да изгонва Сатаната?
\par 24 Ако едно царство се раздели против себе си, това царство не може да устои.
\par 25 И ако един дом се раздели против себе си, тоя дом не ще може да устои.
\par 26 И ако Сатана е възстанал против себе си, и се е разделил, той не може да устои, но дошъл му е краят.
\par 27 Обаче, никой не може да влезе в къщата на силния човек, да ограби покъщнината му, ако първо не върже силния, и тогава ще ограби къщата му.
\par 28 Истина ви казвам, че всичките грехове на човешкия род ще бъдат простени, и всичките хули с които биха богохулствували;
\par 29 но ако някой похули Святия Дух, за него няма прошка до века, но е виновен за вечен грях.
\par 30 Това рече Той, защото казваха: Има нечист дух.
\par 31 Дохождат, прочее, майка Му и братята Му, и като стояха вън, пратиха до Него да Го повикат.
\par 32 А около Него седеше едно множество; и казват Му: Ето, майка Ти и братята Ти вън, търсят Те.
\par 33 И в отговор им каза: Коя е майка Ми? Кои са братята Ми?
\par 34 И като изгледа седящите около Него каза: Ето майка Ми и братята Ми!
\par 35 Защото, който върши Божията воля, той Ми е брат, и сестра, и майка.

\chapter{4}

\par 1 И пак започна да поучава край езерото. И събра се при Него едно твърде голямо множество, така щото Той влезе в една ладия и седеше на езерото; а цялото множество беше на сушата край езерото.
\par 2 И поучаваше ги много с притчи, и казваше им в поучението Си:
\par 3 Слушайте: Ето, сеячът излезе да сее.
\par 4 И когато сееше, някой зърна паднаха край пътя; и птиците дойдоха и ги изкълваха.
\par 5 Други паднаха на канаристо място, гдето нямаше много пръст, и скоро поникнаха, защото нямаше дълбока почва;
\par 6 а когато изгря слънцето, пригоряха, и понеже нямаха корен, изсъхнаха.
\par 7 И други паднаха между тръните; и тръните порастнаха и ги заглушиха, и не дадоха плод.
\par 8 А другите паднаха на добрата земя, и даваха плод, който растеше и се умножаваше, и принесоха кое тридесет, кое шестдесет и кое сто.
\par 9 И каза: Който има уши да слуша, нека слуша.
\par 10 И когато остана сам, ония, които бяха около Него с дванадесетте, Го попитаха за притчите.
\par 11 И каза им: На вас е дадено да познаете тайната на Божието царство; а на ония, външните всичко бива в притчи;
\par 12 тъй щото гледащи да гледат, а да не виждат, и слушащи, да слушат, а да не разбират, да не би да се обърнат и да им се прости [греха].
\par 13 И казва им: Не разбирате ли тая притча? А как ще разберете всичките притчи?
\par 14 Сеячът сее словото.
\par 15 А ония край пътя, гдето се сее словото, са тия, които като чуват, Сатана веднага дохожда и грабва посеяното в тях слово.
\par 16 Също и посеяното на канаристите места са тия които, като чуят словото, веднага с радост го приемат;
\par 17 нямат, обаче, корен в себе си, но са привременни; после, като настане напаст или гонение, поради словото, веднага се съблазняват.
\par 18 Посеяното между тръните са други; те са ония, които са слушали словото;
\par 19 а светските грижи, примамката на богатството, и пожеланията за други работи, като влязат, заглушават словото, и то става безплодно.
\par 20 А посеяното на добрата земя са тия, които слушат словото, приемат го, и дават плод, тридесет, шестдесет, и стократно.
\par 21 И каза им: Затова ли се донася светилото, за да го турят под шиника или под леглото? Не за това ли, да го поставят на светилника?
\par 22 Защото няма нещо тайно, освен за да се яви; нито е имало нещо спотаено, освен за да излезе наяве.
\par 23 Ако има някой уши да слуша, нека слуша.
\par 24 Каза им тоже: Внимавайте в това, което слушате. С каквато мярка мерите, ще ви се отмери, и ще ви се прибави.
\par 25 Защото който има, нему ще се даде, а който няма, от него ще се отнеме и това, което има.
\par 26 И каза: Божието царство е също, както кога човек хвърли семе в земята;
\par 27 и спи, и става нощ и ден; а как никне и расте, той не знае.
\par 28 Земята сама по себе си произвежда, първо ствол, после клас, подир това пълно зърно в класа.
\par 29 А когато узрее плодът, начаса изпраща сърпа, защото е настанала жетва.
\par 30 При това каза: На какво да уприличим Божието царство? Или с каква притча да го представим?
\par 31 То прилича на синапово зърно, което, когато се посее в земята, е по-малко от всичките семена, които са на земята;
\par 32 но когато се посее, расте, и става по-голямо от всичките злакове, и пуска големи клони, така че под сянката му могат да се подслонят небесните птици.
\par 33 С много такива притчи им прогласяше словото, според както можеха да слушат.
\par 34 А без притча не им говореше; но насаме обясняваше всичко на Своите ученици.
\par 35 И в същия ден, когато се свечери, Исус им казва: Да минем на отвъдната страна.
\par 36 И като оставиха народа, вземат Го със себе си в ладията, тъй както бе; и имаше други ладии с Него.
\par 37 И дига се голяма ветрена буря, и вълните се нахвърляха в ладията, тъй че тя вече се пълнеше с вода.
\par 38 А Той беше в задната част, заспал на възглавница; и те Го събуждат и Му казват: Учителю! Нима не Те е грижа че загиваме?
\par 39 И Той, като се събуди, смъмра вятъра и рече на езерото: Мълчи! Утихни! И вятърът престана, и настана голяма тишина.
\par 40 И рече им: Защо сте страхливи? Още ли нямате вяра?
\par 41 И голям страх ги обзе; и те си казаха един на друг: Кой е, прочее, Тоя, че и вятърът и езерото Му се покоряват?

\chapter{5}

\par 1 И тъй минаха отвъд езерото в гадаринската страна.
\par 2 И като излезе от ладията, начаса Го срещна от гробищата човек с нечист дух.
\par 3 Той живееше в гробищата, и никой вече не можеше да го върже нито с верига;
\par 4 защото много пъти бяха го вързвали с окови и с вериги; но той бе разкъсвал веригите и счупвал оковите; и никой нямаше сила да го укроти.
\par 5 И всякога, нощем и денем, в гробищата и по бърдата, той викаше и се изпосичаше с камъни.
\par 6 А като видя Исуса отдалеч, завтече се и Му се поклони;
\par 7 и изкрещя със силен глас и рече: Какво имаш Ти с мен, Исусе, Сине на Всевишния Бог? Заклевам Те в Бога, недей ме мъчи.
\par 8 (Защото му казваше: Излез от човека, душе нечисти).
\par 9 И Исус го попита: Как ти е името? А той Му каза: Легион ми е името; защото сме мнозина.
\par 10 И много Му се моли да не ги отпраща вън от страната.
\par 11 А там по бърдото пасеше голямо стадо свине.
\par 12 И бесовете Му се молиха, казвайки: Прати ни в свинете, за да влезем в тях.
\par 13 Исус им позволи. И нечистите духове излязоха и влязоха в свинете; и стадото, на брой около две хиляди, се спусна по стръмнината в езерото, и се издавиха в езерото.
\par 14 А ония, които ги пасяха, побягнаха и известиха това в града и по селата. И жителите дойдоха да видят какво е станало.
\par 15 И като дохождат при Исуса, виждат хванатия по-преди от бесове, в когото е бил легиона, че седи облечен и смислен; и убояха се.
\par 16 И ония, които бяха видели, разказваха им за станалото с хванатия от бесовете, и за свинете.
\par 17 И те почнаха да Му се молят да си отиде от техните предели.
\par 18 И когато влизаше в ладията, тоя, който бе по-напред хванат от бесове, Му се молеше да бъде заедно с Него.
\par 19 Обаче Той не го допусна, но му каза: Иди си у дома при своите, и кажи им какви неща ти стори Господ и как се смили за тебе.
\par 20 И човекът тръгна и почна да разгласява в Декапол какви неща му стори Исус; и всички се чудеха.
\par 21 Когато Исус пак премина с ладията на отвъдната страна, събра се при Него голямо множество; и Той беше край езерото.
\par 22 И дохожда един от началниците на синагогата, на име Яир, и като Го вижда, пада пред нозете Му,
\par 23 и много Му се моли, казвайки: Малката ми дъщеря бере душа; моля Ти се да дойдеш и положиш ръце на нея, за да оздравее и да живее.
\par 24 И Той отиде с него; и едно голямо множество вървеше подире Му, и хората Го притискаха.
\par 25 И една жена, която бе имала кръвотечение дванадесет години,
\par 26 и беше много пострадала от мнозина лекари, и беше иждивила целия си имот без да види някаква полза, а напротив беше й станало по-зле,
\par 27 като чу отзивите за Исуса, дойде между народа изотзад и се допря до дрехата Му.
\par 28 Защото си казваше: Ако само се допра до дрехата Му, ще оздравея.
\par 29 И на часа пресекна кръвотечението й, и тя усети в тялото си, че се изцели от болестта.
\par 30 И веднага Исус като усети в Себе Си, че е излязла от Него сила, обърна се всред народа и каза: Кой се допря до дрехите Ми?
\par 31 Учениците Му казаха: Ти виждаш, че народът те притиска, и казваш ли: Кой се допря до Мене?
\par 32 Но Той се озърташе за да види тая, която бе сторила това.
\par 33 А жената, уплашена и разтреперана, като знаеше станалото с нея, дойде и падна пред Него и Му каза цялата истина.
\par 34 А Той й рече: Дъщерьо, твоята вяра те изцели; иди си с мир, и бъди здрава от болестта си.
\par 35 Докато Той още говореше, дохождат от къщата на началника на синагогата и казват: Дъщеря ти умря; защо вече затрудняваш Учителя?
\par 36 А Исус, като дочу думата, която говореха, каза на началника на синагогата: Не бой се, само вярвай.
\par 37 И никому не позволи да Го придружи, освен на Петра, Якова и Якововия брат Йоан.
\par 38 И като дохождат до къщата на началника на синагогата, Той вижда вълнение и мнозина, които плачеха и пищяха много.
\par 39 И като влезе, каза им: Защо правите вълнение и плачете? Детето не е умряло, а спи.
\par 40 А те Му се присмиваха. Но Той като изкара навън всичките, взема бащата и майката на детето, и ония, които бяха с Него, и влиза там гдето беше детето.
\par 41 И като хвана детето за ръка, каза му: Талита куми; което значи: Момиче, тебе казвам: Стани.
\par 42 И момичето веднага стана и ходеше, защото беше на дванадесет години. И внезапно те се смаяха твърде много.
\par 43 И много им заръча, никой да не узнае това; и заповяда да й дадат да яде.

\chapter{6}

\par 1 И той излезе оттам и дойде в Своята родина; и учениците Му вървяха подир Него.
\par 2 И когато настана събота, почна да поучава в синагогата; и мнозина, като Го слушаха, се чудеха и думаха: Откъде има Тоя всичко това? И: Каква е дадената на този мъдрост, и какви са тези велики дела извършени от ръцете Му?
\par 3 Тоя не е ли дърводелецът, син на Мария, и брат на Якова и Иосия, на Юда и Симона? И сестрите Му не са ли тука между нас? И те се съблазниха в Него.
\par 4 А Исус им каза: Никой пророк не е без почит, освен в своята родина, и между своите сродници, и в своя си дом.
\par 5 И не можеше да извърши там никакво велико дело, освен дето положи ръце на малцина болни и ги изцели.
\par 6 И чудеше се за тяхното неверие. И обикаляше околните села и поучаваше.
\par 7 И като повика дванадесетте, почна да ги разпраща двама по двама, и даде им власт над нечистите духове.
\par 8 И заповяда им да не вземат нищо за по път, освен една тояга; ни хляб, ни торба, ни пари в пояса;
\par 9 но да се обуват със сандали: И, рече Той, не се обличайте в две дрехи.
\par 10 И каза им: В която къща влезете, оставайте в нея докле си излезете оттам.
\par 11 И ако в някое място не ви приемат, нито ви послушат, като излизате оттам отърсете праха из под нозете си като свидетелство против тях.
\par 12 И те излязоха и проповядваха, че човеците трябва да се покаят.
\par 13 И изгонваха много бесове, и мнозина болни помазваха с масло и ги изцеляваха.
\par 14 И цар Ирод чу за Исуса (защото името Му стана известно), и думаше: Йоан Кръстител е възкръснал от мъртвите и затова тия велики сили действуват чрез Него.
\par 15 А други казваха, че е Илия. Други пък казваха, че Той е пророк, като един от старовременните пророци.
\par 16 Но Ирод, като чу за Него, рече: Това е Йоан, когото аз обезглавих, той е възкръснал.
\par 17 Защото тоя Ирод беше пратил да хванат Йоана, и да го вържат в тъмница, заради Иродиада, жената на брата му Филипа, понеже я беше взел за жена.
\par 18 Защото Йоан казваше на Ирода: Не ти е позволено да имаш братовата си жена.
\par 19 А Иродиада се настрои против него и искаше да го убие, но не можеше;
\par 20 защото Ирод се боеше от Йоана, като знаеше, че той е човек праведен и свят, и го пазеше здраво; и когато го слушаше, вършеше много неща, и с удоволствие го слушаше.
\par 21 И като настана сгоден ден, когато Ирод за рождения си ден направи вечеря на големците си и на хилядниците и на галилейските старейшини,
\par 22 и самата дъщеря на Иродиада влезе и поигра, тя угоди на Ирода и на седящите с него; и царят рече на момичето: Искай от мене каквото щеш, и ще ти го дам.
\par 23 И закле й се: Каквото и да поискаш от мене, ще ти дам, даже до половината на царството ми.
\par 24 А тя излезе и рече на майка си: Какво да поискам? И тя каза: Главата на Йоана Кръстителя.
\par 25 И начаса момичето влезе бързо при царя и поиска, като каза: Искам да ми дадеш още сега, на блюдо, главата на Йоана Кръстителя.
\par 26 И царят се наскърби много; но пак, заради клетвите си и заради седящите с него, не иска да й откаже.
\par 27 И веднага царят прати един телохранител, комуто заповяда да донесе главата му; и той отиде, обезглави го в тъмницата,
\par 28 и донесе главата му на блюдо и я даде на момичето; а момичето я даде на майка си.
\par 29 И учениците му, като чуха това, дойдоха и дигнаха тялото му, и го положиха в гроб.
\par 30 И апостолите се събраха при Исуса и разказаха Му всичко каквото бяха извършили и каквото бяха поучили.
\par 31 И рече им: Дойдете вие сами на уединено място насаме и починете си малко. Защото мнозина дохождаха и отиваха; и нямаха време нито да ядат.
\par 32 И отидоха с ладията на уединено място насаме.
\par 33 А като отидоха, людете ги видяха, и мнозина ги познаха; и от всичките градове се стекоха там пеши, и ги изпревариха.
\par 34 И Исус като излезе, видя едно голямо множество, и смили се за тях, понеже бяха като овце, които нямат пастир; и почна да ги поучава много неща.
\par 35 И когато беше станало вече късно, учениците Му се приближиха при Него и казаха: Мястото е уединено, и вече е късно;
\par 36 разпусни ги за да отидат по околните колиби и села и да си купят нещо за ядене.
\par 37 А Той в отговор им рече: Дайте им вие да ядат. А те Му казаха: Да идем ли да купим за двеста пеняза хляб и да им дадем да ядат?
\par 38 А Той им каза: Колко хляба имате? Идете вижте. И като узнаха, казаха: Пет, и две риби.
\par 39 И заповяда им да насядат всички на групи по зелената трева.
\par 40 И те насядаха на редици, по сто и по петдесет.
\par 41 И като взе петте хляба и двете риби, Исус погледна на небето и благослови; и разчупи хлябовете, и даваше на учениците да наслагат отпреде им; раздели и двете риби на всичките.
\par 42 И всички ядоха и се наситиха.
\par 43 И дигнаха къшеи, дванадесет пълни коша, така и от рибите.
\par 44 А ония които ядоха хлябовете, бяха петхиляди мъже.
\par 45 И веднага накара учениците Си да влязат в ладията и да отидат преди Него на отвъдната страна към Витсаида, докле Той разпусне народа.
\par 46 И след като се прости с тях, отиде на бърдото да се помоли.
\par 47 И когато се свечери, ладията беше всред морето, а Той сам на сушата.
\par 48 И като ги видя, че се мъчат като гребат с веслата, защото им беше противен вятърът, около четвъртата стража на нощта дохожда при тях, като вървеше по езерото; и щеше да ги отмине.
\par 49 А те, като Го видяха да ходи по езерото, помислиха си, че е призрак, и извикаха;
\par 50 защото всички Го видяха и се смутиха. И веднага Той им проговори, като им каза: Дерзайте! Аз съм, не бойте се!
\par 51 И влезе при тях в ладията, и вятърът утихна; и те много се ужасиха в себе си.
\par 52 Защото не бяха се вразумили от чудото с хлябовете, но сърцето им беше закоравяло.
\par 53 И като премина езерото, дойдоха в генисаретската земя и излязоха на сушата.
\par 54 И когато излязоха из ладията, веднага хората Го познаха;
\par 55 и разтичаха се по цялата оная околност и почнаха да носят на легла болните там, гдето чуеха, че се намирал Той.
\par 56 И гдето и да влизаше, в села или в градове или в колиби, туряха болните по пазарите, и молеха Му се да се допрат те поне до полите на дрехите Му; и колкото души се допираха, се изцеляваха.

\chapter{7}

\par 1 Събират се при Исуса фарисеите и някои от книжниците, които бяха дошли от Ерусалим,
\par 2 и бяха видели, че някои от учениците Му ядат хляб с ръце нечисти, сиреч, немити.
\par 3 (Защото фарисеите и всички юдеи, държейки преданието на старейшините, не ядат ако не си омият ръцете до лактите;
\par 4 и когато се връщат от пазар; не ядат ако се не омият. Има и много други неща, които са приели да държат, - измивания на чаши, шулета, и медни съдове [и одрове]).
\par 5 И тъй, фарисеите и книжниците Го запитват: Защо не вървят Твоите ученици по преданието на старейшините, но ядат хляб с нечисти ръце?
\par 6 А той им рече: Добре е пророкувал Исаия за вас лицемерите, както е писано: - "Тези люде Ме почитат с устните си, Но сърцето им отстои далеч от Мене.
\par 7 Обаче напразно Ме почитат, като преподават за поучения човешки заповеди".
\par 8 Вие оставяте Божията заповед и държите човешкото предание, [измивания на шулци и на чаши; и много други неща правите].
\par 9 И каза им: Хубаво! Вие осуетявате Божията заповед за да спазите своето предание!
\par 10 Защото Мойсей рече: "Почитай баща си и майка си" и: "Който злослови баща или майка, непременно да се умъртви".
\par 11 Но вие казвате: Ако рече човек на баща си или на майка си: Това мое имане, с което би могъл да си помогнеш, е курбан, сиреч, подарено Богу, това прави разлика;
\par 12 вие не го оставяте вече да стори нищо за баща си или за майка си.
\par 13 И тъй, осуетявате Божието слово заради вашето предание, което сте предали; и вършите много такива неща подобни на това.
\par 14 И пак повика народа и каза им: Слушайте Ме всички и разбирайте.
\par 15 Няма нищо извън човека, което, като влиза в него, може да го оскверни; но тия неща, които излизат от него, те оскверняват човека.
\par 16 [Ако има някой уши да слуша, нека слуша].
\par 17 И като остави народа и влезе вкъщи учениците Му Го попитаха за притчата.
\par 18 И каза им: Нима сте и вие тъй неразсъдливи? Не разбирате ли, че нищо, което влиза в човека отвън, не може да го оскверни?
\par 19 Защото не влиза в сърцето му, а в корема, и излиза в захода. (Като каза това, Той направи всички ястия чисти).
\par 20 Каза още: Което излиза от човека, то осквернява човека.
\par 21 Защото отвътре, от сърцето на човеците, излизат зли помисли, блудства, кражби, убийства,
\par 22 прелюбодейства, користолюбие, нечестие, коварство, сладострастие, лукаво око, хулене, гордост, безумство.
\par 23 Всички тия зли неща излизат отвътре и оскверняват човека.
\par 24 И като стана оттам, отиде в тирските и сидонските предели; и влезе в една къща, и не искаше да Го знае никой; но не можа да се укрие.
\par 25 А веднага чу за Него една жена, чиято малка дъщеря имаше нечист дух, и тя дойде та падна пред нозете Му.
\par 26 (Жената бе елинка, родом сирофиникианка) И молеше Му се да изгони беса от дъщеря й.
\par 27 А Исус й рече: Остави да се наситят децата; защото не е прилично да се вземе хляба на децата и да се даде на кученцата.
\par 28 А тя в отговор Му каза: Така Господи; но и кученцата под трапезата ядат от трохите паднали от децата.
\par 29 И рече й: 3а тая дума иди си; бесът излезе от дъщеря ти.
\par 30 И като си отиде у дома, намери детето легнало на постелката, и бесът излязъл.
\par 31 И като излезе пак из тирските предели, дойде през Сидон към галилейското езеро, през сред декаполските предели.
\par 32 И довеждат при Него един глух и заекващ човек, и молят Му се да положи ръка на него.
\par 33 Исус, като го отведе от народа насаме, вложи пръстите Си в ушите му, и, като плюна, докосна се до езика му;
\par 34 и погледна към небето, въздъхна и му каза: Еффата, сиреч, Отвори се.
\par 35 И ушите му се отвориха, и връзката на езика му се развърза, и той говореше чисто.
\par 36 И заръча им никому да не кажат това; но колкото повече им заръчваше, толкова повече те го разгласяваха;
\par 37 защото се чудеха твърде много, и думаха: Всичко върши добре; и глухите прави да чуват, и немите да говорят.

\chapter{8}

\par 1 През ония дни, когато пак се беше сбрало голямо множество, и нямаха що да ядат, повика учениците Си и каза им:
\par 2 Жално Ми е за народа, защото три дни вече седят при Мене, и нямат що да ядат;
\par 3 и ако ги разпусна гладни по домовете им, ще им прималее по пътя; а някои от тях са дошли от далеч.
\par 4 И учениците Му отговориха: Отде ще може някой да насити тия с хляб тук в уединено място?
\par 5 И попита ги: Колко хляба имате? А те рекоха: Седем.
\par 6 И заповяда на народа да насядат на земята; и като взе седемте хляба, благодари, разчупи и даваше на учениците Си за да ги сложат. И сложиха ги пред народа.
\par 7 Имаха и малко рибички; и като ги благослови, заповяда да ги сложат и тях.
\par 8 И ядоха и се наситиха; и дигнаха останалите къшеи, седем кошници.
\par 9 А ония, които ядоха, бяха около четирихиляди души; и разпусна ги.
\par 10 И веднага влезе в ладията с учениците Си и дойде в далманутанските предели.
\par 11 И фарисеите излязоха и почнаха да се препират с Него; и като Го изпитваха, поискаха от Него знамение от небето.
\par 12 А Той въздъхна дълбоко от сърце и рече: Защо тоя род иска знамение? Истина ви казвам: На тоя народ няма да се даде знамение.
\par 13 И остави ги, влезе пак в ладията, и мина на отвъдната страна.
\par 14 Но учениците Му забравиха да вземат хляб, и нямаха със себе си в ладията повече от един хляб.
\par 15 И Той им заръча, казвайки: Внимавайте, пазете се от кваса на фарисеите и от кваса на Ирода.
\par 16 И те разискваха помежду си, думайки: Това е защото нямаме хляб.
\par 17 А Исус, като разбра това; рече им: Защо разисквате загдето нямате хляб? Още ли не разбирате, нито разумявате? Окаменено ли е сърцето ви?
\par 18 Като имате очи, не виждате ли? И като имате уши, не чувате ли? И не помните ли?
\par 19 Когато разчупих петте хляба на петте хиляди души, колко кошове пълни с къшеи дигнахте? Казват му: Дванадесет.
\par 20 И когато седемте - на четирите хиляди души, колко кошници, пълни с къшеи дигнахте? Казват му: Седем.
\par 21 И каза им: Не разбирате ли още?
\par 22 Дохождат във Витсаида. И довеждат при Него един слепец и молят Му се да се докосне до него.
\par 23 И Той хвана слепеца за ръка, изведе го вън от селото, и, като плюна на очите му, положи на него ръце и го попита: Виждаш ли нещо?
\par 24 И той, като подигна очи, каза: Виждам човеците; защото виждам неща като дървета, които ходят.
\par 25 После пак положи ръце на очите му; и той втренчи очите си, оздравя, и виждаше всичко ясно.
\par 26 И изпрати го у дома му, и каза: Не влизай в селото, [нито казвай това някому в селото].
\par 27 И излезе Исус с учениците Си по селата на Кесария Филипова; и по пътя попита учениците Си, като им каза: Според както казват хората: Кой съм Аз?
\par 28 А те в отговор Му казаха: Йоан Кръстител; други - Илия; а трети - един от пророците.
\par 29 Тогава ги попита: Но според както вие казвате: Кой съм? Петър в отговор Му каза: Ти си Христос.
\par 30 И заръча им никому да не казват за Него.
\par 31 И почна да ги учи, как Човешкият Син трябва много да пострада, и да бъде отхвърлен от старейшините, главните свещеници, книжниците, и да бъде убит, и след три дни да възкръсне.
\par 32 И явно говореше тая дума. А Петър го взе настрана и почна да Го мъмри.
\par 33 А Той, като се обърна и погледна учениците Си, смъмра Петра, като каза: Махни се, Сатано, и иди зад Мене, защото не мислиш за Божиите неща, но за човешките неща.
\par 34 И повика народа заедно с учениците Си и рече им: Ако иска някой да дойде след Мене, нека се отрече от себе си, и така нека Ме следва.
\par 35 Защото който иска да спаси живота(Или душата; така и до края на главата.) си, ще го изгуби; а който изгуби живота си заради Мене и за благовестието, ще го спаси.
\par 36 Понеже какво се ползува човек като спечели целия свят, а изгуби живота си?
\par 37 Защото какво би дал човек в замяна на живота си?
\par 38 Защото ако се срамува някой поради Мене и поради думите Ми в тоя блуден и грешен род, то и Човешкият Син ще се срамува от него, когато дойде в славата на Отца Си със святите ангели.

\chapter{9}

\par 1 И рече им: Истина ви казвам: Има някои, от тук стоящите, които никак няма да вкусят смърт, докле не видят Божието царство дошло в сила.
\par 2 И след шест дни Исус взема Петра, Якова и Йоана, и завежда само тях на една висока планина насаме; и преобрази се пред тях.
\par 3 Дрехите му станаха бляскави, твърде бели, каквито никой белач на земята не може така да избели.
\par 4 И яви им се Илия с Мойсея, които се разговаряха с Исуса.
\par 5 А Петър проговори, казвайки на Исуса: Учителю, добре е да сме тука; и нека направим три шатри, за Тебе една, за Мойсея една и една за Илия;
\par 6 защото не знаеше какво да отговори, понеже почнаха да се плашат много.
\par 7 И яви се облак и ги засени; и глас дойде из облака, който каза: Този е Моят възлюблен Син; Него слушайте.
\par 8 И внезапно, като се озърнаха, не видяха вече никого при себе си, освен Исуса.
\par 9 И като слизаха от планината, заръча им да не казват никому това що бяха видели, освен когато Човешкият Син бъде възкресен от мъртвите.
\par 10 И те пазеха тая поръчка, като разискваха помежду си що значи да възкръсне от мъртвите.
\par 11 И попитаха Го, казвайки: Защо думат книжниците, че трябва първо Илия да дойде?
\par 12 А Той им каза: Наистина, Илия първо ще дойде и ще възстанови всичко. И как е писано за Човешкия Син? - писано е, че трябва да пострада много и да бъде унизен.
\par 13 Но казвам ви, че Илия е вече дошъл, и те постъпиха с него както си искаха, според както си искаха, според както е писано за него.
\par 14 И когато дойдоха при учениците, видяха около тях едно голямо множество, и книжници, които се препираха с тях.
\par 15 И веднага като Го видя цялото множество, смая се, стекоха се и Го поздравяваха.
\par 16 И Той ги попита: 3а какво се препирате с тях?
\par 17 И един от народа Му отговори: Учителю, доведох при Тебе сина си, който има ням дух.
\par 18 И гдето и да го прехване, тръшка го; и той се запеня, скърца със зъби, и се вцепенява; и говорих на Твоите ученици да изгонят беса, но не можаха.
\par 19 А Той в отговор им каза: О роде невярващ, докога ще бъда с вас? Докога ще ви търпя? Доведете го при Мене.
\par 20 И доведоха го при Него. И като го видя хванатият от бяс, веднага духът го сгърчи; и той падна на земята и се валяше запенен.
\par 21 И попита Исус баща му: Колко време има откак му е станало това? А той каза: От детинство.
\par 22 И много пъти го е хвърлял и в огън, и във вода, за да го погуби; но ако можеш стори нещо, смили се за нас и помогни ни.
\par 23 А Исус му рече: Ако можеш повярва! Всичко е възможно за този, който вярва.
\par 24 Веднага бащата на детето извика, казвайки: Вярвам, [Господи]! Помогни на моето неверие.
\par 25 А Исус, като видя, че се стича народ, смъмра нечистия дух, казвайки му: Душе неми и глухи, Аз ти заповядвам: излез от него, и да не влезеш вече в него.
\par 26 И духът, като изпищя и го сгърчи силно, излезе; и детето стана като мъртво, така щото болшинството думаха, че е умряло.
\par 27 Но Исус го хвана за ръката и го дигна; и то стана.
\par 28 И когато влезе в къщи, учениците Му Го попитаха насаме: Защо ние не можахме да го изгоним?
\par 29 И каза им: Тоя род с нищо не може да излезе, освен с молитва [и пост].
\par 30 И като излязоха оттам, минаваха през Галилея; и Той искаше никой да не узнае това.
\par 31 Защото учеше учениците Си, като им казваше: Човешкият Син ще бъде предаден в ръцете на човеци, и ще Го убият; и след като Го убият, подир три дни ще възкръсне.
\par 32 Но те не разбраха думата, и бояха се да Го попитат.
\par 33 И дойдоха в Капернаум; и когато влезе вкъщи попита ги: Какво разисквахте из пътя?
\par 34 А те мълчаха, защото из пътя бяха се препирали помежду си кой е по-голям?
\par 35 И като седна, повика дванадесетте и каза им: Който иска да бъде пръв, ще бъде от всички последен и на всички служител.
\par 36 Тогава взе едно детенце и го постави посред тях; и като го прегърна, рече им:
\par 37 Който приеме едно от тези дечица в Мое име, и Мене приема; и който приеме Мене, приема не Мене, но Този, Който Ме е пратил.
\par 38 Йоан Му каза: Учителю, видяхме един човек да изгонва бесове в Твое име; и му запретихме, защото не следваше нас.
\par 39 А Исус рече: Недейте му запрещава; защото няма никой, който да извърши велико дело в Мое име, и да може скоро след това да Ме злослови.
\par 40 Понеже оня, който не е против нас, е откъм нас.
\par 41 Защото който ви напои с чаша вода, понеже сте Христови, истина ви казвам: Той никак няма да изгуби наградата си.
\par 42 А който съблазни едно от тия скромните, които вярват в Мене, за него би било по-добре да се окачи голям воденичен камък на врата му и да бъде хвърлен в морето.
\par 43 И ако те съблазни ръката ти, отсечи я; по-добре е за тебе да влезеш в живота недъгав, отколкото да имаш двете си ръце и да отидеш в пъкъла, в неугасимия огън,
\par 44 [дето "червеят им не умира и огънят не угасва"].
\par 45 И ако ногата ти те съблазни, отсечи я; по-добре е за тебе да влезеш в живота куц, отколкото да имаш двете си нозе и да бъдеш хвърлен в пъкъла,
\par 46 [дето "червеят им не умира и огънят не угасва"].
\par 47 И ако окото ти те съблазни, извади го; по-добре е за тебе да влезеш в Божието царство с едно око, отколкото да имаш двете си очи и да бъдеш хвърлен в пъкъла,
\par 48 дето "червеят им не умира, и огънят не угасва".
\par 49 Защото всеки ще се осоли с огън, [и всяка жертва ще се осоли със сол].
\par 50 Добро нещо е солта; но ако солта стане безсолна, с какво ще я подправите? Имайте сол в себе си, и мир имайте помежду си.

\chapter{10}

\par 1 И стана оттам и дойде в юдейските предели, и местата отвъд Йордан; и народ пак се стече при Него; и по обичая Си Той пак ги поучаваше.
\par 2 И някои фарисеи се приближиха и Го попитаха, за да Го изпитат: Позволено ли е на мъж да напусне жена си?
\par 3 В отговор Той им каза: Какво ви е заповядал Моисей?
\par 4 А те рекоха: Моисей е позволил да напише мъжът разводно писмо и да я напусне.
\par 5 А Исус им рече: Поради вашето коравосърдечие ви е написал тая заповед;
\par 6 обаче в началото на създанието, Бог ги е направил мъж и жена.
\par 7 Затова ще остави човек баща си и майка си и ще се привърже към жена си,
\par 8 и двамата ще бъдат една плът; така че не с вече двама, а една плът.
\par 9 И тъй, онова, което Бог е съчетал, човек да го не разлъчва.
\par 10 И вкъщи учениците пак Го попитаха за това.
\par 11 И Той им каза: Който си напусне жената, и се ожени за друга, прелюбодействува против нея.
\par 12 И ако тя напусне мъжа си, и се омъжи за друг, тя прелюбодействува.
\par 13 Тогава доведоха при него дечица, за да се докосне до тях; а учениците ги смъмриха.
\par 14 Но Исус като видя това, възнегодува и рече им: Оставете дечицата да дойдат при Мене: не ги възпирайте, защото на такива е Божието царство.
\par 15 Истина ви казвам: Който не приеме Божието царство като детенце, той никак няма да влезе в него.
\par 16 И прегърна ги и ги благослови, като положи ръцете Си на тях.
\par 17 И когато излизаше на път, някой се завтече та коленичи пред Него и Го попита: Учителю благи, какво да сторя за да наследя вечен живот?
\par 18 А Исус му рече; Защо Ме наричаш благ? Никой не е благ освен един Бог.
\par 19 Знаеш заповедите: "Не убивай; Не прелюбодейстувай; Не кради; Не лъжесвидетелствувай; Не увреждай; Почитай баща си и майка си."
\par 20 А той Му рече: Учителю, всичко това съм упазил от младостта си.
\par 21 А Исус, като го погледна, възлюби го, и му рече: Едно ти не достига; иди продай все що имаш и дай на сиромасите, и ще имаш съкровище на небето: и дойди и Ме следвай.
\par 22 Но лицето му посърна от тая дума, и той си отиде наскърбен, защото беше човек с много имот.
\par 23 А Исус се озърна и каза на учениците: Колко мъчно ще влязат в Божието царство ония, които имат богатство!
\par 24 А учениците се смайваха за Неговите думи. Но в отговор Исус пак им каза: Чада, колко е мъчно да влязат в Божието царство ония, които уповават на богатството.
\par 25 По-лесно е камилата да мине през иглени уши, отколкото богат да влезе в Божието царство.
\par 26 А те се чрезмерно зачудиха и Му казаха: Тогава кой може да се спаси?
\par 27 Исус ги погледна и рече: 3а човеците това е невъзможно, но не и за Бога; защото за Бога всичко е възможно.
\par 28 Петър почна да Му казва: Ето, ние оставихме всичко и Те последвахме.
\par 29 Исус каза: Истина ви казвам: Няма човек, който да е оставил къща, или братя, или сестри, или майка или баща, или чада, или ниви, заради Мене и заради благовестието,
\par 30 и да не получи стократно сега, в настоящето време, къщи и братя, и сестри, и майки, и чада, и ниви, заедно с гонения, а в идещия свят(Или: Век.) вечен живот.
\par 31 Обаче мнозина първи ще бъдат последни, а последните първи.
\par 32 А когато бяха на път, отивайки за Ерусалим, Исус вървеше пред тях; а те се удивяваха, и ония, които вървяха подире, бяха обзети от страх. И като събра дванадесетте, почна да им казва това, което щеше да Го сполети, като рече:
\par 33 Ето, ние възлизаме за Ерусалим, и Човешкият Син ще бъде предаден на главните свещеници и на книжниците; и те, като Го осъдят на смърт, ще Го предадат на езичниците;
\par 34 и ще Му се поругаят, ще Го заплюват, ще Го бият и ще Го убият; а след три дни, ще възкръсне.
\par 35 Тогава се приближават при него Яков и Йоан, Заведеевите синове, и Му казват: Учителю, желаем да ни сториш каквото и да поискаме от Тебе.
\par 36 А Той им казва: Какво желаете да ви сторя?
\par 37 Те му рекоха: Дай ни да седнем, един отдясно Ти, а един отляво Ти в Твоята слава.
\par 38 А Исус им рече: Не знаете какво искате. Можете ли да пиете чашата, която Аз пия, или да се кръстите с кръщението, с което Аз се кръщавам?
\par 39 Те Му рекоха: Можем. А Исус им каза: Чашата, която Аз пия, ще пиете, и с кръщението, с което Аз се кръщавам, ще се кръстите;
\par 40 но да седнете отдясно Ми или отляво Ми, не е Мое да дам, а ще се даде на ония, за които е било приготвено.
\par 41 А десетимата, като чуха това, захванаха да негодуват против Якова и Йоана.
\par 42 Но Исус ги повика и им каза: Вие знаете, че ония, които се считат за управители на народите, господаруват над тях, и големците им властвуват над тях.
\par 43 Но между вас не е така; а който иска да стане големец между вас, ще ви бъде служител;
\par 44 и който иска да бъде пръв между вас, ще бъде слуга на всичките.
\par 45 Защото наистина Човешкият Син не дойде да Му служат, но да служи, и да даде живота Си откуп за мнозина.
\par 46 Дохождат в Ерихон; и когато излизаше из Ерихон с учениците Си и едно голямо множество, Тимеевият син Вартимей, един сляп просяк, седеше край пътя.
\par 47 И като чу, че бил Исус Назарянинът, почна да вика, казвайки: Исусе, сине Давидов, смили се за мене!
\par 48 И мнозина го мъмреха, за да млъкне; но той още повече викаше: Сине Давидов, смили се за мене!
\par 49 И тъй, Исус се спря и рече: Повикайте го. Викат слепеца и му казват: Дерзай, стани, вика те.
\par 50 И той си хвърли дрехата и скокна и дойде при Исуса.
\par 51 И проговори Исус и му каза: Какво искаш да ти сторя? Слепецът му рече: Учителю, да прогледам.
\par 52 А Исус му рече: Иди си, твоята вяра те изцели. И той веднага прогледа, и тръгна подир Него по пътя.

\chapter{11}

\par 1 И когато се приближаваха към Ерусалим, до Витфагия и Витания при Елеонския хълм, изпраща двама от учениците Си и казва им:
\par 2 Идете в селото, което е насреща ви, и щом влезете в него, ще намерите вързано осле, което никой човек не е още възсядал; отвържете го и го докарайте.
\par 3 И ако някой ви рече: Защо правите това? Кажете: На Господа трябва; и веднага ще го прати тук.
\par 4 И тъй, те отидоха и намериха едно осле, вързано до вратата, вън край пътя, и отвързват го.
\par 5 И някои от стоящите там им казаха: Какво правите та отвързвате ослето?
\par 6 А те им казаха, както бе заръчал Исус; и оставиха ги.
\par 7 И докарват ослето при Исуса, и намятат на него дрехите си; и Той го възседна.
\par 8 И мнозина напостлаха дрехите си по пътя, а други - клони, като ги сечеха от нивите.
\par 9 И тия, които вървяха отпред и тия, които идеха изподире, викаха: Осанна! Благословен, Който иде в Господното име!
\par 10 Благословено градущето царство на баща ни Давида [което иде в Господното име]; осанна във висините!
\par 11 И влезе Исус в Ерусалим, в храма, и, като разгледа всичко, понеже вече се бе свечерило, отиде във Витания с дванадесетте.
\par 12 А на сутринта, когато излязоха от Витания, Той огладня.
\par 13 И като видя отдалеч една разлистила се смоковница, дойде дано би намерил нещо на нея; но като дойде до нея, не намери нищо, само едни листа, защото не беше време за смокини.
\par 14 И Той проговори, думайки й: Отсега нататък никой да не яде плод от тебе до века. И учениците Му чуха това.
\par 15 И дойдоха в Ерусалим; Исус като влезе в храма, почна да изпъжда ония, които продаваха, и ония, които купуваха в храма, и прекатури масите на среброменителите и столовете на ония, които продаваха гълъбите.
\par 16 И не позволяваше да пренесе някой какъвто и да било съд през храма.
\par 17 И поучаваше, казвайки им: Не е ли писано, "Домът ми ще се нарече молитвен дом за всичките народи"? а вие го направихте разбойнически вертеп".
\par 18 И главните свещеници и книжниците чуха това; и търсеха начин как да Го погубят, защото се бояха от Него, понеже целият народ се чудеше на учението Му.
\par 19 А всякога на мръкване Той излизаше вън от града.
\par 20 И тъй, като минаваха сутринта, видяха смоковницата изсъхнала от корен.
\par 21 И Петър си спомни и Му каза: Учителю, виж, смоковницата, която ти прокле, изсъхнала.
\par 22 А Исус в отговор им каза: Имайте вяра в Бога.
\par 23 Истина ви казвам: Който рече на тая планина: Дигни се и хвърли се в морето, и не се усъмни в сърцето си, но повярва, че онова, което казва, се сбъдва, ще му стане.
\par 24 Затова ви казвам: Всичко каквото поискате в молитва вярвайте, че сте го получили, и ще ви се сбъдне.
\par 25 И когато се изправяте на молитва, прощавайте, ако имате нещо против някого, за да прости и Отец ви, Който е на небесата, вашите прегрешения.
\par 26 [Но ако вие не прощавате, то нито Отец ви, Който е на небесата, ще ви прости съгрешенията].
\par 27 И дохождат пак в Ерусалим; и когато ходеше в храма, идват при Него главните свещеници, книжниците и старейшините, и Му казват:
\par 28 С каква власт правиш това? Или кой Ти е дал тая власт да правиш това?
\par 29 Исус им рече: Ще ви задам и Аз един въпрос; отговорете Ми, и Аз ще ви кажа с каква власт правя това.
\par 30 Йоановото кръщение от небето ли беше, или от човеците? Отговорете ми.
\par 31 И те разискваха помежду си, думайки: Ако речем - От небето, ще каже - Тогава защо не го повярвахте?
\par 32 Но ако речем: От човеците, - бояха се от народа; защото всички искренно считаха Йоана за пророк.
\par 33 И тъй, в отговор на Исуса казаха: Не знаем. Исус им рече: Нито Аз ви казвам с каква власт правя това.

\chapter{12}

\par 1 И почна да им говори с притчи: Един човек насади лозе, огради го с плет, изкопа лин, и съгради кула, и даде го под наем на земеделците, и отиде в чужбина.
\par 2 И във времето на плода изпрати един слуга до земеделците да прибере от земеделците от плода на лозето.
\par 3 А те го хванаха, биха го, и го отпратиха празен.
\par 4 Пак изпрати до тях друг слуга; и нему счупиха главата, и безсрамно го оскърбиха.
\par 5 Изпрати и друг, когото убиха; и мнозина други, от които едни биха, а други убиха.
\par 6 Още един имаше той, един възлюбен син; него изпрати последен до тях, като думаше, Ще почетат сина ми.
\par 7 А тия земеделци рекоха помежду си: Тоя е наследникът; елате да го убием, и наследството ще бъде наше.
\par 8 И тъй, хванаха го и го убиха, хвърлиха го вън от лозето.
\par 9 Какво, прочее, ще стори стопанинът на лозето? Ще дойде и ще погуби тия земеделци, а лозето ще даде на други.
\par 10 Не сте ли прочели нито това писание: - "Камъкът, който отхвърлиха зидарите, той стана глава на ъгъла,
\par 11 от Господа е това, и чудно е в нашите очи?"
\par 12 И първенците искаха да Го хванат (но се побояха от народа), понеже разбраха, че за тях изрече тая притча. И оставиха Го и си отидоха.
\par 13 Тогава пращат при Него някои от фарисеите и иродианите за да Го впримчат в говоренето Му.
\par 14 И те, като дойдоха, казаха Му: Учителю, знаем, че си искрен и не Те е грижа от никого; защото не гледаш на лицето на човеците, но учиш Божия път според истината. Право ли е да даваме данък на Кесаря, или не?
\par 15 Да даваме ли, или да не даваме? А Той, като разбра лицемерието им, рече им: Защо Ме изпитвате? Донесете ми един пеняз да го видя.
\par 16 И те Му донесоха. Тогава им казва: Чий е тоя образ и надпис? А те му рекоха: Кесарев.
\par 17 Исус им рече: Отдавайте Кесаревото на Кесаря, и Божието на Бога. И те много се зачудиха на Него.
\par 18 След това дохождат при Него садукеи, които казват, че няма възкресение; и питат Го, казвайки:
\par 19 Учителю, Мойсей ни написа, че ако някому умре брат и остави жена, а чадо не остави, то брат му да вземе жена му и да въздигне потомък на брата си.
\par 20 Прочее, имаше седем братя; и първият взе жена; и когато умря не остави потомък.
\par 21 Взе я и вторият, умря, и не остави потомък; също и третият.
\par 22 И седмината не оставиха потомство. А подир всички умря и жената.
\par 23 Във възкресението, на кого от тях ще бъде жена? Защото и седмината я имаха за жена.
\par 24 Исус им рече: Не затова ли се заблуждавате, понеже не знаете писанията, нито Божията сила?
\par 25 Защото когато възкръснат от мъртвите, нито се женят, нито се омъжват, но са като ангели на небесата.
\par 26 А за мъртвите, че биват възкресени, не сте ли чели в книгата на Мойсея, на мястото, при къпината, как Бог му говорй, казвайки: "Аз съм Бог Авраамов, Бог Исааков, и Бог Яковов"?
\par 27 Той не е Бог на мъртвите, а на живите. Вие много се заблуждавате.
\par 28 А един от книжниците, който дойде и ги чу, когато се препираха, като видя, че им отговори добре, пита Го: Коя заповед е първа от всички?
\par 29 Исус отговори: Първата е: "Слушай, Израилю; Господ нашият Бог е един Господ;
\par 30 и да възлюбиш Господа твоя Бог от цялото си сърце, с цялата си душа, с всичкия си ум и с всичката си сила".
\par 31 А ето втората [подобна на нея] заповед: "Да възлюбиш ближния си като себе си". Друга заповед по-голяма от тия няма.
\par 32 И книжникът Му рече: Превъзходно, Учителю! Ти право каза, че Бог е един; и няма друг освен Него;
\par 33 и да Го люби човек от все сърце, с всичкия си разум, и с всичката си сила, и да люби ближния си като себе си, това е много повече от всичките всеизгаряния и жертви.
\par 34 Исус, като видя, че отговори разумно, рече му: Не си далеч от Божието царство. И никой вече не дръзна да Му задава въпроси.
\par 35 И когато поучаваше в храма, Исус проговори казвайки: Как думат книжниците, че Христос е Давидов Син?
\par 36 Сам Давид каза чрез Святия Дух: - "Рече Господ на Моя Господ: Седи отдясно Ми, докле положа враговете Ти за Твое подножие".
\par 37 Сам Давид го нарича Господ; тогава как да е негов син? И голямото множество Го слушаше с удоволствие.
\par 38 И в поучението Си казваше: Пазете се от книжниците, които обичат да ходят пременени, и да приемат поздравите по пазарите,
\par 39 и първите столове по синагогите, и първите места при угощенията;
\par 40 тия, които изпояждат домовете на вдовиците, даже когато за показ принасят дълги молитви. Тия ще приемат по-голямо осъждение.
\par 41 И като седна Исус срещу съкровищницата, гледаше как народът пускаше пари в съкровищницата; и мнозина богаташи пускаха много.
\par 42 А една бедна вдовица дойде и пусна две лепти, сиреч, един кодрант.
\par 43 И повика учениците Си и каза им: Истина ви казвам, тая бедна вдовица пусна повече от всичките, които пускат в съкровищницата;
\par 44 защото те всички пускат от излишъка си, а тя от немотията си пусна всичко що имаше, целия си имот.

\chapter{13}

\par 1 Когато излизаше от храма, един от Неговите ученици Му каза: Учителю, виж, какви камъни и какви здания!
\par 2 А Исус му рече: Виждаш ли тия големи здания! Няма да остане тук камък на камък, който да не се срине.
\par 3 И когато седеше на Елеонския хълм срещу храма, Петър и Яков и Йоан и Андрей Го попитаха насаме:
\par 4 Кажи ни, Кога ще бъде това? И какъв ще бъде знакът, когато всичко това предстои да се изпълни?
\par 5 А Исус почна да им казва: Пазете се да ви не подмами някой.
\par 6 Мнозина ще дойдат в Мое име и ще рекат: Аз съм Христос, и ще подмамят мнозина.
\par 7 А когато чуете за войни и за военни слухове, недейте се смущава; това трябва да стане; но туй не е още свършекът.
\par 8 Защото народ ще се повдигне против народ, и царство против царство, ще има трусове на разни места, ще има и глад; а тия неща са само начало на страданията.
\par 9 А вие внимавайте на себе си, защото ще ви предадат на събори, и в синагоги ще ви бият; и пред управители и царе ще застанете заради Мене, за да свидетелствувате на тях.
\par 10 Обаче, трябва първо да се проповядва благовестието на всичките народи.
\par 11 А когато ви поведат, за да ви предадат, не се безпокойте предварително какво ще говорите; но каквото ви се даде в оня час, това говорете! Защото не сте вие, които говорите, а Святият Дух.
\par 12 Брат, брата ще предаде на смърт, и баща чадо; и чада ще се повдигнат против родители и ще ги умъртвят.
\par 13 И ще бъдете мразени от всички заради Моето име: но който устои до край, той ще бъде спасен.
\par 14 И когато видите мерзостта, която докарва запустение, [за която говори пророк Даниил], стояща там гдето не подобава, (който чете нека разбира), тогава ония, които с в Юдея, нека бягат по планините;
\par 15 и който е на къщния покрив да не слиза в къщата си, нито да влиза да вземе нещо от нея;
\par 16 и който е на нива да не се връща назад да вземе дрехата си.
\par 17 А горко на непразните и на кърмачките през ония дни!
\par 18 При това, молете се да не стане това(Бягането ви.) зиме;
\par 19 защото през ония дни ще има скръб небивала до сега от началото на създанието, което Бог е създал, нито ще има такава.
\par 20 И ако Господ не съкратеше ония дни, не би се избавила ни една твар; но заради избраните, които Той избра, съкратил е дните.
\par 21 Тогава ако ви каже някой: Ето, тук е Христос, или: Ето там! Не вярвайте;
\par 22 защото ще се появят лъжехристи и лъжепророци, които ще покажат знания и чудеса, за да подмамят, ако е възможно и избраните.
\par 23 А вие внимавайте; ето, предсказах ви всичко.
\par 24 Но през ония дни, подир оная скръб, слънцето ще потъмнее, луната няма да даде светлината си,
\par 25 звездите ще падат от небето, и силите, които са на небето ще се разклатят.
\par 26 Тогава ще видят Човешкия Син идещ на облаци с голяма сила и слава.
\par 27 И тогава ще изпрати ангелите, и ще съберат избраните Му от четирите ветрища, от края на земята до края на небето.
\par 28 А научете притчата от смоковницата: Когато клоните й вече омекнат и развият листа, знаете, че е близо лятото;
\par 29 също така и вие, когато видите, че става това, да знаете, че Той е близо при вратата.
\par 30 Истина ви казвам: Това поколение няма да премине, докле не се сбъдне всичко това.
\par 31 Небето и земята ще преминат, но Моите думи няма да преминат.
\par 32 А за оня ден или час никой не знае, нито ангелите на небесата, нито Синът, а само Отец.
\par 33 Внимавайте, бдете, и молете се; защото не знаете кога ще настане времето.
\par 34 Понеже това ще бъде както кога човек, живущ в чужбина, като остави къщата си, и даде власт на слугите си, всекиму особената му работа, заповяда и на вратаря да бди.
\par 35 Бдете, прочее, (защото не знаете кога и ще дойде господарят на къщата - вечерта ли, или в среднощ, или когато пеят петлите, или заранта),
\par 36 да не би, като дойде неочаквано, да ви намери заспали.
\par 37 А каквото казвам на вас, на всички го казвам: Бдете.

\chapter{14}

\par 1 А след два дни щеше да бъде пасхата и празника на безквасните хлябове; и главните свещеници и книжници търсеха случай да Го уловят с хитрост и да Го умъртвят.
\par 2 Защото думаха: Да не стане на празника, за да се не подигне вълнение между народа.
\par 3 И когато Той беше във Витания, и седеше на трапезата в къщата на Симона прокажения, дойде една жена, която имаше алавастрен съд с миро от чист и скъпоценен нард; и като счупи съда, изля мирото на главата Му.
\par 4 А имаше някои, които, негодуващи, думаха помежду си: Защо така се прахоса мирото?
\par 5 защото това миро можеше да се продаде за повече от триста пеняза, и сумата да се раздаде на сиромасите. И роптаеха против нея.
\par 6 Но Исус рече: Оставете я; защо й досаждате? Тя извърши добро дело на Мене.
\par 7 Защото сиромасите всякога се намират между вас, и когато щете можете да им сторите добро; но Аз не се намирам всякога между вас.
\par 8 Тя извърши това, което можеше; предвари да помаже тялото Ми за погребение.
\par 9 Истина ви казвам: Гдето и да се проповядва благовестието по целия свят, ще се разказва за неин спомен и за това, което тя стори.
\par 10 Тогава Юда Искариотски, оня, който бе един от дванадесетте, отиде при главните свещеници за да им Го предаде.
\par 11 Те, като чуха, зарадваха се, и се обещаха да му дадат пари. И той търсеше сгоден случай да Го предаде.
\par 12 А на първия ден на празника на безквасните хлябове, когато колеха жертви за пасхата, учениците Му казаха: Где искаш да отидем и приготвим за да ядеш пасхата?
\par 13 И Той изпраща двама от учениците Си и казва им: Идете в града; и ще ви срещне човек, който носи стомна с вода; вървете подир него.
\par 14 И дето влезе, речете на стопанина тая къща: Учителят казва: Где е приготвената за Мене приемна стая, гдето ще ям пасхата с учениците Си?
\par 15 И той ще ви посочи една голяма горна стая, постлана и готова; там ни пригответе.
\par 16 И тъй, учениците излязоха и дойдоха в града; и намериха както им беше казал; и приготвиха пасхата.
\par 17 И като се свечери, Той дохожда с дванадесетте.
\par 18 И когато седяха на трапезата и ядяха, Исус рече: Истина ви казвам: Един от вас, които яде с Мене, ще Ме предаде.
\par 19 Те почнаха да скърбят и да Му казват един по един: Да не съм аз?
\par 20 А той им рече: Един от дванадесетте е, който топи заедно с Мене в блюдото.
\par 21 Защото Човешкият Син отива, както е писано за Него; но горко на този човек, чрез когото Човешкият Син се предава! Добре би било за този човек, ако не бе се родил.
\par 22 И когато ядяха, Исус взе хляб, и като благослови, разчупи, даде им, и рече: Вземете, [яжте]; това е Моето тяло.
\par 23 Взе и чашата, благослови, и даде им; и те всички пиха от нея.
\par 24 И рече им: Това е Моята кръв на [новия] завет, която се пролива за мнозина.
\par 25 Истина ви казвам, че няма вече да пия от плода на лозата до оня ден, когато ще го пия нов в Божието царство.
\par 26 И като изпяха химн, излязоха на Елеонския хълм.
\par 27 И Исус им каза: Вие всичките ще се съблазните [в Мене тая нощ]; защото е писано: "Ще поразя пастиря, и овците ще се разпръснат".
\par 28 А подир възкресението Си ще ви предваря в Галилея.
\par 29 А Петър Му рече: Ако и всички да се съблазнят, аз, обаче не.
\par 30 Исус му каза: Истина ти казвам, че днес, тая нощ, преди да пее петелът дваж, ти три пъти ще се отречеш от Мене.
\par 31 А той твърде разпалено казваше: Ако стане нужда и да умра с Тебе, пак няма да се отрека от Тебе. Същото казаха и другите.
\par 32 Дохождат на едно място, наречено Гетсимания; и Той каза на учениците Си: Седете тука докле се помоля.
\par 33 И взе със Себе Си Петра, Якова и Йоана, и захвана да се ужасява и да тъгува.
\par 34 И казва им: Душата Ми е прескръбна до смърт; постойте тука и бдете.
\par 35 И като отиде малко напред, падна на земята; и молеше се, ако е възможно, да Го отмине тоя час, казвайки:
\par 36 Авва, Отче, за Тебе всичко е възможно; отмини Ме с тая чаша; не, обаче, както Аз искам, но както Ти искаш.
\par 37 И идва, намира ги заспали; и казва на Петра: Симоне, спиш ли? Не можа ли един час да постоиш буден?
\par 38 Бдете и молете се, за да не паднете в изкушение; духът е бодър, а тялото немощно.
\par 39 И пак отиде и се помоли, като каза същите думи.
\par 40 И като дойде пак, намери ги заспали, защото очите им бяха натегнали; и не знаеха що да Му отговорят.
\par 41 И трети път дохожда и им казва: Още ли спите и почивате? Доста е! Дойде часът! Ето, Човешкият Син се предава в ръцете на грешниците.
\par 42 Станете да вървим; ето, приближи се оня, който Ме предава.
\par 43 И веднага, докато Той говореше, дохожда Юда, един от дванадесетте, и с него едно множество с ножове и сопи, изпратени от главните свещеници, книжниците и старейшините.
\par 44 А оня, който Го предаваше, беше им дал знак, казвайки: Когото целуна, Той е; хванете Го и Го заведете, като Го пазите здраво.
\par 45 И когато дойде, веднага се приближи до Него и каза: Учителю! И целуваше Го.
\par 46 И те туриха ръце на Него и Го хванаха.
\par 47 А един от стоящите там измъкна ножа си и удари слугата на първосвещеника и му отсече ухото.
\par 48 Исус проговори и рече им: Като срещу разбойник ли сте излезли с ножове и сопи да Ме уловите?
\par 49 Всеки ден бях при вас и поучавах в храма, и не Ме хванахте; но това стана, за да се сбъднат писанията.
\par 50 Тогава всички Го оставиха и се разбягаха.
\par 51 И някой си момък следваше подир Него, обвит с плащаница по голо; и те го хванаха.
\par 52 А той, като остави плащаницата, избяга гол.
\par 53 И заведоха Исуса при първосвещеника, при когото се събират и всичките главни свещеници, и старейшините, и книжниците.
\par 54 А Петър Го беше следвал издалеч до вътре в двора на първосвещеника и седеше заедно със служителите и грееше се на пламъка.
\par 55 И главните свещеници и целият синедрион търсеха свидетелство против Исуса, за да Го умъртвят, но не намериха.
\par 56 Защото мнозина лъжесвидетелствуваха против Него, но свидетелствата им не бяха съгласни.
\par 57 Сетне някои станаха и лъжесвидетелствуваха против Него, като казаха:
\par 58 Ние Го чухме да казва: Аз ще разруша тоя ръкотворен храм, и за три дни ще съградя друг неръкотворен.
\par 59 Но и така свидетелствата им не бяха съгласни.
\par 60 Тогава първосвещеникът се изправи насред и попита Исуса, казвайки: Нищо ли не отговаряш? Какво свидетелствуват тия против Тебе?
\par 61 А Той мълчеше и не отговори нищо. Първосвещеникът пак Го попита, като Му каза: Ти ли си Христос, Син на Благословения?
\par 62 А Исус рече: Аз съм; и ще видите Човешкия Син седящ отдясно на силата и идещ с небесните облаци.
\par 63 Тогава първосвещеникът раздра дрехите си и каза: Каква нужда имаме вече от свидетели?
\par 64 Чухте богохулството; как ви се вижда? И те всички Го осъдиха, че се изложи на смъртно наказание.
\par 65 И някои почнаха да Го заплюват, да Му закриват лицето, да Го блъскат и да Му казват: Познай. И служителите, като Го хванаха, удряха Го с плесници.
\par 66 И когато Петър беше долу на двора, дохожда една от слугините на първосвещеника;
\par 67 и като видя Петра че се грее, взря се в него и каза: И ти беше с Назарянина, с Исуса.
\par 68 А той се отрече, казвайки: Нито зная, нито разбирам що говориш. И излезе вън на предверието; и петела изпя.
\par 69 Но слугинята го видя и пак почна да казва на стоящите там: Тоя е от тях.
\par 70 А той пак се отрече. След малко, стоящите там пак казаха на Петра: Наистина от тях си, защото си галилеянин, [и говорът ти съответствува].
\par 71 А той започна да проклина и да се кълне: Не познавам Този човек за Когото говорите.
\par 72 И на часа петелът изпя втори път. И Петър си спомни думата, която Исус му беше рекъл: Преди да пее петелът дваж, три пъти ще се отречеш от Мене. И като размисли за това, заплака.

\chapter{15}

\par 1 И на сутринта главните свещеници със старейшините и книжниците и целият синедрион, незабавно се съвещаха и като вързаха Исуса, заведоха Го и Го предадоха на Пилата.
\par 2 И Пилат Го попита: Ти Юдейският цар ли си? А Той му отговори и рече: Ти казваш.
\par 3 И главните свещеници Го обвиняваха в много неща.
\par 4 А Пилат пак Го попита, казвайки: Нищо ли не отговаряш? Виж за колко неща Те обвиняват!
\par 5 Но Исус нищо вече не отговори, така щото Пилат се чудеше.
\par 6 А на всеки празник той им пущаше по един затворник, когото биха поискали.
\par 7 А в онова време имаше някой си на име Варава, затворен заедно с ония бунтовници, които във време на бунта бяха извършили убийство.
\par 8 И народът се изкачи и почна да иска от Пилата да им направи, каквото имаше обичай да прави.
\par 9 А Пилат в отговор им рече: Искате ли да ви пусна Юдейския цар?
\par 10 (понеже видя, че главните свещеници от завист бяха Го предали).
\par 11 Но главните свещеници подбудиха народа да искат по-добре да им пусне Варава.
\par 12 Пилат пак в отговор им рече: Тогава какво да направя с Този, Когото наричате Юдейски цар?
\par 13 А те пак изкрещяха: Разпни Го!
\par 14 А Пилат им каза: Че какво зло е сторил? Но те много закрещяха: Разпни го!
\par 15 Тогава Пилат, като искаше да угоди на народа, пусна им Варава, а Исуса би и Го предаде на разпятие.
\par 16 И войниците Го заведоха вътре в двора, тоест, в преторията, и свикаха цялата дружина.
\par 17 И облякоха Му морава мантия, сплетоха и венец от тръни, та го положиха на главата Му.
\par 18 и почнаха да Го поздравяват със: Здравей, царю Юдейски!
\par 19 И удряха Го по главата с тръст, заплюваха Го, и коленичейки, кланяха Му се.
\par 20 И след като Му се поругаха, съблякоха Му моравата мантия и Го облякоха в Неговите дрехи и Го изведоха вън да Го разпнат.
\par 21 И накараха да носи кръста Му някой си Симон киринеец, баща на Александра и Руфа, който минаваше на връщане от нива.
\par 22 И завеждат Исуса на мястото Голгота, което значи лобно място.
\par 23 И подаваха Му вино смесено със смирна, но Той не прие.
\par 24 И като Го разпъват, разделят си дрехите Му, и хвърлят жребие за тях, кой какво да вземе.
\par 25 А беше третият час, когато Го разпнаха.
\par 26 А надписът на обвинението Му бе написан така: Юдейският Цар.
\par 27 И с Него разпнаха двама разбойници, един отдясно Му и един отляво Му.
\par 28 [И се изпълни писанието, което казва: "И с беззаконните се числи"].
\par 29 И минаващите оттам Го хулеха, като клатеха глави и казваха: Уха! Ти, който разоряваш храма, и за три дни го пак съграждаш,
\par 30 спаси Себе Си и слез от кръста.
\par 31 Подобно и главните свещеници с книжниците Го ругаеха помежду си, като казваха: Други е избавил, а пък Себе Си не може да избави!
\par 32 Христос Израилевият цар, нека слезе сега от кръста, за да видим и да повярваме. И разпнатите с Него Го ругаеха.
\par 33 А на шестия час, настана тъмнина по цялата земя, трая до деветия час.
\par 34 И на деветия час Исус извика със силен глас: "Елои, Елои, Лама Савахтани?" което значи: Боже Мой, Боже Мой, защо си Ме оставил?
\par 35 И някои от стоящите там, като чуха казаха: Ето вика Илия.
\par 36 И един се завтече, натопи гъба в оцет, надяна я на тръст, и Му даде да пие, като казваше: Оставете! Да видим дали ще дойде Илия да Го снеме.
\par 37 А Исус, като издаде силен глас, издъхна.
\par 38 И завесата на храма се раздра на две, отгоре до долу.
\par 39 А стотникът, който стоеше срещу Него, като видя, че така [извика и] издъхна, рече: Наистина тоя човек беше Син Божи.
\par 40 Имаше още и жени, които гледаха отдалеч, между които бяха и Мария Магдалина, и Мария майката на малкия Яков и на Иосия, и Саломия;
\par 41 които, когато беше в Галилея, вървяха подир Исуса и Му служеха; имаше и много други жени, които бяха възлезли с Него в Ерусалим.
\par 42 И когато вече се свечери, то, понеже беше приготвителен ден, сиреч, срещу събота,
\par 43 дойде Йосиф от Ариматея, един почтен съветник, който и сам ожидаше Божието царство, и осмели се да влезе при Пилата и да поиска Исусовото тяло.
\par 44 А Пилат се почуди дали е вече умрял, и, като повика стотника, попита го дали е от дълго време мъртъв.
\par 45 И като се научи от стотника, отстъпи тялото на Йосифа.
\par 46 И той купи плащаница, и като го сне, обви го в плащаницата, и положи го в гроб, който бе изсечен в скала, и привали камък върху гробната врата.
\par 47 А Мария Магдалина и Мария Иосиевата майка гледаха где го полагаха.

\chapter{16}

\par 1 А когато се мина съботата, Мария Магдалина, Мария Якововата майка, и Саломия купиха аромати за да дойдат и го помажат.
\par 2 И в първия ден на седмицата дохождат на гроба много рано, когато изгрея слънцето.
\par 3 И думаха помежду си: Кой ще ни отвали камъка от гробната врата? - защото беше твърде голям.(Последните четири думи са преместени от края на 4-тия стих.)
\par 4 А като повдигнаха очи, видяха, че камъкът бе отвален.
\par 5 И като влязоха в гроба, видяха, че един юноша седеше отдясно, облечен в бяла одежда; и много се зачудиха.
\par 6 А той им казва: Недейте се учудва; вие търсите Исуса Назарянина, разпнатия. Той възкръсна; няма Го тука; ето мястото гдето Го положиха.
\par 7 Но идете, кажете на учениците Му и на Петра, че отива преди вас в Галилея; там ще Го видите, както ви каза.
\par 8 И те излязоха и побягнаха от гроба, понеже трепет и ужас бяха ги обзели; и никому не казаха нищо, защото се бояха.
\par 9 И като възкръсна рано в първия ден на седмицата, Исус се яви първо на Мария Магдалина, от която бе изгонил седем беса.
\par 10 Тя отиде и извести на тия, които бяха Го придружавали, и които Го жалееха и плачеха.
\par 11 Но те, като чуха, че бил жив, и че тя Го видяла, не повярваха.
\par 12 Подир това се яви в друг образ на двама от тях, когато вървяха, отивайки в село.
\par 13 И те отидоха и известиха на другите; но нито на тях повярваха.
\par 14 После се яви на самите единадесет ученика, когато бяха на трапезата, и смъмра ги за неверието и жестокосърдечието им, дето не повярваха на тия, които Го бяха видели възкръснал.
\par 15 И рече им: Идете по целия свят и проповядвайте благовестието на всяка твар.
\par 16 Който повярва и се кръсти ще бъде спасен; а който не повярва ще бъде осъден.
\par 17 И тия знамения ще придружават повярвалите: В Мое име бесове ще изгонват; нови езици ще говорят;
\par 18 змии ще хващат; а ако изпият нещо смъртоносно, то никак няма да ги повреди; на болни ще възлагат ръце, и те ще оздравяват.
\par 19 И тъй, след като им говори, Господ Исус се възнесе на небето, и седна отдясно на Бога.
\par 20 А те излязоха и проповядваха навсякъде, като им съдействуваше Господ, и потвърдяваше словото със знаменията, които го придружаваха. Амин.

\end{document}