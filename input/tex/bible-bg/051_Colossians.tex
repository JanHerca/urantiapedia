\begin{document}

\title{Колосяни}


\chapter{1}

\par 1 Павел, с Божията воля апостол Исус Христов, и брат Тимотей,
\par 2 до светите и верни братя в Христа, които са в Колоса: Благодат и мир на вас от Бога, нашия Отец, [и Господа Исуса Христа],
\par 3 Благодарим на Бога, Отец на нашия Господ Исус Христос, (като се молим винаги за вас,
\par 4 понеже чухме за вашата вяра в Христа Исуса и за любовта ви към всички светии),
\par 5 по причина на онова, за което се надяваме, което се пази за вас на небесата, за което сте чули от по-напред в истинското слово на благовестието,
\par 6 което дойде до вас; както се принася плод и расте и в целия свят, тъй и между вас, от деня, когато чухте и наистина познахте Божията благодат,
\par 7 както сте я научили от нашия възлюбен съработник Епафраса, който е за нас верен Христов служител;
\par 8 който ни и извести за вашата любов в Духа.
\par 9 Затуй и ние, от деня, когато чухме за това, не преставаме да се молим за вас и да искаме от Бога да се изпълните с познанието на Неговата воля чрез пълна духовна мъдрост и проумяване,
\par 10 за да се обхождате достойно за Господа, да Му угаждате във всичко, като принасяте плод във всяко добро дело и като растете в познаването на Бога;
\par 11 подкрепяни от пълна сила, според Неговото славно могъщество, за да издържите и дълготърпите всичко с радост;
\par 12 като благодарите на Отца, Който ни удостои да участвуваме в наследството на светиите в светлината;
\par 13 Който ни избави от властта на тъмнината и ни пресели в царството на Своя възлюбен Син.
\par 14 В Него имаме изкуплението си, прощението на греховете;
\par 15 в Него, Който е образ на невидимия Бог, първороден преди всяко създание;
\par 16 понеже чрез Него бе създадено всичко, което е на небесата и на земята, видимото и невидимото, било престоли или господства, било началства или власти, всичко чрез Него бе създадено;
\par 17 и Той е преди всичко, и всичко чрез Него се сплотява.
\par 18 И глава на тялото, то ест, на църквата, е Той, Който е началникът, първороден от мъртвите, за да има първенство във всичко.
\par 19 Защото Отец благоволи да всели в Него съвършенната пълнота,
\par 20 и чрез Него да примири всичко със Себе Си, и земните и небесните, като въдвори мир чрез Него с кръвта, пролята на Неговия кръст.
\par 21 И вас, които бяхте някога отстранени и по разположение врагове в злите си дела,
\par 22 примири сега чрез Неговата смърт в плътското Му тяло, да ви представи пред Себе Си свети, непорочни и безупречни,
\par 23 ако останете основани и твърди във вярата, и без да се помръднете от надеждата, открита вам в благовестието, което сте чули, и което е било проповядвано на всяка твар под небесата, на което аз Павел станах служител.
\par 24 Сега се радвам в страданията си за вас, като от моя страна допълням недостатъка на скърбите на Христа в моето тяло заради Неговото тяло, което е църквата;
\par 25 на която аз станах служител, по Божия наредба, която ми бе възложена заради вас, да проповядвам напълно словото на Бога,
\par 26 сиреч, тайната, която е била скрита за векове и поколения, а сега се откри на неговите Светии;
\par 27 на която Божията воля беше да яви, какво е между езичниците богатството на славата на тая тайна, сиреч, Христос между вас, надеждата на славата.
\par 28 Него ние възгласяваме, като съветваме всеки човек, и поучаваме всеки човек с пълна мъдрост, за да предоставим всеки човек съвършен в Христа.
\par 29 Затова се и трудя, като се подвизавам според Неговата сила, която действува в мене мощно.

\chapter{2}

\par 1 Защото желая да знаете, какъв голям подвиг имам за вас и за ония, които са в Лаодикия, и за ония, които не са ме виждали лично,
\par 2 за да се утешат сърцата им, та, свързани заедно в любов за всяко обогатяване със съвършено проумяване, да познаят тайната Божия, сиреч, Христа.
\par 3 в Когото [са скрити] всички съкровища на премъдростта и на знанието.
\par 4 Това ви казвам, за да не ви измами някой с убедителни думи.
\par 5 Защото, ако и да не съм с вас тялом, пак духом съм с вас, и се радвам като гледам вашата уредба и постоянство на вашата вяра спрямо Христа.
\par 6 И тъй, както сте приели Христа, Исуса Господа, така се обхождайте в Него,
\par 7 вкоренени и назидавани в Него, утвърждавани във вярата си, както бяхте научени, и изобилствуващи [в нея с] благодарение.
\par 8 Внимавайте да не ви заплени някой с философията си и с празната си измама, по човешко предание, по първоначалните учения на света, а не по Христа.
\par 9 Защото в Него обитава телесно всичката пълнота на Божеството;
\par 10 и вие имате пълнота в Него, Който е глава на всяко началство и власт.
\par 11 В него бяхте и обрязани с обрязване не от ръка извършено, но с обрязването, което е от Христа, като съблякохте плътското тяло;
\par 12 погребани с Него в кръщението, в което бидохте и възкресени с Него чрез вяра в действието на Бога. Който Го възкреси от мъртвите.
\par 13 И вас, които бяхте мъртви чрез прегрешенията си и необрязаното си плътско естество, вас съживи с Него, като прости всичките ви престъпления;
\par 14 и като изличи противния нам в постановленията Му закон, който беше враждебен нам, махна го отсред и го прикова на кръста;
\par 15 и като ограби всички началства и властите, изведе ги на показ явно, възтържествувайки над тях явно чрез Него.
\par 16 И тъй, никой да не ви осъжда за това, което ядете, или което пиете, или за нещо относно някой празник, или новомесечие, или събота;
\par 17 които са сянка на онова, което ще дойде, в тялото Христово.
\par 18 Никой да ви не отнема наградата с измама, чрез самоволно смиреномъдрие и ангелослужение, като наднича в къща, който не е видял и напразно се надува с плътския си ум,
\par 19 а не държи главата Христа, от когото цялото тяло, снабдявано и сплотено чрез ставите и жилите си, расте с израстване, дадено от Бога.
\par 20 Ако сте умряли с Христа относно първоначалните учения на света, то защо, като че живеете на света, се подчинявате на постановления, като,
\par 21 "Не похващай", "Не вкусвай", "Не пипай",
\par 22 (които всички се развалят от употреба), по човешки заповеди и учения?
\par 23 Тия неща наистина имат вид на мъдрост в произволно богослужение и смирение и в нещадене на тялото, но не струват за нищо в борбата против угождаването на тялото.

\chapter{3}

\par 1 И тъй, ако сте били възкресени заедно с Христа, търсете това, което е горе, гдето седи Христос отдясно на Бога.
\par 2 Мислете за горното, а не за земното;
\par 3 защото умряхте, и животът ви е скрит с Христа в Бога.
\par 4 Когато Христос, нашият живот, се яви, тогава и вие ще се явите с Него в слава.
\par 5 Затова умъртвете природните си части, които действуват за земята: блудство, нечистота, страст, зла пощявка и сребролюбие, което е идолопоклонство;
\par 6 поради които иде Божия гняв върху рода на непокорните;
\par 7 в които и вие някога сте ходили, когато живеехте в тях.
\par 8 Но сега отхвърлете и вие всичко това: гняв, ярост, злоба, хулене, срамотно говорене от устата си.
\par 9 Не се лъжете един друг, понеже сте съблякли стария човек с делата му,
\par 10 и сте се облякли в новия, който се подновява в познание по образа на Този, Който го е създал;
\par 11 гдето не може да има грях и юдеин, обрязани и необрязани, варварин, скит, роб, свободен; но Христос е всичко и в всичко.
\par 12 И тъй, като Божии избрани, свети и възлюблени, облечете се с милосърдие, благост, смирение, кротост, дърготърпение.
\par 13 Претърпявайте си един друг, и един на друг си прощавайте, ако някой има оплакване против някого; както и Господ е простил вам, така прощавайте и вие.
\par 14 А над всичко това облечете се в любовта, която свързва всичко в съвършенство.
\par 15 И нека царува в сърцата ви Христовия мир, за който бяхте и призвани и в едно тяло; и бъдете благодарни.
\par 16 Христовото слово да се вселява във вас богато; с пълна мъдрост учете се и увещавайте се с псалми и химни и духовни песни, като пеете на Бога с благодат в сърцата си.
\par 17 и каквото и да вършите, словом или делом, вършете всичко в името на Господа Исуса, благодарещи чрез Него на Бога Отца.
\par 18 Жени подчинявайте се на мъжете си,както прилича в Господа.
\par 19 Мъже, любете жените си и не се огорчавайте против тях.
\par 20 Деца, покорявайте се на родителите си във всичко, защото това е угодно на Господа.
\par 21 Бащи, не дразнете децата си, за да не се обезсърчават.
\par 22 Слуги, покорявайте се във всичко на господарите си по плът, като работите не за очи като човекоугодници, но със сърдечна простота, боейки се от Господа.
\par 23 Каквото и да вършите работете от сърце, като на Господа, а не като на човеци;
\par 24 понеже знаете, че за награда от Господа ще получите наследството. Слугувайте на Господа Христа.
\par 25 Защото, който струва неправда, ще получи обратно неправдата си, и то без лицеприятие.

\chapter{4}

\par 1 Господари, отдавайте на слугите си безпристрастно това, което е справедливо, като знаете, че и вие имате Господар на небесата.
\par 2 Постоянствувайте в молитва, и бдете в нея с благодарение.
\par 3 Молете се още и за нас, да ни отвори Бог вратата за словото, тъй щото да говоря тайната, която е в Христа, за която съм и в окови,
\par 4 да я изявя така, както трябва да говоря.
\par 5 Обхождайте се мъдро към външните, като изкупувате благоговолението.
\par 6 Това, което говорите, да бъде винаги с благодат подправено със сол, за да знаете как трябва да отговаряте на всекиго.
\par 7 Що се касае до мене, всичко ще ви каже любезният брат и верен служител и съработник в Господа, Тихик,
\par 8 когото пратих до вас нарочно за това, да узнаете моето състояние, и той да утеши сърцата ви.
\par 9 Изпратих с него и верният и възлюблен брат Онисима, който е от вас. Те ще ви кажат всичко за тука.
\par 10 Поздравява ви Аристарх, който е затворен с мене, и Варнавовият сестрин син Марко, (за когото получихте поръчката: ако дойде при вас приемете го),
\par 11 и Исус, наречен Юст. От образованите тия са единствени мои съработници за Божието царство, които са ми били утеха.
\par 12 Поздравява ви служител Исус Христов, Елафрас, който е от вас, и който всякога усърдно се моли за вас, да стоите съвършенни и напълно уверени в всичко що е Божия воля.
\par 13 Защото свидетелствувам за него, че се труди много за вас и за тия, които са в Лаодикия и в Иерапол.
\par 14 Поздравяват ви възлюбленият лекар Лука Димас.
\par 15 Поздравете братята, които са в Лаодикия и Нимфана с домашната й църква.
\par 16 И като прочетете това послание помежду си, наредете да се прочете и в лаодикийската църква; и онова, което е от Лаодикия, да го прочетете и вие.
\par 17 И кажете на Архипа: Внимавай на службата, която си приел от Господа, да я изпълниш.
\par 18 Поздрава пиша аз, Павел, със собствената си ръка. Помнете оковите ми. Благодат да бъде с вас. [Амин].

\end{document}