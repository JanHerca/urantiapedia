\begin{document}

\title{Song of Solomon}


\chapter{1}

\par 1 Соломоновата песен на песните.
\par 2 Нека ме целуне с целувките на устата си, Защото любовта ти е по-желателна от виното.
\par 3 Твоите масла са благоуханни; Името ти е ароматно като изляно масло; Затова те обичат девиците.
\par 4 Привлечи ме; ние ще тичаме след тебе. Царят ме въвежда във вътрешните си стаи; Ще се радваме и ще веселим за тебе, Ще спомняме твоята любов повече от виното; С право те обичат!
\par 5 Черна съм, но хубава, ерусалимски дъщери, Като кидарските шатри, като Соломоновите завеси.
\par 6 Не ме гледайте, че съм почерняла, Понеже слънцето ме е припърлило. Синовете на майка ми, като се разгневиха на мене, Поставиха ме пазачка на лозята; Но своето лозе но опазих.
\par 7 Кажи ми ти, кого люби душата ми, Где пасеш стадото си, где го успокояваш на пладне; Че защо да съм като една, която се скита Край стадата на твоите другари?
\par 8 Ако ти не знаеш, хубавице между жените, Излез по дирите на стадата И паси яретата си при шатрите на овчарите.
\par 9 Уподобих те, любезна моя, На конете от Фараоновите колесници.
\par 10 Красиви са твоите бузи с плетенки, И шията ти с огърлици.
\par 11 Ще ти направим златни плетеници Със сребърни копчета.
\par 12 Докато царят седи на трапезата си, Нардът ми издава благоуханието си.
\par 13 Възлюбеният ми е за мене като китка от смирна, Която лежи между гърдите ми.
\par 14 Възлюбленият ми е за мене като кипрова китка В лозята на Енгади.
\par 15 Ето, хубава си, любезна моя, ето, хубава си; Очите ти са като на гълъбите.
\par 16 Ето, хубав си, любезни ми, да! Приятен си; И постелката ни е зеленината.
\par 17 Гредите на къщите ни са кедрови, Дъските ни са кипарисови.

\chapter{2}

\par 1 Аз съм роза Саронова И долински крем.
\par 2 Както е кремът между тръните; Така е любезната ми между дъщерите.
\par 3 Както ябълката между дърветата на сада, Така е възлюбеният ми между синовете; Пожелах сянката му и седнах под нея, И плодът му бе сладък в устата ми.
\par 4 Доведе ме в дома на пированието, И знамето му над мене бе любов.
\par 5 Подкрепете ме с млинчета, разхладете ме с ябълки Защото съм ранена от любов.
\par 6 Левицата му е под главата ми, И десницата му ме прегръща.
\par 7 Заклевам ви, ерусалимски дъщери В сърните и в полските елени. Да не възбудите и да не събудите любовта ми преди да пожелае.
\par 8 Гласът на възлюбления! ето, иде той, Скача по горите, играе по хълмовете.
\par 9 Възлюбеният ми прилича на сърна или на млад елен; Ето стои, зад стената ни, Гледа в прозорците, Надзърта през решетките.
\par 10 Проговаря възлюбленият ми и казва ми: Стани, любезна моя, прекрасна моя, и дойди;
\par 11 Защото, ето, зимата измина, И дъждът престана и отиде си;
\par 12 Цветята се явяват по земята, Времето на птичето пеене пристигна, И гласът на гургулицата се чува в нашата земя;
\par 13 По смоковницата зреят първите й смокини, И лозята цъфтят и издават благоухание. Стани, любезна моя; прекрасна моя, та дойди.
\par 14 О гълъбице моя, в пукнатините на скалата. В скришните места на стръмнините, Нека видя лицето ти, нека чуя гласа ти; Защото гласът ти е сладък, и лицето ти прекрасно.
\par 15 Хванете ни лисиците, Малките лисици, които погубват лозята; Защото лозята ни цъфтят.
\par 16 Възлюбленият ми е мой, и аз негова; Пасе стадото си между кремовете.
\par 17 Догде повее дневния хладен ветрец и побягнат сенките, Върни се, вълюблени ми, и бъди като сърне Или млад елен по назъбените планини.

\chapter{3}

\par 1 През нощта на леглото си потърсих онзи, когото обича душата ми; Потърсих го, но не го намерих.
\par 2 Рекох: Ще стана сега и ще обиколя града. По улиците и по площадите, Ще търся онзи, когото обича душата ми Потърсих го, но не го намерих.
\par 3 Намериха ме стражарите, които обхождат града; Попитах ги: Видяхте ли онзи, когото обича душата ми?
\par 4 А малко като ги отминах Намерих онзи, когото обича душата ми; Хванах го, и не го напуснах Догде го не въведох в къщата на майка си, И във вътрешната стая на оная, която ме е родила.
\par 5 Заклевам ви, ерусалимски дъщери, В сърните и в полските елени, Да не възбудите и да не събудите любовта ми преди да пожелае.
\par 6 Коя е тая която възлиза от пустинята като стълбове дим, Накадена със смирна и ливан, с всичките благоуханни прахове от търговеца?
\par 7 Ето, носилката е на Соломона; Около нея са шестдесет яки мъже от Израилевите силни
\par 8 Те всички държат меч и са обучени на война; Всеки държи меча си на бедрото си поради нощни страхове.
\par 9 Цар Соломон си направи носилка От ливанско дърво:
\par 10 Стълбчетата й направи от сребро, Легалището й от злато, постелката й от морав плат; Средата й бе бродирана чрез любовта на ерусалимските дъщери.
\par 11 Излезте, сионови дъщери, та вижте цар Соломона С венеца, с който го венча майка му в деня на женитбата му, И в деня, когато сърцето му се веселеше.

\chapter{4}

\par 1 Ето хубава си любезна моя; ето хубава си; Очите ти под булото са като гълъбови; Косите ти са като стадо кози Налягали по Галаадската планина;
\par 2 Зъбите ти са като стадо ново-стрижени овци Възлизащи от къпането; Те са всички като близнета, И не липсва ни един между тях
\par 3 Устните си като червена прежда, И устата ти прекрасни; Челото ти под булото е Като част от нар;
\par 4 Шията ти е като Давидовата кула Съградена за оръжейница, Гдето висят хиляда щитчета.- Всички щитове на силни мъже;
\par 5 Двете ти гърди са като две сърнета близнета, Които пасат между кремовете.
\par 6 Догде повее дневният хладен ветрец и побягнат сенките Аз ще отида в планините на смирната и в хълма на ливана.
\par 7 Ти си все красива, любезна моя; И недостатък няма в тебе.
\par 8 Дойди с мене от Ливан, невясто, С мене от Ливан; Погледни от върха на Амана, От върха на Санир и на Ермон, От леговищата на лъвовете, от планините на рисовете.
\par 9 Пленила си сърцето ми, сестро моя, невясто, Пленила си сърцето ми с един поглед от очите си С една огърлица на шията си.
\par 10 Колко е хубава твоята любов, сестро моя, невясто! Колко по-добра е от виното твоята любов, И благоуханието на твоите масла от всякакъв вид аромати!
\par 11 От устата ти, невясто, капе като мед от пита; Мед и мляко има под езика ти; И благоуханието на дрехите ти е като миризмата на Ливан.
\par 12 Градина затворена е сестра ми, невястата, Извор затворен, източник запечатан.
\par 13 Твоите издънки са рай от нарове С отборни плодове, кипър с нард,
\par 14 Нард и шафран, тръстика и канела, С всичките дървета доставящи благоухания като ливан, Смирна и алой, с всичките най-изрядни аромати.
\par 15 Градински извор си ти, Кладенец с текуща вода, и поточета от Ливан.
\par 16 Събудете се, северни ветрове, и дойди южни, Повей, в градината ми, за да потекат ароматите й. Нека дойде възлюбеният ми в градината си И яде изрядните си плодове.

\chapter{5}

\par 1 Дойдох в градината си, сестро моя, невясто; Обрах смирната си и ароматите си; Ядох медената си пита с меда си; Пих виното си с млякото си. Яжте, приятели; Пийте, да! изобилно пийте, възлюбени.
\par 2 Аз спях, но сърцето ми беше будно; И ето гласът на възлюбения ми; той хлопа и казва: Отвори ми, сестро моя, любезна моя! Гълъбица моя, съвършена моя; Защото главата ми се напълни с роса, Косите ми с нощни капки.
\par 3 Но аз си рекох: Съблякох дрехата си, - как да я облека? Умих нозете си, - как да ги окалям?
\par 4 Възлюбеният ми провря ръката си си през дупката на вратата; И сърцето ми се смути за него.
\par 5 Аз станах да отворя на възлюбения си; И от ръцете ми капеше смирна, И от пръстите ми плавка смирна, Върху дръжките на ключалката.
\par 6 Отворих на възлюбения си; Но възлюбеният ми беше се оттеглил, отишъл бе. Извиках: Душата ми ослабваше когато ми говореше! Потърсих го, но не го намерих; Повиках го, но не ми отговори.
\par 7 Намериха ме стражарите, които обхождат града, Биха ме, раниха ме; Пазачите на стените ми отнеха мантията.
\par 8 Заклевам ви, ерусалимски дъщери, ако намерите възлюбения ми, - то що? Кажете му, че съм ранена от любов.
\par 9 В що различава твоят възлюбен от друг възлюбен, О ти прекрасна между жените? В що различава твоят възлюбен от друг възлюбен, Та ни заклеваш ти така?
\par 10 Възлюбеният ми е бял и румен, Личи и между десет хиляди.
\par 11 Главата му е като най-чисто злато; Косите му са къдрави, черни като гарван;
\par 12 Очите му, умити в мляко, и като прилично вложени скъпоценни камъни, Са подобни на очите на гълъби при водни потоци;
\par 13 Бузите му са като лехи с аромати, Като бряг с благоуханни растения; Устните му са като кремове, от които капе плавка смирна;
\par 14 Ръцете му са като златни цилиндри покрити с хрисолит; Тялото му е като изделие от слонова кост украсено със сапфири;
\par 15 Краката му са като мраморни стълбове Закрепени на подложки от чисто злато; Изгледът му е като Ливан, изящен като кедрове;
\par 16 Устата му са много сладки; и той цял е прелестен. Такъв е възлюбеният ми, и такъв е приятелят ми, О ерусалимски дъщери.

\chapter{6}

\par 1 Где е отишъл твоят възлюбен, О ти прекрасна между жените? Где е свърнал твоят възлюбен, - Та да го търсими ние с тебе?
\par 2 Моят възлюбен слезе в градината си, в лехите с ароматите. За да пасе в градините и да бере крем.
\par 3 Аз съм на възлюбения си, и възлюбеният ми е мой: Той пасе стадото си между кремовете.
\par 4 Хубава си, любезна моя, като Терса, Красива като Ерусалим, Страшна като войска със знамена.
\par 5 Отвърни очите си от мене, Защото те ме обладаха. Косите ти са като стадо кози Налягали по Галаад;
\par 6 Зъбите ти са като стадо овци възлизащи от къпането; Те са всички като близнета, и не липсва ни един между тях,
\par 7 Челото ти под булото е Като част от нар.
\par 8 Има шестдесет царици, и осемдесет наложници, И безброй девойки;
\par 9 Но една е гълъбицата ми, съвършената ми. Тя е безподобната на майка си, отборната на родителите си; Видяха я дъщерите, и рекоха: Блазе й! Да! цариците и наложниците, и те я похвалиха.
\par 10 Коя е тая, която поглежда като зората, Красива като луната, чиста като слънцето, Страшна като войска със знамена?
\par 11 Слязох в градината на орехите За да видя зелените растения в долината. Да видя дали е напъпило лозето, И дали са цъфнали наровете.
\par 12 Без да усетя, ожиданието ми ме постави Между колесниците на благородните ми люде.
\par 13 Върни се, върни се, о суламко; Върни се, върни се, за да те погледаме! Какво ще видите в суламката? Нещо като борба между две дружини!

\chapter{7}

\par 1 Колко са красиви нозете ти с чехлите, дъщерьо княжеска! Твоите закръглени бедра са подобни на огърлица, Изделие на художнически ръце;
\par 2 Пъпът ти е както обла чаша, от която не липсва подправено вино; Коремът ти е като житен копен ограден с кремове;
\par 3 Двете ти гърди са като две сърнета близнета;
\par 4 Шията ти е като стълб от слонова кост; Очите ти са като водоемите в Есевон към портата Бат-рабим; Носът ти е като ливанската кула, Която гледа към Дамаск;
\par 5 Главата ти върху тебе е като Кармил, И косите на главата ти като мораво; Царят е пленен в къдриците им.
\par 6 Колко си хубава и колко приятна, О възлюбена, в очарованията си!
\par 7 Това твое тяло прилича на палма. И гърдите ти на гроздове.
\par 8 Рекох: Ще се възкача на палмата, ще хвана клончетата й; И, ето, гърдите ти ще бъдат като клончета на лоза, И благовонието на дъха ти като ябълки,
\par 9 И устата ти като най-хубаво вино, - Което се поглъща гладко за възлюбения ми, Като се хлъзга през устните на спящите.
\par 10 Аз съм на възлюбения си; И неговото желание е към мене.
\par 11 Дойди, възлюбени мой, нека излезем на полето, Да пренощуваме по селата,
\par 12 Да осъмнем в лозята, да видим напъпила ли е лозата, Появил ли се е крехкият грозд и цъфнали ли са наровете; Там ще ти дам любовта си.
\par 13 Мандрагоровите ябълки издават благоухание; И върху вратата ни има Всякакви изрядни плодове, нови и стари, Които съм запазила за тебе, възлюбени мой.

\chapter{8}

\par 1 Дано ми беше ти като брат, Който е сукал от гърдите на майка ми! Когато те намерих вън щях да те целуна, Да! и никой не щеше да ме презре.
\par 2 Взел бих те и завела В къщата на майка си, за да ме научиш; Напоила бих те с подправено вино И със сок от наровете си.
\par 3 Левицата му би била под главата ми, И десницата му би ме прегръщала.
\par 4 Заклевам ви, ерусалимски дъщери, Да не възбудите нито да събудите любовта ми, Преди да пожелае.
\par 5 Коя е тая, която идва от пустинята Опираща се на възлюбения си? Аз те събудих под ябълката; Там те роди майка ти, Там те роди родителката ти.
\par 6 Положи ме като печат на сърцето си, Като печат на мишцата си; Защото любовта е силна като смъртта. Ревността е остра като преизподнята, Чието святкане е святкане огнено, пламък най-буен.
\par 7 Много води не могат угаси любовта, Нито реките могат я потопи; Ако би дал някой целия имот на дома си за любовта, Съвсем биха го презрели.
\par 8 Ние имаме малка сестра, и тя няма гърди. Що да направим със сестра си в деня, когато стане дума за нея?
\par 9 Ако бъде стена ще съградим на нея сребърни укрепления; И ако бъде врата, ще я оградим с кедрови дъски.
\par 10 Аз съм стена, и гърдите ми са като стълбовете й; Тогава бях пред очите му като една, която е намерила благоволение.
\par 11 Соломон имаше лоза във Ваалхамон; Даде лозето на наематели; За плода му всеки трябваше да донесе хиляда сребърника.
\par 12 Моето лозе, собствеността ми, е под моята власт ; Хилядата нека са на тебе Соломоне, И двете на ония, които пазят плода му.
\par 13 О ти, която седиш в градините, Другарите внимават на гласа ти; Дай ми и аз да го чуя.
\par 14 Бързай, възлюбени мои, И бъди като сърне или еленче Върху планините на ароматите.

\end{document}