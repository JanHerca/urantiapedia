\begin{document}

\title{Ephesians}


\chapter{1}

\par 1 Павел, с божията воля апостол Исус Христов, до светиите и верните в Христа Исуса, които са в Ефес:
\par 2 Благодат и мир да бъде на вас от Бога, нашия Отец, и от Господа Исуса Христа.
\par 3 Благословен да бъде Бог и Отец на нашия Господ Исус Христос, Който в Христа ни е благословил с всяко духовно благословение в небесни места;
\par 4 като ни е избрал в Него преди създанието на света, за да бъдем свети и без недостатък пред Него в любов;
\par 5 като ни е предопределил да Му бъдем осиновени чрез Исуса Христа, по благоволението на Своята воля,
\par 6 за похвала на славната Си благодат, с която ни е обдарил във Възлюбения Си,
\par 7 в Когото имаме изкуплението си чрез кръвта Му, прощението на прегрешенията ни, според богатството на Неговата благодат,
\par 8 която е направил да доставя нам изобълно всяка мъдрост и разумение,
\par 9 като ни е открил тайната на волята Си според благото намерение, което е положил в Себе Си,
\par 10 за да се приложи когато се изпълнят времената, сиреч, да се събере в Христа всичко - това, което е небесно и земно, -
\par 11 в Него казвам, в Когото станахме и наследство, като бяхме предопределени на това според намерението на Бога, Който действува във всичко по решението на Своята воля,
\par 12 тъй щото, да бъдем за похвала на неговата слава ние, които отнапред се надяехме на Христа,
\par 13 в Когото и вие, като чухте словото на истината, сиреч благовестието на нашето спасение, - в Когото като и повярвахте, бяхте запечатани с обещания Свети Дух,
\par 14 който е залог на нашето наследство, догде бъде изкупено притежанието на Бога, - да бъдете за похвала на Неговата слава.
\par 15 Затова и аз като чух за вярата ви в Господа Исуса и за любовта, която сте показали към всичките светии,
\par 16 непрестанно благодаря Богу за вас и ви опоменавам в молитвите си,
\par 17 дано Бог на нашия Господ Исус Христос, славният Отец, ви даде дух на мъдрост и на откровение, за да го познаете,
\par 18 и да просвети очите на сърцето ви, за да познаете, каква е надеждата, към която ви призовава, какво е богатството между светиите на славното от Него наследство,
\par 19 и колко превъзходно велика е силата Му към нас вярващите - сила, която е според действуването на могъщата Негова мощ,
\par 20 с която подействува в Христа, когато Го възкреси от мъртвите и Го тури да седне от дясната Си страна на небесата,
\par 21 далече по-горе от всяко началство и власт, сила и господство, и всяко име, от което се именуват, не само в тоя свят, но и в бъдещия.
\par 22 И всичко покори под нозете Му, и постави Го да бъде глава над всичко в църквата,
\par 23 която е Негово тяло, изпълнено с пълнотата на Този, който изпълнява всичко във всички.

\chapter{2}

\par 1 И съживи вас, когато бяхте мъртви чрез вашите престъпления и грехове,
\par 2 в които сте ходили някога според вървежа на тоя свят, по княза на въздушната власт, на духа, който сега действува в синовете на непокорството;
\par 3 между които и ние всички сме живели някога в нашите плътски страсти, като сме изпълнявали волята на плътта и на помислите, и по естество сме били чада на гнева, както и другите.
\par 4 Бог, обаче, Който е богат с милост, поради голямата любов, с която ни възлюби,
\par 5 даже, когато бяхме мъртви чрез престъпленията си, съживи ни заедно с Христа (по благодат сте спасени),
\par 6 и, като ни съвъзкреси, тури ни да седим с Него в небесни места, в Христа Исуса;
\par 7 за да показва през идните векове премногото богатства на Своята благодат чрез добрината Си към нас в Христа Исуса.
\par 8 Защото по благодат сте спасени чрез вяра, и то не от сами нас; това е дар от Бога;
\par 9 не чрез дела, за да се не похвали никой.
\par 10 Защото сме Негово творение създадени в Христа Исуса за добри дела, в които Бог отнапред е наредил да ходим.
\par 11 Затова помнете, че вие, някога езичници по плът, наричани необрязани от тия, които се наричат обрязани с обрязване на плътта, което се извършва с ръце,
\par 12 в онова време бяхте отделени от Христа, странни на Израилевото гражданство и чужденци към заветите на обещанието, без да имате надежда и без Бог на света.
\par 13 А сега в Христа Исуса вие, които някога сте били далеч, сте поставени близу чрез кръвта на Христа.
\par 14 Защото Той е нашият мир, който направи двата отдела едно, и развали средната стена, която ги отделяше,
\par 15 като в плътта Си унищожи враждата, сиреч, закона със заповедите му изразени в постановления, за да създаде в Себе Си двата в един нов човек, и тъй да направи мир,
\par 16 и в едно тяло да примири и двата с Бога чрез кръста, като уби на него враждата.
\par 17 И като дойде, благовествува мир на вас, които бяхте далеч, и мир на тия, които бяха близу;
\par 18 защото чрез Него и едните и другите имаме своя достъп при Отца в един Дух.
\par 19 Затова вие не сте вече странни и пришелци, но сте съграждани на светиите и членове на Божието семейство:
\par 20 понеже бяхте съградени върху основата на апостолите и пророците, като с краеъгълен камък сам Христос Исус,
\par 21 върху когото всяко здание, стройно сглобено расте за храм свет на Господа;
\par 22 в който и вие се вграждате заедно в Духа на Божие обиталище

\chapter{3}

\par 1 Затова аз, Павел, затворник на Исуса Христа заради вас езичниците, -
\par 2 понеже сте чули за нареденото от Божията благодат, която ми е дадена заради вас,
\par 3 че по откровение ми стана известна тайната, (както и по-преди вкратце ви писах,
\par 4 от което, като прочитате, може да разберете моето проумяване в Христовата тайна)
\par 5 която в други поколения не биде известна на човешкия род, както сега чрез Духа се откри на Неговите свети апостоли и пророци,
\par 6 а именно, че езичниците са сънаследници, като съставляват едно тяло, и са съпричасници на [Неговото] обещание в Христа Исуса чрез благовестието,
\par 7 на което станах служител според Божията благодат, - дар, който ми е дадeн по дейстието на Неговата сила.
\par 8 На мене, най-нищожния от всички светии, се даде тая благодат, да благовестя между езичниците неизследимото Христово богатство;
\par 9 и да осветлявам всичките в наредбата относно тайната, която от векове е била скрита у Бога, създателя на всичко,
\par 10 тъй щото на небесните началства и власти да стане позната сега чрез църквата многообразната премъдрост на Бога,
\par 11 според превечното намерение, което Той изработи в Христа Исуса нашия Господ;
\par 12 в Когото имаме своето дръзновение и достъп с увереност чрез вяра в Него;
\par 13 за която причина ви моля да се не обезсърчавате от моите изпитни за вас, тъй като те са за вас слава, -
\par 14 затова, прекланям коленете си пред Отца [на нашия Господ Исус Христос],
\par 15 от Когото носи името всеки род на небесата и на земята,
\par 16 да ви даде според богатствата на славата Си, да се утвърдите здраво чрез Неговия Дух във вътрешния човек,
\par 17 чрез вяра да се всели Христос във вашите сърца, тъй че закоренени и основани в любовта
\par 18 да бъдете силни, да разберете заедно с всичките светии, що е широчината и дължината, височината и дълбочината,
\par 19 и да повикате Христовата любов, която никое знание не може да обгърне, за да се изпълните в цялата Божия пълнота.
\par 20 А на Този, Който, според действуващата в нас сила, може да направи несравнено повече, отколкото искаме или мислим,
\par 21 на Него да бъде слава в църквата и в Христа Исуса във всичките родове от века до века. Амин.

\chapter{4}

\par 1 И тъй, аз, затворник за Господа, моля ви да се обхождате достойно на званието, към което бяхте призовани,
\par 2 със съвършено смирение и кротост, с дълготърпение, като си претърпявате един друг с любов,
\par 3 и се стараете в свръзката на мира да упазите единството в Духа.
\par 4 Има едно тяло и един Дух, както и бяхте призовани към една надежда на званието ви:
\par 5 един Господ, една вяра, едно кръщение
\par 6 един Бог и Отец на всички, Който е над всички, чрез всички и във всички.
\par 7 А на всеки от нас се даде благодат според мярката на това, което Христос ни е дал
\par 8 Затова казва: - "Като възлезе на високо, плени плен И даде дарове на човеците"
\par 9 (А това "възлезе" що друго значи, освен че бе и [по-напред] слязъл в местата по-долни от земята.
\par 10 Тоя, Който е слязъл е същият, Който и възлезе по-горе от всичките небеса, за да изпълни всичко).
\par 11 И Той даде едни да бъдат апостоли, други пророци, други пък благовестители, а други пастири и учители,
\par 12 за делото на служението, за назиданието на Христовото тяло, с цел да се усъвършенствуват светиите;
\par 13 докле всички достигнем в единството на вярата и на познаването на Божия Син в пълнолетно мъжество, в мярката на ръста на Христовата пълнота;
\par 14 за да не бъдем вече деца, блъскани и завличани от всеки вятър на учение, чрез човешките заблуди, в лукавство, по измамителни хитрости;
\par 15 но, действуващи истинно в любов, да порастнем по всичко в Него, Който е главата, Христос,
\par 16 от Когото цялото тяло, сглобявано и свързано чрез доставяното от всеки става, според съразмерното действие на всяка една част, изработва растенето на тялото за своето назидание в любовта.
\par 17 Прочее, това казвам и заявявам в Господа, да не се обхождате вече, както се обхождат и езичниците, по своя суетен ум,
\par 18 помрачени в разума, и странни на живота от Бога поради невежеството, което е в тях, и поради закоравяването на сърцето им;
\par 19 които, изгубили чувство, са се предали на сладострастие, да вършат ненаситно всякаква нечистота.
\par 20 Но вие не сте така познали Христа;
\par 21 понеже сте чули, и сте научили от Него, (както е истината в Исуса),
\par 22 да съблечете, според предишното си поведение, стария човек, който тлее по измамителните страсти,
\par 23 да се обновите в духа на своя ум,
\par 24 и да се облечете в новия човек, създаден по образа на Бога в правда и светост на истината.
\par 25 Затова, като отхвърлите лъжата, говорете всеки с ближния си истина; защото сме части един на друг.
\par 26 Гневете се, но без да съгрешавате; слънцето да не залезе в разгневяването ви;
\par 27 нито давайте място на дявола.
\par 28 Който е крал, да не краде вече, а по-добре да се труди, като върши с ръцете си нещо полезно, за да има да отделя и на този, който има нужда
\par 29 Никаква гнила дума да не излиза от устата ви, но онова, което е добро, за назидание според нуждата, за да принесе благодат на тия, които слушат;
\par 30 и не оскърбявайте Светия Божий Дух, в когото сте запечатани за деня на изкуплението,
\par 31 всякакво огорчение, ярост, гняв, вик и хула, заедно с всяка злоба да се махне от вас;
\par 32 а бивайте един към друг благи, милосърдни; прощавайте си един на друг, както и Бог в Христа е простил на вас.

\chapter{5}

\par 1 И тъй, бивайте подражатели на Бога, като възлюбени чада;
\par 2 и ходете в любов, както и Христос ви възлюби и предаде Себе Си за нас принос и жертва на Бога за благоуханна миризма.
\par 3 А блудство и всякаква нечистота или сребролюбие да не се даже споменават между вас, както прилича на светии;
\par 4 нито срамотии или празни приказки, нито подигравки, които са неприлични неща, но по-добре благодарение.
\par 5 Защото добре знаете това, че никой блудник, или нечист, или сребролюбец, (който е идолопоклонник), няма наследство в царството на Христа и Бога.
\par 6 Никой да ви не мами с празни думи; понеже поради тия неща иде Божият гняв върху синовете на непокорството.
\par 7 И тъй, не бивайте съучастници на тях.
\par 8 Тъй като някога си бяхте тъмнина, а сега сте светлина в Господа, одхождайте се като чада на светлината;
\par 9 (защото плодът на светлината се състои във всичко що е благо, право и истинно).
\par 10 Опитвайте, що е благоугодно на Господа;
\par 11 и не участвувайте в безплодни дела на тъмнината, а по-добре ги изобличавайте;
\par 12 защото това, което скришом вършат непокорните, срамно е и да се говори.
\par 13 А всичко, което се изобличава, става явно чрез светлината; понеже всяко нещо, което става явно е осветено.
\par 14 Затова казва: Стани, ти, който спиш, И възкресни от мъртвите, И ще те осветли Христос.
\par 15 И тъй, внимавайте добре как се обхождате, не като глупави, но като мъдри,
\par 16 като изкупвате благовремието, защото дните са лоши.
\par 17 Затова не бивайте неомислени, но проумявайте, що е Господната воля.
\par 18 И не се опивайте с вино, следствието от което е разврат, но напълняйте се с Духа;
\par 19 и разговаряйте се със псалми и химни и духовни песни, като пеете и възпявате Господа в сърцето си,
\par 20 и като винаги благодарите за всичко на Бога и Отца в името на нашия Господ Исус Христос,
\par 21 като се подчинявате един на друг в страх от Христа.
\par 22 Жени, подчинявайте се на своите мъже, като длъжност към Господа;
\par 23 защото мъжът е глава на жената, както и Христос е глава на църквата (като само Той е спасител на тялото).
\par 24 Но както църквата се подчинява на Христа, така и жените нека се подчиняват във всичко на своите мъже.
\par 25 Мъже, любете жените си както и Христос възлюби църквата и предаде Себе Си за нея,
\par 26 за да я освети, като я е очистил с водно омиване чрез словото,
\par 27 за да й представи на Себе Си църква славна, без петно, или бръчка, или друго такова нещо, но да бъде света и непорочна.
\par 28 Така са длъжни и мъжете да любят жените си, както своите тела. Който люби жена си, себе си люби.
\par 29 Защото никой никога не е намразил своето тяло, но го храни и се грижи за него, както и Христос за църквата;
\par 30 понеже сме части на Неговото тяло [от Неговата плът и от Неговите кости].
\par 31 "Затова ще остави човек баща си и майка си, и ще се привърже към жена си и двамата ще станат една плът".
\par 32 Тая тайна е голяма; но аз говоря това за Христа и за църквата;
\par 33 Но и вие, всеки до един, да люби своята жена, както себе си; а жената да се бои от мъжа си.

\chapter{6}

\par 1 Деца, покорявайте се на родителите си в Господа;
\par 2 "Почитай баща си и майка си", (което е първата заповед с обещание),
\par 3 "за да ти бъде добре и да живееш много години на земята".
\par 4 И вие, бащи, не дразнете децата си, но възпитавайте ги в учение и наставление Господне.
\par 5 Слуги, покорявайте се на господарите си по плът със страх и трепет в простотата на сърцето си, като към Христа.
\par 6 Не работете само пред очи като човекоугодници, но като Христови слуги изпълнявайте от душа Божията воля;
\par 7 и слугувайте с добра воля като на Господа, а не на човеци;
\par 8 понеже знаете, че всеки слуга или свободен, ще получи от Господа същото добро каквото върши.
\par 9 И вие, господари, струвайте същото на тях, като се въздържате от заплашването; понеже знаете, че и на тях и на вас има Господар на небесата, у когото няма лицеприятие.
\par 10 Най-после, заяквайте в Господа и в силата на Неговото могъщество.
\par 11 Облечете се в Божието всеоръжие, за да можете да устоите срещу хитростите на дявола.
\par 12 Защото нашата борба не е срещу кръв и плът, но срещу началствата, срещу властите, срещу духовните сили на нечестието в небесните места.
\par 13 Затова вземете Божието всеоръжие, за да можете да противостоите в злия ден, и, като надвиете на всичко, да устоите.
\par 14 Стойте, прочее, препасани с истина през кръста си и облечени в правдата за бронен нагръдник,
\par 15 и с нозете си обути с готовност чрез благовестието на мира.
\par 16 А освен всичко това, вземете вярата за щит, с който ще можете да угасите всичките огнени стрели на нечестивия;
\par 17 вземете тоже за шлем спасението и меча на Духа, който е Божието слово;
\par 18 молещи се в Духа на всяко време с всякаква молитва и молба, бидейки бодри в това с неуморно постоянство и моление за всичките светии,
\par 19 и за мене, да ми се даде израз да отворя устата си, за да оповестя дръзновенно тайната на благовестието,
\par 20 за което съм посланник в окови, да говоря за него с дръзновение, както прилича да говоря.
\par 21 Но, за да знайте и вие за моите работи, и как съм, всичко ще ви каже Тихик, любезният брат и верен в Господа служител;
\par 22 когото пратих до вас нарочно за това, да узнаете нашето състояние, и той да утеши сърцата ви.
\par 23 Мир на братята, и любов с вяра, от Бога Отца и Господа Исуса Христа.
\par 24 Благодат да бъде с всички, които исрено любят нашия Господ Исус Христос. [Амин]

\end{document}