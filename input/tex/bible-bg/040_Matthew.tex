\begin{document}

\title{Матей}


\chapter{1}

\par 1 Родословието на Исуса Христа, син на Давида, син на Авраама.
\par 2 Авраам роди Исаак; Исаак роди Якова; Яков роди Юда и братята му;
\par 3 Юда роди Фареса и Зара от Тамар; Фарес роди Есрона; Есрон роди Арама;
\par 4 Арам роди Аминадава; Аминадав
\par 5 Салмон роди Вооза от Рахав; Вооз роди Овида от Рут; Овид роди Есея;
\par 6 а Есей роди цар Давида; цар Давид роди Соломона от Уриевата жена;
\par 7 Соломон роди Ровоама; Ровоам роди Авия; Авия роди Аса;
\par 8 Аса роди Йосафата; Йосафат роди Йорама; Йорам роди Озия;
\par 9 Озия роди Йотама; Йотам роди Ахаза; Ахаз роди Езекия;
\par 10 Езекия роди Манасия; Манасия роди Амона; Амон роди Йосия;
\par 11 а Йосия роди Ехония и братята му във времето на преселението във Вавилон.
\par 12 А след преселението във Вавилон, Ехония роди Салатиила; Салатиил роди Зоровавела;
\par 13 Зоровавел роди Авиуда; Авиуд роди Елиаким; Елиаким роди Азора;
\par 14 Азор роди Садока; Садок роди Ахима; Ахим роди Елиуда;
\par 15 Елиуд роди Елеазара; Елеазар роди Матана; Матан роди Якова;
\par 16 а Яков роди Йосифа, мъжа на Мария, от която се роди Исус, Който се нарича Христос(Помазаник или Месия.)
\par 17 И така, всичките родове от Авраама до Давида са четиринадесет; от Давида до преселението във Вавилон, четиринадесет рода; и от преселението във Вавилон до Христа, четиринадесет рода.
\par 18 А рождението на Исуса Христа(Помазаника) беше така: след като бе сгодена майка Му Мария за Йосифа, преди да бяха се съединили тя се намери непразна от Святия Дух.
\par 19 А мъжът й Йосиф, понеже беше праведен, а пък не искаше да я изложи, намисли да я напусне тайно.
\par 20 Но, когато мислеше това, ето, ангел от Господа му се яви насъне и каза: Йосифе, сине Давидов, не бой се да вземеш жена си Мария; защото зачнатото в нея е от Святия Дух.
\par 21 Тя ще роди син, и ще Му наречеш името Исус(Спасител); защото Той ще спаси людете Си от греховете им.
\par 22 А всичко това стана за да се сбъдне реченото от Господа чрез пророка, който казва:
\par 23 "Ето девицата ще зачне и ще роди син; и ще му нарекат името Емануил" (което значи, Бог с нас).
\par 24 И тъй, Йосиф, като стана от сън, стори както му заповяда ангелът от Господа и взе жена си;
\par 25 но не я познаваше докато тя роди [първородния си] син; и нарече Му името Исус.

\chapter{2}

\par 1 А когато се роди Исус във Витлеем Юдейски, в дните на цар Ирода, ето, мъдреци от изток пристигнаха в Ерусалим.
\par 2 И казаха: Къде е Юдейският цар, който се е родил? Защото видяхме звездата Му на изток, и дойдохме да Му се поклоним.
\par 3 Като чу това Ирод, смути се, и цял Ерусалим се него.
\par 4 Затова събра всички народни главни свещеници и книжници и ги разпитваше, къде трябваше да се роди Христос.
\par 5 А те му казаха: Във Витлеем Юдейски, защото така е писано чрез пророка: -
\par 6 "И ти, Витлееме, земьо Юдова, никак не си най-малък между Юдовите началства, защото от тебе ще произлезе Вожд; Който ще бъде пастир на Моя народ Израил".
\par 7 Тогава Ирод повика тайно мъдреците и внимателно научи от тях времето, когато се е явила звездата.
\par 8 И като ги изпрати във Витлеем, каза им: Идете, разпитайте внимателно за детето; и като Го намерите, известете ми, за да ида и аз да Му се поклоня.
\par 9 А те, като изслушаха царя, тръгнаха си; и, ето, звездата, която бяха видели на изток, вървеше пред тях, докато дойде и спря над мястото където беше детето.
\par 10 Като видяха звездата, зарадваха се твърде много.
\par 11 И като влязоха в къщата, видяха детето с майка Му Мария, паднаха и Му се поклониха; и отваряйки съкровищата си принесоха Му дарове, - злато, ливан и смирна.
\par 12 И понеже им се откри от Бог насъне да се не връщат при Ирода, те си отидоха през друг път в своята страна.
\par 13 А след отиването им, ето, ангел от Господа се явява насъне на Йосифа и казва: Стани, вземи детето и майка Му, и бягай в Египет, и остани там докато ти река, защото Ирод ще потърси детето за да Го погуби.
\par 14 И тъй, той стана, взе детето и майка Му през нощта и отиде в Египет,
\par 15 където остана до Иродовата смърт; за да се сбъдне реченото от Господа чрез пророка, който казва: "От Египет повиках Сина Си".
\par 16 Тогава Ирод, като видя, че беше подигран от мъдреците, разяри се твърде много, и прати да погубят всичките мъжки младенци във Витлеем и във всичките му околности, от две години и по-долу, според времето, което внимателно бе изучил от мъдреците.
\par 17 Тогава се изпълни реченото от пророк Еремия, който казва:
\par 18 "Глас се чу в Рама, плач и писък и голямо ридание; Рахил оплакваше чадата си, и не искаше да се утеши, защото ги няма вече".
\par 19 А като умря Ирод, ето ангел от Господа се явява насъне на Йосифа в Египет и казва:
\par 20 Стани, вземи детето и майка Му, и иди в Израелевата земя; защото измряха ония, които искаха живота на детето.
\par 21 И тъй той стана, взе детето и майка Му, и дойде в Израелевата земя.
\par 22 Но като чу, че над Юдея царувал Архелай, наместо баща си Ирода, страхуваше се да иде там; и, понеже му бе открито от Бога насъне, оттегли се в Галилейските страни,
\par 23 дойде и се засели в един град наречен Назарет; за да се сбъдне реченото чрез пророците, че ще се нарече Назарей.

\chapter{3}

\par 1 В ония дни дойде Йоан Кръстител и проповядваше в Юдейската пустиня, като казваше:
\par 2 Покайте се понеже наближи небесното царство.
\par 3 Защото този беше, за когото се е говорило чрез пророк Исаия, който казва: "Глас на един, който вика в пустинята: Пригответе пътя на Господа, прави направете пътеките за Него".
\par 4 А тоя Йоан носеше облекло от камилска козина и кожен пояс около кръста си; а храната му беше акриди и див мед.
\par 5 Тогава излизаха при него Ерусалим, цяла Юдея и цялата Йорданска околност,
\par 6 и се кръщаваха от него в реката Йордан, като изповядваха греховете си.
\par 7 А като видя, че мнозина от фарисеите и садукеите идеха да се кръстят от него, рече им: Рожби ехидни! Кой ви предупреди да бягате от идещия гняв?
\par 8 Затова, принасяйте плодове достойни за покаяние;
\par 9 и не мислете да думате в себе си: Авраам е нашият баща; защото ви казвам, че Бог може и от тия камъни да въздигне чада на Авраама.
\par 10 А и брадвата лежи вече при корена на дърветата; и тъй всяко дърво, което не дава добър плод, отсича се и в огъня се хвърля.
\par 11 Аз ви кръщавам с вода за покаяние; а Оня, Който иде след мене, е по-силен от мене, Комуто не съм достоен да понеса обущата; Той ще ви кръсти със Святия Дух и с огън.
\par 12 Лопатата е в ръката Му, и Той здраво ще очисти гумното Си, и ще събере житото Си в житницата, а плявата ще изгори в неугасим огън.
\par 13 Тогава идва Исус от Галилея на Йордан при Йоана за да се кръсти от него.
\par 14 А Йоан го възпираше, казвайки: Аз имам нужда да се кръстя от Тебе, и Ти ли идеш при мене?
\par 15 А Исус в отговор му рече: Остави сега, защото така ни е прилично да изпълним всичко що е право. Тогава Йоан Го остави.
\par 16 И като се кръсти, Исус веднага излезе от водата; и, ето, отвориха Му се небесата и видя Божият Дух, че слиза като гълъб и се спускаше на Него;
\par 17 и ето глас от небесата, който казваше: Този е възлюбеният Ми Син, в Когото е Моето благоволение.

\chapter{4}

\par 1 Тогава Исус биде отведен от Духа в пустинята, за да бъде изкушаван от дявола.
\par 2 И след като пости четиридесет дни и четиридесет нощи, най-после огладня.
\par 3 И тъй изкусителят дойде и Му рече: Ако си Божий Син кажи на тия камъни да станат хлябове.
\par 4 А Той в отговор каза: Писано е: "Не само с хляб ще живее човек, на с всяко слово, което излиза от Божиите уста".
\par 5 Тогава дяволът Го завежда в святия град, поставя Го на крилото на храма и Му казва:
\par 6 Ако си Божий Син, хвърли се долу; защото е писано: - "Ще заповяда на ангелите Си за тебе: И на ръце ще Те вдигат. Да не би да препънеш в камък ногата Си".
\par 7 Исус му рече: Писано е още: "Да не изпитваш Господа твоя Бог".
\par 8 Пак Го завежда дяволът на една много висока планина, показва Му всичките царства на света и тяхната слава и казва Му:
\par 9 Всичко това ще ти дам, ако паднеш да ми се поклониш.
\par 10 Тогава Исус му каза: Махни се, Сатано, защото е писано: "На господа твоя Бог да се покланяш, и само Нему да служиш".
\par 11 Тогава дяволът Го остави; и, ето, ангели дойдоха и Му прислужваха.
\par 12 А когато чу Исус, че Йоан бил предаден, отиде в Галилея.
\par 13 И като напусна Назарет, дойде и се настани в Кипернаум край езерото(Гръцки: Морето. Така и навсякъде в това евангелие.), в Завулоновите и Нефталимовите предели;
\par 14 за да се сбъдне реченото чрез пророк Исаия, който казва: -
\par 15 "Завулоновата земя и Нефталимовата земя, Край езерото, отвъд Йордан, Езическа Галилея;
\par 16 Людете, които седяха в тъмнина. Видяха голяма светлина И на ония, които седяха в страната на смъртна сянка, изгря им светлина".
\par 17 От тогава Исус започна да проповядва, казвайки: Покайте се, защото наближи небесното царство.
\par 18 И като ходеше край галилейското езеро, видя двама братя, Симона наречен Петър и брат му Андрея, че хвърляха мрежа в езерото, понеже бяха рибари.
\par 19 И казва им: Дойдете след Мене, и Аз ще ви направя ловци на човеци.
\par 20 И те веднага оставиха мрежите и отидоха след Него.
\par 21 И като отмина от там, видя други двама братя, Якова Заведеев и брат му Йоана, че кърпеха мрежите си в ладията с баща си Заведея; и ги повика.
\par 22 И те на часа оставиха ладията и баща си, и отидоха след Него.
\par 23 Тогава Исус ходеше по цяла Галилея, и поучаваше в синагогите им, и проповядваше благовестието на царството, и изцеляваше всякаква болест и всякаква немощ между людете.
\par 24 И разнесе се слух за Него по цяла Сирия; и довеждаха при Него всички болни, страдащи от разни болести и мъки, хванати от бяс, епилептици и парализирани; и ги изцели.
\par 25 И подире Му вървяха големи множества от Галилея и Декапол, от Ерусалим и Юдея, и из отвъд Йордан.

\chapter{5}

\par 1 А Исус като видя множествата, възкачи се на хълма; и когато седна, учениците Му дойдоха при Него.
\par 2 И като отвори устата Си поучаваше ги казвайки:
\par 3 Блажени нищите по дух, защото е тяхно небесното царство.
\par 4 Блажени скърбящите, защото те ще се утешат.
\par 5 Блажени кротките, защото те ще наследят земята.
\par 6 Блажени които гладуват и жадуват за правдата, защото те ще се наситят.
\par 7 Блажени милостивите, защото на тях ще се показва милост.
\par 8 Блажени чистите по сърце, защото те ще видят Бога.
\par 9 Блажени миротворците, защото те ще се нарекат Божии чада.
\par 10 Блажени гонените заради правдата, защото е тяхно небесното царство.
\par 11 Блажени сте, когато ви хулят и ви гонят, и говорят против вас лъжливо, всякакво зло заради Мене;
\par 12 радвайте се и веселете се, защото голяма е наградата ви на небесата, понеже така гониха пророците, които бяха преди вас.
\par 13 Вие сте солта на земята. Но ако солта обезсолее, с какво ще се осоли? Тя вече за нищо не струва, освен да се изхвърли вън и да се тъпче от хората.
\par 14 Вие сте виделината на света. Град поставен на хълм не може да се укрие.
\par 15 И когато запалят светило, не го турят под шиника, а на светилника, и то свети на всички, които са вкъщи.
\par 16 Също така нека свети вашата виделина пред човеците, за да виждат добрите ви дела, и да прославят вашия Отец, Който е на небесата.
\par 17 Да не мислите, че съм дошъл да разруша закона или пророците; не съм дошъл да разруша, но да изпълня.
\par 18 Защото истина ви казвам: Докле премине небето и земята, ни една йота, ни една точка от закона няма да премине, докато всичко не се сбъдне.
\par 19 И тъй, който наруши една от тия най-малки заповеди, и научи така човеците, най-малък ще се нарече в небесното царство; а който ги изпълни и научи така човеците, той ще се нарече велик в небесното царство.
\par 20 Защото казвам ви, че ако вашата правда не надмине правдата на книжниците и фарисеите, никак няма да влезете в небесното царство.
\par 21 Чули сте, че е било казано на старовременните: "Не убивай; и който убие излага се на съд".
\par 22 А пък Аз ви казвам, че всеки, който се гневи на брата си [без причина], излага се на съд; и който рече на брата си Рака(Безделниче), излага се на Синедриона; а който му рече: Бунтовни безумецо, излага се на огнения пъкъл.
\par 23 И тъй, като принасяш дара си на олтара, ако там си спомниш, че брат ти има нещо против тебе,
\par 24 Остави дара си там пред олтара, и иди, първо се помири с брата си, тогава дойди и принеси дара си.
\par 25 Спогаждай се с противника си по-скоро, докато си на пътя с него към съдилището, да не би противникът ти да те предаде на съдията, а съдията те предаде на служителя, и да бъдеш хвърлен в тъмница.
\par 26 Истина ти казвам: Никак няма да излезеш оттам докле не изплатиш и последния кодрант.
\par 27 Чули сте, че е било казано: "Не прелюбодействувай".
\par 28 Но Аз ви казвам, че всеки, който гледа жена, за да я пожелае, вече е прелюбодействувал с нея в сърцето си.
\par 29 Ако дясното ти око те съблазнява, извади го и хвърли го; защото по-добре е за тебе да погине една от телесните ти части, а не цялото ти тяло да бъде хвърлено в пъкъла.
\par 30 И ако дясната ти ръка те съблазнява, отсечи я и хвърли я: защото по-добре е за тебе да погине една от телесните ти части, а не цялото ти тяло да отиде в пъкъла.
\par 31 Още било казано: "Който си напусне жената, нека й даде разводно писмо".
\par 32 А пък Аз ви казвам, че всеки, който напусне жена си, освен по причина на прелюбодейство, прави я да прелюбодействува; и който се ожени за нея, когато бъде напусната, той прелюбодействува.
\par 33 Чули сте още, че е било казано на старовременните: "Не си престъпвай клетвата, но изпълнявай пред Господа клетвите си".
\par 34 Но Аз ви казвам: Никак да се не кълнете; нито в небето, защото то е престол на Бога;
\par 35 нито в земята, защото е подножието Му; нито в Ерусалим, защото е град на великия Цар.
\par 36 Нито в главата си да се не кълнеш, защото не можеш направи ни един косъм бял или черен.
\par 37 Но говорът ви да бъде: Да, да; Не, не; а каквото е повече от това, е от лукавия.
\par 38 Чули сте, че е било казано: "Око за око, зъб за зъб".
\par 39 А пък Аз ви казвам: Не се противете на злия човек; но, ако те плесне някой по дясната буза, обърни му и другата.
\par 40 На тогова, който би поискал да се съди с тебе и да ти вземе ризата, остави му и горната дреха.
\par 41 Който те принуди да вървиш с него една миля, иди с него две.
\par 42 Дай на оногова, който проси от тебе; и не се отвръщай от оногова, който ти иска на заем.
\par 43 Чули сте, че е било казано: "Обичай ближния си, а мрази неприятеля си".
\par 44 Но Аз ви казвам: Обичайте неприятелите си и молете се за тия, които ви гонят;
\par 45 за да бъдете чада на вашия Отец, Който е на небесата; защото Той прави слънцето Си да изгрява на злите и на добрите, и дава дъжд на праведните и на неправедните.
\par 46 Защото, ако обичате само ония, които обичат вас, каква награда ви се пада? Не правят ли това и бирниците?
\par 47 И ако поздравявате само братята си, какво особено правите? Не правят ли това и езичниците?
\par 48 И тъй бъдете съвършени и вие, както е съвършен вашият небесен Отец.

\chapter{6}

\par 1 Внимавайте да не вършите делата на правдата си пред човеците, за да ви виждат; инак нямате награда при Отца си, Който е на небесата.
\par 2 И тъй, когато правиш милостиня, не тръби пред себе си, както правят лицемерите по синагогите и по улиците, за да бъдат похвалявани от човеците; истина ви казвам: Те са получили вече своята награда.
\par 3 А когато ти правиш милостиня, нека левицата ти не узнае какво прави десницата ти;
\par 4 за да става твоята милостиня в тайно; и твоят Отец, Който вижда в тайно, ще ти въздаде [на яве].
\par 5 И когато се молите, не бивайте като лицемерите; защото те обичат да се молят стоящи по синагогите и по ъглите на улиците, за да ги виждат човеците; истина ви казвам: Те са получили вече своята награда.
\par 6 А ти, когато се молиш, влез във вътрешната си стаичка, и като си затвориш вратата, помоли се на своя Отец, Който е в тайно; и Отец ти, Който вижда в тайно, ще ти въздаде [на яве].
\par 7 А когато се молите, не говорете излишни думи, както езичниците; защото те мислят че ще бъдат послушани заради многото си говорене.
\par 8 И тъй, не бъдете като тях; защото Отец ви знае от що се нуждаете преди вие да Му искате.
\par 9 А вие се молете така: Отче наш, Който си на небесата, да се свети Твоето име!
\par 10 да дойде Твоето царство; да бъде Твоята воля, както на небето така и на земята;
\par 11 Дай ни днес ежедневния хляб;
\par 12 и прости ни дълговете, както и ние простихме на нашите длъжници;
\par 13 и не ни въвеждай в изкушение, но избави ни от лукавия, [защото царството е Твое, и силата и славата, до вековете. Амин].
\par 14 Защото, ако вие простите на човеците съгрешенията им, то и небесният ви Отец ще прости на вас.
\par 15 Но ако вие не простите на човеците съгрешенията им, то и вашият Отец няма да прости вашите съгрешения.
\par 16 А когато постите, не бивайте унили, като лицемерите; защото те помрачават лицата си, за да ги виждат човеците, че постят; истина ви казвам: Те са получили вече своята награда.
\par 17 А ти, когато постиш, помажи главата си и омий лицето си,
\par 18 за да те не виждат човеците, че постиш, но Отец ти, Който е в тайно; и Отец ти, Който вижда в тайно, ще ти въздаде [на яве].
\par 19 Недейте си събира съкровища на земята, гдето молец и ръжда ги изяжда, и гдето крадци подкопават и крадат.
\par 20 Но събирайте си съкровища на небето, гдето молец и ръжда не ги изяжда, и гдето крадци не подкопават нито крадат;
\par 21 защото гдето е съкровището ти, там ще бъде и сърцето ти.
\par 22 Окото е светило на тялото; и тъй, ако окото ти е здраво цялото ти тяло ще бъде осветено;
\par 23 но ако окото ти е болнаво, то цялото ти тяло ще бъде помрачено. Прочее, ако светлината в тебе е тъмнина то колко голяма ще е тъмнината!
\par 24 Никой не може да слугува на двама господари, защото или ще намрази единия, а ще обикне другия, или към единия ще се привърже, а другия ще презира. Не можете да слугувате на Бога и на мамона.
\par 25 Затова ви казвам: Не се безпокойте за живота си, какво ще ядете или какво ще пиете, нито за тялото си, какво ще облечете. Не е ли животът повече от храната, и тялото от облеклото?
\par 26 Погледнете на небесните птици, че не сеят, нито жънат, нито в житници събират: и пак небесният ви Отец ги храни. Вие не сте ли много по-скъпи от тях?
\par 27 И кой от вас може с грижене да прибави един лакът на ръста си?
\par 28 И за облекло, защо се безпокоите? Разгледайте полските кремове как растат; не се трудят; нито предат;
\par 29 но казвам ви, че нито Соломон с всичката си слава не се е обличал като един от тях.
\par 30 Но ако Бог така облича полската трева, която днес я има, а утре я хвърлят в пещ, не ще ли много повече да облича вас маловерци?
\par 31 И тъй не се безпокойте, и не думайте: Какво ще ядем? или: Какво ще пием? или: Какво ще облечем?
\par 32 (Защото всичко това търсят езичниците), понеже небесният ви Отец знае, че се нуждаете от всичко това.
\par 33 Но първо търсете Неговото царство и Неговата правда; и всичко това ще ви се прибави.
\par 34 Затова, не се безпокойте за утре, защото утрешният ден ще се безпокои за себе си. Доста е на деня злото, което му се намери.

\chapter{7}

\par 1 Не съдете, за да не бъдете съдени.
\par 2 Защото с каквато съдба съдите, с такава ще ви съдят, и с каквато мярка мерите, с такава ще ви се мери.
\par 3 И защо гледаш съчицата в окото на брата си, а не внимаваш на гредата в твоето око?
\par 4 Или как ще речеш на брата си: Остави ме да извадя съчицата из окото ти; а ето гредата в твоето око?
\par 5 Лицемерецо, първо извади гредата от твоето око, и тогава ще видиш ясно за да извадиш съчицата от братовото си око.
\par 6 Не давайте свято нещо на кучетата, нито хвърляйте бисерите си пред свинете, да не би да ги стъпчат с краката си и се обърнат да ви разкъсат.
\par 7 Искайте, и ще ви се даде; търсете, и ще намерите; хлопайте, и ще ви се отвори,
\par 8 защото всеки, който иска, получава; който търси, намира; и на тогова, който хлопа, ще се отвори.
\par 9 Има ли между вас човек, който, ако му поиска син му хляб, ще му даде камък?
\par 10 или, ако поиска риба, да му даде змия?
\par 11 И тъй, ако вие, които сте зли, знаете да давате блага на чадата си, колко повече Отец ви, Който е на небесата, ще даде добри неща на тия, които искат от Него!
\par 12 И тъй, всяко нещо, което желаете да правят човеците на вас, така и вие правете на тях; защото това е същината на закона и пророците.
\par 13 Влезте през тясната порта, защото широка е портата и пространен е пътят, който води в погибел, и мнозина са ония, които минават през тях.
\par 14 Понеже тясна е портата и стеснен е пътят, който води в живот, и малцина са ония, които ги намират.
\par 15 Пазете се от лъжливите пророци, които дохождат при вас с овчи дрехи, а отвътре са вълци грабители.
\par 16 От плодовете им ще ги познаете. Бере ли се грозде от тръни, или смокини от репеи?
\par 17 Също така, всяко добро дърво дава добри плодове, а лошото дърво дава лоши плодове.
\par 18 Не може добро дърво да дава лоши плодове; или лошо дърво да дава добри плодове.
\par 19 Всяко дърво, което не дава добър плод, отсича се и се хвърля в огън.
\par 20 И тъй, от плодовете им ще ги познаете.
\par 21 Не всеки, който Ми казва: Господи! Господи! ще влезе в небесното царство, но който върши волята на Отца ми, Който е на небесата.
\par 22 В онзи ден мнозина ще Ми рекат: Господи! Господи! не в Твоето ли име пророкувахме, не в Твоето ли име бесове изгонихме, и не в Твоето ли име направихме много велики дела?
\par 23 Но тогава ще им заявя: Аз никога не съм ви познавал; махнете се от Мене вие, които вършите беззаконие.
\par 24 И тъй, всеки, който чуе тия Мои думи и ги изпълнява, ще се уприличи на разумен човек, който е построил къщата си на канара;
\par 25 и заваля дъждът, придойдоха реките, духнаха ветровете, и устремиха се върху тая къща; но тя не падна, защото бе основана на канара.
\par 26 И всеки, който чуе тези Мои думи и не ги изпълнява, ще се уприличи на неразумен човек, който построи къщата си на пясък;
\par 27 и заваля дъждът, придойдоха реките и духнаха ветровете, и устремиха се върху тая къща; и тя падна, и падането й бе голямо.
\par 28 И когато свърши Исус тия думи, народът се чудеше на учението Му;
\par 29 защото ги поучаваше като един, който има власт, а не като техните книжници.

\chapter{8}

\par 1 А когато слезе от хълма, последваха Го големи множества.
\par 2 И, ето, един прокажен дойде при Него, кланяше Му се и каза: Господи, ако искаш, можеш да ме очистиш.
\par 3 Тогава Исус простря ръка и се допря до него и рече: Искам; бъди очистен. И на часа му се очисти проказата.
\par 4 И Исус му каза: гледай да не кажеш това никому; но, за свидетелство на тях, иди да се покажеш на свещеника, и принеси дара, който Мойсей е заповядал.
\par 5 А когато влезе в Кипернаум, един стотник дойде при Него и Му се молеше, казвайки:
\par 6 Господи, слугата ми лежи у дома парализиран, и много се мъчи.
\par 7 Той му казва: Ще дойда и ще го изцеля.
\par 8 Стотникът в отговор Му рече: Господи не съм достоен да влезеш под стряхата ми; но кажи само една дума, и слугата ми ще оздравее.
\par 9 Защото и аз съм подвластен човек и имам подчинени на мен войници; и казвам на тогова: Иди, и той отива; и на друг: Дойди, и той дохожда; а на слугата си: Стори това, и го струва.
\par 10 Исус, като чу това, почуди се, и рече на ония, които идеха изподире: Истина ви казвам, нито в Израиля съм намерил толкова вяра.
\par 11 Но казвам ви, че мнозина ще дойдат от изток и запад, и ще насядат с Авраама, Исаака и Якова в небесното царство;
\par 12 а чадата на царството ще бъдат изхвърлени във външната тъмнина; там ще бъде плач и скърцане със зъби.
\par 13 Тогава Исус рече на стотника: Иди си; както си повярвал, така нека ти бъде. И слугата оздравя в същия час.
\par 14 И когато дойде Исус в къщата на Петра, видя, че тъща му лежеше болна от треска.
\par 15 И допря се до ръката й, и треската я остави; и тя стана да Му прислужва.
\par 16 А когато се свечери, доведоха при Него мнозина хванати от бяс; и Той изгони духовете с една дума, и изцели всичките болни;
\par 17 за да се сбъдне реченото чрез пророк Исаия, който казва: "Той взе на Себе Си нашите немощи, И болестите ни понесе".
\par 18 И като видя Исус около Себе Си много народ, заповяда да минат отвъд езерото.
\par 19 И дойде един книжник и Му рече: Учителю, ще вървя след Тебе където и да идеш.
\par 20 Исус му каза: Лисиците си имат леговища, и небесните птици гнезда; а Човешкият Син няма где глава да подслони.
\par 21 А друг от учениците Му рече: Господи, позволи ми първо да отида и погреба баща си.
\par 22 Но Исус му рече: Върви след Мене, и остави мъртвите да погребат своите мъртъвци.
\par 23 И когато влезе в една ладия, учениците Му влязоха подир Него.
\par 24 И, ето, голяма буря се подигна на езерото, до толкова щото вълните покриваха ладията; а Той спеше.
\par 25 Тогава се приближиха, събудиха Го и казаха: Господи, спаси! Загиваме!
\par 26 А Той им каза: Защо сте страхливи, маловерци? Тогава стана, смъмри ветровете и вълните, и настана голяма тишина.
\par 27 А човеците се чудеха и казваха: Какъв е Тоя, че и ветровете и вълните(Гръцки: Морето.) Му се покоряват?
\par 28 И когато дойде на отвъдната страна, в гадаринската земя, срещнаха Го двама хванати от бяс, които излизаха от гробищата, твърде свирепи, така щото никой не можеше да мине през онзи път.
\par 29 И, ето, извикаха, казвайки: Какво имаш с нас, Ти Божий Сине? Нима си дошъл тука преди време да ни мъчиш?
\par 30 А надалеч от тях имаше голямо стадо свине, което пасеше.
\par 31 И бесовете му се молеха, думайки: Ако ни изпъдиш, изпрати ни в стадото свине.
\par 32 И рече им: Идете. И те като излязоха, отидоха в свинете; и, ето, цялото стадо се спусна долу по стръмнината в езерото, и загина във водата.
\par 33 А свинарите побягнаха, и като отидоха в града разказаха всичко, и това що бе станало с хванатите от бяс.
\par 34 И, ето, целият град излезе да посрещне Исуса; и като Го видяха, помолиха Му се да си отиде от техните предели.

\chapter{9}

\par 1 Тогава Той влезе в една ладия, премина и дойде в Своя Си град.
\par 2 И, ето, донесоха при Него един паралитик, сложен на постелка; и Исус като видя вярата им, рече на паралитика: Дерзай, синко; прощават ти се греховете.
\par 3 И, ето, някои от книжниците си казаха: Този богохулствува.
\par 4 А Исус, като узна помислите им, рече: Защо мислите зло в сърцата си?
\par 5 Защото кое е по-лесно, да река: Прощават ти се греховете, или да река: Стани и ходи?
\par 6 Но, за да познаете, че Човешкият Син има власт на земята да прощава греховете (тогава каза на паралитика): Стани, вдигни си постелката и иди у дома си.
\par 7 И той стана и отиде у дома си.
\par 8 А множествата, като видяха това, страх ги обзе и прославиха Бога, Който бе дал такава власт на човеците.
\par 9 И като минаваше оттам, Исус видя един човек, на име Матей, седящ в бирничеството; и рече му: Върви след Мене. И той стана да Го последва.
\par 10 И когато бе седнал на трапезата в къщата, ето, мнозина бирници и грешници дойдоха и насядаха с Исуса и с учениците Му.
\par 11 И фарисеите, като видяха това, рекоха на учениците Му: Защо яде вашият учител с бирниците и грешниците?
\par 12 А Той, като чу това, рече: Здравите нямат нужда от лекар, а болните.
\par 13 Но идете и научете се що значи тази дума: "Милост искам, а не жертви", защото не съм дошъл да призова праведните, а грешните [на покаяние].
\par 14 Тогава дохождат при Него Йоановите ученици и казват: Защо ние и фарисеите постим много, а Твоите ученици не постят?
\par 15 Исус им каза: Могат ли сватбарите да жалеят, докато е с тях младоженецът? Ще дойде обаче, време, когато младоженецът ще им се отнеме; и тогава ще постят.
\par 16 Никой не кърпи вехта дреха с невалян плат; защото това, което трябваше да я запълни, отдира от дрехата, и съдраното става по-лошо.
\par 17 Нито наливат ново вино във вехти мехове; инак, меховете се спукват, виното изтича, и меховете се изхабяват. Но наливат ново вино в нови мехове, та и двете се запазват.
\par 18 Когато им говореше това, ето, един началник дойде и им се кланяше, и казваше: Дъщеря ми току що умря; но дойди и възложи ръката Си на нея и тя ще оживее.
\par 19 И, като стана, Исус отиде подир него, тоже и учениците Му.
\par 20 И, ето, една жена, която имаше кръвотечение дванадесет години, приближи се изотзад и се допря до полата на дрехата Му;
\par 21 защото си думаше: Ако се допра до дрехата Му, ще оздравея.
\par 22 А Исус като се обърна и я видя, рече: Дерзай, дъщерьо; твоята вяра те изцели. И от същия час жената оздравя.
\par 23 И когато дойде Исус в къщата на началника, и видя свирачите и народа разтревожен, рече:
\par 24 Идете си, защото момичето не е умряло, но спи. А те Му се присмиваха.
\par 25 А като изпъдиха народа, Той влезе и я хвана за ръката; и момичето стана.
\par 26 И това са разчу по цялата оная страна.
\par 27 И когато Исус си отиваше оттам, подир Него вървяха двама слепи, които викаха, казвайки: Смили се за нас, Сине Давидов!
\par 28 И като влезе вкъщи, слепите се приближиха до Него; и Исус им казва: Вярвате ли че мога да сторя това? Казват Му: Вярваме, Господи.
\par 29 Тогава Той се допря до очите им, и рече: Нека ви бъде според вярата ви.
\par 30 И очите им се отвориха. А Исус им заръча строго, като каза: Внимавайте никой да не знае това.
\par 31 А те, като излязоха разгласиха славата Му по цялата оная страна.
\par 32 И когато те излизаха, ето, доведоха при Него един ням човек, хванат от бяс.
\par 33 И след като бе изгонен бесът, немият проговори; и множествата се чудеха и думаха: Никога не се е виждало такова нещо в Израиля.
\par 34 А фарисеите казваха: Чрез началника на бесовете Той изгонва бесовете.
\par 35 Тогава Исус обикаляше всичките градове и села и поучаваше в синагогите им и проповядваше благовестието на царството; и изцеляваше всякаква болест и всякаква немощ.
\par 36 А когато видя множествата, смили се за тях, защото бяха отрудени и пръснати като овце, които нямат пастир.
\par 37 Тогава рече на учениците Си: Жетвата и изобилна, а работниците малко;
\par 38 затова, молете се на Господаря на жетвата да изпрати работници на жетвата Си.

\chapter{10}

\par 1 И като повика дванадесетте Си ученици, даде им власт над нечистите духове, да ги изгонват, и да изцеляват всякаква болест и всякаква немощ.
\par 2 А ето имената на дванадесетте апостоли: първият, Симон, който се нарича Петър, и Андрей, неговият брат; Яков Заведеев, и Йоан, неговият брат;
\par 3 Филип и Вартоломей; Тома и Матей бирникът; Яков Алфеев и Тадей;
\par 4 Симон Зилот и Юда Искариотски, който Го предаде.
\par 5 Тия дванадесет души Исус изпрати и заповяда им, казвайки: Не пътувайте към езичниците, и в самарийски град не влизайте;
\par 6 но по-добре отивайте при изгубените от Израилевия дом.
\par 7 И като отивате, проповядвайте, казвайки: Небесното царство наближи.
\par 8 Болни изцелявайте, мъртви възкресявайте, прокажени очиствайте, бесове изгонвайте; даром сте приели, даром давайте.
\par 9 Не вземайте нито злато, нито сребро, нито медна монета в пояса си,
\par 10 нито торба за път, нито две ризи, нито обуща, нито тояга; защото работникът заслужава своята прехрана.
\par 11 И в кой да било град или село, като влезете, разпитвайте кой в него е достоен, и там оставайте докле си отидете.
\par 12 А когато влизате в дома, поздравявайте го.
\par 13 И ако домът бъде достоен, нека дойде на него вашият мир; но ако не бъде достоен, мирът ви нека се върне към вас.
\par 14 И ако някой не ви приеме, нито послуша думите ви, когато излизате от дома му, или от онзи град, отърсете праха от нозете си.
\par 15 Истина ви казвам, по-леко ще бъде наказанието на содомската и гоморската земя в съдния ден, отколкото на онзи град.
\par 16 Ето, Аз ви изпращам като овце посред вълци; бъдете, прочее, разумни като змиите, и незлобливи като гълъбите.
\par 17 А пазете се от човеците защото ще ви предават на събори, и в синагогите си ще ви бият.
\par 18 Да! И пред управители и царе ще ви извеждат, поради Мене, за да свидетелствувате на тях и на народите.
\par 19 А когато ви предадат, не се безпокойте, как или какво да говорите, защото в същия час ще ви се даде какво да говорите.
\par 20 Защото не сте вие, които говорите, но Духът на Отца ви, Който говори чрез вас.
\par 21 Брат брата ще предаде на смърт, и баща чадо; и чада ще се подигат против родителите си и ще ги умъртвят.
\par 22 Ще бъдете мразени от всички, поради Моето име; а който устои до край, той ще бъде спасен.
\par 23 А когато ви гонят от тоя град, бягайте в другия; защото истина ви казвам: Няма да изходите Израилевите градове докле дойде Човешкият Син.
\par 24 Ученикът не е по-горен от учителя си, нито слугата е по-горен от господаря си.
\par 25 Доста е на ученика да бъде като учителя си, и слугата като господаря си. Ако стопанинът на дома нарекоха Веезевул, то колко повече домашните Му!
\par 26 И тъй, не бойте се от тях; защото няма нищо покрито, което не ще се открие, и тайно, което не ще се узнае.
\par 27 Това, което ви говоря в тъмно, кажете го на видело; и което чуете на ухо, прогласете го от покрива.
\par 28 Не бойте се от ония, които убиват тялото, а душата не могат да убият; но по-скоро бойте се от оногова, който може и душа и тяло да погуби в пъкъла.
\par 29 Не продават ли се две врабчета за един асарий? И пак ни едно от тях няма да падне на земята без волята на Отца ви.
\par 30 А вам и космите на главата са всички преброени.
\par 31 Не бойте се, прочее, вие сте много по-скъпи от врабчетата.
\par 32 И тъй, всеки, който изповяда Мене пред човеците, ще го изповядам и Аз пред Отца Си, Който е на небесата.
\par 33 Но всеки, който се отрече от Мене пред човеците, ще се отрека и Аз от него пред Отца Си, Който е на небесата.
\par 34 Да не мислите, че дойдох да поставя мир на земята; не дойдох да поставя мир, а нож.
\par 35 Защото дойдох да настроя човек против баща му, дъщеря против майка й, и снаха против свекърва й;
\par 36 и неприятели на човека ще бъдат домашните му.
\par 37 Който люби баща или майка повече от Мене, не е достоен за Мене; и който люби син или дъщеря повече от Мене, не е достоен за Мене.
\par 38 и който не вземе кръста си и не върви след Мене, не е достоен за Мене.
\par 39 Който намери живота си, ще го изгуби; и който изгуби живота си, заради Мене, ще го намери.
\par 40 Който приема вас, Мене приема; и който приема Мене, приема Този, Който Ме е пратил.
\par 41 Който приема пророк в името на пророк, награда на пророк ще получи; и който приема праведник в име на праведник, награда на праведник ще получи.
\par 42 И който напои един от тия скромните само с една чаша студена вода, в име на ученик, истина ви казвам, никак няма да изгуби наградата си.

\chapter{11}

\par 1 А Исус, когато свърши наставленията Си към дванадесетте Си ученика, замина оттам да поучава и проповядва по градовете им.
\par 2 А Йоан като чу в тъмницата за делата на Христа, прати от учениците си да Му кажат:
\par 3 Ти ли си оня, Който има да дойде, или друг да очакваме?
\par 4 Исус в отговор им рече: Идете, съобщете на Йоана това, което чувате и виждате:
\par 5 Слепи прогледват, куци прохождат, прокажени се очистват и глухи прочуват; мъртви биват възкресявани, и на сиромасите се проповядва благовестието.
\par 6 И блажен оня, който не се съблазнява в Мене.
\par 7 И когато те си отиваха, Исус почна да казва на народа за Йоана: Какво излязохте да видите в пустинята? Тръстика ли от вятър разлюлявана?
\par 8 Но какво излязохте да видите? Човек в меки дрехи ли облечен? Ето, тия, които носят меки дрехи, са в царски дворци.
\par 9 Но защо излязохте? Пророк ли да видите? Да, казвам ви, и повече от пророк.
\par 10 Това е онзи, за когото е писано: "Ето, Аз изпращам вестителя Си пред Твоето лице, Който ще устрои пътя Ти пред Тебе."
\par 11 Истина ви казвам: Между родените от жени, не се е въздигнал по-голям от Йоана Кръстителя; обаче, най-малкият в небесното царство, е по-голям от него.
\par 12 А от дните на Йоана Кръстителя до сега небесното царство на сила се взема, и които се насилят го грабват.
\par 13 Защото всичките пророци и законът пророкуваха до Йоана;
\par 14 и, ако искате до го приемете, тоя е Илия, който имаше да дойде.
\par 15 Който има уши да слуша, нека слуша.
\par 16 А на какво да уприлича това поколение? То прилича на деца, седящи по пазарите, които викат на другарите си, казвайки:
\par 17 Свирихме ви, и не играхте; ридахме, и не жалеехте.
\par 18 Защото дойде Йоан, който нито ядеше, нито пиеше; и казват: Бяс има.
\par 19 Дойде Човешкият Син, Който яде и пие; и казват: Ето човек лаком и винопиец, приятел на бирниците и на грешниците! Но пак, мъдростта се оправдава от делата си.
\par 20 Тогава почна да укорява градовете, гдето се извършиха повечето от Неговите велики дела, за гдето не се покаяха:
\par 21 Горко ти Хоразине! Горко ти, Витсаидо! Защото, ако бяха се извършили в Тир и Сидон великите дела, които се извършиха у вас, те отдавна биха се покаяли във вретище и пепел.
\par 22 Но казвам ви, на Тир и Сидон наказанието ще бъде по-леко в съдния ден, отколкото на вас.
\par 23 И ти, Капернауме, до небесата ли ще се издигнеш? До ада ще слезеш! Защото, ако бяха се извършили в Содом великите дела, които се извършиха в тебе, той би и до днес останал.
\par 24 И казвам ви, че в същия ден наказанието на содомската земя ще бъда по-леко отколкото на тебе.
\par 25 В онова време Исус проговори, казвайки: Благодаря Ти, Отче, Господи на небето и земята, за гдето си утаил това от мъдрите и разумните, а си го открил на младенците.
\par 26 Да, Отче защото така Ти се видя угодно.
\par 27 Всичко Ми е предадено от Отца Ми; и, освен Отца, никой не познава Сина; нито познава някой Отца, освен Синът и оня, комуто Синът би благоволил да Го открие.
\par 28 Дойдете при Мене всички, които се трудите и сте обременени, и Аз ще ви успокоя.
\par 29 Вземете Моето иго върху си, и научете се от Мене; защото съм кротък и смирен на сърце; и ще намерите покой на душите си.
\par 30 Защото Моето иго е благо, и Моето бреме е леко.

\chapter{12}

\par 1 По онова време, в една събота, Исус минаваше през посевите; а учениците Му, като огладняха, почнаха да късат класове и да ядат.
\par 2 А фарисеите, като видяха това, рекоха Му: Виж Твоите ученици вършат каквото не е позволено да се върши в събота.
\par 3 А Той им рече: Не сте ли чели що стори Давид, когато огладня той и мъжете, които бяха с него,
\par 4 как влезе в Божия дом и яде от присъствените хлябове, които не бе позволено да яде ни той, нито ония, които бяха с него, а само свещениците?
\par 5 Или не сте ли чели в закона, че в съботен ден свещениците в храма нарушават съботата, и пак не са виновни.
\par 6 Но казвам ви, че тук има повече от храма.
\par 7 Но ако бяхте знаели що значи тая дума: "Милост искам, а не жертва", не бихте осъдили невинните.
\par 8 Защото Човешкият Син е Господар на съботата.
\par 9 И като замина оттам, дойде в синагогата им.
\par 10 И ето човек с изсъхнала ръка; и, за да обвинят Исуса, попитаха Го казвайки: Позволено ли е човек да изцелява в събота?
\par 11 И Той им каза: Кой човек от вас, ако има една овца, и тя в съботен ден падне в яма, не ще я улови и извади?
\par 12 А колко е по-скъп човек от овца! Затова позволено е да се прави добро в съботен ден.
\par 13 Тогава казва на човека: Простри ръката си. И той я простря; и тя стана здрава като другата.
\par 14 А фарисеите, като излязоха, наговориха се против Него, как да Го погубят.
\par 15 Но Исус, като позна това, оттегли се оттам; и мнозина тръгнаха подире Му, и Той ги изцели всички.
\par 16 И заръча им да го не разгласяват;
\par 17 за да се сбъдне реченото чрез пророк Исаия, който казва:
\par 18 "Ето Моят служител, Когото избрах Моят възлюбен, в Когото е благоволението на душата Ми; Ще положа Духа Си на Него, И Той ще възвести съдба на народите.
\par 19 Няма да се скара, нито да извика, Нито ще чуе някой гласа Му по площадите;
\par 20 Смазана тръстика няма да пречупи, И замъждял фитил няма да угаси, Докато изведе правосъдието към победа.
\par 21 И в Неговото име народите ще се надяват".
\par 22 Тогава доведоха при Него един хванат от бяс, сляп и ням; и го изцели, тъй щото немият и проговори и прогледа.
\par 23 И всичките множества се смаяха и думаха: Да не би Този да е Давидовият син?
\par 24 А фарисеите, като чуха това, рекоха: Тоя не изгонва бесовете, освен чрез началника на бесовете, Веелзевула.
\par 25 А Исус, като знаеше техните помисли, рече им: Всяко царство, разделено против себе си, запустява; и никой град или дом, разделен против себе си няма да устои.
\par 26 Ако Сатана изгонва Сатана, той се е разделил против себе си; тогава как ще устои неговото царство?
\par 27 При това, ако аз чрез Веелзевула изгонвам бесовете, чрез кого ги изгонват вашите възпитаници? Затова, те ще ви бъдат съдии.
\par 28 Но ако Аз чрез Божия Дух изгонвам бесовете, то Божието царство е дошло върху вас.
\par 29 Или как може да влезе някой в къщата на силния човек и да му ограби покъщината, ако първо не върже силния? - Тогава ще ограби къщата му.
\par 30 Който не е с Мене, той е против Мене; и който не събира с Мене, разпилява.
\par 31 Затова ви казвам: Всеки грях и хула ще се прости на човеците; но хулата против Духа няма да се прости.
\par 32 И ако някой каже дума против Човешкия Син, ще му се прости; но ако някой каже дума против Святия Дух, няма да му се прости, нито в тоя свят(Или: век.), нито в бъдещия.
\par 33 Или направете дървото добро, и плода му добър; или направете дървото лошо, и плода му лош; защото от плода се познава дървото.
\par 34 Рожби ехидни! Как можахте да говорите добро, като сте зли? Защото от онова, което препълва сърцето, говорят устата.
\par 35 Добрият човек от доброто си съкровище изважда добри неща; а злият човек от злото си съкровище изважда зли неща.
\par 36 И казвам ви, че за всяка празна дума, която кажат човеците, ще отговарят в съдния ден.
\par 37 Защото от думите си ще се оправдаеш, и от думите си ще се осъдиш.
\par 38 Тогава някой от книжниците и фарисеите Му отговориха, казвайки: Учителю, искаме да видим знамение от Тебе.
\par 39 А Той в отговор им рече: Нечестиво и прелюбодейно поколение иска знамение, но друго знамение няма да му се даде, освен знамението на пророк Йона.
\par 40 Защото, както Йона беше в корема на морското чудовище три дни и три нощи, така и Човешкият Син ще бъде в сърцето на земята три дни и три нощи.
\par 41 Ниневийските мъже ще се явят на съда с това поколение, и ще го съдят, защото те се покаяха чрез Йоновата проповед; а ето, тука има повече от Йона.
\par 42 Южната царица ще се яви на съда с това поколение и ще го осъди, защото тя дойде от краищата на земята за да чуе Соломоновата мъдрост; а, ето, тука има повече от Соломона.
\par 43 Когато нечистият дух излезе от човека, той минава през безводни места да търси покой, и не намира.
\par 44 Тогава казва: Ще се върна в къщата си отгдето съм излязъл. И, като дойде намира я празна, пометена и наредена.
\par 45 Тогава отива и взема при себе си седем други духове, по-зли от него, и, като влязат, живеят там; и последното състояние на оня човек става по-лошо от първото. Също така ще бъде и на това нечестиво поколение.
\par 46 Когато Той още говореше на народа, ето, майка Му и братята Му стояха вън и искаха да Му говорят.
\par 47 И някой си Му рече: Ето, майка Ти и братята Ти стоят вън и искат да Ти говорят.
\par 48 А той в отговор рече на този, който Му каза това: Коя е майка Ми? И кои са братята Ми?
\par 49 И като простря ръка към учениците Си рече: Ето майка Ми братята Ми!
\par 50 Защото, който върши волята на Отца Ми, Който е на небесата, той ми е брат и сестра и майка.

\chapter{13}

\par 1 В същия ден Исус излезе из къщи и седна край езерото.
\par 2 И събраха се до Него големи множества, така щото влезе и седна в една ладия; а целият народ стоеше на брега.
\par 3 И говореше им много с притчи, казвайки: Ето, сеячът излезе да сее;
\par 4 и като сееше някои зърна паднаха край пътя: птиците дойдоха и ги изкълваха.
\par 5 А други паднаха на канаристите места, гдето нямаше много пръст; и твърде скоро поникнаха, защото нямаше дълбока почва;
\par 6 и като изгря слънцето, пригоряха, и понеже нямаха корен изсъхнаха.
\par 7 Други пък паднаха между тръните; тръните пораснаха и ги заглушиха.
\par 8 А други паднаха на добра земя, и дадоха плод, кое стократно, кое шестдесет, кое тридесет.
\par 9 Който има уши [да слуша], нека слуша.
\par 10 Тогава се приближиха учениците Му и Му казаха: Защо им говориш с притчи?
\par 11 А Той в отговор им каза: Защото на вас е дадено да знаете тайните на небесното царство, а на тях не е дадено.
\par 12 Защото който има, нему ще се даде, и ще има изобилие; а който няма, от него ще се отнеме и това, което има.
\par 13 Затова им говоря с притчи, защото гледат, а не виждат; чуят а не слушат, нито разбират.
\par 14 На тях се изпълнява Исаевото пророчество, което казва: "С уши ще чуете, а никак няма да разберете; И с очи ще гледате, а никак няма да видите.
\par 15 Защото сърцето на тия люде е задебеляло. И с ушите си тежко чуват, И очите си склопиха; Да не би да видят с очите си, И да чуят с ушите си, И да разберат със сърцето си, И да се обърнат, И Аз да ги изцеля".
\par 16 А вашите очи са блажени, защото виждат, и ушите ви, защото чуват.
\par 17 Защото истина ви казвам, че мнозина пророци и праведници са желали да видят това, което вие виждате, но не видяха, и да чуят това, което вие чувате, но не чуха.
\par 18 Вие, прочее, чуйте какво значи притчата за сеяча.
\par 19 При всекиго, който чуе словото на царството и не го разбира, дохожда лукавият и грабва посяното в сърцето му; той е посяното край пътя.
\par 20 А посяното на канаристите места е оня, който чуе словото и веднага с радост го приема;
\par 21 корен, обаче, няма в себе си, но е привременен; и когато настане напаст или гонение поради словото, на часа се съблазнява.
\par 22 А посяното между тръните е оня, който чува словото; но светските грижи и примамката на богатството заглушават словото, и той става безплоден.
\par 23 А посяното на добра земя е оня, който чуе словото и го разбира, който и дава плод, и принася кой стократно, кой шестдесет, кой тридесет.
\par 24 Друга притча им предложи, като каза: Небесното царство се уприличава на човек, който е посял добро семе на нивата си;
\par 25 но, когато спяха човеците, неприятелят му дойде и пося плевели между житото, и си отиде.
\par 26 А когато поникна стволът и завърза плод, тогава се появиха и плевелите.
\par 27 А слугите на домакина дойдоха и му казаха: Господине, не пося ли добро семе на нивата си? Тогава откъде са плевелите?
\par 28 Той им каза: Някой неприятел е сторил това. А слугите му казаха: Като е тъй искаш ли да идем да го оплевим?
\par 29 А той каза: Не искам; да не би, като плевите плевелите, да изскубете заедно с тях и житото.
\par 30 Оставете да растат и двете заедно до жетва; а във време на жетва ще река на жетварите: Съберете първо плевелите, и вържете ги на снопове за изгаряне, а житото приберете в житницата ми.
\par 31 Друга притча им предложи, казвайки: Небесното царство прилича на синапово зърно, което човек взе и го пося на нивата си;
\par 32 което наистина е по-малко от всичките семена, но, когато порасте, е по-голямо от злаковете, и става дърво, така щото небесните птици дохождат и се подслоняват по клончетата му.
\par 33 Друга притча им каза: Небесното царство прилича на квас, който една жена взе и замеси в три мери брашно, докле вкисна всичкото.
\par 34 Всичко това Исус изказа на народа с притчи, и без притчи не им говореше;
\par 35 за да се изпълни реченото чрез пророка, който казва: "Ще отворя устата Си в притчи; Ще изкажа скритото още от създанието на света".
\par 36 Тогава Той остави народа и дойде вкъщи. И учениците Му се приближиха при Него и казаха: Обясни ни притчата за плевелите на нивата.
\par 37 А в отговор Той каза: Сеячът на доброто семе е Човешкият Син;
\par 38 нивата е светът; доброто семе, това са чадата на царството; а плевелите са чадата на лукавия;
\par 39 неприятелят, който ги пося, е дяволът; жетвата е свършекът на века; а жетварите са ангели.
\par 40 И тъй, както събират плевелите и ги изгарят в огън, така ще бъде и при свършека на века.
\par 41 Човешкият Син ще изпрати ангелите Си, които ще съберат от царството Му всичко що съблазнява, и ония, които вършат беззаконие,
\par 42 и ще ги хвърлят в огнената пещ; там ще бъде плач и скърцане със зъби.
\par 43 Тогава праведните ще блеснат като слънцето в царството на Отца Си. Който има уши [да слуша], нека слуша.
\par 44 Небесното царство прилича на имане скрито в нива, което, като го намери човек, скрива го, и в радостта си отива, продава всичко що има, и купува оная нива.
\par 45 Небесното царство прилича още на търговец, който търсеше хубави бисери,
\par 46 и, като намери един скъпоценен бисер, отиде, продаде всичко що имаше и го купи.
\par 47 Небесното царство прилича още на мрежа, хвърлена в езерото, която събира риби от всякакъв вид,
\par 48 и, като се напълни изтеглиха я на брега, седнаха и прибраха добрите в съдове, а лошите изхвърлиха.
\par 49 Така ще бъде и при свършека на века; ангелите ще излязат и ща отлъчат нечестивите измежду праведните,
\par 50 и ще ги хвърлят в огнената пещ; там ще бъде плач и скърцане със зъби,
\par 51 [Исус им казва]: Разбрахте ли всичко това? Те Му казват: Разбрахме.
\par 52 А Той им рече: Затова, всеки книжник, който е учил за небесното царство, прилича на домакин, който изважда от съкровището си ново и старо.
\par 53 Тогава Исус, когато свърши тия притчи замина си оттам.
\par 54 И като дойде в родината Си, поучаваше ги в синагогите им, така щото те се чудеха и думаха: От къде са на Тогова тая мъдрост и и тия велики дела?
\par 55 Не е ли Тоя син на дърводелеца? Майка Му не казва ли се Мария, и братята Му Яков и Йосиф, Симон и Юда?
\par 56 И сестрите Му не са ли всички при нас? От къде е, прочее, на този всичко това?
\par 57 И съблазняваха се в Него. А Исус им рече: Никой пророк не е без почит, освен в своята родина и в своя дом.
\par 58 И не извърши там много велики дела, поради неверието им.

\chapter{14}

\par 1 В онова време четверовластникът Ирод чу слуха, който се носеше за Исуса;
\par 2 и рече на слугите си: Тоя е Йоан Кръстител; той е възкръснал от мъртвите, и затова тия сили действуват чрез него.
\par 3 Защото Ирод беше хванал Йоана и беше го вързал и турил в тъмница, поради Иродиада, жената на брата си Филипа;
\par 4 понеже Йоан му казваше: Не ти е позволено да я имаш.
\par 5 И искаше да го убие, но се боеше от народа, защото го имаха за пророк.
\par 6 А когато настана рожденият ден на Ирода, Иродиадината дъщеря игра всред събраните и угоди на Ирода.
\par 7 Затова той с клетва се обеща да й даде каквото и да му поиска.
\par 8 А тя, подучена от майка си, каза: Дай ми тука на блюдо главата на Йоана Кръстителя.
\par 9 Царят се наскърби; но заради клетвите си, и заради седящите с него, заповяда да й се даде.
\par 10 И прати да обезглавят Йоана в тъмницата.
\par 11 И донесоха главата му на блюдо и дадоха я на девойката, а тя я занесе на майка си.
\par 12 А учениците му като дойдоха, дигнаха тялото и го погребаха; и отидоха и казаха на Исуса.
\par 13 А Исус, като чу това, оттегли се оттам, с ладия на уединено място настрана; а народът като разбра, отиде подир него пеша от градовете.
\par 14 И Той, като излезе, видя голямо множество, смили се за тях, и изцели болните им.
\par 15 А като се свечери, учениците дойдоха при Него, и рекоха: Мястото е уединено, и времето е вече напреднало; разпусни народа да си отиде пи селата да си купи храна.
\par 16 А Исус им рече: Няма нужда да отидат; дайте им вие да ядат.
\par 17 А те му казаха: Имаме тук само пет хляба и две риби.
\par 18 А Той рече: Донесете ги тук при Мене.
\par 19 Тогава, като заповяда на народа да насяда на тревата, взе петте хляба и двете риби, погледна към небето и благослови; и като разчупи хлябовете, даде ги на учениците, а учениците на народа.
\par 20 И всички ядоха и се наситиха; и дигнаха останалите къшеи, дванадесет пълни коша.
\par 21 А ония, които ядоха, бяха около пет хиляди мъже, освен жени и деца.
\par 22 И на часа Исус накара учениците да влязат в ладията и да отидат преди Него на отвъдната страна, докле разпусне народа.
\par 23 И като разпусна народа, качи се на бърдото да се помоли насаме. И като се свечери, Той беше там сам.
\par 24 А ладията бе вече всред езерото, блъскана от вълните, защото вятърът беше противен.
\par 25 А в четвъртата стража на нощта Той дойде към тях, като вървеше по езерото.
\par 26 А учениците, като Го видяха да ходи по езерото; смутиха се и думаха, че е призрак, и от страх извикаха.
\par 27 И Исус веднага им проговори, казвайки: Дерзайте! Аз съм; не бойте се.
\par 28 И Петър в отговор Му рече: Господи, ако си Ти, кажи ми да дойда при Тебе по водата.
\par 29 А Той рече: Дойди. И Петър слезе от ладията и ходеше по водата да иде при Исуса.
\par 30 Но като виждаше вятърът [силен], уплаши се и, като потъваше, извика, казвайки: Господи избави ме!
\par 31 И Исус веднага простря ръка, хвана го, и му рече: Маловерецо, защо се усъмни?
\par 32 И като влязоха в ладията, вятърът утихна.
\par 33 А ония, които бяха в ладията, Му се поклониха и казаха: Наистина Ти си Божий Син.
\par 34 И като преминаха езерото, дойдоха в генисаретската земя.
\par 35 И когато Го познаха тамошните мъже, разпратиха по цялата оная околност и доведоха при Него всичките болни;
\par 36 и молеха Го да се допрат само до полата на дрехата Му; и колкото се допряха, се изцелиха.

\chapter{15}

\par 1 Тогава дойдоха при Исуса фарисеи и книжници от Ерусалим и казаха:
\par 2 Защо Твоите ученици престъпват преданието на старейшините? Понеже не си мият ръцете, когато ядат хляб.
\par 3 А Той в отговор им каза: Защо и вие заради вашето предание престъпвате Божията заповед?
\par 4 Защото Бог каза: "Почитай баща си и майка си"; и - "Който злослови баща или майка, непременно да се умъртви".
\par 5 Но вие казвате: Който рече на баща си или майка си: Това мое имане, с което би могъл да си помогнеш, е подарено Богу,
\par 6 - той да не почита баща си, [или майка си]. Така, заради вашето предание, осуетихте Божията дума.
\par 7 Лицемери! Добре е пророкувал Исаия за вас, като е казал: -
\par 8 "Тия люде [се приближават при Мене с устата си, и] Ме почитат с устните си; Но сърцето им далеч отстои от Мене.
\par 9 Обаче напразно Ми се кланят, като преподават за поучения човешки заповеди".
\par 10 И като повика народа, рече им: Слушайте и разбирайте!
\par 11 Това, което влиза в устата, не осквернява човека; но това, което излиза от устата, то осквернява човека.
\par 12 Тогава се приближиха учениците Му и рекоха: Знаеш ли, че фарисеите се съблазниха като чуха тая дума?
\par 13 А Той в отговор рече: Всяко растение, което Моят небесен Отец не е насадил, ще се изкорени.
\par 14 Оставете ги; те са слепи водачи; а слепец слепеца ако води, и двамата ще паднат в ямата.
\par 15 Петър в отговор Му рече: Обясни ни тая притча.
\par 16 А Той каза: И вие ли сте още без разумление.
\par 17 Не разбирате ли, че всичко що влиза в устата, минава в корема, и се изхвърля в захода?
\par 18 А онова, което излиза из устата, произхожда от сърцето, и то осквернява човека.
\par 19 Защото от сърцето произлизат зли помисли, убийства, прелюбодейства, блудства, кражби, лъжесвидетелства, хули.
\par 20 Тия са нещата, които оскверняват човека; а да яде с немити ръце, това не го осквернява.
\par 21 И, като излезе оттам, Исус се оттегли в тирските и сидонските страни.
\par 22 И, ето, една ханаанка излезе от ония места и извика, казвайки: Смили се за мене Господи, Сине Давидов; дъщеря ми зле се мъчи от бяс.
\par 23 Но Той не й отговори ни дума. Учениците дойдоха и Му се молеха, като рекоха: Отпрати я, защото вика подире ни.
\par 24 А Той в отговор каза: Аз не съм пратен, освен до загубените овце от Израилевия дом.
\par 25 А тя дойде, кланяше Му се и казваше: Господи помогни ми.
\par 26 Той в отговор рече: Не е прилично да се вземе хляба на децата и да се хвърли на кученцата.
\par 27 А тя рече: Така, Господи; ни и кученцата ядат от трохите, които падат от трапезата на господарите им.
\par 28 Тогава Исус в отговор й рече: О жено, голяма е твоята вяра; нека ти бъде според желанието. И дъщеря й оздравя в същия час.
\par 29 И като замина оттам, Исус дойде при галилейското езеро; и качи се на бърдото и седеше там.
\par 30 И дойдоха при Него големи множества, които имаха със себе си куци, слепи, неми, недъгави и много други, и сложиха ги пред нозете Му; и Той ги изцели;
\par 31 Така щото народът се чудеше, като гледаше неми да говорят, недъгави оздравели, куци да ходят, и слепи да гледат. И прославиха Израилевия Бог.
\par 32 А Исус повика учениците Си и рече: Жално Ми е за народа, защото три дни вече седят при Мене и нямат що да ядат; а не искам да ги разпусна гладни, да не би да им премалее по пътя.
\par 33 Учениците Му казаха: Отгде да имаме в уединено място толкова хляб, че да нахраним такова голямо множество?
\par 34 Исус им каза: Колко хляба имате? А те рекоха: Седем и малко рибки.
\par 35 Тогава заповяда на народа да насядат на земята.
\par 36 И като взе седемте хляба и рибите, благодари и разчупи; и даде на учениците, а учениците на народа.
\par 37 И ядоха всички и наситиха се, и дигнаха останалите къшеи, седем кошници пълни.
\par 38 А ония, които ядоха, бяха четири хиляди мъже, освен жени и деца.
\par 39 И като разпусна народа, влезе в ладията, и дойде в магаданските предели.

\chapter{16}

\par 1 Тогава фарисеите и садукеите дойдоха при Исуса, и, за да Го изпитат, поискаха Му да им покаже знамение от небето.
\par 2 А Той в отговор им рече: Когато се свечери, думате - Времето ще бъде хубаво, защото небето се червенее;
\par 3 а сутрин: Днес времето ще бъде лошо, защото небето се червенее намръщено. Вие знаете да разтълкувате лицето на небето, а знаменията на времената не можете!
\par 4 Зъл и прелюбодеен род иска знамение, но друго знамение няма да му се даде, освен знамението на (пророка) Йона. И остави ги и Си отиде.
\par 5 А учениците, които минаха на отвъдната страна, забравиха да вземат хляб.
\par 6 И Исус им рече: Внимавайте и пазете се от кваса на фарисеите и садукеите.
\par 7 А те разискваха помежду си, думайки: Това е защото не сме взели хляб.
\par 8 А Исус, като разбра това, рече: Маловери, защо разисквате помежду си понеже нямате хляб?
\par 9 Още ли не разбирате, нито помните петте хляба на петте хиляди души, и колко коша събрахте?
\par 10 Нито седемте хляба на четирите хиляди души, и колко кошници събрахте?
\par 11 Как не разбирате, че не заради хляб ви казах да се пазите от кваса на фарисеите и садукеите?
\par 12 Тогава те разбраха, че не им заръча да се пазят от хлебен квас, но от учението на фарисеите и садукеите.
\par 13 А Исус, като дойде в околностите на Кесария Филипова, попита учениците Си, казвайки: Според както казват хората, Човешкият Син Кой е?
\par 14 А те рекоха: Едни казват, че е Йоан Кръстител; други пък - Илия; а други - Еремия, или един от пророците.
\par 15 Казва им: Но според както вие казвате, Кой съм Аз?
\par 16 Симон Петър в отговор рече: Ти си Христос(Т.е. Месия, Помазаник.), Син на живия Бог.
\par 17 Исус в отговор му каза: Блажен си, Симоне, сине Йонов, защото плът и кръв не са ти открили това, но Отец Ми, Който е на небесата.
\par 18 Пък и Аз ти казвам, че ти си Петър(Значи: Канара.) и на тая канара ще съградя Моята църква; и портите на ада няма да й надделеят.
\par 19 Ще ти дам ключовете на небесното царство; и каквото вържеш на земята, ще бъде вързано на небесата, а каквото развържеш на земята, ще бъде развързано на небесата.
\par 20 Тогава заръча на учениците, никому да не казват, че Той е [Исус] Христос.
\par 21 От тогава Исус почна да известява на учениците Си, че трябва да отиде в Ерусалим, и много да пострада от старейшините, главните свещеници и книжниците, и да бъде убит, и на третия ден да бъде възкресен.
\par 22 Тогава Петър Го взе и почна да Го мъмри, като казваше: Бог да Ти се смили, Господи; това никак няма да стане с Тебе.
\par 23 А Той се обърна и рече на Петра: Махни се зад Мене, Сатано; ти си Ми съблазън; защото не мислиш за Божиите неща, а за човешките.
\par 24 Тогава Исус каза на учениците Си: Ако иска някой да дойде след Мене, нека се отрече от себе си, нека дигне кръста си, и така нека Ме последва.
\par 25 Защото, който иска да спаси живота(Или: Душата; и така до края на главата.) си, ще го изгуби; а който изгуби живота си, заради Мене, ще го намери.
\par 26 Понеже какво ще се ползува човек, ако спечели целия свят, а живота си изгуби? Или какво ще даде човек в замяна на живота си?
\par 27 Защото Човешкият Син ще дойде в славата на Отца Си със Своите ангели; и тогава ще въздаде всекиму според делата му.
\par 28 Истина ви казвам: Има някои от стоящите тука, които никак няма да вкусят смърт докле не видят Човешкия Син идещ в царството Си.

\chapter{17}

\par 1 И след шест дни Исус взема Петра, Якова и блата му Йоана, и ги завежда на една висока планина насаме.
\par 2 И преобрази се пред тях; лицето Му светна като слънцето, а дрехите Му станаха бели като светлината.
\par 3 И, ето, явиха им се Мойсей и Илия, които се разговаряха с Него.
\par 4 И Петър проговори, казвайки на Исуса: Господи, добре е да сме тука; ако искаш, аз ще направя тука три скинии(Шатри.), за Тебе една, за Мойсея една и една за Илия.
\par 5 А когато той още говореше, ето, светъл облак ги засени; и ето из облака глас, който каза: Този е Моят възлюбен Син, в Когото е Моето благоволение, Него слушайте.
\par 6 И учениците, като чуха това, паднаха на лицата си, и много се уплашиха.
\par 7 А Исус се приближи при тях, допря се до тях, и рече: Станете, не бойте се.
\par 8 И те, като подигнаха очи, не видяха никой, освен Исуса сам.
\par 9 И като слизаха от планината Исус им заръча, като каза: Никому не съобщавайте за това видение, докле Човешкият Син не възкръсне от мъртвите.
\par 10 Учениците Му го попитаха, казвайки: Защо тогава думат книжниците, че Илия трябва първо да дойде?
\par 11 А Той в отговор рече: Наистина Илия иде, и ще възстанови всичко.
\par 12 Но казвам ви, че Илия вече е дошъл, и не го познаха, но постъпиха с него както си искаха. Също така и Човешкият Син ще пострада от тях.
\par 13 Тогава учениците разбраха, че им говореше за Йоана Кръстителя.
\par 14 И когато дойдоха при народа, приближи се до Него един човек, който коленичи пред Него и каза:
\par 15 Господи, смили се за сина ми, защото е епилептик и зле страда; понеже пада в огъня, и често във водата.
\par 16 И доведох го при Твоите ученици, но те не можаха да го изцелят.
\par 17 Исус в отговор каза: О роде невярващ и извратен, до кога ще бъда с вас? До кога ще ви търпя? Доведете го тука при Мене.
\par 18 И Исус смъмри бяса и той излезе от него; и момчето оздравя в същия час.
\par 19 Тогава учениците дойдоха при Исуса насаме и казаха: Защо ние не можахме да го изгоним?
\par 20 Той им каза: Поради вашето маловерие. Защото истина ви казвам: Ако имате вяра колкото синапово зърно, ще речете на тая планина: Премести се оттука там, и тя ще се премести; и нищо няма да ви бъде невъзможно.
\par 21 [А тоя род не излиза, освен с молитва и пост].
\par 22 И когато седяха в Галилея, Исус им рече: Човешкият Син ще бъде предаден в ръцете на човеците,
\par 23 и ще Го убият; и на третия ден ще бъде възкресен. И те се наскърбиха твърде много.
\par 24 А когато дойдоха в Капернаум, събирачите на двете драхми за храма се приближиха при Петра и казаха: Вашият учител не плаща ли двете драхми?
\par 25 Той рече: Плаща. И когато влезе в къщи, Исус го изпревари и му рече: Какво мислиш, Симоне? Земните царе от кои събират данък или налог? От своите ли хора, или от чужденците?
\par 26 А когато каза: От чужденците, Исус му рече: Като е тъй своите им са свободни.
\par 27 Но, за да не ги съблазним, иди на езерото, хвърли въдица, и измъкни рибата, която първо се закачи, и като разтвориш устата й ще намериш един статир; вземи го и дай им го за Мене и за тебе.

\chapter{18}

\par 1 В същото време учениците дойдоха при Исуса и казаха: Кой е по-голям в небесното царство?
\par 2 А Той повика едно детенце, постави го посред тях, и рече:
\par 3 Истина ви казвам; ако се не обърнете като дечицата, никак няма да влезете в небесното царство.
\par 4 И тъй, който смири себе си като това детенце, той е по-голям в небесното царство.
\par 5 И който приеме едно такова детенце в Мое име, Мене приема.
\par 6 А който съблазни едно от тия малките, които вярват в Мене, за него би било по-добре да се окачеше на врата му един воденичен камък, и да потънеше в морските дълбочини.
\par 7 Горко на света поради съблазните, защото е неизбежно да дойдат съблазните; но горко на онзи човек, чрез когото съблазънта дохожда!
\par 8 Ако те съблазни ръката ти или ногата ти, отсечи я и хвърли я; по-добре е за тебе да влезеш в живота куц или недъгав, отколкото с две ръце или с две нозе да бъдеш хвърлен във вечния огън.
\par 9 И ако те съблазни окото, извади го и хвърли го; по-добре е за тебе да влезеш в живота с едно око, отколкото да имаш две очи и да бъдеш хвърлен в огнения пъкъл.
\par 10 Внимавайте да не презирате ни едно от тия малките, защото ви казвам, че техните ангели на небесата винаги гледат лицето на Отца Ми, Който е на небесата.
\par 11 [защото Човешкият Син дойде да спаси погиналото].
\par 12 Как ви се вижда? Ако някой човек има сто овце и едната от тях се заблуди, не остава ли деветдесетте и девет, и не отива ли по бърдата да търси заблудилата се?
\par 13 И като я намери, истина ви казвам, той се радва за нея повече, отколкото за деветдесетте и девет незаблудили се.
\par 14 Също така не е по волята на Отца ви, Който е на небесата, да загине ни един от тия малките.
\par 15 И ако ти съгреши брат ти, иди, покажи вината му между тебе и него самия. Ако те послуша, спечелил си брата си.
\par 16 Но ако не те послуша, вземи със себе си още един или двама и от устата на двама или трима свидетели да се потвърди всяка дума.
\par 17 И ако не послуша тях, кажи това на църквата; че ако не послуша и църквата, нека ти бъде като езичник и бирник.
\par 18 Истина ви казвам: Каквото вържете на земята, ще бъде вързано на небесата; и каквото развържете на земята, ще бъде развързано на небесата.
\par 19 Пак ви казвам, че ако двама от вас се съгласят на земята за каквото и да било нещо, което да поискат, ще им бъде дадено от Отца Ми, Който е на небесата.
\par 20 Защото, гдето двама или трима са събрани в Мое име, там съм и Аз посред тях.
\par 21 Тогава Петър се приближи и Му рече: Господи, до колко пъти, като ми съгреши брат ми, да му прощавам? До седем пъти ли?
\par 22 Исус му рече: Не ти казвам: До седем пъти - до седемдесет пъти по седем.
\par 23 Затова небесното царство прилича на един цар, който поиска да прегледа сметките на слугите си.
\par 24 И когато почна да преглежда докараха при него един, който му дължеше десет хиляди таланта(Равно на 60 000 000 лева зл.).
\par 25 И понеже нямаше с какво да заплати, господарят му заповяда да продадат него, жена му и децата му, и всичко що имаше, и да се плати дълга.
\par 26 Затова слугата падна, кланяше му се, и каза: Господарю, имай търпение към мене, и ще ти платя всичко.
\par 27 И господарят на тоя слуга, понеже го жалеше, пусна го и му прости заема.
\par 28 Но тоя слуга, като излезе, намери един от съслужителите си, който му дължеше сто пеняза(Равно близо на 90 лева зл.); хвана го и го душеше, и каза: Плати това, което ми дължиш.
\par 29 Затова служителят му падна, молеше му се, и каза: Имай търпение към мене и ще ти платя.
\par 30 Но той не искаше, а отиде и го хвърли в тъмница, да лежи докле изплати дълга.
\par 31 А съслужителите му, като видяха станалото, твърде много се наскърбиха; дойдоха и казаха на господаря си всичко, що бе станало.
\par 32 Тогава господарят му го повика и му каза: Нечестиви слуго, аз ти простих целия оня дълг понеже ми се примоли.
\par 33 Не трябваше ли и ти да се смилиш за съслужителя си, както и аз се смилих за тебе?
\par 34 И господарят му се разгневи и го предаде на мъчителите да го изтезават докле изплати целия дълг.
\par 35 Така и Моят небесен Отец ще постъпи с вас, ако не простите от сърце всеки на брата си.

\chapter{19}

\par 1 Когато Исус свърши тия думи, тръгна от Галилея, и дойде в пределите на Юдея отвъд Йордан.
\par 2 И големи множества вървяха подире Му; и Той ги изцели там.
\par 3 Тогава дойдоха при Него фарисеи, които, изпитвайки Го, казаха: Позволено ли е на човека да напусне жена си по всякаква причина?
\par 4 А Той в отговор рече: Не сте ли чели, че Онзи, Който ги е направил, направил ги е от началото мъжко и женско, и е казал:
\par 5 "Затова ще остави човек баща си и майка си и ще се привърже към жена си; и двамата ще бъдат една плът"?
\par 6 Така щото не са вече двама, а една плът. И тъй, онова, което Бог е съчетал, човек да го не разлъчва.
\par 7 Казват му: Тогава Мойсей защо заповяда, мъжът й да й даде разводно писмо и да я напусне?
\par 8 Каза им: Поради вашето коравосърдечие Мойсей ви е оставил да си напущате жените; но отначало не е било така.
\par 9 И казвам ви: Който напусне жена си, освен за прелюбодейство, и се ожени за друга, той прелюбодействува; и който се се ожени за нея, когато бъде напусната, прелюбодействува.
\par 10 Казват Му учениците: Ако е такова задължението на мъжа към жената, по-добре да се не жени.
\par 11 А Той им рече: Не могат всички да приемат тая дума, но ония, на които е дадено.
\par 12 Защото има скопци, които така са родени от утробата на майка си; има пък скопци, които са били скопени от човеци; а има и скопци, които сами себе си са скопили заради небесното царство. Който може да приеме това, нека приеме.
\par 13 Тогава доведоха при Него дечица, за да възложи ръце на тях и да се помоли; а учениците ги смъмраха.
\par 14 А Исус рече: Оставете дечицата, и не ги възпирайте да дойдат при Мене, защото на такива е небесното царство.
\par 15 И възложи ръце на тях, и замина оттам.
\par 16 И ето един момък дойде при Него и рече: Учителю, какво добро да сторя, за да имам вечен живот?
\par 17 А Той му каза: Защо питаш Мене за доброто? Един [Бог] има, Който е добър. Но ако искаш да влезеш в живота пази заповедите.
\par 18 Казва Му: Кои? Исус рече: Тия: Не убивай; Не прелюбодействувай; Не кради; Не лъжесвидетелствувай;
\par 19 Почитай баща си и майка си; и обичай ближния си както себе си.
\par 20 Момъкът Му каза: Всичко това съм пазил [от младостта си]; какво ми още недостига?
\par 21 Исус Му рече: Ако искаш да бъдеш съвършен, иди, продай имота си, и дай на сиромасите; и ще имаш съкровище на небесата; дойди и Ме следвай.
\par 22 Но момъкът, като чу тая дума, отиде си наскърбен, защото беше човек с много имот.
\par 23 А Исус рече на учениците си: Истина ви казвам: Мъчно ще влезе богат в небесното царство.
\par 24 При това ви казвам: По-лесно е камила да мине през иглени уши, отколкото богат да влезе в Божието царство.
\par 25 А учениците, като чуха това, зачудиха се твърде много и думаха: Като е тъй, кой може да се спаси?
\par 26 А Исус ги погледна и рече им: За човеците това е невъзможно; но за Бога всичко е възможно.
\par 27 Тогава Петър в отговор Му рече: Ето, ние оставихме всичко и Те последвахме; ние, прочее, какво ще имаме?
\par 28 А Исус им рече: Истина ви казвам, че във време на обновлението на всичко, когато Човешкият Син ще седне на славния Си престол, вие, които Ме последвахте, тоже ще седнете на дванадесет престола да съдите дванадесетте Израилеви племена.
\par 29 И всеки, който е оставил къщи, или братя, или сестри, или баща, или майка, [или жена], или чада, или ниви, заради Моето име, ще получи стократно и ще наследи вечен живот.
\par 30 Обаче мнозина първи ще бъдат последни, а последните първи.

\chapter{20}

\par 1 Защото небесното царство прилича на стопанин, който излезе при зазоряване да наеме работници за лозето си.
\par 2 И като се погоди с работниците по един пеняз на ден, прати ги на лозето си.
\par 3 И като излезе около третия час, видя други, че стояха на пазара празни;
\par 4 и на тях рече: Идете и вие на лозето; и каквото е право ще ви дам. И те отидоха.
\par 5 Пак, като излезе около шестия и около деветия час направи същото.
\par 6 А, като излезе около единадесетия час, намери други че стоят, и каза им: Защо стоите тука цял ден празни?
\par 7 Те му казаха: Защото никой не ни е условил. Каза им: Идете и вие на лозето, [и каквото е право ще получите].
\par 8 Като се свечери, стопанинът на лозето каза на настойника си: Повикай работниците и плати им надницата, като почнеш от последните и следваш до първите.
\par 9 И тъй, дойдоха условените около единадесетия час, и получиха по един пеняз.
\par 10 А като дойдоха първите, мислеха си, че ще получат повече от един пеняз но и те получиха по един пеняз.
\par 11 И като го получиха, зароптаха против стопанина, като казаха:
\par 12 Тия последните иждивиха само един час; и пак си ги приравнил с нас, които понесохме теготата на деня и жегата.
\par 13 А той в отговор рече на един от тях: Приятелю, не те онеправдавам. Не се ли погоди с мене за един пеняз?
\par 14 Вземи си своето и иди си; моята воля е да дам на тоя последния както и на тебе.
\par 15 Не ми ли е позволено да сторя със своето каквото искам? Или твоето око е завистливо(Гръцки: Лошо.), защото аз съм добър?
\par 16 Така последните ще бъдат първи, а първите последни.
\par 17 И, когато възлизаше Исус за Ерусалим, взе дванадесетте ученици насаме, и по пътя им рече:
\par 18 Ето, възлизаме за Ерусалим, и Човешкият Син ще бъде предаден на главните свещеници и книжници; и те ще Го осъдят на смърт,
\par 19 и ще го предадат на езичниците, за да Му се поругаят, да Го бият и Го разпнат; и на третия ден ще бъде възкресен.
\par 20 Тогава майката на Заведеевите синове се приближи при Него заедно със синовете си, кланяше Му се и искаше нещо от Него.
\par 21 А Той й рече: Какво искаш? Каза Му: Заповядай тия мои двама сина да седнат, един отдясно Ти, а един от ляво Ти в Твоето царство.
\par 22 А Исус в отговор рече: Не знаете какво искате. Можете ли да пиете чашата, която Аз имам да пия? [и да се кръстите с кръщението с което Аз се кръщавам?] Казват Му: Можем.
\par 23 Той им рече: Моята чаша наистина ще пиете, [и с кръщението с което Аз се кръщавам, ще се кръстите]; но да седнете отдясно Ми и отляво Ми не е Мое да дам, а ще се даде на ония, за които е било приготвено от Отца Ми.
\par 24 И десетимата като чуха това, възнегодуваха против двамата братя.
\par 25 Но Исус ги повика и рече: Вие знаете, че управителите на народите господаруват над тях.
\par 26 Но между вас не ще бъде така; но който иска да стане големец между вас, ще ви бъде служител;
\par 27 и който иска да бъде пръв между вас, ще ви бъде слуга;
\par 28 също както и Човешкият Син не дойде да Му служат, но да служи, и да даде живота Си откуп за мнозина.
\par 29 И като излизаха от Ерихон, голямо множество отиваше подире Му.
\par 30 И, ето, двама слепци, седящи край пътя, като чуха, че Исус минавал, извикаха казвайки: Смили се над нас, Господи Сине Давидов!
\par 31 А народът ги смъмряше, за да млъкнат; но те още по-силно викаха, казвайки: Смили се за нас, Господи Сине Давидов!
\par 32 И тъй, Исус се спря, повика ги и каза: Какво искате да ви сторя!
\par 33 Казват Му: Господи, да се отворят очите ни.
\par 34 А Исус се смили и се допря до очите им; и веднага прогледаха и тръгнаха подире Му.

\chapter{21}

\par 1 И като се приближиха до Ерусалим и дойдоха във Витфагия при Елеонския хълм, Исус изпрати двама ученици и рече им:
\par 2 Идете в селото, което е насреща ви; и веднага ще намерите вързана ослица и осле с нея; отвържете ги и докарайте Ми ги.
\par 3 И ако някой ви рече нещо, кажете: на Господа трябват; и веднага ще ги изпрати.
\par 4 А това стана за да се сбъдне реченото от пророка, който казва:
\par 5 "Речете на Сионовата дъщеря: Ето, твоят Цар иде при тебе, кротък и възседнал на осел и на осле, рожба на ослица".
\par 6 И тъй учениците отидоха и сториха както им заръча Исус;
\par 7 докараха ослицата и ослето, и намятаха на тях дрехите си; и Той възседна върху тях.
\par 8 А по-голямата част от множеството напостлаха дрехите си по пътя; други пък сечеха клони от дърветата и постилаха ги по пътя.
\par 9 А множествата, които вървяха пред Него, и които идеха изподире, викаха казвайки: Осана на Давидовия Син! Благословен, който иде в Господното име! Осана във висините!
\par 10 И когато влезе в Ерусалим целият град се раздвижи; и казваха: Кой е тоя?
\par 11 А народът казваше: Той е пророкът Исус, Който е от Назарет Галилейски.
\par 12 А Исус влезе в Божия храм, и изпъди всички, които продаваха и купуваха в храма, и прекатури масите на среброменителите, и столовете на ония, които продаваха гълъбите, и каза им:
\par 13 Писано е: "Домът ми ще се нарече молитвен дом" а вие го правите разбойнически вертеп.
\par 14 И някои слепи и куци дойдоха при Него в храма; и Той ги изцели.
\par 15 А главните свещеници и книжници, като видяха чудесните дела, които стори и децата, които викаха в храма, казвайки: Осана на Давидовия Син! възнегодуваха и рекоха Му:
\par 16 Чуваш ли какво казват тия? А Исус им каза: Чувам. Не сте ли никога чели тая дума: - "Из устата на младенците и сучещите Приготвил си хвала?"
\par 17 И когато ги остави, излезе вън от града до Витания, гдето и пренощува.
\par 18 А на сутринта, когато се връщаше в града, огладня.
\par 19 И като видя една смоковница край пътя, дойде при нея, но не намери нищо на нея, само едни листа; и рече й: Отсега нататък да няма плод от тебе до века. И смоковницата изсъхна на часа.
\par 20 И учениците, които видяха това, почудиха се и рекоха: Как на часа изсъхна смоковницата?
\par 21 А Исус в отговор им рече: Истина ви казвам: Ако имате вяра, и не се усъмните, не само ще извършите стореното на смоковницата, но даже, ако речете на тоя хълм: Дигни се и хвърли се в морето, ще стане.
\par 22 И всичко, каквото и да поискате в молитва, като вярвате, ще получите.
\par 23 И когато дойде в храма, главните свещеници и народните старейшини дойдоха при Него, като поучаваше, и казаха: С каква власт правиш тия неща? И кой Ти е дал тая власт?
\par 24 А Исус в отговор им каза: Ще ви задам и Аз един въпрос, на който, ако ми отговорите, то и Аз ще ви кажа с каква власт правя тия неща.
\par 25 Йоановото кръщение от къде беше? От небето или от човеците? И те разискваха помежду си, думайки: Ако речем: От небето, Той ще ни каже: Тогава защо не го повярвахте:
\par 26 Но ако речем: От човеците, боим се от народа; защото всички имат Йоана за пророк.
\par 27 И тъй, в отговор на Исуса, казаха: Не знаем. Рече им и Той: Нито Аз ви казвам с каква власт правя тия неща.
\par 28 Но как ви се вижда? Един човек имаше два сина; дойде при първия и му рече: Синко, иди работи днес на лозето.
\par 29 А той в отговор каза: Не искам; но после се разкая и отиде.
\par 30 Дойде и при втория, комуто каза същото. И той в отговор каза: Аз ще ида, господине! Но не отиде.
\par 31 Кой от двамата изпълни бащината си воля? Казват: Първият. Исус им рече: Истина ви казвам, че бирниците и блудниците ви изпреварват в Божието царство.
\par 32 Защото Йоан дойде при вас в пътя на правдата, и не го повярвахте; бирниците обаче и блудниците го повярваха; а вие, като видяхте това, даже не се разкаяхте отпосле да го вярвате.
\par 33 Чуйте друга притча. Имаше един стопанин, който насади лозе, огради го с плет, изкопа в него лин, и съгради кула; и като го даде под наем на земеделци, отиде в чужбина.
\par 34 И когато наближи времето на плодовете изпрати слугите си до земеделците да приберат плодовете му.
\par 35 А земеделците хванаха слугите му, един биха, друг убиха, а трети с камъни замериха.
\par 36 Пак изпрати други слуги, повече на брой от първите; и на тях сториха същото.
\par 37 Най-после изпрати при тях сина си, като думаше: Ще почетат сина ми.
\par 38 Но земеделците, като видяха сина, рекоха помежду си: Той е наследникът; елате да го убием и да присвоим наследството му.
\par 39 И като го хванаха, изхвърлиха го вън от лозето и го убиха.
\par 40 И тъй, когато си дойде стопанинът на лозето, какво ще стори на тия земеделци?
\par 41 Казват Му: Злосторниците люто ще погуби, а лозето ще даде под наем на други земеделци, които ще му дават плодовете на времето им.
\par 42 Исус им каза: Не сте ли никога прочели в писанията тая дума: "Камъкът, който отхвърлиха зидарите Той стана глава на ъгъла; От Господа е това. И чудно е в нашите очи"?
\par 43 Затова ви казвам, че Божието царство ще се отнеме от вас, и ще се даде на народ, който принася плодовете му.
\par 44 И който падне върху тоя камък ще се смаже; а върху когото падне, ще се пръсне.
\par 45 И главните свещеници и фарисеите, като чуха притчите Му, разбраха, че за тях говори;
\par 46 но, когато поискаха да Го хванат, побояха се от народа понеже Го считаше за пророк.

\chapter{22}

\par 1 И Исус почна пак да им говори с притчи, като казваше:
\par 2 Небесното царство прилича на цар, който направи сватба на сина си.
\par 3 Той разпрати слугите си да повикат поканените на сватбата; но те не искаха да дойдат.
\par 4 Пак изпрати други слуги, казвайки: Речете на поканените: Ето, приготвих обяда си; юнците ми и угоените са заклани, и всичко е готово; дойдете на сватба.
\par 5 но те занемариха поканата, и разотидоха се, един на своята нива, а друг на търговията си;
\par 6 а останалите хванаха слугите му и безсрамно ги оскърбиха и убиха.
\par 7 И царят се разгневи, изпрати войските си и погуби ония убийци, и изгори града им.
\par 8 Тогава казва на слугите си: Сватбата е готова, а поканените не бяха достойни.
\par 9 Затова идете по кръстопътищата и колкото намерите, поканете ги на сватба.
\par 10 И тъй, ония слуги излязоха по пътищата, събраха всички колкото намериха, зли и добри; и сватбата се напълни с гости.
\par 11 А царят, като влезе да прегледа гостите, видя там един човек, който не бе облечен в сватбарска дреха.
\par 12 И каза му: Приятелю, ти как си влязъл тук без да имаш сватбарска дреха? А той мълчеше.
\par 13 Тогава царят рече на служителите: Вържете му нозете и ръцете, и хвърлете го във външната тъмнина; там ще бъде плач и скърцане със зъби.
\par 14 Защото мнозина са поканени, а малцина избрани.
\par 15 Тогава фарисеите отидоха и се съветваха как да Го впримчат в говоренето Му.
\par 16 И пращат при Него учениците си, заедно и Иродианите, да кажат: Учителю, знаем, че си искрен, учиш в истина Божият път и не Те е грижа от никого, защото не гледаш на лицето на човеците.
\par 17 Кажи ни, прочее: Ти как мислиш? Право ли е да даваме данък на Кесаря, или не?
\par 18 А Исус разбра лукавството им, и рече: Защо Ме изпитвате, лицемери?
\par 19 Покажете Ми данъчната монета. И те Му донесоха един пеняз.
\par 20 Той им каза: Чий е този образ и надпис?
\par 21 Казват му: Кесарев. Тогава им казва: Като е тъй, отдавайте Кесаревите на Кесаря, а Божиите на Бога.
\par 22 И като чуха това, те се зачудиха, и оставайки Го, си отидоха.
\par 23 В същия ден дойдоха при Него садукеи, които казват, че няма възкресение и попитаха Го, казвайки:
\par 24 Учителю, Мойсей е казал: Ако някой умре бездетен, брат му да се ожени за жена му, и да въздигне потомък на брата си.
\par 25 А между нас имаше седмина братя; и първият се ожени и умря; и, като нямаше потомък, остави жена си на брата си;
\par 26 също и вторият и третият, до седмият.
\par 27 А подир всички умря и жената.
\par 28 И тъй, във възкресението на кого от седмината ще бъде жена? Защото всички те я имаха.
\par 29 А Исус в отговор им рече: Заблуждавате се, като не знаете писанията нито Божията сила.
\par 30 Защото във възкресението нито се женят, нито се омъжват, но са като [Божии] ангели на небето.
\par 31 А за възкресението на мъртвите, не сте ли чели онова, което Бог ви говори, като казва:
\par 32 "Аз съм Бог Авраамов, Бог Исааков и Бог Яковов"? Той не е Бог на мъртвите, а на живите.
\par 33 И множеството, като чуха това, чудеха се на учението Му.
\par 34 А фарисеите, като чуха, че смълчал садукеите, събраха се заедно.
\par 35 И един от тях, законник, за да Го изпита, зададе Му въпрос:
\par 36 Учителю, коя е голямата заповед в закона?
\par 37 А Той му рече: "Да възлюбиш Господа твоя Бог с цялото си сърце, с цялата си душа и с всичкия си ум".
\par 38 Това е голямата и първа заповед.
\par 39 А втора, подобна на нея, е тая: "Да възлюбиш ближния си, както себе си".
\par 40 На тия две заповеди стоят целият закон и пророците.
\par 41 И когато бяха събрани фарисеите, Исус ги попита, казвайки:
\par 42 Какво мислите за Христа? Чий Син е? Казват Му: Давидов.
\par 43 Казва им: Тогава как Давид чрез Духа Го нарича Господ, думайки:
\par 44 "Рече Господ на моя Господ: Седи отдясно Ми. Докле положа враговете Ти под нозете Ти"?
\par 45 Ако, прочее, Давид Го нарича Господ, как е негов син?
\par 46 И никой не можеше да Му отговори ни дума; нито пък дръзна вече някой от тоя ден да Му задава въпроси.

\chapter{23}

\par 1 Тогава Исус продума на народа и на учениците Си казвайки:
\par 2 На Мойсеевото седалище седят книжниците и фарисеите;
\par 3 затова всичко що ви заръчат, правете и пазете, но според делата им не постъпвайте; понеже говорят, а не вършат.
\par 4 Защото свързват тежки и непоносими бремена, и ги налагат върху плещите на хората, а самите те не искат нито с пръста си да ги помръднат.
\par 5 Но вършат всичките си дела, за да ги виждат хората; защото разширяват филактериите си(Напомнителки на Божия закон.), и правят големи полите на дрехите си,
\par 6 и обичат първото място при угощенията, и първите столове в синагогите,
\par 7 и поздравите по пазарите, и да се наричат от хората: учители.
\par 8 Но вие недейте се нарича учители, защото Един е вашият Учител, а вие всички сте братя.
\par 9 И никого на земята недейте нарича свой отец, защото Един е вашият Отец, Небесният.
\par 10 Недейте се нарича нито наставници, защото Един е вашият Наставник, Христос.
\par 11 А по-големият между вас, нека ви бъде служител.
\par 12 Но който възвишава себе си ще се смири; и който смири себе си ще се възвиси.
\par 13 Но горко вам книжници и фарисеи, лицемери! Защото затваряте небесното царство пред човеците, понеже сами вие не влизате, нито влизащите оставяте да влязат.
\par 14 [Горко вам, книжници и фарисеи, лицемери, защото изпояждате домовете на вдовиците, даже, когато за показ правите дълги молитви; затова ще приемете по-голямо осъждане].
\par 15 Горко вам, книжници и фарисеи, лицемери! Защото море и суша обикаляте за да направите един прозелит; и когато стане такъв, правите го рожба на пъкъла два пъти повече от вас.
\par 16 Горко вам слепи водители! които казвате: Ако някой се закълне в храма, не е нищо; но ако някой се закълне в златото на храма задължава се.
\par 17 Безумни и слепи! Че кое е по-голямо, златото или храмът, който е осветил златото?
\par 18 Казвате още: Ако някой се закълне в олтара, не е нищо, но ако някой се закълне в дара, който е върху него, задължава се.
\par 19 [Безумни и] слепи! Че кое е по-голямо, дарът ли, или олтарът, който освещава дара?
\par 20 Прочее, който се кълне в олтара, заклева се в него и във всичко що е върху него.
\par 21 И който се кълне в храма, заклева се в него и в Онзи, Който обитава в него.
\par 22 И който се кълне в небето, заклева се в Божия престол и в Онзи, Който седи на него.
\par 23 Горко вам книжници и фарисеи, лицемери! Защото давате десятък от гйозума, копара и кимнона, а сте пренебрегнали по-важните неща на закона - правосъдието, милостта и верността, но тия трябваше да правите, а ония да не пренебрегвате.
\par 24 Слепи водители! които прецеждате комара, а камилата поглъщате.
\par 25 Горко вам, книжници и фарисеи, лицемери! Защото чистите външността на чашата и блюдото, а отвътре те са пълни с грабеж и насилие.
\par 26 Слепи фарисеино! Очисти първо вътрешността на чашата и блюдото, за да бъде и външността им чиста.
\par 27 Горко вам книжници и фарисеи, лицемери! Защото приличате на варосани гробници, които отвън се виждат хубави, а отвътре са пълни с мъртвешки кости и с всякаква нечистота.
\par 28 Също така и вие отвън се виждате на човеците праведни, но отвътре сте пълни с лицемерие и беззаконие.
\par 29 Горко вам книжници и фарисеи, лицемери; защото зидате гробниците на пророците, и поправяте гробовете на праведните, и казвате:
\par 30 Ние, ако бяхме живели в дните на бащите си, не бихме съучаствували с тях в проливане кръвта на пророците.
\par 31 Така щото свидетелствувате против себе си, че сте синове на ония, които избиха пророците.
\par 32 Допълнете и вие, прочее, мярката на бащите си.
\par 33 Змии! Рожби ехидни! Как ще избегнете от осъждането в пъкъла?
\par 34 Затова, ето, Аз изпращам до вас пророци, мъдри и книжници; едни от тях ще убиете и ще разпънете, а други от тях ще биете в синагогите си, и ще ги гоните от град в град;
\par 35 за да дойде върху вас всичката праведна кръв проляна на земята, от кръвта на праведния Авел до кръвта на Захария, Варахиевия син, когото убихте между светилището и олтара.
\par 36 Истина ви казвам: Всичко това ще дойде върху туй поколение.
\par 37 Ерусалиме! Ерусалиме! Ти, който избиваш пророците, и с камъни убиваш пратените до тебе, колко пъти съм искал да събера твоите чада, както кокошката прибира пилците си под крилата си, но не искахте!
\par 38 Ето, вашият дом се оставя пуст.
\par 39 Защото казвам ви, отсега няма вече да Ме видите, до когато речете: Благословен, Който иде в Господното име.

\chapter{24}

\par 1 И когато излезе Исус от храма и си отиваше, учениците Му се приближиха да Му покажат зданията на храма.
\par 2 А Той в отговор им рече: Не виждате ли всичко това? Истина ви казвам: Няма да остане тук камък на камък, който да се не срине.
\par 3 И когато седеше на Елеонския хълм, учениците дойдоха при Него насаме и рекоха: Кажи ни, кога ще бъде това? И какъв ще бъде белегът на Твоето пришествие и за свършека на века?
\par 4 Исус в отговор им каза: Пазете се да ви не заблуди някой;
\par 5 защото мнозина ще дойдат в Мое име казвайки: Аз съм Христос, и ще заблудят мнозина.
\par 6 И ще чуете за войни и за военни слухове; но внимавайте да се не смущавате; понеже тия неща трябва да станат; но това още не е свършекът.
\par 7 Защото ще се повдигне народ против народ, и царство против царство; и на разни места ще има глад и трусове.
\par 8 Но всичко това ще бъде само начало на страдания.
\par 9 Тогава ще ви предадат на мъки и ще ви убият; и ще бъдете намразени от всичките народи поради Моето име.
\par 10 И тогава мнозина ще се съблазнят, и един друг ще се предадат, и един друг ще се намразят.
\par 11 И много лъжепророци ще се появят и ще заблудят мнозина.
\par 12 Но понеже ще се умножи беззаконието, любовта на мнозинството ще охладнее.
\par 13 Но който устои до край, той ще бъде спасен.
\par 14 И това благовестие на царството ще бъде проповядвано по цялата вселена за свидетелство на всичките народи; и тогава ще дойде свършекът.
\par 15 Затова, когато видите мерзостта, която докарва запустение, за която говори пророк Даниил, стояща на святото място, (който чете нека разбира),
\par 16 тогава ония които са в Юдея, нека бягат по планините;
\par 17 който се намери на къщния покрив да не слиза да вземе нещата от къщата си;
\par 18 и който се намери на нива да се не връща назад да вземе дрехата си.
\par 19 А горко на непразните и на кърмещите в ония дни!
\par 20 При това, молете се да се не случи бягането ви зиме или в съботен ден;
\par 21 защото тогава ще има голяма скръб, небивала от началото на света до сега, и каквато не ще има.
\par 22 И ако да не се съкратеха ония дни, не би се избавила ни една твар; но заради избраните, ония дни ще се съкратят.
\par 23 Тогава ако някой ви каже: Ето тук е Христос, или: Тука, не вярвайте;
\par 24 защото ще се появят лъжехристи, и лъжепророци, които ще покажат големи знамения и чудеса, така щото да заблудят, ако е възможно, и избраните.
\par 25 Ето предсказах ви.
\par 26 Прочее, ако ви кажат: Ето, Той е в пустинята; не излизайте; или: Ето Той е във вътрешните стаи; не вярвайте.
\par 27 Защото както светкавицата излиза от изток и се вижда дори до запад, така ще бъде пришествието на Човешкия Син.
\par 28 Дето бъде мършата, там ще се съберат и орлите.
\par 29 А веднага след скръбта на ония дни, слънцето ще потъмнее, луната няма да даде светлината си, звездите ще паднат от небето и небесните сили ще се разклатят.
\par 30 Тогава ще се яви на небето знамението на Човешкия Син; и тогава ще заплачат всички земни племена като видят Човешкия Син идещ на небесните облаци със сила и голяма слава.
\par 31 Ще изпрати Своите ангели със силен тръбен глас; и те ще съберат избраните Му от четирите ветрища, от единия край на небето до другия.
\par 32 А научете притчата от смоковницата: Когато клоните й вече омекнат и развият листа, знаете, че е близо лятото.
\par 33 Също така и вие, когато видите всичко това, да знаете, че Той е близо при вратата.
\par 34 Истина ви казвам: Това поколение няма да премине, докле не се сбъдне всичко това.
\par 35 Небето и земята ще преминат, но Моите думи няма да преминат.
\par 36 А за оня ден и час никой не знае, нито небесните ангели, нито Синът, а само Отец.
\par 37 И като бяха Ноевите дни, така ще бъде пришествието на Човешкия Син.
\par 38 Защото, както и в ония дни преди потопа, ядяха и пиеха, женеха се и се омъжваха, до деня до когато Ное влезе в ковчега,
\par 39 и не усетиха, до като дойде потопът и завлече всички, така ще бъде и пришествието на Човешкия Син.
\par 40 Тогава двама ще бъдат на полето; единият се взема, а другият се оставя.
\par 41 Две жени ще мелят на мелницата; едната се взема, а другата се оставя.
\par 42 Затова бдете, защото не знаете в кой ден ще дойде вашият Господ.
\par 43 Но това да знаете, че ако домакинът би знаел в кой час щеше да дойде крадецът, бдял би, и не би оставил да му подкопаят къщата.
\par 44 Затова бъдете и вие готови; защото в час, когато го не мислите, Човешкият Син иде.
\par 45 Кой е, прочее, верният и разумен слуга, когото господарят му е поставил над домочадието си; за да им дава храна навреме?
\par 46 Блажен е оня слуга, чийто господар, като си дойде, го намери, че прави така.
\par 47 Истина ви казвам, че ще го постави над целия си имот.
\par 48 Но, ако оня слуга е зъл(Гръцки: Оня зъл слуга каже.), и каже в сърцето си: Господарят ми се забави,
\par 49 и той почне да бие служителите си, и да яде и да пие с пияниците,
\par 50 господарят на оня слуга ще дойде в ден, когато той не го очаква, и в час, когато не знае,
\par 51 и, като го бие тежко, ще определи неговата участ с лицемерите; там ще бъде плач и скърцане със зъби.

\chapter{25}

\par 1 Тогава небесното царство ще се уприличи на десет девици, които взеха светилниците си, и излязоха да посрещнат младоженеца.
\par 2 А от тях пет бяха неразумни и пет разумни.
\par 3 Защото неразумните, като взеха светилниците, не взеха масло със себе си.
\par 4 Но разумните , заедно със светилниците си взеха и масло в съдовете си.
\par 5 И докато се бавеше младоженецът, додряма се на всичките, и заспаха.
\par 6 А по среднощ се нададе вик: Ето младоженецът [иде!] излизайте да го посрещнете!
\par 7 Тогава всички ония девици станаха и приготвиха светилниците си.
\par 8 А неразумните рекоха на разумните: Дайте ни от вашето масло, защото нашите светилници угасват.
\par 9 А разумните в отговор казаха: Да не би да не стигне и за нас и за вас, по-добре идете при продавачите и си купете.
\par 10 А когато те отидоха да купят, младоженецът пристигна; и готовите влязоха с него на сватбата, и вратата се затвори.
\par 11 После дохождат и другите девици и казват: Господи! Господи! Отвори ни.
\par 12 А той в отговор рече: Истина ви казвам: Не ви познавам.
\par 13 И тъй, бдете; защото не знаете ни деня, ни часа, [в който Човешкият Син ще дойде].
\par 14 Защото е както, когато човек при тръгването си за чужбина, свиква своите слуги, и им предаде имота си.
\par 15 На един даде пет таланта, на друг два, на трети един, на всеки според способността му; и тръгна.
\par 16 Веднага тоя, който получи петте таланта, отиде и търгува с тях, и спечели още пет таланта.
\par 17 Също и тоя, който получи двата спечели още два.
\par 18 А тоя, който получи единия, отиде разкопа в земята, и скри парите на господаря си.
\par 19 След дълго време дохожда господарят на тия слуги и прегледа сметката с тях,
\par 20 И когато се приближи тоя, който бе получил петте таланта, донесе още пет таланта, и рече: Господарю, ти ми предаде пет таланта; ето, спечелих още пет.
\par 21 Господарят му рече: Хубаво, добри е верни слуго! В малкото си бил верен, над многото ще те поставя; влез в радостта на господаря си.
\par 22 Приближи се и тоя, който бе получил двата таланта, и рече: Господарю, ти ми даде два таланта, ето, спечелих още два таланта.
\par 23 Господарят му рече: Хубаво, добри и верни слуго! Над малкото си бил верен, над многото ще те поставя; влез в радостта на господаря си.
\par 24 Тогава се приближи тоя който бе получил един талант, и рече: Господарю, аз те знаех, че си строг човек; жънеш гдето не си сял, и събираш гдето не си пръскал;
\par 25 и като се убоях отидох и скрих таланта ти в земята; ето, имаш своето.
\par 26 А Господарят му в отговор каза: Зли и лениви слуго! Знаел си, че жъна гдето не съм сял, и събирам гдето не съм пръскал;
\par 27 ти, прочее, трябваше да внесеш парите ми на банкерите, и когато си дойдех, щях да взема своето с лихва.
\par 28 Затова, вземете от него таланта и дайте го на този, който има десет таланта.
\par 29 Защото на всекиго, който има, ще се даде, и той ще има изобилие; а от този, който няма от него ще се отнеме и това, което има.
\par 30 А тоя безполезен слуга хвърлете във външната тъмнина; там ще бъде плач и скърцане със зъби.
\par 31 А когато дойде Човешкият Син в славата Си, и всичките [свети] ангели с Него, тогава ще седне на славния Си престол.
\par 32 И ще се съберат пред Него всичките народи; и ще ги отлъчи един от други, както овчарят отлъчва овцете от козите;
\par 33 и ще постави овцете от дясната Си страна, а козите от лявата.
\par 34 Тогава Царят ще рече на тия, които са от дясната Му страна: Дойдете вие благословени от Отца Ми, наследете царството, приготвено за вас от създанието на света.
\par 35 Защото огладнях и Ме нахранихте; ожаднях и Ме напоихте; странник бях, и Ме прибрахте;
\par 36 гол бях и Ме облякохте; болен бях и Ме посетихте; в тъмница бях и Ме споходихте.
\par 37 Тогава праведните в отговор ще му кажат: Господи, кога Те видяхме гладен, и Те нахранихме; или жаден, и Те напоихме?
\par 38 И кога Те видяхме странник, и Те прибрахме, или гол и Те облякохме?
\par 39 И кога Те видяхме болен или в тъмница и Те споходихме?
\par 40 А Царят в отговор ще им рече: Истина ви казвам: Понеже сте направили това на един от тия най-скромни Мои братя, на Мене сте го направили.
\par 41 Тогава ще рече и на тия, които са от лявата Му страна: Идете си от Мене, вие проклети, във вечния огън, приготвен за дявола и за неговите ангели.
\par 42 Защото огладнях и не Ме нахранихте; ожаднях и не Ме напоихте;
\par 43 странник бях, и не Ме облякохте; болен и в тъмница бях, и не Ме посетихте.
\par 44 Тогава и те в отговор ще кажат: Господи, кога Те видяхме гладен, или жаден, или странник, или гол, или болен, или в тъмница и не Ти послужихме?
\par 45 Тогава в отговор ще им рече: Истина ви казвам: Понеже не сте направили това на ни един от тия най-скромните, нито на Мене сте го направили.
\par 46 И тия ще отидат във вечно наказание, а праведните във вечен живот.

\chapter{26}

\par 1 Когато Исус свърши тия думи рече на учениците Си:
\par 2 Знаете, че след два дни ще бъде Пасхата, Човешкият Син ще бъде предаден на разпятие.
\par 3 Тогава главните свещеници и народните старейшини се събраха в двора на първосвещеника, който се наричаше Каиафа,
\par 4 и наговаряха се да уловят Исуса с хитрост и да Го умъртвят;
\par 5 но думаха: Да не е на празника, за да не стане вълнение между народа.
\par 6 А когато Исус беше във Витания, в къщата на прокажения Симон,
\par 7 приближи се до Него една жена, която имаше алвастрен съд с много скъпо миро, което изливаше на главата Му, като бе седнал на трапезата.
\par 8 А учениците, като видяха това, възнегодуваха, казвайки: Защо се прахоса това?
\par 9 Защото това миро можеше да се продаде за голяма сума, която да се раздаде на сиромасите.
\par 10 Но Исус като позна това, рече им: Защо досаждате на жената? понеже тя извърши добро дело на Мене.
\par 11 Защото сиромасите всякога се намират между вас, но Аз не всякога се намирам.
\par 12 Защото тя, като изля това миро върху тялото Ми стори го за да Ме приготви за погребение.
\par 13 Истина ви казвам: Гдето и да се проповядва това благовестие по целия свят, ще се разказва за неин спомен и това, което тя стори.
\par 14 Тогава един от дванадесетте, наречен Юда Искариотски отиде при първосвещениците и рече:
\par 15 Какво обичате да ми дадете и аз ще ви Го предам? И те му претеглиха тридесет сребърника.
\par 16 И от тогава той търсеше удобен случай да им Го предаде.
\par 17 А в първия ден на празника на безквасните хлябове учениците отидоха при Исуса и рекоха: Где искаш да Ти приготвим, за да ядеш пасхата?
\par 18 Той каза: Идете в града при еди кого си и речете му: Учителят казва: Времето Ми е близо, у тебе ще празнувам пасхата с учениците Си.
\par 19 И учениците сториха както им заръча Исус, и приготвиха пасхата.
\par 20 И когато се свечери, Той седна на трапезата с дванадесетте ученика.
\par 21 И като ядяха, рече: Истина ви казвам, че един от вас ще Ме предаде.
\par 22 А те, пренаскърбени, почнаха всички един по един да Му казват: Да не съм аз Господи?
\par 23 Той в отговор рече: Който натопи ръката си заедно с Мене в блюдото, той ще Ме предаде.
\par 24 Човешкият Син отива, както е писано за Него; но горко на този човек, чрез когото Човешкият Син ще бъде предаден! Добре щеше да бъде за този човек, ако не бе се родил.
\par 25 И Юда, който Го предаде, в отговор рече: Да не съм аз, Учителю? Исус му каза: Ти рече.
\par 26 И когато ядяха, Исус взе хляб, благослови, и го разчупи и като го даваше на учениците, рече: Вземете яжте; това е Моето тяло.
\par 27 Взе и чашата, и, като благодари, даде им, и рече: Пийте от нея всички!
\par 28 Защото това е Моята кръв на [новия] завет, която се пролива за прощаване на греховете.
\par 29 Но казвам ви, че отсега няма вече да пия от тоя плод на лозата, до оня ден, когато ще го пия д вас нов в царството на Отца Си.
\par 30 И като изпяха химн, излязоха на Елеонския хълм.
\par 31 Тогава Исус им казва: Вие всички ще се съблазните в Мене тая нощ, защото е писано: "Ще поразя пастиря; и овците на стадото ще се разпръснат".
\par 32 А след като бъда възкресен ще ви изпреваря в Галилея.
\par 33 А Петър в отговор Му рече: Ако и всички да се съблазнят в Тебе, аз никога няма да се съблазня.
\par 34 Исус Му рече: Истина ти казвам, че тая нощ, преди да пее петелът, три пъти ще се отречеш от Мене.
\par 35 Петър му казва: Ако станеше нужда и да умра с Тебе, пак няма да се отрека от Тебе. Същото рекоха и всичките ученици.
\par 36 Тогава Исус идва с тях на едно място наречено Гетсимания; и казва на учениците Си: Седете тука, докле отида там да се помоля.
\par 37 И като взе със Себе Си Петра и двамата Заведееви сина, захвана да скърби и да тъгува.
\par 38 Тогава им казва: Душата Ми е прескръбна до смърт; постойте тук и бдете заедно с Мене.
\par 39 И като пристъпи малко напред, падна на лицето Си, и се молеше, казвайки: Отче Мой, ако е възможно, нека Ме отмине тази чаша; не обаче, както Аз искам, но както Ти искаш.
\par 40 Дохожда при учениците, намира ги заспали, и казва на Петра: Как! не можахте ли ни един час да бдите с Мене?
\par 41 Бдете и молете се, за да не паднете в изкушение. Духът е бодър, а тялото - немощно.
\par 42 Пак отиде втори път и се моли, думайки: Отче Мой, ако не е възможно да Ме отмине това, без да го пия, нека бъде Твоята воля.
\par 43 И като дойде пак намери ги заспали; защото очите им бяха натегнали.
\par 44 И пак ги остави и отиде да се помоли трети път, като каза пак същите думи.
\par 45 Тогава дохожда при учениците и казва им: Още ли спите и почивате? Ето, часът наближи, когато Човешкият Син се предава в ръцете на грешници.
\par 46 Станете да вървим; ето, приближи се тоя, който Ме предава.
\par 47 И когато Той говореше, ето, Юда един от дванадесетте, дойде, и с него голямо множество, с ножове и сопи, изпратени от главните свещеници и народните старейшини.
\par 48 А оня, който Го предаваше, беше им дал знак, казвайки: Когото целуна, Той е; хванете Го.
\par 49 И веднага се приближи до Исуса и рече: Здравей, Учителю! и го целува.
\par 50 А Исус му каза: Приятелю, за каквото си дошъл стори го. Тогава пристъпиха, туриха ръце на Исуса, и Го хванаха.
\par 51 И, ето, един от тия, които бяха с Исуса, простря ръка, измъкна ножа си, и, като удари слугата на първосвещеника, отсече му ухото.
\par 52 Тогава Исус му каза: Повърни ножа си на мястото му, защото всички, които се залавят за нож от нож ще загинат.
\par 53 Или мислиш, че не мога да се помоля на Отца Си, и Той би Ми изпратил още сега повече от дванадесет легиона ангели?
\par 54 Но как биха се сбъднали писанията, че това трябва така да бъде?
\par 55 В същия час рече Исус на народа: Като срещу разбойник ли сте излезли с ножове и сопи да Ме уловите? Всеки ден седях и поучавах в храма и не Ме хванахте.
\par 56 Но всичко това стана за да се сбъднат пророческите писания. Тогава всички ученици Го оставиха и се разбягаха.
\par 57 А тия, които бяха хванали Исуса, заведоха Го у първосвещеника Каиафа, гдето бяха събрани книжниците и старейшините.
\par 58 А Петър вървеше подире Му издалеч до двора на първосвещеника; и като влезе вътре, седна със служителите да види края.
\par 59 А главните свещеници и целият синедрион търсеха лъжливо свидетелство против Исуса, за да Го умъртвят;
\par 60 обаче не намериха, при все че дойдоха много лъжесвидетели. Но сетне дойдоха двама и рекоха:
\par 61 Тоя каза: Мога да разруша Божия храм, и за три дни пак да го съградя.
\par 62 Тогава първосвещеникът стана и Му рече: Нищо ли не отговаряш? Какво свидетелствуват тия против Тебе?
\par 63 Но Исус мълчеше. Първосвещеникът му каза: Заклевам Те в живия Бог да ни кажеш: Ти ли Си Христос Божият Син?
\par 64 Исус му каза: Ти рече. Но казвам ви, от сега нататък ще видите Човешкия Син седящ отдясно на силата и идещ на небесните облаци.
\par 65 Тогава първосвещеникът раздра дрехите си и каза: Той богохулствува! Каква нужда имаме вече от свидетели? Ето сега чухме богохулството. Вие какво мислите?
\par 66 А те в отговор рекоха: Изложи се на смъртно наказание.
\par 67 Тогава Го заплюваха в лицето и Го блъскаха; а други Му удряха плесници и Му казваха:
\par 68 Познай ни, Христе, кой Те удари.
\par 69 А Петър седеше вън на двора; и една слугиня дойде при него и му каза: И ти беше с Исуса галилеянина.
\par 70 А той се отрече пред всички, казвайки: Не разбирам що говориш.
\par 71 И когато излезе в предверието, видя го друга слугиня, и каза на тия, които бяха там: И тоя беше с Исуса Назарянина.
\par 72 И Петър пак се отрече с клетва: Не познавам човека.
\par 73 След малко се приближиха и ония, които стояха наблизо, и рекоха на Петра: Наистина и ти си от тях, защото твоят говор те издава.
\par 74 Тогава той започна да проклина и да се кълне: Не познавам човека. И на часа петелът изпя.
\par 75 И спомни си Петър думата на Исуса, Който беше рекъл: Преди да изпее петелът, три пъти ще се отречеш от Мене. И той излезе вън и плака горко.

\chapter{27}

\par 1 А на сутринта всичките главни свещеници и народни старейшини се съвещаваха против Исуса, да Го умъртвят.
\par 2 И когато Го вързаха, заведоха Го и Го предадоха на управителя Пилата.
\par 3 Тогава Юда, който Го беше предал, като видя, че Исус бе осъден, разкая се и върна тридесетте сребърника на главните свещеници и старейшините и каза:
\par 4 Съгреших, че предадох невинна кръв. А те рекоха: Нам що ни е? Ти гледай.
\par 5 И като хвърли сребърниците в храма, излезе и отиде и се обеси.
\par 6 А главните свещеници взеха сребърниците и рекоха: Не е позволено да ги туряме в храмовата каса, понеже са цена на кръв.
\par 7 И като се съветваха, купиха с тях грънчаревата нива, за погребване на чужденци.
\par 8 Затова оная нива се нарече кръвна нива, както се нарича и до днес.
\par 9 Тогава се изпълни реченото чрез пророк Еремия, който каза: "И взеха тридесетте сребърника, цената на оценения, Когото оцениха някои от изралтяните,
\par 10 и дадоха ги за грънчаревата нива, според както ми заповяда Господ".
\par 11 А Исус застана пред управителя; и управителят Го попита, като каза: Ти Юдейският цар ли си? А Исус му рече: Ти казваш.
\par 12 И когато Го обвиняваха главните свещеници и старейшините, нищо не отговаряше.
\par 13 Тогава Пилат Му казва: Не чуваш ли за колко неща свидетелствуват против Тебе?
\par 14 Но Той не му отговори нито на едно нещо; тъй щото управителят се чудеше много.
\par 15 А на всеки празник управителят имаше обичай да пуща на народа един от затворниците, когото биха поискали.
\par 16 А тогава имаха един прочут затворник, на има Варава.
\par 17 И тъй, като бяха събрани, Пилат им каза: Кого искате да ви пусна: Варава ли или Исуса, наречен Христос?
\par 18 (Понеже знаеше, че от завист го предаваха.
\par 19 При това, като седеше на съдийския престол, жена му изпрати до него да кажат: Не струвай нищо на Тоя праведник; защото днес много пострадах насъне поради Него).
\par 20 А главните свещеници и старейшините убедиха народа да изпроси Варава, а Исуса да погубят.
\par 21 Управителят в отговор им рече: Кого от двамата искате да ви пусна? А те рекоха: Варава.
\par 22 Пилат им казва: Тогава какво да правя с Исуса, наречен Христос? Те всички казват: Да бъде разпнат.
\par 23 А той каза: Че какво зло е сторил? А те още повече закрещяха, казвайки: Да бъде разпнат.
\par 24 И тъй Пилат като видя, че никак не помага, а напротив, че се повдига размирие, взе вода, оми си ръцете пред народа, и каза: Аз съм невинен за кръвта на Тоя праведник; вие гледайте.
\par 25 А целият народ в отговор рече: Кръвта Му да бъде на нас и на чадата ни.
\par 26 Тогава им пусна Варава; а Исуса бй и Го предаде на разпятие.
\par 27 Тогава войниците на управителя заведоха Исуса в преторията, и събраха около Него цялата дружина.
\par 28 И като Го съблякоха, облякоха Го в червена мантия.
\par 29 И оплетоха венец от тръни и го наложиха на главата Му, и туриха тръст в десницата Му; и като коленичаха пред Него, ругаеха Му се, казвайки: Здравей, Царю Юдейски!
\par 30 И заплюваха Го, и взеха тръстта и Го удряха по главата.
\par 31 И след като Му се поругаха, съблякоха Му мантията и Го облякоха с Неговите дрехи и Го заведоха да Го разпнат.
\par 32 А на излизане намериха един киринеец, на име Симон; него заставиха да носи кръста Му.
\par 33 И като стигнаха на едно място, наречено Голгота (което значи лобно място),
\par 34 дадоха Му да пие вино размесено с жлъчка; но Той като вкуси, не прие да пие.
\par 35 И когато Го разпнаха, разделиха си дрехите Му, като хвърлиха жребие.
\par 36 И седяха да Го пазят там.
\par 37 И поставиха над главата Му обвинението Му, написано така: Тоя е Исус, Юдейският Цар.
\par 38 Тогава бидоха разпнати с Него двама разбойници, един отдясно, и един отляво.
\par 39 А минаващите оттам Му се подиграваха, като клатеха глави и думаха:
\par 40 Ти, Който разоряваш храма и за три дни пак го съграждаш, спаси Себе Си; ако Си Божий Син, слез от кръста.
\par 41 Подобно и главните свещеници с книжниците и старейшините Го ругаеха, казвайки:
\par 42 Други е избавил, а пък Себе Си не може да избави! Той е Израилевият Цар! нека слезе сега от кръста, и ще повярваме в Него.
\par 43 Упова на Бога; нека Го избави сега, ако Му е угоден; понеже каза: Божий Син съм.
\par 44 Със същия укор Му се присмиваха и разпнатите с Него разбойници.
\par 45 А от шестия час тъмнина покриваше цялата земя до деветия час.
\par 46 А около деветия час Исус извика със силен глас и каза: Или, Или, лама савахтани? сиреч: Боже Мой, Боже Мой, защо си Ме оставил?
\par 47 Някои от стоящите там, като чуха, думаха: Той вика Илия.
\par 48 И веднага един от тях се завтече, взе гъба, натопи я в оцет, и като я надяна на тръст, даде Му да пие.
\par 49 А другите казваха: Остави! да видим дали ще дойде Илия да Го избави.
\par 50 А Исус, като извика пак със силен глас, издъхна.
\par 51 И, ето, завесата на храма се раздра на две от горе до долу, земята се разтресе, скалите се разпукаха,
\par 52 гробовете се разтвориха и много тела на починалите светии бяха възкресени,
\par 53 (които, като излязоха от гробовете след Неговото възкресение, влязоха в святия град, и се явиха на мнозина).
\par 54 А стотникът и ония, които заедно с него пазеха Исуса, като видяха земетресението и всичко що стана, уплашиха се твърде много, и думаха: Наистина тоя беше Син на Бога.
\par 55 Там бяха още и гледаха отдалеч много жени, които бяха следвали Исуса от Галилея, и Му служеха;
\par 56 между които бяха Мария Магдалина, и Мария майка на Якова и на Иосия, и майката на Заведеевите синове.
\par 57 И когато се свечери, дойде един богаташ от Ариматея, на име Йосиф, който също беше се учил при Исуса.
\par 58 Той дойде при Пилата и поиска Исусовото тяло. Тогава Пилат заповяда да Му се даде.
\par 59 Йосиф, като взе тялото, обви го с чиста плащеница,
\par 60 и го положи в своя нов гроб, който бе изсякъл в скалата; и като привали голям камък на гробната врата, отиде си.
\par 61 А там бяха Мария Магдалина и другата Мария, които седяха срещу гроба.
\par 62 И на следващия ден, който бе денят след приготовлението за празника, главните свещеници и фарисеите се събраха при Пилата и казаха:
\par 63 Господарю, спомнихме си, че Оня измамник приживе рече: След три дни ще възкръсна.
\par 64 Заповядай, прочее, гробът да се пази здраво до третия ден, да не би учениците Му да дойдат и да Го откраднат, и кажат на народа: Възкръсна от мъртвите. Така последната измама ще бъде по-лоша от първата.
\par 65 Пилат им рече: Вземете стража; идете, вардете Го както знаете.
\par 66 Те, прочее, отидоха и вардиха гроба, като запечатаха гроба с помощта на стражата.

\chapter{28}

\par 1 А като се мина съботата, на първия ден от седмицата, дойдоха Мария Магдалина и другата Мария да видят гроба.
\par 2 А, ето, стана голям трус; защото ангел от Господа слезе от небето и пристъпи, отвали камъка, и седна на него.
\par 3 Изгледът му беше като блескавица, и облеклото му бяло като сняг.
\par 4 И в страха си от него стражарите трепереха, и станаха като мъртви.
\par 5 А ангелът проговори, като каза на жените: Вие не се бойте, защото зная, че търсите разпнатия Исус.
\par 6 Няма Го тук; защото възкръсна, както и рече: дойдете и вижте мястото, гдето е лежал Господ.
\par 7 Идете скоро да кажете на учениците Му, че е възкръснал от мъртвите; и, ето, Той отива преди вас в Галилея; там ще го видите; ето казах ви.
\par 8 И те излязоха бързо от гроба със страх и голяма радост, и завтекоха се да известят на учениците Му.
\par 9 И ето Исус ги срещна и рече: Здравейте! А те се приближиха и се хванаха за нозете Му и Му се поклониха.
\par 10 Тогава Исус им рече: Не бойте се; идете кажете на братята Ми да идат в Галилея, и там ще Ме видят.
\par 11 Когато те отидоха, ето, някои от стражата дойдоха в града и известиха на главните свещеници всичко що бе станало.
\par 12 И те, като се събраха със старейшините и се съвещаваха, дадоха на войниците доволно пари, и рекоха:
\par 13 Кажете, че учениците Му дойдоха през нощта и Го откраднаха, когато ние спяхме.
\par 14 И ако слух за това стигне до управителя, ние ще го убедим, а вас ще направим да нямате грижа.
\par 15 Те, прочее, взеха парите, и постъпиха както бяха научени. И това що те казаха се разнесе между юдеите, и продължава даже и до днес.
\par 16 А единадесетте ученика отидоха в Галилея, на бърдото, гдето Исус им определи.
\par 17 И като Го видяха, поклониха Му се; а някои се усъмниха.
\par 18 Тогава Исус се приближи към тях и им говори, казвайки: Даде Ми се всяка власт на небето и земята.
\par 19 Идете, прочее, научете всичките народи, и кръщавайте ги в името на Отца и Сина и Святия Дух,
\par 20 като ги учите да пазят всичко що съм ви заповядал; и ето, Аз съм с вас през, всичките дни до свършека на века. [Амин].

\end{document}