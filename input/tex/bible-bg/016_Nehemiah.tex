\begin{document}

\title{Nehemiah}


\chapter{1}

\par 1 Историята на Неемия, Ахалиевия Син. В месец Хаслев, в двадесетата година, когато бяха в столицата Суса,
\par 2 един от братята ми, Ананий, дойде от Юда, той и някои други: и попитах ги за избавените юдеи оцелели от плена, и за Ерусалим.
\par 3 И те ми рекоха: Останалите, които оцеляха от плена в тамошната област, са в голямо тегло и укор; и стената на Ерусалим е съборена, и портите му са изгорени с огън.
\par 4 А когато чух тия думи, седнах та плаках, и тъжах няколко дни; и постих и се молих пред небесния Бог, казвайки:
\par 5 Моля Ти се, Господи, Боже небесни, велики и страшни Боже, Който пазиш завет и милост към тия, които Те любят и изпълняват Твоите заповеди,
\par 6 дано бъде сега ухото Ти внимателно, и очите Ти отворени, за да слушаш, молитвата на Твоя слуга, която принасям сега пред Тебе ден и нощ за Твоите слуги израилтяните, и като изповядвам греховете на израилтяните, с които сме Ти съгрешили. И аз и бащиният ми дом сме съгрешили;
\par 7 много се развратихме пред Тебе, и не опазихме заповедите, повеленията и съдбите, които Ти даде на слугата си Моисея.
\par 8 Спомни си, моля Ти се, словото, което си заповядал на слугата Си Моисея като си рекъл: Ако престъпите, Аз ще ви разпръсна между племената;
\par 9 но ако се обърнете към Мене, и пазите заповедите Ми и ги вършите, то даже ако има от вас изхвърлени до краищата на небесата, и от там ще ги събера, и ще ги доведа на мястото, което избрах за да настаня там името Си.
\par 10 А тия са Твои слуги и Твои люде, които си избавил с голямата Си сила и с мощната Си ръка.
\par 11 Моля Ти се, Господи, да бъде ухото Ти внимателно към молитвата на слугата Ти, и към молитвата на слугите Ти, които обичат да се боят от името Ти; и направи да благоуспее, моля Ти се, слугата Ти днес, и дай му да намери милост пред тоя човек. (Защото аз бях виночерпец на царя).

\chapter{2}

\par 1 А в месец Нисан, в двадесетата година на цар Артаксеркса, като имаше вино пред него, аз взех виното та го дадох на царя. И като не бях изглеждал по-напред посърнал пред него,
\par 2 затова царят ми рече: Защо е посърнало лицето ти, като не си болен? Това не е друго освен скръб на сърцето. Тогава се уплаших твърде много.
\par 3 И рекох на царя: Да живее царят до века! как да не е посърнало лицето ми, когато градът, мястото на гробищата на бащите ми, е запустял, и портите му изгорени с огън.
\par 4 Тогава царят ми каза: За какво правиш прошение? И помолих се на небесния Бог;
\par 5 после рекох на царя: Ако е угодно на царя, и ако слугата ти е придобил твоето благоволение, изпрати ме в Юда, в града на гробищата на бащите ми, за да го съградя.
\par 6 Царят пак ми рече (като седеше при него и царицата): Колко време ще се продължи пътешествието ти? и кога ще се върнеш? И угодно биде на царя да ме изпрати, като му определих срок.
\par 7 Рекох още на царя: Ако е угодно на царя, нека ми се дадат, писма до областните управители отвъд реката, за да ме препращат докле стигна в Юда,
\par 8 и писмо до пазителя на царското бранище Асаф, за да ми даде дървета да направя греди за вратите на крепостта при дома, и за градската стена, и за къщата, в която ще се настаня. И царят ми разреши всичко, понеже добрата ръка на моя Бог беше над мене.
\par 9 И тъй, дойдох при областните управители отвъд реката та им дадох царските писма. (А царят бе пратил с мене военачалници и конници).
\par 10 А когато аронецът Санавалат и слугата Товия, амонецът, чуха това, оскърбиха се твърде много за дето е дошъл човек да се застъпи за доброто на израилтяните.
\par 11 Така дойдох в Ерусалим и седях там три дни.
\par 12 Тогава станах през нощта, аз и неколцина други с мен, без да явя никому що беше турил моят Бог в сърцето ми да направя за Ерусалим; и друг добитък нямаше с мене освен добитъкът, на който яздех.
\par 13 Излязох нощем през портата на долината та дойдох срещу извора на смока и до портата на бунището, та прегледах ерусалимските стени, как бяха съборени, и портите им изгорени с огън.
\par 14 Сетне минах към портата на извора и към царския водоем; но нямаше място от гдето да мине добитъкът, който бе под мене.
\par 15 Тогава възлязох нощем край потока та прегледах стената; после, като се обърнах, влязох през портата на долината та се върнах.
\par 16 А по-видните мъже не знаеха где ходих или що сторих; и до тогава не бях явил това ни на юдеите, ни на свещениците, ни на благородните, ни на по-видните мъже, ни на другите, които вършеха работата.
\par 17 Тогава им рекох: Вие виждате бедствието, в което се намираме, как Ерусалим е опустошен и портите му са изгорени с огън; елате, да съградим стената на Ерусалим, за да не бъдем вече за урок.
\par 18 И разправих им как ръката на моя Бог беше добра над мене, още и за думите, които царят ми беше казал. И те рекоха: Да станем и да градим. Така засилиха ръцете си за добрата работа.
\par 19 А аронецът Санавалат и слугата Товия, амонецът и арабинът Гисам, когато чуха това, присмяха ни се, презряха ни и думаха: Що е това, което правите? искате да се подигнете против царя ли?
\par 20 А аз като им отговорих рекох им: Небесният Бог, Той ще ни направи да благоуспеем; затова ние, слугите Му, ще станем и ще градим. Вие, обаче, нямате дял, нито право, нито спомен в Ерусалим.

\chapter{3}

\par 1 Тогава първосвещеникът Елиасив и братята му свещениците станаха та съградиха овчата порта; те я осветиха и поставиха вратите й, дори до кулата Мея я осветиха, до кулата Ананеил.
\par 2 И до него градяха Ерихонските мъже. И до тях градеше Закхур, Имриевият син.
\par 3 А рибната порта съградиха Сенаевите синове, които, като положиха гредите й, поставиха и вратите й, ключалките й и лостовете й.
\par 4 До тях поправяше Меримот, син на Урия, Акосовият син. До него поправяше Месулам, син на Варахия, Месизавеиловият син. До него поправяше Садок, Ваанаевият син.
\par 5 И до него поправяха текойците; но големците им не се впрегнаха в делото на своя Господ.
\par 6 И старата порта поправиха Иодай, Фасеевият син и Месулам, Весодиевият син, който, като положиха гредите й, поставиха и вратите й, ключалките и лостовете й.
\par 7 До тях гаваонецът Мелатия и меронотецът Ядон, мъже от Гаваон и от Масфа, поправяше до седалището на областния управител отсам реката.
\par 8 До тях поправяше Озиил, Арахиевият син, един от златарите. До него поправяше Анания, един от аптекарите; и те укрепиха Ерусалим до широката стена.
\par 9 До него поправяше Рафаия, Оровият син, началник на половината от Ерусалимския окръг.
\par 10 До тях поправяше, срещу къщата си, Едаия, Арумафовият син. До него поправяше Хатус, Асаваниевият син.
\par 11 А Мелхия, Харимовият син, и Асув, Фаат-моавовият син, поправяха друга част и кулата на пещите.
\par 12 И до тях поправяше, заедно с дъщерите си, Селум, Алоисовият син, началник на половината от Ерусалимския окръг
\par 13 Портата на долината поправиха Анун и жителите на Заноя, които, като я съградиха, поставиха и вратите й, ключалките й и лостовете й; поправиха и хиляда лакти от стената до портата на бунището.
\par 14 А портата на бунището поправи Мелхия, Рихавовият син, началник на Вет-акаремския окръг; и той, като я съгради, постави и вратите й, ключалките й и лостовете й.
\par 15 А портата на извора поправи Селун, Холозеевият син, началник на Масафския окръг, който като я съгради и покри, постави и вратите й, ключалките й и лостовете й; поправи и стената на силоамския водоем при царската градина, дори до стъпалата, които слизат от Давидовия град.
\par 16 След него поправяше Неемия Азвуковият син, началник на половината от Вет-сурския окръг, до мястото срещу Давидовите гробища, и до направения водоем, и до къщата на силните мъже.
\par 17 След него поправяха левитите начело с Реума, Ваниевият син. До него поправяше Асавия, началник на половината от Кеилския окръг, за своя окръг.
\par 18 След него поправяха братята им, начело с Вавая, Инададовият син, началник на другата половина от Кеилския окръг.
\par 19 До него, Есер, Исусовият син, началник на Масфа, поправяше друга част, срещу нагорнището към оръжейницата при ъгъла на стената.
\par 20 След него Варух, Заваевият син, поправяше ревностно друга част, от ъгъла до вратата на къщата на първосвещеник Елиасива.
\par 21 След него, Меримот, син на Урия, Акосовия син, поправяше друга част, от вратата на Елиасивовата къща.
\par 22 След него поправяха свещениците, които живееха в тая околност.
\par 23 След тях поправяха Вениамин и Асув срещу къщата си. След тях поправяше поправяше Азария, син на Маасия, Ананиевият син, при къщата си.
\par 24 След него, Вануй, Инададовият син, поправяше друга част, от къщата на Азария до къта, дори до ъгъла на стената.
\par 25 Фалал Узаевият син поправяше срещу ъгъла и кулата, която се издава от горната царска къща, която бе при двора на стражата; и след него поправяше Фадаия, Фаросовият син.
\par 26 (А нетинимите живееха в Офил до мястото срещу портата на водата на изток, и до издадената кула).
\par 27 След него текойците поправяха друга част срещу голямата издадена кула и до стената на Офил.
\par 28 Над конската порта поправяха свещениците, всеки срещу къщата си.
\par 29 След тях поправяше Садок, Емировият син, срещу къщата си. След него поправяше вратарят на източната порта Семаия, Сеханиевият син.
\par 30 След него, Анания, Селемиевият син и Анун, шестият син на Салафа, поправяха друга част. След тях поправяше Масулам, Варахиевият син, срещу стаята си.
\par 31 След него поправяше Мелхия, син на златаря, до къщата на нетинимите и на търговците, срещу вратата на Мифкада и до нагорнището при ъгъла.
\par 32 А между нагорнището, при ъгъла и овчата порта, поправяха златарите и търговците.

\chapter{4}

\par 1 А Санавалат, когато чу, че ние сме градили стената, разгневи се, много възнегодува, и присмя се на юдеите.
\par 2 И говори пред братята си и пред самарийската войска, казвайки: Що правят тия окаяни юдеи? ще се закрепят ли? ще жертвуват ли? ще свършат ли в един ден? ще съживят ли камъните от купищата пръст, като са изгорели?
\par 3 А амонецът Товия, който бе при него, рече: Даже това що градят, лисица ако се покачи по него, ще събори каменната им стена.
\par 4 Чуй, Боже наш, защото сме презрени, обърни укора им върху собствените им глави, и предай ги на разграбване в земя гдето ще са пленници.
\par 5 Не покривай беззаконието им, и грехът им да се не изличи пред Тебе; защото Те разгневиха пред градящите.
\par 6 Така съградихме стената; и стената се свърза цяла до половината от височината си; защото людете имаха присърце работата.
\par 7 Но когато Санавалат, Товия, арабите, амонците и азотците чуха, че поправянето на ерусалимските стени напредвало, и че проломите почнали да се затварят, разгневиха се много;
\par 8 и всичките заедно се сговориха да дойдат и да воюват против Ерусалим и да му напакостят.
\par 9 А ние се помолихме на своя Бог, и поставихме стражи да пазят против тях денем и нощем, понеже се бояхме от тях.
\par 10 Но Юда рече: Силата на бременосците вече ослабна, а пръстта е много; ние не можем да градим стената.
\par 11 А неприятелите ни рекоха: Няма да усетят, нито да видят докле дойдем всред тях, та ги избием и спрем работата.
\par 12 И като дойдоха юдеите, които живееха при тях, рекоха ни десет пъти: Към която страна и да се обърнете, от там ще дойдат върху нас.
\par 13 Затова поставих зад по-ниските места в стената, зад по-изложените места - поставих людете по семействата им с мечовете им, с копията им и с лъковете им.
\par 14 И като разгледах, станах та рекох на благородните, на по-първите човеци и на другите люде: Не се бойте от тях; помнете великия и страшния Господ, и бийте се за братята си, за синовете си и дъщерите си, за жените си и за домовете си.
\par 15 А когато неприятелите ни чуха, че намерението им било известно нам, и че Бог осуетил проекта им, ние всинца се върнахме на стената, всеки на работата си.
\par 16 И от това време нататък, половината от слугуващите ми бяха на работата, а половината от тях държаха копията, щитчетата, лъковете и броните; и началниците насърчаваха целия Юдов дом.
\par 17 Всеки един от ония, които градяха стената, и които носеха товарите, и които товареха, работеше работата си с едната си ръка, а с другата държеше оръжието;
\par 18 и всеки от зидарите имаше меча си опасан на кръста си и така градеше. А тоя, който свиреше с тръбата беше до мене.
\par 19 И рекох на благородните, на по-първите човеци и на другите люде: Работата е по-голяма и обширна, а ние сме пръснати по стената далеч един от други;
\par 20 затова, гдето чуете гласа на тръбата, там се стичайте при нас; нашият Бог воюва за нас.
\par 21 Така карахме работата; половината от тях държаха копията от зазоряването до появяването на звездите.
\par 22 И в същото време казах на людете: Всеки със слугата си нека нощува всред Ерусалим, за да ни бъдат стражи нощем, а да работят денем.
\par 23 И тъй, нито аз, нито мъжете от стражата, които ме следваха, - никой от нас не събличаше дрехите си; всеки държеше оръжието си даже, когато отиваше на водата да се мие.

\chapter{5}

\par 1 По това време се дигна голям вик от людете и от жените им против братята им юдеите.
\par 2 Защото имаше едни, които казваха: Ние, синовете ни и дъщерите ни сме мнозина; затова, нека ни се даде жито та да ядем и да живеем.
\par 3 А и други имаше, които казваха: Ние заложихме нивите си, лозята си и къщите си, за да вземем жито, по причина на града.
\par 4 Имаше пък още други, които казваха: Срещу нивите и срещу лозята си ние заехме пари за царските данъци.
\par 5 Но пак нашите тела са като телата на братята ни, нашите чада като техните чада; и, ето, ние ще трябва да даваме синовете си и дъщерите си на работата като роби, и някои от дъщерите ни са вече заведени в робство; нито е в силата ни да правим другояче, защото други имат нивите и лозята ни.
\par 6 И оскърбих се твърде много, като чух вика им и тия думи.
\par 7 Тогава, като размислих в себе си, изобличих благородните и по-първите човеци, като им рекох: Вие вземате лихва всеки от брата си. И свиках против тях голямо събрание, та им рекох:
\par 8 Ние според силата си изкупихме братята си юдеите, които бяха предадени на народите; а сами вие ще продадете ли братята си? трябва ли да се продадат нам? А те мълчаха и не намериха какво да отговорят.
\par 9 Пак рекох: Не е добро това, което правите. Не трябва ли да ходите в страха на нашия Бог, за да не ни укоряват езичниците, нашите неприятели?
\par 10 Също и аз - братята ми и слугите ми - сме им заели пари и жито под лихва, но нека оставим, моля, това лихварство.
\par 11 Върнете им, прочее, още днес, нивите им, лозята им, маслините им и къщите им, тоже стотната част от парите, от житото, от виното и от дървеното масло, които изисквате от тях.
\par 12 Тогава рекоха: Ще ги повърнем и нищо не ще искаме от тях; както казваш, така ще направим. Тогава повиках свещениците, чрез които ги заклех, че ще направят според това обещание.
\par 13 Отърсих още и пазухата си, и рекох: Така да отърси Бог от къщата му и от труда му всеки, който не е изпълнил това обещание; да! така да бъде отърсен и изпразнен. И цялото събрание рече: Амин! и прославиха Господа. И людете постъпиха според това обещание.
\par 14 А от времето, когато бях назначен за областен управител над тях в Юдовата земя, от двадесетата година на цар Артаксеркса, дванадесет години, аз и братята ми не сме вземали заплатата на областен управител.
\par 15 А по-раншните областни управители, които бяха преди мене, бяха товар на людете, и, освен четиридесет сикли сребро, вземаха от тях хляб и вино; при това, и слугите им господаруваха над людете. Но аз не правех така, понеже се боях от Бога.
\par 16 Още и залягах върху работата на тая стена, и нива не купихме; и всичките ми слуги бяха събрани там за тая работа.
\par 17 При това, на трапезата ми имаше сто и петдесет мъже от юдеите и по-видните човеци, освен ония, които идеха при нас от околните нам народи.
\par 18 А това що се готвеше за мене всеки ден бе едно говедо и шест отборни овци; и птици се готвеха за мене, а веднъж в десет дни имахме изобилно вино от всякакъв вид; а при всичко това аз не поисках заплатата на областен управител, защото робството тежеше силно върху тия люде.
\par 19 Помни ме, Боже мой, за добро поради всичко, което съм сторил за людете.

\chapter{6}

\par 1 А Санавалат, Товия, арабинът Гисам, и останалите от неприятелите ни, като чуха, че съм бил заградил стената, и че не останало вече пролом в нея, ако и да не бях поставил врати на портите до онова време,
\par 2 Санавалат и Гисам пратиха до мене да кажат: Дойди, нека се срещнем в едно от селата на Оновото поле. Но те възнамеряваха да ми сторят зло.
\par 3 И пратих им човеци да кажат: защо да се спира работата като я оставя и сляза при вас?
\par 4 И пращаха до мене четири пъти по същия начин: но аз им отговарях все така.
\par 5 Тогава за пети път Санавалат прати слугата си до мене по същия начин с отворено писмо в ръката му,
\par 6 в което бе писано: Слух се носи между народите, пък и Гисам казва, че ти и юдеите мислите да се подигнете, за която причина ти и градиш стената; и, според тия думи, ти искаш да им станеш цар.
\par 7 Назначил си още и пророци да разгласяват за тебе в Ерусалим, като казват: Цар има в Юда. И сега ще се извести на царя според тия думи. Дойди сега, прочее, и нека се съветваме заедно.
\par 8 Тогава пратих до него човеци да кажат: Няма такова нещо каквото ти казваш; но ти ги измислюваш от себе си.
\par 9 Защото те всички искаха да ни плашат, казвайки: Ръцете им ще ослабнат от работата, та няма да се свърши. А сега, о Боже, подкрепи Ти ръцете ми.
\par 10 Тогава аз отидох в къщата на Самаия син на Делаия, Метавеиловия син, който бе затворен; и той ми каза: Нека се срещнем в Божия дом, всред храма, и нека затворим вратите на храма; защото тия идат да те убият, да! тая нощ ще дойдат да те убият.
\par 11 Но аз казах: Човек като мене бива ли да бяга? и кой човек като мене би влязъл в храма за да избави живота си? Не ща да вляза.
\par 12 И познах, че, ето, Бог не бе го пратил; но той от себе си произнесе това пророчество против мене, и Товия и Санавалат бяха го подкрепили.
\par 13 С тая цел бе подкупен, да се уплаша та да направя така, и да съгреша, и те да имат причина, да злословят, за да ме укорят.
\par 14 Спомни си, Боже мой, за Товия и Санавалат според тия техни дела, още и за пророчицата Ноадия и за други пророци, които искаха да ме плашат.
\par 15 Така се свърши стената на двадест и петия ден от месец Елул, за петдесет и два дни.
\par 16 И когато чуха това всичките ни неприятели, тогова всичките езичници около нас се уплашиха, и много се снишиха пред своите очи, защото познаха, че това дело стана от нашия Бог.
\par 17 При това, в ония дни Юдовите благородни пращаха често писма до Товия, и писма от Товия дохождаха до тях.
\par 18 Защото в Юда имаше мнозина, които се бяха заклели да му бъдат привързани, понеже бе зет на Сехания Араховия син, и син му Иоанан бе взел за жена дъщеря на Месулама Варахиевия син.
\par 19 Още и приказваха пред мене за неговите благодеяния, а моите думи носеха нему. И Товия пращаше писма за да уплаши.

\chapter{7}

\par 1 А като се съгради стената и поставих вратите, и определиха се вратарите, певците и левитите,
\par 2 предадох Ерусалим под грижата на брата си Анания и на началника на крепостта Анания; защото беше верен човек и боеше се от Бога повече от мнозина.
\par 3 И рекох им: Да се не отварят ерусалимските порти преди да припече слънцето; и вратите да остават заключени и залостени до тогаз, до когато не стоят стражите да пазят при тях; и поставете стражи из ерусалимските жители, всеки на стража си, всеки срещу къщата си.
\par 4 А градът бе широк и голям, и людете в него малцина, и нямаше къщи построени.
\par 5 И моят Бог тури в сърцето ми да събера благородните, по-първите човеци и людете, за да бъдат изброени по родословие. И намерих книгата на родословието на ония, които възлязоха най-напред намерих и писано в нея:
\par 6 Ето човеците на Вавилонската област, които възлязоха от плена на закараните, които вавилонският цар Навуходоносор бе преселил, и който се върнаха в Ерусалим и в Юда, всеки в града си,
\par 7 които дойдоха с Зоровавела, Исуса, Неемия, Азария, Раамия, Наамания, Мардохея, Валасана, Мисиерета, Вагуя, Наума, и Ваана. Числото на мъжете от Израилевите люде беше следното:
\par 8 Фаросови потомци, две хиляди и сто и седемдесет и двама души.
\par 9 Сефатиеви потомци, триста и седемдесет и двама души.
\par 10 Арахови потомци, шестстотин и петдесет и двама души.
\par 11 Фаат-моавови потомци, от Исусовите и Иоавовите потомци, две хиляди и осемстотин и осемнадесет души.
\par 12 Еламови потомци, хиляда и двеста и петдесет и четири души.
\par 13 Затуеви потомци, осемстотин и четиридесет и пет души.
\par 14 Закхееви потомци, седемстотин и шестнадесет души.
\par 15 Вануеви потомци, шестстотин и четиридесет и осем души.
\par 16 Виваеви потомци, шестстотин и двадесет и осем души.
\par 17 Азгадови потомци, две хиляди и триста и двадесет и двама души.
\par 18 Адоникамови потомци, шестстотин и шестдесет и седем души.
\par 19 Вагуеви потомци, две хиляди и шестстотин и седем души.
\par 20 Адинови потомци, шестстотин и петдесет и пет души.
\par 21 Атирови потомци, от Езекия, двадесет и осем души.
\par 22 Асумови потомци, триста и двадесет и осем души.
\par 23 Висаеви потомци, триста и двадесет и четири души.
\par 24 Арифови потомци, сто и дванадесет души.
\par 25 Гаваонски мъже, деветдесет и пет души.
\par 26 Витлеемски и нетофатски мъже, сто и осемдесет и осем души.
\par 27 Анатотски мъже, сто и двадесет и осем души.
\par 28 Вет-асмаветски мъже, четиридесет и двама души.
\par 29 Мъже от Кириатиарим, от Хефира и от Вирот, седемстотин и четиридесет и трима души.
\par 30 Мъже от Рама и от Гава, шестстотин и двадесет и един човека.
\par 31 Мъже от Михмас, сто и двадесет и двама души.
\par 32 Мъже от Ветил и от Гай, сто и двадесет и трима души.
\par 33 Мъже от другия Нево, петдесет и двама души.
\par 34 Потомци на другия Елам, хиляда и двеста и петдесет и четири души.
\par 35 Харимови потомци, триста и двадесет души.
\par 36 Мъже от Ерихон, триста и четиридесет и пет души.
\par 37 Мъже от Лод, от Адид и от Оно, седемстотин и двадесет и един човек.
\par 38 Мъже от Сеная, три хиляди и деветстотин и тридесет души.
\par 39 Свещениците: Едаеви потомци, от Исусовия дом, деветстотин и седемдесет и трима души.
\par 40 Емирови потомци, хиляда и петдесет и двама души.
\par 41 Пасхорови потомци, хиляда и четиридесет и седем души.
\par 42 Харимови потомци, хиляда и седемнадесет души.
\par 43 Левитите: Исусови потомци от Кадмиила, от Одавиевите потомци, седемдесет и четири души.
\par 44 Певците: Асофови потомци, сто и четиридесет и осем души.
\par 45 Вратарите: Селумови потомци, Атирови потомци, Талмонови потомци, Акувови потомци, Атитаеви потомци, Соваеви потомци, сто и тридесет и осем души.
\par 46 Нетинимите: Сихаеви потомци, Асуфови потомци, Таваотови потомци,
\par 47 Киросови потомци, Сиаеви потомци, Фадонови потомци,
\par 48 Леванаеви потомци, Агаваеви потомци, Салмаеви потомци,
\par 49 Ананови потомци, Гедилови потомци, Гаарови потомци,
\par 50 Реаеви потомци, Расинови потомци, Некодаеви потомци,
\par 51 Газамови потомци, Озаеви потомци, Фасееви потомци,
\par 52 Висаеви потомци, Меунимови потомци, Нафусесимови потомци,
\par 53 Ваквукови потомци, Акуфаеви потомци, Арурови потомци,
\par 54 Васалотови потомци, Меидаеви потомци, Арсаеви потомци,
\par 55 Варкосови потомци, Сисарови потомци, Тамаеви потомци,
\par 56 Насиеви потомци и Атифаеви потомци.
\par 57 Потомци на Соломоновите слуги: Сотаиеви потомци, Соферетови потомци, Феридови потомци.
\par 58 Яалаеви потомци, Дарконови потомци, Гедилови потомци,
\par 59 Сафатиеви потомци, Атилови потомци, Фохеретови потомци, от Севаим, Амонови потомци.
\par 60 Всичките нетиними и потомците на Соломоновите слуги бяха триста и деветдесет и двама души.
\par 61 А ето ония, които възлязоха от Тел-мелах, Тел-ариса, Херув, Адон, и Емир, но не можеха да покажат бащините си домове, нито рода си,, дали бяха от Израиля:
\par 62 Делаиеви потомци, Товиеви потомци, Некодаеви потомци, шестстотин и четиридесет и двама души.
\par 63 И от свещениците: Авиеви потомци, Акосови потомци, потомци на Варзелая, който взе жена от дъщерите на галаадеца Варзелай и се нарече с тяхното име;
\par 64 те потърсиха регистра си между преброените по родословие, но не се намери; затова, те бидоха извадени от свещенството като скверни.
\par 65 И управителят им заповяда да не ядат от пресветите неща докле не настане свещеник с Урим и Темим.
\par 66 Всичките купно събрани бяха четиридесет и две хиляди и триста и шестдесет души,
\par 67 освен слугите им и слугините им, които бяха седем хиляди и триста тридесет и седем души. Те имаха и двеста и четиридесет и пет певци и певици.
\par 68 Конете им бяха седемстотин и тридесет и шест; мъските им, двеста и четиридесет и пет;
\par 69 камилите им, четиристотин и тридесет и пет; а ослите им, шест хиляди и седемстотин и двадесет.
\par 70 А някои от началниците на бащините домове дадоха за делото: управителят внесе в съкровищницата хиляда драхми злато, петдесет легени и петстотин и тридесет свещенически одежди.
\par 71 И някои от началниците на бащините домове внесоха в съкровищницата за делото двадесет хиляди драхми злато и две хиляди и двеста фунта сребро.
\par 72 И внесеното от другите люде бе двадесет хиляди драхми злато, две хиляди фунта сребро, и шестдесет и седем свещенически одежди.
\par 73 Така свещениците, левитите, вратарите, певците, някои от людете, нетинимите и целият Израил се заселиха в градовете си; когато настъпи седмият месец, израилтяните бяха вече в градовете си.

\chapter{8}

\par 1 Тогава всичките люде се събраха като един човек на площада, който бе пред портата на водата; и рекоха на книжника Ездра да донесе книгата на Моисеевия закон, който Господ бе заповядал на Израиля.
\par 2 И тъй, на първия ден, от седмия месец, свещеник Ездра донесе закона пред събранието от мъже и жени и от всички, които, слушайки, можеха да разбират.
\par 3 И на площада, който бе пред портата на водата, той чете от него, от зори до пладне, пред мъжете и жените и ония, които можеха да разбират; и вниманието на всички люде беше в книгата на закона.
\par 4 А книжникът Ездра стоеше на висока дървена платформа, която бяха направили за тая цел; и при него стояха Мататия, Сема, Анания, Урия, Хелкия и Маасия отдясно му; а отляво му, Федаия, Мисаил, Мелхия, Асум, Асвадана, Захария и Месулам.
\par 5 И Ездра разгъна книгата пред всичките люде (защото бе над всичките люде); и когато я разгъна, всичките люде станаха прави.
\par 6 И Ездра благослови Господа великия Бог; и всички люде отговориха: Амин, амин! като издигнаха ръцете си; и наведоха се та се поклониха Господу с лицата си до земята.
\par 7 А Исус, Ваний, Серевия, Ямин, Акув, Саветай, Одия, Маасия, Келита, Азария, Иозавад, Анан, Фелаия и левитите тълкуваха закона на людете, като стояха людете на местата си.
\par 8 Четоха ясно от книгата на Божия закон, и дадоха значението като им тълкуваха прочетеното.
\par 9 И Неемия, който бе управител, и свещеник Ездра, книжникът, и левитите, които тълкуваха на людете, рекоха на всичките люде: Тоя ден е свет на Господа вашия Бог; не тъжете нито прочете; (защото всичките люде плачеха като чуха думите на закона).
\par 10 Тогава им рече: Идете, яжте тлъсто и пийте сладко и пратете дялове на тия, които нямат нищо приготвено, защото денят е свет Господу; и не скърбете, защото да се радвате, в Господа, е вашата сила.
\par 11 И левитите усмириха всичките люде, като казваха: Мълчете, защото денят е свет; и не тъжете.
\par 12 И така, всичките отидоха да ядат и да пият, да изпратят дялове, и да направят голямо увеселение, защото бяха разбрали думите, които им се известиха.
\par 13 И на втория ден началниците на бащините домове на всичките люде, и свещениците и левитите, се събраха при книжника Ездра, за да се поучат с думите на закона.
\par 14 И намериха писано в закона, че Господ бе заповядал чрез Моисея на израилтяните да живеят в колиби в празника на седмия месец,
\par 15 и да обнародват и прогласявят това във всичките си градове, па и в Ерусалим, казвайки: Излезте в гората та донесете маслинени клони, клони от дива маслина, мирсинени клони, палмови клони, и клони от гъстолистни дървета, за да направите колиби според предписаното.
\par 16 И тъй людете излязоха та донесоха клони, и направиха се колиби, всеки по покрива на къщата си, в дворовете си, в дворовете на Божия дом, на площада при портата на водата и на площада при Ефремовата порта.
\par 17 Цялото събрание от ония, които бяха се върнали от плена, направиха колиби и седнаха в колибите; защото от времето на Исуса, Навиевия син, до оня ден израилтяните не бяха правили така. И стана много голямо веселие.
\par 18 При това, всеки ден, от първия ден, четеше закона от книгата на Божия закон. И пазеха празника седем дни; а на седмия ден имаше тържествено събрание според наредбата.

\chapter{9}

\par 1 След това, на двадесет и четвъртия ден от същия месец, когато израилтяните бяха събрани с пост, облечени с вретища, и с пръст на себе си,
\par 2 Израилевият род отдели себе си от всичките чужденци; и застанаха та изповядаха своите грехове и беззаконията на бащите си.
\par 3 През една четвърт от деня те ставаха на местата си та четяха от книгата на закона на Господа своя Бог, и през друга четвърт се изповядваха и кланяха се на Господа своя Бог.
\par 4 Тогава някои от левитите, - Исус, Ваний, Кадмиил, Севания, Вуний, Серевия, Ваний и Хананий, - застанаха на платформата та извикаха със силен глас към Господа своя Бог.
\par 5 После левитите Исус, Кадмиил, Ваний, Асавния, Серевия, Одия, Севания и Петаия рекоха: Станете та благославяйте Господа вашия Бог от века до века; и да благославят, Боже, Твоето славно име, което е възвишено по-горе от всяко благословение и хвала.
\par 6 Ти си Господ, само Ти; Ти си направил небето, небето на небесата, и цялото им множество, земята и всичко що е на нея, моретата и всичко що е в тях, и Ти оживотворяваш всичко това; и на Тебе се кланят небесните войнства.
\par 7 Ти си Господ Бог, Който си избрал Аврама, извел си го от Ур халдейски, и си му дал име Авраам;
\par 8 и като си намерил сърцето му вярно пред Тебе, направил си с него завет, че ще дадеш земята на ханаанците, хетейците, аморейците, ферезейците, евусейците и гергесейците, - че ще я дадеш на потомството му; и изпълнил си думите Си, защото си праведен.
\par 9 И Ти видя теглото на бащите ни в Египет, и чу вика им при Червеното море.
\par 10 Ти показа знамения и чудеса над Фараона, над всичките му слуги, и над всичките люде на земята му, понеже Ти позна, че гордо постъпваха против тях; и Ти си придобил име, както е известно днес.
\par 11 Ти и раздели морето пред тях, та преминаха по сухо всред морето; а гонителите им Ти хвърли в дълбочините, като камък в силните води.
\par 12 При това денем Ти ги води с облачен стълб, а нощем с огнен стълб, за да им светиш по пътя, през която трябваше да минат.
\par 13 Тоже Ти слезе на Синайската планина та говори с тях от небето, и им даде справедливи съдби и закони на истината, и добри повеления и заповеди;
\par 14 и им направи позната светата Своя събота, и им даде заповеди, повеления и закони чрез слугата Си Моисея.
\par 15 Ти им даде и хляб от небето, когато бяха гладни, и им извади вода из скала, когато бяха жадни; и заповяда им да влязат, за да завладеят земята, за която беше се клел, че ще им я дадеш.
\par 16 Но те и бащите ни се възгордяха, закоравиха врата си и не послушаха Твоите заповеди;
\par 17 те отказаха да послушат, и не си спомниха чудесата, които Ти извърши за тях; но закоравиха врата си, и в бунта си определиха началник, за да се върнат в робството си. Но понеже си Бог, Който обичаш да прощаваш, милостив и жалостив, дълготърпелив и многомилостив, Ти не ги остави.
\par 18 Даже, когато си направиха леяно теле и рекоха: Ето твоят Бог, който те изведе из Египет, и извършиха предизвикателства,
\par 19 Ти пак в голямото Си милосърдие не ги остави в пустинята; облачният стълб не престана да бъде над тях денем, за да ги води из пътя, нито огненият стълб нощем, за да им свети по пътя, през който трябваше да минат.
\par 20 И ти им даде благия Си Дух, за да ги вразумява и не отне манната Си от устата им, и даде им вода, когато бяха жадни.
\par 21 Даже Ти ги храни четиридесет години в пустинята, та нищо не им липсваше; дрехите им не овехтяха и нозете им не отекоха.
\par 22 При това, Ти им даде царства и племена, които им раздели за дялове; и така завладяха земята на Сиона, и земята на есевонския цар, и земята на васанския цар Ог.
\par 23 И Ти умножи чадата им като небесните звезди, и заведе ги в земята, за която беше казал на бащите им да влязат в нея, за да я завладеят.
\par 24 И те, чадата им влязоха та завладяха земята; и Ти покори пред тях жителите на земята, ханаанците, и предаде ги в ръцете им с царете им и племената на земята, за да им сторят по волята си.
\par 25 И те превзеха укрепени градове и плодовита земя; и притежаваха къщи пълни с всякакви блага, изкопани кладенци, лозя, маслини, и множество плодовити дървета; така те ядоха и се наситиха, затлъстяха, и се насладиха с Твоята голяма благост.
\par 26 Но те не се покоряваха, подигнаха се против Теме, захвърляха Твоя закон зад гърбовете си, и избиваха Твоите пророци, които заявяваха против тях, за да ги обърнат към Тебе, и извършиха големи предизвикателства.
\par 27 Затова, Ти ги предаваше в ръката на притеснителите им, които ги притесняваха; а във време на притеснението им, като викаха към Тебе, Ти слушаше от небето, и според голямото Си милосърдие им даваше избавители, които ги избавяха от ръката на притеснителите им.
\par 28 Но като се успокояваха, те пак вършеха зло пред Тебе; затова Ти ги оставяше в ръката на неприятелите им, които и ги завладяваха; но пак, когато се обръщаха те викаха към Тебе, Ти ги слушаше от небето, и много пъти ги избавяше според милосърдието Си.
\par 29 Ти и заявяваше против тях, за да ги обърнеш към закона Си; но те се гордееха и не слушаха Твоите заповеди, но съгрешаваха против съдбите Ти (чрез които, ако ги изпълнява човек, ще живее), и обръщаха към Тебе гърба си, и закоравяваха врата си та и не слушаха.
\par 30 Въпраки това, Ти много години ги търпеше и заявяваше против тях чрез Духа Си посредством пророците Си, но те не внимаваха; затова Ти ги предаде в ръката на племената на земите.
\par 31 Обаче, поради голямото Твое милосърдие не ги довърши, нито ги остави; защото си Бог щедър и милостив.
\par 32 Сега, прочее, Боже наш, велики, мощни и страшни Боже, Който пазиш завет и милост, да се не счита за малко пред Тебе цялото злострадание, което е постигнало нас, царете ни, първенците ни, свещениците ни, пророците ни, бащите ни и всичките Твои люде от времето на асирийските царе до днес.
\par 33 Наистина Ти си справедлив във всичко, което ни е сполетяло; защото ти си действувал верно, а ние сме постъпвали нечестиво.
\par 34 Още царете ни, първенците ни, свещениците ни и бащите ни не са опазили закона Ти, и не са внимавали в твоите заповеди и заявления, с които си заявявал против тях.
\par 35 Защото те през време на царството си, и при голямата благост, която Ти им показа, и в широката плодовита земя, която Ти постави пред тях, не Ти служиха, нито се обърнаха от нечестивите си дела.
\par 36 Ето, роби сме днес; и в земята, която Ти даде на бащите ни, за да ядат плода й и благото й, ето роби сме в нея.
\par 37 Тя дава голямо изобилие на царете, които Ти си поставил над нас, поради греховете ни; и властвуват над телата ни и над добитъка ни според волята си; а ние сме в голямо притеснение.
\par 38 Поради всичко това ние правим верен завет и го написваме; и първенците ни, свещениците ни го подпечатват.

\chapter{10}

\par 1 А ония, които подпечатаха, бяха: управителят Неемия, Ахалиевият син и Седекия,
\par 2 Сераия, Азария, Еремия,
\par 3 Пасхор, Амария, Мелхия,
\par 4 Хатус, Севания, Малух,
\par 5 Харим, Меримот, Авдия,
\par 6 Даниил, Ганатон, Варух,
\par 7 Месулам, Авия, Миамин,
\par 8 Маазия, Велгай и Семаия; те бяха свещеници.
\par 9 Левити: Исус, Азаниевият син, Вануй от Инададовите потомци и Кадмиил;
\par 10 и братята им Севания, Одия, Келита, Фелаия, Анан,
\par 11 Михей, Реов, Асавия,
\par 12 Закхур, Серевия, Севания,
\par 13 Одия, Ваний и Венину.
\par 14 Началниците на людете: Фарос, Фаат-моав, Елам, Зату, Ваний,
\par 15 Вуний, Азгад, Вивай,
\par 16 Адония, Вагуй, Адин,
\par 17 Атир, Езекия, Азур,
\par 18 Одия, Асум, Висай,
\par 19 Ариф, Анатот, Невай,
\par 20 Матфиас, Месулам, Изир,
\par 21 Месизавеил, Садок, Ядуа,
\par 22 Фелатия, Анан, Анаия,
\par 23 Осия, Анания, Асув,
\par 24 Алоис, Филея, Совив,
\par 25 Реум, Асавна, Маасия,
\par 26 Ахия, Анан, Ганан,
\par 27 Малух, Харим и Ваана.
\par 28 А останалите от людете, свещениците, левитите, вратарите, певците, нетинимите, и всички, които бяха се отделили от племената на земите и се прилепили към Божия закон, жените им, синовете им и дъщерите им, всеки, който знаеше и разбираше,
\par 29 присъединиха се към братята си, големците си, и постъпиха в заклинание и клетва да ходят по Божия закон, който бе даден чрез Божия слуга Моисея, и да пазят и вършат всичките заповеди на Иеова нашия Господ, съдбите Му и повеленията Му;
\par 30 и да не даваме дъщерите си на племената на земята, и да не вземаме техните дъщери за синовете си;
\par 31 и ако племената на земята донесат стоки или каква да било храна за продан в съботен ден, да не купуваме от тях в събота или в свет ден; и да се отказваме от обработването на земята всяка седма година и от изискването на всеки дълг.
\par 32 Наредихме Си още да се задължим да даваме всяка година по една трета от сикъла за службата на дома на нашия Бог,
\par 33 за присъствените хлябове, за постоянния хлебен принос, за постоянното всеизгаряне, в съботите и на новолунията и на празниците, и за светите неща, за приносите за грях в умилостивение за Израиля, и за всичката работа на дома на нашия Бог.
\par 34 И ние - свещениците, левитите и людете - хвърлихме жребие помежду си за приноса на дървата, за да ги докарват в дома на нашия Бог, според бащините ни домове, в определените времена всяка година, за да горят върху олтара на Господа нашия Бог според предписаното в закона;
\par 35 и всяка година да донасяме в Господния дом първите плодове от земята си и първите плодове от рожбата на всяко дърво;
\par 36 и да довеждаме в дома на нашия Бог, на свещениците, които служат в дома на нашия Бог, първородните от синовете си и от добитъка си, според предписаното в закона, и първородните от говедата си и от стадата си;
\par 37 и да донасяме на свещениците, в стаите на дома на нашия Бог първите плодове от нашето тесто, и приносите си, и плодовете от всякакво дърво, тоже и виното и дървеното масло; и да внасяме на левитите десетъците от земята си, и левитите да вземат десетъците по всичките градове, гдето обработваме земята.
\par 38 И някой свещеник, Ааронов потомец да бъде с левитите, когато вземат десетъците; и левитете да донасят десетъка от десетъците в дома на нашия Бог, в стаите на съкровищницата.
\par 39 Защото израилтяните и левийците трябва да донасят приносите от житото, от виното и от дървеното масло в стаите, гдето са съдовете на светилището, и служащите свещеници, вратарите и първенците. И няма да оставим дома на нашия Бог.

\chapter{11}

\par 1 И първенците на людете се заселиха в Ерусалим; а останалите от людете хвърлиха жребие, за да доведат един от десет души да се засели в Ерусалим, в своя град, а девет части от населението в другите градове.
\par 2 И людете благословиха всички ония човеци, които доброволно предложиха себе си да се заселят в Ерусалим.
\par 3 А ето главните мъже от областта, които се заселиха в Ерусалим; (а в Юдовите градове се заселиха, всеки по притежанието си в градовете им, Израил, свещениците, левитите, нетинимите и потомците на Соломоновите слуги);
\par 4 в Ерусалим се заселиха някои от юдейците и от вениаминците. От юдейците: Атаия син на Озия, син на Амария, син на Сафатия, син на Маалалеила, от Фаресовите потомци;
\par 5 и Маасия син на Варуха, син на Холозея, син на Азаия, син на Адаия, син на Иоярива, син на Захария, син на Силония.
\par 6 Всичките Фаресови потомци, които се заселиха в Ерусалим, бяха четиристотин и шестдесет и осем храбри мъже.
\par 7 А вениаминците бяха следните: Салу син на Месулама, син на Иоада, син на Федаия, син на Колаия, син на Маасия, син на Итиила, син на Исаия;
\par 8 и с тях Гавай и Салай; деветстотин двадесет и осем души.
\par 9 Иоил, Захриевият син, беше надзирател над тях, а Юда Сенуиният син, беше втори над града.
\par 10 От свещениците: Едаия, Иояривовият син, Яхин,
\par 11 Сераия син на Хелкия, син на Месулама, син на Садока, син на Мераиота, син на управителя на Божия дом Ахитов;
\par 12 и братята им, които вършеха работата на дома; осемстотин и двадесет и двама души; и Адаия син на Ероама, син на Фелалия, син на Амсия, син на Захария, син на Пасхора, син на Мелхия;
\par 13 и братята му, началници на бащините домове; двеста и четиридесет и двама души; и Амасай син на Азареила, син на Аазая, син на Месилемота, син на Емира;
\par 14 и братята им, силни и храбри мъже, сто и двадесет и осем души; а Завдиил, син на Гедолим, бе надзирател над тях.
\par 15 А от левитите: Семаия син на Асува, син на Азрикама, син на Асавия, син на Вуний;
\par 16 и Саветай и Иозавад, от левитските началници, бяха над външните работи на Божия дом.
\par 17 И Матания син на Михея, син на Завдия, син на Асафа, който ръководеше славословието при молитвата, и Ваквукия, който беше вторият между братята си; и Авда, син на Самуя, син на Галала, син на Едутуна.
\par 18 Всичките левити в светия град бяха двеста и осемдесет и четири души.
\par 19 А вратарите: Акув, Талмон и братята им, които пазеха портите, бяха сто и седемдесет и двама души.
\par 20 А останалите от Израиля, свещениците и левитите, бяха по всичките Юдови градове, всеки в наследството си.
\par 21 А нетинимите се заселиха в Офил; а Сиха и Гесфа бяха над нетинимите.
\par 22 А надзирател над левитите в Ерусалим бе Озий син на Вания, син на Асавия, син на Матания, син на Михея, от Асафовите потомци, певците, над работата на Божия дом.
\par 23 Защото имаше царска заповед за тях, и определен дял за певците според нуждата на всеки ден.
\par 24 А Петаия, син на Месизавеила, от потомците на Юдовия син Зара, беше помощник на царя във всички що се касаеше до людете.
\par 25 А колкото за селата с нивите им, някои от юдеите се заселиха в Кириат-арва и селата й, в Девон и селата му, и в Кавсеил и селата му,
\par 26 и в Иисуя, Молада, Вет-фелет,
\par 27 Асар-суал, Вирсавее и селата му,
\par 28 в Сиклаг, Мекона и селата му,
\par 29 в Ен-римон, Сарая, Ярмут,
\par 30 Заноя, Одолам и селата им, в Лахис и полетата му, и в Азика и селата му. Така се заселиха от Вирсавее до Еномовата долина.
\par 31 А вениаминците се заселиха от Гава нататък, в Михмас, Гаия, Ветил и селата му,
\par 32 в Анатон, Ноб, Анания,
\par 33 Асора, Рама, Гатаим,
\par 34 Адид, Севоим, Невалат,
\par 35 Лод, Оно и в долината на дърводелците.
\par 36 А някои отреди от левитите се заселиха в Юда и във Вениамин.

\chapter{12}

\par 1 А ето свещениците на левитите, които възлязоха със Зоровавела Салатииловия син и с Исуса: Сараия, Еремия, Ездра,
\par 2 Амария, Малух, Хатус,
\par 3 Сехания, Реум, Меримот,
\par 4 Идо, Ганатон, Авия,
\par 5 Миамин, Маадия, Велга,
\par 6 Семаия, Иоярив, Едаия,
\par 7 Салу, Амок, Хелкия и Едаия. Тия бяха началниците на свещениците и на братята им в дните на Исуса.
\par 8 А Левитите: Исус, Вануй, Кадмиил, Серевия, Юда и Матания, който, заедно с братята си, бе над пеенето.
\par 9 А Ваквукия и Уний, братята им, бяха срещу тях в стражите.
\par 10 И Исус Роди Иоакима, а Иоаким роди Елиасива, а Елиасив роди Иодая,
\par 11 а Иодай роди Ионатана, а Ионатан роди Ядуя.
\par 12 А в дните на Иоакима свещеници, които бяха и началници на бащини домове, бяха: началник на бащиния дом на Сараия, Мераия; на Еремия, Анания;
\par 13 на Ездра, Месулам; на Амария, Иоанан;
\par 14 на Мелиху, Ионатан; на Севания, Иосиф;
\par 15 на Харима, Адна; и на Мариота, Хелкай;;
\par 16 на Идо, Захария; на Ганатона, Месулам;
\par 17 на Авия, Зехрий; на Маниамина, от Моадия, Фелтай;
\par 18 на Велга, Самуа; на Самаия, Ионатан;
\par 19 на Иоярива, Матенай; на Едаия, Озий;
\par 20 на Салая, Калай; на Амока, Евер;
\par 21 на Хелкия, Асавия; и на Едаия, Натанаил.
\par 22 В дните на Елиасива, Иодая, Иоанана и Ядуа левитите бяха записани за началници на бащини домове; също и свещениците през царуването на парсиеца Дарий.
\par 23 Левийците, които бяха началници на бащини домове, бяха записани в Книгата на летописите, дори до дните на Иоанана Елиасивовия син.
\par 24 А началниците на левитите бяха: Асавия, Серевия и Исус Кадмииловият син, с братята им срещу тях, назначени да хвалят и песнословят според заповедта на Божия човек Давида, ответно едни срещу други.
\par 25 Матания, Ваквукия, Авдия, Месулам, Талмон и Акув бяха вратари, и пазеха стражата на влагалищата при портите.
\par 26 Тия бяха в дните на Иоакима син на Исуса, син на Иоседека, и в дните на областния управител Неемия, и на свещеникът Ездра книжникът.
\par 27 И при посвещаването на ерусалимската стена потърсиха левитите по всичките им места, за да ги доведат в Ерусалим да празнуват посвещението си с веселие, със славословия и песни, с кимвали, псалми и арфи.
\par 28 И тъй, дружните певци се събраха, както от ерусалимската околност, така и от нетофатските села.
\par 29 от Вет-галгал, и от селата на Гава и на Азмавет; защото певците си бяха съградили села около Ерусалим.
\par 30 И свещениците и левитите, като очистиха себе си, очистиха и людете, портите и стената.
\par 31 Тогава изкачих Юдовите началници на стената, и определих две големи отделения хвалители; едното отиваше в шествие надясно върху стената към портата на бунището;
\par 32 и подир тях вървяха Осаия, и половината от Юдовите първенци
\par 33 и Азария, Ездра, Меулам,
\par 34 Юда, Вениамин, Семаия и Еремия,
\par 35 и някои от синовете на свещениците с тръби: Захария син на Ионатана, син на Семаия, син на Матания, син на Михея, син на Закхура, син на Асафа,
\par 36 и братята му: Семаия, Азареил Милалай, Гилалай, Маай, Натанаил, Юда и Ананий, с музикалните инструменти на Божия човек Давида; и книжника Ездра им беше на чело;
\par 37 и при портата на извора те се изкачиха първо пред себе си по стъпалата на Давидовия град, гдето стената се възвишава над Давидовата къща, дори до портата на водата към изток.
\par 38 А другото отделение хвалители вървеше в противоположна посока, и аз подир тях, тоже и половината от людете, по стената, край кулата на пещите и по широката стена,
\par 39 и над Ефремовата порта, и над старата порта, и над рибната порта, и край кулата Ананеил, и край кулата Мея, до овчата порта, докле застанаха в портата на стражата.
\par 40 Така двете отделения хвалители застанаха в Божия дом, и аз, и половината от видните мъже с мене,
\par 41 и свещениците Елиаким. Маасия, Миниамин, Михей, Елиоинай, Захария и Анания, с тръби,
\par 42 както и Маасия, Семаия, Елеазар, Озий, Иоанан, Малхия, Елам и Езер. И певците запяха със силен глас, с Езраия за водител.
\par 43 И в същия ден принесоха големи жертви и се развеселиха, защото Бог ги развесели премного; още жените и децата се развеселиха; тъй че увеселението на Ерусалим се разчу надалеч.
\par 44 Тоже в същия ден се определиха човеци над стаите за влагалищата, за приносите, за първите плодове, и за десетъците, за да събират в тях от полетата на градовете дяловете узаконени за свещениците и левитите; защото Юда се радваше поради свещениците и левитите, които служеха.
\par 45 Защото те и певците и вратарите пазеха заръчаното от Бога си, и заръчаното за очищението, според заповедта на Давида и сина му Соломона.
\par 46 Защото отдавна, в дните на Давида и на Асафа, имаше главни певци и пеения за хвала и благодарение Богу.
\par 47 И в дните на Зоровавела и в дните на Неемия, целият Израил даваха определените за всеки ден дялове на певците и на вратарите; те посвещаваха даровете си на левитите, а левитите посвещаваха на Аароновите потомци.

\chapter{13}

\par 1 В същия ден като чакаха Моисеевата книга и людете слушаха, намери се писано в нея, че амонците и моавците не трябваше никога да влизат в Божието общество,
\par 2 защото не посрещнаха израилтяните с хляб и вода, а наеха против тях Валаама, за да ги прокълне; обаче нашият Бог обърна проклетията в благословение.
\par 3 И като чуха закона, отлъчиха от Израиля всичките чужденци смесени с него.
\par 4 А въпреки това, свещеник Елиасив, който надзираваше стаите на дома на нашия Бог, като беше сродник на Товия,
\par 5 Приготви за него голяма стая, гдето по-напред туряха хлебните приноси, ливана, вещите и десетъците от житото, от виното и от дървеното масло, които бяха определени за левитите, за певците и за вратарите, също и приносите за свещениците.
\par 6 Но когато ставало всичко това, аз не бях в Ерусалим; защото в тридесет и втората година на вавилонския цар Артаксеркс, отидох при царя. И подир известно време, като изпросих позволение от царя,
\par 7 пак дойдох в Ерусалим, и научих се за злото, което Елиасив беше сторил относно Товия като приготвил за него стая в дворовете на Божия дом.
\par 8 И стана ми много мъчно; за това, изхвърлих вън от стаята всичката покъщнина на Товия.
\par 9 Тогава заповядах, та очистиха стаите, и пак внесох там вещите на Божия дом, хлебните приноси и ливана.
\par 10 После забелязах, че дяловете на левитите не били им давани, така щото левитите и певците, които вършеха работата на служенето, бяха побягнали всеки в селото си.
\par 11 Тогава изобличих по-главните мъже, като рекох: Защо е оставен Божият дом? И събрах побягналите служители та ги поставих на местата им.
\par 12 Тогава целият Юда донесе във влагалищата десетъка от житото, от виното и от дървеното масло.
\par 13 И поставих за пазители на влагалищата свещеник Селемия и секретаря Садок и от левитите Федаия, и при тях Анана син на Закхура, син на Матания, защото се считаха за верни; и работата им бе да раздават на братята си.
\par 14 Помни ме, Боже мой, за това, и не заличавай добрините, които сторих за дома на моя Бог и за наредбите Му.
\par 15 През ония дни видях неколцина в Юда, че в събота тъпчеха грозде в лина, внасяха снопи и товареха на осли вино, грозде, смокини и всякакви товари, които докарваха в Ерусалим в съботен ден; и аз заявих против тях, когато така продаваха храна.
\par 16 Също и тиряните, които живееха в града, донасяха риби и всякакви стоки та продаваха в събота на юдейците в Ерусалим.
\par 17 Тогава изобличих Юдовите благородни, като им рекох: Какво е това зло, което правите, като осквернявате съботния ден?
\par 18 Не постъпиха ли така бащите ви, така щото нашият Бог докара всичкото това зло на нас и на тоя град? А пък вие умножавате гняв върху Израиля като осквернявате съботата.
\par 19 Затова, когато почна да мръкнува в ерусалимските порти преди съботата, заповядах да се затворят вратите, и заповядах да се не отварят до подир съботота; и поставих на портите неколцина от моите слуги, за да не се внася никакъв товар в съботен ден.
\par 20 Тогава един-два пъти търговците и продавачите на всякакви стоки пренощуваха вън от Ерусалим.
\par 21 Аз, прочее, заявих против тях, като им рекох: Защо нощувате пред стената? Ако повторите, ще туря ръка на вас. От тогава не дойдоха вече в събота.
\par 22 И заповядах на левитите да се очистват и да дохождат да вардят портите, за да освещават съботния ден. Помни ме, Боже мой, и за това, и смили се за мене според голямата Си милост.
\par 23 После, в ония дни видях юдеите, които бяха взели жени азотки, амонки и моавки,
\par 24 чиито деца говореха половин азотски, а не знаеха да говорят юдейски, но приказваха по езика на всеки от ония народи.
\par 25 И изобличих ги, проклех ги, бих неколцина от тях, оскубах им космите и заклех ги в Бога, като казах: Да не давате дъщерите си на синовете им, и да не вземате от техните дъщери за синовете си или за себе си.
\par 26 Не съгреши ли така Израилевият цар Соломон? Ако и да не е имало между много народи цар подобен на него, който беше възлюбен от своя Бог, и когото Бог направи цар над целия Израил, но и него чужденките жени накараха да съгреши.
\par 27 А ние да позволим ли на вас да вършите всичко това голямо зло, да ставате престъпници против нашия Бог, като вземате чужденки жени?
\par 28 И един от синовете на Иодая, син на първосвещеник Елиасива, беше зет на оронеца Санавалат; затова го изпъдих от мене.
\par 29 Спомни си за тях, Боже мой, защото са осквернили свещенството и завета на свещенството и левитите.
\par 30 Така ги очистих от всичките чужденци, и определих отреди за свещениците и левитите, за всекиго работата му;
\par 31 наредих и за приноса на дърва в определени времена, за първите плодове. Помни ме, Боже мой, за добро.

\end{document}