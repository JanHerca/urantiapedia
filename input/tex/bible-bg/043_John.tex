\begin{document}

\title{John}


\chapter{1}

\par 1 В началото бе Словото; и Словото беше у Бога; и Словото бе Бог.
\par 2 То в началото беше у Бога.
\par 3 Всичко това чрез Него стана; и без Него не e станало нищо от това, което е станало.
\par 4 В Него бе животът и животът бе светлина на човеците.
\par 5 И светлината свети в тъмнината; а тъмнината я не схвана.
\par 6 Яви се човек изпратен от Бога, на име Иоан.
\par 7 Той дойде за свидетелство, да свидетелствува за светлината, за да повярват всички чрез него.
\par 8 Не беше той светлината, но дойде да свидетелствуо мене беше пръв.
\par 9 Истинската светлина, която осветлява всеки човек, идеше на света.
\par 10 Той бе в света; и светът чрез Него стана; но светът Го не позна.
\par 11 У Своите Си дойде, но Своите Му Го не приеха.
\par 12 А на ония, които Го приеха, даде право да станат Божии чада, сиреч, на тия, които вярват в Неговото име;
\par 13 които се родиха не от кръв, нито от плътска похот, нито от мъжка похот, но от Бога.
\par 14 И словото стана плът и пребиваваше между нас; и видяхме славата Му, слава като на Единородния от Отца, пълно с благодат и истина.
\par 15 Иоан свидетелствува за Него, и викаше казвайки: Ето Онзи за Когото рекох, Който иде подир мене, достигна да бъде пред мене, понеже спрямо мене беше пръв.
\par 16 Защото ние всички приехме от Неговата пълнота, и благодат върху благодат;
\par 17 понеже законът бе даден чрез Моисея, а благодатта и истината дойдоха чрез Исуса Христа.
\par 18 Никой, кога да е, не е видял Бога; Единородният Син, Който е в лоното на Отца, Той Го изяви.
\par 19 И ето, свидетелството, което Иоан даде, когато юдеите пратиха до него свещеници и левити от Ерусалим да го попитат: Ти кой си?
\par 20 Той изповяда, и не отрече, а изповяда: Не съм аз Христос.
\par 21 И попитаха го: Тогава що? Илия ли си? И каза: Не съм. Пророк ли си? И отговори: Не съм.
\par 22 На това му рекоха: Кой си? за да дадем отговор на ония, които са ни пратили. Що казваш за себе си?
\par 23 Той рече: Аз съм глас на едного, който вика в пустинята; Прав правете пътя за Господа, както рече пророк Исаия.
\par 24 А изпратените бяха от фарисеите.
\par 25 И попитаха го, като му рекоха: А защо кръщаваш, ако не си Христос, нито Илия, нито пророкът?
\par 26 В отговор Иоан им рече: Аз кръщавам с вода. Посред вас стои Един, Когото вие не познавате,
\par 27 Онзи, Който иде подир мене, [Който преден ми биде], Комуто аз не съм достоен да развържа ремика на обущата Му.
\par 28 Това стана във Витавара, отвъд Йордан, дето Иоан кръщаваше.
\par 29 На следния ден Иоан вижда Исуса, че иде към него, и казва: Ето Божият Агнец, Който носи греха на света!
\par 30 Тоя е за Когото рекох: Подир мене иде човек, Който достигна да бъде пред мене, защото спрямо мене беше пръв.
\par 31 И аз Го не познавах; но дойдох и кръщавам с вода затова, за да бъде Той изявен на Израиля.
\par 32 И Иоан свидетелствува, казвайки: Видях Духът да слиза като гълъб от небето и да почива върху Него.
\par 33 И аз Го не познах; но Оня, Който ме прати да кръщавам с вода Той ми рече: Онзи, над Когото видиш да слиза Духът и да почива върху Него, Той е Който кръщава със Светия Дух.
\par 34 И видях и свидетелствувам, че Тоя е Божият Син.
\par 35 На следния ден Иоан пак стоеше с двама от учениците си.
\par 36 И като съгледа Исуса когато минаваше каза: Ето Божият Агнец!
\par 37 И двамата ученика го чуха да говори така, и отидоха подир Исуса.
\par 38 И като се обърна Исус и видя, че идат подире Му, каза им: Що търсите? А те Му рекоха: Равви, (което значи, Учителю), где живееш?
\par 39 Казва им: Дойдете и ще видите. Дойдоха, прочее, и видяха где живее, и останаха при Него тоя ден. Беше около десетия час.
\par 40 Единият от двамата, който чуха от Йоана за Него и Го последваха, беше Андрей, брат на Симона Петра.
\par 41 Той първо намира своя брат Симона и му казва: Намерихме Месия (което значи Христос).
\par 42 Като го заведе при Исуса, Исус се вгледа в него и рече: Ти си Симон, син Ионов; ти ще се наричаш Кифа, (което значи Петър - Значи: Канара.).
\par 43 На другия ден Исус възнамери да отиде в Галилея; и намира Филипа и му казва: Дойди след Мене.
\par 44 А Филип беше от Витсаида, от града на Андрея и Петра.
\par 45 Филип намира Натанаила и му казва: Намерихме Онзи, за когото писа Моисей в закона и за Когото писаха пророците, Исуса, Йосифовия син, Който е от Назарет,
\par 46 Натанаил му рече: От Назарет може ли да произлезе нещо добро? Филип му каза: Дойди и виж.
\par 47 Исус видя Натанаила да дохожда към Него, и казва за него: Ето истински израилтянин, у когото няма лукавщина.
\par 48 Натанаил Му каза: Отгде ме познаваш? Исус в отговор му рече: Преди да те повика Филип, видях те като беше под смоковницата.
\par 49 Натанаил му отговори: Учителю, Ти си Божи син, Ти си Израилев цар.
\par 50 Исус в отговор му каза: Понеже ти рекох видях те под смоковницата, вярваш ли? Повече от това ще видиш.
\par 51 И рече му: Истина, истина ви казвам, [Отсега] ще видите небето отворено, и Божиите ангели да възлизат и слизат над Човешкия Син.

\chapter{2}

\par 1 На третия ден имаше сватба в Кана галилейска и Исусовата майка беше там.
\par 2 И Исус и учениците Му бяха поканени на сватбата.
\par 3 И когато се свърши виното, майката на Исуса Му казва: Вино нямат.
\par 4 А Исус - казва: Какво има между Мене и тебе жено? часът Ми още не е дошъл.
\par 5 Майка Му казва на слугите: Каквото ви рече, сторете.
\par 6 А там имаше шест каменни делви, поставени по обичая на юдейското очищение, които побираха по две или три мери.
\par 7 Исус им казва: Напълнете делвите с вода. И напълниха ги до горе.
\par 8 Тогава им казва: Налейте сега та занесете на настойника на угощението. И те занесоха.
\par 9 И когато настойникът на угощението вкуси от водата, сега превърната на вино, и не знаеше от къде беше, (но слугите, които бяха налели водата знаеха), настойникът на угощението повика младоженеца и му каза:
\par 10 Всеки човек слага първо доброто вино, и по-долното след като се понапият; ти си задържал доброто вино до сега.
\par 11 Това извърши Исус в Кана галилейска като начало на знаменията Си, и яви славата Си; и учениците Му повярваха в Него.
\par 12 След това слезе в Капернаум, Той и майка Му, братята Му, и учениците Му, и там преседяха не много дни.
\par 13 И като наближаваше пасхата на юдеите, Исус възлезе в Ерусалим.
\par 14 И намери в храма продавачите на волове, овце и гълъби, и тия, които седяха и разменяха пари;
\par 15 и направи бич от върви и изпъди всички тях от храма, както и овцете и воловете; изсипа парите на среброменителите, и прекатури трапезите им;
\par 16 и на тия, които продаваха гълъбите рече: Дигнете ги оттук; не правете Бащиния Ми дом, дом на търговия.
\par 17 Учениците Му си спомниха, че е писано: Ревността за Твоя дом ще ме изяде.
\par 18 По повод на това, юдеите проговаряйки, Му рекоха: Какво знамение ще ни покажеш, тъй като правиш това?
\par 19 В отговор Исус им рече: Разрушете тоя храм и за три дни ще го издигна.
\par 20 Затова юдеите рекоха: За четиридесет и шест години е бил граден тоя храм, та Ти за три дни ли ще го издигнеш?
\par 21 Но Той говореше за храма на тялото Си.
\par 22 И тъй, когато биде възкресен от мъртвите, учениците Му си спомниха, че беше казал това; и повярваха на писаното, и словото, което Исус беше говорил.
\par 23 И когато беше в Ерусалим, на пасхата през празника мнозина повярваха в Неговото име, като гледаха знаменията, които вършеше.
\par 24 Но Исус не им се доверяваше, защото познаваше всичките човеци,
\par 25 и защото Той нямаше нужда да Му свидетелствува някой за човека, понеже сам знаеше що има в човека.

\chapter{3}

\par 1 Между фарисеите имаше един човек на име Никодим, юдейски началник.
\par 2 Той дойде при Исуса нощем и Му рече: Учителю, знам, че от Бога си дошъл учител; защото никой не може да върши тия знамения, които Ти вършиш, ако Бог не е с него.
\par 3 Исус в отговор му рече: Истина, истина ти казвам, ако се роди някой отгоре (Или: Изново.), не може да види Божието царство.
\par 4 Никодим Му казва: Как може стар човек да се роди? може ли втори път да влезе в утробата на майка си и да се роди?
\par 5 Исус отговори: Истина, истина ти казвам, ако се не роди някой от вода и Дух, не може да влезе в Божието царство.
\par 6 Роденото от плътта е плът, а роденото от Духа е дух.
\par 7 Не се чуди, че ти рекох: трябва да се родите отгоре.
\par 8 Вятърът духа гдето ще, и чуваш шума му; но не знаеш отгде иде и къде отива; така е с всеки, който се е родил от Духа.
\par 9 Никодим в отговор Му рече: Как може да бъде това?
\par 10 Исус в отговор му каза: Ти си Израилев учител, и не знаеш ли това?
\par 11 Истина, истина ти казвам, това, което знаем, говорим, и свидетелствуваме за това, което сме видели, но не приемате свидетелството ни.
\par 12 Ако за земните работи ви говорих и не вярвате, как ще повярвате, ако ви говоря за небесните?
\par 13 И никой не е възлязъл на небето, освен Тоя, Който е слязъл от небето, сиреч, Човешкият Син, Който е на небето.
\par 14 И както Моисей издигна змията в пустинята, така трябва да бъде издигнат Човешкият Син,
\par 15 та всеки, който вярва в Него [да не погине, но] да има вечен живот.
\par 16 Защото Бог толкова възлюби света, че даде Своя Единороден Син, за да не погине ни един, който вярва в Него, но да има вечен живот:
\par 17 Понеже Бог не е пратил Сина на света да съди света, но за да бъде светът спасен чрез Него.
\par 18 Който вярва в Него не е осъден; който не вярва е вече осъден, защото не повярвал в името на Единородния Божий Син.
\par 19 И ето що е осъждането: светлината дойде на света, и човеците обикнаха тъмнината повече от светлината, защото делата им бяха зли.
\par 20 Понеже всеки, който върши зло, мрази светлината, и не отива към светлината, да не би да се открият делата му;
\par 21 но който постъпва според истината отива към светлината, за да се явят делата му, понеже са извършени по Бога.
\par 22 След това дойде Исус с учениците Си в юдейската земя; и там се бавеше с тях и кръщаваше.
\par 23 Също и Иоан кръщаваше в Енон, близо до Салим, защото там имаше много вода; и людете дохождаха и се кръщаваха.
\par 24 Понеже Иоан още не беше хвърлен в тъмница.
\par 25 И така възникна препирня от страна на Иоановите ученици с един юдеин относно очистването.
\par 26 И дойдоха при Иоана и му рекоха: Учителю, Онзи, Който беше с тебе отвъд Иордан, за Когото ти свидетелствува, ето, Той кръщава, и всички отиват при Него.
\par 27 Иоан в отговор рече: Човек не може да вземе върху си нищо, ако не му е дадено от небето.
\par 28 Вие сами сте ми свидетели, че рекох: Не съм аз Христос, но съм пратен пред Него.
\par 29 Младоженецът е, който има невестата, а приятелят на младоженеца, който стои да го слуша, се радва твърде много поради гласа на младоженеца; и така, тая моя радост е пълна.
\par 30 Той трябва да расте, а пък да се смалявам‚
\par 31 Онзи, Който дохожда отгоре, е от всички по-горен; който е от земята, земен е, и земно говори. Който дохожда от небето е от всички по-горен.
\par 32 Каквото е видял и чул, за Него свидетелствува; но никой не приема свидетелството Му.
\par 33 Който е приел Неговото свидетелство потвърдил е с печата си, че Бог е истинен.
\par 34 Защото Тоя, Когото Бог е пратил, говори Божиите думи; понеже Той не Му дава Духа с мярка.
\par 35 Отец люби Сина и е предал всичко в Неговата ръка.
\par 36 Който вярва в Сина има вечен живот; а който не слуша Сина, няма да види живот, но Божият гняв остава върху него.

\chapter{4}

\par 1 Прочее, Когато Господ узна, че фарисеите чули, какво Исус придобивал и кръщавал повече ученици от Иоана,
\par 2 (не че сам Исус кръщаваше, а учениците Му),
\par 3 напусна Юдея и отиде пак в Галилея.
\par 4 И трябваше да мине през Самария.
\par 5 И така, дойде в един самарийски град наречен Сихар, близо до землището, което Яков даде на сина си Иосифа.
\par 6 Там беше Яковият кладенец (Гръцки: Извор.). Исус, прочее, уморен от пътуването, седеше така на кладенеца. Беше около шестият час.
\par 7 Дохожда една самарянка да си начерпи вода. Казва - Исус: Дай Ми да пия.
\par 8 (Защото учениците Му бяха отишли в града да купят храна).
\par 9 Впрочем, самарянката Му казва: Как Ти, Който си юдеин, искаш вода от мене, която съм самарянка? (Защото юдеите не се сношават със самаряните).
\par 10 Исус в отговор - каза: Ако би знаела Божия дар, и Кой е Онзи, Който ти казва: Дай Ми да пия, ти би поискала от Него и Той би ти дал жива вода?
\par 11 Казва Му жената: Господине, нито почерпало имаш, и кладенецът е дълбок; тогава отгде имаш живата вода?
\par 12 Нима си по-голям от баща ни Якова, който ни е дал кладенеца, и сам той е пил от него, и чадата му, и добитъкът му?
\par 13 Исус в отговор - каза: Всеки, който пие от тая вода, пак ще ожаднее;
\par 14 а който пие от водата, която Аз ще му дам, няма да ожаднее до века; но водата, която ще му дам, ще стане в него извор на вода, която извира за вечен живот.
\par 15 Казва Му жената: Господине, дай ми тая вода, за да не ожаднявам нито да извървявам толкова път до тук да изваждам.
\par 16 Казва - Исус: Иди, повикай мъжа си и дойди тука.
\par 17 В отговор жената Му каза: Нямам мъж. Казва - Исус: Право каза, че нямаш мъж;
\par 18 защото петима мъже си водила, и този, който сега имаш не ти е мъж. Това си право казала.
\par 19 Казва Му жената: Господине, виждам, че Ти си пророк.
\par 20 Нашите бащи в тоя хълм са се покланяли; а вие казвате, че в Ерусалим е мястото гдето трябва да се покланяме.
\par 21 Казва - Исус: Жено, вярвай Ми, че иде час, когато нито само в тоя хълм, нито в Ерусалим ще се покланяте на Отца.
\par 22 Вие се покланяте на онова, което не знаете; ние се покланяме не онова, което знаем; защото спасението е от юдеите.
\par 23 Но иде час, и сега е, когато истинските поклонници ще се покланят на Отца с дух и истина; защото такива иска Отец да бъдат поклонниците Му.
\par 24 Бог е дух; и ония, които Му се покланят, с дух и истина трябва да се покланят.
\par 25 Казва му жената: Зная, че ще дойде Месия (който се нарича Христос); Той, когато дойде, ще ни яви всичко.
\par 26 Казва - Исус: Аз, Който се разговарям с тебе съм Месия.
\par 27 В това време дойдоха учениците Му, и се почудиха, че се разговаря с жена; но никой не рече: Какво търсиш? или: Защо разговаряш с нея?
\par 28 Тогава жената остави стомната си, отиде в града и каза на хората:
\par 29 Дойдете да видите човек, който ми каза всичко, което съм сторила. Да не би Той да е Христос?
\par 30 Те излязоха от града и отиваха към Него.
\par 31 Между това учениците молеха Исуса, казвайки: Учителю, яж.
\par 32 А Той им рече: Аз имам храна да ям, за която вие не знаете.
\par 33 Затова учениците думаха помежду си: Да не би някой да Му е донесъл нещо за ядене?
\par 34 Каза им Исус: Моята храна е да върша волята на Онзи, Който ме е пратил, и да върша Неговата работа.
\par 35 Не казвате ли: Още четири месеца и жетвата ще дойде? Ето, казвам ви, подигнете очите си и вижте, че нивите са вече бели за жетва.
\par 36 Който жъне получава заплата, и събира плод за вечен живот, за да се радвате заедно и който сее и който жъне.
\par 37 Защото в това отношение истинна е думата, че един сее, а друг жъне.
\par 38 Аз ви пратих да жънете това, за което не сте се трудили; други се трудиха, а вие влязохте в наследството на техния труд.
\par 39 И от тоя град много самаряни повярваха в Него поради думите на жената, която свидетелствуваше: Той ми каза всичко, което съм сторила.
\par 40 И тъй, когато дойдоха самаряните при Него, помолиха Го да остане при тях; и преседя там два дни.
\par 41 И още мнозина повярваха поради неговото учение;
\par 42 и на жената казаха: Ние вярваме, не вече поради твоето говорене, понеже сами чухме и знаем, че той е наистина [Христос] Спасителят на света.
\par 43 След два дни Той излезе оттам и отиде в Галилея.
\par 44 Защото сам Исус заяви, че пророк няма почит в родината си.
\par 45 И тъй, когато дойде в Галилея, галилеяните Го приеха, като бяха видели всичко що стори в Ерусалим на празника; защото и те бяха отишли на празника.
\par 46 Прочее, Исус пак дойде в Кана Галилейска, гдето беше превърнал водата на вино. И имаше един царски чиновник, чиито син бе болен в Капериаум.
\par 47 Той, като чу, че Исус дошъл от Юдея в Галилея отиде при него и Го помоли да слезе и да изцели сина му, защото беше на умиране.
\par 48 Тогава Исус му рече: Ако не видите знамения и чудеса никак няма да повярвате.
\par 49 Царският чиновник Му каза: Господине, слез докле не е умряло детенцето ми.
\par 50 Каза му Исус: Иди си; син ти е жив. Човекът повярва думата, която му рече Исус, и си отиде.
\par 51 И когато той слизаше към дома си, слугите му го срещнаха и казаха, че син му е жив.
\par 52 А той ги попита в кой час му стана по-леко. Те му казаха: В седмия час треската го остави.
\par 53 И така бащата разбра, че това е станало в същия час, когато Исус му рече: Син ти е жив. И повярва той и целия му дом.
\par 54 Това второ знамение извърши Исус, като дойде пак из Юдея в Галилея.

\chapter{5}

\par 1 Подир това имаше юдейски празник, и Исус влезе в Ерусалим.
\par 2 А в Ерусалим, близо до овчата порта, се намира къпалня, наречена по еврейски Витесда, която има пет предверия.
\par 3 В тях лежаха множество болни, слепи, куци и изсъхнали, [които чакаха да се раздвижи водата.
\par 4 Защото от време на време ангел слизаше в къпалнята и размътваше водата; а който пръв влизаше след раздвижването на водата оздравяваше от каквато и болест да беше болен].
\par 5 И там имаше един човек болен от тридесет и осем години.
\par 6 Исус, като го видя да лежи, и узна, че от дълго време вече боледувал, каза му: Искаш ли да оздравееш?
\par 7 Болният Му отговори: Господине, нямам човек да ме спусне в къпалнята, когато се раздвижи водата, но докато дойда аз, друг слиза преди мене.
\par 8 Исус му казва: Стани, дигни постелката и си ходи.
\par 9 И на часа човекът оздравя, дигна постелката си, и започна да ходи. А тоя ден беше събота.
\par 10 Затова юдеите казаха на изцеления: Събота е, и не ти е позволено да дигнеш постелката си.
\par 11 Но той им отговори: Онзи, Който ме изцели, Той ми рече: Дигни постелката си и ходи?
\par 12 Попитаха го: Кой човек ти рече: Дигни постелката си и ходи?
\par 13 А изцеленият не знаеше Кой е; защото Исус беше се изплъзнал оттам, тъй като имаше множество народ на това място.
\par 14 По-после Исус го намери в храма и му рече: Ето, ти си здрав; не съгрешавай вече, за да те не сполети нещо по-лошо.
\par 15 Човекът отиде и извести на юдеите, че Исус е, Който го изцели.
\par 16 И затова юдеите гонеха Исуса, защото вършеше тия неща в събота.
\par 17 А Исус им отговори: Отец Ми работи до сега, и Аз работя.
\par 18 Затова юдеите искаха още повече да го убият; защото не само нарушаваше съботата, но и правеше Бога Свой Отец, и така правеше Себе Си равен на Бога.
\par 19 Затова Исус им рече: Истина, истина ви казвам, не може Синът да върши от само Себе Си нищо, освен това, което вижда да върши Отец; понеже каквото върши Той, подобно и Синът го върши.
\par 20 Защото Отец люби Сина, и Му показва все що върши сам; ще Му показва и от тия по-големи работи, за да се чудите вие.
\par 21 Понеже, както Отец възкресява мъртвите и ги съживява, така и Синът съживява, тия които иска.
\par 22 Защото, нито Отец не съди никого, но е дал на Сина да съди всички.
\par 23 за да почитат всички Сина, както почитат Отца. Който не почита Сина, не почита Отца, Който Го е пратил.
\par 24 Истина, истина ви казвам, който слуша Моето учение, и вярва в Този, Който Ме е пратил, има вечен живот, и няма да дойде на съд, но е преминал от смъртта в живота.
\par 25 Истина, истина ви казвам, иде час, и сега е, когато мъртвите ще чуят гласа на Божия Син, и които го чуят ще живеят.
\par 26 Защото, както Отец има живот в Себе Си, също така е дал и на Сина да има живот в Себе Си;
\par 27 и дал Му е власт да извършва съдба, защото е Човешкият Син.
\par 28 Недейте се чуди на това; защото иде час, когато всички, които са в гробовете, ще чуят гласа Му,
\par 29 и ще излязат; ония, които са вършили добро, ще възкръснат за живот, а които са вършили зло, ще възкръснат за осъждане.
\par 30 Аз не мога да върша нищо от Себе Си; съдя както чуя; и съдбата Ми е справедлива, защото не искам Моята воля, но волята на Онзи, Който Ме е пратил.
\par 31 Ако свидетелствувам Аз за Себе Си; свидетелството Ми не е истинно.
\par 32 Друг има, Който свидетелствува за Мене; и зная, че свидетелството, което Той дава за Мене е истинно.
\par 33 Вие пратихте до Иоана; и той засвидетелствува за истината.
\par 34 (Обаче свидетелството, което Аз приемам, не е от човека; но казвам това за да се спасите вие).
\par 35 Той беше светилото, което гореше и светеше; и вие пожелахте да се радвате за малко време на неговото светене.
\par 36 Но Аз имам свидетелство по-голямо от Иоановото; защото делата, които Отец Ми е дал да извърша, самите дела, които върша, свидетелствуват за Мене, че Отец Ме е пратил.
\par 37 И Отец, Който Ме е пратил, Той свидетелствува за Мене. Нито гласа сте Му чули някога, нито образа сте Му видели.
\par 38 И нямате Неговото слово постоянно в себе си, защото не вярвате Този, когото Той е пратил.
\par 39 Вие изследвате писанията, понеже мислите, че в тях имате вечен живот, и те са, които свидетелствуват за Мене,
\par 40 и пак не искате да дойдете при Мене, за да имате живот.
\par 41 От човеци слава не приемам;
\par 42 но зная, че вие нямате в себе си любов към Бога.
\par 43 Аз дойдох в името на Отца Си, и не Ме приемате; ако дойде друг в свое име, него ще приемете.
\par 44 Как можете да повярвате вие, които приемате слава един от друг, а не търсите славата, която е от единия Бог.
\par 45 Не мислете, че аз ще ви обвиня пред Отца; има един, който ви обвинява - Моисей, на когото вие се облагахте.
\par 46 Защото, ако вярвахте Моисея, повярвали бихте и Мене; понеже той за мене писа.
\par 47 Но ако не вярвате неговите писания, как ще повярвате Моите думи?

\chapter{6}

\par 1 След това Исус отиде на отвъдната страна на галилейското, то ест, тивериадското езеро .
\par 2 И подир Него вървеше едно голямо множество; защото гледаха знаменията, които вършеше над болните.
\par 3 И Исус се изкачи на хълма, и там седеше с учениците Си.
\par 4 А наближаваше юдейският празник пасхата.
\par 5 Исус, като подигна очи и видя, че иде към Него народ, каза на Филипа: Отгде да купим хляб да ядат тия?
\par 6 (А това каза за да го изпита; защото Той си знаеше какво щеше да направи).
\par 7 Филип Му отговори: За двеста динария хляб не ще им стигне, за да вземе всеки по малко.
\par 8 Един от учениците Му, Андрей, брат на Симона Петра, Му каза:
\par 9 Тук има едно момченце, у което се намират пет ечемичени хляба и две риби; но какво са те за толкова хора.
\par 10 Исус рече: накарайте човеците да насядат. А на това място имаше много трева; и тъй, насядаха около пет хиляди мъже на брой.
\par 11 Исус, прочее, взе хлябовете и, като благодари, раздаде ги на седналите; така и от рибите колкото искаха.
\par 12 И като се наситиха, каза на учениците Си: Съберете останалите къшеи за да не се изгуби нищо.
\par 13 И тъй, от петте ечемичени хляба събраха, и напълниха дванадесет коша с къшеи, останали на тия, които бяха яли.
\par 14 Тогава човеците, като видяха знамението, което Той извърши, казаха: Наистина, Тоя е пророкът, Който щеше да дойде на света.
\par 15 И тъй, Исус като разбра, че ще дойдат да Го вземат на сила, за да Го направят цар, пак се оттегли сам на хълма.
\par 16 А когато се свечери, учениците Му слязоха на езерото,
\par 17 и влязоха в ладия и отиваха отвъд езерото в Капернаум. И беше се вече стъмнило а Исус не бе дошъл още при тях;
\par 18 и езерото се вълнуваше, понеже духаше силен вятър.
\par 19 И като бяха гребали около двадесет и пет или тридесет стадии, видяха Исуса, че ходи по езерото и се приближава към ладията; и уплашиха се.
\par 20 Но той им каза: Аз съм; не бойте се!
\par 21 Затова бяха готови да Го вземат в ладията; и веднага ладията се намери при сушата, към която отиваха.
\par 22 На другия ден, народът, който стоеше отвъд езерото, като бе видял, че там няма друга ладийка, освен едната, и че Исус не беше влязъл с учениците Си в ладийката, но че учениците Му бяха тръгнали сами,
\par 23 (обаче други ладийки бяха дошли от Тивериада близо до мястото гдето бяха яли хляба, след като Господ бе благодарил),
\par 24 и тъй, народът, като видя че няма там Исуса, нито учениците Му, те сами влязоха в ладийката и дойдоха в Капернаум и търсеха Исуса.
\par 25 И като Го намериха отвъд езерото, рекоха Му: Учителю, кога си дошъл тука?
\par 26 В отговор Исус им рече: Истина, истина ви казвам, търсите Ме, не защото видяхте знамения, а защото ядохте от хлябовете и се наситихте.
\par 27 Работете, не за храна, която се разваля, а за храна, която трае за вечен живот, която Човешкият Син ще ви даде; защото Отец, Бог, Него е потвърдил с печата Си.
\par 28 Затова те Му рекоха: Какво да сторим, за да вършим Божиите дела?
\par 29 Исус в отговор им рече: Това е Божието дело, да повярвате в Този, Когото Той е изпратил.
\par 30 Тогава Му рекоха: Че Ти какво знамение правиш, за да видим и да Те повярваме? Какво вършиш?
\par 31 Бащите ни са яли манната в пустинята, както е писано: "Хляб от небето им даде да ядат".
\par 32 На това Исус им рече: Истина, истина ви казвам, не Моисей ви даде хляб от небето; а Отец Ми ви дава истинския хляб от небето.
\par 33 Защото Божият хляб е хлябът, който слиза от небето и дава живот на света.
\par 34 Те, прочее, Му рекоха: Господи, давай ни винаги тоя хляб.
\par 35 Исус им рече: Аз съм хлябът на живота; който дойде при Мене никак няма да огладнее, и който вярва в Мене никак няма да ожаднее.
\par 36 Но казвам ви, че вие Ме видяхте, и пак не вярвате.
\par 37 Всичко което Ми дава Отец, ще дойде при Мене, и който дойде при Мене никак няма да го изпъдя;
\par 38 защото слязох от небето не Моята воля да върша, а волята на Този, Който Ме е изпратил.
\par 39 И ето волята на Този, Който Ме е пратил: от всичко, което Ми е дал, да не изгубя нищо, но да го възкреся в последния ден.
\par 40 Защото това е волята на Отца Ми: всеки, който види Сина и повярва в Него, да има вечен живот, и Аз да го възкреся в последния ден.
\par 41 Тогава юдеите роптаеха против Него, за гдето рече: Аз съм хлябът, който е слязъл от небето.
\par 42 И казаха: Не е ли този Исус, Иосифовият син, Чиито баща и майка ние познаваме? Как казва Той сега: Аз съм слязъл от небето?
\par 43 Исус в отговор им рече: Не роптайте помежду си.
\par 44 Никой не може да дойде при Мене, ако не го привлече Отец, Който Ме е пратил и Аз ще го възкреся в последния ден.
\par 45 Писано е в пророците: "Всички ще бъдат научени от Бога". Всеки, който е чул от Отца, и се е научил, дохожда при Мене.
\par 46 Не, че е видял някой Отца, освен Онзи, Който е от Бога. Той е видял Отца.
\par 47 Истина, истина ви казвам, Който вярва [в Мене] има вечен живот.
\par 48 Аз съм хлябът на живота.
\par 49 Бащите ви ядоха манната в пустинята и все пак измряха.
\par 50 Тоя е хлябът, който слиза от небето, за да яде някой от него и да не умре.
\par 51 Аз съм живият хляб, който е слязъл от небето. Ако яде някой от тоя хляб, ще живее до века; да! и хлябът, който Аз ще дам, е Моята плът [която Аз ще дам] за живота на света.
\par 52 Тогава юдеите взеха да се препират помежду си, казвайки: Как може Този да ни даде да ядем плътта Му?
\par 53 Затова Исус им рече: Истина, истина ви казвам, ако не ядете плътта на Човешкия Син, и не пиете кръвта Му, нямате живот в себе си.
\par 54 Който се храни с плътта Ми и пие кръвта Ми, има вечен живот; и Аз ще го възкреся в последния ден.
\par 55 Защото Моята плът е истинска храна, и Моята кръв е истинско питие.
\par 56 Който се храни с Моята плът и пие Моята кръв, той пребъдва в Мене, и Аз в него.
\par 57 Както живият Отец Ме е пратил, и Аз живея чрез Отца, така и онзи, който се храни с Мене, ще живее чрез Мене.
\par 58 Тоя е хлябът, който слезе от небето; онзи който се храни с тоя хляб, ще живее до века, а не както бащите ви ядоха и измряха.
\par 59 Това рече Исус в синагогата, като поучаваше в Капернаум.
\par 60 И тъй, мнозина от учениците Му, като чуха това, рекоха: Тежко е това учение; кой може да го слуша?
\par 61 Но Исус като знаеше в Себе Си, че учениците Му за туй негодуват, рече им: Това ли ви съблазнява?
\par 62 Тогава, какво ще кажете, ако видите Човешкият Син да възлиза там, гдето е бил изпърво?
\par 63 Духът е, който дава живот; плътта нищо не ползува; думите, които съм ви говорил, дух са и живот са.
\par 64 Но има някои от вас, които не вярват. Защото Исус отначало знаеше кои са невярващите, и кой е тоя, който щеше да Го предаде.
\par 65 И каза: Затова ви рекох, че никой не може да дойде при Мене ако не му е дадено от Отца.
\par 66 Поради това мнозина от учениците Му отстъпиха, и не ходеха вече с Него.
\par 67 За туй Исус рече на дванадесетте: Да не искате и вие да си отидете?
\par 68 Симон Петър Му отговори: Господи, при кого да отидем? Ти имаш думи на вечен живот,
\par 69 и ние вярваме и знаем, че Ти си [Христос, Син на живия Бог] Светият Божий.
\par 70 Исус им отговори: Не Аз ли избрах вас дванадесетте, и един от вас е дявол?
\par 71 А Той говореше за Юда Симонов Искариотски; защото той, един от дванадесетте, щеше да Го предаде.

\chapter{7}

\par 1 След това Исус ходеше по Галилея; защото не искаше да ходи по Юдея, понеже юдеите искаха да Го убият.
\par 2 А наближаваше юдейският празник шатроразпъване.
\par 3 Затова Неговите братя Му рекоха: Замини оттука и иди в Юдея, така че и Твоите ученици да видят делата, които вършиш;
\par 4 защото никой, като иска сам да бъде известен, не върши нещо скришно. Щом вършиш тия дела, яви Себе Си на света.
\par 5 (Защото нито братята Му вярваха в Него).
\par 6 А Исус им каза: Моето време още не е дошло; а вашето време винаги е готово.
\par 7 Вас светът не може да мрази; а Мене мрази, защото Аз заявявам за него, че делата му са нечестиви.
\par 8 Възлезте вие на празника; Аз няма още да възляза на тоя празник, защото времето Ми още не се е навършило.
\par 9 И като им рече това, остана си в Галилея.
\par 10 А когато братята Му бяха възлезли на празника, тогава и Той възлезе, не явно, а тайно някак си.
\par 11 Юдеите, прочее, Го търсеха на празника, и казваха: Къде е Онзи?
\par 12 И имаше за Него много глъчка между народа; едни казваха: Добър човек е; други казваха: Не е, но заблуждава народа.
\par 13 Обаче никой не говореше положително за Него поради страха от юдеите.
\par 14 Но като се преполовяваше вече празникът, Исус възлезе в храма и почна да поучава.
\par 15 Затова юдеите се чудеха и казваха: Как знае Този книга, като не се е учил?
\par 16 Исус, прочее, в отговор им каза: Моето учение не е Мое, а на Онзи, Който Ме е пратил.
\par 17 Ако иска някой да върши Неговата воля, ще познае дали учението е от Бога, или Аз от Себе Си говоря.
\par 18 Който говори от себе си търси своята си слава; а който търси славата на Онзи, Който Го е пратил, Той е истински, и в Него няма неправда.
\par 19 Не даде ли ви Моисей закона? но пак никой от вас не изпълнява закона. Защо искате да Ме убиете?
\par 20 Народът отговори: Бяс имаш. Кой иска да Те убие?
\par 21 Исус в отговор им рече: Едно дело извърших и всички се чудите на Мене поради него.
\par 22 Моисей ви даде обрязването, (не че е от Моисея, но от бащите); и в събота обрязвате човека.
\par 23 Ако се обрязва човек в събота, за да се не наруши Моисеевия закон, на Мене ли се гневите за гдето изцяло оздравих човек в събота.
\par 24 Не съдете по изглед, но съдете справедливо.
\par 25 Тогава някои от ерусалимляните казаха: Не е ли Този човекът, Когото искат да убият?
\par 26 Ето Той явно говори, и нищо не Му казват. Да не би първенците положително да знаят, че Този е Христос?
\par 27 Обаче Този знаем от къде е; а когато дойде Христос, никой няма да знае от къде е.
\par 28 Затова Исус, като поучаваше в храма, извика казвайки: И Мене познавате, и от къде съм знаете; и Аз от само Себе Си не съм дошъл, но истинен е Този, Който Ме е пратил, Когото вие не познавате.
\par 29 Аз Го познавам, защото съм от Него, и Той Ме е пратил.
\par 30 И тъй, искаха да Го хванат; но никой не тури ръка на Него, защото часът Му още не беше дошъл.
\par 31 Обаче, мнозина от народа повярваха в Него, като казваха: Когато дойде Христос, нима ще извърши повече знамения от тия, които Този е извършил?
\par 32 Фарисеите чуха, че тъй шушукал народът за Него; и главните свещеници и фарисеите пратиха служители да Го хванат.
\par 33 Исус, прочее, рече: Още малко време съм с вас, и тогава ще отида при Онзи, Който Ме е пратил.
\par 34 Ще Ме търсите и няма да Ме намерите; и гдето съм Аз, вие не можете да дойдете.
\par 35 На това юдеите рекоха помежду си: Къде ще отиде Тоя та няма да можем да Го намерим? Да не би да отиде при разпръснатите между гърците и да поучава гърците?
\par 36 Какво значи тази дума, която рече: Ще Ме търсите, и няма да Ме намерите; и, гдето съм Аз вие не можете да дойдете?
\par 37 А в последния ден, великия ден на празника, Исус застана и извика казвайки: Ако е някой жаден, нека дойде при Мене и да пие.
\par 38 Ако някой вярва в Мене, реки от жива вода ще потекат от утробата му, както рече писанието.
\par 39 А това каза за Духа, който вярващите в Него щяха да приемат; защото [Светият] Дух още не бе даден, понеже Исус още не бе се прославил.
\par 40 За туй, някои от народа, които чуха тия думи казваха: Наистина Тоя е пророкът.
\par 41 Други казваха: Тоя е Христос. Някои пък казваха: Нима от Галилея ще дойде Христос?
\par 42 Не рече ли писанието, че Христос ще дойде от Давидовото потомство, и от градеца Витлеем, гдето беше Давид?
\par 43 И така възникна раздор между народа за Него.
\par 44 И някои от тях искаха да Го хванат; но никой не тури ръце на Него.
\par 45 Тогава служителите си дойдоха при главните свещеници и фарисеите; и те им рекоха: Защо Го не доведохте?
\par 46 Служителите отговориха: Никога човек не е говорил така [както Тоя Човек].
\par 47 А фарисеите им отговориха: И вие ли сте заблудени?
\par 48 Повярвал ли е в Него някой от първенците или от фарисеите?
\par 49 Но това простолюдие, което не знае закона проклето е.
\par 50 Никодим, който беше един от тях, (същият, който бе ходил при Него нощем по-рано), им казва:
\par 51 Нашият закон осъжда ли човека, ако първо не го изслуша и не разбере що върши?
\par 52 Те в отговор му рекоха: Да не си и ти от Галилея? Изследвай и виж, че от Галилея пророк не се издига.
\par 53 И всеки отиде у дома си.

\chapter{8}

\par 1 А Исус отиде на Елеонския хълм.
\par 2 И рано сутринта пак дойде в храма; а всичките люде дохождаха при Него, и Той седна и ги поучаваше.
\par 3 И книжниците и фарисеите доведоха [при Него] една жена уловена в прелюбодейство и, като я поставиха насред, казаха Му:
\par 4 Учителю, тази жена биде уловена в самото дело на прелюбодейство.
\par 5 А Моисей ни е заповядал в закона да убиваме такива с камъни; Ти, прочее, що казваш за нея?
\par 6 И това казаха да Го изпитват, за да имат за какво да Го обвиняват. А Исус се наведе на долу и пишеше с пръст на земята.
\par 7 Но като постоянствуваха да Го питат, Той се изправи и рече им: Който от вас е безгрешен нека пръв хвърли камък на нея.
\par 8 И пак се наведе на долу, и пишеше с пръст на земята.
\par 9 А те, като чуха това разотидоха се един по един, като почнаха от по-старите и следваха до последните; и Исус остана сам, и жената, гдето си беше, насред.
\par 10 И когато се изправи, Исус й рече: Жено, къде са тези, [които те обвиняваха]? Никой ли не те осъди?
\par 11 И тя отговори: Никой Господи. Исус рече: Нито Аз те осъждам; иди си, отсега не съгрешавай вече.
\par 12 Тогава Исус пак им говори, казвайки: Аз съм светлината на света; който Ме следва няма да ходи в тъмнината, но ще има светлината на живота.
\par 13 Затова фарисеите Му рекоха: Ти сам свидетелствуваш за Себе Си; Твоето свидетелство не е истинно.
\par 14 Исус в отговор им рече: Ако и да свидетелствувам за Себе Си, пак свидетелството Ми е истинно; защото зная от къде съм дошъл и на къде отивам; а вие не знаете от къде ида или на къде отивам.
\par 15 Вие съдите по плът; Аз не съдя никого.
\par 16 И даже ако съдя Моята съдба е истинна, защото не съм сам, но Аз съм, и Отец, Който Ме е изпратил.
\par 17 И в закона, да, във вашия закон е писано, че свидетелството на двама човека е истинно.
\par 18 Аз съм, Който свидетелствувам за Себе Си; и Отец, Който Ме е пратил, свидетелствува за Мене.
\par 19 Тогава Му казаха: Где е Твоят Отец? Исус отговори: Нито Мене познавате, нито Отец Ми; ако познавахте Мене, бихте познавали и Отца Ми.
\par 20 Тия думи Той изговори в съкровищницата, като поучаваше в храма; и никой не Го хвана, защото часът Му не беше още дошъл.
\par 21 И пак им рече Исус: Аз си отивам; и ще Ме търсите, но в греха си ще умрете. Гдето отивам Аз, вие не можете да дойдете?
\par 22 Затова юдеите казаха: Да не би да се самоубие, че казва: Гдето отивам Аз, вие не можете да дойдете?
\par 23 И рече им: Вие сте от тия, които са долу; Аз съм от ония, които са горе. Вие сте от този свят; а Аз не съм от този свят.
\par 24 По тая причина ви рекох, че ще умрете в греховете си; защото ако не повярвате, че съм това, което казвам, в греховете си ще умрете.
\par 25 Те, прочее, Му рекоха: Ти кой си? Исус им каза: Преди всичко, Аз съм именно това, което ви казвам.
\par 26 Много нещо имам да говоря и да съдя за вас; но Този, Който Ме е пратил, е истинен; и каквото съм чул от Него, това говоря на света.
\par 27 Те не разбраха, че им говореше за Отца.
\par 28 Тогава рече Исус: Когато издигнете Човешкия Син, тогава ще познаете, че съм това което казвам, и че от Себе Си нищо не върша, но каквото Ме е научил Отец, това говоря.
\par 29 И Този, Който ме е пратил, с Мене е; не Ме е оставил сам, Аз върша всякога онова, което е Нему угодно.
\par 30 Когато говореше това, мнозина повярваха в Него.
\par 31 Тогава Исус каза на повярвалите в Него юдеи: Ако пребъдвате в Моето учение, наистина сте Мои ученици;
\par 32 и ще познаете истината и истината ще ви направи свободни.
\par 33 Отговориха Му: Ние сме Авраамово потомство и никога никому не сме били слуги; как казваш Ти: Ще станете свободни.
\par 34 Исус им отговори: Истина, истина ви казвам, всеки, който върши грях, слуга е на греха.
\par 35 А слугата не остава вечно в дома; синът остава вечно.
\par 36 Прочее, ако Синът ви освободи, ще бъдете наистина свободни.
\par 37 Зная, че сте Авраамово потомство; но пак искате да Ме убиете, защото за Моето учение няма място във вас.
\par 38 Аз говоря това, което съм видял у Моя Отец; също и вие вършите това, което сте чули от вашия баща.
\par 39 Те в отговор Му казаха: Наш баща е Авраам. Исус им каза: Ако бяхте Авраамови чада, Авраамовите дела щяхте да вършите.
\par 40 А сега искате да убиете Мене, Човека, Който ви казах истината, която чух от Бога. Това Авраам не е сторил.
\par 41 Вие вършите делата на баща си. Те му рекоха: Ние не сме родени от блудство; един Отец имаме, Бога.
\par 42 Исус им рече: Ако беше Бог вашият Отец, то вие щяхте да Ме любите, защото Аз от Бога съм излязъл и дошъл; понеже Аз не съм дошъл от Себе Си, но Той Ме прати.
\par 43 Защо не разбирате Моето говорене? Защото не можете да слушате Моето учение.
\par 44 Вие сте от баща дявола, и желаете да вършите похотите на баща си. Той беше открай човекоубиец, и не устоя в истината; защото в него няма истина. Когато изговаря лъжа, от своите си говори, защото е лъжец, и на лъжата баща.
\par 45 А понеже Аз говоря истината, вие не Ме вярвате.
\par 46 Кой от вас Ме обвинява в грях? Но ако говоря истина, защо не Ме вярвате?
\par 47 Който е от Бога, той слуша Божиите думи; вие затова не слушате, защото не сте от Бога.
\par 48 Юдеите в отговор Му рекоха: Не казваме ли ние право, че си самарянин и имаш бяс?
\par 49 Исус отговори: Нямам бяс; но Аз почитам Отца Си, а вие Ме позорите.
\par 50 Но Аз не търся слава за Себе Си; има Един, Който търси и съди.
\par 51 Истина, истина ви казвам, ако някой опази Моето учение, няма да види смърт до века.
\par 52 Юдеите Му рекоха: Сега знаем, че имаш бяс. Авраам умря, също и пророците; а Ти казваш: Ако някой опази Моето учение, няма да вкуси смърт до века.
\par 53 Нима Ти си по-голям от баща ни Авраама, който умря? И пророците умряха. Ти на какъв се правиш?
\par 54 Исус отговори: Ако славя Аз Себе Си, славата Ми е нищо; Отец Ми е, Който Ме слави, за Когото вие казвате, че е ваш Бог;
\par 55 и пак не сте Го познали. Но Аз Го познавам; и ако река, че не Го познавам, ще бъда като вас лъжец; но Аз го познавам и пазя словото Му.
\par 56 Баща ви Авраам се възхищаваше, че щеше да види Моя ден; и видя го и се зарадва.
\par 57 Юдеите Му рекоха: Петдесет години още нямаш, и Авраам ли си видял?
\par 58 Исус им рече: Истина, истина ви казвам, преди да се е родил Авраам, Аз съм.
\par 59 Тогава взеха камъни да хвърлят върху Му; но Исус се скри и излезе от храма, [минавайки през сред; и така си отиде].

\chapter{9}

\par 1 И когато заминаваше, видя един сляпороден човек.
\par 2 И учениците Му Го попитаха казвайки: Учителю, поради чий грях, негов ли, или на родителите му, той се е родил сляп?
\par 3 Исус отговори: Нито поради негов грях, нито на родителите му, но за да се явят в него Божиите дела.
\par 4 Ние трябва да вършим делата на Този, Който Ме е пратил, докле е ден; иде нощ, когато никой не може да работи.
\par 5 Когато съм в света, светлина съм на света.
\par 6 Като рече това, плюна на земята, направи кал с плюнката и намаза с калта очите му;
\par 7 и рече му: Иди, умий се в къпалнята Силоам (което значи Пратен). И тъй, той отиде, уми се, и дойде прогледал.
\par 8 А съседите и ония, които бяха виждали по-преди, че беше просяк, казваха: Не е ли този, който седеше и просеше?
\par 9 Едни казваха: Той е; други казваха: Не, а прилича на него. Той каза: Аз съм.
\par 10 Затова му рекоха: Тогава как ти се отвориха очите?
\par 11 Той отговори: Човекът който се нарича Исус, направи кал, намаза очите ми, и рече ми: Иди на Силоам и умий се. И тъй отидох, и като се умих прогледах.
\par 12 Тогава му рекоха: Где е Той? Казва: Не зная.
\par 13 Завеждат при фарисеите някогашния слепец.
\par 14 А беше събота, когато Исус направи калта и му отвори очите.
\par 15 И тъй, пак го питаха и фарисеите как е прогледал. И той им рече: Кал тури на очите ми, умих се и гледам.
\par 16 Затова някои от фарисеите казваха: Този човек не е от Бога, защото не пази съботата. А други казваха: Как може грешен човек да върши такива знамения? И възникна раздор по между им.
\par 17 Казват пак на слепеца: Ти що казваш за Него, като ти е отворил очите? И той рече: Пророк е.
\par 18 Но юдеите не вярваха, че той е бил сляп и прогледал, докато не повикаха родителите на прогледалия,
\par 19 та ги попитаха казвайки: Тоя ли е вашият син, за когото казвате, че се е родил сляп? А сега как вижда?
\par 20 Родителите в отговор казаха: Знаем, че този е нашият син, и че се роди сляп;
\par 21 а как сега вижда, не знаем; или кой му е отворил очите, не знаем; него питайте, той е на възраст, сам нека говори за себе си.
\par 22 Това казаха родителите му, защото се бояха от юдеите; понеже юдеите се бяха вече споразумели помежду си да отлъчат от синагогата онзи, който би Го изповядал, че е Христос.
\par 23 По тая причина родителите му рекоха: Той е на възраст, него питайте.
\par 24 И тъй, повикаха втори път човека, който беше сляп, и му рекоха: Въздай слава на Бога; ние знаем, че този човек е грешник.
\par 25 А той отговори: Дали е грешник, не зная; едно зная, че бях сляп, а сега виждам.
\par 26 Затова му казаха: Какво ти направи? Как ти отвори очите?
\par 27 Отговори им: Казах ви ей сега, и не чухте; защо искате пак да чуете? Да не би и вие да искате да Му станете ученици?
\par 28 Тогава те го изругаха, казвайки: Ти си Негов ученик; а ние сме Моисееви ученици.
\par 29 Ние знаем, че на Моисея Бог е говорил, а Този не знаем откъде е.
\par 30 Човекът в отговор им каза: Това е чудно, че вие не знаете от къде е, но пак ми отвори очите.
\par 31 Знаем, че Бог не слуша грешници; но ако някой е благочестив и върши Божията воля, него слуша.
\par 32 А пък от века не се е чуло да е отворил някой очи на сляпороден човек.
\par 33 Ако не беше Този Човек от Бога, не би могъл нищо да стори.
\par 34 Те в отговор му казаха: Ти цял в грехове си роден, и нас ли учиш? И го изпъдиха вън.
\par 35 Чу Исус, че го изпъдили вън, и, като го намери, рече: Ти вярваш ли в Божия Син?
\par 36 Той в отговор рече: А Кой е Той, Господи, за да вярвам в Него?
\par 37 Исус му рече: И видял си Го, и Който говори с тебе, Той е.
\par 38 А той рече: Вярвам Господи; и поклони Му се.
\par 39 И Исус рече: За съдба дойдох Аз на тоя свят, за да виждат невиждащите, а виждащите да ослепеят.
\par 40 Ония от фарисеите, които бяха с Него, като чуха това, рекоха Му: Да не сме и ние слепи?
\par 41 Исус им рече: Ако бяхте слепи, не бихте имали грях, но понеже сега казвате: Виждаме, грехът ви остава.

\chapter{10}

\par 1 Истина, истина ви казвам, който не влиза през вратата на кошарата на овцете, но прескача от другаде, той е крадец и разбойник.
\par 2 А който влиза през вратата, овчар е на овцете.
\par 3 Нему вратарят отваря; и овцете слушат гласа му; и вика своите овце по име и ги извежда.
\par 4 Когато е изкарал всичките свои, върви пред тях; и овцете го следват, защото познават гласа му.
\par 5 А подир чужд човек няма да следват, но ще побягнат от него; защото не познават гласа на чуждите.
\par 6 Тази притча им каза Исус; но те не разбраха какво им говореше.
\par 7 Тогава Исус пак рече: Истина, истина ви казвам, Аз съм вратата на овцете.
\par 8 Всички, които са дошли преди Мене са крадци и разбойници; но овцете не ги послушаха.
\par 9 Аз съм вратата; през Мене ако влезе някой, ще бъде спасен, и ще влиза, и ще излиза, и паша ще намира.
\par 10 Крадецът влиза само да открадне, да заколи и да погуби: Аз дойдох за да имат живот, и да го имат изобилно.
\par 11 Аз съм добрият пастир; добрият пастир живота си дава за овцете.
\par 12 Който е наемник, а не овчар, и не е стопанин на овцете, вижда вълка, че иде, и, като оставя овцете, бяга; и вълкът ги разграбва и разпръсва.
\par 13 Той бяга защото е наемник, и не го е грижа за овцете.
\par 14 Аз съм добрият пастир, и познавам Моите, и Моите Мене познават,
\par 15 също както Отец познава Мене, и Аз познавам Отца; и Аз давам живота Си за овцете.
\par 16 И други овце имам, които не са от тая кошара, и тях трябва да доведа; и ще чуят гласа Ми; и ще станат едно стадо с един пастир.
\par 17 Затова Ме люби Отец, защото Аз давам живота Си, за да го взема пак.
\par 18 Никой не Ми го отнема, но Аз от Себе Си го давам. Имам право да го дам, и имам право пак да го взема. Тая заповед получих от Отца Си.
\par 19 Поради тия думи пак възникна раздор между юдеите.
\par 20 Мнозина от тях казваха: Бяс има, и луд е; защо Го слушате?
\par 21 Други казваха: Тия думи не са на човек хванат от бяс. Може ли бяс да отваря очи на слепи?
\par 22 И настъпи в Ерусалим празникът на освещението на храма. Беше зима;
\par 23 и ходеше Исус в Соломоновия трем на храма.
\par 24 Между това юдеите Го заобиколиха и Му казаха: До кога ще ни държиш в съмнение? Ако си Ти Христос, кажи ни ясно.
\par 25 Исус им отговори: Казах ви, и не вярвате. Делата, които върша в името на Отца Си, те свидетелствуват за Мене.
\par 26 Но вие не вярвате, защото не сте от Моите овце.
\par 27 Моите овце слушат гласа Ми, и Аз ги познавам, и те Ме следват.
\par 28 И Аз им давам вечен живот; и те никога няма да загинат, и никой няма да ги грабне от ръката Ми.
\par 29 Отец Ми, Който Ми ги даде, е по-голям от всички; и никой не може да ги грабне от ръката на Отца.
\par 30 Аз и Отец едно сме.
\par 31 Юдеите пак взеха камъни за да Го убият.
\par 32 Исус им отговори: Много добри дела ви показах от Отца; за кое от тия дела искате да Ме убиете с камъни?
\par 33 Юдеите Му отговориха: Не за добро дело искаме да Те убием с камъни, а за богохулство, и защото Ти, бидейки човек, правиш Себе Си Бог.
\par 34 Исус им отговори: Не е ли писано във вашия закон: "Аз рекох, богове сте вие"?
\par 35 Ако са наречени богове ония, към които дойде Божието слово, (и написаното не може да се наруши),
\par 36 то на Този, Когото Бог освети и прати на света, казвате ли, богохулствуваш, защото рекох, Аз съм Божий Син?
\par 37 Ако не върша делата на Отца Си, недейте Ми вярва;
\par 38 но ако ги върша, то, макар да не вярвате на Мене, вярвайте на делата, за да познаете и разберете, че Отец е в Мене, и Аз в Отца.
\par 39 Пак искаха да Го хванат; но Той избяга от ръката им.
\par 40 И отиде отвъд Иордан, на мястото, гдето Иоан по-преди кръщаваше и остана там.
\par 41 И мнозина дойдоха при Него; и казаха: Иоан не извърши никое знамение; но всичко, що каза Иоан за Този беше истинно.
\par 42 И там мнозина повярваха в Него.

\chapter{11}

\par 1 Един човек на име Лазар, от Витания, от селото на Мария и на сестра й Марта, беше болен.
\par 2 (А Мария, чийто брат Лазар беше болен, бе оная, която помаза Господа с миро и отри нозете Му с косата си).
\par 3 И тъй, сестрите пратиха до Него да Му кажат: Господи, ето този, когото обичаш, е болен.
\par 4 А Исус, като чу това, рече: Тази болест не е смъртоносна, но е за Божията слава, за да се прослави Божият Син чрез нея.
\par 5 А Исус обичаше Марта, и сестра й, и Лазара.
\par 6 Тогава откак чу, че бил болен, престоя два дни на мястото, гдето се намираше.
\par 7 А подир това, каза на учениците: Да отидем пак в Юдея.
\par 8 Казват Му учениците: Учителю, сега юдеите искаха да Те убият с камъни, и пак ли там отиваш?
\par 9 Исус отговори: Нали има дванадесет часа в деня? Ако ходи някой денем, не се препъва, защото вижда виделината на тоя свят.
\par 10 Но ако ходи някой нощем, препъва се, защото не е осветлен.
\par 11 Това изговори, и подир туй им каза: Нашият приятел Лазар заспа; но Аз отивам да го събудя.
\par 12 Затова учениците Му рекоха: Господи, ако е заспал, ще оздравее.
\par 13 Но Исус бе говорил за смъртта му; а те мислеха, че говори за почиване в сън.
\par 14 Тогава Исус им рече ясно: Лазар умря.
\par 15 И заради вас, радвам се, че не бях там, за да повярвате; обаче, нека да отидем при него.
\par 16 Тогава Тома, наречен близнак, каза на съучениците: Да отидем и ние, за да умрем с Него.
\par 17 И тъй, като дойде Исус, намери, че Лазар бил от четири дни в гроба.
\par 18 А Витания беше близо до Ерусалим, колкото петнадесет стадии;
\par 19 и мнозина от юдеите бяха при Марта и Мария да ги утешават за брат им.
\par 20 Марта, прочее, като чу, че идел Исус, отиде да Го посрещне; а Мария още седеше в къщи.
\par 21 Тогава Марта рече на Исуса: Господи, да беше Ти тука, не щеше да умре брат ми.
\par 22 Но и сега зная, че каквото и да поискаш от Бога, Бог ще Ти даде.
\par 23 Казва й Исус: Брат ти ще възкръсне.
\par 24 Казва Му Марта: Зная, че ще възкръсне във възкресението на последния ден.
\par 25 Исус й рече: Аз съм възкресението и живота; който вярва в Мене, ако и да умре, ще живее;
\par 26 и никой, който е жив и вярва в Мене, няма да умре до века. Вярваш ли това?
\par 27 Казва му: Да Господи, вярвам, че Ти си Христос, Божият Син, Който има да дойде на света.
\par 28 И като рече това, отиде да повика скришом сестра си Мария, казвайки: Учителят е дошъл и те вика.
\par 29 И тя, щом чу това, стана бързо и отиде при Него.
\par 30 Исус още не беше дошъл в селото, а беше на мястото, гдето Го посрещна Марта.
\par 31 А юдеите, които бяха с нея в къщи и я утешаваха, като видяха, че Мария стана бързо и излезе, отидоха подире й, като мислеха че отива на гроба да плаче там.
\par 32 И тъй, Мария, като дойде там гдето беше Исус и Го видя, падна пред нозете Му и рече Му: Господи, да беше Ти тука, нямаше да умре брат ми.
\par 33 Исус, като я видя, че плаче, и юдеите, които я придружаваха, че плачат, разтъжи се в духа си и се смути.
\par 34 И рече: Где го положихте? Казват Му: Господи, дойди и виж.
\par 35 Исус се просълзи.
\par 36 Затова юдеите думаха: Виж колко го е обичал!
\par 37 А някои от тях рекоха: Не можеше ли Този, Който отвори очите на слепеца, да направи така, че и този да не умре?
\par 38 Исус, прочее, като тъжеше пак в Себе Си, дохожда на гроба. Беше пещера, и на нея бе привален камък.
\par 39 Казва Исус: Отместете камъка. Марта, сестрата на умрелия, Му казва: Господи, смърди вече, защото е от четири дни в гроба.
\par 40 Казва й Исус: Не рекох ли ти, че ако повярваш ще видиш Божията слава?
\par 41 И тъй, отместиха камъка. А Исус подигна очи нагоре и рече: Отче, благодаря Ти, че Ме послуша.
\par 42 Аз знаех, че Ти винаги Ме слушаш; но това казах заради народа, който стои наоколо, за да повярват, че Ти си Ме пратил.
\par 43 Като каза това, извика със силен глас: Лазаре, излез вън!
\par 44 Умрелият излезе, с ръце и нозе повити в саван, и лицето му забрадено с кърпа. Исус им каза: Разповийте го и оставете го да си иде.
\par 45 Тогава мнозина от юдеите, които бяха дошли при Мария и видяха това що стори Исус, повярваха в Него.
\par 46 А някои от тях отидоха при фарисеите и казаха им какво бе извършил Исус.
\par 47 Затова главните свещеници и фарисеите събраха съвет и казаха: Какво правим ние? Защото Този човек върши много знамения.
\par 48 Ако Го оставим така, всички ще повярват в Него; и римляните като дойдат ще отнемат и страната ни и народа ни.
\par 49 А един от тях на име Каиафа, който беше първосвещеник през тая година, им рече: Вие нищо не знаете,
\par 50 нито вземате в съображение, че за вас е по-добре един човек да умре за людете, а не да загине целият народ.
\par 51 Това не каза от себе си, но бидейки първосвещеник през оная година, предсказа, че Исус ще умре за народа,
\par 52 и не само за народа, но и за да събере в едно разпръснатите Божии чада.
\par 53 И тъй, от онзи ден те се съветваха да Го умъртвят.
\par 54 Затова Исус вече не ходеше явно между юдеите, но оттам отиде в страната близо до пустинята, в един град наречен Ефраим, и там остана с учениците.
\par 55 А наближаваше юдейската пасха; и мнозина от провинцията отидоха в Ерусалим преди пасхата, за да се очистят.
\par 56 И така, те търсеха Исуса, и, стоейки в храма, разговаряха се помежду си: Как ви се вижда? няма ли да дойде на празника?
\par 57 А главните свещеници и фарисеите бяха издали заповед, щото, ако узнае някой къде е, да извести, за да Го уловят.

\chapter{12}

\par 1 А шест дни преди пасхата Исус дойде във Витания, където беше Лазар, когото Той възкреси от мъртвите.
\par 2 Там му направиха вечеря, и Марта прислужваше; а Лазар беше един от тия, които седяха с Него на трапезата.
\par 3 Тогава Мария, като взе един литър миро от чист и скъпоценен нард, помаза нозете на Исуса, и с косата си отри нозете Му; и къщата се изпълни с благоухание от мирото.
\par 4 Но един от учениците Му, Юда Искариотски, който щеше да Го предаде, рече:
\par 5 Защо не се продаде това миро за триста динария, за да се раздадат на сиромасите?
\par 6 А това, рече не защото го беше грижа за сиромасите, а защото бе крадец, и като държеше касата светлината. Това изговори Исус и отиде та се скри от тях.
\par 7 Тогава Исус рече: Оставете я; понеже го е запазила за деня на погребението Ми.
\par 8 Защото сиромасите всякога се намират между вас но Аз не се намирам всякога.
\par 9 А голямо множество от юдеите узнаха, че е там; и дойдоха, не само поради Исуса, но за да видят и Лазара, когото възкресил от мъртвите.
\par 10 А главните свещеници се наговориха да убият и Лазара,
\par 11 защото поради него мнозина от юдеите отиваха към страната на Исуса и вярваха в Него.
\par 12 На следния ден едно голямо множество, което бе дошло на праздника, като чуха, че Исус идел в Ерусалим,
\par 13 взеха палмови клони и излязоха да Го посрещнат, викайки: Осана! благословен, Който иде в Господното име, Израилевия Цар!
\par 14 А Исус като намери едно осле, възседна го, според както е писано: -
\par 15 "Не бой се дъщерьо Сионова. Ето твоят Цар иде, Възседнал на осле";
\par 16 Учениците Му изпърво не разбраха това: а когато се прослави Исус, тогава си спомниха, че това бе писано за Него, и че Му сториха това.
\par 17 Народът, прочее, който беше с Него, когато повика Лазара от гроба и го възкреси от мъртвите, свидетелстваше за това чудо.
\par 18 По същата причина Го посрещна и народът, защото чуха, че извършил това знамение.
\par 19 За туй фарисеите рекоха помежду си: Вижте, че нищо не постигате! Ето, светът отиде след Него.
\par 20 А между ония, които дойдоха на поклонение по праздника, имаше и някои гърци.
\par 21 Те, прочее, дойдоха при Филипа, който беше от Витсаида галилейска, и го помолиха, казвайки: Господине искаме да видим Исуса.
\par 22 Филип дохожда и казва на Андрея; Андрей дохойда, и Филип, и те казват на Исуса.
\par 23 А Исус в отговор им казва: Дойде часът да се прослави Човешкият Син.
\par 24 Истина, истина ви казвам, ако житното зърно не падне в земята и не умре, то си остава самотно; но ако умре, дава много плод.
\par 25 Който обича живота си, ще го изгуби; и който мрази живота си на тоз свят, ще го запази за вечен живот.
\par 26 Ако служи някой на Мене, Мене нека последва; и дето съм Аз, там ще бъде и служителят Ми. Който служи на Мене, него ще почете Отец Ми.
\par 27 Сега душата Ми е развълнувана; и какво да кажа? Отче избави Ме от този час. Но за това дойдох на тоя час.
\par 28 Отче, прослави името Си. Тогава дойде глас от небето: И Го прославих, и пак ще Го прославя.
\par 29 На това, народът, който стоеше там, като чу гласа каза: Гръм е. Други пък казаха: Ангел Му проговори.
\par 30 Исус в отговор рече: Този глас не дойде заради Мене, но заради вас.
\par 31 Сега е съдба на този свят; сега князът на този свят ще бъде извърлен вън.
\par 32 И когато бъда Аз издигнат от земята, ще привлека всички при Себе Си.
\par 33 А като казваше това, Той означаваше от каква смърт щеше да умре.
\par 34 Народът, прочее, Му отговори: Ние сме чули от закона, че Христос пребъдва до века: тогава как казваш Ти, че Човешкият Син трябва да бъде издигнат? Кой е Тоя Човешки Син?
\par 35 Тогава Исус им рече: Още малко време светлината е между вас. Ходете докле имате светлината, за да ви не настигне тъмнината. Който ходи в тъмнината не знае къде отива.
\par 36 Докле имате светлината, вярвайте в свтлината, за да станете просветени чрез светлината. Това изговори Исус, и отиде та се скри от тях.
\par 37 Но ако и да бе извършил толкова знамения пред тях, те пак не вярваха в Него;
\par 38 за да се изпълни казаното от пророк Исаия, който рече: - "Господи, кой от нас е повярвал на онова, което сме чули? И мишцата Господня на кого се е открила"?
\par 39 Те за това не можаха да вярват, защото Исаия пак е рекъл: -
\par 40 "Ослепил е очите им, и закоравил сърцата им, Да не би с очи да видят, и със сърца да разберат, За да се обърнат и да ги изцеля".
\par 41 Това каза Исаия защото видя славата Му и говори за Него.
\par 42 Но пак мнозина от първенците повярваха в Него; но поради фарисеите не Го изповядаха, за да не бъдат отлъчени от синагогата;
\par 43 защото обикнаха похвалата от човеците повече от похвалата от Бога.
\par 44 А Исус извика и рече: Който вярва в Мене, не в Мене вярва, но в Този, Който Ме е пратил.
\par 45 И който гледа Мене, гледа Онзи, Който Ме е пратил.
\par 46 Аз дойдох като светлина на света, за да не остане в тъмнина никой, който вярва в Мене.
\par 47 И ако чуе някой думите Ми и не ги пази, Аз не го съдя; защото не дойдох да съдя света, но да спася света.
\par 48 Който Ме отхвърля, и не приема думите Ми, има кой да го съди; словото, което говорих, то ще го съди в последния ден.
\par 49 Защото Аз от Себе Си не говорих; но Отец, Който Ме прати, Той Ми даде заповед, какво да кажа и що да говоря.
\par 50 И зная, че онова, което Той заповядва, е вечен живот. И тъй, това, което говоря, говоря го така, както Ми е казал Отец.

\chapter{13}

\par 1 А преди празника на пасхата, Исус, знаейки, че е настанал часът Му да премине от този свят към Отца, като беше възлюбил Своите, които бяха на света, до край ги възлюби.
\par 2 И когато беше готова вечерята, (като вече дяволът беше внушил в сърцето на Юда Симонова Искариотски да Го предаде),
\par 3 като знаеше Исус, че Отец, е предал всичко в ръцете Му, и че от Бога е излязъл и при Бога отива,
\par 4 стана от вечерята, сложи мантията Си , че Исус му казва: Купи каквото ни трябва за празника, или: Дай нещо на сиромасите.
\par 5 После наля вода в омивалника и почна да мие нозете на учениците и да ги изтрива с престилката, с която беше препасан.
\par 6 И тъй дохожда при Симона Петра. Той Му казва: Господи, Ти ли ще ми умиеш нозете?
\par 7 Исус в отговор му рече: Това, което Аз правя, ти сега не знаеш, но отпосле ще разбереш.
\par 8 Петър Му каза: Ти няма да омиеш моите нозе до века. Исус му отговори: Ако не те омия нямаш дял с Мене.
\par 9 Симон Петър Му казва: Господи, не само нозете ми, но и ръцете и главата.
\par 10 Исус му казва: Който се е окъпал няма нужда да омие друго освен нозете си, но е цял чист и вие сте чисти, но не всички.
\par 11 Защото Той знаеше онзи, който щеше да Го предаде: затова и рече: Не всички сте чисти.
\par 12 А като оми нозете им и си взе мантията седна пак и рече им: Знаете ли какво ви сторих?
\par 13 Вие Ме наричате Учител и Господ; и добре казвате, защото съм такъв.
\par 14 И тъй, ако Аз, Господ и Учител, ви омих нозете, то и вие сте длъйни един на друг да си миете нозете.
\par 15 Защото ви дадох пример да правите и вие както и Аз направих на вас.
\par 16 Истина, истина ви казвам, слугата не е по-горен от господаря си, нито пратеникът е по-горен от онзи, който го е изпратил.
\par 17 Като знаете това, блажени сте, ако го изпълнявате.
\par 18 Не говоря за всички вас; Аз зная кои съм избрал; но това стана, за да се сбъдне писаното: "Който яде хляба Ми, той дигна своята пета против Мене".
\par 19 Отсега ви казвам това нещо преди да е станало, та когато стане да повярвате, че съм Аз това, което рекох.
\par 20 Истина, истина ви казвам, който приеме онзи, когото Аз пращам, Мене приема; и който приема Мене, приема Този, Който Ме е пратил.
\par 21 Като рече това, Исус се развълнува в духа Си, и заяви, казвайки: Истина, истина ви казвам, че един от вас ще Ме предаде.
\par 22 Учениците се спогледаха помежду си, недоумявайки за кого говори.
\par 23 А на трапезата един от учениците, когото обичаше Исус, беше се облегнал на Исусовото лоно.
\par 24 Затова Симон Петър му кимва и му казва: Кажи ни за кого говори.
\par 25 А той като се обърна така на гърдите на Исуса, каза Му: Господи, кой е?
\par 26 Исус отговори: Той е онзи, за когото ще затопя залъка и ще му го дам. И тъй, като затопи залъка, взема и го подаде на Юда Симонова Искариотски.
\par 27 И тогава подир залъка, сатана влезе в него; и така, Исус му каза: Каквото вършиш, върши го по-скоро.
\par 28 А никой от седещите на трапезата не разбра защо му рече това;
\par 29 защото някои мислеха, понеже Юда държеше касата, че Исус му казва: Купи каквото ни трябва за празника, или: Дай нещо на сиромасите.
\par 30 И тъй, като взе залъка, веднага излезе; а беше нощ.
\par 31 А когато излезе, Исус казва: Сега се прослави Човешкият Син, и Бог се прослави в Него;
\par 32 и Бог ще Го прослави в Себе Си, и скоро ще Го прослави.
\par 33 Дечица, още малко съм с вас. Ще Ме търсите, и както рекох на юдеите, така и вам казвам сега гдето отивам Аз вие не можете да дойдете.
\par 34 Нова заповед ви давам, да се любите един другиго; както Аз ви възлюбих, така и вие да се любите един другиго.
\par 35 По това ще познаят всички, че сте Мои ученици, ако имате любов помежду си.
\par 36 Симон Петър Му казва: Господи, къде отиваш? Исус отговори: Където отивам не можеш сега да дойдеш след Мене, но после ще дойдеш.
\par 37 Петър Му казва: Господи, защо да не мога да дойда след Тебе сега? Животът си ще дам за Тебе.
\par 38 Исус отговори: Животът си ли за Мене ще дадеш? Истина, истина ти казвам, петелът няма да е пропял преди да си се отрекъл три пъти от Мене.

\chapter{14}

\par 1 Да се не смущава сърцето ви; вие вярвате в Бога, вярвайте и Мене.
\par 2 В дома на Отца Ми има много обиталища; ако не беше така, Аз щях да ви кажа, защото отивам да ви приготвя място.
\par 3 И като отида и ви приготвя място, пак ще дойда и ще ви взема при Себе Си, тъй щото гдето съм Аз да бъдете и вие.
\par 4 И вие знаете за къде отивам и пътя [знаете].
\par 5 Тома Му казва: Господи, не знаем къде отиваш; а как знаем пътя?
\par 6 Исус му казва: Аз съм пътят, и истината, и животът; никой не дохожда при Отца, освен чрез Мене.
\par 7 Ако бяхте познали Мене, бихте познали и Отца Ми; отсега Го познавате и сте Го видели.
\par 8 Филип Му казва: Господи, покажи ни Отца, и достатъчно ни е.
\par 9 Исус му казва: Толкова време съм с вас и не познаваш ли Ме Филипе? Който е видял Мене, видял е Отца; как казваш ти: Покажи ми Отца?
\par 10 Не вярваш ли, че Аз съм в Отца, и че Отец е в Мене? Думите, които Аз ви казвам, не от Себе Си ги говоря; но пребъдващият в Мене Отец върши Своите дела.
\par 11 Вярвайте Ме, че Аз съм в Отца и че Отец е в Мене; или пък вярвайте Ме поради самите дела.
\par 12 Истина, истина ви казвам, който вярва в Мене, делата, които върша Аз, и той ще ги върши; защото Аз отивам при Отца.
\par 13 И каквото и да поискате в Мое име, ще го сторя, за да се прослави Отец в Сина.
\par 14 Ако поискате нещо в Мое име, това ще сторя.
\par 15 Ако Ме любите, ще пазите Моите заповеди.
\par 16 И Аз ще поискам от Отца, и Той ще ви даде друг Утешител, за да пребъдва с вас до века.
\par 17 Духът на истината, когото светът не може да приеме, защото го не вижда нито го познава. Вие го познавате, защото той пребъдва във вас, и във вас ще бъде.
\par 18 Няма да ви оставя сираци; ще дойда при вас.
\par 19 Още малко, и светът няма вече да Ме вижда, а вие Ме виждате; понеже Аз живея и вие ще живеете.
\par 20 В оня ден ще познаете, че Аз съм в Отца Си, и вие в Мене, и Аз във вас.
\par 21 Който има Моите заповеди и ги пази, той Ме люби; а който Ме люби ще бъде възлюбен от Отца Ми, и Аз ще го възлюбя, и ще явя Себе Си нему.
\par 22 Юда (не Искариотски) му казва: Господи, по коя причина ще явиш Себе Си на нас, а не на света?
\par 23 Исус в отговор му рече: Ако Ме люби някой, ще пази учението Ми; и Отец Ми ще го възлюби, и Ние ще дойдем при него и ще направим обиталище у него.
\par 24 Който не Ме люби не пази думите Ми; и учението, което слушате, не е Мое, а на Отца, Който Ме е пратил.
\par 25 Това ви изговорих докато още пребъдвам с вас.
\par 26 А Утешителят, Светият Дух, когото Отец ще изпрати в Мое име, той ще ви научи на всичко, и ще ви напомни всичко, което съм ви казал.
\par 27 Мир ви оставям; Моя мир ви давам; Аз не ви давам както светът дава. Да се не смущава сърцето ви, нито да се бои.
\par 28 Чухте как Аз ви рекох, отивам си, и пак ще дойда при вас. Ако Ме любехте, бихте се зарадвали за гдето отивам при Отца; защото Отец е по-голям от Мене.
\par 29 И сега ви казах това преди да е станало, та когато стане, да повярвате.
\par 30 Аз няма вече много да говоря с вас, защото иде князът на [този] свят. Той няма нищо в Мене;
\par 31 но това става, за да познае светът, че Аз любя Отца, и че както Ми е заповядал Отец, така правя. Станете, да си отидем оттук.

\chapter{15}

\par 1 Аз съм истинската лоза, и Отец ми е земеделецът.
\par 2 Всяка пръчка в Мене, която не дава плод, Той я отрязва; и всяка що дава плод, очистя я, за да дава повече плод.
\par 3 Вие сте вече чисти чрез учението, което ви говорих.
\par 4 Пребъдвайте в Мене, и Аз във вас. Както пръчката не може да даде плод от самосебе си, ако не остане на лозата, така и вие не можете, ако не пребъдете в Мене.
\par 5 Аз съм лозата, вие сте пръчките; който пребъдва в Мене, и Аз в него, той дава много плод; защото, отделени от Мене, не можете да сторите нищо.
\par 6 Ако някой не пребъде в Мене, той бива изхвърлен навън като пръчка, и изсъхва; и събират ги та ги хвърлят в огъня, и те изгарят.
\par 7 Ако пребъдете в Мене и думите Ми пребъдат във вас, искайте каквото и да желаете, и ще ви бъде.
\par 8 В това се прославя Отец Ми, да принасяте много плод; и така ще бъдете Мои ученици.
\par 9 Както Отец възлюби Мене, така и Аз възлюбих вас; пребъдвайте в Моята любов.
\par 10 Ако пазите Моите заповеди, ще пребъдвате в любовта Ми, както и Аз опазих заповедите на Отца Си и пребъдвам в Неговата любов.
\par 11 Това ви говорих, за да бъде Моята радост във вас, и вашата радост да стане пълна.
\par 12 Това е Моята заповед, да се любите един друг, както Аз ви възлюбих.
\par 13 Никой няма по-голяма любов от това щото да даде живота си за приятелите си.
\par 14 Вие сте Ми приятели, ако вършите онова, което ви заповядвам.
\par 15 Не ви наричам вече слуги, защото слугата не знае що върши Господарят му; а вас наричам приятели, защото ви явявам всичко що съм чул от Отца Си.
\par 16 Вие не избрахте Мене, но Аз избрах вас, и ви определих да излезете в света и да принасяте плод и плодът ви да бъде траен; та каквото и да поискате от Отца в Мое име, да ви даде.
\par 17 Това ви заповядвам да се любите един друг.
\par 18 Ако светът ви мрази, знайте, че Мене преди вас е намразил.
\par 19 Ако бяхте от света, светът щеше да люби своето; а понеже не сте от света, но Аз ви избрах от света, затова светът ви мрази.
\par 20 Помнете думата, която ви казах, слугата не е по-горен от господаря си. Ако Мене гониха, и вас ще гонят; ако са опазили Моето учение, и вашето ще пазят.
\par 21 Но всичко това ще ви сторят поради Моето име, защото не познават Онзи, Който Ме е пратил.
\par 22 Ако не бях дошъл и не бях им говорил, грях не биха имали; сега, обаче, нямат извинение за греха си.
\par 23 Който мрази Мене, мрази и Отца Ми.
\par 24 Ако не бях сторил между тях делата, които никой друг не е сторил, грях не биха имали; но сега видяха и намразиха и Мене и Отца Ми.
\par 25 Но това става и да се изпълни писаното в закона им слово, "Намразиха Ме без причина".
\par 26 А когато дойде Утешителят, когото Аз ще ви изпратя от Отца, Духът на истината, който изхожда от Отца, той ще свидетелствува за Мене.
\par 27 Но и вие свидетелствувате, защото сте били с Мене отначало.

\chapter{16}

\par 1 Това ви казах, за да се не съблазните.
\par 2 Ще ви отлъчат от синагогите; даже настава час, когато всеки, който ви убие, ще мисли, че принася служба на Бога.
\par 3 И това ще сторят защото не са познали нито Отца, нито Мене.
\par 4 Но Аз ви казах тия неща, та, кога дойде часът им, да помните, че съм ви ги казал. Отначало не ви ги казах, защото бях с вас;
\par 5 а сега отивам при Онзи, Който Ме е пратил; и никой от вас не Ме пита: Къде отиваш?
\par 6 Но понеже ви казах това, скръб изпълни сърцата ви.
\par 7 Обаче Аз ви казвам истината, за вас е по-добре да отида Аз, защото, ако не отида, Утешителят няма да дойде на вас; но ако отида, ще ви го изпратя.
\par 8 И той, когато дойде, ще обвини света за грях, за правда и за съдба;
\par 9 за грях, защото не вярват в Мене;
\par 10 за правда, защото отивам при Отца, и няма вече да Ме виждате;
\par 11 а за съдба, защото князът на тоя свят е осъден.
\par 12 Имам още много неща да ви кажа; но не можете да ги понесете сега.
\par 13 А когато дойде онзи, Духът на истината, ще ви упътва на всяка истина; защото няма да говори от себе си, но каквото чуе, това ще говори, и ще ви извести за идните неща.
\par 14 Той Мене ще прослави, защото от Моето ще взема и ще ви известява.
\par 15 Всичко, що има Отец, е Мое; затова казах, че от Моето като взема, ще ви известява.
\par 16 Още малко, и няма да Ме виждате; и пак малко и ще Ме видите.
\par 17 Затова някои от учениците Му продумаха помежду си: Какво е това що ни казва: Още малко, и няма да Ме виждате; и пак малко и ще ме видите; и това, защото отивам при Отца.
\par 18 И рекоха: Какво е това, което казва: Още малко? Не знаем какво иска да каже.
\par 19 Исус, като разбра, че желаят да Го питат, рече им: За това ли се запитвате помежду си дето рекох: Още малко и няма да Ме виждате; и пак малко и ще Ме видите?
\par 20 Истина, истина ви казвам, че вие ще заплачете и ще заридаете, а светът ще се радва; вие ще скърбите, но скръбта ви ще се обърне в радост.
\par 21 Жена, когато ражда, е в скръб, защото е дошъл часът й; а кога роди детенцето, не помни вече тъгата си поради радостта, че се е родил човек на света.
\par 22 И вие, прочее, сега сте на скръб; но Аз пак ще ви видя, и сърцето ви ще се зарадва, и радостта ви никой няма да ви отнеме.
\par 23 И в онзи ден няма да Ме питате за нищо. Истина, истина ви казвам, ако поискате нещо от Отца, Той ще ви го даде в Мое име.
\par 24 До сега нищо не сте искали в Мое име; искайте и ще получите, за да бъде радостта ви пълна.
\par 25 Това съм ви говорил с притчи. Настава час, когато няма вече да ви говоря с притчи, а ясно ще ви известя за Отца.
\par 26 В оня ден ще искате в Мое име; и не ви казвам, че Аз ще поискам от Отца за вас;
\par 27 защото сам Отец ви люби, понеже вие възлюбихте Мене и повярвахте, че Аз от Отца излязох.
\par 28 Излязох от Отца и дойдох на света; и пак напускам света и отивам при Отца.
\par 29 Казват учениците Му: Ето, сега ясно говориш, и никаква притча не казваш.
\par 30 Сега сме уверени, че Ти всичко знаеш, и няма нужда да Те пита някой, за да му отговаряш. По това вярваме, че си излязъл от Бога.
\par 31 Исус им отговори: Сега ли вярвате?
\par 32 Ето, настава час, даже дошъл е, да се разпръснете всеки при своите си, и Мене да оставите сам; обаче не съм сам, защото Отец е с Мене.
\par 33 Това ви казах, за да имате в Мене мир. В света имате скръб; но дерзайте Аз победих света.

\chapter{17}

\par 1 Това като изговори Исус, дигна очите си към небето, и рече: Отче настана часът; прослави Сина Си, за да Те прослави и Сина Ти,
\par 2 според както си Му дал власт над всяка твар да даде вечен живот на всички, които си Му дал.
\par 3 А това е вечен живот, да познаят Тебе, единия истинен Бог, и Исуса Христа, Когото си изпратил.
\par 4 Аз те прославих на земята, като свърших делото, което Ти Ми даде да върша.
\par 5 И сега прослави Ме, Отче, у Себе Си със славата, която имах у Тебе преди създанието на света.
\par 6 Изявих името Ти на човеците, които Ми даде от света. Те бяха Твои, и Ти ги даде на Мене, и те опазиха Твоето слово.
\par 7 Сега знаят, че всичко, което си Ми дал е от Тебе;
\par 8 защото думите, които Ми даде Ти, Аз ги предадох на тях, и те ги приеха; и наистина знаят че, от Тебе излязох, и вярват, че Ти си Ме пратил.
\par 9 Аз за тях се моля; не се моля за света, а за тия, които си Ми дал, защото са Твои.
\par 10 И всичко Мое е Твое, и Твоето Мое, и Аз се прославям в тях.
\par 11 Не съм вече на света, а тия са на света, и Аз ида при Тебе, Отче свети, опази в името Си тия, които си Ми дал, за да бъдат едно, както сме и Ние.
\par 12 До като бях с тях, Аз пазех в Твоето име тия, които Ми даде; опазих ги, и нито един от тях не погина, освен сина на погибелта, за да се изпълни писанието.
\par 13 А сега ида при Тебе; но догдето съм още на света казвам това, за да имат Моята радост пълна в себе си.
\par 14 Аз им предадох Твоето слово; и светът ги намрази, защото те не са от света, както и Аз не съм от него.
\par 15 Не се моля да ги вземеш от света, но да ги пазиш от лукавия.
\par 16 Те не са от света както и Аз не съм от света.
\par 17 Освети ги чрез истината; Твоето слово е истина.
\par 18 Както Ти прати Мене в света, така и Аз пратих тях в света;
\par 19 и заради тях Аз освещавам Себе Си, за да бъдат и те осветени чрез истината.
\par 20 И не само за тях се моля, но и за ония, които биха повярвали в Мене чрез тяхното учение,
\par 21 да бъдат всички едно; както Ти, Отче, си в Мене и Аз в Тебе, тъй и те да бъдат в Нас, за да повярва светът, че Ти си Ме пратил.
\par 22 И славата, която Ти Ми даде, Аз я дадох на тях; за да бъдат едно, както и Ние сме едно;
\par 23 Аз в тях, и Ти в мене, за да бъдат съвършени в единство; за да познае светът, че Ти си Ме пратил, и си възлюбил тях както си възлюбил и Мене.
\par 24 Отче, желая гдето съм Аз, да бъдат с Мене и тия, които си Ми дал, за да гледат Моята слава, която си Ми дал; защото си Ме възлюбил преди създанието на света.
\par 25 Отче праведни, светът не Те е познал, но Аз Те познах; и тия познаха, че Ти си Ме пратил.
\par 26 И явих им Твоето име, и ще явя, та любовта, с която си Ме възлюбил, да бъде в тях, и Аз в тях.

\chapter{18}

\par 1 Като изговори това, Исус излезе с учениците Си отвъд потока Кедрон, гдето имаше градина, в която влезе Той и учениците Му.
\par 2 А и Юда, който Го предаваше, знаеше това място; защото Исус често се събираше там с учениците Си.
\par 3 И тъй, Юда като взе една чета войници и служители от главните свещеници и фарисеите, дойде там с фенери, факли и оръжия.
\par 4 А Исус като знаеше всичко, което щеше да Го сполети, излезе и им рече: Кого търсите?
\par 5 Отговориха Му: Исуса Назарянина. Исус им каза: Аз съм. С тях стоеше и Юда, който Го предаваше.
\par 6 И когато им каза: Аз съм, те се дръпнаха назад и паднаха на земята.
\par 7 Пак ги попита: Кого търсите? А те рекоха: Исуса Назарянина.
\par 8 Исус отговори: Рекох ви, че съм Аз; прочее, ако Мене търсите оставете тия да си отидат;
\par 9 (за да се изпълни думата казана от Него: От тия, които си Ми дал, ни един не изгубих).
\par 10 А Симон Петър, като имаше нож, измъкна го, удари слугата на първосвещеника, и му отсече дясното ухо; а името на слугата беше Малх.
\par 11 Тогава Исус рече на Петра: Тури ножа в ножницата. Чашата, която Ми даде Отец, да я не пия ли?
\par 12 И тъй, четата, хилядникът и юдейските служители хванаха Исуса и Го вързаха.
\par 13 И заведоха Го първо при Анна; защото той беше тъст на Каиафа, който беше първосвещеник през тая година.
\par 14 А Каиафа, беше онзи, който беше съветвал юдеите, че е по-добре един човек да загине за людете.
\par 15 И подир Исуса вървяха Симон Петър и един друг ученик; и този ученик, като беше познат на първосвещеника, влезе с Исуса в двора на първосвещеника.
\par 16 А Петър стоеше вън до вратата; и тъй другият ученик, който беше познат на първосвещеника, излезе та каза на вратарката и въведе Петра.
\par 17 И слугинята вратарка казва на Петра: И ти ли си от учениците на Този човек? Той казва: Не съм.
\par 18 А слугите и служителите бяха наклали огън, защото беше студено, и стояха та се грееха; а и Петър стоеше с тях и се грееше.
\par 19 А първосвещеникът попита Исуса за учениците Му и за учението Му.
\par 20 Исус му отговори: Аз говорих явно на света, винаги поучавах в синагогите и в храма, гдето всички юдеи се събират, и нищо не съм говорил скришно.
\par 21 Защо питаш Мене? питай ония, които са Ме слушали, какво съм им говорил; ето, те знаят що съм казвал.
\par 22 Когато рече това, един от служителите, който стоеше наблизо, удари плесница на Исуса и рече: Така ли отговаряш на първосвещеника?
\par 23 Исус му отговори: Ако съм продумал нещо зло, покажи злото; но ако добро, защо ме биеш?
\par 24 Анна, прочее, Го прати вързан при първосвещеника Каиафа.
\par 25 А Симон Петър стоеше и се грееше; и рекоха му: Не си ли и ти от Неговите ученици? Той отрече казвайки: Не съм.
\par 26 Един от слугите на първосвещеника, сродник на онзи, комуто Петър отсече ухото, казва: Нали те видях аз в градината с Него?
\par 27 И Петър пак се отрече; и на часа изпя петел.
\par 28 Тогава поведоха Исуса от Каиафа в преторията; а беше рано. Но сами те не влязоха в преторията, за да се не осквернят, та да могат да ядат пасхата.
\par 29 Затова Пилат излезе при тях и каза: В какво обвинявате Тоя човек?
\par 30 В отговор му рекоха: Ако не беше Той злодеец, не щяхме да Го предадем на тебе.
\par 31 А Пилат им рече: Вземете Го вие и Го съдете според вашия закон. Юдеите му рекоха: Нам не е позволено да умъртвим никого,
\par 32 (за да се изпълни думата, която рече Исус, като означаваше с каква смърт щеше да умре).
\par 33 И тъй, Пилат пак влезе в преторията, повика Исуса и Му каза: Ти юдейски Цар ли си?
\par 34 Исус отговори: От себе си ли казваш това, или други са ти говорили за Мене?
\par 35 Пилат отговори: Че аз юдеин ли съм? Твоят народ и главните свещеници Те предадоха на мене. Какво си сторил?
\par 36 Исус отговори: Моето царство не е от този свят; ако беше царството Ми от този свят, служителите Ми щяха да се борят да не бъда предаден на юдеите. А сега царството Ми не е оттук.
\par 37 Затова Пилат Му каза: Тогава, Ти цар ли си? Исус отговори: Ти право казваш, защото Аз съм цар. Аз за това се родих, и за това дойдох на света, да свидетелствувам за истината. Всеки, който е от истината, слуша Моя глас.
\par 38 Пилат Му каза: Що е истина? И като рече това, пак излезе при юдеите и каза им: Аз не намирам никаква вина в него.
\par 39 А у вас има обичай да ви пущам по един на пасхата; желаете ли, прочее, да ви пусна юдейския цар?
\par 40 Тогава те пак закрещяха, казвайки: Не Тоя, но Варава. А Варава беше разбойник.

\chapter{19}

\par 1 Тогава Пилат взе Исуса и го би.
\par 2 И войниците сплетоха венец от тръни, наложиха го на главата Му, и като Му облякоха морава дреха,
\par 3 приближиха се при Него и казваха: Здравей, царю юдейски! и удряха Му плесници.
\par 4 Тогава Пилат пак излезе вън и каза им: Ето, извеждам ви Го вън за да познаете, че не намирам никаква вина в Него.
\par 5 Исус, прочее, излезе вън носещ трънения венец и моравата дреха. Пилат им казва: Ето човекът!
\par 6 А като Го видяха главните свещеници и служители извикаха, казвайки: Разпни Го! разпни Го! Пилат им каза: Вземете Го вие и разпнете Го; защото аз не намирам вина в Него.
\par 7 Юдеите му отговориха: Ние си имаме закон и по тоя закон Той трябва да умре, защото направи Себе Си Божий Син.
\par 8 А Пилат като чу тая дума още повече се уплаши.
\par 9 И пак влезе в преторията и каза на Исуса: Ти от къде си? А Исус не му даде отговор.
\par 10 Затова Пилат Му казва: На мене ли не говориш? Не знаеш ли че имам власт да Те пусна, и имам власт да Те разпна?
\par 11 Исус му отговори: Ти не би имал никаква власт над Мене, ако не бе ти дадено от горе; затова, по-голям грях има оня, който Ме предаде на тебе.
\par 12 Поради това Пилат търсеше начин да Го пусне; юдеите, обаче, викаха, казвайки: Ако пуснеш Тогова, не си Кесарев приятел; всеки, който прави себе си цар, е противник на Кесаря.
\par 13 А Пилат, като чу тия думи, изведе Исуса вън и седна на съдийския стол, на мястото наречено Каменно настлание, а по еврейски, Гавата.
\par 14 Беше денят на приготовление за пасхата, около шестия час; и той казва на юдеите: Ето вашият цар!
\par 15 А те извикаха: Махни Го! махни! разпни Го! Пилат им казва: Вашия цар ли да разпна? Главните свещеници отговориха: Нямаме друг цар освен Кесаря.
\par 16 Затова той им Го предаде да бъде разпнат.
\par 17 И така, взеха Исуса; и Той сам носейки кръста Си излезе; и дойде на мястото наречено Лобно, което по еврейски се казва Голгота,
\par 18 гдето Го разпнаха, и с Него други двама, от едната и от другата страна, а Исус посред.
\par 19 А Пилат написа и надпис, който постави над Него на кръста. А писаното бе: Исус Назарянин, юдейският цар.
\par 20 Тоя надпис прочетоха мнозина от юдеите, защото мястото гдето разпнаха Исуса беше близо до града, и написаното бе на еврейски, на латински и на гръцки.
\par 21 А юдейските главни свещеници казаха на Пилата: Недей писа: юдейски цар, но - Самозваният юдейски цар.
\par 22 Пилат отговори: Каквото писах, писах.
\par 23 А войниците като разпнаха Исуса, взеха дрехите Му и ги разделиха на четири дяла, на всеки войник по един дял; взеха и дрехата. А дрехата не беше шита, а изтъкана цяла от горе до долу;
\par 24 затова те рекоха помежду си: Да не я раздираме, а да хвърлим жребие за нея чия да бъде; за да се изпълни написаното, което казва: - "Разделиха си дрехите Ми, И за облеклото Ми хвърлиха жребие". Войниците, прочее, сториха това.
\par 25 А при кръста на Исуса стояха майка Му, и сестрата на майка Му, Мария Клеопова и Мария Магдалина.
\par 26 А Исус, като видя майка Си и ученика, когото обичаше, който стоеше близо, каза на майка Си: Жено, ето син ти!
\par 27 После каза на ученика: Ето майка ти! И от онзи час ученикът я прибра у дома си.
\par 28 След това, Исус, като знаеше, че всичко вече е свършено, за да се сбъдне писанието рече: Жаден съм.
\par 29 А като беше сложен там съд пълен с оцет, натъкнаха на исопова тръст една гъба натопена в оцет, и я поднесоха до устата Му.
\par 30 А Исус като прие оцета рече: Свърши се, и наведе глава, и предаде дух.
\par 31 И понеже беше Приготвителният ден, то, за да не останат телата на кръста в съботата, (защото оная събота беше голям ден), юдеите помолиха Пилата да им се пребият пищялите, и да ги дигнат оттам.
\par 32 Затова дойдоха войниците и пребиха пищялите на единия и на другия, които бяха разпнати с Исуса.
\par 33 Но когато дойдоха при Исуса и Го видяха вече умрял не Му пребиха пищялите.
\par 34 Обаче, един от войниците прободе с копие ребрата Му; и веднага изтече кръв и вода.
\par 35 И тоя, който видя, свидетелствува за това, и неговото свидетелство е вярно; и той знае, че говори истината, за да повярвате и вие.
\par 36 Защото това стана, за да се изпълни написаното: "Кост Негова няма да се строши";
\par 37 и пак на друго място писанието казва: "Ще погледнат на Него, Когото прободоха".
\par 38 След това Иосиф от Ариматея, който беше Исусов ученик, но таен, поради страха от юдеите, помоли Пилата да му позволи да вземе Исусовото тяло; и Пилат позволи. Той, прочее, дойде та вдигна тялото Му.
\par 39 Дойде и Никодим, който бе дохождал изпърво при Него нощем, и донесе около сто литри смес от смирна и алой.
\par 40 И тъй, взеха Исусовото тяло и Го обвиха в плащаница с ароматите, според юдейския обичай на погребване.
\par 41 А на мястото, гдето бе разпнат, имаше градина, и в градината нов гроб, в който още никой не бе полаган.
\par 42 Там, прочее, положиха Исуса поради юдейския Приготвителен ден; защото гробът беше наблизо.

\chapter{20}

\par 1 В първия ден на седмицата Мария Магдалина дохожда на гроба сутринта, като беше още тъмно, и вижда, че камъкът е дигнат от гроба.
\par 2 Затова се затича и дохожда при Симона Петра и при другия ученик, когото обичаше Исус, и им казва: Дигнали Господа от гроба, и не знаем где са Го положили.
\par 3 И тъй, Петър и другият ученик излязоха и отиваха на гроба;
\par 4 и двамата тичаха заедно, но другият ученик надвари Петра и стигна пръв на гроба.
\par 5 И като надникна, видя плащаниците сложени, но не влезе вътре.
\par 6 След него дойде Симон Петър и влезе в гроба; видя плащаниците сложени,
\par 7 и кърпата, която беше на главата Му, не сложена с плащаниците, а свита на отделно място.
\par 8 Тогава влезе другият ученик, който пръв стигна на гроба; и видя и повярва.
\par 9 Защото още не бяха разбрали писанието, че Той трябваше да възкръсне от мъртвите.
\par 10 И тъй, учениците се върнаха пак у тях си.
\par 11 А Мария стоеше до гроба отвън и плачеше; и така, като плачеше надникна в гроба,
\par 12 и вижда два ангела в бели дрехи седнали там гдето бе лежало Исусовото тяло, един откъм главата, и един откъм нозете.
\par 13 И те й казват: Жено, защо плачеш? Казва им: Защото дигнали Господа мой, и не знам где са Го положили.
\par 14 Като рече това, тя се обърна назад и видя Исуса, че стои, но не позна, че беше Исус.
\par 15 Казва й Исус: Жено, защо плачеш? кого търсиш? Тя, като мислеше, че е градинарят, казва Му: Господине, ако ти си го изнесъл, кажи ми где си Го положил, и аз ще Го дигна.
\par 16 Казва й Исус: Марийо! Тя се обърна и Му рече на еврейски: Равуни! което значи, Учителю!
\par 17 Казва й Исус: Не се допирай до Мене, защото още не съм се възнесъл при Отца; но иди при братята Ми и кажи им: Възнасям се при Моя Отец и вашия Отец, при Моя Бог и вашия Бог.
\par 18 Мария Магдалина дохожда и известява на учениците, че видяла Господа, и че Той й казал това.
\par 19 А вечерта на същия ден, първия на седмицата, когато вратата на стаята, гдето бяха учениците, беше заключена поради страха от юдеите, Исус дойде, застана посред, и каза им: Мир вам!
\par 20 И като рече това, показа им ръцете и ребрата Си. И зарадваха се учениците като видяха Господа.
\par 21 И Исус пак им рече: Мир вам! Както Отец изпрати Мене, така и Аз изпращам вас.
\par 22 И като рече това, духна върху тях и им каза: Приемете Светия Дух.
\par 23 На които простите греховете, простени им са, на които задържите задържани са.
\par 24 А Тома, един от дванадесетте наречен Близнак, не беше с тях, когато дохожда Исус.
\par 25 Затова другите ученици му казаха: Видяхме Господа. А той им рече: Ако не видя на ръцете Му раните от гвоздеите, и не туря ръката си в ребрата Му, няма да повярвам.
\par 26 И подир осем дни учениците Му пак бяха вътре, и Тома с тях. Исус дохожда, като беше заключена вратата, застана насред, и рече: Мир вам!
\par 27 Тогава каза на Тома: Дай си пръста тука и виж ръцете Ми, и дай ръката си и тури я в ребрата Ми; и не бъди невярващ, а вярващ.
\par 28 Тома в отговор Му рече: Господ мой и Бог мой!
\par 29 Исус му казва: Понеже Ме видя, [Томо], ти повярва, блажени ония, които, без да видят, са повярвали.
\par 30 А Исус извърши пред учениците още много други знамения, които не са вписани в тая книга.
\par 31 А тия са написани за да повярвате, че Исус е Христос, Божият Син, и, като вярвате, да имате живот в Неговото име.

\chapter{21}

\par 1 Подир това Исус пак се яви на учениците на Тивериадското езеро; и ето как им се яви:
\par 2 там бяха заедно Симон Петър, Тома наречен Близнак, Натанаил от Кана галилейска, Заведеевите синове, и други двама от учениците Му.
\par 3 Симон Петър им казва: Отивам да ловя риба. Казват му: Ще дойдем и ние с тебе. Излязоха и се качиха на ладията; и през оная нощ не уловиха нищо.
\par 4 А като се разсъмваше вече Исус застана на брега; учениците, обаче, не познаха, че е Исус.
\par 5 Исус им казва: Момчета, имате ли нещо за ядене? Отговориха Му: Нямаме.
\par 6 А Той им рече: Хвърлете мрежата отдясно на ладията и ще намерите. Те, прочее, хвърлиха; и вече не можаха да я извлекат поради многото риби.
\par 7 Тогава оня ученик, когото обичаше Исус, казва на Петра: Господ е. А Симон Петър, като чу, че бил Господ, препаса си връхната дреха (защото беше гол) и се хвърли в езерото.
\par 8 А другите ученици дойдоха в ладията, (защото не бяха далеч от сушата, на около двеста лакти), и влачеха мрежата с рибата.
\par 9 И като излязоха на сушата, видяха жарава положена, и риба турена на нея и хляб.
\par 10 Исус им казва: Донесете от рибите, които сега уловихте.
\par 11 Затова Симон Петър се качи на ладията та извлече мрежата на сушата, пълна с едри риби на брой сто и петдесет и три; и при все, че бяха толкова, мрежата не се съдра.
\par 12 Исус им казва: Дойдете да закусите. (И никой от учениците не смееше да Го попита: Ти Кой си? понеже знаеха, че е Господ).
\par 13 Дохожда Исус, взема хляба, и им дава, също и рибата.
\par 14 Това беше вече трети път как Исус се яви на учениците след като възкръсна от мъртвите.
\par 15 А като позакусиха, Исус казва на Симона Петра: Симоне Ионов, любиш ли Ме повече отколкото Ме любят тия? Казва Му: Да, Господи, Ти знаеш, че Те обичам. Той му казва: Паси агънцата Ми.
\par 16 Пак му каза втори път: Симоне Ионов, любиш ли Ме? Казва Му: Да, Господи, Ти знаеш, че Те обичам. Той Му казва: Паси овцете Ми.
\par 17 Казва му трети път: Симоне Ионов, Обичаш ли Ме? Петър се наскърби за гдето трети път му рече: Обичаш ли Ме? и Му рече: Господи, Ти всичко знаеш, Ти знаеш, че Те обичам. Исус му казва: Паси овцете Ми.
\par 18 Истина, истина ти казвам, когато беше по-млад, ти сам се опасваше и ходеше където си щеше; но когато остарееш ще простреш ръцете си, и друг ще те опасва, и ще те води където не щеш.
\par 19 А това рече като означаваше с каква смърт Петър щеше да прослави Бога. И като рече това, казва му: Върви след Мене.
\par 20 Петър обръщайки се вижда, че иде подире му ученикът, когото обичаше Исус, този, който на вечерята се обърна на гърдите Му и каза: Господи, кой е този, който ще Те предаде?
\par 21 Него, прочее, като видя, Петър казва на Исуса, Господи, а на този какво ще стане?
\par 22 Исус му каза: Ако искам да остане той докле дойда, тебе що ти е? Ти върви след Мене.
\par 23 И така, разнесе се между братята тази дума, че този ученик нямаше да умре. Исус, обаче, не му рече, че няма да умре, но: Ако искам да остане той докле дойда, тебе що ти е?
\par 24 Този е ученикът, който свидетелствува за тия неща, който и написа тия неща; и знаем, че неговото свидетелство е истинно.
\par 25 Има още и много други дела, които извърши Исус; но ако се напишеха едно по едно, струва ми се, че цял свят не щеше да побере написаните книги. [Амин].

\end{document}