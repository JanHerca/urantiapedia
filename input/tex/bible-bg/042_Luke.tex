\begin{document}

\title{Luke}


\chapter{1}

\par 1 Понеже мнозина предприеха да съчинят повест за съвършено потвърдените между нас събития,
\par 2 както ни ги предадоха ония, които от началото са били очевидци и служители на евангелското слово,
\par 3 видя се добре и на мене, който изследвах подробно всичко от началото, да ти пиша наред за това, почтени Теофиле,
\par 4 за да познаеш достоверността на това, в което си бил поучаван.
\par 5 В дните на Юдейския цар Ирод имаше един свещеник от Авиевия отред, на име Захария; и жена му беше от Аароновите потомци, и се наричаше Елисавета.
\par 6 Те и двамата бяха праведни пред Бога, като ходеха непорочно във всичките Господни заповеди и наредби.
\par 7 И нямаха чадо, понеже Елисавета беше неплодна, а и двамата бяха в напреднала възраст.
\par 8 И като свещенодействуваше той пред Бога по реда на своя отред,
\par 9 по обичая на свещеничеството, нему се падна по жребие да влезе в Господния храм и да покади.
\par 10 И в часа на каденето цялото множество на людете се молеше отвън.
\par 11 И яви му се ангел от Господа, стоящ отдясно на кадилния олтар.
\par 12 И Захарий като го видя, смути се, и страх го обзе.
\par 13 Но ангелът му рече: Не бой се, Захари; защото твоята молитва е чута, и жена ти Елисавета ще ти роди син, когото ще наречеш Иоан.
\par 14 Той ще ти бъде за радост и веселие; и мнозина ще се зарадват за неговото рождение.
\par 15 Защото ще бъде велик пред Господа; вино и спиртно питие няма да пие; и ще се изпълни със Святия Дух още от зачатието си.
\par 16 И ще обърне мнозина от израилтяните към Господа техния Бог.
\par 17 Той ще предиде пред лицето Му в духа и силата на Илия, за да обърна сърцата на бащите към чадата, и непокорните към мъдростта на праведните, да приготви за Господа благоразположен народ.
\par 18 А Захария рече на ангела: По какво ще узная това? Защото аз съм стар, и жена ми е в напреднала възраст.
\par 19 Ангелът в отговор му каза: Аз съм Гавриил, който стои пред Бога; и съм изпратен да ти говоря и да ти благовестя това.
\par 20 И, ето, ще млъкнеш, и не ще можеш да говориш до деня, когато ще се сбъдне това, защото не повярва думите ми, които ще се сбъднат своевременно.
\par 21 И людете чакаха Захария, и се чудеха, че се бави в храма.
\par 22 А когато излезе, не можеше да им продума; и те разбраха, че е видял видение в храма, защото той им правеше знакове и оставаше ням.
\par 23 И като се навършиха дните на службата му, той отиде у дома си.
\par 24 А след тия дни жена му Елисавета зачна; и криеше се пет месеца, като казваше:
\par 25 Така ми стори Господ в дните, когато погледна милостиво, за да отнеме от човеците причината да ме корят.
\par 26 А в шестия месец ангел Гавриил бе изпратен от Бога в галилейския град, наречен Назарет,
\par 27 при една девица, сгодена за мъж на име Йосиф от Давидовия дом; а името на девицата бе Мария.
\par 28 И като дойде ангелът при нея, рече: Здравей благодатна! Господ е с тебе, [благословена си ти между жените].
\par 29 А тя много се смути от думата му и в недоумение беше, какъв ли ще бъде тоя поздрав.
\par 30 И ангелът - рече: Не бой се, Марио, защото си придобила Божието благоволение.
\par 31 И ето, ще зачнеш в утробата си и ще родиш син, Когото ще наречеш Исус.
\par 32 Той ще бъде велик, и ще се нарече Син на Всевишния; и Господ Бог ще Му даде престола на баща Му Давида.
\par 33 Ще царува над Якововия дом до века; и царството Му не ще има край.
\par 34 А Мария рече на ангела: Как ще бъде това, тъй като мъж не познавам?
\par 35 И ангелът в отговор - рече: Святият дух ще дойде върху ти и силата но Всевишния ще те осени; за туй, и святото Онова, Което ще се роди от тебе, ще се нарече Божий Син.
\par 36 И, ето, твоята сродница Елисавета, и тя в старините си е зачнала син; и това е шестия месец за нея, която се казваше неплодна.
\par 37 Защото за Бога няма невъзможно нещо.
\par 38 И Мария рече: Ето Господната слугиня; нека ми бъде според както си казал. И ангелът си отиде от нея.
\par 39 През тия дни Мария стана и отиде бързо към хълмистата страна, в един Юдов град,
\par 40 и влезе в Захариевата къща и поздрави Елисавета.
\par 41 И щом чу Елисавета Марииния поздрав, младенецът заигра в утробата -; и Елисавета се изпълни със Святия Дух,
\par 42 и като извика със силен глас рече: Благословена си ти между жените, и благословен е плодът на твоята утроба!
\par 43 И от какво ми е тая чест, да дойде при мене майката на моя Господ?
\par 44 Защото, ето, щом стигна гласът на твоя поздрав до ушите ми, младенецът заигра радостно в утробата ми.
\par 45 И блажена е тая, която е повярвала, че ще се сбъдне казаното - от Господа.
\par 46 И Мария каза: Величае душата ми Господа,
\par 47 и зарадва се духът ми в Господа, Спасителя мой,
\par 48 Защото погледна милостиво на ниското положение на слугинята си; и ето, отсега ще ме ублажават всичките родове.
\par 49 Защото Силният извърши за мене велики дела; и свято е Неговото име.
\par 50 И, през родове и родове, Неговата милост е върху ония, които Му се боят.
\par 51 Извърши силни дела със Своята мишца; разпръсна ония, които са горделиви в мислите на сърцето си.
\par 52 Свали владетели от престолите им. И въздигна смирени.
\par 53 Гладните напълни с блага. А богатите отпрати празни.
\par 54 Помогна на слугата Си Израиля, за да помни да покаже милост.
\par 55 (Както бе говорил на бащите ни). Към Авраама и към неговото потомство до века.
\par 56 А Мария, като преседя с нея около три месеца, върна се у дома си.
\par 57 А на Елисавета се навърши времето да роди; и роди син.
\par 58 И като чуха съседите и роднините -, че Господ показал към нея велика милост, радваха се с нея.
\par 59 И на осмия ден дойдоха да обрежат детенцето; и щяха да го нарекат Захария, по бащиното му име.
\par 60 Но майка му в отговор каза: Не, но ще се нарече Иоан.
\par 61 И рекоха -: Няма никой в рода ти, който се нарича с това име.
\par 62 И запитаха баща му със знакове, как би искал той да го нарекат.
\par 63 А той поиска дъсчица и написа тия думи: Иоан е името му. И те всички се почудиха.
\par 64 И начаса му се отвориха устата, и езикът му се развърза, и той проговори и благославяше Бога.
\par 65 И страх обзе всичките им съседи; и за всичко това се говореше по цялата хълмиста страна на Юдея.
\par 66 И всички, които чуха, пазеха това в сърцата си, казвайки: Какво ли ще бъде това детенце? Защото Господната ръка беше с него.
\par 67 Тогава баща му Захария се изпълни със Святия Дух и пророкува, казвайки:
\par 68 Благословен Господ, израилевия Бог, защото посети Своите люде и извърши изкупление за тях.
\par 69 И въздигна рог на спасение за нас в дома на слугата Си Давида,
\par 70 (Както е говорил чрез устата на святите Си от века пророци).
\par 71 Избавление от неприятелите ни и от ръката на всички, които ни мразят,
\par 72 за да покаже милост към бащите ни и да спомни святия Свой завет,
\par 73 клетвата, с която се закле на баща ни Авраама.
\par 74 Да даде нам, бидейки освободени от ръката на неприятелите ни. Да му служим без страх,
\par 75 в святост и правда пред Него, през всичките си дни.
\par 76 Да! И ти, детенце, пророк на Всевишния ще се наречеш; защото ще вървиш пред лицето на Господа да приготвиш пътищата за Него.
\par 77 3а да дадеш на Неговите люде да познаят спасение чрез прощаване греховете им,
\par 78 поради милосърдието на нашия Бог, с което ще ни посети зора отгоре,
\par 79 за да осияе седящите в тъмнина и в мрачна сянка; така щото да отправи нозете ни в пътя на мира.
\par 80 А детенцето растеше и крепнеше по дух; и беше в пустините до деня, когато се яви на Израиля.

\chapter{2}

\par 1 А в ония дни излезе заповед от Кесаря Августа да се запише цялата вселена.
\par 2 Това беше първото записване, откакто Квириний управляваше Сирия.
\par 3 И всички отиваха да се записват, всеки в своя град.
\par 4 И тъй, отиде и Йосиф от Галилея, от града Назарет, в Юдея, в Давидовия град, който се нарича Витлеем, (понеже той беше от дома и рода Давидов),
\par 5 за да се запише с Мария, която беше сгодена за него и беше непразна.
\par 6 И когато бяха там, навършиха се дните - да роди.
\par 7 И роди първородния си Син, пови Го, и положи Го в ясли, защото нямаше място за тях в страноприемницата.
\par 8 И на същото място имаше овчари, които живееха в полето, и пазеха нощна стража около стадото си.
\par 9 И ангел от Господа застана пред тях, и Господната слава ги осия; и те се уплашиха много.
\par 10 Но ангелът им рече: Не бойте се, защото, ето, благовестявам ви голяма радост, която ще бъде за всичките люде.
\par 11 Защото днес ви се роди в Давидовия град Спасител, Който е Христос Господ.
\par 12 И това ще ви бъде знакът: ще намерите Младенец, повит и лежащ в ясли.
\par 13 И внезапно, заедно с ангела, се намери множество небесно воинство, което хвалеше Бога, казвайки:
\par 14 Слава на Бога във висините. И на земята мир между човеците, в които е Неговото благоволение.
\par 15 Щом ангелите си отидоха от тях на небето, овчарите си рекоха един на друг: Нека отидем тогава във Витлеем, и нека видим това, що е станало, което Господ ни изяви.
\par 16 И дойдоха бързо и намериха Мария и Йосифа, и Младенеца лежащ в яслите.
\par 17 И като видяха, разказаха каквото им беше известено за това детенце.
\par 18 И всички които чуха, се зачудиха за това, което овчарите им казаха.
\par 19 А Мария спазваше всички тия думи и размишляваше за тях в сърцето си.
\par 20 И овчарите се върнаха, славещи и хвалещи Бога за всичко, що бяха чули и видели, според както им беше казано.
\par 21 И като се навършиха осем дни и трябваше да обрежат Детенцето, дадоха Му името Исус, както беше наречено от ангела преди да е било зачнато в утробата.
\par 22 Като се навършиха и дните за очистването им, според Мойсеевия закон, занесоха Го в Ерусалим за да Го представят пред Господа,
\par 23 (както е писано в Господния закон, че всеки първороден младенец от мъжки пол ще бъде свят на Господа),
\par 24 и да принесат жертва според казаното в Господния закон две гургулици и две гълъбчета.
\par 25 И ето, имаше в Ерусалим един човек на име Симеон; и тоя човек бе праведен и благочестив, и чакаше утехата на Израиля; и Святият Дух беше в него.
\par 26 Нему бе открито от Святия Дух, че няма да види смърт, докле не види Христа Господен.
\par 27 И по внушението на Духа той дойде в храма; и когато родителите внесоха детенцето Исус, за да сторят за Него по обичая на закона,
\par 28 той Го взе на ръцете си и благослови Бога, като каза:
\par 29 Сега, Владико, отпущаш слугата Си в мир, според думата си;
\par 30 защото видяха очите ми спасението,
\par 31 което си приготвил пред всички люде;
\par 32 светлина да просвещава народите. И слава на Твоите люде Израил.
\par 33 А баща Му и майка Му се чудеха на това, което се говореше за Него.
\par 34 И Симеон го благослови, и рече на майка Му Мария: Ето, това детенце е поставено за падане и за ставане на мнозина в Израиля, и за белег, против който ще се говори.
\par 35 Да! И на сама тебе меч ще прониже душата ти, за да се открият помислите на много сърца.
\par 36 Имаше и някоя си Анна, Фануилова дъщеря, от Асировото племе; (тя беше в много напреднала възраст, като бе живяла с мъжа си седем години от девството си.
\par 37 И беше вдовица за цели осемдесет и четири години), която не се отделяше от храма, дето нощем и денем служеше Богу в пост и молитва.
\par 38 И тя, като се приближи, в същия час, благодареше Богу, и говореше за Него на всички, които ожидаха изкуплението на Ерусалим.
\par 39 И като свършиха всичко, що беше според Господния закон, върнаха се в Галилея, в града си Назарет.
\par 40 А Детенцето растеше, крепнеше, и се изпълваше с мъдрост; и Божията благодат бе на Него.
\par 41 И родителите Му ходеха всяка година в Ерусалим за празника на Пасхата.
\par 42 И когато Той беше на дванадесет години, като отидоха по обичая на празника,
\par 43 и като изкараха дните и се връщаха, Момчето Исус остана в Ерусалим, без да знаят родителите Му.
\par 44 А те, понеже мислеха, че Той е с дружината, изминаха един ден път, като Го търсеха между роднините и познатите си.
\par 45 И като не Го намериха, върнаха се в Ерусалим и Го търсеха.
\par 46 И след три дни Го намериха в храма, седнал между законоучителите, че ги слушаше и ги запитваше.
\par 47 И всички, които Го слушаха, се учудваха на разума Му и на отговорите Му.
\par 48 И като Го видяха, смаяха се; и рече Му майка Му: Синко, защо постъпи тъй с нас? Ето, баща Ти и аз, наскърбени, Те търсехме.
\par 49 А Той им рече: Защо да Ме търсите? Не знаете ли, че трябва да се намеря около дома на Отца Ми?
\par 50 А те не разбраха думата, която им рече.
\par 51 И Той слезе с тях, и дойде в Назарет, и там им се покоряваше. А майка Му спазваше всички тия думи в сърцето си.
\par 52 А Исус напредваше в мъдрост, в ръст и в благоволение пред Бога и човеците.

\chapter{3}

\par 1 В петнадесетата година на царуването на Тиверия Кесаря, когато Понтий Пилат беше управител на Юдея, а Ирод четверовластник в Галилея и брат му Филип четверовластник в Итурейската и Трахонитската страна, и Лисаний четверовластник в Авилиния,
\par 2 при първосвещенството на Анна Каиафа, Божието слово дойде до Иоана, Захариевия син, в пустинята.
\par 3 И той отиваше по цялата страна около Йордан и проповядваше кръщение на покаяние за прощаване на греховете,
\par 4 както е писано в книгата с думите на пророк Исаия: - Глас на един, който вика в пустинята: Пригответе пътя за Господа, Прави направете пътеките Му.
\par 5 Всяка долина ще се напълни. И всяка планина и хълм ще се сниши; Кривите пътеки ще станат прави, И неравните места - гладки пътища;
\par 6 И всяка твар ще види Божието спасение.
\par 7 И тъй, той казваше на множествата, които излизаха да се кръщават от него: Рожби ехиднини! Кой ви предупреди да бягате от идещия гняв?
\par 8 Прочее, принасяйте плодове, достойни за покаяние; и не почвайте да си думате, имаме Авраама за баща; защото ви казвам, че Бог може и от тия камъни да въздигне чада на Авраама.
\par 9 А и брадвата лежи вече при корена на дърветата: и тъй, всяко дърво, което не дава добър плод, отсича се и в огън се хвърля.
\par 10 И множествата го питаха, казвайки: Тогава какво да правим?
\par 11 А той в отговор им рече: Който има две ризи, нека даде на този, който няма; и който има храна, нека прави същото.
\par 12 Дойдоха и бирниците да се кръстят, и му рекоха: Учителю, ние какво да правим?
\par 13 Той им каза: Не изисквайте нищо повече от това, що ви е определено.
\par 14 Питаха го и военослужащи, казвайки: а ние какво да правим? Каза им: Не насилвайте никого, нито наклеветявайте; и задоволявайте се със заплатите си.
\par 15 И понеже людете бяха в недоумение, и всички размишляваха в сърцата си за Иоана, да не би той да е Христос,
\par 16 Иоан отговори на всички, като каза: Аз ви кръщавам с вода, но иде Оня, Който е по-силен от мене, Комуто не съм достоен да развържа ремъка на обущата Му; Той ще ви кръсти със Святия Дух и с огън.
\par 17 Той държи лопатата в ръката Си, за да очисти добре гумното Си и да събере житото в житницата Си; а плявата ще изгори в неугасимия огън.
\par 18 И с много други увещания той благовестяваше на людете.
\par 19 А четверовластникът Ирод, бидейки изобличаван от него, поради Иродиада, братовата си жена, и поради всичките други злини, които беше сторил Ирод,
\par 20 прибави над всичко друго и това, че затвори Иоана в тъмницата.
\par 21 И когато се кръстиха всичките люде, като се кръсти и Исус и се молеше, отвори се небето,
\par 22 и Святият Дух слезе върху Него в телесен образ като гълъб, и глас дойде от небето, който казваше: Ти си Моят възлюбен Син; в тебе е Моето благоволение.
\par 23 И сам Исус беше на около тридесет години, когато почна да поучава и, както мислеха, беше син на Йосифа, който бе син Илиев.
\par 24 А, Илий, Мататов; Матат Левиев; Левий, Мелхиев; Мелхий, Яанайев; Яанай, Йосифов;
\par 25 Йосиф Мататиев; Мататия, Амосов; Амос, Наумов; Наум, Еслиев; Еслий, Нагеев;
\par 26 Нагей, Маатов; Маат, Мататиев; Мататия, Семеинов; Семеин, Иосехов; Иосех, Иодов;
\par 27 Иода, Иоананов; Иоанан, Рисов; Риса, Зоровавелов; Зоровавел, Салатиилев; Салатиил, Нириев;
\par 28 Нирий, Мелхиев; Мелхий, Адиев; Адий, Косамов; Косам, Елмадамов; Елмадам, Иров;
\par 29 Ир, Исусов; Исус, Елиезеров; Елиезер, Иоримов; Иорим, Мататов; Матат, Левиев;
\par 30 Левий, Симеонов; Симеон, Юдов; Юда, Йосифов; Йосиф, Иоанамов; Ионам, Елиакимов;
\par 31 Елиаким, Мелеев; Мелеа, Менов; Мена, Мататов; Матата, Натанов; Натан, Давидов;
\par 32 Давид, Есеев; Есей, Овидов; Овид, Воозов; Вооз, Салмонов; Салмон, Наасонов;
\par 33 Наасон, Аминадавов; Аминадав, Арниев; Арний, Есронов; Есрон, Фаресов; Фарес, Юдов;
\par 34 Юда, Яковов; Яков, Исааков; Исаак, Авраамов; Авраам, Таров; Тара, Нахоров;
\par 35 Нахор, Серухов; Серух, Рагавов; Рагав, Фалеков; Фалек, Еверов; Евер, Салов;
\par 36 Сала, Каинанов; Каинан, Арфаксадов; Арфаксад, Симов; Сим, Ноев; Ное, Ламехов;
\par 37 Ламех, Матусалов; Матусал, Енохов; Енох, Яредов; Яред, Малелеилов; Малелеил, Каинанов;
\par 38 Каинан, Еносов; Енос, Ситов; Сит, Адамов; а Адам, Божий.

\chapter{4}

\par 1 А Исус, пълен с Святия Дух, когато се върна, от Йордан, бе воден от Духа из пустинята четиридесет дена,
\par 2 дето бе изкушаван от дявола. И не яде нищо през тия дни; и като се изминаха те, Той огладня.
\par 3 И дяволът Му рече: Ако си Божий Син, заповядай на тоя камък да стане хляб.
\par 4 А Исус му отговори: Писано е: "Не само с хляб ще живее човек, [но с всяко Божие слово"].
\par 5 Тогава, като Го възведе [на една планина] на високо и Му показа всичките царства на вселената, в един миг време, дяволът Му рече:
\par 6 На тебе ще дам всичката власт и слава на тия царства, (защото на мене е предадена, и аз я давам комуто ща), -
\par 7 и тъй, ако ми се поклониш, всичко ще бъде Твое.
\par 8 А Исус в отговор му каза: Писано е, "На Господа твоя Бог, да се кланяш, и само Нему да служиш".
\par 9 Тогава Го заведе в Ерусалим, постави Го на крилото на храма и Му рече: Ако си Божий Син, хвърли се от тук долу;
\par 10 защото е писано: - "Ще заповяда на ангелите Си за Тебе да Те пазят:
\par 11 и на ръце ще Те дигат, да не би да удариш о камък ногата Си".
\par 12 А Исус в отговор му рече: Казано е: "Да не изпиташ Господа, твоя Бог".
\par 13 И като изчерпи всяко изкушение, дяволът се оттегли от Него за известно време.
\par 14 А Исус се върна в Галилея със силата на Духа; и слух се разнесе за Него по цялата околност.
\par 15 И той поучаваше по синагогите им; и всички Го прославяха.
\par 16 И дойде в Назарет, дето беше отхранен, и по обичая Си влезе в синагогата един съботен ден и стана да чете.
\par 17 И подадоха Му книгата на пророк Исаия; и Той, като отвори книгата, намери мястото дето бе писано: -
\par 18 "Духът на Господа е на Мене, Защото Ме е помазал да благовестявам на сиромасите; Прати Ме да проглася освобождение на пленниците, И прогледване на слепите, Да пусна на свобода угнетените,
\par 19 да проглася благоприятната Господна година".
\par 20 И като затвори книгата, върна я на служителя и седна; а очите на всички в синагогата бяха впити в Него.
\par 21 И почна да им казва: Днес се изпълни това писание във вашите уши.
\par 22 И всички му засвидетелствуваха, чудещи се на благодатните думи, които излизаха из устата Му. И думаха: Тоя не е ли Йосифовият син?
\par 23 А той им рече: Без друго ще Ми кажете тая поговорка: Лекарю, изцери себе си; каквото сме чули, че става в Капернаум, стори го и тука в Своята родина.
\par 24 И пак рече: Истина ви казвам, че никой пророк не е приет в родината си.
\par 25 А казвам ви наистина, много вдовици имаше в Израил в дните на Илия, когато се затвори небето за три години и шест месеца, и настана голям глад по цялата земя;
\par 26 а нито при една от тях не бе пратен Илия, а само при една вдовица в Сарепта Сидонска.
\par 27 Тъй също много прокажени имаше в Израил във времето на пророк Елисея; но никой от тях не бе очистен, а само сириецът Нееман.
\par 28 Като чуха това тия, които бяха в синагогата, всички се изпълниха с гняв,
\par 29 и, като станаха, изкараха Го вън из града, и заведоха Го при стръмнината на хълма, на който градът им беше съграден, за да Го хвърлят долу.
\par 30 Но Той мина посред тях и си отиде.
\par 31 И слезе в Галилейския град Капернаум и поучаваше ги в съботен ден;
\par 32 и учудваха се на учението Му, защото Неговото слово беше с власт.
\par 33 И в синагогата имаше човек, хванат от духа на нечист бяс; и той извика със силен глас:
\par 34 Ех, какво имаш Ти с нас, Исусе Назарянине? Нима си дошъл да ни погубиш? Познавам Те Кой си Ти, Светият Божий.
\par 35 Но Исус го смъмра, казвайки: Млъкни и излез из него. И бесът, като го повали насред, излезе из него, без да го повреди никак.
\par 36 И всички се смаяха, и разговаряха се помежду си, думайки: Какво е това слово, дето Той с власт и сила заповядва на нечистите духове, и те излизат?
\par 37 И слух се разнесе за Него по всичките околни места.
\par 38 И като стана та излезе от синагогата, влезе в Симоновата къща. А Симоновата тъща беше хваната от силна треска; и молиха Го за нея.
\par 39 И Той, като застана над нея, смъмра треската, и тя я остави; и на часа стана та им прислужваше.
\par 40 И когато залязваше слънцето, всички, които имаха болни от разни болести, доведоха ги при Него; а Той, като положи ръце на всеки от тях, изцели ги.
\par 41 Още и бесове с крясък излизаха из мнозина, и казваха: Ти си Божия Син. А Той ги мъмреше, и не ги оставяше да говорят, понеже знаеха, че Той е Христос.
\par 42 И като се съмна, Той излезе и отиде в уединено място; а народът Го търсеше, дохождаше при Него и искаше да Го задържи, за да си не отива от тях.
\par 43 Но Той им рече: И на другите градове трябва да благовестя Божието царство, понеже за това съм изпратен.
\par 44 И проповядваше в галилейските синагоги.

\chapter{5}

\par 1 А еднаж, когато народът Го притискаше да слуша Божието слово, Той стоеше при Генисаретското езеро.
\par 2 И видя две ладии, спрени край езерото; а рибарите бяха излезли от тях и изпираха мрежите си.
\par 3 И като влезе в една от ладиите, която беше Симонова, помоли го да я отдалечи малко от сушата; и седна та поучаваше народа от ладията.
\par 4 И като престана да говори, рече на Симона: Оттегли ладията към дълбокото и хвърлете мрежите за ловитба.
\par 5 А Симон в отговор рече: Учителю, цяла нощ се трудихме и нищо не уловихме; но по Твоята дума ще хвърля мрежите.
\par 6 И като сториха това, уловиха твърде много риба, така щото се прокъсваха мрежите им.
\par 7 И извикаха съдружниците си от другата ладия да им дойдат на помощ; и те дойдоха и напълниха и двете ладии, до толкова, щото щяха да потънат.
\par 8 А Симон Петър, като видя това, падна пред Исусовите колене и каза: Иди си от мене, Господи, защото съм грешен човек.
\par 9 Понеже той и всички, които бяха с него, се учудиха на ловитбата на рибите що уловиха,
\par 10 също и Яков и Иоан, синове на Заведея, които бяха Симонови съдружници. А Исус рече на Симона: Не бой се; отсега човеци ще ловиш.
\par 11 И когато извлякоха ладиите на сушата, оставиха всичко и отидоха след Него.
\par 12 И когато беше в един от градовете, ето, човек, който беше цял прокажен, като видя Исуса, падна на лицето си и Му се помоли, казвайки: Господи, ако искаш можеш да ме очистиш.
\par 13 А Той простря ръка и се допря до него, и рече: Искам: бъди очистен! И на часа проказата го остави.
\par 14 И Той му заръча, никому да не каже това: Но, за свидетелство на тях, иди, каза, и се покажи на свещеника, и принеси за очистването си, според както е заповядал Моисей.
\par 15 Но още повече се разнасяше вестта за Него; и големи множества се събираха да слушат и да се изцеляват от болестите си.
\par 16 А Той се оттегляше в пустините и се молеше.
\par 17 И през един от тия дни, когато Той поучаваше, там седяха фарисеи и законоучители, надошли от всяко село на Галилея, Юдея и Ерусалим; и сила от Господа бе с Него да изцелява.
\par 18 И ето мъже, които носеха на постелка един болен човек, който беше паралитик; и опитаха се да го внесат вътре и да го сложат пред Него.
\par 19 Но понеже не намериха през где да го внесат вътре, поради народа, качиха се на покрива, и през керемидите го спуснаха с постелката насред пред Исуса.
\par 20 И Той, като видя вярата им рече: Човече, прощават ти се греховете.
\par 21 Тогава книжниците и фарисеите почнаха да се препират, казвайки: Кой е Тоя, Който богохулствува? Кой може да прощава грехове, освен един Бог?
\par 22 Но Исус, като видя разискванията им, каза им в отговор: Защо разисквате в сърцата си?
\par 23 Кое е по-лесно, да река: Прощават ти се греховете, или да река: Стани и ходи?
\par 24 Но за да познаете, че Човешкият Син има власт на земята да прощава грехове, (рече на паралитика): Казвам ти: Стани, дигни постелката си, и иди у дома си.
\par 25 И начаса той стана пред тях, дигна това, на което лежеше, и отиде у дома си, като славеше Бога.
\par 26 И те всички се учудиха и славеха Бога, и, изпълнени със страх, казваха: Днес видяхме пречудни неща.
\par 27 След това Исус като излезе, видя един бирник, на име Левий, седящ в бирничеството, и рече му: Върви след Мене.
\par 28 Той остави всичко, стана и тръгна след Него.
\par 29 А Левий Му направи голямо угощение в къщата си; и имаше голямо множество бирници и други, които седяха на трапезата с тях.
\par 30 А фарисеите и техните книжници роптаеха против учениците Му, казвайки: Защо ядете и пиете с бирниците и грешниците?
\par 31 Исус в отговор им рече: Здравите нямат нужда от лекар, а болните.
\par 32 Не съм дошъл да призова праведните, но грешните на покаяние.
\par 33 И те Му рекоха: Иоановите ученици често постят и правят молитви, така и фарисейските, а Твоите ядат и пият.
\par 34 Исус им рече: Можете ли да накарате сватбарите да постят докато е с тях младоженецът?
\par 35 Ще дойдат, обаче дни, когато младоженецът ще се отнеме от тях; тогава, през ония дни, ще постят.
\par 36 Каза им още и притча: Никой не отдира кръпка от нова дреха да я тури на вехта дреха; инак и новата дреха се съдира, и кръпката от новата не прилича на вехтата.
\par 37 И никой не налива ново вино в стари мехове; инак, новото вино ще пръсне меховете, и то само ще изтече, и меховете ще се изхабят.
\par 38 Но трябва да се налива ново вино в нови мехове.
\par 39 И никой, след като е пил старо вино, не иска ново, защото казва: Старото е по-добро.

\chapter{6}

\par 1 И една събота, [първата след втория ден на Пасхата], като минаваше Той през посевите, учениците Му късаха класове и ядяха, като ги стриваха с ръце.
\par 2 А някои от фарисеите рекоха: Защо правите това, което не е позволено да се прави в събота?
\par 3 Исус в отговор им рече: Не сте ли чели това, което стори Давид, когато огладня, той и мъжете, които бяха с него,
\par 4 как влезе в Божия дом, взе присъствените хлябове и яде, и даде на ония, които бяха с него, които хлябове не е позволено никой да яде, а само свещениците?
\par 5 И каза им: Човешкият Син е господар на съботата.
\par 6 И в друга събота влезе в синагогата и поучаваше; и там имаше един човек, чиято дясна ръка бе изсъхнала.
\par 7 И книжниците и фарисеите Го наблюдаваха, дали в събота ще го изцели, за да могат да Го обвинят.
\par 8 Но Той знаеше помислите им, и каза на човека с изсъхналата ръка: Стани и се изправи насред. И той стана и се изправи.
\par 9 Тогава им рече Исус: Питам ви: Какво е позволено да прави човек в събота? Добро ли да прави или зло? Да спаси ли живот или да погуби?
\par 10 И като ги изгледа всички, рече на човека: Простри ръката си. И той направи така; и ръката му оздравя.
\par 11 А те се изпълниха с луда ярост и се разговаряха помежду си какво биха могли да сторят на Исуса.
\par 12 През ония дни Исус излезе на бърдото да се помоли и прекара цяла нощ в молитва към Бога.
\par 13 И като се съмна, повика учениците Си, и избра от тях дванадесет души, които и нарече апостоли:
\par 14 Симона, когото и нарече Петър, и брата му Андрея, Якова и Иоана, Филипа и Вартоломея,
\par 15 Матея и Тома, Якова Алфеев и Симона, наречен Зилот.
\par 16 Юда, Якововия брат, и Юда Искариотски, който и стана предател.
\par 17 И като слезе заедно с тях, Той се спря на едно равно място; спряха се там и голямо множество от учениците Му и голяма навалица от люде от цяла Юдея и Ерусалим и от Тирското и Сидонското крайморие, които бяха дошли да Го чуят и да се изцелят от болестите си;
\par 18 тоже и измъчваните от нечисти духове се изцеляваха.
\par 19 И целият народ се стараеше да се допре до Него, защото сила излизаше от Него и изцеляваше всичките.
\par 20 И Той подигна очи към учениците Си и каза: Блажени вие сиромаси; защото е ваше Божието царство.
\par 21 Блажени, които гладувате сега, защото ще се наситите. Блажени, които плачете сега, защото ще се разсмеете.
\par 22 Блажени сте, когато ви намразят човеците, и когато ви отлъчат от себе си и ви похулят и отхвърлят името ви като лошо, поради Човешкия Син;
\par 23 възрадвайте се в оня ден и заиграйте, защото, ето, голяма е наградата ви на небесата; понеже бащите им така правеха на пророците.
\par 24 Но горко на вас богатите; защото сте приели вече утехата си.
\par 25 Горко на вас, които сега сте наситени; защото скоро ще изгладнеете. Горко на вас, които сега се смеете, защото ще жалеете и ще плачете.
\par 26 Горко на вас, когато всички човеци ви захвалят, защото бащите им така правеха на лъжепророците.
\par 27 Но на вас, които слушате, казвам: Обичайте неприятелите си, правете добро на тия, които ви мразят,
\par 28 благославяйте тия, които ви кълнат, молете се за тия, които ви правят пакост.
\par 29 На този, който те плесне по едната буза, обърни и другата; и на този, който ти вземе горната дреха, не отказвай и ризата си.
\par 30 Дай на всеки, който ти поиска; и не изисквай нещата си от този, който ги отнема.
\par 31 И както желаете да правят човеците на вас, така и вие правете на тях.
\par 32 Понеже ако обичате само ония, които обичат вас, каква благодарност ви се пада? Защото и грешниците обичат ония, които тях обичат.
\par 33 И ако правите добро само на ония, които на вас правят добро, каква благодарност ви се пада? Защото и грешниците правят същото.
\par 34 И ако заемете само на тия, от които се надявате да вземете, каква благодарност ви се пада? Защото и грешни на грешни заемат, за да вземат назад равното.
\par 35 Но вие обичайте неприятелите си, правете добро, и заемайте, без да очаквате да приемете назад; и наградата ви ще бъде голяма, и ще бъдете чада на Всевишния; защото Той е благ към неблагодарните и злите.
\par 36 Бъдете [прочее] милосърдни, както и Отец ваш е милосърден.
\par 37 Не съдете, и няма да бъдете съдени; не осъждайте, и няма да бъдете осъждани; прощавайте, и ще бъдете простени;
\par 38 давайте, и ще ви се дава; добра мярка, натъпкана, стърсена, препълнена ще ви дават в пазухата; защото с каквато мярка мерите, с такава ще ви се отмерва.
\par 39 Рече им една притча: Може ли слепец слепеца да води? Не ще ли паднат и двамата в яма?
\par 40 Ученикът не е по-горен от учителя си; а всеки ученик, когато се усъвършенствува, ще бъде като учителя си.
\par 41 И защо гледаш съчицата в окото на брата си, а не забелязваш гредата в твоето око?
\par 42 Или как можеш да речеш на брата си: Брате, остави ме да извадя съчицата, която е в окото ти, когато ти сам не виждаш гредата, която е в твоето око? Лицемерецо, първо извади гредата от своето око, и тогава ще видиш ясно, за да извадиш съчицата, която е в братовото ти око.
\par 43 Защото няма добро дърво, което дава лош плод, нито пък лошо дърво, което дава добър плод.
\par 44 Понеже всяко дърво от своя плод се познава; защото не берат смокини от тръни, нито късат грозде от къпина.
\par 45 Добрият човек от доброто съкровище на сърцето си изнася доброто; а злият човек от злото си съкровище изнася злото; защото от онова, което препълва сърцето му, говорят неговите уста.
\par 46 И защо Ме зовете: Господи, Господи, и не вършите това, което казвам?
\par 47 Всеки, който дохожда при Мене, и слуша Моите думи, и ги изпълнява, ще ви покажа на кого прилича.
\par 48 Прилича на човек, който като построи къща, изкопа и задълбочи, и положи основа на канара; и когато стана наводнение, реката се устреми върху оная къща, но не можа да я поклати, защото беше здраво построена.
\par 49 А който слуша и не изпълнява, прилича на човек, който е построил къща на земята, без основа; върху която се устреми реката, и начаса рухна; и срутването на оная къща беше голямо.

\chapter{7}

\par 1 След като свърши всичките Свои речи към людете, които Го слушаха(Гръцки: В ушите на людете.), Исус влезе в Капернаум.
\par 2 А на някой си стотник слугата, който му беше мил, боледуваше на умиране.
\par 3 И като чу за Исуса, изпрати до Него някои юдейски старейшини да Го помолят да дойде и да оздрави слугата му.
\par 4 Те, прочее, дойдоха при Исуса и Му се молеха усърдно, като казваха: Той заслужава да му сториш това;
\par 5 защото обича нашия народ, и той ни е построил синагогата.
\par 6 А когато Исус вървеше, с тях и не беше вече далеч от къщата, стотникът изпрати до Него приятели да Му кажат: Господи, не си прави труд, защото не съм достоен да влезеш под стряхата ми;
\par 7 затова нито счетох себе си достоен да дойда при тебе; кажи само дума, и слугата ми ще оздравее.
\par 8 Защото и аз съм човек, поставен под власт, и имам подчинени на мен войници; и казвам на един: Иди, и той отива; и на друг: Дойди, и дохожда; и на слугата си: Направи това, и го прави.
\par 9 Като чу това, Исус му се почуди; обърна се и рече на вървящия подир Него народ: Казвам ви, нито в Израил съм намерил толкова вяра.
\par 10 И когато изпратените се върнаха в къщата, намериха слугата оздравял.
\par 11 А скоро след това, Исус отиде в един град, наречен Наин; и с Него вървяха учениците Му и едно голямо множество.
\par 12 И когато се приближи до градската порта, ето, изнасяха мъртвец, едничък син на майка си, която беше и вдовица; и с нея имаше голямо множество от града.
\par 13 И Господ, като я видя, смили се за нея и рече -: Недей плака!
\par 14 Тогава се приближи и се допря до носилото; а носачите се спряха. И рече: Момче, казвам ти, стани!
\par 15 И мъртвият се подигна и седна, и почна да говори. И Исус го даде на майка му.
\par 16 И страх обзе всички, и славеха Бога, казвайки: Велик пророк се издигна между нас; и Бог посети Своите люде.
\par 17 И това, което казваха, се разнесе за Него по цяла Юдея и по цялата околност.
\par 18 И учениците на Иоана му известиха за всичко това.
\par 19 И Иоан повика двама от учениците си и ги прати при Господа да кажат: Ти ли си Оня, Който има да дойде, или друг да очакваме?
\par 20 И като дойдоха при Него човеците рекоха: Иоан Кръстител ни прати до Тебе да кажем: Ти ли си Оня, Който има да дойде, или друг да очакваме?
\par 21 И в същия час Той изцели мнозина от болести и язви, и зли духове, и на мнозина слепи подари зрение.
\par 22 Тогава в отговор им каза: Идете и разкажете на Иоана това, което видяхте и чухте, че слепи прогледват, куци прохождат, прокажени се очистват, и глухи прочуват; мъртви биват възкресени, и на сиромасите се проповядва благовестието.
\par 23 И блажен е оня, който не се съблазни в Мене.
\par 24 А когато си отидоха пратените от Иоана, Исус почна да говори на народа за Иоана: Какво излязохте да видите в пустинята? Тръстика ли от вятър разлюлявана?
\par 25 Но какво излязохте да видите? Човек ли в меки дрехи облечен? Ето великолепно облечените и ония, които живеят разкошно, са в царските дворци.
\par 26 Но какво излязохте да видите? Пророк ли? Да, казвам ви, и повече от пророк.
\par 27 Това е оня, за когото е писано: - "то, Аз изпращам вестителя Си пред Твоето лице, Който ще устрои пътя Ми пред Тебе"
\par 28 Казвам ви: Между родените от жена няма по-голям от Иоана; обаче, по-малкият в Божието царство е по-голям от него.
\par 29 (И всички люде и бирници, като чуха това, признаха Божията правда, като бяха се кръстили с Иоановото кръщение.
\par 30 А фарисеите и законниците осуетиха Божията воля спрямо себе си, като не бяха се кръстили от него).
\par 31 А на какво да уподобя човеците от това поколение? И на какво приличат?
\par 32 Те приличат на деца, седящи на пазаря, които викат едно на друго, казвайки: Свирихме ви, и не играхте; ридахме, и не плакахте.
\par 33 Защото Иоан Кръстител дойде, който нито хляб яде, нито вино пие, и казвате: Бяс има.
\par 34 Дойде Човешкият Син, който яде и пие, и казвате: Ето човек лаком и винопиец, приятел на бирниците и на грешниците.
\par 35 Но пак мъдростта се оправдава от всичките си чада.
\par 36 И един от фарисеите го покани да яде с него; и Той влезе във фарисеевата къща и седна на трапезата.
\par 37 И, ето, една жена от града, която беше грешница, като разбра, че седи на трапезата във фарисеевата къща, донесе алавастрен съд с миро.
\par 38 И като застана отзаде при нозете Му и плачеше, почна да облива нозете Му със сълзи и да ги изтрива с косата си, целуваше нозете Му, и мажеше ги с мирото.
\par 39 А като видя това фарисеят, който Го бе поканил, думаше в себе си, казвайки: Тоя ако беше пророк, щеше да знае коя и каква е жената, която се допира до Него, че е грешница.
\par 40 А Исус в отговор му рече: Симоне, имам нещо да ти кажа. А той рече: Учителю, кажи!
\par 41 Някой си заемодавец имаше двама длъжника; единият дължеше петстотин динара, а другият - петдесет.
\par 42 И понеже нямаха с какво да му платят, той прости и на двамата. И тъй, кой от тях ще го обикне повече?
\par 43 В отговор Симон рече: Мисля, че оня, комуто е простил повечето. А Той му рече: Право си отсъдил.
\par 44 И като се обърна към жената, рече на Симона: Видиш ли тая жена? Влязох в къщата ти, и ти вода за нозете Ми не даде; а тя със сълзи обля нозете Ми, и с косата си ги изтри.
\par 45 Ти целувка ми не даде; а тя, откак съм влязъл не е престанала да целува нозете Ми.
\par 46 Ти с масло не помаза главата Ми; а тя с миро помаза нозете Ми.
\par 47 Затова ти казвам: Прощават - се многото грехове; (защото тя обикна много); а комуто малко се прощава, той малко обича.
\par 48 И рече -: Прощават ти се греховете.
\par 49 И тия, които седяха с Него на трапезата, почнаха да казват помежду си: Кой е тоя, който и греховете прощава?
\par 50 И рече на жената: Твоята вяра те спаси; иди си с мир.

\chapter{8}

\par 1 И скоро след това Исус ходеше по градове и села да проповядва и да благовестява Божието царство; и с Него бяха дванадесетте ученика,
\par 2 и някои жени, които бяха изцелени от зли духове и болести: Мария, наречена Магдалина, от която бяха излезли седем бяса,
\par 3 и Иоана жената на Иродовия настойник Хуза и Сусана и много други, които им услужваха с имота си.
\par 4 И понеже се събра голямо множество, и дохождаха при Него от всеки град, рече с притча:
\par 5 Сеячът излезе да сее семето си; и когато сееше, едно падна край пътя; и затъпка се, и небесните птици го изкълваха.
\par 6 А друго падна на канарата; и щом поникна изсъхна, защото нямаше влага.
\par 7 Друго пък падна всред тръните; и заедно с него порастнаха тръните и го заглушиха.
\par 8 А друго падна на добра земя и като порастна, даде стократен плод. Като каза това, извика: Който има уши да слуша, нека слуша.
\par 9 А учениците Му Го попитаха за значението на тая притча.
\par 10 Той каза: На вас е дадено да знаете тайните на Божието царство; а на другите се проповядва с притчи, тъй щото, като гледат, да не виждат, и като слушат, да не разбират.
\par 11 Прочее, ето значението на притчата: Семето е Божието слово.
\par 12 А посяното край пътя са тия, които са слушали; тогава дохожда дяволът и отнема словото от сърцата им, да не би да повярват и се спасят.
\par 13 Падналото на канарата са тия, които, когато чуят, приемат словото с радост; но те, като нямат корен, временно вярват, а когато настане изпитание, отстъпват.
\par 14 Падналото всред тръните са ония, които са слушали, и, като си отиват, заглъхват от грижи и богатства и житейски удоволствия и не дават узрял плод.
\par 15 А посятото на добра земя са тия, които, като чуят словото, държат го в искрено и добро сърце, и дават плод с търпение.
\par 16 И никой, като запали светило, не го покрива със съд, нито го туря под легло, но го туря на светилник, за да видят светлината тия, които влизат.
\par 17 Защото няма нещо тайно, което не ще стане явно, нито потаено, което не ще се узнае и не ще излезе наяве.
\par 18 Затова, внимавайте как слушате; защото който има, нему ще се даде; а който няма, от него ще се отнеме и това, което мисли че има.
\par 19 И дойдоха при Него майка Му и братята Му, но поради множеството не можаха да се приближат до Него.
\par 20 И известиха Му: Майка Ти и братята Ти стоят вън и искат да Те видят.
\par 21 А Той в отговор им рече: Майка Моя и братя Мои са тия, които слушат Божието слово и го изпълняват.
\par 22 А в един от тия дни Той влезе в ладия с учениците Си, и рече им: Да минем на отвъдната страна на езерото. И отплуваха.
\par 23 А като плуваха, Той заспа; и ветрена буря се устреми върху езерото, и вълните ги заплашваха така щото бяха в опасност.
\par 24 И дойдоха, разбудиха Го и казаха: Наставниче! Наставниче! Загиваме! А Той се събуди и смъмра вятъра и развълнуваната вода; и успокоиха се, и настана тишина.
\par 25 И рече им: Где е вярата ви? А те уплашени се чудеха, и казваха си един на друг: Кой ли ще е Тоя, който заповядва и на ветровете и на водата, и те Му се покоряват?
\par 26 И пристигнаха в Герасинската страна, която е срещу Галилея.
\par 27 И като излезе на сушата, срещна Го от града някой си човек, който имаше бесове, и за дълго време не беше обличал дреха, и в къщи не живееше, но в гробищата.
\par 28 Той, като видя Исуса, извика, падна пред Него, и рече със силен глас: Какво имаш Ти с мене Исусе, Сине на Всевишния Бог? Моля Ти се недей ме мъчи!
\par 29 Защото Исус беше заповядал на нечистия дух да излезе от човека. Понеже много пъти бе го прихващал; и връзваха го с вериги и окови и го пазеха; но той разкъсваше връзките, и бесът го гонеше по пустините.
\par 30 Исус го попита: Как ти е името? А той каза: Легион, защото много бесове бяха влезли в него.
\par 31 И молеха Го да не им заповяда да отидат в бездната.
\par 32 А там имаше голямо стадо свини, което пасеше по хълма, и бесовете Го помолиха да им позволи да влязат в тях. И позволи им.
\par 33 И като излязоха бесовете из човека и влязоха в свините, стадото се спусна по стръмнината в езерото и се издави.
\par 34 А свинарите, като видяха станалото, прибягнаха и известиха за това в града и в селата.
\par 35 И като излязоха да видят станалото и дойдоха при Исуса, намериха човека, от когото бяха излезли бесовете, седнал при Исусовите нозе, облечен и смислен; и изплашиха се.
\par 36 И тия, които бяха видели това, разказаха им как излекувал бесния.
\par 37 И цялото множество от Герасинската околност Му се помоли да си отиде от тях, защото бяха обзети от голям страх; и Той влезе в ладията и се завърна.
\par 38 А човекът, от когото бяха излезли бесовете, молеше Му се да бъде с него; но Исус го изпрати като каза:
\par 39 Върни се у дома си и разкажи, какви неща ти стори Бог. И той отиде и разгласи по целия град, какви неща му стори Исус.
\par 40 А когато се върна Исус, народът Го посрещна радостно, защото всички Го чакаха.
\par 41 И, ето, дойде един човек на име Яир, който беше началник на синагогата, и падна пред Исусовите нозе и Му се молеше да влезе в къщата му;
\par 42 защото имаше едничка дъщеря, на около дванадесет години, и тя беше на умиране. И когато отиваше, народът Го притискаше.
\par 43 И една жена, която имаше кръвотечение от дванадесет години, и бе иждивила за лекари целия си имот, без да може да се излекува от никого,
\par 44 се приближи изотзад та се допря до полата на дрехата Му; и на часа престана кръвотечението -.
\par 45 И рече Исус: Кой се допря до Мене? И когато всички се отричаха, Петър и ония, които бяха с Него, казаха: Наставниче, народът Те притиска и гнети, [а Ти казваш: Кой се допря до Мене?]
\par 46 Но Исус каза: Някой се допря до Мене; защото Аз усетих, че сила излезе от Мене.
\par 47 И жената, като видя, че не се укри, дойде разтреперана и падна пред Него и извика пред всичките люде, по коя причина се допря до Него, и как на часа оздравя.
\par 48 А той - рече: Дъщерьо, твоята вяра те изцели; иди си с мир!
\par 49 Докато Той още говореше, дохожда някой си от къщата на началника на синагогата и казва: Дъщеря ти умря; не затруднявай Учителя.
\par 50 А Исус като дочу това, отговори му: Не бой се; само вярвай, и тя ще се избави.
\par 51 И когато дойде в къщата, не остави никой да влезе с Него, освен Петра, Иоана, Якова и бащата и майката на момичето.
\par 52 И всички плачеха и го оплакваха. А Той им рече: Не плачете; защото не е умряло, а спи.
\par 53 А те Му се присмиваха, понеже знаеха, че е умряло.
\par 54 Но Той го хвана за ръката, и извика: Момиче, стани!
\par 55 И върна се духът му, и то на часа стана; и Той заповяда да му дадат нещо да яде.
\par 56 И родителите му се учудиха; а Той им заръча да не казват никому за станалото.

\chapter{9}

\par 1 И като свика дванадесетте, даде им сила и власт над всички бесове, и да изцеляват болести.
\par 2 И изпрати ги да проповядват Божието царство и да изцеляват болните.
\par 3 И каза им: Не вземайте нищо за път, ни тояга, ни торба, ни хляб, ни пари, нито да имате по две ризи.
\par 4 И в която къща влезете, там седете, и от там тръгвайте на път.
\par 5 И ако някои не ви приемат, когато излизате от оня град отърсете и праха от нозете си, за свидетелство против тях.
\par 6 И те тръгнаха и отиваха по селата и проповядваха благовестието и изцеляваха навсякъде.
\par 7 А четверовластникът Ирод чу за всичко що ставало и беше в недоумение; защото някои казваха, че Иоан е възкръснал от мъртвите;
\par 8 други пък, че Илия се е явил; а други, че един от старовременните пророци е възкръснал.
\par 9 И рече Ирод: Иоана аз обезглавих; но Кой е Тоя, за Когото слушам такива неща? И желаеше да Го види.
\par 10 И като се върнаха апостолите, разказаха на Исуса все що бяха извършили; и Той ги взе и се оттегли насаме [в уединено място] до един град, наречен Витсаида.
\par 11 А множествата, като разбраха това, отидоха подире Му; и Той ги посрещна с готовност, и им говореше за Божието царство, и изцеляваше ония, които имаха нужда от изцеление.
\par 12 И като почна денят да преваля, дванадесетте се приближиха и Му рекоха: Разпусни народа за да отидат в околните села и колиби да нощуват и да си намерят храна, защото тука сме в уединено място.
\par 13 Но Той им каза: Дайте им вие да ядат. А те рекоха: Нямаме повече от пет хляба и две риби, освен, - да отидем ли и да купим храна за всички тия люде?
\par 14 (Защото имаше около пет хиляди мъже). И каза на учениците Си: Накарайте ги да насядат на групи по петдесет души.
\par 15 Те сториха така и накараха всички да насядат.
\par 16 А той взе петте хляба и двете риби, и погледна към небето и ги благослови; и като ги разчупи, даваше на учениците да сложат пред народа.
\par 17 И ядоха и всички се наситиха; и дигнаха къшеите, които им останаха, дванадесет коша.
\par 18 И когато Той се молеше насаме, и учениците бяха с Него, попита ги, казвайки: Според както казва народът, Кой съм аз?
\par 19 А в отговор те рекоха: Едни казват, че си Иоан Кръстител; а други - Илия; трети пък, - че един от старовременните пророци е възкръснал.
\par 20 Тогава им каза: А според, както вие мислите, Кой съм Аз? Петър в отговор рече: Божият Христос.
\par 21 А Той им заръча, и заповяда да не казват никому, като рече:
\par 22 Човешкият Син трябва много да пострада, и да бъде отхвърлен от старейшините, главните свещеници и книжниците, да бъде убит, и на третия ден да бъде възкресен.
\par 23 Каза още и на всички: Ако иска някой да дойде след Мене, нека се отрече от себе си, нека носи кръста си всеки ден, и нека Ме следва.
\par 24 Защото, който иска да спаси живота си, ще го изгуби; а който изгуби живота си заради Мене, той ще го спаси.
\par 25 Понеже какво се ползва човек, ако спечели целия свят, а изгуби или ощети себе си?
\par 26 Защото, ако се срамува някой от Мене и от думите ми, то и Човешкият Син ще се срамува от него, когато дойде в Своята слава и в славата на Отца и на Своите ангели.
\par 27 А казвам ви наистина, има някои от тук стоящите, които никак няма да вкусят смърт, докле не видят Божието царство.
\par 28 И около осем дни след като каза това, Той взе със Себе Си Петра, Йоана и Якова, и се качи на планината да се моли.
\par 29 И като се молеше, видът на лицето Му се измени, и облеклото Му стана бяло и бляскаво.
\par 30 И, ето, двама мъже се разговаряха с Него; те бяха Моисей и Илия,
\par 31 които се явиха в слава и говореха за смъртта Му(Гръцки: Изхода.), която Му предстоеше да изпълни в Ерусалим.
\par 32 А Петър и ония, които бяха с него, ги беше налегнал сън; но когато се разбудиха, видяха славата Му и двамата мъже, които стояха с Него.
\par 33 И когато те се разделиха с Него, Петър рече на Исуса: Наставниче, добре е да сме тука; и нека направим три шатри, за Тебе една, за Моисея една, и една за Илия, без да знае какво дума.
\par 34 И докато казваше това, дойде облак та ги засени; и учениците се уплашиха като влязоха в облака.
\par 35 И дойде из облака глас, който каза: Този е Моят Син, Избраник Мой; Него слушайте!
\par 36 И когато престана гласът, Исус се намери сам. И те замълчаха, и през ония дни не казваха никому нищо от това, що бяха видели.
\par 37 И на следния ден, когато слязоха от планината, срещна Го голямо множество.
\par 38 И, ето, един човек от народа извика, казвайки: Учителю, моля Ти се да погледнеш на сина ми, защото ми е единствено чадо.
\par 39 И, ето, дух го прехваща и той изведнъж закрещява и духът го сгърчва така, че той се запеня, и като го смазва, едвам го напуща.
\par 40 И помолих Твоите ученици да го изгонят, но не можаха.
\par 41 Исус в отговор рече: О роде неверен и извратен! докога ще бъда с вас и ще ви търпя? Доведи сина си тука.
\par 42 И когато още идваше, бесът го тръшна и сгърчи силно; а Исус смъмра нечистия дух, изцели момчето, и върна го на баща му.
\par 43 И всички се учудваха на Божието величие. А докато всички се чудеха на всичко, което правеше, Той рече на учениците Си:
\par 44 Вложете в ушите си тия думи, защото Човешкият Син ще бъде предаден в човешки ръце.
\par 45 Но те не разбираха тая дума; тя бе скрита от тях за да не я разберат; а бояха се да Го попитат за тая дума.
\par 46 И повдигна се между тях разискване, кой от тях ще бъде по-голям.
\par 47 А Исус, като видя мисълта на сърцето им, взе едно детенце, постави го при Себе Си, и рече им:
\par 48 Който приеме това детенце в Мое име, Мене приема; и който приеме Мене, приема Този, който Ме е изпратил; защото който е най-малък между всички вас, той е голям.
\par 49 А Йоан продума и рече: Наставниче, видяхме един човек да изгонва бесове в Твое име: и му запретихме, защото не следва с нас.
\par 50 А Исус му рече: Недейте му запрещава; защото, който не е против вас откъм вас е.
\par 51 И когато се навършваха дните да се възнесе, Той насочи лицето Си да пътува към Ерусалим.
\par 52 И проводи пред Себе Си пратеници, които отидоха и влязоха в едно самарянско село да приготвят за Него.
\par 53 Но не Го приеха, защото лицето Му беше обърнато към Ерусалим.
\par 54 Като видяха това учениците Му Яков и Йоан, рекоха: Господи, искаш ли да заповядаме да падне огън от небето и да ги изтреби [както стори и Илия]?
\par 55 А Той се обърна та ги смъмра; [и рече: Вие не знаете на какъв сте дух; защото Човешкият Син не е дошъл да погуби човешки души, но да спаси].
\par 56 И отидоха в друго село.
\par 57 А като вървяха в пътя, един човек Му рече: Ще те следвам дето и да идеш.
\par 58 Исус му каза: Лисиците си имат легловища, и небесните птици гнезда; а Човешкият Син няма где глава да подслони.
\par 59 А на друг каза: Върви след Мене. А той рече: Господи, позволи ми първо да отида и погреба баща си.
\par 60 Но Той му каза: Остави мъртвите да погребват своите мъртви; а ти иди и разгласявай Божието царство.
\par 61 Рече и друг: Ще дойда след Тебе, Господи; но първо ми позволи да се сбогувам с домашните си.
\par 62 А Исус му каза: Никой, който е турил ръката си на ралото и гледа назад, не е годен за Божието царство.

\chapter{10}

\par 1 След това Господ определи други седемдесет души, и ги разпрати по двама пред Себе Си във всеки град и място, гдето сам Той щеше да отиде.
\par 2 И каза им: Жетвата е изобилна, а работниците малко; затова молете се на Господаря на жетвата да изпрати работници в жетвата Си.
\par 3 Идете: Ето, Аз ви изпращам като агнета посред вълци.
\par 4 Не носете ни кесия, ни торба, ни обуща, и никого по пътя не поздравявайте.
\par 5 И в която къща влезете, първо казвайте: Мир на тоя дом!
\par 6 И ако бъде там някой син на мира, вашият мир ще почива на него; но ако няма, ще се върне на вас.
\par 7 И в същата къща седете, и яжте, и пийте каквото ви сложат; защото работникът заслужава своята заплата. Недейте се премества от къща в къща.
\par 8 И като влизате в някой град, и те ви приемат, яжте каквото ви сложат,
\par 9 и изцелявайте болните в него, и казвайте им: Божието царство е наближило до вас.
\par 10 А като влезете в някой град, и те не ви приемат, излезте на улиците му и речете:
\par 11 И праха, който е полепнал по нозете ни от вашия град, ви отърсваме; но това да знаете, че Божието царство е наближило.
\par 12 Казвам ви: По-леко ще бъде наказанието на Содом в оня ден, отколкото на тоя град.
\par 13 Горко ти Хоразине! Горко ти Витсаидо! Защото, ако бяха се извършили в Тир и Сидон великите дела, които се извършиха у вас, отдавна те биха се покаяли, седящи във вретище и пепел.
\par 14 Обаче на Тир и Сидон ще бъде по-леко в съда, отколкото на вас.
\par 15 И ти, Капернауме, до небесата ли ще се издигнеш? До ада ще се смъкнеш.
\par 16 Който слуша вас, Мене слуша; и който отхвърля вас, Мене отхвърля; а който отхвърля Мене, отхвърля Онзи, Който Ме е пратил.
\par 17 И седемдесетте се върнаха с радост, и казаха: Господи, в Твоето име и бесовете се покоряват на нас.
\par 18 А Той им рече: Видях сатана, паднал от небето като светкавица.
\par 19 Ето, давам ви власт да настъпвате на змии, и на скорпии, и власт над цялата сила на врага; и нищо няма да ви повреди.
\par 20 Обаче, недейте се радва на това, че духовете ви се покоряват; а радвайте се, че имената ви са написани на небесата.
\par 21 В същия час Исус се зарадва чрез Святия Дух и каза: Благодаря Ти, Отче, Господи на небето и земята, задето си утаил това от мъдрите и разумните, а си го открил на младенците. Да, Отче, защото така Ти се видя угодно.
\par 22 Всичко Ми е предадено от Отца Ми; и освен Отец, никой не знае Кой е Синът; и никой не знае Кой е Отец, освен Синът и оня, комуто Синът би благоволил да Го открие.
\par 23 И като се обърна към учениците, рече частно: Блажени очите, които виждат това, което вие виждате.
\par 24 Защото ви казвам, че много пророци и царе пожелаха да видят това, което вие виждате, и не видяха, и да чуят това, което вие чувате, и не чуха.
\par 25 И, ето, някой законник стана и Го изпитваше, казвайки: Учителю, какво да правя, за да наследя вечен живот?
\par 26 А Той му рече: Какво е писано в закона, как четеш?
\par 27 А той в отговор каза: "Да възлюбиш Господа, твоя Бог, с цялото си сърце, с цялата си душа, с всичката си сила, и с всичкия си ум, и ближния си както себе си".
\par 28 Исус му рече: Право си отговорил; това стори, и ще живееш.
\par 29 Но той, понеже искаше да оправдае себе си, рече на Исуса: А кой е моят ближен?
\par 30 В отговор Исус каза: Някой си човек слизаше от Ерусалим в Ерихон; и налетя на разбойници, които го съблякоха и нараниха и отидоха си, като го оставиха полумъртъв.
\par 31 А случайно някой си свещеник слизаше по оня път, и, като го видя, замина си от срещната страна.
\par 32 Също и един левит, като стигна на това място и го видя, замина си от срещната страна.
\par 33 Но един самарянин, като пътуваше дойде на мястото, дето беше той, и като го видя, смили се,
\par 34 приближи се и превърза раните му, като изливаше на тях масло и вино. После го качи на собственото си добиче, закара го в една страноприемница и се погрижи за него.
\par 35 И на следния ден извади два динара та ги даде на съдържателя и рече: Погрижи се за него; и каквото повече иждивиш, на връщане аз ще ти заплатя.
\par 36 Кой от тия трима ти се вижда да се е показал ближен на изпадналия всред разбойниците?
\par 37 Той рече: Онзи, който му показа милост. Исус му каза: Иди и ти прави също така!
\par 38 И като вървяха по пътя, Той влезе в едно село; и някоя си жена на име Марта Го прие у дома си.
\par 39 Тя имаше сестра на име Мария, която седна при нозете на Господа и слушаше словото Му.
\par 40 А Марта като се залисваше с много шетане, пристъпи и рече: Господи, не те ли е грижа, че сестра ми ме остави сама да шетам? Кажи -, прочее, да ми помогне.
\par 41 Но Господ в отговор - рече: Марто, Марто, ти се грижиш и безпокоиш за много неща,
\par 42 но едно е потребно; и Мария избра добрата част, която не ще се отнеме от нея.

\chapter{11}

\par 1 И когато Той се молеше на едно място, като престана, един от Неговите ученици Му рече: Господи, научи ни да се молим както и Иоан е научил своите ученици.
\par 2 А Той им каза: Когато се молите, думайте: Отче [наш, Който си на небесата], да се свети Твоето име; да дойде Твоето царство, [да бъде Твоята воля, както на небето, така и на земята];
\par 3 давай ни всеки ден ежедневния ни хляб;
\par 4 и прости греховете ни, защото и сами ние прощаваме на всеки наш длъжник; и не въвеждай ни в изкушение, [но избави ни от лукавия].
\par 5 И каза им: Ако някой от вас има приятел и отиде при него посред нощ, и му рече: Приятелю, дай ми на заем три хляба,
\par 6 понеже един мой приятел дойде у дома от път, и нямам какво да сложа пред него;
\par 7 и ако той отвътре в отговор рече: Не ме безпокой; вратата е вече заключена, и децата ми са с мене на легло; не мога да стана да ти дам;
\par 8 казвам ви, че даже ако не стане да му даде, защото му е приятел, то поради неговата настойчивост ще стане и ще му даде колкото му трябва.
\par 9 И аз ви казвам: Искайте, и ще ви се даде; търсете, и ще намерите; хлопайте и ще ви се отвори.
\par 10 Защото всеки, който иска, получава; който търси, намира; и на онзи, който хлопа, ще се отвори.
\par 11 И кой е оня баща между вас, който, ако му поиска син му хляб, ще му даде камък? Или, ако му поиска риба, наместо риба, ще му даде змия?
\par 12 Или, ако поиска яйце, ще му даде скорпия?
\par 13 И тъй, ако вие, които сте зли знаете да давате блага на чадата си, колко повече Небесният Отец ще даде Святия Дух на ония, които искат от Него!
\par 14 Еднаж Той изгонваше един ням бяс; и като излезе бясът, немият проговори, и народът се почуди.
\par 15 А някои от тях рекоха: Чрез началника на бесовете, Веелзевул, изгонва бесовете.
\par 16 А други, изпитвайки Го, искаха от Него знамение от небето.
\par 17 Но той, като знаеше техните помисли, каза им: Всяко царство, разделено против себе си, запустява, и дом, разделен против себе си,(Гръцки: Против дома.) пада.
\par 18 Така също, ако Сатана се раздели против себе си, как ще устои царството му? Понеже казвате, че изгонвам бесовете, чрез Веелзевула,
\par 19 и ако Аз чрез Веелзевула изгонвам бесовете, вашите синове чрез кого ги изгонват? Затова, те ще ви бъдат съдии.
\par 20 Но ако Аз с Божия пръст изгонвам бесовете, то Божието царство е достигнало до вас.
\par 21 Когато силният въоръжен човек пази двора си, имотът му е в безопасност.
\par 22 Но когато един по-силен от него нападне и му надвие, взема му всичкото оръжие, на което се е надявал, и разподеля каквото грабне от него.
\par 23 Който не е с Мене, той е против Мене; и който не събира заедно с Мене, той разпилява.
\par 24 Когато нечистият дух излезе от човека, той минава през безводни места и търси спокойствие; и като не намери, казва: Ще се върна в къщата си, отдето съм излязъл.
\par 25 И като дойде, намира я пометена и наредена.
\par 26 Тогава отива и взема със себе си седем други духове по-зли от него, и като влязат, живеят там; и последното състояние на оня човек става по-лошо от първото.
\par 27 Когато говореше това, една жена от множеството със силен глас Му рече: Блажена утробата, която Те е носила, и съсците, които си сукал.
\par 28 А Той рече: По-добре кажи: Блажени ония, които слушат Божието слово и го пазят.
\par 29 А когато народът се събираше около Него, почна да казва: Това поколение е нечестиво поколение; иска знамение, но друго знамение няма да му се даде, освен знамението на [пророка] Иона.
\par 30 Защото както Иона стана знамение на ниневийците, така и Човешкият Син ще бъде на това поколение.
\par 31 Южната царица ще се яви на съда с човеците от това поколение и ще ги осъди, защото дойде от краищата на земята да чуе Соломоновата мъдрост; а, ето, тук има повече от Соломона.
\par 32 Ниневийските мъже ще се явят на съда с това поколение и ще го осъдят, защото се покаяха чрез Ионовата проповед; а, ето, тук има повече от Иона.
\par 33 Никой, като запали светило, не го туря в зимника, нито под шиника, но на светилника, за да виждат светлината ония, които влизат.
\par 34 Светило на тялото ти е твоето око; когато окото ти е здраво, то и цялото ти тяло е осветлено; а когато е болнаво, и тялото ти е в мрак.
\par 35 Затова, внимавай, да не би светлината в тебе да е тъмнина.
\par 36 Ако, прочее, цялото твое тяло бъде осветено, без да има тъмна част, то цяло ще бъде осветено, както кога светилото се осветява със сиянието си.
\par 37 Като говореше, един фарисей Го покани да обядва у него; и Той влезе и седна на трапезата.
\par 38 Фарисеят се почуди, като видя, че Той не се оми първо преди обеда.
\par 39 И Господ му каза: Сега вие фарисеите чистите външното на чашата и на паницата; а вашето вътрешно е пълно с грабеж и нечестие.
\par 40 Несмислени! Тоя, който е направил външното, не е ли направил и вътрешното?
\par 41 По-добре чистете вътрешното; и ето, всичко ще ви бъде чисто.
\par 42 Но горко на вас фарисеи, защото давате десетък от гьозума, от седефчето и от всякакъв зеленчук, и пренебрегвате правосъдието и Божията любов. Но тия трябваше да правите, а ония да не пренебрегвате.
\par 43 Горко вам фарисеи! Защото обичате първите столове в синагогите и поздравите по пазарите.
\par 44 Горко вам! Защото сте като гробове, които не личат, тъй щото човеците, които ходят по тях, не знаят.
\par 45 И един от законниците в отговор Му рече: Учителю, като казваш това, и нас кориш.
\par 46 А Той каза: Горко и на вас, законниците, защото товарите човеците с бремена, които тежко се носят; а сами вие нито с един пръст не се допирате до бремената.
\par 47 Горко вам! Защото градите гробниците на пророците, а бащите ви ги избиха.
\par 48 И тъй, вие свидетелствувате за делата на бащите си и се съгласявате с тях; защото те ги избиха, а вие им градите гробниците.
\par 49 Затова и Божията премъдрост рече: Ще им пращам пророци и апостоли и едни от тях ще убият и изгонят;
\par 50 за да се изиска от това поколение кръвта на всичките пророци, която е проливана от създаването на света, -
\par 51 от кръвта на Авела до кръвта на Захария, който загина между олтара и светилището. Да! Казвам ви, ще се изиска от това поколение.
\par 52 Горко но вас, законници! Защото отнехте ключа на знанието; сами вие не влязохте, и на влизащите попречихте.
\par 53 И като излезе оттам, книжниците и фарисеите почнаха яростно да Го преследват и да Го предизвикват да говори за още много неща,
\par 54 като Го дебнеха, за да уловят нещо от думите Му.

\chapter{12}

\par 1 Между това, като се събра едно многохилядно множество, дотолкова, че един други се тъпчеха, Той почна да говори на учениците Си: Преди всичко пазете се от фарисейския квас, който е лицемерие.
\par 2 Няма нищо покрито, което не ще се открие, и тайно, което не ще се узнае.
\par 3 Затова, каквото сте говорили на тъмно, ще се чуе на видело; и каквото сте казали на ухо във вътрешните стаи, ще се разгласи от покрива.
\par 4 А на вас, Моите приятели, казвам: Не бойте се от тия, които убиват тялото, и след това не могат нищо повече да сторят.
\par 5 Но ще ви предупредя от кого да се боите: Бойте се от онзи, който, след като е убил, има власт да хвърля в пъкъла. Да! Казвам ви, от него да се боите.
\par 6 Не продават ли се пет врабчета за два асария? И ни едно от тях не е забравено пред Бога.
\par 7 Но вам и космите на главата са всички преброени. Не бойте се; вие сте много по-скъпи от врабчетата.
\par 8 И казвам ви: Всеки, който изповяда Мене пред човеците, ще го изповяда и Човешкият Син пред Божиите ангели;
\par 9 но ако се отрече някой от Мене пред човеците, ще бъде отречен пред Божиите ангели.
\par 10 И всекиму, който би казал дума против Човешкия Син, ще му се прости; но ако някой похули Святия Дух, няма да му се прости.
\par 11 И когато ви заведат в синагогите и пред началствата и властите, не се безпокойте как или какво ще отговорите, или какво ще кажете.
\par 12 Защото Святият Дух ще ви научи в същия час, какво трябва да кажете.
\par 13 И някой си от множеството Му рече: Учителю, кажи на брат ми да раздели с мене наследството.
\par 14 А Той му каза: Човече, кой Ме е поставил съдия или делач над вас?
\par 15 И каза им: Внимавайте и пазете се от всяко користолюбие; защото животът на човека не се състои в изобилието на имота му.
\par 16 И каза им притча, като рече: Нивите на един богаташ родиха много плод.
\par 17 И той размишляваше в себе си, думайки: Какво да правя, защото нямам где да събера плодовете си.
\par 18 И рече: Ето какво ще направя: Ще съборя житниците си, и ще построя по-големи, и там ще събера всичките си жита и благата си.
\par 19 И ще река на душата си: Душо, имаш много блага, натрупани за много години; успокой се, яж, пий, весели се.
\par 20 А Бог му рече: Глупецо! Тая нощ ще ти изискат душата; а това, което си приготвил, чие ще бъде?
\par 21 Така става с този, който събира имот за себе си, и не богатее в Бога.
\par 22 Рече още на учениците Си: Затова ви казвам, не се безпокойте за живота си, какво ще ядете, нито за тялото си, какво ще облечете.
\par 23 Защото животът е повече от храната, и тялото - от облеклото.
\par 24 Разгледайте враните, че не сеят, нито жънат; те нямат нито скривалище, нито житница, но пак Бог ги храни. Колко по-скъпи сте вие от птиците!
\par 25 И кой от вас може с грижене да прибави един лакът на ръста си?
\par 26 И тъй, ако и най-малкото нещо не можете стори, защо се безпокоите за друго?
\par 27 Разгледайте кремовете как растат; не се трудят, нито предат; но казвам ви, нито Соломон във всичката си слава не се е обличал както един от тях.
\par 28 И ако Бог така облича полската трева, която днес я има, а утре я хвърлят в пещ, колко повече ще облича вас, маловери!
\par 29 И тъй, не търсете какво да ядете и какво да пиете, и не се съмнявайте;
\par 30 защото всичко това търсят народите на света; а Отец ви знае, че се нуждаете от това.
\par 31 Но търсете [Божието] царство и [всичко] това ще ви се прибави.
\par 32 Не бой се, малко стадо, защото Отец ви благоволи да ви даде царството.
\par 33 Продайте имота си и давайте милостиня; направете си кесии, които не овехтяват, неизчерпаемо съкровище на небесата, гдето крадец не се приближава, нито молец изяжда.
\par 34 Защото гдето е съкровището ви, там ще бъде и сърцето ви.
\par 35 Кръстът ви да бъде препасан и светилниците ви запалени;
\par 36 и сами вие да приличате на човеци, които чакат господаря си, когато се върне от сватба, за да му отворят незабавно, щом дойде и похлопа.
\par 37 Блажени ония слуги, чийто господар ги намери будни, когато си дойде; истина ви казвам, че той ще се препаше, ще ги накара да седнат на трапезата и ще дойде да им прислужи.
\par 38 И ако дойде на втора стража, или на трета стража, и ги намери така, блажени са ония слуги.
\par 39 Но това да знаете, че ако домакинът беше знаел в кой час щеше да дойде крадецът, бдял би и не би оставил да му подкопаят къщата.
\par 40 Бъдете, прочее, и вие готови; защото в час, когато Го не мислите, Човешкият Син ще дойде.
\par 41 Тогава Петър каза: Господи, само на нас ли казваш тая притча, или и на всичките?
\par 42 Господ каза: Кой е, прочее, онзи верен и благоразумен настойник, когото господаря му ще постави над домочадието си, да им дава на време определената храна?
\par 43 Блажен онзи слуга, чийто господар, когато си дойде, го намери, че прави така.
\par 44 Истина ви казвам, че ще го постави над целия си имот.
\par 45 Но ако онзи слуга рече в сърцето си: Господарят ми се забави, и почне да бие момчетата и момичетата, да яде, да пие и да се опива,
\par 46 то господарят на онзи слуга ще дойде в ден, когато той не го очаква и в час, който не знае, и, като го бие тежко, ще определи неговата участ с неверните.
\par 47 И онзи слуга, като е знаел волята на господаря си, но не е приготвил, нито постъпил по волята му, ще бъде много бит.
\par 48 А онзи, който не е знаел и е сторил нещо, което заслужава бой, малко ще бъде бит. И от всеки, комуто много е дадено, много и ще се изисква; и комуто са много поверили, от него повече ще изискват.
\par 49 Огън дойдох да хвърля на земята; и какво повече да искам, ако се е вече запалил?
\par 50 Но имам кръщение, с което трябва да се кръстя; и колко се утеснявам докле се извърши!
\par 51 Мислите ли, че съм дошъл да дам мир на земята? Не, казвам ви, но по-скоро раздяла.
\par 52 Защото отсега нататък петима в една къща ще бъдат разделени, трима против двама и двама против трима.
\par 53 Ще се разделят баща против син и син против баща; майка против дъщеря и дъщеря против майка; свекърва против снаха си и снаха против свекърва си.
\par 54 Казваше още на народа: Когато видите облак да се издига от запад, веднага казвате: Дъжд ще вали, и така става.
\par 55 И когато духа южен вятър казвате: Ще стане жега, и става.
\par 56 Лицемери! Лицето на земята и на небето знаете да разтълкувате, а това време как не знаете да разтълкувате?
\par 57 А защо и от само себе си не съдите що е право?
\par 58 Защото, когато отиваш с противника си да се я явиш пред управителя, постарай се по пътя да се отървеш от него, да не би да те завлече при съдията и съдията да те предаде на служителя, и служителят да те хвърли в тъмница.
\par 59 Казвам ти, никак няма да излезеш от там, докле не изплатиш и най-последното петаче.

\chapter{13}

\par 1 В същото време присъствуваха някои, които известиха на Исуса за галилеяните, чиято кръв Пилат смесил с жертвите им.
\par 2 И Той в отговор им рече: Мислите ли, че тия галилеяни са били най-грешни от всичките галилеяни, понеже са пострадали така?
\par 3 Казвам ви, не; но ако се не покаете, всички така ще загинете.
\par 4 Или мислите ли, че ония осемнадесет души, върху които падна Силоамската кула и ги уби, бяха престъпници повече от всички човеци, които живеят в Ерусалим?
\par 5 Казвам ви, не; но ако не се покаете, всички така ще загинете.
\par 6 Каза и тая притча: Някой си имаше в лозето си посадена смоковница; и дойде да търси плод на нея, но не намери.
\par 7 И рече на лозаря: Ето, три години как дохождам да търся плод на тая смоковница, но не намирам; отсечи я; защо да запразня земята?
\par 8 А той в отговор му рече: Господарю, остави я и това лято, докле разкопая около нея и насипя тор;
\par 9 и ако подир това даде плод, добре, но ако не, ще я отсечеш.
\par 10 И една събота Той поучаваше в една от синагогите;
\par 11 и ето една жена, която имаше дух, който - беше причинявал немощи за осемнадесет години; тя беше сгърбена и не можеше никак да се изправи.
\par 12 А Исус, като я видя, повика я и рече -: Жено, освободена си от немощта си.
\par 13 И положи ръце на нея; и на часа тя се изправи и славеше Бога.
\par 14 А началникът на синагогата, като негодуваше за дето Исус в събота я изцели, проговаряйки рече на народа: Има шест дни, в които трябва да се работи; в тях, прочее, дохождайте и целете се, а не в съботен ден.
\par 15 Но Господ в отговор му рече: Лицемери! В събота не отвързва ли всеки един от вас вола или осеела си от яслите и го завежда да го напоява?
\par 16 А тая, като е Авраамова дъщеря, която Сатана е държал цели осемнадесет години, не трябваше ли да бъде развързана от тая връзка в съботен ден?
\par 17 И като каза това, всичките Му противници се посрамиха, и целият народ се радваше за всичките славни дела, които се вършеха от Него.
\par 18 Каза прочее: На какво прилича Божието царство? И на що да го уподобя?
\par 19 Прилича на синапово зърно, което човек взе и пося в градината си; и то растеше и стана дърво, и небесните птици се подслоняваха по клончетата му.
\par 20 И пак каза: На какво да уподобя Божието царство?
\par 21 Прилича на квас, който една жена взе и замеси в три мери брашно, докле вкисна всичкото.
\par 22 И по пътя Си за Ерусалим, минаваше през градовете и през селата и поучаваше.
\par 23 И някой си Му рече: Господи, малцина ли са, които се спасяват? А Той им каза:
\par 24 Подвизавайте се да влезете през тесните врата; защото ви казвам, мнозина ще се стараят да влязат, и не ще могат,
\par 25 след като стане домакинът и затвори вратата, и вие, като останете вън, почнете да хлопате на вратата и да казвате: Господи отвори; а Той в отговор ви каже: Не ви зная откъде сте.
\par 26 Тогава ще почнете да казвате: Ядохме и пихме пред Тебе; и в нашите улици Си поучавал.
\par 27 А Той рече: Казвам ви, не зная откъде сте; махнете се от Мене всички вие, които вършите неправда.
\par 28 Там ще бъде плач и скърцане със зъби, когато видите Авраама, Исаака, Якова и всички пророци в Божието царство, а себе си, изпъдени вън.
\par 29 И ще дойдат от изток и запад, от север и юг, и ще седнат в Божието царство.
\par 30 И, ето, има последни, които ще бъдат първи, и има първи, които ще бъдат последни.
\par 31 В същия час дойдоха някои фарисеи, които Му казаха: Излез и иди Си оттук, защото Ирод иска да Те убие.
\par 32 И рече им: Идете, кажете на тая лисица: Ето, изгонвам бесове, и изцелявам днес и утре, и в третия ден свършвам.
\par 33 Обаче трябва днес и утре, и други ден да пътувам; защото не е възможно пророк да загине вън от Ерусалим.
\par 34 Ерусалиме! Ерусалиме! Ти, който избиваш пророците, и с камъни убиваш пратените до тебе, колко пъти съм искал да събера твоите чада, както кокошка прибира пилците си, под крилете си, но не искахте!
\par 35 Ето, оставя се вам дома ви пуст; и казвам ви, няма да Ме видите до когато кажете: Благословен, Който иде в Господното име.

\chapter{14}

\par 1 Една събота, когато влезе да яде хляб в къщата на един от фарисейските началници, те Го наблюдаваха.
\par 2 И, ето, пред Него имаше някой красничав човек.
\par 3 И Исус продума на законниците и фарисеите, и рече: Позволено ли е да лекува някой в събота, или не?
\par 4 А те мълчаха. И Той като хвана човека, изцели го и го пусна.
\par 5 И рече им: Ако паднеше оселът или волът на някого от вас в кладенец, не щеше ли той начаса да го извлече в съботен ден?
\par 6 И не можаха да отговорят на това.
\par 7 И като забелязваше как поканените избираха първите столове, каза им притча, думайки:
\par 8 Когато те покани някой на сватба, не сядай на първия стол, да не би да е бил поканен от него по-почетен от тебе,
\par 9 и дойде този, който е поканил и тебе и него, и ти рече: Отстъпи това място на този човек; и тогава ще почнеш със срам да заемаш последното място.
\par 10 Но когато те поканят, иди и седни на последното място, и когато дойде този, който те е поканил, да ти рече: Приятелю, мини по-горе. Тогава ще имаш почит пред всички тия, които сядат с тебе на трапезата.
\par 11 Защото всеки, който възвишава себе си, ще се смири, а който смирява себе си, ще се възвиси.
\par 12 Каза и на този, който го беше поканил: Когато даваш обед или вечеря, недей кани приятелите си, ни братята си, ни роднините си, нито богати съседи, да не би и те да те поканят, и ти бъде отплатено.
\par 13 Но когато даваш угощение, поканвай сиромаси, недъгави, куци, слепи;
\par 14 и ще бъдеш блажен, защото, понеже те нямат с какво да ти отплатят, ще ти бъде отплатено във възкресението на праведните.
\par 15 И като чу това един от седящите с Него, рече му: Блажен е онзи, който ще яде хляб в Божието царство.
\par 16 А Той му рече: Някой си човек даде голяма вечеря и покани мнозина.
\par 17 И в часа на вечерята изпрати слугата си да рече на поканените: Дойдете, понеже всичко е вече готово.
\par 18 А те всички почнаха единодушно да се извиняват. Първият му рече: Купих си нива и трябва да изляза да я видя; моля ти се, извини ме.
\par 19 Друг рече: Купих си пет чифта волове, и отивам да ги опитам; моля ти се извини ме.
\par 20 А друг рече: Ожених се, и затова не мога да дойда.
\par 21 И слугата дойде и каза това на господаря си. Тогава стопанинът, разгневен рече на слугата си: Излез скоро на градските улици и пътеки, и доведи тука сиромасите, недъгавите, слепите и куците.
\par 22 И слугата рече: Господарю, каквото си заповядал стана, и още място има.
\par 23 И господарят рече на слугата: Излез по пътищата и по оградите, и колкото намериш, накарай ги да влязат, за да се напълни къщата ми;
\par 24 защото ви казвам, че никой от поканените няма да вкуси от вечерята ми.
\par 25 А големи множества вървяха заедно с Него; и Той се обърна и им рече:
\par 26 Ако дойде някой при Мене, и не намрази баща си и майка си, жена си, чадата си, братята си, и сестрите си, а още и собствения си живот, не може да бъде Мой ученик.
\par 27 Който не носи своя кръст и не върви след Мене, не може да бъде Мой ученик.
\par 28 Защото кой от вас, когато иска да съгради кула, не сяда първо да пресметне разноските, дали ще има с какво да я доизкара?
\par 29 Да не би, като положи основа, а не може да доизкара, всички които гледат, да почнат да му се присмиват и да казват:
\par 30 Този човек почна да гради, но не можа да доизкара.
\par 31 Или кой цар, като отива на война срещу друг цар, не ще седне първо да се съветва, може ли с десет хиляди да стои против този, който иде срещу него с двадесет хиляди?
\par 32 Иначе, докато другият е още далеч, изпраща посланици да искат условия за мир.
\par 33 И тъй, ако някой от вас не се отрече от всичко що има, не може да бъде Мой ученик.
\par 34 Прочее, добро нещо е солта, но ако самата сол обезсолее, с какво ще се поправи?
\par 35 Тя не струва нито за земята, нито за тор; изхвърлят я вън. Който има уши да слуша, нека слуша!

\chapter{15}

\par 1 А всичките бирници и грешници се приближаваха при Него да Го слушат.
\par 2 А фарисеите и книжниците роптаеха, казвайки: Тоя приема грешниците и яде с тях.
\par 3 И Той им изговори тая притча, като каза:
\par 4 Кой от вас, ако има сто овце, и му се изгуби една от тях, не оставя деветдесетте и девет в пустинята, и не отива след изгубената докле я намери?
\par 5 И като я намери, вдига я на рамената си радостен.
\par 6 И като си дойде у дома, свиква приятелите си и съседите си и им казва: Радвайте се с мене, че си намерих изгубената овца.
\par 7 Казвам ви, че също така ще има повече радост на небето за един грешник, който се кае, нежели за деветдесет и девет праведници, които нямат нужда от покаяние.
\par 8 Или коя жена, ако има десет драхми, и изгуби една драхма, не запаля светило, не помита къщата, и не търси грижливо докле я намери?
\par 9 И като я намери, свиква приятелките и съседките си и казва: Радвайте се с мене, защото намерих драхмата, която бях изгубила.
\par 10 Също така, казвам ви, има радост пред Божите ангели за един грешник, който се кае.
\par 11 Каза още: Някой си човек имаше двама сина.
\par 12 И по-младият от тях рече на баща си: Тате, дай ми дела, който ми се пада от имота. И той им раздели имота.
\par 13 И не след много дни по-младият син си събра всичко и отиде в далечна страна, и там разпиля имота си с разпуснатия си живот.
\par 14 И като иждиви всичко, настана голям глад в оная страна; и той изпадна в лишение.
\par 15 И отиде да се пристави на един от гражданите на оная страна, който го прати на полетата си да пасе свини.
\par 16 И желаеше да се насити с рошковите, от които ядяха свините; но никой не му даваше.
\par 17 А като дойде на себе си, рече: Колко наемници на баща ми имат излишьк от хляб, а пък аз умирам от глад!
\par 18 Ще стана да ида при баща си и ще му река: Тате, съгреших против небето и пред тебе;
\par 19 не съм вече достоен да се наричам твой син; направи ме като един от наемниците си.
\par 20 И стана и дойде при баща си. А когато бе още далеч, видя го баща му, смили се, и като се завтече, хвърли се на врата му и го целуваше.
\par 21 Рече му синът: Тате, съгреших против небето и пред тебе; не съм достоен да се наричам твой син.
\par 22 Но бащата рече на слугите си: Скоро изнесете най-хубавата премяна и облечете го, и турете пръстен на ръката му и обуща на нозете му;
\par 23 докарайте угоеното теле и го заколете и нека ядем и се веселим;
\par 24 защото този мой син бе мъртъв, и оживя, изгубен бе, и се намери. И почнаха да се веселят.
\par 25 А по-старият му син беше на нивата; и като си идеше и се приближи до къщата, чу песни и игри.
\par 26 И повика един от слугите и попита, що е това.
\par 27 А той му рече: Брат ти си дойде и баща ти закла угоеното теле, защото го прие здрав.
\par 28 И той се разсърди и не искаше да влезе и баща му излезе и го молеше.
\par 29 А той в отговор рече на баща си: Ето, толкова години ти работя, и никога не съм престъпил някоя твоя заповед; но пак на мене нито яре не си дал някога да се повеселя с приятелите си;
\par 30 а щом си дойде този твой син, който изпояде имота ти с блудниците, за него си заклал угоеното теле.
\par 31 А той му каза: Синко ти си винаги с мене, и всичко мое твое е.
\par 32 Но прилично беше да се развеселим и да се зарадваме; защото този твой брат бе мъртъв, и оживя, и изгубен бе, и се намери.

\chapter{16}

\par 1 Каза още на учениците Си: Някой си богаташ имаше настойник, когото наклеветиха пред него, че разпилявал имота му.
\par 2 И той го повика и му рече: Какво е това що слушам за тебе? Дай сметка за настойничеството си; защото не можеш вече да бъдеш настойник.
\par 3 Тогава настойникът си рече: Що да сторя, тъй като господарят ми отнема от мене настойничеството? Нямам сила да копая, да прося срам ме е.
\par 4 Сетих се що да сторя, за да ме приемат в къщите си, когато бъда отстранен от настойничеството.
\par 5 И тъй, като повика всеки от длъжниците на господаря си, каза на първия: Колко дължиш на господаря ми?
\par 6 А той рече: Сто мери масло. И каза му: Вземи записа си и седни скоро та пиши петдесет.
\par 7 После каза на друг: А ти колко дължиш? И той рече: Сто мери жито. Казва му: Вземи записа си и пиши осемдесет.
\par 8 И господарят му похвали неверния настойник за гдето остроумно постъпил; защото човеците на тоя век са по-остроумни спрямо своето поколение от просветените чрез виделината.
\par 9 И Аз ви казвам, спечелете си приятели посредством неправедното богатство, та, когато се привърши да ви приемат във вечните жилища.
\par 10 Верният в най-малкото и в многото е верен, а неверният в най-малкото и в многото е неверен.
\par 11 И тъй, ако в неправедното богатство не бяхте верни, кой ще ви повери истинското богатство?
\par 12 И ако в чуждото не бяхте верни, кой ще ви даде вашето?
\par 13 Никой служител не може да служи на двама господари; защото или ще намрази единия и другия ще обикне, или ще се привърже към единия, а другия ще презира. Не можете да служите на Бога и на мамона.
\par 14 Всичко това слушаха фарисеите, които бяха сребролюбци, и Му се присмиваха.
\par 15 И рече им: Вие сте, които се показвате праведни пред човеците; но Бог знае сърцата ви; защото онова, което се цени високо между човеците, е мерзост пред Бога.
\par 16 Законът и пророците бяха до Йоана; оттогава Божието царство се благовествува, и всеки на сила влиза в него.
\par 17 Но по-лесно е небето и земята да преминат, отколкото една точка от закона да падне.
\par 18 Всеки, който напусне жена си, и се ожени за друга, прелюбодействува; и който се ожени за напусната от мъж, той прелюбодействува.
\par 19 Имаше някой си богаташ, който се обличаше в мораво и висон, и всеки ден се веселеше бляскаво.
\par 20 Имаше и един сиромах, на име Лазар, покрит със струпеи, когото туряха да лежи пред портата му,
\par 21 като желаеше да се нахрани от падналото от трапезата на богаташа; и кучетата дохождаха та лижеха раните му.
\par 22 Умря сиромахът; и ангелите го занесоха в Авраамовото лоно. Умря и богаташа и бе погребан.
\par 23 И в пъкъла, като беше на мъки и подигна очи, видя отдалеч Авраама и Лазаря в неговите обятия.
\par 24 И той извика, казвайки: Отче Аврааме, смили се за мене, и изпрати Лазара да натопи края на пръста си във вода и да разхлади езика ми; защото съм на мъки в тоя пламък.
\par 25 Но Авраам рече: Синко, спомни си, че ти си получил своите блага приживе, така и Лазар злините; но сега той тук се утешава, а ти се мъчиш.
\par 26 И освен всичко това, между нас и вас е утвърдена голяма бездна, така че ония, които биха искали да минат оттук към вас, да не могат, нито пък оттам да преминат към нас.
\par 27 А той рече: Като е тъй, моля ти се, отче, да го пратиш в бащиния ми дом;
\par 28 защото имам петима братя, за да им засвидетелствува, да не би да дойдат и те на това мъчително място.
\par 29 Но Авраам каза: Имат Мойсея и пророците; нека слушат тях.
\par 30 А той рече: Не, отче Аврааме, но ако отиде при тях някой от мъртвите, ще се покаят.
\par 31 И той му каза: Ако не слушат Моисея и пророците, то и от мъртвите да възкръсне някой, пак няма да се убедят.

\chapter{17}

\par 1 И рече на учениците Си: Не е възможно да не дойдат съблазните; но горко на онзи, чрез когото дохождат!
\par 2 По-добре би било за него да се окачи един голям воденичен камък на врата му и да бъде хвърлен в морето, а не да съблазни един от тия малките.
\par 3 Внимавайте на себе си. Ако прегреши брат ти, смъмри го; и ако се покае прости му.
\par 4 И седем пъти на ден ако ти сгреши, и седем пъти се обърне към тебе и каже: Покайвам се, прощавай му.
\par 5 И апостолите рекоха на Господа: Придай ни вяра.
\par 6 А Господ рече: Ако имате вяра колкото синапово зърно, казали бихте на тая черница: Изкорени се и насади се в морето, и тя би ви послушала.
\par 7 А кой от вас, ако има слуга да му оре или да му пасе, ще му рече веднага, щом си дойде от нива: Влез да ядеш?
\par 8 Напротив, не ще ли му рече: Приготви нещо да вечерям, стегни се та ми пошетай, докато ям и пия, и подир това ти ще ядеш и пиеш?
\par 9 Нима ще благодари на слугата за дето е извършил каквото е било заповядано? [Не вярвам].
\par 10 Също така и вие, когато извършите все що ви е заповядано, казвайте: Ние сме безполезни слуги; извършихме само това, което бяхме длъжни да извършим.
\par 11 И в пътуването Си към Ерусалим Той минаваше границата между Самария и Галилея.
\par 12 И като влизаше в едно село, срещнаха Го десетина прокажени, които, като се спряха отдалеч,
\par 13 извикаха със силен глас, казвайки: Исусе наставниче, смили се за нас!
\par 14 И като ги видя, рече им: Идете покажете се на свещениците. И като отиваха, очистиха се.
\par 15 И един от тях, като видя, че изцеля, върна се и със силен глас славеше Бога,
\par 16 и падна на лице при нозете на Исуса и му благодареше. И той бе самарянин.
\par 17 А Исус в отговор му рече: Нали се очистиха десетимата? А где са деветимата?
\par 18 Не намериха ли се други да се върнат и въздадат слава на Бога, освен тоя иноплеменник?
\par 19 И рече му: Стани и си иди; твоята вяра те изцели.
\par 20 А Исус, попитан от фарисеите, кога ще дойде Божието царство, в отговор им каза: Божието царство не иде така щото да се забелязва:
\par 21 нито ще рекат: Ето тук е! Или: Там е! Защото, ето Божието царство е всред вас.
\par 22 И рече на учениците: Ще дойдат дни, когато ще пожелаете да видите поне един от дните на Човешкия Син, и няма да видите.
\par 23 И като ви рекат: Ето, тук е! Да не отидете, нито да тичате подрие им.
\par 24 Защото, както светкавицата, когато блесне от единия край на хоризонта, свети до другия край на хоризонта, така ще бъде и Човешкият Син в Своя ден.
\par 25 Но първо Той трябва да пострада много и да бъде отхвърлен от това поколение.
\par 26 И както стана в Ноевите дни, така ще бъде и в дните на Човешкия Син;
\par 27 ядяха, пиеха, женеха се и се омъжваха до деня, когато Ное влезе в ковчега, и дойде потопът ги погуби всички.
\par 28 Така също, както стана в Лотовите дни; ядяха, пиеха, купуваха, продаваха, садяха и градяха,
\par 29 а в деня, когато Лот излезе от Содом, огън и сяра наваляха от небето и ги погубиха всички, -
\par 30 подобно на това ще бъде и в деня, когато Човешкият Син ще се яви.
\par 31 В оня ден, който се намери на къщния покрив, ако вещите му са в къщи, да не слиза да ги вземе; също и който е на нива, да не се връща назад.
\par 32 Помнете Лотовата жена!
\par 33 Който иска да спечели живота(Или: Душата.) си, ще го изгуби; а който го изгуби, ще го опази.
\par 34 Казвам ви, в онази нощ двама ще бъдат на едно легло; единият ще се вземе, а другият ще се остави.
\par 35 Две жени ще мелят заедно; едната ще се вземе, а другата ще се остави.
\par 36 Двама ще бъдат на нива; единият ще се вземе, а другият ще се остави].
\par 37 Отговарят му, казвайки: Къде, Господи? А Той им рече: Гдето е трупът, там ще се съберат и орлите.

\chapter{18}

\par 1 Каза им една притча за как трябва всякога да се молят, да не отслабват, думайки:
\par 2 В някой си град имаше един съдия, който от Бога се не боеше и човека не зачиташе.
\par 3 В същия град имаше и една вдовица, която дохождаше при него и му казваше: Отдай ми правото спрямо противника ми.
\par 4 Но той за известно време не искаше. А после си каза: При все, че от Бога не се боя и човеците не зачитам,
\par 5 пак, понеже тая вдовица ми досажда, ще - отдам правото, да не би да ме измори с безкрайното си дохождане.
\par 6 И Господ рече: Слушайте що каза неправедният съдия!
\par 7 А Бог няма ли да отдаде правото на Своите избрани, които викат към Него ден и нощ, ако и да се бави спрямо тях?
\par 8 Казвам ви, че ще им отдаде правото скоро. Обаче, когато дойде Човешкият Син ще намери ли вяра на земята?
\par 9 И на някои, които уповаваха на себе си, че са праведни, и презираха другите, каза и тая притча:
\par 10 Двама души влязоха в храма да се помолят, единият фарисей, а другият бирник.
\par 11 Фарисеят, като се изправи, молеше се в себе си така: Боже, благодаря Ти, че не съм като другите човеци, грабители, неправедни, прелюбодейци и особено не като тоя бирник.
\par 12 Постя дваж в седмицата, давам десятък от всичко що придобия.
\par 13 А бирникът като стоеше издалеч, не щеше нито очите си да подигне към небето, но удряше се в гърди и казваше: Боже бъди милостив към мене грешника.
\par 14 Казвам ви, че този слезе у дома си оправдан, а не онзи; защото всеки, който възвишава себе си, ще се смири, а който смирява себе си, ще се възвиси.
\par 15 Донесоха още при Него младенците си, за да се докосне до тях; а учениците, като видяха, смъмриха ги.
\par 16 Но Исус ги повика и рече: Оставете дечицата да дойдат при Мене и не ги възпирайте; защото на такива е Божието царство.
\par 17 Истина ви казвам: Който не приеме като детенце Божието царство, той никак няма да влезе в него.
\par 18 И някой си началник Го попита, казвайки: Благи Учителю, какво да сторя, за да наследя вечен живот?
\par 19 А Исус му рече: Защо Ме наричаш благ? Никой не е благ, освен един Бог.
\par 20 Знаеш заповедите: "Не прелюбодействувай", "Не убивай"; "Не кради"; "Не лъжесвидетелствувай"; "Почитай баща си и майка си";
\par 21 а той каза: Всичко това съм опазил от младостта си.
\par 22 Исус, като го чу, рече му: Едно още ти не достига. Продай все що имаш и раздай на сиромасите и ще имаш съкровище на небето; дойди и Ме следвай!
\par 23 Но той, като чу това, наскърби се много, защото беше твърде богат.
\par 24 И Исус като го видя, каза: Колко мъчно ще влязат в Божието царство ония, които имат богатство!
\par 25 Защото по-лесно е камила да мине през иглени уши, отколкото богат да влезе в Божието царство.
\par 26 А ония, които чуха това рекоха: Тогава кой може да се спаси?
\par 27 А Той каза: Невъзможното за човеците за Бога е възможно.
\par 28 А Петър рече: Ето, ние оставихме своето и Те последвахме.
\par 29 А Той им рече: Истина ви казвам, няма никой, който да е оставил къща или жена, или братя, или родители, или чада, заради Божието царство,
\par 30 който да не получи многократно повече в сегашно време, а в идещия свят вечен живот.
\par 31 И като взе дванадесетте при Себе Си, рече им: Ето, възлизаме в Ерусалим и ще се изпълни в Човешкия Син всичко, що е писано чрез пророците.
\par 32 Защото ще бъде предаден на езичниците, които ще Му се поругаят и безсрамно ще Го оскърбят, и ще Го заплюят,
\par 33 и, когато Го бият, ще Го убият; и на третия ден ще възкръсне.
\par 34 Но те не разбираха нищо от това; и тая дума беше скрита за тях, и не разбираха това, което се казваше.
\par 35 А когато Той се приближаваше до Ерихон, един слепец седеше край пътя да проси.
\par 36 И като чу, че минава народ, попита какво е това.
\par 37 И казаха му, че Исус назарянинът минава.
\par 38 Тогава той извика, казвайки: Исусе, сине Давидов, смили се за мене!
\par 39 А тия, които вървяха отпред го смъмриха, за да млъкне; но той още повече викаше: Сине Давидов, смили се за мене!
\par 40 И тъй, Исус се спря и заповяда да Му го доведат. И като се приближи, попита го:
\par 41 Какво искаш да ти сторя? А той каза: Господи, да прогледам,
\par 42 Исус му рече: Прогледай; твоята вяра те изцели.
\par 43 И той веднага прогледа, и тръгна след Него, като славеше Бога. И всичките люде, като видяха това, въздадоха хвала на Бога.

\chapter{19}

\par 1 След това Исус влезе в Ерихон и минаваше през града.
\par 2 И, ето, един човек, на име Закхей, който беше началник на бирниците, и богат,
\par 3 искаше да види Исуса Кой е, но не можеше поради народа, защото беше малък на ръст.
\par 4 И завтече се напред и се покачи на една черница за да Го види; понеже през оня път щеше да мине.
\par 5 Исус, като дойде на това място, погледна нагоре и му рече: Закхее, слез скоро, защото днес трябва да престоя у дома ти.
\par 6 И той побърза да слезе, и прие Го с радост.
\par 7 И като видяха това, всички роптаеха, казвайки: При грешен човек влезе да преседи.
\par 8 А Закхей стана и рече на Господа: Господи, ето отсега давам половината от имота си на сиромасите; и ако някак съм ограбил някого, връщам му четверократно.
\par 9 И Исус му рече: Днес стана спасение на този дом; защото и този е Авраамов син.
\par 10 Понеже Човешкият Син дойде да потърси и да спаси погиналото.
\par 11 И като слушаха това, Той прибави и каза една притча, защото беше близо до Ерусалим, и те си мислеха, че Божието царство щеше веднага да се яви.
\par 12 Затова каза: Някой си благородник отиде в далечна страна да получи за себе си царска власт, и да се върне.
\par 13 И повика десетима от слугите си и даде им десет мнаси; и рече им: Търгувайте с това, докле дойда.
\par 14 Но неговите граждани го мразеха, и изпратиха след него посланници да кажат: Не щем този да царува над нас.
\par 15 А като получи царската власт и се върна, заповяда да му покажат ония слуги, на който бе дал парите, за да знае какво са припечелили чрез търгуване.
\par 16 Дойде, прочее, първият и рече: Господарю, твоята мнаса спечели десет мнаси.
\par 17 И рече му: Хубаво, добри слуго! Понеже на твърде малкото се показа верен, имай власт над десет града.
\par 18 Дойде и вторият, и рече: Господарю, твоята мнаса принесе пет мнаси.
\par 19 А рече и на него: Бъди и ти над пет града.
\par 20 Дойде и друг и рече: Господарю, ето твоята мнаса, която пазех скътана в кърпа;
\par 21 защото се боях от тебе, понеже си строг човек; задигаш това, което не си положил, и жънеш, което не си посял.
\par 22 Господарят му казва: От устата ти ще съдя, зли слуго. Знаел си, че съм строг човек, който задигам това, което не съм положил, и жъна което не съм сял;
\par 23 тогава защо не вложи парите ми в банката, и аз като си дойдех, щях да ги прибера с лихвата?
\par 24 И рече на предстоящите: Вземете от него мнасата и дайте я на този, който има десетте мнаси,
\par 25 (Рекоха му: Господарю, той има вече десет мнаси!)
\par 26 Казвам ви, че на всеки който има, ще се даде; а от този, който няма, от него ще се отнеме и това, което има.
\par 27 А ония мои неприятели, които не искаха да царувам над тях, доведете ги тука и посечете ги пред мене.
\par 28 И като изрече това, Исус вървеше напред, възлизайки за Ерусалим.
\par 29 И когато се приближи до Витфагия н Витания, до хълма, наречен Елеонски, прати двама от учениците и рече им:
\par 30 Идете в селото, което е насреща ви, в което като влизате ще намерите едно осле, вързано, което никой човек не е възсядал; отвържете го и го докарайте.
\par 31 И ако някой ви попита: Защо го отвързвате? Кажете така: На Господа трябва.
\par 32 И изпратените отидоха и намериха както им беше казал.
\par 33 И като отвързаха ослето, рекоха стопаните му: Защо отвързвате ослето?
\par 34 А те казаха: На Господа трябва.
\par 35 И докараха го при Исуса; и като намятаха дрехите си на ослето, качиха Исуса.
\par 36 И като вървеше Той, людете постилаха дрехите си по пътя.
\par 37 И когато вече се приближаваше до превалата на Елеонския хълм, цялото множество ученици почнаха да се радват и велегласно да славят Бога за всичките велики дела, които бяха видели, казвайки:
\par 38 Благословен Царят, Който иде в Господното име; мир на небето, и слава във висините!
\par 39 А някои от фарисеите между народа Му рекоха: Учителю, смъмри учениците Си.
\par 40 И Той в отговор рече: Казвам ви, че ако тия млъкнат, то камъните ще извикат.
\par 41 И като се приближи и видя града, плака за него и каза:
\par 42 Да беше знаел ти, да! Ти, поне в този [твой] ден, това което служи за мира ти; но сега е скрито от очите ти.
\par 43 Защото ще дойдат върху тебе дни, когато твоите неприятели, ще издигнат окопи около тебе, ще те обсадят, ще те стеснят отвред,
\par 44 и ще те разорят и ще избият жителите ти в тебе и няма да оставят в тебе камък на камък; защото ти не позна времето, когато беше посетен.
\par 45 И като влезе в храма, почна да изпъжда ония, които продаваха; и казваше им:
\par 46 Писано е: "И домът ми ще бъде молитвен дом", а вие го направихте "разбойнически вертеп".
\par 47 И поучаваше всеки ден в храма. А главните свещеници, книжниците и народните първенци се стараеха да Го погубят;
\par 48 но не намериха какво да сторят, понеже всичките люде бяха прилепнали при Него да Го слушат.

\chapter{20}

\par 1 И в един от дните, когато Той поучаваше людете в храма и проповядваше благовестието, надойдоха главните свещеници и книжниците със старейшините и Му рекоха:
\par 2 Кажи ни с каква власт правиш това? Или, кой е онзи, който Ти е дал тази власт.
\par 3 И в отговор им рече: Ще ви задам и Аз един въпрос, и отговорете Ми:
\par 4 Иоановото кръщение от небето ли беше, или от човеците?
\par 5 А те разискваха помежду си, думайки: Ако речем: От небето, ще каже: Защо го не повярвахте?
\par 6 Но ако речем: От човеците, всичките люде ще ни убият с камъни, защото са убедени, че Иоан беше пророк.
\par 7 И отговориха, че не знаят от къде беше.
\par 8 Тогава Исус им рече: Нито Аз ви казвам с каква власт правя това.
\par 9 И почна да говори на людете тая притча: Един човек насади лозе, даде го под наем на земеделци, и отиде в чужбина за дълго време.
\par 10 И във време на беритбата прати един слуга при земеделците, за да му дадат от плода на лозето; но земеделците го биха и отпратиха го празен.
\par 11 И изпрати друг слуга; а те и него биха, срамно го оскърбиха, и го отпратиха празен.
\par 12 Изпрати и трети; но те и него нараниха и изхвърлиха.
\par 13 Тогава стопанинът на лозето рече: Що да сторя? Ще изпратя любезния си син; може него да почетат.
\par 14 Но земеделците, като го видяха, разискваха по между си, като думаха: Това е наследникът; нека го убием, за да стане наследството наше.
\par 15 Изхвърлиха го вън от лозето и го убиха. И тъй, какво ще им стори стопанинът на лозето?
\par 16 Ще дойде и ще погуби тия земеделци, и ще даде лозето на други. А като чуха това рекоха: Дано не бъде!
\par 17 А тоя ги погледна и рече: Тогава що значи това, което е писано: "Камъкът, който отхвърлиха зидарите, той стана глава на ъгъла"?
\par 18 Всеки, който падне върху този камък, ще се смаже, а върху когото падне, ще го пръсне.
\par 19 И в същия час книжниците и главните свещеници се стараеха да турят ръце на Него, защото разбраха, че Той каза тая притча против тях, но се бояха от народа.
\par 20 И като Го наблюдаваха, пратиха издебници, които се преструваха, че са праведни, за да уловят някоя Негова дума, тъй щото да Го предадат на началството и на властта на управителя.
\par 21 И те Го попитаха, казвайки: Учителю, знаем, че право говориш и учиш, и у Тебе няма лицеприятие, но учиш Божия път според истината;
\par 22 право ли е за нас да даваме данък на Кесаря, или не?
\par 23 А Той разбра лукавството им, и рече им:
\par 24 Покажете ми един динарий. Чий образ и надпис има? И [в отговор] казаха: Кесарев.
\par 25 А Той рече: Тогава отдавайте Кесаревото на Кесаря, а Божието на Бога.
\par 26 И не можаха да уловят нещо в думата пред народа; и, зачудени на отговора Му, млъкнаха.
\par 27 Тогава се приближиха някои от садукеите, които твърдят, че няма възкресение, и Го попитаха, казвайки:
\par 28 Учителю, Моисей ни е писал: "Ако умре на някого брат му, който е женен, но е бездетен, брат му да вземе жената и да въздигне потомък на брата си".
\par 29 А имаше седмина братя; и първия взе жена и умря бездетен.
\par 30 И вторият и третият я взеха;
\par 31 така също и седмината я взеха и умряха без да оставят деца.
\par 32 А после умря и жената.
\par 33 И тъй, във възкресението, на кого от тях ще бъде жена? Защото и седмината я имаха за жена.
\par 34 А Исус им рече: Човеците на този свят се женят и се омъжват;
\par 35 а ония, които се удостоят да достигнат онзи свят и възкресението от мъртвите, нито се женят, нито се омъжват.
\par 36 И не могат вече да умрат, понеже са равни на ангелите; и, като участници на възкресението, са чада на Бога.
\par 37 А че мъртвите биват възкресени, това и Моисей показа в мястото, дето писа за къпината, когато нарече Господа "Бог Авраамов, Бог Исааков и Бог Яковов".
\par 38 Но Той не е Бог на мъртвите, а на живите; защото за Него всички са живи.
\par 39 А някои от книжниците в отговор рекоха: Учителю, Ти добре каза.
\par 40 Защото не смееха вече за нищо да Го попитат.
\par 41 И рече им: Как казват, че Христос е Давидов син?
\par 42 Защото сам Давид казва в книгата на псалмите: - Рече Господ на моя Господ: Седи отдясно Ми,
\par 43 докле положа враговете Ти за Твое подножие.
\par 44 И тъй, Давид Го нарича Господ, тогава как е негов син?
\par 45 И когато слушаха всичките люде, Той рече на учениците Си:
\par 46 Пазете се от книжниците, които обичат да ходят пременени, и обичат поздрави по пазарите, първите столове в синагогите, и първите места по угощенията;
\par 47 които изпояждат домовете на вдовиците, и за показ принасят дълги молитви. Те ще получат по-голямо осъждане.

\chapter{21}

\par 1 Като подигна очи, Исус видя богатите, че пускат даровете си в съкровищницата.
\par 2 А видя и една бедна вдовица, че пускаше там две лепти(Две лепти са равни на 2 1/2 стотинки.)
\par 3 и рече: Истина ви казвам, че тая бедна вдовица пусна повече от всички;
\par 4 защото всички тия пуснаха в даровете [за Бога] от излишъка си, а тя от немотията си пусна целия имот, що имаше.
\par 5 И когато някои говореха за храма, че е украсен с хубави камъни и с посветени приноси, рече:
\par 6 Ще дойдат дни, когато от това, което гледате, няма да остане тук камък, който да се не срине.
\par 7 И попитаха Го, казвайки: Учителю, а кога ще бъде това? И какъв ще бъде белегът, когато предстои да стане това?
\par 8 А Той каза: Внимавайте да не ви заблудят; защото мнозина ще дойдат в Мое име и ще казват: Аз съм Христос, и, че времето е наближило. Да не отидете подир тях.
\par 9 И когато чуете за войни и размирици, да не се уплашите; защото тия неща трябва първо да станат, но не е веднага свършекът.
\par 10 Тогава им каза: Народ, ще се повдигне против народ, и царство против царство;
\par 11 и ще има големи трусове, и в разни места глад и мор; ще има и ужаси и големи знамения от небето.
\par 12 А преди да стане всичко това ще турят ръце на вас и ще ви изгонят, като ви предадат на синагоги и в тъмници, и ще ви извеждат пред царе и пред управители поради Моето име.
\par 13 Това ще ви служи за свидетелство.
\par 14 И тъй, решете в сърцата си да не обмисляте предварително що да отговаряте;
\par 15 защото Аз ще ви дам тъй мъдро да отговорите(Гръцки: Уста и мъдрост.), щото всичките ви противници ще бъдат безсилни да ви противостоят или противоречат.
\par 16 И ще бъдете предадени и даже от родители и братя, от роднини и приятели; и ще умъртвят някои от вас.
\par 17 И ще бъдете мразени от всички, поради Моето име.
\par 18 Но и косъм от главата ви няма да загине.
\par 19 Чрез твърдостта си ще придобиете душите си.
\par 20 А когато видите Ерусалим, че е заобиколен от войски, това да знаете, че е наближило запустяването му.
\par 21 Тогава ония, които са в Юдея, нека бягат в планините, а които са всред града, нека излязат вън, а които са в околностите, да не влизат в него.
\par 22 Защото това са дни на въздаяние, за да се изпълни всичко, което е писано.
\par 23 Горко на непразните и на кърмачките, през ония дни! Защото, ще има голямо бедствие в страната, и гняв върху тия люде.
\par 24 Те ще паднат под острието на ножа, и ще бъдат откарани в плен по всичките народи; и Ерусалим ще бъде тъпкан от народите, докле се изпълнят времената на езичниците.
\par 25 И ще станат знамения в слънцето, в луната и в звездите, а по земята бедствие на народите, като ще бъдат в недоумение, поради бучението на морето и вълните.
\par 26 Човеците ще примират от страх и от очакване онова, което ще постигне вселената, защото небесните сили ще са разклатят.
\par 27 И тогава ще видят Човешкия Син, идещ в облак със сила и голяма слава.
\par 28 А когато почне да става това, изправете се и подигнете главите си, защото изкуплението ви наближава.
\par 29 И каза им притча: Погледнете смоковницата и всичките дървета.
\par 30 Когато вече покарат, вие, като видите това, сами знаете, че лятото е вече близо.
\par 31 Също така и вие, когато видите, че става това, да знаете, че е близо Божието царство.
\par 32 Истина ви казвам, че това поколение няма да премине докле не се сбъдне всичко това.
\par 33 Небето и земята ще преминат, но Моите думи няма да преминат.
\par 34 Но внимавайте на себе си, да не би да натегнат сърцата ви от преяждане, пиянство и житейски грижи, и ви постигне оня ден внезапно като примка;
\par 35 защото така ще дойде върху всички, които живеят по лицето на цялата земя.
\par 36 Но бдете всякога, и молете се, за да сполучите да избегнете всичко, което предстои да стане, и да стоите пред Човешкия Син.
\par 37 И всеки ден Той поучаваше в храма; а всяка нощ излизаше и нощуваше на хълма, наречен Елеонски.
\par 38 И всичките люде подраняваха при Него в храма да Го слушат.

\chapter{22}

\par 1 А наближаваше празникът на безквасните хлябове, който се нарича Пасха.
\par 2 И главните свещеници и книжниците обмисляха как да Го умъртвят; защото се бояха от людете.
\par 3 Тогава влезе Сатаната в Юда, наречен Искариот, който беше от числото на дванадесетте;
\par 4 и той отиде и се сговори с главните свещеници и началниците на стражата, как да им Го предаде.
\par 5 И те се зарадваха, и се обещаха да му дадат пари.
\par 6 И той се съгласи, и търсеше сгоден случай да Го предаде, когато би отсъствувало множеството.
\par 7 И настана денят на безквасните хлябове, когато трябваше да жертвуват пасхата.
\par 8 И прати Исус Петра и Иоана, и рече: Идете и ни пригответе, за да ядем пасхата.
\par 9 А те Му казаха: Где искаш да приготвим?
\par 10 А Той им рече: Ето, като влезте в града, ще ви срещне човек, който носи стомна с вода; идете подир него в къщата, в която влезе,
\par 11 и речете на стопанина на къщата: Учителят ти казва: Где е приемната стая, в която ще ям пасхата с учениците Си?
\par 12 И той ще ви посочи голяма горна стая, постлана; там пригответе.
\par 13 И като отидоха, намериха както им беше казал; и приготвиха пасхата.
\par 14 И като дойде часът, Той седна на трапезата, и апостолите с Него.
\par 15 И рече им: Твърде много съм желал да ям тази пасха с вас преди да страдам;
\par 16 защото ви казвам, че няма вече да я ям докле се не изпълни в Божието царство.
\par 17 И като прие чаша, благодари и рече: Вземете това и разделете го помежду си;
\par 18 защото ви казвам, че няма вече да пия от плода на лозата, докато не дойде Божието царство.
\par 19 И взе хляб, и, като благодари, разчупи го, даде им, и рече: Това е Моето тяло, което за вас се дава; това правете за Мое възпоменание.
\par 20 Така взе и чашата подир вечерята, и рече: Тази чаша е новият завет в Моята кръв, която за вас се пролива.
\par 21 Но, ето, ръката на този, който Ме предава, е с Мене на трапезата.
\par 22 Защото Човешкият Син наистина отива, според както е било определено; но горко на този човек, чрез когото се предава!
\par 23 И те почнаха да се питат помежду си, кой ли от тях ще е този, който ще стори това.
\par 24 Стана още и препирня помежду им, кой от тях се счита за по-голям.
\par 25 А Той им рече: Царете на народите господаруват над тях, и тия, които ги владеят се наричат благодетели.
\par 26 Но вие недейте така; а по-големият между вас нека стане като по-младия, и който началствува - като онзи, който слугува.
\par 27 Защото кой е по-голям, този, който седи на трапезата ли, или онзи, който слугува? Не е ли този, който седи на трапезата? Но Аз съм всред вас, като онзи, който слугува.
\par 28 А вие сте ония, които устояхте с Мене в Моите изпитни.
\par 29 Затова, както Моят Отец завещава царство на Мене, а Аз завещавам на вас,
\par 30 да ядете и да пиете на трапезата Ми в Моето царство; и ще седнете на престола да съдите дванадесетте Израилеви племена.
\par 31 [И Рече Господ]: Симоне, Симоне, ето, Сатана ви изиска всички, за да ви пресее като жито;
\par 32 но Аз се молих за тебе, да не отслабне твоята вяра; и ти, когато се обърнеш, утвърди братята си.
\par 33 Петър Му рече: Господи, готов съм да отида с Тебе и в тъмница и на смърт.
\par 34 А Той рече: Казвам ти, Петре, петелът няма да пее днес, докато не си се отрекъл три пъти, че Ме не познаваш.
\par 35 И рече им: Когато ви пратих без кесия, без торба и без обуща, останахте ли лишени от нещо? А те казаха: От нищо.
\par 36 И рече им: Но сега, който има кесия, нека я вземе, така и торба; и който няма кесия нека продаде дрехата си и нека си купи нож;
\par 37 защото ви казвам, че трябва да се изпълни в Мене и това писание: "И към беззаконници биде причислен", защото писаното за Мене наближава към своето изпълнение.
\par 38 И те рекоха: Господи, ето тук има два ножа. А Той им рече: Доволно е.
\par 39 И излезе да отиде по обичая Си на Елеонския хълм; подир Него отидоха и учениците.
\par 40 И като се намери на мястото, рече им: Молете се да не паднете в изкушение.
\par 41 И Той се отдели от тях колкото един хвърлей камък, и като коленичи, молеше се, думайки:
\par 42 Отче, ако щеш, отмини Ме с тази чаша; обаче, не Моята воля, но Твоята да бъде.
\par 43 И яви Му се ангел от небето и Го укрепяваше.
\par 44 И като беше на мъка, молеше се по-усърдно; и потта Му стана като големи капки кръв, които капеха на земята.
\par 45 И като стана от молитвата, дойде при учениците и ги намери заспали от скръб; и рече им:
\par 46 Защо спите? Станете и молете се, за да не паднете в изкушение.
\par 47 Докато още говореше, ето едно множество; и този, който се наричаше Юда, един от дванадесетте, вървеше пред тях; и приближи се до Исуса, за да Го целуне.
\par 48 А Исус му рече: Юдо, с целувка ли предаваш Човешкия Син?
\par 49 И тия, които бяха около Исуса, като видяха какво щеше да стане, рекоха: Господи, да ударим ли с нож?
\par 50 И един от тях удари слугата на първосвещеника и му отсече дясното ухо,
\par 51 а Исус проговори, казвайки: Оставете до тука; и допря се до ухото му и го изцели.
\par 52 А на дошлите против Него главни свещеници, началници на храмовата стража и на старейшините Исус рече: Като срещу разбойник ли сте излезли с ножове и сопи?
\par 53 когато бях всеки ден с вас в храма, не простряхте ръце против Мене. Но сега е вашият час и на властта на тъмнината.
\par 54 И като Го хванаха, заведоха Го, и въведоха Го в къщата на първосвещеника. А Петър вървеше подире издалеч.
\par 55 И когато бяха наклали огън насред двора и бяха насядали около него, то и Петър седна между тях.
\par 56 И една слугиня, като го видя седнал до пламъка, вгледа се в него и рече: И тоя беше с Него.
\par 57 А той се отрече, казвайки: Жено, не Го познавам.
\par 58 След малко друг го видя и рече: И ти си от тях. Но Петър рече: Човече, не съм.
\par 59 И като се мина около един час, друг някой настоятелно казваше: Наистина и той беше с Него, защото е галилеянин.
\par 60 А Петър рече: Човече, не зная що казваш. И начаса, докато още говореше, един петел изпя.
\par 61 И Господ се обърна та погледна Петра. И Петър си спомни думата на Господа, как му беше казал: Преди да пропее петела днес, ти три пъти ще се отречеш от Мене.
\par 62 И излезе вън, та плака горко.
\par 63 И мъжете, които държаха Исуса, ругаеха Го и Го биеха,
\par 64 и като Го закриваха [удряха Го по лицето и] питаха Го, казвайки: Познай кой Те удари.
\par 65 И много други хули изговориха против Него.
\par 66 И като се разсъмна, събраха се народните старейшини, главни свещеници и книжници, и Го заведоха в синедриона си и Му рекоха:
\par 67 Ако си Ти Христос, кажи ни. А Той рече: Ако ви кажа, няма да повярвате;
\par 68 и ако ви задам въпрос, не ще отговорите.
\par 69 Но отсега нататък Човешкият Син ще седи отдясно на Божията сила.
\par 70 И те всички казаха: Тогава Божият Син ли си Ти? А Той им рече: Вие право казвате, защото Съм.
\par 71 А те рекоха: Каква нужда имаме вече от свидетелство? Защото сами ние чухме от устата Му.

\chapter{23}

\par 1 Тогава цялото множество техни хора стана и Го заведе при Пилата.
\par 2 И почнаха да Го обвиняват, казвайки: Намерихме този, че развращава народа ни, забранява да се дава данък на Кесаря, и казва за Себе Си, че е Христос Цар.
\par 3 А Пилат Го попита, думайки: Ти ли си Юдейският цар? А Той в отговор му каза: Ти право казваш.
\par 4 И Пилат рече на главните свещеници и на народа: Аз не намирам никаква вина в Тоя човек.
\par 5 А те по-настойчиво казваха: Той вълнува людете, понеже поучава по цяла Юдея, като е почнал от Галилея и е следвал даже до тук.
\par 6 А Пилат, като чу това, попита дали е галилеянин човекът.
\par 7 И като узна, че е от Иродовата област, изпрати Го до Ирода, който и той беше през тия дни в Ерусалим.
\par 8 А Ирод, като видя Исуса, много се зарадва, защото отдавна желаеше да Го види, понеже бе слушал за Него; и надяваше се да види някое знамение от Него.
\par 9 И запитваше Го с много думи, но Той нищо не му отговори.
\par 10 А главните свещеници и книжниците стояха и силно Го обвиняваха.
\par 11 Но Ирод с войниците си, презирайки Го, след като Му се поруга, облече Го във великолепна дреха, и Го прати обратно на Пилата.
\par 12 В същия ден Ирод и Пилат се сприятелиха помежду си, защото преди това враждуваха един против друг.
\par 13 Тогава Пилат свика главните свещеници, началници и народа, и рече им:
\par 14 Доведохте ми Тоя човек като един, който развращава людете; но, ето, аз Го разпитах пред вас, и не намерих в Тоя човек никаква вина относно онова, за което Го обвинявате.
\par 15 Нито пък Ирод е намерил; защото Го е изпратил обратно до нас; и ето, Той не е сторил нищо, което заслужава смъртно наказание.
\par 16 И тъй, като Го накажа, ще Го пусна.
\par 17 А той се задължаваше да им пуща на всеки празник по един затворник].
\par 18 Но те всички едногласно изкрещяха, казвайки: Махни Този и пусни ни Варава.
\par 19 (който, за някаква размирица, станала в града, и за убийство, бе хвърлен в тъмница).
\par 20 И Пилат пак им извика, като желаеше да пусне Исуса.
\par 21 А те крещяха, казвайки: Разпни Го! Разпни Го!
\par 22 А той трети път им каза: Че какво зло е сторил Той? Аз не намирам в Него нищо, за което да заслужава смърт; и тъй, като Го накажа, ще Го пусна.
\par 23 Но те настояваха със силни гласове, искайки да бъде разпнат; и техните гласове надделяха.
\par 24 И Пилат реши да изпълни искането им:
\par 25 Пусна онзи, когото искаха, който за размирица и убийство бе хвърлен в тъмница; а Исуса предаде на волята им.
\par 26 И когато Го поведоха, хванаха някого си Симона киринеец, който се връщаше от нива, и сложиха на него кръста, за да го носи подир Исуса.
\par 27 И след Него вървяха голямо множество от народ и от жени, които плачеха за Него.
\par 28 А Исус се обърна към тях и рече: Дъщери ерусалимски, недейте плака за Мене, но плачете за себе си и за чадата си;
\par 29 защото, ето идат дни, когато ще рекат: Блажени неплодните, и утробите, които не са раждали, и съсците, които не са кърмили.
\par 30 Тогава ще почнат да казват на планините: Паднете върху нас, и на хълмовете: Покрийте ни.
\par 31 Защото, ако правят това с суровото дърво, какво ще правят със сухото?
\par 32 И с Него караха и други двама, които бяха злодейци, за да ги погубят.
\par 33 И когато стигнаха на мястото, наречено Лобно, там разпнаха Него и злодейците, единият отдясно Му, а другият отляво.
\par 34 А Исус каза: Отче, прости им, защото не знаят какво правят. И като разделиха дрехите Му, хвърлиха жребие за тях.
\par 35 И людете стояха та гледаха. Още и началниците Го ругаеха, казвайки: Други е избавил; нека избави Себе Си, ако Този е Божият Христос, Неговият Избраник.
\par 36 Тоже и войниците Му се подиграваха като се приближаваха и Му поднасяха оцет, и казваха:
\par 37 Ако си Юдейският цар, избави Себе Си.
\par 38 А над него имаше и надпис: Тоя е Юдейският Цар.
\par 39 И един от обесените злодейци Го хулеше, казвайки: Нали си Ти Христос? Избави Себе Си и нас!
\par 40 А другият в отговор го смъмра, като каза: Ни от Бога ли се не боиш ти, който си под същото осъждение?
\par 41 И ние справедливо сме осъдени, защото получаваме заслуженото от това, което сме сторили; а Този не е сторил нищо лошо.
\par 42 И каза: [Господи] Исусе, спомни си за мене, когато дойдеш в Царството Си.
\par 43 А [Исус] му рече: Истина ти казвам, днес ще бъдеш с Мене в рая.
\par 44 А беше вече около шестият час, и тъмнина покриваше цялата земя до деветия час,
\par 45 като потъмня слънцето; и завесата на храма се раздра през средата.
\par 46 И Исус извика със силен глас и рече: Отче, в Твоите ръце предавам духа Си. И това като рече, издъхна.
\par 47 И стотникът като видя станалото, прослави Бога, като каза: Наистина този човек бе праведен.
\par 48 А всичките множества, надошли на това зрелище, като виждаха какво стана, връщаха се биещи се в гърди.
\par 49 А всичките негови познайници и жените, които бяха дошли подир Него от Галилея, стояха надалеч и гледаха това.
\par 50 И ето, един човек, на име Иосиф, който беше съветник, човек добър и праведен,
\par 51 който не беше се съгласил с намерението и делото им, - от юдейския град Ариматея, човек, който ожидаше Божието царство, -
\par 52 той отиде при Пилата и поиска Исусовото тяло.
\par 53 И като го сне, обви го с плащаница, и положи го в гроб, изсечен в скала, гдето никой не бе още полаган.
\par 54 И беше приготвителен ден, и съботата настъпваше.
\par 55 И жените, които бяха дошли с Него от Галилея, вървяха изподире и видяха гроба и как беше положено тялото Му.
\par 56 И като се върнаха, приготвиха аромати и миро; и в съботата си почиваха според заповедта.

\chapter{24}

\par 1 А в първия ден на седмицата, сутринта рано, жените дойдоха на гроба, носещи ароматите, които бяха приготвили.
\par 2 И намериха камъка отвален от гроба.
\par 3 И като влязоха, не намериха тялото на Господа Исуса.
\par 4 И когато бяха в недоумение за това, ето застанаха пред тях двама мъже с ослепително облекло.
\par 5 И обзети от страх, те наведоха лицата си до земята; а мъжете им рекоха: Защо търсите живия между мъртвите?
\par 6 Няма Го тука, но възкръсна. Спомнете си какво ви говореше, когато беше още в Галилея,
\par 7 като казваше, че Човешкият Син трябва да бъде предаден в ръцете на грешни човеци, да бъде разпнат, и в третия ден да възкръсне.
\par 8 И спомниха си думите му.
\par 9 И като се върнаха от гроба, известиха всичко това на единадесетте и на всичките други.
\par 10 А бяха Магдалина Мария, и Иоанна, и Мария Якововата майка и другите жени с тях, които казаха тия неща на апостолите.
\par 11 А тия думи им се видяха като празни приказки, и не вярваха.
\par 12 А Петър стана и изтича на гроба, и, като надникна, видя саваните, сложени отделно; и отиде у дома си, чудейки се за станалото.
\par 13 И, ето, в същия ден двама от тях отиваха в едно село на име Емаус, шестдесет стадии(Около 11 километра.) далеч от Ерусалим.
\par 14 И те се разговаряха помежду си за всичко онова, що бе станало.
\par 15 И като се разговаряха и разискваха, сам Исус се приближи и вървеше с тях;
\par 16 Но очите им се удържаха да Го не познаят.
\par 17 И рече им: Какви са тия думи, които разменявате помежду си, като пътувате? И те се спряха натъжени.
\par 18 И един от тях, на име Клеопа, в отговор Му рече: Само ти ли си пришелец в Ерусалим и не знаеш станалото там тия дни?
\par 19 И рече им: Кое? А те му рекоха: Станалото с Исуса Назарянина, Който бе пророк, силен в дело и в слово пред Бога и пред всичките люде;
\par 20 и как нашите главни свещеници и началници Го предадоха да бъде осъден на смърт и Го разпнаха.
\par 21 А ние се надявахме, че Той е Онзи, Който ще избави Израиля. И освен всичко това, вече е трети ден, откак стана това.
\par 22 При туй и някои жени измежду нас ни смаяха, които като отишли отзарана на гроба,
\par 23 и не намерили тялото Му, дойдоха и казаха, че видели и видение на ангели, които казали, че Той бил жив.
\par 24 И някои от ония, които бяха с нас, отидоха на гроба, и намериха тъй както рекоха жените; а Него не видели.
\par 25 И Той им рече: О несмислени и мудни по сърце да вярвате всичко, което са говорили пророците!
\par 26 Не трябваше ли Христос да пострада така и да влезе в славата Си?
\par 27 И като почна от Моисея и от всичките пророци, тълкуваше им писаното за Него във всичките писания.
\par 28 И приближиха селото, в което отиваха; Той се държеше, като че отива по-надалеч.
\par 29 Но те Го нудеха, казвайки: Остани с нас, защото е привечер, и денят вече е превалил. И Той влезе да остане с тях.
\par 30 И като седна с тях на трапезата, взе хляба и благослови, разчупи и даде им.
\par 31 И очите им се отвориха и Го познаха; а Той стана невидим за тях.
\par 32 И рекоха помежду си: Не гореше ли в нас сърцето, когато ни говореше по пътя и когато ни тълкуваше писанията?
\par 33 И в същия час станаха и се върнаха в Ерусалим, гдето намериха събрани единадесетте и тия, които бяха с тях,
\par 34 които и казаха: Господ наистина възкръснал и се явил на Симона.
\par 35 Те пък разправиха станалото по пътя, и как Го познаха, когато разчупваше хляба.
\par 36 И когато говореха за това, сам Исус застана посред тях и каза им: Мир вам!
\par 37 А те се стреснаха и се уплашиха, като мислеха, че виждат дух.
\par 38 И Той им рече: Защо се смущавате? И защо се пораждат такива мисли в сърцата ви?
\par 39 Погледнете ръцете Ми и нозете Ми, че съм Аз същият; попипайте Ме и вижте, защото дух няма меса и кости, както виждате, че Аз имам.
\par 40 И като рече това, показа им ръцете и нозете си.
\par 41 Но понеже те от радост още не вярваха и се чудеха, Той рече: Имате ли тук нещо за ядене?
\par 42 И дадоха му част от печена риба [и меден сок].
\par 43 И взе та яде пред тях.
\par 44 И рече им: Тия са думите, които ви говорих, когато бях още с вас, че трябва да се изпълни всичко, което е писано за Мене в Моисеевия закон, в пророците и в псалмите.
\par 45 Тогава им отвори ума, за да разберат писанията.
\par 46 И рече им: Така е писано, че Христос трябва да пострада и да възкръсне от мъртвите в третия ден,
\par 47 и че трябва да се проповядва в Негово име покаяние и прощение на греховете между всичките народи, като се започне от Ерусалим.
\par 48 Вие сте свидетели на това.
\par 49 И, ето, Аз изпращам върху вас обещанието на Отца Ми; а вие стойте в града [Ерусалим], докато се облечете със сила от горе.
\par 50 И ги заведе до едно място срещу Витания; и дигна ръцете Си да ги благослови.
\par 51 И като ги благославяше, отдели се от тях, и се възнесе на небето.
\par 52 И те Му се поклониха, и върнаха се в Ерусалим с голяма радост;
\par 53 и бяха постоянно в храма, [хвалещи и] благославящи Бога. [Амин].

\end{document}