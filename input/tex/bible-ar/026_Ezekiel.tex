\begin{document}

\title{حزقيال}


\chapter{1}

\par 1 كَانَ فِي سَنَةِ الثَّلاَثِينَ, فِي الشَّهْرِ الرَّابِعِ, فِي الْخَامِسِ مِنَ الشَّهْرِ, وَأَنَا بَيْنَ الْمَسْبِيِّينَ عِنْدَ نَهْرِ خَابُورَ أَنَّ السَّمَاوَاتِ انْفَتَحَتْ, فَرَأَيْتُ رُؤَى اللَّهِ.
\par 2 فِي الْخَامِسِ مِنَ الشَّهْرِ, وَهِيَ السَّنَةُ الْخَامِسَةُ مِنْ سَبْيِ يُويَاكِينَ الْمَلِكِ,
\par 3 صَارَ كَلاَمُ الرَّبِّ إِلَى حِزْقِيَالَ الْكَاهِنِ ابْنِ بُوزِي فِي أَرْضِ الْكِلْدَانِيِّينَ عِنْدَ نَهْرِ خَابُورَ. وَكَانَتْ عَلَيْهِ هُنَاكَ يَدُ الرَّبِّ.
\par 4 فَنَظَرْتُ وَإِذَا بِرِيحٍ عَاصِفَةٍ جَاءَتْ مِنَ الشِّمَالِ. سَحَابَةٌ عَظِيمَةٌ وَنَارٌ مُتَوَاصِلَةٌ وَحَوْلَهَا لَمَعَانٌ, وَمِنْ وَسَطِهَا كَمَنْظَرِ النُّحَاسِ اللاَّمِعِ مِنْ وَسَطِ النَّارِ.
\par 5 وَمِنْ وَسَطِهَا شِبْهُ أَرْبَعَةِ حَيَوَانَاتٍ. وَهَذَا مَنْظَرُهَا: لَهَا شِبْهُ إِنْسَانٍ.
\par 6 وَلِكُلِّ وَاحِدٍ أَرْبَعَةُ أَوْجُهٍ, وَلِكُلِّ وَاحِدٍ أَرْبَعَةُ أَجْنِحَةٍ.
\par 7 وَأَرْجُلُهَا أَرْجُلٌ قَائِمَةٌ, وَأَقْدَامُ أَرْجُلِهَا كَقَدَمِ رِجْلِ الْعِجْلِ, وَبَارِقَةٌ كَمَنْظَرِ النُّحَاسِ الْمَصْقُولِ.
\par 8 وَأَيْدِي إِنْسَانٍ تَحْتَ أَجْنِحَتِهَا عَلَى جَوَانِبِهَا الأَرْبَعَةِ. وَوُجُوهُهَا وَأَجْنِحَتُهَا لِجَوَانِبِهَا الأَرْبَعَةِ.
\par 9 وَأَجْنِحَتُهَا مُتَّصِلَةٌ الْوَاحِدُ بِأَخِيهِ. لَمْ تَدُرْ عِنْدَ سَيْرِهَا. كُلُّ وَاحِدٍ يَسِيرُ إِلَى جِهَةِ وَجْهِهِ.
\par 10 أَمَّا شِبْهُ وُجُوهِهَا فَوَجْهُ إِنْسَانٍ وَوَجْهُ أَسَدٍ لِلْيَمِينِ لأَرْبَعَتِهَا, وَوَجْهُ ثَوْرٍ مِنَ الشِّمَالِ لأَرْبَعَتِهَا, وَوَجْهُ نَسْرٍ لأَرْبَعَتِهَا.
\par 11 فَهَذِهِ أَوْجُهُهَا. أَمَّا أَجْنِحَتُهَا فَمَبْسُوطَةٌ مِنْ فَوْقُ. لِكُلِّ وَاحِدٍ اثْنَانِ مُتَّصِلاَنِ أَحَدُهُمَا بِأَخِيهِ, وَاثْنَانِ يُغَطِّيَانِ أَجْسَامَهَا.
\par 12 وَكُلُّ وَاحِدٍ كَانَ يَسِيرُ إِلَى جِهَةِ وَجْهِهِ. إِلَى حَيْثُ تَكُونُ الرُّوحُ لِتَسِيرَ تَسِيرُ. لَمْ تَدُرْ عِنْدَ سَيْرِهَا.
\par 13 أَمَّا شِبْهُ الْحَيَوَانَاتِ فَمَنْظَرُهَا كَجَمْرِ نَارٍ مُتَّقِدَةٍ, كَمَنْظَرِ مَصَابِيحَ هِيَ سَالِكَةٌ بَيْنَ الْحَيَوَانَاتِ. وَلِلنَّارِ لَمَعَانٌ, وَمِنَ النَّارِ كَانَ يَخْرُجُ بَرْقٌ.
\par 14 الْحَيَوَانَاتُ رَاكِضَةٌ وَرَاجِعَةٌ كَمَنْظَرِ الْبَرْقِ.
\par 15 فَنَظَرْتُ الْحَيَوَانَاتِ وَإِذَا بَكَرَةٌ وَاحِدَةٌ عَلَى الأَرْضِ بِجَانِبِ الْحَيَوَانَاتِ بِأَوْجُهِهَا الأَرْبَعَةِ.
\par 16 مَنْظَرُ الْبَكَرَاتِ وَصَنْعَتُهَا كَمَنْظَرِ الزَّبَرْجَدِ. وَلِلأَرْبَعِ شَكْلٌ وَاحِدٌ, وَمَنْظَرُهَا وَصَنْعَتُهَا كَأَنَّهَا كَانَتْ بَكَرَةً وَسَطَ بَكَرَةٍ.
\par 17 لَمَّا سَارَتْ سَارَتْ عَلَى جَوَانِبِهَا الأَرْبَعَةِ. لَمْ تَدُرْ عِنْدَ سَيْرِهَا.
\par 18 أَمَّا أُطُرُهَا فَعَالِيَةٌ وَمُخِيفَةٌ. وَأُطُرُهَا مَلآنَةٌ عُيُوناً حَوَالَيْهَا لِلأَرْبَعِ.
\par 19 فَإِذَا سَارَتِ الْحَيَوَانَاتُ سَارَتِ الْبَكَرَاتُ بِجَانِبِهَا, وَإِذَا ارْتَفَعَتِ الْحَيَوَانَاتُ عَنِ الأَرْضِ ارْتَفَعَتِ الْبَكَرَاتُ.
\par 20 إِلَى حَيْثُ تَكُونُ الرُّوحُ لِتَسِيرَ يَسِيرُونَ. إِلَى حَيْثُ الرُّوحُ لِتَسِيرَ وَالْبَكَرَاتُ تَرْتَفِعُ مَعَهَا. لأَنَّ رُوحَ الْحَيَوَانَاتِ كَانَتْ فِي الْبَكَرَاتِ.
\par 21 فَإِذَا سَارَتْ تِلْكَ سَارَتْ هَذِهِ, وَإِذَا وَقَفَتْ تِلْكَ وَقَفَتْ. وَإِذَا ارْتَفَعَتْ تِلْكَ عَنِ الأَرْضِ ارْتَفَعَتِ الْبَكَرَاتُ مَعَهَا, لأَنَّ رُوحَ الْحَيَوَانَاتِ كَانَتْ فِي الْبَكَرَاتِ.
\par 22 وَعَلَى رُؤُوسِ الْحَيَوَانَاتِ شِبْهُ مُقَبَّبٍ كَمَنْظَرِ الْبَلُّورِ الْهَائِلِ مُنْتَشِراً عَلَى رُؤُوسِهَا مِنْ فَوْقُ.
\par 23 وَتَحْتَ الْمُقَبَّبِ أَجْنِحَتُهَا مُسْتَقِيمَةٌ الْوَاحِدُ نَحْوَ أَخِيهِ. لِكُلِّ وَاحِدٍ اثْنَانِ يُغَطِّيَانِ مِنْ هُنَا, وَلِكُلِّ وَاحِدٍ اثْنَانِ يُغَطِّيَانِ مِنْ هُنَاكَ أَجْسَامَهَا.
\par 24 فَلَمَّا سَارَتْ سَمِعْتُ صَوْتَ أَجْنِحَتِهَا كَخَرِيرِ مِيَاهٍ كَثِيرَةٍ, كَصَوْتِ الْقَدِيرِ, صَوْتَ ضَجَّةٍ كَصَوْتِ جَيْشٍ. وَلَمَّا وَقَفَتْ أَرْخَتْ أَجْنِحَتَهَا.
\par 25 فَكَانَ صَوْتٌ مِنْ فَوْقِ الْمُقَبَّبِ الَّذِي عَلَى رُؤُوسِهَا. إِذَا وَقَفَتْ أَرْخَتْ أَجْنِحَتَهَا.
\par 26 وَفَوْقَ الْمُقَبَّبِ الَّذِي عَلَى رُؤُوسِهَا شِبْهُ عَرْشٍ كَمَنْظَرِ حَجَرِ الْعَقِيقِ الأَزْرَقِ, وَعَلَى شِبْهِ الْعَرْشِ شِبْهٌ كَمَنْظَرِ إِنْسَانٍ عَلَيْهِ مِنْ فَوْقُ.
\par 27 وَرَأَيْتُ مِثْلَ مَنْظَرِ النُّحَاسِ اللاَّمِعِ كَمَنْظَرِ نَارٍ دَاخِلَهُ مِنْ حَوْلِهِ, مِنْ مَنْظَرِ حَقَوَيْهِ إِلَى فَوْقُ, وَمِنْ مَنْظَرِ حَقَوَيْهِ إِلَى تَحْتُ. رَأَيْتُ مِثْلَ مَنْظَرِ نَارٍ وَلَهَا لَمَعَانٌ مِنْ حَوْلِهَا
\par 28 كَمَنْظَرِ الْقَوْسِ الَّتِي فِي السَّحَابِ يَوْمَ مَطَرٍ. هَكَذَا مَنْظَرُ اللَّمَعَانِ مِنْ حَوْلِهِ. هَذَا مَنْظَرُ شِبْهِ مَجْدِ الرَّبِّ. وَلَمَّا رَأَيْتُهُ خَرَرْتُ عَلَى وَجْهِي. وَسَمِعْتُ صَوْتَ مُتَكَلِّمٍ.

\chapter{2}

\par 1 فَقَالَ لِي: [يَا ابْنَ آدَمَ, قُمْ عَلَى قَدَمَيْكَ فَأَتَكَلَّمَ مَعَكَ».
\par 2 فَدَخَلَ فِيَّ رُوحٌ لَمَّا تَكَلَّمَ مَعِي. وَأَقَامَنِي عَلَى قَدَمَيَّ فَسَمِعْتُ الْمُتَكَلِّمَ مَعِي.
\par 3 وَقَالَ لِي: [يَا ابْنَ آدَمَ, أَنَا مُرْسِلُكَ إِلَى بَنِي إِسْرَائِيلَ, إِلَى أُمَّةٍ مُتَمَرِّدَةٍ قَدْ تَمَرَّدَتْ عَلَيَّ. هُمْ وَآبَاؤُهُمْ عَصُوا عَلَيَّ إِلَى ذَاتِ هَذَا الْيَوْمِ.
\par 4 وَالْبَنُونَ الْقُسَاةُ الْوُجُوهِ وَالصِّلاَبُ الْقُلُوبِ أَنَا مُرْسِلُكَ إِلَيْهِمْ. فَتَقُولُ لَهُمْ: هَكَذَا قَالَ السَّيِّدُ الرَّبُّ.
\par 5 وَهُمْ إِنْ سَمِعُوا وَإِنِ امْتَنَعُوا (لأَنَّهُمْ بَيْتٌ مُتَمَرِّدٌ) فَإِنَّهُمْ يَعْلَمُونَ أَنَّ نَبِيّاً كَانَ بَيْنَهُمْ.
\par 6 أَمَّا أَنْتَ يَا ابْنَ آدَمَ فَلاَ تَخَفْ مِنْهُمْ, وَمِنْ كَلاَمِهِمْ لاَ تَخَفْ, لأَنَّهُمْ قَرِيسٌ وَسُلاَّءٌ لَدَيْكَ, وَأَنْتَ سَاكِنٌ بَيْنَ الْعَقَارِبِ. مِنْ كَلاَمِهِمْ لاَ تَخَفْ وَمِنْ وُجُوهِهِمْ لاَ تَرْتَعِبْ, لأَنَّهُمْ بَيْتٌ مُتَمَرِّدٌ.
\par 7 وَتَتَكَلَّمُ مَعَهُمْ بِكَلاَمِي إِنْ سَمِعُوا وَإِنِ امْتَنَعُوا, لأَنَّهُمْ مُتَمَرِّدُونَ.
\par 8 [وَأَنْتَ يَا ابْنَ آدَمَ فَاسْمَعْ مَا أَنَا مُكَلِّمُكَ بِهِ. لاَ تَكُنْ مُتَمَرِّداً كَالْبَيْتِ الْمُتَمَرِّدِ. افْتَحْ فَمَكَ وَكُلْ مَا أَنَا مُعْطِيكَهُ».
\par 9 فَنَظَرْتُ وَإِذَا بِيَدٍ مَمْدُودَةٍ إِلَيَّ, وَإِذَا بِدَرْجِ سِفْرٍ فِيهَا.
\par 10 فَنَشَرَهُ أَمَامِي وَهُوَ مَكْتُوبٌ مِنْ دَاخِلٍ وَمِنْ قَفَاهُ, وَكُتِبَ فِيهِ مَرَاثٍ وَنَحِيبٌ وَوَيْلٌ.

\chapter{3}

\par 1 فَقَالَ لِي: [يَا ابْنَ آدَمَ, كُلْ مَا تَجِدُهُ. كُلْ هَذَا الدَّرْجَ, وَاذْهَبْ كَلِّمْ بَيْتَ إِسْرَائِيلَ».
\par 2 فَفَتَحْتُ فَمِي فَأَطْعَمَنِي ذَلِكَ الدَّرْجَ.
\par 3 وَقَالَ لِي: [يَا ابْنَ آدَمَ, أَطْعِمْ بَطْنَكَ وَامْلَأْ جَوْفَكَ مِنْ هَذَا الدَّرْجِ الَّذِي أَنَا مُعْطِيكَهُ». فَأَكَلْتُهُ فَصَارَ فِي فَمِي كَالْعَسَلِ حَلاَوَةً.
\par 4 فَقَالَ لِي: [يَا ابْنَ آدَمَ, اذْهَبِ امْضِ إِلَى بَيْتِ إِسْرَائِيلَ وَكَلِّمْهُمْ بِكَلاَمِي.
\par 5 لأَنَّكَ غَيْرُ مُرْسَلٍ إِلَى شَعْبٍ غَامِضِ اللُّغَةِ وَثَقِيلِ اللِّسَانِ, بَلْ إِلَى بَيْتِ إِسْرَائِيلَ.
\par 6 لاَ إِلَى شُعُوبٍ كَثِيرَةٍ غَامِضَةِ اللُّغَةِ وَثَقِيلَةِ اللِّسَانِ لَسْتَ تَفْهَمُ كَلاَمَهُمْ. فَلَوْ أَرْسَلْتُكَ إِلَى هَؤُلاَءِ لَسَمِعُوا لَكَ.
\par 7 لَكِنَّ بَيْتَ إِسْرَائِيلَ لاَ يَشَاءُ أَنْ يَسْمَعَ لَكَ, لأَنَّهُمْ لاَ يَشَاؤُونَ أَنْ يَسْمَعُوا لِي. لأَنَّ كُلَّ بَيْتِ إِسْرَائِيلَ صِلاَبُ الْجِبَاهِ وَقُسَاةُ الْقُلُوبِ.
\par 8 هَئَنَذَا قَدْ جَعَلْتُ وَجْهَكَ صُلْباً مِثْلَ وُجُوهِهِمْ وَجِبْهَتَكَ صُلْبَةً مِثْلَ جِبَاهِهِمْ,
\par 9 قَدْ جَعَلْتُ جِبْهَتَكَ كَالْمَاسِ أَصْلَبَ مِنَ الصَّوَّانِ, فَلاَ تَخَفْهُمْ وَلاَ تَرْتَعِبْ مِنْ وُجُوهِهِمْ لأَنَّهُمْ بَيْتٌ مُتَمَرِّد
\par 10 وَقَالَ لِي: [يَا ابْنَ آدَمَ, كُلُّ الْكَلاَمِ الَّذِي أُكَلِّمُكَ بِهِ أَوْعِهِ فِي قَلْبِكَ وَاسْمَعْهُ بِأُذُنَيْكَ.
\par 11 وَامْضِ اذْهَبْ إِلَى الْمَسْبِيِّينَ إِلَى بَنِي شَعْبِكَ وَكَلِّمْهُمْ وَقُلْ لَهُمْ: هَكَذَا قَالَ السَّيِّدُ الرَّبُّ, إِنْ سَمِعُوا وَإِنِ امْتَنَعُوا».
\par 12 ثُمَّ حَمَلَنِي رُوحٌ فَسَمِعْتُ خَلْفِي صَوْتَ رَعْدٍ عَظِيمٍ: [مُبَارَكٌ مَجْدُ الرَّبِّ مِنْ مَكَانِهِ».
\par 13 وَصَوْتَ أَجْنِحَةِ الْحَيَوَانَاتِ الْمُتَلاَصِقَةِ الْوَاحِدُ بِأَخِيهِ وَصَوْتَ الْبَكَرَاتِ مَعَهَا وَصَوْتَ رَعْدٍ عَظِيمٍ.
\par 14 فَحَمَلَنِي الرُّوحُ وَأَخَذَنِي, فَذَهَبْتُ مُرّاً فِي حَرَارَةِ رُوحِي, وَيَدُ الرَّبِّ كَانَتْ شَدِيدَةً عَلَيَّ.
\par 15 فَجِئْتُ إِلَى الْمَسْبِيِّينَ عِنْدَ تَلِّ أَبِيبَ, السَّاكِنِينَ عَُِنْدَ نَهْرِ خَابُورَ. وَحَيْثُ سَكَنُوا هُنَاكَ سَكَنْتُ سَبْعَةَ أَيَّامٍ مُتَحَيِّراً فِي وَسَطِهِمْ.
\par 16 وَكَانَ عِنْدَ تَمَامِ السَّبْعَةِ الأَيَّامِ أَنَّ كَلِمَةَ الرَّبِّ صَارَتْ إِلَيَّ:
\par 17 [يَا ابْنَ آدَمَ, قَدْ جَعَلْتُكَ رَقِيباً لِبَيْتِ إِسْرَائِيلَ. فَاسْمَعِ الْكَلِمَةَ مِنْ فَمِي وَأَنْذِرْهُمْ مِنْ قِبَلِي.
\par 18 إِذَا قُلْتُ لِلشِّرِّيرِ: مَوْتاً تَمُوتُ وَمَا أَنْذَرْتَهُ أَنْتَ وَلاَ تَكَلَّمْتَ إِنْذَاراً لِلشِّرِّيرِ مِنْ طَرِيقِهِ الرَّدِيئَةِ لإِحْيَائِهِ, فَذَلِكَ الشِّرِّيرُ يَمُوتُ بِإِثْمِهِ, أَمَّا دَمُهُ فَمِنْ يَدِكَ أَطْلُبُهُ.
\par 19 وَإِنْ أَنْذَرْتَ أَنْتَ الشِّرِّيرَ وَلَمْ يَرْجِعْ عَنْ شَرِّهِ وَلاَ عَنْ طَرِيقِهِ الرَّدِيئَةِ, فَإِنَّهُ يَمُوتُ بِإِثْمِهِ. أَمَّا أَنْتَ فَقَدْ نَجَّيْتَ نَفْسَكَ.
\par 20 وَالْبَارُّ إِنْ رَجَعَ عَنْ بِرِّهِ وَعَمِلَ إِثْماً وَجَعَلْتُ مَعْثَرَةً أَمَامَهُ فَإِنَّهُ يَمُوتُ. لأَنَّكَ لَمْ تُنْذِرْهُ يَمُوتُ فِي خَطِيَّتِهِ وَلاَ يُذْكَرُ بِرُّهُ الَّذِي عَمِلَهُ. أَمَّا دَمُهُ فَمِنْ يَدِكَ أَطْلُبُهُ.
\par 21 وَإِنْ أَنْذَرْتَ أَنْتَ الْبَارَّ مِنْ أَنْ يُخْطِئَ الْبَارُّ, وَهُوَ لَمْ يُخْطِئْ, فَإِنَّهُ حَيَاةً يَحْيَا لأَنَّهُ أُنْذِرَ, وَأَنْتَ تَكُونُ قَدْ نَجَّيْتَ نَفْسَكَ
\par 22 وَكَانَتْ يَدُ الرَّبِّ عَلَيَّ هُنَاكَ. وَقَالَ لِي: [قُمُ اخْرُجْ إِلَى الْبُقْعَةِ وَهُنَاكَ أُكَلِّمُكَ».
\par 23 فَقُمْتُ وَخَرَجْتُ إِلَى الْبُقْعَةِ, وَإِذَا بِمَجْدِ الرَّبِّ وَاقِفٌ هُنَاكَ كَالْمَجْدِ الَّذِي رَأَيْتُهُ عِنْدَ نَهْرِ خَابُورَ. فَخَرَرْتُ عَلَى وَجْهِي.
\par 24 فَدَخَلَ فِيَّ رُوحٌ وَأَقَامَنِي عَلَى قَدَمَيَّ. ثُمَّ قَالَ لِي: [اِذْهَبْ أَغْلِقْ عَلَى نَفْسِكَ فِي وَسَطِ بَيْتِكَ.
\par 25 وَأَنْتَ يَا ابْنَ آدَمَ فَهَا هُمْ يَضَعُونَ عَلَيْكَ رُبُطاً وَيُقَيِّدُونَكَ بِهَا, فَلاَ تَخْرُجُ فِي وَسَطِهِمْ.
\par 26 وَأُلْصِقُ لِسَانَكَ بِحَنَكِكَ فَتَبْكَمُ وَلاَ تَكُونُ لَهُمْ رَجُلاً مُوَبِّخاً, لأَنَّهُمْ بَيْتٌ مُتَمَرِّدٌ.
\par 27 فَإِذَا كَلَّمْتُكَ أَفْتَحُ فَمَكَ فَتَقُولُ لَهُمْ: هَكَذَا قَالَ السَّيِّدُ الرَّبُّ. مَنْ يَسْمَعْ فَلْيَسْمَعْ, وَمَنْ يَمْتَنِعْ فَلْيَمْتَنِعْ. لأَنَّهُمْ بَيْتٌ مُتَمَرِّدٌ.

\chapter{4}

\par 1 [وَأَنْتَ يَا ابْنَ آدَمَ فَخُذْ لِنَفْسِكَ لِبْنَةً وَضَعْهَا أَمَامَكَ, وَارْسِمْ عَلَيْهَا مَدِينَةَ أُورُشَلِيمَ.
\par 2 وَاجْعَلْ عَلَيْهَا حِصَاراً, وَابْنِ عَلَيْهَا بُرْجاً, وَأَقِمْ عَلَيْهَا مِتْرَسَةً, وَاجْعَلْ عَلَيْهَا جُيُوشاً, وَأَقِمْ عَلَيْهَا مَجَانِقَ حَوْلَهَا.
\par 3 وَخُذْ أَنْتَ لِنَفْسِكَ صَاجاً مِنْ حَدِيدٍ وَانْصِبْهُ سُوراً مِنْ حَدِيدٍ بَيْنَكَ وَبَيْنَ الْمَدِينَةِ, وَثَبِّتْ وَجْهَكَ عَلَيْهَا فَتَكُونَ فِي حِصَارٍ وَتُحَاصِرَهَا. تِلْكَ آيَةٌ لِبَيْتِ إِسْرَائِيلَ
\par 4 وَاتَّكِئْ أَنْتَ عَلَى جَنْبِكَ الْيَسَارِ وَضَعْ عَلَيْهِ إِثْمَ بَيْتِ إِسْرَائِيلَ. عَلَى عَدَدِ الأَيَّامِ الَّتِي فِيهَا تَتَّكِئُ عَلَيْهِ تَحْمِلُ إِثْمَهُمْ.
\par 5 وَأَنَا قَدْ جَعَلْتُ لَكَ سِنِي إِثْمِهِمْ حَسَبَ عَدَدِ الأَيَّامِ, ثَلاَثَ مِئَةِ يَوْمٍ وَتِسْعِينَ يَوْماً, فَتَحْمِلُ إِثْمَ بَيْتِ إِسْرَائِيلَ.
\par 6 فَإِذَا أَتْمَمْتَهَا فَاتَّكِئْ عَلَى جَنْبِكَ الْيَمِينِ أَيْضاً, فَتَحْمِلَ إِثْمَ بَيْتِ يَهُوذَا أَرْبَعِينَ يَوْماً. فَقَدْ جَعَلْتُ لَكَ كُلَّ يَوْمٍ عِوَضاً عَنْ سَنَةٍ.
\par 7 فَثَبِّتْ وَجْهَكَ عَلَى حِصَارِ أُورُشَلِيمَ وَذِرَاعُكَ مَكْشُوفَةٌ وَتَنَبَّأْ عَلَيْهَا.
\par 8 وَهَئَنَذَا أَجْعَلُ عَلَيْكَ رُبُطاً فَلاَ تَقْلِبُ مِنْ جَنْبٍ إِلَى جَنْبٍ حَتَّى تُتَمِّمَ أَيَّامَ حِصَارِكَ
\par 9 وَخُذْ أَنْتَ لِنَفْسِكَ قَمْحاً وَشَعِيراً وَفُولاً وَعَدَساً وَدُخْناً وَكَرْسَنَّةَ وَضَعْهَا فِي وِعَاءٍ وَاحِدٍ, وَاصْنَعْهَا لِنَفْسِكَ خُبْزاً كَعَدَدِ الأَيَّامِ الَّتِي تَتَّكِئُ فِيهَا عَلَى جَنْبِكَ. ثَلاَثَ مِئَةِ يَوْمٍ وَتِسْعِينَ يَوْماً تَأْكُلُهُ.
\par 10 وَطَعَامُكَ الَّذِي تَأْكُلُهُ يَكُونُ بِالْوَزْنِ. كُلَّ يَوْمٍ عِشْرِينَ شَاقِلاً. مِنْ وَقْتٍ إِلَى وَقْتٍ تَأْكُلُهُ.
\par 11 وَتَشْرَبُ الْمَاءَ بِالْكَيْلِ. سُدْسَ الْهِينِ. مِنْ وَقْتٍ إِلَى وَقْتٍ تَشْرَبُهُ.
\par 12 وَتَأْكُلُ كَعْكاً مِنَ الشَّعِيرِ. عَلَى الْخُرْءِ الَّذِي يَخْرُجُ مِنَ الإِنْسَانِ تَخْبِزُهُ أَمَامَ عُيُونِهِمْ».
\par 13 وَقَالَ الرَّبُّ: [هَكَذَا يَأْكُلُ بَنُو إِسْرَائِيلَ خُبْزَهُمُ النَّجِسَ بَيْنَ الأُمَمِ الَّذِينَ أَطْرُدُهُمْ إِلَيْهِمْ».
\par 14 فَقُلْتُ: [آهِ يَا سَيِّدُ الرَّبُّ, هَا نَفْسِي لَمْ تَتَنَجَّسْ. وَمِنْ صِبَايَ إِلَى الآنَ لَمْ آكُلْ مِيتَةً أَوْ فَرِيسَةً, وَلاَ دَخَلَ فَمِي لَحْمٌ نَجِسٌ».
\par 15 فَقَالَ لِي: [اُنْظُرْ. قَدْ جَعَلْتُ لَكَ خِثْيَ الْبَقَرِ بَدَلَ خُرْءِ الإِنْسَانِ فَتَصْنَعُ خُبْزَكَ عَلَيْهِ».
\par 16 وَقَالَ لِي: [يَا ابْنَ آدَمَ, هَئَنَذَا أُكَسِّرُ قِوَامَ الْخُبْزِ فِي أُورُشَلِيمَ, فَيَأْكُلُونَ الْخُبْزَ بِالْوَزْنِ وَبِالْغَمِّ, وَيَشْرَبُونَ الْمَاءَ بِالْكَيْلِ وَبِالْحَيْرَةِ
\par 17 لِكَيْ يُعْوِزَهُمُ الْخُبْزُ وَالْمَاءُ, وَيَتَحَيَّرُوا الرَّجُلُ وَأَخُوهُ وَيَفْنُوا بِإِثْمِهِمْ].

\chapter{5}

\par 1 وَأَنْتَ يَا ابْنَ آدَمَ فَخُذْ لِنَفْسِكَ سِكِّيناً حَادّاً, مُوسَى الْحَلاَّقِ تَأْخُذُ لِنَفْسِكَ. وَأَمْرِرْهَا عَلَى رَأْسِكَ وَعَلَى لِحْيَتِكَ. وَخُذْ لِنَفْسِكَ مِيزَاناً لِلْوَزْنِ وَاقْسِمْهُ
\par 2 وَأَحْرِقْ بِالنَّارِ ثُلْثَهُ فِي وَسَطِ الْمَدِينَةِ إِذَا تَمَّتْ أَيَّامُ الْحِصَارِ. وَخُذْ ثُلْثاً وَاضْرِبْهُ بِالسَّيْفِ حَوَالَيْهِ, وَذَرِّ ثُلْثاً إِلَى الرِّيحِ. وَأَنَا أَسْتَلُّ سَيْفاً وَرَاءَهُمْ.
\par 3 وَخُذْ مِنْهُ قَلِيلاً بِالْعَدَدِ وَصُرَّهُ فِي أَذْيَالِكَ.
\par 4 وَخُذْ مِنْهُ أَيْضاً وَأَلْقِهِ فِي وَسَطِ النَّارِ وَأَحْرِقْهُ بِالنَّارِ. مِنْهُ تَخْرُجُ نَارٌ عَلَى كُلِّ بَيْتِ إِسْرَائِيلَ».
\par 5 هَكَذَا قَالَ السَّيِّدُ الرَّبُّ: [هَذِهِ أُورُشَلِيمُ. فِي وَسَطِ الشُّعُوبِ قَدْ أَقَمْتُهَا وَحَوَالَيْهَا الأَرَاضِي.
\par 6 فَخَالَفَتْ أَحْكَامِي بِأَشَرَّ مِنَ الأُمَمِ, وَفَرَائِضِي بِأَشَرَّ مِنَ الأَرَاضِي الَّتِي حَوَالَيْهَا. لأَنَّ أَحْكَامِي رَفَضُوهَا وَفَرَائِضِي لَمْ يَسْلُكُوا فِيهَا.
\par 7 لأَجْلِ ذَلِكَ هَكَذَا قَالَ السَّيِّدُ الرَّبُّ: مِنْ أَجْلِ أَنَّكُمْ ضَجَجْتُمْ أَكْثَرَ مِنَ الأُمَمِ الَّتِي حَوَالَيْكُمْ وَلَمْ تَسْلُكُوا فِي فَرَائِضِي وَلَمْ تَعْمَلُوا حَسَبَ أَحْكَامِي, وَلاَ عَمِلْتُمْ حَسَبَ أَحْكَامِ الأُمَمِ الَّتِي حَوَالَيْكُمْ,
\par 8 لِذَلِكَ هَكَذَا قَالَ السَّيِّدُ الرَّبُّ: هَا إِنِّي أَنَا أَيْضاً عَلَيْكِ, وَسَأُجْرِي فِي وَسَطِكِ أَحْكَاماً أَمَامَ عُيُونِ الأُمَمِ,
\par 9 وَأَفْعَلُ بِكِ مَا لَمْ أَفْعَلْ وَمَا لَنْ أَفْعَلَ مِثْلَهُ بَعْدُ بِسَبَبِ كُلِّ أَرْجَاسِكِ.
\par 10 لأَجْلِ ذَلِكَ تَأْكُلُ الآبَاءُ الأَبْنَاءَ فِي وَسَطِكِ, وَالأَبْنَاءُ يَأْكُلُونَ آبَاءَهُمْ. وَأُجْرِي فِيكِ أَحْكَاماً, وَأُذَرِّي بَقِيَّتَكِ كُلَّهَا فِي كُلِّ رِيحٍ.
\par 11 مِنْ أَجْلِ ذَلِكَ حَيٌّ أَنَا يَقُولُ السَّيِّدُ الرَّبُّ, مِنْ أَجْلِ أَنَّكِ قَدْ نَجَّسْتِ مَقْدِسِي بِكُلِّ مَكْرُهَاتِكِ وَبِكُلِّ أَرْجَاسِكِ, فَأَنَا أَيْضاً أَجُزُّ وَلاَ تُشْفِقُ عَيْنِي. وَأَنَا أَيْضاً لاَ أَعْفُو.
\par 12 ثُلْثُكِ يَمُوتُ بِالْوَبَإِ, وَبِالْجُوعِ يَفْنُونَ فِي وَسَطِكِ. وَثُلْثٌ يَسْقُطُ بِالسَّيْفِ مِنْ حَوْلِكِ, وَثُلْثٌ أُذَرِّيهِ فِي كُلِّ رِيحٍ, وَأَسْتَلُّ سَيْفاً وَرَاءَهُمْ.
\par 13 وَإِذَا تَمَّ غَضَبِي وَأَحْلَلْتُ سَخَطِي عَلَيْهِمْ وَتَشَفَّيْتُ, يَعْلَمُونَ أَنِّي أَنَا الرَّبُّ تَكَلَّمْتُ فِي غَيْرَتِي إِذَا أَتْمَمْتُ سَخَطِي فِيهِمْ.
\par 14 وَأَجْعَلُكِ خَرَاباً وَعَاراً بَيْنَ الأُمَمِ الَّتِي حَوَالَيْكِ أَمَامَ عَيْنَيْ كُلِّ عَابِرٍ,
\par 15 فَتَكُونِينَ عَاراً وَلَعْنَةً وَتَأْدِيباً وَدَهَشاً لِلأُمَمِ الَّتِي حَوَالَيْكِ, إِذَا أَجْرَيْتُ فِيكِ أَحْكَاماً بِغَضَبٍ وَبِسَخَطٍ وَبِتَوْبِيخَاتٍ حَامِيَةٍ. أَنَا الرَّبُّ تَكَلَّمْتُ.
\par 16 إِذَا أَرْسَلْتُ عَلَيْهِمْ سِهَامَ الْجُوعِ الشِّرِّيرَةَ الَّتِي تَكُونُ لِلْخَرَابِ الَّتِي أُرْسِلُهَا لِخَرَابِكُمْ, وَأَزِيدُ الْجُوعَ عَلَيْكُمْ وَأُكَسِّرُ لَكُمْ قِوَامَ الْخُبْزِ,
\par 17 وَإِذَا أَرْسَلْتُ عَلَيْكُمُ الْجُوعَ وَالْوُحُوشَ الرَّدِيئَةَ فَتُثْكِلُكِ, وَيَعْبُرُ فِيكِ الْوَبَأُ وَالدَّمُ وَأَجْلُبُ عَلَيْكِ سَيْفاً. أَنَا الرَّبُّ تَكَلَّمْتُ.

\chapter{6}

\par 1 وَكَانَ إِلَيَّ كَلاَمُ الرَّبِّ:
\par 2 [يَا ابْنَ آدَمَ, اجْعَلْ وَجْهَكَ نَحْوَ جِبَالِ إِسْرَائِيلَ وَتَنَبَّأْ عَلَيْهَا
\par 3 وَقُلْ: يَا جِبَالَ إِسْرَائِيلَ, اسْمَعِي كَلِمَةَ السَّيِّدِ الرَّبِّ. هَكَذَا قَالَ السَّيِّدُ الرَّبُّ لِلْجِبَالِ وَلِلآكَامِ, لِلأَوْدِيَةِ وَلِلأَوْطِئَةِ, هَئَنَذَا جَالِبٌ عَلَيْكُمْ سَيْفاً وَأُبِيدُ مُرْتَفَعَاتِكُمْ.
\par 4 فَتَخْرَبُ مَذَابِحُكُمْ, وَتَتَكَسَّرُ شَمْسَاتُكُمْ, وَأَطْرَحُ قَتْلاَكُمْ قُدَّامَ أَصْنَامِكُمْ.
\par 5 وَأَضَعُ جُثَثَ بَنِي إِسْرَائِيلَ قُدَّامَ أَصْنَامِهِمْ, وَأُذَرِّي عِظَامَكُمْ حَوْلَ مَذَابِحِكُمْ.
\par 6 فِي كُلِّ مَسَاكِنِكُمْ تُقْفِرُ الْمُدُنُ, وَتَخْرَبُ الْمُرْتَفَعَاتُ, لِتُقْفِرَ وَتَخْرَبَ مَذَابِحُكُمْ وَتَنْكَسِرَ وَتَزُولَ أَصْنَامُكُمْ وَتُقْطَعَ شَمْسَاتُكُمْ وَتُمْحَى أَعْمَالُكُمْ.
\par 7 وَتَسْقُطُ الْقَتْلَى فِي وَسَطِكُمْ فَتَعْلَمُونَ أَنِّي أَنَا الرَّبُّ
\par 8 وَأُبْقِي بَقِيَّةً, إِذْ يَكُونُ لَكُمْ نَاجُونَ مِنَ السَّيْفِ بَيْنَ الأُمَمِ عِنْدَ تَذَرِّيكُمْ فِي الأَرَاضِي.
\par 9 وَالنَّاجُونَ مِنْكُمْ يَذْكُرُونَنِي بَيْنَ الأُمَمِ الَّذِينَ يُسْبَوْنَ إِلَيْهِمْ, إِذَا كَسَرْتُ قَلْبَهُمُ الزَّانِيَ الَّذِي حَادَ عَنِّي وَعُيُونَهُمُ الزَّانِيَةَ وَرَاءَ أَصْنَامِهِمْ, وَمَقَتُوا أَنْفُسَهُمْ لأَجْلِ الشُّرُورِ الَّتِي فَعَلُوهَا فِي كُلِّ رَجَاسَاتِهِمْ.
\par 10 وَيَعْلَمُونَ أَنِّي أَنَا الرَّبُّ. لَمْ أَقُلْ بَاطِلاً إِنِّي أَفْعَلُ بِهِمْ هَذَا الشَّرَّ].
\par 11 هَكَذَا قَالَ السَّيِّدُ الرَّبُّ: [اضْرِبْ بِيَدِكَ وَاخْبِطْ بِرِجْلِكَ, وَقُلْ: آهِ عَلَى كُلِّ رَجَاسَاتِ بَيْتِ إِسْرَائِيلَ الشِّرِّيرَةِ حَتَّى يَسْقُطُوا بِالسَّيْفِ وَبِالْجُوعِ وَبِالْوَبَإِ!
\par 12 اَلْبَعِيدُ يَمُوتُ بِالْوَبَإِ, وَالْقَرِيبُ يَسْقُطُ بِالسَّيْفِ, وَالْبَاقِي وَالْمُنْحَصِرُ يَمُوتُ بِالْجُوعِ, فَأُتَمِّمُ غَضَبِي عَلَيْهِمْ.
\par 13 فَتَعْلَمُونَ أَنِّي أَنَا الرَّبُّ إِذَا كَانَتْ قَتْلاَهُمْ وَسَطَ أَصْنَامِهِمْ حَوْلَ مَذَابِحِهِمْ عَلَى كُلِّ أَكَمَةٍ عَالِيَةٍ, وَفِي رُؤُوسِ كُلِّ الْجِبَالِ, وَتَحْتَ كُلِّ شَجَرَةٍ خَضْرَاءَ, وَتَحْتَ كُلِّ بَلُّوطَةٍ غَبْيَاءَ, الْمَوْضِعِ الَّذِي قَرَّبُوا فِيهِ رَائِحَةَ سُرُورٍ لِكُلِّ أَصْنَامِهِمْ.
\par 14 وَأَمُدُّ يَدِي عَلَيْهِمْ, وَأُصَيِّرُ الأَرْضَ مُقْفِرَةً وَخَرِبَةً مِنَ الْقَفْرِ إِلَى دَبْلَةَ فِي كُلِّ مَسَاكِنِهِمْ, فَيَعْلَمُونَ أَنِّي أَنَا الرَّبُّ].

\chapter{7}

\par 1 وَكَانَ إِلَيَّ كَلاَمُ الرَّبِّ:
\par 2 [وَأَنْتَ يَا ابْنَ آدَمَ فَهَكَذَا قَالَ السَّيِّدُ الرَّبُّ لأَرْضِ إِسْرَائِيلَ: نِهَايَةٌ. قَدْ جَاءَتِ النِّهَايَةُ عَلَى زَوَايَا الأَرْضِ الأَرْبَعِ.
\par 3 اَلآنَ النِّهَايَةُ عَلَيْكِ, وَأُرْسِلُ غَضَبِي عَلَيْكِ, وَأَحْكُمُ عَلَيْكِ كَطُرُقِكِ وَأَجْلِبُ عَلَيْكِ كُلَّ رَجَاسَاتِكِ.
\par 4 فَلاَ تُشْفِقُ عَلَيْكِ عَيْنِي, وَلاَ أَعْفُو بَلْ أَجْلِبُ عَلَيْكِ طُرُقَكِ وَتَكُونُ رَجَاسَاتُكِ فِي وَسَطِكِ, فَتَعْلَمُونَ أَنِّي أَنَا الرَّبُّ».
\par 5 هَكَذَا قَالَ السَّيِّدُ الرَّبُّ: [شَرٌّ شَرٌّ وَحِيدٌ هُوَذَا قَدْ أَتَى.
\par 6 نِهَايَةٌ قَدْ جَاءَتْ. جَاءَتِ النِّهَايَةُ. انْتَبَهَتْ إِلَيْكِ. هَا هِيَ قَدْ جَاءَتْ.
\par 7 انْتَهَى الدَّوْرُ إِلَيْكَ أَيُّهَا السَّاكِنُ فِي الأَرْضِ. بَلَغَ الْوَقْتُ. اقْتَرَبَ يَوْمُ اضْطِرَابٍ, لاَ هُتَافُ الْجِبَالِ.
\par 8 اَلآنَ عَنْ قَرِيبٍ أَصُبُّ رِجْزِي عَلَيْكِ, وَأُتَمِّمُ سَخَطِي عَلَيْكِ, وَأَحْكُمُ عَلَيْكِ كَطُرُقِكِ, وَأَجْلِبُ عَلَيْكِ كُلَّ رَجَاسَاتِكِ.
\par 9 فَلاَ تُشْفِقُ عَيْنِي وَلاَ أَعْفُو بَلْ أَجْلِبُ عَلَيْكِ كَطُرُقِكِ, وَرَجَاسَاتُكِ تَكُونُ فِي وَسَطِكِ. فَتَعْلَمُونَ أَنِّي أَنَا الرَّبُّ الضَّارِب
\par 10 [هَا هُوَذَا الْيَوْمُ, هَا هُوَذَا قَدْ جَاءَ. دَارَتِ الدَّائِرَةُ. أَزْهَرَتِ الْعَصَا. أَفْرَخَتِ الْكِبْرِيَاءُ.
\par 11 قَامَ الظُّلْمُ إِلَى عَصَا الشَّرِّ. لاَ يَبْقَى مِنْهُمْ وَلاَ مِنْ ثَرْوَتِهِمْ وَلاَ مِنْ ضَجِيجِهِمْ, وَلاَ نَوْحٌ عَلَيْهِمْ.
\par 12 قَدْ جَاءَ الْوَقْتُ. بَلَغَ الْيَوْمُ. فَلاَ يَفْرَحَنَّ الشَّارِي وَلاَ يَحْزَنَنَّ الْبَائِعُ, لأَنَّ الْغَضَبَ عَلَى كُلِّ جُمْهُورِهِمْ.
\par 13 لأَنَّ الْبَائِعَ لَنْ يَعُودَ إِلَى الْمَبِيعِ وَإِنْ كَانُوا بَعْدُ بَيْنَ الأَحْيَاءِ. لأَنَّ الرُّؤْيَا عَلَى كُلِّ جُمْهُورِهَا فَلاَ يَعُودُ, وَالإِنْسَانُ بِإِثْمِهِ لاَ يُشَدِّدُ حَيَاتَهُ.
\par 14 قَدْ نَفَخُوا فِي الْبُوقِ وَأَعَدُّوا الْكُلَّ, وَلاَ ذَاهِبَ إِلَى الْقِتَالِ. لأَنَّ غَضَبِي عَلَى كُلِّ جُمْهُورِهِمْ.
\par 15 اَلسَّيْفُ مِنْ خَارِجٍ, وَالْوَبَأُ وَالْجُوعُ مِنْ دَاخِلٍ. الَّذِي هُوَ فِي الْحَقْلِ يَمُوتُ بِالسَّيْفِ, وَالَّذِي هُوَ فِي الْمَدِينَةِ يَأْكُلُهُ الْجُوعُ وَالْوَبَأُ.
\par 16 وَيَنْفَلِتُ مِنْهُمْ مُنْفَلِتُونَ وَيَكُونُونَ عَلَى الْجِبَالِ كَحَمَامِ الأَوْطِئَةِ. كُلُّهُمْ يَهْدِرُونَ كُلُّ وَاحِدٍ عَلَى إِثْمِهِ.
\par 17 كُلُّ الأَيْدِي تَرْتَخِي, وَكُلُّ الرُّكَبِ تَصِيرُ مَاءً.
\par 18 وَيَتَنَطَّقُونَ بِالْمِسْحِ وَيَغْشَاهُمْ رُعْبٌ, وَعَلَى جَمِيعِ الْوُجُوهِ خِزْيٌ وَعَلَى جَمِيعِ رُؤُوسِهِمْ قَرَعٌ.
\par 19 يُلْقُونَ فِضَّتَهُمْ فِي الشَّوَارِعِ وَذَهَبُهُمْ يَكُونُ لِنَجَاسَةٍ. لاَ تَسْتَطِيعُ فِضَّتُهُمْ وَذَهَبُهُمْ إِنْقَاذَهُمْ فِي يَوْمِ غَضَبِ الرَّبِّ. لاَ يُشْبِعُونَ مِنْهُمَا أَنْفُسَهُمْ وَلاَ يَمْلأُونَ جَوْفَهُمْ, لأَنَّهُمَا صَارَا مَعْثَرَةَ إِثْمِهِمْ.
\par 20 أَمَّا بَهْجَةُ زِينَتِهِ فَجَعَلَهَا لِلْكِبْرِيَاءِ. جَعَلُوا فِيهَا أَصْنَامَ مَكْرُهَاتِهِمْ, رَجَاسَاتِهِمْ. لأَجْلِ ذَلِكَ جَعَلْتُهَا لَهُمْ نَجَاسَةً.
\par 21 أُسْلِمُهَا إِلَى أَيْدِي الْغُرَبَاءِ لِلنَّهْبِ, وَإِلَى أَشْرَارِ الأَرْضِ سَلْباً فَيُنَجِّسُونَهَا.
\par 22 وَأُحَوِّلُ وَجْهِي عَنْهُمْ فَيُنَجِّسُونَ سِرِّي, وَيَدْخُلُهُ الْمُعْتَنِفُونَ وَيُنَجِّسُونَهُ.
\par 23 اِصْنَعِ السِّلْسِلَةَ لأَنَّ الأَرْضَ قَدِ امْتَلأَتْ مِنْ أَحْكَامِ الدَّمِ, وَالْمَدِينَةُ امْتَلأَتْ مِنَ الظُّلْمِ.
\par 24 فَآتِي بِأَشَرِّ الأُمَمِ فَيَرِثُونَ بُيُوتَهُمْ, وَأُبِيدُ كِبْرِيَاءَ الأَشِدَّاءِ فَتَتَنَجَّسُ مَقَادِسُهُمْ.
\par 25 اَلرُّعْبُ آتٍ فَيَطْلُبُونَ السَّلاَمَ وَلاَ يَكُونُ.
\par 26 سَتَأْتِي مُصِيبَةٌ عَلَى مُصِيبَةٍ. وَيَكُونُ خَبَرٌ عَلَى خَبَرٍ. فَيَطْلُبُونَ رُؤْيَا مِنَ النَّبِيِّ. وَالشَّرِيعَةُ تُبَادُ عَنِ الْكَاهِنِ وَالْمَشُورَةُ عَنِ الشُّيُوخِ.
\par 27 الْمَلِكُ يَنُوحُ وَالرَّئِيسُ يَلْبَسُ حَيْرَةً, وَأَيْدِي شَعْبِ الأَرْضِ تَرْجُفُ. كَطَرِيقِهِمْ أَصْنَعُ بِهِمْ, وَكَأَحْكَامِهِمْ أَحْكُمُ عَلَيْهِمْ فَيَعْلَمُونَ أَنِّي أَنَا الرَّبُّ].

\chapter{8}

\par 1 وَكَانَ فِي السَّنَةِ السَّادِسَةِ فِي الشَّهْرِ السَّادِسِ فِي الْخَامِسِ مِنَ الشَّهْرِ, وَأَنَا جَالِسٌ فِي بَيْتِي وَمَشَايِخُ يَهُوذَا جَالِسُونَ أَمَامِي, أَنَّ يَدَ السَّيِّدِ الرَّبِّ وَقَعَتْ عَلَيَّ هُنَاكَ.
\par 2 فَنَظَرْتُ وَإِذَا شَبَهٌ كَمَنْظَرِ نَارٍ, مِنْ مَنْظَرِ حَقَوَيْهِ إِلَى تَحْتُ نَارٌ, وَمِنْ حَقَوَيْهِ إِلَى فَوْقُ كَمَنْظَرِ لَمَعَانٍ كَشَبَهِ النُّحَاسِ اللاَّمِعِ.
\par 3 وَمَدَّ شَبَهَ يَدٍ وَأَخَذَنِي بِنَاصِيَةِ رَأْسِي, وَرَفَعَنِي رُوحٌ بَيْنَ الأَرْضِ وَالسَّمَاءِ, وَأَتَى بِي فِي رُؤَى اللَّهِ إِلَى أُورُشَلِيمَ إِلَى مَدْخَلِ الْبَابِ الدَّاخِلِيِّ الْمُتَّجِهِ نَحْوَ الشِّمَالِ حَيْثُ مَجْلِسُ تِمْثَالِ الْغَيْرَةِ, الْمُهَيِّجِ الْغَيْرَةِ.
\par 4 وَإِذَا مَجْدُ إِلَهِ إِسْرَائِيلَ هُنَاكَ مِثْلُ الرُّؤْيَا الَّتِي رَأَيْتُهَا فِي الْبُقْعَةِ.
\par 5 ثُمَّ قَالَ لِي: [يَا ابْنَ آدَمَ, ارْفَعْ عَيْنَيْكَ نَحْوَ طَرِيقِ الشِّمَالِ». فَرَفَعْتُ عَيْنَيَّ نَحْوَ طَرِيقِ الشِّمَالِ وَإِذَا مِنْ شِمَالِيِّ بَابِ الْمَذْبَحِ تِمْثَالُ الْغَيْرَةِ هَذَا فِي الْمَدْخَلِ.
\par 6 وَقَالَ لِي: [يَا ابْنَ آدَمَ, هَلْ رَأَيْتَ مَا هُمْ عَامِلُونَ؟ الرَّجَاسَاتِ الْعَظِيمَةَ الَّتِي بَيْتُ إِسْرَائِيلَ عَامِلُهَا هُنَا لإِبْعَادِي عَنْ مَقْدِسِي. وَبَعْدُ تَعُودُ تَنْظُرُ رَجَاسَاتٍ أَعْظَمَ».
\par 7 ثُمَّ جَاءَ بِي إِلَى بَابِ الدَّارِ فَنَظَرْتُ وَإِذَا ثَقْبٌ فِي الْحَائِطِ.
\par 8 ثُمَّ قَالَ لِي: [يَا ابْنَ آدَمَ, انْقُبْ فِي الْحَائِطِ». فَنَقَبْتُ فِي الْحَائِطِ, فَإِذَا بَابٌ.
\par 9 وَقَالَ لِي: [ادْخُلْ وَانْظُرِ الرَّجَاسَاتِ الشِّرِّيرَةَ الَّتِي هُمْ عَامِلُوهَا هُنَا».
\par 10 فَدَخَلْتُ وَنَظَرْتُ وَإِذَا كُلُّ شَكْلِ دَبَّابَاتٍ وَحَيَوَانٍ نَجِسٍ, وَكُلُّ أَصْنَامِ بَيْتِ إِسْرَائِيلَ, مَرْسُومَةٌ عَلَى الْحَائِطِ عَلَى دَائِرِهِ.
\par 11 وَوَاقِفٌ قُدَّامَهَا سَبْعُونَ رَجُلاً مِنْ شُيُوخِ بَيْتِ إِسْرَائِيلَ, وَيَازَنْيَا بْنُ شَافَانَ قَائِمٌ فِي وَسَطِهِمْ, وَكُلُّ وَاحِدٍ مِجْمَرَتُهُ فِي يَدِهِ وَعِطْرُ عَنَانِ الْبَخُورِ صَاعِدٌ.
\par 12 ثُمَّ قَالَ لِي: [أَرَأَيْتَ يَا ابْنَ آدَمَ مَا تَفْعَلُهُ شُيُوخُ بَيْتِ إِسْرَائِيلَ فِي الظَّلاَمِ, كُلُّ وَاحِدٍ فِي مَخَادِعِ تَصَاوِيرِهِ؟ لأَنَّهُمْ يَقُولُونَ: الرَّبُّ لاَ يَرَانَا! الرَّبُّ قَدْ تَرَكَ الأَرْضَ!].
\par 13 وَقَالَ لِي: [بَعْدُ تَعُودُ تَنْظُرُ رَجَاسَاتٍ أَعْظَمَ هُمْ عَامِلُوهَا».
\par 14 فَجَاءَ بِي إِلَى مَدْخَلِ بَابِ بَيْتِ الرَّبِّ الَّذِي مِنْ جِهَةِ الشِّمَالِ, وَإِذَا هُنَاكَ نِسْوَةٌ جَالِسَاتٌ يَبْكِينَ عَلَى تَمُّوزَ.
\par 15 فَقَالَ لِي: [أَرَأَيْتَ هَذَا يَا ابْنَ آدَمَ؟ بَعْدُ تَعُودُ تَنْظُرُ رَجَاسَاتٍ أَعْظَمَ مِنْ هَذِهِ].
\par 16 فَجَاءَ بِي إِلَى دَارِ بَيْتِ الرَّبِّ الدَّاخِلِيَّةِ, وَإِذَا عِنْدَ بَابِ هَيْكَلِ الرَّبِّ بَيْنَ الرِّوَاقِ وَالْمَذْبَحِ نَحْوُ خَمْسَةٍ وَعِشْرِينَ رَجُلاً ظُهُورُهُمْ نَحْوَ هَيْكَلِ الرَّبِّ وَوُجُوهُهُمْ نَحْوَ الشَّرْقِ, وَهُمْ سَاجِدُونَ لِلشَّمْسِ نَحْوَ الشَّرْقِ.
\par 17 وَقَالَ لِي: [أَرَأَيْتَ يَا ابْنَ آدَمَ؟ أَقَلِيلٌ لِبَيْتِ يَهُوذَا عَمَلُ الرَّجَاسَاتِ الَّتِي عَمِلُوهَا هُنَا؟ لأَنَّهُمْ قَدْ مَلأُوا الأَرْضَ ظُلْماً وَيَعُودُونَ لإِغَاظَتِي, وَهَا هُمْ يُقَرِّبُونَ الْغُصْنَ إِلَى أَنْفِهِمْ.
\par 18 فَأَنَا أَيْضاً أُعَامِلُ بِالْغَضَبِ. لاَ تُشْفِقُ عَيْنِي وَلاَ أَعْفُو. وَإِنْ صَرَخُوا فِي أُذُنَيَّ بِصَوْتٍ عَالٍ لاَ أَسْمَعُهُمْ].

\chapter{9}

\par 1 وَصَرَخَ فِي سَمْعِي بِصَوْتٍ عَالٍ: [قَرِّبْ وُكَلاَءَ الْمَدِينَةِ, كُلَّ وَاحِدٍ وَعُدَّتَهُ الْمُهْلِكَةَ بِيَدِهِ».
\par 2 وَإِذَا بِسِتَّةِ رِجَالٍ مُقْبِلِينَ مِنْ طَرِيقِ الْبَابِ الأَعْلَى الَّذِي هُوَ مِنْ جِهَةِ الشِّمَالِ, وَكُلُّ وَاحِدٍ عُدَّتُهُ السَّاحِقَةُ بِيَدِهِ, وَفِي وَسَطِهِمْ رَجُلٌ لاَبِسٌ الْكَتَّانَ, وَعَلَى جَانِبِهِ دَوَاةُ كَاتِبٍ. فَدَخَلُوا وَوَقَفُوا جَانِبَ مَذْبَحِ النُّحَاسِ.
\par 3 وَمَجْدُ إِلَهِ إِسْرَائِيلَ صَعِدَ عَنِ الْكَرُوبِ الَّذِي كَانَ عَلَيْهِ إِلَى عَتَبَةِ الْبَيْتِ. فَدَعَا الرَّبُّ الرَّجُلَ اللاَّبِسَ الْكَتَّانِ الَّذِي دَوَاةُ الْكَاتِبِ عَلَى جَانِبِهِ,
\par 4 وَقَالَ لَهُ: [اعْبُرْ فِي وَسَطِ الْمَدِينَةِ أُورُشَلِيمَ, وَسِمْ سِمَةً عَلَى جِبَاهِ الرِّجَالِ الَّذِينَ يَئِنُّونَ وَيَتَنَهَّدُونَ عَلَى كُلِّ الرَّجَاسَاتِ الْمَصْنُوعَةِ فِي وَسَطِهَا».
\par 5 وَقَالَ لأُولَئِكَ فِي سَمْعِي: [اعْبُرُوا فِي الْمَدِينَةِ وَرَاءَهُ وَاضْرِبُوا. لاَ تُشْفِقْ أَعْيُنُكُمْ وَلاَ تَعْفُوا.
\par 6 اَلشَّيْخَ وَالشَّابَّ وَالْعَذْرَاءَ وَالطِّفْلَ وَالنِّسَاءَ. اقْتُلُوا لِلْهَلاَكِ. وَلاَ تَقْرُبُوا مِنْ إِنْسَانٍ عَلَيْهِ السِّمَةُ, وَابْتَدِئُوا مِنْ مَقْدِسِي». فَابْتَدَأُوا بِالرِّجَالِ الشُّيُوخِ الَّذِينَ أَمَامَ الْبَيْتِ.
\par 7 وَقَالَ لَهُمْ: [نَجِّسُوا الْبَيْتَ, وَامْلأُوا الدُّورَ قَتْلَى. اخْرُجُوا». فَخَرَجُوا وَقَتَلُوا فِي الْمَدِينَةِ.
\par 8 وَكَانَ بَيْنَمَا هُمْ يَقْتُلُونَ وَأُبْقِيتُ أَنَا, أَنِّي خَرَرْتُ عَلَى وَجْهِي وَصَرَخْتُ: [آهِ يَا سَيِّدُ الرَّبُّ! هَلْ أَنْتَ مُهْلِكٌ بَقِيَّةَ إِسْرَائِيلَ كُلَّهَا بِصَبِّ رِجْزِكَ عَلَى أُورُشَلِيمَ؟»
\par 9 فَقَالَ لِي: [إِنَّ إِثْمَ بَيْتِ إِسْرَائِيلَ وَيَهُوذَا عَظِيمٌ جِدّاً جِدّاً, وَقَدِ امْتَلأَتِ الأَرْضُ دِمَاءً, وَامْتَلأَتِ الْمَدِينَةُ جَنَفاً. لأَنَّهُمْ يَقُولُونَ: الرَّبُّ قَدْ تَرَكَ الأَرْضَ, وَالرَّبُّ لاَ يَرَى.
\par 10 وَأَنَا أَيْضاً عَيْنِي لاَ تُشْفِقُ وَلاَ أَعْفُو. أَجْلِبُ طَرِيقَهُمْ عَلَى رُؤُوسِهِمْ».
\par 11 وَإِذَا بِالرَّجُلِ اللاَّبِسِ الْكَتَّانِ الَّذِي الدَّوَاةُ عَلَى جَانِبِهِ رَدَّ قَائِلاً: [قَدْ فَعَلْتُ كَمَا أَمَرْتَنِي].

\chapter{10}

\par 1 ثُمَّ نَظَرْتُ وَإِذَا عَلَى الْمُقَبَّبِ الَّذِي عَلَى رَأْسِ الْكَرُوبِيمِ شَيْءٌ كَحَجَرِ الْعَقِيقِ الأَزْرَقِ, كَمَنْظَرِ شَبَهِ عَرْشٍ.
\par 2 وَقَالَ لِلرَّجُلِ اللاَّبِسِ الْكَتَّانِ: [ادْخُلْ بَيْنَ الْبَكَرَاتِ تَحْتَ الْكَرُوبِ وَامْلأْ حُفْنَتَيْكَ جَمْرَ نَارٍ مِنْ بَيْنِ الْكَرُوبِيمِ وَذَرِّهَا عَلَى الْمَدِينَةِ». فَدَخَلَ قُدَّامَ عَيْنَيَّ.
\par 3 وَالْكَرُوبِيمُ وَاقِفُونَ عَنْ يَمِينِ الْبَيْتِ حِينَ دَخَلَ الرَّجُلُ, وَالسَّحَابَةُ مَلأَتِ الدَّارَ الدَّاخِلِيَّةَ.
\par 4 فَارْتَفَعَ مَجْدُ الرَّبِّ عَنِ الْكَرُوبِ إِلَى عَتَبَةِ الْبَيْتِ. فَامْتَلأَ الْبَيْتُ مِنَ السَّحَابَةِ, وَامْتَلأَتِ الدَّارُ مِنْ لَمَعَانِ مَجْدِ الرَّبِّ.
\par 5 وَسُمِعَ صَوْتُ أَجْنِحَةِ الْكَرُوبِيمِ إِلَى الدَّارِ الْخَارِجِيَّةِ كَصَوْتِ اللَّهِ الْقَدِيرِ إِذَا تَكَلَّمَ.
\par 6 وَكَانَ لَمَّا أَمَرَ الرَّجُلَ اللاَّبِسَ الْكَتَّانِ: [خُذْ نَاراً مِنْ بَيْنِ الْبَكَرَاتِ مِنْ بَيْنِ الْكَرُوبِيمِ» أَنَّهُ دَخَلَ وَوَقَفَ بِجَانِبِ الْبَكَرَةِ.
\par 7 وَمَدَّ كَرُوبٌ يَدَهُ مِنْ بَيْنِ الْكَرُوبِيمِ, إِلَى النَّارِ الَّتِي بَيْنَ الْكَرُوبِيمِ فَرَفَعَ مِنْهَا وَوَضَعَهَا فِي حُفْنَتَيِ اللاَّبِسِ الْكَتَّانِ, فَأَخَذَهَا وَخَرَجَ.
\par 8 فَظَهَرَ فِي الْكَرُوبِيمِ شَبَهُ يَدِ إِنْسَانٍ مِنْ تَحْتِ أَجْنِحَتِهَا.
\par 9 وَنَظَرْتُ وَإِذَا أَرْبَعُ بَكَرَاتٍ بِجَانِبِ الْكَرُوبِيمِ. بَكَرَةٌ وَاحِدَةٌ بِجَانِبِ الْكَرُوبِ الْوَاحِدِ, وَبَكَرَةٌ أُخْرَى بِجَانِبِ الْكَرُوبِ الآخَرِ. وَمَنْظَرُ الْبَكَرَاتِ كَشَبَهِ حَجَرِ الزَّبَرْجَدِ.
\par 10 وَمَنْظَرُهُنَّ شَكْلٌ وَاحِدٌ لِلأَرْبَعِ. كَأَنَّهُ كَانَ بَكَرَةً وَسَطَ بَكَرَةٍ.
\par 11 لَمَّا سَارَتْ سَارَتْ عَلَى جَوَانِبِهَا الأَرْبَعَةِ. لَمْ تَدُرْ عِنْدَ سَيْرِهَا. بَلْ إِلَى الْمَوْضِعِ الَّذِي تَوَجَّهَ إِلَيْهِ الرَّأْسُ ذَهَبَتْ وَرَاءَهُ. لَمْ تَدُرْ عِنْدَ سَيْرِهَا.
\par 12 وَكُلُّ جِسْمِهَا وَظُهُورِهَا وَأَيْدِيهَا وَأَجْنِحَتِهَا وَالْبَكَرَاتِ مَلآنَةٌ عُيُوناً حَوَالَيْهَا لِبَكَرَاتِهَا الأَرْبَعِ.
\par 13 أَمَّا الْبَكَرَاتُ فَنُودِيَ إِلَيْهَا فِي سَمَاعِي: [يَا بَكَرَةُ».
\par 14 وَلِكُلِّ وَاحِدٍ أَرْبَعَةُ أَوْجُهٍ. الْوَجْهُ الأَوَّلُ وَجْهُ كَرُوبٍ, وَالْوَجْهُ الثَّانِي وَجْهُ إِنْسَانٍ, وَالثَّالِثُ وَجْهُ أَسَدٍ, وَالرَّابِعُ وَجْهُ نَسْرٍ.
\par 15 ثُمَّ صَعِدَ الْكَرُوبِيمُ. هَذَا هُوَ الْحَيَوَانُ الَّذِي رَأَيْتُهُ عِنْدَ نَهْرِ خَابُورَ.
\par 16 وَعِنْدَ سَيْرِ الْكَرُوبِيمِ سَارَتِ الْبَكَرَاتُ بِجَانِبِهَا, وَعِنْدَ رَفْعِ الْكَرُوبِيمِ أَجْنِحَتَهَا لِلاِرْتِفَاعِ عَنِ الأَرْضِ لَمْ تَدُرِ الْبَكَرَاتُ أَيْضاً عَنْ جَانِبِهَا.
\par 17 عِنْدَ وُقُوفِهَا وَقَفَتْ هَذِهِ, وَعِنْدَ ارْتِفَاعِهَا ارْتَفَعَتْ مَعَهَا, لأَنَّ فِيهَا رُوحَ الْحَيَوَانِ.
\par 18 وَخَرَجَ مَجْدُ الرَّبِّ مِنْ عَلَى عَتَبَةِ الْبَيْتِ وَوَقَفَ عَلَى الْكَرُوبِيمِ.
\par 19 فَرَفَعَتِ الْكَرُوبِيمُ أَجْنِحَتَهَا وَصَعِدَتْ عَنِ الأَرْضِ قُدَّامَ عَيْنَيَّ. عِنْدَ خُرُوجِهَا كَانَتِ الْبَكَرَاتُ مَعَهَا, وَوَقَفَتْ عِنْدَ مَدْخَلِ بَابِ بَيْتِ الرَّبِّ الشَّرْقِيِّ, وَمَجْدُ إِلَهِ إِسْرَائِيلَ عَلَيْهَا مِنْ فَوْقُ.
\par 20 هَذَا هُوَ الْحَيَوَانُ الَّذِي رَأَيْتُهُ تَحْتَ إِلَهِ إِسْرَائِيلَ عِنْدَ نَهْرِ خَابُورَ. وَعَلِمْتُ أَنَّهَا هِيَ الْكَرُوبِيمُ.
\par 21 لِكُلِّ وَاحِدٍ أَرْبَعَةُ أَوْجُهٍ, وَلِكُلِّ وَاحِدٍ أَرْبَعَةُ أَجْنِحَةٍ, وَشَبَهُ أَيْدِي إِنْسَانٍ تَحْتَ أَجْنِحَتِهَا.
\par 22 وَشَكْلُ وُجُوهِهَا هُوَ شَكْلُ الْوُجُوهِ الَّتِي رَأَيْتُهَا عِنْدَ نَهْرِ خَابُورَ, مَنَاظِرُهَا وَذَوَاتُهَا. كُلُّ وَاحِدٍ يَسِيرُ إِلَى جِهَةِ وَجْهِهِ

\chapter{11}

\par 1 ثُمَّ رَفَعَنِي رُوحٌ وَأَتَى بِي إِلَى بَابِ بَيْتِ الرَّبِّ الشَّرْقِيِّ الْمُتَّجِهِ نَحْوَ الشَّرْقِ, وَإِذَا عِنْدَ مَدْخَلِ الْبَابِ خَمْسَةٌ وَعِشْرُونَ رَجُلاً, وَرَأَيْتُ بَيْنَهُمْ يَازَنْيَا بْنَ عَزُورَ, وَفَلَطْيَا بْنَ بَنَايَا رَئِيسَيِ الشَّعْبِ.
\par 2 فَقَالَ لِي: [يَا ابْنَ آدَمَ, هَؤُلاَءِ هُمُ الرِّجَالُ الْمُفَكِّرُونَ بِالإِثْمِ, الْمُشِيرُونَ مَشُورَةً رَدِيئَةً فِي هَذِهِ الْمَدِينَةِ.
\par 3 اَلْقَائِلُونَ: مَا هُوَ قَرِيبٌ بِنَاءُ الْبُيُوتِ! هِيَ الْقِدْرُ وَنَحْنُ اللَّحْمُ!
\par 4 لأَجْلِ ذَلِكَ تَنَبَّأْ عَلَيْهِمْ. تَنَبَّأْ يَا ابْنَ آدَمَ].
\par 5 وَحَلَّ عَلَيَّ رُوحُ الرَّبِّ وَقَالَ لِي: [قُلْ هَكَذَا قَالَ الرَّبُّ: هَكَذَا قُلْتُمْ يَا بَيْتَ إِسْرَائِيلَ, وَمَا يَخْطُرُ بِبَالِكُمْ قَدْ عَلِمْتُهُ.
\par 6 قَدْ كَثَّرْتُمْ قَتْلاَكُمْ فِي هَذِهِ الْمَدِينَةِ وَمَلأْتُمْ أَزِقَّتَهَا بِالْقَتْلَى.
\par 7 لِذَلِكَ هَكَذَا قَالَ السَّيِّدُ الرَّبُّ: قَتْلاَكُمُ الَّذِينَ طَرَحْتُمُوهُمْ فِي وَسَطِهَا هُمُ اللَّحْمُ وَهِيَ الْقِدْرُ. وَإِيَّاكُمْ أُخْرِجُ مِنْ وَسَطِهَا.
\par 8 قَدْ فَزِعْتُمْ مِنَ السَّيْفِ, فَالسَّيْفُ أَجْلِبُهُ عَلَيْكُمْ يَقُولُ السَّيِّدُ الرَّبُّ.
\par 9 وَأُخْرِجُكُمْ مِنْ وَسَطِهَا وَأُسَلِّمُكُمْ إِلَى أَيْدِي الْغُرَبَاءِ, وَأُجْرِي فِيكُمْ أَحْكَاماً.
\par 10 بِالسَّيْفِ تَسْقُطُونَ. فِي تُخُمِ إِسْرَائِيلَ أَقْضِي عَلَيْكُمْ فَتَعْلَمُونَ أَنِّي أَنَا الرَّبُّ.
\par 11 هَذِهِ لاَ تَكُونُ لَكُمْ قِدْراً وَلاَ أَنْتُمْ تَكُونُونَ اللَّحْمَ فِي وَسَطِهَا. فِي تُخُمِ إِسْرَائِيلَ أَقْضِي عَلَيْكُمْ
\par 12 فَتَعْلَمُونَ أَنِّي أَنَا الرَّبُّ الَّذِي لَمْ تَسْلُكُوا فِي فَرَائِضِهِ وَلَمْ تَعْمَلُوا بِأَحْكَامِهِ, بَلْ عَمِلْتُمْ حَسَبَ أَحْكَامِ الأُمَمِ الَّذِينَ حَوْلَكُمْ].
\par 13 وَكَانَ لَمَّا تَنَبَّأْتُ أَنَّ فَلَطْيَا بْنَ بَنَايَا مَاتَ. فَخَرَرْتُ عَلَى وَجْهِي وَصَرَخْتُ بِصَوْتٍ عَظِيمٍ: [آهِ يَا سَيِّدُ الرَّبُّ! هَلْ تُفْنِي أَنْتَ بَقِيَّةَ إِسْرَائِيلَ؟]
\par 14 وَكَانَ إِلَيَّ كَلاَمُ الرَّبِّ:
\par 15 [يَا ابْنَ آدَمَ, إِخْوَتُكَ ذَوُو قَرَابَتِكَ, وَكُلُّ بَيْتِ إِسْرَائِيلَ بِأَجْمَعِهِ, هُمُ الَّذِينَ قَالَ لَهُمْ سُكَّانُ أُورُشَلِيمَ: ابْتَعِدُوا عَنِ الرَّبِّ. لَنَا أُعْطِيَتْ هَذِهِ الأَرْضُ مِيرَاثاً.
\par 16 لِذَلِكَ قُلْ: هَكَذَا قَالَ السَّيِّدُ الرَّبُّ: وَإِنْ كُنْتُ قَدْ أَبْعَدْتُهُمْ بَيْنَ الأُمَمِ, وَإِنْ كُنْتُ قَدْ بَدَّدْتُهُمْ فِي الأَرَاضِي, فَإِنِّي أَكُونُ لَهُمْ مَقْدِساً صَغِيراً فِي الأَرَاضِي الَّتِي يَأْتُونَ إِلَيْهَا.
\par 17 لِذَلِكَ قُلْ: هَكَذَا قَالَ السَّيِّدُ الرَّبُّ: إِنِّي أَجْمَعُكُمْ مِنْ بَيْنِ الشُّعُوبِ, وَأَحْشُرُكُمْ مِنَ الأَرَاضِي الَّتِي تَبَدَّدْتُمْ فِيهَا, وَأُعْطِيكُمْ أَرْضَ إِسْرَائِيلَ.
\par 18 فَيَأْتُونَ إِلَى هُنَاكَ وَيُزِيلُونَ جَمِيعَ مَكْرُهَاتِهَا وَجَمِيعَ رَجَاسَاتِهَا مِنْهَا.
\par 19 وَأُعْطِيهِمْ قَلْباً وَاحِداً, وَأَجْعَلُ فِي دَاخِلِكُمْ رُوحاً جَدِيداً, وَأَنْزِعُ قَلْبَ الْحَجَرِ مِنْ لَحْمِهِمْ وَأُعْطِيهِمْ قَلْبَ لَحْمٍ
\par 20 لِيَسْلُكُوا فِي فَرَائِضِي وَيَحْفَظُوا أَحْكَامِي وَيَعْمَلُوا بِهَا, وَيَكُونُوا لِي شَعْباً فَأَنَا أَكُونُ لَهُمْ إِلَهاً.
\par 21 أَمَّا الَّذِينَ قَلْبُهُمْ ذَاهِبٌ وَرَاءَ قَلْبِ مَكْرُهَاتِهِمْ وَرَجَاسَاتِهِمْ, فَإِنِّي أَجْلِبُ طَرِيقَهُمْ عَلَى رُؤُوسِهِمْ يَقُولُ السَّيِّدُ الرَّبُّ].
\par 22 ثُمَّ رَفَعَتِ الْكَرُوبِيمُ أَجْنِحَتَهَا وَالْبَكَرَاتِ مَعَهَا وَمَجْدُ إِلَهِ إِسْرَائِيلَ عَلَيْهَا مِنْ فَوْقُ.
\par 23 وَصَعِدَ مَجْدُ الرَّبِّ مِنْ عَلَى وَسَطِ الْمَدِينَةِ وَوَقَفَ عَلَى الْجَبَلِ الَّذِي عَلَى شَرْقِيِّ الْمَدِينَةِ.
\par 24 وَحَمَلَنِي رُوحٌ وَجَاءَ بِي فِي الرُّؤْيَا بِرُوحِ اللَّهِ إِلَى أَرْضِ الْكِلْدَانِيِّينَ إِلَى الْمَسْبِيِّينَ. فَصَعِدَتْ عَنِّي الرُّؤْيَا الَّتِي رَأَيْتُهَا.
\par 25 فَكَلَّمْتُ الْمَسْبِيِّينَ بِكُلِّ كَلاَمِ الرَّبِّ الَّذِي أَرَانِي إِيَّاهُ

\chapter{12}

\par 1 وَكَانَ إِلَيَّ كَلاَمُ الرَّبِّ:
\par 2 [يَا ابْنَ آدَمَ, أَنْتَ سَاكِنٌ فِي وَسَطِ بَيْتٍ مُتَمَرِّدٍ, الَّذِينَ لَهُمْ أَعْيُنٌ لِيَنْظُرُوا وَلاَ يَنْظُرُونَ. لَهُمْ آذَانٌ لِيَسْمَعُوا وَلاَ يَسْمَعُونَ لأَنَّهُمْ بَيْتٌ مُتَمَرِّدٌ.
\par 3 وَأَنْتَ يَا ابْنَ آدَمَ فَهَيِّئْ لِنَفْسِكَ أُهْبَةَ جَلاَءٍ, وَارْتَحِلْ قُدَّامَ عُيُونِهِمْ نَهَاراً, وَارْتَحِلْ مِنْ مَكَانِكَ إِلَى مَكَانٍ آخَرَ قُدَّامَ عُيُونِهِمْ لَعَلَّهُمْ يَنْظُرُونَ أَنَّهُمْ بَيْتٌ مُتَمَرِّدٌ.
\par 4 فَتُخْرِجُ أُهْبَتَكَ كَأُهْبَةِ الْجَلاَءِ قُدَّامَ عُيُونِهِمْ نَهَاراً, وَأَنْتَ تَخْرُجُ مَسَاءً قُدَّامَ عُيُونِهِمْ كَالْخَارِجِينَ إِلَى الْجَلاَءِ.
\par 5 وَانْقُبْ لِنَفْسِكَ فِي الْحَائِطِ قُدَّامَ عُيُونِهِمْ وَأَخْرِجْهَا مِنْهُ.
\par 6 وَاحْمِلْ عَلَى كَتِفِكَ قُدَّامَ عُيُونِهِمْ. فِي الْعَتَمَةِ تُخْرِجُهَا. تُغَطِّي وَجْهَكَ فَلاَ تَرَى الأَرْضَ. لأَنِّي جَعَلْتُكَ آيَةً لِبَيْتِ إِسْرَائِيلَ».
\par 7 فَفَعَلْتُ هَكَذَا كَمَا أُمِرْتُ, فَأَخْرَجْتُ أُهْبَتِي كَأُهْبَةِ الْجَلاَءِ نَهَاراً, وَفِي الْمَسَاءِ نَقَبْتُ لِنَفْسِي فِي الْحَائِطِ بِيَدِي, وَأَخْرَجْتُ فِي الْعَتَمَةِ وَحَمَلْتُ عَلَى كَتِفِي قُدَّامَ عُيُونِهِمْ.
\par 8 وَفِي الصَّبَاحِ كَانَتْ إِلَيَّ كَلِمَةُ الرَّبِّ:
\par 9 [يَا ابْنَ آدَمَ, أَلَمْ يَقُلْ لَكَ بَيْتُ إِسْرَائِيلَ, الْبَيْتُ الْمُتَمَرِّدُ: مَاذَا تَصْنَعُ؟
\par 10 قُلْ لَهُمْ: هَكَذَا قَالَ السَّيِّدُ الرَّبُّ. هَذَا الْوَحْيُ هُوَ الرَّئِيسُ فِي أُورُشَلِيمَ وَكُلِّ بَيْتِ إِسْرَائِيلَ وَالَّذِينَ هُمْ فِي وَسَطِهِمْ.
\par 11 قُلْ: أَنَا آيَةٌ لَكُمْ. كَمَا صَنَعْتُ هَكَذَا يُصْنَعُ بِهِمْ. إِلَى الْجَلاَءِ إِلَى السَّبْيِ يَذْهَبُونَ.
\par 12 وَالرَّئِيسُ الَّذِي فِي وَسَطِهِمْ يَحْمِلُ عَلَى الْكَتِفِ فِي الْعَتَمَةِ وَيَخْرُجُ. يَنْقُبُونَ فِي الْحَائِطِ لِيُخْرِجُوا مِنْهُ. يُغَطِّي وَجْهَهُ لِكَيْلاَ يَنْظُرَ الأَرْضَ بِعَيْنَيْهِ.
\par 13 وَأَبْسُطُ شَبَكَتِي عَلَيْهِ فَيُؤْخَذُ فِي شَرَكِي وَآتِي بِهِ إِلَى بَابِلَ إِلَى أَرْضِ الْكِلْدَانِيِّينَ, وَلَكِنْ لاَ يَرَاهَا وَهُنَاكَ يَمُوتُ.
\par 14 وَأُذَرِّي فِي كُلِّ رِيحٍ جَمِيعَ الَّذِينَ حَوْلَهُ لِنَصْرِهِ, وَكُلَّ جُيُوشِهِ, وَأَسْتَلُّ السَّيْفَ وَرَاءَهُمْ.
\par 15 فَيَعْلَمُونَ أَنِّي أَنَا الرَّبُّ حِينَ أُبَدِّدُهُمْ بَيْنَ الأُمَمِ وَأُذَرِّيهِمْ فِي الأَرَاضِي.
\par 16 وَأُبْقِي مِنْهُمْ رِجَالاً مَعْدُودِينَ مِنَ السَّيْفِ وَمِنَ الْجُوعِ وَمِنَ الْوَبَإِ, لِيُحَدِّثُوا بِكُلِّ رَجَاسَاتِهِمْ بَيْنَ الأُمَمِ الَّتِي يَأْتُونَ إِلَيْهَا, فَيَعْلَمُونَ أَنِّي أَنَا الرَّبُّ].
\par 17 وَكَانَتْ إِلَيَّ كَلِمَةُ الرَّبِّ:
\par 18 [يَا ابْنَ آدَمَ, كُلْ خُبْزَكَ بِارْتِعَاشٍ, وَاشْرَبْ مَاءَكَ بِارْتِعَادٍ وَغَمٍّ.
\par 19 وَقُلْ لِشَعْبِ الأَرْضِ: هَكَذَا قَالَ السَّيِّدُ الرَّبُّ عَلَى سُكَّانِ أُورُشَلِيمَ فِي أَرْضِ إِسْرَائِيلَ: يَأْكُلُونَ خُبْزَهُمْ بِالْغَمِّ, وَيَشْرَبُونَ مَاءَهُمْ بِحَيْرَةٍ, لِتَخْرَبَ أَرْضُهَا عَنْ مِلْئِهَا مِنْ ظُلْمِ كُلِّ السَّاكِنِينَ فِيهَا.
\par 20 وَالْمُدُنُ الْمَسْكُونَةُ تَخْرَبُ, وَالأَرْضُ تُقْفِرُ, فَتَعْلَمُونَ أَنِّي أَنَا الرَّبُّ].
\par 21 وَكَانَ إِلَيَّ كَلاَمُ الرَّبِّ:
\par 22 [يَا ابْنَ آدَمَ, مَا هَذَا الْمَثَلُ الَّذِي لَكُمْ عَلَى أَرْضِ إِسْرَائِيلَ, الْقَائِلُ: قَدْ طَالَتِ الأَيَّامُ وَخَابَتْ كُلُّ رُؤْيَا.
\par 23 لِذَلِكَ قُلْ لَهُمْ: هَكَذَا قَالَ السَّيِّدُ الرَّبُّ: أُبَطِّلُ هَذَا الْمَثَلَ فَلاَ يُمَثِّلُونَ بِهِ بَعْدُ فِي إِسْرَائِيلَ. بَلْ قُلْ لَهُمْ: قَدِ اقْتَرَبَتِ الأَيَّامُ وَكَلاَمُ كُلِّ رُؤْيَا.
\par 24 لأَنَّهُ لاَ تَكُونُ بَعْدُ رُؤْيَا بَاطِلَةٌ وَلاَ عِرَافَةٌ مَلِقَةٌ فِي وَسَطِ بَيْتِ إِسْرَائِيلَ.
\par 25 لأَنِّي أَنَا الرَّبُّ أَتَكَلَّمُ, وَالْكَلِمَةُ الَّتِي أَتَكَلَّمُ بِهَا تَكُونُ. لاَ تَطُولُ بَعْدُ. لأَنِّي فِي أَيَّامِكُمْ أَيُّهَا الْبَيْتُ الْمُتَمَرِّدُ أَقُولُ الْكَلِمَةَ وَأُجْرِيهَا, يَقُولُ السَّيِّدُ الرَّبُّ].
\par 26 وَكَانَ إِلَيَّ كَلاَمُ الرَّبِّ:
\par 27 [يَا ابْنَ آدَمَ, هُوَذَا بَيْتُ إِسْرَائِيلَ قَائِلُونَ: الرُّؤْيَا الَّتِي هُوَ رَائِيهَا هِيَ إِلَى أَيَّامٍ كَثِيرَةٍ, وَهُوَ مُتَنَبِّئٌ لأَزْمِنَةٍ بَعِيدَةٍ.
\par 28 لِذَلِكَ قُلْ لَهُمْ: هَكَذَا قَالَ السَّيِّدُ الرَّبُّ: لاَ يَطُولُ بَعْدُ شَيْءٌ مِنْ كَلاَمِي. الْكَلِمَةُ الَّتِي تَكَلَّمْتُ بِهَا تَكُونُ, يَقُولُ السَّيِّدُ الرَّبُّ].

\chapter{13}

\par 1 وَكَانَ إِلَيَّ كَلاَمُ الرَّبِّ:
\par 2 [يَا ابْنَ آدَمَ, تَنَبَّأْ عَلَى أَنْبِيَاءِ إِسْرَائِيلَ الَّذِينَ يَتَنَبَّأُونَ وَقُلْ لِلَّذِينَ هُمْ أَنْبِيَاءُ مِنْ تِلْقَاءِ ذَوَاتِهِمِ: اسْمَعُوا كَلِمَةَ الرَّبِّ.
\par 3 هَكَذَا قَالَ السَّيِّدُ الرَّبُّ: وَيْلٌ لِلأَنْبِيَاءِ الْحَمْقَى الذَّاهِبِينَ وَرَاءَ رُوحِهِمْ وَلَمْ يَرُوا شَيْئاً.
\par 4 أَنْبِيَاؤُكَ يَا إِسْرَائِيلُ صَارُوا كَالثَّعَالِبِ فِي الْخِرَبِ.
\par 5 لَمْ تَصْعَدُوا إِلَى الثُّغَرِ, وَلَمْ تَبْنُوا جِدَاراً لِبَيْتِ إِسْرَائِيلَ لِلْوُقُوفِ فِي الْحَرْبِ فِي يَوْمِ الرَّبِّ.
\par 6 رَأُوا بَاطِلاً وَعِرَافَةً كَاذِبَةً. الْقَائِلُونَ: وَحْيُ الرَّبِّ وَالرَّبُّ لَمْ يُرْسِلْهُمْ, وَانْتَظَرُوا إِثْبَاتَ الْكَلِمَةِ.
\par 7 أَلَمْ تَرُوا رُؤْيَا بَاطِلَةً, وَتَكَلَّمْتُمْ بِعِرَافَةٍ كَاذِبَةٍ, قَائِلِينَ: وَحْيُ الرَّبِّ وَأَنَا لَمْ أَتَكَلَّمْ؟
\par 8 لِذَلِكَ هَكَذَا قَالَ السَّيِّدُ الرَّبُّ: لأَنَّكُمْ تَكَلَّمْتُمْ بِالْبَاطِلِ وَرَأَيْتُمْ كَذِباً, فَلِذَلِكَ هَا أَنَا عَلَيْكُمْ يَقُولُ السَّيِّدُ الرَّبُّ.
\par 9 وَتَكُونُ يَدِي عَلَى الأَنْبِيَاءِ الَّذِينَ يَرُونَ الْبَاطِلَ وَالَّذِينَ يَعْرِفُونَ بِالْكَذِبِ. فِي مَجْلِسِ شَعْبِي لاَ يَكُونُونَ, وَفِي كِتَابِ بَيْتِ إِسْرَائِيلَ لاَ يُكْتَبُونَ, وَإِلَى أَرْضِ إِسْرَائِيلَ لاَ يَدْخُلُونَ, فَتَعْلَمُونَ أَنِّي أَنَا السَّيِّدُ الرَّبُّ.
\par 10 مِنْ أَجْلِ أَنَّهُمْ أَضَلُّوا شَعْبِي قَائِلِينَ: سَلاَمٌ وَلَيْسَ سَلاَمٌ, وَوَاحِدٌ مِنْهُمْ يَبْنِي حَائِطاً وَهَا هُمْ يُمَلِّطُونَهُ بِالطُّفَالِ.
\par 11 فَقُلْ لِلَّذِينَ يُمَلِّطُونَهُ بِالطُّفَالِ إِنَّهُ يَسْقُطُ. يَكُونُ مَطَرٌ جَارِفٌ, وَأَنْتُنَّ يَا حِجَارَةَ الْبَرَدِ تَسْقُطْنَ, وَرِيحٌ عَاصِفَةٌ تُشَقِّقُهُ.
\par 12 وَهُوَذَا إِذَا سَقَطَ الْحَائِطُ, أَفَلاَ يُقَالُ لَكُمْ: أَيْنَ الطِّينُ الَّذِي طَيَّنْتُمْ بِهِ؟
\par 13 لِذَلِكَ هَكَذَا قَالَ السَّيِّدُ الرَّبُّ: إِنِّي أُشَقِّقُهُ بِرِيحٍ عَاصِفَةٍ فِي غَضَبِي, وَيَكُونُ مَطَرٌ جَارِفٌ فِي سَخَطِي وَحِجَارَةُ بَرَدٍ فِي غَيْظِي لإِفْنَائِهِ.
\par 14 فَأَهْدِمُ الْحَائِطَ الَّذِي مَلَّطْتُمُوهُ بِالطُّفَالِ, وَأُلْصِقُهُ بِالأَرْضِ, وَيَنْكَشِفُ أَسَاسُهُ فَيَسْقُطُ, وَتَفْنُونَ أَنْتُمْ فِي وَسَطِهِ, فَتَعْلَمُونَ أَنِّي أَنَا الرَّبُّ.
\par 15 فَأُتِمُّ غَضَبِي عَلَى الْحَائِطِ وَعَلَى الَّذِينَ مَلَّطُوهُ بِالطُّفَالِ, وَأَقُولُ لَكُمْ: لَيْسَ الْحَائِطُ بِمَوْجُودٍ وَلاَ الَّذِينَ مَلَّطُوهُ!
\par 16 (أَيْ أَنْبِيَاءُ إِسْرَائِيلَ الَّذِينَ يَتَنَبَّأُونَ لأُورُشَلِيمَ وَيَرُونَ لَهَا رُؤَى سَلاَمٍ, وَلاَ سَلاَمَ) يَقُولُ السَّيِّدُ الرَّبّ!
\par 17 [وَأَنْتَ يَا ابْنَ آدَمَ, فَاجْعَلْ وَجْهَكَ ضِدَّ بَنَاتِ شَعْبِكَ اللَّوَاتِي يَتَنَبَّأْنَ مِنْ تِلْقَاءِ ذَوَاتِهِنَّ وَتَنَبَّأْ عَلَيْهِنَّ,
\par 18 وَقُلْ: هَكَذَا قَالَ السَّيِّدُ الرَّبُّ: وَيْلٌ لِلَّوَاتِي يَخُطْنَ وَسَائِدَ لِكُلِّ أَوْصَالِ الأَيْدِي, وَيَصْنَعْنَ مِخَدَّاتٍ لِرَأْسِ كُلِّ قَامَةٍ لاِصْطِيَادِ النُّفُوسِ. أَفَتَصْطَدْنَ نُفُوسَ شَعْبِي وَتَسْتَحْيِينَ أَنْفُسَكُنَّ,
\par 19 وَتُنَجِّسْنَنِي عِنْدَ شَعْبِي لأَجْلِ حُفْنَةِ شَعِيرٍ وَلأَجْلِ فُتَاتٍ مِنَ الْخُبْزِ, لإِمَاتَةِ نُفُوسٍ لاَ يَنْبَغِي أَنْ تَمُوتَ, وَاسْتِحْيَاءِ نُفُوسٍ لاَ يَنْبَغِي أَنْ تَحْيَا, بِكِذْبِكُنَّ عَلَى شَعْبِي السَّامِعِينَ لِلْكَذِبِ؟
\par 20 [لِذَلِكَ هَكَذَا قَالَ السَّيِّدُ الرَّبُّ: هَا أَنَا ضِدُّ وَسَائِدِكُنَّ الَّتِي تَصْطَدْنَ بِهَا النُّفُوسَ كَالْفِرَاخِ, وَأُمَزِّقُهَا عَنْ أَذْرُعِكُنَّ وَأُطْلِقُ النُّفُوسَ الَّتِي تَصْطَدْنَهَا كَالْفِرَاخِ.
\par 21 وَأُمَزِّقُ مِخَدَّاتِكُنَّ وَأُنْقِذُ شَعْبِي مِنْ أَيْدِيكُنَّ, فَلاَ يَكُونُونَ بَعْدُ فِي أَيْدِيكُنَّ لِلصَّيْدِ, فَتَعْلَمْنَ أَنِّي أَنَا الرَّبُّ.
\par 22 لأَنَّكُنَّ أَحْزَنْتُنَّ قَلْبَ الصِّدِّيقِ كَذِباً وَأَنَا لَمْ أُحْزِنْهُ, وَشَدَّدْتُنَّ أَيْدِي الشِّرِّيرِ حَتَّى لاَ يَرْجِعَ عَنْ طَرِيقِهِ الرَّدِيئَةِ فَيَحْيَا
\par 23 فَلِذَلِكَ لَنْ تَعُدْنَ تَرَيْنَ الْبَاطِلَ وَلاَ تَعْرِفْنَ عِرَافَةً بَعْدُ, وَأُنْقِذُ شَعْبِي مِنْ أَيْدِيكُنَّ فَتَعْلَمْنَ أَنِّي أَنَا الرَّبُّ

\chapter{14}

\par 1 فَجَاءَ إِلَيَّ رِجَالٌ مِنْ شُيُوخِ إِسْرَائِيلَ وَجَلَسُوا أَمَامِي.
\par 2 فَصَارَتْ إِلَيَّ كَلِمَةُ الرَّبِّ:
\par 3 [يَا ابْنَ آدَمَ, هَؤُلاَءِ الرِّجَالُ قَدْ أَصْعَدُوا أَصْنَامَهُمْ إِلَى قُلُوبِهِمْ, وَوَضَعُوا مَعْثَرَةَ إِثْمِهِمْ تِلْقَاءَ أَوْجُهِهِمْ. فَهَلْ أُسْأَلُ مِنْهُمْ سُؤَالاً؟
\par 4 لأَجْلِ ذَلِكَ كَلِّمْهُمْ وَقُلْ لَهُمْ: هَكَذَا قَالَ السَّيِّدُ الرَّبُّ: كُلُّ إِنْسَانٍ مِنْ بَيْتِ إِسْرَائِيلَ الَّذِي يُصْعِدُ أَصْنَامَهُ إِلَى قَلْبِهِ, وَيَضَعُ مَعْثَرَةَ إِثْمِهِ تِلْقَاءَ وَجْهِهِ, ثُمَّ يَأْتِي إِلَى النَّبِيِّ, فَإِنِّي أَنَا الرَّبُّ أُجِيبُهُ حَسَبَ كَثْرَةِ أَصْنَامِهِ
\par 5 لِكَيْ آخُذَ بَيْتَ إِسْرَائِيلَ بِقُلُوبِهِمْ, لأَنَّهُمْ كُلَّهُمْ قَدِ ارْتَدُّوا عَنِّي بِأَصْنَامِهِمْ.
\par 6 لِذَلِكَ قُلْ لِبَيْتِ إِسْرَائِيلَ: هَكَذَا قَالَ السَّيِّدُ الرَّبُّ: تُوبُوا وَارْجِعُوا عَنْ أَصْنَامِكُمْ, وَعَنْ كُلِّ رَجَاسَاتِكُمُ اصْرِفُوا وُجُوهَكُمْ.
\par 7 لأَنَّ كُلَّ إِنْسَانٍ مِنْ بَيْتِ إِسْرَائِيلَ أَوْ مِنَ الْغُرَبَاءِ الْمُتَغَرِّبِينَ فِي إِسْرَائِيلَ, إِذَا ارْتَدَّ عَنِّي وَأَصْعَدَ أَصْنَامَهُ إِلَى قَلْبِهِ, وَوَضَعَ مَعْثَرَةَ إِثْمِهِ تِلْقَاءَ وَجْهِهِ, ثُمَّ جَاءَ إِلَى النَّبِيِّ لِيَسْأَلَهُ عَنِّي, فَإِنِّي أَنَا الرَّبُّ أُجِيبُهُ بِنَفْسِي.
\par 8 وَأَجْعَلُ وَجْهِي ضِدَّ ذَلِكَ الإِنْسَانِ وَأَجْعَلُهُ آيَةً وَمَثَلاً, وَأَسْتَأْصِلُهُ مِنْ وَسَطِ شَعْبِي, فَتَعْلَمُونَ أَنِّي أَنَا الرَّبُّ.
\par 9 فَإِذَا ضَلَّ النَّبِيُّ وَتَكَلَّمَ كَلاَماً فَأَنَا الرَّبَّ قَدْ أَضْلَلْتُ ذَلِكَ النَّبِيَّ, وَسَأَمُدُّ يَدِي عَلَيْهِ وَأُبِيدُهُ مِنْ وَسَطِ شَعْبِي إِسْرَائِيلَ.
\par 10 وَيَحْمِلُونَ إِثْمَهُمْ. كَإِثْمِ السَّائِلِ يَكُونُ إِثْمُ النَّبِيِّ.
\par 11 لِكَيْ لاَ يَعُودَ يَضِلُّ عَنِّي بَيْتُ إِسْرَائِيلَ, وَلِكَيْ لاَ يَعُودُوا يَتَنَجَّسُونَ بِكُلِّ مَعَاصِيهِمْ, بَلْ لِيَكُونُوا لِي شَعْباً وَأَنَا أَكُونُ لَهُمْ إِلَهاً, يَقُولُ السَّيِّدُ الرَّبُّ].
\par 12 وَكَانَتْ إِلَيَّ كَلِمَةُ الرَّبِّ:
\par 13 [يَا ابْنَ آدَمَ, إِنْ أَخْطَأَتْ إِلَيَّ أَرْضٌ وَخَانَتْ خِيَانَةً, فَمَدَدْتُ يَدِي عَلَيْهَا وَكَسَرْتُ لَهَا قِوَامَ الْخُبْزِ, وَأَرْسَلْتُ عَلَيْهَا الْجُوعَ, وَقَطَعْتُ مِنْهَا الإِنْسَانَ وَالْحَيَوَانَ,
\par 14 وَكَانَ فِيهَا هَؤُلاَءِ الرِّجَالُ الثَّلاَثَةُ: نُوحٌ وَدَانِيآلُ وَأَيُّوبُ, فَإِنَّهُمْ إِنَّمَا يُخَلِّصُونَ أَنْفُسَهُمْ بِبِرِّهِمْ يَقُولُ السَّيِّدُ الرَّبُّ.
\par 15 إِنْ عَبَّرْتُ فِي الأَرْضِ وُحُوشاً رَدِيئَةً فَأَثْكَلُوهَا وَصَارَتْ خَرَاباً بِلاَ عَابِرٍ بِسَبَبِ الْوُحُوشِ,
\par 16 وَفِي وَسَطِهَا هَؤُلاَءِ الرِّجَالُ الثَّلاَثَةُ, فَحَيٌّ أَنَا يَقُولُ السَّيِّدُ الرَّبُّ إِنَّهُمْ لاَ يُخَلِّصُونَ بَنِينَ وَلاَ بَنَاتٍ. هُمْ وَحْدَهُمْ يَخْلُصُونَ وَالأَرْضُ تَصِيرُ خَرِبَةً.
\par 17 أَوْ إِنْ جَلَبْتُ سَيْفاً عَلَى تِلْكَ الأَرْضِ وَقُلْتُ: يَا سَيْفُ اعْبُرْ فِي الأَرْضِ, وَقَطَعْتُ مِنْهَا الإِنْسَانَ وَالْحَيَوَانَ,
\par 18 وَفِي وَسَطِهَا هَؤُلاَءِ الرِّجَالُ الثَّلاَثَةُ, فَحَيٌّ أَنَا يَقُولُ السَّيِّدُ الرَّبُّ إِنَّهُمْ لاَ يُخَلِّصُونَ بَنِينَ وَلاَ بَنَاتٍ, بَلْ هُمْ وَحْدَهُمْ يَخْلُصُونَ.
\par 19 أَوْ إِنْ أَرْسَلْتُ وَبَأً عَلَى تِلْكَ الأَرْضِ وَسَكَبْتُ غَضَبِي عَلَيْهَا بِالدَّمِ لأَقْطَعَ مِنْهَا الإِنْسَانَ وَالْحَيَوَانَ,
\par 20 وَفِي وَسَطِهَا نُوحٌ وَدَانِيآلُ وَأَيُّوبُ, فَحَيٌّ أَنَا يَقُولُ السَّيِّدُ الرَّبُّ إِنَّهُمْ لاَ يُخَلِّصُونَ ابْناً وَلاَ ابْنَةً. إِنَّمَا يُخَلِّصُونَ أَنْفُسَهُمْ بِبِرِّهِمْ.
\par 21 لأَنَّهُ هَكَذَا قَالَ السَّيِّدُ الرَّبُّ: كَمْ بِالْحَرِيِّ إِنْ أَرْسَلْتُ أَحْكَامِي الرَّدِيئَةَ عَلَى أُورُشَلِيمَ سَيْفاً وَجُوعاً وَوَحْشاً رَدِيئاً وَوَبَأً, لأَقْطَعَ مِنْهَا الإِنْسَانَ وَالْحَيَوَانَ,
\par 22 فَهُوَذَا بَقِيَّةٌ فِيهَا نَاجِيَةٌ تُخْرَجُ بَنُونَ وَبَنَاتٌ. هُوَذَا يَخْرُجُونَ إِلَيْكُمْ فَتَنْظُرُونَ طَرِيقَهُمْ وَأَعْمَالَهُمْ, وَتَتَعَزُّونَ عَنِ الشَّرِّ الَّذِي جَلَبْتُهُ عَلَى أُورُشَلِيمَ عَنْ كُلِّ مَا جَلَبْتُهُ عَلَيْهَا.
\par 23 وَيُعَزُّونَكُمْ إِذْ تَرُونَ طَرِيقَهُمْ وَأَعْمَالَهُمْ, فَتَعْلَمُونَ أَنِّي لَمْ أَصْنَعْ بِلاَ سَبَبٍ كُلَّ مَا صَنَعْتُهُ فِيهَا يَقُولُ السَّيِّدُ الرَّبُّ

\chapter{15}

\par 1 وَكَانَ إِلَيَّ كَلاَمُ الرَّبِّ:
\par 2 [يَا ابْنَ آدَمَ, مَاذَا يَكُونُ عُودُ الْكَرْمِ فَوْقَ كُلِّ عُودٍ أَوْ فَوْقَ الْقَضِيبِ الَّذِي مِنْ شَجَرِ الْوَعْرِ؟
\par 3 هَلْ يُؤْخَذُ مِنْهُ عُودٌ لاِصْطِنَاعِ عَمَلٍ مَا, أَوْ يَأْخُذُونَ مِنْهُ وَتَداً لِيُعَلَّقَ عَلَيْهِ إِنَاءٌ مَا؟
\par 4 هُوَذَا يُطْرَحُ أَكْلاً لِلنَّارِ. تَأْكُلُ النَّارُ طَرَفَيْهِ وَيُحْرَقُ وَسَطُهُ. فَهَلْ يَصْلُحُ لِعَمَلٍ؟
\par 5 هُوَذَا حِينَ كَانَ صَحِيحاً لَمْ يَكُنْ يَصْلُحُ لِعَمَلٍ مَا, فَكَم بالْحَرِيِّ لاَ يَصْلُحُ بَعْدُ لِعَمَلٍ إِذْ أَكَلَتْهُ النَّارُ فَاحْتَرَقَ؟]
\par 6 لِذَلِكَ هَكَذَا قَالَ السَّيِّدُ الرَّبُّ: [مِثْلَ عُودِ الْكَرْمِ بَيْنَ عِيدَانِ الْوَعْرِ الَّتِي بَذَلْتُهَا أَكْلاً لِلنَّارِ كَذَلِكَ أَبْذِلُ سُكَّانَ أُورُشَلِيمَ.
\par 7 وَأَجْعَلُ وَجْهِي ضِدَّهُمْ. يَخْرُجُونَ مِنْ نَارٍ فَتَأْكُلُهُمْ نَارٌ, فَتَعْلَمُونَ أَنِّي أَنَا الرَّبُّ حِينَ أَجْعَلُ وَجْهِي ضِدَّهُمْ.
\par 8 وَأَجْعَلُ الأَرْضَ خَرَاباً لأَنَّهُمْ خَانُوا خِيَانَةً يَقُولُ السَّيِّدُ الرَّبُّ].

\chapter{16}

\par 1 وَكَانَتْ إِلَيَّ كَلِمَةُ الرَّبِّ:
\par 2 [يَا ابْنَ آدَمَ, عَرِّفْ أُورُشَلِيمَ بِرَجَاسَاتِهَا
\par 3 وَقُلْ: هَكَذَا قَالَ السَّيِّدُ الرَّبُّ لأُورُشَلِيمَ: مَخْرَجُكِ وَمَوْلِدُكِ مِنْ أَرْضِ كَنْعَانَ. أَبُوكِ أَمُورِيٌّ وَأُمُّكِ حِثِّيَّةٌ.
\par 4 أَمَّا مِيلاَدُكِ يَوْمَ وُلِدْتِ فَلَمْ تُقْطَعْ سُرَّتُكِ, وَلَمْ تُغْسَلِي بِالْمَاءِ لِلتَّنَظُّفِ, وَلَمْ تُمَلَّحِي تَمْلِيحاً, وَلَمْ تُقَمَّطِي تَقْمِيطاً.
\par 5 لَمْ تُشْفِقْ عَلَيْكِ عَيْنٌ لِتَصْنَعَ لَكِ وَاحِدَةً مِنْ هَذِهِ لِتَرِقَّ لَكِ. بَلْ طُرِحْتِ عَلَى وَجْهِ الْحَقْلِ بِكَرَاهَةِ نَفْسِكِ يَوْمَ وُلِدْتِ.
\par 6 فَمَرَرْتُ بِكِ وَرَأَيْتُكِ مَدُوسَةً بِدَمِكِ, فَقُلْتُ لَكِ: بِدَمِكِ عِيشِي. قُلْتُ لَكِ بِدَمِكِ عِيشِي.
\par 7 جَعَلْتُكِ رَبْوَةً كَنَبَاتِ الْحَقْلِ, فَرَبَوْتِ وَكَبِرْتِ وَبَلَغْتِ زِينَةَ الأَزْيَانِ. نَهَدَ ثَدْيَاكِ وَنَبَتَ شَعْرُكِ وَقَدْ كُنْتِ عُرْيَانَةً وَعَارِيَةً.
\par 8 فَمَرَرْتُ بِكِ وَرَأَيْتُكِ, وَإِذَا زَمَنُكِ زَمَنُ الْحُبِّ. فَبَسَطْتُ ذَيْلِي عَلَيْكِ وَسَتَرْتُ عَوْرَتَكِ, وَحَلَفْتُ لَكِ وَدَخَلْتُ مَعَكِ فِي عَهْدٍ يَقُولُ السَّيِّدُ الرَّبُّ, فَصِرْتِ لِي.
\par 9 فَحَمَّمْتُكِ بِالْمَاءِ وَغَسَلْتُ عَنْكِ دِمَاءَكِ وَمَسَحْتُكِ بِالزَّيْتِ,
\par 10 وَأَلْبَسْتُكِ مُطَرَّزَةً, وَنَعَلْتُكِ بِالتُّخَسِ, وَأَزَّرْتُكِ بِالْكَتَّانِ وَكَسَوْتُكِ بَزّاً,
\par 11 وَحَلَّيْتُكِ بِالْحُلِيِّ, فَوَضَعْتُ أَسْوِرَةً فِي يَدَيْكِ وَطَوْقاً فِي عُنُقِكِ.
\par 12 وَوَضَعْتُ خِزَامَةً فِي أَنْفِكِ وَأَقْرَاطاً فِي أُذُنَيْكِ وَتَاجَ جَمَالٍ عَلَى رَأْسِكِ.
\par 13 فَتَحَلَّيْتِ بِالذَّهَبِ وَالْفِضَّةِ وَلِبَاسُكِ الْكَتَّانُ وَالْبَزُّ وَالْمُطَرَّزُ. وَأَكَلْتِ السَّمِيذَ وَالْعَسَلَ وَالزَّيْتَ, وَجَمُلْتِ جِدّاً جِدّاً فَصَلُحْتِ لِمَمْلَكَةٍ.
\par 14 وَخَرَجَ لَكِ اسْمٌ فِي الأُمَمِ لِجَمَالِكِ, لأَنَّهُ كَانَ كَامِلاً بِبَهَائِي الَّذِي جَعَلْتُهُ عَلَيْكِ يَقُولُ السَّيِّدُ الرَّبُّ.
\par 15 [فَاتَّكَلْتِ عَلَى جَمَالِكِ وَزَنَيْتِ عَلَى اسْمِكِ, وَسَكَبْتِ زِنَاكِ عَلَى كُلِّ عَابِرٍ فَكَانَ لَهُ.
\par 16 وَأَخَذْتِ مِنْ ثِيَابِكِ وَصَنَعْتِ لِنَفْسِكِ مُرْتَفَعَاتٍ مُوَشَّاةٍ وَزَنَيْتِ عَلَيْهَا. أَمْرٌ لَمْ يَأْتِ وَلَمْ يَكُنْ.
\par 17 وَأَخَذْتِ أَمْتِعَةَ زِينَتِكِ مِنْ ذَهَبِي وَمِنْ فِضَّتِي الَّتِي أَعْطَيْتُكِ, وَصَنَعْتِ لِنَفْسِكِ صُوَرَ ذُكُورٍ وَزَنَيْتِ بِهَا.
\par 18 وَأَخَذْتِ ثِيَابَكِ الْمُطَرَّزَةَ وَغَطَّيْتِهَا بِهَا وَوَضَعْتِ أَمَامَهَا زَيْتِي وَبَخُورِي.
\par 19 وَخُبْزِي الَّذِي أَعْطَيْتُكِ, السَّمِيذَ وَالزَّيْتَ وَالْعَسَلَ الَّذِي أَطْعَمْتُكِ, وَضَعْتِهَا أَمَامَهَا رَائِحَةَ سُرُورٍ. وَهَكَذَا كَانَ يَقُولُ السَّيِّدُ الرَّبُّ.
\par 20 [أَخَذْتِ بَنِيكِ وَبَنَاتِكِ الَّذِينَ وَلَدْتِهِمْ لِي وَذَبَحْتِهِمْ لَهَا طَعَاماً. أَهُوَ قَلِيلٌ مِنْ زِنَاكِ
\par 21 أَنَّكِ ذَبَحْتِ بَنِيَّ وَجَعَلْتِهِمْ يَجُوزُونَ فِي النَّارِ لَهَا؟
\par 22 وَفِي كُلِّ رَجَاسَاتِكِ وَزِنَاكِ لَمْ تَذْكُرِي أَيَّامَ صِبَاكِ, إِذْ كُنْتِ عُرْيَانَةً وَعَارِيَةً وَكُنْتِ مَدُوسَةً بِدَمِكِ.
\par 23 وَكَانَ بَعْدَ كُلِّ شَرِّكِ. وَيْلٌ وَيْلٌ لَكِ يَقُولُ السَّيِّدُ الرَّبُّ,
\par 24 أَنَّكِ بَنَيْتِ لِنَفْسِكِ قُبَّةً وَصَنَعْتِ لِنَفْسِكِ مُرْتَفَعَةً فِي كُلِّ شَارِعٍ.
\par 25 فِي رَأْسِ كُلِّ طَرِيقٍ بَنَيْتِ مُرْتَفَعَتَكِ وَرَجَّسْتِ جَمَالَكِ, وَفَرَّجْتِ رِجْلَيْكِ لِكُلِّ عَابِرٍ وَأَكْثَرْتِ زِنَاكِ.
\par 26 وَزَنَيْتِ مَعَ جِيرَانِكِ بَنِي مِصْرَ الْغِلاَظِ اللَّحْمِ, وَزِدْتِ فِي زِنَاكِ لإِغَاظَتِي.
\par 27 فَهَئَنَذَا قَدْ مَدَدْتُ يَدِي عَلَيْكِ, وَمَنَعْتُ عَنْكِ فَرِيضَتَكِ, وَأَسْلَمْتُكِ لِمَرَامِ مُبْغِضَاتِكِ بَنَاتِ الْفِلِسْطِينِيِّينَ اللَّوَاتِي يَخْجَلْنَ مِنْ طَرِيقِكِ الرَّذِيلَةِ.
\par 28 وَزَنَيْتِ مَعَ بَنِي أَشُّورَ إِذْ كُنْتِ لَمْ تَشْبَعِي فَزَنَيْتِ بِهِمْ, وَلَمْ تَشْبَعِي أَيْضاً.
\par 29 وَكَثَّرْتِ زِنَاكِ فِي أَرْضِ كَنْعَانَ إِلَى أَرْضِ الْكِلْدَانِيِّينَ, وَبِهَذَا أَيْضاً لَمْ تَشْبَعِي.
\par 30 مَا أَمْرَضَ قَلْبَكِ يَقُولُ السَّيِّدُ الرَّبُّ, إِذْ فَعَلْتِ كُلَّ هَذَا فِعْلَ امْرَأَةٍ زَانِيَةٍ سَلِيطَةٍ!
\par 31 بِبِنَائِكِ قُبَّتَكِ فِي رَأْسِ كُلِّ طَرِيقٍ, وَصُنْعِكِ مُرْتَفَعَتَكِ فِي كُلِّ شَارِعٍ. وَلَمْ تَكُونِي كَزَانِيَةٍ, بَلْ مُحْتَقِرَةً الأُجْرَةَ.
\par 32 أَيَّتُهَا الزَّوْجَةُ الْفَاسِقَةُ, تَأْخُذُ أَجْنَبِيِّينَ مَكَانَ زَوْجِهَا.
\par 33 لِكُلِّ الزَّوَانِي يُعْطُونَ هَدِيَّةً, أَمَّا أَنْتِ فَقَدْ أَعْطَيْتِ كُلَّ مُحِبِّيكِ هَدَايَاكِ, وَرَشَيْتِهِمْ لِيَأْتُوكِ مِنْ كُلِّ جَانِبٍ لِلزِّنَا بِكِ.
\par 34 وَصَارَ فِيكِ عَكْسُ عَادَةِ النِّسَاءِ فِي زِنَاكِ, إِذْ لَمْ يُزْنَ وَرَاءَكِ, بَلْ أَنْتِ تُعْطِينَ أُجْرَةً وَلاَ أُجْرَةَ تُعْطَى لَكِ, فَصِرْتِ بِالْعَكْس!
\par 35 [فَلِذَلِكَ يَا زَانِيَةُ اسْمَعِي كَلاَمَ الرَّبِّ.
\par 36 هَكَذَا قَالَ السَّيِّدُ الرَّبُّ: مِنْ أَجْلِ أَنَّهُ قَدْ أُنْفِقَ نُحَاسُكِ وَانْكَشَفَتْ عَوْرَتُكِ بِزِنَاكِ بِمُحِبِّيكِ وَبِكُلِّ أَصْنَامِ رَجَاسَاتِكِ, وَلِدِمَاءِ بَنِيكِ الَّذِينَ بَذَلْتِهِمْ لَهَا,
\par 37 لِذَلِكَ هَئَنَذَا أَجْمَعُ جَمِيعَ مُحِبِّيكِ الَّذِينَ لَذَذْتِ لَهُمْ, وَكُلَّ الَّذِينَ أَحْبَبْتِهِمْ مَعَ كُلِّ الَّذِينَ أَبْغَضْتِهِمْ, فَأَجْمَعُهُمْ عَلَيْكِ مِنْ حَوْلِكِ, وَأَكْشِفُ عَوْرَتَكِ لَهُمْ لِيَنْظُرُوا كُلَّ عَوْرَتِكِ.
\par 38 وَأَحْكُمُ عَلَيْكِ أَحْكَامَ الْفَاسِقَاتِ السَّافِكَاتِ الدَّمِ, وَأَجْعَلُكِ دَمَ السَّخَطِ وَالْغَيْرَةِ.
\par 39 وَأُسَلِّمُكِ لِيَدِهِمْ فَيَهْدِمُونَ قُبَّتَكِ وَيُهَدِّمُونَ مُرْتَفَعَاتِكِ وَيَنْزِعُونَ عَنْكِ ثِيَابَكِ وَيَأْخُذُونَ أَدَوَاتِ زِينَتِكِ, وَيَتْرُكُونَكِ عُرْيَانَةً وَعَارِيَةً.
\par 40 وَيُصْعِدُونَ عَلَيْكِ جَمَاعَةً وَيَرْجُمُونَكِ بِالْحِجَارَةِ وَيَقْطَعُونَكِ بِسُيُوفِهِمْ,
\par 41 وَيُحْرِقُونَ بُيُوتَكِ بِالنَّارِ وَيُجْرُونَ عَلَيْكِ أَحْكَاماً قُدَّامَ عُيُونِ نِسَاءٍ كَثِيرَةٍ. وَأَكُفُّكِ عَنِ الزِّنَا, وَأَيْضاً لاَ تُعْطِينَ أُجْرَةً بَعْدُ.
\par 42 وَأُحِلُّ غَضَبِي بِكِ فَتَنْصَرِفُ غَيْرَتِي عَنْكِ فَأَسْكُنُ وَلاَ أَغْضَبُ بَعْدُ.
\par 43 مِنْ أَجْلِ أَنَّكِ لَمْ تَذْكُرِي أَيَّامَ صِبَاكِ بَلْ أَسْخَطْتِنِي فِي كُلِّ هَذِهِ, فَهَئَنَذَا أَيْضاً أَجْلِبُ طَرِيقَكِ عَلَى رَأْسِكِ يَقُولُ السَّيِّدُ الرَّبُّ. فَلاَ تَفْعَلِينَ هَذِهِ الرَّذِيلَةَ فَوْقَ رَجَاسَاتِكِ كُلِّهَا.
\par 44 [هُوَذَا كُلُّ ضَارِبِ مَثَلٍ يَضْرِبُ مَثَلاً عَلَيْكِ قَائِلاً: مِثْلُ الأُمِّ بِنْتُهَا.
\par 45 اِبْنَةُ أُمِّكِ أَنْتِ الْكَارِهَةُ زَوْجَهَا وَبَنِيهَا. وَأَنْتِ أُخْتُ أَخَوَاتِكِ اللَّوَاتِي كَرِهْنَ أَزْوَاجَهُنَّ وَأَبْنَاءَهُنَّ. أُمُّكُنَّ حِثِّيَّةٌ وَأَبُوكُنَّ أَمُورِيٌّ,
\par 46 وَأُخْتُكِ الْكُبْرَى السَّامِرَةُ هِيَ وَبَنَاتُهَا السَّاكِنَةُ عَنْ شِمَالِكِ. وَأُخْتُكِ الصُّغْرَى السَّاكِنَةُ عَنْ يَمِينِكِ هِيَ سَدُومُ وَبَنَاتُهَا.
\par 47 وَلاَ فِي طَرِيقِهِنَّ سَلَكْتِ, وَلاَ مِثْلَ رَجَاسَاتِهِنَّ فَعَلْتِ, كَأَنَّ ذَلِكَ قَلِيلٌ فَقَطْ, فَفَسَدْتِ أَكْثَرَ مِنْهُنَّ فِي كُلِّ طُرُقِكِ.
\par 48 حَيٌّ أَنَا يَقُولُ السَّيِّدُ الرَّبُّ, إِنَّ سَدُومَ أُخْتَكِ لَمْ تَفْعَلْ هِيَ وَلاَ بَنَاتُهَا كَمَا فَعَلْتِ أَنْتِ وَبَنَاتُكِ!
\par 49 هَذَا كَانَ إِثْمَ أُخْتِكِ سَدُومَ: الْكِبْرِيَاءُ وَالشَّبَعُ مِنَ الْخُبْزِ وَسَلاَمُ الاِطْمِئْنَانِ كَانَ لَهَا وَلِبَنَاتِهَا, وَلَمْ تُشَدِّدْ يَدَ الْفَقِيرِ وَالْمِسْكِينِ.
\par 50 وَتَكَبَّرْنَ وَعَمِلْنَ الرِّجْسَ أَمَامِي فَنَزَعْتُهُنَّ كَمَا رَأَيْتُ.
\par 51 وَلَمْ تُخْطِئِ السَّامِرَةُ نِصْفَ خَطَايَاكِ. بَلْ زِدْتِ رَجَاسَاتِكِ أَكْثَرَ مِنْهُنَّ, وَبَرَّرْتِ أَخَوَاتِكِ بِكُلِّ رَجَاسَاتِكِ الَّتِي فَعَلْتِ.
\par 52 فَاحْمِلِي أَيْضاً خِزْيَكِ, أَنْتِ الْقَاضِيَةُ عَلَى أَخَوَاتِكِ بِخَطَايَاكِ الَّتِي بِهَا رَجَسْتِ أَكْثَرَ مِنْهُنَّ. هُنَّ أَبَرُّ مِنْكِ. فَاخْجَلِي أَنْتِ أَيْضاً وَاحْمِلِي عَارَكِ بِتَبْرِيرِكِ أَخَوَاتِكِ.
\par 53 وَأُرَجِّعُ سَبْيَهُنَّ, سَبْيَ سَدُومَ وَبَنَاتِهَا, وَسَبْيَ السَّامِرَةِ وَبَنَاتِهَا, وَسَبْيَ مَسْبِيِّيكِ فِي وَسَطِهَا,
\par 54 لِتَحْمِلِي عَارَكِ وَتَخْزِي مِنْ كُلِّ مَا فَعَلْتِ بِتَعْزِيَتِكِ إِيَّاهُنَّ.
\par 55 وَأَخَوَاتُكِ سَدُومُ وَبَنَاتُهَا يَرْجِعْنَ إِلَى حَالَتِهِنَّ الْقَدِيمَةِ, وَالسَّامِرَةُ وَبَنَاتُهَا يَرْجِعْنَ إِلَى حَالَتِهِنَّ الْقَدِيمَةِ, وَأَنْتِ وَبَنَاتُكِ تَرْجِعْنَ إِلَى حَالَتِكُنَّ الْقَدِيمَةِ.
\par 56 وَأُخْتُكِ سَدُومُ لَمْ تَكُنْ تُذْكَرْ فِي فَمِكِ يَوْمَ كِبْرِيَائِكِ
\par 57 قَبْلَ مَا انْكَشَفَ شَرُّكِ, كَمَا فِي زَمَانِ تَعْيِيرِ بَنَاتِ أَرَامَ وَكُلِّ مَنْ حَوْلَهَا, بَنَاتِ الْفِلِسْطِينِيِّينَ اللَّوَاتِي يَحْتَقِرْنَكِ مِنْ كُلِّ جِهَةٍ.
\par 58 رَذِيلَتُكِ وَرَجَاسَاتُكِ أَنْتِ تَحْمِلِينَهَا يَقُولُ الرَّبُّ].
\par 59 لأَنَّهُ هَكَذَا قَالَ السَّيِّدُ الرَّبُّ: [إِنِّي أَفْعَلُ بِكِ كَمَا فَعَلْتِ, إِذِ ازْدَرَيْتِ بِالْقَسَمِ لِنَكْثِ الْعَهْدِ.
\par 60 وَلَكِنِّي أَذْكُرُ عَهْدِي مَعَكِ فِي أَيَّامِ صِبَاكِ, وَأُقِيمُ لَكِ عَهْداً أَبَدِيّاً.
\par 61 فَتَتَذَكَّرِينَ طُرُقَكِ وَتَخْجَلِينَ إِذْ تَقْبَلِينَ أَخَوَاتِكِ الْكِبَرَ وَالصِّغَرَ, وَأَجْعَلُهُنَّ لَكِ بَنَاتٍ وَلَكِنْ لاَ بِعَهْدِكِ.
\par 62 وَأَنَا أُقِيمُ عَهْدِي مَعَكِ فَتَعْلَمِينَ أَنِّي أَنَا الرَّبُّ.
\par 63 لِتَتَذَكَّرِي فَتَخْزِي وَلاَ تَفْتَحِي فَاكِ بَعْدُ بِسَبَبِ خِزْيِكِ, حِينَ أَغْفِرُ لَكِ كُلَّ مَا فَعَلْتِ يَقُولُ السَّيِّدُ الرَّبُّ].

\chapter{17}

\par 1 وَكَانَ إِلَيَّ كَلاَمُ الرَّبِّ:
\par 2 [يَا ابْنَ آدَمَ, حَاجِ أُحْجِيَّةً وَمَثِّلْ مَثَلاً لِبَيْتِ إِسْرَائِيلَ
\par 3 وَقُلْ: هَكَذَا قَالَ السَّيِّدُ الرَّبُّ: نَسْرٌ عَظِيمٌ كَبِيرُ الْجَنَاحَيْنِ طَوِيلُ الْقَوَادِمِ وَاسِعُ الْمَنَاكِبِ ذُو تَهَاوِيلَ, جَاءَ إِلَى لُبْنَانَ وَأَخَذَ فَرْعَ الأَرْزِ.
\par 4 قَصَفَ رَأْسَ خَرَاعِيبِهِ, وَجَاءَ بِهِ إِلَى أَرْضِ كَنْعَانَ وَجَعَلَهُ فِي مَدِينَةِ التُّجَّارِ.
\par 5 وَأَخَذَ مِنْ زَرْعِ الأَرْضِ وَأَلْقَاهُ فِي حَقْلِ الزَّرْعِ, وَجَعَلَهُ عَلَى مِيَاهٍ كَثِيرَةٍ. أَقَامَهُ كَالصَّفْصَافِ,
\par 6 فَنَبَتَ وَصَارَ كَرْمَةً مُنْتَشِرَةً قَصِيرَةَ السَّاقِ. انْعَطَفَتْ عَلَيْهِ زَرَاجِينُهَا وَكَانَتْ أُصُولُهَا تَحْتَهُ, فَصَارَتْ كَرْمَةً وَأَنْبَتَتْ فُرُوعاً وَأَفْرَخَتْ أَغْصَاناً.
\par 7 وَكَانَ نَسْرٌ آخَرُ عَظِيمٌ كَبِيرُ الْجَنَاحَيْنِ وَاسِعُ الْمَنْكَبِ, فَإِذَا بِهَذِهِ الْكَرْمَةِ عَطَفَتْ عَلَيْهِ أُصُولَهَا وَأَنْبَتَتْ نَحْوَهُ زَرَاجِينَهَا لِيَسْقِيهَا فِي خَمَائِلِ غَرْسِهَا.
\par 8 فِي حَقْلٍ جَيِّدٍ عَلَى مِيَاهٍ كَثِيرَةٍ هِيَ مَغْرُوسَةٌ لِتُنْبِتَ أَغْصَانَهَا وَتَحْمِلَ ثَمَراً, فَتَكُونَ كَرْمَةً وَاسِعَةً.
\par 9 قُلْ: هَكَذَا قَالَ السَّيِّدُ الرَّبُّ: هَلْ تَنْجَحُ؟ أَفَلاَ يَقْلَعُ أُصُولَهَا وَيَقْطَعُ ثَمَرَهَا فَتَيْبَسَ؟ كُلٌّ مِنْ أَوْرَاقِ أَغْصَانِهَا تَيْبَسُ, وَلَيْسَ بِذِرَاعٍ عَظِيمَةٍ أَوْ بِشَعْبٍ كَثِيرٍ لِيَقْلَعُوهَا مِنْ أُصُولِهَا.
\par 10 هَا هِيَ الْمَغْرُوسَةُ, فَهَلْ تَنْجَحُ؟ أَلاَ تَيْبَسُ يَبَساً كَأَنَّ رِيحاً شَرْقِيَّةً أَصَابَتْهَا؟ فِي خَمَائِلِ نَبْتِهَا تَيْبَسُ].
\par 11 وَكَانَ إِلَيَّ كَلاَمُ الرَّبِّ:
\par 12 [قُلْ لِلْبَيْتِ الْمُتَمَرِّدِ: أَمَا عَلِمْتُمْ مَا هَذِهِ؟ قُلْ: هُوَذَا مَلِكُ بَابِلَ قَدْ جَاءَ إِلَى أُورُشَلِيمَ وَأَخَذَ مَلِكَهَا وَرُؤَسَاءَهَا وَجَاءَ بِهِمْ إِلَيْهِ إِلَى بَابِلَ.
\par 13 وَأَخَذَ مِنَ الزَّرْعِ الْمَلِكِيِّ وَقَطَعَ مَعَهُ عَهْداً وَأَدْخَلَهُ فِي قَسَمٍ, وَأَخَذَ أَقْوِيَاءَ الأَرْضِ
\par 14 لِتَكُونَ الْمَمْلَكَةُ حَقِيرَةً وَلاَ تَرْتَفِعَ. لِتَحْفَظَ الْعَهْدَ فَتَثْبُتَ.
\par 15 فَتَمَرَّدَ عَلَيْهِ بِإِرْسَالِهِ رُسُلَهُ إِلَى مِصْرَ لِيُعْطُوهُ خَيْلاً وَشَعْباً كَثِيرِينَ. فَهَلْ يَنْجَحُ؟ هَلْ يُفْلِتُ فَاعِلُ هَذَا, أَوْ يَنْقُضُ عَهْداً وَيُفْلِتُ؟
\par 16 حَيٌّ أَنَا يَقُولُ السَّيِّدُ الرَّبُّ, إِنَّ فِي مَوْضِعِ الْمَلِكِ الَّذِي مَلَّكَهُ, الَّذِي ازْدَرَى قَسَمَهُ وَنَقَضَ عَهْدَهُ, فَعِنْدَهُ فِي وَسَطِ بَابِلَ يَمُوتُ.
\par 17 وَلاَ بِجَيْشٍ عَظِيمٍ وَجَمْعٍ غَفِيرٍ يُعِينُهُ فِرْعَوْنُ فِي الْحَرْبِ, بِإِقَامَةِ مِتْرَسَةٍ وَبِبِنَاءِ بُرْجٍ لِقَطْعِ نُفُوسٍ كَثِيرَةٍ.
\par 18 إِذِ ازْدَرَى الْقَسَمَ لِنَقْضِ الْعَهْدِ, وَهُوَذَا قَدْ أَعْطَى يَدَهُ وَفَعَلَ هَذَا كُلَّهُ فَلاَ يُفْلِتُ.
\par 19 لأَجْلِ ذَلِكَ هَكَذَا قَالَ السَّيِّدُ الرَّبُّ: حَيٌّ أَنَا, إِنَّ قَسَمِي الَّذِي ازْدَرَاهُ, وَعَهْدِي الَّذِي نَقَضَهُ, أَرُدُّهُمَا عَلَى رَأْسِهِ.
\par 20 وَأَبْسُطُ شَبَكَتِي عَلَيْهِ فَيُؤْخَذُ فِي شَرَكِي, وَآتِي بِهِ إِلَى بَابِلَ وَأُحَاكِمُهُ هُنَاكَ عَلَى خِيَانَتِهِ الَّتِي خَانَنِي بِهَا.
\par 21 وَكُلُّ هَارِبِيهِ وَكُلُّ جُيُوشِهِ يَسْقُطُونَ بِالسَّيْفِ, وَالْبَاقُونَ يُذَرُّونَ فِي كُلِّ رِيحٍ, فَتَعْلَمُونَ أَنِّي أَنَا الرَّبُّ تَكَلَّمْتُ].
\par 22 هَكَذَا قَالَ السَّيِّدُ الرَّبُّ: [وَآخُذُ أَنَا مِنْ فَرْعِ الأَرْزِ الْعَالِي وَأَغْرِسُهُ, وَأَقْطِفُ مِنْ رَأْسِ خَرَاعِيبِهِ غُصْناً وَأَغْرِسُهُ عَلَى جَبَلٍ عَالٍ وَشَامِخٍ.
\par 23 فِي جَبَلِ إِسْرَائِيلَ الْعَالِي أَغْرِسُهُ, فَيُنْبِتُ أَغْصَاناً وَيَحْمِلُ ثَمَراً وَيَكُونُ أَرْزاً وَاسِعاً, فَيَسْكُنُ تَحْتَهُ كُلُّ طَائِرٍ. كُلُّ ذِي جَنَاحٍ يَسْكُنُ فِي ظِلِّ أَغْصَانِهِ.
\par 24 فَتَعْلَمُ جَمِيعُ أَشْجَارِ الْحَقْلِ أَنِّي أَنَا الرَّبُّ وَضَعْتُ الشَّجَرَةَ الرَّفِيعَةَ, وَرَفَعْتُ الشَّجَرَةَ الْوَضِيعَةَ, وَيَبَّسْتُ الشَّجَرَةَ الْخَضْرَاءَ, وَأَفْرَخْتُ الشَّجَرَةَ الْيَابِسَةَ. أَنَا الرَّبَّ تَكَلَّمْتُ وَفَعَلْتُ].

\chapter{18}

\par 1 وَكَانَ إِلَيَّ كَلاَمُ الرَّبِّ:
\par 2 [مَا لَكُمْ أَنْتُمْ تَضْرِبُونَ هَذَا الْمَثَلَ عَلَى أَرْضِ إِسْرَائِيلَ, قَائِلِينَ: الآبَاءُ أَكَلُوا الْحِصْرِمَ وَأَسْنَانُ الأَبْنَاءِ ضَرِسَتْ؟
\par 3 حَيٌّ أَنَا يَقُولُ السَّيِّدُ الرَّبُّ, لاَ يَكُونُ لَكُمْ مِنْ بَعْدُ أَنْ تَضْرِبُوا هَذَا الْمَثَلَ فِي إِسْرَائِيلَ.
\par 4 هَا كُلُّ النُّفُوسِ هِيَ لِي. نَفْسُ الأَبِ كَنَفْسِ الاِبْنِ. كِلاَهُمَا لِي. النَّفْسُ الَّتِي تُخْطِئُ هِيَ تَمُوتُ.
\par 5 وَالإِنْسَانُ الَّذِي كَانَ بَارّاً وَفَعَلَ حَقّاً وَعَدْلاً,
\par 6 لَمْ يَأْكُلْ عَلَى الْجِبَالِ وَلَمْ يَرْفَعْ عَيْنَيْهِ إِلَى أَصْنَامِ بَيْتِ إِسْرَائِيلَ, وَلَمْ يُنَجِّسِ امْرَأَةَ قَرِيبِهِ وَلَمْ يَقْرُبِ امْرَأَةً طَامِثاً,
\par 7 وَلَمْ يَظْلِمْ إِنْسَاناً, بَلْ رَدَّ لِلْمَدْيُونِ رَهْنَهُ, وَلَمْ يَغْتَصِبِ اغْتِصَاباً بَلْ بَذَلَ خُبْزَهُ لِلْجَوْعَانِ وَكَسَا الْعُرْيَانَ ثَوْباً,
\par 8 وَلَمْ يُعْطِ بِالرِّبَا, وَلَمْ يَأْخُذْ مُرَابَحَةً, وَكَفَّ يَدَهُ عَنِ الْجَوْرِ, وَأَجْرَى الْعَدْلَ الْحَقَّ بَيْنَ الإِنْسَانِ, وَالإِنْسَانِ
\par 9 وَسَلَكَ فِي فَرَائِضِي وَحَفِظَ أَحْكَامِي لِيَعْمَلَ بِالْحَقِّ فَهُوَ بَارٌّ. حَيَاةً يَحْيَا يَقُولُ السَّيِّدُ الرَّبُّ.
\par 10 [فَإِنْ وَلَدَ ابْناً مُعْتَنِفاً سَفَّاكَ دَمٍ, فَفَعَلَ شَيْئاً مِنْ هَذِهِ
\par 11 وَلَمْ يَفْعَلْ كُلَّ تِلْكَ, بَلْ أَكَلَ عَلَى الْجِبَالِ وَنَجَّسَ امْرَأَةَ قَرِيبِهِ
\par 12 وَظَلَمَ الْفَقِيرَ وَالْمِسْكِينَ, وَاغْتَصَبَ اغْتِصَاباً, وَلَمْ يَرُدَّ الرَّهْنَ, وَقَدْ رَفَعَ عَيْنَيْهِ إِلَى الأَصْنَامِ وَفَعَلَ الرِّجْسَ,
\par 13 وَأَعْطَى بِالرِّبَا وَأَخَذَ الْمُرَابَحَةَ, أَفَيَحْيَا؟ لاَ يَحْيَا! قَدْ عَمِلَ كُلَّ هَذِهِ الرَّجَاسَاتِ فَمَوْتاً يَمُوتُ. دَمُهُ يَكُونُ عَلَى نَفْسِهِ!
\par 14 [وَإِنْ وَلَدَ ابْناً رَأَى جَمِيعَ خَطَايَا أَبِيهِ الَّتِي فَعَلَهَا فَرَآهَا وَلَمْ يَفْعَلْ مِثْلَهَا.
\par 15 لَمْ يَأْكُلْ عَلَى الْجِبَالِ وَلَمْ يَرْفَعْ عَيْنَيْهِ إِلَى أَصْنَامِ بَيْتِ إِسْرَائِيلَ وَلاَ نَجَّسَ امْرَأَةَ قَرِيبِهِ
\par 16 وَلاَ ظَلَمَ إِنْسَاناً وَلاَ ارْتَهَنَ رَهْناً وَلاَ اغْتَصَبَ اغْتِصَاباً, بَلْ بَذَلَ خُبْزَهُ لِلْجَوْعَانِ وَكَسَا الْعُرْيَانَ ثَوْباً
\par 17 وَرَفَعَ يَدَهُ عَنِ الْفَقِيرِ وَلَمْ يَأْخُذْ رِباً وَلاَ مُرَابَحَةً, بَلْ أَجْرَى أَحْكَامِي وَسَلَكَ فِي فَرَائِضِي, فَإِنَّهُ لاَ يَمُوتُ بِإِثْمِ أَبِيهِ. حَيَاةً يَحْيَا.
\par 18 أَمَّا أَبُوهُ فَلأَنَّهُ ظَلَمَ ظُلْماً وَاغْتَصَبَ أَخَاهُ اغْتِصَاباً, وَعَمِلَ غَيْرَ الصَّالِحِ بَيْنَ شَعْبِهِ, فَهُوَذَا يَمُوتُ بِإِثْمِهِ.
\par 19 [وَأَنْتُمْ تَقُولُونَ: لِمَاذَا لاَ يَحْمِلُ الاِبْنُ مِنْ إِثْمِ الأَبِ؟ أَمَّا الاِبْنُ فَقَدْ فَعَلَ حَقّاً وَعَدْلاً. حَفِظَ جَمِيعَ فَرَائِضِي وَعَمِلَ بِهَا فَحَيَاةً يَحْيَا.
\par 20 اَلنَّفْسُ الَّتِي تُخْطِئُ هِيَ تَمُوتُ. الاِبْنُ لاَ يَحْمِلُ مِنْ إِثْمِ الأَبِ وَالأَبُ لاَ يَحْمِلُ مِنْ إِثْمِ الاِبْنِ. بِرُّ الْبَارِّ عَلَيْهِ يَكُونُ وَشَرُّ الشِّرِّيرِ عَلَيْهِ يَكُونُ.
\par 21 فَإِذَا رَجَعَ الشِّرِّيرُ عَنْ جَمِيعِ خَطَايَاهُ الَّتِي فَعَلَهَا وَحَفِظَ كُلَّ فَرَائِضِي وَفَعَلَ حَقّاً وَعَدْلاً فَحَيَاةً يَحْيَا. لاَ يَمُوتُ.
\par 22 كُلُّ مَعَاصِيهِ الَّتِي فَعَلَهَا لاَ تُذْكَرُ عَلَيْهِ. فِي بِرِّهِ الَّذِي عَمِلَ يَحْيَا.
\par 23 هَلْ مَسَرَّةً أُسَرُّ بِمَوْتِ الشِّرِّيرِ يَقُولُ السَّيِّدُ الرَّبُّ؟ أَلاَ بِرُجُوعِهِ عَنْ طُرُقِهِ فَيَحْيَا؟
\par 24 وَإِذَا رَجَعَ الْبَارُّ عَنْ بِرِّهِ وَعَمِلَ إِثْماً وَفَعَلَ مِثْلَ كُلِّ الرَّجَاسَاتِ الَّتِي يَفْعَلُهَا الشِّرِّيرُ, أَفَيَحْيَا؟ كُلُّ بِرِّهِ الَّذِي عَمِلَهُ لاَ يُذْكَرُ. فِي خِيَانَتِهِ الَّتِي خَانَهَا وَفِي خَطِيَّتِهِ الَّتِي أَخْطَأَ بِهَا يَمُوتُ.
\par 25 [وَأَنْتُمْ تَقُولُونَ: لَيْسَتْ طَرِيقُ الرَّبِّ مُسْتَوِيَةً. فَاسْمَعُوا الآنَ يَا بَيْتَ إِسْرَائِيلَ. أَطَرِيقِي هِيَ غَيْرُ مُسْتَوِيَةٍ؟ أَلَيْسَتْ طُرُقُكُمْ غَيْرَ مُسْتَوِيَةٍ؟
\par 26 إِذَا رَجَعَ الْبَارُّ عَنْ بِرِّهِ وَعَمِلَ إِثْماً وَمَاتَ فِيهِ, فَبِإِثْمِهِ الَّذِي عَمِلَهُ يَمُوتُ.
\par 27 وَإِذَا رَجَعَ الشِّرِّيرُ عَنْ شَرِّهِ الَّذِي فَعَلَ, وَعَمِلَ حَقّاً وَعَدْلاً, فَهُوَ يُحْيِي نَفْسَهُ.
\par 28 رَأَى فَرَجَعَ عَنْ كُلِّ مَعَاصِيهِ الَّتِي عَمِلَهَا فَحَيَاةً يَحْيَا. لاَ يَمُوتُ.
\par 29 وَبَيْتُ إِسْرَائِيلَ يَقُولُ: لَيْسَتْ طَرِيقُ الرَّبِّ مُسْتَوِيَةً. أَطُرُقِي غَيْرُ مُسْتَقِيمَةٍ يَا بَيْتَ إِسْرَائِيلَ؟ أَلَيْسَتْ طُرُقُكُمْ غَيْرَ مُسْتَقِيمَةٍ؟
\par 30 مِنْ أَجْلِ ذَلِكَ أَقْضِي عَلَيْكُمْ يَا بَيْتَ إِسْرَائِيلَ كُلِّ وَاحِدٍ كَطُرُقِهِ يَقُولُ السَّيِّدُ الرَّبُّ. تُوبُوا وَارْجِعُوا عَنْ كُلِّ مَعَاصِيكُمْ, وَلاَ يَكُونُ لَكُمُ الإِثْمُ مَهْلَكَةً.
\par 31 اِطْرَحُوا عَنْكُمْ كُلَّ مَعَاصِيكُمُ الَّتِي عَصِيْتُمْ بِهَا, وَاعْمَلُوا لأَنْفُسِكُمْ قَلْباً جَدِيداً وَرُوحاً جَدِيدَةً. فَلِمَاذَا تَمُوتُونَ يَا بَيْتَ إِسْرَائِيلَ؟
\par 32 لأَنِّي لاَ أُسَرُّ بِمَوْتِ مَنْ يَمُوتُ يَقُولُ السَّيِّدُ الرَّبُّ. فَارْجِعُوا وَاحْيُوا].

\chapter{19}

\par 1 [أَمَّا أَنْتَ فَارْفَعْ مَرْثَاةً عَلَى رُؤَسَاءِ إِسْرَائِيلَ
\par 2 وَقُلْ: مَا هِيَ أُمُّكَ؟ لَبْوَةٌ رَبَضَتْ بَيْنَ الأُسُودِ, وَرَبَّتْ جِرَاءَهَا بَيْنَ الأَشْبَالِ!
\par 3 رَبَّتْ وَاحِداً مِنْ جِرَائِهَا فَصَارَ شِبْلاً وَتَعَلَّمَ افْتِرَاسَ الْفَرِيسَةِ. أَكَلَ النَّاسَ.
\par 4 فَلَمَّا سَمِعَتْ بِهِ الأُمَمُ أُخِذَ فِي حُفْرَتِهِمْ, فَأَتُوا بِهِ بِخَزَائِمَ إِلَى أَرْضِ مِصْرَ.
\par 5 فَلَمَّا رَأَتْ أَنَّهَا قَدِ انْتَظَرَتْ وَهَلَكَ رَجَاؤُهَا, أَخَذَتْ آخَرَ مِنْ جِرَائِهَا وَصَيَّرَتْهُ شِبْلاً.
\par 6 فَتَمَشَّى بَيْنَ الأُسُودِ. صَارَ شِبْلاً وَتَعَلَّمَ افْتِرَاسَ الْفَرِيسَةِ. أَكَلَ النَّاسَ.
\par 7 وَعَرَفَ قُصُورَهُمْ وَخَرَّبَ مُدُنَهُمْ, فَأَقْفَرَتِ الأَرْضُ وَمِلْؤُهَا مِنْ صَوْتِ زَمْجَرَتِهِ.
\par 8 فَاتَّفَقَ عَلَيْهِ الأُمَمُ مِنْ كُلِّ جِهَةٍ مِنَ الْبُلْدَانِ, وَبَسَطُوا عَلَيْهِ شَبَكَتَهُمْ, فَأُخِذَ فِي حُفْرَتِهِمْ,
\par 9 فَوَضَعُوهُ فِي قَفَصٍ بِخَزَائِمَ وَأَحْضَرُوهُ إِلَى مَلِكِ بَابِلَ, وَأَتُوا بِهِ إِلَى الْقِلاَعِ لِكَيْ لاَ يُسْمَعَ صَوْتُهُ بَعْدُ عَلَى جِبَالِ إِسْرَائِيلَ.
\par 10 [أُمُّكَ كَكَرْمَةٍ, مِثْلِكَ غُرِسَتْ عَلَى الْمِيَاهِ. كَانَتْ مُثْمِرَةً مُفْرِخَةً مِنْ كَثْرَةِ الْمِيَاهِ.
\par 11 وَكَانَ لَهَا فُرُوعٌ قَوِيَّةٌ لِقُضْبَانِ الْمُتَسَلِّطِينَ, وَارْتَفَعَ سَاقُهَا بَيْنَ الأَغْصَانِ الْغَبْيَاءِ, وَظَهَرَتْ فِي ارْتِفَاعِهَا بِكَثْرَةِ زَرَاجِينِهَا.
\par 12 لَكِنَّهَا اقْتُلِعَتْ بِغَيْظٍ وَطُرِحَتْ عَلَى الأَرْضِ, وَقَدْ يَبَّسَتْ رِيحٌ شَرْقِيَّةٌ ثَمَرَهَا. قُصِفَتْ وَيَبِسَتْ فُرُوعُهَا الْقَوِيَّةُ. أَكَلَتْهَا النَّارُ.
\par 13 وَالآنَ غُرِسَتْ فِي الْقَفْرِ فِي أَرْضٍ يَابِسَةٍ عَطْشَانَةٍ.
\par 14 وَخَرَجَتْ نَارٌ مِنْ فَرْعِ عِصِيِّهَا أَكَلَتْ ثَمَرَهَا. وَلَيْسَ لَهَا الآنَ فَرْعٌ قَوِيٌّ لِقَضِيبِ تَسَلُّطٍ. هِيَ رِثَاءٌ وَتَكُونُ لِمَرْثَاةٍ.

\chapter{20}

\par 1 وَكَانَ فِي السَّنَةِ السَّابِعَةِ فِي الشَّهْرِ الْخَامِسِ فِي الْعَاشِرِ مِنَ الشَّهْرِ, أَنَّ أُنَاساً مِنْ شُيُوخِ إِسْرَائِيلَ جَاءُوا لِيَسْأَلُوا الرَّبَّ, فَجَلَسُوا أَمَامِي.
\par 2 فَكَانَ إِلَيَّ كَلاَمُ الرَّبِّ:
\par 3 [يَا ابْنَ آدَمَ, كَلِّمْ شُيُوخَ إِسْرَائِيلَ وَقُلْ لَهُمْ: هَكَذَا قَالَ السَّيِّدُ الرَّبُّ: هَلْ أَنْتُمْ آتُونَ لِتَسْأَلُونِي؟ حَيٌّ أَنَا لاَ أُسْأَلُ مِنْكُمْ يَقُولُ السَّيِّدُ الرَّبُّ.
\par 4 هَلْ تَدِينُهُمْ؟ هَلْ تَدِينُ يَا ابْنَ آدَمَ؟ عَرِّفْهُمْ رَجَاسَاتِ آبَائِهِمْ,
\par 5 وَقُلْ لَهُمْ: هَكَذَا قَالَ السَّيِّدُ الرَّبُّ: فِي يَوْمِ اخْتَرْتُ إِسْرَائِيلَ وَرَفَعْتُ يَدِي لِنَسْلِ بَيْتِ يَعْقُوبَ, وَعَرَّفْتُهُمْ نَفْسِي فِي أَرْضِ مِصْرَ, وَرَفَعْتُ لَهُمْ يَدِي قَائِلاً: أَنَا الرَّبُّ إِلَهُكُمْ,
\par 6 فِي ذَلِكَ الْيَوْمِ رَفَعْتُ لَهُمْ يَدِي لأُخْرِجَهُمْ مِنْ أَرْضِ مِصْرَ إِلَى الأَرْضِ الَّتِي تَجَسَّسْتُهَا لَهُمْ, تَفِيضُ لَبَناً وَعَسَلاً. هِيَ فَخْرُ كُلِّ الأَرَاضِي
\par 7 وَقُلْتُ لَهُمُ: اطْرَحُوا كُلُّ إِنْسَانٍ مِنْكُمْ أَرْجَاسَ عَيْنَيْهِ وَلاَ تَتَنَجَّسُوا بِأَصْنَامِ مِصْرَ. أَنَا الرَّبُّ إِلَهُكُمْ.
\par 8 فَتَمَرَّدُوا عَلَيَّ وَلَمْ يُرِيدُوا أَنْ يَسْمَعُوا لِي, وَلَمْ يَطْرَحِ الإِنْسَانُ مِنْهُمْ أَرْجَاسَ عَيْنَيْهِ وَلَمْ يَتْرُكُوا أَصْنَامَ مِصْرَ. فَقُلْتُ: إِنِّي أَسْكُبُ رِجْزِي عَلَيْهِمْ لأُتِمَّ عَلَيْهِمْ سَخَطِي فِي وَسَطِ أَرْضِ مِصْرَ.
\par 9 لَكِنْ صَنَعْتُ لأَجْلِ اسْمِي لِكَيْلاَ يَتَنَجَّسَ أَمَامَ عُيُونِ الأُمَمِ الَّذِينَ هُمْ فِي وَسَطِهِمِ, الَّذِينَ عَرَّفْتُهُمْ نَفْسِي أَمَامَ عُيُونِهِمْ بِإِخْرَاجِهِمْ مِنْ أَرْضِ مِصْرَ.
\par 10 فَأَخْرَجْتُهُمْ مِنْ أَرْضِ مِصْرَ وَأَتَيْتُ بِهِمْ إِلَى الْبَرِّيَّةِ.
\par 11 وَأَعْطَيْتُهُمْ فَرَائِضِي وَعَرَّفْتُهُمْ أَحْكَامِي الَّتِي إِنْ عَمِلَهَا إِنْسَانٌ يَحْيَا بِهَا.
\par 12 وَأَعْطَيْتُهُمْ أَيْضاً سُبُوتِي لِتَكُونَ عَلاَمَةً بَيْنِي وَبَيْنَهُمْ, لِيَعْلَمُوا أَنِّي أَنَا الرَّبُّ مُقَدِّسُهُمْ.
\par 13 [فَتَمَرَّدَ عَلَيَّ بَيْتُ إِسْرَائِيلَ فِي الْبَرِّيَّةِ. لَمْ يَسْلُكُوا فِي فَرَائِضِي وَرَفَضُوا أَحْكَامِي الَّتِي إِنْ عَمِلَهَا إِنْسَانٌ يَحْيَا بِهَا, وَنَجَّسُوا سُبُوتِي كَثِيراً. فَقُلْتُ: إِنِّي أَسْكُبُ رِجْزِي عَلَيْهِمْ فِي الْبَرِّيَّةِ لإِفْنَائِهِمْ.
\par 14 لَكِنْ صَنَعْتُ لأَجْلِ اسْمِي لِكَيْلاَ يَتَنَجَّسَ أَمَامَ عُيُونِ الأُمَمِ الَّذِينَ أَخْرَجْتُهُمْ أَمَامَ عُيُونِهِمْ.
\par 15 وَرَفَعْتُ أَيْضاً يَدِي لَهُمْ فِي الْبَرِّيَّةِ بِأَنِّي لاَ آتِي بِهِمْ إِلَى الأَرْضِ الَّتِي أَعْطَيْتُهُمْ إِيَّاهَا تَفِيضُ لَبَناً وَعَسَلاً. هِيَ فَخْرُ كُلِّ الأَرَاضِي.
\par 16 لأَنَّهُمْ رَفَضُوا أَحْكَامِي وَلَمْ يَسْلُكُوا فِي فَرَائِضِي, بَلْ نَجَّسُوا سُبُوتِي, لأَنَّ قَلْبَهُمْ ذَهَبَ وَرَاءَ أَصْنَامِهِمْ.
\par 17 لَكِنَّ عَيْنِي أَشْفَقَتْ عَلَيْهِمْ عَنْ إِهْلاَكِهِمْ, فَلَمْ أُفْنِهِمْ فِي الْبَرِّيَّةِ.
\par 18 وَقُلْتُ لأَبْنَائِهِمْ فِي الْبَرِّيَّةِ: لاَ تَسْلُكُوا فِي فَرَائِضِ آبَائِكُمْ وَلاَ تَحْفَظُوا أَحْكَامَهُمْ وَلاَ تَتَنَجَّسُوا بِأَصْنَامِهِمْ.
\par 19 أَنَا الرَّبُّ إِلَهُكُمْ فَاسْلُكُوا فِي فَرَائِضِي وَاحْفَظُوا أَحْكَامِي وَاعْمَلُوا بِهَا
\par 20 وَقَدِّسُوا سُبُوتِي فَتَكُونَ عَلاَمَةً بَيْنِي وَبَيْنَكُمْ, لِتَعْلَمُوا أَنِّي أَنَا الرَّبُّ إِلَهُكُمْ.
\par 21 فَتَمَرَّدَ الأَبْنَاءُ عَلَيَّ. لَمْ يَسْلُكُوا فِي فَرَائِضِي وَلَمْ يَحْفَظُوا أَحْكَامِي لِيَعْمَلُوهَا, الَّتِي إِنْ عَمِلَهَا إِنْسَانٌ يَحْيَا بِهَا, وَنَجَّسُوا سُبُوتِي. فَقُلْتُ: إِنِّي أَسْكُبُ رِجْزِي عَلَيْهِمْ لأُتِمَّ سَخَطِي عَلَيْهِمْ فِي الْبَرِّيَّةِ.
\par 22 ثُمَّ كَفَفْتُ يَدِي وَصَنَعْتُ لأَجْلِ اسْمِي لِكَيْلاَ يَتَنَجَّسَ أَمَامَ عُيُونِ الأُمَمِ الَّذِينَ أَخْرَجْتُهُمْ أَمَامَ عُيُونِهِمْ.
\par 23 وَرَفَعْتُ أَيْضاً يَدِي لَهُمْ فِي الْبَرِّيَّةِ لأُفَرِّقَهُمْ فِي الأُمَمِ وَأُذَرِّيَهُمْ فِي الأَرَاضِي,
\par 24 لأَنَّهُمْ لَمْ يَصْنَعُوا أَحْكَامِي, بَلْ رَفَضُوا فَرَائِضِي وَنَجَّسُوا سُبُوتِي وَكَانَتْ عُيُونُهُمْ وَرَاءَ أَصْنَامِ آبَائِهِمْ.
\par 25 وَأَعْطَيْتُهُمْ أَيْضاً فَرَائِضَ غَيْرَ صَالِحَةٍ وَأَحْكَاماً لاَ يَحْيُونَ بِهَا
\par 26 وَنَجَّسْتُهُمْ بِعَطَايَاهُمْ إِذْ أَجَازُوا فِي النَّارِ كُلَّ فَاتِحِ رَحِمٍ لأُبِيدَهُمْ, حَتَّى يَعْلَمُوا أَنِّي أَنَا الرَّبُّ.
\par 27 [لأَجْلِ ذَلِكَ كَلِّمْ بَيْتَ إِسْرَائِيلَ يَا ابْنَ آدَمَ وَقُلْ لَهُمْ: هَكَذَا قَالَ السَّيِّدُ الرَّبُّ: فِي هَذَا أَيْضاً جَدَّفَ عَلَيَّ آبَاؤُكُمْ, إِذْ خَانُونِي خِيَانَةً
\par 28 لَمَّا أَتَيْتُ بِهِمْ إِلَى الأَرْضِ الَّتِي رَفَعْتُ لَهُمْ يَدِي لأُعْطِيَهُمْ إِيَّاهَا, فَرَأُوا كُلَّ تَلٍّ عَالٍ وَكُلَّ شَجَرَةٍ غَبْيَاءَ, فَذَبَحُوا هُنَاكَ ذَبَائِحَهُمْ وَقَرَّبُوا هُنَاكَ قَرَابِينَهُمُ الْمُغِيظَةَ, وَقَدَّمُوا هُنَاكَ رَوَائِحَ سُرُورِهِمْ, وَسَكَبُوا هُنَاكَ سَكَائِبَهُمْ.
\par 29 فَقُلْتُ لَهُمْ: مَا هَذِهِ الْمُرْتَفَعَةُ الَّتِي تَأْتُونَ إِلَيْهَا؟ فَدُعِيَ اسْمُهَا [مُرْتَفَعَةً» إِلَى هَذَا الْيَوْمِ.
\par 30 لِذَلِكَ قُلْ لِبَيْتِ إِسْرَائِيلَ: هَكَذَا قَالَ السَّيِّدُ الرَّبُّ: هَلْ تَنَجَّسْتُمْ بِطَرِيقِ آبَائِكُمْ وَزَنَيْتُمْ وَرَاءَ أَرْجَاسِهِمْ؟
\par 31 وَبِتَقْدِيمِ عَطَايَاكُمْ وَإِجَازَةِ أَبْنَائِكُمْ فِي النَّارِ تَتَنَجَّسُونَ بِكُلِّ أَصْنَامِكُمْ إِلَى الْيَوْمِ. فَهَلْ أُسْأَلُ مِنْكُمْ يَا بَيْتَ إِسْرَائِيلَ؟ حَيٌّ أَنَا يَقُولُ السَّيِّدُ الرَّبُّ لاَ أُسْأَلُ مِنْكُمْ.
\par 32 وَالَّذِي يَخْطُرُ بِبَالِكُمْ لَنْ يَكُونَ, إِذْ تَقُولُونَ: نَكُونُ كَالأُمَمِ, كَقَبَائِلِ الأَرَاضِي فَنَعْبُدُ الْخَشَبَ وَالْحَجَرَ.
\par 33 حَيٌّ أَنَا يَقُولُ السَّيِّدُ الرَّبُّ إِنِّي بِيَدٍ قَوِيَّةٍ وَبِذِرَاعٍ مَمْدُودَةٍ وَبِسَخَطٍ مَسْكُوبٍ أَمْلِكُ عَلَيْكُمْ.
\par 34 وَأُخْرِجُكُمْ مِنْ بَيْنِ الشُّعُوبِ, وَأَجْمَعُكُمْ مِنَ الأَرَاضِي الَّتِي تَفَرَّقْتُمْ فِيهَا بِيَدٍ قَوِيَّةٍ وَبِذِرَاعٍ مَمْدُودَةٍ, وَبِسَخَطٍ مَسْكُوبٍ.
\par 35 وَآتِي بِكُمْ إِلَى بَرِّيَّةِ الشُّعُوبِ وَأُحَاكِمُكُمْ هُنَاكَ وَجْهاً لِوَجْهٍ.
\par 36 كَمَا حَاكَمْتُ آبَاءَكُمْ فِي بَرِّيَّةِ أَرْضِ مِصْرَ كَذَلِكَ أُحَاكِمُكُمْ, يَقُولُ السَّيِّدُ الرَّبُّ.
\par 37 وَأُمِرُّكُمْ تَحْتَ الْعَصَا, وَأُدْخِلُكُمْ فِي رِبَاطِ الْعَهْدِ.
\par 38 وَأَعْزِلُ مِنْكُمُ الْمُتَمَرِّدِينَ وَالْعُصَاةَ عَلَيَّ. أُخْرِجُهُمْ مِنْ أَرْضِ غُرْبَتِهِمْ وَلاَ يَدْخُلُونَ أَرْضَ إِسْرَائِيلَ, فَتَعْلَمُونَ أَنِّي أَنَا الرَّبُّ.
\par 39 [أَمَّا أَنْتُمْ يَا بَيْتَ إِسْرَائِيلَ فَهَكَذَا قَالَ السَّيِّدُ الرَّبُّ: اذْهَبُوا اعْبُدُوا كُلُّ إِنْسَانٍ أَصْنَامَهُ. وَبَعْدُ إِنْ لَمْ تَسْمَعُوا لِي فَلاَ تُنَجِّسُوا اسْمِي الْقُدُّوسَ بَعْدُ بِعَطَايَاكُمْ وَبِأَصْنَامِكُمْ.
\par 40 لأَنَّهُ فِي جَبَلِ قُدْسِي, فِي جَبَلِ إِسْرَائِيلَ الْعَالِي يَقُولُ السَّيِّدُ الرَّبُّ هُنَاكَ يَعْبُدُنِي كُلُّ بَيْتِ إِسْرَائِيلَ, كُلُّهُمْ فِي الأَرْضِ. هُنَاكَ أَرْضَى عَنْهُمْ, وَهُنَاكَ أَطْلُبُ تَقْدِمَاتِكُمْ وَبَاكُورَاتِ جِزَاكُمْ مَعَ جَمِيعِ مُقَدَّسَاتِكُمْ.
\par 41 بِرَائِحَةِ سُرُورِكُمْ أَرْضَى عَنْكُمْ, حِينَ أُخْرِجُكُمْ مِنْ بَيْنِ الشُّعُوبِ وَأَجْمَعُكُمْ مِنَ الأَرَاضِي الَّتِي تَفَرَّقْتُمْ فِيهَا, وَأَتَقَدَّسُ فِيكُمْ أَمَامَ عُيُونِ الأُمَمِ,
\par 42 فَتَعْلَمُونَ أَنِّي أَنَا الرَّبُّ, حِينَ آتِي بِكُمْ إِلَى أَرْضِ إِسْرَائِيلَ, إِلَى الأَرْضِ الَّتِي رَفَعْتُ يَدِي لأُعْطِي آبَاءَكُمْ إِيَّاهَا.
\par 43 وَهُنَاكَ تَذْكُرُونَ طُرُقَكُمْ وَكُلَّ أَعْمَالِكُمُ الَّتِي تَنَجَّسْتُمْ بِهَا, وَتَمْقُتُونَ أَنْفُسَكُمْ لِجَمِيعِ الشُّرُورِ الَّتِي فَعَلْتُمْ.
\par 44 فَتَعْلَمُونَ أَنِّي أَنَا الرَّبُّ إِذَا فَعَلْتُ بِكُمْ مِنْ أَجْلِ اسْمِي. لاَ كَطُرُقِكُمُ الشِّرِّيرَةِ, وَلاَ كَأَعْمَالِكُمُ الْفَاسِدَةِ يَا بَيْتَ إِسْرَائِيلَ, يَقُولُ السَّيِّدُ الرَّبُّ].
\par 45 وَكَانَ إِلَيَّ كَلاَمُ الرَّبِّ:
\par 46 [يَا ابْنَ آدَمَ, اجْعَلْ وَجْهَكَ نَحْوَ التَّيْمَنِ وَتَكَلَّمْ نَحْوَ الْجَنُوبِ, وَتَنَبَّأْ عَلَى وَعْرِ الْحَقْلِ فِي الْجَنُوبِ
\par 47 وَقُلْ لِوَعْرِ الْجَنُوبِ اسْمَعْ كَلاَمَ الرَّبِّ. هَكَذَا قَالَ السَّيِّدُ الرَّبُّ: هَئَنَذَا أُضْرِمُ فِيكَ نَاراً فَتَأْكُلُ كُلَّ شَجَرَةٍ خَضْرَاءَ فِيكَ وَكُلَّ شَجَرَةٍ يَابِسَةٍ. لاَ يُطْفَأُ لَهِيبُهَا الْمُلْتَهِبُ, وَتُحْرَقُ بِهَا كُلُّ الْوُجُوهِ مِنَ الْجَنُوبِ إِلَى الشِّمَالِ.
\par 48 فَيَرَى كُلُّ بَشَرٍ أَنِّي أَنَا الرَّبُّ أَضْرَمْتُهَا. لاَ تُطْفَأُ».
\par 49 فَقُلْتُ: [آهِ يَا سَيِّدُ الرَّبُّ! هُمْ يَقُولُونَ: أَمَا يُمَثِّلُ هُوَ أَمْثَالاً؟].

\chapter{21}

\par 1 وَكَانَ إِلَيَّ كَلاَمُ الرَّبِّ:
\par 2 [يَا ابْنَ آدَمَ, اجْعَلْ وَجْهَكَ نَحْوَ أُورُشَلِيمَ وَتَكَلَّمْ عَلَى الْمَقَادِسِ وَتَنَبَّأْ عَلَى أَرْضِ إِسْرَائِيلَ,
\par 3 وَقُلْ لأَرْضِ إِسْرَائِيلَ: هَكَذَا قَالَ الرَّبُّ: هَئَنَذَا عَلَيْكِ, وَأَسْتَلُّ سَيْفِي مِنْ غِمْدِهِ فَأَقْطَعُ مِنْكِ الصِّدِّيقَ وَالشِّرِّيرَ
\par 4 مِنْ حَيْثُ أَنِّي أَقْطَعُ مِنْكِ الصِّدِّيقَ وَالشِّرِّيرَ, فَلِذَلِكَ يَخْرُجُ سَيْفِي مِنْ غِمْدِهِ عَلَى كُلِّ بَشَرٍ مِنَ الْجَنُوبِ إِلَى الشِّمَالِ.
\par 5 فَيَعْلَمُ كُلُّ بَشَرٍ أَنِّي أَنَا الرَّبُّ, سَلَلْتُ سَيْفِي مِنْ غِمْدِهِ. لاَ يَرْجِعُ أَيْضاً.
\par 6 أَمَّا أَنْتَ يَا ابْنَ آدَمَ فَتَنَهَّدْ بِانْكِسَارِ الْحَقَوَيْنِ, وَبِمَرَارَةٍ تَنَهَّدْ أَمَامَ عُيُونِهِمْ.
\par 7 وَيَكُونُ إِذَا سَأَلُوكَ: عَلَى مَ تَتَنَهَّدُ؟ أَنَّكَ تَقُولُ: عَلَى الْخَبَرِ, لأَنَّهُ جَاءٍ فَيَذُوبُ كُلُّ قَلْبٍ, وَتَرْتَخِي كُلُّ الأَيْدِي وَتَيْأَسُ كُلُّ رُوحٍ, وَكُلُّ الرُّكَبِ تَصِيرُ كَالْمَاءِ, هَا هِيَ آتِيَةٌ وَتَكُونُ, يَقُولُ السَّيِّدُ الرَّبُّ].
\par 8 وَكَانَ إِلَيَّ كَلاَمُ الرَّبِّ:
\par 9 [يَا ابْنَ آدَمَ, تَنَبَّأْ وَقُلْ: هَكَذَا قَالَ الرَّبُّ: سَيْفٌ سَيْفٌ حُدِّدَ وَصُقِلَ أَيْضاً.
\par 10 قَدْ حُدِّدَ لِيَذْبَحَ ذَبْحاً. قَدْ صُقِلَ لِيَبْرُقَ. فَهَلْ نَبْتَهِجُ؟ عَصَا ابْنِي تَزْدَرِي بِكُلِّ عُودٍ.
\par 11 وَقَدْ أَعْطَاهُ لِيُصْقَلَ لِيُمْسَكَ بِالْكَفِّ. هَذَا السَّيْفُ قَدْ حُدِّدَ وَهُوَ مَصْقُولٌ لِيُسَلَّمَ لِيَدِ الْقَاتِلِ.
\par 12 اصْرُخْ وَوَلْوِلْ يَا ابْنَ آدَمَ, لأَنَّهُ يَكُونُ عَلَى شَعْبِي وَعَلَى كُلِّ رُؤَسَاءِ إِسْرَائِيلَ. أَهْوَالٌ بِسَبَبِ السَّيْفِ تَكُونُ عَلَى شَعْبِي. لِذَلِكَ اصْفِقْ عَلَى فَخْذِكَ.
\par 13 لأَنَّهُ امْتِحَانٌ. وَمَاذَا إِنْ لَمْ تَكُنْ أَيْضاً الْعَصَا الْمُزْدَرِيَةُ يَقُولُ السَّيِّدُ الرَّبُّ؟
\par 14 فَتَنَبَّأْ أَنْتَ يَا ابْنَ آدَمَ وَاصْفِقْ كَفّاً عَلَى كَفٍّ, وَلْيُعَدِ السَّيْفُ ثَالِثَةً. هُوَ سَيْفُ الْقَتْلَى, سَيْفُ الْقَتْلِ الْعَظِيمِ الْمُحِيقُ بِهِمْ.
\par 15 لِذَوَبَانِ الْقَلْبِ وَتَكْثِيرِ الْمَهَالِكِ, لِذَلِكَ جَعَلْتُ عَلَى كُلِّ الأَبْوَابِ سَيْفاً مُتَقَلِّباً. آهِ! قَدْ جُعِلَ بَرَّاقاً. هُوَ مَصْقُولٌ لِلذَّبْحِ.
\par 16 انْضَمَّ. يَمِّنِ. انْتَصِبْ. شَمِّلْ حَيْثُمَا تَوَجَّهَ حَدُّكَ.
\par 17 وَأَنَا أَيْضاً أُصَفِّقُ كَفِّي عَلَى كَفِّي وَأُسَكِّنُ غَضَبِي. أَنَا الرَّبُّ تَكَلَّمْتُ].
\par 18 وَكَانَ إِلَيَّ كَلاَمُ الرَّبِّ:
\par 19 [وَأَنْتَ يَا ابْنَ آدَمَ عَيِّنْ لِنَفْسِكَ طَرِيقَيْنِ لِمَجِيءِ سَيْفِ مَلِكِ بَابِلَ. مِنْ أَرْضٍ وَاحِدَةٍ تَخْرُجُ الاِثْنَتَانِ. وَاصْنَعْ صُوَّةً عَلَى رَأْسِ طَرِيقِ الْمَدِينَةِ.
\par 20 عَيِّنْ طَرِيقاً لِيَأْتِيَ السَّيْفُ عَلَى رَبَّةِ بَنِي عَمُّونَ وَعَلَى يَهُوذَا فِي أُورُشَلِيمَ الْمَنِيعَةِ.
\par 21 لأَنَّ مَلِكَ بَابِلَ قَدْ وَقَفَ عَلَى أُمِّ الطَّرِيقِ, عَلَى رَأْسِ الطَّرِيقَيْنِ لِيَعْرِفَ عِرَافَةً. صَقَلَ السِّهَامَ. سَأَلَ بِالتَّرَافِيمِ. نَظَرَ إِلَى الْكَبِدِ.
\par 22 عَنْ يَمِينِهِ كَانَتِ الْعِرَافَةُ عَلَى أُورُشَلِيمَ لِوَضْعِ الْمَجَانِقِ, لِفَتْحِ الْفَمِ فِي الْقَتْلِ, وَلِرَفْعِ الصَّوْتِ بِالْهُتَافِ, لِوَضْعِ الْمَجَانِقِ عَلَى الأَبْوَابِ, لإِقَامَةِ مِتْرَسَةٍ لِبِنَاءِ بُرْجٍ.
\par 23 وَتَكُونُ لَهُمْ مِثْلَ عِرَافَةٍ كَاذِبَةٍ فِي عُيُونِهِمِ الْحَالِفِينَ لَهُمْ حَلْفاً. لَكِنَّهُ يَذْكُرُ الإِثْمَ حَتَّى يُؤْخَذُوا».
\par 24 لِذَلِكَ هَكَذَا قَالَ السَّيِّدُ الرَّبُّ: [مِنْ أَجْلِ أَنَّكُمْ ذَكَّرْتُمْ بِإِثْمِكُمْ عِنْدَ انْكِشَافِ مَعَاصِيكُمْ لإِظْهَارِ خَطَايَاكُمْ فِي جَمِيعِ أَعْمَالِكُمْ, فَمِنْ تَذْكِيرِكُمْ تُؤْخَذُونَ بِالْيَدِ.
\par 25 [وَأَنْتَ أَيُّهَا النَّجِسُ الشِّرِّيرُ, رَئِيسُ إِسْرَائِيلَ الَّذِي قَدْ جَاءَ يَوْمُهُ فِي زَمَانِ إِثْمِ النِّهَايَةِ,
\par 26 هَكَذَا قَالَ السَّيِّدُ الرَّبُّ: انْزِعِ الْعَمَامَةَ. ارْفَعِ التَّاجَ. هَذِهِ لاَ تِلْكَ. ارْفَعِ الْوَضِيعَ, وَضَعِ الرَّفِيعَ.
\par 27 مُنْقَلِباً مُنْقَلِباً مُنْقَلِباً أَجْعَلُهُ. هَذَا أَيْضاً لاَ يَكُونُ حَتَّى يَأْتِيَ الَّذِي لَهُ الْحُكْمُ فَأُعْطِيَهُ إِيَّاهُ.
\par 28 [وَأَنْتَ يَا ابْنَ آدَمَ فَتَنَبَّأْ وَقُلْ: هَكَذَا قَالَ السَّيِّدُ الرَّبُّ فِي بَنِي عَمُّونَ وَفِي تَعْيِيرِهِمْ: سَيْفٌ! سَيْفٌ مَسْلُولٌ لِلذَّبْحِ. مَصْقُولٌ لِلْغَايَةِ لِلْبَرِيقِ.
\par 29 إِذْ يَرُونَ لَكَ بَاطِلاً, إِذْ يَعْرِفُونَ لَكَ كَذِباً لِيَجْعَلُوكَ عَلَى أَعْنَاقِ الْقَتْلَى الأَشْرَارِ الَّذِينَ جَاءَ يَوْمُهُمْ فِي زَمَانِ إِثْمِ النِّهَايَةِ.
\par 30 فَهَلْ أُعِيدُهُ إِلَى غِمْدِهِ؟ أَلاَ فِي الْمَوْضِعِ الَّذِي خُلِقْتِ فِيهِ فِي مَوْلِدِكِ أُحَاكِمُكِ!
\par 31 وَأَسْكُبُ عَلَيْكِ غَضَبِي, وَأَنْفُخُ عَلَيْكِ بِنَارِ غَيْظِي, وَأُسَلِّمُكِ لِيَدِ رِجَالٍ مُتَحَرِّقِينَ مَاهِرِينَ لِلإِهْلاَكِ.
\par 32 تَكُونِينَ أَكْلَةً لِلنَّارِ. دَمُكِ يَكُونُ فِي وَسَطِ الأَرْضِ. لاَ تُذْكَرِينَ, لأَنِّي أَنَا الرَّبُّ تَكَلَّمْتُ].

\chapter{22}

\par 1 وَكَانَ إِلَيَّ كَلاَمُ الرَّبِّ:
\par 2 [وَأَنْتَ يَا ابْنَ آدَمَ, هَلْ تَدِينُ, هَلْ تَدِينُ مَدِينَةَ الدِّمَاءِ؟ فَعَرِّفْهَا كُلَّ رَجَاسَاتِهَا
\par 3 وَقُلْ: هَكَذَا قَالَ السَّيِّدُ الرَّبُّ: أَيَّتُهَا الْمَدِينَةُ السَّافِكَةُ الدَّمِ فِي وَسَطِهَا لِيَأْتِيَ وَقْتُهَا, الصَّانِعَةُ أَصْنَاماً لِنَفْسِهَا لِتَتَنَجَّسَ بِهَا,
\par 4 قَدْ أَثِمْتِ بِدَمِكِ الَّذِي سَفَكْتِ, وَنَجَّسْتِ نَفْسَكِ بِأَصْنَامِكِ الَّتِي عَمِلْتِ, وَقَرَّبْتِ أَيَّامَكِ وَبَلَغْتِ سِنِيكِ. فَلِذَلِكَ جَعَلْتُكِ عَاراً لِلأُمَمِ وَسُخْرَةً لِجَمِيعِ الأَرَاضِي.
\par 5 الْقَرِيبَةُ إِلَيْكِ وَالْبَعِيدَةُ عَنْكِ يَسْخَرُونَ مِنْكِ, يَا نَجِسَةَ الاِسْمِ يَا كَثِيرَةَ الشَّغَبِ.
\par 6 هُوَذَا رُؤَسَاءُ إِسْرَائِيلَ, كُلُّ وَاحِدٍ حَسَبَ اسْتِطَاعَتِهِ, كَانُوا فِيكِ لأَجْلِ سَفْكِ الدَّمِ.
\par 7 فِيكِ أَهَانُوا أَباً وَأُمّاً. فِي وَسَطِكِ عَامَلُوا الْغَرِيبَ بِالظُّلْمِ. فِيكِ اضْطَهَدُوا الْيَتِيمَ وَالأَرْمَلَةَ.
\par 8 ازْدَرَيْتِ أَقْدَاسِي وَنَجَّسْتِ سُبُوتِي.
\par 9 كَانَ فِيكِ أُنَاسٌ وُشَاةٌ لِسَفْكِ الدَّمِ, وَفِيكِ أَكَلُوا عَلَى الْجِبَالِ. فِي وَسَطِكِ عَمِلُوا رَذِيلَةً.
\par 10 فِيكِ كَشَفَ الإِنْسَانُ عَوْرَةَ أَبِيهِ. فِيكِ أَذَلُّوا الْمُتَنَجِّسَةَ بِطَمْثِهَا.
\par 11 إِنْسَانٌ فَعَلَ الرِّجْسَ بِامْرَأَةِ قَرِيبِهِ. إِنْسَانٌ نَجَّسَ كَنَّتَهُ بِرَذِيلَةٍ. إِنْسَانٌ أَذَلَّ فِيكِ أُخْتَهُ بِنْتَ أَبِيهِ.
\par 12 فِيكِ أَخَذُوا الرَّشْوَةَ لِسَفْكِ الدَّمِ. أَخَذْتِ الرِّبَا وَالْمُرَابَحَةَ وَسَلَبْتِ أَقْرِبَاءَكِ بِالظُّلْمِ, وَنَسِيتِنِي يَقُولُ السَّيِّدُ الرَّبُّ.
\par 13 فَهَئَنَذَا قَدْ صَفَّقْتُ بِكَفِّي بِسَبَبِ خَطْفِكِ الَّذِي خَطَفْتِ, وَبِسَبَبِ دَمِكِ الَّذِي كَانَ فِي وَسَطِكِ.
\par 14 فَهَلْ يَثْبُتُ قَلْبُكِ أَوْ تَقْوَى يَدَاكِ فِي الأَيَّامِ الَّتِي فِيهَا أُعَامِلُكِ؟ أَنَا الرَّبَّ تَكَلَّمْتُ وَسَأَفْعَلُ.
\par 15 وَأُبَدِّدُكِ بَيْنَ الأُمَمِ, وَأُذَرِّيكِ فِي الأَرَاضِي, وَأُزِيلُ نَجَاسَتَكِ مِنْكِ.
\par 16 وَتَتَدَنَّسِينَ بِنَفْسِكِ أَمَامَ عُيُونِ الأُمَمِ, وَتَعْلَمِينَ أَنِّي أَنَا الرَّبُّ].
\par 17 وَكَانَ إِلَيَّ كَلاَمُ الرَّبِّ:
\par 18 [يَا ابْنَ آدَمَ, قَدْ صَارَ لِي بَيْتُ إِسْرَائِيلَ زَغَلاً. كُلُّهُمْ نُحَاسٌ وَقَصْدِيرٌ وَحَدِيدٌ وَرَصَاصٌ فِي وَسَطِ كُورٍ. صَارُوا زَغَلَ فِضَّةٍ.
\par 19 لأَجْلِ ذَلِكَ هَكَذَا قَالَ السَّيِّدُ الرَّبُّ: مِنْ حَيْثُ إِنَّكُمْ كُلَّكُمْ صِرْتُمْ زَغَلاً, فَلِذَلِكَ هَئَنَذَا أَجْمَعُكُمْ فِي وَسَطِ أُورُشَلِيمَ
\par 20 جَمْعَ فِضَّةٍ وَنُحَاسٍ وَحَدِيدٍ وَرَصَاصٍ وَقَصْدِيرٍ إِلَى وَسَطِ كُورٍ لِنَفْخِ النَّارِ عَلَيْهَا لِسَبْكِهَا, كَذَلِكَ أَجْمَعُكُمْ بِغَضَبِي وَسَخَطِي وَأَطْرَحُكُمْ وَأَسْبِكُكُمْ.
\par 21 فَأَجْمَعُكُمْ وَأَنْفُخُ عَلَيْكُمْ فِي نَارِ غَضَبِي, فَتُسْبَكُونَ فِي وَسَطِهَا.
\par 22 كَمَا تُسْبَكُ الْفِضَّةُ فِي وَسَطِ الْكُورِ كَذَلِكَ تُسْبَكُونَ فِي وَسَطِهَا, فَتَعْلَمُونَ أَنِّي أَنَا الرَّبُّ سَكَبْتُ سَخَطِي عَلَيْكُمْ].
\par 23 وَكَانَ إِلَيَّ كَلاَمُ الرَّبِّ:
\par 24 [يَا ابْنَ آدَمَ, قُلْ لَهَا: أَنْتِ الأَرْضُ الَّتِي لَمْ تَطْهُرْ. لَمْ يُمْطَرْ عَلَيْهَا فِي يَوْمِ الْغَضَبِ.
\par 25 فِتْنَةُ أَنْبِيَائِهَا فِي وَسَطِهَا كَأَسَدٍ مُزَمْجِرٍ يَخْطُفُ الْفَرِيسَةَ. أَكَلُوا نُفُوساً. أَخَذُوا الْكَنْزَ وَالنَّفِيسَ. أَكْثَرُوا أَرَامِلَهَا فِي وَسَطِهَا.
\par 26 كَهَنَتُهَا خَالَفُوا شَرِيعَتِي وَنَجَّسُوا أَقْدَاسِي. لَمْ يُمَيِّزُوا بَيْنَ الْمُقَدَّسِ وَالْمُحَلَّلِ, وَلَمْ يَعْلَمُوا الْفَرْقَ بَيْنَ النَّجِسِ وَالطَّاهِرِ, وَحَجَبُوا عُيُونَهُمْ عَنْ سُبُوتِي فَتَدَنَّسْتُ فِي وَسَطِهِمْ.
\par 27 رُؤَسَاؤُهَا فِي وَسَطِهَا كَذِئَابٍ خَاطِفَةٍ خَطْفاً لِسَفْكِ الدَّمِ, لإِهْلاَكِ النُّفُوسِ لاِكْتِسَابِ كَسْبٍ.
\par 28 وَأَنْبِيَاؤُهَا قَدْ طَيَّنُوا لَهُمْ بِالطُّفَالِ, رَائِينَ بَاطِلاً وَعَارِفِينَ لَهُمْ كَذِباً, قَائِلِينَ: هَكَذَا قَالَ السَّيِّدُ الرَّبُّ وَالرَّبُّ لَمْ يَتَكَلَّمْ!
\par 29 شَعْبُ الأَرْضِ ظَلَمُوا ظُلْماً وَغَصَبُوا غَصْباً, وَاضْطَهَدُوا الْفَقِيرَ وَالْمِسْكِينَ, وَظَلَمُوا الْغَرِيبَ بِغَيْرِ الْحَقِّ.
\par 30 وَطَلَبْتُ مِنْ بَيْنِهِمْ رَجُلاً يَبْنِي جِدَاراً وَيَقِفُ فِي الثَّغْرِ أَمَامِي عَنِ الأَرْضِ لِكَيْلاَ أَخْرِبَهَا, فَلَمْ أَجِدْ!
\par 31 فَسَكَبْتُ سَخَطِي عَلَيْهِمْ. أَفْنَيْتُهُمْ بِنَارِ غَضَبِي. جَلَبْتُ طَرِيقَهُمْ عَلَى رُؤُوسِهِمْ يَقُولُ السَّيِّدُ الرَّبُّ].

\chapter{23}

\par 1 وَكَانَ إِلَيَّ كَلاَمُ الرَّبِّ:
\par 2 [يَا ابْنَ آدَمَ, كَانَتِ امْرَأَتَانِ ابْنَتَا أُمٍّ وَاحِدَةٍ,
\par 3 زَنَتَا بِمِصْرَ فِي صِبَاهُمَا. هُنَاكَ دُغْدِغَتْ ثُدِيُّهُمَا, وَهُنَاكَ تَزَغْزَغَتْ تَرَائِبُ عُذْرَتِهِمَا.
\par 4 وَاسْمُهُمَا: أُهُولَةُ الْكَبِيرَةُ, وَأُهُولِيبَةُ أُخْتُهَا. وَكَانَتَا لِي, وَوَلَدَتَا بَنِينَ وَبَنَاتٍ. وَاسْمَاهُمَا: السَّامِرَةُ أُهُولَةُ, وَأُورُشَلِيمُ أُهُولِيبَةُ.
\par 5 وَزَنَتْ أُهُولَةُ مِنْ تَحْتِي وَعَشِقَتْ مُحِبِّيهَا, أَشُّورَ الأَبْطَالَ
\par 6 اللاَّبِسِينَ الأَسْمَانْجُونِيَّ وُلاَةً وَشِحَناً, كُلُّهُمْ شُبَّانُ شَهْوَةٍ, فُرْسَانٌ رَاكِبُونَ الْخَيْلَ.
\par 7 فَدَفَعَتْ لَهُمْ عُقْرَهَا لِمُخْتَارِي بَنِي أَشُّورَ كُلِّهِمْ, وَتَنَجَّسَتْ بِكُلِّ مَنْ عَشِقَتْهُمْ بِكُلِّ أَصْنَامِهِمْ.
\par 8 وَلَمْ تَتْرُكْ زِنَاهَا مِنْ مِصْرَ أَيْضاً, لأَنَّهُمْ ضَاجَعُوهَا فِي صِبَاهَا وَزَغْزَغُوا تَرَائِبَ عُذْرَتِهَا وَسَكَبُوا عَلَيْهَا زِنَاهُمْ.
\par 9 لِذَلِكَ سَلَّمْتُهَا لِيَدِ عُشَّاقِهَا, لِيَدِ بَنِي أَشُّورَ الَّذِينَ عَشِقَتْهُمْ.
\par 10 هُمْ كَشَفُوا عَوْرَتَهَا. أَخَذُوا بَنِيهَا وَبَنَاتِهَا وَذَبَحُوهَا بِالسَّيْفِ, فَصَارَتْ عِبْرَةً لِلنِّسَاءِ. وَأَجْرُوا عَلَيْهَا حُكْماً.
\par 11 [فَلَمَّا رَأَتْ أُخْتُهَا أُهُولِيبَةُ ذَلِكَ أَفْسَدَتْ فِي عِشْقِهَا أَكْثَرَ مِنْهَا, وَفِي زِنَاهَا أَكْثَرَ مِنْ زِنَا أُخْتِهَا.
\par 12 عَشِقَتْ بَنِي أَشُّورَ الْوُلاَةَ وَالشِّحَنَ الأَبْطَالَ اللاَّبِسِينَ أَفْخَرَ لِبَاسٍ, فُرْسَاناً رَاكِبِينَ الْخَيْلَ كُلُّهُمْ شُبَّانُ شَهْوَةٍ.
\par 13 فَرَأَيْتُ أَنَّهَا قَدْ تَنَجَّسَتْ, وَلِكِلْتَيْهِمَا طَرِيقٌ وَاحِدَةٌ.
\par 14 وَزَادَتْ زِنَاهَا. وَلَمَّا نَظَرَتْ إِلَى رِجَالٍ مُصَوَّرِينَ عَلَى الْحَائِطِ, صُوَرُ الْكِلْدَانِيِّينَ مُصَوَّرَةًٍ بِمُغْرَةٍ,
\par 15 مُنَطَّقِينَ بِمَنَاطِقَ عَلَى أَحْقَائِهِمْ, عَمَائِمُهُمْ مَسْدُولَةٌ عَلَى رُؤُوسِهِمْ. كُلُّهُمْ فِي الْمَنْظَرِ رُؤَسَاءُ مَرْكَبَاتٍ شِبْهُ بَنِي بَابِلَ الْكِلْدَانِيِّينَ أَرْضُ مِيلاَدِهِمْ
\par 16 عَشِقَتْهُمْ عِنْدَ لَمْحِ عَيْنَيْهَا إِيَّاهُمْ, وَأَرْسَلَتْ إِلَيْهِمْ رُسُلاً إِلَى أَرْضِ الْكِلْدَانِيِّينَ.
\par 17 فَأَتَاهَا بَنُو بَابِلَ فِي مَضْجَعِ الْحُبِّ وَنَجَّسُوهَا بِزِنَاهُمْ, فَتَنَجَّسَتْ بِهِمْ وَجَفَتْهُمْ نَفْسُهَا.
\par 18 وَكَشَفَتْ زِنَاهَا وَكَشَفَتْ عَوْرَتَهَا, فَجَفَتْهَا نَفْسِي كَمَا جَفَتْ نَفْسِي أُخْتَهَا.
\par 19 وَأَكْثَرَتْ زِنَاهَا بِذِكْرِهَا أَيَّامَ صِبَاهَا الَّتِي فِيهَا زَنَتْ بِأَرْضِ مِصْرَ.
\par 20 وَعَشِقَتْ مَعْشُوقِيهِمِ الَّذِينَ لَحْمُهُمْ كَلَحْمِ الْحَمِيرِ وَمَنِيُّهُمْ كَمَنِيِّ الْخَيْلِ.
\par 21 وَافْتَقَدْتِ رَذِيلَةَ صِبَاكِ بِزَغْزَغَةِ الْمِصْرِيِّينَ تَرَائِبَكِ لأَجْلِ ثَدْيِ صِبَاكِ.
\par 22 [لأَجْلِ ذَلِكَ يَا أُهُولِيبَةُ, هَكَذَا قَالَ السَّيِّدُ الرَّبُّ: هَئَنَذَا أُهَيِّجُ عَلَيْكِ عُشَّاقَكِ الَّذِينَ جَفَتْهُمْ نَفْسُكِ, وَآتِي بِهِمْ عَلَيْكِ مِنْ كُلِّ جِهَةٍ:
\par 23 بَنِي بَابِلَ وَكُلَّ الْكِلْدَانِيِّينَ, فَقُودَ وَشُوعَ وَقُوعَ, وَمَعَهُمْ كُلُّ بَنِي أَشُّورَ, شُبَّانُ شَهْوَةٍ, وُلاَةٌ وَشِحَنٌ كُلُّهُمْ رُؤَسَاءُ مَرْكَبَاتٍ وَشُهَرَاءُ. كُلُّهُمْ رَاكِبُونَ الْخَيْلَ.
\par 24 فَيَأْتُونَ عَلَيْكِ بِأَسْلِحَةٍ: مَرْكَبَاتٍ وَعَجَلاَتٍ, وَبِجَمَاعَةِ شُعُوبٍ يُقِيمُونَ عَلَيْكِ التُّرْسَ وَالْمِجَنَّ وَالْخُوذَةَ مِنْ حَوْلِكِ, وَأُسَلِّمُ لَهُمُ الْحُكْمَ فَيَحْكُمُونَ عَلَيْكِ بِأَحْكَامِهِمْ.
\par 25 وَأَجْعَلُ غَيْرَتِي عَلَيْكِ فَيُعَامِلُونَكِ بِالسَّخَطِ. يَقْطَعُونَ أَنْفَكِ وَأُذُنَيْكِ, وَبَقِيَّتُكِ تَسْقُطُ بِالسَّيْفِ. يَأْخُذُونَ بَنِيكِ وَبَنَاتِكِ, وَتُؤْكَلُ بَقِيَّتُكِ بِالنَّارِ.
\par 26 وَيَنْزِعُونَ عَنْكِ ثِيَابَكِ وَيَأْخُذُونَ أَدَوَاتِ زِينَتِكِ.
\par 27 وَأُبَطِّلُ رَذِيلَتَكِ عَنْكِ وَزِنَاكِ مِنْ أَرْضِ مِصْرَ, فَلاَ تَرْفَعِينَ عَيْنَيْكِ إِلَيْهِمْ وَلاَ تَذْكُرِينَ مِصْرَ بَعْدُ.
\par 28 لأَنَّهُ هَكَذَا قَالَ السَّيِّدُ الرَّبُّ: هَئَنَذَا أُسَلِّمُكِ لِيَدِ الَّذِينَ أَبْغَضْتِهِمْ, لِيَدِ الَّذِينَ جَفَتْهُمْ نَفْسُكِ.
\par 29 فَيُعَامِلُونَكِ بِالْبَغْضَاءِ وَيَأْخُذُونَ كُلَّ تَعَبِكِ, وَيَتْرُكُونَكِ عُرْيَانَةً وَعَارِيَةً فَتَنْكَشِفُ عَوْرَةُ زِنَاكِ وَرَذِيلَتُكِ وَزِنَاكِ.
\par 30 أَفْعَلُ بِكِ هَذَا لأَنَّكِ زَنَيْتِ وَرَاءَ الأُمَمِ. لأَنَّكِ تَنَجَّسْتِ بِأَصْنَامِهِمْ.
\par 31 فِي طَرِيقِ أُخْتِكِ سَلَكْتِ فَأَدْفَعُ كَأْسَهَا لِيَدِكِ.
\par 32 هَكَذَا قَالَ السَّيِّدُ الرَّبُّ: إِنَّكِ تَشْرَبِينَ كَأْسَ أُخْتِكِ الْعَمِيقَةَ الْكَبِيرَةَ. تَكُونِينَ لِلضِّحْكِ وَالاِسْتِهْزَاءِ. تَسَعُ كَثِيراً.
\par 33 تَمْتَلِئِينَ سُكْراً وَحُزْناً, كَأْسَ التَّحَيُّرِ وَالْخَرَابِ, كَأْسَ أُخْتِكِ السَّامِرَةِ.
\par 34 فَتَشْرَبِينَهَا وَتَمْتَصِّينَهَا وَتَقْضَمِينَ شُقَفَهَا وَتَجْتَثِّينَ ثَدْيَيْكِ, لأَنِّي تَكَلَّمْتُ يَقُولُ السَّيِّدُ الرَّبُّ.
\par 35 لِذَلِكَ هَكَذَا قَالَ السَّيِّدُ الرَّبُّ: مِنْ أَجْلِ أَنَّكِ نَسِيتِنِي وَطَرَحْتِنِي وَرَاءَ ظَهْرِكِ فَتَحْمِلِي أَيْضاً رَذِيلَتَكِ وَزِنَاك].
\par 36 وَقَالَ الرَّبُّ لِي: [يَا ابْنَ آدَمَ, أَتَحْكُمُ عَلَى أُهُولَةَ وَأُهُولِيبَةَ؟ بَلْ أَخْبِرْهُمَا بِرَجَاسَاتِهِمَا
\par 37 لأَنَّهُمَا قَدْ زَنَتَا وَفِي أَيْدِيهِمَا دَمٌ, وَزَنَتَا بِأَصْنَامِهِمَا وَأَيْضاً أَجَازَتَا بَنِيهِمَا الَّذِينَ وَلَدَتَاهُمْ لِي النَّارَ أَكْلاً لَهَا.
\par 38 وَفَعَلَتَا أَيْضاً بِي هَذَا: نَجَّسَتَا مَقْدِسِي فِي ذَلِكَ الْيَوْمِ وَدَنَّسَتَا سُبُوتِي.
\par 39 وَلَمَّا ذَبَحَتَا بَنِيهِمَا لأَصْنَامِهِمَا أَتَتَا فِي ذَلِكَ الْيَوْمِ إِلَى مَقْدِسِي لِتُنَجِّسَاهُ. فَهُوَذَا هَكَذَا فَعَلَتَا فِي وَسَطِ بَيْتِي.
\par 40 بَلْ أَرْسَلْتُمَا إِلَى رِجَالٍ آتِينَ مِنْ بَعِيدٍ. الَّذِينَ أُرْسِلَ إِلَيْهِمْ رَسُولٌ فَهُوَذَا جَاءُوا. هُمُ الَّذِينَ لأَجْلِهِمِ اسْتَحْمَمْتِ وَكَحَّلْتِ عَيْنَيْكِ وَتَحَلَّيْتِ بِالْحُلِيِّ
\par 41 وَجَلَسْتِ عَلَى سَرِيرٍ فَاخِرٍ أَمَامَهُ مَائِدَةٌ مُنَضَّضَةٌ, وَوَضَعْتِ عَلَيْهَا بَخُورِي وَزَيْتِي.
\par 42 وَصَوْتُ جُمْهُورٍ مُتَرَفِّهِينَ مَعَهَا, مَعَ أُنَاسٍ مِنْ رَعَاعِ الْخَلْقِ. أُتِيَ بِسُكَارَى مِنَ الْبَرِّيَّةِ الَّذِينَ جَعَلُوا أَسْوِرَةً عَلَى أَيْدِيهِمَا وَتَاجَ جَمَالٍ عَلَى رُؤُوسِهِمَا.
\par 43 فَقُلْتُ عَنِ الْبَالِيَةِ فِي الزِّنَى: آلآنَ يَزْنُونَ مَعَهَا أَيْضاً.
\par 44 فَدَخَلُوا عَلَيْهَا كَمَا يُدْخَلُ عَلَى امْرَأَةٍ زَانِيَةٍ. هَكَذَا دَخَلُوا عَلَى أُهُولَةَ وَعَلَى أُهُولِيبَةَ الْمَرْأَتَيْنِ الزَّانِيَتَيْنِ.
\par 45 وَالرِّجَالُ الصِّدِّيقُونَ هُمْ يَحْكُمُونَ عَلَيْهِمَا حُكْمَ زَانِيَةٍ وَحُكْمَ سَفَّاكَةِ الدَّمِ, لأَنَّهُمَا زَانِيَتَانِ وَفِي أَيْدِيهِمَا دَمٌ.
\par 46 لأَنَّهُ هَكَذَا قَالَ السَّيِّدُ الرَّبُّ: إِنِّي أُصْعِدُ عَلَيْهِمَا جَمَاعَةً وَأُسَلِّمُهُمَا لِلْجَوْرِ وَالنَّهْبِ.
\par 47 وَتَرْجُمُهُمَا الْجَمَاعَةُ بِالْحِجَارَةِ وَيُقَطِّعُونَهُمَا بِسُيُوفِهِمْ, وَيَذْبَحُونَ أَبْنَاءَهُمَا وَبَنَاتِهِمَا, وَيُحْرِقُونَ بُيُوتَهُمَا بِالنَّارِ.
\par 48 فَأُبَطِّلُ الرَّذِيلَةَ مِنَ الأَرْضِ, فَتَتَأَدَّبُ جَمِيعُ النِّسَاءِ وَلاَ يَفْعَلْنَ مِثْلَ رَذِيلَتِكُمَا.
\par 49 وَيَرُدُّونَ عَلَيْكُمَا رَذِيلَتَكُمَا, فَتَحْمِلاَنِ خَطَايَا أَصْنَامِكُمَا, وَتَعْلَمَانِ أَنِّي أَنَا السَّيِّدُ الرَّبُّ].

\chapter{24}

\par 1 وَكَانَ كَلاَمُ الرَّبِّ إِلَيَّ فِي السَّنَةِ التَّاسِعَةِ فِي الشَّهْرِ الْعَاشِرِ فِي الْعَاشِرِ مِنَ الشَّهْرِ:
\par 2 [يَا ابْنَ آدَمَ, اكْتُبْ لِنَفْسِكَ اسْمَ الْيَوْمِ, هَذَا الْيَوْمَ بِعَيْنِهِ. فَإِنَّ مَلِكَ بَابِلَ قَدِ اقْتَرَبَ إِلَى أُورُشَلِيمَ هَذَا الْيَوْمَ بِعَيْنِهِ.
\par 3 وَاضْرِبْ مَثَلاً لِلْبَيْتِ الْمُتَمَرِّدِ وَقُلْ لَهُمْ: هَكَذَا قَالَ السَّيِّدُ الرَّبُّ: ضَعِ الْقِدْرَ. ضَعْهَا وَأَيْضاً صُبَّ فِيهَا مَاءً.
\par 4 اِجْمَعْ إِلَيْهَا قِطَعَهَا, كُلَّ قِطْعَةٍ طَيِّبَةٍ. الْفَخْذَ وَالْكَتِفَ. امْلأُوهَا بِخِيَارِ الْعِظَامِ.
\par 5 خُذْ مِنْ خِيَارِ الْغَنَمِ وَكُومَةَ الْعِظَامِ تَحْتَهَا. أَغْلِهَا إِغْلاَءً فَتُسْلَقَ أَيْضاً عِظَامُهَا فِي وَسَطِهَا].
\par 6 لِذَلِكَ هَكَذَا قَالَ السَّيِّدُ الرَّبُّ: [وَيْلٌ لِمَدِينَةِ الدِّمَاءِ, الْقِدْرِ الَّتِي فِيهَا زِنْجَارُهَا وَمَا خَرَجَ مِنْهَا زِنْجَارُهَا. أَخْرِجُوهَا قِطْعَةً قِطْعَةً. لاَ تَقَعُ عَلَيْهَا قُرْعَةٌ.
\par 7 لأَنَّ دَمَهَا فِي وَسَطِهَا. قَدْ وَضَعَتْهُ عَلَى ضِحِّ الصَّخْرِ. لَمْ تُرِقْهُ عَلَى الأَرْضِ لِتُوارِيهِ بِالتُّرَابِ.
\par 8 لِصُعُودِ الْغَضَبِ, لِتُنْقَمَ نَقْمَةً, وَضَعْتُ دَمَهَا عَلَى ضِحِّ الصَّخْرِ لِئَلاَّ يُوارَى.
\par 9 لِذَلِكَ هَكَذَا قَالَ السَّيِّدُ الرَّبُّ: وَيْلٌ لِمَدِينَةِ الدِّمَاءِ. إِنِّي أَنَا أُعَظِّمُ كُومَتَهَا.
\par 10 كَثِّرِ الْحَطَبَ. أَضْرِمِ النَّارَ. أَنْضِجِ اللَّحْمَ. تَبِّلْهُ تَتْبِيلاً, وَلْتُحْرَقِ الْعِظَامُ.
\par 11 ثُمَّ ضَعْهَا فَارِغَةً عَلَى الْجَمْرِ لِيَحْمَى نُحَاسُهَا وَيُحْرَقَ فَيَذُوبَ قَذَرُهَا فِيهَا وَيَفْنَى زِنْجَارُهَا.
\par 12 بِمَشَقَّاتٍ تَعِبَتْ وَلَمْ تَخْرُجْ مِنْهَا كَثْرَةُ زِنْجَارِهَا. فِي النَّارِ زِنْجَارُهَا.
\par 13 فِي نَجَاسَتِكِ رَذِيلَةٌ لأَنِّي طَهَّرْتُكِ فَلَمْ تَطْهُرِي وَلَنْ تَطْهُرِي بَعْدُ مِنْ نَجَاسَتِكِ حَتَّى أُحِلَّ غَضَبِي عَلَيْكِ.
\par 14 أَنَا الرَّبَّ تَكَلَّمْتُ. يَأْتِي فَأَفْعَلُهُ. لاَ أُطْلِقُ وَلاَ أُشْفِقُ وَلاَ أَنْدَمُ. حَسَبَ طُرُقِكِ وَحَسَبَ أَعْمَالِكِ يَحْكُمُونَ عَلَيْكِ, يَقُولُ السَّيِّدُ الرَّبُّ].
\par 15 وَكَانَ إِلَيَّ كَلاَمُ الرَّبِّ:
\par 16 [يَا ابْنَ آدَمَ, هَئَنَذَا آخُذُ عَنْكَ شَهْوَةَ عَيْنَيْكَ بِضَرْبَةٍ, فَلاَ تَنُحْ وَلاَ تَبْكِ وَلاَ تَنْزِلْ دُمُوعُكَ.
\par 17 تَنَهَّدْ سَاكِتاً. لاَ تَعْمَلْ مَنَاحَةً عَلَى أَمْوَاتٍ. لُفَّ عِصَابَتَكَ عَلَيْكَ وَاجْعَلْ نَعْلَيْكَ فِي رِجْلَيْكَ وَلاَ تُغَطِّ شَارِبَيْكَ وَلاَ تَأْكُلْ مِنْ خُبْزِ النَّاسِ».
\par 18 فَكَلَّمْتُ الشَّعْبَ صَبَاحاً وَمَاتَتْ زَوْجَتِي مَسَاءً. وَفَعَلْتُ فِي الْغَدِ كَمَا أُمِرْتُ.
\par 19 فَسَأَلَنِي الشَّعْبُ: [أَلاَ تُخْبِرُنَا مَا لَنَا وَهَذِهِ الَّتِي أَنْتَ صَانِعُهَا؟»
\par 20 فَأَجَبْتُهُمْ: [قَدْ كَانَ إِلَيَّ كَلاَمُ الرَّبِّ:
\par 21 كَلِّمْ بَيْتَ إِسْرَائِيلَ. هَكَذَا قَالَ السَّيِّدُ الرَّبُّ: هَئَنَذَا مُنَجِّسٌ مَقْدِسِي فَخْرَ عِزِّكُمْ شَهْوَةَ أَعْيُنِكُمْ وَلَذَّةَ نُفُوسِكُمْ. وَأَبْنَاؤُكُمْ وَبَنَاتُكُمُ الَّذِينَ خَلَّفْتُمْ يَسْقُطُونَ بِالسَّيْفِ,
\par 22 وَتَفْعَلُونَ كَمَا فَعَلْتُ: لاَ تُغَطُّونَ شَوَارِبَكُمْ وَلاَ تَأْكُلُونَ مِنْ خُبْزِ النَّاسِ.
\par 23 وَتَكُونُ عَصَائِبُكُمْ عَلَى رُؤُوسِكُمْ وَنِعَالُكُمْ فِي أَرْجُلِكُمْ. لاَ تَنُوحُونَ وَلاَ تَبْكُونَ وَتَفْنُونَ بِآثَامِكُمْ. تَئِنُّونَ بَعْضُكُمْ عَلَى بَعْضٍ.
\par 24 وَيَكُونُ حِزْقِيَالُ لَكُمْ آيَةً. مِثْلَ كُلِّ مَا صَنَعَ تَصْنَعُونَ. إِذَا جَاءَ هَذَا تَعْلَمُونَ أَنِّي أَنَا السَّيِّدُ الرَّبُّ.
\par 25 وَأَنْتَ يَا ابْنَ آدَمَ, أَفَلاَ يَكُونُ فِي يَوْمٍ آخُذُ عَنْهُمْ عِزَّهُمْ, سُرُورَ فَخْرِهِمْ, شَهْوَةَ عُيُونِهِمْ وَرَفْعَةَ نَفْسِهِمْ: أَبْنَاءَهُمْ وَبَنَاتِهِمْ,
\par 26 أَنْ يَأْتِيَ إِلَيْكَ فِي ذَلِكَ الْيَوْمِ الْمُنْفَلِتُ لِيُسْمِعَ أُذُنَيْكَ.
\par 27 فِي ذَلِكَ الْيَوْمِ يَنْفَتِحُ فَمُكَ لِلْمُنْفَلِتِ وَتَتَكَلَّمُ, وَلاَ تَكُونُ مِنْ بَعْدُ أَبْكَمَ. وَتَكُونُ لَهُمْ آيَةً, فَيَعْلَمُونَ أَنِّي أَنَا الرَّبُّ].

\chapter{25}

\par 1 وَكَانَ إِلَيَّ كَلاَمُ الرَّبِّ:
\par 2 [يَا ابْنَ آدَمَ, اجْعَلْ وَجْهَكَ نَحْوَ بَنِي عَمُّونَ وَتَنَبَّأْ عَلَيْهِمْ,
\par 3 وَقُلْ لِبَنِي عَمُّونَ: اسْمَعُوا كَلاَمَ السَّيِّدِ الرَّبِّ. هَكَذَا قَالَ السَّيِّدُ الرَّبُّ: مِنْ أَجْلِ أَنَّكِ قُلْتِ: هَهْ! عَلَى مَقْدِسِي لأَنَّهُ تَنَجَّسَ, وَعَلَى أَرْضِ إِسْرَائِيلَ لأَنَّهَا خَرِبَتْ, وَعَلَى بَيْتِ يَهُوذَا لأَنَّهُمْ ذَهَبُوا إِلَى السَّبْيِ,
\par 4 فَلِذَلِكَ هَئَنَذَا أُسَلِّمُكِ لِبَنِي الْمَشْرِقِ مِلْكاً فَيُقِيمُونَ صِيَرَهُمْ فِيكِ وَيَجْعَلُونَ مَسَاكِنَهُمْ فِيكِ. هُمْ يَأْكُلُونَ غَلَّتَكِ وَهُمْ يَشْرَبُونَ لَبَنَكِ.
\par 5 وَأَجْعَلُ [رَبَّةَ» مَنَاخاً لِلإِبِلِ, وَبَنِي عَمُّونَ مَرْبِضاً لِلْغَنَمِ, فَتَعْلَمُونَ أَنِّي أَنَا الرَّبُّ.
\par 6 لأَنَّهُ هَكَذَا قَالَ السَّيِّدُ الرَّبُّ: مِنْ أَجْلِ أَنَّكَ صَفَّقْتَ بِيَدَيْكَ وَخَبَطْتَ بِرِجْلَيْكَ وَفَرِحْتَ بِكُلِّ إِهَانَتِكَ لِلْمَوْتِ عَلَى أَرْضِ إِسْرَائِيلَ.
\par 7 فَلِذَلِكَ هَئَنَذَا أَمُدُّ يَدِي عَلَيْكَ وَأُسَلِّمُكَ غَنِيمَةً لِلأُمَمِ وَأَسْتَأْصِلُكَ مِنَ الشُّعُوبِ وَأُبِيدُكَ مِنَ الأَرَاضِي. أَخْرِبُكَ فَتَعْلَمُ أَنِّي أَنَا الرَّبُّ].
\par 8 هَكَذَا قَالَ السَّيِّدُ الرَّبُّ: [مِنْ أَجْلِ أَنَّ مُوآبَ وَسَعِيرَ يَقُولُونَ: هُوَذَا بَيْتُ يَهُوذَا مِثْلُ كُلِّ الأُمَمِ.
\par 9 لِذَلِكَ هَئَنَذَا أَفْتَحُ جَانِبَ مُوآبَ مِنَ الْمُدُنِ, مِنْ مُدُنِهِ مِنْ أَقْصَاهَا, بَهَاءِ الأَرْضِ, بَيْتِ بَشِيمُوتَ وَبَعْلِ مَعُونَ وَقَرْيَتَايِمَ,
\par 10 لِبَنِي الْمَشْرِقِ عَلَى بَنِي عَمُّونَ, وَأَجْعَلُهُمْ مُلْكاً لِكَيْلاَ يُذْكَرَ بَنُو عَمُّونَ بَيْنَ الأُمَمِ.
\par 11 وَبِمُوآبَ أُجْرِي أَحْكَاماً, فَيَعْلَمُونَ أَنِّي أَنَا الرَّبُّ].
\par 12 هَكَذَا قَالَ السَّيِّدُ الرَّبُّ: [مِنْ أَجْلِ أَنَّ أَدُومَ قَدْ عَمِلَ بِالاِنْتِقَامِ عَلَى بَيْتِ يَهُوذَا وَأَسَاءَ إِسَاءَةً وَانْتَقَمَ مِنْهُ,
\par 13 لِذَلِكَ هَكَذَا قَالَ السَّيِّدُ الرَّبُّ: وَأَمُدُّ يَدِي عَلَى أَدُومَ وَأَقْطَعُ مِنْهَا الإِنْسَانَ وَالْحَيَوَانَ, وَأُصَيِّرُهَا خَرَاباً. مِنَ التَّيْمَنِ وَإِلَى دَدَانَ يَسْقُطُونَ بِالسَّيْفِ.
\par 14 وَأَجْعَلُ نَقْمَتِي فِي أَدُومَ بِيَدِ شَعْبِي إِسْرَائِيلَ, فَيَفْعَلُونَ بِأَدُومَ كَغَضَبِي وَكَسَخَطِي, فَيَعْرِفُونَ نَقْمَتِي يَقُولُ السَّيِّدُ الرَّبُّ].
\par 15 هَكَذَا قَالَ السَّيِّدُ الرَّبُّ: [مِنْ أَجْلِ أَنَّ الْفِلِسْطِينِيِّينَ قَدْ عَمِلُوا بِالاِنْتِقَامِ وَانْتَقَمُوا نَقْمَةً بِالإِهَانَةِ إِلَى الْمَوْتِ لِلْخَرَابِ مِنْ عَدَاوَةٍ أَبَدِيَّةٍ,
\par 16 فَلِذَلِكَ هَكَذَا قَالَ السَّيِّدُ الرَّبُّ: هَئَنَذَا أَمُدُّ يَدِي عَلَى الْفِلِسْطِينِيِّينَ وَأَسْتَأْصِلُ الْكَرِيتِيِّينَ وَأُهْلِكُ بَقِيَّةَ سَاحِلِ الْبَحْرِ.
\par 17 وَأُجْرِي عَلَيْهِمْ نَقْمَاتٍ عَظِيمَةً بِتَأْدِيبِ سَخَطٍ فَيَعْلَمُونَ أَنِّي أَنَا الرَّبُّ, إِذْ أَجْعَلُ نَقْمَتِي عَلَيْهِمْ].

\chapter{26}

\par 1 وَكَانَ فِي السَّنَةِ الْحَادِيَةِ عَشَرَةَ فِي أَوَّلِ الشَّهْرِ أَنَّ كَلاَمَ الرَّبِّ كَانَ إِلَيَّ:
\par 2 [يَا ابْنَ آدَمَ, مِنْ أَجْلِ أَنَّ صُورَ قَالَتْ عَلَى أُورُشَلِيمَ: هَهْ! قَدِ انْكَسَرَتْ مَصَارِيعُ الشُّعُوبِ. قَدْ تَحَوَّلَتْ إِلَيَّ. أَمْتَلِئُ إِذْ خَرِبَتْ.
\par 3 لِذَلِكَ هَكَذَا قَالَ السَّيِّدُ الرَّبُّ: هَئَنَذَا عَلَيْكِ يَا صُورُ فَأُصْعِدُ عَلَيْكِ أُمَماً كَثِيرَةً كَمَا يُعَلِّي الْبَحْرُ أَمْوَاجَهُ.
\par 4 فَيَخْرِبُونَ أَسْوَارَ صُورَ وَيَهْدِمُونَ أَبْرَاجَهَا. وَأَسْحِي تُرَابَهَا عَنْهَا وَأُصَيِّرُهَا ضِحَّ الصَّخْرِ.
\par 5 فَتَصِيرُ مَبْسَطاً لِلشِّبَاكِ فِي وَسَطِ الْبَحْرِ, لأَنِّي أَنَا تَكَلَّمْتُ يَقُولُ السَّيِّدُ الرَّبُّ. وَتَكُونُ غَنِيمَةً لِلأُمَمِ.
\par 6 وَبَنَاتُهَا اللَّوَاتِي فِي الْحَقْلِ تُقْتَلُ بِالسَّيْفِ, فَيَعْلَمُونَ أَنِّي أَنَا الرَّبُّ].
\par 7 لأَنَّهُ هَكَذَا قَالَ السَّيِّدُ الرَّبُّ: [هَئَنَذَا أَجْلِبُ عَلَى صُورَ نَبُوخَذْنَصَّرَ مَلِكَ بَابِلَ مِنَ الشِّمَالِ مَلِكَ الْمُلُوكِ بِخَيْلٍ وَمَرْكَبَاتٍ وَفُرْسَانٍ وَجَمَاعَةٍ وَشَعْبٍ كَثِيرٍ,
\par 8 فَيَقْتُلُ بَنَاتِكِ فِي الْحَقْلِ بِالسَّيْفِ, وَيَبْنِي عَلَيْكِ مَعَاقِلَ وَيَبْنِي عَلَيْكِ بُرْجاً وَيُقِيمُ عَلَيْكِ مِتْرَسَةً وَيَرْفَعُ عَلَيْكِ تُرْساً,
\par 9 وَيَجْعَلُ مَجَانِقَ عَلَى أَسْوَارِكِ وَيَهْدِمُ أَبْرَاجَكِ بِأَدَوَاتِ حَرْبِهِ.
\par 10 وَلِكَثْرَةِ خَيْلِهِ يُغَطِّيكِ غُبَارُهَا. مِنْ صَوْتِ الْفُرْسَانِ وَالْعَجَلاَتِ وَالْمَرْكَبَاتِ تَتَزَلْزَلُ أَسْوَارُكِ عِنْدَ دُخُولِهِ أَبْوَابَكِ كَمَا تُدْخَلُ مَدِينَةٌ مَثْغُورَةٌ.
\par 11 بِحَوَافِرِ خَيْلِهِ يَدُوسُ كُلَّ شَوَارِعِكِ. يَقْتُلُ شَعْبَكِ بِالسَّيْفِ فَتَسْقُطُ إِلَى الأَرْضِ أَنْصَابُ عِزِّكِ.
\par 12 وَيَنْهَبُونَ ثَرْوَتَكِ وَيَغْنَمُونَ تِجَارَتَكِ وَيَهُدُّونَ أَسْوَارَكِ وَيَهْدِمُونَ بُيُوتَكِ الْبَهِيجَةَ وَيَضَعُونَ حِجَارَتَكِ وَخَشَبَكِ وَتُرَابَكِ فِي وَسَطِ الْمِيَاهِ.
\par 13 وَأُبَطِّلُ قَوْلَ أَغَانِيكِ, وَصَوْتُ أَعْوَادِكِ لَنْ يُسْمَعَ بَعْدُ.
\par 14 وَأُصَيِّرُكِ كَضِحِّ الصَّخْرِ فَتَكُونِينَ مَبْسَطاً لِلشِّبَاكِ. لاَ تُبْنَيْنَ بَعْدُ, لأَنِّي أَنَا الرَّبُّ تَكَلَّمْتُ يَقُولُ السَّيِّدُ الرَّبُّ].
\par 15 هَكَذَا قَالَ السَّيِّدُ الرَّبُّ لِصُورَ: [أَمَا تَتَزَلْزَلُ الْجَزَائِرُ عِنْدَ صَوْتِ سُقُوطِكِ, عِنْدَ صُرَاخِ الْجَرْحَى, عِنْدَ وُقُوعِ الْقَتْلِ فِي وَسَطِكِ؟
\par 16 فَتَنْزِلُ جَمِيعُ رُؤَسَاءِ الْبَحْرِ عَنْ كَرَاسِيِّهِمْ وَيَخْلَعُونَ جُبَبَهُمْ وَيَنْزِعُونَ ثِيَابَهُمُ الْمُطَرَّزَةَ. يَلْبِسُونَ رَعْدَاتٍ وَيَجْلِسُونَ عَلَى الأَرْضِ وَيَرْتَعِدُونَ كُلَّ لَحْظَةٍ وَيَتَحَيَّرُونَ مِنْكِ.
\par 17 وَيَرْفَعُونَ عَلَيْكِ مَرْثَاةً وَيَقُولُونَ لَكِ: كَيْفَ بِدْتِ يَا مَعْمُورَةُ مِنَ الْبِحَارِ, الْمَدِينَةُ الشَّهِيرَةُ الَّتِي كَانَتْ قَوِيَّةً فِي الْبَحْرِ هِيَ وَسُكَّانُهَا الَّذِينَ أَوْقَعُوا رُعْبَهُمْ عَلَى جَمِيعِ جِيرَانِهَا؟
\par 18 اَلآنَ تَرْتَعِدُ الْجَزَائِرُ يَوْمَ سُقُوطِكِ وَتَضْطَرِبُ الْجَزَائِرُ الَّتِي فِي الْبَحْرِ لِزَوَالِكِ.
\par 19 لأَنَّهُ هَكَذَا قَالَ السَّيِّدُ الرَّبُّ: حِينَ أُصَيِّرُكِ مَدِينَةً خَرِبَةً كَالْمُدُنِ غَيْرِ الْمَسْكُونَةِ, حِينَ أُصْعِدُ عَلَيْكِ الْغَمْرَ فَتَغْشَاكِ الْمِيَاهُ الْكَثِيرَةُ,
\par 20 أُهْبِطُكِ مَعَ الْهَابِطِينَ فِي الْجُبِّ, إِلَى شَعْبِ الْقِدَمِ, وَأُجْلِسُكِ فِي أَسَافِلِ الأَرْضِ فِي الْخِرَبِ الأَبَدِيَّةِ مَعَ الْهَابِطِينَ فِي الْجُبِّ, لِتَكُونِي غَيْرَ مَسْكُونَةٍ وَأَجْعَلُ فَخْراً فِي أَرْضِ الأَحْيَاءِ.
\par 21 أُصَيِّرُكِ أَهْوَالاً وَلاَ تَكُونِينَ, وَتُطْلَبِينَ فَلاَ تُوجَدِينَ بَعْدُ إِلَى الأَبَدِ يَقُولُ السَّيِّدُ الرَّبُّ].

\chapter{27}

\par 1 وَكَانَ إِلَيَّ كَلاَمُ الرَّبِّ:
\par 2 [وَأَنْتَ يَا ابْنَ آدَمَ فَارْفَعْ مَرْثَاةً عَلَى صُورَ,
\par 3 وَقُلْ لِصُورَ: أَيَّتُهَا السَّاكِنَةُ عِنْدَ مَدَاخِلِ الْبَحْرِ, تَاجِرَةُ الشُّعُوبِ إِلَى جَزَائِرَ كَثِيرَةٍ, هَكَذَا قَالَ السَّيِّدُ الرَّبُّ: يَا صُورُ, أَنْتِ قُلْتِ: أَنَا كَامِلَةُ الْجَمَالِ.
\par 4 تُخُومُكِ فِي قَلْبِ الْبُحُورِ. بَنَّاؤُوكِ تَمَّمُوا جَمَالَكِ.
\par 5 عَمِلُوا كُلَّ أَلْوَاحِكِ مِنْ سَرْوِ سَنِيرَ. أَخَذُوا أَرْزاً مِنْ لُبْنَانَ لِيَصْنَعُوهُ لَكِ سَوَارِيَ.
\par 6 صَنَعُوا مِنْ بَلُّوطِ بَاشَانَ مَجَاذِيفَكِ. صَنَعُوا مَقَاعِدَكِ مِنْ عَاجٍ مُطَعَّمٍ فِي الْبَقْسِ مِنْ جَزَائِرِ كِتِّيمَ.
\par 7 كَتَّانٌ مُطَرَّزٌ مِنْ مِصْرَ هُوَ شِرَاعُكِ لِيَكُونَ لَكِ رَايَةً. الأَسْمَانْجُونِيُّ وَالأُرْجُوانُ مِنْ جَزَائِرِ أَلِيشَةَ كَانَا غِطَاءَكِ.
\par 8 أَهْلُ صَيْدُونَ وَإِرْوَادَ كَانُوا مَلاَّحِيكِ. حُكَمَاؤُكِ يَا صُورُ الَّذِينَ كَانُوا فِيكِ هُمْ رَبَابِينُكِ.
\par 9 شُيُوخُ جُبَيْلَ وَحُكَمَاؤُهَا كَانُوا فِيكِ قَلاَّفُوكِ. جَمِيعُ سُفُنِ الْبَحْرِ وَمَلاَّحُوهَا كَانُوا فِيكِ لِيُتَاجِرُوا بِتِجَارَتِكِ.
\par 10 فَارِسُ وَلُودُ وَفُوطُ كَانُوا فِي جَيْشِكِ, رِجَالَ حَرْبِكِ. عَلَّقُوا فِيكِ تُرْساً وَخُوذَةً. هُمْ صَيَّرُوا بَهَاءَكِ.
\par 11 بَنُو إِرْوَادَ مَعَ جَيْشِكِ عَلَى الأَسْوَارِ مِنْ حَوْلِكِ, وَالأَبْطَالِ كَانُوا فِي بُرُوجِكِ. عَلَّقُوا أَتْرَاسَهُمْ عَلَى أَسْوَارِكِ مِنْ حَوْلِكِ. هُمْ تَمَّمُوا جَمَالَكِ.
\par 12 تَرْشِيشُ تَاجِرَتُكِ بِكَثْرَةِ كُلِّ غِنًى. بِالْفِضَّةِ وَالْحَدِيدِ وَالْقَصْدِيرِ وَالرَّصَاصِ أَقَامُوا أَسْوَاقَكِ.
\par 13 يَاوَانُ وَتُوبَالُ وَمَاشِكُ هُمْ تُجَّارُكِ. بِنُفُوسِ النَّاسِ وَبِآنِيَةِ النُّحَاسِ أَقَامُوا تِجَارَتَكِ.
\par 14 وَمِنْ بَيْتِ تُوجَرْمَةَ بِالْخَيْلِ وَالْفُرْسَانِ وَالْبِغَالِ أَقَامُوا أَسْوَاقَكِ.
\par 15 بَنُو دَدَانَ تُجَّارُكِ. جَزَائِرُ كَثِيرَةٌ تُجَّارُ يَدِكِ. أَدُّوا هَدِيَّتَكِ قُرُوناً مِنَ الْعَاجِ وَالآبْنُوسِ.
\par 16 أَرَامُ تَاجِرَتُكِ بِكَثْرَةِ صَنَائِعِكِ. تَاجَرُوا فِي أَسْوَاقِكِ بِالْبَهْرَمَانِ وَالأُرْجُوانِ وَالْمُطَرَّزِ وَالْبُوصِ وَالْمُرْجَانِ وَالْيَاقُوتِ.
\par 17 يَهُوذَا وَأَرْضُ إِسْرَائِيلَ هُمْ تُجَّارُكِ. تَاجَرُوا فِي سُوقِكِ بِحِنْطَةِ مِنِّيتَ وَحَلاَوَى وَعَسَلٍ وَزَيْتٍ وَبَلَسَانٍ.
\par 18 دِمَشْقُ تَاجِرَتُكِ بِكَثْرَةِ صَنَائِعِكِ وَكَثْرَةِ كُلِّ غِنًى. بِخَمْرِ حَلْبُونَ وَالصُّوفِ الأَبْيَضِ.
\par 19 وَدَانُ وَيَاوَانُ قَدَّمُوا غَزْلاً فِي أَسْوَاقِكِ. حَدِيدٌ مَشْغُولٌ وَسَلِيخَةٌ وَقَصَبُ الذَّرِيرَةِ كَانَتْ فِي سُوقِكِ.
\par 20 دَدَانُ تَاجِرَتُكِ بِطَنَافِسَ لِلرُّكُوبِ.
\par 21 اَلْعَرَبُ وَكُلُّ رُؤَسَاءِ قِيدَارَ هُمْ تُجَّارُ يَدِكِ بِالْخِرْفَانِ وَالْكِبَاشِ وَالأَعْتِدَةِ. فِي هَذِهِ كَانُوا تُجَّارَكِ.
\par 22 تُجَّارُ شَبَا وَرَعْمَةَ هُمْ تُجَّارُكِ. بِأَفْخَرِ كُلِّ أَنْوَاعِ الطِّيبِ وَبِكُلِّ حَجَرٍ كَرِيمٍ وَالذَّهَبِ أَقَامُوا أَسْوَاقَكِ.
\par 23 حُرَّانُ وَكِنَّةُ وَعَدَنُ تُجَّارُ شَبَا وَأَشُّورَ وَكِلْمَدَ تُجَّارُكِ.
\par 24 هَؤُلاَءِ تُجَّارُكِ بِنَفَائِسَ بِأَرْدِيَةٍ أَسْمَانْجُونِيَّةٍ وَمُطَرَّزَةٍ وَأَصْوِنَةٍ مُبْرَمٍ مَعْكُومَةٍ بِالْحِبَالِ مَصْنُوعَةٍ مِنَ الأَرْزِ بَيْنَ بَضَائِعِكِ.
\par 25 [سُفُنُ تَرْشِيشَ قَوَافِلُكِ لِتِجَارَتِكِ, فَامْتَلَأْتِ وَتَمَجَّدْتِ جِدّاً فِي قَلْبِ الْبِحَارِ.
\par 26 مَلاَّحُوكِ قَدْ أَتُوا بِكِ إِلَى مِيَاهٍ كَثِيرَةٍ. كَسَرَتْكِ الرِّيحُ الشَّرْقِيَّةُ فِي قَلْبِ الْبِحَارِ.
\par 27 ثَرْوَتُكِ وَأَسْوَاقُكِ وَبِضَاعَتُكِ وَمَلاَّحُوكِ وَرَبَابِينُكِ وَقَلاَّفُوكِ وَالْمُتَاجِرُونَ بِمَتْجَرِكِ وَجَمِيعُ رِجَالِ حَرْبِكِ الَّذِينَ فِيكِ وَكُلُّ جَمْعِكِ الَّذِي فِي وَسَطِكِ يَسْقُطُونَ فِي قَلْبِ الْبِحَارِ فِي يَوْمِ سُقُوطِكِ.
\par 28 مِنْ صَوْتِ صُرَاخِ رَبَابِينِكِ تَتَزَلْزَلُ الْمَسَارِحُ.
\par 29 وَكُلُّ مُمْسِكِي الْمِجْدَافِ وَالْمَلاَّحُونَ وَكُلُّ رَبَابِينِ الْبَحْرِ يَنْزِلُونَ مِنْ سُفُنِهِمْ وَيَقِفُونَ عَلَى الْبَرِّ
\par 30 وَيُسْمِعُونَ صَوْتَهُمْ عَلَيْكِ وَيَصْرُخُونَ بِمَرَارَةٍ وَيُذَرُّونَ تُرَاباً فَوْقَ رُؤُوسِهِمْ وَيَتَمَرَّغُونَ فِي الرَّمَادِ.
\par 31 وَيَجْعَلُونَ فِي أَنْفُسِهِمْ قَرْعَةً عَلَيْكِ, وَيَتَنَطَّقُونَ بِالْمُسُوحِ وَيَبْكُونَ عَلَيْكِ بِمَرَارَةِ نَفْسٍ نَحِيباً مُرّاً.
\par 32 وَفِي نَوْحِهِمْ يَرْفَعُونَ عَلَيْكِ مَنَاحَةً وَيَرْثُونَكِ, وَيَقُولُونَ: أَيَّةُ مَدِينَةٍ كَصُورَ كَالْمُسْكَتَةِ فِي قَلْبِ الْبَحْرِ؟
\par 33 عِنْدَ خُرُوجِ بَضَائِعِكِ مِنَ الْبِحَارِ أَشْبَعْتِ شُعُوباً كَثِيرِينَ. بِكَثْرَةِ ثَرْوَتِكِ وَتِجَارَتِكِ أَغْنَيْتِ مُلُوكَ الأَرْضِ.
\par 34 حِينَ انْكِسَارِكِ مِنَ الْبِحَارِ فِي أَعْمَاقِ الْمِيَاهِ سَقَطَ مَتْجَرُكِ وَكُلُّ جَمْعِكِ.
\par 35 كُلُّ سُكَّانِ الْجَزَائِرِ يَتَحَيَّرُونَ عَلَيْكِ, وَمُلُوكِهِنَّ يَقْشَعِرُّونَ اقْشِعْرَاراً. يَضْطَرِبُونَ فِي الْوُجُوهِ.
\par 36 اَلتُّجَّارُ بَيْنَ الشُّعُوبِ يَصْفِرُونَ عَلَيْكِ فَتَكُونِينَ أَهْوَالاً, وَلاَ تَكُونِينَ بَعْدُ إِلَى الأَبَدِ].

\chapter{28}

\par 1 وَكَانَ إِلَيَّ كَلاَمُ الرَّبِّ:
\par 2 [يَا ابْنَ آدَمَ, قُلْ لِرَئِيسِ صُورَ. هَكَذَا قَالَ السَّيِّدُ الرَّبُّ: مِنْ أَجْلِ أَنَّهُ قَدِ ارْتَفَعَ قَلْبُكَ وَقُلْتَ: أَنَا إِلَهٌ. فِي مَجْلِسِ الآلِهَةِ أَجْلِسُ فِي قَلْبِ الْبِحَارِ. وَأَنْتَ إِنْسَانٌ لاَ إِلَهٌ, وَإِنْ جَعَلْتَ قَلْبَكَ كَقَلْبِ الآلِهَةِ.
\par 3 هَا أَنْتَ أَحْكَمُ مِنْ دَانِيآلَ! سِرٌّ مَا لاَ يَخْفَى عَلَيْكَ!
\par 4 وَبِحِكْمَتِكَ وَبِفَهْمِكَ حَصَّلْتَ لِنَفْسِكَ ثَرْوَةً, وَحَصَّلْتَ الذَّهَبَ وَالْفِضَّةَ فِي خَزَائِنِكَ!
\par 5 بِكَثْرَةِ حِكْمَتِكَ فِي تِجَارَتِكَ كَثَّرْتَ ثَرْوَتَكَ, فَارْتَفَعَ قَلْبُكَ بِسَبَبِ غِنَاكَ!
\par 6 فَلِذَلِكَ هَكَذَا قَالَ السَّيِّدُ الرَّبُّ: مِنْ أَجْلِ أَنَّكَ جَعَلْتَ قَلْبَكَ كَقَلْبِ الآلِهَةِ,
\par 7 لِذَلِكَ هَئَنَذَا أَجْلِبُ عَلَيْكَ غُرَبَاءَ, عُتَاةَ الأُمَمِ, فَيُجَرِّدُونَ سُيُوفَهُمْ عَلَى بَهْجَةِ حِكْمَتِكَ وَيُدَنِّسُونَ جَمَالَكَ.
\par 8 يُنَزِّلُونَكَ إِلَى الْحُفْرَةِ فَتَمُوتُ مَوْتَ الْقَتْلَى فِي قَلْبِ الْبِحَارِ.
\par 9 هَلْ تَقُولُ قَوْلاً أَمَامَ قَاتِلِكَ: أَنَا إِلَهٌ. وَأَنْتَ إِنْسَانٌ لاَ إِلَهٌ فِي يَدِ طَاعِنِكَ؟
\par 10 مَوْتَ الْغُلْفِ تَمُوتُ بِيَدِ الْغُرَبَاءِ, لأَنِّي أَنَا تَكَلَّمْتُ يَقُولُ السَّيِّدُ الرَّبُّ].
\par 11 وَكَانَ إِلَيَّ كَلاَمُ الرَّبِّ:
\par 12 [يَا ابْنَ آدَمَ, ارْفَعْ مَرْثَاةً عَلَى مَلِكِ صُورَ وَقُلْ لَهُ: هَكَذَا قَالَ السَّيِّدُ الرَّبُّ: أَنْتَ خَاتِمُ الْكَمَالِ, مَلآنٌ حِكْمَةً وَكَامِلُ الْجَمَالِ.
\par 13 كُنْتَ فِي عَدْنٍ جَنَّةِ اللَّهِ. كُلُّ حَجَرٍ كَرِيمٍ سِتَارَتُكَ, عَقِيقٌ أَحْمَرُ وَيَاقُوتٌ أَصْفَرُ وَعَقِيقٌ أَبْيَضُ وَزَبَرْجَدٌ وَجَزْعٌ وَيَشْبٌ وَيَاقُوتٌ أَزْرَقُ وَبَهْرَمَانُ وَزُمُرُّدٌ وَذَهَبٌ. أَنْشَأُوا فِيكَ صَنْعَةَ صِيغَةِ الفُصُوصِ وَتَرْصِيعِهَا يَوْمَ خُلِقْتَ.
\par 14 أَنْتَ الْكَرُوبُ الْمُنْبَسِطُ الْمُظَلِّلُ. وَأَقَمْتُكَ. عَلَى جَبَلِ اللَّهِ الْمُقَدَّسِ كُنْتَ. بَيْنَ حِجَارَةِ النَّارِ تَمَشَّيْتَ.
\par 15 أَنْتَ كَامِلٌ فِي طُرُقِكَ مِنْ يَوْمَ خُلِقْتَ حَتَّى وُجِدَ فِيكَ إِثْمٌ.
\par 16 بِكَثْرَةِ تِجَارَتِكَ مَلأُوا جَوْفَكَ ظُلْماً فَأَخْطَأْتَ. فَأَطْرَحُكَ مِنْ جَبَلِ اللَّهِ وَأُبِيدُكَ أَيُّهَا الْكَرُوبُ الْمُظَلِّلُ مِنْ بَيْنِ حِجَارَةِ النَّارِ.
\par 17 قَدِ ارْتَفَعَ قَلْبُكَ لِبَهْجَتِكَ. أَفْسَدْتَ حِكْمَتَكَ لأَجْلِ بَهَائِكَ. سَأَطْرَحُكَ إِلَى الأَرْضِ وَأَجْعَلُكَ أَمَامَ الْمُلُوكِ لِيَنْظُرُوا إِلَيْكَ.
\par 18 قَدْ نَجَّسْتَ مَقَادِسَكَ بِكَثْرَةِ آثَامِكَ بِظُلْمِ تِجَارَتِكَ, فَأُخْرِجُ نَاراً مِنْ وَسَطِكَ فَتَأْكُلُكَ, وَأُصَيِّرُكَ رَمَاداً عَلَى الأَرْضِ أَمَامَ عَيْنَيْ كُلِّ مَنْ يَرَاكَ.
\par 19 فَيَتَحَيَّرُ مِنْكَ جَمِيعُ الَّذِينَ يَعْرِفُونَكَ بَيْنَ الشُّعُوبِ وَتَكُونُ أَهْوَالاً وَلاَ تُوجَدُ بَعْدُ إِلَى الأَبَدِ].
\par 20 وَكَانَ إِلَيَّ كَلاَمُ الرَّبِّ:
\par 21 [يَا ابْنَ آدَمَ, اجْعَلْ وَجْهَكَ نَحْوَ صَيْدُونَ وَتَنَبَّأْ عَلَيْهَا
\par 22 وَقُلْ: هَكَذَا قَالَ السَّيِّدُ الرَّبُّ: هَأَنَذَا عَلَيْكِ يَا صَيْدُونُ وَسَأَتَمَجَّدُ فِي وَسَطِكِ, فَيَعْلَمُونَ أَنِّي أَنَا الرَّبُّ حِينَ أُجْرِي فِيهَا أَحْكَاماً وَأَتَقَدَّسُ فِيهَا.
\par 23 وَأُرْسِلُ عَلَيْهَا وَبَأً وَدَماً إِلَى أَزِقَّتِهَا وَيُسْقَطُ الْجَرْحَى فِي وَسَطِهَا بِالسَّيْفِ الَّذِي عَلَيْهَا مِنْ كُلِّ جَانِبٍ, فَيَعْلَمُونَ أَنِّي أَنَا الرَّبُّ.
\par 24 [فَلاَ يَكُونُ بَعْدُ لِبَيْتِ إِسْرَائِيلَ سُلاَّءٌ مُمَرِّرٌ وَلاَ شَوْكَةٌ مُوجِعَةٌ مِنْ كُلِّ الَّذِينَ حَوْلَهُمُ الَّذِينَ يُبْغِضُونَهُمْ, فَيَعْلَمُونَ أَنِّي أَنَا السَّيِّدُ الرَّبُّ.
\par 25 عِنْدَمَا أَجْمَعُ بَيْتَ إِسْرَائِيلَ مِنَ الشُّعُوبِ الَّذِينَ تَفَرَّقُوا بَيْنَهُمْ, وَأَتَقَدَّسُ فِيهِمْ أَمَامَ عُيُونِ الأُمَمِ, يَسْكُنُونَ فِي أَرْضِهِمِ الَّتِي أَعْطَيْتُهَا لِعَبْدِي يَعْقُوبَ,
\par 26 وَيَسْكُنُونَ فِيهَا آمِنِينَ وَيَبْنُونَ بُيُوتاً وَيَغْرِسُونَ كُرُوماً وَيَسْكُنُونَ فِي أَمْنٍ عِنْدَمَا أُجْرِي أَحْكَاماً عَلَى جَمِيعِ مُبْغِضِيهِمْ مِنْ حَوْلِهِمْ, فَيَعْلَمُونَ أَنِّي أَنَا الرَّبُّ إِلَهُهُمْ].

\chapter{29}

\par 1 فِي السَّنَةِ الْعَاشِرَةِ فِي الثَّانِي عَشَرَ مِنَ الشَّهْرِ الْعَاشِرِ كَانَ إِلَيَّ كَلاَمُ الرَّبِّ:
\par 2 [يَا ابْنَ آدَمَ, اجْعَلْ وَجْهَكَ نَحْوَ فِرْعَوْنَ مَلِكِ مِصْرَ وَتَنَبَّأْ عَلَيْهِ وَعَلَى مِصْرَ كُلِّهَا.
\par 3 تَكَلَّمْ وَقُلْ: هَكَذَا قَالَ السَّيِّدُ الرَّبُّ: هَئَنَذَا عَلَيْكَ يَا فِرْعَوْنُ مَلِكُ مِصْرَ, التِّمْسَاحُ الْكَبِيرُ الرَّابِضُ فِي وَسَطِ أَنْهَارِهِ, الَّذِي قَالَ: نَهْرِي لِي وَأَنَا عَمِلْتُهُ لِنَفْسِي.
\par 4 فَأَجْعَلُ خَزَائِمَ فِي فَكَّيْكَ وَأُلْزِقُ سَمَكَ أَنْهَارِكَ بِحَرْشَفِكَ, وَأُطْلِعُكَ مِنْ وَسَطِ أَنْهَارِكَ وَكُلُّ سَمَكِ أَنْهَارِكَ مُلْزَقٌ بِحَرْشَفِكَ.
\par 5 وَأَتْرُكُكَ فِي الْبَرِّيَّةِ أَنْتَ وَجَمِيعَ سَمَكِ أَنْهَارِكَ. عَلَى وَجْهِ الْحَقْلِ تَسْقُطُ فَلاَ تُجْمَعُ وَلاَ تُلَمُّ. بَذَلْتُكَ طَعَاماً لِوُحُوشِ الْبَرِّ وَلِطُيُورِ السَّمَاءِ.
\par 6 وَيَعْلَمُ كُلُّ سُكَّانِ مِصْرَ أَنِّي أَنَا الرَّبُّ مِنْ أَجْلِ كَوْنِهِمْ عُكَّازَ قَصَبٍ لِبَيْتِ إِسْرَائِيلَ.
\par 7 عِنْدَ مَسْكِهِمْ بِكَ بِالْكَفِّ انْكَسَرْتَ وَمَزَّقْتَ لَهُمْ كُلَّ كَتِفٍ, وَلَمَّا تَوَكَّأُوا عَلَيْكَ انْكَسَرْتَ وَقَلْقَلْتَ كُلَّ مُتُونِهِمْ].
\par 8 لِذَلِكَ هَكَذَا قَالَ السَّيِّدُ الرَّبُّ: [هَئَنَذَا أَجْلِبُ عَلَيْكَ سَيْفاً, وَأَسْتَأْصِلُ مِنْكَ الإِنْسَانَ وَالْحَيَوَانَ.
\par 9 وَتَكُونُ أَرْضُ مِصْرَ مُقْفِرَةً وَخَرِبَةً, فَيَعْلَمُونَ أَنِّي أَنَا الرَّبُّ لأَنَّهُ قَالَ: النَّهْرُ لِي وَأَنَا عَمِلْتُهُ.
\par 10 لِذَلِكَ هَئَنَذَا عَلَيْكَ وَعَلَى أَنْهَارِكَ, وَأَجْعَلُ أَرْضَ مِصْرَ خِرَباً خَرِبَةً مُقْفِرَةً مِنْ مَجْدَلَ إِلَى أَسْوَانَ إِلَى تُخُمِ كُوشَ.
\par 11 لاَ تَمُرُّ فِيهَا رِجْلُ إِنْسَانٍ, وَلاَ تَمُرُّ فِيهَا رِجْلُ بَهِيمَةٍ, وَلاَ تُسْكَنُ أَرْبَعِينَ سَنَةً.
\par 12 وَأَجْعَلُ أَرْضَ مِصْرَ مُقْفِرَةً فِي وَسَطِ الأَرَاضِي الْمُقْفِرَةِ, وَمُدُنَهَا فِي وَسَطِ الْمُدُنِ الْخَرِبَةِ تَكُونُ مُقْفِرَةً أَرْبَعِينَ سَنَةً. وَأُشَتِّتُ الْمِصْرِيِّينَ بَيْنَ الأُمَمِ وَأُبَدِّدُهُمْ فِي الأَرَاضِي.
\par 13 لأَنَّهُ هَكَذَا قَالَ السَّيِّدُ الرَّبُّ: عِنْدَ نَهَايَةِ أَرْبَعِينَ سَنَةً أَجْمَعُ الْمِصْرِيِّينَ مِنَ الشُّعُوبِ الَّذِينَ تَشَتَّتُوا بَيْنَهُمْ
\par 14 وَأَرُدُّ سَبْيَ مِصْرَ, وَأُرْجِعُهُمْ إِلَى أَرْضِ فَتْرُوسَ إِلَى أَرْضِ مِيلاَدِهِمْ, وَيَكُونُونَ هُنَاكَ مَمْلَكَةً حَقِيرَةً.
\par 15 تَكُونُ أَحْقَرَ الْمَمَالِكِ فَلاَ تَرْتَفِعُ بَعْدُ عَلَى الأُمَمِ, وَأُقَلِّلُهُمْ لِكَيْلاَ يَتَسَلَّطُوا عَلَى الأُمَمِ.
\par 16 فَلاَ تَكُونُ بَعْدُ مُعْتَمَداً لِبَيْتِ إِسْرَائِيلَ, مُذَكِّرَةَ الإِثْمِ بِانْصِرَافِهِمْ وَرَاءَهُمْ, وَيَعْلَمُونَ أَنِّي أَنَا السَّيِّدُ الرَّبُّ].
\par 17 وَكَانَ فِي السَّنَةِ السَّابِعَةِ وَالْعِشْرِينَ فِي الشَّهْرِ الأَوَّلِ فِي أَوَّلِ الشَّهْرِ أَنَّ كَلاَمَ الرَّبِّ كَانَ إِلَيَّ:
\par 18 [يَا ابْنَ آدَمَ, إِنَّ نَبُوخَذْنَصَّرَ مَلِكَ بَابِلَ اسْتَخْدَمَ جَيْشَهُ خِدْمَةً شَدِيدَةً عَلَى صُورَ. كُلُّ رَأْسٍ قَرِعَ, وَكُلُّ كَتِفٍ تَجَرَّدَتْ, وَلَمْ تَكُنْ لَهُ وَلاَ لِجَيْشِهِ أُجْرَةٌ مِنْ صُورَ لأَجْلِ خِدْمَتِهِ الَّتِي خَدَمَ بِهَا عَلَيْهَا.
\par 19 لِذَلِكَ هَكَذَا قَالَ السَّيِّدُ الرَّبُّ: هَئَنَذَا أَبْذُلُ أَرْضَ مِصْرَ لِنَبُوخَذْنَصَّرَ مَلِكِ بَابِلَ فَيَأْخُذُ ثَرْوَتَهَا وَيَغْنَمُ غَنِيمَتَهَا وَيَنْهَبُ نَهْبَهَا فَتَكُونُ أُجْرَةً لِجَيْشِهِ.
\par 20 قَدْ أَعْطَيْتُهُ أَرْضَ مِصْرَ لأَجْلِ شُغْلِهِ الَّذِي خَدَمَ بِهِ لأَنَّهُمْ عَمِلُوا لأَجْلِي يَقُولُ السَّيِّدُ الرَّبُّ.
\par 21 فِي ذَلِكَ الْيَوْمِ أُنْبِتُ قَرْناً لِبَيْتِ إِسْرَائِيلَ. وَأَجْعَلُ لَكَ فَتْحَ الْفَمِ فِي وَسَطِهِمْ فَيَعْلَمُونَ أَنِّي أَنَا الرَّبُّ].

\chapter{30}

\par 1 وَكَانَ إِلَيَّ كَلاَمُ الرَّبِّ:
\par 2 [يَا ابْنَ آدَمَ تَنَبَّأْ وَقُلْ: هَكَذَا قَالَ السَّيِّدُ الرَّبُّ: وَلْوِلُوا: يَا لَلْيَوْمِ!
\par 3 لأَنَّ الْيَوْمَ قَرِيبٌ. وَيَوْمٌ لِلرَّبِّ قَرِيبٌ يَوْمُ غَيْمٍ. يَكُونُ وَقْتاً لِلأُمَمِ.
\par 4 وَيَأْتِي سَيْفٌ عَلَى مِصْرَ, وَيَكُونُ فِي كُوشَ خَوْفٌ شَدِيدٌ عِنْدَ سُقُوطِ الْقَتْلَى فِي مِصْرَ, وَيَأْخُذُونَ ثَرْوَتَهَا وَتُهْدَمُ أُسُسُهَا.
\par 5 يَسْقُطُ مَعَهُمْ بِالسَّيْفِ كُوشُ وَفُوطُ وَلُودُ وَكُلُّ اللَّفِيفِ, وَكُوبُ وَبَنُو أَرْضِ الْعَهْدِ.
\par 6 هَكَذَا قَالَ الرَّبُّ: وَيَسْقُطُ عَاضِدُو مِصْرَ وَتَنْحَطُّ كِبْرِيَاءُ عِزَّتِهَا. مِنْ مَجْدَلَ إِلَى أَسْوَانَ يَسْقُطُونَ فِيهَا بِالسَّيْفِ يَقُولُ السَّيِّدُ الرَّبُّ.
\par 7 فَتُقْفِرُ فِي وَسَطِ الأَرَاضِي الْمُقْفِرَةِ, وَتَكُونُ مُدُنُهَا فِي وَسَطِ الْمُدُنِ الْخَرِبَةِ.
\par 8 فَيَعْلَمُونَ أَنِّي أَنَا الرَّبُّ عِنْدَ إِضْرَامِي نَاراً فِي مِصْرَ وَيُكْسَرُ جَمِيعُ أَعْوَانِهَا.
\par 9 فِي ذَلِكَ الْيَوْمِ يَخْرُجُ مِنْ قِبَلِي رُسُلٌ فِي سُفُنٍ لِتَخْوِيفِ كُوشَ الْمُطْمَئِنَّةِ, فَيَأْتِي عَلَيْهِمْ خَوْفٌ عَظِيمٌ كَمَا فِي يَوْمِ مِصْرَ. لأَنَّهُ هُوَذَا يَأْتِي].
\par 10 هَكَذَا قَالَ السَّيِّدُ الرَّبُّ: [إِنِّي أُبِيدُ ثَرْوَةَ مِصْرَ بِيَدِ نَبُوخَذْنَصَّرَ مَلِكِ بَابِلَ.
\par 11 هُوَ وَشَعْبُهُ مَعَهُ عُتَاةُ الأُمَمِ يُؤْتَى بِهِمْ لِخَرَابِ الأَرْضِ, فَيُجَرِّدُونَ سُيُوفَهُمْ عَلَى مِصْرَ وَيَمْلأُونَ الأَرْضَ مِنَ الْقَتْلَى.
\par 12 وَأَجْعَلُ الأَنْهَارَ يَابِسَةً وَأَبِيعُ الأَرْضَ لِيَدِ الأَشْرَارِ وَأُخْرِبُ الأَرْضَ وَمِلأَهَا بِيَدِ الْغُرَبَاءِ. أَنَا الرَّبَّ تَكَلَّمْتُ.
\par 13 هَكَذَا قَالَ السَّيِّدُ الرَّبُّ. وَأُبِيدُ الأَصْنَامَ وَأُبَطِّلُ الأَوْثَانَ مِنْ نُوفَ. وَلاَ يَكُونُ بَعْدُ رَئِيسٌ مِنْ أَرْضِ مِصْرَ, وَأُلْقِي الرُّعْبَ فِي أَرْضِ مِصْرَ.
\par 14 وَأُخْرِبُ فَتْرُوسَ وَأُضْرِمُ نَاراً في صُوعَنَ وَأُجْرِي أَحْكَاماً فِي نُوَ.
\par 15 وَأَسْكُبُ غَضَبِي عَلَى سِينَ حِصْنِ مِصْرَ, وَأَسْتَأْصِلُ جُمْهُورَ نُوَ.
\par 16 وَأُضْرِمُ نَاراً فِي مِصْرَ. سِينُ تَتَوَجَّعُ تَوَجُّعاً, وَنُوَ تَكُونُ لِلتَّمْزِيقِ وَلِنُوفَ ضِيقَاتٌ كُلَّ يَوْمٍ.
\par 17 شُبَّانُ آوَنَ وَفِيبِسْتَةَ يَسْقُطُونَ بِالسَّيْفِ وَهُمَا تَذْهَبَانِ إِلَى السَّبْيِ.
\par 18 وَيُظْلِمُ النَّهَارُ فِي تَحْفَنِيسَ عِنْدَ كَسْرِي أَنْيَارَ مِصْرَ هُنَاكَ. وَتَبْطُلُ فِيهَا كِبْرِيَاءُ عِزِّهَا. أَمَّا هِيَ فَتَغْشَاهَا سَحَابَةٌ وَتَذْهَبُ بَنَاتُهَا إِلَى السَّبْيِ.
\par 19 فَأُجْرِي أَحْكَاماً فِي مِصْرَ, فَيَعْلَمُونَ أَنِّي أَنَا الرَّبُّ].
\par 20 وَكَانَ فِي السَّنَةِ الْحَادِيَةِ عَشَرَةَ فِي الشَّهْرِ الأَوَّلِ فِي السَّابِعِ مِنَ الشَّهْرِ أَنَّ كَلاَمَ الرَّبِّ صَارَ إِلَيَّ:
\par 21 [يَا ابْنَ آدَمَ, إِنِّي كَسَرْتُ ذِرَاعَ فِرْعَوْنَ مَلِكِ مِصْرَ, وَهَا هِيَ لَنْ تُجْبَرُ بِوَضْعِ رَفَائِدَ وَلاَ بِوَضْعِ عِصَابَةٍ لِتُجْبَرَ فَتُمْسِكَ السَّيْفَ.
\par 22 لِذَلِكَ هَكَذَا قَالَ السَّيِّدُ الرَّبُّ: هَئَنَذَا عَلَى فِرْعَوْنَ مَلِكِ مِصْرَ فَأُكَسِّرُ ذِرَاعَيْهِ الْقَوِيَّةَ وَالْمَكْسُورَةَ, وَأُسْقِطُ السَّيْفَ مِنْ يَدِهِ.
\par 23 وَأُشَتِّتُ الْمِصْرِيِّينَ بَيْنَ الأُمَمِ وَأُذَرِّيهِمْ فِي الأَرَاضِي.
\par 24 وَأُشَدِّدُ ذِرَاعَيْ مَلِكِ بَابِلَ وَأَجْعَلُ سَيْفِي فِي يَدِهِ. وَأُكَسِّرُ ذِرَاعَيْ فِرْعَوْنَ فَيَئِنُّ قُدَّامَهُ أَنِينَ الْجَرِيحِ.
\par 25 وَأُشَدِّدُ ذِرَاعَيْ مَلِكِ بَابِلَ. أَمَّا ذِرَاعَا فِرْعَوْنَ فَتَسْقُطَانِ, فَيَعْلَمُونَ أَنِّي أَنَا الرَّبُّ حِينَ أَجْعَلُ سَيْفِي فِي يَدِ مَلِكِ بَابِلَ فَيَمُدُّهُ عَلَى أَرْضِ مِصْرَ.
\par 26 وَأُشَتِّتُ الْمِصْرِيِّينَ بَيْنَ الأُمَمِ وَأُذَرِّيهِمْ فِي الأَرَاضِي, فَيَعْلَمُونَ أَنِّي أَنَا الرَّبُّ».

\chapter{31}

\par 1 وَكَانَ فِي السَّنَةِ الْحَادِيَةِ عَشَرَةَ فِي الشَّهْرِ الثَّالِثِ فِي أَوَّلِ الشَّهْرِ أَنَّ كَلاَمَ الرَّبِّ كَانَ إِلَيَّ:
\par 2 [يَا ابْنَ آدَمَ, قُلْ لِفِرْعَوْنَ مَلِكِ مِصْرَ وَجُمْهُورِهِ: مَنْ أَشْبَهْتَ فِي عَظَمَتِكَ؟
\par 3 هُوَذَا أَعْلَى الأَرْزِ فِي لُبْنَانَ جَمِيلُ الأَغْصَانِ وَأَغْبَى الظِّلِّ وَقَامَتُهُ طَوِيلَةٌ وَكَانَ فَرْعُهُ بَيْنَ الْغُيُومِ.
\par 4 قَدْ عَظَّمَتْهُ الْمِيَاهُ وَرَفَعَهُ الْغَمْرُ. أَنْهَارُهُ جَرَتْ مِنْ حَوْلِ مَغْرِسِهِ وَأَرْسَلَتْ جَدَاوِلَهَا إِلَى كُلِّ أَشْجَارِ الْحَقْلِ.
\par 5 فَلِذَلِكَ ارْتَفَعَتْ قَامَتُهُ عَلَى جَمِيعِ أَشْجَارِ الْحَقْلِ, وَكَثُرَتْ أَغْصَانُهُ وَطَالَتْ فُرُوعُهُ لِكَثْرَةِ الْمِيَاهِ إِذْ نَبَتَ.
\par 6 وَعَشَّشَتْ فِي أَغْصَانِهِ كُلُّ طُيُورِ السَّمَاءِ, وَتَحْتَ فُرُوعِهِ وَلَدَتْ كُلُّ حَيَوَانِ الْبَرِّ, وَسَكَنَ تَحْتَ ظِلِّهِ كُلُّ الأُمَمِ الْعَظِيمَةِ.
\par 7 فَكَانَ جَمِيلاً فِي عَظَمَتِهِ وَفِي طُولِ قُضْبَانِهِ, لأَنَّ أَصْلَهُ كَانَ عَلَى مِيَاهٍ كَثِيرَةٍ.
\par 8 اَلأَرْزُ فِي جَنَّةِ اللَّهِ لَمْ يَفُقْهُ, السَّرْوُ لَمْ يُشْبِهْ أَغْصَانَهُ, وَالدُّلْبُ لَمْ يَكُنْ مِثْلَ فُرُوعِهِ. كُلُّ الأَشْجَارِ فِي جَنَّةِ اللَّهِ لَمْ تُشْبِهْهُ فِي حُسْنِهِ.
\par 9 جَعَلْتُهُ جَمِيلاً بِكَثْرَةِ قُضْبَانِهِ حَتَّى حَسَدَتْهُ كُلُّ أَشْجَارِ عَدْنٍ الَّتِي فِي جَنَّةِ اللَّهِ].
\par 10 لِذَلِكَ هَكَذَا قَالَ السَّيِّدُ الرَّبُّ: [مِنْ أَجْلِ أَنَّكَ ارْتَفَعَتْ قَامَتُكَ وَقَدْ جَعَلَ فَرْعَهُ بَيْنَ الْغُيُومِ وَارْتَفَعَ قَلْبُهُ بِعُلُوِّهِ,
\par 11 أَسْلَمْتُهُ إِلَى يَدِ قَوِيِّ الأُمَمِ فَيَفْعَلُ بِهِ فِعْلاً. لِشَرِّهِ طَرَدْتُهُ.
\par 12 وَيَسْتَأْصِلُهُ الْغُرَبَاءُ عُتَاةُ الأُمَمِ وَيَتْرُكُونَهُ, فَتَتَسَاقَطُ قُضْبَانُهُ عَلَى الْجِبَالِ وَفِي جَمِيعِ الأَوْدِيَةِ, وَتَنْكَسِرُ قُضْبَانُهُ عِنْدَ كُلِّ أَنْهَارِ الأَرْضِ, وَيَنْزِلُ عَنْ ظِلِّهِ كُلُّ شُعُوبِ الأَرْضِ وَيَتْرُكُونَهُ.
\par 13 عَلَى هَشِيمِهِ تَسْتَقِرُّ جَمِيعُ طُيُورِ السَّمَاءِ, وَجَمِيعُ حَيَوَانِ الْبَرِّ تَكُونُ عَلَى قُضْبَانِهِ.
\par 14 لِكَيْلاَ تَرْتَفِعَ شَجَرَةٌ مَا وَهِيَ عَلَى الْمِيَاهِ لِقَامَتِهَا, وَلاَ تَجْعَلُ فَرْعَهَا بَيْنَ الْغُيُومِ, وَلاَ تَقُومُ بَلُّوطَاتُهَا فِي ارْتِفَاعِهَا كُلُّ شَارِبَةٍ مَاءً, لأَنَّهَا قَدْ أُسْلِمَتْ جَمِيعاً إِلَى الْمَوْتِ إِلَى الأَرْضِ السُّفْلَى فِي وَسَطِ بَنِي آدَمَ مَعَ الْهَابِطِينَ فِي الْجُبِّ.
\par 15 فِي يَوْمِ نُزُولِهِ إِلَى الْهَاوِيَةِ أَقَمْتُ نَوْحاً. كَسَوْتُ عَلَيْهِ الْغَمْرَ وَمَنَعْتُ أَنْهَارَهُ وَفَنِيَتِ الْمِيَاهُ الْكَثِيرَةُ, وَأَحْزَنْتُ لُبْنَانَ عَلَيْهِ, وَكُلُّ أَشْجَارِ الْحَقْلِ ذَبُلَتْ عَلَيْهِ.
\par 16 مِنْ صَوْتِ سُقُوطِهِ أَرْجَفْتُ الأُمَمَ عِنْدَ إِنْزَالِي إِيَّاهُ إِلَى الْهَاوِيَةِ مَعَ الْهَابِطِينَ فِي الْجُبِّ, فَتَتَعَزَّى فِي الأَرْضِ السُّفْلَى كُلُّ أَشْجَارِ عَدْنٍ مُخْتَارُ لُبْنَانَ وَخِيَارُهُ كُلُّ شَارِبَةٍ مَاءً.
\par 17 هُمْ أَيْضاً نَزَلُوا إِلَى الْهَاوِيَةِ مَعَهُ, إِلَى الْقَتْلَى بِالسَّيْفِ, وَزَرْعُهُ السَّاكِنُونَ تَحْتَ ظِلِّهِ فِي وَسَطِ الأُمَمِ.
\par 18 مَنْ أَشْبَهْتَ فِي الْمَجْدِ وَالْعَظَمَةِ هَكَذَا بَيْنَ أَشْجَارِ عَدْنٍ؟ سَتُحْدَرُ مَعَ أَشْجَارِ عَدْنٍ إِلَى الأَرْضِ السُّفْلَى, وَتَضْطَجِعُ بَيْنَ الْغُلْفِ مَعَ الْمَقْتُولِينَ بِالسَّيْفِ. هَذَا فِرْعَوْنُ وَكُلُّ جُمْهُورِهِ يَقُولُ السَّيِّدُ الرَّبُّ].

\chapter{32}

\par 1 وَكَانَ فِي السَّنَةِ الثَّانِيَةِ عَشَرَةَ فِي الشَّهْرِ الثَّانِي عَشَرَ فِي أَوَّلِ الشَّهْرِ أَنَّ كَلاَمَ الرَّبِّ صَارَ إِلَيَّ:
\par 2 [يَا ابْنَ آدَمَ, ارْفَعْ مَرْثَاةً عَلَى فِرْعَوْنَ مَلِكِ مِصْرَ وَقُلْ لَهُ: أَشْبَهْتَ شِبْلَ الأُمَمِ وَأَنْتَ نَظِيرُ تِمْسَاحٍ فِي الْبِحَارِ. انْدَفَقْتَ بِأَنْهَارِكَ وَكَدَّرْتَ الْمَاءَ بِرِجْلَيْكَ وَعَكَّرْتَ أَنْهَارَهُمْ.
\par 3 هَكَذَا قَالَ السَّيِّدُ الرَّبُّ: إِنِّي أَبْسُطُ عَلَيْكَ شَبَكَتِي مَعَ جَمَاعَةِ شُعُوبٍ كَثِيرَةٍ, وَهُمْ يُصْعِدُونَكَ فِي مِجْزَفَتِي
\par 4 وَأَتْرُكُكَ عَلَى الأَرْضِ وَأَطْرَحُكَ عَلَى وَجْهِ الْحَقْلِ وَأُقِرُّ عَلَيْكَ كُلَّ طُيُورِ السَّمَاءِ وَأُشْبِعُ مِنْكَ وُحُوشَ الأَرْضِ كُلَّهَا.
\par 5 وَأُلْقِي لَحْمَكَ عَلَى الْجِبَالِ, وَأَمْلأُ الأَوْدِيَةَ مِنْ جِيَفِكَ.
\par 6 وَأَسْقِي أَرْضَ فَيَضَانِكَ مِنْ دَمِكَ إِلَى الْجِبَالِ, وَتَمْتَلِئُ مِنْكَ الآفَاقُ.
\par 7 وَعِنْدَ إِطْفَائِي إِيَّاكَ أَحْجُبُ السَّمَاوَاتِ وَأُظْلِمُ نُجُومَهَا, وَأُغْشِي الشَّمْسَ بِسَحَابٍ, وَالْقَمَرُ لاَ يُضِيءُ ضُوءَهُ.
\par 8 وَأُظْلِمُ فَوْقَكَ كُلَّ أَنْوَارِ السَّمَاءِ الْمُنِيرَةِ, وَأَجْعَلُ الظُّلْمَةَ عَلَى أَرْضِكَ يَقُولُ السَّيِّدُ الرَّبُّ.
\par 9 وَأَغُمُّ قُلُوبَ شُعُوبٍ كَثِيرِينَ عِنْدَ إِتْيَانِي بِكَسْرِكَ بَيْنَ الأُمَمِ فِي أَرَاضٍ لَمْ تَعْرِفْهَا.
\par 10 وَأُحَيِّرُ مِنْكَ شُعُوباً كَثِيرِينَ, مُلُوكُهُمْ يَقْشَعِرُّونَ عَلَيْكَ اقْشِعْرَاراً عِنْدَمَا أَخْطِرُ بِسَيْفِي قُدَّامَ وُجُوهِهِمْ, فَيَرْجِفُونَ كُلَّ لَحْظَةٍ كُلُّ وَاحِدٍ عَلَى نَفْسِهِ فِي يَوْمِ سُقُوطِكَ].
\par 11 لأَنَّهُ هَكَذَا قَالَ السَّيِّدُ الرَّبُّ: [سَيْفُ مَلِكِ بَابِلَ يَأْتِي عَلَيْكَ.
\par 12 بِسُيُوفِ الْجَبَابِرَةِ أُسْقِطُ جُمْهُورَكَ. كُلُّهُمْ عُتَاةُ الأُمَمِ فَيَسْلُبُونَ كِبْرِيَاءَ مِصْرَ وَيَهْلِكُ كُلُّ جُمْهُورِهَا.
\par 13 وَأُبِيدُ جَمِيعَ بَهَائِمِهَا عَنِ الْمِيَاهِ الْكَثِيرَةِ, فَلاَ تُكَدِّرُهَا مِنْ بَعْدُ رِجْلُ إِنْسَانٍ, وَلاَ تُعَكِّرُهَا أَظْلاَفُ بَهِيمَةٍ.
\par 14 حِينَئِذٍ أُنْضِبُ مِيَاهَهُمْ وَأُجْرِي أَنْهَارَهُمْ كَالزَّيْتِ يَقُولُ السَّيِّدُ الرَّبُّ.
\par 15 حِينَ أَجْعَلُ أَرْضَ مِصْرَ خَرَاباً, وَتَخْلُو الأَرْضُ مِنْ مِلْئِهَا. عِنْدَ ضَرْبِي جَمِيعَ سُكَّانِهَا يَعْلَمُونَ أَنِّي أَنَا الرَّبُّ.
\par 16 هَذِهِ مَرْثَاةٌ يَرْثُونَ بِهَا. بَنَاتُ الأُمَمِ تَرْثُو بِهَا. عَلَى مِصْرَ وَعَلَى كُلِّ جُمْهُورِهَا تَرْثُو بِهَا يَقُولُ السَّيِّدُ الرَّبُّ].
\par 17 وَكَانَ فِي السَّنَةِ الثَّانِيَةِ عَشَرَةَ فِي الْخَامِسِ عَشَرَ مِنَ الشَّهْرِ أَنَّ كَلاَمَ الرَّبِّ كَانَ إِلَيَّ:
\par 18 [يَا ابْنَ آدَمَ, وَلْوِلْ عَلَى جُمْهُورِ مِصْرَ وَأَحْدِرْهُ هُوَ وَبَنَاتِ الأُمَمِ الْعَظِيمَةِ إِلَى الأَرْضِ السُّفْلَى مَعَ الْهَابِطِينَ فِي الْجُبِّ.
\par 19 مِمَّنْ نَعِمْتَ أَكْثَرَ؟ انْزِلْ وَاضْطَجِعْ مَعَ الْغُلْفِ.
\par 20 يَسْقُطُونَ فِي وَسَطِ الْقَتْلَى بِالسَّيْفِ. قَدْ أُسْلِمَ السَّيْفُ. أَمْسِكُوهَا مَعَ كُلِّ جُمْهُورِهَا.
\par 21 يُكَلِّمُهُ أَقْوِيَاءُ الْجَبَابِرَةِ مِنْ وَسَطِ الْهَاوِيَةِ مَعَ أَعْوَانِهِ. قَدْ نَزَلُوا. اضْطَجَعُوا غُلْفاً قَتْلَى بِالسَّيْفِ.
\par 22 هُنَاكَ أَشُّورُ وَكُلُّ جَمَاعَتِهَا. قُبُورُهُ مِنْ حَوْلِهِ. كُلُّهُمْ قَتْلَى سَاقِطُونَ بِالسَّيْفِ.
\par 23 اَلَّذِينَ جُعِلَتْ قُبُورُهُمْ فِي أَسَافِلِ الْجُبِّ وَجَمَاعَتُهَا حَوْلَ قَبْرِهَا, كُلُّهُمْ قَتْلَى سَاقِطُونَ بِالسَّيْفِ الَّذِينَ جَعَلُوا رُعْباً فِي أَرْضِ الأَحْيَاءِ.
\par 24 هُنَاكَ عِيلاَمُ وَكُلُّ جُمْهُورِهَا حَوْلَ قَبْرِهَا, كُلُّهُمْ قَتْلَى سَاقِطُونَ بِالسَّيْفِ الَّذِينَ هَبَطُوا غُلْفاً إِلَى الأَرْضِ السُّفْلَى, الَّذِينَ جَعَلُوا رُعْبَهُمْ فِي أَرْضِ الأَحْيَاءِ. فَحَمَلُوا خِزْيَهُمْ مَعَ الْهَابِطِينَ فِي الْجُبِّ.
\par 25 قَدْ جَعَلُوا لَهَا مَضْجَعاً بَيْنَ الْقَتْلَى مَعَ كُلِّ جُمْهُورِهَا. حَوْلَهُ قُبُورُهُمْ كُلُّهُمْ غُلْفٌ قَتْلَى بِالسَّيْفِ, مَعَ أَنَّهُ قَدْ جُعِلَ رُعْبُهُمْ فِي أَرْضِ الأَحْيَاءِ. قَدْ حَمَلُوا خِزْيَهُمْ مَعَ الْهَابِطِينَ فِي الْجُبِّ. قَدْ جُعِلَ فِي وَسَطِ الْقَتْلَى.
\par 26 هُنَاكَ مَاشِكُ وَتُوبَالُ وَكُلُّ جُمْهُورِهَا. حَوْلَهُ قُبُورُهَا. كُلُّهُمْ غُلْفٌ قَتْلَى بِالسَّيْفِ, مَعَ أَنَّهُمْ جَعَلُوا رُعْبَهُمْ فِي أَرْضِ الأَحْيَاءِ.
\par 27 وَلاَ يَضْطَجِعُونَ مَعَ الْجَبَابِرَةِ السَّاقِطِينَ مِنَ الْغُلْفِ النَّازِلِينَ إِلَى الْهَاوِيَةِ بِأَدَوَاتِ حَرْبِهِمْ, وَقَدْ وُضِعَتْ سُيُوفُهُمْ تَحْتَ رُؤُوسِهِمْ, فَتَكُونُ آثَامُهُمْ عَلَى عِظَامِهِمْ مَعَ أَنَّهُمْ رُعْبُ الْجَبَابِرَةِ فِي أَرْضِ الأَحْيَاءِ.
\par 28 أَمَّا أَنْتَ فَفِي وَسَطِ الْغُلْفِ تَنْكَسِرُ وَتَضْطَجِعُ مَعَ الْقَتْلَى بِالسَّيْفِ.
\par 29 هُنَاكَ أَدُومُ وَمُلُوكُهَا وَكُلُّ رُؤَسَائِهَا الَّذِينَ مَعَ جَبَرُوتِهِمْ قَدْ أُلْقُوا مَعَ الْقَتْلَى بِالسَّيْفِ, فَيَضْطَجِعُونَ مَعَ الْغُلْفِ وَمَعَ الْهَابِطِينَ فِي الْجُبِّ.
\par 30 هُنَاكَ أُمَرَاءُ الشِّمَالِ كُلُّهُمْ وَجَمِيعُ الصَّيْدُونِيِّينَ الْهَابِطِينَ مَعَ الْقَتْلَى بِرُعْبِهِمْ, خَزُوا مِنْ جَبَرُوتِهِمْ وَاضْطَجَعُوا غُلْفاً مَعَ قَتْلَى السَّيْفِ, وَحَمَلُوا خِزْيَهُمْ مَعَ الْهَابِطِينَ إِلَى الْجُبِّ.
\par 31 يَرَاهُمْ فِرْعَوْنُ وَيَتَعَزَّى عَنْ كُلِّ جُمْهُورِهِ. فِرْعَوْنُ وَكُلُّ جُمْهُورِهِ قَتْلَى بِالسَّيْفِ يَقُولُ السَّيِّدُ الرَّبُّ.
\par 32 لأَنِّي جَعَلْتُ رُعْبَهُ فِي أَرْضِ الأَحْيَاءِ, فَيُضْجَعُ بَيْنَ الْغُلْفِ مَعَ قَتْلَى السَّيْفِ, فِرْعَوْنُ وَكُلُّ جُمْهُورِهِ يَقُولُ السَّيِّدُ الرَّبُّ].

\chapter{33}

\par 1 وَكَانَ إِلَيَّ كَلاَمُ الرَّبِّ:
\par 2 [يَا ابْنَ آدَمَ, قُلْ لِبَنِي شَعْبِكَ: إِذَا جَلَبْتُ السَّيْفَ عَلَى أَرْضٍ, فَإِنْ أَخَذَ شَعْبُ الأَرْضِ رَجُلاً مِنْ بَيْنِهِمْ وَجَعَلُوهُ رَقِيباً لَهُمْ,
\par 3 فَإِذَا رَأَى السَّيْفَ مُقْبِلاً عَلَى الأَرْضِ نَفَخَ فِي الْبُوقِ وَحَذَّرَ الشَّعْبَ,
\par 4 وَسَمِعَ السَّامِعُ صَوْتَ الْبُوقِ وَلَمْ يَتَحَذَّرْ, فَجَاءَ السَّيْفُ وَأَخَذَهُ, فَدَمُهُ يَكُونُ عَلَى رَأْسِهِ.
\par 5 سَمِعَ صَوْتَ الْبُوقِ وَلَمْ يَتَحَذَّرْ, فَدَمُهُ يَكُونُ عَلَى نَفْسِهِ. لَوْ تَحَذَّرَ لَخَلَّصَ نَفْسَهُ.
\par 6 فَإِنْ رَأَى الرَّقِيبُ السَّيْفَ مُقْبِلاً وَلَمْ يَنْفُخْ فِي الْبُوقِ وَلَمْ يَتَحَذَّرِ الشَّعْبُ, فَجَاءَ السَّيْفُ وَأَخَذَ نَفْساً مِنْهُمْ, فَهُوَ قَدْ أُخِذَ بِذَنْبِهِ, أَمَّا دَمُهُ فَمِنْ يَدِ الرَّقِيبِ أَطْلُبُهُ.
\par 7 [وَأَنْتَ يَا ابْنَ آدَمَ فَقَدْ جَعَلْتُكَ رَقِيباً لِبَيْتِ إِسْرَائِيلَ, فَتَسْمَعُ الْكَلاَمَ مِنْ فَمِي وَتُحَذِّرُهُمْ مِنْ قِبَلِي.
\par 8 إِذَا قُلْتُ لِلشِّرِّيرِ: يَا شِرِّيرُ مَوْتاً تَمُوتُ! فَإِنْ لَمْ تَتَكَلَّمْ لِتُحَذِّرَ الشِّرِّيرَ مِنْ طَرِيقِهِ, فَذَلِكَ الشِّرِّيرُ يَمُوتُ بِذَنْبِهِ, أَمَّا دَمُهُ فَمِنْ يَدِكَ أَطْلُبُهُ.
\par 9 وَإِنْ حَذَّرْتَ الشِّرِّيرَ مِنْ طَرِيقِهِ لِيَرْجِعَ عَنْهُ وَلَمْ يَرْجِعْ عَنْ طَرِيقِهِ, فَهُوَ يَمُوتُ بِذَنْبِهِ. أَمَّا أَنْتَ فَقَدْ خَلَّصْتَ نَفْسَكَ.
\par 10 وَأَنْتَ يَا ابْنَ آدَمَ قُلْ لِبَيْتِ إِسْرَائِيلَ: أَنْتُمْ تَقُولُونَ: إِنَّ مَعَاصِيَنَا وَخَطَايَانَا عَلَيْنَا, وَبِهَا نَحْنُ فَانُونَ, فَكَيْفَ نَحْيَا؟
\par 11 قُلْ لَهُمْ: حَيٌّ أَنَا يَقُولُ السَّيِّدُ الرَّبُّ, إِنِّي لاَ أُسَرُّ بِمَوْتِ الشِّرِّيرِ, بَلْ بِأَنْ يَرْجِعَ الشِّرِّيرُ عَنْ طَرِيقِهِ وَيَحْيَا. إِرْجِعُوا ارْجِعُوا عَنْ طُرُقِكُمُ الرَّدِيئَةِ. فَلِمَاذَا تَمُوتُونَ يَا بَيْتَ إِسْرَائِيلَ؟
\par 12 وَأَنْتَ يَا ابْنَ آدَمَ فَقُلْ لِبَنِي شَعْبِكَ: إِنَّ بِرَّ الْبَارِّ لاَ يُنَجِّيهِ فِي يَوْمِ مَعْصِيَتِهِ, وَالشِّرِّيرُ لاَ يَعْثُرُ بِشَرِّهِ فِي يَوْمِ رُجُوعِهِ عَنْ شَرِّهِ. وَلاَ يَسْتَطِيعُ الْبَارُّ أَنْ يَحْيَا بِبِرِّهِ فِي يَوْمِ خَطِيئَتِهِ.
\par 13 إِذَا قُلْتُ لِلْبَارِّ حَيَاةً تَحْيَا, فَاتَّكَلَ هُوَ عَلَى بِرِّهِ وَأَثِمَ, فَبِرُّهُ كُلُّهُ لاَ يُذْكَرُ, بَلْ بِإِثْمِهِ الَّذِي فَعَلَهُ يَمُوتُ.
\par 14 وَإِذَا قُلْتُ لِلشِّرِّيرِ: مَوْتاً تَمُوتُ! فَإِنْ رَجَعَ عَنْ خَطِيَّتِهِ وَعَمِلَ بِالْعَدْلِ وَالْحَقِّ,
\par 15 إِنْ رَدَّ الشِّرِّيرُ الرَّهْنَ وَعَوَّضَ عَنِ الْمُغْتَصَبِ وَسَلَكَ فِي فَرَائِضِ الْحَيَاةِ بِلاَ عَمَلِ إِثْمٍ, فَإِنَّهُ حَيَاةً يَحْيَا. لاَ يَمُوتُ.
\par 16 كُلُّ خَطِيَّتِهِ الَّتِي أَخْطَأَ بِهَا لاَ تُذْكَرُ عَلَيْهِ. عَمِلَ بِالْعَدْلِ وَالْحَقِّ فَيَحْيَا حَيَاةً.
\par 17 وَأَبْنَاءُ شَعْبِكَ يَقُولُونَ: لَيْسَتْ طَرِيقُ الرَّبِّ مُسْتَوِيَةً. بَلْ هُمْ طَرِيقُهُمْ غَيْرُ مُسْتَوِيَةٍ!
\par 18 عِنْدَ رُجُوعِ الْبَارِّ عَنْ بِرِّهِ وَعِنْدَ عَمَلِهِ إِثْماً فَإِنَّهُ يَمُوتُ بِهِ.
\par 19 وَعِنْدَ رُجُوعِ الشِّرِّيرِ عَنْ شَرِّهِ وَعِنْدَ عَمَلِهِ بِالْعَدْلِ وَالْحَقِّ, فَإِنَّهُ يَحْيَا بِهِمَا.
\par 20 وَأَنْتُمْ تَقُولُونَ: إِنَّ طَرِيقَ الرَّبِّ غَيْرُ مُسْتَوِيَةٍ. إِنِّي أَحْكُمُ عَلَى كُلِّ وَاحِدٍ مِنْكُمْ كَطُرُقِهِ يَا بَيْتَ إِسْرَائِيلَ].
\par 21 وَكَانَ فِي السَّنَةِ الثَّانِيَةِ عَشَرَةَ مِنْ سَبْيِنَا, فِي الشَّهْرِ الْعَاشِرِ فِي الْخَامِسِ مِنَ الشَّهْرِ, أَنَّهُ جَاءَ إِلَيَّ مُنْفَلِتٌ مِنْ أُورُشَلِيمَ فَقَالَ: [قَدْ ضُرِبَتِ الْمَدِينَةُ».
\par 22 وَكَانَتْ يَدُ الرَّبِّ عَلَيَّ مَسَاءً قَبْلَ مَجِيءِ الْمُنْفَلِتِ, وَفَتَحَتْ فَمِي حَتَّى جَاءَ إِلَيَّ صَبَاحاً, فَانْفَتَحَ فَمِي وَلَمْ أَكُنْ بَعْدُ أَبْكَمَ.
\par 23 فَكَانَ إِلَيَّ كَلاَمُ الرَّبِّ:
\par 24 [يَا ابْنَ آدَمَ, إِنَّ السَّاكِنِينَ فِي هَذِهِ الْخِرَبِ فِي أَرْضِ إِسْرَائِيلَ يَقُولُونَ: إِنَّ إِبْرَاهِيمَ كَانَ وَاحِداً وَقَدْ وَرَثَ الأَرْضَ. وَنَحْنُ كَثِيرُونَ. لَنَا أُعْطِيَتِ الأَرْضُ مِيرَاثاً.
\par 25 لِذَلِكَ قُلْ لَهُمْ. هَكَذَا قَالَ السَّيِّدُ الرَّبُّ: تَأْكُلُونَ بِالدَّمِ وَتَرْفَعُونَ أَعْيُنَكُمْ إِلَى أَصْنَامِكُمْ وَتَسْفِكُونَ الدَّمَ. أَفَتَرِثُونَ الأَرْضَ؟
\par 26 وَقَفْتُمْ عَلَى سَيْفِكُمْ. فَعَلْتُمُ الرِّجْسَ وَكُلٌّ مِنْكُمْ نَجَّسَ امْرَأَةَ صَاحِبِهِ. أَفَتَرِثُونَ الأَرْضَ؟
\par 27 قُلْ لَهُمْ. هَكَذَا قَالَ السَّيِّدُ الرَّبُّ: حَيٌّ أَنَا, إِنَّ الَّذِينَ فِي الْخِرَبِ يَسْقُطُونَ بِالسَّيْفِ, وَالَّذِي هُوَ عَلَى وَجْهِ الْحَقْلِ أَبْذِلُهُ لِلْوَحْشِ مَأْكَلاً, وَالَّذِينَ فِي الْحُصُونِ وَفِي الْمَغَايِرِ يَمُوتُونَ بِالْوَبَإِ.
\par 28 فَأَجْعَلُ الأَرْضَ خَرِبَةً مُقْفِرَةً, وَتَبْطُلُ كِبْرِيَاءُ عِزَّتِهَا, وَتَخْرَبُ جِبَالُ إِسْرَائِيلَ بِلاَ عَابِرٍ.
\par 29 فَيَعْلَمُونَ أَنِّي أَنَا الرَّبُّ حِينَ أَجْعَلُ الأَرْضَ خَرِبَةً مُقْفِرَةً عَلَى كُلِّ رَجَاسَاتِهِمِ الَّتِي فَعَلُوهَا.
\par 30 [وَأَنْتَ يَا ابْنَ آدَمَ, فَإِنَّ بَنِي شَعْبِكَ يَتَكَلَّمُونَ عَلَيْكَ بِجَانِبِ الْجُدْرَانِ وَفِي أَبْوَابِ الْبُيُوتِ وَيَتَكَلَّمُ الْوَاحِدُ مَعَ الآخَرِ, الرَّجُلُ مَعَ أَخِيهِ قَائِلِينَ: هَلُمَّ اسْمَعُوا مَا هُوَ الْكَلاَمُ الْخَارِجُ مِنْ عَُِنْدِ الرَّبِّ!
\par 31 وَيَأْتُونَ إِلَيْكَ كَمَا يَأْتِي الشَّعْبُ, وَيَجْلِسُونَ أَمَامَكَ كَشَعْبِي, وَيَسْمَعُونَ كَلاَمَكَ وَلاَ يَعْمَلُونَ بِهِ, لأَنَّهُمْ بِأَفْوَاهِهِمْ يُظْهِرُونَ أَشْوَاقاً وَقَلْبُهُمْ ذَاهِبٌ وَرَاءَ كَسْبِهِمْ.
\par 32 وَهَا أَنْتَ لَهُمْ كَشِعْرِ أَشْوَاقٍ لِجَمِيلِ الصَّوْتِ يُحْسِنُ الْعَزْفَ, فَيَسْمَعُونَ كَلاَمَكَ وَلاَ يَعْمَلُونَ بِهِ.
\par 33 وَإِذَا جَاءَ هَذَا (لأَنَّهُ يَأْتِي) فَيَعْلَمُونَ أَنَّ نَبِيّاً كَانَ فِي وَسَطِهِمْ].

\chapter{34}

\par 1 وَكَانَ إِلَيَّ كَلاَمُ الرَّبِّ:
\par 2 [يَا ابْنَ آدَمَ تَنَبَّأْ عَلَى رُعَاةِ إِسْرَائِيلَ, وَقُلْ لَهُمْ هَكَذَا قَالَ السَّيِّدُ الرَّبُّ لِلرُّعَاةِ: وَيْلٌ لِرُعَاةِ إِسْرَائِيلَ الَّذِينَ كَانُوا يَرْعُونَ أَنْفُسَهُمْ. أَلاَ يَرْعَى الرُّعَاةُ الْغَنَمَ؟
\par 3 تَأْكُلُونَ الشَّحْمَ وَتَلْبِسُونَ الصُّوفَ وَتَذْبَحُونَ السَّمِينَ وَلاَ تَرْعُونَ الْغَنَمَ.
\par 4 الْمَرِيضُ لَمْ تُقَوُّوهُ, وَالْمَجْرُوحُ لَمْ تَعْصِبُوهُ, وَالْمَكْسُورُ لَمْ تَجْبُرُوهُ, وَالْمَطْرُودُ لَمْ تَسْتَرِدُّوهُ, وَالضَّالُّ لَمْ تَطْلُبُوهُ, بَلْ بِشِدَّةٍ وَبِعُنْفٍ تَسَلَّطْتُمْ عَلَيْهِمْ.
\par 5 فَتَشَتَّتَتْ بِلاَ رَاعٍ وَصَارَتْ مَأْكَلاً لِجَمِيعِ وُحُوشِ الْحَقْلِ, وَتَشَتَّتَتْ.
\par 6 ضَلَّتْ غَنَمِي فِي كُلِّ الْجِبَالِ وَعَلَى كُلِّ تَلٍّ عَالٍ وَعَلَى كُلِّ وَجْهِ الأَرْضِ. تَشَتَّتَتْ غَنَمِي وَلَمْ يَكُنْ مَنْ يَسْأَلُ أَوْ يُفَتِّشُ.
\par 7 [فَلِذَلِكَ أَيُّهَا الرُّعَاةُ اسْمَعُوا كَلاَمَ الرَّبِّ:
\par 8 حَيٌّ أَنَا يَقُولُ السَّيِّدُ الرَّبُّ, مِنْ حَيْثُ إِنَّ غَنَمِي صَارَتْ غَنِيمَةً وَمَأْكَلاً لِكُلِّ وَحْشِ الْحَقْلِ, إِذْ لَمْ يَكُنْ رَاعٍ وَلاَ سَأَلَ رُعَاتِي عَنْ غَنَمِي, وَرَعَى الرُّعَاةُ أَنْفُسَهُمْ وَلَمْ يَرْعُوا غَنَمِي,
\par 9 فَلِذَلِكَ أَيُّهَا الرُّعَاةُ اسْمَعُوا كَلاَمَ الرَّبِّ.
\par 10 هَكَذَا قَالَ السَّيِّدُ الرَّبُّ: هَئَنَذَا عَلَى الرُّعَاةِ وَأَطْلُبُ غَنَمِي مِنْ يَدِهِمْ, وَأَكُفُّهُمْ عَنْ رَعْيِ الْغَنَمِ, وَلاَ يَرْعَى الرُّعَاةُ أَنْفُسَهُمْ بَعْدُ, فَأُخَلِّصُ غَنَمِي مِنْ أَفْوَاهِهِمْ فَلاَ تَكُونُ لَهُمْ مَأْكَلاً.
\par 11 لأَنَّهُ هَكَذَا قَالَ السَّيِّدُ الرَّبُّ: هَئَنَذَا أَسْأَلُ عَنْ غَنَمِي وَأَفْتَقِدُهَا.
\par 12 كَمَا يَفْتَقِدُ الرَّاعِي قَطِيعَهُ يَوْمَ يَكُونُ فِي وَسَطِ غَنَمِهِ الْمُشَتَّتَةِ, هَكَذَا أَفْتَقِدُ غَنَمِي وَأُخَلِّصُهَا مِنْ جَمِيعِ الأَمَاكِنِ الَّتِي تَشَتَّتَتْ إِلَيْهَا فِي يَوْمِ الْغَيْمِ وَالضَّبَابِ.
\par 13 وَأُخْرِجُهَا مِنَ الشُّعُوبِ وَأَجْمَعُهَا مِنَ الأَرَاضِي, وَآتِي بِهَا إِلَى أَرْضِهَا وَأَرْعَاهَا عَلَى جِبَالِ إِسْرَائِيلَ وَفِي الأَوْدِيَةِ وَفِي جَمِيعِ مَسَاكِنِ الأَرْضِ.
\par 14 أَرْعَاهَا فِي مَرْعًى جَيِّدٍ, وَيَكُونُ مَرَاحُهَا عَلَى جِبَالِ إِسْرَائِيلَ الْعَالِيَةِ. هُنَالِكَ تَرْبُضُ فِي مَرَاحٍ حَسَنٍ, وَفِي مَرْعًى دَسِمٍ يَرْعُونَ عَلَى جِبَالِ إِسْرَائِيلَ.
\par 15 أَنَا أَرْعَى غَنَمِي وَأُرْبِضُهَا يَقُولُ السَّيِّدُ الرَّبُّ.
\par 16 وَأَطْلُبُ الضَّالَّ, وَأَسْتَرِدُّ الْمَطْرُودَ, وَأَجْبِرُ الْكَسِيرَ, وَأَعْصِبُ الْجَرِيحَ, وَأُبِيدُ السَّمِينَ وَالْقَوِيَّ, وَأَرْعَاهَا بِعَدْلٍ.
\par 17 وَأَنْتُمْ يَا غَنَمِي فَهَكَذَا قَالَ السَّيِّدُ الرَّبُّ: هَئَنَذَا أَحْكُمُ بَيْنَ شَاةٍ وَشَاةٍ. بَيْنَ كِبَاشٍ وَتُيُوسٍ.
\par 18 أَهُوَ صَغِيرٌ عِنْدَكُمْ أَنْ تَرْعُوا الْمَرْعَى الْجَيِّدَ وَبَقِيَّةُ مَرَاعِيكُمْ تَدُوسُونَهَا بِأَرْجُلِكُمْ, وَأَنْ تَشْرَبُوا مِنَ الْمِيَاهِ الْعَمِيقَةِ, وَالْبَقِيَّةُ تُكَدِّرُونَهَا بِأَقْدَامِكُمْ؟
\par 19 وَغَنَمِي تَرْعَى مِنْ دَوْسِ أَقْدَامِكُمْ, وَتَشْرَبُ مِنْ كَدَرِ أَرْجُلِكُمْ!].
\par 20 لِذَلِكَ هَكَذَا قَالَ السَّيِّدُ الرَّبُّ لَهُمْ: [هَئَنَذَا أَحْكُمُ بَيْنَ الشَّاةِ السَّمِينَةِ وَالشَّاةِ الْمَهْزُولَةِ.
\par 21 لأَنَّكُمْ بَهَزْتُمْ بِالْجَنْبِ وَالْكَتِفِ, وَنَطَحْتُمُ الْمَرِيضَةَ بِقُرُونِكُمْ حَتَّى شَتَّتْتُمُوهَا إِلَى خَارِجٍ.
\par 22 فَأُخَلِّصُ غَنَمِي فَلاَ تَكُونُ مِنْ بَعْدُ غَنِيمَةً, وَأَحْكُمُ بَيْنَ شَاةٍ وَشَاةٍ.
\par 23 وَأُقِيمُ عَلَيْهَا رَاعِياً وَاحِداً فَيَرْعَاهَا عَبْدِي دَاوُدُ. هُوَ يَرْعَاهَا وَهُوَ يَكُونُ لَهَا رَاعِياً.
\par 24 وَأَنَا الرَّبُّ أَكُونُ لَهُمْ إِلَهاً, وَعَبْدِي دَاوُدُ رَئِيساً فِي وَسَطِهِمْ. أَنَا الرَّبُّ تَكَلَّمْتُ.
\par 25 وَأَقْطَعُ مَعَهُمْ عَهْدَ سَلاَمٍ, وَأَنْزِعُ الْوُحُوشَ الرَّدِيئَةَ مِنَ الأَرْضِ, فَيَسْكُنُونَ فِي الْبَرِّيَّةِ مُطْمَئِنِّينَ وَيَنَامُونَ فِي الْوُعُورِ.
\par 26 وَأَجْعَلُهُمْ وَمَا حَوْلَ أَكَمَتِي بَرَكَةً, وَأُنْزِلُ عَلَيْهِمِ الْمَطَرَ فِي وَقْتِهِ فَتَكُونُ أَمْطَارَ بَرَكَةٍ.
\par 27 وَتُعْطِي شَجَرَةُ الْحَقْلِ ثَمَرَتَهَا, وَتُعْطِي الأَرْضُ غَلَّتَهَا, وَيَكُونُونَ آمِنِينَ فِي أَرْضِهِمْ, وَيَعْلَمُونَ أَنِّي أَنَا الرَّبُّ عِنْدَ تَكْسِيرِي رُبُطَ نِيرِهِمْ, وَإِذَا أَنْقَذْتُهُمْ مِنْ يَدِ الَّذِينَ اسْتَعْبَدُوهُمْ.
\par 28 فَلاَ يَكُونُونَ بَعْدُ غَنِيمَةً لِلأُمَمِ, وَلاَ يَأْكُلُهُمْ وَحْشُ الأَرْضِ. بَلْ يَسْكُنُونَ آمِنِينَ وَلاَ مُخِيفٌ.
\par 29 وَأُقِيمُ لَهُمْ غَرْساً لِصِيتٍ فَلاَ يَكُونُونَ بَعْدُ مَفْنِيِّي الْجُوعِ فِي الأَرْضِ, وَلاَ يَحْمِلُونَ بَعْدُ تَعْيِيرَ الأُمَمِ.
\par 30 فَيَعْلَمُونَ أَنِّي أَنَا الرَّبُّ إِلَهُهُمْ مَعَهُمْ, وَهُمْ شَعْبِي بَيْتُ إِسْرَائِيلَ يَقُولُ السَّيِّدُ الرَّبُّ.
\par 31 وَأَنْتُمْ يَا غَنَمِي, غَنَمُ مَرْعَايَ, أُنَاسٌ أَنْتُمْ. أَنَا إِلَهُكُمْ يَقُولُ السَّيِّدُ الرَّبُّ].

\chapter{35}

\par 1 وَكَانَ إِلَيَّ كَلاَمُ الرَّبِّ:
\par 2 [يَا ابْنَ آدَمَ, اجْعَلْ وَجْهَكَ نَحْوَ جَبَلِ سَعِيرَ وَتَنَبَّأْ عَلَيْهِ
\par 3 وَقُلْ لَهُ هَكَذَا قَالَ السَّيِّدُ الرَّبُّ: هَئَنَذَا عَلَيْكَ يَا جَبَلَ سَعِيرَ, وَأَمُدُّ يَدِي عَلَيْكَ وَأَجْعَلُكَ خَرَاباً مُقْفِراً.
\par 4 أَجْعَلُ مُدُنَكَ خَرِبَةً, وَتَكُونُ أَنْتَ مُقْفِراً وَتَعْلَمُ أَنِّي أَنَا الرَّبُّ.
\par 5 لأَنَّهُ كَانَتْ لَكَ بُغْضَةٌ أَبَدِيَّةٌ, وَدَفَعْتَ بَنِي إِسْرَائِيلَ إِلَى يَدِ السَّيْفِ فِي وَقْتِ مُصِيبَتِهِمْ, وَقْتِ إِثْمِ النِّهَايَةِ.
\par 6 لِذَلِكَ حَيٌّ أَنَا يَقُولُ السَّيِّدُ الرَّبُّ إِنِّي أُهَيِّئُكَ لِلدَّمِ وَالدَّمُ يَتْبَعُكَ. إِذْ لَمْ تَكْرَهِ الدَّمَ فَالدَّمُ يَتْبَعُكَ.
\par 7 فَأَجْعَلُ جَبَلَ سَعِيرَ خَرَاباً وَمُقْفِراً, وَأَسْتَأْصِلُ مِنْهُ الذَّاهِبَ وَالآئِبَ.
\par 8 وَأَمْلأُ جِبَالَهُ مِنْ قَتْلاَهُ. تِلاَلُكَ وَأَوْدِيَتُكَ وَجَمِيعُ أَنْهَارِكَ يَسْقُطُونَ فِيهَا قَتْلَى بِالسَّيْفِ.
\par 9 وَأُصَيِّرُكَ خِرَباً أَبَدِيَّةً, وَمُدُنُكَ لَنْ تَعُودَ, فَتَعْلَمُونَ أَنِّي أَنَا الرَّبُّ.
\par 10 لأَنَّكَ قُلْتَ: إِنَّ هَاتَيْنِ الأُمَّتَيْنِ وَهَاتَيْنِ الأَرْضَيْنِ تَكُونَانِ لِي فَنَمْتَلِكُهُمَا وَالرَّبُّ كَانَ هُنَاكَ,
\par 11 فَلِذَلِكَ حَيٌّ أَنَا يَقُولُ السَّيِّدُ الرَّبُّ, لأَفْعَلَنَّ كَغَضَبِكَ وَحَسَدِكَ اللَّذَيْنِ عَامَلْتَ بِهِمَا مِنْ بُغْضَتِكَ لَهُمْ, وَأُعَرِّفُ بِنَفْسِي بَيْنَهُمْ عِنْدَمَا أَحْكُمُ عَلَيْكَ,
\par 12 فَتَعْلَمُ أَنِّي أَنَا الرَّبُّ قَدْ سَمِعْتُ كُلَّ إِهَانَتِكَ الَّتِي تَكَلَّمْتَ بِهَا عَلَى جِبَالِ إِسْرَائِيلَ قَائِلاً: قَدْ خَرِبَتْ. قَدْ أُعْطِينَاهَا مَأْكَلاً.
\par 13 قَدْ تَعَظَّمْتُمْ عَلَيَّ بِأَفْوَاهِكُمْ وَكَثَّرْتُمْ كَلاَمَكُمْ عَلَيَّ. أَنَا سَمِعْتُ.
\par 14 هَكَذَا قَالَ السَّيِّدُ الرَّبُّ: عِنْدَ فَرَحِ كُلِّ الأَرْضِ أَجْعَلُكَ مُقْفِراً.
\par 15 كَمَا فَرِحْتَ عَلَى مِيرَاثِ بَيْتِ إِسْرَائِيلَ لأَنَّهُ خَرِبَ كَذَلِكَ أَفْعَلُ بِكَ. تَكُونُ خَرَاباً يَا جَبَلَ سَِعِيرَ أَنْتَ وَكُلُّ أَدُومَ بِأَجْمَعِهَا, فَيَعْلَمُونَ أَنِّي أَنَا الرَّبُّ.

\chapter{36}

\par 1 [وَأَنْتَ يَا ابْنَ آدَمَ فَتَنَبَّأْ لِجِبَالِ إِسْرَائِيلَ وَقُلْ: يَا جِبَالَ إِسْرَائِيلَ اسْمَعِي كَلِمَةَ الرَّبِّ.
\par 2 هَكَذَا قَالَ السَّيِّدُ الرَّبُّ: مِنْ أَجْلِ أَنَّ الْعَدُوَّ قَالَ عَلَيْكُمْ: هَهْ! إِنَّ الْمُرْتَفَعَاتِ الْقَدِيمَةَ صَارَتْ لَنَا مِيرَاثاً
\par 3 فَلِذَلِكَ هَكَذَا قَالَ السَّيِّدُ الرَّبُّ: مِنْ أَجْلِ أَنَّهُمْ قَدْ أَخْرَبُوكُمْ وَتَهَمَّمُوكُمْ مِنْ كُلِّ جَانِبٍ لِتَكُونُوا مِيرَاثاً لِبَقِيَّةِ الأُمَمِ, وَأُصْعِدْتُمْ عَلَى شِفَاهِ اللِّسَانِ وَصِرْتُمْ مَذَمَّةَ الشَّعْبِ,
\par 4 لِذَلِكَ فَاسْمَعِي يَا جِبَالَ إِسْرَائِيلَ كَلِمَةَ السَّيِّدِ الرَّبِّ. هَكَذَا قَالَ السَّيِّدُ الرَّبُّ لِلْجِبَالِ وَالآكَامِ وَالأَنْهَارِ وَالأَوْدِيَةِ وَالْخِرَبِ الْمُقْفِرَةِ وَالْمُدُنِ الْمَهْجُورَةِ الَّتِي صَارَتْ لِلنَّهْبِ وَالاِسْتِهْزَاءِ, لِبَقِيَّةِ الأُمَمِ الَّذِينَ حَوْلَهَا.
\par 5 مِنْ أَجْلِ ذَلِكَ هَكَذَا قَالَ السَّيِّدُ الرَّبُّ: إِنِّي فِي نَارِ غَيْرَتِي تَكَلَّمْتُ عَلَى بَقِيَّةِ الأُمَمِ وَعَلَى أَدُومَ كُلِّهَا الَّذِينَ جَعَلُوا أَرْضِي مِيرَاثاً لَهُمْ بِفَرَحِ كُلِّ الْقَلْبِ وَبُغْضَةِ نَفْسٍ لِنَهْبِهَا غَنِيمَةً.
\par 6 فَتَنَبَّأْ عَلَى أَرْضِ إِسْرَائِيلَ وَقُلْ لِلْجِبَالِ وَالتِّلاَلِ وَالأَنْهَارِ وَالأَوْدِيَةِ هَكَذَا قَالَ السَّيِّدُ الرَّبُّ: هَئَنَذَا فِي غَيْرَتِي وَفِي غَضَبِي تَكَلَّمْتُ مِنْ أَجْلِ أَنَّكُمْ حَمَلْتُمْ تَعْيِيرَ الأُمَمِ.
\par 7 لِذَلِكَ هَكَذَا قَالَ السَّيِّدُ الرَّبُّ: إِنِّي رَفَعْتُ يَدِي, فَالأُمَمُ الَّذِينَ حَوْلَكُمْ هُمْ يَحْمِلُونَ تَعْيِيرَهُمْ.
\par 8 أَمَّا أَنْتُمْ يَا جِبَالَ إِسْرَائِيلَ فَإِنَّكُمْ تُنْبِتُونَ فُرُوعَكُمْ وَتُثْمِرُونَ ثَمَرَكُمْ لِشَعْبِي إِسْرَائِيلَ, لأَنَّهُ قَرِيبُ الإِتْيَانِ.
\par 9 لأَنِّي أَنَا لَكُمْ وَأَلْتَفِتُ إِلَيْكُمْ فَتُحْرَثُونَ وَتُزْرَعُونَ.
\par 10 وَأُكَثِّرُ النَّاسَ عَلَيْكُمْ كُلَّ بَيْتِ إِسْرَائِيلَ بِأَجْمَعِهِ, فَتُعْمَرُ الْمُدُنُ وَتُبْنَى الْخِرَبُ.
\par 11 وَأُكَثِّرُ عَلَيْكُمُ الإِنْسَانَ وَالْبَهِيمَةَ فَيَكْثُرُونَ وَيُثْمِرُونَ, وَأُسَكِّنُكُمْ حَسَبَ حَالَتِكُمُ الْقَدِيمَةِ, وَأُحْسِنُ إِلَيْكُمْ أَكْثَرَ مِمَّا فِي أَوَائِلِكُمْ, فَتَعْلَمُونَ أَنِّي أَنَا الرَّبُّ.
\par 12 وَأُمَشِّي النَّاسَ عَلَيْكُمْ شَعْبِي إِسْرَائِيلَ, فَيَرِثُونَكَ فَتَكُونُ لَهُمْ مِيرَاثاً وَلاَ تَعُودُ بَعْدُ تُثْكِلُهُمْ.
\par 13 هَكَذَا قَالَ السَّيِّدُ الرَّبُّ: مِنْ أَجْلِ أَنَّهُمْ قَالُوا لَكُمْ: أَنْتِ أَكَّالَةُ النَّاسِ وَمُثْكِلَةُ شُعُوبِكِ.
\par 14 لِذَلِكَ لَنْ تَأْكُلِي النَّاسَ بَعْدُ وَلاَ تُثْكِلِي شُعُوبَكِ بَعْدُ يَقُولُ السَّيِّدُ الرَّبُّ.
\par 15 وَلاَ أُسَمِّعُ فِيكِ مِنْ بَعْدُ تَعْيِيرَ الأُمَمِ, وَلاَ تَحْمِلِينَ تَعْيِيرَ الشُّعُوبِ بَعْدُ, وَلاَ تُعْثِرِينَ شُعُوبَكِ بَعْدُ يَقُولُ السَّيِّدُ الرَّبُّ].
\par 16 وَكَانَ إِلَيَّ كَلاَمُ الرَّبِّ:
\par 17 [يَا ابْنَ آدَمَ, إِنَّ بَيْتَ إِسْرَائِيلَ لَمَّا سَكَنُوا أَرْضَهُمْ نَجَّسُوهَا بِطَرِيقِهِمْ وَأَفْعَالِهِمْ. كَانَتْ طَرِيقُهُمْ أَمَامِي كَنَجَاسَةِ الطَّامِثِ,
\par 18 فَسَكَبْتُ غَضَبِي عَلَيْهِمْ لأَجْلِ الدَّمِ الَّذِي سَفَكُوهُ عَلَى الأَرْضِ, وَبِأَصْنَامِهِمْ نَجَّسُوهَا.
\par 19 فَبَدَّدْتُهُمْ فِي الأُمَمِ فَتَذَرُّوا فِي الأَرَاضِي. كَطَرِيقِهِمْ وَأَفْعَالِهِمْ دِنْتُهُمْ.
\par 20 فَلَمَّا جَاءُوا إِلَى الأُمَمِ حَيْثُ جَاءُوا نَجَّسُوا اسْمِي الْقُدُّوسَ, إِذْ قَالُوا لَهُمْ: هَؤُلاَءِ شَعْبُ الرَّبِّ وَقَدْ خَرَجُوا مِنْ أَرْضِهِ.
\par 21 فَتَحَنَّنْتُ عَلَى اسْمِي الْقُدُّوسِ الَّذِي نَجَّسَهُ بَيْتُ إِسْرَائِيلَ فِي الأُمَمِ حَيْثُ جَاءُوا]
\par 22 لِذَلِكَ فَقُلْ لِبَيْتِ إِسْرَائِيلَ. هَكَذَا قَالَ السَّيِّدُ الرَّبُّ: [لَيْسَ لأَجْلِكُمْ أَنَا صَانِعٌ يَا بَيْتَ إِسْرَائِيلَ, بَلْ لأَجْلِ اسْمِي الْقُدُّوسِ الَّذِي نَجَّسْتُمُوهُ فِي الأُمَمِ حَيْثُ جِئْتُمْ.
\par 23 فَأُقَدِّسُ اسْمِي الْعَظِيمَ الْمُنَجَّسَ فِي الأُمَمِ الَّذِي نَجَّسْتُمُوهُ فِي وَسَطِهِمْ, فَتَعْلَمُ الأُمَمُ أَنِّي أَنَا الرَّبُّ يَقُولُ السَّيِّدُ الرَّبُّ حِينَ أَتَقَدَّسُ فِيكُمْ قُدَّامَ أَعْيُنِهِمْ.
\par 24 وَآخُذُكُمْ مِنْ بَيْنِ الأُمَمِ وَأَجْمَعُكُمْ مِنْ جَمِيعِ الأَرَاضِي وَآتِي بِكُمْ إِلَى أَرْضِكُمْ.
\par 25 وَأَرُشُّ عَلَيْكُمْ مَاءً طَاهِراً فَتُطَهَّرُونَ. مِنْ كُلِّ نَجَاسَتِكُمْ وَمِنْ كُلِّ أَصْنَامِكُمْ أُطَهِّرُكُمْ.
\par 26 وَأُعْطِيكُمْ قَلْباً جَدِيداً, وَأَجْعَلُ رُوحاً جَدِيدَةً فِي دَاخِلِكُمْ, وَأَنْزِعُ قَلْبَ الْحَجَرِ مِنْ لَحْمِكُمْ وَأُعْطِيكُمْ قَلْبَ لَحْمٍ.
\par 27 وَأَجْعَلُ رُوحِي فِي دَاخِلِكُمْ, وَأَجْعَلُكُمْ تَسْلُكُونَ فِي فَرَائِضِي وَتَحْفَظُونَ أَحْكَامِي وَتَعْمَلُونَ بِهَا.
\par 28 وَتَسْكُنُونَ الأَرْضَ الَّتِي أَعْطَيْتُ آبَاءَكُمْ إِيَّاهَا, وَتَكُونُونَ لِي شَعْباً وَأَنَا أَكُونُ لَكُمْ إِلَهاً.
\par 29 وَأُخَلِّصُكُمْ مِنْ كُلِّ نَجَاسَاتِكُمْ. وَأَدْعُو الْحِنْطَةَ وَأُكَثِّرُهَا وَلاَ أَضَعُ عَلَيْكُمْ جُوعاً.
\par 30 وَأُكَثِّرُ ثَمَرَ الشَّجَرِ وَغَلَّةَ الْحَقْلِ لِكَيْلاَ تَنَالُوا بَعْدُ عَارَ الْجُوعِ بَيْنَ الأُمَمِ.
\par 31 فَتَذْكُرُونَ طُرُقَكُمُ الرَّدِيئَةَ وَأَعْمَالَكُمْ غَيْرَ الصَّالِحَةِ, وَتَمْقُتُونَ أَنْفُسَكُمْ أَمَامَ وُجُوهِكُمْ مِنْ أَجْلِ آثَامِكُمْ وَعَلَى رَجَاسَاتِكُمْ.
\par 32 لاَ مِنْ أَجْلِكُمْ أَنَا صَانِعٌ يَقُولُ السَّيِّدُ الرَّبُّ, فَلْيَكُنْ مَعْلُوماً لَكُمْ. فَاخْجَلُوا وَاخْزُوا مِنْ طُرُقِكُمْ يَا بَيْتَ إِسْرَائِيلَ.
\par 33 هَكَذَا قَالَ السَّيِّدُ الرَّبُّ: فِي يَوْمِ تَطْهِيرِي إِيَّاكُمْ مِنْ كُلِّ آثَامِكُمْ أُسْكِنُكُمْ فِي الْمُدُنِ, فَتُبْنَى الْخِرَبُ.
\par 34 وَتُفْلَحُ الأَرْضُ الْخَرِبَةُ عِوَضاً عَنْ كَوْنِهَا خَرِبَةً أَمَامَ عَيْنَيْ كُلِّ عَابِرٍ.
\par 35 فَيَقُولُونَ: هَذِهِ الأَرْضُ الْخَرِبَةُ صَارَتْ كَجَنَّةِ عَدْنٍ, وَالْمُدُنُ الْخَرِبَةُ وَالْمُقْفِرَةُ وَالْمُنْهَدِمَةُ مُحَصَّنَةً مَعْمُورَةً.
\par 36 فَتَعْلَمُ الأُمَمُ الَّذِينَ تُرِكُوا حَوْلَكُمْ أَنِّي أَنَا الرَّبُّ, بَنَيْتُ الْمُنْهَدِمَةَ وَغَرَسْتُ الْمُقْفِرَةَ. أَنَا الرَّبُّ تَكَلَّمْتُ وَسَأَفْعَلُ.
\par 37 بَعْدَ هَذِهِ أُطْلَبُ مِنْ بَيْتِ إِسْرَائِيلَ لأَفْعَلَ لَهُمْ. أُكَثِّرُهُمْ كَغَنَمِ أُنَاسٍ.
\par 38 كَغَنَمِ مَقْدِسٍ, كَغَنَمِ أُورُشَلِيمَ فِي مَوَاسِمِهَا, فَتَكُونُ الْمُدُنُ الْخَرِبَةُ مَلآنَةً غَنَمَ أُنَاسٍ, فَيَعْلَمُونَ أَنِّي أَنَا الرَّبُّ].

\chapter{37}

\par 1 كَانَتْ عَلَيَّ يَدُ الرَّبِّ فَأَخْرَجَني بِرُوحِ الرَّبِّ وَأَنْزَلَنِي فِي وَسَطِ الْبُقْعَةِ, وَهِيَ مَلآنَةٌ عِظَاماً.
\par 2 وَأَمَرَّنِي عَلَيْهَا مِنْ حَوْلِهَا وَإِذَا هِيَ كَثِيرَةٌ جِدّاً عَلَى وَجْهِ الْبُقْعَةِ, وَإِذَا هِيَ يَابِسَةٌ جِدّاً.
\par 3 فَقَالَ لِي: [يَا ابْنَ آدَمَ, أَتَحْيَا هَذِهِ الْعِظَامُ؟» فَقُلْتُ: [يَا سَيِّدُ الرَّبُّ أَنْتَ تَعْلَمُ».
\par 4 فَقَالَ لِي: [تَنَبَّأْ عَلَى هَذِهِ الْعِظَامِ وَقُلْ لَهَا: أَيَّتُهَا الْعِظَامُ الْيَابِسَةُ, اسْمَعِي كَلِمَةَ الرَّبِّ.
\par 5 هَكذَا قَالَ السَّيِّدُ الرَّبُّ لِهَذِهِ الْعِظَامِ: هَئَنَذَا أُدْخِلُ فِيكُمْ رُوحاً فَتَحْيُونَ.
\par 6 وَأَضَعُ عَلَيْكُمْ عَصَباً وأَكْسِيكُمْ لَحْماً وَأَبْسُطُ عَلَيْكُمْ جِلْداً وَأَجْعَلُ فِيكُمْ رُوحاً فَتَحْيُونَ وَتَعْلَمُونَ أَنِّي أَنَا الرَّبُّ].
\par 7 فَتَنَبَّأْتُ كمَا أُمِرتُ. وَبَيْنَمَا أَنَا أَتنَبَّأُ كَانَ صَوْتٌ وَإِذَا رَعْشٌ فَتَقَارَبَتِ الْعِظَامُ كُلُّ عَظْمٍ إِلَى عَظْمِهِ.
\par 8 ونَظَرْتُ وَإِذَا بِالْعَصَبِ وَاللَّحْمِ كَسَاهَا, وبُسِطَ الْجِلْدُ علَيْهَا مِنْ فَوْقُ, وَلَيْسَ فِيهَا رُوحٌ.
\par 9 فَقَالَ لِي: [تَنَبَّأْ لِلرُّوحِ, تَنَبَّأْ يَا ابْنَ آدَمَ, وَقُلْ لِلرُّوحِ: هَكذَا قَالَ السَّيِّدُ الرَّبُّ: هَلُمَّ يَا رُوحُ مِنَ الرِّيَاحِ الأَرْبَعِ وَهُبَّ عَلَى هَؤُلاَءِ الْقَتْلَى لِيَحْيُوا».
\par 10 فَتَنَبَّأْتُ كَمَا أَمَرَني, فَدَخَلَ فِيهِمِ الرُّوحُ, فَحَيُوا وَقَامُوا عَلَى أَقدَامِهِمْ جَيْشٌ عَظيمٌ جِدّاً جِدّاً.
\par 11 ثُمَّ قَالَ لِي: [يَا ابْنَ آدَمَ, هَذِهِ العِظَامُ هِيَ كُلُّ بَيتِ إِسْرَائِيلَ. هَا هُمْ يَقُولُونَ: يَبِسَتْ عِظَامُنَا وَهَلَكَ رَجَاؤُنَا. قَدِ انْقَطَعْنَا.
\par 12 لِذَلِكَ تَنَبَّأْ وَقُلْ لَهُمْ: هَكذَا قَالَ السَّيِّدُ الرَّبُّ: هَئَنَذَا أَفتَحُ قُبُورَكُمْ وأُصْعِدُكُمْ مِنْ قُبُورِكُمْ يَا شَعْبِي وَآتِي بِكُمْ إِلَى أَرْضِ إِسْرَائِيلَ.
\par 13 فَتَعْلَمُونَ أَنِّي أَنَا الرَّبُّ عِنْدَ فَتْحِي قُبُورَكُمْ وَإِصْعَادِي إِيَّاكُمْ مِنْ قُبُورِكُمْ يَا شَعْبِي.
\par 14 وأَجْعَلُ رُوحِي فِيكُمْ فتَحْيُونَ, وَأَجْعَلُكُمْ فِي أَرْضِكُمْ, فَتَعْلَمُونَ أَنِّي أنَا الرَّبُّ تَكَلَّمْتُ وَأَفْعَلُ, يَقُولُ الرَّبُّ».
\par 15 وَكَانَ إِلَيَّ كَلاَمُ الرَّبِّ:
\par 16 [وَأَنْتَ يَا ابْنَ آدَمَ, خُذْ لِنَفْسِكَ عَصاً وَاحِدَةً وَاكْتُبْ عَلَيْهَا: لِيَهُوذَا وَلِبَنِي إِسْرَائِيلَ رُفَقَائِهِ. وَخُذْ عَصاً أُخْرَى وَاكْتُبْ عَلَيْهَا: لِيُوسُفَ عَصَا أَفْرَايِمَ وَكُلِّ بَيْتِ إِسْرَائِيلَ رُفَقَائِهِ.
\par 17 وَاقْرِنْهُمَا الْوَاحِدَةَ بِالأُخْرَى كَعَصاً وَاحِدَةٍ, فَتَصِيرَا وَاحِدَةً فِي يَدِكَ.
\par 18 فَإِذَا سَأَلَكَ أَبْنَاءُ شَعْبِكَ: أَمَا تُخْبِرُنَا مَا لَكَ وَهَذَا؟
\par 19 فَقُلْ لَهُمْ: هَكَذَا قَالَ السَّيِّدُ الرَّبُّ: هَئَنَذَا آخُذُ عَصَا يُوسُفَ الَّتِي فِي يَدِ أَفْرَايِمَ, وَأَسْبَاطَ إِسْرَائِيلَ رُفَقَاءَهُ, وَأَضُمُّ إِلَيْهَا عَصَا يَهُوذَا, وَأَجْعَلُهُمْ عَصاً وَاحِدَةً فَيَصِيرُونَ وَاحِدَةً فِي يَدِي.
\par 20 وَتَكُونُ الْعَصَوَانِ اللَّتَانِ كَتَبْتَ عَلَيْهِمَا فِي يَدِكَ أَمَامَ أَعْيُنِهِمْ.
\par 21 وَقُلْ لَهُمْ: هَكَذَا قَالَ السَّيِّدُ الرَّبُّ: هَئَنَذَا آخُذُ بَنِي إِسْرَائِيلَ مِنْ بَيْنِ الأُمَمِ الَّتِي ذَهَبُوا إِلَيْهَا, وَأَجْمَعُهُمْ مِنْ كُلِّ نَاحِيَةٍ, وَآتِي بِهِمْ إِلَى أَرْضِهِمْ.
\par 22 وَأُصَيِّرُهُمْ أُمَّةً وَاحِدَةً فِي الأَرْضِ عَلَى جِبَالِ إِسْرَائِيلَ, وَمَلِكٌ وَاحِدٌ يَكُونُ مَلِكاً عَلَيْهِمْ كُلِّهِمْ, وَلاَ يَكُونُونَ بَعْدُ أُمَّتَيْنِ, وَلاَ يَنْقَسِمُونَ بَعْدُ إِلَى مَمْلَكَتَيْنِ.
\par 23 وَلاَ يَتَنَجَّسُونَ بَعْدُ بِأَصْنَامِهِمْ وَلاَ بِرَجَاسَاتِهِمْ وَلاَ بِشَيْءٍ مِنْ مَعَاصِيهِمْ, بَلْ أُخَلِّصُهُمْ مِنْ كُلِّ مَسَاكِنِهِمِ الَّتِي فِيهَا أَخْطَأُوا وَأُطَهِّرُهُمْ فَيَكُونُونَ لِي شَعْباً وَأَنَا أَكُونُ لَهُمْ إِلَهاً.
\par 24 وَدَاوُدُ عَبْدِي يَكُونُ مَلِكاً عَلَيْهِمْ, وَيَكُونُ لِجَمِيعِهِمْ رَاعٍ وَاحِدٌ, فَيَسْلُكُونَ فِي أَحْكَامِي وَيَحْفَظُونَ فَرَائِضِي وَيَعْمَلُونَ بِهَا.
\par 25 وَيَسْكُنُونَ فِي الأَرْضِ الَّتِي أَعْطَيْتُ عَبْدِي يَعْقُوبَ إِيَّاهَا, الَّتِي سَكَنَهَا آبَاؤُكُمْ, وَيَسْكُنُونَ فِيهَا هُمْ وَبَنُوهُمْ وَبَنُو بَنِيهِمْ إِلَى الأَبَدِ, وَعَبْدِي دَاوُدُ رَئِيسٌ عَلَيْهِمْ إِلَى الأَبَدِ.
\par 26 وَأَقْطَعُ مَعَهُمْ عَهْدَ سَلاَمٍ, فَيَكُونُ مَعَهُمْ عَهْداً مُؤَبَّداً, وَأُقِرُّهُمْ وَأُكَثِّرُهُمْ وَأَجْعَلُ مَقْدِسِي فِي وَسَطِهِمْ إِلَى الأَبَدِ.
\par 27 وَيَكُونُ مَسْكَنِي فَوْقَهُمْ, وَأَكُونُ لَهُمْ إِلَهاً وَيَكُونُونَ لِي شَعْباً.
\par 28 فَتَعْلَمُ الأُمَمُ أَنِّي أَنَا الرَّبُّ مُقَدِّسُ إِسْرَائِيلَ, إِذْ يَكُونُ مَقْدِسِي فِي وَسَطِهِمْ إِلَى الأَبَدِ].

\chapter{38}

\par 1 وَكَانَ إِلَيَّ كَلاَمُ الرَّبِّ:
\par 2 [يَا ابْنَ آدَمَ, اجْعَلْ وَجْهَكَ عَلَى جُوجٍ أَرْضِ مَاجُوجَ رَئِيسِ رُوشٍ مَاشِكَ وَتُوبَالَ وَتَنَبَّأْ عَلَيْهِ
\par 3 وَقُلْ: هَكَذَا قَالَ السَّيِّدُ الرَّبُّ: هَئَنَذَا عَلَيْكَ يَا جُوجُ رَئِيسُ رُوشٍ مَاشِكَ وَتُوبَالَ.
\par 4 وَأُرْجِعُكَ, وَأَضَعُ شَكَائِمَ فِي فَكَّيْكَ, وَأُخْرِجُكَ أَنْتَ وَكُلَّ جَيْشِكَ خَيْلاً وَفُرْسَاناً كُلَّهُمْ لاَبِسِينَ أَفْخَرَ لِبَاسٍ, جَمَاعَةً عَظِيمَةً مَعَ أَتْرَاسٍ وَمَجَانَّ, كُلَّهُمْ مُمْسِكِينَ السُّيُوفَ
\par 5 فَارِسَ وَكُوشَ وَفُوطَ مَعَهُمْ, كُلَّهُمْ بِمِجَنٍّ وَخُوذَةٍ,
\par 6 وَجُومَرَ وَكُلَّ جُيُوشِهِ, وَبَيْتَ تُوجَرْمَةَ مِنْ أَقَاصِي الشِّمَالِ مَعَ كُلِّ جَيْشِهِ, شُعُوباً كَثِيرِينَ مَعَكَ.
\par 7 اسْتَعِدَّ وَهَيِّئْ لِنَفْسِكَ أَنْتَ وَكُلُّ جَمَاعَاتِكَ الْمُجْتَمِعَةِ إِلَيْكَ فَصِرْتَ لَهُمْ مُوَقَّراً.
\par 8 بَعْدَ أَيَّامٍ كَثِيرَةٍ تُفْتَقَدُ. فِي السِّنِينَ الأَخِيرَةِ تَأْتِي إِلَى الأَرْضِ الْمُسْتَرَدَّةِ مِنَ السَّيْفِ الْمَجْمُوعَةِ مِنْ شُعُوبٍ كَثِيرَةٍ عَلَى جِبَالِ إِسْرَائِيلَ الَّتِي كَانَتْ دَائِمَةً خَرِبَةً, لِلَّذِينَ أُخْرِجُوا مِنَ الشُّعُوبِ وَسَكَنُوا آمِنِينَ كُلُّهُمْ.
\par 9 وَتَصْعَدُ وَتَأْتِي كَزَوْبَعَةٍ, وَتَكُونُ كَسَحَابَةٍ تُغَشِّي الأَرْضَ أَنْتَ وَكُلُّ جُيُوشِكَ وَشُعُوبٌ كَثِيرُونَ مَعَكَ.
\par 10 وَيَكُونُ فِي ذَلِكَ الْيَوْمِ أَنَّ أُمُوراً تَخْطُرُ بِبَالِكَ فَتُفَكِّرُ فِكْراً رَدِيئاً,
\par 11 وَتَقُولُ: إِنِّي أَصْعَدُ عَلَى أَرْضٍ أَعْرَاءٍ. آتِي الْهَادِئِينَ السَّاكِنِينَ فِي أَمْنٍ, كُلُّهُمْ سَاكِنُونَ بِغَيْرِ سُورٍ وَلَيْسَ لَهُمْ عَارِضَةٌ وَلاَ مَصَارِيعُ
\par 12 لِسَلْبِ السَّلْبِ وَلِغُنْمِ الْغَنِيمَةِ, لِرَدِّ يَدِكَ عَلَى خِرَبٍ مَعْمُورَةٍ وَعَلَى شَعْبٍ مَجْمُوعٍ مِنَ الأُمَمِ, الْمُقْتَنِي مَاشِيَةً وَقُنْيَةً, السَّاكِنُ فِي أَعَالِي الأَرْضِ.
\par 13 شَبَا وَدَدَانُ وَتُجَّارُ تَرْشِيشَ وَكُلُّ أَشْبَالِهَا يَسْأَلُونَكَ: هَلْ لِسَلْبِ سَلْبٍ أَنْتَ جَاءٍ؟ هَلْ لِغُنْمِ غَنِيمَةٍ جَمَعْتَ جَمَاعَتَكَ, لِحَمْلِ الْفِضَّةِ وَالذَّهَبِ, لأَخْذِ الْمَاشِيَةِ وَالْقُنْيَةِ, لِنَهْبِ نَهْبٍ عَظِيمٍ؟
\par 14 [لِذَلِكَ تَنَبَّأْ يَا ابْنَ آدَمَ وَقُلْ لِجُوجٍ: هَكَذَا قَالَ السَّيِّدُ الرَّبُّ: فِي ذَلِكَ الْيَوْمِ عِنْدَ سُكْنَى شَعْبِي إِسْرَائِيلَ آمِنِينَ, أَفَلاَ تَعْلَمُ؟
\par 15 وَتَأْتِي مِنْ مَوْضِعِكَ مِنْ أَقَاصِي الشِّمَالِ أَنْتَ وَشُعُوبٌ كَثِيرُونَ مَعَكَ, كُلُّهُمْ رَاكِبُونَ خَيْلاً جَمَاعَةٌ عَظِيمَةٌ وَجَيْشٌ كَثِيرٌ.
\par 16 وَتَصْعَدُ عَلَى شَعْبِي إِسْرَائِيلَ كَسَحَابَةٍ تُغَشِّي الأَرْضَ. فِي الأَيَّامِ الأَخِيرَةِ يَكُونُ. وَآتِي بِكَ عَلَى أَرْضِي لِتَعْرِفَنِي الأُمَمُ, حِينَ أَتَقَدَّسُ فِيكَ أَمَامَ أَعْيُنِهِمْ يَا جُوجُ].
\par 17 هَكَذَا قَالَ السَّيِّدُ الرَّبُّ: [هَلْ أَنْتَ هُوَ الَّذِي تَكَلَّمْتُ عَنْهُ فِي الأَيَّامِ الْقَدِيمَةِ عَنْ يَدِ عَبِيدِي أَنْبِيَاءِ إِسْرَائِيلَ, الَّذِينَ تَنَبَّأُوا فِي تِلْكَ الأَيَّامِ سِنِيناً أَنْ آتِيَ بِكَ عَلَيْهِمْ؟
\par 18 وَيَكُونُ فِي ذَلِكَ الْيَوْمِ, يَوْمَ مَجِيءِ جُوجٍ عَلَى أَرْضِ إِسْرَائِيلَ يَقُولُ السَّيِّدُ الرَّبُّ, أَنَّ غَضَبِي يَصْعَدُ فِي أَنْفِي.
\par 19 وَفِي غَيْرَتِي فِي نَارِ سَخَطِي تَكَلَّمْتُ, أَنَّهُ فِي ذَلِكَ الْيَوْمِ يَكُونُ رَعْشٌ عَظِيمٌ فِي أَرْضِ إِسْرَائِيلَ.
\par 20 فَتَرْعَشُ أَمَامِي سَمَكُ الْبَحْرِ وَطُيُورُ السَّمَاءِ وَوُحُوشُ الْحَقْلِ وَالدَّابَّاتُ الَّتِي تَدِبُّ عَلَى الأَرْضِ, وَكُلُّ النَّاسِ الَّذِينَ عَلَى وَجْهِ الأَرْضِ, وَتَنْدَكُّ الْجِبَالُ وَتَسْقُطُ الْمَعَاقِلُ وَتَسْقُطُ كُلُّ الأَسْوَارِ إِلَى الأَرْضِ.
\par 21 وَأَسْتَدْعِي السَّيْفَ عَلَيْهِ فِي كُلِّ جِبَالِي يَقُولُ السَّيِّدُ الرَّبُّ. فَيَكُونُ سَيْفُ كُلِّ وَاحِدٍ عَلَى أَخِيهِ.
\par 22 وَأُعَاقِبُهُ بِالْوَبَإِ وَبِالدَّمِ, وَأُمْطِرُ عَلَيْهِ وَعَلَى جَيْشِهِ وَعَلَى الشُّعُوبِ الْكَثِيرَةِ الَّذِينَ مَعَهُ مَطَراً جَارِفاً وَحِجَارَةَ بَرَدٍ عَظِيمَةً وَنَاراً وَكِبْرِيتاً.
\par 23 فَأَتَعَظَّمُ وَأَتَقَدَّسُ وَأُعْرَفُ فِي عُيُونِ أُمَمٍ كَثِيرَةٍ, فَيَعْلَمُونَ أَنِّي أَنَا الرَّبُّ».

\chapter{39}

\par 1 [وَأَنْتَ يَا ابْنَ آدَمَ تَنَبَّأْ عَلَى جُوجٍ وَقُلْ: هَكَذَا قَالَ السَّيِّدُ الرَّبُّ: هَئَنَذَا عَلَيْكَ يَا جُوجُ رَئِيسُ رُوشٍ مَاشِكَ وَتُوبَالَ.
\par 2 وَأَرُدُّكَ وَأَقُودُكَ وَأُصْعِدُكَ مِنْ أَقَاصِي الشِّمَالِ وَآتِي بِكَ عَلَى جِبَالِ إِسْرَائِيلَ.
\par 3 وَأَضْرِبُ قَوْسَكَ مِنْ يَدِكَ الْيُسْرَى, وَأُسْقِطُ سِهَامَكَ مِنْ يَدِكَ الْيُمْنَى.
\par 4 فَتَسْقُطُ عَلَى جِبَالِ إِسْرَائِيلَ أَنْتَ وَكُلُّ جَيْشِكَ وَالشُّعُوبُ الَّذِينَ مَعَكَ. أَبْذِلُكَ مَأْكَلاً لِلطُّيُورِ الْكَاسِرَةِ مِنْ كُلِّ نَوْعٍ وَلِوُحُوشِ الْحَقْلِ.
\par 5 عَلَى وَجْهِ الْحَقْلِ تَسْقُطُ لأَنِّي تَكَلَّمْتُ يَقُولُ السَّيِّدُ الرَّبُّ.
\par 6 وَأُرْسِلُ نَاراً عَلَى مَاجُوجَ وَعَلَى السَّاكِنِينَ فِي الْجَزَائِرِ آمِنِينَ, فَيَعْلَمُونَ أَنِّي أَنَا الرَّبُّ.
\par 7 وَأُعَرِّفُ بِاسْمِي الْمُقَدَّسِ فِي وَسَطِ شَعْبِي إِسْرَائِيلَ, وَلاَ أَدَعُ اسْمِي الْمُقَدَّسَ يُنَجَّسُ بَعْدُ, فَتَعْلَمُ الأُمَمُ أَنِّي أَنَا الرَّبُّ قُدُّوسُ إِسْرَائِيلَ
\par 8 [هَا هُوَ قَدْ أَتَى وَصَارَ يَقُولُ السَّيِّدُ الرَّبُّ. هَذَا هُوَ الْيَوْمُ الَّذِي تَكَلَّمْتُ عَنْهُ.
\par 9 وَيَخْرُجُ سُكَّانُ مُدُنِ إِسْرَائِيلَ وَيُشْعِلُونَ وَيُحْرِقُونَ السِّلاَحَ وَالْمَجَانَّ وَالأَتْرَاسَ وَالْقِسِيَّ وَالسِّهَامَ وَالْحِرَابَ وَالرِّمَاحَ, وَيُوقِدُونَ بِهَا النَّارَ سَبْعَ سِنِينَ.
\par 10 فَلاَ يَأْخُذُونَ مِنَ الْحَقْلِ عُوداً, وَلاَ يَحْتَطِبُونَ مِنَ الْوُعُورِ لأَنَّهُمْ يُحْرِقُونَ السِّلاَحَ بِالنَّارِ, وَيَنْهَبُونَ الَّذِينَ نَهَبُوهُمْ وَيَسْلُبُونَ الَّذِينَ سَلَبُوهُمْ, يَقُولُ السَّيِّدُ الرَّبُّ.
\par 11 وَيَكُونُ فِي ذَلِكَ الْيَوْمِ, أَنِّي أُعْطِي جُوجاً مَوْضِعاً هُنَاكَ لِلْقَبْرِ فِي إِسْرَائِيلَ, وَوَادِي عَبَارِيمَ بِشَرْقِيِّ الْبَحْرِ, فَيَسُدُّ نَفَسَ الْعَابِرِينَ. وَهُنَاكَ يَدْفِنُونَ جُوجاً وَجُمْهُورَهُ كُلَّهُ, وَيُسَمُّونَهُ [وَادِيَ جُمْهُورِ جُوجٍ».
\par 12 وَيَقْبِرُهُمْ بَيْتُ إِسْرَائِيلَ لِيُطَهِّرُوا الأَرْضَ سَبْعَةَ أَشْهُرٍ.
\par 13 كُلُّ شَعْبِ الأَرْضِ يَقْبِرُونَ, وَيَكُونُ لَهُمْ يَوْمُ تَمْجِيدِي مَشْهُوراً يَقُولُ السَّيِّدُ الرَّبُّ.
\par 14 وَيُفْرِزُونَ أُنَاساً مُسْتَدِيمِينَ عَابِرِينَ فِي الأَرْضِ, قَابِرِينَ مَعَ الْعَابِرِينَ أُولَئِكَ الَّذِينَ بَقُوا عَلَى وَجْهِ الأَرْضِ. تَطْهِيراً لَهَا. بَعْدَ سَبْعَةِ أَشْهُرٍ يَفْحَصُونَ
\par 15 فَيَعْبُرُ الْعَابِرُونَ فِي الأَرْضِ وَإِذَا رَأَى أَحَدٌ عَظْمَ إِنْسَانٍ يَبْنِي بِجَانِبِهِ صُوَّةً حَتَّى يَقْبِرَهُ الْقَابِرُونَ فِي وَادِي جُمْهُورِ جُوجٍ
\par 16 وَأَيْضاً اسْمُ الْمَدِينَةِ [هَمُونَةُ» فَيُطَهِّرُونَ الأَرْضَ.
\par 17 [وَأَنْتَ يَا ابْنَ آدَمَ, فَهَكَذَا قَالَ السَّيِّدُ الرَّبُّ: قُلْ لِطَائِرِ كُلِّ جَنَاحٍ, وَلِكُلِّ وُحُوشِ الْبَرِّ: اجْتَمِعُوا, وَتَعَالُوا احْتَشِدُوا مِنْ كُلِّ جِهَةٍ, إِلَى ذَبِيحَتِي الَّتِي أَنَا ذَابِحُهَا لَكُمْ, ذَبِيحَةً عَظِيمَةً عَلَى جِبَالِ إِسْرَائِيلَ لِتَأْكُلُوا لَحْماً وَتَشْرَبُوا دَماً.
\par 18 تَأْكُلُونَ لَحْمَ الْجَبَابِرَةِ وَتَشْرَبُونَ دَمَ رُؤَسَاءِ الأَرْضِ, كِبَاشٌ وَحُمْلاَنٌ وَأَعْتِدَةٌ وَثِيرَانٌ كُلُّهَا مِنْ مُسَمَّنَاتِ بَاشَانَ.
\par 19 وَتَأْكُلُونَ الشَّحْمَ إِلَى الشَّبَعِ, وَتَشْرَبُونَ الدَّمَ إِلَى السُّكْرِ مِنْ ذَبِيحَتِي الَّتِي ذَبَحْتُهَا لَكُمْ.
\par 20 فَتَشْبَعُونَ عَلَى مَائِدَتِي مِنَ الْخَيْلِ وَالْمَرْكَبَاتِ وَالْجَبَابِرَةِ وَكُلِّ رِجَالِ الْحَرْبِ, يَقُولُ السَّيِّدُ الرَّبُّ.
\par 21 وَأَجْعَلُ مَجْدِي فِي الأُمَمِ, وَجَمِيعُ الأُمَمِ يَرُونَ حُكْمِي الَّذِي أَجْرَيْتُهُ وَيَدِي الَّتِي جَعَلْتُهَا عَلَيْهِمْ,
\par 22 فَيَعْلَمُ بَيْتُ إِسْرَائِيلَ أَنِّي أَنَا الرَّبُّ إِلَهُهُمْ مِنْ ذَلِكَ الْيَوْمِ فَصَاعِداً.
\par 23 وَتَعْلَمُ الأُمَمُ أَنَّ بَيْتَ إِسْرَائِيلَ قَدْ أُجْلُوا بِإِثْمِهِمْ لأَنَّهُمْ خَانُونِي, فَحَجَبْتُ وَجْهِي عَنْهُمْ وَسَلَّمْتُهُمْ لِيَدِ مُضَايِقِيهِمْ, فَسَقَطُوا كُلُّهُمْ بِالسَّيْفِ.
\par 24 كَنَجَاسَتِهِمْ وَكَمَعَاصِيهِمْ فَعَلْتُ مَعَهُمْ وَحَجَبْتُ وَجْهِي عَنْهُمْ].
\par 25 لِذَلِكَ هَكَذَا قَالَ السَّيِّدُ الرَّبُّ: [الآنَ أَرُدُّ سَبْيَ يَعْقُوبَ وَأَرْحَمُ كُلَّ بَيْتِ إِسْرَائِيلَ وَأَغَارُ عَلَى اسْمِي الْقُدُّوسِ.
\par 26 فَيَحْمِلُونَ خِزْيَهُمْ وَكُلَّ خِيَانَتِهِمِ الَّتِي خَانُونِي إِيَّاهَا عِنْدَ سَكَنِهِمْ فِي أَرْضِهِمْ مُطْمَئِنِّينَ وَلاَ مُخِيفٌ.
\par 27 عِنْدَ إِرْجَاعِي إِيَّاهُمْ مِنَ الشُّعُوبِ وَجَمْعِي إِيَّاهُمْ مِنْ أَرَاضِي أَعْدَائِهِمْ, وَتَقْدِيسِي فِيهِمْ أَمَامَ عُيُونِ أُمَمٍ كَثِيرِينَ,
\par 28 يَعْلَمُونَ أَنِّي أَنَا الرَّبُّ إِلَهُهُمْ بِإِجْلاَئِي إِيَّاهُمْ إِلَى الأُمَمِ, ثُمَّ جَمْعِهِمْ إِلَى أَرْضِهِمْ. وَلاَ أَتْرُكُ بَعْدُ هُنَاكَ أَحَداً مِنْهُمْ,
\par 29 وَلاَ أَحْجُبُ وَجْهِي عَنْهُمْ بَعْدُ, لأَنِّي سَكَبْتُ رُوحِي عَلَى بَيْتِ إِسْرَائِيلَ يَقُولُ السَّيِّدُ الرَّبُّ].

\chapter{40}

\par 1 فِي السَّنَةِ الْخَامِسَةِ وَالْعِشْرِينَ مِنْ سَبْيِنَا فِي رَأْسِ السَّنَةِ, فِي الْعَاشِرِ مِنَ الشَّهْرِ فِي السَّنَةِ الرَّابِعَةِ عَشَرَةَ, بَعْدَ مَا ضُرِبَتِ الْمَدِينَةُ فِي نَفْسِ ذَلِكَ الْيَوْمِ, كَانَتْ عَلَيَّ يَدُ الرَّبِّ وَأَتَى بِي إِلَى هُنَاكَ.
\par 2 فِي رُؤَى اللَّهِ أَتَى بِي إِلَى أَرْضِ إِسْرَائِيلَ وَوَضَعَنِي عَلَى جَبَلٍ عَالٍ جِدّاً, عَلَيْهِ كَبِنَاءِ مَدِينَةٍ مِنْ جِهَةِ الْجَنُوبِ.
\par 3 وَلَمَّا أَتَى بِي إِلَى هُنَاكَ إِذَا بِرَجُلٍ مَنْظَرُهُ كَمَنْظَرِ النُّحَاسِ, وَبِيَدِهِ خَيْطُ كَتَّانٍ وَقَصَبَةُ الْقِيَاسِ, وَهُوَ وَاقِفٌ بِالْبَابِ.
\par 4 فَقَالَ لِي الرَّجُلُ: [يَا ابْنَ آدَمَ, انْظُرْ بِعَيْنَيْكَ وَاسْمَعْ بِأُذُنَيْكَ وَاجْعَلْ قَلْبَكَ إِلَى كُلِّ مَا أُرِيكَهُ, لأَنَّهُ لأَجْلِ إِرَاءَتِكَ أُتِيَ بِكَ إِلَى هُنَا. أَخْبِرْ بَيْتَ إِسْرَائِيلَ بِكُلِّ مَا تَرَى».
\par 5 وَإِذَا بِسُورٍ خَارِجَ الْبَيْتِ مُحِيطٍ بِهِ, وَبِيَدِ الرَّجُلِ قَصَبَةُ الْقِيَاسِ سِتُّ أَذْرُعٍ طُولاً بِالذِّرَاعِ وَشِبْرٌ. فَقَاسَ عَرْضَ الْبِنَاءِ قَصَبَةً وَاحِدَةً وَسُمْكَهُ قَصَبَةً وَاحِدَةً.
\par 6 ثُمَّ جَاءَ إِلَى الْبَابِ الَّذِي وَجْهُهُ نَحْوَ الشَّرْقِ وَصَعِدَ فِي دَرَجِهِ, وَقَاسَ عَتَبَةَ الْبَابِ قَصَبَةً وَاحِدَةً عَرْضاً وَالْعَتَبَةَ الأُخْرَى قَصَبَةً وَاحِدَةً عَرْضاً.
\par 7 وَالْغُرْفَةَ قَصَبَةً وَاحِدَةً طُولاً وَقَصَبَةً وَاحِدَةً عَرْضاً, وَبَيْنَ الْغُرُفَاتِ خَمْسُ أَذْرُعٍ. وَعَتَبَةُ الْبَابِ بِجَانِبِ رِوَاقِ الْبَابِ مِنْ دَاخِلٍ قَصَبَةٌ وَاحِدَةٌ.
\par 8 وَقَاسَ رِوَاقَ الْبَابِ مِنْ دَاخِلٍ قَصَبَةً وَاحِدَةً.
\par 9 وَقَاسَ رِوَاقَ الْبَابِ ثَمَانِيَ أَذْرُعٍ, وَعَضَائِدَهُ ذِرَاعَيْنِ, وَرُِوَاقُ الْبَابِ مِنْ دَاخِلٍ.
\par 10 وَغُرُفَاتُ الْبَابِ نَحْوَ الشَّرْقِ ثَلاَثٌ مِنْ هُنَا وَثَلاَثٌ مِنْ هُنَاكَ. لِلثَّلاَثِ قِيَاسٌ وَاحِدٌ, وَلِلْعَضَائِدِ قِيَاسٌ وَاحِدٌ مِنْ هُنَا وَمِنْ هُنَاكَ.
\par 11 وَقَاسَ عَرْضَ مَدْخَلِ الْبَابِ عَشَرَ أَذْرُعٍ, وَطُولَ الْبَابِ ثَلاَثَ عَشَرَةَ ذِرَاعاً.
\par 12 وَالْحَافَّةُ أَمَامَ الْغُرُفَاتِ ذِرَاعٌ وَاحِدَةٌ مِنْ هُنَا وَالْحَافَّةُ ذِرَاعٌ وَاحِدَةٌ مِنْ هُنَاكَ. وَالْغُرْفَةُ سِتُّ أَذْرُعٍ مِنْ هُنَا وَسِتُّ أَذْرُعٍ مِنْ هُنَاكَ.
\par 13 ثُمَّ قَاسَ الْبَابَ مِنْ سَقْفِ الْغُرْفَةِ الْوَاحِدَةِ إِلَى سَقْفِ الأُخْرَى عَرْضَ خَمْسٍ وَعِشْرِينَ ذِرَاعاً. الْبَابُ مُقَابِلُ الْبَابِ.
\par 14 وَعَمِلَ عَضَائِدَ سِتِّينَ ذِرَاعاً إِلَى عَضَادَةِ الدَّارِ حَوْلَ الْبَابِ.
\par 15 وَقُدَّامَ بَابِ الْمَدْخَلِ إِلَى قُدَّامِ رِوَاقِ الْبَابِ الدَّاخِلِيِّ خَمْسُونَ ذِرَاعاً.
\par 16 وَلِلْغُرُفَاتِ كُوًى مُشَبَّكَةٌ, وَلِلْعَضَائِدِ مِنْ دَاخِلِ الْبَابِ حَوَالَيْهِ, وَهَكَذَا فِي الْقُبَبِ أَيْضاً, كُوًى حَوَالَيْهَا مِنْ دَاخِلٍ, وَعَلَى الْعَضَادَةِ نَخِيلٌ.
\par 17 ثُمَّ أَتَى بِي إِلَى الدَّارِ الْخَارِجِيَّةِ وَإِذَا بِمَخَادِعَ وَمُجَزَّعٍ مَصْنُوعٍ لِلدَّارِ حَوَالَيْهَا. عَلَى الْمُجَزَّعِ ثَلاَثُونَ مِخْدَعاً.
\par 18 وَالْمُجَزَّعُ بِجَانِبِ الأَبْوَابِ مُقَابِلَ طُولِ الأَبْوَابِ الْمُجَزَّعُ الأَسْفَلُ.
\par 19 وَقَاسَ الْعَرْضَ مِنْ قُدَّامِ الْبَابِ الأَسْفَلِ إِلَى قُدَّامِ الدَّارِ الدَّاخِلِيَّةِ مِنْ خَارِجٍ مِئَةَ ذِرَاعٍ إِلَى الشَّرْقِ وَإِلَى الشِّمَالِ.
\par 20 وَالْبَابُ الْمُتَّجِهُ نَحْوَ الشِّمَالِ الَّذِي لِلدَّارِ الْخَارِجِيَّةِ قَاسَ طُولَهُ وَعَرْضَهُ.
\par 21 وَغُرُفَاتُهُ ثَلاَثٌ مِنْ هُنَا وَثَلاَثٌ مِنْ هُنَاكَ, وَعَضَائِدُهُ وَمُقَبَّبُهُ كَانَتْ عَلَى قِيَاسِ الْبَابِ الأَوَّلِ, طُولُهَا خَمْسُونَ ذِرَاعاً وَعَرْضُهَا خَمْسٌ وَعِشْرُونَ ذِرَاعاً.
\par 22 وَكُواهَا وَمُقَبَّبُهَا وَنَخِيلُهَا عَلَى قِيَاسِ الْبَابِ الْمُتَّجِهِ نَحْوَ الشَّرْقِ, وَكَانُوا يَصْعَدُونَ إِلَيْهِ فِي سَبْعِ دَرَجَاتٍ, وَمُقَبَّبُهُ أَمَامَهُ.
\par 23 وَلِلدَّارِ الدَّاخِلِيَّةِ بَابٌ مُقَابِلُ بَابٍ لِلشِّمَالِ وَلِلشَّرْقِ. وَقَاسَ مِنْ بَابٍ إِلَى بَابٍ مِئَةَ ذِرَاعٍ.
\par 24 ثُمَّ ذَهَبَ بِي نَحْوَ الْجَنُوبِ, وَإِذَا بِبَابٍ نَحْوَ الْجَنُوبِ, فَقَاسَ عَضَائِدَهُ وَمُقَبَّبَهُ كَهَذِهِ الأَقْيِسَةِ.
\par 25 وَفِيهِ كُوًى وَفِي مُقَبَّبِهِ مِنْ حَوَالَيْهِ كَتِلْكَ الْكُوَى. الطُّولُ خَمْسُونَ ذِرَاعاً وَالْعَرْضُ خَمْسٌ وَعِشْرُونَ ذِرَاعاً.
\par 26 وَسَبْعُ دَرَجَاتٍ مَصْعَدُهُ وَمُقَبَّبُهُ قُدَّامَهُ, وَلَهُ نَخِيلٌ وَاحِدَةٌ مِنْ هُنَا وَوَاحِدَةٌ مِنْ هُنَاكَ عَلَى عَضَائِدِهِ.
\par 27 وَلِلدَّارِ الدَّاخِلِيَّةِ بَابٌ نَحْوَ الْجَنُوبِ. وَقَاسَ مِنَ الْبَابِ إِلَى الْبَابِ نَحْوَ الْجَنُوبِ مِئَةَ ذِرَاعٍ.
\par 28 وَأَتَى بِي إِلَى الدَّارِ الدَّاخِلِيَّةِ مِنْ بَابِ الْجَنُوبِ, وَقَاسَ بَابَ الْجَنُوبِ كَهَذِهِ الأَقْيِسَةِ.
\par 29 وَغُرُفَاتُهُ وَعَضَائِدُهُ وَمُقَبَّبُهُ كَهَذِهِ الأَقْيِسَةِ. وَفِيهِ وَفِي مُقَبَّبِهِ كُوًى حَوَالَيْهِ. الطُّولُ خَمْسُونَ ذِرَاعاً وَالْعَرْضُ خَمْسٌ وَعِشْرُونَ ذِرَاعاً.
\par 30 وَحَوَالَيْهِ مُقَبَّبٌ خَمْسٌ وَعِشْرُونَ ذِرَاعاً طُولاً وَخَمْسُ أَذْرُعٍ عَرْضاً.
\par 31 وَمُقَبَّبُهُ نَحْوَ الدَّارِ الْخَارِجِيَّةِ, وَعَلَى عَضَائِدِهِ نَخِيلٌ, وَمَصْعَدُهُ ثَمَانِي دَرَجَاتٍ.
\par 32 وَأَتَى بِي إِلَى الدَّارِ الدَّاخِلِيَّةِ نَحْوَ الْمَشْرِقِ وَقَاسَ الْبَابَ كَهَذِهِ الأَقْيِسَةِ.
\par 33 وَغُرُفَاتُهُ وَعَضَائِدُهُ وَمُقَبَّبُهُ كَهَذِهِ الأَقْيِسَةِ. وَفِيهِ وَفِي مُقَبَّبِهِ كُوًى حَوَالَيْهِ. الطُّولُ خَمْسُونَ ذِرَاعاً وَالْعَرْضُ خَمْسٌ وَعِشْرُونَ ذِرَاعاً.
\par 34 وَمُقَبَّبُهُ نَحْوَ الدَّارِ الْخَارِجِيَّةِ وَعَلَى عَضَائِدِهِ نَخِيلٌ مِنْ هُنَا وَمِنْ هُنَاكَ, وَمَصْعَدُهُ ثَمَانِي دَرَجَاتٍ.
\par 35 وَأَتَى بِي إِلَى بَابِ الشِّمَالِ وَقَاسَ كَهَذِهِ الأَقْيِسَةِ.
\par 36 غُرُفَاتُهُ وَعَضَائِدُهُ وَمُقَبَّبُهُ وَالْكُوى الَّتِي لَهُ حَوَالَيْهِ. الطُّولُ خَمْسُونَ ذِرَاعاً وَالْعَرْضُ خَمْسٌ وَعِشْرُونَ ذِرَاعاً.
\par 37 وَعَضَائِدُهُ نَحْوَ الدَّارِ الْخَارِجِيَّةِ, وَعَلَى عَضَائِدِهِ نَخِيلٌ مِنْ هُنَا وَمِنْ هُنَاكَ, وَمَصْعَدُهُ ثَمَانِي دَرَجَاتٍ.
\par 38 وَعِنْدَ عَضَائِدِ الأَبْوَابِ مِخْدَعٌ وَمَدْخَلُهُ. هُنَاكَ يَغْسِلُونَ الْمُحْرَقَةَ.
\par 39 وَفِي رِوَاقِ الْبَابِ مَائِدَتَانِ مِنْ هُنَا وَمَائِدَتَانِ مِنْ هُنَاكَ لِتُذْبَحَ عَلَيْهَا الْمُحْرَقَةُ وَذَبِيحَةُ الْخَطِيئَةِ وَذَبِيحَةُ الإِثْمِ.
\par 40 وَعَلَى الْجَانِبِ مِنْ خَارِجٍ حَيْثُ يُصْعَدُ إِلَى مَدْخَلِ بَابِ الشِّمَالِ مَائِدَتَانِ, وَعَلَى الْجَانِبِ الآخَرِ الَّذِي لِرِوَاقِ الْبَابِ مَائِدَتَانِ.
\par 41 أَرْبَعُ مَوَائِدَ مِنْ هُنَا وَأَرْبَعُ مَوَائِدَ مِنْ هُنَاكَ عَلَى جَانِبِ الْبَابِ. ثَمَانِي مَوَائِدَ كَانُوا يَذْبَحُونَ عَلَيْهَا.
\par 42 وَالْمَوَائِدُ الأَرْبَعُ لِلْمُحْرَقَةِ مِنْ حَجَرٍ نَحِيتٍ, الطُّولُ ذِرَاعٌ وَنِصْفٌ وَالْعَرْضُ ذِرَاعٌ وَنِصْفٌ وَالسَّمْكُ ذِرَاعٌ وَاحِدَةٌ. كَانُوا يَضَعُونَ عَلَيْهَا الأَدَوَاتِ الَّتِي يَذْبَحُونَ بِهَا الْمُحْرَقَةَ وَالذَّبِيحَةَ.
\par 43 وَالْمَآزِيبُ شِبْرٌ وَاحِدٌ مُمَكَّنَةً فِي الْبَيْتِ مِنْ حَوْلِهِ. وَعَلَى الْمَوَائِدِ لَحْمُ الْقُرْبَانِ.
\par 44 وَمِنْ خَارِجِ الْبَابِ الدَّاخِلِيِّ مَخَادِعُ الْمُغَنِّينَ فِي الدَّارِ الدَّاخِلِيَّةِ الَّتِي بِجَانِبِ بَابِ الشِّمَالِ, وَوُجُوهُهَا نَحْوَ الْجَنُوبِ. وَاحِدٌ بِجَانِبِ بَابِ الشَّرْقِ مُتَّجِهٌ نَحْوَ الشِّمَالِ.
\par 45 وَقَالَ لِي: [هَذَا الْمِخْدَعُ الَّذِي وَجْهُهُ نَحْوَ الْجَنُوبِ هُوَ لِلْكَهَنَةِ حَارِسِي حِرَاسَةِ الْبَيْتِ.
\par 46 وَالْمِخْدَعُ الَّذِي وَجْهُهُ نَحْوَ الشِّمَالِ لِلْكَهَنَةِ حَارِسِي حِرَاسَةِ الْمَذْبَحِ. هُمْ بَنُو صَادُوقَ الْمُقَرَّبُونَ مِنْ بَنِي لاَوِي إِلَى الرَّبِّ لِيَخْدِمُوهُ».
\par 47 فَقَاسَ الدَّارَ مِئَةَ ذِرَاعٍ طُولاً وَمِئَةَ ذِرَاعٍ عَرْضاً, مُرَبَّعَةً, وَالْمَذْبَحَ أَمَامَ الْبَيْتِ.
\par 48 وَأَتَى بِي إِلَى رِوَاقِ الْبَيْتِ وَقَاسَ عَضَادَةَ الرِّوَاقِ, خَمْسَ أَذْرُعٍ مِنْ هُنَا وَخَمْسَ أَذْرُعٍ مِنْ هُنَاكَ, وَعَرْضَ الْبَابِ ثَلاَثَ أَذْرُعٍ مِنْ هُنَا وَثَلاَثَ أَذْرُعٍ مِنْ هُنَاكَ.
\par 49 طُولُ الرِّوَاقِ عِشْرُونَ ذِرَاعاً, وَالْعَرْضُ إِحْدَى عَشَرَةَ ذِرَاعاً عِنْدَ الدَّرَجِ الَّذِي بِهِ كَانُوا يَصْعَدُونَ إِلَيْهِ. وَعِنْدَ الْعَضَائِدِ أَعْمِدَةٌ, وَاحِدٌ مِنْ هُنَا وَوَاحِدٌ مِنْ هُنَاكَ.

\chapter{41}

\par 1 وَأَتَى بِي إِلَى الْهَيْكَلِ وَقَاسَ الْعَضَائِدَ, عَرْضُهَا مِنْ هُنَا سِتُّ أَذْرُعٍ وَمِنْ هُنَاكَ سِتُّ أَذْرُعٍ عَرْضُ الْخَيْمَةِ.
\par 2 وَعَرْضُ الْمَدْخَلِ عَشَرُ أَذْرُعٍ, وَجَوَانِبُ الْمَدْخَلِ مِنْ هُنَا خَمْسُ أَذْرُعٍ وَمِنْ هُنَاكَ خَمْسُ أَذْرُعٍ. وَقَاسَ طُولَهُ أَرْبَعِينَ ذِرَاعاً وَالْعَرْضَ عِشْرِينَ ذِرَاعاً.
\par 3 ثُمَّ جَاءَ إِلَى دَاخِلٍ وَقَاسَ عَضَادَةَ الْمَدْخَلِ ذِرَاعَيْنِ, وَالْمَدْخَلَ سِتَّ أَذْرُعٍ, وَعَرْضَ الْمَدْخَلِ سَبْعَ أَذْرُعٍ.
\par 4 وَقَاسَ طُولَهُ عِشْرِينَ ذِرَاعاً, وَالْعَرْضَ عِشْرِينَ ذِرَاعاً إِلَى قُدَّامِ الْهَيْكَلِ. وَقَالَ لِي: [هَذَا قُدْسُ الأَقْدَاسِ».
\par 5 وَقَاسَ حَائِطَ الْبَيْتِ سِتَّ أَذْرُعٍ, وَعَرْضَ الْغُرْفَةِ أَرْبَعَ أَذْرُعٍ حَوْلَ الْبَيْتِ مِنْ كُلِّ جِهَةٍ.
\par 6 وَالْغُرُفَاتُ غُرْفَةٌ إِلَى غُرْفَةٍ ثَلاَثاً وَثَلاَثِينَ مَرَّةً, وَدَخَلَتْ فِي الْحَائِطِ الَّذِي لِلْبَيْتِ لِلْغُرُفَاتِ حَوْلَهُ لِتَتَمَكَّنَ وَلاَ تَتَمَكَّنَ فِي حَائِطِ الْبَيْتِ.
\par 7 وَاتَّسَعَتِ الْغُرُفَاتُ وَأَحَاطَتْ صَاعِداً فَصَاعِداً, لأَنَّ مُحِيطَ الْبَيْتِ كَانَ صَاعِداً فَصَاعِداً حَوْلَ الْبَيْتِ. لِذَلِكَ عَرْضُ الْبَيْتِ إِلَى فَوْقُ وَهَكَذَا مِنَ الأَسْفَلِ يُصْعَدُ إِلَى الأَعْلَى فِي الْوَسَطِ.
\par 8 وَرَأَيْتُ سَمْكَ الْبَيْتِ حَوَالَيْهِ. أُسُسُ الْغُرُفَاتِ قَصَبَةٌ تَامَّةٌ سِتُّ أَذْرُعٍ إِلَى الْمَفْصَلِ.
\par 9 عَرْضُ الْحَائِطِ الَّذِي لِلْغُرْفَةِ مِنْ خَارِجٍ خَمْسُ أَذْرُعٍ, وَمَا بَقِيَ فَفَسْحَةٌ لِغُرُفَاتِ الْبَيْتِ.
\par 10 وَبَيْنَ الْمَخَادِعِ عَرْضُ عِشْرِينَ ذِرَاعاً حَوْلَ الْبَيْتِ مِنْ كُلِّ جَانِبٍ.
\par 11 وَمَدْخَلُ الْغُرْفَةِ فِي الْفَسْحَةِ مَدْخَلٌ وَاحِدٌ نَحْوَ الشِّمَالِ, وَمَدْخَلٌ آخَرُ نَحْوَ الْجَنُوبِ. وَعَرْضُ مَكَانِ الْفَسْحَةِ خَمْسُ أَذْرُعٍ حَوَالَيْهِ.
\par 12 وَالْبِنَاءُ الَّذِي أَمَامَ الْمَكَانِ الْمُنْفَصِلِ عِنْدَ الطَّرَفِ نَحْوَ الْغَرْبِ سَبْعُونَ ذِرَاعاً عَرْضاً, وَحَائِطِ الْبِنَاءِ خَمْسُ أَذْرُعٍ عَرْضاً مِنْ حَوْلِهِ, وَطُولُهُ تِسْعُونَ ذِرَاعاً.
\par 13 وَقَاسَ الْبَيْتَ مِئَةَ ذِرَاعٍ طُولاً, وَالْمَكَانَ الْمُنْفَصِلَ وَالْبِنَاءَ مَعَ حِيطَانِهِ مِئَةَ ذِرَاعٍ طُولاً.
\par 14 وَعَرْضَ وَجْهَ الْبَيْتِ وَالْمَكَانِ الْمُنْفَصِلِ نَحْوَ الشَّرْقِ مِئَةَ ذِرَاعٍ.
\par 15 وَقَاسَ طُولَ الْبِنَاءِ إِلَى قُدَّامِ الْمَكَانِ الْمُنْفَصِلِ الَّذِي وَرَاءَهُ وَأَسَاطِينَهُ مِنْ جَانِبٍ إِلَى جَانِبٍ مِئَةَ ذِرَاعٍ مَعَ الْهَيْكَلِ الدَّاخِلِيِّ وَأَرْوِقَةِ الدَّارِ.
\par 16 الْعَتَبَاتُ وَالْكُوَى الْمُشَبَّكَةُ وَالأَسَاطِينُ حَوَالَيِ الطَّبَقَاتِ الثَّلاَثِ مُقَابِلُ الْعَتَبَةِ مِنْ أَلْوَاحِ خَشَبٍ مِنْ كُلِّ جَانِبٍ وَمِنَ الأَرْضِ إِلَى الْكُوَى - وَالْكُوَى مُغَطَّاةٌ -
\par 17 إِلَى مَا فَوْقَ الْمَدْخَلِ وَإِلَى الْبَيْتِ الدَّاخِلِيِّ وَإِلَى الْخَارِجِ وَإِلَى الْحَائِطِ كُلِّهِ حَوَالَيْهِ مِنْ دَاخِلٍ وَمِنْ خَارِجٍ بِهَذِهِ الأَقْيِسَةِ.
\par 18 وَعُمِلَ فِيهِ كَرُوبِيمُ وَنَخِيلٌ. نَخْلَةٌ بَيْنَ كَرُوبٍ وَكَرُوبٍ, وَلِكُلِّ كَرُوبٍ وَجْهَانِ.
\par 19 فَوَجْهُ الإِنْسَانِ نَحْوَ نَخْلَةٍ مِنْ هُنَا, وَوَجْهُ الشِّبْلِ نَحْوَ نَخْلَةٍ مِنْ هُنَالِكَ. عُمِلَ فِي كُلِّ الْبَيْتِ حَوَالَيْهِ.
\par 20 مِنَ الأَرْضِ إِلَى مَا فَوْقَ الْمَدْخَلِ عُمِلَ كَرُوبِيمُ وَنَخِيلٌ وَعَلَى حَائِطِ الْهَيْكَلِ.
\par 21 وَقَوَائِمُ الْهَيْكَلِ مُرَبَّعَةٌ, وَوَجْهُ الْقُدْسِ مَنْظَرُهُ كَمَنْظَرِ وَجْهِ الْهَيْكَلِ.
\par 22 اَلْمَذْبَحُ مِنْ خَشَبٍ ثَلاَثُ أَذْرُعٍ ارْتِفَاعاً, وَطُولُهُ ذِرَاعَانِ, وَزَوَايَاهُ وَطُولُهُ وَحِيطَانُهُ مِنْ خَشَبٍ. وَقَالَ لِي: [هَذِهِ الْمَائِدَةُ أَمَامَ الرَّبِّ».
\par 23 وَلِلْهَيْكَلِ وَلِلْقُدْسِ بَابَانِ.
\par 24 وَلِلْبَابَيْنِ مِصْرَاعَانِ مِصْرَاعَانِ يَنْطَوِيَانِ, مِصْرَاعَانِ لِلْبَابِ الْوَاحِدِ وَمِصْرَاعَانِ لِلْبَابِ الآخَرِ.
\par 25 وَعُمِلَ عَلَيْهَا عَلَى مَصَارِيعِ الْهَيْكَلِ كَرُوبِيمُ وَنَخِيلٌ كَمَا عُمِلَ عَلَى الْحِيطَانِ, وَغِشَاءٌ مِنْ خَشَبٍ عَلَى وَجْهِ الرِّوَاقِ مِنْ خَارِجٍ,
\par 26 وَكُوىً مُشَبَّكَةٌ وَنَخِيلٌ مِنْ هُنَا وَمِنْ هُنَاكَ عَلَى جَوَانِبِ الرِّوَاقِ وَعَلَى غُرُفَاتِ الْبَيْتِ وَعَلَى الأَفَارِيزِ.

\chapter{42}

\par 1 وَأَخْرَجَنِي إِلَى الدَّارِ الْخَارِجِيَّةِ مِنْ طَرِيقِ جِهَةِ الشِّمَالِ وَأَدْخَلَنِي إِلَى الْمِخْدَعِ الَّذِي هُوَ تُجَاهَ الْمَكَانِ الْمُنْفَصِلِ, وَالَّذِي هُوَ قُدَّامَ الْبِنَاءِ إِلَى الشِّمَالِ.
\par 2 إِلَى قُدَّامِ طُولِ مِئَةِ ذِرَاعٍ مَدْخَلُ الشِّمَالِ, وَالْعَرْضُ خَمْسُونَ ذِرَاعاً.
\par 3 تُجَاهَ الْعِشْرِينَ الَّتِي لِلدَّارِ الدَّاخِلِيَّةِ وَتُجَاهَ الْمُجَزَّعِ الَّذِي لِلدَّارِ الْخَارِجِيَّةِ أُسْطُوانَةٌ تُجَاهَ أُسْطُوانَةٍ فِي الطَّبَقَاتِ الثَّلاَثِ.
\par 4 وَأَمَامَ الْمَخَادِعِ مَمْشًى عَشَرُ أَذْرُعٍ عَرْضاً. وَإِلَى الدَّاخِلِيَّةِ طَرِيقٌ, ذِرَاعٌ وَاحِدَةٌ عَرْضاً وَأَبْوَابُهَا نَحْوَ الشِّمَالِ.
\par 5 وَالْمَخَادِعُ الْعُلْيَا أَقْصَرُ. لأَنَّ الأَسَاطِينَ أَكَلَتْ مِنْ هَذِهِ. مِنْ أَسَافِلِ الْبِنَاءِ وَمِنْ أَوَاسِطِهِ.
\par 6 لأَنَّهَا ثَلاَثُ طَبَقَاتٍ, وَلَمْ يَكُنْ لَهَا أَعْمِدَةٌ كَأَعْمِدَةِ الدُّورِ, لِذَلِكَ تَضِيقُ مِنَ الأَسَافِلِ وَمِنَ الأَوَاسِطِ مِنَ الأَرْضِ.
\par 7 وَالْحَائِطُ الَّذِي مِنْ خَارِجٍ مَعَ الْمَخَادِعِ نَحْوَ الدَّارِ الْخَارِجِيَّةِ إِلَى قُدَّامِ الْمَخَادِعِ طُولُهُ خَمْسُونَ ذِرَاعاً.
\par 8 لأَنَّ طُولَ الْمَخَادِعِ الَّتِي لِلدَّارِ الْخَارِجِيَّةِ خَمْسُونَ ذِرَاعاً. وَهُوَذَا أَمَامَ الْهَيْكَلِ مِئَةُ ذِرَاعٍ.
\par 9 وَمِنْ تَحْتِ هَذِهِ الْمَخَادِعِ مَدْخَلٌ مِنَ الشَّرْقِ مِنْ حَيْثُ يُدْخَلُ إِلَيْهَا مِنَ الدَّارِ الْخَارِجِيَّةِ.
\par 10 اَلْمَخَادِعُ كَانَتْ فِي عَرْضِ جِدَارِ الدَّارِ نَحْوَ الشَّرْقِ قُدَّامَ الْمَكَانِ الْمُنْفَصِلِ وَقُبَالَةَ الْبِنَاءِ.
\par 11 وَأَمَامَهَا طَرِيقٌ كَمِثْلِ الْمَخَادِعِ الَّتِي نَحْوَ الشِّمَالِ, كَطُولِهَا هَكَذَا عَرْضُهَا وَجَمِيعُ مَخَارِجِهَا وَكَأَشْكَالِهَا وَكَأَبْوَابِهَا,
\par 12 وَكَأَبْوَابِ الْمَخَادِعِ الَّتِي نَحْوَ الْجَنُوبِ بَابٌ عَلَى رَأْسِ الطَّرِيقِ. الطَّرِيقِ أَمَامَ الْجِدَارِ الْمُوافِقِ نَحْوَ الشَّرْقِ مِنْ حَيْثُ يُدْخَلُ إِلَيْهَا.
\par 13 وَقَالَ لِي: [مَخَادِعُ الشِّمَالِ وَمَخَادِعُ الْجَنُوبِ الَّتِي أَمَامَ الْمَكَانِ الْمُنْفَصِلِ هِيَ مَخَادِعُ مُقَدَّسَةٌ حَيْثُ يَأْكُلُ الْكَهَنَةُ الَّذِينَ يَتَقَرَّبُونَ إِلَى الرَّبِّ قُدْسَ الأَقْدَاسِ. هُنَاكَ يَضَعُونَ قُدْسَ الأَقْدَاسِ وَالتَّقْدِمَةَ وَذَبِيحَةَ الْخَطِيَّةِ وَذَبِيحَةَ الإِثْمِ لأَنَّ الْمَكَانَ مُقَدَّسٌ.
\par 14 عِنْدَ دُخُولِ الْكَهَنَةِ لاَ يَخْرُجُونَ مِنَ الْقُدْسِ إِلَى الدَّارِ الْخَارِجِيَّةِ, بَلْ يَضَعُونَ هُنَاكَ ثِيَابَهُمُ الَّتِي يَخْدِمُونَ بِهَا لأَنَّهَا مُقَدَّسَةٌ, وَيَلْبِسُونَ ثِيَاباً غَيْرَهَا وَيَتَقَدَّمُونَ إِلَى مَا هُوَ لِلشَّعْبِ].
\par 15 فَلَمَّا أَتَمَّ قِيَاسَ الْبَيْتِ الدَّاخِلِيِّ أَخْرَجَنِي نَحْوَ الْبَابِ الْمُتَّجِهِ نَحْوَ الْمَشْرِقِ وَقَاسَهُ حَوَالَيْهِ.
\par 16 قَاسَ جَانِبَ الْمَشْرِقِ بِقَصَبَةِ الْقِيَاسِ خَمْسَ مِئَةِ قَصَبَةٍ بِقَصَبَةِ الْقِيَاسِ حَوَالَيْهِ.
\par 17 وَقَاسَ جَانِبَ الشِّمَالِ خَمْسَ مِئَةِ قَصَبَةٍ بِقَصَبَةِ الْقِيَاسِ حَوَالَيْهِ.
\par 18 وَقَاسَ جَانِبَ الْجَنُوبِ خَمْسَ مِئَةِ قَصَبَةٍ بِقَصَبَةِ الْقِيَاسِ.
\par 19 ثُمَّ دَارَ إِلَى جَانِبِ الْغَرْبِ وَقَاسَ خَمْسَ مِئَةِ قَصَبَةٍ بِقَصَبَةِ الْقِيَاسِ.
\par 20 قَاسَهُ مِنَ الْجَوَانِبِ الأَرْبَعَةِ. لَهُ سُورٌ حَوَالَيْهِ خَمْسُ مِئَةٍ طُولاً وَخَمْسُ مِئَةٍ عَرْضاً, لِلْفَصْلِ بَيْنَ الْمُقَدَّسِ وَالْمُحَلَّلِ.

\chapter{43}

\par 1 ثُمَّ ذَهَبَ بِي إِلَى الْبَابِ الْمُتَّجِهِ نَحْوَ الشَّرْقِ.
\par 2 وَإِذَا بِمَجْدِ إِلَهِ إِسْرَائِيلَ جَاءَ مِنْ طَرِيقِ الشَّرْقِ وَصَوْتُهُ كَصَوْتِ مِيَاهٍ كَثِيرَةٍ, وَالأَرْضُ أَضَاءَتْ مِنْ مَجْدِهِ.
\par 3 وَالْمَنْظَرُ كَالْمَنْظَرِ الَّذِي رَأَيْتُهُ لَمَّا جِئْتُ لأُخْرِبَ الْمَدِينَةَ, وَالْمَنَاظِرُ كَالْمَنْظَرِ الَّذِي رَأَيْتُ عِنْدَ نَهْرِ خَابُورَ, فَخَرَرْتُ عَلَى وَجْهِي.
\par 4 فَجَاءَ مَجْدُ الرَّبِّ إِلَى الْبَيْتِ مِنْ طَرِيقِ الْبَابِ الْمُتَّجِهِ نَحْوَ الشَّرْقِ.
\par 5 فَحَمَلَنِي رُوحٌ وَأَتَى بِي إِلَى الدَّارِ الدَّاخِلِيَّةِ, وَإِذَا بِمَجْدِ الرَّبِّ قَدْ مَلأَ الْبَيْتَ.
\par 6 وَسَمِعْتُهُ يُكَلِّمُنِي مِنَ الْبَيْتِ. وَكَانَ رَجُلٌ وَاقِفاً عِنْدِي.
\par 7 وَقَالَ لِي: [يَا ابْنَ آدَمَ, هَذَا مَكَانُ كُرْسِيِّي وَمَكَانُ بَاطِنِ قَدَمَيَّ حَيْثُ أَسْكُنُ فِي وَسَطِ بَنِي إِسْرَائِيلَ إِلَى الأَبَدِ, وَلاَ يُنَجِّسُ بَعْدُ بَيْتُ إِسْرَائِيلَ اسْمِي الْقُدُّوسَ, لاَ هُمْ وَلاَ مُلُوكُهُمْ, لاَ بِزِنَاهُمْ وَلاَ بِجُثَثِ مُلُوكِهِمْ فِي مُرْتَفَعَاتِهِمْ.
\par 8 بِجَعْلِهِمْ عَتَبَتَهُمْ لَدَى عَتَبَتِي وَقَوَائِمَهُمْ لَدَى قَوَائِمِي وَبَيْنِي وَبَيْنَهُمْ حَائِطٌ, فَنَجَّسُوا اسْمِي الْقُدُّوسَ بِرَجَاسَاتِهِمِ الَّتِي فَعَلُوهَا, فَأَفْنَيْتُهُمْ بِغَضَبِي.
\par 9 فَلْيُبْعِدُوا عَنِّي الآنَ زِنَاهُمْ وَجُثَثَ مُلُوكِهِمْ فَأَسْكُنَ فِي وَسَطِهِمْ إِلَى الأَبَدِ.
\par 10 [وَأَنْتَ يَا ابْنَ آدَمَ فَأَخْبِرْ بَيْتَ إِسْرَائِيلَ عَنِ الْبَيْتِ لِيَخْزُوا مِنْ آثَامِهِمْ. وَلْيَقِيسُوا الرَّسْمَ.
\par 11 فَإِنْ خَزُوا مِنْ كُلِّ مَا فَعَلُوهُ فَعَرِّفْهُمْ صُورَةَ الْبَيْتِ وَرَسْمَهُ وَمَخَارِجَهُ وَمَدَاخِلَهُ وَكُلَّ أَشْكَالِهِ وَكُلَّ فَرَائِضِهِ وَكُلَّ أَشْكَالِهِ وَكُلَّ شَرَائِعِهِ. وَاكْتُبْ ذَلِكَ قُدَّامَ أَعْيُنِهِمْ لِيَحْفَظُوا كُلَّ رُسُومِهِ وَكُلَّ فَرَائِضِهِ وَيَعْمَلُوا بِهَا.
\par 12 هَذِهِ سُنَّةُ الْبَيْتِ. عَلَى رَأْسِ الْجَبَلِ كُلُّ تُخُمِهِ حَوَالَيْهِ قُدْسُ أَقْدَاسٍ. هَذِهِ هِيَ سُنَّةُ الْبَيْتِ.
\par 13 [وَهَذِهِ أَقْيِسَةُ الْمَذْبَحِ بِالأَذْرُعِ (وَالذِّرَاعُ هِيَ ذِرَاعٌ وَفِتْرٌ): الْحِضْنُ ذِرَاعٌ, وَالْعَرْضُ ذِرَاعٌ, وَحَاشِيَتُهُ إِلَى شَفَتِهِ حَوَالَيْهِ شِبْرٌ وَاحِدٌ. هَذَا ظَهْرُ الْمَذْبَحِ.
\par 14 وَمِنَ الْحِضْنِ عِنْدَ الأَرْضِ إِلَى الْخُصْمِ الأَسْفَلِ ذِرَاعَانِ, وَالْعَرْضُ ذِرَاعٌ. وَمِنَ الْخُصْمِ الأَصْغَرِ إِلَى الْخُصْمِ الأَكْبَرِ أَرْبَعُ أَذْرُعٍ, وَالْعَرْضُ ذِرَاعٌ.
\par 15 وَالْمَوْقِدُ أَرْبَعُ أَذْرُعٍ. وَمِنَ الْمَوْقِدِ إِلَى فَوْقُ أَرْبَعَةُ قُرُونٍ.
\par 16 وَالْمَوْقِدُ اثْنَتَا عَشَرَةَ طُولاً, بِاثْنَتَيْ عَشَرَةَ عَرْضاً, مُرَبَّعاً عَلَى جَوَانِبِهِ الأَرْبَعَةِ.
\par 17 وَالْخُصْمُ أَرْبَعَ عَشَرَةَ طُولاً بِأَرْبَعَ عَشَْرَةَ عَرْضاً عَلَى جَوَانِبِهِ الأَرْبَعَةِ. وَالْحَاشِيَةُ حَوَالَيْهِ نِصْفُ ذِرَاعٍ, وَحِضْنُهُ ذِرَاعٌ حَوَالَيْهِ, وَدَرَجَاتُهُ تُجَاهَ الْمَشْرِقِ].
\par 18 وَقَالَ لِي: [يَا ابْنَ آدَمَ, هَكَذَا قَالَ السَّيِّدُ الرَّبُّ: هَذِهِ فَرَائِضُ الْمَذْبَحِ يَوْمَ صُنْعِهِ لإِصْعَادِ الْمُحْرَقَةِ عَلَيْهِ وَلِرَشِّ الدَّمِ عَلَيْهِ.
\par 19 فَتُعْطِي الْكَهَنَةَ اللاَّوِيِّينَ الَّذِينَ مِنْ نَسْلِ صَادُوقَ الْمُقْتَرِبِينَ إِلَيَّ لِيَخْدِمُونِي, ثَوْراً مِنَ الْبَقَرِ لِذَبِيحَةِ خَطِيَّةٍ.
\par 20 وَتَأْخُذُ مِنْ دَمِهِ وَتَضَعُهُ عَلَى قُرُونِهِ الأَرْبَعَةِ, وَعَلَى أَرْبَعِ زَوَايَا الْخُصْمِ وَعَلَى الْحَاشِيَةِ حَوَالَيْهَا, فَتُطَهِّرُهُ وَتُكَفِّرُ عَنْهُ.
\par 21 وَتَأْخُذُ ثَوْرَ الْخَطِيَّةِ فَيُحْرَقُ فِي الْمَوْضِعِ الْمُعَيَّنِ مِنَ الْبَيْتِ خَارِجَ الْمَقْدِسِ.
\par 22 وَفِي الْيَوْمِ الثَّانِي تُقَرِّبُ تَيْساً مِنَ الْمَعْزِ صَحِيحاً ذَبِيحَةَ خَطِيَّةٍ, فَيُطَهِّرُونَ الْمَذْبَحَ كَمَا طَهَّرُوهُ بِالثَّوْرِ.
\par 23 وَإِذَا أَكْمَلْتَ التَّطْهِيرَ تُقَرِّبُ ثَوْراً مِنَ الْبَقَرِ صَحِيحاً, وَكَبْشاً مِنَ الضَّأْنِ صَحِيحاً.
\par 24 وَتُقَرِّبُهُمَا قُدَّامَ الرَّبِّ, وَيُلْقِي عَلَيْهِمَا الْكَهَنَةُ مِلْحاً وَيُصْعِدُونَهُمَا مُحْرَقَةً لِلرَّبِّ.
\par 25 سَبْعَةَ أَيَّامٍ تَعْمَلُ فِي كُلِّ يَوْمٍ تَيْسَ الْخَطِيَّةِ. وَيَعْمَلُونَ ثَوْراً مِنَ الْبَقَرِ وَكَبْشاً مِنَ الضَّأْنِ صَحِيحَيْنِ.
\par 26 سَبْعَةَ أَيَّامٍ يُكَفِّرُونَ عَنِ الْمَذْبَحِ وَيُطَهِّرُونَهُ وَيَمْلأُونَ يَدَهُ.
\par 27 فَإِذَا تَمَّتْ هَذِهِ الأَيَّامُ يَكُونُ فِي الْيَوْمِ الثَّامِنِ فَصَاعِداً أَنَّ الْكَهَنَةَ يَعْمَلُونَ عَلَى الْمَذْبَحِ مُحْرَقَاتِكُمْ وَذَبَائِحَكُمْ السَّلاَمِيَّةَ, فَأَرْضَى عَنْكُمْ يَقُولُ السَّيِّدُ الرَّبُّ].

\chapter{44}

\par 1 ثُمَّ أَرْجَعَنِي إِلَى طَرِيقِ بَابِ الْمَقْدِسِ الْخَارِجِيِّ الْمُتَّجِهِ لِلْمَشْرِقِ وَهُوَ مُغْلَقٌ.
\par 2 فَقَالَ لِيَ الرَّبُّ: [هَذَا الْبَابُ يَكُونُ مُغْلَقاً, لاَ يُفْتَحُ وَلاَ يَدْخُلُ مِنْهُ إِنْسَانٌ, لأَنَّ الرَّبَّ إِلَهَ إِسْرَائِيلَ دَخَلَ مِنْهُ فَيَكُونُ مُغْلَقاً.
\par 3 اَلرَّئِيسُ الرَّئِيسُ هُوَ يَجْلِسُ فِيهِ لِيَأْكُلَ خُبْزاً أَمَامَ الرَّبِّ. مِنْ طَرِيقِ رِوَاقِ الْبَابِ يَدْخُلُ, وَمِنْ طَرِيقِهِ يَخْرُجُ».
\par 4 ثُمَّ أَتَى بِي فِي طَرِيقِ بَابِ الشِّمَالِ إِلَى قُدَّامِ الْبَيْتِ. فَنَظَرْتُ وَإِذَا بِمَجْدِ الرَّبِّ قَدْ مَلأَ بَيْتَ الرَّبِّ. فَخَرَرْتُ عَلَى وَجْهِي.
\par 5 فَقَالَ لِي الرَّبُّ: [يَا ابْنَ آدَمَ, اجْعَلْ قَلْبَكَ وَانْظُرْ بِعَيْنَيْكَ وَاسْمَعْ بِأُذُنَيْكَ كُلَّ مَا أَقُولُهُ لَكَ عَنْ كُلِّ فَرَائِضِ بَيْتِ الرَّبِّ وَعَنْ كُلِّ سُنَنِهِ, وَاجْعَلْ قَلْبَكَ عَلَى مَدْخَلِ الْبَيْتِ مَعَ كُلِّ مَخَارِجِ الْمَقْدِسِ.
\par 6 وَقُلْ لِلْمُتَمَرِّدِينَ, لِبَيْتِ إِسْرَائِيلَ: هَكَذَا قَالَ السَّيِّدُ الرَّبُّ: يَكْفِيكُمْ كُلُّ رَجَاسَاتِكُمْ يَا بَيْتَ إِسْرَائِيلَ
\par 7 بِإِدْخَالِكُمْ أَبْنَاءَ الْغَرِيبِ الْغُلْفَ الْقُلُوبِ الْغُلْفَ اللَّحْمِ لِيَكُونُوا فِي مَقْدِسِي, فَيُنَجِّسُوا بَيْتِي بِتَقْرِيبِكُمْ خُبْزِي الشَّحْمَ وَالدَّمَ. فَنَقَضُوا عَهْدِي فَوْقَ كُلِّ رَجَاسَاتِكُمْ.
\par 8 وَلَمْ تَحْرُسُوا حِرَاسَةَ أَقْدَاسِي, بَلْ أَقَمْتُمْ حُرَّاساً يَحْرُسُونَ عَنْكُمْ فِي مَقْدِسِي].
\par 9 هَكَذَا قَالَ السَّيِّدُ الرَّبُّ: [ابْنُ الْغَرِيبِ أَغْلَفُ الْقَلْبِ وَأَغْلَفُ اللَّحْمِ لاَ يَدْخُلُ مَقْدِسِي مِنْ كُلِّ ابْنٍ غَرِيبٍ الَّذِي مِنْ وَسَطِ بَنِي إِسْرَائِيلَ.
\par 10 بَلِ اللاَّوِيُّونَ الَّذِينَ ابْتَعَدُوا عَنِّي حِينَ ضَلَّ إِسْرَائِيلُ, فَضَلُّوا عَنِّي وَرَاءَ أَصْنَامِهِمْ, يَحْمِلُونَ إِثْمَهُمْ.
\par 11 وَيَكُونُونَ خُدَّاماً فِي مَقْدِسِي حُرَّاسَ أَبْوَابِ الْبَيْتِ وَخُدَّامَ الْبَيْتِ. هُمْ يَذْبَحُونَ الْمُحْرَقَةَ وَالذَّبِيحَةَ لِلشَّعْبِ, وَهُمْ يَقِفُونَ أَمَامَهُمْ لِيَخْدِمُوهُمْ.
\par 12 لأَنَّهُمْ خَدَمُوهُمْ أَمَامَ أَصْنَامِهِمْ وَكَانُوا مَعْثَرَةَ إِثْمٍ لِبَيْتِ إِسْرَائِيلَ. لِذَلِكَ رَفَعْتُ يَدِي عَلَيْهِمْ يَقُولُ السَّيِّدُ الرَّبُّ فَيَحْمِلُونَ إِثْمَهُمْ.
\par 13 وَلاَ يَتَقَرَّبُونَ إِلَيَّ لِيَكْهَنُوا لِي, وَلاَ لِلاِقْتِرَابِ إِلَى شَيْءٍ مِنْ أَقْدَاسِي إِلَى قُدْسِ الأَقْدَاسِ, بَلْ يَحْمِلُونَ خِزْيَهُمْ وَرَجَاسَاتِهِمِ الَّتِي فَعَلُوهَا.
\par 14 وَأَجْعَلُهُمْ حَارِسِي حِرَاسَةَ الْبَيْتِ لِكُلِّ خِدْمَةٍ لِكُلِّ مَا يُعْمَلُ فِيهِ.
\par 15 [أَمَّا الْكَهَنَةُ اللاَّوِيُّونَ أَبْنَاءُ صَادُوقَ الَّذِينَ حَرَسُوا حِرَاسَةَ مَقْدِسِي حِينَ ضَلَّ عَنِّي بَنُو إِسْرَائِيلَ فَهُمْ يَتَقَدَّمُونَ إِلَيَّ لِيَخْدِمُونِي, وَيَقِفُونَ أَمَامِي لِيُقَرِّبُوا لِي الشَّحْمَ وَالدَّمَ يَقُولُ السَّيِّدُ الرَّبُّ.
\par 16 هُمْ يَدْخُلُونَ مَقْدِسِي وَيَتَقَدَّمُونَ إِلَى مَائِدَتِي لِيَخْدِمُونِي وَيَحْرُسُوا حِرَاسَتِي.
\par 17 وَيَكُونُ عِنْدَ دُخُولِهِمْ أَبْوَابَ الدَّارِ الدَّاخِلِيَّةِ أَنَّهُمْ يَلْبِسُونَ ثِيَاباً مِنْ كَتَّانٍ, وَلاَ يَأْتِي عَلَيْهِمْ صُوفٌ عِنْدَ خِدْمَتِهِمْ فِي أَبْوَابِ الدَّارِ الدَّاخِلِيَّةِ وَمِنْ دَاخِلٍ.
\par 18 وَلْتَكُنْ عَصَائِبُ مِنْ كَتَّانٍ عَلَى رُؤُوسِهِمْ, وَلْتَكُنْ سَرَاوِيلُ مِنْ كَتَّانٍ عَلَى أَحْقَائِهِمْ. لاَ يَتَنَطَّقُونَ بِمَا يُعَرِّقُ.
\par 19 وَعِنْدَ خُرُوجِهِمْ إِلَى الدَّارِ الْخَارِجِيَّةِ إِلَى الشَّعْبِ إِلَى الدَّارِ الْخَارِجِيَّةِ يَخْلَعُونَ ثِيَابَهُمُ الَّتِي خَدَمُوا بِهَا, وَيَضَعُونَهَا فِي مَخَادِعِ الْقُدْسِ, ثُمَّ يَلْبِسُونَ ثِيَاباً أُخْرَى وَلاَ يُقَدِّسُونَ الشَّعْبَ بِثِيَابِهِمْ.
\par 20 وَلاَ يَحْلِقُونَ رُؤُوسَهُمْ, وَلاَ يُرَبُّونَ خُصَلاً, بَلْ يَجُزُّونَ شَعْرَ رُؤُوسِهِمْ جَزّاً.
\par 21 وَلاَ يَشْرَبُ كَاهِنٌ خَمْراً عِنْدَ دُخُولِهِ إِلَى الدَّارِ الدَّاخِلِيَّةِ.
\par 22 وَلاَ يَأْخُذُونَ أَرْمَلَةً وَلاَ مُطَلَّقَةً زَوْجَةً, بَلْ يَتَّخِذُونَ عَذَارَى مِنْ نَسْلِ بَيْتِ إِسْرَائِيلَ, أَوْ أَرْمَلَةً الَّتِي كَانَتْ أَرْمَلَةَ كَاهِنٍ.
\par 23 وَيُرُونَ شَعْبِي التَّمْيِيزَ بَيْنَ الْمُقَدَّسِ وَالْمُحَلَّلِ, وَيُعَلِّمُونَهُمُ التَّمْيِيزَ بَيْنَ النَّجِسِ وَالطَّاهِرِ.
\par 24 وَفِي الْخِصَامِ هُمْ يَقِفُونَ لِلْحُكْمِ وَيَحْكُمُونَ حَسَبَ أَحْكَامِي, وَيَحْفَظُونَ شَرَائِعِي وَفَرَائِضِي فِي كُلِّ مَوَاسِمِي, وَيُقَدِّسُونَ سُبُوتِي.
\par 25 وَلاَ يَدْنُوا مِنْ إِنْسَانٍ مَيِّتٍ فَيَتَنَجَّسُوا. أَمَّا لأَبٍ أَوْ أُمٍّ أَوِ ابْنٍ أَوِ ابْنَةٍ أَوْ أَخٍ أَوْ أُخْتٍ لَمْ تَكُنْ لِرَجُلٍ يَتَنَجَّسُونَ.
\par 26 وَبَعْدَ تَطْهِيرِهِ يَحْسِبُونَ لَهُ سَبْعَةَ أَيَّامٍ.
\par 27 وَفِي يَوْمِ دُخُولِهِ إِلَى الْقُدْسِ إِلَى الدَّارِ الدَّاخِلِيَّةِ لِيَخْدِمَ فِي الْقُدْسِ يُقَرِّبُ ذَبِيحَتَهُ عَنِ الْخَطِيَّةِ يَقُولُ السَّيِّدُ الرَّبُّ.
\par 28 وَيَكُونُ لَهُمْ مِيرَاثاً. أَنَا مِيرَاثُهُمْ, وَلاَ تُعْطُونَهُمْ مِلْكاً فِي إِسْرَائِيلَ. أَنَا مِلْكُهُمْ.
\par 29 يَأْكُلُونَ التَّقْدِمَةَ وَذَبِيحَةَ الْخَطِيَّةِ وَذَبِيحَةَ الإِثْمِ, وَكُلُّ مُحَرَّمٍ فِي إِسْرَائِيلَ يَكُونُ لَهُمْ.
\par 30 وَأَوَائِلُ كُلِّ الْبَاكُورَاتِ جَمِيعِهَا وَكُلُّ رَفِيعَةٍ مِنْ كُلِّ رَفَائِعِكُمْ تَكُونُ لِلْكَهَنَةِ. وَتُعْطُونَ الْكَاهِنَ أَوَائِلَ عَجِينِكُمْ لِتَحِلَّ الْبَرَكَةُ عَلَى بَيْتِكَ.
\par 31 لاَ يَأْكُلُ الْكَاهِنُ مِنْ مَيِّتَةٍ وَلاَ مِنْ فَرِيسَةٍ, طَيْراً كَانَتْ أَوْ بَهِيمَةً.

\chapter{45}

\par 1 [وَإِذَا قَسَمْتُمُ الأَرْضَ مِلْكاً تُقَدِّمُونَ تَقْدِمَةً لِلرَّبِّ قُدْساً مِنَ الأَرْضِ طُولُهُ خَمْسَةٌ وَعِشْرُونَ أَلْفاً طُولاً, وَالْعَرْضُ عَشَرَةُ آلاَفٍ. هَذَا قُدْسٌ بِكُلِّ تُخُومِهِ حَوَالَيْهِ.
\par 2 يَكُونُ لِلْقُدْسِ مِنْ هَذَا خَمْسُ مِئَةٍ فِي خَمْسِ مِئَةٍ, مُرَبَّعَةٍ حَوَالَيْهِ, وَخَمْسُونَ ذِرَاعاً مَسْرَحاً لَهُ حَوَالَيْهِ.
\par 3 مِنْ هَذَا الْقِيَاسِ تَقِيسُ طُولَ خَمْسَةٍ وَعِشْرِينَ أَلْفاً وَعَرْضَ عَشَرَةِ آلاَفٍ, وَفِيهِ يَكُونُ الْمَقْدِسُ قُدْسُ الأَقْدَاسِ.
\par 4 قُدْسٌ مِنَ الأَرْضِ هُوَ. يَكُونُ لِلْكَهَنَةِ خُدَّامِ الْمَقْدِسِ الْمُقْتَرِبِينَ لِخِدْمَةِ الرَّبِّ, وَيَكُونُ لَهُمْ مَوْضِعاً لِلْبُيُوتِ وَمُقَدَّساً لِلْمَقْدِسِ.
\par 5 وَخَمْسَةٌ وَعِشْرُونَ أَلْفاً فِي الطُّولِ وَعَشَرَةُ آلاَفٍ فِي الْعَرْضِ تَكُونُ لِلاَّوِيِّينَ خُدَّامِ الْبَيْتِ لَهُمْ مِلْكاً. عِشْرُونَ مِخْدَعاً.
\par 6 وَتَجْعَلُونَ مِلْكَ الْمَدِينَةِ خَمْسَةَ آلاَفٍ عَرْضاً وَخَمْسَةً وَعِشْرِينَ أَلْفاً طُولاً, مُوازِياً تَقْدِمَةَ الْقُدْسِ, فَيَكُونُ لِكُلِّ بَيْتِ إِسْرَائِيلَ.
\par 7 [وَلِلرَّئِيسِ مِنْ هُنَا وَمِنْ هُنَاكَ مِنْ تَقْدِمَةِ الْقُدْسِ وَمِنْ مِلْكِ الْمَدِينَةِ قُدَّامَ تَقْدِمَةِ الْقُدْسِ وَقُدَّامَ مِلْكِ الْمَدِينَةِ مِنْ جِهَةِ الْغَرْبِ غَرْباً, وَمِنْ جِهَةِ الشَّرْقِ شَرْقاً وَالطُّولُ مُوازٍ أَحَدَ الْقِسْمَيْنِ مِنْ تُخُمِ الْغَرْبِ إِلَى تُخُمِ الشَّرْقِ.
\par 8 تَكُونُ لَهُ أَرْضاً مِلْكاً فِي إِسْرَائِيلَ, وَلاَ تَعُودُ رُؤَسَائِي يَظْلِمُونَ شَعْبِي, وَالأَرْضُ يُعْطُونَهَا لِبَيْتِ إِسْرَائِيلَ لأَسْبَاطِهِمْ].
\par 9 هَكَذَا قَالَ السَّيِّدُ الرَّبُّ: [يَكْفِيكُمْ يَا رُؤَسَاءَ إِسْرَائِيلَ. أَزِيلُوا الْجَوْرَ وَالاِغْتِصَابَ, وَأَجْرُوا الْحَقَّ وَالْعَدْلَ. ارْفَعُوا الظُّلْمَ عَنْ شَعْبِي يَقُولُ السَّيِّدُ الرَّبُّ.
\par 10 مَوَازِينُ حَقٍّ وَإِيفَةُ حَقٍّ وَبَثُّ حَقٍّ تَكُونُ لَكُمْ.
\par 11 تَكُونُ الإِيفَةُ وَالْبَثُّ مِقْدَاراً وَاحِداً, لِكَيْ يَسَعَ الْبَثُّ عُشْرَ الْحُومَرِ, وَالإِيفَةُ عُشْرُ الْحُومَرِ. عَلَى الْحُومَرِ يَكُونُ مِقْدَارُهُمَا.
\par 12 وَالشَّاقِلُ عِشْرُونَ جِيرَةً. عِشْرُونَ شَاقِلاً وَخَمْسَةٌ وَعِشْرُونَ شَاقِلاً وَخَمْسَةَ عَشَرَ شَاقِلاً تَكُونُ مَنَّكُمْ.
\par 13 هَذِهِ هِيَ التَّقْدِمَةُ الَّتِي تُقَدِّمُونَهَا. سُدْسَ الإِيفَةِ مِنْ حُومَرِ الْحِنْطَةِ. وَتُعْطُونَ سُدْسَ الإِيفَةِ مِنْ حُومَرِ الشَّعِيرِ.
\par 14 وَفَرِيضَةُ الزَّيْتِ بَثٌّ مِنْ زَيْتٍ. الْبَثُّ عُشْرٌ مِنَ الْكُرِّ مِنْ عَشَرَةِ أَبْثَاثٍ لِلْحُومَرِ, لأَنَّ عَشَرَةَ أَبْثَاثٍ حُومَرٌ.
\par 15 وَشَاةٌ وَاحِدَةٌ مِنَ الضَّأْنِ. مِنَ الْمِئَتَيْنِ مِنْ سَقْيِ إِسْرَائِيلَ تَقْدِمَةً وَمُحْرَقَةً وَذَبَائِحَ سَلاَمَةٍ, لِلْكَفَّارَةِ عَنْهُمْ يَقُولُ السَّيِّدُ الرَّبُّ.
\par 16 وَهَذِهِ التَّقْدِمَةُ لِلرَّئِيسِ فِي إِسْرَائِيلَ تَكُونُ عَلَى كُلِّ شَعْبِ الأَرْضِ.
\par 17 وَعَلَى الرَّئِيسِ تَكُونُ الْمُحْرَقَاتُ وَالتَّقْدِمَةُ وَالسَّكِيبُ فِي الأَعْيَادِ وَفِي الشُّهُورِ وَفِي السُّبُوتِ وَفِي كُلِّ مَوَاسِمِ بَيْتِ إِسْرَائِيلَ. وَهُوَ يَعْمَلُ ذَبِيحَةَ الْخَطِيَّةِ وَالتَّقْدِمَةَ وَالْمُحْرَقَةَ وَذَبَائِحَ السَّلاَمَةِ لِلْكَفَّارَةِ عَنْ بَيْتِ إِسْرَائِيلَ].
\par 18 هَكَذَا قَالَ السَّيِّدُ الرَّبُّ: [فِي الشَّهْرِ الأَوَّلِ فِي أَوَّلِ الشَّهْرِ تَأْخُذُ ثَوْراً مِنَ الْبَقَرِ صَحِيحاً وَتُطَهِّرُ الْمَقْدِسَ.
\par 19 وَيَأْخُذُ الْكَاهِنُ مِنْ دَمِ ذَبِيحَةِ الْخَطِيَّةِ وَيَضَعُهُ عَلَى قَوَائِمِ الْبَيْتِ, وَعَلَى زَوَايَا خُصْمِ الْمَذْبَحِ الأَرْبَعِ, وَعَلَى قَوَائِمِ بَابِ الدَّارِ الدَّاخِلِيَّةِ.
\par 20 وَهَكَذَا تَفْعَلُ فِي سَابِعِ الشَّهْرِ عَنِ الرَّجُلِ السَّاهِي أَوِ الْغَوِيِّ, فَتُكَفِّرُونَ عَنِ الْبَيْتِ.
\par 21 فِي الشَّهْرِ الأَوَّلِ فِي الْيَوْمِ الرَّابِعِ عَشَرَ مِنَ الشَّهْرِ يَكُونُ لَكُمُ الْفِصْحُ عِيداً. سَبْعَةَ أَيَّامٍ يُؤْكَلُ الْفَطِيرُ.
\par 22 وَيَعْمَلُ الرَّئِيسُ فِي ذَلِكَ الْيَوْمِ عَنْ نَفْسِهِ وَعَنْ كُلِّ شَعْبِ الأَرْضِ ثَوْراً ذَبِيحَةَ خَطِيَّةٍ.
\par 23 وَفِي سَبْعَةِ أَيَّامِ الْعِيدِ يَعْمَلُ مُحْرَقَةً لِلرَّبِّ: سَبْعَةَ ثِيرَانٍ وَسَبْعَةَ كِبَاشٍ صَحِيحَةٍ, كُلَّ يَوْمٍ مِنَ السَّبْعَةِ الأَيَّامِ. وَكُلَّ يَوْمٍ تَيْساً مِنَ الْمَعْزِ ذَبِيحَةَ خَطِيَّةٍ.
\par 24 وَيَعْمَلُ التَّقْدِمَةَ إِيفَةً لِلثَّوْرِ, وَإِيفَةً لِلْكَبْشِ, وَهِيناً مِن زَيْتٍ لِلإِيفَةِ.
\par 25 في الشَّهْرِ السَّابِعِ في الْيَوْمِ الْخَامِسِ عَشَرَ مِنَ الشَّهْرِ فِي الْعِيدِ يَعْمَلُ مِثْلَ ذَلِكَ سَبْعَةَ أَيَّامٍ كَذَبِيحَةِ الْخَطِيَّةِ وَكَالْمُحْرَقَةِ وَكَالتَّقْدِمَةِ وَكَالزَّيْتِ].

\chapter{46}

\par 1 هَكَذَا قَالَ السَّيِّدُ الرَّبُّ: [بَابُ الدَّارِ الدَّاخِلِيَّةِ الْمُتَّجِهُ لِلْمَشْرِقِ يَكُونُ مُغْلَقاً سِتَّةَ أَيَّامِ الْعَمَلِ, وَفِي السَّبْتِ يُفْتَحُ. وَأَيْضاً فِي يَوْمِ رَأْسِ الشَّهْرِ يُفْتَحُ.
\par 2 وَيَدْخُلُ الرَّئِيسُ مِنْ طَرِيقِ رِوَاقِ الْبَابِ مِنْ خَارِجٍ وَيَقِفُ عِنْدَ قَائِمَةِ الْبَابِ, وَتَعْمَلُ الْكَهَنَةُ مُحْرَقَتَهُ وَذَبَائِحَهُ السَّلاَمِيَّةَ, فَيَسْجُدُ عَلَى عَتَبَةِ الْبَابِ ثُمَّ يَخْرُجُ. أَمَّا الْبَابُ فَلاَ يُغْلَقُ إِلَى الْمَسَاءِ.
\par 3 وَيَسْجُدُ شَعْبُ الأَرْضِ عِنْدَ مَدْخَلِ هَذَا الْبَابِ قُدَّامَ الرَّبِّ فِي السُّبُوتِ وَفِي رُؤُوسِ الشُّهُورِ.
\par 4 وَالْمُحْرَقَةُ الَّتِي يُقَرِّبُهَا الرَّئِيسُ لِلرَّبِّ فِي يَوْمِ السَّبْتِ: سِتَّةُ حُمْلاَنٍ صَحِيحَةٍ وَكَبْشٌ صَحِيحٌ.
\par 5 وَالتَّقْدِمَةُ إِيفَةٌ لِلْكَبْشِ, وَلِلْحُمْلاَنِ تَقْدِمَةُ عَطِيَّةِ يَدِهِ, وَهِينُ زَيْتٍ لِلإِيفَةِ.
\par 6 وَفِي يَوْمِ رَأْسِ الشَّهْرِ: ثَوْرٌ ابْنُ بَقَرٍ صَحِيحٌ, وَسِتَّةُ حُمْلاَنٍ, وَكَبْشٌ تَكُونُ صَحِيحَةً.
\par 7 وَيَعْمَلُ تَقْدِمَةً إِيفَةً لِلثَّوْرِ وَإِيفَةً لِلْكَبْشِ. أَمَّا لِلْحُمْلاَنِ فَحَسْبَمَا تَنَالُ يَدُهُ. وَلِلإِيفَةِ هِينُ زَيْتٍ.
\par 8 [وَعَُِنْدَ دُخُولِ الرَّئِيسِ يَدْخُلُ مِنْ طَرِيقِ رِوَاقِ الْبَابِ, وَمِنْ طَرِيقِهِ يَخْرُجُ.
\par 9 وَعِنْدَ دُخُولِ شَعْبِ الأَرْضِ قُدَّامَ الرَّبِّ فِي الْمَوَاسِمِ فَالدَّاخِلُ مِنْ طَرِيقِ بَابِ الشِّمَالِ لِيَسْجُدَ يَخْرُجُ مِنْ طَرِيقِ بَابِ الْجَنُوبِ. وَالدَّاخِلُ مِنْ طَرِيقِ بَابِ الْجَنُوبِ يَخْرُجُ مِنْ طَرِيقِ بَابِ الشِّمَالِ. لاَ يَرْجِعُ مِنْ طَرِيقِ الْبَابِ الَّذِي دَخَلَ مِنْهُ, بَلْ يَخْرُجُ مُقَابِلَهُ.
\par 10 وَالرَّئِيسُ فِي وَسَطِهِمْ يَدْخُلُ عِنْدَ دُخُولِهِمْ, وَعِنْدَ خُرُوجِهِمْ يَخْرُجُونَ مَعاً.
\par 11 وَفِي الأَعْيَادِ وَفِي الْمَوَاسِمِ تَكُونُ التَّقْدِمَةُ إِيفَةً لِلثَّوْرِ وَإِيفَةً لِلْكَبْشِ. وَلِلْحُمْلاَنِ عَطِيَّةُ يَدِهِ, وَلِلإِيفَةِ هِينُ زَيْتٍ.
\par 12 وَإِذَا عَمِلَ الرَّئِيسُ نَافِلَةً, مُحْرَقَةً أَوْ ذَبَائِحَ سَلاَمَةٍ, نَافِلَةً لِلرَّبِّ, يُفْتَحُ لَهُ الْبَابُ الْمُتَّجِهُ لِلْمَشْرِقِ, فَيَعْمَلُ مُحْرَقَتَهُ وَذَبَائِحَهُ السَّلاَمِيَّةَ كَمَا يَعْمَلُ فِي يَوْمِ السَّبْتِ ثُمَّ يَخْرُجُ. وَبَعْدَ خُرُوجِهِ يُغْلَقُ الْبَابُ.
\par 13 وَتَعْمَلُ كُلَّ يَوْمٍ مُحْرَقَةً لِلرَّبِّ حَمَلاً حَوْلِيّاً صَحِيحاً. صَبَاحاً صَبَاحاً تَعْمَلُهُ.
\par 14 وَتَعْمَلُ عَلَيْهِ تَقْدِمَةً صَبَاحاً صَبَاحاً سُدْسَ الإِيفَةِ وَزَيْتاً ثُلْثَ الْهِينِ لِرَشِّ الدَّقِيقِ. تَقْدِمَةً لِلرَّبِّ, فَرِيضَةً أَبَدِيَّةً دَائِمَةً.
\par 15 وَيَعْمَلُونَ الْحَمَلَ وَالتَّقْدِمَةَ وَالزَّيْتَ صَبَاحاً صَبَاحاً مُحْرَقَةً دَائِمَةً».
\par 16 هَكَذَا قَالَ السَّيِّدُ الرَّبُّ: [إِنْ أَعْطَى الرَّئِيسُ رَجُلاً مِنْ بَنِيهِ عَطِيَّةً, فَإِرْثُهَا يَكُونُ لِبَنِيهِ. مُلْكُهُمْ هِيَ بِالْوَرَاثَةِ.
\par 17 فَإِنْ أَعْطَى أَحَداً مِنْ عَبِيدِهِ عَطِيَّةً مِنْ مِيرَاثِهِ فَتَكُونُ لَهُ إِلَى سَنَةِ الْعِتْقِ, ثُمَّ تَرْجِعُ لِلرَّئِيسِ. وَلَكِنَّ مِيرَاثَهُ يَكُونُ لأَوْلاَدِهِ.
\par 18 وَلاَ يَأْخُذُ الرَّئِيسُ مِنْ مِيرَاثِ الشَّعْبِ طَرْداً لَهُمْ مِنْ مُلْكِهِمْ. مِنْ مُلْكِهِ يُورِثُ بَنِيهِ, لِكَيْلاَ يُفَرَّقَ شَعْبِي الرَّجُلُ عَنْ مُلْكِهِ].
\par 19 0ثُمَّ أَدْخَلَنِي بِالْمَدْخَلِ الَّذِي بِجَانِبِ الْبَابِ إِلَى مَخَادِعِ الْقُدْسِ الَّتِي لِلْكَهَنَةِ الْمُتَّجِهَةِ لِلشِّمَالِ, وَإِذَا هُنَاكَ مَوْضِعٌ عَلَى الْجَانِبَيْنِ إِلَى الْغَرْبِ.
\par 20 وَقَالَ لِي: [هَذَا هُوَ الْمَوْضِعُ الَّذِي تَطْبُخُ فِيهِ الْكَهَنَةُ ذَبِيحَةَ الإِثْمِ وَذَبِيحَةَ الْخَطِيَّةِ, وَحَيْثُ يَخْبِزُونَ التَّقْدِمَةَ, لِئَلاَّ يَخْرُجُوا بِهَا إِلَى الدَّارِ الْخَارِجِيَّةِ لِيُقَدِّسُوا الشَّعْبَ».
\par 21 ثُمَّ أَخْرَجَنِي إِلَى الدَّارِ الْخَارِجِيَّةِ وَعَبَّرَنِي عَلَى زَوَايَا الدَّارِ الأَرْبَعِ, فَإِذَا فِي كُلِّ زَاوِيَةٍ مِنَ الدَّارِ دَارٌ.
\par 22 فِي زَوَايَا الدَّارِ الأَرْبَعِ دُورٌ مُصَوَّنَةٌ طُولُهَا أَرْبَعُونَ وَعَرْضُهَا ثَلاَثُونَ. لِلزَّوَايَا الأَرْبَعِ قِيَاسٌ وَاحِدٌ.
\par 23 وَمُحِيطَةٌ بِهَا حَافَةٌ حَوْلَ الأَرْبَعَةِ, وَمَطَابِخُ مَعْمُولَةٌ تَحْتَ الْحَافَاتِ الْمُحِيطَةِ بِهَا.
\par 24 ثُمَّ قَالَ لِي: [هَذَا بَيْتُ الطَّبَّاخِينَ حَيْثُ يَطْبُخُ خُدَّامُ الْبَيْتِ ذَبِيحَةَ الشَّعْبِ].

\chapter{47}

\par 1 ثُمَّ أَرْجَعَنِي إِلَى مَدْخَلِ الْبَيْتِ وَإِذَا بِمِيَاهٍ تَخْرُجُ مِنْ تَحْتِ عَتَبَةِ الْبَيْتِ نَحْوَ الْمَشْرِقِ, لأَنَّ وَجْهَ الْبَيْتِ نَحْوَ الْمَشْرِقِ. وَالْمِيَاهُ نَازِلَةٌ مِنْ تَحْتِ جَانِبِ الْبَيْتِ الأَيْمَنِ عَنْ جَنُوبِ الْمَذْبَحِ.
\par 2 ثُمَّ أَخْرَجَنِي مِنْ طَرِيقِ بَابِ الشِّمَالِ وَدَارَ بِي فِي الطَّرِيقِ مِنْ خَارِجٍ إِلَى الْبَابِ الْخَارِجِيِّ مِنَ الطَّرِيقِ الَّذِي يَتَّجِهُ نَحْوَ الْمَشْرِقِ, وَإِذَا بِمِيَاهٍ جَارِيَةٍ مِنَ الْجَانِبِ الأَيْمَنِ.
\par 3 وَعِنْدَ خُرُوجِ الرَّجُلِ نَحْوَ الْمَشْرِقِ وَالْخَيْطُ بِيَدِهِ, قَاسَ أَلْفَ ذِرَاعٍ وَعَبَّرَنِي فِي الْمِيَاهِ, وَالْمِيَاهُ إِلَى الْكَعْبَيْنِ.
\par 4 ثُمَّ قَاسَ أَلْفاً وَعَبَّرَنِي فِي الْمِيَاهِ, وَالْمِيَاهُ إِلَى الرُّكْبَتَيْنِ. ثُمَّ قَاسَ أَلْفاً وَعَبَّرَنِي, وَالْمِيَاهُ إِلَى الْحَقَوَيْنِ.
\par 5 ثُمَّ قَاسَ أَلْفاً, وَإِذَا بِنَهْرٍ لَمْ أَسْتَطِعْ عُبُورَهُ, لأَنَّ الْمِيَاهَ طَمَتْ, مِيَاهَ سِبَاحَةٍ, نَهْرٌٍ لاَ يُعْبَرُ.
\par 6 وَقَالَ لِي: [أَرَأَيْتَ يَا ابْنَ آدَمَ؟» ثُمَّ ذَهَبَ بِي وَأَرْجَعَنِي إِلَى شَاطِئِ النَّهْرِ.
\par 7 وَعِنْدَ رُجُوعِي إِذَا عَلَى شَاطِئِ النَّهْرِ أَشْجَارٌ كَثِيرَةٌ جِدّاً مِنْ هُنَا وَمِنْ هُنَاكَ.
\par 8 وَقَالَ لِي: [هَذِهِ الْمِيَاهُ خَارِجَةٌ إِلَى الدَّائِرَةِ الشَّرْقِيَّةِ وَتَنْزِلُ إِلَى الْعَرَبَةِ وَتَذْهَبُ إِلَى الْبَحْرِ. إِلَى الْبَحْرِ هِيَ خَارِجَةٌ فَتُشْفَى الْمِيَاهُ.
\par 9 وَيَكُونُ أَنَّ كُلَّ نَفْسٍ حَيَّةٍ تَدِبُّ حَيْثُمَا يَأْتِي النَّهْرَانِ تَحْيَا. وَيَكُونُ السَّمَكُ كَثِيراً جِدّاً لأَنَّ هَذِهِ الْمِيَاهَ تَأْتِي إِلَى هُنَاكَ فَتُشْفَى, وَيَحْيَا كُلُّ مَا يَأْتِي النَّهْرُ إِلَيْهِ.
\par 10 وَيَكُونُ الصَّيَّادُونَ وَاقِفِينَ عَلَيْهِ. مِنْ عَيْنِ جَدْيٍ إِلَى عَيْنِ عِجْلاَيِمَ يَكُونُ لِبَسْطِ الشِّبَاكِ, وَيَكُونُ سَمَكُهُمْ عَلَى أَنْوَاعِهِ كَسَمَكِ الْبَحْرِ الْعَظِيمِ كَثِيراً جِدّاً.
\par 11 أَمَّا غَمِقَاتُهُ وَبِرَكُهُ فَلاَ تُشْفَى. تُجْعَلُ لِلْمِلْحِ.
\par 12 وَعَلَى النَّهْرِ يَنْبُتُ عَلَى شَاطِئِهِ مِنْ هُنَا وَمِنْ هُنَاكَ كُلُّ شَجَرٍ لِلأَكْلِ, لاَ يَذْبُلُ وَرَقُهُ وَلاَ يَنْقَطِعُ ثَمَرُهُ. كُلَّ شَهْرٍ يُبَكِّرُ لأَنَّ مِيَاهَهُ خَارِجَةٌ مِنَ الْمَقْدِسِ, وَيَكُونُ ثَمَرُهُ لِلأَكْلِ وَوَرَقُهُ لِلدَّوَاءِ].
\par 13 هَكَذَا قَالَ السَّيِّدُ الرَّبُّ: [هَذَا هُوَ التُّخُمُ الَّذِي بِهِ تَمْتَلِكُونَ الأَرْضَ بِحَسَبِ أَسْبَاطِ إِسْرَائِيلَ الاِثْنَيْ عَشَرَ: يُوسُفُ قِسْمَانِ.
\par 14 وَتَمْتَلِكُونَهَا أَحَدُكُمْ كَصَاحِبِهِ الَّتِي رَفَعْتُ يَدِي لأُعْطِيَ آبَاءَكُمْ إِيَّاهَا. وَهَذِهِ الأَرْضُ تَقَعُ لَكُمْ نَصِيباً.
\par 15 وَهَذَا تُخُمُ الأَرْضِ: نَحْوَ الشِّمَالِ مِنَ الْبَحْرِ الْكَبِيرِ طَرِيقُ حِثْلُونَ إِلَى الْمَجِيءِ إِلَى صَدَدَ
\par 16 حَمَاةُ وَبَيْرُوثَةُ وَسِبْرَائِمُ (الَّتِي بَيْنَ تُخُمِ دِمَشْقَ وَتُخُمِ حَمَاةَ) وَحَصْرُ الْوُسْطَى (الَّتِي عَلَى تُخُمِ حَوْرَانَ).
\par 17 وَيَكُونُ التُّخُمُ مِنَ الْبَحْرِ حَصْرَ عِينَانَ تُخُمَ دِمَشْقَ وَالشِّمَالُ شِمَالاً وَتُخُمَ حَمَاةَ. وَهَذَا جَانِبُ الشِّمَالِ.
\par 18 وَجَانِبُ الشَّرْقِ بَيْنَ حَوْرَانَ وَدِمَشْقَ وَجِلْعَادَ وَأَرْضَ إِسْرَائِيلَ الأُرْدُنُّ. مِنَ التُّخُمِ إِلَى الْبَحْرِ الشَّرْقِيِّ تَقِيسُونَ. وَهَذَا جَانِبُ الْمَشْرِقِ.
\par 19 وَجَانِبُ الْجَنُوبِ يَمِيناً مِنْ ثَامَارَ إِلَى مِيَاهِ مَرِيبُوثَ قَادِشَ النَّهْرُ إِلَى الْبَحْرِ الْكَبِيرِ. وَهَذَا جَانِبُ الْيَمِينِ جَنُوباً.
\par 20 وَجَانِبُ الْغَرْبِ الْبَحْرُ الْكَبِيرُ مِنَ التُّخُمِ إِلَى مُقَابِلِ مَدْخَلِ حَمَاةَ. وَهَذَا جَانِبُ الْغَرْبِ.
\par 21 فَتَقْتَسِمُونَ هَذِهِ الأَرْضَ لَكُمْ لأَسْبَاطِ إِسْرَائِيلَ.
\par 22 وَيَكُونُ أَنَّكُمْ تَقْسِمُونَهَا بِالْقُرْعَةِ لَكُمْ وَلِلْغُرَبَاءِ الْمُتَغَرِّبِينَ فِي وَسَطِكُمُ الَّذِينَ يَلِدُونَ بَنِينَ فِي وَسَطِكُمْ, فَيَكُونُونَ لَكُمْ كَالْوَطَنِيِّينَ مِنْ بَنِي إِسْرَائِيلَ. يُقَاسِمُونَكُمُ الْمِيرَاثَ فِي وَسَطِ أَسْبَاطِ إِسْرَائِيلَ.
\par 23 وَيَكُونُ أَنَّهُ فِي السِّبْطِ الَّذِي فِيهِ يَتَغَرَّبُ غَرِيبٌ هُنَاكَ تُعْطُونَهُ مِيرَاثَهُ يَقُولُ السَّيِّدُ الرَّبُّ.

\chapter{48}

\par 1 [وَهَذِهِ أَسْمَاءُ الأَسْبَاطِ: مِنْ طَرَفِ الشِّمَالِ إِلَى جَانِبِ طَرِيقِ حِثْلُونَ إِلَى مَدْخَلِ حَمَاةَ حَصْرُ عِينَانَ تُخُمُ دِمَشْقَ شِمَالاً إِلَى جَانِبِ حَمَاةَ لِدَانٍ. فَيَكُونُ لَهُ مِنَ الشَّرْقِ إِلَى الْبَحْرِ قِسْمٌ وَاحِدٌ.
\par 2 وَعَلَى تُخُمِ دَانٍ مِنْ جَانِبِ الْمَشْرِقِ إِلَى جَانِبِ الْبَحْرِ لأَشِيرَ قِسْمٌ وَاحِدٌ.
\par 3 وَعَلَى تُخُمِ أَشِيرَ مِنْ جَانِبِ الشَّرْقِ إِلَى جَانِبِ الْبَحْرِ لِنَفْتَالِي قِسْمٌ وَاحِدٌ.
\par 4 وَعَلَى تُخُمِ نَفْتَالِي مِنْ جَانِبِ الشَّرْقِ إِلَى جَانِبِ الْبَحْرِ لِمَنَسَّى قِسْمٌ وَاحِدٌ.
\par 5 وَعَلَى تُخُمِ مَنَسَّى مِنْ جَانِبِ الشَّرْقِ إِلَى جَانِبِ الْبَحْرِ لأَفْرَايِمَ قِسْمٌ وَاحِدٌ.
\par 6 وَعَلَى تُخُمِ أَفْرَايِمَ مِنْ جَانِبِ الشَّرْقِ إِلَى جَانِبِ الْبَحْرِ لِرَأُوبَيْنَ قِسْمٌ وَاحِدٌ.
\par 7 وَعَلَى تُخُمِ رَأُوبَيْنَ مِنْ جَانِبِ الشَّرْقِ إِلَى جَانِبِ الْبَحْرِ لِيَهُوذَا قِسْمٌ وَاحِدٌ.
\par 8 وَعَلَى تُخُمِ يَهُوذَا مِنْ جَانِبِ الشَّرْقِ إِلَى جَانِبِ الْبَحْرِ تَكُونُ التَّقْدِمَةُ الَّتِي تُقَدِّمُونَهَا خَمْسَةً وَعِشْرِينَ أَلْفاً عَرْضاً, وَالطُّولُ كَأَحَدِ الأَقْسَامِ مِنْ جَانِبِ الشَّرْقِ إِلَى جَانِبِ الْبَحْرِ, وَيَكُونُ الْمَقْدِسُ فِي وَسَطِهَا.
\par 9 التَّقْدِمَةُ الَّتِي تُقَدِّمُونَهَا لِلرَّبِّ تَكُونُ خَمْسَةً وَعِشْرِينَ أَلْفاً طُولاً وَعَشَرَةَ آلاَفٍ عَرْضاً.
\par 10 وَلِهَؤُلاَءِ تَكُونُ تَقْدِمَةَ الْقُدْسِ لِلْكَهَنَةِ. مِنْ جِهَةِ الشِّمَالِ خَمْسَةٌ وَعِشْرُونَ أَلْفاً فِي الطُّولِ, وَمِنْ جِهَةِ الْبَحْرِ عَشَرَةُ آلاَفٍ فِي الْعَرْضِ, وَمِنْ جِهَةِ الشَّرْقِ عَشَرَةُ آلاَفٍ فِي الْعَرْضِ, وَمِنْ جِهَةِ الْجَنُوبِ خَمْسَةٌ وَعِشْرُونَ أَلْفاً فِي الطُّولِ. وَيَكُونُ مَقْدِسُ الرَّبِّ فِي وَسَطِهَا.
\par 11 أَمَّا الْمُقَدَّسُ فَلِلْكَهَنَةِ مِنْ بَنِي صَادُوقَ الَّذِينَ حَرَسُوا حِرَاسَتِي, الَّذِينَ لَمْ يَضِلُّوا حِينَ ضَلَّ بَنُو إِسْرَائِيلَ كَمَا ضَلَّ اللاَّوِيُّونَ.
\par 12 وَتَكُونُ لَهُمْ تَقْدِمَةً مِنْ تَقْدِمَةِ الأَرْضِ, قُدْسُ أَقْدَاسٍ عَلَى تُخُمِ اللاَّوِيِّينَ.
\par 13 [وَلِلاَّوِيِّينَ عَلَى مُوازَاةِ تُخُمِ الْكَهَنَةِ خَمْسَةٌ وَعِشْرُونَ أَلْفاً فِي الطُّولِ, وَعَشَرَةُ آلاَفٍ فِي الْعَرْضِ. الطُّولُ كُلُّهُ خَمْسَةٌ وَعِشْرُونَ أَلْفاً, وَالْعَرْضُ عَشَرَةُ آلاَفٍ.
\par 14 وَلاَ يَبِيعُونَ مِنْهُ وَلاَ يُبَدِّلُونَ, وَلاَ يَصْرِفُونَ بَاكُورَاتِ الأَرْضِ لأَنَّهَا مُقَدَّسَةٌ لِلرَّبِّ.
\par 15 وَالْخَمْسَةُ الآلاَفِ الْفَاضِلَةُ مِنَ الْعَرْضِ قُدَّامَ الْخَمْسَةِ وَالْعِشْرِينَ أَلْفاً هِيَ مُحَلَّلَةٌ لِلْمَدِينَةِ لِلسُّكْنَى وَلِلْمَسْرَحِ, وَالْمَدِينَةُ تَكُونُ فِي وَسَطِهَا.
\par 16 وَهَذِهِ أَقْيِسَتُهَا: جَانِبُ الشِّمَالِ أَرْبَعَةُ آلاَفٍ وَخَمْسُ مِئَةٍ, وَجَانِبُ الْجَنُوبِ أَرْبَعَةُ آلاَفٍ وَخَمْسُ مِئَةٍ وَجَانِبُ الشَّرْقِ أَرْبَعَةُ آلاَفٍ وَخَمْسُ مِئَةٍ وَجَانِبُ الْغَرْبِ أَرْبَعَةُ آلاَفٍ وَخَمْسُ مِئَةٍ.
\par 17 وَيَكُونُ مَسْرَحٌ لِلْمَدِينَةِ نَحْوَ الشِّمَالِ مِئَتَيْنِ وَخَمْسِينَ, وَنَحْوَ الْجَنُوبِ مِئَتَيْنِ وَخَمْسِينَ, وَنَحْوَ الشَّرْقِ مِئَتَيْنِ وَخَمْسِينَ, وَنَحْوَ الْغَرْبِ مِئَتَيْنِ وَخَمْسِينَ.
\par 18 وَالْبَاقِي مِنَ الطُّولِ مُوازِياً تَقْدِمَةَ الْقُدْسِ عَشَرَةُ آلاَفٍ نَحْوَ الشَّرْقِ, وَعَشَرَةُ آلاَفٍ نَحْوَ الْغَرْبِ. وَيَكُونُ مُوازِياً تَقْدِمَةَ الْقُدْسِ, وَغَلَّتُهُ تَكُونُ أَكْلاً لِخِدْمَةِ الْمَدِينَةِ.
\par 19 أَمَّا خَدَمَةُ الْمَدِينَةِ فَيَخْدِمُونَهَا مِنْ كُلِّ أَسْبَاطِ إِسْرَائِيلَ.
\par 20 كُلُّ التَّقْدِمَةِ خَمْسَةٌ وَعِشْرُونَ أَلْفاً بِخَمْسَةٍ وَعِشْرِينَ أَلْفاً. مُرَبَّعَةً تُقَدِّمُونَ تَقْدِمَةَ الْقُدْسِ مَعَ مُلْكِ الْمَدِينَةِ.
\par 21 وَالْبَقِيَّةُ لِلرَّئِيسِ مِنْ هُنَا وَمِنْ هُنَاكَ لِتَقْدِمَةِ الْقُدْسِ وَلِمُلْكِ الْمَدِينَةِ قُدَّامَ الْخَمْسَةِ وَالْعِشْرِينَ أَلْفاً لِلتَّقْدِمَةِ إِلَى تَُخُمِ الشَّرْقِ, وَمِنْ جِهَةِ الْغَرْبِ قُدَّامَ الْخَمْسَةِ وَالْعِشْرِينَ أَلْفاً عَلَى تُخُمِ الْغَرْبِ مُوازِياً أَمْلاَكَ الرَّئِيسِ, وَتَكُونُ تَقْدِمَةُ الْقُدْسِ وَمَقْدِسُ الْبَيْتِ فِي وَسَطِهَا.
\par 22 وَمِنْ مُلْكِ اللاَّوِيِّينَ مِنْ مُلْكِ الْمَدِينَةِ فِي وَسَطِ الَّذِي هُوَ لِلرَّئِيسِ, مَا بَيْنَ تُخُمِ يَهُوذَا وَتُخُمِ بِنْيَامِينَ, يَكُونُ لِلرَّئِيسِ.
\par 23 وَبَاقِي الأَسْبَاطِ: فَمِنْ جَانِبِ الشَّرْقِ إِلَى جَانِبِ الْبَحْرِ لِبِنْيَامِينَ قِسْمٌ وَاحِدٌ.
\par 24 وَعَلَى تُخُمِ بِنْيَامِينَ, مِنْ جَانِبِ الشَّرْقِ إِلَى جَانِبِ الْبَحْرِ لِشَمْعُونَ قِسْمٌ وَاحِدٌ.
\par 25 وَعَلَى تُخُمِ شَمْعُونَ مِنْ جَانِبِ الشَّرْقِ إِلَى جَانِبِ الْبَحْرِ لِيَسَّاكَرَ قِسْمٌ وَاحِدٌ.
\par 26 وَعَلَى تُخُمِ يَسَّاكَرَ مِنْ جَانِبِ الشَّرْقِ إِلَى جَانِبِ الْبَحْرِ لِزَبُولُونَ قِسْمٌ وَاحِدٌ.
\par 27 وَعَلَى تُخُمِ زَبُولُونَ مِنْ جَانِبِ الشَّرْقِ إِلَى جَانِبِ الْبَحْرِ لِجَادٍ قِسْمٌ وَاحِدٌ.
\par 28 وَعَلَى تُخُمِ جَادٍ مِنْ جَانِبِ الْجَنُوبِ يَمِيناً يَكُونُ التُّخُمُ مِنْ ثَامَارَ إِلَى مِيَاهِ مَرِيبَةَِ قَادِشِ النَّهْرِ إِلَى الْبَحْرِ الْكَبِيرِ.
\par 29 هَذِهِ هِيَ الأَرْضُ الَّتِي تَقْسِمُونَهَا مُلْكاً لأَسْبَاطِ إِسْرَائِيلَ, وَهَذِهِ حِصَصُهُمْ يَقُولُ السَّيِّدُ الرَّبُّ.
\par 30 [وَهَذِهِ مَخَارِجُ الْمَدِينَةِ: مِنْ جَانِبِ الشِّمَالِ أَرْبَعَةُ آلاَفٍ وَخَمْسُ مِئَةِ مِقْيَاسٍ.
\par 31 (وَأَبْوَابُ الْمَدِينَةِ عَلَى أَسْمَاءِ أَسْبَاطِ إِسْرَائِيلَ). ثَلاَثَةُ أَبْوَابٍ نَحْوَ الشِّمَالِ: بَابُ رَأُوبَيْنَ وَبَابُ يَهُوذَا وَبَابُ لاَوِي.
\par 32 وَإِلَى جَانِبِ الشَّرْقِ أَرْبَعَةُ آلاَفٍ وَخَمْسُ مِئَةٍ. وَثَلاَثَةُ أَبْوَابٍ: بَابُ يُوسُفَ وَبَابُ بِنْيَامِينَ وَبَابُ دَانٍ.
\par 33 وَجَانِبُ الْجَنُوبِ أَرْبَعَةُ آلاَفٍ وَخَمْسُ مِئَةِ مِقْيَاسٍ. وَثَلاَثَةُ أَبْوَابٍ: بَابُ شَمْعُونَ وَبَابُ يَسَّاكَرَ وَبَابُ زَبُولُونَ.
\par 34 وَجَانِبُ الْغَرْبِ أَرْبَعَةُ آلاَفٍ وَخَمْسُ مِئَةٍ. وَثَلاَثَةُ أَبْوَابٍ: بَابُ جَادٍ وَبَابُ أَشِيرَ وَبَابُ نَفْتَالِي.
\par 35 الْمُحِيطُ ثَمَانِيَةَ عَشَرَ أَلْفاً, وَاسْمُ الْمَدِينَةِ مِنْ ذَلِكَ الْيَوْمِ [يَهْوَهْ شَمَّهْ».

\end{document}