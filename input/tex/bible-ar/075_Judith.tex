\begin{document}

\title{جوديث}


\chapter{1}

\par 1 وفي السنة الثانية عشرة من ملك نبوخذنصر الذي ملك في نينوى المدينة العظيمة، في أيام أرفكشاد الذي ملك على الميديين في أحمتان،
\par 2 وبنى في إكباتان أسوارًا مستديرة من حجارة منحوتة، عرضها ثلاثة أذرع وطولها ستة أذرع، وجعل ارتفاع السور سبعين ذراعًا، وعرضه خمسين ذراعًا
\par 3 وجعل أبراجها على أبوابها مئة ذراع ارتفاعًا، وعرضها في الأساس ستون ذراعًا
\par 4 وعمل أبوابها أبوابًا مرتفعة إلى ارتفاع سبعين ذراعًا، وعرضها أربعون ذراعًا، لخروج جيوشه العظيمة، ولصف مشاته
\par 5 حتى في تلك الأيام، خاض الملك نبوخذنصر حربًا مع الملك أرفكشاد في السهل العظيم، وهو السهل الواقع في حدود راغاو
\par 6 فجاء إليه جميع سكان الجبال، وجميع سكان الفرات ودجلة وهيدسبيس وسهول أريوك ملك إيليمان، وأمم كثيرة جدًا من بني خلود، واجتمعوا للقتال
\par 7 ثم أرسل نبوخذنصر ملك أشور إلى جميع سكان فارس، وجميع سكان الغرب، وسكان كيليكية، ودمشق، ولبنان، وبلدة لبنان، وجميع سكان ساحل البحر،
\par 8 وإلى أولئك الذين من بين الأمم الذين كانوا من الكرمل، وجلعاد، والجليل الأعلى، وسهل يسدريلوم العظيم،
\par 9 وإلى كل من في السامرة ومدنها، وفي عبر الأردن إلى أورشليم، وبيتان، وخلوس، وقادس، ونهر مصر، وتفتنيس، ورعمسى، وكل أرض جاسم،
\par 10 حتى تصلوا إلى ما وراء تانيس وممفيس، وإلى جميع سكان مصر، حتى تصلوا إلى حدود إثيوبيا
\par 11 لكن جميع سكان الأرض استهانوا بأمر نبوخذنصر ملك آشور، ولم يذهبوا معه إلى المعركة؛ لأنهم لم يخافوه. نعم، كان أمامهم كرجل واحد، وأرسلوا سفراءه عنهم دون جدوى، وبخزي
\par 12 لذلك غضب نبوخذنصر غضبًا شديدًا على كل هذه البلاد، وأقسم بعرشه ومملكته أنه سينتقم انتقامًا لا محالة من جميع سواحل كيليكية ودمشق وسوريا، وأنه سيقتل بحد السيف جميع سكان أرض موآب وبني عمون وكل يهودا وكل ما في مصر حتى تصلوا إلى حدود البحرين
\par 13 ثم سار في صفوف قتالية بقوته على الملك أرفكشاد في السنة السابعة عشرة، وانتصر في معركته، لأنه هزم كل قوة أرفكشاد، وكل فرسانه، وكل مركباته،
\par 14 وأصبح سيدًا لمدنه، وجاء إلى إكباتان، واستولى على الأبراج، ونهب شوارعها، وحول جمالها إلى عار
\par 15 وأخذ أرفكشاد أيضًا في جبال راغاو وضربه بسهامه وحرمه في ذلك اليوم.
\par 16 ثم عاد بعد ذلك إلى نينوى، وكان هو وكل جماعته من الأمم المختلفة جمهورًا كبيرًا جدًا من رجال الحرب، وهناك استراح، وأقام وليمة هو وجيشه مئة وعشرين يومًا

\chapter{2}

\par 1 وفي السنة الثامنة عشرة، في اليوم الثاني والعشرين من الشهر الأول، دار حديث في بيت نبوخذنصر ملك أشور أنه، كما قال، سينتقم لنفسه من كل الأرض
\par 2 فدعا إليه جميع ضباطه وجميع نبلائه، وأبلغهم بمشورته السرية، واختتم بفمه إِذْلال الأرض كلها
\par 3 ثم أمروا بإهلاك كل ذي جسد لا يطيع أمر فمه
\par 4 ولما انتهى من مشورته، دعا نبوخذنصر ملك آشور أليفانا رئيس جيشه الذي كان بجانبه، وقال له
\par 5 هكذا قال الملك العظيم، سيد الأرض كلها: ها أنت تخرج من أمامي، وتأخذ معك رجالاً يعتمدون على قوتهم، من المشاة مئة وعشرين ألفًا، وعدد الخيل وفرسانها اثنا عشر ألفًا
\par 6 وتذهب ضد كل بلاد الغرب، لأنهم عصوا أمري
\par 7 وتُخبرهم أنهم يُهيئون لي ترابًا وماءً، لأني أخرج بغضبي عليهم، وأُغطي كل وجه الأرض بأقدام جيشي، وأُسلمهم لهم غنيمة
\par 8 لكي تملأ قتلاهم أوديتهم وجداولهم، ويمتلئ النهر بموتاهم حتى يفيض:
\par 9 وسأسبيهم إلى أقصى بقاع الأرض كلها.
\par 10 "لذلك تخرج وتأخذ لي جميع تخومهم مقدما. فإن سلموا أنفسهم لك، فإنك تحتفظ بها لي إلى يوم عقابهم."
\par 11 وأما المتمردون فلا تشفق عينك عليهم، بل اذبحهم وسلبهم أينما ذهبت
\par 12 لأني حيٌّ أنا، وبقوة ملكوتي، فإن كل ما تكلمت به سأفعله بيدي
\par 13 واحذر أن تتعدى شيئًا من وصايا ربك، بل أتمها تمامًا كما أمرتك، ولا تتأخر عن فعلها
\par 14 ثم خرج أليفانا من عند سيده، ودعا جميع الولاة والرؤساء وضباط جيش أشور
\par 15 وحشد الرجال المختارين للمعركة، كما أمره سيده، إلى مائة وعشرين ألفًا، واثني عشر ألفًا من رماة السهام على ظهور الخيل؛
\par 16 ورتبهم كما يرتب جيش عظيم للحرب.
\par 17 فأخذ جمالًا وحميرًا لعرباتهم، عددًا كبيرًا جدًا، وغنمًا وبقرًا ومعزًا لا عدد لها لزادهم
\par 18 وكان طعام كثير لكل رجل من رجال الجيش، وذهب وفضة كثيرين جداً من بيت الملك.
\par 19 ثم خرج بكل قوته ليذهب أمام الملك نبوخذنصر في الرحلة، وليغطي كل وجه الأرض غربًا بمركباتهم وفرسانهم ومشاةهم المختارين
\par 20 وجاء معهم أيضًا عدد كبير من بلدان مختلفة كالجراد ومثل رمل الأرض، لأن الجمع كان بلا عدد
\par 21 وخرجوا من نينوى مسيرة ثلاثة أيام نحو سهل بكتيلث، ونزلوا من بكتيلث بالقرب من الجبل الذي على يسار كيليكية العليا
\par 22 ثم أخذ كل جيشه، مشاته وفرسانه ومركباته، وانطلق من هناك إلى منطقة الجبال؛
\par 23 ودمر فؤاد ولود، وسلب جميع بني رسس وبني إسرائيل الذين كانوا نحو البرية جنوبي أرض الكيليان
\par 24 ثم عبر الفرات، وجاب بلاد ما بين النهرين، ودمر جميع المدن المرتفعة التي على نهر أربوناي، حتى وصل إلى البحر
\par 25 واستولى على حدود كيليكية، وقتل كل من قاومه، وجاء إلى حدود يافث، التي كانت نحو الجنوب، مقابل الجزيرة العربية
\par 26 وأحاط أيضًا بجميع بني مديان، وأحرق خيامهم، ونهب حظائر غنمهم
\par 27 ثم نزل إلى سهل دمشق في وقت حصاد القمح، وأحرق جميع حقولهم، وأهلك غنمهم وبقرهم، ونهب مدنهم، ودمر أراضيهم، وضرب جميع شبانهم بحد السيف
\par 28 فوقع خوفه ورعبه على جميع سكان سواحل البحر الذين في صيدا وصور، وسكان صور وأوكينا، وجميع سكان يمنان. وخافه سكان أشدود وعسقلان خوفًا شديدًا

\chapter{3}

\par 1 فأرسلوا إليه رسلاً للصلح قائلين:
\par 2 هوذا نحن عبيد نبوخذنصر الملك العظيم راقدون أمامك، فاستخدمنا كما يحسن في عينيك
\par 3 هوذا بيوتنا، وجميع أماكننا، وجميع حقول قمحنا، وقطعاننا، وبقرنا، وجميع مساكن خيامنا أمام وجهك. فاستخدمها كما يحلو لك
\par 4 هوذا مدننا وسكانها عبيدك. تعال وافعل بهم ما يحسن في عينيك
\par 5 فجاء الرجال إلى أليفارنيس، وأخبروه بهذا الكلام
\par 6 ثم نزل نحو ساحل البحر هو وجيشه، وأقام حاميات في المدن العليا، وأخذ منها رجالاً مختارين للعون
\par 7 فاستقبلهم هم وجميع أهل البلاد المحيطة بالأكاليل والرقصات والدفوف
\par 8 ولكنه هدم حدودهم وقطع بساتينهم لأنه أمر بتدمير جميع آلهة الأرض، وأن تعبد جميع الأمم نبوخذنصر وحده، وأن تدعوه جميع الألسنة والقبائل كإله.
\par 9 وجاء أيضًا مقابل عزرالون بالقرب من يهودا، مقابل مضيق يهودا العظيم
\par 10 ونزل بين جبع وسكيثوبوليس، ومكث هناك شهرًا كاملاً، ليجمع كل مركبات جيشه

\chapter{4}

\par 1 فسمع بنو إسرائيل الساكنون في اليهودية بكل ما فعله أليفانا رئيس قواد نبوخذنصر ملك أشور بالأمم، وكيف نهب جميع هياكلهم وأبادهم
\par 2 لذلك خافوا منه خوفًا شديدًا، واضطربوا على أورشليم وعلى هيكل الرب إلههم
\par 3 لأنهم كانوا قد رجعوا حديثًا من السبي، وجمع كل شعب اليهودية حديثًا. وقدست الآنية والمذبح والبيت بعد التدنيس
\par 4 فأرسلوا إلى جميع تخوم السامرة، والقرى، وإلى بيت حورون، وبلمن، وأريحا، وخوبا، وعاسورة، ووادي ساليم،
\par 5 وامتلكوا مسبقًا جميع رؤوس الجبال العالية، وحصنوا القرى التي فيها، وادخروا مؤونة للحرب، لأن حقولهم حُصدت مؤخرًا
\par 6 وكتب أيضًا يوياقيم رئيس الكهنة الذي كان في تلك الأيام في أورشليم، إلى الساكنين في بيت فلوى وبيت ميسثام التي مقابل عزرالون في ناحية البرية التي بالقرب من دوثايم،
\par 7 تكليفهم بحراسة ممرات الجبال، لأنه كان منها مدخل إلى اليهودية، وكان من السهل إيقاف الصاعدين، لأن الممر كان مستقيمًا، لرجلين على الأكثر
\par 8 ففعل بنو إسرائيل كما أمرهم يوياقيم رئيس الكهنة، مع شيوخ كل شعب إسرائيل الساكنين في أورشليم
\par 9 فصرخ كل رجل من إسرائيل إلى الله بحرارة عظيمة، وتواضعوا بشدة عظيمة.
\par 10 وهم ونساؤهم وأبناؤهم ومواشيهم وكل غريب وأجير وعبيدهم المبتاعين بفضة وضعوا المسوح على أحقائهم.
\par 11 وهكذا سقط كل رجل وامرأة، والأطفال، وسكان أورشليم أمام الهيكل، وذروا الرماد على رؤوسهم، وفرشوا مسوحهم أمام وجه الرب. ووضعوا المسوح حول المذبح،
\par 12 وصرخوا إلى إله إسرائيل جميعًا بصوت واحد، أن لا يعطي أولادهم غنيمة، ولا نساءهم غنيمة، ولا مدن ميراثهم للدمار، ولا المقدسات للتدنيس والعار، ولا الأمم للفرح بها
\par 13 فاستمع الله صلواتهم ونظر إلى ضيقاتهم، لأن الشعب صاموا أياماً كثيرة في كل اليهودية وأورشليم أمام مقدس الرب القدير.
\par 14 وكان يهوياقيم رئيس الكهنة، وجميع الكهنة الواقفين أمام الرب، والذين يخدمون الرب، قد شدوا أحقاءهم بمسوح، وكانوا يقدمون المحرقات اليومية مع نذور الشعب وعطاياه
\par 15 وكان الرماد على عماماتهم، وصرخوا إلى الرب بكل قوتهم، أن ينظر إلى كل بيت إسرائيل برحمة

\chapter{5}

\par 1 ثم أُخبر أليفانا، رئيس جيش أشور، أن بني إسرائيل قد استعدوا للحرب، وأغلقوا معابر الجبال، وحصنوا جميع رؤوس التلال العالية، ووضعوا متاريس في بلاد الشام
\par 2 فاستشاط غضبًا، ودعا جميع رؤساء موآب، وقادة بني عمون، وجميع حكام ساحل البحر،
\par 3 فقال لهم: أخبروني يا بني كنعان، من هو هذا الشعب الساكن في الجبال، وما هي المدن التي يسكنونها، وما هي كثرة جيشهم، وفي أي قوتهم وسلطانهم، وأي ملك مُوَكَّل عليهم، أو رئيس جيشهم؟
\par 4 ولماذا قرروا عدم المجيء لمقابلتي، أكثر من جميع سكان الغرب
\par 5 فقال أحيور، رئيس جميع بني عمون: «ليسمع سيدي كلمة من فم عبدك، فأخبرك بالحق عن هذا الشعب الساكن بالقرب منك والساكن في الجبال، ولا يخرج كذب من فم عبدك».
\par 6 هذا الشعب ينحدر من الكلدانيين:
\par 7 وكانوا قبل ذلك متغربين في بلاد ما بين النهرين، لأنهم لم يريدوا أن يسيروا وراء آلهة آبائهم الذين في أرض الكلدانيين.
\par 8 لأنهم تركوا طريق آبائهم، وعبدوا إله السماء، الإله الذي عرفوه. فطردوهم من أمام آلهتهم، وهربوا إلى بلاد ما بين النهرين، وأقاموا هناك أيامًا كثيرة
\par 9 ثم أمرهم إلههم بالخروج من المكان الذي تغربوا فيه، والذهاب إلى أرض كنعان، حيث سكنوا، وازدادوا ذهبًا وفضةً ومواشي كثيرة جدًا
\par 10 ولما غطت المجاعة كل أرض كنعان، نزلوا إلى مصر، وأقاموا هناك حتى تغذوا، وصاروا هناك جمعًا كبيرًا، حتى لم يكن أحد يستطيع إحصاء أمتهم
\par 11 لذلك قام ملك مصر عليهم، وعاملهم بمكر، وأذلهم بالعمل في صناعة الآجر، وجعلهم عبيدًا
\par 12 فصرخوا إلى إلههم، فضرب كل أرض مصر بأوبئة لا شفاء منها، فطرحهم المصريون من أمام أعينهم
\par 13 فجفف الله البحر الأحمر أمامهم،
\par 14 وأحضرهم إلى جبل سيناء وقادش وأخرج جميع سكان البرية.
\par 15 فسكنوا في أرض الأموريين، وأهلكوا بقوتهم جميع أهل حشبون، وعبروا الأردن وامتلكوا كل الجبال
\par 16 فطردوا من أمامهم الكنعانيين والفرزيين واليبوسيين والشكيميين وجميع الجرجسيين، وأقاموا في تلك الأرض أيامًا كثيرة
\par 17 وبينما لم يخطئوا أمام إلههم، نجحوا، لأن الإله الذي يبغض الإثم كان معهم
\par 18 ولكن لما حادوا عن الطريق الذي رسمه لهم، هلكوا في معارك كثيرة شرسة، وساقوا أسرى إلى أرض ليست لهم، ودُمر هيكل إلههم، واستولى الأعداء على مدنهم
\par 19 ولكنهم الآن قد رجعوا إلى إلههم، وصعدوا من الأماكن التي تشتتوا فيها، وورثوا أورشليم حيث مقدسهم، وجلسوا في الجبل لأنها كانت خربة
\par 20 والآن يا سيدي والوالي، إن كان هناك أي خطأ ضد هذا الشعب، وأخطأوا إلى إلههم، فلنعتبر أن هذا سيكون هلاكهم، ولنصعد فنتغلب عليهم
\par 21 ولكن إن لم يكن هناك إثم في أمتهم، فليمر سيدي الآن، لئلا يدافع عنهم ربهم، ويكون إلههم لهم، فنصبح عارا أمام العالم أجمع
\par 22 ولما فرغ أحيور من هذه الأقوال، تذمر جميع الشعب الواقفين حول الخيمة، وتكلم رؤساء أليفانا وجميع سكان البحر وفي موآب بأن يقتلوه
\par 23 لأنهم يقولون إننا لا نخاف من وجه بني إسرائيل، لأنه هوذا شعب لا قوة له ولا سلطان على معركة شديدة
\par 24 والآن، يا سيد هولوفرنيس، سنصعد، وسيكونون فريسة يلتهمها كل جيشك

\chapter{6}

\par 1 ولما هدأت ضجة الرجال الذين كانوا حول المجلس، قال أليفانا، رئيس جيش أشور، لأحيور وجميع الموآبيين أمام كل جماعة الأمم الأخرى،
\par 2 ومن أنت يا أحيور وأجراء أفرايم، حتى أنك تنبأت علينا اليوم وقلت: لا نحارب بني إسرائيل لأن إلههم سيدافع عنهم؟ ومن هو الله إلا نبوخذنصر؟
\par 3 يُرسِلُ قُوَّتَهُ، وَيُهْلِكُهُمْ عَنِ وَجْهِ الأَرْضِ، وَلاَ يُنْقِذُهُمْ إِلهُهُمْ. وَأَمَّا نَحْنُ عِبَادُهُ فَنَهْلِكُهُمْ كَرِجْلٍ وَاحِدٍ، لأَنَّهُمْ لاَ يَقْدِرُونَ أَنْ يُحْمِلُوا قُوَّةً لِخَيْلِنَا
\par 4 "فبهم ندوسهم، فتسكر جبالهم من دمائهم، وتمتلئ حقولهم من جثثهم، ولا تستطيع آثار أقدامهم أن تقف أمامنا، لأنهم يبيدون إلى الأبد، يقول الملك نبوخذنصر سيد كل الأرض، لأنه قال: لا يكون شيء من كلامي باطلا".
\par 5 وأنت يا أخيور، أجير عمون، الذي تكلمت بهذا الكلام في يوم إثمك، لن ترى وجهي بعد من هذا اليوم، حتى أنتقم من هذه الأمة التي خرجت من مصر
\par 6 وحينئذٍ يمر سيف جيشي، وجمهور الذين يخدمونني، من بين جنبيك، وتسقط بين قتلاهم عند عودتي
\par 7 والآن سيعيدك عبيدي إلى الجبل، ويسكنونك في إحدى مدن الممرات
\par 8 ولن تهلك حتى تهلك معهم.
\par 9 وإن أقنعت نفسك في قلبك أنهم سيؤخذون فلا يسقط وجهك. لقد تكلمت، وكل كلامي لن يذهب عبثا.
\par 10 ثم أمر أليفانا عبيده النائمين في خيمته أن يأخذوا أخيور ويأتوا به إلى بيت فلوى ويسلموه إلى أيدي بني إسرائيل
\par 11 فأخذه عبيده وأخرجوه من المحلة إلى السهل، وذهبوا من وسط السهل إلى الجبل، وأتوا إلى العيون التي تحت بيت فلوى
\par 12 فلما رآهم رجال المدينة، أخذوا سلاحهم وخرجوا إلى خارج المدينة إلى رأس التل، وكان كل رجل يحمل مقلاعًا يمنعهم من الصعود برجمهم بالحجارة
\par 13 ومع ذلك، بعد أن دخلوا خلسةً تحت التل، ربطوا أحيور وألقوه إلى أسفل، وتركوه عند سفح التل، ثم رجعوا إلى سيدهم
\par 14 فنزل بنو إسرائيل من مدينتهم، وأتوا إليه، وحلوه، وأتوا به إلى بيت فلوى، وأسلموه إلى حكام المدينة
\par 15 الذين كانوا في تلك الأيام عزيا بن ميخا من سبط شمعون، وخافريس بن جوثونيئيل، وكارميس بن ملكيئيل
\par 16 فجمعوا كل شيوخ المدينة، وركض كل شبابهم ونساؤهم إلى المجمع، وأقاموا أحيور في وسط كل شعبهم. فسأله عزيا عما جرى
\par 17 فأجاب وأخبرهم بكلام مجلس أليفانا، وكل الكلام الذي تكلم به في وسط رؤساء أشور، وكل ما تكلم به أليفانا بتكبر على بيت إسرائيل
\par 18 فخرّ الشعب وسجدوا لله، وصرخوا إلى الله قائلين:
\par 19 يا رب إله السماء، انظر إلى كبريائهم، واشفق على وضع أمتنا المتواضع، وانظر إلى وجوه أولئك الذين قُدِّسوا لك هذا اليوم
\par 20 ثم عزوا أحيور ومدحوه كثيرًا.
\par 21 فأخرجه عزيا من الجماعة إلى بيته وصنع وليمة للشيوخ، ودعوا إله إسرائيل ذلك الليل كله.

\chapter{7}

\par 1 وفي الغد أمر أليفانا كل جيشه وكل شعبه الذين جاؤوا لمناصرته أن ينقلوا معسكرهم إلى بيت فلوى، وأن يستولوا على مرتفعات الجبل، وأن يحاربوا بني إسرائيل
\par 2 ثم رحل أبطالهم عن معسكراتهم في ذلك اليوم، وكان جيش رجال الحرب مئة وسبعين ألف راجل، واثني عشر ألف فارس، بالإضافة إلى الأمتعة، ورجال آخرين مشاة بينهم، جمع عظيم جدًا
\par 3 ونزلوا في الوادي بالقرب من بيت فلوى، عند العين، وامتدوا في العرض على دوثايم حتى بلمايم، وفي الطول من بيت فلوى إلى كينامون، التي مقابل عزرالون
\par 4 فلما رأوا بنو إسرائيل كثرة عددهم، اضطربوا جدًا، وقال كل واحد لصاحبه: الآن يلعق هؤلاء الرجال وجه الأرض، لأنه لا الجبال العالية ولا الأودية ولا التلال تستطيع أن تحمل ثقلهم
\par 5 ثم حمل كل رجل أسلحته الحربية، وأشعلوا النيران في أبراجهم، وبقوا يراقبون طوال تلك الليلة
\par 6 وفي اليوم الثاني أخرج أليفانا جميع فرسانه أمام أعين بني إسرائيل الذين في بيت فلوى،
\par 7 ونظر إلى الممرات المؤدية إلى المدينة، وجاء إلى ينابيع مياهها، فاستول عليها، وجعل عليها حاميات من رجال الحرب، ثم سار هو نحو قومه
\par 8 فتقدم إليه جميع رؤساء بني عيسو، وجميع حكام شعب موآب، وقواد ساحل البحر، وقالوا:
\par 9 فليسمع سيدنا الآن كلمة، حتى لا يكون هناك انقلاب في جيشك
\par 10 لأن هذا الشعب من بني إسرائيل لا يتكل على رماحهم، بل على ارتفاع الجبال التي يسكنون فيها، لأنه ليس من السهل الصعود إلى قمم جبالهم
\par 11 والآن يا سيدي، لا تقاتلهم في صفوف قتالية، ولن يهلك من شعبك رجل واحد
\par 12 ابق في معسكرك، واحفظ جميع رجال جيشك، وليأخذ عبيدك بأيديهم نبع الماء الذي يخرج من سفح الجبل
\par 13 لأن جميع سكان بيت فلوى يحصلون على مياههم من هناك، فيقتلهم العطش، ويتخلون عن مدينتهم، وسنصعد نحن وشعبنا إلى قمم الجبال القريبة، ونخيم عليها، لنراقب لئلا يخرج أحد من المدينة
\par 14 فيُحرقون هم ونساؤهم وأطفالهم بالنار، وقبل أن يأتي السيف عليهم يُطرحون في الشوارع التي يسكنون فيها
\par 15 هكذا تجازيهم جزاءً سيئًا، لأنهم تمردوا ولم يلتقوا بشخصك بسلام
\par 16 فحسن هذا الكلام في عيني أليفانا وفي عيني كل عبيده، فأوصى أن يفعلوا كما تكلموا.
\par 17 فارتحل جيش بني عمون ومعهم خمسة آلاف من الأشوريين، ونزلوا في الوادي، وأخذوا المياه وينابيع مياه بني إسرائيل
\par 18 ثم صعد بنو عيسو مع بني عمون، ونزلوا في الجبل مقابل دوثايم. وأرسلوا بعضهم نحو الجنوب، ونحو الشرق مقابل عقربيل، التي قرب حوشي التي على وادي موكمور. ونزل بقية جيش الأشوريين في السهل، وغطوا وجه كل الأرض. ونصبوا خيامهم ومركباتهم لجمهور عظيم جدًا
\par 19 فصرخ بنو إسرائيل إلى الرب إلههم لأن قلوبهم قد انهارت، إذ أحاط بهم جميع أعدائهم، ولم يكن لهم مخرج من وسطهم
\par 20 وهكذا بقيت كل جماعة آشور حولهم، مشاةهم ومركباتهم وفرسانهم، أربعة وثلاثين يومًا، حتى أن جميع أوانيهم المائية نفدت من جميع موانئ بيت فلوى
\par 21 فأفرغت الآبار، ولم يكن فيها ماء ليشربوا حتى تمتلئ يومًا واحدًا، لأنهم كانوا يسقونهم بالكيل
\par 22 لذلك فقد أطفالهم قلوبهم، وضعف شبابهم ونساؤهم من العطش، وسقطوا في شوارع المدينة وعند ممرات الأبواب، ولم تعد فيهم قوة
\par 23 فاجتمع كل الشعب إلى عزيا وإلى رئيس المدينة، الشبان والنساء والأطفال، وصرخوا بصوت عظيم، وقالوا أمام جميع الشيوخ:
\par 24 فليحكم الله بيننا وبينكم، لأنكم ألحقتم بنا ضررًا عظيمًا، إذ لم تطلبوا السلام من بني آشور
\par 25 لأنه ليس لنا الآن معين، ولكن الله باعنا في أيديهم لكي نلقى أمامهم عطشًا وهلاكًا عظيمًا
\par 26 والآن فادعهم إليك، وسلم المدينة كلها غنيمة لشعب أليفانا ولكل جيشه
\par 27 لأنه خير لنا أن نصبح غنيمة لهم من أن نموت عطشًا. لأننا سنكون عبيدًا له، لكي تحيا نفوسنا، ولا نرى موت أطفالنا أمام أعيننا، ولا زوجاتنا ولا أولادنا يموتون
\par 28 نشهد عليكم السماء والأرض وإلهنا ورب آبائنا، الذي يعاقبنا حسب خطايانا وخطايا آبائنا، أنه لا يفعل كما قلنا في هذا اليوم
\par 29 فحدث بكاء عظيم بصوت واحد في وسط الجماعة، وصرخوا إلى الرب الإله بصوت عظيم
\par 30 ثم قال لهم عزيا: أيها الإخوة، تشجعوا، فلنصبر خمسة أيام أيضًا، فحينئذٍ يرد الرب إلهنا رحمته علينا، لأنه لا يتركنا إلى الأبد
\par 31 وإن مضت هذه الأيام ولم تأتنا مساعدة فإني أفعل حسب قولك.
\par 32 فشتّت الشعب، كل واحد على نفقته، فذهبوا إلى أسوار مدينتهم وأبراجها، وأرسلوا النساء والأطفال إلى بيوتهم، فدخلوا المدينة وهم في حالة يرثى لها

\chapter{8}

\par 1 في ذلك الوقت سمعت يهوديت، وهي ابنة مراري بن أوكس بن يوسف بن أوزيل بن إلكيا بن حننيا بن جدعون بن رافائيم بن أكيثو بن إيليو بن أليآب بن نثنائيل بن سمائل بن شلاسادال بن إسرائيل
\par 2 وكان منسى زوجها من سبطها وعشيرتها، ومات في حصاد الشعير
\par 3 وبينما كان واقفًا يُشرف على حازمي الحزم في الحقل، أصابته الحرّ رأسه، فسقط على فراشه، ومات في مدينة بيت فلوى. فدفنوه مع آبائه في الحقل بين دوثايم وبلامو
\par 4 فكانت يهوديت أرملة في بيتها ثلاث سنين وأربعة أشهر.
\par 5 فصنعت لنفسها خيمة على سطح بيتها ووضعت مسحا على حقويها ولبست ثياب ترملها.
\par 6 وصامت كل أيام ترملها، ما عدا عشية السبوت، وعشية رؤوس الشهور، ورأس السنة، والأعياد والمناسبات لبيت إسرائيل
\par 7 وكانت أيضًا حسنة المنظر وجميلة المنظر جدًا. وترك لها زوجها منسى ذهبًا وفضة وعبيدًا وإماءً وماشيةً وأراضي، فأقامت عليها
\par 8 ولم يكن هناك من يشتمها، بل كانت تخشى الله خوفًا شديدًا
\par 9 ولما سمعت كلام الشعب الشرير على الوالي، أنهم فقدوا الوعي من قلة الماء، لأن يهوديت سمعت كل الكلام الذي كلمهم به عزيا، وأنه أقسم أن يسلم المدينة إلى الأشوريين بعد خمسة أيام؛
\par 10 ثم أرسلت خادمتها، التي كانت مسؤولة عن كل ما لديها، لتستدعي عزيا وكابريس وخارميس، شيوخ المدينة
\par 11 فجاءوا إليها، فقالت لهم: اسمعوا لي يا حكام سكان بيت فلوى، فإن كلامكم الذي تكلمتم به أمام الشعب اليوم ليس صحيحًا فيما يتعلق بهذا القسم الذي أقسمتموه بين الله وبينكم، ووعدتم بتسليم المدينة لأعدائنا، ما لم يرجع الرب إليكم في هذه الأيام
\par 12 والآن، من أنتم الذين جربتم الله اليوم، ووقفتم بدلًا من الله بين بني البشر؟
\par 13 والآن جرّبوا الرب القدير، لكنكم لن تعرفوا شيئًا أبدًا.
\par 14 لأنكم لا تستطيعون أن تجدوا عمق قلب الإنسان، ولا تدركوا ما يفكر فيه. فكيف تستطيعون أن تبحثوا عن الله الذي صنع كل هذه الأشياء، وتعرفوا فكره، أو تدركوا قصده؟ كلا يا إخوتي، لا تغضبوا الرب إلهنا.
\par 15 لأنه إذا لم يساعدنا خلال هذه الأيام الخمسة، فهو قادر على الدفاع عنا عندما يشاء، بل كل يوم، أو أن يدمرنا أمام أعدائنا.
\par 16 لا تُقيّدوا مشورات الرب إلهنا، لأن الله ليس كالإنسان فيُهدد، ولا كابن الإنسان فيتردد
\par 17 لذلك فلننتظر خلاصه، ولنطلب منه عونًا لنا، فيسمع صوتنا إن شاء
\par 18 لأنه لم يقم في عصرنا، ولا يوجد الآن في هذه الأيام، لا قبيلة، ولا عائلة، ولا شعب، ولا مدينة بيننا، يعبد آلهة مصنوعة بأيدي، كما كان من قبل
\par 19 لأجل ذلك سُلِّم آباؤنا إلى السيف والغنيمة، وسقطوا سقوطًا عظيمًا أمام أعدائنا
\par 20 لكننا لا نعرف إلهًا آخر، لذلك نثق أنه لن يحتقرنا، ولا أي فرد من أمتنا
\par 21 لأنه إن اتخذنا هكذا، فستُخرب كل اليهودية، ويُنهب مقدسنا، وسيُطالب بتدنيسه من فمنا
\par 22 وسيُرجع قتل إخوتنا، وسبي البلاد، وخراب ميراثنا على رؤوسنا بين الأمم، حيثما نكون مستعبدين، ونكون عثرة وعارا لكل من يمتلكنا
\par 23 لأن عبوديتنا لن تُوجَّه إلى نعمة، بل الرب إلهنا سيحولها إلى إهانة
\par 24 والآن أيها الإخوة، لنُعطي مثالاً لإخوتنا، لأن قلوبهم علينا، والقدس والبيت والمذبح يستقر علينا
\par 25 فلنشكر الرب إلهنا الذي يمتحننا كما امتحن آباءنا
\par 26 تذكر ما فعله بإبراهيم، وكيف امتحن إسحاق، وما حدث ليعقوب في بلاد ما بين النهرين في سورية، عندما كان يرعى غنم لابان، أخي أمه
\par 27 لأنه لم يجربنا بالنار كما جربهم لامتحان قلوبهم، ولم ينتقم منا. لكن الرب يجلد الذين يقتربون منه لينذرهم
\par 28 فقال لها عزيا: كل ما تكلمتِ به قد تكلمتِ به بقلب سليم، وليس من يقاوم كلامكِ
\par 29 لأن هذا ليس اليوم الأول الذي تظهر فيه حكمتك، ​​بل منذ بدء أيامك عرف جميع الشعب فهمك، لأن تدبير قلبك صالح
\par 30 ولكن الشعب كان عطشانًا جدًا، فألزمونا أن نفعل بهم كما تكلمنا، وأن نأتي على أنفسنا بقسم لا نخلفه
\par 31 لذلك صلي الآن لأجلنا، لأنك امرأة تقية، وسيرسل لنا الرب المطر ليملأ آبارنا، ولن نكل بعد الآن
\par 32 فقالت لهم يهوديت: اسمعوا لي، فأفعل أمرًا يكون عبر الأجيال لأبناء أمتنا
\par 33 "أنت تقف هذه الليلة في الباب، وأنا أخرج مع خادمتي. وفي الأيام التي وعدتم فيها بتسليم المدينة لأعدائنا، يفتقد الرب إسرائيل بيدي."
\par 34 لكن لا تسألوا أنتم عن أفعالي، لأني لن أخبركم بها حتى أكمل ما أنا أفعله
\par 35 ثم قال لها عزيا والرؤساء: اذهبي بسلام، وليكن الرب الإله أمامك، لينتقم من أعدائنا
\par 36 فرجعوا من الخيمة وذهبوا إلى حُجُنِهم.

\chapter{9}

\par 1 سقطت يهوديت على وجهها، ووضعت رمادًا على رأسها، وكشفت عن المسح الذي كانت تلبسه. ونحو وقت إشعال البخور في ذلك المساء في أورشليم في بيت الرب، صرخت يهوديت بصوت عظيم وقالت:
\par 2 يا رب إله أبي شمعون، الذي أعطيته سيفًا للانتقام من الغرباء، الذين حلوا منطقة العذراء لتنجيسها، وكشفوا فخذها لعارها، ودنسوا عذريتها لعارها؛ لأنك قلت: لا يكون هكذا؛ ومع ذلك فعلوا هكذا
\par 3 لذلك سلمت حكامهم للقتل، حتى صبغوا فراشهم بالدم، مخدوعين، وضربوا العبيد مع أسيادهم، والأسياد على عروشهم؛
\par 4 وأسلمت نساءهم غنيمة، وبناتهم سبايا، وكل غنائمهم لتُقسم بين أبنائك الأعزاء، الذين تأثروا بغيرتك، وكرهوا نجاسة دمائهم، ولجأوا إليك طلبًا للمساعدة. يا الله، يا إلهي، استجب لي أنا أيضًا أنا الأرملة
\par 5 لأنك لم تصنع تلك الأشياء فحسب، بل صنعت أيضًا الأشياء التي وقعت من قبل، والتي حدثت بعد ذلك؛ لقد فكرت في الأشياء التي هي الآن، والتي ستأتي
\par 6 نعم، الأشياء التي قصدتها أصبحت جاهزة، وقلت: ها نحن هنا، لأن جميع طرقك مُعَدَّة، وأحكامك في علمك السابق
\par 7 لأنه هوذا الأشوريون قد كثروا في قوتهم، وقد تعظموا بالخيل والرجال، ويفتخرون بقوة مشاتهم، ويتوكلون على الترس والرمح والقوس والمقلاع، ولا يعلمون أنك أنت الرب الذي يكسر المعارك، الرب اسمك
\par 8 ألقِ قوتهم في قدرتك، وأنزل سلطتهم في غضبك. لأنهم فكروا في تدنيس مقدسك، ودنس المسكن الذي يستقر فيه اسمك المجيد، وإسقاط قرن مذبحك بالسيف
\par 9 انظر إلى كبريائهم، وأرسل غضبك على رؤوسهم. أعطِ في يدي، أنا الأرملة، السلطة التي حملتها
\par 10 اضرب بمكر شفتي العبد مع الأمير، والأمير مع العبد. اهدم هيبتهما بيد امرأة
\par 11 لأن قدرتك لا تقوم بالكثرة، ولا جبروتك بالأقوياء. لأنك أنت إله البائسين، ومعين المظلومين، وداعم الضعفاء، وحامي البائسين، ومخلص الذين لا رجاء لهم
\par 12 أتوسل إليك، أتوسل إليك، يا إله أبي، وإله ميراث إسرائيل، رب السماوات والأرض، خالق المياه، ملك كل خليقة، اسمع صلاتي:
\par 13 واجعل كلامي ومكره جرحًا وضربة لهم، الذين تآمروا على عهدك، وعلى بيتك المقدس، وعلى رأس صهيون، وعلى بيت ملك أبنائك
\par 14 واجعل كل أمة وقبيلة تعترف بأنك أنت إله كل قوة وقدرة، وأنه لا يوجد غيرك يحمي شعب إسرائيل

\chapter{10}

\par 1 وبعد ذلك كفّت عن الصراخ إلى إله إسرائيل، ووضعت حدًا لكل هذه الكلمات
\par 2 فقامت حيث سقطت، ودعت خادمتها، ونزلت إلى البيت الذي كانت تبيت فيه أيام السبت وأعيادها،
\par 3 وخلعت المسح الذي كان عليها، وخلعت ثياب ترملها، وغسلت جسدها كله بالماء، ودهنت نفسها بطيب طيب، وضفرت شعر رأسها، ووضعت عليه طوقًا، ولبست ثياب الفرح التي كانت تلبسها في حياة منسى زوجها
\par 4 وأخذت نعالاً على قدميها، ولبست أساورها، وسلاسلها، وخواتمها، وأقراطها، وكل زينتها، وتزينت بشجاعة، لتجذب أعين كل من يراها من الرجال
\par 5 ثم أعطت جاريتها زقاق خمر، وكوز زيت، وملأت كيسًا بذرةً مجففة، وكتل تين، وخبزًا فاخرًا، ثم طوت كل هذه الأشياء معًا، ووضعتها عليها
\par 6 فخرجوا إلى بوابة مدينة بيت فلوى، فوجدوا عزيا وشيخي المدينة، خابريس وخارميس، واقفين هناك
\par 7 فلما رأوها أن منظرها قد تغير ولباسها قد تغير، تعجبوا من جمالها جدًا وقالوا لها:
\par 8 الله، إله آبائنا، يُعطيك نعمة، ويُتمم أعمالك لمجد بني إسرائيل، ولرفعة أورشليم. ثم سجدوا لله
\par 9 فقالت لهم: «مروا أن تُفتح لي أبواب المدينة، لأخرج وأُنجز ما تحدثتم معي عنه». فأمروا الشبان أن يفتحوا لها كما قالت
\par 10 ولما فعلوا ذلك، خرجت يهوديت هي وخادمتها معها، واعتنى بها رجال المدينة حتى نزلت من الجبل، وحتى عبرت الوادي، ولم يعد بإمكانهم رؤيتها
\par 11 وهكذا انطلقوا في خط مستقيم في الوادي، والتقت بها أول حراسة للآشوريين،
\par 12 فأخذها وسألها: من أي شعب أنتِ؟ ومن أين أتيتِ؟ وإلى أين تذهبين؟ فقالت: أنا امرأة عبرانية وقد هربت منهم لأنهم سيعطونكم للهلاك.
\par 13 وأنا آتي إلى أليفانا رئيس جيشك لأتكلم بكلام الحق، وأريه الطريق الذي يسير فيه وينتصر على كل الجبال، دون أن يفقد جسد أو نفس أحد من رجاله.
\par 14 فلما سمع الرجال كلامها ونظروا إلى وجهها، تعجبوا من جمالها، وقالوا لها:
\par 15 لقد أنقذت حياتك، إذ أسرعت بالنزول إلى حضرة سيدنا. فالآن تعال إلى خيمته، وسيقودك بعضنا حتى يسلموك إلى يديه
\par 16 ومتى وقفت أمامه، فلا تخف في قلبك، بل أظهر له حسب كلامك، فيُحسن إليك
\par 17 ثم اختاروا منهم مئة رجل ليرافقوها هي وخادمتها، وأتوا بها إلى خيمة أليفانا
\par 18 فحدث ضجيج في كل المحلة، لأن مجيئها سُمع بين الخيام، فأحاطوا بها وهي واقفة خارج خيمة أليفانا، حتى أخبروه عنها
\par 19 فتعجبوا من جمالها، وأعجب بنو إسرائيل بسببها، وقال كل واحد لصاحبه: من يحتقر هذا الشعب الذي بينهم مثل هؤلاء النساء؟ إنه ليس جيدًا أن يبقى منهم رجل واحد، فإذا أطلق، خدع الأرض كلها
\par 20 فخرج الذين كانوا يضطجعون بالقرب من أليفانا، وجميع عبيده، وأدخلوها إلى الخيمة
\par 21 وكان أليفارنس مستريحًا على سريره تحت مظلة منسوجة بأرجوان وذهب وزمرّد وأحجار كريمة
\par 22 فأروه إياها، فخرج إلى أمام خيمته ومصابيح فضة أمامه
\par 23 ولما جاءت يهوديت أمامه وأمام عبيده، تعجب الجميع من جمال وجهها، فسقطت على وجهها وسجدت له، فحملها عبيده

\chapter{11}

\par 1 ثم قال لها أليفارنس: يا امرأة، تعزّي، لا تخافي في قلبك، لأني لم أؤذِ أحدًا قط ممن أرادوا خدمة نبوخذنصر، ملك كل الأرض
\par 2 والآن، لو لم يستهين بي شعبك الساكن في الجبال، لما رفعت رمحي عليهم. لكنهم فعلوا بأنفسهم هذه الأشياء
\par 3 لكن أخبرني الآن لماذا هربت منهم، ولماذا أتيت إلينا؟ لأنك أتيت للحماية. كن مطمئنًا، ستعيش هذه الليلة، وبعدها
\par 4 لأنه لن يؤذيك أحد، بل يطلبون منك الخير، كما يفعلون مع عبيد الملك نبوخذنصر سيدي
\par 5 فقالت له يهوديت: خذ كلام أمتك، ودع أمتك تتكلم أمامك، ولن أكذب على سيدي هذه الليلة
\par 6 وإن اتبعتِ كلام أمتكِ، فسيُتم الله الأمر بكِ على أكمل وجه، ولن يفشل سيدي في تحقيق مقاصده
\par 7 حي هو نبوخذنصر ملك كل الأرض، وحية هي قوته الذي أرسلك لصيانة كل حي. لأنه ليس الناس فقط يتعبدون لك، بل أيضا وحوش البرية والبهائم وطيور السماء تعيش بقوتك تحت حكم نبوخذنصر وكل بيته.
\par 8 لأننا سمعنا بحكمتك وسياساتك، وأُشيع في كل الأرض أنك وحدك متميز في كل المملكة، وعظيم المعرفة، وعجيب في أعمال الحرب
\par 9 وأما الأمر الذي تكلم به أخيور في مجلسك، فقد سمعنا كلامه، لأن رجال بيت فلوى أنقذوه، وأخبرهم بكل ما كلمك به
\par 10 لذلك، أيها السيد والحاكم، لا تحترم كلمته؛ بل ضعها في قلبك، لأنها حق: لأن أمتنا لن تُعاقب، ولا يقوى عليهم السيف، إلا إذا أخطأوا إلى إلههم
\par 11 والآن، لكي لا يُهزم سيدي ويُحبط من قصده، فقد سقط عليهم الموت الآن، وأدركتهم خطيئتهم، التي سيُغضبون بها إلههم كلما فعلوا ما لا يليق
\par 12 لأن مؤونتهم تنفد، ومياههم شحيحة، وقد عزموا على الاستيلاء على ماشيتهم، وقصدوا أن يأكلوا كل ما حرم الله عليهم أكله بشرائعه:
\par 13 وقد عزموا على إنفاق باكورة أعشار الخمر والزيت التي قدسوها وحفظوها للكهنة الذين يخدمون في أورشليم أمام وجه إلهنا، وهي أشياء لا يجوز لأحد من الشعب أن يلمسها بأيديه
\par 14 لأنهم أرسلوا بعضهم إلى أورشليم، لأن الساكنين هناك فعلوا مثل ذلك، ليأتوا لهم رخصة من مجلس الشيوخ
\par 15 والآن عندما يأتونهم بالخبر، سيفعلونه على الفور، وسيُسلمون إليك للهلاك في ذلك اليوم
\par 16 لذلك، أنا أمتك، عالمة بكل هذا، هربت من أمامهم. وقد أرسلني الله لأعمل معك أمورًا تُدهش منها كل الأرض وكل من يسمع
\par 17 لأن عبدك متدين، ويخدم إله السماء نهارًا وليلاً. والآن يا سيدي، سأبقى معك، وسيخرج عبدك ليلًا إلى الوادي، وسأصلي إلى الله، فيخبرني متى ارتكبوا خطاياهم
\par 18 وأنا آتي وأخبرك، فتخرج بكل جيشك، ولا يكون أحد يقاومك
\par 19 وسأقودك في وسط اليهودية حتى تصل إلى أورشليم، وأضع عرشك في وسطها، وتسوقهم كغنم لا راعي لها، ولا يفتح الكلب فمه عليك. لأن هذه الأمور أُخبرت بها حسب علمي السابق، وأُبلغت لي، وأُرسلت لأخبرك بها
\par 20 فسرت كلماتها أليفانا وجميع عبيده، فتعجبوا من حكمتها، وقالوا:
\par 21 لا توجد مثل هذه المرأة من أقاصي الأرض إلى أقاصيها، سواء في جمال الوجه أو حكمة الكلام
\par 22 كذلك قال لها هولوفرنيس: لقد أحسن الله بإرسالك أمام الشعب، لتكون القوة في أيدينا والهلاك على الذين يستهينون بسيدي
\par 23 والآن أنتِ جميلة في وجهك، وذكية في كلامك. إن فعلتِ كما تكلمتِ، فسيكون إلهكِ إلهي، وستسكنين في بيت الملك نبوخذنصر، وستكونين مشهورة في كل الأرض

\chapter{12}

\par 1 ثم أمر بإحضارها إلى حيث وُضع طبقه، وأمر أن يُعدّوا لها من أطعمته، وأن تشرب من خمره
\par 2 فقالت يهوديت: لا آكل منه لئلا تكون عثرة، ولكن يُصنع لي تدبير مما أحضرته
\par 3 فقال لها أليفارنس: إذا نفدت رزقك، فكيف نعطيك مثله؟ لأنه ليس معنا أحد من أمتك
\par 4 فقالت له يهوديت: «حية هي نفسك يا سيدي، إن أمتك لا تنفق ما عندي قبل أن يعمل الرب بيدي ما قد قرره».
\par 5 فأدخلها عبيد أليفانا إلى الخيمة، فنامت إلى منتصف الليل، ثم استيقظت عند طلوع الفجر،
\par 6 وأرسل إلى هولوفرنيس، قائلاً: فليأمر سيدي الآن أن تخرج أمتك إلى الصلاة
\par 7 ثم أمر أليفانا حراسه ألا يمنعوها، فأقامت في المحلة ثلاثة أيام، ثم خرجت ليلاً إلى وادي بيت فلوى، واغتسلت في عين ماء بجانب المحلة
\par 8 ولما خرجت، طلبت إلى الرب إله إسرائيل أن يرشد طريقها إلى تربية أبناء شعبها
\par 9 فدخلت نظيفة، وبقيت في الخيمة حتى أكلت لحمها في المساء
\par 10 وفي اليوم الرابع صنع أليفانا وليمة لعبيده فقط، ولم يدعُ أحدًا من الخدم إلى الوليمة
\par 11 ثم قال لباغواس الخصي، الذي كان مسؤولاً عن كل ما كان له: اذهب الآن وأقنع هذه المرأة العبرية التي معك أن تأتي إلينا وتأكل وتشرب معنا
\par 12 لأنه، ها هو ذا، سيكون عارًا علينا أن نترك مثل هذه المرأة دون أن نرافقها؛ لأنه إذا لم نجذبها إلينا، فسوف تسخر منا
\par 13 ثم ذهب باغواس من أمام أليفرن، وجاء إليها، وقال: لا تخف هذه الفتاة الجميلة من المجيء إلى سيدي، وأن تُكرّم في حضرته، وتشرب الخمر، وتفرح معنا، وتكون اليوم كإحدى بنات الآشوريين اللواتي يخدمن في بيت نبوخذنصر
\par 14 فقالت له يهوديت: من أنا الآن حتى أقاوم سيدي؟ إنما أفعل كل ما يحسن في عينيه سريعاً، ويكون ذلك فرحي إلى يوم وفاتي.
\par 15 فقامت وتزينت بملابسها وكل ملابسها النسائية، وذهبت خادمتها ووضعت لها جلودًا ناعمة على الأرض مقابل هولوفرن، كانت قد أخذتها من باجوا لاستخدامها اليومي، لتجلس وتأكل عليها
\par 16 ولما دخلت يهوديت وجلست، تأثر أليفانا بها، وتحركت نفسه، واشتهى ​​صحبتها بشدة، لأنه انتظر وقتًا ليخدعها، منذ اليوم الذي رآها فيه
\par 17 ثم قال لها هولوفرنيس: اشربي الآن، وافرحي معنا.
\par 18 فقالت يهوديت: أشرب الآن يا سيدي، لأن حياتي قد عظمت في هذا اليوم أكثر من كل الأيام منذ ولدت.
\par 19 ثم أخذت وأكلت وشربت أمامه مما أعدته خادمتها
\par 20 وكان أليفارنس مسرورًا بها جدًا، وشرب من النبيذ أكثر مما شرب في أي وقت في يوم واحد منذ ولادته

\chapter{13}

\par 1 ولما حل المساء، أسرع عبيده للمغادرة، وأغلق باغواس خيمته في الخارج، وصرف الخدام من أمام سيده، وذهبوا إلى فراشهم، لأنهم كانوا جميعًا متعبين، لأن الوليمة طال أمدها
\par 2 وبقيت يهوديت في الخيمة، وأليفانا مُستلقيًا على فراشه، لأنه كان قد شبع من الخمر
\par 3 كانت جوديث قد أمرت خادمتها بالوقوف خارج حجرة نومها، وانتظار خروجها، كما كانت تفعل يوميًا: لأنها قالت إنها ستخرج إلى صلواتها، وكلمت باجواس وفقًا لنفس الغرض
\par 4 فخرج الجميع ولم يبق في حجرة النوم أحد، لا صغير ولا كبير. ثم وقفت يهوديت عند فراشه، وقالت في قلبها: أيها الرب إله كل القدرة، انظر إلى هذه الهدية على أعمال يدي لرفع أورشليم
\par 5 الآن هو الوقت المناسب لدعم ميراثك، وتنفيذ مشاريعك لتدمير الأعداء الذين ثاروا علينا
\par 6 ثم جاءت إلى عمود السرير الذي كان عند رأس أليفرن، وأنزلت سريره من هناك،
\par 7 وتقدم إلى فراشه وأمسك بشعر رأسه وقال: أيدني يا رب إله إسرائيل اليوم
\par 8 وضربت على رقبته بكل قوتها مرتين، وأخذت رأسه عنه
\par 9 وأسقطت جسده عن السرير، وسحبت المظلة عن الأعمدة، وبعد قليل خرجت، وأعطت هولوفرن رأسه لخادمتها؛
\par 10 فوضعته في جرابها، وذهبتا معًا كعادتهما إلى الصلاة. ولما عبرتا المحلة، دارتا حول الوادي، وصعدتا إلى جبل بيت فلوى، وأتيتا إلى أبوابه
\par 11 فقالت يهوديت من بعيد للحراس على الباب: افتحوا افتحوا الآن الباب. الله إلهنا معنا ليظهر قوته بعد في أورشليم وجنوده ضد العدو كما فعل اليوم أيضا.
\par 12 فلما سمع رجال مدينتها صوتها، أسرعوا بالنزول إلى باب مدينتهم، ونادوا شيوخ المدينة
\par 13 ثم ركضوا جميعًا معًا، صغارًا وكبارًا، لأنه كان غريبًا عليهم أنها جاءت. ففتحوا الباب، واستقبلوهم، وأشعلوا نارًا للضوء، ووقفوا حولهم
\par 14 ثم قالت لهم بصوت عظيم: سبحوا، سبحوا الله، سبحوا الله، لأنه لم ينزع رحمته عن بيت إسرائيل، بل أهلك أعداءنا بيدي هذه الليلة
\par 15 فأخرجت الرأس من الكيس وأرته، وقالت لهم: هوذا رأس أليفانا، رئيس جيش أشور، وهوذا المظلة التي كان مضطجعًا فيها في سكره، وقد ضربه الرب بيد امرأة
\par 16 حي هو الرب الذي حفظني في طريقي الذي سلكته، إن وجهي قد أضله إلى الهلاك، ومع ذلك لم يفعل معي خطية لتدنيسي وإخزائي
\par 17 فدهش جميع الشعب دهشة عجيبة، وسجدوا لله، وقالوا بصوت واحد: مبارك أنت يا إلهنا الذي أهلكتَ أعداء شعبك اليوم
\par 18 فقال لها عزيا: يا ابنتي، مباركة أنت من الله العلي فوق كل نساء الأرض، ومبارك الرب الإله خالق السماوات والأرض، الذي أمرك بقطع رأس رئيس أعدائنا
\par 19 لأن ثقتك لن تزول عن قلوب البشر الذين يذكرون قدرة الله إلى الأبد
\par 20 ويحول الله هذه الأمور إليك حمدًا أبديًا، ليفتقدك بالخير، لأنك لم تبخل بحياتك من أجل مذلة أمتنا، بل انتقمت لخرابنا، سالكًا طريقًا مستقيمًا أمام إلهنا. فقال جميع الشعب: ليكن، ليكن

\chapter{14}

\par 1 فقالت لهم يهوديت: اسمعوا لي الآن يا إخوتي، وخذوا هذا الرأس وعلقوه على أعلى أسواركم
\par 2 وبمجرد أن يطلع الصباح، وتشرق الشمس على الأرض، خذوا كل واحد سلاحه، واخرجوا كل رجل شجاع من المدينة، وضعوا عليهم قائدًا، كما لو كنتم تريدون النزول إلى الحقل نحو حراسة الآشوريين؛ ولكن لا تنزلوا
\par 3 فيأخذون أسلحةهم ويدخلون معسكرهم، ويقيمون رؤساء جيش أشور، ويركضون إلى خيمة أليفانا، لكنهم لا يجدونه. فيقع عليهم الرعب، ويهربون من أمام وجهك
\par 4 فتطردونهم أنتم وجميع سكان ساحل إسرائيل، وتسقطونهم في طريقهم
\par 5 ولكن قبل أن تفعلوا هذا، ادع لي أخيور العموني، لكي يرى ويعرف من احتقر بيت إسرائيل، وأرسله إلينا كما لو كان إلى موته.
\par 6 فدعوا أخيور من بيت عزيا، فلما جاء ورأى رأس أليفانا في يد إنسان في جماعة الشعب، سقط على وجهه وفسدت روحه
\par 7 فلما استعادوه، سقط عند قدمي يهوديت وسجد لها، وقال: مباركة أنت في جميع خيام يهوذا، وفي جميع الأمم، الذين إذا سمعوا اسمك تعجبوا
\par 8 فأخبرني الآن بكل الأمور التي فعلتها في هذه الأيام. فأخبرته يهوديت في وسط الشعب بكل ما فعلته من اليوم الذي خرجت فيه إلى تلك الساعة التي كلمتهم فيها
\par 9 ولما انتهت من الكلام، هتف الناس بصوت عظيم، وأحدثوا ضجة كبيرة في مدينتهم
\par 10 ولما رأى أحيور كل ما صنع إله إسرائيل، آمن بالله إيمانًا عظيمًا، وختن لحم غرلته، وانضم إلى بيت إسرائيل إلى هذا اليوم
\par 11 ولما طلع الصباح، علقوا رأس أليفرن على السور، وأخذ كل رجل سلاحه، وخرجوا فرقًا إلى مضيق الجبل
\par 12 فلما رآهم الآشوريون أرسلوا إلى رؤسائهم، فجاءوا إلى قوادّهم وولاة أمرهم، وإلى كل واحد من حكامهم
\par 13 فجاءوا إلى خيمة أليفانا، وقالوا للذي كان مسؤولاً عن جميع أغراضه: أيقظ الآن سيدنا، لأن العبيد تجرأوا على النزول إلينا للقتال، لكي يُهلكوا تمامًا
\par 14 ثم دخل باجواس، وطرق باب الخيمة، لأنه ظن أنه نام مع يهوديت
\par 15 ولكن لأنه لم يُجب أحد، فتح الباب ودخل إلى غرفة النوم، فوجده ملقى على الأرض ميتًا، ورأسه مقطوع عنه
\par 16 لذلك صرخ بصوت عظيم، وبكاء وتنهد وصراخ شديد، ومزق ثيابه
\par 17 وبعد أن دخل الخيمة التي كانت تقيم فيها يهوديت، ولما لم يجدها، قفز إلى الناس ونادى،
\par 18 لقد خان هؤلاء العبيد؛ فقد جلبت امرأة عبرانية واحدة العار على بيت الملك نبوخذنصر، لأنه هوذا أليفانا مُلقى على الأرض بلا رأس
\par 19 عندما سمع قادة جيش آشور هذه الكلمات، مزقوا ثيابهم واضطربت نفوسهم بشدة، وكان هناك صراخ وضجيج عظيم جدًا في جميع أنحاء المحلة

\chapter{15}

\par 1 فلما سمع الذين في الخيام تعجبوا مما حدث
\par 2 فسقط عليهم الخوف والرعدة، حتى لم يجرؤ أحد على البقاء أمام قريبه، بل اندفعوا جميعًا وهربوا إلى كل طريق من طرق السهل والجبل
\par 3 وهرب أيضًا الذين نزلوا في الجبال المحيطة ببيت فلوى، فانقض عليهم بنو إسرائيل، كل من كان منهم محاربًا.
\par 4 ثم أرسل عزيا إلى بيت مستيم، وإلى باباي، وخوباي، وكولا، وإلى جميع تخوم إسرائيل، ليخبروا بما جرى، وليهجموا على أعدائهم ليهلكوهم
\par 5 فلما سمع بنو إسرائيل ذلك، انقضوا عليهم جميعًا بكلمة واحدة، وقتلوهم حتى شوباي. وكذلك أيضًا الذين أتوا من أورشليم ومن جميع الجبال، لأن الناس أخبروهم بما جرى في معسكر أعدائهم، والذين في جلعاد والجليل، طاردوهم بضربة عظيمة حتى تجاوزوا دمشق وحدودها
\par 6 وأما الباقون الذين سكنوا بيت فلوى، فسقطوا على محلة أشور، ونهبوهم، وأثروا كثيراً
\par 7 وكان لبنو إسرائيل الذين رجعوا من المذبحة ما بقى، أما القرى والمدن التي في الجبال وفي السهل، فقد غنمت غنائم كثيرة، لأن الجمهور كان كثيرًا جدًا
\par 8 فجاء يهوياقيم رئيس الكهنة وشيوخ بني إسرائيل الساكنين في أورشليم لينظروا الخيرات التي أظهرها الله لإسرائيل، وليروا يهوديت وليسلموا عليها
\par 9 ولما جاءوا إليها باركوها بنفس واحدة، وقالوا لها: أنتِ رفعة أورشليم، أنتِ مجد إسرائيل العظيم، أنتِ فرح أمتنا العظيم
\par 10 لقد فعلتَ كل هذه الأمور بيدك، وقد أحسنتَ إلى إسرائيل، ورضي الله بذلك. لتكن مباركًا من الرب القدير إلى الأبد. فقال جميع الشعب: ليكن
\par 11 ونهب الشعب المحلة مدة ثلاثين يومًا، وأعطوا يهوديت أليفرن خيمته، وكل صحنه، وأسرته، وأوانيه، وكل أمتعته، فأخذتها ووضعتها على بغلها، وأعدت عرباتها ووضعتها عليها
\par 12 فركضت جميع نساء إسرائيل لينظرنها، وباركنها، ورقصن لها بينهن، وأخذت في يدها أغصانًا، وأعطت أيضًا للنساء اللواتي معها
\par 13 ووضعوا عليها وعلى جاريتها التي معها إكليلاً من زيتون، وكانت تسير أمام كل الشعب في الرقص، تقود جميع النساء، وكان جميع رجال إسرائيل يسيرون وراءهم بدروعهم، وأكاليل، وأغاني في أفواههم

\chapter{16}

\par 1 ثم بدأت يهوديت تغني هذا الشكر في كل إسرائيل، وغنى جميع الشعب وراءها هذا النشيد
\par 2 فقالت يهوديت: ابتدئوا لإلهي بالدفوف، غنوا لربي بالصنوج، أنشدوا له مزمورًا جديدًا، ارفعوه وادعوا باسمه
\par 3 لأن الله يكسر الحروب، لأنه في المعسكرات، في وسط الشعب، أنقذني من أيدي الذين اضطهدونني
\par 4 فخرج آشور من الجبال من الشمال، وجاء بعشرات الآلاف من جيشه، فمنع هذا الجمع السيول، وغطت فرسانه الجبال.
\par 5 كان يتفاخر بأنه سيحرق تخومي، ويقتل شباني بالسيف، ويسحق الأطفال الرضع على الأرض، ويجعل أطفالي غنيمة، وعذارى غنيمة
\par 6 لكن الرب القدير خيب آمالهم بيد امرأة.
\par 7 لأنه لم يسقط الجبار على يد الشبان، ولم يضربه أبناء الجبابرة، ولا هاجمه عمالقة عظماء، لكن يهوديت ابنة مراري أضعفته بجمال وجهها.
\par 8 لأنها خلعت ثوب ترملها من أجل رفعة المظلومين في إسرائيل، ودهنت وجهها بطيب، وربطت شعرها بعصبة، وأخذت ثوبًا من كتان لتخدعه
\par 9 سحرت صندلها عينيه، وأسر جمالها عقله، ومر الوشاح عبر رقبته
\par 10 ارتعد الفرس من جرأتها، وخاف الميديون من جرأتها
\par 11 حينئذٍ هتف بائسي فرحًا، وصرخ ضعفائي بصوت عالٍ، لكنهم دهشوا. رفعوا أصواتهم، لكنهم انهزموا
\par 12 لقد طعنهم أبناء الفتيات، وجرحوهم كأبناء الهاربين: هلكوا في معركة الرب
\par 13 سأرنم للرب ترنيمة جديدة: يا رب، أنت عظيم ومجيد، عجيب القوة، ولا يُقهر
\par 14 لتخدمك جميع المخلوقات: لأنك أنت تكلمت فخلقت، أنت أرسلت روحك فخلقتهم، وليس هناك من يستطيع مقاومة صوتك
\par 15 لأن الجبال ستتحرك من أساساتها مع المياه، والصخور ستذوب كالشمع أمام وجهك، ومع ذلك فأنت رحيمٌ على خائفيك
\par 16 لأن كل ذبيحة قليلة لرائحة سرور لك، وكل الشحم لا يكفي لمحرقتك. أما من يتقي الرب فهو عظيم في كل وقت
\par 17 ويل للأمم التي تثور على إخوتي! سينتقم منهم الرب القدير يوم القيامة، بوضع النار والديدان في أجسادهم، وسيشعرون بها ويبكون إلى الأبد
\par 18 فلما دخلوا أورشليم سجدوا للرب، ولما تطهر الشعب قدموا محرقاتهم وتقدماتهم وعطاياهم
\par 19 وقدست يهوديت أيضًا جميع أمتعة أليفارنيس التي أعطاها لها الشعب، وأعطت المظلة التي أخرجتها من حجرة نومه هدية للرب
\par 20 فأقام الشعب في أورشليم أمام الحرم ثلاثة أشهر، وبقيت يهوديت معهم
\par 21 بعد هذا الوقت، عاد كل واحد إلى ميراثه، وذهبت يهوديت إلى بيت فلوى، وبقيت في ملكها، وكانت في عهدها محترمة في كل البلاد
\par 22 وكان كثيرون يشتهونها، ولكن لم يعرفها أحد كل أيام حياتها، بعد أن مات منسى رجلها وانضم إلى قومه.
\par 23 ولكنها ازدادت كرامة وشاخَت في بيت زوجها وهي ابنة مئة وخمس سنين، وأعتقت خادمتها، فماتت في بيت فلوى، ودفنوها في مغارة زوجها منسى
\par 24 ورثاها بيت إسرائيل سبعة أيام، وقبل أن تموت، وزعت أملاكها على جميع أقرب أقربائها إلى منسى زوجها، وعلى أقرب أقربائها
\par 25 ولم يكن هناك من يُخيف بني إسرائيل بعد في أيام يهوديت، ولا بعد وفاتها بزمن طويل


\end{document}