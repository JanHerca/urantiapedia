\begin{document}

\title{1كورنثوس}


\chapter{1}

\par 1 بُولُسُ الْمَدْعُوُّ رَسُولاً لِيَسُوعَ الْمَسِيحِ بِمَشِيئَةِ اللهِ وَسُوسْتَانِيسُ الأَخُ
\par 2 إِلَى كَنِيسَةِ اللهِ الَّتِي فِي كُورِنْثُوسَ الْمُقَدَّسِينَ فِي الْمَسِيحِ يَسُوعَ الْمَدْعُوِّينَ قِدِّيسِينَ مَعَ جَمِيعِ الَّذِينَ يَدْعُونَ بِاسْمِ رَبِّنَا يَسُوعَ الْمَسِيحِ فِي كُلِّ مَكَانٍ لَهُمْ وَلَنَا.
\par 3 نِعْمَةٌ لَكُمْ وَسَلاَمٌ مِنَ اللهِ أَبِينَا وَالرَّبِّ يَسُوعَ الْمَسِيحِ.
\par 4 أَشْكُرُ إِلَهِي فِي كُلِّ حِينٍ مِنْ جِهَتِكُمْ عَلَى نِعْمَةِ اللهِ الْمُعْطَاةِ لَكُمْ فِي يَسُوعَ الْمَسِيحِ
\par 5 أَنَّكُمْ فِي كُلِّ شَيْءٍ اسْتَغْنَيْتُمْ فِيهِ فِي كُلِّ كَلِمَةٍ وَكُلِّ عِلْمٍ
\par 6 كَمَا ثُبِّتَتْ فِيكُمْ شَهَادَةُ الْمَسِيحِ
\par 7 حَتَّى إِنَّكُمْ لَسْتُمْ نَاقِصِينَ فِي مَوْهِبَةٍ مَا وَأَنْتُمْ مُتَوَقِّعُونَ اسْتِعْلاَنَ رَبِّنَا يَسُوعَ الْمَسِيحِ
\par 8 الَّذِي سَيُثْبِتُكُمْ أَيْضاً إِلَى النِّهَايَةِ بِلاَ لَوْمٍ فِي يَوْمِ رَبِّنَا يَسُوعَ الْمَسِيحِ.
\par 9 أَمِينٌ هُوَ اللهُ الَّذِي بِهِ دُعِيتُمْ إِلَى شَرِكَةِ ابْنِهِ يَسُوعَ الْمَسِيحِ رَبِّنَا.
\par 10 وَلَكِنَّنِي أَطْلُبُ إِلَيْكُمْ أَيُّهَا الإِخْوَةُ بِاسْمِ رَبِّنَا يَسُوعَ الْمَسِيحِ أَنْ تَقُولُوا جَمِيعُكُمْ قَوْلاً وَاحِداً وَلاَ يَكُونَ بَيْنَكُمُ انْشِقَاقَاتٌ بَلْ كُونُوا كَامِلِينَ فِي فِكْرٍ وَاحِدٍ وَرَأْيٍ وَاحِدٍ
\par 11 لأَنِّي أُخْبِرْتُ عَنْكُمْ يَا إِخْوَتِي مِنْ أَهْلِ خُلُوِي أَنَّ بَيْنَكُمْ خُصُومَاتٍ.
\par 12 فَأَنَا أَعْنِي هَذَا: أَنَّ كُلَّ وَاحِدٍ مِنْكُمْ يَقُولُ: «أَنَا لِبُولُسَ وَأَنَا لأَبُلُّوسَ وَأَنَا لِصَفَا وَأَنَا لِلْمَسِيحِ».
\par 13 هَلِ انْقَسَمَ الْمَسِيحُ؟ أَلَعَلَّ بُولُسَ صُلِبَ لأَجْلِكُمْ أَمْ بِاسْمِ بُولُسَ اعْتَمَدْتُمْ؟
\par 14 أَشْكُرُ اللهَ أَنِّي لَمْ أُعَمِّدْ أَحَداً مِنْكُمْ إِلاَّ كِرِيسْبُسَ وَغَايُسَ
\par 15 حَتَّى لاَ يَقُولَ أَحَدٌ إِنِّي عَمَّدْتُ بِاسْمِي.
\par 16 وَعَمَّدْتُ أَيْضاً بَيْتَ اسْتِفَانُوسَ. عَدَا ذَلِكَ لَسْتُ أَعْلَمُ هَلْ عَمَّدْتُ أَحَداً آخَرَ
\par 17 لأَنَّ الْمَسِيحَ لَمْ يُرْسِلْنِي لأُعَمِّدَ بَلْ لأُبَشِّرَ - لاَ بِحِكْمَةِ كَلاَمٍ لِئَلاَّ يَتَعَطَّلَ صَلِيبُ الْمَسِيحِ.
\par 18 فَإِنَّ كَلِمَةَ الصَّلِيبِ عِنْدَ الْهَالِكِينَ جَهَالَةٌ وَأَمَّا عِنْدَنَا نَحْنُ الْمُخَلَّصِينَ فَهِيَ قُوَّةُ اللهِ
\par 19 لأَنَّهُ مَكْتُوبٌ: «سَأُبِيدُ حِكْمَةَ الْحُكَمَاءِ وَأَرْفُضُ فَهْمَ الْفُهَمَاءِ».
\par 20 أَيْنَ الْحَكِيمُ؟ أَيْنَ الْكَاتِبُ؟ أَيْنَ مُبَاحِثُ هَذَا الدَّهْرِ؟ أَلَمْ يُجَهِّلِ اللهُ حِكْمَةَ هَذَا الْعَالَمِ؟
\par 21 لأَنَّهُ إِذْ كَانَ الْعَالَمُ فِي حِكْمَةِ اللهِ لَمْ يَعْرِفِ اللهَ بِالْحِكْمَةِ اسْتَحْسَنَ اللهُ أَنْ يُخَلِّصَ الْمُؤْمِنِينَ بِجَهَالَةِ الْكِرَازَةِ
\par 22 لأَنَّ الْيَهُودَ يَسْأَلُونَ آيَةً وَالْيُونَانِيِّينَ يَطْلُبُونَ حِكْمَةً
\par 23 وَلَكِنَّنَا نَحْنُ نَكْرِزُ بِالْمَسِيحِ مَصْلُوباً: لِلْيَهُودِ عَثْرَةً وَلِلْيُونَانِيِّينَ جَهَالَةً!
\par 24 وَأَمَّا لِلْمَدْعُوِّينَ: يَهُوداً وَيُونَانِيِّينَ فَبِالْمَسِيحِ قُوَّةِ اللهِ وَحِكْمَةِ اللهِ.
\par 25 لأَنَّ جَهَالَةَ اللهِ أَحْكَمُ مِنَ النَّاسِ! وَضَعْفَ اللهِ أَقْوَى مِنَ النَّاسِ!
\par 26 فَانْظُرُوا دَعْوَتَكُمْ أَيُّهَا الإِخْوَةُ أَنْ لَيْسَ كَثِيرُونَ حُكَمَاءُ حَسَبَ الْجَسَدِ. لَيْسَ كَثِيرُونَ أَقْوِيَاءُ. لَيْسَ كَثِيرُونَ شُرَفَاءُ.
\par 27 بَلِ اخْتَارَ اللهُ جُهَّالَ الْعَالَمِ لِيُخْزِيَ الْحُكَمَاءَ وَاخْتَارَ اللهُ ضُعَفَاءَ الْعَالَمِ لِيُخْزِيَ الأَقْوِيَاءَ
\par 28 وَاخْتَارَ اللهُ أَدْنِيَاءَ الْعَالَمِ وَالْمُزْدَرَى وَغَيْرَ الْمَوْجُودِ لِيُبْطِلَ الْمَوْجُودَ
\par 29 لِكَيْ لاَ يَفْتَخِرَ كُلُّ ذِي جَسَدٍ أَمَامَهُ.
\par 30 وَمِنْهُ أَنْتُمْ بِالْمَسِيحِ يَسُوعَ الَّذِي صَارَ لَنَا حِكْمَةً مِنَ اللهِ وَبِرّاً وَقَدَاسَةً وَفِدَاءً.
\par 31 حَتَّى كَمَا هُوَ مَكْتُوبٌ: «مَنِ افْتَخَرَ فَلْيَفْتَخِرْ بِالرَّبِّ».

\chapter{2}

\par 1 وَأَنَا لَمَّا أَتَيْتُ إِلَيْكُمْ أَيُّهَا الإِخْوَةُ أَتَيْتُ لَيْسَ بِسُمُوِّ الْكَلاَمِ أَوِ الْحِكْمَةِ مُنَادِياً لَكُمْ بِشَهَادَةِ اللهِ
\par 2 لأَنِّي لَمْ أَعْزِمْ أَنْ أَعْرِفَ شَيْئاً بَيْنَكُمْ إِلاَّ يَسُوعَ الْمَسِيحَ وَإِيَّاهُ مَصْلُوباً.
\par 3 وَأَنَا كُنْتُ عِنْدَكُمْ فِي ضُعْفٍ وَخَوْفٍ وَرِعْدَةٍ كَثِيرَةٍ.
\par 4 وَكَلاَمِي وَكِرَازَتِي لَمْ يَكُونَا بِكَلاَمِ الْحِكْمَةِ الإِنْسَانِيَّةِ الْمُقْنِعِ بَلْ بِبُرْهَانِ الرُّوحِ وَالْقُوَّةِ
\par 5 لِكَيْ لاَ يَكُونَ إِيمَانُكُمْ بِحِكْمَةِ النَّاسِ بَلْ بِقُوَّةِ اللهِ.
\par 6 لَكِنَّنَا نَتَكَلَّمُ بِحِكْمَةٍ بَيْنَ الْكَامِلِينَ وَلَكِنْ بِحِكْمَةٍ لَيْسَتْ مِنْ هَذَا الدَّهْرِ وَلاَ مِنْ عُظَمَاءِ هَذَا الدَّهْرِ الَّذِينَ يُبْطَلُونَ.
\par 7 بَلْ نَتَكَلَّمُ بِحِكْمَةِ اللهِ فِي سِرٍّ: الْحِكْمَةِ الْمَكْتُومَةِ الَّتِي سَبَقَ اللهُ فَعَيَّنَهَا قَبْلَ الدُّهُورِ لِمَجْدِنَا
\par 8 الَّتِي لَمْ يَعْلَمْهَا أَحَدٌ مِنْ عُظَمَاءِ هَذَا الدَّهْرِ - لأَنْ لَوْ عَرَفُوا لَمَا صَلَبُوا رَبَّ الْمَجْدِ.
\par 9 بَلْ كَمَا هُوَ مَكْتُوبٌ: «مَا لَمْ تَرَ عَيْنٌ وَلَمْ تَسْمَعْ أُذُنٌ وَلَمْ يَخْطُرْ عَلَى بَالِ إِنْسَانٍ: مَا أَعَدَّهُ اللهُ لِلَّذِينَ يُحِبُّونَهُ».
\par 10 فَأَعْلَنَهُ اللهُ لَنَا نَحْنُ بِرُوحِهِ. لأَنَّ الرُّوحَ يَفْحَصُ كُلَّ شَيْءٍ حَتَّى أَعْمَاقَ اللهِ.
\par 11 لأَنْ مَنْ مِنَ النَّاسِ يَعْرِفُ أُمُورَ الإِنْسَانِ إِلاَّ رُوحُ الإِنْسَانِ الَّذِي فِيهِ؟ هَكَذَا أَيْضاً أُمُورُ اللهِ لاَ يَعْرِفُهَا أَحَدٌ إِلاَّ رُوحُ اللهِ.
\par 12 وَنَحْنُ لَمْ نَأْخُذْ رُوحَ الْعَالَمِ بَلِ الرُّوحَ الَّذِي مِنَ اللهِ لِنَعْرِفَ الأَشْيَاءَ الْمَوْهُوبَةَ لَنَا مِنَ اللهِ
\par 13 الَّتِي نَتَكَلَّمُ بِهَا أَيْضاً لاَ بِأَقْوَالٍ تُعَلِّمُهَا حِكْمَةٌ إِنْسَانِيَّةٌ بَلْ بِمَا يُعَلِّمُهُ الرُّوحُ الْقُدُسُ قَارِنِينَ الرُّوحِيَّاتِ بِالرُّوحِيَّاتِ.
\par 14 وَلَكِنَّ الإِنْسَانَ الطَّبِيعِيَّ لاَ يَقْبَلُ مَا لِرُوحِ اللهِ لأَنَّهُ عِنْدَهُ جَهَالَةٌ وَلاَ يَقْدِرُ أَنْ يَعْرِفَهُ لأَنَّهُ إِنَّمَا يُحْكَمُ فِيهِ رُوحِيّاً.
\par 15 وَأَمَّا الرُّوحِيُّ فَيَحْكُمُ فِي كُلِّ شَيْءٍ وَهُوَ لاَ يُحْكَمُ فِيهِ مِنْ أَحَدٍ.
\par 16 لأَنَّهُ مَنْ عَرَفَ فِكْرَ الرَّبِّ فَيُعَلِّمَهُ؟ وَأَمَّا نَحْنُ فَلَنَا فِكْرُ الْمَسِيحِ.

\chapter{3}

\par 1 وَأَنَا أَيُّهَا الإِخْوَةُ لَمْ أَسْتَطِعْ أَنْ أُكَلِّمَكُمْ كَرُوحِيِّينَ بَلْ كَجَسَدِيِّينَ كَأَطْفَالٍ فِي الْمَسِيحِ
\par 2 سَقَيْتُكُمْ لَبَناً لاَ طَعَاماً لأَنَّكُمْ لَمْ تَكُونُوا بَعْدُ تَسْتَطِيعُونَ بَلِ الآنَ أَيْضاً لاَ تَسْتَطِيعُونَ
\par 3 لأَنَّكُمْ بَعْدُ جَسَدِيُّونَ. فَإِنَّهُ إِذْ فِيكُمْ حَسَدٌ وَخِصَامٌ وَانْشِقَاقٌ أَلَسْتُمْ جَسَدِيِّينَ وَتَسْلُكُونَ بِحَسَبِ الْبَشَرِ؟
\par 4 لأَنَّهُ مَتَى قَالَ وَاحِدٌ: «أَنَا لِبُولُسَ» وَآخَرُ: «أَنَا لأَبُلُّوسَ» أَفَلَسْتُمْ جَسَدِيِّينَ؟
\par 5 فَمَنْ هُوَ بُولُسُ وَمَنْ هُوَ أَبُلُّوسُ؟ بَلْ خَادِمَانِ آمَنْتُمْ بِوَاسِطَتِهِمَا وَكَمَا أَعْطَى الرَّبُّ لِكُلِّ وَاحِدٍ:
\par 6 أَنَا غَرَسْتُ وَأَبُلُّوسُ سَقَى لَكِنَّ اللهَ كَانَ يُنْمِي.
\par 7 إِذاً لَيْسَ الْغَارِسُ شَيْئاً وَلاَ السَّاقِي بَلِ اللهُ الَّذِي يُنْمِي.
\par 8 وَالْغَارِسُ وَالسَّاقِي هُمَا وَاحِدٌ وَلَكِنَّ كُلَّ وَاحِدٍ سَيَأْخُذُ أُجْرَتَهُ بِحَسَبِ تَعَبِهِ.
\par 9 فَإِنَّنَا نَحْنُ عَامِلاَنِ مَعَ اللهِ وَأَنْتُمْ فَلاَحَةُ اللهِ بِنَاءُ اللهِ.
\par 10 حَسَبَ نِعْمَةِ اللهِ الْمُعْطَاةِ لِي كَبَنَّاءٍ حَكِيمٍ قَدْ وَضَعْتُ أَسَاساً وَآخَرُ يَبْنِي عَلَيْهِ. وَلَكِنْ فَلْيَنْظُرْ كُلُّ وَاحِدٍ كَيْفَ يَبْنِي عَلَيْهِ.
\par 11 فَإِنَّهُ لاَ يَسْتَطِيعُ أَحَدٌ أَنْ يَضَعَ أَسَاساً آخَرَ غَيْرَ الَّذِي وُضِعَ الَّذِي هُوَ يَسُوعُ الْمَسِيحُ.
\par 12 وَلَكِنْ إِنْ كَانَ أَحَدُ يَبْنِي عَلَى هَذَا الأَسَاسِ ذَهَباً فِضَّةً حِجَارَةً كَرِيمَةً خَشَباً عُشْباً قَشّاً
\par 13 فَعَمَلُ كُلِّ وَاحِدٍ سَيَصِيرُ ظَاهِراً لأَنَّ الْيَوْمَ سَيُبَيِّنُهُ. لأَنَّهُ بِنَارٍ يُسْتَعْلَنُ وَسَتَمْتَحِنُ النَّارُ عَمَلَ كُلِّ وَاحِدٍ مَا هُوَ.
\par 14 إِنْ بَقِيَ عَمَلُ أَحَدٍ قَدْ بَنَاهُ عَلَيْهِ فَسَيَأْخُذُ أُجْرَةً.
\par 15 إِنِ احْتَرَقَ عَمَلُ أَحَدٍ فَسَيَخْسَرُ وَأَمَّا هُوَ فَسَيَخْلُصُ وَلَكِنْ كَمَا بِنَارٍ.
\par 16 أَمَا تَعْلَمُونَ أَنَّكُمْ هَيْكَلُ اللهِ وَرُوحُ اللهِ يَسْكُنُ فِيكُمْ؟
\par 17 إِنْ كَانَ أَحَدٌ يُفْسِدُ هَيْكَلَ اللهِ فَسَيُفْسِدُهُ اللهُ لأَنَّ هَيْكَلَ اللهِ مُقَدَّسٌ الَّذِي أَنْتُمْ هُوَ.
\par 18 لاَ يَخْدَعَنَّ أَحَدٌ نَفْسَهُ. إِنْ كَانَ أَحَدٌ يَظُنُّ أَنَّهُ حَكِيمٌ بَيْنَكُمْ فِي هَذَا الدَّهْرِ فَلْيَصِرْ جَاهِلاً لِكَيْ يَصِيرَ حَكِيماً!
\par 19 لأَنَّ حِكْمَةَ هَذَا الْعَالَمِ هِيَ جَهَالَةٌ عِنْدَ اللهِ لأَنَّهُ مَكْتُوبٌ: «الآخِذُ الْحُكَمَاءَ بِمَكْرِهِمْ».
\par 20 وَأَيْضاً: «الرَّبُّ يَعْلَمُ أَفْكَارَ الْحُكَمَاءِ أَنَّهَا بَاطِلَةٌ».
\par 21 إِذاً لاَ يَفْتَخِرَنَّ أَحَدٌ بِالنَّاسِ فَإِنَّ كُلَّ شَيْءٍ لَكُمْ:
\par 22 أَبُولُسُ أَمْ أَبُلُّوسُ أَمْ صَفَا أَمِ الْعَالَمُ أَمِ الْحَيَاةُ أَمِ الْمَوْتُ أَمِ الأَشْيَاءُ الْحَاضِرَةُ أَمِ الْمُسْتَقْبَِلَةُ. كُلُّ شَيْءٍ لَكُمْ.
\par 23 وَأَمَّا أَنْتُمْ فَلِلْمَسِيحِ وَالْمَسِيحُ لِلَّهِ.

\chapter{4}

\par 1 هَكَذَا فَلْيَحْسِبْنَا الإِنْسَانُ كَخُدَّامِ الْمَسِيحِ وَوُكَلاَءِ سَرَائِرِ اللهِ
\par 2 ثُمَّ يُسْأَلُ فِي الْوُكَلاَءِ لِكَيْ يُوجَدَ الإِنْسَانُ أَمِيناً.
\par 3 وَأَمَّا أَنَا فَأَقَلُّ شَيْءٍ عِنْدِي أَنْ يُحْكَمَ فِيَّ مِنْكُمْ أَوْ مِنْ يَوْمِ بَشَرٍ. بَلْ لَسْتُ أَحْكُمُ فِي نَفْسِي أَيْضاً.
\par 4 فَإِنِّي لَسْتُ أَشْعُرُ بِشَيْءٍ فِي ذَاتِي. لَكِنَّنِي لَسْتُ بِذَلِكَ مُبَرَّراً. وَلَكِنَّ الَّذِي يَحْكُمُ فِيَّ هُوَ الرَّبُّ.
\par 5 إِذاً لاَ تَحْكُمُوا فِي شَيْءٍ قَبْلَ الْوَقْتِ حَتَّى يَأْتِيَ الرَّبُّ الَّذِي سَيُنِيرُ خَفَايَا الظَّلاَمِ وَيُظْهِرُ آرَاءَ الْقُلُوبِ. وَحِينَئِذٍ يَكُونُ الْمَدْحُ لِكُلِّ وَاحِدٍ مِنَ اللهِ.
\par 6 فَهَذَا أَيُّهَا الإِخْوَةُ حَوَّلْتُهُ تَشْبِيهاً إِلَى نَفْسِي وَإِلَى أَبُلُّوسَ مِنْ أَجْلِكُمْ لِكَيْ تَتَعَلَّمُوا فِينَا أَنْ لاَ تَفْتَكِرُوا فَوْقَ مَا هُوَ مَكْتُوبٌ كَيْ لاَ يَنْتَفِخَ أَحَدٌ لأَجْلِ الْوَاحِدِ عَلَى الآخَرِ.
\par 7 لأَنَّهُ مَنْ يُمَيِّزُكَ؟ وَأَيُّ شَيْءٍ لَكَ لَمْ تَأْخُذْهُ؟ وَإِنْ كُنْتَ قَدْ أَخَذْتَ فَلِمَاذَا تَفْتَخِرُ كَأَنَّكَ لَمْ تَأْخُذْ؟
\par 8 إِنَّكُمْ قَدْ شَبِعْتُمْ! قَدِ اسْتَغْنَيْتُمْ! مَلَكْتُمْ بِدُونِنَا! وَلَيْتَكُمْ مَلَكْتُمْ لِنَمْلِكَ نَحْنُ أَيْضاً مَعَكُمْ!
\par 9 فَإِنِّي أَرَى أَنَّ اللهَ أَبْرَزَنَا نَحْنُ الرُّسُلَ آخِرِينَ كَأَنَّنَا مَحْكُومٌ عَلَيْنَا بِالْمَوْتِ. لأَنَّنَا صِرْنَا مَنْظَراً لِلْعَالَمِ لِلْمَلاَئِكَةِ وَالنَّاسِ.
\par 10 نَحْنُ جُهَّالٌ مِنْ أَجْلِ الْمَسِيحِ وَأَمَّا أَنْتُمْ فَحُكَمَاءُ فِي الْمَسِيحِ! نَحْنُ ضُعَفَاءُ وَأَمَّا أَنْتُمْ فَأَقْوِيَاءُ! أَنْتُمْ مُكَرَّمُونَ وَأَمَّا نَحْنُ فَبِلاَ كَرَامَةٍ!
\par 11 إِلَى هَذِهِ السَّاعَةِ نَجُوعُ وَنَعْطَشُ وَنَعْرَى وَنُلْكَمُ وَلَيْسَ لَنَا إِقَامَةٌ
\par 12 وَنَتْعَبُ عَامِلِينَ بِأَيْدِينَا. نُشْتَمُ فَنُبَارِكُ. نُضْطَهَدُ فَنَحْتَمِلُ.
\par 13 يُفْتَرَى عَلَيْنَا فَنَعِظُ. صِرْنَا كَأَقْذَارِ الْعَالَمِ وَوَسَخِ كُلِّ شَيْءٍ إِلَى الآنَ.
\par 14 لَيْسَ لِكَيْ أُخَجِّلَكُمْ أَكْتُبُ بِهَذَا بَلْ كَأَوْلاَدِي الأَحِبَّاءِ أُنْذِرُكُمْ.
\par 15 لأَنَّهُ وَإِنْ كَانَ لَكُمْ رَبَوَاتٌ مِنَ الْمُرْشِدِينَ فِي الْمَسِيحِ لَكِنْ لَيْسَ آبَاءٌ كَثِيرُونَ. لأَنِّي أَنَا وَلَدْتُكُمْ فِي الْمَسِيحِ يَسُوعَ بِالإِنْجِيلِ.
\par 16 فَأَطْلُبُ إِلَيْكُمْ أَنْ تَكُونُوا مُتَمَثِّلِينَ بِي.
\par 17 لِذَلِكَ أَرْسَلْتُ إِلَيْكُمْ تِيمُوثَاوُسَ الَّذِي هُوَ ابْنِي الْحَبِيبُ وَالأَمِينُ فِي الرَّبِّ الَّذِي يُذَكِّرُكُمْ بِطُرُقِي فِي الْمَسِيحِ كَمَا أُعَلِّمُ فِي كُلِّ مَكَانٍ فِي كُلِّ كَنِيسَةٍ.
\par 18 فَانْتَفَخَ قَوْمٌ كَأَنِّي لَسْتُ آتِياً إِلَيْكُمْ.
\par 19 وَلَكِنِّي سَآتِي إِلَيْكُمْ سَرِيعاً إِنْ شَاءَ الرَّبُّ فَسَأَعْرِفُ لَيْسَ كَلاَمَ الَّذِينَ انْتَفَخُوا بَلْ قُوَّتَهُمْ.
\par 20 لأَنَّ مَلَكُوتَ اللهِ لَيْسَ بِكَلاَمٍ بَلْ بِقُوَّةٍ.
\par 21 مَاذَا تُرِيدُونَ؟ أَبِعَصاً آتِي إِلَيْكُمْ أَمْ بِالْمَحَبَّةِ وَرُوحِ الْوَدَاعَةِ؟

\chapter{5}

\par 1 يُسْمَعُ مُطْلَقاً أَنَّ بَيْنَكُمْ زِنًى! وَزِنًى هَكَذَا لاَ يُسَمَّى بَيْنَ الأُمَمِ حَتَّى أَنْ تَكُونَ لِلإِنْسَانِ امْرَأَةُ أَبِيهِ.
\par 2 أَفَأَنْتُمْ مُنْتَفِخُونَ وَبِالْحَرِيِّ لَمْ تَنُوحُوا حَتَّى يُرْفَعَ مِنْ وَسَطِكُمُ الَّذِي فَعَلَ هَذَا الْفِعْلَ؟
\par 3 فَإِنِّي أَنَا كَأَنِّي غَائِبٌ بِالْجَسَدِ وَلَكِنْ حَاضِرٌ بِالرُّوحِ قَدْ حَكَمْتُ كَأَنِّي حَاضِرٌ فِي الَّذِي فَعَلَ هَذَا هَكَذَا
\par 4 بِاسْمِ رَبِّنَا يَسُوعَ الْمَسِيحِ - إِذْ أَنْتُمْ وَرُوحِي مُجْتَمِعُونَ مَعَ قُوَّةِ رَبِّنَا يَسُوعَ الْمَسِيحِ -
\par 5 أَنْ يُسَلَّمَ مِثْلُ هَذَا لِلشَّيْطَانِ لِهَلاَكِ الْجَسَدِ لِكَيْ تَخْلُصَ الرُّوحُ فِي يَوْمِ الرَّبِّ يَسُوعَ.
\par 6 لَيْسَ افْتِخَارُكُمْ حَسَناً. أَلَسْتُمْ تَعْلَمُونَ أَنَّ خَمِيرَةً صَغِيرَةً تُخَمِّرُ الْعَجِينَ كُلَّهُ؟
\par 7 إِذاً نَقُّوا مِنْكُمُ الْخَمِيرَةَ الْعَتِيقَةَ لِكَيْ تَكُونُوا عَجِيناً جَدِيداً كَمَا أَنْتُمْ فَطِيرٌ. لأَنَّ فِصْحَنَا أَيْضاً الْمَسِيحَ قَدْ ذُبِحَ لأَجْلِنَا.
\par 8 إِذاً لِنُعَيِّدْ لَيْسَ بِخَمِيرَةٍ عَتِيقَةٍ وَلاَ بِخَمِيرَةِ الشَّرِّ وَالْخُبْثِ بَلْ بِفَطِيرِ الإِخْلاَصِ وَالْحَقِّ.
\par 9 كَتَبْتُ إِلَيْكُمْ فِي الرِّسَالَةِ أَنْ لاَ تُخَالِطُوا الزُّنَاةَ.
\par 10 وَلَيْسَ مُطْلَقاً زُنَاةَ هَذَا الْعَالَمِ أَوِ الطَّمَّاعِينَ أَوِ الْخَاطِفِينَ أَوْ عَبَدَةَ الأَوْثَانِ وَإِلاَّ فَيَلْزَمُكُمْ أَنْ تَخْرُجُوا مِنَ الْعَالَمِ.
\par 11 وَأَمَّا الآنَ فَكَتَبْتُ إِلَيْكُمْ: إِنْ كَانَ أَحَدٌ مَدْعُوٌّ أَخاً زَانِياً أَوْ طَمَّاعاً أَوْ عَابِدَ وَثَنٍ أَوْ شَتَّاماً أَوْ سِكِّيراً أَوْ خَاطِفاً أَنْ لاَ تُخَالِطُوا وَلاَ تُؤَاكِلُوا مِثْلَ هَذَا.
\par 12 لأَنَّهُ مَاذَا لِي أَنْ أَدِينَ الَّذِينَ مِنْ خَارِجٍ أَلَسْتُمْ أَنْتُمْ تَدِينُونَ الَّذِينَ مِنْ دَاخِلٍ.
\par 13 أَمَّا الَّذِينَ مِنْ خَارِجٍ فَاللَّهُ يَدِينُهُمْ. فَاعْزِلُوا الْخَبِيثَ مِنْ بَيْنِكُمْ.

\chapter{6}

\par 1 أَيَتَجَاسَرُ مِنْكُمْ أَحَدٌ لَهُ دَعْوَى عَلَى آخَرَ أَنْ يُحَاكَمَ عِنْدَ الظَّالِمِينَ وَلَيْسَ عِنْدَ الْقِدِّيسِينَ؟
\par 2 أَلَسْتُمْ تَعْلَمُونَ أَنَّ الْقِدِّيسِينَ سَيَدِينُونَ الْعَالَمَ؟ فَإِنْ كَانَ الْعَالَمُ يُدَانُ بِكُمْ أَفَأَنْتُمْ غَيْرُ مُسْتَأْهِلِينَ لِلْمَحَاكِمِ الصُّغْرَى؟
\par 3 أَلَسْتُمْ تَعْلَمُونَ أَنَّنَا سَنَدِينُ مَلاَئِكَةً؟ فَبِالأَوْلَى أُمُورَ هَذِهِ الْحَيَاةِ!
\par 4 فَإِنْ كَانَ لَكُمْ مَحَاكِمُ فِي أُمُورِ هَذِهِ الْحَيَاةِ فَأَجْلِسُوا الْمُحْتَقَرِينَ فِي الْكَنِيسَةِ قُضَاةً!
\par 5 لِتَخْجِيلِكُمْ أَقُولُ. أَهَكَذَا لَيْسَ بَيْنَكُمْ حَكِيمٌ وَلاَ وَاحِدٌ يَقْدِرُ أَنْ يَقْضِيَ بَيْنَ إِخْوَتِهِ؟
\par 6 لَكِنَّ الأَخَ يُحَاكِمُ الأَخَ وَذَلِكَ عِنْدَ غَيْرِ الْمُؤْمِنِينَ.
\par 7 فَالآنَ فِيكُمْ عَيْبٌ مُطْلَقاً لأَنَّ عِنْدَكُمْ مُحَاكَمَاتٍ بَعْضِكُمْ مَعَ بَعْضٍ. لِمَاذَا لاَ تُظْلَمُونَ بِالْحَرِيِّ؟ لِمَاذَا لاَ تُسْلَبُونَ بِالْحَرِيِّ؟
\par 8 لَكِنْ أَنْتُمْ تَظْلِمُونَ وَتَسْلُبُونَ وَذَلِكَ لِلإِخْوَةِ.
\par 9 أَمْ لَسْتُمْ تَعْلَمُونَ أَنَّ الظَّالِمِينَ لاَ يَرِثُونَ مَلَكُوتَ اللهِ؟ لاَ تَضِلُّوا! لاَ زُنَاةٌ وَلاَ عَبَدَةُ أَوْثَانٍ وَلاَ فَاسِقُونَ وَلاَ مَأْبُونُونَ وَلاَ مُضَاجِعُو ذُكُورٍ
\par 10 وَلاَ سَارِقُونَ وَلاَ طَمَّاعُونَ وَلاَ سِكِّيرُونَ وَلاَ شَتَّامُونَ وَلاَ خَاطِفُونَ يَرِثُونَ مَلَكُوتَ اللهِ.
\par 11 وَهَكَذَا كَانَ أُنَاسٌ مِنْكُمْ. لَكِنِ اغْتَسَلْتُمْ بَلْ تَقَدَّسْتُمْ بَلْ تَبَرَّرْتُمْ بِاسْمِ الرَّبِّ يَسُوعَ وَبِرُوحِ إِلَهِنَا.
\par 12 كُلُّ الأَشْيَاءِ تَحِلُّ لِي لَكِنْ لَيْسَ كُلُّ الأَشْيَاءِ تُوافِقُ. كُلُّ الأَشْيَاءِ تَحِلُّ لِي لَكِنْ لاَ يَتَسَلَّطُ عَلَيَّ شَيْءٌ.
\par 13 اَلأَطْعِمَةُ لِلْجَوْفِ وَالْجَوْفُ لِلأَطْعِمَةِ وَاللهُ سَيُبِيدُ هَذَا وَتِلْكَ. وَلَكِنَّ الْجَسَدَ لَيْسَ لِلزِّنَا بَلْ لِلرَّبِّ وَالرَّبُّ لِلْجَسَدِ.
\par 14 وَاللَّهُ قَدْ أَقَامَ الرَّبَّ وَسَيُقِيمُنَا نَحْنُ أَيْضاً بِقُوَّتِهِ.
\par 15 أَلَسْتُمْ تَعْلَمُونَ أَنَّ أَجْسَادَكُمْ هِيَ أَعْضَاءُ الْمَسِيحِ؟ أَفَآخُذُ أَعْضَاءَ الْمَسِيحِ وَأَجْعَلُهَا أَعْضَاءَ زَانِيَةٍ؟ حَاشَا!
\par 16 أَمْ لَسْتُمْ تَعْلَمُونَ أَنَّ مَنِ الْتَصَقَ بِزَانِيَةٍ هُوَ جَسَدٌ وَاحِدٌ لأَنَّهُ يَقُولُ: «يَكُونُ الِاثْنَانِ جَسَداً وَاحِداً».
\par 17 وَأَمَّا مَنِ الْتَصَقَ بِالرَّبِّ فَهُوَ رُوحٌ وَاحِدٌ.
\par 18 اُهْرُبُوا مِنَ الزِّنَا. كُلُّ خَطِيَّةٍ يَفْعَلُهَا الإِنْسَانُ هِيَ خَارِجَةٌ عَنِ الْجَسَدِ لَكِنَّ الَّذِي يَزْنِي يُخْطِئُ إِلَى جَسَدِهِ.
\par 19 أَمْ لَسْتُمْ تَعْلَمُونَ أَنَّ جَسَدَكُمْ هُوَ هَيْكَلٌ لِلرُّوحِ الْقُدُسِ الَّذِي فِيكُمُ الَّذِي لَكُمْ مِنَ اللهِ وَأَنَّكُمْ لَسْتُمْ لأَنْفُسِكُمْ؟
\par 20 لأَنَّكُمْ قَدِ اشْتُرِيتُمْ بِثَمَنٍ. فَمَجِّدُوا اللهَ فِي أَجْسَادِكُمْ وَفِي أَرْوَاحِكُمُ الَّتِي هِيَ لِلَّهِ.

\chapter{7}

\par 1 وَأَمَّا مِنْ جِهَةِ الأُمُورِ الَّتِي كَتَبْتُمْ لِي عَنْهَا فَحَسَنٌ لِلرَّجُلِ أَنْ لاَ يَمَسَّ امْرَأَةً.
\par 2 وَلَكِنْ لِسَبَبِ الزِّنَا لِيَكُنْ لِكُلِّ وَاحِدٍ امْرَأَتُهُ وَلْيَكُنْ لِكُلِّ وَاحِدَةٍ رَجُلُهَا.
\par 3 لِيُوفِ الرَّجُلُ الْمَرْأَةَ حَقَّهَا الْوَاجِبَ وَكَذَلِكَ الْمَرْأَةُ أَيْضاً الرَّجُلَ.
\par 4 لَيْسَ لِلْمَرْأَةِ تَسَلُّطٌ عَلَى جَسَدِهَا بَلْ لِلرَّجُلِ وَكَذَلِكَ الرَّجُلُ أَيْضاً لَيْسَ لَهُ تَسَلُّطٌ عَلَى جَسَدِهِ بَلْ لِلْمَرْأَةِ.
\par 5 لاَ يَسْلِبْ أَحَدُكُمُ الآخَرَ إِلاَّ أَنْ يَكُونَ عَلَى مُوافَقَةٍ إِلَى حِينٍ لِكَيْ تَتَفَرَّغُوا لِلصَّوْمِ وَالصَّلاَةِ ثُمَّ تَجْتَمِعُوا أَيْضاً مَعاً لِكَيْ لاَ يُجَرِّبَكُمُ الشَّيْطَانُ لِسَبَبِ عَدَمِ نَزَاهَتِكُمْ.
\par 6 وَلَكِنْ أَقُولُ هَذَا عَلَى سَبِيلِ الإِذْنِ لاَ عَلَى سَبِيلِ الأَمْرِ.
\par 7 لأَنِّي أُرِيدُ أَنْ يَكُونَ جَمِيعُ النَّاسِ كَمَا أَنَا. لَكِنَّ كُلَّ وَاحِدٍ لَهُ مَوْهِبَتُهُ الْخَاصَّةُ مِنَ اللهِ. الْوَاحِدُ هَكَذَا وَالآخَرُ هَكَذَا.
\par 8 وَلَكِنْ أَقُولُ لِغَيْرِ الْمُتَزَوِّجِينَ وَلِلأَرَامِلِ إِنَّهُ حَسَنٌ لَهُمْ إِذَا لَبِثُوا كَمَا أَنَا.
\par 9 وَلَكِنْ إِنْ لَمْ يَضْبِطُوا أَنْفُسَهُمْ فَلْيَتَزَوَّجُوا لأَنَّ التَّزَوُّجَ أَصْلَحُ مِنَ التَّحَرُّقِ.
\par 10 وَأَمَّا الْمُتَزَوِّجُونَ فَأُوصِيهِمْ لاَ أَنَا بَلِ الرَّبُّ أَنْ لاَ تُفَارِقَ الْمَرْأَةُ رَجُلَهَا.
\par 11 وَإِنْ فَارَقَتْهُ فَلْتَلْبَثْ غَيْرَ مُتَزَوِّجَةٍ أَوْ لِتُصَالِحْ رَجُلَهَا. وَلاَ يَتْرُكِ الرَّجُلُ امْرَأَتَهُ.
\par 12 وَأَمَّا الْبَاقُونَ فَأَقُولُ لَهُمْ أَنَا لاَ الرَّبُّ: إِنْ كَانَ أَخٌ لَهُ امْرَأَةٌ غَيْرُ مُؤْمِنَةٍ وَهِيَ تَرْتَضِي أَنْ تَسْكُنَ مَعَهُ فَلاَ يَتْرُكْهَا.
\par 13 وَالْمَرْأَةُ الَّتِي لَهَا رَجُلٌ غَيْرُ مُؤْمِنٍ وَهُوَ يَرْتَضِي أَنْ يَسْكُنَ مَعَهَا فَلاَ تَتْرُكْهُ.
\par 14 لأَنَّ الرَّجُلَ غَيْرَ الْمُؤْمِنِ مُقَدَّسٌ فِي الْمَرْأَةِ وَالْمَرْأَةُ غَيْرُ الْمُؤْمِنَةِ مُقَدَّسَةٌ فِي الرَّجُلِ - وَإِلاَّ فَأَوْلاَدُكُمْ نَجِسُونَ. وَأَمَّا الآنَ فَهُمْ مُقَدَّسُونَ.
\par 15 وَلَكِنْ إِنْ فَارَقَ غَيْرُ الْمُؤْمِنِ فَلْيُفَارِقْ. لَيْسَ الأَخُ أَوِ الأُخْتُ مُسْتَعْبَداً فِي مِثْلِ هَذِهِ الأَحْوَالِ. وَلَكِنَّ اللهَ قَدْ دَعَانَا فِي السَّلاَمِ.
\par 16 لأَنَّهُ كَيْفَ تَعْلَمِينَ أَيَّتُهَا الْمَرْأَةُ هَلْ تُخَلِّصِينَ الرَّجُلَ؟ أَوْ كَيْفَ تَعْلَمُ أَيُّهَا الرَّجُلُ هَلْ تُخَلِّصُ الْمَرْأَةَ؟
\par 17 غَيْرَ أَنَّهُ كَمَا قَسَمَ اللهُ لِكُلِّ وَاحِدٍ كَمَا دَعَا الرَّبُّ كُلَّ وَاحِدٍ هَكَذَا لِيَسْلُكْ. وَهَكَذَا أَنَا آمُرُ فِي جَمِيعِ الْكَنَائِسِ.
\par 18 دُعِيَ أَحَدٌ وَهُوَ مَخْتُونٌ فَلاَ يَصِرْ أَغْلَفَ. دُعِيَ أَحَدٌ فِي الْغُرْلَةِ فَلاَ يَخْتَتِنْ.
\par 19 لَيْسَ الْخِتَانُ شَيْئاً وَلَيْسَتِ الْغُرْلَةُ شَيْئاً بَلْ حِفْظُ وَصَايَا اللهِ.
\par 20 اَلدَّعْوَةُ الَّتِي دُعِيَ فِيهَا كُلُّ وَاحِدٍ فَلْيَلْبَثْ فِيهَا.
\par 21 دُعِيتَ وَأَنْتَ عَبْدٌ فَلاَ يَهُمَّكَ. بَلْ وَإِنِ اسْتَطَعْتَ أَنْ تَصِيرَ حُرّاً فَاسْتَعْمِلْهَا بِالْحَرِيِّ.
\par 22 لأَنَّ مَنْ دُعِيَ فِي الرَّبِّ وَهُوَ عَبْدٌ فَهُوَ عَتِيقُ الرَّبِّ. كَذَلِكَ أَيْضاً الْحُرُّ الْمَدْعُوُّ هُوَ عَبْدٌ لِلْمَسِيحِ.
\par 23 قَدِ اشْتُرِيتُمْ بِثَمَنٍ فَلاَ تَصِيرُوا عَبِيداً لِلنَّاسِ.
\par 24 مَا دُعِيَ كُلُّ وَاحِدٍ فِيهِ أَيُّهَا الإِخْوَةُ فَلْيَلْبَثْ فِي ذَلِكَ مَعَ اللهِ.
\par 25 وَأَمَّا الْعَذَارَى فَلَيْسَ عِنْدِي أَمْرٌ مِنَ الرَّبِّ فِيهِنَّ وَلَكِنَّنِي أُعْطِي رَأْياً كَمَنْ رَحِمَهُ الرَّبُّ أَنْ يَكُونَ أَمِيناً.
\par 26 فَأَظُنُّ أَنَّ هَذَا حَسَنٌ لِسَبَبِ الضِّيقِ الْحَاضِرِ. أَنَّهُ حَسَنٌ لِلإِنْسَانِ أَنْ يَكُونَ هَكَذَا:
\par 27 أَنْتَ مُرْتَبِطٌ بِامْرَأَةٍ فَلاَ تَطْلُبْ الِانْفِصَالَ. أَنْتَ مُنْفَصِلٌ عَنِ امْرَأَةٍ فَلاَ تَطْلُبِ امْرَأَةً.
\par 28 لَكِنَّكَ وَإِنْ تَزَوَّجْتَ لَمْ تُخْطِئْ. وَإِنْ تَزَوَّجَتِ الْعَذْرَاءُ لَمْ تُخْطِئْ. وَلَكِنَّ مِثْلَ هَؤُلاَءِ يَكُونُ لَهُمْ ضِيقٌ فِي الْجَسَدِ. وَأَمَّا أَنَا فَإِنِّي أُشْفِقُ عَلَيْكُمْ.
\par 29 فَأَقُولُ هَذَا أَيُّهَا الإِخْوَةُ: الْوَقْتُ مُنْذُ الآنَ مُقَصَّرٌ لِكَيْ يَكُونَ الَّذِينَ لَهُمْ نِسَاءٌ كَأَنْ لَيْسَ لَهُمْ
\par 30 وَالَّذِينَ يَبْكُونَ كَأَنَّهُمْ لاَ يَبْكُونَ وَالَّذِينَ يَفْرَحُونَ كَأَنَّهُمْ لاَ يَفْرَحُونَ وَالَّذِينَ يَشْتَرُونَ كَأَنَّهُمْ لاَ يَمْلِكُونَ
\par 31 وَالَّذِينَ يَسْتَعْمِلُونَ هَذَا الْعَالَمَ كَأَنَّهُمْ لاَ يَسْتَعْمِلُونَهُ. لأَنَّ هَيْئَةَ هَذَا الْعَالَمِ تَزُولُ.
\par 32 فَأُرِيدُ أَنْ تَكُونُوا بِلاَ هَمٍّ. غَيْرُ الْمُتَزَوِّجِ يَهْتَمُّ فِي مَا لِلرَّبِّ كَيْفَ يُرْضِي الرَّبَّ
\par 33 وَأَمَّا الْمُتَزَوِّجُ فَيَهْتَمُّ فِي مَا لِلْعَالَمِ كَيْفَ يُرْضِي امْرَأَتَهُ.
\par 34 إِنَّ بَيْنَ الزَّوْجَةِ وَالْعَذْرَاءِ فَرْقاً: غَيْرُ الْمُتَزَوِّجَةِ تَهْتَمُّ فِي مَا لِلرَّبِّ لِتَكُونَ مُقَدَّسَةً جَسَداً وَرُوحاً. وَأَمَّا الْمُتَزَوِّجَةُ فَتَهْتَمُّ فِي مَا لِلْعَالَمِ كَيْفَ تُرْضِي رَجُلَهَا.
\par 35 هَذَا أَقُولُهُ لِخَيْرِكُمْ لَيْسَ لِكَيْ أُلْقِيَ عَلَيْكُمْ وَهَقاً بَلْ لأَجْلِ اللِّيَاقَةِ وَالْمُثَابَرَةِ لِلرَّبِّ مِنْ دُونِ ارْتِبَاكٍ.
\par 36 وَلَكِنْ إِنْ كَانَ أَحَدٌ يَظُنُّ أَنَّهُ يَعْمَلُ بِدُونِ لِيَاقَةٍ نَحْوَ عَذْرَائِهِ إِذَا تَجَاوَزَتِ الْوَقْتَ وَهَكَذَا لَزِمَ أَنْ يَصِيرَ فَلْيَفْعَلْ مَا يُرِيدُ. إِنَّهُ لاَ يُخْطِئُ. فَلْيَتَزَوَّجَا.
\par 37 وَأَمَّا مَنْ أَقَامَ رَاسِخاً فِي قَلْبِهِ وَلَيْسَ لَهُ اضْطِرَارٌ بَلْ لَهُ سُلْطَانٌ عَلَى إِرَادَتِهِ وَقَدْ عَزَمَ عَلَى هَذَا فِي قَلْبِهِ أَنْ يَحْفَظَ عَذْرَاءَهُ فَحَسَناً يَفْعَلُ.
\par 38 إِذاً مَنْ زَوَّجَ فَحَسَناً يَفْعَلُ وَمَنْ لاَ يُزَوِّجُ يَفْعَلُ أَحْسَنَ.
\par 39 الْمَرْأَةُ مُرْتَبِطَةٌ بِالنَّامُوسِ مَا دَامَ رَجُلُهَا حَيّاً. وَلَكِنْ إِنْ مَاتَ رَجُلُهَا فَهِيَ حُرَّةٌ لِكَيْ تَتَزَوَّجَ بِمَنْ تُرِيدُ فِي الرَّبِّ فَقَطْ.
\par 40 وَلَكِنَّهَا أَكْثَرُ غِبْطَةً إِنْ لَبِثَتْ هَكَذَا بِحَسَبِ رَأْيِي. وَأَظُنُّ أَنِّي أَنَا أَيْضاً عِنْدِي رُوحُ اللهِ.

\chapter{8}

\par 1 وَأَمَّا مِنْ جِهَةِ مَا ذُبِحَ لِلأَوْثَانِ فَنَعْلَمُ أَنَّ لِجَمِيعِنَا عِلْماً. الْعِلْمُ يَنْفُخُ وَلَكِنَّ الْمَحَبَّةَ تَبْنِي.
\par 2 فَإِنْ كَانَ أَحَدٌ يَظُنُّ أَنَّهُ يَعْرِفُ شَيْئاً فَإِنَّهُ لَمْ يَعْرِفْ شَيْئاً بَعْدُ كَمَا يَجِبُ أَنْ يَعْرِفَ!
\par 3 وَلَكِنْ إِنْ كَانَ أَحَدٌ يُحِبُّ اللهَ فَهَذَا مَعْرُوفٌ عِنْدَهُ.
\par 4 فَمِنْ جِهَةِ أَكْلِ مَا ذُبِحَ لِلأَوْثَانِ نَعْلَمُ أَنْ لَيْسَ وَثَنٌ فِي الْعَالَمِ وَأَنْ لَيْسَ إِلَهٌ آخَرُ إِلاَّ وَاحِداً.
\par 5 لأَنَّهُ وَإِنْ وُجِدَ مَا يُسَمَّى آلِهَةً سِوَاءٌ كَانَ فِي السَّمَاءِ أَوْ عَلَى الأَرْضِ كَمَا يُوجَدُ آلِهَةٌ كَثِيرُونَ وَأَرْبَابٌ كَثِيرُونَ.
\par 6 لَكِنْ لَنَا إِلَهٌ وَاحِدٌ: الآبُ الَّذِي مِنْهُ جَمِيعُ الأَشْيَاءِ وَنَحْنُ لَهُ. وَرَبٌّ وَاحِدٌ: يَسُوعُ الْمَسِيحُ الَّذِي بِهِ جَمِيعُ الأَشْيَاءِ وَنَحْنُ بِهِ.
\par 7 وَلَكِنْ لَيْسَ الْعِلْمُ فِي الْجَمِيعِ. بَلْ أُنَاسٌ بِالضَّمِيرِ نَحْوَ الْوَثَنِ إِلَى الآنَ يَأْكُلُونَ كَأَنَّهُ مِمَّا ذُبِحَ لِوَثَنٍ. فَضَمِيرُهُمْ إِذْ هُوَ ضَعِيفٌ يَتَنَجَّسُ.
\par 8 وَلَكِنَّ الطَّعَامَ لاَ يُقَدِّمُنَا إِلَى اللهِ لأَنَّنَا إِنْ أَكَلْنَا لاَ نَزِيدُ وَإِنْ لَمْ نَأْكُلْ لاَ نَنْقُصُ.
\par 9 وَلَكِنِ انْظُرُوا لِئَلاَّ يَصِيرَ سُلْطَانُكُمْ هَذَا مَعْثَرَةً لِلضُّعَفَاءِ.
\par 10 لأَنَّهُ إِنْ رَآكَ أَحَدٌ يَا مَنْ لَهُ عِلْمٌ مُتَّكِئاً فِي هَيْكَلِ وَثَنٍ أَفَلاَ يَتَقَوَّى ضَمِيرُهُ إِذْ هُوَ ضَعِيفٌ حَتَّى يَأْكُلَ مَا ذُبِحَ لِلأَوْثَانِ؟
\par 11 فَيَهْلِكَ بِسَبَبِ عِلْمِكَ الأَخُ الضَّعِيفُ الَّذِي مَاتَ الْمَسِيحُ مِنْ أَجْلِهِ.
\par 12 وَهَكَذَا إِذْ تُخْطِئُونَ إِلَى الإِخْوَةِ وَتَجْرَحُونَ ضَمِيرَهُمُ الضَّعِيفَ تُخْطِئُونَ إِلَى الْمَسِيحِ.
\par 13 لِذَلِكَ إِنْ كَانَ طَعَامٌ يُعْثِرُ أَخِي فَلَنْ آكُلَ لَحْماً إِلَى الأَبَدِ لِئَلاَّ أُعْثِرَ أَخِي.

\chapter{9}

\par 1 أَلَسْتُ أَنَا رَسُولاً؟ أَلَسْتُ أَنَا حُرّاً؟ أَمَا رَأَيْتُ يَسُوعَ الْمَسِيحَ رَبَّنَا؟ أَلَسْتُمْ أَنْتُمْ عَمَلِي فِي الرَّبِّ؟!
\par 2 إِنْ كُنْتُ لَسْتُ رَسُولاً إِلَى آخَرِينَ فَإِنَّمَا أَنَا إِلَيْكُمْ رَسُولٌ لأَنَّكُمْ أَنْتُمْ خَتْمُ رِسَالَتِي فِي الرَّبِّ.
\par 3 هَذَا هُوَ احْتِجَاجِي عِنْدَ الَّذِينَ يَفْحَصُونَنِي.
\par 4 أَلَعَلَّنَا لَيْسَ لَنَا سُلْطَانٌ أَنْ نَأْكُلَ وَنَشْرَبَ؟
\par 5 أَلَعَلَّنَا لَيْسَ لَنَا سُلْطَانٌ أَنْ نَجُولَ بِأُخْتٍ زَوْجَةً كَبَاقِي الرُّسُلِ وَإِخْوَةِ الرَّبِّ وَصَفَا؟
\par 6 أَمْ أَنَا وَبَرْنَابَا وَحْدَنَا لَيْسَ لَنَا سُلْطَانٌ أَنْ لاَ نَشْتَغِلَ؟
\par 7 مَنْ تَجَنَّدَ قَطُّ بِنَفَقَةِ نَفْسِهِ؟ وَمَنْ يَغْرِسُ كَرْماً وَمِنْ ثَمَرِهِ لاَ يَأْكُلُ؟ أَوْ مَنْ يَرْعَى رَعِيَّةً وَمِنْ لَبَنِ الرَّعِيَّةِ لاَ يَأْكُلُ؟
\par 8 أَلَعَلِّي أَتَكَلَّمُ بِهَذَا كَإِنْسَانٍ؟ أَمْ لَيْسَ النَّامُوسُ أَيْضاً يَقُولُ هَذَا؟
\par 9 فَإِنَّهُ مَكْتُوبٌ فِي نَامُوسِ مُوسَى: «لاَ تَكُمَّ ثَوْراً دَارِساً». أَلَعَلَّ اللهَ تُهِمُّهُ الثِّيرَانُ؟
\par 10 أَمْ يَقُولُ مُطْلَقاً مِنْ أَجْلِنَا؟ إِنَّهُ مِنْ أَجْلِنَا مَكْتُوبٌ. لأَنَّهُ يَنْبَغِي لِلْحَرَّاثِ أَنْ يَحْرُثَ عَلَى رَجَاءٍ وَلِلدَّارِسِ عَلَى الرَّجَاءِ أَنْ يَكُونَ شَرِيكاً فِي رَجَائِهِ.
\par 11 إِنْ كُنَّا نَحْنُ قَدْ زَرَعْنَا لَكُمُ الرُّوحِيَّاتِ أَفَعَظِيمٌ إِنْ حَصَدْنَا مِنْكُمُ الْجَسَدِيَّاتِ؟
\par 12 إِنْ كَانَ آخَرُونَ شُرَكَاءَ فِي السُّلْطَانِ عَلَيْكُمْ أَفَلَسْنَا نَحْنُ بِالأَوْلَى؟ لَكِنَّنَا لَمْ نَسْتَعْمِلْ هَذَا السُّلْطَانَ بَلْ نَتَحَمَّلُ كُلَّ شَيْءٍ لِئَلاَّ نَجْعَلَ عَائِقاً لِإِنْجِيلِ الْمَسِيحِ.
\par 13 أَلَسْتُمْ تَعْلَمُونَ أَنَّ الَّذِينَ يَعْمَلُونَ فِي الأَشْيَاءِ الْمُقَدَّسَةِ مِنَ الْهَيْكَلِ يَأْكُلُونَ؟ الَّذِينَ يُلاَزِمُونَ الْمَذْبَحَ يُشَارِكُونَ الْمَذْبَحَ.
\par 14 هَكَذَا أَيْضاً أَمَرَ الرَّبُّ: أَنَّ الَّذِينَ يُنَادُونَ بِالإِنْجِيلِ مِنَ الإِنْجِيلِ يَعِيشُونَ.
\par 15 أَمَّا أَنَا فَلَمْ أَسْتَعْمِلْ شَيْئاً مِنْ هَذَا وَلاَ كَتَبْتُ هَذَا لِكَيْ يَصِيرَ فِيَّ هَكَذَا. لأَنَّهُ خَيْرٌ لِي أَنْ أَمُوتَ مِنْ أَنْ يُعَطِّلَ أَحَدٌ فَخْرِي.
\par 16 لأَنَّهُ إِنْ كُنْتُ أُبَشِّرُ فَلَيْسَ لِي فَخْرٌ إِذِ الضَّرُورَةُ مَوْضُوعَةٌ عَلَيَّ فَوَيْلٌ لِي إِنْ كُنْتُ لاَ أُبَشِّرُ.
\par 17 فَإِنَّهُ إِنْ كُنْتُ أَفْعَلُ هَذَا طَوْعاً فَلِي أَجْرٌ وَلَكِنْ إِنْ كَانَ كَرْهاً فَقَدِ اسْتُؤْمِنْتُ عَلَى وَكَالَةٍ.
\par 18 فَمَا هُوَ أَجْرِي؟ إِذْ وَأَنَا أُبَشِّرُ أَجْعَلُ إِنْجِيلَ الْمَسِيحِ بِلاَ نَفَقَةٍ حَتَّى لَمْ أَسْتَعْمِلْ سُلْطَانِي فِي الإِنْجِيلِ.
\par 19 فَإِنِّي إِذْ كُنْتُ حُرّاً مِنَ الْجَمِيعِ اسْتَعْبَدْتُ نَفْسِي لِلْجَمِيعِ لأَرْبَحَ الأَكْثَرِينَ.
\par 20 فَصِرْتُ لِلْيَهُودِ كَيَهُودِيٍّ لأَرْبَحَ الْيَهُودَ وَلِلَّذِينَ تَحْتَ النَّامُوسِ كَأَنِّي تَحْتَ النَّامُوسِ لأَرْبَحَ الَّذِينَ تَحْتَ النَّامُوسِ
\par 21 وَلِلَّذِينَ بِلاَ نَامُوسٍ كَأَنِّي بِلاَ نَامُوسٍ - مَعَ أَنِّي لَسْتُ بِلاَ نَامُوسٍ لِلَّهِ بَلْ تَحْتَ نَامُوسٍ لِلْمَسِيحِ - لأَرْبَحَ الَّذِينَ بِلاَ نَامُوسٍ.
\par 22 صِرْتُ لِلضُّعَفَاءِ كَضَعِيفٍ لأَرْبَحَ الضُّعَفَاءَ. صِرْتُ لِلْكُلِّ كُلَّ شَيْءٍ لأُخَلِّصَ عَلَى كُلِّ حَالٍ قَوْماً.
\par 23 وَهَذَا أَنَا أَفْعَلُهُ لأَجْلِ الإِنْجِيلِ لأَكُونَ شَرِيكاً فِيهِ.
\par 24 أَلَسْتُمْ تَعْلَمُونَ أَنَّ الَّذِينَ يَرْكُضُونَ فِي الْمَِيْدَانِ جَمِيعُهُمْ يَرْكُضُونَ وَلَكِنَّ وَاحِداً يَأْخُذُ الْجَعَالَةَ؟ هَكَذَا ارْكُضُوا لِكَيْ تَنَالُوا.
\par 25 وَكُلُّ مَنْ يُجَاهِدُ يَضْبِطُ نَفْسَهُ فِي كُلِّ شَيْءٍ. أَمَّا أُولَئِكَ فَلِكَيْ يَأْخُذُوا إِكْلِيلاً يَفْنَى وَأَمَّا نَحْنُ فَإِكْلِيلاً لاَ يَفْنَى.
\par 26 إِذاً أَنَا أَرْكُضُ هَكَذَا كَأَنَّهُ لَيْسَ عَنْ غَيْرِ يَقِينٍ. هَكَذَا أُضَارِبُ كَأَنِّي لاَ أَضْرِبُ الْهَوَاءَ.
\par 27 بَلْ أَقْمَعُ جَسَدِي وَأَسْتَعْبِدُهُ حَتَّى بَعْدَ مَا كَرَزْتُ لِلآخَرِينَ لاَ أَصِيرُ أَنَا نَفْسِي مَرْفُوضاً.

\chapter{10}

\par 1 فَإِنِّي لَسْتُ أُرِيدُ أَيُّهَا الإِخْوَةُ أَنْ تَجْهَلُوا أَنَّ آبَاءَنَا جَمِيعَهُمْ كَانُوا تَحْتَ السَّحَابَةِ وَجَمِيعَهُمُ اجْتَازُوا فِي الْبَحْرِ
\par 2 وَجَمِيعَهُمُ اعْتَمَدُوا لِمُوسَى فِي السَّحَابَةِ وَفِي الْبَحْرِ
\par 3 وَجَمِيعَهُمْ أَكَلُوا طَعَاماً وَاحِداً رُوحِيّاً
\par 4 وَجَمِيعَهُمْ شَرِبُوا شَرَاباً وَاحِداً رُوحِيّاً - لأَنَّهُمْ كَانُوا يَشْرَبُونَ مِنْ صَخْرَةٍ رُوحِيَّةٍ تَابِعَتِهِمْ وَالصَّخْرَةُ كَانَتِ الْمَسِيحَ.
\par 5 لَكِنْ بِأَكْثَرِهِمْ لَمْ يُسَرَّ اللهُ لأَنَّهُمْ طُرِحُوا فِي الْقَفْرِ.
\par 6 وَهَذِهِ الأُمُورُ حَدَثَتْ مِثَالاً لَنَا حَتَّى لاَ نَكُونَ نَحْنُ مُشْتَهِينَ شُرُوراً كَمَا اشْتَهَى أُولَئِكَ.
\par 7 فَلاَ تَكُونُوا عَبَدَةَ أَوْثَانٍ كَمَا كَانَ أُنَاسٌ مِنْهُمْ كَمَا هُوَ مَكْتُوبٌ: «جَلَسَ الشَّعْبُ لِلأَكْلِ وَالشُّرْبِ ثُمَّ قَامُوا لِلَّعِبِ».
\par 8 وَلاَ نَزْنِ كَمَا زَنَى أُنَاسٌ مِنْهُمْ فَسَقَطَ فِي يَوْمٍ وَاحِدٍ ثَلاَثَةٌ وَعِشْرُونَ أَلْفاً.
\par 9 وَلاَ نُجَرِّبِ الْمَسِيحَ كَمَا جَرَّبَ أَيْضاً أُنَاسٌ مِنْهُمْ فَأَهْلَكَتْهُمُ الْحَيَّاتُ.
\par 10 وَلاَ تَتَذَمَّرُوا كَمَا تَذَمَّرَ أَيْضاً أُنَاسٌ مِنْهُمْ فَأَهْلَكَهُمُ الْمُهْلِكُ.
\par 11 فَهَذِهِ الأُمُورُ جَمِيعُهَا أَصَابَتْهُمْ مِثَالاً وَكُتِبَتْ لِإِنْذَارِنَا نَحْنُ الَّذِينَ انْتَهَتْ إِلَيْنَا أَوَاخِرُ الدُّهُورِ.
\par 12 إِذاً مَنْ يَظُنُّ أَنَّهُ قَائِمٌ فَلْيَنْظُرْ أَنْ لاَ يَسْقُطَ.
\par 13 لَمْ تُصِبْكُمْ تَجْرِبَةٌ إِلاَّ بَشَرِيَّةٌ. وَلَكِنَّ اللهَ أَمِينٌ الَّذِي لاَ يَدَعُكُمْ تُجَرَّبُونَ فَوْقَ مَا تَسْتَطِيعُونَ بَلْ سَيَجْعَلُ مَعَ التَّجْرِبَةِ أَيْضاً الْمَنْفَذَ لِتَسْتَطِيعُوا أَنْ تَحْتَمِلُوا.
\par 14 لِذَلِكَ يَا أَحِبَّائِي اهْرُبُوا مِنْ عِبَادَةِ الأَوْثَانِ.
\par 15 أَقُولُ كَمَا لِلْحُكَمَاءِ: احْكُمُوا أَنْتُمْ فِي مَا أَقُولُ.
\par 16 كَأْسُ الْبَرَكَةِ الَّتِي نُبَارِكُهَا أَلَيْسَتْ هِيَ شَرِكَةَ دَمِ الْمَسِيحِ؟ الْخُبْزُ الَّذِي نَكْسِرُهُ أَلَيْسَ هُوَ شَرِكَةَ جَسَدِ الْمَسِيحِ؟
\par 17 فَإِنَّنَا نَحْنُ الْكَثِيرِينَ خُبْزٌ وَاحِدٌ جَسَدٌ وَاحِدٌ لأَنَّنَا جَمِيعَنَا نَشْتَرِكُ فِي الْخُبْزِ الْوَاحِدِ.
\par 18 انْظُرُوا إِسْرَائِيلَ حَسَبَ الْجَسَدِ. أَلَيْسَ الَّذِينَ يَأْكُلُونَ الذَّبَائِحَ هُمْ شُرَكَاءَ الْمَذْبَحِ؟
\par 19 فَمَاذَا أَقُولُ؟ أَإِنَّ الْوَثَنَ شَيْءٌ أَوْ إِنَّ مَا ذُبِحَ لِلْوَثَنِ شَيْءٌ؟
\par 20 بَلْ إِنَّ مَا يَذْبَحُهُ الأُمَمُ فَإِنَّمَا يَذْبَحُونَهُ لِلشَّيَاطِينِ لاَ لِلَّهِ. فَلَسْتُ أُرِيدُ أَنْ تَكُونُوا أَنْتُمْ شُرَكَاءَ الشَّيَاطِينِ.
\par 21 لاَ تَقْدِرُونَ أَنْ تَشْرَبُوا كَأْسَ الرَّبِّ وَكَأْسَ شَيَاطِينَ. لاَ تَقْدِرُونَ أَنْ تَشْتَرِكُوا فِي مَائِدَةِ الرَّبِّ وَفِي مَائِدَةِ شَيَاطِينَ.
\par 22 أَمْ نُغِيرُ الرَّبَّ؟ أَلَعَلَّنَا أَقْوَى مِنْهُ؟
\par 23 كُلُّ الأَشْيَاءِ تَحِلُّ لِي لَكِنْ لَيْسَ كُلُّ الأَشْيَاءِ تُوافِقُ. كُلُّ الأَشْيَاءِ تَحِلُّ لِي وَلَكِنْ لَيْسَ كُلُّ الأَشْيَاءِ تَبْنِي.
\par 24 لاَ يَطْلُبْ أَحَدٌ مَا هُوَ لِنَفْسِهِ بَلْ كُلُّ وَاحِدٍ مَا هُوَ لِلآخَرِ.
\par 25 كُلُّ مَا يُبَاعُ فِي الْمَلْحَمَةِ كُلُوهُ غَيْرَ فَاحِصِينَ عَنْ شَيْءٍ مِنْ أَجْلِ الضَّمِيرِ
\par 26 لأَنَّ لِلرَّبِّ الأَرْضَ وَمِلأَهَا.
\par 27 وَإِنْ كَانَ أَحَدٌ مِنْ غَيْرِ الْمُؤْمِنِينَ يَدْعُوكُمْ وَتُرِيدُونَ أَنْ تَذْهَبُوا فَكُلُّ مَا يُقَدَّمُ لَكُمْ كُلُوا مِنْهُ غَيْرَ فَاحِصِينَ مِنْ أَجْلِ الضَّمِيرِ.
\par 28 وَلَكِنْ إِنْ قَالَ لَكُمْ أَحَدٌ: «هَذَا مَذْبُوحٌ لِوَثَنٍ» فَلاَ تَأْكُلُوا مِنْ أَجْلِ ذَاكَ الَّذِي أَعْلَمَكُمْ وَالضَّمِيرِ. لأَنَّ لِلرَّبِّ الأَرْضَ وَمِلأَهَا
\par 29 أَقُولُ الضَّمِيرُ - لَيْسَ ضَمِيرَكَ أَنْتَ بَلْ ضَمِيرُ الآخَرِ. لأَنَّهُ لِمَاذَا يُحْكَمُ فِي حُرِّيَّتِي مِنْ ضَمِيرِ آخَرَ؟
\par 30 فَإِنْ كُنْتُ أَنَا أَتَنَاوَلُ بِشُكْرٍ فَلِمَاذَا يُفْتَرَى عَلَيَّ لأَجْلِ مَا أَشْكُرُ عَلَيْهِ؟
\par 31 فَإِذَا كُنْتُمْ تَأْكُلُونَ أَوْ تَشْرَبُونَ أَوْ تَفْعَلُونَ شَيْئاً فَافْعَلُوا كُلَّ شَيْءٍ لِمَجْدِ اللهِ.
\par 32 كُونُوا بِلاَ عَثْرَةٍ لِلْيَهُودِ وَلِلْيُونَانِيِّينَ وَلِكَنِيسَةِ اللهِ.
\par 33 كَمَا أَنَا أَيْضاً أُرْضِي الْجَمِيعَ فِي كُلِّ شَيْءٍ غَيْرَ طَالِبٍ مَا يُوافِقُ نَفْسِي بَلِ الْكَثِيرِينَ لِكَيْ يَخْلُصُوا.

\chapter{11}

\par 1 كُونُوا مُتَمَثِّلِينَ بِي كَمَا أَنَا أَيْضاً بِالْمَسِيحِ.
\par 2 فَأَمْدَحُكُمْ أَيُّهَا الإِخْوَةُ عَلَى أَنَّكُمْ تَذْكُرُونَنِي فِي كُلِّ شَيْءٍ وَتَحْفَظُونَ التَّعَالِيمَ كَمَا سَلَّمْتُهَا إِلَيْكُمْ.
\par 3 وَلَكِنْ أُرِيدُ أَنْ تَعْلَمُوا أَنَّ رَأْسَ كُلِّ رَجُلٍ هُوَ الْمَسِيحُ. وَأَمَّا رَأْسُ الْمَرْأَةِ فَهُوَ الرَّجُلُ. وَرَأْسُ الْمَسِيحِ هُوَ اللهُ.
\par 4 كُلُّ رَجُلٍ يُصَلِّي أَوْ يَتَنَبَّأُ وَلَهُ عَلَى رَأْسِهِ شَيْءٌ يَشِينُ رَأْسَهُ.
\par 5 وَأَمَّا كُلُّ امْرَأَةٍ تُصَلِّي أَوْ تَتَنَبَّأُ وَرَأْسُهَا غَيْرُ مُغَطّىً فَتَشِينُ رَأْسَهَا لأَنَّهَا وَالْمَحْلُوقَةَ شَيْءٌ وَاحِدٌ بِعَيْنِهِ.
\par 6 إِذِ الْمَرْأَةُ إِنْ كَانَتْ لاَ تَتَغَطَّى فَلْيُقَصَّ شَعَرُهَا. وَإِنْ كَانَ قَبِيحاً بِالْمَرْأَةِ أَنْ تُقَصَّ أَوْ تُحْلَقَ فَلْتَتَغَطَّ.
\par 7 فَإِنَّ الرَّجُلَ لاَ يَنْبَغِي أَنْ يُغَطِّيَ رَأْسَهُ لِكَوْنِهِ صُورَةَ اللهِ وَمَجْدَهُ. وَأَمَّا الْمَرْأَةُ فَهِيَ مَجْدُ الرَّجُلِ.
\par 8 لأَنَّ الرَّجُلَ لَيْسَ مِنَ الْمَرْأَةِ بَلِ الْمَرْأَةُ مِنَ الرَّجُلِ.
\par 9 وَلأَنَّ الرَّجُلَ لَمْ يُخْلَقْ مِنْ أَجْلِ الْمَرْأَةِ بَلِ الْمَرْأَةُ مِنْ أَجْلِ الرَّجُلِ.
\par 10 لِهَذَا يَنْبَغِي لِلْمَرْأَةِ أَنْ يَكُونَ لَهَا سُلْطَانٌ عَلَى رَأْسِهَا مِنْ أَجْلِ الْمَلاَئِكَةِ.
\par 11 غَيْرَ أَنَّ الرَّجُلَ لَيْسَ مِنْ دُونِ الْمَرْأَةِ وَلاَ الْمَرْأَةُ مِنْ دُونِ الرَّجُلِ فِي الرَّبِّ.
\par 12 لأَنَّهُ كَمَا أَنَّ الْمَرْأَةَ هِيَ مِنَ الرَّجُلِ هَكَذَا الرَّجُلُ أَيْضاً هُوَ بِالْمَرْأَةِ. وَلَكِنَّ جَمِيعَ الأَشْيَاءِ هِيَ مِنَ اللهِ.
\par 13 احْكُمُوا فِي أَنْفُسِكُمْ: هَلْ يَلِيقُ بِالْمَرْأَةِ أَنْ تُصَلِّيَ إِلَى اللهِ وَهِيَ غَيْرُ مُغَطَّاةٍ؟
\par 14 أَمْ لَيْسَتِ الطَّبِيعَةُ نَفْسُهَا تُعَلِّمُكُمْ أَنَّ الرَّجُلَ إِنْ كَانَ يُرْخِي شَعْرَهُ فَهُوَ عَيْبٌ لَهُ؟
\par 15 وَأَمَّا الْمَرْأَةُ إِنْ كَانَتْ تُرْخِي شَعْرَهَا فَهُوَ مَجْدٌ لَهَا لأَنَّ الشَّعْرَ قَدْ أُعْطِيَ لَهَا عِوَضَ بُرْقُعٍ.
\par 16 وَلَكِنْ إِنْ كَانَ أَحَدٌ يُظْهِرُ أَنَّهُ يُحِبُّ الْخِصَامَ فَلَيْسَ لَنَا نَحْنُ عَادَةٌ مِثْلُ هَذِهِ وَلاَ لِكَنَائِسِ اللهِ.
\par 17 وَلَكِنَّنِي إِذْ أُوصِي بِهَذَا لَسْتُ أَمْدَحُ كَوْنَكُمْ تَجْتَمِعُونَ لَيْسَ لِلأَفْضَلِ بَلْ لِلأَرْدَإِ.
\par 18 لأَنِّي أَوَّلاً حِينَ تَجْتَمِعُونَ فِي الْكَنِيسَةِ أَسْمَعُ أَنَّ بَيْنَكُمُ انْشِقَاقَاتٍ وَأُصَدِّقُ بَعْضَ التَّصْدِيقِ.
\par 19 لأَنَّهُ لاَ بُدَّ أَنْ يَكُونَ بَيْنَكُمْ بِدَعٌ أَيْضاً لِيَكُونَ الْمُزَكَّوْنَ ظَاهِرِينَ بَيْنَكُمْ.
\par 20 فَحِينَ تَجْتَمِعُونَ مَعاً لَيْسَ هُوَ لأَكْلِ عَشَاءِ الرَّبِّ.
\par 21 لأَنَّ كُلَّ وَاحِدٍ يَسْبِقُ فَيَأْخُذُ عَشَاءَ نَفْسِهِ فِي الأَكْلِ فَالْوَاحِدُ يَجُوعُ وَالآخَرُ يَسْكَرُ.
\par 22 أَفَلَيْسَ لَكُمْ بُيُوتٌ لِتَأْكُلُوا فِيهَا وَتَشْرَبُوا؟ أَمْ تَسْتَهِينُونَ بِكَنِيسَةِ اللهِ وَتُخْجِلُونَ الَّذِينَ لَيْسَ لَهُمْ؟ مَاذَا أَقُولُ لَكُمْ! أَأَمْدَحُكُمْ عَلَى هَذَا؟ لَسْتُ أَمْدَحُكُمْ!
\par 23 لأَنَّنِي تَسَلَّمْتُ مِنَ الرَّبِّ مَا سَلَّمْتُكُمْ أَيْضاً: إِنَّ الرَّبَّ يَسُوعَ فِي اللَّيْلَةِ الَّتِي أُسْلِمَ فِيهَا أَخَذَ خُبْزاً
\par 24 وَشَكَرَ فَكَسَّرَ وَقَالَ: «خُذُوا كُلُوا هَذَا هُوَ جَسَدِي الْمَكْسُورُ لأَجْلِكُمُ. اصْنَعُوا هَذَا لِذِكْرِي».
\par 25 كَذَلِكَ الْكَأْسَ أَيْضاً بَعْدَمَا تَعَشَّوْا قَائِلاً: «هَذِهِ الْكَأْسُ هِيَ الْعَهْدُ الْجَدِيدُ بِدَمِي. اصْنَعُوا هَذَا كُلَّمَا شَرِبْتُمْ لِذِكْرِي».
\par 26 فَإِنَّكُمْ كُلَّمَا أَكَلْتُمْ هَذَا الْخُبْزَ وَشَرِبْتُمْ هَذِهِ الْكَأْسَ تُخْبِرُونَ بِمَوْتِ الرَّبِّ إِلَى أَنْ يَجِيءَ.
\par 27 إِذاً أَيُّ مَنْ أَكَلَ هَذَا الْخُبْزَ أَوْ شَرِبَ كَأْسَ الرَّبِّ بِدُونِ اسْتِحْقَاقٍ يَكُونُ مُجْرِماً فِي جَسَدِ الرَّبِّ وَدَمِهِ.
\par 28 وَلَكِنْ لِيَمْتَحِنِ الإِنْسَانُ نَفْسَهُ وَهَكَذَا يَأْكُلُ مِنَ الْخُبْزِ وَيَشْرَبُ مِنَ الْكَأْسِ.
\par 29 لأَنَّ الَّذِي يَأْكُلُ وَيَشْرَبُ بِدُونِ اسْتِحْقَاقٍ يَأْكُلُ وَيَشْرَبُ دَيْنُونَةً لِنَفْسِهِ غَيْرَ مُمَيِّزٍ جَسَدَ الرَّبِّ.
\par 30 مِنْ أَجْلِ هَذَا فِيكُمْ كَثِيرُونَ ضُعَفَاءُ وَمَرْضَى وَكَثِيرُونَ يَرْقُدُونَ.
\par 31 لأَنَّنَا لَوْ كُنَّا حَكَمْنَا عَلَى أَنْفُسِنَا لَمَا حُكِمَ عَلَيْنَا
\par 32 وَلَكِنْ إِذْ قَدْ حُكِمَ عَلَيْنَا نُؤَدَّبُ مِنَ الرَّبِّ لِكَيْ لاَ نُدَانَ مَعَ الْعَالَمِ.
\par 33 إِذاً يَا إِخْوَتِي حِينَ تَجْتَمِعُونَ لِلأَكْلِ انْتَظِرُوا بَعْضُكُمْ بَعْضاً.
\par 34 إِنْ كَانَ أَحَدٌ يَجُوعُ فَلْيَأْكُلْ فِي الْبَيْتِ كَيْ لاَ تَجْتَمِعُوا لِلدَّيْنُونَةِ. وَأَمَّا الأُمُورُ الْبَاقِيَةُ فَعِنْدَمَا أَجِيءُ أُرَتِّبُهَا.

\chapter{12}

\par 1 وَأَمَّا مِنْ جِهَةِ الْمَوَاهِبِ الرُّوحِيَّةِ أَيُّهَا الإِخْوَةُ فَلَسْتُ أُرِيدُ أَنْ تَجْهَلُوا.
\par 2 أَنْتُمْ تَعْلَمُونَ أَنَّكُمْ كُنْتُمْ أُمَماً مُنْقَادِينَ إِلَى الأَوْثَانِ الْبُكْمِ كَمَا كُنْتُمْ تُسَاقُونَ.
\par 3 لِذَلِكَ أُعَرِّفُكُمْ أَنْ لَيْسَ أَحَدٌ وَهُوَ يَتَكَلَّمُ بِرُوحِ اللهِ يَقُولُ: «يَسُوعُ أَنَاثِيمَا». وَلَيْسَ أَحَدٌ يَقْدِرُ أَنْ يَقُولَ: «يَسُوعُ رَبٌّ» إِلاَّ بِالرُّوحِ الْقُدُسِ.
\par 4 فَأَنْوَاعُ مَوَاهِبَ مَوْجُودَةٌ وَلَكِنَّ الرُّوحَ وَاحِدٌ.
\par 5 وَأَنْوَاعُ خِدَمٍ مَوْجُودَةٌ وَلَكِنَّ الرَّبَّ وَاحِدٌ.
\par 6 وَأَنْوَاعُ أَعْمَالٍ مَوْجُودَةٌ وَلَكِنَّ اللهَ وَاحِدٌ الَّذِي يَعْمَلُ الْكُلَّ فِي الْكُلِّ.
\par 7 وَلَكِنَّهُ لِكُلِّ وَاحِدٍ يُعْطَى إِظْهَارُ الرُّوحِ لِلْمَنْفَعَةِ.
\par 8 فَإِنَّهُ لِوَاحِدٍ يُعْطَى بِالرُّوحِ كَلاَمُ حِكْمَةٍ. وَلِآخَرَ كَلاَمُ عِلْمٍ بِحَسَبِ الرُّوحِ الْوَاحِدِ.
\par 9 وَلِآخَرَ إِيمَانٌ بِالرُّوحِ الْوَاحِدِ. وَلِآخَرَ مَوَاهِبُ شِفَاءٍ بِالرُّوحِ الْوَاحِدِ.
\par 10 وَلِآخَرَ عَمَلُ قُوَّاتٍ وَلِآخَرَ نُبُوَّةٌ وَلِآخَرَ تَمْيِيزُ الأَرْوَاحِ وَلِآخَرَ أَنْوَاعُ أَلْسِنَةٍ وَلِآخَرَ تَرْجَمَةُ أَلْسِنَةٍ.
\par 11 وَلَكِنَّ هَذِهِ كُلَّهَا يَعْمَلُهَا الرُّوحُ الْوَاحِدُ بِعَيْنِهِ قَاسِماً لِكُلِّ وَاحِدٍ بِمُفْرَدِهِ كَمَا يَشَاءُ.
\par 12 لأَنَّهُ كَمَا أَنَّ الْجَسَدَ هُوَ وَاحِدٌ وَلَهُ أَعْضَاءٌ كَثِيرَةٌ وَكُلُّ أَعْضَاءِ الْجَسَدِ الْوَاحِدِ إِذَا كَانَتْ كَثِيرَةً هِيَ جَسَدٌ وَاحِدٌ كَذَلِكَ الْمَسِيحُ أَيْضاً.
\par 13 لأَنَّنَا جَمِيعَنَا بِرُوحٍ وَاحِدٍ أَيْضاً اعْتَمَدْنَا إِلَى جَسَدٍ وَاحِدٍ يَهُوداً كُنَّا أَمْ يُونَانِيِّينَ عَبِيداً أَمْ أَحْرَاراً. وَجَمِيعُنَا سُقِينَا رُوحاً وَاحِداً.
\par 14 فَإِنَّ الْجَسَدَ أَيْضاً لَيْسَ عُضْواً وَاحِداً بَلْ أَعْضَاءٌ كَثِيرَةٌ.
\par 15 إِنْ قَالَتِ الرِّجْلُ: «لأَنِّي لَسْتُ يَداً لَسْتُ مِنَ الْجَسَدِ». أَفَلَمْ تَكُنْ لِذَلِكَ مِنَ الْجَسَدِ؟
\par 16 وَإِنْ قَالَتِ الأُذُنُ: «لأَنِّي لَسْتُ عَيْناً لَسْتُ مِنَ الْجَسَدِ». أَفَلَمْ تَكُنْ لِذَلِكَ مِنَ الْجَسَدِ؟
\par 17 لَوْ كَانَ كُلُّ الْجَسَدِ عَيْناً فَأَيْنَ السَّمْعُ؟ لَوْ كَانَ الْكُلُّ سَمْعاً فَأَيْنَ الشَّمُّ؟
\par 18 وَأَمَّا الآنَ فَقَدْ وَضَعَ اللهُ الأَعْضَاءَ كُلَّ وَاحِدٍ مِنْهَا فِي الْجَسَدِ كَمَا أَرَادَ.
\par 19 وَلَكِنْ لَوْ كَانَ جَمِيعُهَا عُضْواً وَاحِداً أَيْنَ الْجَسَدُ؟
\par 20 فَالآنَ أَعْضَاءٌ كَثِيرَةٌ وَلَكِنْ جَسَدٌ وَاحِدٌ.
\par 21 لاَ تَقْدِرُ الْعَيْنُ أَنْ تَقُولَ لِلْيَدِ: «لاَ حَاجَةَ لِي إِلَيْكِ». أَوِ الرَّأْسُ أَيْضاً لِلرِّجْلَيْنِ: «لاَ حَاجَةَ لِي إِلَيْكُمَا».
\par 22 بَلْ بِالأَوْلَى أَعْضَاءُ الْجَسَدِ الَّتِي تَظْهَرُ أَضْعَفَ هِيَ ضَرُورِيَّةٌ.
\par 23 وَأَعْضَاءُ الْجَسَدِ الَّتِي نَحْسِبُ أَنَّهَا بِلاَ كَرَامَةٍ نُعْطِيهَا كَرَامَةً أَفْضَلَ. وَالأَعْضَاءُ الْقَبِيحَةُ فِينَا لَهَا جَمَالٌ أَفْضَلُ.
\par 24 وَأَمَّا الْجَمِيلَةُ فِينَا فَلَيْسَ لَهَا احْتِيَاجٌ. لَكِنَّ اللهَ مَزَجَ الْجَسَدَ مُعْطِياً النَّاقِصَ كَرَامَةً أَفْضَلَ
\par 25 لِكَيْ لاَ يَكُونَ انْشِقَاقٌ فِي الْجَسَدِ بَلْ تَهْتَمُّ الأَعْضَاءُ اهْتِمَاماً وَاحِداً بَعْضُهَا لِبَعْضٍ.
\par 26 فَإِنْ كَانَ عُضْوٌ وَاحِدٌ يَتَأَلَّمُ فَجَمِيعُ الأَعْضَاءِ تَتَأَلَّمُ مَعَهُ. وَإِنْ كَانَ عُضْوٌ وَاحِدٌ يُكَرَّمُ فَجَمِيعُ الأَعْضَاءِ تَفْرَحُ مَعَهُ.
\par 27 وَأَمَّا أَنْتُمْ فَجَسَدُ الْمَسِيحِ وَأَعْضَاؤُهُ أَفْرَاداً.
\par 28 فَوَضَعَ اللهُ أُنَاساً فِي الْكَنِيسَةِ: أَوَّلاً رُسُلاً ثَانِياً أَنْبِيَاءَ ثَالِثاً مُعَلِّمِينَ ثُمَّ قُوَّاتٍ وَبَعْدَ ذَلِكَ مَوَاهِبَ شِفَاءٍ أَعْوَاناً تَدَابِيرَ وَأَنْوَاعَ أَلْسِنَةٍ.
\par 29 أَلَعَلَّ الْجَمِيعَ رُسُلٌ؟ أَلَعَلَّ الْجَمِيعَ أَنْبِيَاءُ؟ أَلَعَلَّ الْجَمِيعَ مُعَلِّمُونَ؟ أَلَعَلَّ الْجَمِيعَ أَصْحَابُ قُوَّاتٍ؟
\par 30 أَلَعَلَّ لِلْجَمِيعِ مَوَاهِبَ شِفَاءٍ؟ أَلَعَلَّ الْجَمِيعَ يَتَكَلَّمُونَ بِأَلْسِنَةٍ؟ أَلَعَلَّ الْجَمِيعَ يُتَرْجِمُونَ؟
\par 31 وَلَكِنْ جِدُّوا لِلْمَوَاهِبِ الْحُسْنَى. وَأَيْضاً أُرِيكُمْ طَرِيقاً أَفْضَلَ:

\chapter{13}

\par 1 إِنْ كُنْتُ أَتَكَلَّمُ بِأَلْسِنَةِ النَّاسِ وَالْمَلاَئِكَةِ وَلَكِنْ لَيْسَ لِي مَحَبَّةٌ فَقَدْ صِرْتُ نُحَاساً يَطِنُّ أَوْ صَنْجاً يَرِنُّ.
\par 2 وَإِنْ كَانَتْ لِي نُبُوَّةٌ وَأَعْلَمُ جَمِيعَ الأَسْرَارِ وَكُلَّ عِلْمٍ وَإِنْ كَانَ لِي كُلُّ الإِيمَانِ حَتَّى أَنْقُلَ الْجِبَالَ وَلَكِنْ لَيْسَ لِي مَحَبَّةٌ فَلَسْتُ شَيْئاً.
\par 3 وَإِنْ أَطْعَمْتُ كُلَّ أَمْوَالِي وَإِنْ سَلَّمْتُ جَسَدِي حَتَّى أَحْتَرِقَ وَلَكِنْ لَيْسَ لِي مَحَبَّةٌ فَلاَ أَنْتَفِعُ شَيْئاً.
\par 4 الْمَحَبَّةُ تَتَأَنَّى وَتَرْفُقُ. الْمَحَبَّةُ لاَ تَحْسِدُ. الْمَحَبَّةُ لاَ تَتَفَاخَرُ وَلاَ تَنْتَفِخُ
\par 5 وَلاَ تُقَبِّحُ وَلاَ تَطْلُبُ مَا لِنَفْسِهَا وَلاَ تَحْتَدُّ وَلاَ تَظُنُّ السُّؤَ
\par 6 وَلاَ تَفْرَحُ بِالإِثْمِ بَلْ تَفْرَحُ بِالْحَقِّ.
\par 7 وَتَحْتَمِلُ كُلَّ شَيْءٍ وَتُصَدِّقُ كُلَّ شَيْءٍ وَتَرْجُو كُلَّ شَيْءٍ وَتَصْبِرُ عَلَى كُلِّ شَيْءٍ.
\par 8 اَلْمَحَبَّةُ لاَ تَسْقُطُ أَبَداً. وَأَمَّا النُّبُوَّاتُ فَسَتُبْطَلُ وَالأَلْسِنَةُ فَسَتَنْتَهِي وَالْعِلْمُ فَسَيُبْطَلُ.
\par 9 لأَنَّنَا نَعْلَمُ بَعْضَ الْعِلْمِ وَنَتَنَبَّأُ بَعْضَ التَّنَبُّؤِ.
\par 10 وَلَكِنْ مَتَى جَاءَ الْكَامِلُ فَحِينَئِذٍ يُبْطَلُ مَا هُوَ بَعْضٌ.
\par 11 لَمَّا كُنْتُ طِفْلاً كَطِفْلٍ كُنْتُ أَتَكَلَّمُ وَكَطِفْلٍ كُنْتُ أَفْطَنُ وَكَطِفْلٍ كُنْتُ أَفْتَكِرُ. وَلَكِنْ لَمَّا صِرْتُ رَجُلاً أَبْطَلْتُ مَا لِلطِّفْلِ.
\par 12 فَإِنَّنَا نَنْظُرُ الآنَ فِي مِرْآةٍ فِي لُغْزٍ لَكِنْ حِينَئِذٍ وَجْهاً لِوَجْهٍ. الآنَ أَعْرِفُ بَعْضَ الْمَعْرِفَةِ لَكِنْ حِينَئِذٍ سَأَعْرِفُ كَمَا عُرِفْتُ.
\par 13 أَمَّا الآنَ فَيَثْبُتُ الإِيمَانُ وَالرَّجَاءُ وَالْمَحَبَّةُ هَذِهِ الثَّلاَثَةُ وَلَكِنَّ أَعْظَمَهُنَّ الْمَحَبَّةُ.

\chapter{14}

\par 1 اِتْبَعُوا الْمَحَبَّةَ وَلَكِنْ جِدُّوا لِلْمَوَاهِبِ الرُّوحِيَّةِ وَبِالأَوْلَى أَنْ تَتَنَبَّأُوا.
\par 2 لأَنَّ مَنْ يَتَكَلَّمُ بِلِسَانٍ لاَ يُكَلِّمُ النَّاسَ بَلِ اللهَ لأَنْ لَيْسَ أَحَدٌ يَسْمَعُ. وَلَكِنَّهُ بِالرُّوحِ يَتَكَلَّمُ بِأَسْرَارٍ.
\par 3 وَأَمَّا مَنْ يَتَنَبَّأُ فَيُكَلِّمُ النَّاسَ بِبُنْيَانٍ وَوَعْظٍ وَتَسْلِيَةٍ.
\par 4 مَنْ يَتَكَلَّمُ بِلِسَانٍ يَبْنِي نَفْسَهُ وَأَمَّا مَنْ يَتَنَبَّأُ فَيَبْنِي الْكَنِيسَةَ.
\par 5 إِنِّي أُرِيدُ أَنَّ جَمِيعَكُمْ تَتَكَلَّمُونَ بِأَلْسِنَةٍ وَلَكِنْ بِالأَوْلَى أَنْ تَتَنَبَّأُوا. لأَنَّ مَنْ يَتَنَبَّأُ أَعْظَمُ مِمَّنْ يَتَكَلَّمُ بِأَلْسِنَةٍ إِلاَّ إِذَا تَرْجَمَ حَتَّى تَنَالَ الْكَنِيسَةُ بُنْيَاناً.
\par 6 فَالآنَ أَيُّهَا الإِخْوَةُ إِنْ جِئْتُ إِلَيْكُمْ مُتَكَلِّماً بِأَلْسِنَةٍ فَمَاذَا أَنْفَعُكُمْ إِنْ لَمْ أُكَلِّمْكُمْ إِمَّا بِإِعْلاَنٍ أَوْ بِعِلْمٍ أَوْ بِنُبُوَّةٍ أَوْ بِتَعْلِيمٍ؟
\par 7 اَلأَشْيَاءُ الْعَادِمَةُ النُّفُوسِ الَّتِي تُعْطِي صَوْتاً: مِزْمَارٌ أَوْ قِيثَارَةٌ مَعَ ذَلِكَ إِنْ لَمْ تُعْطِ فَرْقاً لِلنَّغَمَاتِ فَكَيْفَ يُعْرَفُ مَا زُمِّرَ أَوْ مَا عُزِفَ بِهِ؟
\par 8 فَإِنَّهُ إِنْ أَعْطَى الْبُوقُ أَيْضاً صَوْتاً غَيْرَ وَاضِحٍ فَمَنْ يَتَهَيَّأُ لِلْقِتَالِ؟
\par 9 هَكَذَا أَنْتُمْ أَيْضاً إِنْ لَمْ تُعْطُوا بِاللِّسَانِ كَلاَماً يُفْهَمُ فَكَيْفَ يُعْرَفُ مَا تُكُلِّمَ بِهِ؟ فَإِنَّكُمْ تَكُونُونَ تَتَكَلَّمُونَ فِي الْهَوَاءِ!
\par 10 رُبَّمَا تَكُونُ أَنْوَاعُ لُغَاتٍ هَذَا عَدَدُهَا فِي الْعَالَمِ وَلَيْسَ شَيْءٌ مِنْهَا بِلاَ مَعْنىً.
\par 11 فَإِنْ كُنْتُ لاَ أَعْرِفُ قُوَّةَ اللُّغَةِ أَكُونُ عِنْدَ الْمُتَكَلِّمِ أَعْجَمِيّاً وَالْمُتَكَلِّمُ أَعْجَمِيّاً عِنْدِي.
\par 12 هَكَذَا أَنْتُمْ أَيْضاً إِذْ إِنَّكُمْ غَيُورُونَ لِلْمَوَاهِبِ الرُّوحِيَّةِ اطْلُبُوا لأَجْلِ بُنْيَانِ الْكَنِيسَةِ أَنْ تَزْدَادُوا.
\par 13 لِذَلِكَ مَنْ يَتَكَلَّمُ بِلِسَانٍ فَلْيُصَلِّ لِكَيْ يُتَرْجِمَ.
\par 14 لأَنَّهُ إِنْ كُنْتُ أُصَلِّي بِلِسَانٍ فَرُوحِي تُصَلِّي وَأَمَّا ذِهْنِي فَهُوَ بِلاَ ثَمَرٍ.
\par 15 فَمَا هُوَ إِذاً؟ أُصَلِّي بِالرُّوحِ وَأُصَلِّي بِالذِّهْنِ أَيْضاً. أُرَتِّلُ بِالرُّوحِ وَأُرَتِّلُ بِالذِّهْنِ أَيْضاً.
\par 16 وَإِلاَّ فَإِنْ بَارَكْتَ بِالرُّوحِ فَالَّذِي يُشْغِلُ مَكَانَ الْعَامِّيِّ كَيْفَ يَقُولُ «آمِينَ» عِنْدَ شُكْرِكَ؟ لأَنَّهُ لاَ يَعْرِفُ مَاذَا تَقُولُ!
\par 17 فَإِنَّكَ أَنْتَ تَشْكُرُ حَسَناً! وَلَكِنَّ الآخَرَ لاَ يُبْنَى.
\par 18 أَشْكُرُ إِلَهِي أَنِّي أَتَكَلَّمُ بِأَلْسِنَةٍ أَكْثَرَ مِنْ جَمِيعِكُمْ.
\par 19 وَلَكِنْ فِي كَنِيسَةٍ أُرِيدُ أَنْ أَتَكَلَّمَ خَمْسَ كَلِمَاتٍ بِذِهْنِي لِكَيْ أُعَلِّمَ آخَرِينَ أَيْضاً أَكْثَرَ مِنْ عَشْرَةِ آلاَفِ كَلِمَةٍ بِلِسَانٍ.
\par 20 أَيُّهَا الإِخْوَةُ لاَ تَكُونُوا أَوْلاَداً فِي أَذْهَانِكُمْ بَلْ كُونُوا أَوْلاَداً فِي الشَّرِّ وَأَمَّا فِي الأَذْهَانِ فَكُونُوا كَامِلِينَ.
\par 21 مَكْتُوبٌ فِي النَّامُوسِ: «إِنِّي بِذَوِي أَلْسِنَةٍ أُخْرَى وَبِشِفَاهٍ أُخْرَى سَأُكَلِّمُ هَذَا الشَّعْبَ وَلاَ هَكَذَا يَسْمَعُونَ لِي يَقُولُ الرَّبُّ».
\par 22 إِذاً الأَلْسِنَةُ آيَةٌ لاَ لِلْمُؤْمِنِينَ بَلْ لِغَيْرِ الْمُؤْمِنِينَ. أَمَّا النُّبُوَّةُ فَلَيْسَتْ لِغَيْرِ الْمُؤْمِنِينَ بَلْ لِلْمُؤْمِنِينَ.
\par 23 فَإِنِ اجْتَمَعَتِ الْكَنِيسَةُ كُلُّهَا فِي مَكَانٍ وَاحِدٍ وَكَانَ الْجَمِيعُ يَتَكَلَّمُونَ بِأَلْسِنَةٍ فَدَخَلَ عَامِّيُّونَ أَوْ غَيْرُ مُؤْمِنِينَ أَفَلاَ يَقُولُونَ إِنَّكُمْ تَهْذُونَ؟
\par 24 وَلَكِنْ إِنْ كَانَ الْجَمِيعُ يَتَنَبَّأُونَ فَدَخَلَ أَحَدٌ غَيْرُ مُؤْمِنٍ أَوْ عَامِّيٌّ فَإِنَّهُ يُوَبَّخُ مِنَ الْجَمِيعِ. يُحْكَمُ عَلَيْهِ مِنَ الْجَمِيعِ.
\par 25 وَهَكَذَا تَصِيرُ خَفَايَا قَلْبِهِ ظَاهِرَةً. وَهَكَذَا يَخِرُّ عَلَى وَجْهِهِ وَيَسْجُدُ لِلَّهِ مُنَادِياً أَنَّ اللهَ بِالْحَقِيقَةِ فِيكُمْ.
\par 26 فَمَا هُوَ إِذاً أَيُّهَا الإِخْوَةُ؟ مَتَى اجْتَمَعْتُمْ فَكُلُّ وَاحِدٍ مِنْكُمْ لَهُ مَزْمُورٌ لَهُ تَعْلِيمٌ لَهُ لِسَانٌ لَهُ إِعْلاَنٌ لَهُ تَرْجَمَةٌ: فَلْيَكُنْ كُلُّ شَيْءٍ لِلْبُنْيَانِ.
\par 27 إِنْ كَانَ أَحَدٌ يَتَكَلَّمُ بِلِسَانٍ فَاثْنَيْنِ اثْنَيْنِ أَوْ عَلَى الأَكْثَرِ ثَلاَثَةً ثَلاَثَةً وَبِتَرْتِيبٍ وَلْيُتَرْجِمْ وَاحِدٌ.
\par 28 وَلَكِنْ إِنْ لَمْ يَكُنْ مُتَرْجِمٌ فَلْيَصْمُتْ فِي الْكَنِيسَةِ وَلْيُكَلِّمْ نَفْسَهُ وَاللهَ.
\par 29 أَمَّا الأَنْبِيَاءُ فَلْيَتَكَلَّمِ اثْنَانِ أَوْ ثَلاَثَةٌ وَلْيَحْكُمِ الآخَرُونَ.
\par 30 وَلَكِنْ إِنْ أُعْلِنَ لِآخَرَ جَالِسٍ فَلْيَسْكُتِ الأَوَّلُ.
\par 31 لأَنَّكُمْ تَقْدِرُونَ جَمِيعُكُمْ أَنْ تَتَنَبَّأُوا وَاحِداً وَاحِداً لِيَتَعَلَّمَ الْجَمِيعُ وَيَتَعَزَّى الْجَمِيعُ.
\par 32 وَأَرْوَاحُ الأَنْبِيَاءِ خَاضِعَةٌ لِلأَنْبِيَاءِ.
\par 33 لأَنَّ اللهَ لَيْسَ إِلَهَ تَشْوِيشٍ بَلْ إِلَهُ سَلاَمٍ كَمَا فِي جَمِيعِ كَنَائِسِ الْقِدِّيسِينَ.
\par 34 لِتَصْمُتْ نِسَاؤُكُمْ فِي الْكَنَائِسِ لأَنَّهُ لَيْسَ مَأْذُوناً لَهُنَّ أَنْ يَتَكَلَّمْنَ بَلْ يَخْضَعْنَ كَمَا يَقُولُ النَّامُوسُ أَيْضاً.
\par 35 وَلَكِنْ إِنْ كُنَّ يُرِدْنَ أَنْ يَتَعَلَّمْنَ شَيْئاً فَلْيَسْأَلْنَ رِجَالَهُنَّ فِي الْبَيْتِ لأَنَّهُ قَبِيحٌ بِالنِّسَاءِ أَنْ تَتَكَلَّمَ فِي كَنِيسَةٍ.
\par 36 أَمْ مِنْكُمْ خَرَجَتْ كَلِمَةُ اللهِ؟ أَمْ إِلَيْكُمْ وَحْدَكُمُ انْتَهَتْ؟
\par 37 إِنْ كَانَ أَحَدٌ يَحْسِبُ نَفْسَهُ نَبِيّاً أَوْ رُوحِيّاً فَلْيَعْلَمْ مَا أَكْتُبُهُ إِلَيْكُمْ أَنَّهُ وَصَايَا الرَّبِّ.
\par 38 وَلَكِنْ إِنْ يَجْهَلْ أَحَدٌ فَلْيَجْهَلْ!
\par 39 إِذاً أَيُّهَا الإِخْوَةُ جِدُّوا لِلتَّنَبُّؤِ وَلاَ تَمْنَعُوا التَّكَلُّمَ بِأَلْسِنَةٍ.
\par 40 وَلْيَكُنْ كُلُّ شَيْءٍ بِلِيَاقَةٍ وَبِحَسَبِ تَرْتِيبٍ.

\chapter{15}

\par 1 وَأُعَرِّفُكُمْ أَيُّهَا الإِخْوَةُ بِالإِنْجِيلِ الَّذِي بَشَّرْتُكُمْ بِهِ وَقَبِلْتُمُوهُ وَتَقُومُونَ فِيهِ
\par 2 وَبِهِ أَيْضاً تَخْلُصُونَ إِنْ كُنْتُمْ تَذْكُرُونَ أَيُّ كَلاَمٍ بَشَّرْتُكُمْ بِهِ. إِلاَّ إِذَا كُنْتُمْ قَدْ آمَنْتُمْ عَبَثاً!
\par 3 فَإِنَّنِي سَلَّمْتُ إِلَيْكُمْ فِي الأَوَّلِ مَا قَبِلْتُهُ أَنَا أَيْضاً: أَنَّ الْمَسِيحَ مَاتَ مِنْ أَجْلِ خَطَايَانَا حَسَبَ الْكُتُبِ
\par 4 وَأَنَّهُ دُفِنَ وَأَنَّهُ قَامَ فِي الْيَوْمِ الثَّالِثِ حَسَبَ الْكُتُبِ
\par 5 وَأَنَّهُ ظَهَرَ لِصَفَا ثُمَّ لِلِاثْنَيْ عَشَرَ.
\par 6 وَبَعْدَ ذَلِكَ ظَهَرَ دَفْعَةً وَاحِدَةً لأَكْثَرَ مِنْ خَمْسِمِئَةِ أَخٍ أَكْثَرُهُمْ بَاقٍ إِلَى الآنَ. وَلَكِنَّ بَعْضَهُمْ قَدْ رَقَدُوا.
\par 7 وَبَعْدَ ذَلِكَ ظَهَرَ لِيَعْقُوبَ ثُمَّ لِلرُّسُلِ أَجْمَعِينَ.
\par 8 وَآخِرَ الْكُلِّ كَأَنَّهُ لِلسِّقْطِ ظَهَرَ لِي أَنَا.
\par 9 لأَنِّي أَصْغَرُ الرُّسُلِ أَنَا الَّذِي لَسْتُ أَهْلاً لأَنْ أُدْعَى رَسُولاً لأَنِّي اضْطَهَدْتُ كَنِيسَةَ اللهِ.
\par 10 وَلَكِنْ بِنِعْمَةِ اللهِ أَنَا مَا أَنَا وَنِعْمَتُهُ الْمُعْطَاةُ لِي لَمْ تَكُنْ بَاطِلَةً بَلْ أَنَا تَعِبْتُ أَكْثَرَ مِنْهُمْ جَمِيعِهِمْ. وَلَكِنْ لاَ أَنَا بَلْ نِعْمَةُ اللهِ الَّتِي مَعِي.
\par 11 فَسَوَاءٌ أَنَا أَمْ أُولَئِكَ هَكَذَا نَكْرِزُ وَهَكَذَا آمَنْتُمْ.
\par 12 وَلَكِنْ إِنْ كَانَ الْمَسِيحُ يُكْرَزُ بِهِ أَنَّهُ قَامَ مِنَ الأَمْوَاتِ فَكَيْفَ يَقُولُ قَوْمٌ بَيْنَكُمْ إِنْ لَيْسَ قِيَامَةُ أَمْوَاتٍ؟
\par 13 فَإِنْ لَمْ تَكُنْ قِيَامَةُ أَمْوَاتٍ فَلاَ يَكُونُ الْمَسِيحُ قَدْ قَامَ!
\par 14 وَإِنْ لَمْ يَكُنِ الْمَسِيحُ قَدْ قَامَ فَبَاطِلَةٌ كِرَازَتُنَا وَبَاطِلٌ أَيْضاً إِيمَانُكُمْ
\par 15 وَنُوجَدُ نَحْنُ أَيْضاً شُهُودَ زُورٍ لِلَّهِ لأَنَّنَا شَهِدْنَا مِنْ جِهَةِ اللهِ أَنَّهُ أَقَامَ الْمَسِيحَ وَهُوَ لَمْ يُقِمْهُ - إِنْ كَانَ الْمَوْتَى لاَ يَقُومُونَ.
\par 16 لأَنَّهُ إِنْ كَانَ الْمَوْتَى لاَ يَقُومُونَ فَلاَ يَكُونُ الْمَسِيحُ قَدْ قَامَ.
\par 17 وَإِنْ لَمْ يَكُنِ الْمَسِيحُ قَدْ قَامَ فَبَاطِلٌ إِيمَانُكُمْ. أَنْتُمْ بَعْدُ فِي خَطَايَاكُمْ!
\par 18 إِذاً الَّذِينَ رَقَدُوا فِي الْمَسِيحِ أَيْضاً هَلَكُوا!
\par 19 إِنْ كَانَ لَنَا فِي هَذِهِ الْحَيَاةِ فَقَطْ رَجَاءٌ فِي الْمَسِيحِ فَإِنَّنَا أَشْقَى جَمِيعِ النَّاسِ.
\par 20 وَلَكِنِ الآنَ قَدْ قَامَ الْمَسِيحُ مِنَ الأَمْوَاتِ وَصَارَ بَاكُورَةَ الرَّاقِدِينَ.
\par 21 فَإِنَّهُ إِذِ الْمَوْتُ بِإِنْسَانٍ بِإِنْسَانٍ أَيْضاً قِيَامَةُ الأَمْوَاتِ.
\par 22 لأَنَّهُ كَمَا فِي آدَمَ يَمُوتُ الْجَمِيعُ هَكَذَا فِي الْمَسِيحِ سَيُحْيَا الْجَمِيعُ.
\par 23 وَلَكِنَّ كُلَّ وَاحِدٍ فِي رُتْبَتِهِ. الْمَسِيحُ بَاكُورَةٌ ثُمَّ الَّذِينَ لِلْمَسِيحِ فِي مَجِيئِهِ.
\par 24 وَبَعْدَ ذَلِكَ النِّهَايَةُ مَتَى سَلَّمَ الْمُلْكَ لِلَّهِ الآبِ مَتَى أَبْطَلَ كُلَّ رِيَاسَةٍ وَكُلَّ سُلْطَانٍ وَكُلَّ قُوَّةٍ.
\par 25 لأَنَّهُ يَجِبُ أَنْ يَمْلِكَ حَتَّى يَضَعَ جَمِيعَ الأَعْدَاءِ تَحْتَ قَدَمَيْهِ.
\par 26 آخِرُ عَدُوٍّ يُبْطَلُ هُوَ الْمَوْتُ.
\par 27 لأَنَّهُ أَخْضَعَ كُلَّ شَيْءٍ تَحْتَ قَدَمَيْهِ. وَلَكِنْ حِينَمَا يَقُولُ «إِنَّ كُلَّ شَيْءٍ قَدْ أُخْضِعَ» فَوَاضِحٌ أَنَّهُ غَيْرُ الَّذِي أَخْضَعَ لَهُ الْكُلَّ.
\par 28 وَمَتَى أُخْضِعَ لَهُ الْكُلُّ فَحِينَئِذٍ الِابْنُ نَفْسُهُ أَيْضاً سَيَخْضَعُ لِلَّذِي أَخْضَعَ لَهُ الْكُلَّ كَيْ يَكُونَ اللهُ الْكُلَّ فِي الْكُلِّ.
\par 29 وَإِلاَّ فَمَاذَا يَصْنَعُ الَّذِينَ يَعْتَمِدُونَ مِنْ أَجْلِ الأَمْوَاتِ؟ إِنْ كَانَ الأَمْوَاتُ لاَ يَقُومُونَ الْبَتَّةَ فَلِمَاذَا يَعْتَمِدُونَ مِنْ أَجْلِ الأَمْوَاتِ؟
\par 30 وَلِمَاذَا نُخَاطِرُ نَحْنُ كُلَّ سَاعَةٍ؟
\par 31 إِنِّي بِافْتِخَارِكُمُ الَّذِي لِي فِي يَسُوعَ الْمَسِيحِ رَبِّنَا أَمُوتُ كُلَّ يَوْمٍ.
\par 32 إِنْ كُنْتُ كَإِنْسَانٍ قَدْ حَارَبْتُ وُحُوشاً فِي أَفَسُسَ فَمَا الْمَنْفَعَةُ لِي؟ إِنْ كَانَ الأَمْوَاتُ لاَ يَقُومُونَ فَلْنَأْكُلْ وَنَشْرَبْ لأَنَّنَا غَداً نَمُوتُ!
\par 33 لاَ تَضِلُّوا! فَإِنَّ الْمُعَاشَرَاتِ الرَّدِيَّةَ تُفْسِدُ الأَخْلاَقَ الْجَيِّدَةَ.
\par 34 اُصْحُوا لِلْبِرِّ وَلاَ تُخْطِئُوا لأَنَّ قَوْماً لَيْسَتْ لَهُمْ مَعْرِفَةٌ بِاللَّهِ. أَقُولُ ذَلِكَ لِتَخْجِيلِكُمْ!
\par 35 لَكِنْ يَقُولُ قَائِلٌ: «كَيْفَ يُقَامُ الأَمْوَاتُ وَبِأَيِّ جِسْمٍ يَأْتُونَ؟»
\par 36 يَا غَبِيُّ! الَّذِي تَزْرَعُهُ لاَ يُحْيَا إِنْ لَمْ يَمُتْ.
\par 37 وَالَّذِي تَزْرَعُهُ لَسْتَ تَزْرَعُ الْجِسْمَ الَّذِي سَوْفَ يَصِيرُ بَلْ حَبَّةً مُجَرَّدَةً رُبَّمَا مِنْ حِنْطَةٍ أَوْ أَحَدِ الْبَوَاقِي.
\par 38 وَلَكِنَّ اللهَ يُعْطِيهَا جِسْماً كَمَا أَرَادَ. وَلِكُلِّ وَاحِدٍ مِنَ الْبُزُورِ جِسْمَهُ.
\par 39 لَيْسَ كُلُّ جَسَدٍ جَسَداً وَاحِداً بَلْ لِلنَّاسِ جَسَدٌ وَاحِدٌ وَلِلْبَهَائِمِ جَسَدٌ آخَرُ وَلِلسَّمَكِ آخَرُ وَلِلطَّيْرِ آخَرُ.
\par 40 وَأَجْسَامٌ سَمَاوِيَّةٌ وَأَجْسَامٌ أَرْضِيَّةٌ. لَكِنَّ مَجْدَ السَّمَاوِيَّاتِ شَيْءٌ وَمَجْدَ الأَرْضِيَّاتِ آخَرُ.
\par 41 مَجْدُ الشَّمْسِ شَيْءٌ وَمَجْدُ الْقَمَرِ آخَرُ وَمَجْدُ النُّجُومِ آخَرُ. لأَنَّ نَجْماً يَمْتَازُ عَنْ نَجْمٍ فِي الْمَجْدِ.
\par 42 هَكَذَا أَيْضاً قِيَامَةُ الأَمْوَاتِ: يُزْرَعُ فِي فَسَادٍ وَيُقَامُ فِي عَدَمِ فَسَادٍ.
\par 43 يُزْرَعُ فِي هَوَانٍ وَيُقَامُ فِي مَجْدٍ. يُزْرَعُ فِي ضُعْفٍ وَيُقَامُ فِي قُوَّةٍ.
\par 44 يُزْرَعُ جِسْماً حَيَوَانِيّاً وَيُقَامُ جِسْماً رُوحَانِيّاً. يُوجَدُ جِسْمٌ حَيَوَانِيٌّ وَيُوجَدُ جِسْمٌ رُوحَانِيٌّ.
\par 45 هَكَذَا مَكْتُوبٌ أَيْضاً: «صَارَ آدَمُ الإِنْسَانُ الأَوَّلُ نَفْساً حَيَّةً وَآدَمُ الأَخِيرُ رُوحاً مُحْيِياً».
\par 46 لَكِنْ لَيْسَ الرُّوحَانِيُّ أَوَّلاً بَلِ الْحَيَوَانِيُّ وَبَعْدَ ذَلِكَ الرُّوحَانِيُّ.
\par 47 الإِنْسَانُ الأَوَّلُ مِنَ الأَرْضِ تُرَابِيٌّ. الإِنْسَانُ الثَّانِي الرَّبُّ مِنَ السَّمَاءِ.
\par 48 كَمَا هُوَ التُّرَابِيُّ هَكَذَا التُّرَابِيُّونَ أَيْضاً وَكَمَا هُوَ السَّمَاوِيُّ هَكَذَا السَّمَاوِيُّونَ أَيْضاً.
\par 49 وَكَمَا لَبِسْنَا صُورَةَ التُّرَابِيِّ سَنَلْبَسُ أَيْضاً صُورَةَ السَّمَاوِيِّ.
\par 50 فَأَقُولُ هَذَا أَيُّهَا الإِخْوَةُ: إِنَّ لَحْماً وَدَماً لاَ يَقْدِرَانِ أَنْ يَرِثَا مَلَكُوتَ اللهِ وَلاَ يَرِثُ الْفَسَادُ عَدَمَ الْفَسَادِ.
\par 51 هُوَذَا سِرٌّ أَقُولُهُ لَكُمْ: لاَ نَرْقُدُ كُلُّنَا وَلَكِنَّنَا كُلَّنَا نَتَغَيَّرُ
\par 52 فِي لَحْظَةٍ فِي طَرْفَةِ عَيْنٍ عِنْدَ الْبُوقِ الأَخِيرِ. فَإِنَّهُ سَيُبَوَّقُ فَيُقَامُ الأَمْوَاتُ عَدِيمِي فَسَادٍ وَنَحْنُ نَتَغَيَّرُ.
\par 53 لأَنَّ هَذَا الْفَاسِدَ لاَ بُدَّ أَنْ يَلْبَسَ عَدَمَ فَسَادٍ وَهَذَا الْمَائِتَ يَلْبَسُ عَدَمَ مَوْتٍ.
\par 54 وَمَتَى لَبِسَ هَذَا الْفَاسِدُ عَدَمَ فَسَادٍ وَلَبِسَ هَذَا الْمَائِتُ عَدَمَ مَوْتٍ فَحِينَئِذٍ تَصِيرُ الْكَلِمَةُ الْمَكْتُوبَةُ: «ابْتُلِعَ الْمَوْتُ إِلَى غَلَبَةٍ».
\par 55 أَيْنَ شَوْكَتُكَ يَا مَوْتُ؟ أَيْنَ غَلَبَتُكِ يَا هَاوِيَةُ؟
\par 56 أَمَّا شَوْكَةُ الْمَوْتِ فَهِيَ الْخَطِيَّةُ وَقُوَّةُ الْخَطِيَّةِ هِيَ النَّامُوسُ.
\par 57 وَلَكِنْ شُكْراً لِلَّهِ الَّذِي يُعْطِينَا الْغَلَبَةَ بِرَبِّنَا يَسُوعَ الْمَسِيحِ.
\par 58 إِذاً يَا إِخْوَتِي الأَحِبَّاءَ كُونُوا رَاسِخِينَ غَيْرَ مُتَزَعْزِعِينَ مُكْثِرِينَ فِي عَمَلِ الرَّبِّ كُلَّ حِينٍ عَالِمِينَ أَنَّ تَعَبَكُمْ لَيْسَ بَاطِلاً فِي الرَّبِّ.

\chapter{16}

\par 1 وَأَمَّا مِنْ جِهَةِ الْجَمْعِ لأَجْلِ الْقِدِّيسِينَ فَكَمَا أَوْصَيْتُ كَنَائِسَ غَلاَطِيَّةَ هَكَذَا افْعَلُوا أَنْتُمْ أَيْضاً.
\par 2 فِي كُلِّ أَوَّلِ أُسْبُوعٍ لِيَضَعْ كُلُّ وَاحِدٍ مِنْكُمْ عِنْدَهُ خَازِناً مَا تَيَسَّرَ حَتَّى إِذَا جِئْتُ لاَ يَكُونُ جَمْعٌ حِينَئِذٍ.
\par 3 وَمَتَى حَضَرْتُ فَالَّذِينَ تَسْتَحْسِنُونَهُمْ أُرْسِلُهُمْ بِرَسَائِلَ لِيَحْمِلُوا إِحْسَانَكُمْ إِلَى أُورُشَلِيمَ.
\par 4 وَإِنْ كَانَ يَسْتَحِقُّ أَنْ أَذْهَبَ أَنَا أَيْضاً فَسَيَذْهَبُونَ مَعِي.
\par 5 وَسَأَجِيءُ إِلَيْكُمْ مَتَى اجْتَزْتُ بِمَكِدُونِيَّةَ لأَنِّي أَجْتَازُ بِمَكِدُونِيَّةَ.
\par 6 وَرُبَّمَا أَمْكُثُ عِنْدَكُمْ أَوْ أُشَتِّي أَيْضاً لِكَيْ تُشَيِّعُونِي إِلَى حَيْثُمَا أَذْهَبُ.
\par 7 لأَنِّي لَسْتُ أُرِيدُ الآنَ أَنْ أَرَاكُمْ فِي الْعُبُورِ لأَنِّي أَرْجُو أَنْ أَمْكُثَ عِنْدَكُمْ زَمَاناً إِنْ أَذِنَ الرَّبُّ.
\par 8 وَلَكِنَّنِي أَمْكُثُ فِي أَفَسُسَ إِلَى يَوْمِ الْخَمْسِينَ
\par 9 لأَنَّهُ قَدِ انْفَتَحَ لِي بَابٌ عَظِيمٌ فَعَّالٌ وَيُوجَدُ مُعَانِدُونَ كَثِيرُونَ.
\par 10 ثُمَّ إِنْ أَتَى تِيمُوثَاوُسُ فَانْظُرُوا أَنْ يَكُونَ عِنْدَكُمْ بِلاَ خَوْفٍ. لأَنَّهُ يَعْمَلُ عَمَلَ الرَّبِّ كَمَا أَنَا أَيْضاً.
\par 11 فَلاَ يَحْتَقِرْهُ أَحَدٌ بَلْ شَيِّعُوهُ بِسَلاَمٍ لِيَأْتِيَ إِلَيَّ لأَنِّي أَنْتَظِرُهُ مَعَ الإِخْوَةِ.
\par 12 وَأَمَّا مِنْ جِهَةِ أَبُلُّوسَ الأَخِ فَطَلَبْتُ إِلَيْهِ كَثِيراً أَنْ يَأْتِيَ إِلَيْكُمْ مَعَ الإِخْوَةِ وَلَمْ تَكُنْ لَهُ إِرَادَةٌ الْبَتَّةَ أَنْ يَأْتِيَ الآنَ. وَلَكِنَّهُ سَيَأْتِي مَتَى تَوَفَّقَ الْوَقْتُ.
\par 13 اِسْهَرُوا. اثْبُتُوا فِي الإِيمَانِ. كُونُوا رِجَالاً. تَقَوُّوا.
\par 14 لِتَصِرْ كُلُّ أُمُورِكُمْ فِي مَحَبَّةٍ.
\par 15 وَأَطْلُبُ إِلَيْكُمْ أَيُّهَا الإِخْوَةُ: أَنْتُمْ تَعْرِفُونَ بَيْتَ اسْتِفَانَاسَ أَنَّهُمْ بَاكُورَةُ أَخَائِيَةَ وَقَدْ رَتَّبُوا أَنْفُسَهُمْ لِخِدْمَةِ الْقِدِّيسِينَ
\par 16 كَيْ تَخْضَعُوا أَنْتُمْ أَيْضاً لِمِثْلِ هَؤُلاَءِ وَكُلِّ مَنْ يَعْمَلُ مَعَهُمْ وَيَتْعَبُ.
\par 17 ثُمَّ إِنِّي أَفْرَحُ بِمَجِيءِ اسْتِفَانَاسَ وَفُرْتُونَاتُوسَ وَأَخَائِيكُوسَ لأَنَّ نُقْصَانَكُمْ هَؤُلاَءِ قَدْ جَبَرُوهُ
\par 18 إِذْ أَرَاحُوا رُوحِي وَرُوحَكُمْ. فَاعْرِفُوا مِثْلَ هَؤُلاَءِ.
\par 19 تُسَلِّمُ عَلَيْكُمْ كَنَائِسُ أَسِيَّا. يُسَلِّمُ عَلَيْكُمْ فِي الرَّبِّ كَثِيراً أَكِيلاَ وَبِرِيسْكِلاَّ مَعَ الْكَنِيسَةِ الَّتِي فِي بَيْتِهِمَا.
\par 20 يُسَلِّمُ عَلَيْكُمُ الإِخْوَةُ أَجْمَعُونَ. سَلِّمُوا بَعْضُكُمْ عَلَى بَعْضٍ بِقُبْلَةٍ مُقَدَّسَةٍ.
\par 21 اَلسَّلاَمُ بِيَدِي أَنَا بُولُسَ.
\par 22 إِنْ كَانَ أَحَدٌ لاَ يُحِبُّ الرَّبَّ يَسُوعَ الْمَسِيحَ فَلْيَكُنْ أَنَاثِيمَا. مَارَانْ أَثَا.
\par 23 نِعْمَةُ الرَّبِّ يَسُوعَ الْمَسِيحِ مَعَكُمْ.
\par 24 مَحَبَّتِي مَعَ جَمِيعِكُمْ فِي الْمَسِيحِ يَسُوعَ. آمِينَ

\end{document}