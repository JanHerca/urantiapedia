\begin{document}

\title{2تسالونيكي}


\chapter{1}

\par 1 بُولُسُ وَسِلْوَانُسُ وَتِيمُوثَاوُسُ، إِلَى كَنِيسَةِ التَّسَالُونِيكِيِّينَ، فِي اللهِ أَبِينَا وَالرَّبِّ يَسُوعَ الْمَسِيحِ.
\par 2 نِعْمَةٌ لَكُمْ وَسَلاَمٌ مِنَ اللهِ أَبِينَا وَالرَّبِّ يَسُوعَ الْمَسِيحِ.
\par 3 يَنْبَغِي لَنَا أَنْ نَشْكُرَ اللهَ كُلَّ حِينٍ مِنْ جِهَتِكُمْ أَيُّهَا الإِخْوَةُ كَمَا يَحِقُّ، لأَنَّ إِيمَانَكُمْ يَنْمُو كَثِيراً، وَمَحَبَّةُ كُلِّ وَاحِدٍ مِنْكُمْ جَمِيعاً بَعْضِكُمْ لِبَعْضٍ تَزْدَادُ،
\par 4 حَتَّى إِنَّنَا نَحْنُ أَنْفُسَنَا نَفْتَخِرُ بِكُمْ فِي كَنَائِسِ اللهِ، مِنْ أَجْلِ صَبْرِكُمْ وَإِيمَانِكُمْ فِي جَمِيعِ اضْطِهَادَاتِكُمْ وَالضِّيقَاتِ الَّتِي تَحْتَمِلُونَهَا،
\par 5 بَيِّنَةً عَلَى قَضَاءِ اللهِ الْعَادِلِ، أَنَّكُمْ تُؤَهَّلُونَ لِمَلَكُوتِ اللهِ الَّذِي لأَجْلِهِ تَتَأَلَّمُونَ أَيْضاً،
\par 6 إِذْ هُوَ عَادِلٌ عِنْدَ اللهِ أَنَّ الَّذِينَ يُضَايِقُونَكُمْ يُجَازِيهِمْ ضِيقاً،
\par 7 وَإِيَّاكُمُ الَّذِينَ تَتَضَايَقُونَ رَاحَةً مَعَنَا عِنْدَ اسْتِعْلاَنِ الرَّبِّ يَسُوعَ مِنَ السَّمَاءِ مَعَ مَلاَئِكَةِ قُوَّتِهِ،
\par 8 فِي نَارِ لَهِيبٍ، مُعْطِياً نَقْمَةً لِلَّذِينَ لاَ يَعْرِفُونَ اللهَ وَالَّذِينَ لاَ يُطِيعُونَ إِنْجِيلَ رَبِّنَا يَسُوعَ الْمَسِيحِ،
\par 9 الَّذِينَ سَيُعَاقَبُونَ بِهَلاَكٍ أَبَدِيٍّ مِنْ وَجْهِ الرَّبِّ وَمِنْ مَجْدِ قُوَّتِهِ،
\par 10 مَتَى جَاءَ لِيَتَمَجَّدَ فِي قِدِّيسِيهِ وَيُتَعَجَّبَ مِنْهُ فِي جَمِيعِ الْمُؤْمِنِينَ. لأَنَّ شَهَادَتَنَا عِنْدَكُمْ صُدِّقَتْ فِي ذَلِكَ الْيَوْمِ،
\par 11 الأَمْرُ الَّذِي لأَجْلِهِ نُصَلِّي أَيْضاً كُلَّ حِينٍ مِنْ جِهَتِكُمْ: أَنْ يُؤَهِّلَكُمْ إِلَهُنَا لِلدَّعْوَةِ، وَيُكَمِّلَ كُلَّ مَسَرَّةِ الصَّلاَحِ وَعَمَلَ الإِيمَانِ بِقُوَّةٍ،
\par 12 لِكَيْ يَتَمَجَّدَ اسْمُ رَبِّنَا يَسُوعَ الْمَسِيحِ فِيكُمْ، وَأَنْتُمْ فِيهِ، بِنِعْمَةِ إِلَهِنَا وَالرَّبِّ يَسُوعَ الْمَسِيحِ.

\chapter{2}

\par 1 ثُمَّ نَسْأَلُكُمْ أَيُّهَا الإِخْوَةُ مِنْ جِهَةِ مَجِيءِ رَبِّنَا يَسُوعَ الْمَسِيحِ وَاجْتِمَاعِنَا إِلَيْهِ،
\par 2 أَنْ لاَ تَتَزَعْزَعُوا سَرِيعاً عَنْ ذِهْنِكُمْ، وَلاَ تَرْتَاعُوا، لاَ بِرُوحٍ وَلاَ بِكَلِمَةٍ وَلاَ بِرِسَالَةٍ كَأَنَّهَا مِنَّا: أَيْ أَنَّ يَوْمَ الْمَسِيحِ قَدْ حَضَرَ.
\par 3 لاَ يَخْدَعَنَّكُمْ أَحَدٌ عَلَى طَرِيقَةٍ مَا، لأَنَّهُ لاَ يَأْتِي إِنْ لَمْ يَأْتِ الاِرْتِدَادُ أَوَّلاً، وَيُسْتَعْلَنَ إِنْسَانُ الْخَطِيَّةِ، ابْنُ الْهَلاَكِ،
\par 4 الْمُقَاوِمُ وَالْمُرْتَفِعُ عَلَى كُلِّ مَا يُدْعَى إِلَهاً أَوْ مَعْبُوداً، حَتَّى إِنَّهُ يَجْلِسُ فِي هَيْكَلِ اللهِ كَإِلَهٍ مُظْهِراً نَفْسَهُ أَنَّهُ إِلَهٌ.
\par 5 أَمَا تَذْكُرُونَ أَنِّي وَأَنَا بَعْدُ عِنْدَكُمْ كُنْتُ أَقُولُ لَكُمْ هَذَا؟
\par 6 وَالآنَ تَعْلَمُونَ مَا يَحْجِزُ حَتَّى يُسْتَعْلَنَ فِي وَقْتِهِ.
\par 7 لأَنَّ سِرَّ الإِثْمِ الآنَ يَعْمَلُ فَقَطْ، إِلَى أَنْ يُرْفَعَ مِنَ الْوَسَطِ الَّذِي يَحْجِزُ الآنَ،
\par 8 وَحِينَئِذٍ سَيُسْتَعْلَنُ الأَثِيمُ، الَّذِي الرَّبُّ يُبِيدُهُ بِنَفْخَةِ فَمِهِ، وَيُبْطِلُهُ بِظُهُورِ مَجِيئِهِ.
\par 9 الَّذِي مَجِيئُهُ بِعَمَلِ الشَّيْطَانِ، بِكُلِّ قُوَّةٍ، وَبِآيَاتٍ وَعَجَائِبَ كَاذِبَةٍ،
\par 10 وَبِكُلِّ خَدِيعَةِ الإِثْمِ، فِي الْهَالِكِينَ، لأَنَّهُمْ لَمْ يَقْبَلُوا مَحَبَّةَ الْحَقِّ حَتَّى يَخْلُصُوا.
\par 11 وَلأَجْلِ هَذَا سَيُرْسِلُ إِلَيْهِمُ اللهُ عَمَلَ الضَّلاَلِ، حَتَّى يُصَدِّقُوا الْكَذِبَ،
\par 12 لِكَيْ يُدَانَ جَمِيعُ الَّذِينَ لَمْ يُصَدِّقُوا الْحَقَّ، بَلْ سُرُّوا بِالإِثْمِ.
\par 13 وَأَمَّا نَحْنُ فَيَنْبَغِي لَنَا أَنْ نَشْكُرَ اللهَ كُلَّ حِينٍ لأَجْلِكُمْ أَيُّهَا الإِخْوَةُ الْمَحْبُوبُونَ مِنَ الرَّبِّ، أَنَّ اللهَ اخْتَارَكُمْ مِنَ الْبَدْءِ لِلْخَلاَصِ، بِتَقْدِيسِ الرُّوحِ وَتَصْدِيقِ الْحَقِّ.
\par 14 الأَمْرُ الَّذِي دَعَاكُمْ إِلَيْهِ بِإِنْجِيلِنَا، لاِقْتِنَاءِ مَجْدِ رَبِّنَا يَسُوعَ الْمَسِيحِ.
\par 15 فَاثْبُتُوا إِذاً أَيُّهَا الإِخْوَةُ وَتَمَسَّكُوا بِالتَّعَالِيمِ الَّتِي تَعَلَّمْتُمُوهَا، سَوَاءٌ كَانَ بِالْكَلاَمِ أَمْ بِرِسَالَتِنَا.
\par 16 وَرَبُّنَا نَفْسُهُ يَسُوعُ الْمَسِيحُ، وَاللهُ أَبُونَا الَّذِي أَحَبَّنَا وَأَعْطَانَا عَزَاءً أَبَدِيّاً وَرَجَاءً صَالِحاً بِالنِّعْمَةِ،
\par 17 يُعَزِّي قُلُوبَكُمْ وَيُثَبِّتُكُمْ فِي كُلِّ كَلاَمٍ وَعَمَلٍ صَالِحٍ.

\chapter{3}

\par 1 أَخِيراً أَيُّهَا الإِخْوَةُ صَلُّوا لأَجْلِنَا، لِكَيْ تَجْرِيَ كَلِمَةُ الرَّبِّ وَتَتَمَجَّدَ، كَمَا عِنْدَكُمْ أَيْضاً،
\par 2 وَلِكَيْ نُنْقَذَ مِنَ النَّاسِ الأَرْدِيَاءِ الأَشْرَارِ. لأَنَّ الإِيمَانَ لَيْسَ لِلْجَمِيعِ.
\par 3 أَمِينٌ هُوَ الرَّبُّ الَّذِي سَيُثَبِّتُكُمْ وَيَحْفَظُكُمْ مِنَ الشِّرِّيرِ.
\par 4 وَنَثِقُ بِالرَّبِّ مِنْ جِهَتِكُمْ أَنَّكُمْ تَفْعَلُونَ مَا نُوصِيكُمْ بِهِ وَسَتَفْعَلُونَ أَيْضاً.
\par 5 وَالرَّبُّ يَهْدِي قُلُوبَكُمْ إِلَى مَحَبَّةِ اللهِ وَإِلَى صَبْرِ الْمَسِيحِ.
\par 6 ثُمَّ نُوصِيكُمْ أَيُّهَا الإِخْوَةُ، بِاسْمِ رَبِّنَا يَسُوعَ الْمَسِيحِ، أَنْ تَتَجَنَّبُوا كُلَّ أَخٍ يَسْلُكُ بِلاَ تَرْتِيبٍ، وَلَيْسَ حَسَبَ التَّعْلِيمِ الَّذِي أَخَذَهُ مِنَّا.
\par 7 إِذْ أَنْتُمْ تَعْرِفُونَ كَيْفَ يَجِبُ أَنْ يُتَمَثَّلَ بِنَا، لأَنَّنَا لَمْ نَسْلُكْ بِلاَ تَرْتِيبٍ بَيْنَكُمْ،
\par 8 وَلاَ أَكَلْنَا خُبْزاً مَجَّاناً مِنْ أَحَدٍ، بَلْ كُنَّا نَشْتَغِلُ بِتَعَبٍ وَكَدٍّ لَيْلاً وَنَهَاراً، لِكَيْ لاَ نُثَقِّلَ عَلَى أَحَدٍ مِنْكُمْ.
\par 9 لَيْسَ أَنْ لاَ سُلْطَانَ لَنَا، بَلْ لِكَيْ نُعْطِيَكُمْ أَنْفُسَنَا قُدْوَةً حَتَّى تَتَمَثَّلُوا بِنَا.
\par 10 فَإِنَّنَا أَيْضاً حِينَ كُنَّا عِنْدَكُمْ أَوْصَيْنَاكُمْ بِهَذَا: أَنَّهُ إِنْ كَانَ أَحَدٌ لاَ يُرِيدُ أَنْ يَشْتَغِلَ فَلاَ يَأْكُلْ أَيْضاً.
\par 11 لأَنَّنَا نَسْمَعُ أَنَّ قَوْماً يَسْلُكُونَ بَيْنَكُمْ بِلاَ تَرْتِيبٍ، لاَ يَشْتَغِلُونَ شَيْئاً بَلْ هُمْ فُضُولِيُّونَ.
\par 12 فَمِثْلُ هَؤُلاَءِ نُوصِيهِمْ وَنَعِظُهُمْ بِرَبِّنَا يَسُوعَ الْمَسِيحِ أَنْ يَشْتَغِلُوا بِهُدُوءٍ، وَيَأْكُلُوا خُبْزَ أَنْفُسِهِمْ.
\par 13 أَمَّا أَنْتُمْ أَيُّهَا الإِخْوَةُ فَلاَ تَفْشَلُوا فِي عَمَلِ الْخَيْرِ.
\par 14 وَإِنْ كَانَ أَحَدٌ لاَ يُطِيعُ كَلاَمَنَا بِالرِّسَالَةِ، فَسِمُوا هَذَا وَلاَ تُخَالِطُوهُ لِكَيْ يَخْجَلَ،
\par 15 وَلَكِنْ لاَ تَحْسِبُوهُ كَعَدُوٍّ، بَلْ أَنْذِرُوهُ كَأَخٍ.
\par 16 وَرَبُّ السَّلاَمِ نَفْسُهُ يُعْطِيكُمُ السَّلاَمَ دَائِماً مِنْ كُلِّ وَجْهٍ. الرَّبُّ مَعَ جَمِيعِكُمْ.
\par 17 اَلسَّلاَمُ بِيَدِي أَنَا بُولُسَ، الَّذِي هُوَ عَلاَمَةٌ فِي كُلِّ رِسَالَةٍ. هَكَذَا أَنَا أَكْتُبُ.
\par 18 نِعْمَةُ رَبِّنَا يَسُوعَ الْمَسِيحِ مَعَ جَمِيعِكُمْ. آمِينَ.

\end{document}