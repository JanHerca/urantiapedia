\begin{document}

\title{مزامير}


\chapter{1}

\par 1 طُوبَى لِلرَّجُلِ الَّذِي لَمْ يَسْلُكْ فِي مَشُورَةِ الأَشْرَارِ وَفِي طَرِيقِ الْخُطَاةِ لَمْ يَقِفْ وَفِي مَجْلِسِ الْمُسْتَهْزِئِينَ لَمْ يَجْلِسْ.
\par 2 لَكِنْ فِي نَامُوسِ الرَّبِّ مَسَرَّتُهُ وَفِي نَامُوسِهِ يَلْهَجُ نَهَاراً وَلَيْلاً.
\par 3 فَيَكُونُ كَشَجَرَةٍ مَغْرُوسَةٍ عِنْدَ مَجَارِيِِ الْمِيَاهِ الَّتِي تُعْطِي ثَمَرَهَا فِي أَوَانِهِ وَوَرَقُهَا لاَ يَذْبُلُ. وَكُلُّ مَا يَصْنَعُهُ يَنْجَحُ.
\par 4 لَيْسَ كَذَلِكَ الأَشْرَارُ لَكِنَّهُمْ كَالْعُصَافَةِ الَّتِي تُذَرِّيهَا الرِّيحُ.
\par 5 لِذَلِكَ لاَ تَقُومُ الأَشْرَارُ فِي الدِّينِ وَلاَ الْخُطَاةُ فِي جَمَاعَةِ الأَبْرَارِ.
\par 6 لأَنَّ الرَّبَّ يَعْلَمُ طَرِيقَ الأَبْرَارِ أَمَّا طَرِيقُ الأَشْرَارِ فَتَهْلِكُ.

\chapter{2}

\par 1 لِمَاذَا ارْتَجَّتِ الأُمَمُ وَتَفَكَّرَ الشُّعُوبُ فِي الْبَاطِلِ؟
\par 2 قَامَ مُلُوكُ الأَرْضِ وَتَآمَرَ الرُّؤَسَاءُ مَعاً عَلَى الرَّبِّ وَعَلَى مَسِيحِهِ قَائِلِينَ:
\par 3 [لِنَقْطَعْ قُيُودَهُمَا وَلْنَطْرَحْ عَنَّا رُبُطَهُمَا].
\par 4 اَلسَّاكِنُ فِي السَّمَاوَاتِ يَضْحَكُ. الرَّبُّ يَسْتَهْزِئُ بِهِمْ.
\par 5 حِينَئِذٍ يَتَكَلَّمُ عَلَيْهِمْ بِغَضَبِهِ وَيَرْجُفُهُمْ بِغَيْظِهِ.
\par 6 أَمَّا أَنَا فَقَدْ مَسَحْتُ مَلِكِي عَلَى صِهْيَوْنَ جَبَلِ قُدْسِي.
\par 7 إِنِّي أُخْبِرُ مِنْ جِهَةِ قَضَاءِ الرَّبِّ. قَالَ لِي: [أَنْتَ ابْنِي. أَنَا الْيَوْمَ وَلَدْتُكَ.
\par 8 اِسْأَلْنِي فَأُعْطِيَكَ الأُمَمَ مِيرَاثاً لَكَ وَأَقَاصِيَ الأَرْضِ مُلْكاً لَكَ.
\par 9 تُحَطِّمُهُمْ بِقَضِيبٍ مِنْ حَدِيدٍ. مِثْلَ إِنَاءِ خَزَّافٍ تُكَسِّرُهُمْ].
\par 10 فَالآنَ يَا أَيُّهَا الْمُلُوكُ تَعَقَّلُوا. تَأَدَّبُوا يَا قُضَاةَ الأَرْضِ.
\par 11 اعْبُدُوا الرَّبَّ بِخَوْفٍ وَاهْتِفُوا بِرَعْدَةٍ.
\par 12 قَبِّلُوا الاِبْنَ لِئَلاَّ يَغْضَبَ فَتَبِيدُوا مِنَ الطَّرِيقِ. لأَنَّهُ عَنْ قَلِيلٍ يَتَّقِدُ غَضَبُهُ. طُوبَى لِجَمِيعِ الْمُتَّكِلِينَ عَلَيْهِ.

\chapter{3}

\par 1 مَزْمُورٌ لِدَاوُدَ حِينَمَا هَرَبَ مِنْ وَجْهِ أَبْشَالُومَ ابْنِهِ يَا رَبُّ مَا أَكْثَرَ مُضَايِقِيَّ. كَثِيرُونَ قَائِمُونَ عَلَيَّ.
\par 2 كَثِيرُونَ يَقُولُونَ لِنَفْسِي: [لَيْسَ لَهُ خَلاَصٌ بِإِلَهِهِ]. سِلاَهْ.
\par 3 أَمَّا أَنْتَ يَا رَبُّ فَتُرْسٌ لِي. مَجْدِي وَرَافِعُ رَأْسِي.
\par 4 بِصَوْتِي إِلَى الرَّبِّ أَصْرُخُ فَيُجِيبُنِي مِنْ جَبَلِ قُدْسِهِ. سِلاَهْ.
\par 5 أَنَا اضْطَجَعْتُ وَنِمْتُ. اسْتَيْقَظْتُ لأَنَّ الرَّبَّ يَعْضُدُنِي.
\par 6 لاَ أَخَافُ مِنْ رَبَوَاتِ الشُّعُوبِ الْمُصْطَفِّينَ عَلَيَّ مِنْ حَوْلِي.
\par 7 قُمْ يَا رَبُّ. خَلِّصْنِي يَا إِلَهِي. لأَنَّكَ ضَرَبْتَ كُلَّ أَعْدَائِي عَلَى الْفَكِّ. هَشَّمْتَ أَسْنَانَ الأَشْرَارِ.
\par 8 لِلرَّبِّ الْخَلاَصُ. عَلَى شَعْبِكَ بَرَكَتُكَ. سِلاَهْ.

\chapter{4}

\par 1 لإِمَامِ الْمُغَنِّينَ عَلَى ذَوَاتِ الأَوْتَارِ. مَزْمُورٌ لِدَاوُدَ عِنْدَ دُعَائِيَ اسْتَجِبْ لِي يَا إِلَهَ بِرِّي. فِي الضِّيقِ رَحَّبْتَ لِي. تَرَاءَفْ عَلَيَّ وَاسْمَعْ صَلاَتِي.
\par 2 يَا بَنِي الْبَشَرِ حَتَّى مَتَى يَكُونُ مَجْدِي عَاراً! حَتَّى مَتَى تُحِبُّونَ الْبَاطِلَ وَتَبْتَغُونَ الْكَذِبَ! سِلاَهْ.
\par 3 فَاعْلَمُوا أَنَّ الرَّبَّ قَدْ مَيَّزَ تَقِيَّهُ. الرَّبُّ يَسْمَعُ عِنْدَ مَا أَدْعُوهُ.
\par 4 اِرْتَعِدُوا وَلاَ تُخْطِئُوا. تَكَلَّمُوا فِي قُلُوبِكُمْ عَلَى مَضَاجِعِكُمْ وَاسْكُتُوا. سِلاَهْ.
\par 5 اِذْبَحُوا ذَبَائِحَ الْبِرِّ وَتَوَكَّلُوا عَلَى الرَّبِّ.
\par 6 كَثِيرُونَ يَقُولُونَ: [مَنْ يُرِينَا خَيْراً؟] ارْفَعْ عَلَيْنَا نُورَ وَجْهِكَ يَا رَبُّ.
\par 7 جَعَلْتَ سُرُوراً فِي قَلْبِي أَعْظَمَ مِنْ سُرُورِهِمْ إِذْ كَثُرَتْ حِنْطَتُهُمْ وَخَمْرُهُمْ.
\par 8 بِسَلاَمَةٍ أَضْطَجِعُ بَلْ أَيْضاً أَنَامُ لأَنَّكَ أَنْتَ يَا رَبُّ مُنْفَرِداً فِي طُمَأْنِينَةٍ تُسَكِّنُنِي.

\chapter{5}

\par 1 لإِمَامِ الْمُغَنِّينَ عَلَى ذَوَاتِ النَّفْخِ. مَزْمُورٌ لِدَاوُدَ لِكَلِمَاتِي أَصْغِ يَا رَبُّ. تَأَمَّلْ صُرَاخِي.
\par 2 اسْتَمِعْ لِصَوْتِ دُعَائِي يَا مَلِكِي وَإِلَهِي لأَنِّي إِلَيْكَ أُصَلِّي.
\par 3 يَا رَبُّ بِالْغَدَاةِ تَسْمَعُ صَوْتِي. بِالْغَدَاةِ أُوَجِّهُ صَلاَتِي نَحْوَكَ وَأَنْتَظِرُ.
\par 4 لأَنَّكَ أَنْتَ لَسْتَ إِلَهاً يُسَرُّ بِالشَّرِّ لاَ يُسَاكِنُكَ الشِّرِّيرُ.
\par 5 لاَ يَقِفُ الْمُفْتَخِرُونَ قُدَّامَ عَيْنَيْكَ. أَبْغَضْتَ كُلَّ فَاعِلِي الإِثْمِ.
\par 6 تُهْلِكُ الْمُتَكَلِّمِينَ بِالْكَذِبِ. رَجُلُ الدِّمَاءِ وَالْغِشِّ يَكْرَهُهُ الرَّبُّ.
\par 7 أَمَّا أَنَا فَبِكَثْرَةِ رَحْمَتِكَ أَدْخُلُ بَيْتَكَ. أَسْجُدُ فِي هَيْكَلِ قُدْسِكَ بِخَوْفِكَ.
\par 8 يَا رَبُّ اهْدِنِي إِلَى بِرِّكَ بِسَبَبِ أَعْدَائِي. سَهِّلْ قُدَّامِي طَرِيقَكَ.
\par 9 لأَنَّهُ لَيْسَ فِي أَفْوَاهِهِمْ صِدْقٌ. جَوْفُهُمْ هُوَّةٌ. حَلْقُهُمْ قَبْرٌ مَفْتُوحٌ. أَلْسِنَتُهُمْ صَقَلُوهَا.
\par 10 دِنْهُمْ يَا اللهُ. لِيَسْقُطُوا مِنْ مُؤَامَرَاتِهِمْ بِكَثْرَةِ ذُنُوبِهِمْ. طَوِّحْ بِهِمْ لأَنَّهُمْ تَمَرَّدُوا عَلَيْكَ.
\par 11 وَيَفْرَحُ جَمِيعُ الْمُتَّكِلِينَ عَلَيْكَ. إِلَى الأَبَدِ يَهْتِفُونَ وَتُظَلِّلُهُمْ. وَيَبْتَهِجُ بِكَ مُحِبُّو اسْمِكَ.
\par 12 لأَنَّكَ أَنْتَ تُبَارِكُ الصِّدِّيقَ يَا رَبُّ. كَأَنَّهُ بِتُرْسٍ تُحِيطُهُ بِالرِّضَا.

\chapter{6}

\par 1 لإِمَامِ الْمُغَنِّينَ عَلَى ذَوَاتِ الأَوْتَارِ عَلَى الْقَرَارِ. مَزْمُورٌ لِدَاوُدَ يَا رَبُّ لاَ تُوَبِّخْنِي بِغَضَبِكَ وَلاَ تُؤَدِّبْنِي بِغَيْظِكَ.
\par 2 ارْحَمْنِي يَا رَبُّ لأَنِّي ضَعِيفٌ. اشْفِنِي يَا رَبُّ لأَنَّ عِظَامِي قَدْ رَجَفَتْ
\par 3 وَنَفْسِي قَدِ ارْتَاعَتْ جِدّاً. وَأَنْتَ يَا رَبُّ فَحَتَّى مَتَى!
\par 4 عُدْ يَا رَبُّ. نَجِّ نَفْسِي. خَلِّصْنِي مِنْ أَجْلِ رَحْمَتِكَ.
\par 5 لأَنَّهُ لَيْسَ فِي الْمَوْتِ ذِكْرُكَ. فِي الْهَاوِيَةِ مَنْ يَحْمَدُكَ؟
\par 6 تَعِبْتُ فِي تَنَهُّدِي. أُعَوِّمُ فِي كُلِّ لَيْلَةٍ سَرِيرِي بِدُمُوعِي. أُذَوِّبُ فِرَاشِي.
\par 7 سَاخَتْ مِنَ الْغَمِّ عَيْنِي. شَاخَتْ مِنْ كُلِّ مُضَايِقِيَّ.
\par 8 اُبْعُدُوا عَنِّي يَا جَمِيعَ فَاعِلِي الإِثْمِ لأَنَّ الرَّبَّ قَدْ سَمِعَ صَوْتَ بُكَائِي.
\par 9 سَمِعَ الرَّبُّ تَضَرُّعِي. الرَّبُّ يَقْبَلُ صَلاَتِي.
\par 10 جَمِيعُ أَعْدَائِي يُخْزَوْنَ وَيَرْتَاعُونَ جِدّاً. يَعُودُونَ وَيُخْزَوْنَ بَغْتَةً.

\chapter{7}

\par 1 شَجَوِيَّةٌ لِدَاوُدَ غَنَّاهَا لِلرَّبِّ بِسَبَبِ كَلاَمِ كُوشَ الْبِنْيَامِينِيِّ يَا رَبُّ إِلَهِي عَلَيْكَ تَوَكَّلْتُ. خَلِّصْنِي مِنْ كُلِّ الَّذِينَ يَطْرُدُونَنِي وَنَجِّنِي
\par 2 لِئَلاَّ يَفْتَرِسَ كَأَسَدٍ نَفْسِي هَاشِماً إِيَّاهَا وَلاَ مُنْقِذَ.
\par 3 يَا رَبُّ إِلَهِي إِنْ كُنْتُ قَدْ فَعَلْتُ هَذَا. إِنْ وُجِدَ ظُلْمٌ فِي يَدَيَّ.
\par 4 إِنْ كَافَأْتُ مُسَالِمِي شَرّاً وَسَلَبْتُ مُضَايِقِي بِلاَ سَبَبٍ
\par 5 فَلْيُطَارِدْ عَدُوٌّ نَفْسِي وَلْيُدْرِكْهَا وَلْيَدُسْ إِلَى الأَرْضِ حَيَاتِي وَلْيَحُطَّ إِلَى التُّرَابِ مَجْدِي. سِلاَهْ.
\par 6 قُمْ يَا رَبُّ بِغَضَبِكَ. ارْتَفِعْ عَلَى سَخَطِ مُضَايِقِيَّ وَانْتَبِهْ لِي. بِالْحَقِّ أَوْصَيْتَ.
\par 7 وَمَجْمَعُ الْقَبَائِلِ يُحِيطُ بِكَ فَعُدْ فَوْقَهَا إِلَى الْعُلَى.
\par 8 الرَّبُّ يَدِينُ الشُّعُوبَ. اقْضِ لِي يَا رَبُّ كَحَقِّي وَمِثْلَ كَمَالِي الَّذِي فِيَّ.
\par 9 لِيَنْتَهِ شَرُّ الأَشْرَارِ وَثَبِّتِ الصِّدِّيقَ. فَإِنَّ فَاحِصَ الْقُلُوبِ وَالْكُلَى اللهُ الْبَارُّ.
\par 10 تُرْسِي عِنْدَ اللهِ مُخَلِّصِ مُسْتَقِيمِي الْقُلُوبِ.
\par 11 اَللهُ قَاضٍ عَادِلٌ وَإِلَهٌ يَسْخَطُ فِي كُلِّ يَوْمٍ.
\par 12 إِنْ لَمْ يَرْجِعْ يُحَدِّدْ سَيْفَهُ. مَدَّ قَوْسَهُ وَهَيَّأَهَا
\par 13 وَسَدَّدَ نَحْوَهُ آلَةَ الْمَوْتِ. يَجْعَلُ سِهَامَهُ مُلْتَهِبَةً.
\par 14 هُوَذَا يَمْخَضُ بِالإِثْمِ. حَمَلَ تَعَباً وَوَلَدَ كَذِباً.
\par 15 كَرَا جُبّاً. حَفَرَهُ فَسَقَطَ فِي الْهُوَّةِ الَّتِي صَنَعَ.
\par 16 يَرْجِعُ تَعَبُهُ عَلَى رَأْسِهِ وَعَلَى هَامَتِهِ يَهْبِطُ ظُلْمُهُ.
\par 17 أَحْمَدُ الرَّبَّ حَسَبَ بِرِّهِ. وَأُرَنِّمُ لاِسْمِ الرَّبِّ الْعَلِيِّ.

\chapter{8}

\par 1 لإِمَامِ الْمُغَنِّينَ عَلَى الْجَتِّيَّةِ. مَزْمُورٌ لِدَاوُدَ أَيُّهَا الرَّبُّ سَيِّدُنَا مَا أَمْجَدَ اسْمَكَ فِي كُلِّ الأَرْضِ حَيْثُ جَعَلْتَ جَلاَلَكَ فَوْقَ السَّمَاوَاتِ!
\par 2 مِنْ أَفْوَاهِ الأَطْفَالِ وَالرُّضَّعِ أَسَّسْتَ حَمْداً بِسَبَبِ أَضْدَادِكَ لِتَسْكِيتِ عَدُوٍّ وَمُنْتَقِمٍ.
\par 3 إِذَا أَرَى سَمَاوَاتِكَ عَمَلَ أَصَابِعِكَ الْقَمَرَ وَالنُّجُومَ الَّتِي كَوَّنْتَهَا
\par 4 فَمَنْ هُوَ الإِنْسَانُ حَتَّى تَذْكُرَهُ وَابْنُ آدَمَ حَتَّى تَفْتَقِدَهُ!
\par 5 وَتَنْقُصَهُ قَلِيلاً عَنِ الْمَلاَئِكَةِ وَبِمَجْدٍ وَبَهَاءٍ تُكَلِّلُهُ.
\par 6 تُسَلِّطُهُ عَلَى أَعْمَالِ يَدَيْكَ. جَعَلْتَ كُلَّ شَيْءٍ تَحْتَ قَدَمَيْهِ.
\par 7 الْغَنَمَ وَالْبَقَرَ جَمِيعاً وَبَهَائِمَ الْبَرِّ أَيْضاً
\par 8 وَطُيُورَ السَّمَاءِ وَسَمَكَ الْبَحْرِ السَّالِكَ فِي سُبُلِ الْمِيَاهِ.
\par 9 أَيُّهَا الرَّبُّ سَيِّدُنَا مَا أَمْجَدَ اسْمَكَ فِي كُلِّ الأَرْضِ!

\chapter{9}

\par 1 لإِمَامِ الْمُغَنِّينَ. عَلَى [مَوْتِ الاِبْنِ]. مَزْمُورٌ لِدَاوُدَ أَحْمَدُ الرَّبَّ بِكُلِّ قَلْبِي. أُحَدِّثُ بِجَمِيعِ عَجَائِبِكَ.
\par 2 أَفْرَحُ وَأَبْتَهِجُ بِكَ. أُرَنِّمُ لاِسْمِكَ أَيُّهَا الْعَلِيُّ.
\par 3 عِنْدَ رُجُوعِ أَعْدَائِي إِلَى خَلْفٍ يَسْقُطُونَ وَيَهْلِكُونَ مِنْ قُدَّامِ وَجْهِكَ
\par 4 لأَنَّكَ أَقَمْتَ حَقِّي وَدَعْوَايَ. جَلَسْتَ عَلَى الْكُرْسِيِّ قَاضِياً عَادِلاً.
\par 5 انْتَهَرْتَ الأُمَمَ. أَهْلَكْتَ الشِّرِّيرَ. مَحَوْتَ اسْمَهُمْ إِلَى الدَّهْرِ وَالأَبَدِ.
\par 6 اَلْعَدُوُّ تَمَّ خَرَابُهُ إِلَى الأَبَدِ. وَهَدَمْتَ مُدُناً. بَادَ ذِكْرُهُ نَفْسُهُ.
\par 7 أَمَّا الرَّبُّ فَإِلَى الدَّهْرِ يَجْلِسُ. ثَبَّتَ لِلْقَضَاءِ كُرْسِيَّهُ
\par 8 وَهُوَ يَقْضِي لِلْمَسْكُونَةِ بِالْعَدْلِ. يَدِينُ الشُّعُوبَ بِالاِسْتِقَامَةِ.
\par 9 وَيَكُونُ الرَّبُّ مَلْجَأً لِلْمُنْسَحِقِ. مَلْجَأً فِي أَزْمِنَةِ الضِّيقِ.
\par 10 وَيَتَّكِلُ عَلَيْكَ الْعَارِفُونَ اسْمَكَ. لأَنَّكَ لَمْ تَتْرُكْ طَالِبِيكَ يَا رَبُّ.
\par 11 رَنِّمُوا لِلرَّبِّ السَّاكِنِ فِي صِهْيَوْنَ. أَخْبِرُوا بَيْنَ الشُّعُوبِ بِأَفْعَالِهِ.
\par 12 لأَنَّهُ مُطَالِبٌ بِالدِّمَاءِ. ذَكَرَهُمْ. لَمْ يَنْسَ صُرَاخَ الْمَسَاكِينِ.
\par 13 اِرْحَمْنِي يَا رَبُّ. انْظُرْ مَذَلَّتِي مِنْ مُبْغِضِيَّ يَا رَافِعِي مِنْ أَبْوَابِ الْمَوْتِ.
\par 14 لِكَيْ أُحَدِّثَ بِكُلِّ تَسَابِيحِكَ فِي أَبْوَابِ ابْنَةِ صِهْيَوْنَ مُبْتَهِجاً بِخَلاَصِكَ.
\par 15 تَوَرَّطَتِ الأُمَمُ فِي الْحُفْرَةِ الَّتِي عَمِلُوهَا. فِي الشَّبَكَةِ الَّتِي أَخْفُوهَا انْتَشَبَتْ أَرْجُلُهُمْ.
\par 16 مَعْرُوفٌ هُوَ الرَّبُّ. قَضَاءً أَمْضَى. الشِّرِّيرُ يَعْلَقُ بِعَمَلِ يَدَيْهِ. (ضَرْبُ الأَوْتَارِ). سِلاَهْ.
\par 17 اَلأَشْرَارُ يَرْجِعُونَ إِلَى الْهَاوِيَةِ كُلُّ الأُمَمِ النَّاسِينَ اللهَ.
\par 18 لأَنَّهُ لاَ يُنْسَى الْمِسْكِينُ إِلَى الأَبَدِ. رَجَاءُ الْبَائِسِينَ لاَ يَخِيبُ إِلَى الدَّهْرِ.
\par 19 قُمْ يَا رَبُّ. لاَ يَعْتَزَّ الإِنْسَانُ. لِتُحَاكَمِ الأُمَمُ قُدَّامَكَ.
\par 20 يَا رَبُّ اجْعَلْ عَلَيْهِمْ رُعْباً لِيَعْلَمِ الأُمَمُ أَنَّهُمْ بَشَرٌ. سِلاَهْ

\chapter{10}

\par 1 يَا رَبُّ لِمَاذَا تَقِفُ بَعِيداً؟ لِمَاذَا تَخْتَفِي فِي أَزْمِنَةِ الضِّيقِ؟
\par 2 فِي كِبْرِيَاءِ الشِّرِّيرِ يَحْتَرِقُ الْمِسْكِينُ. يُؤْخَذُونَ بِالْمُؤَامَرَةِ الَّتِي فَكَّرُوا بِهَا.
\par 3 لأَنَّ الشِّرِّيرَ يَفْتَخِرُ بِشَهَوَاتِ نَفْسِهِ وَالْخَاطِفُ يُجَدِّفُ. يُهِينُ الرَّبَّ.
\par 4 الشِّرِّيرُ حَسَبَ تَشَامُخِ أَنْفِهِ يَقُولُ: [لاَ يُطَالِبُ]. كُلُّ أَفْكَارِهِ أَنَّهُ لاَ إِلَهَ.
\par 5 تَثْبُتُ سُبُلُهُ فِي كُلِّ حِينٍ. عَالِيَةٌ أَحْكَامُكَ فَوْقَهُ. كُلُّ أَعْدَائِهِ يَنْفُثُ فِيهِمْ.
\par 6 قَالَ فِي قَلْبِهِ: [لاَ أَتَزَعْزَعُ. مِنْ دَوْرٍ إِلَى دَوْرٍ بِلاَ سُوءٍ].
\par 7 فَمُهُ مَمْلُوءٌ لَعْنَةً وَغِشّاً وَظُلْماً. تَحْتَ لِسَانِهِ مَشَقَّةٌ وَإِثْمٌ.
\par 8 يَجْلِسُ فِي مَكْمَنِ الدِّيَارِ فِي الْمُخْتَفَيَاتِ يَقْتُلُ الْبَرِيءَ. عَيْنَاهُ تُرَاقِبَانِ الْمِسْكِينَ.
\par 9 يَكْمُنُ فِي الْمُخْتَفَى كَأَسَدٍ فِي عِرِّيسِهِ. يَكْمُنُ لِيَخْطُفَ الْمِسْكِينَ. يَخْطُفُ الْمِسْكِينَ بِجَذْبِهِ فِي شَبَكَتِهِ
\par 10 فَتَنْسَحِقُ وَتَنْحَنِي وَتَسْقُطُ الْمَسَاكِينُ بِبَرَاثِنِهِ.
\par 11 قَالَ فِي قَلْبِهِ: [إِنَّ اللهَ قَدْ نَسِيَ. حَجَبَ وَجْهَهُ. لاَ يَرَى إِلَى الأَبَدِ].
\par 12 قُمْ يَا رَبُّ. يَا اللهُ ارْفَعْ يَدَكَ. لاَ تَنْسَ الْمَسَاكِينَ.
\par 13 لِمَاذَا أَهَانَ الشِّرِّيرُ اللهَ؟ لِمَاذَا قَالَ فِي قَلْبِهِ: [لاَ تُطَالِبُ]؟
\par 14 قَدْ رَأَيْتَ. لأَنَّكَ تُبْصِرُ الْمَشَقَّةَ وَالْغَمَّ لِتُجَازِيَ بِيَدِكَ. إِلَيْكَ يُسَلِّمُ الْمِسْكِينُ أَمْرَهُ. أَنْتَ صِرْتَ مُعِينَ الْيَتِيمِ.
\par 15 اِحْطِمْ ذِرَاعَ الْفَاجِرِ. وَالشِّرِّيرُ تَطْلُبُ شَرَّهُ وَلاَ تَجِدُهُ.
\par 16 الرَّبُّ مَلِكٌ إِلَى الدَّهْرِ وَالأَبَدِ. بَادَتِ الأُمَمُ مِنْ أَرْضِهِ.
\par 17 تَأَوُّهَ الْوُدَعَاءِ قَدْ سَمِعْتَ يَا رَبُّ. تُثَبِّتُ قُلُوبَهُمْ. تُمِيلُ أُذْنَكَ
\par 18 لِحَقِّ الْيَتِيمِ وَالْمُنْسَحِقِ لِكَيْ لاَ يَعُودَ أَيْضاً يُرْعِبُهُمْ إِنْسَانٌ مِنَ الأَرْضِ.

\chapter{11}

\par 1 لإِمَامِ الْمُغَنِّينَ. لِدَاوُدَ عَلَى الرَّبِّ تَوَكَّلْتُ. كَيْفَ تَقُولُونَ لِنَفْسِي: [اهْرُبُوا إِلَى جِبَالِكُمْ كَعُصْفُورٍ]؟
\par 2 لأَنَّهُ هُوَذَا الأَشْرَارُ يَمُدُّونَ الْقَوْسَ. فَوَّقُوا السَّهْمَ فِي الْوَتَرِ لِيَرْمُوا فِي الدُّجَى مُسْتَقِيمِي الْقُلُوبِ.
\par 3 إِذَا انْقَلَبَتِ الأَعْمِدَةُ فَالصِّدِّيقُ مَاذَا يَفْعَلُ؟
\par 4 اَلرَّبُّ فِي هَيْكَلِ قُدْسِهِ. الرَّبُّ فِي السَّمَاءِ كُرْسِيُّهُ. عَيْنَاهُ تَنْظُرَانِ. أَجْفَانُهُ تَمْتَحِنُ بَنِي آدَمَ.
\par 5 الرَّبُّ يَمْتَحِنُ الصِّدِّيقَ. أَمَّا الشِّرِّيرُ وَمُحِبُّ الظُّلْمِ فَتُبْغِضُهُ نَفْسُهُ.
\par 6 يُمْطِرُ عَلَى الأَشْرَارِ فِخَاخاً نَاراً وَكِبْرِيتاً وَرِيحَ السَّمُومِ نَصِيبَ كَأْسِهِمْ.
\par 7 لأَنَّ الرَّبَّ عَادِلٌ وَيُحِبُّ الْعَدْلَ. الْمُسْتَقِيمُ يُبْصِرُ وَجْهَهُ.

\chapter{12}

\par 1 لإِمَامِ الْمُغَنِّينَ عَلَى [الْقَرَارِ]. مَزْمُورٌ لِدَاوُدَ خَلِّصْ يَا رَبُّ لأَنَّهُ قَدِ انْقَرَضَ التَّقِيُّ لأَنَّهُ قَدِ انْقَطَعَ الأُمَنَاءُ مِنْ بَنِي الْبَشَرِ.
\par 2 يَتَكَلَّمُونَ بِالْكَذِبِ كُلُّ وَاحِدٍ مَعَ صَاحِبِهِ بِشِفَاهٍ مَلِقَةٍ بِقَلْبٍ فَقَلْبٍ يَتَكَلَّمُونَ.
\par 3 يَقْطَعُ الرَّبُّ جَمِيعَ الشِّفَاهِ الْمَلِقَةِ وَاللِّسَانَ الْمُتَكَلِّمَ بِالْعَظَائِمِ
\par 4 الَّذِينَ قَالُوا: [بِأَلْسِنَتِنَا نَتَجَبَّرُ. شِفَاهُنَا مَعَنَا. مَنْ هُوَ سَيِّدٌ عَلَيْنَا؟].
\par 5 [مِنِ اغْتِصَابِ الْمَسَاكِينِ مِنْ صَرْخَةِ الْبَائِسِينَ الآنَ أَقُومُ يَقُولُ الرَّبُّ. [أَجْعَلُ فِي وُسْعٍ الَّذِي يُنْفُثُ فِيهِ].
\par 6 كَلاَمُ الرَّبِّ كَلاَمٌ نَقِيٌّ كَفِضَّةٍ مُصَفَّاةٍ فِي بُوطَةٍ فِي الأَرْضِ مَمْحُوصَةٍ سَبْعَ مَرَّاتٍ.
\par 7 أَنْتَ يَا رَبُّ تَحْفَظُهُمْ. تَحْرُسُهُمْ مِنْ هَذَا الْجِيلِ إِلَى الدَّهْرِ.
\par 8 الأَشْرَارُ يَتَمَشُّونَ مِنْ كُلِّ نَاحِيَةٍ عِنْدَ ارْتِفَاعِ الأَرْذَالِ بَيْنَ النَّاسِ.

\chapter{13}

\par 1 لإِمَامِ الْمُغَنِّينَ. مَزْمُورٌ لِدَاوُدَ إِلَى مَتَى يَا رَبُّ تَنْسَانِي كُلَّ النِّسْيَانِ! إِلَى مَتَى تَحْجُبُ وَجْهَكَ عَنِّي!
\par 2 إِلَى مَتَى أَجْعَلُ هُمُوماً فِي نَفْسِي وَحُزْناً فِي قَلْبِي كُلَّ يَوْمٍ! إِلَى مَتَى يَرْتَفِعُ عَدُوِّي عَلَيَّ!
\par 3 انْظُرْ وَاسْتَجِبْ لِي يَا رَبُّ إِلَهِي. أَنِرْ عَيْنَيَّ لِئَلاَّ أَنَامَ نَوْمَ الْمَوْتِ
\par 4 لِئَلاَّ يَقُولَ عَدُوِّي: [قَدْ قَوِيتُ عَلَيْهِ]. لِئَلاَّ يَهْتِفَ مُضَايِقِيَّ بِأَنِّي تَزَعْزَعْتُ.
\par 5 أَمَّا أَنَا فَعَلَى رَحْمَتِكَ تَوَكَّلْتُ. يَبْتَهِجُ قَلْبِي بِخَلاَصِكَ.
\par 6 أُغَنِّي لِلرَّبِّ لأَنَّهُ أَحْسَنَ إِلَيَّ.

\chapter{14}

\par 1 لإِمَامِ الْمُغَنِّينَ. لِدَاوُدَ قَالَ الْجَاهِلُ فِي قَلْبِهِ: [لَيْسَ إِلَهٌ]. فَسَدُوا وَرَجِسُوا بِأَفْعَالِهِمْ. لَيْسَ مَنْ يَعْمَلُ صَلاَحاً.
\par 2 اَلرَّبُّ مِنَ السَّمَاءِ أَشْرَفَ عَلَى بَنِي الْبَشَرِ لِيَنْظُرَ: هَلْ مِنْ فَاهِمٍ طَالِبِ اللهِ؟
\par 3 الْكُلُّ قَدْ زَاغُوا مَعاً فَسَدُوا. لَيْسَ مَنْ يَعْمَلُ صَلاَحاً لَيْسَ وَلاَ وَاحِدٌ.
\par 4 أَلَمْ يَعْلَمْ كُلُّ فَاعِلِي الإِثْمِ الَّذِينَ يَأْكُلُونَ شَعْبِي كَمَا يَأْكُلُونَ الْخُبْزَ وَالرَّبَّ لَمْ يَدْعُوا.
\par 5 هُنَاكَ خَافُوا خَوْفاً لأَنَّ اللهَ فِي الْجِيلِ الْبَارِّ.
\par 6 رَأْيَ الْمَِسْكِينِ نَاقَضْتُمْ لأَنَّ الرَّبَّ مَلْجَأُهُ.
\par 7 لَيْتَ مِنْ صِهْيَوْنَ خَلاَصَ إِسْرَائِيلَ. عِنْدَ رَدِّ الرَّبِّ سَبْيَ شَعْبِهِ يَهْتِفُ يَعْقُوبُ وَيَفْرَحُ إِسْرَائِيلُ.

\chapter{15}

\par 1 مَزْمُورٌ لِدَاوُدَ يَا رَبُّ مَنْ يَنْزِلُ فِي مَسْكَنِكَ؟ مَنْ يَسْكُنُ فِي جَبَلِ قُدْسِكَ؟
\par 2 السَّالِكُ بِالْكَمَالِ وَالْعَامِلُ الْحَقَّ وَالْمُتَكَلِّمُ بِالصِّدْقِ فِي قَلْبِهِ.
\par 3 الَّذِي لاَ يَشِي بِلِسَانِهِ وَلاَ يَصْنَعُ شَرّاً بِصَاحِبِهِ وَلاَ يَحْمِلُ تَعْيِيراً عَلَى قَرِيبِهِ.
\par 4 وَالرَّذِيلُ مُحْتَقَرٌ فِي عَيْنَيْهِ وَيُكْرِمُ خَائِفِي الرَّبِّ. يَحْلِفُ لِلضَّرَرِ وَلاَ يُغَيِّرُ.
\par 5 فِضَّتُهُ لاَ يُعْطِيهَا بِالرِّبَا وَلاَ يَأْخُذُ الرَّشْوَةَ عَلَى الْبَرِيءِ. الَّذِي يَصْنَعُ هَذَا لاَ يَتَزَعْزَعُ إِلَى الدَّهْرِ.

\chapter{16}

\par 1 مُذَهَّبَةٌ لِدَاوُدَ اِحْفَظْنِي يَا اللهُ لأَنِّي عَلَيْكَ تَوَكَّلْتُ.
\par 2 قُلْتُ لِلرَّبِّ: [أَنْتَ سَيِّدِي. خَيْرِي لاَ شَيْءَ غَيْرُكَ.
\par 3 الْقِدِّيسُونَ الَّذِينَ فِي الأَرْضِ وَالأَفَاضِلُ كُلُّ مَسَرَّتِي بِهِمْ].
\par 4 تَكْثُرُ أَوْجَاعُهُمُ الَّذِينَ أَسْرَعُوا وَرَاءَ آخَرَ. لاَ أَسْكُبُ سَكَائِبَهُمْ مِنْ دَمٍ وَلاَ أَذْكُرُ أَسْمَاءَهُمْ بِشَِفَتَيَّ.
\par 5 الرَّبُّ نَصِيبُ قِسْمَتِي وَكَأْسِي. أَنْتَ قَابِضُ قُرْعَتِي.
\par 6 حِبَالٌ وَقَعَتْ لِي فِي النُّعَمَاءِ فَالْمِيرَاثُ حَسَنٌ عِنْدِي.
\par 7 أُبَارِكُ الرَّبَّ الَّذِي نَصَحَنِي وَأَيْضاً بِاللَّيْلِ تُنْذِرُنِي كُلْيَتَايَ.
\par 8 جَعَلْتُ الرَّبَّ أَمَامِي فِي كُلِّ حِينٍ. لأَنَّهُ عَنْ يَمِينِي فَلاَ أَتَزَعْزَعُ.
\par 9 لِذَلِكَ فَرِحَ قَلْبِي وَابْتَهَجَتْ رُوحِي. جَسَدِي أَيْضاً يَسْكُنُ مُطْمَئِنّاً.
\par 10 لأَنَّكَ لَنْ تَتْرُكَ نَفْسِي فِي الْهَاوِيَةِ. لَنْ تَدَعَ تَقِيَّكَ يَرَى فَسَاداً.
\par 11 تُعَرِّفُنِي سَبِيلَ الْحَيَاةِ. أَمَامَكَ شِبَعُ سُرُورٍ. فِي يَمِينِكَ نِعَمٌ إِلَى الأَبَدِ.

\chapter{17}

\par 1 صَلاَةٌ لِدَاوُدَ اِسْمَعْ يَا رَبُّ لِلْحَقِّ. أُنْصُتْ إِلَى صُرَاخِي. أَصْغِ إِلَى صَلاَتِي مِنْ شَفَتَيْنِ بِلاَ غِشٍّ.
\par 2 مِنْ قُدَّامِكَ يَخْرُجُ قَضَائِي. عَيْنَاكَ تَنْظُرَانِ الْمُسْتَقِيمَاتِ.
\par 3 جَرَّبْتَ قَلْبِي. تَعَهَّدْتَهُ لَيْلاً. مَحَّصْتَنِي. لاَ تَجِدُ فِيَّ ذُمُوماً. لاَ يَتَعَدَّى فَمِي.
\par 4 مِنْ جِهَةِ أَعْمَالِ النَّاسِ فَبِكَلاَمِ شَفَتَيْكَ أَنَا تَحَفَّظْتُ مِنْ طُرُقِ الْمُعْتَنِفِ.
\par 5 تَمَسَّكَتْ خَطَواتِي بِآثَارِكَ فَمَا زَلَّتْ قَدَمَايَ.
\par 6 أَنَا دَعَوْتُكَ لأَنَّكَ تَسْتَجِيبُ لِي يَا اللهُ. أَمِلْ أُذُنَيْكَ إِلَيَّ. اسْمَعْ كَلاَمِي.
\par 7 مَيِّزْ مَرَاحِمَكَ يَا مُخَلِّصَ الْمُتَّكِلِينَ عَلَيْكَ بِيَمِينِكَ مِنَ الْمُقَاوِمِينَ.
\par 8 احْفَظْنِي مِثْلَ حَدَقَةِ الْعَيْنِ. بِظِلِّ جَنَاحَيْكَ اسْتُرْنِي
\par 9 مِنْ وَجْهِ الأَشْرَارِ الَّذِينَ يُخْرِبُونَنِي أَعْدَائِي بِالنَّفْسِ الَّذِينَ يَكْتَنِفُونَنِي.
\par 10 قَلْبَهُمُ السَّمِينَ قَدْ أَغْلَقُوا. بِأَفْوَاهِهِمْ قَدْ تَكَلَّمُوا بِالْكِبْرِيَاءِ.
\par 11 فِي خَطَواتِنَا الآنَ قَدْ أَحَاطُوا بِنَا. نَصَبُوا أَعْيُنَهُمْ لِيُزْلِقُونَا إِلَى الأَرْضِ.
\par 12 مَثَلُهُ مَثَلُ الأَسَدِ الْقَرِمِ إِلَى الاِفْتِرَاسِ وَكَالشِّبْلِ الْكَامِنِ فِي عِرِّيسِهِ.
\par 13 قُمْ يَا رَبُّ. تَقَدَّمْهُ. اصْرَعْهُ. نَجِّ نَفْسِي مِنَ الشِّرِّيرِ بِسَيْفِكَ
\par 14 مِنَ النَّاسِ بِيَدِكَ يَا رَبُّ مِنْ أَهْلِ الدُّنْيَا. نَصِيبُهُمْ فِي حَيَاتِهِمْ. بِذَخَائِرِكَ تَمْلَأُ بُطُونَهُمْ. يَشْبَعُونَ أَوْلاَداً وَيَتْرُكُونَ فُضَالَتَهُمْ لأَطْفَالِهِمْ.
\par 15 أَمَّا أَنَا فَبِالْبِرِّ أَنْظُرُ وَجْهَكَ. أَشْبَعُ إِذَا اسْتَيْقَظْتُ بِشَبَهِكَ.

\chapter{18}

\par 1 لإِمَامِ الْمُغَنِّينَ. لِعَبْدِ الرَّبِّ دَاوُدَ الَّذِي كَلَّمَ الرَّبَّ بِكَلاَمِ هَذَا النَّشِيدِ فِي الْيَوْمِ الَّذِي أَنْقَذَهُ فِيهِ الرَّبُّ مِنْ أَيْدِي كُلِّ أَعْدَائِهِ وَمِنْ يَدِ شَاوُلَ. فَقَالَ: أُحِبُّكَ يَا رَبُّ يَا قُوَّتِي.
\par 2 الرَّبُّ صَخْرَتِي وَحِصْنِي وَمُنْقِذِي. إِلَهِي صَخْرَتِي بِهِ أَحْتَمِي. تُرْسِي وَقَرْنُ خَلاَصِي وَمَلْجَإِي.
\par 3 أَدْعُو الرَّبَّ الْحَمِيدَ فَأَتَخَلَّصُ مِنْ أَعْدَائِي.
\par 4 اِكْتَنَفَتْنِي حِبَالُ الْمَوْتِ وَسُيُولُ الْهَلاَكِ أَفْزَعَتْنِي.
\par 5 حِبَالُ الْهَاوِيَةِ حَاقَتْ بِي. أَشْرَاكُ الْمَوْتِ انْتَشَبَتْ بِي.
\par 6 فِي ضِيقِي دَعَوْتُ الرَّبَّ وَإِلَى إِلَهِي صَرَخْتُ فَسَمِعَ مِنْ هَيْكَلِهِ صَوْتِي وَصُرَاخِي قُدَّامَهُ دَخَلَ أُذُنَيْهِ.
\par 7 فَارْتَجَّتِ الأَرْضُ وَارْتَعَشَتْ أُسُسُ الْجِبَالِ. ارْتَعَدَتْ وَارْتَجَّتْ لأَنَّهُ غَضِبَ.
\par 8 صَعِدَ دُخَانٌ مِنْ أَنْفِهِ وَنَارٌ مِنْ فَمِهِ أَكَلَتْ. جَمْرٌ اشْتَعَلَتْ مِنْهُ.
\par 9 طَأْطَأَ السَّمَاوَاتِ وَنَزَلَ وَضَبَابٌ تَحْتَ رِجْلَيْهِ.
\par 10 رَكِبَ عَلَى كَرُوبٍ وَطَارَ وَهَفَّ عَلَى أَجْنِحَةِ الرِّيَاحِ.
\par 11 جَعَلَ الظُّلْمَةَ سِتْرَهُ. حَوْلَهُ مَظَلَّتَهُ ضَبَابَ الْمِيَاهِ وَظَلاَمَ الْغَمَامِ.
\par 12 مِنَ الشُّعَاعِ قُدَّامَهُ عَبَرَتْ سُحُبُهُ. بَرَدٌ وَجَمْرُ نَارٍ.
\par 13 أَرْعَدَ الرَّبُّ مِنَ السَّمَاوَاتِ وَالْعَلِيُّ أَعْطَى صَوْتَهُ بَرَداً وَجَمْرَ نَارٍ.
\par 14 أَرْسَلَ سِهَامَهُ فَشَتَّتَهُمْ وَبُرُوقاً كَثِيرَةً فَأَزْعَجَهُمْ
\par 15 فَظَهَرَتْ أَعْمَاقُ الْمِيَاهِ وَانْكَشَفَتْ أُسُسُ الْمَسْكُونَةِ مِنْ زَجْرِكَ يَا رَبُّ مِنْ نَسَمَةِ رِيحِ أَنْفِكَ.
\par 16 أَرْسَلَ مِنَ الْعُلَى فَأَخَذَنِي. نَشَلَنِي مِنْ مِيَاهٍ كَثِيرَةٍ.
\par 17 أَنْقَذَنِي مِنْ عَدُوِّي الْقَوِيِّ وَمِنْ مُبْغِضِيَّ لأَنَّهُمْ أَقْوَى مِنِّي.
\par 18 أَصَابُونِي فِي يَوْمِ بَلِيَّتِي وَكَانَ الرَّبُّ سَنَدِي.
\par 19 أَخْرَجَنِي إِلَى الرُّحْبِ. خَلَّصَنِي لأَنَّهُ سُرَّ بِي.
\par 20 يُكَافِئُنِي الرَّبُّ حَسَبَ بِرِّي. حَسَبَ طَهَارَةِ يَدَيَّ يَرُدُّ لِي.
\par 21 لأَنِّي حَفِظْتُ طُرُقَ الرَّبِّ وَلَمْ أَعْصِ إِلَهِي.
\par 22 لأَنَّ جَمِيعَ أَحْكَامِهِ أَمَامِي وَفَرَائِضَهُ لَمْ أُبْعِدْهَا عَنْ نَفْسِي.
\par 23 وَأَكُونُ كَامِلاً مَعَهُ وَأَتَحَفَّظُ مِنْ إِثْمِي.
\par 24 فَيَرُدُّ الرَّبُّ لِي كَبِرِّي وَكَطَهَارَةِ يَدَيَّ أَمَامَ عَيْنَيْهِ.
\par 25 مَعَ الرَّحِيمِ تَكُونُ رَحِيماً. مَعَ الرَّجُلِ الْكَامِلِ تَكُونُ كَامِلاً.
\par 26 مَعَ الطَّاهِرِ تَكُونُ طَاهِراً. وَمَعَ الأَعْوَجِ تَكُونُ مُلْتَوِياً.
\par 27 لأَنَّكَ أَنْتَ تُخَلِّصُ الشَّعْبَ الْبَائِسَ وَالأَعْيُنُ الْمُرْتَفِعَةُ تَضَعُهَا.
\par 28 لأَنَّكَ أَنْتَ تُضِيءُ سِرَاجِي. الرَّبُّ إِلَهِي يُنِيرُ ظُلْمَتِي.
\par 29 لأَنِّي بِكَ اقْتَحَمْتُ جَيْشاً وَبِإِلَهِي تَسَوَّرْتُ أَسْوَاراً.
\par 30 اَللهُ طَرِيقُهُ كَامِلٌ. قَوْلُ الرَّبِّ نَقِيٌّ. تُرْسٌ هُوَ لِجَمِيعِ الْمُحْتَمِينَ بِهِ.
\par 31 لأَنَّهُ مَنْ هُوَ إِلَهٌ غَيْرُ الرَّبِّ! وَمَنْ هُوَ صَخْرَةٌ سِوَى إِلَهِنَا!
\par 32 الإِلَهُ الَّذِي يُمَنْطِقُنِي بِالْقُوَّةِ وَيُصَيِّرُ طَرِيقِي كَامِلاً.
\par 33 الَّذِي يَجْعَلُ رِجْلَيَّ كَالإِيَّلِ وَعَلَى مُرْتَفِعَاتِي يُقِيمُنِي.
\par 34 الَّذِي يُعَلِّمُ يَدَيَّ الْقِتَالَ فَتُحْنَى بِذِرَاعَيَّ قَوْسٌ مِنْ نُحَاسٍ.
\par 35 وَتَجْعَلُ لِي تُرْسَ خَلاَصِكَ وَيَمِينُكَ تَعْضُدُنِي وَلُطْفُكَ يُعَظِّمُنِي.
\par 36 تُوَسِّعُ خُطُوَاتِي تَحْتِي فَلَمْ تَتَقَلْقَلْ عَقِبَايَ.
\par 37 أَتْبَعُ أَعْدَائِي فَأُدْرِكُهُمْ وَلاَ أَرْجِعُ حَتَّى أُفْنِيَهُمْ.
\par 38 أَسْحَقُهُمْ فَلاَ يَسْتَطِيعُونَ الْقِيَامَ. يَسْقُطُونَ تَحْتَ رِجْلَيَّ.
\par 39 تُمَنْطِقُنِي بِقُوَّةٍ لِلْقِتَالِ. تَصْرَعُ تَحْتِي الْقَائِمِينَ عَلَيَّ.
\par 40 وَتُعْطِينِي أَقْفِيَةَ أَعْدَائِي وَمُبْغِضِيَّ أُفْنِيهِمْ.
\par 41 يَصْرُخُونَ وَلاَ مُخَلِّصَ. إِلَى الرَّبِّ فَلاَ يَسْتَجِيبُ لَهُمْ.
\par 42 فَأَسْحَقُهُمْ كَالْغُبَارِ قُدَّامَ الرِّيحِ. مِثْلَ طِينِ الأَسْوَاقِ أَطْرَحُهُمْ.
\par 43 تُنْقِذُنِي مِنْ مُخَاصَمَاتِ الشَّعْبِ. تَجْعَلُنِي رَأْساً لِلأُمَمِ. شَعْبٌ لَمْ أَعْرِفْهُ يَتَعَبَّدُ لِي.
\par 44 مِنْ سَمَاعِ الأُذُنِ يَسْمَعُونَ لِي. بَنُو الْغُرَبَاءِ يَتَذَلَّلُونَ لِي.
\par 45 بَنُو الْغُرَبَاءِ يَبْلُونَ وَيَزْحَفُونَ مِنْ حُصُونِهِمْ.
\par 46 حَيٌّ هُوَ الرَّبُّ وَمُبَارَكٌ صَخْرَتِي وَمُرْتَفِعٌ إِلَهُ خَلاَصِي
\par 47 اَلإِلَهُ الْمُنْتَقِمُ لِي وَالَّذِي يُخْضِعُ الشُّعُوبَ تَحْتِي.
\par 48 مُنَجِّيَّ مِنْ أَعْدَائِي. رَافِعِي أَيْضاً فَوْقَ الْقَائِمِينَ عَلَيَّ. مِنَ الرَّجُلِ الظَّالِمِ تُنْقِذُنِي.
\par 49 لِذَلِكَ أَحْمَدُكَ يَا رَبُّ فِي الأُمَمِ وَأُرَنِّمُ لاِسْمِكَ.
\par 50 بُرْجُ خَلاَصٍ لِمَلِكِهِ وَالصَّانِعُ رَحْمَةً لِمَسِيحِهِ لِدَاوُدَ وَنَسْلِهِ إِلَى الأَبَدِ.

\chapter{19}

\par 1 لإِمَامِ الْمُغَنِّينَ. مَزْمُورٌ لِدَاوُدَ اَلسَّمَاوَاتُ تُحَدِّثُ بِمَجْدِ اللهِ وَالْفَلَكُ يُخْبِرُ بِعَمَلِ يَدَيْهِ.
\par 2 يَوْمٌ إِلَى يَوْمٍ يُذِيعُ كَلاَماً وَلَيْلٌ إِلَى لَيْلٍ يُبْدِي عِلْماً.
\par 3 لاَ قَوْلَ وَلاَ كَلاَمَ. لاَ يُسْمَعُ صَوْتُهُمْ.
\par 4 فِي كُلِّ الأَرْضِ خَرَجَ مَنْطِقُهُمْ وَإِلَى أَقْصَى الْمَسْكُونَةِ كَلِمَاتُهُمْ. جَعَلَ لِلشَّمْسِ مَسْكَناً فِيهَا
\par 5 وَهِيَ مِثْلُ الْعَرُوسِ الْخَارِجِ مِنْ حَجَلَتِهِ. يَبْتَهِجُ مِثْلَ الْجَبَّارِ لِلسِّبَاقِ فِي الطَّرِيقِ.
\par 6 مِنْ أَقْصَى السَّمَاوَاتِ خُرُوجُهَا وَمَدَارُهَا إِلَى أَقَاصِيهَا وَلاَ شَيْءَ يَخْتَفِي مِنْ حَرِّهَا.
\par 7 نَامُوسُ الرَّبِّ كَامِلٌ يَرُدُّ النَّفْسَ. شَهَادَاتُ الرَّبِّ صَادِقَةٌ تُصَيِّرُ الْجَاهِلَ حَكِيماً.
\par 8 وَصَايَا الرَّبِّ مُسْتَقِيمَةٌ تُفَرِّحُ الْقَلْبَ. أَمْرُ الرَّبِّ طَاهِرٌ يُنِيرُ الْعَيْنَيْنِ.
\par 9 خَوْفُ الرَّبِّ نَقِيٌّ ثَابِتٌ إِلَى الأَبَدِ. أَحْكَامُ الرَّبِّ حَقٌّ عَادِلَةٌ كُلُّهَا.
\par 10 أَشْهَى مِنَ الذَّهَبِ وَالإِبْرِيزِ الْكَثِيرِ وَأَحْلَى مِنَ الْعَسَلِ وَقَطْرِ الشِّهَادِ.
\par 11 أَيْضاً عَبْدُكَ يُحَذَّرُ بِهَا وَفِي حِفْظِهَا ثَوَابٌ عَظِيمٌ.
\par 12 اَلسَّهَوَاتُ مَنْ يَشْعُرُ بِهَا! مِنَ الْخَطَايَا الْمُسْتَتِرَةِ أَبْرِئْنِي.
\par 13 أَيْضاً مِنَ الْمُتَكَبِّرِينَ احْفَظْ عَبْدَكَ فَلاَ يَتَسَلَّطُوا عَلَيَّ. حِينَئِذٍ أَكُونُ كَامِلاً وَأَتَبَرَّأُ مِنْ ذَنْبٍ عَظِيمٍ.
\par 14 لِتَكُنْ أَقْوَالُ فَمِي وَفِكْرُ قَلْبِي مَرْضِيَّةً أَمَامَكَ يَا رَبُّ صَخْرَتِي وَوَلِيِّي.

\chapter{20}

\par 1 لإِمَامِ الْمُغَنِّينَ. مَزْمُورٌ لِدَاوُدَ لِيَسْتَجِبْ لَكَ الرَّبُّ فِي يَوْمِ الضِّيقِ. لِيَرْفَعْكَ اسْمُ إِلَهِ يَعْقُوبَ.
\par 2 لِيُرْسِلْ لَكَ عَوْناً مِنْ قُدْسِهِ وَمِنْ صِهْيَوْنَ لِيَعْضُدْكَ.
\par 3 لِيَذْكُرْ كُلَّ تَقْدِمَاتِكَ وَيَسْتَسْمِنْ مُحْرَقَاتِكَ. سِلاَهْ.
\par 4 لِيُعْطِكَ حَسَبَ قَلْبِكَ وَيُتَمِّمْ كُلَّ رَأْيِكَ.
\par 5 نَتَرَنَّمُ بِخَلاَصِكَ وَبِاسْمِ إِلَهِنَا نَرْفَعُ رَايَتَنَا. لِيُكَمِّلِ الرَّبُّ كُلَّ سُؤْلِكَ.
\par 6 اَلآنَ عَرَفْتُ أَنَّ الرَّبَّ مُخَلِّصُ مَسِيحِهِ. يَسْتَجِيبُهُ مِنْ سَمَاءِ قُدْسِهِ بِجَبَرُوتِ خَلاَصِ يَمِينِهِ.
\par 7 هَؤُلاَءِ بِالْمَرْكَبَاتِ وَهَؤُلاَءِ بِالْخَيْلِ - أَمَّا نَحْنُ فَاسْمَ الرَّبِّ إِلَهِنَا نَذْكُرُ.
\par 8 هُمْ جَثُوا وَسَقَطُوا أَمَّا نَحْنُ فَقُمْنَا وَانْتَصَبْنَا.
\par 9 يَا رَبُّ خَلِّصْ. لِيَسْتَجِبْ لَنَا الْمَلِكُ فِي يَوْمِ دُعَائِنَا.

\chapter{21}

\par 1 لإِمَامِ الْمُغَنِّينَ. مَزْمُورٌ لِدَاوُدَ يَا رَبُّ بِقُوَّتِكَ يَفْرَحُ الْمَلِكُ وَبِخَلاَصِكَ كَيْفَ لاَ يَبْتَهِجُ جِدّاً!
\par 2 شَهْوَةَ قَلْبِهِ أَعْطَيْتَهُ وَمُلْتَمَسَ شَفَتَيْهِ لَمْ تَمْنَعْهُ. سِلاَهْ.
\par 3 لأَنَّكَ تَتَقَدَّمُهُ بِبَرَكَاتِ خَيْرٍ. وَضَعْتَ عَلَى رَأْسِهِ تَاجاً مِنْ إِبْرِيزٍ.
\par 4 حَيَاةً سَأَلَكَ فَأَعْطَيْتَهُ. طُولَ الأَيَّامِ إِلَى الدَّهْرِ وَالأَبَدِ.
\par 5 عَظِيمٌ مَجْدُهُ بِخَلاَصِكَ جَلاَلاً وَبَهَاءً تَضَعُ عَلَيْهِ.
\par 6 لأَنَّكَ جَعَلْتَهُ بَرَكَاتٍ إِلَى الأَبَدِ. تُفَرِّحُهُ ابْتِهَاجاً أَمَامَكَ.
\par 7 لأَنَّ الْمَلِكَ يَتَوَكَّلُ عَلَى الرَّبِّ وَبِنِعْمَةِ الْعَلِيِّ لاَ يَتَزَعْزَعُ.
\par 8 تُصِيبُ يَدُكَ جَمِيعَ أَعْدَائِكَ. يَمِينُكَ تُصِيبُ كُلَّ مُبْغِضِيكَ.
\par 9 تَجْعَلُهُمْ مِثْلَ تَنُّورِ نَارٍ فِي زَمَانِ حُضُورِكَ. الرَّبُّ بِسَخَطِهِ يَبْتَلِعُهُمْ وَتَأْكُلُهُمُ النَّارُ.
\par 10 تُبِيدُ ثَمَرَهُمْ مِنَ الأَرْضِ وَذُرِّيَّتَهُمْ مِنْ بَيْنِ بَنِي آدَمَ.
\par 11 لأَنَّهُمْ نَصَبُوا عَلَيْكَ شَرّاً. تَفَكَّرُوا بِمَكِيدَةٍ. لَمْ يَسْتَطِيعُوهَا.
\par 12 لأَنَّكَ تَجْعَلُهُمْ يَتَوَلُّونَ. تُفَوِّقُ السِّهَامَ عَلَى أَوْتَارِكَ تِلْقَاءَ وُجُوهِهِمْ.
\par 13 ارْتَفِعْ يَا رَبُّ بِقُوَّتِكَ. نُرَنِّمُ وَنُنَغِّمُ بِجَبَرُوتِكَ.

\chapter{22}

\par 1 لإِمَامِ الْمُغَنِّينَ عَلَى [أَيِّلَةِ الصُّبْحِ]. مَزْمُورٌ لِدَاوُدَ إِلَهِي! إِلَهِي لِمَاذَا تَرَكْتَنِي بَعِيداً عَنْ خَلاَصِي عَنْ كَلاَمِ زَفِيرِي؟
\par 2 إِلَهِي فِي النَّهَارِ أَدْعُو فَلاَ تَسْتَجِيبُ. فِي اللَّيْلِ أَدْعُو فَلاَ هُدُوءَ لِي.
\par 3 وَأَنْتَ الْقُدُّوسُ الْجَالِسُ بَيْنَ تَسْبِيحَاتِ إِسْرَائِيلَ.
\par 4 عَلَيْكَ اتَّكَلَ آبَاؤُنَا. اتَّكَلُوا فَنَجَّيْتَهُمْ.
\par 5 إِلَيْكَ صَرَخُوا فَنَجُوا. عَلَيْكَ اتَّكَلُوا فَلَمْ يَخْزُوا.
\par 6 أَمَّا أَنَا فَدُودَةٌ لاَ إِنْسَانٌ. عَارٌ عِنْدَ الْبَشَرِ وَمُحْتَقَرُ الشَّعْبِ.
\par 7 كُلُّ الَّذِينَ يَرُونَنِي يَسْتَهْزِئُونَ بِي. يَفْغَرُونَ الشِّفَاهَ وَيُنْغِضُونَ الرَّأْسَ قَائِلِينَ:
\par 8 [اتَّكَلَ عَلَى الرَّبِّ فَلْيُنَجِّهِ. لِيُنْقِذْهُ لأَنَّهُ سُرَّ بِهِ].
\par 9 لأَنَّكَ أَنْتَ جَذَبْتَنِي مِنَ الْبَطْنِ. جَعَلْتَنِي مُطْمَئِنّاً عَلَى ثَدْيَيْ أُمِّي.
\par 10 عَلَيْكَ أُلْقِيتُ مِنَ الرَّحِمِ. مِنْ بَطْنِ أُمِّي أَنْتَ إِلَهِي.
\par 11 لاَ تَتَبَاعَدْ عَنِّي لأَنَّ الضِّيقَ قَرِيبٌ. لأَنَّهُ لاَ مُعِينَ.
\par 12 أَحَاطَتْ بِي ثِيرَانٌ كَثِيرَةٌ. أَقْوِيَاءُ بَاشَانَ اكْتَنَفَتْنِي.
\par 13 فَغَرُوا عَلَيَّ أَفْوَاهَهُمْ كَأَسَدٍ مُفْتَرِسٍ مُزَمْجِرٍ.
\par 14 كَالْمَاءِ انْسَكَبْتُ. انْفَصَلَتْ كُلُّ عِظَامِي. صَارَ قَلْبِي كَالشَّمْعِ. قَدْ ذَابَ فِي وَسَطِ أَمْعَائِي.
\par 15 يَبِسَتْ مِثْلَ شَقْفَةٍ قُوَّتِي وَلَصِقَ لِسَانِي بِحَنَكِي وَإِلَى تُرَابِ الْمَوْتِ تَضَعُنِي.
\par 16 لأَنَّهُ قَدْ أَحَاطَتْ بِي كِلاَبٌ. جَمَاعَةٌ مِنَ الأَشْرَارِ اكْتَنَفَتْنِي. ثَقَبُوا يَدَيَّ وَرِجْلَيَّ.
\par 17 أُحْصِي كُلَّ عِظَامِي وَهُمْ يَنْظُرُونَ وَيَتَفَرَّسُونَ فِيَّ.
\par 18 يَقْسِمُونَ ثِيَابِي بَيْنَهُمْ وَعَلَى لِبَاسِي يَقْتَرِعُونَ.
\par 19 أَمَّا أَنْتَ يَا رَبُّ فَلاَ تَبْعُدْ. يَا قُوَّتِي أَسْرِعْ إِلَى نُصْرَتِي.
\par 20 أَنْقِذْ مِنَ السَّيْفِ نَفْسِي. مِنْ يَدِ الْكَلْبِ وَحِيدَتِي.
\par 21 خَلِّصْنِي مِنْ فَمِ الأَسَدِ وَمِنْ قُرُونِ بَقَرِ الْوَحْشِ اسْتَجِبْ لِي.
\par 22 أُخْبِرْ بِاسْمِكَ إِخْوَتِي. فِي وَسَطِ الْجَمَاعَةِ أُسَبِّحُكَ.
\par 23 يَا خَائِفِي الرَّبِّ سَبِّحُوهُ. مَجِّدُوهُ يَا مَعْشَرَ ذُرِّيَّةِ يَعْقُوبَ. وَاخْشُوهُ يَا زَرْعَ إِسْرَائِيلَ جَمِيعاً.
\par 24 لأَنَّهُ لَمْ يَحْتَقِرْ وَلَمْ يَرْذُلْ مَسْكَنَةَ الْمَِسْكِينِ وَلَمْ يَحْجِبْ وَجْهَهُ عَنْهُ بَلْ عِنْدَ صُرَاخِهِ إِلَيْهِ اسْتَمَعَ.
\par 25 مِنْ قِبَلِكَ تَسْبِيحِي فِي الْجَمَاعَةِ الْعَظِيمَةِ. أُوفِي بِنُذُورِي قُدَّامَ خَائِفِيهِ.
\par 26 يَأْكُلُ الْوُدَعَاءُ وَيَشْبَعُونَ. يُسَبِّحُ الرَّبَّ طَالِبُوهُ. تَحْيَا قُلُوبُكُمْ إِلَى الأَبَدِ.
\par 27 تَذْكُرُ وَتَرْجِعُ إِلَى الرَّبِّ كُلُّ أَقَاصِي الأَرْضِ. وَتَسْجُدُ قُدَّامَكَ كُلُّ قَبَائِلِ الأُمَمِ.
\par 28 لأَنَّ لِلرَّبِّ الْمُلْكَ وَهُوَ الْمُتَسَلِّطُ عَلَى الأُمَمِ.
\par 29 أَكَلَ وَسَجَدَ كُلُّ سَمِينِي الأَرْضِ. قُدَّامَهُ يَجْثُو كُلُّ مَنْ يَنْحَدِرُ إِلَى التُّرَابِ وَمَنْ لَمْ يُحْيِ نَفْسَهُ.
\par 30 الذُّرِّيَّةُ تَتَعَبَّدُ لَهُ. يُخَبَّرُ عَنِ الرَّبِّ الْجِيلُ الآتِي.
\par 31 يَأْتُونَ وَيُخْبِرُونَ بِبِرِّهِ شَعْباً سَيُولَدُ بِأَنَّهُ قَدْ فَعَلَ.

\chapter{23}

\par 1 مَزْمُورٌ لِدَاوُدَ اَلرَّبُّ رَاعِيَّ فَلاَ يُعْوِزُنِي شَيْءٌ.
\par 2 فِي مَرَاعٍ خُضْرٍ يُرْبِضُنِي. إِلَى مِيَاهِ الرَّاحَةِ يُورِدُنِي.
\par 3 يَرُدُّ نَفْسِي. يَهْدِينِي إِلَى سُبُلِ الْبِرِّ مِنْ أَجْلِ اسْمِهِ.
\par 4 أَيْضاً إِذَا سِرْتُ فِي وَادِي ظِلِّ الْمَوْتِ لاَ أَخَافُ شَرّاً لأَنَّكَ أَنْتَ مَعِي. عَصَاكَ وَعُكَّازُكَ هُمَا يُعَزِّيَانِنِي.
\par 5 تُرَتِّبُ قُدَّامِي مَائِدَةً تُجَاهَ مُضَايِقِيَّ. مَسَحْتَ بِالدُّهْنِ رَأْسِي. كَأْسِي رَيَّا.
\par 6 إِنَّمَا خَيْرٌ وَرَحْمَةٌ يَتْبَعَانِنِي كُلَّ أَيَّامِ حَيَاتِي وَأَسْكُنُ فِي بَيْتِ الرَّبِّ إِلَى مَدَى الأَيَّامِ.

\chapter{24}

\par 1 لِدَاوُدَ. مَزْمُورٌ لِلرَّبِّ الأَرْضُ وَمِلْؤُهَا. الْمَسْكُونَةُ وَكُلُّ السَّاكِنِينَ فِيهَا.
\par 2 لأَنَّهُ عَلَى الْبِحَارِ أَسَّسَهَا وَعَلَى الأَنْهَارِ ثَبَّتَهَا.
\par 3 مَنْ يَصْعَدُ إِلَى جَبَلِ الرَّبِّ وَمَنْ يَقُومُ فِي مَوْضِعِ قُدْسِهِ؟
\par 4 اَلطَّاهِرُ الْيَدَيْنِ وَالنَّقِيُّ الْقَلْبِ الَّذِي لَمْ يَحْمِلْ نَفْسَهُ إِلَى الْبَاطِلِ وَلاَ حَلَفَ كَذِباً.
\par 5 يَحْمِلُ بَرَكَةً مِنْ عِنْدِ الرَّبِّ وَبِرّاً مِنْ إِلَهِ خَلاَصِهِ.
\par 6 هَذَا هُوَ الْجِيلُ الطَّالِبُهُ الْمُلْتَمِسُونَ وَجْهَكَ يَا يَعْقُوبُ. سِلاَهْ.
\par 7 اِرْفَعْنَ أَيَّتُهَا الأَرْتَاجُ رُؤُوسَكُنَّ وَارْتَفِعْنَ أَيَّتُهَا الأَبْوَابُ الدَّهْرِيَّاتُ فَيَدْخُلَ مَلِكُ الْمَجْدِ.
\par 8 مَنْ هُوَ هَذَا مَلِكُ الْمَجْدِ؟ الرَّبُّ الْقَدِيرُ الْجَبَّارُ الرَّبُّ الْجَبَّارُ فِي الْقِتَالِ!
\par 9 ارْفَعْنَ أَيَّتُهَا الأَرْتَاجُ رُؤُوسَكُنَّ وَارْفَعْنَهَا أَيَّتُهَا الأَبْوَابُ الدَّهْرِيَّاتُ فَيَدْخُلَ مَلِكُ الْمَجْدِ؟
\par 10 مَنْ هُوَ هَذَا مَلِكُ الْمَجْدِ! رَبُّ الْجُنُودِ هُوَ مَلِكُ الْمَجْدِ. سِلاَهْ

\chapter{25}

\par 1 لِدَاوُدَ إِلَيْكَ يَا رَبُّ أَرْفَعُ نَفْسِي.
\par 2 يَا إِلَهِي عَلَيْكَ تَوَكَّلْتُ فَلاَ تَدَعْنِي أَخْزَى. لاَ تَشْمَتْ بِي أَعْدَائِي.
\par 3 أَيْضاً كُلُّ مُنْتَظِرِيكَ لاَ يَخْزَوْا. لِيَخْزَ الْغَادِرُونَ بِلاَ سَبَبٍ.
\par 4 طُرُقَكَ يَا رَبُّ عَرِّفْنِي. سُبُلَكَ عَلِّمْنِي.
\par 5 دَرِّبْنِي فِي حَقِّكَ وَعَلِّمْنِي. لأَنَّكَ أَنْتَ إِلَهُ خَلاَصِي. إِيَّاكَ انْتَظَرْتُ الْيَوْمَ كُلَّهُ.
\par 6 اذْكُرْ مَرَاحِمَكَ يَا رَبُّ وَإِحْسَانَاتِكَ لأَنَّهَا مُنْذُ الأَزَلِ هِيَ.
\par 7 لاَ تَذْكُرْ خَطَايَا صِبَايَ وَلاَ مَعَاصِيَّ. كَرَحْمَتِكَ اذْكُرْنِي أَنْتَ مِنْ أَجْلِ جُودِكَ يَا رَبُّ.
\par 8 اَلرَّبُّ صَالِحٌ وَمُسْتَقِيمٌ لِذَلِكَ يُعَلِّمُ الْخُطَاةَ الطَّرِيقَ.
\par 9 يُدَرِّبُ الْوُدَعَاءَ فِي الْحَقِّ وَيُعَلِّمُ الْوُدَعَاءَ طُرُقَهُ.
\par 10 كُلُّ سُبُلِ الرَّبِّ رَحْمَةٌ وَحَقٌّ لِحَافِظِي عَهْدِهِ وَشَهَادَاتِهِ.
\par 11 مِنْ أَجْلِ اسْمِكَ يَا رَبُّ اغْفِرْ إِثْمِي لأَنَّهُ عَظِيمٌ.
\par 12 مَنْ هُوَ الإِنْسَانُ الْخَائِفُ الرَّبَّ؟ يُعَلِّمُهُ طَرِيقاً يَخْتَارُهُ.
\par 13 نَفْسُهُ فِي الْخَيْرِ تَبِيتُ وَنَسْلُهُ يَرِثُ الأَرْضَ.
\par 14 سِرُّ الرَّبِّ لِخَائِفِيهِ وَعَهْدُهُ لِتَعْلِيمِهِمْ.
\par 15 عَيْنَايَ دَائِماً إِلَى الرَّبِّ لأَنَّهُ هُوَ يُخْرِجُ رِجْلَيَّ مِنَ الشَّبَكَةِ.
\par 16 اِلْتَفِتْ إِلَيَّ وَارْحَمْنِي لأَنِّي وَحْدٌ وَمِسْكِينٌ أَنَا.
\par 17 اُفْرُجْ ضِيقَاتِ قَلْبِي. مِنْ شَدَائِدِي أَخْرِجْنِي.
\par 18 انْظُرْ إِلَى ذُلِّي وَتَعَبِي وَاغْفِرْ جَمِيعَ خَطَايَايَ.
\par 19 انْظُرْ إِلَى أَعْدَائِي لأَنَّهُمْ قَدْ كَثُرُوا. وَبُغْضاً ظُلْماً أَبْغَضُونِي.
\par 20 احْفَظْ نَفْسِي وَأَنْقِذْنِي. لاَ أُخْزَى لأَنِّي عَلَيْكَ تَوَكَّلْتُ.
\par 21 يَحْفَظُنِي الْكَمَالُ وَالاِسْتِقَامَةُ لأَنِّي انْتَظَرْتُكَ.
\par 22 يَا اللهُ افْدِ إِسْرَائِيلَ مِنْ كُلِّ ضِيقَاتِهِ.

\chapter{26}

\par 1 لِدَاوُدَ اِقْضِ لِي يَا رَبُّ لأَنِّي بِكَمَالِي سَلَكْتُ وَعَلَى الرَّبِّ تَوَكَّلْتُ بِلاَ تَقَلْقُلٍ.
\par 2 جَرِّبْنِي يَا رَبُّ وَامْتَحِنِّي. صَفِّ كُلْيَتَيَّ وَقَلْبِي.
\par 3 لأَنَّ رَحْمَتَكَ أَمَامَ عَيْنِي. وَقَدْ سَلَكْتُ بِحَقِّكَ.
\par 4 لَمْ أَجْلِسْ مَعَ أُنَاسِ السُّوءِ وَمَعَ الْمَاكِرِينَ لاَ أَدْخُلُ.
\par 5 أَبْغَضْتُ جَمَاعَةَ الأَثَمَةِ وَمَعَ الأَشْرَارِ لاَ أَجْلِسُ.
\par 6 أَغْسِلُ يَدَيَّ فِي النَّقَاوَةِ فَأَطُوفُ بِمَذْبَحِكَ يَا رَبُّ
\par 7 لِأُسَمِّعَ بِصَوْتِ الْحَمْدِ وَأُحَدِّثَ بِجَمِيعِ عَجَائِبِكَ.
\par 8 يَا رَبُّ أَحْبَبْتُ مَحَلَّ بَيْتِكَ وَمَوْضِعَ مَسْكَنِ مَجْدِكَ.
\par 9 لاَ تَجْمَعْ مَعَ الْخُطَاةِ نَفْسِي وَلاَ مَعَ رِجَالِ الدِّمَاءِ حَيَاتِي.
\par 10 الَّذِينَ فِي أَيْدِيهِمْ رَذِيلَةٌ وَيَمِينُهُمْ مَلآنَةٌ رَشْوَةً.
\par 11 أَمَّا أَنَا فَبِكَمَالِي أَسْلُكُ. افْدِنِي وَارْحَمْنِي.
\par 12 رِجْلِي وَاقِفَةٌ عَلَى سَهْلٍ. فِي الْجَمَاعَاتِ أُبَارِكُ الرَّبَّ.

\chapter{27}

\par 1 لِدَاوُدَ اَلرَّبُّ نُورِي وَخَلاَصِي مِمَّنْ أَخَافُ؟ الرَّبُّ حِصْنُ حَيَاتِي مِمَّنْ أَرْتَعِبُ؟
\par 2 عِنْدَ مَا اقْتَرَبَ إِلَيَّ الأَشْرَارُ لِيَأْكُلُوا لَحْمِي مُضَايِقِيَّ وَأَعْدَائِي عَثَرُوا وَسَقَطُوا.
\par 3 إِنْ نَزَلَ عَلَيَّ جَيْشٌ لاَ يَخَافُ قَلْبِي. إِنْ قَامَتْ عَلَيَّ حَرْبٌ فَفِي ذَلِكَ أَنَا مُطْمَئِنٌّ.
\par 4 وَاحِدَةً سَأَلْتُ مِنَ الرَّبِّ وَإِيَّاهَا أَلْتَمِسُ: أَنْ أَسْكُنَ فِي بَيْتِ الرَّبِّ كُلَّ أَيَّامِ حَيَاتِي لِكَيْ أَنْظُرَ إِلَى جَمَالِ الرَّبِّ وَأَتَفَرَّسَ فِي هَيْكَلِهِ.
\par 5 لأَنَّهُ يُخَبِّئُنِي فِي مَظَلَّتِهِ فِي يَوْمِ الشَّرِّ. يَسْتُرُنِي بِسِتْرِ خَيْمَتِهِ. عَلَى صَخْرَةٍ يَرْفَعُنِي.
\par 6 وَالآنَ يَرْتَفِعُ رَأْسِي عَلَى أَعْدَائِي حَوْلِي فَأَذْبَحُ فِي خَيْمَتِهِ ذَبَائِحَ الْهُتَافِ. أُغَنِّي وَأُرَنِّمُ لِلرَّبِّ.
\par 7 اِسْتَمِعْ يَا رَبُّ. بِصَوْتِي أَدْعُو فَارْحَمْنِي وَاسْتَجِبْ لِي.
\par 8 لَكَ قَالَ قَلْبِي: [قُلْتَ اطْلُبُوا وَجْهِي. وَجْهَكَ يَا رَبُّ أَطْلُبُ].
\par 9 لاَ تَحْجُبْ وَجْهَكَ عَنِّي. لاَ تُخَيِّبْ بِسَخَطٍ عَبْدَكَ. قَدْ كُنْتَ عَوْنِي فَلاَ تَرْفُضْنِي وَلاَ تَتْرُكْنِي يَا إِلَهَ خَلاَصِي.
\par 10 إِنَّ أَبِي وَأُمِّي قَدْ تَرَكَانِي وَالرَّبُّ يَضُمُّنِي.
\par 11 عَلِّمْنِي يَا رَبُّ طَرِيقَكَ وَاهْدِنِي فِي سَبِيلٍ مُسْتَقِيمٍ بِسَبَبِ أَعْدَائِي.
\par 12 لاَ تُسَلِّمْنِي إِلَى مَرَامِ مُضَايِقِيَّ لأَنَّهُ قَدْ قَامَ عَلَيَّ شُهُودُ زُورٍ وَنَافِثُ ظُلْمٍ.
\par 13 لَوْلاَ أَنَّنِي آمَنْتُ بِأَنْ أَرَى جُودَ الرَّبِّ فِي أَرْضِ الأَحْيَاءِ
\par 14 انْتَظِرِ الرَّبَّ. لِيَتَشَدَّدْ وَلْيَتَشَجَّعْ قَلْبُكَ وَانْتَظِرِ الرَّبَّ.

\chapter{28}

\par 1 لِدَاوُدَ إِلَيْكَ يَا رَبُّ أَصْرُخُ. يَا صَخْرَتِي لاَ تَتَصَامَمْ مِنْ جِهَتِي لِئَلاَّ تَسْكُتَ عَنِّي فَأُشْبِهَ الْهَابِطِينَ فِي الْجُبِّ.
\par 2 اسْتَمِعْ صَوْتَ تَضَرُّعِي إِذْ أَسْتَغِيثُ بِكَ وَأَرْفَعُ يَدَيَّ إِلَى مِحْرَابِ قُدْسِكَ.
\par 3 لاَ تَجْذِبْنِي مَعَ الأَشْرَارِ وَمَعَ فَعَلَةِ الإِثْمِ الْمُخَاطِبِينَ أَصْحَابَهُمْ بِالسَّلاَمِ وَالشَّرُّ فِي قُلُوبِهِمْ.
\par 4 أَعْطِهِمْ حَسَبَ فِعْلِهِمْ وَحَسَبَ شَرِّ أَعْمَالِهِمْ. حَسَبَ صُنْعِ أَيْدِيهِمْ أَعْطِهِمْ. رُدَّ عَلَيْهِمْ مُعَامَلَتَهُمْ.
\par 5 لأَنَّهُمْ لَمْ يَنْتَبِهُوا إِلَى أَفْعَالِ الرَّبِّ وَلاَ إِلَى أَعْمَالِ يَدَيْهِ يَهْدِمُهُمْ وَلاَ يَبْنِيهِمْ.
\par 6 مُبَارَكٌ الرَّبُّ لأَنَّهُ سَمِعَ صَوْتَ تَضَرُّعِي.
\par 7 الرَّبُّ عِزِّي وَتُرْسِي. عَلَيْهِ اتَّكَلَ قَلْبِي فَانْتَصَرْتُ. وَيَبْتَهِجُ قَلْبِي وَبِأُغْنِيَتِي أَحْمَدُهُ.
\par 8 الرَّبُّ عِزٌّ لَهُمْ وَحِصْنُ خَلاَصِ مَسِيحِهِ هُوَ.
\par 9 خَلِّصْ شَعْبَكَ وَبَارِكْ مِيرَاثَكَ وَارْعَهُم وَاحْمِلْهُم إِلَى الأَبَدِ.

\chapter{29}

\par 1 مَزْمُورٌ لِدَاوُدَ قَدِّمُوا لِلرَّبِّ يَا أَبْنَاءَ اللهِ قَدِّمُوا لِلرَّبِّ مَجْداً وَعِزّاً.
\par 2 قَدِّمُوا لِلرَّبِّ مَجْدَ اسْمِهِ. اسْجُدُوا لِلرَّبِّ فِي زِينَةٍ مُقَدَّسَةٍ.
\par 3 صَوْتُ الرَّبِّ عَلَى الْمِيَاهِ. إِلَهُ الْمَجْدِ أَرْعَدَ. الرَّبُّ فَوْقَ الْمِيَاهِ الْكَثِيرَةِ.
\par 4 صَوْتُ الرَّبِّ بِالْقُوَّةِ. صَوْتُ الرَّبِّ بِالْجَلاَلِ.
\par 5 صَوْتُ الرَّبِّ مُكَسِّرُ الأَرْزِ وَيُكَسِّرُ الرَّبُّ أَرْزَ لُبْنَانَ
\par 6 وَيُمْرِحُهَا مِثْلَ عِجْلٍ. لُبْنَانَ وَسِرْيُونَ مِثْلَ فَرِيرِ الْبَقَرِ الْوَحْشِيِّ.
\par 7 صَوْتُ الرَّبِّ يَقْدَحُ لُهُبَ نَارٍ.
\par 8 صَوْتُ الرَّبِّ يُزَلْزِلُ الْبَرِّيَّةَ. يُزَلْزِلُ الرَّبُّ بَرِيَّةَ قَادِشَ.
\par 9 صَوْتُ الرَّبِّ يُوَلِّدُ الإِيَّلَ وَيَكْشِفُ الْوُعُورَ وَفِي هَيْكَلِهِ الْكُلُّ قَائِلٌ: [مَجْدٌ].
\par 10 الرَّبُّ بِالطُّوفَانِ جَلَسَ وَيَجْلِسُ الرَّبُّ مَلِكاً إِلَى الأَبَدِ.
\par 11 الرَّبُّ يُعْطِي عِزّاً لِشَعْبِهِ. الرَّبُّ يُبَارِكُ شَعْبَهُ بِالسَّلاَمِ.

\chapter{30}

\par 1 مَزْمُورٌ أُغْنِيَةُ تَدْشِينِ الْبَيْتِ. لِدَاوُدَ أُعَظِّمُكَ يَا رَبُّ لأَنَّكَ نَشَلْتَنِي وَلَمْ تُشْمِتْ بِي أَعْدَائِي.
\par 2 يَا رَبُّ إِلَهِي اسْتَغَثْتُ بِكَ فَشَفَيْتَنِي.
\par 3 يَا رَبُّ أَصْعَدْتَ مِنَ الْهَاوِيَةِ نَفْسِي. أَحْيَيْتَنِي مِنْ بَيْنِ الْهَابِطِينَ فِي الْجُبِّ.
\par 4 رَنِّمُوا لِلرَّبِّ يَا أَتْقِيَاءَهُ وَاحْمَدُوا ذِكْرَ قُدْسِهِ.
\par 5 لأَنَّ لِلَحْظَةٍ غَضَبَهُ. حَيَاةٌ فِي رِضَاهُ. عِنْدَ الْمَسَاءِ يَبِيتُ الْبُكَاءُ وَفِي الصَّبَاحِ تَرَنُّمٌ.
\par 6 وَأَنَا قُلْتُ فِي طُمَأْنِينَتِي: [لاَ أَتَزَعْزَعُ إِلَى الأَبَدِ].
\par 7 يَا رَبُّ بِرِضَاكَ ثَبَّتَّ لِجَبَلِي عِزّاً. حَجَبْتَ وَجْهَكَ فَصِرْتُ مُرْتَاعاً.
\par 8 إِلَيْكَ يَا رَبُّ أَصْرُخُ وَإِلَى السَّيِّدِ أَتَضَرَّعُ.
\par 9 مَا الْفَائِدَةُ مِنْ دَمِي إِذَا نَزَلْتُ إِلَى الْحُفْرَةِ؟ هَلْ يَحْمَدُكَ التُّرَابُ؟ هَلْ يُخْبِرُ بِحَقِّكَ؟
\par 10 اسْتَمِعْ يَا رَبُّ وَارْحَمْنِي. يَا رَبُّ كُنْ مُعِيناً لِي.
\par 11 حَوَّلْتَ نَوْحِي إِلَى رَقْصٍ لِي. حَلَلْتَ مِسْحِي وَمَنْطَقْتَنِي فَرَحاً
\par 12 لِكَيْ تَتَرَنَّمَ لَكَ رُوحِي وَلاَ تَسْكُتَ. يَا رَبُّ إِلَهِي إِلَى الأَبَدِ أَحْمَدُكَ.

\chapter{31}

\par 1 لإِمَامِ الْمُغَنِّينَ. مَزْمُورٌ لِدَاوُدَ عَلَيْكَ يَا رَبُّ تَوَكَّلْتُ. لاَ تَدَعْنِي أَخْزَى مَدَى الدَّهْرِ. بِعَدْلِكَ نَجِّنِي.
\par 2 أَمِلْ إِلَيَّ أُذْنَكَ. سَرِيعاً أَنْقِذْنِي. كُنْ لِي صَخْرَةَ حِصْنٍ بَيْتَ مَلْجَأٍ لِتَخْلِيصِي.
\par 3 لأَنَّ صَخْرَتِي وَمَعْقِلِي أَنْتَ. مِنْ أَجْلِ اسْمِكَ تَهْدِينِي وَتَقُودُنِي.
\par 4 أَخْرِجْنِي مِنَ الشَّبَكَةِ الَّتِي خَبَّأُوهَا لِي لأَنَّكَ أَنْتَ حِصْنِي.
\par 5 فِي يَدِكَ أَسْتَوْدِعُ رُوحِي. فَدَيْتَنِي يَا رَبُّ إِلَهَ الْحَقِّ.
\par 6 أَبْغَضْتُ الَّذِينَ يُرَاعُونَ أَبَاطِيلَ كَاذِبَةً. أَمَّا أَنَا فَعَلَى الرَّبِّ تَوَكَّلْتُ.
\par 7 أَبْتَهِجُ وَأَفْرَحُ بِرَحْمَتِكَ لأَنَّكَ نَظَرْتَ إِلَى مَذَلَّتِي وَعَرَفْتَ فِي الشَّدَائِدِ نَفْسِي
\par 8 وَلَمْ تَحْبِسْنِي فِي يَدِ الْعَدُوِّ بَلْ أَقَمْتَ فِي الرَُّحْبِ رِجْلِي.
\par 9 اِرْحَمْنِي يَا رَبُّ لأَنِّي فِي ضِيقٍ. خَسَفَتْ مِنَ الْغَمِّ عَيْنِي. نَفْسِي وَبَطْنِي.
\par 10 لأَنَّ حَيَاتِي قَدْ فَنِيَتْ بِالْحُزْنِ وَسِنِينِي بِالتَّنَهُّدِ. ضَعُفَتْ بِشَقَاوَتِي قُوَّتِي وَبَلِيَتْ عِظَامِي.
\par 11 عِنْدَ كُلِّ أَعْدَائِي صِرْتُ عَاراً وَعِنْدَ جِيرَانِي بِالْكُلِّيَّةِ وَرُعْباً لِمَعَارِفِي. الَّذِينَ رَأُونِي خَارِجاً هَرَبُوا عَنِّي.
\par 12 نُسِيتُ مِنَ الْقَلْبِ مِثْلَ الْمَيْتِ. صِرْتُ مِثْلَ إِنَاءٍ مُتْلَفٍ.
\par 13 لأَنِّي سَمِعْتُ مَذَمَّةً مِنْ كَثِيرِينَ. الْخَوْفُ مُسْتَدِيرٌ بِي بِمُؤَامَرَتِهِمْ مَعاً عَلَيَّ. تَفَكَّرُوا فِي أَخْذِ نَفْسِي.
\par 14 أَمَّا أَنَا فَعَلَيْكَ تَوَكَّلْتُ يَا رَبُّ. قُلْتُ: [إِلَهِي أَنْتَ].
\par 15 فِي يَدِكَ آجَالِي. نَجِّنِي مِنْ يَدِ أَعْدَائِي وَمِنَ الَّذِينَ يَطْرُدُونَنِي.
\par 16 أَضِئْ بِوَجْهِكَ عَلَى عَبْدِكَ. خَلِّصْنِي بِرَحْمَتِكَ.
\par 17 يَا رَبُّ لاَ تَدَعْنِي أَخْزَى لأَنِّي دَعَوْتُكَ. لِيَخْزَ الأَشْرَارُ. لِيَسْكُتُوا فِي الْهَاوِيَةِ.
\par 18 لِتُبْكَمْ شِفَاهُ الْكَذِبِ الْمُتَكَلِّمَةُ عَلَى الصِّدِّيقِ بِوَقَاحَةٍ بِكِبْرِيَاءَ وَاسْتِهَانَةٍ.
\par 19 مَا أَعْظَمَ جُودَكَ الَّذِي ذَخَرْتَهُ لِخَائِفِيكَ وَفَعَلْتَهُ لِلْمُتَّكِلِينَ عَلَيْكَ تُجَاهَ بَنِي الْبَشَرِ.
\par 20 تَسْتُرُهُمْ بِسِتْرِ وَجْهِكَ مِنْ مَكَايِدِ النَّاسِ. تُخْفِيهِمْ فِي مَظَلَّةٍ مِنْ مُخَاصَمَةِ الأَلْسُنِ.
\par 21 مُبَارَكٌ الرَّبُّ لأَنَّهُ قَدْ جَعَلَ عَجَباً رَحْمَتَهُ لِي فِي مَدِينَةٍ مُحَصَّنَةٍ.
\par 22 وَأَنَا قُلْتُ فِي حَيْرَتِي: [إِنِّي قَدِ انْقَطَعْتُ مِنْ قُدَّامِ عَيْنَيْكَ]. وَلَكِنَّكَ سَمِعْتَ صَوْتَ تَضَرُّعِي إِذْ صَرَخْتُ إِلَيْكَ.
\par 23 أَحِبُّوا الرَّبَّ يَا جَمِيعَ أَتْقِيَائِهِ. الرَّبُّ حَافِظُ الأَمَانَةِ وَمُجَازٍ بِكَِثْرَةٍ الْعَامِلَ بِالْكِبْرِيَاءِ.
\par 24 لِتَتَشَدَّدْ وَلْتَتَشَجَّعْ قُلُوبُكُمْ يَا جَمِيعَ الْمُنْتَظِرِينَ الرَّبَّ

\chapter{32}

\par 1 لِدَاوُدَ. قَصِيدَةٌ طُوبَى لِلَّذِي غُفِرَ إِثْمُهُ وَسُتِرَتْ خَطِيَّتُهُ.
\par 2 طُوبَى لِرَجُلٍ لاَ يَحْسِبُ لَهُ الرَّبُّ خَطِيَّةً وَلاَ فِي رُوحِهِ غِشٌّ.
\par 3 لَمَّا سَكَتُّ بَلِيَتْ عِظَامِي مِنْ زَفِيرِي الْيَوْمَ كُلَّهُ
\par 4 لأَنَّ يَدَكَ ثَقُلَتْ عَلَيَّ نَهَاراً وَلَيْلاً. تَحَوَّلَتْ رُطُوبَتِي إِلَى يُبُوسَةِ الْقَيْظِ. سِلاَهْ.
\par 5 أَعْتَرِفُ لَكَ بِخَطِيَّتِي وَلاَ أَكْتُمُ إِثْمِي. قُلْتُ: [أَعْتَرِفُ لِلرَّبِّ بِذَنْبِي] وَأَنْتَ رَفَعْتَ أَثَامَ خَطِيَّتِي. سِلاَهْ.
\par 6 لِهَذَا يُصَلِّي لَكَ كُلُّ تَقِيٍّ فِي وَقْتٍ يَجِدُكَ فِيهِ. عِنْدَ غَمَارَةِ الْمِيَاهِ الْكَثِيرَةِ إِيَّاهُ لاَ تُصِيبُ.
\par 7 أَنْتَ سِتْرٌ لِي. مِنَ الضِّيقِ تَحْفَظُنِي. بِتَرَنُّمِ النَّجَاةِ تَكْتَنِفُنِي. سِلاَهْ.
\par 8 أُعَلِّمُكَ وَأُرْشِدُكَ الطَّرِيقَ الَّتِي تَسْلُكُهَا. أَنْصَحُكَ. عَيْنِي عَلَيْكَ.
\par 9 لاَ تَكُونُوا كَفَرَسٍ أَوْ بَغْلٍ بِلاَ فَهْمٍ. بِلِجَامٍ وَزِمَامٍ زِينَتِهِ يُكَمُّ لِئَلاَّ يَدْنُوَ إِلَيْكَ.
\par 10 كَثِيرَةٌ هِيَ نَكَبَاتُ الشِّرِّيرِ أَمَّا الْمُتَوَكِّلُ عَلَى الرَّبِّ فَالرَّحْمَةُ تُحِيطُ بِهِ.
\par 11 افْرَحُوا بِالرَّبِّ وَابْتَهِجُوا يَا أَيُّهَا الصِّدِّيقُونَ وَاهْتِفُوا يَا جَمِيعَ الْمُسْتَقِيمِي الْقُلُوبِ.

\chapter{33}

\par 1 اِهْتِفُوا أَيُّهَا الصِّدِّيقُونَ بِالرَّبِّ. بِالْمُسْتَقِيمِينَ يَلِيقُ التَّسْبِيحُ.
\par 2 احْمَدُوا الرَّبَّ بِالْعُودِ. بِرَبَابَةٍ ذَاتِ عَشَرَةِ أَوْتَارٍ رَنِّمُوا لَهُ.
\par 3 غَنُّوا لَهُ أُغْنِيَةً جَدِيدَةً. أَحْسِنُوا الْعَزْفَ بِهُتَافٍ.
\par 4 لأَنَّ كَلِمَةَ الرَّبِّ مُسْتَقِيمَةٌ وَكُلَّ صُنْعِهِ بِالأَمَانَةِ.
\par 5 يُحِبُّ الْبِرَّ وَالْعَدْلَ. امْتَلَأَتِ الأَرْضُ مِنْ رَحْمَةِ الرَّبِّ.
\par 6 بِكَلِمَةِ الرَّبِّ صُنِعَتِ السَّمَاوَاتُ وَبِنَسَمَةِ فَمِهِ كُلُّ جُنُودِهَا.
\par 7 يَجْمَعُ كَنَدٍّ أَمْوَاهَ الْيَمِّ. يَجْعَلُ اللُّجَجَ فِي أَهْرَاءٍ.
\par 8 لِتَخْشَ الرَّبَّ كُلُّ الأَرْضِ وَمِنْهُ لِيَخَفْ كُلُّ سُكَّانِ الْمَسْكُونَةِ.
\par 9 لأَنَّهُ قَالَ فَكَانَ. هُوَ أَمَرَ فَصَارَ.
\par 10 الرَّبُّ أَبْطَلَ مُؤَامَرَةَ الأُمَمِ. لاَشَى أَفْكَارَ الشُّعُوبِ.
\par 11 أَمَّا مُؤَامَرَةُ الرَّبِّ فَإِلَى الأَبَدِ تَثْبُتُ. أَفْكَارُ قَلْبِهِ إِلَى دَوْرٍ فَدَوْرٍ.
\par 12 طُوبَى لِلأُمَّةِ الَّتِي الرَّبُّ إِلَهُهَا الشَّعْبِ الَّذِي اخْتَارَهُ مِيرَاثاً لِنَفْسِهِ.
\par 13 مِنَ السَّمَاوَاتِ نَظَرَ الرَّبُّ. رَأَى جَمِيعَ بَنِي الْبَشَرِ.
\par 14 مِنْ مَكَانِ سُكْنَاهُ تَطَلَّعَ إِلَى جَمِيعِ سُكَّانِ الأَرْضِ.
\par 15 الْمُصَوِّرُ قُلُوبَهُمْ جَمِيعاً الْمُنْتَبِهُ إِلَى كُلِّ أَعْمَالِهِمْ.
\par 16 لَنْ يَخْلُصَ الْمَلِكُ بِكَِثْرَةِ الْجَيْشِ. الْجَبَّارُ لاَ يُنْقَذُ بِعِظَمِ الْقُوَّةِ.
\par 17 بَاطِلٌ هُوَ الْفَرَسُ لأَجْلِ الْخَلاَصِ وَبِشِدَّةِ قُوَّتِهِ لاَ يُنَجِّي.
\par 18 هُوَذَا عَيْنُ الرَّبِّ عَلَى خَائِفِيهِ الرَّاجِينَ رَحْمَتَهُ
\par 19 لِيُنَجِّيَ مِنَ الْمَوْتِ أَنْفُسَهُمْ وَلِيَسْتَحْيِيَهُمْ فِي الْجُوعِ.
\par 20 أَنْفُسُنَا انْتَظَرَتِ الرَّبَّ. مَعُونَتُنَا وَتُرْسُنَا هُوَ.
\par 21 لأَنَّهُ بِهِ تَفْرَحُ قُلُوبُنَا لأَنَّنَا عَلَى اسْمِهِ الْقُدُّوسِ اتَّكَلْنَا.
\par 22 لِتَكُنْ يَا رَبُّ رَحْمَتُكَ عَلَيْنَا حَسْبَمَا انْتَظَرْنَاكَ.

\chapter{34}

\par 1 لِدَاوُدَ عِنْدَمَا غَيَّرَ عَقْلَهُ قُدَّامَ أَبِيمَالِكَ فَطَرَدَهُ فَانْطَلَقَ أُبَارِكُ الرَّبَّ فِي كُلِّ حِينٍ. دَائِماً تَسْبِيحُهُ فِي فَمِي.
\par 2 بِالرَّبِّ تَفْتَخِرُ نَفْسِي. يَسْمَعُ الْوُدَعَاءُ فَيَفْرَحُونَ.
\par 3 عَظِّمُوا الرَّبَّ مَعِي وَلْنُعَلِّ اسْمَهُ مَعاً.
\par 4 طَلَبْتُ إِلَى الرَّبِّ فَاسْتَجَابَ لِي وَمِنْ كُلِّ مَخَاوِفِي أَنْقَذَنِي.
\par 5 نَظَرُوا إِلَيْهِ وَاسْتَنَارُوا وَوُجُوهُهُمْ لَمْ تَخْجَلْ.
\par 6 هَذَا الْمِسْكِينُ صَرَخَ وَالرَّبُّ اسْتَمَعَهُ وَمِنْ كُلِّ ضِيقَاتِهِ خَلَّصَهُ.
\par 7 مَلاَكُ الرَّبِّ حَالٌّ حَوْلَ خَائِفِيهِ وَيُنَجِّيهِمْ.
\par 8 ذُوقُوا وَانْظُرُوا مَا أَطْيَبَ الرَّبَّ! طُوبَى لِلرَّجُلِ الْمُتَوَكِّلِ عَلَيْهِ.
\par 9 اتَّقُوا الرَّبَّ يَا قِدِّيسِيهِ لأَنَّهُ لَيْسَ عَوَزٌ لِمُتَّقِيهِ.
\par 10 الأَشْبَالُ احْتَاجَتْ وَجَاعَتْ وَأَمَّا طَالِبُو الرَّبِّ فَلاَ يُعْوِزُهُمْ شَيْءٌ مِنَ الْخَيْرِ.
\par 11 هَلُمَّ أَيُّهَا الْبَنُونَ اسْتَمِعُوا إِلَيَّ فَأُعَلِّمَكُمْ مَخَافَةَ الرَّبِّ.
\par 12 مَنْ هُوَ الإِنْسَانُ الَّذِي يَهْوَى الْحَيَاةَ وَيُحِبُّ كَثْرَةَ الأَيَّامِ لِيَرَى خَيْراً؟
\par 13 صُنْ لِسَانَكَ عَنِ الشَّرِّ وَشَفَتَيْكَ عَنِ التَّكَلُّمِ بِالْغِشِّ.
\par 14 حِدْ عَنِ الشَّرِّ وَاصْنَعِ الْخَيْرَ. اطْلُبِ السَّلاَمَةَ وَاسْعَ وَرَاءَهَا.
\par 15 عَيْنَا الرَّبِّ نَحْوَ الصِّدِّيقِينَ وَأُذُنَاهُ إِلَى صُرَاخِهِمْ.
\par 16 وَجْهُ الرَّبِّ ضِدُّ عَامِلِي الشَّرِّ لِيَقْطَعَ مِنَ الأَرْضِ ذِكْرَهُمْ.
\par 17 أُولَئِكَ صَرَخُوا وَالرَّبُّ سَمِعَ وَمِنْ كُلِّ شَدَائِدِهِمْ أَنْقَذَهُمْ.
\par 18 قَرِيبٌ هُوَ الرَّبُّ مِنَ الْمُنْكَسِرِي الْقُلُوبِ وَيُخَلِّصُ الْمُنْسَحِقِي الرُّوحِ.
\par 19 كَثِيرَةٌ هِيَ بَلاَيَا الصِّدِّيقِ وَمِنْ جَمِيعِهَا يُنَجِّيهِ الرَّبُّ.
\par 20 يَحْفَظُ جَمِيعَ عِظَامِهِ. وَاحِدٌ مِنْهَا لاَ يَنْكَسِرُ.
\par 21 الشَّرُّ يُمِيتُ الشِّرِّيرَ وَمُبْغِضُو الصِّدِّيقِ يُعَاقَبُونَ.
\par 22 الرَّبُّ فَادِي نُفُوسِ عَبِيدِهِ وَكُلُّ مَنِ اتَّكَلَ عَلَيْهِ لاَ يُعَاقَبُ.

\chapter{35}

\par 1 لِدَاوُدَ خَاصِمْ يَا رَبُّ مُخَاصِمِيَّ. قَاتِلْ مُقَاتِلِيَّ.
\par 2 أَمْسِكْ مِجَنّاً وَتُرْساً وَانْهَضْ إِلَى مَعُونَتِي
\par 3 وَأَشْرِعْ رُمْحاً وَصُدَّ تِلْقَاءَ مُطَارِدِيَّ. قُلْ لِنَفْسِي: [خَلاَصُكِ أَنَا].
\par 4 لِيَخْزَ وَلْيَخْجَلِ الَّذِينَ يَطْلُبُونَ نَفْسِي. لِيَرْتَدَّ إِلَى الْوَرَاءِ وَيَخْجَلِ الْمُتَفَكِّرُونَ بِإِسَاءَتِي.
\par 5 لِيَكُونُوا مِثْلَ الْعُصَافَةِ قُدَّامَ الرِّيحِ وَمَلاَكُ الرَّبِّ دَاحِرُهُمْ.
\par 6 لِيَكُنْ طَرِيقُهُمْ ظَلاَماً وَزَلَقاً وَمَلاَكُ الرَّبِّ طَارِدُهُمْ.
\par 7 لأَنَّهُمْ بِلاَ سَبَبٍ أَخْفُوا لِي هُوَّةَ شَبَكَتِهِمْ. بِلاَ سَبَبٍ حَفَرُوا لِنَفْسِي.
\par 8 لِتَأْتِهِ التَّهْلُكَةُ وَهُوَ لاَ يَعْلَمُ وَلْتَنْشَبْ بِهِ الشَّبَكَةُ الَّتِي أَخْفَاهَا وَفِي التَّهْلُكَةِ نَفْسِهَا لِيَقَعْ.
\par 9 أَمَّا نَفْسِي فَتَفْرَحُ بِالرَّبِّ وَتَبْتَهِجُ بِخَلاَصِهِ.
\par 10 جَمِيعُ عِظَامِي تَقُولُ: [يَا رَبُّ مَنْ مِثْلُكَ الْمُنْقِذُ الْمِسْكِينَ مِمَّنْ هُوَ أَقْوَى مِنْهُ وَالْفَقِيرَ وَالْبَائِسَ مِنْ سَالِبِهِ؟].
\par 11 شُهُودُ زُورٍ يَقُومُونَ وَعَمَّا لَمْ أَعْلَمْ يَسْأَلُونَنِي.
\par 12 يُجَازُونَنِي عَنِ الْخَيْرِ شَرّاً ثَكَلاً لِنَفْسِي.
\par 13 أَمَّا أَنَا فَفِي مَرَضِهِمْ كَانَ لِبَاسِي مِسْحاً. أَذْلَلْتُ بِالصَّوْمِ نَفْسِي. وَصَلاَتِي إِلَى حِضْنِي تَرْجِعُ.
\par 14 كَأَنَّهُ قَرِيبٌ كَأَنَّهُ أَخِي كُنْتُ أَتَمَشَّى. كَمَنْ يَنُوحُ عَلَى أُمِّهِ انْحَنَيْتُ حَزِيناً.
\par 15 وَلَكِنَّهُمْ فِي ظَلْعِي فَرِحُوا وَاجْتَمَعُوا. اجْتَمَعُوا عَلَيَّ شَاتِمِينَ وَلَمْ أَعْلَمْ. مَزَّقُوا وَلَمْ يَكُفُّوا.
\par 16 بَيْنَ الْفُجَّارِ الْمُجَّانِ لأَجْلِ كَعْكَةٍ حَرَّقُوا عَلَيَّ أَسْنَانَهُمْ.
\par 17 يَا رَبُّ إِلَى مَتَى تَنْظُرُ؟ اسْتَرِدَّ نَفْسِي مِنْ تَهْلُكَاتِهِمْ وَحِيدَتِي مِنَ الأَشْبَالِ.
\par 18 أَحْمَدُكَ فِي الْجَمَاعَةِ الْكَثِيرَةِ. فِي شَعْبٍ عَظِيمٍ أُسَبِّحُكَ.
\par 19 لاَ يَشْمَتْ بِي الَّذِينَ هُمْ أَعْدَائِي بَاطِلاً وَلاَ يَتَغَامَزْ بِالْعَيْنِ الَّذِينَ يُبْغِضُونَنِي بِلاَ سَبَبٍ.
\par 20 لأَنَّهُمْ لاَ يَتَكَلَّمُونَ بِالسَّلاَمِ وَعَلَى الْهَادِئِينَ فِي الأَرْضِ يَتَفَكَّرُونَ بِكَلاَمِ مَكْرٍ.
\par 21 فَغَرُوا عَلَيَّ أَفْوَاهَهُمْ. قَالُوا: [هَهْ هَهْ! قَدْ رَأَتْ أَعْيُنُنَا].
\par 22 قَدْ رَأَيْتَ يَا رَبُّ. لاَ تَسْكُتْ يَا سَيِّدُ. لاَ تَبْتَعِدْ عَنِّي.
\par 23 اسْتَيْقِظْ وَانْتَبِهْ إِلَى حُكْمِي يَا إِلَهِي وَسَيِّدِي إِلَى دَعْوَايَ.
\par 24 اقْضِ لِي حَسَبَ عَدْلِكَ يَا رَبُّ إِلَهِي فَلاَ يَشْمَتُوا بِي.
\par 25 لاَ يَقُولُوا فِي قُلُوبِهِمْ: [هَهْ! شَهْوَتُنَا]. لاَ يَقُولُوا: [قَدِ ابْتَلَعْنَاهُ!]
\par 26 لِيَخْزَ وَلْيَخْجَلْ مَعاً الْفَرِحُونَ بِمُصِيبَتِي. لِيَلْبِسِ الْخِزْيَ وَالْخَجَلَ الْمُتَعَظِّمُونَ عَلَيَّ.
\par 27 لِيَهْتِفْ وَيَفْرَحِ الْمُبْتَغُونَ حَقِّي وَلْيَقُولُوا دَائِماً: [لِيَتَعَظَّمِ الرَّبُّ الْمَسْرُورُ بِسَلاَمَةِ عَبْدِهِ].
\par 28 وَلِسَانِي يَلْهَجُ بِعَدْلِكَ. الْيَوْمَ كُلَّهُ بِحَمْدِكَ.

\chapter{36}

\par 1 لإِمَامِ الْمُغَنِّينَ. لِعَبْدِ الرَّبِّ دَاوُدَ نَأْمَةُ مَعْصِيَةِ الشِّرِّيرِ فِي دَاخِلِ قَلْبِي أَنْ لَيْسَ خَوْفُ اللهِ أَمَامَ عَيْنَيْهِ.
\par 2 لأَنَّهُ مَلَّقَ نَفْسَهُ لِنَفْسِهِ مِنْ جِهَةِ وِجْدَانِ إِثْمِهِ وَبُغْضِهِ.
\par 3 كَلاَمُ فَمِهِ إِثْمٌ وَغِشٌّ. كَفَّ عَنِ التَّعَقُّلِ عَنْ عَمَلِ الْخَيْرِ.
\par 4 يَتَفَكَّرُ بِالإِثْمِ عَلَى مَضْجَعِهِ. يَقِفُ فِي طَرِيقٍ غَيْرِ صَالِحٍ. لاَ يَرْفُضُ الشَّرَّ.
\par 5 يَا رَبُّ فِي السَّمَاوَاتِ رَحْمَتُكَ. أَمَانَتُكَ إِلَى الْغَمَامِ.
\par 6 عَدْلُكَ مِثْلُ جِبَالِ اللهِ وَأَحْكَامُكَ لُجَّةٌ عَظِيمَةٌ. النَّاسَ وَالْبَهَائِمَ تُخَلِّصُ يَا رَبُّ.
\par 7 مَا أَكْرَمَ رَحْمَتَكَ يَا اللهُ فَبَنُو الْبَشَرِ فِي ظِلِّ جَنَاحَيْكَ يَحْتَمُونَ.
\par 8 يَرْوُونَ مِنْ دَسَمِ بَيْتِكَ وَمِنْ نَهْرِ نِعَمِكَ تَسْقِيهِمْ.
\par 9 لأَنَّ عِنْدَكَ يَنْبُوعَ الْحَيَاةِ. بِنُورِكَ نَرَى نُوراً.
\par 10 أَدِمْ رَحْمَتَكَ لِلَّذِينَ يَعْرِفُونَكَ وَعَدْلَكَ لِلْمُسْتَقِيمِي الْقَلْبِ.
\par 11 لاَ تَأْتِنِي رِجْلُ الْكِبْرِيَاءِ وَيَدُ الأَشْرَارِ لاَ تُزَحْزِحْنِي.
\par 12 هُنَاكَ سَقَطَ فَاعِلُو الإِثْمِ. دُحِرُوا فَلَمْ يَسْتَطِيعُوا الْقِيَامَ.

\chapter{37}

\par 1 لِدَاوُدَ لاَ تَغَرْ مِنَ الأَشْرَارِ وَلاَ تَحْسِدْ عُمَّالَ الإِثْمِ
\par 2 فَإِنَّهُمْ مِثْلَ الْحَشِيشِ سَرِيعاً يُقْطَعُونَ وَمِثْلَ الْعُشْبِ الأَخْضَرِ يَذْبُلُونَ.
\par 3 اتَّكِلْ عَلَى الرَّبِّ وَافْعَلِ الْخَيْرَ. اسْكُنِ الأَرْضَ وَارْعَ الأَمَانَةَ.
\par 4 وَتَلَذَّذْ بِالرَّبِّ فَيُعْطِيَكَ سُؤْلَ قَلْبِكَ.
\par 5 سَلِّمْ لِلرَّبِّ طَرِيقَكَ وَاتَّكِلْ عَلَيْهِ وَهُوَ يُجْرِي
\par 6 وَيُخْرِجُ مِثْلَ النُّورِ بِرَّكَ وَحَقَّكَ مِثْلَ الظَّهِيرَةِ.
\par 7 انْتَظِرِ الرَّبَّ وَاصْبِرْ لَهُ وَلاَ تَغَرْ مِنَ الَّذِي يَنْجَحُ فِي طَرِيقِهِ مِنَ الرَّجُلِ الْمُجْرِي مَكَايِدَ.
\par 8 كُفَّ عَنِ الْغَضَبِ وَاتْرُكِ السَّخَطَ وَلاَ تَغَرْ لِفِعْلِ الشَّرِّ
\par 9 لأَنَّ عَامِلِي الشَّرِّ يُقْطَعُونَ وَالَّذِينَ يَنْتَظِرُونَ الرَّبَّ هُمْ يَرِثُونَ الأَرْضَ.
\par 10 بَعْدَ قَلِيلٍ لاَ يَكُونُ الشِّرِّيرُ. تَطَّلِعُ فِي مَكَانِهِ فَلاَ يَكُونُ.
\par 11 أَمَّا الْوُدَعَاءُ فَيَرِثُونَ الأَرْضَ وَيَتَلَذَّذُونَ فِي كَثْرَةِ السَّلاَمَةِ.
\par 12 الشِّرِّيرُ يَتَفَكَّرُ ضِدَّ الصِّدِّيقِ وَيُحَرِّقُ عَلَيْهِ أَسْنَانَهُ.
\par 13 الرَّبُّ يَضْحَكُ بِهِ لأَنَّهُ رَأَى أَنَّ يَوْمَهُ آتٍ!
\par 14 الأَشْرَارُ قَدْ سَلُّوا السَّيْفَ وَمَدُّوا قَوْسَهُمْ لِرَمْيِ الْمِسْكِينِ وَالْفَقِيرِ لِقَتْلِ الْمُسْتَقِيمِ طَرِيقُهُمْ.
\par 15 سَيْفُهُمْ يَدْخُلُ فِي قَلْبِهِمْ وَقِسِيُّهُمْ تَنْكَسِرُ.
\par 16 اَلْقَلِيلُ الَّذِي لِلصِّدِّيقِ خَيْرٌ مِنْ ثَرْوَةِ أَشْرَارٍ كَثِيرِينَ.
\par 17 لأَنَّ سَوَاعِدَ الأَشْرَارِ تَنْكَسِرُ وَعَاضِدُ الصِّدِّيقِينَ الرَّبُّ.
\par 18 الرَّبُّ عَارِفٌ أَيَّامَ الْكَمَلَةِ وَمِيرَاثُهُمْ إِلَى الأَبَدِ يَكُونُ.
\par 19 لاَ يُخْزَوْنَ فِي زَمَنِ السُّوءِ وَفِي أَيَّامِ الْجُوعِ يَشْبَعُونَ.
\par 20 لأَنَّ الأَشْرَارَ يَهْلِكُونَ وَأَعْدَاءُ الرَّبِّ كَبَهَاءِ الْمَرَاعِي. فَنُوا. كَالدُّخَانِ فَنُوا.
\par 21 الشِّرِّيرُ يَسْتَقْرِضُ وَلاَ يَفِي أَمَّا الصِّدِّيقُ فَيَتَرَأَّفُ وَيُعْطِي.
\par 22 لأَنَّ الْمُبَارَكِينَ مِنْهُ يَرِثُونَ الأَرْضَ وَالْمَلْعُونِينَ مِنْهُ يُقْطَعُونَ.
\par 23 مِنْ قِبَلِ الرَّبِّ تَتَثَبَّتُ خَطَوَاتُ الإِنْسَانِ وَفِي طَرِيقِهِ يُسَرُّ.
\par 24 إِذَا سَقَطَ لاَ يَنْطَرِحُ لأَنَّ الرَّبَّ مُسْنِدٌ يَدَهُ.
\par 25 أَيْضاً كُنْتُ فَتىً وَقَدْ شِخْتُ وَلَمْ أَرَ صِدِّيقاً تُخُلِّيَ عَنْهُ وَلاَ ذُرِّيَّةً لَهُ تَلْتَمِسُ خُبْزاً.
\par 26 الْيَوْمَ كُلَّهُ يَتَرَأَّفُ وَيُقْرِضُ وَنَسْلُهُ لِلْبَرَكَةِ.
\par 27 حِدْ عَنِ الشَّرِّ وَافْعَلِ الْخَيْرَ وَاسْكُنْ إِلَى الأَبَدِ.
\par 28 لأَنَّ الرَّبَّ يُحِبُّ الْحَقَّ وَلاَ يَتَخَلَّى عَنْ أَتْقِيَائِهِ. إِلَى الأَبَدِ يُحْفَظُونَ. أَمَّا نَسْلُ الأَشْرَارِ فَيَنْقَطِعُ.
\par 29 الصِّدِّيقُونَ يَرِثُونَ الأَرْضَ وَيَسْكُنُونَهَا إِلَى الأَبَدِ.
\par 30 فَمُ الصِّدِّيقِ يَلْهَجُ بِالْحِكْمَةِ وَلِسَانُهُ يَنْطِقُ بِالْحَقِّ.
\par 31 شَرِيعَةُ إِلَهِهِ فِي قَلْبِهِ. لاَ تَتَقَلْقَلُ خَطَوَاتُهُ.
\par 32 الشِّرِّيرُ يُرَاقِبُ الصِّدِّيقَ مُحَاوِلاً أَنْ يُمِيتَهُ.
\par 33 الرَّبُّ لاَ يَتْرُكُهُ فِي يَدِهِ وَلاَ يَحْكُمُ عَلَيْهِ عِنْدَ مُحَاكَمَتِهِ.
\par 34 انْتَظِرِ الرَّبَّ وَاحْفَظْ طَرِيقَهُ فَيَرْفَعَكَ لِتَرِثَ الأَرْضَ. إِلَى انْقِرَاضِ الأَشْرَارِ تَنْظُرُ.
\par 35 قَدْ رَأَيْتُ الشِّرِّيرَ عَاتِياً وَارِفاً مِثْلَ شَجَرَةٍ شَارِقَةٍ نَاضِرَةٍ.
\par 36 عَبَرَ فَإِذَا هُوَ لَيْسَ بِمَوْجُودٍ وَالْتَمَسْتُهُ فَلَمْ يُوجَدْ.
\par 37 لاَحِظِ الْكَامِلَ وَانْظُرِ الْمُسْتَقِيمَ فَإِنَّ الْعَقِبَ لإِنْسَانِ السَّلاَمَةِ.
\par 38 أَمَّا الأَشْرَارُ فَيُبَادُونَ جَمِيعاً. عَقِبُ الأَشْرَارِ يَنْقَطِعُ.
\par 39 أَمَّا خَلاَصُ الصِّدِّيقِينَ فَمِنْ قِبَلِ الرَّبِّ حِصْنِهُمْ فِي زَمَانِ الضِّيقِ.
\par 40 وَيُعِينُهُمُ الرَّبُّ وَيُنَجِّيهِمْ. يُنْقِذُهُمْ مِنَ الأَشْرَارِ وَيُخَلِّصُهُمْ لأَنَّهُمُ احْتَمُوا بِهِ.

\chapter{38}

\par 1 مَزْمُورٌ لِدَاوُدَ لِلتَّذْكِيرِ يَا رَبُّ لاَ تُوَبِّخْنِي بِسَخَطِكَ وَلاَ تُؤَدِّبْنِي بِغَيْظِكَ
\par 2 لأَنَّ سِهَامَكَ قَدِ انْتَشَبَتْ فِيَّ وَنَزَلَتْ عَلَيَّ يَدُكَ.
\par 3 لَيْسَتْ فِي جَسَدِي صِحَّةٌ مِنْ جِهَةِ غَضَبِكَ. لَيْسَتْ فِي عِظَامِي سَلاَمَةٌ مِنْ جِهَةِ خَطِيَّتِي.
\par 4 لأَنَّ آثَامِي قَدْ طَمَتْ فَوْقَ رَأْسِي. كَحِمْلٍ ثَقِيلٍ أَثْقَلَ مِمَّا أَحْتَمِلُ.
\par 5 قَدْ أَنْتَنَتْ قَاحَتْ حُبُرُ ضَرْبِي مِنْ جِهَةِ حَمَاقَتِي.
\par 6 لَوِيتُ. انْحَنَيْتُ إِلَى الْغَايَةِ. الْيَوْمَ كُلَّهُ ذَهَبْتُ حَزِيناً.
\par 7 لأَنَّ خَاصِرَتَيَّ قَدِ امْتَلَأَتَا احْتِرَاقاً وَلَيْسَتْ فِي جَسَدِي صِحَّةٌ.
\par 8 خَدِرْتُ وَانْسَحَقْتُ إِلَى الْغَايَةِ. كُنْتُ أَئِنُّ مِنْ زَفِيرِ قَلْبِي.
\par 9 يَا رَبُّ أَمَامَكَ كُلُّ تَأَوُّهِي وَتَنَهُّدِي لَيْسَ بِمَسْتُورٍ عَنْكَ.
\par 10 قَلْبِي خَافِقٌ. قُوَّتِي فَارَقَتْنِي وَنُورُ عَيْنِي أَيْضاً لَيْسَ مَعِي.
\par 11 أَحِبَّائِي وَأَصْحَابِي يَقِفُونَ تُجَاهَ ضَرْبَتِي وَأَقَارِبِي وَقَفُوا بَعِيداً.
\par 12 وَطَالِبُو نَفْسِي نَصَبُوا شَرَكاً وَالْمُلْتَمِسُونَ لِيَ الشَّرَّ تَكَلَّمُوا بِالْمَفَاسِدِ وَالْيَوْمَ كُلَّهُ يَلْهَجُونَ بِالْغِشِّ.
\par 13 وَأَمَّا أَنَا فَكَأَصَمَّ لاَ أَسْمَعُ. وَكَأَبْكَمَ لاَ يَفْتَحُ فَاهُ.
\par 14 وَأَكُونُ مِثْلَ إِنْسَانٍ لاَ يَسْمَعُ وَلَيْسَ فِي فَمِهِ حُجَّةٌ.
\par 15 لأَنِّي لَكَ يَا رَبُّ صَبِرْتُ أَنْتَ تَسْتَجِيبُ يَا رَبُّ إِلَهِي.
\par 16 لأَنِّي قُلْتُ: [لِئَلاَّ يَشْمَتُوا بِي]. عِنْدَمَا زَلَّتْ قَدَمِي تَعَظَّمُوا عَلَيَّ.
\par 17 لأَنِّي مُوشِكٌ أَنْ أَظْلَعَ وَوَجَعِي مُقَابِلِي دَائِماً.
\par 18 لأَنَّنِي أُخْبِرُ بِإِثْمِي وَأَغْتَمُّ مِنْ خَطِيَّتِي.
\par 19 وَأَمَّا أَعْدَائِي فَأَحْيَاءٌ. عَظُمُوا. وَالَّذِينَ يُبْغِضُونَنِي ظُلْماً كَثُرُوا.
\par 20 وَالْمُجَازُونَ عَنِ الْخَيْرِ بِشَرٍّ يُقَاوِمُونَنِي لأَجْلِ اتِّبَاعِي الصَّلاَحَ.
\par 21 لاَ تَتْرُكْنِي يَا رَبُّ. يَا إِلَهِي لاَ تَبْعُدْ عَنِّي.
\par 22 أَسْرِعْ إِلَى مَعُونَتِي يَا رَبُّ يَا خَلاَصِي.

\chapter{39}

\par 1 لإِمَامِ الْمُغَنِّينَ. لِيَدُوثُونَ. مَزْمُورٌ لِدَاوُدَ قُلْتُ أَتَحَفَّظُ لِسَبِيلِي مِنَ الْخَطَإِ بِلِسَانِي. أَحْفَظُ لِفَمِي كِمَامَةً فِيمَا الشِّرِّيرُ مُقَابِلِي.
\par 2 صَمَتُّ صَمْتاً سَكَتُّ عَنِ الْخَيْرِ فَتَحَرَّكَ وَجَعِي.
\par 3 حَمِيَ قَلْبِي فِي جَوْفِي. عِنْدَ لَهَجِي اشْتَعَلَتِ النَّارُ. تَكَلَّمْتُ بِلِسَانِي.
\par 4 عَرِّفْنِي يَا رَبُّ نِهَايَتِي وَمِقْدَارَ أَيَّامِي كَمْ هِيَ فَأَعْلَمَ كَيْفَ أَنَا زَائِلٌ.
\par 5 هُوَذَا جَعَلْتَ أَيَّامِي أَشْبَاراً وَعُمْرِي كَلاَ شَيْءَ قُدَّامَكَ. إِنَّمَا نَفْخَةً كُلُّ إِنْسَانٍ قَدْ جُعِلَ. سِلاَهْ.
\par 6 إِنَّمَا كَخَيَالٍ يَتَمَشَّى الإِنْسَانُ. إِنَّمَا بَاطِلاً يَضِجُّونَ. يَذْخَرُ ذَخَائِرَ وَلاَ يَدْرِي مَنْ يَضُمُّهَا.
\par 7 وَالآنَ مَاذَا انْتَظَرْتُ يَا رَبُّ؟ رَجَائِي فِيكَ هُوَ.
\par 8 مِنْ كُلِّ مَعَاصِيَّ نَجِّنِي. لاَ تَجْعَلْنِي عَاراً عِنْدَ الْجَاهِلِ.
\par 9 صَمَتُّ. لاَ أَفْتَحُ فَمِي لأَنَّكَ أَنْتَ فَعَلْتَ.
\par 10 ارْفَعْ عَنِّي ضَرْبَكَ. مِنْ مُهَاجَمَةِ يَدِكَ أَنَا قَدْ فَنِيتُ.
\par 11 بِتَأْدِيبَاتٍ إِنْ أَدَّبْتَ الإِنْسَانَ مِنْ أَجْلِ إِثْمِهِ أَفْنَيْتَ مِثْلَ الْعُثِّ مُشْتَهَاهُ. إِنَّمَا كُلُّ إِنْسَانٍ نَفْخَةٌ. سِلاَهْ.
\par 12 اِسْتَمِعْ صَلاَتِي يَا رَبُّ وَاصْغَ إِلَى صُرَاخِي. لاَ تَسْكُتْ عَنْ دُمُوعِي. لأَنِّي أَنَا غَرِيبٌ عِنْدَكَ. نَزِيلٌ مِثْلُ جَمِيعِ آبَائِي.
\par 13 اقْتَصِرْ عَنِّي فَأَتَبَلَّجَ قَبْلَ أَنْ أَذْهَبَ فَلاَ أُوجَدَ.

\chapter{40}

\par 1 لإِمَامِ الْمُغَنِّينَ. مَزْمُورٌ لِدَاوُدَ اِنْتِظَاراً انْتَظَرْتُ الرَّبَّ فَمَالَ إِلَيَّ وَسَمِعَ صُرَاخِي
\par 2 وَأَصْعَدَنِي مِنْ جُبِّ الْهَلاَكِ مِنْ طِينِ الْحَمْأَةِ وَأَقَامَ عَلَى صَخْرَةٍ رِجْلَيَّ. ثَبَّتَ خُطُواتِي
\par 3 وَجَعَلَ فِي فَمِي تَرْنِيمَةً جَدِيدَةً تَسْبِيحَةً لإِلَهِنَا. كَثِيرُونَ يَرُونَ وَيَخَافُونَ وَيَتَوَكَّلُونَ عَلَى الرَّبِّ.
\par 4 طُوبَى لِلرَّجُلِ الَّذِي جَعَلَ الرَّبَّ مُتَّكَلَهُ وَلَمْ يَلْتَفِتْ إِلَى الْغَطَارِيسِ وَالْمُنْحَرِفِينَ إِلَى الْكَذِبِ.
\par 5 كَثِيراً مَا جَعَلْتَ أَنْتَ أَيُّهَا الرَّبُّ إِلَهِي عَجَائِبَكَ وَأَفْكَارَكَ مِنْ جِهَتِنَا. لاَ تُقَوَّمُ لَدَيْكَ. لَأُخْبِرَنَّ وَأَتَكَلَّمَنَّ بِهَا. زَادَتْ عَنْ أَنْ تُعَدَّ.
\par 6 بِذَبِيحَةٍ وَتَقْدِمَةٍ لَمْ تُسَرَّ. أُذُنَيَّ فَتَحْتَ. مُحْرَقَةً وَذَبِيحَةَ خَطِيَّةٍ لَمْ تَطْلُبْ.
\par 7 حِينَئِذٍ قُلْتُ: [هَئَنَذَا جِئْتُ. بِدَرْجِ الْكِتَابِ مَكْتُوبٌ عَنِّي
\par 8 أَنْ أَفْعَلَ مَشِيئَتَكَ يَا إِلَهِي سُرِرْتُ. وَشَرِيعَتُكَ فِي وَسَطِ أَحْشَائِي].
\par 9 بَشَّرْتُ بِبِرٍّ فِي جَمَاعَةٍ عَظِيمَةٍ. هُوَذَا شَفَتَايَ لَمْ أَمْنَعْهُمَا. أَنْتَ يَا رَبُّ عَلِمْتَ.
\par 10 لَمْ أَكْتُمْ عَدْلَكَ فِي وَسَطِ قَلْبِي. تَكَلَّمْتُ بِأَمَانَتِكَ وَخَلاَصِكَ. لَمْ أُخْفِ رَحْمَتَكَ وَحَقَّكَ عَنِ الْجَمَاعَةِ الْعَظِيمَةِ.
\par 11 أَمَّا أَنْتَ يَا رَبُّ فَلاَ تَمْنَعْ رَأْفَتَكَ عَنِّي. تَنْصُرُنِي رَحْمَتُكَ وَحَقُّكَ دَائِماً.
\par 12 لأَنَّ شُرُوراً لاَ تُحْصَى قَدِ اكْتَنَفَتْنِي. حَاقَتْ بِي آثَامِي وَلاَ أَسْتَطِيعُ أَنْ أُبْصِرَ. كَثُرَتْ أَكْثَرَ مِنْ شَعْرِ رَأْسِي وَقَلْبِي قَدْ تَرَكَنِي.
\par 13 اِرْتَضِ يَا رَبُّ بِأَنْ تُنَجِّيَنِي. يَا رَبُّ إِلَى مَعُونَتِي أَسْرِعْ.
\par 14 لِيَخْزَ وَلْيَخْجَلْ مَعاً الَّذِينَ يَطْلُبُونَ نَفْسِي لإِهْلاَكِهَا. لِيَرْتَدَّ إِلَى الْوَرَاءِ وَلْيَخْزَ الْمَسْرُورُونَ بِأَذِيَّتِي.
\par 15 لِيَسْتَوْحِشْ مِنْ أَجْلِ خِزْيِهِمِ الْقَائِلُونَ لِي: [هَهْ هَهْ!]
\par 16 لِيَبْتَهِجْ وَيَفْرَحْ بِكَ جَمِيعُ طَالِبِيكَ. لِيَقُلْ أَبَداً مُحِبُّو خَلاَصِكَ: [يَتَعَظَّمُ الرَّبُّ].
\par 17 أَمَّا أَنَا فَمِسْكِينٌ وَبَائِسٌ. الرَّبُّ يَهْتَمُّ بِي. عَوْنِي وَمُنْقِذِي أَنْتَ. يَا إِلَهِي لاَ تُبْطِئ.

\chapter{41}

\par 1 لإِمَامِ الْمُغَنِّينَ. مَزْمُورٌ لِدَاوُدَ طُوبَى لِلَّذِي يَنْظُرُ إِلَى الْمَِسْكِينِ. فِي يَوْمِ الشَّرِّ يُنَجِّيهِ الرَّبُّ.
\par 2 الرَّبُّ يَحْفَظُهُ وَيُحْيِيهِ. يَغْتَبِطُ فِي الأَرْضِ وَلاَ يُسَلِّمُهُ إِلَى مَرَامِ أَعْدَائِهِ.
\par 3 الرَّبُّ يَعْضُدُهُ وَهُوَ عَلَى فِرَاشِ الضُّعْفِ. مَهَّدْتَ مَضْجَعَهُ كُلَّهُ فِي مَرَضِهِ.
\par 4 أَنَا قُلْتُ: [يَا رَبُّ ارْحَمْنِي. اشْفِ نَفْسِي لأَنِّي قَدْ أَخْطَأْتُ إِلَيْكَ].
\par 5 أَعْدَائِي يَتَقَاوَلُونَ عَلَيَّ بِشَرٍّ: [مَتَى يَمُوتُ وَيَبِيدُ اسْمُهُ؟]
\par 6 وَإِنْ دَخَلَ لِيَرَانِي يَتَكَلَّمُ بِالْكَذِبِ. قَلْبُهُ يَجْمَعُ لِنَفْسِهِ إِثْماً. يَخْرُجُ فِي الْخَارِجِ يَتَكَلَّمُ.
\par 7 كُلُّ مُبْغِضِيَّ يَتَنَاجُونَ مَعاً عَلَيَّ. عَلَيَّ تَفَكَّرُوا بِأَذِيَّتِي.
\par 8 يَقُولُونَ: [أَمْرٌ رَدِيءٌ قَدِ انْسَكَبَ عَلَيْهِ. حَيْثُ اضْطَجَعَ لاَ يَعُودُ يَقُومُ].
\par 9 أَيْضاً رَجُلُ سَلاَمَتِي الَّذِي وَثَقْتُ بِهِ آكِلُ خُبْزِي رَفَعَ عَلَيَّ عَقِبَهُ!
\par 10 أَمَّا أَنْتَ يَا رَبُّ فَارْحَمْنِي وَأَقِمْنِي فَأُجَازِيَهُمْ.
\par 11 بِهَذَا عَلِمْتُ أَنَّكَ سُرِرْتَ بِي أَنَّهُ لَمْ يَهْتِفْ عَلَيَّ عَدُوِّي.
\par 12 أَمَّا أَنَا فَبِكَمَالِي دَعَمْتَنِي وَأَقَمْتَنِي قُدَّامَكَ إِلَى الأَبَدِ.
\par 13 مُبَارَكٌ الرَّبُّ إِلَهُ إِسْرَائِيلَ مِنَ الأَزَلِ وَإِلَى الأَبَدِ. آمِينَ فَآمِينَ.

\chapter{42}

\par 1 لإِمَامِ الْمُغَنِّينَ. قَصِيدَةٌ لِبَنِي قُورَحَ كَمَا يَشْتَاقُ الإِيَّلُ إِلَى جَدَاوِلِ الْمِيَاهِ هَكَذَا تَشْتَاقُ نَفْسِي إِلَيْكَ يَا اللهُ.
\par 2 عَطِشَتْ نَفْسِي إِلَى اللهِ إِلَى الإِلَهِ الْحَيِّ. مَتَى أَجِيءُ وَأَتَرَاءَى قُدَّامَ اللهِ!
\par 3 صَارَتْ لِي دُمُوعِي خُبْزاً نَهَاراً وَلَيْلاً إِذْ قِيلَ لِي كُلَّ يَوْمٍ أَيْنَ إِلَهُكَ
\par 4 هَذِهِ أَذْكُرُهَا فَأَسْكُبُ نَفْسِي عَلَيَّ. لأَنِّي كُنْتُ أَمُرُّ مَعَ الْجُمَّاعِ أَتَدَرَّجُ مَعَهُمْ إِلَى بَيْتِ اللهِ بِصَوْتِ تَرَنُّمٍ وَحَمْدٍ جُمْهُورٌ مُعَيِّدٌ.
\par 5 لِمَاذَا أَنْتِ مُنْحَنِيَةٌ يَا نَفْسِي وَلِمَاذَا تَئِنِّينَ فِيَّ؟ ارْتَجِي اللهَ لأَنِّي بَعْدُ أَحْمَدُهُ لأَجْلِ خَلاَصِ وَجْهِهِ.
\par 6 يَا إِلَهِي نَفْسِي مُنْحَنِيَةٌ فِيَّ لِذَلِكَ أَذْكُرُكَ مِنْ أَرْضِ الأُرْدُنِّ وَجِبَالِ حَرْمُونَ مِنْ جَبَلِ مِصْعَرَ.
\par 7 غَمْرٌ يُنَادِي غَمْراً عِنْدَ صَوْتِ مَيَازِيبِكَ. كُلُّ تَيَّارَاتِكَ وَلُجَجِكَ طَمَتْ عَلَيَّ.
\par 8 بِالنَّهَارِ يُوصِي الرَّبُّ رَحْمَتَهُ وَبِاللَّيْلِ تَسْبِيحُهُ عِنْدِي صَلاَةٌ لإِلَهِ حَيَاتِي.
\par 9 أَقُولُ لِلَّهِ صَخْرَتِي: [لِمَاذَا نَسِيتَنِي؟ لِمَاذَا أَذْهَبُ حَزِيناً مِنْ مُضَايَقَةِ الْعَدُوِّ؟]
\par 10 بِسَحْقٍ فِي عِظَامِي عَيَّرَنِي مُضَايِقِيَّ بِقَوْلِهِمْ لِي كُلَّ يَوْمٍ: [أَيْنَ إِلَهُكَ؟]
\par 11 لِمَاذَا أَنْتِ مُنْحَنِيَةٌ يَا نَفْسِي وَلِمَاذَا تَئِنِّينَ فِيَّ؟ تَرَجَّيِ اللهَ لأَنِّي بَعْدُ أَحْمَدُهُ خَلاَصَ وَجْهِي وَإِلَهِي.

\chapter{43}

\par 1 اِقْضِ لِي يَا اللهُ وَخَاصِمْ مُخَاصَمَتِي مَعَ أُمَّةٍ غَيْرِ رَاحِمَةٍ وَمِنْ إِنْسَانِ غِشٍّ وَظُلْمٍ نَجِّنِي.
\par 2 لأَنَّكَ أَنْتَ إِلَهُ حِصْنِي. لِمَاذَا رَفَضْتَنِي؟ لِمَاذَا أَتَمَشَّى حَزِيناً مِنْ مُضَايَقَةِ الْعَدُوِّ؟
\par 3 أَرْسِلْ نُورَكَ وَحَقَّكَ هُمَا يَهْدِيَانِنِي وَيَأْتِيَانِ بِي إِلَى جَبَلِ قُدْسِكَ وَإِلَى مَسَاكِنِكَ.
\par 4 فَآتِي إِلَى مَذْبَحِ اللهِ إِلَى اللهِ بَهْجَةِ فَرَحِي وَأَحْمَدُكَ بِالْعُودِ يَا اللهُ إِلَهِي.
\par 5 لِمَاذَا أَنْتِ مُنْحَنِيَةٌ يَا نَفْسِي وَلِمَاذَا تَئِنِّينَ فِيَّ؟ تَرَجَّيِ اللهَ لأَنِّي بَعْدُ أَحْمَدُهُ خَلاَصَ وَجْهِي وَإِلَهِي.

\chapter{44}

\par 1 لإِمَامِ الْمُغَنِّينَ. لِبَنِي قُورَحَ. قَصِيدَةٌ اَللهُمَّ بِآذَانِنَا قَدْ سَمِعْنَا. آبَاؤُنَا أَخْبَرُونَا بِعَمَلٍ عَمِلْتَهُ فِي أَيَّامِهِمْ فِي أَيَّامِ الْقِدَمِ.
\par 2 أَنْتَ بِيَدِكَ اسْتَأْصَلْتَ الأُمَمَ وَغَرَسْتَهُمْ. حَطَّمْتَ شُعُوباً وَمَدَدْتَهُمْ.
\par 3 لأَنَّهُ لَيْسَ بِسَيْفِهِمُِ امْتَلَكُوا الأَرْضَ وَلاَ ذِرَاعُهُمْ خَلَّصَتْهُمْ لَكِنْ يَمِينُكَ وَذِرَاعُكَ وَنُورُ وَجْهِكَ لأَنَّكَ رَضِيتَ عَنْهُمْ.
\par 4 أَنْتَ هُوَ مَلِكِي يَا اللهُ. فَأْمُرْ بِخَلاَصِ يَعْقُوبَ.
\par 5 بِكَ نَنْطَحُ مُضَايِقِينَا. بِاسْمِكَ نَدُوسُ الْقَائِمِينَ عَلَيْنَا.
\par 6 لأَنِّي عَلَى قَوْسِي لاَ أَتَّكِلُ وَسَيْفِي لاَ يُخَلِّصُنِي.
\par 7 لأَنَّكَ أَنْتَ خَلَّصْتَنَا مِنْ مُضَايِقِينَا وَأَخْزَيْتَ مُبْغِضِينَا.
\par 8 بِاللهِ نَفْتَخِرُ الْيَوْمَ كُلَّهُ وَاسْمَكَ نَحْمَدُ إِلَى الدَّهْرِ. سِلاَهْ.
\par 9 لَكِنَّكَ قَدْ رَفَضْتَنَا وَأَخْجَلْتَنَا وَلاَ تَخْرُجُ مَعَ جُنُودِنَا.
\par 10 تُرْجِعُنَا إِلَى الْوَرَاءِ عَنِ الْعَدُوِّ وَمُبْغِضُونَا نَهَبُوا لأَنْفُسِهِمْ.
\par 11 جَعَلْتَنَا كَالضَّأْنِ أَكْلاً. ذَرَّيْتَنَا بَيْنَ الأُمَمِ.
\par 12 بِعْتَ شَعْبَكَ بِغَيْرِ مَالٍ وَمَا رَبِحْتَ بِثَمَنِهِمْ.
\par 13 تَجْعَلُنَا عَاراً عِنْدَ جِيرَانِنَا هُزْأَةً وَسُخْرَةً لِلَّذِينَ حَوْلَنَا.
\par 14 تَجْعَلُنَا مَثَلاً بَيْنَ الشُّعُوبِ. لإِنْغَاضِ الرَّأْسِ بَيْنَ الأُمَمِ.
\par 15 الْيَوْمَ كُلَّهُ خَجَلِي أَمَامِي وَخِزْيُ وَجْهِي قَدْ غَطَّانِي.
\par 16 مِنْ صَوْتِ الْمُعَيِّرِ وَالشَّاتِمِ. مِنْ وَجْهِ عَدُوٍّ وَمُنْتَقِمٍ.
\par 17 هَذَا كُلُّهُ جَاءَ عَلَيْنَا وَمَا نَسِينَاكَ وَلاَ خُنَّا فِي عَهْدِكَ.
\par 18 لَمْ يَرْتَدَّ قَلْبُنَا إِلَى وَرَاءٍ وَلاَ مَالَتْ خَطْوَتُنَا عَنْ طَرِيقِكَ
\par 19 حَتَّى سَحَقْتَنَا فِي مَكَانِ التَّنَانِينِ وَغَطَّيْتَنَا بِظِلِّ الْمَوْتِ.
\par 20 إِنْ نَسِينَا اسْمَ إِلَهِنَا أَوْ بَسَطْنَا أَيْدِيَنَا إِلَى إِلَهٍ غَرِيبٍ
\par 21 أَفَلاَ يَفْحَصُ اللهُ عَنْ هَذَا لأَنَّهُ هُوَ يَعْرِفُ خَفِيَّاتِ الْقَلْبِ؟
\par 22 لأَنَّنَا مِنْ أَجْلِكَ نُمَاتُ الْيَوْمَ كُلَّهُ. قَدْ حُسِبْنَا مِثْلَ غَنَمٍ لِلذَّبْحِ.
\par 23 اِسْتَيْقِظْ. لِمَاذَا تَتَغَافَى يَا رَبُّ؟ انْتَبِهْ. لاَ تَرْفُضْ إِلَى الأَبَدِ.
\par 24 لِمَاذَا تَحْجُبُ وَجْهَكَ وَتَنْسَى مَذَلَّتَنَا وَضِيقَنَا؟
\par 25 لأَنَّ أَنْفُسَنَا مُنْحَنِيَةٌ إِلَى التُّرَابِ. لَصِقَتْ فِي الأَرْضِ بُطُونُنَا.
\par 26 قُمْ عَوْناً لَنَا وَافْدِنَا مِنْ أَجْلِ رَحْمَتِكَ.

\chapter{45}

\par 1 لإِمَامِ الْمُغَنِّينَ. عَلَى السَّوْسَنِّ. لِبَنِي قُورَحَ. قَصِيدَةٌ. تَرْنِيمَةُ مَحَبَّةٍ فَاضَ قَلْبِي بِكَلاَمٍ صَالِحٍ. مُتَكَلِّمٌ أَنَا بِإِنْشَائِي لِلْمَلِكِ. لِسَانِي قَلَمُ كَاتِبٍ مَاهِرٍ.
\par 2 أَنْتَ أَبْرَعُ جَمَالاً مِنْ بَنِي الْبَشَرِ. انْسَكَبَتِ النِّعْمَةُ عَلَى شَفَتَيْكَ لِذَلِكَ بَارَكَكَ اللهُ إِلَى الأَبَدِ.
\par 3 تَقَلَّدْ سَيْفَكَ عَلَى فَخْذِكَ أَيُّهَا الْجَبَّارُ جَلاَلَكَ وَبَهَاءَكَ.
\par 4 وَبِجَلاَلِكَ اقْتَحِمِ. ارْكَبْ. مِنْ أَجْلِ الْحَقِّ وَالدَّعَةِ وَالْبِرِّ فَتُرِيَكَ يَمِينُكَ مَخَاوِفَ.
\par 5 نَبْلُكَ الْمَسْنُونَةُ فِي قَلْبِ أَعْدَاءِ الْمَلِكِ. شُعُوبٌ تَحْتَكَ يَسْقُطُونَ.
\par 6 كُرْسِيُّكَ يَا اللهُ إِلَى دَهْرِ الدُّهُورِ. قَضِيبُ اسْتِقَامَةٍ قَضِيبُ مُلْكِكَ.
\par 7 أَحْبَبْتَ الْبِرَّ وَأَبْغَضْتَ الإِثْمَ مِنْ أَجْلِ ذَلِكَ مَسَحَكَ اللهُ إِلَهُكَ بِدُهْنِ الاِبْتِهَاجِ أَكْثَرَ مِنْ رُفَقَائِكَ.
\par 8 كُلُّ ثِيَابِكَ مُرٌّ وَعُودٌ وَسَلِيخَةٌ. مِنْ قُصُورِ الْعَاجِ سَرَّتْكَ الأَوْتَارُ.
\par 9 بَنَاتُ مُلُوكٍ بَيْنَ حَظِيَّاتِكَ. جُعِلَتِ الْمَلِكَةُ عَنْ يَمِينِكَ بِذَهَبِ أُوفِيرٍ.
\par 10 اِسْمَعِي يَا بِنْتُ وَانْظُرِي وَأَمِيلِي أُذْنَكِ وَانْسَيْ شَعْبَكِ وَبَيْتَ أَبِيكِ
\par 11 فَيَشْتَهِيَ الْمَلِكُ حُسْنَكِ لأَنَّهُ هُوَ سَيِّدُكِ فَاسْجُدِي لَهُ.
\par 12 وَبِنْتُ صُورٍ أَغْنَى الشُّعُوبِ تَتَرَضَّى وَجْهَكِ بِهَدِيَّةٍ.
\par 13 كُلُّهَا مَجْدٌ ابْنَةُ الْمَلِكِ فِي خِدْرِهَا. مَنْسُوجَةٌ بِذَهَبٍ مَلاَبِسُهَا.
\par 14 بِمَلاَبِسَ مُطَرَّزَةٍ تُحْضَرُ إِلَى الْمَلِكِ. فِي أَثَرِهَا عَذَارَى صَاحِبَاتُهَا. مُقَدَّمَاتٍ إِلَيْكَ
\par 15 يُحْضَرْنَ بِفَرَحٍ وَابْتِهَاجٍ. يَدْخُلْنَ إِلَى قَصْرِ الْمَلِكِ.
\par 16 عِوَضاً عَنْ آبَائِكَ يَكُونُ بَنُوكَ تُقِيمُهُمْ رُؤَسَاءَ فِي كُلِّ الأَرْضِ.
\par 17 أَذْكُرُ اسْمَكَ فِي كُلِّ دَوْرٍ فَدَوْرٍ. مِنْ أَجْلِ ذَلِكَ تَحْمَدُكَ الشُّعُوبُ إِلَى الدَّهْرِ وَالأَبَدِ.

\chapter{46}

\par 1 لإِمَامِ الْمُغَنِّينَ. لِبَنِي قُورَحَ. عَلَى الْجَوَابِ. تَرْنِيمَةٌ اَللهُ لَنَا مَلْجَأٌ وَقُوَّةٌ. عَوْناً فِي الضِّيقَاتِ وُجِدَ شَدِيداً.
\par 2 لِذَلِكَ لاَ نَخْشَى وَلَوْ تَزَحْزَحَتِ الأَرْضُ وَلَوِ انْقَلَبَتِ الْجِبَالُ إِلَى قَلْبِ الْبِحَارِ.
\par 3 تَعِجُّ وَتَجِيشُ مِيَاهُهَا. تَتَزَعْزَعُ الْجِبَالُ بِطُمُوِّهَا. سِلاَهْ.
\par 4 نَهْرٌ سَوَاقِيهِ تُفَرِّحُ مَدِينَةَ اللهِ مَقْدِسَ مَسَاكِنِ الْعَلِيِّ.
\par 5 اللهُ فِي وَسَطِهَا فَلَنْ تَتَزَعْزَعَ. يُعِينُهَا اللهُ عِنْدَ إِقْبَالِ الصُّبْحِ.
\par 6 عَجَّتِ الأُمَمُ. تَزَعْزَعَتِ الْمَمَالِكُ. أَعْطَى صَوْتَهُ ذَابَتِ الأَرْضُ.
\par 7 رَبُّ الْجُنُودِ مَعَنَا. مَلْجَأُنَا إِلَهُ يَعْقُوبَ. سِلاَهْ.
\par 8 هَلُمُّوا انْظُرُوا أَعْمَالَ اللهِ كَيْفَ جَعَلَ خِرَباً فِي الأَرْضِ.
\par 9 مُسَكِّنُ الْحُرُوبِ إِلَى أَقْصَى الأَرْضِ. يَكْسِرُ الْقَوْسَ وَيَقْطَعُ الرُّمْحَ. الْمَرْكَبَاتِ يُحْرِقُهَا بِالنَّارِ.
\par 10 كُفُّوا وَاعْلَمُوا أَنِّي أَنَا اللهُ. أَتَعَالَى بَيْنَ الأُمَمِ. أَتَعَالَى فِي الأَرْضِ.
\par 11 رَبُّ الْجُنُودِ مَعَنَا. مَلْجَأُنَا إِلَهُ يَعْقُوبَ. سِلاَهْ.

\chapter{47}

\par 1 لإِمَامِ الْمُغَنِّينَ. لِبَنِي قُورَحَ. مَزْمُورٌ يَا جَمِيعَ الأُمَمِ صَفِّقُوا بِالأَيَادِي. اهْتِفُوا لِلَّهِ بِصَوْتِ الاِبْتِهَاجِ.
\par 2 لأَنَّ الرَّبَّ عَلِيٌّ مَخُوفٌ مَلِكٌ كَبِيرٌ عَلَى كُلِّ الأَرْضِ.
\par 3 يُخْضِعُ الشُّعُوبَ تَحْتَنَا وَالأُمَمَ تَحْتَ أَقْدَامِنَا.
\par 4 يَخْتَارُ لَنَا نَصِيبَنَا فَخْرَ يَعْقُوبَ الَّذِي أَحَبَّهُ. سِلاَهْ.
\par 5 صَعِدَ اللهُ بِهُتَافٍ الرَّبُّ بِصَوْتِ الصُّورِ.
\par 6 رَنِّمُوا لِلَّهِ رَنِّمُوا. رَنِّمُوا لِمَلِكِنَا رَنِّمُوا.
\par 7 لأَنَّ اللهَ مَلِكُ الأَرْضِ كُلِّهَا رَنِّمُوا قَصِيدَةً.
\par 8 مَلَكَ اللهُ عَلَى الأُمَمِ. اللهُ جَلَسَ عَلَى كُرْسِيِّ قُدْسِهِ.
\par 9 شُرَفَاءُ الشُّعُوبِ اجْتَمَعُوا. شَعْبُ إِلَهِ إِبْرَاهِيمَ. لأَنَّ لِلَّهِ مَجَانَّ الأَرْضِ. هُوَ مُتَعَالٍ جِدّاً.

\chapter{48}

\par 1 تَسْبِيحَةٌ. مَزْمُورٌ لِبَنِي قُورَحَ عَظِيمٌ هُوَ الرَّبُّ وَحَمِيدٌ جِدّاً فِي مَدِينَةِ إِلَهِنَا جَبَلِ قُدْسِهِ.
\par 2 جَمِيلُ الاِرْتِفَاعِ فَرَحُ كُلِّ الأَرْضِ جَبَلُ صِهْيَوْنَ. فَرَحُ أَقَاصِي الشِّمَالِ مَدِينَةُ الْمَلِكِ الْعَظِيمِ.
\par 3 اَللهُ فِي قُصُورِهَا يُعْرَفُ مَلْجَأً.
\par 4 لأَنَّهُ هُوَذَا الْمُلُوكُ اجْتَمَعُوا. مَضُوا جَمِيعاً.
\par 5 لَمَّا رَأُوا بُهِتُوا ارْتَاعُوا فَرُّوا.
\par 6 أَخَذَتْهُمُ الرَّعْدَةُ هُنَاكَ وَالْمَخَاضُ كَوَالِدَةٍ
\par 7 بِرِيحٍ شَرْقِيَّةٍ تَكْسِرُ سُفُنَ تَرْشِيشَ.
\par 8 كَمَا سَمِعْنَا هَكَذَا رَأَيْنَا فِي مَدِينَةِ رَبِّ الْجُنُودِ فِي مَدِينَةِ إِلَهِنَا. اللهُ يُثَبِّتُهَا إِلَى الأَبَدِ. سِلاَهْ.
\par 9 ذَكَرْنَا يَا اللهُ رَحْمَتَكَ فِي وَسَطِ هَيْكَلِكَ.
\par 10 نَظِيرُ اسْمِكَ يَا اللهُ تَسْبِيحُكَ إِلَى أَقَاصِي الأَرْضِ. يَمِينُكَ مَلآنَةٌ بِرّاً.
\par 11 يَفْرَحُ جَبَلُ صِهْيَوْنَ تَبْتَهِجُ بَنَاتُ يَهُوذَا مِنْ أَجْلِ أَحْكَامِكَ.
\par 12 طُوفُوا بِصِهْيَوْنَ وَدُورُوا حَوْلَهَا. عُدُّوا أَبْرَاجَهَا.
\par 13 ضَعُوا قُلُوبَكُمْ عَلَى مَتَارِسِهَا. تَأَمَّلُوا قُصُورَهَا لِكَيْ تُحَدِّثُوا بِهَا جِيلاً آخَرَ.
\par 14 لأَنَّ اللهَ هَذَا هُوَ إِلَهُنَا إِلَى الدَّهْرِ وَالأَبَدِ. هُوَ يَهْدِينَا حَتَّى إِلَى الْمَوْتِ.

\chapter{49}

\par 1 لإِمَامِ الْمُغَنِّينَ. لِبَنِي قُورَحَ. مَزْمُورٌ اِسْمَعُوا هَذَا يَا جَمِيعَ الشُّعُوبِ. أَصْغُوا يَا جَمِيعَ سُكَّانِ الدُّنْيَا
\par 2 عَالٍ وَدُونٍ أَغْنِيَاءَ وَفُقَرَاءَ سَوَاءً.
\par 3 فَمِي يَتَكَلَّمُ بِالْحِكَمِ وَلَهَجُ قَلْبِي فَهْمٌ.
\par 4 أُمِيلُ أُذُنِي إِلَى مَثَلٍ وَأُوضِّحُ بِعُودٍ لُغْزِي.
\par 5 لِمَاذَا أَخَافُ فِي أَيَّامِ الشَّرِّ عِنْدَمَا يُحِيطُ بِي إِثْمُ مُتَعَقِّبِيَّ؟
\par 6 الَّذِينَ يَتَّكِلُونَ عَلَى ثَرْوَتِهِمْ وَبِكَثْرَةِ غِنَاهُمْ يَفْتَخِرُونَ.
\par 7 الأَخُ لَنْ يَفْدِيَ الإِنْسَانَ فِدَاءً وَلاَ يُعْطِيَ اللهَ كَفَّارَةً عَنْهُ.
\par 8 وَكَرِيمَةٌ هِيَ فِدْيَةُ نُفُوسِهِمْ فَغَلِقَتْ إِلَى الدَّهْرِ -
\par 9 حَتَّى يَحْيَا إِلَى الأَبَدِ فَلاَ يَرَى الْقَبْرَ.
\par 10 بَلْ يَرَاهُ! الْحُكَمَاءُ يَمُوتُونَ. كَذَلِكَ الْجَاهِلُ وَالْبَلِيدُ يَهْلِكَانِ وَيَتْرُكَانِ ثَرْوَتَهُمَا لآخَرِينَ.
\par 11 بَاطِنُهُمْ أَنَّ بُيُوتَهُمْ إِلَى الأَبَدِ مَسَاكِنَهُمْ إِلَى دَوْرٍ فَدَوْرٍ. يُنَادُونَ بِأَسْمَائِهِمْ فِي الأَرَاضِي.
\par 12 وَالإِنْسَانُ فِي كَرَامَةٍ لاَ يَبِيتُ. يُشْبِهُ الْبَهَائِمَ الَّتِي تُبَادُ.
\par 13 هَذَا طَرِيقُهُمُ اعْتِمَادُهُمْ وَخُلَفَاؤُهُمْ يَرْتَضُونَ بِأَقْوَالِهِمْ. سِلاَهْ.
\par 14 مِثْلُ الْغَنَمِ لِلْهَاوِيَةِ يُسَاقُونَ. الْمَوْتُ يَرْعَاهُمْ وَيَسُودُهُمُ الْمُسْتَقِيمُونَ. غَدَاةً وَصُورَتُهُمْ تَبْلَى. الْهَاوِيَةُ مَسْكَنٌ لَهُمْ.
\par 15 إِنَّمَا اللهُ يَفْدِي نَفْسِي مِنْ يَدِ الْهَاوِيَةِ لأَنَّهُ يَأْخُذُنِي. سِلاَهْ.
\par 16 لاَ تَخْشَ إِذَا اسْتَغْنَى إِنْسَانٌ إِذَا زَادَ مَجْدُ بَيْتِهِ.
\par 17 لأَنَّهُ عِنْدَ مَوْتِهِ كُلُّهُ لاَ يَأْخُذُ. لاَ يَنْزِلُ وَرَاءَهُ مَجْدُهُ.
\par 18 لأَنَّهُ فِي حَيَاتِهِ يُبَارِكُ نَفْسَهُ. وَيَحْمَدُونَكَ إِذَا أَحْسَنْتَ إِلَى نَفْسِكَ.
\par 19 تَدْخُلُ إِلَى جِيلِ آبَائِهِ الَّذِينَ لاَ يُعَايِنُونَ النُّورَ إِلَى الأَبَدِ.
\par 20 إِنْسَانٌ فِي كَرَامَةٍ وَلاَ يَفْهَمُ يُشْبِهُ الْبَهَائِمَ الَّتِي تُبَادُ.

\chapter{50}

\par 1 مَزْمُورٌ لآسَافَ إِلَهُ الآلِهَةِ الرَّبُّ تَكَلَّمَ وَدَعَا الأَرْضَ مِنْ مَشْرِقِ الشَّمْسِ إِلَى مَغْرِبِهَا.
\par 2 مِنْ صِهْيَوْنَ كَمَالِ الْجَمَالِ اللهُ أَشْرَقَ.
\par 3 يَأْتِي إِلَهُنَا وَلاَ يَصْمُتُ. نَارٌ قُدَّامَهُ تَأْكُلُ وَحَوْلَهُ عَاصِفٌ جِدّاً.
\par 4 يَدْعُو السَّمَاوَاتِ مِنْ فَوْقُ وَالأَرْضَ إِلَى مُدَايَنَةِ شَعْبِهِ.
\par 5 اجْمَعُوا إِلَيَّ أَتْقِيَائِي الْقَاطِعِينَ عَهْدِي عَلَى ذَبِيحَةٍ.
\par 6 وَتُخْبِرُ السَّمَاوَاتُ بِعَدْلِهِ لأَنَّ اللهَ هُوَ الدَّيَّانُ. سِلاَهْ.
\par 7 اِسْمَعْ يَا شَعْبِي فَأَتَكَلَّمَ. يَا إِسْرَائِيلُ فَأَشْهَدَ عَلَيْكَ. اللهُ إِلَهُكَ أَنَا.
\par 8 لاَ عَلَى ذَبَائِحِكَ أُوَبِّخُكَ فَإِنَّ مُحْرَقَاتِكَ هِيَ دَائِماً قُدَّامِي.
\par 9 لاَ آخُذُ مِنْ بَيْتِكَ ثَوْراً وَلاَ مِنْ حَظَائِرِكَ أَعْتِدَةً.
\par 10 لأَنَّ لِي حَيَوَانَ الْوَعْرِ وَالْبَهَائِمَ عَلَى الْجِبَالِ الأُلُوفِ.
\par 11 قَدْ عَلِمْتُ كُلَّ طُيُورِ الْجِبَالِ وَوُحُوشُ الْبَرِّيَّةِ عِنْدِي.
\par 12 إِنْ جُعْتُ فَلاَ أَقُولُ لَكَ لأَنَّ لِي الْمَسْكُونَةَ وَمِلأَهَا.
\par 13 هَلْ آكُلُ لَحْمَ الثِّيرَانِ أَوْ أَشْرَبُ دَمَ التُّيُوسِ؟
\par 14 اِذْبَحْ لِلَّهِ حَمْداً وَأَوْفِ الْعَلِيَّ نُذُورَكَ
\par 15 وَادْعُنِي فِي يَوْمِ الضِّيقِ أُنْقِذْكَ فَتُمَجِّدَنِي.
\par 16 وَلِلشِّرِّيرِ قَالَ اللهُ: [مَا لَكَ تُحَدِّثُ بِفَرَائِضِي وَتَحْمِلُ عَهْدِي عَلَى فَمِكَ
\par 17 وَأَنْتَ قَدْ أَبْغَضْتَ التَّأْدِيبَ وَأَلْقَيْتَ كَلاَمِي خَلْفَكَ.
\par 18 إِذَا رَأَيْتَ سَارِقاً وَافَقْتَهُ وَمَعَ الزُّنَاةِ نَصِيبُكَ.
\par 19 أَطْلَقْتَ فَمَكَ بِالشَّرِّ وَلِسَانُكَ يَخْتَرِعُ غِشّاً.
\par 20 تَجْلِسُ تَتَكَلَّمُ عَلَى أَخِيكَ. لاِبْنِ أُمِّكَ تَضَعُ مَعْثَرَةً.
\par 21 هَذِهِ صَنَعْتَ وَسَكَتُّ. ظَنَنْتَ أَنِّي مِثْلُكَ. أُوَبِّخُكَ وَأَصُفُّ خَطَايَاكَ أَمَامَ عَيْنَيْكَ.
\par 22 افْهَمُوا هَذَا يَا أَيُّهَا النَّاسُونَ اللهَ لِئَلاَّ أَفْتَرِسَكُمْ وَلاَ مُنْقِذَ.
\par 23 ذَابِحُ الْحَمْدِ يُمَجِّدُنِي وَالْمُقَوِّمُ طَرِيقَهُ أُرِيهِ خَلاَصَ اللهِ].

\chapter{51}

\par 1 لإِمَامِ الْمُغَنِّينَ. مَزْمُورٌ لِدَاوُدَ عِنْدَمَا جَاءَ إِلَيْهِ نَاثَانُ النَّبِيُّ بَعْدَ مَا دَخَلَ إِلَى بَثْشَبَعَ اِرْحَمْنِي يَا اللهُ حَسَبَ رَحْمَتِكَ. حَسَبَ كَثْرَةِ رَأْفَتِكَ امْحُ مَعَاصِيَّ.
\par 2 اغْسِلْنِي كَثِيراً مِنْ إِثْمِي وَمِنْ خَطِيَّتِي طَهِّرْنِي.
\par 3 لأَنِّي عَارِفٌ بِمَعَاصِيَّ وَخَطِيَّتِي أَمَامِي دَائِماً.
\par 4 إِلَيْكَ وَحْدَكَ أَخْطَأْتُ وَالشَّرَّ قُدَّامَ عَيْنَيْكَ صَنَعْتُ لِكَيْ تَتَبَرَّرَ فِي أَقْوَالِكَ وَتَزْكُوَ فِي قَضَائِكَ.
\par 5 هَئَنَذَا بِالإِثْمِ صُوِّرْتُ وَبِالْخَطِيَّةِ حَبِلَتْ بِي أُمِّي.
\par 6 هَا قَدْ سُرِرْتَ بِالْحَقِّ فِي الْبَاطِنِ فَفِي السَّرِيرَةِ تُعَرِّفُنِي حِكْمَةً.
\par 7 طَهِّرْنِي بِالزُوّفَا فَأَطْهُرَ. اغْسِلْنِي فَأَبْيَضَّ أَكْثَرَ مِنَ الثَّلْجِ.
\par 8 أَسْمِعْنِي سُرُوراً وَفَرَحاً فَتَبْتَهِجَ عِظَامٌ سَحَقْتَهَا.
\par 9 اسْتُرْ وَجْهَكَ عَنْ خَطَايَايَ وَامْحُ كُلَّ آثَامِي.
\par 10 قَلْباً نَقِيّاً اخْلُقْ فِيَّ يَا اللهُ وَرُوحاً مُسْتَقِيماً جَدِّدْ فِي دَاخِلِي.
\par 11 لاَ تَطْرَحْنِي مِنْ قُدَّامِ وَجْهِكَ وَرُوحَكَ الْقُدُّوسَ لاَ تَنْزِعْهُ مِنِّي.
\par 12 رُدَّ لِي بَهْجَةَ خَلاَصِكَ وَبِرُوحٍ مُنْتَدِبَةٍ اعْضُدْنِي.
\par 13 فَأُعَلِّمَ الأَثَمَةَ طُرُقَكَ وَالْخُطَاةُ إِلَيْكَ يَرْجِعُونَ.
\par 14 نَجِّنِي مِنَ الدِّمَاءِ يَا اللهُ إِلَهَ خَلاَصِي فَيُسَبِّحَ لِسَانِي بِرَّكَ.
\par 15 يَا رَبُّ افْتَحْ شَفَتَيَّ فَيُخْبِرَ فَمِي بِتَسْبِيحِكَ.
\par 16 لأَنَّكَ لاَ تُسَرُّ بِذَبِيحَةٍ وَإِلاَّ فَكُنْتُ أُقَدِّمُهَا. بِمُحْرَقَةٍ لاَ تَرْضَى.
\par 17 ذَبَائِحُ اللهِ هِيَ رُوحٌ مُنْكَسِرَةٌ. الْقَلْبُ الْمُنْكَسِرُ وَالْمُنْسَحِقُ يَا اللهُ لاَ تَحْتَقِرُهُ.
\par 18 أَحْسِنْ بِرِضَاكَ إِلَى صِهْيَوْنَ. ابْنِ أَسْوَارَ أُورُشَلِيمَ.
\par 19 حِينَئِذٍ تُسَرُّ بِذَبَائِحِ الْبِرِّ مُحْرَقَةٍ وَتَقْدِمَةٍ تَامَّةٍ. حِينَئِذٍ يُصْعِدُونَ عَلَى مَذْبَحِكَ عُجُولاً.

\chapter{52}

\par 1 لإِمَامِ الْمُغَنِّينَ. قَصِيدَةٌ لِدَاوُدَ_عِنْدَمَا جَاءَ دُوَاغُ الأَدُومِيُّ وَأَخْبَرَ شَاوُلَ وَقَالَ لَهُ: [جَاءَ دَاوُدُ إِلَى بَيْتِ أَخِيمَالِكَ]. لِمَاذَا تَفْتَخِرُ بِالشَّرِّ أَيُّهَا الْجَبَّارُ؟ رَحْمَةُ اللهِ هِيَ كُلَّ يَوْمٍ!
\par 2 لِسَانُكَ يَخْتَرِعُ مَفَاسِدَ. كَمُوسَى مَسْنُونَةٍ يَعْمَلُ بِالْغِشِّ.
\par 3 أَحْبَبْتَ الشَّرَّ أَكْثَرَ مِنَ الْخَيْرِ الْكَذِبَ أَكْثَرَ مِنَ التَّكَلُّمِ بِالصِّدْقِ. سِلاَهْ.
\par 4 أَحْبَبْتَ كُلَّ كَلاَمٍ مُهْلِكٍ وَلِسَانِ غِشٍّ.
\par 5 أَيْضاً يَهْدِمُكَ اللهُ إِلَى الأَبَدِ. يَخْطُفُكَ وَيَقْلَعُكَ مِنْ مَسْكَنِكَ وَيَسْتَأْصِلُكَ مِنْ أَرْضِ الأَحْيَاءِ. سِلاَهْ.
\par 6 فَيَرَى الصِّدِّيقُونَ وَيَخَافُونَ وَعَلَيْهِ يَضْحَكُونَ:
\par 7 [هُوَذَا الإِنْسَانُ الَّذِي لَمْ يَجْعَلِ اللهَ حِصْنَهُ بَلِ اتَّكَلَ عَلَى كَثْرَةِ غِنَاهُ وَاعْتَزَّ بِفَسَادِهِ].
\par 8 أَمَّا أَنَا فَمِثْلُ زَيْتُونَةٍ خَضْرَاءَ فِي بَيْتِ اللهِ. تَوَكَّلْتُ عَلَى رَحْمَةِ اللهِ إِلَى الدَّهْرِ وَالأَبَدِ.
\par 9 أَحْمَدُكَ إِلَى الدَّهْرِ لأَنَّكَ فَعَلْتَ وَأَنْتَظِرُ اسْمَكَ فَإِنَّهُ صَالِحٌ قُدَّامَ أَتْقِيَائِكَ.

\chapter{53}

\par 1 لإِمَامِ الْمُغَنِّينَ عَلَى الْعُودِ. قَصِيدَةٌ لِدَاوُدَ قَالَ الْجَاهِلُ فِي قَلْبِهِ: [لَيْسَ إِلَهٌ]. فَسَدُوا وَرَجِسُوا رَجَاسَةً. لَيْسَ مَنْ يَعْمَلُ صَلاَحاً.
\par 2 اَللهُ مِنَ السَّمَاءِ أَشْرَفَ عَلَى بَنِي الْبَشَرِ لِيَنْظُرَ: هَلْ مِنْ فَاهِمٍ طَالِبِ اللهِ؟
\par 3 كُلُّهُمْ قَدِ ارْتَدُّوا مَعاً فَسَدُوا لَيْسَ مَنْ يَعْمَلُ صَلاَحاً لَيْسَ وَلاَ وَاحِدٌ.
\par 4 أَلَمْ يَعْلَمْ فَاعِلُو الإِثْمِ الَّذِينَ يَأْكُلُونَ شَعْبِي كَمَا يَأْكُلُونَ الْخُبْزَ وَاللهَ لَمْ يَدْعُوا؟
\par 5 هُنَاكَ خَافُوا خَوْفاً وَلَمْ يَكُنْ خَوْفٌ لأَنَّ اللهَ قَدْ بَدَّدَ عِظَامَ مُحَاصِرِكَ. أَخْزَيْتَهُمْ لأَنَّ اللهَ قَدْ رَفَضَهُمْ.
\par 6 لَيْتَ مِنْ صِهْيَوْنَ خَلاَصَ إِسْرَائِيلَ. عِنْدَ رَدِّ اللهِ سَبْيَ شَعْبِهِ يَهْتِفُ يَعْقُوبُ وَيَفْرَحُ إِسْرَائِيلُ.اَلأَصْحَاحُ الرَّابِعُ وَالْخَمْسُونَ لإِمَامِ الْمُغَنِّينَ عَلَى ذَوَاتِ الأَوْتَارِ. قَصِيدَةٌ لِدَاوُدَ عِنْدَمَا أَتَى الزِّيفِيُّونَ وَقَالُوا لِشَاوُلَ: [أَلَيْسَ دَاوُدُ مُخْتَبِئاً عِنْدَنَا؟]

\chapter{54}

\par 1 اَللهُمَّ بِاسْمِكَ خَلِّصْنِي وَبِقُوَّتِكَ احْكُمْ لِي.
\par 2 اسْمَعْ يَا اللهُ صَلاَتِي. اصْغَ إِلَى كَلاَمِ فَمِي.
\par 3 لأَنَّ غُرَبَاءَ قَدْ قَامُوا عَلَيَّ وَعُتَاةً طَلَبُوا نَفْسِي. لَمْ يَجْعَلُوا اللهَ أَمَامَهُمْ. سِلاَهْ.
\par 4 هُوَذَا اللهُ مُعِينٌ لِي. الرَّبُّ بَيْنَ عَاضِدِي نَفْسِي.
\par 5 يَرْجِعُ الشَّرُّ عَلَى أَعْدَائِي. بِحَقِّكَ أَفْنِهِمْ.
\par 6 أَذْبَحُ لَكَ مُنْتَدِباً. أَحْمَدُ اسْمَكَ يَا رَبُّ لأَنَّهُ صَالِحٌ.
\par 7 لأَنَّهُ مِنْ كُلِّ ضِيقٍ نَجَّانِي وَبِأَعْدَائِي رَأَتْ عَيْنِي.

\chapter{55}

\par 1 لإِمَامِ الْمُغَنِّينَ عَلَى ذَوَاتِ الأَوْتَارِ. قَصِيدَةٌ لِدَاوُدَ اِصْغَ يَا اللهُ إِلَى صَلاَتِي وَلاَ تَتَغَاضَ عَنْ تَضَرُّعِي.
\par 2 اسْتَمِعْ لِي وَاسْتَجِبْ لِي. أَتَحَيَّرُ فِي كُرْبَتِي وَأَضْطَرِبُ
\par 3 مِنْ صَوْتِ الْعَدُوِّ مِنْ قِبَلِ ظُلْمِ الشِّرِّيرِ. لأَنَّهُمْ يُحِيلُونَ عَلَيَّ إِثْماً وَبِغَضَبٍ يَضْطَهِدُونَنِي.
\par 4 يَمْخَضُ قَلْبِي فِي دَاخِلِي وَأَهْوَالُ الْمَوْتِ سَقَطَتْ عَلَيَّ
\par 5 خَوْفٌ وَرَعْدَةٌ أَتَيَا عَلَيَّ وَغَشِيَنِي رُعْبٌ.
\par 6 فَقُلْتُ: [لَيْتَ لِي جَنَاحاً كَالْحَمَامَةِ فَأَطِيرَ وَأَسْتَرِيحَ!
\par 7 هَئَنَذَا كُنْتُ أَبْعُدُ هَارِباً وَأَبِيتُ فِي الْبَرِّيَّةِ. سِلاَهْ.
\par 8 كُنْتُ أُسْرِعُ فِي نَجَاتِي مِنَ الرِّيحِ الْعَاصِفَةِ وَمِنَ النَّوْءِ].
\par 9 أَهْلِكْ يَا رَبُّ فَرِّقْ أَلْسِنَتَهُمْ لأَنِّي قَدْ رَأَيْتُ ظُلْماً وَخِصَاماً فِي الْمَدِينَةِ.
\par 10 نَهَاراً وَلَيْلاً يُحِيطُونَ بِهَا عَلَى أَسْوَارِهَا وَإِثْمٌ وَمَشَقَّةٌ فِي وَسَطِهَا.
\par 11 مَفَاسِدُ فِي وَسَطِهَا وَلاَ يَبْرَحُ مِنْ سَاحَتِهَا ظُلْمٌ وَغِشٌّ.
\par 12 لأَنَّهُ لَيْسَ عَدُوٌّ يُعَيِّرُنِي فَأَحْتَمِلَ. لَيْسَ مُبْغِضِي تَعَظَّمَ عَلَيَّ فَأَخْتَبِئَ مِنْهُ.
\par 13 بَلْ أَنْتَ إِنْسَانٌ عَدِيلِي إِلْفِي وَصَدِيقِي
\par 14 الَّذِي مَعَهُ كَانَتْ تَحْلُو لَنَا الْعِشْرَةُ. إِلَى بَيْتِ اللهِ كُنَّا نَذْهَبُ فِي الْجُمْهُورِ.
\par 15 لِيَبْغَتْهُمُ الْمَوْتُ. لِيَنْحَدِرُوا إِلَى الْهَاوِيَةِ أَحْيَاءً لأَنَّ فِي مَسَاكِنِهِمْ فِي وَسَطِهِمْ شُرُوراً.
\par 16 أَمَّا أَنَا فَإِلَى اللهِ أَصْرُخُ وَالرَّبُّ يُخَلِّصُنِي.
\par 17 مَسَاءً وَصَبَاحاً وَظُهْراً أَشْكُو وَأَنُوحُ فَيَسْمَعُ صَوْتِي.
\par 18 فَدَى بِسَلاَمٍ نَفْسِي مِنْ قِتَالٍ عَلَيَّ لأَنَّهُمْ بِكَثْرَةٍ كَانُوا حَوْلِي.
\par 19 يَسْمَعُ اللهُ فَيُذِلُّهُمْ وَالْجَالِسُ مُنْذُ الْقِدَمِ. سِلاَهْ. الَّذِينَ لَيْسَ لَهُمْ تَغَيُّرٌ وَلاَ يَخَافُونَ اللهَ.
\par 20 أَلْقَى يَدَيْهِ عَلَى مُسَالِمِيهِ. نَقَضَ عَهْدَهُ.
\par 21 أَنْعَمُ مِنَ الزُّبْدَةِ فَمُهُ وَقَلْبُهُ قِتَالٌ. أَلْيَنُ مِنَ الزَّيْتِ كَلِمَاتُهُ وَهِيَ سُيُوفٌ مَسْلُولَةٌ.
\par 22 أَلْقِ عَلَى الرَّبِّ هَمَّكَ فَهُوَ يَعُولُكَ. لاَ يَدَعُ الصِّدِّيقَ يَتَزَعْزَعُ إِلَى الأَبَدِ.
\par 23 وَأَنْتَ يَا اللهُ تُحَدِّرُهُمْ إِلَى جُبِّ الْهَلاَكِ. رِجَالُ الدِّمَاءِ وَالْغِشِّ لاَ يَنْصُفُونَ أَيَّامَهُمْ. أَمَّا أَنَا فَأَتَّكِلُ عَلَيْكَ.

\chapter{56}

\par 1 لإِمَامِ الْمُغَنِّينَ عَلَى [الْحَمَامَةِ الْبَكْمَاءِ بَيْنَ الْغُرَبَاءِ]. مُذَهَّبَةٌ لِدَاوُدَ عِنْدَمَا أَخَذَهُ الْفِلِسْطِينِيُّونَ فِي جَتٍَّ. اِرْحَمْنِي يَا اللهُ لأَنَّ الإِنْسَانَ يَتَهَمَّمُنِي وَالْيَوْمَ كُلَّهُ مُحَارِباً يُضَايِقُنِي.
\par 2 تَهَمَّمَنِي أَعْدَائِي الْيَوْمَ كُلَّهُ لأَنَّ كَثِيرِينَ يُقَاوِمُونَنِي بِكِبْرِيَاءٍ.
\par 3 فِي يَوْمِ خَوْفِي أَنَا عَلَيْكَ أَتَّكِلُ.
\par 4 اَللهُ أَفْتَخِرُ بِكَلاَمِهِ. عَلَى اللهِ تَوَكَّلْتُ فَلاَ أَخَافُ. مَاذَا يَصْنَعُهُ بِي الْبَشَرُ!
\par 5 الْيَوْمَ كُلَّهُ يُحَرِّفُونَ كَلاَمِي. عَلَيَّ كُلُّ أَفْكَارِهِمْ بِالشَّرِّ.
\par 6 يَجْتَمِعُونَ يَخْتَفُونَ يُلاَحِظُونَ خَطَواتِي عِنْدَمَا تَرَصَّدُوا نَفْسِي.
\par 7 عَلَى إِثْمِهِمْ جَازِهِمْ. بِغَضَبٍ أَخْضِعِ الشُّعُوبَ يَا اللهُ.
\par 8 تَيَهَانِي رَاقَبْتَ. اجْعَلْ أَنْتَ دُمُوعِي فِي زِقِّكَ. أَمَا هِيَ فِي سِفْرِكَ؟
\par 9 حِينَئِذٍ تَرْتَدُّ أَعْدَائِي إِلَى الْوَرَاءِ فِي يَوْمٍ أَدْعُوكَ فِيهِ. هَذَا قَدْ عَلِمْتُهُ لأَنَّ اللهَ لِي.
\par 10 اَللهُ أَفْتَخِرُ بِكَلاَمِهِ. الرَّبُّ أَفْتَخِرُ بِكَلاَمِهِ.
\par 11 عَلَى اللهِ تَوَكَّلْتُ فَلاَ أَخَافُ. مَاذَا يَصْنَعُهُ بِي الإِنْسَانُ؟
\par 12 اَللهُمَّ عَلَيَّ نُذُورُكَ. أُوفِي ذَبَائِحَ شُكْرٍ لَكَ.
\par 13 لأَنَّكَ نَجَّيْتَ نَفْسِي مِنَ الْمَوْتِ. نَعَمْ وَرِجْلَيَّ مِنَ الزَّلَقِ لِكَيْ أَسِيرَ قُدَّامَ اللهِ فِي نُورِ الأَحْيَاءِ.

\chapter{57}

\par 1 لإِمَامِ الْمُغَنِّينَ. عَلَى [لاَ تُهْلِكْ]. مُذَهَّبَةٌ لِدَاوُدَ عِنْدَمَا هَرَبَ مِنْ قُدَّامِ شَاوُلَ فِي الْمَغَارَةِ. اِرْحَمْنِي يَا اللهُ ارْحَمْنِي لأَنَّهُ بِكَ احْتَمَتْ نَفْسِي وَبِظِلِّ جَنَاحَيْكَ أَحْتَمِي إِلَى أَنْ تَعْبُرَ الْمَصَائِبُ.
\par 2 أَصْرُخُ إِلَى اللهِ الْعَلِيِّ إِلَى اللهِ الْمُحَامِي عَنِّي.
\par 3 يُرْسِلُ مِنَ السَّمَاءِ وَيُخَلِّصُنِي. عَيَّرَ الَّذِي يَتَهَمَّمُنِي. سِلاَهْ. يُرْسِلُ اللهُ رَحْمَتَهُ وَحَقَّهُ.
\par 4 نَفْسِي بَيْنَ الأَشْبَالِ. أَضْطَجِعُ بَيْنَ الْمُتَّقِدِينَ بَنِي آدَمَ. أَسْنَانُهُمْ أَسِنَّةٌ وَسِهَامٌ وَلِسَانُهُمْ سَيْفٌ مَاضٍ.
\par 5 ارْتَفِعِ اللهُمَّ عَلَى السَّمَاوَاتِ. لِيَرْتَفِعْ عَلَى كُلِّ الأَرْضِ مَجْدُكَ.
\par 6 هَيَّأُوا شَبَكَةً لِخَطَوَاتِي. انْحَنَتْ نَفْسِي. حَفَرُوا قُدَّامِي حُفْرَةً. سَقَطُوا فِي وَسَطِهَا. سِلاَهْ.
\par 7 ثَابِتٌ قَلْبِي يَا اللهُ ثَابِتٌ قَلْبِي. أُغَنِّي وَأُرَنِّمُ.
\par 8 اسْتَيْقِظْ يَا مَجْدِي. اسْتَيْقِظِي يَا رَبَابُ وَيَا عُودُ. أَنَا أَسْتَيْقِظُ سَحَراً.
\par 9 أَحْمَدُكَ بَيْنَ الشُّعُوبِ يَا رَبُّ. أُرَنِّمُ لَكَ بَيْنَ الأُمَمِ.
\par 10 لأَنَّ رَحْمَتَكَ قَدْ عَظُمَتْ إِلَى السَّمَاوَاتِ وَإِلَى الْغَمَامِ حَقُّكَ.
\par 11 ارْتَفِعِ اللهُمَّ عَلَى السَّمَاوَاتِ. لِيَرْتَفِعْ عَلَى كُلِّ الأَرْضِ مَجْدُكَ.

\chapter{58}

\par 1 لإِمَامِ الْمُغَنِّينَ. عَلَى [لاَ تُهْلِكْ]. لِدَاوُدَ. مُذَهَّبَةٌ أَحَقّاً بِالْحَقِّ الأَخْرَسِ تَتَكَلَّمُونَ بِالْمُسْتَقِيمَاتِ تَقْضُونَ يَا بَنِي آدَمَ؟
\par 2 بَلْ بِالْقَلْبِ تَعْمَلُونَ شُرُوراً فِي الأَرْضِ. ظُلْمَ أَيْدِيكُمْ تَزِنُونَ.
\par 3 زَاغَ الأَشْرَارُ مِنَ الرَّحِمِ. ضَلُّوا مِنَ الْبَطْنِ مُتَكَلِّمِينَ كَذِباً.
\par 4 لَهُمْ حُمَةٌ مِثْلُ حُمَةِ الْحَيَّةِ. مِثْلُ الصِّلِّ الأَصَمِّ يَسُدُّ أُذْنَهُ
\par 5 الَّذِي لاَ يَسْتَمِعُ إِلَى صَوْتِ الْحُواةِ الرَّاقِينَ رُقَى حَكِيمٍ.
\par 6 اَللهُمَّ كَسِّرْ أَسْنَانَهُمْ فِي أَفْوَاهِهِمِ. اهْشِمْ أَضْرَاسَ الأَشْبَالِ يَا رَبُّ.
\par 7 لِيَذُوبُوا كَالْمَاءِ لِيَذْهَبُوا. إِذَا فَوَّقَ سِهَامَهُ فَلْتَنْبُ.
\par 8 كَمَا يَذُوبُ الْحَلَزُونُ مَاشِياً. مِثْلَ سِقْطِ الْمَرْأَةِ لاَ يُعَايِنُوا الشَّمْسَ.
\par 9 قَبْلَ أَنْ تَشْعُرَ قُدُورُكُمْ بِالشَّوْكِ نِيئاً أَوْ مَحْرُوقاً يَجْرُفُهُمْ.
\par 10 يَفْرَحُ الصِّدِّيقُ إِذَا رَأَى النَّقْمَةَ. يَغْسِلُ خَطَواتِهِ بِدَمِ الشِّرِّيرِ.
\par 11 وَيَقُولُ الإِنْسَانُ: [إِنَّ لِلصِّدِّيقِ ثَمَراً. إِنَّهُ يُوجَدُ إِلَهٌ قَاضٍ فِي الأَرْضِ].

\chapter{59}

\par 1 لإِمَامِ الْمُغَنِّينَ. عَلَى [لاَ تُهْلِكْ]. مُذَهَّبَةٌ لِدَاوُدَ لَمَّا أَرْسَلَ شَاوُلُ وَرَاقَبُوا الْبَيْتَ لِيَقْتُلُوهُ. أَنْقِذْنِي مِنْ أَعْدَائِي يَا إِلَهِي. مِنْ مُقَاوِمِيَّ احْمِنِي.
\par 2 نَجِّنِي مِنْ فَاعِلِي الإِثْمِ وَمِنْ رِجَالِ الدِّمَاءِ خَلِّصْنِي
\par 3 لأَنَّهُمْ يَكْمُنُونَ لِنَفْسِي. الأَقْوِيَاءُ يَجْتَمِعُونَ عَلَيَّ لاَ لإِثْمِي وَلاَ لِخَطِيَّتِي يَا رَبُّ.
\par 4 بِلاَ إِثْمٍ مِنِّي يَجْرُونَ وَيُعِدُّونَ أَنْفُسَهُمُ. اسْتَيْقِظْ إِلَى لِقَائِي وَانْظُرْ.
\par 5 وَأَنْتَ يَا رَبُّ إِلَهَ الْجُنُودِ إِلَهَ إِسْرَائِيلَ انْتَبِهْ لِتُطَالِبَ كُلَّ الأُمَمِ. كُلَّ غَادِرٍ أَثِيمٍ لاَ تَرْحَمْ. سِلاَهْ.
\par 6 يَعُودُونَ عِنْدَ الْمَسَاءِ يَهِرُّونَ مِثْلَ الْكَلْبِ وَيَدُورُونَ فِي الْمَدِينَةِ.
\par 7 هُوَذَا يُبِقُّونَ بِأَفْوَاهِهِمْ. سُيُوفٌ فِي شِفَاهِهِمْ. لأَنَّهُمْ يَقُولُونَ: [مَنْ سَامِعٌ؟]
\par 8 أَمَّا أَنْتَ يَا رَبُّ فَتَضْحَكُ بِهِمْ. تَسْتَهْزِئُ بِجَمِيعِ الأُمَمِ.
\par 9 مِنْ قُوَّتِهِ إِلَيْكَ أَلْتَجِئُ لأَنَّ اللهَ مَلْجَإِي.
\par 10 إِلَهِي رَحْمَتُهُ تَتَقَدَّمُنِي. اللهُ يُرِينِي بِأَعْدَائِي.
\par 11 لاَ تَقْتُلْهُمْ لِئَلاَّ يَنْسَى شَعْبِي. تَيِّهْهُمْ بِقُوَّتِكَ وَأَهْبِطْهُمْ يَا رَبُّ تُرْسَنَا.
\par 12 خَطِيَّةُ أَفْوَاهِهِمْ هِيَ كَلاَمُ شِفَاهِهِمْ. وَلْيُؤْخَذُوا بِكِبْرِيَائِهِمْ وَمِنَ اللَّعْنَةِ وَمِنَ الْكَذِبِ الَّذِي يُحَدِّثُونَ بِهِ.
\par 13 أَفْنِ بِحَنَقٍ أَفْنِ وَلاَ يَكُونُوا وَلْيَعْلَمُوا أَنَّ اللهَ مُتَسَلِّطٌ فِي يَعْقُوبَ إِلَى أَقَاصِي الأَرْضِ. سِلاَهْ.
\par 14 وَيَعُودُونَ عِنْدَ الْمَسَاءِ. يَهِرُّونَ مِثْلَ الْكَلْبِ وَيَدُورُونَ فِي الْمَدِينَةِ.
\par 15 هُمْ يَتِيهُونَ لِلأَكْلِ. إِنْ لَمْ يَشْبَعُوا وَيَبِيتُوا.
\par 16 أَمَّا أَنَا فَأُغَنِّي بِقُوَّتِكَ وَأُرَنِّمُ بِالْغَدَاةِ بِرَحْمَتِكَ لأَنَّكَ كُنْتَ مَلْجَأً لِي وَمَنَاصاً فِي يَوْمِ ضِيقِي.
\par 17 يَا قُوَّتِي لَكَ أُرَنِّمُ لأَنَّ اللهَ مَلْجَإِي إِلَهُ رَحْمَتِي.

\chapter{60}

\par 1 لإِمَامِ الْمُغَنِّينَ عَلَى السَّوْسَنِّ. شَهَادَةٌ مُذَهَّبَةٌ لِدَاوُدَ لِلتَّعْلِيمِ. عِنْدَ مُحَارَبَتِهِ أَرَامَ النَّهْرَيْنِ وَأَرَامَ صُوبَةَ فَرَجَعَ يُوآبُ وَضَرَبَ مِنْ أَدُومَ فِي وَادِي الْمِلْحِ اثْنَيْ عَشَرَ أَلْفاً. يَا اللهُ رَفَضْتَنَا. اقْتَحَمْتَنَا. سَخِطْتَ. أَرْجِعْنَا.
\par 2 زَلْزَلْتَ الأَرْضَ. فَصَمْتَهَا. اجْبُرْ كَسْرَهَا لأَنَّهَا مُتَزَعْزِعَةٌ.
\par 3 أَرَيْتَ شَعْبَكَ عُسْراً. سَقَيْتَنَا خَمْرَ التَّرَنُّحِ.
\par 4 أَعْطَيْتَ خَائِفِيكَ رَايَةً تُرْفَعُ لأَجْلِ الْحَقِّ. سِلاَهْ.
\par 5 لِكَيْ يَنْجُوَ أَحِبَّاؤُكَ. خَلِّصْ بِيَمِينِكَ وَاسْتَجِبْ لِي.
\par 6 اَللهُ قَدْ تَكَلَّمَ بِقُدْسِهِ. أَبْتَهِجُ. أَقْسِمُ شَكِيمَ وَأَقِيسُ وَادِيَ سُكُّوتَ.
\par 7 لِي جِلْعَادُ وَلِي مَنَسَّى وَأَفْرَايِمُ خُوذَةُ رَأْسِي. يَهُوذَا صَوْلَجَانِي.
\par 8 مُوآبُ مِرْحَضَتِي. عَلَى أَدُومَ أَطْرَحُ نَعْلِي. يَا فَلَسْطِينُ اهْتِفِي عَلَيَّ.
\par 9 مَنْ يَقُودُنِي إِلَى الْمَدِينَةِ الْمُحَصَّنَةِ؟ مَنْ يَهْدِينِي إِلَى أَدُومَ؟
\par 10 أَلَيْسَ أَنْتَ يَا اللهُ الَّذِي رَفَضْتَنَا وَلاَ تَخْرُجُ يَا اللهُ مَعَ جُيُوشِنَا؟
\par 11 أَعْطِنَا عَوْناً فِي الضِّيقِ فَبَاطِلٌ هُوَ خَلاَصُ الإِنْسَانِ.
\par 12 بِاللهِ نَصْنَعُ بِبَأْسٍ وَهُوَ يَدُوسُ أَعْدَاءَنَا.

\chapter{61}

\par 1 لإِمَامِ الْمُغَنِّينَ عَلَى ذَوَاتِ الأَوْتَارِ. لِدَاوُدَ اِسْمَعْ يَا اللهُ صُرَاخِي وَاصْغَ إِلَى صَلاَتِي.
\par 2 مِنْ أَقْصَى الأَرْضِ أَدْعُوكَ إِذَا غُشِيَ عَلَى قَلْبِي. إِلَى صَخْرَةٍ أَرْفَعَ مِنِّي تَهْدِينِي.
\par 3 لأَنَّكَ كُنْتَ مَلْجَأً لِي بُرْجَ قُوَّةٍ مِنْ وَجْهِ الْعَدُوِّ.
\par 4 لَأَسْكُنَنَّ فِي مَسْكَنِكَ إِلَى الدُّهُورِ. أَحْتَمِي بِسِتْرِ جَنَاحَيْكَ. سِلاَهْ.
\par 5 لأَنَّكَ أَنْتَ يَا اللهُ اسْتَمَعْتَ نُذُورِي. أَعْطَيْتَ مِيرَاثَ خَائِفِي اسْمِكَ.
\par 6 إِلَى أَيَّامِ الْمَلِكِ تُضِيفُ أَيَّاماً. سِنِينُهُ كَدَوْرٍ فَدَوْرٍ.
\par 7 يَجْلِسُ قُدَّامَ اللهِ إِلَى الدَّهْرِ. اجْعَلْ رَحْمَةً وَحَقّاً يَحْفَظَانِهِ.
\par 8 هَكَذَا أُرَنِّمُ لاِسْمِكَ إِلَى الأَبَدِ. لِوَفَاءِ نُذُورِي يَوْماً فَيَوْماً.

\chapter{62}

\par 1 لإِمَامِ الْمُغَنِّينَ عَلَى [يَدُوثُونَ]. مَزْمُورٌ لِدَاوُدَ إِنَّمَا لِلَّهِ انْتَظَرَتْ نَفْسِي. مِنْ قِبَلِهِ خَلاَصِي.
\par 2 إِنَّمَا هُوَ صَخْرَتِي وَخَلاَصِي مَلْجَإِي. لاَ أَتَزَعْزَعُ كَثِيراً.
\par 3 إِلَى مَتَى تَهْجِمُونَ عَلَى الإِنْسَانِ؟ تَهْدِمُونَهُ كُلُّكُمْ كَحَائِطٍ مُنْقَضٍّ كَجِدَارٍ وَاقِعٍ!
\par 4 إِنَّمَا يَتَآمَرُونَ لِيَدْفَعُوهُ عَنْ شَرَفِهِ. يَرْضُونَ بِالْكَذِبِ. بِأَفْوَاهِهِمْ يُبَارِكُونَ وَبِقُلُوبِهِمْ يَلْعَنُونَ. سِلاَهْ.
\par 5 إِنَّمَا لِلَّهِ انْتَظِرِي يَا نَفْسِي لأَنَّ مِنْ قِبَلِهِ رَجَائِي.
\par 6 إِنَّمَا هُوَ صَخْرَتِي وَخَلاَصِي. مَلْجَإِي فَلاَ أَتَزَعْزَعُ.
\par 7 عَلَى اللهِ خَلاَصِي وَمَجْدِي. صَخْرَةُ قُوَّتِي مُحْتَمَايَ فِي اللهِ.
\par 8 تَوَكَّلُوا عَلَيْهِ فِي كُلِّ حِينٍ يَا قَوْمُ. اسْكُبُوا قُدَّامَهُ قُلُوبَكُمْ. اللهُ مَلْجَأٌ لَنَا. سِلاَهْ.
\par 9 إِنَّمَا بَاطِلٌ بَنُو آدَمَ. كَذِبٌ بَنُو الْبَشَرِ. فِي الْمَوَازِينِ هُمْ إِلَى فَوْقُ. هُمْ مِنْ بَاطِلٍ أَجْمَعُونَ.
\par 10 لاَ تَتَّكِلُوا عَلَى الظُّلْمِ وَلاَ تَصِيرُوا بَاطِلاً فِي الْخَطْفِ. إِنْ زَادَ الْغِنَى فَلاَ تَضَعُوا عَلَيْهِ قَلْباً.
\par 11 مَرَّةً وَاحِدَةً تَكَلَّمَ الرَّبُّ وَهَاتَيْنِ الاِثْنَتَيْنِ سَمِعْتُ أَنَّ الْعِزَّةَ لِلَّهِ.
\par 12 وَلَكَ يَا رَبُّ الرَّحْمَةُ لأَنَّكَ أَنْتَ تُجَازِي الإِنْسَانَ كَعَمَلِهِ.

\chapter{63}

\par 1 مَزْمُورٌ لِدَاوُدَ لَمَّا كَانَ فِي بَرِّيَّةِ يَهُوذَا يَا اللهُ إِلَهِي أَنْتَ. إِلَيْكَ أُبَكِّرُ. عَطِشَتْ إِلَيْكَ نَفْسِي يَشْتَاقُ إِلَيْكَ جَسَدِي فِي أَرْضٍ نَاشِفَةٍ وَيَابِسَةٍ بِلاَ مَاءٍ
\par 2 لِكَيْ أُبْصِرَ قُوَّتَكَ وَمَجْدَكَ كَمَا قَدْ رَأَيْتُكَ فِي قُدْسِكَ.
\par 3 لأَنَّ رَحْمَتَكَ أَفْضَلُ مِنَ الْحَيَاةِ. شَفَتَايَ تُسَبِّحَانِكَ.
\par 4 هَكَذَا أُبَارِكُكَ فِي حَيَاتِي. بِاسْمِكَ أَرْفَعُ يَدَيَّ.
\par 5 كَمَا مِنْ شَحْمٍ وَدَسَمٍ تَشْبَعُ نَفْسِي وَبِشَفَتَيْ الاِبْتِهَاجِ يُسَبِّحُكَ فَمِي.
\par 6 إِذَا ذَكَرْتُكَ عَلَى فِرَاشِي فِي السُّهْدِ أَلْهَجُ بِكَ
\par 7 لأَنَّكَ كُنْتَ عَوْناً لِي وَبِظِلِّ جَنَاحَيْكَ أَبْتَهِجُ.
\par 8 اِلْتَصَقَتْ نَفْسِي بِكَ. يَمِينُكَ تَعْضُدُنِي.
\par 9 أَمَّا الَّذِينَ هُمْ لِلتَّهْلُكَةِ يَطْلُبُونَ نَفْسِي فَيَدْخُلُونَ فِي أَسَافِلِ الأَرْضِ.
\par 10 يُدْفَعُونَ إِلَى يَدَيِ السَّيْفِ. يَكُونُونَ نَصِيباً لِبَنَاتِ آوَى.
\par 11 أَمَّا الْمَلِكُ فَيَفْرَحُ بِاللهِ. يَفْتَخِرُ كُلُّ مَنْ يَحْلِفُ بِهِ. لأَنَّ أَفْوَاهَ الْمُتَكَلِّمِينَ بِالْكَذِبِ تُسَدُّ.

\chapter{64}

\par 1 لإِمَامِ الْمُغَنِّينَ. مَزْمُورٌ لِدَاوُدَ اِسْتَمِعْ يَا اللهُ صَوْتِي فِي شَكْوَايَ. مِنْ خَوْفِ الْعَدُوِّ احْفَظْ حَيَاتِي.
\par 2 اسْتُرْنِي مِنْ مُؤَامَرَةِ الأَشْرَارِ مِنْ جُمْهُورِ فَاعِلِي الإِثْمِ
\par 3 الَّذِينَ صَقَلُوا أَلْسِنَتَهُمْ كَالسَّيْفِ. فَوَّقُوا سَهْمَهُمْ كَلاَماً مُرّاً
\par 4 لِيَرْمُوا الْكَامِلَ فِي الْمُخْتَفَى بَغْتَةً. يَرْمُونَهُ وَلاَ يَخْشُونَ.
\par 5 يُشَدِّدُونَ أَنْفُسَهُمْ لأَمْرٍ رَدِيءٍ. يَتَحَادَثُونَ بِطَمْرِ فِخَاخٍ. قَالُوا: [مَنْ يَرَاهُمْ؟]
\par 6 يَخْتَرِعُونَ إِثْماً تَمَّمُوا اخْتِرَاعاً مُحْكَماً. وَدَاخِلُ الإِنْسَانِ وَقَلْبُهُ عَمِيقٌ.
\par 7 فَيَرْمِيهِمِ اللهُ بِسَهْمٍ. بَغْتَةً كَانَتْ ضَرْبَتُهُمْ.
\par 8 وَيُوقِعُونَ أَلْسِنَتَهُمْ عَلَى أَنْفُسِهِمْ. يُنْغِضُ الرَّأْسَ كُلُّ مَنْ يَنْظُرُ إِلَيْهِمْ.
\par 9 وَيَخْشَى كُلُّ إِنْسَانٍ وَيُخْبِرُ بِفِعْلِ اللهِ وَبِعَمَلِهِ يَفْطَنُونَ.
\par 10 يَفْرَحُ الصِّدِّيقُ بِالرَّبِّ وَيَحْتَمِي بِهِ وَيَبْتَهِجُ كُلُّ الْمُسْتَقِيمِي الْقُلُوبِ.

\chapter{65}

\par 1 لإِمَامِ الْمُغَنِّينَ. مَزْمُورٌ لِدَاوُدَ. تَسْبِيحَةٌ لَكَ يَنْبَغِي التَّسْبِيحُ يَا اللهُ فِي صِهْيَوْنَ وَلَكَ يُوفَى النَّذْرُ.
\par 2 يَا سَامِعَ الصَّلاَةِ إِلَيْكَ يَأْتِي كُلُّ بَشَرٍ.
\par 3 آثَامٌ قَدْ قَوِيَتْ عَلَيَّ. مَعَاصِينَا أَنْتَ تُكَفِّرُ عَنْهَا.
\par 4 طُوبَى لِلَّذِي تَخْتَارُهُ وَتُقَرِّبُهُ لِيَسْكُنَ فِي دِيَارِكَ. لَنَشْبَعَنَّ مِنْ خَيْرِ بَيْتِكَ قُدْسِ هَيْكَلِكَ.
\par 5 بِمَخَاوِفَ فِي الْعَدْلِ تَسْتَجِيبُنَا يَا إِلَهَ خَلاَصِنَا يَا مُتَّكَلَ جَمِيعِ أَقَاصِي الأَرْضِ وَالْبَحْرِ الْبَعِيدَةِ.
\par 6 الْمُثْبِتُ الْجِبَالَِ بِقُوَّتِهِ الْمُتَنَطِّقُ بِالْقُدْرَةِ
\par 7 الْمُهَدِّئُ عَجِيجَ الْبِحَارِ عَجِيجَ أَمْوَاجِهَا وَضَجِيجَ الأُمَمِ.
\par 8 وَتَخَافُ سُكَّانُ الأَقَاصِي مِنْ آيَاتِكَ. تَجْعَلُ مَطَالِعَ الصَّبَاحِ وَالْمَسَاءِ تَبْتَهِجُ.
\par 9 تَعَهَّدْتَ الأَرْضَ وَجَعَلْتَهَا تَفِيضُ. تُغْنِيهَا جِدّاً. سَوَاقِي اللهِ مَلآنَةٌ مَاءً. تُهَيِّئُ طَعَامَهُمْ لأَنَّكَ هَكَذَا تُعِدُّهَا.
\par 10 أَرْوِ أَتْلاَمَهَا. مَهِّدْ أَخَادِيدَهَا. بِالْغُيُوثِ تُحَلِّلُهَا. تُبَارِكُ غَلَّتَهَا.
\par 11 كَلَّلْتَ السَّنَةَ بِجُودِكَ وَآثَارُكَ تَقْطُرُ دَسَماً.
\par 12 تَقْطُرُ مَرَاعِي الْبَرِّيَّةِ وَتَتَنَطَّقُ الآكَامُ بِالْبَهْجَةِ.
\par 13 اكْتَسَتِ الْمُرُوجُ غَنَماً وَالأَوْدِيَةُ تَتَعَطَّفُ بُرّاً. تَهْتِفُ وَأَيْضاً تُغَنِّي.

\chapter{66}

\par 1 لإِمَامِ الْمُغَنِّينَ. تَسْبِيحَةٌ. مَزْمُورٌ اِهْتِفِي لِلَّهِ يَا كُلَّ الأَرْضِ.
\par 2 رَنِّمُوا بِمَجْدِ اسْمِهِ. اجْعَلُوا تَسْبِيحَهُ مُمَجَّداً.
\par 3 قُولُوا لِلَّهِ: [مَا أَهْيَبَ أَعْمَالَكَ. مِنْ عِظَمِ قُوَّتِكَ تَتَمَلَّقُ لَكَ أَعْدَاؤُكَ.
\par 4 كُلُّ الأَرْضِ تَسْجُدُ لَكَ وَتُرَنِّمُ لَكَ. تُرَنِّمُ لاِسْمِكَ]. سِلاَهْ.
\par 5 هَلُمَّ انْظُرُوا أَعْمَالَ اللهِ. فِعْلَهُ الْمُرْهِبَ نَحْوَ بَنِي آدَمَ.
\par 6 حَوَّلَ الْبَحْرَ إِلَى يَبَسٍ وَفِي النَّهْرِ عَبَرُوا بِالرِّجْلِ. هُنَاكَ فَرِحْنَا بِهِ.
\par 7 مُتَسَلِّطٌ بِقُوَّتِهِ إِلَى الدَّهْرِ. عَيْنَاهُ تُرَاقِبَانِ الأُمَمَ. الْمُتَمَرِّدُونَ لاَ يَرْفَعُنَّ أَنْفُسَهُمْ. سِلاَهْ.
\par 8 بَارِكُوا إِلَهَنَا يَا أَيُّهَا الشُّعُوبُ وَسَمِّعُوا صَوْتَ تَسْبِيحِهِ.
\par 9 الْجَاعِلَ أَنْفُسَنَا فِي الْحَيَاةِ وَلَمْ يُسَلِّمْ أَرْجُلَنَا إِلَى الزَّلَلِ.
\par 10 لأَنَّكَ جَرَّبْتَنَا يَا اللهُ. مَحَصْتَنَا كَمَحْصِ الْفِضَّةِ.
\par 11 أَدْخَلْتَنَا إِلَى الشَّبَكَةِ. جَعَلْتَ ضَغْطاً عَلَى مُتُونِنَا.
\par 12 رَكَّبْتَ أُنَاساً عَلَى رُؤُوسِنَا. دَخَلْنَا فِي النَّارِ وَالْمَاءِ ثُمَّ أَخْرَجْتَنَا إِلَى الْخِصْبِ.
\par 13 أَدْخُلُ إِلَى بَيْتِكَ بِمُحْرَقَاتٍ أُوفِيكَ نُذُورِي
\par 14 الَّتِي نَطَقَتْ بِهَا شَفَتَايَ وَتَكَلَّمَ بِهَا فَمِي فِي ضِيقِي.
\par 15 أُصْعِدُ لَكَ مُحْرَقَاتٍ سَمِينَةً مَعَ بَخُورِ كِبَاشٍ. أُقَدِّمُ بَقَراً مَعَ تُيُوسٍ. سِلاَهْ
\par 16 هَلُمَّ اسْمَعُوا فَأُخْبِرَكُمْ يَا كُلَّ الْخَائِفِينَ اللهَ بِمَا صَنَعَ لِنَفْسِي.
\par 17 صَرَخْتُ إِلَيْهِ بِفَمِي وَتَبْجِيلٌ عَلَى لِسَانِي.
\par 18 إِنْ رَاعَيْتُ إِثْماً فِي قَلْبِي لاَ يَسْتَمِعُ لِيَ الرَّبُّ.
\par 19 لَكِنْ قَدْ سَمِعَ اللهُ. أَصْغَى إِلَى صَوْتِ صَلاَتِي.
\par 20 مُبَارَكٌ اللهُ الَّذِي لَمْ يُبْعِدْ صَلاَتِي وَلاَ رَحْمَتَهُ عَنِّي.

\chapter{67}

\par 1 لإِمَامِ الْمُغَنِّينَ عَلَى ذَوَاتِ الأَوْتَارِ. مَزْمُورٌ. تَسْبِيحَةٌ لِيَتَحَنَّنِ اللهُ عَلَيْنَا وَلْيُبَارِكْنَا. لِيُنِرْ بِوَجْهِهِ عَلَيْنَا. سِلاَهْ.
\par 2 لِكَيْ يُعْرَفَ فِي الأَرْضِ طَرِيقُكَ وَفِي كُلِّ الأُمَمِ خَلاَصُكَ.
\par 3 يَحْمَدُكَ الشُّعُوبُ يَا اللهُ. يَحْمَدُكَ الشُّعُوبُ كُلُّهُمْ.
\par 4 تَفْرَحُ وَتَبْتَهِجُ الأُمَمُ لأَنَّكَ تَدِينُ الشُّعُوبَ بِالاِسْتِقَامَةِ وَأُمَمَ الأَرْضِ تَهْدِيهِمْ. سِلاَهْ.
\par 5 يَحْمَدُكَ الشُّعُوبُ يَا اللهُ. يَحْمَدُكَ الشُّعُوبُ كُلُّهُمُ.
\par 6 الأَرْضُ أَعْطَتْ غَلَّتَهَا. يُبَارِكُنَا اللهُ إِلَهُنَا.
\par 7 يُبَارِكُنَا اللهُ وَتَخْشَاهُ كُلُّ أَقَاصِي الأَرْضِ.

\chapter{68}

\par 1 لإِمَامِ الْمُغَنِّينَ. لِدَاوُدَ. مَزْمُورٌ. تَسْبِيحَةٌ يَقُومُ اللهُ. يَتَبَدَّدُ أَعْدَاؤُهُ وَيَهْرُبُ مُبْغِضُوهُ مِنْ أَمَامِ وَجْهِهِ.
\par 2 كَمَا يُذْرَى الدُّخَانُ تُذْرِيهِمْ. كَمَا يَذُوبُ الشَّمْعُ قُدَّامَ النَّارِ يَبِيدُ الأَشْرَارُ قُدَّامَ اللهِ.
\par 3 وَالصِّدِّيقُونَ يَفْرَحُونَ. يَبْتَهِجُونَ أَمَامَ اللهِ وَيَطْفِرُونَ فَرَحاً.
\par 4 غَنُّوا لِلَّهِ. رَنِّمُوا لاِسْمِهِ. أَعِدُّوا طَرِيقاً لِلرَّاكِبِ فِي الْقِفَارِ بِاسْمِهِ يَاهْ وَاهْتِفُوا أَمَامَهُ.
\par 5 أَبُو الْيَتَامَى وَقَاضِي الأَرَامِلِ اللهُ فِي مَسْكَنِ قُدْسِهِ.
\par 6 اَللهُ مُسْكِنُ الْمُتَوَحِّدِينَ فِي بَيْتٍ. مُخْرِجُ الأَسْرَى إِلَى فَلاَحٍ. إِنَّمَا الْمُتَمَرِّدُونَ يَسْكُنُونَ الرَّمْضَاءَ.
\par 7 اَللهُمَّ عِنْدَ خُرُوجِكَ أَمَامَ شَعْبِكَ عِنْدَ صُعُودِكَ فِي الْقَفْرِ - سِلاَهْ.
\par 8 الأَرْضُ ارْتَعَدَتِ. السَّمَاوَاتُ أَيْضاً قَطَرَتْ أَمَامَ وَجْهِ اللهِ. سِينَاءُ نَفْسُهُ مِنْ وَجْهِ اللهِ إِلَهِ إِسْرَائِيلَ.
\par 9 مَطَراً غَزِيراً نَضَحْتَ يَا اللهُ. مِيرَاثُكَ وَهُوَ مُعْيٍ أَنْتَ أَصْلَحْتَهُ.
\par 10 قَطِيعُكَ سَكَنَ فِيهِ. هَيَّأْتَ بِجُودِكَ لِلْمَسَاكِينِ يَا اللهُ.
\par 11 الرَّبُّ يُعْطِي كَلِمَةً. الْمُبَشِّرَاتُ بِهَا جُنْدٌ كَثِيرٌ:
\par 12 [مُلُوكُ جُيُوشٍ يَهْرُبُونَ يَهْرُبُونَ. الْمُلاَزِمَةُ الْبَيْتَ تَقْسِمُ الْغَنَائِمَ.
\par 13 إِذَا اضْطَجَعْتُمْ بَيْنَ الْحَظَائِرِ فَأَجْنِحَةُ حَمَامَةٍ مُغَشَّاةٌ بِفِضَّةٍ وَرِيشُهَا بِصُفْرَةِ الذَّهَبِ].
\par 14 عِنْدَمَا شَتَّتَ الْقَدِيرُ مُلُوكاً فِيهَا أَثْلَجَتْ فِي صَلْمُونَ.
\par 15 جَبَلُ اللهِ جَبَلُ بَاشَانَ. جَبَلُ أَسْنِمَةٍ جَبَلُ بَاشَانَ.
\par 16 لِمَاذَا أَيَّتُهَا الْجِبَالُ الْمُسَنَّمَةُ تَرْصُدْنَ الْجَبَلَ الَّذِي اشْتَهَاهُ اللهُ لِسَكَنِهِ؟ بَلِ الرَّبُّ يَسْكُنُ فِيهِ إِلَى الأَبَدِ.
\par 17 مَرْكَبَاتُ اللهِ رَبَوَاتٌ أُلُوفٌ مُكَرَّرَةٌ. الرَّبُّ فِيهَا. سِينَا فِي الْقُدْسِ.
\par 18 صَعِدْتَ إِلَى الْعَلاَءِ. سَبَيْتَ سَبْياً. قَبِلْتَ عَطَايَا بَيْنَ النَّاسِ وَأَيْضاً الْمُتَمَرِّدِينَ لِلسَّكَنِ أَيُّهَا الرَّبُّ الإِلَهُ.
\par 19 مُبَارَكٌ الرَّبُّ يَوْماً فَيَوْماً. يُحَمِّلُنَا إِلَهُ خَلاَصِنَا. سِلاَهْ.
\par 20 اَللهُ لَنَا إِلَهُ خَلاَصٍ وَعِنْدَ الرَّبِّ السَّيِّدِ لِلْمَوْتِ مَخَارِجُ.
\par 21 وَلَكِنَّ اللهَ يَسْحَقُ رُؤُوسَ أَعْدَائِهِ الْهَامَةَ الشَّعْرَاءَ لِلسَّالِكِ فِي ذُنُوبِهِ.
\par 22 قَالَ الرَّبُّ: [مِنْ بَاشَانَ أُرْجِعُ. أُرْجِعُ مِنْ أَعْمَاقِ الْبَحْرِ
\par 23 لِكَيْ تَصْبِغَ رِجْلَكَ بِالدَّمِ. أَلْسُنُ كِلاَبِكَ مِنَ الأَعْدَاءِ نَصِيبُهُمْ].
\par 24 رَأُوا طُرُقَكَ يَا اللهُ طُرُقَ إِلَهِي مَلِكِي فِي الْقُدْسِ.
\par 25 مِنْ قُدَّامٍ الْمُغَنُّونَ. مِنْ وَرَاءٍ ضَارِبُو الأَوْتَارِ. فِي الْوَسَطِ فَتَيَاتٌ ضَارِبَاتُ الدُّفُوفِ.
\par 26 فِي الْجَمَاعَاتِ بَارِكُوا اللهَ الرَّبَّ أَيُّهَا الْخَارِجُونَ مِنْ عَيْنِ إِسْرَائِيلَ.
\par 27 هُنَاكَ بِنْيَامِينُ الصَّغِيرُ مُتَسَلِّطُهُمْ رُؤَسَاءُ يَهُوذَا جُلُّهُمْ رُؤَسَاءُ زَبُولُونَ رُؤَسَاءُ نَفْتَالِي.
\par 28 قَدْ أَمَرَ إِلَهُكَ بِعِزِّكَ. أَيِّدْ يَا اللهُ هَذَا الَّذِي فَعَلْتَهُ لَنَا.
\par 29 مِنْ هَيْكَلِكَ فَوْقَ أُورُشَلِيمَ لَكَ تُقَدِّمُ مُلُوكٌ هَدَايَا.
\par 30 انْتَهِرْ وَحْشَ الْقَصَبِ صِوَارَ الثِّيرَانِ مَعَ عُجُولِ الشُّعُوبِ الْمُتَرَامِينَ بِقِطَعِ فِضَّةٍ. شَتِّتِ الشُّعُوبَ الَّذِينَ يُسَرُّونَ بِالْقِتَالِ.
\par 31 يَأْتِي شُرَفَاءُ مِنْ مِصْرَ. كُوشُ تُسْرِعُ بِيَدَيْهَا إِلَى اللهِ.
\par 32 يَا مَمَالِكَ الأَرْضِ غَنُّوا لِلَّهِ. رَنِّمُوا لِلسَّيِّدِ. سِلاَهْ.
\par 33 لِلرَّاكِبِ عَلَى سَمَاءِ السَّمَاوَاتِ الْقَدِيمَةِ. هُوَذَا يُعْطِي صَوْتَهُ صَوْتَ قُوَّةٍ.
\par 34 أَعْطُوا عِزّاً لِلَّهِ. عَلَى إِسْرَائِيلَ جَلاَلُهُ وَقُوَّتُهُ فِي الْغَمَامِ.
\par 35 مَخُوفٌ أَنْتَ يَا اللهُ مِنْ مَقَادِسِكَ. إِلَهُ إِسْرَائِيلَ هُوَ الْمُعْطِي قُوَّةً وَشِدَّةً لِلشَّعْبِ. مُبَارَكٌ اللهُ!

\chapter{69}

\par 1 لإِمَامِ الْمُغَنِّينَ. عَلَى السَّوْسَنِّ. لِدَاوُدَ خَلِّصْنِي يَا اللهُ لأَنَّ الْمِيَاهَ قَدْ دَخَلَتْ إِلَى نَفْسِي.
\par 2 غَرِقْتُ فِي حَمْأَةٍ عَمِيقَةٍ وَلَيْسَ مَقَرٌّ. دَخَلْتُ إِلَى أَعْمَاقِ الْمِيَاهِ وَالسَّيْلُ غَمَرَنِي.
\par 3 تَعِبْتُ مِنْ صُرَاخِي. يَبِسَ حَلْقِي. كَلَّتْ عَيْنَايَ مِنِ انْتِظَارِ إِلَهِي.
\par 4 أَكْثَرُ مِنْ شَعْرِ رَأْسِي الَّذِينَ يُبْغِضُونَنِي بِلاَ سَبَبٍ. اعْتَزَّ مُسْتَهْلِكِيَّ أَعْدَائِي ظُلْماً. حِينَئِذٍ رَدَدْتُ الَّذِي لَمْ أَخْطَفْهُ.
\par 5 يَا اللهُ أَنْتَ عَرَفْتَ حَمَاقَتِي وَذُنُوبِي عَنْكَ لَمْ تَخْفَ.
\par 6 لاَ يَخْزَ بِي مُنْتَظِرُوكَ يَا سَيِّدُ رَبَّ الْجُنُودِ. لاَ يَخْجَلْ بِي مُلْتَمِسُوكَ يَا إِلَهَ إِسْرَائِيلَ.
\par 7 لأَنِّي مِنْ أَجْلِكَ احْتَمَلْتُ الْعَارَ. غَطَّى الْخَجَلُ وَجْهِي.
\par 8 صِرْتُ أَجْنَبِيّاً عِنْدَ إِخْوَتِي وَغَرِيباً عِنْدَ بَنِي أُمِّي.
\par 9 لأَنَّ غَيْرَةَ بَيْتِكَ أَكَلَتْنِي وَتَعْيِيرَاتِ مُعَيِّرِيكَ وَقَعَتْ عَلَيَّ.
\par 10 وَأَبْكَيْتُ بِصَوْمٍ نَفْسِي فَصَارَ ذَلِكَ عَاراً عَلَيَّ.
\par 11 جَعَلْتُ لِبَاسِي مِسْحاً وَصِرْتُ لَهُمْ مَثَلاً.
\par 12 يَتَكَلَّمُ فِيَّ الْجَالِسُونَ فِي الْبَابِ وَأَغَانِيُّ شَرَّابِي الْمُسْكِرِ.
\par 13 أَمَّا أَنَا فَلَكَ صَلاَتِي يَا رَبُّ فِي وَقْتِ رِضًى. يَا اللهُ بِكَِثْرَةِ رَحْمَتِكَ اسْتَجِبْ لِي بِحَقِّ خَلاَصِكَ.
\par 14 نَجِّنِي مِنَ الطِّينِ فَلاَ أَغْرَقَ. نَجِّنِي مِنْ مُبْغِضِيَّ وَمِنْ أَعْمَاقِ الْمِيَاهِ.
\par 15 لاَ يَغْمُرَنِّي سَيْلُ الْمِيَاهِ وَلاَ يَبْتَلِعَنِّي الْعُمْقُ وَلاَ تُطْبِقِ الْهَاوِيَةُ عَلَيَّ فَاهَا.
\par 16 اسْتَجِبْ لِي يَا رَبُّ لأَنَّ رَحْمَتَكَ صَالِحَةٌ. كَكَِثْرَةِ مَرَاحِمِكَ الْتَفِتْ إِلَيَّ.
\par 17 وَلاَ تَحْجُبْ وَجْهَكَ عَنْ عَبْدِكَ لأَنَّ لِي ضِيقاً. اسْتَجِبْ لِي سَرِيعاً.
\par 18 اقْتَرِبْ إِلَى نَفْسِي. فُكَّهَا. بِسَبَبِ أَعْدَائِي افْدِنِي.
\par 19 أَنْتَ عَرَفْتَ عَارِي وَخِزْيِي وَخَجَلِي. قُدَّامَكَ جَمِيعُ مُضَايِقِيَّ.
\par 20 الْعَارُ قَدْ كَسَرَ قَلْبِي فَمَرِضْتُ. انْتَظَرْتُ رِقَّةً فَلَمْ تَكُنْ وَمُعَزِّينَ فَلَمْ أَجِدْ.
\par 21 وَيَجْعَلُونَ فِي طَعَامِي عَلْقَماً وَفِي عَطَشِي يَسْقُونَنِي خَلاًّ.
\par 22 لِتَصِرْ مَائِدَتُهُمْ قُدَّامَهُمْ فَخّاً وَلِلآمِنِينَ شَرَكاً.
\par 23 لِتُظْلِمْ عُيُونُهُمْ عَنِ الْبَصَرِ وَقَلْقِلْ مُتُونَهُمْ دَائِماً.
\par 24 صُبَّ عَلَيْهِمْ سَخَطَكَ وَلْيُدْرِكْهُمْ حُمُوُّ غَضَبِكَ.
\par 25 لِتَصِرْ دَارُهُمْ خَرَاباً وَفِي خِيَامِهِمْ لاَ يَكُنْ سَاكِنٌ.
\par 26 لأَنَّ الَّذِي ضَرَبْتَهُ أَنْتَ هُمْ طَرَدُوهُ وَبِوَجَعِ الَّذِينَ جَرَحْتَهُمْ يَتَحَدَّثُونَ.
\par 27 اِجْعَلْ إِثْماً عَلَى إِثْمِهِمْ وَلاَ يَدْخُلُوا فِي بِرِّكَ.
\par 28 لِيُمْحَوْا مِنْ سِفْرِ الأَحْيَاءِ وَمَعَ الصِّدِّيقِينَ لاَ يُكْتَبُوا.
\par 29 أَمَّا أَنَا فَمِسْكِينٌ وَكَئِيبٌ. خَلاَصُكَ يَا اللهُ فَلْيُرَفِّعْنِي.
\par 30 أُسَبِّحُ اسْمَ اللهِ بِتَسْبِيحٍ وَأُعَظِّمُهُ بِحَمْدٍ.
\par 31 فَيُسْتَطَابُ عِنْدَ الرَّبِّ أَكْثَرَ مِنْ ثَوْرِ بَقَرٍ ذِي قُرُونٍ وَأَظْلاَفٍ.
\par 32 يَرَى ذَلِكَ الْوُدَعَاءُ فَيَفْرَحُونَ وَتَحْيَا قُلُوبُكُمْ يَا طَالِبِي اللهِ.
\par 33 لأَنَّ الرَّبَّ سَامِعٌ لِلْمَسَاكِينِ وَلاَ يَحْتَقِرُ أَسْرَاهُ.
\par 34 تُسَبِّحُهُ السَّمَاوَاتُ وَالأَرْضُ الْبِحَارُ وَكُلُّ مَا يَدِبُّ فِيهَا.
\par 35 لأَنَّ اللهَ يُخَلِّصُ صِهْيَوْنَ وَيَبْنِي مُدُنَ يَهُوذَا فَيَسْكُنُونَ هُنَاكَ وَيَرِثُونَهَا.
\par 36 وَنَسْلُ عَبِيدِهِ يَمْلِكُونَهَا وَمُحِبُّو اسْمِهِ يَسْكُنُونَ فِيهَا.

\chapter{70}

\par 1 لإِمَامِ الْمُغَنِّينَ. لِدَاوُدَ لِلتَّذْكِيرِ اَللهُمَّ إِلَى تَنْجِيَتِي يَا رَبُّ إِلَى مَعُونَتِي أَسْرِعْ.
\par 2 لِيَخْزَ وَيَخْجَلْ طَالِبُو نَفْسِي. لِيَرْتَدَّ إِلَى خَلْفٍ وَيَخْجَلِ الْمُشْتَهُونَ لِي شَرّاً.
\par 3 لِيَرْجِعْ مِنْ أَجْلِ خِزْيِهِمُ الْقَائِلُونَ: [هَهْ هَهْ!]
\par 4 وَلْيَبْتَهِجْ وَيَفْرَحْ بِكَ كُلُّ طَالِبِيكَ وَلْيَقُلْ دَائِماً مُحِبُّو خَلاَصِكَ: [لِيَتَعَظَّمِ الرَّبُّ!]
\par 5 أَمَّا أَنَا فَمِسْكِينٌ وَفَقِيرٌ. اللهُمَّ أَسْرِعْ إِلَيَّ. مُعِينِي وَمُنْقِذِي أَنْتَ. يَا رَبُّ لاَ تَبْطُؤْ.

\chapter{71}

\par 1 بِكَ يَا رَبُّ احْتَمَيْتُ فَلاَ أَخْزَى إِلَى الدَّهْرِ.
\par 2 بِعَدْلِكَ نَجِّنِي وَأَنْقِذْنِي. أَمِلْ إِلَيَّ أُذْنَكَ وَخَلِّصْنِي.
\par 3 كُنْ لِي صَخْرَةَ مَلْجَأٍ أَدْخُلُهُ دَائِماً. أَمَرْتَ بِخَلاَصِي لأَنَّكَ صَخْرَتِي وَحِصْنِي.
\par 4 يَا إِلَهِي نَجِّنِي مِنْ يَدِ الشِّرِّيرِ مِنْ كَفِّ فَاعِلِ الشَّرِّ وَالظَّالِمِ.
\par 5 لأَنَّكَ أَنْتَ رَجَائِي يَا سَيِّدِي. الرَّبَّ مُتَّكَلِي مُنْذُ صِبَايَ.
\par 6 عَلَيْكَ اسْتَنَدْتُ مِنَ الْبَطْنِ وَأَنْتَ مُخْرِجِي مِنْ أَحْشَاءِ أُمِّي. بِكَ تَسْبِيحِي دَائِماً.
\par 7 صِرْتُ كَآيَةٍ لِكَثِيرِينَ. أَمَّا أَنْتَ فَمَلْجَإِي الْقَوِيُّ.
\par 8 يَمْتَلِئُ فَمِي مِنْ تَسْبِيحِكَ الْيَوْمَ كُلَّهُ مِنْ مَجْدِكَ.
\par 9 لاَ تَرْفُضْنِي فِي زَمَنِ الشَّيْخُوخَةِ. لاَ تَتْرُكْنِي عِنْدَ فَنَاءِ قُوَّتِي.
\par 10 لأَنَّ أَعْدَائِي تَقَاوَلُوا عَلَيَّ وَالَّذِينَ يَرْصُدُونَ نَفْسِي تَآمَرُوا مَعاً
\par 11 قَائِلِينَ: [إِنَّ اللهَ قَدْ تَرَكَهُ. الْحَقُوهُ وَأَمْسِكُوهُ لأَنَّهُ لاَ مُنْقِذَ لَهُ].
\par 12 يَا اللهُ لاَ تَبْعُدْ عَنِّي. يَا إِلَهِي إِلَى مَعُونَتِي أَسْرِعْ.
\par 13 لِيَخْزَ وَيَفْنَ مُخَاصِمُو نَفْسِي. لِيَلْبِسِ الْعَارَ وَالْخَجَلَ الْمُلْتَمِسُونَ لِي شَرّاً.
\par 14 أَمَّا أَنَا فَأَرْجُو دَائِماً وَأَزِيدُ عَلَى كُلِّ تَسْبِيحِكَ.
\par 15 فَمِي يُحَدِّثُ بِعَدْلِكَ الْيَوْمَ كُلَّهُ بِخَلاَصِكَ لأَنِّي لاَ أَعْرِفُ لَهَا أَعْدَاداً.
\par 16 آتِي بِجَبَرُوتِ السَّيِّدِ الرَّبِّ. أَذْكُرُ بِرَّكَ وَحْدَكَ.
\par 17 اَللهُمَّ قَدْ عَلَّمْتَنِي مُنْذُ صِبَايَ وَإِلَى الآنَ أُخْبِرُ بِعَجَائِبِكَ.
\par 18 وَأَيْضاً إِلَى الشَّيْخُوخَةِ وَالشَّيْبِ يَا اللهُ لاَ تَتْرُكْنِي حَتَّى أُخْبِرَ بِذِرَاعِكَ الْجِيلَ الْمُقْبِلَ وَبِقُوَّتِكَ كُلَّ آتٍ.
\par 19 وَبِرُّكَ إِلَى الْعَلْيَاءِ يَا اللهُ الَّذِي صَنَعْتَ الْعَظَائِمَ. يَا اللهُ مَنْ مِثْلُكَ!
\par 20 أَنْتَ الَّذِي أَرَيْتَنَا ضِيقَاتٍ كَثِيرَةً وَرَدِيئَةً تَعُودُ فَتُحْيِينَا وَمِنْ أَعْمَاقِ الأَرْضِ تَعُودُ فَتُصْعِدُنَا.
\par 21 تَزِيدُ عَظَمَتِي وَتَرْجِعُ فَتُعَزِّينِي.
\par 22 فَأَنَا أَيْضاً أَحْمَدُكَ بِرَبَابٍ حَقَّكَ يَا إِلَهِي. أُرَنِّمُ لَكَ بِالْعُودِ يَا قُدُّوسَ إِسْرَائِيلَ.
\par 23 تَبْتَهِجُ شَفَتَايَ إِذْ أُرَنِّمُ لَكَ وَنَفْسِي الَّتِي فَدَيْتَهَا
\par 24 وَلِسَانِي أَيْضاً الْيَوْمَ كُلَّهُ يَلْهَجُ بِبِرِّكَ. لأَنَّهُ قَدْ خَزِيَ لأَنَّهُ قَدْ خَجِلَ الْمُلْتَمِسُونَ لِي شَرّاً.

\chapter{72}

\par 1 لِسُلَيْمَانَ اَللهُمَّ أَعْطِ أَحْكَامَكَ لِلْمَلِكِ وَبِرَّكَ لاِبْنِ الْمَلِكِ.
\par 2 يَدِينُ شَعْبَكَ بِالْعَدْلِ وَمَسَاكِينَكَ بِالْحَقِّ.
\par 3 تَحْمِلُ الْجِبَالُ سَلاَماً لِلشَّعْبِ وَالآكَامُ بِالْبِرِّ.
\par 4 يَقْضِي لِمَسَاكِينِ الشَّعْبِ. يُخَلِّصُ بَنِي الْبَائِسِينَ وَيَسْحَقُ الظَّالِمَ.
\par 5 يَخْشُونَكَ مَا دَامَتِ الشَّمْسُ وَقُدَّامَ الْقَمَرِ إِلَى دَوْرٍ فَدَوْرٍ.
\par 6 يَنْزِلُ مِثْلَ الْمَطَرِ عَلَى الْجُزَازِ وَمِثْلَ الْغُيُوثِ الذَّارِفَةِ عَلَى الأَرْضِ.
\par 7 يُشْرِقُ فِي أَيَّامِهِ الصِّدِّيقُ وَكَثْرَةُ السَّلاَمِ إِلَى أَنْ يَضْمَحِلَّ الْقَمَرُ.
\par 8 وَيَمْلِكُ مِنَ الْبَحْرِ إِلَى الْبَحْرِ وَمِنَ النَّهْرِ إِلَى أَقَاصِي الأَرْضِ.
\par 9 أَمَامَهُ تَجْثُو أَهْلُ الْبَرِّيَّةِ وَأَعْدَاؤُهُ يَلْحَسُونَ التُّرَابَ.
\par 10 مُلُوكُ تَرْشِيشَ وَالْجَزَائِرِ يُرْسِلُونَ تَقْدِمَةً. مُلُوكُ شَبَا وَسَبَأٍ يُقَدِّمُونَ هَدِيَّةً
\par 11 وَيَسْجُدُ لَهُ كُلُّ الْمُلُوكِ. كُلُّ الأُمَمِ تَتَعَبَّدُ لَهُ
\par 12 لأَنَّهُ يُنَجِّي الْفَقِيرَ الْمُسْتَغِيثَ وَالْمَِسْكِينَ إِذْ لاَ مُعِينَ لَهُ.
\par 13 يُشْفِقُ عَلَى الْمَِسْكِينِ وَالْبَائِسِ وَيُخَلِّصُ أَنْفُسَ الْفُقَرَاءِ.
\par 14 مِنَ الظُّلْمِ وَالْخَطْفِ يَفْدِي أَنْفُسَهُمْ وَيُكْرَمُ دَمُهُمْ فِي عَيْنَيْهِ.
\par 15 وَيَعِيشُ وَيُعْطِيهِ مِنْ ذَهَبِ شَبَا. وَيُصَلِّي لأَجْلِهِ دَائِماً. الْيَوْمَ كُلَّهُ يُبَارِكُهُ.
\par 16 تَكُونُ حُفْنَةُ بُرٍّ فِي الأَرْضِ فِي رُؤُوسِ الْجِبَالِ. تَتَمَايَلُ مِثْلَ لُبْنَانَ ثَمَرَتُهَا وَيُزْهِرُونَ مِنَ الْمَدِينَةِ مِثْلَ عُشْبِ الأَرْضِ.
\par 17 يَكُونُ اسْمُهُ إِلَى الدَّهْرِ. قُدَّامَ الشَّمْسِ يَمْتَدُّ اسْمُهُ. وَيَتَبَارَكُونَ بِهِ. كُلُّ أُمَمِ الأَرْضِ يُطَوِّبُونَهُ.
\par 18 مُبَارَكٌ الرَّبُّ اللهُ إِلَهُ إِسْرَائِيلَ الصَّانِعُ الْعَجَائِبَ وَحْدَهُ.
\par 19 وَمُبَارَكٌ اسْمُ مَجْدِهِ إِلَى الدَّهْرِ وَلِْتَمْتَلِئِ الأَرْضُ كُلُّهَا مِنْ مَجْدِهِ. آمِينَ ثُمَّ آمِينَ.
\par 20 تَمَّتْ صَلَوَاتُ دَاوُدَ بْنِ يَسَّى

\chapter{73}

\par 1 مَزْمُورٌ. لآسَافَ إِنَّمَا صَالِحٌ اللهُ لإِسْرَائِيلَ لأَنْقِيَاءِ الْقَلْبِ.
\par 2 أَمَّا أَنَا فَكَادَتْ تَزِلُّ قَدَمَايَ. لَوْلاَ قَلِيلٌ لَزَلِقَتْ خَطَوَاتِي
\par 3 لأَنِّي غِرْتُ مِنَ الْمُتَكَبِّرِينَ إِذْ رَأَيْتُ سَلاَمَةَ الأَشْرَارِ.
\par 4 لأَنَّهُ لَيْسَتْ فِي مَوْتِهِمْ شَدَائِدُ وَجِسْمُهُمْ سَمِينٌ.
\par 5 لَيْسُوا فِي تَعَبِ النَّاسِ وَمَعَ الْبَشَرِ لاَ يُصَابُونَ.
\par 6 لِذَلِكَ تَقَلَّدُوا الْكِبْرِيَاءَ. لَبِسُوا كَثَوْبٍ ظُلْمَهُمْ.
\par 7 جَحَظَتْ عُيُونُهُمْ مِنَ الشَّحْمِ. جَاوَزُوا تَصَوُّرَاتِ الْقَلْبِ.
\par 8 يَسْتَهْزِئُونَ وَيَتَكَلَّمُونَ بِالشَّرِّ ظُلْماً. مِنَ الْعَلاَءِ يَتَكَلَّمُونَ.
\par 9 جَعَلُوا أَفْوَاهَهُمْ فِي السَّمَاءِ وَأَلْسِنَتُهُمْ تَتَمَشَّى فِي الأَرْضِ.
\par 10 لِذَلِكَ يَرْجِعُ شَعْبُهُ إِلَى هُنَا وَكَمِيَاهٍ مُرْوِيَةٍ يُمْتَصُّونَ مِنْهُمْ.
\par 11 وَقَالُوا: [كَيْفَ يَعْلَمُ اللهُ وَهَلْ عِنْدَ الْعَلِيِّ مَعْرِفَةٌ؟]
\par 12 هُوَذَا هَؤُلاَءِ هُمُ الأَشْرَارُ وَمُسْتَرِيحِينَ إِلَى الدَّهْرِ يُكْثِرُونَ ثَرْوَةً.
\par 13 حَقّاً قَدْ زَكَّيْتُ قَلْبِي بَاطِلاً وَغَسَلْتُ بِالنَّقَاوَةِ يَدَيَّ.
\par 14 وَكُنْتُ مُصَاباً الْيَوْمَ كُلَّهُ وَتَأَدَّبْتُ كُلَّ صَبَاحٍ.
\par 15 لَوْ قُلْتُ أُحَدِّثُ هَكَذَا لَغَدَرْتُ بِجِيلِ بَنِيكَ.
\par 16 فَلَمَّا قَصَدْتُ مَعْرِفَةَ هَذَا إِذَا هُوَ تَعَبٌ فِي عَيْنَيَّ.
\par 17 حَتَّى دَخَلْتُ مَقَادِسَ اللهِ وَانْتَبَهْتُ إِلَى آخِرَتِهِمْ.
\par 18 حَقّاً فِي مَزَالِقَ جَعَلْتَهُمْ. أَسْقَطْتَهُمْ إِلَى الْبَوَارِ.
\par 19 كَيْفَ صَارُوا لِلْخَرَابِ بَغْتَةً! اضْمَحَلُّوا فَنُوا مِنَ الدَّوَاهِي.
\par 20 كَحُلْمٍ عِنْدَ التَّيَقُّظِ يَا رَبُّ عِنْدَ التَّيَقُّظِ تَحْتَقِرُ خَيَالَهُمْ.
\par 21 لأَنَّهُ تَمَرْمَرَ قَلْبِي وَانْتَخَسْتُ فِي كُلْيَتَيَّ.
\par 22 وَأَنَا بَلِيدٌ وَلاَ أَعْرِفُ. صِرْتُ كَبَهِيمٍ عِنْدَكَ.
\par 23 وَلَكِنِّي دَائِماً مَعَكَ. أَمْسَكْتَ بِيَدِي الْيُمْنَى.
\par 24 بِرَأْيِكَ تَهْدِينِي وَبَعْدُ إِلَى مَجْدٍ تَأْخُذُنِي.
\par 25 مَنْ لِي فِي السَّمَاءِ؟ وَمَعَكَ لاَ أُرِيدُ شَيْئاً فِي الأَرْضِ.
\par 26 قَدْ فَنِيَ لَحْمِي وَقَلْبِي. صَخْرَةُ قَلْبِي وَنَصِيبِي اللهُ إِلَى الدَّهْرِ.
\par 27 لأَنَّهُ هُوَذَا الْبُعَدَاءُ عَنْكَ يَبِيدُونَ. تُهْلِكُ كُلَّ مَنْ يَزْنِي عَنْكَ.
\par 28 أَمَّا أَنَا فَالاِقْتِرَابُ إِلَى اللهِ حَسَنٌ لِي. جَعَلْتُ بِالسَّيِّدِ الرَّبِّ مَلْجَإِي لِأُخْبِرَ بِكُلِّ صَنَائِعِكَ.

\chapter{74}

\par 1 قَصِيدَةٌ لآسَافَ لِمَاذَا رَفَضْتَنَا يَا اللهُ إِلَى الأَبَدِ؟ لِمَاذَا يُدَخِّنُ غَضَبُكَ عَلَى غَنَمِ مَرْعَاكَ؟
\par 2 اذْكُرْ جَمَاعَتَكَ الَّتِي اقْتَنَيْتَهَا مُنْذُ الْقِدَمِ وَفَدَيْتَهَا سِبْطَ مِيرَاثِكَ جَبَلَ صِهْيَوْنَ هَذَا الَّذِي سَكَنْتَ فِيهِ.
\par 3 ارْفَعْ خَطَوَاتِكَ إِلَى الْخِرَبِ الأَبَدِيَّةِ. الْكُلَّ قَدْ حَطَّمَ الْعَدُوُّ فِي الْمَقْدِسِ.
\par 4 قَدْ زَمْجَرَ مُقَاوِمُوكَ فِي وَسَطِ مَعْهَدِكَ جَعَلُوا آيَاتِهِمْ آيَاتٍ.
\par 5 يَبَانُ كَأَنَّهُ رَافِعُ فُؤُوسٍ عَلَى الأَشْجَارِ الْمُشْتَبِكَةِ.
\par 6 وَالآنَ مَنْقُوشَاتِهِ مَعاً بِالْفُؤُوسِ وَالْمَعَاوِلِ يَكْسِرُونَ.
\par 7 أَطْلَقُوا النَّارَ فِي مَقْدِسِكَ. دَنَّسُوا لِلأَرْضِ مَسْكَنَ اسْمِكَ.
\par 8 قَالُوا فِي قُلُوبِهِمْ: [لِنُفْنِينَّهُمْ مَعاً]. أَحْرَقُوا كُلَّ مَعَاهِدِ اللهِ فِي الأَرْضِ.
\par 9 آيَاتِنَا لاَ نَرَى. لاَ نَبِيَّ بَعْدُ. وَلاَ بَيْنَنَا مَنْ يَعْرِفُ حَتَّى مَتَى.
\par 10 حَتَّى مَتَى يَا اللهُ يُعَيِّرُ الْمُقَاوِمُ وَيُهِينُ الْعَدُوُّ اسْمَكَ إِلَى الْغَايَةِ؟
\par 11 لِمَاذَا تَرُدُّ يَدَكَ وَيَمِينَكَ؟ أَخْرِجْهَا مِنْ وَسَطِ حِضْنِكَ. أَفْنِ.
\par 12 وَاللهُ مَلِكِي مُنْذُ الْقِدَمِ فَاعِلُ الْخَلاَصِ فِي وَسَطِ الأَرْضِ.
\par 13 أَنْتَ شَقَقْتَ الْبَحْرَ بِقُوَّتِكَ. كَسَرْتَ رُؤُوسَ التَّنَانِينِ عَلَى الْمِيَاهِ.
\par 14 أَنْتَ رَضَضْتَ رُؤُوسَ لَوِيَاثَانَ. جَعَلْتَهُ طَعَاماً لِلشَّعْبِ لأَهْلِ الْبَرِّيَّةِ.
\par 15 أَنْتَ فَجَّرْتَ عَيْناً وَسَيْلاً. أَنْتَ يَبَّسْتَ أَنْهَاراً دَائِمَةَ الْجَرَيَانِ.
\par 16 لَكَ النَّهَارُ وَلَكَ أَيْضاً اللَّيْلُ. أَنْتَ هَيَّأْتَ النُّورَ وَالشَّمْسَ.
\par 17 أَنْتَ نَصَبْتَ كُلَّ تُخُومِ الأَرْضِ. الصَّيْفَ وَالشِّتَاءَ أَنْتَ خَلَقْتَهُمَا.
\par 18 اُذْكُرْ هَذَا: أَنَّ الْعَدُوَّ قَدْ عَيَّرَ الرَّبَّ وَشَعْباً جَاهِلاً قَدْ أَهَانَ اسْمَكَ.
\par 19 لاَ تُسَلِّمْ لِلْوَحْشِ نَفْسَ يَمَامَتِكَ. قَطِيعَ بَائِسِيكَ لاَ تَنْسَ إِلَى الأَبَدِ.
\par 20 انْظُرْ إِلَى الْعَهْدِ. لأَنَّ مُظْلِمَاتِ الأَرْضِ امْتَلَأَتْ مِنْ مَسَاكِنِ الظُّلْمِ.
\par 21 لاَ يَرْجِعَنَّ الْمُنْسَحِقُ خَازِياً. الْفَقِيرُ وَالْبَائِسُ لِيُسَبِّحَا اسْمَكَ.
\par 22 قُمْ يَا اللهُ. أَقِمْ دَعْوَاكَ. اذْكُرْ تَعْيِيرَ الْجَاهِلِ إِيَّاكَ الْيَوْمَ كُلَّهُ.
\par 23 لاَ تَنْسَ صَوْتَ أَضْدَادِكَ ضَجِيجَ مُقَاوِمِيكَ الصَّاعِدَ دَائِماً.

\chapter{75}

\par 1 لإِمَامِ الْمُغَنِّينَ. عَلَى [لاَ تُهْلِكْ]. مَزْمُورٌ لآسَافَ. تَسْبِيحَةٌ نَحْمَدُكَ يَا اللهُ. نَحْمَدُكَ وَاسْمُكَ قَرِيبٌ. يُحَدِّثُونَ بِعَجَائِبِكَ.
\par 2 [لأَنِّي أُعَيِّنُ مِيعَاداً. أَنَا بِالْمُسْتَقِيمَاتِ أَقْضِي.
\par 3 ذَابَتِ الأَرْضُ وَكُلُّ سُكَّانِهَا. أَنَا وَزَنْتُ أَعْمِدَتَهَا]. سِلاَهْ.
\par 4 قُلْتُ لِلْمُفْتَخِرِينَ: [لاَ تَفْتَخِرُوا] وَلِلأَشْرَارِ: [لاَ تَرْفَعُوا قَرْناً.
\par 5 لاَ تَرْفَعُوا إِلَى الْعُلَى قَرْنَكُمْ. لاَ تَتَكَلَّمُوا بِعُنُقٍ مُتَصَلِّبٍ].
\par 6 لأَنَّهُ لاَ مِنَ الْمَشْرِقِ وَلاَ مِنَ الْمَغْرِبِ وَلاَ مِنْ بَرِّيَّةِ الْجِبَالِ.
\par 7 وَلَكِنَّ اللهَ هُوَ الْقَاضِي. هَذَا يَضَعُهُ وَهَذَا يَرْفَعُهُ.
\par 8 لأَنَّ فِي يَدِ الرَّبِّ كَأْساً وَخَمْرُهَا مُخْتَمِرَةٌ. مَلآنَةٌ شَرَاباً مَمْزُوجاً. وَهُوَ يَسْكُبُ مِنْهَا. لَكِنْ عَكَرُهَا يَمَصُّهُ يَشْرَبُهُ كُلُّ أَشْرَارِ الأَرْضِ.
\par 9 أَمَّا أَنَا فَأُخْبِرُ إِلَى الدَّهْرِ. أُرَنِّمُ لإِلَهِ يَعْقُوبَ.
\par 10 وَكُلَّ قُرُونِ الأَشْرَارِ أَعْضِبُ. قُرُونُ الصِّدِّيقِ تَنْتَصِبُ.

\chapter{76}

\par 1 لإِمَامِ الْمُغَنِّينَ عَلَى ذَوَاتِ الأَوْتَارِ. مَزْمُورٌ لآسَافَ. تَسْبِيحَةٌ اَللهُ مَعْرُوفٌ فِي يَهُوذَا. اسْمُهُ عَظِيمٌ فِي إِسْرَائِيلَ.
\par 2 كَانَتْ فِي سَالِيمَ مَظَلَّتُهُ وَمَسْكَنُهُ فِي صِهْيَوْنَ.
\par 3 هُنَاكَ سَحَقَ الْقِسِيَّ الْبَارِقَةَ. الْمِجَنَّ وَالسَّيْفَ وَالْقِتَالَ. سِلاَهْ.
\par 4 أَبْهَى أَنْتَ أَمْجَدُ مِنْ جِبَالِ السَّلَبِ.
\par 5 سُلِبَ أَشِدَّاءُ الْقَلْبِ. نَامُوا سِنَتَهُمْ. كُلُّ رِجَالِ الْبَأْسِ لَمْ يَجِدُوا أَيْدِيَهُمْ.
\par 6 مِنِ انْتِهَارِكَ يَا إِلَهَ يَعْقُوبَ يُسَبَّخُ فَارِسٌ وَخَيْلٌ.
\par 7 أَنْتَ مَهُوبٌ أَنْتَ. فَمَنْ يَقِفُ قُدَّامَكَ حَالَ غَضَبِكَ؟
\par 8 مِنَ السَّمَاءِ أَسْمَعْتَ حُكْماً. الأَرْضُ فَزِعَتْ وَسَكَتَتْ
\par 9 عِنْدَ قِيَامِ اللهِ لِلْقَضَاءِ لِتَخْلِيصِ كُلِّ وُدَعَاءِ الأَرْضِ. سِلاَهْ.
\par 10 لأَنَّ غَضَبَ الإِنْسَانِ يَحْمَدُكَ. بَقِيَّةُ الْغَضَبِ تَتَمَنْطَقُ بِهَا.
\par 11 اُنْذُرُوا وَأَوْفُوا لِلرَّبِّ إِلَهِكُمْ يَا جَمِيعَ الَّذِينَ حَوْلَهُ. لِيُقَدِّمُوا هَدِيَّةً لِلْمَهُوبِ.
\par 12 يَقْطِفُ رُوحَ الرُّؤَسَاءِ. هُوَ مَهُوبٌ لِمُلُوكِ الأَرْضِ.

\chapter{77}

\par 1 لإِمَامِ الْمُغَنِّينَ عَلَى [يَدُوثُونَ]. لآسَافَ. مَزْمُورٌ صَوْتِي إِلَى اللهِ فَأَصْرُخُ. صَوْتِي إِلَى اللهِ فَأَصْغَى إِلَيَّ.
\par 2 فِي يَوْمِ ضِيقِي الْتَمَسْتُ الرَّبَّ. يَدِي فِي اللَّيْلِ انْبَسَطَتْ وَلَمْ تَخْدَرْ. أَبَتْ نَفْسِي التَّعْزِيَةَ.
\par 3 أَذْكُرُ اللهَ فَأَئِنُّ. أُنَاجِي نَفْسِي فَيُغْشَى عَلَى رُوحِي. سِلاَهْ.
\par 4 أَمْسَكْتَ أَجْفَانَ عَيْنَيَّ. انْزَعَجْتُ فَلَمْ أَتَكَلَّمْ.
\par 5 تَفَكَّرْتُ فِي أَيَّامِ الْقِدَمِ السِّنِينَ الدَّهْرِيَّةِ.
\par 6 أَذْكُرُ تَرَنُّمِي فِي اللَّيْلِ. مَعَ قَلْبِي أُنَاجِي وَرُوحِي تَبْحَثُ.
\par 7 هَلْ إِلَى الدُّهُورِ يَرْفُضُ الرَّبُّ وَلاَ يَعُودُ لِلرِّضَا بَعْدُ؟
\par 8 هَلِ انْتَهَتْ إِلَى الأَبَدِ رَحْمَتُهُ؟ هَلِ انْقَطَعَتْ كَلِمَتُهُ إِلَى دَوْرٍ فَدَوْرٍ؟
\par 9 هَلْ نَسِيَ اللهُ رَأْفَةً أَوْ قَفَصَ بِرِجْزِهِ مَرَاحِمَهُ؟ سِلاَهْ.
\par 10 فَقُلْتُ: [هَذَا مَا يُعِلُّنِي: تَغَيُّرُ يَمِينِ الْعَلِيِّ].
\par 11 أَذْكُرُ أَعْمَالَ الرَّبِّ إِذْ أَتَذَكَّرُ عَجَائِبَكَ مُنْذُ الْقِدَمِ
\par 12 وَأَلْهَجُ بِجَمِيعِ أَفْعَالِكَ وَبِصَنَائِعِكَ أُنَاجِي.
\par 13 اَللهُمَّ فِي الْقُدْسِ طَرِيقُكَ. أَيُّ إِلَهٍ عَظِيمٌ مِثْلُ اللهِ!
\par 14 أَنْتَ الإِلَهُ الصَّانِعُ الْعَجَائِبَ. عَرَّفْتَ بَيْنَ الشُّعُوبِ قُوَّتَكَ.
\par 15 فَكَكْتَ بِذِرَاعِكَ شَعْبَكَ بَنِي يَعْقُوبَ وَيُوسُفَ. سِلاَهْ.
\par 16 أَبْصَرَتْكَ الْمِيَاهُ يَا اللهُ أَبْصَرَتْكَ الْمِيَاهُ فَفَزِعَتْ. ارْتَعَدَتْ أَيْضاً اللُّجَجُ.
\par 17 سَكَبَتِ الْغُيُومُ مِيَاهاً. أَعْطَتِ السُّحُبُ صَوْتاً. أَيْضاً سِهَامُكَ طَارَتْ.
\par 18 صَوْتُ رَعْدِكَ فِي الزَّوْبَعَةِ. الْبُرُوقُ أَضَاءَتِ الْمَسْكُونَةَ. ارْتَعَدَتْ وَرَجَفَتِ الأَرْضُ.
\par 19 فِي الْبَحْرِ طَرِيقُكَ وَسُبُلُكَ فِي الْمِيَاهِ الْكَثِيرَةِ وَآثَارُكَ لَمْ تُعْرَفْ.
\par 20 هَدَيْتَ شَعْبَكَ كَالْغَنَمِ بِيَدِ مُوسَى وَهَارُونَ.

\chapter{78}

\par 1 قَصِيدَةٌ لآسَافَ اِصْغَ يَا شَعْبِي إِلَى شَرِيعَتِي. أَمِيلُوا آذَانَكُمْ إِلَى كَلاَمِ فَمِي.
\par 2 أَفْتَحُ بِمَثَلٍ فَمِي. أُذِيعُ أَلْغَازاً مُنْذُ الْقِدَمِ.
\par 3 الَّتِي سَمِعْنَاهَا وَعَرَفْنَاهَا وَآبَاؤُنَا أَخْبَرُونَا.
\par 4 لاَ نُخْفِي عَنْ بَنِيهِمْ إِلَى الْجِيلِ الآخِرِ مُخْبِرِينَ بِتَسَابِيحِ الرَّبِّ وَقُوَّتِهِ وَعَجَائِبِهِ الَّتِي صَنَعَ.
\par 5 أَقَامَ شَهَادَةً فِي يَعْقُوبَ وَوَضَعَ شَرِيعَةً فِي إِسْرَائِيلَ الَّتِي أَوْصَى آبَاءَنَا أَنْ يُعَرِّفُوا بِهَا أَبْنَاءَهُمْ
\par 6 لِكَيْ يَعْلَمَ الْجِيلُ الآخِرُ. بَنُونَ يُولَدُونَ فَيَقُومُونَ وَيُخْبِرُونَ أَبْنَاءَهُمْ
\par 7 فَيَجْعَلُونَ عَلَى اللّهِ اعْتِمَادَهُم وَلاَ يَنْسُونَ أَعْمَالَ اللهِ بَلْ يَحْفَظُونَ وَصَايَاهُ
\par 8 وَلاَ يَكُونُونَ مِثْلَ آبَائِهِمْ جِيلاً زَائِغاً وَمَارِداً جِيلاً لَمْ يُثَبِّتْ قَلْبَهُ وَلَمْ تَكُنْ رُوحُهُ أَمِينَةً لِلَّهِ.
\par 9 بَنُو أَفْرَايِمَ النَّازِعُونَ فِي الْقَوْسِ الرَّامُونَ انْقَلَبُوا فِي يَوْمِ الْحَرْبِ.
\par 10 لَمْ يَحْفَظُوا عَهْدَ اللهِ وَأَبُوا السُّلُوكَ فِي شَرِيعَتِهِ
\par 11 وَنَسُوا أَفْعَالَهُ وَعَجَائِبَهُ الَّتِي أَرَاهُمْ.
\par 12 قُدَّامَ آبَائِهِمْ صَنَعَ أُعْجُوبَةً فِي أَرْضِ مِصْرَ بِلاَدِ صُوعَنَ.
\par 13 شَقَّ الْبَحْرَ فَعَبَّرَهُمْ وَنَصَبَ الْمِيَاهَ كَنَدٍّ.
\par 14 وَهَدَاهُمْ بِالسَّحَابِ نَهَاراً وَاللَّيْلَ كُلَّهُ بِنُورِ نَارٍ.
\par 15 شَقَّ صُخُوراً فِي الْبَرِّيَّةِ وَسَقَاهُمْ كَأَنَّهُ مِنْ لُجَجٍ عَظِيمَةٍ.
\par 16 أَخْرَجَ مَجَارِيَ مِنْ صَخْرَةٍ وَأَجْرَى مِيَاهاً كَالأَنْهَارِ.
\par 17 ثُمَّ عَادُوا أَيْضاً لِيُخْطِئُوا إِلَيْهِ لِعِصْيَانِ الْعَلِيِّ فِي الأَرْضِ النَّاشِفَةِ.
\par 18 وَجَرَّبُوا اللهَ فِي قُلُوبِهِمْ بِسُؤَالِهِمْ طَعَاماً لِشَهْوَتِهِمْ.
\par 19 فَوَقَعُوا فِي اللهِ. قَالُوا: [هَلْ يَقْدِرُ اللهُ أَنْ يُرَتِّبَ مَائِدَةً فِي الْبَرِّيَّةِ؟
\par 20 هُوَذَا ضَرَبَ الصَّخْرَةَ فَجَرَتِ الْمِيَاهُ وَفَاضَتِ الأَوْدِيَةُ. هَلْ يَقْدِرُ أَيْضاً أَنْ يُعْطِيَ خُبْزاً أَوْ يُهَيِّئَ لَحْماً لِشَعْبِهِ؟]
\par 21 لِذَلِكَ سَمِعَ الرَّبُّ فَغَضِبَ وَاشْتَعَلَتْ نَارٌ فِي يَعْقُوبَ وَسَخَطٌ أَيْضاً صَعِدَ عَلَى إِسْرَائِيلَ
\par 22 لأَنَّهُمْ لَمْ يُؤْمِنُوا بِاللهِ وَلَمْ يَتَّكِلُوا عَلَى خَلاَصِهِ.
\par 23 فَأَمَرَ السَّحَابَ مِنْ فَوْقُ وَفَتَحَ مَصَارِيعَ السَّمَاوَاتِ
\par 24 وَأَمْطَرَ عَلَيْهِمْ مَنّاً لِلأَكْلِ وَبُرَّ السَّمَاءِ أَعْطَاهُمْ.
\par 25 أَكَلَ الإِنْسَانُ خُبْزَ الْمَلاَئِكَةِ. أَرْسَلَ عَلَيْهِمْ زَاداً لِلشِّبَعِ.
\par 26 أَهَاجَ رِيحَاً شَرْقِيَّةً فِي السَّمَاءِ وَسَاقَ بِقُوَّتِهِ جَنُوبِيَّةً
\par 27 وَأَمْطَرَ عَلَيْهِمْ لَحْماً مِثْلَ التُّرَابِ وَكَرَمْلِ الْبَحْرِ طُيُوراً ذَوَاتِ أَجْنِحَةٍ.
\par 28 وَأَسْقَطَهَا فِي وَسَطِ مَحَلَّتِهِمْ حَوَالَيْ مَسَاكِنِهِمْ.
\par 29 فَأَكَلُوا وَشَبِعُوا جِدّاً وَأَتَاهُمْ بِشَهْوَتِهِمْ.
\par 30 لَمْ يَزُوغُوا عَنْ شَهْوَتِهِمْ. طَعَامُهُمْ بَعْدُ فِي أَفْوَاهِهِمْ
\par 31 فَصَعِدَ عَلَيْهِمْ غَضَبُ اللهِ وَقَتَلَ مِنْ أَسْمَنِهِمْ. وَصَرَعَ مُخْتَارِي إِسْرَائِيلَ.
\par 32 فِي هَذَا كُلِّهِ أَخْطَأُوا بَعْدُ وَلَمْ يُؤْمِنُوا بِعَجَائِبِهِ.
\par 33 فَأَفْنَى أَيَّامَهُمْ بِالْبَاطِلِ وَسِنِيهِمْ بِالرُّعْبِ.
\par 34 إِذْ قَتَلَهُمْ طَلَبُوهُ وَرَجَعُوا وَبَكَّرُوا إِلَى اللهِ
\par 35 وَذَكَرُوا أَنَّ اللهَ صَخْرَتُهُمْ وَاللهَ الْعَلِيَّ وَلِيُّهُمْ.
\par 36 فَخَادَعُوهُ بِأَفْوَاهِهِمْ وَكَذَبُوا عَلَيْهِ بِأَلْسِنَتِهِمْ.
\par 37 أَمَّا قُلُوبُهُمْ فَلَمْ تُثَبَّتْ مَعَهُ وَلَمْ يَكُونُوا أُمَنَاءَ فِي عَهْدِهِ.
\par 38 أَمَّا هُوَ فَرَأُوفٌ يَغْفِرُ الإِثْمَ وَلاَ يُهْلِكُ وَكَثِيراً مَا رَدَّ غَضَبَهُ وَلَمْ يُشْعِلْ كُلَّ سَخَطِهِ.
\par 39 ذَكَرَ أَنَّهُمْ بَشَرٌ. رِيحٌ تَذْهَبُ وَلاَ تَعُودُ.
\par 40 كَمْ عَصُوهُ فِي الْبَرِّيَّةِ وَأَحْزَنُوهُ فِي الْقَفْرِ!
\par 41 رَجَعُوا وَجَرَّبُوا اللهَ وَعَنُّوا قُدُّوسَ إِسْرَائِيلَ.
\par 42 لَمْ يَذْكُرُوا يَدَهُ يَوْمَ فَدَاهُمْ مِنَ الْعَدُوِّ
\par 43 حَيْثُ جَعَلَ فِي مِصْرَ آيَاتِهِ وَعَجَائِبَهُ فِي بِلاَدِ صُوعَنَ
\par 44 إِذْ حَوَّلَ خُلْجَانَهُمْ إِلَى دَمٍ وَمَجَارِيَهُمْ لِكَيْ لاَ يَشْرَبُوا.
\par 45 أَرْسَلَ عَلَيْهِمْ بَعُوضاً فَأَكَلَهُمْ وَضَفَادِعَ فَأَفْسَدَتْهُمْ.
\par 46 أَسْلَمَ لِلْجَرْدَمِ غَلَّتَهُمْ وَتَعَبَهُمْ لِلْجَرَادِ.
\par 47 أَهْلَكَ بِالْبَرَدِ كُرُومَهُمْ وَجُمَّيْزَهُمْ بِالصَّقِيعِ.
\par 48 وَدَفَعَ إِلَى الْبَرَدِ بَهَائِمَهُمْ وَمَوَاشِيَهُمْ لِلْبُرُوقِ.
\par 49 أَرْسَلَ عَلَيْهِمْ حُمُوَّ غَضَبِهِ سَخَطاً وَرِجْزاً وَضِيقاً جَيْشَ مَلاَئِكَةٍ أَشْرَارٍ.
\par 50 مَهَّدَ سَبِيلاً لِغَضَبِهِ. لَمْ يَمْنَعْ مِنَ الْمَوْتِ أَنْفُسَهُمْ بَلْ دَفَعَ حَيَاتَهُمْ لِلْوَبَإِ.
\par 51 وَضَرَبَ كُلَّ بِكْرٍ فِي مِصْرَ. أَوَائِلَ الْقُدْرَةِ فِي خِيَامِ حَامٍ.
\par 52 وَسَاقَ مِثْلَ الْغَنَمِ شَعْبَهُ وَقَادَهُمْ مِثْلَ قَطِيعٍ فِي الْبَرِّيَّةِ.
\par 53 وَهَدَاهُمْ آمِنِينَ فَلَمْ يَجْزَعُوا. أَمَّا أَعْدَاؤُهُمْ فَغَمَرَهُمُ الْبَحْرُ.
\par 54 وَأَدْخَلَهُمْ فِي تُخُومِ قُدْسِهِ هَذَا الْجَبَلِ الَّذِي اقْتَنَتْهُ يَمِينُهُ.
\par 55 وَطَرَدَ الأُمَمَ مِنْ قُدَّامِهِمْ وَقَسَمَهُمْ بِالْحَبْلِ مِيرَاثاً وَأَسْكَنَ فِي خِيَامِهِمْ أَسْبَاطَ إِسْرَائِيلَ.
\par 56 فَجَرَّبُوا وَعَصُوا اللهَ الْعَلِيَّ وَشَهَادَاتِهِ لَمْ يَحْفَظُوا
\par 57 بَلِ ارْتَدُّوا وَغَدَرُوا مِثْلَ آبَائِهِمْ. انْحَرَفُوا كَقَوْسٍ مُخْطِئَةٍ.
\par 58 أَغَاظُوهُ بِمُرْتَفَعَاتِهِمْ وَأَغَارُوهُ بِتَمَاثِيلِهِمْ.
\par 59 سَمِعَ اللهُ فَغَضِبَ وَرَذَلَ إِسْرَائِيلَ جِدّاً
\par 60 وَرَفَضَ مَسْكَنَ شِيلُوهَ الْخَيْمَةَ الَّتِي نَصَبَهَا بَيْنَ النَّاسِ.
\par 61 وَسَلَّمَ لِلسَّبْيِ عِزَّهُ وَجَلاَلَهُ لِيَدِ الْعَدُوِّ.
\par 62 وَدَفَعَ إِلَى السَّيْفِ شَعْبَهُ وَغَضِبَ عَلَى مِيرَاثِهِ.
\par 63 مُخْتَارُوهُ أَكَلَتْهُمُ النَّارُ وَعَذَارَاهُ لَمْ يُحْمَدْنَ.
\par 64 كَهَنَتُهُ سَقَطُوا بِالسَّيْفِ وَأَرَامِلُهُ لَمْ يَبْكِينَ.
\par 65 فَاسْتَيْقَظَ الرَّبُّ كَنَائِمٍ كَجَبَّارٍ مُعَيِّطٍ مِنَ الْخَمْرِ.
\par 66 فَضَرَبَ أَعْدَاءَهُ إِلَى الْوَرَاءِ. جَعَلَهُمْ عَاراً أَبَدِيّاً.
\par 67 وَرَفَضَ خَيْمَةَ يُوسُفَ وَلَمْ يَخْتَرْ سِبْطَ أَفْرَايِمَ.
\par 68 بَلِ اخْتَارَ سِبْطَ يَهُوذَا جَبَلَ صِهْيَوْنَ الَّذِي أَحَبَّهُ.
\par 69 وَبَنَى مِثْلَ مُرْتَفَعَاتٍ مَقْدِسَهُ كَالأَرْضِ الَّتِي أَسَّسَهَا إِلَى الأَبَدِ.
\par 70 وَاخْتَارَ دَاوُدَ عَبْدَهُ وَأَخَذَهُ مِنْ حَظَائِرِ الْغَنَمِ.
\par 71 مِنْ خَلْفِ الْمُرْضِعَاتِ أَتَى بِهِ لِيَرْعَى يَعْقُوبَ شَعْبَهُ وَإِسْرَائِيلَ مِيرَاثَهُ.
\par 72 فَرَعَاهُمْ حَسَبَ كَمَالِ قَلْبِهِ وَبِمَهَارَةِ يَدَيْهِ هَدَاهُمْ.

\chapter{79}

\par 1 مَزْمُورٌ. لآسَافَ اَللهُمَّ إِنَّ الأُمَمَ قَدْ دَخَلُوا مِيرَاثَكَ. نَجَّسُوا هَيْكَلَ قُدْسِكَ. جَعَلُوا أُورُشَلِيمَ أَكْوَاماً.
\par 2 دَفَعُوا جُثَثَ عَبِيدِكَ طَعَاماً لِطُيُورِ السَّمَاءِ لَحْمَ أَتْقِيَائِكَ لِوُحُوشِ الأَرْضِ.
\par 3 سَفَكُوا دَمَهُمْ كَالْمَاءِ حَوْلَ أُورُشَلِيمَ وَلَيْسَ مَنْ يَدْفِنُ.
\par 4 صِرْنَا عَاراً عِنْدَ جِيرَانِنَا هُزْءاً وَسُخْرَةً لِلَّذِينَ حَوْلَنَا.
\par 5 إِلَى مَتَى يَا رَبُّ تَغْضَبُ كُلَّ الْغَضَبِ وَتَتَّقِدُ كَالنَّارِ غَيْرَتُكَ؟
\par 6 أَفِضْ رِجْزَكَ عَلَى الأُمَمِ الَّذِينَ لاَ يَعْرِفُونَكَ وَعَلَى الْمَمَالِكِ الَّتِي لَمْ تَدْعُ بِاسْمِكَ.
\par 7 لأَنَّهُمْ قَدْ أَكَلُوا يَعْقُوبَ وَأَخْرَبُوا مَسْكَنَهُ.
\par 8 لاَ تَذْكُرْ عَلَيْنَا ذُنُوبَ الأَوَّلِينَ. لِتَتَقَدَّمْنَا مَرَاحِمُكَ سَرِيعاً لأَنَّنَا قَدْ تَذَلَّلْنَا جِدّاً.
\par 9 أَعِنَّا يَا إِلَهَ خَلاَصِنَا مِنْ أَجْلِ مَجْدِ اسْمِكَ وَنَجِّنَا وَاغْفِرْ خَطَايَانَا مِنْ أَجْلِ اسْمِكَ.
\par 10 لِمَاذَا يَقُولُ الأُمَمُ: [أَيْنَ هُوَ إِلَهُهُمْ؟] لِتُعْرَفْ عِنْدَ الأُمَمِ قُدَّامَ أَعْيُنِنَا نَقْمَةُ دَمِ عَبِيدِكَ الْمُهْرَاقِ.
\par 11 لِيَدْخُلْ قُدَّامَكَ أَنِينُ الأَسِيرِ. كَعَظَمَةِ ذِرَاعِكَ اسْتَبْقِ بَنِي الْمَوْتِ.
\par 12 وَرُدَّ عَلَى جِيرَانِنَا سَبْعَةَ أَضْعَافٍ فِي أَحْضَانِهِمِ الْعَارَ الَّذِي عَيَّرُوكَ بِهِ يَا رَبُّ.
\par 13 أَمَّا نَحْنُ شَعْبُكَ وَغَنَمُ رِعَايَتِكَ نَحْمَدُكَ إِلَى الدَّهْرِ. إِلَى دَوْرٍ فَدَوْرٍ نُحَدِّثُ بِتَسْبِيحِكَ.

\chapter{80}

\par 1 لإِمَامِ الْمُغَنِّينَ عَلَى السَّوْسَنِّ. شَهَادَةٌ. لآسَافَ. مَزْمُورٌ يَا رَاعِيَ إِسْرَائِيلَ اصْغَ يَا قَائِدَ يُوسُفَ كَالضَّأْنِ يَا جَالِساً عَلَى الْكَرُوبِيمِ أَشْرِقْ.
\par 2 قُدَّامَ أَفْرَايِمَ وَبِنْيَامِينَ وَمَنَسَّى أَيْقِظْ جَبَرُوتَكَ وَهَلُمَّ لِخَلاَصِنَا.
\par 3 يَا اللهُ أَرْجِعْنَا وَأَنِرْ بِوَجْهِكَ فَنَخْلُصَ.
\par 4 يَا رَبُّ إِلَهَ الْجُنُودِ إِلَى مَتَى تُدَخِّنُ عَلَى صَلاَةِ شَعْبِكَ؟
\par 5 قَدْ أَطْعَمْتَهُمْ خُبْزَ الدُّمُوعِ وَسَقَيْتَهُمُ الدُّمُوعَ بِالْكَيْلِ.
\par 6 جَعَلْتَنَا نِزَاعاً عِنْدَ جِيرَانِنَا وَأَعْدَاؤُنَا يَسْتَهْزِئُونَ بَيْنَ أَنْفُسِهِمْ.
\par 7 يَا إِلَهَ الْجُنُودِ أَرْجِعْنَا وَأَنِرْ بِوَجْهِكَ فَنَخْلُصَ.
\par 8 كَرْمَةً مِنْ مِصْرَ نَقَلْتَ. طَرَدْتَ أُمَماً وَغَرَسْتَهَا.
\par 9 هَيَّأْتَ قُدَّامَهَا فَأَصَّلَتْ أُصُولَهَا فَمَلَأَتِ الأَرْضَ.
\par 10 غَطَّى الْجِبَالَ ظِلُّهَا وَأَغْصَانُهَا أَرْزَ اللهِ.
\par 11 مَدَّتْ قُضْبَانَهَا إِلَى الْبَحْرِ وَإِلَى النَّهْرِ فُرُوعَهَا.
\par 12 فَلِمَاذَا هَدَمْتَ جُدْرَانَهَا فَيَقْطِفَهَا كُلُّ عَابِرِي الطَّرِيقِ؟
\par 13 يُفْسِدُهَا الْخِنْزِيرُ مِنَ الْوَعْرِ وَيَرْعَاهَا وَحْشُ الْبَرِّيَّةِ!
\par 14 يَا إِلَهَ الْجُنُودِ ارْجِعَنَّ. اطَّلِعْ مِنَ السَّمَاءِ وَانْظُرْ وَتَعَهَّدْ هَذِهِ الْكَرْمَةَ
\par 15 وَالْغَرْسَ الَّذِي غَرَسَتْهُ يَمِينُكَ وَالاِبْنَ الَّذِي اخْتَرْتَهُ لِنَفْسِكَ.
\par 16 هِيَ مَحْرُوقَةٌ بِنَارٍ مَقْطُوعَةٌ. مِنِ انْتِهَارِ وَجْهِكَ يَبِيدُونَ.
\par 17 لِتَكُنْ يَدُكَ عَلَى رَجُلِ يَمِينِكَ وَعَلَى ابْنِ آدَمَ الَّذِي اخْتَرْتَهُ لِنَفْسِكَ
\par 18 فَلاَ نَرْتَدَّ عَنْكَ. أَحْيِنَا فَنَدْعُوَ بِاسْمِكَ.
\par 19 يَا رَبُّ إِلَهَ الْجُنُودِ أَرْجِعْنَا. أَنِرْ بِوَجْهِكَ فَنَخْلُصَ.

\chapter{81}

\par 1 لإِمَامِ الْمُغَنِّينَ عَلَى الْجَتِّيَّةِ. لآسَافَ رَنِّمُوا لِلَّهِ قُوَّتِنَا. اهْتِفُوا لإِلَهِ يَعْقُوبَ.
\par 2 ارْفَعُوا نَغْمَةً وَهَاتُوا دُفّاً عُوداً حُلْواً مَعَ رَبَابٍ.
\par 3 انْفُخُوا فِي رَأْسِ الشَّهْرِ بِالْبُوقِ عِنْدَ الْهِلاَلِ لِيَوْمِ عِيدِنَا.
\par 4 لأَنَّ هَذَا فَرِيضَةٌ لإِسْرَائِيلَ حُكْمٌ لإِلَهِ يَعْقُوبَ.
\par 5 جَعَلَهُ شَهَادَةً فِي يُوسُفَ عِنْدَ خُرُوجِهِ عَلَى أَرْضِ مِصْرَ. سَمِعْتُ لِسَاناً لَمْ أَعْرِفْهُ.
\par 6 [أَبْعَدْتُ مِنَ الْحِمْلِ كَتِفَهُ. يَدَاهُ تَحَوَّلَتَا عَنِ السَّلِّ.
\par 7 فِي الضِّيقِ دَعَوْتَ فَنَجَّيْتُكَ. اسْتَجَبْتُكَ فِي سِتْرِ الرَّعْدِ. جَرَّبْتُكَ عَلَى مَاءِ مَرِيبَةَ]. سِلاَهْ.
\par 8 [اِسْمَعْ يَا شَعْبِي فَأُحَذِّرَكَ. يَا إِسْرَائِيلُ إِنْ سَمِعْتَ لِي.
\par 9 لاَ يَكُنْ فِيكَ إِلَهٌ غَرِيبٌ وَلاَ تَسْجُدْ لإِلَهٍ أَجْنَبِيٍّ.
\par 10 أَنَا الرَّبُّ إِلَهُكَ الَّذِي أَصْعَدَكَ مِنْ أَرْضِ مِصْرَ. أَفْغِرْ فَاكَ فَأَمْلَأَهُ.
\par 11 فَلَمْ يَسْمَعْ شَعْبِي لِصَوْتِي وَإِسْرَائِيلُ لَمْ يَرْضَ بِي.
\par 12 فَسَلَّمْتُهُمْ إِلَى قَسَاوَةِ قُلُوبِهِمْ لِيَسْلُكُوا فِي مُؤَامَرَاتِ أَنْفُسِهِمْ.
\par 13 لَوْ سَمِعَ لِي شَعْبِي وَسَلَكَ إِسْرَائِيلُ فِي طُرُقِي
\par 14 سَرِيعاً كُنْتُ أُخْضِعُ أَعْدَاءَهُمْ وَعَلَى مُضَايِقِيهِمْ كُنْتُ أَرُدُّ يَدِي.
\par 15 مُبْغِضُو الرَّبِّ يَتَذَلَّلُونَ لَهُ وَيَكُونُ وَقْتُهُمْ إِلَى الدَّهْرِ.
\par 16 وَكَانَ أَطْعَمَهُ مِنْ شَحْمِ الْحِنْطَةِ وَمِنَ الصَّخْرَةِ كُنْتُ أُشْبِعُكَ عَسَلاً].

\chapter{82}

\par 1 مَزْمُورٌ لآسَافَ اَللهُ قَائِمٌ فِي مَجْمَعِ اللهِ. فِي وَسَطِ الآلِهَةِ يَقْضِي.
\par 2 حَتَّى مَتَى تَقْضُونَ جَوْراً وَتَرْفَعُونَ وُجُوهَ الأَشْرَارِ؟ سِلاَهْ.
\par 3 اِقْضُوا لِلذَّلِيلِ وَلِلْيَتِيمِ. أَنْصِفُوا الْمَِسْكِينَ وَالْبَائِسَ.
\par 4 نَجُّوا الْمَِسْكِينَ وَالْفَقِيرَ. مِنْ يَدِ الأَشْرَارِ أَنْقِذُوا.
\par 5 لاَ يَعْلَمُونَ وَلاَ يَفْهَمُونَ. فِي الظُّلْمَةِ يَتَمَشُّونَ. تَتَزَعْزَعُ كُلُّ أُسُسِ الأَرْضِ.
\par 6 أَنَا قُلْتُ إِنَّكُمْ آلِهَةٌ وَبَنُو الْعَلِيِّ كُلُّكُمْ.
\par 7 لَكِنْ مِثْلَ النَّاسِ تَمُوتُونَ وَكَأَحَدِ الرُّؤَسَاءِ تَسْقُطُونَ.
\par 8 قُمْ يَا اللهُ. دِنِ الأَرْضَ لأَنَّكَ أَنْتَ تَمْتَلِكُ كُلَّ الأُمَمِ.

\chapter{83}

\par 1 تَسْبِيحَةٌ. مَزْمُورٌ لآسَافَ اَللهُمَّ لاَ تَصْمُتْ لاَ تَسْكُتْ وَلاَ تَهْدَأْ يَا اللهُ
\par 2 فَهُوَذَا أَعْدَاؤُكَ يَعِجُّونَ وَمُبْغِضُوكَ قَدْ رَفَعُوا الرَّأْسَ.
\par 3 عَلَى شَعْبِكَ مَكَرُوا مُؤَامَرَةً وَتَشَاوَرُوا عَلَى أَحْمِيَائِكَ.
\par 4 قَالُوا: [هَلُمَّ نُبِدْهُمْ مِنْ بَيْنِ الشُّعُوبِ وَلاَ يُذْكَرُ اسْمُ إِسْرَائِيلَ بَعْدُ].
\par 5 لأَنَّهُمْ تَآمَرُوا بِالْقَلْبِ مَعاً. عَلَيْكَ تَعَاهَدُوا عَهْداً.
\par 6 خِيَامُ أَدُومَ وَالإِسْمَاعيلِيِّينَ. مُوآبُ وَالْهَاجَرِيُّونَ.
\par 7 جِبَالُ وَعَمُّونُ وَعَمَالِيقُ. فَلَسْطِينُ مَعَ سُكَّانِ صُورٍ.
\par 8 أَشُّورُ أَيْضاً اتَّفَقَ مَعَهُمْ. صَارُوا ذِرَاعاً لِبَنِي لُوطٍ. سِلاَهْ
\par 9 اِفْعَلْ بِهِمْ كَمَا بِمِدْيَانَ كَمَا بِسِيسَرَا كَمَا بِيَابِينَ فِي وَادِي قِيشُونَ.
\par 10 بَادُوا فِي عَيْنِ دُورٍ. صَارُوا دِمْناً لِلأَرْضِ.
\par 11 اجْعَلْ شُرَفَاءَهُمْ مِثْلَ غُرَابٍ وَمِثْلَ ذِئْبٍ. وَمِثْلَ زَبَحَ وَمِثْلَ صَلْمُنَّاعَ كُلَّ أُمَرَائِهِمُِ.
\par 12 الَّذِينَ قَالُوا: [لِنَمْتَلِكْ لأَنْفُسِنَا مَسَاكِنَ اللهِ].
\par 13 يَا إِلَهِي اجْعَلْهُمْ مِثْلَ الْجُلِّ مِثْلَ الْقَشِّ أَمَامَ الرِّيحِ.
\par 14 كَنَارٍ تُحْرِقُ الْوَعْرَ كَلَهِيبٍ يُشْعِلُ الْجِبَالَ.
\par 15 هَكَذَا اطْرُدْهُمْ بِعَاصِفَتِكَ وَبِزَوْبَعَتِكَ رَوِّعْهُمُِ.
\par 16 امْلَأْ وُجُوهَهُمْ خِزْياً فَيَطْلُبُوا اسْمَكَ يَا رَبُّ.
\par 17 لِيَخْزُوا وَيَرْتَاعُوا إِلَى الأَبَدِ وَلْيَخْجَلُوا وَيَبِيدُوا
\par 18 وَيَعْلَمُوا أَنَّكَ اسْمُكَ يَهْوَهُ وَحْدَكَ الْعَلِيُّ عَلَى كُلِّ الأَرْضِ.

\chapter{84}

\par 1 لإِمَامِ الْمُغَنِّينَ عَلَى الْجَتِّيَّةِ. لِبَنِي قُورَحَ. مَزْمُورٌ مَا أَحْلَى مَسَاكِنَكَ يَا رَبَّ الْجُنُودِ.
\par 2 تَشْتَاقُ بَلْ تَتُوقُ نَفْسِي إِلَى دِيَارِ الرَّبِّ. قَلْبِي وَلَحْمِي يَهْتِفَانِ بِالإِلَهِ الْحَيِّ.
\par 3 اَلْعُصْفُورُ أَيْضاً وَجَدَ بَيْتاً وَالسُّنُونَةُ عُشّاً لِنَفْسِهَا حَيْثُ تَضَعُ أَفْرَاخَهَا مَذَابِحَكَ يَا رَبَّ الْجُنُودِ مَلِكِي وَإِلَهِي.
\par 4 طُوبَى لِلسَّاكِنِينَ فِي بَيْتِكَ أَبَداً يُسَبِّحُونَكَ. سِلاَهْ.
\par 5 طُوبَى لِأُنَاسٍ عِزُّهُمْ بِكَ. طُرُقُ بَيْتِكَ فِي قُلُوبِهِمْ.
\par 6 عَابِرِينَ فِي وَادِي الْبُكَاءِ يُصَيِّرُونَهُ يَنْبُوعاً. أَيْضاً بِبَرَكَاتٍ يُغَطُّونَ مُورَةَ.
\par 7 يَذْهَبُونَ مِنْ قُوَّةٍ إِلَى قُوَّةٍ. يُرَوْنَ قُدَّامَ اللهِ فِي صِهْيَوْنَ.
\par 8 يَا رَبُّ إِلَهَ الْجُنُودِ اسْمَعْ صَلاَتِي وَاصْغَ يَا إِلَهَ يَعْقُوبَ. سِلاَهْ.
\par 9 يَا مِجَنَّنَا انْظُرْ يَا اللهُ وَالْتَفِتْ إِلَى وَجْهِ مَسِيحِكَ.
\par 10 لأَنَّ يَوْماً وَاحِداً فِي دِيَارِكَ خَيْرٌ مِنْ أَلْفٍ. اخْتَرْتُ الْوُقُوفَ عَلَى الْعَتَبَةِ فِي بَيْتِ إِلَهِي عَلَى السَّكَنِ فِي خِيَامِ الأَشْرَارِ.
\par 11 لأَنَّ الرَّبَّ اللهَ شَمْسٌ وَمِجَنٌّ. الرَّبُّ يُعْطِي رَحْمَةً وَمَجْداً. لاَ يَمْنَعُ خَيْراً عَنِ السَّالِكِينَ بِالْكَمَالِ.
\par 12 يَا رَبَّ الْجُنُودِ طُوبَى لِلإِنْسَانِ الْمُتَّكِلِ عَلَيْكَ!

\chapter{85}

\par 1 لإِمَامِ الْمُغَنِّينَ. لِبَنِي قُورَحَ. مَزْمُورٌ رَضِيتَ يَا رَبُّ عَلَى أَرْضِكَ. أَرْجَعْتَ سَبْيَ يَعْقُوبَ.
\par 2 غَفَرْتَ إِثْمَ شَعْبِكَ. سَتَرْتَ كُلَّ خَطِيَّتِهِمْ. سِلاَهْ.
\par 3 حَجَزْتَ كُلَّ رِجْزِكَ. رَجَعْتَ عَنْ حُمُوِّ غَضَبِكَ.
\par 4 أَرْجِعْنَا يَا إِلَهَ خَلاَصِنَا وَانْفِ غَضَبَكَ عَنَّا.
\par 5 هَلْ إِلَى الدَّهْرِ تَسْخَطُ عَلَيْنَا؟ هَلْ تُطِيلُ غَضَبَكَ إِلَى دَوْرٍ فَدَوْرٍ؟
\par 6 أَلاَ تَعُودُ أَنْتَ فَتُحْيِينَا فَيَفْرَحَ بِكَ شَعْبُكَ؟
\par 7 أَرِنَا يَا رَبُّ رَحْمَتَكَ وَأَعْطِنَا خَلاَصَكَ.
\par 8 إِنِّي أَسْمَعُ مَا يَتَكَلَّمُ بِهِ اللهُ الرَّبُّ. لأَنَّهُ يَتَكَلَّمُ بِالسَّلاَمِ لِشَعْبِهِ وَلأَتْقِيَائِهِ فَلاَ يَرْجِعُنَّ إِلَى الْحَمَاقَةِ.
\par 9 لأَنَّ خَلاَصَهُ قَرِيبٌ مِنْ خَائِفِيهِ لِيَسْكُنَ الْمَجْدُ فِي أَرْضِنَا.
\par 10 الرَّحْمَةُ وَالْحَقُّ الْتَقَيَا. الْبِرُّ وَالسَّلاَمُ تَلاَثَمَا.
\par 11 الْحَقُّ مِنَ الأَرْضِ يَنْبُتُ وَالْبِرُّ مِنَ السَّمَاءِ يَطَّلِعُ.
\par 12 أَيْضاً الرَّبُّ يُعْطِي الْخَيْرَ وَأَرْضُنَا تُعْطِي غَلَّتَهَا.
\par 13 الْبِرُّ قُدَّامَهُ يَسْلُكُ وَيَطَأُ فِي طَرِيقِ خَطَوَاتِهِ.

\chapter{86}

\par 1 صَلاَةٌ لِدَاوُدَ أَمِلْ يَا رَبُّ أُذْنَكَ. اسْتَجِبْ لِي لأَنِّي مَِسْكِينٌ وَبَائِسٌ أَنَا.
\par 2 احْفَظْ نَفْسِي لأَنِّي تَقِيٌّ. يَا إِلَهِي خَلِّصْ أَنْتَ عَبْدَكَ الْمُتَّكِلَ عَلَيْكَ.
\par 3 ارْحَمْنِي يَا رَبُّ لأَنِّي إِلَيْكَ أَصْرُخُ الْيَوْمَ كُلَّهُ.
\par 4 فَرِّحْ نَفْسَ عَبْدِكَ لأَنَّنِي إِلَيْكَ يَا رَبُّ أَرْفَعُ نَفْسِي.
\par 5 لأَنَّكَ أَنْتَ يَا رَبُّ صَالِحٌ وَغَفُورٌ وَكَثِيرُ الرَّحْمَةِ لِكُلِّ الدَّاعِينَ إِلَيْكَ.
\par 6 اِصْغَ يَا رَبُّ إِلَى صَلاَتِي وَأَنْصِتْ إِلَى صَوْتِ تَضَرُّعَاتِي.
\par 7 فِي يَوْمِ ضِيقِي أَدْعُوكَ لأَنَّكَ تَسْتَجِيبُ لِي.
\par 8 لاَ مِثْلَ لَكَ بَيْنَ الآلِهَةِ يَا رَبُّ وَلاَ مِثْلَ أَعْمَالِكَ.
\par 9 كُلُّ الأُمَمِ الَّذِينَ صَنَعْتَهُمْ يَأْتُونَ وَيَسْجُدُونَ أَمَامَكَ يَا رَبُّ وَيُمَجِّدُونَ اسْمَكَ.
\par 10 لأَنَّكَ عَظِيمٌ أَنْتَ وَصَانِعٌ عَجَائِبَ. أَنْتَ اللهُ وَحْدَكَ.
\par 11 عَلِّمْنِي يَا رَبُّ طَرِيقَكَ أَسْلُكْ فِي حَقِّكَ. وَحِّدْ قَلْبِي لِخَوْفِ اسْمِكَ.
\par 12 أَحْمَدُكَ يَا رَبُّ إِلَهِي مِنْ كُلِّ قَلْبِي وَأُمَجِّدُ اسْمَكَ إِلَى الدَّهْرِ.
\par 13 لأَنَّ رَحْمَتَكَ عَظِيمَةٌ نَحْوِي وَقَدْ نَجَّيْتَ نَفْسِي مِنَ الْهَاوِيَةِ السُّفْلَى.
\par 14 اَللهُمَّ الْمُتَكَبِّرُونَ قَدْ قَامُوا عَلَيَّ وَجَمَاعَةُ الْعُتَاةِ طَلَبُوا نَفْسِي وَلَمْ يَجْعَلُوكَ أَمَامَهُمْ.
\par 15 أَمَّا أَنْتَ يَا رَبُّ فَإِلَهٌ رَحِيمٌ وَرَأُوفٌ طَوِيلُ الرُّوحِ وَكَثِيرُ الرَّحْمَةِ وَالْحَقِّ.
\par 16 الْتَفِتْ إِلَيَّ وَارْحَمْنِي. أَعْطِ عَبْدَكَ قُوَّتَكَ وَخَلِّصِ ابْنَ أَمَتِكَ.
\par 17 اصْنَعْ مَعِي آيَةً لِلْخَيْرِ فَيَرَى ذَلِكَ مُبْغِضِيَّ فَيَخْزُوا لأَنَّكَ أَنْتَ يَا رَبُّ أَعَنْتَنِي وَعَزَّيْتَنِي.

\chapter{87}

\par 1 لِبَنِي قُورَحَ. مَزْمُورُ تَسْبِيحَةٍ أَسَاسُهُ فِي الْجِبَالِ الْمُقَدَّسَةِ.
\par 2 الرَّبُّ أَحَبَّ أَبْوَابَ صِهْيَوْنَ أَكْثَرَ مِنْ جَمِيعِ مَسَاكِنِ يَعْقُوبَ.
\par 3 قَدْ قِيلَ بِكِ أَمْجَادٌ يَا مَدِينَةَ اللهِ. سِلاَهْ
\par 4 أَذْكُرُ رَهَبَ وَبَابِلَ عَارِفَتَيَّ. هُوَذَا فَلَسْطِينُ وَصُورُ مَعَ كُوشَ. هَذَا وُلِدَ هُنَاكَ.
\par 5 وَلِصِهْيَوْنَ يُقَالُ: [هَذَا الإِنْسَانُ وَهَذَا الإِنْسَانُ وُلِدَ فِيهَا وَهِيَ الْعَلِيُّ يُثَبِّتُهَا].
\par 6 الرَّبُّ يَعُدُّ فِي كِتَابَةِ الشُّعُوبِ أَنَّ هَذَا وُلِدَ هُنَاكَ. سِلاَهْ.
\par 7 وَمُغَنُّونَ كَعَازِفِينَ كُلُّ السُّكَّانِ فِيكِ.

\chapter{88}

\par 1 تَسْبِيحَةٌ. مَزْمُورٌ لِبَنِي قُورَحَ. لإِمَامِ الْمُغَنِّينَ عَلَى الْعُودِ لِلْغِنَاءِ. قَصِيدَةٌ لِهَيْمَانَ الأَزْرَاحِيِّ يَا رَبُّ إِلَهَ خَلاَصِي بِالنَّهَارِ وَاللَّيْلِ صَرَخْتُ أَمَامَكَ
\par 2 فَلْتَأْتِ قُدَّامَكَ صَلاَتِي. أَمِلْ أُذْنَكَ إِلَى صُرَاخِي
\par 3 لأَنَّهُ قَدْ شَبِعَتْ مِنَ الْمَصَائِبِ نَفْسِي وَحَيَاتِي إِلَى الْهَاوِيَةِ دَنَتْ.
\par 4 حُسِبْتُ مِثْلَ الْمُنْحَدِرِينَ إِلَى الْجُبِّ. صِرْتُ كَرَجُلٍ لاَ قُوَّةَ لَهُ.
\par 5 بَيْنَ الأَمْوَاتِ فِرَاشِي مِثْلُ الْقَتْلَى الْمُضْطَجِعِينَ فِي الْقَبْرِ الَّذِينَ لاَ تَذْكُرُهُمْ بَعْدُ وَهُمْ مِنْ يَدِكَ انْقَطَعُوا.
\par 6 وَضَعْتَنِي فِي الْجُبِّ الأَسْفَلِ فِي ظُلُمَاتٍ فِي أَعْمَاقٍ.
\par 7 عَلَيَّ اسْتَقَرَّ غَضَبُكَ وَبِكُلِّ تَيَّارَاتِكَ ذَلَّلْتَنِي. سِلاَهْ.
\par 8 أَبْعَدْتَ عَنِّي مَعَارِفِي. جَعَلْتَنِي رِجْساً لَهُمْ. أُغْلِقَ عَلَيَّ فَمَا أَخْرُجُ.
\par 9 عَيْنِي ذَابَتْ مِنَ الذُّلِّ. دَعَوْتُكَ يَا رَبُّ كُلَّ يَوْمٍ. بَسَطْتُ إِلَيْكَ يَدَيَّ.
\par 10 أَفَلَعَلَّكَ لِلأَمْوَاتِ تَصْنَعُ عَجَائِبَ أَمِ الأَخِيلَةُ تَقُومُ تُمَجِّدُكَ؟ سِلاَهْ.
\par 11 هَلْ يُحَدَّثُ فِي الْقَبْرِ بِرَحْمَتِكَ أَوْ بِحَقِّكَ فِي الْهَلاَكِ؟
\par 12 هَلْ تُعْرَفُ فِي الظُّلْمَةِ عَجَائِبُكَ وَبِرُّكَ فِي أَرْضِ النِّسْيَانِ؟
\par 13 أَمَّا أَنَا فَإِلَيْكَ يَا رَبُّ صَرَخْتُ وَفِي الْغَدَاةِ صَلاَتِي تَتَقَدَّمُكَ.
\par 14 لِمَاذَا يَا رَبُّ تَرْفُضُ نَفْسِي؟ لِمَاذَا تَحْجُبُ وَجْهَكَ عَنِّي؟
\par 15 أَنَا مَِسْكِينٌ وَمُسَلِّمُ الرُّوحِ مُنْذُ صِبَايَ. احْتَمَلْتُ أَهْوَالَكَ. تَحَيَّرْتُ.
\par 16 عَلَيَّ عَبَرَ سَخَطُكَ. أَهْوَالُكَ أَهْلَكَتْنِي.
\par 17 أَحَاطَتْ بِي كَالْمِيَاهِ الْيَوْمَ كُلَّهُ. اكْتَنَفَتْنِي مَعاً.
\par 18 أَبْعَدْتَ عَنِّي مُحِبّاً وَصَاحِباً. مَعَارِفِي فِي الظُّلْمَةِ.

\chapter{89}

\par 1 قَصِيدَةٌ لأَيْثَانَ الأَزْرَاحِيِّ بِمَرَاحِمِ الرَّبِّ أُغَنِّي إِلَى الدَّهْرِ. لِدَوْرٍ فَدَوْرٍ أُخْبِرُ عَنْ حَقِّكَ بِفَمِي.
\par 2 لأَنِّي قُلْتُ: [إِنَّ الرَّحْمَةَ إِلَى الدَّهْرِ تُبْنَى. السَّمَاوَاتُ تُثْبِتُ فِيهَا حَقَّكَ].
\par 3 [قَطَعْتُ عَهْداً مَعَ مُخْتَارِي. حَلَفْتُ لِدَاوُدَ عَبْدِي.
\par 4 إِلَى الدَّهْرِ أُثَبِّتُ نَسْلَكَ وَأَبْنِي إِلَى دَوْرٍ فَدَوْرٍ كُرْسِيَّكَ]. سِلاَهْ.
\par 5 وَالسَّمَاوَاتُ تَحْمَدُ عَجَائِبَكَ يَا رَبُّ وَحَقَّكَ أَيْضاً فِي جَمَاعَةِ الْقِدِّيسِينَ.
\par 6 لأَنَّهُ مَنْ فِي السَّمَاءِ يُعَادِلُ الرَّبَّ. مَنْ يُشْبِهُ الرَّبَّ بَيْنَ أَبْنَاءِ اللهِ؟
\par 7 إِلَهٌ مَهُوبٌ جِدّاً فِي مُؤَامَرَةِ الْقِدِّيسِينَ وَمَخُوفٌ عِنْدَ جَمِيعِ الَّذِينَ حَوْلَهُ.
\par 8 يَا رَبُّ إِلَهَ الْجُنُودِ مَنْ مِثْلُكَ قَوِيٌّ رَبٌّ وَحَقُّكَ مِنْ حَوْلِكَ؟
\par 9 أَنْتَ مُتَسَلِّطٌ عَلَى كِبْرِيَاءِ الْبَحْرِ. عِنْدَ ارْتِفَاعِ لُجَجِهِ أَنْتَ تُسَكِّنُهَا.
\par 10 أَنْتَ سَحَقْتَ رَهَبَ مِثْلَ الْقَتِيلِ. بِذِرَاعِ قُوَّتِكَ بَدَّدْتَ أَعْدَاءَكَ.
\par 11 لَكَ السَّمَاوَاتُ. لَكَ أَيْضاً الأَرْضُ. الْمَسْكُونَةُ وَمِلْؤُهَا أَنْتَ أَسَّسْتَهُمَا.
\par 12 الشِّمَالُ وَالْجَنُوبُ أَنْتَ خَلَقْتَهُمَا. تَابُورُ وَحَرْمُونُ بِاسْمِكَ يَهْتِفَانِ.
\par 13 لَكَ ذِرَاعُ الْقُدْرَةِ. قَوِيَّةٌ يَدُكَ. مُرْتَفِعَةٌ يَمِينُكَ.
\par 14 الْعَدْلُ وَالْحَقُّ قَاعِدَةُ كُرْسِيِّكَ. الرَّحْمَةُ وَالأَمَانَةُ تَتَقَدَّمَانِ أَمَامَ وَجْهِكَ.
\par 15 طُوبَى لِلشَّعْبِ الْعَارِفِينَ الْهُتَافَ. يَا رَبُّ بِنُورِ وَجْهِكَ يَسْلُكُونَ.
\par 16 بِاسْمِكَ يَبْتَهِجُونَ الْيَوْمَ كُلَّهُ وَبِعَدْلِكَ يَرْتَفِعُونَ.
\par 17 لأَنَّكَ أَنْتَ فَخْرُ قُوَّتِهِمْ وَبِرِضَاكَ يَنْتَصِبُ قَرْنُنَا.
\par 18 لأَنَّ الرَّبَّ مِجَنُّنَا وَقُدُّوسَ إِسْرَائِيلَ مَلِكُنَا.
\par 19 حِينَئِذٍ كَلَّمْتَ بِرُؤْيَا تَقِيَّكَ وَقُلْتَ جَعَلْتُ: [عَوْناً عَلَى قَوِيٍّ. رَفَعْتُ مُخْتَاراً مِنْ بَيْنِ الشَّعْبِ.
\par 20 وَجَدْتُ دَاوُدَ عَبْدِي. بِدُهْنِ قُدْسِي مَسَحْتُهُ.
\par 21 الَّذِي تَثْبُتُ يَدِي مَعَهُ. أَيْضاً ذِرَاعِي تُشَدِّدُهُ.
\par 22 لاَ يُرْغِمُهُ عَدُوٌّ وَابْنُ الإِثْمِ لاَ يُذَلِّلُهُ.
\par 23 وَأَسْحَقُ أَعْدَاءَهُ أَمَامَ وَجْهِهِ وَأَضْرِبُ مُبْغِضِيهِ.
\par 24 أَمَّا أَمَانَتِي وَرَحْمَتِي فَمَعَهُ وَبِاسْمِي يَنْتَصِبُ قَرْنُهُ.
\par 25 وَأَجْعَلُ عَلَى الْبَحْرِ يَدَهُ وَعَلَى الأَنْهَارِ يَمِينَهُ.
\par 26 هُوَ يَدْعُونِي: أَبِي أَنْتَ. إِلَهِي وَصَخْرَةُ خَلاَصِي.
\par 27 أَنَا أَيْضاً أَجْعَلُهُ بِكْراً أَعْلَى مِنْ مُلُوكِ الأَرْضِ.
\par 28 إِلَى الدَّهْرِ أَحْفَظُ لَهُ رَحْمَتِي. وَعَهْدِي يُثَبَّتُ لَهُ.
\par 29 وَأَجْعَلُ إِلَى الأَبَدِ نَسْلَهُ وَكُرْسِيَّهُ مِثْلَ أَيَّامِ السَّمَاوَاتِ.
\par 30 إِنْ تَرَكَ بَنُوهُ شَرِيعَتِي وَلَمْ يَسْلُكُوا بِأَحْكَامِي
\par 31 إِنْ نَقَضُوا فَرَائِضِي وَلَمْ يَحْفَظُوا وَصَايَايَ
\par 32 أَفْتَقِدُ بِعَصاً مَعْصِيَتَهُمْ وَبِضَرَبَاتٍ إِثْمَهُمْ.
\par 33 أَمَّا رَحْمَتِي فَلاَ أَنْزِعُهَا عَنْهُ وَلاَ أَكْذِبُ مِنْ جِهَةِ أَمَانَتِي.
\par 34 لاَ أَنْقُضُ عَهْدِي وَلاَ أُغَيِّرُ مَا خَرَجَ مِنْ شَفَتَيَّ.
\par 35 مَرَّةً حَلَفْتُ بِقُدْسِي أَنِّي لاَ أَكْذِبُ لِدَاوُدَ.
\par 36 نَسْلُهُ إِلَى الدَّهْرِ يَكُونُ وَكُرْسِيُّهُ كَالشَّمْسِ أَمَامِي.
\par 37 مِثْلَ الْقَمَرِ يُثَبَّتُ إِلَى الدَّهْرِ. وَالشَّاهِدُ فِي السَّمَاءِ أَمِينٌ]. سِلاَهْ
\par 38 لَكِنَّكَ رَفَضْتَ وَرَذَلْتَ. غَضِبْتَ عَلَى مَسِيحِكَ.
\par 39 نَقَضْتَ عَهْدَ عَبْدِكَ. نَجَّسْتَ تَاجَهُ فِي التُّرَابِ.
\par 40 هَدَمْتَ كُلَّ جُدْرَانِهِ. جَعَلْتَ حُصُونَهُ خَرَاباً.
\par 41 أَفْسَدَهُ كُلُّ عَابِرِي الطَّرِيقِ. صَارَ عَاراً عِنْدَ جِيرَانِهِ.
\par 42 رَفَعْتَ يَمِينَ مُضَايِقِيهِ. فَرَّحْتَ جَمِيعَ أَعْدَائِهِ.
\par 43 أَيْضاً رَدَدْتَ حَدَّ سَيْفِهِ وَلَمْ تَنْصُرْهُ فِي الْقِتَالِ.
\par 44 أَبْطَلْتَ بَهَاءَهُ وَأَلْقَيْتَ كُرْسِيَّهُ إِلَى الأَرْضِ.
\par 45 قَصَّرْتَ أَيَّامَ شَبَابِهِ. غَطَّيْتَهُ بِالْخِزْيِ. سِلاَهْ.
\par 46 حَتَّى مَتَى يَا رَبُّ تَخْتَبِئُ كُلَّ الاِخْتِبَاءِ؟ حَتَّى مَتَى يَتَّقِدُ كَالنَّارِ غَضَبُكَ؟
\par 47 اذْكُرْ كَيْفَ أَنَا زَائِلٌ. إِلَى أَيِّ بَاطِلٍ خَلَقْتَ جَمِيعَ بَنِي آدَمَ؟
\par 48 أَيُّ إِنْسَانٍ يَحْيَا وَلاَ يَرَى الْمَوْتَ؟ أَيٌّ يُنَجِّي نَفْسَهُ مِنْ يَدِ الْهَاوِيَةِ؟ سِلاَهْ.
\par 49 أَيْنَ مَرَاحِمُكَ الأُوَلُ يَا رَبُّ الَّتِي حَلَفْتَ بِهَا لِدَاوُدَ بِأَمَانَتِكَ؟
\par 50 اذْكُرْ يَا رَبُّ عَارَ عَبِيدِكَ الَّذِي أَحْتَمِلُهُ فِي حِضْنِي مِنْ كَثْرَةِ الأُمَمِ كُلِّهَا
\par 51 الَّذِي بِهِ عَيَّرَ أَعْدَاؤُكَ يَا رَبُّ الَّذِينَ عَيَّرُوا آثَارَ مَسِيحِكَ.
\par 52 مُبَارَكٌ الرَّبُّ إِلَى الدَّهْرِ. آمِينَ فَآمِينَ.

\chapter{90}

\par 1 صَلاَةٌ لِمُوسَى رَجُلِ اللهِ يَا رَبُّ مَلْجَأً كُنْتَ لَنَا فِي دَوْرٍ فَدَوْرٍ.
\par 2 مِنْ قَبْلِ أَنْ تُولَدَ الْجِبَالُ أَوْ أَبْدَأْتَ الأَرْضَ وَالْمَسْكُونَةَ مُنْذُ الأَزَلِ إِلَى الأَبَدِ أَنْتَ اللهُ.
\par 3 تُرْجِعُ الإِنْسَانَ إِلَى الْغُبَارِ وَتَقُولُ: [ارْجِعُوا يَا بَنِي آدَمَ].
\par 4 لأَنَّ أَلْفَ سَنَةٍ فِي عَيْنَيْكَ مِثْلُ يَوْمِ أَمْسِ بَعْدَ مَا عَبَرَ وَكَهَزِيعٍ مِنَ اللَّيْلِ.
\par 5 جَرَفْتَهُمْ. كَسِنَةٍ يَكُونُونَ. بِالْغَدَاةِ كَعُشْبٍ يَزُولُ.
\par 6 بِالْغَدَاةِ يُزْهِرُ فَيَزُولُ. عِنْدَ الْمَسَاءِ يُجَزُّ فَيَيْبَسُ.
\par 7 لأَنَّنَا قَدْ فَنِينَا بِسَخَطِكَ وَبِغَضَبِكَ ارْتَعَبْنَا.
\par 8 قَدْ جَعَلْتَ آثَامَنَا أَمَامَكَ خَفِيَّاتِنَا فِي ضُوءِ وَجْهِكَ.
\par 9 لأَنَّ كُلَّ أَيَّامِنَا قَدِ انْقَضَتْ بِرِجْزِكَ. أَفْنَيْنَا سِنِينَا كَقِصَّةٍ.
\par 10 أَيَّامُ سِنِينَا هِيَ سَبْعُونَ سَنَةً وَإِنْ كَانَتْ مَعَ الْقُوَّةِ فَثَمَانُونَ سَنَةً وَأَفْخَرُهَا تَعَبٌ وَبَلِيَّةٌ لأَنَّهَا تُقْرَضُ سَرِيعاً فَنَطِيرُ.
\par 11 مَنْ يَعْرِفُ قُوَّةَ غَضِبَكَ وَكَخَوْفِكَ سَخَطُكَ.
\par 12 إِحْصَاءَ أَيَّامِنَا هَكَذَا عَلِّمْنَا فَنُؤْتَى قَلْبَ حِكْمَةٍ.
\par 13 اِرْجِعْ يَا رَبُّ. حَتَّى مَتَى؟ وَتَرَأَّفْ عَلَى عَبِيدِكَ.
\par 14 أَشْبِعْنَا بِالْغَدَاةِ مِنْ رَحْمَتِكَ فَنَبْتَهِجَ وَنَفْرَحَ كُلَّ أَيَّامِنَا.
\par 15 فَرِّحْنَا كَالأَيَّامِ الَّتِي فِيهَا أَذْلَلْتَنَا كَالسِّنِينِ الَّتِي رَأَيْنَا فِيهَا شَرّاً.
\par 16 لِيَظْهَرْ فِعْلُكَ لِعَبِيدِكَ وَجَلاَلُكَ لِبَنِيهِمْ.
\par 17 وَلْتَكُنْ نِعْمَةُ الرَّبِّ إِلَهِنَا عَلَيْنَا وَعَمَلَ أَيْدِينَا ثَبِّتْ عَلَيْنَا وَعَمَلَ أَيْدِينَا ثَبِّتْهُ.

\chapter{91}

\par 1 اَلسَّاكِنُ فِي سِتْرِ الْعَلِيِّ فِي ظِلِّ الْقَدِيرِ يَبِيتُ.
\par 2 أَقُولُ لِلرَّبِّ: [مَلْجَإِي وَحِصْنِي. إِلَهِي فَأَتَّكِلُ عَلَيْهِ].
\par 3 لأَنَّهُ يُنَجِّيكَ مِنْ فَخِّ الصَّيَّادِ وَمِنَ الْوَبَإِ الْخَطِرِ.
\par 4 بِخَوَافِيهِ يُظَلِّلُكَ وَتَحْتَ أَجْنِحَتِهِ تَحْتَمِي. تُرْسٌ وَمِجَنٌّ حَقُّهُ.
\par 5 لاَ تَخْشَى مِنْ خَوْفِ اللَّيْلِ وَلاَ مِنْ سَهْمٍ يَطِيرُ فِي النَّهَارِ
\par 6 وَلاَ مِنْ وَبَأٍ يَسْلُكُ فِي الدُّجَى وَلاَ مِنْ هَلاَكٍ يُفْسِدُ فِي الظَّهِيرَةِ.
\par 7 يَسْقُطُ عَنْ جَانِبِكَ أَلْفٌ وَرَبَوَاتٌ عَنْ يَمِينِكَ. إِلَيْكَ لاَ يَقْرُبُ.
\par 8 إِنَّمَا بِعَيْنَيْكَ تَنْظُرُ وَتَرَى مُجَازَاةَ الأَشْرَارِ.
\par 9 لأَنَّكَ قُلْتَ: [أَنْتَ يَا رَبُّ مَلْجَإِي]. جَعَلْتَ الْعَلِيَّ مَسْكَنَكَ
\par 10 لاَ يُلاَقِيكَ شَرٌّ وَلاَ تَدْنُو ضَرْبَةٌ مِنْ خَيْمَتِكَ.
\par 11 لأَنَّهُ يُوصِي مَلاَئِكَتَهُ بِكَ لِكَيْ يَحْفَظُوكَ فِي كُلِّ طُرْقِكَ.
\par 12 عَلَى الأَيْدِي يَحْمِلُونَكَ لِئَلاَّ تَصْدِمَ بِحَجَرٍ رِجْلَكَ.
\par 13 عَلَى الأَسَدِ وَالصِّلِّ تَطَأُ. الشِّبْلَ وَالثُّعْبَانَ تَدُوسُ.
\par 14 لأَنَّهُ تَعَلَّقَ بِي أُنَجِّيهِ. أُرَفِّعُهُ لأَنَّهُ عَرَفَ اسْمِي.
\par 15 يَدْعُونِي فَأَسْتَجِيبُ لَهُ. مَعَهُ أَنَا فِي الضِّيقِ. أُنْقِذُهُ وَأُمَجِّدُهُ.
\par 16 مِنْ طُولِ الأَيَّامِ أُشْبِعُهُ وَأُرِيهِ خَلاَصِي.

\chapter{92}

\par 1 مَزْمُورُ تَسْبِيحَةٍ. لِيَوْمِ السَّبْتِ حَسَنٌ هُوَ الْحَمْدُ لِلرَّبِّ وَالتَّرَنُّمُ لاِسْمِكَ أَيُّهَا الْعَلِيُّ.
\par 2 أَنْ يُخْبَرَ بِرَحْمَتِكَ فِي الْغَدَاةِ وَأَمَانَتِكَ كُلَّ لَيْلَةٍ
\par 3 عَلَى ذَاتِ عَشْرَةِ أَوْتَارٍ وَعَلَى الرَّبَابِ عَلَى عَزْفِ الْعُودِ.
\par 4 لأَنَّكَ فَرَّحْتَنِي يَا رَبُّ بِصَنَائِعِكَ. بِأَعْمَالِ يَدَيْكَ أَبْتَهِجُ.
\par 5 مَا أَعْظَمَ أَعْمَالَكَ يَا رَبُّ وَأَعْمَقَ جِدّاً أَفْكَارَكَ.
\par 6 الرَّجُلُ الْبَلِيدُ لاَ يَعْرِفُ وَالْجَاهِلُ لاَ يَفْهَمُ هَذَا.
\par 7 إِذَا زَهَا الأَشْرَارُ كَالْعُشْبِ وَأَزْهَرَ كُلُّ فَاعِلِي الإِثْمِ فَلِكَيْ يُبَادُوا إِلَى الدَّهْرِ.
\par 8 أَمَّا أَنْتَ يَا رَبُّ فَمُتَعَالٍ إِلَى الأَبَدِ.
\par 9 لأَنَّهُ هُوَذَا أَعْدَاؤُكَ يَا رَبُّ لأَنَّهُ هُوَذَا أَعْدَاؤُكَ يَبِيدُونَ. يَتَبَدَّدُ كُلُّ فَاعِلِي الإِثْمِ.
\par 10 وَتَنْصِبُ مِثْلَ الْبَقَرِ الْوَحْشِيِّ قَرْنِي. تَدَهَّنْتُ بِزَيْتٍ طَرِيٍّ.
\par 11 وَتُبْصِرُ عَيْنِي بِمُرَاقِبِيَّ وَبِالْقَائِمِينَ عَلَيَّ بِالشَّرِّ تَسْمَعُ أُذُنَايَ.
\par 12 اَلصِّدِّيقُ كَالنَّخْلَةِ يَزْهُو كَالأَرْزِ فِي لُبْنَانَ يَنْمُو.
\par 13 مَغْرُوسِينَ فِي بَيْتِ الرَّبِّ فِي دِيَارِ إِلَهِنَا يُزْهِرُونَ.
\par 14 أَيْضاً يُثْمِرُونَ فِي الشَّيْبَةِ. يَكُونُونَ دِسَاماً وَخُضْراً
\par 15 لِيُخْبِرُوا بِأَنَّ الرَّبَّ مُسْتَقِيمٌ. صَخْرَتِي هُوَ وَلاَ ظُلْمَ فِيهِ.

\chapter{93}

\par 1 اَلرَّبُّ قَدْ مَلَكَ. لَبِسَ الْجَلاَلَ. لَبِسَ الرَّبُّ الْقُدْرَةَ. اتَّزَرَ بِهَا. أَيْضاً تَثَبَّتَتِ الْمَسْكُونَةُ. لاَ تَتَزَعْزَعُ.
\par 2 كُرْسِيُّكَ مُثْبَتَةٌ مُنْذُ الْقِدَمِ. مُنْذُ الأَزَلِ أَنْتَ.
\par 3 رَفَعَتِ الأَنْهَارُ يَا رَبُّ رَفَعَتِ الأَنْهَارُ صَوْتَهَا. تَرْفَعُ الأَنْهَارُ عَجِيجَهَا.
\par 4 مِنْ أَصْوَاتِ مِيَاهٍ كَثِيرَةٍ مِنْ غِمَارِ أَمْوَاجِ الْبَحْرِ الرَّبُّ فِي الْعُلَى أَقْدَرُ.
\par 5 شَهَادَاتُكَ ثَابِتَةٌ جِدّاً. بِبَيْتِكَ تَلِيقُ الْقَدَاسَةُ يَا رَبُّ إِلَى طُولِ الأَيَّامِ.

\chapter{94}

\par 1 يَا إِلَهَ النَّقَمَاتِ يَا رَبُّ يَا إِلَهَ النَّقَمَاتِ أَشْرِقِ.
\par 2 ارْتَفِعْ يَا دَيَّانَ الأَرْضِ. جَازِ صَنِيعَ الْمُسْتَكْبِرِينَ.
\par 3 حَتَّى مَتَى الْخُطَاةُ يَا رَبُّ حَتَّى مَتَى الْخُطَاةُ يَشْمَتُونَ؟
\par 4 يُبِقُّونَ يَتَكَلَّمُونَ بِوَقَاحَةٍ. كُلُّ فَاعِلِي الإِثْمِ يَفْتَخِرُونَ.
\par 5 يَسْحَقُونَ شَعْبَكَ يَا رَبُّ وَيُذِلُّونَ مِيرَاثَكَ.
\par 6 يَقْتُلُونَ الأَرْمَلَةَ وَالْغَرِيبَ وَيُمِيتُونَ الْيَتِيمَ.
\par 7 وَيَقُولُونَ: [الرَّبُّ لاَ يُبْصِرُ وَإِلَهُ يَعْقُوبَ لاَ يُلاَحِظُ].
\par 8 اِفْهَمُوا أَيُّهَا الْبُلَدَاءُ فِي الشَّعْبِ وَيَا جُهَلاَءُ مَتَى تَعْقِلُونَ؟
\par 9 الْغَارِسُ الأُذُنَِ أَلاَ يَسْمَعُ؟ الصَّانِعُ الْعَيْنَ أَلاَ يُبْصِرُ؟
\par 10 الْمُؤَدِّبُ الأُمَمَ أَلاَ يُبَكِّتُ؟ الْمُعَلِّمُ الإِنْسَانَ مَعْرِفَةً.
\par 11 الرَّبُّ يَعْرِفُ أَفْكَارَ الإِنْسَانِ أَنَّهَا بَاطِلَةٌ.
\par 12 طُوبَى لِلرَّجُلِ الَّذِي تُؤَدِّبُهُ يَا رَبُّ وَتُعَلِّمُهُ مِنْ شَرِيعَتِكَ
\par 13 لِتُرِيحَهُ مِنْ أَيَّامِ الشَّرِّ حَتَّى تُحْفَرَ لِلشِّرِّيرِ حُفْرَةٌ.
\par 14 لأَنَّ الرَّبَّ لاَ يَرْفُضُ شَعْبَهُ وَلاَ يَتْرُكُ مِيرَاثَهُ.
\par 15 لأَنَّهُ إِلَى الْعَدْلِ يَرْجِعُ الْقَضَاءُ وَعَلَى أَثَرِهِ كُلُّ مُسْتَقِيمِي الْقُلُوبِ.
\par 16 مَنْ يَقُومُ لِي عَلَى الْمُسِيئِينَ؟ مَنْ يَقِفُ لِي ضِدَّ فَعَلَةِ الإِثْمِ؟
\par 17 لَوْلاَ أَنَّ الرَّبَّ مُعِينِي لَسَكَنَتْ نَفْسِي سَرِيعاً أَرْضَ السُّكُوتِ.
\par 18 إِذْ قُلْتُ: [قَدْ زَلَّتْ قَدَمِي] فَرَحْمَتُكَ يَا رَبُّ تَعْضُدُنِي.
\par 19 عِنْدَ كَثْرَةِ هُمُومِي فِي دَاخِلِي تَعْزِيَاتُكَ تُلَذِّذُ نَفْسِي.
\par 20 هَلْ يُعَاهِدُكَ كُرْسِيُّ الْمَفَاسِدِ الْمُخْتَلِقُ إِثْماً عَلَى فَرِيضَةٍ؟
\par 21 يَزْدَحِمُونَ عَلَى نَفْسِ الصِّدِّيقِ وَيَحْكُمُونَ عَلَى دَمٍ زَكِيٍّ.
\par 22 فَكَانَ الرَّبُّ لِي صَرْحاً وَإِلَهِي صَخْرَةَ مَلْجَإِي
\par 23 وَيَرُدُّ عَلَيْهِمْ إِثْمَهُمْ وَبِشَرِّهِمْ يُفْنِيهِمْ. يُفْنِيهِمُ الرَّبُّ إِلَهُنَا.

\chapter{95}

\par 1 هَلُمَّ نُرَنِّمُ لِلرَّبِّ نَهْتِفُ لِصَخْرَةِ خَلاَصِنَا.
\par 2 نَتَقَدَّمُ أَمَامَهُ بِحَمْدٍ وَبِتَرْنِيمَاتٍ نَهْتِفُ لَهُ.
\par 3 لأَنَّ الرَّبَّ إِلَهٌ عَظِيمٌ مَلِكٌ كَبِيرٌ عَلَى كُلِّ الآلِهَةِ.
\par 4 الَّذِي بِيَدِهِ مَقَاصِيرُ الأَرْضِ وَخَزَائِنُ الْجِبَالِ لَهُ.
\par 5 الَّذِي لَهُ الْبَحْرُ وَهُوَ صَنَعَهُ وَيَدَاهُ سَبَكَتَا الْيَابِسَةَ.
\par 6 هَلُمَّ نَسْجُدُ وَنَرْكَعُ وَنَجْثُو أَمَامَ الرَّبِّ خَالِقِنَا
\par 7 لأَنَّهُ هُوَ إِلَهُنَا وَنَحْنُ شَعْبُ مَرْعَاهُ وَغَنَمُ يَدِهِ. الْيَوْمَ إِنْ سَمِعْتُمْ صَوْتَهُ
\par 8 فَلاَ تُقَسُّوا قُلُوبَكُمْ كَمَا فِي مَرِيبَةَ مِثْلَ يَوْمِ مَسَّةَ فِي الْبَرِّيَّةِ
\par 9 حَيْثُ جَرَّبَنِي آبَاؤُكُمُ. اخْتَبَرُونِي. أَبْصَرُوا أَيْضاً فِعْلِي
\par 10 أَرْبَعِينَ سَنَةً مَقَتُّ ذَلِكَ الْجِيلَ وَقُلْتُ: [هُمْ شَعْبٌ ضَالٌّ قَلْبُهُمْ وَهُمْ لَمْ يَعْرِفُوا سُبُلِي].
\par 11 فَأَقْسَمْتُ فِي غَضَبِي لاَ يَدْخُلُونَ رَاحَتِي!

\chapter{96}

\par 1 رَنِّمُوا لِلرَّبِّ تَرْنِيمَةً جَدِيدَةً. رَنِّمِي لِلرَّبِّ يَا كُلَّ الأَرْضِ.
\par 2 رَنِّمُوا لِلرَّبِّ بَارِكُوا اسْمَهُ بَشِّرُوا مِنْ يَوْمٍ إِلَى يَوْمٍ بِخَلاَصِهِ.
\par 3 حَدِّثُوا بَيْنَ الأُمَمِ بِمَجْدِهِ بَيْنَ جَمِيعِ الشُّعُوبِ بِعَجَائِبِهِ.
\par 4 لأَنَّ الرَّبَّ عَظِيمٌ وَحَمِيدٌ جِدّاً مَهُوبٌ هُوَ عَلَى كُلِّ الآلِهَةِ.
\par 5 لأَنَّ كُلَّ آلِهَةِ الشُّعُوبِ أَصْنَامٌ أَمَّا الرَّبُّ فَقَدْ صَنَعَ السَّمَاوَاتِ.
\par 6 مَجْدٌ وَجَلاَلٌ قُدَّامَهُ. الْعِزُّ وَالْجَمَالُ فِي مَقْدِسِهِ.
\par 7 قَدِّمُوا لِلرَّبِّ يَا قَبَائِلَ الشُّعُوبِ قَدِّمُوا لِلرَّبِّ مَجْداً وَقُوَّةً.
\par 8 قَدِّمُوا لِلرَّبِّ مَجْدَ اسْمِهِ. هَاتُوا تَقْدِمَةً وَادْخُلُوا دِيَارَهُ.
\par 9 اسْجُدُوا لِلرَّبِّ فِي زِينَةٍ مُقَدَّسَةٍ. ارْتَعِدِي قُدَّامَهُ يَا كُلَّ الأَرْضِ.
\par 10 قُولُوا بَيْنَ الأُمَمِ: [الرَّبُّ قَدْ مَلَكَ. أَيْضاً تَثَبَّتَتِ الْمَسْكُونَةُ فَلاَ تَتَزَعْزَعُ. يَدِينُ الشُّعُوبَ بِالاِسْتِقَامَةِ].
\par 11 لِتَفْرَحِ السَّمَاوَاتُ وَلْتَبْتَهِجِ الأَرْضُ لِيَعِجَّ الْبَحْرُ وَمِلْؤُهُ.
\par 12 لِيَجْذَلِ الْحَقْلُ وَكُلُّ مَا فِيهِ. لِتَتَرَنَّمْ حِينَئِذٍ كُلُّ أَشْجَارِ الْوَعْرِ
\par 13 أَمَامَ الرَّبِّ لأَنَّهُ جَاءَ. جَاءَ لِيَدِينَ الأَرْضَ. يَدِينُ الْمَسْكُونَةَ بِالْعَدْلِ وَالشُّعُوبَ بِأَمَانَتِهِ

\chapter{97}

\par 1 اَلرَّبُّ قَدْ مَلَكَ فَلْتَبْتَهِجِ الأَرْضُ وَلْتَفْرَحِ الْجَزَائِرُ الْكَثِيرَةُ.
\par 2 السَّحَابُ وَالضَّبَابُ حَوْلَهُ. الْعَدْلُ وَالْحَقُّ قَاعِدَةُ كُرْسِيِّهِ.
\par 3 قُدَّامَهُ تَذْهَبُ نَارٌ وَتُحْرِقُ أَعْدَاءَهُ حَوْلَهُ.
\par 4 أَضَاءَتْ بُرُوقُهُ الْمَسْكُونَةَ. رَأَتِ الأَرْضُ وَارْتَعَدَتْ.
\par 5 ذَابَتِ الْجِبَالُ مِثْلَ الشَّمْعِ قُدَّامَ الرَّبِّ قُدَّامَ سَيِّدِ الأَرْضِ كُلِّهَا.
\par 6 أَخْبَرَتِ السَّمَاوَاتُ بِعَدْلِهِ وَرَأَى جَمِيعُ الشُّعُوبِ مَجْدَهُ.
\par 7 يَخْزَى كُلُّ عَابِدِي تِمْثَالٍ مَنْحُوتٍ الْمُفْتَخِرِينَ بِالأَصْنَامِ. اسْجُدُوا لَهُ يَا جَمِيعَ الآلِهَةِ.
\par 8 سَمِعَتْ صِهْيَوْنُ فَفَرِحَتْ وَابْتَهَجَتْ بَنَاتُ يَهُوذَا مِنْ أَجْلِ أَحْكَامِكَ يَا رَبُّ.
\par 9 لأَنَّكَ أَنْتَ يَا رَبُّ عَلِيٌّ عَلَى كُلِّ الأَرْضِ. عَلَوْتَ جِدّاً عَلَى كُلِّ الآلِهَةِ.
\par 10 يَا مُحِبِّي الرَّبِّ أَبْغِضُوا الشَّرَّ. هُوَ حَافِظٌ نُفُوسَ أَتْقِيَائِهِ. مِنْ يَدِ الأَشْرَارِ يُنْقِذُهُمْ.
\par 11 نُورٌ قَدْ زُرِعَ لِلصِّدِّيقِ وَفَرَحٌ لِلْمُسْتَقِيمِي الْقَلْبِ.
\par 12 افْرَحُوا أَيُّهَا الصِّدِّيقُونَ بِالرَّبِّ وَاحْمَدُوا ذِكْرَ قُدْسِهِ.

\chapter{98}

\par 1 رَنِّمُوا لِلرَّبِّ تَرْنِيمَةً جَدِيدَةً لأَنَّهُ صَنَعَ عَجَائِبَ. خَلَّصَتْهُ يَمِينُهُ وَذِرَاعُ قُدْسِهِ.
\par 2 أَعْلَنَ الرَّبُّ خَلاَصَهُ. لِعُيُونِ الأُمَمِ كَشَفَ بِرَّهُ.
\par 3 ذَكَرَ رَحْمَتَهُ وَأَمَانَتَهُ لِبَيْتِ إِسْرَائِيلَ. رَأَتْ كُلُّ أَقَاصِي الأَرْضِ خَلاَصَ إِلَهِنَا.
\par 4 اِهْتِفِي لِلرَّبِّ يَا كُلَّ الأَرْضِ. اهْتِفُوا وَرَنِّمُوا وَغَنُّوا.
\par 5 رَنِّمُوا لِلرَّبِّ بِعُودٍ. بِعُودٍ وَصَوْتِ نَشِيدٍ.
\par 6 بِالأَبْوَاقِ وَصَوْتِ الصُّورِ اهْتِفُوا قُدَّامَ الْمَلِكِ الرَّبِّ.
\par 7 لِيَعِجَّ الْبَحْرُ وَمِلْؤُهُ الْمَسْكُونَةُ وَالسَّاكِنُونَ فِيهَا.
\par 8 الأَنْهَارُ لِتُصَفِّقْ بِالأَيَادِي الْجِبَالُ لِتُرَنِّمْ مَعاً
\par 9 أَمَامَ الرَّبِّ لأَنَّهُ جَاءَ لِيَدِينَ الأَرْضَ. يَدِينُ الْمَسْكُونَةَ بِالْعَدْلِ وَالشُّعُوبَ بِالاِسْتِقَامَةِ.

\chapter{99}

\par 1 اَلرَّبُّ قَدْ مَلَكَ. تَرْتَعِدُ الشُّعُوبُ. هُوَ جَالِسٌ عَلَى الْكَرُوبِيمِ. تَتَزَلْزَلُ الأَرْضُ.
\par 2 الرَّبُّ عَظِيمٌ فِي صِهْيَوْنَ وَعَالٍ هُوَ عَلَى كُلِّ الشُّعُوبِ.
\par 3 يَحْمَدُونَ اسْمَكَ الْعَظِيمَ وَالْمَهُوبَ. قُدُّوسٌ هُوَ.
\par 4 وَعِزُّ الْمَلِكِ أَنْ يُحِبَّ الْحَقَّ. أَنْتَ ثَبَّتَّ الاِسْتِقَامَةَ. أَنْتَ أَجْرَيْتَ حَقّاً وَعَدْلاً فِي يَعْقُوبَ.
\par 5 عَلُّوا الرَّبَّ إِلَهَنَا وَاسْجُدُوا عِنْدَ مَوْطِئِ قَدَمَيْهِ. قُدُّوسٌ هُوَ.
\par 6 مُوسَى وَهَارُونُ بَيْنَ كَهَنَتِهِ وَصَمُوئِيلُ بَيْنَ الَّذِينَ يَدْعُونَ بِاسْمِهِ. دَعُوا الرَّبَّ وَهُوَ اسْتَجَابَ لَهُمْ.
\par 7 بِعَمُودِ السَّحَابِ كَلَّمَهُمْ. حَفِظُوا شَهَادَاتِهِ وَالْفَرِيضَةَ الَّتِي أَعْطَاهُمْ.
\par 8 أَيُّهَا الرَّبُّ إِلَهُنَا أَنْتَ اسْتَجَبْتَ لَهُمْ. إِلَهاً غَفُوراً كُنْتَ لَهُمْ وَمُنْتَقِماً عَلَى أَفْعَالِهِمْ.
\par 9 عَلُّوا الرَّبَّ إِلَهَنَا وَاسْجُدُوا فِي جَبَلِ قُدْسِهِ لأَنَّ الرَّبَّ إِلَهَنَا قُدُّوسٌ.

\chapter{100}

\par 1 مَزْمُورُ حَمْدٍ اِهْتِفِي لِلرَّبِّ يَا كُلَّ الأَرْضِ.
\par 2 اعْبُدُوا الرَّبَّ بِفَرَحٍ. ادْخُلُوا إِلَى حَضْرَتِهِ بِتَرَنُّمٍ.
\par 3 اعْلَمُوا أَنَّ الرَّبَّ هُوَ اللهُ. هُوَ صَنَعَنَا وَلَهُ نَحْنُ شَعْبُهُ وَغَنَمُ مَرْعَاهُ.
\par 4 ادْخُلُوا أَبْوَابَهُ بِحَمْدٍ دِيَارَهُ بِالتَّسْبِيحِ. احْمَدُوهُ بَارِكُوا اسْمَهُ
\par 5 لأَنَّ الرَّبَّ صَالِحٌ. إِلَى الأَبَدِ رَحْمَتُهُ وَإِلَى دَوْرٍ فَدَوْرٍ أَمَانَتُهُ.

\chapter{101}

\par 1 لِدَاوُدَ. مَزْمُورٌ رَحْمَةً وَحُكْماً أُغَنِّي. لَكَ يَا رَبُّ أُرَنِّمُ.
\par 2 أَتَعَقَّلُ فِي طَرِيقٍ كَامِلٍ. مَتَى تَأْتِي إِلَيَّ؟ أَسْلُكُ فِي كَمَالِ قَلْبِي فِي وَسَطِ بَيْتِي.
\par 3 لاَ أَضَعُ قُدَّامَ عَيْنَيَّ أَمْراً رَدِيئاً. عَمَلَ الزَّيَغَانِ أَبْغَضْتُ. لاَ يَلْصَقُ بِي.
\par 4 قَلْبٌ مُعَوَّجٌ يَبْعُدُ عَنِّي. الشِّرِّيرُ لاَ أَعْرِفُهُ.
\par 5 الَّذِي يَغْتَابُ صَاحِبَهُ سِرّاً هَذَا أَقْطَعُهُ. مُسْتَكْبِرُ الْعَيْنِ وَمُنْتَفِخُ الْقَلْبِ لاَ أَحْتَمِلُهُ.
\par 6 عَيْنَايَ عَلَى أُمَنَاءِ الأَرْضِ لِكَيْ أُجْلِسَهُمْ مَعِي. السَّالِكُ طَرِيقاً كَامِلاً هُوَ يَخْدِمُنِي.
\par 7 لاَ يَسْكُنُ وَسَطَ بَيْتِي عَامِلُ غِشٍّ. الْمُتَكَلِّمُ بِالْكَذِبِ لاَ يَثْبُتُ أَمَامَ عَيْنَيَّ.
\par 8 بَاكِراً أُبِيدُ جَمِيعَ أَشْرَارِ الأَرْضِ لأَقْطَعَ مِنْ مَدِينَةِ الرَّبِّ كُلَّ فَاعِلِي الإِثْمِ.

\chapter{102}

\par 1 صَلاَةٌ لِمِسْكِينٍ إِذَا أَعْيَا وَسَكَبَ شَكْوَاهُ قُدَّامَ اللهِ يَا رَبُّ اسْتَمِعْ صَلاَتِي وَلْيَدْخُلْ إِلَيْكَ صُرَاخِي.
\par 2 لاَ تَحْجُبْ وَجْهَكَ عَنِّي فِي يَوْمِ ضِيقِي. أَمِلْ إِلَيَّ أُذُنَكَ فِي يَوْمِ أَدْعُوكَ. اسْتَجِبْ لِي سَرِيعاً.
\par 3 لأَنَّ أَيَّامِي قَدْ فَنِيَتْ فِي دُخَانٍ وَعِظَامِي مِثْلُ وَقِيدٍ قَدْ يَبِسَتْ.
\par 4 مَلْفُوحٌ كَالْعُشْبِ وَيَابِسٌ قَلْبِي حَتَّى سَهَوْتُ عَنْ أَكْلِ خُبْزِي.
\par 5 مِنْ صَوْتِ تَنَهُّدِي لَصِقَ عَظْمِي بِلَحْمِي.
\par 6 أَشْبَهْتُ قُوقَ الْبَرِّيَّةِ. صِرْتُ مِثْلَ بُومَةِ الْخِرَبِ.
\par 7 سَهِدْتُ وَصِرْتُ كَعُصْفُورٍ مُنْفَرِدٍ عَلَى السَّطْحِ.
\par 8 الْيَوْمَ كُلَّهُ عَيَّرَنِي أَعْدَائِيَ. الْحَنِقُونَ عَلَيَّ حَلَفُوا عَلَيَّ.
\par 9 إِنِّي قَدْ أَكَلْتُ الرَّمَادَ مِثْلَ الْخُبْزِ وَمَزَجْتُ شَرَابِي بِدُمُوعٍ
\par 10 بِسَبَبِ غَضَبِكَ وَسَخَطِكَ لأَنَّكَ حَمَلْتَنِي وَطَرَحْتَنِي.
\par 11 أَيَّامِي كَظِلٍّ مَائِلٍ وَأَنَا مِثْلُ الْعُشْبِ يَبِسْتُ.
\par 12 أَمَّا أَنْتَ يَا رَبُّ فَإِلَى الدَّهْرِ جَالِسٌ وَذِكْرُكَ إِلَى دَوْرٍ فَدَوْرٍ.
\par 13 أَنْتَ تَقُومُ وَتَرْحَمُ صِهْيَوْنَ لأَنَّهُ وَقْتُ الرَّأْفَةِ لأَنَّهُ جَاءَ الْمِيعَادُ.
\par 14 لأَنَّ عَبِيدَكَ قَدْ سُرُّوا بِحِجَارَتِهَا وَحَنُّوا إِلَى تُرَابِهَا.
\par 15 فَتَخْشَى الأُمَمُ اسْمَ الرَّبِّ وَكُلُّ مُلُوكِ الأَرْضِ مَجْدَكَ.
\par 16 إِذَا بَنَى الرَّبُّ صِهْيَوْنَ يُرَى بِمَجْدِهِ.
\par 17 الْتَفَتَ إِلَى صَلاَةِ الْمُضْطَرِّ وَلَمْ يَرْذُلْ دُعَاءَهُمْ.
\par 18 يُكْتَبُ هَذَا لِلدَّوْرِ الآخِرِ وَشَعْبٌ سَوْفَ يُخْلَقُ يُسَبِّحُ الرَّبَّ.
\par 19 لأَنَّهُ أَشْرَفَ مِنْ عُلْوِ قُدْسِهِ. الرَّبُّ مِنَ السَّمَاءِ إِلَى الأَرْضِ نَظَرَ
\par 20 لِيَسْمَعَ أَنِينَ الأَسِيرِ لِيُطْلِقَ بَنِي الْمَوْتِ
\par 21 لِكَيْ يُحَدَّثَ فِي صِهْيَوْنَ بِاسْمِ الرَّبِّ وَبِتَسْبِيحِهِ فِي أُورُشَلِيمَ
\par 22 عِنْدَ اجْتِمَاعِ الشُّعُوبِ مَعاً وَالْمَمَالِكِ لِعِبَادَةِ الرَّبِّ.
\par 23 ضَعَّفَ فِي الطَّرِيقِ قُوَّتِي. قَصَّرَ أَيَّامِي.
\par 24 أَقُولُ: [يَا إِلَهِي لاَ تَقْبِضْنِي فِي نِصْفِ أَيَّامِي. إِلَى دَهْرِ الدُّهُورِ سِنُوكَ.
\par 25 مِنْ قِدَمٍ أَسَّسْتَ الأَرْضَ وَالسَّمَاوَاتُ هِيَ عَمَلُ يَدَيْكَ.
\par 26 هِيَ تَبِيدُ وَأَنْتَ تَبْقَى وَكُلُّهَا كَثَوْبٍ تَبْلَى كَرِدَاءٍ تُغَيِّرُهُنَّ فَتَتَغَيَّرُ.
\par 27 وَأَنْتَ هُوَ وَسِنُوكَ لَنْ تَنْتَهِيَ.
\par 28 أَبْنَاءُ عَبِيدِكَ يَسْكُنُونَ وَذُرِّيَّتُهُمْ تُثَبَّتُ أَمَامَكَ].

\chapter{103}

\par 1 لِدَاوُدَ بَارِكِي يَا نَفْسِي الرَّبَّ وَكُلُّ مَا فِي بَاطِنِي لِيُبَارِكِ اسْمَهُ الْقُدُّوسَ.
\par 2 بَارِكِي يَا نَفْسِي الرَّبَّ وَلاَ تَنْسَيْ كُلَّ حَسَنَاتِهِ.
\par 3 الَّذِي يَغْفِرُ جَمِيعَ ذُنُوبِكِ. الَّذِي يَشْفِي كُلَّ أَمْرَاضِكِ.
\par 4 الَّذِي يَفْدِي مِنَ الْحُفْرَةِ حَيَاتَكِ. الَّذِي يُكَلِّلُكِ بِالرَّحْمَةِ وَالرَّأْفَةِ.
\par 5 الَّذِي يُشْبِعُ بِالْخَيْرِ عُمْرَكِ فَيَتَجَدَّدُ مِثْلَ النَّسْرِ شَبَابُكِ.
\par 6 اَلرَّبُّ مُجْرِي الْعَدْلَ وَالْقَضَاءَ لِجَمِيعِ الْمَظْلُومِينَ.
\par 7 عَرَّفَ مُوسَى طُرُقَهُ وَبَنِي إِسْرَائِيلَ أَفْعَالَهُ.
\par 8 الرَّبُّ رَحِيمٌ وَرَأُوفٌ طَوِيلُ الرُّوحِ وَكَثِيرُ الرَّحْمَةِ.
\par 9 لاَ يُحَاكِمُ إِلَى الأَبَدِ وَلاَ يَحْقِدُ إِلَى الدَّهْرِ.
\par 10 لَمْ يَصْنَعْ مَعَنَا حَسَبَ خَطَايَانَا وَلَمْ يُجَازِنَا حَسَبَ آثَامِنَا.
\par 11 لأَنَّهُ مِثْلُ ارْتِفَاعِ السَّمَاوَاتِ فَوْقَ الأَرْضِ قَوِيَتْ رَحْمَتُهُ عَلَى خَائِفِيهِ.
\par 12 كَبُعْدِ الْمَشْرِقِ مِنَ الْمَغْرِبِ أَبْعَدَ عَنَّا مَعَاصِيَنَا.
\par 13 كَمَا يَتَرَأَّفُ الأَبُ عَلَى الْبَنِينَ يَتَرَأَّفُ الرَّبُّ عَلَى خَائِفِيهِ.
\par 14 لأَنَّهُ يَعْرِفُ جِبْلَتَنَا. يَذْكُرُ أَنَّنَا تُرَابٌ نَحْنُ.
\par 15 الإِنْسَانُ مِثْلُ الْعُشْبِ أَيَّامُهُ. كَزَهْرِ الْحَقْلِ كَذَلِكَ يُزْهِرُ.
\par 16 لأَنَّ رِيحاً تَعْبُرُ عَلَيْهِ فَلاَ يَكُونُ وَلاَ يَعْرِفُهُ مَوْضِعُهُ بَعْدُ.
\par 17 أَمَّا رَحْمَةُ الرَّبِّ فَإِلَى الدَّهْرِ وَالأَبَدِ عَلَى خَائِفِيهِ وَعَدْلُهُ عَلَى بَنِي الْبَنِينَ
\par 18 لِحَافِظِي عَهْدِهِ وَذَاكِرِي وَصَايَاهُ لِيَعْمَلُوهَا.
\par 19 اَلرَّبُّ فِي السَّمَاوَاتِ ثَبَّتَ كُرْسِيَّهُ وَمَمْلَكَتُهُ عَلَى الْكُلِّ تَسُودُ.
\par 20 بَارِكُوا الرَّبَّ يَا مَلاَئِكَتَهُ الْمُقْتَدِرِينَ قُوَّةً الْفَاعِلِينَ أَمْرَهُ عِنْدَ سَمَاعِ صَوْتِ كَلاَمِهِ.
\par 21 بَارِكُوا الرَّبَّ يَا جَمِيعَ جُنُودِهِ خُدَّامَهُ الْعَامِلِينَ مَرْضَاتَهُ.
\par 22 بَارِكُوا الرَّبَّ يَا جَمِيعَ أَعْمَالِهِ. فِي كُلِّ مَوَاضِعِ سُلْطَانِهِ بَارِكِي يَا نَفْسِيَ الرَّبَّ.

\chapter{104}

\par 1 بَارِكِي يَا نَفْسِي الرَّبَّ. يَا رَبُّ إِلَهِي قَدْ عَظُمْتَ جِدّاً. مَجْداً وَجَلاَلاً لَبِسْتَ.
\par 2 اللاَّبِسُ النُّورَ كَثَوْبٍ الْبَاسِطُ السَّمَاوَاتِ كَشُقَّةٍ.
\par 3 الْمُسَقِّفُ عَلاَلِيَهُ بِالْمِيَاهِ. الْجَاعِلُ السَّحَابَ مَرْكَبَتَهُ. الْمَاشِي عَلَى أَجْنِحَةِ الرِّيحِ.
\par 4 الصَّانِعُ مَلاَئِكَتَهُ رِيَاحاً وَخُدَّامَهُ نَاراً مُلْتَهِبَةً.
\par 5 الْمُؤَسِّسُ الأَرْضَ عَلَى قَوَاعِدِهَا فَلاَ تَتَزَعْزَعُ إِلَى الدَّهْرِ وَالأَبَدِ.
\par 6 كَسَوْتَهَا الْغَمْرَ كَثَوْبٍ. فَوْقَ الْجِبَالِ تَقِفُ الْمِيَاهُ.
\par 7 مِنِ انْتِهَارِكَ تَهْرُبُ مِنْ صَوْتِ رَعْدِكَ تَفِرُّ.
\par 8 تَصْعَدُ إِلَى الْجِبَالِ. تَنْزِلُ إِلَى الْبِقَاعِ إِلَى الْمَوْضِعِ الَّذِي أَسَّسْتَهُ لَهَا.
\par 9 وَضَعْتَ لَهَا تُخُماً لاَ تَتَعَدَّاهُ. لاَ تَرْجِعُ لِتُغَطِّيَ الأَرْضَ.
\par 10 اَلْمُفَجِّرُ عُيُوناً فِي الأَوْدِيَةِ. بَيْنَ الْجِبَالِ تَجْرِي.
\par 11 تَسْقِي كُلَّ حَيَوَانِ الْبَرِّ. تَكْسِرُ الْفِرَاءُ ظَمَأَهَا.
\par 12 فَوْقَهَا طُيُورُ السَّمَاءِ تَسْكُنُ. مِنْ بَيْنِ الأَغْصَانِ تُسَمِّعُ صَوْتاً.
\par 13 السَّاقِي الْجِبَالَ مِنْ عَلاَلِيهِ. مِنْ ثَمَرِ أَعْمَالِكَ تَشْبَعُ الأَرْضُ.
\par 14 الْمُنْبِتُ عُشْباً لِلْبَهَائِمِ وَخُضْرَةً لِخِدْمَةِ الإِنْسَانِ لإِخْرَاجِ خُبْزٍ مِنَ الأَرْضِ
\par 15 وَخَمْرٍ تُفَرِّحُ قَلْبَ الإِنْسَانِ لإِلْمَاعِ وَجْهِهِ أَكْثَرَ مِنَ الزَّيْتِ وَخُبْزٍ يُسْنِدُ قَلْبَ الإِنْسَانِ.
\par 16 تَشْبَعُ أَشْجَارُ الرَّبِّ أَرْزُ لُبْنَانَ الَّذِي نَصَبَهُ.
\par 17 حَيْثُ تُعَشِّشُ هُنَاكَ الْعَصَافِيرُ. أَمَّا اللَّقْلَقُ فَالسَّرْوُ بَيْتُهُ.
\par 18 الْجِبَالُ الْعَالِيَةُ لِلْوُعُولِ. الصُّخُورُ مَلْجَأٌ لِلْوِبَارِ.
\par 19 صَنَعَ الْقَمَرَ لِلْمَوَاقِيتِ. الشَّمْسُ تَعْرِفُ مَغْرِبَهَا.
\par 20 تَجْعَلُ ظُلْمَةً فَيَصِيرُ لَيْلٌ. فِيهِ يَدِبُّ كُلُّ حَيَوَانِ الْوَعْرِ.
\par 21 الأَشْبَالُ تُزَمْجِرُ لِتَخْطُفَ وَلِتَلْتَمِسَ مِنَ اللهِ طَعَامَهَا.
\par 22 تُشْرِقُ الشَّمْسُ فَتَجْتَمِعُ وَفِي مَآوِيهَا تَرْبِضُ.
\par 23 الإِنْسَانُ يَخْرُجُ إِلَى عَمَلِهِ وَإِلَى شُغْلِهِ إِلَى الْمَسَاءِ.
\par 24 مَا أَعْظَمَ أَعْمَالَكَ يَا رَبُّ! كُلَّهَا بِحِكْمَةٍ صَنَعْتَ. مَلآنَةٌ الأَرْضُ مِنْ غِنَاكَ.
\par 25 هَذَا الْبَحْرُ الْكَبِيرُ الْوَاسِعُ الأَطْرَافِ. هُنَاكَ دَبَّابَاتٌ بِلاَ عَدَدٍ. صِغَارُ حَيَوَانٍ مَعَ كِبَارٍ.
\par 26 هُنَاكَ تَجْرِي السُّفُنُ. لَوِيَاثَانُ هَذَا خَلَقْتَهُ لِيَلْعَبَ فِيهِ.
\par 27 كُلُّهَا إِيَّاكَ تَتَرَجَّى لِتَرْزُقَهَا قُوتَهَا فِي حِينِهِ.
\par 28 تُعْطِيهَا فَتَلْتَقِطُ. تَفْتَحُ يَدَكَ فَتَشْبَعُ خَيْراً.
\par 29 تَحْجُبُ وَجْهَكَ فَتَرْتَاعُ. تَنْزِعُ أَرْوَاحَهَا فَتَمُوتُ وَإِلَى تُرَابِهَا تَعُودُ.
\par 30 تُرْسِلُ رُوحَكَ فَتُخْلَقُ. وَتُجَدِّدُ وَجْهَ الأَرْضِ.
\par 31 يَكُونُ مَجْدُ الرَّبِّ إِلَى الدَّهْرِ. يَفْرَحُ الرَّبُّ بِأَعْمَالِهِ.
\par 32 النَّاظِرُ إِلَى الأَرْضِ فَتَرْتَعِدُ. يَمَسُّ الْجِبَالَ فَتُدَخِّنُ.
\par 33 أُغَنِّي لِلرَّبِّ فِي حَيَاتِي. أُرَنِّمُ لإِلَهِي مَا دُمْتُ مَوْجُوداً
\par 34 فَيَلَذُّ لَهُ نَشِيدِي وَأَنَا أَفْرَحُ بِالرَّبِّ.
\par 35 لِتُبَدِ الْخُطَاةُ مِنَ الأَرْضِ وَالأَشْرَارُ لاَ يَكُونُوا بَعْدُ. بَارِكِي يَا نَفْسِي الرَّبَّ. هَلِّلُويَا.

\chapter{105}

\par 1 اِحْمَدُوا الرَّبَّ. ادْعُوا بِاسْمِهِ. عَرِّفُوا بَيْنَ الأُمَمِ بِأَعْمَالِهِ.
\par 2 غَنُّوا لَهُ. رَنِّمُوا لَهُ. أَنْشِدُوا بِكُلِّ عَجَائِبِهِ.
\par 3 افْتَخِرُوا بِاسْمِهِ الْقُدُّوسِ. لِتَفْرَحْ قُلُوبُ الَّذِينَ يَلْتَمِسُونَ الرَّبَّ.
\par 4 اُطْلُبُوا الرَّبَّ وَقُدْرَتَهُ. الْتَمِسُوا وَجْهَهُ دَائِماً.
\par 5 اذْكُرُوا عَجَائِبَهُ الَّتِي صَنَعَ آيَاتِهِ وَأَحْكَامَ فَمِهِ
\par 6 يَا ذُرِّيَّةَ إِبْرَاهِيمَ عَبْدِهِ يَا بَنِي يَعْقُوبَ مُخْتَارِيهِ.
\par 7 هُوَ الرَّبُّ إِلَهُنَا فِي كُلِّ الأَرْضِ أَحْكَامُهُ.
\par 8 ذَكَرَ إِلَى الدَّهْرِ عَهْدَهُ كَلاَماً أَوْصَى بِهِ إِلَى أَلْفِ دَوْرٍ
\par 9 الَّذِي عَاهَدَ بِهِ إِبْرَاهِيمَ وَقَسَمَهُ لإِسْحَاقَ
\par 10 فَثَبَّتَهُ لِيَعْقُوبَ فَرِيضَةً وَلإِسْرَائِيلَ عَهْداً أَبَدِيّاً
\par 11 قَائِلاً: [لَكَ أُعْطِي أَرْضَ كَنْعَانَ حَبْلَ مِيرَاثِكُمْ].
\par 12 إِذْ كَانُوا عَدَداً يُحْصَى قَلِيلِينَ وَغُرَبَاءَ فِيهَا.
\par 13 ذَهَبُوا مِنْ أُمَّةٍ إِلَى أُمَّةٍ مِنْ مَمْلَكَةٍ إِلَى شَعْبٍ آخَرَ.
\par 14 فَلَمْ يَدَعْ إِنْسَاناً يَظْلِمُهُمْ بَلْ وَبَّخَ مُلُوكاً مِنْ أَجْلِهِمْ
\par 15 قَائِلاً: [لاَ تَمَسُّوا مُسَحَائِي وَلاَ تُسِيئُوا إِلَى أَنْبِيَائِي].
\par 16 دَعَا بِالْجُوعِ عَلَى الأَرْضِ. كَسَرَ قِوَامَ الْخُبْزِ كُلَّهُ.
\par 17 أَرْسَلَ أَمَامَهُمْ رَجُلاً. بِيعَ يُوسُفُ عَبْداً.
\par 18 آذُوا بِالْقَيْدِ رِجْلَيْهِ. فِي الْحَدِيدِ دَخَلَتْ نَفْسُهُ
\par 19 إِلَى وَقْتِ مَجِيءِ كَلِمَتِهِ. قَوْلُ الرَّبِّ امْتَحَنَهُ.
\par 20 أَرْسَلَ الْمَلِكُ فَحَلَّهُ. أَرْسَلَ سُلْطَانُ الشَّعْبِ فَأَطْلَقَهُ.
\par 21 أَقَامَهُ سَيِّداً عَلَى بَيْتِهِ وَمُسَلَّطاً عَلَى كُلِّ مُلْكِهِ
\par 22 لِيَأْسِرَ رُؤَسَاءَهُ حَسَبَ إِرَادَتِهِ وَيُعَلِّمَ مَشَايِخَهُ حِكْمَةً.
\par 23 فَجَاءَ إِسْرَائِيلُ إِلَى مِصْرَ وَيَعْقُوبُ تَغَرَّبَ فِي أَرْضِ حَامٍ.
\par 24 جَعَلَ شَعْبَهُ مُثْمِراً جِدّاً وَأَعَزَّهُ عَلَى أَعْدَائِهِ.
\par 25 حَوَّلَ قُلُوبَهُمْ لِيُبْغِضُوا شَعْبَهُ لِيَحْتَالُوا عَلَى عَبِيدِهِ.
\par 26 أَرْسَلَ مُوسَى عَبْدَهُ وَهَارُونَ الَّذِي اخْتَارَهُ.
\par 27 أَقَامَا بَيْنَهُمْ كَلاَمَ آيَاتِهِ وَعَجَائِبَ فِي أَرْضِ حَامٍ.
\par 28 أَرْسَلَ ظُلْمَةً فَأَظْلَمَتْ وَلَمْ يَعْصُوا كَلاَمَهُ.
\par 29 حَوَّلَ مِيَاهَهُمْ إِلَى دَمٍ وَقَتَلَ أَسْمَاكَهُمْ.
\par 30 أَفَاضَتْ أَرْضُهُمْ ضَفَادِعَ حَتَّى فِي مَخَادِعِ مُلُوكِهِمْ.
\par 31 أَمَرَ فَجَاءَ الذُّبَّانُ وَالْبَعُوضُ فِي كُلِّ تُخُومِهِمْ.
\par 32 جَعَلَ أَمْطَارَهُمْ بَرَداً وَنَاراً مُلْتَهِبَةً فِي أَرْضِهِمْ.
\par 33 ضَرَبَ كُرُومَهُمْ وَتِينَهُمْ وَكَسَّرَ كُلَّ أَشْجَارِ تُخُومِهِمْ.
\par 34 أَمَرَ فَجَاءَ الْجَرَادُ وَغَوْغَاءُ بِلاَ عَدَدٍ
\par 35 فَأَكَلَ كُلَّ عُشْبٍ فِي بِلاَدِهِمْ. وَأَكَلَ أَثْمَارَ أَرْضِهِمْ.
\par 36 قَتَلَ كُلَّ بِكْرٍ فِي أَرْضِهِمْ أَوَائِلَ كُلِّ قُوَّتِهِمْ.
\par 37 فَأَخْرَجَهُمْ بِفِضَّةٍ وَذَهَبٍ وَلَمْ يَكُنْ فِي أَسْبَاطِهِمْ عَاثِرٌ.
\par 38 فَرِحَتْ مِصْرُ بِخُرُوجِهِمْ لأَنَّ رُعْبَهُمْ سَقَطَ عَلَيْهِمْ.
\par 39 بَسَطَ سَحَاباً سَجْفاً وَنَاراً لِتُضِيءَ اللَّيْلَ.
\par 40 سَأَلُوا فَأَتَاهُمْ بِالسَّلْوَى وَخُبْزَ السَّمَاءِ أَشْبَعَهُمْ.
\par 41 شَقَّ الصَّخْرَةَ فَانْفَجَرَتِ الْمِيَاهُ. جَرَتْ فِي الْيَابِسَةِ نَهْراً.
\par 42 لأَنَّهُ ذَكَرَ كَلِمَةَ قُدْسِهِ مَعَ إِبْرَاهِيمَ عَبْدِهِ
\par 43 فَأَخْرَجَ شَعْبَهُ بِابْتِهَاجٍ وَمُخْتَارِيهِ بِتَرَنُّمٍ.
\par 44 وَأَعْطَاهُمْ أَرَاضِيَ الأُمَمِ. وَتَعَبَ الشُّعُوبِ وَرَثُوهُ
\par 45 لِكَيْ يَحْفَظُوا فَرَائِضَهُ وَيُطِيعُوا شَرَائِعَهُ. هَلِّلُويَا.

\chapter{106}

\par 1 هَلِّلُويَا. احْمَدُوا الرَّبَّ لأَنَّهُ صَالِحٌ لأَنَّ إِلَى الأَبَدِ رَحْمَتَهُ.
\par 2 مَنْ يَتَكَلَّمُ بِجَبَرُوتِ الرَّبِّ؟ مَنْ يُخْبِرُ بِكُلِّ تَسَابِيحِهِ؟
\par 3 طُوبَى لِلْحَافِظِينَ الْحَقَّ وَلِلصَّانِعِ الْبِرَّ فِي كُلِّ حِينٍ.
\par 4 اذْكُرْنِي يَا رَبُّ بِرِضَا شَعْبِكَ. تَعَهَّدْنِي بِخَلاَصِكَ
\par 5 لأَرَى خَيْرَ مُخْتَارِيكَ. لأَفْرَحَ بِفَرَحِ أُمَّتِكَ. لأَفْتَخِرَ مَعَ مِيرَاثِكَ.
\par 6 أَخْطَأْنَا مَعَ آبَائِنَا. أَسَأْنَا وَأَذْنَبْنَا.
\par 7 آبَاؤُنَا فِي مِصْرَ لَمْ يَفْهَمُوا عَجَائِبَكَ. لَمْ يَذْكُرُوا كَثْرَةَ مَرَاحِمِكَ فَتَمَرَّدُوا عِنْدَ الْبَحْرِ عِنْدَ بَحْرِ سُوفٍ.
\par 8 فَخَلَّصَهُمْ مِنْ أَجْلِ اسْمِهِ لِيُعَرِّفَ بِجَبَرُوتِهِ.
\par 9 وَانْتَهَرَ بَحْرَ سُوفٍ فَيَبِسَ وَسَيَّرَهُمْ فِي اللُّجَجِ كَالْبَرِّيَّةِ.
\par 10 وَخَلَّصَهُمْ مِنْ يَدِ الْمُبْغِضِ وَفَدَاهُمْ مِنْ يَدِ الْعَدُوِّ.
\par 11 وَغَطَّتِ الْمِيَاهُ مُضَايِقِيهِمْ. وَاحِدٌ مِنْهُمْ لَمْ يَبْقَ.
\par 12 فَآمَنُوا بِكَلاَمِهِ. غَنُّوا بِتَسْبِيحِهِ.
\par 13 أَسْرَعُوا فَنَسُوا أَعْمَالَهُ. لَمْ يَنْتَظِرُوا مَشُورَتَهُ.
\par 14 بَلِ اشْتَهُوا شَهْوَةً فِي الْبَرِّيَّةِ وَجَرَّبُوا اللهَ فِي الْقَفْرِ.
\par 15 فَأَعْطَاهُمْ سُؤْلَهُمْ وَأَرْسَلَ هُزَالاً فِي أَنْفُسِهِمْ.
\par 16 وَحَسَدُوا مُوسَى فِي الْمَحَلَّةِ وَهَارُونَ قُدُّوسَ الرَّبِّ.
\par 17 فَتَحَتِ الأَرْضُ وَابْتَلَعَتْ دَاثَانَ وَطَبَقَتْ عَلَى جَمَاعَةِ أَبِيرَامَ
\par 18 وَاشْتَعَلَتْ نَارٌ فِي جَمَاعَتِهِمْ. اللهِيبُ أَحْرَقَ الأَشْرَارَ.
\par 19 صَنَعُوا عِجْلاً فِي حُورِيبَ وَسَجَدُوا لِتِمْثَالٍ مَسْبُوكٍ
\par 20 وَأَبْدَلُوا مَجْدَهُمْ بِمِثَالِ ثَوْرٍ آكِلِ عُشْبٍ.
\par 21 نَسُوا اللهَ مُخَلِّصَهُمُ الصَّانِعَ عَظَائِمَ فِي مِصْرَ
\par 22 وَعَجَائِبَ فِي أَرْضِ حَامٍ وَمَخَاوِفَ عَلَى بَحْرِ سُوفٍ
\par 23 فَقَالَ بِإِهْلاَكِهِمْ. لَوْلاَ مُوسَى مُخْتَارُهُ وَقَفَ فِي الثَّغْرِ قُدَّامَهُ لِيَصْرِفَ غَضَبَهُ عَنْ إِتْلاَفِهِمْ.
\par 24 وَرَذَلُوا الأَرْضَ الشَّهِيَّةَ. لَمْ يُؤْمِنُوا بِكَلِمَتِهِ.
\par 25 بَلْ تَمَرْمَرُوا فِي خِيَامِهِمْ. لَمْ يَسْمَعُوا لِصَوْتِ الرَّبِّ
\par 26 فَرَفَعَ يَدَهُ عَلَيْهِمْ لِيُسْقِطَهُمْ فِي الْبَرِّيَّةِ
\par 27 وَلِيُسْقِطَ نَسْلَهُمْ بَيْنَ الأُمَمِ وَلِيُبَدِّدَهُمْ فِي الأَرَاضِي.
\par 28 وَتَعَلَّقُوا بِبَعْلِ فَغُورَ وَأَكَلُوا ذَبَائِحَ الْمَوْتَى.
\par 29 وَأَغَاظُوهُ بِأَعْمَالِهِمْ فَاقْتَحَمَهُمُ الْوَبَأُ.
\par 30 فَوَقَفَ فِينَحَاسُ وَدَانَ فَامْتَنَعَ الْوَبَأُ.
\par 31 فَحُسِبَ لَهُ ذَلِكَ بِرّاً إِلَى دَوْرٍ فَدَوْرٍ إِلَى الأَبَدِ.
\par 32 وَأَسْخَطُوهُ عَلَى مَاءِ مَرِيبَةَ حَتَّى تَأَذَّى مُوسَى بِسَبَبِهِمْ.
\par 33 لأَنَّهُمْ أَمَرُّوا رُوحَهُ حَتَّى فَرَطَ بِشَفَتَيْهِ.
\par 34 لَمْ يَسْتَأْصِلُوا الأُمَمَ الَّذِينَ قَالَ لَهُمُ الرَّبُّ عَنْهُمْ
\par 35 بَلِ اخْتَلَطُوا بِالأُمَمِ وَتَعَلَّمُوا أَعْمَالَهُمْ
\par 36 وَعَبَدُوا أَصْنَامَهُمْ فَصَارَتْ لَهُمْ شَرَكاً.
\par 37 وَذَبَحُوا بَنِيهِمْ وَبَنَاتِهِمْ لِلأَوْثَانِ
\par 38 وَأَهْرَقُوا دَماً زَكِيّاً دَمَ بَنِيهِمْ وَبَنَاتِهِمِ الَّذِينَ ذَبَحُوهُمْ لأَصْنَامِ كَنْعَانَ وَتَدَنَّسَتِ الأَرْضُ بِالدِّمَاءِ
\par 39 وَتَنَجَّسُوا بِأَعْمَالِهِمْ وَزَنُوا بِأَفْعَالِهِمْ.
\par 40 فَحَمِيَ غَضَبُ الرَّبِّ عَلَى شَعْبِهِ وَكَرِهَ مِيرَاثَهُ
\par 41 وَأَسْلَمَهُمْ لِيَدِ الأُمَمِ وَتَسَلَّطَ عَلَيْهِمْ مُبْغِضُوهُمْ.
\par 42 وَضَغَطَهُمْ أَعْدَاؤُهُمْ فَذَلُّوا تَحْتَ يَدِهِمْ
\par 43 مَرَّاتٍ كَثِيرَةً أَنْقَذَهُمْ. أَمَّا هُمْ فَعَصُوهُ بِمَشُورَتِهِمْ وَانْحَطُّوا بِإِثْمِهِمْ.
\par 44 فَنَظَرَ إِلَى ضِيقِهِمْ إِذْ سَمِعَ صُرَاخَهُمْ
\par 45 وَذَكَرَ لَهُمْ عَهْدَهُ وَنَدِمَ حَسَبَ كَثْرَةِ رَحْمَتِهِ.
\par 46 وَأَعْطَاهُمْ نِعْمَةً قُدَّامَ كُلِّ الَّذِينَ سَبُوهُمْ.
\par 47 خَلِّصْنَا أَيُّهَا الرَّبُّ إِلَهُنَا وَاجْمَعْنَا مِنْ بَيْنِ الأُمَمِ لِنَحْمَدَ اسْمَ قُدْسِكَ وَنَتَفَاخَرَ بِتَسْبِيحِكَ.
\par 48 مُبَارَكٌ الرَّبُّ إِلَهُ إِسْرَائِيلَ مِنَ الأَزَلِ وَإِلَى الأَبَدِ. وَيَقُولُ كُلُّ الشَّعْبِ: [آمِينَ]. هَلِّلُويَا.

\chapter{107}

\par 1 اِحْمَدُوا الرَّبَّ لأَنَّهُ صَالِحٌ لأَنَّ إِلَى الأَبَدِ رَحْمَتَهُ.
\par 2 لِيَقُلْ مَفْدِيُّو الرَّبِّ الَّذِينَ فَدَاهُمْ مِنْ يَدِ الْعَدُوِّ
\par 3 وَمِنَ الْبُلْدَانِ جَمَعَهُمْ مِنَ الْمَشْرِقِ وَمِنَ الْمَغْرِبِ مِنَ الشِّمَالِ وَمِنَ الْبَحْرِ.
\par 4 تَاهُوا فِي الْبَرِّيَّةِ فِي قَفْرٍ بِلاَ طَرِيقٍ. لَمْ يَجِدُوا مَدِينَةَ سَكَنٍ.
\par 5 جِيَاعٌ عِطَاشٌ أَيْضاً أَعْيَتْ أَنْفُسُهُمْ فِيهِمْ.
\par 6 فَصَرَخُوا إِلَى الرَّبِّ فِي ضِيقِهِمْ فَأَنْقَذَهُمْ مِنْ شَدَائِدِهِمْ
\par 7 وَهَدَاهُمْ طَرِيقاً مُسْتَقِيماً لِيَذْهَبُوا إِلَى مَدِينَةِ سَكَنٍ.
\par 8 فَلْيَحْمَدُوا الرَّبَّ عَلَى رَحْمَتِهِ وَعَجَائِبِهِ لِبَنِي آدَمَ.
\par 9 لأَنَّهُ أَشْبَعَ نَفْساً مُشْتَهِيَةً وَمَلَأَ نَفْساً جَائِعَةً خُبْزاً
\par 10 الْجُلُوسَ فِي الظُّلْمَةِ وَظِلاَلِ الْمَوْتِ مُوثَقِينَ بِالذُّلِّ وَالْحَدِيدِ.
\par 11 لأَنَّهُمْ عَصُوا كَلاَمَ اللهِ وَأَهَانُوا مَشُورَةَ الْعَلِيِّ فَأَذَلَّ قُلُوبَهُمْ بِتَعَبٍ.
\par 12 عَثَرُوا وَلاَ مَعِينَ.
\par 13 ثُمَّ صَرَخُوا إِلَى الرَّبِّ فِي ضِيقِهِمْ فَخَلَّصَهُمْ مِنْ شَدَائِدِهِمْ.
\par 14 أَخْرَجَهُمْ مِنَ الظُّلْمَةِ وَظِلاَلِ الْمَوْتِ وَقَطَّعَ قُيُودَهُمْ.
\par 15 فَلْيَحْمَدُوا الرَّبَّ عَلَى رَحْمَتِهِ وَعَجَائِبِهِ لِبَنِي آدَمَ.
\par 16 لأَنَّهُ كَسَّرَ مَصَارِيعَ نُحَاسٍ وَقَطَّعَ عَوَارِضَ حَدِيدٍ.
\par 17 وَالْجُهَّالُ مِنْ طَرِيقِ مَعْصِيَتِهِمْ وَمِنْ آثَامِهِمْ يُذَلُّونَ.
\par 18 كَرِهَتْ أَنْفُسُهُمْ كُلَّ طَعَامٍ وَاقْتَرَبُوا إِلَى أَبْوَابِ الْمَوْتِ.
\par 19 فَصَرَخُوا إِلَى الرَّبِّ فِي ضِيقِهِمْ فَخَلَّصَهُمْ مِنْ شَدَائِدِهِمْ.
\par 20 أَرْسَلَ كَلِمَتَهُ فَشَفَاهُمْ وَنَجَّاهُمْ مِنْ تَهْلُكَاتِهِمْ.
\par 21 فَلْيَحْمَدُوا الرَّبَّ عَلَى رَحْمَتِهِ وَعَجَائِبِهِ لِبَنِي آدَمَ
\par 22 وَلْيَذْبَحُوا لَهُ ذَبَائِحَ الْحَمْدِ وَلْيَعُدُّوا أَعْمَالَهُ بِتَرَنُّمٍ.
\par 23 اَلنَّازِلُونَ إِلَى الْبَحْرِ فِي السُّفُنِ الْعَامِلُونَ عَمَلاً فِي الْمِيَاهِ الْكَثِيرَةِ
\par 24 هُمْ رَأُوا أَعْمَالَ الرَّبِّ وَعَجَائِبَهُ فِي الْعُمْقِ.
\par 25 أَمَرَ فَأَهَاجَ رِيحاً عَاصِفَةً فَرَفَعَتْ أَمْوَاجَهُ.
\par 26 يَصْعَدُونَ إِلَى السَّمَاوَاتِ يَهْبِطُونَ إِلَى الأَعْمَاقِ. ذَابَتْ أَنْفُسُهُمْ بِالشَّقَاءِ.
\par 27 يَتَمَايَلُونَ وَيَتَرَنَّحُونَ مِثْلَ السَّكْرَانِ وَكُلُّ حِكْمَتِهِمِ ابْتُلِعَتْ.
\par 28 فَيَصْرُخُونَ إِلَى الرَّبِّ فِي ضِيقِهِمْ وَمِنْ شَدَائِدِهِمْ يُخَلِّصُهُمْ.
\par 29 يُهَدِّئُ الْعَاصِفَةَ فَتَسْكُنُ وَتَسْكُتُ أَمْوَاجُهَا.
\par 30 فَيَفْرَحُونَ لأَنَّهُمْ هَدَأُوا فَيَهْدِيهِمْ إِلَى الْمَرْفَإِ الَّذِي يُرِيدُونَهُ.
\par 31 فَلْيَحْمَدُوا الرَّبَّ عَلَى رَحْمَتِهِ وَعَجَائِبِهِ لِبَنِي آدَمَ.
\par 32 وَلْيَرْفَعُوهُ فِي مَجْمَعِ الشَّعْبِ وَلْيُسَبِّحُوهُ فِي مَجْلِسِ الْمَشَايِخِ.
\par 33 يَجْعَلُ الأَنْهَارَ قِفَاراً وَمَجَارِيَ الْمِيَاهِ مَعْطَشَةً
\par 34 وَالأَرْضَ الْمُثْمِرَةَ سَبِخَةً مِنْ شَرِّ السَّاكِنِينَ فِيهَا.
\par 35 يَجْعَلُ الْقَفْرَ غَدِيرَ مِيَاهٍ وَأَرْضاً يَبَساً يَنَابِيعَ مِيَاهٍ.
\par 36 وَيُسْكِنُ هُنَاكَ الْجِيَاعَ فَيُهَيِّئُونَ مَدِينَةَ سَكَنٍ.
\par 37 وَيَزْرَعُونَ حُقُولاً وَيَغْرِسُونَ كُرُوماً فَتَصْنَعُ ثَمَرَ غَلَّةٍ.
\par 38 وَيُبَارِكُهُمْ فَيَكْثُرُونَ جِدّاً وَلاَ يُقَلِّلُ بَهَائِمَهُمْ.
\par 39 ثُمَّ يَقِلُّونَ وَيَنْحَنُونَ مِنْ ضَغْطِ الشَّرِّ وَالْحُزْنِ.
\par 40 يَسْكُبُ هَوَاناً عَلَى رُؤَسَاءَ وَيُضِلُّهُمْ فِي تِيهٍ بِلاَ طَرِيقٍ
\par 41 وَيُعَلِّي الْمِسْكِينَ مِنَ الذُّلِّ وَيَجْعَلُ الْقَبَائِلَ مِثْلَ قُطْعَانِ الْغَنَمِ.
\par 42 يَرَى ذَلِكَ الْمُسْتَقِيمُونَ فَيَفْرَحُونَ وَكُلُّ إِثْمٍ يَسُدُّ فَاهُ.
\par 43 مَنْ كَانَ حَكِيماً يَحْفَظُ هَذَا وَيَتَعَقَّلُ مَرَاحِمَ الرَّبِّ.

\chapter{108}

\par 1 تَسْبِيحَةٌ. مَزْمُورٌ لِدَاوُدَ ثَابِتٌ قَلْبِي يَا اللهُ. أُغَنِّي وَأُرَنِّمُ. كَذَلِكَ مَجْدِي.
\par 2 اسْتَيْقِظِي أَيَّتُهَا الرَّبَابُ وَالْعُودُ. أَنَا أَسْتَيْقِظُ سَحَراً.
\par 3 أَحْمَدُكَ بَيْنَ الشُّعُوبِ يَا رَبُّ وَأُرَنِّمُ لَكَ بَيْنَ الأُمَمِ.
\par 4 لأَنَّ رَحْمَتَكَ قَدْ عَظُمَتْ فَوْقَ السَّمَاوَاتِ وَإِلَى الْغَمَامِ حَقُّكَ.
\par 5 ارْتَفِعِ اللهُمَّ عَلَى السَّمَاوَاتِ وَلْيَرْتَفِعْ عَلَى كُلِّ الأَرْضِ مَجْدُكَ.
\par 6 لِكَيْ يَنْجُوَ أَحِبَّاؤُكَ. خَلِّصْ بِيَمِينِكَ وَاسْتَجِبْ لِي.
\par 7 اَللهُ قَدْ تَكَلَّمَ بِقُدْسِهِ. أَبْتَهِجُ أَقْسِمُ شَكِيمَ وَأَقِيسُ وَادِيَ سُكُّوتَ.
\par 8 لِي جِلْعَادُ لِي مَنَسَّى. أَفْرَايِمُ خُوذَةُ رَأْسِي. يَهُوذَا صَوْلَجَانِي.
\par 9 مُوآبُ مِرْحَضَتِي. عَلَى أَدُومَ أَطْرَحُ نَعْلِي. يَا فَلَسْطِينُ اهْتِفِي عَلَيَّ.
\par 10 مَنْ يَقُودُنِي إِلَى الْمَدِينَةِ الْمُحَصَّنَةِ؟ مَنْ يَهْدِينِي إِلَى أَدُومَ؟
\par 11 أَلَيْسَ أَنْتَ يَا اللهُ الَّذِي رَفَضْتَنَا وَلاَ تَخْرُجُ يَا اللهُ مَعَ جُيُوشِنَا؟
\par 12 أَعْطِنَا عَوْناً فِي الضِّيقِ فَبَاطِلٌ هُوَ خَلاَصُ الإِنْسَانِ.
\par 13 بِاللهِ نَصْنَعُ بِبَأْسٍ وَهُوَ يَدُوسُ أَعْدَاءَنَا.

\chapter{109}

\par 1 لإِمَامِ الْمُغَنِّينَ. لِدَاوُدَ. مَزْمُورٌ يَا إِلَهَ تَسْبِيحِي لاَ تَسْكُتْ
\par 2 لأَنَّهُ قَدِ انْفَتَحَ عَلَيَّ فَمُ الشِّرِّيرِ وَفَمُ الْغِشِّ. تَكَلَّمُوا مَعِي بِلِسَانِ كَذِبٍ
\par 3 بِكَلاَمِ بُغْضٍ أَحَاطُوا بِي وَقَاتَلُونِي بِلاَ سَبَبٍ.
\par 4 بَدَلَ مَحَبَّتِي يُخَاصِمُونَنِي. أَمَّا أَنَا فَصَلاَةً.
\par 5 وَضَعُوا عَلَيَّ شَرّاً بَدَلَ خَيْرٍ وَبُغْضاً بَدَلَ حُبِّي.
\par 6 فَأَقِمْ أَنْتَ عَلَيْهِ شِرِّيراً وَلْيَقِفْ شَيْطَانٌ عَنْ يَمِينِهِ.
\par 7 إِذَا حُوكِمَ فَلْيَخْرُجْ مُذْنِباً وَصَلاَتُهُ فَلْتَكُنْ خَطِيَّةً.
\par 8 لِتَكُنْ أَيَّامُهُ قَلِيلَةً وَوَظِيفَتُهُ لِيَأْخُذْهَا آخَرُ.
\par 9 لِيَكُنْ بَنُوهُ أَيْتَاماً وَامْرَأَتُهُ أَرْمَلَةً.
\par 10 لِيَتِهْ بَنُوهُ تَيَهَاناً وَيَسْتَعْطُوا وَيَلْتَمِسُوا خَيْراً مِنْ خِرَبِهِمْ.
\par 11 لِيَصْطَدِ الْمُرَابِي كُلَّ مَا لَهُ وَلْيَنْهَبِ الْغُرَبَاءُ تَعَبَهُ.
\par 12 لاَ يَكُنْ لَهُ بَاسِطٌ رَحْمَةً وَلاَ يَكُنْ مُتَرَغِّفٌ عَلَى يَتَامَاهُ.
\par 13 لِتَنْقَرِضْ ذُرِّيَّتُهُ. فِي الْجِيلِ الْقَادِمِ لِيُمْحَ اسْمُهُمْ.
\par 14 لِيُذْكَرْ إِثْمُ آبَائِهِ لَدَى الرَّبِّ وَلاَ تُمْحَ خَطِيَّةُ أُمِّهِ.
\par 15 لِتَكُنْ أَمَامَ الرَّبِّ دَائِماً وَلْيَقْرِضْ مِنَ الأَرْضِ ذِكْرَهُمْ.
\par 16 مِنْ أَجْلِ أَنَّهُ لَمْ يَذْكُرْ أَنْ يَصْنَعَ رَحْمَةً بَلْ طَرَدَ إِنْسَاناً مَِسْكِيناً وَفَقِيراً وَالْمُنْسَحِقَ الْقَلْبِ لِيُمِيتَهُ.
\par 17 وَأَحَبَّ اللَّعْنَةَ فَأَتَتْهُ وَلَمْ يُسَرَّ بِالْبَرَكَةِ فَتَبَاعَدَتْ عَنْهُ.
\par 18 وَلَبِسَ اللَّعْنَةَ مِثْلَ ثَوْبِهِ فَدَخَلَتْ كَمِيَاهٍ فِي حَشَاهُ وَكَزَيْتٍ فِي عِظَامِهِ.
\par 19 لِتَكُنْ لَهُ كَثَوْبٍ يَتَعَطَّفُ بِهِ وَكَمِنْطَقَةٍ يَتَنَطَّقُ بِهَا دَائِماً.
\par 20 هَذِهِ أُجْرَةُ مُبْغِضِيَّ مِنْ عِنْدِ الرَّبِّ وَأُجْرَةُ الْمُتَكَلِّمِينَ شَرّاً عَلَى نَفْسِي.
\par 21 أَمَّا أَنْتَ يَا رَبُّ السَّيِّدُ فَاصْنَعْ مَعِي مِنْ أَجْلِ اسْمِكَ. لأَنَّ رَحْمَتَكَ طَيِّبَةٌ نَجِّنِي.
\par 22 فَإِنِّي فَقِيرٌ وَمِسْكِينٌ أَنَا وَقَلْبِي مَجْرُوحٌ فِي دَاخِلِي.
\par 23 كَظِلٍّ عِنْدَ مَيْلِهِ ذَهَبْتُ. انْتَفَضْتُ كَجَرَادَةٍ.
\par 24 رُكْبَتَايَ ارْتَعَشَتَا مِنَ الصَّوْمِ وَلَحْمِي هُزِلَ عَنْ سِمَنٍ.
\par 25 وَأَنَا صِرْتُ عَاراً عِنْدَهُمْ. يَنْظُرُونَ إِلَيَّ وَيُنْغِضُونَ رُؤُوسَهُمْ.
\par 26 أَعِنِّي يَا رَبُّ إِلَهِي. خَلِّصْنِي حَسَبَ رَحْمَتِكَ.
\par 27 وَلْيَعْلَمُوا أَنَّ هَذِهِ هِيَ يَدُكَ. أَنْتَ يَا رَبُّ فَعَلْتَ هَذَا.
\par 28 أَمَّا هُمْ فَيَلْعَنُونَ وَأَمَّا أَنْتَ فَتُبَارِكُ. قَامُوا وَخَزُوا أَمَّا عَبْدُكَ فَيَفْرَحُ.
\par 29 لِيَلْبِسْ خُصَمَائِي خَجَلاً وَلْيَتَعَطَّفُوا بِخِزْيِهِمْ كَالرِّدَاءِ.
\par 30 أَحْمَدُ الرَّبَّ جِدّاً بِفَمِي وَفِي وَسَطِ كَثِيرِينَ أُسَبِّحُهُ.
\par 31 لأَنَّهُ يَقُومُ عَنْ يَمِينِ الْمَِسْكِينِ لِيُخَلِّصَهُ مِنَ الْقَاضِينَ عَلَى نَفْسِهِ.

\chapter{110}

\par 1 لِدَاوُدَ. مَزْمُورٌ قَالَ الرَّبُّ لِرَبِّي: [اجْلِسْ عَنْ يَمِينِي حَتَّى أَضَعَ أَعْدَاءَكَ مَوْطِئاً لِقَدَمَيْكَ].
\par 2 يُرْسِلُ الرَّبُّ قَضِيبَ عِزِّكَ مِنْ صِهْيَوْنَ. تَسَلَّطْ فِي وَسَطِ أَعْدَائِكَ.
\par 3 شَعْبُكَ مُنْتَدَبٌ فِي يَوْمِ قُوَّتِكَ فِي زِينَةٍ مُقَدَّسَةٍ مِنْ رَحِمِ الْفَجْرِ. لَكَ طَلُّ حَدَاثَتِكَ.
\par 4 أَقْسَمَ الرَّبُّ وَلَنْ يَنْدَمَ: [أَنْتَ كَاهِنٌ إِلَى الأَبَدِ عَلَى رُتْبَةِ مَلْكِي صَادِقَ].
\par 5 الرَّبُّ عَنْ يَمِينِكَ يُحَطِّمُ فِي يَوْمِ رِجْزِهِ مُلُوكاً.
\par 6 يَدِينُ بَيْنَ الأُمَمِ. مَلَأَ جُثَثاً أَرْضاً وَاسِعَةً. سَحَقَ رُؤُوسَهَا.
\par 7 مِنَ النَّهْرِ يَشْرَبُ فِي الطَّرِيقِ لِذَلِكَ يَرْفَعُ الرَّأْسَ.

\chapter{111}

\par 1 هَلِّلُويَا. أَحْمَدُ الرَّبَّ بِكُلِّ قَلْبِي فِي مَجْلِسِ الْمُسْتَقِيمِينَ وَجَمَاعَتِهِمْ.
\par 2 عَظِيمَةٌ هِيَ أَعْمَالُ الرَّبِّ. مَطْلُوبَةٌ لِكُلِّ الْمَسْرُورِينَ بِهَا.
\par 3 جَلاَلٌ وَبَهَاءٌ عَمَلُهُ وَعَدْلُهُ قَائِمٌ إِلَى الأَبَدِ.
\par 4 صَنَعَ ذِكْراً لِعَجَائِبِهِ. حَنَّانٌ وَرَحِيمٌ هُوَ الرَّبُّ.
\par 5 أَعْطَى خَائِفِيهِ طَعَاماً. يَذْكُرُ إِلَى الأَبَدِ عَهْدَهُ.
\par 6 أَخْبَرَ شَعْبَهُ بِقُوَّةِ أَعْمَالِهِ لِيُعْطِيَهُمْ مِيرَاثَ الأُمَمِ.
\par 7 أَعْمَالُ يَدَيْهِ أَمَانَةٌ وَحَقٌّ. كُلُّ وَصَايَاهُ أَمِينَةٌ
\par 8 ثَابِتَةٌ مَدَى الدَّهْرِ وَالأَبَدِ مَصْنُوعَةٌ بِالْحَقِّ وَالاِسْتِقَامَةِ.
\par 9 أَرْسَلَ فِدَاءً لِشَعْبِهِ. أَقَامَ إِلَى الأَبَدِ عَهْدَهُ. قُدُّوسٌ وَمَهُوبٌ اسْمُهُ.
\par 10 رَأْسُ الْحِكْمَةِ مَخَافَةُ الرَّبِّ. فِطْنَةٌ جَيِّدَةٌ لِكُلِّ عَامِلِيهَا. تَسْبِيحُهُ قَائِمٌ إِلَى الأَبَدِ.

\chapter{112}

\par 1 هَلِّلُويَا. طُوبَى لِلرَّجُلِ الْمُتَّقِي الرَّبَِّ الْمَسْرُورِ جِدّاً بِوَصَايَاهُ.
\par 2 نَسْلُهُ يَكُونُ قَوِيّاً فِي الأَرْضِ. جِيلُ الْمُسْتَقِيمِينَ يُبَارَكُ.
\par 3 رَغْدٌ وَغِنًى فِي بَيْتِهِ وَبِرُّهُ قَائِمٌ إِلَى الأَبَدِ.
\par 4 نُورٌ أَشْرَقَ فِي الظُّلْمَةِ لِلْمُسْتَقِيمِينَ. هُوَ حَنَّانٌ وَرَحِيمٌ وَصِدِّيقٌ.
\par 5 سَعِيدٌ هُوَ الرَّجُلُ الَّذِي يَتَرَأَّفُ وَيُقْرِضُ. يُدَبِّرُ أُمُورَهُ بِالْحَقِّ.
\par 6 لأَنَّهُ لاَ يَتَزَعْزَعُ إِلَى الدَّهْرِ. الصِّدِّيقُ يَكُونُ لِذِكْرٍ أَبَدِيٍّ.
\par 7 لاَ يَخْشَى مِنْ خَبَرِ سُوءٍ. قَلْبُهُ ثَابِتٌ مُتَّكِلاً عَلَى الرَّبِّ.
\par 8 قَلْبُهُ مُمَكَّنٌ فَلاَ يَخَافُ حَتَّى يَرَى بِمُضَايِقِيهِ.
\par 9 فَرَّقَ أَعْطَى الْمَسَاكِينَ. بِرُّهُ قَائِمٌ إِلَى الأَبَدِ. قَرْنُهُ يَنْتَصِبُ بِالْمَجْدِ.
\par 10 الشِّرِّيرُ يَرَى فَيَغْضَبُ. يُحَرِّقُ أَسْنَانَهُ وَيَذُوبُ. شَهْوَةُ الشِّرِّيرِ تَبِيدُ.

\chapter{113}

\par 1 هَلِّلُويَا. سَبِّحُوا يَا عَبِيدَ الرَّبِّ. سَبِّحُوا اسْمَ الرَّبِّ.
\par 2 لِيَكُنِ اسْمُ الرَّبِّ مُبَارَكاً مِنَ الآنَ وَإِلَى الأَبَدِ.
\par 3 مِنْ مَشْرِقِ الشَّمْسِ إِلَى مَغْرِبِهَا اسْمُ الرَّبِّ مُسَبَّحٌ.
\par 4 الرَّبُّ عَالٍ فَوْقَ كُلِّ الأُمَمِ. فَوْقَ السَّمَاوَاتِ مَجْدُهُ.
\par 5 مَنْ مِثْلُ الرَّبِّ إِلَهِنَا السَّاكِنِ فِي الأَعَالِي
\par 6 النَّاظِرِ الأَسَافِلَ فِي السَّمَاوَاتِ وَفِي الأَرْضِ
\par 7 الْمُقِيمِ الْمَِسْكِينَ مِنَ التُّرَابِ الرَّافِعِ الْبَائِسَ مِنَ الْمَزْبَلَةِ
\par 8 لِيُجْلِسَهُ مَعَ أَشْرَافٍ مَعَ أَشْرَافِ شَعْبِهِ.
\par 9 الْمُسْكِنِ الْعَاقِرَِ فِي بَيْتٍ أُمَّ أَوْلاَدٍ فَرْحَانَةً! هَلِّلُويَا.

\chapter{114}

\par 1 عِنْدَ خُرُوجِ إِسْرَائِيلَ مِنْ مِصْرَ وَبَيْتِ يَعْقُوبَ مِنْ شَعْبٍ أَعْجَمَ
\par 2 كَانَ يَهُوذَا مَقْدِسَهُ وَإِسْرَائِيلُ مَحَلَّ سُلْطَانِهِ.
\par 3 الْبَحْرُ رَآهُ فَهَرَبَ. الأُرْدُنُّ رَجَعَ إِلَى خَلْفٍ.
\par 4 الْجِبَالُ قَفَزَتْ مِثْلَ الْكِبَاشِ وَالآكَامُ مِثْلَ حُمْلاَنِ الْغَنَمِ.
\par 5 مَا لَكَ أَيُّهَا الْبَحْرُ قَدْ هَرَبْتَ وَمَا لَكَ أَيُّهَا الأُرْدُنُّ قَدْ رَجَعْتَ إِلَى خَلْفٍ
\par 6 وَمَا لَكُنَّ أَيَّتُهَا الْجِبَالُ قَدْ قَفَزْتُنَّ مِثْلَ الْكِبَاشِ وَأَيَّتُهَا التِّلاَلُ مِثْلَ حُمْلاَنِ الْغَنَمِ؟
\par 7 أَيَّتُهَا الأَرْضُ تَزَلْزَلِي مِنْ قُدَّامِ الرَّبِّ مِنْ قُدَّامِ إِلَهِ يَعْقُوبَ!
\par 8 الْمُحَوِّلِ الصَّخْرَةَ إِلَى غُدْرَانِ مِيَاهٍ الصَّوَّانَ إِلَى يَنَابِيعِ مِيَاهٍ.

\chapter{115}

\par 1 لَيْسَ لَنَا يَا رَبُّ لَيْسَ لَنَا لَكِنْ لاِسْمِكَ أَعْطِ مَجْداً مِنْ أَجْلِ رَحْمَتِكَ مِنْ أَجْلِ أَمَانَتِكَ.
\par 2 لِمَاذَا يَقُولُ الأُمَمُ: [أَيْنَ هُوَ إِلَهُهُمْ؟]
\par 3 إِنَّ إِلَهَنَا فِي السَّمَاءِ. كُلَّمَا شَاءَ صَنَعَ.
\par 4 أَصْنَامُهُمْ فِضَّةٌ وَذَهَبٌ عَمَلُ أَيْدِي النَّاسِ.
\par 5 لَهَا أَفْوَاهٌ وَلاَ تَتَكَلَّمُ. لَهَا أَعْيُنٌ وَلاَ تُبْصِرُ.
\par 6 لَهَا آذَانٌ وَلاَ تَسْمَعُ. لَهَا مَنَاخِرُ وَلاَ تَشُمُّ.
\par 7 لَهَا أَيْدٍ وَلاَ تَلْمِسُ. لَهَا أَرْجُلٌ وَلاَ تَمْشِي وَلاَ تَنْطِقُ بِحَنَاجِرِهَا.
\par 8 مِثْلَهَا يَكُونُ صَانِعُوهَا بَلْ كُلُّ مَنْ يَتَّكِلُ عَلَيْهَا.
\par 9 يَا إِسْرَائِيلُ اتَّكِلْ عَلَى الرَّبِّ. هُوَ مُعِينُهُمْ وَمِجَنُّهُمْ.
\par 10 يَا بَيْتَ هَارُونَ اتَّكِلُوا عَلَى الرَّبِّ. هُوَ مُعِينُهُمْ وَمِجَنُّهُمْ.
\par 11 يَا مُتَّقِي الرَّبِّ اتَّكِلُوا عَلَى الرَّبِّ. هُوَ مُعِينُهُمْ وَمِجَنُّهُمْ.
\par 12 الرَّبُّ قَدْ ذَكَرَنَا فَيُبَارِكُ. يُبَارِكُ بَيْتَ إِسْرَائِيلَ. يُبَارِكُ بَيْتَ هَارُونَ.
\par 13 يُبَارِكُ مُتَّقِي الرَّبِّ الصِّغَارَ مَعَ الْكِبَارِ.
\par 14 لِيَزِدِ الرَّبُّ عَلَيْكُمْ. عَلَيْكُمْ وَعَلَى أَبْنَائِكُمْ.
\par 15 أَنْتُمْ مُبَارَكُونَ لِلرَّبِّ الصَّانِعِ السَّمَاوَاتِ وَالأَرْضِ.
\par 16 السَّمَاوَاتُ سَمَاوَاتٌ لِلرَّبِّ أَمَّا الأَرْضُ فَأَعْطَاهَا لِبَنِي آدَمَ.
\par 17 لَيْسَ الأَمْوَاتُ يُسَبِّحُونَ الرَّبَّ وَلاَ مَنْ يَنْحَدِرُ إِلَى أَرْضِ السُّكُوتِ.
\par 18 أَمَّا نَحْنُ فَنُبَارِكُ الرَّبَّ مِنَ الآنَ وَإِلَى الدَّهْرِ. هَلِّلُويَا.

\chapter{116}

\par 1 أَحْبَبْتُ لأَنَّ الرَّبَّ يَسْمَعُ صَوْتِي تَضَرُّعَاتِي.
\par 2 لأَنَّهُ أَمَالَ أُذْنَهُ إِلَيَّ فَأَدْعُوهُ مُدَّةَ حَيَاتِي.
\par 3 اكْتَنَفَتْنِي حِبَالُ الْمَوْتِ. أَصَابَتْنِي شَدَائِدُ الْهَاوِيَةِ. كَابَدْتُ ضِيقاً وَحُزْناً.
\par 4 وَبِاسْمِ الرَّبِّ دَعَوْتُ: [آهِ يَا رَبُّ نَجِّ نَفْسِي].
\par 5 الرَّبُّ حَنَّانٌ وَصِدِّيقٌ وَإِلَهُنَا رَحِيمٌ.
\par 6 الرَّبُّ حَافِظُ الْبُسَطَاءِ. تَذَلَّلْتُ فَخَلَّصَنِي.
\par 7 ارْجِعِي يَا نَفْسِي إِلَى رَاحَتِكِ لأَنَّ الرَّبَّ قَدْ أَحْسَنَ إِلَيْكِ.
\par 8 لأَنَّكَ أَنْقَذْتَ نَفْسِي مِنَ الْمَوْتِ وَعَيْنِي مِنَ الدَّمْعَةِ وَرِجْلَيَّ مِنَ الزَّلَقِ.
\par 9 أَسْلُكُ قُدَّامَ الرَّبِّ فِي أَرْضِ الأَحْيَاءِ.
\par 10 آمَنْتُ لِذَلِكَ تَكَلَّمْتُ. أَنَا تَذَلَّلْتُ جِدّاً.
\par 11 أَنَا قُلْتُ فِي حَيْرَتِي: [كُلُّ إِنْسَانٍ كَاذِبٌ].
\par 12 مَاذَا أَرُدُّ لِلرَّبِّ مِنْ أَجْلِ كُلِّ حَسَنَاتِهِ لِي؟
\par 13 كَأْسَ الْخَلاَصِ أَتَنَاوَلُ وَبِاسْمِ الرَّبِّ أَدْعُو.
\par 14 أُوفِي نُذُورِي لِلرَّبِّ مُقَابِلَ كُلِّ شَعْبِهِ.
\par 15 عَزِيزٌ فِي عَيْنَيِ الرَّبِّ مَوْتُ أَتْقِيَائِهِ.
\par 16 آهِ يَا رَبُّ. لأَنِّي عَبْدُكَ. أَنَا عَبْدُكَ ابْنُ أَمَتِكَ. حَلَلْتَ قُيُودِي.
\par 17 فَلَكَ أَذْبَحُ ذَبِيحَةَ حَمْدٍ وَبِاسْمِ الرَّبِّ أَدْعُو.
\par 18 أُوفِي نُذُورِي لِلرَّبِّ مُقَابِلَ شَعْبِهِ
\par 19 فِي دِيَارِ بَيْتِ الرَّبِّ فِي وَسَطِكِ يَا أُورُشَلِيمُ. هَلِّلُويَا.

\chapter{117}

\par 1 سَبِّحُوا الرَّبَّ يَا كُلَّ الأُمَمِ. حَمِّدُوهُ يَا كُلَّ الشُّعُوبِ.
\par 2 لأَنَّ رَحْمَتَهُ قَدْ قَوِيَتْ عَلَيْنَا وَأَمَانَةُ الرَّبِّ إِلَى الدَّهْرِ. هَلِّلُويَا.

\chapter{118}

\par 1 اِحْمَدُوا الرَّبَّ لأَنَّهُ صَالِحٌ لأَنَّ إِلَى الأَبَدِ رَحْمَتَهُ.
\par 2 لِيَقُلْ إِسْرَائِيلُ: [إِنَّ إِلَى الأَبَدِ رَحْمَتَهُ].
\par 3 لِيَقُلْ بَيْتُ هَارُونَ: [إِنَّ إِلَى الأَبَدِ رَحْمَتَهُ].
\par 4 لِيَقُلْ مُتَّقُو الرَّبِّ: [إِنَّ إِلَى الأَبَدِ رَحْمَتَهُ].
\par 5 مِنَ الضِّيقِ دَعَوْتُ الرَّبَّ فَأَجَابَنِي مِنَ الرُّحْبِ.
\par 6 الرَّبُّ لِي فَلاَ أَخَافُ. مَاذَا يَصْنَعُ بِي الإِنْسَانُ؟
\par 7 الرَّبُّ لِي بَيْنَ مُعِينِيَّ وَأَنَا سَأَرَى بِأَعْدَائِي.
\par 8 الاِحْتِمَاءُ بِالرَّبِّ خَيْرٌ مِنَ التَّوَكُّلِ عَلَى إِنْسَانٍ.
\par 9 الاِحْتِمَاءُ بِالرَّبِّ خَيْرٌ مِنَ التَّوَكُّلِ عَلَى الرُّؤَسَاءِ.
\par 10 كُلُّ الأُمَمِ أَحَاطُوا بِي. بِاسْمِ الرَّبِّ أُبِيدُهُمْ.
\par 11 أَحَاطُوا بِي وَاكْتَنَفُونِي. بِاسْمِ الرَّبِّ أُبِيدُهُمْ.
\par 12 أَحَاطُوا بِي مِثْلَ النَّحْلِ. انْطَفَأُوا كَنَارِ الشَّوْكِ. بِاسْمِ الرَّبِّ أُبِيدُهُمْ.
\par 13 دَحَرْتَنِي دُحُوراً لأَسْقُطَ. أَمَّا الرَّبُّ فَعَضَدَنِي.
\par 14 قُوَّتِي وَتَرَنُّمِي الرَّبُّ وَقَدْ صَارَ لِي خَلاَصاً.
\par 15 صَوْتُ تَرَنُّمٍ وَخَلاَصٍ فِي خِيَامِ الصِّدِّيقِينَ. يَمِينُ الرَّبِّ صَانِعَةٌ بِبَأْسٍ.
\par 16 يَمِينُ الرَّبِّ مُرْتَفِعَةٌ. يَمِينُ الرَّبِّ صَانِعَةٌ بِبَأْسٍ.
\par 17 لاَ أَمُوتُ بَلْ أَحْيَا وَأُحَدِّثُ بِأَعْمَالِ الرَّبِّ.
\par 18 تَأْدِيباً أَدَّبَنِي الرَّبُّ وَإِلَى الْمَوْتِ لَمْ يُسْلِمْنِي.
\par 19 اِفْتَحُوا لِي أَبْوَابَ الْبِرِّ. أَدْخُلْ فِيهَا وَأَحْمَدِ الرَّبَّ.
\par 20 هَذَا الْبَابُ لِلرَّبِّ. الصِّدِّيقُونَ يَدْخُلُونَ فِيهِ.
\par 21 أَحْمَدُكَ لأَنَّكَ اسْتَجَبْتَ لِي وَصِرْتَ لِي خَلاَصاً.
\par 22 الْحَجَرُ الَّذِي رَفَضَهُ الْبَنَّاؤُونَ قَدْ صَارَ رَأْسَ الزَّاوِيَةِ.
\par 23 مِنْ قِبَلِ الرَّبِّ كَانَ هَذَا وَهُوَ عَجِيبٌ فِي أَعْيُنِنَا.
\par 24 هَذَا هُوَ الْيَوْمُ الَّذِي صَنَعَهُ الرَّبُّ. نَبْتَهِجُ وَنَفْرَحُ فِيهِ.
\par 25 آهِ يَا رَبُّ خَلِّصْ! آهِ يَا رَبُّ أَنْقِذْ!
\par 26 مُبَارَكٌ الآتِي بِاسْمِ الرَّبِّ. بَارَكْنَاكُمْ مِنْ بَيْتِ الرَّبِّ.
\par 27 الرَّبُّ هُوَ اللهُ وَقَدْ أَنَارَ لَنَا. أَوْثِقُوا الذَّبِيحَةَ بِرُبُطٍ إِلَى قُرُونِ الْمَذْبَحِ.
\par 28 إِلَهِي أَنْتَ فَأَحْمَدُكَ. إِلَهِي فَأَرْفَعُكَ.
\par 29 احْمَدُوا الرَّبَّ لأَنَّهُ صَالِحٌ لأَنَّ إِلَى الأَبَدِ رَحْمَتَهُ.

\chapter{119}

\par 1 طُوبَى لِلْكَامِلِينَ طَرِيقاً السَّالِكِينَ فِي شَرِيعَةِ الرَّبِّ.
\par 2 طُوبَى لِحَافِظِي شَهَادَاتِهِ. مِنْ كُلِّ قُلُوبِهِمْ يَطْلُبُونَهُ.
\par 3 أَيْضاً لاَ يَرْتَكِبُونَ إِثْماً. فِي طُرُقِهِ يَسْلُكُونَ.
\par 4 أَنْتَ أَوْصَيْتَ بِوَصَايَاكَ أَنْ تُحْفَظَ تَمَاماً.
\par 5 لَيْتَ طُرُقِي تُثَبَّتُ فِي حِفْظِ فَرَائِضِكَ.
\par 6 حِينَئِذٍ لاَ أَخْزَى إِذَا نَظَرْتُ إِلَى كُلِّ وَصَايَاكَ.
\par 7 أَحْمَدُكَ بِاسْتِقَامَةِ قَلْبٍ عِنْدَ تَعَلُّمِي أَحْكَامَ عَدْلِكَ.
\par 8 وَصَايَاكَ أَحْفَظُ. لاَ تَتْرُكْنِي إِلَى الْغَايَةِ.
\par 9 ب بِمَ يُزَكِّي الشَّابُّ طَرِيقَهُ؟ بِحِفْظِهِ إِيَّاهُ حَسَبَ كَلاَمِكَ.
\par 10 بِكُلِّ قَلْبِي طَلَبْتُكَ. لاَ تُضِلَّنِي عَنْ وَصَايَاكَ.
\par 11 خَبَّأْتُ كَلاَمَكَ فِي قَلْبِي لِكَيْلاَ أُخْطِئَ إِلَيْكَ.
\par 12 مُبَارَكٌ أَنْتَ يَا رَبُّ. عَلِّمْنِي فَرَائِضَكَ.
\par 13 بِشَفَتَيَّ حَسَبْتُ كُلَّ أَحْكَامِ فَمِكَ.
\par 14 بِطَرِيقِ شَهَادَاتِكَ فَرِحْتُ كَمَا عَلَى كُلِّ الْغِنَى.
\par 15 بِوَصَايَاكَ أَلْهَجُ وَأُلاَحِظُ سُبُلَكَ.
\par 16 بِفَرَائِضِكَ أَتَلَذَّذُ. لاَ أَنْسَى كَلاَمَكَ.
\par 17 ج أَحْسِنْ إِلَى عَبْدِكَ فَأَحْيَا وَأَحْفَظَ أَمْرَكَ.
\par 18 اكْشِفْ عَنْ عَيْنَيَّ فَأَرَى عَجَائِبَ مِنْ شَرِيعَتِكَ.
\par 19 غَرِيبٌ أَنَا فِي الأَرْضِ. لاَ تُخْفِ عَنِّي وَصَايَاكَ.
\par 20 انْسَحَقَتْ نَفْسِي شَوْقاً إِلَى أَحْكَامِكَ فِي كُلِّ حِينٍ.
\par 21 انْتَهَرْتَ الْمُتَكَبِّرِينَ الْمَلاَعِينَ الضَّالِّينَ عَنْ وَصَايَاكَ.
\par 22 دَحْرِجْ عَنِّي الْعَارَ وَالإِهَانَةَ لأَنِّي حَفِظْتُ شَهَادَاتِكَ.
\par 23 جَلَسَ أَيْضاً رُؤَسَاءُ تَقَاوَلُوا عَلَيَّ. أَمَّا عَبْدُكَ فَيُنَاجِي بِفَرَائِضِكَ.
\par 24 أَيْضاً شَهَادَاتُكَ هِيَ لَذَّتِي أَهْلُ مَشُورَتِي.
\par 25 د لَصِقَتْ بِالتُّرَابِ نَفْسِي فَأَحْيِنِي حَسَبَ كَلِمَتِكَ.
\par 26 قَدْ صَرَّحْتُ بِطُرُقِي فَاسْتَجَبْتَ لِي. عَلِّمْنِي فَرَائِضَكَ.
\par 27 طَرِيقَ وَصَايَاكَ فَهِّمْنِي فَأُنَاجِيَ بِعَجَائِبِكَ.
\par 28 قَطَرَتْ نَفْسِي مِنَ الْحُزْنِ. أَقِمْنِي حَسَبَ كَلاَمِكَ.
\par 29 طَرِيقَ الْكَذِبِ أَبْعِدْ عَنِّي وَبِشَرِيعَتِكَ ارْحَمْنِي.
\par 30 اخْتَرْتُ طَرِيقَ الْحَقِّ. جَعَلْتُ أَحْكَامَكَ قُدَّامِي.
\par 31 لَصِقْتُ بِشَهَادَاتِكَ. يَا رَبُّ لاَ تُخْزِنِي.
\par 32 فِي طَرِيقِ وَصَايَاكَ أَجْرِي لأَنَّكَ تُرَحِّبُ قَلْبِي.
\par 33 هـ عَلِّمْنِي يَا رَبُّ طَرِيقَ فَرَائِضِكَ فَأَحْفَظَهَا إِلَى النِّهَايَةِ.
\par 34 فَهِّمْنِي فَأُلاَحِظَ شَرِيعَتَكَ وَأَحْفَظَهَا بِكُلِّ قَلْبِي.
\par 35 دَرِّبْنِي فِي سَبِيلِ وَصَايَاكَ لأَنِّي بِهِ سُرِرْتُ.
\par 36 أَمِلْ قَلْبِي إِلَى شَهَادَاتِكَ لاَ إِلَى الْمَكْسَبِ.
\par 37 حَوِّلْ عَيْنَيَّ عَنِ النَّظَرِ إِلَى الْبَاطِلِ. فِي طَرِيقِكَ أَحْيِنِي.
\par 38 أَقِمْ لِعَبْدِكَ قَوْلَكَ الَّذِي لِمُتَّقِيكَ.
\par 39 أَزِلْ عَارِي الَّذِي حَذِرْتُ مِنْهُ لأَنَّ أَحْكَامَكَ طَيِّبَةٌ.
\par 40 هَئَنَذَا قَدِ اشْتَهَيْتُ وَصَايَاكَ. بِعَدْلِكَ أَحْيِنِي.
\par 41 و لِتَأْتِنِي رَحْمَتُكَ يَا رَبُّ خَلاَصُكَ حَسَبَ قَوْلِكَ
\par 42 فَأُجَاوِبَ مُعَيِّرِي كَلِمَةً لأَنِّي اتَّكَلْتُ عَلَى كَلاَمِكَ.
\par 43 وَلاَ تَنْزِعْ مِنْ فَمِي كَلاَمَ الْحَقِّ كُلَّ النَّزْعِ لأَنِّي انْتَظَرْتُ أَحْكَامَكَ.
\par 44 فَأَحْفَظَ شَرِيعَتَكَ دَائِماً إِلَى الدَّهْرِ وَالأَبَدِ
\par 45 وَأَتَمَشَّى فِي رُحْبٍ لأَنِّي طَلَبْتُ وَصَايَاكَ
\par 46 وَأَتَكَلَّمُ بِشَهَادَاتِكَ قُدَّامَ مُلُوكٍ وَلاَ أَخْزَى
\par 47 وَأَتَلَذَّذُ بِوَصَايَاكَ الَّتِي أَحْبَبْتُ
\par 48 وَأَرْفَعُ يَدَيَّ إِلَى وَصَايَاكَ الَّتِي وَدِدْتُ وَأُنَاجِي بِفَرَائِضِكَ.
\par 49 ز اُذْكُرْ لِعَبْدِكَ الْقَوْلَ الَّذِي جَعَلْتَنِي أَنْتَظِرُهُ.
\par 50 هَذِهِ هِيَ تَعْزِيَتِي فِي مَذَلَّتِي لأَنَّ قَوْلَكَ أَحْيَانِي.
\par 51 الْمُتَكَبِّرُونَ اسْتَهْزَأُوا بِي إِلَى الْغَايَةِ. عَنْ شَرِيعَتِكَ لَمْ أَمِلْ.
\par 52 تَذَكَّرْتُ أَحْكَامَكَ مُنْذُ الدَّهْرِ يَا رَبُّ فَتَعَزَّيْتُ.
\par 53 الْحَمِيَّةُ أَخَذَتْنِي بِسَبَبِ الأَشْرَارِ تَارِكِي شَرِيعَتِكَ.
\par 54 تَرْنِيمَاتٍ صَارَتْ لِي فَرَائِضُكَ فِي بَيْتِ غُرْبَتِي.
\par 55 ذَكَرْتُ فِي اللَّيْلِ اسْمَكَ يَا رَبُّ وَحَفِظْتُ شَرِيعَتَكَ.
\par 56 هَذَا صَارَ لِي لأَنِّي حَفِظْتُ وَصَايَاكَ.
\par 57 ح نَصِيبِي الرَّبُّ قُلْتُ لِحِفْظِ كَلاَمِكَ.
\par 58 تَرَضَّيْتُ وَجْهَكَ بِكُلِّ قَلْبِي. ارْحَمْنِي حَسَبَ قَوْلِكَ.
\par 59 تَفَكَّرْتُ فِي طُرُقِي وَرَدَدْتُ قَدَمَيَّ إِلَى شَهَادَاتِكَ.
\par 60 أَسْرَعْتُ وَلَمْ أَتَوَانَ لِحِفْظِ وَصَايَاكَ.
\par 61 حِبَالُ الأَشْرَارِ الْتَفَّتْ عَلَيَّ. أَمَّا شَرِيعَتُكَ فَلَمْ أَنْسَهَا.
\par 62 فِي مُنْتَصَفِ اللَّيْلِ أَقُومُ لأَحْمَدَكَ عَلَى أَحْكَامِ بِرِّكَ.
\par 63 رَفِيقٌ أَنَا لِكُلِّ الَّذِينَ يَتَّقُونَكَ وَلِحَافِظِي وَصَايَاكَ.
\par 64 رَحْمَتُكَ يَا رَبُّ قَدْ مَلَأَتِ الأَرْضَ. عَلِّمْنِي فَرَائِضَكَ.
\par 65 ط خَيْراً صَنَعْتَ مَعَ عَبْدِكَ يَا رَبُّ حَسَبَ كَلاَمِكَ.
\par 66 ذَوْقاً صَالِحاً وَمَعْرِفَةً عَلِّمْنِي لأَنِّي بِوَصَايَاكَ آمَنْتُ.
\par 67 قَبْلَ أَنْ أُذَلَّلَ أَنَا ضَلَلْتُ أَمَّا الآنَ فَحَفِظْتُ قَوْلَكَ.
\par 68 صَالِحٌ أَنْتَ وَمُحْسِنٌ. عَلِّمْنِي فَرَائِضَكَ.
\par 69 الْمُتَكَبِّرُونَ قَدْ لَفَّقُوا عَلَيَّ كَذِباً أَمَّا أَنَا فَبِكُلِّ قَلْبِي أَحْفَظُ وَصَايَاكَ.
\par 70 سَمِنَ مِثْلَ الشَّحْمِ قَلْبُهُمْ أَمَّا أَنَا فَبِشَرِيعَتِكَ أَتَلَذَّذُ.
\par 71 خَيْرٌ لِي أَنِّي تَذَلَّلْتُ لِكَيْ أَتَعَلَّمَ فَرَائِضَكَ.
\par 72 شَرِيعَةُ فَمِكَ خَيْرٌ لِي مِنْ أُلُوفِ ذَهَبٍ وَفِضَّةٍ.
\par 73 ي يَدَاكَ صَنَعَتَانِي وَأَنْشَأَتَانِي. فَهِّمْنِي فَأَتَعَلَّمَ وَصَايَاكَ.
\par 74 مُتَّقُوكَ يَرُونَنِي فَيَفْرَحُونَ لأَنِّي انْتَظَرْتُ كَلاَمَكَ.
\par 75 قَدْ عَلِمْتُ يَا رَبُّ أَنَّ أَحْكَامَكَ عَدْلٌ وَبِالْحَقِّ أَذْلَلْتَنِي.
\par 76 فَلْتَصِرْ رَحْمَتُكَ لِتَعْزِيَتِي حَسَبَ قَوْلِكَ لِعَبْدِكَ.
\par 77 لِتَأْتِنِي مَرَاحِمُكَ فَأَحْيَا لأَنَّ شَرِيعَتَكَ هِيَ لَذَّتِي.
\par 78 لِيَخْزَ الْمُتَكَبِّرُونَ لأَنَّهُمْ زُوراً افْتَرُوا عَلَيَّ. أَمَّا أَنَا فَأُنَاجِي بِوَصَايَاكَ.
\par 79 لِيَرْجِعْ إِلَيَّ مُتَّقُوكَ وَعَارِفُو شَهَادَاتِكَ.
\par 80 لِيَكُنْ قَلْبِي كَامِلاً فِي فَرَائِضِكَ لِكَيْ لاَ أَخْزَى.
\par 81 ك تَاقَتْ نَفْسِي إِلَى خَلاَصِكَ. كَلاَمَكَ انْتَظَرْتُ.
\par 82 كَلَّتْ عَيْنَايَ مِنَ النَّظَرِ إِلَى قَوْلِكَ فَأَقُولُ: [مَتَى تُعَزِّينِي؟]
\par 83 لأَنِّي قَدْ صِرْتُ كَزِقٍّ فِي الدُّخَانِ. أَمَّا فَرَائِضُكَ فَلَمْ أَنْسَهَا.
\par 84 كَمْ هِيَ أَيَّامُ عَبْدِكَ؟ مَتَى تُجْرِي حُكْماً عَلَى مُضْطَهِدِيَّ؟
\par 85 الْمُتَكَبِّرُونَ قَدْ كَرُوا لِي حَفَائِرَ. ذَلِكَ لَيْسَ حَسَبَ شَرِيعَتِكَ.
\par 86 كُلُّ وَصَايَاكَ أَمَانَةٌ. زُوراً يَضْطَهِدُونَنِي. أَعِنِّي.
\par 87 لَوْلاَ قَلِيلٌ لَأَفْنُونِي مِنَ الأَرْضِ. أَمَّا أَنَا فَلَمْ أَتْرُكْ وَصَايَاكَ.
\par 88 حَسَبَ رَحْمَتِكَ أَحْيِنِي فَأَحْفَظَ شَهَادَاتِ فَمِكَ.
\par 89 ل إِلَى الأَبَدِ يَا رَبُّ كَلِمَتُكَ مُثَبَّتَةٌ فِي السَّمَاوَاتِ.
\par 90 إِلَى دَوْرٍ فَدَوْرٍ أَمَانَتُكَ. أَسَّسْتَ الأَرْضَ فَثَبَتَتْ.
\par 91 عَلَى أَحْكَامِكَ ثَبَتَتِ الْيَوْمَ لأَنَّ الْكُلَّ عَبِيدُكَ.
\par 92 لَوْ لَمْ تَكُنْ شَرِيعَتُكَ لَذَّتِي لَهَلَكْتُ حِينَئِذٍ فِي مَذَلَّتِي.
\par 93 إِلَى الدَّهْرِ لاَ أَنْسَى وَصَايَاكَ لأَنَّكَ بِهَا أَحْيَيْتَنِي.
\par 94 لَكَ أَنَا فَخَلِّصْنِي لأَنِّي طَلَبْتُ وَصَايَاكَ.
\par 95 إِيَّايَ انْتَظَرَ الأَشْرَارُ لِيُهْلِكُونِي. بِشَهَادَاتِكَ أَفْطَنُ.
\par 96 لِكُلِّ كَمَالٍ رَأَيْتُ حَدّاً أَمَّا وَصِيَّتُكَ فَوَاسِعَةٌ جِدّاً.
\par 97 م كَمْ أَحْبَبْتُ شَرِيعَتَكَ! الْيَوْمَ كُلَّهُ هِيَ لَهَجِي.
\par 98 وَصِيَّتُكَ جَعَلَتْنِي أَحْكَمَ مِنْ أَعْدَائِي لأَنَّهَا إِلَى الدَّهْرِ هِيَ لِي.
\par 99 أَكْثَرَ مِنْ كُلِّ مُعَلِّمِيَّ تَعَقَّلْتُ لأَنَّ شَهَادَاتِكَ هِيَ لَهَجِي.
\par 100 أَكْثَرَ مِنَ الشُّيُوخِ فَطِنْتُ لأَنِّي حَفِظْتُ وَصَايَاكَ.
\par 101 مِنْ كُلِّ طَرِيقِ شَرٍّ مَنَعْتُ رِجْلَيَّ لِكَيْ أَحْفَظَ كَلاَمَكَ.
\par 102 عَنْ أَحْكَامِكَ لَمْ أَمِلْ لأَنَّكَ أَنْتَ عَلَّمْتَنِي.
\par 103 مَا أَحْلَى قَوْلَكَ لِحَنَكِي! أَحْلَى مِنَ الْعَسَلِ لِفَمِي.
\par 104 مِنْ وَصَايَاكَ أَتَفَطَّنُ لِذَلِكَ أَبْغَضْتُ كُلَّ طَرِيقِ كَذِبٍ.
\par 105 ن سِرَاجٌ لِرِجْلِي كَلاَمُكَ وَنُورٌ لِسَبِيلِي.
\par 106 حَلَفْتُ فَأَبِرُّهُ أَنْ أَحْفَظَ أَحْكَامَ بِرِّكَ.
\par 107 تَذَلَّلْتُ إِلَى الْغَايَةِ. يَا رَبُّ أَحْيِنِي حَسَبَ كَلاَمِكَ.
\par 108 ارْتَضِ بِمَنْدُوبَاتِ فَمِي يَا رَبُّ وَأَحْكَامَكَ عَلِّمْنِي.
\par 109 نَفْسِي دَائِماً فِي كَفِّي أَمَّا شَرِيعَتُكَ فَلَمْ أَنْسَهَا.
\par 110 الأَشْرَارُ وَضَعُوا لِي فَخّاً أَمَّا وَصَايَاكَ فَلَمْ أَضِلَّ عَنْهَا.
\par 111 وَرَثْتُ شَهَادَاتِكَ إِلَى الدَّهْرِ لأَنَّهَا هِيَ بَهْجَةُ قَلْبِي.
\par 112 عَطَفْتُ قَلْبِي لأَصْنَعَ فَرَائِضَكَ إِلَى الدَّهْرِ إِلَى النِّهَايَةِ.
\par 113 س الْمُتَقَلِّبِينَ أَبْغَضْتُ وَشَرِيعَتَكَ أَحْبَبْتُ.
\par 114 سِتْرِي وَمِجَنِّي أَنْتَ. كَلاَمَكَ انْتَظَرْتُ.
\par 115 انْصَرِفُوا عَنِّي أَيُّهَا الأَشْرَارُ فَأَحْفَظَ وَصَايَا إِلَهِي.
\par 116 اعْضُدْنِي حَسَبَ قَوْلِكَ فَأَحْيَا وَلاَ تُخْزِنِي مِنْ رَجَائِي.
\par 117 أَسْنِدْنِي فَأَخْلُصَ وَأُرَاعِيَ فَرَائِضَكَ دَائِماً.
\par 118 احْتَقَرْتَ كُلَّ الضَّالِّينَ عَنْ فَرَائِضِكَ لأَنَّ مَكْرَهُمْ بَاطِلٌ.
\par 119 كَزَغَلٍ عَزَلْتَ كُلَّ أَشْرَارِ الأَرْضِ لِذَلِكَ أَحْبَبْتُ شَهَادَاتِكَ.
\par 120 قَدِ اقْشَعَرَّ لَحْمِي مِنْ رُعْبِكَ وَمِنْ أَحْكَامِكَ جَزِعْتُ.
\par 121 ع أَجْرَيْتُ حُكْماً وَعَدْلاً. لاَ تُسْلِمْنِي إِلَى ظَالِمِيَّ.
\par 122 كُنْ ضَامِنَ عَبْدِكَ لِلْخَيْرِ لِكَيْ لاَ يَظْلِمَنِي الْمُسْتَكْبِرُونَ.
\par 123 كَلَّتْ عَيْنَايَ اشْتِيَاقاً إِلَى خَلاَصِكَ وَإِلَى كَلِمَةِ بِرِّكَ.
\par 124 اصْنَعْ مَعَ عَبْدِكَ حَسَبَ رَحْمَتِكَ وَفَرَائِضَكَ عَلِّمْنِي.
\par 125 عَبْدُكَ أَنَا. فَهِّمْنِي فَأَعْرِفَ شَهَادَاتِكَ.
\par 126 إِنَّهُ وَقْتُ عَمَلٍ لِلرَّبِّ. قَدْ نَقَضُوا شَرِيعَتَكَ.
\par 127 لأَجْلِ ذَلِكَ أَحْبَبْتُ وَصَايَاكَ أَكْثَرَ مِنَ الذَّهَبِ وَالإِبْرِيزِ.
\par 128 لأَجْلِ ذَلِكَ حَسِبْتُ كُلَّ وَصَايَاكَ فِي كُلِّ شَيْءٍ مُسْتَقِيمَةً. كُلَّ طَرِيقِ كَذِبٍ أَبْغَضْتُ.
\par 129 ف عَجِيبَةٌ هِيَ شَهَادَاتُكَ لِذَلِكَ حَفِظَتْهَا نَفْسِي.
\par 130 فَتْحُ كَلاَمِكَ يُنِيرُ يُعَقِّلُ الْجُهَّالَ.
\par 131 فَغَرْتُ فَمِي وَلَهَثْتُ لأَنِّي إِلَى وَصَايَاكَ اشْتَقْتُ.
\par 132 الْتَفِتْ إِلَيَّ وَارْحَمْنِي كَحَقِّ مُحِبِّي اسْمِكَ.
\par 133 ثَبِّتْ خَطَواتِي فِي كَلِمَتِكَ وَلاَ يَتَسَلَّطْ عَلَيَّ إِثْمٌ.
\par 134 افْدِنِي مِنْ ظُلْمِ الإِنْسَانِ فَأَحْفَظَ وَصَايَاكَ.
\par 135 أَضِيءْ بِوَجْهِكَ عَلَى عَبْدِكَ وَعَلِّمْنِي فَرَائِضَكَ.
\par 136 جَدَاوِلُ مِيَاهٍ جَرَتْ مِنْ عَيْنَيَّ لأَنَّهُمْ لَمْ يَحْفَظُوا شَرِيعَتَكَ.
\par 137 ص بَارٌّ أَنْتَ يَا رَبُّ وَأَحْكَامُكَ مُسْتَقِيمَةٌ.
\par 138 عَدْلاً أَمَرْتَ بِشَهَادَاتِكَ وَحَقّاً إِلَى الْغَايَةِ.
\par 139 أَهْلَكَتْنِي غَيْرَتِي لأَنَّ أَعْدَائِي نَسُوا كَلاَمَكَ.
\par 140 كَلِمَتُكَ مُمَحَّصَةٌ جِدّاً وَعَبْدُكَ أَحَبَّهَا.
\par 141 صَغِيرٌ أَنَا وَحَقِيرٌ أَمَّا وَصَايَاكَ فَلَمْ أَنْسَهَا.
\par 142 عَدْلُكَ عَدْلٌ إِلَى الدَّهْرِ وَشَرِيعَتُكَ حَقٌّ.
\par 143 ضِيقٌ وَشِدَّةٌ أَصَابَانِي أَمَّا وَصَايَاكَ فَهِيَ لَذَّاتِي.
\par 144 عَادِلَةٌ شَهَادَاتُكَ إِلَى الدَّهْرِ. فَهِّمْنِي فَأَحْيَا.
\par 145 ق صَرَخْتُ مِنْ كُلِّ قَلْبِي. اسْتَجِبْ لِي يَا رَبُّ. فَرَائِضَكَ أَحْفَظُ.
\par 146 دَعَوْتُكَ. خَلِّصْنِي فَأَحْفَظَ شَهَادَاتِكَ.
\par 147 تَقَدَّمْتُ فِي الصُّبْحِ وَصَرَخْتُ. كَلاَمَكَ انْتَظَرْتُ.
\par 148 تَقَدَّمَتْ عَيْنَايَ الْهُزُعَ لِكَيْ أَلْهَجَ بِأَقْوَالِكَ.
\par 149 صَوْتِيَ اسْتَمِعْ حَسَبَ رَحْمَتِكَ. يَا رَبُّ حَسَبَ أَحْكَامِكَ أَحْيِنِي.
\par 150 اقْتَرَبَ التَّابِعُونَ الرَّذِيلَةَ. عَنْ شَرِيعَتِكَ بَعُدُوا.
\par 151 قَرِيبٌ أَنْتَ يَا رَبُّ وَكُلُّ وَصَايَاكَ حَقٌّ.
\par 152 مُنْذُ زَمَانٍ عَرَفْتُ مِنْ شَهَادَاتِكَ أَنَّكَ إِلَى الدَّهْرِ أَسَّسْتَهَا.
\par 153 ر اُنْظُرْ إِلَى ذُلِّي وَأَنْقِذْنِي لأَنِّي لَمْ أَنْسَ شَرِيعَتَكَ.
\par 154 أَحْسِنْ دَعْوَايَ وَفُكَّنِي. حَسَبَ كَلِمَتِكَ أَحْيِنِي.
\par 155 الْخَلاَصُ بَعِيدٌ عَنِ الأَشْرَارِ لأَنَّهُمْ لَمْ يَلْتَمِسُوا فَرَائِضَكَ.
\par 156 كَثِيرَةٌ هِيَ مَرَاحِمُكَ يَا رَبُّ. حَسَبَ أَحْكَامِكَ أَحْيِنِي.
\par 157 كَثِيرُونَ مُضْطَهِدِيَّ وَمُضَايِقِيَّ. أَمَّا شَهَادَاتُكَ فَلَمْ أَمِلْ عَنْهَا.
\par 158 رَأَيْتُ الْغَادِرِينَ وَمَقَتُّ لأَنَّهُمْ لَمْ يَحْفَظُوا كَلِمَتَكَ.
\par 159 انْظُرْ أَنِّي أَحْبَبْتُ وَصَايَاكَ. يَا رَبُّ حَسَبَ رَحْمَتِكَ أَحْيِنِي.
\par 160 رَأْسُ كَلاَمِكَ حَقٌّ وَإِلَى الدَّهْرِ كُلُّ أَحْكَامِ عَدْلِكَ.
\par 161 ش رُؤَسَاءُ اضْطَهَدُونِي بِلاَ سَبَبٍ وَمِنْ كَلاَمِكَ جَزِعَ قَلْبِي.
\par 162 أَبْتَهِجُ أَنَا بِكَلاَمِكَ كَمَنْ وَجَدَ غَنِيمَةً وَافِرَةً.
\par 163 أَبْغَضْتُ الْكَذِبَ وَكَرِهْتُهُ أَمَّا شَرِيعَتُكَ فَأَحْبَبْتُهَا.
\par 164 سَبْعَ مَرَّاتٍ فِي النَّهَارِ سَبَّحْتُكَ عَلَى أَحْكَامِ عَدْلِكَ.
\par 165 سَلاَمَةٌ جَزِيلَةٌ لِمُحِبِّي شَرِيعَتِكَ وَلَيْسَ لَهُمْ مَعْثَرَةٌ.
\par 166 رَجَوْتُ خَلاَصَكَ يَا رَبُّ وَوَصَايَاكَ عَمِلْتُ.
\par 167 حَفِظَتْ نَفْسِي شَهَادَاتِكَ وَأُحِبُّهَا جِدّاً.
\par 168 حَفِظْتُ وَصَايَاكَ وَشَهَادَاتِكَ لأَنَّ كُلَّ طُرُقِي أَمَامَكَ.
\par 169 ت لِيَبْلُغْ صُرَاخِي إِلَيْكَ يَا رَبُّ. حَسَبَ كَلاَمِكَ فَهِّمْنِي.
\par 170 لِتَدْخُلْ طِلْبَتِي إِلَى حَضْرَتِكَ. كَكَلِمَتِكَ نَجِّنِي.
\par 171 تُنَبِّعُ شَفَتَايَ تَسْبِيحاً إِذَا عَلَّمْتَنِي فَرَائِضَكَ.
\par 172 يُغَنِّي لِسَانِي بِأَقْوَالِكَ لأَنَّ كُلَّ وَصَايَاكَ عَدْلٌ.
\par 173 لِتَكُنْ يَدُكَ لِمَعُونَتِي لأَنَّنِي اخْتَرْتُ وَصَايَاكَ.
\par 174 اشْتَقْتُ إِلَى خَلاَصِكَ يَا رَبُّ وَشَرِيعَتُكَ هِيَ لَذَّتِي.
\par 175 لِتَحْيَ نَفْسِي وَتُسَبِّحَكَ وَأَحْكَامُكَ لِتُعِنِّي.
\par 176 ضَلَلْتُ كَشَاةٍ ضَالَّةٍ. اطْلُبْ عَبْدَكَ لأَنِّي لَمْ أَنْسَ وَصَايَاكَ.

\chapter{120}

\par 1 تَرْنِيمَةُ الْمَصَاعِدِ إِلَى الرَّبِّ فِي ضِيقِي صَرَخْتُ فَاسْتَجَابَ لِي.
\par 2 يَا رَبُّ نَجِّ نَفْسِي مِنْ شِفَاهِ الْكَذِبِ مِنْ لِسَانِ غِشٍّ.
\par 3 مَاذَا يُعْطِيكَ وَمَاذَا يَزِيدُ لَكَ لِسَانُ الْغِشِّ؟
\par 4 سِهَامَ جَبَّارٍ مَسْنُونَةً مَعَ جَمْرِ الرَّتَمِ.
\par 5 وَيْلِي لِغُرْبَتِي فِي مَاشِكَ لِسَكَنِي فِي خِيَامِ قِيدَارَ!
\par 6 طَالَ عَلَى نَفْسِي سَكَنُهَا مَعَ مُبْغِضِ السَّلاَمِ.
\par 7 أَنَا سَلاَمٌ وَحِينَمَا أَتَكَلَّمُ فَهُمْ لِلْحَرْبِ.

\chapter{121}

\par 1 تَرْنِيمَةُ الْمَصَاعِدِ أَرْفَعُ عَيْنَيَّ إِلَى الْجِبَالِ مِنْ حَيْثُ يَأْتِي عَوْنِي.
\par 2 مَعُونَتِي مِنْ عِنْدِ الرَّبِّ صَانِعِ السَّمَاوَاتِ وَالأَرْضِ.
\par 3 لاَ يَدَعُ رِجْلَكَ تَزِلُّ. لاَ يَنْعَسُ حَافِظُكَ.
\par 4 إِنَّهُ لاَ يَنْعَسُ وَلاَ يَنَامُ حَافِظُ إِسْرَائِيلَ.
\par 5 الرَّبُّ حَافِظُكَ. الرَّبُّ ظِلٌّ لَكَ عَنْ يَدِكَ الْيُمْنَى.
\par 6 لاَ تَضْرِبُكَ الشَّمْسُ فِي النَّهَارِ وَلاَ الْقَمَرُ فِي اللَّيْلِ.
\par 7 الرَّبُّ يَحْفَظُكَ مِنْ كُلِّ شَرٍّ. يَحْفَظُ نَفْسَكَ.
\par 8 الرَّبُّ يَحْفَظُ خُرُوجَكَ وَدُخُولَكَ مِنَ الآنَ وَإِلَى الدَّهْرِ.

\chapter{122}

\par 1 تَرْنِيمَةُ الْمَصَاعِدِ. لِدَاوُدَ فَرِحْتُ بِالْقَائِلِينَ لِي: [إِلَى بَيْتِ الرَّبِّ نَذْهَبُ].
\par 2 تَقِفُ أَرْجُلُنَا فِي أَبْوَابِكِ يَا أُورُشَلِيمُ.
\par 3 أُورُشَلِيمُ الْمَبْنِيَّةُ كَمَدِينَةٍ مُتَّصِلَةٍ كُلِّهَا
\par 4 حَيْثُ صَعِدَتِ الأَسْبَاطُ أَسْبَاطُ الرَّبِّ شَهَادَةً لإِسْرَائِيلَ لِيَحْمَدُوا اسْمَ الرَّبِّ.
\par 5 لأَنَّهُ هُنَاكَ اسْتَوَتِ الْكَرَاسِيُّ لِلْقَضَاءِ كَرَاسِيُّ بَيْتِ دَاوُدَ.
\par 6 اسْأَلُوا سَلاَمَةَ أُورُشَلِيمَ. لِيَسْتَرِحْ مُحِبُّوكِ.
\par 7 لِيَكُنْ سَلاَمٌ فِي أَبْرَاجِكِ رَاحَةٌ فِي قُصُورِكِ.
\par 8 مِنْ أَجْلِ إِخْوَتِي وَأَصْحَابِي لَأَقُولَنَّ: [سَلاَمٌ بِكِ].
\par 9 مِنْ أَجْلِ بَيْتِ الرَّبِّ إِلَهِنَا أَلْتَمِسُ لَكِ خَيْراً.

\chapter{123}

\par 1 تَرْنِيمَةُ الْمَصَاعِدِ إِلَيْكَ رَفَعْتُ عَيْنَيَّ يَا سَاكِناً فِي السَّمَاوَاتِ.
\par 2 هُوَذَا كَمَا أَنَّ عُيُونَ الْعَبِيدِ نَحْوَ أَيْدِي سَادَتِهِمْ كَمَا أَنَّ عَيْنَيِ الْجَارِيَةِ نَحْوَ يَدِ سَيِّدَتِهَا هَكَذَا عُيُونُنَا نَحْوَ الرَّبِّ إِلَهِنَا حَتَّى يَتَرَأَّفَ عَلَيْنَا.
\par 3 ارْحَمْنَا يَا رَبُّ ارْحَمْنَا لأَنَّنَا كَثِيراً مَا امْتَلَأْنَا هَوَاناً.
\par 4 كَثِيراً مَا شَبِعَتْ أَنْفُسُنَا مِنْ هُزْءِ الْمُسْتَرِيحِينَ وَإِهَانَةِ الْمُسْتَكْبِرِينَ.

\chapter{124}

\par 1 تَرْنِيمَةُ الْمَصَاعِدِ. لِدَاوُ [لَوْلاَ الرَّبُّ الَّذِي كَانَ لَنَا]. لِيَقُلْ إِسْرَائِيلُ:
\par 2 [لَوْلاَ الرَّبُّ الَّذِي كَانَ لَنَا عِنْدَ مَا قَامَ النَّاسُ عَلَيْنَا
\par 3 إِذاً لاَبْتَلَعُونَا أَحْيَاءً عِنْدَ احْتِمَاءِ غَضَبِهِمْ عَلَيْنَا
\par 4 إِذاً لَجَرَفَتْنَا الْمِيَاهُ لَعَبَرَ السَّيْلُ عَلَى أَنْفُسِنَا.
\par 5 إِذاً لَعَبَرَتْ عَلَى أَنْفُسِنَا الْمِيَاهُ الطَّامِيَةُ].
\par 6 مُبَارَكٌ الرَّبُّ الَّذِي لَمْ يُسْلِمْنَا فَرِيسَةً لأَسْنَانِهِمْ.
\par 7 انْفَلَتَتْ أَنْفُسُنَا مِثْلَ الْعُصْفُورِ مِنْ فَخِّ الصَّيَّادِينَ. الْفَخُّ انْكَسَرَ وَنَحْنُ انْفَلَتْنَا.
\par 8 عَوْنُنَا بِاسْمِ الرَّبِّ الصَّانِعِ السَّمَاوَاتِ وَالأَرْضَ.

\chapter{125}

\par 1 تَرْنِيمَةُ الْمَصَاعِدِ اَلْمُتَوَكِّلُونَ عَلَى الرَّبِّ مِثْلُ جَبَلِ صِهْيَوْنَ الَّذِي لاَ يَتَزَعْزَعُ بَلْ يَسْكُنُ إِلَى الدَّهْرِ.
\par 2 أُورُشَلِيمُ الْجِبَالُ حَوْلَهَا وَالرَّبُّ حَوْلَ شَعْبِهِ مِنَ الآنَ وَإِلَى الدَّهْرِ.
\par 3 لأَنَّهُ لاَ تَسْتَقِرُّ عَصَا الأَشْرَارِ عَلَى نَصِيبِ الصِّدِّيقِينَ لِكَيْ لاَ يَمُدَّ الصِّدِّيقُونَ أَيْدِيَهُمْ إِلَى الإِثْمِ.
\par 4 أَحْسِنْ يَا رَبُّ إِلَى الصَّالِحِينَ وَإِلَى الْمُسْتَقِيمِي الْقُلُوبِ.
\par 5 أَمَّا الْعَادِلُونَ إِلَى طُرُقٍ مُعَوَّجَةٍ فَيُذْهِبُهُمُِ الرَّبُّ مَعَ فَعَلَةِ الإِثْمِ. سَلاَمٌ عَلَى إِسْرَائِيلَ.

\chapter{126}

\par 1 تَرْنِيمَةُ الْمَصَاعِدِ عِنْدَمَا رَدَّ الرَّبُّ سَبْيَ صِهْيَوْنَ صِرْنَا مِثْلَ الْحَالِمِينَ.
\par 2 حِينَئِذٍ امْتَلَأَتْ أَفْوَاهُنَا ضِحْكاً وَأَلْسِنَتُنَا تَرَنُّماً. حِينَئِذٍ قَالُوا بَيْنَ الأُمَمِ: [إِنَّ الرَّبَّ قَدْ عَظَّمَ الْعَمَلَ مَعَ هَؤُلاَءِ].
\par 3 عَظَّمَ الرَّبُّ الْعَمَلَ مَعَنَا وَصِرْنَا فَرِحِينَ.
\par 4 ارْدُدْ يَا رَبُّ سَبْيَنَا مِثْلَ السَّوَاقِي فِي الْجَنُوبِ.
\par 5 الَّذِينَ يَزْرَعُونَ بِالدُّمُوعِ يَحْصُدُونَ بِالاِبْتِهَاجِ.
\par 6 الذَّاهِبُ ذِهَاباً بِالْبُكَاءِ حَامِلاً مِبْذَرَ الزَّرْعِ مَجِيئاً يَجِيءُ بِالتَّرَنُّمِ حَامِلاً حُزَمَهُ.

\chapter{127}

\par 1 تَرْنِيمَةُ الْمَصَاعِدِ. لِسُلَيْمَانَ إِنْ لَمْ يَبْنِ الرَّبُّ الْبَيْتَ فَبَاطِلاً يَتْعَبُ الْبَنَّاؤُونَ. إِنْ لَمْ يَحْفَظِ الرَّبُّ الْمَدِينَةَ فَبَاطِلاً يَسْهَرُ الْحَارِسُ.
\par 2 بَاطِلٌ هُوَ لَكُمْ أَنْ تُبَكِّرُوا إِلَى الْقِيَامِ مُؤَخِّرِينَ الْجُلُوسَ آكِلِينَ خُبْزَ الأَتْعَابِ. لَكِنَّهُ يُعْطِي حَبِيبَهُ نَوْماً.
\par 3 هُوَذَا الْبَنُونَ مِيرَاثٌ مِنْ عِنْدِ الرَّبِّ ثَمَرَةُ الْبَطْنِ أُجْرَةٌ.
\par 4 كَسِهَامٍ بِيَدِ جَبَّارٍ هَكَذَا أَبْنَاءُ الشَّبِيبَةِ.
\par 5 طُوبَى لِلَّذِي مَلَأَ جُعْبَتَهُ مِنْهُمْ. لاَ يَخْزُونَ بَلْ يُكَلِّمُونَ الأَعْدَاءَ فِي الْبَابِ.

\chapter{128}

\par 1 تَرْنِيمَةُ الْمَصَاعِدِ طُوبَى لِكُلِّ مَنْ يَتَّقِي الرَّبَّ وَيَسْلُكُ فِي طُرُقِهِ
\par 2 لأَنَّكَ تَأْكُلُ تَعَبَ يَدَيْكَ. طُوبَاكَ وَخَيْرٌ لَكَ.
\par 3 امْرَأَتُكَ مِثْلُ كَرْمَةٍ مُثْمِرَةٍ فِي جَوَانِبِ بَيْتِكَ. بَنُوكَ مِثْلُ غُرُوسِ الزَّيْتُونِ حَوْلَ مَائِدَتِكَ.
\par 4 هَكَذَا يُبَارَكُ الرَّجُلُ الْمُتَّقِي الرَّبَّ.
\par 5 يُبَارِكُكَ الرَّبُّ مِنْ صِهْيَوْنَ وَتُبْصِرُ خَيْرَ أُورُشَلِيمَ كُلَّ أَيَّامِ حَيَاتِكَ
\par 6 وَتَرَى بَنِي بَنِيكَ. سَلاَمٌ عَلَى إِسْرَائِيلَ.

\chapter{129}

\par 1 تَرْنِيمَةُ الْمَصَاعِدِ [كَثِيراً مَا ضَايَقُونِي مُنْذُ شَبَابِي]. لِيَقُلْ إِسْرَائِيلُ:
\par 2 [كَثِيراً مَا ضَايَقُونِي مُنْذُ شَبَابِي لَكِنْ لَمْ يَقْدِرُوا عَلَيَّ.
\par 3 عَلَى ظَهْرِي حَرَثَ الْحُرَّاثُ. طَوَّلُوا أَتْلاَمَهُمْ].
\par 4 الرَّبُّ صِدِّيقٌ. قَطَعَ رُبُطَ الأَشْرَارِ.
\par 5 فَلْيَخْزَ وَلْيَرْتَدَّ إِلَى الْوَرَاءِ كُلُّ مُبْغِضِي صِهْيَوْنَ.
\par 6 لِيَكُونُوا كَعُشْبِ السُّطُوحِ الَّذِي يَيْبَسُ قَبْلَ أَنْ يُقْلَعَ
\par 7 الَّذِي لاَ يَمْلَأُ الْحَاصِدُ كَفَّهُ مِنْهُ وَلاَ الْمُحَزِّمُ حِضْنَهُ.
\par 8 وَلاَ يَقُولُ الْعَابِرُونَ: [بَرَكَةُ الرَّبِّ عَلَيْكُمْ. بَارَكْنَاكُمْ بِاسْمِ الرَّبِّ].

\chapter{130}

\par 1 تَرْنِيمَةُ الْمَصَاعِدِ مِنَ الأَعْمَاقِ صَرَخْتُ إِلَيْكَ يَا رَبُّ.
\par 2 يَا رَبُّ اسْمَعْ صَوْتِي. لِتَكُنْ أُذُنَاكَ مُصْغِيَتَيْنِ إِلَى صَوْتِ تَضَرُّعَاتِي.
\par 3 إِنْ كُنْتَ تُرَاقِبُ الآثَامَ يَا رَبُّ يَا سَيِّدُ فَمَنْ يَقِفُ؟
\par 4 لأَنَّ عِنْدَكَ الْمَغْفِرَةَ. لِكَيْ يُخَافَ مِنْكَ.
\par 5 انْتَظَرْتُكَ يَا رَبُّ. انْتَظَرَتْ نَفْسِي وَبِكَلاَمِهِ رَجَوْتُ.
\par 6 نَفْسِي تَنْتَظِرُ الرَّبَّ أَكْثَرَ مِنَ الْمُرَاقِبِينَ الصُّبْحَ. أَكْثَرَ مِنَ الْمُرَاقِبِينَ الصُّبْحَ.
\par 7 لِيَرْجُ إِسْرَائِيلُ الرَّبَّ لأَنَّ عِنْدَ الرَّبِّ الرَّحْمَةَ وَعِنْدَهُ فِدًى كَثِيرٌ
\par 8 وَهُوَ يَفْدِي إِسْرَائِيلَ مِنْ كُلِّ آثَامِهِ.

\chapter{131}

\par 1 تَرْنِيمَةُ الْمَصَاعِدِ. لِدَاوُدَ يَا رَبُّ لَمْ يَرْتَفِعْ قَلْبِي وَلَمْ تَسْتَعْلِ عَيْنَايَ وَلَمْ أَسْلُكْ فِي الْعَظَائِمِ وَلاَ فِي عَجَائِبَ فَوْقِي.
\par 2 بَلْ هَدَّأْتُ وَسَكَّتُّ نَفْسِي كَفَطِيمٍ نَحْوَ أُمِّهِ. نَفْسِي نَحْوِي كَفَطِيمٍ.
\par 3 لِيَرْجُ إِسْرَائِيلُ الرَّبَّ مِنَ الآنَ وَإِلَى الدَّهْرِ.

\chapter{132}

\par 1 تَرْنِيمَةُ الْمَصَاعِدِ اُذْكُرْ يَا رَبُّ دَاوُدَ كُلَّ ذُلِّهِ.
\par 2 كَيْفَ حَلَفَ لِلرَّبِّ نَذَرَ لِعَزِيزِ يَعْقُوبَ:
\par 3 [لاَ أَدْخُلُ خَيْمَةَ بَيْتِي. لاَ أَصْعَدُ عَلَى سَرِيرِ فِرَاشِي.
\par 4 لاَ أُعْطِي وَسَناً لِعَيْنَيَّ وَلاَ نَوْماً لأَجْفَانِي
\par 5 أَوْ أَجِدَ مَقَاماً لِلرَّبِّ مَسْكَناً لِعَزِيزِ يَعْقُوبَ].
\par 6 هُوَذَا قَدْ سَمِعْنَا بِهِ فِي أَفْرَاتَةَ. وَجَدْنَاهُ فِي حُقُولِ الْوَعْرِ.
\par 7 لِنَدْخُلْ إِلَى مَسَاكِنِهِ. لِنَسْجُدْ عِنْدَ مَوْطِئِ قَدَمَيْهِ.
\par 8 قُمْ يَا رَبُّ إِلَى رَاحَتِكَ أَنْتَ وَتَابُوتُ عِزِّكَ.
\par 9 كَهَنَتُكَ يَلْبِسُونَ الْبِرَّ وَأَتْقِيَاؤُكَ يَهْتِفُونَ.
\par 10 مِنْ أَجْلِ دَاوُدَ عَبْدِكَ لاَ تَرُدَّ وَجْهَ مَسِيحِكَ.
\par 11 أَقْسَمَ الرَّبُّ لِدَاوُدَ بِالْحَقِّ لاَ يَرْجِعُ عَنْهُ: [مِنْ ثَمَرَةِ بَطْنِكَ أَجْعَلُ عَلَى كُرْسِيِّكَ.
\par 12 إِنْ حَفِظَ بَنُوكَ عَهْدِي وَشَهَادَاتِي الَّتِي أُعَلِّمُهُمْ إِيَّاهَا فَبَنُوهُمْ أَيْضاً إِلَى الأَبَدِ يَجْلِسُونَ عَلَى كُرْسِيِّكَ].
\par 13 لأَنَّ الرَّبَّ قَدِ اخْتَارَ صِهْيَوْنَ. اشْتَهَاهَا مَسْكَناً لَهُ:
\par 14 [هَذِهِ هِيَ رَاحَتِي إِلَى الأَبَدِ. هَهُنَا أَسْكُنُ لأَنِّي اشْتَهَيْتُهَا.
\par 15 طَعَامَهَا أُبَارِكُ بَرَكَةً. مَسَاكِينَهَا أُشْبِعُ خُبْزاً.
\par 16 كَهَنَتَهَا أُلْبِسُ خَلاَصاً وَأَتْقِيَاؤُهَا يَهْتِفُونَ هُتَافاً.
\par 17 هُنَاكَ أُنْبِتُ قَرْناً لِدَاوُدَ. رَتَّبْتُ سِرَاجاً لِمَسِيحِي.
\par 18 أَعْدَاءَهُ أُلْبِسُ خِزْياً وَعَلَيْهِ يُزْهِرُ إِكْلِيلُهُ].

\chapter{133}

\par 1 تَرْنِيمَةُ الْمَصَاعِدِ. لِدَاوُدَ هُوَذَا مَا أَحْسَنَ وَمَا أَجْمَلَ أَنْ يَسْكُنَ الإِخْوَةُ مَعاً!
\par 2 مِثْلُ الدُّهْنِ الطَّيِّبِ عَلَى الرَّأْسِ النَّازِلِ عَلَى اللِّحْيَةِ لِحْيَةِ هَارُونَ النَّازِلِ إِلَى طَرَفِ ثِيَابِهِ.
\par 3 مِثْلُ نَدَى حَرْمُونَ النَّازِلِ عَلَى جَبَلِ صِهْيَوْنَ. لأَنَّهُ هُنَاكَ أَمَرَ الرَّبُّ بِالْبَرَكَةِ حَيَاةٍ إِلَى الأَبَدِ.

\chapter{134}

\par 1 تَرْنِيمَةُ الْمَصَاعِدِ هُوَذَا بَارِكُوا الرَّبَّ يَا جَمِيعَ عَبِيدِ الرَّبِّ الْوَاقِفِينَ فِي بَيْتِ الرَّبِّ بِاللَّيَالِي.
\par 2 ارْفَعُوا أَيْدِيَكُمْ نَحْوَ الْقُدْسِ وَبَارِكُوا الرَّبَّ.
\par 3 يُبَارِكُكَ الرَّبُّ مِنْ صِهْيَوْنَ الصَّانِعُ السَّمَاوَاتِ وَالأَرْضَ.

\chapter{135}

\par 1 هَلِّلُويَا. سَبِّحُوا اسْمَ الرَّبِّ. سَبِّحُوا يَا عَبِيدَ الرَّبِّ
\par 2 الْوَاقِفِينَ فِي بَيْتِ الرَّبِّ فِي دِيَارِ بَيْتِ إِلَهِنَا.
\par 3 سَبِّحُوا الرَّبَّ لأَنَّ الرَّبَّ صَالِحٌ. رَنِّمُوا لاِسْمِهِ لأَنَّ ذَاكَ حُلْوٌ.
\par 4 لأَنَّ الرَّبَّ قَدِ اخْتَارَ يَعْقُوبَ لِذَاتِهِ وَإِسْرَائِيلَ لِخَاصَّتِهِ.
\par 5 لأَنِّي أَنَا قَدْ عَرَفْتُ أَنَّ الرَّبَّ عَظِيمٌ وَرَبَّنَا فَوْقَ جَمِيعِ الآلِهَةِ.
\par 6 كُلَّ مَا شَاءَ الرَّبُّ صَنَعَ فِي السَّمَاوَاتِ وَفِي الأَرْضِ فِي الْبِحَارِ وَفِي كُلِّ اللُّجَجِ.
\par 7 الْمُصْعِدُ السَّحَابَ مِنْ أَقَاصِي الأَرْضِ. الصَّانِعُ بُرُوقاً لِلْمَطَرِ. الْمُخْرِجُ الرِّيحَ مِنْ خَزَائِنِهِ.
\par 8 الَّذِي ضَرَبَ أَبْكَارَ مِصْرَ مِنَ النَّاسِ إِلَى الْبَهَائِمِ.
\par 9 أَرْسَلَ آيَاتٍ وَعَجَائِبَ فِي وَسَطِكِ يَا مِصْرُ عَلَى فِرْعَوْنَ وَعَلَى كُلِّ عَبِيدِهِ.
\par 10 الَّذِي ضَرَبَ أُمَماً كَثِيرَةً وَقَتَلَ مُلُوكاً أَعِزَّاءَ:
\par 11 سِيحُونَ مَلِكَ الأَمُورِيِّينَ وَعُوجَ مَلِكَ بَاشَانَ وَكُلَّ مَمَالِكِ كَنْعَانَ.
\par 12 وَأَعْطَى أَرْضَهُمْ مِيرَاثاً مِيرَاثاً لإِسْرَائِيلَ شَعْبِهِ.
\par 13 يَا رَبُّ اسْمُكَ إِلَى الدَّهْرِ. يَا رَبُّ ذِكْرُكَ إِلَى دَوْرٍ فَدَوْرٍ.
\par 14 لأَنَّ الرَّبَّ يَدِينُ شَعْبَهُ وَعَلَى عَبِيدِهِ يُشْفِقُ.
\par 15 أَصْنَامُ الأُمَمِ فِضَّةٌ وَذَهَبٌ عَمَلُ أَيْدِي النَّاسِ.
\par 16 لَهَا أَفْوَاهٌ وَلاَ تَتَكَلَّمُ. لَهَا أَعْيُنٌ وَلاَ تُبْصِرُ.
\par 17 لَهَا آذَانٌ وَلاَ تَسْمَعُ. كَذَلِكَ لَيْسَ فِي أَفْوَاهِهَا نَفَسٌ!
\par 18 مِثْلَهَا يَكُونُ صَانِعُوهَا وَكُلُّ مَنْ يَتَّكِلُ عَلَيْهَا.
\par 19 يَا بَيْتَ إِسْرَائِيلَ بَارِكُوا الرَّبَّ. يَا بَيْتَ هَارُونَ بَارِكُوا الرَّبَّ.
\par 20 يَا بَيْتَ لاَوِي بَارِكُوا الرَّبَّ. يَا خَائِفِي الرَّبِّ بَارِكُوا الرَّبَّ.
\par 21 مُبَارَكٌ الرَّبُّ مِنْ صِهْيَوْنَ السَّاكِنُ فِي أُورُشَلِيمَ. هَلِّلُويَا.

\chapter{136}

\par 1 اِحْمَدُوا الرَّبَّ لأَنَّهُ صَالِحٌ لأَنَّ إِلَى الأَبَدِ رَحْمَتَهُ.
\par 2 احْمَدُوا إِلَهَ الآلِهَةِ لأَنَّ إِلَى الأَبَدِ رَحْمتَهُ.
\par 3 احْمَدُوا رَبَّ الأَرْبَابِ لأَنَّ إِلَى الأَبَدِ رَحْمَتَهُ.
\par 4 الصَّانِعَ الْعَجَائِبَ الْعِظَامَ وَحْدَهُ لأَنَّ إِلَى الأَبَدِ رَحْمَتَهُ.
\par 5 الصَّانِعَ السَّمَاوَاتِ بِفَهْمٍ لأَنَّ إِلَى الأَبَدِ رَحْمَتَهُ.
\par 6 الْبَاسِطَ الأَرْضَ عَلَى الْمِيَاهِ لأَنَّ إِلَى الأَبَدِ رَحْمَتَهُ.
\par 7 الصَّانِعَ أَنْوَاراً عَظِيمَةً لأَنَّ إِلَى الأَبَدِ رَحْمَتَهُ.
\par 8 الشَّمْسَ لِحُكْمِ النَّهَارِ لأَنَّ إِلَى الأَبَدِ رَحْمَتَهُ.
\par 9 الْقَمَرَ وَالْكَوَاكِبَ لِحُكْمِ اللَّيْلِ لأَنَّ إِلَى الأَبَدِ رَحْمَتَهُ.
\par 10 الَّذِي ضَرَبَ مِصْرَ مَعَ أَبْكَارِهَا لأَنَّ إِلَى الأَبَدِ رَحْمَتَهُ.
\par 11 وَأَخْرَجَ إِسْرَائِيلَ مِنْ وَسَطِهِمْ لأَنَّ إِلَى الأَبَدِ رَحْمَتَهُ.
\par 12 بِيَدٍ شَدِيدَةٍ وَذِرَاعٍ مَمْدُودَةٍ لأَنَّ إِلَى الأَبَدِ رَحْمَتَهُ.
\par 13 الَّذِي شَقَّ بَحْرَ سُوفٍ إِلَى شُقَقٍ لأَنَّ إِلَى الأَبَدِ رَحْمَتَهُ.
\par 14 وَعَبَّرَ إِسْرَائِيلَ فِي وَسَطِهِ لأَنَّ إِلَى الأَبَدِ رَحْمَتَهُ.
\par 15 وَدَفَعَ فِرْعَوْنَ وَقُوَّتَهُ فِي بَحْرِ سُوفٍ لأَنَّ إِلَى الأَبَدِ رَحْمَتَهُ.
\par 16 الَّذِي سَارَ بِشَعْبِهِ فِي الْبَرِّيَّةِ لأَنَّ إِلَى الأَبَدِ رَحْمَتَهُ.
\par 17 الَّذِي ضَرَبَ مُلُوكاً عُظَمَاءَ لأَنَّ إِلَى الأَبَدِ رَحْمَتَهُ.
\par 18 وَقَتَلَ مُلُوكاً أَعِزَّاءَ لأَنَّ إِلَى الأَبَدِ رَحْمَتَهُ:
\par 19 سِيحُونَ مَلِكَ الأَمُورِيِّينَ لأَنَّ إِلَى الأَبَدِ رَحْمَتَهُ
\par 20 وَعُوجَ مَلِكَ بَاشَانَ لأَنَّ إِلَى الأَبَدِ رَحْمَتَهُ.
\par 21 وَأَعْطَى أَرْضَهُمْ مِيرَاثاً لأَنَّ إِلَى الأَبَدِ رَحْمَتَهُ
\par 22 مِيرَاثاً لإِسْرَائِيلَ عَبْدِهِ لأَنَّ إِلَى الأَبَدِ رَحْمَتَهُ.
\par 23 الَّذِي فِي مَذَلَّتِنَا ذَكَرَنَا لأَنَّ إِلَى الأَبَدِ رَحْمَتَهُ.
\par 24 وَنَجَّانَا مِنْ أَعْدَائِنَا لأَنَّ إِلَى الأَبَدِ رَحْمَتَهُ.
\par 25 الَّذِي يُعْطِي خُبْزاً لِكُلِّ بَشَرٍ لأَنَّ إِلَى الأَبَدِ رَحْمَتَهُ.
\par 26 احْمَدُوا إِلَهَ السَّمَاوَاتِ لأَنَّ إِلَى الأَبَدِ رَحْمَتَهُ.

\chapter{137}

\par 1 عَلَى أَنْهَارِ بَابِلَ هُنَاكَ جَلَسْنَا. بَكَيْنَا أَيْضاً عِنْدَ مَا تَذَكَّرْنَا صِهْيَوْنَ.
\par 2 عَلَى الصَّفْصَافِ فِي وَسَطِهَا عَلَّقْنَا أَعْوَادَنَا.
\par 3 لأَنَّهُ هُنَاكَ سَأَلَنَا الَّذِينَ سَبُونَا كَلاَمَ تَرْنِيمَةٍ وَمُعَذِّبُونَا سَأَلُونَا فَرَحاً: [رَنِّمُوا لَنَا مِنْ تَرْنِيمَاتِ صِهْيَوْنَ].
\par 4 كَيْفَ نُرَنِّمُ تَرْنِيمَةَ الرَّبِّ فِي أَرْضٍ غَرِيبَةٍ؟
\par 5 إِنْ نَسِيتُكِ يَا أُورُشَلِيمُ تَنْسَى يَمِينِي - لِيَلْتَصِقْ لِسَانِي بِحَنَكِي إِنْ لَمْ أَذْكُرْكِ!
\par 6 إِنْ لَمْ أُفَضِّلْ أُورُشَلِيمَ عَلَى أَعْظَمِ فَرَحِي!
\par 7 اُذْكُرْ يَا رَبُّ لِبَنِي أَدُومَ يَوْمَ أُورُشَلِيمَ الْقَائِلِينَ: [هُدُّوا هُدُّوا حَتَّى إِلَى أَسَاسِهَا].
\par 8 يَا بِنْتَ بَابِلَ الْمُخْرَبَةَ طُوبَى لِمَنْ يُجَازِيكِ جَزَاءَكِ الَّذِي جَازَيْتِنَا!
\par 9 طُوبَى لِمَنْ يُمْسِكُ أَطْفَالَكِ وَيَضْرِبُ بِهِمُ الصَّخْرَةَ!

\chapter{138}

\par 1 لِدَاوُدَ أَحْمَدُكَ مِنْ كُلِّ قَلْبِي. قُدَّامَ الآلِهَةِ أُرَنِّمُ لَكَ.
\par 2 أَسْجُدُ فِي هَيْكَلِ قُدْسِكَ وَأَحْمَدُ اسْمَكَ عَلَى رَحْمَتِكَ وَحَقِّكَ لأَنَّكَ قَدْ عَظَّمْتَ كَلِمَتَكَ عَلَى كُلِّ اسْمِكَ.
\par 3 فِي يَوْمَ دَعَوْتُكَ أَجَبْتَنِي. شَجَّعْتَنِي قُوَّةً فِي نَفْسِي.
\par 4 يَحْمَدُكَ يَا رَبُّ كُلُّ مُلُوكِ الأَرْضِ إِذَا سَمِعُوا كَلِمَاتِ فَمِكَ.
\par 5 وَيُرَنِّمُونَ فِي طُرُقِ الرَّبِّ لأَنَّ مَجْدَ الرَّبِّ عَظِيمٌ.
\par 6 لأَنَّ الرَّبَّ عَالٍ وَيَرَى الْمُتَوَاضِعَ. أَمَّا الْمُتَكَبِّرُ فَيَعْرِفُهُ مِنْ بَعِيدٍ.
\par 7 إِنْ سَلَكْتُ فِي وَسَطِ الضِّيقِ تُحْيِنِي. عَلَى غَضَبِ أَعْدَائِي تَمُدُّ يَدَكَ وَتُخَلِّصُنِي يَمِينُكَ.
\par 8 الرَّبُّ يُحَامِي عَنِّي. يَا رَبُّ رَحْمَتُكَ إِلَى الأَبَدِ. عَنْ أَعْمَالِ يَدَيْكَ لاَ تَتَخَلَّ.

\chapter{139}

\par 1 لإِمَامِ الْمُغَنِّينَ. لِدَاوُدَ. مَزْمُورٌ يَا رَبُّ قَدِ اخْتَبَرْتَنِي وَعَرَفْتَنِي.
\par 2 أَنْتَ عَرَفْتَ جُلُوسِي وَقِيَامِي. فَهِمْتَ فِكْرِي مِنْ بَعِيدٍ.
\par 3 مَسْلَكِي وَمَرْبَضِي ذَرَّيْتَ وَكُلَّ طُرُقِي عَرَفْتَ.
\par 4 لأَنَّهُ لَيْسَ كَلِمَةٌ فِي لِسَانِي إِلاَّ وَأَنْتَ يَا رَبُّ عَرَفْتَهَا كُلَّهَا.
\par 5 مِنْ خَلْفٍ وَمِنْ قُدَّامٍ حَاصَرْتَنِي وَجَعَلْتَ عَلَيَّ يَدَكَ.
\par 6 عَجِيبَةٌ هَذِهِ الْمَعْرِفَةُ فَوْقِي. ارْتَفَعَتْ لاَ أَسْتَطِيعُهَا.
\par 7 أَيْنَ أَذْهَبُ مِنْ رُوحِكَ وَمِنْ وَجْهِكَ أَيْنَ أَهْرُبُ؟
\par 8 إِنْ صَعِدْتُ إِلَى السَّمَاوَاتِ فَأَنْتَ هُنَاكَ وَإِنْ فَرَشْتُ فِي الْهَاوِيَةِ فَهَا أَنْتَ.
\par 9 إِنْ أَخَذْتُ جَنَاحَيِ الصُّبْحِ وَسَكَنْتُ فِي أَقَاصِي الْبَحْرِ
\par 10 فَهُنَاكَ أَيْضاً تَهْدِينِي يَدُكَ وَتُمْسِكُنِي يَمِينُكَ.
\par 11 فَقُلْتُ: [إِنَّمَا الظُّلْمَةُ تَغْشَانِي]. فَاللَّيْلُ يُضِيءُ حَوْلِي!
\par 12 الظُّلْمَةُ أَيْضاً لاَ تُظْلِمُ لَدَيْكَ وَاللَّيْلُ مِثْلَ النَّهَارِ يُضِيءُ. كَالظُّلْمَةِ هَكَذَا النُّورُ.
\par 13 لأَنَّكَ أَنْتَ اقْتَنَيْتَ كُلْيَتَيَّ. نَسَجْتَنِي فِي بَطْنِ أُمِّي.
\par 14 أَحْمَدُكَ مِنْ أَجْلِ أَنِّي قَدِ امْتَزْتُ عَجَباً. عَجِيبَةٌ هِيَ أَعْمَالُكَ وَنَفْسِي تَعْرِفُ ذَلِكَ يَقِيناً.
\par 15 لَمْ تَخْتَفِ عَنْكَ عِظَامِي حِينَمَا صُنِعْتُ فِي الْخَفَاءِ وَرُقِمْتُ فِي أَعْمَاقِ الأَرْضِ.
\par 16 رَأَتْ عَيْنَاكَ أَعْضَائِي وَفِي سِفْرِكَ كُلُّهَا كُتِبَتْ يَوْمَ تَصَوَّرَتْ إِذْ لَمْ يَكُنْ وَاحِدٌ مِنْهَا.
\par 17 مَا أَكْرَمَ أَفْكَارَكَ يَا اللهُ عِنْدِي! مَا أَكْثَرَ جُمْلَتَهَا!
\par 18 إِنْ أُحْصِهَا فَهِيَ أَكْثَرُ مِنَ الرَّمْلِ. اسْتَيْقَظْتُ وَأَنَا بَعْدُ مَعَكَ.
\par 19 لَيْتَكَ تَقْتُلُ الأَشْرَارَ يَا اللهُ. فَيَا رِجَالَ الدِّمَاءِ ابْعُدُوا عَنِّي.
\par 20 الَّذِينَ يُكَلِّمُونَكَ بِالْمَكْرِ نَاطِقِينَ بِالْكَذِبِ هُمْ أَعْدَاؤُكَ.
\par 21 أَلاَ أُبْغِضُ مُبْغِضِيكَ يَا رَبُّ وَأَمْقُتُ مُقَاوِمِيكَ.
\par 22 بُغْضاً تَامّاً أَبْغَضْتُهُمْ. صَارُوا لِي أَعْدَاءً.
\par 23 اخْتَبِرْنِي يَا اللهُ وَاعْرِفْ قَلْبِي. امْتَحِنِّي وَاعْرِفْ أَفْكَارِي.
\par 24 وَانْظُرْ إِنْ كَانَ فِيَّ طَرِيقٌ بَاطِلٌ وَاهْدِنِي طَرِيقاً أَبَدِيّاً.

\chapter{140}

\par 1 لإِمَامِ الْمُغَنِّينَ. مَزْمُورٌ لِدَاوُدَ أَنْقِذْنِي يَا رَبُّ مِنْ أَهْلِ الشَّرِّ. مِنْ رَجُلِ الظُّلْمِ احْفَظْنِي.
\par 2 الَّذِينَ يَتَفَكَّرُونَ بِشُرُورٍ فِي قُلُوبِهِمْ. الْيَوْمَ كُلَّهُ يَجْتَمِعُونَ لِلْقِتَالِ.
\par 3 سَنُّوا أَلْسِنَتَهُمْ كَحَيَّةٍ. حُمَةُ الأُفْعُوانِ تَحْتَ شِفَاهِهِمْ. سِلاَهْ.
\par 4 احْفَظْنِي يَا رَبُّ مِنْ يَدَيِ الشِّرِّيرِ. مِنْ رَجُلِ الظُّلْمِ أَنْقِذْنِي. الَّذِينَ تَفَكَّرُوا فِي تَعْثِيرِ خُطُواتِي.
\par 5 أَخْفَى لِي الْمُسْتَكْبِرُونَ فَخّاً وَحِبَالاً. مَدُّوا شَبَكَةً بِجَانِبِ الطَّرِيقِ. وَضَعُوا لِي أَشْرَاكاً. سِلاَهْ.
\par 6 قُلْتُ لِلرَّبِّ: [أَنْتَ إِلَهِي. أَصْغِ يَا رَبُّ إِلَى صَوْتِ تَضَرُّعَاتِي.
\par 7 يَا رَبُّ السَّيِّدُ قُوَّةَ خَلاَصِي ظَلَّلْتَ رَأْسِي فِي يَوْمِ الْقِتَالِ.
\par 8 لاَ تُعْطِ يَا رَبُّ شَهَوَاتِ الشِّرِّيرِ. لاَ تُنَجِّحْ مَقَاصِدَهُ. يَتَرَفَّعُونَ. سِلاَهْ.
\par 9 أَمَّا رُؤُوسُ الْمُحِيطِينَ بِي فَشَقَاءُ شِفَاهِهِمْ يُغَطِّيهِمْ.
\par 10 لِيَسْقُطْ عَلَيْهِمْ جَمْرٌ. لِيُسْقَطُوا فِي النَّارِ وَفِي غَمَرَاتٍ فَلاَ يَقُومُوا.
\par 11 رَجُلُ لِسَانٍ لاَ يَثْبُتُ فِي الأَرْضِ. رَجُلُ الظُّلْمِ يَصِيدُهُ الشَّرُّ إِلَى هَلاَكِهِ].
\par 12 قَدْ عَلِمْتُ أَنَّ الرَّبَّ يُجْرِي حُكْماً لِلْمَسَاكِينِ وَحَقّاً لِلْبَائِسِينَ.
\par 13 إِنَّمَا الصِّدِّيقُونَ يَحْمَدُونَ اسْمَكَ. الْمُسْتَقِيمُونَ يَجْلِسُونَ فِي حَضْرَتِكَ.

\chapter{141}

\par 1 مَزْمُورٌ لِدَاوُدَ يَا رَبُّ إِلَيْكَ صَرَخْتُ. أَسْرِعْ إِلَيَّ. أَصْغِ إِلَى صَوْتِي عِنْدَ مَا أَصْرُخُ إِلَيْكَ.
\par 2 لِتَسْتَقِمْ صَلاَتِي كَالْبَخُورِ قُدَّامَكَ. لِيَكُنْ رَفْعُ يَدَيَّ كَذَبِيحَةٍ مَسَائِيَّةٍ.
\par 3 اجْعَلْ يَا رَبُّ حَارِساً لِفَمِي. احْفَظْ بَابَ شَفَتَيَّ.
\par 4 لاَ تُمِلْ قَلْبِي إِلَى أَمْرٍ رَدِيءٍ لأَتَعَلَّلَ بِعِلَلِ الشَّرِّ مَعَ أُنَاسٍ فَاعِلِي إِثْمٍ وَلاَ آكُلْ مِنْ نَفَائِسِهِمْ.
\par 5 لِيَضْرِبْنِي الصِّدِّيقُ فَرَحْمَةٌ وَلْيُوَبِّخْنِي فَزَيْتٌ لِلرَّأْسِ. لاَ يَأْبَى رَأْسِي. لأَنَّ صَلاَتِي بَعْدُ فِي مَصَائِبِهِمْ.
\par 6 قَدِ انْطَرَحَ قُضَاتُهُمْ مِنْ عَلَى الصَّخْرَةِ وَسَمِعُوا كَلِمَاتِي لأَنَّهَا لَذِيذَةٌ.
\par 7 كَمَنْ يَفْلَحُ وَيَشُقُّ الأَرْضَ تَبَدَّدَتْ عِظَامُنَا عِنْدَ فَمِ الْهَاوِيَةِ.
\par 8 لأَنَّهُ إِلَيْكَ يَا سَيِّدُ يَا رَبُّ عَيْنَايَ. بِكَ احْتَمَيْتُ. لاَ تُفْرِغْ نَفْسِي.
\par 9 احْفَظْنِي مِنَ الْفَخِّ الَّذِي قَدْ نَصَبُوهُ لِي وَمِنْ أَشْرَاكِ فَاعِلِي الإِثْمِ.
\par 10 لِيَسْقُطِ الأَشْرَارُ فِي شِبَاكِهِمْ حَتَّى أَنْجُوَ أَنَا بِالْكُلِّيَّةِ.

\chapter{142}

\par 1 قَصِيدَةٌ لِدَاوُدَ لَمَّا كَانَ فِي الْمَغَارَةِ. صَلاَةٌ بِصَوْتِي إِلَى الرَّبِّ أَصْرُخُ. بِصَوْتِي إِلَى الرَّبِّ أَتَضَرَّعُ.
\par 2 أَسْكُبُ أَمَامَهُ شَكْوَايَ. بِضِيقِي قُدَّامَهُ أُخْبِرُ.
\par 3 عِنْدَ مَا أَعْيَتْ رُوحِي فِيَّ وَأَنْتَ عَرَفْتَ مَسْلَكِي - فِي الطَّرِيقِ الَّتِي أَسْلُكُ أَخْفُوا لِي فَخّاً.
\par 4 اُنْظُرْ إِلَى الْيَمِينِ وَأَبْصِرْ فَلَيْسَ لِي عَارِفٌ. بَادَ عَنِّي الْمَنَاصُ. لَيْسَ مَنْ يَسْأَلُ عَنْ نَفْسِي.
\par 5 صَرَخْتُ إِلَيْكَ يَا رَبُّ. قُلْتُ: [أَنْتَ مَلْجَإِي نَصِيبِي فِي أَرْضِ الأَحْيَاءِ.
\par 6 أَصْغِ إِلَى صُرَاخِي لأَنِّي قَدْ تَذَلَّلْتُ جِدّاً. نَجِّنِي مِنْ مُضْطَهِدِيَّ لأَنَّهُمْ أَشَدُّ مِنِّي.
\par 7 أَخْرِجْ مِنَ الْحَبْسِ نَفْسِي لِتَحْمِيدِ اسْمِكَ. الصِّدِّيقُونَ يَكْتَنِفُونَنِي لأَنَّكَ تُحْسِنُ إِلَيَّ].

\chapter{143}

\par 1 مَزْمُورٌ لِدَاوُدَ يَا رَبُّ اسْمَعْ صَلاَتِي وَأَصْغِ إِلَى تَضَرُّعَاتِي. بِأَمَانَتِكَ اسْتَجِبْ لِي بِعَدْلِكَ.
\par 2 وَلاَ تَدْخُلْ فِي الْمُحَاكَمَةِ مَعَ عَبْدِكَ فَإِنَّهُ لَنْ يَتَبَرَّرَ قُدَّامَكَ حَيٌّ.
\par 3 لأَنَّ الْعَدُوَّ قَدِ اضْطَهَدَ نَفْسِي. سَحَقَ إِلَى الأَرْضِ حَيَاتِي. أَجْلَسَنِي فِي الظُّلُمَاتِ مِثْلَ الْمَوْتَى مُنْذُ الدَّهْرِ.
\par 4 أَعْيَتْ فِيَّ رُوحِي. تَحَيَّرَ فِي دَاخِلِي قَلْبِي.
\par 5 تَذَكَّرْتُ أَيَّامَ الْقِدَمِ. لَهَجْتُ بِكُلِّ أَعْمَالِكَ. بِصَنَائِعِ يَدَيْكَ أَتَأَمَّلُ.
\par 6 بَسَطْتُ إِلَيْكَ يَدَيَّ. نَفْسِي نَحْوَكَ كَأَرْضٍ يَابِسَةٍ. سِلاَهْ.
\par 7 أَسْرِعْ أَجِبْنِي يَا رَبُّ. فَنِيَتْ رُوحِي. لاَ تَحْجُبْ وَجْهَكَ عَنِّي فَأُشْبِهَ الْهَابِطِينَ فِي الْجُبِّ.
\par 8 أَسْمِعْنِي رَحْمَتَكَ فِي الْغَدَاةِ لأَنِّي عَلَيْكَ تَوَكَّلْتُ. عَرِّفْنِي الطَّرِيقَ الَّتِي أَسْلُكُ فِيهَا لأَنِّي إِلَيْكَ رَفَعْتُ نَفْسِي.
\par 9 أَنْقِذْنِي مِنْ أَعْدَائِي يَا رَبُّ. إِلَيْكَ الْتَجَأْتُ.
\par 10 عَلِّمْنِي أَنْ أَعْمَلَ رِضَاكَ لأَنَّكَ أَنْتَ إِلَهِي. رُوحُكَ الصَّالِحُ يَهْدِينِي فِي أَرْضٍ مُسْتَوِيَةٍ.
\par 11 مِنْ أَجْلِ اسْمِكَ يَا رَبُّ تُحْيِينِي. بِعَدْلِكَ تُخْرِجُ مِنَ الضِّيقِ نَفْسِي
\par 12 وَبِرَحْمَتِكَ تَسْتَأْصِلُ أَعْدَائِي وَتُبِيدُ كُلَّ مُضَايِقِي نَفْسِي لأَنِّي أَنَا عَبْدُكَ.

\chapter{144}

\par 1 لِدَاوُدَ مُبَارَكٌ الرَّبُّ صَخْرَتِي الَّذِي يُعَلِّمُ يَدَيَّ الْقِتَالَ وَأَصَابِعِي الْحَرْبَ.
\par 2 رَحْمَتِي وَمَلْجَإِي صَرْحِي وَمُنْقِذِي مِجَنِّي وَالَّذِي عَلَيْهِ تَوَكَّلْتُ الْمُخْضِعُ شَعْبِي تَحْتِي.
\par 3 يَا رَبُّ أَيُّ شَيْءٍ هُوَ الإِنْسَانُ حَتَّى تَعْرِفَهُ أَوِ ابْنُ الإِنْسَانِ حَتَّى تَفْتَكِرَ بِهِ؟
\par 4 الإِنْسَانُ أَشْبَهَ نَفْخَةً. أَيَّامُهُ مِثْلُ ظِلٍّ عَابِرٍ.
\par 5 يَا رَبُّ طَأْطِئْ سَمَاوَاتِكَ وَانْزِلِ. الْمِسِ الْجِبَالَ فَتُدَخِّنَ.
\par 6 أَبْرِقْ بُرُوقاً وَبَدِّدْهُمْ. أَرْسِلْ سِهَامَكَ وَأَزْعِجْهُمْ.
\par 7 أَرْسِلْ يَدَكَ مِنَ الْعَلاَءِ. أَنْقِذْنِي وَنَجِّنِي مِنَ الْمِيَاهِ الْكَثِيرَةِ مِنْ أَيْدِي الْغُرَبَاءِ
\par 8 الَّذِينَ تَكَلَّمَتْ أَفْوَاهُهُمْ بِالْبَاطِلِ وَيَمِينُهُمْ يَمِينُ كَذِبٍ.
\par 9 يَا اللهُ أُرَنِّمُ لَكَ تَرْنِيمَةً جَدِيدَةً. بِرَبَابٍ ذَاتِ عَشَرَةِ أَوْتَارٍ أُرَنِّمُ لَكَ.
\par 10 الْمُعْطِي خَلاَصاً لِلْمُلُوكِ. الْمُنْقِذُ دَاوُدَ عَبْدَهُ مِنَ السَّيْفِ السُّوءِ.
\par 11 أَنْقِذْنِي وَنَجِّنِي مِنْ أَيْدِي الْغُرَبَاءِ الَّذِينَ تَكَلَّمَتْ أَفْوَاهُهُمْ بِالْبَاطِلِ وَيَمِينُهُمْ يَمِينُ كَذِبٍ.
\par 12 لِكَيْ يَكُونَ بَنُونَا مِثْلَ الْغُرُوسِ النَّامِيَةِ فِي شَبِيبَتِهَا. بَنَاتُنَا كَأَعْمِدَةِ الزَّوَايَا مَنْحُوتَاتٍ حَسَبَ بِنَاءِ هَيْكَلٍ.
\par 13 أَهْرَاؤُنَا مَلآنَةً تَفِيضُ مِنْ صِنْفٍ فَصِنْفٍ. أَغْنَامُنَا تُنْتِجُ أُلُوفاً وَرَبَوَاتٍ فِي شَوَارِعِنَا.
\par 14 بَقَرُنَا مُحَمَّلَةً. لاَ اقْتِحَامَ وَلاَ هُجُومَ وَلاَ شَكْوَى فِي شَوَارِعِنَا.
\par 15 طُوبَى لِلشَّعْبِ الَّذِي لَهُ كَهَذَا. طُوبَى لِلشَّعْبِ الَّذِي الرَّبُّ إِلَهُهُ.

\chapter{145}

\par 1 تَسْبِيحَةٌ لِدَاوُدَ أَرْفَعُكَ يَا إِلَهِي الْمَلِكَ وَأُبَارِكُ اسْمَكَ إِلَى الدَّهْرِ وَالأَبَدِ.
\par 2 فِي كُلِّ يَوْمٍ أُبَارِكُكَ وَأُسَبِّحُ اسْمَكَ إِلَى الدَّهْرِ وَالأَبَدِ.
\par 3 عَظِيمٌ هُوَ الرَّبُّ وَحَمِيدٌ جِدّاً وَلَيْسَ لِعَظَمَتِهِ اسْتِقْصَاءٌ.
\par 4 دَوْرٌ إِلَى دَوْرٍ يُسَبِّحُ أَعْمَالَكَ وَبِجَبَرُوتِكَ يُخْبِرُونَ.
\par 5 بِجَلاَلِ مَجْدِ حَمْدِكَ وَأُمُورِ عَجَائِبِكَ أَلْهَجُ.
\par 6 بِقُوَّةِ مَخَاوِفِكَ يَنْطِقُونَ وَبِعَظَمَتِكَ أُحَدِّثُ.
\par 7 ذِكْرَ كَثْرَةِ صَلاَحِكَ يُبْدُونَ وَبِعَدْلِكَ يُرَنِّمُونَ.
\par 8 اَلرَّبُّ حَنَّانٌ وَرَحِيمٌ طَوِيلُ الرُّوحِ وَكَثِيرُ الرَّحْمَةِ.
\par 9 الرَّبُّ صَالِحٌ لِلْكُلِّ وَمَرَاحِمُهُ عَلَى كُلِّ أَعْمَالِهِ.
\par 10 يَحْمَدُكَ يَا رَبُّ كُلُّ أَعْمَالِكَ وَيُبَارِكُكَ أَتْقِيَاؤُكَ.
\par 11 بِمَجْدِ مُلْكِكَ يَنْطِقُونَ وَبِجَبَرُوتِكَ يَتَكَلَّمُونَ
\par 12 لِيُعَرِّفُوا بَنِي آدَمَ قُدْرَتَكَ وَمَجْدَ جَلاَلِ مُلْكِكَ.
\par 13 مُلْكُكَ مُلْكُ كُلِّ الدُّهُورِ وَسُلْطَانُكَ فِي كُلِّ دَوْرٍ فَدَوْرٍ.
\par 14 اَلرَّبُّ عَاضِدٌ كُلَّ السَّاقِطِينَ وَمُقَوِّمٌ كُلَّ الْمُنْحَنِينَ.
\par 15 أَعْيُنُ الْكُلِّ إِيَّاكَ تَتَرَجَّى وَأَنْتَ تُعْطِيهِمْ طَعَامَهُمْ فِي حِينِهِ.
\par 16 تَفْتَحُ يَدَكَ فَتُشْبِعُ كُلَّ حَيٍّ رِضىً.
\par 17 الرَّبُّ بَارٌّ فِي كُلِّ طُرُقِهِ وَرَحِيمٌ فِي كُلِّ أَعْمَالِهِ.
\par 18 الرَّبُّ قَرِيبٌ لِكُلِّ الَّذِينَ يَدْعُونَهُ الَّذِينَ يَدْعُونَهُ بِالْحَقِّ.
\par 19 يَعْمَلُ رِضَى خَائِفِيهِ وَيَسْمَعُ تَضَرُّعَهُمْ فَيُخَلِّصُهُمْ.
\par 20 يَحْفَظُ الرَّبُّ كُلَّ مُحِبِّيهِ وَيُهْلِكُ جَمِيعَ الأَشْرَارِ.
\par 21 بِتَسْبِيحِ الرَّبِّ يَنْطِقُ فَمِي وَلِْيُبَارِكْ كُلُّ بَشَرٍ اسْمَهُ الْقُدُّوسَ إِلَى الدَّهْرِ وَالأَبَدِ.

\chapter{146}

\par 1 هَلِّلُويَا. سَبِّحِي يَا نَفْسِي الرَّبَّ.
\par 2 أُسَبِّحُ الرَّبَّ فِي حَيَاتِي. وَأُرَنِّمُ لإِلَهِي مَا دُمْتُ مَوْجُوداً.
\par 3 لاَ تَتَّكِلُوا عَلَى الرُّؤَسَاءِ وَلاَ عَلَى ابْنِ آدَمَ حَيْثُ لاَ خَلاَصَ عِنْدَهُ.
\par 4 تَخْرُجُ رُوحُهُ فَيَعُودُ إِلَى تُرَابِهِ. فِي ذَلِكَ الْيَوْمِ نَفْسِهِ تَهْلِكُ أَفْكَارُهُ.
\par 5 طُوبَى لِمَنْ إِلَهُ يَعْقُوبَ مُعِينُهُ وَرَجَاؤُهُ عَلَى الرَّبِّ إِلَهِهِ
\par 6 الصَّانِعِ السَّمَاوَاتِ وَالأَرْضَ الْبَحْرَ وَكُلَّ مَا فِيهَا. الْحَافِظِ الأَمَانَةَ إِلَى الأَبَدِ.
\par 7 الْمُجْرِي حُكْماً لِلْمَظْلُومِينَ الْمُعْطِي خُبْزاً لِلْجِيَاعِ. الرَّبُّ يُطْلِقُ الأَسْرَى.
\par 8 الرَّبُّ يَفْتَحُ أَعْيُنَ الْعُمْيِ. الرَّبُّ يُقَوِّمُ الْمُنْحَنِينَ. الرَّبُّ يُحِبُّ الصِّدِّيقِينَ.
\par 9 الرَّبُّ يَحْفَظُ الْغُرَبَاءَ. يَعْضُدُ الْيَتِيمَ وَالأَرْمَلَةَ. أَمَّا طَرِيقُ الأَشْرَارِ فَيُعَوِّجُهُ.
\par 10 يَمْلِكُ الرَّبُّ إِلَى الأَبَدِ إِلَهُكِ يَا صِهْيَوْنُ إِلَى دَوْرٍ فَدَوْرٍ. هَلِّلُويَا.

\chapter{147}

\par 1 سَبِّحُوا الرَّبَّ لأَنَّ التَّرَنُّمَ لإِلَهِنَا صَالِحٌ. لأَنَّهُ مُلِذٌّ. التَّسْبِيحُ لاَئِقٌ.
\par 2 الرَّبُّ يَبْنِي أُورُشَلِيمَ. يَجْمَعُ مَنْفِيِّي إِسْرَائِيلَ.
\par 3 يَشْفِي الْمُنْكَسِرِي الْقُلُوبِ وَيَجْبُرُ كَسْرَهُمْ.
\par 4 يُحْصِي عَدَدَ الْكَوَاكِبِ. يَدْعُو كُلَّهَا بِأَسْمَاءٍ.
\par 5 عَظِيمٌ هُوَ رَبُّنَا وَعَظِيمُ الْقُوَّةِ. لِفَهْمِهِ لاَ إِحْصَاءَ.
\par 6 الرَّبُّ يَرْفَعُ الْوُدَعَاءَ وَيَضَعُ الأَشْرَارَ إِلَى الأَرْضِ.
\par 7 أَجِيبُوا الرَّبَّ بِحَمْدٍ. رَنِّمُوا لإِلَهِنَا بِعُودٍ.
\par 8 الْكَاسِي السَّمَاوَاتِ سَحَاباً الْمُهَيِّئِ لِلأَرْضِ مَطَراً الْمُنْبِتِ الْجِبَالَ عُشْباً
\par 9 الْمُعْطِي لِلْبَهَائِمِ طَعَامَهَا لِفِرَاخِ الْغِرْبَانِ الَّتِي تَصْرُخُ.
\par 10 لاَ يُسَرُّ بِقُوَّةِ الْخَيْلِ. لاَ يَرْضَى بِسَاقَيِ الرَّجُلِ.
\par 11 يَرْضَى الرَّبُّ بِأَتْقِيَائِهِ بِالرَّاجِينَ رَحْمَتَهُ.
\par 12 سَبِّحِي يَا أُورُشَلِيمُ الرَّبَّ. سَبِّحِي إِلَهَكِ يَا صِهْيَوْنُ.
\par 13 لأَنَّهُ قَدْ شَدَّدَ عَوَارِضَ أَبْوَابِكِ. بَارَكَ أَبْنَاءَكِ دَاخِلَكِ.
\par 14 الَّذِي يَجْعَلُ تُخُومَكِ سَلاَماً وَيُشْبِعُكِ مِنْ شَحْمِ الْحِنْطَةِ.
\par 15 يُرْسِلُ كَلِمَتَهُ فِي الأَرْضِ. سَرِيعاً جِدّاً يُجْرِي قَوْلَهُ.
\par 16 الَّذِي يُعْطِي الثَّلْجَ كَالصُّوفِ وَيُذَرِّي الصَّقِيعَ كَالرَّمَادِ.
\par 17 يُلْقِي جَمْدَهُ كَفُتَاتٍ. قُدَّامَ بَرْدِهِ مَنْ يَقِفُ؟
\par 18 يُرْسِلُ كَلِمَتَهُ فَيُذِيبُهَا. يَهُبُّ بِرِيحِهِ فَتَسِيلُ الْمِيَاهُ.
\par 19 يُخْبِرُ يَعْقُوبَ بِكَلِمَتِهِ وَإِسْرَائِيلَ بِفَرَائِضِهِ وَأَحْكَامِهِ.
\par 20 لَمْ يَصْنَعْ هَكَذَا بِإِحْدَى الأُمَمِ وَأَحْكَامُهُ لَمْ يَعْرِفُوهَا. هَلِّلُويَا.

\chapter{148}

\par 1 هَلِّلُويَا. سَبِّحُوا الرَّبَّ مِنَ السَّمَاوَاتِ. سَبِّحُوهُ فِي الأَعَالِي.
\par 2 سَبِّحُوهُ يَا جَمِيعَ مَلاَئِكَتِهِ. سَبِّحُوهُ يَا كُلَّ جُنُودِهِ.
\par 3 سَبِّحِيهِ يَا أَيَّتُهَا الشَّمْسُ وَالْقَمَرُ. سَبِّحِيهِ يَا جَمِيعَ كَوَاكِبِ النُّورِ.
\par 4 سَبِّحِيهِ يَا سَمَاءَ السَّمَاوَاتِ وَيَا أَيَّتُهَا الْمِيَاهُ الَّتِي فَوْقَ السَّمَاوَاتِ.
\par 5 لِتُسَبِّحِ اسْمَ الرَّبِّ لأَنَّهُ أَمَرَ فَخُلِقَتْ
\par 6 وَثَبَّتَهَا إِلَى الدَّهْرِ وَالأَبَدِ وَضَعَ لَهَا حَدّاً فَلَنْ تَتَعَدَّاهُ.
\par 7 سَبِّحِي الرَّبَّ مِنَ الأَرْضِ يَا أَيَّتُهَا التَّنَانِينُ وَكُلَّ اللُّجَجِ.
\par 8 النَّارُ وَالْبَرَدُ الثَّلْجُ وَالضَّبَابُ الرِّيحُ الْعَاصِفَةُ الصَّانِعَةُ كَلِمَتَهُ
\par 9 الْجِبَالُ وَكُلُّ الآكَامِ الشَّجَرُ الْمُثْمِرُ وَكُلُّ الأَرْزِ
\par 10 الْوُحُوشُ وَكُلُّ الْبَهَائِمِ الدَّبَّابَاتُ وَالطُّيُورُ ذَوَاتُ الأَجْنِحَةِ
\par 11 مُلُوكُ الأَرْضِ وَكُلُّ الشُّعُوبِ الرُّؤَسَاءُ وَكُلُّ قُضَاةِ الأَرْضِ
\par 12 الأَحْدَاثُ وَالْعَذَارَى أَيْضاً الشُّيُوخُ مَعَ الْفِتْيَانِ
\par 13 لِيُسَبِّحُوا اسْمَ الرَّبِّ لأَنَّهُ قَدْ تَعَالَى اسْمُهُ وَحْدَهُ. مَجْدُهُ فَوْقَ الأَرْضِ وَالسَّمَاوَاتِ.
\par 14 وَيَنْصِبُ قَرْناً لِشَعْبِهِ فَخْراً لِجَمِيعِ أَتْقِيَائِهِ لِبَنِي إِسْرَائِيلَ الشَّعْبِ الْقَرِيبِ إِلَيْهِ. هَلِّلُويَا.

\chapter{149}

\par 1 هَلِّلُويَا. غَنُّوا لِلرَّبِّ تَرْنِيمَةً جَدِيدَةً تَسْبِيحَتَهُ فِي جَمَاعَةِ الأَتْقِيَاءِ.
\par 2 لِيَفْرَحْ إِسْرَائِيلُ بِخَالِقِهِ. لِيَبْتَهِجْ بَنُو صِهْيَوْنَ بِمَلِكِهِمْ.
\par 3 لِيُسَبِّحُوا اسْمَهُ بِرَقْصٍ. بِدُفٍّ وَعُودٍ لِيُرَنِّمُوا لَهُ.
\par 4 لأَنَّ الرَّبَّ رَاضٍ عَنْ شَعْبِهِ. يُجَمِّلُ الْوُدَعَاءَ بِالْخَلاَصِ.
\par 5 لِيَبْتَهِجِ الأَتْقِيَاءُ بِمَجْدٍ. لِيُرَنِّمُوا عَلَى مَضَاجِعِهِمْ.
\par 6 تَنْوِيهَاتُ اللهِ فِي أَفْوَاهِهِمْ وَسَيْفٌ ذُو حَدَّيْنِ فِي يَدِهِمْ.
\par 7 لِيَصْنَعُوا نَقْمَةً فِي الأُمَمِ وَتَأْدِيبَاتٍ فِي الشُّعُوبِ.
\par 8 لأَسْرِ مُلُوكِهِمْ بِقُيُودٍ وَشُرَفَائِهِمْ بِكُبُولٍ مِنْ حَدِيدٍ.
\par 9 لِيُجْرُوا بِهِمُِ الْحُكْمَ الْمَكْتُوبَ. كَرَامَةٌ هَذَا لِجَمِيعِ أَتْقِيَائِهِ. هَلِّلُويَا.

\chapter{150}

\par 1 هَلِّلُويَا. سَبِّحُوا اللهَ فِي قُدْسِهِ. سَبِّحُوهُ فِي فَلَكِ قُوَّتِهِ.
\par 2 سَبِّحُوهُ عَلَى قُوَّاتِهِ. سَبِّحُوهُ حَسَبَ كَثْرَةِ عَظَمَتِهِ.
\par 3 سَبِّحُوهُ بِصَوْتِ الصُّورِ. سَبِّحُوهُ بِرَبَابٍ وَعُودٍ.
\par 4 سَبِّحُوهُ بِدُفٍّ وَرَقْصٍ. سَبِّحُوهُ بِأَوْتَارٍ وَمِزْمَارٍ.
\par 5 سَبِّحُوهُ بِصُنُوجِ التَّصْوِيتِ. سَبِّحُوهُ بِصُنُوجِ الْهُتَافِ.
\par 6 كُلُّ نَسَمَةٍ فَلْتُسَبِّحِ الرَّبَّ. هَلِّلُويَا.

\end{document}