\begin{document}

\title{2 باروخ}

\chapter{1}

\par \textit{إعلان عن خراب أورشليم القادم لباروخ}

\par 1 وفي السنة الخامسة والعشرين ليكنيا ملك يهوذا، كانت كلمة الرب إلى باروخ بن نيريا قائلة له:

\par 2 هل رأيت كل ما يفعله هذا الشعب بي، حتى إن شرور هذين السبطين الباقيين أعظم من شرور الأسباط العشرة الذين سُبيوا؟

\par 3 فالقبائل السابقة أُجبرت من قبل ملوكها على ارتكاب الخطيئة، أما هذان الاثنان فقد أجبرا ملوكهما وأجبروهما على ارتكاب الخطيئة

\par 4 لذلك، ها أنا جالب شرًا على هذه المدينة وعلى سكانها، وستزول من أمامي إلى حين، وسأشتت هذا الشعب بين الأمم ليحسنوا إلى الأمم. وسيتأدب شعبي، وسيأتي الوقت الذي يطلبون فيه خير عصرهم

\chapter{2}

\par 1 لأني قلت لكم هذا لكي تأمروا إرميا وكل من هم مثلكم بالخروج من هذه المدينة

\par 2 لأن أعمالكم هي عمود ثابت لهذه المدينة،

"و صلواتك كالسور المتين."

\chapter{3}

\par 1 فقلت: يا رب، يا رب، هل جئت إلى الدنيا لأرى شرور أمي؟ ليس يا رب

\par 2 إن كنت قد وجدت نعمة في عينيك فخذ روحي أولاً لكي أذهب إلى آبائي ولا أنظر هلاك أمي.

\par 3 فإن شيئين يضغطان عليّ بشدة: فأنا لا أستطيع مقاومتك، وعلاوة على ذلك، فإن روحي لا تستطيع أن ترى شرور أمي.

\par 4 ولكن شيئاً واحداً أقوله أمامك يا رب.

\par 5 فماذا يكون بعد هذه الأمور؟ لأنه إذا أهلكت مدينتك، وسلمت أرضك إلى مبغضينا، فكيف يُذكر اسم إسرائيل من جديد؟

\par 6 أو كيف يتكلم المرء عن تسبيحك؟ أو لمن يُفسَّر ما في شريعتك؟ أم يعود العالم إلى طبيعته السابقة، ويعود العصر إلى الصمت البدائي؟ وهل تُؤخذ النفوس الكثيرة، ولا تُسمَّى طبيعة الإنسان مرة أخرى؟ وأين كل ما قلته عنا؟

\chapter{4}

\par 1 وقال لي الرب:

\par «هذه المدينة ستُسلَّم إلى حين،

\par ويُؤدَّب الشعب إلى حين،

\par والعالم لن يُسلم إلى النسيان.

\par \textit{أورشليم السماوية}

\par 2 [أتظن أن هذه هي المدينة التي قلت عنها: على راحتي نقشتك؟]

\par 3 هذا البناء الذي بني الآن في وسطكم ليس هو الذي أُعلن عندي، والذي أُعِدَّ مسبقًا هنا منذ الوقت الذي اتخذت فيه المشورة لإنشاء الفردوس، وأريته لآدم قبل أن يخطئ، ولكن عندما تجاوز الوصية أُزيل منه، كما أُزيلت منه الفردوس أيضًا.

\par 4 وبعد هذه الأمور أريتها لعبدي إبراهيم ليلا بين أجزاء الضحايا.

\par 5 وأريته أيضاً لموسى على جبل سيناء حين أريته شبه المسكن وكل آنيته.

\par 6 والآن، ها هي محفوظَةٌ لديّ، كفردوس.

\par 7 فاذهب إذن وافعل كما آمرك.

\chapter{5}

\par \textit{شكوى باروخ وطمأنينة الله}

\par 1 فأجبت وقلت:

\par 'لذلك أنا مقدر أن أحزن على صهيون،

\par لأن أعدائك سيأتون إلى هذا المكان ويدنسون مقدسك،

\par وتأسر ميراثك إلى السبي،

\par ويجعلون أنفسهم سادة على الذين أحببتهم،

\par ثم ينصرفون إلى مكان أصنامهم،

\par ويفتخر أمامهم:

\par وماذا ستفعل من أجل اسمك العظيم؟

\par 2 وقال لي الرب:

\par اسمي ومجدي إلى الأبد

\par وسيأتي حكمي في وقته.

\par 3 وسترى بعينيك

\par أن العدو لن يقلب صهيون،

ولا يحرقون أورشليم،

ولكن كونوا وزراء للقاضي إلى هذا الوقت.

\par 4 ولكن هل تذهب وتفعل ما قلته لك؟


\par 5 فذهبت وأخذت إرميا وعدو وسرياه ويابيش وجدليا وكل رجال الشعب الشرفاء، وجئت بهم إلى وادي قدرون، وأخبرتهم بكل ما قيل لي.

\par 6 ورفعوا أصواتهم وبكوا جميعهم.

\par 7 وجلسنا هناك وصمنا حتى المساء.

\chapter{6}

\par \textit{غزو الكلدانيين ودخولهم المدينة بعد إخفاء الأواني المقدسة وهدم الملائكة لأسوار المدينة}

\par 1 وفي الغد إذا جيش الكلدانيين قد أحاط بالمدينة، وفي وقت المساء تركت أنا باروخ من بين الشعب وخرجت ووقفت بجانب البلوطة.

\par 2 وكنت أحزن على صهيون، وأنوح على السبي الذي أصاب الشعب.

\par 3 وإذا بروح قوي حملني بغتة وحملني فوق سور أورشليم.

\par 4 ونظرت وإذا بأربعة ملائكة واقفين على أربع زوايا المدينة، وكل واحد منهم معه مشعل نار في يده.

\par 5 ونزل ملاك آخر من السماء وقال لهم: «امسكوا مصابيحكم ولا توقدوها حتى أقول لكم».

\par 6 لأني أُرسلت أولاً لأتكلم بكلمة إلى الأرض، وأضع فيها ما أمرني به الرب العلي.

\par 7 ورأيته ينزل إلى قدس الأقداس، ويأخذ من هناك الحجاب، وتابوت القدس، والغطاء، والمائدتين، والثياب المقدسة للكهنة، ومذبح البخور، والثمانية والأربعين حجراً كريماً التي كان الكاهن يتزين بها، وكل آنية القدس للمسكن.

\par 8 وخاطب الأرض بصوت عظيم:

\par «يا أرض، يا أرض، يا أرض، اسمعي كلام الله القدير،

\par وخذ ما أعهده إليك،

\par واحفظهم إلى يوم الدين

\par لكي إذا أمرتم تستطيعون إعادتهم،

\par حتى لا يتمكن الغرباء من امتلاكها.

\par 9 لأنه يأتي وقت تحرر فيه أورشليم أيضاً إلى حين.

\par حتى يقال أنه قد تم استعادته إلى الأبد.

\par 10 ففتحت الأرض فاها وابتلعتهم.

\chapter{7}

\par 1 وبعد هذه الأمور، سمعتُ ذلك الملاك يقول لأولئك الملائكة الذين يحملون المصابيح: «اهدموا إذن، وانقضوا جداره إلى أساساته، لئلا يفتخر العدو ويقول:

\par “لقد هدمنا سور صهيون،

"وأحرقنا مكان الله العظيم."

\par 2 وقد استولوا على المكان الذي كنت أقف فيه من قبل.

\chapter{8}

\par 1 ففعل الملائكة كما أمرهم، ولما نقضوا زوايا السور، سمع صوت من داخل الهيكل بعد سقوط السور قائلا:

\par 2 ادخلوا أيها الأعداء،

وتعالوا أيها الخصوم؛

لأن حارس البيت قد هجره

\par 3 ثم انطلقت أنا باروخ.

\par 4 وحدث بعد هذه الأمور أن جيش الكلدانيين دخل واستولوا على البيت وكل ما حوله، وساقوا الشعب سبايا، وقتلوا منهم من فرط صبرهم، وقيدوا الملك صدقيا، وأرسلوه إلى ملك بابل

\chapter{9}

\par \textit{الصوم الأول لمدة سبعة أيام: باروخ يبقى وسط أنقاض أورشليم وإرميا يرافق المنفيين إلى بابل. مرثية باروخ على أورشليم}

\par 1 فجئت أنا باروخ وإرميا الذي وجد قلبه نقيا من الخطايا ولم يؤسر في الاستيلاء على المدينة.

\par 2 فمزقنا ثيابنا وبكينا ونوحنا وصمنا سبعة أيام.

\chapter{10}

\par 1 وحدث بعد سبعة أيام أن كلمة الله جاءت إليّ وقالت لي:

\par 2 قل لإرميا أن يذهب ويساند سبي الشعب إلى بابل. أما أنتم فابقوا هنا وسط خراب صهيون، فسأريكم بعد هذه الأيام ما سيحدث في آخر الزمان. فقلت لإرميا كما أمرني الرب. فانطلق هو والشعب، أما أنا باروخ، فرجعت وجلست أمام أبواب الهيكل، وندبت صهيون ندبًا كهذا، وقلت:

\par 3 [...]

\par 4 [...]

\par 5 [...]

\par 6 طوبى لمن لم يولد،

أو من ولد ومات.

\par 7 أما نحن الأحياء، فويل لنا،

\par لأننا نرى ضيقات صهيون،

\par وماذا حل بالقدس.

\par 8 سأنادي حوريات البحر من البحر،

\par وأنتِ يا ليلين، تعالي من الصحراء،

\par وأنتم شيديم والتنينات من الغابات:

\par استيقظوا وشدوا أحقاءكم للحزن،

\par وخذ معي الترانيم،

\par واصنع معي رثاءً.

\par 9 أيها المزارعون، لا تزرعوا أيضًا.

\par ويا أرض، لماذا تعطين حصادك ثمارًا؟

احتفظ في داخلك بحلاوة رزقك.

\par 10 وأنتِ أيتها الكرمة لماذا تعطين خمرك بعد؟

لأنه لن يُقدم من هناك أيضًا قربان في صهيون.

ولا تُقَدَّمُ البواكيرُ أيضاً.

\par 11 وأنتِ أيتها السماوات، امنعِي ندائكِ،

\par ولا تفتحي خزائن المطر:

\par 12 وأنتِ يا شمس، احجبي نور أشعتك.

وأنتِ يا قمر، أطفئي كثرة نورك؛

\par لماذا يجب أن يرتفع الضوء مرة أخرى؟

\par أين أظلم نور صهيون؟

\par 13 وأنتم أيها العرسان، لا تدخلوا،

\par ولا تزين العرائس أنفسهن بأكاليل الزهور؛

وأنتم أيها النساء لا تصلين لكي تلدين.

\par 14 لأن العاقر يفرح فوق الجميع،

\par ويفرح من ليس له أبناء،

وأما الذين لهم أبناء فسوف يكون لهم ضيق.

\par 15 فلماذا يتحملون الألم،

\par فقط ليدفنوا في الحزن؟

\par 16 أو لماذا، مرة أخرى، ينبغي للبشرية أن يكون لها أبناء؟

\par أو لماذا ينبغي تسمية نسل جنسهم مرة أخرى،

\par حيث هذه الأم مهجورة،

\par وأبناؤها يُساقون إلى السبي؟

\par 17 من الآن فصاعدًا، لا تتحدث عن الجمال،

\par ولا تتحدث عن الرقة

\par 18 علاوة على ذلك، أيها الكهنة، تأخذون لأنفسكم مفاتيح الهيكل،

\par وتطرحونها في أعالي السماء،

\par وأعطهم للرب وقل:

\par “احرس منزلك بنفسك،

"يا إلهي! لقد وجدنا وكلاء كاذبين."

\par 19 وأنتِ أيتها العذارى، المنسوجات الكتان

\par والحرير بذهب أوفير،

\par خذ كل هذه الأشياء على عجل

\par وألقوهم في النار

\par لكي يحملهم إلى الذي خلقهم،

\par وأرسلهم اللهيب إلى الذي خلقهم،

\par لئلا يسيطر عليهم العدو.

\chapter{11}

\par 1 وأنا باروخ أقول هذا عليكِ يا بابل:

\par 'لو كنت قد نجحت،

\par وسكنت صهيون في مجدها،

\par ومع ذلك كان الحزن علينا عظيما

\par أن تكون مساويا لصهيون.

\par 2 ولكن الآن، انظر! الحزن لا نهاية له،

\par والرثاء لا حدود له،

\par يا إلهي! لقد ازدهرت

\par وصهيون خربة.

\par 3 من يحكم في هذه الأمور؟

\par أم إلى من نشكو مما أصابنا؟

يا رب كيف تحملت؟

\par 4 ذهب آباؤنا إلى الراحة دون حزن،

\par وها هو الأبرار ينامون في الأرض في طمأنينة؛

\par 5 لأنهم لم يعرفوا هذا العذاب،

\par ولم يسمعوا بعد بما حل بنا

\par 6 ليت لك آذانًا يا أرض،

\par وليتك لك قلبًا يا تراب:

\par لكي تذهب وتبشر في الهاوية،

\par وقل للموتى:

\par 7 «طوبى لكم أكثر منا نحن الأحياء.»

\chapter{12}

\par \textit{OXY = OXYRHYNCHUS قطعة يونانية، من برديات أوكسيرينخوس لغرينفيل وهانت، المجلد الثالث، 3-7، 1903. الوجه الخلفي.}

\par 1 [لكنني سأقول هذا كما أعتقد.] [OXY: لكنني سأقول هذا كما أعتقد،]

\par [وأتكلم عليكِ أيتها الأرض التي ما زالت مزدهرة.] [OXY: وأتكلم عليكِ أيتها الأرض التي ما زالت مزدهرة.]

\par 2 [لا يحترق الظهيرة دائمًا.] [OXY: لا يحترق الظهيرة دائمًا.]

\par [ولا تعطي أشعة الشمس الضوء باستمرار.] [OXY: ولا تعطي أشعة الشمس الضوء باستمرار.]

\par 3 [لا تتوقع أن تأمل في أن تكون دائمًا مزدهرًا ومبتهجًا.] [أوكسي: ولا تتوقع أن تفرح،]

\par [ولا تكن متكبرًا ومتباهيًا إلى حد كبير.] [OXY: ولا تدين كثيرًا.]

\par 4 [لأنه لا محالة في حينه سيستيقظ الغضب (الإلهي) ضدكم.] [أوكسي: لأنه لا محالة في حينه سيستيقظ الغضب (الإلهي) ضدكم،]

\par [الذي أصبح الآن في حالة من الصبر الطويل مقيدًا كما لو كان بزمام.] [OXY: والذي أصبح الآن مقيدًا من خلال الصبر الطويل كما لو كان بزمام.]

\par 5 [وبعد أن قلت هذه الأشياء، صمت سبعة أيام.] [أوكسي: وبعد أن قلت هذه الأشياء، صمت سبعة أيام.]


\chapter{13}

\par \textit{OXY = OXYRHYNCHUS قطعة يونانية، من برديات أوكسيرينخوس لغرينفيل وهانت، المجلد الثالث، 3-7، 1903. الوجه الخلفي.}

الصوم الثاني. رؤيا يوحنا بشأن الدينونة القادمة على الوثنيين.

\par 1 [وحدث بعد هذه الأمور أنني أنا باروخ كنت واقفًا على جبل صهيون، وإذا بصوت من العُلى وقال لي:] [أوكسي: وحدث بعد هذه الأمور أنني أنا باروخ كنت واقفًا على جبل صهيون، وإذا بصوت من العُلى وقال لي:]

\par 2 [«قف على قدميك يا باروخ، واسمع كلام الله القدير.»] [أوكسي: «قف على قدميك يا باروخ، واسمع كلام الله القدير.»]

\par 3 لأنكم دهشتم مما حل بصهيون، فستُحفظون بالتأكيد إلى انقضاء الأزمنة، لكي تكونوا شهادة

\par 4 لذلك، إذا كانت تلك المدن المزدهرة تقول:

\par 5 "لماذا أنزل الله القدير علينا هذا العقاب؟" قل لهم أنت ومن مثلك ممن رأوا هذا الشر: "هذا هو الشر والعقاب الذي يأتي عليك وعلى شعبك في وقته (المقدر) لكي تُضرب الأمم ضربة كاملة.

\par 6 وحينئذٍ يكونون في عذاب.

\par 7 وإن قالوا في ذلك الوقت:

\par 8 إلى متى؟ ستقول لهم:

\par «يا من شربتم الخمر المصفى،

\par اشربوا أيضًا من رواسبه،

\par حكم العلي

"من لا يحترم الأشخاص."

\par 9 لهذا السبب لم يرحم أبناءه من قبل،

\par بل أذلهم كأعدائه، لأنهم أخطأوا،

\par 10 لذلك تم تأديبهم

\par لكي يتقدسوا

\par 11 [ولكن الآن، أيها الشعوب والأمم، أنتم مذنبون] [OXY: (أنتم) الشعوب و...]

\par [لأنك قد وطئت الأرض دائمًا،] [OXY: لقد وطئت الأرض]

\par [واستخدموا الخلق بغير حق] [OXY: واستعملوا فيه ما خلقوا.]

\par 12 [لأني كنتُ أُفيدك دائمًا.] [أوكسي: لأنك كنتَ دائمًا تُفيد]

\par [ولقد كنت دائمًا جاحدًا للنعمة.] [OXY: ولكنك كنت دائمًا جاحدًا للنعمة.]

\chapter{14}

لم يُفدِ برُّ الصالحين لا هم ولا مدينتهم؛ أحكام الله غامضة؛ خُلِقَ العالمُ للصالحين، ومع ذلك تزول، والعالمُ باقي (١٤). الجواب: الإنسانُ يعلم أحكام الله، وقد أخطأ طواعيةً. هذا العالمُ مُرهِقٌ للصالحين، أما العالمُ القادمُ فهو لهم (١٥)، يُنالونه بالأخلاق، سواءٌ طال أمدُ الإنسانِ هنا أم قصر (١٦-١٧). النعيمُ أو الويلُ الأخيران - السؤالُ الأعظم (١٨-١٩).

\par 1 [فأجبت وقلت: هوذا قد أريتني منهج الأزمنة وما سيكون بعد هذه الأمور، وقلت لي إن العقاب الذي تحدثت عنه سيأتي على الأمم.] [أوكسي: فأجبت وقلت: هوذا قد أريتني منهج الأزمنة وما سيكون. وقلت لي إن العقاب الذي تحدثت عنه سيتحمله الأمم.]

\par 2 [والآن أعلم أن الذين أخطأوا كثيرون، وأنهم عاشوا في رخاء، ورحلوا عن العالم، ولكن ستبقى أمم قليلة في تلك الأوقات، الذين سيُقال لهم تلك الكلمات التي قلتها.] [أوكسي: والآن أعلم أن الذين أخطأوا كثيرون، وأنهم عاشوا... ورحلوا عن العالم، ولكن ستبقى أمم قليلة في تلك الأوقات الذين... الكلمات التي قلتها.]

\par 3 [فما الفائدة في هذا، أو ما هو (الشر)، الأسوأ مما رأيناه يصيبنا، والذي نتوقع رؤيته؟] [أوكسي: وما الفائدة في هذا أو ما هو أسوأ من (هذه؟)]

\par 4 ولكنني سأتحدث مرة أخرى في حضورك:

\par 5 ماذا نفع الذين عرفوا المعرفة قبلك ولم يسلكوا في الباطل مثل باقي الأمم ولم يقولوا للموتى: أعطنا حيا، بل خافوك كل حين ولم يتركوا طرقك؟

\par 6 وها هم قد سُلبوا، ولم ترحم صهيون بسببهم

\par 7 وإن كان آخرون قد فعلوا الشر، فمن حق صهيون أن تُغفر لها أعمال الذين عملوا الصالحات، ولا تُغمر بسبب أعمال الذين عملوا الإثم

\par 8 ولكن من يا رب، يا سيدي، يفهم حكمك؟

\par أو من يبحث عن عمق طريقك؟

\par أو من يفكر في ثقل طريقك؟

\par 9 أو من يستطيع أن يفكر في مشورتك غير المفهومة؟

\par أو من من الذين ولدوا وجد

\par بداية أو نهاية حكمتك؟


\par 10 لأننا جميعًا قد خُلقنا مثل نسمة.

\par 11 فكما أن النفس تصعد لا إرادياً ثم تموت، كذلك هي طبيعة البشر الذين لا يرحلون حسب إرادتهم، ولا يعرفون ما الذي سيصيبهم في النهاية.

\par 12 لأن الصديقين يرجون النهاية بحق، ويغادرون هذا المسكن بلا خوف، لأن لديهم معكم كنزًا من الأعمال محفوظًا في الكنوز

\par 13 ولذلك فإن هؤلاء أيضاً يتركون هذا العالم بلا خوف، واثقين بفرح يرجون أن ينالوا العالم الذي وعدتهم به.

\par 14 أما نحن، فويل لنا، نحن أيضًا الذين نُعامل الآن بخزٍ، وفي ذلك الوقت لا ننظر إلا إلى الشرور

\par 15 لكنك تعلم جيدًا ما فعلته بواسطة عبادك؛ لأننا لسنا قادرين على فهم ما هو صالح مثلك يا خالقنا

\par 16 ولكني سأتكلم مرة أخرى في حضرتك يا رب، يا سيدي.

\par 17 عندما لم يكن هناك عالم بسكانه في القديم، كنت تبتكر وتتكلم بكلمة، وعلى الفور وقفت أعمال الخلق أمامك.

\par 18 وقلتَ إنك ستجعل لعالمك إنسانًا مديرًا لأعمالك، حتى يكون معلومًا أنه لم يُخلق من أجل العالم بأي حال من الأحوال، بل خُلق العالم من أجله

\par 19 والآن أرى أن العالم الذي خُلق من أجلنا، هوذا هو باقي؛ أما نحن الذين خُلق من أجلنا، فنرحل

\chapter{15}

\par 1 فأجاب الرب وقال لي: «لقد دهشت حقًا من رحيل الإنسان، لكنك لم تحكم جيدًا في الشرور التي تصيب أولئك الذين يخطئون

\par 2 وأما ما قلته من أن الصالحين يُؤخذون والأشرار يزدهرون،

\par 3 وأما ما قلته: "الإنسان لا يعرف حكمك" - لهذا السبب اسمع، وسأكلمك، وأنصت، وسأجعلك تسمع كلماتي

\par 4 [...]

\par 5 لم يكن الإنسان ليفهم حكمي فهمًا صحيحًا، لو لم يكن قد قبل الشريعة، وأرشدته إلى الفهم

\par 6 لكن الآن، لأنه تجاوز عن عمد، نعم، فقط على هذا الأساس الذي يعلم به، فسوف يُعذب

\par 7 وأما ما قلته عن الأبرار، من أنه من أجلهم جاء هذا العالم، فكذلك سيأتي أيضًا من أجلهم

\par 8 لأن هذا العالم بالنسبة لهم هو صراع وكدح مع مشقة كبيرة، والذي سيأتي بعده هو إكليل ذو مجد عظيم

\chapter{16}

\par 1 فأجبتُ وقلتُ: يا رب، يا سيدي، هوذا سنو هذا الزمان قليلةٌ وشرير، فمن ذا الذي يستطيع في وقته القليل أن يكتسب ما لا يُحصى؟

\chapter{17}

\par 1 فأجاب الرب وقال لي: «عند العلي لا يُؤخذ حسابٌ لوقتٍ ولا لبضع سنين

\par 2 فماذا نفع آدم أن عاش تسعمائة وثلاثين سنةً وتجاوز ما أُمر به؟ فلم ينفعه كثرة الأيام التي عاشها، بل جلبت الموت وقطعت سني الذين ولدوا منه. فأي خسارة لحقت بموسى إذ عاش مئة وعشرين سنةً فقط، ولأنه كان خاضعًا لمبدعه، أوصل الشريعة إلى نسل يعقوب، وأضاء سراجًا لأمة إسرائيل؟

\chapter{18}

\par 1 فأجبت وقلت: «الذي أضاء أخذ من النور، وقليلون هم الذين قلدوه. أما الذين أضاءهم فقد أخذوا من ظلمة آدم ولم يفرحوا بنور السراج».

\chapter{19}

\par 1 فأجاب وقال لي: «لذلك وضع لهم في ذلك الوقت عهدًا وقال:

\par "ها أنا قد وضعت أمامك الحياة والموت"

وأشهد عليهم السماء والأرض.

\par 2 لأنه كان يعلم أن وقته قصير،

\par لكن السماء والأرض تبقى إلى الأبد

\par 3 ولكن بعد موته أخطأوا وتعدوا،

\par مع أنهم كانوا يعلمون أن الناموس يوبخهم،

\par والنور الذي لا يمكن أن يخطئ فيه شيء،

\par أيضا المجالات التي تشهد، وأنا.

\par 4 أما فيما يتعلق بكل ما هو كائن، فأنا أحكم عليه، ولكن لا تتشاور في نفسك بشأن هذه الأمور، ولا تحزن بسبب ما كان

\par 5 الآن، ينبغي النظر في اكتمال الوقت، سواء كان عملاً، أو ازدهارًا، أو عارًا، وليس بدايته

\par 6 لأنه إذا ازدهر الإنسان في بداياته وعومل بخجل في شيخوخته، فإنه ينسى كل الرخاء الذي كان يتمتع به

\par 7 وأيضًا، إذا عومل إنسان بإهانة في بداياته، وازدهر في نهايته، فإنه لا يتذكر معاملته السيئة مرة أخرى

\par 8 واسمع أيضًا: لو أن كل واحد منهم كان ناجحًا طوال ذلك الوقت - طوال الوقت من اليوم الذي حُكم فيه بالموت على المخالفين - وهلك في نهايته، لكان كل شيء عبثًا

\chapter{20}

\par \textit{لقد أُزيلت صهيون لتسريع مجيء الدينونة}

\par 1 "لذلك، انظروا! الأيام تأتي،

\par والأزمنة أسرع من السابقة،

\par وستمضي الفصول أسرع من تلك التي مضت،

\par وتمضي الأعوام أسرع من التي الآن.

\par 2 لذلك أخذت الآن صهيون،

\par حتى أتمكن من زيارة العالم في موسمه بشكل أسرع.

\par 3 فالآن احتفظوا في قلوبكم بكل ما أوصيكم به،

\par وأغلقه في أعماق عقلك.

\par 4 "ثم سأريكم حكم قوتي،

\par وطرقي التي لا يمكن البحث فيها.

\par 5 فاذهب وتقدس سبعة أيام ولا تأكل خبزا ولا تشرب ماء ولا تكلم أحدا.

\par 6 "ثم تعالوا إلى ذلك المكان فأظهر لكم ذاتي وأتكلم معكم بالحق وأعطيكم الوصية فيما يتعلق بأسلوب الأزمنة لأنها تأتي ولا تتأخر."

\chapter{21}

\par \textit{صوم سبعة أيام: صلاة باروخ: إجابة الله}

\par \textit{صلاة باروخ بن نيريا.}

\par 1 فذهبت إلى هناك وجلست في وادي قدرون في مغارة الأرض، وقدست نفسي هناك، ولم آكل خبزاً ولم أجوع، ولم أشرب ماءً ولم أعطش، وكنت هناك إلى اليوم السابع كما أمرني.

\par 2 وبعد ذلك وصلت إلى المكان الذي تحدث معي فيه.

\par 3 وحدث عند غروب الشمس أن روحي فكرت كثيرًا، وبدأت أتحدث في حضرة القدير، وقلت:

\par 4 يا من صنعتم الأرض، اسمعوني، يا من ثبتم الجلد بالكلمة، وثبّتم علو السماء بالروح، يا من دعوتم منذ بداية العالم ما لم يكن موجودًا بعد، وهم يطيعونكم

\par 5 أنت الذي تحكمت في الهواء بإيماءتك، ورأيت الأشياء التي ستكون مثل تلك التي تفعلها

\par 6 أنت الذي تحكم بفكر عظيم، الجيوش التي تقف أمامك: وأيضًا الكائنات المقدسة التي لا تعد ولا تحصى، التي خلقتها منذ البداية، من لهيب ونار، والتي تقف حول عرشك، تحكمها بسخط

\par 7 لك وحدك الحق في أن تفعل فورًا كل ما تريده

\par 8 الذي يُنزل قطرات المطر بالعدد على الأرض، والذي وحده يعلم انتهاء الأزمنة قبل مجيئها؛ انظر إلى صلاتي

\par 9 أنت وحدك القادر على إعالة جميع الموجودين، والذين رحلوا، والذين سيكونون، والذين يخطئون، والذين هم أبرار [كأحياء وكائنين لا يُكتشفون]. لأنك وحدك تعيش خالدًا لا يُكتشف، وتعرف عدد البشر. وإن أخطأ كثيرون مع مرور الوقت، فإن آخرين ليسوا قليلين صاروا أبرارًا

\par \textit{استخفاف باروخ بهذه الحياة.}

\par 10 [...]

\par 11 [...]

\par 12 أنت تعلم أين تحفظ نهاية الذين أخطأوا، أو اكتمال الذين كانوا صالحين

\par 13 لأنه لو كانت هناك هذه الحياة فقط، التي تخص جميع البشر، لما كان هناك شيء أكثر مرارة من هذا

\par 14 فأي نفع للقوة التي تتحول إلى مرض؟

\par أو للشبع الذي يتحول إلى مجاعة؟

\par أو الجمال الذي يتحول إلى قبح.

\par 15 لأن طبيعة الإنسان قابلة للتغيير دائمًا.

\par 16 فما كنا عليه في السابق لم نعد عليه، وما نحن عليه الآن لن نبقى عليه بعد ذلك.

\par 17 لأنه لو لم يُهَيَّأ للجميع إتمام، لكانت بدايتهم عبثًا. لكنك تُخبرني بكل ما يأتي منك، وتُنيرني بكل ما أسألك عنه

\par \textit{باروخ يصلي إلى الله لتسريع الحكم وتحقيق وعده}

\par 18 [...]

\par 19 إلى متى سيبقى ما هو قابل للفساد، وإلى متى سيزدهر زمن البشر، وإلى متى سيتلوث أولئك الذين يرتكبون المعصية في العالم بالكثير من الشر؟

\par 20 فأمر برحمة، وأتم كل ما قلت إنك تريد تحقيقه، لكي تُعرف قوتك لأولئك الذين يظنون أن طول أناتك ضعف

\par 21 وأظهر للذين لا يعرفون أن كل ما أصابنا ولمدينتنا إلى الآن كان حسب أناة قدرتك، لأنك من أجل اسمك دعوتنا شعباً محبوباً.

\par 22 لذا، ضع حدًا للفناء من الآن فصاعدًا.

\par 23 وبخ ملاك الموت وفقًا لذلك، فلتظهر مجدك، ولتكن قوة جمالك معروفة، وليُغلق الهاوية حتى لا تقبل من الآن فصاعدًا الموتى، ولتُردّ خزائن النفوس أولئك المحبوسين فيها.

\par 24 لأنه قد مضت سنون كثيرة مثل تلك التي هي خراب من أيام إبراهيم وإسحاق ويعقوب، وكل من هم مثلهم، الراقدين على الأرض، الذين قلت إنك خلقت العالم بسببهم

\par 25 والآن أظهر مجدك بسرعة، ولا تؤخر ما وعدت به

\par 26 ولما انتهيتُ من كلمات هذه الصلاة، ضعفتُ كثيرًا

\chapter{22}

\par \textit{رد الله على صلاة باروخ. سيُحقق وعده: الوقت اللازم لتحقيقه: يجب الحكم على الأشياء في ضوء اكتمالها (22). حتى تولد جميع النفوس، لا يمكن أن تأتي النهاية (23).}

\par 1 وحدث بعد هذه الأمور أنه إذا السماوات انفتحت، ونظرت، وأُعطيت قوة، وسمع صوت من العلاء وقال لي:

\par 2 باروخ، باروخ، لماذا أنت مضطرب؟

\par 3 من يسافر في طريق ولا يُكمله، أو من يُغادر بحرًا ولا يصل إلى الميناء، فهل يُعزى؟

\par 4 أو من وعد آخر بهدية ولم يوف بها، أليس هو سرقة؟

\par 5 أم من يزرع الأرض ولا يحصد ثمرها في حينه، ألا يخسر كل شيء؟

\par 6 أم من يغرس غرسًا لم يُنبت إلى حينه، أينتظر من غرسه أن يثمر منه؟

\par 7 أو امرأة حملت، فإن أسقطت، ألا تقتل ولدها قتلاً؟

\par 8 أو من يبني بيتًا، فإن لم يُسقِّفه ويُكمله، فهل يُسمَّى بيتًا؟ أخبرني بذلك أولًا

\chapter{23}

\par 1 فأجبتُ وقلتُ: «لا يا رب، يا سيدي».

\par 2 فأجابني وقال لي: «لماذا إذن تضطرب مما لا تعرفه، ولماذا أنت مضطرب بشأن ما تجهله؟»

\par 3 لأنه كما أنك لم تنسَ الذين هم الآن والذين رحلوا، كذلك أتذكر أنا الذين هم في المستقبل.

\par 4 لأنه لما أخطأ آدم وحُكم عليه بالموت، أُحصي عدد المولودين، وأُعدّ له مكانٌ يُؤوي الأحياء ويُحفظ فيه الأموات. فقبل أن يُكمل العدد المذكور، لن تحيا الخليقة [لأن روحي هو خالق الحياة]، وستستقبل الهاوية الأموات.

\par 5 [...]

\par 6 ويُعطى لكم أيضًا أن تسمعوا ما هي الأمور التي ستأتي بعد هذه الأوقات.

\par 7 لأن خلاصي قد اقترب حقًا، ولم يعد بعيدًا كما كان من قبل

\chapter{24}

\par \textit{الدينونة القادمة}

\par 1 «لأنه هوذا أيام تأتي وتُفتح الأسفار التي كُتبت فيها خطايا جميع الذين أخطأوا، وأيضًا الكنوز التي جُمعت فيها بر جميع الذين كانوا صالحين في الخليقة.»

\par 2 لأنه سيحدث في ذلك الوقت أنكم ترون - وكثيرين الذين معكم - طول أناة العلي، التي كانت طوال الأجيال، والذي كان طويل الأناة تجاه كل من يولدون، سواء كانوا خاطئين أو أبرارًا

\par 3 فأجبتُ وقلتُ: "لكن يا رب، لا أحد يعلم عدد الأشياء التي مضت ولا عدد الأشياء التي ستأتي."

\par 4 لأني أعلم حقًا ما أصابنا، لكنني لا أعلم ماذا سيحدث لأعدائنا، ومتى ستفتقد أعمالك

\chapter{25}

\par \textit{علامة الدينونة القادمة}

\par 1 فأجاب وقال لي: «أنت أيضًا ستُحفظ إلى ذلك الوقت، إلى تلك العلامة التي سيصنعها العلي لسكان الأرض في آخر الأيام».

\par 2 وهذه ستكون العلامة.

\par 3 عندما يُصيب سُكّان الأرض ذهول، ويسقطون في محنٍ كثيرة، ثم يقعون في عذاباتٍ عظيمة. وسيحدث ذلك عندما يقولون في خواطرهم من شدة محنتهم: «الربّ لا يذكر الأرض بعد الآن» - نعم، سيحدث ذلك عندما يتخلى عن الأمل، أن الوقت سيستيقظ حينها.

\chapter{26}

\par \textit{الويلات الاثنتا عشرة القادمة على الأرض: المسيح والمملكة المسيحانية المؤقتة}

\par 1 فأجبت وقلت: هل يستمر هذا الضيق الذي سيأتي طويلاً، وهل ستستمر هذه الضرورة لسنوات عديدة؟

\chapter{27}

\par 1 فأجابني وقال لي: «يُقسَّم ذلك الزمان إلى اثني عشر جزءًا، وكل جزء منها مخصص لما قُدِّر له».

\par 2 في الجزء الأول سيكون بداية الاضطرابات.

\par 3 وفي الجزء الثاني قتلى العظماء.

\par 4 وفي الجزء الثالث سقوط كثيرين بالموت.

\par 5 وفي الجزء الرابع إرسال السيف.

\par 6 وفي الجزء الخامس المجاعة ومنع المطر.

\par 7 وفي الجزء السادس الزلازل والرعب.

\par 8 [الرغبة.]

\par 9 وفي الجزء الثامن، كثرة من الأشباح وهجمات الشديم

\par 10 وفي الجزء التاسع سقوط النار.

\par 11 وفي العاشر نهب وظلم كثير.

\par 12 وفي الجزء الحادي عشر: الفجور والعفة.

\par 13 وفي الجزء الثاني عشر الاضطراب الناتج عن اختلاط كل تلك الأشياء المذكورة آنفاً.

\par 14 لأن هذه الأجزاء من ذلك الوقت محجوزة، ويجب أن تمتزج بعضها ببعض وتخدم بعضها بعضًا

\par 15 "فإن بعضهم يتركون بعضاً من خاصتهم ويأخذون بدلاً منه من الآخرين، وبعضهم يكملون خاصتهم وخاصية الآخرين، حتى لا يفهم أولئك الذين على الأرض في تلك الأيام أن هذا هو انقضاء الأزمنة."

\chapter{28}

\par 1 «ولكن من يفهم يصبح حكيماً.»

\par 2 لأن قياس ذلك الوقت وحسابه هما جزآن في أسبوع من سبعة أسابيع

\par 3 فأجبت وقلت: «جيد للإنسان أن يأتي وينظر، ولكن من الأفضل ألا يأتي لئلا يسقط».

\par 4 [ولكنني سأقول هذا أيضًا:

\par 5 هل من هو غير قابل للفساد سيحتقر الأشياء القابلة للفساد، وكل ما يحدث في شأن تلك القابلة للفساد، حتى ينظر فقط إلى الأشياء غير القابلة للفساد؟]

\par 6 ولكن، يا رب، إن كانت تلك الأمور التي تنبأت بها لي ستحدث بالتأكيد، فأرِني هذا أيضًا إن كنت قد وجدت نعمة في عينيك

\par 7 هل تحدث هذه الأشياء في مكان واحد أو في أحد أجزاء الأرض، أم ستشهدها الأرض كلها؟

\chapter{29}

\par 1 فأجابني وقال لي: «ما سيحدث حينئذٍ للأرض كلها؛ لذلك سيختبره كل من يعيش».

\par 2 لأني في ذلك الوقت سأحمي فقط أولئك الذين يوجدون في تلك الأيام نفسها في هذه الأرض.

\par 3 ويكون عندما يتم كل ما كان يجب أن يحدث في تلك الأجزاء أن المسيح يبدأ في الظهور.

\par 4 "ويظهر بهيموت من مكانه، ويصعد ليفياثان من البحر، هذان الوحشان العظيمان اللذان خلقتهما في اليوم الخامس من الخلق، وأبقيتهما إلى ذلك الوقت. وحينئذ يكونان طعاماً لكل من بقي."

\par 5 والأرض أيضاً تعطي ثمرها عشرة آلاف ضعف، وعلى كل كرمة ألف غصن، وكل غصن ينتج ألف عنقود، وكل عنقود ينتج ألف حبة عنب، وكل حبة عنب تنتج كوراً من النبيذ.

\par 6 والذين جائعون يفرحون، وأيضا سوف يرون عجائب كل يوم.

\par 7 فإنه ستخرج من أمامي الرياح لتجلب كل صباح رائحة الفواكه العطرة، وفي نهاية اليوم السحب التي تقطر ندى الصحة.

\par 8 ويكون في ذلك الوقت نفسه أن كنز المن ينزل مرة أخرى من العلاء، فيأكلون منه في تلك السنين، لأن هؤلاء هم الذين بلغوا انقضاء الزمان

\chapter{30}

\par \textit{القيامة}

\par 1 ويكون بعد هذه الأمور، عندما يتم وقت مجيء المسيح، أنه يعود في مجد

\par 2 حينئذٍ يقوم كلُّ من ناموا على رجاءِه. ويكونُ في ذلك الوقتِ أن تُفتَحَ الخزائنُ التي فيها حُفِظَ عددُ نفوسِ الأبرار، فيخرجونَ، وتُرى جموعٌ من النفوسِ معًا في اجتماعٍ واحدٍ بفكرٍ واحد، فيفرحُ الأولونَ ولا يحزنُ الآخرون.

\par 3 لأنهم يعلمون أن الوقت الذي قيل عنه إنه انقضاء الأزمنة قد جاء

\par 4 لكن نفوس الأشرار، عندما ترى كل هذه الأشياء، ستذبل أكثر فأكثر

\par 5 لأنهم سيعلمون أن عذابهم قد جاء وهلاكهم قد وصل

\chapter{31}

\par \textit{باروخ يحث الشعب على الاستعداد لشرور أسوأ}

\par 1 وحدث بعد هذه الأمور أنني ذهبت إلى الشعب وقلت لهم: «اجمعوا إليّ جميع شيوخكم فأخاطبهم بكلام».

\par 2 فاجتمعوا كلهم ​​في وادي قدرون.

\par 3 فأجبت وقلت لهم:

\par اسمع يا إسرائيل فأكلمك،

وأصغوا يا ذرية يعقوب فأعلمكم.

\par 4 لا تنسَ صهيون،

بل تذكر ضيق أورشليم.

\par 5 لأنه هوذا الأيام قادمة،

\par عندما يصبح كل ما هو موجود فريسة للفساد

\par ويكون كأنه لم يكن.

\chapter{32}

\par 1 أما أنتم، فإن هيأتم قلوبكم لزرع ثمار الناموس فيها، فإنه سيحميكم في ذلك الوقت الذي سيزلزل فيه القدير الخليقة كلها

\par 2 لأنه بعد قليلٍ سيهتزّ بناء صهيون ليُبنى من جديد. لكن هذا البناء لن يبقى، بل سيُقتلع بعد حين، ويبقى خرابًا إلى الأبد.

\par 3 [...]

\par 4 وبعد ذلك يجب أن يتجدد في المجد، ويكتمل إلى الأبد.]

\par 5 لذلك لا ينبغي لنا أن نحزن على الشر الذي جاء الآن بقدر ما نحزن على الشر الذي سيأتي.

\par 6 لأنه ستكون هناك محنة أعظم من هاتين المحنتين عندما يجدد القدير خلقته

\par 7 والآن لا تقترب مني لبضعة أيام، ولا تطلبني حتى آتي إليك

\par 8 ولما كلمتهم بكل هذا الكلام، ذهبت أنا باروخ في طريقي، ولما رآني الشعب منطلقًا، رفعوا أصواتهم وناحوا وقالوا:

\par 9 إلى أين تبتعد عنا يا باروخ، وتتخلى عنا كما يتخلى الأب عن أبنائه الأيتام ويبتعد عنهم؟

\chapter{33}

\par 1 «هل هذه هي الوصايا التي أمرك بها رفيقك، إرميا النبي، وقال لك: انظر إلى هذا الشعب حتى أذهب وأهيئ بقية الإخوة في بابل الذين صدر عليهم الحكم بسبيهم؟ والآن إن تركتنا أنت أيضًا، لكان خيرًا لنا جميعًا أن نموت أمامك، ثم تنسحب منا.»

\chapter{34}

\par \textit{مرثاة باروخ}

\par 1 "فأجبت وقلت للشعب: حاشا لي أن أترككم أو أنسحب منكم. ولكنني أذهب إلى قدس الأقداس فقط لأسأل القدير عنكم وعن صهيون، هل أنال في شيء من الاستنارة أكثر. وبعد هذه الأمور أعود إليكم."

\chapter{35}

\par 1 وأنا باروخ، ذهبتُ إلى القدس، وجلستُ على الأنقاض وبكيتُ، وقلتُ:

\par 2 ليت عيني كانتا ينابيع،

\par وجفوني ينبوع دموع

\par 3 فكيف أنوح على صهيون؟

\par وكيف أنوح على أورشليم؟

\par 4 لأنه في ذلك المكان الذي أسجد فيه الآن،

\par كان رئيس الكهنة في القديم يقدم ذبائح مقدسة،

\par ووضع عليها بخورا طيب الرائحة.

\par 5 ولكن الآن قد تحول افتخارنا إلى تراب،

\par ورغبة أرواحنا إلى رمال

\chapter{36}

\par \textit{رؤية الغابة والكرمة والنافورة والأرز}

\par 1 وبعد أن قلت هذه الأشياء نمت هناك، ورأيت رؤيا في الليل.

\par 2 وإذا بغابة من الأشجار مزروعة في السهل، وتحيط بها جبال صخرية عالية ووعرة، وكانت تلك الغابة تشغل مساحة كبيرة.

\par 3 وإذا مقابلها كرمة نابتة، ومن تحتها خرج ينبوع هادئ.

\par 4 ثم جاءت تلك النافورة إلى الغابة وتحركت إلى أمواج عظيمة، وغمرت تلك الأمواج تلك الغابة، وفجأة اقتلعت الجزء الأكبر من تلك الغابة، وقلبت كل الجبال التي كانت حولها.

\par 5 فبدأ ارتفاع الغابة ينخفض، وانخفضت قمم الجبال، وغلبت تلك العين جدًا، حتى إنها لم تترك شيئًا من تلك الغابة العظيمة إلا أرزًا واحدًا فقط.

\par 6 أيضًا عندما أسقطتها ودمرت واقتلعت الجزء الأكبر من تلك الغابة، حتى لم يبق منها شيء، ولم يعد من الممكن التعرف على مكانها، بدأت تلك الكرمة تأتي مع النبع بسلام وهدوء عظيم، ووصلت إلى مكان ليس بعيدًا عن ذلك الأرز، وأحضروا الأرز الذي أُسقط إليها

\par 7 ونظرتُ وإذا بالكرمة قد فتحت فاها وتكلمت وقالت لذلك الأرز: ألستَ أنت ذلك الأرز الذي بقي من غابة الشر، والذي استمر الشر بواسطته، وصنع كل تلك السنين، ولم يكن هناك خير أبدًا

\par 8 وظللت تنتصر على ما ليس لك، ولم تُظهر أبدًا أي شفقة تجاه ما كان لك، وظللت تمتد سلطانك على أولئك البعيدين عنك، وعلى أولئك الذين اقتربوا منك، وتمسكت بشدة بأتعاب شرّك، ورفعت نفسك دائمًا كشخص لا يمكن اقتلاعه!

\par 9 ولكن الآن قد انقضى وقتك وحان وقتك.

\par 10 فهل تذهبين أنت أيضا أيها الأرز وراء الغابة التي ذهبت أمامك وتصبحين ترابًا معها ويختلط رمادك بها.

\par 11 والآن استلقِ في الضيق واسترح في العذاب حتى يأتي وقتك الأخير، حيث ستأتي مرة أخرى، وتتعذب أكثر.

\chapter{37}

\par 1 وبعد هذه الأشياء رأيتُ الأرز يحترق، والكرمة تنمو، هي نفسها وما حولها، والسهل مليء بأزهار لا تذبل. وبالفعل استيقظتُ وقمتُ

\chapter{38}

\par \textit{تفسير الرؤيا}

\par 1 وصليتُ وقلتُ: يا رب، يا سيدي، أنت تُنير دائمًا من ينقاد بالفهم

\par 2 شريعتك هي الحياة، وحكمتك هي الرشاد.

\par 3 فأعلمني بتفسير هذه الرؤيا.

\par 4 لأنكَ تَعلَمُ أنَّ نَفْسِي سَلِكَتْ فِي شَرِيعَتِكَ دَائِمًا، وَمِنْذُ أَيَّامِي (الأُولَى) لَمْ أَحْدْ عَنْ حِكْمَتِكَ

\chapter{39}

\par 1 فأجاب وقال لي: «يا باروخ، هذا هو تفسير الرؤيا التي رأيتها

\par 2 كما رأيتم الغابة العظيمة التي تحيط بها الجبال الشاهقة والوعرة، هذه هي الكلمة.

\par 3 هوذا أيام تأتي، وتُدمر هذه المملكة التي أهلكت صهيون، وتخضع للتي تأتي بعدها.

\par 4 وأيضاً فإنه بعد وقت أيضاً سوف يهلك، ويقوم آخر، ثالث، ويكون له سلطان لوقته، وسوف يهلك.

\par 5 وبعد هذه الأمور، ستقوم مملكة رابعة، يكون سلطانها قاسيًا وشرًا يفوق بكثير ما كان قبلها، وستحكم مرات عديدة كالغابات في السهل، وستصمد لأزمنة، وسترتفع أكثر من أرز لبنان

\par 6 وبه تُخفى الحقيقة، ويهرب إليها كل من تلوث بالإثم، كما تهرب الوحوش الشريرة وتزحف إلى الغابة

\par 7 ويكون عندما يقترب وقت اكتماله لسقوطه، حينئذٍ ستظهر رئاسة مسيّ، التي هي مثل الينبوع والكرمة، وعندما تظهر ستستأصل جمهور جندها

\par 8 وأما ما رأيتموه، من الأرز الشامخ الذي بقي من تلك الغابة، وحقيقة أن الكرمة تكلمت معه بتلك الكلمات التي سمعتموها، فهذه هي الكلمة

\chapter{40}

\par 1 سيبقى آخر قائد في ذلك الوقت على قيد الحياة، عندما يُقتل جمهور جيوشه بالسيف، ويُربط، ويصعدون به إلى جبل صهيون، وسيُدينه مسيّا على كل ذنوبه، ويجمع ويعرض أمامه جميع أعمال جيوشه

\par 2 وبعد ذلك يقتله ويحمي بقية شعبي الذين يوجدون في المكان الذي اخترته.

\par 3 وستبقى إمارته قائمة إلى الأبد، حتى ينتهي عالم الفساد، وحتى تتم الأزمنة المذكورة.

\par 4 هذه رؤيتك، وهذا تفسيرها.


\chapter{41}

\par \textit{مصير المرتدين والمهتدين}

\par 1 فأجبت وقلت: لمن وكم تكون هذه الأشياء؟ أو من سيكون مستحقا للحياة في ذلك الوقت؟

\par 2 لأني سأتكلم أمامك بكل ما أفكر فيه، وأسألك عن الأمور التي أفكر فيها.

\par 3 ها أنا أرى كثيرين من شعبك قد نكثوا عهدك، وألقوا عنهم نير شريعتك

\par 4 لكنني رأيت آخرين أيضًا تخلوا عن غرورهم، وهربوا للاحتماء تحت جناحيك

\par 5 فماذا سيكون لهم؟ أو كيف ستكون آخر مرة يستقبلون فيها؟

\par 6 أو لعل زمن هؤلاء يُوزن، فيُحكم عليهم كما تميل العارضة؟

\chapter{42}

\par 1 فأجاب وقال لي: «سأريك هذه أيضًا».

\par 2 أما ما قلته - "لمن ستكون هذه الأشياء، وكم عددها؟" - فسيكون للذين آمنوا الخير الذي قيل عنه سابقًا، وسيكون للذين يحتقرون عكس هذه الأشياء

\par 3 وأما ما قلته فيمن اقترب ومن تراجع فهذا في الكلمة.

\par 4 وأما الذين كانوا خاضعين من قبل، ثم انسحبوا واختلطوا ببذور الشعوب المختلطة، فكان زمانهم هو الأول، وكان يعد شيئا عظيما.

\par 5 وأما الذين لم يعرفوا من قبل وعرفوا الحياة من بعد واختلطوا بذرية القوم الذين انفصلوا، فإن زمان هؤلاء هو الأخير، وهو شيء عظيم.

\par 6 "ويتعاقب الزمان على الزمان، والموسم على المواسم، ويأخذ الواحد من الآخر، وحينئذٍ، من أجل الكمال، تُقارن كل الأشياء حسب قياس الأزمنة وساعات المواسم."

\par 7 لأن الفساد سيأخذ من ينتمي إليه، والحياة من ينتمي إليه

\par 8 ويُدعى التراب، ويُقال له: «أَرْدِ مَا لَيْسَ لَكَ، وَأَرْسِلْ كُلَّ مَا احْتَفَظْتَ إِلَى أَوْانِهِ».


\chapter{43}

\par \textit{أخبر باروخ عن وفاته وأمر بإعطاء أوامره الأخيرة للشعب}

\par 1 ولكن، يا باروخ، وجه قلبك إلى ما قيل لك،

\par وافهموا الأشياء التي أريتكم.

لأن هناك الكثير من التعزيات الأبدية لك.

\par 2 لأنك ستغادر هذا المكان،

\par وتجتازون المناطق التي ترونها الآن،

\par وتنسى كل ما هو قابل للفساد،

\par ولن يتذكر بعد الآن تلك الأمور التي تحدث بين البشر.

\par 3 فاذهب وأمر شعبك وتعالوا إلى هذا المكان وبعد ذلك صوموا سبعة أيام، ثم آتي إليك وأكلمك.

\par \textit{باروخ يخبر الشيوخ بموته الوشيك، لكنه يشجعهم على توقع عزاء صهيون}

\chapter{44}

\par 1 ثم انطلقت أنا باروخ من هناك، وجئت إلى شعبي، ودعوت ابني البكر، وأصدقائي [الجدليا]، وسبعة من شيوخ الشعب، وقلت لهم:

\par 2 ها أنا ذاهب إلى آبائي

\par حسب طريق كل الأرض.

\par 3 ولكن لا تبتعدوا عن طريق الناموس،

\par ولكن احذروا وانذروا الشعب الذي بقي،

لئلا يبتعدوا عن وصايا القدير.

\par 4 فإنكم ترون أن من نعبده عادل،

\par وخالقنا لا يحابي الوجوه

\par 5 وانظر ماذا حل بصهيون،

\par وماذا حدث لأورشليم

\par 6 لأنه (بذلك) يُعلن حكم القدير،

\par وطرقه التي، وإن كانت غير قابلة للاكتشاف، فهي مستقيمة

\par 7 لأنه إن صبرتم وثابرتم في مخافته،

\par ولا تنسوا شريعته،

\par سوف تتغير الأوقات بالنسبة لك إلى الأفضل.

\par وترى عزاء صهيون.

\par 8 لأن كل ما هو الآن ليس شيئًا،

\par ولكن ما سيكون فهو عظيم جدًا

\par لأن كل ما هو قابل للفساد يزول،

\par 9 وكل ما يموت سيرحل،

\par وسيُنسى كل الوقت الحاضر،

ولا يكون ذكر للزمن الحاضر الملوث بالشرور.

\par 10 لأن ما يجري الآن يجري إلى الباطل،

\par وما ينجح سرعان ما يسقط ويذل

\par 11 لأن ما سيكون سيكون موضع الرغبة،

\par وفيما سيأتي بعد ذلك سنرجو؛

\par لأنه وقت لا يمضي،

\par 12 وتأتي الساعة الباقية إلى الأبد.

\par والعالم الجديد (يأتي) الذي لا يفسد من يذهب إلى نعيمه،

\par ولا يرحم من ذهب إلى العذاب،

ولا يهلك من يعيش فيه.

\par 13 لأن هؤلاء هم الذين سيرثون ذلك الزمان الذي قيل عنه،

\par ولهم ميراث الزمان الموعود.

\par 14 هؤلاء هم الذين اقتنوا لأنفسهم كنوزًا من الحكمة،

\par وعندهم مخازن من الفهم،

\par ومن الرحمة لم ينتزعوا

\par وحفظوا حقيقة الناموس.

\par 15 لأن لهم يُعطون العالم الآتي،

وأما الباقين وهم كثيرون، فتكون مساكنهم في النار

\chapter{45}

\par 1 «فأعلموا الشعب على قدر استطاعتكم، لأن هذا العمل عملنا. لأنكم إن علمتموهم، فإنكم تحيونهم.»

\chapter{46}

\par 1 فأجاب ابني وشيوخ الشعب وقالوا لي:

\par 'هل أذلنا الجبار إلى هذا الحد؟

\par كيف تأخذك منا بسرعة؟

\par 2 وحقًا سنكون في ظلام،

\par ولن يكون هناك نور للشعب الذي بقي،

\par 3 فأين نطلب الناموس أيضًا؟

\par أو من يميز لنا بين الموت والحياة؟

\par 4 فقلت لهم: إن عرش القدير لا أستطيع مقاومته.

ومع ذلك لن ينقص إسرائيل رجل حكيم

\par ولا ابنًا للشريعة من نسل يعقوب.

\par 5 "ولكن فقط أعدوا قلوبكم لكي تعملوا بالناموس،

واخضعوا للخائفين الحكماء والفهماء.

وأعدوا أنفسكم لا تفارقوها

\par 6 لأنه إذا فعلتم هذه الأمور تأتيكم البشارة الطيبة.

\par [الذي أخبرتك به من قبل، ولا تقع في العذاب الذي شهدت لك به من قبل.]

\par 7 أما فيما يتعلق بالكلمة التي سأُؤخذ بها، فلم أُخبرهم بها ولا ابني.]

\chapter{47}

\par 1 ولما خرجتُ وصرفتهم، ذهبتُ إلى هناك وقلتُ لهم: ها أنا ذاهبٌ إلى حبرون، لأن القدير أرسلني إلى هناك

\par 2 فجئت إلى المكان الذي قيل لي فيه الكلمة، وجلست هناك وصمت سبعة أيام

\chapter{48}

\par \textit{صلاة باروخ}

\par 1 وكان بعد اليوم السابع أنني صليت أمام القدير وقلت

\par 2 يا سيدي أنت تدعو إلى ظهور الزمان،

\par وهم يقفون أمامك؛

\par أنت تجعل قوة العصور تزول،

\par ولا يقاومونكم؛

\par أنت ترتب طريقة الفصول،

\par وهم يطيعونك.

\par 3 أنت وحدك تعلم مدة الأجيال،

\par ولا تكشف أسرارك للكثيرين

\par 4 أنت تُعرِّف كثرة النار،

\par وأنت تزن خفة الريح

\par 5 تستكشف حدود المرتفعات،

\par وتفحص أعماق الظلام

\par 6 أنت تهتم بالعدد الذي يرحلون لكي يتم حفظهم، وأنت تعد مسكنًا لأولئك الذين سيموتون

\par 7 تتذكر البداية التي بدأتها،

\par والدمار الذي سيكون لا تنساه.

\par 8 بإيماءات من الخوف والسخط، أنت تأمر النيران،

\par وتتحول إلى أرواح،

\par وبكلمة تُحيي ما لم يكن،

وبيدك ما لم يأتِ بعد بقوة عظيمة.

\par 9 تُعَلِّمُ الْخَلْقَ بِفَهْمِكَ،

\par وتُحَكِّمُ الأَفْلاكُ لِتَخْدِمَ فِي أَمْرِهَا

\par 10 تقف أمامك جيوش لا حصر لها

\par ويخدمون أوامرهم بهدوء بناءً على إيماءتك

\par 11 اسمع عبدك

\par وأصغِ إلى طلبتي

\par 12 لأننا ولدنا بعد قليل،

\par ونعود بعد قليل

\par 13 لكن معك الساعات كوقت،

\par والأيام كأجيال

\par 14 فلا تغضب على الإنسان، لأنه ليس بشيء

\par 15 ولا تنظر إلى أعمالنا، فما نحن؟

\par ها نحن نأتي إلى العالم بفضل هديتك،

ولا نخرج من هنا بإرادتنا.

\par 16 لأننا لم نقل لآبائنا: أنجبونا،

\par ولا أرسلنا إلى الهاوية قائلين: اقبلونا

\par 17 فما هي قوتنا حتى نحتمل غضبك؟

\par أو ماذا نحن حتى نحتمل دينونتك؟

\par 18 احمنا برحمتك،

\par وبرحمتك أعنا

\par 19 انظر إلى الصغار الخاضعين لك،

\par وخلص كل من يقترب منك:

ولا تدمر أمل شعبنا،

ولا تقصروا أوقات عوننا.

\par 20 لأن هذه هي الأمة التي اخترتها،

\par وهؤلاء هم الشعب الذي لا تجد لهم نداً

\par 21 لكنني سأتحدث الآن أمامكم،

\par وسأقول كما يفكر قلبي

\par 22 عليك توكلنا، لأن شريعتك معنا،

\par ونعلم أننا لن نسقط ما دمنا نحفظ أحكامك

\par 23 [نحن مباركون دائمًا في كل الأحوال لأننا لم نختلط بالأمم.]

\par 24 لأننا جميعًا شعب واحد مشهور،

\par الذين تلقوا قانونًا واحدًا من واحد:

\par والشريعة التي بيننا سوف تساعدنا،

"والحكمة الفائقة التي فينا تساعدنا."

\par 25 ولما صليت وقلت هذه الأشياء، ضعفت جدًا

\par 26 فأجاب وقال لي:

\par لقد صليت ببساطة يا باروخ،

\par وقد سمعت كل كلماتك.

\par 27 لكن حكمي يُطالب بحقوقه

\par وشريعتي تُطالب بحقوقها

\par 28 لأني من كلامك سأجيبك،

\par ومن صلاتك سأكلمك

\par 29 لأن هذا هو كما يلي: من فسد ليس فاسدًا على الإطلاق؛ لقد فعل الإثم بقدر ما يستطيع أن يفعل أي شيء، ولم يتذكر صلاحي، ولم يقبل طول أناتي

\par 30 لذلك ستُرفعون بالتأكيد، كما أخبرتكم سابقًا.

\par 31 فإنه سيأتي ذلك الزمان الذي يجلب الضيق، لأنه سيأتي ويمر بسرعة شديدة، وسيكون مضطربًا في حرارة الغضب.

\par 32 ويكون في تلك الأيام أن جميع سكان الأرض يثورون بعضهم على بعض، لأنهم لم يعلموا أن دينونتي قد اقتربت

\par 33 لأنه لن يوجد كثير من الحكماء في ذلك الوقت،

\par ولن يكون الأذكياء إلا قليلين:

علاوة على ذلك، فإن أولئك الذين يعرفون سوف يلتزمون الصمت في الغالب.

\par 34 وستكون هناك شائعات وأخبار كثيرة ليست بقليلة،

\par وسيظهر فعل الأوهام،

\par والوعود لا تعد ولا تحصى،

\par بعضهم (سيثبت) عاطلاً عن العمل،

\par وبعضها سوف يتم التأكد منه.

\par 35 ويتحول الشرف إلى عار،

\par وتذل القوة إلى ازدراء،

\par ودمرت النزاهة،

\par ويصبح الجمال قبحاً.

\par 36 وسيقول كثيرون لكثيرين في ذلك الوقت:

\par «أين اختبأت كثرة الذكاء،

\par وإلى أين ذهبت جماعة الحكمة؟

\par 37 وبينما هم يتأملون هذه الأمور،

\par عندها سينشأ الحسد في أولئك الذين لم يفكروا في أنفسهم (؟)

\par والعاطفة ستسيطر على من هو مسالم،

\par فيثور كثيرون غضبا لإيذاء كثيرين،

\par ويُثيرون جيوشًا لسفك الدماء،

وفي النهاية سوف يهلكون معهم.

\par 38 وسيحدث في الوقت نفسه،

\par أن تغيير الأوقات سيجذب كل إنسان بشكل واضح،

\par لأنهم في كل تلك الأوقات لوثوا أنفسهم

\par ومارسوا الظلم،

\par وسار كل واحد في أعماله،

\par ولم يذكر شريعة القدير.

\par 39 لذلك سوف تلتهم النار أفكارهم،

\par وفي اللهيب ستُختبر تأملات أعناقهم؛

لأن القاضي يأتي ولا يتأخر.

\par 40 لأن كل واحد من سكان الأرض كان يعلم متى كان يتعدى

ولكنهم لم يعرفوا شريعتي بسبب كبريائهم.

\par 41 لكن كثيرين سيبكون حينئذٍ بالتأكيد،

\par نعم، على الأحياء أكثر من الأموات

\par 42 فأجبت وقلت:

\par «يا آدم، ماذا فعلت بكل من ولد منك؟»

\par وماذا يقال لحواء الأولى التي استمعت إلى الحية؟

\par 43 لأن كل هذا الجمع ذاهب إلى الفساد،

\par ولا يوجد إحصاء لمن تأكلهم النار

\par 44 ولكنني سأتحدث مرة أخرى في حضورك.

\par 45 أنت يا رب ربي تعلم ما في خليقتك.

\par 46 لأنك أمرت التراب قديمًا أن يُنتج آدم، وأنت تعلم عدد الذين ولدوا منه، ومدى خطاياهم قبلك، الذين وُجدوا ولم يعترفوا بك خالقهم

\par 47 وعلى كل هؤلاء فإن نهايتهم ستدينهم، وشريعتك التي تعدوها ستجازيهم في يومك

\par \textit{جزء من خطاب باروخ للشعب}

\par 48 [«ولكن الآن فلنطرد الأشرار ولنسأل عن الأبرار.»

\par 49 وسأروي نعيمهم

\par ولا تسكتوا عن ذكر مجدهم الذي هو لهم.

\par 50 فكما أنكم في وقت قصير في هذا العالم الزائل الذي تعيشون فيه، قد تحملتم الكثير من التعب،

\par لذا في ذلك العالم الذي ليس له نهاية، سوف تتلقى نورًا عظيمًا.']

\chapter{49}

\par \textit{طبيعة جسد القيامة: المصائر النهائية للأبرار والأشرار}

\par 1 ومع ذلك، سأطلب منك مرة أخرى، أيها القدير، نعم، سأطلب من كل شيء صنعته

\par 2 «بأي حالٍ سيعيش أولئك الذين يعيشون في أيامك؟»

\par أو كيف سيستمر بهاء أولئك الذين بعد ذلك الزمان؟

\par 3 هل سيستأنفون هذا الشكل من الحاضر؟

\par ويضعون هذه الأعضاء المتشابكة،

\par والتي أصبحت الآن متورطة في الشرور،

\par وفيها تتم الشرور،

\par أم أنك قد تغير هذه الأشياء التي كانت في العالم؟

\par كما هو الحال في العالم أيضًا؟

\chapter{50}

\par 1 فأجاب وقال لي:

\par «اسمع يا باروخ هذه الكلمة،

واكتب في ذاكرة قلبك كل ما تتعلم.

\par 2 لأن الأرض ستعيد الموتى بالتأكيد،

\par [الذي يتلقاه الآن، من أجل الحفاظ عليها].

\par لا يجوز أن يحدث أي تغيير في شكلها،

\par ولكن كما استقبلت، فسوف تعيدهم،

وكما أسلمتهم إليه، كذلك سيُقيمهم أيضاً.

\par 3 لأنه حينئذٍ سيكون من الضروري إظهار الأحياء أن الموتى قد عادوا إلى الحياة، وأن الذين رحلوا قد عادوا (مرة أخرى).

\par 4 "ويحدث عندما يتعرفون على كل واحد على حدة على أولئك الذين يعرفونهم الآن، فحينئذٍ ستقوى الحكمة، وستأتي الأمور التي سبق أن تحدثنا عنها."

\chapter{51}

\par 1 ويكون، متى انقضى ذلك اليوم المعين، أن صورة المدانين ستتغير فيما بعد، ومجد المبررين

\par 2 لأن مظهر أولئك الذين يتصرفون الآن بالشر سيصبح أسوأ مما هو عليه، لأنهم سيعانون من العذاب

\par 3 كذلك (أما) مجد أولئك الذين تبرروا الآن في شريعتي، الذين كان لديهم فهم في حياتهم، والذين غرسوا في قلوبهم جذر الحكمة، فحينئذٍ سيتمجد بهاؤهم في التغييرات، وسيتحول شكل وجوههم إلى نور جمالهم، حتى يتمكنوا من اقتناء وقبول العالم الذي لا يموت، والذي وُعدوا به حينئذٍ

\par 4 لأنه فوق هذا كله، سيبكي القادمون حينئذٍ، لأنهم رفضوا شريعتي، وسدُّوا آذانهم لئلا يسمعوا الحكمة أو ينالوا الفهم

\par 5 لذلك عندما يرون أولئك الذين يُرفعون عليهم الآن، (ولكن) الذين سيُرفعون ويُمجدون أكثر منهم، سيتحولون على التوالي، هؤلاء إلى بهاء الملائكة، وسيتلاشى هؤلاء أكثر في دهشة من الرؤى وعند رؤية الأشكال

\par 6 لأنهم سينظرون أولاً ثم يذهبون إلى العذاب.

\par 7 وأما الذين خلصوا بأعمالهم،

\par والذين صار لهم الناموس الآن رجاء،

\par وفهم التوقع،

\par والحكمة ثقة،

هل تظهر العجائب في وقتها؟

\par 8 لأنهم سيرون العالم الذي هو الآن غير مرئي لهم،

\par وسيرون الزمن الذي هو الآن مخفي عنهم:

\par 9 ولن يُشيخهم الزمن بعد الآن.

\par 10 لأنهم سيسكنون في أعالي ذلك العالم،

\par فيكونون مثل الملائكة،

\par وأن يكون مساويا للنجوم،

\par ويتم تحويلهم إلى كل صورة يشتهون

\par من الجمال إلى الروعة،

\par ومن النور إلى بهاء المجد.

\par 11 لأنه ستُبسط أمامهم مساحات الفردوس، وسيُرى لهم جمال جلال المخلوقات الحية التي تحت العرش، وجميع جيوش الملائكة، الذين يُمسكون الآن بكلمتي خشية أن يظهروا، ويُمسكون بأمر، حتى يقفوا في أماكنهم حتى يأتي مجيئهم

\par 12 ثم يكون في الأبرار فضلٌ يفوق فضل الملائكة

\par 13 لأن الأولين سيستقبلون الآخرين، أولئك الذين كانوا ينتظرونهم، والآخرين الذين اعتادوا أن يسمعوا عنهم أنهم رحلوا

\par 14 لأنهم قد تحرروا من عالم الضيق هذا،

\par ووضعوا عنهم عبء الضيق

\par 15 لماذا إذن فقد الرجال حياتهم؟

\par وبماذا استبدل الذين كانوا على الأرض أنفسهم؟

\par 16 لأنهم اختاروا (لا) لأنفسهم هذه المرة،

\par التي، بعيدًا عن متناول الألم، لا يمكن أن تزول:

\par ولكنهم اختاروا لأنفسهم ذلك الوقت،

\par التي كانت إصداراتها مليئة بالرثاء والشرور،

\par وأنكروا الدنيا التي لا تهرم من يأتيها،

\par ورفضوا زمن المجد،

\par لكي لا يصلوا إلى الشرف الذي قلت لكم عنه من قبل.

\chapter{52}

\par 1 فأجبت وقلت:

\par «كيف ننسى من أعد لهم الويل؟»

\par 2 فلماذا نبكي أيضًا على الذين يموتون؟

\par أو لماذا نبكي على الذين يذهبون إلى الهاوية؟

\par 3 فلتكن المراثي محفوظة لبداية ذلك العذاب القادم،

\par ولتكن الدموع محفوظة لمجيء هلاك ذلك الزمان.

\par 4 [ولكن حتى في مواجهة هذه الأمور سأتحدث.]

\par 5 وأما الصالحون فماذا يفعلون الآن؟

\par 6 افرحوا في المعاناة التي تعانينها الآن:

\par فلماذا تنتظرون هلاك أعدائكم؟

\par 7 هيئوا أنفسكم لما هو مُدْرَجٌ لكم،

\par وهيئوا أنفسكم للأجر المُدْرَج لكم.]

\chapter{53}

\par \textit{نهاية العالم للمسيح}

\par \textit{رؤية السحابة مع المياه السوداء والبيضاء}


\par 1 وبعد أن قلت هذه الأشياء، نمت هناك، ورأيت رؤيا، وإذا سحابة تصعد من بحر عظيم جدًا، وظللت أنظر إليها، وإذا بها مليئة بمياه بيضاء وسوداء، وكانت هناك ألوان كثيرة في تلك المياه نفسها، وكان يشبه البرق العظيم الذي شوهد في قمتها.

\par 2 ورأيت السحابة تمر سريعا وتقطع خطوات سريعة، وغطت كل الأرض.

\par 3 وحدث بعد هذه الأمور أن تلك السحابة ابتدأت تسكب على الأرض المياه التي فيها.

\par 4 ورأيت أنه لم يكن هناك صورة واحدة في المياه التي نزلت منه.

\par 5 ففي البدء كانت سوداء وكثيرة، ثم رأيتُ بعد ذلك أن المياه أصبحت لامعة، ولكنها لم تكن كثيرة، ثم رأيتُ بعد ذلك مياهًا سوداء، ثم بعد ذلك لامعة، ثم سوداء، ثم لامعة.

\par 6 لقد تم ذلك اثنتي عشرة مرة، لكن الأسود كان دائمًا أكثر عددًا من الساطعين.

\par 7 وحدث في نهاية السحابة أنه أمطرت مياهًا سوداء، وكانت أغمق من كل تلك المياه التي كانت من قبل، واختلطت بها النار، وحيث نزلت تلك المياه أحدثت الخراب والدمار.

\par 8 وبعد هذه الأمور، رأيت كيف أن البرق الذي رأيته على قمة السحابة، أمسك بها وألقى بها إلى الأرض

\par 9 الآن، أشرق البرق بشدة، حتى أضاء الأرض كلها، وشفى تلك المناطق التي نزلت فيها المياه الأخيرة وأحدثت دمارًا

\par 10 فامتلك كل الأرض وتسلط عليها.

\par 11 ونظرت بعد هذا، وإذا باثني عشر نهرًا تصعد من البحر، وبدأت تُحيط بذلك البرق وتصبح خاضعة له

\par 12 ومن خوفي استيقظت.

\chapter{54}

\par \textit{صلاة باروخ لتفسير الرؤيا: مجيء راميئيل لهذا الغرض}


\par 1 وتضرعت إلى القدير وقلت:

\par «أنت وحدك يا ​​رب، تعلم من قبل أعماق العالم،

\par والأشياء التي تحدث في أوقاتها تجلبها بكلمتك، وضد أعمال سكان الأرض تعجل ببدايات الأزمنة،

\par ونهاية الفصول أنت وحدك من يعرفها.

\par 2 (أنت) الذي لا شيء صعب عليك،

\par لكنك تفعل كل شيء بسهولة بإيماءة:

\par 3 (أنت) الذي تأتي إليه الأعماق كالأعالي،

\par والذي تخدم كلمته بدايات العصور:

\par 4 (أنت) الذي تكشف للذين يخافونك ما هو معد لهم،

\par لكي يتعزوا من الآن فصاعدًا

\par 5 تُري أعمالاً عظيمةً لمن لا يعلم؛

\par تُهدم حصار الجاهلين،

\par وأضيئ ما هو مظلم،

\par وأكشف ما خفي على المطهرين،

\par [الذين بالإيمان أسلموا أنفسهم لك ولشريعتك.]

\par 6 لقد أريت عبدك هذه الرؤيا؛

\par فأخبرني أيضًا بتفسيرها

\par 7 لأني أعلم أنني قد استجبت لكم فيما طلبته منكم،

وأما ما طلبته فقد أوحيت لي بالصوت الذي يجب أن أمدحك به،

\par ومن أي الأعضاء يجب أن أصعد لك التسبيح والتهليل.

\par 8 لو كانت أعضائي أفواهًا،

\par وشعر رأسي أصواتًا،

\par ومع ذلك لم أستطع أن أعطيك مكافأة المديح،

\par ولا أثني عليك كما يليق،

\par ولا أستطيع أن أروي مديحك،

\par ولا تخبر بمجد جمالك.

\par 9 ما أنا بين الناس؟

\par أو لماذا أُحسب من بين من هم أفضل مني؟

\par أنني سمعت كل هذه الأشياء العجيبة من العلي،

\par و وعود لا تعد ولا تحصى من الذي خلقني؟

\par 10 طوبى لأمي بين الوالدات،

\par وممدوحة بين النساء من ولدتني

\par 11 لأني لا أسكت عن تسبيح القدير،

وبصوت التسبيح أذكر عجائبه.

\par 12 فمن يحب عجائبك يا الله؟

\par أو من يفهم فكرك العميق في الحياة.

\par 13 لأنك بمشورتك تحكم جميع الخليقة التي خلقتها يمينك

\par وأقمت كل ينبوع نور بجانبك،

وأعددت كنوز الحكمة تحت عرشك.

\par 14 ويَهلِكُ بِحَقٍّ مَنْ لَمْ يُحِبُّوا شَرِيعَتَكَ،

\par وسينتظر عذابُ الدَّينونةِ مَنْ لَمْ يُخْضِعُوا لِقُوَّتِكَ

\par 15 فمع أن آدم أخطأ أولًا

\par وجلب الموت المبكر على الجميع،

\par ومع ذلك من الذين ولدوا منه

\par كل واحد منهم قد استعد لعذاب نفسه القادم،

وأيضاً كل واحد منهم اختار لنفسه الأمجاد الآتية.

\par 16 [فمن يؤمن سينال أجرًا بالتأكيد.

\par 17 وأما أنتم الآن أيها الأشرار فارجعوا إلى الهلاك لأنكم ستفتقدون سريعا لأنكم رفضتم سابقا فهم العلي.

\par 18 لأن أعماله لم تُعلِّمك،

\par ولا مهارة خلقه التي تُقنعك دائمًا.]

\par 19 آدم إذن ليس هو السبب، إلا روحه فقط،

\par لكن كل واحد منا كان آدم روحه

\par 20 لكن أنت يا رب، هل تشرح لي ما كشفته لي؟

وأبلغني فيما سألتك عنه.

\par 21 لأنه عند نهاية العالم، سيُنتقم من أولئك الذين فعلوا الشر حسب شرهم،

\par وتُمجِّدُ الأَمْنَاءَ حَسَبَ أَمانَتِهِمْ.

\par 22 فإنك تحكم من بين خاصتك،

\par ومن يخطئ فامحه من بين خاصتك

\chapter{55}

\par 1 ولما انتهيت من تلاوة كلمات هذه الصلاة، جلست هناك تحت شجرة، لأستريح في ظل الأغصان

\par 2 فتعجبتُ ودهشتُ، وتأملتُ في أفكاري كثرة الخير الذي رفضه الخطاة الذين على الأرض، والعذاب العظيم الذي احتقروه، مع أنهم كانوا يعلمون أنهم سيُعذبون بسبب الخطيئة التي ارتكبوها. وبينما كنتُ أفكر في هذه الأمور وما شابهها، إذا بالملاك راميئيل الذي يرأس الرؤى الحقيقية قد أُرسل إليّ، وقال لي:

\par 3 [...]

\par 4 لماذا يحزنك قلبك يا باروخ؟

\par ولماذا يزعجك فكرك؟

\par 5 لأنه إن كنتَ متأثرًا بهذا بسبب التقرير الذي سمعتَه فقط عن الحكم،

ماذا تكون حينما ترى ذلك جليا بأعينك؟

\par 6 وإن غلبك ما تتوقعه من يوم القدير،

ماذا تكونون حين تأتين مجيئه؟

\par 7 وإذا كنتَ مضطربًا للغاية عند سماعك كلمة إعلان عذاب أولئك الذين تصرفوا بحماقة،

كم بالحري عندما يكشف الحدث عن أشياء عجيبة؟

\par 8 وإن سمعتم أخبارًا عن الخير والشر الآتية وحزنتم،

ماذا ستكون عندما ترى ما سيكشفه الجلال، والذي سيدين هؤلاء ويجعل أولئك يفرحون.

\chapter{56}

\par \textit{تفسير الرؤية. ترمز المياه السوداء والمشرقة إلى تاريخ العالم من آدم إلى مجيء المسيح.}


\par 1 ولكن بما أنك طلبت من الله العلي أن يكشف لك تفسير الرؤيا التي رأيتها، فقد أُرسلت لأخبرك بها.

\par 2 ولقد بيّن لكم القدير سبل الأزمنة التي مضت، وتلك التي ستمضي في عالمه من بداية خلقه إلى انتهائه، من الخداع والصدق

\par 3 فكما رأيتم سحابة عظيمة صعدت من البحر وذهبت وغطت الأرض، فهذه هي مدة العالم (= αιων) التي صنعها القدير عندما قصد أن يصنع العالم

\par 4 وحدث عندما خرجت الكلمة من حضرته أن مدة العالم قد وُجدت إلى حد صغير، وتحددت وفقًا لكثرة ذكاء من أرسلها

\par 5 وكما رأيتم سابقًا على قمة السحابة المياه السوداء التي نزلت سابقًا على الأرض، فهذه هي المعصية التي تعداها آدم الإنسان الأول

\par 6 لأنه [منذ] عندما تعدى

\par وُجد الموت المفاجئ،

\par تم تسمية الحزن

\par وكان الألم مُعدّاً،

\par و تم خلق الألم،

\par وانتهى الأمر بالمتاعب،

\par وبدأ المرض يتوطد،

\par وكانت الهاوية تطلب أن يتجدد دمها،

\par وحدث إنجاب الأولاد،

\par وأنتج شغف الوالدين،

\par وأُهينت عظمة الإنسانية،

\par والخير يذبل.

\par 7 ما الذي يمكن أن يكون أكثر سوادًا أو قتامة من هذه الأشياء؟

\par 8 هذه هي بداية المياه السوداء التي رأيتها.

\par 9 ومن هذه (المياه) السوداء انبثقت سوداء مرة أخرى، ونتج ظلمة الظلام

\par 10 لأنه أصبح خطرًا على نفسه: حتى على الملائكة

\par 11 لأنه علاوة على ذلك، في ذلك الوقت الذي تم خلقه، كانوا يتمتعون بالحرية.

\par 12 وأصبح خطرًا، فنزل بعضهم واختلطوا بالنساء

\par 13 ثم عُذِّب من فعل ذلك بالسلاسل.

\par 14 وأما بقية جمهور الملائكة الذين لا عدد لهم، فتمالكوا.

\par 15 وهلك من كان على الأرض معهم بمياه الطوفان

\par 16 هذه هي المياه السوداء الأولى.

\chapter{57}

\par 1 وبعد هذه (المياه) رأيت مياها ناصعة. هذا هو ينبوع إبراهيم، وأجياله، ومجيء ابنه، وابن ابنه، وأمثالهم.

\par 2 لأنه في ذلك الوقت كان يسمى القانون غير المكتوب بينهم،

\par فتمَّت أعمال الوصايا،

\par فنشأ الإيمان بالدينونة القادمة،

\par وبناءً على ذلك، تم بناء الأمل في العالم الذي كان من المقرر أن يتجدد،

\par ووعد بالحياة الآخرة.

\par 3 هذه هي المياه المضيئة التي رأيتها.

\chapter{58}

\par 1 "وأما المياه الثالثة السوداء التي رأيتموها فهي مزيج من كل الخطايا التي فعلتها الأمم بعد موت أولئك الرجال الصالحين، وشر أرض مصر الذي ارتكبوه شراً في الخدمة التي جعلوا أبناءهم يخدمونها."

\par 2 ومع ذلك، فقد هلك هؤلاء أيضًا في النهاية.

\chapter{59}

\par 1 والمياه الرابعة المضيئة التي رأيتها هي مجيء موسى وهارون ومريم ويشوع بن نون وكالب وكل من أمثالهم

\par 2 لأنه في ذلك الوقت أضاء سراج الناموس الأبدي على جميع الجالسين في الظلمة، الذي أعلن للذين يؤمنون بوعد مكافأتهم، وللذين ينكرون عذاب النار الذي أعد لهم.

\par 3 لكن السماوات أيضًا اهتزت في ذلك الوقت من مكانها، واضطرب الذين كانوا تحت عرش القدير عندما أخذ موسى إليه

\par 4 لأنه أراه تحذيرات كثيرة مع مبادئ الشريعة واكتمال الأزمنة، كما أراه لك أيضًا، وكذلك نموذج صهيون ومقاييسها، التي على نموذجها سيُصنع مقدس الزمان الحاضر

\par 5 ثم أراه أيضًا مقاييس النار، وأعماق الهاوية، وثقل الرياح، وعدد قطرات المطر:

\par 6 وكظم الغضب، وكثرة الصبر، وصدق القضاء

\par 7 وأصل الحكمة، وغنى الفهم، وينبوع المعرفة:

\par 8 وعلو الهواء، وعظمة الفردوس، وانقضاء الدهور، وبداية يوم الدينونة:

\par 9 وعدد القرابين والأراضي التي لم تأت بعد:

\par 10 وفم جهنم، ومقام الانتقام، وموضع الإيمان، وأرض الرجاء: وشبه العذاب الآتي، وحشد الملائكة الذين لا يُحصى عددهم، والجحافل المشتعلة، وبهاء البروق، وصوت الرعود، وأوامر رؤساء الملائكة، وخزائن النور، وتغيرات الأزمنة، وبحوث الشريعة

\par 11 [...]

\par 12 هذه هي المياه الرابعة المضيئة التي رأيتها

\chapter{60}

\par 1 والمياه السوداء الخامسة التي رأيتموها تمطر هي الأعمال التي صنعها الأموريون، وتعاويذهم التي صنعوها، وشر أسرارهم، واختلاط نجاستهم

\par 2 ولكن إسرائيل أيضًا كانت ملوثة بالخطايا في أيام القضاة، على الرغم من أنهم رأوا العديد من العلامات التي كانت من قبل الذي خلقهم.

\chapter{61}

\par 1 والمياه السادسة الساطعة التي رأيتها، هذا هو الوقت الذي وُلد فيه داود وسليمان

\par 2 وكان في ذلك الوقت بناء صهيون،

\par وتدشين الحرم،

\par وسفك دماء كثيرة من الأمم التي أخطأت آنذاك،

وكثير من القرابين التي قدمت حينئذ عند تدشين الهيكل.

\par 3 وكان السلام والهدوء سائدين في ذلك الوقت،

\par 4 وسمعت الحكمة في الجماعة.

\par وتزايد غنى الفهم في الجماعات،

\par 5 وأُكملت الأعياد المقدسة بفرح وبهجة غامرة.

\par 6 ولقد تبين حينئذ أن حكم الحكام كان بلا غش،

\par وبر وصايا القدير تم بالحق.

\par 7 والأرض التي كانت محبوبة لدى الرب آنذاك،

\par ولأن سكانها لم يخطئوا، فقد مجدت فوق كل الأراضي، وحكمت مدينة صهيون آنذاك على كل الأراضي والمناطق

\par 8 هذه هي المياه الزاهية التي رأيتها.

\chapter{62}

\par 1 والمياه السابعة السوداء التي رأيتموها، هي الانحراف (الذي أحدثته) مشورة يربعام، الذي تشاور لصنع عجلين من ذهب:

\par 2 وجميع الآثام التي فعلها الملوك الذين بعده

\par 3 ولعنة إيزابل وعبادة الأصنام التي كان إسرائيل يفعلها في ذلك الوقت.

\par 4 ومنع المطر والمجاعات التي حدثت حتى أكلت النساء ثمرة أرحامهن.

\par 5 وأما زمان سبيهم الذي جاء على التسعة أسباط والنصف، لأنهم كانوا في خطايا كثيرة.

\par 6 فجاء شلمناصَّر ملك أشور وسبىهم.

\par 7 وأما فيما يتعلق بالأمم، فقد كان من الممل أن نقول كيف أنهم كانوا دائماً يمارسون الفجور والشر، ولم يمارسوا البر أبداً.

\par 8 هذه هي المياه السابعة السوداء التي رأيتها.

\chapter{63}

\par 1 والمياه الثامنة الناصعة التي رأيتها، هي استقامة حزقيا ملك يهوذا وصلاحه، ونعمة الله التي حلت عليه

\par 2 "فإن سنحاريب ثار لكي يهلك، وضايقه غضبه لكي يهلك بذلك، لأن جمهور الأمم الذين كانوا معه أيضا."

\par 3 "ولما سمع الملك حزقيا ما خطط له ملك آشور، أي أن يأتي ويقبض عليه ويهلك شعبه، السبطين والنصف اللذين بقيا، بل أراد أن يهلك صهيون أيضا، فوثق حزقيا بأعماله، ورجى بره، وتكلم مع القدير وقال:

\par 4 «هوذا سنحاريب مُستعدٌّ لإهلاكنا، وسيفتخر ويعتزّ عندما يُدمّر صهيون.»

\par 5 فسمعه الجبار، لأن حزقيا كان حكيماً،

وكان يحترم صلاته لأنه كان بارًا.

\par 6 ثم أمر القدير ملاكه الذي يتكلم معك راميئيل.

\par 7 فخرجت فأهلكت جمعهم، وكان عدد رؤسائهم مائة وخمسة وثمانين ألفًا فقط، وكان لكل رجل منهم مثله.

\par 8 "وفي ذلك الوقت أحرقت أجسادهم في الداخل، ولكنني احتفظت بملابسهم وأسلحتهم في الخارج، حتى تظهر أعمال القدير الأكثر عجبًا، وبذلك يتم التحدث عن اسمه في جميع أنحاء الأرض.

\par 9 وخلصت صهيون، وتحررت أورشليم، وتحررت إسرائيل أيضًا من الضيق

\par 10 وفرح جميع الذين في الأرض المقدسة، وتمجد اسم القدير حتى صار يُذكر

\par 11 هذه هي المياه الزاهية التي رأيتها.

\chapter{64}

\par 1 والمياه التاسعة السوداء التي رأيتها، هي كل الشر الذي كان في أيام منسى بن حزقيا

\par 2 لأنه فعل الكثير من الفجور، وقتل الصديق، وانتحل الحق، وسفك دماء الأبرياء، وتزوج نساءً دنسهن بعنف، وقلب المذابح، ودمر قرابينهن، وطرد كهنةهم لئلا يخدموا في المقدس

\par 3 وصنع تمثالاً له خمسة أوجه: أربعة منها تنظر إلى الرياح الأربع، والخامس على رأس التمثال كخصم لغيرة القدير

\par 4 ثم خرج الغضب من وجه القدير ليُقتلع صهيون، كما حدث أيضًا في أيامكم. ولكن أيضًا على السبطين والنصف خرج أمرٌ بأن يُؤخذوا سبيًا أيضًا، كما رأيتم الآن

\par 5 وازداد كفر منسى لدرجة أنه أزال تسبيح العلي من المقدس

\par 6 [...]

\par 7 لهذا السبب سُمي منسى في ذلك الوقت "الكافر"، وأخيرًا كان مسكنه في النار

\par 8 فمع أن صلاته قد استُجيبت عند العلي، إلا أنه أخيرًا، عندما أُلقي في الحصان النحاسي وسُوِّي الحصان النحاسي، كان ذلك بمثابة علامة له للساعة

\par 9 لأنه لم يعش حياة كاملة، لأنه لم يكن مستحقًا، ولكن لكي يعرف من الآن فصاعدًا من الذي سوف يعذبه في النهاية.

\par 10 فمن كان قادراً على النفع فهو قادر أيضاً على العذاب.

\chapter{65}

\par 1 وهكذا، علاوة على ذلك، تصرف منسى بتقوى، وظن أنه في زمانه لن يتحقق القدير من هذه الأمور

\par 2 هذه هي المياه التاسعة السوداء التي رأيتها.

\chapter{66}

\par 1 والمياه العشرة الناصعة التي رأيتها: هذه هي طهارة أجيال يوشيا ملك يهوذا، الذي كان الوحيد في ذلك الوقت الذي خضع لله القدير بكل قلبه وكل نفسه

\par 2 وطهر الأرض من الأصنام وقدس كل الآنية التي كانت نجسة، وأعاد القرابين إلى المذبح، ورفع قرن المقدسين، ورفع الصديقين، وكرم كل حكيم الفهم، وأعاد الكهنة إلى خدمتهم، ودمر وأزال السحرة والعرافين من الأرض.

\par 3 ولم يكتف بقتل الأشرار الأحياء، بل أخذوا أيضاً من القبور عظام الموتى وأحرقوها بالنار.

\par 4 [وأقام الأعياد والسبوت في قدسها]، وأحرق نجسيها في النار، والأنبياء الكذبة الذين ضلوا الشعب، هؤلاء أيضًا أحرقهم في النار، والشعب الذي كان يستمع إليهم وهم أحياء، طرحهم في وادي قدرون، ورجم عليهم حجارة

\par 5 وكان غارًا غيرةً للقدير من كل نفسه، وكان هو وحده ثابتًا في الشريعة في ذلك الوقت، حتى إنه لم يترك أحدًا غير مختون، أو من يفعل فجورًا في كل الأرض، كل أيام حياته

\par 6 لذلك سينال مكافأة أبدية، وسيتمجد عند القدير أكثر من كثيرين في وقت لاحق

\par 7 لأنه بفضله وبفضل أمثاله خُلقت وأُعدت الأمجاد الجليلة التي أُخبرتم عنها سابقًا. هذه هي المياه الناصعة التي رأيتموها

\chapter{67}

\par 1 والمياه الحادية عشرة السوداء التي رأيتها: هذه هي الكارثة التي تصيب صهيون الآن

\par 2 أتظن أنه لا ضائقة على الملائكة عند القدير؟

\par أن صهيون قد سلمت هكذا،

\par وأن الأمم تفتخر في قلوبها،

\par واجتمعوا إلى أصنامهم وقولوا

\par "إنها تُداس من تُداس في كثير من الأحيان،

\par وقد استعبدت من استعبد؟

\par 3 أتظنون أن العلي يفرح بهذه الأمور،

\par أو أن اسمه يُمجَّد؟

\par 4 [ولكن كيف يخدم ذلك حكمه العادل؟]

\par 5 "ولكن بعد هذه الأمور سيُقبض على المتفرقين بين الأمم بالضيق،

\par وفي الخزي يسكنون في كل مكان.

\par 6 لأنه حتى الآن تم تسليم صهيون

\par وخربت أورشليم،

هل تنجح الأصنام في مدن الأمم،

\par فيطفأ بخار دخان بخور البر الذي حسب الشريعة في صهيون،

وفي كورة صهيون في كل مكان هوذا دخان الكفر.

\par 7 "ولكن يقوم ملك بابل الذي أهلك صهيون الآن،

\par ويفتخر على الشعب،

\par ويتكلم بعظائم في قلبه أمام العلي.

\par 8 لكنه سيسقط أيضًا في النهاية. هذه هي المياه السوداء.

\chapter{68}

\par 1 والمياه الثانية عشرة المضيئة التي رأيتموها، هذه هي الكلمة. فبعد هذه الأمور سيأتي وقتٌ يقع فيه شعبكم في ضيق، حتى أنهم جميعًا معرضون للهلاك معًا.

\par 2 [...]

\par 3 ومع ذلك، سيتم إنقاذهم، وسيسقط أعداؤهم في حضرتهم

\par 4 ولهم فرح عظيم حينئذ.

\par 5 وفي ذلك الوقت بعد زمان يسير ستُبنى صهيون أيضاً، وستُعاد قرابينها، وسيعود الكهنة إلى خدمتهم، وأيضاً ستأتي الأمم لتمجيدها.

\par 6 ومع ذلك، ليس بشكل كامل كما في البداية.

\par 7 ولكن سيكون بعد هذه الأمور سقوط أمم كثيرة

\par 8 هذه هي المياه الزاهية التي رأيتها.

\chapter{69}

\par 1 لأن المياه الأخيرة التي رأيتها، والتي كانت أغمق من كل ما كان قبلها، تلك التي كانت بعد العدد الثاني عشر، والتي جُمعت معًا، تنتمي إلى العالم كله

\par 2 لأن العليَّ قسَّم منذ البداية، لأنه وحده يعلم ما سيحدث

\par 3 أما بالنسبة للفظائع والعيوب التي ستُرتكب أمامه، فقد تنبأ بستة أنواع منها

\par 4 ومن أعمال الصالحين التي ينبغي أن تُنجز أمامه، فقد سبق فرأى ستة أنواع منها، بالإضافة إلى تلك التي ينبغي أن يعملها عند اكتمال الدهر

\par 5 على حسابه لم تكن هناك مياه سوداء مع سوداء، ولا لامعة مع لامعة؛ لأنه هو الكمال

\chapter{70}

\par 1 فاسمع إذًا تفسير المياه السوداء الأخيرة التي ستأتي [بعد السوداء]: هذه هي الكلمة

\par 2 هوذا الأيام تأتي، ويكون حين ينضج وقت العصر،

\par وجاء حصاد بذورها الشريرة والطيبة،

\par أن القدير سينزل على الأرض وسكانها وحكامها

\par اضطراب الروح وذهول القلب.

\par 3 وسوف يكرهون بعضهم بعضا،

\par واستفزاز بعضهم البعض للقتال،

\par والمتوسط ​​يحكم الشريف،

\par وأولئك من الدرجة الدنيا سوف يتم رفعهم فوق المشهورين.

\par 4 وسوف يتم تسليم الكثيرين إلى أيدي القليلين،

\par والذين لم يكونوا شيئا سيتسلطون على الأقوياء،

\par ويكون للفقراء فضل يفوق الغني،

\par وسوف يرتفع الأشرار فوق الأبطال.

\par 5 والحكماء يصمتون

\par ويتكلم الجاهل،

\par ولا يمكن حينئذ تأكيد فكر الرجال،

\par ولا مشورة الأقوياء،

ولا يثبت رجاء المنتظرين.

\par 6 وعندما تتحقق تلك الأمور التي تنبأ عنها،

\par حينئذٍ سيقع الاضطراب على جميع الناس،

\par وبعضهم يسقطون في المعركة،

\par وبعضهم يهلك في الضيق،

\par 7 وبعضهم يُهلك على يد خاصته، ثم الشعوب العليا التي أعدها من قبل،

\par فيأتون ويحاربون الرؤساء الذين يبقون حينئذ.

\par 8 ويكون أن كل من ينجو من الحرب يموت بالزلزال،

ومن نجا من الزلزال فإنه يحرق بالنار،

ومن ينجو من النار يهلك بالجوع.

\par 9 [وسيحدث أن كل من ينجو من الغالبين والمغلوبين وينجو من كل هذه الأمور المذكورة أعلاه، سيُسلم إلى يدي عبدي المسيح.]

\par 10 لأن كل الأرض ستلتهم سكانها.

\chapter{71}

\par 1 وترحم الأرض المقدسة نفسها، وتحمي سكانها في ذلك الوقت.

\par 2 هذه هي الرؤيا التي رأيتها، وهذا هو تفسيرها

\par 3 لأني جئت لأخبركم بهذا، لأن صلاتكم قد سمعت عند العلي

\chapter{72}

\par 1 اسمع الآن أيضًا عن البرق الساطع الذي سيأتي عند اكتمال هذه (المياه) السوداء: هذه هي الكلمة

\par 2 وبعد أن تأتي العلامات التي أخبرتم عنها من قبل، عندما تصبح الأمم مضطربة، ويأتي وقت مسيّ، فإنه يدعو كل الأمم، فيستبقي بعضهم، ويقتل بعضهم.

\par 3 وسوف تأتي هذه الأشياء على الأمم الذين سيخلصهم.

\par 4 كل أمة لا تعرف إسرائيل ولم تدوس نسل يعقوب، ستُنجى حقًا

\par 5 وذلك لأن بعضًا من كل أمة سيخضعون لشعبك

\par 6 ولكن جميع الذين تسلطوا عليك أو عرفوك يُسلَّمون إلى السيف

\chapter{73}

\par 1 وسيحدث، عندما يُخضِع كل ما في العالم،

\par وجلس بسلام مدى الدهر على عرش ملكوته،

\par ثم ينكشف هذا الفرح،

\par والباقي سوف يظهر.

\par 2 ثم ينزل الشفاء في الندى،

\par وسوف ينسحب المرض،

\par ويزول القلق والضيق والندم من بين البشر،

\par وينتشر الفرح في كل الأرض.

\par 3 ولن يموت أحد مرة أخرى قبل الأوان،

ولا ينبغي لأي محنة أن تقع فجأة.

\par 4 والأحكام، والكلام المسيء، والمخاصمات، والانتقام،

\par والدم، والعواطف، والحسد، والكراهية،

وكل ما يشبه هذا فإنه يدخل في الدينونة عندما يُنزع.

\par 5 لأن هذه الأشياء ذاتها هي التي ملأت هذا العالم بالشرور،

وبسبب هذه الأمور أصبحت حياة الإنسان مضطربة للغاية.

\par 6 "وتأتي الوحوش البرية من الغابة وتخدم البشر"

\par وسوف تخرج الأفاعي والتنينات من جحورها لتخضع لطفل صغير.

\par 7 ولن تتألم النساء بعد الآن عندما يلدن،

\par ولن يعانين من عذاب عندما يضعن ثمرة البطن

\chapter{74}

\par 1 ويكون في تلك الأيام أن الحاصدين لا يكلون،

ولا يتعب الذين يبنون.

\par لأن الأعمال سوف تتقدم بسرعة من تلقاء نفسها

\par مع الذين يقومون بها في هدوء شديد.

\par 2 لأن ذلك الوقت هو اكتمال ما هو قابل للفساد،

\par وبداية ما لا يفنى.

\par 3 لذلك فإن الأشياء التي تنبأ عنها سوف تنتمي إليه:

لذلك فهو بعيد عن الشرور، وقريب من الأشياء التي لا تموت.

\par 4 هذا هو البرق الساطع الذي جاء بعد المياه المظلمة الأخيرة.

\chapter{75}

\par \textit{ترنيمة باروخ عن غموض طرق الله وعن رحمته التي من خلالها ينال المؤمنون الخاتمة المباركة}

\par 1 فأجبت وقلت:

من يفهم صلاحك يا رب؟

\par لأنه غير مفهوم.

\par 2 أو من يستطيع البحث في تعاطفك،

\par التي هي لا نهائية؟

\par 3 أو من يقدر على إدراك عقلك؟

\par 4 أو من يقدر على سرد أفكار عقلك؟

\par 5 أو من من المولودين يستطيع أن يأمل في الوصول إلى تلك الأشياء،

\par إلا إذا كان ممن تكونون له رحماء ورؤوفين؟

\par 6 لأنه إن لم ترحم الإنسان،

\par الذين هم تحت يمينك،

\par لم يتمكنوا من الوصول إلى تلك الأشياء،

\par ولكن يمكن استدعاء أولئك الموجودين في الأرقام المذكورة.

\par 7 ولكن إن كنا نحن الموجودين نعرف حقًا لماذا أتينا،

\par ونخضع لمن أخرجنا من مصر،

\par سوف نعود مرة أخرى ونتذكر تلك الأشياء التي مرت،

\par ويفرح بما كان.

\par 8 ولكن إن كنا لا نعرف الآن لماذا أتينا،

\par ولا نعرف تدبير الذي أخرجنا من مصر، فسنعود ونطلب ما كان الآن،

\par ويحزنون من الألم بسبب الأشياء التي حدثت.

\chapter{76}

\par \textit{أمر باروخ بتعليم الشعب لمدة أربعين يومًا ثم الاستعداد لانتقاله عند مجيء المسيح}

\par 1 فأجاب وقال لي: [بما أن وحي هذه الرؤيا قد فُسِّر لك كما طلبت]، فاسمع كلام العلي لتعلم ما سيحدث لك بعد هذه الأمور

\par 2 لأنك ستغادر هذه الأرض لا محالة، ولكن ليس إلى الموت، بل ستُحفظ إلى انقضاء الدهر.

\par 3 فاصعد إذن إلى رأس ذلك الجبل، فيمر أمامك كل أراضي تلك الأرض، وشكل العالم المسكون، وقمم الجبال، وأعماق الأودية، ولجج البحار، وعدد الأنهار، لكي تنظر إلى ما أنت تاركه وإلى أين أنت ذاهب.

\par 4 وهذا سيحدث بعد أربعين يومًا. فاذهب الآن في هذه الأيام، وعلم الشعب ما تستطيع، حتى يتعلموا فلا يموتوا في آخر الزمان، بل ليحيوا في آخر الزمان.

\chapter{77}

\par \textit{تحذير باروخ للشعب وكتابته رسالتين - واحدة إلى القبائل التسعة والنصف في آشور والأخرى إلى القبائل والنصف في بابل}


\par 1 فذهبت أنا باروخ إلى هناك وأتيت إلى الشعب وجمعتهم من الكبير إلى الصغير وقلت لهم:

\par 2 اسمعوا يا بني إسرائيل، انظروا كم أنتم كثيرون الباقون من أسباط إسرائيل الاثني عشر.

\par 3 لأن الرب أعطى لكم ولآبائكم شريعة أعظم من جميع الشعوب.

\par 4 ولأن إخوتكم تعدوا وصايا العلي،

\par فأنزل الانتقام عليك وعلى منهم.

\par ولم يشفق على الأولين

\par وأسلم الأخير أيضاً إلى السبي.

ولم يترك منهم باقيا

\par 5 ولكن ها أنت هنا معي.

\par 6 فإن سددت طرقك،

ولا ترحلوا أنتم أيضًا كما رحل إخوتكم،

\par ولكنهم سيأتون إليك.

\par 7 فهو رحيم بمن تعبدون،

\par وهو كريم بمن ترجون،

\par وهو صادق حتى أنه يفعل الخير ولا يفعل الشر.

\par 8 ألم ترَ هنا ما حلَّ بصهيون؟

\par 9 أم تظن أن المكان قد أخطأ،

\par وأنه لهذا السبب تم الانقلاب عليه؟

\par أو أن الأرض قد فعلت الحماقة،

\par وأنه لذلك تم تسليمه؟

\par 10 وألم تعلم أنه بسببك أنت الذي أخطأت،

\par من لم يخطئ سقط،

\par وبسبب أولئك الذين عملوا الشر،

\par ما لم يصنع جهالة سُلِّم إلى أعدائه؟

\par 11 فأجاب كل الشعب وقالوا لي: «بقدر ما نستطيع أن نتذكر الخيرات التي صنعها القدير لنا، فإننا نتذكرها؛ وما لا نتذكره فهو يعلمه برحمته.»

\par 12 لكن، افعل هذا لنا نحن شعبك: اكتب أيضًا إلى إخوتنا في بابل رسالة تعليم وسفر رجاء، لكي تثبتهم أيضًا قبل أن تفارقنا

\par 13 لأن رعاة إسرائيل قد هلكوا،

\par وانطفأت المصابيح التي كانت تنير،

\par وحبس الينابيع مجرى مياهها الذي كنا نشرب منه.

\par 14 وتُركنا في الظلام،

\par ووسط أشجار الغابة،

\par وعطش البرية.

\par 15 فأجبت وقلت لهم

الرعاة والمصابيح والينابيع تأتي من الناموس:

وإن ذهبنا، فالناموس يبقى.

\par 16 فإن كنتم تحترمون القانون،

\par وتحرصون على الحكمة،

\par لن ينقص المصباح،

\par والراعي لا يفشل،

\par ولن تجف النافورة.

\par 17 ومع ذلك، كما قلت لي، سأكتب أيضًا إلى إخوتكم في بابل، وسأرسل بواسطة رجال، وسأكتب بنفس الطريقة إلى الأسباط التسعة والنصف، وسأرسل بواسطة طائر

\par 18 وفي اليوم الحادي والعشرين من الشهر الثامن، أتيتُ أنا باروخ وجلستُ تحت البطمة تحت ظل الأغصان، ولم يكن معي أحد، بل كنتُ وحدي

\par 19 وكتبتُ هاتين الرسالتين: واحدة أرسلتها بنسر إلى الأسباط التسعة والنصف، والأخرى أرسلتها إلى الذين في بابل بواسطة ثلاثة رجال

\par 20 وناديت النسر وقلت له هذه الكلمات:

\par 21 "لقد خلقك العلي لتكون أعلى من جميع الطيور."

\par 22 والآن اذهب ولا تمكث في أي مكان، ولا تدخل عشًا، ولا تستقر على أي شجرة، حتى تعبر عرض مياه نهر الفرات الكثيرة، وتذهب إلى القوم الساكنين هناك، وتُلقي عليهم هذه الرسالة

\par 23 وأذكر أيضًا أنه في وقت الطوفان، تلقى نوح من الحمامة ثمرة الزيتون عندما أرسلها من الفلك.

\par 24 نعم، والغربان أيضًا خدمت إيليا، حاملةً له طعامًا كما أُمرت

\par 25 سليمان أيضًا، في زمن ملكه، أينما أراد أن يرسل أو يبحث عن أي شيء، أمر طائرًا (بالذهاب إلى هناك)، فأطاعه كما أمره

\par 26 والآن لا يرهقكم هذا، ولا تميلوا يمينًا ولا يسارًا، بل اهربوا واذهبوا في طريق مستقيم، لكي تحفظوا أمر القدير، كما قلت لكم

\chapter{78}

\par \textit{رسالة باروخ بن نيريا التي كتبها إلى الأسباط التسعة والنصف}

\par 1 هذه هي كلمات تلك الرسالة التي أرسلها باروخ بن نيريا إلى الأسباط التسعة والنصف، الذين كانوا عبر نهر الفرات، والتي كُتبت فيها هذه الأمور

\par 2 هكذا قال باروخ بن نيريا للإخوة المأسورين: «رحمة وسلام». إنني أذكر، يا إخوتي، محبة من خلقنا، الذي أحبنا منذ القدم، ولم يكرهنا قط، بل ربانا قبل كل شيء

\par 3 وأنا أعلم حقًا أننا جميعنا الأسباط الاثني عشر مرتبطون برباط واحد، لأننا مولودون من أب واحد

\par 4 لذلك حرصتُ أكثر على ترك كلمات هذه الرسالة لكم قبل أن أموت، لكي تتعزوا بشأن الشرور التي حلت بكم، ولكي تحزنوا أيضًا بشأن الشر الذي أصاب إخوتكم؛ وأيضًا، لكي تبرروا دينونته التي

\par 5 لقد قضى عليك أن تُؤسر - لأن ما عانيته لا يتناسب مع ما فعلته - حتى تُوجد في آخر الزمان أهلاً لآبائك

\par 6 لذلك، إذا كنت تعتقد أنك قد عانيت الآن من تلك الأمور من أجل خيرك، حتى لا تُدان وتُعذب في النهاية، فستحصل على رجاء أبدي؛ إذا كنت قبل كل شيء تدمر من قلبك الضلال الباطل الذي بسببه رحلت من هنا

\par 7 لأنه إن فعلتم هذه الأمور، فسوف يتذكركم دائمًا، هو الذي وعد دائمًا نيابةً عنا أولئك الذين كانوا أفضل منا، أنه لن ينسانا أو يتركنا أبدًا، بل سيجمع مرة أخرى المشتتين برحمة كبيرة

\chapter{79}

\par 1 والآن يا إخوتي، تعلموا أولاً ما حل بصهيون: كيف صعد علينا نبوخذنصر ملك بابل

\par 2 لأننا أخطأنا إلى صانعنا ولم نحفظ الوصايا التي أوصانا بها ولم يؤدبنا كما نستحق.

\par 3 فإن ما أصابكم نعاني منه أيضاً بدرجة كبيرة، لأنه أصابنا أيضاً.

\chapter{80}

\par 1 والآن يا إخوتي، أُعلِمُكُم أنه عندما حاصر العدو المدينة، أُرسِلَ ملائكة العلي، وهدموا تحصينات السور القوي، ودمروا الزوايا الحديدية المتينة التي لم يكن من الممكن اقتلاعها

\par 2 ولكنهم أخفوا جميع آنية المقدس لئلا يستولي عليها العدو.

\par 3 ولما فعلوا ذلك، سلموا للعدو السور المنهدم، والبيت المنهوب، والهيكل المحروق، والشعب المنهزم لأنهم أُسلموا، لئلا يفتخر العدو ويقول:

\par 4 «هكذا استطعنا بالقوة أن نهدم حتى بيت العلي بالحرب.» إخوتكم أيضًا قيدوهم وساقوهم إلى بابل، وأسكنوهم هناك

\par 5 لكننا تُركنا هنا، لأننا قليلون جدًا.

\par 6 هذا هو الضيق الذي كتبت إليكم عنه.

\par 7 لأني أعلم يقينًا أن (تعزية) سكان صهيون تعزيكم: بقدر ما علمتم أنها نجحت (تعزيتكم) كانت أعظم من الضيق الذي تحملتموه في اضطراركم إلى مغادرتها

\chapter{81}

\par 1 وأما التعزية فاسمع الكلمة

\par 2 لأني كنتُ حزينًا على صهيون، وصليتُ من أجل الرحمة من العلي، وقلتُ:

\par 3 "إلى متى ستستمر هذه الأمور علينا؟

\par وهل تأتي علينا هذه الشرور دائما؟

\par 4 "وفعل القدير حسب كثرة رحمته،

\par والعلي حسب عظم رحمته،

\par فأعلن لي الكلمة لكي أنال عزاءً،

\par وأراني رؤىً أن لا أحتمل الضيق بعد،

وأعلمني بسر الأزمنة.

\par ومجيء الساعات أراني.

\chapter{82}

\par 1 لذلك، يا إخوتي، كتبت إليكم لكي تعزيوا أنفسكم بشأن كثرة ضيقاتكم

\par 2 "لأنكم تعلمون أن خالقنا سينتقم لنا من جميع أعدائنا حسب كل ما فعلوا بنا، وأن مجيء العلي قريب جداً، ورحمته الآتية ومجيء دينونته ليسا بعيدين."

\par 3 فها نحن نرى الآن كثرة ازدهار الأمم،

\par على الرغم من أنهم يتصرفون بطريقة غير تقية،

\par ولكنهم يكونون كالبخار.

\par 4 ونحن نرى كثرة قوتهم،

\par على الرغم من أنهم يفعلون الشر،

\par ولكنهم يصيرون مثل قطرة.

\par 5 ونرى قوة جبروتهم.

\par على الرغم من أنهم يقاومون الجبار في كل ساعة،

ولكنهم يحسبون كالبصاق.

\par 6 ونحن ننظر إلى مجد عظمتهم،

\par مع أنهم لا يحفظون شرائع العلي،

ولكن كالدخان يزولون.

\par 7 ونحن نتأمل في جمال رشاقتهم،

\par على الرغم من أن لها علاقة بالتلوث،

ولكن كما يذبل العشب يذبل.

\par 8 وننظر إلى شدة قسوتهم،

\par مع أنهم لا يتذكرون النهاية،

ولكن كما تمر الموجة سوف ينكسرون.

\par 9 ونلاحظ فخر جبروتهم،

\par مع أنهم ينكرون نعمة الله الذي وهبهم إياها،

ولكنهم يزولون كسحابة عابرة.

\chapter{83}

\par 1 لأن العلي سيعجل أوقاته بالتأكيد،

\par وسيأتي بساعاته بالتأكيد.

\par 2 وسيدين بالتأكيد أولئك الذين في عالمه،

\par وسيفتقد كل الأشياء بالحق من خلال جميع أعمالها الخفية

\par 3 وسيفحص الأفكار الخفية بالتأكيد،

\par وما هو مدفون في الغرف السرية لجميع أعضاء البريد. وسيظهرها أمام الجميع بالتوبيخ

\par 4 فلا يصعد شيء من هذه الأمور الحاضرة إلى قلوبكم، بل قبل كل شيء لننتظر، لأن ما وُعدنا به سيأتي

\par 5 دعونا لا ننظر الآن إلى مسرات الأمم في الحاضر، بل دعونا نتذكر ما وُعدنا به في النهاية

\par 6 لأن نهايات الأزمنة والأزمنة وكل ما معها ستمر معًا لا محالة

\par 7 علاوة على ذلك، سيُظهر اكتمال العصر حينها القوة العظيمة لحاكمه، عندما تأتي كل الأشياء إلى الدينونة

\par 8 فأعدّوا قلوبكم لما آمنتم به من قبل، لئلا تصبحوا عبيدًا في كلا العالمين، فتُؤسرون هنا وتُعذبون هناك

\par 9 لأن ما هو موجود الآن، أو ما مضى، أو ما سيأتي، في كل هذه الأشياء، لا الشر شرير تمامًا، ولا الخير خير تمامًا

\par 10 لأن جميع صحة هذا العصر تتحول إلى أمراض،

\par 11 وكل قوة هذا الزمان تتحول إلى ضعف،

\par وكل قوة هذا الزمن تتحول إلى عجز،

\par 12 وكل طاقة شباب تتحول إلى شيخوخة وكمال.

\par وكل جمال رشاقة في هذا الزمان يتحول إلى باهت ومكروه،

\par 13 وكل سيادة فخورة في الحاضر تتحول إلى إذلال وعار،

\par 14 وكل مديح لمجد هذا الزمان يتحول إلى عار الصمت،

وكل روعة وغرور ووقاحة في هذا الزمان تتحول إلى خراب صامت.

\par 15 وكل بهجة وفرح في هذا الزمان يتحول إلى ديدان وفساد،

\par 16 وكل ضجيج كبرياء هذا الزمان يتحول إلى غبار وسكون

\par 17 وكل ممتلكات ثروات هذا الزمان تتحول إلى الهاوية وحدها،

\par 18 وكل نهب العاطفة في هذا الزمان يتحول إلى موت لا إرادي،

وكل هوى من شهوات هذا الزمان يتحول إلى دينونة عذاب.

\par 19 وكل حيلة ومكر في هذا الزمان يتحول إلى دليل على الحق،

\par 20 وكل حلاوة مرهم هذا الزمان تتحول إلى دينونة وإدانة،

\par 21 وكل حب للكذب يتحول إلى احتقار بالصدق.

\par 22 [فبما أن كل هذه الأمور قد حدثت الآن، فهل يظن أحد أنه لن يُنتقم؟ ولكن في نهاية الأمر، سيتحقق الحق.]

\chapter{84}

\par 1 ها أنا أعلمتكم بهذه الأمور وأنا حي، لأني قلت لكم أن تتعلموا الأمور الممتازة، لأن القدير أمرني أن أعلمكم، وسأضع أمامكم بعض وصايا دينونته قبل أن أموت.

\par 2 تذكروا أن موسى قد شهد عليكم سابقًا السماء والأرض وقال: "إن خالفتم الناموس فسوف تُشتتون، وإن حفظتموه فسوف تُحفظون".

\par 3 وأشياء أخرى كان يقولها لكم أيضًا حين كنتم أنتم الأسباط الاثنا عشر معًا في البرية

\par 4 وبعد موته، طردتموهم منكم: ولذلك جاءكم ما تنبأ عنه

\par 5 وكان موسى يُخبركم بها قبل أن تُصيبكم، وها هي قد أصابتكم: لأنكم تركتم الشريعة

\par 6 ها أنا أقول لكم أيضًا بعد أن تتألموا: إن أطعتم ما قيل لكم، تنالون من القدير كل ما هو مدفون ومحفوظ لكم

\par 7 ولتكن هذه الرسالة شهادة بيني وبينكم، لكي تتذكروا وصايا القدير، ولكي يكون لي أيضًا حجة لدى الذي أرسلني

\par 8 واذكروا الشريعة وصهيون، والأرض المقدسة وإخوتكم، وعهد آبائكم، ولا تنسوا الأعياد والسبوت. وسلموا هذه الرسالة وتقاليد الشريعة إلى أبنائكم من بعدكم، كما سلمها إليكم آباؤكم أيضًا

\par 9 [...]

\par 10 واطلبوا في كل وقت بإلحاح، وصلّوا باجتهاد من كل قلبكم، أن يصالحكم القدير، ولا يحسب كثرة خطاياكم، بل يذكر استقامة آبائكم

\par 11 لأنه إن لم يحكم علينا حسب كثرة رحمته، فويل لنا جميعًا الذين ولدوا

\chapter{85}

\par 1 واعلم أيضًا أن

\par كان لآبائنا في الأزمنة السابقة وفي الأجيال القديمة معينون،

\par الرجال الصالحين والأنبياء القديسين:

\par 2 لم يعد هناك، كنا في أرضنا

\par [وساعدونا حين أخطأنا],

\end{document}