\begin{document}

\title{2كورنثوس}


\chapter{1}

\par 1 بُولُسُ، رَسُولُ يَسُوعَ الْمَسِيحِ بِمَشِيئَةِ اللهِ، وَتِيمُوثَاوُسُ الأَخُ، إِلَى كَنِيسَةِ اللهِ الَّتِي فِي كُورِنْثُوسَ، مَعَ الْقِدِّيسِينَ أَجْمَعِينَ الَّذِينَ فِي جَمِيعِ أَخَائِيَةَ.
\par 2 نِعْمَةٌ لَكُمْ وَسَلاَمٌ مِنَ اللهِ أَبِينَا وَالرَّبِّ يَسُوعَ الْمَسِيحِ.
\par 3 مُبَارَكٌ اللهُ أَبُو رَبِّنَا يَسُوعَ الْمَسِيحِ، أَبُو الرَّأْفَةِ وَإِلَهُ كُلِّ تَعْزِيَةٍ،
\par 4 الَّذِي يُعَزِّينَا فِي كُلِّ ضِيقَتِنَا، حَتَّى نَسْتَطِيعَ أَنْ نُعَزِّيَ الَّذِينَ هُمْ فِي كُلِّ ضِيقَةٍ بِالتَّعْزِيَةِ الَّتِي نَتَعَزَّى نَحْنُ بِهَا مِنَ اللهِ.
\par 5 لأَنَّهُ كَمَا تَكْثُرُ آلاَمُ الْمَسِيحِ فِينَا، كَذَلِكَ بِالْمَسِيحِ تَكْثُرُ تَعْزِيَتُنَا أَيْضاً.
\par 6 فَإِنْ كُنَّا نَتَضَايَقُ فَلأَجْلِ تَعْزِيَتِكُمْ وَخَلاَصِكُمُ، الْعَامِلِ فِي احْتِمَالِ نَفْسِ الآلاَمِ الَّتِي نَتَأَلَّمُ بِهَا نَحْنُ أَيْضاً. أَوْ نَتَعَزَّى فَلأَجْلِ تَعْزِيَتِكُمْ وَخَلاَصِكُمْ.
\par 7 فَرَجَاؤُنَا مِنْ أَجْلِكُمْ ثَابِتٌ. عَالِمِينَ أَنَّكُمْ كَمَا أَنْتُمْ شُرَكَاءُ فِي الآلاَمِ، كَذَلِكَ فِي التَّعْزِيَةِ أَيْضاً.
\par 8 فَإِنَّنَا لاَ نُرِيدُ أَنْ تَجْهَلُوا أَيُّهَا الإِخْوَةُ مِنْ جِهَةِ ضِيقَتِنَا الَّتِي أَصَابَتْنَا فِي أَسِيَّا، أَنَّنَا تَثَقَّلْنَا جِدّاً فَوْقَ الطَّاقَةِ، حَتَّى أَيِسْنَا مِنَ الْحَيَاةِ أَيْضاً.
\par 9 لَكِنْ كَانَ لَنَا فِي أَنْفُسِنَا حُكْمُ الْمَوْتِ، لِكَيْ لاَ نَكُونَ مُتَّكِلِينَ عَلَى أَنْفُسِنَا بَلْ عَلَى اللهِ الَّذِي يُقِيمُ الأَمْوَاتَ،
\par 10 الَّذِي نَجَّانَا مِنْ مَوْتٍ مِثْلِ هَذَا، وَهُوَ يُنَجِّي. الَّذِي لَنَا رَجَاءٌ فِيهِ أَنَّهُ سَيُنَجِّي أَيْضاً فِيمَا بَعْدُ.
\par 11 وَأَنْتُمْ أَيْضاً مُسَاعِدُونَ بِالصَّلاَةِ لأَجْلِنَا، لِكَيْ يُؤَدَّى شُكْرٌ لأَجْلِنَا مِنْ أَشْخَاصٍ كَثِيرِينَ، عَلَى مَا وُهِبَ لَنَا بِوَاسِطَةِ كَثِيرِينَ.
\par 12 لأَنَّ فَخْرَنَا هُوَ هَذَا: شَهَادَةُ ضَمِيرِنَا أَنَّنَا فِي بَسَاطَةٍ وَإِخْلاَصِ اللهِ، لاَ فِي حِكْمَةٍ جَسَدِيَّةٍ بَلْ فِي نِعْمَةِ اللهِ، تَصَرَّفْنَا فِي الْعَالَمِ، وَلاَ سِيَّمَا مِنْ نَحْوِكُمْ.
\par 13 فَإِنَّنَا لاَ نَكْتُبُ إِلَيْكُمْ بِشَيْءٍ آخَرَ سِوَى مَا تَقْرَأُونَ أَوْ تَعْرِفُونَ. وَأَنَا أَرْجُو أَنَّكُمْ سَتَعْرِفُونَ إِلَى النِّهَايَةِ أَيْضاً،
\par 14 كَمَا عَرَفْتُمُونَا أَيْضاً بَعْضَ الْمَعْرِفَةِ أَنَّنَا فَخْرُكُمْ، كَمَا أَنَّكُمْ أَيْضاً فَخْرُنَا فِي يَوْمِ الرَّبِّ يَسُوعَ.
\par 15 وَبِهَذِهِ الثِّقَةِ كُنْتُ أَشَاءُ أَنْ آتِيَ إِلَيْكُمْ أَوَّلاً، لِتَكُونَ لَكُمْ نِعْمَةٌ ثَانِيَةٌ.
\par 16 وَأَنْ أَمُرَّ بِكُمْ إِلَى مَكِدُونِيَّةَ، وَآتِيَ أَيْضاً مِنْ مَكِدُونِيَّةَ إِلَيْكُمْ، وَأُشَيَّعَ مِنْكُمْ إِلَى الْيَهُودِيَّةِ.
\par 17 فَإِذْ أَنَا عَازِمٌ عَلَى هَذَا، أَلَعَلِّي اسْتَعْمَلْتُ الْخِفَّةَ، أَمْ أَعْزِمُ عَلَى مَا أَعْزِمُ بِحَسَبِ الْجَسَدِ، كَيْ يَكُونَ عِنْدِي نَعَمْ نَعَمْ وَلاَ لاَ؟
\par 18 لَكِنْ أَمِينٌ هُوَ اللهُ إِنَّ كَلاَمَنَا لَكُمْ لَمْ يَكُنْ نَعَمْ وَلاَ.
\par 19 لأَنَّ ابْنَ اللهِ يَسُوعَ الْمَسِيحَ، الَّذِي كُرِزَ بِهِ بَيْنَكُمْ بِوَاسِطَتِنَا، أَنَا وَسِلْوَانُسَ وَتِيمُوثَاوُسَ، لَمْ يَكُنْ نَعَمْ وَلاَ، بَلْ قَدْ كَانَ فِيهِ نَعَمْ.
\par 20 لأَنْ مَهْمَا كَانَتْ مَوَاعِيدُ اللهِ فَهُوَ فِيهِ النَّعَمْ وَفِيهِ الآمِينُ، لِمَجْدِ اللهِ، بِوَاسِطَتِنَا.
\par 21 وَلَكِنَّ الَّذِي يُثَبِّتُنَا مَعَكُمْ فِي الْمَسِيحِ، وَقَدْ مَسَحَنَا، هُوَ اللهُ
\par 22 الَّذِي خَتَمَنَا أَيْضاً، وَأَعْطَى عَرْبُونَ الرُّوحِ فِي قُلُوبِنَا.
\par 23 وَلَكِنِّي أَسْتَشْهِدُ اللهَ عَلَى نَفْسِي أَنِّي إِشْفَاقاً عَلَيْكُمْ لَمْ آتِ إِلَى كُورِنْثُوسَ.
\par 24 لَيْسَ أَنَّنَا نَسلُودُ عَلَى إِيمَانِكُمْ بَلْ نَحْنُ مُوازِرُونَ لِسُرُورِكُمْ. لأَنَّكُمْ بِالإِيمَانِ تَثْبُتُونَ.

\chapter{2}

\par 1 وَلَكِنِّي جَزَمْتُ بِهَذَا فِي نَفْسِي أَنْ لاَ آتِيَ إِلَيْكُمْ أَيْضاً فِي حُزْنٍ.
\par 2 لأَنَّهُ إِنْ كُنْتُ أُحْزِنُكُمْ أَنَا، فَمَنْ هُوَ الَّذِي يُفَرِّحُنِي إِلاَّ الَّذِي أَحْزَنْتُهُ؟
\par 3 وَكَتَبْتُ لَكُمْ هَذَا عَيْنَهُ حَتَّى إِذَا جِئْتُ لاَ يَكُونُ لِي حُزْنٌ مِنَ الَّذِينَ كَانَ يَجِبُ أَنْ أَفْرَحَ بِهِمْ، وَاثِقاً بِجَمِيعِكُمْ أَنَّ فَرَحِي هُوَ فَرَحُ جَمِيعِكُمْ.
\par 4 لأَنِّي مِنْ حُزْنٍ كَثِيرٍ وَكَآبَةِ قَلْبٍ كَتَبْتُ إِلَيْكُمْ بِدُمُوعٍ كَثِيرَةٍ، لاَ لِكَيْ تَحْزَنُوا، بَلْ لِكَيْ تَعْرِفُوا الْمَحَبَّةَ الَّتِي عِنْدِي وَلاَ سِيَّمَا مِنْ نَحْوِكُمْ.
\par 5 وَلَكِنْ إِنْ كَانَ أَحَدٌ قَدْ أَحْزَنَ، فَإِنَّهُ لَمْ يُحْزِنِّي، بَلْ أَحْزَنَ جَمِيعَكُمْ بَعْضَ الْحُزْنِ لِكَيْ لاَ أُثَقِّلَ.
\par 6 مِثْلُ هَذَا يَكْفِيهِ هَذَا الْقِصَاصُ الَّذِي مِنَ الأَكْثَرِينَ،
\par 7 حَتَّى تَكُونُوا - بِالْعَكْسِ - تُسَامِحُونَهُ بِالْحَرِيِّ وَتُعَزُّونَهُ، لِئَلاَّ يُبْتَلَعَ مِثْلُ هَذَا مِنَ الْحُزْنِ الْمُفْرِطِ.
\par 8 لِذَلِكَ أَطْلُبُ أَنْ تُمَكِّنُوا لَهُ الْمَحَبَّةَ.
\par 9 لأَنِّي لِهَذَا كَتَبْتُ لِكَيْ أَعْرِفَ تَزْكِيَتَكُمْ، هَلْ أَنْتُمْ طَائِعُونَ فِي كُلِّ شَيْءٍ؟
\par 10 وَالَّذِي تُسَامِحُونَهُ بِشَيْءٍ فَأَنَا أَيْضاً. لأَنِّي أَنَا مَا سَامَحْتُ بِهِ - إِنْ كُنْتُ قَدْ سَامَحْتُ بِشَيْءٍ - فَمِنْ أَجْلِكُمْ بِحَضْرَةِ الْمَسِيحِ،
\par 11 لِئَلاَّ يَطْمَعَ فِينَا الشَّيْطَانُ، لأَنَّنَا لاَ نَجْهَلُ أَفْكَارَهُ.
\par 12 وَلَكِنْ لَمَّا جِئْتُ إِلَى تَرُوَاسَ، لأَجْلِ إِنْجِيلِ الْمَسِيحِ وَانْفَتَحَ لِي بَابٌ فِي الرَّبِّ،
\par 13 لَمْ تَكُنْ لِي رَاحَةٌ فِي رُوحِي، لأَنِّي لَمْ أَجِدْ تِيطُسَ أَخِي. لَكِنْ وَدَّعْتُهُمْ فَخَرَجْتُ إِلَى مَكِدُونِيَّةَ.
\par 14 وَلَكِنْ شُكْراً لِلَّهِ الَّذِي يَقُودُنَا فِي مَوْكِبِ نُصْرَتِهِ فِي الْمَسِيحِ كُلَّ حِينٍ، وَيُظْهِرُ بِنَا رَائِحَةَ مَعْرِفَتِهِ فِي كُلِّ مَكَانٍ.
\par 15 لأَنَّنَا رَائِحَةُ الْمَسِيحِ الذَّكِيَّةِ لِلَّهِ، فِي الَّذِينَ يَخْلُصُونَ وَفِي الَّذِينَ يَهْلِكُونَ.
\par 16 لِهَؤُلاَءِ رَائِحَةُ مَوْتٍ لِمَوْتٍ، وَلأُولَئِكَ رَائِحَةُ حَيَاةٍ لِحَيَاةٍ. وَمَنْ هُوَ كُفْوءٌ لِهَذِهِ الأُمُورِ؟
\par 17 لأَنَّنَا لَسْنَا كَالْكَثِيرِينَ غَاشِّينَ كَلِمَةَ اللهِ، لَكِنْ كَمَا مِنْ إِخْلاَصٍ، بَلْ كَمَا مِنَ اللهِ نَتَكَلَّمُ أَمَامَ اللهِ فِي الْمَسِيحِ.

\chapter{3}

\par 1 أَفَنَبْتَدِئُ نَمْدَحُ أَنْفُسَنَا، أَمْ لَعَلَّنَا نَحْتَاجُ كَقَوْمٍ رَسَائِلَ تَوْصِيَةٍ إِلَيْكُمْ، أَوْ رَسَائِلَ تَوْصِيَةٍ مِنْكُمْ؟
\par 2 أَنْتُمْ رِسَالَتُنَا، مَكْتُوبَةً فِي قُلُوبِنَا، مَعْرُوفَةً وَمَقْرُوءَةً مِنْ جَمِيعِ النَّاسِ.
\par 3 ظَاهِرِينَ أَنَّكُمْ رِسَالَةُ الْمَسِيحِ، مَخْدُومَةً مِنَّا، مَكْتُوبَةً لاَ بِحِبْرٍ بَلْ بِرُوحِ اللهِ الْحَيِّ، لاَ فِي أَلْوَاحٍ حَجَرِيَّةٍ بَلْ فِي أَلْوَاحِ قَلْبٍ لَحْمِيَّةٍ.
\par 4 وَلَكِنْ لَنَا ثِقَةٌ مِثْلُ هَذِهِ بِالْمَسِيحِ لَدَى اللهِ.
\par 5 لَيْسَ أَنَّنَا كُفَاةٌ مِنْ أَنْفُسِنَا أَنْ نَفْتَكِرَ شَيْئاً كَأَنَّهُ مِنْ أَنْفُسِنَا، بَلْ كِفَايَتُنَا مِنَ اللهِ،
\par 6 الَّذِي جَعَلَنَا كُفَاةً لأَنْ نَكُونَ خُدَّامَ عَهْدٍ جَدِيدٍ. لاَ الْحَرْفِ بَلِ الرُّوحِ. لأَنَّ الْحَرْفَ يَقْتُلُ وَلَكِنَّ الرُّوحَ يُحْيِي.
\par 7 ثُمَّ إِنْ كَانَتْ خِدْمَةُ الْمَوْتِ، الْمَنْقُوشَةُ بِأَحْرُفٍ فِي حِجَارَةٍ، قَدْ حَصَلَتْ فِي مَجْدٍ، حَتَّى لَمْ يَقْدِرْ بَنُو إِسْرَائِيلَ أَنْ يَنْظُرُوا إِلَى وَجْهِ مُوسَى لِسَبَبِ مَجْدِ وَجْهِهِ الزَّائِلِ،
\par 8 فَكَيْفَ لاَ تَكُونُ بِالأَوْلَى خِدْمَةُ الرُّوحِ فِي مَجْدٍ؟
\par 9 لأَنَّهُ إِنْ كَانَتْ خِدْمَةُ الدَّيْنُونَةِ مَجْداً، فَبِالأَوْلَى كَثِيراً تَزِيدُ خِدْمَةُ الْبِرِّ فِي مَجْدٍ.
\par 10 فَإِنَّ الْمُمَجَّدَ أَيْضاً لَمْ يُمَجَّدْ مِنْ هَذَا الْقَبِيلِ لِسَبَبِ الْمَجْدِ الْفَائِقِ.
\par 11 لأَنَّهُ إِنْ كَانَ الزَّائِلُ فِي مَجْدٍ، فَبِالأَوْلَى كَثِيراً يَكُونُ الدَّائِمُ فِي مَجْدٍ.
\par 12 فَإِذْ لَنَا رَجَاءٌ مِثْلُ هَذَا نَسْتَعْمِلُ مُجَاهَرَةً كَثِيرَةً.
\par 13 وَلَيْسَ كَمَا كَانَ مُوسَى يَضَعُ بُرْقُعاً عَلَى وَجْهِهِ لِكَيْ لاَ يَنْظُرَ بَنُو إِسْرَائِيلَ إِلَى نِهَايَةِ الزَّائِلِ.
\par 14 بَلْ أُغْلِظَتْ أَذْهَانُهُمْ، لأَنَّهُ حَتَّى الْيَوْمِ ذَلِكَ الْبُرْقُعُ نَفْسُهُ عِنْدَ قِرَاءَةِ الْعَهْدِ الْعَتِيقِ بَاقٍ غَيْرُ مُنْكَشِفٍ، الَّذِي يُبْطَلُ فِي الْمَسِيحِ.
\par 15 لَكِنْ حَتَّى الْيَوْمِ، حِينَ يُقْرَأُ مُوسَى، الْبُرْقُعُ مَوْضُوعٌ عَلَى قَلْبِهِمْ.
\par 16 وَلَكِنْ عِنْدَمَا يَرْجِعُ إِلَى الرَّبِّ يُرْفَعُ الْبُرْقُعُ.
\par 17 وَأَمَّا الرَّبُّ فَهُوَ الرُّوحُ، وَحَيْثُ رُوحُ الرَّبِّ هُنَاكَ حُرِّيَّةٌ.
\par 18 وَنَحْنُ جَمِيعاً نَاظِرِينَ مَجْدَ الرَّبِّ بِوَجْهٍ مَكْشُوفٍ، كَمَا فِي مِرْآةٍ، نَتَغَيَّرُ إِلَى تِلْكَ الصُّورَةِ عَيْنِهَا، مِنْ مَجْدٍ إِلَى مَجْدٍ، كَمَا مِنَ الرَّبِّ الرُّوحِ.

\chapter{4}

\par 1 مِنْ أَجْلِ ذَلِكَ، إِذْ لَنَا هَذِهِ الْخِدْمَةُ كَمَا رُحِمْنَا، لاَ نَفْشَلُ.
\par 2 بَلْ قَدْ رَفَضْنَا خَفَايَا الْخِزْيِ، غَيْرَ سَالِكِينَ فِي مَكْرٍ، وَلاَ غَاشِّينَ كَلِمَةَ اللهِ، بَلْ بِإِظْهَارِ الْحَقِّ، مَادِحِينَ أَنْفُسَنَا لَدَى ضَمِيرِ كُلِّ إِنْسَانٍ قُدَّامَ اللهِ.
\par 3 وَلَكِنْ إِنْ كَانَ إِنْجِيلُنَا مَكْتُوماً، فَإِنَّمَا هُوَ مَكْتُومٌ فِي الْهَالِكِينَ،
\par 4 الَّذِينَ فِيهِمْ إِلَهُ هَذَا الدَّهْرِ قَدْ أَعْمَى أَذْهَانَ غَيْرِ الْمُؤْمِنِينَ، لِئَلاَّ تُضِيءَ لَهُمْ إِنَارَةُ إِنْجِيلِ مَجْدِ الْمَسِيحِ، الَّذِي هُوَ صُورَةُ اللهِ.
\par 5 فَإِنَّنَا لَسْنَا نَكْرِزُ بِأَنْفُسِنَا، بَلْ بِالْمَسِيحِ يَسُوعَ رَبّاً، وَلَكِنْ بِأَنْفُسِنَا عَبِيداً لَكُمْ مِنْ أَجْلِ يَسُوعَ.
\par 6 لأَنَّ اللهَ الَّذِي قَالَ أَنْ يُشْرِقَ نُورٌ مِنْ ظُلْمَةٍ، هُوَ الَّذِي أَشْرَقَ فِي قُلُوبِنَا، لِإِنَارَةِ مَعْرِفَةِ مَجْدِ اللهِ فِي وَجْهِ يَسُوعَ الْمَسِيحِ.
\par 7 وَلَكِنْ لَنَا هَذَا الْكَنْزُ فِي أَوَانٍ خَزَفِيَّةٍ، لِيَكُونَ فَضْلُ الْقُوَّةِ لِلَّهِ لاَ مِنَّا.
\par 8 مُكْتَئِبِينَ فِي كُلِّ شَيْءٍ، لَكِنْ غَيْرَ مُتَضَايِقِينَ. مُتَحَيِّرِينَ، لَكِنْ غَيْرَ يَائِسِينَ.
\par 9 مُضْطَهَدِينَ، لَكِنْ غَيْرَ مَتْرُوكِينَ. مَطْرُوحِينَ، لَكِنْ غَيْرَ هَالِكِينَ.
\par 10 حَامِلِينَ فِي الْجَسَدِ كُلَّ حِينٍ إِمَاتَةَ الرَّبِّ يَسُوعَ، لِكَيْ تُظْهَرَ حَيَاةُ يَسُوعَ أَيْضاً فِي جَسَدِنَا.
\par 11 لأَنَّنَا نَحْنُ الأَحْيَاءَ نُسَلَّمُ دَائِماً لِلْمَوْتِ مِنْ أَجْلِ يَسُوعَ، لِكَيْ تَظْهَرَ حَيَاةُ يَسُوعَ أَيْضاً فِي جَسَدِنَا الْمَائِتِ.
\par 12 إِذاً الْمَوْتُ يَعْمَلُ فِينَا، وَلَكِنِ الْحَيَاةُ فِيكُمْ.
\par 13 فَإِذْ لَنَا رُوحُ الإِيمَانِ عَيْنُهُ، حَسَبَ الْمَكْتُوبِ «آمَنْتُ لِذَلِكَ تَكَلَّمْتُ» - نَحْنُ أَيْضاً نُؤْمِنُ وَلِذَلِكَ نَتَكَلَّمُ أَيْضاً.
\par 14 عَالِمِينَ أَنَّ الَّذِي أَقَامَ الرَّبَّ يَسُوعَ سَيُقِيمُنَا نَحْنُ أَيْضاً بِيَسُوعَ، وَيُحْضِرُنَا مَعَكُمْ.
\par 15 لأَنَّ جَمِيعَ الأَشْيَاءِ هِيَ مِنْ أَجْلِكُمْ، لِكَيْ تَكُونَ النِّعْمَةُ وَهِيَ قَدْ كَثُرَتْ بِالأَكْثَرِينَ، تَزِيدُ الشُّكْرَ لِمَجْدِ اللهِ.
\par 16 لِذَلِكَ لاَ نَفْشَلُ. بَلْ وَإِنْ كَانَ إِنْسَانُنَا الْخَارِجُ يَفْنَى، فَالدَّاخِلُ يَتَجَدَّدُ يَوْماً فَيَوْماً.
\par 17 لأَنَّ خِفَّةَ ضِيقَتِنَا الْوَقْتِيَّةَ تُنْشِئُ لَنَا أَكْثَرَ فَأَكْثَرَ ثِقَلَ مَجْدٍ أَبَدِيّاً.
\par 18 وَنَحْنُ غَيْرُ نَاظِرِينَ إِلَى الأَشْيَاءِ الَّتِي تُرَى، بَلْ إِلَى الَّتِي لاَ تُرَى. لأَنَّ الَّتِي تُرَى وَقْتِيَّةٌ، وَأَمَّا الَّتِي لاَ تُرَى فَأَبَدِيَّةٌ.

\chapter{5}

\par 1 لأَنَّنَا نَعْلَمُ أَنَّهُ إِنْ نُقِضَ بَيْتُ خَيْمَتِنَا الأَرْضِيُّ، فَلَنَا فِي السَّمَاوَاتِ بِنَاءٌ مِنَ اللهِ، بَيْتٌ غَيْرُ مَصْنُوعٍ بِيَدٍ، أَبَدِيٌّ.
\par 2 فَإِنَّنَا فِي هَذِهِ أَيْضاً نَئِنُّ مُشْتَاقِينَ إِلَى أَنْ نَلْبَسَ فَوْقَهَا مَسْكَنَنَا الَّذِي مِنَ السَّمَاءِ.
\par 3 وَإِنْ كُنَّا لاَبِسِينَ لاَ نُوجَدُ عُرَاةً.
\par 4 فَإِنَّنَا نَحْنُ الَّذِينَ فِي الْخَيْمَةِ نَئِنُّ مُثْقَلِينَ، إِذْ لَسْنَا نُرِيدُ أَنْ نَخْلَعَهَا بَلْ أَنْ نَلْبَسَ فَوْقَهَا، لِكَيْ يُبْتَلَعَ الْمَائِتُ مِنَ الْحَيَاةِ.
\par 5 وَلَكِنَّ الَّذِي صَنَعَنَا لِهَذَا عَيْنِهِ هُوَ اللهُ، الَّذِي أَعْطَانَا أَيْضاً عَرْبُونَ الرُّوحِ.
\par 6 فَإِذاً نَحْنُ وَاثِقُونَ كُلَّ حِينٍ وَعَالِمُونَ أَنَّنَا وَنَحْنُ مُسْتَوْطِنُونَ فِي الْجَسَدِ فَنَحْنُ مُتَغَرِّبُونَ عَنِ الرَّبِّ.
\par 7 لأَنَّنَا بِالإِيمَانِ نَسْلُكُ لاَ بِالْعَيَانِ.
\par 8 فَنَثِقُ وَنُسَرُّ بِالأَوْلَى أَنْ نَتَغَرَّبَ عَنِ الْجَسَدِ وَنَسْتَوْطِنَ عِنْدَ الرَّبِّ.
\par 9 لِذَلِكَ نَحْتَرِصُ أَيْضاً مُسْتَوْطِنِينَ كُنَّا أَوْ مُتَغَرِّبِينَ أَنْ نَكُونَ مَرْضِيِّينَ عِنْدَهُ.
\par 10 لأَنَّهُ لاَ بُدَّ أَنَّنَا جَمِيعاً نُظْهَرُ أَمَامَ كُرْسِيِّ الْمَسِيحِ، لِيَنَالَ كُلُّ وَاحِدٍ مَا كَانَ بِالْجَسَدِ بِحَسَبِ مَا صَنَعَ، خَيْراً كَانَ أَمْ شَرّاً.
\par 11 فَإِذْ نَحْنُ عَالِمُونَ مَخَافَةَ الرَّبِّ نُقْنِعُ النَّاسَ. وَأَمَّا اللهُ فَقَدْ صِرْنَا ظَاهِرِينَ لَهُ، وَأَرْجُو أَنَّنَا قَدْ صِرْنَا ظَاهِرِينَ فِي ضَمَائِرِكُمْ أَيْضاً.
\par 12 لأَنَّنَا لَسْنَا نَمْدَحُ أَنْفُسَنَا أَيْضاً لَدَيْكُمْ، بَلْ نُعْطِيكُمْ فُرْصَةً لِلاِفْتِخَارِ مِنْ جِهَتِنَا، لِيَكُونَ لَكُمْ جَوَابٌ عَلَى الَّذِينَ يَفْتَخِرُونَ بِالْوَجْهِ لاَ بِالْقَلْبِ.
\par 13 لأَنَّنَا إِنْ صِرْنَا مُخْتَلِّينَ فَلِلَّهِ، أَوْ كُنَّا عَاقِلِينَ فَلَكُمْ.
\par 14 لأَنَّ مَحَبَّةَ الْمَسِيحِ تَحْصُرُنَا. إِذْ نَحْنُ نَحْسِبُ هَذَا: أَنَّهُ إِنْ كَانَ وَاحِدٌ قَدْ مَاتَ لأَجْلِ الْجَمِيعِ. فَالْجَمِيعُ إِذاً مَاتُوا.
\par 15 وَهُوَ مَاتَ لأَجْلِ الْجَمِيعِ كَيْ يَعِيشَ الأَحْيَاءُ فِيمَا بَعْدُ لاَ لأَنْفُسِهِمْ، بَلْ لِلَّذِي مَاتَ لأَجْلِهِمْ وَقَامَ.
\par 16 إِذاً نَحْنُ مِنَ الآنَ لاَ نَعْرِفُ أَحَداً حَسَبَ الْجَسَدِ. وَإِنْ كُنَّا قَدْ عَرَفْنَا الْمَسِيحَ حَسَبَ الْجَسَدِ، لَكِنِ الآنَ لاَ نَعْرِفُهُ بَعْدُ.
\par 17 إِذاً إِنْ كَانَ أَحَدٌ فِي الْمَسِيحِ فَهُوَ خَلِيقَةٌ جَدِيدَةٌ. الأَشْيَاءُ الْعَتِيقَةُ قَدْ مَضَتْ. هُوَذَا الْكُلُّ قَدْ صَارَ جَدِيداً.
\par 18 وَلَكِنَّ الْكُلَّ مِنَ اللهِ، الَّذِي صَالَحَنَا لِنَفْسِهِ بِيَسُوعَ الْمَسِيحِ، وَأَعْطَانَا خِدْمَةَ الْمُصَالَحَةِ،
\par 19 أَيْ إِنَّ اللهَ كَانَ فِي الْمَسِيحِ مُصَالِحاً الْعَالَمَ لِنَفْسِهِ، غَيْرَ حَاسِبٍ لَهُمْ خَطَايَاهُمْ، وَوَاضِعاً فِينَا كَلِمَةَ الْمُصَالَحَةِ.
\par 20 إِذاً نَسْعَى كَسُفَرَاءَ عَنِ الْمَسِيحِ، كَأَنَّ اللهَ يَعِظُ بِنَا. نَطْلُبُ عَنِ الْمَسِيحِ: تَصَالَحُوا مَعَ اللهِ.
\par 21 لأَنَّهُ جَعَلَ الَّذِي لَمْ يَعْرِفْ خَطِيَّةً، خَطِيَّةً لأَجْلِنَا، لِنَصِيرَ نَحْنُ بِرَّ اللهِ فِيهِ.

\chapter{6}

\par 1 فَإِذْ نَحْنُ عَامِلُونَ مَعَهُ نَطْلُبُ أَنْ لاَ تَقْبَلُوا نِعْمَةَ اللهِ بَاطِلاً.
\par 2 لأَنَّهُ يَقُولُ: «فِي وَقْتٍ مَقْبُولٍ سَمِعْتُكَ، وَفِي يَوْمِ خَلاَصٍ أَعَنْتُكَ». هُوَذَا الآنَ وَقْتٌ مَقْبُولٌ. هُوَذَا الآنَ يَوْمُ خَلاَصٍ.
\par 3 وَلَسْنَا نَجْعَلُ عَثْرَةً فِي شَيْءٍ لِئَلاَّ تُلاَمَ الْخِدْمَةُ.
\par 4 بَلْ فِي كُلِّ شَيْءٍ نُظْهِرُ أَنْفُسَنَا كَخُدَّامِ اللهِ، فِي صَبْرٍ كَثِيرٍ، فِي شَدَائِدَ، فِي ضَرُورَاتٍ، فِي ضِيقَاتٍ،
\par 5 فِي ضَرَبَاتٍ، فِي سُجُونٍ، فِي اضْطِرَابَاتٍ، فِي أَتْعَابٍ، فِي أَسْهَارٍ، فِي أَصْوَامٍ،
\par 6 فِي طَهَارَةٍ، فِي عِلْمٍ، فِي أَنَاةٍ، فِي لُطْفٍ، فِي الرُّوحِ الْقُدُسِ، فِي مَحَبَّةٍ بِلاَ رِيَاءٍ،
\par 7 فِي كَلاَمِ الْحَقِّ، فِي قُوَّةِ اللهِ بِسِلاَحِ الْبِرِّ لِلْيَمِينِ وَلِلْيَسَارِ.
\par 8 بِمَجْدٍ وَهَوَانٍ. بِصِيتٍ رَدِيءٍ وَصِيتٍ حَسَنٍ. كَمُضِلِّينَ وَنَحْنُ صَادِقُونَ.
\par 9 كَمَجْهُولِينَ وَنَحْنُ مَعْرُوفُونَ. كَمَائِتِينَ وَهَا نَحْنُ نَحْيَا. كَمُؤَدَّبِينَ وَنَحْنُ غَيْرُ مَقْتُولِينَ.
\par 10 كَحَزَانَى وَنَحْنُ دَائِماً فَرِحُونَ. كَفُقَرَاءَ وَنَحْنُ نُغْنِي كَثِيرِينَ. كَأَنْ لاَ شَيْءَ لَنَا وَنَحْنُ نَمْلِكُ كُلَّ شَيْءٍ.
\par 11 فَمُنَا مَفْتُوحٌ إِلَيْكُمْ أَيُّهَا الْكُورِنْثِيُّونَ. قَلْبُنَا مُتَّسِعٌ.
\par 12 لَسْتُمْ مُتَضَيِّقِينَ فِينَا بَلْ مُتَضَيِّقِينَ فِي أَحْشَائِكُمْ.
\par 13 فَجَزَاءً لِذَلِكَ أَقُولُ كَمَا لأَوْلاَدِي: كُونُوا أَنْتُمْ أَيْضاً مُتَّسِعِينَ!
\par 14 لاَ تَكُونُوا تَحْتَ نِيرٍ مَعَ غَيْرِ الْمُؤْمِنِينَ، لأَنَّهُ أَيَّةُ خِلْطَةٍ لِلْبِرِّ وَالإِثْمِ؟ وَأَيَّةُ شَرِكَةٍ لِلنُّورِ مَعَ الظُّلْمَةِ؟
\par 15 وَأَيُّ اتِّفَاقٍ لِلْمَسِيحِ مَعَ بَلِيعَالَ؟ وَأَيُّ نَصِيبٍ لِلْمُؤْمِنِ مَعَ غَيْرِ الْمُؤْمِنِ؟
\par 16 وَأَيَّةُ مُوَافَقَةٍ لِهَيْكَلِ اللهِ مَعَ الأَوْثَانِ؟ فَإِنَّكُمْ أَنْتُمْ هَيْكَلُ اللهِ الْحَيِّ، كَمَا قَالَ اللهُ: «إِنِّي سَأَسْكُنُ فِيهِمْ وَأَسِيرُ بَيْنَهُمْ، وَأَكُونُ لَهُمْ إِلَهاً وَهُمْ يَكُونُونَ لِي شَعْباً.
\par 17 لِذَلِكَ اخْرُجُوا مِنْ وَسَطِهِمْ وَاعْتَزِلُوا، يَقُولُ الرَّبُّ. وَلاَ تَمَسُّوا نَجِساً فَأَقْبَلَكُمْ،
\par 18 وَأَكُونَ لَكُمْ أَباً وَأَنْتُمْ تَكُونُونَ لِي بَنِينَ وَبَنَاتٍ» يَقُولُ الرَّبُّ الْقَادِرُ عَلَى كُلِّ شَيْءٍ.

\chapter{7}

\par 1 فَإِذْ لَنَا هَذِهِ الْمَوَاعِيدُ أَيُّهَا الأَحِبَّاءُ لِنُطَهِّرْ ذَوَاتِنَا مِنْ كُلِّ دَنَسِ الْجَسَدِ وَالرُّوحِ، مُكَمِّلِينَ الْقَدَاسَةَ فِي خَوْفِ اللهِ.
\par 2 اِقْبَلُونَا. لَمْ نَظْلِمْ أَحَداً. لَمْ نُفْسِدْ أَحَداً. لَمْ نَطْمَعْ فِي أَحَدٍ.
\par 3 لاَ أَقُولُ هَذَا لأَجْلِ دَيْنُونَةٍ، لأَنِّي قَدْ قُلْتُ سَابِقاً إِنَّكُمْ فِي قُلُوبِنَا لِنَمُوتَ مَعَكُمْ وَنَعِيشَ مَعَكُمْ.
\par 4 لِي ثِقَةٌ كَثِيرَةٌ بِكُمْ. لِي افْتِخَارٌ كَثِيرٌ مِنْ جِهَتِكُمْ. قَدِ امْتَلَأْتُ تَعْزِيَةً وَازْدَدْتُ فَرَحاً جِدّاً فِي جَمِيعِ ضِيقَاتِنَا.
\par 5 لأَنَّنَا لَمَّا أَتَيْنَا إِلَى مَكِدُونِيَّةَ لَمْ يَكُنْ لِجَسَدِنَا شَيْءٌ مِنَ الرَّاحَةِ بَلْ كُنَّا مُكْتَئِبِينَ فِي كُلِّ شَيْءٍ. مِنْ خَارِجٍ خُصُومَاتٌ. مِنْ دَاخِلٍ مَخَاوِفُ.
\par 6 لَكِنَّ اللهَ الَّذِي يُعَزِّي الْمُتَّضِعِينَ عَزَّانَا بِمَجِيءِ تِيطُسَ.
\par 7 وَلَيْسَ بِمَجِيئِهِ فَقَطْ بَلْ أَيْضاً بِالتَّعْزِيَةِ الَّتِي تَعَزَّى بِهَا بِسَبَبِكُمْ وَهُوَ يُخْبِرُنَا بِشَوْقِكُمْ وَنَوْحِكُمْ وَغَيْرَتِكُمْ لأَجْلِي، حَتَّى إِنِّي فَرِحْتُ أَكْثَرَ.
\par 8 لأَنِّي وَإِنْ كُنْتُ قَدْ أَحْزَنْتُكُمْ بِالرِّسَالَةِ لَسْتُ أَنْدَمُ، مَعَ أَنِّي نَدِمْتُ. فَإِنِّي أَرَى أَنَّ تِلْكَ الرِّسَالَةَ أَحْزَنَتْكُمْ وَلَوْ إِلَى سَاعَةٍ.
\par 9 اَلآنَ أَنَا أَفْرَحُ، لاَ لأَنَّكُمْ حَزِنْتُمْ، بَلْ لأَنَّكُمْ حَزِنْتُمْ لِلتَّوْبَةِ. لأَنَّكُمْ حَزِنْتُمْ بِحَسَبِ مَشِيئَةِ اللهِ لِكَيْ لاَ تَتَخَسَّرُوا مِنَّا فِي شَيْءٍ.
\par 10 لأَنَّ الْحُزْنَ الَّذِي بِحَسَبِ مَشِيئَةِ اللهِ يُنْشِئُ تَوْبَةً لِخَلاَصٍ بِلاَ نَدَامَةٍ، وَأَمَّا حُزْنُ الْعَالَمِ فَيُنْشِئُ مَوْتاً.
\par 11 فَإِنَّهُ هُوَذَا حُزْنُكُمْ هَذَا عَيْنُهُ بِحَسَبِ مَشِيئَةِ اللهِ، كَمْ أَنْشَأَ فِيكُمْ مِنَ الاِجْتِهَادِ، بَلْ مِنَ الاِحْتِجَاجِ، بَلْ مِنَ الْغَيْظِ، بَلْ مِنَ الْخَوْفِ، بَلْ مِنَ الشَّوْقِ، بَلْ مِنَ الْغَيْرَةِ، بَلْ مِنَ الاِنْتِقَامِ. فِي كُلِّ شَيْءٍ أَظْهَرْتُمْ أَنْفُسَكُمْ أَنَّكُمْ أَبْرِيَاءُ فِي هَذَا الأَمْرِ.
\par 12 إِذاً وَإِنْ كُنْتُ قَدْ كَتَبْتُ إِلَيْكُمْ، فَلَيْسَ لأَجْلِ الْمُذْنِبِ وَلاَ لأَجْلِ الْمُذْنَبِ إِلَيْهِ، بَلْ لِكَيْ يَظْهَرَ لَكُمْ أَمَامَ اللهِ اجْتِهَادُنَا لأَجْلِكُمْ.
\par 13 مِنْ أَجْلِ هَذَا قَدْ تَعَزَّيْنَا بِتَعْزِيَتِكُمْ. وَلَكِنْ فَرِحْنَا أَكْثَرَ جِدّاً بِسَبَبِ فَرَحِ تِيطُسَ، لأَنَّ رُوحَهُ قَدِ اسْتَرَاحَتْ بِكُمْ جَمِيعاً.
\par 14 فَإِنِّي إِنْ كُنْتُ افْتَخَرْتُ شَيْئاً لَدَيْهِ مِنْ جِهَتِكُمْ لَمْ أُخْجَلْ، بَلْ كَمَا كَلَّمْنَاكُمْ بِكُلِّ شَيْءٍ بِالصِّدْقِ، كَذَلِكَ افْتِخَارُنَا أَيْضاً لَدَى تِيطُسَ صَارَ صَادِقاً.
\par 15 وَأَحْشَاؤُهُ هِيَ نَحْوَكُمْ بِالزِّيَادَةِ، مُتَذَكِّراً طَاعَةَ جَمِيعِكُمْ، كَيْفَ قَبِلْتُمُوهُ بِخَوْفٍ وَرَِعْدَةٍ.
\par 16 أَنَا أَفْرَحُ إِذاً أَنِّي أَثِقُ بِكُمْ فِي كُلِّ شَيْءٍ.

\chapter{8}

\par 1 ثُمَّ نُعَرِّفُكُمْ أَيُّهَا الإِخْوَةُ نِعْمَةَ اللهِ الْمُعْطَاةَ فِي كَنَائِسِ مَكِدُونِيَّةَ،
\par 2 أَنَّهُ فِي اخْتِبَارِ ضِيقَةٍ شَدِيدَةٍ فَاضَ وُفُورُ فَرَحِهِمْ وَفَقْرِهِمِ الْعَمِيقِ لِغِنَى سَخَائِهِمْ،
\par 3 لأَنَّهُمْ أَعْطَوْا حَسَبَ الطَّاقَةِ، أَنَا أَشْهَدُ، وَفَوْقَ الطَّاقَةِ، مِنْ تِلْقَاءِ أَنْفُسِهِمْ،
\par 4 مُلْتَمِسِينَ مِنَّا، بِطِلْبَةٍ كَثِيرَةٍ، أَنْ نَقْبَلَ النِّعْمَةَ وَشَرِكَةَ الْخِدْمَةِ الَّتِي لِلْقِدِّيسِينَ.
\par 5 وَلَيْسَ كَمَا رَجَوْنَا، بَلْ أَعْطَوْا أَنْفُسَهُمْ أَوَّلاً لِلرَّبِّ، وَلَنَا، بِمَشِيئَةِ اللهِ.
\par 6 حَتَّى إِنَّنَا طَلَبْنَا مِنْ تِيطُسَ أَنَّهُ كَمَا سَبَقَ فَابْتَدَأَ، كَذَلِكَ يُتَمِّمُ لَكُمْ هَذِهِ النِّعْمَةَ أَيْضاً.
\par 7 لَكِنْ كَمَا تَزْدَادُونَ فِي كُلِّ شَيْءٍ: فِي الإِيمَانِ وَالْكَلاَمِ وَالْعِلْمِ وَكُلِّ اجْتِهَادٍ وَمَحَبَّتِكُمْ لَنَا، لَيْتَكُمْ تَزْدَادُونَ فِي هَذِهِ النِّعْمَةِ أَيْضاً.
\par 8 لَسْتُ أَقُولُ عَلَى سَبِيلِ الأَمْرِ، بَلْ بِاجْتِهَادِ آخَرِينَ، مُخْتَبِراً إِخْلاَصَ مَحَبَّتِكُمْ أَيْضاً.
\par 9 فَإِنَّكُمْ تَعْرِفُونَ نِعْمَةَ رَبِّنَا يَسُوعَ الْمَسِيحِ، أَنَّهُ مِنْ أَجْلِكُمُ افْتَقَرَ وَهُوَ غَنِيٌّ، لِكَيْ تَسْتَغْنُوا أَنْتُمْ بِفَقْرِهِ.
\par 10 أُعْطِي رَأْياً فِي هَذَا أَيْضاً، لأَنَّ هَذَا يَنْفَعُكُمْ أَنْتُمُ الَّذِينَ سَبَقْتُمْ فَابْتَدَأْتُمْ مُنْذُ الْعَامِ الْمَاضِي، لَيْسَ أَنْ تَفْعَلُوا فَقَطْ بَلْ أَنْ تُرِيدُوا أَيْضاً.
\par 11 وَلَكِنِ الآنَ تَمِّمُوا الْعَمَلَ أَيْضاً، حَتَّى إِنَّهُ كَمَا أَنَّ النَّشَاطَ لِلإِرَادَةِ، كَذَلِكَ يَكُونُ التَّتْمِيمُ أَيْضاً حَسَبَ مَا لَكُمْ.
\par 12 لأَنَّهُ إِنْ كَانَ النَّشَاطُ مَوْجُوداً فَهُوَ مَقْبُولٌ عَلَى حَسَبِ مَا لِلإِنْسَانِ، لاَ عَلَى حَسَبِ مَا لَيْسَ لَهُ.
\par 13 فَإِنَّهُ لَيْسَ لِكَيْ يَكُونَ لِلآخَرِينَ رَاحَةٌ وَلَكُمْ ضِيقٌ،
\par 14 بَلْ بِحَسَبِ الْمُسَاوَاةِ. لِكَيْ تَكُونَ فِي هَذَا الْوَقْتِ فُضَالَتُكُمْ لِإِعْوَازِهِمْ، كَيْ تَصِيرَ فُضَالَتُهُمْ لِإِعْوَازِكُمْ، حَتَّى تَحْصُلَ الْمُسَاوَاةُ.
\par 15 كَمَا هُوَ مَكْتُوبٌ: «الَّذِي جَمَعَ كَثِيراً لَمْ يُفْضِلْ، وَالَّذِي جَمَعَ قَلِيلاً لَمْ يُنْقِصْ».
\par 16 وَلَكِنْ شُكْراً لِلَّهِ الَّذِي جَعَلَ هَذَا الاِجْتِهَادَ عَيْنَهُ لأَجْلِكُمْ فِي قَلْبِ تِيطُسَ،
\par 17 لأَنَّهُ قَبِلَ الطِّلْبَةَ. وَإِذْ كَانَ أَكْثَرَ اجْتِهَاداً مَضَى إِلَيْكُمْ مِنْ تِلْقَاءِ نَفْسِهِ.
\par 18 وَأَرْسَلْنَا مَعَهُ الأَخَ الَّذِي مَدْحُهُ فِي الإِنْجِيلِ فِي جَمِيعِ الْكَنَائِسِ.
\par 19 وَلَيْسَ ذَلِكَ فَقَطْ، بَلْ هُوَ مُنْتَخَبٌ أَيْضاً مِنَ الْكَنَائِسِ رَفِيقاً لَنَا فِي السَّفَرِ، مَعَ هَذِهِ النِّعْمَةِ الْمَخْدُومَةِ مِنَّا لِمَجْدِ ذَاتِ الرَّبِّ الْوَاحِدِ، وَلِنَشَاطِكُمْ.
\par 20 مُتَجَنِّبِينَ هَذَا أَنْ يَلُومَنَا أَحَدٌ فِي جَسَامَةِ هَذِهِ الْمَخْدُومَةِ مِنَّا.
\par 21 مُعْتَنِينَ بِأُمُورٍ حَسَنَةٍ، لَيْسَ قُدَّامَ الرَّبِّ فَقَطْ، بَلْ قُدَّامَ النَّاسِ أَيْضاً.
\par 22 وَأَرْسَلْنَا مَعَهُمَا أَخَانَا، الَّذِي اخْتَبَرْنَا مِرَاراً فِي أُمُورٍ كَثِيرَةٍ أَنَّهُ مُجْتَهِدٌ، وَلَكِنَّهُ الآنَ أَشَدُّ اجْتِهَاداً كَثِيراً بِالثِّقَةِ الْكَثِيرَةِ بِكُمْ.
\par 23 أَمَّا مِنْ جِهَةِ تِيطُسَ فَهُوَ شَرِيكٌ لِي وَعَامِلٌ مَعِي لأَجْلِكُمْ. وَأَمَّا أَخَوَانَا فَهُمَا رَسُولاَ الْكَنَائِسِ، وَمَجْدُ الْمَسِيحِ.
\par 24 فَبَيِّنُوا لَهُمْ، وَقُدَّامَ الْكَنَائِسِ، بَيِّنَةَ مَحَبَّتِكُمْ، وَافْتِخَارِنَا مِنْ جِهَتِكُمْ

\chapter{9}

\par 1 فَإِنَّهُ مِنْ جِهَةِ الْخِدْمَةِ لِلْقِدِّيسِينَ هُوَ فُضُولٌ مِنِّي أَنْ أَكْتُبَ إِلَيْكُمْ.
\par 2 لأَنِّي أَعْلَمُ نَشَاطَكُمُ الَّذِي أَفْتَخِرُ بِهِ مِنْ جِهَتِكُمْ لَدَى الْمَكِدُونِيِّينَ، أَنَّ أَخَائِيَةَ مُسْتَعِدَّةٌ مُنْذُ الْعَامِ الْمَاضِي. وَغَيْرَتُكُمْ قَدْ حَرَّضَتِ الأَكْثَرِينَ.
\par 3 وَلَكِنْ أَرْسَلْتُ الإِخْوَةَ لِئَلاَّ يَتَعَطَّلَ افْتِخَارُنَا مِنْ جِهَتِكُمْ مِنْ هَذَا الْقَبِيلِ، كَيْ تَكُونُوا مُسْتَعِدِّينَ كَمَا قُلْتُ.
\par 4 حَتَّى إِذَا جَاءَ مَعِي مَكِدُونِيُّونَ وَوَجَدُوكُمْ غَيْرَ مُسْتَعِدِّينَ لاَ نُخْجَلُ نَحْنُ - حَتَّى لاَ أَقُولُ أَنْتُمْ - فِي جَسَارَةِ الاِفْتِخَارِ هَذِهِ.
\par 5 فَرَأَيْتُ لاَزِماً أَنْ أَطْلُبَ إِلَى الإِخْوَةِ أَنْ يَسْبِقُوا إِلَيْكُمْ، وَيُهَيِّئُوا قَبْلاً بَرَكَتَكُمُ الَّتِي سَبَقَ التَّخْبِيرُ بِهَا، لِتَكُونَ هِيَ مُعَدَّةً هَكَذَا كَأَنَّهَا بَرَكَةٌ، لاَ كَأَنَّهَا بُخْلٌ.
\par 6 هَذَا وَإِنَّ مَنْ يَزْرَعُ بِالشُّحِّ فَبِالشُّحِّ أَيْضاً يَحْصُدُ، وَمَنْ يَزْرَعُ بِالْبَرَكَاتِ فَبِالْبَرَكَاتِ أَيْضاً يَحْصُدُ.
\par 7 كُلُّ وَاحِدٍ كَمَا يَنْوِي بِقَلْبِهِ، لَيْسَ عَنْ حُزْنٍ أَوِ اضْطِرَارٍ. لأَنَّ الْمُعْطِيَ الْمَسْرُورَ يُحِبُّهُ اللهُ.
\par 8 وَاللَّهُ قَادِرٌ أَنْ يَزِيدَكُمْ كُلَّ نِعْمَةٍ، لِكَيْ تَكُونُوا وَلَكُمْ كُلُّ اكْتِفَاءٍ كُلَّ حِينٍ فِي كُلِّ شَيْءٍ، تَزْدَادُونَ فِي كُلِّ عَمَلٍ صَالِحٍ.
\par 9 كَمَا هُوَ مَكْتُوبٌ: «فَرَّقَ. أَعْطَى الْمَسَاكِينَ. بِرُّهُ يَبْقَى إِلَى الأَبَدِ».
\par 10 وَالَّذِي يُقَدِّمُ بِذَاراً لِلزَّارِعِ وَخُبْزاً لِلأَكْلِ، سَيُقَدِّمُ وَيُكَثِّرُ بِذَارَكُمْ وَيُنْمِي غَلاَّتِ بِرِّكُمْ.
\par 11 مُسْتَغْنِينَ فِي كُلِّ شَيْءٍ لِكُلِّ سَخَاءٍ يُنْشِئُ بِنَا شُكْراً لِلَّهِ.
\par 12 لأَنَّ افْتِعَالَ هَذِهِ الْخِدْمَةِ لَيْسَ يَسُدُّ إِعْوَازَ الْقِدِّيسِينَ فَقَطْ، بَلْ يَزِيدُ بِشُكْرٍ كَثِيرٍ لِلَّهِ
\par 13 إِذْ هُمْ بِاخْتِبَارِ هَذِهِ الْخِدْمَةِ يُمَجِّدُونَ اللهَ عَلَى طَاعَةِ اعْتِرَافِكُمْ لِإِنْجِيلِ الْمَسِيحِ، وَسَخَاءِ التَّوْزِيعِ لَهُمْ وَلِلْجَمِيعِ.
\par 14 وَبِدُعَائِهِمْ لأَجْلِكُمْ، مُشْتَاقِينَ إِلَيْكُمْ مِنْ أَجْلِ نِعْمَةِ اللهِ الْفَائِقَةِ لَدَيْكُمْ.
\par 15 فَشُكْراً لِلَّهِ عَلَى عَطِيَّتِهِ الَّتِي لاَ يُعَبَّرُ عَنْهَا.

\chapter{10}

\par 1 ثُمَّ أَطْلُبُ إِلَيْكُمْ بِوَدَاعَةِ الْمَسِيحِ وَحِلْمِهِ، أَنَا نَفْسِي بُولُسُ الَّذِي فِي الْحَضْرَةِ ذَلِيلٌ بَيْنَكُمْ، وَأَمَّا فِي الْغَيْبَةِ فَمُتَجَاسِرٌ عَلَيْكُمْ.
\par 2 وَلَكِنْ أَطْلُبُ أَنْ لاَ أَتَجَاسَرَ وَأَنَا حَاضِرٌ بِالثِّقَةِ الَّتِي بِهَا أَرَى أَنِّي سَأَجْتَرِئُ عَلَى قَوْمٍ يَحْسِبُونَنَا كَأَنَّنَا نَسْلُكُ حَسَبَ الْجَسَدِ.
\par 3 لأَنَّنَا وَإِنْ كُنَّا نَسْلُكُ فِي الْجَسَدِ، لَسْنَا حَسَبَ الْجَسَدِ نُحَارِبُ.
\par 4 إِذْ أَسْلِحَةُ مُحَارَبَتِنَا لَيْسَتْ جَسَدِيَّةً، بَلْ قَادِرَةٌ بِاللَّهِ عَلَى هَدْمِ حُصُونٍ.
\par 5 هَادِمِينَ ظُنُوناً وَكُلَّ عُلْوٍ يَرْتَفِعُ ضِدَّ مَعْرِفَةِ اللهِ، وَمُسْتَأْسِرِينَ كُلَّ فِكْرٍ إِلَى طَاعَةِ الْمَسِيحِ،
\par 6 وَمُسْتَعِدِّينَ لأَنْ نَنْتَقِمَ عَلَى كُلِّ عِصْيَانٍ، مَتَى كَمِلَتْ طَاعَتُكُمْ.
\par 7 أَتَنْظُرُونَ إِلَى مَا هُوَ حَسَبَ الْحَضْرَةِ؟ إِنْ وَثِقَ أَحَدٌ بِنَفْسِهِ أَنَّهُ لِلْمَسِيحِ، فَلْيَحْسِبْ هَذَا أَيْضاً مِنْ نَفْسِهِ: أَنَّهُ كَمَا هُوَ لِلْمَسِيحِ، كَذَلِكَ نَحْنُ أَيْضاً لِلْمَسِيحِ!
\par 8 فَإِنِّي وَإِنِ افْتَخَرْتُ شَيْئاً أَكْثَرَ بِسُلْطَانِنَا الَّذِي أَعْطَانَا إِيَّاهُ الرَّبُّ لِبُنْيَانِكُمْ لاَ لِهَدْمِكُمْ، لاَ أُخْجَلُ.
\par 9 لِئَلاَّ أَظْهَرَ كَأَنِّي أُخِيفُكُمْ بِالرَّسَائِلِ.
\par 10 لأَنَّهُ يَقُولُ: «الرَّسَائِلُ ثَقِيلَةٌ وَقَوِيَّةٌ، وَأَمَّا حُضُورُ الْجَسَدِ فَضَعِيفٌ وَالْكَلاَمُ حَقِيرٌ».
\par 11 مِثْلُ هَذَا فَلْيَحْسِبْ أَنَّنَا كَمَا نَحْنُ فِي الْكَلاَمِ بِالرَّسَائِلِ وَنَحْنُ غَائِبُونَ، هَكَذَا نَكُونُ أَيْضاً بِالْفِعْلِ وَنَحْنُ حَاضِرُونَ.
\par 12 لأَنَّنَا لاَ نَجْتَرِئُ أَنْ نَعُدَّ أَنْفُسَنَا بَيْنَ قَوْمٍ مِنَ الَّذِينَ يَمْدَحُونَ أَنْفُسَهُمْ، وَلاَ أَنْ نُقَابِلَ أَنْفُسَنَا بِهِمْ. بَلْ هُمْ إِذْ يَقِيسُونَ أَنْفُسَهُمْ عَلَى أَنْفُسِهِمْ، وَيُقَابِلُونَ أَنْفُسَهُمْ بِأَنْفُسِهِمْ، لاَ يَفْهَمُونَ.
\par 13 وَلَكِنْ نَحْنُ لاَ نَفْتَخِرُ إِلَى مَا لاَ يُقَاسُ، بَلْ حَسَبَ قِيَاسِ الْقَانُونِ الَّذِي قَسَمَهُ لَنَا اللهُ، قِيَاساً لِلْبُلُوغِ إِلَيْكُمْ أَيْضاً.
\par 14 لأَنَّنَا لاَ نُمَدِّدُ أَنْفُسَنَا كَأَنَّنَا لَسْنَا نَبْلُغُ إِلَيْكُمْ. إِذْ قَدْ وَصَلْنَا إِلَيْكُمْ أَيْضاً فِي إِنْجِيلِ الْمَسِيحِ.
\par 15 غَيْرَ مُفْتَخِرِينَ إِلَى مَا لاَ يُقَاسُ فِي أَتْعَابِ آخَرِينَ، بَلْ رَاجِينَ إِذَا نَمَا إِيمَانُكُمْ أَنْ نَتَعَظَّمَ بَيْنَكُمْ حَسَبَ قَانُونِنَا بِزِيَادَةٍ،
\par 16 لِنُبَشِّرَ إِلَى مَا وَرَاءَكُمْ. لاَ لِنَفْتَخِرَ بِالأُمُورِ الْمُعَدَّةِ فِي قَانُونِ غَيْرِنَا.
\par 17 وَأَمَّا مَنِ افْتَخَرَ فَلْيَفْتَخِرْ بِالرَّبِّ.
\par 18 لأَنَّهُ لَيْسَ مَنْ مَدَحَ نَفْسَهُ هُوَ الْمُزَكَّى، بَلْ مَنْ يَمْدَحُهُ الرَّبُّ.

\chapter{11}

\par 1 لَيْتَكُمْ تَحْتَمِلُونَ غَبَاوَتِي قَلِيلاً! بَلْ أَنْتُمْ مُحْتَمِلِيَّ.
\par 2 فَإِنِّي أَغَارُ عَلَيْكُمْ غَيْرَةَ اللهِ، لأَنِّي خَطَبْتُكُمْ لِرَجُلٍ وَاحِدٍ، لأُقَدِّمَ عَذْرَاءَ عَفِيفَةً لِلْمَسِيحِ.
\par 3 وَلَكِنَّنِي أَخَافُ أَنَّهُ كَمَا خَدَعَتِ الْحَيَّةُ حَوَّاءَ بِمَكْرِهَا، هَكَذَا تُفْسَدُ أَذْهَانُكُمْ عَنِ الْبَسَاطَةِ الَّتِي فِي الْمَسِيحِ.
\par 4 فَإِنَّهُ إِنْ كَانَ الآتِي يَكْرِزُ بِيَسُوعٍ آخَرَ لَمْ نَكْرِزْ بِهِ، أَوْ كُنْتُمْ تَأْخُذُونَ رُوحاً آخَرَ لَمْ تَأْخُذُوهُ، أَوْ إِنْجِيلاً آخَرَ لَمْ تَقْبَلُوهُ، فَحَسَناً كُنْتُمْ تَحْتَمِلُونَ.
\par 5 لأَنِّي أَحْسِبُ أَنِّي لَمْ أَنْقُصْ شَيْئاً عَنْ فَائِقِي الرُّسُلِ.
\par 6 وَإِنْ كُنْتُ عَامِّيّاً فِي الْكَلاَمِ فَلَسْتُ فِي الْعِلْمِ، بَلْ نَحْنُ فِي كُلِّ شَيْءٍ ظَاهِرُونَ لَكُمْ بَيْنَ الْجَمِيعِ.
\par 7 أَمْ أَخْطَأْتُ خَطِيَّةً إِذْ أَذْلَلْتُ نَفْسِي كَيْ تَرْتَفِعُوا أَنْتُمْ، لأَنِّي بَشَّرْتُكُمْ مَجَّاناً بِإِنْجِيلِ اللهِ؟
\par 8 سَلَبْتُ كَنَائِسَ أُخْرَى آخِذاً أُجْرَةً لأَجْلِ خِدْمَتِكُمْ، وَإِذْ كُنْتُ حَاضِراً عِنْدَكُمْ وَاحْتَجْتُ، لَمْ أُثَقِّلْ عَلَى أَحَدٍ.
\par 9 لأَنَّ احْتِيَاجِي سَدَّهُ الإِخْوَةُ الَّذِينَ أَتَوْا مِنْ مَكِدُونِيَّةَ. وَفِي كُلِّ شَيْءٍ حَفِظْتُ نَفْسِي غَيْرَ ثَقِيلٍ عَلَيْكُمْ، وَسَأَحْفَظُهَا.
\par 10 حَقُّ الْمَسِيحِ فِيَّ. إِنَّ هَذَا الاِفْتِخَارَ لاَ يُسَدُّ عَنِّي فِي أَقَالِيمِ أَخَائِيَةَ.
\par 11 لِمَاذَا؟ أَلأَنِّي لاَ أُحِبُّكُمْ؟ اللَّهُ يَعْلَمُ.
\par 12 وَلَكِنْ مَا أَفْعَلُهُ سَأَفْعَلُهُ لأَقْطَعَ فُرْصَةَ الَّذِينَ يُرِيدُونَ فُرْصَةً كَيْ يُوجَدُوا كَمَا نَحْنُ أَيْضاً فِي مَا يَفْتَخِرُونَ بِهِ.
\par 13 لأَنَّ مِثْلَ هَؤُلاَءِ هُمْ رُسُلٌ كَذَبَةٌ، فَعَلَةٌ مَاكِرُونَ، مُغَيِّرُونَ شَكْلَهُمْ إِلَى شِبْهِ رُسُلِ الْمَسِيحِ.
\par 14 وَلاَ عَجَبَ. لأَنَّ الشَّيْطَانَ نَفْسَهُ يُغَيِّرُ شَكْلَهُ إِلَى شِبْهِ مَلاَكِ نُورٍ!
\par 15 فَلَيْسَ عَظِيماً إِنْ كَانَ خُدَّامُهُ أَيْضاً يُغَيِّرُونَ شَكْلَهُمْ كَخُدَّامٍ لِلْبِرِّ. الَّذِينَ نِهَايَتُهُمْ تَكُونُ حَسَبَ أَعْمَالِهِمْ.
\par 16 أَقُولُ أَيْضاً: لاَ يَظُنَّ أَحَدٌ أَنِّي غَبِيٌّ. وَإِلاَّ فَاقْبَلُونِي وَلَوْ كَغَبِيٍّ، لأَفْتَخِرَ أَنَا أَيْضاً قَلِيلاً.
\par 17 الَّذِي أَتَكَلَّمُ بِهِ لَسْتُ أَتَكَلَّمُ بِهِ بِحَسَبِ الرَّبِّ، بَلْ كَأَنَّهُ فِي غَبَاوَةٍ، فِي جَسَارَةِ الاِفْتِخَارِ هَذِهِ.
\par 18 بِمَا أَنَّ كَثِيرِينَ يَفْتَخِرُونَ حَسَبَ الْجَسَدِ أَفْتَخِرُ أَنَا أَيْضاً.
\par 19 فَإِنَّكُمْ بِسُرُورٍ تَحْتَمِلُونَ الأَغْبِيَاءَ، إِذْ أَنْتُمْ عُقَلاَءُ!
\par 20 لأَنَّكُمْ تَحْتَمِلُونَ إِنْ كَانَ أَحَدٌ يَسْتَعْبِدُكُمْ! إِنْ كَانَ أَحَدٌ يَأْكُلُكُمْ! إِنْ كَانَ أَحَدٌ يَأْخُذُكُمْ! إِنْ كَانَ أَحَدٌ يَرْتَفِعُ! إِنْ كَانَ أَحَدٌ يَضْرِبُكُمْ عَلَى وُجُوهِكُمْ!
\par 21 عَلَى سَبِيلِ الْهَوَانِ أَقُولُ كَيْفَ أَنَّنَا كُنَّا ضُعَفَاءَ. وَلَكِنَّ الَّذِي يَجْتَرِئُ فِيهِ أَحَدٌ، أَقُولُ فِي غَبَاوَةٍ: أَنَا أَيْضاً أَجْتَرِئُ فِيهِ.
\par 22 أَهُمْ عِبْرَانِيُّونَ؟ فَأَنَا أَيْضاً. أَهُمْ إِسْرَائِيلِيُّونَ؟ فَأَنَا أَيْضاً. أَهُمْ نَسْلُ إِبْرَاهِيمَ؟ فَأَنَا أَيْضاً.
\par 23 أَهُمْ خُدَّامُ الْمَسِيحِ؟ أَقُولُ كَمُخْتَلِّ الْعَقْلِ: فَأَنَا أَفْضَلُ. فِي الأَتْعَابِ أَكْثَرُ. فِي الضَّرَبَاتِ أَوْفَرُ. فِي السُّجُونِ أَكْثَرُ. فِي الْمِيتَاتِ مِرَاراً كَثِيرَةً.
\par 24 مِنَ الْيَهُودِ خَمْسَ مَرَّاتٍ قَبِلْتُ أَرْبَعِينَ جَلْدَةً إِلاَّ وَاحِدَةً.
\par 25 ثَلاَثَ مَرَّاتٍ ضُرِبْتُ بِالْعِصِيِّ. مَرَّةً رُجِمْتُ. ثَلاَثَ مَرَّاتٍ انْكَسَرَتْ بِيَ السَّفِينَةُ. لَيْلاً وَنَهَاراً قَضَيْتُ فِي الْعُمْقِ.
\par 26 بِأَسْفَارٍ مِرَاراً كَثِيرَةً. بِأَخْطَارِ سُيُولٍ. بِأَخْطَارِ لُصُوصٍ. بِأَخْطَارٍ مِنْ جِنْسِي. بِأَخْطَارٍ مِنَ الأُمَمِ. بِأَخْطَارٍ فِي الْمَدِينَةِ. بِأَخْطَارٍ فِي الْبَرِّيَّةِ. بِأَخْطَارٍ فِي الْبَحْرِ. بِأَخْطَارٍ مِنْ إِخْوَةٍ كَذَبَةٍ.
\par 27 فِي تَعَبٍ وَكَدٍّ. فِي أَسْهَارٍ مِرَاراً كَثِيرَةً. فِي جُوعٍ وَعَطَشٍ. فِي أَصْوَامٍ مِرَاراً كَثِيرَةً. فِي بَرْدٍ وَعُرْيٍ.
\par 28 عَدَا مَا هُوَ دُونَ ذَلِكَ: التَّرَاكُمُ عَلَيَّ كُلَّ يَوْمٍ. الاِهْتِمَامُ بِجَمِيعِ الْكَنَائِسِ.
\par 29 مَنْ يَضْعُفُ وَأَنَا لاَ أَضْعُفُ؟ مَنْ يَعْثُرُ وَأَنَا لاَ أَلْتَهِبُ؟
\par 30 إِنْ كَانَ يَجِبُ الاِفْتِخَارُ، فَسَأَفْتَخِرُ بِأُمُورِ ضُعْفِي.
\par 31 اَللَّهُ أَبُو رَبِّنَا يَسُوعَ الْمَسِيحِ، الَّذِي هُوَ مُبَارَكٌ إِلَى الأَبَدِ، يَعْلَمُ أَنِّي لَسْتُ أَكْذِبُ.
\par 32 فِي دِمَشْقَ وَالِي الْحَارِثِ الْمَلِكِ كَانَ يَحْرُسُ مدِينَةَ الدِّمَشْقِيِّينَ يُرِيدُ أَنْ يُمْسِكَنِي،
\par 33 فَتَدَلَّيْتُ مِنْ طَاقَةٍ فِي زَنْبِيلٍ مِنَ السُّورِ، وَنَجَوْتُ مِنْ يَدَيْهِ.

\chapter{12}

\par 1 إِنَّهُ لاَ يُوافِقُنِي أَنْ أَفْتَخِرَ. فَإِنِّي آتِي إِلَى مَنَاظِرِ الرَّبِّ وَإِعْلاَنَاتِهِ.
\par 2 أَعْرِفُ إِنْسَاناً فِي الْمَسِيحِ قَبْلَ أَرْبَعَ عَشْرَةَ سَنَةً. أَفِي الْجَسَدِ؟ لَسْتُ أَعْلَمُ، أَمْ خَارِجَ الْجَسَدِ؟ لَسْتُ أَعْلَمُ. اللهُ يَعْلَمُ. اخْتُطِفَ هَذَا إِلَى السَّمَاءِ الثَّالِثَةِ.
\par 3 وَأَعْرِفُ هَذَا الإِنْسَانَ. أَفِي الْجَسَدِ أَمْ خَارِجَ الْجَسَدِ؟ لَسْتُ أَعْلَمُ. اللهُ يَعْلَمُ.
\par 4 أَنَّهُ اخْتُطِفَ إِلَى الْفِرْدَوْسِ، وَسَمِعَ كَلِمَاتٍ لاَ يُنْطَقُ بِهَا، وَلاَ يَسُوغُ لِإِنْسَانٍ أَنْ يَتَكَلَّمَ بِهَا.
\par 5 مِنْ جِهَةِ هَذَا أَفْتَخِرُ. وَلَكِنْ مِنْ جِهَةِ نَفْسِي لاَ أَفْتَخِرُ إِلاَّ بِضَعَفَاتِي.
\par 6 فَإِنِّي إِنْ أَرَدْتُ أَنْ أَفْتَخِرَ لاَ أَكُونُ غَبِيّاً، لأَنِّي أَقُولُ الْحَقَّ. وَلَكِنِّي أَتَحَاشَى لِئَلاَّ يَظُنَّ أَحَدٌ مِنْ جِهَتِي فَوْقَ مَا يَرَانِي أَوْ يَسْمَعُ مِنِّي.
\par 7 وَلِئَلاَّ أَرْتَفِعَ بِفَرْطِ الإِعْلاَنَاتِ، أُعْطِيتُ شَوْكَةً فِي الْجَسَدِ، مَلاَكَ الشَّيْطَانِ، لِيَلْطِمَنِي لِئَلاَّ أَرْتَفِعَ.
\par 8 مِنْ جِهَةِ هَذَا تَضَرَّعْتُ إِلَى الرَّبِّ ثَلاَثَ مَرَّاتٍ أَنْ يُفَارِقَنِي.
\par 9 فَقَالَ لِي: «تَكْفِيكَ نِعْمَتِي، لأَنَّ قُوَّتِي فِي الضُّعْفِ تُكْمَلُ». فَبِكُلِّ سُرُورٍ أَفْتَخِرُ بِالْحَرِيِّ فِي ضَعَفَاتِي، لِكَيْ تَحِلَّ عَلَيَّ قُوَّةُ الْمَسِيحِ.
\par 10 لِذَلِكَ أُسَرُّ بِالضَّعَفَاتِ وَالشَّتَائِمِ وَالضَّرُورَاتِ وَالاِضْطِهَادَاتِ وَالضِّيقَاتِ لأَجْلِ الْمَسِيحِ. لأَنِّي حِينَمَا أَنَا ضَعِيفٌ فَحِينَئِذٍ أَنَا قَوِيٌّ.
\par 11 قَدْ صِرْتُ غَبِيّاً وَأَنَا أَفْتَخِرُ. أَنْتُمْ أَلْزَمْتُمُونِي! لأَنَّهُ كَانَ يَنْبَغِي أَنْ أُمْدَحَ مِنْكُمْ، إِذْ لَمْ أَنْقُصْ شَيْئاً عَنْ فَائِقِي الرُّسُلِ، وَإِنْ كُنْتُ لَسْتُ شَيْئاً.
\par 12 إِنَّ عَلاَمَاتِ الرَّسُولِ صُنِعَتْ بَيْنَكُمْ فِي كُلِّ صَبْرٍ، بِآيَاتٍ وَعَجَائِبَ وَقُوَّاتٍ.
\par 13 لأَنَّهُ مَا هُوَ الَّذِي نَقَصْتُمْ عَنْ سَائِرِ الْكَنَائِسِ، إِلاَّ أَنِّي أَنَا لَمْ أُثَقِّلْ عَلَيْكُمْ؟ سَامِحُونِي بِهَذَا الظُّلْمِ.
\par 14 هُوَذَا الْمَرَّةُ الثَّالِثَةُ أَنَا مُسْتَعِدٌّ أَنْ آتِيَ إِلَيْكُمْ وَلاَ أُثَقِّلَ عَلَيْكُمْ. لأَنِّي لَسْتُ أَطْلُبُ مَا هُوَ لَكُمْ بَلْ إِيَّاكُمْ. لأَنَّهُ لاَ يَنْبَغِي أَنَّ الأَوْلاَدَ يَذْخَرُونَ لِلْوَالِدِينَ بَلِ الْوَالِدُونَ لِلأَوْلاَدِ.
\par 15 وَأَمَّا أَنَا فَبِكُلِّ سُرُورٍ أُنْفِقُ وَأُنْفَقُ لأَجْلِ أَنْفُسِكُمْ، وَإِنْ كُنْتُ كُلَّمَا أُحِبُّكُمْ أَكْثَرَ أُحَبُّ أَقَلَّ!
\par 16 فَلْيَكُنْ. أَنَا لَمْ أُثَقِّلْ عَلَيْكُمْ. لَكِنْ إِذْ كُنْتُ مُحْتَالاً أَخَذْتُكُمْ بِمَكْرٍ!
\par 17 هَلْ طَمِعْتُ فِيكُمْ بِأَحَدٍ مِنَ الَّذِينَ أَرْسَلْتُهُمْ إِلَيْكُمْ؟
\par 18 طَلَبْتُ إِلَى تِيطُسَ وَأَرْسَلْتُ مَعَهُ الأَخَ. هَلْ طَمِعَ فِيكُمْ تِيطُسُ؟ أَمَا سَلَكْنَا بِذَاتِ الرُّوحِ الْوَاحِدِ؟ أَمَا بِذَاتِ الْخَطَوَاتِ الْوَاحِدَةِ؟
\par 19 أَتَظُنُّونَ أَيْضاً أَنَّنَا نَحْتَجُّ لَكُمْ؟ أَمَامَ اللهِ فِي الْمَسِيحِ نَتَكَلَّمُ. وَلَكِنَّ الْكُلَّ أَيُّهَا الأَحِبَّاءُ لأَجْلِ بُنْيَانِكُمْ.
\par 20 لأَنِّي أَخَافُ إِذَا جِئْتُ أَنْ لاَ أَجِدَكُمْ كَمَا أُرِيدُ، وَأُوجَدَ مِنْكُمْ كَمَا لاَ تُرِيدُونَ. أَنْ تُوجَدَ خُصُومَاتٌ وَمُحَاسَدَاتٌ وَسَخَطَاتٌ وَتَحَزُبَاتٌ وَمَذَمَّاتٌ وَنَمِيمَاتٌ وَتَكَبُّرَاتٌ وَتَشْوِيشَاتٌ -
\par 21 أَنْ يُذِلَّنِي إِلَهِي عِنْدَكُمْ، إِذَا جِئْتُ أَيْضاً وَأَنُوحُ عَلَى كَثِيرِينَ مِنَ الَّذِينَ أَخْطَأُوا مِنْ قَبْلُ وَلَمْ يَتُوبُوا عَنِ النَّجَاسَةِ وَالزِّنَا وَالْعَهَارَةِ الَّتِي فَعَلُوهَا.

\chapter{13}

\par 1 هَذِهِ الْمَرَّةُ الثَّالِثَةُ آتِي إِلَيْكُمْ. عَلَى فَمِ شَاهِدَيْنِ وَثَلاَثَةٍ تَقُومُ كُلُّ كَلِمَةٍ.
\par 2 قَدْ سَبَقْتُ فَقُلْتُ، وَأَسْبِقُ فَأَقُولُ كَمَا وَأَنَا حَاضِرٌ الْمَرَّةَ الثَّانِيَةَ، وَأَنَا غَائِبٌ الآنَ، أَكْتُبُ لِلَّذِينَ أَخْطَأُوا مِنْ قَبْلُ، وَلِجَمِيعِ الْبَاقِينَ: أَنِّي إِذَا جِئْتُ أَيْضاً لاَ أُشْفِقُ.
\par 3 إِذْ أَنْتُمْ تَطْلُبُونَ بُرْهَانَ الْمَسِيحِ الْمُتَكَلِّمِ فِيَّ، الَّذِي لَيْسَ ضَعِيفاً لَكُمْ بَلْ قَوِيٌّ فِيكُمْ.
\par 4 لأَنَّهُ وَإِنْ كَانَ قَدْ صُلِبَ مِنْ ضُعْفٍ لَكِنَّهُ حَيٌّ بِقُوَّةِ اللهِ. فَنَحْنُ أَيْضاً ضُعَفَاءُ فِيهِ، لَكِنَّنَا سَنَحْيَا مَعَهُ بِقُوَّةِ اللهِ مِنْ جِهَتِكُمْ.
\par 5 جَرِّبُوا أَنْفُسَكُمْ، هَلْ أَنْتُمْ فِي الإِيمَانِ؟ امْتَحِنُوا أَنْفُسَكُمْ. أَمْ لَسْتُمْ تَعْرِفُونَ أَنْفُسَكُمْ أَنَّ يَسُوعَ الْمَسِيحَ هُوَ فِيكُمْ، إِنْ لَمْ تَكُونُوا مَرْفُوضِينَ؟
\par 6 لَكِنَّنِي أَرْجُو أَنَّكُمْ سَتَعْرِفُونَ أَنَّنَا نَحْنُ لَسْنَا مَرْفُوضِينَ.
\par 7 وَأُصَلِّي إِلَى اللهِ أَنَّكُمْ لاَ تَعْمَلُونَ شَيْئاً رَدِيّاً، لَيْسَ لِكَيْ نَظْهَرَ نَحْنُ مُزَكَّيْنَ، بَلْ لِكَيْ تَصْنَعُوا أَنْتُمْ حَسَناً، وَنَكُونَ نَحْنُ كَأَنَّنَا مَرْفُوضُونَ.
\par 8 لأَنَّنَا لاَ نَسْتَطِيعُ شَيْئاً ضِدَّ الْحَقِّ بَلْ لأَجْلِ الْحَقِّ.
\par 9 لأَنَّنَا نَفْرَحُ حِينَمَا نَكُونُ نَحْنُ ضُعَفَاءَ وَأَنْتُمْ تَكُونُونَ أَقْوِيَاءَ. وَهَذَا أَيْضاً نَطْلُبُهُ كَمَا لَكُمْ.
\par 10 لِذَلِكَ أَكْتُبُ بِهَذَا وَأَنَا غَائِبٌ، لِكَيْ لاَ أَسْتَعْمِلَ جَزْماً وَأَنَا حَاضِرٌ، حَسَبَ السُّلْطَانِ الَّذِي أَعْطَانِي إِيَّاهُ الرَّبُّ لِلْبُنْيَانِ لاَ لِلْهَدْمِ.
\par 11 أَخِيراً أَيُّهَا الإِخْوَةُ افْرَحُوا. اكْمَلُوا. تَعَزَّوْا. اهْتَمُّوا اهْتِمَاماً وَاحِداً. عِيشُوا بِالسَّلاَمِ، وَإِلَهُ الْمَحَبَّةِ وَالسَّلاَمِ سَيَكُونُ مَعَكُمْ.
\par 12 سَلِّمُوا بَعْضُكُمْ عَلَى بَعْضٍ بِقُبْلَةٍ مُقَدَّسَةٍ.
\par 13 يُسَلِّمُ عَلَيْكُمْ جَمِيعُ الْقِدِّيسِينَ.
\par 14 نِعْمَةُ رَبِّنَا يَسُوعَ الْمَسِيحِ، وَمَحَبَّةُ اللهِ، وَشَرِكَةُ الرُّوحِ الْقُدُسِ مَعَ جَمِيعِكُمْ. آمِينَ.


\end{document}