\begin{document}

\title{2 بطرس}


\chapter{1}

\par 1 سِمْعَانُ بُطْرُسُ عَبْدُ يَسُوعَ الْمَسِيحِ وَرَسُولُهُ، إِلَى الَّذِينَ نَالُوا مَعَنَا إِيمَاناً ثَمِيناً مُسَاوِياً لَنَا، بِبِرِّ إِلَهِنَا وَالْمُخَلِّصِ يَسُوعَ الْمَسِيحِ.
\par 2 لِتَكْثُرْ لَكُمُ النِّعْمَةُ وَالسَّلاَمُ بِمَعْرِفَةِ اللَّهِ وَيَسُوعَ رَبِّنَا.
\par 3 كَمَا أَنَّ قُدْرَتَهُ الإِلَهِيَّةَ قَدْ وَهَبَتْ لَنَا كُلَّ مَا هُوَ لِلْحَيَاةِ وَالتَّقْوَى، بِمَعْرِفَةِ الَّذِي دَعَانَا بِالْمَجْدِ وَالْفَضِيلَةِ،
\par 4 اللَّذَيْنِ بِهِمَا قَدْ وَهَبَ لَنَا الْمَوَاعِيدَ الْعُظْمَى وَالثَّمِينَةَ لِكَيْ تَصِيرُوا بِهَا شُرَكَاءَ الطَّبِيعَةِ الإِلَهِيَّةِ، هَارِبِينَ مِنَ الْفَسَادِ الَّذِي فِي الْعَالَمِ بِالشَّهْوَةِ.
\par 5 وَلِهَذَا عَيْنِهِ وَأَنْتُمْ بَاذِلُونَ كُلَّ اجْتِهَادٍ قَدِّمُوا فِي إِيمَانِكُمْ فَضِيلَةً، وَفِي الْفَضِيلَةِ مَعْرِفَةً،
\par 6 وَفِي الْمَعْرِفَةِ تَعَفُّفاً، وَفِي التَّعَفُّفِ صَبْراً، وَفِي الصَّبْرِ تَقْوَى،
\par 7 وَفِي التَّقْوَى مَوَدَّةً أَخَوِيَّةً، وَفِي الْمَوَدَّةِ الأَخَوِيَّةِ مَحَبَّةً.
\par 8 لأَنَّ هَذِهِ إِذَا كَانَتْ فِيكُمْ وَكَثُرَتْ، تُصَيِّرُكُمْ لاَ مُتَكَاسِلِينَ وَلاَ غَيْرَ مُثْمِرِينَ لِمَعْرِفَةِ رَبِّنَا يَسُوعَ الْمَسِيحِ.
\par 9 لأَنَّ الَّذِي لَيْسَ عِنْدَهُ هَذِهِ هُوَ أَعْمَى قَصِيرُ الْبَصَرِ، قَدْ نَسِيَ تَطْهِيرَ خَطَايَاهُ السَّالِفَةِ.
\par 10 لِذَلِكَ بِالأَكْثَرِ اجْتَهِدُوا أَيُّهَا الإِخْوَةُ أَنْ تَجْعَلُوا دَعْوَتَكُمْ وَاخْتِيَارَكُمْ ثَابِتَيْنِ. لأَنَّكُمْ إِذَا فَعَلْتُمْ ذَلِكَ لَنْ تَزِلُّوا أَبَداً.
\par 11 لأَنَّهُ هَكَذَا يُقَدَّمُ لَكُمْ بِسِعَةٍ دُخُولٌ إِلَى مَلَكُوتِ رَبِّنَا وَمُخَلِّصِنَا يَسُوعَ الْمَسِيحِ الأَبَدِيِّ.
\par 12 لِذَلِكَ لاَ أُهْمِلُ أَنْ أُذَكِّرَكُمْ دَائِماً بِهَذِهِ الأُمُورِ، وَإِنْ كُنْتُمْ عَالِمِينَ وَمُثَبَّتِينَ فِي الْحَقِّ الْحَاضِرِ.
\par 13 وَلَكِنِّي أَحْسِبُهُ حَقّاً مَا دُمْتُ فِي هَذَا الْمَسْكَنِ أَنْ أُنْهِضَكُمْ بِالتَّذْكِرَةِ،
\par 14 عَالِماً أَنَّ خَلْعَ مَسْكَنِي قَرِيبٌ كَمَا أَعْلَنَ لِي رَبُّنَا يَسُوعُ الْمَسِيحُ أَيْضاً.
\par 15 فَأَجْتَهِدُ أَيْضاً أَنْ تَكُونُوا بَعْدَ خُرُوجِي تَتَذَكَّرُونَ كُلَّ حِينٍ بِهَذِهِ الأُمُورِ.
\par 16 لأَنَّنَا لَمْ نَتْبَعْ خُرَافَاتٍ مُصَنَّعَةً إِذْ عَرَّفْنَاكُمْ بِقُوَّةِ رَبِّنَا يَسُوعَ الْمَسِيحِ وَمَجِيئِهِ، بَلْ قَدْ كُنَّا مُعَايِنِينَ عَظَمَتَهُ.
\par 17 لأَنَّهُ أَخَذَ مِنَ اللَّهِ الآبِ كَرَامَةً وَمَجْداً، إِذْ أَقْبَلَ عَلَيْهِ صَوْتٌ كَهَذَا مِنَ الْمَجْدِ الأَسْنَى: «هَذَا هُوَ ابْنِي الْحَبِيبُ الَّذِي أَنَا سُرِرْتُ بِهِ».
\par 18 وَنَحْنُ سَمِعْنَا هَذَا الصَّوْتَ مُقْبِلاً مِنَ السَّمَاءِ إِذْ كُنَّا مَعَهُ فِي الْجَبَلِ الْمُقَدَّسِ.
\par 19 وَعِنْدَنَا الْكَلِمَةُ النَّبَوِيَّةُ، وَهِيَ أَثْبَتُ، الَّتِي تَفْعَلُونَ حَسَناً إِنِ انْتَبَهْتُمْ إِلَيْهَا كَمَا إِلَى سِرَاجٍ مُنِيرٍ فِي مَوْضِعٍ مُظْلِمٍ، إِلَى أَنْ يَنْفَجِرَ النَّهَارُ وَيَطْلَعَ كَوْكَبُ الصُّبْحِ فِي قُلُوبِكُمْ،
\par 20 عَالِمِينَ هَذَا أَوَّلاً: أَنَّ كُلَّ نُبُوَّةِ الْكِتَابِ لَيْسَتْ مِنْ تَفْسِيرٍ خَاصٍّ،
\par 21 لأَنَّهُ لَمْ تَأْتِ نُبُوَّةٌ قَطُّ بِمَشِيئَةِ إِنْسَانٍ، بَلْ تَكَلَّمَ أُنَاسُ اللَّهِ الْقِدِّيسُونَ مَسُوقِينَ مِنَ الرُّوحِ الْقُدُسِ.

\chapter{2}

\par 1 وَلَكِنْ كَانَ أَيْضاً فِي الشَّعْبِ أَنْبِيَاءُ كَذَبَةٌ، كَمَا سَيَكُونُ فِيكُمْ أَيْضاً مُعَلِّمُونَ كَذَبَةٌ، الَّذِينَ يَدُسُّونَ بِدَعَ هَلاَكٍ. وَإِذْ هُمْ يُنْكِرُونَ الرَّبَّ الَّذِي اشْتَرَاهُمْ، يَجْلِبُونَ عَلَى أَنْفُسِهِمْ هَلاَكاً سَرِيعاً.
\par 2 وَسَيَتْبَعُ كَثِيرُونَ تَهْلُكَاتِهِمْ. الَّذِينَ بِسَبَبِهِمْ يُجَدَّفُ عَلَى طَرِيقِ الْحَقِّ.
\par 3 وَهُمْ فِي الطَّمَعِ يَتَّجِرُونَ بِكُمْ بِأَقْوَالٍ مُصَنَّعَةٍ، الَّذِينَ دَيْنُونَتُهُمْ مُنْذُ الْقَدِيمِ لاَ تَتَوَانَى وَهَلاَكُهُمْ لاَ يَنْعَسُ.
\par 4 لأَنَّهُ إِنْ كَانَ اللَّهُ لَمْ يُشْفِقْ عَلَى مَلاَئِكَةٍ قَدْ أَخْطَأُوا، بَلْ فِي سَلاَسِلِ الظَّلاَمِ طَرَحَهُمْ فِي جَهَنَّمَ، وَسَلَّمَهُمْ مَحْرُوسِينَ لِلْقَضَاءِ،
\par 5 وَلَمْ يُشْفِقْ عَلَى الْعَالَمِ الْقَدِيمِ، بَلْ إِنَّمَا حَفِظَ نُوحاً ثَامِناً كَارِزاً لِلْبِرِّ إِذْ جَلَبَ طُوفَاناً عَلَى عَالَمِ الْفُجَّارِ.
\par 6 وَإِذْ رَمَّدَ مَدِينَتَيْ سَدُومَ وَعَمُورَةَ حَكَمَ عَلَيْهِمَا بِالاِنْقِلاَبِ، وَاضِعاً عِبْرَةً لِلْعَتِيدِينَ أَنْ يَفْجُرُوا،
\par 7 وَأَنْقَذَ لُوطاً الْبَارَّ مَغْلُوباً مِنْ سِيرَةِ الأَرْدِيَاءِ فِي الدَّعَارَةِ.
\par 8 إِذْ كَانَ الْبَارُّ بِالنَّظَرِ وَالسَّمْعِ وَهُوَ سَاكِنٌ بَيْنَهُمْ يُعَذِّبُ يَوْماً فَيَوْماً نَفْسَهُ الْبَارَّةَ بِالأَفْعَالِ الأَثِيمَةِ.
\par 9 يَعْلَمُ الرَّبُّ أَنْ يُنْقِذَ الأَتْقِيَاءَ مِنَ التَّجْرِبَةِ وَيَحْفَظَ الأَثَمَةَ إِلَى يَوْمِ الدِّينِ مُعَاقَبِينَ،
\par 10 وَلاَ سِيَّمَا الَّذِينَ يَذْهَبُونَ وَرَاءَ الْجَسَدِ فِي شَهْوَةِ النَّجَاسَةِ، وَيَسْتَهِينُونَ بِالسِّيَادَةِ. جَسُورُونَ، مُعْجِبُونَ بِأَنْفُسِهِمْ، لاَ يَرْتَعِبُونَ أَنْ يَفْتَرُوا عَلَى ذَوِي الأَمْجَادِ
\par 11 حَيْثُ مَلاَئِكَةٌ، وَهُمْ أَعْظَمُ قُوَّةً وَقُدْرَةً - لاَ يُقَدِّمُونَ عَلَيْهِمْ لَدَى الرَّبِّ حُكْمَ افْتِرَاءٍ.
\par 12 أَمَّا هَؤُلاَءِ فَكَحَيَوَانَاتٍ غَيْرِ نَاطِقَةٍ، طَبِيعِيَّةٍ، مَوْلُودَةٍ لِلصَّيْدِ وَالْهَلاَكِ، يَفْتَرُونَ عَلَى مَا يَجْهَلُونَ، فَسَيَهْلِكُونَ فِي فَسَادِهِمْ
\par 13 آخِذِينَ أُجْرَةَ الإِثْمِ. الَّذِينَ يَحْسِبُونَ تَنَعُّمَ يَوْمٍ لَذَّةً. أَدْنَاسٌ وَعُيُوبٌ، يَتَنَعَّمُونَ فِي غُرُورِهِمْ صَانِعِينَ وَلاَئِمَ مَعَكُمْ.
\par 14 لَهُمْ عُيُونٌ مَمْلُوَّةٌ فِسْقاً لاَ تَكُفُّ عَنِ الْخَطِيَّةِ، خَادِعُونَ النُّفُوسَ غَيْرَ الثَّابِتَةِ. لَهُمْ قَلْبٌ مُتَدَرِّبٌ فِي الطَّمَعِ. أَوْلاَدُ اللَّعْنَةِ.
\par 15 قَدْ تَرَكُوا الطَّرِيقَ الْمُسْتَقِيمَ، فَضَلُّوا تَابِعِينَ طَرِيقَ بَلْعَامَ بْنِ بَصُورَ الَّذِي أَحَبَّ أُجْرَةَ الإِثْمِ.
\par 16 وَلَكِنَّهُ حَصَلَ عَلَى تَوْبِيخِ تَعَدِّيهِ، إِذْ مَنَعَ حَمَاقَةَ النَّبِيِّ حِمَارٌ أَعْجَمُ نَاطِقاً بِصَوْتِ إِنْسَانٍ.
\par 17 هَؤُلاَءِ هُمْ آبَارٌ بِلاَ مَاءٍ، غُيُومٌ يَسُوقُهَا النَّوْءُ. الَّذِينَ قَدْ حُفِظَ لَهُمْ قَتَامُ الظَّلاَمِ إِلَى الأَبَدِ.
\par 18 لأَنَّهُمْ إِذْ يَنْطِقُونَ بِعَظَائِمِ الْبُطْلِ، يَخْدَعُونَ بِشَهَوَاتِ الْجَسَدِ فِي الدَّعَارَةِ مَنْ هَرَبَ قَلِيلاً مِنَ الَّذِينَ يَسِيرُونَ فِي الضَّلاَلِ،
\par 19 وَاعِدِينَ إِيَّاهُمْ بِالْحُرِّيَّةِ، وَهُمْ أَنْفُسُهُمْ عَبِيدُ الْفَسَادِ. لأَنَّ مَا انْغَلَبَ مِنْهُ أَحَدٌ فَهُوَ لَهُ مُسْتَعْبَدٌ أَيْضاً!
\par 20 لأَنَّهُ إِذَا كَانُوا بَعْدَمَا هَرَبُوا مِنْ نَجَاسَاتِ الْعَالَمِ، بِمَعْرِفَةِ الرَّبِّ وَالْمُخَلِّصِ يَسُوعَ الْمَسِيحِ، يَرْتَبِكُونَ أَيْضاً فِيهَا، فَيَنْغَلِبُونَ، فَقَدْ صَارَتْ لَهُمُ الأَوَاخِرُ أَشَرَّ مِنَ الأَوَائِلِ.
\par 21 لأَنَّهُ كَانَ خَيْراً لَهُمْ لَوْ لَمْ يَعْرِفُوا طَرِيقَ الْبِرِّ، مِنْ أَنَّهُمْ بَعْدَمَا عَرَفُوا يَرْتَدُّونَ عَنِ الْوَصِيَّةِ الْمُقَدَّسَةِ الْمُسَلَّمَةِ لَهُمْ.
\par 22 قَدْ أَصَابَهُمْ مَا فِي الْمَثَلِ الصَّادِقِ: «كَلْبٌ قَدْ عَادَ إِلَى قَيْئِهِ، وَخِنْزِيرَةٌ مُغْتَسِلَةٌ إِلَى مَرَاغَةِ الْحَمْأَةِ».

\chapter{3}

\par 1 هَذِهِ أَكْتُبُهَا الآنَ إِلَيْكُمْ رِسَالَةً ثَانِيَةً أَيُّهَا الأَحِبَّاءُ، فِيهِمَا أُنْهِضُ بِالتَّذْكِرَةِ ذِهْنَكُمُ النَّقِيَّ،
\par 2 لِتَذْكُرُوا الأَقْوَالَ الَّتِي قَالَهَا سَابِقاً الأَنْبِيَاءُ الْقِدِّيسُونَ، وَوَصِيَّتَنَا نَحْنُ الرُّسُلَ، وَصِيَّةَ الرَّبِّ وَالْمُخَلِّصِ.
\par 3 عَالِمِينَ هَذَا أَوَّلاً: أَنَّهُ سَيَأْتِي فِي آخِرِ الأَيَّامِ قَوْمٌ مُسْتَهْزِئُونَ، سَالِكِينَ بِحَسَبِ شَهَوَاتِ أَنْفُسِهِمْ،
\par 4 وَقَائِلِينَ: «أَيْنَ هُوَ مَوْعِدُ مَجِيئِهِ؟ لأَنَّهُ مِنْ حِينَ رَقَدَ الآبَاءُ كُلُّ شَيْءٍ بَاقٍ هَكَذَا مِنْ بَدْءِ الْخَلِيقَةِ».
\par 5 لأَنَّ هَذَا يَخْفَى عَلَيْهِمْ بِإِرَادَتِهِمْ: أَنَّ السَّمَاوَاتِ كَانَتْ مُنْذُ الْقَدِيمِ وَالأَرْضَ بِكَلِمَةِ اللَّهِ قَائِمَةً مِنَ الْمَاءِ وَبِالْمَاءِ،
\par 6 اللَّوَاتِي بِهِنَّ الْعَالَمُ الْكَائِنُ حِينَئِذٍ فَاضَ عَلَيْهِ الْمَاءُ فَهَلَكَ.
\par 7 وَأَمَّا السَّمَاوَاتُ وَالأَرْضُ الْكَائِنَةُ الآنَ فَهِيَ مَخْزُونَةٌ بِتِلْكَ الْكَلِمَةِ عَيْنِهَا، مَحْفُوظَةً لِلنَّارِ إِلَى يَوْمِ الدِّينِ وَهَلاَكِ النَّاسِ الْفُجَّارِ.
\par 8 وَلَكِنْ لاَ يَخْفَ عَلَيْكُمْ هَذَا الشَّيْءُ الْوَاحِدُ أَيُّهَا الأَحِبَّاءُ، أَنَّ يَوْماً وَاحِداً عِنْدَ الرَّبِّ كَأَلْفِ سَنَةٍ، وَأَلْفَ سَنَةٍ كَيَوْمٍ وَاحِدٍ.
\par 9 لاَ يَتَبَاطَأُ الرَّبُّ عَنْ وَعْدِهِ كَمَا يَحْسِبُ قَوْمٌ التَّبَاطُؤَ، لَكِنَّهُ يَتَأَنَّى عَلَيْنَا، وَهُوَ لاَ يَشَاءُ أَنْ يَهْلِكَ أُنَاسٌ، بَلْ أَنْ يُقْبِلَ الْجَمِيعُ إِلَى التَّوْبَةِ.
\par 10 وَلَكِنْ سَيَأْتِي كَلِصٍّ فِي اللَّيْلِ، يَوْمُ الرَّبِّ، الَّذِي فِيهِ تَزُولُ السَّمَاوَاتُ بِضَجِيجٍ، وَتَنْحَلُّ الْعَنَاصِرُ مُحْتَرِقَةً، وَتَحْتَرِقُ الأَرْضُ وَالْمَصْنُوعَاتُ الَّتِي فِيهَا.
\par 11 فَبِمَا أَنَّ هَذِهِ كُلَّهَا تَنْحَلُّ، أَيَّ أُنَاسٍ يَجِبُ أَنْ تَكُونُوا أَنْتُمْ فِي سِيرَةٍ مُقَدَّسَةٍ وَتَقْوَى؟
\par 12 مُنْتَظِرِينَ وَطَالِبِينَ سُرْعَةَ مَجِيءِ يَوْمِ الرَّبِّ، الَّذِي بِهِ تَنْحَلُّ السَّمَاوَاتُ مُلْتَهِبَةً، وَالْعَنَاصِرُ مُحْتَرِقَةً تَذُوبُ.
\par 13 وَلَكِنَّنَا بِحَسَبِ وَعْدِهِ نَنْتَظِرُ سَمَاوَاتٍ جَدِيدَةً وَأَرْضاً جَدِيدَةً، يَسْكُنُ فِيهَا الْبِرُّ.
\par 14 لِذَلِكَ أَيُّهَا الأَحِبَّاءُ، إِذْ أَنْتُمْ مُنْتَظِرُونَ هَذِهِ، اجْتَهِدُوا لِتُوجَدُوا عِنْدَهُ بِلاَ دَنَسٍ وَلاَ عَيْبٍ، فِي سَلاَمٍ.
\par 15 وَاحْسِبُوا أَنَاةَ رَبِّنَا خَلاَصاً، كَمَا كَتَبَ إِلَيْكُمْ أَخُونَا الْحَبِيبُ بُولُسُ أَيْضاً بِحَسَبِ الْحِكْمَةِ الْمُعْطَاةِ لَهُ،
\par 16 كَمَا فِي الرَّسَائِلِ كُلِّهَا أَيْضاً، مُتَكَلِّماً فِيهَا عَنْ هَذِهِ الأُمُورِ، الَّتِي فِيهَا أَشْيَاءُ عَسِرَةُ الْفَهْمِ، يُحَرِّفُهَا غَيْرُ الْعُلَمَاءِ وَغَيْرُ الثَّابِتِينَ كَبَاقِي الْكُتُبِ أَيْضاً، لِهَلاَكِ أَنْفُسِهِمْ.
\par 17 فَأَنْتُمْ أَيُّهَا الأَحِبَّاءُ إِذْ قَدْ سَبَقْتُمْ فَعَرَفْتُمُ، احْتَرِسُوا مِنْ أَنْ تَنْقَادُوا بِضَلاَلِ الأَرْدِيَاءِ فَتَسْقُطُوا مِنْ ثَبَاتِكُمْ.
\par 18 وَلَكِنِ انْمُوا فِي النِّعْمَةِ وَفِي مَعْرِفَةِ رَبِّنَا وَمُخَلِّصِنَا يَسُوعَ الْمَسِيحِ. لَهُ الْمَجْدُ الآنَ وَإِلَى يَوْمِ الدَّهْرِ. آمِينَ.

\end{document}