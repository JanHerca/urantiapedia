\begin{document}

\title{1 عزرا}


\chapter{1}

\par 1 وأقام يوشيا عيد الفصح في أورشليم لسيده، وأصعد الفصح في اليوم الرابع عشر من الشهر الأول.
\par 2 وأقام الكهنة حسب طقوسهم اليومية، لابسين ثياباً طويلة في هيكل الرب.
\par 3 وكلم اللاويين خدام إسرائيل القديسين أن يقدسوا أنفسهم للرب لكي يضعوا تابوت قدس الرب في البيت الذي بناه الملك سليمان بن داود.
\par 4 وقال لا تحملون التابوت على أكتافك بعد. فالآن اعبد الرب إلهك واخدم شعبه إسرائيل وأعد نفسك حسب عائلاتك وأجدادك.
\par 5 كما أمر داود ملك إسرائيل، وحسب بهاء سليمان ابنه، ووقوفكم في الهيكل حسب رتب عشائركم اللاويين، الذين يخدمون بين إخوتكم بني إسرائيل،
\par 6 قدموا الفصح حسب الترتيب، وأعدوا الذبائح لإخوتكم، واحفظوا الفصح حسب وصية الرب التي أعطيت لموسى.
\par 7 وأعطى يوشيا للشعب الذي وجد هناك ثلاثين ألف خروف وجدي وثلاثة آلاف عجل. هذه أعطيت من عطية الملك كما وعد للشعب والكهنة واللاويين.
\par 8 وأعطى حلقيا وزكريا وسيلسوس رؤساء الهيكل للكهنة للفصح ألفين وستمائة شاة وثلاثمائة عجل.
\par 9 وكان يكنيا وشمعيا ونثنائيل أخوه وأسابيا وأوكيئيل ويورام رؤساء الألوف أعطوا للاويين في الفصح خمسة آلاف شاة وسبع مئة عجل.
\par 10 ولما تم ذلك وقف الكهنة واللاويون عند الفطير حسب ترتيب جميل جدا حسب عشائرهم.
\par 11 وحسب مراتب الآباء أمام الشعب ليقدموا للرب كما هو مكتوب في سفر موسى. وهكذا فعلوا في الصباح.
\par 12 "وخبزوا الفصح بالنار كما هو معلوم، وأما الذبائح فكانوا يحرقونها في قدور نحاس ومقالي برائحة طيبة،
\par 13 وأوقفوها أمام جميع الشعب، وبعد ذلك هيأوا لأنفسهم وللكهنة إخوتهم بني هارون.
\par 14 وكان الكهنة يذبحون الشحم إلى الليل، فأعدّ اللاويون لأنفسهم، والكهنة إخوتهم بنو هارون.
\par 15 وكان المغنون القديسون أيضاً بنو آساف في ترتيبهم حسب ترتيب داود، وهم آساف وزكريا ويدوثون الذي كان من حاشية الملك.
\par 16 وكان البوابون على كل باب، ولم يكن يحل لأحد أن يخرج من خدمته المعتادة، لأن إخوتهم اللاويين كانوا يعدون لهم.
\par 17 فكانت تتم أعمال ذبائح الرب في ذلك اليوم لإقامة الفصح.
\par 18 ويقدمون ذبائح على مذبح الرب حسب أمر الملك يوشيا.
\par 19 فصنع بنو إسرائيل الذين كانوا هناك الفصح في ذلك الوقت وعيد الخبز الحلو سبعة أيام.
\par 20 ولم يكن مثل هذا الفصح موجوداً في إسرائيل منذ أيام النبي صموئيل.
\par 21 ولم يعمل جميع ملوك إسرائيل فصحا مثل فصح يوشيا والكهنة واللاويون واليهود مع كل إسرائيل الذين وجدوا ساكنين في أورشليم.
\par 22 وفي السنة الثامنة عشرة من ملك يوشيا تم عمل هذا الفصح.
\par 23 وكانت أعمال يوشيا مستقيمة أمام ربه بقلب ممتلئ تقوى.
\par 24 وأما الأمور التي حدثت في زمانه فقد كتبت في الأيام الأولى عن الذين أخطأوا وعملوا الشر ضد الرب أكثر من جميع الشعوب والممالك وكيف أحزنوه جدا حتى ارتفعت كلمات الرب على إسرائيل.
\par 25 وبعد كل هذه الأعمال التي عملها يوشيا، حدث أن فرعون ملك مصر جاء ليحارب كركميش عند الفرات، فخرج يوشيا للقتال.
\par 26 فأرسل إليه ملك مصر قائلا: ما لي ولك يا ملك اليهودية؟
\par 27 أنا لست مرسلاً من قبل الرب الإله ضدك، لأن حربي على الفرات، والآن الرب معي، والرب معي يُعجِّلني. اذهبوا عني ولا تكونوا على الرب.
\par 28 ولكن يوشيا لم يرد مركبته عنه، بل بدأ يحاربه، غير مبال بكلام النبي إرميا الذي تكلم به الرب بفم الرب:
\par 29 ولكن انضم إليه في الحرب في سهل مجدو، فجاء الرؤساء لمحاربة الملك يوشيا.
\par 30 فقال الملك لعبيده: أخرجوني من المعركة، فإني ضعيف جدًا. فأخذه عبيده في الحال إلى خارج المعركة.
\par 31 ثم ركب مركبته الثانية، وأتى بها إلى أورشليم ومات، ودفن في قبر أبيه.
\par 32 "وكانوا في كل يهوذا ينوحون على يوشيا، حتى أن إرميا النبي رثى يوشيا، وكان رؤساء الرجال والنساء يرثونه إلى هذا اليوم، فأُعطي هذا فريضة ليعمل بها دائما في كل أمة إسرائيل."
\par 33 "هذه الأمور مكتوبة في سفر أخبار ملوك يهوذا، وكل الأعمال التي عملها يوشيا، ومجده، وفهمه في شريعة الرب، والأمور التي عملها من قبل، والأمور التي تليت الآن، مسجلة في سفر ملوك إسرائيل ويهوذا."
\par 34 فأخذ الشعب يهوحاز بن يوشيا وملكوه مكان يوشيا أبيه وكان عمره ثلاثا وعشرين سنة.
\par 35 وملك في اليهودية وأورشليم ثلاثة أشهر، ثم عزله ملك مصر عن الملك في أورشليم.
\par 36 ففرض على الأرض ضريبة قدرها مائة وزنة من الفضة ووزنة واحدة من الذهب.
\par 37 وجعل ملك مصر الملك يهوياقيم أخاه ملكاً على اليهودية وأورشليم.
\par 38 فأوثق يهوياقيم والأشراف، وأما زاراكيس أخاه فأمسكه وأخرجه من مصر.
\par 39 وكان يهوياقيم ابن خمس وعشرين سنة حين ملك على أرض اليهودية وأورشليم، وعمل الشر في عيني الرب.
\par 40 لذلك صعد عليه نبوخذناصر ملك بابل وقيده بسلسلة من نحاس وجاء به إلى بابل.
\par 41 وأخذ نبوخذنصر أيضاً من آنية الرب المقدسة، وحملها ووضعها في هيكله في بابل.
\par 42 وأما الأشياء التي سجلت عنه وعن نجاسته وفجوره فهي مكتوبة في أخبار الملوك.
\par 43 وملك يهوياقيم ابنه عوضا عنه، وملك وهو ابن ثماني عشرة سنة.
\par 44 وملك ثلاثة أشهر وعشرة أيام في أورشليم، وعمل الشر في عيني الرب.
\par 45 وبعد سنة أرسل نبوخذنصر فأحضره إلى بابل مع أواني الرب المقدسة.
\par 46 وجعل صدقيا ملكا على اليهودية وأورشليم وهو ابن إحدى وعشرين سنة، وملك إحدى عشرة سنة.
\par 47 وعمل الشر أيضاً في عيني الرب ولم يبال بالكلام الذي قيل له عن يد النبي إرميا من فم الرب.
\par 48 وبعد أن أقسمه الملك نبوخذناصر باسم الرب، حلف وتمرد وصلب عنقه وقلبه، وتعدى على شرائع الرب إله إسرائيل.
\par 49 وحكام الشعب والكهنة فعلوا أموراً كثيرة ضد الشريعة، ونشروا كل نجاسات كل الأمم، ونجّسوا هيكل الرب الذي قدس في أورشليم.
\par 50 ولكن إله آبائهم أرسل بيد رسوله ليدعوهم، لأنه أشفق عليهم وعلى مسكنه أيضاً.
\par 51 "ولكنهم سخروا من رسله، وانظروا، عندما تكلم الرب إليهم، سخروا من أنبيائه."
\par 52 حتى أنه غضب على شعبه بسبب كفرهم الشديد، فأمر ملوك الكلدانيين أن يصعدوا إليهم.
\par 53 الذي قتل شبانهم بالسيف حتى في محيط هيكل قدسهم، ولم يشفق على شاب ولا عذراء ولا شيخ ولا طفل بينهم، لأنه سلم الجميع في أيديهم.
\par 54 فأخذوا جميع آنية قدس الرب الكبيرة والصغيرة وآنية تابوت الله وخزائن الملك، وحملوها إلى بابل.
\par 55 وأما بيت الرب فأحرقوه وهدموا أسوار أورشليم وأحرقوا أبراجها بالنار.
\par 56 وأما مجدها فلم ينته حتى أبادها كلها، والشعب الذي لم يقتل بالسيف ساقه إلى بابل.
\par 57 فصاروا له ولأولاده عبيداً إلى أن ملك الفرس لكي يتمموا كلام الرب الذي تكلم به إرميا.
\par 58 إلى أن تستوفي الأرض سبوتها تستريح كل أيام خرابها إلى تمام سبعين سنة.

\chapter{2}

\par 1 في السنة الأولى لكورش ملك فارس، لكي يتم كلام الرب الذي تكلم به بفم إرميا.
\par 2 فأثار الرب روح كورش ملك فارس، فأطلق نداءً في كل مملكته، وكتب أيضاً:
\par 3 قائلا هكذا قال كورش ملك الفرس: إن رب إسرائيل الرب العلي جعلني ملكا على كل الأرض،
\par 4 وأمرني أن أبني له بيتاً في أورشليم في اليهودية.
\par 5 فإن كان أحد منكم من شعبه، فليكن الرب سيده معه، وليصعد إلى أورشليم التي في اليهودية، ويبني بيت رب إسرائيل، لأنه هو الرب الساكن في أورشليم.
\par 6 "فمن كان ساكنًا في تلك الأماكن، فليساعده جيرانه بالذهب والفضة،
\par 7 مع الهدايا، والخيول، والماشية، وسائر الأشياء التي نُذرت لهيكل الرب في أورشليم.
\par 8 "فقام رؤساء آباء اليهودية وبنيامين والكهنة واللاويون وكل من حرك الرب قلبه للصعود وبناء بيت للرب في أورشليم،
\par 9 والذين سكنوا حولهم، وساعدوهم في كل شيء بالفضة والذهب، وبالخيول والماشية، وبهدايا مجانية كثيرة من عدد كبير ممن تحركت عقولهم لذلك.
\par 10 وأخرج الملك كورش أيضًا الأواني المقدسة التي أخذها نبوخذنصر من أورشليم وأقامها في هيكل الأصنام.
\par 11 "ولما أخرجهم كورش ملك الفرس، سلمهم إلى ميثريداتس أمين خزنته.
\par 12 فأسلموهم إلى سنباصر والي اليهودية.
\par 13 وهذا كان عددها: ألف كأس من ذهب، وألف مجمرة من فضة، وتسعة وعشرون مجمرة من فضة، وثلاثون مجمرة من ذهب، وألفان وأربعمائة وعشرة مجامر من فضة، وألفان وأربعمائة وعشرة مجامر من ذهب.
\par 14 فكانت جميع آنية الذهب والفضة التي أخذت خمسة آلاف وأربعمائة وتسعة وستون.
\par 15 وقد أعاد سنباصر هؤلاء مع أهل السبي من بابل إلى أورشليم.
\par 16 "ولكن في أيام أرتحشستا ملك الفرس، كتب بيليموس وميثريداتس وتابليوس وراثومس وبيلتيثموس وسيميليوس الكاتب، مع آخرين كانوا في الخدمة معهم، وكانوا يسكنون السامرة وأماكن أخرى، هذه الرسائل إليه ضد الذين يسكنون في اليهودية وأورشليم:
\par 17 إلى الملك أرتحشستا سيدنا، عبيدك، وراثومس الكاتب، وسيميليوس الكاتب، وبقية مجلسهما، والقضاة الذين في بقاع سورية وفينيقيا.
\par 18 فليكن معلوما الآن لدى السيد الملك أن اليهود الذين صعدوا من عندك إلينا، قد أتوا إلى أورشليم المدينة المتمردة الشريرة، وهم يبنون الأسواق، ويرممون أسوارها، ويضعون أساس الهيكل.
\par 19 والآن إذا بنيت هذه المدينة وأسوارها، فلن يرفضوا فقط إعطاء الجزية، بل سيثورون أيضًا على الملوك.
\par 20 وبما أن الأمور المتعلقة بالهيكل أصبحت الآن تحت السيطرة، فإننا نعتقد أنه من المناسب عدم إهمال مثل هذا الأمر،
\par 21 "ولكن لكي نتحدث إلى سيدنا الملك، لكي يتم البحث عنه في كتب آبائك إذا رغبت في ذلك:"
\par 22 "وتجد في أخبار الأيام ما هو مكتوب في هذا الشأن، وتفهم أن تلك المدينة كانت متمردة، مقلقة الملوك والمدن.
\par 23 وأن اليهود كانوا متمردين، وأثاروا الحروب هناك دائمًا، ولذلك أصبحت هذه المدينة خرابة.
\par 24 ولذلك نعلن لك الآن يا سيد الملك أنه إذا بنيت هذه المدينة مرة أخرى وأقيمت أسوارها من جديد فلن يكون لك من الآن فصاعدا أي طريق إلى بلاد الشام وفينيقيا.
\par 25 "ثم كتب الملك أيضًا إلى راثوموس كاتب القصص، وإلى بيلتيثموس، وإلى سميليوس الكاتب، وإلى بقية القائمين على الأمر، والساكنين في السامرة وسورية وفينيقية، على هذا النحو."
\par 26 وقد قرأت الرسالة التي أرسلتموها إليّ، فأمرت بالبحث جيداً، فوجدت أن تلك المدينة كانت منذ البدء تعمل ضد الملوك.
\par 27 وكان الرجال فيها منخرطين في التمرد والحرب، وكان في أورشليم ملوك أشداء وأشداء، وكانوا يحكمون ويأخذون الجزية في بقاع سورية وفينيقية.
\par 28 والآن أمرت بمنع هؤلاء الرجال من بناء المدينة، والحرص على عدم عمل أي شيء آخر فيها.
\par 29 وأن هؤلاء العمال الأشرار لا يستمرون في إزعاج الملوك،
\par 30 "ثم لما قرئت رسائل الملك أرتحشستا، سار راثومس وسميلوس الكاتب وبقية الذين كانوا في عهدهما مسرعين نحو أورشليم بجيش من الفرسان وحشد من الشعب في صفوف القتال، وابتدأوا يمنعون البنائين، فتوقف بناء الهيكل في أورشليم إلى السنة الثانية من ملك داريوس ملك الفرس."

\chapter{3}

\par 1 "ولما ملك داريوس صنع وليمة عظيمة لجميع رعيته ولكل بيته ولكل رؤساء مادي وفارس،
\par 2 وإلى جميع الولاة والقواد والنواب الذين تحت قيادته من الهند إلى كوش، مائة وسبعة وعشرين إقليماً.
\par 3 وبعد أن أكلوا وشربوا وشبعوا ورجعوا إلى بيوتهم، دخل الملك داريوس إلى حجرة نومه ونام، ثم استيقظ بعد قليل.
\par 4 ثم تحدث ثلاثة شبان من الحرس الذين يحرسون جسد الملك بعضهم مع بعض.
\par 5 "فليتكلم كل واحد منا بحكمة: من يغلب، والذي يبدو حكمه أكثر حكمة من الآخرين، فسوف يمنحه الملك داريوس هدايا عظيمة، وأشياء عظيمة كرمز للنصر.
\par 6 كأنه يلبس الأرجوان ويشرب الذهب وينام على الذهب ومركبة بلجام من ذهب وطوق من كتان وسلسلة حول عنقه.
\par 7 ويجلس بجانب داريوس لحكمته، ويُدعى داريوس ابن عمه.
\par 8 وبعد ذلك كتب كل واحد حكمه وختمه ووضعه تحت وسادة الملك داريوس.
\par 9 وقال أنه عندما يقوم الملك، سوف يعطونه البعض الكتابات؛ ومن جانبه سوف يحكم الملك وثلاثة من أمراء فارس بأن حكمه هو الأكثر حكمة، فسوف يُعطى النصر، كما تم تعيينه.
\par 10 كتب الأول: الخمر هو الأقوى.
\par 11 وكتب الثاني: الملك هو الأقوى.
\par 12 وكتب الثالث: النساء أقوى، ولكن فوق كل شيء الحقيقة تسلب النصر.
\par 13 "ولما قام الملك أخذوا كتبهم ودفعوها إليه، فقرأها."
\par 14 ثم أرسل ودعا جميع رؤساء فارس ومادي والولاة والقادة والولاة والرؤساء.
\par 15 وأجلسه على كرسي القضاء الملكي، وقُرأت الكتب أمامهم.
\par 16 فقال: ادع الشباب ليُبَيِّنوا أحكامهم، فدُعوا ودخلوا.
\par 17 فقال لهم: أخبرونا برأيكم في الكتب. فبدأ الأول الذي تكلم عن قوة الخمر.
\par 18 فقال هكذا أيها الرجال ما أعظم الخمر حتى إنها تضل كل من يشربها.
\par 19 "إنه يجعل عقل الملك والطفل اليتيم واحدًا؛ عقل العبد والحر، عقل الفقير والغني.
\par 20 ويحول كل فكر إلى فرح وسرور، حتى لا يتذكر الإنسان حزنًا ولا دينًا.
\par 21 ويجعل كل قلب غنيًا، حتى أن الإنسان لا يتذكر ملكًا ولا حاكمًا، ويجعله يتكلم بكل شيء بالمواهب.
\par 22 "وعندما يكونون في كؤوسهم، ينسون محبتهم للأصدقاء والإخوة، وبعد قليل يستلون سيوفهم.
\par 23 ولكن عندما يخرجون من الخمر لا يتذكرون ما فعلوا.
\par 24 أيها الرجال، أليس الخمر أقوى من أن يُجبر على فعل هذا؟ فلما قال ذلك سكت.

\chapter{4}

\par 1 ثم بدأ الثاني الذي تحدث عن قوة الملك يقول:
\par 2 أيها الرجال، أليس الرجال يتميزون بالقوة عندما يسيطرون على البحر والبر وكل ما فيهما؟
\par 3 ولكن الملك أعظم، لأنه سيد كل هذه الأشياء، وله سلطان عليها، وكل ما يأمرهم به يفعلونه.
\par 4 إذا أمرهم أن يصنعوا حربًا بعضهم ضد بعض، يفعلونها. إذا أرسلهم لمحاربة الأعداء، يذهبون ويهدمون الجبال والأسوار والأبراج.
\par 5 إنهم يقتلون ويُقتلون، ولا يتعدون على أمر الملك. إذا حصلوا على النصر، فإنهم يقدمون كل شيء إلى الملك، بما في ذلك الغنيمة، مثل كل شيء آخر.
\par 6 وكذلك بالنسبة لأولئك الذين ليسوا جنودًا، وليس لهم علاقة بالحروب، ولكنهم يعملون في الزراعة، عندما يحصدون ما زرعوه، يقدمونه إلى الملك، ويجبرون بعضهم البعض على دفع الجزية للملك.
\par 7 ومع ذلك فهو رجل واحد فقط: إذا أمر بالقتل، يقتلون، وإذا أمر بالإبقاء، يبقون، وإذا أمر بالاستبقاء، يبقون.
\par 8 إذا أمر بالضرب ضربوا، وإذا أمر بالتخريب خربوا، وإذا أمر بالبناء بنوا،
\par 9 إذا أمر بالقطع قطعوا، وإذا أمر بالغرس غرسوا.
\par 10 فكل شعبه وجيوشه يطيعونه، وهو يضطجع ويأكل ويشرب ويستريح.
\par 11 وهم يحرسونه من حوله، فلا يستطيع أحد أن يخرج ليعمل ما يريد، ولا يعصيه في أي شيء.
\par 12 يا أيها الرجال، كيف لا يكون الملك أعظم، وهو مطاع في مثل هذا؟ فأمسك لسانه.
\par 13 ثم بدأ الثالث الذي تكلم عن النساء والحقيقة (كان هذا زربابل) يتكلم.
\par 14 أيها الرجال، ليس الملك العظيم، ولا جمهور الرجال، ولا الخمر، هو الذي يفوق. فمن هو الذي يحكمهم أو له السيادة عليهم؟ أليسوا نساء؟
\par 15 لقد حملت النساء الملك وكل الشعوب التي تحمل الحكم في البحر والبر.
\par 16 ومنهم جاءوا، وربوا غارسي الكروم التي منها يأتي الخمر.
\par 17 وهذه تصنع أيضاً ملابس للرجال، وهذه تجلب المجد للرجال، وبدون النساء لا يمكن أن يكون الرجال.
\par 18 نعم، وإذا جمع الرجال ذهباً وفضة أو أي شيء صالح آخر، أفلا يحبون المرأة الجميلة المنظر والجمال؟
\par 19 "وإذا تركوا كل تلك الأشياء، أفلا يفتحون أفواههم، وحتى بأفواه مفتوحة يثبتون أعينهم عليها؛ أليس كل الرجال يرغبون فيها أكثر من رغبتهم في الفضة أو الذهب، أو أي شيء طيب على الإطلاق؟
\par 20 يترك الرجل أباه الذي رباه ووطنه ويلتصق بامرأته.
\par 21 لا يلتزم بقضاء حياته مع زوجته ولا يتذكر أباه ولا أمه ولا وطنه.
\par 22 وبهذا أيضاً يجب أن تعلموا أن النساء متسلطات عليكم: ألا تتعبون وتتعبون وتعطون وتأتون بكل شيء للمرأة؟
\par 23 نعم، يأخذ الرجل سيفه ويمضي في طريقه ليسلب ويسرق، ويسافر على البحر والأنهار.
\par 24 وينظر إلى الأسد ويذهب في الظلمة، وعندما يسرق وينهب ويسلب، فإنه يحضره إلى محبته.
\par 25 لذلك يحب الرجل امرأته أكثر من أبيه أو أمه.
\par 26 نعم، هناك كثيرون قد فقدوا عقولهم من أجل النساء، وأصبحوا خدماً لهن.
\par 27 وكثيرون أيضاً هلكوا وأخطأوا وأخطأوا من أجل النساء.
\par 28 والآن ألا تصدقونني؟ أليس الملك عظيما في قدرته؟ ألا تخشى كل المناطق أن تمسه؟
\par 29 ولكني رأيته وأفامية سرية الملك، ابنة برتاكوس العظيم، جالسين عن يمين الملك،
\par 30 فأخذت التاج عن رأس الملك ووضعته على رأسها، وضربت الملك بيدها اليسرى.
\par 31 ومع كل هذا كان الملك فاغر الفم وينظر إليها بفم مفتوح: إذا ضحكت عليه، ضحك هو أيضًا: ولكن إذا شعرت بأي استياء منه، كان الملك مستعدًا للتملق، حتى تتصالح معه مرة أخرى.
\par 32 يا أيها الرجال، كيف لا تكون النساء قويات، وهن يفعلن ذلك؟
\par 33 ثم نظر الملك والأمراء بعضهم إلى بعض، فابتدأ يتكلم بالحق.
\par 34 يا أيها الرجال، أليست النساء قويات؟ عظيمة هي الأرض، عالية هي السماء، سريعة هي الشمس في مجراها، لأنها تدور حول السماء، وتعود إلى مكانها في يوم واحد.
\par 35 أليس عظيماً هو الذي يصنع هذه الأشياء؟ لذلك فإن الحقيقة عظيمة وأقوى من كل الأشياء.
\par 36 كل الأرض تصرخ على الحقيقة، والسماء تباركها: كل الأعمال تهتز وترتجف بسببها، ولا يوجد معها أي شيء غير صالح.
\par 37 الخمر شريرة، والملك شرير، والنساء شريرات، وكل أبناء البشر أشرار، وهذه هي كل أعمالهم الشريرة، وليس فيهم حق، وفي إثمهم أيضاً يهلكون.
\par 38 وأما الحقيقة فهي باقية وقوية دائماً، وتعيش وتنتصر إلى الأبد.
\par 39 ليس عندها قبول للأشخاص أو المكافآت؛ بل تفعل الأشياء العادلة، وتتجنب كل الأشياء غير العادلة والشريرة؛ وجميع الناس يفعلون الخير مثل أعمالها.
\par 40 لا ظلم في حكمها، وهي القوة والملك والسلطان والجلال إلى الأبد. تبارك إله الحق.
\par 41 ثم سكت. فصاح الناس جميعاً وقالوا: "عظيم هو الحق، وعظيم فوق كل شيء".
\par 42 فقال له الملك: اطلب ما تريد أكثر مما هو مكتوب في الكتابة فنعطيك لأنك وجدت أحكم، وتجلس بجانبي وتدعى ابن عمي.
\par 43 ثم قال للملك اذكر نذرك الذي نذرته لبناء أورشليم يوم جئت إلى ملكك.
\par 44 "وأن يرسل جميع الآنية التي أخذت من أورشليم التي خصصها كورش حين نذر أن يهلك بابل، ويرسلها أيضا إلى هناك."
\par 45 وأنت أيضاً قد نذرت أن تبني الهيكل الذي أحرقه الأدوميون حين خربت اليهودية على يد الكلدانيين.
\par 46 "والآن يا سيدي الملك، هذا هو ما أطلبه منك، وهذا هو الكرم الأميري الصادر منك: أريد إذن أن تفي بالنذر الذي نذرته بفمك لملك السماء.
\par 47 فقام الملك داريوس وقبله وكتب له رسائل إلى جميع الخزنة والولاة والرؤساء والولاة لكي يوصلوه سالماً هو وجميع الذين يصعدون معه لبناء أورشليم.
\par 48 وكتب أيضاً رسائل إلى الولاة الذين في بقاع سورية وفينيقية والذين في لبنان لكي يأتوا بخشب الأرز من لبنان إلى أورشليم، وأن يبنوا المدينة معه.
\par 49 "وكتب أيضًا إلى جميع اليهود الذين خرجوا من مملكته إلى اليهودية، فيما يتعلق بحريتهم، أنه لا يجوز لأي ضابط، ولا حاكم، ولا ملازم، ولا أمين صندوق، أن يدخل أبوابهم عنوة؛
\par 50 وأن كل البلاد التي في أيديهم تكون حرة بلا جزية، وأن الأدوميين يسلمون قرى اليهود التي كانوا في أيديهم حينئذ.
\par 51 نعم، أنه ينبغي أن يعطى عشرين وزنة في كل سنة لبناء الهيكل، إلى الوقت الذي يتم بناؤه.
\par 52 وعشر وزنات أخرى في كل سنة لإيقاد المحرقات على المذبح كل يوم، كما كان لهم الوصية بتقديم سبعة عشر وزنة.
\par 53 وأن جميع الذين ذهبوا من بابل لبناء المدينة يجب أن تكون لهم الحرية الكاملة، هم وذريتهم، وجميع الكهنة الذين ذهبوا.
\par 54 وكتب أيضًا عن الواجبات والثياب التي يخدمون بها؛
\par 55 وكذلك تكليف اللاويين أن يعطوا إلى اليوم الذي يكمل فيه البيت وتبنى أورشليم.
\par 56 وأمر بإعطاء كل من يعمل في المدينة معاشات وأجور.
\par 57 وأرسل أيضا جميع الآنية من بابل التي خصصها كورش، وكل ما أمر به كورش أمر أيضا أن يتم وأرسله إلى أورشليم.
\par 58 "ولما خرج هذا الشاب رفع وجهه نحو السماء نحو أورشليم وسبح ملك السماء،
\par 59 وقال من عندك النصرة ومنك الحكمة ولك المجد وأنا عبدك.
\par 60 طوبى لك يا من أعطيتني الحكمة، لأني لك أشكرك يا رب آبائنا.
\par 61 فأخذ الكتب وخرج وجاء إلى بابل وأخبر جميع إخوته.
\par 62 فسبحوا إله آبائهم لأنه أعطاهم الحرية والتحرر.
\par 63 ليصعدوا ويبنوا أورشليم والهيكل الذي دعي باسمه. وأكلوا بآلات غناء وفرح سبعة أيام.

\chapter{5}

\par 1 وبعد ذلك صعد رؤوس الآباء حسب أسباطهم مع نسائهم وأبنائهم وبناتهم وعبيدهم وإماءهم ومواشيهم.
\par 2 فأرسل داريوس معهم ألف فارس حتى ردوهم إلى أورشليم سالمين، ومعهم الطبول والمزامير.
\par 3 ولعب جميع إخوتهم، فأصعدهم معهم جميعا.
\par 4 وهذه أسماء الرجال الذين صعدوا حسب بيوت آبائهم في أسباطهم، بعد رؤوسهم.
\par 5 وأما الكهنة بنو فينحاس بن هارون: يسوع بن يهوذادق بن سرايا، ويواقيم بن زربابل بن شألتيئيل من بيت داود من عشيرة فارص من سبط يهوذا.
\par 6 الذي تكلم بكلام حكيم أمام داريوس ملك فارس في السنة الثانية من ملكه في شهر نيسان الذي هو الشهر الأول.
\par 7 وهؤلاء هم اليهود الصاعدون من السبي الذي كانوا فيه غرباء، الذين سباهم نبوخذناصر ملك بابل إلى بابل.
\par 8 ثم رجعوا إلى أورشليم وسائر نواحي اليهودية، كل واحد إلى مدينته، ​​الذين جاءوا مع زربابل، ويسوع، ونحميا، وزكريا، ورئشعيا، وعينينيوس، ومردخاوس، وبعلساروس، وأسفراسوس، ورئيليوس، ورويمس، وبعنة، مرشديهم.
\par 9 عددهم من الأمة وولاتهم بنو فورس ألفان ومئة واثنان وسبعون وبنو سافات أربع مئة واثنان وسبعون.
\par 10 أبناء آريس سبعمائة وستة وخمسون:
\par 11 بنو فاعث موآب ألفان وثمانمائة واثنا عشر.
\par 12 بنو عيلام ألف ومائتان وأربعة وخمسون. بنو زثول تسعمائة وخمسة وأربعون. بنو كورب سبعمائة وخمسة. بنو باني ستمائة وثمانية وأربعون.
\par 13 بنو باباي ستمائة وثلاثة وعشرون. بنو صداش ثلاثة آلاف ومائتان واثنان وعشرون.
\par 14 بنو أدونيقام ستمائة وسبعة وستون. بنو باغوي ألفان وستة وستون. بنو عادين أربعمائة وأربعة وخمسون.
\par 15 بنو أترزياس اثنان وتسعون، بنو سيلان وأزطاس سبعة وستون، بنو أزوران أربعمائة واثنان وثلاثون.
\par 16 بنو حننيا مائة وواحد، بنو أروم اثنان وثلاثون، وبنو بصة ثلاثمائة وثلاثة وعشرون، بنو صفوريث مئة واثنان.
\par 17 بنو ميتروس ثلاثة آلاف وخمسة، بنو بيتلومون مائة وثلاثة وعشرون،
\par 18 وهم من نطوفة خمسة وخمسون، وهم من عناثوث مائة وثمانية وخمسون، وهم من بيت ساموس اثنان وأربعون.
\par 19 وهم من قرياتياريوس خمسة وعشرون، وهم من كافيرة وبيروت سبعمائة وثلاثة وأربعون، وهم من بيرة سبعمائة.
\par 20 وهم من قاديا وأمديدوي أربعمائة واثنان وعشرون، وهم من سيراما وجابدس ستمائة وواحد وعشرون.
\par 21 وهم من مكالون مائة واثنان وعشرون، وهم من بيتوليوس اثنان وخمسون، وأبناء نافيس مائة وستة وخمسون.
\par 22 بنو كالامولالوس وأونوس سبعمائة وخمسة وعشرون. بنو أريحا مئتان وخمسة وأربعون.
\par 23 بنو حنان ثلاثة آلاف وثلاثمائة وثلاثون.
\par 24 الكهنة: بنو جدو بن عيسى من بني سناسيب تسعمائة واثنان وسبعون: بنو ميروث ألف واثنان وخمسون:
\par 25 بنو فسارون ألف وسبعة وأربعون. بنو كرمي ألف وسبعة عشر.
\par 26 واللاويون بنو يشوع وقدميئيل وبانواس وسوديا أربعة وسبعون.
\par 27 والمغنون القديسون: بنو آساف مائة وثمانية وعشرون.
\par 28 والبوابون: بنو شلوم، بنو يطال، بنو طلمون، بنو دعقوبي، بنو تيتا، بنو سامي، جميعهم مائة وتسعة وثلاثون.
\par 29 عبيد الهيكل: بنو عيسو، بنو أسيفا، بنو تابوث، بنو سيراش، بنو سود، بنو فالياس، بنو لبانة، بنو جرابا،
\par 30 بنو أكوا، بنو أوتا، بنو كتّاب، بنو أغابا، بنو سوباي، بنو عنان، بنو كاثوا، بنو جدور،
\par 31 بنو عايروس، بنو دايسان، بنو نويبا، بنو كسيبا، بنو جازيرا، بنو عزيا، بنو فينحاس، بنو عزرا، بنو بستاي، بنو أسنا، بنو مياني، بنو نفيشي، بنو عاقب، بنو عكيفا، بنو أشور، بنو فراسيم، بنو بصلوت،
\par 32 بنو ميدا، بنو كوثا، بنو شاريا، بنو شركس، بنو آسيرر، بنو توموي، بنو ناسيث، بنو عطيفة.
\par 33 بنو عبيد سليمان: بنو أصفيون، بنو فريرة، بنو يعيلي، بنو لوزون، بنو إسرائيل، بنو سافث،
\par 34 أبناء هاجيا، أبناء فاراكيرث، أبناء سابي، أبناء ساروثي، أبناء ماسياس، أبناء جار، أبناء أدوس، أبناء سوبا، أبناء أفرا، أبناء باروديس، أبناء ساباط، أبناء ألوم.
\par 35 وكان جميع خدام البيت وبني عبيد سليمان ثلاث مئة واثنان وسبعون.
\par 36 هؤلاء صعدوا من ثيرميليث وثيلرساس، وكان يقودهم كاراثالار، وآلاار.
\par 37 ولم يستطيعوا أن يظهروا قبائلهم ولا أصولهم أنهم من إسرائيل: بنو لدان بن بان، بنو نقودان، ستمائة واثنان وخمسون.
\par 38 ومن الكهنة الذين اغتصبوا الكهنوت ولم يوجدوا: بنو عوبديا، وبنو أكوز، وبنو أدوس الذي تزوج أوجيا إحدى بنات برزلس، فسمي باسمه.
\par 39 ولما تم البحث عن وصف عائلة هؤلاء الرجال في السجل ولم يتم العثور عليهم، تم عزلهم من أداء وظيفة الكهنوت.
\par 40 فإن نحميا وأثاريا قالا لهما: إنهما لا يشتركان في الأشياء المقدسة حتى يقوم رئيس كهنة يلبس التعليم والحق.
\par 41 فكان من إسرائيل من ابن اثنتي عشرة سنة فصاعدا كلهم ​​أربعين ألفا، ما عدا العبيد والإماء ألفين وثلاثمائة وستون.
\par 42 وكان عبيدهم وإماؤهم سبعة آلاف وثلاثمائة وسبعة وأربعين والمغنون والمغنيات مائتين وخمسة وأربعين.
\par 43 أربعمائة وخمسة وثلاثون من الإبل، وسبعة آلاف وستة وثلاثون من الخيول، ومائتان وخمسة وأربعون من البغال، وخمسة آلاف وخمسمائة وخمسة وعشرون من الحيوانات التي تجرها النير.
\par 44 "وبعض رؤساء آبائهم لما جاءوا إلى هيكل الله الذي في أورشليم نذروا أن يقيموا البيت في مكانه حسب طاقتهم،
\par 45 وإعطاء الخزانة المقدسة للأعمال ألف منا من الذهب، وخمسة آلاف من الفضة، ومائة ثوب للكهنة.
\par 46 وهكذا سكن الكهنة واللاويون والشعب في أورشليم وفي الحقول، والمغنون والبوابون أيضاً، وكل إسرائيل في قراهم.
\par 47 "ولما اقترب الشهر السابع، وكان بنو إسرائيل كل واحد في مكانه، جاءوا جميعا بنفس واحدة إلى ساحة الباب الأول الذي هو نحو الشرق."
\par 48 فقام يسوع بن يهوذادق وإخوته الكهنة وزروبابل بن شألتيئيل وإخوته وهيأوا مذبح إله إسرائيل،
\par 49 لإصعاد محرقات عليه، كما هو مأمور به صراحة في كتاب موسى رجل الله.
\par 50 واجتمع إليهم من سائر أمم الأرض، فأقاموا المذبح في مكانه، لأن كل أمم الأرض كانوا في عداوة معهم وضايقوهم. وكانوا يذبحون ذبائح حسب الوقت، ومحرقات للرب صباحا ومساء.
\par 51 وكانوا يحتفلون بعيد المظال كما هو مأمور به في الناموس، ويقدمون الذبائح كل يوم كما ينبغي.
\par 52 وبعد ذلك، التقدمات الدائمة، وذبائح السبوت، والأهلة، وجميع الأعياد المقدسة.
\par 53 وكل الذين نذروا نذرا لله ابتدأوا يذبحون ذبائح لله من اليوم الأول من الشهر السابع، مع أن هيكل الرب لم يكن قد بني بعد.
\par 54 وأعطوا البنائين والنجارين الفضة والطعام والشراب بفرح.
\par 55 وأعطوا أهل صيدون وصور أيضا سفناً لكي يأتوا بأخشاب أرز من لبنان، فيحملونها في سفن إلى ميناء يافا كما أمرهم كورش ملك الفرس.
\par 56 وفي السنة الثانية والشهر الثاني بعد مجيئه إلى هيكل الله في أورشليم، ابتدأ زربابل بن شألتيئيل ويسوع بن يهوذادق وإخوتهما والكهنة واللاويون وكل الذين أتوا إلى أورشليم من السبي.
\par 57 ووضعوا أساس بيت الله في اليوم الأول من الشهر الثاني، في السنة الثانية لمجيئهم إلى اليهودية وأورشليم.
\par 58 وأقاموا اللاويين من ابن عشرين سنة على أعمال الرب. فقام يسوع وبنوه وإخوته، وقدميئيل أخوه، وأبناء مديابون، وأبناء يهوذا بن أليادون، وأبناؤهم وإخوتهم، جميع اللاويين، بقلب واحد، عاملين على العمل، يعملون على إتمام أعمال بيت الله. فبنى العمال هيكل الرب.
\par 59 "ووقف الكهنة مرتدين ثيابهم مع آلات الغناء والأبواق، وكان اللاويون بنو آساف يحملون الصنوج،
\par 60 ترنيم الحمد والتسبيح للرب كما أمر داود ملك إسرائيل.
\par 61 وكانوا يغنون بأصوات عظيمة ترانيم لتسبيح الرب لأن رحمته ومجده إلى الأبد في كل إسرائيل.
\par 62 فنفخ جميع الشعب في الأبواق وهتفوا بصوت عظيم وهم يغنون الحمد للرب لأجل إقامة بيت الرب.
\par 63 وأيضاً من الكهنة واللاويين ورؤساء آبائهم الشيوخ الذين رأوا البيت الأول جاءوا إلى بناء هذا بالبكاء والصراخ العظيم.
\par 64 ولكن كثيرين هتفوا بصوت عظيم مع الأبواق والفرح،
\par 65 حتى أن البوق لم يُسمَع لبكاء الشعب، ولكن الجمع نفخ نفخة عظيمة حتى سُمِعَ من بعيد.
\par 66 ولذلك عندما سمع أعداء سبط يهوذا وبنيامين ذلك، عرفوا ما يعنيه صوت الأبواق.
\par 67 وعلموا أن الذين من السبي هم الذين بنوا هيكل الرب إله إسرائيل.
\par 68 فذهبوا إلى زربابل ويسوع ورؤساء العائلات وقالوا لهم: نبني معكم جميعا.
\par 69 فإننا نحن أيضاً مثلكم نطيع سيدنا ونذبح له منذ أيام عزبازرة ملك أشور الذي أتى بنا إلى هنا.
\par 70 فقال لهم زربابل ويسوع ورؤساء آباء إسرائيل ليس لنا ولكم أن نبني بيتا للرب إلهنا.
\par 71 نحن وحدنا نبني للرب إسرائيل كما أمرنا كورش ملك فارس.
\par 72 ولكن أمميي الأرض كانوا يثقلون كاهل سكان اليهودية، ويقيدونهم، مما أعاق بناءهم.
\par 73 وبمؤامراتهم السرية وإقناعاتهم الشعبية واضطراباتهم، أعاقوا إتمام البناء كل أيام الملك كورش. فأوقفوا عن البناء مدة سنتين حتى ملك داريوس.

\chapter{6}

\par 1 وفي السنة الثانية من ملك داريوس حجوس وزكريا بن عدو تنبأ النبيان لليهود في يهوذا وأورشليم باسم الرب إله إسرائيل الذي كان عليهم.
\par 2 حينئذ قام زربابل بن سالاتيئيل ويسوع بن يهوذادق وابتدآ يبنيان بيت الرب في أورشليم وكان معهما أنبياء الرب يساعدونهما.
\par 3 وفي ذلك الوقت جاء إليهم سيسينيس حاكم سورية وفينيقية، ومعه سترازوسان ورفاقه، وقال لهم:
\par 4 من الذي أمركم أن تبنوا هذا البيت وهذا السقف وكل هذه الأعمال الأخرى؟ ومن هم العمال الذين يعملون هذه الأعمال؟
\par 5 ولكن شيوخ اليهود نالوا نعمة لأن الرب افتقد السبي.
\par 6 ولم يمنعهم ذلك من البناء حتى أعطي لداريوس علم عنهم وتلقى جوابا.
\par 7 نسخة الرسائل التي كتبها سيسينيس حاكم سورية وفينيقية وستر بوزنيس ورفاقهما حكام سورية وفينيقية وأرسلوها إلى داريوس إلى الملك داريوس تحية:
\par 8 فليكن معلوماً لدى سيدنا الملك أنه لما أتينا إلى بلاد اليهودية ودخلنا مدينة أورشليم وجدنا في مدينة أورشليم شيوخ اليهود الذين كانوا من السبي.
\par 9 بناء بيت للرب، عظيماً وجديداً، من حجارة منحوتة وحجارة كريمة، وخشب موضوع على الحيطان.
\par 10 وتتم هذه الأعمال بسرعة كبيرة، ويسير العمل بنجاح في أيديهم، وبكل مجد واجتهاد يتم ذلك.
\par 11 فسألنا هؤلاء الشيوخ قائلين: بأمر من تبنون هذا البيت وتضعون أسس هذه الأعمال؟
\par 12 ولذلك لكي نعلمك كتابةً، طلبنا من القائمين على الأمر كتابة أسماء كبار رجالهم.
\par 13 فأجابونا: نحن عبيد الرب صانع السماء والأرض.
\par 14 وأما هذا البيت فقد بناه قبل سنين كثيرة ملك إسرائيل عظيم وقوي وأكمل.
\par 15 ولكن عندما أغضب آباؤنا الله وأخطأوا إلى رب إسرائيل الذي في السماء أسلمهم إلى يد نبوخذنصر ملك بابل الكلدانيين.
\par 16 الذي هدم البيت وأحرقه وسبى الشعب إلى بابل.
\par 17 ولكن في السنة الأولى لملك كورش الملك على بلاد بابل كتب الملك كورش لبناء هذا البيت.
\par 18 "وأما الآنية المقدسة من الذهب والفضة التي أخرجها نبوخذناصر من بيت أورشليم ووضعها في هيكله فأخرجها الملك كورش أيضاً من هيكل بابل ودفعها إلى زربابل وسناباساروس الملك،
\par 19 وأمر أن يأخذ تلك الآنية ويضعها في هيكل أورشليم، ويبنى هيكل الرب في مكانه.
\par 20 ثم جاء سنباساروس هذا إلى هنا، ووضع أساسات بيت الرب في أورشليم، ومن ذلك الوقت إلى الآن لم يكتمل هذا البناء بعد.
\par 21 والآن إن حسن عند الملك فليبحث في سجلات الملك كورش.
\par 22 وإن وجد أن بناء بيت الرب في أورشليم قد تم بموافقة الملك كورش، وإذا كان سيدنا الملك مهتماً بهذا فليخبرنا بذلك.
\par 23 ثم أمر الملك داريوس أن يبحث في الكتب التي في بابل، فوجد في أحمتان القصر الذي في بلاد ميديا ​​درجاً مكتوباً فيه هذه الأمور.
\par 24 وفي السنة الأولى من ملك كورش أمر الملك كورش أن يعاد بناء بيت الرب في أورشليم حيث يذبحون بالنار الدائمة.
\par 25 ويكون ارتفاعها ستين ذراعا وعرضها ستين ذراعا، بثلاثة صفوف من الحجارة المنحوتة وصف واحد من خشب جديد من تلك الأرض، ونفقتها تعطى من بيت الملك كورش.
\par 26 وأن آنية القدس لبيت الرب من الذهب والفضة التي أخرجها نبوخذناصر من بيت أورشليم وجاء بها إلى بابل تعاد إلى بيت أورشليم وتوضع في المكان الذي كانت فيه أولا.
\par 27 وأمر أيضا أن سيسينيس والي سورية وفينيقية، وستر بوزنيس ورفقائهما، والذين أقيموا حكاما على سورية وفينيقية، أن يحذروا من المساس بهذا المكان، بل يتركوا زربابل، خادم الرب ووالي اليهودية، وشيوخ اليهود، يبنون بيت الرب في ذلك المكان.
\par 28 وقد أمرت أيضاً أن يبنوا كاملاً أيضاً، وأن يهتموا بمساعدة الذين من سبي اليهود حتى يكمل بيت الرب.
\par 29 ومن جزية بقاع سورية وفينيقية قسمة مجزية تُعطى لهؤلاء الرجال عن ذبائح الرب، أي لزروبابل الوالي، من العجول والكباش والحملان.
\par 30 وأيضاً القمح والملح والخمر والزيت، وذلك دائماً كل سنة بلا جدال، كما يشير الكهنة الذين في أورشليم إلى الإنفاق اليومي.
\par 31 لكي تقدم القرابين إلى الله العلي من أجل الملك وأولاده، ولكي يصلوا من أجل حياتهم.
\par 32 وأمر بأن كل من يتعدى أو يستهين بأي شيء سبق أن قيل أو كتب، يجب أن تؤخذ من بيته شجرة ويعلق عليها، وتصادر جميع ممتلكاته للملك.
\par 33 لذلك، الرب الذي دُعي اسمه هناك، يُهلك كل ملك وكل أمة يمد يده ليمنع أو يُهدم بيت الرب في أورشليم.
\par 34 أنا داريوس الملك أمرت أن يتم ذلك بكل اجتهاد.

\chapter{7}

\par 1 ثم سيسينس حاكم بلاد الشام وفينيقية، وسثرابوزانيس ورفاقهما، باتباع أوامر الملك داريوس،
\par 2 أشرف بعناية شديدة على الأعمال المقدسة، وساعد شيوخ اليهود وحكام الهيكل.
\par 3 وهكذا ازدهرت الأعمال المقدسة عندما تنبأ النبيان أجيوس وزكريا.
\par 4 وأكملوا هذه الأمور حسب أمر الرب إله إسرائيل وموافقة كورش وداريوس وأرتحشستا ملوك فارس.
\par 5 وهكذا تم إكمال البيت المقدس في اليوم الثالث والعشرين من شهر أذار في السنة السادسة لداريوس ملك فارس.
\par 6 ففعل بنو إسرائيل الكهنة واللاويون وسائر السبي الذين انضموا إليهم حسب ما هو مكتوب في سفر موسى.
\par 7 ولتدشين هيكل الرب قدموا مئة ثور ومائتي كبش وأربعمائة خروف.
\par 8 واثني عشر تيسا لخطية كل إسرائيل حسب عدد رؤساء أسباط إسرائيل.
\par 9 ووقف الكهنة واللاويون في ثيابهم حسب عشائرهم لخدمة الرب إله إسرائيل حسب سفر موسى والبوابون على كل باب.
\par 10 وعمل بنو إسرائيل الذين من السبي الفصح في اليوم الرابع عشر من الشهر الأول، بعد أن تقدس الكهنة واللاويون.
\par 11 ولم يكن جميع الذين من السبي مقدسين معاً، وأما اللاويون فكانوا جميعهم مقدسين معاً.
\par 12 فصنعوا الفصح عن جميع المسبيين وعن إخوتهم الكهنة وعن أنفسهم.
\par 13 فأكل بنو إسرائيل الخارجون من السبي جميع الذين انفصلوا عن رجاسات شعب الأرض وطلبوا الرب.
\par 14 فاحتفلوا بعيد الفطير سبعة أيام فرحين أمام الرب.
\par 15 لأنه حول مشورة ملك أشور نحوهم لتشديد أيديهم في أعمال الرب إله إسرائيل.

\chapter{8}

\par 1 وبعد هذه الأمور، عندما ملك أرتحشستا ملك الفرس، جاء عزرا بن سرايا بن عزريا بن حلخيا بن شلوم،
\par 2 ابن صدوق، بن أخيطوب، بن أمريا، بن عزيا، بن مريموث، بن زارايا، بن ساوياس، بن بوكاس، بن أبيسوم، بن فينحاس، بن العازار، بن هارون رئيس الكهنة.
\par 3 هذا عزرا صعد من بابل كاتباً، وكان على دراية تامة بشريعة موسى التي أعطيت من إله إسرائيل.
\par 4 فأكرمه الملك لأنه وجد نعمة في عينيه في كل طلباته.
\par 5 وصعد معه أيضاً بعض بني إسرائيل من كهنة اللاويين والمغنين المقدسين والبوابين وخدام الهيكل إلى أورشليم،
\par 6 وفي السنة السابعة من ملك أرتحشستا، في الشهر الخامس، كانت هذه السنة السابعة للملك. لأنهم خرجوا من بابل في اليوم الأول من الشهر الأول، وجاءوا إلى أورشليم حسب الطريق الصالح الذي أعطاهم الرب إياه.
\par 7 وكان عزرا ماهراً جداً، حتى أنه لم يهمل شيئاً من شريعة الرب ووصاياه، بل كان يعلم كل إسرائيل الأحكام والفرائض.
\par 8 وأما نسخة الأمر الذي كتبه الملك أرتحشستا ووصل إلى عزرا الكاهن وقارئ شريعة الرب فهي كالتالي:
\par 9 الملك أرتحشستا يرسل سلامه إلى عزرا الكاهن وقارئ شريعة الرب.
\par 10 وبعد أن قررت أن أتعامل بلطف، فقد أصدرت أمراً بأن يذهب معك إلى أورشليم أولئك من أمة اليهود والكهنة واللاويين الموجودين في مملكتنا، الذين يرغبون ويرغبون في ذلك.
\par 11 "فكل من له رأي في ذلك فليذهب معك، كما رأيت أنا وأصدقائي السبعة المستشارين."
\par 12 لكي ينظروا إلى أمور اليهودية وأورشليم حسب ما هو في ناموس الرب.
\par 13 "وحملوا إلى أورشليم الهدايا التي نذرتها أنا وأصدقائي للرب إسرائيل، وكل الذهب والفضة الموجودة في بلاد بابل، للرب في أورشليم،
\par 14 وكذلك ما يُعطى من الشعب لمعبد الرب إلههم في أورشليم، حتى يُجمع الفضة والذهب للثيران والكباش والحملان وما يتعلق بها.
\par 15 لكي يقدموا ذبائح للرب على مذبح الرب إلههم الذي في أورشليم.
\par 16 وكل ما تريد أنت وإخوتك أن تفعلوه بالفضة والذهب، فافعلوه بحسب مشيئة إلهكم.
\par 17 وأما آنية الرب المقدسة التي أعطيت لك لبيت إلهك الذي في أورشليم فتضعها أمام إلهك في أورشليم.
\par 18 وأما كل ما تذكره من أجل هيكل إلهك فأعطه من خزائن الملك.
\par 19 وأنا الملك أرتحشستا أمرت أيضًا أمناء الكنوز في سورية وفينيقيا أن كل ما يطلبه عزرا الكاهن وقارئ شريعة الله الأعظم، فليعطوه إياه بسرعة.
\par 20 إلى مجموع مائة وزنة من الفضة، وكذلك أيضًا من القمح حتى مائة كور، ومئة قطعة من الخمر، وأشياء أخرى بكثرة.
\par 21 لتكن كل الأشياء حسب شريعة الله باجتهاد أمام الله العلي، لكي لا يأتي الغضب على مملكة الملك وبنيه.
\par 22 وأنا أوصيكم أيضا أن لا تطلبوا ضريبة أو أي فرض آخر من أي من الكهنة أو اللاويين أو المغنين المقدسين أو البوابين أو خدام الهيكل أو من أي من الذين لهم أعمال في هذا الهيكل، وأن لا يكون لأحد سلطان أن يفرض عليهم أي شيء.
\par 23 وأنت يا عزرا، بحسب حكمة الله، تقيم قضاةً وقضاةً، لكي يقضوا في كل سورية وفينيقية لكل من يعرف شريعة إلهك. ومن لا يعرفها فتعلمه.
\par 24 وكل من يتعدى على شريعة إلهك وشريعة الملك، فإنه يعاقب بشدة، سواء بالموت أو بعقوبة أخرى، أو بغرامة مالية، أو بالسجن.
\par 25 ثم قال عزرا الكاتب تبارك الرب إله آبائي الوحيد الذي وضع هذه الأمور في قلب الملك لأجل تمجيد بيته الذي في أورشليم.
\par 26 وأكرمني أمام الملك ومشيريه وجميع أصدقائه وعظمائه.
\par 27 لذلك تشجعت بمعونة الرب إلهي، وجمعت رجال إسرائيل ليصعدوا معي.
\par 28 وهؤلاء هم الرؤساء حسب عشائرهم ورتبهم الذين صعدوا معي من بابل في ملك أرتحشستا الملك.
\par 29 من بني فينحاس جرشون، ومن بني إيثامار جمائيل، ومن بني داود لطس بن سكنياس.
\par 30 ومن بني فارص زكريا ومعه مائة وخمسون رجلاً.
\par 31 ومن بني فحث موآب إيلياونيا بن زارايا ومعه مئتا رجل.
\par 32 من بني زاتوه سكنياس بن يزليوس ومعه ثلاثمائة رجل. ومن بني عادين عوبيث بن يوناثان ومعه مئتان وخمسون رجلاً.
\par 33 ومن بني عيلام يوشيا بن جتوليا ومعه سبعون رجلاً.
\par 34 ومن بني سفطيا زارايا بن ميخائيل ومعه سبعون رجلاً.
\par 35 ومن بني يوآب أبدياس بن يزلوش ومعه مئتان واثنا عشر رجلاً.
\par 36 ومن بني بنياد أساليموث بن يوشافيا ومعه مائة وستون رجلاً.
\par 37 ومن أبناء بابي زكريا بن باباي ومعه ثمانية وعشرون رجلاً.
\par 38 ومن بني آشتات يوحنا بن أكاتان ومعه مائة وعشرة من الرجال.
\par 39 ومن بني أدونيقام الأخيرين وهذه أسماؤهم: إليفلط وجويل وسامايا ومعهم سبعون رجلاً.
\par 40 ومن أبناء باغو، أوثي بن إستالكوروس، ومعه سبعون رجلاً.
\par 41 فجمعتهم إلى النهر الذي يدعى ثيراس، حيث نصبنا خيامنا ثلاثة أيام، ثم تفقدتهم.
\par 42 ولكن لما لم أجد هناك أحدا من الكهنة واللاويين،
\par 43 ثم أرسلت إلى العازار وإيدويل ومسمان،
\par 44 وألناثان، وماميا، ويوريباس، وناثان، ويوناتان، وزكريا، وموسلامون، رجال رؤساء وعلماء.
\par 45 وأمرتهم أن يذهبوا إلى صدّيوس القائد الذي كان في مكان الخزانة.
\par 46 وأمرهم أن يكلموا داديوس وإخوته وأمناء الخزنة في ذلك المكان لكي يرسلوا إلينا رجالاً يقومون بخدمة الكهنوت في بيت الرب.
\par 47 وبيد ربنا القوية أتوا إلينا رجالاً ماهرين من بني مولي بن لاوي بن إسرائيل أشيبيا وبنوه وإخوته ثمانية عشر رجلاً.
\par 48 وأسيبيا وأنوس وأوسايا أخوه من بني كنونيوس وبنوهم عشرون رجلاً.
\par 49 ومن خدام الهيكل الذين أقامهم داود، والرجال الأوائل لخدمة اللاويين، خدام الهيكل مائتين وعشرين، وقد أُظهِرت أسماءهم.
\par 50 وهناك نذرت صوماً للشباب أمام ربنا، لأطلب منه رحلة موفقة لنا ولمن معنا، ولأولادنا وللمواشي.
\par 51 لأني خجلت من أن أطلب من الملك جنوداً وفرساناً وقائداً لحماية أنفسنا من أعدائنا.
\par 52 لأننا قلنا للملك أن قوة الرب إلهنا تكون مع الذين يطلبونه ليعينهم في كل شيء.
\par 53 ثم طلبنا من الرب أيضًا أن يطلب منا هذه الأمور، فوجدناه مقبولًا لدينا.
\par 54 ثم فصلت اثني عشر من رؤساء الكهنة: اسبرياس وأسانياس، وعشرة رجال من إخوتهما معهما.
\par 55 ووزنت لهم الذهب والفضة وآنية القدس لبيت ربنا التي أعطاها الملك ومشورته والرؤساء وكل إسرائيل.
\par 56 "ولما وزنتها، سلمت إليهم ستمائة وخمسين وزنة من الفضة، وآنية فضة مئة وزنة، ومئة وزنة من الذهب،
\par 57 وعشرين من آنية من ذهب، واثنتي عشرة آنية من نحاس، نحاس نقي لامع كالذهب.
\par 58 فقلت لهم أنتم مقدسون للرب، والآنية مقدسة، والذهب والفضة نذر للرب رب آبائنا.
\par 59 اسهروا واحفظوها حتى تدفعوها إلى رؤساء الكهنة واللاويين ورؤساء آباء إسرائيل في أورشليم في مخادع بيت إلهنا.
\par 60 فأتى الكهنة واللاويون الذين أخذوا الفضة والذهب والآنية بها إلى أورشليم، إلى هيكل الرب.
\par 61 "ومن نهر ثيراس ارتحلنا في اليوم الثاني عشر من الشهر الأول، وأتينا إلى أورشليم بيد ربنا القوية التي كانت معنا. ومن ابتداء رحلتنا أنقذنا الرب من كل عدو، وهكذا أتينا إلى أورشليم."
\par 62 وبعد أن أقمنا هناك ثلاثة أيام، أُعطي الذهب والفضة التي وُزنت في بيت ربنا في اليوم الرابع لمرموث الكاهن ابن عيري.
\par 63 وكان معه العازار بن فينحاس ومعهم يوساباد بن يسوع وموئت بن شبان اللاويان. فأسلم إليهم الجميع بالعدد والوزن.
\par 64 وكل وزنهم كتب في نفس الساعة.
\par 65 ثم إن الذين خرجوا من السبي قدموا ذبائح للرب إله إسرائيل اثني عشر ثورا عن كل إسرائيل وستة وثمانين كبشا.
\par 66 اثنان وستون خروفاً وتيوساً لذبيحة السلامة اثنا عشر كلها ذبيحة للرب.
\par 67 وأبلغوا أوامر الملك إلى وكلاء الملك وولاة بقاع سورية وفينيقية، فكانوا يكرمون الشعب وهيكل الله.
\par 68 وبعد أن تم ذلك، جاء إلي الرؤساء وقالوا:
\par 69 ولم ينزع شعب إسرائيل، الرؤساء والكهنة واللاويون، من بينهم الغرباء من أهل الأرض، ولا نجاسات الأمم من الكنعانيين والحثيين والفرسيين واليبوسيين والموآبيين والمصريين والأدوميين.
\par 70 لأنهم هم وأبناؤهم تزوجوا ببناتهم، واختلط النسل المقدس بغرباء الأرض، ومنذ بدء هذا الأمر كان الرؤساء والعظماء شركاء في هذا الإثم.
\par 71 ولما سمعت هذه الأمور مزقت ملابسي والثوب المقدس، ونتفت شعر رأسي ولحيتي، وجلست حزيناً وثقيلاً جداً.
\par 72 فاجتمع إليّ كل الذين تأثروا بكلام الرب إله إسرائيل، وأنا حزنت على الإثم، وجلست حزيناً إلى تقدمة المساء.
\par 73 ثم أقوم من الصوم وملابسي والثوب المقدس ممزقان وأحني ركبتي وأمد يدي إلى الرب،
\par 74 فقلت يا رب إني أخجل وأخجل من وجهك.
\par 75 لأن خطايانا قد كثرت فوق رؤوسنا، وجهالتنا وصلت إلى السماء.
\par 76 فمنذ أيام آبائنا كنا وما زلنا في خطيئة عظيمة إلى هذا اليوم.
\par 77 ومن أجل خطايانا وخطايا آبائنا، سُلِمنا نحن وإخوتنا وملوكنا وكهنتنا لملوك الأرض للسيف والسبي والنهب بالعار إلى هذا اليوم.
\par 78 والآن فقد أظهر لنا بعض الرحمة منك يا رب، حتى يبقى لنا جذر واسم في مكان قدسك.
\par 79 "وليظهر لنا سراجاً في بيت الرب إلهنا، ويعطينا طعاماً في زمن عبوديتنا."
\par 80 نعم، حين كنا في العبودية، لم يتركنا الرب، بل جعلنا منعمين أمام ملوك فارس حتى أعطونا طعامًا.
\par 81 نعم، وكرموا هيكل ربنا، وأقاموا صهيون الخربة، حتى أعطونا إقامة أكيدة في اليهودية وأورشليم.
\par 82 والآن يا رب ماذا نقول وقد بلغنا هذا؟ لأننا تعدينا وصاياك التي أمرت بها عن يد عبيدك الأنبياء قائلا:
\par 83 إن الأرض التي أنتم داخلون إليها لكي تمتلكوها ميراثا هي أرض نجسة بنجاسات غرباء الأرض وقد ملأوها من نجاساتهم.
\par 84 لذلك الآن لا تلحقوا بناتكم ببنيهم ولا تأخذوا بناتهم لبنيكم.
\par 85 ولا تطلبوا السلام معهم أبدًا، لكي تقوىوا وتأكلوا خيرات الأرض، ولكي تتركوا ميراث الأرض لأولادكم إلى الأبد.
\par 86 وكل ما أصابنا إنما هو بسبب أعمالنا الشريرة وخطايانا العظيمة، لأنك يا رب خففت عنا خطايانا،
\par 87 فأعطيتنا مثل هذا الجذر، لكننا رجعنا أيضًا لنتعدى على شريعتك ونختلط بنجاسة أمم الأرض.
\par 88 ألا يجوز لك أن تغضب علينا لتهلكنا حتى لا تترك لنا جذرًا ولا بذرة ولا اسمًا؟
\par 89 يا رب إسرائيل أنت صادق لأننا بقينا اليوم أصلاً.
\par 90 هوذا نحن الآن أمامك في آثامنا، لأنه لا يمكننا أن نقف بعد أمامك بسبب هذه الأمور.
\par 91 وبينما كان عزرا يعترف في صلاته، باكياً، وملقى على الأرض أمام الهيكل، اجتمع إليه من أورشليم جمع عظيم من الرجال والنساء والأطفال، لأنه كان هناك بكاء عظيم بين الجمع.
\par 92 فنادى يكنيا بن ييئيلوش واحد من بني إسرائيل وقال يا عزرا قد أخطأنا إلى الرب الإله واتخذنا نساء غريبة من أمم الأرض والآن كل إسرائيل في العلاء.
\par 93 فلنحلف للرب أن نخلع جميع نسائنا اللواتي أخذناهنّ من الأمم مع أولادهنّ.
\par 94 كما قضت، وكما يطيعون شريعة الرب.
\par 95 قم وانفذ، لأن هذا الأمر يعود إليك، ونحن سنكون معك. افعل ذلك بشجاعة.
\par 96 فقام عزرا واستحلف رؤساء الكهنة واللاويين من كل إسرائيل أن يعملوا مثل هذه الأمور فحلفوا كذلك.

\chapter{9}

\par 1 ثم قام عزرا من دار الهيكل وذهب إلى حجرة يوحنان بن الياشيب،
\par 2 وأقام هناك ولم يأكل لحماً ولم يشرب ماءً، وكان يبكي على ذنوب الجمع الكثيرة.
\par 3 "وكان نداء في كل اليهودية وأورشليم إلى جميع المسبيين أن يجمعوا إلى أورشليم.
\par 4 "ومن لم يلتقِ هناك في يومين أو ثلاثة أيام حسب ما عيّنه الشيوخ الذين تولوا الأمر، تُؤخذ مواشيه إلى الهيكل، ويُطرد هو من بين المسبيين."
\par 5 وفي ثلاثة أيام اجتمع جميع سبط يهوذا وبنيامين إلى أورشليم في اليوم العشرين من الشهر التاسع.
\par 6 وكان كل الجمع جالسين مرتجفين في دار الهيكل الواسعة بسبب نوء الطقس.
\par 7 فقام عزرا وقال لهم: لقد خالفتم الناموس بالزواج من نساء أجنبيات لكي تزيدوا بذلك خطايا إسرائيل.
\par 8 والآن بالاعتراف أعطوا المجد للرب إله آبائنا،
\par 9 فافعلوا مشيئته، وانفصلوا عن أمم الأرض وعن النساء الأجنبيات.
\par 10 فصرخ كل الجمع بصوت عظيم وقالوا كما تكلمت كذلك نفعل.
\par 11 ولكن بما أن الناس كثيرون، والجو كان بارداً، فلا نستطيع أن نقف خارجاً، وهذا ليس عملاً يستغرق يوماً أو يومين، لأن خطيتنا في هذه الأمور قد انتشرت على نطاق واسع:
\par 12 "فليبق رؤساء الجمهور، وليأتِ جميع أهل بيوتنا الذين لهم زوجات أجنبيات في الوقت المحدد،
\par 13 ومعهم رؤساء وقضاة كل مكان حتى نصرف غضب الرب عنا في هذا الأمر.
\par 14 فأخذ يوناتان بن عزائيل وحزقيا بن ثيوكانوس الأمر عليهما، وساعدهما موسلام ولاوي وساباثيوس.
\par 15 ففعل الذين من السبي حسب كل هذا.
\par 16 فاختار عزرا الكاهن لنفسه وجوه آبائهم كلهم ​​بأسمائهم، وفي اليوم الأول من الشهر العاشر جلسوا معا لينظروا في الأمر.
\par 17 وبذلك انتهت قضيتهم التي تتعلق بالزوجات الأجنبيات في اليوم الأول من الشهر الأول.
\par 18 ومن الكهنة الذين اجتمعوا وكان لهم نساء غريبات، وجدوا:
\par 19 من أبناء يسوع بن يهوذادق وإخوته متيلاوس وألعازار ويوريبوس ويوادانوس.
\par 20 فأعطوا أيديهم لخلع نساءهم وتقديم الكباش للتكفير عن خطاياهم.
\par 21 ومن بني إمير حننيا وزبداوس ويأنس وساميوس وهيريئيل وعزريا.
\par 22 ومن بني فايشور: إليوناس، ومسياس إسرائيل، ونثنائيل، وأوقيدل، وتالساس.
\par 23 ومن اللاويين يوزاباد وساميس وكوليوس الذي يدعى كاليتاس وفاتيوس ويهوذا ويونس.
\par 24 من المغنين القديسين: إليعازورس، باخوس.
\par 25 من البوابين: سالوموس، وتولبانيس.
\par 26 ومن إسرائيل من بني فورس: هيرمص، وأديا، وملكيا، ومايلوس، وأليعازار، وأسيبيا، وبعنيا.
\par 27 من بني أيلة: متانيا، وزكريا، وإيريلوس، وهيريموث، وعيدية.
\par 28 ومن بني زاموث؛ إلياداس وأليسيمس وأثونيا ويريموث وساباتوس وساردس.
\par 29 من أبناء باباي يوحنا وحننيا ويوزاباد وأماثايس.
\par 30 من بني ماني أولاموس، وماموخوس، ويديوس، وياسوبوس، وياشائيل، وهيريموث.
\par 31 ومن بني عدي: ناثوس وموسيا ولاكونوس ونايدوس ​​ومتنيا وسيسثيل وبالنوس ومنسى.
\par 32 ومن بني حنان: إيليوناس وآسياس وملخيا وسابوس وسمعان كوساموس.
\par 33 ومن بني عاسوم: ألتانيوس ومتياس وبعنايا وإيفلط ومنسى وشمعي.
\par 34 ومن بني معني: إرميا، وموديس، وعومروس، ويعوئيل، ومبدائي، وفليا، وأنوس، وقربازيون، وعناسيبوس، وممنيتنايموس، وإيلياسيوس، وبانوس، وأليعالي، وشاميش، وشلميا، ونثنيا. ومن بني عذر: سيسيس، وإسريل، وأزيلوس، وساماطوس، وزمبيس ، ويوسيفوس.
\par 35 ومن بني أثمة مزيتيا وزبدايا وإيدس ويوعيل وبنايا.
\par 36 وكان كل هؤلاء قد اتخذوا نساءً غريبات، فتركوهن مع أولادهم.
\par 37 وأقام الكهنة واللاويون وأبناء إسرائيل في أورشليم وفي الحقول في اليوم الأول من الشهر السابع وكان بنو إسرائيل في مساكنهم.
\par 38 "وكان الجمع كله يجتمع بصوت واحد إلى ساحة الرواق المقدس نحو الشرق.
\par 39 فكلموا عزرا الكاهن والقارئ أن يأتي بشريعة موسى التي أعطيت من الرب إله إسرائيل.
\par 40 فأتى عزرا رئيس الكهنة بالشريعة إلى كل الجمهور من الرجال والنساء وجميع الكهنة ليسمعوا الشريعة في اليوم الأول من الشهر السابع.
\par 41 وكان يقرأ في الدار الواسعة أمام الرواق المقدس من الصباح إلى الظهر أمام الرجال والنساء، وكان الجمع يصغون إلى الشريعة.
\par 42 فقام عزرا الكاهن وقارئ الشريعة على منبر من خشب صنع لهذا الغرض.
\par 43 ووقف لديه متثيا وسموس وحنانيا وعزريا وأوريا وحزقيا وبلعسامس عن اليمين.
\par 44 وعن يده اليسرى وقف فلدايوس وميصائيل وملكيا ولوثاسوبس ونباريا.
\par 45 ثم أخذ عزرا سفر الشريعة أمام الجمع، لأنه جلس في الموضع الأول بشرف أمام الجميع.
\par 46 ولما فتح الناموس، وقفوا جميعاً على قدميهم. فبارك عزرا الرب الإله العلي، إله الجنود القادر على كل شيء.
\par 47 فأجاب جميع الشعب: آمين، ورفعوا أيديهم وسقطوا على الأرض وسجدوا للرب.
\par 48 وأيضاً يسوع، وآنوس، وسارابيا، وأدينوس، ويعقوب، وساباتياس، وأوطياس، ومايانياس، وكاليطا، وأسرياس، ويوازابدوس، وحننيا، وبياتاس، اللاويون، علموا ناموس الرب، وجعلوهم يفهمونه.
\par 49 ثم كلم عطارات عزرا رئيس الكهنة والقارئ واللاويين الذين كانوا يعلمون الجموع، حتى الجميع، قائلا:
\par 50 "هذا اليوم مقدس للرب. (لأنهم بكوا جميعاً حين سمعوا الشريعة)"
\par 51 اذهبوا فكلوا الشحم واشربوا الحلو وأرسلوا جزءا إلى الذين ليس لهم شيء.
\par 52 لأن هذا اليوم مقدس للرب فلا تحزنوا لأن الرب يكرمكم.
\par 53 فأخبر اللاويون الشعب بكل شيء قائلين: هذا اليوم مقدس للرب، لا تحزنوا.
\par 54 ثم انطلقوا كل واحد ليأكل ويشرب ويفرح ويعطي الذين ليس لهم شيء وليفرحوا كثيرا.
\par 55 لأنهم فهموا الكلام الذي علموهم إياه والذي لأجله اجتمعوا.

\end{document}