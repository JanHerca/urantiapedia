\begin{document}

\title{تثنية}


\chapter{1}

\par 1 هَذَا هُوَ الكَلامُ الذِي كَلمَ بِهِ مُوسَى جَمِيعَ إِسْرَائِيل فِي عَبْرِ الأُرْدُنِّ فِي البَرِّيَّةِ فِي العَرَبَةِ قُبَالةَ سُوفٍَ بَيْنَ فَارَانَ وَتُوفَل وَلابَانَ وَحَضَيْرُوتَ وَذِي ذَهَبٍ.
\par 2 أَحَدَ عَشَرَ يَوْماً مِنْ حُورِيبَ عَلى طَرِيقِ جَبَلِ سَعِيرَ إِلى قَادِشَ بَرْنِيعَ.
\par 3 فَفِي السَّنَةِ الأَرْبَعِينَ فِي الشَّهْرِ الحَادِي عَشَرَ فِي الأَوَّلِ مِنَ الشَّهْرِ كَلمَ مُوسَى بَنِي إِسْرَائِيل حَسَبَ كُلِّ مَا أَوْصَاهُ الرَّبُّ إِليْهِمْ.
\par 4 بَعْدَ مَا ضَرَبَ سِيحُونَ مَلِكَ الأَمُورِيِّينَ السَّاكِنَ فِي حَشْبُونَ وَعُوجَ مَلِكَ بَاشَانَ السَّاكِنَ فِي عَشْتَارُوثَ فِي إِذْرَعِي.
\par 5 فِي عَبْرِ الأُرْدُنِّ فِي أَرْضِ مُوآبَ ابْتَدَأَ مُوسَى يَشْرَحُ هَذِهِ الشَّرِيعَةَ قَائِلاً:
\par 6 «اَلرَّبُّ إِلهُنَا كَلمَنَا فِي حُورِيبَ قَائِلاً: كَفَاكُمْ قُعُودٌ فِي هَذَا الجَبَلِ!
\par 7 تَحَوَّلُوا وَارْتَحِلُوا وَادْخُلُوا جَبَل الأَمُورِيِّينَ وَكُل مَا يَلِيهِ مِنَ العَرَبَةِ وَالجَبَلِ وَالسَّهْلِ وَالجَنُوبِ وَسَاحِلِ البَحْرِ أَرْضَ الكَنْعَانِيِّ وَلُبْنَانَ إِلى النَّهْرِ الكَبِيرِ نَهْرِ الفُرَاتِ.
\par 8 اُنْظُرْ قَدْ جَعَلتُ أَمَامَكُمُ الأَرْضَ. ادْخُلُوا وَتَمَلكُوا الأَرْضَ التِي أَقْسَمَ الرَّبُّ لآِبَائِكُمْ إِبْرَاهِيمَ وَإِسْحَاقَ وَيَعْقُوبَ أَنْ يُعْطِيَهَا لهُمْ وَلِنَسْلِهِمْ مِنْ بَعْدِهِمْ.
\par 9 وَكَلمْتُكُمْ فِي ذَلِكَ الوَقْتِ قَائِلاً: لا أَقْدِرُ وَحْدِي أَنْ أَحْمِلكُمْ.
\par 10 اَلرَّبُّ إِلهُكُمْ قَدْ كَثَّرَكُمْ. وَهُوَذَا أَنْتُمُ اليَوْمَ كَنُجُومِ السَّمَاءِ فِي الكَثْرَةِ.
\par 11 الرَّبُّ إِلهُ آبَائِكُمْ يَزِيدُ عَليْكُمْ مِثْلكُمْ أَلفَ مَرَّةٍ وَيُبَارِكُكُمْ كَمَا كَلمَكُمْ.
\par 12 كَيْفَ أَحْمِلُ وَحْدِي ثِقْلكُمْ وَحِمْلكُمْ وَخُصُومَتَكُمْ؟
\par 13 هَاتُوا مِنْ أَسْبَاطِكُمْ رِجَالاً حُكَمَاءَ وَعُقَلاءَ وَمَعْرُوفِينَ فَأَجْعَلُهُمْ رُؤُوسَكُمْ.
\par 14 فَأَجَبْتُمُونِي: حَسَنٌ الأَمْرُ الذِي تَكَلمْتَ بِهِ أَنْ يُعْمَل.
\par 15 فَأَخَذْتُ رُؤُوسَ أَسْبَاطِكُمْ رِجَالاً حُكَمَاءَ وَمَعْرُوفِينَ وَجَعَلتُهُمْ رُؤُوساً عَليْكُمْ رُؤَسَاءَ أُلُوفٍ وَرُؤَسَاءَ مِئَاتٍ وَرُؤَسَاءَ خَمَاسِينَ وَرُؤَسَاءَ عَشَرَاتٍ وَعُرَفَاءَ لأَسْبَاطِكُمْ.
\par 16 وَأَمَرْتُ قُضَاتَكُمْ فِي ذَلِكَ الوَقْتِ قَائِلاً: اسْمَعُوا بَيْنَ إِخْوَتِكُمْ وَاقْضُوا بِالحَقِّ بَيْنَ الإِنْسَانِ وَأَخِيهِ وَنَزِيلِهِ.
\par 17 لا تَنْظُرُوا إِلى الوُجُوهِ فِي القَضَاءِ. لِلصَّغِيرِ كَالكَبِيرِ تَسْمَعُونَ. لا تَهَابُوا وَجْهَ إِنْسَانٍ لأَنَّ القَضَاءَ لِلهِ. وَالأَمْرُ الذِي يَعْسُرُ عَليْكُمْ تُقَدِّمُونَهُ إِليَّ لأَسْمَعَهُ.
\par 18 وَأَمَرْتُكُمْ فِي ذَلِكَ الوَقْتِ بِكُلِّ الأُمُورِ التِي تَعْمَلُونَهَا.
\par 19 «ثُمَّ ارْتَحَلنَا مِنْ حُورِيبَ وَسَلكْنَا كُل ذَلِكَ القَفْرِ العَظِيمِ المَخُوفِ الذِي رَأَيْتُمْ فِي طَرِيقِ جَبَلِ الأَمُورِيِّينَ كَمَا أَمَرَنَا الرَّبُّ إِلهُنَا. وَجِئْنَا إِلى قَادِشَ بَرْنِيعَ.
\par 20 فَقُلتُ لكُمْ: قَدْ جِئْتُمْ إِلى جَبَلِ الأَمُورِيِّينَ الذِي أَعْطَانَا الرَّبُّ إِلهُنَا.
\par 21 اُنْظُرْ. قَدْ جَعَل الرَّبُّ إِلهُكَ الأَرْضَ أَمَامَكَ. اصْعَدْ تَمَلكْ كَمَا كَلمَكَ الرَّبُّ إِلهُ آبَائِكَ! لا تَخَفْ وَلا تَرْتَعِبْ!
\par 22 فَتَقَدَّمْتُمْ إِليَّ جَمِيعُكُمْ وَقُلتُمْ: دَعْنَا نُرْسِل رِجَالاً قُدَّامَنَا لِيَتَجَسَّسُوا لنَا الأَرْضَ وَيَرُدُّوا إِليْنَا خَبَراً عَنِ الطَّرِيقِ التِي نَصْعَدُ فِيهَا وَالمُدُنِ التِي نَأْتِي إِليْهَا.
\par 23 فَحَسُنَ الكَلامُ لدَيَّ فَأَخَذْتُ مِنْكُمُ اثْنَيْ عَشَرَ رَجُلاً. رَجُلاً وَاحِداً مِنْ كُلِّ سِبْطٍ.
\par 24 فَانْصَرَفُوا وَصَعِدُوا إِلى الجَبَلِ وَأَتُوا إِلى وَادِي أَشْكُول وَتَجَسَّسُوهُ
\par 25 وَأَخَذُوا فِي أَيْدِيهِمْ مِنْ أَثْمَارِ الأَرْضِ وَنَزَلُوا بِهِ إِليْنَا وَرَدُّوا لنَا خَبَراً وَقَالُوا: جَيِّدَةٌ هِيَ الأَرْضُ التِي أَعْطَانَا الرَّبُّ إِلهُنَا.
\par 26 «لكِنَّكُمْ لمْ تَشَاءُوا أَنْ تَصْعَدُوا وَعَصَيْتُمْ قَوْل الرَّبِّ إِلهِكُمْ
\par 27 وَتَمَرْمَرْتُمْ فِي خِيَامِكُمْ وَقُلتُمُ: الرَّبُّ بِسَبَبِ بُغْضَتِهِ لنَا قَدْ أَخْرَجَنَا مِنْ أَرْضِ مِصْرَ لِيَدْفَعَنَا إِلى أَيْدِي الأَمُورِيِّينَ لِيُهْلِكَنَا.
\par 28 إِلى أَيْنَ نَحْنُ صَاعِدُونَ؟ قَدْ أَذَابَ إِخْوَتُنَا قُلُوبَنَا قَائِلِينَ: شَعْبٌ أَعْظَمُ وَأَطْوَلُ مِنَّا. مُدُنٌ عَظِيمَةٌ مُحَصَّنَةٌ إِلى السَّمَاءِ وَأَيْضاً قَدْ رَأَيْنَا بَنِي عَنَاقَ هُنَاكَ.
\par 29 فَقُلتُ لكُمْ: لا تَرْهَبُوا وَلا تَخَافُوا مِنْهُمُ!
\par 30 الرَّبُّ إِلهُكُمُ السَّائِرُ أَمَامَكُمْ هُوَ يُحَارِبُ عَنْكُمْ حَسَبَ كُلِّ مَا فَعَل مَعَكُمْ فِي مِصْرَ أَمَامَ أَعْيُنِكُمْ
\par 31 وَفِي البَرِّيَّةِ حَيْثُ رَأَيْتَ كَيْفَ حَمَلكَ الرَّبُّ إِلهُكَ كَمَا يَحْمِلُ الإِنْسَانُ ابْنَهُ فِي كُلِّ الطَّرِيقِ التِي سَلكْتُمُوهَا حَتَّى جِئْتُمْ إِلى هَذَا المَكَانِ.
\par 32 وَلكِنْ فِي هَذَا الأَمْرِ لسْتُمْ وَاثِقِينَ بِالرَّبِّ إِلهِكُمُ
\par 33 السَّائِرِ أَمَامَكُمْ فِي الطَّرِيقِ لِيَلتَمِسَ لكُمْ مَكَاناً لِنُزُولِكُمْ فِي نَارٍ ليْلاً لِيُرِيَكُمُ الطَّرِيقَ التِي تَسِيرُونَ فِيهَا وَفِي سَحَابٍ نَهَاراً.
\par 34 وَسَمِعَ الرَّبُّ صَوْتَ كَلامِكُمْ فَسَخِطَ وَأَقْسَمَ قَائِلاً:
\par 35 لنْ يَرَى إِنْسَانٌ مِنْ هَؤُلاءِ النَّاسِ مِنْ هَذَا الجِيلِ الشِّرِّيرِ الأَرْضَ الجَيِّدَةَ التِي أَقْسَمْتُ أَنْ أُعْطِيَهَا لآِبَائِكُمْ
\par 36 مَا عَدَا كَالِبَ بْنَ يَفُنَّةَ. هُوَ يَرَاهَا وَلهُ أُعْطِي الأَرْضَ التِي وَطِئَهَا وَلِبَنِيهِ لأَنَّهُ قَدِ اتَّبَعَ الرَّبَّ تَمَاماً.
\par 37 وَعَليَّ أَيْضاً غَضِبَ الرَّبُّ بِسَبَبِكُمْ قَائِلاً: وَأَنْتَ أَيْضاً لا تَدْخُلُ إِلى هُنَاكَ.
\par 38 يَشُوعُ بْنُ نُونَ الوَاقِفُ أَمَامَكَ هُوَ يَدْخُلُ إِلى هُنَاكَ. شَدِّدْهُ لأَنَّهُ هُوَ يَقْسِمُهَا لِإِسْرَائِيل.
\par 39 وَأَمَّا أَطْفَالُكُمُ الذِينَ قُلتُمْ يَكُونُونَ غَنِيمَةً وَبَنُوكُمُ الذِينَ لمْ يَعْرِفُوا اليَوْمَ الخَيْرَ وَالشَّرَّ فَهُمْ يَدْخُلُونَ إِلى هُنَاكَ وَلهُمْ أُعْطِيهَا وَهُمْ يَمْلِكُونَهَا.
\par 40 وَأَمَّا أَنْتُمْ فَتَحَوَّلُوا وَارْتَحِلُوا إِلى البَرِّيَّةِ عَلى طَرِيقِ بَحْرِ سُوفٍَ.
\par 41 «فَأَجَبْتُمْ: قَدْ أَخْطَأْنَا إِلى الرَّبِّ. نَحْنُ نَصْعَدُ وَنُحَارِبُ حَسَبَ كُلِّ مَا أَمَرَنَا الرَّبُّ إِلهُنَا. وَتَنَطَّقْتُمْ كُلُّ وَاحِدٍ بِعُدَّةِ حَرْبِهِ وَاسْتَخْفَفْتُمُ الصُّعُودَ إِلى الجَبَلِ.
\par 42 فَقَال الرَّبُّ لِي: قُل لهُمْ لا تَصْعَدُوا وَلا تُحَارِبُوا لأَنِّي لسْتُ فِي وَسَطِكُمْ لِئَلا تَنْكَسِرُوا أَمَامَ أَعْدَائِكُمْ.
\par 43 فَكَلمْتُكُمْ وَلمْ تَسْمَعُوا بَل عَصَيْتُمْ قَوْل الرَّبِّ وَطَغَيْتُمْ وَصَعِدْتُمْ إِلى الجَبَلِ.
\par 44 فَخَرَجَ الأَمُورِيُّونَ السَّاكِنُونَ فِي ذَلِكَ الجَبَلِ لِلِقَائِكُمْ وَطَرَدُوكُمْ كَمَا يَفْعَلُ النَّحْلُ وَكَسَرُوكُمْ فِي سَعِيرَ إِلى حُرْمَةَ.
\par 45 فَرَجَعْتُمْ وَبَكَيْتُمْ أَمَامَ الرَّبِّ وَلمْ يَسْمَعِ الرَّبُّ لِصَوْتِكُمْ وَلا أَصْغَى إِليْكُمْ.
\par 46 وَقَعَدْتُمْ فِي قَادِشَ أَيَّاماً كَثِيرَةً كَالأَيَّامِ التِي قَعَدْتُمْ فِيهَا».

\chapter{2}

\par 1 «ثُمَّ تَحَوَّلنَا وَارْتَحَلنَا إِلى البَرِّيَّةِ عَلى طَرِيقِ بَحْرِ سُوفَ كَمَا كَلمَنِي الرَّبُّ وَدُرْنَا بِجَبَلِ سَعِيرَ أَيَّاماً كَثِيرَةً.
\par 2 ثُمَّ كَلمَنِي الرَّبُّ:
\par 3 كَفَاكُمْ دَوَرَانٌ بِهَذَا الجَبَلِ. تَحَوَّلُوا نَحْوَ الشِّمَالِ.
\par 4 وَأَوْصِ الشَّعْبَ قَائِلاً: أَنْتُمْ مَارُّونَ بِتُخُمِ إِخْوَتِكُمْ بَنِي عِيسُو السَّاكِنِينَ فِي سَعِيرَ فَيَخَافُونَ مِنْكُمْ. فَاحْتَرِزُوا جِدّاً.
\par 5 لا تَهْجِمُوا عَليْهِمْ. لأَنِّي لا أُعْطِيكُمْ مِنْ أَرْضِهِمْ وَلا وَطْأَةَ قَدَمٍ لأَنِّي لِعِيسُو قَدْ أَعْطَيْتُ جَبَل سَعِيرَ مِيرَاثاً.
\par 6 طَعَاماً تَشْتَرُونَ مِنْهُمْ بِالفِضَّةِ لِتَأْكُلُوا وَمَاءً أَيْضاً تَبْتَاعُونَ مِنْهُمْ بِالفِضَّةِ لِتَشْرَبُوا.
\par 7 لأَنَّ الرَّبَّ إِلهَكَ قَدْ بَارَكَكَ فِي كُلِّ عَمَلِ يَدِكَ عَارِفاً مَسِيرَكَ فِي هَذَا القَفْرِ العَظِيمِ. الآنَ أَرْبَعُونَ سَنَةً لِلرَّبِّ إِلهِكَ مَعَكَ لمْ يَنْقُصْ عَنْكَ شَيْءٌ.
\par 8 فَعَبَرْنَا عَنْ إِخْوَتِنَا بَنِي عِيسُو السَّاكِنِينَ فِي سَعِيرَ عَلى طَرِيقِ العَرَبَةِ عَلى أَيْلةَ وَعَلى عِصْيُونَِ جَابِرَ ثُمَّ تَحَوَّلنَا وَمَرَرْنَا فِي طَرِيقِ بَرِّيَّةِ مُوآبَ.
\par 9 «فَقَال لِي الرَّبُّ: لا تُعَادِ مُوآبَ وَلا تُثِرْ عَليْهِمْ حَرْباً لأَنِّي لا أُعْطِيكَ مِنْ أَرْضِهِمْ مِيرَاثاً. لأَنِّي لِبَنِي لُوطٍَ قَدْ أَعْطَيْتُ «عَارَ» مِيرَاثاً.
\par 10 (الإِيمِيُّونَ سَكَنُوا فِيهَا قَبْلاً. شَعْبٌ كَبِيرٌ وَكَثِيرٌ وَطَوِيلٌ كَالعَنَاقِيِّينَ.
\par 11 هُمْ أَيْضاً يُحْسَبُونَ رَفَائِيِّينَ كَالعَنَاقِيِّينَ لكِنَّ المُوآبِيِّينَ يَدْعُونَهُمْ إِيمِيِّينَ.
\par 12 وَفِي سَعِيرَ سَكَنَ قَبْلاً الحُورِيُّونَ فَطَرَدَهُمْ بَنُو عِيسُو وَأَبَادُوهُمْ مِنْ قُدَّامِهِمْ وَسَكَنُوا مَكَانَهُمْ كَمَا فَعَل إِسْرَائِيلُ بِأَرْضِ مِيرَاثِهِمِ التِي أَعْطَاهُمُ الرَّبُّ).
\par 13 اَلآنَ قُومُوا وَاعْبُرُوا وَادِيَ زَارَدَ. فَعَبَرْنَا وَادِيَ زَارَدَ.
\par 14 وَالأَيَّامُ التِي سِرْنَا فِيهَا مِنْ قَادِشَ بَرْنِيعَ حَتَّى عَبَرْنَا وَادِيَ زَارَدَ كَانَتْ ثَمَانِيَ وَثَلاثِينَ سَنَةً حَتَّى فَنِيَ كُلُّ الجِيلِ رِجَالُ الحَرْبِ مِنْ وَسَطِ المَحَلةِ كَمَا أَقْسَمَ الرَّبُّ لهُمْ.
\par 15 وَيَدُ الرَّبِّ أَيْضاً كَانَتْ عَليْهِمْ لِإِبَادَتِهِمْ مِنْ وَسَطِ المَحَلةِ حَتَّى فَنُوا.
\par 16 «فَعِنْدَمَا فَنِيَ جَمِيعُ رِجَالِ الحَرْبِ بِالمَوْتِ مِنْ وَسَطِ الشَّعْبِ
\par 17 قَال لِي الرَّبُّ:
\par 18 أَنْتَ مَارٌّ اليَوْمَ بِتُخُمِ مُوآبَ بِعَارَ.
\par 19 فَمَتَى قَرُبْتَ إِلى تُجَاهِ بَنِي عَمُّونَ لا تُعَادِهِمْ وَلا تَهْجِمُوا عَليْهِمْ لأَنِّي لا أُعْطِيكَ مِنْ أَرْضِ بَنِي عَمُّونَ مِيرَاثاً - لأَنِّي لِبَنِي لُوطٍ قَدْ أَعْطَيْتُهَا مِيرَاثاً.
\par 20 (هِيَ أَيْضاً تُحْسَبُ أَرْضَ رَفَائِيِّينَ. سَكَنَ الرَّفَائِيُّونَ فِيهَا قَبْلاً لكِنَّ العَمُّونِيِّينَ يَدْعُونَهُمْ زَمْزُمِيِّينَ.
\par 21 شَعْبٌ كَبِيرٌ وَكَثِيرٌ وَطَوِيلٌ كَالعَنَاقِيِّينَ أَبَادَهُمُ الرَّبُّ مِنْ قُدَّامِهِمْ فَطَرَدُوهُمْ وَسَكَنُوا مَكَانَهُمْ.
\par 22 كَمَا فَعَل لِبَنِي عِيسُو السَّاكِنِينَ فِي سَعِيرَ الذِينَ أَتْلفَ الحُورِيِّينَ مِنْ قُدَّامِهِمْ فَطَرَدُوهُمْ وَسَكَنُوا مَكَانَهُمْ إِلى هَذَا اليَوْمِ.
\par 23 وَالعُوِّيُّونَ السَّاكِنُونَ فِي القُرَى إِلى غَزَّةَ أَبَادَهُمُ الكَفْتُورِيُّونَ الذِينَ خَرَجُوا مِنْ كَفْتُورَ وَسَكَنُوا مَكَانَهُمْ).
\par 24 قُومُوا ارْتَحِلُوا وَاعْبُرُوا وَادِيَ أَرْنُونَ. انْظُرْ. قَدْ دَفَعْتُ إِلى يَدِكَ سِيحُونَ مَلِكَ حَشْبُونَ الأَمُورِيَّ وَأَرْضَهُ. ابْتَدِئْ تَمَلكْ وَأَثِرْ عَليْهِ حَرْباً.
\par 25 فِي هَذَا اليَوْمِ أَبْتَدِئُ أَجْعَلُ خَشْيَتَكَ وَخَوْفَكَ أَمَامَ وُجُوهِ الشُّعُوبِ تَحْتَ كُلِّ السَّمَاءِ. الذِينَ يَسْمَعُونَ خَبَرَكَ يَرْتَعِدُونَ وَيَجْزَعُونَ أَمَامَكَ.
\par 26 «فَأَرْسَلتُ رُسُلاً مِنْ بَرِّيَّةِ قَدِيمُوتَ إِلى سِيحُونَ مَلِكِ حَشْبُونَ بِكَلامِ سَلامٍ قَائِلاً:
\par 27 أَمُرُّ فِي أَرْضِكَ. أَسْلُكُ الطَّرِيقَ الطَّرِيقَ. لا أَمِيلُ يَمِيناً وَلا شِمَالاً.
\par 28 طَعَاماً بِالفِضَّةِ تَبِيعُنِي لآِكُل وَمَاءً بِالفِضَّةِ تُعْطِينِي لأَشْرَبَ. أَمُرُّ بِرِجْليَّ فَقَطْ.
\par 29 كَمَا فَعَل بِي بَنُو عِيسُو السَّاكِنُونَ فِي سَعِيرَ وَالمُوآبِيُّونَ السَّاكِنُونَ فِي عَارَ إِلى أَنْ أَعْبُرَ الأُرْدُنَّ إِلى الأَرْضِ التِي أَعْطَانَا الرَّبُّ إِلهُنَا.
\par 30 لكِنْ لمْ يَشَأْ سِيحُونُ مَلِكُ حَشْبُونَ أَنْ يَدَعَنَا نَمُرَّ بِهِ لأَنَّ الرَّبَّ إِلهَكَ قَسَّى رُوحَهُ وَقَوَّى قَلبَهُ لِيَدْفَعَهُ إِلى يَدِكَ كَمَا فِي هَذَا اليَوْمِ.
\par 31 وَقَال الرَّبُّ لِي: انْظُرْ! قَدِ ابْتَدَأْتُ أَدْفَعُ أَمَامَكَ سِيحُونَ وَأَرْضَهُ. ابْتَدِئْ تَمَلكْ حَتَّى تَمْتَلِكَ أَرْضَهُ.
\par 32 فَخَرَجَ سِيحُونُ لِلِقَائِنَا هُوَ وَجَمِيعُ قَوْمِهِ لِلحَرْبِ إِلى يَاهَصَ
\par 33 فَدَفَعَهُ الرَّبُّ إِلهُنَا أَمَامَنَا فَضَرَبْنَاهُ وَبَنِيهِ وَجَمِيعَ قَوْمِهِ.
\par 34 وَأَخَذْنَا كُل مُدُنِهِ فِي ذَلِكَ الوَقْتِ وَحَرَّمْنَا مِنْ كُلِّ مَدِينَةٍ الرِّجَال وَالنِّسَاءَ وَالأَطْفَال. لمْ نُبْقِ شَارِداً.
\par 35 لكِنَّ البَهَائِمَ نَهَبْنَاهَا لأَنْفُسِنَا وَغَنِيمَةَ المُدُنِ التِي أَخَذْنَا
\par 36 مِنْ عَرُوعِيرَ التِي عَلى حَافَةِ وَادِي أَرْنُونَ وَالمَدِينَةِ التِي فِي الوَادِي إِلى جِلعَادَ لمْ تَكُنْ قَرْيَةٌ قَدِ امْتَنَعَتْ عَليْنَا. الجَمِيعُ دَفَعَهُ الرَّبُّ إِلهُنَا أَمَامَنَا.
\par 37 وَلكِنَّ أَرْضَ بَنِي عَمُّونَ لمْ نَقْرَبْهَا. كُل نَاحِيَةِ وَادِي يَبُّوقَ وَمُدُنَ الجَبَلِ وَكُل مَا أَوْصَى الرَّبُّ إِلهُنَا».

\chapter{3}

\par 1 «ثُمَّ تَحَوَّلنَا وَصَعِدْنَا فِي طَرِيقِ بَاشَانَ فَخَرَجَ عُوجُ مَلِكُ بَاشَانَ لِلِقَائِنَا هُوَ وَجَمِيعُ قَوْمِهِ لِلحَرْبِ فِي إِذْرَعِي.
\par 2 فَقَال لِي الرَّبُّ: لا تَخَفْ مِنْهُ لأَنِّي قَدْ دَفَعْتُهُ إِلى يَدِكَ وَجَمِيعَ قَوْمِهِ وَأَرْضِهِ فَتَفْعَلُ بِهِ كَمَا فَعَلتَ بِسِيحُونَ مَلِكِ الأَمُورِيِّينَ الذِي كَانَ سَاكِناً فِي حَشْبُونَ.
\par 3 فَدَفَعَ الرَّبُّ إِلهُنَا إِلى أَيْدِينَا عُوجَ أَيْضاً مَلِكَ بَاشَانَ وَجَمِيعَ قَوْمِهِ فَضَرَبْنَاهُ حَتَّى لمْ يَبْقَ لهُ شَارِدٌ.
\par 4 وَأَخَذْنَا كُل مُدُنِهِ فِي ذَلِكَ الوَقْتِ. لمْ تَكُنْ قَرْيَةٌ لمْ نَأْخُذْهَا مِنْهُمْ. سِتُّونَ مَدِينَةً كُلُّ كُورَةِ أَرْجُوبَ مَمْلكَةُ عُوجٍ فِي بَاشَانَ.
\par 5 كُلُّ هَذِهِ كَانَتْ مُدُناً مُحَصَّنَةً بِأَسْوَارٍ شَامِخَةٍ وَأَبْوَابٍ وَمَزَالِيجَ. سِوَى قُرَى الصَّحْرَاءِ الكَثِيرَةِ جِدّاً.
\par 6 فَحَرَّمْنَاهَا كَمَا فَعَلنَا بِسِيحُونَ مَلِكِ حَشْبُونَ مُحَرِّمِينَ كُل مَدِينَةٍ الرِّجَال: وَالنِّسَاءَ وَالأَطْفَال.
\par 7 لكِنَّ كُل البَهَائِمِ وَغَنِيمَةِ المُدُنِ نَهَبْنَاهَا لأَنْفُسِنَا.
\par 8 وَأَخَذْنَا فِي ذَلِكَ الوَقْتِ مِنْ يَدِ مَلِكَيِ الأَمُورِيِّينَ الأَرْضَ التِي فِي عَبْرِ الأُرْدُنِّ مِنْ وَادِي أَرْنُونَ إِلى جَبَلِ حَرْمُونَ.
\par 9 (وَالصَّيْدُونِيُّونَ يَدْعُونَ حَرْمُونَ سِرْيُونَ وَالأَمُورِيُّونَ يَدْعُونَهُ سَنِيرَ).
\par 10 كُل مُدُنِ السَّهْلِ وَكُل جِلعَادَ وَكُل بَاشَانَ إِلى سَلخَةَ وَإِذْرَعِي مَدِينَتَيْ مَمْلكَةِ عُوجٍ فِي بَاشَانَ.
\par 11 إِنَّ عُوجَ مَلِكَ بَاشَانَ وَحْدَهُ بَقِيَ مِنْ بَقِيَّةِ الرَّفَائِيِّينَ. هُوَذَا سَرِيرُهُ سَرِيرٌ مِنْ حَدِيدٍ. (أَليْسَ هُوَ فِي رَبَّةِ بَنِي عَمُّونَ؟) طُولُهُ تِسْعُ أَذْرُعٍ وَعَرْضُهُ أَرْبَعُ أَذْرُعٍ بِذِرَاعِ رَجُلٍ.
\par 12 فَهَذِهِ الأَرْضُ امْتَلكْنَاهَا فِي ذَلِكَ الوَقْتِ مِنْ عَرُوعِيرَ التِي عَلى وَادِي أَرْنُونَ وَنِصْفَ جَبَلِ جِلعَادَ وَمُدُنَهُ أَعْطَيْتُ لِلرَّأُوبَيْنِيِّينَ وَالجَادِيِّينَ.
\par 13 وَبَقِيَّةَ جِلعَادَ وَكُل بَاشَانَ مَمْلكَةَ عُوجٍَ أَعْطَيْتُ لِنِصْفِ سِبْطِ مَنَسَّى. (كُل كُورَةِ أَرْجُوبَ مَعَ كُلِّ بَاشَانَ وَهِيَ تُدْعَى أَرْضَ الرَّفَائِيِّينَ.
\par 14 يَائِيرُ بْنُ مَنَسَّى أَخَذَ كُل كُورَةِ أَرْجُوبَ إِلى تُخُمِ الجَشُورِيِّينَ وَالمَعْكِيِّينَ وَدَعَاهَا عَلى اسْمِهِ بَاشَانَ «حَوُّوثَ يَائِيرَ» إِلى هَذَا اليَوْمِ).
\par 15 وَلِمَاكِيرَ أَعْطَيْتُ جِلعَادَ.
\par 16 وَلِلرَّأُوبَيْنِيِّينَ وَالجَادِيِّينَ أَعْطَيْتُ مِنْ جِلعَادَ إِلى وَادِي أَرْنُونَ وَسَطَ الوَادِي تُخُماً. وَإِلى وَادِي يَبُّوقَ تُخُمِ بَنِي عَمُّونَ.
\par 17 وَالعَرَبَةَ وَالأُرْدُنَّ تُخُماً مِنْ كِنَّارَةَ إِلى بَحْرِ العَرَبَةِ (بَحْرِ المِلحِ) تَحْتَ سُفُوحِ الفِسْجَةِ نَحْوَ الشَّرْقِ.
\par 18 «وَأَمَرْتُكُمْ فِي ذَلِكَ الوَقْتِ قَائِلاً: الرَّبُّ إِلهُكُمْ قَدْ أَعْطَاكُمْ هَذِهِ الأَرْضَ لِتَمْتَلِكُوهَا. مُتَجَرِّدِينَ تَعْبُرُونَ أَمَامَ إِخْوَتِكُمْ بَنِي إِسْرَائِيل كُلُّ ذَوِي بَأْسٍ.
\par 19 أَمَّا نِسَاؤُكُمْ وَأَطْفَالُكُمْ وَمَوَاشِيكُمْ. (قَدْ عَرَفْتُ أَنَّ لكُمْ مَوَاشِيَ كَثِيرَةً) فَتَمْكُثُ فِي مُدُنِكُمُ التِي أَعْطَيْتُكُمْ
\par 20 حَتَّى يُرِيحَ الرَّبُّ إِخْوَتَكُمْ مِثْلكُمْ وَيَمْتَلِكُوا هُمْ أَيْضاً الأَرْضَ التِي الرَّبُّ إِلهُكُمْ يُعْطِيهِمْ فِي عَبْرِ الأُرْدُنِّ. ثُمَّ تَرْجِعُونَ كُلُّ وَاحِدٍ إِلى مُلكِهِ الذِي أَعْطَيْتُكُمْ.
\par 21 وَأَمَرْتُ يَشُوعَ فِي ذَلِكَ الوَقْتِ قَائِلاً: عَيْنَاكَ قَدْ أَبْصَرَتَا كُل مَا فَعَل الرَّبُّ إِلهُكُمْ بِهَذَيْنِ المَلِكَيْنِ. هَكَذَا يَفْعَلُ الرَّبُّ بِجَمِيعِ المَمَالِكِ التِي أَنْتَ عَابِرٌ إِليْهَا.
\par 22 لا تَخَافُوا مِنْهُمْ لأَنَّ الرَّبَّ إِلهَكُمْ هُوَ المُحَارِبُ عَنْكُمْ.
\par 23 «وَتَضَرَّعْتُ إِلى الرَّبِّ فِي ذَلِكَ الوَقْتِ قَائِلاً:
\par 24 يَا سَيِّدُ الرَّبُّ أَنْتَ قَدِ ابْتَدَأْتَ تُرِي عَبْدَكَ عَظَمَتَكَ وَيَدَكَ الشَّدِيدَةَ. فَإِنَّهُ أَيُّ إِلهٍ فِي السَّمَاءِ وَعَلى الأَرْضِ يَعْمَلُ كَأَعْمَالِكَ وَكَجَبَرُوتِكَ؟
\par 25 دَعْنِي أَعْبُرْ وَأَرَى الأَرْضَ الجَيِّدَةَ التِي فِي عَبْرِ الأُرْدُنِّ هَذَا الجَبَل الجَيِّدَ وَلُبْنَانَ.
\par 26 لكِنَّ الرَّبَّ غَضِبَ عَليَّ بِسَبَبِكُمْ وَلمْ يَسْمَعْ لِي بَل قَال لِي الرَّبُّ: كَفَاكَ! لا تَعُدْ تُكَلِّمُنِي أَيْضاً فِي هَذَا الأَمْرِ.
\par 27 اصْعَدْ إِلى رَأْسِ الفِسْجَةِ وَارْفَعْ عَيْنَيْكَ إِلى الغَرْبِ وَالشِّمَالِ وَالجَنُوبِ وَالشَّرْقِ وَانْظُرْ بِعَيْنَيْكَ لكِنْ لا تَعْبُرُ هَذَا الأُرْدُنَّ!
\par 28 وَأَمَّا يَشُوعُ فَأَوْصِهِ وَشَدِّدْهُ وَشَجِّعْهُ لأَنَّهُ هُوَ يَعْبُرُ أَمَامَ هَذَا الشَّعْبِ وَهُوَ يَقْسِمُ لهُمُ الأَرْضَ التِي تَرَاهَا.
\par 29 فَمَكَثْنَا فِي الجِوَاءِ مُقَابِل بَيْتِ فَغُورَ».

\chapter{4}

\par 1 «فَالآنَ يَا إِسْرَائِيلُ اسْمَعِ الفَرَائِضَ وَالأَحْكَامَ التِي أَنَا أُعَلِّمُكُمْ لِتَعْمَلُوهَا لِتَحْيُوا وَتَدْخُلُوا وَتَمْتَلِكُوا الأَرْضَ التِي الرَّبُّ إِلهُ آبَائِكُمْ يُعْطِيكُمْ.
\par 2 لا تَزِيدُوا عَلى الكَلامِ الذِي أَنَا أُوصِيكُمْ بِهِ وَلا تُنَقِّصُوا مِنْهُ لِتَحْفَظُوا وَصَايَا الرَّبِّ إِلهِكُمُ التِي أَنَا أُوصِيكُمْ بِهَا.
\par 3 أَعْيُنُكُمْ قَدْ أَبْصَرَتْ مَا فَعَلهُ الرَّبُّ بِبَعْل فَغُورَ. إِنَّ كُل مَنْ ذَهَبَ وَرَاءَ بَعْل فَغُورَ أَبَادَهُ الرَّبُّ إِلهُكُمْ مِنْ وَسَطِكُمْ.
\par 4 وَأَمَّا أَنْتُمُ المُلتَصِقُونَ بِالرَّبِّ إِلهِكُمْ فَجَمِيعُكُمْ أَحْيَاءٌ اليَوْمَ.
\par 5 اُنْظُرْ. قَدْ عَلمْتُكُمْ فَرَائِضَ وَأَحْكَاماً كَمَا أَمَرَنِي الرَّبُّ إِلهِي لِتَعْمَلُوا هَكَذَا فِي الأَرْضِ التِي أَنْتُمْ دَاخِلُونَ إِليْهَا لِتَمْتَلِكُوهَا.
\par 6 فَاحْفَظُوا وَاعْمَلُوا. لأَنَّ ذَلِكَ حِكْمَتُكُمْ وَفِطْنَتُكُمْ أَمَامَ أَعْيُنِ الشُّعُوبِ الذِينَ يَسْمَعُونَ كُل هَذِهِ الفَرَائِضِ فَيَقُولُونَ: هَذَا الشَّعْبُ العَظِيمُ إِنَّمَا هُوَ شَعْبٌ حَكِيمٌ وَفَطِنٌ.
\par 7 لأَنَّهُ أَيُّ شَعْبٍ هُوَ عَظِيمٌ لهُ آلِهَةٌ قَرِيبَةٌ مِنْهُ كَالرَّبِّ إِلهِنَا فِي كُلِّ أَدْعِيَتِنَا إِليْهِ؟
\par 8 وَأَيُّ شَعْبٍ هُوَ عَظِيمٌ لهُ فَرَائِضُ وَأَحْكَامٌ عَادِلةٌ مِثْلُ كُلِّ هَذِهِ الشَّرِيعَةِ التِي أَنَا وَاضِعٌ أَمَامَكُمُ اليَوْمَ؟
\par 9 «إِنَّمَا احْتَرِزْ وَاحْفَظْ نَفْسَكَ جِدّاً لِئَلا تَنْسَى الأُمُورَ التِي أَبْصَرَتْ عَيْنَاكَ وَلِئَلا تَزُول مِنْ قَلبِكَ كُل أَيَّامِ حَيَاتِكَ. وَعَلِّمْهَا أَوْلادَكَ وَأَوْلادَ أَوْلادِكَ.
\par 10 فِي اليَوْمِ الذِي وَقَفْتَ فِيهِ أَمَامَ الرَّبِّ إِلهِكَ فِي حُورِيبَ حِينَ قَال لِي الرَّبُّ: اجْمَعْ لِي الشَّعْبَ فَأُسْمِعَهُمْ كَلامِي لِيَتَعَلمُوا أَنْ يَخَافُونِي كُل الأَيَّامِ التِي هُمْ فِيهَا أَحْيَاءٌ عَلى الأَرْضِ وَيُعَلِّمُوا أَوْلادَهُمْ.
\par 11 فَتَقَدَّمْتُمْ وَوَقَفْتُمْ فِي أَسْفَلِ الجَبَلِ وَالجَبَلُ يَضْطَرِمُ بِالنَّارِ إِلى كَبِدِ السَّمَاءِ بِظَلامٍ وَسَحَابٍ وَضَبَابٍ.
\par 12 فَكَلمَكُمُ الرَّبُّ مِنْ وَسَطِ النَّارِ وَأَنْتُمْ سَامِعُونَ صَوْتَ كَلامٍ وَلكِنْ لمْ تَرُوا صُورَةً بَل صَوْتاً.
\par 13 وَأَخْبَرَكُمْ بِعَهْدِهِ الذِي أَمَرَكُمْ أَنْ تَعْمَلُوا بِهِ الكَلِمَاتِ العَشَرِ وَكَتَبَهُ عَلى لوْحَيْ حَجَرٍ.
\par 14 وَإِيَّايَ أَمَرَ الرَّبُّ فِي ذَلِكَ الوَقْتِ أَنْ أُعَلِّمَكُمْ فَرَائِضَ وَأَحْكَاماً لِتَعْمَلُوهَا فِي الأَرْضِ التِي أَنْتُمْ عَابِرُونَ إِليْهَا لِتَمْتَلِكُوهَا.
\par 15 فَاحْتَفِظُوا جِدّاً لأَنْفُسِكُمْ. فَإِنَّكُمْ لمْ تَرُوا صُورَةً مَا يَوْمَ كَلمَكُمُ الرَّبُّ فِي حُورِيبَ مِنْ وَسَطِ النَّارِ.
\par 16 لِئَلا تَفْسُدُوا وَتَعْمَلُوا لأَنْفُسِكُمْ تِمْثَالاً مَنْحُوتاً صُورَةَ مِثَالٍ مَا شِبْهَ ذَكَرٍ أَوْ أُنْثَى
\par 17 شِبْهَ بَهِيمَةٍ مَا مِمَّا عَلى الأَرْضِ شِبْهَ طَيْرٍ مَا ذِي جَنَاحٍ مِمَّا يَطِيرُ فِي السَّمَاءِ
\par 18 شِبْهَ دَبِيبٍ مَا عَلى الأَرْضِ شِبْهَ سَمَكٍ مَا مِمَّا فِي المَاءِ مِنْ تَحْتِ الأَرْضِ.
\par 19 وَلِئَلا تَرْفَعَ عَيْنَيْكَ إِلى السَّمَاءِ وَتَنْظُرَ الشَّمْسَ وَالقَمَرَ وَالنُّجُومَ كُل جُنْدِ السَّمَاءِ التِي قَسَمَهَا الرَّبُّ إِلهُكَ لِجَمِيعِ الشُّعُوبِ التِي تَحْتَ كُلِّ السَّمَاءِ فَتَغْتَرَّ وَتَسْجُدَ لهَا وَتَعْبُدَهَا.
\par 20 وَأَنْتُمْ قَدْ أَخَذَكُمُ الرَّبُّ وَأَخْرَجَكُمْ مِنْ كُورِ الحَدِيدِ مِنْ مِصْرَ لِتَكُونُوا لهُ شَعْبَ مِيرَاثٍ كَمَا فِي هَذَا اليَوْمِ.
\par 21 وَغَضِبَ الرَّبُّ عَليَّ بِسَبَبِكُمْ وَأَقْسَمَ إِنِّي لا أَعْبُرُ الأُرْدُنَّ وَلا أَدْخُلُ الأَرْضَ الجَيِّدَةَ التِي الرَّبُّ إِلهُكَ يُعْطِيكَ نَصِيباً.
\par 22 فَأَمُوتُ أَنَا فِي هَذِهِ الأَرْضِ. لا أَعْبُرُ الأُرْدُنَّ. وَأَمَّا أَنْتُمْ فَتَعْبُرُونَ وَتَمْتَلِكُونَ تِلكَ الأَرْضَ الجَيِّدَةَ.
\par 23 اِحْتَرِزُوا مِنْ أَنْ تَنْسُوا عَهْدَ الرَّبِّ إِلهِكُمُ الذِي قَطَعَهُ مَعَكُمْ وَتَصْنَعُوا لأَنْفُسِكُمْ تِمْثَالاً مَنْحُوتاً صُورَةَ كُلِّ مَا نَهَاكَ عَنْهُ الرَّبُّ إِلهُكَ.
\par 24 لأَنَّ الرَّبَّ إِلهَكَ هُوَ نَارٌ آكِلةٌ إِلهٌ غَيُورٌ.
\par 25 «إِذَا وَلدْتُمْ أَوْلاداً وَأَحفَاداً وَأَطَلتُمُ الزَّمَانَ فِي الأَرْضِ وَفَسَدْتُمْ وَصَنَعْتُمْ تِمْثَالاً مَنْحُوتاً صُورَةَ شَيْءٍ مَا وَفَعَلتُمُ الشَّرَّ فِي عَيْنَيِ الرَّبِّ إِلهِكُمْ لِإِغَاظَتِهِ
\par 26 أُشْهِدُ عَليْكُمُ اليَوْمَ السَّمَاءَ وَالأَرْضَ أَنَّكُمْ تَبِيدُونَ سَرِيعاً عَنِ الأَرْضِ التِي أَنْتُمْ عَابِرُونَ الأُرْدُنَّ إِليْهَا لِتَمْتَلِكُوهَا. لا تُطِيلُونَ الأَيَّامَ عَليْهَا بَل تَهْلِكُونَ لا مَحَالةَ.
\par 27 وَيُبَدِّدُكُمُ الرَّبُّ فِي الشُّعُوبِ فَتَبْقُونَ عَدَداً قَلِيلاً بَيْنَ الأُمَمِ التِي يَسُوقُكُمُ الرَّبُّ إِليْهَا.
\par 28 وَتَصْنَعُونَ هُنَاكَ آلِهَةً صَنْعَةَ أَيْدِي النَّاسِ مِنْ خَشَبٍ وَحَجَرٍ مِمَّا لا يُبْصِرُ وَلا يَسْمَعُ وَلا يَأْكُلُ وَلا يَشُمُّ.
\par 29 ثُمَّ إِنْ طَلبْتَ مِنْ هُنَاكَ الرَّبَّ إِلهَكَ تَجِدْهُ إِذَا التَمَسْتَهُ بِكُلِّ قَلبِكَ وَبِكُلِّ نَفْسِكَ.
\par 30 عِنْدَمَا ضُيِّقَ عَليْكَ وَأَصَابَتْكَ كُلُّ هَذِهِ الأُمُورِ فِي آخِرِ الأَيَّامِ تَرْجِعُ إِلى الرَّبِّ إِلهِكَ وَتَسْمَعُ لِقَوْلِهِ
\par 31 لأَنَّ الرَّبَّ إِلهَكَ إِلهٌ رَحِيمٌ لا يَتْرُكُكَ وَلا يُهْلِكُكَ وَلا يَنْسَى عَهْدَ آبَائِكَ الذِي أَقْسَمَ لهُمْ عَليْهِ.
\par 32 «فَاسْأَل عَنِ الأَيَّامِ الأُولى التِي كَانَتْ قَبْلكَ مِنَ اليَوْمِ الذِي خَلقَ اللهُ فِيهِ الإِنْسَانَ عَلى الأَرْضِ وَمِنْ أَقْصَاءِ السَّمَاءِ إِلى أَقْصَائِهَا. هَل جَرَى مِثْلُ هَذَا الأَمْرِ العَظِيمِ أَوْ هَل سُمِعَ نَظِيرُهُ؟
\par 33 هَل سَمِعَ شَعْبٌ صَوْتَ اللهِ يَتَكَلمُ مِنْ وَسَطِ النَّارِ كَمَا سَمِعْتَ أَنْتَ وَعَاشَ؟
\par 34 أَوْ هَل شَرَعَ اللهُ أَنْ يَأْتِيَ وَيَأْخُذَ لِنَفْسِهِ شَعْباً مِنْ وَسَطِ شَعْبٍ بِتَجَارِبَ وَآيَاتٍ وَعَجَائِبَ وَحَرْبٍ وَيَدٍ شَدِيدَةٍ وَذِرَاعٍ رَفِيعَةٍ وَمَخَاوِفَ عَظِيمَةٍ مِثْل كُلِّ مَا فَعَل لكُمُ الرَّبُّ إِلهُكُمْ فِي مِصْرَ أَمَامَ أَعْيُنِكُمْ؟
\par 35 إِنَّكَ قَدْ أُرِيتَ لِتَعْلمَ أَنَّ الرَّبَّ هُوَ الإِلهُ. ليْسَ آخَرَ سِوَاهُ.
\par 36 مِنَ السَّمَاءِ أَسْمَعَكَ صَوْتَهُ لِيُنْذِرَكَ وَعَلى الأَرْضِ أَرَاكَ نَارَهُ العَظِيمَةَ وَسَمِعْتَ كَلامَهُ مِنْ وَسَطِ النَّارِ.
\par 37 وَلأَجْلِ أَنَّهُ أَحَبَّ آبَاءَكَ وَاخْتَارَ نَسْلهُمْ مِنْ بَعْدِهِمْ أَخْرَجَكَ بِحَضْرَتِهِ بِقُوَّتِهِ العَظِيمَةِ مِنْ مِصْرَ
\par 38 لِيَطْرُدَ مِنْ أَمَامِكَ شُعُوباً أَكْبَرَ وَأَعْظَمَ مِنْكَ وَيَأْتِيَ بِكَ وَيُعْطِيَكَ أَرْضَهُمْ نَصِيباً كَمَا فِي هَذَا اليَوْمِ.
\par 39 فَاعْلمِ اليَوْمَ وَرَدِّدْ فِي قَلبِكَ أَنَّ الرَّبَّ هُوَ الإِلهُ فِي السَّمَاءِ مِنْ فَوْقُ وَعَلى الأَرْضِ مِنْ أَسْفَلُ. ليْسَ سِوَاهُ.
\par 40 وَاحْفَظْ فَرَائِضَهُ وَوَصَايَاهُ التِي أَنَا أُوصِيكَ بِهَا اليَوْمَ لِيُحْسَنَ إِليْكَ وَإِلى أَوْلادِكَ مِنْ بَعْدِكَ وَلِتُطِيل أَيَّامَكَ عَلى الأَرْضِ التِي الرَّبُّ إِلهُكَ يُعْطِيكَ إِلى الأَبَدِ».
\par 41 حِينَئِذٍ أَفْرَزَ مُوسَى ثَلاثَ مُدُنٍ فِي عَبْرِ الأُرْدُنِّ نَحْوَ شُرُوقِ الشَّمْسِ
\par 42 لِكَيْ يَهْرُبَ إِليْهَا القَاتِلُ الذِي يَقْتُلُ صَاحِبَهُ بِغَيْرِ عِلمٍ وَهُوَ غَيْرُ مُبْغِضٍ لهُ مُنْذُ أَمْسِ وَمَا قَبْلهُ. يَهْرُبُ إِلى إِحْدَى تِلكَ المُدُنِ فَيَحْيَا.
\par 43 بَاصَرَ فِي البَرِّيَّةِ فِي أَرْضِ السَّهْلِ لِلرَّأُوبَيْنِيِّينَ وَرَامُوتَ فِي جِلعَادَ لِلجَادِيِّينَ وَجُولانَ فِي بَاشَانَ لِلمَنَسِّيِّينَ.
\par 44 وَهَذِهِ هِيَ الشَّرِيعَةُ التِي وَضَعَهَا مُوسَى أَمَامَ بَنِي إِسْرَائِيل.
\par 45 هَذِهِ هِيَ الشَّهَادَاتُ وَالفَرَائِضُ وَالأَحْكَامُ التِي كَلمَ بِهَا مُوسَى بَنِي إِسْرَائِيل عِنْدَ خُرُوجِهِمْ مِنْ مِصْرَ
\par 46 فِي عَبْرِ الأُرْدُنِّ فِي الجِوَاءِ مُقَابِل بَيْتِ فَغُورَ فِي أَرْضِ سِيحُونَ مَلِكِ الأَمُورِيِّينَ الذِي كَانَ سَاكِناً فِي حَشْبُونَ الذِي ضَرَبَهُ مُوسَى وَبَنُو إِسْرَائِيل عِنْدَ خُرُوجِهِمْ مِنْ مِصْرَ
\par 47 وَامْتَلكُوا أَرْضَهُ وَأَرْضَ عُوجٍَ مَلِكِ بَاشَانَ مَلِكَيِ الأَمُورِيِّينَ اللذَيْنِ فِي عَبْرِ الأُرْدُنِّ نَحْوَ شُرُوقِ الشَّمْسِ.
\par 48 مِنْ عَرُوعِيرَ التِي عَلى حَافَةِ وَادِي أَرْنُونَ إِلى جَبَلِ سِيئُونَ (الذِي هُوَ حَرْمُونُ)
\par 49 وَكُل العَرَبَةِ فِي عَبْرِ الأُرْدُنِّ نَحْوَ الشُّرُوقِ إِلى بَحْرِ العَرَبَةِ تَحْتَ سُفُوحِ الفِسْجَةِ.

\chapter{5}

\par 1 وَدَعَا مُوسَى جَمِيعَ إِسْرَائِيل وَقَال لهُمْ: «اِسْمَعْ يَا إِسْرَائِيلُ الفَرَائِضَ وَالأَحْكَامَ التِي أَتَكَلمُ بِهَا فِي مَسَامِعِكُمُ اليَوْمَ وَتَعَلمُوهَا وَاحْتَرِزُوا لِتَعْمَلُوهَا.
\par 2 اَلرَّبُّ إِلهُنَا قَطَعَ مَعَنَا عَهْداً فِي حُورِيبَ.
\par 3 ليْسَ مَعَ آبَائِنَا قَطَعَ الرَّبُّ هَذَا العَهْدَ بَل مَعَنَا نَحْنُ الذِينَ هُنَا اليَوْمَ جَمِيعُنَا أَحْيَاءٌ.
\par 4 وَجْهاً لِوَجْهٍ تَكَلمَ الرَّبُّ مَعَنَا فِي الجَبَلِ مِنْ وَسَطِ النَّارِ.
\par 5 أَنَا كُنْتُ وَاقِفاً بَيْنَ الرَّبِّ وَبَيْنَكُمْ فِي ذَلِكَ الوَقْتِ لِأُخْبِرَكُمْ بِكَلامِ الرَّبِّ لأَنَّكُمْ خِفْتُمْ مِنْ أَجْلِ النَّارِ وَلمْ تَصْعَدُوا إِلى الجَبَلِ. فَقَال:
\par 6 أَنَا هُوَ الرَّبُّ إِلهُكَ الذِي أَخْرَجَكَ مِنْ أَرْضِ مِصْرَ مِنْ بَيْتِ العُبُودِيَّةِ.
\par 7 لا يَكُنْ لكَ آلِهَةٌ أُخْرَى أَمَامِي.
\par 8 لا تَصْنَعْ لكَ تِمْثَالاً مَنْحُوتاً صُورَةً مَا مِمَّا فِي السَّمَاءِ مِنْ فَوْقُ وَمَا فِي الأَرْضِ مِنْ أَسْفَلُ وَمَا فِي المَاءِ مِنْ تَحْتِ الأَرْضِ.
\par 9 لا تَسْجُدْ لهُنَّ وَلا تَعْبُدْهُنَّ لأَنِّي أَنَا الرَّبُّ إِلهُكَ إِلهٌ غَيُورٌ أَفْتَقِدُ ذُنُوبَ الآبَاءِ فِي الأَبْنَاءِ وَفِي الجِيلِ الثَّالِثِ وَالرَّابِعِ مِنَ الذِينَ يُبْغِضُونَنِي
\par 10 وَأَصْنَعُ إِحْسَاناً إِلى أُلُوفٍ مِنْ مُحِبِّيَّ وَحَافِظِي وَصَايَايَ.
\par 11 لا تَنْطِقْ بِاسْمِ الرَّبِّ إِلهِكَ بَاطِلاً لأَنَّ الرَّبَّ لا يُبْرِئُ مَنْ نَطَقَ بِاسْمِهِ بَاطِلاً.
\par 12 اِحْفَظْ يَوْمَ السَّبْتِ لِتُقَدِّسَهُ كَمَا أَوْصَاكَ الرَّبُّ إِلهُكَ.
\par 13 سِتَّةَ أَيَّامٍ تَشْتَغِلُ وَتَعْمَلُ جَمِيعَ أَعْمَالِكَ
\par 14 وَأَمَّا اليَوْمُ السَّابِعُ فَسَبْتٌ لِلرَّبِّ إِلهِكَ لا تَعْمَل فِيهِ عَمَلاً مَا أَنْتَ وَابْنُكَ وَابْنَتُكَ وَعَبْدُكَ وَأَمَتُكَ وَثَوْرُكَ وَحِمَارُكَ وَكُلُّ بَهَائِمِكَ وَنَزِيلُكَ الذِي فِي أَبْوَابِكَ لِيَسْتَرِيحَ عَبْدُكَ وَأَمَتُكَ مِثْلكَ.
\par 15 وَاذْكُرْ أَنَّكَ كُنْتَ عَبْداً فِي أَرْضِ مِصْرَ فَأَخْرَجَكَ الرَّبُّ إِلهُكَ مِنْ هُنَاكَ بِيَدٍ شَدِيدَةٍ وَذِرَاعٍ مَمْدُودَةٍ. لأَجْلِ ذَلِكَ أَوْصَاكَ الرَّبُّ إِلهُكَ أَنْ تَحْفَظَ يَوْمَ السَّبْتِ.
\par 16 أَكْرِمْ أَبَاكَ وَأُمَّكَ كَمَا أَوْصَاكَ الرَّبُّ إِلهُكَ لِتَطُول أَيَّامُكَ وَلِيَكُونَ لكَ خَيْرٌ على الأَرْضِ التِي يُعْطِيكَ الرَّبُّ إِلهُكَ.
\par 17 لا تَقْتُل
\par 18 وَلا تَزْن�
\par 19 وَلا تَسْرِقْ
\par 20 وَلا تَشْهَدْ عَلى قَرِيبِكَ شَهَادَةَ زُورٍ
\par 21 وَلا تَشْتَهِ امْرَأَةَ قَرِيبِكَ وَلا تَشْتَهِ بَيْتَ قَرِيبِكَ وَلا حَقْلهُ وَلا عَبْدَهُ وَلا أَمَتَهُ وَلا ثَوْرَهُ وَلا حِمَارَهُ وَلا كُل مَا لِقَرِيبِكَ.
\par 22 هَذِهِ الكَلِمَاتُ كَلمَ بِهَا الرَّبُّ كُل جَمَاعَتِكُمْ فِي الجَبَلِ مِنْ وَسَطِ النَّارِ وَالسَّحَابِ وَالضَّبَابِ وَصَوْتٍ عَظِيمٍ وَلمْ يَزِدْ. وَكَتَبَهَا عَلى لوْحَيْنِ مِنْ حَجَرٍ وَأَعْطَانِي إِيَّاهَا.
\par 23 «فَلمَّا سَمِعْتُمُ الصَّوْتَ مِنْ وَسَطِ الظَّلامِ وَالجَبَلُ يَشْتَعِلُ بِالنَّارِ تَقَدَّمْتُمْ إِليَّ جَمِيعُ رُؤَسَاءِ أَسْبَاطِكُمْ وَشُيُوخُكُمْ
\par 24 وَقُلتُمْ: هُوَذَا الرَّبُّ إِلهُنَا قَدْ أَرَانَا مَجْدَهُ وَعَظَمَتَهُ وَسَمِعْنَا صَوْتَهُ مِنْ وَسَطِ النَّارِ. هَذَا اليَوْمَ قَدْ رَأَيْنَا أَنَّ اللهَ يُكَلِّمُ الإِنْسَانَ وَيَحْيَا.
\par 25 وَأَمَّا الآنَ فَلِمَاذَا نَمُوتُ؟ لأَنَّ هَذِهِ النَّارَ العَظِيمَةَ تَأْكُلُنَا. إِنْ عُدْنَا نَسْمَعُ صَوْتَ الرَّبِّ إِلهِنَا أَيْضاً نَمُوتُ!
\par 26 لأَنَّهُ مَنْ هُوَ مِنْ جَمِيعِ البَشَرِ الذِي سَمِعَ صَوْتَ اللهِ الحَيِّ يَتَكَلمُ مِنْ وَسَطِ النَّارِ مِثْلنَا وَعَاشَ؟
\par 27 تَقَدَّمْ أَنْتَ وَاسْمَعْ كُل مَا يَقُولُ لكَ الرَّبُّ إِلهُنَا وَكَلِّمْنَا بِكُلِّ مَا يُكَلِّمُكَ بِهِ الرَّبُّ إِلهُنَا فَنَسْمَعَ وَنَعْمَل.
\par 28 فَسَمِعَ الرَّبُّ صَوْتَ كَلامِكُمْ حِينَ كَلمْتُمُونِي وَقَال لِي الرَّبُّ: سَمِعْتُ صَوْتَ كَلامِ هَؤُلاءِ الشَّعْبِ الذِي كَلمُوكَ بِهِ. قَدْ أَحْسَنُوا فِي كُلِّ مَا تَكَلمُوا.
\par 29 يَا ليْتَ قَلبَهُمْ كَانَ هَكَذَا فِيهِمْ حَتَّى يَتَّقُونِي وَيَحْفَظُوا جَمِيعَ وَصَايَايَ كُل الأَيَّامِ لِيَكُونَ لهُمْ وَلأَوْلادِهِمْ خَيْرٌ إِلى الأَبَدِ.
\par 30 اِذْهَبْ قُل لهُمْ: ارْجِعُوا إِلى خِيَامِكُمْ.
\par 31 وَأَمَّا أَنْتَ فَقِفْ هُنَا مَعِي فَأُكَلِّمَكَ بِجَمِيعِ الوَصَايَا وَالفَرَائِضِ وَالأَحْكَامِ التِي تُعَلِّمُهُمْ فَيَعْمَلُونَهَا فِي الأَرْضِ التِي أَنَا أُعْطِيهِمْ لِيَمْتَلِكُوهَا.
\par 32 فَاحْتَرِزُوا لِتَعْمَلُوا كَمَا أَمَرَكُمُ الرَّبُّ إِلهُكُمْ. لا تَزِيغُوا يَمِيناً وَلا يَسَاراً.
\par 33 فِي جَمِيعِ الطَّرِيقِ التِي أَوْصَاكُمْ بِهَا الرَّبُّ إِلهُكُمْ تَسْلُكُونَ لِتَحْيُوا وَيَكُونَ لكُمْ خَيْرٌ وَتُطِيلُوا الأَيَّامَ فِي الأَرْضِ التِي تَمْتَلِكُونَهَا».

\chapter{6}

\par 1 «وَهَذِهِ هِيَ الوَصَايَا وَالفَرَائِضُ وَالأَحْكَامُ التِي أَمَرَ الرَّبُّ إِلهُكُمْ أَنْ أُعَلِّمَكُمْ لِتَعْمَلُوهَا فِي الأَرْضِ التِي أَنْتُمْ عَابِرُونَ إِليْهَا لِتَمْتَلِكُوهَا
\par 2 لِتَتَّقِيَ الرَّبَّ إِلهَكَ وَتَحْفَظَ جَمِيعَ فَرَائِضِهِ وَوَصَايَاهُ التِي أَنَا أُوصِيكَ بِهَا أَنْتَ وَابْنُكَ وَابْنُ ابْنِكَ كُل أَيَّامِ حَيَاتِكَ وَلِتَطُول أَيَّامُكَ.
\par 3 فَاسْمَعْ يَا إِسْرَائِيلُ وَاحْتَرِزْ لِتَعْمَل لِيَكُونَ لكَ خَيْرٌ وَتَكْثُرَ جِدّاً كَمَا كَلمَكَ الرَّبُّ إِلهُ آبَائِكَ فِي أَرْضٍ تَفِيضُ لبَناً وَعَسَلاً.
\par 4 «إِسْمَعْ يَا إِسْرَائِيلُ: الرَّبُّ إِلهُنَا رَبٌّ وَاحِدٌ.
\par 5 فَتُحِبُّ الرَّبَّ إِلهَكَ مِنْ كُلِّ قَلبِكَ وَمِنْ كُلِّ نَفْسِكَ وَمِنْ كُلِّ قُوَّتِكَ.
\par 6 وَلتَكُنْ هَذِهِ الكَلِمَاتُ التِي أَنَا أُوصِيكَ بِهَا اليَوْمَ عَلى قَلبِكَ
\par 7 وَقُصَّهَا عَلى أَوْلادِكَ وَتَكَلمْ بِهَا حِينَ تَجْلِسُ فِي بَيْتِكَ وَحِينَ تَمْشِي فِي الطَّرِيقِ وَحِينَ تَنَامُ وَحِينَ تَقُومُ
\par 8 وَارْبُطْهَا عَلامَةً عَلى يَدِكَ وَلتَكُنْ عَصَائِبَ بَيْنَ عَيْنَيْكَ
\par 9 وَاكْتُبْهَا عَلى قَوَائِمِ أَبْوَابِ بَيْتِكَ وَعَلى أَبْوَابِكَ.
\par 10 «وَمَتَى أَتَى بِكَ الرَّبُّ إِلهُكَ إِلى الأَرْضِ التِي حَلفَ لآِبَائِكَ إِبْرَاهِيمَ وَإِسْحَاقَ وَيَعْقُوبَ أَنْ يُعْطِيَكَ إِلى مُدُنٍ عَظِيمَةٍ جَيِّدَةٍ لمْ تَبْنِهَا
\par 11 وَبُيُوتٍ مَمْلُوءَةٍ كُل خَيْرٍ لمْ تَمْلأْهَا وَآبَارٍ مَحْفُورَةٍ لمْ تَحْفُرْهَا وَكُرُومٍ وَزَيْتُونٍ لمْ تَغْرِسْهَا وَأَكَلتَ وَشَبِعْتَ
\par 12 فَاحْتَرِزْ لِئَلا تَنْسَى الرَّبَّ الذِي أَخْرَجَكَ مِنْ أَرْضِ مِصْرَ مِنْ بَيْتِ العُبُودِيَّةِ.
\par 13 الرَّبَّ إِلهَكَ تَتَّقِي وَإِيَّاهُ تَعْبُدُ وَبِاسْمِهِ تَحْلِفُ.
\par 14 لا تَسِيرُوا وَرَاءَ آلِهَةٍ أُخْرَى مِنْ آلِهَةِ الأُمَمِ التِي حَوْلكُمْ
\par 15 لأَنَّ الرَّبَّ إِلهَكُمْ إِلهٌ غَيُورٌ فِي وَسَطِكُمْ لِئَلا يَحْمَى غَضَبُ الرَّبِّ إِلهِكُمْ عَليْكُمْ فَيُبِيدَكُمْ عَنْ وَجْهِ الأَرْضِ.
\par 16 لا تُجَرِّبُوا الرَّبَّ إِلهَكُمْ كَمَا جَرَّبْتُمُوهُ فِي مَسَّةَ.
\par 17 احْفَظُوا وَصَايَا الرَّبِّ إِلهِكُمْ وَشَهَادَاتِهِ وَفَرَائِضِهِ التِي أَوْصَاكُمْ بِهَا.
\par 18 وَاعْمَلِ الصَّالِحَ وَالحَسَنَ فِي عَيْنَيِ الرَّبِّ لِيَكُونَ لكَ خَيْرٌ وَتَدْخُل وَتَمْتَلِكَ الأَرْضَ الجَيِّدَةَ التِي حَلفَ الرَّبُّ لآِبَائِكَ
\par 19 أَنْ يَنْفِيَ جَمِيعَ أَعْدَائِكَ مِنْ أَمَامِكَ. كَمَا تَكَلمَ الرَّبُّ.
\par 20 «إِذَا سَأَلكَ ابْنُكَ غَداً: مَا هِيَ الشَّهَادَاتُ وَالفَرَائِضُ وَالأَحْكَامُ التِي أَوْصَاكُمْ بِهَا الرَّبُّ إِلهُنَا؟
\par 21 تَقُولُ لاِبْنِكَ: كُنَّا عَبِيداً لِفِرْعَوْنَ فِي مِصْرَ فَأَخْرَجَنَا الرَّبُّ مِنْ مِصْرَ بِيَدٍ شَدِيدَةٍ
\par 22 وَصَنَعَ الرَّبُّ آيَاتٍ وَعَجَائِبَ عَظِيمَةً وَرَدِيئَةً بِمِصْرَ بِفِرْعَوْنَ وَجَمِيعِ بَيْتِهِ أَمَامَ أَعْيُنِنَا
\par 23 وَأَخْرَجَنَا مِنْ هُنَاكَ لِيَأْتِيَ بِنَا وَيُعْطِيَنَا الأَرْضَ التِي حَلفَ لآِبَائِنَا.
\par 24 فَأَمَرَنَا الرَّبُّ أَنْ نَعْمَل جَمِيعَ هَذِهِ الفَرَائِضَ وَنَتَّقِيَ الرَّبَّ إِلهَنَا لِيَكُونَ لنَا خَيْرٌ كُل الأَيَّامِ وَيَسْتَبْقِيَنَا كَمَا فِي هَذَا اليَوْمِ.
\par 25 وَإِنَّهُ يَكُونُ لنَا بِرٌّ إِذَا حَفِظْنَا جَمِيعَ هَذِهِ الوَصَايَا لِنَعْمَلهَا أَمَامَ الرَّبِّ إِلهِنَا كَمَا أَوْصَانَا».

\chapter{7}

\par 1 «مَتَى أَتَى بِكَ الرَّبُّ إِلهُكَ إِلى الأَرْضِ التِي أَنْتَ دَاخِلٌ إِليْهَا لِتَمْتَلِكَهَا وَطَرَدَ شُعُوباً كَثِيرَةً مِنْ أَمَامِكَ: الحِثِّيِّينَ وَالجِرْجَاشِيِّينَ وَالأَمُورِيِّينَ وَالكَنْعَانِيِّينَ وَالفِرِزِّيِّينَ وَالحِوِّيِّينَ وَاليَبُوسِيِّينَ سَبْعَ شُعُوبٍ أَكْثَرَ وَأَعْظَمَ مِنْكَ
\par 2 وَدَفَعَهُمُ الرَّبُّ إِلهُكَ أَمَامَكَ وَضَرَبْتَهُمْ فَإِنَّكَ تُحَرِّمُهُمْ. لا تَقْطَعْ لهُمْ عَهْداً وَلا تُشْفِقْ عَليْهِمْ
\par 3 وَلا تُصَاهِرْهُمْ. ابْنَتَكَ لا تُعْطِ لاِبْنِهِ وَابْنَتَهُ لا تَأْخُذْ لاِبْنِكَ.
\par 4 لأَنَّهُ يَرُدُّ ابْنَكَ مِنْ وَرَائِي فَيَعْبُدُ آلِهَةً أُخْرَى فَيَحْمَى غَضَبُ الرَّبِّ عَليْكُمْ وَيُهْلِكُكُمْ سَرِيعاً.
\par 5 وَلكِنْ هَكَذَا تَفْعَلُونَ بِهِمْ: تَهْدِمُونَ مَذَابِحَهُمْ وَتُكَسِّرُونَ أَنْصَابَهُمْ وَتُقَطِّعُونَ سَوَارِيَهُمْ وَتُحْرِقُونَ تَمَاثِيلهُمْ بِالنَّارِ.
\par 6 لأَنَّكَ أَنْتَ شَعْبٌ مُقَدَّسٌ لِلرَّبِّ إِلهِكَ. إِيَّاكَ قَدِ اخْتَارَ الرَّبُّ إِلهُكَ لِتَكُونَ لهُ شَعْباً أَخَصَّ مِنْ جَمِيعِ الشُّعُوبِ الذِينَ عَلى وَجْهِ الأَرْضِ
\par 7 ليْسَ مِنْ كَوْنِكُمْ أَكْثَرَ مِنْ سَائِرِ الشُّعُوبِ التَصَقَ الرَّبُّ بِكُمْ وَاخْتَارَكُمْ لأَنَّكُمْ أَقَلُّ مِنْ سَائِرِ الشُّعُوبِ.
\par 8 بَل مِنْ مَحَبَّةِ الرَّبِّ إِيَّاكُمْ وَحِفْظِهِ القَسَمَ الذِي أَقْسَمَ لآِبَائِكُمْ أَخْرَجَكُمُ الرَّبُّ بِيَدٍ شَدِيدَةٍ وَفَدَاكُمْ مِنْ بَيْتِ العُبُودِيَّةِ مِنْ يَدِ فِرْعَوْنَ مَلِكِ مِصْرَ.
\par 9 فَاعْلمْ أَنَّ الرَّبَّ إِلهَكَ هُوَ اللهُ الإِلهُ الأَمِينُ الحَافِظُ العَهْدَ وَالإِحْسَانَ لِلذِينَ يُحِبُّونَهُ وَيَحْفَظُونَ وَصَايَاهُ إِلى أَلفِ جِيلٍ
\par 10 وَالمُجَازِي الذِينَ يُبْغِضُونَهُ بِوُجُوهِهِمْ لِيُهْلِكَهُمْ. لا يُمْهِلُ مَنْ يُبْغِضُهُ. بِوَجْهِهِ يُجَازِيهِ.
\par 11 فَاحْفَظِ الوَصَايَا وَالفَرَائِضَ وَالأَحْكَامَ التِي أَنَا أُوصِيكَ اليَوْمَ لِتَعْمَلهَا.
\par 12 «وَمِنْ أَجْلِ أَنَّكُمْ تَسْمَعُونَ هَذِهِ الأَحْكَامَ وَتَحْفَظُونَ وَتَعْمَلُونَهَا يَحْفَظُ لكَ الرَّبُّ إِلهُكَ العَهْدَ وَالإِحْسَانَ اللذَيْنِ أَقْسَمَ لآِبَائِكَ
\par 13 وَيُحِبُّكَ وَيُبَارِكُكَ وَيُكَثِّرُكَ وَيُبَارِكُ ثَمَرَةَ بَطْنِكَ وَثَمَرَةَ أَرْضِكَ: قَمْحَكَ وَخَمْرَكَ وَزَيْتَكَ وَنِتَاجَ بَقَرِكَ وَإِنَاثَ غَنَمِكَ عَلى الأَرْضِ التِي أَقْسَمَ لآِبَائِكَ أَنَّهُ يُعْطِيكَ إِيَّاهَا.
\par 14 مُبَارَكاً تَكُونُ فَوْقَ جَمِيعِ الشُّعُوبِ. لا يَكُونُ عَقِيمٌ وَلا عَاقِرٌ فِيكَ وَلا فِي بَهَائِمِكَ.
\par 15 وَيَرُدُّ الرَّبُّ عَنْكَ كُل مَرَضٍ وَكُل أَدْوَاءِ مِصْرَ الرَّدِيئَةِ التِي عَرَفْتَهَا لا يَضَعُهَا عَليْكَ بَل يَجْعَلُهَا عَلى كُلِّ مُبْغِضِيكَ.
\par 16 وَتَأْكُلُ كُل الشُّعُوبِ الذِينَ الرَّبُّ إِلهُكَ يَدْفَعُ إِليْكَ. لا تُشْفِقْ عَيْنَاكَ عَليْهِمْ وَلا تَعْبُدْ آلِهَتَهُمْ لأَنَّ ذَلِكَ شَرَكٌ لكَ.
\par 17 إِنْ قُلتَ فِي قَلبِكَ: هَؤُلاءِ الشُّعُوبُ أَكْثَرُ مِنِّي. كَيْفَ أَقْدِرُ أَنْ أَطْرُدَهُمْ؟
\par 18 فَلا تَخَفْ مِنْهُمُ. اذْكُرْ مَا فَعَلهُ الرَّبُّ إِلهُكَ بِفِرْعَوْنَ وَبِجَمِيعِ المِصْرِيِّينَ.
\par 19 التَّجَارِبَ العَظِيمَةَ التِي أَبْصَرَتْهَا عَيْنَاكَ وَالآيَاتِ وَالعَجَائِبَ وَاليَدَ الشَّدِيدَةَ وَالذِّرَاعَ الرَّفِيعَةَ التِي بِهَا أَخْرَجَكَ الرَّبُّ إِلهُكَ. هَكَذَا يَفْعَلُ الرَّبُّ إِلهُكَ بِجَمِيعِ الشُّعُوبِ التِي أَنْتَ خَائِفٌ مِنْ وَجْهِهَا.
\par 20 «وَالزَّنَابِيرُ أَيْضاً يُرْسِلُهَا الرَّبُّ إِلهُكَ عَليْهِمْ حَتَّى يَفْنَى البَاقُونَ وَالمُخْتَفُونَ مِنْ أَمَامِكَ.
\par 21 لا تَرْهَبْ وُجُوهَهُمْ لأَنَّ الرَّبَّ إِلَهَكَ فِي وَسَطِكَ إِلَهٌ عَظِيمٌ وَمَخُوفٌ.
\par 22 وَلكِنَّ الرَّبَّ إِلهَكَ يَطْرُدُ هَؤُلاءِ الشُّعُوبَ مِنْ أَمَامِكَ قَلِيلاً قَلِيلاً. لا تَسْتَطِيعُ أَنْ تُفْنِيَهُمْ سَرِيعاً لِئَلا تَكْثُرَ عَليْكَ وُحُوشُ البَرِّيَّةِ.
\par 23 وَيَدْفَعُهُمُ الرَّبُّ إِلهُكَ أَمَامَكَ وَيُوقِعُ بِهِمِ اضْطِرَاباً عَظِيماً حَتَّى يَفْنُوا.
\par 24 وَيَدْفَعُ مُلُوكَهُمْ إِلى يَدِكَ فَتَمْحُو اسْمَهُمْ مِنْ تَحْتِ السَّمَاءِ. لا يَقِفُ إِنْسَانٌ فِي وَجْهِكَ حَتَّى تُفْنِيَهُمْ.
\par 25 وَتَمَاثِيل آلِهَتِهِمْ تُحْرِقُونَ بِالنَّارِ. لا تَشْتَهِ فِضَّةً وَلا ذَهَباً مِمَّا عَليْهَا لِتَأْخُذَ لكَ لِئَلا تُصَادَ بِهِ لأَنَّهُ رِجْسٌ عِنْدَ الرَّبِّ إِلهِكَ.
\par 26 وَلا تُدْخِل رِجْساً إِلى بَيْتِكَ لِئَلا تَكُونَ مُحَرَّماً مِثْلهُ. تَسْتَقْبِحُهُ وَتَكْرَهُهُ لأَنَّهُ مُحَرَّمٌ».

\chapter{8}

\par 1 «جَمِيعَ الوَصَايَا التِي أَنَا أُوصِيكُمْ بِهَا اليَوْمَ تَحْفَظُونَ لِتَعْمَلُوهَا لِتَحْيُوا وَتَكْثُرُوا وَتَدْخُلُوا وَتَمْتَلِكُوا الأَرْضَ التِي أَقْسَمَ الرَّبُّ لآِبَائِكُمْ.
\par 2 وَتَتَذَكَّرُ كُل الطَّرِيقِ التِي فِيهَا سَارَ بِكَ الرَّبُّ إِلهُكَ هَذِهِ الأَرْبَعِينَ سَنَةً فِي القَفْرِ لِيُذِلكَ وَيُجَرِّبَكَ لِيَعْرِفَ مَا فِي قَلبِكَ أَتَحْفَظُ وَصَايَاهُ أَمْ لا؟
\par 3 فَأَذَلكَ وَأَجَاعَكَ وَأَطْعَمَكَ المَنَّ الذِي لمْ تَكُنْ تَعْرِفُهُ وَلا عَرَفَهُ آبَاؤُكَ لِيُعَلِّمَكَ أَنَّهُ ليْسَ بِالخُبْزِ وَحْدَهُ يَحْيَا الإِنْسَانُ بَل بِكُلِّ مَا يَخْرُجُ مِنْ فَمِ الرَّبِّ يَحْيَا الإِنْسَانُ.
\par 4 ثِيَابُكَ لمْ تَبْل عَليْكَ وَرِجْلُكَ لمْ تَتَوَرَّمْ هَذِهِ الأَرْبَعِينَ سَنَةً.
\par 5 فَاعْلمْ فِي قَلبِكَ أَنَّهُ كَمَا يُؤَدِّبُ الإِنْسَانُ ابْنَهُ قَدْ أَدَّبَكَ الرَّبُّ إِلهُكَ.
\par 6 وَاحْفَظْ وَصَايَا الرَّبِّ إِلهِكَ لِتَسْلُكَ فِي طُرُقِهِ وَتَتَّقِيَهُ
\par 7 لأَنَّ الرَّبَّ إِلهَكَ آتٍ بِكَ إِلى أَرْضٍ جَيِّدَةٍ أَرْضِ أَنْهَارٍ مِنْ عُيُونٍ وَغِمَارٍ تَنْبَعُ فِي البِقَاعِ وَالجِبَالِ.
\par 8 أَرْضِ حِنْطَةٍ وَشَعِيرٍ وَكَرْمٍ وَتِينٍ وَرُمَّانٍ. أَرْضِ زَيْتُونِ زَيْتٍ وَعَسَلٍ.
\par 9 أَرْضٌ ليْسَ بِالمَسْكَنَةِ تَأْكُلُ فِيهَا خُبْزاً وَلا يُعْوِزُكَ فِيهَا شَيْءٌ. أَرْضٌ حِجَارَتُهَا حَدِيدٌ وَمِنْ جِبَالِهَا تَحْفُرُ نُحَاساً.
\par 10 فَمَتَى أَكَلتَ وَشَبِعْتَ تُبَارِكُ الرَّبَّ إِلهَكَ لأَجْلِ الأَرْضِ الجَيِّدَةِ التِي أَعْطَاكَ.
\par 11 اِحْتَرِزْ مِنْ أَنْ تَنْسَى الرَّبَّ إِلهَكَ وَلا تَحْفَظَ وَصَايَاهُ وَأَحْكَامَهُ وَفَرَائِضَهُ التِي أَنَا أُوصِيكَ بِهَا اليَوْمَ.
\par 12 لِئَلا إِذَا أَكَلتَ وَشَبِعْتَ وَبَنَيْتَ بُيُوتاً جَيِّدَةً وَسَكَنْتَ
\par 13 وَكَثُرَتْ بَقَرُكَ وَغَنَمُكَ وَكَثُرَتْ لكَ الفِضَّةُ وَالذَّهَبُ وَكَثُرَ كُلُّ مَا لكَ
\par 14 يَرْتَفِعُ قَلبُكَ وَتَنْسَى الرَّبَّ إِلهَكَ الذِي أَخْرَجَكَ مِنْ أَرْضِ مِصْرَ مِنْ بَيْتِ العُبُودِيَّةِ
\par 15 الذِي سَارَ بِكَ فِي القَفْرِ العَظِيمِ المَخُوفِ مَكَانِ حَيَّاتٍ مُحْرِقَةٍ وَعَقَارِبَ وَعَطَشٍ حَيْثُ ليْسَ مَاءٌ. الذِي أَخْرَجَ لكَ مَاءً مِنْ صَخْرَةِ الصَّوَّانِ
\par 16 الذِي أَطْعَمَكَ فِي البَرِّيَّةِ المَنَّ الذِي لمْ يَعْرِفْهُ آبَاؤُكَ لِيُذِلكَ وَيُجَرِّبَكَ لِيُحْسِنَ إِليْكَ فِي آخِرَتِكَ.
\par 17 وَلِئَلا تَقُول فِي قَلبِكَ: قُوَّتِي وَقُدْرَةُ يَدِيَ اصْطَنَعَتْ لِي هَذِهِ الثَّرْوَةَ.
\par 18 بَلِ اذْكُرِ الرَّبَّ إِلهَكَ أَنَّهُ هُوَ الذِي يُعْطِيكَ قُوَّةً لاِصْطِنَاعِ الثَّرْوَةِ لِيَفِيَ بِعَهْدِهِ الذِي أَقْسَمَ لآِبَائِكَ كَمَا فِي هَذَا اليَوْمِ.
\par 19 وَإِنْ نَسِيتَ الرَّبَّ إِلهَكَ وَذَهَبْتَ وَرَاءَ آلِهَةٍ أُخْرَى وَعَبَدْتَهَا وَسَجَدْتَ لهَا أُشْهِدُ عَليْكُمُ اليَوْمَ أَنَّكُمْ تَبِيدُونَ لا مَحَالةَ.
\par 20 كَالشُّعُوبِ الذِينَ يُبِيدُهُمُ الرَّبُّ مِنْ أَمَامِكُمْ كَذَلِكَ تَبِيدُونَ لأَجْلِ أَنَّكُمْ لمْ تَسْمَعُوا لِقَوْلِ الرَّبِّ إِلهِكُمْ».

\chapter{9}

\par 1 «إِسْمَعْ يَا إِسْرَائِيلُ أَنْتَ اليَوْمَ عَابِرٌ الأُرْدُنَّ لِتَدْخُل وَتَمْتَلِكَ شُعُوباً أَكْبَرَ وَأَعْظَمَ مِنْكَ وَمُدُناً عَظِيمَةً وَمُحَصَّنَةً إِلى السَّمَاءِ.
\par 2 قَوْماً عِظَاماً وَطِوَالاً بَنِي عَنَاقٍَ الذِينَ عَرَفْتَهُمْ وَسَمِعْتَ: مَنْ يَقِفُ فِي وَجْهِ بَنِي عَنَاقٍَ؟
\par 3 فَاعْلمِ اليَوْمَ أَنَّ الرَّبَّ إِلهَكَ هُوَ العَابِرُ أَمَامَكَ نَاراً آكِلةً. هُوَ يُبِيدُهُمْ وَيُذِلُّهُمْ أَمَامَكَ فَتَطْرُدُهُمْ وَتُهْلِكُهُمْ سَرِيعاً كَمَا كَلمَكَ الرَّبُّ.
\par 4 لا تَقُل فِي قَلبِكَ حِينَ يَنْفِيهِمِ الرَّبُّ إِلهُكَ مِنْ أَمَامِكَ: لأَجْلِ بِرِّي أَدْخَلنِي الرَّبُّ لأَمْتَلِكَ هَذِهِ الأَرْضَ. وَلأَجْلِ إِثْمِ هَؤُلاءِ الشُّعُوبِ يَطْرُدُهُمُ الرَّبُّ مِنْ أَمَامِكَ.
\par 5 ليْسَ لأَجْلِ بِرِّكَ وَعَدَالةِ قَلبِكَ تَدْخُلُ لِتَمْتَلِكَ أَرْضَهُمْ بَل لأَجْلِ إِثْمِ أُولئِكَ الشُّعُوبِ يَطْرُدُهُمُ الرَّبُّ إِلهُكَ مِنْ أَمَامِكَ وَلِيَفِيَ بِالكَلامِ الذِي أَقْسَمَ الرَّبُّ عَليْهِ لآِبَائِكَ إِبْرَاهِيمَ وَإِسْحَاقَ وَيَعْقُوبَ.
\par 6 فَاعْلمْ أَنَّهُ ليْسَ لأَجْلِ بِرِّكَ يُعْطِيكَ الرَّبُّ إِلهُكَ هَذِهِ الأَرْضَ الجَيِّدَةَ لِتَمْتَلِكَهَا لأَنَّكَ شَعْبٌ صُلبُ الرَّقَبَةِ.
\par 7 «اُذْكُرْ. لا تَنْسَ كَيْفَ أَسْخَطْتَ الرَّبَّ إِلهَكَ فِي البَرِّيَّةِ. مِنَ اليَوْمِ الذِي خَرَجْتَ فِيهِ مِنْ أَرْضِ مِصْرَ حَتَّى أَتَيْتُمْ إِلى هَذَا المَكَانِ كُنْتُمْ تُقَاوِمُونَ الرَّبَّ.
\par 8 حَتَّى فِي حُورِيبَ أَسْخَطْتُمُ الرَّبَّ فَغَضِبَ الرَّبُّ عَليْكُمْ لِيُبِيدَكُمْ.
\par 9 حِينَ صَعِدْتُ إِلى الجَبَلِ لآِخُذَ لوْحَيِ الحَجَرِ لوْحَيِ العَهْدِ الذِي قَطَعَهُ الرَّبُّ مَعَكُمْ أَقَمْتُ فِي الجَبَلِ أَرْبَعِينَ نَهَاراً وَأَرْبَعِينَ ليْلةً لا آكُلُ خُبْزاً وَلا أَشْرَبُ مَاءً.
\par 10 وَأَعْطَانِيَ الرَّبُّ لوْحَيِ الحَجَرِ المَكْتُوبَيْنِ بِإِصْبِعِ اللهِ وَعَليْهِمَا مِثْلُ جَمِيعِ الكَلِمَاتِ التِي كَلمَكُمْ بِهَا الرَّبُّ فِي الجَبَلِ مِنْ وَسَطِ النَّارِ فِي يَوْمِ الاِجْتِمَاعِ.
\par 11 وَفِي نِهَايَةِ الأَرْبَعِينَ نَهَاراً وَالأَرْبَعِينَ ليْلةً لمَّا أَعْطَانِيَ الرَّبُّ لوْحَيِ الحَجَرِ لوْحَيِ العَهْدِ
\par 12 قَال الرَّبُّ لِي: قُمِ انْزِل عَاجِلاً مِنْ هُنَا لأَنَّهُ قَدْ فَسَدَ شَعْبُكَ الذِي أَخْرَجْتَهُ مِنْ مِصْرَ. زَاغُوا سَرِيعاً عَنِ الطَّرِيقِ التِي أَوْصَيْتُهُمْ. صَنَعُوا لأَنْفُسِهِمْ تِمْثَالاً مَسْبُوكاً.
\par 13 وَقَال الرَّبُّ لِي: رَأَيْتُ هَذَا الشَّعْبَ وَإِذَا هُوَ شَعْبٌ صُلبُ الرَّقَبَةِ.
\par 14 أُتْرُكْنِي فَأُبِيدَهُمْ وَأَمْحُوَ اسْمَهُمْ مِنْ تَحْتِ السَّمَاءِ وَأَجْعَلكَ شَعْباً أَعْظَمَ وَأَكْثَرَ مِنْهُمْ.
\par 15 فَانْصَرَفْتُ وَنَزَلتُ مِنَ الجَبَلِ وَالجَبَلُ يَشْتَعِلُ بِالنَّارِ وَلوْحَا العَهْدِ فِي يَدَيَّ.
\par 16 «فَنَظَرْتُ وَإِذَا أَنْتُمْ قَدْ أَخْطَأْتُمْ إِلى الرَّبِّ إِلهِكُمْ وَصَنَعْتُمْ لأَنْفُسِكُمْ عِجْلاً مَسْبُوكاً وَزُغْتُمْ سَرِيعاً عَنِ الطَّرِيقِ التِي أَوْصَاكُمْ بِهَا الرَّبُّ.
\par 17 فَأَخَذْتُ اللوْحَيْنِ وَطَرَحْتُهُمَا مِنْ يَدَيَّ وَكَسَّرْتُهُمَا أَمَامَ أَعْيُنِكُمْ.
\par 18 ثُمَّ سَقَطْتُ أَمَامَ الرَّبِّ كَالأَوَّلِ أَرْبَعِينَ نَهَاراً وَأَرْبَعِينَ ليْلةً لا آكُلُ خُبْزاً وَلا أَشْرَبُ مَاءً مِنْ أَجْلِ كُلِّ خَطَايَاكُمُ التِي أَخْطَأْتُمْ بِهَا بِعَمَلِكُمُ الشَّرَّ أَمَامَ الرَّبِّ لِإِغَاظَتِهِ.
\par 19 لأَنِّي فَزِعْتُ مِنَ الغَضَبِ وَالغَيْظِ الذِي سَخِطَهُ الرَّبُّ عَليْكُمْ لِيُبِيدَكُمْ. فَسَمِعَ لِيَ الرَّبُّ تِلكَ المَرَّةَ أَيْضاً.
\par 20 وَعَلى هَارُونَ غَضِبَ الرَّبُّ جِدّاً لِيُبِيدَهُ. فَصَليْتُ أَيْضاً مِنْ أَجْلِ هَارُونَ فِي ذَلِكَ الوَقْتِ.
\par 21 وَأَمَّا خَطِيَّتُكُمُ العِجْلُ الذِي صَنَعْتُمُوهُ فَأَخَذْتُهُ وَأَحْرَقْتُهُ بِالنَّارِ وَرَضَضْتُهُ وَطَحَنْتُهُ جَيِّداً حَتَّى نَعِمَ كَالغُبَارِ. ثُمَّ طَرَحْتُ غُبَارَهُ فِي النَّهْرِ المُنْحَدِرِ مِنَ الجَبَلِ.
\par 22 «وَفِي تَبْعِيرَةَ وَمَسَّةَ وَقَبَرُوتَ هَتَّأَوَةَ أَسْخَطْتُمُ الرَّبَّ.
\par 23 وَحِينَ أَرْسَلكُمُ الرَّبُّ مِنْ قَادِشَ بَرْنِيعَ قَائِلاً: اصْعَدُوا امْتَلِكُوا الأَرْضَ التِي أَعْطَيْتُكُمْ عَصَيْتُمْ قَوْل الرَّبِّ إِلهِكُمْ وَلمْ تُصَدِّقُوهُ وَلمْ تَسْمَعُوا لِقَوْلِهِ.
\par 24 قَدْ كُنْتُمْ تَعْصُونَ الرَّبَّ مُنْذُ يَوْمَ عَرَفْتُكُمْ.
\par 25 «فَسَقَطْتُ أَمَامَ الرَّبِّ الأَرْبَعِينَ نَهَاراً وَالأَرْبَعِينَ ليْلةً التِي سَقَطْتُهَا لأَنَّ الرَّبَّ قَال إِنَّهُ يُهْلِكُكُمْ.
\par 26 وَصَليْتُ لِلرَّبِّ: يَا سَيِّدُ الرَّبُّ لا تُهْلِكْ شَعْبَكَ وَمِيرَاثَكَ الذِي فَدَيْتَهُ بِعَظَمَتِكَ الذِي أَخْرَجْتَهُ مِنْ مِصْرَ بِيَدٍ شَدِيدَةٍ.
\par 27 اُذْكُرْ عَبِيدَكَ إِبْرَاهِيمَ وَإِسْحَاقَ وَيَعْقُوبَ. لا تَلتَفِتْ إِلى غَلاظَةِ هَذَا الشَّعْبِ وَإِثْمِهِ وَخَطِيَّتِهِ
\par 28 لِئَلا تَقُول الأَرْضُ التِي أَخْرَجْتَنَا مِنْهَا: لأَجْلِ أَنَّ الرَّبَّ لمْ يَقْدِرْ أَنْ يُدْخِلهُمُ الأَرْضَ التِي كَلمَهُمْ عَنْهَا وَلأَجْلِ أَنَّهُ أَبْغَضَهُمْ أَخْرَجَهُمْ لِيُمِيتَهُمْ فِي البَرِّيَّةِ.
\par 29 وَهُمْ شَعْبُكَ وَمِيرَاثُكَ الذِي أَخْرَجْتَهُ بِقُوَّتِكَ العَظِيمَةِ وَبِذِرَاعِكَ الرَّفِيعَةِ».

\chapter{10}

\par 1 «فِي ذَلِكَ الوَقْتِ قَال لِيَ الرَّبُّ: انْحَتْ لكَ لوْحَيْنِ مِنْ حَجَرٍ مِثْل الأَوَّليْنِ وَاصْعَدْ إِليَّ إِلى الجَبَلِ وَاصْنَعْ لكَ تَابُوتاً مِنْ خَشَبٍ.
\par 2 فَأَكْتُبُ عَلى اللوْحَيْنِ الكَلِمَاتِ التِي كَانَتْ عَلى اللوْحَيْنِ الأَوَّليْنِ اللذَيْنِ كَسَرْتَهُمَا وَتَضَعُهُمَا فِي التَّابُوتِ.
\par 3 فَصَنَعْتُ تَابُوتاً مِنْ خَشَبِ السَّنْطِ وَنَحَتُّ لوْحَيْنِ مِنْ حَجَرٍ مِثْل الأَوَّليْنِ وَصَعِدْتُ إِلى الجَبَلِ وَاللوْحَانِ فِي يَدِي.
\par 4 فَكَتَبَ عَلى اللوْحَيْنِ مِثْل الكِتَابَةِ الأُولى الكَلِمَاتِ العَشَرَ التِي كَلمَكُمْ بِهَا الرَّبُّ فِي الجَبَلِ مِنْ وَسَطِ النَّارِ فِي يَوْمِ الاِجْتِمَاعِ وَأَعْطَانِيَ الرَّبُّ إِيَّاهَا.
\par 5 ثُمَّ انْصَرَفْتُ وَنَزَلتُ مِنَ الجَبَلِ وَوَضَعْتُ اللوْحَيْنِ فِي التَّابُوتِ الذِي صَنَعْتُ فَكَانَا هُنَاكَ كَمَا أَمَرَنِيَ الرَّبُّ.
\par 6 (وَبَنُو إِسْرَائِيل ارْتَحَلُوا مِنْ آبَارِ بَنِي يَعْقَانَ إِلى مُوسِيرَ. هُنَاكَ مَاتَ هَارُونُ وَهُنَاكَ دُفِنَ. فَكَهَنَ أَلِعَازَارُ ابْنُهُ عِوَضاً عَنْهُ.
\par 7 مِنْ هُنَاكَ ارْتَحَلُوا إِلى الجِدْجَادِ وَمِنَ الجِدْجَادِ إِلى يُطْبَاتَ أَرْضِ أَنْهَارِ مَاءٍ.
\par 8 فِي ذَلِكَ الوَقْتِ أَفْرَزَ الرَّبُّ سِبْطَ لاوِي لِيَحْمِلُوا تَابُوتَ عَهْدِ الرَّبِّ وَلِيَقِفُوا أَمَامَ الرَّبِّ لِيَخْدِمُوهُ وَيُبَارِكُوا بِاسْمِهِ إِلى هَذَا اليَوْمِ.
\par 9 لأَجْلِ ذَلِكَ لمْ يَكُنْ لِلاوِي قِسْمٌ وَلا نَصِيبٌ مَعَ إِخْوَتِهِ. الرَّبُّ هُوَ نَصِيبُهُ كَمَا كَلمَهُ الرَّبُّ إِلهُكَ).
\par 10 «وَأَنَا مَكَثْتُ فِي الجَبَلِ كَالأَيَّامِ الأُولى أَرْبَعِينَ نَهَاراً وَأَرْبَعِينَ ليْلةً. وَسَمِعَ الرَّبُّ لِي تِلكَ المَرَّةَ أَيْضاً وَلمْ يَشَإِ الرَّبُّ أَنْ يُهْلِكَكَ.
\par 11 ثُمَّ قَال لِيَ الرَّبُّ: قُمِ اذْهَبْ لِلاِرْتِحَالِ أَمَامَ الشَّعْبِ فَيَدْخُلُوا وَيَمْتَلِكُوا الأَرْضَ التِي حَلفْتُ لآِبَائِهِمْ أَنْ أُعْطِيَهُمْ.
\par 12 «فَالآنَ يَا إِسْرَائِيلُ مَاذَا يَطْلُبُ مِنْكَ الرَّبُّ إِلهُكَ إِلا أَنْ تَتَّقِيَ الرَّبَّ إِلهَكَ لِتَسْلُكَ فِي كُلِّ طُرُقِهِ وَتُحِبَّهُ وَتَعْبُدَ الرَّبَّ إِلهَكَ مِنْ كُلِّ قَلبِكَ وَمِنْ كُلِّ نَفْسِكَ
\par 13 وَتَحْفَظَ وَصَايَا الرَّبِّ وَفَرَائِضَهُ التِي أَنَا أُوصِيكَ بِهَا اليَوْمَ لِخَيْرِكَ.
\par 14 هُوَذَا لِلرَّبِّ إِلهِكَ السَّمَاوَاتُ وَسَمَاءُ السَّمَاوَاتِ وَالأَرْضُ وَكُلُّ مَا فِيهَا.
\par 15 وَلكِنَّ الرَّبَّ إِنَّمَا التَصَقَ بِآبَائِكَ لِيُحِبَّهُمْ فَاخْتَارَ مِنْ بَعْدِهِمْ نَسْلهُمُ الذِي هُوَ أَنْتُمْ فَوْقَ جَمِيعِ الشُّعُوبِ كَمَا فِي هَذَا اليَوْمِ.
\par 16 فَاخْتِنُوا غُرْلةَ قُلُوبِكُمْ وَلا تُصَلِّبُوا رِقَابَكُمْ بَعْدُ.
\par 17 لأَنَّ الرَّبَّ إِلهَكُمْ هُوَ إِلهُ الآلِهَةِ وَرَبُّ الأَرْبَابِ الإِلهُ العَظِيمُ الجَبَّارُ المَهِيبُ الذِي لا يَأْخُذُ بِالوُجُوهِ وَلا يَقْبَلُ رَشْوَةً
\par 18 الصَّانِعُ حَقَّ اليَتِيمِ وَالأَرْمَلةِ وَالمُحِبُّ الغَرِيبَ لِيُعْطِيَهُ طَعَاماً وَلِبَاساً.
\par 19 فَأَحِبُّوا الغَرِيبَ لأَنَّكُمْ كُنْتُمْ غُرَبَاءَ فِي أَرْضِ مِصْرَ.
\par 20 الرَّبَّ إِلهَكَ تَتَّقِي. إِيَّاهُ تَعْبُدُ وَبِهِ تَلتَصِقُ وَبِاسْمِهِ تَحْلِفُ.
\par 21 هُوَ فَخْرُكَ وَهُوَ إِلهُكَ الذِي صَنَعَ مَعَكَ تِلكَ العَظَائِمَ وَالمَخَاوِفَ التِي أَبْصَرَتْهَا عَيْنَاكَ.
\par 22 سَبْعِينَ نَفْساً نَزَل آبَاؤُكَ إِلى مِصْرَ وَالآنَ قَدْ جَعَلكَ الرَّبُّ إِلهُكَ كَنُجُومِ السَّمَاءِ فِي الكَثْرَةِ».

\chapter{11}

\par 1 «فَأَحْبِبِ الرَّبَّ إِلهَكَ وَاحْفَظْ حُقُوقَهُ وَفَرَائِضَهُ وَأَحْكَامَهُ وَوَصَايَاهُ كُل الأَيَّامِ.
\par 2 وَاعْلمُوا اليَوْمَ أَنِّي لسْتُ أُرِيدُ بَنِيكُمُ الذِينَ لمْ يَعْرِفُوا وَلا رَأُوا تَأْدِيبَ الرَّبِّ إِلهِكُمْ عَظَمَتَهُ وَيَدَهُ الشَّدِيدَةَ وَذِرَاعَهُ الرَّفِيعَةَ
\par 3 وَآيَاتِهِ وَصَنَائِعَهُ التِي عَمِلهَا فِي مِصْرَ بِفِرْعَوْنَ مَلِكِ مِصْرَ وَبِكُلِّ أَرْضِهِ
\par 4 وَالتِي عَمِلهَا بِجَيْشِ مِصْرَ بِخَيْلِهِمْ وَمَرَاكِبِهِمْ حَيْثُ أَطَافَ مِيَاهَ بَحْرِ سُوفٍ عَلى وُجُوهِهِمْ حِينَ سَعُوا وَرَاءَكُمْ فَأَبَادَهُمُ الرَّبُّ إِلى هَذَا اليَوْمِ
\par 5 وَالتِي عَمِلهَا لكُمْ فِي البَرِّيَّةِ حَتَّى جِئْتُمْ إِلى هَذَا المَكَانِ
\par 6 وَالتِي عَمِلهَا بِدَاثَانَ وَأَبِيرَامَ ابْنَيْ أَلِيآبَ ابْنِ رَأُوبَيْنَ اللذَيْنِ فَتَحَتِ الأَرْضُ فَاهَا وَابْتَلعَتْهُمَا مَعَ بُيُوتِهِمَا وَخِيَامِهِمَا وَكُلِّ المَوْجُودَاتِ التَّابِعَةِ لهُمَا فِي وَسْطِ كُلِّ إِسْرَائِيل.
\par 7 لأَنَّ أَعْيُنَكُمْ هِيَ التِي أَبْصَرَتْ كُل صَنَائِعِ الرَّبِّ العَظِيمَةِ التِي عَمِلهَا.
\par 8 «فَاحْفَظُوا كُل الوَصَايَا التِي أَنَا أُوصِيكُمْ بِهَا اليَوْمَ لِتَتَشَدَّدُوا وَتَدْخُلُوا وَتَمْتَلِكُوا الأَرْضَ التِي أَنْتُمْ عَابِرُونَ إِليْهَا لِتَمْتَلِكُوهَا
\par 9 وَلِتُطِيلُوا الأَيَّامَ عَلى الأَرْضِ التِي أَقْسَمَ الرَّبُّ لآِبَائِكُمْ أَنْ يُعْطِيَهَا لهُمْ وَلِنَسْلِهِمْ أَرْضٌ تَفِيضُ لبَناً وَعَسَلاً.
\par 10 لأَنَّ الأَرْضَ التِي أَنْتَ دَاخِلٌ إِليْهَا لِتَمْتَلِكَهَا ليْسَتْ مِثْل أَرْضِ مِصْرَ التِي خَرَجْتَ مِنْهَا حَيْثُ كُنْتَ تَزْرَعُ زَرْعَكَ وَتَسْقِيهِ بِرِجْلِكَ كَبُسْتَانِ بُقُولٍ.
\par 11 بَل هِيَ أَرْضُ جِبَالٍ وَبِقَاعٍ. مِنْ مَطَرِ السَّمَاءِ تَشْرَبُ مَاءً.
\par 12 أَرْضٌ يَعْتَنِي بِهَا الرَّبُّ إِلهُكَ. عَيْنَا الرَّبِّ إِلهِكَ عَليْهَا دَائِماً مِنْ أَوَّلِ السَّنَةِ إِلى آخِرِهَا.
\par 13 «فَإِذَا سَمِعْتُمْ لِوَصَايَايَ التِي أَنَا أُوصِيكُمْ بِهَا اليَوْمَ لِتُحِبُّوا الرَّبَّ إِلهَكُمْ وَتَعْبُدُوهُ مِنْ كُلِّ قُلُوبِكُمْ وَمِنْ كُلِّ أَنْفُسِكُمْ
\par 14 أُعْطِي مَطَرَ أَرْضِكُمْ فِي حِينِهِ: المُبَكِّرَ وَالمُتَأَخِّرَ. فَتَجْمَعُ حِنْطَتَكَ وَخَمْرَكَ وَزَيْتَكَ.
\par 15 وَأُعْطِي لِبَهَائِمِكَ عُشْباً فِي حَقْلِكَ فَتَأْكُلُ أَنْتَ وَتَشْبَعُ.
\par 16 فَاحْتَرِزُوا مِنْ أَنْ تَنْغَوِيَ قُلُوبُكُمْ فَتَزِيغُوا وَتَعْبُدُوا آلِهَةً أُخْرَى وَتَسْجُدُوا لهَا
\par 17 فَيَحْمَى غَضَبُ الرَّبِّ عَليْكُمْ وَيُغْلِقُ السَّمَاءَ فَلا يَكُونُ مَطَرٌ وَلا تُعْطِي الأَرْضُ غَلتَهَا فَتَبِيدُونَ سَرِيعاً عَنِ الأَرْضِ الجَيِّدَةِ التِي يُعْطِيكُمُ الرَّبُّ.
\par 18 «فَضَعُوا كَلِمَاتِي هَذِهِ عَلى قُلُوبِكُمْ وَنُفُوسِكُمْ وَارْبُطُوهَا عَلامَةً عَلى أَيْدِيكُمْ وَلتَكُنْ عَصَائِبَ بَيْنَ عُيُونِكُمْ
\par 19 وَعَلِّمُوهَا أَوْلادَكُمْ مُتَكَلِّمِينَ بِهَا حِينَ تَجْلِسُونَ فِي بُيُوتِكُمْ وَحِينَ تَمْشُونَ فِي الطَّرِيقِ وَحِينَ تَنَامُونَ وَحِينَ تَقُومُونَ.
\par 20 وَاكْتُبْهَا عَلى قَوَائِمِ أَبْوَابِ بَيْتِكَ وَعَلى أَبْوَابِكَ
\par 21 لِتَكْثُرَ أَيَّامُكَ وَأَيَّامُ أَوْلادِكَ عَلى الأَرْضِ التِي أَقْسَمَ الرَّبُّ لآِبَائِكَ أَنْ يُعْطِيَهُمْ إِيَّاهَا كَأَيَّامِ السَّمَاءِ عَلى الأَرْضِ.
\par 22 لأَنَّهُ إِذَا حَفِظْتُمْ جَمِيعَ هَذِهِ الوَصَايَا التِي أَنَا أُوصِيكُمْ بِهَا لِتَعْمَلُوهَا لِتُحِبُّوا الرَّبَّ إِلهَكُمْ وَتَسْلُكُوا فِي جَمِيعِ طُرُقِهِ وَتَلتَصِقُوا بِهِ
\par 23 يَطْرُدُ الرَّبُّ جَمِيعَ هَؤُلاءِ الشُّعُوبِ مِنْ أَمَامِكُمْ فَتَرِثُونَ شُعُوباً أَكْبَرَ وَأَعْظَمَ مِنْكُمْ.
\par 24 كُلُّ مَكَانٍ تَدُوسُهُ بُطُونُ أَقْدَامِكُمْ يَكُونُ لكُمْ. مِنَ البَرِّيَّةِ وَلُبْنَانَ. مِنَ نَهْرِ الفُرَاتِ إِلى البَحْرِ الغَرْبِيِّ يَكُونُ تُخُمُكُمْ.
\par 25 لا يَقِفُ إِنْسَانٌ فِي وَجْهِكُمْ. الرَّبُّ إِلهُكُمْ يَجْعَلُ خَشْيَتَكُمْ وَرُعْبَكُمْ عَلى كُلِّ الأَرْضِ التِي تَدُوسُونَهَا كَمَا كَلمَكُمْ.
\par 26 «اُنْظُرْ! أَنَا وَاضِعٌ أَمَامَكُمُ اليَوْمَ بَرَكَةً وَلعْنَةً.
\par 27 البَرَكَةُ إِذَا سَمِعْتُمْ لِوَصَايَا الرَّبِّ إِلهِكُمُ التِي أَنَا أُوصِيكُمْ بِهَا اليَوْمَ.
\par 28 وَاللعْنَةُ إِذَا لمْ تَسْمَعُوا لِوَصَايَا الرَّبِّ إِلهِكُمْ وَزُغْتُمْ عَنِ الطَّرِيقِ التِي أَنَا أُوصِيكُمْ بِهَا اليَوْمَ لِتَذْهَبُوا وَرَاءَ آلِهَةٍ أُخْرَى لمْ تَعْرِفُوهَا.
\par 29 وَإِذَا جَاءَ بِكَ الرَّبُّ إِلهُكَ إِلى الأَرْضِ التِي أَنْتَ دَاخِلٌ إِليْهَا لِتَمْتَلِكَهَا فَاجْعَلِ البَرَكَةَ عَلى جَبَلِ جِرِزِّيمَ وَاللعْنَةَ عَلى جَبَلِ عِيبَال.
\par 30 أَمَا هُمَا فِي عَبْرِ الأُرْدُنِّ وَرَاءَ طَرِيقِ غُرُوبِ الشَّمْسِ فِي أَرْضِ الكَنْعَانِيِّينَ السَّاكِنِينَ فِي العَرَبَةِ مُقَابِل الجِلجَالِ بِجَانِبِ بَلُّوطَاتِ مُورَةَ؟
\par 31 لأَنَّكُمْ عَابِرُونَ الأُرْدُنَّ لِتَدْخُلُوا وَتَمْتَلِكُوا الأَرْضَ التِي الرَّبُّ إِلهُكُمْ يُعْطِيكُمْ. تَمْتَلِكُونَهَا وَتَسْكُنُونَهَا.
\par 32 فَاحْفَظُوا جَمِيعَ الفَرَائِضِ وَالأَحْكَامِ التِي أَنَا وَاضِعٌ أَمَامَكُمُ اليَوْمَ لِتَعْمَلُوهَا».

\chapter{12}

\par 1 «هَذِهِ هِيَ الفَرَائِضُ وَالأَحْكَامُ التِي تَحْفَظُونَ لِتَعْمَلُوهَا فِي الأَرْضِ التِي أَعْطَاكَ الرَّبُّ إِلهُ آبَائِكَ لِتَمْتَلِكَهَا؛ كُل الأَيَّامِ التِي تَحْيُونَ عَلى الأَرْضِ:
\par 2 تُخْرِبُونَ جَمِيعَ الأَمَاكِنِ حَيْثُ عَبَدَتِ الأُمَمُ التِي تَرِثُونَهَا آلِهَتَهَا عَلى الجِبَالِ الشَّامِخَةِ وَعَلى التِّلالِ وَتَحْتَ كُلِّ شَجَرَةٍ خَضْرَاءَ.
\par 3 وَتَهْدِمُونَ مَذَابِحَهُمْ وَتُكَسِّرُونَ أَنْصَابَهُمْ وَتُحْرِقُونَ سَوَارِيَهُمْ بِالنَّارِ وَتُقَطِّعُونَ تَمَاثِيل آلِهَتِهِمْ وَتَمْحُونَ اسْمَهُمْ مِنْ ذَلِكَ المَكَانِ.
\par 4 لا تَفْعَلُوا هَكَذَا لِلرَّبِّ إِلهِكُمْ.
\par 5 بَلِ المَكَانُ الذِي يَخْتَارُهُ الرَّبُّ إِلهُكُمْ مِنْ جَمِيعِ أَسْبَاطِكُمْ لِيَضَعَ اسْمَهُ فِيهِ سُكْنَاهُ تَطْلُبُونَ وَإِلى هُنَاكَ تَأْتُونَ
\par 6 وَتُقَدِّمُونَ إِلى هُنَاكَ مُحْرَقَاتِكُمْ وَذَبَائِحَكُمْ وَعُشُورَكُمْ وَرَفَائِعَ أَيْدِيكُمْ وَنُذُورَكُمْ وَنَوَافِلكُمْ وَأَبْكَارَ بَقَرِكُمْ وَغَنَمِكُمْ
\par 7 وَتَأْكُلُونَ هُنَاكَ أَمَامَ الرَّبِّ إِلهِكُمْ وَتَفْرَحُونَ بِكُلِّ مَا تَمْتَدُّ إِليْهِ أَيْدِيكُمْ أَنْتُمْ وَبُيُوتُكُمْ كَمَا بَارَكَكُمُ الرَّبُّ إِلهُكُمْ.
\par 8 «لا تَعْمَلُوا حَسَبَ كُلِّ مَا نَحْنُ عَامِلُونَ هُنَا اليَوْمَ أَيْ كُلُّ إِنْسَانٍ مَهْمَا صَلحَ فِي عَيْنَيْهِ.
\par 9 لأَنَّكُمْ لمْ تَدْخُلُوا حَتَّى الآنَ إِلى المَقَرِّ وَالنَّصِيبِ اللذَيْنِ يُعْطِيكُمُ الرَّبُّ إِلهُكُمْ.
\par 10 فَمَتَى عَبَرْتُمُ الأُرْدُنَّ وَسَكَنْتُمُ الأَرْضَ التِي يَقْسِمُهَا لكُمُ الرَّبُّ إِلهُكُمْ وَأَرَاحَكُمْ مِنْ جَمِيعِ أَعْدَائِكُمُ الذِينَ حَوَاليْكُمْ وَسَكَنْتُمْ آمِنِينَ
\par 11 فَالمَكَانُ الذِي يَخْتَارُهُ الرَّبُّ إِلهُكُمْ لِيَحِل اسْمَهُ فِيهِ تَحْمِلُونَ إِليْهِ كُل مَا أَنَا أُوصِيكُمْ بِهِ: مُحْرَقَاتِكُمْ وَذَبَائِحَكُمْ وَعُشُورَكُمْ وَرَفَائِعَ أَيْدِيكُمْ وَكُل خِيَارِ نُذُورِكُمُ التِي تَنْذُرُونَهَا لِلرَّبِّ.
\par 12 وَتَفْرَحُونَ أَمَامَ الرَّبِّ إِلهِكُمْ أَنْتُمْ وَبَنُوكُمْ وَبَنَاتُكُمْ وَعَبِيدُكُمْ وَإِمَاؤُكُمْ وَاللاوِيُّ الذِي فِي أَبْوَابِكُمْ لأَنَّهُ ليْسَ لهُ قِسْمٌ وَلا نَصِيبٌ مَعَكُمْ.
\par 13 «اِحْتَرِزْ مِنْ أَنْ تُصْعِدَ مُحْرَقَاتِكَ فِي كُلِّ مَكَانٍ تَرَاهُ.
\par 14 بَل فِي المَكَانِ الذِي يَخْتَارُهُ الرَّبُّ فِي أَحَدِ أَسْبَاطِكَ. هُنَاكَ تُصْعِدُ مُحْرَقَاتِكَ وَهُنَاكَ تَعْمَلُ كُل مَا أَنَا أُوصِيكَ بِهِ.
\par 15 وَلكِنْ مِنْ كُلِّ مَا تَشْتَهِي نَفْسُكَ تَذْبَحُ وَتَأْكُلُ لحْماً فِي جَمِيعِ أَبْوَابِكَ حَسَبَ بَرَكَةِ الرَّبِّ إِلهِكَ التِي أَعْطَاكَ. النَّجِسُ وَالطَّاهِرُ يَأْكُلانِهِ كَالظَّبْيِ وَالإِيَّلِ.
\par 16 وَأَمَّا الدَّمُ فَلا تَأْكُلهُ. عَلى الأَرْضِ تَسْفِكُهُ كَالمَاءِ.
\par 17 لا يَحِلُّ لكَ أَنْ تَأْكُل فِي أَبْوَابِكَ عُشْرَ حِنْطَتِكَ وَخَمْرِكَ وَزَيْتِكَ وَلا أَبْكَارَ بَقَرِكَ وَغَنَمِكَ وَلا شَيْئاً مِنْ نُذُورِكَ التِي تَنْذُرُ وَنَوَافِلِكَ وَرَفَائِعِ يَدِكَ.
\par 18 بَل أَمَامَ الرَّبِّ إِلهِكَ تَأْكُلُهَا فِي المَكَانِ الذِي يَخْتَارُهُ الرَّبُّ إِلهُكَ أَنْتَ وَابْنُكَ وَابْنَتُكَ وَعَبْدُكَ وَأَمَتُكَ وَاللاوِيُّ الذِي فِي أَبْوَابِكَ وَتَفْرَحُ أَمَامَ الرَّبِّ إِلهِكَ بِكُلِّ مَا امْتَدَّتْ إِليْهِ يَدُكَ.
\par 19 اِحْتَرِزْ مِنْ أَنْ تَتْرُكَ اللاوِيَّ كُل أَيَّامِكَ عَلى أَرْضِكَ.
\par 20 «إِذَا وَسَّعَ الرَّبُّ إِلهُكَ تُخُومَكَ كَمَا كَلمَكَ وَقُلتَ: آكُلُ لحْماً لأَنَّ نَفْسَكَ تَشْتَهِي أَنْ تَأْكُل لحْماً. فَمِنْ كُلِّ مَا تَشْتَهِي نَفْسُكَ تَأْكُلُ لحْماً.
\par 21 إِذَا كَانَ المَكَانُ الذِي يَخْتَارُهُ الرَّبُّ إِلهُكَ لِيَضَعَ اسْمَهُ فِيهِ بَعِيداً عَنْكَ فَاذْبَحْ مِنْ بَقَرِكَ وَغَنَمِكَ التِي أَعْطَاكَ الرَّبُّ كَمَا أَوْصَيْتُكَ وَكُل فِي أَبْوَابِكَ مِنْ كُلِّ مَا اشْتَهَتْ نَفْسُكَ.
\par 22 كَمَا يُؤْكَلُ الظَّبْيُ وَالإِيَّلُ هَكَذَا تَأْكُلُهُ. النَّجِسُ وَالطَّاهِرُ يَأْكُلانِهِ سَوَاءً.
\par 23 لكِنِ احْتَرِزْ أَنْ لا تَأْكُل الدَّمَ لأَنَّ الدَّمَ هُوَ النَّفْسُ. فَلا تَأْكُلِ النَّفْسَ مَعَ اللحْمِ.
\par 24 لا تَأْكُلهُ. عَلى الأَرْضِ تَسْفِكُهُ كَالمَاءِ.
\par 25 لا تَأْكُلهُ لِيَكُونَ لكَ وَلأَوْلادِكَ مِنْ بَعْدِكَ خَيْرٌ إِذَا عَمِلتَ الحَقَّ فِي عَيْنَيِ الرَّبِّ.
\par 26 وَأَمَّا أَقْدَاسُكَ التِي لكَ وَنُذُورُكَ فَتَحْمِلُهَا وَتَذْهَبُ إِلى المَكَانِ الذِي يَخْتَارُهُ الرَّبُّ.
\par 27 فَتَعْمَلُ مُحْرَقَاتِكَ: اللحْمَ وَالدَّمَ عَلى مَذْبَحِ الرَّبِّ إِلهِكَ. وَأَمَّا ذَبَائِحُكَ فَيُسْفَكُ دَمُهَا عَلى مَذْبَحِ الرَّبِّ إِلهِكَ وَاللحْمُ تَأْكُلُهُ.
\par 28 اِحْفَظْ وَاسْمَعْ جَمِيعَ هَذِهِ الكَلِمَاتِ التِي أَنَا أُوصِيكَ بِهَا لِيَكُونَ لكَ وَلأَوْلادِكَ مِنْ بَعْدِكَ خَيْرٌ إِلى الأَبَدِ إِذَا عَمِلتَ الصَّالِحَ وَالحَقَّ فِي عَيْنَيِ الرَّبِّ إِلهِكَ.
\par 29 «مَتَى قَرَضَ الرَّبُّ إِلهُكَ مِنْ أَمَامِكَ الأُمَمَ الذِينَ أَنْتَ ذَاهِبٌ إِليْهِمْ لِتَرِثَهُمْ وَوَرِثْتَهُمْ وَسَكَنْتَ أَرْضَهُمْ
\par 30 فَاحْتَرِزْ مِنْ أَنْ تُصَادَ وَرَاءَهُمْ مِنْ بَعْدِ مَا بَادُوا مِنْ أَمَامِكَ وَمِنْ أَنْ تَسْأَل عَنْ آلِهَتِهِمْ: كَيْفَ عَبَدَ هَؤُلاءِ الأُمَمُ آلِهَتَهُمْ فَأَنَا أَيْضاً أَفْعَلُ هَكَذَا؟
\par 31 لا تَعْمَل هَكَذَا لِلرَّبِّ إِلهِكَ لأَنَّهُمْ قَدْ عَمِلُوا لآِلِهَتِهِمْ كُل رِجْسٍ لدَى الرَّبِّ مِمَّا يَكْرَهُهُ إِذْ أَحْرَقُوا حَتَّى بَنِيهِمْ وَبَنَاتِهِمْ بِالنَّارِ لآِلِهَتِهِمْ.
\par 32 كُلُّ الكَلامِ الذِي أُوصِيكُمْ بِهِ احْرِصُوا لِتَعْمَلُوهُ. لا تَزِدْ عَليْهِ وَلا تُنَقِّصْ مِنْهُ».

\chapter{13}

\par 1 «إِذَا قَامَ فِي وَسَطِكَ نَبِيٌّ أَوْ حَالِمٌ حُلماً وَأَعْطَاكَ آيَةً أَوْ أُعْجُوبَةً
\par 2 وَلوْ حَدَثَتِ الآيَةُ أَوِ الأُعْجُوبَةُ التِي كَلمَكَ عَنْهَا قَائِلاً: لِنَذْهَبْ وَرَاءَ آلِهَةٍ أُخْرَى لمْ تَعْرِفْهَا وَنَعْبُدْهَا
\par 3 فَلا تَسْمَعْ لِكَلامِ ذَلِكَ النَّبِيِّ أَوِ الحَالِمِ ذَلِكَ الحُلمَ لأَنَّ الرَّبَّ إِلهَكُمْ يَمْتَحِنُكُمْ لِيَعْلمَ هَل تُحِبُّونَ الرَّبَّ إِلهَكُمْ مِنْ كُلِّ قُلُوبِكُمْ وَمِنْ كُلِّ أَنْفُسِكُمْ.
\par 4 وَرَاءَ الرَّبِّ إِلهِكُمْ تَسِيرُونَ وَإِيَّاهُ تَتَّقُونَ وَوَصَايَاهُ تَحْفَظُونَ وَصَوْتَهُ تَسْمَعُونَ وَإِيَّاهُ تَعْبُدُونَ وَبِهِ تَلتَصِقُونَ.
\par 5 وَذَلِكَ النَّبِيُّ أَوِ الحَالِمُ ذَلِكَ الحُلمَ يُقْتَلُ لأَنَّهُ تَكَلمَ بِالزَّيْغِ مِنْ وَرَاءِ الرَّبِّ إِلهِكُمُ الذِي أَخْرَجَكُمْ مِنْ أَرْضِ مِصْرَ وَفَدَاكُمْ مِنْ بَيْتِ العُبُودِيَّةِ لِيُطَوِّحَكُمْ عَنِ الطَّرِيقِ التِي أَمَرَكُمُ الرَّبُّ إِلهُكُمْ أَنْ تَسْلُكُوا فِيهَا. فَتَنْزِعُونَ الشَّرَّ مِنْ بَيْنِكُمْ.
\par 6 «وَإِذَا أَغْوَاكَ سِرّاً أَخُوكَ ابْنُ أُمِّكَ أَوِ ابْنُكَ أَوِ ابْنَتُكَ أَوِ امْرَأَةُ حِضْنِكَ أَوْ صَاحِبُكَ الذِي مِثْلُ نَفْسِكَ قَائِلاً: نَذْهَبُ وَنَعْبُدُ آلِهَةً أُخْرَى لمْ تَعْرِفْهَا أَنْتَ وَلا آبَاؤُكَ
\par 7 مِنْ آلِهَةِ الشُّعُوبِ الذِينَ حَوْلكَ القَرِيبِينَ مِنْكَ أَوِ البَعِيدِينَ عَنْكَ مِنْ أَقْصَاءِ الأَرْضِ إِلى أَقْصَائِهَا
\par 8 فَلا تَرْضَ مِنْهُ وَلا تَسْمَعْ لهُ وَلا تُشْفِقْ عَيْنُكَ عَليْهِ وَلا تَرِقَّ لهُ وَلا تَسْتُرْهُ
\par 9 بَل قَتْلاً تَقْتُلُهُ. يَدُكَ تَكُونُ عَليْهِ أَوَّلاً لِقَتْلِهِ ثُمَّ أَيْدِي جَمِيعِ الشَّعْبِ أَخِيراً.
\par 10 تَرْجُمُهُ بِالحِجَارَةِ حَتَّى يَمُوتَ لأَنَّهُ التَمَسَ أَنْ يُطَوِّحَكَ عَنِ الرَّبِّ إِلهِكَ الذِي أَخْرَجَكَ مِنْ أَرْضِ مِصْرَ مِنْ بَيْتِ العُبُودِيَّةِ.
\par 11 فَيَسْمَعُ جَمِيعُ إِسْرَائِيل وَيَخَافُونَ وَلا يَعُودُونَ يَعْمَلُونَ مِثْل هَذَا الأَمْرِ الشِّرِّيرِ فِي وَسَطِكَ.
\par 12 «إِنْ سَمِعْتَ عَنْ إِحْدَى مُدُنِكَ التِي يُعْطِيكَ الرَّبُّ إِلهُكَ لِتَسْكُنَ فِيهَا قَوْلاً:
\par 13 قَدْ خَرَجَ أُنَاسٌ بَنُو لئِيمٍ مِنْ وَسَطِكَ وَطَوَّحُوا سُكَّانَ مَدِينَتِهِمْ قَائِلِينَ: نَذْهَبُ وَنَعْبُدُ آلِهَةً أُخْرَى لمْ تَعْرِفُوهَا.
\par 14 وَفَحَصْتَ وَفَتَّشْتَ وَسَأَلتَ جَيِّداً وَإِذَا الأَمْرُ صَحِيحٌ وَأَكِيدٌ قَدْ عُمِل ذَلِكَ الرِّجْسُ فِي وَسَطِكَ
\par 15 فَضَرْباً تَضْرِبُ سُكَّانَ تِلكَ المَدِينَةِ بِحَدِّ السَّيْفِ وَتُحَرِّمُهَا بِكُلِّ مَا فِيهَا مَعَ بَهَائِمِهَا بِحَدِّ السَّيْفِ.
\par 16 تَجْمَعُ كُل أَمْتِعَتِهَا إِلى وَسَطِ سَاحَتِهَا وَتُحْرِقُ بِالنَّارِ المَدِينَةَ وَكُل أَمْتِعَتِهَا كَامِلةً لِلرَّبِّ إِلهِكَ فَتَكُونُ تَلاًّ إِلى الأَبَدِ لا تُبْنَى بَعْدُ.
\par 17 وَلا يَلتَصِقْ بِيَدِكَ شَيْءٌ مِنَ المُحَرَّمِ لِيَرْجِعَ الرَّبُّ مِنْ حُمُوِّ غَضَبِهِ وَيُعْطِيَكَ رَحْمَةً. يَرْحَمُكَ وَيُكَثِّرُكَ كَمَا حَلفَ لآِبَائِكَ
\par 18 إِذَا سَمِعْتَ لِصَوْتِ الرَّبِّ إِلهِكَ لِتَحْفَظَ جَمِيعَ وَصَايَاهُ التِي أَنَا أُوصِيكَ بِهَا اليَوْمَ لِتَعْمَل الحَقَّ فِي عَيْنَيِ الرَّبِّ إِلهِكَ».

\chapter{14}

\par 1 «أَنْتُمْ أَوْلادٌ لِلرَّبِّ إِلهِكُمْ. لا تَخْمِشُوا أَجْسَامَكُمْ وَلا تَجْعَلُوا قَرْعَةً بَيْنَ أَعْيُنِكُمْ لأَجْلِ مَيِّتٍ.
\par 2 لأَنَّكَ شَعْبٌ مُقَدَّسٌ لِلرَّبِّ إِلهِكَ وَقَدِ اخْتَارَكَ الرَّبُّ لِتَكُونَ لهُ شَعْباً خَاصّاً فَوْقَ جَمِيعِ الشُّعُوبِ الذِينَ عَلى وَجْهِ الأَرْضِ.
\par 3 «لا تَأْكُل رِجْساً مَا.
\par 4 هَذِهِ هِيَ البَهَائِمُ التِي تَأْكُلُونَهَا: البَقَرُ وَالضَّأْنُ وَالمَعْزُ
\par 5 وَالإِيَّلُ وَالظَّبْيُ وَاليَحْمُورُ وَالوَعْلُ وَالرِّئْمُ وَالثَّيْتَلُ وَالمَهَاةُ.
\par 6 وَكُلُّ بَهِيمَةٍ مِنَ البَهَائِمِ تَشُقُّ ظِلفاً وَتَقْسِمُهُ ظِلفَيْنِ وَتَجْتَرُّ فَإِيَّاهَا تَأْكُلُونَ.
\par 7 إِلا هَذِهِ فَلا تَأْكُلُوهَا مِمَّا يَجْتَرُّ وَمِمَّا يَشُقُّ الظِّلفَ المُنْقَسِمَ: الجَمَلُ وَالأَرْنَبُ وَالوَبْرُ لأَنَّهَا تَجْتَرُّ لكِنَّهَا لا تَشُقُّ ظِلفاً فَهِيَ نَجِسَةٌ لكُمْ.
\par 8 وَالخِنْزِيرُ لأَنَّهُ يَشُقُّ الظِّلفَ لكِنَّهُ لا يَجْتَرُّ فَهُوَ نَجِسٌ لكُمْ. فَمِنْ لحْمِهَا لا تَأْكُلُوا وَجُثَثَهَا لا تَلمِسُوا.
\par 9 «وَهَذَا تَأْكُلُونَهُ مِنْ كُلِّ مَا فِي المِيَاهِ: كُلُّ مَا لهُ زَعَانِفُ وَحَرْشَفٌ تَأْكُلُونَهُ.
\par 10 لكِنْ كُلُّ مَا ليْسَ لهُ زَعَانِفُ وَحَرْشَفٌ لا تَأْكُلُوهُ. إِنَّهُ نَجِسٌ لكُمْ.
\par 11 «كُل طَيْرٍ طَاهِرٍ تَأْكُلُونَ.
\par 12 وَهَذَا مَا لا تَأْكُلُونَ مِنْهُ: النَّسْرُ وَالأَنُوقُ وَالعُقَابُ
\par 13 وَالحِدَأَةُ وَالبَاشِقُ وَالشَّاهِينُ عَلى أَجْنَاسِهِ
\par 14 وَكُلُّ غُرَابٍ عَلى أَجْنَاسِهِ
\par 15 وَالنَّعَامَةُ وَالظَّلِيمُ وَالسَّأَفُ وَالبَازُ عَلى أَجْنَاسِهِ
\par 16 وَالبُومُ وَالكُرْكِيُّ وَالبَجَعُ
\par 17 وَالقُوقُ وَالرَّخَمُ وَالغَوَّاصُ
\par 18 وَاللقْلقُ وَالبَبْغَاءُ عَلى أَجْنَاسِهِ وَالهُدْهُدُ وَالخُفَّاشُ.
\par 19 وَكُلُّ دَبِيبِ الطَّيْرِ نَجِسٌ لكُمْ. لا يُؤْكَلُ.
\par 20 كُل طَيْرٍ طَاهِرٍ تَأْكُلُونَ.
\par 21 «لا تَأْكُلُوا جُثَّةً مَا. تُعْطِيهَا لِلغَرِيبِ الذِي فِي أَبْوَابِكَ فَيَأْكُلُهَا أَوْ يَبِيعُهَا لأَجْنَبِيٍّ لأَنَّكَ شَعْبٌ مُقَدَّسٌ لِلرَّبِّ إِلهِكَ. لا تَطْبُخْ جَدْياً بِلبَنِ أُمِّهِ.
\par 22 «تَعْشِيراً تُعَشِّرُ كُل مَحْصُولِ زَرْعِكَ الذِي يَخْرُجُ مِنَ الحَقْلِ سَنَةً بِسَنَةٍ.
\par 23 وَتَأْكُلُ أَمَامَ الرَّبِّ إِلهِكَ فِي المَكَانِ الذِي يَخْتَارُهُ لِيُحِل اسْمَهُ فِيهِ عُشْرَ حِنْطَتِكَ وَخَمْرِكَ وَزَيْتِكَ وَأَبْكَارِ بَقَرِكَ وَغَنَمِكَ لِتَتَعَلمَ أَنْ تَتَّقِيَ الرَّبَّ إِلهَكَ كُل الأَيَّامِ.
\par 24 وَلكِنْ إِذَا طَال عَليْكَ الطَّرِيقُ حَتَّى لا تَقْدِرَ أَنْ تَحْمِلهُ. إِذَا كَانَ بَعِيداً عَليْكَ المَكَانُ الذِي يَخْتَارُهُ الرَّبُّ إِلهُكَ لِيَجْعَل اسْمَهُ فِيهِ إِذْ يُبَارِكُكَ الرَّبُّ إِلهُكَ
\par 25 فَبِعْهُ بِفِضَّةٍ وَصُرَّ الفِضَّةَ فِي يَدِكَ وَاذْهَبْ إِلى المَكَانِ الذِي يَخْتَارُهُ الرَّبُّ إِلهُكَ
\par 26 وَأَنْفِقِ الفِضَّةَ فِي كُلِّ مَا تَشْتَهِي نَفْسُكَ فِي البَقَرِ وَالغَنَمِ وَالخَمْرِ وَالمُسْكِرِ وَكُلِّ مَا تَطْلُبُ مِنْكَ نَفْسُكَ وَكُل هُنَاكَ أَمَامَ الرَّبِّ إِلهِكَ وَافْرَحْ أَنْتَ وَبَيْتُكَ.
\par 27 وَاللاوِيُّ الذِي فِي أَبْوَابِكَ لا تَتْرُكْهُ لأَنَّهُ ليْسَ لهُ قِسْمٌ وَلا نَصِيبٌ مَعَكَ.
\par 28 «فِي آخِرِ ثَلاثِ سِنِينَ تُخْرِجُ كُل عُشْرِ مَحْصُولِكَ فِي تِلكَ السَّنَةِ وَتَضَعُهُ فِي أَبْوَابِكَ.
\par 29 فَيَأْتِي اللاوِيُّ لأَنَّهُ ليْسَ لهُ قِسْمٌ وَلا نَصِيبٌ مَعَكَ وَالغَرِيبُ وَاليَتِيمُ وَالأَرْمَلةُ الذِينَ فِي أَبْوَابِكَ وَيَأْكُلُونَ وَيَشْبَعُونَ لِيُبَارِكَكَ الرَّبُّ إِلهُكَ فِي كُلِّ عَمَلِ يَدِكَ الذِي تَعْمَلُ».

\chapter{15}

\par 1 «فِي آخِرِ سَبْعِ سِنِينَ تَعْمَلُ إِبْرَاءً.
\par 2 وَهَذَا هُوَ حُكْمُ الإِبْرَاءِ: يُبْرِئُ كُلُّ صَاحِبِ دَيْنٍ يَدَهُ مِمَّا أَقْرَضَ صَاحِبَهُ. لا يُطَالِبُ صَاحِبَهُ وَلا أَخَاهُ لأَنَّهُ قَدْ نُودِيَ بِإِبْرَاءٍ لِلرَّبِّ.
\par 3 الأَجْنَبِيَّ تُطَالِبُ وَأَمَّا مَا كَانَ لكَ عِنْدَ أَخِيكَ فَتُبْرِئُهُ يَدُكَ مِنْهُ.
\par 4 إِلا إِنْ لمْ يَكُنْ فِيكَ فَقِيرٌ. لأَنَّ الرَّبَّ إِنَّمَا يُبَارِكُكَ فِي الأَرْضِ التِي يُعْطِيكَ الرَّبُّ إِلهُكَ نَصِيباً لِتَمْتَلِكَهَا.
\par 5 إِذَا سَمِعْتَ صَوْتَ الرَّبِّ إِلهِكَ لِتَحْفَظَ وَتَعْمَل كُل هَذِهِ الوَصَايَا التِي أَنَا أُوصِيكَ اليَوْمَ
\par 6 يُبَارِكُكَ الرَّبُّ إِلهُكَ كَمَا قَال لكَ. فَتُقْرِضُ أُمَماً كَثِيرَةً وَأَنْتَ لا تَقْتَرِضُ وَتَتَسَلطُ عَلى أُمَمٍ كَثِيرَةٍ وَهُمْ عَليْكَ لا يَتَسَلطُونَ.
\par 7 «إِنْ كَانَ فِيكَ فَقِيرٌ أَحَدٌ مِنْ إِخْوَتِكَ فِي أَحَدِ أَبْوَابِكَ فِي أَرْضِكَ التِي يُعْطِيكَ الرَّبُّ إِلهُكَ فَلا تُقَسِّ قَلبَكَ وَلا تَقْبِضْ يَدَكَ عَنْ أَخِيكَ الفَقِيرِ
\par 8 بَلِ افْتَحْ يَدَكَ لهُ وَأَقْرِضْهُ مِقْدَارَ مَا يَحْتَاجُ إِليْهِ.
\par 9 احْتَرِزْ مِنْ أَنْ يَكُونَ مَعَ قَلبِكَ كَلامٌ لئِيمٌ قَائِلاً: قَدْ قَرُبَتِ السَّنَةُ السَّابِعَةُ سَنَةُ الإِبْرَاءِ وَتَسُوءُ عَيْنُكَ بِأَخِيكَ الفَقِيرِ وَلا تُعْطِيهِ فَيَصْرُخَ عَليْكَ إِلى الرَّبِّ فَتَكُونُ عَليْكَ خَطِيَّةٌ.
\par 10 أَعْطِهِ وَلا يَسُوءُ قَلبُكَ عِنْدَمَا تُعْطِيهِ لأَنَّهُ بِسَبَبِ هَذَا الأَمْرِ يُبَارِكُكَ الرَّبُّ إِلهُكَ فِي كُلِّ أَعْمَالِكَ وَجَمِيعِ مَا تَمْتَدُّ إِليْهِ يَدُكَ.
\par 11 لأَنَّهُ لا تُفْقَدُ الفُقَرَاءُ مِنَ الأَرْضِ. لِذَلِكَ أَنَا أُوصِيكَ قَائِلاً: افْتَحْ يَدَكَ لأَخِيكَ المِسْكِينِ وَالفَقِيرِ فِي أَرْضِكَ.
\par 12 «إِذَا بِيعَ لكَ أَخُوكَ العِبْرَانِيُّ أَوْ أُخْتُكَ العِبْرَانِيَّةُ وَخَدَمَكَ سِتَّ سِنِينَ فَفِي السَّنَةِ السَّابِعَةِ تُطْلِقُهُ حُرّاً مِنْ عِنْدِكَ.
\par 13 وَحِينَ تُطْلِقُهُ حُرّاً مِنْ عِنْدِكَ لا تُطْلِقُهُ فَارِغاً.
\par 14 تُزَوِّدُهُ مِنْ غَنَمِكَ وَمِنْ بَيْدَرِكَ وَمِنْ مَعْصَرَتِكَ. كَمَا بَارَكَكَ الرَّبُّ إِلهُكَ تُعْطِيهِ.
\par 15 وَاذْكُرْ أَنَّكَ كُنْتَ عَبْداً فِي أَرْضِ مِصْرَ فَفَدَاكَ الرَّبُّ إِلهُكَ. لِذَلِكَ أَنَا أُوصِيكَ بِهَذَا الأَمْرِ اليَوْمَ.
\par 16 وَلكِنْ إِذَا قَال لكَ: لا أَخْرُجُ مِنْ عِنْدِكَ لأَنَّهُ قَدْ أَحَبَّكَ وَبَيْتَكَ إِذْ كَانَ لهُ خَيْرٌ عِنْدَكَ
\par 17 فَخُذِ المِخْرَزَ وَاجْعَلهُ فِي أُذُنِهِ وَفِي البَابِ فَيَكُونَ لكَ عَبْداً مُؤَبَّداً. وَهَكَذَا تَفْعَلُ لأَمَتِكَ أَيْضاً.
\par 18 لا يَصْعُبْ عَليْكَ أَنْ تُطْلِقَهُ حُرّاً مِنْ عِنْدِكَ لأَنَّهُ ضِعْفَيْ أُجْرَةِ الأَجِيرِ خَدَمَكَ سِتَّ سِنِينَ. فَيُبَارِكُكَ الرَّبُّ إِلهُكَ فِي كُلِّ مَا تَعْمَلُ.
\par 19 «كُلُّ بِكْرٍ ذَكَرٍ يُولدُ مِنْ بَقَرِكَ وَمِنْ غَنَمِكَ تُقَدِّسُهُ لِلرَّبِّ إِلهِكَ. لا تَشْتَغِل عَلى بِكْرِ بَقَرِكَ وَلا تَجُزَّ بِكْرَ غَنَمِكَ.
\par 20 أَمَامَ الرَّبِّ إِلهِكَ تَأْكُلُهُ سَنَةً بِسَنَةٍ فِي المَكَانِ الذِي يَخْتَارُهُ الرَّبُّ أَنْتَ وَبَيْتُكَ.
\par 21 وَلكِنْ إِذَا كَانَ فِيهِ عَيْبٌ عَرَجٌ أَوْ عَمىً عَيْبٌ مَا رَدِيءٌ فَلا تَذْبَحْهُ لِلرَّبِّ إِلهِكَ.
\par 22 فِي أَبْوَابِكَ تَأْكُلُهُ. النَّجِسُ وَالطَّاهِرُ سَوَاءً كَالظَّبْيِ وَالإِيَّلِ.
\par 23 وَأَمَّا دَمُهُ فَلا تَأْكُلُهُ. عَلى الأَرْضِ تَسْفِكُهُ كَالمَاءِ».

\chapter{16}

\par 1 «اِحْفَظْ شَهْرَ أَبِيبَ وَاعْمَل فِصْحاً لِلرَّبِّ إِلهِكَ لأَنَّهُ فِي شَهْرِ أَبِيبَ أَخْرَجَكَ الرَّبُّ إِلهُكَ مِنْ مِصْرَ ليْلاً.
\par 2 فَتَذْبَحُ الفِصْحَ لِلرَّبِّ إِلهِكَ غَنَماً وَبَقَراً فِي المَكَانِ الذِي يَخْتَارُهُ الرَّبُّ لِيُحِل اسْمَهُ فِيهِ.
\par 3 لا تَأْكُل عَليْهِ خَمِيراً. سَبْعَةَ أَيَّامٍ تَأْكُلُ عَليْهِ فَطِيراً خُبْزَ المَشَقَّةِ (لأَنَّكَ بِعَجَلةٍ خَرَجْتَ مِنْ أَرْضِ مِصْرَ) لِتَذْكُرَ يَوْمَ خُرُوجِكَ مِنْ أَرْضِ مِصْرَ كُل أَيَّامِ حَيَاتِكَ.
\par 4 وَلا يُرَ عِنْدَكَ خَمِيرٌ فِي جَمِيعِ تُخُومِكَ سَبْعَةَ أَيَّامٍ وَلا يَبِتْ شَيْءٌ مِنَ اللحْمِ الذِي تَذْبَحُ مَسَاءً فِي اليَوْمِ الأَوَّلِ إِلى الغَدِ.
\par 5 لا يَحِلُّ لكَ أَنْ تَذْبَحَ الفِصْحَ فِي أَحَدِ أَبْوَابِكَ التِي يُعْطِيكَ الرَّبُّ إِلهُكَ
\par 6 بَل فِي المَكَانِ الذِي يَخْتَارُهُ الرَّبُّ إِلهُكَ لِيُحِل اسْمَهُ فِيهِ. هُنَاكَ تَذْبَحُ الفِصْحَ مَسَاءً نَحْوَ غُرُوبِ الشَّمْسِ فِي مِيعَادِ خُرُوجِكَ مِنْ مِصْرَ
\par 7 وَتَطْبُخُ وَتَأْكُلُ فِي المَكَانِ الذِي يَخْتَارُهُ الرَّبُّ إِلهُكَ ثُمَّ تَنْصَرِفُ فِي الغَدِ وَتَذْهَبُ إِلى خِيَامِكَ.
\par 8 سِتَّةَ أَيَّامٍ تَأْكُلُ فَطِيراً وَفِي اليَوْمِ السَّابِعِ اعْتِكَافٌ لِلرَّبِّ إِلهِكَ. لا تَعْمَل فِيهِ عَمَلاً.
\par 9 «سَبْعَةَ أَسَابِيعَ تَحْسِبُ لكَ. مِنِ ابْتِدَاءِ المِنْجَلِ فِي الزَّرْعِ تَبْتَدِئُ أَنْ تَحْسِبَ سَبْعَةَ أَسَابِيعَ.
\par 10 وَتَعْمَلُ عِيدَ أَسَابِيعَ لِلرَّبِّ إِلهِكَ عَلى قَدْرِ مَا تَسْمَحُ يَدُكَ أَنْ تُعْطِيَ كَمَا يُبَارِكُكَ الرَّبُّ إِلهُكَ.
\par 11 وَتَفْرَحُ أَمَامَ الرَّبِّ إِلهِكَ أَنْتَ وَابْنُكَ وَابْنَتُكَ وَعَبْدُكَ وَأَمَتُكَ وَاللاوِيُّ الذِي فِي أَبْوَابِكَ وَالغَرِيبُ وَاليَتِيمُ وَالأَرْمَلةُ الذِينَ فِي وَسْطِكَ فِي المَكَانِ الذِي يَخْتَارُهُ الرَّبُّ إِلهُكَ لِيُحِل اسْمَهُ فِيهِ.
\par 12 وَتَذْكُرُ أَنَّكَ كُنْتَ عَبْداً فِي مِصْرَ وَتَحْفَظُ وَتَعْمَلُ هَذِهِ الفَرَائِضَ.
\par 13 «تَعْمَلُ لِنَفْسِكَ عِيدَ المَظَالِّ سَبْعَةَ أَيَّامٍ عِنْدَمَا تَجْمَعُ مِنْ بَيْدَرِكَ وَمِنْ مِعْصَرَتِكَ.
\par 14 وَتَفْرَحُ فِي عِيدِكَ أَنْتَ وَابْنُكَ وَابْنَتُكَ وَعَبْدُكَ وَأَمَتُكَ وَاللاوِيُّ وَالغَرِيبُ وَاليَتِيمُ وَالأَرْمَلةُ الذِينَ فِي أَبْوَابِكَ.
\par 15 سَبْعَةَ أَيَّامٍ تُعَيِّدُ لِلرَّبِّ إِلهِكَ فِي المَكَانِ الذِي يَخْتَارُهُ الرَّبُّ لأَنَّ الرَّبَّ إِلهَكَ يُبَارِكُكَ فِي كُلِّ مَحْصُولِكَ وَفِي كُلِّ عَمَلِ يَدَيْكَ فَلا تَكُونُ إِلا فَرِحاً.
\par 16 «ثَلاثَ مَرَّاتٍ فِي السَّنَةِ يَحْضُرُ جَمِيعُ ذُكُورِكَ أَمَامَ الرَّبِّ إِلهِكَ فِي المَكَانِ الذِي يَخْتَارُهُ فِي عِيدِ الفَطِيرِ وَعِيدِ الأَسَابِيعِ وَعِيدِ المَظَالِّ. وَلا يَحْضُرُوا أَمَامَ الرَّبِّ فَارِغِينَ.
\par 17 كُلُّ وَاحِدٍ حَسْبَمَا تُعْطِي يَدُهُ كَبَرَكَةِ الرَّبِّ إِلهِكَ التِي أَعْطَاكَ.
\par 18 «قُضَاةً وَعُرَفَاءَ تَجْعَلُ لكَ فِي جَمِيعِ أَبْوَابِكَ التِي يُعْطِيكَ الرَّبُّ إِلهُكَ حَسَبَ أَسْبَاطِكَ فَيَقْضُونَ لِلشَّعْبِ قَضَاءً عَادِلاً.
\par 19 لا تُحَرِّفِ القَضَاءَ وَلا تَنْظُرْ إِلى الوُجُوهِ وَلا تَأْخُذْ رَشْوَةً لأَنَّ الرَّشْوَةَ تُعْمِي أَعْيُنَ الحُكَمَاءِ وَتُعَوِّجُ كَلامَ الصِّدِّيقِينَ.
\par 20 العَدْل العَدْل تَتَّبِعُ لِكَيْ تَحْيَا وَتَمْتَلِكَ الأَرْضَ التِي يُعْطِيكَ الرَّبُّ إِلهُكَ.
\par 21 «لا تَنْصُبْ لِنَفْسِكَ سَارِيَةً مِنْ شَجَرَةٍ مَا بِجَانِبِ مَذْبَحِ الرَّبِّ إِلهِكَ الذِي تَصْنَعُهُ لكَ
\par 22 وَلا تُقِمْ لكَ نَصَباً. الشَّيْءَ الذِي يُبْغِضُهُ الرَّبُّ إِلهُكَ».

\chapter{17}

\par 1 «لا تَذْبَحْ لِلرَّبِّ إِلهِكَ ثَوْراً أَوْ شَاةً فِيهِ عَيْبٌ شَيْءٌ مَا رَدِيءٌ لأَنَّ ذَلِكَ رِجْسٌ لدَى الرَّبِّ إِلهِكَ.
\par 2 «إِذَا وُجِدَ فِي وَسَطِكَ فِي أَحَدِ أَبْوَابِكَ التِي يُعْطِيكَ الرَّبُّ إِلهُكَ رَجُلٌ أَوِ امْرَأَةٌ يَفْعَلُ شَرّاً فِي عَيْنَيِ الرَّبِّ إِلهِكَ بِتَجَاوُزِ عَهْدِهِ
\par 3 وَيَذْهَبُ وَيَعْبُدُ آلِهَةً أُخْرَى وَيَسْجُدُ لهَا أَوْ لِلشَّمْسِ أَوْ لِلقَمَرِ أَوْ لِكُلٍّ مِنْ جُنْدِ السَّمَاءِ - الشَّيْءَ الذِي لمْ أُوصِ بِهِ
\par 4 وَأُخْبِرْتَ وَسَمِعْتَ وَفَحَصْتَ جَيِّداً وَإِذَا الأَمْرُ صَحِيحٌ أَكِيدٌ. قَدْ عُمِل ذَلِكَ الرِّجْسُ فِي إِسْرَائِيل
\par 5 فَأَخْرِجْ ذَلِكَ الرَّجُل أَوْ تِلكَ المَرْأَةَ الذِي فَعَل ذَلِكَ الأَمْرَ الشِّرِّيرَ إِلى أَبْوَابِكَ الرَّجُل أَوِ المَرْأَةَ وَارْجُمْهُ بِالحِجَارَةِ حَتَّى يَمُوتَ.
\par 6 عَلى فَمِ شَاهِدَيْنِ أَوْ ثَلاثَةِ شُهُودٍ يُقْتَلُ الذِي يُقْتَلُ. لا يُقْتَل عَلى فَمِ شَاهِدٍ وَاحِدٍ.
\par 7 أَيْدِي الشُّهُودِ تَكُونُ عَليْهِ أَوَّلاً لِقَتْلِهِ ثُمَّ أَيْدِي جَمِيعِ الشَّعْبِ أَخِيراً فَتَنْزِعُ الشَّرَّ مِنْ وَسَطِكَ.
\par 8 «إِذَا عَسِرَ عَليْكَ أَمْرٌ فِي القَضَاءِ بَيْنَ دَمٍ وَدَمٍ أَوْ بَيْنَ دَعْوَى وَدَعْوَى أَوْ بَيْنَ ضَرْبَةٍ وَضَرْبَةٍ مِنْ أُمُورِ الخُصُومَاتِ فِي أَبْوَابِكَ فَقُمْ وَاصْعَدْ إِلى المَكَانِ الذِي يَخْتَارُهُ الرَّبُّ إِلهُكَ
\par 9 وَاذْهَبْ إِلى الكَهَنَةِ اللاوِيِّينَ وَإِلى القَاضِي الذِي يَكُونُ فِي تِلكَ الأَيَّامِ وَاسْأَل فَيُخْبِرُوكَ بِأَمْرِ القَضَاءِ.
\par 10 فَتَعْمَلُ حَسَبَ الأَمْرِ الذِي يُخْبِرُونَكَ بِهِ مِنْ ذَلِكَ المَكَانِ الذِي يَخْتَارُهُ الرَّبُّ وَتَحْرِصُ أَنْ تَعْمَل حَسَبَ كُلِّ مَا يُعَلِّمُونَكَ.
\par 11 حَسَبَ الشَّرِيعَةِ التِي يُعَلِّمُونَكَ وَالقَضَاءِ الذِي يَقُولُونَهُ لكَ تَعْمَلُ. لا تَحِدْ عَنِ الأَمْرِ الذِي يُخْبِرُونَكَ بِهِ يَمِيناً أَوْ شِمَالاً.
\par 12 وَالرَّجُلُ الذِي يَعْمَلُ بِطُغْيَانٍ فَلا يَسْمَعُ لِلكَاهِنِ الوَاقِفِ هُنَاكَ لِيَخْدِمَ الرَّبَّ إِلهَكَ أَوْ لِلقَاضِي يُقْتَلُ ذَلِكَ الرَّجُلُ فَتَنْزِعُ الشَّرَّ مِنْ إِسْرَائِيل.
\par 13 فَيَسْمَعُ جَمِيعُ الشَّعْبِ وَيَخَافُونَ وَلا يَطْغُونَ بَعْدُ.
\par 14 «مَتَى أَتَيْتَ إِلى الأَرْضِ التِي يُعْطِيكَ الرَّبُّ إِلهُكَ وَامْتَلكْتَهَا وَسَكَنْتَ فِيهَا فَإِنْ قُلتَ: أَجْعَلُ عَليَّ مَلِكاً كَجَمِيعِ الأُمَمِ الذِينَ حَوْلِي.
\par 15 فَإِنَّكَ تَجْعَلُ عَليْكَ مَلِكاً الذِي يَخْتَارُهُ الرَّبُّ إِلهُكَ. مِنْ وَسَطِ إِخْوَتِكَ تَجْعَلُ عَليْكَ مَلِكاً. لا يَحِلُّ لكَ أَنْ تَجْعَل عَليْكَ رَجُلاً أَجْنَبِيّاً ليْسَ هُوَ أَخَاكَ.
\par 16 وَلكِنْ لا يُكَثِّرْ لهُ الخَيْل وَلا يَرُدُّ الشَّعْبَ إِلى مِصْرَ لِكَيْ يُكَثِّرَ الخَيْل وَالرَّبُّ قَدْ قَال لكُمْ: لا تَعُودُوا تَرْجِعُونَ فِي هَذِهِ الطَّرِيقِ أَيْضاً.
\par 17 وَلا يُكَثِّرْ لهُ نِسَاءً لِئَلا يَزِيغَ قَلبُهُ. وَفِضَّةً وَذَهَباً لا يُكَثِّرْ لهُ كَثِيراً.
\par 18 وَعِنْدَمَا يَجْلِسُ عَلى كُرْسِيِّ مَمْلكَتِهِ يَكْتُبُ لِنَفْسِهِ نُسْخَةً مِنْ هَذِهِ الشَّرِيعَةِ فِي كِتَابٍ مِنْ عِنْدِ الكَهَنَةِ اللاوِيِّينَ
\par 19 فَتَكُونُ مَعَهُ وَيَقْرَأُ فِيهَا كُل أَيَّامِ حَيَاتِهِ لِيَتَعَلمَ أَنْ يَتَّقِيَ الرَّبَّ إِلهَهُ وَيَحْفَظَ جَمِيعَ كَلِمَاتِ هَذِهِ الشَّرِيعَةِ وَهَذِهِ الفَرَائِضَ لِيَعْمَل بِهَا
\par 20 لِئَلا يَرْتَفِعَ قَلبُهُ عَلى إِخْوَتِهِ وَلِئَلا يَحِيدَ عَنِ الوَصِيَّةِ يَمِيناً أَوْ شِمَالاً. لِكَيْ يُطِيل الأَيَّامَ عَلى مَمْلكَتِهِ هُوَ وَبَنُوهُ فِي وَسَطِ إِسْرَائِيل».

\chapter{18}

\par 1 «لا يَكُونُ لِلكَهَنَةِ اللاوِيِّينَ كُلِّ سِبْطِ لاوِي قِسْمٌ وَلا نَصِيبٌ مَعَ إِسْرَائِيل. يَأْكُلُونَ وَقَائِدَ الرَّبِّ وَنَصِيبَهُ.
\par 2 فَلا يَكُونُ لهُ نَصِيبٌ فِي وَسَطِ إِخْوَتِهِ. الرَّبُّ هُوَ نَصِيبُهُ كَمَا قَال لهُ.
\par 3 «وَهَذَا يَكُونُ حَقُّ الكَهَنَةِ مِنَ الشَّعْبِ مِنَ الذِينَ يَذْبَحُونَ الذَّبَائِحَ بَقَراً كَانَتْ أَوْ غَنَماً. يُعْطُونَ الكَاهِنَ السَّاعِدَ وَالفَكَّيْنِ وَالكِرْشَ.
\par 4 وَتُعْطِيهِ أَوَّل حِنْطَتِكَ وَخَمْرِكَ وَزَيْتِكَ وَأَوَّل جَزَازِ غَنَمِكَ.
\par 5 لأَنَّ الرَّبَّ إِلهَكَ قَدِ اخْتَارَهُ مِنْ جَمِيعِ أَسْبَاطِكَ لِيَقِفَ وَيَخْدِمَ بِاسْمِ الرَّبِّ هُوَ وَبَنُوهُ كُل الأَيَّامِ.
\par 6 «وَإِذَا جَاءَ لاوِيٌّ مِنْ أَحَدِ أَبْوَابِكَ مِنْ جَمِيعِ إِسْرَائِيل حَيْثُ هُوَ مُتَغَرِّبٌ وَجَاءَ بِكُلِّ رَغْبَةِ نَفْسِهِ إِلى المَكَانِ الذِي يَخْتَارُهُ الرَّبُّ
\par 7 وَخَدَمَ بِاسْمِ الرَّبِّ إِلهِكَ مِثْل جَمِيعِ إِخْوَتِهِ اللاوِيِّينَ الوَاقِفِينَ هُنَاكَ أَمَامَ الرَّبِّ
\par 8 يَأْكُلُونَ أَقْسَاماً مُتَسَاوِيَةً عَدَا مَا يَبِيعُهُ عَنْ آبَائِهِ.
\par 9 «مَتَى دَخَلتَ الأَرْضَ التِي يُعْطِيكَ الرَّبُّ إِلهُكَ لا تَتَعَلمْ أَنْ تَفْعَل مِثْل رِجْسِ أُولئِكَ الأُمَمِ.
\par 10 لا يُوجَدْ فِيكَ مَنْ يُجِيزُ ابْنَهُ أَوِ ابْنَتَهُ فِي النَّارِ وَلا مَنْ يَعْرُفُ عِرَافَةً وَلا عَائِفٌ وَلا مُتَفَائِلٌ وَلا سَاحِرٌ
\par 11 وَلا مَنْ يَرْقِي رُقْيَةً وَلا مَنْ يَسْأَلُ جَانّاً أَوْ تَابِعَةً وَلا مَنْ يَسْتَشِيرُ المَوْتَى.
\par 12 لأَنَّ كُل مَنْ يَفْعَلُ ذَلِكَ مَكْرُوهٌ عِنْدَ الرَّبِّ. وَبِسَبَبِ هَذِهِ الأَرْجَاسِ الرَّبُّ إِلهُكَ طَارِدُهُمْ مِنْ أَمَامِكَ.
\par 13 تَكُونُ كَامِلاً لدَى الرَّبِّ إِلهِكَ.
\par 14 إِنَّ هَؤُلاءِ الأُمَمَ الذِينَ تَخْلُفُهُمْ يَسْمَعُونَ لِلعَائِفِينَ وَالعَرَّافِينَ. وَأَمَّا أَنْتَ فَلمْ يَسْمَحْ لكَ الرَّبُّ إِلهُكَ هَكَذَا.
\par 15 «يُقِيمُ لكَ الرَّبُّ إِلهُكَ نَبِيّاً مِنْ وَسَطِكَ مِنْ إِخْوَتِكَ مِثْلِي. لهُ تَسْمَعُونَ.
\par 16 حَسَبَ كُلِّ مَا طَلبْتَ مِنَ الرَّبِّ إِلهِكَ فِي حُورِيبَ يَوْمَ الاِجْتِمَاعِ قَائِلاً: لا أَعُودُ أَسْمَعُ صَوْتَ الرَّبِّ إِلهِي وَلا أَرَى هَذِهِ النَّارَ العَظِيمَةَ أَيْضاً لِئَلا أَمُوتَ
\par 17 قَال لِيَ الرَّبُّ: قَدْ أَحْسَنُوا فِي مَا تَكَلمُوا.
\par 18 أُقِيمُ لهُمْ نَبِيّاً مِنْ وَسَطِ إِخْوَتِهِمْ مِثْلكَ وَأَجْعَلُ كَلامِي فِي فَمِهِ فَيُكَلِّمُهُمْ بِكُلِّ مَا أُوصِيهِ بِهِ.
\par 19 وَيَكُونُ أَنَّ الإِنْسَانَ الذِي لا يَسْمَعُ لِكَلامِي الذِي يَتَكَلمُ بِهِ بِاسْمِي أَنَا أُطَالِبُهُ.
\par 20 وَأَمَّا النَّبِيُّ الذِي يُطْغِي فَيَتَكَلمُ بِاسْمِي كَلاماً لمْ أُوصِهِ أَنْ يَتَكَلمَ بِهِ أَوِ الذِي يَتَكَلمُ بِاسْمِ آلِهَةٍ أُخْرَى فَيَمُوتُ ذَلِكَ النَّبِيُّ.
\par 21 وَإِنْ قُلتَ فِي قَلبِكَ: كَيْفَ نَعْرِفُ الكَلامَ الذِي لمْ يَتَكَلمْ بِهِ الرَّبُّ؟
\par 22 فَمَا تَكَلمَ بِهِ النَّبِيُّ بِاسْمِ الرَّبِّ وَلمْ يَحْدُثْ وَلمْ يَصِرْ فَهُوَ الكَلامُ الذِي لمْ يَتَكَلمْ بِهِ الرَّبُّ بَل بِطُغْيَانٍ تَكَلمَ بِهِ النَّبِيُّ فَلا تَخَفْ مِنْهُ».

\chapter{19}

\par 1 «مَتَى قَرَضَ الرَّبُّ إِلهُكَ الأُمَمَ الذِينَ الرَّبُّ إِلهُكَ يُعْطِيكَ أَرْضَهُمْ وَوَرِثْتَهُمْ وَسَكَنْتَ مُدُنَهُمْ وَبُيُوتَهُمْ
\par 2 تَفْرِزُ لِنَفْسِكَ ثَلاثَ مُدُنٍ فِي وَسَطِ أَرْضِكَ التِي يُعْطِيكَ الرَّبُّ إِلهُكَ لِتَمْتَلِكَهَا.
\par 3 تُصْلِحُ الطَّرِيقَ وَتُثَلِّثُ تُخُومَ أَرْضِكَ التِي يَقْسِمُ لكَ الرَّبُّ إِلهُكَ فَتَكُونُ مَهْرَبَاً لِكُلِّ قَاتِلٍ.
\par 4 وَهَذَا هُوَ حُكْمُ القَاتِلِ الذِي يَهْرُبُ إِلى هُنَاكَ فَيَحْيَا: مَنْ ضَرَبَ صَاحِبَهُ بِغَيْرِ عِلمٍ وَهُوَ غَيْرُ مُبْغِضٍ لهُ مُنْذُ أَمْسِ وَمَا قَبْلهُ.
\par 5 وَمَنْ ذَهَبَ مَعَ صَاحِبِهِ فِي الوَعْرِ لِيَحْتَطِبَ حَطَباً فَانْدَفَعَتْ يَدُهُ بِالفَأْسِ لِيَقْطَعَ الحَطَبَ وَأَفْلتَ الحَدِيدُ مِنَ الخَشَبِ وَأَصَابَ صَاحِبَهُ فَمَاتَ فَهُوَ يَهْرُبُ إِلى إِحْدَى تِلكَ المُدُنِ فَيَحْيَا.
\par 6 لِئَلا يَسْعَى وَلِيُّ الدَّمِ وَرَاءَ القَاتِلِ حِينَ يَحْمَى قَلبُهُ وَيُدْرِكَهُ إِذَا طَال الطَّرِيقُ وَيَقْتُلهُ وَليْسَ عَليْهِ حُكْمُ المَوْتِ لأَنَّهُ غَيْرُ مُبْغِضٍ لهُ مُنْذُ أَمْسِ وَمَا قَبْلهُ.
\par 7 لأَجْلِ ذَلِكَ أَنَا آمُرُكَ: ثَلاثَ مُدُنٍ تَفْرِزُ لِنَفْسِكَ.
\par 8 وَإِنْ وَسَّعَ الرَّبُّ إِلهُكَ تُخُومَكَ كَمَا حَلفَ لآِبَائِكَ وَأَعْطَاكَ جَمِيعَ الأَرْضِ التِي قَال إِنَّهُ يُعْطِي لآِبَائِكَ
\par 9 إِذْ حَفِظْتَ كُل هَذِهِ الوَصَايَا لِتَعْمَلهَا كَمَا أَنَا أُوصِيكَ اليَوْمَ لِتُحِبَّ الرَّبَّ إِلهَكَ وَتَسْلُكَ فِي طُرُقِهِ كُل الأَيَّامِ فَزِدْ لِنَفْسِكَ أَيْضاً ثَلاثَ مُدُنٍ عَلى هَذِهِ الثَّلاثِ
\par 10 حَتَّى لا يُسْفَكُ دَمُ بَرِيءٍ فِي وَسَطِ أَرْضِكَ التِي يُعْطِيكَ الرَّبُّ إِلهُكَ نَصِيباً فَيَكُونَ عَليْكَ دَمٌ.
\par 11 «وَلكِنْ إِذَا كَانَ إِنْسَانٌ مُبْغِضاً لِصَاحِبِهِ فَكَمَنَ لهُ وَقَامَ عَليْهِ وَضَرَبَهُ ضَرْبَةً قَاتِلةً فَمَاتَ ثُمَّ هَرَبَ إِلى إِحْدَى تِلكَ المُدُنِ
\par 12 يُرْسِلُ شُيُوخُ مَدِينَتِهِ وَيَأْخُذُونَهُ مِنْ هُنَاكَ وَيَدْفَعُونَهُ إِلى يَدِ وَلِيِّ الدَّمِ فَيَمُوتُ.
\par 13 لا تُشْفِقْ عَيْنُكَ عَليْهِ. فَتَنْزِعَ دَمَ البَرِيءِ مِنْ إِسْرَائِيل فَيَكُونَ لكَ خَيْرٌ.
\par 14 لا تَنْقُل تُخُمَ صَاحِبِكَ الذِي نَصَبَهُ الأَوَّلُونَ فِي نَصِيبِكَ الذِي تَنَالُهُ فِي الأَرْضِ التِي يُعْطِيكَ الرَّبُّ إِلهُكَ لِتَمْتَلِكَهَا.
\par 15 «لا يَقُومُ شَاهِدٌ وَاحِدٌ عَلى إِنْسَانٍ فِي ذَنْبٍ مَا أَوْ خَطِيَّةٍ مَا مِنْ جَمِيعِ الخَطَايَا التِي يُخْطِئُ بِهَا. عَلى فَمِ شَاهِدَيْنِ أَوْ عَلى فَمِ ثَلاثَةِ شُهُودٍ يَقُومُ الأَمْرُ.
\par 16 إِذَا قَامَ شَاهِدُ زُورٍ عَلى إِنْسَانٍ لِيَشْهَدَ عَليْهِ بِزَيْغٍ
\par 17 يَقِفُ الرَّجُلانِ اللذَانِ بَيْنَهُمَا الخُصُومَةُ أَمَامَ الرَّبِّ أَمَامَ الكَهَنَةِ وَالقُضَاةِ الذِينَ يَكُونُونَ فِي تِلكَ الأَيَّامِ.
\par 18 فَإِنْ فَحَصَ القُضَاةُ جَيِّداً وَإِذَا الشَّاهِدُ شَاهِدٌ كَاذِبٌ. قَدْ شَهِدَ بِالكَذِبِ عَلى أَخِيهِ
\par 19 فَافْعَلُوا بِهِ كَمَا نَوَى أَنْ يَفْعَل بِأَخِيهِ. فَتَنْزِعُونَ الشَّرَّ مِنْ وَسْطِكُمْ.
\par 20 وَيَسْمَعُ البَاقُونَ فَيَخَافُونَ وَلا يَعُودُونَ يَفْعَلُونَ مِثْل ذَلِكَ الأَمْرِ الخَبِيثِ فِي وَسَطِكَ.
\par 21 لا تُشْفِقْ عَيْنُكَ. نَفْسٌ بِنَفْسٍ. عَيْنٌ بِعَيْنٍ. سِنٌّ بِسِنٍّ. يَدٌ بِيَدٍ. رِجْلٌ بِرِجْلٍ».

\chapter{20}

\par 1 «إِذَا خَرَجْتَ لِلحَرْبِ عَلى عَدُوِّكَ وَرَأَيْتَ خَيْلاً وَمَرَاكِبَ قَوْماً أَكْثَرَ مِنْكَ فَلا تَخَفْ مِنْهُمْ لأَنَّ مَعَكَ الرَّبَّ إِلهَكَ الذِي أَصْعَدَكَ مِنْ أَرْضِ مِصْرَ.
\par 2 وَعِنْدَمَا تَقْرُبُونَ مِنَ الحَرْبِ يَتَقَدَّمُ الكَاهِنُ وَيَقُولُ لِلشَّعْبِ:
\par 3 اسْمَعْ يَا إِسْرَائِيلُ: أَنْتُمْ قَرُبْتُمُ اليَوْمَ مِنَ الحَرْبِ عَلى أَعْدَائِكُمْ. لا تَضْعُفْ قُلُوبُكُمْ. لا تَخَافُوا وَلا تَرْتَعِدُوا وَلا تَرْهَبُوا وُجُوهَهُمْ
\par 4 لأَنَّ الرَّبَّ إِلهَكُمْ سَائِرٌ مَعَكُمْ لِيُحَارِبَ عَنْكُمْ أَعْدَاءَكُمْ لِيُخَلِّصَكُمْ.
\par 5 ثُمَّ يَقُولُ العُرَفَاءُ لِلشَّعْبِ: مَنْ هُوَ الرَّجُلُ الذِي بَنَى بَيْتاً جَدِيداً وَلمْ يُدَشِّنْهُ؟ لِيَذْهَبْ وَيَرْجِعْ إِلى بَيْتِهِ لِئَلا يَمُوتَ فِي الحَرْبِ فَيُدَشِّنَهُ رَجُلٌ آخَرُ.
\par 6 وَمَنْ هُوَ الرَّجُلُ الذِي غَرَسَ كَرْماً وَلمْ يَبْتَكِرْهُ؟ لِيَذْهَبْ وَيَرْجِعْ إِلى بَيْتِهِ لِئَلا يَمُوتَ فِي الحَرْبِ فَيَبْتَكِرَهُ رَجُلٌ آخَرُ.
\par 7 وَمَنْ هُوَ الرَّجُلُ الذِي خَطَبَ امْرَأَةً وَلمْ يَأْخُذْهَا؟ لِيَذْهَبْ وَيَرْجِعْ إِلى بَيْتِهِ لِئَلا يَمُوتَ فِي الحَرْبِ فَيَأْخُذَهَا رَجُلٌ آخَرُ.
\par 8 ثُمَّ يَعُودُ العُرَفَاءُ يُخَاطِبُونَ الشَّعْبَ: مَنْ هُوَ الرَّجُلُ الخَائِفُ وَالضَّعِيفُ القَلبِ؟ لِيَذْهَبْ وَيَرْجِعْ إِلى بَيْتِهِ لِئَلا تَذُوبَ قُلُوبُ إِخْوَتِهِ مِثْل قَلبِهِ.
\par 9 وَعِنْدَ فَرَاغِ العُرَفَاءِ مِنْ مُخَاطَبَةِ الشَّعْبِ يُقِيمُونَ رُؤَسَاءَ جُنُودٍ عَلى رَأْسِ الشَّعْبِ.
\par 10 «حِينَ تَقْرُبُ مِنْ مَدِينَةٍ لِتُحَارِبَهَا اسْتَدْعِهَا لِلصُّلحِ
\par 11 فَإِنْ أَجَابَتْكَ إِلى الصُّلحِ وَفَتَحَتْ لكَ فَكُلُّ الشَّعْبِ المَوْجُودِ فِيهَا يَكُونُ لكَ لِلتَّسْخِيرِ وَيُسْتَعْبَدُ لكَ.
\par 12 وَإِنْ لمْ تُسَالِمْكَ بَل عَمِلتْ مَعَكَ حَرْباً فَحَاصِرْهَا.
\par 13 وَإِذَا دَفَعَهَا الرَّبُّ إِلهُكَ إِلى يَدِكَ فَاضْرِبْ جَمِيعَ ذُكُورِهَا بِحَدِّ السَّيْفِ.
\par 14 وَأَمَّا النِّسَاءُ وَالأَطْفَالُ وَالبَهَائِمُ وَكُلُّ مَا فِي المَدِينَةِ كُلُّ غَنِيمَتِهَا فَتَغْتَنِمُهَا لِنَفْسِكَ وَتَأْكُلُ غَنِيمَةَ أَعْدَائِكَ التِي أَعْطَاكَ الرَّبُّ إِلهُكَ.
\par 15 هَكَذَا تَفْعَلُ بِجَمِيعِ المُدُنِ البَعِيدَةِ مِنْكَ جِدّاً التِي ليْسَتْ مِنْ مُدُنِ هَؤُلاءِ الأُمَمِ هُنَا.
\par 16 وَأَمَّا مُدُنُ هَؤُلاءِ الشُّعُوبِ التِي يُعْطِيكَ الرَّبُّ إِلهُكَ نَصِيباً فَلا تَسْتَبْقِ مِنْهَا نَسَمَةً مَا
\par 17 بَل تُحَرِّمُهَا تَحْرِيماً: الحِثِّيِّينَ وَالأَمُورِيِّينَ وَالكَنْعَانِيِّينَ وَالفِرِزِّيِّينَ وَالحِوِّيِّينَ وَاليَبُوسِيِّينَ كَمَا أَمَرَكَ الرَّبُّ إِلهُكَ
\par 18 لِكَيْ لا يُعَلِّمُوكُمْ أَنْ تَعْمَلُوا حَسَبَ جَمِيعِ أَرْجَاسِهِمِ التِي عَمِلُوا لآِلِهَتِهِمْ فَتُخْطِئُوا إِلى الرَّبِّ إِلهِكُمْ.
\par 19 «إِذَا حَاصَرْتَ مَدِينَةً أَيَّاماً كَثِيرَةً مُحَارِباً إِيَّاهَا لِتَأْخُذَهَا فَلا تُتْلِفْ شَجَرَهَا بِوَضْعِ فَأْسٍ عَليْهِ. إِنَّكَ مِنْهُ تَأْكُلُ. فَلا تَقْطَعْهُ. لأَنَّهُ هَل شَجَرَةُ الحَقْلِ إِنْسَانٌ حَتَّى يَذْهَبَ قُدَّامَكَ فِي الحِصَارِ؟
\par 20 وَأَمَّا الشَّجَرُ الذِي تَعْرِفُ أَنَّهُ ليْسَ شَجَراً يُؤْكَلُ مِنْهُ فَإِيَّاهُ تُتْلِفُ وَتَقْطَعُ وَتَبْنِي حِصْناً عَلى المَدِينَةِ التِي تَعْمَلُ مَعَكَ حَرْباً حَتَّى تَسْقُطَ».

\chapter{21}

\par 1 «إِذَا وُجِدَ قَتِيلٌ فِي الأَرْضِ التِي يُعْطِيكَ الرَّبُّ إِلهُكَ لِتَمْتَلِكَهَا وَاقِعاً فِي الحَقْلِ لا يُعْلمُ مَنْ قَتَلهُ
\par 2 يَخْرُجُ شُيُوخُكَ وَقُضَاتُكَ وَيَقِيسُونَ إِلى المُدُنِ التِي حَوْل القَتِيلِ.
\par 3 فَالمَدِينَةُ القُرْبَى مِنَ القَتِيلِ يَأْخُذُ شُيُوخُ تِلكَ المَدِينَةِ عِجْلةً مِنَ البَقَرِ لمْ يُحْرَثْ عَليْهَا لمْ تَجُرَّ بِالنِّيرِ.
\par 4 وَيَنْحَدِرُ شُيُوخُ تِلكَ المَدِينَةِ بِالعِجْلةِ إِلى وَادٍ دَائِمِ السَّيَلانِ لمْ يُحْرَثْ فِيهِ وَلمْ يُزْرَعْ وَيَكْسِرُونَ عُنُقَ العِجْلةِ فِي الوَادِي.
\par 5 ثُمَّ يَتَقَدَّمُ الكَهَنَةُ بَنُو لاوِي - لأَنَّهُ إِيَّاهُمُ اخْتَارَ الرَّبُّ إِلهُكَ لِيَخْدِمُوهُ وَيُبَارِكُوا بِاسْمِ الرَّبِّ وَحَسَبَ قَوْلِهِمْ تَكُونُ كُلُّ خُصُومَةٍ وَكُلُّ ضَرْبَةٍ -
\par 6 وَيَغْسِلُ جَمِيعُ شُيُوخِ تِلكَ المَدِينَةِ القَرِيبِينَ مِنَ القَتِيلِ أَيْدِيَهُمْ عَلى العِجْلةِ المَكْسُورَةِ العُنُقُِ فِي الوَادِي
\par 7 وَيَقُولُونَ: أَيْدِينَا لمْ تَسْفِكْ هَذَا الدَّمَ وَأَعْيُنُنَا لمْ تُبْصِرْ.
\par 8 اِغْفِرْ لِشَعْبِكَ إِسْرَائِيل الذِي فَدَيْتَ يَا رَبُّ وَلا تَجْعَل دَمَ بَرِيءٍ فِي وَسَطِ شَعْبِكَ إِسْرَائِيل. فَيُغْفَرُ لهُمُ الدَّمُ.
\par 9 فَتَنْزِعُ الدَّمَ البَرِيءَ مِنْ وَسَطِكَ إِذَا عَمِلتَ الصَّالِحَ فِي عَيْنَيِ الرَّبِّ.
\par 10 «إِذَا خَرَجْتَ لِمُحَارَبَةِ أَعْدَائِكَ وَدَفَعَهُمُ الرَّبُّ إِلهُكَ إِلى يَدِكَ وَسَبَيْتَ مِنْهُمْ سَبْياً
\par 11 وَرَأَيْتَ فِي السَّبْيِ امْرَأَةً جَمِيلةَ الصُّورَةِ وَالتَصَقْتَ بِهَا وَاتَّخَذْتَهَا لكَ زَوْجَةً
\par 12 فَحِينَ تُدْخِلُهَا إِلى بَيْتِكَ تَحْلِقُ رَأْسَهَا وَتُقَلِّمُ أَظْفَارَهَا
\par 13 وَتَنْزِعُ ثِيَابَ سَبْيِهَا عَنْهَا وَتَقْعُدُ فِي بَيْتِكَ وَتَبْكِي أَبَاهَا وَأُمَّهَا شَهْراً مِنَ الزَّمَانِ ثُمَّ بَعْدَ ذَلِكَ تَدْخُلُ عَليْهَا وَتَتَزَوَّجُ بِهَا فَتَكُونُ لكَ زَوْجَةً.
\par 14 وَإِنْ لمْ تُسَرَّ بِهَا فَأَطْلِقْهَا لِنَفْسِهَا. لا تَبِعْهَا بَيْعاً بِفِضَّةٍ وَلا تَسْتَرِقَّهَا مِنْ أَجْلِ أَنَّكَ قَدْ أَذْللتَهَا.
\par 15 «إِذَا كَانَ لِرَجُلٍ امْرَأَتَانِ إِحْدَاهُمَا مَحْبُوبَةٌ وَالأُخْرَى مَكْرُوهَةٌ فَوَلدَتَا لهُ بَنِينَ المَحْبُوبَةُ وَالمَكْرُوهَةُ. فَإِنْ كَانَ الاِبْنُ البِكْرُ لِلمَكْرُوهَةِ
\par 16 فَيَوْمَ يَقْسِمُ لِبَنِيهِ مَا كَانَ لهُ لا يَحِلُّ لهُ أَنْ يُقَدِّمَ ابْنَ المَحْبُوبَةِ بِكْراً عَلى ابْنِ المَكْرُوهَةِ البِكْرِ
\par 17 بَل يَعْرِفُ ابْنَ المَكْرُوهَةِ بِكْراً لِيُعْطِيَهُ نَصِيبَ اثْنَيْنِ مِنْ كُلِّ مَا يُوجَدُ عِنْدَهُ لأَنَّهُ هُوَ أَوَّلُ قُدْرَتِهِ. لهُ حَقُّ البَكُورِيَّةِ.
\par 18 «إِذَا كَانَ لِرَجُلٍ ابْنٌ مُعَانِدٌ وَمَارِدٌ لا يَسْمَعُ لِقَوْلِ أَبِيهِ وَلا لِقَوْلِ أُمِّهِ وَيُؤَدِّبَانِهِ فَلا يَسْمَعُ لهُمَا.
\par 19 يُمْسِكُهُ أَبُوهُ وَأُمُّهُ وَيَأْتِيَانِ بِهِ إِلى شُيُوخِ مَدِينَتِهِ وَإِلى بَابِ مَكَانِهِ
\par 20 وَيَقُولانِ لِشُيُوخِ مَدِينَتِهِ: ابْنُنَا هَذَا مُعَانِدٌ وَمَارِدٌ لا يَسْمَعُ لِقَوْلِنَا وَهُوَ مُسْرِفٌ وَسِكِّيرٌ.
\par 21 فَيَرْجُمُهُ جَمِيعُ رِجَالِ مَدِينَتِهِ بِحِجَارَةٍ حَتَّى يَمُوتَ. فَتَنْزِعُ الشَّرَّ مِنْ بَيْنِكُمْ وَيَسْمَعُ كُلُّ إِسْرَائِيل وَيَخَافُونَ.
\par 22 «وَإِذَا كَانَ عَلى إِنْسَانٍ خَطِيَّةٌ حَقُّهَا المَوْتُ فَقُتِل وَعَلقْتَهُ عَلى خَشَبَةٍ
\par 23 فَلا تَبِتْ جُثَّتُهُ عَلى الخَشَبَةِ بَل تَدْفِنُهُ فِي ذَلِكَ اليَوْمِ لأَنَّ المُعَلقَ مَلعُونٌ مِنَ اللهِ. فَلا تُنَجِّسْ أَرْضَكَ التِي يُعْطِيكَ الرَّبُّ إِلهُكَ نَصِيباً».

\chapter{22}

\par 1 «لا تَنْظُرْ ثَوْرَ أَخِيكَ أَوْ شَاتَهُ شَارِداً وَتَتَغَاضَى عَنْهُ بَل تَرُدُّهُ إِلى أَخِيكَ لا مَحَالةَ.
\par 2 وَإِنْ لمْ يَكُنْ أَخُوكَ قَرِيباً مِنْكَ أَوْ لمْ تَعْرِفْهُ فَضُمَّهُ إِلى دَاخِلِ بَيْتِكَ. وَيَكُونُ عِنْدَكَ حَتَّى يَطْلُبَهُ أَخُوكَ حِينَئِذٍ تَرُدُّهُ إِليْهِ.
\par 3 وَهَكَذَا تَفْعَلُ بِحِمَارِهِ وَهَكَذَا تَفْعَلُ بِثِيَابِهِ. وَهَكَذَا تَفْعَلُ بِكُلِّ مَفْقُودٍ لأَخِيكَ يُفْقَدُ مِنْهُ وَتَجِدُهُ. لا يَحِلُّ لكَ أَنْ تَتَغَاضَى.
\par 4 لا تَنْظُرْ حِمَارَ أَخِيكَ أَوْ ثَوْرَهُ وَاقِعاً فِي الطَّرِيقِ وَتَتَغَافَلُ عَنْهُ بَل تُقِيمُهُ مَعَهُ لا مَحَالةَ.
\par 5 «لا يَكُنْ مَتَاعُ رَجُلٍ عَلى امْرَأَةٍ وَلا يَلبِسْ رَجُلٌ ثَوْبَ امْرَأَةٍ لأَنَّ كُل مَنْ يَعْمَلُ ذَلِكَ مَكْرُوهٌ لدَى الرَّبِّ إِلهِكَ.
\par 6 «إِذَا اتَّفَقَ قُدَّامَكَ عُشُّ طَائِرٍ فِي الطَّرِيقِ فِي شَجَرَةٍ مَا أَوْ عَلى الأَرْضِ فِيهِ فِرَاخٌ أَوْ بَيْضٌ وَالأُمُّ حَاضِنَةٌ الفِرَاخَ أَوِ البَيْضَ فَلا تَأْخُذِ الأُمَّ مَعَ الأَوْلادِ.
\par 7 أَطْلِقِ الأُمَّ وَخُذْ لِنَفْسِكَ الأَوْلادَ لِيَكُونَ لكَ خَيْرٌ وَتُطِيل الأَيَّامَ.
\par 8 «إِذَا بَنَيْتَ بَيْتاً جَدِيداً فَاعْمَل حَائِطاً لِسَطْحِكَ لِئَلا تَجْلِبَ دَماً عَلى بَيْتِكَ إِذَا سَقَطَ عَنْهُ سَاقِطٌ.
\par 9 «لا تَزْرَعْ حَقْلكَ صِنْفَيْنِ لِئَلا يَتَقَدَّسَ المِلءُ: الزَّرْعُ الذِي تَزْرَعُ وَمَحْصُولُ الحَقْلِ.
\par 10 لا تَحْرُثْ عَلى ثَوْرٍ وَحِمَارٍ مَعاً.
\par 11 لا تَلبَسْ ثَوْباً مُخْتَلطاً صُوفاً وَكَتَّاناً مَعاً.
\par 12 «اِعْمَل لِنَفْسِكَ جَدَائِل عَلى أَرْبَعَةِ أَطْرَافِ ثَوْبِكَ الذِي تَتَغَطَّى بِهِ.
\par 13 «إِذَا اتَّخَذَ رَجُلٌ امْرَأَةً وَحِينَ دَخَل عَليْهَا أَبْغَضَهَا
\par 14 وَنَسَبَ إِليْهَا أَسْبَابَ كَلامٍ وَأَشَاعَ عَنْهَا اسْماً رَدِيئاً وَقَال: هَذِهِ المَرْأَةُ اتَّخَذْتُهَا وَلمَّا دَنَوْتُ مِنْهَا لمْ أَجِدْ لهَا عُذْرَةً.
\par 15 يَأْخُذُ الفَتَاةَ أَبُوهَا وَأُمُّهَا وَيُخْرِجَانِ عَلامَةَ عُذْرَتِهَا إِلى شُيُوخِ المَدِينَةِ إِلى البَابِ
\par 16 وَيَقُولُ أَبُو الفَتَاةِ لِلشُّيُوخِ: أَعْطَيْتُ هَذَا الرَّجُل ابْنَتِي زَوْجَةً فَأَبْغَضَهَا.
\par 17 وَهَا هُوَ قَدْ جَعَل أَسْبَابَ كَلامٍ قَائِلاً: لمْ أَجِدْ لِبِنْتِكَ عُذْرَةً. وَهَذِهِ عَلامَةُ عُذْرَةِ ابْنَتِي. وَيَبْسُطَانِ الثَّوْبَ أَمَامَ شُيُوخِ المَدِينَةِ.
\par 18 فَيَأْخُذُ شُيُوخُ تِلكَ المَدِينَةِ الرَّجُل وَيُؤَدِّبُونَهُ
\par 19 وَيُغَرِّمُونَهُ بِمِئَةٍ مِنَ الفِضَّةِ وَيُعْطُونَهَا لأَبِي الفَتَاةِ لأَنَّهُ أَشَاعَ اسْماً رَدِيئَاً عَنْ عَذْرَاءَ مِنْ إِسْرَائِيل. فَتَكُونُ لهُ زَوْجَةً. لا يَقْدِرُ أَنْ يُطَلِّقَهَا كُل أَيَّامِهِ.
\par 20 «وَلكِنْ إِنْ كَانَ هَذَا الأَمْرُ صَحِيحاً لمْ تُوجَدْ عُذْرَةٌ لِلفَتَاةِ.
\par 21 يُخْرِجُونَ الفَتَاةَ إِلى بَابِ بَيْتِ أَبِيهَا وَيَرْجُمُهَا رِجَالُ مَدِينَتِهَا بِالحِجَارَةِ حَتَّى تَمُوتَ لأَنَّهَا عَمِلتْ قَبَاحَةً فِي إِسْرَائِيل بِزِنَاهَا فِي بَيْتِ أَبِيهَا. فَتَنْزِعُ الشَّرَّ مِنْ وَسَطِكَ.
\par 22 «إِذَا وُجِدَ رَجُلٌ مُضْطَجِعاً مَعَ امْرَأَةٍ زَوْجَةِ بَعْلٍ يُقْتَلُ الاِثْنَانِ: الرَّجُلُ المُضْطَجِعُ مَعَ المَرْأَةِ وَالمَرْأَةُ. فَتَنْزِعُ الشَّرَّ مِنْ إِسْرَائِيل.
\par 23 «إِذَا كَانَتْ فَتَاةٌ عَذْرَاءُ مَخْطُوبَةً لِرَجُلٍ فَوَجَدَهَا رَجُلٌ فِي المَدِينَةِ وَاضْطَجَعَ مَعَهَا
\par 24 فَأَخْرِجُوهُمَا كِليْهِمَا إِلى بَابِ تِلكَ المَدِينَةِ وَارْجُمُوهُمَا بِالحِجَارَةِ حَتَّى يَمُوتَا. الفَتَاةُ مِنْ أَجْلِ أَنَّهَا لمْ تَصْرُخْ فِي المَدِينَةِ وَالرَّجُلُ مِنْ أَجْلِ أَنَّهُ أَذَل امْرَأَةَ صَاحِبِهِ. فَتَنْزِعُ الشَّرَّ مِنْ وَسَطِكَ.
\par 25 وَلكِنْ إِنْ وَجَدَ الرَّجُلُ الفَتَاةَ المَخْطُوبَةَ فِي الحَقْلِ وَأَمْسَكَهَا الرَّجُلُ وَاضْطَجَعَ مَعَهَا يَمُوتُ الرَّجُلُ الذِي اضْطَجَعَ مَعَهَا وَحْدَهُ.
\par 26 وَأَمَّا الفَتَاةُ فَلا تَفْعَل بِهَا شَيْئاً. ليْسَ عَلى الفَتَاةِ خَطِيَّةٌ لِلمَوْتِ بَل كَمَا يَقُومُ رَجُلٌ عَلى صَاحِبِهِ وَيَقْتُلُهُ قَتْلاً. هَكَذَا هَذَا الأَمْرُ.
\par 27 إِنَّهُ فِي الحَقْلِ وَجَدَهَا فَصَرَخَتِ الفَتَاةُ المَخْطُوبَةُ فَلمْ يَكُنْ مَنْ يُخَلِّصُهَا.
\par 28 «إِذَا وَجَدَ رَجُلٌ فَتَاةً عَذْرَاءَ غَيْرَ مَخْطُوبَةٍ فَأَمْسَكَهَا وَاضْطَجَعَ مَعَهَا فَوُجِدَا.
\par 29 يُعْطِي الرَّجُلُ الذِي اضْطَجَعَ مَعَهَا لأَبِي الفَتَاةِ خَمْسِينَ مِنَ الفِضَّةِ وَتَكُونُ هِيَ لهُ زَوْجَةً مِنْ أَجْلِ أَنَّهُ قَدْ أَذَلهَا. لا يَقْدِرُ أَنْ يُطَلِّقَهَا كُل أَيَّامِهِ.
\par 30 «لا يَتَّخِذْ رَجُلٌ امْرَأَةَ أَبِيهِ وَلا يَكْشِفْ ذَيْل أَبِيهِ».

\chapter{23}

\par 1 «لا يَدْخُل مَخْصِيٌّ بِالرَّضِّ أَوْ مَجْبُوبٌ فِي جَمَاعَةِ الرَّبِّ.
\par 2 لا يَدْخُلِ ابْنُ زِنىً فِي جَمَاعَةِ الرَّبِّ. حَتَّى الجِيلِ العَاشِرِ لا يَدْخُل مِنْهُ أَحَدٌ فِي جَمَاعَةِ الرَّبِّ.
\par 3 لا يَدْخُل عَمُّونِيٌّ وَلا مُوآبِيٌّ فِي جَمَاعَةِ الرَّبِّ. حَتَّى الجِيلِ العَاشِرِ لا يَدْخُل مِنْهُمْ أَحَدٌ فِي جَمَاعَةِ الرَّبِّ إِلى الأَبَدِ
\par 4 مِنْ أَجْلِ أَنَّهُمْ لمْ يُلاقُوكُمْ بِالخُبْزِ وَالمَاءِ فِي الطَّرِيقِ عِنْدَ خُرُوجِكُمْ مِنْ مِصْرَ وَلأَنَّهُمُ اسْتَأْجَرُوا عَليْكَ بَلعَامَ بْنَ بَعُورَ مِنْ فَتُورِ أَرَامِ النَّهْرَيْنِ لِيَلعَنَكَ.
\par 5 وَلكِنْ لمْ يَشَإِ الرَّبُّ إِلهُكَ أَنْ يَسْمَعَ لِبَلعَامَ فَحَوَّل لأَجْلِكَ الرَّبُّ إِلهُكَ اللعْنَةَ إِلى بَرَكَةٍ لأَنَّ الرَّبَّ إِلهَكَ قَدْ أَحَبَّكَ.
\par 6 لا تَلتَمِسْ سَلامَهُمْ وَلا خَيْرَهُمْ كُل أَيَّامِكَ إِلى الأَبَدِ.
\par 7 لا تَكْرَهْ أَدُومِيّاً لأَنَّهُ أَخُوكَ. لا تَكْرَهْ مِصْرِيّاً لأَنَّكَ كُنْتَ نَزِيلاً فِي أَرْضِهِ.
\par 8 الأَوْلادُ الذِينَ يُولدُونَ لهُمْ فِي الجِيلِ الثَّالِثِ يَدْخُلُونَ مِنْهُمْ فِي جَمَاعَةِ الرَّبِّ.
\par 9 «إِذَا خَرَجْتَ فِي جَيْشٍ عَلى أَعْدَائِكَ فَاحْتَرِزْ مِنْ كُلِّ شَيْءٍ رَدِيءٍ.
\par 10 إِنْ كَانَ فِيكَ رَجُلٌ غَيْرَ طَاهِرٍ مِنْ عَارِضِ الليْلِ يَخْرُجُ إِلى خَارِجِ المَحَلةِ. لا يَدْخُل إِلى دَاخِلِ المَحَلةِ.
\par 11 وَنَحْوَ إِقْبَالِ المَسَاءِ يَغْتَسِلُ بِمَاءٍ وَعِنْدَ غُرُوبِ الشَّمْسِ يَدْخُلُ إِلى دَاخِلِ المَحَلةِ.
\par 12 وَيَكُونُ لكَ مَوْضِعٌ خَارِجَ المَحَلةِ لِتَخْرُجَ إِليْهِ خَارِجاً.
\par 13 وَيَكُونُ لكَ وَتَدٌ مَعَ عُدَّتِكَ لِتَحْفُرَ بِهِ عِنْدَمَا تَجْلِسُ خَارِجاً وَتَرْجِعُ وَتُغَطِّي بُرَازَكَ.
\par 14 لأَنَّ الرَّبَّ إِلهَكَ سَائِرٌ فِي وَسَطِ مَحَلتِكَ لِيُنْقِذَكَ وَيَدْفَعَ أَعْدَاءَكَ أَمَامَكَ. فَلتَكُنْ مَحَلتُكَ مُقَدَّسَةً لِئَلا يَرَى فِيكَ قَذَرَ شَيْءٍ فَيَرْجِعَ عَنْكَ.
\par 15 «عَبْداً أَبَقَ إِليْكَ مِنْ مَوْلاهُ لا تُسَلِّمْ إِلى مَوْلاهُ.
\par 16 عِنْدَكَ يُقِيمُ فِي وَسَطِكَ فِي المَكَانِ الذِي يَخْتَارُهُ فِي أَحَدِ أَبْوَابِكَ حَيْثُ يَطِيبُ لهُ. لا تَظْلِمْهُ.
\par 17 «لا تَكُنْ زَانِيَةٌ مِنْ بَنَاتِ إِسْرَائِيل وَلا يَكُنْ مَأْبُونٌ مِنْ بَنِي إِسْرَائِيل.
\par 18 لا تُدْخِل أُجْرَةَ زَانِيَةٍ وَلا ثَمَنَ كَلبٍ إِلى بَيْتِ الرَّبِّ إِلهِكَ عَنْ نَذْرٍ مَا لأَنَّهُمَا كِليْهِمَا رِجْسٌ لدَى الرَّبِّ إِلهِكَ.
\par 19 «لا تُقْرِضْ أَخَاكَ بِرِباً رِبَا فِضَّةٍ أَوْ رِبَا طَعَامٍ أَوْ رِبَا شَيْءٍ مَا مِمَّا يُقْرَضُ بِرِباً
\par 20 لِلأَجْنَبِيِّ تُقْرِضُ بِرِباً وَلكِنْ لأَخِيكَ لا تُقْرِضْ بِرِباً لِيُبَارِكَكَ الرَّبُّ إِلهُكَ فِي كُلِّ مَا تَمْتَدُّ إِليْهِ يَدُكَ فِي الأَرْضِ التِي أَنْتَ دَاخِلٌ إِليْهَا لِتَمْتَلِكَهَا.
\par 21 «إِذَا نَذَرْتَ نَذْراً لِلرَّبِّ إِلهِكَ فَلا تُؤَخِّرْ وَفَاءَهُ لأَنَّ الرَّبَّ إِلهَكَ يَطْلُبُهُ مِنْكَ فَتَكُونُ عَليْكَ خَطِيَّةٌ.
\par 22 وَلكِنْ إِذَا امْتَنَعْتَ أَنْ تَنْذُرَ لا تَكُونُ عَليْكَ خَطِيَّةٌ.
\par 23 مَا خَرَجَ مِنْ شَفَتَيْكَ احْفَظْ وَاعْمَل كَمَا نَذَرْتَ لِلرَّبِّ إِلهِكَ تَبَرُّعاً كَمَا تَكَلمَ فَمُكَ.
\par 24 «إِذَا دَخَلتَ كَرْمَ صَاحِبِكَ فَكُل عِنَباً حَسَبَ شَهْوَةِ نَفْسِكَ شَبْعَتَكَ. وَلكِنْ فِي وِعَائِكَ لا تَجْعَل.
\par 25 إِذَا دَخَلتَ زَرْعَ صَاحِبِكَ فَاقْطِفْ سَنَابِل بِيَدِكَ وَلكِنْ مِنْجَلاً لا تَرْفَعْ عَلى زَرْعِ صَاحِبِكَ».

\chapter{24}

\par 1 «إِذَا أَخَذَ رَجُلٌ امْرَأَةً وَتَزَوَّجَ بِهَا فَإِنْ لمْ تَجِدْ نِعْمَةً فِي عَيْنَيْهِ لأَنَّهُ وَجَدَ فِيهَا عَيْبَ شَيْءٍ وَكَتَبَ لهَا كِتَابَ طَلاقٍ وَدَفَعَهُ إِلى يَدِهَا وَأَطْلقَهَا مِنْ بَيْتِهِ
\par 2 وَمَتَى خَرَجَتْ مِنْ بَيْتِهِ ذَهَبَتْ وَصَارَتْ لِرَجُلٍ آخَرَ
\par 3 فَإِنْ أَبْغَضَهَا الرَّجُلُ الأَخِيرُ وَكَتَبَ لهَا كِتَابَ طَلاقٍ وَدَفَعَهُ إِلى يَدِهَا وَأَطْلقَهَا مِنْ بَيْتِهِ أَوْ إِذَا مَاتَ الرَّجُلُ الأَخِيرُ الذِي اتَّخَذَهَا لهُ زَوْجَةً
\par 4 لا يَقْدِرُ زَوْجُهَا الأَوَّلُ الذِي طَلقَهَا أَنْ يَعُودَ يَأْخُذُهَا لِتَصِيرَ لهُ زَوْجَةً بَعْدَ أَنْ تَنَجَّسَتْ. لأَنَّ ذَلِكَ رِجْسٌ لدَى الرَّبِّ. فَلا تَجْلِبْ خَطِيَّةً عَلى الأَرْضِ التِي يُعْطِيكَ الرَّبُّ إِلهُكَ نَصِيباً.
\par 5 «إِذَا اتَّخَذَ رَجُلٌ امْرَأَةً جَدِيدَةً فَلا يَخْرُجْ فِي الجُنْدِ وَلا يُحْمَل عَليْهِ أَمْرٌ مَا. حُرّاً يَكُونُ فِي بَيْتِهِ سَنَةً وَاحِدَةً وَيَسُرُّ امْرَأَتَهُ التِي أَخَذَهَا.
\par 6 «لا يَسْتَرْهِنْ أَحَدٌ رَحىً أَوْ مِرْدَاتَهَا لأَنَّهُ إِنَّمَا يَسْتَرْهِنُ حَيَاةً.
\par 7 «إِذَا وُجِدَ رَجُلٌ قَدْ سَرَقَ نَفْساً مِنْ إِخْوَتِهِ بَنِي إِسْرَائِيل وَاسْتَرَقَّهُ وَبَاعَهُ يَمُوتُ ذَلِكَ السَّارِقُ فَتَنْزِعُ الشَّرَّ مِنْ وَسَطِكَ.
\par 8 «اِحْرِصْ فِي ضَرْبَةِ البَرَصِ لِتَحْفَظَ جِدّاً وَتَعْمَل حَسَبَ كُلِّ مَا يُعَلِّمُكَ الكَهَنَةُ اللاوِيُّونَ. كَمَا أَمَرْتُهُمْ تَحْرِصُونَ أَنْ تَعْمَلُوا.
\par 9 اُذْكُرْ مَا صَنَعَ الرَّبُّ إِلهُكَ بِمَرْيَمَ فِي الطَّرِيقِ عِنْدَ خُرُوجِكُمْ مِنْ مِصْرَ.
\par 10 «إِذَا أَقْرَضْتَ صَاحِبَكَ قَرْضاً مَا فَلا تَدْخُل بَيْتَهُ لِتَرْتَهِنَ رَهْناً مِنْهُ.
\par 11 فِي الخَارِجِ تَقِفُ وَالرَّجُلُ الذِي تُقْرِضُهُ يُخْرِجُ إِليْكَ الرَّهْنَ إِلى الخَارِجِ.
\par 12 وَإِنْ كَانَ رَجُلاً فَقِيراً فَلا تَنَمْ فِي رَهْنِهِ.
\par 13 رُدَّ إِليْهِ الرَّهْنَ عِنْدَ غُرُوبِ الشَّمْسِ لِيَنَامَ فِي ثَوْبِهِ وَيُبَارِكَكَ فَيَكُونَ لكَ بِرٌّ لدَى الرَّبِّ إِلهِكَ.
\par 14 «لا تَظْلِمْ أَجِيراً مِسْكِيناً وَفَقِيراً مِنْ إِخْوَتِكَ أَوْ مِنَ الغُرَبَاءِ الذِينَ فِي أَرْضِكَ فِي أَبْوَابِكَ.
\par 15 فِي يَوْمِهِ تُعْطِيهِ أُجْرَتَهُ وَلا تَغْرُبْ عَليْهَا الشَّمْسُ لأَنَّهُ فَقِيرٌ وَإِليْهَا حَامِلٌ نَفْسَهُ لِئَلا يَصْرُخَ عَليْكَ إِلى الرَّبِّ فَتَكُونَ عَليْكَ خَطِيَّةٌ.
\par 16 «لا يُقْتَلُ الآبَاءُ عَنِ الأَوْلادِ وَلا يُقْتَلُ الأَوْلادُ عَنِ الآبَاءِ. كُلُّ إِنْسَانٍ بِخَطِيَّتِهِ يُقْتَلُ.
\par 17 «لا تُعَوِّجْ حُكْمَ الغَرِيبِ وَاليَتِيمِ وَلا تَسْتَرْهِنْ ثَوْبَ الأَرْمَلةِ.
\par 18 وَاذْكُرْ أَنَّكَ كُنْتَ عَبْداً فِي مِصْرَ فَفَدَاكَ الرَّبُّ إِلهُكَ مِنْ هُنَاكَ. لِذَلِكَ أَنَا أُوصِيكَ أَنْ تَعْمَل هَذَا الأَمْرَ.
\par 19 «إِذَا حَصَدْتَ حَصِيدَكَ فِي حَقْلِكَ وَنَسِيتَ حُزْمَةً فِي الحَقْلِ فَلا تَرْجِعْ لِتَأْخُذَهَا. لِلغَرِيبِ وَاليَتِيمِ وَالأَرْمَلةِ تَكُونُ لِيُبَارِكَكَ الرَّبُّ إِلهُكَ فِي كُلِّ عَمَلِ يَدَيْكَ.
\par 20 وَإِذَا خَبَطْتَ زَيْتُونَكَ فَلا تُرَاجِعِ الأَغْصَانَ وَرَاءَكَ. لِلغَرِيبِ وَاليَتِيمِ وَالأَرْمَلةِ يَكُونُ.
\par 21 إِذَا قَطَفْتَ كَرْمَكَ فَلا تُعَلِّلهُ وَرَاءَكَ. لِلغَرِيبِ وَاليَتِيمِ وَالأَرْمَلةِ يَكُونُ.
\par 22 وَاذْكُرْ أَنَّكَ كُنْتَ عَبْداً فِي أَرْضِ مِصْرَ. لِذَلِكَ أَنَا أُوصِيكَ أَنْ تَعْمَل هَذَا الأَمْرَ».

\chapter{25}

\par 1 «إِذَا كَانَتْ خُصُومَةٌ بَيْنَ أُنَاسٍ وَتَقَدَّمُوا إِلى القَضَاءِ لِيَقْضِيَ القُضَاةُ بَيْنَهُمْ فَليُبَرِّرُوا البَارَّ وَيَحْكُمُوا عَلى المُذْنِبِ.
\par 2 فَإِنْ كَانَ المُذْنِبُ مُسْتَوْجِبَ الضَّرْبِ يَطْرَحُهُ القَاضِي وَيَجْلِدُونَهُ أَمَامَهُ عَلى قَدَرِ ذَنْبِهِ بِالعَدَدِ.
\par 3 أَرْبَعِينَ يَجْلِدُهُ. لا يَزِدْ لِئَلا إِذَا زَادَ فِي جَلدِهِ عَلى هَذِهِ ضَرَبَاتٍ كَثِيرَةً يُحْتَقَرَ أَخُوكَ فِي عَيْنَيْكَ.
\par 4 لا تَكُمَّ الثَّوْرَ فِي دِرَاسِهِ.
\par 5 «إِذَا سَكَنَ إِخْوَةٌ مَعاً وَمَاتَ وَاحِدٌ مِنْهُمْ وَليْسَ لهُ ابْنٌ فَلا تَصِرِ امْرَأَةُ المَيِّتِ إِلى خَارِجٍ لِرَجُلٍ أَجْنَبِيٍّ. أَخُو زَوْجِهَا يَدْخُلُ عَليْهَا وَيَتَّخِذُهَا لِنَفْسِهِ زَوْجَةً وَيَقُومُ لهَا بِوَاجِبِ أَخِي الزَّوْجِ.
\par 6 وَالبِكْرُ الذِي تَلِدُهُ يَقُومُ بِاسْمِ أَخِيهِ المَيِّتِ لِئَلا يُمْحَى اسْمُهُ مِنْ إِسْرَائِيل.
\par 7 «وَإِنْ لمْ يَرْضَ الرَّجُلُ أَنْ يَأْخُذَ امْرَأَةَ أَخِيهِ تَصْعَدُ امْرَأَةُ أَخِيهِ إِلى البَابِ إِلى الشُّيُوخِ وَتَقُولُ: قَدْ أَبَى أَخُو زَوْجِي أَنْ يُقِيمَ لأَخِيهِ اسْماً فِي إِسْرَائِيل. لمْ يَشَأْ أَنْ يَقُومَ لِي بِوَاجِبِ أَخِي الزَّوْجِ.
\par 8 فَيَدْعُوهُ شُيُوخُ مَدِينَتِهِ وَيَتَكَلمُونَ مَعَهُ. فَإِنْ أَصَرَّ وَقَال: لا أَرْضَى أَنْ أَتَّخِذَهَا
\par 9 تَتَقَدَّمُ امْرَأَةُ أَخِيهِ إِليْهِ أَمَامَ أَعْيُنِ الشُّيُوخِ وَتَخْلعُ نَعْلهُ مِنْ رِجْلِهِ وَتَبْصُقُ فِي وَجْهِهِ وَتَقُولُ: هَكَذَا يُفْعَلُ بِالرَّجُلِ الذِي لا يَبْنِي بَيْتَ أَخِيهِ.
\par 10 فَيُدْعَى اسْمُهُ فِي إِسْرَائِيل «بَيْتَ مَخْلُوعِ النَّعْلِ».
\par 11 «إِذَا تَخَاصَمَ رَجُلانِ رَجُلٌ وَأَخُوهُ وَتَقَدَّمَتِ امْرَأَةُ أَحَدِهِمَا لِتُخَلِّصَ رَجُلهَا مِنْ يَدِ ضَارِبِهِ وَمَدَّتْ يَدَهَا وَأَمْسَكَتْ بِعَوْرَتِهِ
\par 12 فَاقْطَعْ يَدَهَا وَلا تُشْفِقْ عَيْنُكَ.
\par 13 «لا يَكُنْ لكَ فِي كِيسِكَ أَوْزَانٌ مُخْتَلِفَةٌ كَبِيرَةٌ وَصَغِيرَةٌ.
\par 14 لا يَكُنْ لكَ فِي بَيْتِكَ مَكَايِيلُ مُخْتَلِفَةٌ كَبِيرَةٌ وَصَغِيرَةٌ.
\par 15 وَزْنٌ صَحِيحٌ وَحَقٌّ يَكُونُ لكَ وَمِكْيَالٌ صَحِيحٌ وَحَقٌّ يَكُونُ لكَ لِتَطُول أَيَّامُكَ عَلى الأَرْضِ التِي يُعْطِيكَ الرَّبُّ إِلهُكَ.
\par 16 لأَنَّ كُل مَنْ عَمِل ذَلِكَ كُل مَنْ عَمِل غِشّاً مَكْرُوهٌ لدَى الرَّبِّ إِلهِكَ.
\par 17 «اُذْكُرْ مَا فَعَلهُ بِكَ عَمَالِيقُ فِي الطَّرِيقِ عِنْدَ خُرُوجِكَ مِنْ مِصْرَ.
\par 18 كَيْفَ لاقَاكَ فِي الطَّرِيقِ وَقَطَعَ مِنْ مُؤَخَّرِكَ كُل المُسْتَضْعِفِينَ وَرَاءَكَ وَأَنْتَ كَلِيلٌ وَمُتْعَبٌ وَلمْ يَخَفِ اللهَ.
\par 19 فَمَتَى أَرَاحَكَ الرَّبُّ إِلهُكَ مِنْ جَمِيعِ أَعْدَائِكَ حَوْلكَ فِي الأَرْضِ التِي يُعْطِيكَ الرَّبُّ إِلهُكَ نَصِيباً لِتَمْتَلِكَهَا تَمْحُو ذِكْرَ عَمَالِيقَ مِنْ تَحْتِ السَّمَاءِ. لا تَنْسَ».

\chapter{26}

\par 1 «وَمَتَى أَتَيْتَ إِلى الأَرْضِ التِي يُعْطِيكَ الرَّبُّ إِلهُكَ نَصِيباً وَامْتَلكْتَهَا وَسَكَنْتَ فِيهَا
\par 2 فَتَأْخُذُ مِنْ أَوَّلِ كُلِّ ثَمَرِ الأَرْضِ الذِي تُحَصِّلُ مِنْ أَرْضِكَ التِي يُعْطِيكَ الرَّبُّ إِلهُكَ وَتَضَعُهُ فِي سَلةٍ وَتَذْهَبُ إِلى المَكَانِ الذِي يَخْتَارُهُ الرَّبُّ إِلهُكَ لِيُحِل اسْمَهُ فِيهِ.
\par 3 وَتَأْتِي إِلى الكَاهِنِ الذِي يَكُونُ فِي تِلكَ الأَيَّامِ وَتَقُولُ لهُ: أَعْتَرِفُ اليَوْمَ لِلرَّبِّ إِلهِكَ أَنِّي قَدْ دَخَلتُ الأَرْضَ التِي حَلفَ الرَّبُّ لآِبَائِنَا أَنْ يُعْطِيَنَا إِيَّاهَا.
\par 4 فَيَأْخُذُ الكَاهِنُ السَّلةَ مِنْ يَدِكَ وَيَضَعُهَا أَمَامَ مَذْبَحِ الرَّبِّ إِلهِكَ.
\par 5 ثُمَّ تَقُولُ أَمَامَ الرَّبِّ إِلهِكَ: أَرَامِيّاً تَائِهاً كَانَ أَبِي فَانْحَدَرَ إِلى مِصْرَ وَتَغَرَّبَ هُنَاكَ فِي نَفَرٍ قَلِيلٍ فَصَارَ هُنَاكَ أُمَّةً كَبِيرَةً وَعَظِيمَةً وَكَثِيرَةً.
\par 6 فَأَسَاءَ إِليْنَا المِصْرِيُّونَ وَثَقَّلُوا عَليْنَا وَجَعَلُوا عَليْنَا عُبُودِيَّةً قَاسِيَةً.
\par 7 فَلمَّا صَرَخْنَا إِلى الرَّبِّ إِلهِ آبَائِنَا سَمِعَ الرَّبُّ صَوْتَنَا وَرَأَى مَشَقَّتَنَا وَتَعَبَنَا وَضِيقَنَا.
\par 8 فَأَخْرَجَنَا مِنْ مِصْرَ بِيَدٍ شَدِيدَةٍ وَذِرَاعٍ رَفِيعَةٍ وَمَخَاوِفَ عَظِيمَةٍ وَآيَاتٍ وَعَجَائِبَ
\par 9 وَأَدْخَلنَا هَذَا المَكَانَ وَأَعْطَانَا هَذِهِ الأَرْضَ أَرْضاً تَفِيضُ لبَناً وَعَسَلاً.
\par 10 فَالآنَ هَئَنَذَا قَدْ أَتَيْتُ بِأَوَّلِ ثَمَرِ الأَرْضِ التِي أَعْطَيْتَنِي يَا رَبُّ. ثُمَّ تَضَعُهُ أَمَامَ الرَّبِّ إِلهِكَ وَتَسْجُدُ أَمَامَ الرَّبِّ إِلهِكَ.
\par 11 وَتَفْرَحُ بِجَمِيعِ الخَيْرِ الذِي أَعْطَاهُ الرَّبُّ إِلهُكَ لكَ وَلِبَيْتِكَ أَنْتَ وَاللاوِيُّ وَالغَرِيبُ الذِي فِي وَسْطِكَ.
\par 12 «مَتَى فَرَغْتَ مِنْ تَعْشِيرِ كُلِّ عُشُورِ مَحْصُولِكَ فِي السَّنَةِ الثَّالِثَةِ سَنَةِ العُشُورِ وَأَعْطَيْتَ اللاوِيَّ وَالغَرِيبَ وَاليَتِيمَ وَالأَرْمَلةَ فَأَكَلُوا فِي أَبْوَابِكَ وَشَبِعُوا
\par 13 تَقُولُ أَمَامَ الرَّبِّ إِلهِكَ: قَدْ نَزَعْتُ المُقَدَّسَ مِنَ البَيْتِ وَأَيْضاً أَعْطَيْتُهُ لِلاوِيِّ وَالغَرِيبِ وَاليَتِيمِ وَالأَرْمَلةِ حَسَبَ كُلِّ وَصِيَّتِكَ التِي أَوْصَيْتَنِي بِهَا. لمْ أَتَجَاوَزْ وَصَايَاكَ وَلا نَسِيتُهاَ.
\par 14 لمْ آكُل مِنْهُ فِي حُزْنِي وَلا أَخَذْتُ مِنْهُ فِي نَجَاسَةٍ وَلا أَعْطَيْتُ مِنْهُ لأَجْلِ مَيِّتٍ بَل سَمِعْتُ لِصَوْتِ الرَّبِّ إِلهِي وَعَمِلتُ حَسَبَ كُلِّ مَا أَوْصَيْتَنِي.
\par 15 اِطَّلِعْ مِنْ مَسْكَنِ قُدْسِكَ مِنَ السَّمَاءِ وَبَارِكْ شَعْبَكَ إِسْرَائِيل وَالأَرْضَ التِي أَعْطَيْتَنَا كَمَا حَلفْتَ لآِبَائِنَا أَرْضاً تَفِيضُ لبَناً وَعَسَلاً.
\par 16 «هَذَا اليَوْمَ قَدْ أَمَرَكَ الرَّبُّ إِلهُكَ أَنْ تَعْمَل بِهَذِهِ الفَرَائِضِ وَالأَحْكَامِ. فَاحْفَظْ وَاعْمَل بِهَا مِنْ كُلِّ قَلبِكَ وَمِنْ كُلِّ نَفْسِكَ.
\par 17 قَدْ وَاعَدْتَ الرَّبَّ اليَوْمَ أَنْ يَكُونَ لكَ إِلهاً وَأَنْ تَسْلُكَ فِي طُرُقِهِ وَتَحْفَظَ فَرَائِضَهُ وَوَصَايَاهُ وَأَحْكَامَهُ وَتَسْمَعَ لِصَوْتِهِ.
\par 18 وَوَاعَدَكَ الرَّبُّ اليَوْمَ أَنْ تَكُونَ لهُ شَعْباً خَاصّاً كَمَا قَال لكَ وَتَحْفَظَ جَمِيعَ وَصَايَاهُ
\par 19 وَأَنْ يَجْعَلكَ مُسْتَعْلِياً عَلى جَمِيعِ القَبَائِلِ التِي عَمِلهَا فِي الثَّنَاءِ وَالاِسْمِ وَالبَهَاءِ وَأَنْ تَكُونَ شَعْباً مُقَدَّساً لِلرَّبِّ إِلهِكَ كَمَا قَال».

\chapter{27}

\par 1 وَأَوْصَى مُوسَى وَشُيُوخُ إِسْرَائِيل الشَّعْبَ: «احْفَظُوا جَمِيعَ الوَصَايَا التِي أَنَا أُوصِيكُمْ بِهَا اليَوْمَ.
\par 2 فَيَوْمَ تَعْبُرُونَ الأُرْدُنَّ إِلى الأَرْضِ التِي يُعْطِيكَ الرَّبُّ إِلهُكَ تُقِيمُ لِنَفْسِكَ حِجَارَةً كَبِيرَةً وَتَشِيدُهَا بِالشِّيدِ
\par 3 وَتَكْتُبُ عَليْهَا جَمِيعَ كَلِمَاتِ هَذَا النَّامُوسِ حِينَ تَعْبُرُ لِتَدْخُل الأَرْضَ التِي يُعْطِيكَ الرَّبُّ إِلهُكَ أَرْضاً تَفِيضُ لبَناً وَعَسَلاً كَمَا قَال لكَ الرَّبُّ إِلهُ آبَائِكَ.
\par 4 حِينَ تَعْبُرُونَ الأُرْدُنَّ تُقِيمُونَ هَذِهِ الحِجَارَةَ التِي أَنَا أُوصِيكُمْ بِهَا اليَوْمَ فِي جَبَلِ عِيبَال وَتُكَلِّسُهَا بِالكِلسِ.
\par 5 وَتَبْنِي هُنَاكَ مَذْبَحاً لِلرَّبِّ إِلهِكَ مَذْبَحاً مِنْ حِجَارَةٍ لا تَرْفَعْ عَليْهَا حَدِيداً.
\par 6 مِنْ حِجَارَةٍ صَحِيحَةٍ تَبْنِي مَذْبَحَ الرَّبِّ إِلهِكَ وَتُصْعِدُ عَليْهِ مُحْرَقَاتٍ لِلرَّبِّ إِلهِكَ.
\par 7 وَتَذْبَحُ ذَبَائِحَ سَلامَةٍ وَتَأْكُلُ هُنَاكَ وَتَفْرَحُ أَمَامَ الرَّبِّ إِلهِكَ.
\par 8 وَتَكْتُبُ عَلى الحِجَارَةِ جَمِيعَ كَلِمَاتِ هَذَا النَّامُوسِ نَقْشاً جَيِّداً».
\par 9 ثُمَّ قَال مُوسَى وَالكَهَنَةُ اللاوِيُّونَ لِجَمِيعِ إِسْرَائِيل: «اُنْصُتْ وَاسْمَعْ يَا إِسْرَائِيلُ. اليَوْمَ صِرْتَ شَعْباً لِلرَّبِّ إِلهِكَ.
\par 10 فَاسْمَعْ لِصَوْتِ الرَّبِّ إِلهِكَ وَاعْمَل بِوَصَايَاهُ وَفَرَائِضِهِ التِي أَنَا أُوصِيكَ بِهَا اليَوْمَ».
\par 11 وَأَوْصَى مُوسَى الشَّعْبَ فِي ذَلِكَ اليَوْمِ:
\par 12 «هَؤُلاءِ يَقِفُونَ عَلى جَبَلِ جِرِزِّيمَ لِيُبَارِكُوا الشَّعْبَ حِينَ تَعْبُرُونَ الأُرْدُنَّ. شَمْعُونُ وَلاوِي وَيَهُوذَا وَيَسَّاكَرُ وَيُوسُفُ وَبِنْيَامِينُ.
\par 13 وَهَؤُلاءِ يَقِفُونَ عَلى جَبَلِ عِيبَال لِلعْنَةِ. رَأُوبَيْنُ وَجَادُ وَأَشِيرُ وَزَبُولُونُ وَدَانُ وَنَفْتَالِي.
\par 14 فَيَقُولُ اللاوِيُّونَ لِجَمِيعِ قَوْمِ إِسْرَائِيل بِصَوْتٍ عَالٍ:
\par 15 مَلعُونٌ الإِنْسَانُ الذِي يَصْنَعُ تِمْثَالاً مَنْحُوتاً أَوْ مَسْبُوكاً رِجْساً لدَى الرَّبِّ عَمَل يَدَيْ نَحَّاتٍ وَيَضَعُهُ فِي الخَفَاءِ. وَيُجِيبُ جَمِيعُ الشَّعْبِ وَيَقُولُونَ: آمِينَ.
\par 16 مَلعُونٌ مَنْ يَسْتَخِفُّ بِأَبِيهِ أَوْ أُمِّهِ. وَيَقُولُ جَمِيعُ الشَّعْبِ: آمِينَ.
\par 17 مَلعُونٌ مَنْ يَنْقُلُ تُخْمَ صَاحِبِهِ. وَيَقُولُ جَمِيعُ الشَّعْبِ: آمِينَ.
\par 18 مَلعُونٌ مَنْ يُضِلُّ الأَعْمَى عَنِ الطَّرِيقِ. وَيَقُولُ جَمِيعُ الشَّعْبِ: آمِينَ.
\par 19 مَلعُونٌ مَنْ يُعَوِّجُ حَقَّ الغَرِيبِ وَاليَتِيمِ وَالأَرْمَلةِ. وَيَقُولُ جَمِيعُ الشَّعْبِ: آمِينَ.
\par 20 مَلعُونٌ مَنْ يَضْطَجِعُ مَعَ امْرَأَةِ أَبِيهِ لأَنَّهُ يَكْشِفُ ذَيْل أَبِيهِ. وَيَقُولُ جَمِيعُ الشَّعْبِ: آمِينَ.
\par 21 مَلعُونٌ مَنْ يَضْطَجِعُ مَعَ بَهِيمَةٍ مَا. وَيَقُولُ جَمِيعُ الشَّعْبِ: آمِينَ.
\par 22 مَلعُونٌ مَنْ يَضْطَجِعُ مَعَ أُخْتِهِ ابْنَةِ أَبِيهِ أَوْ ابْنَةِ أُمِّهِ. وَيَقُولُ جَمِيعُ الشَّعْبِ: آمِينَ.
\par 23 مَلعُونٌ مَنْ يَضْطَجِعُ مَعَ حَمَاتِهِ. وَيَقُولُ جَمِيعُ الشَّعْبِ: آمِينَ.
\par 24 مَلعُونٌ مَنْ يَقْتُلُ قَرِيبَهُ فِي الخَفَاءِ. وَيَقُولُ جَمِيعُ الشَّعْبِ: آمِينَ.
\par 25 مَلعُونٌ مَنْ يَأْخُذُ رَشْوَةً لِيَقْتُل دَماً بَرِيئاً. وَيَقُولُ جَمِيعُ الشَّعْبِ: آمِينَ.
\par 26 مَلعُونٌ مَنْ لا يُقِيمُ كَلِمَاتِ هَذَا النَّامُوسِ لِيَعْمَل بِهَا. وَيَقُولُ جَمِيعُ الشَّعْبِ: آمِينَ».

\chapter{28}

\par 1 «وَإِنْ سَمِعْتَ سَمْعاً لِصَوْتِ الرَّبِّ إِلهِكَ لِتَحْرِصَ أَنْ تَعْمَل بِجَمِيعِ وَصَايَاهُ التِي أَنَا أُوصِيكَ بِهَا اليَوْمَ يَجْعَلُكَ الرَّبُّ إِلهُكَ مُسْتَعْلِياً عَلى جَمِيعِ قَبَائِلِ الأَرْضِ
\par 2 وَتَأْتِي عَليْكَ جَمِيعُ هَذِهِ البَرَكَاتِ وَتُدْرِكُكَ إِذَا سَمِعْتَ لِصَوْتِ الرَّبِّ إِلهِكَ.
\par 3 مُبَارَكاً تَكُونُ فِي المَدِينَةِ وَمُبَارَكاً تَكُونُ فِي الحَقْلِ.
\par 4 وَمُبَارَكَةً تَكُونُ ثَمَرَةُ بَطْنِكَ وَثَمَرَةُ أَرْضِكَ وَثَمَرَةُ بَهَائِمِكَ نِتَاجُ بَقَرِكَ وَإِنَاثُ غَنَمِكَ.
\par 5 مُبَارَكَةً تَكُونُ سَلتُكَ وَمِعْجَنُكَ.
\par 6 مُبَارَكاً تَكُونُ فِي دُخُولِكَ وَمُبَارَكاً تَكُونُ فِي خُرُوجِكَ.
\par 7 يَجْعَلُ الرَّبُّ أَعْدَاءَكَ القَائِمِينَ عَليْكَ مُنْهَزِمِينَ أَمَامَكَ. فِي طَرِيقٍ وَاحِدَةٍ يَخْرُجُونَ عَليْكَ وَفِي سَبْعِ طُرُقٍ يَهْرُبُونَ أَمَامَكَ.
\par 8 يَأْمُرُ لكَ الرَّبُّ بِالبَرَكَةِ فِي خَزَائِنِكَ وَفِي كُلِّ مَا تَمْتَدُّ إِليْهِ يَدُكَ وَيُبَارِكُكَ فِي الأَرْضِ التِي يُعْطِيكَ الرَّبُّ إِلهُكَ.
\par 9 يُقِيمُكَ الرَّبُّ لِنَفْسِهِ شَعْباً مُقَدَّساً كَمَا حَلفَ لكَ إِذَا حَفِظْتَ وَصَايَا الرَّبِّ إِلهِكَ وَسَلكْتَ فِي طُرُقِهِ.
\par 10 فَيَرَى جَمِيعُ شُعُوبِ الأَرْضِ أَنَّ اسْمَ الرَّبِّ قَدْ سُمِّيَ عَليْكَ وَيَخَافُونَ مِنْكَ.
\par 11 وَيَزِيدُكَ الرَّبُّ خَيْراً فِي ثَمَرَةِ بَطْنِكَ وَثَمَرَةِ بَهَائِمِكَ وَثَمَرَةِ أَرْضِكَ عَلى الأَرْضِ التِي حَلفَ الرَّبُّ لآِبَائِكَ أَنْ يُعْطِيَكَ.
\par 12 يَفْتَحُ لكَ الرَّبُّ كَنْزَهُ الصَّالِحَ السَّمَاءَ لِيُعْطِيَ مَطَرَ أَرْضِكَ فِي حِينِهِ وَليُبَارِكَ كُل عَمَلِ يَدِكَ فَتُقْرِضُ أُمَماً كَثِيرَةً وَأَنْتَ لا تَقْتَرِضُ.
\par 13 وَيَجْعَلُكَ الرَّبُّ رَأْساً لا ذَنَباً وَتَكُونُ فِي الاِرْتِفَاعِ فَقَطْ وَلا تَكُونُ فِي الاِنْحِطَاطِ إِذَا سَمِعْتَ لِوَصَايَا الرَّبِّ إِلهِكَ التِي أَنَا أُوصِيكَ بِهَا اليَوْمَ لِتَحْفَظَ وَتَعْمَل
\par 14 وَلا تَزِيغَ عَنْ جَمِيعِ الكَلِمَاتِ التِي أَنَا أُوصِيكَ بِهَا اليَوْمَ يَمِيناً أَوْ شِمَالاً لِتَذْهَبَ وَرَاءَ آلِهَةٍ أُخْرَى لِتَعْبُدَهَا.
\par 15 «وَلكِنْ إِنْ لمْ تَسْمَعْ لِصَوْتِ الرَّبِّ إِلهِكَ لِتَحْرِصَ أَنْ تَعْمَل بِجَمِيعِ وَصَايَاهُ وَفَرَائِضِهِ التِي أَنَا أُوصِيكَ بِهَا اليَوْمَ تَأْتِي عَليْكَ جَمِيعُ هَذِهِ اللعْنَاتِ وَتُدْرِكُكَ.
\par 16 مَلعُوناً تَكُونُ فِي المَدِينَةِ وَمَلعُوناً تَكُونُ فِي الحَقْلِ.
\par 17 مَلعُونَةً تَكُونُ سَلتُكَ وَمِعْجَنُكَ.
\par 18 مَلعُونَةً تَكُونُ ثَمَرَةُ بَطْنِكَ وَثَمَرَةُ أَرْضِكَ نِتَاجُ بَقَرِكَ وَإِنَاثُ غَنَمِكَ.
\par 19 مَلعُوناً تَكُونُ فِي دُخُولِكَ وَمَلعُوناً تَكُونُ فِي خُرُوجِكَ.
\par 20 يُرْسِلُ الرَّبُّ عَليْكَ اللعْنَ وَالاِضْطِرَابَ وَالزَّجْرَ فِي كُلِّ مَا تَمْتَدُّ إِليْهِ يَدُكَ لِتَعْمَلهُ حَتَّى تَهْلِكَ وَتَفْنَى سَرِيعاً مِنْ أَجْلِ سُوءِ أَفْعَالِكَ إِذْ تَرَكْتَنِي.
\par 21 يُلصِقُ بِكَ الرَّبُّ الوَبَأَ حَتَّى يُبِيدَكَ عَنِ الأَرْضِ التِي أَنْتَ دَاخِلٌ إِليْهَا لِتَمْتَلِكَهَا.
\par 22 يَضْرِبُكَ الرَّبُّ بِالسِّلِّ وَالحُمَّى وَالبُرَدَاءِ وَالاِلتِهَابِ وَالجَفَافِ وَاللفْحِ وَالذُّبُولِ فَتَتَّبِعُكَ حَتَّى تُفْنِيَكَ.
\par 23 وَتَكُونُ سَمَاؤُكَ التِي فَوْقَ رَأْسِكَ نُحَاساً وَالأَرْضُ التِي تَحْتَكَ حَدِيداً.
\par 24 وَيَجْعَلُ الرَّبُّ مَطَرَ أَرْضِكَ غُبَاراً وَتُرَاباً يُنَزِّلُ عَليْكَ مِنَ السَّمَاءِ حَتَّى تَهْلِكَ.
\par 25 يَجْعَلُكَ الرَّبُّ مُنْهَزِماً أَمَامَ أَعْدَائِكَ. فِي طَرِيقٍ وَاحِدَةٍ تَخْرُجُ عَليْهِمْ وَفِي سَبْعِ طُرُقٍ تَهْرُبُ أَمَامَهُمْ وَتَكُونُ قَلِقاً فِي جَمِيعِ مَمَالِكِ الأَرْضِ.
\par 26 وَتَكُونُ جُثَّتُكَ طَعَاماً لِجَمِيعِ طُيُورِ السَّمَاءِ وَوُحُوشِ الأَرْضِ وَليْسَ مَنْ يُزْعِجُهَا.
\par 27 يَضْرِبُكَ الرَّبُّ بِقُرْحَةِ مِصْرَ وَبِالبَوَاسِيرِ وَالجَرَبِ وَالحِكَّةِ حَتَّى لا تَسْتَطِيعَ الشِّفَاءَ.
\par 28 يَضْرِبُكَ الرَّبُّ بِجُنُونٍ وَعَمىً وَحَيْرَةِ قَلبٍ
\par 29 فَتَتَلمَّسُ فِي الظُّهْرِ كَمَا يَتَلمَّسُ الأَعْمَى فِي الظَّلامِ وَلا تَنْجَحُ فِي طُرُقِكَ بَل لا تَكُونُ إِلا مَظْلُوماً مَغْصُوباً كُل الأَيَّامِ وَليْسَ مُخَلِّصٌ.
\par 30 تَخْطُبُ امْرَأَةً وَرَجُلٌ آخَرُ يَضْطَجِعُ مَعَهَا. تَبْنِي بَيْتاً وَلا تَسْكُنُ فِيهِ. تَغْرِسُ كَرْماً وَلا تَسْتَغِلُّهُ.
\par 31 يُذْبَحُ ثَوْرُكَ أَمَامَ عَيْنَيْكَ وَلا تَأْكُلُ مِنْهُ. يُغْتَصَبُ حِمَارُكَ مِنْ أَمَامِ وَجْهِكَ وَلا يَرْجِعُ إِليْكَ. تُدْفَعُ غَنَمُكَ إِلى أَعْدَائِكَ وَليْسَ لكَ مُخَلِّصٌ.
\par 32 يُسَلمُ بَنُوكَ وَبَنَاتُكَ لِشَعْبٍ آخَرَ وَعَيْنَاكَ تَنْظُرَانِ إِليْهِمْ طُول النَّهَارِ فَتَكِلانِ وَليْسَ فِي يَدِكَ طَائِلةٌ.
\par 33 ثَمَرُ أَرْضِكَ وَكُلُّ تَعَبِكَ يَأْكُلُهُ شَعْبٌ لا تَعْرِفُهُ فَلا تَكُونُ إِلا مَظْلُوماً وَمَسْحُوقاً كُل الأَيَّامِ.
\par 34 وَتَكُونُ مَجْنُوناً مِنْ مَنْظَرِ عَيْنَيْكَ الذِي تَنْظُرُ.
\par 35 يَضْرِبُكَ الرَّبُّ بِقُرْحٍ خَبِيثٍ عَلى الرُّكْبَتَيْنِ وَعَلى السَّاقَيْنِ حَتَّى لا تَسْتَطِيعَ الشِّفَاءَ مِنْ أَسْفَلِ قَدَمِكَ إِلى قِمَّةِ رَأْسِكَ.
\par 36 يَذْهَبُ بِكَ الرَّبُّ وَبِمَلِكِكَ الذِي تُقِيمُهُ عَليْكَ إِلى أُمَّةٍ لمْ تَعْرِفْهَا أَنْتَ وَلا آبَاؤُكَ وَتَعْبُدُ هُنَاكَ آلِهَةً أُخْرَى مِنْ خَشَبٍ وَحَجَرٍ
\par 37 وَتَكُونُ دَهَشاً وَمَثَلاً وَهُزْأَةً فِي جَمِيعِ الشُّعُوبِ الذِينَ يَسُوقُكَ الرَّبُّ إِليْهِمْ.
\par 38 بِذَاراً كَثِيراً تُخْرِجُ إِلى الحَقْلِ وَقَلِيلاً تَجْمَعُ لأَنَّ الجَرَادَ يَأْكُلُهُ.
\par 39 كُرُوماً تَغْرِسُ وَتَشْتَغِلُ وَخَمْراً لا تَشْرَبُ وَلا تَجْنِي لأَنَّ الدُّودَ يَأْكُلُهَا.
\par 40 يَكُونُ لكَ زَيْتُونٌ فِي جَمِيعِ تُخُومِكَ وَبِزَيْتٍ لا تَدَّهِنُ لأَنَّ زَيْتُونَكَ يَنْتَثِرُ.
\par 41 بَنِينَ وَبَنَاتٍ تَلِدُ وَلا يَكُونُونَ لكَ لأَنَّهُمْ إِلى السَّبْيِ يَذْهَبُونَ.
\par 42 جَمِيعُ أَشْجَارِكَ وَأَثْمَارِ أَرْضِكَ يَتَوَلاهُ الصَّرْصَرُ.
\par 43 اَلغَرِيبُ الذِي فِي وَسَطِكَ يَسْتَعْلِي عَليْكَ مُتَصَاعِداً وَأَنْتَ تَنْحَطُّ مُتَنَازِلاً.
\par 44 هُوَ يُقْرِضُكَ وَأَنْتَ لا تُقْرِضُهُ. هُوَ يَكُونُ رَأْساً وَأَنْتَ تَكُونُ ذَنَباً.
\par 45 وَتَأْتِي عَليْكَ جَمِيعُ هَذِهِ اللعَنَاتِ وَتَتَّبِعُكَ وَتُدْرِكُكَ حَتَّى تَهْلِكَ لأَنَّكَ لمْ تَسْمَعْ لِصَوْتِ الرَّبِّ إِلهِكَ لِتَحْفَظَ وَصَايَاهُ وَفَرَائِضَهُ التِي أَوْصَاكَ بِهَا.
\par 46 فَتَكُونُ فِيكَ آيَةً وَأُعْجُوبَةً وَفِي نَسْلِكَ إِلى الأَبَدِ.
\par 47 مِنْ أَجْلِ أَنَّكَ لمْ تَعْبُدِ الرَّبَّ إِلهَكَ بِفَرَحٍ وَبِطِيبَةِ قَلبٍ لِكَثْرَةِ كُلِّ شَيْءٍ.
\par 48 تُسْتَعْبَدُ لأَعْدَائِكَ الذِينَ يُرْسِلُهُمُ الرَّبُّ عَليْكَ فِي جُوعٍ وَعَطَشٍ وَعُرْيٍ وَعَوَزِ كُلِّ شَيْءٍ. فَيَجْعَلُ نِيرَ حَدِيدٍ عَلى عُنُقِكَ حَتَّى يُهْلِكَكَ.
\par 49 يَجْلِبُ الرَّبُّ عَليْكَ أُمَّةً مِنْ بَعِيدٍ مِنْ أَقْصَاءِ الأَرْضِ كَمَا يَطِيرُ النَّسْرُ أُمَّةً لا تَفْهَمُ لِسَانَهَا
\par 50 أُمَّةً جَافِيَةَ الوَجْهِ لا تَهَابُ الشَّيْخَ وَلا تَحِنُّ إِلى الوَلدِ
\par 51 فَتَأْكُلُ ثَمَرَةَ بَهَائِمِكَ وَثَمَرَةَ أَرْضِكَ حَتَّى تَهْلِكَ وَلا تُبْقِي لكَ قَمْحاً وَلا خَمْراً وَلا زَيْتاً وَلا نِتَاجَ بَقَرِكَ وَلا إِنَاثَ غَنَمِكَ حَتَّى تُفْنِيَكَ.
\par 52 وَتُحَاصِرُكَ فِي جَمِيعِ أَبْوَابِكَ حَتَّى تَهْبِطَ أَسْوَارُكَ الشَّامِخَةُ الحَصِينَةُ التِي أَنْتَ تَثِقُ بِهَا فِي كُلِّ أَرْضِكَ. تُحَاصِرُكَ فِي جَمِيعِ أَبْوَابِكَ فِي كُلِّ أَرْضِكَ التِي يُعْطِيكَ الرَّبُّ إِلهُكَ.
\par 53 فَتَأْكُلُ ثَمَرَةَ بَطْنِكَ لحْمَ بَنِيكَ وَبَنَاتِكَ الذِينَ أَعْطَاكَ الرَّبُّ إِلهُكَ فِي الحِصَارِ وَالضِّيقَةِ التِي يُضَايِقُكَ بِهَا عَدُوُّكَ.
\par 54 الرَّجُلُ المُتَنَعِّمُ فِيكَ وَالمُتَرَفِّهُ جِدّاً تَبْخَلُ عَيْنُهُ عَلى أَخِيهِ وَامْرَأَةِ حِضْنِهِ وَبَقِيَّةِ أَوْلادِهِ الذِينَ يُبْقِيهِمْ
\par 55 بِأَنْ يُعْطِيَ أَحَدَهُمْ مِنْ لحْمِ بَنِيهِ الذِي يَأْكُلُهُ لأَنَّهُ لمْ يُبْقَ لهُ شَيْءٌ فِي الحِصَارِ وَالضِّيقَةِ التِي يُضَايِقُكَ بِهَا عَدُوُّكَ فِي جَمِيعِ أَبْوَابِكَ.
\par 56 وَالمَرْأَةُ المُتَنَعِّمَةُ فِيكَ وَالمُتَرَفِّهَةُ التِي لمْ تُجَرِّبْ أَنْ تَضَعَ أَسْفَل قَدَمِهَا عَلى الأَرْضِ لِلتَّنَعُّمِ وَالتَّرَفُّهِ تَبْخَلُ عَيْنُهَا عَلى رَجُلِ حِضْنِهَا وَعَلى ابْنِهَا وَابْنَتِهَا
\par 57 بِمَشِيمَتِهَا الخَارِجَةِ مِنْ بَيْنِ رِجْليْهَا وَبِأَوْلادِهَا الذِينَ تَلِدُهُمْ لأَنَّهَا تَأْكُلُهُمْ سِرّاً فِي عَوَزِ كُلِّ شَيْءٍ فِي الحِصَارِ وَالضِّيقَةِ التِي يُضَايِقُكَ بِهَا عَدُوُّكَ فِي أَبْوَابِكَ.
\par 58 إِنْ لمْ تَحْرِصْ لِتَعْمَل بِجَمِيعِ كَلِمَاتِ هَذَا النَّامُوسِ المَكْتُوبَةِ فِي هَذَا السِّفْرِ لِتَهَابَ هَذَا الاِسْمَ الجَلِيل المَرْهُوبَ الرَّبَّ إِلهَكَ
\par 59 يَجْعَلُ الرَّبُّ ضَرَبَاتِكَ وَضَرَبَاتِ نَسْلِكَ عَجِيبَةً. ضَرَبَاتٍ عَظِيمَةً رَاسِخَةً وَأَمْرَاضاً رَدِيئَةً ثَابِتَةً.
\par 60 وَيَرُدُّ عَليْكَ جَمِيعَ أَدْوَاءِ مِصْرَ التِي فَزِعْتَ مِنْهَا فَتَلتَصِقُ بِكَ.
\par 61 أَيْضاً كُلُّ مَرَضٍ وَكُلُّ ضَرْبَةٍ لمْ تُكْتَبْ فِي سِفْرِ النَّامُوسِ هَذَا يُسَلِّطُهُ الرَّبُّ عَليْكَ حَتَّى تَهْلكَ.
\par 62 فَتَبْقُونَ نَفَراً قَلِيلاً عِوَضَ مَا كُنْتُمْ كَنُجُومِ السَّمَاءِ فِي الكَثْرَةِ لأَنَّكَ لمْ تَسْمَعْ لِصَوْتِ الرَّبِّ إِلهِكَ.
\par 63 وَكَمَا فَرِحَ الرَّبُّ لكُمْ لِيُحْسِنَ إِليْكُمْ وَيُكَثِّرَكُمْ كَذَلِكَ يَفْرَحُ الرَّبُّ لكُمْ لِيُفْنِيَكُمْ وَيُهْلِكَكُمْ فَتُسْتَأْصَلُونَ مِنَ الأَرْضِ التِي أَنْتَ دَاخِلٌ إِليْهَا لِتَمْتَلِكَهَا.
\par 64 وَيُبَدِّدُكَ الرَّبُّ فِي جَمِيعِ الشُّعُوبِ مِنْ أَقْصَاءِ الأَرْضِ إِلى أَقْصَائِهَا وَتَعْبُدُ هُنَاكَ آلِهَةً أُخْرَى لمْ تَعْرِفْهَا أَنْتَ وَلا آبَاؤُكَ مِنْ خَشَبٍ وَحَجَرٍ.
\par 65 وَفِي تِلكَ الأُمَمِ لا تَطْمَئِنُّ وَلا يَكُونُ قَِرَارٌ لِقَدَمِكَ بَل يُعْطِيكَ الرَّبُّ هُنَاكَ قَلباً مُرْتَجِفاً وَكَلال العَيْنَيْنِ وَذُبُول النَّفْسِ.
\par 66 وَتَكُونُ حَيَاتُكَ مُعَلقَةً قُدَّامَكَ وَتَرْتَعِبُ ليْلاً وَنَهَاراً وَلا تَأْمَنُ عَلى حَيَاتِكَ.
\par 67 فِي الصَّبَاحِ تَقُولُ: يَا ليْتَهُ المَسَاءُ! وَفِي المَسَاءِ تَقُولُ: يَا ليْتَهُ الصَّبَاحُ! مِنِ ارْتِعَابِ قَلبِكَ الذِي تَرْتَعِبُ وَمِنْ مَنْظَرِ عَيْنَيْكَ الذِي تَنْظُرُ.
\par 68 وَيَرُدُّكَ الرَّبُّ إِلى مِصْرَ فِي سُفُنٍ فِي الطَّرِيقِ التِي قُلتُ لكَ لا تَعُدْ تَرَاهَا فَتُبَاعُونَ هُنَاكَ لأَعْدَائِكَ عَبِيداً وَإِمَاءً وَليْسَ مَنْ يَشْتَرِي». (29:1)هَذِهِ كَلِمَاتُ العَهْدِ الذِي أَمَرَ الرَّبُّ مُوسَى أَنْ يَقْطَعَهُ مَعَ بَنِي إِسْرَائِيل فِي أَرْضِ مُوآبَ فَضْلاً عَنِ العَهْدِ الذِي قَطَعَهُ مَعَهُمْ فِي حُورِيبَ.

\chapter{29}

\par 1 هذه هي كلمات العهد الذي امر الرب موسى ان يقطعه مع بني اسرائيل في ارض موآب فضلا عن العهد الذي قطعه معهم في حوريب
\par 2 وَدَعَا مُوسَى جَمِيعَ إِسْرَائِيل وَقَال لهُمْ: «أَنْتُمْ شَاهَدْتُمْ مَا فَعَل الرَّبُّ أَمَامَ أَعْيُنِكُمْ فِي أَرْضِ مِصْرَ بِفِرْعَوْنَ وَبِجَمِيعِ عَبِيدِهِ وَبِكُلِّ أَرْضِهِ
\par 3 التَّجَارِبَ العَظِيمَةَ التِي أَبْصَرَتْهَا عَيْنَاكَ وَتِلكَ الآيَاتِ وَالعَجَائِبَ العَظِيمَةَ.
\par 4 وَلكِنْ لمْ يُعْطِكُمُ الرَّبُّ قَلباً لِتَفْهَمُوا وَأَعْيُناً لِتُبْصِرُوا وَآذَاناً لِتَسْمَعُوا إِلى هَذَا اليَوْمِ.
\par 5 فَقَدْ سِرْتُ بِكُمْ أَرْبَعِينَ سَنَةً فِي البَرِّيَّةِ لمْ تَبْل ثِيَابُكُمْ عَليْكُمْ وَنَعْلُكَ لمْ تَبْل عَلى رِجْلِكَ.
\par 6 لمْ تَأْكُلُوا خُبْزاً وَلمْ تَشْرَبُوا خَمْراً وَلا مُسْكِراً لِتَعْلمُوا أَنِّي أَنَا الرَّبُّ إِلهُكُمْ.
\par 7 وَلمَّا جِئْتُمْ إِلى هَذَا المَكَانِ خَرَجَ سِيحُونُ مَلِكُ حَشْبُونَ وَعُوجُ مَلِكُ بَاشَانَ لِلِقَائِنَا لِلحَرْبِ فَكَسَّرْنَاهُمَا
\par 8 وَأَخَذْنَا أَرْضَهُمَا وَأَعْطَيْنَاهَا نَصِيباً لِرَأُوبَيْنَ وَجَادَ وَنِصْفِ سِبْطِ مَنَسَّى.
\par 9 فَاحْفَظُوا كَلِمَاتِ هَذَا العَهْدِ وَاعْمَلُوا بِهَا لِتَفْلِحُوا فِي كُلِّ مَا تَفْعَلُونَ.
\par 10 «أَنْتُمْ وَاقِفُونَ اليَوْمَ جَمِيعُكُمْ أَمَامَ الرَّبِّ إِلهِكُمْ رُؤَسَاؤُكُمْ أَسْبَاطُكُمْ شُيُوخُكُمْ وَعُرَفَاؤُكُمْ وَكُلُّ رِجَالِ إِسْرَائِيل
\par 11 وَأَطْفَالُكُمْ وَنِسَاؤُكُمْ وَغَرِيبُكُمُ الذِي فِي وَسَطِ مَحَلتِكُمْ مِمَّنْ يَحْتَطِبُ حَطَبَكُمْ إِلى مَنْ يَسْتَقِي مَاءَكُمْ
\par 12 لِتَدْخُل فِي عَهْدِ الرَّبِّ إِلهِكَ وَقَسَمِهِ الذِي يَقْطَعُهُ الرَّبُّ إِلهُكَ مَعَكَ اليَوْمَ
\par 13 لِيُقِيمَكَ اليَوْمَ لِنَفْسِهِ شَعْباً وَهُوَ يَكُونُ لكَ إِلهاً كَمَا قَال لكَ وَكَمَا حَلفَ لآِبَائِكَ إِبْرَاهِيمَ وَإِسْحَاقَ وَيَعْقُوبَ.
\par 14 وَليْسَ مَعَكُمْ وَحْدَكُمْ أَقْطَعُ أَنَا هَذَا العَهْدَ وَهَذَا القَسَمَ
\par 15 بَل مَعَ الذِي هُوَ هُنَا مَعَنَا وَاقِفاً اليَوْمَ أَمَامَ الرَّبِّ إِلهِنَا وَمَعَ الذِي ليْسَ هُنَا مَعَنَا اليَوْمَ.
\par 16 (لأَنَّكُمْ قَدْ عَرَفْتُمْ كَيْفَ أَقَمْنَا فِي أَرْضِ مِصْرَ؛ وَكَيْفَ اجْتَزْنَا فِي وَسَطِ الأُمَمِ الذِينَ مَرَرْتُمْ بِهِمْ؛
\par 17 وَرَأَيْتُمْ أَرْجَاسَهُمْ وَأَصْنَامَهُمُ التِي عِنْدَهُمْ مِنْ خَشَبٍ وَحَجَرٍ وَفِضَّةٍ وَذَهَبٍ)
\par 18 لِئَلا يَكُونَ فِيكُمْ رَجُلٌ أَوِ امْرَأَةٌ أَوْ عَشِيرَةٌ أَوْ سِبْطٌ قَلبُهُ اليَوْمَ مُنْصَرِفٌ عَنِ الرَّبِّ إِلهِنَا لِكَيْ يَذْهَبَ لِيَعْبُدَ آلِهَةَ تِلكَ الأُمَمِ. لِئَلا يَكُونَ فِيكُمْ أَصْلٌ يُثْمِرُ عَلقَماً وَأَفْسَنْتِيناً.
\par 19 فَيَكُونُ مَتَى سَمِعَ كَلامَ هَذِهِ اللعْنَةِ يُبَارِكُ نَفْسَهُ فِي قَلبِهِ وَيَقُولُ: يَكُونُ لِي سَلامٌ وَإِنْ سِرْتُ بِتَصَلُّبِ قَلبِي - فَيَفْنَى الرَّيَّانُ مَعَ العَطْشَانِ.
\par 20 مِثْلُ هَذا لا يَشَاءُ الرَّبُّ أَنْ يَرْفُقَ بِهِ بَل يُدَخِّنُ حِينَئِذٍ غَضَبُ الرَّبِّ وَغَيْرَتُهُ عَلى ذَلِكَ الرَّجُلِ فَتَحِلُّ عَليْهِ كُلُّ اللعَنَاتِ المَكْتُوبَةِ فِي هَذَا الكِتَابِ وَيَمْحُو الرَّبُّ اسْمَهُ مِنْ تَحْتِ السَّمَاءِ.
\par 21 وَيُفْرِزُهُ الرَّبُّ لِلشَّرِّ مِنْ جَمِيعِ أَسْبَاطِ إِسْرَائِيل حَسَبَ جَمِيعِ لعَنَاتِ العَهْدِ المَكْتُوبَةِ فِي كِتَابِ الشَّرِيعَةِ هَذَا.
\par 22 فَيَقُولُ الجِيلُ الأَخِيرُ بَنُوكُمُ الذِينَ يَقُومُونَ بَعْدَكُمْ وَالأَجْنَبِيُّ الذِي يَأْتِي مِنْ أَرْضٍ بَعِيدَةٍ حِينَ يَرُونَ ضَرَبَاتِ تِلكَ الأَرْضِ وَأَمْرَاضَهَا التِي يُمْرِضُهَا بِهَا الرَّبُّ -
\par 23 كِبْرِيتٌ وَمِلحٌ كُلُّ أَرْضِهَا حَرِيقٌ لا تُزْرَعُ وَلا تُنْبِتُ وَلا يَطْلُعُ فِيهَا عُشْبٌ مَا كَانْقِلابِ سَدُومَ وَعَمُورَةَ وَأَدْمَةَ وَصَبُويِيمَ التِي قَلبَهَا الرَّبُّ بِغَضَبِهِ وَسَخَطِهِ.
\par 24 وَيَقُولُ جَمِيعُ الأُمَمِ: لِمَاذَا فَعَل الرَّبُّ هَكَذَا بِهَذِهِ الأَرْضِ؟ لِمَاذَا حُمُوُّ هَذَا الغَضَبِ العَظِيمِ؟
\par 25 فَيَقُولُونَ: لأَنَّهُمْ تَرَكُوا عَهْدَ الرَّبِّ إِلهِ آبَائِهِمِ الذِي قَطَعَهُ مَعَهُمْ حِينَ أَخْرَجَهُمْ مِنْ أَرْضِ مِصْرَ
\par 26 وَذَهَبُوا وَعَبَدُوا آلِهَةً أُخْرَى وَسَجَدُوا لهَا. آلِهَةً لمْ يَعْرِفُوهَا وَلا قُسِمَتْ لهُمْ.
\par 27 فَاشْتَعَل غَضَبُ الرَّبِّ عَلى تِلكَ الأَرْضِ حَتَّى جَلبَ عَليْهَا كُل اللعَنَاتِ المَكْتُوبَةِ فِي هَذَا السِّفْرِ.
\par 28 وَاسْتَأْصَلهُمُ الرَّبُّ مِنْ أَرْضِهِمْ بِغَضَبٍ وَسَخَطٍ وَغَيْظٍ عَظِيمٍ وَأَلقَاهُمْ إِلى أَرْضٍ أُخْرَى كَمَا فِي هَذَا اليَوْمِ.
\par 29 السَّرَائِرُ لِلرَّبِّ إِلهِنَا وَالمُعْلنَاتُ لنَا وَلِبَنِينَا إِلى الأَبَدِ لِنَعْمَل بِجَمِيعِ كَلِمَاتِ هَذِهِ الشَّرِيعَةِ».

\chapter{30}

\par 1 «وَمَتَى أَتَتْ عَليْكَ كُلُّ هَذِهِ الأُمُورِ البَرَكَةُ وَاللعْنَةُ اللتَانِ جَعَلتُهُمَا قُدَّامَكَ فَإِنْ رَدَدْتَ فِي قَلبِكَ بَيْنَ جَمِيعِ الأُمَمِ الذِينَ طَرَدَكَ الرَّبُّ إِلهُكَ إِليْهِمْ
\par 2 وَرَجَعْتَ إِلى الرَّبِّ إِلهِكَ وَسَمِعْتَ لِصَوْتِهِ حَسَبَ كُلِّ مَا أَنَا أُوصِيكَ بِهِ اليَوْمَ أَنْتَ وَبَنُوكَ بِكُلِّ قَلبِكَ وَبِكُلِّ نَفْسِكَ
\par 3 يَرُدُّ الرَّبُّ إِلهُكَ سَبْيَكَ وَيَرْحَمُكَ وَيَعُودُ فَيَجْمَعُكَ مِنْ جَمِيعِ الشُّعُوبِ الذِينَ بَدَّدَكَ إِليْهِمِ الرَّبُّ إِلهُكَ.
\par 4 إِنْ يَكُنْ قَدْ بَدَّدَكَ إِلى أَقْصَاءِ السَّمَاوَاتِ فَمِنْ هُنَاكَ يَجْمَعُكَ الرَّبُّ إِلهُكَ وَمِنْ هُنَاكَ يَأْخُذُكَ.
\par 5 وَيَأْتِي بِكَ الرَّبُّ إِلهُكَ إِلى الأَرْضِ التِي امْتَلكَهَا آبَاؤُكَ فَتَمْتَلِكُهَا وَيُحْسِنُ إِليْكَ وَيُكَثِّرُكَ أَكْثَرَ مِنْ آبَائِكَ.
\par 6 وَيَخْتِنُ الرَّبُّ إِلهُكَ قَلبَكَ وَقَلبَ نَسْلِكَ لِكَيْ تُحِبَّ الرَّبَّ إِلهَكَ مِنْ كُلِّ قَلبِكَ وَمِنْ كُلِّ نَفْسِكَ لِتَحْيَا.
\par 7 وَيَجْعَلُ الرَّبُّ إِلهُكَ كُل هَذِهِ اللعَنَاتِ عَلى أَعْدَائِكَ وَعَلى مُبْغِضِيكَ الذِينَ طَرَدُوكَ.
\par 8 وَأَمَّا أَنْتَ فَتَعُودُ تَسْمَعُ لِصَوْتِ الرَّبِّ وَتَعْمَلُ بِجَمِيعِ وَصَايَاهُ التِي أَنَا أُوصِيكَ بِهَا اليَوْمَ
\par 9 فَيَزِيدُكَ الرَّبُّ إِلهُكَ خَيْراً فِي كُلِّ عَمَلِ يَدِكَ فِي ثَمَرَةِ بَطْنِكَ وَثَمَرَةِ بَهَائِمِكَ وَثَمَرَةِ أَرْضِكَ. لأَنَّ الرَّبَّ يَرْجِعُ لِيَفْرَحَ لكَ بِالخَيْرِ كَمَا فَرِحَ لآِبَائِكَ
\par 10 إِذَا سَمِعْتَ لِصَوْتِ الرَّبِّ إِلهِكَ لِتَحْفَظَ وَصَايَاهُ وَفَرَائِضَهُ المَكْتُوبَةَ فِي سِفْرِ الشَّرِيعَةِ هَذَا. إِذَا رَجَعْتَ إِلى الرَّبِّ إِلهِكَ بِكُلِّ قَلبِكَ وَبِكُلِّ نَفْسِكَ.
\par 11 «إِنَّ هَذِهِ الوَصِيَّةَ التِي أُوصِيكَ بِهَا اليَوْمَ ليْسَتْ عَسِرَةً عَليْكَ وَلا بَعِيدَةً مِنْكَ.
\par 12 ليْسَتْ هِيَ فِي السَّمَاءِ حَتَّى تَقُول: مَنْ يَصْعَدُ لأَجْلِنَا إِلى السَّمَاءِ وَيَأْخُذُهَا لنَا وَيُسْمِعُنَا إِيَّاهَا لِنَعْمَل بِهَا؟
\par 13 وَلا هِيَ فِي عَبْرِ البَحْرِ حَتَّى تَقُول: مَنْ يَعْبُرُ لأَجْلِنَا البَحْرَ وَيَأْخُذُهَا لنَا وَيُسْمِعُنَا إِيَّاهَا لِنَعْمَل بِهَا؟
\par 14 بَلِ الكَلِمَةُ قَرِيبَةٌ مِنْكَ جِدّاً فِي فَمِكَ وَفِي قَلبِكَ لِتَعْمَل بِهَا.
\par 15 «اُنْظُرْ. قَدْ جَعَلتُ اليَوْمَ قُدَّامَكَ الحَيَاةَ وَالخَيْرَ وَالمَوْتَ وَالشَّرَّ
\par 16 بِمَا أَنِّي أَوْصَيْتُكَ اليَوْمَ أَنْ تُحِبَّ الرَّبَّ إِلهَكَ وَتَسْلُكَ فِي طُرُقِهِ وَتَحْفَظَ وَصَايَاهُ وَفَرَائِضَهُ وَأَحْكَامَهُ لِتَحْيَا وَتَنْمُوَ وَيُبَارِكَكَ الرَّبُّ إِلهُكَ فِي الأَرْضِ التِي أَنْتَ دَاخِلٌ إِليْهَا لِتَمْتَلِكَهَا.
\par 17 فَإِنِ انْصَرَفَ قَلبُكَ وَلمْ تَسْمَعْ بَل غَوَيْتَ وَسَجَدْتَ لآِلِهَةٍ أُخْرَى وَعَبَدْتَهَا
\par 18 فَإِنِّي أُنْبِئُكُمُ اليَوْمَ أَنَّكُمْ لا مَحَالةَ تَهْلِكُونَ. لا تُطِيلُ الأَيَّامَ عَلى الأَرْضِ التِي أَنْتَ عَابِرٌ الأُرْدُنَّ لِتَدْخُلهَا وَتَمْتَلِكَهَا.
\par 19 أُشْهِدُ عَليْكُمُ اليَوْمَ السَّمَاءَ وَالأَرْضَ. قَدْ جَعَلتُ قُدَّامَكَ الحَيَاةَ وَالمَوْتَ. البَرَكَةَ وَاللعْنَةَ. فَاخْتَرِ الحَيَاةَ لِتَحْيَا أَنْتَ وَنَسْلُكَ
\par 20 إِذْ تُحِبُّ الرَّبَّ إِلهَكَ وَتَسْمَعُ لِصَوْتِهِ وَتَلتَصِقُ بِهِ لأَنَّهُ هُوَ حَيَاتُكَ وَالذِي يُطِيلُ أَيَّامَكَ لِتَسْكُنَ عَلى الأَرْضِ التِي حَلفَ الرَّبُّ لآِبَائِكَ إِبْرَاهِيمَ وَإِسْحَاقَ وَيَعْقُوبَ أَنْ يُعْطِيَهُمْ إِيَّاهَا».

\chapter{31}

\par 1 فَذَهَبَ مُوسَى وَكَلمَ بِهَذِهِ الكَلِمَاتِ جَمِيعَ إِسْرَائِيل
\par 2 وَقَال لهُمْ: «أَنَا اليَوْمَ ابْنُ مِئَةٍ وَعِشْرِينَ سَنَةً. لا أَسْتَطِيعُ الخُرُوجَ وَالدُّخُول بَعْدُ وَالرَّبُّ قَدْ قَال لِي: لا تَعْبُرُ هَذَا الأُرْدُنَّ.
\par 3 الرَّبُّ إِلهُكَ هُوَ عَابِرٌ قُدَّامَكَ. هُوَ يُبِيدُ هَؤُلاءِ الأُمَمَ مِنْ قُدَّامِكَ فَتَرِثُهُمْ. يَشُوعُ عَابِرٌ قُدَّامَكَ كَمَا قَال الرَّبُّ.
\par 4 وَيَفْعَلُ الرَّبُّ بِهِمْ كَمَا فَعَل بِسِيحُونَ وَعُوجَ مَلِكَيِ الأَمُورِيِّينَ اللذَيْنِ أَهْلكَهُمَا وَبِأَرْضِهِمَا.
\par 5 فَمَتَى دَفَعَهُمُ الرَّبُّ أَمَامَكُمْ تَفْعَلُونَ بِهِمْ حَسَبَ كُلِّ الوَصَايَا التِي أَوْصَيْتُكُمْ بِهَا.
\par 6 تَشَدَّدُوا وَتَشَجَّعُوا. لا تَخَافُوا وَلا تَرْهَبُوا وُجُوهَهُمْ لأَنَّ الرَّبَّ إِلهَكَ سَائِرٌ مَعَكَ. لا يُهْمِلُكَ وَلا يَتْرُكُكَ».
\par 7 فَدَعَا مُوسَى يَشُوعَ وَقَال لهُ أَمَامَ أَعْيُنِ جَمِيعِ إِسْرَائِيل: «تَشَدَّدْ وَتَشَجَّعْ لأَنَّكَ أَنْتَ تَدْخُلُ مَعَ هَذَا الشَّعْبِ الأَرْضَ التِي أَقْسَمَ الرَّبُّ لآِبَائِهِمْ أَنْ يُعْطِيَهُمْ إِيَّاهَا. وَأَنْتَ تَقْسِمُهَا لهُمْ.
\par 8 وَالرَّبُّ سَائِرٌ أَمَامَكَ. هُوَ يَكُونُ مَعَكَ. لا يُهْمِلُكَ وَلا يَتْرُكُكَ. لا تَخَفْ وَلا تَرْتَعِبْ».
\par 9 وَكَتَبَ مُوسَى هَذِهِ التَّوْرَاةَ وَسَلمَهَا لِلكَهَنَةِ بَنِي لاوِي حَامِلِي تَابُوتِ عَهْدِ الرَّبِّ وَلِجَمِيعِ شُيُوخِ إِسْرَائِيل.
\par 10 وَأَمَرَهُمْ مُوسَى: «فِي نِهَايَةِ السَّبْعِ السِّنِينَ فِي مِيعَادِ سَنَةِ الإِبْرَاءِ فِي عِيدِ المَظَالِّ
\par 11 حِينَمَا يَجِيءُ جَمِيعُ إِسْرَائِيل لِيَظْهَرُوا أَمَامَ الرَّبِّ إِلهِكَ فِي المَكَانِ الذِي يَخْتَارُهُ تَقْرَأُ هَذِهِ التَّوْرَاةَ أَمَامَ كُلِّ إِسْرَائِيل فِي مَسَامِعِهِمْ.
\par 12 اِجْمَعِ الشَّعْبَ الرِّجَال وَالنِّسَاءَ وَالأَطْفَال وَالغَرِيبَ الذِي فِي أَبْوَابِكَ لِيَسْمَعُوا وَيَتَعَلمُوا أَنْ يَتَّقُوا الرَّبَّ إِلهَكُمْ وَيَحْرَِصُوا أَنْ يَعْمَلُوا بِجَمِيعِ كَلِمَاتِ هَذِهِ التَّوْرَاةِ.
\par 13 وَأَوْلادُهُمُ الذِينَ لمْ يَعْرِفُوا يَسْمَعُونَ وَيَتَعَلمُونَ أَنْ يَتَّقُوا الرَّبَّ إِلهَكُمْ كُل الأَيَّامِ التِي تَحْيُونَ فِيهَا عَلى الأَرْضِ التِي أَنْتُمْ عَابِرُونَ الأُرْدُنَّ إِليْهَا لِتَمْتَلِكُوهَا».
\par 14 وَقَال الرَّبُّ لِمُوسَى: «هُوَذَا أَيَّامُكَ قَدْ قَرُبَتْ لِتَمُوتَ. ادْعُ يَشُوعَ وَقِفَا فِي خَيْمَةِ الاِجْتِمَاعِ لِكَيْ أُوصِيَهُ». فَانْطَلقَ مُوسَى وَيَشُوعُ وَوَقَفَا فِي خَيْمَةِ الاِجْتِمَاعِ
\par 15 فَتَرَاءَى الرَّبُّ فِي الخَيْمَةِ فِي عَمُودِ سَحَابٍ وَوَقَفَ عَمُودُ السَّحَابِ عَلى بَابِ الخَيْمَةِ.
\par 16 وَقَال الرَّبُّ لِمُوسَى: «هَا أَنْتَ تَرْقُدُ مَعَ آبَائِكَ فَيَقُومُ هَذَا الشَّعْبُ وَيَفْجُرُ وَرَاءَ آلِهَةِ الأَجْنَبِيِّينَ فِي الأَرْضِ التِي هُوَ دَاخِلٌ إِليْهَا فِي مَا بَيْنَهُمْ وَيَتْرُكُنِي وَيَنْكُثُ عَهْدِي الذِي قَطَعْتُهُ مَعَهُ.
\par 17 فَيَشْتَعِلُ غَضَبِي عَليْهِ فِي ذَلِكَ اليَوْمِ وَأَتْرُكُهُ وَأَحْجُبُ وَجْهِي عَنْهُ فَيَكُونُ مَأْكُلةً وَتُصِيبُهُ شُرُورٌ كَثِيرَةٌ وَشَدَائِدُ حَتَّى يَقُول فِي ذَلِكَ اليَوْمِ: أَمَا لأَنَّ إِلهِي ليْسَ فِي وَسَطِي أَصَابَتْنِي هَذِهِ الشُّرُورُ!.
\par 18 وَأَنَا أَحْجُبُ وَجْهِي فِي ذَلِكَ اليَوْمِ لأَجْلِ جَمِيعِ الشَّرِّ الذِي عَمِلهُ إِذِ التَفَتَ إِلى آلِهَةٍ أُخْرَى.
\par 19 فَالآنَ اكْتُبُوا لأَنْفُسِكُمْ هَذَا النَّشِيدَ وَعَلِّمْ بَنِي إِسْرَائِيل إِيَّاهُ. ضَعْهُ فِي أَفْوَاهِهِمْ لِيَكُونَ لِي هَذَا النَّشِيدُ شَاهِداً عَلى بَنِي إِسْرَائِيل.
\par 20 لأَنِّي أُدْخِلُهُمُ الأَرْضَ التِي أَقْسَمْتُ لآِبَائِهِمِ الفَائِضَةَ لبَناً وَعَسَلاً فَيَأْكُلُونَ وَيَشْبَعُونَ وَيَسْمَنُونَ ثُمَّ يَلتَفِتُونَ إِلى آلِهَةٍ أُخْرَى وَيَعْبُدُونَهَا وَيَزْدَرُونَ بِي وَيَنْكُثُونَ عَهْدِي.
\par 21 فَمَتَى أَصَابَتْهُ شُرُورٌ كَثِيرَةٌ وَشَدَائِدُ يُجَاوِبُ هَذَا النَّشِيدُ أَمَامَهُ شَاهِداً لأَنَّهُ لا يُنْسَى مِنْ أَفْوَاهِ نَسْلِهِ. إِنِّي عَرَفْتُ فِكْرَهُ الذِي يُفَكِّرُ بِهِ اليَوْمَ قَبْل أَنْ أُدْخِلهُ إِلى الأَرْضِ كَمَا أَقْسَمْتُ».
\par 22 فَكَتَبَ مُوسَى هَذَا النَّشِيدَ فِي ذَلِكَ اليَوْمِ وَعَلمَ بَنِي إِسْرَائِيل إِيَّاهُ.
\par 23 وَأَوْصَى يَشُوعَ بْنَ نُونَ وَقَال: «تَشَدَّدْ وَتَشَجَّعْ لأَنَّكَ أَنْتَ تَدْخُلُ بِبَنِي إِسْرَائِيل الأَرْضَ التِي أَقْسَمْتُ لهُمْ عَنْهَا وَأَنَا أَكُونُ مَعَكَ».
\par 24 فَعِنْدَمَا كَمَّل مُوسَى كِتَابَةَ كَلِمَاتِ هَذِهِ التَّوْرَاةِ فِي كِتَابٍ إِلى تَمَامِهَا
\par 25 أَمَرَ مُوسَى اللاوِيِّينَ حَامِلِي تَابُوتِ عَهْدِ الرَّبِّ:
\par 26 «خُذُوا كِتَابَ التَّوْرَاةِ هَذَا وَضَعُوهُ بِجَانِبِ تَابُوتِ عَهْدِ الرَّبِّ إِلهِكُمْ لِيَكُونَ هُنَاكَ شَاهِداً عَليْكُمْ.
\par 27 لأَنِّي أَنَا عَارِفٌ تَمَرُّدَكُمْ وَرِقَابَكُمُ الصُّلبَةَ. هُوَذَا وَأَنَا بَعْدُ حَيٌّ مَعَكُمُ اليَوْمَ قَدْ صِرْتُمْ تُقَاوِمُونَ الرَّبَّ فَكَمْ بِالحَرِيِّ بَعْدَ مَوْتِي!
\par 28 اِجْمَعُوا إِليَّ كُل شُيُوخِ أَسْبَاطِكُمْ وَعُرَفَاءَكُمْ لأَنْطِقَ فِي مَسَامِعِهِمْ بِهَذِهِ الكَلِمَاتِ وَأُشْهِدَ عَليْهِمِ السَّمَاءَ وَالأَرْضَ.
\par 29 لأَنِّي عَارِفٌ أَنَّكُمْ بَعْدَ مَوْتِي تَفْسِدُونَ وَتَزِيغُونَ عَنِ الطَّرِيقِ الذِي أَوْصَيْتُكُمْ بِهِ وَيُصِيبُكُمُ الشَّرُّ فِي آخِرِ الأَيَّامِ لأَنَّكُمْ تَعْمَلُونَ الشَّرَّ أَمَامَ الرَّبِّ حَتَّى تُغِيظُوهُ بِأَعْمَالِ أَيْدِيكُمْ».
\par 30 فَنَطَقَ مُوسَى فِي مَسَامِعِ كُلِّ جَمَاعَةِ إِسْرَائِيل بِكَلِمَاتِ هَذَا النَّشِيدِ إِلى تَمَامِهِ:

\chapter{32}

\par 1 «اُنْصُتِي أَيَّتُهَا السَّمَاوَاتُ فَأَتَكَلمَ وَلتَسْمَعِ الأَرْضُ أَقْوَال فَمِي.
\par 2 يَهْطِلُ كَالمَطَرِ تَعْلِيمِي وَيَقْطُرُ كَالنَّدَى كَلامِي. كَالطَّلِّ عَلى الكَلإِ وَكَالوَابِلِ عَلى العُشْبِ.
\par 3 إِنِّي بِاسْمِ الرَّبِّ أُنَادِي. أَعْطُوا عَظَمَةً لِإِلهِنَا.
\par 4 هُوَ الصَّخْرُ الكَامِلُ صَنِيعُهُ. إِنَّ جَمِيعَ سُبُلِهِ عَدْلٌ. إِلهُ أَمَانَةٍ لا جَوْرَ فِيهِ. صِدِّيقٌ وَعَادِلٌ هُوَ.
\par 5 «فَسَدُوا تِجَاهَهُ الذِينَ هُمْ عَارٌ وَليْسُوا أَوْلادَهُ جِيلٌ أَعْوَجُ مُلتَوٍ.
\par 6 هَل تُكَافِئُونَ الرَّبَّ بِهَذَا يَا شَعْباً غَبِيّاً غَيْرَ حَكِيمٍ؟ أَليْسَ هُوَ أَبَاكَ وَمُقْتَنِيَكَ هُوَ عَمِلكَ وَأَنْشَأَكَ؟
\par 7 اُذْكُرْ أَيَّامَ القِدَمِ وَتَأَمَّلُوا سِنِي دَوْرٍ فَدَوْرٍ. اسْأَل أَبَاكَ فَيُخْبِرَكَ وَشُيُوخَكَ فَيَقُولُوا لكَ.
\par 8 «حِينَ قَسَمَ العَلِيُّ لِلأُمَمِ حِينَ فَرَّقَ بَنِي آدَمَ نَصَبَ تُخُوماً لِشُعُوبٍ حَسَبَ عَدَدِ بَنِي إِسْرَائِيل.
\par 9 إِنَّ قِسْمَ الرَّبِّ هُوَ شَعْبُهُ. يَعْقُوبُ حَبْلُ نَصِيبِهِ.
\par 10 وَجَدَهُ فِي أَرْضِ قَفْرٍ وَفِي خَلاءٍ مُسْتَوْحِشٍ خَرِبٍ. أَحَاطَ بِهِ وَلاحَظَهُ وَصَانَهُ كَحَدَقَةِ عَيْنِهِ.
\par 11 كَمَا يُحَرِّكُ النَّسْرُ عُشَّهُ وَعَلى فِرَاخِهِ يَرِفُّ وَيَبْسُطُ جَنَاحَيْهِ وَيَأْخُذُهَا وَيَحْمِلُهَا عَلى مَنَاكِبِهِ
\par 12 هَكَذَا الرَّبُّ وَحْدَهُ اقْتَادَهُ وَليْسَ مَعَهُ إِلهٌ أَجْنَبِيٌّ.
\par 13 أَرْكَبَهُ عَلى مُرْتَفَعَاتِ الأَرْضِ فَأَكَل ثِمَارَ الصَّحْرَاءِ وَأَرْضَعَهُ عَسَلاً مِنْ حَجَرٍ وَزَيْتاً مِنْ صَوَّانِ الصَّخْرِ
\par 14 وَزُبْدَةَ بَقَرٍ وَلبَنَ غَنَمٍ مَعَ شَحْمِ خِرَافٍ وَكِبَاشٍ أَوْلادِ بَاشَانَ وَتُيُوسٍ مَعَ دَسَمِ لُبِّ الحِنْطَةِ وَدَمَ العِنَبِ شَرِبْتَهُ خَمْراً.
\par 15 «فَسَمِنَ يَشُورُونَ وَرَفَسَ. سَمِنْتَ وَغَلُظْتَ وَاكْتَسَيْتَ شَحْماً! فَرَفَضَ الإِلهَ الذِي عَمِلهُ وَغَبِيَ عَنْ صَخْرَةِ خَلاصِهِ.
\par 16 أَغَارُوهُ بِالأَجَانِبِ وَأَغَاظُوهُ بِالأَرْجَاسِ.
\par 17 ذَبَحُوا لأَوْثَانٍ ليْسَتِ اللهَ. لآِلِهَةٍ لمْ يَعْرِفُوهَا أَحْدَاثٍ قَدْ جَاءَتْ مِنْ قَرِيبٍ لمْ يَرْهَبْهَا آبَاؤُكُمْ.
\par 18 الصَّخْرُ الذِي وَلدَكَ تَرَكْتَهُ وَنَسِيتَ اللهَ الذِي أَبْدَأَكَ.
\par 19 «فَرَأَى الرَّبُّ وَرَذَل مِنَ الغَيْظِ بَنِيهِ وَبَنَاتِهِ.
\par 20 وَقَال أَحْجُبُ وَجْهِي عَنْهُمْ وَأَنْظُرُ مَاذَا تَكُونُ آخِرَتُهُمْ. إِنَّهُمْ جِيلٌ مُتَقَلِّبٌ أَوْلادٌ لا أَمَانَةَ فِيهِمْ.
\par 21 هُمْ أَغَارُونِي بِمَا ليْسَ إِلهاً أَغَاظُونِي بِأَبَاطِيلِهِمْ. فَأَنَا أُغِيرُهُمْ بِمَا ليْسَ شَعْباً بِأُمَّةٍ غَبِيَّةٍ أُغِيظُهُمْ.
\par 22 إِنَّهُ قَدِ اشْتَعَلتْ نَارٌ بِغَضَبِي فَتَتَّقِدُ إِلى الهَاوِيَةِ السُّفْلى وَتَأْكُلُ الأَرْضَ وَغَلتَهَا وَتُحْرِقُ أُسُسَ الجِبَالِ.
\par 23 أَجْمَعُ عَليْهِمْ شُرُوراً وَأُنْفِدُ سِهَامِي فِيهِمْ
\par 24 إِذْ هُمْ خَاوُونَ مِنْ جُوعٍ وَمَنْهُوكُونَ مِنْ حُمَّى وَدَاءٍ سَامٍّ. أُرْسِلُ فِيهِمْ أَنْيَابَ الوُحُوشِ مَعَ حُمَةِ زَوَاحِفِ الأَرْضِ.
\par 25 مِنْ خَارِجٍ السَّيْفُ يُثْكِلُ وَمِنْ دَاخِلِ الخُدُورِ الرُّعْبَةُ. الفَتَى مَعَ الفَتَاةِ وَالرَّضِيعُ مَعَ الأَشْيَبِ.
\par 26 قُلتُ أُبَدِّدُهُمْ إِلى الزَّوَايَا وَأُبَطِّلُ مِنَ النَّاسِ ذِكْرَهُمْ.
\par 27 لوْ لمْ أَخَفْ مِنْ إِغَاظَةِ العَدُوِّ مِنْ أَنْ يُنْكِرَ أَضْدَادُهُمْ مِنْ أَنْ يَقُولُوا: يَدُنَا ارْتَفَعَتْ وَليْسَ الرَّبُّ فَعَل كُل هَذِهِ.
\par 28 «إِنَّهُمْ أُمَّةٌ عَدِيمَةُ الرَّأْيِ وَلا بَصِيرَةَ فِيهِمْ.
\par 29 لوْ عَقَلُوا لفَطِنُوا بِهَذِهِ وَتَأَمَّلُوا آخِرَتَهُمْ.
\par 30 كَيْفَ يَطْرُدُ وَاحِدٌ أَلفاً وَيَهْزِمُ اثْنَانِ رَبْوَةً لوْلا أَنَّ صَخْرَهُمْ بَاعَهُمْ وَالرَّبَّ سَلمَهُمْ؟
\par 31 لأَنَّهُ ليْسَ كَصَخْرِنَا صَخْرُهُمْ وَلوْ كَانَ أَعْدَاؤُنَا حَاكِمِينَ.
\par 32 لأَنَّ مِنْ جَفْنَةِ سَدُومَ جَفْنَتَهُمْ وَمِنْ كُرُومِ عَمُورَةَ. عِنَبُهُمْ عِنَبُ سُمٍّ وَلهُمْ عَنَاقِيدُ مَرَارَةٍ.
\par 33 خَمْرُهُمْ حُمَةُ الثَّعَابِينِ وَسِمُّ الأَصْلالِ القَاتِلُ.
\par 34 «أَليْسَ ذَلِكَ مَكْنُوزاً عِنْدِي مَخْتُوماً عَليْهِ فِي خَزَائِنِي؟
\par 35 لِيَ النَّقْمَةُ وَالجَزَاءُ. فِي وَقْتٍ تَزِلُّ أَقْدَامُهُمْ. إِنَّ يَوْمَ هَلاكِهِمْ قَرِيبٌ وَالمُهَيَّئَاتُ لهُمْ مُسْرِعَةٌ.
\par 36 لأَنَّ الرَّبَّ يَدِينُ شَعْبَهُ وَعَلى عَبِيدِهِ يُشْفِقُ. حِينَ يَرَى أَنَّ اليَدَ قَدْ مَضَتْ وَلمْ يَبْقَ مَحْجُوزٌ وَلا مُطْلقٌ
\par 37 يَقُولُ: أَيْنَ آلِهَتُهُمُ الصَّخْرَةُ التِي التَجَأُوا إِليْهَا
\par 38 التِي كَانَتْ تَأْكُلُ شَحْمَ ذَبَائِحِهِمْ وَتَشْرَبُ خَمْرَ سَكَائِبِهِمْ؟ لِتَقُمْ وَتُسَاعِدْكُمْ وَتَكُنْ عَليْكُمْ حِمَايَةً.
\par 39 اُنْظُرُوا الآنَ! أَنَا أَنَا هُوَ وَليْسَ إِلهٌ مَعِي. أَنَا أُمِيتُ وَأُحْيِي. سَحَقْتُ وَإِنِّي أَشْفِي وَليْسَ مِنْ يَدِي مُخَلِّصٌ.
\par 40 إِنِّي أَرْفَعُ إِلى السَّمَاءِ يَدِي وَأَقُولُ: حَيٌّ أَنَا إِلى الأَبَدِ.
\par 41 إِذَا سَنَنْتُ سَيْفِي البَارِقَ وَأَمْسَكَتْ بِالقَضَاءِ يَدِي أَرُدُّ نَقْمَةً عَلى أَضْدَادِي وَأُجَازِي مُبْغِضِيَّ.
\par 42 أُسْكِرُ سِهَامِي بِدَمٍ وَيَأْكُلُ سَيْفِي لحْماً. بِدَمِ القَتْلى وَالسَّبَايَا وَمِنْ رُؤُوسِ قُوَّادِ العَدُوِّ.
\par 43 «تَهَللُوا أَيُّهَا الأُمَمُ شَعْبُهُ لأَنَّهُ يَنْتَقِمُ بِدَمِ عَبِيدِهِ وَيَرُدُّ نَقْمَةً عَلى أَضْدَادِهِ وَيَصْفَحُ عَنْ أَرْضِهِ عَنْ شَعْبِهِ».
\par 44 فَأَتَى مُوسَى وَنَطَقَ بِجَمِيعِ كَلِمَاتِ هَذَا النَّشِيدِ فِي مَسَامِعِ الشَّعْبِ هُوَ وَيَشُوعُ بْنُ نُونَ.
\par 45 وَلمَّا فَرَغَ مُوسَى مِنْ مُخَاطَبَةِ جَمِيعِ إِسْرَائِيل بِكُلِّ هَذِهِ الكَلِمَاتِ
\par 46 قَال لهُمْ: «وَجِّهُوا قُلُوبَكُمْ إِلى جَمِيعِ الكَلِمَاتِ التِي أَنَا أَشْهَدُ عَليْكُمْ بِهَا اليَوْمَ لِكَيْ تُوصُوا بِهَا أَوْلادَكُمْ لِيَحْرِصُوا أَنْ يَعْمَلُوا بِجَمِيعِ كَلِمَاتِ هَذِهِ التَّوْرَاةِ.
\par 47 لأَنَّهَا ليْسَتْ أَمْراً بَاطِلاً عَليْكُمْ بَل هِيَ حَيَاتُكُمْ. وَبِهَذَا الأَمْرِ تُطِيلُونَ الأَيَّامَ عَلى الأَرْضِ التِي أَنْتُمْ عَابِرُونَ الأُرْدُنَّ إِليْهَا لِتَمْتَلِكُوهَا».
\par 48 وَقَال الرَّبُّ لِمُوسَى فِي نَفْسِ ذَلِكَ اليَوْمِ:
\par 49 «اِصْعَدْ إِلى جَبَلِ عَبَارِيمَ هَذَا جَبَلِ نَبُو الذِي فِي أَرْضِ مُوآبَ الذِي قُبَالةَ أَرِيحَا وَانْظُرْ أَرْضَ كَنْعَانَ التِي أَنَا أُعْطِيهَا لِبَنِي إِسْرَائِيل مُلكاً
\par 50 وَمُتْ فِي الجَبَلِ الذِي تَصْعَدُ إِليْهِ وَانْضَمَّ إِلى قَوْمِكَ كَمَا مَاتَ هَارُونُ أَخُوكَ فِي جَبَلِ هُورٍ وَضُمَّ إِلى قَوْمِهِ.
\par 51 لأَنَّكُمَا خُنْتُمَانِي فِي وَسَطِ بَنِي إِسْرَائِيل عِنْدَ مَاءِ مَرِيبَةِ قَادِشَ فِي بَرِّيَّةِ صِينٍ إِذْ لمْ تُقَدِّسَانِي فِي وَسَطِ بَنِي إِسْرَائِيل.
\par 52 فَإِنَّكَ تَنْظُرُ الأَرْضَ مِنْ قُبَالتِهَا وَلكِنَّكَ لا تَدْخُلُ إِلى هُنَاكَ إِلى الأَرْضِ التِي أَنَا أُعْطِيهَا لِبَنِي إِسْرَائِيل».

\chapter{33}

\par 1 وَهَذِهِ هِيَ البَرَكَةُ التِي بَارَكَ بِهَا مُوسَى رَجُلُ اللهِ بَنِي إِسْرَائِيل قَبْل مَوْتِهِ
\par 2 فَقَال: «جَاءَ الرَّبُّ مِنْ سِينَاءَ وَأَشْرَقَ لهُمْ مِنْ سَعِيرَ وَتَلأْلأَ مِنْ جَبَلِ فَارَانَ وَأَتَى مِنْ رَبَوَاتِ القُدْسِ وَعَنْ يَمِينِهِ نَارُ شَرِيعَةٍ لهُمْ.
\par 3 فَأَحَبَّ الشَّعْبَ. جَمِيعُ قِدِّيسِيهِ فِي يَدِكَ وَهُمْ جَالِسُونَ عِنْدَ قَدَمِكَ يَتَقَبَّلُونَ مِنْ أَقْوَالِكَ.
\par 4 بِنَامُوسٍ أَوْصَانَا مُوسَى مِيرَاثاً لِجَمَاعَةِ يَعْقُوبَ.
\par 5 وَكَانَ فِي يَشُورُونَ مَلِكاً حِينَ اجْتَمَعَ رُؤَسَاءُ الشَّعْبِ أَسْبَاطُ إِسْرَائِيل مَعاً.
\par 6 لِيَحْيَ رَأُوبَيْنُ وَلا يَمُتْ وَلا يَكُنْ رِجَالُهُ قَلِيلِينَ».
\par 7 وَهَذِهِ عَنْ يَهُوذَا: «قَال اسْمَعْ يَا رَبُّ صَوْتَ يَهُوذَا وَأْتِ بِهِ إِلى قَوْمِهِ. بِيَدَيْهِ يُقَاتِلُ لِنَفْسِهِ فَكُنْ عَوْناً عَلى أَضْدَادِهِ».
\par 8 وَلِلاوِي قَال: «تُمِّيمُكَ وَأُورِيمُكَ لِرَجُلِكَ الصِّدِّيقِ الذِي جَرَّبْتَهُ فِي مَسَّةَ وَخَاصَمْتَهُ عِنْدَ مَاءِ مَرِيبَةَ.
\par 9 الذِي قَال عَنْ أَبِيهِ وَأُمِّهِ: لمْ أَرَهُمَا وَبِإِخْوَتِهِ لمْ يَعْتَرِفْ وَأَوْلادَهُ لمْ يَعْرِفْ بَل حَفِظُوا كَلامَكَ وَصَانُوا عَهْدَكَ.
\par 10 يُعَلِّمُونَ يَعْقُوبَ أَحْكَامَكَ وَإِسْرَائِيل نَامُوسَكَ. يَضَعُونَ بَخُوراً فِي أَنْفِكَ وَمُحْرَقَاتٍ عَلى مَذْبَحِكَ.
\par 11 بَارِكْ يَا رَبُّ قُوَّتَهُ وَارْتَضِ بِعَمَلِ يَدَيْهِ. احْطِمْ مُتُونَ مُقَاوِمِيهِ وَمُبْغِضِيهِ حَتَّى لا يَقُومُوا».
\par 12 وَلِبِنْيَامِينَ قَال: «حَبِيبُ الرَّبِّ يَسْكُنُ لدَيْهِ آمِناً. يَسْتُرُهُ طُول النَّهَارِ وَبَيْنَ مَنْكِبَيْهِ يَسْكُنُ».
\par 13 وَلِيُوسُفَ قَال: «مُبَارَكَةٌ مِنَ الرَّبِّ أَرْضُهُ بِنَفَائِسِ السَّمَاءِ بِالنَّدَى وَبِاللُّجَّةِ الرَّابِضَةِ تَحْتُ
\par 14 وَنَفَائِسِ مُغَلاتِ الشَّمْسِ وَنَفَائِسِ مُنْبَتَاتِ الأَقْمَارِ.
\par 15 وَمِنْ مَفَاخِرِ الجِبَالِ القَدِيمَةِ وَمِنْ نَفَائِسِ الإِكَامِ الأَبَدِيَّةِ
\par 16 وَمِنْ نَفَائِسِ الأَرْضِ وَمِلئِهَا وَرِضَى السَّاكِنِ فِي العُليْقَةِ. فَلتَأْتِ عَلى رَأْسِ يُوسُفَ وَعَلى قِمَّةِ نَذِيرِ إِخْوَتِهِ.
\par 17 بِكْرُ ثَوْرِهِ زِينَةٌ لهُ وَقَرْنَاهُ قَرْنَا رِئْمٍ. بِهِمَا يَنْطَحُ الشُّعُوبَ مَعاً إِلى أَقَاصِي الأَرْضِ. هُمَا رَبَوَاتُ أَفْرَايِمَ وَأُلُوفُ مَنَسَّى».
\par 18 وَلِزَبُولُونَ قَال: «اِفْرَحْ يَا زَبُولُونُ بِخُرُوجِكَ وَأَنْتَ يَا يَسَّاكَرُ بِخِيَامِكَ.
\par 19 إِلى الجَبَلِ يَدْعُوانِ القَبَائِل. هُنَاكَ يَذْبَحَانِ ذَبَائِحَ البِرِّ لأَنَّهُمَا يَرْتَضِعَانِ مِنْ فَيْضِ البِحَارِ وَذَخَائِرَ مَطْمُورَةٍ فِي الرَّمْلِ».
\par 20 وَلِجَادَ قَال: «مُبَارَكٌ الذِي وَسَّعَ جَادَ. كَلبْوَةٍ سَكَنَ وَافْتَرَسَ الذِّرَاعَ مَعَ قِمَّةِ الرَّأْسِ.
\par 21 وَرَأَى الأَوَّل لِنَفْسِهِ لأَنَّهُ هُنَاكَ قِسْمٌ مِنَ الشَّارِعِ مَحْفُوظاً فَأَتَى رَأْساً لِلشَّعْبِ يَعْمَلُ حَقَّ الرَّبِّ وَأَحْكَامَهُ مَعَ إِسْرَائِيل».
\par 22 وَلِدَانَ قَال: «دَانُ شِبْلُ أَسَدٍ يَثِبُ مِنْ بَاشَانَ».
\par 23 وَلِنَفْتَالِي قَال: « يَا نَفْتَالِي اشْبَعْ رِضىً وَامْتَلِئْ بَرَكَةً مِنَ الرَّبِّ وَامْلِكِ الغَرْبَ وَالجَنُوبَ».
\par 24 وَلأَشِيرَ قَال: «مُبَارَكٌ مِنَ البَنِينَ أَشِيرُ. لِيَكُنْ مَقْبُولاً مِنْ إِخْوَتِهِ وَيَغْمِسْ فِي الزَّيْتِ رِجْلهُ.
\par 25 حَدِيدٌ وَنُحَاسٌ مَزَالِيجُكَ وَكَأَيَّامِكَ رَاحَتُكَ.
\par 26 «ليْسَ مِثْل اللهِ يَا يَشُورُونُ. يَرْكَبُ السَّمَاءَ فِي مَعُونَتِكَ وَالغَمَامَ فِي عَظَمَتِهِ.
\par 27 الإِلهُ القَدِيمُ مَلجَأٌ وَالأَذْرُعُ الأَبَدِيَّةُ مِنْ تَحْتُ. فَطَرَدَ مِنْ قُدَّامِكَ العَدُوَّ وَقَال: أَهْلِكْ.
\par 28 فَيَسْكُنَ إِسْرَائِيلُ آمِناً وَحْدَهُ. تَكُونُ عَيْنُ يَعْقُوبَ إِلى أَرْضِ حِنْطَةٍ وَخَمْرٍ وَسَمَاؤُهُ تَقْطُرُ نَدىً.
\par 29 طُوبَاكَ يَا إِسْرَائِيلُ! مَنْ مِثْلُكَ يَا شَعْباً مَنْصُوراً بِالرَّبِّ تُرْسِ عَوْنِكَ وَسَيْفِ عَظَمَتِكَ! فَيَتَذَللُ لكَ أَعْدَاؤُكَ وَأَنْتَ تَطَأُ مُرْتَفَعَاتِهِمْ».

\chapter{34}

\par 1 وَصَعِدَ مُوسَى مِنْ عَرَبَاتِ مُوآبَ إِلى جَبَلِ نَبُو إِلى رَأْسِ الفِسْجَةِ الذِي قُبَالةَ أَرِيحَا فَأَرَاهُ الرَّبُّ جَمِيعَ الأَرْضِ مِنْ جِلعَادَ إِلى دَانَ
\par 2 وَجَمِيعَ نَفْتَالِي وَأَرْضَ أَفْرَايِمَ وَمَنَسَّى وَجَمِيعَ أَرْضِ يَهُوذَا إِلى البَحْرِ الغَرْبِيِّ
\par 3 وَالجَنُوبَ وَالدَّائِرَةَ بُقْعَةَ أَرِيحَا مَدِينَةِ النَّخْلِ إِلى صُوغَرَ.
\par 4 وَقَال لهُ الرَّبُّ: «هَذِهِ هِيَ الأَرْضُ التِي أَقْسَمْتُ لِإِبْرَاهِيمَ وَإِسْحَاقَ وَيَعْقُوبَ قَائِلاً: لِنَسْلِكَ أُعْطِيهَا. قَدْ أَرَيْتُكَ إِيَّاهَا بِعَيْنَيْكَ وَلكِنَّكَ إِلى هُنَاكَ لا تَعْبُرُ».
\par 5 فَمَاتَ هُنَاكَ مُوسَى عَبْدُ الرَّبِّ فِي أَرْضِ مُوآبَ حَسَبَ قَوْلِ الرَّبِّ.
\par 6 وَدَفَنَهُ فِي الجِوَاءِ فِي أَرْضِ مُوآبَ مُقَابِل بَيْتِ فَغُورَ. وَلمْ يَعْرِفْ إِنْسَانٌ قَبْرَهُ إِلى هَذَا اليَوْمِ.
\par 7 وَكَانَ مُوسَى ابْنَ مِئَةٍ وَعِشْرِينَ سَنَةً حِينَ مَاتَ وَلمْ تَكِل عَيْنُهُ وَلا ذَهَبَتْ نَضَارَتُهُ.
\par 8 فَبَكَى بَنُو إِسْرَائِيل مُوسَى فِي عَرَبَاتِ مُوآبَ ثَلاثِينَ يَوْماً. فَكَمُلتْ أَيَّامُ بُكَاءِ مَنَاحَةِ مُوسَى.
\par 9 وَيَشُوعُ بْنُ نُونٍ كَانَ قَدِ امْتَلأَ رُوحَ حِكْمَةٍ إِذْ وَضَعَ مُوسَى عَليْهِ يَدَيْهِ فَسَمِعَ لهُ بَنُو إِسْرَائِيل وَعَمِلُوا كَمَا أَوْصَى الرَّبُّ مُوسَى.
\par 10 وَلمْ يَقُمْ بَعْدُ نَبِيٌّ فِي إِسْرَائِيل مِثْلُ مُوسَى الذِي عَرَفَهُ الرَّبُّ وَجْهاً لِوَجْهٍ
\par 11 فِي جَمِيعِ الآيَاتِ وَالعَجَائِبِ التِي أَرْسَلهُ الرَّبُّ لِيَعْمَلهَا فِي أَرْضِ مِصْرَ بِفِرْعَوْنَ وَبِجَمِيعِ عَبِيدِهِ وَكُلِّ أَرْضِهِ
\par 12 وَفِي كُلِّ اليَدِ الشَّدِيدَةِ وَكُلِّ المَخَاوِفِ العَظِيمَةِ التِي صَنَعَهَا مُوسَى أَمَامَ أَعْيُنِ جَمِيعِ إِسْرَائِيلَ.


\end{document}