\begin{document}

\title{2 اخبار}


\chapter{1}

\par 1 وَتَشَدَّدَ سُلَيْمَانُ بْنُ دَاوُدَ عَلَى مَمْلَكَتِهِ وَكَانَ الرَّبُّ إِلَهُهُ مَعَهُ وَعَظَّمَهُ جِدّاً.
\par 2 وَكَلَّمَ سُلَيْمَانُ جَمِيعَ إِسْرَائِيلَ رُؤَسَاءَ الأُلُوفِ وَالْمِئَاتِ وَالْقُضَاةَ وَكُلَّ رَئِيسٍ فِي كُلِّ إِسْرَائِيلَ رُؤُوسَ الآبَاءِ
\par 3 فَذَهَبَ سُلَيْمَانُ وَكُلُّ الْجَمَاعَةِ مَعَهُ إِلَى الْمُرْتَفَعَةِ الَّتِي فِي جِبْعُونَ لأَنَّهُ هُنَاكَ كَانَتْ خَيْمَةُ الاِجْتِمَاعِ خَيْمَةُ اللَّهِ الَّتِي عَمِلَهَا مُوسَى عَبْدُ الرَّبِّ فِي الْبَرِّيَّةِ.
\par 4 وَأَمَّا تَابُوتُ اللَّهِ فَأَصْعَدَهُ دَاوُدُ مِنْ قَرْيَةِ يَعَارِيمَ عِنْدَمَا هَيَّأَ لَهُ دَاوُدُ لأَنَّهُ نَصَبَ لَهُ خَيْمَةً فِي أُورُشَلِيمَ.
\par 5 وَمَذْبَحُ النُّحَاسِ الَّذِي عَمِلَهُ بَصَلْئِيلُ بْنُ أُورِي بْنِ حُورَ وَضَعَهُ أَمَامَ مَسْكَنِ الرَّبِّ. وَطَلَبَ إِلَيْهِ سُلَيْمَانُ وَالْجَمَاعَةُ.
\par 6 وَصَعِدَ سُلَيْمَانُ هُنَاكَ إِلَى مَذْبَحِ النُّحَاسِ أَمَامَ الرَّبِّ الَّذِي كَانَ فِي خَيْمَةِ الاِجْتِمَاعِ وَأَصْعَدَ عَلَيْهِ أَلْفَ مُحْرَقَةٍ.
\par 7 فِي تِلْكَ اللَّيْلَةِ تَرَاءَى اللَّهُ لِسُلَيْمَانَ وَقَالَ لَهُ: [اسْأَلْ مَاذَا أُعْطِيكَ].
\par 8 فَقَالَ سُلَيْمَانُ لِلَّهِ: [إِنَّكَ قَدْ فَعَلْتَ مَعَ دَاوُدَ أَبِي رَحْمَةً عَظِيمَةً وَمَلَّكْتَنِي مَكَانَهُ.
\par 9 فَالآنَ أَيُّهَا الرَّبُّ الإِلَهُ لِيَثْبُتْ كَلاَمُكَ مَعَ دَاوُدَ أَبِي لأَنَّكَ قَدْ مَلَّكْتَنِي عَلَى شَعْبٍ كَثِيرٍ كَتُرَابِ الأَرْضِ.
\par 10 فَأَعْطِنِي الآنَ حِكْمَةً وَمَعْرِفَةً لأَخْرُجَ أَمَامَ هَذَا الشَّعْبِ وَأَدْخُلَ لأَنَّهُ مَنْ يَقْدِرُ أَنْ يَحْكُمَ عَلَى شَعْبِكَ هَذَا الْعَظِيمِ]
\par 11 فَقَالَ اللَّهُ لِسُلَيْمَانَ: [مِنْ أَجْلِ أَنَّ هَذَا كَانَ فِي قَلْبِكَ وَلَمْ تَسْأَلْ غِنىً وَلاَ أَمْوَالاً وَلاَ كَرَامَةً وَلاَ أَنْفُسَ مُبْغِضِيكَ وَلاَ سَأَلْتَ أَيَّاماً كَثِيرَةً بَلْ إِنَّمَا سَأَلْتَ لِنَفْسِكَ حِكْمَةً وَمَعْرِفَةً تَحْكُمُ بِهِمَا عَلَى شَعْبِي الَّذِي مَلَّكْتُكَ عَلَيْهِ
\par 12 قَدْ أَعْطَيْتُكَ حِكْمَةً وَمَعْرِفَةً وَأُعْطِيكَ غِنىً وَأَمْوَالاً وَكَرَامَةً لَمْ يَكُنْ مِثْلُهَا لِلْمُلُوكِ الَّذِينَ قَبْلَكَ وَلاَ يَكُونُ مِثْلُهَا لِمَنْ بَعْدَكَ].
\par 13 فَجَاءَ سُلَيْمَانُ مِنَ الْمُرْتَفَعَةِ الَّتِي فِي جِبْعُونَ إِلَى أُورُشَلِيمَ مِنْ أَمَامِ خَيْمَةِ الاِجْتِمَاعِ وَمَلَكَ عَلَى إِسْرَائِيلَ.
\par 14 وَجَمَعَ سُلَيْمَانُ مَرْكَبَاتٍ وَفُرْسَاناً فَكَانَ لَهُ أَلْفٌ وَأَرْبَعُ مِئَةِ مَرْكَبَةٍ وَاثْنَا عَشَرَ أَلْفَ فَارِسٍ فَجَعَلَهَا فِي مُدُنِ الْمَرْكَبَاتِ وَمَعَ الْمَلِكِ فِي أُورُشَلِيمَ.
\par 15 وَجَعَلَ الْمَلِكُ الْفِضَّةَ وَالذَّهَبَ فِي أُورُشَلِيمَ مِثْلَ الْحِجَارَةِ وَجَعَلَ الأَرْزَ كَالْجُمَّيْزِ الَّذِي فِي السَّهْلِ فِي الْكَثْرَةِ.
\par 16 وَكَانَ مُخْرَجُ الْخَيْلِ الَّتِي لِسُلَيْمَانَ مِنْ مِصْرَ. وَجَمَاعَةُ تُجَّارِ الْمَلِكِ أَخَذُوا جَلِيبَةً بِثَمَنٍ
\par 17 فَأَصْعَدُوا وَأَخْرَجُوا مِنْ مِصْرَ الْمَرْكَبَةَ بِسِتِّ مِئَةِ شَاقِلٍ مِنَ الْفِضَّةِ وَالْفَرَسَ بِمِئَةٍ وَخَمْسِينَ وَهَكَذَا لِجَمِيعِ مُلُوكِ الْحِثِّيِّينَ وَمُلُوكِ أَرَامَ كَانُوا يُخْرِجُونَ عَنْ يَدِهِمْ.

\chapter{2}

\par 1 وَأَمَرَ سُلَيْمَانُ بِبِنَاءِ بَيْتٍ لاِسْمِ الرَّبِّ وَبَيْتٍ لِمُلْكِهِ.
\par 2 وَأَحْصَى سُلَيْمَانُ سَبْعِينَ أَلْفَ رَجُلٍ حَمَّالٍ وَثَمَانِينَ أَلْفَ رَجُلٍ نَحَّاتٍ فِي الْجَبَلِ وَوُكَلاَءَ عَلَيْهِمْ ثَلاَثَةَ آلاَفٍ وَسِتَّ مِئَةٍ.
\par 3 وَأَرْسَلَ سُلَيْمَانُ إِلَى حُورَامَ مَلِكِ صُورَ قَائِلاً: [كَمَا فَعَلْتَ مَعَ دَاوُدَ أَبِي إِذْ أَرْسَلْتَ لَهُ أَرْزاً لِيَبْنِيَ لَهُ بَيْتاً يَسْكُنُ فِيهِ
\par 4 فَهَئَنَذَا أَبْنِي بَيْتاً لاِسْمِ الرَّبِّ إِلَهِي لِأُقَدِّسَهُ لَهُ لِأُوقِدَ أَمَامَهُ بَخُوراً عَطِراً وَلِخُبْزِ الْوُجُوهِ الدَّائِمِ وَلِلْمُحْرَقَاتِ صَبَاحاً وَمَسَاءً وَلِلسُّبُوتِ وَالأَهِلَّةِ وَمَوَاسِمِ الرَّبِّ إِلَهِنَا. هَذَا عَلَى إِسْرَائِيلَ إِلَى الأَبَدِ.
\par 5 وَالْبَيْتُ الَّذِي أَنَا بَانِيهِ عَظِيمٌ لأَنَّ إِلَهَنَا أَعْظَمُ مِنْ جَمِيعِ الآلِهَةِ.
\par 6 وَمَنْ يَسْتَطِيعُ أَنْ يَبْنِيَ لَهُ بَيْتاً لأَنَّ السَّمَاوَاتِ وَسَمَاءَ السَّمَاوَاتِ لاَ تَسَعُهُ وَمَنْ أَنَا حَتَّى أَبْنِيَ لَهُ بَيْتاً إِلاَّ لِلإِيقَادِ أَمَامَهُ!
\par 7 فَالآنَ أَرْسِلْ لِي رَجُلاً حَكِيماً فِي صَنَاعَةِ الذَّهَبِ وَالْفِضَّةِ وَالنُّحَاسِ وَالْحَدِيدِ وَالأُرْجُوانِ وَالْقِرْمِزِ وَالأَسْمَانْجُونِيِّ مَاهِراً فِي النَّقْشِ مَعَ الْحُكَمَاءِ الَّذِينَ عِنْدِي فِي يَهُوذَا وَفِي أُورُشَلِيمَ الَّذِينَ أَعَدَّهُمْ دَاوُدُ أَبِي.
\par 8 وَأَرْسِلْ لِي خَشَبَ أَرْزٍ وَسَرْوٍ وَصَنْدَلٍ مِنْ لُبْنَانَ لأَنِّي أَعْلَمُ أَنَّ عَبِيدَكَ مَاهِرُونَ فِي قَطْعِ خَشَبِ لُبْنَانَ. وَهُوَذَا عَبِيدِي مَعَ عَبِيدِكَ.
\par 9 وَلْيُعِدُّوا لِي خَشَباً بِكَثْرَةٍ لأَنَّ الْبَيْتَ الَّذِي أَبْنِيهِ عَظِيمٌ وَعَجِيبٌ.
\par 10 وَهَئَنَذَا أُعْطِي لِلْقَطَّاعِينَ الْقَاطِعِينَ الْخَشَبَ عِشْرِينَ أَلْفَ كُرٍّ مِنَ الْحِنْطَةِ طَعَاماً لِعَبِيدِكَ وَعِشْرِينَ أَلْفَ كُرِّ شَعِيرٍ وَعِشْرِينَ أَلْفَ بَثِّ خَمْرٍ وَعِشْرِينَ أَلْفَ بَثِّ زَيْتٍ].
\par 11 فَأَجَابَ حُورَامُ مَلِكُ صُورَ بِرِسَالَةٍ إِلَى سُلَيْمَانَ: [لأَنَّ الرَّبَّ قَدْ أَحَبَّ شَعْبَهُ جَعَلَكَ عَلَيْهِمْ مَلِكاً].
\par 12 وَقَالَ حُورَامُ: [مُبَارَكٌ الرَّبُّ إِلَهُ إِسْرَائِيلَ الَّذِي صَنَعَ السَّمَاءَ وَالأَرْضَ الَّذِي أَعْطَى دَاوُدَ الْمَلِكَ ابْناً حَكِيماً صَاحِبَ مَعْرِفَةٍ وَفَهْمٍ الَّذِي يَبْنِي بَيْتاً لِلرَّبِّ وَبَيْتاً لِمُلْكِهِ.
\par 13 وَالآنَ أَرْسَلْتُ رَجُلاً حَكِيماً صَاحِبَ فَهْمٍ اسْمُهُ حُورَامَ أَبِي
\par 14 (ابْنَ امْرَأَةٍ مِنْ بَنَاتِ دَانَ وَأَبُوهُ رَجُلٌ صُورِيٌّ) مَاهِرٌ فِي صَنَاعَةِ الذَّهَبِ وَالْفِضَّةِ وَالنُّحَاسِ وَالْحَدِيدِ وَالْحِجَارَةِ وَالْخَشَبِ وَالأُرْجُوانِ وَالأَسْمَانْجُونِيِّ وَالْكَتَّانِ وَالْقِرْمِزِ وَنَقْشِ كُلِّ نَوْعٍ مِنَ النَّقْشِ وَاخْتِرَاعِ كُلِّ اخْتِرَاعٍ يُلْقَى عَلَيْهِ مَعَ حُكَمَائِكَ وَحُكَمَاءِ سَيِّدِي دَاوُدَ أَبِيكَ.
\par 15 وَالآنَ الْحِنْطَةُ وَالشَّعِيرُ وَالزَّيْتُ وَالْخَمْرُ الَّتِي ذَكَرَهَا سَيِّدِي فَلْيُرْسِلْهَا لِعَبِيدِهِ.
\par 16 وَنَحْنُ نَقْطَعُ خَشَباً مِنْ لُبْنَانَ حَسَبَ كُلِّ احْتِيَاجِكَ وَنَأْتِي بِهِ إِلَيْكَ أَرْمَاثاً عَلَى الْبَحْرِ إِلَى يَافَا. وَأَنْتَ تُصْعِدُهُ إِلَى أُورُشَلِيمَ].
\par 17 وَعَدَّ سُلَيْمَانُ جَمِيعَ الرِّجَالِ الأَجْنَبِيِّينَ الَّذِينَ فِي أَرْضِ إِسْرَائِيلَ بَعْدَ الْعَدِّ الَّذِي عَدَّهُمْ إِيَّاهُ دَاوُدُ أَبُوهُ فَوُجِدُوا مِئَةً وَثَلاَثَةً وَخَمْسِينَ أَلْفاً وَسِتَّ مِئَةٍ.
\par 18 فَجَعَلَ مِنْهُمْ سَبْعِينَ أَلْفَ حَمَّالٍ وَثَمَانِينَ أَلْفَ قَطَّاعٍ عَلَى الْجَبَلِ وَثَلاَثَةَ آلاَفٍ وَسِتَّ مِئَةٍ وُكَلاَءَ لِتَشْغِيلِ الشَّعْبِ.

\chapter{3}

\par 1 وَشَرَعَ سُلَيْمَانُ فِي بِنَاءِ بَيْتِ الرَّبِّ فِي أُورُشَلِيمَ فِي جَبَلِ الْمُرِيَّا حَيْثُ تَرَاءَى لِدَاوُدَ أَبِيهِ حَيْثُ هَيَّأَ دَاوُدُ مَكَاناً فِي بَيْدَرِ أُرْنَانَ الْيَبُوسِيِّ.
\par 2 وَشَرَعَ فِي الْبِنَاءِ فِي ثَانِي الشَّهْرِ الثَّانِي فِي السَّنَةِ الرَّابِعَةِ لِمُلْكِهِ.
\par 3 وَهَذِهِ أَسَّسَهَا سُلَيْمَانُ لِبِنَاءِ بَيْتِ اللَّهِ: الطُّولُ (بِـ/لذِّرَاعِ عَلَى الْقِيَاسِ الأَوَّلِ) سِتُّونَ ذِرَاعاً وَالْعَرْضُ عِشْرُون ذِرَاعاً.
\par 4 وَالرِّواقُ الَّذِي قُدَّامَ الطُّولِ حَسَبَ عَرْضِ الْبَيْتِ عِشْرُونَ ذِرَاعاً وَارْتِفَاعُهُ مِئَةٌ وَعِشْرُونَ وَغَشَّاهُ مِنْ دَاخِلٍ بِذَهَبٍ خَالِصٍ.
\par 5 وَالْبَيْتُ الْعَظِيمُ غَشَّاهُ بِخَشَبِ سَرْوٍ غَشَّاهُ بِذَهَبٍ خَالِصٍ وَجَعَلَ عَلَيْهِ نَخِيلاً وَسَلاَسِلَ.
\par 6 وَرَصَّعَ الْبَيْتَ بِحِجَارَةٍ كَرِيمَةٍ لِلْجَمَالِ. وَالذَّهَبُ ذَهَبُ فَرَوَايِمَ.
\par 7 وَغَشَّى الْبَيْتَ: أَخْشَابَهُ وَأَعْتَابَهُ وَحِيطَانَهُ وَمَصَارِيعَهُ بِذَهَبٍ وَنَقَشَ كَرُوبِيمَ عَلَى الْحِيطَانِ.
\par 8 وَعَمِلَ بَيْتَ قُدْسِ الأَقْدَاسِ طُولُهُ حَسَبَ عَرْضِ الْبَيْتِ عِشْرُونَ ذِرَاعاً وَعَرْضُهُ عِشْرُونَ ذِرَاعاً وَغَشَّاهُ بِذَهَبٍ جَيِّدٍ سِتِّ مِئَةِ وَزْنَةٍ.
\par 9 وَكَانَ وَزْنُ الْمَسَامِيرِ خَمْسِينَ شَاقِلاً مِنْ ذَهَبٍ وَغَشَّى الْعَلاَلِيَّ بِذَهَبٍ.
\par 10 وَعَمِلَ فِي بَيْتِ قُدْسِ الأَقْدَاسِ كَرُوبَيْنِ صَنَاعَةَ الصِّيَاغَةِ وَغَشَّاهُمَا بِذَهَبٍ.
\par 11 وَأَجْنِحَةُ الْكَرُوبَيْنِ طُولُهَا عِشْرُونَ ذِرَاعاً الْجَنَاحُ الْوَاحِدُ خَمْسُ أَذْرُعٍ يَمَسُّ حَائِطَ الْبَيْتِ وَالْجَنَاحُ الآخَرُ خَمْسُ أَذْرُعٍ يَمَسُّ جَنَاحَ الْكَرُوبِ الآخَرِ.
\par 12 وَجَنَاحُ الْكَرُوبِ الآخَرِ خَمْسُ أَذْرُعٍ يَمَسُّ حَائِطَ الْبَيْتِ وَالْجَنَاحُ الآخَرُ خَمْسُ أَذْرُعٍ يَتَّصِلُ بِجَنَاحِ الْكَرُوبِ الآخَرِ.
\par 13 وَأَجْنِحَةُ هَذَيْنِ الْكَرُوبَيْنِ مُنْبَسِطَةً عِشْرُونَ ذِرَاعاً وَهُمَا وَاقِفَانِ عَلَى أَرْجُلِهِمَا وَوَجْهُهُمَا إِلَى دَاخِلٍ.
\par 14 وَعَمِلَ الْحِجَابَ مِنْ أَسْمَانْجُونِيٍّ وَأُرْجُوانٍ وَقِرْمِزٍ وَكَتَّانٍ وَجَعَلَ عَلَيْهِ كَرُوبِيمَ.
\par 15 وَعَمِلَ أَمَامَ الْبَيْتِ عَمُودَيْنِ طُولُهُمَا خَمْسٌ وَثَلاَثُونَ ذِرَاعاً وَالتَّاجَانِ اللَّذَانِ عَلَى رَأْسَيْهِمَا خَمْسُ أَذْرُعٍ.
\par 16 وَعَمِلَ سَلاَسِلَ كَمَا فِي الْمِحْرَابِ وَجَعَلَهَا عَلَى رَأْسَيِ الْعَمُودَيْنِ وَعَمِلَ مِئَةَ رُمَّانَةٍ وَجَعَلَهَا فِي السَّلاَسِلِ.
\par 17 وَأَوْقَفَ الْعَمُودَيْنِ أَمَامَ الْهَيْكَلِ وَاحِداً عَنِ الْيَمِينِ وَوَاحِداً عَنِ الْيَسَارِ وَدَعَا اسْمَ الأَيْمَنِ [يَاكِينَ] وَاسْمَ الأَيْسَرِ [بُوعَزَ].

\chapter{4}

\par 1 وَعَمِلَ مَذْبَحَ نُحَاسٍ طُولُهُ عِشْرُونَ ذِرَاعاً وَعَرْضُهُ عِشْرُونَ ذِرَاعاً وَارْتِفَاعُهُ عَشَرُ أَذْرُعٍ.
\par 2 وَعَمِلَ الْبَحْرَ مَسْبُوكاً عَشَرَ أَذْرُعٍ مِنْ شَفَتِهِ إِلَى شَفَتِهِ وَكَانَ مُدَوَّراً مُسْتَدِيراً وَارْتِفَاعُهُ خَمْسُ أَذْرُعٍ وَخَيْطٌ ثَلاَثُونَ ذِرَاعاً يُحِيطُ بِدَائِرِهِ.
\par 3 وَشِبْهُ قُثَّاءٍ تَحْتَهُ مُسْتَدِيراً يُحِيطُ بِهِ عَلَى اسْتِدَارَتِهِ لِلذِّرَاعِ عَشَرٌ تُحِيطُ بِالْبَحْرِ مُسْتَدِيرَةً وَالْقِثَّاءُ صَفَّانِ قَدْ سُبِكَتْ بِسَبْكِهِ
\par 4 كَانَ قَائِماً عَلَى اثْنَيْ عَشَرَ ثَوْراً ثَلاَثَةٌ مُتَّجِهَةٌ إِلَى الشِّمَالِ وَثَلاَثَةٌ مُتَّجِهَةٌ إِلَى الْغَرْبِ وَثَلاَثَةٌ مُتَّجِهَةٌ إِلَى الْجَنُوبِ وَثَلاَثَةٌ مُتَّجِهَةٌ إِلَى الشَّرْقِ وَالْبَحْرُ عَلَيْهَا مِنْ فَوْقُ وَجَمِيعُ مُؤَخَّرَاتِهَا إِلَى دَاخِلٍ.
\par 5 وَسُمْكُهُ شِبْرٌ وَشَفَتُهُ كَعَمَلِ شَفَةِ كَأْسٍ بِزَهْرِ سَوْسَنٍّ. يَأْخُذُ وَيَسَعُ ثَلاَثَةَ آلاَفِ بَثٍّ.
\par 6 وَعَمِلَ عَشَْرَ مَرَاحِضَ وَجَعَلَ خَمْساً عَنِ الْيَمِينِ وَخَمْساً عَنِ الْيَسَارِ لِلاِغْتِسَالِ فِيهَا. كَانُوا يَغْسِلُونَ فِيهَا مَا يُقَرِّبُونَهُ مُحْرَقَةً وَالْبَحْرُ لِيَغْتَسِلَ فِيهِ الْكَهَنَةُ.
\par 7 وَعَمِلَ مَنَائِرَ ذَهَبٍ عَشَراً كَرَسْمِهَا وَجَعَلَهَا فِي الْهَيْكَلِ خَمْساً عَنِ الْيَمِينِ وَخَمْساً عَنِ الْيَسَارِ.
\par 8 وَعَمِلَ عَشَرَ مَوَائِدَ وَوَضَعَهَا فِي الْهَيْكَلِ خَمْساً عَنِ الْيَمِينِ وَخَمْساً عَنِ الْيَسَارِ. وَعَمِلَ مِئَةَ مِنْضَحَةٍ مِنْ ذَهَبٍ.
\par 9 وَعَمِلَ دَارَ الْكَهَنَةِ وَالدَّارَ الْعَظِيمَةَ وَمَصَارِيعَ الدَّارِ وَغَشَّى مَصَارِيعَهَا بِنُحَاسٍ.
\par 10 وَجَعَلَ الْبَحْرَ إِلَى الْجَانِبِ الأَيْمَنِ إِلَى الشَّرْقِ مِنْ جِهَةِ الْجَنُوبِ.
\par 11 وَعَمِلَ حُورَامُ الْقُدُورَ وَالرُّفُوشَ وَالْمَنَاضِحَ وَانْتَهَى حُورَامُ مِنْ عَمَلِ الْعَمَلِ الَّذِي صَنَعَهُ لِلْمَلِكِ سُلَيْمَانَ فِي بَيْتِ اللَّهِ:
\par 12 الْعَمُودَيْنِ وَكُرَتَيِ التَّاجَيْنِ عَلَى رَأْسَيِ الْعَمُودَيْنِ وَالشَّبَكَتَيْنِ لِتَغْطِيَةِ كُرَتَيِ التَّاجَيْنِ اللَّذَيْنِ عَلَى رَأْسَيِ الْعَمُودَيْنِ
\par 13 وَالرُّمَّانَاتِ الأَرْبَعِ مِئَةٍ لِلشَّبَكَتَيْنِ (صَفَّيْ رُمَّانٍ لِلشَّبَكَةِ الْوَاحِدَةِ لِتَغْطِيَةِ كُرَتَيِ التَّاجَيْنِ اللَّذَيْنِ عَلَى الْعَمُودَيْنِ).
\par 14 وَعَمِلَ الْقَوَاعِدَ وَعَمِلَ الْمَرَاحِضَ عَلَى الْقَوَاعِدِ
\par 15 وَالْبَحْرَ الْوَاحِدَ وَالاِثْنَيْ عَشَرَ ثَوْراً تَحْتَهُ
\par 16 وَالْقُدُورَ وَالرُّفُوشَ وَالْمَنَاشِلَ وَكُلَّ آنِيَتِهَا عَمِلَهَا لِلْمَلِكِ سُلَيْمَانَ [حُورَامُ أَبِي] لِبَيْتِ الرَّبِّ مِنْ نُحَاسٍ مَجْلِيٍّ.
\par 17 فِي غَوْرِ الأُرْدُنِّ سَبَكَهَا الْمَلِكُ فِي أَرْضِ الْخَزَفِ بَيْنَ سُكُّوتَ وَصَرَدَةَ.
\par 18 وَعَمِلَ سُلَيْمَانُ كُلَّ هَذِهِ الآنِيَةِ كَثِيرَةً جِدّاً لأَنَّهُ لَمْ يُتَحَقَّقْ وَزْنُ النُّحَاسِ.
\par 19 وَعَمِلَ سُلَيْمَانُ كُلَّ الآنِيَةِ الَّتِي لِبَيْتِ اللَّهِ وَمَذْبَحَ الذَّهَبِ وَالْمَوَائِدَ وَعَلَيْهَا خُبْزُ الْوُجُوهِ
\par 20 وَالْمَنَائِرَ وَسُرُجَهَا لِتَتَّقِدَ حَسَبَ الْمَرْسُومِ أَمَامَ الْمِحْرَابِ مِنْ ذَهَبٍ خَالِصٍ
\par 21 وَالأَزْهَارَ وَالسُّرُجَ وَالْمَلاَقِطَ مِنْ ذَهَبٍ. وَهُوَ ذَهَبٌ كَامِلٌ.
\par 22 وَالْمَقَاصَّ وَالْمَنَاضِحَ وَالصُّحُونَ وَالْمَجَامِرَ مِنْ ذَهَبٍ خَالِصٍ. وَبَابَ الْبَيْتِ وَمَصَارِيعَهُ الدَّاخِلِيَّةَ لِقُدْسِ الأَقْدَاسِ وَمَصَارِيعَ بَيْتِ الْهَيْكَلِ مِنْ ذَهَبٍ.

\chapter{5}

\par 1 وَكَمِلَ جَمِيعُ الْعَمَلِ الَّذِي عَمِلَهُ سُلَيْمَانُ لِبَيْتِ الرَّبِّ. وَأَدْخَلَ سُلَيْمَانُ أَقْدَاسَ دَاوُدَ أَبِيهِ. وَالْفِضَّةُ وَالذَّهَبُ وَجَمِيعُ الآنِيَةِ جَعَلَهَا فِي خَزَائِنِ بَيْتِ اللَّهِ.
\par 2 حِينَئِذٍ جَمَعَ سُلَيْمَانُ شُيُوخَ إِسْرَائِيلَ وَكُلَّ رُؤُوسِ الأَسْبَاطِ رُؤَسَاءَ الآبَاءِ لِبَنِي إِسْرَائِيلَ إِلَى أُورُشَلِيمَ لإِصْعَادِ تَابُوتِ عَهْدِ الرَّبِّ مِنْ مَدِينَةِ دَاوُدَ (هِيَ صِهْيَوْنُ).
\par 3 فَاجْتَمَعَ إِلَى الْمَلِكِ جَمِيعُ رِجَالِ إِسْرَائِيلَ فِي الْعِيدِ الَّذِي فِي الشَّهْرِ السَّابِعِ.
\par 4 وَجَاءَ جَمِيعُ شُيُوخِ إِسْرَائِيلَ. وَحَمَلَ اللاَّوِيُّونَ التَّابُوتَ
\par 5 وَأَصْعَدُوا التَّابُوتَ وَخَيْمَةَ الاِجْتِمَاعِ مَعَ جَمِيعِ آنِيَةِ الْقُدْسِ الَّتِي فِي الْخَيْمَةِ أَصْعَدَهَا الْكَهَنَةُ وَاللاَّوِيُّونَ.
\par 6 وَالْمَلِكُ سُلَيْمَانُ وَكُلُّ جَمَاعَةِ إِسْرَائِيلَ الْمُجْتَمِعِينَ إِلَيْهِ أَمَامَ التَّابُوتِ كَانُوا يَذْبَحُونَ غَنَماً وَبَقَراً مَا لاَ يُحْصَى وَلاَ يُعَدُّ مِنَ الْكَثْرَةِ.
\par 7 وَأَدْخَلَ الْكَهَنَةُ تَابُوتَ عَهْدِ الرَّبِّ إِلَى مَكَانِهِ فِي مِحْرَابِ الْبَيْتِ فِي قُدْسِ الأَقْدَاسِ إِلَى تَحْتِ جَنَاحَيِ الْكَرُوبَيْنِ.
\par 8 وَكَانَ الْكَرُوبَانِ بَاسِطَيْنِ أَجْنِحَتَهُمَا عَلَى مَوْضِعِ التَّابُوتِ. وَظَلَّلَ الْكَرُوبَانِ التَّابُوتَ وَعِصِيَّهُ مِنْ فَوْقُ.
\par 9 وَجَذَبُوا الْعِصِيَّ فَتَرَاءَتْ رُؤُوسُ الْعِصِيِّ مِنَ التَّابُوتِ أَمَامَ الْمِحْرَابِ وَلَمْ تُرَ خَارِجاً وَهِيَ هُنَاكَ إِلَى هَذَا الْيَوْمِ.
\par 10 لَمْ يَكُنْ فِي التَّابُوتِ إِلاَّ اللَّوْحَانِ اللَّذَانِ وَضَعَهُمَا مُوسَى فِي حُورِيبَ حِينَ عَاهَدَ الرَّبُّ بَنِي إِسْرَائِيلَ عِنْدَ خُرُوجِهِمْ مِنْ مِصْرَ.
\par 11 وَكَانَ لَمَّا خَرَجَ الْكَهَنَةُ مِنَ الْقُدْسِ. (لأَنَّ جَمِيعَ الْكَهَنَةِ الْمَوْجُودِينَ تَقَدَّسُوا. لَمْ تُلاَحَظِ الْفِرَقُ).
\par 12 وَاللاَّوِيُّونَ الْمُغَنُّونَ أَجْمَعُونَ: آسَافُ وَهَيْمَانُ وَيَدُوثُونُ وَبَنُوهُمْ وَإِخْوَتُهُمْ لاَبِسِينَ كَتَّاناً بِالصُّنُوجِ وَالرَّبَابِ وَالْعِيدَانِ وَاقِفِينَ شَرْقِيَّ الْمَذْبَحِ وَمَعَهُمْ مِنَ الْكَهَنَةِ مِئَةٌ وَعِشْرُونَ يَنْفُخُونَ فِي الأَبْوَاقِ.
\par 13 وَكَانَ لَمَّا صَوَّتَ الْمُبَوِّقُونَ وَالْمُغَنُّونَ كَوَاحِدٍ صَوْتاً وَاحِداً لِتَسْبِيحِ الرَّبِّ وَحَمْدِهِ وَرَفَعُوا صَوْتاً بِالأَبْوَاقِ وَالصُّنُوجِ وَآلاَتِ الْغِنَاءِ وَالتَّسْبِيحِ لِلرَّبِّ لأَنَّهُ صَالِحٌ لأَنَّ إِلَى الأَبَدِ رَحْمَتَهُ أَنَّ بَيْتَ الرَّبِّ امْتَلَأَ سَحَاباً.
\par 14 وَلَمْ يَسْتَطِعِ الْكَهَنَةُ أَنْ يَقِفُوا لِلْخِدْمَةِ بِسَبَبِ السَّحَابِ لأَنَّ مَجْدَ الرَّبِّ مَلَأَ بَيْتَ اللَّهِ.

\chapter{6}

\par 1 حِينَئِذٍ قَالَ سُلَيْمَانُ: [قَالَ الرَّبُّ إِنَّهُ يَسْكُنُ فِي الضَّبَابِ.
\par 2 وَأَنَا بَنَيْتُ لَكَ بَيْتَ سُكْنَى مَكَاناً لِسُكْنَاكَ إِلَى الأَبَدِ].
\par 3 وَحَوَّلَ الْمَلِكُ وَجْهَهُ وَبَارَكَ كُلَّ جُمْهُورِ إِسْرَائِيلَ وَكُلَّ جُمْهُورِ إِسْرَائِيلَ وَاقِفٌ.
\par 4 وَقَالَ: [مُبَارَكٌ الرَّبُّ إِلَهُ إِسْرَائِيلَ الَّذِي كَلَّمَ بِفَمِهِ دَاوُدَ أَبِي وَأَكْمَلَ بِيَدَيْهِ قَائِلاً:
\par 5 مُنْذُ يَوْمَ أَخْرَجْتُ شَعْبِي مِنْ أَرْضِ مِصْرَ لَمْ أَخْتَرْ مَدِينَةً مِنْ جَمِيعِ أَسْبَاطِ إِسْرَائِيلَ لِبِنَاءِ بَيْتٍ لِيَكُونَ اسْمِي هُنَاكَ وَلاَ اخْتَرْتُ رَجُلاً يَكُونُ رَئِيساً لِشَعْبِي إِسْرَائِيلَ.
\par 6 بَلِ اخْتَرْتُ أُورُشَلِيمَ لِيَكُونَ اسْمِي فِيهَا وَاخْتَرْتُ دَاوُدَ لِيَكُونَ عَلَى شَعْبِي إِسْرَائِيلَ.
\par 7 وَكَانَ فِي قَلْبِ دَاوُدَ أَبِي أَنْ يَبْنِيَ بَيْتاً لاِسْمِ الرَّبِّ إِلَهِ إِسْرَائِيلَ
\par 8 فَقَالَ الرَّبُّ لِدَاوُدَ أَبِي: مِنْ أَجْلِ أَنَّهُ كَانَ فِي قَلْبِكَ أَنْ تَبْنِيَ بَيْتاً لاِسْمِي قَدْ أَحْسَنْتَ بِكَوْنِ ذَلِكَ فِي قَلْبِكَ.
\par 9 إِلاَّ أَنَّكَ أَنْتَ لاَ تَبْنِي الْبَيْتَ بَلِ ابْنُكَ الْخَارِجُ مِنْ صُلْبِكَ هُوَ يَبْنِي الْبَيْتَ لاِسْمِي.
\par 10 وَأَقَامَ الرَّبُّ كَلاَمَهُ الَّذِي تَكَلَّمَ بِهِ وَقَدْ قُمْتُ أَنَا مَكَانَ دَاوُدَ أَبِي وَجَلَسْتُ عَلَى كُرْسِيِّ إِسْرَائِيلَ كَمَا تَكَلَّمَ الرَّبُّ وَبَنَيْتُ الْبَيْتَ لاِسْمِ الرَّبِّ إِلَهِ إِسْرَائِيلَ.
\par 11 وَوَضَعْتُ هُنَاكَ التَّابُوتَ الَّذِي فِيهِ عَهْدُ الرَّبِّ الَّذِي قَطَعَهُ مَعَ بَنِي إِسْرَائِيلَ].
\par 12 وَوَقَفَ أَمَامَ مَذْبَحِ الرَّبِّ تُجَاهَ كُلِّ جَمَاعَةِ إِسْرَائِيلَ وَبَسَطَ يَدَيْهِ.
\par 13 (لأَنَّ سُلَيْمَانَ صَنَعَ مِنْبَراً مِنْ نُحَاسٍ وَجَعَلَهُ فِي وَسَطِ الدَّارِ طُولُهُ خَمْسُ أَذْرُعٍ وَعَرْضُهُ خَمْسُ أَذْرُعٍ وَارْتِفَاعُهُ ثَلاَثُ أَذْرُعٍ وَوَقَفَ عَلَيْهِ ثُمَّ جَثَا عَلَى رُكْبَتَيْهِ تُجَاهَ كُلِّ جَمَاعَةِ إِسْرَائِيلَ وَبَسَطَ يَدَيْهِ نَحْوَ السَّمَاءِ)
\par 14 وَقَالَ: [أَيُّهَا الرَّبُّ إِلَهُ إِسْرَائِيلَ لاَ إِلَهَ مِثْلُكَ فِي السَّمَاءِ وَالأَرْضِ حَافِظُ الْعَهْدِ وَالرَّحْمَةِ لِعَبِيدِكَ السَّائِرِينَ أَمَامَكَ بِكُلِّ قُلُوبِهِمْ.
\par 15 الَّذِي قَدْ حَفِظْتَ لِعَبْدِكَ دَاوُدَ أَبِي مَا كَلَّمْتَهُ بِهِ فَتَكَلَّمْتَ بِفَمِكَ وأَكْمَلْتَ بِيَدِكَ كَهَذَا الْيَوْمِ.
\par 16 وَالآنَ أَيُّهَا الرَّبُّ إِلَهُ إِسْرَائِيلَ احْفَظْ لِعَبْدِكَ دَاوُدَ أَبِي مَا كَلَّمْتَهُ بِهِ قَائِلاً: لاَ يُعْدَمُ لَكَ أَمَامِي رَجُلٌ يَجْلِسُ عَلَى كُرْسِيِّ إِسْرَائِيلَ إِنْ حَفِظَ بَنُوكَ طُرُقَهُمْ حَتَّى يَسِيرُوا فِي شَرِيعَتِي كَمَا سِرْتَ أَنْتَ أَمَامِي.
\par 17 وَالآنَ أَيُّهَا الرَّبُّ إِلَهُ إِسْرَائِيلَ فَلْيَتَحَقَّقْ كَلاَمُكَ الَّذِي كَلَّمْتَ بِهِ عَبْدَكَ دَاوُدَ.
\par 18 لأَنَّهُ هَلْ يَسْكُنُ اللَّهُ حَقّاً مَعَ الإِنْسَانِ عَلَى الأَرْضِ؟ هُوَذَا السَّمَاوَاتُ وَسَمَاءُ السَّمَاوَاتِ لاَ تَسَعُكَ فَكَمْ بِالأَقَلِّ هَذَا الْبَيْتُ الَّذِي بَنَيْتُ!
\par 19 فَالْتَفِتْ إِلَى صَلاَةِ عَبْدِكَ وَإِلَى تَضَرُّعِهِ أَيُّهَا الرَّبُّ إِلَهِي وَاسْمَعِ الصُّرَاخَ وَالصَّلاَةَ الَّتِي يُصَلِّيهَا عَبْدُكَ أَمَامَكَ.
\par 20 لِتَكُونَ عَيْنَاكَ مَفْتُوحَتَيْنِ عَلَى هَذَا الْبَيْتِ نَهَاراً وَلَيْلاً عَلَى الْمَوْضِعِ الَّذِي قُلْتَ إِنَّكَ تَضَعُ اسْمَكَ فِيهِ لِتَسْمَعَ الصَّلاَةَ الَّتِي يُصَلِّيهَا عَبْدُكَ فِي هَذَا الْمَوْضِعِ
\par 21 وَاسْمَعْ تَضَرُّعَاتِ عَبْدِكَ وَشَعْبِكَ إِسْرَائِيلَ الَّذِينَ يُصَلُّونَ فِي هَذَا الْمَوْضِعِ وَاسْمَعْ أَنْتَ مِنْ مَوْضِعِ سُكْنَاكَ مِنَ السَّمَاءِ وَإِذَا سَمِعْتَ فَاغْفِرْ.
\par 22 إِنْ أَخْطَأَ أَحَدٌ إِلَى صَاحِبِهِ وَوُضِعَ عَلَيْهِ حَلْفٌ لِيُحَلِّفَهُ وَجَاءَ الْحَلْفُ أَمَامَ مَذْبَحِكَ فِي هَذَا الْبَيْتِ
\par 23 فَاسْمَعْ أَنْتَ مِنَ السَّمَاءِ وَاعْمَلْ وَاقْضِ بَيْنَ عَبِيدِكَ إِذْ تُعَاقِبُ الْمُذْنِبَ فَتَجْعَلُ طَرِيقَهُ عَلَى رَأْسِهِ وَتُبَرِّرُ الْبَارَّ إِذْ تُعْطِيهِ حَسَبَ بِرِّهِ.
\par 24 وَإِنِ انْكَسَرَ شَعْبُكَ إِسْرَائِيلُ أَمَامَ الْعَدُوِّ لأَنَّهُمْ أَخْطَأُوا إِلَيْكَ ثُمَّ رَجَعُوا وَاعْتَرَفُوا بِاسْمِكَ وَصَلُّوا وَتَضَرَّعُوا أَمَامَكَ نَحْوَ هَذَا الْبَيْتِ
\par 25 فَاسْمَعْ أَنْتَ مِنَ السَّمَاءِ وَاغْفِرْ خَطِيَّةَ شَعْبِكَ إِسْرَائِيلَ وَأَرْجِعْهُمْ إِلَى الأَرْضِ الَّتِي أَعْطَيْتَهَا لَهُمْ وَلآبَائِهِمْ.
\par 26 [إِذَا أُغْلِقَتِ السَّمَاءُ وَلَمْ يَكُنْ مَطَرٌ لأَنَّهُمْ أَخْطَأُوا إِلَيْكَ ثُمَّ صَلُّوا فِي هَذَا الْمَكَانِ وَاعْتَرَفُوا بِاسْمِكَ وَرَجَعُوا عَنْ خَطِيَّتِهِمْ لأَنَّكَ ضَايَقْتَهُمْ
\par 27 فَاسْمَعْ أَنْتَ مِنَ السَّمَاءِ وَاغْفِرْ خَطِيَّةَ عَبِيدِكَ وَشَعْبِكَ إِسْرَائِيلَ فَتُعَلِّمَهُمُ الطَّرِيقَ الصَّالِحَ الَّذِي يَسْلُكُونَ فِيهِ وَأَعْطِ مَطَراً عَلَى أَرْضِكَ الَّتِي أَعْطَيْتَهَا لِشَعْبِكَ مِيرَاثاً.
\par 28 إِذَا صَارَ فِي الأَرْضِ جُوعٌ إِذَا صَارَ وَبَأٌ أَوْ لَفْحٌ أَوْ يَرَقَانٌ أَوْ جَرَادٌ أَوْ جَرْدَمٌ أَوْ إِذَا حَاصَرَهُمْ أَعْدَاؤُهُمْ فِي أَرْضِ مُدُنِهِمْ فِي كُلِّ ضَرْبَةٍ وَكُلِّ مَرَضٍ
\par 29 فَكُلُّ صَلاَةٍ وَكُلُّ تَضَرُّعٍ تَكُونُ مِنْ أَيِّ إِنْسَانٍ كَانَ أَوْ مِنْ كُلِّ شَعْبِكَ إِسْرَائِيلَ الَّذِينَ يَعْرِفُونَ كُلُّ وَاحِدٍ ضَرْبَتَهُ وَوَجَعَهُ فَيَبْسُطُ يَدَيْهِ نَحْوَ هَذَا الْبَيْتِ
\par 30 فَاسْمَعْ أَنْتَ مِنَ السَّمَاءِ مَكَانِ سُكْنَاكَ وَاغْفِرْ وَأَعْطِ كُلَّ إِنْسَانٍ حَسَبَ كُلِّ طُرُقِهِ كَمَا تَعْرِفُ قَلْبَهُ. لأَنَّكَ أَنْتَ وَحْدَكَ تَعْرِفُ قُلُوبَ بَنِي الْبَشَرِ.
\par 31 لِيَخَافُوكَ وَيَسِيرُوا فِي طُرُقِكَ كُلَّ الأَيَّامِ الَّتِي يَحْيُونَ فِيهَا عَلَى وَجْهِ الأَرْضِ الَّتِي أَعْطَيْتَ لآبَائِنَا.
\par 32 وَكَذَلِكَ الأَجْنَبِيُّ الَّذِي لَيْسَ هُوَ مِنْ شَعْبِكَ إِسْرَائِيلَ وَقَدْ جَاءَ مِنْ أَرْضٍ بَعِيدَةٍ مِنْ أَجْلِ اسْمِكَ الْعَظِيمِ وَيَدِكَ الْقَوِيَّةِ وَذِرَاعِكَ الْمَمْدُودَةِ فَمَتَى جَاءُوا وَصَلُّوا فِي هَذَا الْبَيْتِ
\par 33 فَاسْمَعْ أَنْتَ مِنَ السَّمَاءِ مَكَانِ سُكْنَاكَ وَافْعَلْ حَسَبَ كُلِّ مَا يَدْعُوكَ بِهِ الأَجْنَبِيُّ لِيَعْلَمَ كُلُّ شُعُوبِ الأَرْضِ اسْمَكَ فَيَخَافُوكَ كَشَعْبِكَ إِسْرَائِيلَ وَلِيَعْلَمُوا أَنَّ اسْمَكَ قَدْ دُعِيَ عَلَى هَذَا الْبَيْتِ الَّذِي بَنَيْتُ.
\par 34 [إِذَا خَرَجَ شَعْبُكَ لِمُحَارَبَةِ أَعْدَائِهِ فِي الطَّرِيقِ الَّذِي تُرْسِلُهُمْ فِيهِ وَصَلُّوا إِلَيْكَ نَحْوَ هَذِهِ الْمَدِينَةِ الَّتِي اخْتَرْتَهَا وَالْبَيْتِ الَّذِي بَنَيْتُ لاِسْمِكَ
\par 35 فَاسْمَعْ مِنَ السَّمَاءِ صَلاَتَهُمْ وَتَضَرُّعَهُمْ وَاقْضِ قَضَاءَهُمْ.
\par 36 إِذَا أَخْطَأُوا إِلَيْكَ (لأَنَّهُ لَيْسَ إِنْسَانٌ لاَ يُخْطِئُ) وَغَضِبْتَ عَلَيْهِمْ وَدَفَعْتَهُمْ أَمَامَ الْعَدُوِّ وَسَبَاهُمْ سَابُوهُمْ إِلَى أَرْضٍ بَعِيدَةٍ أَوْ قَرِيبَةٍ
\par 37 فَإِذَا رَدُّوا إِلَى قُلُوبِهِمْ فِي الأَرْضِ الَّتِي يُسْبَوْنَ إِلَيْهَا وَرَجَعُوا وَتَضَرَّعُوا إِلَيْكَ فِي أَرْضِ سَبْيِهِمْ قَائِلِينَ: قَدْ أَخْطَأْنَا وَعَوَّجْنَا وَأَذْنَبْنَا
\par 38 وَرَجَعُوا إِلَيْكَ مِنْ كُلِّ قُلُوبِهِمْ وَمِنْ كُلِّ أَنْفُسِهِمْ فِي أَرْضِ سَبْيِهِمِ الَّتِي سَبُوهُمْ إِلَيْهَا وَصَلُّوا نَحْوَ أَرْضِهِمِ الَّتِي أَعْطَيْتَهَا لآبَائِهِمْ وَالْمَدِينَةِ الَّتِي اخْتَرْتَ وَالْبَيْتِ الَّذِي بَنَيْتُ لاِسْمِكَ
\par 39 فَاسْمَعْ مِنَ السَّمَاءِ مِنْ مَكَانِ سُكْنَاكَ صَلاَتَهُمْ وَتَضَرُّعَاتِهِمْ وَاقْضِ قَضَاءَهُمْ وَاغْفِرْ لِشَعْبِكَ مَا أَخْطَأُوا بِهِ إِلَيْكَ.
\par 40 الآنَ يَا إِلَهِي لِتَكُنْ عَيْنَاكَ مَفْتُوحَتَيْنِ وَأُذُنَاكَ مُصْغِيَتَيْنِ لِصَلاَةِ هَذَا الْمَكَانِ.
\par 41 وَالآنَ قُمْ أَيُّهَا الرَّبُّ الإِلَهُ إِلَى رَاحَتِكَ أَنْتَ وَتَابُوتُ عِزِّكَ. كَهَنَتُكَ أَيُّهَا الرَّبُّ الإِلَهُ يَلْبِسُونَ الْخَلاَصَ وَأَتْقِيَاؤُكَ يَبْتَهِجُونَ بِالْخَيْرِ.
\par 42 أَيُّهَا الرَّبُّ الإِلَهُ لاَ تَرُدَّ وَجْهَ مَسِيحِكَ. اذْكُرْ مَرَاحِمَ دَاوُدَ عَبْدِكَ].

\chapter{7}

\par 1 وَلَمَّا انْتَهَى سُلَيْمَانُ مِنَ الصَّلاَةِ نَزَلَتِ النَّارُ مِنَ السَّمَاءِ وَأَكَلَتِ الْمُحْرَقَةَ وَالذَّبَائِحَ وَمَلَأَ مَجْدُ الرَّبِّ الْبَيْتَ.
\par 2 وَلَمْ يَسْتَطِعِ الْكَهَنَةُ أَنْ يَدْخُلُوا بَيْتَ الرَّبِّ لأَنَّ مَجْدَ الرَّبِّ مَلَأَ بَيْتَ الرَّبِّ.
\par 3 وَكَانَ جَمِيعُ بَنِي إِسْرَائِيلَ يَنْظُرُونَ عِنْدَ نُزُولِ النَّارِ وَمَجْدِ الرَّبِّ عَلَى الْبَيْتِ وَخَرُّوا عَلَى وُجُوهِهِمْ إِلَى الأَرْضِ عَلَى الْبَلاَطِ الْمُجَزَّعِ وَسَجَدُوا وَحَمَدُوا الرَّبَّ لأَنَّهُ صَالِحٌ وَإِلَى الأَبَدِ رَحْمَتُهُ.
\par 4 ثُمَّ إِنَّ الْمَلِكَ وَكُلَّ الشَّعْبِ ذَبَحُوا ذَبَائِحَ أَمَامَ الرَّبِّ.
\par 5 وَذَبَحَ الْمَلِكُ سُلَيْمَانُ ذَبَائِحَ مِنَ الْبَقَرِ: اثْنَيْنِ وَعِشْرِينَ أَلْفاً وَمِنَ الْغَنَمِ مِئَةً وَعِشْرِينَ أَلْفاً وَدَشَّنَ الْمَلِكُ وَكُلُّ الشَّعْبِ بَيْتَ اللَّهِ.
\par 6 وَكَانَ الْكَهَنَةُ وَاقِفِينَ عَلَى مَحَارِسِهِمْ وَاللاَّوِيُّونَ بِآلاَتِ غِنَاءِ الرَّبِّ الَّتِي عَمِلَهَا دَاوُدُ الْمَلِكُ لأَجْلِ حَمْدِ الرَّبِّ [لأَنَّ إِلَى الأَبَدِ رَحْمَتَهُ] حِينَ سَبَّحَ دَاوُدُ بِهَا وَالْكَهَنَةُ يَنْفُخُونَ فِي الأَبْوَاقِ مُقَابِلَهُمْ وَكُلُّ إِسْرَائِيلَ وَاقِفٌ.
\par 7 وَقَدَّسَ سُلَيْمَانُ وَسَطَ الدَّارِ الَّتِي أَمَامَ بَيْتِ الرَّبِّ لأَنَّهُ قَرَّبَ هُنَاكَ الْمُحْرَقَاتِ وَشَحْمَ ذَبَائِحِ السَّلاَمَةِ لأَنَّ مَذْبَحَ النُّحَاسِ الَّذِي عَمِلَهُ سُلَيْمَانُ لَمْ يَكْفِ لأَنْ يَسَعَ الْمُحْرَقَاتِ وَالتَّقْدِمَاتِ وَالشَّحْمَ.
\par 8 وَعَيَّدَ سُلَيْمَانُ الْعِيدَ فِي ذَلِكَ الْوَقْتِ سَبْعَةَ أَيَّامٍ وَكُلُّ إِسْرَائِيلَ مَعَهُ وَجُمْهُورٌ عَظِيمٌ جِدّاً مِنْ مَدْخَلِ حَمَاةَ إِلَى وَادِي مِصْرَ.
\par 9 وَعَمِلُوا فِي الْيَوْمِ الثَّامِنِ اعْتِكَافاً لأَنَّهُمْ عَمِلُوا تَدْشِينَ الْمَذْبَحِ سَبْعَةَ أَيَّامٍ وَالْعِيدَ سَبْعَةَ أَيَّامٍ.
\par 10 وَفِي الْيَوْمِ الثَّالِثِ وَالْعِشْرِينَ مِنَ الشَّهْرِ السَّابِعِ صَرَفَ الشَّعْبَ إِلَى خِيَامِهِمْ فَرِحِينَ وَطَيِّبِي الْقُلُوبِ لأَجْلِ الْخَيْرِ الَّذِي عَمِلَهُ الرَّبُّ لِدَاوُدَ وَلِسُلَيْمَانَ وَلإِسْرَائِيلَ شَعْبِهِ.
\par 11 وَأَكْمَلَ سُلَيْمَانُ بَيْتَ الرَّبِّ وَبَيْتَ الْمَلِكِ. وَكُلَّ مَا خَطَرَ بِبَالِ سُلَيْمَانَ أَنْ يَعْمَلَهُ فِي بَيْتِ الرَّبِّ وَفِي بَيْتِهِ نَجَحَ فِيهِ.
\par 12 وَتَرَاءَى الرَّبُّ لِسُلَيْمَانَ لَيْلاً وَقَالَ لَهُ: [قَدْ سَمِعْتُ صَلاَتَكَ وَاخْتَرْتُ هَذَا الْمَكَانَ لِي بَيْتَ ذَبِيحَةٍ.
\par 13 إِنْ أَغْلَقْتُ السَّمَاءَ وَلَمْ يَكُنْ مَطَرٌ وَإِنْ أَمَرْتُ الْجَرَادَ أَنْ يَأْكُلَ الأَرْضَ وَإِنْ أَرْسَلْتُ وَبَأً عَلَى شَعْبِي
\par 14 فَإِذَا تَوَاضَعَ شَعْبِي الَّذِينَ دُعِيَ اسْمِي عَلَيْهِمْ وَصَلُّوا وَطَلَبُوا وَجْهِي وَرَجَعُوا عَنْ طُرُقِهِمِ الرَّدِيئَةِ فَإِنِّي أَسْمَعُ مِنَ السَّمَاءِ وَأَغْفِرُ خَطِيَّتَهُمْ وَأُبْرِئُ أَرْضَهُمْ.
\par 15 اَلآنَ عَيْنَايَ تَكُونَانِ مَفْتُوحَتَيْنِ وَأُذُنَايَ مُصْغِيَتَيْنِ إِلَى صَلاَةِ هَذَا الْمَكَانِ.
\par 16 وَالآنَ قَدِ اخْتَرْتُ وَقَدَّسْتُ هَذَا الْبَيْتَ لِيَكُونَ اسْمِي فِيهِ إِلَى الأَبَدِ وَتَكُونُ عَيْنَايَ وَقَلْبِي هُنَاكَ كُلَّ الأَيَّامِ.
\par 17 وَأَنْتَ إِنْ سَلَكْتَ أَمَامِي كَمَا سَلَكَ دَاوُدُ أَبُوكَ وَعَمِلْتَ حَسَبَ كُلِّ مَا أَمَرْتُكَ بِهِ وَحَفِظْتَ فَرَائِضِي وَأَحْكَامِي
\par 18 فَإِنِّي أُثَبِّتُ كُرْسِيَّ مُلْكِكَ كَمَا عَاهَدْتُ دَاوُدَ أَبَاكَ قَائِلاً: لاَ يُعْدَمُ لَكَ رَجُلٌ يَتَسَلَّطُ عَلَى إِسْرَائِيلَ.
\par 19 وَلَكِنْ إِنِ انْقَلَبْتُمْ وَتَرَكْتُمْ فَرَائِضِي وَوَصَايَايَ الَّتِي جَعَلْتُهَا أَمَامَكُمْ وَذَهَبْتُمْ وَعَبَدْتُمْ آلِهَةً أُخْرَى وَسَجَدْتُمْ لَهَا
\par 20 فَإِنِّي أَقْلَعُهُمْ مِنْ أَرْضِي الَّتِي أَعْطَيْتُهُمْ إِيَّاهَا وَهَذَا الْبَيْتُ الَّذِي قَدَّسْتُهُ لاِسْمِي أَطْرَحُهُ مِنْ أَمَامِي وَأَجْعَلُهُ مَثَلاً وَهُزْأَةً فِي جَمِيعِ الشُّعُوبِ.
\par 21 وَهَذَا الْبَيْتُ الَّذِي كَانَ مُرْتَفِعاً كُلُّ مَنْ يَمُرُّ بِهِ يَتَعَجَّبُ وَيَقُولُ: لِمَاذَا عَمِلَ الرَّبُّ هَكَذَا لِهَذِهِ الأَرْضِ وَلِهَذَا الْبَيْتِ؟
\par 22 فَيَقُولُونَ: مِنْ أَجْلِ أَنَّهُمْ تَرَكُوا الرَّبَّ إِلَهَ آبَائِهِمِ الَّذِي أَخْرَجَهُمْ مِنْ أَرْضِ مِصْرَ وَتَمَسَّكُوا بِآلِهَةٍ أُخْرَى وَسَجَدُوا لَهَا وَعَبَدُوهَا لِذَلِكَ جَلَبَ عَلَيْهِمْ كُلَّ هَذَا الشَّرِّ].

\chapter{8}

\par 1 وَبَعْدَ نِهَايَةِ عِشْرِينَ سَنَةً بَعْدَ أَنْ بَنَى سُلَيْمَانُ بَيْتَ الرَّبِّ وَبَيْتَهُ
\par 2 بَنَى سُلَيْمَانُ الْمُدُنَ الَّتِي أَعْطَاهَا حُورَامُ لِسُلَيْمَانَ وَأَسْكَنَ فِيهَا بَنِي إِسْرَائِيلَ.
\par 3 وَذَهَبَ سُلَيْمَانُ إِلَى حَمَاةِ صُوبَةَ وَقَوِيَ عَلَيْهَا.
\par 4 وَبَنَى تَدْمُرَ فِي الْبَرِّيَّةِ وَجَمِيعَ مُدُنِ الْمَخَازِنِ الَّتِي بَنَاهَا فِي حَمَاةَ.
\par 5 وَبَنَى بَيْتَ حُورُونَ الْعُلْيَا وَبَيْتَ حُورُونَ السُّفْلَى مُدُناً حَصِينَةً بِأَسْوَارٍ وَأَبْوَابٍ وَعَوَارِضَ.
\par 6 وَبَعْلَةَ وَكُلَّ مُدُنِ الْمَخَازِنِ الَّتِي كَانَتْ لِسُلَيْمَانَ وَجَمِيعَ مُدُنِ الْمَرْكَبَاتِ وَمُدُنِ الْفُرْسَانِ وَكُلَّ مَرْغُوبِ سُلَيْمَانَ الَّذِي رَغِبَ أَنْ يَبْنِيَهُ فِي أُورُشَلِيمَ وَفِي لُبْنَانَ وَفِي كُلِّ أَرْضِ سُلْطَانِهِ.
\par 7 أَمَّا جَمِيعُ الشَّعْبِ الْبَاقِي مِنَ الْحِثِّيِّينَ وَالأَمُورِيِّينَ وَالْفِرِزِّيِّينَ وَالْحِوِّيِّينَ وَالْيَبُوسِيِّينَ الَّذِينَ لَيْسُوا مِنْ إِسْرَائِيلَ
\par 8 مِنْ بَيْنِهِمِ الَّذِينَ بَقُوا بَعْدَهُمْ فِي الأَرْضِ الَّذِينَ لَمْ يُفْنِهِمْ بَنُو إِسْرَائِيلَ فَجَعَلَ سُلَيْمَانُ عَلَيْهِمْ سُخْرَةً إِلَى هَذَا الْيَوْمِ.
\par 9 وَأَمَّا بَنُو إِسْرَائِيلَ فَلَمْ يَجْعَلْ سُلَيْمَانُ مِنْهُمْ عَبِيداً لِشُغْلِهِ لأَنَّهُمْ رِجَالُ الْقِتَالِ وَرُؤَسَاءُ قُوَّادِهِ وَرُؤَسَاءُ مَرْكَبَاتِهِ وَفُرْسَانِهِ.
\par 10 وَهَؤُلاَءِ رُؤَسَاءُ الْمُوَكَّلِينَ الَّذِينَ لِلْمَلِكِ سُلَيْمَانَ مِئَتَانِ وَخَمْسُونَ الْمُتَسَلِّطُونَ عَلَى الشَّعْبِ.
\par 11 وَأَمَّا بِنْتُ فِرْعَوْنَ فَأَصْعَدَهَا سُلَيْمَانُ مِنْ مَدِينَةِ دَاوُدَ إِلَى الْبَيْتِ الَّذِي بَنَاهُ لَهَا لأَنَّهُ قَالَ: [لاَ تَسْكُنِ امْرَأَةٌ لِي فِي بَيْتِ دَاوُدَ مَلِكِ إِسْرَائِيلَ لأَنَّ الأَمَاكِنَ الَّتِي دَخَلَ إِلَيْهَا تَابُوتُ الرَّبِّ إِنَّمَا هِيَ مُقَدَّسَةٌ].
\par 12 حِينَئِذٍ أَصْعَدَ سُلَيْمَانُ مُحْرَقَاتٍ لِلرَّبِّ عَلَى مَذْبَحِ الرَّبِّ الَّذِي بَنَاهُ قُدَّامَ الرِّواقِ.
\par 13 أَمْرَ كُلِّ يَوْمٍ بِيَوْمِهِ مِنَ الْمُحْرَقَاتِ حَسَبَ وَصِيَّةِ مُوسَى فِي السُّبُوتِ وَالأَهِلَّةِ وَالْمَوَاسِمِ ثَلاَثَ مَرَّاتٍ فِي السَّنَةِ فِي عِيدِ الْفَطِيرِ وَعِيدِ الأَسَابِيعِ وَعِيدِ الْمَظَالِّ.
\par 14 وَأَوْقَفَ حَسَبَ قَضَاءِ دَاوُدَ أَبِيهِ فِرَقَ الْكَهَنَةِ عَلَى خِدْمَتِهِمْ وَاللاَّوِيِّينَ عَلَى حِرَاسَاتِهِمْ (لِلتَّسْبِيحِ وَالْخِدْمَةِ أَمَامَ الْكَهَنَةِ) عَمَلِ كُلِّ يَوْمٍ بِيَوْمِهِ وَالْبَوَّابِينَ حَسَبَ فِرَقِهِمْ عَلَى كُلِّ بَابٍ. لأَنَّهُ هَكَذَا هِيَ وَصِيَّةُ دَاوُدَ رَجُلِ اللَّهِ.
\par 15 وَلَمْ يَحِيدُوا عَنْ وَصِيَّةِ الْمَلِكِ عَلَى الْكَهَنَةِ وَاللاَّوِيِّينَ فِي كُلِّ أَمْرٍ وَفِي الْخَزَائِنِ.
\par 16 فَتَهَيَّأَ كُلُّ عَمَلِ سُلَيْمَانَ إِلَى يَوْمِ تَأْسِيسِ بَيْتِ الرَّبِّ وَإِلَى نِهَايَتِهِ. فَكَمُلَ بَيْتُ الرَّبِّ.
\par 17 حِينَئِذٍ ذَهَبَ سُلَيْمَانُ إِلَى عِصْيُونَ جَابِرَ وَإِلَى أَيْلَةَ عَلَى شَاطِئِ الْبَحْرِ فِي أَرْضِ أَدُومَ.
\par 18 وَأَرْسَلَ لَهُ حُورَامُ بِيَدِ عَبِيدِهِ سُفُناً وَعَبِيداً يَعْرِفُونَ الْبَحْرَ فَأَتُوا مَعَ عَبِيدِ سُلَيْمَانَ إِلَى أُوفِيرَ وَأَخَذُوا مِنْ هُنَاكَ أَرْبَعَ مِئَةٍ وَخَمْسِينَ وَزْنَةَ ذَهَبٍ وَأَتُوا بِهَا إِلَى الْمَلِكِ سُلَيْمَانَ.

\chapter{9}

\par 1 وَسَمِعَتْ مَلِكَةُ سَبَا بِخَبَرِ سُلَيْمَانَ فَأَتَتْ لِتَمْتَحِنَ سُلَيْمَانَ بِمَسَائِلَ إِلَى أُورُشَلِيمَ بِمَوْكِبٍ عَظِيمٍ جِدّاً وَجِمَالٍ حَامِلَةٍ أَطْيَاباً وَذَهَباً بِكَثْرَةٍ وَحِجَارَةً كَرِيمَةً فَأَتَتْ إِلَى سُلَيْمَانَ وَكَلَّمَتْهُ عَنْ كُلِّ مَا فِي قَلْبِهَا.
\par 2 فَأَخْبَرَهَا سُلَيْمَانُ بِكُلِّ كَلاَمِهَا. وَلَمْ يُخْفَ عَنْ سُلَيْمَانَ أَمْرٌ إِلاَّ وَأَخْبَرَهَا بِهِ.
\par 3 فَلَمَّا رَأَتْ مَلِكَةُ سَبَا حِكْمَةَ سُلَيْمَانَ وَالْبَيْتَ الَّذِي بَنَاهُ
\par 4 وَطَعَامَ مَائِدَتِهِ وَمَجْلِسَ عَبِيدِهِ وَمَوْقِفَ خُدَّامِهِ وَمَلاَبِسَهُمْ وَسُقَاتَهُ وَمَلاَبِسَهُمْ وَمُحْرَقَاتِهِ الَّتِي كَانَ يُصْعِدُهَا فِي بَيْتِ الرَّبِّ لَمْ تَبْقَ فِيهَا رُوحٌ بَعْدُ.
\par 5 فَقَالَتْ لِلْمَلِكِ: [صَحِيحٌ الْخَبَرُ الَّذِي سَمِعْتُهُ فِي أَرْضِي عَنْ أُمُورِكَ وَعَنْ حِكْمَتِكَ!
\par 6 وَلَمْ أُصَدِّقْ كَلاَمَهُمْ حَتَّى جِئْتُ وَأَبْصَرَتْ عَيْنَايَ فَهُوَذَا لَمْ أُخْبَرْ بِنِصْفِ كَثْرَةِ حِكْمَتِكَ. زِدْتَ عَلَى الْخَبَرِ الَّذِي سَمِعْتُهُ.
\par 7 فَطُوبَى لِرِجَالِكَ وَطُوبَى لِعَبِيدِكَ هَؤُلاَءِ الْوَاقِفِينَ أَمَامَكَ دَائِماً وَالسَّامِعِينَ حِكْمَتَكَ.
\par 8 لِيَكُنْ مُبَارَكاً الرَّبُّ إِلَهُكَ الَّذِي سُرَّ بِكَ وَجَعَلَكَ عَلَى كُرْسِيِّهِ مَلِكاً لِلرَّبِّ إِلَهِكَ. لأَنَّ إِلَهَكَ أَحَبَّ إِسْرَائِيلَ لِيُثْبِتَهُ إِلَى الأَبَدِ قَدْ جَعَلَكَ عَلَيْهِمْ مَلِكاً لِتُجْرِيَ حُكْماً وَعَدْلاً].
\par 9 وَأَهْدَتْ لِلْمَلِكِ مِئَةً وَعِشْرِينَ وَزْنَةَ ذَهَبٍ وَأَطْيَاباً كَثِيرَةً جِدّاً وَحِجَارَةً كَرِيمَةً وَلَمْ يَكُنْ مِثْلُ ذَلِكَ الطِّيبِ الَّذِي أَهْدَتْهُ مَلِكَةُ سَبَا لِلْمَلِكِ سُلَيْمَانَ.
\par 10 وَكَذَا عَبِيدُ حُورَامَ وَعَبِيدُ سُلَيْمَانَ الَّذِينَ جَلَبُوا ذَهَباً مِنْ أُوفِيرَ أَتُوا بِخَشَبِ الصَّنْدَلِ وَحِجَارَةٍ كَرِيمَةٍ.
\par 11 وَعَمِلَ الْمَلِكُ خَشَبَ الصَّنْدَلِ دَرَجاً لِبَيْتِ الرَّبِّ وَبَيْتِ الْمَلِكِ وَأَعْوَاداً وَرَبَاباً وَلَمْ يُرَ مِثْلُهَا قَبْلُ فِي أَرْضِ يَهُوذَا.
\par 12 وَأَعْطَى الْمَلِكُ سُلَيْمَانُ مَلِكَةَ سَبَا كُلَّ مُشْتَهَاهَا الَّذِي طَلَبَتْ فَضْلاً عَمَّا أَتَتْ بِهِ إِلَى الْمَلِكِ. فَانْصَرَفَتْ وَذَهَبَتْ إِلَى أَرْضِهَا هِيَ وَعَبِيدُهَا.
\par 13 وَكَانَ وَزْنُ الذَّهَبِ الَّذِي جَاءَ سُلَيْمَانَ فِي سَنَةٍ وَاحِدَةٍ سِتَّ مِئَةٍ وَسِتّاً وَسِتِّينَ وَزْنَةَ ذَهَبٍ
\par 14 فَضْلاً عَنِ الَّذِي جَاءَ بِهِ التُّجَّارُ وَالْمُسْتَبْضِعُونَ. وَكُلُّ مُلُوكِ الْعَرَبِ وَوُلاَةُ الأَرْضِ كَانُوا يَأْتُونَ بِذَهَبٍ وَفِضَّةٍ إِلَى سُلَيْمَانَ.
\par 15 وَعَمِلَ الْمَلِكُ سُلَيْمَانُ مِئَتَيْ تُرْسٍ مِنْ ذَهَبٍ مُطَرَّقٍ خَصَّ التُّرْسَ الْوَاحِدَ سِتُّ مِئَةِ شَاقِلٍ مِنَ الذَّهَبِ الْمُطَرَّقِ
\par 16 وَثَلاَثَ مِئَةِ مِجَنٍّ مِنْ ذَهَبٍ مُطَرَّقٍ خَصَّ الْمِجَنَّ الْوَاحِدَ ثَلاَثُ مِئَةِ شَاقِلٍ مِنَ الذَّهَبِ. وَجَعَلَهَا الْمَلِكُ فِي بَيْتِ وَعْرِ لُبْنَانَ.
\par 17 وَعَمِلَ الْمَلِكُ كُرْسِيّاً عَظِيماً مِنْ عَاجٍ وَغَشَّاهُ بِذَهَبٍ خَالِصٍ.
\par 18 وَلِلْكُرْسِيِّ سِتُّ دَرَجَاتٍ. وَلِلْكُرْسِيِّ مَوْطِئٌ مِنْ ذَهَبٍ كُلُّهَا مُتَّصِلَةٌ وَيَدَانِ مِنْ هُنَا وَمِنْ هُنَاكَ عَلَى مَكَانِ الْجُلُوسِ وَأَسَدَانِ وَاقِفَانِ بِجَانِبِ الْيَدَيْنِ.
\par 19 وَاثْنَا عَشَرَ أَسَداً وَاقِفَةٌ هُنَاكَ عَلَى الدَّرَجَاتِ السِّتِّ مِنْ هُنَا وَمِنْ هُنَاكَ. لَمْ يُعْمَلْ مِثْلُهُ فِي جَمِيعِ الْمَمَالِكِ.
\par 20 وَجَمِيعُ آنِيَةِ شُرْبِ الْمَلِكِ سُلَيْمَانَ مِنْ ذَهَبٍ وَجَمِيعُ آنِيَةِ بَيْتِ وَعْرِ لُبْنَانَ مِنْ ذَهَبٍ خَالِصٍ. لَمْ تُحْسَبِ الْفِضَّةُ شَيْئاً فِي أَيَّامِ سُلَيْمَانَ
\par 21 لأَنَّ سُفُنَ الْمَلِكِ كَانَتْ تَسِيرُ إِلَى تَرْشِيشَ مَعَ عَبِيدِ حُورَامَ وَكَانَتْ سُفُنُ تَرْشِيشَ تَأْتِي مَرَّةً فِي كُلِّ ثَلاَثِ سِنِينَ حَامِلَةً ذَهَباً وَفِضَّةً وَعَاجاً وَقُرُوداً وَطَوَاوِيسَ.
\par 22 فَتَعَظَّمَ الْمَلِكُ سُلَيْمَانُ عَلَى كُلِّ مُلُوكِ الأَرْضِ فِي الْغِنَى وَالْحِكْمَةِ.
\par 23 وَكَانَ جَمِيعُ مُلُوكِ الأَرْضِ يَلْتَمِسُونَ وَجْهَ سُلَيْمَانَ لِيَسْمَعُوا حِكْمَتَهُ الَّتِي جَعَلَهَا اللَّهُ فِي قَلْبِهِ.
\par 24 وَكَانُوا يَأْتُونَ كُلُّ وَاحِدٍ بِهَدِيَّتِهِ بِآنِيَةِ فِضَّةٍ وَآنِيَةِ ذَهَبٍ وَحُلَلٍ وَسِلاَحٍ وَأَطْيَابٍ وَخَيْلٍ وَبِغَالٍ سَنَةً فَسَنَةً.
\par 25 وَكَانَ لِسُلَيْمَانَ أَرْبَعَةُ آلاَفِ مِذْوَدِ خَيْلٍ وَمَرْكَبَاتٍ وَاثْنَا عَشَرَ أَلْفَ فَارِسٍ فَجَعَلَهَا فِي مُدُنِ الْمَرْكَبَاتِ وَمَعَ الْمَلِكِ فِي أُورُشَلِيمَ.
\par 26 وَكَانَ مُتَسَلِّطاً عَلَى جَمِيعِ الْمُلُوكِ مِنَ النَّهْرِ إِلَى أَرْضِ الْفِلِسْطِينِيِّينَ وَإِلَى تُخُومِ مِصْرَ.
\par 27 وَجَعَلَ الْمَلِكُ الْفِضَّةَ فِي أُورُشَلِيمَ مِثْلَ الْحِجَارَةِ وَجَعَلَ الأَرْزَ مِثْلَ الْجُمَّيْزِ الَّذِي فِي السَّهْلِ فِي الْكَثْرَةِ.
\par 28 وَكَانَ مُخْرَجُ خَيْلِ سُلَيْمَانَ مِنْ مِصْرَ وَمِنْ جَمِيعِ الأَرَاضِي.
\par 29 وَبَقِيَّةُ أُمُورِ سُلَيْمَانَ الأُولَى وَالأَخِيرَةِ مَكْتُوبَةٌ فِي أَخْبَارِ نَاثَانَ النَّبِيِّ وَفِي نُبُوَّةِ أَخِيَّا الشِّيلُونِيِّ وَفِي رُؤَى يَعْدُو الرَّائِي عَلَى يَرُبْعَامَ بْنِ نَبَاطَ.
\par 30 وَمَلَكَ سُلَيْمَانُ فِي أُورُشَلِيمَ عَلَى كُلِّ إِسْرَائِيلَ أَرْبَعِينَ سَنَةً.
\par 31 ثُمَّ اضْطَجَعَ سُلَيْمَانُ مَعَ آبَائِهِ فَدَفَنُوهُ فِي مَدِينَةِ دَاوُدَ أَبِيهِ. وَمَلَكَ رَحُبْعَامُ ابْنُهُ عِوَضاً عَنْهُ.

\chapter{10}

\par 1 وَذَهَبَ رَحُبْعَامُ إِلَى شَكِيمَ لأَنَّهُ جَاءَ إِلَى شَكِيمَ كُلُّ إِسْرَائِيلَ لِيُمَلِّكُوهُ.
\par 2 وَلَمَّا سَمِعَ يَرُبْعَامُ بْنُ نَبَاطَ (وَهُوَ فِي مِصْرَ حَيْثُ هَرَبَ مِنْ وَجْهِ سُلَيْمَانَ الْمَلِكِ) رَجَعَ يَرُبْعَامُ مِنْ مِصْرَ.
\par 3 فَأَرْسَلُوا وَدَعُوهُ فَأَتَى يَرُبْعَامُ وَكُلُّ إِسْرَائِيلَ وَقَالُوا لِرَحُبْعَامَ:
\par 4 [إِنَّ أَبَاكَ قَسَّى نِيرَنَا فَالآنَ خَفِّفْ مِنْ عُبُودِيَّةِ أَبِيكَ الْقَاسِيَةِ وَمِنْ نِيرِهِ الثَّقِيلِ الَّذِي جَعَلَهُ عَلَيْنَا فَنَخْدِمَكَ].
\par 5 فَقَالَ لَهُمُ: [ارْجِعُوا إِلَيَّ بَعْدَ ثَلاَثَةِ أَيَّامٍ]. فَذَهَبَ الشَّعْبُ.
\par 6 فَاسْتَشَارَ الْمَلِكُ رَحُبْعَامُ الشُّيُوخَ الَّذِينَ كَانُوا يَقِفُونَ أَمَامَ سُلَيْمَانَ أَبِيهِ وَهُوَ حَيٌّ قَائِلاً: [كَيْفَ تُشِيرُونَ أَنْ أَرُدَّ جَوَاباً عَلَى هَذَا الشَّعْبِ؟]
\par 7 فَقَالُوا: [إِنْ كُنْتَ صَالِحاً نَحْوَ هَذَا الشَّعْبِ وَأَرْضَيْتَهُمْ وَكَلَّمْتَهُمْ كَلاَماً حَسَناً يَكُونُونَ لَكَ عَبِيداً كُلَّ الأَيَّامِ].
\par 8 فَتَرَكَ مَشُورَةَ الشُّيُوخِ الَّتِي أَشَارُوا بِهَا عَلَيْهِ وَاسْتَشَارَ الأَحْدَاثَ الَّذِينَ نَشَأُوا مَعَهُ وَوَقَفُوا أَمَامَهُ
\par 9 وَسَأَلَهُمْ: [بِمَاذَا تُشِيرُونَ أَنْتُمْ فَنَرُدَّ جَوَاباً عَلَى هَذَا الشَّعْبِ الَّذِينَ كَلَّمُونِي قَائِلِينَ: خَفِّفْ مِنَ النِّيرِ الَّذِي جَعَلَهُ عَلَيْنَا أَبُوكَ؟]
\par 10 فَأَجَابَ الأَحْدَاثُ الَّذِينَ نَشَأُوا مَعَهُ: [هَكَذَا تَقُولُ لِلشَّعْبِ الَّذِينَ قَالُوا إِنَّ أَبَاكَ ثَقَّلَ نِيرَنَا وَأَمَّا أَنْتَ فَخَفِّفْ عَنَّا: إِنَّ خِنْصَرِي أَغْلَظُ مِنْ وَسْطِ أَبِي.
\par 11 وَالآنَ أَبِي حَمَّلَكُمْ نِيراً ثَقِيلاً وَأَنَا أَزِيدُ عَلَى نِيرِكُمْ. أَبِي أَدَّبَكُمْ بِالسِّيَاطِ وَأَمَّا أَنَا فَبِالْعَقَارِبِ].
\par 12 فَجَاءَ يَرُبْعَامُ وَجَمِيعُ الشَّعْبِ إِلَى رَحُبْعَامَ فِي الْيَوْمِ الثَّالِثِ كَمَا أَمَرَ الْمَلِكُ: [ارْجِعُوا إِلَيَّ فِي الْيَوْمِ الثَّالِثِ].
\par 13 فَأَجَابَهُمُ الْمَلِكُ بِقَسَاوَةٍ وَتَرَكَ الْمَلِكُ رَحُبْعَامُ مَشُورَةَ الشُّيُوخِ
\par 14 وَكَلَّمَهُمْ حَسَبَ مَشُورَةِ الأَحْدَاثِ قَائِلاً: [أَبِي ثَقَّلَ نِيرَكُمْ وَأَنَا أَزِيدُ عَلَيْهِ. أَبِي أَدَّبَكُمْ بِالسِّيَاطِ وَأَمَّا أَنَا فَبِالْعَقَارِبِ].
\par 15 وَلَمْ يَسْمَعِ الْمَلِكُ لِلشَّعْبِ لأَنَّ السَّبَبَ كَانَ مِنْ قِبَلِ اللَّهِ لِيُقِيمَ الرَّبُّ كَلاَمَهُ الَّذِي تَكَلَّمَ بِهِ عَنْ يَدِ أَخِيَّا الشِّيلُونِيِّ إِلَى يَرُبْعَامَ بْنِ نَبَاطَ.
\par 16 فَلَمَّا رَأَى كُلُّ إِسْرَائِيلَ أَنَّ الْمَلِكَ لَمْ يَسْمَعْ لَهُمْ قَالَ الشَّعْبُ لِلْمَلِكِ: [أَيُّ قِسْمٍ لَنَا فِي دَاوُدَ! وَلاَ نَصِيبَ لَنَا فِي ابْنِ يَسَّى. كُلُّ وَاحِدٍ إِلَى خَيْمَتِهِ يَا إِسْرَائِيلُ. الآنَ انْظُرْ إِلَى بَيْتِكَ يَا دَاوُدُ!] وَذَهَبَ كُلُّ إِسْرَائِيلَ إِلَى خِيَامِهِمْ.
\par 17 وَأَمَّا بَنُو إِسْرَائِيلَ السَّاكِنُونَ فِي مُدُنِ يَهُوذَا فَمَلَكَ عَلَيْهِمْ رَحُبْعَامُ.
\par 18 ثُمَّ أَرْسَلَ الْمَلِكُ رَحُبْعَامُ هَدُورَامَ الَّذِي عَلَى التَّسْخِيرِ فَرَجَمَهُ بَنُو إِسْرَائِيلَ بِالْحِجَارَةِ فَمَاتَ. فَبَادَرَ الْمَلِكُ رَحُبْعَامُ وَصَعِدَ إِلَى الْمَرْكَبَةِ لِيَهْرُبَ إِلَى أُورُشَلِيمَ
\par 19 فَعَصَى إِسْرَائِيلُ بَيْتَ دَاوُدَ إِلَى هَذَا الْيَوْمِ.اَلأَصْحَاحُ الْحَادِي عَشَرَ

\chapter{11}

\par 1 وَلَمَّا جَاءَ رَحُبْعَامُ إِلَى أُورُشَلِيمَ جَمَعَ مِنْ بَيْتِ يَهُوذَا وَبِنْيَامِينَ مِئَةً وَثَمَانِينَ أَلْفَ مُخْتَارٍ مُحَارِبٍ لِيُحَارِبَ إِسْرَائِيلَ لِيَرُدَّ الْمُلْكَ إِلَى رَحُبْعَامَ.
\par 2 وَكَانَ كَلاَمُ الرَّبِّ إِلَى شَمَعْيَا رَجُلِ اللَّهِ
\par 3 [قُلْ لِرَحُبْعَامَ بْنِ سُلَيْمَانَ مَلِكِ يَهُوذَا وَكُلِّ إِسْرَائِيلَ فِي يَهُوذَا وَبِنْيَامِينَ:
\par 4 هَكَذَا قَالَ الرَّبُّ: لاَ تَصْعَدُوا وَلاَ تُحَارِبُوا إِخْوَتَكُمْ. ارْجِعُوا كُلُّ وَاحِدٍ إِلَى بَيْتِهِ لأَنَّهُ مِنْ قِبَلِي صَارَ هَذَا الأَمْرُ] فَسَمِعُوا لِكَلاَمِ الرَّبِّ وَرَجَعُوا عَنِ الذَّهَابِ ضِدَّ يَرُبْعَامَ.
\par 5 وَأَقَامَ رَحُبْعَامُ فِي أُورُشَلِيمَ وَبَنَى مُدُناً لِلْحِصَارِ فِي يَهُوذَا.
\par 6 فَبَنَى بَيْتَ لَحْمٍ وَعِيطَامَ وَتَقُوعَ
\par 7 وَبَيْتَ صُورَ وَسُوكُوَ وَعَدُلاَّمَ
\par 8 وَجَتَّ وَمَرِيشَةَ وَزِيفَ
\par 9 وَأَدُورَايِمَ وَلَخِيشَ وَعَزِيقَةَ
\par 10 وَصَرْعَةَ وَأَيَّلُونَ وَحَبْرُونَ الَّتِي فِي يَهُوذَا وَبِنْيَامِينَ مُدُناً حَصِينَةً.
\par 11 وَشَدَّدَ الْحُصُونَ وَجَعَلَ فِيهَا قُوَّاداً وَخَزَائِنَ مَأْكَلٍ وَزَيْتٍ وَخَمْرٍ
\par 12 وَأَتْرَاساً فِي كُلِّ مَدِينَةٍ وَرِمَاحاً وَشَدَّدَهَا كَثِيراً جِدّاً وَكَانَ لَهُ يَهُوذَا وَبَنْيَامِينُ.
\par 13 وَالْكَهَنَةُ وَاللاَّوِيُّونَ الَّذِينَ فِي كُلِّ إِسْرَائِيلَ مَثَلُوا بَيْنَ يَدَيْهِ مِنْ جَمِيعِ تُخُومِهِمْ
\par 14 لأَنَّ اللاَّوِيِّينَ تَرَكُوا مَرَاعِيَهُمْ وَأَمْلاَكَهُمْ وَانْطَلَقُوا إِلَى يَهُوذَا وَأُورُشَلِيمَ لأَنَّ يَرُبْعَامَ وَبَنِيهِ رَفَضُوهُمْ مِنْ أَنْ يَكْهَنُوا لِلرَّبِّ
\par 15 وَأَقَامَ لِنَفْسِهِ كَهَنَةً لِلْمُرْتَفَعَاتِ وَلِلتُّيُوسِ وَلِلْعُجُولِ الَّتِي عَمِلَ.
\par 16 وَبَعْدَهُمْ جَاءَ إِلَى أُورُشَلِيمَ مِنْ جَمِيعِ أَسْبَاطِ إِسْرَائِيلَ الَّذِينَ وَجَّهُوا قُلُوبَهُمْ إِلَى طَلَبِ الرَّبِّ إِلَهِ إِسْرَائِيلَ لِيَذْبَحُوا لِلرَّبِّ إِلَهِ آبَائِهِمْ.
\par 17 وَشَدَّدُوا مَمْلَكَةَ يَهُوذَا وَقَوُّوا رَحُبْعَامَ بْنَ سُلَيْمَانَ ثَلاَثَ سِنِينَ لأَنَّهُمْ سَارُوا فِي طَرِيقِ دَاوُدَ وَسُلَيْمَانَ ثَلاَثَ سِنِينَ.
\par 18 وَاتَّخَذَ رَحُبْعَامُ لِنَفْسِهِ امْرَأَةً: مَحْلَةَ بِنْتَ يَرِيمُوثَ بْنِ دَاوُدَ وَأَبِيجَايِلَ بِنْتَ أَلِيآبَ بْنِ يَسَّى.
\par 19 فَوَلَدَتْ لَهُ بَنِينَ: يَعُوشَ وَشَمَرْيَا وَزَاهَمَ
\par 20 ثُمَّ بَعْدَهَا أَخَذَ مَعْكَةَ بِنْتَ أَبْشَالُومَ فَوَلَدَتْ لَهُ أَبِيَّا وَعَتَّايَ وَزِيزَا وَشَلُومِيثَ.
\par 21 وَأَحَبَّ رَحُبْعَامُ مَعْكَةَ بِنْتَ أَبْشَالُومَ أَكْثَرَ مِنْ جَمِيعِ نِسَائِهِ وَسَرَارِيهِ لأَنَّهُ اتَّخَذَ ثَمَانِيَ عَشَرَةَ امْرَأَةً وَسِتِّينَ سُرِّيَّةً وَوَلَدَ ثَمَانِيَةً وَعِشْرِينَ ابْناً وَسِتِّينَ ابْنَةً.
\par 22 وَأَقَامَ رَحُبْعَامُ أَبِيَّا ابْنَ مَعْكَةَ رَأْساً وَقَائِداً بَيْنَ إِخْوَتِهِ لِيُمَلِّكَهُ.
\par 23 وَكَانَ فَهِيماً وَفَرَّقَ مِنْ كُلِّ بَنِيهِ فِي جَمِيعِ أَرَاضِي يَهُوذَا وَبِنْيَامِينَ فِي كُلِّ الْمُدُنِ الْحَصِينَةِ وَأَعْطَاهُمْ زَاداً بِكَثْرَةٍ. وَطَلَبَ نِسَاءً كَثِيرَةً.

\chapter{12}

\par 1 وَلَمَّا تَثَبَّتَتْ مَمْلَكَةُ رَحُبْعَامَ وَتَشَدَّدَتْ تَرَكَ شَرِيعَةَ الرَّبِّ هُوَ وَكُلُّ إِسْرَائِيلَ مَعَهُ.
\par 2 وَفِي السَّنَةِ الْخَامِسَةِ لِلْمَلِكِ رَحُبْعَامَ صَعِدَ شِيشَقُ مَلِكُ مِصْرَ عَلَى أُورُشَلِيمَ - لأَنَّهُمْ خَانُوا الرَّبَّ -
\par 3 بِأَلْفٍ وَمِئَتَيْ مَرْكَبَةٍ وَسِتِّينَ أَلْفَ فَارِسٍ وَلَمْ يَكُنْ عَدَدٌ لِلشَّعْبِ الَّذِينَ جَاءُوا مَعَهُ مِنْ مِصْرَ: لُوبِيِّينَ وَسُكِّيِّينَ وَكُوشِيِّينَ.
\par 4 وَأَخَذَ الْمُدُنَ الْحَصِينَةَ الَّتِي لِيَهُوذَا وَأَتَى إِلَى أُورُشَلِيمَ.
\par 5 فَجَاءَ شَمَعْيَا النَّبِيُّ إِلَى رَحُبْعَامَ وَرُؤَسَاءِ يَهُوذَا الَّذِينَ اجْتَمَعُوا فِي أُورُشَلِيمَ مِنْ وَجْهِ شِيشَقَ وَقَالَ لَهُمْ: [هَكَذَا قَالَ الرَّبُّ: أَنْتُمْ تَرَكْتُمُونِي وَأَنَا أَيْضاً تَرَكْتُكُمْ لِيَدِ شِيشَقَ].
\par 6 فَتَذَلَّلَ رُؤَسَاءُ إِسْرَائِيلَ وَالْمَلِكُ وَقَالُوا: [بَارٌّ هُوَ الرَّبُّ].
\par 7 فَلَمَّا رَأَى الرَّبُّ أَنَّهُمْ تَذَلَّلُوا كَانَ كَلاَمُ الرَّبِّ إِلَى شَمَعْيَا: [قَدْ تَذَلَّلُوا فَلاَ أُهْلِكُهُمْ بَلْ أُعْطِيهِمْ قَلِيلاً مِنَ النَّجَاةِ وَلاَ يَنْصَبُّ غَضَبِي عَلَى أُورُشَلِيمَ بِيَدِ شِيشَقَ
\par 8 لَكِنَّهُمْ يَكُونُونَ لَهُ عَبِيداً وَيَعْلَمُونَ خِدْمَتِي وَخِدْمَةَ مَمَالِكِ الأَرَاضِي].
\par 9 فَصَعِدَ شِيشَقُ مَلِكُ مِصْرَ عَلَى أُورُشَلِيمَ وَأَخَذَ خَزَائِنَ بَيْتِ الرَّبِّ وَخَزَائِنَ بَيْتِ الْمَلِكِ أَخَذَ الْجَمِيعَ وَأَخَذَ أَتْرَاسَ الذَّهَبِ الَّتِي عَمِلَهَا سُلَيْمَانُ.
\par 10 فَعَمِلَ الْمَلِكُ رَحُبْعَامُ عِوَضاً عَنْهَا أَتْرَاسَ نُحَاسٍ وَسَلَّمَهَا إِلَى أَيْدِي رُؤَسَاءِ السُّعَاةِ الْحَافِظِينَ بَابَ بَيْتِ الْمَلِكِ.
\par 11 وَكَانَ إِذَا دَخَلَ الْمَلِكُ بَيْتَ الرَّبِّ يَأْتِي السُّعَاةُ وَيَحْمِلُونَهَا ثُمَّ يُرْجِعُونَهَا إِلَى غُرْفَةِ السُّعَاةِ.
\par 12 وَلَمَّا تَذَلَّلَ ارْتَدَّ عَنْهُ غَضَبُ الرَّبِّ فَلَمْ يُهْلِكْهُ تَمَاماً. وَكَذَلِكَ كَانَ فِي يَهُوذَا أُمُورٌ حَسَنَةٌ.
\par 13 فَتَشَدَّدَ الْمَلِكُ رَحُبْعَامُ فِي أُورُشَلِيمَ وَمَلَكَ لأَنَّ رَحُبْعَامَ كَانَ ابْنَ إِحْدَى وَأَرْبَعِينَ سَنَةً حِينَ مَلَكَ وَمَلَكَ سَبْعَ عَشْرَةَ سَنَةً فِي أُورُشَلِيمَ الْمَدِينَةِ الَّتِي اخْتَارَهَا الرَّبُّ لِيَضَعَ اسْمَهُ فِيهَا دُونَ جَمِيعِ أَسْبَاطِ إِسْرَائِيلَ. وَاسْمُ أُمِّهِ نَعْمَةُ الْعَمُّونِيَّةُ.
\par 14 وَعَمِلَ الشَّرَّ لأَنَّهُ لَمْ يُهَيِّئْ قَلْبَهُ لِطَلَبِ الرَّبِّ.
\par 15 وَأُمُورُ رَحُبْعَامَ الأُولَى وَالأَخِيرَةُ مَكْتُوبَةٌ فِي أَخْبَارِ شَمَعْيَا النَّبِيِّ وَعِدُّو الرَّائِي عَنِ الاِنْتِسَابِ. وَكَانَتْ حُرُوبٌ بَيْنَ رَحُبْعَامَ وَيَرُبْعَامَ كُلَّ الأَيَّامِ.
\par 16 ثُمَّ اضْطَجَعَ رَحُبْعَامُ مَعَ آبَائِهِ وَدُفِنَ فِي مَدِينَةِ دَاوُدَ وَمَلَكَ أَبِيَّا ابْنُهُ عِوَضاً عَنْهُ.

\chapter{13}

\par 1 فِي السَّنَةِ الثَّامِنَةَ عَشَرَةَ لِلْمَلِكِ يَرُبْعَامَ مَلَكَ أَبِيَّا عَلَى يَهُوذَا.
\par 2 مَلَكَ ثَلاَثَ سِنِينَ فِي أُورُشَلِيمَ. وَاسْمُ أُمِّهِ مِيخَايَا بِنْتُ أُورِيئِيلَ مِنْ جَبْعَةَ. وَكَانَتْ حَرْبٌ بَيْنَ أَبِيَّا وَيَرُبْعَامَ.
\par 3 وَابْتَدَأَ أَبِيَّا فِي الْحَرْبِ بِجَيْشٍ مِنْ جَبَابِرَةِ الْقِتَالِ أَرْبَعِ مِئَةِ أَلْفِ رَجُلٍ مُخْتَارٍ وَيَرُبْعَامُ اصْطَفَّ لِمُحَارَبَتِهِ بِثَمَانِ مِئَةِ أَلْفِ رَجُلٍ مُخْتَارٍ جَبَابِرَةِ بَأْسٍ.
\par 4 وَقَامَ أَبِيَّا عَلَى جَبَلِ صَمَارَايِمَ الَّذِي فِي جَبَلِ أَفْرَايِمَ وَقَالَ: [اسْمَعُونِي يَا يَرُبْعَامُ وَكُلَّ إِسْرَائِيلَ.
\par 5 أَمَا لَكُمْ أَنْ تَعْرِفُوا أَنَّ الرَّبَّ إِلَهَ إِسْرَائِيلَ أَعْطَى الْمُلْكَ عَلَى إِسْرَائِيلَ لِدَاوُدَ إِلَى الأَبَدِ وَلِبَنِيهِ بِعَهْدِ مِلْحٍ؟
\par 6 فَقَامَ يَرُبْعَامُ بْنُ نَبَاطَ عَبْدُ سُلَيْمَانَ بْنِ دَاوُدَ وَعَصَى سَيِّدَهُ.
\par 7 فَاجْتَمَعَ إِلَيْهِ رِجَالٌ بَطَّالُونَ بَنُو بَلِيَّعَالَ وَتَشَدَّدُوا عَلَى رَحُبْعَامَ بْنِ سُلَيْمَانَ وَكَانَ رَحُبْعَامُ فَتىً رَقِيقَ الْقَلْبِ فَلَمْ يَثْبُتْ أَمَامَهُمْ.
\par 8 وَالآنَ أَنْتُمْ تَقُولُونَ إِنَّكُمْ تَثْبُتُونَ أَمَامَ مَمْلَكَةِ الرَّبِّ بِيَدِ بَنِي دَاوُدَ وَأَنْتُمْ جُمْهُورٌ كَثِيرٌ وَمَعَكُمْ عُجُولُ ذَهَبٍ قَدْ عَمِلَهَا يَرُبْعَامُ لَكُمْ آلِهَةً.
\par 9 أَمَا طَرَدْتُمْ كَهَنَةَ الرَّبِّ بَنِي هَارُونَ وَاللاَّوِيِّينَ وَعَمِلْتُمْ لأَنْفُسِكُمْ كَهَنَةً كَشُعُوبِ الأَرَاضِي كُلُّ مَنْ أَتَى لِيَمْلَأَ يَدَهُ بِثَوْرٍ ابْنِ بَقَرٍ وَسَبْعَةِ كِبَاشٍ صَارَ كَاهِناً لِلَّذِينَ لَيْسُوا آلِهَةً!
\par 10 وَأَمَّا نَحْنُ فَالرَّبُّ هُوَ إِلَهُنَا وَلَمْ نَتْرُكْهُ. وَالْكَهَنَةُ الْخَادِمُونَ الرَّبَّ هُمْ بَنُو هَارُونَ وَاللاَّوِيُّونَ فِي الْعَمَلِ
\par 11 وَيُوقِدُونَ لِلرَّبِّ مُحْرَقَاتٍ كُلَّ صَبَاحٍ وَمَسَاءٍ وَبَخُورُ أَطْيَابٍ وَخُبْزُ الْوُجُوهِ عَلَى الْمَائِدَةِ الطَّاهِرَةِ وَمَنَارَةُ الذَّهَبِ وَسُرُجُهَا لِلإِيقَادِ كُلَّ مَسَاءٍ لأَنَّنَا نَحْنُ حَارِسُونَ حِرَاسَةَ الرَّبِّ إِلَهِنَا. وَأَمَّا أَنْتُمْ فَقَدْ تَرَكْتُمُوهُ.
\par 12 وَهُوَذَا مَعَنَا اللَّهُ رَئِيساً وَكَهَنَتُهُ وَأَبْوَاقُ الْهُتَافِ لِلْهُتَافِ عَلَيْكُمْ. فَيَا بَنِي إِسْرَائِيلَ لاَ تُحَارِبُوا الرَّبَّ إِلَهَ آبَائِكُمْ لأَنَّكُمْ لاَ تُفْلِحُونَ].
\par 13 وَلَكِنْ يَرُبْعَامُ جَعَلَ الْكَمِينَ يَدُورُ لِيَأْتِيَ مِنْ خَلْفِهِمْ. فَكَانُوا أَمَامَ يَهُوذَا وَالْكَمِينُ خَلْفَهُمْ.
\par 14 فَالْتَفَتَ يَهُوذَا وَإِذَا الْحَرْبُ عَلَيْهِمْ مِنْ قُدَّامٍ وَمِنْ خَلْفٍ. فَصَرَخُوا إِلَى الرَّبِّ وَبَوَّقَ الْكَهَنَةُ بِالأَبْوَاقِ
\par 15 وَهَتَفَ رِجَالُ يَهُوذَا. وَلَمَّا هَتَفَ رِجَالُ يَهُوذَا ضَرَبَ اللَّهُ يَرُبْعَامَ وَكُلَّ إِسْرَائِيلَ أَمَامَ أَبِيَّا وَيَهُوذَا.
\par 16 فَانْهَزَمَ بَنُو إِسْرَائِيلَ مِنْ أَمَامِ يَهُوذَا وَدَفَعَهُمُ اللَّهُ لِيَدِهِمْ.
\par 17 وَضَرَبَهُمْ أَبِيَّا وَقَوْمُهُ ضَرْبَةً عَظِيمَةً فَسَقَطَ قَتْلَى مِنْ إِسْرَائِيلَ خَمْسُ مِئَةِ أَلْفِ رَجُلٍ مُخْتَارٍ.
\par 18 فَذَلَّ بَنُو إِسْرَائِيلَ فِي ذَلِكَ الْوَقْتِ وَتَشَجَّعَ بَنُو يَهُوذَا لأَنَّهُمُ اتَّكَلُوا عَلَى الرَّبِّ إِلَهِ آبَائِهِمْ.
\par 19 وَطَارَدَ أَبِيَّا يَرُبْعَامَ وَأَخَذَ مِنْهُ مُدُناً: بَيْتَ إِيلَ وَقُرَاهَا وَيَشَانَةَ وَقُرَاهَا وَعَفْرُونَ وَقُرَاهَا.
\par 20 وَلَمْ يَقْوَ يَرُبْعَامُ بَعْدُ فِي أَيَّامِ أَبِيَّا فَضَرَبَهُ الرَّبُّ وَمَاتَ.
\par 21 وَتَشَدَّدَ أَبِيَّا وَاتَّخَذَ لِنَفْسِهِ أَرْبَعَ عَشَرَةَ امْرَأَةً وَوَلَدَ اثْنَيْنِ وَعِشْرِينَ ابْناً وَسِتَّ عَشَرَةَ بِنْتاً.
\par 22 وَبَقِيَّةُ أُمُورِ أَبِيَّا وَطُرُقُهُ وَأَقْوَالُهُ مَكْتُوبَةٌ فِي مِدْرَسِ النَّبِيِّ عِدُّو.

\chapter{14}

\par 1 ثُمَّ اضْطَجَعَ أَبِيَّا مَعَ آبَائِهِ فَدَفَنُوهُ فِي مَدِينَةِ دَاوُدَ وَمَلَكَ آسَا ابْنُهُ عِوَضاً عَنْهُ. فِي أَيَّامِهِ اسْتَرَاحَتِ الأَرْضُ عَشَرَ سِنِينَ.
\par 2 وَعَمِلَ آسَا مَا هُوَ صَالِحٌ وَمُسْتَقِيمٌ فِي عَيْنَيِ الرَّبِّ إِلَهِهِ.
\par 3 وَنَزَعَ الْمَذَابِحَ الْغَرِيبَةَ وَالْمُرْتَفَعَاتِ وَكَسَّرَ التَّمَاثِيلَ وَقَطَّعَ السَّوَارِيَ
\par 4 وَقَالَ لِيَهُوذَا أَنْ يَطْلُبُوا الرَّبَّ إِلَهَ آبَائِهِمْ وَأَنْ يَعْمَلُوا حَسَبَ الشَّرِيعَةِ وَالْوَصِيَّةِ.
\par 5 وَنَزَعَ مِنْ كُلِّ مُدُنِ يَهُوذَا الْمُرْتَفَعَاتِ وَتَمَاثِيلَ الشَّمْسِ وَاسْتَرَاحَتِ الْمَمْلَكَةُ أَمَامَهُ.
\par 6 وَبَنَى مُدُناً حَصِينَةً فِي يَهُوذَا لأَنَّ الأَرْضَ اسْتَرَاحَتْ وَلَمْ تَكُنْ عَلَيْهِ حَرْبٌ فِي تِلْكَ السِّنِينَ لأَنَّ الرَّبَّ أَرَاحَهُ.
\par 7 وَقَالَ لِيَهُوذَا: [لِنَبْنِ هَذِهِ الْمُدُنَ وَنُحَوِّطْهَا بِأَسْوَارٍ وَأَبْرَاجٍ وَأَبْوَابٍ وَعَوَارِضَ مَا دَامَتِ الأَرْضُ أَمَامَنَا لأَنَّنَا قَدْ طَلَبْنَا الرَّبَّ إِلَهَنَا. طَلَبْنَاهُ فَأَرَاحَنَا مِنْ كُلِّ جِهَةٍ]. فَبَنُوا وَنَجَحُوا.
\par 8 وَكَانَ لآسَا جَيْشٌ يَحْمِلُونَ أَتْرَاساً وَرِمَاحاً مِنْ يَهُوذَا ثَلاَثُ مِئَةِ أَلْفٍ وَمِنْ بَنْيَامِينَ مِنَ الَّذِينَ يَحْمِلُونَ الأَتْرَاسَ وَيَشُدُّونَ الْقِسِيَّ مِئَتَانِ وَثَمَانُونَ أَلْفاً. كُلُّ هَؤُلاَءِ جَبَابِرَةُ بَأْسٍ.
\par 9 فَخَرَجَ إِلَيْهِمْ زَارَحُ الْكُوشِيُّ بِجَيْشٍ أَلْفِ أَلْفٍ وَبِمَرْكَبَاتٍ ثَلاَثِ مِئَةٍ وَأَتَى إِلَى مَرِيشَةَ.
\par 10 وَخَرَجَ آسَا لِلِقَائِهِ وَاصْطَفُّوا لِلْقِتَالِ فِي وَادِي صَفَاتَةَ عِنْدَ مَرِيشَةَ.
\par 11 وَدَعَا آسَا الرَّبَّ إِلَهَهُ: [أَيُّهَا الرَّبُّ لَيْسَ فَرْقاً عِنْدَكَ أَنْ تُسَاعِدَ الْكَثِيرِينَ وَمَنْ لَيْسَ لَهُمْ قُوَّةٌ. فَسَاعِدْنَا أَيُّهَا الرَّبُّ إِلَهُنَا لأَنَّنَا عَلَيْكَ اتَّكَلْنَا وَبِاسْمِكَ قَدُمْنَا عَلَى هَذَا الْجَيْشِ. أَيُّهَا الرَّبُّ أَنْتَ إِلَهُنَا. لاَ يَقْوَ عَلَيْكَ إِنْسَانٌ].
\par 12 فَضَرَبَ الرَّبُّ الْكُوشِيِّينَ أَمَامَ آسَا وَأَمَامَ يَهُوذَا فَهَرَبَ الْكُوشِيُّونَ.
\par 13 وَطَرَدَهُمْ آسَا وَالشَّعْبُ الَّذِي مَعَهُ إِلَى جَرَارَ وَسَقَطَ مِنَ الْكُوشِيِّينَ حَتَّى لَمْ يَكُنْ لَهُمْ حَيٌّ لأَنَّهُمُ انْكَسَرُوا أَمَامَ الرَّبِّ وَأَمَامَ جَيْشِهِ. فَحَمَلُوا غَنِيمَةً كَثِيرَةً جِدّاً.
\par 14 وَضَرَبُوا جَمِيعَ الْمُدُنِ الَّتِي حَوْلَ جَرَارَ لأَنَّ رُعْبَ الرَّبِّ كَانَ عَلَيْهِمْ وَنَهَبُوا كُلَّ الْمُدُنِ لأَنَّهُ كَانَ فِيهَا نَهْبٌ كَثِيرٌ.
\par 15 وَضَرَبُوا أَيْضاً خِيَامَ الْمَاشِيَةِ وَسَاقُوا غَنَماً كَثِيراً وَجِمَالاً ثُمَّ رَجَعُوا إِلَى أُورُشَلِيمَ.

\chapter{15}

\par 1 وَكَانَ رُوحُ اللَّهِ عَلَى عَزَرْيَا بْنِ عُودِيدَ
\par 2 فَخَرَجَ لِلِقَاءِ آسَا وَقَالَ لَهُ: [اسْمَعُوا لِي يَا آسَا وَجَمِيعَ يَهُوذَا وَبِنْيَامِينَ. الرَّبُّ مَعَكُمْ مَا كُنْتُمْ مَعَهُ وَإِنْ طَلَبْتُمُوهُ يُوجَدْ لَكُمْ وَإِنْ تَرَكْتُمُوهُ يَتْرُكْكُمْ.
\par 3 وَلإِسْرَائِيلَ أَيَّامٌ كَثِيرَةٌ بِلاَ إِلَهٍ حَقٍّ وَبِلاَ كَاهِنٍ مُعَلِّمٍ وَبِلاَ شَرِيعَةٍ.
\par 4 وَلَكِنْ لَمَّا رَجَعُوا عِنْدَمَا تَضَايَقُوا إِلَى الرَّبِّ إِلَهِ إِسْرَائِيلَ وَطَلَبُوهُ وُجِدَ لَهُمْ.
\par 5 وَفِي تِلْكَ الأَزْمَانِ لَمْ يَكُنْ أَمَانٌ لِلْخَارِجِ وَلاَ لِلدَّاخِلِ لأَنَّ اضْطِرَابَاتٍ كَثِيرَةً كَانَتْ عَلَى كُلِّ سُكَّانِ الأَرَاضِي.
\par 6 فَأُفْنِيَتْ أُمَّةٌ بِأُمَّةٍ وَمَدِينَةٌ بِمَدِينَةٍ لأَنَّ اللَّهَ أَزْعَجَهُمْ بِكُلِّ ضِيقٍ.
\par 7 فَتَشَدَّدُوا أَنْتُمْ وَلاَ تَرْتَخِ أَيْدِيكُمْ لأَنَّ لِعَمَلِكُمْ أَجْراً].
\par 8 فَلَمَّا سَمِعَ آسَا هَذَا الْكَلاَمَ وَنُبُوَّةَ عُودِيدَ النَّبِيِّ تَشَدَّدَ وَنَزَعَ الرَّجَاسَاتِ مِنْ كُلِّ أَرْضِ يَهُوذَا وَبِنْيَامِينَ وَمِنَ الْمُدُنِ الَّتِي أَخَذَهَا مِنْ جَبَلِ أَفْرَايِمَ وَجَدَّدَ مَذْبَحَ الرَّبِّ الَّذِي أَمَامَ رِوَاقِ الرَّبِّ.
\par 9 وَجَمَعَ كُلَّ يَهُوذَا وَبِنْيَامِينَ وَالْغُرَبَاءَ مَعَهُمْ مِنْ أَفْرَايِمَ وَمَنَسَّى وَمِنْ شَمْعُونَ لأَنَّهُمْ سَقَطُوا إِلَيْهِ مِنْ إِسْرَائِيلَ بِكَثْرَةٍ حِينَ رَأُوا أَنَّ الرَّبَّ إِلَهَهُ مَعَهُ.
\par 10 فَاجْتَمَعُوا فِي أُورُشَلِيمَ فِي الشَّهْرِ الثَّالِثِ فِي السَّنَةِ الْخَامِسَةَ عَشَرَةَ لِمُلْكِ آسَا
\par 11 وَذَبَحُوا لِلرَّبِّ فِي ذَلِكَ الْيَوْمِ مِنَ الْغَنِيمَةِ الَّتِي جَلَبُوا سَبْعَ مِئَةٍ مِنَ الْبَقَرِ وَسَبْعَةَ آلاَفٍ مِنَ الضَّأْنِ.
\par 12 وَدَخَلُوا فِي عَهْدٍ أَنْ يَطْلُبُوا الرَّبَّ إِلَهَ آبَائِهِمْ بِكُلِّ قُلُوبِهِمْ وَكُلِّ أَنْفُسِهِمْ.
\par 13 حَتَّى إِنَّ كُلَّ مَنْ لاَ يَطْلُبُ الرَّبَّ إِلَهَ إِسْرَائِيلَ يُقْتَلُ مِنَ الصَّغِيرِ إِلَى الْكَبِيرِ مِنَ الرِّجَالِ وَالنِّسَاءِ.
\par 14 وَحَلَفُوا لِلرَّبِّ بِصَوْتٍ عَظِيمٍ وَهُتَافٍ وَبِأَبْوَاقٍ وَقُرُونٍ.
\par 15 وَفَرِحَ كُلُّ يَهُوذَا مِنْ أَجْلِ الْحَلْفِ لأَنَّهُمْ حَلَفُوا بِكُلِّ قُلُوبِهِمْ وَطَلَبُوهُ بِكُلِّ رِضَاهُمْ فَوُجِدَ لَهُمْ وَأَرَاحَهُمُ الرَّبُّ مِنْ كُلِّ جِهَةٍ.
\par 16 حَتَّى إِنَّ مَعْكَةَ أُمَّ آسَا الْمَلِكِ خَلَعَهَا مِنْ أَنْ تَكُونَ مَلِكَةً لأَنَّهَا عَمِلَتْ لِسَارِيَةٍ تِمْثَالاً وَقَطَعَ آسَا تِمْثَالَهَا وَدَقَّهُ وَأَحْرَقَهُ فِي وَادِي قَدْرُونَ.
\par 17 وَأَمَّا الْمُرْتَفَعَاتُ فَلَمْ تُنْزَعْ مِنْ إِسْرَائِيلَ. إِلاَّ أَنَّ قَلْبَ آسَا كَانَ كَامِلاً كُلَّ أَيَّامِهِ.
\par 18 وَأَدْخَلَ أَقْدَاسَ أَبِيهِ وَأَقْدَاسَهُ إِلَى بَيْتِ اللَّهِ مِنَ الْفِضَّةِ وَالذَّهَبِ وَالآنِيَةِ.
\par 19 وَلَمْ تَكُنْ حَرْبٌ إِلَى السَّنَةِ الْخَامِسَةِ وَالثَّلاَثِينَ لِمُلْكِ آسَا.

\chapter{16}

\par 1 فِي السَّنَةِ السَّادِسَةِ وَالثَّلاَثِينَ لِمُلْكِ آسَا صَعِدَ بَعْشَا مَلِكُ إِسْرَائِيلَ عَلَى يَهُوذَا وَبَنَى الرَّامَةَ لِكَيْلاَ يَدَعَ أَحَداً يَخْرُجُ أَوْ يَدْخُلُ إِلَى آسَا مَلِكِ يَهُوذَا.
\par 2 وَأَخْرَجَ آسَا فِضَّةً وَذَهَباً مِنْ خَزَائِنِ بَيْتِ الرَّبِّ وَبَيْتِ الْمَلِكِ وَأَرْسَلَ إِلَى بَنْهَدَدَ مَلِكِ أَرَامَ السَّاكِنِ فِي دِمَشْقَ قَائِلاً:
\par 3 [إِنَّ بَيْنِي وَبَيْنَكَ وَبَيْنَ أَبِي وَأَبِيكَ عَهْداً. هُوَذَا قَدْ أَرْسَلْتُ لَكَ فِضَّةً وَذَهَباً فَتَعَالَ انْقُضْ عَهْدَكَ مَعَ بَعْشَا مَلِكِ إِسْرَائِيلَ فَيَصْعَدَ عَنِّي].
\par 4 فَسَمِعَ بَنْهَدَدُ لِلْمَلِكِ آسَا وَأَرْسَلَ رُؤَسَاءَ الْجُيُوشِ الَّتِي لَهُ عَلَى مُدُنِ إِسْرَائِيلَ فَضَرَبُوا عُيُونَ وَدَانَ وَآبَلَ الْمِيَاهِ وَجَمِيعَ مَخَازِنِ مُدُنِ نَفْتَالِي.
\par 5 فَلَمَّا سَمِعَ بَعْشَا كَفَّ عَنْ بِنَاءِ الرَّامَةِ وَتَرَكَ عَمَلَهُ.
\par 6 فَأَخَذَ آسَا الْمَلِكُ كُلَّ يَهُوذَا فَحَمَلُوا حِجَارَةَ الرَّامَةِ وَأَخْشَابَهَا الَّتِي بَنَى بِهَا بَعْشَا وَبَنَى بِهَا جَبْعَ وَالْمِصْفَاةَ.
\par 7 وَفِي ذَلِكَ الزَّمَانِ جَاءَ حَنَانِي الرَّائِي إِلَى آسَا مَلِكِ يَهُوذَا وَقَالَ لَهُ: [مِنْ أَجْلِ أَنَّكَ اسْتَنَدْتَ عَلَى مَلِكِ أَرَامَ وَلَمْ تَسْتَنِدْ عَلَى الرَّبِّ إِلَهِكَ لِذَلِكَ قَدْ نَجَا جَيْشُ مَلِكِ أَرَامَ مِنْ يَدِكَ.
\par 8 أَلَمْ يَكُنِ الْكُوشِيُّونَ وَاللُّوبِيُّونَ جَيْشاً كَثِيراً بِمَرْكَبَاتٍ وَفُرْسَانٍ كَثِيرَةٍ جِدّاً؟ فَمِنْ أَجْلِ أَنَّكَ اسْتَنَدْتَ عَلَى الرَّبِّ دَفَعَهُمْ لِيَدِكَ.
\par 9 لأَنَّ عَيْنَيِ الرَّبِّ تَجُولاَنِ فِي كُلِّ الأَرْضِ لِيَتَشَدَّدَ مَعَ الَّذِينَ قُلُوبُهُمْ كَامِلَةٌ نَحْوَهُ فَقَدْ حَمِقْتَ فِي هَذَا حَتَّى إِنَّهُ مِنَ الآنَ تَكُونُ عَلَيْكَ حُرُوبٌ.
\par 10 فَغَضِبَ آسَا عَلَى الرَّائِي وَوَضَعَهُ فِي السِّجْنِ لأَنَّهُ اغْتَاظَ مِنْهُ مِنْ أَجْلِ هَذَا وَضَايَقَ آسَا بَعْضاً مِنَ الشَّعْبِ فِي ذَلِكَ الْوَقْتِ.
\par 11 وَأُمُورُ آسَا الأُولَى وَالأَخِيرَةُ مَكْتُوبَةٌ فِي سِفْرِ الْمُلُوكِ لِيَهُوذَا وَإِسْرَائِيلَ.
\par 12 وَمَرِضَ آسَا فِي السَّنَةِ التَّاسِعَةِ وَالثَّلاَثِينَ مِنْ مُلْكِهِ فِي رِجْلَيْهِ حَتَّى اشْتَدَّ مَرَضُهُ وَفِي مَرَضِهِ أَيْضاً لَمْ يَطْلُبِ الرَّبَّ بَلِ الأَطِبَّاءَ.
\par 13 ثُمَّ اضْطَجَعَ آسَا مَعَ آبَائِهِ وَمَاتَ فِي السَّنَةِ الْحَادِيَةِ وَالأَرْبَعِينَ لِمُلْكِهِ
\par 14 فَدَفَنُوهُ فِي قُبُورِهِ الَّتِي حَفَرَهَا لِنَفْسِهِ فِي مَدِينَةِ دَاوُدَ وَأَضْجَعُوهُ فِي سَرِيرٍ كَانَ مَمْلُوّاً أَطْيَاباً وَأَصْنَافاً عَطِرَةً حَسَبَ صِنَاعَةِ الْعِطَارَةِ. وَأَحْرَقُوا لَهُ حَرِيقَةً عَظِيمَةً جِدّاً.

\chapter{17}

\par 1 وَمَلَكَ يَهُوشَافَاطُ ابْنُهُ عِوَضاً عَنْهُ وَتَشَدَّدَ عَلَى إِسْرَائِيلَ.
\par 2 وَجَعَلَ جَيْشاً فِي جَمِيعِ مُدُنِ يَهُوذَا الْحَصِينَةِ وَجَعَلَ وُكَلاَءَ فِي أَرْضِ يَهُوذَا وَفِي مُدُنِ أَفْرَايِمَ الَّتِي أَخَذَهَا آسَا أَبُوهُ.
\par 3 وَكَانَ الرَّبُّ مَعَ يَهُوشَافَاطَ لأَنَّهُ سَارَ فِي طُرُقِ دَاوُدَ أَبِيهِ الأُولَى وَلَمْ يَطْلُبِ الْبَعْلِيمَ
\par 4 وَلَكِنَّهُ طَلَبَ إِلَهَ أَبِيهِ وَسَارَ فِي وَصَايَاهُ لاَ حَسَبَ أَعْمَالِ إِسْرَائِيلَ.
\par 5 فَثَبَّتَ الرَّبُّ الْمَمْلَكَةَ فِي يَدِهِ وَقَدَّمَ كُلُّ يَهُوذَا هَدَايَا لِيَهُوشَافَاطَ. وَكَانَ لَهُ غِنىً وَكَرَامَةً بِكَثْرَةٍ.
\par 6 وَتَقَوَّى قَلْبُهُ فِي طُرُقِ الرَّبِّ وَنَزَعَ أَيْضاً الْمُرْتَفَعَاتِ وَالسَّوَارِيَ مِنْ يَهُوذَا.
\par 7 وَفِي السَّنَةِ الثَّالِثَةِ لِمُلْكِهِ أَرْسَلَ إِلَى رُؤَسَائِهِ إِلَى بِنْحَائِلَ وَعُوبَدْيَا وَزَكَرِيَّا وَنَثَنْئِيلَ وَمِيخَايَا أَنْ يُعَلِّمُوا فِي مُدُنِ يَهُوذَا
\par 8 وَمَعَهُمُ اللاَّوِيُّونَ شَمَعْيَا وَنَثَنْيَا وَزَبَدْيَا وَعَسَائِيلُ وَشَمِيرَامُوثُ وَيَهُونَاثَانُ وَأَدُونِيَّا وَطُوبِيَّا وَطُوبُ أَدُونِيَّا اللاَّوِيُّونَ وَمَعَهُمْ أَلِيشَمَعُ وَيَهُورَامُ الْكَاهِنَانِ.
\par 9 فَعَلَّمُوا فِي يَهُوذَا وَمَعَهُمْ سِفْرُ شَرِيعَةِ الرَّبِّ وَجَالُوا فِي جَمِيعِ مُدُنِ يَهُوذَا وَعَلَّمُوا الشَّعْبَ.
\par 10 وَكَانَتْ هَيْبَةُ الرَّبِّ عَلَى جَمِيعِ مَمَالِكِ الأَرَاضِي الَّتِي حَوْلَ يَهُوذَا فَلَمْ يُحَارِبُوا يَهُوشَافَاطَ.
\par 11 وَبَعْضُ الْفِلِسْطِينِيِّينَ أَتُوا يَهُوشَافَاطَ بِهَدَايَا وَحِمْلِ فِضَّةٍ وَالْعُرْبَانُ أَيْضاً أَتُوهُ بِغَنَمٍ مِنَ الْكِبَاشِ سَبْعَةُ آلاَفٍ وَسَبْعُ مِئَةٍ وَمِنَ التُّيُوسِ سَبْعَةُ آلاَفٍ وَسَبْعُ مِئَةٍ.
\par 12 وَكَانَ يَهُوشَافَاطُ يَتَعَظَّمُ جِدّاً وَبَنَى فِي يَهُوذَا حُصُوناً وَمُدُنَ مَخَازِنَ.
\par 13 وَكَانَ لَهُ شُغْلٌ كَثِيرٌ فِي مُدُنِ يَهُوذَا وَرِجَالُ حَرْبٍ جَبَابِرَةُ بَأْسٍ فِي أُورُشَلِيمَ.
\par 14 وَهَذَا عَدَدُهُمْ حَسَبَ بُيُوتِ آبَائِهِمْ مِنْ يَهُوذَا رُؤَسَاءِ أُلُوفٍ: عَدَنَةُ الرَّئِيسُ وَمَعَهُ جَبَابِرَةُ بَأْسٍ ثَلاَثُ مِئَةِ أَلْفٍ.
\par 15 وَبِجَانِبِهِ يَهُونَاثَانُ الرَّئِيسُ وَمَعَهُ مِئَتَانِ وَثَمَانُونَ أَلْفاً.
\par 16 وَبِجَانِبِهِ عَمَسْيَا بْنُ زِكْرِي الْمُنْتَدِبُ لِلرَّبِّ وَمَعَهُ مِئَتَا أَلْفِ جَبَّارِ بَأْسٍ.
\par 17 وَمِنْ بِنْيَامِينَ أَلِيَادَاعُ جَبَّارُ بَأْسٍ وَمَعَهُ مِنَ الْمُتَسَلِّحِينَ بِالْقِسِيِّ وَالأَتْرَاسِ مِئَتَا أَلْفٍ.
\par 18 وَبِجَانِبِهِ يَهُوزَابَادُ وَمَعَهُ مِئَةٌ وَثَمَانُونَ أَلْفاً مُتَجَرِّدُونَ لِلْحَرْبِ.
\par 19 هَؤُلاَءِ خُدَّامُ الْمَلِكِ فَضْلاً عَنِ الَّذِينَ جَعَلَهُمُ الْمَلِكُ فِي الْمُدُنِ الْحَصِينَةِ فِي كُلِّ يَهُوذَا.

\chapter{18}

\par 1 وَكَانَ لِيَهُوشَافَاطَ غِنًى وَكَرَامَةٌ بِكَثْرَةٍ. وَصَاهَرَ أَخْآبَ.
\par 2 وَنَزَلَ بَعْدَ سِنِينَ إِلَى أَخْآبَ إِلَى السَّامِرَةِ فَذَبَحَ أَخْآبُ غَنَماً وَبَقَراً بِكَثْرَةٍ لَهُ وَلِلشَّعْبِ الَّذِي مَعَهُ وَأَغْوَاهُ أَنْ يَصْعَدَ إِلَى رَامُوتِ جِلْعَادَ.
\par 3 وَقَالَ أَخْآبُ مَلِكُ إِسْرَائِيلَ لِيَهُوشَافَاطَ مَلِكِ يَهُوذَا: [أَتَذْهَبُ مَعِي إِلَى رَامُوتَ جِلْعَادَ؟] فَقَالَ لَهُ: [مَثَلِي مَثَلُكَ وَشَعْبِي كَشَعْبِكَ وَمَعَكَ فِي الْقِتَالِ].
\par 4 ثُمَّ قَالَ يَهُوشَافَاطُ لِمَلِكِ إِسْرَائِيلَ: [اسْأَلِ الْيَوْمَ عَنْ كَلاَمِ الرَّبِّ].
\par 5 فَجَمَعَ مَلِكُ إِسْرَائِيلَ الأَنْبِيَاءَ أَرْبَعَ مِئَةِ رَجُلٍ وَقَالَ لَهُمْ: [أَنَذْهَبُ إِلَى رَامُوتِ جِلْعَادَ لِلْقِتَالِ أَمْ أَمْتَنِعُ؟] فَقَالُوا: [اصْعَدْ فَيَدْفَعَهَا اللَّهُ لِيَدِ الْمَلِكِ].
\par 6 فَقَالَ يَهُوشَافَاطُ: [أَلَيْسَ هُنَا أَيْضاً نَبِيٌّ لِلرَّبِّ فَنَسْأَلَ مِنْهُ؟]
\par 7 فَقَالَ مَلِكُ إِسْرَائِيلَ لِيَهُوشَافَاطَ: [بَعْدُ رَجُلٌ وَاحِدٌ لِسُؤَالِ الرَّبِّ بِهِ وَلَكِنَّنِي أُبْغِضُهُ لأَنَّهُ لاَ يَتَنَبَّأُ عَلَيَّ خَيْراً بَلْ شَرّاً كُلَّ أَيَّامِهِ وَهُوَ مِيخَا بْنُ يَمْلَةَ]. فَقَالَ يَهُوشَافَاطُ: [لاَ يَقُلِ الْمَلِكُ هَكَذَا].
\par 8 فَدَعَا مَلِكُ إِسْرَائِيلَ خَصِيّاً وَقَالَ: [أَسْرِعْ بِمِيخَا بْنِ يَمْلَةَ].
\par 9 وَكَانَ مَلِكُ إِسْرَائِيلَ وَيَهُوشَافَاطُ مَلِكُ يَهُوذَا جَالِسَيْنِ كُلُّ وَاحِدٍ عَلَى كُرْسِيِّهِ لاَبِسَيْنِ ثِيَابَهُمَا وَجَالِسَيْنِ فِي سَاحَةٍ عِنْدَ مَدْخَلِ بَابِ السَّامِرَةِ وَجَمِيعُ الأَنْبِيَاءِ يَتَنَبَّأُونَ أَمَامَهُمَا.
\par 10 وَعَمِلَ صِدْقِيَّا بْنُ كَنْعَنَةَ لِنَفْسِهِ قُرُونَ حَدِيدٍ وَقَالَ: [هَكَذَا قَالَ الرَّبُّ: بِهَذِهِ تَنْطَحُ الأَرَامِيِّينَ حَتَّى يَفْنُوا].
\par 11 وَتَنَبَّأَ جَمِيعُ الأَنْبِيَاءِ هَكَذَا قَائِلِينَ: [اصْعَدْ إِلَى رَامُوتِ جِلْعَادَ وَأَفْلِحْ فَيَدْفَعَهَا الرَّبُّ لِيَدِ الْمَلِكِ].
\par 12 وَقَالَ الرَّسُولُ الَّذِي ذَهَبَ لِيَدْعُوَ مِيخَا: [هُوَذَا كَلاَمُ جَمِيعِ الأَنْبِيَاءِ بِفَمٍ وَاحِدٍ خَيْرٌ لِلْمَلِكِ. فَلِْيَكُنْ كَلاَمُكَ كَوَاحِدٍ مِنْهُمْ وَتَكَلَّمْ بِخَيْرٍ].
\par 13 فَقَالَ مِيخَا: [حَيٌّ هُوَ الرَّبُّ إِنَّ مَا يَقُولُهُ إِلَهِي فَبِهِ أَتَكَلَّمُ].
\par 14 وَلَمَّا جَاءَ إِلَى الْمَلِكِ قَالَ لَهُ الْمَلِكُ: [يَا مِيخَا أَنَذْهَبُ إِلَى رَامُوتَ جِلْعَادَ لِلْقِتَالِ أَمْ أَمْتَنِعُ؟] فَقَالَ: [اصْعَدُوا وَأَفْلِحُوا فَيُدْفَعُوا لِيَدِكُمْ].
\par 15 فَقَالَ لَهُ الْمَلِكُ: [كَمْ مَرَّةٍ أَسْتَحْلِفُكَ أَنْ لاَ تَقُولَ لِي إِلاَّ الْحَقَّ بِاسْمِ الرَّبِّ!]
\par 16 فَقَالَ: [رَأَيْتُ كُلَّ إِسْرَائِيلَ مُشَتَّتِينَ عَلَى الْجِبَالِ كَخِرَافٍ لاَ رَاعِيَ لَهَا. فَقَالَ الرَّبُّ: لَيْسَ لِهَؤُلاَءِ أَصْحَابٌ فَلْيَرْجِعُوا كُلُّ وَاحِدٍ إِلَى بَيْتِهِ بِسَلاَمٍ].
\par 17 فَقَالَ مَلِكُ إِسْرَائِيلَ لِيَهُوشَافَاطَ: [أَمَا قُلْتُ لَكَ إِنَّهُ لاَ يَتَنَبَّأُ عَلَيَّ خَيْراً بَلْ شَرّاً؟]
\par 18 وَقَالَ: [فَاسْمَعْ إِذاً كَلاَمَ الرَّبِّ. قَدْ رَأَيْتُ الرَّبَّ جَالِساً عَلَى كُرْسِيِّهِ وَكُلُّ جُنْدِ السَّمَاءِ وُقُوفٌ عَنْ يَمِينِهِ وَعَنْ يَسَارِهِ.
\par 19 فَقَالَ الرَّبُّ: [مَنْ يُغْوِي أَخْآبَ مَلِكَ إِسْرَائِيلَ فَيَصْعَدَ وَيَسْقُطَ فِي رَامُوتَ جِلْعَادَ؟ فَقَالَ هَذَا هَكَذَا وَقَالَ ذَاكَ هَكَذَا.
\par 20 ثُمَّ خَرَجَ الرُّوحُ وَوَقَفَ أَمَامَ الرَّبِّ وَقَالَ: [أَنَا أُغْوِيهِ. فَسَأَلَهُ الرَّبُّ: [بِمَاذَا؟]
\par 21 فَقَالَ: [أَخْرُجُ وَأَكُونُ رُوحَ كَذِبٍ فِي أَفْوَاهِ جَمِيعِ أَنْبِيَائِهِ]. فَقَالَ: [إِنَّكَ تُغْوِيهِ وَتَقْتَدِرُ. فَاخْرُجْ وَافْعَلْ هَكَذَا.
\par 22 وَالآنَ هُوَذَا قَدْ جَعَلَ الرَّبُّ رُوحَ كَذِبٍ فِي أَفْوَاهِ أَنْبِيَائِكَ هَؤُلاَءِ وَالرَّبُّ تَكَلَّمَ عَلَيْكَ بِشَرٍّ].
\par 23 فَتَقَدَّمَ صِدْقِيَّا بْنُ كَنْعَنَةَ وَضَرَبَ مِيخَا عَلَى الْفَكِّ وَقَالَ: [مِنْ أَيِّ طَرِيقٍ عَبَرَ رُوحُ الرَّبِّ مِنِّي لِيُكَلِّمَك؟].
\par 24 فَقَالَ مِيخَا: [إِنَّكَ سَتَرَى فِي ذَلِكَ الْيَوْمِ الَّذِي تَدْخُلُ فِيهِ مِنْ مِخْدَعٍ إِلَى مِخْدَعٍ لِتَخْتَبِئَ].
\par 25 فَقَالَ مَلِكُ إِسْرَائِيلَ: [خُذُوا مِيخَا وَرُدُّوهُ إِلَى أَمُّونَ رَئِيسِ الْمَدِينَةِ وَإِلَى يُوآشَ ابْنِ الْمَلِكِ
\par 26 وَقُولُوا هَكَذَا يَقُولُ الْمَلِكُ: ضَعُوا هَذَا فِي السِّجْنِ وَأَطْعِمُوهُ خُبْزَ الضِّيقِ وَمَاءَ الضِّيقِ حَتَّى أَرْجِعَ بِسَلاَمٍ].
\par 27 فَقَالَ مِيخَا: [إِنْ رَجَعْتَ رُجُوعاً بِسَلاَمٍ فَلَمْ يَتَكَلَّمِ الرَّبُّ بِي]. وَقَالَ: [اسْمَعُوا أَيُّهَا الشُّعُوبُ أَجْمَعُونَ].
\par 28 فَصَعِدَ مَلِكُ إِسْرَائِيلَ وَيَهُوشَافَاطُ مَلِكُ يَهُوذَا إِلَى رَامُوتَ جِلْعَادَ.
\par 29 وَقَالَ مَلِكُ إِسْرَائِيلَ لِيَهُوشَافَاطَ: [إِنِّي أَتَنَكَّرُ وَأَدْخُلُ الْحَرْبَ وَأَمَّا أَنْتَ فَالْبَسْ ثِيَابَكَ]. فَتَنَكَّرَ مَلِكُ إِسْرَائِيلَ وَدَخَلاَ الْحَرْبَ.
\par 30 وَأَمَرَ مَلِكُ أَرَامَ رُؤَسَاءَ الْمَرْكَبَاتِ الَّتِي لَهُ: [لاَ تُحَارِبُوا صَغِيراً وَلاَ كَبِيراً إِلاَّ مَلِكَ إِسْرَائِيلَ وَحْدَهُ].
\par 31 فَلَمَّا رَأَى رُؤَسَاءُ الْمَرْكَبَاتِ يَهُوشَافَاطَ قَالُوا إِنَّهُ مَلِكُ إِسْرَائِيلَ فَحَاوَطُوهُ لِلْقِتَالِ فَصَرَخَ يَهُوشَافَاطُ وَسَاعَدَهُ الرَّبُّ وَحَوَّلَهُمُ عَنْهُ.
\par 32 فَلَمَّا رَأَى رُؤَسَاءُ الْمَرْكَبَاتِ أَنَّهُ لَيْسَ مَلِكَ إِسْرَائِيلَ رَجَعُوا عَنْهُ.
\par 33 وَإِنَّ رَجُلاً نَزَعَ فِي قَوْسِهِ غَيْرَ مُتَعَمِّدٍ وَضَرَبَ مَلِكَ إِسْرَائِيلَ بَيْنَ أَوْصَالِ الدِّرْعِ فَقَالَ لِمُدِيرِ الْمَرْكَبَةِ: [رُدَّ يَدَكَ وَأَخْرِجْنِي مِنَ الْجَيْشِ لأَنِّي قَدْ جُرِحْتُ].
\par 34 وَاشْتَدَّ الْقِتَالُ فِي ذَلِكَ الْيَوْمِ وَأُوقِفَ مَلِكُ إِسْرَائِيلَ فِي الْمَرْكَبَةِ مُقَابِلَ أَرَامَ إِلَى الْمَسَاءِ وَمَاتَ عِنْدَ غُرُوبِ الشَّمْسِ.

\chapter{19}

\par 1 وَرَجَعَ يَهُوشَافَاطُ مَلِكُ يَهُوذَا إِلَى بَيْتِهِ بِسَلاَمٍ إِلَى أُورُشَلِيمَ.
\par 2 وَخَرَجَ لِلِقَائِهِ يَاهُو بْنُ حَنَانِي الرَّائِي وَقَالَ لِلْمَلِكِ يَهُوشَافَاطَ: [أَتُسَاعِدُ الشِّرِّيرَ وَتُحِبُّ مُبْغِضِي الرَّبِّ؟ فَلِذَلِكَ الْغَضَبُ عَلَيْكَ مِنْ قِبَلِ الرَّبِّ.
\par 3 غَيْرَ أَنَّهُ وُجِدَتْ فِيكَ أُمُورٌ صَالِحَةٌ لأَنَّكَ نَزَعْتَ السَّوَارِيَ مِنَ الأَرْضِ وَهَيَّأْتَ قَلْبَكَ لِطَلَبِ اللَّهِ].
\par 4 وَأَقَامَ يَهُوشَافَاطُ فِي أُورُشَلِيمَ ثُمَّ رَجَعَ وَخَرَجَ أَيْضاً بَيْنَ الشَّعْبِ مِنْ بِئْرِ سَبْعٍ إِلَى جَبَلِ أَفْرَايِمَ وَرَدَّهُمْ إِلَى الرَّبِّ إِلَهِ آبَائِهِمْ.
\par 5 وَأَقَامَ قُضَاةً فِي الأَرْضِ فِي كُلِّ مُدُنِ يَهُوذَا الْمُحَصَّنَةِ فِي كُلِّ مَدِينَةٍ فَمَدِينَةٍ.
\par 6 وَقَالَ لِلْقُضَاةِ: [انْظُرُوا مَا أَنْتُمْ فَاعِلُونَ لأَنَّكُمْ لاَ تَقْضُونَ لِلإِنْسَانِ بَلْ لِلرَّبِّ وَهُوَ مَعَكُمْ فِي أَمْرِ الْقَضَاءِ.
\par 7 وَالآنَ لِتَكُنْ هَيْبَةُ الرَّبِّ عَلَيْكُمُ. احْذَرُوا وَافْعَلُوا. لأَنَّهُ لَيْسَ عِنْدَ الرَّبِّ إِلَهِنَا ظُلْمٌ وَلاَ مُحَابَاةٌ وَلاَ ارْتِشَاءٌ].
\par 8 وَكَذَا فِي أُورُشَلِيمَ أَقَامَ يَهُوشَافَاطُ مِنَ اللاَّوِيِّينَ وَالْكَهَنَةِ وَمِنْ رُؤُوسِ آبَاءِ إِسْرَائِيلَ لِقَضَاءِ الرَّبِّ وَالدَّعَاوِي. وَرَجَعُوا إِلَى أُورُشَلِيمَ.
\par 9 وَأَمَرَهُمْ: [هَكَذَا تَفْعَلُونَ بِتَقْوَى الرَّبِّ بِأَمَانَةٍ وَقَلْبٍ كَامِلٍ.
\par 10 وَفِي كُلِّ دَعْوَى تَأْتِي إِلَيْكُمْ مِنْ إِخْوَتِكُمُ السَّاكِنِينَ فِي مُدُنِهِمْ بَيْنَ دَمٍ وَدَمٍ بَيْنَ شَرِيعَةٍ وَوَصِيَّةٍ مِنْ جِهَةِ فَرَائِضَ أَوْ أَحْكَامٍ حَذِّرُوهُمْ فَلاَ يَأْثَمُوا إِلَى الرَّبِّ فَيَكُونَ غَضَبٌ عَلَيْكُمْ وَعَلَى إِخْوَتِكُمْ. هَكَذَا افْعَلُوا فَلاَ تَأْثَمُوا.
\par 11 وَهُوَذَا أَمَرْيَا الْكَاهِنُ الرَّأْسُ عَلَيْكُمْ فِي كُلِّ أُمُورِ الرَّبِّ وَزَبَدْيَا بْنُ يَشْمَعِئِيلَ الرَّئِيسُ عَلَى بَيْتِ يَهُوذَا فِي كُلِّ أُمُورِ الْمَلِكِ وَالْعُرَفَاءُ اللاَّوِيُّونَ أَمَامَكُمْ. تَشَدَّدُوا وَافْعَلُوا وَلِْيَكُنِ الرَّبُّ مَعَ الصَّالِحِ].

\chapter{20}

\par 1 ثُمَّ بَعْدَ ذَلِكَ أَتَى بَنُو مُوآبَ وَبَنُو عَمُّونَ وَمَعَهُمُ الْعَمُّونِيُّونَ عَلَى يَهُوشَافَاطَ لِلْمُحَارَبَةِ.
\par 2 فَجَاءَ أُنَاسٌ وَأَخْبَرُوا يَهُوشَافَاطَ: [قَدْ جَاءَ عَلَيْكَ جُمْهُورٌ كَثِيرٌ مِنْ عَبْرِ الْبَحْرِ مِنْ أَرَامَ وَهَا هُمْ فِي حَصُّونَ تَامَارَ] (هِيَ عَيْنُ جَدْيٍ).
\par 3 فَخَافَ يَهُوشَافَاطُ وَجَعَلَ وَجْهَهُ لِيَطْلُبَ الرَّبَّ وَنَادَى بِصَوْمٍ فِي كُلِّ يَهُوذَا.
\par 4 وَاجْتَمَعَ يَهُوذَا لِيَسْأَلُوا الرَّبَّ. جَاءُوا أَيْضاً مِنْ كُلِّ مُدُنِ يَهُوذَا لِيَسْأَلُوا الرَّبَّ.
\par 5 فَوَقَفَ يَهُوشَافَاطُ فِي جَمَاعَةِ يَهُوذَا وَأُورُشَلِيمَ فِي بَيْتِ الرَّبِّ أَمَامَ الدَّارِ الْجَدِيدَةِ
\par 6 وَقَالَ: [يَا رَبُّ إِلَهَ آبَائِنَا أَمَا أَنْتَ هُوَ اللَّهُ فِي السَّمَاءِ وَأَنْتَ الْمُتَسَلِّطُ عَلَى جَمِيعِ مَمَالِكِ الأُمَمِ وَبِيَدِكَ قُوَّةٌ وَجَبَرُوتٌ وَلَيْسَ مَنْ يَقِفُ مَعَكَ؟
\par 7 أَلَسْتَ أَنْتَ إِلَهَنَا الَّذِي طَرَدْتَ سُكَّانَ هَذِهِ الأَرْضِ مِنْ أَمَامِ شَعْبِكَ إِسْرَائِيلَ وَأَعْطَيْتَهَا لِنَسْلِ إِبْرَاهِيمَ خَلِيلِكَ إِلَى الأَبَدِ
\par 8 فَسَكَنُوا فِيهَا وَبَنُوا لَكَ فِيهَا مَقْدِساً لاِسْمِكَ قَائِلِينَ:
\par 9 إِذَا جَاءَ عَلَيْنَا شَرٌّ سَيْفٌ قَضَاءٌ أَوْ وَبَأٌ أَوْ جُوعٌ وَوَقَفْنَا أَمَامَ هَذَا الْبَيْتِ وَأَمَامَكَ (لأَنَّ اسْمَكَ فِي هَذَا الْبَيْتِ) وَصَرَخْنَا إِلَيْكَ مِنْ ضِيقِنَا فَإِنَّكَ تَسْمَعُ وَتُخَلِّصُ؟
\par 10 وَالآنَ هُوَذَا بَنُو عَمُّونَ وَمُوآبُ وَجَبَلُ سَاعِيرَ الَّذِينَ لَمْ تَدَعْ إِسْرَائِيلَ يَدْخُلُونَ إِلَيْهِمْ حِينَ جَاءُوا مِنْ أَرْضِ مِصْرَ بَلْ مَالُوا عَنْهُمْ وَلَمْ يُهْلِكُوهُمْ
\par 11 فَهُوَذَا هُمْ يُكَافِئُونَنَا بِمَجِيئِهِمْ لِطَرْدِنَا مِنْ مُلْكِكَ الَّذِي مَلَّكْتَنَا إِيَّاهُ.
\par 12 يَا إِلَهَنَا أَمَا تَقْضِي عَلَيْهِمْ لأَنَّهُ لَيْسَ فِينَا قُوَّةٌ أَمَامَ هَذَا الْجُمْهُورِ الْكَثِيرِ الآتِي عَلَيْنَا وَنَحْنُ لاَ نَعْلَمُ مَاذَا نَعْمَلُ وَلَكِنْ نَحْوَكَ أَعْيُنُنَا].
\par 13 وَكَانَ كُلُّ يَهُوذَا وَاقِفِينَ أَمَامَ الرَّبِّ مَعَ أَطْفَالِهِمْ وَنِسَائِهِمْ وَبَنِيهِمْ.
\par 14 وَإِنَّ يَحْزَئِيلَ بْنَ زَكَرِيَّا بْنِ بَنَايَا بْنِ يَعِيئِيلَ بْنِ مَتَّنِيَّا اللاَّوِيَِّ مِنْ بَنِي آسَافَ كَانَ عَلَيْهِ رُوحُ الرَّبِّ فِي وَسَطِ الْجَمَاعَةِ
\par 15 فَقَالَ: [اصْغُوا يَا جَمِيعَ يَهُوذَا وَسُكَّانَ أُورُشَلِيمَ وَأَيُّهَا الْمَلِكُ يَهُوشَافَاطُ. هَكَذَا قَالَ الرَّبُّ لَكُمْ: لاَ تَخَافُوا وَلاَ تَرْتَاعُوا بِسَبَبِ هَذَا الْجُمْهُورِ الْكَثِيرِ لأَنَّ الْحَرْبَ لَيْسَتْ لَكُمْ بَلْ لِلَّهِ.
\par 16 غَداً انْزِلُوا عَلَيْهِمْ. هُوَذَا هُمْ صَاعِدُونَ فِي عَقَبَةِ صِيصَ فَتَجِدُوهُمْ فِي أَقْصَى الْوَادِي أَمَامَ بَرِّيَّةِ يَرُوئِيلَ.
\par 17 لَيْسَ عَلَيْكُمْ أَنْ تُحَارِبُوا فِي هَذِهِ. قِفُوا اثْبُتُوا وَانْظُرُوا خَلاَصَ الرَّبِّ مَعَكُمْ يَا يَهُوذَا وَأُورُشَلِيمُ لاَ تَخَافُوا وَلاَ تَرْتَاعُوا. غَداً اخْرُجُوا لِلِقَائِهِمْ وَالرَّبُّ مَعَكُمْ].
\par 18 فَخَرَّ يَهُوشَافَاطُ لِوَجْهِهِ عَلَى الأَرْضِ وَكُلُّ يَهُوذَا وَسُكَّانُ أُورُشَلِيمَ سَقَطُوا أَمَامَ الرَّبِّ سُجُوداً لِلرَّبِّ.
\par 19 فَقَامَ اللاَّوِيُّونَ مِنْ بَنِي الْقَهَاتِيِّينَ وَمِنْ بَنِي الْقُورَحِيِّينَ لِيُسَبِّحُوا الرَّبَّ إِلَهَ إِسْرَائِيلَ بِصَوْتٍ عَظِيمٍ جِدّاً.
\par 20 وَبَكَّرُوا صَبَاحاً وَخَرَجُوا إِلَى بَرِّيَّةِ تَقُوعَ. وَعِنْدَ خُرُوجِهِمْ وَقَفَ يَهُوشَافَاطُ وَقَالَ: [اسْمَعُوا يَا يَهُوذَا وَسُكَّانَ أُورُشَلِيمَ آمِنُوا بِالرَّبِّ إِلَهِكُمْ فَتَأْمَنُوا. آمِنُوا بِأَنْبِيَائِهِ فَتُفْلِحُوا].
\par 21 وَلَمَّا اسْتَشَارَ الشَّعْبَ أَقَامَ مُغَنِّينَ لِلرَّبِّ وَمُسَبِّحِينَ فِي زِينَةٍ مُقَدَّسَةٍ عِنْدَ خُرُوجِهِمْ أَمَامَ الْمُتَجَرِّدِينَ وَقَائِلِينَ: [احْمَدُوا الرَّبَّ لأَنَّ إِلَى الأَبَدِ رَحْمَتَهُ].
\par 22 وَلَمَّا ابْتَدَأُوا فِي الْغِنَاءِ وَالتَّسْبِيحِ جَعَلَ الرَّبُّ أَكْمِنَةً عَلَى بَنِي عَمُّونَ وَمُوآبَ وَجَبَلِ سَاعِيرَ الآتِينَ عَلَى يَهُوذَا فَانْكَسَرُوا.
\par 23 وَقَامَ بَنُو عَمُّونَ وَمُوآبُ عَلَى سُكَّانِ جَبَلِ سَاعِيرَ لِيُحَرِّمُوهُمْ وَيُهْلِكُوهُمْ. وَلَمَّا فَرَغُوا مِنْ سُكَّانِ سَاعِيرَ سَاعَدَ بَعْضُهُمْ عَلَى إِهْلاَكِ بَعْضٍ.
\par 24 وَلَمَّا جَاءَ يَهُوذَا إِلَى الْمَرْقَبِ فِي الْبَرِّيَّةِ تَطَلَّعُوا نَحْوَ الْجُمْهُورِ وَإِذَا هُمْ جُثَثٌ سَاقِطَةٌ عَلَى الأَرْضِ وَلَمْ يَنْفَلِتْ أَحَدٌ.
\par 25 فَأَتَى يَهُوشَافَاطُ وَشَعْبُهُ لِنَهْبِ أَمْوَالِهِمْ فَوَجَدُوا بَيْنَهُمْ أَمْوَالاً وَجُثَثاً وَأَمْتِعَةً ثَمِينَةً بِكَثْرَةٍ فَأَخَذُوهَا لأَنْفُسِهِمْ حَتَّى لَمْ يَقْدِرُوا أَنْ يَحْمِلُوهَا. وَكَانُوا ثَلاَثَةَ أَيَّامٍ يَنْهَبُونَ الْغَنِيمَةَ لأَنَّهَا كَانَتْ كَثِيرَةً.
\par 26 وَفِي الْيَوْمِ الرَّابِعِ اجْتَمَعُوا فِي وَادِي بَرَكَةَ لأَنَّهُمْ هُنَاكَ بَارَكُوا الرَّبَّ لِذَلِكَ دَعَوُا اسْمَ ذَلِكَ الْمَكَانِ [وَادِي بَرَكَةَ] إِلَى الْيَوْمِ.
\par 27 ثُمَّ ارْتَدَّ كُلُّ رِجَالِ يَهُوذَا وَأُورُشَلِيمَ وَيَهُوشَافَاطُ بِرَأْسِهِمْ لِيَرْجِعُوا إِلَى أُورُشَلِيمَ بِفَرَحٍ لأَنَّ الرَّبَّ فَرَّحَهُمْ عَلَى أَعْدَائِهِمْ.
\par 28 وَدَخَلُوا أُورُشَلِيمَ بِالرَّبَابِ وَالْعِيدَانِ وَالأَبْوَاقِ إِلَى بَيْتِ الرَّبِّ.
\par 29 وَكَانَتْ هَيْبَةُ اللَّهِ عَلَى كُلِّ مَمَالِكِ الأَرَاضِي حِينَ سَمِعُوا أَنَّ الرَّبَّ حَارَبَ أَعْدَاءَ إِسْرَائِيلَ.
\par 30 وَاسْتَرَاحَتْ مَمْلَكَةُ يَهُوشَافَاطَ وَأَرَاحَهُ إِلَهُهُ مِنْ كُلِّ جِهَةٍ.
\par 31 وَمَلَكَ يَهُوشَافَاطُ عَلَى يَهُوذَا. كَانَ ابْنَ خَمْسٍ وَثَلاَثِينَ سَنَةً حِينَ مَلَكَ وَمَلَكَ خَمْساً وَعِشْرِينَ سَنَةً فِي أُورُشَلِيمَ وَاسْمُ أُمِّهِ عَزُوبَةُ بِنْتُ شَلْحِي.
\par 32 وَسَارَ فِي طَرِيقِ أَبِيهِ آسَا وَلَمْ يَحِدْ عَنْهَا إِذْ عَمِلَ الْمُسْتَقِيمَ فِي عَيْنَيِ الرَّبِّ.
\par 33 إِلاَّ أَنَّ الْمُرْتَفَعَاتِ لَمْ تُنْتَزَعْ بَلْ كَانَ الشَّعْبُ لَمْ يُعِدُّوا بَعْدُ قُلُوبَهُمْ لإِلَهِ آبَائِهِمْ.
\par 34 وَبَقِيَّةُ أُمُورِ يَهُوشَافَاطَ الأُولَى وَالأَخِيرَةِ مَكْتُوبَةٌ فِي أَخْبَارِ يَاهُوَ بْنِ حَنَانِي الْمَذْكُورِ فِي سِفْرِ مُلُوكِ إِسْرَائِيلَ.
\par 35 ثُمَّ بَعْدَ ذَلِكَ اتَّحَدَ يَهُوشَافَاطُ مَلِكُ يَهُوذَا مَعَ أَخَزْيَا مَلِكِ إِسْرَائِيلَ الَّذِي أَسَاءَ فِي عَمَلِهِ.
\par 36 فَاتَّحَدَ مَعَهُ فِي عَمَلِ سُفُنٍ تَسِيرُ إِلَى تَرْشِيشَ فَعَمِلاَ السُّفُنَ فِي عِصْيُونَ جَابِرَ.
\par 37 وَتَنَبَّأَ أَلِيعَزَرُ بْنُ دُودَاوَا مِنْ مَرِيشَةَ عَلَى يَهُوشَافَاطَ قَائِلاً: [لأَنَّكَ اتَّحَدْتَ مَعَ أَخَزْيَا قَدِ اقْتَحَمَ الرَّبُّ أَعْمَالَكَ]. فَتَكَسَّرَتِ السُّفُنُ وَلَمْ تَسْتَطِعِ السَّيْرَ إِلَى تَرْشِيشَ.

\chapter{21}

\par 1 وَاضْطَجَعَ يَهُوشَافَاطُ مَعَ آبَائِهِ فَدُفِنَ مَعَ آبَائِهِ فِي مَدِينَةِ دَاوُدَ وَمَلَكَ يَهُورَامُ ابْنُهُ عِوَضاً عَنْهُ.
\par 2 وَكَانَ لَهُ إِخْوَةٌ بَنُو يَهُوشَافَاطَ: عَزَرْيَا وَيَحِيئِيلُ وَزَكَرِيَّا وَعَزَرْيَاهُو وَمِيخَائِيلُ وَشَفَطْيَا. كُلُّ هَؤُلاَءِ بَنُو يَهُوشَافَاطَ مَلِكِ إِسْرَائِيلَ.
\par 3 وَأَعْطَاهُمْ أَبُوهُمْ عَطَايَا كَثِيرَةً مِنْ فِضَّةٍ وَذَهَبٍ وَتُحَفٍ مَعَ مُدُنٍ حَصِينَةٍ فِي يَهُوذَا. وَأَمَّا الْمَمْلَكَةُ فَأَعْطَاهَا لِيَهُورَامَ لأَنَّهُ الْبِكْرُ.
\par 4 فَقَامَ يَهُورَامُ عَلَى مَمْلَكَةِ أَبِيهِ وَتَشَدَّدَ وَقَتَلَ جَمِيعَ إِخْوَتِهِ بِالسَّيْفِ وَأَيْضاً بَعْضاً مِنْ رُؤَسَاءِ إِسْرَائِيلَ.
\par 5 كَانَ يَهُورَامُ ابْنَ اثْنَتَيْنِ وَثَلاَثِينَ سَنَةً حِينَ مَلَكَ وَمَلَكَ ثَمَانِيَ سِنِينَ فِي أُورُشَلِيمَ.
\par 6 وَسَارَ فِي طَرِيقِ مُلُوكِ إِسْرَائِيلَ كَمَا فَعَلَ بَيْتُ أَخْآبَ لأَنَّ بِنْتَ أَخْآبَ كَانَتْ لَهُ امْرَأَةً. وَعَمِلَ الشَّرَّ فِي عَيْنَيِ الرَّبِّ.
\par 7 وَلَمْ يَشَإِ الرَّبُّ أَنْ يُبِيدَ بَيْتَ دَاوُدَ لأَجْلِ الْعَهْدِ الَّذِي قَطَعَهُ مَعَ دَاوُدَ وَلأَنَّهُ قَالَ إِنَّهُ يُعْطِيهِ وَبَنِيهِ سِرَاجاً كُلَّ الأَيَّامِ.
\par 8 فِي أَيَّامِهِ عَصَى أَدُومُ عَلَى يَهُوذَا وَمَلَّكُوا عَلَى أَنْفُسِهِمْ مَلِكاً.
\par 9 وَعَبَرَ يَهُورَامُ مَعَ رُؤَسَائِهِ وَجَمِيعُ الْمَرْكَبَاتِ مَعَهُ وَقَامَ لَيْلاً وَضَرَبَ أَدُومَ الْمُحِيطَ بِهِ وَرُؤَسَاءَ الْمَرْكَبَاتِ.
\par 10 فَعَصَى أَدُومُ عَلَى يَهُوذَا إِلَى هَذَا الْيَوْمِ. حِينَئِذٍ عَصَتْ لِبْنَةُ فِي ذَلِكَ الْوَقْتِ عَلَيْهِ لأَنَّهُ تَرَكَ الرَّبَّ إِلَهَ آبَائِهِ.
\par 11 وَهُوَ أَيْضاً عَمِلَ مُرْتَفَعَاتٍ فِي جِبَالِ يَهُوذَا وَجَعَلَ سُكَّانَ أُورُشَلِيمَ يَزْنُونَ وَطَوَّحَ يَهُوذَا.
\par 12 وَأَتَتْ إِلَيْهِ كِتَابَةٌ مِنْ إِيلِيَّا النَّبِيِّ تَقُولُ: [هَكَذَا قَالَ الرَّبُّ إِلَهُ دَاوُدَ أَبِيكَ: مِنْ أَجْلِ أَنَّكَ لَمْ تَسْلُكْ فِي طُرُقِ يَهُوشَافَاطَ أَبِيكَ وَطُرُقِ آسَا مَلِكِ يَهُوذَا
\par 13 بَلْ سَلَكْتَ فِي طُرُقِ مُلُوكِ إِسْرَائِيلَ وَجَعَلْتَ يَهُوذَا وَسُكَّانَ أُورُشَلِيمَ يَزْنُونَ كَزِنَا بَيْتِ أَخْآبَ وَقَتَلْتَ أَيْضاً إِخْوَتَكَ مِنْ بَيْتِ أَبِيكَ الَّذِينَ هُمْ أَفْضَلُ مِنْكَ
\par 14 هُوَذَا يَضْرِبُ الرَّبُّ شَعْبَكَ وَبَنِيكَ وَنِسَاءَكَ وَكُلَّ مَالِكَ ضَرْبَةً عَظِيمَةً.
\par 15 وَإِيَّاكَ بِأَمْرَاضٍ كَثِيرَةٍ بِدَاءِ أَمْعَائِكَ حَتَّى تَخْرُجَ أَمْعَاؤُكَ بِسَبَبِ الْمَرَضِ يَوْماً فَيَوْماً].
\par 16 وَأَهَاجَ الرَّبُّ عَلَى يَهُورَامَ رُوحَ الْفِلِسْطِينِيِّينَ وَالْعَرَبَ الَّذِينَ بِجَانِبِ الْكُوشِيِّينَ
\par 17 فَصَعِدُوا إِلَى يَهُوذَا وَافْتَتَحُوهَا وَسَبُوا كُلَّ الأَمْوَالِ الْمَوْجُودَةِ فِي بَيْتِ الْمَلِكِ مَعَ بَنِيهِ وَنِسَائِهِ أَيْضاً وَلَمْ يَبْقَ لَهُ ابْنٌ إِلاَّ يَهُوآحَازُ أَصْغَرَ بَنِيهِ.
\par 18 وَبَعْدَ هَذَا كُلِّهِ ضَرَبَهُ الرَّبُّ فِي أَمْعَائِهِ بِمَرَضٍ لَيْسَ لَهُ شِفَاءٌ.
\par 19 وَكَانَ مِنْ يَوْمٍ إِلَى يَوْمٍ وَحَسَبَ ذِهَابِ الْمُدَّةِ عِنْدَ نَهَايَةِ سَنَتَيْنِ أَنَّ أَمْعَاءَهُ خَرَجَتْ بِسَبَبِ مَرَضِهِ فَمَاتَ بِأَمْرَاضٍ رَدِيئَةٍ وَلَمْ يَعْمَلْ لَهُ شَعْبُهُ حَرِيقَةً كَحَرِيقَةِ آبَائِهِ.
\par 20 كَانَ ابْنَ اثْنَتَيْنِ وَثَلاَثِينَ سَنَةً حِينَ مَلَكَ وَمَلَكَ ثَمَانِيَ سِنِينَ فِي أُورُشَلِيمَ وَذَهَبَ غَيْرَ مَأْسُوفٍ عَلَيْهِ وَدَفَنُوهُ فِي مَدِينَةِ دَاوُدَ وَلَكِنْ لَيْسَ فِي قُبُورِ الْمُلُوكِ.

\chapter{22}

\par 1 وَمَلَّكَ سُكَّانُ أُورُشَلِيمَ أَخَزْيَا ابْنَهُ الأَصْغَرَ عِوَضاً عَنْهُ لأَنَّ جَمِيعَ الأَوَّلِينَ قَتَلَهُمُ الْغُزَاةُ الَّذِينَ جَاءُوا مَعَ الْعَرَبِ إِلَى الْمَحَلَّةِ. فَمَلَكَ أَخَزْيَا بْنُ يَهُورَامَ مَلِكِ يَهُوذَا.
\par 2 كَانَ أَخَزْيَا ابْنَ اثْنَتَيْنِ وَأَرْبَعِينَ سَنَةً حِينَ مَلَكَ وَمَلَكَ سَنَةً وَاحِدَةً فِي أُورُشَلِيمَ وَاسْمُ أُمِّهِ عَثَلْيَا بِنْتُ عُمْرِي.
\par 3 وَهُوَ أَيْضاً سَلَكَ فِي طُرُقِ بَيْتِ أَخْآبَ لأَنَّ أُمَّهُ كَانَتْ تُشِيرُ عَلَيْهِ بِفِعْلِ الشَّرِّ.
\par 4 فَعَمِلَ الشَّرَّ فِي عَيْنَيِ الرَّبِّ مِثْلَ بَيْتِ أَخْآبَ لأَنَّهُمْ كَانُوا لَهُ مُشِيرِينَ بَعْدَ وَفَاةِ أَبِيهِ لإِبَادَتِهِ.
\par 5 فَسَلَكَ بِمَشُورَتِهِمْ وَذَهَبَ مَعَ يُورَامَ بْنِ أَخْآبَ مَلِكِ إِسْرَائِيلَ لِمُحَارَبَةِ حَزَائِيلَ مَلِكِ أَرَامَ فِي رَامُوتَ جِلْعَادَ. وَضَرَبَ الأَرَامِيُّونَ يُورَامَ
\par 6 فَرَجَعَ لِيَبْرَأَ فِي يَزْرَعِيلَ بِسَبَبِ الضَّرَبَاتِ الَّتِي ضَرَبُوهُ بِهَا فِي الرَّامَةِ عِنْدَ مُحَارَبَتِهِ حَزَائِيلَ مَلِكَ أَرَامَ. وَنَزَلَ أَخَزْيَا بْنُ يَهُورَامَ مَلِكُ يَهُوذَا لِيَزُورَ يُورَامَ بْنِ أَخْآبَ فِي يَزْرَعِيلَ لأَنَّهُ كَانَ مَرِيضاً.
\par 7 فَمِنْ قِبَلِ اللَّهِ كَانَ هَلاَكُ أَخَزْيَا بِمَجِيئِهِ إِلَى يُورَامَ. فَإِنَّهُ حِينَ جَاءَ خَرَجَ مَعَ يُورَامَ إِلَى يَاهُوَ بْنِ نِمْشِي الَّذِي مَسَحَهُ الرَّبُّ لِقَطْعِ بَيْتِ أَخْآبَ.
\par 8 وَإِذْ كَانَ يَاهُو يَقْضِي عَلَى بَيْتِ أَخْآبَ وَجَدَ رُؤَسَاءَ يَهُوذَا وَبَنِي إِخْوَةِ أَخَزْيَا الَّذِينَ كَانُوا يَخْدُِمُونَ أَخَزْيَا فَقَتَلَهُمْ.
\par 9 وَطَلَبَ أَخَزْيَا فَأَمْسَكُوهُ وَهُوَ مُخْتَبِئٌ فِي السَّامِرَةِ وَأَتُوا بِهِ إِلَى يَاهُو وَقَتَلُوهُ وَدَفَنُوهُ لأَنَّهُمْ قَالُوا إِنَّهُ ابْنُ يَهُوشَافَاطَ الَّذِي طَلَبَ الرَّبَّ بِكُلِّ قَلْبِهِ. فَلَمْ يَكُنْ لِبَيْتِ أَخَزْيَا مَنْ يَقْوَى عَلَى الْمَمْلَكَةِ.
\par 10 وَلَمَّا رَأَتْ عَثَلْيَا أُمُّ أَخَزْيَا أَنَّ ابْنَهَا قَدْ مَاتَ قَامَتْ وَأَبَادَتْ جَمِيعَ النَّسْلِ الْمَلِكِيِّ مِنْ بَيْتِ يَهُوذَا.
\par 11 أَمَّا يَهُوشَبْعَةُ بِنْتُ الْمَلِكِ فَأَخَذَتْ يَهُوآشَ بْنَ أَخَزْيَا وَسَرَقَتْهُ مِنْ وَسَطِ بَنِي الْمَلِكِ الَّذِينَ قُتِلُوا وَجَعَلَتْهُ هُوَ وَمُرْضِعَتَهُ فِي مِخْدَعِ السَّرِيرِ وَخَبَّأَتْهُ يَهُوشَبْعَةُ بِنْتُ الْمَلِكِ يَهُورَامَ امْرَأَةُ يَهُويَادَاعَ الْكَاهِنِ. (لأَنَّهَا كَانَتْ أُخْتَ أَخَزْيَا) مِنْ وَجْهِ عَثَلْيَا فَلَمْ تَقْتُلْهُ.
\par 12 وَكَانَ مَعَهُمْ فِي بَيْتِ اللَّهِ مُخْتَبِئاً سِتَّ سِنِينٍ وَعَثَلْيَا مَالِكَةٌ عَلَى الأَرْضِ.

\chapter{23}

\par 1 وَفِي السَّنَةِ السَّابِعَةِ تَشَدَّدَ يَهُويَادَاعُ وَأَخَذَ مَعَهُ فِي الْعَهْدِ رُؤَسَاءَ الْمِئَاتِ: عَزَرْيَا بْنَ يَرُوحَامَ وَإِسْمَاعِيلَ بْنَ يَهُوحَانَانَ وَعَزَرْيَا بْنَ عُوبِيدَ وَمَعْسِيَّا بْنَ عَدَايَا وَأَلِيشَافَاطَ بْنَ زِكْرِي
\par 2 وَجَالُوا فِي يَهُوذَا وَجَمَعُوا اللاَّوِيِّينَ مِنْ جَمِيعِ مُدُنِ يَهُوذَا وَرُؤُوسَ آبَاءِ إِسْرَائِيلَ وَجَاءُوا إِلَى أُورُشَلِيمَ.
\par 3 وَقَطَعَ كُلُّ الْمَجْمَعِ عَهْداً فِي بَيْتِ اللَّهِ مَعَ الْمَلِكِ. وَقَالَ لَهُمْ: [هُوَذَا ابْنُ الْمَلِكِ يَمْلِكُ كَمَا تَكَلَّمَ الرَّبُّ عَنْ بَنِي دَاوُدَ.
\par 4 هَذَا هُوَ الأَمْرُ الَّذِي تَعْمَلُونَهُ. الثُّلْثُ مِنْكُمُ الَّذِينَ يَدْخُلُونَ فِي السَّبْتِ مِنَ الْكَهَنَةِ وَاللاَّوِيِّينَ يَكُونُونَ بَوَّابِينَ لِلأَبْوَابِ
\par 5 وَالثُّلْثُ فِي بَيْتِ الْمَلِكِ وَالثُّلْثُ فِي بَابِ الأَسَاسِ وَجَمِيعُ الشَّعْبِ فِي دِيَارِ بَيْتِ الرَّبِّ.
\par 6 وَلاَ يَدْخُلْ بَيْتَ الرَّبِّ إِلاَّ الْكَهَنَةُ وَالَّذِينَ يَخْدِمُونَ مِنَ اللاَّوِيِّينَ فَهُمْ يَدْخُلُونَ لأَنَّهُمْ مُقَدَّسُونَ وَكُلُّ الشَّعْبِ يَحْرُسُونَ حِرَاسَةَ الرَّبِّ.
\par 7 وَيُحِيطُ اللاَّوِيُّونَ بِالْمَلِكِ مُسْتَدِيرِينَ كُلُّ وَاحِدٍ سِلاَحُهُ بِيَدِهِ. وَالَّذِي يَدْخُلُ الْبَيْتَ يُقْتَلُ. وَكُونُوا مَعَ الْمَلِكِ فِي دُخُولِهِ وَفِي خُرُوجِهِ].
\par 8 فَعَمِلَ اللاَّوِيُّونَ وَكُلُّ يَهُوذَا حَسَبَ كُلِّ مَا أَمَرَ بِهِ يَهُويَادَاعُ الْكَاهِنُ. وَأَخَذُوا كُلُّ وَاحِدٍ رِجَالَهُ الدَّاخِلِينَ فِي السَّبْتِ مَعَ الْخَارِجِينَ فِي السَّبْتِ لأَنَّ يَهُويَادَاعَ الْكَاهِنَ لَمْ يَصْرِفِ الْفِرَقَ.
\par 9 وَأَعْطَى يَهُويَادَاعُ الْكَاهِنُ رُؤَسَاءَ الْمِئَاتِ الْحِرَابَ وَالْمَجَانَّ وَالأَتْرَاسَ الَّتِي لِلْمَلِكِ دَاوُدَ الَّتِي فِي بَيْتِ اللَّهِ.
\par 10 وَأَوْقَفَ جَمِيعَ الشَّعْبِ وَكُلُّ وَاحِدٍ سِلاَحُهُ بِيَدِهِ مِنْ جَانِبِ الْبَيْتِ الأَيْمَنِ إِلَى جَانِبِ الْبَيْتِ الأَيْسَرِ حَوْلَ الْمَذْبَحِ وَالْبَيْتِ حَوْلَ الْمَلِكِ مُسْتَدِيرِينَ.
\par 11 ثُمَّ أَخْرَجُوا ابْنَ الْمَلِكِ وَوَضَعُوا عَلَيْهِ التَّاجَ وَأَعْطُوهُ الشَّهَادَةَ وَمَلَّكُوهُ. وَمَسَحَهُ يَهُويَادَاعُ وَبَنُوهُ وَقَالُوا: [لِيَحْيَ الْمَلِكُ!].
\par 12 وَلَمَّا سَمِعَتْ عَثَلْيَا صَوْتَ الشَّعْبِ يَرْكُضُونَ وَيَمْدَحُونَ الْمَلِكَ دَخَلَتْ إِلَى الشَّعْبِ فِي بَيْتِ الرَّبِّ.
\par 13 وَنَظَرَتْ وَإِذَا الْمَلِكُ وَاقِفٌ عَلَى مِنْبَرِهِ فِي الْمَدْخَلِ وَالرُّؤَسَاءُ وَالأَبْوَاقُ عِنْدَ الْمَلِكِ وَكُلُّ شَعْبِ الأَرْضِ يَفْرَحُونَ وَيَنْفُخُونَ بِالأَبْوَاقِ وَالْمُغَنُّونَ بِآلاَتِ الْغِنَاءِ وَالْمُعَلِّمُونَ التَّسْبِيحَ. فَشَقَّتْ عَثَلْيَا ثِيَابَهَا وَقَالَتْ: [خِيَانَةٌ! خِيَانَةٌ!]
\par 14 فَأَخْرَجَ يَهُويَادَاعُ الْكَاهِنُ رُؤَسَاءَ الْمِئَاتِ الْمُوَكَّلِينَ عَلَى الْجَيْشِ وَقَالَ لَهُمْ: [أَخْرِجُوهَا إِلَى خَارِجِ الصُّفُوفِ وَالَّذِي يَتَّبِعُهَا يُقْتَلُ بِالسَّيْفِ]. لأَنَّ الْكَاهِنَ قَالَ: [لاَ تَقْتُلُوهَا فِي بَيْتِ الرَّبِّ].
\par 15 فَأَلْقُوا عَلَيْهَا الأَيَادِيَ. وَلَمَّا أَتَتْ إِلَى مَدْخَلِ بَابِ الْخَيْلِ إِلَى بَيْتِ الْمَلِكِ قَتَلُوهَا هُنَاكَ.
\par 16 فَقَطَعَ يَهُويَادَاعُ عَهْداً بَيْنَهُ وَبَيْنَ كُلِّ الشَّعْبِ وَبَيْنَ الْمَلِكِ أَنْ يَكُونُوا شَعْباً لِلرَّبِّ.
\par 17 وَدَخَلَ جَمِيعُ الشَّعْبِ إِلَى بَيْتِ الْبَعْلِ وَهَدَمُوهُ وَكَسَّرُوا مَذَابِحَهُ وَتَمَاثِيلَهُ وَقَتَلُوا مَتَّانَ كَاهِنَ الْبَعْلِ أَمَامَ الْمَذْبَحِ.
\par 18 وَجَعَلَ يَهُويَادَاعُ حُرَّاساً عَلَى بَيْتِ الرَّبِّ عَنْ يَدِ الْكَهَنَةِ اللاَّوِيِّينَ الَّذِينَ قَسَمَهُمْ دَاوُدُ عَلَى بَيْتِ الرَّبِّ لإِصْعَادِ مُحْرَقَاتِ الرَّبِّ كَمَا هُوَ مَكْتُوبٌ فِي شَرِيعَةِ مُوسَى بِالْفَرَحِ وَالْغِنَاءِ حَسَبَ أَمْرِ دَاوُدَ.
\par 19 وَأَوْقَفَ الْبَوَّابِينَ عَلَى أَبْوَابِ بَيْتِ الرَّبِّ لِئَلاَّ يَدْخُلَ نَجِسٌ فِي أَمْرٍ مَا.
\par 20 وَأَخَذَ رُؤَسَاءَ الْمِئَاتِ وَالْعُظَمَاءَ وَالْمُتَسَلِّطِينَ عَلَى الشَّعْبِ وَكُلَّ شَعْبِ الأَرْضِ وَأَنْزَلَ الْمَلِكَ مِنْ بَيْتِ الرَّبِّ وَدَخَلُوا مِنْ وَسَطِ الْبَابِ الأَعْلَى إِلَى بَيْتِ الْمَلِكِ وَأَجْلَسُوا الْمَلِكَ عَلَى كُرْسِيِّ الْمَمْلَكَةِ.
\par 21 فَفَرِحَ كُلُّ شَعْبِ الأَرْضِ وَاسْتَرَاحَتِ الْمَدِينَةُ وَقَتَلُوا عَثَلْيَا بِالسَّيْفِ.

\chapter{24}

\par 1 كَانَ يَهُوآشُ ابْنَ سَبْعِ سِنِينَ حِينَ مَلَكَ وَمَلَكَ أَرْبَعِينَ سَنَةً فِي أُورُشَلِيمَ وَاسْمُ أُمِّهِ ظَبْيَةُ مِنْ بِئْرِ سَبْعٍ.
\par 2 وَعَمِلَ يَهُوآشُ الْمُسْتَقِيمَ فِي عَيْنَيِ الرَّبِّ كُلَّ أَيَّامِ يَهُويَادَاعَ الْكَاهِنِ.
\par 3 وَاتَّخَذَ يَهُويَادَاعُ لَهُ امْرَأَتَيْنِ فَوَلَدَ بَنِينَ وَبَنَاتٍ.
\par 4 وَحَدَثَ بَعْدَ ذَلِكَ أَنَّهُ كَانَ فِي قَلْبِ يَهُوآشَ أَنْ يُجَدِّدَ بَيْتَ الرَّبِّ.
\par 5 فَجَمَعَ الْكَهَنَةَ وَاللاَّوِيِّينَ وَقَالَ لَهُمُ: [اخْرُجُوا إِلَى مُدُنِ يَهُوذَا وَاجْمَعُوا مِنْ جَمِيعِ إِسْرَائِيلَ فِضَّةً لأَجْلِ تَرْمِيمِ بَيْتِ إِلَهِكُمْ مِنْ سَنَةٍ إِلَى سَنَةٍ وَبَادِرُوا أَنْتُمْ إِلَى هَذَا الأَمْرِ]. فَلَمْ يُبَادِرِ اللاَّوِيُّونَ.
\par 6 فَدَعَا الْمَلِكُ يَهُويَادَاعَ الرَّئِيسَ وَسَأَلَهُ: [لِمَاذَا لَمْ تَطْلُبْ مِنَ اللاَّوِيِّينَ أَنْ يَأْتُوا مِنْ يَهُوذَا وَأُورُشَلِيمَ بِجِزْيَةِ مُوسَى عَبْدِ الرَّبِّ وَجَمَاعَةِ إِسْرَائِيلَ لِخَيْمَةِ الشَّهَادَةِ؟
\par 7 لأَنَّ بَنِي عَثَلْيَا الْخَبِيثَةِ قَدْ هَدَمُوا بَيْتَ اللَّهِ وَصَيَّرُوا كُلَّ أَقْدَاسِ بَيْتِ الرَّبِّ لِلْبَعْلِيمِ].
\par 8 وَأَمَرَ الْمَلِكُ فَعَمِلُوا صُنْدُوقاً وَجَعَلُوهُ فِي بَابِ بَيْتِ الرَّبِّ خَارِجاً
\par 9 وَنَادُوا فِي يَهُوذَا وَأُورُشَلِيمَ بِأَنْ يَأْتُوا إِلَى الرَّبِّ بِجِزْيَةِ مُوسَى عَبْدِ الرَّبِّ الْمَفْرُوضَةِ عَلَى إِسْرَائِيلَ فِي الْبَرِّيَّةِ.
\par 10 فَفَرِحَ كُلُّ الرُّؤَسَاءِ وَكُلُّ الشَّعْبِ وَأَدْخَلُوا وَأَلْقُوا فِي الصُّنْدُوقِ حَتَّى امْتَلَأَ.
\par 11 وَحِينَمَا كَانَ يُؤْتَى بِالصُّنْدُوقِ إِلَى وَكَالَةِ الْمَلِكِ بِيَدِ اللاَّوِيِّينَ عِنْدَمَا يَرُونَ أَنَّ الْفِضَّةَ قَدْ كَثُرَتْ كَانَ يَأْتِي كَاتِبُ الْمَلِكِ وَوَكِيلُ الْكَاهِنِ الرَّئِيسِ وَيُفْرِغَانِ الصُّنْدُوقَ ثُمَّ يَحْمِلاَنِهِ وَيَرُدَّانِهِ إِلَى مَكَانِهِ. هَكَذَا كَانُوا يَفْعَلُونَ يَوْماً فَيَوْماً حَتَّى جَمَعُوا فِضَّةً بِكَثْرَةٍ.
\par 12 وَدَفَعَهَا الْمَلِكُ وَيَهُويَادَاعُ لِعَامِلِي شُغْلِ خِدْمَةِ بَيْتِ الرَّبِّ وَكَانُوا يَسْتَأْجِرُونَ نَحَّاتِينَ وَنَجَّارِينَ لِتَجْدِيدِ بَيْتِ الرَّبِّ وَلِلْعَامِلِينَ فِي الْحَدِيدِ وَالنُّحَاسِ أَيْضاً لِتَرْمِيمِ بَيْتِ الرَّبِّ.
\par 13 فَعَمِلَ عَامِلُو الشُّغْلِ وَنَجَحَ الْعَمَلُ بِأَيْدِيهِمْ وَأَقَامُوا بَيْتَ اللَّهِ عَلَى رَسْمِهِ وَثَبَّتُوهُ.
\par 14 وَلَمَّا أَكْمَلُوا أَتُوا إِلَى مَا بَيْنَ يَدَيِ الْمَلِكِ وَيَهُويَادَاعَ بِبَقِيَّةِ الْفِضَّةِ وَعَمِلُوهَا آنِيَةً لِبَيْتِ الرَّبِّ آنِيَةَ خِدْمَةٍ وَإِصْعَادٍ وَصُحُوناً وَآنِيَةَ ذَهَبٍ وَفِضَّةٍ. وَكَانُوا يُصْعِدُونَ مُحْرَقَاتٍ فِي بَيْتِ الرَّبِّ دَائِماً كُلَّ أَيَّامِ يَهُويَادَاعَ.
\par 15 وَشَاخَ يَهُويَادَاعُ وَشَبِعَ مِنَ الأَيَّامِ وَمَاتَ. كَانَ ابْنَ مِئَةٍ وَثَلاَثِينَ سَنَةً عِنْدَ وَفَاتِهِ.
\par 16 فَدَفَنُوهُ فِي مَدِينَةِ دَاوُدَ مَعَ الْمُلُوكِ لأَنَّهُ عَمِلَ خَيْراً فِي إِسْرَائِيلَ وَمَعَ اللَّهِ وَبَيْتِهِ.
\par 17 وَبَعْدَ مَوْتِ يَهُويَادَاعَ جَاءَ رُؤَسَاءُ يَهُوذَا وَسَجَدُوا لِلْمَلِكِ. حِينَئِذٍ سَمِعَ الْمَلِكُ لَهُمْ.
\par 18 وَتَرَكُوا بَيْتَ الرَّبِّ إِلَهِ آبَائِهِمْ وَعَبَدُوا السَّوَارِيَ وَالأَصْنَامَ فَكَانَ غَضَبٌ عَلَى يَهُوذَا وَأُورُشَلِيمَ لأَجْلِ إِثْمِهِمْ هَذَا.
\par 19 وَأَرْسَلَ إِلَيْهِمْ أَنْبِيَاءَ لإِرْجَاعِهِمْ إِلَى الرَّبِّ وَأَشْهَدُوا عَلَيْهِمْ فَلَمْ يُصْغُوا.
\par 20 وَلَبِسَ رُوحُ اللَّهِ زَكَرِيَّا بْنَ يَهُويَادَاعَ الْكَاهِنَ فَوَقَفَ فَوْقَ الشَّعْبِ وَقَالَ لَهُمْ: [هَكَذَا يَقُولُ اللَّهُ: لِمَاذَا تَتَعَدَّوْنَ وَصَايَا الرَّبِّ فَلاَ تُفْلِحُونَ؟ لأَنَّكُمْ تَرَكْتُمُ الرَّبَّ قَدْ تَرَكَكُمْ].
\par 21 فَفَتَنُوا عَلَيْهِ وَرَجَمُوهُ بِحِجَارَةٍ بِأَمْرِ الْمَلِكِ فِي دَارِ بَيْتِ الرَّبِّ.
\par 22 وَلَمْ يَذْكُرْ يَهُوآشُ الْمَلِكُ الْمَعْرُوفَ الَّذِي عَمِلَهُ يَهُويَادَاعُ أَبُوهُ مَعَهُ بَلْ قَتَلَ ابْنَهُ. وَعِنْدَ مَوْتِهِ قَالَ: [الرَّبُّ يَنْظُرُ وَيُطَالِبُ].
\par 23 وَفِي مَدَارِ السَّنَةِ صَعِدَ عَلَيْهِ جَيْشُ أَرَامَ وَأَتُوا إِلَى يَهُوذَا وَأُورُشَلِيمَ وَأَهْلَكُوا كُلَّ رُؤَسَاءِ الشَّعْبِ مِنَ الشَّعْبِ وَجَمِيعَ غَنِيمَتِهِمْ أَرْسَلُوهَا إِلَى مَلِكِ دِمَشْقَ.
\par 24 لأَنَّ جَيْشَ أَرَامَ جَاءَ بِشِرْذِمَةٍ قَلِيلَةٍ وَدَفَعَ الرَّبُّ لِيَدِهِمْ جَيْشاً كَثِيراً جِدّاً لأَنَّهُمْ تَرَكُوا الرَّبَّ إِلَهَ آبَائِهِمْ. فَأَجْرُوا قَضَاءً عَلَى يَهُوآشَ.
\par 25 وَعِنْدَ ذَهَابِهِمْ عَنْهُ - لأَنَّهُمْ تَرَكُوهُ بِأَمْرَاضٍ كَثِيرَةٍ - فَتَنَ عَلَيْهِ عَبِيدُهُ مِنْ أَجْلِ دِمَاءِ بَنِي يَهُويَادَاعَ الْكَاهِنِ وَقَتَلُوهُ عَلَى سَرِيرِهِ فَمَاتَ. فَدَفَنُوهُ فِي مَدِينَةِ دَاوُدَ وَلَمْ يَدْفِنُوهُ فِي قُبُورِ الْمُلُوكِ.
\par 26 وَهَذَانِ هُمَا الْفَاتِنَانِ عَلَيْهِ: زَابَادُ بْنُ شِمْعَةَ الْعَمُّونِيَّةِ وَيَهُوزَابَادُ بْنُ شِمْرِيتَ الْمُوآبِيَّةِ.
\par 27 وَأَمَّا بَنُوهُ وَكَثْرَةُ مَا حُمِلَ عَلَيْهِ وَمَرَمَّةُ بَيْتِ اللَّهِ مَمَكْتُوبَةٌ فِي مِدْرَسِ سِفْرِ الْمُلُوكِ. وَمَلَكَ أَمَصْيَا ابْنُهُ عِوَضاً عَنْهُ.

\chapter{25}

\par 1 مَلَكَ أَمَصْيَا وَهُوَ ابْنُ خَمْسٍ وَعِشْرِينَ سَنَةً وَمَلَكَ تِسْعاً وَعِشْرِينَ سَنَةً فِي أُورُشَلِيمَ وَاسْمُ أُمِّهِ يَهُوعَدَّانُ مِنْ أُورُشَلِيمَ.
\par 2 وَعَمِلَ الْمُسْتَقِيمَ فِي عَيْنَيِ الرَّبِّ وَلَكِنْ لَيْسَ بِقَلْبٍ كَامِلٍ.
\par 3 وَلَمَّا تَثَبَّتَتِ الْمَمْلَكَةُ عَلَيْهِ قَتَلَ عَبِيدَهُ الَّذِينَ قَتَلُوا الْمَلِكَ أَبَاهُ.
\par 4 وَأَمَّا بَنُوهُمْ فَلَمْ يَقْتُلْهُمْ بَلْ كَمَا هُوَ مَكْتُوبٌ فِي الشَّرِيعَةِ فِي سِفْرِ مُوسَى حَيْثُ أَمَرَ الرَّبُّ: [لاَ تَمُوتُ الآبَاءُ لأَجْلِ الْبَنِينَ وَلاَ الْبَنُونَ يَمُوتُونَ لأَجْلِ الآبَاءِ بَلْ كُلُّ وَاحِدٍ يَمُوتُ لأَجْلِ خَطِيَّتِهِ].
\par 5 وَجَمَعَ أَمَصْيَا يَهُوذَا وَأَقَامَهُمْ حَسَبَ بُيُوتِ الآبَاءِ رُؤَسَاءَ أُلُوفٍ وَرُؤَسَاءَ مِئَاتٍ فِي كُلِّ يَهُوذَا وَبِنْيَامِينَ وَأَحْصَاهُمْ مِنِ ابْنِ عِشْرِينَ سَنَةً فَمَا فَوْقُ فَوَجَدَهُمْ ثَلاَثَ مِئَةِ أَلْفِ مُخْتَارٍ خَارِجٍ لِلْحَرْبِ حَامِلِ رُمْحٍ وَتُرْسٍ.
\par 6 وَاسْتَأْجَرَ مِنْ إِسْرَائِيلَ مِئَةَ أَلْفِ جَبَّارِ بَأْسٍ بِمِئَةِ وَزْنَةٍ مِنَ الْفِضَّةِ.
\par 7 وَجَاءَ إِلَيْهِ رَجُلُ اللَّهِ قَائِلاً: [أَيُّهَا الْمَلِكُ لاَ يَأْتِي مَعَكَ جَيْشُ إِسْرَائِيلَ لأَنَّ الرَّبَّ لَيْسَ مَعَ إِسْرَائِيلَ مَعَ كُلِّ بَنِي أَفْرَايِمَ.
\par 8 وَإِنْ ذَهَبْتَ أَنْتَ فَاعْمَلْ وَتَشَدَّدْ لِلْقِتَالِ لأَنَّ اللَّهَ يُسْقِطُكَ أَمَامَ الْعَدُوِّ لأَنَّ عِنْدَ اللَّهِ قُوَّةً لِلْمُسَاعَدَةِ وَلِلإِسْقَاطِ].
\par 9 فَقَالَ أَمَصْيَا لِرَجُلِ اللَّهِ: [فَمَاذَا يُعْمَلُ لأَجْلِ الْمِئَةِ الْوَزْنَةِ الَّتِي أَعْطَيْتُهَا لِغُزَاةِ إِسْرَائِيلَ؟] فَقَالَ رَجُلُ اللَّهِ: [إِنَّ الرَّبَّ قَادِرٌ أَنْ يُعْطِيَكَ أَكْثَرَ مِنْهَا].
\par 10 فَأَفْرَزَ أَمَصْيَا الْغُزَاةَ الَّذِينَ جَاءُوا إِلَيْهِ مِنْ أَفْرَايِمَ لِيَنْطَلِقُوا إِلَى مَكَانِهِمْ فَحَمِيَ غَضَبُهُمْ جِدّاً عَلَى يَهُوذَا وَرَجَعُوا إِلَى مَكَانِهِمْ بِحُمُوِّ الْغَضَبِ.
\par 11 وَأَمَّا أَمَصْيَا فَتَشَدَّدَ وَاقْتَادَ شَعْبَهُ وَذَهَبَ إِلَى وَادِي الْمِلْحِ وَضَرَبَ مِنْ بَنِي سَاعِيرَ عَشَرَةَ آلاَفٍ
\par 12 وَعَشَرَةَ آلاَفٍ أَحْيَاءَ سَبَاهُمْ بَنُو يَهُوذَا وَأَتُوا بِهِمْ إِلَى رَأْسِ سَالِعَ وَطَرَحُوهُمْ عَنْ رَأْسِ سَالِعَ فَتَكَسَّرُوا أَجْمَعُونَ.
\par 13 وَأَمَّا الْغُزَاةُ الَّذِينَ أَرْجَعَهُمْ أَمَصْيَا عَنِ الذَّهَابِ مَعَهُ إِلَى الْقِتَالِ فَاقْتَحَمُوا مُدُنَ يَهُوذَا مِنَ السَّامِرَةِ إِلَى بَيْتِ حُورُونَ وَضَرَبُوا مِنْهُمْ ثَلاَثَةَ آلاَفٍ وَنَهَبُوا نَهْباً كَثِيراً.
\par 14 ثُمَّ بَعْدَ مَجِيءِ أَمَصْيَا مِنْ ضَرْبِ الأَدُومِيِّينَ أَتَى بِآلِهَةِ بَنِي سَاعِيرَ وَأَقَامَهُمْ لَهُ آلِهَةً وَسَجَدَ أَمَامَهُمْ وَأَوْقَدَ لَهُمْ.
\par 15 فَحَمِيَ غَضَبُ الرَّبِّ عَلَى أَمَصْيَا وَأَرْسَلَ إِلَيْهِ نَبِيّاً فَقَالَ لَهُ: [لِمَاذَا طَلَبْتَ آلِهَةَ الشَّعْبِ الَّذِينَ لَمْ يُنْقِذُوا شَعْبَهُمْ مِنْ يَدِكَ؟]
\par 16 وَفِيمَا هُوَ يُكَلِّمُهُ قَالَ لَهُ: [هَلْ جَعَلُوكَ مُشِيراً لِلْمَلِكِ؟ كُفَّ! لِمَاذَا يَقْتُلُونَكَ؟] فَكَفَّ النَّبِيُّ وَقَالَ: [قَدْ عَلِمْتُ أَنَّ اللَّهَ قَدْ قَضَى بِهَلاَكِكَ لأَنَّكَ عَمِلْتَ هَذَا وَلَمْ تَسْمَعْ لِمَشُورَتِي].
\par 17 فَاسْتَشَارَ أَمَصْيَا مَلِكُ يَهُوذَا وَأَرْسَلَ إِلَى يُوآشَ بْنِ يَهُوآحَازَ بْنِ يَاهُو مَلِكِ إِسْرَائِيلَ قَائِلاً: [هَلُمَّ نَتَرَاءَ مُواجَهَةً].
\par 18 فَأَرْسَلَ يُوآشُ مَلِكُ إِسْرَائِيلَ إِلَى أَمَصْيَا مَلِكِ يَهُوذَا قَائِلاً: [الْعَوْسَجُ الَّذِي فِي لُبْنَانَ أَرْسَلَ إِلَى الأَرْزِ الَّذِي فِي لُبْنَانَ يَقُولُ: [أَعْطِ ابْنَتَكَ لاِبْنِي امْرَأَةً. فَعَبَرَ حَيَوَانٌ بَرِّيٌّ كَانَ فِي لُبْنَانَ وَدَاسَ الْعَوْسَجَ.
\par 19 تَقُولُ: هَئَنَذَا قَدْ ضَرَبْتُ أَدُومَ. فَرَفَّعَكَ قَلْبُكَ لِلتَّمَجُّدِ! فَالآنَ أَقِمْ فِي بَيْتِكَ. لِمَاذَا تَهْجُمُ عَلَى الشَّرِّ فَتَسْقُطَ أَنْتَ وَيَهُوذَا مَعَكَ؟].
\par 20 فَلَمْ يَسْمَعْ أَمَصْيَا لأَنَّهُ كَانَ مِنْ قِبَلِ اللَّهِ أَنْ يُسَلِّمَهُمْ لأَنَّهُمْ طَلَبُوا آلِهَةَ أَدُومَ.
\par 21 وَصَعِدَ يُوآشُ مَلِكُ إِسْرَائِيلَ فَتَرَاءَيَا مُواجَهَةً هُوَ وَأَمَصْيَا مَلِكُ يَهُوذَا فِي بَيْتِ شَمْسٍ الَّتِي لِيَهُوذَا.
\par 22 فَانْهَزَمَ يَهُوذَا أَمَامَ إِسْرَائِيلَ وَهَرَبُوا كُلُّ وَاحِدٍ إِلَى خَيْمَتِهِ.
\par 23 وَأَمَّا أَمَصْيَا مَلِكُ يَهُوذَا فَأَمْسَكَهُ يُوآشُ مَلِكُ إِسْرَائِيلَ فِي بَيْتِ شَمْسٍ وَجَاءَ بِهِ إِلَى أُورُشَلِيمَ وَهَدَمَ سُورَ أُورُشَلِيمَ مِنْ بَابِ أَفْرَايِمَ إِلَى بَابِ الزَّاوِيَةِ أَرْبَعَ مِئَةِ ذِرَاعٍ.
\par 24 وَأَخَذَ كُلَّ الذَّهَبِ وَالْفِضَّةِ وَكُلَّ الآنِيَةِ الْمَوْجُودَةِ فِي بَيْتِ اللَّهِ مَعَ عُوبِيدَ أَدُومَ وَخَزَائِنِ بَيْتِ الْمَلِكِ وَالرُّهَنَاءَ وَرَجَعَ إِلَى السَّامِرَةِ.
\par 25 وَعَاشَ أَمَصْيَا بْنُ يَهُوآشَ مَلِكُ يَهُوذَا بَعْدَ مَوْتِ يُوآشَ بْنِ يَهُوآحَازَ مَلِكِ إِسْرَائِيلَ خَمْسَ عَشَرَةَ سَنَةً.
\par 26 وَبَقِيَّةُ أُمُورِ أَمَصْيَا الأُولَى وَالأَخِيرَةُِ مَكْتُوبَةٌ فِي سِفْرِ مُلُوكِ يَهُوذَا وَإِسْرَائِيلَ.
\par 27 وَمِنْ حِينَ حَادَ أَمَصْيَا مِنْ وَرَاءِ الرَّبِّ فَتَنُوا عَلَيْهِ فِي أُورُشَلِيمَ فَهَرَبَ إِلَى لَخِيشَ فَأَرْسَلُوا وَرَاءَهُ إِلَى لَخِيشَ وَقَتَلُوهُ هُنَاكَ
\par 28 وَحَمَلُوهُ عَلَى الْخَيْلِ وَدَفَنُوهُ مَعَ آبَائِهِ فِي مَدِينَةِ يَهُوذَا.

\chapter{26}

\par 1 وَأَخَذَ كُلُّ شَعْبِ يَهُوذَا عُزِّيَّا وَهُوَ ابْنُ سِتَّ عَشَرَةَ سَنَةً وَمَلَّكُوهُ عِوَضاً عَنْ أَبِيهِ أَمَصْيَا.
\par 2 هُوَ بَنَى أَيْلَةَ وَرَدَّهَا لِيَهُوذَا بَعْدَ اضْطِجَاعِ الْمَلِكِ مَعَ آبَائِهِ.
\par 3 كَانَ عُزِّيَّا ابْنَ سِتَّ عَشَرَةَ سَنَةً حِينَ مَلَكَ وَمَلَكَ اثْنَتَيْنِ وَخَمْسِينَ سَنَةً فِي أُورُشَلِيمَ. وَاسْمُ أُمِّهِ يَكُلْيَا مِنْ أُورُشَلِيمَ.
\par 4 وَعَمِلَ الْمُسْتَقِيمَ فِي عَيْنَيِ الرَّبِّ حَسَبَ كُلِّ مَا عَمِلَ أَمَصْيَا أَبُوهُ.
\par 5 وَكَانَ يَطْلُبُ اللَّهَ فِي أَيَّامِ زَكَرِيَّا الْفَاهِمِ بِمَنَاظِرِ اللَّهِ. وَفِي أَيَّامِ طَلَبِهِ الرَّبَّ أَنْجَحَهُ اللَّهُ.
\par 6 وَخَرَجَ وَحَارَبَ الْفِلِسْطِينِيِّينَ وَهَدَمَ سُورَ جَتَّ وَسُورَ يَبْنَةَ وَسُورَ أَشْدُودَ وَبَنَى مُدُناً فِي أَرْضِ أَشْدُودَ وَالْفِلِسْطِينِيِّينَ.
\par 7 وَسَاعَدَهُ اللَّهُ عَلَى الْفِلِسْطِينِيِّينَ وَعَلَى الْعَرَبِ السَّاكِنِينَ فِي جُورَبَعْلَ وَالْمَعُونِيِّينَ.
\par 8 وَأَعْطَى الْعَمُّونِيُّونَ عُزِّيَّا هَدَايَا وَامْتَدَّ اسْمُهُ إِلَى مَدْخَلِ مِصْرَ لأَنَّهُ تَشَدَّدَ جِدّاً.
\par 9 وَبَنَى عُزِّيَّا أَبْرَاجاً فِي أُورُشَلِيمَ عِنْدَ بَابِ الزَّاوِيَةِ وَعِنْدَ بَابِ الْوَادِي وَعِنْدَ الزَّاوِيَةِ وَحَصَّنَهَا.
\par 10 وَبَنَى أَبْرَاجاً فِي الْبَرِّيَّةِ وَحَفَرَ آبَاراً كَثِيرَةً لأَنَّهُ كَانَ لَهُ مَاشِيَةٌ كَثِيرَةٌ فِي السَّاحِلِ وَالسَّهْلِ وَفَلاَّحُونَ وَكَرَّامُونَ فِي الْجِبَالِ وَفِي الْكَرْمَلِ لأَنَّهُ كَانَ يُحِبُّ الْفِلاَحَةَ.
\par 11 وَكَانَ لِعُزِّيَّا جَيْشٌ مِنَ الْمُقَاتِلِينَ يَخْرُجُونَ لِلْحَرْبِ أَحْزَاباً حَسَبَ عَدَدِ إِحْصَائِهِمْ عَنْ يَدِ يَعِيئِيلَ الْكَاتِبِ وَمَعْسِيَّا الْعَرِيفِ تَحْتَ يَدِ حَنَنْيَا وَاحِدٍ مِنْ رُؤَسَاءِ الْمَلِكِ.
\par 12 كُلُّ عَدَدِ رُؤُوسِ الآبَاءِ مِنْ جَبَابِرَةِ الْبَأْسِ أَلْفَانِ وَسِتُّ مِئَةٍ.
\par 13 وَتَحْتَ يَدِهِمْ جَيْشُ جُنُودٍ ثَلاَثُ مِئَةِ أَلْفٍ وَسَبْعَةُ آلاَفٍ وَخَمْسُ مِئَةٍ مِنَ الْمُقَاتِلِينَ بِقُوَّةٍ شَدِيدَةٍ لِمُسَاعَدَةِ الْمَلِكِ عَلَى الْعَدُوِّ.
\par 14 وَهَيَّأَ لَهُمْ عُزِّيَّا لِكُلِّ الْجَيْشِ أَتْرَاساً وَرِمَاحاً وَخُوَذاً وَدُرُوعاً وَقِسِيّاً وَحِجَارَةَ مَقَالِيعَ.
\par 15 وَعَمِلَ فِي أُورُشَلِيمَ مَنْجَنِيقَاتٍ اخْتِرَاعَ مُخْتَرِعِينَ لِتَكُونَ عَلَى الأَبْرَاجِ وَعَلَى الزَّوَايَا لِتُرْمَى بِهَا السِّهَامُ وَالْحِجَارَةُ الْعَظِيمَةُ. وَامْتَدَّ اسْمُهُ إِلَى بَعِيدٍ إِذْ عَجِبَتْ مُسَاعَدَتُهُ حَتَّى تَشَدَّدَ.
\par 16 وَلَمَّا تَشَدَّدَ ارْتَفَعَ قَلْبُهُ إِلَى الْهَلاَكِ وَخَانَ الرَّبَّ إِلَهَهُ وَدَخَلَ هَيْكَلَ الرَّبِّ لِيُوقِدَ عَلَى مَذْبَحِ الْبَخُورِ.
\par 17 وَدَخَلَ وَرَاءَهُ عَزَرْيَا الْكَاهِنُ وَمَعَهُ ثَمَانُونَ مِنْ كَهَنَةِ الرَّبِّ بَنِي الْبَأْسِ.
\par 18 وَقَاوَمُوا عُزِّيَّا الْمَلِكَ وَقَالُوا لَهُ: [لَيْسَ لَكَ يَا عُزِّيَّا أَنْ تُوقِدَ لِلرَّبِّ بَلْ لِلْكَهَنَةِ بَنِي هَارُونَ الْمُقَدَّسِينَ لِلإِيقَادِ. اخْرُجْ مِنَ الْمَقْدِسِ لأَنَّكَ خُنْتَ وَلَيْسَ لَكَ مِنْ كَرَامَةٍ مِنْ عِنْدِ الرَّبِّ الإِلَهِ].
\par 19 فَحَنِقَ عُزِّيَّا. وَكَانَ فِي يَدِهِ مِجْمَرَةٌ لِلإِيقَادِ. وَعِنْدَ حَنَقِهِ عَلَى الْكَهَنَةِ خَرَجَ بَرَصٌ فِي جَبْهَتِهِ أَمَامَ الْكَهَنَةِ فِي بَيْتِ الرَّبِّ بِجَانِبِ مَذْبَحِ الْبَخُورِ.
\par 20 فَالْتَفَتَ نَحْوَهُ عَزَرْيَاهُو الْكَاهِنُ الرَّأْسُ وَكُلُّ الْكَهَنَةِ وَإِذَا هُوَ أَبْرَصُ فِي جَبْهَتِهِ فَطَرَدُوهُ مِنْ هُنَاكَ حَتَّى إِنَّهُ هُوَ نَفْسُهُ بَادَرَ إِلَى الْخُرُوجِ لأَنَّ الرَّبَّ ضَرَبَهُ.
\par 21 وَكَانَ عُزِّيَّا الْمَلِكُ أَبْرَصَ إِلَى يَوْمِ وَفَاتِهِ وَأَقَامَ فِي بَيْتِ الْمَرَضِ أَبْرَصَ لأَنَّهُ قُطِعَ مِنْ بَيْتِ الرَّبِّ وَكَانَ يُوثَامُ ابْنُهُ عَلَى بَيْتِ الْمَلِكِ يَحْكُمُ عَلَى شَعْبِ الأَرْضِ.
\par 22 وَبَقِيَّةُ أُمُورِ عُزِّيَّا الأُولَى وَالأَخِيرَةُ كَتَبَهَا إِشَعْيَاءُ بْنُ آمُوصَ النَّبِيُّ.
\par 23 ثُمَّ اضْطَجَعَ عُزِّيَّا مَعَ آبَائِهِ وَدَفَنُوهُ مَعَ آبَائِهِ فِي حَقْلِ الْمَِقْبَرَةِ الَّتِي لِلْمُلُوكِ لأَنَّهُمْ قَالُوا إِنَّهُ أَبْرَصُ. وَمَلَكَ يُوثَامُ ابْنُهُ عِوَضاً عَنْهُ.

\chapter{27}

\par 1 كَانَ يُوثَامُ ابْنَ خَمْسٍ وَعِشْرِينَ سَنَةً حِينَ مَلَكَ وَمَلَكَ سِتَّ عَشَرَةَ سَنَةً فِي أُورُشَلِيمَ وَاسْمُ أُمِّهِ يَرُوشَةُ بِنْتُ صَادُوقَ.
\par 2 وَعَمِلَ الْمُسْتَقِيمَ فِي عَيْنَيِ الرَّبِّ حَسَبَ كُلِّ مَا عَمِلَ عُزِّيَّا أَبُوهُ (إِلاَّ أَنَّهُ لَمْ يَدْخُلْ هَيْكَلَ الرَّبِّ). وَكَانَ الشَّعْبُ يُفْسِدُونَ بَعْدُ.
\par 3 هُوَ بَنَى الْبَابَ الأَعْلَى لِبَيْتِ الرَّبِّ وَبَنَى كَثِيراً عَلَى سُورِ الأَكَمَةِ.
\par 4 وَبَنَى مُدُناً فِي جَبَلِ يَهُوذَا وَبَنَى فِي الْغَابَاتِ قِلَعاً وَأَبْرَاجاً.
\par 5 وَهُوَ حَارَبَ مَلِكَ بَنِي عَمُّونَ وَقَوِيَ عَلَيْهِمْ فَأَعْطَاهُ بَنُو عَمُّونَ فِي تِلْكَ السَّنَةِ مِئَةَ وَزْنَةٍ مِنَ الْفِضَّةِ وَعَشَرَةَ آلاَفِ كُرِّ قَمْحٍ وَعَشَرَةَ آلاَفٍ مِنَ الشَّعِيرِ. هَذَا مَا أَدَّاهُ لَهُ بَنُو عَمُّونَ وَكَذَلِكَ فِي السَّنَةِ الثَّانِيَةِ وَالثَّالِثَةِ.
\par 6 وَتَشَدَّدَ يُوثَامُ لأَنَّهُ هَيَّأَ طُرُقَهُ أَمَامَ الرَّبِّ إِلَهِهِ.
\par 7 وَبَقِيَّةُ أُمُورِ يُوثَامَ وَكُلُّ حُرُوبِهِ وَطُرُقِهِ مَكْتُوبَةٌ فِي سِفْرِ مُلُوكِ إِسْرَائِيلَ وَيَهُوذَا.
\par 8 كَانَ ابْنَ خَمْسٍ وَعِشْرِينَ سَنَةً حِينَ مَلَكَ وَمَلَكَ سِتَّ عَشَرَةَ سَنَةً فِي أُورُشَلِيمَ.
\par 9 ثُمَّ اضْطَجَعَ يُوثَامُ مَعَ آبَائِهِ فَدَفَنُوهُ فِي مَدِينَةِ دَاوُدَ وَمَلَكَ آحَازُ ابْنُهُ عِوَضاً عَنْهُ.

\chapter{28}

\par 1 كَانَ آحَازُ ابْنَ عِشْرِينَ سَنَةً حِينَ مَلَكَ وَمَلَكَ سِتَّ عَشَرَةَ سَنَةً فِي أُورُشَلِيمَ وَلَمْ يَفْعَلِ الْمُسْتَقِيمَ فِي عَيْنَيِ الرَّبِّ كَدَاوُدَ أَبِيهِ
\par 2 بَلْ سَارَ فِي طُرُقِ مُلُوكِ إِسْرَائِيلَ وَعَمِلَ أَيْضاً تَمَاثِيلَ مَسْبُوكَةً لِلْبَعْلِيمِ.
\par 3 وَهُوَ أَوْقَدَ فِي وَادِي ابْنِ هِنُّومَ وَأَحْرَقَ بَنِيهِ بِالنَّارِ حَسَبَ رَجَاسَاتِ الأُمَمِ الَّذِينَ طَرَدَهُمُ الرَّبُّ مِنْ أَمَامِ بَنِي إِسْرَائِيلَ.
\par 4 وَذَبَحَ وَأَوْقَدَ عَلَى الْمُرْتَفَعَاتِ وَعَلَى التِّلاَلِ وَتَحْتَ كُلِّ شَجَرَةٍ خَضْرَاءَ.
\par 5 فَدَفَعَهُ الرَّبُّ إِلَهُهُ لِيَدِ مَلِكِ أَرَامَ فَضَرَبُوهُ وَسَبُوا مِنْهُ سَبْياً عَظِيماً وَأَتُوا بِهِمْ إِلَى دِمَشْقَ. وَدُفِعَ أَيْضاً لِيَدِ مَلِكِ إِسْرَائِيلَ فَضَرَبَهُ ضَرْبَةً عَظِيمَةً.
\par 6 وَقَتَلَ فَقْحُ بْنُ رَمَلْيَا فِي يَهُوذَا مِئَةً وَعِشْرِينَ أَلْفاً فِي يَوْمٍ وَاحِدٍ - الْجَمِيعُ بَنُو بَأْسٍ - لأَنَّهُمْ تَرَكُوا الرَّبَّ إِلَهَ آبَائِهِمْ.
\par 7 وَقَتَلَ زِكْرِي جَبَّارُ أَفْرَايِمَ مَعْسِيَّا ابْنَ الْمَلِكِ وَعَزْرِيقَامَ رَئِيسَ الْبَيْتِ وَأَلْقَانَةَ ثَانِيَ الْمَلِكِ.
\par 8 وَسَبَى بَنُو إِسْرَائِيلَ مِنْ إِخْوَتِهِمْ مِئَتَيْ أَلْفٍ مِنَ النِّسَاءِ وَالْبَنِينَ وَالْبَنَاتِ وَنَهَبُوا أَيْضاً مِنْهُمْ غَنِيمَةً وَافِرَةً وَأَتُوا بِالْغَنِيمَةِ إِلَى السَّامِرَةِ.
\par 9 وَكَانَ هُنَاكَ نَبِيٌّ لِلرَّبِّ اسْمُهُ عُودِيدُ فَخَرَجَ لِلِقَاءِ الْجَيْشِ الآتِي إِلَى السَّامِرَةِ وَقَالَ لَهُمْ: [هُوَذَا مِنْ أَجْلِ غَضَبِ الرَّبِّ إِلَهِ آبَائِكُمْ عَلَى يَهُوذَا قَدْ دَفَعَهُمْ لِيَدِكُمْ وَقَدْ قَتَلْتُمُوهُمْ بِغَضَبٍ بَلَغَ السَّمَاءَ.
\par 10 وَالآنَ أَنْتُمْ عَازِمُونَ عَلَى إِخْضَاعِ بَنِي يَهُوذَا وَأُورُشَلِيمَ عَبِيداً وَإِمَاءً لَكُمْ. أَمَا عِنْدَكُمْ أَنْتُمْ آثَامٌ لِلرَّبِّ إِلَهِكُمْ؟
\par 11 وَالآنَ اسْمَعُوا لِي وَرُدُّوا السَّبْيَ الَّذِي سَبَيْتُمُوهُ مِنْ إِخْوَتِكُمْ لأَنَّ حُمُوَّ غَضَبِ الرَّبِّ عَلَيْكُمْ].
\par 12 ثُمَّ قَامَ رِجَالٌ مِنْ رُؤُوسِ بَنِي أَفْرَايِمَ: عَزَرْيَا بْنُ يَهُوحَانَانَ وَبَرَخْيَا بْنُ مَشُلِّيمُوتَ وَيَحَزْقِيَا بْنُ شَلُّومَ وَعَمَاسَا بْنُ حِدْلاَيَ عَلَى الْمُقْبِلِينَ مِنَ الْجَيْشِ
\par 13 وَقَالُوا لَهُمْ: [لاَ تَدْخُلُونَ بِالسَّبْيِ إِلَى هُنَا لأَنَّ عَلَيْنَا إِثْماً لِلرَّبِّ وَأَنْتُمْ عَازِمُونَ أَنْ تَزِيدُوا عَلَى خَطَايَانَا وَعَلَى إِثْمِنَا لأَنَّ لَنَا إِثْماً كَثِيراً وَعَلَى إِسْرَائِيلَ حُمُوُّ غَضَبٍ].
\par 14 فَتَرَكَ الْمُتَجَرِّدُونَ السَّبْيَ وَالنَّهْبَ أَمَامَ الرُّؤَسَاءِ وَكُلِّ الْجَمَاعَةِ.
\par 15 وَقَامَ الرِّجَالُ الْمُعَيَّنَةُ أَسْمَاؤُهُمْ وَأَخَذُوا الْمَسْبِيِّينَ وَأَلْبَسُوا كُلَّ عُرَاتِهِمْ مِنَ الْغَنِيمَةِ وَكَسُوهُمْ وَحَذُوهُمْ وَأَطْعَمُوهُمْ وَأَسْقُوهُمْ وَدَهَّنُوهُمْ وَحَمَلُوا عَلَى حَمِيرٍ جَمِيعَ الْمُعْيِينَ مِنْهُمْ وَأَتُوا بِهِمْ إِلَى أَرِيحَا مَدِينَةِ النَّخْلِ إِلَى إِخْوَتِهِمْ. ثُمَّ رَجَعُوا إِلَى السَّامِرَةِ.
\par 16 فِي ذَلِكَ الْوَقْتِ أَرْسَلَ الْمَلِكُ آحَازُ إِلَى مُلُوكِ أَشُّورَ لِيُسَاعِدُوهُ.
\par 17 فَإِنَّ الأَدُومِيِّينَ أَتُوا أَيْضاً وَضَرَبُوا يَهُوذَا وَسَبُوا سَبْياً.
\par 18 وَاقْتَحَمَ الْفِلِسْطِينِيُّونَ مُدُنَ السَّوَاحِلِ وَجَنُوبِيَّ يَهُوذَا وَأَخَذُوا بَيْتَ شَمْسٍ وَأَيَّلُونَ وَجَدِيرُوتَ وَسُوكُو وَقُرَاهَا وَتِمْنَةَ وَقُرَاهَا وَحِمْزُو وَقُرَاهَا وَسَكَنُوا هُنَاكَ.
\par 19 لأَنَّ الرَّبَّ ذَلَّلَ يَهُوذَا بِسَبَبِ آحَازَ مَلِكِ إِسْرَائِيلَ لأَنَّهُ أَجْمَحَ يَهُوذَا وَخَانَ الرَّبَّ خِيَانَةً.
\par 20 فَجَاءَ عَلَيْهِ تَغْلَثَ فَلاَسَرُ مَلِكُ أَشُّورَ وَضَايَقَهُ وَلَمْ يُشَدِّدْهُ.
\par 21 لأَنَّ آحَازَ أَخَذَ قِسْماً مِنْ بَيْتِ الرَّبِّ وَمِنْ بَيْتِ الْمَلِكِ وَمِنَ الرُّؤَسَاءِ وَأَعْطَاهُ لِمَلِكِ أَشُّورَ وَلَكِنَّهُ لَمْ يُسَاعِدْهُ.
\par 22 وَفِي ضِيقِهِ زَادَ خِيَانَةً لِلرَّبِّ (الْمَلِكُ آحَازُ هَذَا)
\par 23 وَذَبَحَ لآلِهَةِ دِمَشْقَ الَّذِينَ ضَارَبُوهُ وَقَالَ: [لأَنَّ آلِهَةَ مُلُوكِ أَرَامَ تُسَاعِدُهُمْ أَنَا أَذْبَحُ لَهُمْ فَيُسَاعِدُونَنِي]. وَأَمَّا هُمْ فَكَانُوا سَبَبَ سُقُوطٍ لَهُ وَلِكُلِّ إِسْرَائِيلَ.
\par 24 وَجَمَعَ آحَازُ آنِيَةَ بَيْتِ اللَّهِ وَقَطَّعَهَا وَأَغْلَقَ أَبْوَابَ بَيْتِ الرَّبِّ وَعَمِلَ لِنَفْسِهِ مَذَابِحَ فِي كُلِّ زَاوِيَةٍ فِي أُورُشَلِيمَ.
\par 25 وَفِي كُلِّ مَدِينَةٍ فَمَدِينَةٍ مِنْ يَهُوذَا عَمِلَ مُرْتَفَعَاتٍ لِلإِيقَادِ لآلِهَةٍ أُخْرَى وَأَسْخَطَ الرَّبَّ إِلَهَ آبَائِهِ.
\par 26 وَبَقِيَّةُ أُمُورِهِ وَكُلُّ طُرُقِهِ الأُولَى وَالأَخِيرَةُ مَكْتُوبَةٌ فِي سِفْرِ مُلُوكِ يَهُوذَا وَإِسْرَائِيلَ.
\par 27 ثُمَّ اضْطَجَعَ آحَازُ مَعَ آبَائِهِ فَدَفَنُوهُ فِي الْمَدِينَةِ فِي أُورُشَلِيمَ لأَنَّهُمْ لَمْ يَأْتُوا بِهِ إِلَى قُبُورِ مُلُوكِ إِسْرَائِيلَ. وَمَلَكَ حَزَقِيَّا ابْنُهُ عِوَضاً عَنْهُ.

\chapter{29}

\par 1 مَلَكَ حَزَقِيَّا وَهُوَ ابْنُ خَمْسٍ وَعِشْرِينَ سَنَةً وَمَلَكَ تِسْعاً وَعِشْرِينَ سَنَةً فِي أُورُشَلِيمَ وَاسْمُ أُمِّهِ أَبِيَّةُ بِنْتُ زَكَرِيَّا.
\par 2 وَعَمِلَ الْمُسْتَقِيمَ فِي عَيْنَيِ الرَّبِّ حَسَبَ كُلِّ مَا عَمِلَ دَاوُدُ أَبُوهُ.
\par 3 هُوَ فِي السَّنَةِ الأُولَى مِنْ مُلْكِهِ فِي الشَّهْرِ الأَوَّلِ فَتَحَ أَبْوَابَ بَيْتِ الرَّبِّ وَرَمَّمَهَا.
\par 4 وَأَدْخَلَ الْكَهَنَةَ وَاللاَّوِيِّينَ وَجَمَعَهُمْ إِلَى السَّاحَةِ الشَّرْقِيَّةِ
\par 5 وَقَالَ لَهُمُ: [اسْمَعُوا لِي أَيُّهَا اللاَّوِيُّونَ تَقَدَّسُوا الآنَ وَقَدِّسُوا بَيْتَ الرَّبِّ إِلَهِ آبَائِكُمْ وَأَخْرِجُوا النَّجَاسَةَ مِنَ الْقُدْسِ
\par 6 لأَنَّ آبَاءَنَا خَانُوا وَعَمِلُوا الشَّرَّ فِي عَيْنَيِ الرَّبِّ إِلَهِنَا وَتَرَكُوهُ وَحَوَّلُوا وُجُوهَهُمْ عَنْ مَسْكَنِ الرَّبِّ وَأَعْطُوا قَفاً
\par 7 وَأَغْلَقُوا أَيْضاً أَبْوَابَ الرِّوَاقِ وَأَطْفَأُوا السُّرُجَ وَلَمْ يُوقِدُوا بَخُوراً وَلَمْ يُصْعِدُوا مُحْرَقَةً فِي الْقُدْسِ لإِلَهِ إِسْرَائِيلَ.
\par 8 فَكَانَ غَضَبُ الرَّبِّ عَلَى يَهُوذَا وَأُورُشَلِيمَ وَأَسْلَمَهُمْ لِلْقَلَقِ وَالدَّهْشِ وَالصَّفِيرِ كَمَا أَنْتُمْ رَاؤُونَ بِأَعْيُنِكُمْ.
\par 9 وَهُوَذَا قَدْ سَقَطَ آبَاؤُنَا بِالسَّيْفِ وَبَنُونَا وَبَنَاتُنَا وَنِسَاؤُنَا فِي السَّبْيِ لأَجْلِ هَذَا.
\par 10 فَالآنَ فِي قَلْبِي أَنْ أَقْطَعَ عَهْداً مَعَ الرَّبِّ إِلَهِ إِسْرَائِيلَ فَيَرُدُّ عَنَّا حُمُوَّ غَضَبِهِ.
\par 11 يَا بَنِيَّ لاَ تَضِلُّوا الآنَ لأَنَّ الرَّبَّ اخْتَارَكُمْ لِتَقِفُوا أَمَامَهُ وَتَخْدِمُوهُ وَتَكُونُوا خَادِمِينَ وَمُوقِدِينَ لَهُ].
\par 12 فَقَامَ اللاَّوِيُّونَ مَحَثُ بْنُ عَمَاسَايَ وَيُوئِيلُ بْنُ عَزَرْيَا مِنْ بَنِي الْقَهَاتِيِّينَ وَمِنْ بَنِي مَرَارِي قَيْسُ بْنُ عَبْدِي وَعَزَرْيَا بْنُ يَهْلَلْئِيلَ وَمِنَ الْجَرْشُونِيِّينَ يُوآخُ بْنُ زِمَّةَ وَعِيدَنُ بْنُ يُوآخَ
\par 13 وَمِنْ بَنِي أَلِيصَافَانَ شِمْرِي وَيَعِيئِيلُ وَمِنْ بَنِي آسَافَ زَكَرِيَّا وَمَتَّنْيَا
\par 14 وَمِنْ بَنِي هَيْمَانَ يَحِيئِيلُ وَشَمْعِي وَمِنْ بَنِي يَدُوثُونَ شَمَعْيَا وَعُزِّيئِيلُ.
\par 15 وَجَمَعُوا إِخْوَتَهُمْ وَتَقَدَّسُوا وَأَتُوا حَسَبَ أَمْرِ الْمَلِكِ بِكَلاَمِ الرَّبِّ لِيُطَهِّرُوا بَيْتَ الرَّبِّ.
\par 16 وَدَخَلَ الْكَهَنَةُ إِلَى دَاخِلِ بَيْتِ الرَّبِّ لِيُطَهِّرُوهُ وَأَخْرَجُوا كُلَّ النَّجَاسَةِ الَّتِي وَجَدُوهَا فِي هَيْكَلِ الرَّبِّ إِلَى دَارِ بَيْتِ الرَّبِّ وَتَنَاوَلَهَا اللاَّوِيُّونَ لِيُخْرِجُوهَا إِلَى الْخَارِجِ إِلَى وَادِي قَدْرُونَ.
\par 17 وَشَرَعُوا فِي التَّقْدِيسِ فِي أَوَّلِ الشَّهْرِ الأَوَّلِ. وَفِي الْيَوْمِ الثَّامِنِ مِنَ الشَّهْرِ انْتَهُوا إِلَى رِوَاقِ الرَّبِّ وَقَدَّسُوا بَيْتَ الرَّبِّ فِي ثَمَانِيَةِ أَيَّامٍ وَفِي الْيَوْمِ السَّادِسَ عَشَرَ مِنَ الشَّهْرِ الأَوَّلِ انْتَهُوا.
\par 18 وَدَخَلُوا إِلَى دَاخِلٍ إِلَى حَزَقِيَّا الْمَلِكِ وَقَالُوا: [قَدْ طَهَّرْنَا كُلَّ بَيْتِ الرَّبِّ وَمَذْبَحَ الْمُحْرَقَةِ وَكُلَّ آنِيَتِهِ وَمَائِدَةَ خُبْزِ الْوُجُوهِ وَكُلَّ آنِيَتِهَا.
\par 19 وَجَمِيعُ الآنِيَةِ الَّتِي طَرَحَهَا الْمَلِكُ آحَازُ فِي مُلْكِهِ بِخِيَانَتِهِ قَدْ هَيَّأْنَاهَا وَقَدَّسْنَاهَا وَهَا هِيَ أَمَامَ مَذْبَحِ الرَّبِّ].
\par 20 وَبَكَّرَ حَزَقِيَّا الْمَلِكُ وَجَمَعَ رُؤَسَاءَ الْمَدِينَةِ وَصَعِدَ إِلَى بَيْتِ الرَّبِّ.
\par 21 فَأَتُوا بِسَبْعَةِ ثِيرَانٍ وَسَبْعَةِ كِبَاشٍ وَسَبْعَةِ خِرْفَانٍ وَسَبْعَةِ تُيُوسِ مِعْزًى ذَبِيحَةَ خَطِيَّةٍ عَنِ الْمَمْلَكَةِ وَعَنِ الْمَقْدِسِ وَعَنْ يَهُوذَا. وَقَالَ لِبَنِي هَارُونَ الْكَهَنَةِ أَنْ يُصْعِدُوهَا عَلَى مَذْبَحِ الرَّبِّ.
\par 22 فَذَبَحُوا الثِّيرَانَ وَتَنَاوَلَ الْكَهَنَةُ الدَّمَ وَرَشُّوهُ عَلَى الْمَذْبَحِ ثُمَّ ذَبَحُوا الْكِبَاشَ وَرَشُّوا الدَّمَ عَلَى الْمَذْبَحِ ثُمَّ ذَبَحُوا الْخِرْفَانَ وَرَشُّوا الدَّمَ عَلَى الْمَذْبَحِ.
\par 23 ثُمَّ تَقَدَّمُوا بِتُيُوسِ ذَبِيحَةِ الْخَطِيَّةِ أَمَامَ الْمَلِكِ وَالْجَمَاعَةِ وَوَضَعُوا أَيْدِيَهُمْ عَلَيْهَا
\par 24 وَذَبَحَهَا الْكَهَنَةُ وَكَفَّرُوا بِدَمِهَا عَلَى الْمَذْبَحِ تَكْفِيراً عَنْ جَمِيعِ إِسْرَائِيلَ لأَنَّ الْمَلِكَ قَالَ إِنَّ الْمُحْرَقَةَ وَذَبِيحَةَ الْخَطِيَّةِ هُمَا عَنْ كُلِّ إِسْرَائِيلَ.
\par 25 وَأَوْقَفَ اللاَّوِيِّينَ فِي بَيْتِ الرَّبِّ بِصُنُوجٍ وَرَبَابٍ وَعِيدَانٍ حَسَبَ أَمْرِ دَاوُدَ وَجَادَ رَائِي الْمَلِكِ وَنَاثَانَ النَّبِيِّ لأَنَّ مِنْ قِبَلِ الرَّبِّ الْوَصِيَّةَ عَنْ يَدِ أَنْبِيَائِهِ.
\par 26 فَوَقَفَ اللاَّوِيُّونَ بِآلاَتِ دَاوُدَ وَالْكَهَنَةُ بِالأَبْوَاقِ.
\par 27 وَأَمَرَ حَزَقِيَّا بِإِصْعَادِ الْمُحْرَقَةِ عَلَى الْمَذْبَحِ. وَعِنْدَ ابْتِدَاءِ الْمُحْرَقَةِ ابْتَدَأَ نَشِيدُ الرَّبِّ وَالأَبْوَاقُ بِوَاسِطَةِ آلاَتِ دَاوُدَ مَلِكِ إِسْرَائِيلَ.
\par 28 وَكَانَ كُلُّ الْجَمَاعَةِ يَسْجُدُونَ وَالْمُغَنُّونَ يُغَنُّونَ وَالْمُبَوِّقُونَ يُبَوِّقُونَ. الْجَمِيعُ إِلَى أَنِ انْتَهَتِ الْمُحْرَقَةُ.
\par 29 وَعِنْدَ انْتِهَاءِ الْمُحْرَقَةِ خَرَّ الْمَلِكُ وَكُلُّ الْمَوْجُودِينَ مَعَهُ وَسَجَدُوا.
\par 30 وَقَالَ حَزَقِيَّا الْمَلِكُ وَالرُّؤَسَاءُ لِلاَّوِيِّينَ أَنْ يُسَبِّحُوا الرَّبَّ بِكَلاَمِ دَاوُدَ وَآسَافَ الرَّائِي فَسَبَّحُوا بِابْتِهَاجٍ وَخَرُّوا وَسَجَدُوا.
\par 31 ثُمَّ قَالَ حَزَقِيَّا: [الآنَ مَلَأْتُمْ أَيْدِيَكُمْ لِلرَّبِّ. تَقَدَّمُوا وَأْتُوا بِذَبَائِحَ وَقَرَابِينِ شُكْرٍ لِبَيْتِ الرَّبِّ]. فَأَتَتِ الْجَمَاعَةُ بِذَبَائِحَِ وَقَرَابِينِ شُكْرٍ وَكُلُّ سَمُوحِ الْقَلْبِ أَتَى بِمُحْرَقَاتٍ.
\par 32 وَكَانَ عَدَدُ الْمُحْرَقَاتِ الَّتِي أَتَى بِهَا الْجَمَاعَةُ سَبْعِينَ ثَوْراً وَمِئَةَ كَبْشٍ وَمِئَتَيْ خَرُوفٍ. كُلُّ هَذِهِ مُحْرَقَةٌ لِلرَّبِّ.
\par 33 وَالأَقْدَاسُ سِتُّ مِئَةٍ مِنَ الْبَقَرِ وَثَلاَثَةُ آلاَفٍ مِنَ الضَّأْنِ.
\par 34 إِلاَّ إِنَّ الْكَهَنَةَ كَانُوا قَلِيلِينَ فَلَمْ يَقْدِرُوا أَنْ يَسْلُخُوا كُلَّ الْمُحْرَقَاتِ فَسَاعَدَهُمْ إِخْوَتُهُمُ اللاَّوِيُّونَ حَتَّى كَمِلَ الْعَمَلُ وَحَتَّى تَقَدَّسَ الْكَهَنَةُ. لأَنَّ اللاَّوِيِّينَ كَانُوا أَكْثَرَ اسْتِقَامَةَ قَلْبٍ مِنَ الْكَهَنَةِ فِي التَّقَدُّسِ.
\par 35 وَأَيْضاً كَانَتِ الْمُحْرَقَاتُ كَثِيرَةً بِشَحْمِ ذَبَائِحِ السَّلاَمَةِ وَسَكَائِبِ الْمُحْرَقَاتِ. فَاسْتَقَامَتْ خِدْمَةُ بَيْتِ الرَّبِّ.
\par 36 وَفَرِحَ حَزَقِيَّا وَكُلُّ الشَّعْبِ مِنْ أَجْلِ أَنَّ اللَّهَ أَعَدَّ الشَّعْبَ لأَنَّ الأَمْرَ كَانَ بَغْتَةً.

\chapter{30}

\par 1 وَأَرْسَلَ حَزَقِيَّا إِلَى جَمِيعِ إِسْرَائِيلَ وَيَهُوذَا وَكَتَبَ أَيْضاً رَسَائِلَ إِلَى أَفْرَايِمَ وَمَنَسَّى أَنْ يَأْتُوا إِلَى بَيْتِ الرَّبِّ فِي أُورُشَلِيمَ لِيَعْمَلُوا فِصْحاً لِلرَّبِّ إِلَهِ إِسْرَائِيلَ.
\par 2 فَتَشَاوَرَ الْمَلِكُ وَرُؤَسَاؤُهُ وَكُلُّ الْجَمَاعَةِ فِي أُورُشَلِيمَ أَنْ يَعْمَلُوا الْفِصْحَ فِي الشَّهْرِ الثَّانِي
\par 3 لأَنَّهُمْ لَمْ يَقْدِرُوا أَنْ يَعْمَلُوهُ فِي ذَلِكَ الْوَقْتِ لأَنَّ الْكَهَنَةَ لَمْ يَتَقَدَّسُوا بِالْكِفَايَةِ وَالشَّعْبَ لَمْ يَجْتَمِعُوا إِلَى أُورُشَلِيمَ.
\par 4 فَحَسُنَ الأَمْرُ فِي عَيْنَيِ الْمَلِكِ وَعُيُونِ كُلِّ الْجَمَاعَةِ.
\par 5 فَاعْتَمَدُوا عَلَى إِطْلاَقِ النِّدَاءِ فِي جَمِيعِ إِسْرَائِيلَ مِنْ بِئْرِ سَبْعٍ إِلَى دَانَ أَنْ يَأْتُوا لِعَمَلِ الْفِصْحِ لِلرَّبِّ إِلَهِ إِسْرَائِيلَ فِي أُورُشَلِيمَ لأَنَّهُمْ لَمْ يَعْمَلُوهُ كَمَا هُوَ مَكْتُوبٌ مُنْذُ زَمَانٍ كَثِيرٍ.
\par 6 فَذَهَبَ السُّعَاةُ بِالرَّسَائِلِ مِنْ يَدِ الْمَلِكِ وَرُؤَسَائِهِ فِي جَمِيعِ إِسْرَائِيلَ وَيَهُوذَا وَحَسَبَ وَصِيَّةِ الْمَلِكِ كَانُوا يَقُولُونَ: [يَا بَنِي إِسْرَائِيلَ ارْجِعُوا إِلَى الرَّبِّ إِلَهِ إِبْرَاهِيمَ وَإِسْحَاقَ وَإِسْرَائِيلَ فَيَرْجِعَ إِلَى النَّاجِينَ الْبَاقِينَ لَكُمْ مِنْ يَدِ مُلُوكِ أَشُّورَ.
\par 7 وَلاَ تَكُونُوا كَآبَائِكُمْ وَكَإِخْوَتِكُمُ الَّذِينَ خَانُوا الرَّبَّ إِلَهَ آبَائِهِمْ فَجَعَلَهُمْ دَهْشَةً كَمَا أَنْتُمْ تَرُونَ.
\par 8 الآنَ لاَ تُصَلِّبُوا رِقَابَكُمْ كَآبَائِكُمْ بَلِ اخْضَعُوا لِلرَّبِّ وَادْخُلُوا مَقْدِسَهُ الَّذِي قَدَّسَهُ إِلَى الأَبَدِ وَاعْبُدُوا الرَّبَّ إِلَهَكُمْ فَيَرْتَدَّ عَنْكُمْ حُمُوُّ غَضَبِهِ.
\par 9 لأَنَّهُ بِرُجُوعِكُمْ إِلَى الرَّبِّ يَجِدُ إِخْوَتُكُمْ وَبَنُوكُمْ رَحْمَةً أَمَامَ الَّذِينَ يَسْبُونَهُمْ فَيَرْجِعُونَ إِلَى هَذِهِ الأَرْضِ لأَنَّ الرَّبَّ إِلَهَكُمْ حَنَّانٌ وَرَحِيمٌ وَلاَ يُحَوِّلُ وَجْهَهُ عَنْكُمْ إِذَا رَجَعْتُمْ إِلَيْهِ].
\par 10 فَكَانَ السُّعَاةُ يَعْبُرُونَ مِنْ مَدِينَةٍ إِلَى مَدِينَةٍ فِي أَرْضِ أَفْرَايِمَ وَمَنَسَّى حَتَّى زَبُولُونَ فَكَانُوا يَضْحَكُونَ عَلَيْهِمْ وَيَهْزَأُونَ بِهِمْ.
\par 11 إِلاَّ إِنَّ قَوْماً مِنْ أَشِيرَ وَمَنَسَّى وَزَبُولُونَ تَوَاضَعُوا وَأَتُوا إِلَى أُورُشَلِيمَ.
\par 12 وَكَانَتْ يَدُ اللَّهِ فِي يَهُوذَا أَيْضاً فَأَعْطَاهُمْ قَلْباً وَاحِداً لِيَعْمَلُوا بِأَمْرِ الْمَلِكِ وَالرُّؤَسَاءِ حَسَبَ قَوْلِ الرَّبِّ.
\par 13 فَاجْتَمَعَ إِلَى أُورُشَلِيمَ شَعْبٌ كَثِيرٌ لِعَمَلِ عِيدِ الْفَطِيرِ فِي الشَّهْرِ الثَّانِي جَمَاعَةٌ كَثِيرَةٌ جِدّاً.
\par 14 وَقَامُوا وَأَزَالُوا الْمَذَابِحَ الَّتِي فِي أُورُشَلِيمَ وَأَزَالُوا كُلَّ مَذَابِحِ التَّبْخِيرِ وَطَرَحُوهَا إِلَى وَادِي قَدْرُونَ.
\par 15 وَذَبَحُوا الْفِصْحَ فِي الرَّابِعِ عَشَرَ مِنَ الشَّهْرِ الثَّانِي. وَالْكَهَنَةُ وَاللاَّوِيُّونَ خَجِلُوا وَتَقَدَّسُوا وَأَدْخَلُوا الْمُحْرَقَاتِ إِلَى بَيْتِ الرَّبِّ
\par 16 وَأَقَامُوا عَلَى مَقَامِهِمْ حَسَبَ حُكْمِهِمْ كَنَامُوسِ مُوسَى رَجُلِ اللَّهِ. كَانَ الْكَهَنَةُ يَرُشُّونَ الدَّمَ مِنْ يَدِ اللاَّوِيِّينَ.
\par 17 لأَنَّهُ كَانَ كَثِيرُونَ فِي الْجَمَاعَةِ لَمْ يَتَقَدَّسُوا فَكَانَ اللاَّوِيُّونَ عَلَى ذَبْحِ الْفِصْحِ عَنْ كُلِّ مَنْ لَيْسَ بِطَاهِرٍ لِتَقْدِيسِهِمْ لِلرَّبِّ.
\par 18 لأَنَّ كَثِيرِينَ مِنَ الشَّعْبِ كَثِيرِينَ مِنْ أَفْرَايِمَ وَمَنَسَّى وَيَسَّاكَرَ وَزَبُولُونَ لَمْ يَتَطَهَّرُوا بَلْ أَكَلُوا الْفِصْحَ لَيْسَ كَمَا هُوَ مَكْتُوبٌ. إِلاَّ إِنَّ حَزَقِيَّا صَلَّى عَنْهُمْ قَائِلاً: [الرَّبُّ الصَّالِحُ يُكَفِّرُ عَنْ
\par 19 كُلِّ مَنْ هَيَّأَ قَلْبَهُ لِطَلَبِ اللَّهِ الرَّبِّ إِلَهِ آبَائِهِ وَلَيْسَ كَطَهَارَةِ الْقُدْسِ].
\par 20 فَسَمِعَ الرَّبُّ لِحَزَقِيَّا وَشَفَى الشَّعْبَ.
\par 21 وَعَمِلَ بَنُو إِسْرَائِيلَ الْمَوْجُودُونَ فِي أُورُشَلِيمَ عِيدَ الْفَطِيرِ سَبْعَةَ أَيَّامٍ بِفَرَحٍ عَظِيمٍ وَكَانَ اللاَّوِيُّونَ وَالْكَهَنَةُ يُسَبِّحُونَ الرَّبَّ يَوْماً فَيَوْماً بِآلاَتِ حَمْدٍ لِلرَّبِّ.
\par 22 وَطَيَّبَ حَزَقِيَّا قُلُوبَ جَمِيعِ اللاَّوِيِّينَ الْفَطِنِينَ فِطْنَةً صَالِحَةً لِلرَّبِّ وَأَكَلُوا الْمَوْسِمَ سَبْعَةَ أَيَّامٍ يَذْبَحُونَ ذَبَائِحَ سَلاَمَةٍ وَيَحْمَدُونَ الرَّبَّ إِلَهَ آبَائِهِمْ.
\par 23 وَتَشَاوَرَ كُلُّ الْجَمَاعَةِ أَنْ يَعْمَلُوا سَبْعَةَ أَيَّامٍ أُخْرَى فَعَمِلُوا سَبْعَةَ أَيَّامٍ بِفَرَحٍ.
\par 24 لأَنَّ حَزَقِيَّا مَلِكَ يَهُوذَا قَدَّمَ لِلْجَمَاعَةِ أَلْفَ ثَوْرٍ وَسَبْعَةَ آلاَفٍ مِنَ الضَّأْنِ وَالرُّؤَسَاءُ قَدَّمُوا لِلْجَمَاعَةِ أَلْفَ ثَوْرٍ وَعَشَرَةَ آلاَفٍ مِنَ الضَّأْنِ وَتَقَدَّسَ كَثِيرُونَ مِنَ الْكَهَنَةِ.
\par 25 وَفَرِحَ كُلُّ جَمَاعَةِ يَهُوذَا وَالْكَهَنَةُ وَاللاَّوِيُّونَ وَكُلُّ الْجَمَاعَةِ الآتِينَ مِنْ إِسْرَائِيلَ وَالْغُرَبَاءُ الآتُونَ مِنْ أَرْضِ إِسْرَائِيلَ وَالسَّاكِنُونَ فِي يَهُوذَا.
\par 26 وَكَانَ فَرَحٌ عَظِيمٌ فِي أُورُشَلِيمَ لأَنَّهُ مِنْ أَيَّامِ سُلَيْمَانَ بْنِ دَاوُدَ مَلِكِ إِسْرَائِيلَ لَمْ يَكُنْ كَهَذَا فِي أُورُشَلِيمَ.
\par 27 وَقَامَ الْكَهَنَةُ اللاَّوِيُّونَ وَبَارَكُوا الشَّعْبَ فَسُمِعَ صَوْتُهُمْ وَدَخَلَتْ صَلاَتُهُمْ إِلَى مَسْكَنِ قُدْسِهِ إِلَى السَّمَاءِ.

\chapter{31}

\par 1 وَلَمَّا كَمِلَ هَذَا خَرَجَ كُلُّ إِسْرَائِيلَ الْحَاضِرِينَ إِلَى مُدُنِ يَهُوذَا وَكَسَّرُوا الأَنْصَابَ وَقَطَعُوا السَّوَارِيَ وَهَدَمُوا الْمُرْتَفَعَاتِ وَالْمَذَابِحَ مِنْ كُلِّ يَهُوذَا وَبِنْيَامِينَ وَمِنْ أَفْرَايِمَ وَمَنَسَّى حَتَّى أَفْنُوهَا ثُمَّ رَجَعَ كُلُّ إِسْرَائِيلَ كُلُّ وَاحِدٍ إِلَى مُلْكِهِ إِلَى مُدُنِهِمْ.
\par 2 وَأَقَامَ حَزَقِيَّا فِرَقَ الْكَهَنَةِ وَاللاَّوِيِّينَ حَسَبَ أَقْسَامِهِمْ كُلُّ وَاحِدٍ حَسَبَ خِدْمَتِهِ: الْكَهَنَةَ وَاللاَّوِيِّينَ لِلْمُحْرَقَاتِ وَذَبَائِحِ السَّلاَمَةِ لِلْخِدْمَةِ وَالْحَمْدِ وَالتَّسْبِيحِ فِي أَبْوَابِ مَحَلاَّتِ الرَّبِّ.
\par 3 وَأَعْطَى الْمَلِكُ حِصَّةً مِنْ مَالِهِ لِلْمُحْرَقَاتِ مُحْرَقَاتِ الصَّبَاحِ وَالْمَسَاءِ وَالْمُحْرَقَاتِ لِلسُّبُوتِ وَالأَشْهُرِ وَالْمَوَاسِمِ كَمَا هُوَ مَكْتُوبٌ فِي شَرِيعَةِ الرَّبِّ.
\par 4 وَقَالَ لِلشَّعْبِ سُكَّانِ أُورُشَلِيمَ أَنْ يُعْطُوا حِصَّةَ الْكَهَنَةِ وَاللاَّوِيِّينَ لِيَتَمَسَّكُوا بِشَرِيعَةِ الرَّبِّ.
\par 5 وَلَمَّا شَاعَ الأَمْرُ كَثَّرَ بَنُو إِسْرَائِيلَ مِنْ أَوَائِلِ الْحِنْطَةِ وَالْمُِسْطَارِ وَالزَّيْتِ وَالْعَسَلِ وَمِنْ كُلِّ غَلَّةِ الْحَقْلِ وَأَتُوا بِعُشْرِ الْجَمِيعِ بِكَِثْرَةٍ.
\par 6 وَبَنُو إِسْرَائِيلَ وَيَهُوذَا السَّاكِنُونَ فِي مُدُنِ يَهُوذَا أَتُوا هُمْ أَيْضاً بِعُشْرِ الْبَقَرِ وَالضَّأْنِ وَعُشْرِ الأَقْدَاسِ الْمُقَدَّسَةِ لِلرَّبِّ إِلَهِهِمْ وَجَعَلُوهَا كَوْمَةً كَوْمَةً.
\par 7 فِي الشَّهْرِ الثَّالِثِ ابْتَدَأُوا بِتَأْسِيسِ الْكُوَمِ وَفِي الشَّهْرِ السَّابِعِ أَكْمَلُوا.
\par 8 وَجَاءَ حَزَقِيَّا وَالرُّؤَسَاءُ وَرَأُوا الْكُوَمَ فَبَارَكُوا الرَّبَّ وَشَعْبَهُ إِسْرَائِيلَ.
\par 9 وَسَأَلَ حَزَقِيَّا الْكَهَنَةَ وَاللاَّوِيِّينَ عَنِ الْكُوَمِ
\par 10 فَأَجَابَ عَزَرْيَا الْكَاهِنُ الرَّأْسُ لِبَيْتِ صَادُوقَ: [مُنْذُ ابْتَدَأَ بِجَلْبِ التَّقْدِمَةِ إِلَى بَيْتِ الرَّبِّ أَكَلْنَا وَشَبِعْنَا وَفَضَلَ عَنَّا بِكَِثْرَةٍ لأَنَّ الرَّبَّ بَارَكَ شَعْبَهُ وَالَّذِي فَضَلَ هُوَ هَذِهِ الْكَثْرَةُ].
\par 11 وَأَمَرَ حَزَقِيَّا بِإِعْدَادِ مَخَادِعَ فِي بَيْتِ الرَّبِّ فَأَعَدُّوا.
\par 12 وَأَتُوا بِالتَّقْدِمَةِ وَالْعُشْرِ وَالأَقْدَاسِ بِأَمَانَةٍ. وَكَانَ رَئِيساً عَلَيْهِمْ كُونَنْيَا اللاَّوِيُّ وَشَمْعِي أَخُوهُ الثَّانِي
\par 13 وَيَحِيئِيلُ وَعَزَزْيَا وَنَحَثُ وَعَسَائِيلُ وَيَرِيمُوثُ وَيُوزَابَادُ وَإِيلِيئِيلُ وَيَسْمَخْيَا وَمَحَثُ وَبَنَايَا وُكَلاَءَ تَحْتَ يَدِ كُونَنْيَا وَشَمْعِي أَخِيهِ حَسَبَ تَعْيِينِ حَزَقِيَّا الْمَلِكِ وَعَزَرْيَا رَئِيسِ بَيْتِ اللَّهِ.
\par 14 وَقُورِي بْنُ يَمْنَةَ اللاَّوِيُّ الْبَوَّابُ نَحْوَ الشَّرْقِ كَانَ عَلَى الْمُتَبَرَّعِ بِهِ لِلَّهِ لإِعْطَاءِ تَقْدِمَةِ الرَّبِّ وَأَقْدَاسِ الأَقْدَاسِ.
\par 15 وَتَحْتَ يَدِهِ: عَدْنُ وَبِنْيَامِينُ وَيَشُوعُ وَشَمَعْيَا وَأَمَرْيَا وَشَكَنْيَا فِي مُدُنِ الْكَهَنَةِ بِأَمَانَةٍ لِيُعْطُوا لإِخْوَتِهِمْ حَسَبَ الْفِرَقِ الْكَبِيرِ كَالصَّغِيرِ
\par 16 فَضْلاً عَنِ انْتِسَابِ ذُكُورِهِمْ مِنِ ابْنِ ثَلاَثِ سِنِينٍَ فَمَا فَوْقُ مِنْ كُلِّ دَاخِلٍ بَيْتَ الرَّبِّ أَمْرَ كُلِّ يَوْمٍ بِيَوْمِهِ حَسَبَ خِدْمَتِهِمْ فِي حِرَاسَاتِهِمْ حَسَبَ أَقْسَامِهِمْ
\par 17 وَانْتِسَابِ الْكَهَنَةِ حَسَبَ بُيُوتِ آبَائِهِمْ وَاللاَّوِيِّينَ مِنِ ابْنِ عِشْرِينَ سَنَةً فَمَا فَوْقُ حَسَبَ حَرَاسَاتِهِمْ وَأَقْسَامِهِمْ
\par 18 وَانْتِسَابِ جَمِيعِ أَطْفَالِهِمْ وَنِسَائِهِمْ وَبَنِيهِمْ وَبَنَاتِهِمْ فِي كُلِّ الْجَمَاعَةِ لأَنَّهُمْ بِأَمَانَتِهِمْ تَقَدَّسُوا تَقَدُّساً.
\par 19 وَمِنْ بَنِي هَارُونَ الْكَهَنَةِ فِي حُقُولِ مَرَاعِي مُدُنِهِمْ فِي كُلِّ مَدِينَةٍ فَمَدِينَةٍ الرِّجَالُ الْمُعَيَّنَةُ أَسْمَاؤُهُمْ لإِعْطَاءِ حِصَصٍ لِكُلِّ ذَكَرٍ مِنَ الْكَهَنَةِ وَلِكُلِّ مَنِ انْتَسَبَ مِنَ اللاَّوِيِّينَ.
\par 20 هَكَذَا عَمِلَ حَزَقِيَّا فِي كُلِّ يَهُوذَا وَعَمِلَ مَا هُوَ صَالِحٌ وَمُسْتَقِيمٌ وَحَقٌّ أَمَامَ الرَّبِّ إِلَهِهِ.
\par 21 وَكُلُّ عَمَلٍ ابْتَدَأَ بِهِ فِي خِدْمَةِ بَيْتِ اللَّهِ وَفِي الشَّرِيعَةِ وَالْوَصِيَّةِ لِيَطْلُبَ إِلَهَهُ إِنَّمَا عَمِلَهُ بِكُلِّ قَلْبِهِ وَأَفْلَحَ.

\chapter{32}

\par 1 وَبَعْدَ هَذِهِ الأُمُورِ وَهَذِهِ الأَمَانَةِ أَتَى سَنْحَارِيبُ مَلِكُ أَشُّورَ وَدَخَلَ يَهُوذَا وَنَزَلَ عَلَى الْمُدُنِ الْحَصِينَةِ وَطَمِعَ بِإِخْضَاعِهَا لِنَفْسِهِ.
\par 2 وَلَمَّا رَأَى حَزَقِيَّا أَنَّ سَنْحَارِيبَ قَدْ أَتَى وَوَجْهُهُ عَلَى مُحَارَبَةِ أُورُشَلِيمَ
\par 3 تَشَاوَرَ هُوَ وَرُؤَسَاؤُهُ وَجَبَابِرَتُهُ عَلَى طَمِّ مِيَاهِ الْعُيُونِ الَّتِي هِيَ خَارِجَ الْمَدِينَةِ فَسَاعَدُوهُ.
\par 4 فَتَجَمَّعَ شَعْبٌ كَثِيرٌ وَطَمُّوا جَمِيعَ الْيَنَابِيعِ وَالنَّهْرَ الْجَارِيَ فِي وَسَطِ الأَرْضِ قَائِلِينَ: [لِمَاذَا يَأْتِي مُلُوكُ أَشُّورَ وَيَجِدُونَ مِيَاهاً غَزِيرَةً؟]
\par 5 وَتَشَدَّدَ وَبَنَى كُلَّ السُّورِ الْمُنْهَدِمِ وَأَعْلاَهُ إِلَى الأَبْرَاجِ وَسُوراً آخَرَ خَارِجاً وَحَصَّنَ الْقَلْعَةَ مَدِينَةَ دَاوُدَ وَعَمِلَ سِلاَحاً بِكَِثْرَةٍ وَأَتْرَاساً.
\par 6 وَجَعَلَ رُؤَسَاءَ قِتَالٍ عَلَى الشَّعْبِ وَجَمَعَهُمْ إِلَيْهِ إِلَى سَاحَةِ بَابِ الْمَدِينَةِ وَطَيَّبَ قُلُوبَهُمْ قَائِلاً:
\par 7 [تَشَدَّدُوا وَتَشَجَّعُوا. لاَ تَخَافُوا وَلاَ تَرْتَاعُوا مِنْ مَلِكِ أَشُّورَ وَمِنْ كُلِّ الْجُمْهُورِ الَّذِي مَعَهُ لأَنَّ مَعَنَا أَكْثَرَ مِمَّا مَعَهُ.
\par 8 مَعَهُ ذِرَاعُ بَشَرٍ وَمَعَنَا الرَّبُّ إِلَهُنَا لِيُسَاعِدَنَا وَيُحَارِبَ حُرُوبَنَا]. فَاسْتَنَدَ الشَّعْبُ عَلَى كَلاَمِ حَزَقِيَّا مَلِكِ يَهُوذَا.
\par 9 بَعْدَ هَذَا أَرْسَلَ سَنْحَارِيبُ مَلِكُ أَشُّورَ عَبِيدَهُ إِلَى أُورُشَلِيمَ. (وَهُوَ عَلَى لَخِيشَ وَكُلُّ سَلْطَنَتِهِ مَعَهُ) إِلَى حَزَقِيَّا مَلِكِ يَهُوذَا وَإِلَى كُلِّ يَهُوذَا الَّذِينَ فِي أُورُشَلِيمَ يَقُولُونَ:
\par 10 [هَكَذَا يَقُولُ سَنْحَارِيبُ مَلِكُ أَشُّورَ: عَلَى مَاذَا تَتَّكِلُونَ وَتُقِيمُونَ فِي الْحِصَارِ فِي أُورُشَلِيمَ؟
\par 11 أَلَيْسَ حَزَقِيَّا يُغْوِيكُمْ لِيَدْفَعَكُمْ لِلْمَوْتِ بِالْجُوعِ وَالْعَطَشِ قَائِلاً: الرَّبُّ إِلَهُنَا يُنْقِذُنَا مِنْ يَدِ مَلِكِ أَشُّورَ.
\par 12 أَلَيْسَ حَزَقِيَّا هُوَ الَّذِي أَزَالَ مُرْتَفَعَاتِهِ وَمَذَابِحَهُ وَقَالَ لِيَهُوذَا وَأُورُشَلِيمَ: أَمَامَ مَذْبَحٍ وَاحِدٍ تَسْجُدُونَ وَعَلَيْهِ تُوقِدُونَ؟
\par 13 أَمَا تَعْلَمُونَ مَا فَعَلْتُهُ أَنَا وَآبَائِي بِجَمِيعِ شُعُوبِ الأَرَاضِي؟ فَهَلْ قَدِرَتْ آلِهَةُ أُمَمِ الأَرَاضِي أَنْ تُنْقِذَ أَرْضَهَا مِنْ يَدِي.
\par 14 مَنْ مِنْ جَمِيعِ آلِهَةِ هَؤُلاَءِ الأُمَمِ الَّذِينَ حَرَّمَهُمْ آبَائِي اسْتَطَاعَ أَنْ يُنْقِذَ شَعْبَهُ مِنْ يَدِي حَتَّى يَسْتَطِيعَ إِلَهُكُمْ أَنْ يُنْقِذَكُمْ مِنْ يَدِي؟
\par 15 وَالآنَ لاَ يَخْدَعَنَّكُمْ حَزَقِيَّا وَلاَ يُغْوِيَنَّكُمْ هَكَذَا وَلاَ تُصَدِّقُوهُ لأَنَّهُ لَمْ يَقْدِرْ إِلَهُ أُمَّةٍ أَوْ مَمْلَكَةٍ أَنْ يُنْقِذَ شَعْبَهُ مِنْ يَدِي وَيَدِ آبَائِي. فَكَمْ بِالْحَرِيِّ إِلَهُكُمْ لاَ يُنْقِذُكُمْ مِنْ يَدِي!].
\par 16 وَتَكَلَّمَ عَبِيدُهُ أَكْثَرَ ضِدَّ الرَّبِّ الإِلَهِ وَضِدَّ حَزَقِيَّا عَبْدِهِ.
\par 17 وَكَتَبَ رَسَائِلَ لِتَعْيِيرِ الرَّبِّ إِلَهِ إِسْرَائِيلَ وَلِلتَّكَلُّمِ ضِدَّهُ قَائِلاً: [كَمَا أَنَّ آلِهَةَ أُمَمِ الأَرَاضِي لَمْ تُنْقِذْ شُعُوبَهَا مِنْ يَدِي كَذَلِكَ لاَ يُنْقِذُ إِلَهُ حَزَقِيَّا شَعْبَهُ مِنْ يَدِي].
\par 18 وَصَرَخُوا بِصَوْتٍ عَظِيمٍ بِالْيَهُودِيِّ إِلَى شَعْبِ أُورُشَلِيمَ الَّذِينَ عَلَى السُّورِ لِتَخْوِيفِهِمْ وَتَرْوِيعِهِمْ لِيَأْخُذُوا الْمَدِينَةَ.
\par 19 وَتَكَلَّمُوا عَلَى إِلَهِ أُورُشَلِيمَ كَمَا عَلَى آلِهَةِ شُعُوبِ الأَرْضِ صَنْعَةِ أَيْدِي النَّاسِ.
\par 20 فَصَلَّى حَزَقِيَّا الْمَلِكُ وَإِشَعْيَاءُ بْنُ آمُوصَ النَّبِيُّ لِذَلِكَ وَصَرَخَا إِلَى السَّمَاءِ
\par 21 فَأَرْسَلَ الرَّبُّ مَلاَكاً فَأَبَادَ كُلَّ جَبَّارِ بَأْسٍ وَرَئِيسٍ وَقَائِدٍ فِي مَحَلَّةِ مَلِكِ أَشُّورَ. فَرَجَعَ بِخِزْيِ الْوَجْهِ إِلَى أَرْضِهِ. وَلَمَّا دَخَلَ بَيْتَ إِلَهِهِ قَتَلَهُ هُنَاكَ بِالسَّيْفِ الَّذِينَ خَرَجُوا مِنْ أَحْشَائِهِ.
\par 22 وَخَلَّصَ الرَّبُّ حَزَقِيَّا وَسُكَّانَ أُورُشَلِيمَ مِنْ سَنْحَارِيبَ مَلِكِ أَشُّورَ وَمِنْ يَدِ الْجَمِيعِ وَحَمَاهُمْ مِنْ كُلِّ نَاحِيَةٍ.
\par 23 وَكَانَ كَثِيرُونَ يَأْتُونَ بِتَقْدِمَاتِ الرَّبِّ إِلَى أُورُشَلِيمَ وَتُحَفٍ لِحَزَقِيَّا مَلِكِ يَهُوذَا وَاعْتُبِرَ فِي أَعْيُنِ جَمِيعِ الأُمَمِ بَعْدَ ذَلِكَ.
\par 24 فِي تِلْكَ الأَيَّامِ مَرِضَ حَزَقِيَّا إِلَى حَدِّ الْمَوْتِ وَصَلَّى إِلَى الرَّبِّ فَكَلَّمَهُ وَأَعْطَاهُ عَلاَمَةً.
\par 25 وَلَكِنْ لَمْ يَرُدَّ حَزَقِيَّا حَسْبَمَا أُنْعِمَ عَلَيْهِ لأَنَّ قَلْبَهُ ارْتَفَعَ فَكَانَ غَضَبٌ عَلَيْهِ وَعَلَى يَهُوذَا وَأُورُشَلِيمَ.
\par 26 ثُمَّ تَوَاضَعَ حَزَقِيَّا بِسَبَبِ ارْتِفَاعِ قَلْبِهِ هُوَ وَسُكَّانُ أُورُشَلِيمَ فَلَمْ يَأْتِ عَلَيْهِمْ غَضَبُ الرَّبِّ فِي أَيَّامِ حَزَقِيَّا.
\par 27 وَكَانَ لِحَزَقِيَّا غِنًى وَكَرَامَةٌ كَثِيرَةٌ جِدّاً وَعَمِلَ لِنَفْسِهِ خَزَائِنَ لِلْفِضَّةِ وَالذَّهَبِ وَالْحِجَارَةِ الْكَرِيمَةِ وَالأَطْيَابِ وَالأَتْرَاسِ وَكُلِّ آنِيَةٍ ثَمِينَةٍ
\par 28 وَمَخَازِنَ لِغَلَّةِ الْحِنْطَةِ وَالْمِسْطَارِ وَالزَّيْتِ وَإِسْطَبْلاَتٍ لِكُلِّ أَنْوَاعِ الْبَهَائِمِ وَلِلْقُطْعَانِ.
\par 29 وَعَمِلَ لِنَفْسِهِ أَبْرَاجاً وَمَوَاشِيَ غَنَمٍ وَبَقَرٍ بِكَثْرَةٍ لأَنَّ اللَّهَ أَعْطَاهُ أَمْوَالاً كَثِيرَةً جِدّاً.
\par 30 وَحَزَقِيَّا هَذَا سَدَّ مَخْرَجَ مِيَاهِ جَيْحُونَ الأَعْلَى وَأَجْرَاهَا تَحْتَ الأَرْضِ إِلَى الْجِهَةِ الْغَرْبِيَّةِ مِنْ مَدِينَةِ دَاوُدَ. وَأَفْلَحَ حَزَقِيَّا فِي كُلِّ عَمَلِهِ.
\par 31 وَهَكَذَا فِي أَمْرِ سُفَرَاءِ رُؤَسَاءِ بَابِلَ الَّذِينَ أَرْسَلُوا إِلَيْهِ لِيَسْأَلُوا عَنِ الأُعْجُوبَةِ الَّتِي كَانَتْ فِي الأَرْضِ تَرَكَهُ اللَّهُ لِيُجَرِّبَهُ لِيَعْلَمَ كُلَّ مَا فِي قَلْبِهِ.
\par 32 وَبَقِيَّةُ أُمُورِ حَزَقِيَّا وَمَرَاحِمُهُ مَكْتُوبَةٌ فِي رُؤْيَا إِشَعْيَاءَ بْنِ آمُوصَ النَّبِيِّ فِي سِفْرِ مُلُوكِ يَهُوذَا وَإِسْرَائِيلَ.
\par 33 ثُمَّ اضْطَجَعَ حَزَقِيَّا مَعَ آبَائِهِ فَدَفَنُوهُ فِي عَقَبَةِ قُبُورِ بَنِي دَاوُدَ وَعَمِلَ لَهُ إِكْرَاماً عِنْدَ مَوْتِهِ كُلُّ يَهُوذَا وَسُكَّانِ أُورُشَلِيمَ. وَمَلَكَ مَنَسَّى ابْنُهُ عِوَضاً عَنْهُ.

\chapter{33}

\par 1 كَانَ مَنَسَّى ابْنَ اثْنَتَيْ عَشْرَةَ سَنَةً حِينَ مَلَكَ وَمَلَكَ خَمْساً وَخَمْسِينَ سَنَةً فِي أُورُشَلِيمَ.
\par 2 وَعَمِلَ الشَّرَّ فِي عَيْنَيِ الرَّبِّ حَسَبَ رَجَاسَاتِ الأُمَمِ الَّذِينَ طَرَدَهُمُ الرَّبُّ مِنْ أَمَامِ بَنِي إِسْرَائِيلَ.
\par 3 وَعَادَ فَبَنَى الْمُرْتَفَعَاتِ الَّتِي هَدَمَهَا حَزَقِيَّا أَبُوهُ وَأَقَامَ مَذَابِحَ لِلْبَعْلِيمِ وَعَمِلَ سَوَارِيَ وَسَجَدَ لِكُلِّ جُنْدِ السَّمَاءِ وَعَبَدَهَا.
\par 4 وَبَنَى مَذَابِحَ فِي بَيْتِ الرَّبِّ الَّذِي قَالَ عَنْهُ الرَّبُّ [فِي أُورُشَلِيمَ يَكُونُ اسْمِي إِلَى الأَبَدِ].
\par 5 وَبَنَى مَذَابِحَ لِكُلِّ جُنْدِ السَّمَاءِ فِي دَارَيْ بَيْتِ الرَّبِّ.
\par 6 وَعَبَّرَ بَنِيهِ فِي النَّارِ فِي وَادِي ابْنِ هِنُّومَ وَعَافَ وَتَفَاءَلَ وَسَحَرَ وَاسْتَخْدَمَ جَانّاً وَتَابِعَةً وَأَكْثَرَ عَمَلَ الشَّرِّ فِي عَيْنَيِ الرَّبِّ لإِغَاظَتِهِ.
\par 7 وَوَضَعَ تِمْثَالَ الشَّكْلِ الَّذِي عَمِلَهُ فِي بَيْتِ اللَّهِ الَّذِي قَالَ اللَّهُ عَنْهُ لِدَاوُدَ وَلِسُلَيْمَانَ ابْنِهِ [فِي هَذَا الْبَيْتِ وَفِي أُورُشَلِيمَ الَّتِي اخْتَرْتُ مِنْ جَمِيعِ أَسْبَاطِ إِسْرَائِيلَ أَضَعُ اسْمِي إِلَى الأَبَدِ.
\par 8 وَلاَ أَعُودُ أُزَحْزِحُ رِجْلَ إِسْرَائِيلَ عَنِ الأَرْضِ الَّتِي عَيَّنْتُ لآبَائِهِمْ وَذَلِكَ إِذَا حَفِظُوا وَعَمِلُوا كُلَّ مَا أَوْصَيْتُهُمْ بِهِ كُلَّ الشَّرِيعَةِ وَالْفَرَائِضِ وَالأَحْكَامِ عَنْ يَدِ مُوسَى].
\par 9 وَلَكِنْ مَنَسَّى أَضَلَّ يَهُوذَا وَسُكَّانَ أُورُشَلِيمَ لِيَعْمَلُوا أَشَرَّ مِنَ الأُمَمِ الَّذِينَ طَرَدَهُمُ الرَّبُّ مِنْ أَمَامِ بَنِي إِسْرَائِيلَ.
\par 10 وَكَلَّمَ الرَّبُّ مَنَسَّى وَشَعْبَهُ فَلَمْ يُصْغُوا.
\par 11 فَجَلَبَ الرَّبُّ عَلَيْهِمْ رُؤَسَاءَ ْجُنْدِ مَلِكِ أَشُّورَ فَأَخَذُوا مَنَسَّى بِخِزَامَةٍ وَقَيَّدُوهُ بِسَلاَسِلِ نُحَاسٍ وَذَهَبُوا بِهِ إِلَى بَابِلَ.
\par 12 وَلَمَّا تَضَايَقَ طَلَبَ وَجْهَ الرَّبِّ إِلَهِهِ وَتَوَاضَعَ جِدّاً أَمَامَ إِلَهِ آبَائِهِ
\par 13 وَصَلَّى إِلَيْهِ فَاسْتَجَابَ لَهُ وَسَمِعَ تَضَرُّعَهُ وَرَدَّهُ إِلَى أُورُشَلِيمَ إِلَى مَمْلَكَتِهِ. فَعَلِمَ مَنَسَّى أَنَّ الرَّبَّ هُوَ اللَّهُ.
\par 14 وَبَعْدَ ذَلِكَ بَنَى سُوراً خَارِجَ مَدِينَةِ دَاوُدَ غَرْباً إِلَى جِيحُونَ فِي الْوَادِي وَإِلَى مَدْخَلِ بَابِ السَّمَكِ وَحَوَّطَ الأَكَمَةَ بِسُورٍ وَعَلاَّهُ جِدّاً. وَوَضَعَ رُؤَسَاءَ جُيُوشٍ فِي جَمِيعِ الْمُدُنِ الْحَصِينَةِ فِي يَهُوذَا.
\par 15 وَأَزَالَ الآلِهَةَ الْغَرِيبَةَ وَالصَّنَمَ مِنْ بَيْتِ الرَّبِّ وَجَمِيعَ الْمَذَابِحِ الَّتِي بَنَاهَا فِي جَبَلِ بَيْتِ الرَّبِّ وَفِي أُورُشَلِيمَ وَطَرَحَهَا خَارِجَ الْمَدِينَةِ.
\par 16 وَرَمَّمَ مَذْبَحَ الرَّبِّ وَذَبَحَ عَلَيْهِ ذَبَائِحَ سَلاَمَةٍ وَشُكْرٍ وَأَمَرَ يَهُوذَا أَنْ يَعْبُدُوا الرَّبَّ إِلَهَ إِسْرَائِيلَ.
\par 17 إِلاَّ أَنَّ الشَّعْبَ كَانُوا بَعْدُ يَذْبَحُونَ عَلَى الْمُرْتَفَعَاتِ إِنَّمَا لِلرَّبِّ إِلَهِهِمْ.
\par 18 وَبَقِيَّةُ أُمُورِ مَنَسَّى وَصَلاَتُهُ إِلَى إِلَهِهِ وَكَلاَمُ الرَّائِينَ الَّذِينَ كَلَّمُوهُ بِاسْمِ الرَّبِّ إِلَهِ إِسْرَائِيلَ هِيَ فِي أَخْبَارِ مُلُوكِ إِسْرَائِيلَ.
\par 19 وَصَلاَتُهُ وَالاِسْتِجَابَةُ لَهُ وَكُلُّ خَطَايَاهُ وَخِيَانَتُهُ وَالأَمَاكِنُ الَّتِي بَنَى فِيهَا مُرْتَفَعَاتٍ وَأَقَامَ سَوَارِيَ وَتَمَاثِيلَ قَبْلَ تَوَاضُعِهِ مَكْتُوبَةٌ فِي أَخْبَارِ الرَّائِينَ.
\par 20 ثُمَّ اضْطَجَعَ مَنَسَّى مَعَ آبَائِهِ فَدَفَنُوهُ فِي بَيْتِهِ وَمَلَكَ آمُونُ ابْنُهُ عِوَضاً عَنْهُ.
\par 21 كَانَ آمُونُ ابْنَ اثْنَتَيْنِ وَعِشْرِينَ سَنَةً حِينَ مَلَكَ وَمَلَكَ سَنَتَيْنِ فِي أُورُشَلِيمَ.
\par 22 وَعَمِلَ الشَّرَّ فِي عَيْنَيِ الرَّبِّ كَمَا عَمِلَ مَنَسَّى أَبُوهُ وَذَبَحَ آمُونُ لِجَمِيعِ التَّمَاثِيلِ الَّتِي عَمِلَ مَنَسَّى أَبُوهُ وَعَبَدَهَا.
\par 23 وَلَمْ يَتَوَاضَعْ أَمَامَ الرَّبِّ كَمَا تَوَاضَعَ مَنَسَّى أَبُوهُ بَلِ ازْدَادَ آمُونُ إِثْماً.
\par 24 وَفَتَنَ عَلَيْهِ عَبِيدُهُ وَقَتَلُوهُ فِي بَيْتِهِ.
\par 25 وَقَتَلَ شَعْبُ الأَرْضِ جَمِيعَ الْفَاتِنِينَ عَلَى الْمَلِكِ آمُونَ وَمَلَّكَ شَعْبُ الأَرْضِ يُوشِيَّا ابْنَهُ عِوَضاً عَنْهُ.

\chapter{34}

\par 1 كَانَ يُوشِيَّا ابْنَ ثَمَانِيَ سِنِينَ حِينَ مَلَكَ وَمَلَكَ إِحْدَى وَثَلاَثِينَ سَنَةً فِي أُورُشَلِيمَ.
\par 2 وَعَمِلَ الْمُسْتَقِيمَ فِي عَيْنَيِ الرَّبِّ وَسَارَ فِي طُرُقِ دَاوُدَ أَبِيهِ وَلَمْ يَحِدْ يَمِيناً وَلاَ شِمَالاً.
\par 3 وَفِي السَّنَةِ الثَّامِنَةِ مِنْ مُلْكِهِ إِذْ كَانَ بَعْدُ فَتىً ابْتَدَأَ يَطْلُبُ إِلَهَ دَاوُدَ أَبِيهِ. وَفِي السَّنَةِ الثَّانِيَةِ عَشَرَةَ ابْتَدَأَ يُطَهِّرُ يَهُوذَا وَأُورُشَلِيمَ مِنَ الْمُرْتَفَعَاتِ وَالسَّوَارِي وَالتَّمَاثِيلِ وَالْمَسْبُوكَاتِ.
\par 4 وَهَدَمُوا أَمَامَهُ مَذَابِحَ الْبَعْلِيمِ وَتَمَاثِيلَ الشَّمْسِ الَّتِي عَلَيْهَا مِنْ فَوْقُ قَطَعَهَا وَكَسَّرَ السَّوَارِيَ وَالتَّمَاثِيلَ وَالْمَسْبُوكَاتِ وَدَقَّهَا وَرَشَّهَا عَلَى قُبُورِ الَّذِينَ ذَبَحُوا لَهَا.
\par 5 وَأَحْرَقَ عِظَامَ الْكَهَنَةِ عَلَى مَذَابِحِهِمْ وَطَهَّرَ يَهُوذَا وَأُورُشَلِيمَ.
\par 6 وَفِي مُدُنِ مَنَسَّى وَأَفْرَايِمَ وَشَمْعُونَ إِلَى نَفْتَالِي مَعَ خَرَائِبِهَا حَوْلَهَا
\par 7 هَدَمَ الْمَذَابِحَ وَالسَّوَارِيَ وَدَقَّ التَّمَاثِيلَ نَاعِماً وَقَطَعَ جَمِيعَ تَمَاثِيلِ الشَّمْسِ فِي كُلِّ أَرْضِ إِسْرَائِيلَ ثُمَّ رَجَعَ إِلَى أُورُشَلِيمَ.
\par 8 وَفِي السَّنَةِ الثَّامِنَةَ عَشَرَةَ مِنْ مُلْكِهِ بَعْدَ أَنْ طَهَّرَ الأَرْضَ وَالْبَيْتَ أَرْسَلَ شَافَانَ بْنَ أَصَلْيَا وَمَعْسِيَّا رَئِيسَ الْمَدِينَةِ وَيُوآخَ بْنَ يُوآحَازَ الْمُسَجِّلَ لأَجْلِ تَرْمِيمِ بَيْتِ الرَّبِّ إِلَهِهِ.
\par 9 فَجَاءُوا إِلَى حِلْقِيَّا الْكَاهِنِ الْعَظِيمِ وَأَعْطُوهُ الْفِضَّةَ الْمُدْخَلَةَ إِلَى بَيْتِ اللَّهِ الَّتِي جَمَعَهَا اللاَّوِيُّونَ حَارِسُو الْبَابِ مِنْ مَنَسَّى وَأَفْرَايِمَ وَمِنْ كُلِّ بَقِيَّةِ إِسْرَائِيلَ وَمِنْ كُلِّ يَهُوذَا وَبِنْيَامِينَ ثُمَّ رَجَعُوا إِلَى أُورُشَلِيمَ.
\par 10 وَدَفَعُوهَا لأَيْدِي عَامِلِي الشُّغْلِ الْمُوَكَّلِينَ فِي بَيْتِ الرَّبِّ فَدَفَعُوهَا لِعَامِلِي الشُّغْلِ الَّذِينَ كَانُوا يَعْمَلُونَ فِي بَيْتِ الرَّبِّ لأَجْلِ إِصْلاَحِ الْبَيْتِ وَتَرْمِيمِهِ.
\par 11 وَأَعْطُوهَا لِلنَّجَّارِينَ وَالْبَنَّائِينَ لِيَشْتَرُوا حِجَارَةً مَنْحُوتَةً وَأَخْشَاباً لِلْوُصَلِ وَلأَجْلِ تَسْقِيفِ الْبُيُوتِ الَّتِي أَخْرَبَهَا مُلُوكُ يَهُوذَا.
\par 12 وَكَانَ الرِّجَالُ يَعْمَلُونَ الْعَمَلَ بِأَمَانَةٍ وَعَلَيْهِمْ وُكَلاَءُ يَحَثُ وَعُوبَدْيَا اللاَّوِيَّانِ مِنْ بَنِي مَرَارِي وَزَكَرِيَّا وَمَشُلاَّمُ مِنْ بَنِي الْقَهَاتِيِّينَ لأَجْلِ الْمُنَاظَرَةِ وَمِنَ اللاَّوِيِّينَ كُلُّ مَاهِرٍ بِآلاَتِ الْغِنَاءِ.
\par 13 وَكَانُوا عَلَى الْحُمَّالِ وَوُكَلاَءَ عَلَى كُلِّ عَامِلِ شُغْلٍ فِي خِدْمَةٍ فَخِدْمَةٍ. وَكَانَ مِنَ اللاَّوِيِّينَ كُتَّابٌ وَعُرَفَاءُ وَبَوَّابُونَ.
\par 14 وَعِنْدَ إِخْرَاجِهِمِ الْفِضَّةَ الْمُدْخَلَةَ إِلَى بَيْتِ الرَّبِّ وَجَدَ حِلْقَّيَا الْكَاهِنُ سِفْرَ شَرِيعَةِ الرَّبِّ بِيَدِ مُوسَى.
\par 15 فَقَالَ حِلْقِيَّا لِشَافَانَ الْكَاتِبِ: [قَدْ وَجَدْتُ سِفْرَ الشَّرِيعَةِ فِي بَيْتِ الرَّبِّ]. وَسَلَّمَ حِلْقِيَّا السِّفْرَ إِلَى شَافَانَ.
\par 16 فَجَاءَ شَافَانُ بِالسِّفْرِ إِلَى الْمَلِكِ وَقَالَ: [كُلُّ مَا أُسْلِمَ لِيَدِ عَبِيدِكَ هُمْ يَفْعَلُونَهُ.
\par 17 وَقَدْ أَفْرَغُوا الْفِضَّةَ الْمَوْجُودَةَ فِي بَيْتِ الرَّبِّ وَدَفَعُوهَا لِيَدِ الْوُكَلاَءِ وَيَدِ عَامِلِي الشُّغْلِ].
\par 18 وَأَخْبَرَ شَافَانُ الْكَاتِبُ الْمَلِكَ: [قَدْ أَعْطَانِي حِلْقِيَّا الْكَاهِنُ سِفْراً]. وَقَرَأَ فِيهِ شَافَانُ أَمَامَ الْمَلِكِ.
\par 19 فَلَمَّا سَمِعَ الْمَلِكُ كَلاَمَ الشَّرِيعَةِ مَزَّقَ ثِيَابَهُ
\par 20 وَأَمَرَ الْمَلِكُ حِلْقِيَا وَأَخِيقَامَ بْنَ شَافَانَ وَعَبْدُونَ بْنَ مِيخَا وَشَافَانَ الْكَاتِبَ وَعَسَايَا عَبْدَ الْمَلِكِ:
\par 21 [اذْهَبُوا اسْأَلُوا الرَّبَّ مِنْ أَجْلِي وَمِنْ أَجْلِ مَنْ بَقِيَ مِنْ إِسْرَائِيلَ وَيَهُوذَا عَنْ كَلاَمِ السِّفْرِ الَّذِي وُجِدَ لأَنَّهُ عَظِيمٌ غَضَبُ الرَّبِّ الَّذِي انْسَكَبَ عَلَيْنَا مِنْ أَجْلِ أَنَّ آبَاءَنَا لَمْ يَحْفَظُوا كَلاَمَ الرَّبِّ لِيَعْمَلُوا حَسَبَ كُلِّ مَا هُوَ مَكْتُوبٌ فِي هَذَا السِّفْرِ].
\par 22 فَذَهَبَ حِلْقِيَّا وَالَّذِينَ أَمَرَهُمُ الْمَلِكُ إِلَى خَلْدَةَ النَّبِيَّةِ امْرَأَةِ شَلُّومَ بْنِ تُوقَهَةَ بْنِ حَسْرَةَ حَارِسِ الثِّيَابِ وَهِيَ سَاكِنَةٌ فِي أُورُشَلِيمَ فِي الْقِسْمِ الثَّانِي وَكَلَّمُوهَا هَكَذَا.
\par 23 فَقَالَتْ لَهُمْ: [هَكَذَا قَالَ الرَّبُّ إِلَهُ إِسْرَائِيلَ: قُولُوا لِلرَّجُلِ الَّذِي أَرْسَلَكُمْ إِلَيَّ.
\par 24 هَكَذَا قَالَ الرَّبُّ: هَئَنَذَا جَالِبٌ شَرّاً عَلَى هَذَا الْمَوْضِعِ وَعَلَى سُكَّانِهِ جَمِيعَ اللَّعَنَاتِ الْمَكْتُوبَةِ فِي السِّفْرِ الَّذِي قَرَأُوهُ أَمَامَ مَلِكِ يَهُوذَا.
\par 25 مِنْ أَجْلِ أَنَّهُمْ تَرَكُونِي وَأَوْقَدُوا لآلِهَةٍ أُخْرَى لِيَغِيظُونِي بِكُلِّ أَعْمَالِ أَيْدِيهِمْ. وَيَنْسَكِبُ غَضَبِي عَلَى هَذَا الْمَوْضِعِ وَلاَ يَنْطَفِئُ.
\par 26 وَأَمَّا مَلِكُ يَهُوذَا الَّذِي أَرْسَلَكُمْ لِتَسْأَلُوا مِنَ الرَّبِّ فَهَكَذَا تَقُولُونَ لَهُ: هَكَذَا قَالَ الرَّبُّ إِلَهُ إِسْرَائِيلَ مِنْ جِهَةِ الْكَلاَمِ الَّذِي سَمِعْتَ:
\par 27 مِنْ أَجْلِ أَنَّهُ قَدْ رَقَّ قَلْبُكَ وَتَوَاضَعْتَ أَمَامَ اللَّهِ حِينَ سَمِعْتَ كَلاَمَهُ عَلَى هَذَا الْمَوْضِعِ وَعَلَى سُكَّانِهِ وَتَوَاضَعْتَ أَمَامِي وَمَزَّقْتَ ثِيَابَكَ وَبَكَيْتَ أَمَامِي يَقُولُ الرَّبُّ قَدْ سَمِعْتُ أَنَا أَيْضاً.
\par 28 هَئَنَذَا أَضُمُّكَ إِلَى آبَائِكَ فَتُضَمُّ إِلَى قَبْرِكَ بِسَلاَمٍ وَكُلَّ الشَّرِّ الَّذِي أَجْلِبُهُ عَلَى هَذَا الْمَوْضِعِ وَعَلَى سُكَّانِهِ لاَ تَرَى عَيْنَاكَ]. فَرَدُّوا عَلَى الْمَلِكِ الْجَوَابَ.
\par 29 وَأَرْسَلَ الْمَلِكُ وَجَمَعَ كُلَّ شُيُوخِ يَهُوذَا وَأُورُشَلِيمَ
\par 30 وَصَعِدَ الْمَلِكُ إِلَى بَيْتِ الرَّبِّ مَعَ كُلِّ رِجَالِ يَهُوذَا وَسُكَّانِ أُورُشَلِيمَ وَالْكَهَنَةِ وَاللاَّوِيِّينَ وَكُلِّ الشَّعْبِ مِنَ الْكَبِيرِ إِلَى الصَّغِيرِ وَقَرَأَ فِي آذَانِهِمْ كُلَّ كَلاَمِ سِفْرِ الْعَهْدِ الَّذِي وُجِدَ فِي بَيْتِ الرَّبِّ.
\par 31 وَوَقَفَ الْمَلِكُ عَلَى مِنْبَرِهِ وَقَطَعَ عَهْداً أَمَامَ الرَّبِّ لِلذَّهَابِ وَرَاءَ الرَّبِّ وَلِحِفْظِ وَصَايَاهُ وَشَهَادَاتِهِ وَفَرَائِضِهِ بِكُلِّ قَلْبِهِ وَكُلِّ نَفْسِهِ لِيَعْمَلَ كَلاَمَ الْعَهْدِ الْمَكْتُوبِ فِي هَذَا السِّفْرِ.
\par 32 وَأَوْقَفَ كُلَّ الْمَوْجُودِينَ فِي أُورُشَلِيمَ وَبِنْيَامِينَ فَعَمِلَ سُكَّانُ أُورُشَلِيمَ حَسَبَ عَهْدِ اللَّهِ إِلَهِ آبَائِهِمْ.
\par 33 وَأَزَالَ يُوشِيَّا جَمِيعَ الرَّجَاسَاتِ مِنْ كُلِّ الأَرَاضِي الَّتِي لِبَنِي إِسْرَائِيلَ وَجَعَلَ جَمِيعَ الْمَوْجُودِينَ فِي أُورُشَلِيمَ يَعْبُدُونَ الرَّبَّ إِلَهَهُمْ. كُلَّ أَيَّامِهِ لَمْ يَحِيدُوا مِنْ وَرَاءِ الرَّبِّ إِلَهِ آبَائِهِمْ.

\chapter{35}

\par 1 وَعَمِلَ يُوشِيَّا فِي أُورُشَلِيمَ فِصْحاً لِلرَّبِّ وَذَبَحُوا الْفِصْحَ فِي الرَّابِعَ عَشَرَ مِنَ الشَّهْرِ الأَوَّلِ.
\par 2 وَأَقَامَ الْكَهَنَةَ عَلَى حِرَاسَاتِهِمْ وَشَدَّدَهُمْ لِخِدْمَةِ بَيْتِ الرَّبِّ.
\par 3 وَقَالَ لِلاَّوِيِّينَ الَّذِينَ كَانُوا يُعَلِّمُونَ كُلَّ إِسْرَائِيلَ الَّذِينَ كَانُوا مُقَدَّسِينَ لِلرَّبِّ: [اجْعَلُوا تَابُوتَ الْقُدْسِ فِي الْبَيْتِ الَّذِي بَنَاهُ سُلَيْمَانُ بْنُ دَاوُدَ مَلِكُ إِسْرَائِيلَ. لَيْسَ لَكُمْ أَنْ تَحْمِلُوا عَلَى الأَكْتَافِ. الآنَ اخْدِمُوا الرَّبَّ إِلَهَكُمْ وَشَعْبَهُ إِسْرَائِيلَ.
\par 4 وَأَعِدُّوا بُيُوتَ آبَائِكُمْ حَسَبَ فِرَقِكُمْ حَسَبَ كِتَابَةِ دَاوُدَ مَلِكِ إِسْرَائِيلَ وَحَسَبَ كِتَابَةِ سُلَيْمَانَ ابْنِهِ.
\par 5 وَقِفُوا فِي الْقُدْسِ حَسَبَ أَقْسَامِ بُيُوتِ آبَاءِ إِخْوَتِكُمْ بَنِي الشَّعْبِ وَفِرَقِ بُيُوتِ آبَاءِ اللاَّوِيِّينَ
\par 6 وَاذْبَحُوا الْفِصْحَ وَتَقَدَّسُوا وَأَعِدُّوا إِخْوَتَكُمْ لِيَعْمَلُوا حَسَبَ كَلاَمِ الرَّبِّ عَنْ يَدِ مُوسَى].
\par 7 وَأَعْطَى يُوشِيَّا لِبَنِي الشَّعْبِ غَنَماً حُمْلاَناً وَجِدَاءً جَمِيعَ ذَلِكَ لِلْفِصْحِ لِكُلِّ الْمَوْجُودِينَ إِلَى عَدَدِ ثَلاَثِينَ أَلْفاً وَثَلاَثَةِ آلاَفٍ مِنَ الْبَقَرِ. هَذِهِ مِنْ مَالِ الْمَلِكِ.
\par 8 وَرُؤَسَاؤُهُ قَدَّمُوا تَبَرُّعاً لِلشَّعْبِ وَالْكَهَنَةِ وَاللاَّوِيِّينَ حِلْقِيَا وَزَكَرِيَّا وَيَحْيِئِيلَ رُؤَسَاءِ بَيْتِ اللَّهِ. أَعْطُوا الْكَهَنَةَ لِلْفِصْحِ أَلْفَيْنِ وَسِتَّ مِئَةٍ مِنَ الْغَنَمِ وَمِنَ الْبَقَرِ ثَلاَثَ مِئَةٍ.
\par 9 وَكُونَنْيَا وَشَمَعْيَا وَنِثْنِئِيلُ أَخَوَاهُ وَحَشَبْيَا وَيَعِيئِيلُ وَيُوزَابَادُ رُؤَسَاءُ اللاَّوِيِّينَ قَدَّمُوا لِلاَّوِيِّينَ لِلْفِصْحِ خَمْسَةَ آلاَفٍ مِنَ الْغَنَمِ وَمِنَ الْبَقَرِ خَمْسَ مِئَةٍ.
\par 10 فَتَهَيَّأَتِ الْخِدْمَةُ وَقَامَ الْكَهَنَةُ فِي مَقَامِهِمْ وَاللاَّوِيُّونَ فِي فِرَقِهِمْ حَسَبَ أَمْرِ الْمَلِكِ
\par 11 وَذَبَحُوا الْفِصْحَ. وَرَشَّ الْكَهَنَةُ مِنْ أَيْدِيهِمْ. وَأَمَّا اللاَّوِيُّونَ فَكَانُوا يَسْلُخُونَ.
\par 12 وَرَفَعُوا الْمُحْرَقَةَ لِيُعْطُوا حَسَبَ أَقْسَامِ بُيُوتِ الآبَاءِ لِبَنِي الشَّعْبِ لِيُقَرِّبُوا لِلرَّبِّ كَمَا هُوَ مَكْتُوبٌ فِي سِفْرِ مُوسَى. وَهَكَذَا بِالْبَقَرِ.
\par 13 وَشَوُوا الْفِصْحَ بِالنَّارِ كَالْمَرْسُومِ. وَأَمَّا الأَقْدَاسُ فَطَبَخُوهَا فِي الْقُدُورِ وَالْمَرَاجِلِ وَالصِّحَافِ وَبَادَرُوا بِهَا إِلَى جَمِيعِ بَنِي الشَّعْبِ.
\par 14 وَبَعْدُ أَعَدُّوا لأَنْفُسِهِمْ وَلِلْكَهَنَةِ لأَنَّ الْكَهَنَةَ بَنِي هَارُونَ كَانُوا عَلَى إِصْعَادِ الْمُحْرَقَةِ وَالشَّحْمِ إِلَى اللَّيْلِ. فَأَعَدَّ اللاَّوِيُّونَ لأَنْفُسِهِمْ وَلِلْكَهَنَةِ بَنِي هَارُونَ.
\par 15 وَالْمُغَنُّونَ بَنُو آسَافَ كَانُوا فِي مَقَامِهِمْ حَسَبَ أَمْرِ دَاوُدَ وَآسَافَ وَهَيْمَانَ وَيَدُوثُونَ رَائِي الْمَلِكِ. وَالْبَوَّابُونَ عَلَى بَابٍ فَبَابٍ لَمْ يَكُنْ لَهُمْ أَنْ يَحِيدُوا عَنْ خِدْمَتِهِمْ لأَنَّ إِخْوَتَهُمُ اللاَّوِيِّينَ أَعَدُّوا لَهُمْ.
\par 16 فَتَهَيَّأَ كُلُّ خَدَمَةِ الرَّبِّ فِي ذَلِكَ الْيَوْمِ لِعَمَلِ الْفِصْحِ وَإِصْعَادِ الْمُحْرَقَاتِ عَلَى مَذْبَحِ الرَّبِّ حَسَبَ أَمْرِ الْمَلِكِ يُوشِيَّا.
\par 17 وَعَمِلَ بَنُو إِسْرَائِيلَ الْمَوْجُودُونَ الْفِصْحَ فِي ذَلِكَ الْوَقْتِ وَعِيدَ الْفَطِيرِ سَبْعَةَ أَيَّامٍ.
\par 18 وَلَمْ يُعْمَلْ فِصْحٌ مِثْلُهُ فِي إِسْرَائِيلَ مِنْ أَيَّامِ صَمُوئِيلَ النَّبِيِّ. وَكُلُّ مُلُوكِ إِسْرَائِيلَ لَمْ يَعْمَلُوا كَالْفِصْحِ الَّذِي عَمِلَهُ يُوشِيَّا وَالْكَهَنَةُ وَاللاَّوِيُّونَ وَكُلُّ يَهُوذَا وَإِسْرَائِيلَ الْمَوْجُودِينَ وَسُكَّانِ أُورُشَلِيمَ.
\par 19 فِي السَّنَةِ الثَّامِنَةَ عَشَرَةَ لِمُلْكِ يُوشِيَّا عُمِلَ هَذَا الْفِصْحُ.
\par 20 بَعْدَ كُلِّ هَذَا حِينَ هَيَّأَ يُوشِيَّا الْبَيْتَ صَعِدَ نَخُو مَلِكُ مِصْرَ إِلَى كَرْكَمِيشَ لِيُحَارِبَ عِنْدَ الْفُرَاتِ. فَخَرَجَ يُوشِيَّا لِلِقَائِهِ.
\par 21 فَأَرْسَلَ إِلَيْهِ رُسُلاً يَقُولُ: [مَا لِي وَلَكَ يَا مَلِكَ يَهُوذَا! لَسْتُ عَلَيْكَ أَنْتَ الْيَوْمَ وَلَكِنْ عَلَى بَيْتٍ آخَرَ أُحَارِبُهُ وَاللَّهُ أَمَرَ بِإِسْرَاعِي. فَكُفَّ عَنِ اللَّهِ الَّذِي مَعِي فَلاَ يُهْلِكَكَ].
\par 22 وَلَمْ يُحَوِّلْ يُوشِيَّا وَجْهَهُ عَنْهُ بَلْ تَنَكَّرَ لِمُقَاتَلَتِهِ وَلَمْ يَسْمَعْ لِكَلاَمِ نَخُو مِنْ فَمِ اللَّهِ بَلْ جَاءَ لِيُحَارِبَ فِي بُقْعَةِ مَجِدُّو.
\par 23 وَأَصَابَ الرُّمَاةُ الْمَلِكَ يُوشِيَّا فَقَالَ الْمَلِكُ لِعَبِيدِهِ: [انْقُلُونِي لأَنِّي جُرِحْتُ جِدّاً].
\par 24 فَنَقَلَهُ عَبِيدُهُ مِنَ الْمَرْكَبَةِ وَأَرْكَبُوهُ عَلَى الْمَرْكَبَةِ الثَّانِيَةِ الَّتِي لَهُ وَسَارُوا بِهِ إِلَى أُورُشَلِيمَ فَمَاتَ وَدُفِنَ فِي قُبُورِ آبَائِهِ. وَكَانَ كُلُّ يَهُوذَا وَأُورُشَلِيمَ يَنُوحُونَ عَلَى يُوشِيَّا.
\par 25 وَرَثَى إِرْمِيَا يُوشِيَّا. وَكَانَ جَمِيعُ الْمُغَنِّينَ وَالْمُغَنِّيَاتِ يَنْدُبُونَ يُوشِيَّا فِي مَرَاثِيهِمْ إِلَى الْيَوْمِ وَجَعَلُوهَا فَرِيضَةً عَلَى إِسْرَائِيلَ. وَهَا هِيَ مَكْتُوبَةٌ فِي الْمَرَاثِي.
\par 26 وَبَقِيَّةُ أُمُورِ يُوشِيَّا وَمَرَاحِمُهُ حَسْبَمَا هُوَ مَكْتُوبٌ فِي نَامُوسِ الرَّبِّ.
\par 27 وَأُمُورُهُ الأُولَى وَالأَخِيرَةُ مَكْتُوبَةٌ فِي سِفْرِ مُلُوكِ إِسْرَائِيلَ وَيَهُوذَا.

\chapter{36}

\par 1 وَأَخَذَ شَعْبُ الأَرْضِ يَهُوآحَازَ بْنَ يُوشِيَّا وَمَلَّكُوهُ عِوَضاً عَنْ أَبِيهِ فِي أُورُشَلِيمَ.
\par 2 كَانَ يَهُوآحَازُ ابْنَ ثَلاَثٍ وَعِشْرِينَ سَنَةً حِينَ مَلَكَ وَمَلَكَ ثَلاَثَةَ أَشْهُرٍ فِي أُورُشَلِيمَ
\par 3 وَعَزَلَهُ مَلِكُ مِصْرَ فِي أُورُشَلِيمَ وَغَرَّمَ الأَرْضَ بِمِئَةِ وَزْنَةٍ مِنَ الْفِضَّةِ وَبِوَزْنَةٍ مِنَ الذَّهَبِ.
\par 4 وَمَلَّكَ مَلِكُ مِصْرَ أَلِيَاقِيمَ أَخَاهُ عَلَى يَهُوذَا وَأُورُشَلِيمَ وَغَيَّرَ اسْمَهُ إِلَى يَهُويَاقِيمَ. وَأَمَّا يَهُوآحَازُ أَخُوهُ فَأَخَذَهُ نَخُو وَأَتَى بِهِ إِلَى مِصْرَ.
\par 5 كَانَ يَهُويَاقِيمُ ابْنَ خَمْسٍ وَعِشْرِينَ سَنَةً حِينَ مَلَكَ وَمَلَكَ إِحْدَى عَشَرَةَ سَنَةً فِي أُورُشَلِيمَ وَعَمِلَ الشَّرَّ فِي عَيْنَيِ الرَّبِّ إِلَهِهِ.
\par 6 عَلَيْهِ صَعِدَ نَبُوخَذْنَصَّرُ مَلِكُ بَابِلَ وَقَيَّدَهُ بِسَلاَسِلِ نُحَاسٍ لِيَذْهَبَ بِهِ إِلَى بَابِلَ
\par 7 وَأَتَى نَبُوخَذْنَصَّرُ بِبَعْضِ آنِيَةِ بَيْتِ الرَّبِّ إِلَى بَابِلَ وَجَعَلَهَا فِي هَيْكَلِهِ فِي بَابِلَ.
\par 8 وَبَقِيَّةُ أُمُورِ يَهُويَاقِيمَ وَرَجَاسَاتُهُ الَّتِي عَمِلَ وَمَا وُجِدَ فِيهِ مَكْتُوبَةٌ فِي سِفْرِ مُلُوكِ إِسْرَائِيلَ وَيَهُوذَا. وَمَلَكَ يَهُويَاكِينُ ابْنُهُ عِوَضاً عَنْهُ.
\par 9 كَانَ يَهُويَاكِينُ ابْنَ ثَمَانِي سِنِينَ حِينَ مَلَكَ وَمَلَكَ ثَلاَثَةَ أَشْهُرٍ وَعَشَرَةَ أَيَّامٍ فِي أُورُشَلِيمَ. وَعَمِلَ الشَّرَّ فِي عَيْنَيِ الرَّبِّ.
\par 10 وَعِنْدَ رُجُوعِ السَّنَةِ أَرْسَلَ الْمَلِكُ نَبُوخَذْنَصَّرُ فَأَتَى بِهِ إِلَى بَابِلَ مَعَ آنِيَةِ بَيْتِ الرَّبِّ الثَّمِينَةِ وَمَلَّكَ صِدْقِيَّا أَخَاهُ عَلَى يَهُوذَا وَأُورُشَلِيمَ.
\par 11 كَانَ صِدْقِيَّا ابْنَ إِحْدَى وَعِشْرِينَ سَنَةً حِينَ مَلَكَ وَمَلَكَ إِحْدَى عَشَرَةَ سَنَةً فِي أُورُشَلِيمَ.
\par 12 وَعَمِلَ الشَّرَّ فِي عَيْنَيِ الرَّبِّ إِلَهِهِ وَلَمْ يَتَوَاضَعْ أَمَامَ إِرْمِيَا النَّبِيِّ مِنْ فَمِ الرَّبِّ.
\par 13 وَتَمَرَّدَ أَيْضاً عَلَى الْمَلِكِ نَبُوخَذْنَصَّرَ الَّذِي حَلَّفَهُ بِاللَّهِ وَصَلَّبَ عُنُقَهُ وَقَوَّى قَلْبَهُ عَنِ الرُّجُوعِ إِلَى الرَّبِّ إِلَهِ إِسْرَائِيلَ
\par 14 حَتَّى أَنَّ جَمِيعَ رُؤَسَاءِ الْكَهَنَةِ وَالشَّعْبِ أَكْثَرُوا الْخِيَانَةَ حَسَبَ كُلِّ رَجَاسَاتِ الأُمَمِ وَنَجَّسُوا بَيْتَ الرَّبِّ الَّذِي قَدَّسَهُ فِي أُورُشَلِيمَ.
\par 15 فَأَرْسَلَ الرَّبُّ إِلَهُ آبَائِهِمْ إِلَيْهِمْ عَنْ يَدِ رُسُلِهِ مُبَكِّراً وَمُرْسِلاً لأَنَّهُ شَفِقَ عَلَى شَعْبِهِ وَعَلَى مَسْكَنِهِ
\par 16 فَكَانُوا يَهْزَأُونَ بِرُسُلِ اللَّهِ وَرَذَلُوا كَلاَمَهُ وَتَهَاوَنُوا بِأَنْبِيَائِهِ حَتَّى ثَارَ غَضَبُ الرَّبِّ عَلَى شَعْبِهِ حَتَّى لَمْ يَكُنْ شِفَاءٌ.
\par 17 فَأَصْعَدَ عَلَيْهِمْ مَلِكَ الْكِلْدَانِيِّينَ فَقَتَلَ مُخْتَارِيهِمْ بِالسَّيْفِ فِي بَيْتِ مَقْدِسِهِمْ. وَلَمْ يُشْفِقْ عَلَى فَتًى أَوْ عَذْرَاءَ وَلاَ عَلَى شَيْخٍ أَوْ أَشْيَبَ بَلْ دَفَعَ الْجَمِيعَ لِيَدِهِ.
\par 18 وَجَمِيعُ آنِيَةِ بَيْتِ اللَّهِ الْكَبِيرَةِ وَالصَّغِيرَةِ وَخَزَائِنِ بَيْتِ الرَّبِّ وَخَزَائِنِ الْمَلِكِ وَرُؤَسَائِهِ أَتَى بِهَا جَمِيعاً إِلَى بَابِلَ.
\par 19 وَأَحْرَقُوا بَيْتَ اللَّهِ وَهَدَمُوا سُورَ أُورُشَلِيمَ وَأَحْرَقُوا جَمِيعَ قُصُورِهَا بِالنَّارِ وَأَهْلَكُوا جَمِيعَ آنِيَتِهَا الثَّمِينَةِ.
\par 20 وَسَبَى الَّذِينَ بَقُوا مِنَ السَّيْفِ إِلَى بَابِلَ فَكَانُوا لَهُ وَلِبَنِيهِ عَبِيداً إِلَى أَنْ مَلَكَتْ مَمْلَكَةُ فَارِسَ
\par 21 لإِكْمَالِ كَلاَمِ الرَّبِّ بِفَمِ إِرْمِيَا حَتَّى اسْتَوْفَتِ الأَرْضُ سُبُوتَهَا لأَنَّهَا سَبَتَتْ فِي كُلِّ أَيَّامِ خَرَابِهَا لإِكْمَالِ سَبْعِينَ سَنَةً.
\par 22 وَفِي السَّنَةِ الأُولَى لِكُورَشَ مَلِكِ فَارِسَ لأَجْلِ تَكْمِيلِ كَلاَمِ الرَّبِّ بِفَمِ إِرْمِيَا نَبَّهَ الرَّبُّ رُوحَ كُورَشَ مَلِكِ فَارِسَ فَأَطْلَقَ نِدَاءً فِي كُلِّ مَمْلَكَتِهِ وَكَذَا بِالْكِتَابَةِ قَائِلاً:
\par 23 [هَكَذَا قَالَ كُورَشُ مَلِكُ فَارِسَ إِنَّ الرَّبَّ إِلَهَ السَّمَاءِ قَدْ أَعْطَانِي جَمِيعَ مَمَالِكِ الأَرْضِ وَهُوَ أَوْصَانِي أَنْ أَبْنِيَ لَهُ بَيْتاً فِي أُورُشَلِيمَ الَّتِي فِي يَهُوذَا. مَنْ مِنْكُمْ مِنْ جَمِيعِ شَعْبِهِ الرَّبُّ إِلَهُهُ مَعَهُ وَلْيَصْعَدْ].

\end{document}