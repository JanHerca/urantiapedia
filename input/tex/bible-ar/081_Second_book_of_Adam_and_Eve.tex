\begin{document}

\title{الكتاب الثاني لآدم وحواء}

\chapter{1}

\par \textit{العائلة الحزينة. تزوج قابيل من لولوة وانتقلا للعيش بعيدًا.}

\par 1 عندما سمعت لولوة كلمات قابيل، بكت وذهبت إلى أبيها وأمها، وأخبرتهما كيف قتل قابيل أخاه هابيل

\par 2 فصرخوا جميعهم بصوت عالٍ ورفعوا أصواتهم ولطموا وجوههم وذروا ترابا على رؤوسهم ومزقوا ثيابهم وخرجوا وجاءوا إلى المكان الذي قُتل فيه هابيل

\par 3 فوجدوه ملقى على الأرض مقتولاً، والوحوش حوله، وهم يبكون ويصرخون على هذا الواحد. ومن جسده، بسبب نقائه، خرجت رائحة أطياب ذكية

\par 4 فحمله آدم ودموعه تنهمر على وجهه، وذهب إلى مغارة الكنوز، حيث وضعه، ولفه بالطيب والمر

\par 5 واستمر آدم وحواء عند دفنه في حزن شديد مئة وأربعين يومًا. كان هابيل ابن خمس عشرة سنة ونصف، وكان قابيل ابن سبع عشرة سنة ونصف

\par 6 أما قابيل، فلما انتهى منادته على أخيه، أخذ أخته لولوة وتزوجها دون إذن من أبيه وأمه، لأنهما لم يستطيعا أن يمنعاه عنها بسبب ثقل قلبهما

\par 7 ثم نزل إلى أسفل الجبل، بعيدًا عن الحديقة، بالقرب من المكان الذي قتل فيه أخاه

\par 8 وفي ذلك المكان كانت هناك أشجار فاكهة كثيرة وأشجار غابات. أنجبت له أخته أطفالًا، بدأوا بدورهم يتكاثرون تدريجيًا حتى ملأوا ذلك المكان

\par 9 أما آدم وحواء، فلم يجتمعا بعد جنازة هابيل لمدة سبع سنوات. ولكن بعد ذلك، حملت حواء. وبينما كانت حاملاً، قال لها آدم: "هلمّي، لنأخذ قربانًا ونرفعه إلى الله، ونسأله أن يرزقنا ولدًا جميلًا، نجد فيه العزاء، ونزوجه من أخت هابيل."

\par 10 ثم أعدوا تقدمة وأصعدوها إلى المذبح، وقدموها أمام الرب، وابتدأوا يتضرعون إليه أن يقبل تقدمتهم، وأن يرزقهم نسلا صالحا

\par 11 فسمع الله آدم وقبل قربانه. ثم سجد آدم وحواء وابنتهما، ونزلوا إلى مغارة الكنوز ووضعوا فيها سراجًا ليوقد ليلًا ونهارًا أمام جسد هابيل

\par 12 ثم استمر آدم وحواء في الصيام والصلاة حتى حان وقت ولادة حواء، عندما قالت لآدم: "أريد أن أذهب إلى الكهف في الصخرة، لألد فيه."

\par 13 فقال: «اذهب وخذ معك ابنتك لتخدمك، أما أنا فسأبقى في مغارة الكنوز هذه أمام جثة ابني هابيل».

\par 14 ثم استمعت حواء لآدم، وذهبت هي وابنتها. أما آدم فبقي وحده في مغارة الكنوز

\chapter{2}

\par \textit{يولد ابن ثالث لآدم وحواء.}

\par 1 فولدت حواء ابنًا كاملًا جميلًا في الصورة والمنظر، وكان جماله كجمال أبيه آدم، بل أجمل.

\par 2 ثم تعزّت حواء عندما رأته، وبقيت ثمانية أيام في الكهف؛ ثم أرسلت ابنتها إلى آدم لتخبره أن يأتي ويرى الطفل ويسميه. لكن الابنة بقيت في مكانه بجانب جثة أخيها، حتى عاد آدم. وكذلك فعلت

\par 3 فلما جاء آدم ورأى حسن الصبي وجماله وقوامه الكامل، فرح به وتعزى عن هابيل. ثم دعا الطفل شيثًا، وهذا يعني "أن الله قد سمع صلاتي وأنقذني من ضيقي". ولكنه يعني أيضًا "القدرة والقدرة".

\par 4 ثم بعد أن سمى آدم الطفل، عاد إلى كهف الكنوز، وعادت ابنته إلى أمها

\par 5 أما حواء فقد بقيت في كهفها حتى انقضت أربعون يومًا، ثم أتت إلى آدم، وأحضرت معها الطفل وابنتها

\par 6 وأتوا إلى نهر ماء، حيث اغتسل آدم وابنته بسبب حزنهما على هابيل؛ أما حواء والطفل فقد اغتسلتا للتطهير

\par 7 ثم رجعوا وأخذوا تقدمة، وذهبوا إلى الجبل وقدموها عن الطفل، فقبل الله تقدمتهم، وأرسل بركته عليهم وعلى ابنهم شيث، ورجعوا إلى مغارة الكنوز

\par 8 أما آدم، فلم يعرف زوجته حواء كل أيام حياته، ولم يولد منها ذرية، إلا هؤلاء الخمسة فقط: قابيل، ولولوة، وهابيل، وعقلية، وشيث وحدهم

\par 9 أما شيث فكان يزداد في القامة والقوة، وبدأ يصوم ويصلي بحرارة

\chapter{3}

\par \textit{يظهر الشيطان كامرأة جميلة تُغري آدم، قائلةً له إنه لا يزال شابًا. "اقضِ شبابك في مرح وسرور." (12) الأشكال المختلفة التي يتخذها الشيطان (15).}

\par 1 أما أبونا آدم، ففي نهاية سبع سنوات من اليوم الذي انفصل فيه عن زوجته حواء، حسده الشيطان عندما رآه منفصلاً عنها، وسعى ليجعله يعيش معها مرة أخرى

\par 2 ثم قام آدم وصعد فوق مغارة الكنوز، وظل ينام هناك ليلًا نهارًا. وما إن أشرق النهار حتى نزل إلى المغارة ليصلي فيها ويأخذ منها بركة.

\par 3 ولما كان المساء صعد إلى سطح المغارة، حيث كان نائمًا بمفرده، خوفًا من أن يغلبه الشيطان، ومكث على هذا الحال تسعة وثلاثين يومًا.

\par 4 ثم إن الشيطان، كاره كل خير، لما رأى آدم وحيدًا هكذا، صائمًا ومصليًا، ظهر له في صورة امرأة جميلة، فجاءت ووقفت أمامه في ليلة الأربعين، وقالت له:

\par 5 يا آدم، منذ سكنت هذا الكهف، ونحن ننعم عليك بسلام عظيم، ووصلت إلينا دعواتك، واسترحنا منك.

\par 6 والآن يا آدم، وقد صعدتَ فوق سقف الكهف لتنام، شككنا فيك، وحلّ بنا حزنٌ عظيمٌ لفراقك حواء. ثم، وأنت على سطح هذا الكهف، تُسكب صلاتك، ويهيم قلبك من جانب إلى آخر.

\par 7 «ولكن عندما كنتِ في الكهف، كانت صلاتكِ كالنار المتجمعة، نزلت إلينا، ووجدتِ الراحة

\par 8 ثم حزنتُ أيضًا على أولادكِ الذين انفصلوا عنكِ، وحزني عظيم على مقتل ابنكِ هابيل، لأنه كان بارًا، وعلى الرجل البار سيحزن كل واحد

\par 9 «لكنني فرحتُ بميلاد ابنك شيث، ولكن بعد قليل حزنتُ حزنًا شديدًا على حواء، لأنها أختي. لأنه عندما أرسل الله عليكِ سباتًا عميقًا، وانتزعها من جنبكِ، أخرجني أنا أيضًا معها. لكنه أقامها بوضعها معكِ، بينما أنزلني

\par 10 «لقد فرحتُ بأختي لوجودها معك. لكن الله كان قد قطع لي وعدًا من قبل، وقال: لا تحزني؛ عندما يصعد آدم إلى سطح كهف الكنوز، وينفصل عن حواء زوجته، سأرسلكِ إليه، وستتزوجينه، وتنجبين له خمسة أطفال، كما أنجبت له حواء خمسة.»

\par 11 «والآن، ها هوذا وعد الله لي قد تحقق؛ لأنه هو الذي أرسلني إليكِ للزواج؛ لأنه إذا تزوجتني، فسأنجبك أطفالًا أجمل وأفضل من أطفال حواء

\par 12 «ومع ذلك، فأنت ما زلت شابًا؛ لا تُنهِ شبابك في هذا العالم بالحزن؛ بل اقضِ أيام شبابك في الفرح والسرور. لأن أيامك قليلة ومحنتك عظيمة. كن قويًا؛ وأنهِ أيامك في هذا العالم بالفرح. سأسعد بك، وستفرح معي بهذه الحكمة، ودون خوف.»

\par 13 ثم اقتربت من آدم واحتضنته قائلة: «قُم إذن، وأنفذ أمر إلهك».

\par 14 ولكن عندما رأى آدم أنها ستُغلب عليه، صلى إلى الله بقلب حار أن يُنقذه منها

\par 15 ثم أرسل الله كلمته إلى آدم قائلًا: "يا آدم، هذا الشخص هو الذي وعدك بالألوهية والجلال؛ إنه ليس راغبًا فيك؛ بل يظهر لك تارة في صورة امرأة؛ وتارة أخرى في صورة ملاك؛ وتارة أخرى في صورة حية؛ وتارة أخرى في صورة إله؛ لكنه يفعل كل ذلك فقط لتدمير روحك

\par 16 «الآن، يا آدم، إذ أفهم قلبك، فقد أنقذتك من يديه مرات عديدة؛ لأريك أنني إله رحيم؛ وأنني أتمنى لك الخير، ولا أريد لك الهلاك.»

\chapter{4}

\par \textit{آدم يرى الشيطان بألوانه الحقيقية.}

\par 1 ثم أمر الله الشيطان أن يظهر لآدم بكل وضوح، في صورته البشعة.

\par 2 فلما رآه آدم خاف وارتعد من رؤيته.

\par 3 وقال الله لآدم: انظر إلى هذا الشيطان، وإلى مظهره البغيض، واعلم أنه هو الذي أسقطك من النور إلى الظلمة، ومن السلام والراحة إلى العناء والشقاء.

\par 4 وانظر يا آدم إلى من قال عن نفسه إنه إله! هل يمكن أن يكون الله أسود؟ هل يتخذ الله صورة امرأة؟ هل يوجد من هو أقوى من الله؟ وهل يمكن التغلب عليه؟

\par 5 «انظر إذن يا آدم، وانظر إليه مقيدًا في حضرتك، في الهواء، غير قادر على الفرار! لذلك أقول لك: لا تخف منه؛ من الآن فصاعدًا، احترس منه، واحذر منه، في كل ما قد يفعله بك.»

\par 6 ثم طرد الله الشيطان من أمام آدم، الذي قواه، وعزّى قلبه، قائلاً له: "انزل إلى مغارة الكنوز، ولا تنفصل عن حواء؛ سأطفئ فيك كل شهوة حيوانية."

\par 7 من تلك الساعة، غادر آدم وحواء، وتمتعا بالراحة بأمر الله. لكن الله لم يفعل مثل ذلك لأي من نسل آدم؛ بل لآدم وحواء فقط

\par 8 ثم سجد آدم أمام الرب لأنه خلصه، ولأنه وضع آلامه. ونزل من فوق الكهف، وسكن مع حواء كما كان من قبل

\par 9 انتهى بذلك أربعون يومًا من انفصاله عن حواء.

\chapter{5}

يرسم الشيطان صورة رائعة لسيث ليتلذذ بأفكاره عنها.

\par 1 أما شيث، فلما كان في السابعة من عمره، كان يعرف الخير والشر، وكان مواظبًا على الصيام والصلاة، وكان يقضي لياليه كلها في التوسل إلى الله طلبًا للرحمة والمغفرة

\par 2 وكان يصوم أيضًا عند تقديم قربانه كل يوم، أكثر من والده؛ لأنه كان حسن المنظر، كملاك الله. وكان لديه أيضًا قلب طيب، حافظ على أجود صفات نفسه: ولهذا السبب كان يقدم قربانه كل يوم

\par 3 فرض الله بتقدمته، ورضي أيضًا بطهارته. واستمر هكذا يعمل مشيئة الله ومشيئة أبيه وأمه حتى بلغ السابعة من عمره

\par 4 وبعد ذلك، بينما كان نازلاً من المذبح، بعد أن فرغ من تقديم ذبيحته، ظهر له الشيطان في شكل ملاك جميل منير، وبيده عصا من نور، وهو متنطق بمنطقة من نور.

\par 5 استقبل شيث بابتسامة جميلة، وبدأ يغازله بكلمات لطيفة، قائلاً له: "يا شيث، لماذا تقيم في هذا الجبل؟ إنه جبل وعر، مليء بالحجارة والرمال، وأشجار بلا ثمار طيبة؛ برية بلا مساكن ولا مدن؛ لا مكان مناسب للسكن. كل شيء فيه حر وتعب وشقاء."

\par 6 قال أيضًا: "لكننا نعيش في أماكن جميلة، في عالم آخر غير هذه الأرض. عالمنا نور، وحالنا في أحسن حال؛ ونساؤنا أجمل من غيرهن؛ وأتمنى منك يا سيث أن تتزوج إحداهن؛ لأني أرى أنك جميل المنظر، وفي هذه الأرض لا توجد امرأة واحدة تناسبك. علاوة على ذلك، كل من يعيش في هذا العالم خمس أرواح فقط."

\par 7 لكن في عالمنا رجالٌ ونساءٌ كثيرات، كلٌّ منهن أجمل من الأخرى. لذا، أودّ أن أرحل عنكِ، لتتمكّني من رؤية أقاربي والزواج بمن تشاءين.

\par 8 «حينئذٍ ستلتزم بي وتكون في سلام؛ وستمتلئ بالروعة والنور، مثلنا

\par 9 «ستبقى في عالمنا. وسترتاح من هذا العالم وبؤسه؛ ولن تشعر بالضعف والتعب مرة أخرى؛ ولن تُقدم قربانًا أبدًا، ولا تطلب الرحمة؛ لأنك لن ترتكب خطيئة بعد الآن، ولن تتأثر بالأهواء.»

\par 10 «وإن سمعتَ ما أقول، فتزوج إحدى بناتي؛ لأنه ليس عندنا خطيئة أن نفعل ذلك؛ ولا يُحسب شهوة حيوانية

\par 11 «لأنه في عالمنا ليس لدينا إله؛ لكننا جميعًا آلهة؛ نحن جميعًا من النور، سماويون، أقوياء، أقوياء ومجيدون.»

\chapter{6}

\par \textit{يساعد ضمير شيث. يعود إلى آدم وحواء.}

\par 1 عندما سمع شيث هذه الكلمات، اندهش، وأمال قلبه إلى كلام الشيطان الغادر، وقال له: "أقلتَ أن هناك عالمًا آخر مخلوقًا غير هذا؛ ومخلوقات أخرى أجمل من المخلوقات التي في هذا العالم؟"

\par 2 فقال الشيطان: «نعم، ها قد سمعتني، ولكني سأمدحهم وطرقهم في مسمعك».

\par 3 فقال له شيث: "لقد أذهلني كلامك، ووصفك الجميل لكل شيء

\par 4 «ومع ذلك، لا أستطيع الذهاب معك اليوم؛ ليس حتى أذهب إلى أبي آدم وأمي حواء، وأخبرهما بكل ما قلته لي. ثم إذا سمحا لي بالذهاب معك، فسآتي.»

\par 5 قال شيث مرة أخرى: "أخشى أن أفعل أي شيء دون إذن أبي وأمي، خشية أن أهلك مثل أخي قابيل، ومثل أبي آدم، اللذين خالفا وصية الله. ولكن، انظر، أنت تعرف هذا المكان؛ تعال وقابلني هنا غدًا."

\par 6 فلما سمع الشيطان هذا، قال لشيث: "إذا أخبرت أباك آدم بما قلته لك، فلن يدعك تأتي معي

\par 7 لكن اسمع لي؛ لا تخبر أباك وأمك بما قلته لك؛ بل تعال معي اليوم، إلى عالمنا؛ حيث سترى أشياء جميلة وتستمتع هناك، وتحتفل بهذا اليوم بين أبنائي، تنظر إليها وتشبع من البهجة؛ وتفرح أكثر فأكثر. ثم سأعيدك إلى هذا المكان غدًا؛ ولكن إذا كنت تفضل البقاء معي، فليكن

\par 8 فأجاب شيث: «إن روح أبي وأمي معلقة بي، وإذا اختفيت عنهما يومًا ما، فإنهما يموتان، وسيحاسبني الله على خطيئتي ضدهما.

\par 9 ولولا أنهم علموا أني جئتُ إلى هذا المكان لأُقدّم له قرباني، لما ابتعدوا عني ساعةً واحدة؛ ولا أذهب إلى أي مكان آخر إلا إذا سمحوا لي. لكنهم يعاملونني بلطفٍ بالغ، لأني أعود إليهم سريعًا.

\par 10 فقال له الشيطان: ماذا يحدث لك إن اختفيت عنهم ليلة ثم رجعت إليهم عند طلوع الفجر؟

\par 11 ولكن شيث، عندما رأى كيف استمر في الكلام، وأنه لن يتركه، ركض وصعد إلى المذبح، وبسط يديه إلى الله، وطلب الخلاص منه

\par 12 ثم أرسل الله كلمته، ولعن الشيطان الذي هرب منه.

\par 13 أما شيث، فقد صعد إلى المذبح، قائلاً في قلبه: «المذبح هو موضع التقدمة، والله هناك؛ ستحرقه نار إلهية، ولن يقدر الشيطان على إيذائي، ولن يأخذني منه».

\par 14 ثم نزل شيث عن المذبح وذهب إلى أبيه وأمه حيث وجدهما في الطريق، مشتاقًا إلى سماع صوته، لأنه تأخر قليلًا

\par 15 ثم بدأ يخبرهم بما أصابه من الشيطان، في صورة ملاك

\par 16 ولكن عندما سمع آدم قصته، قبّل وجهه، وحذره من ذلك الملاك، قائلاً له إن الشيطان هو الذي ظهر له هكذا. ثم أخذ آدم شيث، وذهبا إلى مغارة الكنوز، وفرحوا بها

\par 17 ولكن منذ ذلك اليوم فصاعدًا، لم ينفصل آدم وحواء عنه أبدًا، إلى أي مكان قد يذهب إليه، سواء من أجل قربانه أو من أجل أي شيء آخر

\par 18 حدثت هذه العلامة لسيث عندما كان في التاسعة من عمره.

\chapter{7}

\par \textit{سيث يتزوج أكليا. آدم يعيش ليرى أحفاده وأبناء أحفاده.}

\par 1 عندما رأى أبونا آدم أن شيثًا كان كامل القلب، أراد أن يتزوجه؛ لئلا يظهر له العدو مرة أخرى ويتغلب عليه

\par 2 فقال آدم لابنه شيث: "أريد يا بني أن تتزوج أختك أكلية، أخت هابيل، لتلد لك أولادًا يملؤون الأرض، حسب وعد الله لنا

\par 3 «لا تخف يا بني، ليس في ذلك عيب. أريدك أن تتزوج خوفًا من أن يتغلب عليك العدو.»

\par 4 ومع ذلك، لم يرغب سيث في الزواج؛ ولكن طاعةً لأبيه وأمه، لم ينطق بكلمة واحدة

\par 5 فزوجه آدم من أكليا. وكان عمره خمس عشرة سنة.

\par 6 ولما بلغت العشرين من عمرها، ولد ابنا سماه أنوش، ثم ولد أولاد آخرين غيره.

\par 7 ثم كبر أنوش وتزوج وولد قينان.

\par 8 وكبر قينان أيضًا وتزوج وأنجب مهلائيل.

\par 9 وُلِد هؤلاء الآباء في حياة آدم، وسكنوا بجوار كهف الكنوز

\par 10 ثم كانت أيام آدم تسعمائة وثلاثين سنة، وأيام مهللئيل مئة سنة. ولكن مهللئيل، لما كبر، أحب الصوم والصلاة والتعب الشديد، إلى أن اقتربت نهاية أيام أبينا آدم

\chapter{8}

\par \textit{كلمات آدم الأخيرة الرائعة. يتنبأ بالطوفان. يحث نسله على الخير. يكشف بعض أسرار الحياة.}

\par 1 ولما رأى أبونا آدم أن نهايته قريبة، دعا ابنه شيث، فجاء إليه في مغارة الكنوز، وقال له:

\par 2 "يا شيث، يا بني أحضر لي أولادك وأولاد أولادك، حتى أتمكن من إفساح المجال لهم قبل أن أموت."

\par 3 عندما سمع شيث هذه الكلمات من أبيه آدم، ابتعد عنه، وذرف سيلًا من الدموع على وجهه، وجمع أولاده وأولاد أولاده، وأتى بهم إلى أبيه آدم

\par 4 ولكن عندما رآهم أبونا آدم حوله، بكى لاضطراره إلى الانفصال عنهم

\par 5 فلما رأوه يبكي، بكوا جميعهم معًا، وسقطوا على وجهه قائلين: "كيف تُفصل عنا يا أبانا؟ وكيف تقبلك الأرض وتخفيك عن أعيننا؟" هكذا ندبوا كثيرًا، وبكلمات مماثلة

\par 6 ثم باركهم أبونا آدم جميعاً، وقال لشيث بعد أن باركهم:

\par 7 يا سيث، يا بني، أنت تعلم أن هذا العالم مليء بالحزن والتعب، وتعلم كل ما حل بنا من تجاربنا فيه. لذلك أوصيك الآن بهذه الكلمات: أن تحافظ على البراءة، وأن تكون طاهرًا وعادلًا، وأن تتوكل على الله، ولا تعتمد على أحاديث الشيطان، ولا على الأشباح التي سيظهر لك فيها

\par 8 لكن احفظ الوصايا التي أنا أوصيك بها اليوم، ثم أعطها لابنك أنوش، وليعطها أنوش لابنه قينان، وقينان لابنه مهللئيل، حتى تبقى هذه الوصية ثابتة في جميع أبنائك

\par 9 يا سيث، يا بني، في اللحظة التي أموت فيها، خذ جسدي ولفه بالمر والعود والقرفة، واتركني هنا في كهف الكنوز هذا الذي يحتوي على كل هذه الرموز التي أعطانا إياها الله من الجنة

\par 10 «يا بني، سيأتي بعد ذلك طوفانٌ يغمر جميع المخلوقات، ولا يترك سوى ثماني أرواح

\par 11 «ولكن يا بني، فليأخذ من يُترك من بين أبنائك في ذلك الوقت جسدي معهم من هذا الكهف؛ وعندما يأخذونه معهم، فليأمر أكبرهم سنًا أبناءه بوضع جسدي في سفينة حتى يهدأ الطوفان ويخرجوا من السفينة

\par 12 ثم يأخذون جسدي ويضعونه في وسط الأرض، بعد وقت قصير من نجاتهم من مياه الطوفان

\par 13 «لأن المكان الذي سيُوضع فيه جسدي هو وسط الأرض، ومن هناك سيأتي الله ويخلص جميع أقربائنا

\par 14 «ولكن الآن يا شيث يا بني، ضع نفسك على رأس شعبك؛ ارعهم واحرسهم في مخافة الله؛ واهدهم في الطريق الصالح؛ وأمرهم بالصوم لله؛ وأفهمهم أنه لا ينبغي لهم الاستماع إلى الشيطان لئلا يهلكهم

\par 15 «ثم افصل أولادك وأولاد أولادك عن أولاد قابيل؛ لا تدعهم يختلطون بهم أبدًا، ولا يقتربون منهم لا بأقوالهم ولا بأفعالهم.»

\par 16 ثم أنزل آدم بركاته على شيث، وعلى أبنائه، وعلى جميع أبناء أبنائه

\par 17 ثم التفت إلى ابنه شيث وزوجته حواء، وقال لهما: «احفظا هذا الذهب، وهذا البخور، وهذا المر، الذي أعطانا إياه الله علامة؛ لأنه في الأيام القادمة، سيغمر طوفان الخليقة كلها. أما الذين يدخلون الفلك فسيأخذون معهم الذهب والبخور والمر مع جسدي، ويضعون الذهب والبخور والمر مع جسدي في وسط الأرض.»

\par 18 «ثم بعد زمن طويل، تُنهب المدينة التي وُجد فيها الذهب والبخور والمر مع جسدي. ولكن عندما تُنهب، يُؤخذ الذهب والبخور والمر مع الغنيمة المحفوظة؛ ولن يهلك شيء منها، حتى يأتي كلمة الله المتجسد؛ عندما يأخذها الملوك ويقدمون له، ذهبًا رمزًا لكونه ملكًا؛ وبخورًا رمزًا لكونه إله السماء والأرض؛ ومرًا رمزًا لآلامه.»

\par 19 «البرد أيضًا، كرمز لتغلبه على الشيطان، وجميع أعدائنا؛ والبخور كرمز لقيامته من بين الأموات، وتعاليه فوق ما في السماء وما على الأرض؛ والمر، كرمز لشربه المرارة؛ وشعوره بآلام الجحيم من الشيطان

\par 20 «والآن يا شيث، يا بني، ها قد كشفت لك أسرارًا خفية كشفها الله لي. احفظ وصيتي لنفسك ولشعبك.»

\chapter{9}

\par \textit{موت آدم.}

\par 1 عندما انتهى آدم من وصيته لشيث، ارتخى جسده، وفقدت يداه وقدماه كل قوة، وأصبح فمه أخرس، وتوقف لسانه تمامًا عن الكلام. أغمض عينيه وأسلم الروح

\par 2 ولكن عندما رأى أبناؤه أنه مات، ألقوا أنفسهم عليه، رجالاً ونساءً، شيوخًا وشبابًا، يبكون

\par 3 حدثت وفاة آدم في نهاية تسعمائة وثلاثين عامًا عاشها على الأرض؛ في اليوم الخامس عشر من برمودة، بعد حساب كسوف الشمس، في الساعة التاسعة

\par 4 كان يوم جمعة، وهو نفس اليوم الذي خُلِق فيه، والذي فيه استراح؛ والساعة التي مات فيها هي نفس الساعة التي خرج فيها من الجنة

\par 5 ثم لفه شيث جيدًا، وحنطه بكمية كبيرة من الأطياب العطرة، من الأشجار المقدسة ومن الجبل المقدس؛ ووضع جسده على الجانب الشرقي من داخل الكهف، جانب البخور؛ ووضع أمامه منارة مشتعلة

\par 6 ثم وقف أبناؤه أمامه يبكون وينوحون عليه طوال الليل حتى طلوع الفجر

\par 7 فخرج شيث وابنه الكبير أنوش وقينان بن أنوش، وأخذوا تقدمات جيدة ليقدموها للرب، وجاءوا إلى المذبح الذي كان آدم يقدم عليه تقدمات لله، حين قدمها

\par 8 فقالت لهم حواء: "انتظروا حتى نطلب من الله أولاً أن يقبل قرباننا، وأن يحفظ عنده نفس آدم عبده، وأن يأخذها إلى الراحة."

\par 9 وقاموا جميعًا وصلوا.

\chapter{10}

\par \textit{"آدم كان الأول...."}

\par 1 ولما انتهوا من صلاتهم، جاءت كلمة الله وعزاهم عن أبيهم آدم

\par 2 بعد ذلك، قدموا هداياهم لأنفسهم ولأبيهم

\par 3 ولما انتهوا من تقديم قربانهم، جاء كلام الله إلى شيث، أكبرهم، قائلاً له: "يا شيث، شيث، شيث، ثلاث مرات. كما كنت مع أبيك، سأكون معك أيضًا، حتى يتم الوفاء بالوعد الذي قطعته له - قال أبوك: سأرسل كلمتي وأخلصك ونسلك

\par 4 وأما أنت فأبوك آدم، فاحفظ الوصية التي أوصاك بها، واقطع نسلك من نسل قابيل أخيك

\par 5 وسحب الله كلمته من شيث.

\par 6 ثم نزل شيث وحواء وأبناؤهما من الجبل إلى مغارة الكنوز

\par 7 لكن آدم كان أول من ماتت روحه في أرض عدن، في مغارة الكنوز؛ لأنه لم يمت أحد قبله إلا ابنه هابيل، الذي مات مقتولاً

\par 8 فقام جميع بني آدم وبكوا على أبيهم آدم، وقدموا له قرابين مئة وأربعين يومًا

\chapter{11}

\par \textit{أصبح سيث رئيسًا لأكثر قبيلة من الناس سعادةً وعدلاً على الإطلاق.}

\par 1 بعد موت آدم وحواء، فصل شيث أبناءه وأبناء أبنائه عن أبناء قابيل. فنزل قابيل ونسله وسكنوا غربًا، أسفل المكان الذي قتل فيه أخاه هابيل.

\par 2 وأما شيث وأولاده، فسكنوا شمالاً على جبل مغارة الكنوز، لكي يكونوا قريبين من أبيهم آدم.

\par 3 "ووقف شيث الأكبر، طويل القامة، حسن النفس، قوي النفس، على رأس قومه، وكان يربيهم ببراءة وتوبة ووداعة، ولم يدع أحداً منهم ينزل إلى بني قايين."

\par 4 ولكن بسبب طهارتهم، سُموا "أبناء الله"، وكانوا مع الله، بدلاً من جيوش الملائكة الذين سقطوا؛ لأنهم استمروا في التسبيح لله، وفي غناء المزامير له، في كهفهم - مغارة الكنوز.

\par 5 ثم وقف شيث أمام جسد أبيه آدم وأمه حواء، وصلى ليلاً ونهاراً، وطلب الرحمة لنفسه ولأولاده، وأنه عندما يواجه صعوبة في التعامل مع طفل، سيعطيه النصيحة.

\par 6 ولكن شيث وأولاده لم يحبوا العمل الأرضي، بل انصرفوا إلى الأمور السماوية؛ إذ لم يكن لديهم أي فكر آخر غير التسبيح والتمجيد والمزامير لله.

\par 7 ولذلك كانوا يسمعون في كل وقت أصوات الملائكة، يسبحون ويمجدون الله؛ من داخل الجنة، أو عندما أرسلهم الله في مهمة، أو عندما كانوا صاعدين إلى السماء.

\par 8 لأن شيث وأولاده، بفضل نقائهم، سمعوا ورأوا هؤلاء الملائكة. ثم، لم تكن الجنة بعيدة عنهم، بل كانت على بُعد خمسة عشر ذراعًا روحية فقط.

\par 9 والآن، فإن ذراعًا روحية واحدة تعادل ثلاثة أذرع من الإنسان، أي ما مجموعه خمسة وأربعون ذراعًا

\par 10 سكن شيث وأبناؤه على الجبل أسفل الجنة، فلم يزرعوا ولم يحصدوا، ولم يصنعوا طعامًا للجسد، ولا حتى قمحًا، بل تقدمات فقط. وكانوا يأكلون من ثمار وأشجار طيبة الطعم تنمو على الجبل الذي سكنوا فيه.

\par 11 ثم كان شيث يصوم كثيرًا كل أربعين يومًا، وكذلك كان يفعل أبناؤه الأكبر. لأن عائلة شيث كانت تشتم رائحة أشجار الحديقة عندما تهب الريح في اتجاهها

\par 12 كانوا سعداء، أبرياء، بلا خوف مفاجئ، لم تكن بينهم غيرة، ولا فعل شرير، ولا كراهية. لم تكن بينهم هوى حيواني؛ ولم تخرج من أفواههم كلمات بذيئة أو لعنات؛ ولا مشورة شريرة ولا احتيال. لأن رجال ذلك الزمان لم يقسموا قط، ولكن في ظل الظروف الصعبة، عندما كان على الرجال أن يقسموا، كانوا يقسمون بدم هابيل البار

\par 13 لكنهم أجبروا أطفالهم ونساءهم كل يوم في الكهف على الصيام والصلاة وعبادة الله العلي. وتباركوا في جسد أبيهم آدم، ودهنوا أنفسهم به

\par 14 وفعلوا ذلك حتى اقتربت نهاية شيث.

\chapter{12}

شؤون عائلة شيث. وفاته. رئاسة أنوش. كيف كان حال الفرع المنبوذ من عائلة آدم؟

\par 1 ثم دعا شيث البار ابنه أنوش، وقينان بن أنوش، ومهللئيل بن قينان، وقال لهم:

\par 2 «بما أن نهايتي قريبة، أرغب في بناء سقف فوق المذبح لتقديم الهدايا عليه.»

\par 3 سمعوا أمره وخرجوا جميعًا، كبارًا وصغارًا، واجتهدوا فيه، وبنوا سقفًا جميلًا فوق المذبح

\par 4 وكان فكر شيث في القيام بذلك هو أن تحل البركة على أبنائه على الجبل، وأنه يجب عليه أن يقدم لهم قربانًا قبل وفاته

\par 5 ولما انتهى بناء السقف، أمرهم بتقديم قرابين. فاجتهدوا في ذلك، وأتوا بها إلى شيث أبيهم، فأخذها وقدمها على المذبح، وصلى إلى الله أن يقبل قرابينهم، وأن يرحم نفوس أبنائه، وأن يحفظهم من يد الشيطان

\par 6 فقبل الله تقدمته، وأرسل بركاته عليه وعلى أبنائه. ثم قطع الله وعدًا لشيث قائلًا: "في نهاية الأيام الخمسة والنصف العظيمة، التي وعدتك بها أنت وأبيك، سأرسل كلمتي وأخلصك ونسلك."

\par 7 ثم اجتمع شيث وأبناؤه، وأبناء أبنائه، ونزلوا من المذبح، وذهبوا إلى مغارة الكنوز، حيث صلوا، وتباركوا في جسد أبينا آدم، ومسحوا أنفسهم به

\par 8 لكن شيث مكث في كهف الكنوز بضعة أيام، ثم عانى - معاناة حتى الموت

\par 9 فجاء إليه أنوش ابنه البكر وقينان ابنه ومهللئيل بن قينان ويارد بن مهللئيل وحنوك بن يارد ونسائهم وأولادهم ليأخذوا بركة من شيث.

\par 10 ثم صلى عليهم شيث وباركهم وحلف لهم بدم هابيل البار قائلاً: "أتوسل إليكم يا أبنائي ألا تدع أحداً منكم ينزل من هذا الجبل المقدس الطاهر

\par 11 لا تشتركوا مع أبناء قابيل القاتل والخاطئ الذي قتل أخاه، لأنكم تعلمون يا أبنائي أننا نهرب منه ومن كل خطيئته بكل قوتنا لأنه قتل أخاه هابيل

\par 12 بعد أن قال هذا، بارك شيث أنوش، ابنه البكر، وأمره أن يخدم بطهارة أمام جسد أبينا آدم كل أيام حياته؛ ثم أن يذهب أيضًا أحيانًا إلى المذبح الذي بناه شيث. وأمره أن يرعى شعبه بالبر والعدل والطهارة كل أيام حياته

\par 13 ثم انحلت أطراف شيث، وفقدت يداه ورجلاه كل قوة، وأصبح فمه أخرس عاجزًا عن النطق، وأسلم الروح ومات في اليوم التالي لسنته الثانية عشرة والتسعمائة، في اليوم السابع والعشرين من شهر أبيب، وكان أخنوخ آنذاك ابن عشرين عامًا

\par 14 ثم لفوا جسد شيث بعناية، وحنطوه بأطياب عطرة، ووضعوه في مغارة الكنوز، على الجانب الأيمن من جسد أبينا آدم، وندبوه أربعين يومًا. وقدموا له هدايا كما فعلوا لأبينا آدم

\par 15 بعد وفاة شيث، تولى أنوش قيادة شعبه، فأطعمهم بالبر والعدل، كما أمره أبوه

\par 16 ولكن عندما بلغ أنوش ثمانمائة وعشرين عامًا، كان لقابيل ذرية كبيرة؛ لأنهم كانوا يتزوجون كثيرًا، منغمسين في الشهوات الحيوانية؛ حتى امتلأت الأرض أسفل الجبل بهم

\chapter{13}

\par \textit{"وكان بين أبناء قابيل سرقة وقتل وخطيئة كثيرة."}

\par 1 في تلك الأيام عاش لامك الأعمى، وهو من أبناء قايين. وكان له ابن اسمه أتون، وكان لهما ماشية كثيرة

\par 2 لكن لامك كان معتادًا على إرسالهم ليرعوا عند راعٍ شاب، كان يرعاهم؛ وكان عندما يعود إلى المنزل في المساء يبكي أمام جده، وأمام أبيه أتون وأمه حزينة، ويقول لهم: "أما أنا، فلا أستطيع أن أرعى تلك الماشية وحدي، خشية أن يسلبني أحد بعضها، أو يقتلني من أجلها." لأنه كان بين أبناء قابيل الكثير من السرقة والقتل والخطيئة

\par 3 ثم أشفق عليه لامك، وقال: «حقًا، إنه وهو وحيد، يغلبه رجال هذا المكان».

\par 4 فقام لامك، وأخذ قوسًا كان يحفظه منذ صغره، قبل أن يُصاب بالعمى، وأخذ سهامًا كبيرة وحجارةً ملساء ومقلاعًا كان معه، وذهب إلى الحقل مع الراعي الشاب، ووقف خلف الماشية، بينما كان الراعي الشاب يراقب الماشية. وهكذا فعل لامك أيامًا كثيرة.

\par 5 في هذه الأثناء، لم يستطع قابيل، منذ أن طرده الله، ولعنه بالرعدة والرعب، أن يستقر أو يجد راحة في أي مكان؛ بل كان يتجول من مكان إلى آخر

\par 6 وفي تجواله، جاء إلى زوجات لامك، وسألهن عنه. فقلن له: «هو في الحقل مع الماشية».

\par 7 فذهب قابيل ليبحث عنه، ولما دخل الحقل، سمع الراعي الشاب ضجيجه، ورعي الماشية من أمامه

\par 8 ثم قال للامك: «يا سيدي، أهذا وحش بري أم لص؟»

\par 9 فقال له لامك: «أفهمني في أي اتجاه ينظر حين يصعد».

\par 10 فثنى لامك قوسه، ووضع عليه سهمًا، وثبت حجرًا في المقلاع. ولما خرج قايين من البرية، قال الراعي للامك: «ارمِ، هوذا قادم».

\par 11 فرمى لامك قابيل بسهمه فأصابه في جنبه. فضربه لامك بحجر من مقلاعه فسقط على وجهه ففقأ عينيه، فسقط قابيل في الحال ومات

\par 12 فتقدم إليه لامك والراعي الشاب، فوجداه ملقى على الأرض. فقال له الراعي الشاب: «هو قابيل جدنا الذي قتلته يا سيدي!»

\par 13 فحزن لامك على ذلك، ومن مرارة ندمه، صفق بيديه معًا، وضرب بكفه المسطحة رأس الشاب، فسقط كما لو كان ميتًا؛ لكن لامك ظن أنها خدعة؛ فأخذ حجرًا وضربه، وهشم رأسه حتى مات

\chapter{14}

\par \textit{الزمن، كجدول متدفق، يحمل جيلًا آخر من الرجال.}

\par 1 ولما بلغ أنوش تسعمائة سنة، اجتمع حوله جميع أبناء شيث وقينان وبكره، مع زوجاتهم وأولادهم، طالبين منه البركة

\par 2 ثم صلى عليهم وباركهم، وأقسم عليهم بدم هابيل البار قائلاً لهم: «لا ينزل أحد من أولادكم من هذا الجبل المقدس، ولا يخالط أبناء قايين القاتل».

\par 3 ثم دعا أنوش ابنه قينان وقال له: «انظر يا ابني، واجعل قلبك على شعبك، وثبتهم في البر والطهارة، وقف خادمًا أمام جسد أبينا آدم كل أيام حياتك».

\par 4 بعد ذلك، دخل أنوش في الراحة، وكان عمره تسعمائة وخمسة وثمانين عامًا، ولفه قينان ووضعه في مغارة الكنوز عن يسار أبيه آدم، وقدم له قرابين، كعادة آبائه

\chapter{15}

\par \textit{يستمر أبناء آدم في الاحتفاظ بكهف الكنوز كمزار عائلي.}

\par 1 بعد وفاة أنوش، وقف قينان على رأس شعبه في بر وبراءة، كما أمره أبوه؛ كما استمر في الخدمة أمام جسد آدم، داخل مغارة الكنوز

\par 2 ثم لما عاش تسعمائة وعشر سنوات، أصابه الضيق والشدة. ولما كاد أن يدخل الراحة، جاء إليه جميع الآباء مع نسائهم وأولادهم، فباركهم، وأقسم عليهم بدم هابيل البار، قائلاً لهم: «لا ينزل أحد منكم من هذا الجبل المقدس، ولا تصاحبوا أبناء قايين القاتل».

\par 3 تلقى مهللئيل، ابنه البكر، هذه الوصية من أبيه، الذي باركه ومات

\par 4 فحنطه مهللئيل بأطياب عطرة، ووضعه في مغارة الكنوز مع آبائه، وقدموا له قرابين كعادة آبائهم



\chapter{16}

\par \textit{الفرع الصالح من العائلة لا يزال خائفًا من أبناء قابيل.}

\par 1 ثم وقف مهللئيل على شعبه، وأطعمهم بالبر والطهارة، وراقبهم ليتأكد من أنهم لا يخالطون أبناء قابيل.

\par 2 واستمر أيضاً في مغارة الكنوز يصلي ويخدم أمام جسد أبينا آدم، طالباً من الله الرحمة لنفسه ولشعبه، إلى أن بلغ ثمانمائة وسبعين سنة حين مرض.

\par 3 ثم اجتمع إليه جميع أولاده لرؤيته وطلب بركاته عليهم جميعًا قبل أن يغادر هذا العالم.

\par 4 فقام مهللئيل وجلس على سريره ودموعه تسيل على وجهه، ودعا ابنه الأكبر يارد، فجاء إليه.

\par 5 ثم قبل وجهه وقال له: "يا يارد، يا بني، أستحلفك بمن خلق السماء والأرض، أن تحرس شعبك، وأن تطعمهم بالبر والبراءة، وأن لا تدع أحدًا منهم ينزل من هذا الجبل المقدس إلى بني قابيل، لئلا يهلك معهم

\par 6 «اسمع يا بني، سيأتي بعد ذلك دمار عظيم على هذه الأرض بسببهم؛ سيغضب الله على العالم، وسيهلكهم بالمياه

\par 7 «ولكني أعلم أيضًا أن أولادك لن يسمعوا لك، وأنهم سينزلون من هذا الجبل ويعاشرون أولاد قابيل، فيهلكون معهم

\par 8 «يا بني! علمهم، واحذرهم، حتى لا يلحقك بهم ذنب.»

\par 9 قال مهللئيل، علاوة على ذلك، لابنه يارد: "عندما أموت، حنط جسدي وضعه في مغارة الكنوز، بجانب أجساد آبائي؛ ثم قف بجانب جسدي وصلِّ إلى الله؛ واعتنِ بهم، وأتم خدمتك أمامهم، حتى تدخل أنت إلى الراحة."

\par 10 ثم بارك مهللئيل جميع أبنائه، ثم اضطجع على فراشه، ودخل في راحة مثل آبائه

\par 11 ولما رأى يارد أن أباه مهللئيل قد مات بكى وحزن واحتضنه وقبل يديه ورجليه، وكذلك فعل كل بنيه.

\par 12 فحنطه أبناؤه بتأنٍ، ووضعوه بجانب أجساد آبائه. ثم قاموا وندبوه أربعين يومًا

\chapter{17}

\par \textit{يتحول جاريد إلى رجل صارم. يتم استدراجه إلى أرض قابيل حيث يرى العديد من المناظر الشهوانية. ينجو جاريد بصعوبة بقلب نقي.}

\par 1 ثم حفظ يارد وصية أبيه، وقام كالأسد على شعبه. رعاهم بالبر والبراءة، وأمرهم ألا يفعلوا شيئًا بدون مشورته. لأنه كان خائفًا عليهم لئلا يذهبوا إلى بني قايين

\par 2 ولهذا السبب أصدر لهم الأوامر مرارًا وتكرارًا، واستمر في ذلك حتى نهاية السنة الرابعة والثمانين والخمسة والثمانين من حياته

\par 3 في نهاية هذه السنوات المذكورة، ظهرت له هذه العلامة. بينما كان يارد واقفًا كالأسد أمام جثث آبائه، يصلي ويحذر شعبه، حسده الشيطان، وصنع له ظهورًا جميلًا، لأن يارد لم يكن ليدع أولاده يفعلون أي شيء دون مشورته

\par 4 ثم ظهر له الشيطان مع ثلاثين رجلاً من جنوده، في صورة رجال وسيمين؛ وكان الشيطان نفسه أكبرهم سنًا وأطولهم، وله لحية جميلة

\par 5 وقفوا عند مدخل الكهف، ونادوا جاريد من داخله

\par 6 فخرج إليهم، فوجدهم كأنهم رجالٌ جميلون، مليئون بالنور، وذوي جمالٍ عظيم. فتعجب من جمالهم ومن منظرهم، وفكر في نفسه: هل هم من بني قابيل؟

\par 7 وقال أيضًا في قلبه: "بما أن أبناء قابيل لا يستطيعون الصعود إلى ارتفاع هذا الجبل، وليس فيهم من هو وسيم مثل هؤلاء، وليس بين هؤلاء الرجال أحد من أقاربي، فلا بد أنهم غرباء."

\par 8 ثم تبادل يارد وهم السلام، وقال لشيخهم: «يا أبي، فسّر لي العجيبة التي فيك، وأخبرني من هؤلاء الذين معك؛ فإنهم يبدون لي كأشخاص غرباء».

\par 9 ثم بدأ الشيخ يبكي، وبكى الباقون معه، وقال ليارد: "أنا آدم الذي خلقه الله أولاً، وهذا هو هابيل ابني، الذي قتله أخوه قابيل، الذي وضع الشيطان في قلبه ليقتله

\par 10 «فهذا هو ابني شيث الذي سألته من الرب فأعطاني إياه ليعزيني بدلًا من هابيل

\par 11 «فهذا ابني أنوش بن شيث، وهذا الآخر قينان بن أنوش، وهذا الآخر مهللئيل بن قينان أبوك.»

\par 12 وأما يارد فبقي متعجباً من منظرهم ومن كلام الشيخ له.

\par 13 ثم قال له الشيخ: "لا تتعجب يا بني، نحن نعيش في الأرض الواقعة شمال الجنة التي خلقها الله قبل العالم. لم يدعنا نعيش هناك، بل وضعنا داخل الجنة التي أنتم تسكنون أسفلها الآن

\par 14 "ولكن بعد أن عصيت، أخرجني من ذلك، وتركت ساكنًا في هذا الكهف؛ وحدثت لي مصاعب عظيمة ومؤلمة؛ وعندما اقتربت مني الوفاة، أمرت ابني شيث أن يعتني بشعبه جيدًا؛ وهذه وصيتي يجب أن تُسلم من واحد إلى آخر، إلى نهاية الأجيال القادمة.

\par 15 «لكن يا يارد يا بني، نحن نعيش في مناطق جميلة، بينما أنت تعيش هنا في بؤس، كما أخبرني أبوك مهللئيل، إذ أخبرني أن طوفانًا عظيمًا سيأتي ويغمر الأرض كلها

\par 16 «لذلك يا بني، خوفًا عليك، نهضتُ وأخذتُ أطفالي معي، وجئتُ إلى هنا لنزورك أنت وأطفالك؛ لكنني وجدتُك واقفًا في هذا الكهف تبكي، وأطفالك مشتتون حول هذا الجبل، في الحرّ والبؤس

\par 17 «لكن يا بني، عندما ضللنا طريقنا، ووصلنا إلى هذا الحد، وجدنا رجالًا آخرين أسفل هذا الجبل؛ يسكنون بلدًا جميلًا، مليئًا بالأشجار والفواكه، وجميع أنواع الخضرة؛ إنه مثل حديقة؛ لذلك عندما وجدناهم ظننا أنهم أنت؛ إلى أن أخبرني والدك مهللئيل أنهم ليسوا كذلك

\par 18 «الآن يا بني، استمع لنصيحتي، وانزل إليهم أنت وأولادك. سترتاحون من كل هذه المعاناة التي أنتم فيها. ولكن إن لم تنزل إليهم، فقم، خذ أولادك، وتعال معنا إلى حديقتنا؛ ستعيشون في أرضنا الجميلة، وسترتاحون من كل هذا العناء الذي تتحملونه أنتم وأولادك الآن.»

\par 19 ولكن يارد عندما سمع هذا الحديث من الشيخ، تعجب، وذهب هنا وهناك، ولكن في تلك اللحظة لم يجد أحدًا من أبنائه

\par 20 ثم أجاب وقال للشيخ: "لماذا أخفيتما أنفسكما إلى هذا اليوم؟"

\par 21 فأجاب الشيخ: "لو لم يخبرنا أبوك، لما علمنا."

\par 22 ثم صدق جاريد أن كلماته صحيحة.

\par 23 فقال الشيخ ليارد: «لماذا رجعت يا فلان؟» فقال: «كنت أبحث عن أحد أبنائي لأخبره عن ذهابي معك وعن نزولهم إلى الذين تحدثت عنهم».

\par 24 عندما سمع الشيخ نية يارد، قال له: "دع هذا الهدف الآن، وتعال معنا؛ سترى بلادنا. إذا كانت الأرض التي نعيش فيها ترضيك، فسنعود نحن وأنت إلى هنا ونأخذ عائلتك معنا. ولكن إذا لم تكن بلادنا ترضيك، فستعود إلى مكانك."

\par 25 وحث الشيخ يارد على الذهاب قبل أن يأتي أحد أبنائه لينصحه بخلاف ذلك

\par 26 فخرج يارد من الكهف ومضى معهم، وفي وسطهم. فعزاه حتى وصلوا إلى قمة جبل بني قابيل

\par 27 ثم قال الشيخ لأحد رفاقه: "لقد نسينا شيئًا عند مدخل الكهف، وهو الثوب المختار الذي أحضرناه لنلبس به يارد."

\par 28 ثم قال لأحدهم: «ارجع يا رجل، وسننتظرك هنا حتى تعود. ثم سنلبس يارد، فيكون مثلنا، صالحًا، وسيمًا، وصالحًا للقدوم معنا إلى بلادنا».

\par 29 ثم عاد ذلك الشخص.

\par 30 ولكن عندما كان على مسافة قصيرة، ناداه الشيخ وقال له: "انتظر حتى أصعد وأتحدث إليك."

\par 31 ثم وقف ساكنًا، فتقدم إليه الشيخ وقال له: "شيء واحد نسيناه في الكهف، وهو هذا - إطفاء المصباح الذي يشتعل بداخله، فوق الجثث التي فيه. إذًا عُد إلينا سريعًا."

\par 32 فذهب ذاك، ورجع الشيخ إلى أصحابه وإلى يارد. فنزلوا من الجبل، ويارد معهم، وأقاموا عند عين ماء، قرب بيوت بني قابيل، وانتظروا صاحبهم حتى أحضر الرداء ليارد

\par 33 ثم عاد إلى الكهف، وأطفأ المصباح، وجاء إليهم وأحضر معه شبحًا وأراهم إياه. وعندما رآه يارد، تعجب من جماله وجماله، وفرح في قلبه معتقدًا أن كل ذلك صحيح

\par 34 وفيما هم مقيمون هناك، دخل ثلاثة منهم بيوت بني قابيل، وقالوا لهم: «هاتوا لنا اليوم طعامًا من عند عين الماء، لنأكل نحن وأصحابنا».

\par 35 فلما رآهم بنو قابيل تعجبوا منهم وقالوا: "إن هؤلاء جميلو المنظر، لم نرَ مثلهم من قبل". فقاموا وجاءوا معهم إلى عين الماء ليروا أصحابهم

\par 36 وجدوهم وسيمين للغاية، لدرجة أنهم بكوا بصوت عالٍ حول أماكنهم ليجتمع الآخرون ويأتوا وينظروا إلى هذه المخلوقات الجميلة. ثم اجتمع حولهم رجال ونساء

\par 37 ثم قال لهم الشيخ: "نحن غرباء في أرضكم، فأتونا بطعام جيد واشربوا أنتم ونسائكم لنتغذّى معكم."

\par 38 عندما سمع أولئك الرجال كلمات الشيخ هذه، أحضر كل واحد من أبناء قابيل زوجته، وأحضر آخر ابنته، وهكذا جاءت إليهم نساء كثيرات، كل واحدة منهن تخاطب يارد إما باسم نفسها أو باسم زوجته، جميعهن على حد سواء

\par 39 فلما رأى يارد ما فعلوه، انتزع نفسه منهم، فلم يذق طعامهم ولا شرابهم

\par 40 رأى الشيخ تلميحًا وهو ينتزع نفسه منهم، وقال له: "لا تحزن؛ أنا الشيخ العظيم، كما ستراني أفعل، افعل أنت أيضًا مثلي."

\par 41 ثم بسط يديه وأخذ إحدى النساء، ففعل خمسة من أصحابه الشيء نفسه أمام يارد، ليفعل مثلهن

\par 42 ولكن عندما رآهم جاريد يرتكبون العار، بكى وقال في نفسه: لم يفعل آبائي مثل هذا قط

\par 43 ثم بسط يديه وصلى بقلب حار وبكاء كثير، وتوسل إلى الله أن ينقذه من أيديهم

\par 44 وما إن بدأ يارد بالصلاة حتى هرب الشيخ مع رفاقه، لأنهم لم يستطيعوا أن يمكثوا في مكان للصلاة.

\par 45 ثم التفت يارد ولم يستطع رؤيتهم، بل وجد نفسه واقفًا في وسط بني قابيل

\par 46 ثم بكى وقال: "يا رب، لا تهلكني بهذا العرق الذي حذرني منه آبائي؛ لأني الآن، يا ربي، كنت أظن أن الذين ظهروا لي هم آبائي؛ لكنني اكتشفت أنهم شياطين، أغووني بهذا الظهور الجميل، حتى صدقتهم

\par 47 «لكنني الآن أسألك يا الله أن تنقذني من هذا الجيل الذي أقيم بينهم الآن، كما أنقذتني من هؤلاء الشياطين. أرسل ملاكك ليخرجني من وسطهم؛ لأني لا أملك القدرة على الفرار منهم.»

\par 48 ولما انتهى يارد من صلاته، أرسل الله ملاكه في وسطهم، فأخذ يارد وأقامه على الجبل، وأراه الطريق، وأعطاه المشورة، ثم انصرف عنه

\chapter{18}

\par \textit{ارتباك في كهف الكنوز. خطاب معجزي لآدم الميت.}

\par 1 اعتاد أبناء يارد زيارته ساعة بعد ساعة، لتلقي بركاته وطلب نصيحته في كل شيء يفعلونه؛ وعندما كان لديه عمل يقوم به، كانوا يقومون به من أجله

\par 2 لكن هذه المرة عندما دخلوا الكهف، لم يجدوا يارد، بل وجدوا السراج مطفأً، وأجساد الآباء ملقاة في كل مكان، وخرجت منهم أصوات بقوة الله تقول: "لقد خدع الشيطان ابننا في ظهور، راغبًا في إهلاكه كما أهلك ابننا قابيل".

\par 3 وقالوا أيضًا: «أيها الرب إله السماء والأرض، نجِّ ابننا من يد الشيطان الذي صنع أمامه ظهورًا عظيمًا وكاذبًا». وتحدثوا أيضًا عن أمور أخرى، بقوة الله

\par 4 فلما سمع بنو يارد هذه الأصوات خافوا، ووقفوا يبكون على أبيهم، لأنهم لم يعرفوا ما الذي أصابه

\par 5 فبكوا عليه ذلك اليوم حتى غروب الشمس.

\par 6 ثم جاء يارد بوجه حزين، بائس في عقله وجسده، حزينًا لأنه انفصل عن أجساد آبائه.

\par 7 ولكن بينما كان يقترب من الكهف، رآه أبناؤه، فأسرعوا إلى الكهف، وتعلقوا برقبته، يصرخون، قائلين له: "يا أبتاه، أين كنت، ولماذا تركتنا كما لم تكن معتادًا؟" ومرة ​​أخرى: "يا أبتاه، عندما اختفيت، انطفأ السراج فوق أجساد آبائنا، وتطايرت الأجساد، وخرجت منها أصوات."

\par 8 عندما سمع يارد ذلك، ندم، ودخل الكهف؛ وهناك وجد الجثث ملقاة، والمصباح مطفأ، والآباء أنفسهم يصلون من أجل خلاصه من يد الشيطان

\par 9 ثم سقط يارد على الجثث واحتضنها، وقال: «يا آبائي، بشفاعتكم، فليُنجّني الله من يد الشيطان! وأرجو منكم أن تطلبوا من الله أن يحفظني ويحفظني منه إلى يوم مماتي».

\par 10 ثم سكتت جميع الأصوات إلا صوت أبينا آدم، الذي خاطب يارد بقوة الله، كما خاطب أحدٌ صاحبه، قائلاً: "يا يارد، يا بني، قدّم قرابين لله لأنه خلصك من يد الشيطان؛ وعندما تُقدّم تلك القرابين، فليكن أن تُقدّمها على المذبح الذي قدّمتُ عليه. ثم احذر أيضًا من الشيطان؛ لأنه خدعني مرات عديدة بظهوراته، راغبًا في إهلاكي، لكن الله خلصني من يده

\par 11 «أَمْرِ شَعْبِكَ أَنْ يَكُونُوا عَلَى حَرَاسَةٍ مِنْهُ، وَلاَ يَكُفُّوا عَنْ تَقْدِيمِ الْقُرْبَانِ لِلَّهِ.»

\par 12 ثم سكت صوت آدم أيضًا، وتعجب يارد وأبناؤه من ذلك. ثم وضعوا الجثث كما كانت أولًا، ووقف يارد وأبناؤه يصلون تلك الليلة كلها حتى طلوع الفجر

\par 13 ثم قدم يارد ذبيحة وقدمها على المذبح كما أمره آدم. ولما صعد إلى المذبح صلى إلى الله طالبًا الرحمة ومغفرة خطيته في انطفاء السراج

\par 14 ثم ظهر الله ليارد على المذبح وباركه وأولاده، وقبل قرابينهم، وأمر يارد أن يأخذ من النار المقدسة من المذبح، وأن يضيء بها السراج الذي ينير جسد آدم

\chapter{19}

\par \textit{أولاد يارد ضلوا الطريق.}

\par 1 ثم كشف الله له مرة أخرى الوعد الذي قطعه لآدم، وشرح له الخمسمائة سنة، وكشف له سر مجيئه إلى الأرض.

\par 2 فقال الله ليارد: «أما النار التي أخذتها من المذبح لإضاءة السراج، فتبقى معك لتضيء للأجساد، ولا تخرج من الكهف حتى يخرج جسد آدم منه

\par 3 لكن يا يارد، اعتنِ بالنار، حتى تظل مشتعلة في المصباح؛ ولا تخرج من الكهف مرة أخرى، حتى تتلقى أمرًا من خلال رؤيا، وليس في شبح، عندما تراه

\par 4 «ثم أوصِ شعبك أيضًا ألا يُخالطوا أبناء قابيل، ولا يتعلموا طرقهم؛ لأني أنا الله الذي لا يُحب البغضاء وأعمال الإثم.»

\par 5 أعطى الله أيضًا العديد من الوصايا الأخرى ليارد، وباركه. ثم سحب كلمته منه

\par 6 ثم تقدم يارد مع أبنائه، وأخذ نارًا، ونزل إلى الكهف، وأضاء السراج أمام جسد آدم، وأعطى شعبه الوصايا كما أمره الله

\par 7 حدثت هذه الآية ليارد في نهاية عامه الأربعمائة والخمسين؛ كما حدث أيضًا مع عجائب أخرى كثيرة لا نسجلها. لكننا نسجل هذه الآية فقط من أجل الاختصار، وحتى لا نطيل سردنا

\par 8 واستمر يارد يُعلّم أولاده ثمانين عامًا، ولكنهم بعد ذلك بدأوا يُخالفون الوصايا التي أوصاهم بها، ويفعلون أشياءً كثيرةً دون مشورته. فبدأوا ينزلون من الجبل المقدس واحدًا تلو الآخر، ويختلطون بأبناء قابيل في شراكاتٍ دنيئة.

\par 9 والسبب الذي من أجله نزل أبناء يارد من الجبل المقدس هو هذا الذي سنكشفه لكم الآن



\chapter{20}

\par \textit{موسيقى فاتنة؛ شراب قوي سُفك بين أبناء قابيل. يرتدون ملابس ملونة. ينظر بنو شيث بعيون مشتاقة. يثورون على المشورة الحكيمة؛ ينزلون الجبل إلى وادي الإثم. لا يستطيعون صعود الجبل مرة أخرى.}

\par 1 بعد أن نزل قابيل إلى أرض ذات تربة داكنة، وتكاثر فيها أولاده، كان هناك واحد منهم اسمه جينون، ابن لامك الأعمى، هو الذي قتل قابيل

\par 2 أما بالنسبة لهذا الجنون، فقد دخل عليه الشيطان في طفولته؛ فصنع أنواعًا مختلفة من الأبواق والقرون، والآلات الوترية، والصنج، والزمار، والقيثارات، والقيثارات، والمزامير؛ وكان يعزف عليها في جميع الأوقات وفي كل ساعة

\par 3 ولما لعب عليهم، دخل الشيطان فيهم، فسمع من بينهم أصوات جميلة وعذبة، فتلفت القلوب

\par 4 ثم جمع فرقًا فوق فرق ليلعب عليها؛ وعندما لعبوا، سُرَّ أبناء قابيل، الذين أشعلوا الخطيئة فيما بينهم، واحترقوا كالنار؛ بينما أشعل الشيطان قلوبهم، بعضهم ببعض، وزاد الشهوة بينهم

\par 5 علّم الشيطان أيضًا جينون أن يُخرج مشروبًا قويًا من الكومة؛ وكان هذا الجينون يجمع جماعاتٍ فوق جماعات في حانات الشرب؛ ويجلب في أيديهم جميع أنواع الفاكهة والزهور؛ وكانوا يشربون معًا

\par 6 وهكذا ضاعف هذا الجينون الخطيئة بشكل كبير؛ كما تصرف بكبرياء، وعلم أبناء قابيل ارتكاب كل أنواع الشرور الجسيمة التي لم يعرفوها؛ ودفعهم إلى أفعال متعددة لم يعرفوها من قبل

\par 7 ثم لما رأى الشيطان أنهم استسلموا لجينون وأصغوا إليه في كل ما قاله لهم، فرح فرحًا عظيمًا، وزاد فهم جينون، حتى أخذ الحديد وصنع منه أسلحة حرب

\par 8 ثم عندما سكروا، ازدادت الكراهية والقتل بينهم؛ استخدم رجل العنف ضد آخر لتعليمه الشر، فأخذ أطفاله ودنسهم أمامه

\par 9 ولما رأى الناس أنهم قد غلبوا، ورأوا آخرين لم يغلبوا، جاء المنهزمون إلى جينون، ولجأوا إليه، فاتخذهم شركاء له.

\par 10 ثم كثرت الخطيئة بينهم جدا حتى تزوج الرجل أخته أو ابنته أو أمه أو غيرهن أو ابنة أخت أبيه، فلم يعد هناك تمييز في القرابة، ولم يعرفوا ما هو الإثم، بل عملوا الشر، وتنجست الأرض بالخطيئة، وأغضبوا الله القاضي الذي خلقهم.

\par 11 لكن جينون جمع فرقًا تلو الأخرى، تعزف على الأبواق وجميع الآلات الأخرى التي ذكرناها سابقًا، عند سفح الجبل المقدس؛ وفعلوا ذلك حتى يسمعها أبناء شيث الذين كانوا على الجبل المقدس

\par 12 ولكن عندما سمع بنو شيث الضجيج، تعجبوا، وجاءوا جماعات، ووقفوا على قمة الجبل لينظروا إلى من هم في الأسفل؛ وفعلوا ذلك لمدة عام كامل

\par 13 عندما رأى جينون، في نهاية ذلك العام، أنهم كانوا ينجذبون إليه شيئًا فشيئًا، دخل الشيطان فيه، وعلمه صنع مواد صباغة للملابس ذات الأنماط المتنوعة، وجعله يفهم كيفية صباغة القرمزي والأرجواني وما إلى ذلك

\par 14 واجتمع بنو قابيل الذين صنعوا كل هذا، وكانوا يتألقون في الجمال والملابس الفاخرة، عند سفح الجبل في بهاء، بالقرون والملابس الفاخرة، وسباقات الخيل، مرتكبين كل أنواع الرجاسات

\par 15 في هذه الأثناء، كان بنو شيث، الذين كانوا على الجبل المقدس، يصلون ويسبحون الله، بدلاً من جحافل الملائكة الذين سقطوا؛ ولذلك دعاهم الله "ملائكة"، لأنه فرح بهم كثيرًا

\par 16 ولكن بعد ذلك، لم يعودوا يحفظون وصيته، ولا يلتزمون بالوعد الذي قطعه لآبائهم؛ بل استراحوا من صيامهم وصلاتهم، ومن مشورة يارد أبيهم. وظلوا يجتمعون على قمة الجبل، لينظروا إلى بني قابيل، من الصباح إلى المساء، وما فعلوه، إلى ثيابهم الجميلة وزينتهم

\par 17 ثم رفع بنو قابيل نظرهم من أسفل، فرأوا بني شيث واقفين في صفوف على رأس الجبل، فنادوا عليهم أن ينزلوا إليهم

\par 18 فقال لهم بنو شيث من فوق: «لا نعرف الطريق». فسمعهم جينون بن لامك يقولون إنهم لا يعرفون الطريق، ففكر كيف ينزلهم

\par 19 ثم ظهر له الشيطان ليلاً قائلاً: «لا سبيل لهم للنزول من الجبل الذي يسكنون فيه. ولكن متى جاؤوا غدًا، فقل لهم: تعالوا إلى الجانب الغربي من الجبل، هناك ستجدون طريق جدول ماء ينزل إلى سفح الجبل بين تلّين. انزلوا إلينا من هناك».

\par 20 ثم عندما طلع النهار، نفخ جينون في الأبواق وقرع الطبول أسفل الجبل، كعادته. فسمعه أبناء شيث، وجاءوا كما اعتادوا أن يفعلوا

\par 21 ثم قال لهم جينون من أسفل: اذهبوا إلى الجانب الغربي من الجبل، وهناك تجدون الطريق للنزول.

\par 22 ولكن عندما سمع أبناء شيث هذه الكلمات منه، عادوا إلى الكهف إلى يارد، ليخبروه بكل ما سمعوه

\par 23 فلما سمع يارد ذلك، حزن، لأنه علم أنهم سيتعدون على مشورته

\par 24 بعد ذلك اجتمع مئة رجل من بني شيث، وقالوا فيما بينهم: "هلموا ننزل إلى بني قابيل، ونرى ماذا يفعلون، ونتلذذ معهم."

\par 25 ولكن عندما سمع يارد هذا عن المئة رجل، تحركت نفسه، وحزن قلبه. ثم نهض بحرارة عظيمة، ووقف في وسطهم، وحلف لهم بدم هابيل البار: "لا ينزل أحد منكم من هذا الجبل المقدس الطاهر، الذي أمر آباؤنا أن يسكن فيه."

\par 26 فلما رأى يارد أنهم لم يقبلوا كلامه، قال لهم: "يا أبنائي الصالحين الأبرياء القديسين، اعلموا أنه عندما تنزلون من هذا الجبل المقدس، فلن يسمح لكم الله بالعودة إليه مرة أخرى."

\par 27 ثم أقسم عليهم مرة أخرى قائلاً: «أقسم بموت أبينا آدم، وبدم هابيل، وشيث، وأنوش، وقينان، ومهللئيل، أن يسمعوا لي، ولا ينزلوا من هذا الجبل المقدس؛ لأنكم في اللحظة التي تتركونه فيها تُحرمون من الحياة والرحمة؛ ولن تُدعون بعد الآن «أبناء الله»، بل «أبناء إبليس».

\par 28 لكنهم لم يصغوا إلى كلماته.

\par 29 وكان أخنوخ في ذلك الوقت قد كبر بالفعل، وفي غيرته على الله، قام وقال: "اسمعوا لي يا بني شيث، الصغار والكبار، عندما تتعدون على وصية آبائنا وتنزلون من هذا الجبل المقدس، فلن تصعدوا إلى هنا أيضًا إلى الأبد".

\par 30 لكنهم ثاروا على أخنوخ، ولم يسمعوا لكلامه، بل نزلوا من الجبل المقدس

\par 31 ولما نظروا إلى بنات قابيل، إلى صورهن الجميلة، وإلى أيديهن وأرجلهن المصبوغة بالألوان، والموشومة بالحلي على وجوههن، اشتعلت فيهن نار الخطيئة

\par 32 ثم جعلهم الشيطان يبدون أجمل ما يكون في عيون بني شيث، كما جعل بني شيث أيضًا يبدون أجمل ما يكون في عيون بنات قابيل، حتى اشتهت بنات قابيل بني شيث كالوحوش المفترسة، وشتهى بنو شيث بنات قابيل، حتى ارتكبوا الرجس معهم

\par 33 ولكن بعد أن سقطوا في هذا الدنس، عادوا من الطريق الذي أتوا منه، وحاولوا صعود الجبل المقدس. لكنهم لم يتمكنوا، لأن حجارة ذلك الجبل المقدس كانت من نار تلمع أمامهم، ولذلك لم يتمكنوا من الصعود مرة أخرى

\par 34 وغضب الله عليهم، وتاب عنهم لأنهم نزلوا من المجد، وبالتالي فقدوا أو تخلوا عن طهارتهم أو براءتهم، وسقطوا في دنس الخطيئة

\par 35 ثم أرسل الله كلمته إلى يارد قائلًا: "هؤلاء أولادك الذين دعوتهم "أولادي"، ها هم قد تعدوا على وصيتي، ونزلوا إلى دار الهلاك والخطيئة. أرسل رسولًا إلى الذين بقوا، لئلا ينزلوا ويهلكوا."

\par 36 ثم بكى يارد أمام الرب، وطلب منه الرحمة والمغفرة. لكنه تمنى أن تفارق روحه جسده، بدلاً من سماع هذه الكلمات من الله عن نزول أبنائه من الجبل المقدس

\par 37 لكنه اتبع أمر الله، ووعظهم ألا ينزلوا من ذلك الجبل المقدس، وألا يخالطوا أبناء قابيل

\par 38 لكنهم لم يصغوا إلى رسالته، ولم يطيعوا مشورته.

\chapter{21}

\par \textit{مات يارد حزنًا على أبنائه الذين ضلوا الطريق. نبوءة بالطوفان.}

\par 1 بعد ذلك، اجتمعت مجموعة أخرى، وذهبوا لرعاية إخوتهم؛ لكنهم هلكوا مثلهم. وهكذا كان الأمر، مجموعة بعد مجموعة، حتى لم يبق منهم إلا عدد قليل

\par 2 ثم مرض يارد من الحزن، وكان مرضه شديدًا حتى اقترب يوم وفاته

\par 3 ثم دعا حنوك ابنه الأكبر، ومتوشالح ابن حنوك، ولامك ابن متوشالح، ونوح ابن لامك

\par 4 ولما جاءوا إليه صلى عليهم وباركهم، وقال لهم: «أنتم أبناء صالحون وأبرياء. لا تنزلوا من هذا الجبل المقدس، لأنه هوذا أبناؤكم وأبناء أبنائكم قد نزلوا من هذا الجبل المقدس، وزاغوا عن هذا الجبل المقدس بشهواتهم الرجسة ومخالفتهم لوصية الله

\par 5 «ولكني أعلم بقوة الله أنه لن يترككم على هذا الجبل المقدس، لأن أولادكم تعدوا وصيته ووصية آبائنا التي أخذناها منهم

\par 6 «لكن يا أبنائي، سيأخذكم الله إلى أرض غريبة، ولن تعودوا أبدًا لتنظروا بأعينكم هذه الجنة وهذا الجبل المقدس

\par 7 «لذلك يا أبنائي، ضعوا قلوبكم على أنفسكم، واحفظوا وصية الله التي معكم. ومتى ذهبتم من هذا الجبل المقدس، إلى أرض غريبة لا تعرفونها، خذوا معكم جسد أبينا آدم، ومعه هذه الهدايا الثلاث الثمينة والقرابين: الذهب والبخور والمر. ولتكن في المكان الذي سيوضع فيه جسد أبينا آدم.»

\par 8 «وإلى من بقي منكم يا أبنائي، يأتي كلمة الله، وعندما يخرج من هذه الأرض يأخذ معه جسد أبينا آدم، ويضعه في وسط الأرض، المكان الذي سيُصنع فيه الخلاص.»

\par 9 فقال له نوح: «من هو الذي يبقى منا؟»

\par 10 فأجاب يارد: «أنت هو الباقي. وأنت تأخذ جسد أبينا آدم من المغارة وتضعه معك في الفلك عندما يأتي الطوفان.

\par 11 «وابنك سام، الذي سيخرج من صلبك، هو الذي سيضع جسد أبينا آدم في وسط الأرض، في المكان الذي سيأتي منه الخلاص.»

\par 12 ثم التفت يارد إلى ابنه حنوك، وقال له: "أنت يا ابني، أمكث في هذا الكهف، واخدم باجتهاد أمام جسد أبينا آدم كل أيام حياتك، وأطعم شعبك بالبر والبراءة."

\par 13 ولم يزد يارد على ذلك. فُرغت يداه، وأُغمِضَت عيناه، ودخل في راحة كآبائه. ووقعت وفاته في السنة الستين والثلاثمائة من عهد نوح، وفي السنة التاسعة والثمانين والتسعمائة من حياته؛ في الثاني عشر من شهر تخسس يوم جمعة

\par 14 ولكن عندما مات يارد، انهمرت الدموع على وجهه من شدة حزنه على أبناء شيث الذين سقطوا في أيامه

\par 15 ثم بكى عليه أخنوخ ومتوشالح ولامك ونوح، هؤلاء الأربعة، وحنطوه بعناية ، ثم وضعوه في مغارة الكنوز. ثم قاموا وندبوه أربعين يومًا

\par 16 ولما انقضت أيام الحداد هذه، بقي حنوك ومتوشالح ولامك ونوح في حزن شديد، لأن أباهم فارقهم ولم يروه بعد ذلك

\chapter{22}

\par \textit{لم يبقَ في العالم سوى ثلاثة رجال صالحين. أحوال البشر الشريرة قبل الطوفان.}

\par 1 لكن حنوك حفظ وصية يارد أبيه، واستمر في الخدمة في الكهف

\par 2 هذا هو أخنوخ الذي حدثت له عجائب كثيرة، والذي كتب أيضًا كتابًا مشهورًا؛ لكن لا يجوز سرد تلك العجائب في هذا المكان

\par 3 ثم بعد ذلك، ضلّ بنو شيث وسقطوا، هم وأبناؤهم ونساؤهم. ولما رآهم أخنوخ ومتوشالح ولامك ونوح، تألمت قلوبهم بسبب سقوطهم في الشك المليء بعدم الإيمان؛ فبكوا وطلبوا من الله الرحمة، أن يحفظهم، ويخرجهم من ذلك الجيل الشرير

\par 4 استمر أخنوخ في خدمته أمام الرب ثلاثمائة وخمسة وثمانين عامًا، وفي نهاية تلك الفترة أدرك بنعمة الله أن الله ينوي إزالته من الأرض

\par 5 ثم قال لابنه: "يا بني، أعلم أن الله يريد أن يجلب مياه الطوفان على الأرض، وأن يهلك خليقتنا

\par 6 «وأنتم آخر الحكام على هذا الشعب في هذا الجبل، لأني أعلم أنه لن يُترك لكم أحدٌ ليولد على هذا الجبل المقدس، ولا يتسلط أحدٌ منكم على أبناء شعبه، ولا يُترك منكم جمعٌ عظيمٌ على هذا الجبل.»

\par 7 وقال لهم أخنوخ أيضًا: "احترسوا من نفوسكم، وتمسكوا بخوف الله وخدمته، واعبدوه بإيمان مستقيم، واخدموه في البر والبراءة والدينونة، وفي التوبة وأيضًا في الطهارة".

\par 8 عندما أنهى أخنوخ وصاياه لهم، نقله الله من ذلك الجبل إلى أرض الحياة، إلى قصور الصالحين والمختارين، دار فردوس الفرح، في نور يصل إلى السماء؛ نور خارج نور هذا العالم؛ لأنه نور الله، الذي يملأ العالم كله، ولكن لا يمكن لأي مكان أن يحتويه

\par 9 وهكذا، ولأن أخنوخ كان في نور الله، وجد نفسه بعيدًا عن متناول الموت؛ حتى أراد الله أن يموته

\par 10 لم يبق على ذلك الجبل المقدس أحد من آبائنا أو من أبنائهم، إلا هؤلاء الثلاثة: متوشالح، ولامك، ونوح. أما الباقون جميعًا فقد نزلوا من الجبل وسقطوا في الخطيئة مع أبناء قابيل. لذلك مُنعوا من ذلك الجبل، ولم يبق عليه أحد إلا هؤلاء الرجال الثلاثة


\end{document}