\begin{document}

\title{عوبديا}


\chapter{1}

\par 1 رُؤْيَا عُوبَدْيَا: هَكَذَا قَالَ السَّيِّدُ الرَّبُّ عَنْ أَدُومَ (سَمِعْنَا خَبَراً مِنْ قِبَلِ الرَّبِّ وَأُرْسِلَ رَسُولٌ بَيْنَ الأُمَمِ: «قُومُوا وَلْنَقُمْ عَلَيْهَا لِلْحَرْبِ»):
\par 2 «إِنِّي قَدْ جَعَلْتُكَ صَغِيراً بَيْنَ الأُمَمِ. أَنْتَ مُحْتَقَرٌ جِدّاً.
\par 3 تَكَبُّرُ قَلْبِكَ قَدْ خَدَعَكَ أَيُّهَا السَّاكِنُ فِي مَحَاجِئِ الصَّخْرِ رِفْعَةَ مَقْعَدِهِ الْقَائِلُ فِي قَلْبِهِ: مَنْ يُحْدِرُنِي إِلَى الأَرْضِ؟»
\par 4 إِنْ كُنْتَ تَرْتَفِعُ كَالنَّسْرِ وَإِنْ كَانَ عُشُّكَ مَوْضُوعاً بَيْنَ النُّجُومِ فَمِنْ هُنَاكَ أُحْدِرُكَ يَقُولُ الرَّبُّ.
\par 5 إِنْ أَتَاكَ سَارِقُونَ أَوْ لُصُوصُ لَيْلٍ. كَيْفَ هَلَكْتَ. أَفَلاَ يَسْرِقُونَ حَاجَتَهُمْ؟ إِنْ أَتَاكَ قَاطِفُونَ أَفَلاَ يُبْقُونَ خُصَاصَةً؟
\par 6 كَيْفَ فُتِّشَ عِيسُو وَفُحِصَتْ مَخَابِئُهُ؟
\par 7 طَرَدَكَ إِلَى التُّخُمِ كُلُّ مُعَاهِدِيكَ. خَدَعَكَ وَغَلَبَ عَلَيْكَ مُسَالِمُوكَ. أَهْلُ خُبْزِكَ وَضَعُوا شَرَكاً تَحْتَكَ. لاَ فَهْمَ فِيهِ.
\par 8 أَلاَ أُبِيدُ فِي ذَلِكَ الْيَوْمِ يَقُولُ الرَّبُّ الْحُكَمَاءَ مِنْ أَدُومَ وَالْفَهْمَ مِنْ جَبَلِ عِيسُو؟
\par 9 فَيَرْتَاعُ أَبْطَالُكَ يَا تَيْمَانُ لِيَنْقَرِضَ كُلُّ وَاحِدٍ مِنْ جَبَلِ عِيسُو بِالْقَتْلِ.
\par 10 مِنْ أَجْلِ ظُلْمِكَ لأَخِيكَ يَعْقُوبَ يَغْشَاكَ الْخِزْيُ وَتَنْقَرِضُ إِلَى الأَبَدِ.
\par 11 يَوْمَ وَقَفْتَ مُقَابِلَهُ يَوْمَ سَبَتِ الأَعَاجِمُ قُدْرَتَهُ وَدَخَلَتِ الْغُرَبَاءُ أَبْوَابَهُ وَأَلْقُوا قُرْعَةً عَلَى أُورُشَلِيمَ كُنْتَ أَنْتَ أَيْضاً كَوَاحِدٍ مِنْهُمْ.
\par 12 وَيَجِبُ أَنْ لاَ تَنْظُرَ إِلَى يَوْمِ أَخِيكَ يَوْمَ مُصِيبَتِهِ وَلاَ تَشْمَتَ بِبَنِي يَهُوذَا يَوْمَ هَلاَكِهِمْ وَلاَ تَفْغَرَ فَمَكَ يَوْمَ الضِّيقِ
\par 13 وَلاَ تَدْخُلَ بَابَ شَعْبِي يَوْمَ بَلِيَّتِهِمْ وَلاَ تَنْظُرَ أَنْتَ أَيْضاً إِلَى مُصِيبَتِهِ يَوْمَ بَلِيَّتِهِ وَلاَ تَمُدَّ يَداً إِلَى قُدْرَتِهِ يَوْمَ بَلِيَّتِهِ
\par 14 وَلاَ تَقِفَ عَلَى الْمَفْرَقِ لِتَقْطَعَ مُنْفَلِتِيهِ وَلاَ تُسَلِّمَ بَقَايَاهُ يَوْمَ الضِّيقِ.
\par 15 فَإِنَّهُ قَرِيبٌ يَوْمُ الرَّبِّ عَلَى كُلِّ الأُمَمِ. كَمَا فَعَلْتَ يُفْعَلُ بِكَ. عَمَلُكَ يَرْتَدُّ عَلَى رَأْسِكَ.
\par 16 لأَنَّهُ كَمَا شَرِبْتُمْ عَلَى جَبَلِ قُدْسِي يَشْرَبُ جَمِيعُ الأُمَمِ دَائِماً يَشْرَبُونَ وَيَجْرَعُونَ وَيَكُونُونَ كَأَنَّهُمْ لَمْ يَكُونُوا.
\par 17 وَأَمَّا جَبَلُ صِهْيَوْنَ فَتَكُونُ عَلَيْهِ نَجَاةٌ وَيَكُونُ مُقَدَّساً وَيَرِثُ بَيْتُ يَعْقُوبَ مَوَارِيثَهُمْ.
\par 18 وَيَكُونُ بَيْتُ يَعْقُوبَ نَاراً وَبَيْتُ يُوسُفَ لَهِيباً وَبَيْتُ عِيسُو قَشّاً فَيُشْعِلُونَهُمْ وَيَأْكُلُونَهُمْ وَلاَ يَكُونُ بَاقٍ مِنْ بَيْتِ عِيسُو لأَنَّ الرَّبَّ تَكَلَّمَ.
\par 19 وَيَرِثُ أَهْلُ الْجَنُوبِ جَبَلَ عِيسُو وَأَهْلُ السَّهْلِ الْفِلِسْطِينِيِّينَ وَيَرِثُونَ بِلاَدَ أَفْرَايِمَ وَبِلاَدَ السَّامِرَةِ وَيَرِثُ بِنْيَامِينُ جِلْعَادَ.
\par 20 وَسَبْيُ هَذَا الْجَيْشِ مِنْ بَنِي إِسْرَائِيلَ يَرِثُونَ الَّذِينَ هُمْ مِنَ الْكَنْعَانِيِّينَ إِلَى صَرْفَةَ. وَسَبْيُ أُورُشَلِيمَ الَّذِينَ فِي صَفَارِدَ يَرِثُونَ مُدُنَ الْجَنُوبِ.
\par 21 وَيَصْعَدُ مُخَلِّصُونَ عَلَى جَبَلِ صِهْيَوْنَ لِيَدِينُوا جَبَلَ عِيسُو وَيَكُونُ الْمُلْكُ لِلرَّبِّ.


\end{document}