\begin{document}

\title{3 باروخ}

\chapter{1}

\par \textit{مقدمة}

\par 1 رواية وكشف لباروخ، بشأن تلك الأشياء التي لا تُوصف والتي رآها بأمر الله. باركك يا رب

\par 2 وحي من باروخ، الذي وقف على نهر جل يبكي على سبي

\par 3 أورشليم، حين حُفظ أبيمالك أيضًا بيد الله، في مزرعة أغريبا. وكان جالسًا هكذا عند الأبواب الجميلة، حيث قدس الأقداس

\chapter{1b}

\par \textit{مقدمة}

\par 1 حَقًّا أَنَا بَارُوخُ كُنْتُ أَبْكِي فِي قَلْبِي وَأَتَأَلَّمُ عَلَى الْقَوْمِ، وَذَلِكَ

\par 2 لقد سمح الله للملك نبوخذ نصر بتدمير مدينته قائلاً: يا رب، لماذا أشعلت النار في كرمك وجعلته خرابا؟ لماذا فعلت هذا؟ ولماذا يا رب لم تجازنا بتأديب آخر، بل سلمتنا إلى أمم مثل هؤلاء، لكي لا يسلمونا إلى عقابك؟

\par 3 "وإذا أنا أبكي وأقول هذا، فرأيت ملاك الرب مقبلاً وقال لي: افهم أيها الإنسان المحبوب، ولا تتعب نفسك كثيراً بشأن خلاص أورشليم، لأنه هكذا قال السيد الرب:

\par 4 القدير. لأنه أرسلني أمامك لأعرفك وأُريك كل (الأشياء)

\par 5 [...]

\par 6 من الله. لأن صلاتك سُمعت أمامه، ودخلت مسامع الرب الإله. ولما قال لي هذه الأشياء، صمت. فقال لي الملاك: كف عن الاستفزاز

\par 7 يا الله، سأريك أسرارًا أخرى أعظم من هذه. فقلت أنا باروخ: حي هو الرب الإله، إن أريتني وسمعت كلامك، فلن أتكلم بعد

\par 8 سيزيد الله من دينونتي يوم القيامة، إذا تكلمت فيما بعد. وقال لي ملاك القوات: تعال، وسأريك أسرار الله

\chapter{2}

\par \textit{السماء الأولى.}

\par 1 وأخذني وقادني إلى حيث ثُبتت السماء، وحيث كان هناك نهر لا يستطيع أحد عبوره، ولا أي نسمة غريبة من كل ما خلقه الله. وأخذني وقادني إلى السماء الأولى، وأراني بابًا كبيرًا. وقال لي: لندخل

\par 2 [...]

\par 3 عبرناها، ودخلنا كأننا محمولون على أجنحة، مسافة نحو ثلاثين يومًا. وأراني في السماء سهلًا، وكان فيه أناس يسكنون، وجوههم كالكلاب.

\par 4 ثيران، وقرون أيائل، وأرجل معزى، وأفخاذ حملان. فسأل باروخ الملاك: "أخبرني، أرجوك، ما هو سمك السماء التي سافرنا فيها؟"

\par 5 أو ما هو مداه، أو ما هو السهل، حتى أتمكن من إخبار بني البشر أيضًا. وقال لي الملاك الذي اسمه فمعئيل: هذا الباب الذي تراه هو باب السماء، وكما أن المسافة من الأرض إلى السماء عظيمة كذلك سمكه؛ وأيضًا كما أن المسافة (من الشمال إلى الجنوب، عظيمة جدًا) هي طول السهل الذي رأيته. وقال لي ملاك القوات أيضًا: تعال، وسأريك أسرارًا أعظم. لكن

\par 6 [...]

\par 7  قلتُ: أرني من هم هؤلاء الرجال. فقال لي: هؤلاء هم الذين بنوا برج الخصام على الله، فنفاهم الرب

\chapter{3}

\par \textit{السماء الثانية.}

\par 1 وأخذني ملاك الرب وقادني إلى سماء ثانية. وأراني هناك

\par 2 أيضًا بابًا مثل الأول وقال: لندخل منه. فدخلنا محمولين على أجنحة

\par 3 مسافة رحلة حوالي ستين يومًا. وأراني هناك أيضًا سهلًا، وكان مليئًا بـ

\par 4 رجال، مظهرهم كشكل الكلاب، وأقدامهم كأقدام الغزلان. وسألت

\par 5 الملاك: أتوسل إليك يا رب، قل لي من هؤلاء. فقال: هؤلاء هم الذين أشاروا ببناء البرج، لأن الذين تراهم طردوا جموعًا من الرجال والنساء لصنع الطوب؛ ومن بينهم امرأة تصنع الطوب لم يُسمح لها بالخروج في ساعة الولادة، بل وُلدت وهي تصنع الطوب، وحملت طفلها في مئزرها، و

\par 6 استمروا في صنع الطوب. وظهر لهم الرب وأبلبل كلامهم، عندما

\par 7 بنى برجًا بارتفاع أربعمائة وثلاثة وستين ذراعًا. فأخذوا مِثقابًا، وحاولوا أن يثقبوا السماء، قائلين: لنرَ السماء من طين أم من طين؟

\par 8 من نحاس، أو من حديد. فلما رأى الله ذلك لم يسمح لهم، بل ضربهم بالعمى واضطراب النطق، وجعلهم كما ترى

\chapter{4}

\par \textit{السماء الثالثة.}

\par 1 فقلت أنا باروخ: هوذا يا رب، قد أريتني أمورًا عظيمة وعجائب، والآن

\par 2 أرني كل شيء من أجل الرب. فقال لي الملاك: تعالَ ننطلق. (وسرتُ) مع الملاك من ذلك المكان نحو مائة وخمسة وثمانين يومًا.

\par 3 وأراني سهلاً وثعبانًا، بدا أن طولهما مائتي ثعبان.

\par 4 وأراني الجحيم، وكان منظره مظلمًا وبغيضًا. فقلت:

\par 5 من هو هذا التنين ومن هو هذا الوحش الذي حوله فقال الملاك التنين هو

\par 6 الذي يأكل أجساد الذين يقضون حياتهم في الشر، ويتغذى منها. وهذا هو الجحيم، الذي يشبهه أيضًا عن كثب، إذ يشرب أيضًا نحو ذراع من

\par 7 البحر الذي لا يغرق إطلاقًا. قال باروخ: وكيف (يحدث هذا)؟ فقال الملاك: اسمع، صنع الرب الإله ثلاثمائة وستين نهرًا، منها نهران رئيسيان.

\par 8 جميعهم ألفياس وأبيريس وجيريكوس؛ وبسببهم لا يغرق البحر. فقلتُ: أرجوك أن تُريني أي شجرة أضلّت آدم. فقال لي الملاك: إنها الكرمة التي غرسها الملاك صموئيل، فغضب الرب الإله ولعنه هو وغرسه، ولهذا السبب أيضًا لم يسمح لآدم بلمسها، ولذلك...

\par 9 حسد الشيطان فخدعه من خلال كرمته. [وقلت أنا باروخ: بما أن الكرمة أيضًا كانت سببًا لمثل هذا الشر العظيم، وهي تحت دينونة لعنة الله، وكانت

\par 10 تدمير المخلوق الأول، كيف يكون مفيدًا الآن؟ فقال الملاك: لقد سألتَ بشكل صحيح. عندما تسبب الله في الطوفان على الأرض، وأهلك كل ذي جسد، وأربعمائة وتسعة آلاف عملاق، وارتفع الماء خمسة عشر ذراعًا فوق أعلى الجبال، ثم دخل الماء إلى الجنة ودمر كل زهرة؛ لكنه أزال تمامًا وبلا حدود البرعم

\par 11 من الكرمة وألقاها خارجًا. ولما ظهرت الأرض من الماء، وخرج نوح

\par 12 فأخذ الفلك، فبدأ يغرس من النباتات التي وجدها. فوجد أيضًا غصن الكرمة، فأخذه وفكر في نفسه: ما هو إذن؟ فجئت وكلمته

\par 13 له الأمور المتعلقة بها. فقال: أغرسها أم ماذا أفعل؟ بما أن آدم هلك بسببها، فلا أُلاقي أنا أيضًا غضب الله بسببها. وقوله

\par 14 هذه الأمور، صلى أن يكشف الله له ما يجب أن يفعله بشأنها. وبعد أن أكمل الصلاة التي استمرت أربعين يومًا، وبعد أن توسل كثيرًا وبكى،

\par 15 قال: يا رب، أتوسل إليك أن تكشف لي ما سأفعله بشأن هذه النبتة. لكن الله أرسل ملاكه سرسائيل، وقال له: قم يا نوح، واغرس غصن الكرمة، لأنه هكذا قال الرب: ستتحول مرارتها إلى حلاوة، وستصبح لعنتها نعمة، وسيصبح ما ينتج منها دم الله؛ وكما نال الجنس البشري من خلالها الدينونة، كذلك من خلال يسوع المسيح عمانوئيل سينالون فيه

\par 16 الدعوة إلى العلاء، ودخول الفردوس]. فاعلم يا باروخ أنه كما نال آدم من خلال هذه الشجرة ذاتها الإدانة، وجُرِّد من مجد الله، كذلك فإن الرجال الذين يشربون الآن بلا نهم الخمر الناتج منها، يرتكبون مخالفات أسوأ من آدم، وهم بعيدون عن

\par 17 مجد الله، ويسلمون أنفسهم للنار الأبدية. لأنه لا خير يأتي من خلالها. فإن الذين يشربونها بإفراط يفعلون هذه الأمور: لا يشفق الأخ على أخيه، ولا الأب على ابنه، ولا الأبناء على والديهم، بل من شرب الخمر تأتي كل الشرور، مثل القتل، والزنا، والفسق، وشهادة الزور، والسرقة، وما شابه ذلك. ولا يقوم بها شيء صالح

\chapter{5}

\par 1 فقلت أنا باروخ للملاك:

\par 2 دعني أسألك شيئًا واحدًا يا رب. بما أنك قلت لي

\par 3 إن التنين يشرب ذراعًا واحدًا من البحر، فقل لي أيضًا، كم هو عظيم بطنه؟ فقال الملاك: بطنه هو الجحيم، وبقدر ما يُرمى شاقولة من قبل ثلاثمائة رجل، يكون بطنه كذلك عظيمًا. تعالَ إذن لأريكَ أيضًا أعمالًا أعظم من هذه

\chapter{6}

\par 1 وأخذني وقادني إلى حيث تشرق الشمس؛

\par 2 وأراني عربة وأربع عربات، تحتها نار مشتعلة، وفي العربة كان يجلس رجل يرتدي تاجًا من نار، وكانت العربة يجرها أربعون ملائكة. وإذا بطائر يحلق أمام الشمس، حوالي تسع

\par 3 ذراعًا. فقلت للملاك: ما هذا الطائر؟ فقال لي: هذا هو

\par 4 [...]

\par 5 حارس الأرض. فقلت: يا رب، كيف يكون حارس الأرض؟ علمني. فقال لي الملاك: هذا الطائر يطير بجانب الشمس، ويفتح جناحيه ويستقبل ضوءها الناري

\par 6 أشعة. لأنه لو لم يكن يستقبلها، لما حُفظ الجنس البشري، ولا أي كائن آخر

\par 7 مخلوق حي. لكن الله سخر له هذا الطائر. فبسط جناحيه، فرأيت على جناحه الأيمن حروفًا كبيرة جدًا، بحجم مساحة البيدر، حجمها نحو أربعة

\par 8 ألف مودي؛ وكانت الحروف من ذهب. وقال لي الملاك: اقرأها. وقرأت

\par 9 وركضوا هكذا: لا الأرض ولا السماء تخرجني، بل أجنحة من نار تخرجني. فقلت: يا رب، ما هذا الطائر، وما اسمه؟ فقال لي الملاك: اسمه يُدعى

\par 10 [...]

\par 11 فينيكس. (وقلت): وماذا يأكل؟ فقال لي: من السماء و

\par 12 ندى الأرض. فقلت: هل يُخرج الطائر برازًا؟ فقال لي: إنه يُخرج دودة، وبراز الدودة هو القرفة التي يستخدمها الملوك والأمراء. لكن انتظر وسترى

\par 13 انظر إلى مجد الله. وبينما كان يُحادثني، حدث دويٌّ كصوت الرعد، واهتز المكان الذي كنا واقفين عليه. فسألتُ الملاك: يا سيدي، ما هذا الصوت؟ فقال لي الملاك: حتى الآن، تفتح الملائكة البوابات الثلاثمائة والخمسة والستين

\par 14 من السماء، والنور ينفصل عن الظلمة. وجاء صوت يقول: نور

\par 15 يا معطي، امنح العالم إشعاعًا. ولما سمعت صوت الطائر، قلت: يا رب، ما هذا؟

\par 16 ضجيج وقال: هذا هو الطائر الذي يوقظ الديوك على الأرض من سباتها. فكما يفعل البشر من خلال الفم، كذلك يدل الديك لأولئك الذين في العالم، في كلامه. لأن الملائكة تُهيئ الشمس، والديك يصيح

\chapter{7}

\par 1 وقلتُ: ومن أين تبدأ الشمس أعمالها بعد صياح الديك؟

\par 2 فقال لي الملاك: اسمع يا باروخ: كل ما أريتك إياه موجود في السماء الأولى والثانية، وفي السماء الثالثة تمر الشمس وتنير العالم. لكن انتظر، وسترى ما هو مكتوب.

\par 3 سأرى مجد الله. وبينما كنت أتحدث معه، رأيت الطائر، وظهر لي

\par 4 في المقدمة، ونما حجمه أكثر فأكثر، ثم عاد أخيرًا إلى حجمه الكامل. وخلفه رأيت الشمس الساطعة، والملائكة الذين يرسمونها، وتاجًا على خرزتها، كنا نراه

\par 5 غير قادر على النظر إليه والتأمل. وبمجرد أن أشرقت الشمس، بسط طائر الفينيق جناحيه أيضًا. أما أنا، فلما رأيت مثل هذا المجد العظيم، فقد خارت قواي خوفًا عظيمًا، وهربت و

\par 6 اختبأوا في أجنحة الملاك. فقال لي الملاك: لا تخف يا باروخ، بل انتظر وسترى أيضًا غروبهم

\chapter{8}

\par 1 وأخذني وقادني نحو الغرب، وعندما حان وقت الغروب، رأيت الطائر مرة أخرى قادمًا أمامه، وبمجرد أن جاء رأيت الملائكة وقد رفعوا التاج

\par 2 [...]

\par 3 من رأسه. أما الطائر فوقف منهكًا وجناحيه منقبضين. ونظرت إلى هذه الأشياء، فقلت: يا رب، لماذا رفعوا التاج عن رأس الشمس، ولماذا

\par 4 قال لي الملاك: «إكليل الشمس، عندما يجتاز النهار، يأخذه أربعة ملائكة، ويرفعونه إلى السماء، ويجددونه، لأنه هو وأشعته قد تدنس على الأرض؛ علاوة على ذلك، فهو يتجدد كل يوم». فقلت أنا باروخ: «يا رب، ولماذا؟»

\par 5 قال لي الملاك: "لأنها تنظر إلى إثم الناس وظلمهم، أي الزنا والسرقة والاغتصاب وعبادة الأصنام والسكر والقتل والخصام والحسد والتجديف والتذمر والهمس والعرافة وما شابه ذلك مما لا يرضي الله. فبسبب هذه الأمور تنجس، ولذلك تتجدد".

\par 6 لكنك تسأل عن الطائر، كيف يُنهك؟ لأنه يُنهك بحجب أشعة الشمس عن النار وحرارة النهار. وكما ذكرنا سابقًا، لولا أن أجنحته حجبت أشعة الشمس، لما نجا كائن حي.

\chapter{9}

\par 1 ولما انصرفوا، حل الليل أيضًا، وفي الوقت نفسه جاءت مركبة القمر مع النجوم

\par 2 فقلت أنا باروخ: يا رب أرني أيضا كيف

\par 3 يخرج، وأين يرحل، وبأي هيئة يسلك. فقال الملاك: انتظر، فستراه أيضًا قريبًا. وفي الغد رأيته أيضًا في صورة امرأة جالسة على عربة ذات عجلات. وكان أمامه ثيران وحملان في العربة، وحشد من

\par 4 الملائكة كذلك. فقلت: يا رب، ما البقر والحملان؟ فقال لي:

\par 5 هم أيضًا ملائكة. وسألت مرة أخرى: لماذا يزداد هذا أحيانًا، ولكنه يختفي أحيانًا أخرى؟

\par 6 ينقص الوقت وقال لي اسمع يا باروخ هذا الذي تراه كان مكتوبا

\par 7 والله جميلٌ لا مثيل له. وعند معصية آدم الأول، اقترب صموئيل حين اتخذ الحية ثوبًا. فلم تختفِ بل ازدادت، وكان الله

\par 8 غاضبًا عليه، وأحزنه، وقصّر أيامه. فقلت: وكيف لا يضيء دائمًا، بل في الليل فقط؟ فقال الملاك: اسمعوا: كما لا يستطيع رجال الحاشية التحدث بحرية في حضرة الملك، كذلك لا يستطيع القمر والنجوم أن يتألقوا في حضرة الشمس؛ لأن النجوم معلقة دائمًا، لكنها محجوبة بالشمس، والقمر، على الرغم من أنه غير مصاب، يحترق بحرارة الشمس

\chapter{10}

\par \textit{السماء الرابعة.}

\par 1 وبعد أن تعلمت كل هذه الأشياء من رئيس الملائكة، أخذني وقادني إلى السماء الرابعة

\par 2 [...]

\par 3 السماء. ورأيت سهلًا رتيبًا، وفي وسطه بركة ماء. وكان فيه جموع من الطيور من كل نوع، ولكن ليس مثل تلك الموجودة هنا على الأرض. لكنني رأيت كركيًا كبيرًا مثل

\par 4 ثيران ضخمة؛ وكانت جميع الطيور أكبر من تلك الموجودة في العالم. وسألت الملاك: ماذا

\par 5 هو السهل، وما هي البركة، وما هي كثرة الطيور حولها. وقال الملاك: اسمع يا باروخ: السهل الذي يحتوي على البركة وعجائب أخرى هو المكان الذي فيه

\par 6 تأتي أرواح الصالحين، عندما يتحدثون، يعيشون معًا في جوقات. لكن الماء

\par 7 ما تستقبله السحب، فتمطر على الأرض، وتكثر الثمار. فقلتُ مرة أخرى لملاك الرب: ولكن (ما) هذه الطيور؟ فقال لي: هي التي

\par 8 سبحوا الرب باستمرار. فقلت: يا رب، كيف يقول الناس إن الماء الذي

\par 9 الماء الذي ينزل في المطر هو من البحر. وقال الملاك: الماء الذي ينزل في المطر هو أيضًا من البحر، ومن مياه الأرض؛ لكن ما يُنبئ بالثمار هو (فقط) من

\par 10 المصدر الأخير. فاعلم من الآن فصاعدًا أن من هذا المصدر ما يُسمى ندى السماء

\chapter{11}

\par \textit{السماء الخامسة.}

\par 1 فأخذني الملاك وقادني من هناك إلى سماء خامسة. وكانت البوابة مغلقة. فقلت: يا رب، أليس هذا الباب مفتوحًا لندخل؟ فقال لي الملاك: لا يمكننا الدخول حتى يأتي ميخائيل، الذي يحمل مفاتيح ملكوت السماوات؛ ولكن انتظر وسترى

\par 2 [...]

\par 3 مجد الله. وكان صوت عظيم كالرعد. فقلت: يا رب، ما هذا الصوت؟

\par 4 فقال لي الآن ينزل ميخائيل رئيس الملائكة ليأخذ

\par 5 صلوات الناس. وإذا بصوتٍ يقول: "لتُفتح الأبواب". ففتحوها،

\par 6 كان هناك هدير كما لو كان رعدًا. وجاء ميخائيل، والملاك الذي كان معي وقف أمامي.

\par 7 واجهه وقال: سلام يا قائدي وقائد كل رتبتنا. فقال القائد ميخائيل: سلام أنت أيضًا يا أخانا وترجمان الوحي للذين يمرون في الحياة.

\par 8 بفضيلة. وبعد أن سلموا على بعضهم البعض هكذا، وقفوا. ورأيت القائد ميخائيل يحمل إناءً عظيمًا جدًا، كان عمقه كبعد السماء إلى

\par 9 الأرض، وعرضها كالمسافة من الشمال إلى الجنوب. فقلت: يا رب، ما هذا الذي يحمله ميخائيل رئيس الملائكة؟ فقال لي: هذا هو المكان الذي تدخل فيه فضائل الصالحين، وأعمالهم الصالحة التي يعملونها، والتي تُرفع أمام الله السماوي

\chapter{12}

\par 1 وبينما كنت أتحدث معهم، إذا بالملائكة قد أتوا يحملون سلالاً مليئة بالزهور. و

\par 2 أعطوها لميخائيل. وسألت الملاك: يا رب، من هؤلاء، وما هي الأشياء

\par 3 أُحضر من بينهم وقال لي: هؤلاء ملائكة على الأرض

\par 4 [...]

\par 5 البار. فأخذ رئيس الملائكة السلال وألقاها في الإناء. والملاك

\par 6 قال لي: هذه الزهور من فضائل الصالحين. ورأيت ملائكة آخرين يحملون سلالاً لم تكن فارغة ولا ممتلئة. فبدأوا ينوحون، ولم يجرؤوا على الاقتراب،

\par 7 لأنه لم تكن لديهم الجوائز كاملة. فنادى ميخائيل وقال: تعالوا أنتم أيضًا إلى هنا

\par 8 أيها الملائكة، أحضروا ما أحضرتموه. وحزن ميخائيل والملاك الذي كان معي حزنًا شديدًا، لأنهما لم يملأا الإناء

\chapter{13}

\par 1 ثم جاء ملائكة آخرون على نفس المنوال يبكون ويندبون، ويقولون بخوف: انظر كيف أننا مُثقلون يا رب، لأننا سُلِّمنا إلى أناس أشرار، ونريد أن نبتعد عن

\par 2 فقال ميخائيل لا تستطيعون أن تفارقوهم لئلا يغلب العدو عليكم.

\par 3 النهاية، لكن أخبروني بما تطلبون. فقالوا: نسألك يا ميخائيل قائدنا أن تنقلنا عنهم، لأننا لا نستطيع أن نقيم مع أناس أشرار وسُفهاء، لأنه لا خير فينا.

\par 4 فيهم، بل كل أنواع الإثم والجشع. لأننا لا نراهم يدخلون الكنيسة إطلاقًا، ولا بين الآباء الروحيين، ولا في أي عمل صالح. ولكن حيث يوجد القتل، فهم أيضًا في الوسط، وحيث يوجد الزنا، والزنى، والسرقة، والقذف، وشهادة الزور، والغيرة، والسكر، والخصام، والحسد، والتذمر، والهمس، وعبادة الأصنام، والعرافة، وما شابه ذلك،

\par 5 فهم يعملون أعمالاً كهذه، وأعمالاً أخرى أسوأ. لذلك نطلب أن نتركهم. فقال ميخائيل للملائكة: انتظروا حتى أعلم من الرب ما سيحدث.

\chapter{14}

\par 1 وفي تلك الساعة عينها، انصرف ميخائيل، وأُغلقت الأبواب. وحدث صوتٌ كـ

\par 2 فسألت الملاك: ما هذا الصوت؟ فقال لي: إن ميخائيل الآن يعرض محاسن البشر على الله.

\chapter{15}

\par 1 وفي تلك الساعة نزل ميخائيل، فانفتح الباب، وأحضر زيتًا

\par 2 وأما الملائكة الذين قدموا السلال الممتلئة فملأها زيتا وقال: ارفعوها وأعطوا مائة ضعف لأصدقائنا وللذين تعبوا في أعمال صالحة.

\par 3 فمن زرع بفضيلة، حصد بفضيلة أيضًا. وقال أيضًا للذين أحضروا السلال نصف الفارغة: تعالوا أنتم أيضًا، خذوا الأجرة كما أحضرتم،

\par 4 سلموها لبني البشر. [ثم قال أيضًا للذين أحضروا السلال الممتلئة وللذين أحضروا السلال نصف الفارغة: اذهبوا وباركوا أصدقاءنا، وقولوا للذين قال الرب: أنتم أمناء على القليل، فأقيمكم على الكثير. ادخلوا إلى فرح سيدكم.]

\chapter{16}

\par 1 ثم التفت وقال أيضًا للذين لم يقدموا شيئًا: هكذا قال الرب: لا تحزنوا

\par 2 لا تبكوا، ولا تدعوا أبناء البشر وشأنهم. ولكن بما أنهم أغضبوني بأعمالهم، فاذهب وأثير حسودهم وغضبهم وغيظهم على شعب ليس بشعب، شعب

\par 3 قوم ليس لديهم فهم. علاوة على ذلك، بالإضافة إلى هؤلاء، أرسل اليرقة والجراد غير المجنح، والعفن، والجراد العادي (و) بردًا مع البرق والغضب، و

\par 4 عاقبهم عقابًا شديدًا بالسيف والموت، وأولادهم بالشياطين. لأنهم لم يسمعوا لصوتي، ولم يحفظوا وصاياي، ولم يعملوا بها، بل استهزأوا بوصاياي، وتكبروا على الكهنة الذين بشروهم بكلامي.

\chapter{17}

\par 1 وبينما هو يتكلم، أُغلق الباب، فانسحبنا.

\par 2 وأخذني الملاك و

\par 3 أعادني إلى حيث كنت في البداية. ولما عدتُ إلى نفسي، مجدتُ

\par 4 لله الذي حسبني أهلاً لهذا التكريم. فلماذا أنتم أيضاً، أيها الإخوة الذين نلتم هذا الوحي، تمجّدون الله أنفسكم أيضاً، لكي يمجّدكم هو أيضاً، الآن وكل أوان وإلى دهر الدهور. آمين.

\end{document}