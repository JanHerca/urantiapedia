\begin{document}

\title{4 المكابيين}

\chapter{1}

\par \textit{ملخص للفلسفة من العصور القديمة فيما يتعلق بالعقل الملهم. لم تصل الحضارة قط إلى فكر أعلى. مناقشة "القمع". تلخص الآية 48 فلسفة البشرية بأكملها.}

\par 1 السؤال الذي أقترح مناقشته فلسفي إلى أقصى حد، ألا وهو ما إذا كان العقل المُلهم هو الحاكم الأعلى على الأهواء؛ وأود أن أطلب منكم بجدية اهتمامًا جادًا بفلسفته

\par 2 فليس الموضوع ضروريًا بشكل عام كفرع من فروع المعرفة فحسب، بل إنه يشمل أيضًا مدح أعظم الفضائل، وأعني بذلك ضبط النفس

\par 3 أي أنه إذا ثبت أن العقل يتحكم في الأهواء المعادية للاعتدال، والشراهة، والشهوة، فقد ثبت أيضًا بوضوح أنه يسود على الأهواء، مثل الحقد، المعارضة للعدالة، وعلى تلك المعارضة للرجولة، وهي الغضب والألم والخوف

\par 4 ولكن قد يتساءل البعض، إذا كان العقل سيد الأهواء، فلماذا لا يتحكم في النسيان والجهل؟ هدفهما هو السخرية

\par 5 الجواب هو أن العقل ليس سيدًا على العيوب الكامنة في العقل نفسه، بل على الأهواء أو العيوب الأخلاقية التي تتعارض مع العدالة والرجولة والاعتدال والحكمة؛ وعمله في هذه الحالة ليس استئصال الأهواء، بل تمكيننا من مقاومتها بنجاح

\par 6 يمكنني أن أعرض عليكم العديد من الأمثلة، المأخوذة من مصادر مختلفة، حيث أثبت العقل أنه سيد الأهواء، لكن أفضل مثال يمكنني تقديمه على الإطلاق هو السلوك النبيل لأولئك الذين ماتوا من أجل الفضيلة، إليعازار، والإخوة السبعة والأم

\par 7 فهؤلاء جميعًا، من خلال ازدرائهم للآلام، بل حتى الموت، أثبتوا أن العقل يتفوق على الأهواء

\par 8 قد أُوسّع هنا مدحًا لفضائلهم، هم، الرجال ذوو الأم، الذين يموتون في هذا اليوم الذي نحتفل به من أجل حب الجمال الأخلاقي والخير، لكنني أود أن أهنئهم على التكريم الذي نالوه

\par 9 إن الإعجاب بشجاعتهم وقدرتهم على التحمل، ليس فقط من قبل العالم أجمع، بل من قبل جلاديهم أنفسهم، جعلهم سبب سقوط الطغيان الذي كانت ترزح تحته أمتنا، إذ هزموا الطاغية بصبرهم، فتم تطهير بلادهم من خلالهم

\par 10 ولكنني سأغتنم الفرصة الآن لمناقشة هذا الأمر، بعد أن نبدأ بالنظرية العامة، كما اعتدت أن أفعل، ثم سأنتقل إلى قصتهم، مُمجِّدًا الله الحكيم

\par 11 سؤالنا إذن هو ما إذا كان العقل هو السيد الأسمى على الأهواء

\par 12 ولكن يجب علينا أن نحدد بدقة ما هو العقل وما هي العاطفة، وكم عدد أشكال العاطفة الموجودة، وما إذا كان العقل أسمى منها جميعًا

\par 13 أعتقد أن السبب هو أن العقل يفضل بتأنٍّ واضح حياة الحكمة

\par 14 الحكمة هي معرفة الأشياء الإلهية والبشرية وأسبابها.

\par 15 أعتبر هذا هو الثقافة المكتسبة بموجب الناموس، والتي من خلالها نتعلم باحترام ما يخص الله، ولمنفعتنا الدنيوية ما يخص الإنسان

\par 16 تتجلى الحكمة الآن في أشكال الحكمة والعدل والشجاعة والاعتدال

\par 17 لكن الحكم أو ضبط النفس هو الذي يُسيطر عليهم جميعًا، لأنه من خلاله، في الحقيقة، يُؤكد العقل سلطته على الأهواء

\par 18 ولكن للأهواء مصدران شاملان، هما اللذة والألم، وكلاهما ينتمي أساسًا إلى الروح وكذلك إلى الجسد

\par 19 وفيما يتعلق بكل من اللذة والألم، هناك العديد من الحالات التي يكون فيها للعواطف تسلسلات معينة

\par 20 وهكذا، بينما تسبق الرغبة المتعة، يتبعها الرضا، وبينما يسبق الخوف الألم، يأتي الحزن بعد الألم

\par 21 الغضب، مرة أخرى، إذا أراد الرجل أن يتتبع مسار مشاعره، هو شغف يمتزج فيه كل من المتعة والألم

\par 22 تحت المتعة، أيضًا، يأتي ذلك الانحطاط الأخلاقي الذي يُظهر أوسع تنوع في المشاعر

\par 23 يتجلى في النفس كالتباهي، والطمع، والمجد الباطل، والشجار، والغيبة، وفي الجسد كأكل لحوم غريبة، والشراهة، والتهام الطعام في الخفاء

\par 24 الآن، بما أن اللذة والألم هما شجرتان، تنموان من الجسد والروح، فإن العديد من فروع هذه المشاعر تنبت؛ وعقل كل إنسان كبستاني ماهر، يزيل الأعشاب الضارة ويقلمها ويربطها، ويدير الماء ويوجهه هنا وهناك، يُخضع غابة التصرفات والمشاعر للتدجين

\par 25 فبينما العقل هو دليل الفضائل، فهو سيد الأهواء

\par 26 لاحظ الآن، أولًا، أن العقل يصبح أسمى من الأهواء بفضل الفعل المثبط للاعتدال

\par 27 أعتقد أن الاعتدال هو قمع الرغبات؛ ولكن من بين الرغبات، بعضها عقلي وبعضها جسدي، وكلا النوعين يتحكم فيهما العقل بوضوح؛ عندما نميل إلى تناول اللحوم المحرمة، كيف نتخلى عن الملذات المستمدة منها؟

\par 28 أليس للعقل القدرة على قمع الشهوات؟ في رأيي، إنه كذلك

\par 29 وبناءً على ذلك، عندما نشعر برغبة في أكل حيوانات الماء والطيور والوحوش واللحوم من كل نوع محرم علينا بموجب الشريعة، فإننا نمتنع بحكم غلبة العقل

\par 30 لأن نزعات شهواتنا تُكبح وتُثبط بالعقل المعتدل، وجميع حركات الجسد تخضع لقيود العقل

\par 31 وما الذي يدعو للدهشة إذا أُخمدت الرغبة الطبيعية للروح في الاستمتاع بثمار الجمال؟

\par 32 "ولهذا السبب نمدح يوسف الفاضل، لأنه بعقله وبجهد ذهني، كبح الدافع الجسدي. 1 لأنه، وهو شاب في سن تكون فيه الرغبة الجسدية قوية، أطفأ بعقله دافع أهوائه.

\par 33 وقد ثبت أن العقل يُخضع ليس فقط دافع الرغبة الجنسية، بل جميع أنواع الطمع

\par 34 لأن الناموس يقول: «لا تشتهِ امرأة قريبك، ولا شيئًا مما لقريبك».

\par 35 حقًا، عندما يأمرنا القانون بعدم الطمع، فإنه، في رأيي، ينبغي أن يؤكد بقوة الحجة القائلة بأن العقل قادر على التحكم في الرغبات الجشعة، تمامًا كما يفعل مع الأهواء التي تعارض العدالة

\par 36 وإلا فكيف يمكن تعليم رجل، بطبيعته شره وجشع وسكير، أن يغير طبيعته، إذا لم يكن العقل هو سيد الأهواء بشكل واضح؟

\par 37 بالتأكيد، بمجرد أن ينظم الإنسان حياته وفقًا للشريعة، إذا كان بخيلًا، فإنه يتصرف على عكس طبيعته، ويقرض المال للمحتاجين دون فائدة، وفي فترات السنة السابعة يلغي الدين

\par 38 وإذا كان بخيلاً، فإنه يخضع للقانون بفعل العقل، ويمتنع عن جمع بقايا حشيشه أو قطف آخر حبات العنب من كرومه

\par 39 وفيما يتعلق بكل ما تبقى، يمكننا أن ندرك أن العقل هو المسيطر على العواطف أو المشاعر

\par 40 لأن الشريعة تتفوق على محبة الوالدين، فلا يجوز للرجل أن يتنازل عن فضيلته من أجلهما، وتتجاوز محبة الزوجة، فإذا أخطأت يجب على الرجل أن يوبخها، وتتحكم في محبة الأبناء، فإذا أساءوا عاقبهم، وتتحكم في ادعاءات الصداقة، بحيث يجب على الرجل أن يوبخ أصدقاءه إذا فعلوا الشر

\par 41 ولا تظنوا أنه أمر متناقض أن يكون العقل من خلال القانون قادرًا على التغلب حتى على الكراهية، بحيث يمتنع الإنسان عن قطع بساتين العدو، ويحمي ممتلكات العدو من المفسدين، ويجمع سلعهم التي تفرقت

\par 42 وبالمثل، ثبت أن حكم العقل يمتد من خلال الأهواء أو الرذائل الأكثر عدوانية، والطموح، والغرور، والتباهي، والكبرياء، والغيبة

\par 43 لأن العقل المعتدل يصد كل هذه الأهواء المنحطة، كما يفعل مع الغضب، لأنه ينتصر حتى على هذا

\par 44 نعم، لم يُطلق موسى العنان لغضبه عندما غضب على داثان وأبيرام، بل سيطر على غضبه بعقله

\par 45 فالعقل المعتدل قادر، كما قلت، على الانتصار على الأهواء، فيعدل بعضها، ويسحق بعضها الآخر سحقًا مطلقًا

\par 46 لماذا أيضًا ألقى أبونا الحكيم يعقوب باللوم على بيتي شمعون ولاوي في مذبحتهم غير المعقولة لسبط شكيم، قائلاً: "ملعون غضبهم!"

\par 47 لأنه لو لم يكن للعقل القدرة على كبح غضبهم لما تكلم هكذا

\par 48 "ففي اليوم الذي خلق الله فيه الإنسان، غرس فيه أهوائه وميوله، وفي الوقت نفسه، وضع العقل على عرش بين الحواس ليكون دليله المقدس في كل شيء؛ وأعطى العقل القانون، الذي إذا نظم الإنسان نفسه به، فسوف يحكم مملكة معتدلة وعادلة وفاضلة وشجاعة.

\par \textit{الحواشي}

\par \textit{179:1 انظر وصية يوسف، الصفحة 260.}

\chapter{2}

\par \textit{حكم الشهوة والغضب. قصة عطش داود. فصول مثيرة من التاريخ القديم. محاولات وحشية لإجبار اليهود على أكل الخنازير. إشارات مثيرة للاهتمام إلى بنك قديم (الآية 21)}

\par 1 حسنًا، قد يتساءل أحدهم: إذا كان العقل سيد الأهواء، فلماذا لا يكون سيد النسيان والجهل؟

\par 2 لكن الحجة سخيفة للغاية. إذ لا يُظهر العقل أنه مسيطر على الأهواء أو العيوب في ذاته، بل على تلك الموجودة في الجسد

\par 3 على سبيل المثال، لا أحد منكم قادر على استئصال رغبته الطبيعية، لكن العقل قادر على تمكينه من الهروب من أن يصبح عبدًا للرغبة

\par 4 لا أحد منكم قادر على استئصال الغضب من النفس، ولكن من الممكن أن يأتي العقل لمساعدته على الغضب

\par 5 لا يستطيع أحد منكم استئصال النزعة الخبيثة، لكن العقل يمكن أن يكون حليفه القوي ضد التأثر بالخبث

\par 6 العقل ليس استئصالًا للأهواء، بل هو خصمها.

\par 7 ولعل قضية عطش الملك داود تساعد على توضيح هذا الأمر على الأقل.

\par 8 لأنه بعد أن حارب داود الفلسطينيين طوال حياته، وبمساعدة محاربي بلادنا قتل منهم كثيرين، جاء في المساء، منهكًا من العرق والتعب، إلى الخيمة الملكية، التي كان يعسكر حولها كل جيش آبائنا

\par 9 فتناول كل الجيش عشاءهم، لكن الملك، إذ كان منهكًا من العطش الشديد، لم يستطع أن يروي عطشه، على الرغم من وفرة الماء لديه

\par 10 بدلًا من ذلك، فإن الرغبة غير المنطقية في الماء الذي كان بحوزة العدو، والتي كانت تزداد شدةً، أحرقته وأفقدته رباطة جأشه واستهلكته

\par 11 ثم عندما تذمر حارسه الشخصي من رغبة الملك، خجل شابان، محاربان عظيمان، من أن ملكهما يفتقر إلى رغبته، وارتديا جميع دروعهما، وأخذا إناء ماء، وتسلقا أسوار العدو؛ وتسللا دون أن يكتشفهما الحراس عند البوابة، وفتشا في جميع معسكر العدو

\par 12 ووجدوا النبع بشجاعة، واستقوا منه ماءً للملك

\par 13 لكن داود، على الرغم من أنه لا يزال يحترق عطشًا، اعتبر أن مثل هذه الجرعة، التي تُعتبر بمثابة دم، تشكل خطرًا جسيمًا على روحه

\par 14 لذلك، عارض عقله رغبته، وسكب الماء قربانًا لله

\par 15 "فإن العقل المعتدل قادر على التغلب على إملاءات العواطف، وإطفاء نيران الرغبة، والمصارعة منتصراً مع آلام أجسادنا على الرغم من أنها قوية للغاية، وبجمال العقل الأخلاقي وصلاحه، قادر على تحدي كل سيطرة العواطف بازدراء.

\par 16 والآن تدعونا المناسبة إلى عرض قصة العقل المتحكم في نفسه

\par 17 في وقت كان فيه آباؤنا يتمتعون بسلام عظيم من خلال مراعاة الشريعة على النحو الواجب، وكانوا في حالة جيدة، لدرجة أن سلوقس نيكانور، ملك آسيا، أقر ضريبة خدمة الهيكل، واعترف بسياستنا، في ذلك الوقت بالتحديد، قام بعض الرجال، الذين تصرفوا بشكل عدائي ضد الوفاق العام، بتوريطنا في العديد من الكوارث المتنوعة

\par 18 كان أونيا، رجلاً ذا مكانة عالية، رئيسًا للكهنة آنذاك، وكان يشغل المنصب مدى الحياة، فثار عليه رجل اسمه سمعان، ولكن لأنه على الرغم من كل أنواع الافتراء، لم يُؤذِه بسبب الشعب، هرب إلى خارج البلاد قاصدًا خيانة وطنه

\par 19 فجاء إلى أبولونيوس، حاكم سورية وفينيقيا وكيليكية، وقال: "بصفتي مخلصًا للملك، أتيت لأخبرك أن خزائن أورشليم تحتوي على آلاف عديدة من الودائع الخاصة، التي لا تنتمي إلى حساب الهيكل، بل هي ملك للملك سلوقس بحق."

\par 20 بعد أن استفسر أبولونيوس عن تفاصيل الأمر، أشاد بسيمون على خدمته المخلصة للملك، وأسرع إلى بلاط سلوقس، وكشف له عن الكنز الثمين؛ ثم بعد أن حصل على تفويض للتعامل مع الأمر، سار على الفور إلى بلادنا، برفقة سمعان الملعون وجيش قوي جدًا، وأعلن أنه موجود هناك بأمر الملك للاستيلاء على الودائع الخاصة في الخزانة

\par 21 غضب شعبنا بشدة من هذا الإعلان، واحتج بشدة، معتبرين أنه أمر شائن أن تُسلب ودائع أولئك الذين أوكلوا ودائعهم إلى خزانة الهيكل، ووضعوا كل العقبات الممكنة في طريقه

\par 22 ومع ذلك، ومع التهديدات، شق أبولونيوس طريقه إلى الهيكل.

\par 23 "ثم توسل الكهنة في الهيكل والنساء والأطفال إلى الله أن يأتي لإنقاذ مكانه المقدس الذي كان يُنتهك؛ وعندما سار أبولونيوس مع جيشه المسلح للاستيلاء على الأموال، ظهر من السماء ملائكة، يركبون على الخيول، والبرق يلمع من أذرعهم، وألقوا خوفًا عظيمًا ورعدة عليهم.

\par 24 فسقط أبولونيوس نصف ميت في ساحة الأمم، ومد يديه إلى السماء، وبدموع توسل إلى العبرانيين أن يشفعوا له ويوقفوا غضب الجند السماوي

\par 25 لأنه قال إنه أخطأ وأنه يستحق حتى الموت، وأنه إذا أُعطيت له حياته فسوف يُمجِّد لجميع الناس نعمة المكان المقدس

\par 26 وقد تأثر أونياس، رئيس الكهنة، بهذه الكلمات، رغم أنه كان شديد التدقيق في حالات أخرى، فتشفع له حتى لا يظن الملك سلوقس أن أبولونيوس قد أُطيح به بمكيدة بشرية وليس بعدالة إلهية.

\par 27 بناءً على ذلك، غادر أبولونيوس، بعد نجاته المذهلة، ليبلغ الملك بالأمور التي حلت به

\par 28 ولكن بعد وفاة سلوقس، كان خليفته على العرش ابنه أنطيوخس أبيفانس، وهو رجل متغطرس رهيب؛ الذي طرد أونيا من منصبه المقدس، وجعل شقيقه ياسون رئيس كهنة بدلاً منه، بشرط أن يدفع له ياسون ثلاثة آلاف وستمائة وستين وزنة سنويًا مقابل هذا التعيين

\par 29 فعيّن ياسون رئيسًا للكهنة وجعله رئيسًا على الشعب

\par 30 وقد أدخل (جايسون) إلى شعبنا أسلوب حياة جديدًا ودستورًا جديدًا في تحدٍّ تام للشريعة؛ حتى أنه لم يقم فقط بإنشاء صالة للألعاب الرياضية على جبل آبائنا، بل ألغى أيضًا خدمة الهيكل

\par 31 لذلك اشتعلت العدالة الإلهية غضبًا، وجلبت أنطيوخس نفسه عدوًا لنا

\par 32 لأنه عندما كان يخوض حربًا مع بطليموس في مصر وسمع أن أهل أورشليم قد فرحوا فرحًا شديدًا بخبر وفاته، سار على الفور عائدًا لمواجهتهم

\par 33 وبعد أن نهب المدينة، أصدر مرسومًا يندد بعقوبة الإعدام على كل من يُرى أنه يعيش وفقًا لشريعة آبائنا

\par 34 لكنه وجد جميع أحكامه بلا جدوى في هدم ثبات شعبنا على الشريعة، ورأى جميع تهديداته وعقوباته محتقرة تمامًا، حتى أن النساء اللواتي يختنّ أبنائهن، على الرغم من معرفتهن مسبقًا بمصيرهن، أُلقين مع ذريتهن من فوق الصخور

\par 35 لذلك، عندما استمرت مراسيمه في الازدراء من قبل عامة الشعب، حاول شخصيًا إجبار كل رجل على حدة، عن طريق التعذيب، على أكل اللحوم النجسة، وبالتالي التخلي عن الدين اليهودي

\par 36 وبناءً على ذلك، جلس الطاغية أنطيوخس، برفقة مستشاريه، للحكم على مكان مرتفع، وقد اصطفت قواته حوله بكامل دروعها، وأمر حراسه بسحب كل رجل من العبرانيين إلى هناك وإجبارهم على أكل لحم الخنزير وما ذبح للأصنام؛ ولكن إذا رفض أحد أن يتنجس بالأشياء النجسة، فإنه يُعذب ويُقتل

\par 37 وبعد أن أُخذ كثيرون بالقوة، أُحضر رجل واحد أولاً من بين الجماعة إلى أنطيوخس، وهو رجل عبراني اسمه أليعازار، كاهن بالولادة، متدرب على معرفة الشريعة، رجل متقدم في السن ومعروف لدى كثيرين من بلاط الطاغية بفلسفته

\par 38 فنظر إليه أنطيوخس وقال: "قبل أن أسمح بتعذيبك أيها الرجل الجليل، أود أن أقدم لك هذه النصيحة، أن تأكل من لحم الخنزير وتنقذ حياتك؛ لأني أحترم عمرك وشعرك الرمادي، على الرغم من أنك ارتديته لفترة طويلة، ولا تزال متمسكًا بالدين اليهودي، يجعلني أعتقد أنك لست فيلسوفًا".

\par 39 «فإن لحم هذا الحيوان الذي أنعمت به علينا الطبيعة برحمتها هو أطيب لحم، فلماذا تكرهه؟ حقًا إنه من الحماقة عدم الاستمتاع بالملذات البريئة، ومن الخطأ رفض نعم الطبيعة.»

\par 40 «لكنني أعتقد أن الأمر سيكون حماقة أكبر من جانبك إذا واصلتَ التباهي بالحقيقة حتى تتحداني أنا، مما سيعرضك لعقوبة خاصة بك.»

\par 41 ألن تستيقظ من فلسفتك السخيفة؟ ألن تتخلى عن هراء حساباتك، وتتبنى إطارًا ذهنيًا آخر يناسب سنوات نضجك، وتتعلم الفلسفة الحقيقية للمصلحة، وكيفية اتباع نصيحتي الخيرية، وتشفق على عمرك الجليل؟

\par 42 «فكّر في هذا أيضًا، أنه حتى لو كانت هناك قوة تراقب دينك هذا، فسوف تغفر لك دائمًا أي تجاوز ارتكبته تحت الإكراه.»

\par 43 بعد أن حثه الطاغية على أكل اللحوم النجسة، طلب أليعازار الإذن بالكلام؛ وعندما حصل عليه، بدأ حديثه أمام المحكمة على النحو التالي:

\par 44 نحن، يا أنطيوخس، وقد قبلنا الشريعة الإلهية قانونًا لبلادنا، لا نعتقد أن هناك ضرورة أقوى تقع علينا من طاعتنا للشريعة

\par 45 «لذلك نرى أنه من الصواب عدم مخالفة القانون بأي شكل من الأشكال.»

\par 46 ومع ذلك، لو لم يكن قانوننا، كما تقترح، إلهيًا حقًا، بينما كنا نعتقد عبثًا أنه إلهي، فلن يكون من الصواب لنا أن ندمر سمعتنا بالتقوى

\par 47 «فلا تظنوا إذن أن أكل النجس خطيئة صغيرة، لأن مخالفة الناموس، سواء كانت صغيرة أو كبيرة، شنيعة بنفس القدر؛ لأنه في كلتا الحالتين يُحتقر الناموس على حد سواء.»

\par 48 «وأنت تسخر من فلسفتنا، كما لو كنا نعيش في ظلها بطريقة تتعارض مع العقل.»

\par 49 ليس الأمر كذلك، لأن الناموس يعلمنا ضبط النفس، حتى نكون سادة على جميع ملذاتنا ورغباتنا، ونكون مدربين تدريبًا كاملاً على الرجولة حتى نتحمل كل ألم باستعداد؛ ويعلمنا العدل، حتى نتصرف بعدل مع جميع تصرفاتنا المختلفة، ويعلمنا البر، حتى نعبد باحترام فقط الإله الكائن

\par 50 «لذلك لا نأكل لحمًا نجسًا؛ لأننا نؤمن بأن شريعتنا مُعطاة من الله، ونعلم أيضًا أن خالق العالم، بصفته المشرع، يشعر بنا وفقًا لطبيعتنا.»

\par 51 «أمرنا أن نأكل ما يُريح نفوسنا، ونهانا عن أكل ما يخالف ذلك».

\par 52 "ولكن هذا هو عمل الطاغية الذي تجبرنا ليس فقط على انتهاك القانون، بل وتجبرنا أيضًا على تناول الطعام بطريقة تجعلك تسخر من هذا النجاسة البغيضة تمامًا بالنسبة لنا."

\par 53 «لكنك لا تسخر مني هكذا، ولن أنكث العهود المقدسة لأسلافي لحفظ الشريعة، حتى لو اقتلعت عينيّ وحرقت أحشائي.»

\par 54 «أنا لستُ عاجزًا بسبب الشيخوخة، ولكن عندما يكون الحق على المحك، تعود قوة الشباب إلى عقلي.»

\par 55 «لذا، لُوِّ رفوفك بقوة وانفخ فرنك أكثر سخونة. أنا لا أشفق على شيخوختي لدرجة أن أخالف قانون آبائي في شخصيتي.»

\par 56 «لن أكذبك أيها الناموس الذي كان معلّمي؛ لن أهجرك أيها ضبط النفس المحبوب؛ لن أخزيك أيها العقل المحب للحكمة، ولن أنكرك أيها الكهنوت المبجل ومعرفة الناموس.»

\par 57 «ولا تُدنّس فمي الطاهر لشيخوختي وثباتي مدى الحياة على الشريعة. سيقبلني آبائي طاهرين، غير خائفين من عذاباتك حتى الموت.»

\par 58 «لأنك قد تكون طاغية على الظالمين، لكنك لن تتحكم في قراري في أمر البر سواء بأقوالك أو بأفعالك.»

\chapter{3}

\par \textit{أظهر إليعازار، الرجل العجوز اللطيف الروح، قوة تحمل كبيرة لدرجة أنه حتى عندما نقرأ هذه الكلمات بعد 2000 عام، تبدو وكأنها نار لا تنطفئ.}

\par 1 ولكن عندما أجاب أليعازار ببلاغة على تحذيرات الطغاة، جرّه الحراس من حوله بقسوة إلى مكان التعذيب

\par 2 وخلعوا أولاً ثياب الرجل العجوز، الذي كان مزينًا بجمال القداسة

\par 3 ثم قيدوا ذراعيه من كلا الجانبين وجلدوه، وكان هناك منادٍ يقف وينادي في مواجهته: "أطع أوامر الملك!"

\par 4 لكن الرجل النبيل ذو الروح العظيمة، وهو إليعازار في الحقيقة، لم يكن يتأثر في عقله أكثر مما لو كان يتعذب في حلم؛ نعم، الرجل العجوز الذي أبقا عينيه مرفوعتين بثبات إلى السماء، سمح لجسده أن يُجلد بالسياط حتى غمرته الدماء وأصبحت جنبيه كتلة من الجروح؛ وحتى عندما سقط على الأرض لأن جسده لم يعد قادرًا على تحمل الألم، فقد حافظ على عقله منتصبًا وغير مرن

\par 5 بقدمه، ركله أحد حراس القنينة بوحشية في جنبه أثناء سقوطه ليجعله ينهض

\par 6 لكنه تحمل الألم، واحتقر الإكراه، وتحمل العذابات، ومثل رياضي شجاع يتقبل العقاب، تفوق الرجل العجوز على معذبيه

\par 7 وقف العرق على جبينه، وأخذ أنفاسه في شهقات قوية، حتى استحوذ نبل روحه على إعجاب معذبيه أنفسهم

\par 8 عندئذٍ، ومن باب الشفقة على شيخوخته، ومن باب التعاطف مع صديقهم، ومن باب الإعجاب بشجاعته، ذهب بعض رجال حاشية الملك إليه وقالوا:

\par 9 لماذا يا إليعازار تُهلك نفسك بهذه المأساة؟ سنُحضر لك بعض اللحوم المسلوقة، ولكنك تتظاهر بأنك تأكل لحم الخنزير فقط، وهكذا تُنقذ نفسك.

\par 10 وصرخ أليعازار بصوت عالٍ، وكأن مشورتهم لم تفعل سوى زيادة عذاباته: "لا. أتمنى ألا يخطر ببالنا نحن أبناء إبراهيم مثل هذه الفكرة الشريرة التي تدفعنا بقلب ضعيف إلى تزييف جزء غير لائق بنا."

\par 11 «على عكس المنطق، لو كان الأمر كذلك بالنسبة لنا، بعد أن عشنا على الحق حتى سن الشيخوخة، وحافظنا في زي شرعي على سمعة العيش بهذه الطريقة، أن نتغير الآن ونصبح في أشخاصنا قدوة للشباب في عدم التقوى، بحيث نشجعهم على أكل اللحوم النجسة.»

\par 12 «يا للعار لو عشنا لفترة أطول قليلاً، بينما كنا نتعرض للسخرية من جميع الرجال بسبب جبنهم، وبينما كان الطاغية يحتقرنا باعتبارهم غير رجوليين، فشلنا في الدفاع عن القانون الإلهي حتى الموت.»

\par 13 «لذلك يا أبناء إبراهيم، تموتون موتًا شريفًا من أجل البر؛ أما أنتم يا أتباع الطاغية، فلماذا تتوقفون عن عملكم؟»

\par 14 فلما رأوه منتصرًا على التعذيب وغير متأثر حتى بشفقة جلاديه، جروه إلى النار

\par 15 هناك ألقوه عليه، وأحرقوه بأساليب ماكرة قاسية، وسكبوا مرقًا كريه الرائحة في أنفه

\par 16 ولكن عندما وصلت النار إلى عظامه وكان على وشك أن يسلم الروح، رفع عينيه إلى الله وقال:

\par 17 «يا الله، أنت تعلم أنني على الرغم من أنني قد أنقذ نفسي، إلا أنني أموت بعذابات نارية من أجل شريعتك. ارحم شعبك، وليكن عقابنا كفارة لهم. اجعل دمي طهارتهم، وخذ روحي فديةً لنفوسهم.»

\par 18 «وبهذه الكلمات، أسلم الرجل المقدس روحه بنبل تحت وطأة التعذيب، ومن أجل القانون الذي يدافع عنه عقله، حتى ضد العذابات حتى الموت.»

\par 19 لا شك إذن أن العقل المُلهم هو سيد الأهواء؛ لأنه لو سادت أهواءه أو آلامه على عقله لكنا اعتبرناها دليلاً على قوتها المتفوقة

\par 20 ولكن الآن، بعد أن تغلب عقله على أهوائه، فإننا ننسب إليه بحق القدرة على التحكم فيها

\par 21 ومن الصواب أن نعترف بأن السيادة تكمن في العقل، على الأقل في الحالات التي يتغلب فيها على الآلام التي تأتي من خارج أنفسنا؛ لأنه من السخافة إنكاره

\par 22 ودليلي لا يشمل تفوق العقل على الآلام فحسب، بل يشمل أيضًا تفوقه على الملذات؛ كما أنه لا يستسلم لها

\chapter{4}

\par \textit{قد يُقرأ في هذا الفصل ما يُسمى "عصر العقل" أن فلسفة العقل عمرها 2000 عام. قصة سبعة أبناء وأمهم.}

\par 1 "لأن سبب أبينا العازار، مثل ربان ماهر يقود سفينة القداسة في بحر الآلام، على الرغم من تعرضه لتهديدات الطاغية واجتياحه لأمواج التعذيب المتصاعدة، لم يحرك دفة القداسة لحظة واحدة حتى أبحر إلى ميناء النصر على الموت.

\par 2 لم تدافع أي مدينة محاصرة بالعديد من الآلات الماكرة عن نفسها بشكل جيد كما فعل ذلك الرجل المقدس عندما تعرضت روحه المقدسة للهجوم بالسوط والضرب واللهب، وحرك أولئك الذين كانوا يحاصرون روحه من خلال عقله الذي كان درع القداسة

\par 3 لأن أبانا العازار، إذ جعل عقله فيلمًا كجرف بحري خنفساء، كسر البداية المجنونة لموجات العواطف

\par 4 أيها الكاهن المستحق لكهنوتك، لم تُنجس أسنانك المقدسة، ولم تُدنس بلحم نجس بطنك الذي لم يكن فيه مكان إلا للتقوى والطهارة

\par 5 يا مُعترفًا بالناموس وفيلسوفًا للحياة الإلهية! هكذا ينبغي أن يكون أولئك الذين مهمتهم خدمة الناموس والدفاع عنه بدمائهم وعرقهم الشريف في مواجهة المعاناة حتى الموت

\par 6 أنت أيها الآب، عززت إخلاصنا للشريعة من خلال ثباتك نحو المجد؛ وبعد أن تكلمت في شرف القداسة، لم تكذب كلامك، وأيدت كلمات الفلسفة الإلهية بأفعالك، أيها الرجل العجوز الذي كنت أقوى من التعذيب

\par 7 أيها الشيخ الموقر الذي كنت أشد توترًا من اللهب، أيها الملك العظيم على العواطف، إليعازار

\par 8 فكما أن أبونا هارون، مسلحًا بالمبخرة، ركض عبر الجماعة الحاشدة ضد الملاك الناري وتغلب عليه، كذلك ابن هارون، أليعازار، إذ التهمته حرارة النار الذائبة، ظل ثابتًا في عقله

\par 9 ومع ذلك، فإن الأكثر روعة من كل ذلك، أنه، وهو رجل عجوز، بأوتار جسده المتوترة وعضلاته المسترخية وأعصابه الضعيفة، عاد شابًا مرة أخرى بروح عقله، وبعقل يشبه عقل إسحاق، حوّل عذابه متعدد الرؤوس إلى عجز

\par 10 يا أيها العصر المبارك، يا أيها الرأس الرمادي المبجل، يا أيها الحياة الأمينة على الشريعة والمكتملة بختم الموت!

\par 11 بالتأكيد، إذن، إذا احتقر رجل عجوز العذابات حتى الموت من أجل البر، فيجب الاعتراف بأن العقل الموحى به قادر على توجيه الأهواء

\par 12 لكن ربما يجيب البعض بأن ليس كل الرجال سادة على الأهواء لأن ليس كل الرجال لديهم عقولهم المستنير

\par 13 ولكن كل من يجعل البر أول فكره من كل قلبه، هؤلاء وحدهم قادرون على التغلب على ضعف الجسد، مؤمنين أنهم لا يموتون لله، كما لم يمت آباؤنا إبراهيم وإسحاق ويعقوب، بل يحيون لله

\par 14 لذلك، لا يوجد تناقض في أن يبدو بعض الأشخاص عبيدًا للعاطفة نتيجة لضعف عقولهم

\par 15 فمن هو الفيلسوف الذي يتبع باستقامة كل قواعد الفلسفة، والذي وضع ثقته في الله، والذي يعلم أن تحمل كل المشقة من أجل الفضيلة هو أمر مبارك، ولا يتغلب على أهوائه من أجل البر؟

\par 16 فالإنسان الحكيم وضبط النفس وحده هو المسيطر الشجاع على الأهواء

\par 17 نعم، بهذه الطريقة، حتى الصبية الصغار، لكونهم فلاسفة بحكم العقل الذي يتوافق مع الصلاح، قد انتصروا على عذابات أشد وطأة

\par 18 لأنه عندما وجد الطاغية نفسه مهزومًا بشكل ملحوظ في محاولته الأولى، وعاجزًا عن إجبار رجل عجوز على أكل لحم نجس، أمر الحراس في غضب شديد بإحضار آخرين من شباب العبرانيين، وإذا أرادوا أكل لحم نجس، أن يطلقوا سراحهم بعد أكله، وإذا رفضوا، أن يعذبوهم بوحشية أكبر

\par 19 وتحت أوامر الطاغية، أُحضر سبعة إخوة مع أمهم المسنة سجناء أمامه، وكانوا جميعًا وسيمين ومتواضعين وذوي أصول حسنة، وجذابين بشكل عام

\par 20 وعندما رآهم الطاغية هناك، واقفين كما لو كانوا جوقة احتفالية وأمهم في الوسط، لاحظهم، وأُعجب بمظهرهم النبيل والمتميز، فابتسم لهم، ودعاهم للاقتراب وقال:

\par 21 «أيها الشباب، أتمنى الخير لكل واحد منكم، وأُعجب بجمالكم، وأُكَرِّم بشدة هذه المجموعة الكبيرة من الإخوة؛ لذلك لا أنصحكم فقط بعدم الاستمرار في جنون ذلك الرجل العجوز الذي عانى بالفعل، بل أتوسل إليكم أيضًا أن تستسلموا لي وتصبحوا شركاء في صداقتي.»

\par 22 «لأنني، كما أستطيع معاقبة من يعصون أوامري، كذلك أستطيع ترقية من يطيعونني.»

\par 23 «كن على يقين إذن من أنه سيتم منحك مناصب ذات أهمية وسلطة في خدمتي إذا رفضت القانون القديم لنظامك السياسي.»

\par 24 «شاركوا في الحياة اليونانية، وامشوا في طريق جديد، واستمتعوا بشبابكم؛ لأنكم إذا أغضبتموني بعصيانكم، فستجبرونني على اللجوء إلى عقوبات رهيبة، وحكم على كل واحد منكم بالموت تحت التعذيب.»

\par 25 «إذن ارحموا أنفسكم، فأنا حتى أنا خصمكم، أشفق على شبابكم وجمالكم.»

\par 26 ألا تفكرون في أنفسكم في هذا الأمر، أنه إذا عصيتموني فلا شيء أمامكم سوى الموت في العذاب؟

\par 27 بهذه الكلمات أمر بإحضار أدوات التعذيب لإقناعهم بالخوف على أكل اللحوم النجسة

\par 28 ولكن عندما أخرج الحراس العجلات، وأدوات خلع المفاصل، والرفوف، وكسارات العظام، والمقاليع، والمراجل، والمواقد، والبراغي، والمخالب الحديدية، والأوتاد، وحديد الوسم، تحدث الطاغية مرة أخرى وقال:

\par 29 «من الأفضل أن تشعروا بالخوف يا رفاق، والعدالة التي تعبدونها ستغفر لكم تجاوزكم غير المقصود.»

\par 30 ولكنهم، عندما سمعوا إقناعاته، ورأوا مكائده المروعة، لم يظهروا أي خوف فحسب، بل نظموا فلسفتهم في معارضة الطاغية، وبفضل عقلهم الحق، حط من قدر طغيانه.

\par 31 ومع ذلك، فكروا؛ لو افترضنا أن بعضهم كان ضعيف القلب وجبانًا، فأي نوع من اللغة كانوا سيستخدمونها؟ ألن يكون الأمر كذلك؟

\par 32 «يا للأسف! نحن مخلوقات بائسة وحمقى إلى أبعد الحدود! عندما يدعونا الملك ويناشدنا بشروط المعاملة اللطيفة، ألا نطيعه؟»

\par 33 لماذا نشجع أنفسنا برغبات باطلة ونجرؤ على عصيان سيكلفنا حياتنا؟ ألا نخشى، يا إخوتي، أدوات الرعب ونزن تهديداته بالتعذيب جيدًا، ونتخلى عن هذه التباهي الفارغ وهذا التباهي المميت؟

\par 34 «فلنشفق على شبابنا، ونرحم شيخوخة أمهاتنا؛ ولنضع في قلوبنا أننا إن عصينا فسنموت.»

\par 35 «وحتى العدالة الإلهية سترحمنا، إذا اضطررنا للخضوع للملك خوفًا. لماذا نتخلص من هذه الحياة العزيزة ونحرم أنفسنا من هذا العالم الجميل؟»

\par 36 دعونا لا نقاوم الضرورة ولا ندعو إلى عذابنا بثقة باطلة

\par 37 «حتى الناموس نفسه لا يحكم علينا بالموت طوعًا، لأننا في خوف من أدوات التعذيب.»

\par 38 لماذا يُلهبنا هذا الجدل، ويحظى عنادنا القاتل بقبولنا، بينما قد نحظى بحياة سلمية بطاعة الملك؟

\par 39 لكن مثل هذه الكلمات لم تفلت من هؤلاء الشباب عند احتمال تعرضهم للتعذيب، ولم تخطر ببالهم مثل هذه الأفكار

\par 40 لأنهم كانوا يحتقرون الأهواء ويسيطرون على الألم.

\chapter{5}

فصلٌ مليءٌ بالرعب والتعذيب يكشف عن وحشية الطغيان القديم. الآية ٢٦ حقيقةٌ عميقة.

\par 1 وهكذا، ما إن انتهى الطاغية من حثهم على أكل اللحوم النجسة حتى قال له الجميع بصوت واحد، وكأنهم نفس واحدة:

\par 2 لماذا تتأخر أيها الطاغية؟ نحن مستعدون للموت على أن نخالف وصايا آبائنا

\par 3 «لأنه ينبغي لنا أن نخجل آباءنا أيضًا إن لم نسلك في طاعة الناموس ونتخذ موسى مشيرًا لنا.»

\par 4 «يا أيها الطاغية الذي ينصحنا بانتهاك القانون، لا تشفق علينا، فأنت تكرهنا، وتتجاوز شفقتك علينا.»

\par 5 «لأننا نقدر رحمتك، إذ وهبتنا حياتنا مقابل خرق الناموس، وهو أمر أشد من الموت نفسه.»

\par 6 «كنت تريد أن ترعبنا بتهديداتك بالموت تحت التعذيب، كما لو أنك لم تتعلم شيئًا من أليعازار قبل قليل.»

\par 7 «ولكن إذا كان شيوخ العبرانيين قد تحملوا العذابات من أجل البر، نعم، حتى ماتوا، فمن الأجدر بنا نحن الشباب أن نموت محتقرين عذابات إكراهك، التي انتصر عليها هو أيضًا معلمنا الشيخ.»

\par 8 "فحاكمنا أيها الطاغية. وإن كنت تقتلنا من أجل الحق، فلا تظن أنك تؤذينا بعذاباتك."

\par 9 «لأننا من خلال معاملتنا السيئة هذه وتحملنا لها سنفوز بجائزة الفضيلة؛ أما أنتم، بسبب قتلنا الوحشي، فستعانون على يد العدالة الإلهية عذابًا كافيًا بالنار إلى الأبد.»

\par 10 ضاعفت كلمات الشباب هذه غضب الطاغية، ليس فقط على عصيانهم، بل أيضًا على ما اعتبره جحودًا منهم

\par 11 فبأمره، أحضر الجلادون أكبرهم سنًا وجردوه من ثوبه وربطوا يديه وذراعيه من كل جانب بأحزمة

\par 12 ولكن بعد أن جلدوه حتى تعبوا، ولم ينالوا شيئًا، ألقوه على الدولاب

\par 13 وعليها عذب الشاب النبيل حتى تمزقت عظامه. وكلما تمزق مفصل تلو الآخر، شجب الطاغية بهذه الكلمات:

\par 14 «يا أيها الطاغية البغيض، يا عدو عدالة السماء، يا دموي العقل، إنك تعذبني بهذه الطريقة ليس بسبب القتل أو الكفر، بل بسبب دفاعي عن شريعة الله.»

\par 15 وعندما قال له الحراس: "وافق على الأكل، حتى تتحرر من عذاباتك"، قال لهم: "إن طريقتكم، أيها الأتباع البائسون، ليست قوية بما يكفي لأسر عقلي. اقطعوا أطرافي، واحرقوا لحمي، ولووا مفاصلي؛ من خلال كل العذابات سأريكم أنه من أجل الفضيلة، فإن أبناء العبرانيين وحدهم لا يُقهرون."

\par 16 وبينما كان يتكلم هكذا، أشعلوا عليه جمرًا ساخنًا، واشتدت وطأة التعذيب عليه، مما زاد من شدته على العجلة

\par 17 وكانت العجلة بأكملها ملطخة بدمه، وأُخمدت الجمر المتراكم بفعل أخلاط جسده المتساقطة، ودار اللحم الممزق حول محاور الآلة

\par 18 ومع انحلال جسده، لم يتأوه هذا الشاب العظيم الروح، كابن حقيقي لإبراهيم، على الإطلاق؛ بل كما لو كان يعاني من تغيير بالنار إلى عدم الفساد، فقد تحمل العذاب بنبل، قائلاً:

\par 19 «اتبعوا قدوتي يا إخوتي. لا تتخلوا عني إلى الأبد، ولا تتخلوا عن أخوتنا في نبل الروح.»

\par 20 «خُض حربًا مقدسة ومشرفة من أجل البر، والتي من خلالها تُصبح العناية الإلهية العادلة التي سهرت على آبائنا رحيمة بشعبها وتنتقم من الطاغية الملعون.»

\par 21 وبهذه الكلمات أسلم الشاب المقدس روحه.

\par 22 ولكن بينما كان الجميع يتعجبون من ثبات روحه، أحضر الحراس الثاني في العمر من بين أبنائه، وأمسكوه بأيدي حادة من الحديد وربطوه بالمحركات والمنجنيق.

\par 23 ولكن عندما سمعوا عزمه النبيل ردًا على سؤالهم: "أيأكل بدلًا من أن يُعذب؟" مزقت هذه الوحوش الشبيهة بالفهود أوتاره بمخالب من حديد، ومزقت كل لحم خديه، وسلخت الجلد عن رأسه

\par 24 ولكنه صبر على هذا العذاب وقال: ما أحلى كل موت من أجل بر آبائنا!

\par 25 وقال للطاغية: يا أشد الطغاة قسوة، ألا يبدو لك أنك في هذه اللحظة تعاني من عذابات أشد من عذابي عندما ترى نية طغيانك المتغطرسة تتغلب على صبري من أجل الصلاح؟

\par 26 «لأني أستمد قوتي من الألم بفضل الأفراح التي تأتي من الفضيلة، بينما أنت في عذاب وأنت تفتخر بكفرك؛ ولن تنجو، أيها الطاغية البغيض، من عقوبات الغضب الإلهي.»

\par 27 وبعد أن لقي حتفه المجيد بشجاعة، قُدِّم الابن الثالث، وتوسل إليه الكثيرون بإلحاح أن يذوق، وبالتالي أن ينقذ نفسه

\par 28 فأجاب بصوت عظيم: «أتجهلون أن نفس الأب ولدني وإخوتي الأموات، وأن نفس الأم ولدتنا، وعلى نفس التعليم رُبيت؟»

\par 29 «لا أقسم برباط الأخوة النبيل.»

\par 30 "لذلك إذا كان لديكم أي وسيلة للتعذيب، فاستخدموها على جسدي؛ لأن روحي لا يمكنكم الوصول إليها، ليس إذا أردتم ذلك."

\par 31 لكنهم غضبوا بشدة من كلام الرجل الجريء، فخلعوا يديه وقدميه بمكائنهم المخلوعة، وانتزعوا أطرافه من محاجرها، وفكّوها، ولفّوا أصابعه وذراعيه وساقيه ومفاصل مرفقيه

\par 32 ولأنهم لم يتمكنوا من خنق روحه بأي شكل من الأشكال، فقد نزعوا جلده، آخذين معه أطراف الأصابع، ومزقوا فروة رأسه على الطريقة السكيثية، وأحضروه على الفور إلى العجلة

\par 33 وعلى هذا لوى عموده الفقري حتى رأى لحمه معلقًا في شرائط وكتلًا كبيرة من الدم تتدفق من أحشائه

\par 34 وعند اقترابه من الموت قال: "نحن، أيها الطاغية البغيض، نعاني هكذا من أجل تربيتنا وفضيلتنا التي هي من الله؛ أما أنت، فستعاني من أجل كفرك وقسوتك عذابات لا نهاية لها."

\par 35 ولما مات هذا الرجل مستحقًا إخوته، أحضروا الرابع، وقالوا له: "لا تصب أنت أيضًا بنفس جنون إخوتك، بل أطع الملك وخلص نفسك."

\par 36 فقال لهم: «ليس عندكم لي نارٌ حارةٌ تجعلني جبانًا».

\par 37 «بموت إخوتي المبارك، وبالهلاك الأبدي للطاغية، وبالحياة المجيدة للصالحين، لن أنكر أخوتي النبيلة.»

\par 38 «اخترع عذابات أيها الطاغية، لتعلم بذلك أنني أخ لمن عُذبوا بالفعل.»

\par 39 عندما سمع أنطيوخس المتعطش للدماء، القاتل، والبغيض للغاية هذا، أمرهم بقطع لسانه

\par 40 لكنه قال: «حتى لو نزعتَ عني عضو نطقي، فإن الله سامعٌ أيضًا للبكماء».

\par 41 «ها أنا ذا أُخرج لساني: اقطعه، لأنك لن تُسكت عقلي بذلك.»

\par 42 بكل سرور نعطي أعضاء أجسادنا ليتم تمثيلها في سبيل الله

\par 43 «لكن الله سيطاردك سريعًا؛ لأنك قطعت اللسان الذي غنى له أغاني التسبيح.»

\par 44 ولكن عندما مات هذا الرجل أيضًا من شدة العذاب، اندفع الخامس إلى الأمام قائلًا: "لا أتردد، أيها الطاغية، في طلب التعذيب من أجل الفضيلة."

\par 45 «نعم، لقد تقدمت من نفسي، لكي تتمكن، بقتلي أيضًا، من خلال المزيد من الأفعال السيئة، من زيادة العقوبة التي تدين بها لعدالة السماء.»

\par 46 يا عدو الفضيلة وعدو الإنسان، بأي جرم تدمرنا بهذه الطريقة؟

\par 47 «هل يبدو لك شرًا أن نعبد خالق الجميع ونعيش وفقًا لشريعته الفاضلة؟»

\par 48 «لكن هذه الأشياء تستحق التكريم لا العذاب، إذا كنت تفهم التطلعات البشرية وكان لديك أمل في الخلاص أمام الله.»

\par 49 «ها أنت الآن عدو الله، وتحارب من يعبدون الله.»

\par 50 وبينما كان يتكلم هكذا، قيده الحراس وأحضروه أمام المنجنيق؛ وربطوه به على ركبتيه، وربطوهما هناك بمشابك حديدية، وشدوا حقويه فوق "الإسفين" المتدحرج حتى التفت إلى الخلف تمامًا مثل العقرب، وانفصلت كل مفصل منه

\par 51 وهكذا، في ضيق شديد في التنفس وعذاب جسدي، صرخ قائلًا: "مجيد، أيها الطاغية، مجيد رغماً عنك، هي النعم التي تمنحني إياها، مما مكنني من إظهار إخلاصي للقانون من خلال المزيد من التعذيبات المشرفة."

\par 52 وعندما مات هذا الرجل أيضًا، أُحضر السادس، وهو مجرد صبي، والذي أجاب على سؤال الطاغية عما إذا كان على استعداد لتناول الطعام وإطلاق سراحه، فقال:

\par 53 «أنا لستُ كبيرًا في السن مثل إخوتي، لكنني كبير في السن في العقل. لأننا وُلدنا وتربينا لنفس الغرض، وملزمون أيضًا بالموت لنفس السبب؛ فإذا اخترتَ تعذيبنا لعدم أكل لحم نجس، فعذبنا.»

\par 54 وبينما كان ينطق بهذه الكلمات، أحضروه إلى الدولاب، ومدّوه بعناية وخلعوا عظام ظهره وأشعلوا النار تحته

\par 55 وصنعوا أسياخًا حادة ساخنة وغرزوها في ظهره، وثقبوا جانبيه وأحرقوا أحشائه أيضًا

\par 56 لكنه في خضم عذاباته صرخ قائلًا: "يا لها من مسابقة تليق بالقديسين، حيث دخل الكثير منا نحن الإخوة، في سبيل البر، في مسابقة عذابات، ولم يُهزموا!"

\par 57 لأن الفهم المستقيم، أيها الطاغية، لا يُقهر.

\par 58 في درع الفضيلة أذهب للانضمام إلى إخوتي في الموت، ولأضيف إلى نفسي منتقمًا قويًا آخر لمعاقبتك، يا مخترع التعذيب وعدو الصالحين حقًا.

\par 59 لقد أطاحنا نحن الشباب الستة بطغيانك. أليس عجزك عن تغيير عقولنا أو إجبارنا على أكل لحم نجس انقلابًا عليك؟

\par 60 "إن نارك باردة علينا، وأدوات تعذيبك لا تعذبنا، وعنفك عاجز."

\par 61 «لأن الحراس كانوا ضباطًا لنا، ليس لطاغية، بل للقانون الإلهي؛ ولذلك لم يُقهر عقلنا بعد.»

\chapter{6}

\par \textit{روابط أخوية وحب أم.}

\par 1 "وعندما مات هذا أيضًا موتًا مباركًا، حيث أُلقي في المرجل، تقدم الابن السابع، وهو أصغرهم جميعًا،."

\par 2 لكن الطاغية، على الرغم من غضبه الشديد من إخوته، شعر بالشفقة على الصبي، ولما رآه مقيدًا هناك بالفعل، أمر بإحضاره، وسعى لإقناعه قائلاً:

\par 3 «أنت ترى نهاية حماقة إخوتك؛ لأنهم بسبب عصيانهم قد تعرضوا للتعذيب حتى الموت. أنت أيضًا، إذا لم تطع، فسوف تُعذب بشدة وتُقتل قبل أوانك؛ ولكن إذا أطعت، فستكون صديقي، وستُرقّى إلى منصب رفيع في أعمال المملكة.»

\par 4 وبينما كان يناشده بهذه الطريقة، أرسل في طلب والدة الصبي، حتى تتمكن، في حزنها على فقدان هذا العدد الكبير من الأبناء، من حث الناجي على الطاعة والنجاة

\par 5 لكن الأم، وهي تتحدث باللغة العبرية، كما سأخبرك لاحقًا، شجعت الصبي، فقال للحراس: "أطلقوني لأتحدث إلى الملك وجميع أصدقائه معه".

\par 6 ففرحوا بطلب الصبي، وسارعوا إلى إطلاق سراحه

\par 7 وركض نحو المِجمر المُلتهب، وصاح قائلًا: "أيها الطاغية الكافر، وأكثر الخطاة كفرًا، ألا تخجل من أن تأخذ بركاتك ومُلكك على يد الله، وأن تقتل عبيده وتعذب أتباع البر؟"

\par 8 «لهذا السبب تُسلمك العدالة الإلهية إلى نار أبدية أسرع وعذابات لن تُبقيك مُسيطرًا إلى الأبد.»

\par 9 ألا تخجل، وأنت رجل، أيها الوغد ذو قلب وحش بري، من أن تأخذ معك رجالًا ذوي مشاعر مماثلة، مصنوعين من نفس العناصر، وتقطع ألسنتهم، وتجلدهم وتعذبهم بهذه الطريقة؟

\par 10 «ولكن بينما يُكملون برهم تجاه الله في موتهم النبيل، ستصرخ بحزن: «ويلٌ قد حل» لقتلك الظالم لأبطال الفضيلة.»

\par 11 ثم وقف على شفا الموت وقال: "أنا لستُ منكرًا للشهادة التي أداها إخوتي."

\par 12 «وأنا أدعو إله آبائي أن يرحم أمتي».

\par 13 'وَلَيُعَذِّبْكَ فِي الْحَيَاةِ الْأَدْنَى وَبَعْدَ مَا مَوْتٍ'

\par 14 ومع هذه الصلاة، ألقى بنفسه في الموقدة الملتهبة، وهكذا أسلم الروح

\par 15 إذا كان الإخوة السبعة قد احتقروا التعذيب حتى الموت، فقد ثبت عالميًا أن العقل المُلهم هو السيد الأعلى على الأهواء

\par 16 لأنه لو استسلموا لأهوائهم أو آلامهم وأكلوا لحمًا نجسًا، لقلنا إنهم قد غلبوا بذلك.

\par 17 ولكن في هذه الحالة لم يكن الأمر كذلك؛ بل على العكس، فبسبب عقولهم التي أُشيد بها في نظر الله، ارتقوا فوق أهوائهم

\par 18 ومن المستحيل إنكار سيادة العقل؛ لأنهم انتصروا على أهوائهم وآلامهم

\par 19 كيف يمكننا أن نفعل غير ذلك سوى الاعتراف بسيادة العقل السليم على العاطفة مع هؤلاء الرجال الذين لم يتراجعوا أمام آلام الحرق؟

\par 20 فكما أن الأبراج على خلد الميناء تصد هجمات الأمواج وتوفر مدخلاً هادئًا لمن يدخلون الميناء، كذلك دافع عقل الشباب الأيمن ذو الأبراج السبعة عن ميناء البر وصدّ عاصفة الأهواء

\par 21 شكلوا جوقة مقدسة من البر وهم يهتفون لبعضهم البعض قائلين:

\par 22 «لنمتْ كإخوة، أيها الإخوة، من أجل الناموس.»

\par 23 "دعونا نحاكي الأطفال الثلاثة في البلاط الآشوري الذين احتقروا نفس محنة الفرن."

\par 24 «لا نتراجع جبناء أمام برهان البر.»

\par 25 وقال واحد: "أيها الأخ، كن مطمئنًا"، وقال آخر: "تحمل الأمر بشرف"، وقال آخر متذكرًا الماضي: "تذكروا من أي أصل أنتم، ومن على يد أبيكم سلم إسحق نفسه ذبيحة من أجل البر".

\par 26 فقال كل واحد منهم معًا، وهو ينظر إلى الآخر بنظرة مشرقة وجرأة شديدة: "بقلب كامل سنكرس أنفسنا لله الذي أعطانا أرواحنا، ولنقرض أجسادنا لحفظ الناموس."

\par 27 «لا نخشَ من يظن أنه يقتل؛ لأن صراعًا عظيمًا وخطرًا على النفس ينتظر في عذاب أبدي أولئك الذين يخالفون شرع الله.»

\par 28 فلنتسلح إذًا بسيادة العقل الإلهي على الأهواء

\par 29 «بعد هذه آلامنا، سيقبلنا إبراهيم وإسحاق ويعقوب، وسيمدحنا جميع آبائنا.»

\par 30 وبينما كان يُسحب كل واحد من الإخوة على حدة، قال الذين لم يأتِ دورهم بعد: «لا تُخجِلنا يا أخي، ولا تكذب على إخوتنا الذين ماتوا».

\par 31 أنتم لستم جاهلين بمحبة الإخوة، التي وهبتها العناية الإلهية الحكيمة ميراثًا لأولئك الذين ولدوا من آبائهم، وغرستها فيهم حتى من خلال رحم أمهاتهم؛ حيث يعيش الإخوة في نفس الفترة، ويتخذون شكلهم خلال نفس الوقت، ويتغذون من نفس الدم، ويحيون بنفس الروح، ويأتون إلى العالم بعد نفس المكان، ويستمدون الحليب من نفس الينابيع، حيث تحتضن أرواحهم الأخوية معًا؛ وهم مرتبطون بشكل أوثق من خلال الرعاية المشتركة والرفقة اليومية وغيرها من التعليم، ومن خلال تأديبنا بموجب شريعة الله

\par 32 لما كان شعورهم بالمحبة الأخوية قويًا بطبيعتهم، ازداد انسجامهم المتبادل قوة. فبتدريبهم على شريعة واحدة، وتأديبهم على الفضائل نفسها، وتربيتهم معًا على حياة مستقيمة، ازداد حبهم لبعضهم البعض سخاءً. وقد زاد حماسهم المشترك للجمال الأخلاقي والخير من انسجامهم المتبادل، إذ زاد، إلى جانب تقواهم، من اشتعال محبتهم الأخوية.

\par 33 ولكن على الرغم من أن الطبيعة والرفقة وتصرفاتهم الفاضلة زادت من حماسة حبهم الأخوي، إلا أن الأبناء الباقين على قيد الحياة من خلال دينهم أيدوا رؤية إخوانهم الذين كانوا على وشك الموت وهم يتعرضون للتعذيب؛ بل وأكثر من ذلك، فقد شجعوهم على مواجهة العذاب، ليس فقط لكي يحتقروا تعذيبهم، ولكن أيضًا للتغلب على شغفهم بالمودة الأخوية تجاه إخوانهم

\par 34 يا أيها العقول العاقلة، الأكثر ملكية من الملوك، والأكثر حرية من الأحرار، من انسجام الإخوة السبعة، القديسين والمنسجمين جيدًا مع النغمة الأساسية للتقوى!

\par 35 لم يجبن أي من الشباب السبعة، ولم يتقلص أي منهم في وجه الموت، بل سارعوا جميعًا إلى الموت تحت وطأة التعذيب كما لو كانوا يركضون في طريق الخلود

\par 36 فكما تتحرك الأيدي والأقدام بانسجام مع إيحاءات الروح، كذلك ذهب أولئك الشباب المقدسون، وكأنهم مدفوعون بروح الدين الخالدة، إلى الموت بانسجام من أجله

\par 37 يا رفقة الإخوة السبعة المقدسة في وئام!

\par 38 فكما أن الأيام السبعة لخلق العالم كانت بمثابة احتفال بالدين، كذلك كان الشباب مثل جوقة احتفالاً بصحبتهم السبعة، وجعلوا رعب التعذيب بلا قيمة.

\par 39 نرتعد الآن عندما نسمع عن معاناة هؤلاء الشباب؛ لكنهم لم يروا ذلك بأعينهم فحسب، ولا سمعوا التهديد الوشيك المنطوق فحسب، بل شعروا بالألم بالفعل، وتحملوه؛ وذلك في عذاب النار، أي عذاب أعظم يمكن أن نجده؟

\par 40 لأن قوة النار حادة وقاسية، وسرعان ما أدت إلى تحلل أجسادهم

\par 41 ولا تظنوا أنه من العجيب أن ينتصر العقل مع هؤلاء الرجال على التعذيب، في حين أن روح المرأة تحتقر تنوعًا أكبر من الآلام؛ لأن والدة الشباب السبعة تحملت العذابات التي تعرض لها كل واحد من أطفالها

\par 42 لكن تأملوا كم تتعدد أشواق قلب الأم، حتى أن شعورها تجاه صغارها يصبح محور عالمها كله؛ وفي الواقع، هنا، حتى الحيوانات غير العاقلة لديها تجاه صغارها عاطفة وحب مماثلين للبشر

\par 43 على سبيل المثال، من بين الطيور، تدافع الطيور الأليفة التي تحتمي تحت أسقفنا عن صغارها؛ وتلك التي تعشش على قمم الجبال، وفي شقوق الصخور، وفي تجاويف الأشجار، وفي الأغصان، وتفقس صغارها هناك، تطرد أيضًا الدخيل

\par 44 ثم إذا لم يتمكنوا من إبعاده، فإنهم يرفرفون حول الفراخ في شغف من الحب، وينادون عليهم بلغتهم الخاصة، ويقدمون المساعدة لصغارهم بأي طريقة ممكنة.

\par 45 وما حاجتنا إلى أمثلة على حب النسل بين الحيوانات غير العاقلة، عندما تقوم النحلات، في موسم صنع الشفرات تقريبًا، بصد الدخلاء، وتطعن بلسعتها، كما لو كانت بالسيف، أولئك الذين يقتربون من صغارها، وتقاتلهم حتى الموت؟

\par 46 لكنها، أم هؤلاء الشباب، ذات الروح مثل إبراهيم، لم تتأثر بمشاعرها تجاه أطفالها

\chapter{7}

\par \textit{مقارنة بين عاطفة الأم والأب، في هذا الفصل بعض قمم البلاغة.}

\par 1 عقل الأبناء، سيد الأهواء! يا للدين، الذي كان أعز على الأم من أبنائها!

\par 2 الأم، إذ كان أمامها خياران، الدين وإنقاذ أبنائها السبعة أحياءً وفقًا لوعد الطاغية، أحبت الدين الذي يخلص للحياة الأبدية وفقًا لله

\par 3 كيف لي أن أعبر عن حب الوالدين الشغوف لأبنائهم؟ فنحن نطبع صورة رائعة لروحنا وشكلنا على طبيعة الطفل الرقيقة، والأهم من ذلك كله أن تعاطف الأم مع أبنائها أعمق من تعاطف الأب

\par 4 فالنساء أرق نفسًا من الرجال، وكلما زاد عدد الأطفال الذين أنجبنهم، زاد حبهن لهم

\par 5 ولكن من بين جميع الأمهات، كانت هي من بين أبنائها السبعة مليئة بالمحبة أكثر من البقية، إذ رأت أنها، بعد أن شعرت في سبع ولادة بعطف أموميّ تجاه ثمرة بطنها، وبعد أن أُجبرت بسبب الآلام الكثيرة التي حملتها بكل واحد منهم على عاطفة وثيقة، إلا أنها مع ذلك، من خلال مخافة الله، رفضت الأمان الحالي لأطفالها

\par 6 بل وأكثر من ذلك، من خلال الجمال الأخلاقي وصلاح أبنائها وطاعتهم للشريعة، أصبح حبها الأمومي لهم أقوى

\par 7 لأنهم كانوا عادلين، ومعتدلين، وشجعان، وذوي نفوس عظيمة، ومحبين لبعضهم البعض ولأمهم بطريقة جعلتهم يطيعونها في حفظ الشريعة حتى الموت

\par 8 ولكن مع ذلك، وعلى الرغم من أنها واجهت الكثير من الإغراءات للاستسلام لغرائزها الأمومية، إلا أنه في أي حالة واحدة لم يكن لمجموعة التعذيبات المروعة القدرة على تغيير عقلها؛ بل حثت الأم كل ابن على حدة، وجميعهم معًا، على الموت من أجل دينهم

\par 9 يا للطبيعة المقدسة، وحب الوالدين، وشوق الوالدين إلى الأبناء، وأجر الرضاعة، وحنان الأمهات الذي لا يُقهر!

\par 10 عندما رأتهم الأم يُعذبون ويُحرقون واحدًا تلو الآخر، ظلت ثابتة في روحها من أجل الدين

\par 11 فرأت أجساد أبنائها تحترق في النار، وأطراف أيديهم وأرجلهم متناثرة على الأرض، والأغطية التي كانوا يرتدونها ممزقة من رؤوسهم إلى خدودهم، مبعثرة مثل الأقنعة.

\par 12 يا أمّي، يا من عرفتِ الآن آلامًا أشدّ من آلام المخاض! يا امرأة، يا الوحيدة بين النساء، التي كانت ثمرة بطنها الدين الكامل!

\par 13 لم يغير ابنك البكر، الذي فارق الحياة، عزمك، ولا ابنك الثاني، الذي نظر إليك بعيون الشفقة تحت عذاباته، ولا ابنك الثالث، الذي زفر روحه

\par 14 ولم تبكي عندما رأيت عيون كل واحد منهم وسط العذابات تنظر بجرأة إلى نفس الألم، ورأيت في أنوفهم المرتعشة علامات اقتراب الموت

\par 15 عندما رأيت لحم ابن يُقطع تلو الآخر، ويدًا تلو الأخرى تُقطع، ورأسًا تلو الآخر يُسلخ، وجثة تلو الأخرى تُلقى، والمكان يزدحم بالمتفرجين بسبب تعذيب أبنائك، لم تذرف دمعة واحدة

\par 16 لا تسحر ألحان حوريات البحر ولا أغاني البجع ذات الصوت العذب آذان السامع، كما تسحر أصوات الأبناء وهم يتحدثون إلى الأم من وسط العذاب

\par 17 كم وكم كانت عظيمة من العذابات التي عذّبت بها الأم بينما كان أبناؤها يُعذّبون بعذابات الغلي والنار!

\par 18 لكن العقل المُلهم منح قلبها قوة رجل تحت وطأة شغفها بالمعاناة، ورفعها إلى درجة أنها لم تعد تُراعي التطلعات الحالية لحب الأم

\par 19 وعلى الرغم من أنها رأت تدمير أطفالها السبعة وأشكال عذابهم العديدة والمتنوعة، إلا أن الأم النبيلة استسلمت لهم طواعيةً من خلال الإيمان بالله

\par 20 لأنها رأت في عقلها، كما لو كان محامين ماكرين في قاعة المجلس، الطبيعة، والأبوة، وحب الأم، وأطفالها على المحك، وكان الأمر كما لو أنها، الأم، لديها الاختيار بين صوتين في قضية أطفالها، صوت لموتهم وآخر لإنقاذهم أحياء، لذلك لم تفكر في إنقاذ أبنائها السبعة لفترة قصيرة، ولكن، كابنة حقيقية لإبراهيم، تذكرت شجاعته في تقوى الله

\par 21 يا أمَّ العِرق، يا مُدافعةَ شريعتنا، يا مُدافعةَ ديننا، يا فائزةً بالجائزة في النضالِ داخلَ ذاتِك!

\par 22 يا امرأة، أنبل من الرجال في المقاومة، وأشجع من المحاربين في التحمل!

\par 23 فكما صمدت سفينة نوح، بكل ما فيها من عالم حي، في الطوفان الذي اجتاح العالم، أمام الأمواج العاتية، كذلك أنت، يا حافظ الشريعة، وقد ضربتك أمواج الأهواء العاتية من كل جانب، وتعرضت لضربات قوية من عذابات أبنائك، فقد نجوت بنبل من العواصف التي هاجمتك من أجل الدين

\par 24 وهكذا، إذا كانت امرأة متقدمة في السن، وأم لسبعة أبناء، تتحمل رؤية أطفالها يتعرضون للتعذيب حتى الموت، فإن العقل الملهم يجب أن يكون بلا شك الحاكم الأعلى على العواطف.

\par 25 لقد أثبتتُ، بناءً على ذلك، أن الرجال لم ينتصروا على معاناتهم فحسب، بل إن المرأة أيضًا احتقرت أشد أنواع التعذيب فظاعة

\par 26 ولم تكن الأسود حول دانيال شرسة، ولم يكن أتون ميشائيل المشتعل حارقًا لدرجة أنها أحرقت غريزة الأمومة عند رؤية أبنائها السبعة وهم يُعذبون

\par 27 ولكن بفضل عقلها الموجه بالدين، أطفأت الأم عواطفها، مهما كانت كثيرة وقوية

\par 28 هناك أيضًا ما يجب مراعاته، وهو أنه لو كانت المرأة ضعيفة الروح، على الرغم من أمومتها، لربما بكت عليهم، وربما قالت هذا:

\par 29 «آه، لقد بئستُ ثلاث مرات، وأكثر من ثلاث مرات! لقد أنجبتُ سبعة أطفال وبقيت بلا أطفال!»

\par 30 «عبثًا حملتُ سبع مرات، وعبثًا حملتُ عشرة أشهر سبع مرات، وكانت رضاعتي غير مثمرة، ورضّاعاتي حزينة.»

\par 31 عبثًا، يا أبنائي، تحملتُ آلام العمل الكثيرة، وهموم تربيتكم الأكثر صعوبة

\par 32 يا للأسف على أبنائي، بعضهم لم يتزوج بعد، والذين تزوجوا لم ينجبوا أطفالًا؛ لن أرى أطفالًا لكم أبدًا، ولن أُدعى باسم الجد

\par 33 آه، يا من كان لي العديد من الأطفال الجميلين، وأنا أرملة وبائسة في بؤسي! ولن يكون هناك ابن ليدفنني بعد وفاتي!

\par 34 لكن الأم القديسة المتقية لم تندب أحدًا منهم بهذا الرثاء، ولم تتوسل لأحد أن ينجو من الموت، ولم تندب عليهم كرجال يحتضرون؛ بل كما لو كانت تمتلك روحًا من الألماس، وأنها ستلد عددًا من أبنائها، للمرة الثانية، إلى الحياة الخالدة، فقد توسلت إليهم وتوسلت إليهم أن يموتوا من أجل الدين

\par 35 يا أماه، يا محاربة الله في سبيل الدين، عجوزًا وامرأة، لقد هزمتِ الطاغية بصبركِ، ووجدتِ أقوى من الرجل، في الأفعال كما في الأقوال

\par 36 لأنه حقًا عندما وُضعتِ في قيود مع أبنائك، وقفتِ هناك ورأيتِ أليعازار يُعذب، وكلمتِ أبنائك باللغة العبرية:

\par 37 «يا أبنائي، إن القتال نبيل؛ وأنتم مدعوون إليه للشهادة لأمتنا، فقاتلوا فيه بحماسة من أجل شريعة آبائنا.»

\par 38 «لأنه سيكون من المخزي أن تتقلصوا أنتم الشباب أمام الألم، بينما يتحمل هذا الرجل المسن هذا العذاب من أجل الدين.»

\par 39 "تذكروا أنكم من أجل الله أتيتم إلى العالم، واستمتعتم بالحياة، ولذلك فإنكم مدينون لله أن تتحملوا كل الألم من أجله؛ الذي من أجله سارع أبونا إبراهيم إلى التضحية بابنه إسحق، جد أمتنا؛ وإسحق، عندما رأى يد أبيه ترفع السكين عليه، لم يتراجع."

\par 40 «وطرح دانيال البار إلى الأسود، وألقي حننيا وعزريا وميشائيل في أتون النار، وصبروا من أجل الله.»

\par 41 «وأنتم أيضًا، إذ لكم إيمانٌ واحدٌ بالله، لا ترتاعوا، لأنه كان ضد العقل أن لا تصمدوا أمام الآلام، وأنتم عارفون البر.»

\par 42 بهذه الكلمات، شجعت أم السبعة كل واحد من أبنائها على الموت بدلاً من مخالفة أمر الله؛ وهم أنفسهم يعلمون جيدًا أن الرجال الذين يموتون من أجل الله يعيشون لله، كما يعيش إبراهيم وإسحاق ويعقوب وجميع الآباء



\chapter{8}

\par \textit{"الرياضيون الصالحون" المشهورون. هنا تنتهي قصة الشجاعة المسماة بسفر المكابيين الرابع.}

\par 1 أعلن بعض الحراس أنه عندما كانوا على وشك القبض عليها وإعدامها، ألقت بنفسها على المحرقة حتى لا يتمكن أي رجل من لمس جسدها

\par 2 يا أمّي، يا من حطمتِ مع أبنائك السبعة قوة الطاغية، وأبطلتِ مكائده الشريرة، وأعطيتِ مثالاً على نبل الإيمان

\par 3 لقد وُضِعتَ بنبلٍ كسقفٍ على أبنائك كأعمدة، ولم يهزك زلزال العذاب إطلاقًا

\par 4 افرحي إذن، أيتها الأم الطاهرة النفس، إذ أن رجاء صبركِ أكيد لدى الله

\par 5 ليس القمر مهيبًا وسط النجوم في السماء كما أنت، بعد أن أنرت طريق أبنائك السبعة الشبيهين بالنجوم إلى البر، تقف في شرف أمام الله؛ وأنت جالس في السماء معهم

\par 6 لأن ولادتك كانت من ابن إبراهيم.

\par 7 ولو كان من المشروع لنا أن نرسم، كما قد يفعل بعض الفنانين، قصة تقواك، ألن يرتعد المشاهدون عندما يرون أمًا لسبعة أبناء تعاني من عذابات لا تعد ولا تحصى حتى الموت من أجل البر؟

\par 8 وكان من المناسب بالفعل نقش هذه الكلمات فوق مثواهم، متحدثةً عن نصب تذكاري للأجيال القادمة من شعبنا:

\par هنا يرقد كاهن مسن
\par وامرأة مليئة بالسنوات
\par وأبناؤها السبعة
\par من خلال عنف الطاغية
\par الرغبة في تدمير الأمة العبرية.
\par لقد دافعوا عن حقوق شعبنا
\par النظر إلى الله والصبر
\par العذابات حتى
\par الموت.

\par 9 لأنها كانت حقًا حربًا مقدسة خاضوها. ففي ذلك اليوم، امتحنتهم الفضيلة بالصبر، ووضعوا أمامهم جائزة النصر في عدم الفساد في الحياة الأبدية

\par 10 لكن أليعازار كان أول من قاتل، ولعبت أم الأبناء السبعة دورها، وقاتل الإخوة

\par 11 وكان الطاغية عدوهم، وكان العالم وحياة الإنسان متفرجين.

\par 12 وانتصر البر على الغالب وأعطى الإكليل لرياضييه. فمن ذا الذي تعجب من رياضيي الشريعة الحقة؟

\par 13 من لم يُعجب بهم؟ لقد أُعجب الطاغية نفسه ومجلسه بأكمله بصمودهم، حيث يقفون الآن بجانب عرش الله ويعيشون العصر المبارك

\par 14 لأن موسى يقول: «وجميع الذين قدسوا أنفسهم هم تحت يديك».

\par 15 ولذلك، بعد أن قدس هؤلاء الرجال أنفسهم من أجل الله، لم ينالوا هذا الشرف فحسب، بل نالوا أيضًا الشرف بأنه من خلالهم لم يعد للعدو أي سلطة على شعبنا، وأن الطاغية عانى من العقاب، وأن بلادنا طُهِّرت، إذ أصبحوا كما لو كانوا فدية عن خطيئة أمتنا؛ ومن خلال دماء هؤلاء الرجال الصالحين وكفارة موتهم، أنقذت العناية الإلهية إسرائيل التي كانت تُعامل معاملة سيئة من قبل

\par 16 لأنه عندما رأى الطاغية أنطيوخس بطولة فضيلتهم، وصبرهم تحت التعذيب، رفع صبرهم علنًا أمام جنوده كمثال؛ وهكذا ألهم رجاله شعورًا بالشرف والبطولة في ساحة المعركة وفي أعمال الحصار، حتى أنه نهب وهزم جميع أعدائه

\par 17 يا بني إسرائيل، يا أبناء نسل إبراهيم، أطيعوا هذه الشريعة، وكونوا صالحين في كل شيء، مدركين أن العقل الموحى به هو سيد الأهواء والآلام، ليس فقط من داخلنا، بل من خارجنا أيضًا؛ وبهذه الطريقة، فإن هؤلاء الرجال، الذين سلموا أجسادهم للعذاب من أجل البر، لم يكتسبوا إعجاب البشرية فحسب، بل اعتُبروا جديرين بميراث إلهي

\par 18 ومن خلالهم، حصل الشعب على السلام، وأعاد مراعاة القانون في بلدنا، واستعاد المدينة من العدو

\par 19 وقد لاحق الانتقام الطاغية أنطيوخس على الأرض، وفي الموت يعاني العقاب

\par 20 لأنه عندما فشل تمامًا في إجبار شعب أورشليم على العيش مثل الأمم والتخلي عن عادات آبائنا، غادر أورشليم عندئذٍ وسار بعيدًا ضد الفرس

\par 21 وهذه هي الكلمات التي قالتها أم الأبناء السبعة، المرأة الصالحة، لأولادها:

\par 22 «كنتُ عذراءً طاهرة، ولم أضل عن بيت أبي، وحافظتُ على الضلع الذي بُني في حواء.»

\par 23 «لم يُفسدني مُغوٍ في البرية، ولا مُخادع في الحقل؛ ولم يُدنّس الثعبان الكاذب المُضلّل نقاءَ عذريتي؛ لقد عشتُ مع زوجي كل أيام شبابي؛ ولكن عندما كبر أبنائي، مات والدهم.»

\par 24 «كان سعيدًا؛ لأنه عاش حياة مباركة بالأطفال، ولم يعرف أبدًا ألم فقدانهم.»

\par 25 «الذي كان بعد عندنا يعلمكم الناموس والأنبياء، وقرأ لنا قصة هابيل الذي قتله قابيل، وإسحاق الذي قُدِّم محرقة، ويوسف في السجن».

\par 26 «وحدثنا عن فينحاس الكاهن الغيور، وعلمكم ترنيمة حننيا وعزريا وميشائيل في النار».

\par 27 «ومجد دانيال أيضًا في جب الأسود وباركه، وأعاد إلى أذهانكم قول إشعياء،»

\par 28 «نعم، حتى لو مررت بالنار، فلن يؤذيك اللهب.»

\par 29 لقد غنى لنا كلمات داود صاحب المزمور: "كثيرة هي بلايا الصديقين".

\par 30 «لقد اقتبس لنا مثل سليمان، «هو شجرة حياة لكل من يفعل مشيئته».»

\par 31 «وأكد كلام حزقيال: «هل تحيا هذه العظام اليابسة؟» لأنه لم ينس النشيد الذي علمه موسى، الذي يقول: «سأقتل وأحيي. هذه هي حياتك وبركة أيامك».

\par 32 آه، كم كان يومًا قاسيًا، ومع ذلك لم يكن قاسيًا، عندما أشعل طاغية الإغريق القاسي النار لمجمراته البربرية، ومع غليان مشاعره، جلب إلى المنجنيق وأعاد أبناء ابنة إبراهيم السبعة إلى عذاباته، وأعمى مقل عيونهم، وقطع ألسنتهم، وقتلهم بأنواع عديدة من العذاب

\par 33 ولهذا السبب لاحقت دينونة الله، وستلاحق، ذلك الوغد الملعون

\par 34 لكن أبناء إبراهيم، مع أمهم المنتصرة، جُمعوا إلى مكان أجدادهم، بعد أن نالوا أرواحًا طاهرة وخالدة من الله، الذي له المجد إلى أبد الآبدين


\end{document}