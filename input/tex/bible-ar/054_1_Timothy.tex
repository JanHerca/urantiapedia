\begin{document}

\title{تيموثاوس1}


\chapter{1}

\par 1 بُولُسُ، رَسُولُ يَسُوعَ الْمَسِيحِ، بِحَسَبِ أَمْرِ اللهِ مُخَلِّصِنَا وَرَبِّنَا يَسُوعَ الْمَسِيحِ، رَجَائِنَا.
\par 2 إِلَى تِيمُوثَاوُسَ، الاِبْنِ الصَّرِيحِ فِي الإِيمَانِ. نِعْمَةٌ وَرَحْمَةٌ وَسَلاَمٌ مِنَ اللهِ أَبِينَا وَالْمَسِيحِ يَسُوعَ رَبِّنَا.
\par 3 كَمَا طَلَبْتُ إِلَيْكَ أَنْ تَمْكُثَ فِي أَفَسُسَ، إِذْ كُنْتُ أَنَا ذَاهِباً إِلَى مَكِدُونِيَّةَ، لِكَيْ تُوصِيَ قَوْماً أَنْ لاَ يُعَلِّمُوا تَعْلِيماً آخَرَ،
\par 4 وَلاَ يُصْغُوا إِلَى خُرَافَاتٍ وَأَنْسَابٍ لاَ حَدَّ لَهَا، تُسَبِّبُ مُبَاحَثَاتٍ دُونَ بُنْيَانِ اللهِ الَّذِي فِي الإِيمَانِ.
\par 5 وَأَمَّا غَايَةُ الْوَصِيَّةِ فَهِيَ الْمَحَبَّةُ مِنْ قَلْبٍ طَاهِرٍ، وَضَمِيرٍ صَالِحٍ، وَإِيمَانٍ بِلاَ رِيَاءٍ.
\par 6 الأُمُورُ الَّتِي إِذْ زَاغَ قَوْمٌ عَنْهَا انْحَرَفُوا إِلَى كَلاَمٍ بَاطِلٍ.
\par 7 يُرِيدُونَ أَنْ يَكُونُوا مُعَلِّمِي النَّامُوسِ، وَهُمْ لاَ يَفْهَمُونَ مَا يَقُولُونَ وَلاَ مَا يُقَرِّرُونَهُ.
\par 8 وَلَكِنَّنَا نَعْلَمُ أَنَّ النَّامُوسَ صَالِحٌ، إِنْ كَانَ أَحَدٌ يَسْتَعْمِلُهُ نَامُوسِيّاً.
\par 9 عَالِماً هَذَا: أَنَّ النَّامُوسَ لَمْ يُوضَعْ لِلْبَارِّ، بَلْ لِلأَثَمَةِ وَالْمُتَمَرِّدِينَ، لِلْفُجَّارِ وَالْخُطَاةِ، لِلدَّنِسِينَ وَالْمُسْتَبِيحِينَ، لِقَاتِلِي الآبَاءِ وَقَاتِلِي الأُمَّهَاتِ، لِقَاتِلِي النَّاسِ،
\par 10 لِلزُّنَاةِ، لِمُضَاجِعِي الذُّكُورِ، لِسَارِقِي النَّاسِ، لِلْكَذَّابِينَ، لِلْحَانِثِينَ، وَإِنْ كَانَ شَيْءٌ آخَرُ يُقَاوِمُ التَّعْلِيمَ الصَّحِيحَ،
\par 11 حَسَبَ إِنْجِيلِ مَجْدِ اللهِ الْمُبَارَكِ الَّذِي اؤْتُمِنْتُ أَنَا عَلَيْهِ.
\par 12 وَأَنَا أَشْكُرُ الْمَسِيحَ يَسُوعَ رَبَّنَا الَّذِي قَوَّانِي، أَنَّهُ حَسِبَنِي أَمِيناً، إِذْ جَعَلَنِي لِلْخِدْمَةِ،
\par 13 أَنَا الَّذِي كُنْتُ قَبْلاً مُجَدِّفاً وَمُضْطَهِداً وَمُفْتَرِياً. وَلَكِنَّنِي رُحِمْتُ، لأَنِّي فَعَلْتُ بِجَهْلٍ فِي عَدَمِ إِيمَانٍ.
\par 14 وَتَفَاضَلَتْ نِعْمَةُ رَبِّنَا جِدّاً مَعَ الإِيمَانِ وَالْمَحَبَّةِ الَّتِي فِي الْمَسِيحِ يَسُوعَ.
\par 15 صَادِقَةٌ هِيَ الْكَلِمَةُ وَمُسْتَحِقَّةٌ كُلَّ قُبُولٍ: أَنَّ الْمَسِيحَ يَسُوعَ جَاءَ إِلَى الْعَالَمِ لِيُخَلِّصَ الْخُطَاةَ الَّذِينَ أَوَّلُهُمْ أَنَا.
\par 16 لَكِنَّنِي لِهَذَا رُحِمْتُ: لِيُظْهِرَ يَسُوعُ الْمَسِيحُ فِيَّ أَنَا أَوَّلاً كُلَّ أَنَاةٍ، مِثَالاً لِلْعَتِيدِينَ أَنْ يُؤْمِنُوا بِهِ لِلْحَيَاةِ الأَبَدِيَّةِ.
\par 17 وَمَلِكُ الدُّهُورِ الَّذِي لاَ يَفْنَى وَلاَ يُرَى، الإِلَهُ الْحَكِيمُ وَحْدَهُ، لَهُ الْكَرَامَةُ وَالْمَجْدُ إِلَى دَهْرِ الدُّهُورِ. آمِينَ.
\par 18 هَذِهِ الْوَصِيَّةُ أَيُّهَا الاِبْنُ تِيمُوثَاوُسُ أَسْتَوْدِعُكَ إِيَّاهَا حَسَبَ النُّبُوَّاتِ الَّتِي سَبَقَتْ عَلَيْكَ، لِكَيْ تُحَارِبَ فِيهَا الْمُحَارَبَةَ الْحَسَنَةَ،
\par 19 وَلَكَ إِيمَانٌ وَضَمِيرٌ صَالِحٌ، الَّذِي إِذْ رَفَضَهُ قَوْمٌ انْكَسَرَتْ بِهِمِ السَّفِينَةُ مِنْ جِهَةِ الإِيمَانِ أَيْضاً،
\par 20 الَّذِينَ مِنْهُمْ هِيمِينَايُسُ وَالإِسْكَنْدَرُ، اللَّذَانِ أَسْلَمْتُهُمَا لِلشَّيْطَانِ لِكَيْ يُؤَدَّبَا حَتَّى لاَ يُجَدِّفَا.

\chapter{2}

\par 1 فَأَطْلُبُ أَوَّلَ كُلِّ شَيْءٍ أَنْ تُقَامَ طِلْبَاتٌ وَصَلَوَاتٌ وَابْتِهَالاَتٌ وَتَشَكُّرَاتٌ لأَجْلِ جَمِيعِ النَّاسِ،
\par 2 لأَجْلِ الْمُلُوكِ وَجَمِيعِ الَّذِينَ هُمْ فِي مَنْصِبٍ، لِكَيْ نَقْضِيَ حَيَاةً مُطْمَئِنَّةً هَادِئَةً فِي كُلِّ تَقْوَى وَوَقَارٍ،
\par 3 لأَنَّ هَذَا حَسَنٌ وَمَقْبُولٌ لَدَى مُخَلِّصِنَا اللهِ،
\par 4 الَّذِي يُرِيدُ أَنَّ جَمِيعَ النَّاسِ يَخْلُصُونَ وَإِلَى مَعْرِفَةِ الْحَقِّ يُقْبِلُونَ.
\par 5 لأَنَّهُ يُوجَدُ إِلَهٌ وَاحِدٌ وَوَسِيطٌ وَاحِدٌ بَيْنَ اللهِ وَالنَّاسِ: الإِنْسَانُ يَسُوعُ الْمَسِيحُ،
\par 6 الَّذِي بَذَلَ نَفْسَهُ فِدْيَةً لأَجْلِ الْجَمِيعِ، الشَّهَادَةُ فِي أَوْقَاتِهَا الْخَاصَّةِ،
\par 7 الَّتِي جُعِلْتُ أَنَا لَهَا كَارِزاً وَرَسُولاً. الْحَقَّ أَقُولُ فِي الْمَسِيحِ وَلاَ أَكْذِبُ، مُعَلِّماً لِلأُمَمِ فِي الإِيمَانِ وَالْحَقِّ.
\par 8 فَأُرِيدُ أَنْ يُصَلِّيَ الرِّجَالُ فِي كُلِّ مَكَانٍ رَافِعِينَ أَيَادِيَ طَاهِرَةً، بِدُونِ غَضَبٍ وَلاَ جِدَالٍ.
\par 9 وَكَذَلِكَ أَنَّ النِّسَاءَ يُزَيِّنَّ ذَوَاتِهِنَّ بِلِبَاسِ الْحِشْمَةِ مَعَ وَرَعٍ وَتَعَقُّلٍ، لاَ بِضَفَائِرَ أَوْ ذَهَبٍ أَوْ لَآلِئَ أَوْ مَلاَبِسَ كَثِيرَةِ الثَّمَنِ،
\par 10 بَلْ كَمَا يَلِيقُ بِنِسَاءٍ مُتَعَاهِدَاتٍ بِتَقْوَى اللهِ بِأَعْمَالٍ صَالِحَةٍ.
\par 11 لِتَتَعَلَّمِ الْمَرْأَةُ بِسُكُوتٍ فِي كُلِّ خُضُوعٍ.
\par 12 وَلَكِنْ لَسْتُ آذَنُ لِلْمَرْأَةِ أَنْ تُعَلِّمَ وَلاَ تَتَسَلَّطَ عَلَى الرَّجُلِ، بَلْ تَكُونُ فِي سُكُوتٍ،
\par 13 لأَنَّ آدَمَ جُبِلَ أَوَّلاً ثُمَّ حَوَّاءُ،
\par 14 وَآدَمُ لَمْ يُغْوَ لَكِنَّ الْمَرْأَةَ أُغْوِيَتْ فَحَصَلَتْ فِي التَّعَدِّي،
\par 15 وَلَكِنَّهَا سَتَخْلُصُ بِوِلاَدَةِ الأَوْلاَدِ، إِنْ ثَبَتْنَ فِي الإِيمَانِ وَالْمَحَبَّةِ وَالْقَدَاسَةِ مَعَ التَّعَقُّلِ.

\chapter{3}

\par 1 صَادِقَةٌ هِيَ الْكَلِمَةُ: إِنِ ابْتَغَى أَحَدٌ الأُسْقُفِيَّةَ فَيَشْتَهِي عَمَلاً صَالِحاً.
\par 2 فَيَجِبُ أَنْ يَكُونَ الأُسْقُفُ بِلاَ لَوْمٍ، بَعْلَ امْرَأَةٍ وَاحِدَةٍ، صَاحِياً، عَاقِلاً، مُحْتَشِماً، مُضِيفاً لِلْغُرَبَاءِ، صَالِحاً لِلتَّعْلِيمِ،
\par 3 غَيْرَ مُدْمِنِ الْخَمْرِ، وَلاَ ضَرَّابٍ، وَلاَ طَامِعٍ بِالرِّبْحِ الْقَبِيحِ، بَلْ حَلِيماً، غَيْرَ مُخَاصِمٍ، وَلاَ مُحِبٍّ لِلْمَالِ،
\par 4 يُدَبِّرُ بَيْتَهُ حَسَناً، لَهُ أَوْلاَدٌ فِي الْخُضُوعِ بِكُلِّ وَقَارٍ.
\par 5 وَإِنَّمَا إِنْ كَانَ أَحَدٌ لاَ يَعْرِفُ أَنْ يُدَبِّرَ بَيْتَهُ، فَكَيْفَ يَعْتَنِي بِكَنِيسَةِ اللهِ؟
\par 6 غَيْرَ حَدِيثِ الإِيمَانِ لِئَلاَّ يَتَصَلَّفَ فَيَسْقُطَ فِي دَيْنُونَةِ إِبْلِيسَ.
\par 7 وَيَجِبُ أَيْضاً أَنْ تَكُونَ لَهُ شَهَادَةٌ حَسَنَةٌ مِنَ الَّذِينَ هُمْ مِنْ خَارِجٍ، لِئَلاَّ يَسْقُطَ فِي تَعْيِيرٍ وَفَخِّ إِبْلِيسَ.
\par 8 كَذَلِكَ يَجِبُ أَنْ يَكُونَ الشَّمَامِسَةُ ذَوِي وَقَارٍ، لاَ ذَوِي لِسَانَيْنِ، غَيْرَ مُولَعِينَ بِالْخَمْرِ الْكَثِيرِ، وَلاَ طَامِعِينَ بِالرِّبْحِ الْقَبِيحِ،
\par 9 وَلَهُمْ سِرُّ الإِيمَانِ بِضَمِيرٍ طَاهِرٍ.
\par 10 وَإِنَّمَا هَؤُلاَءِ أَيْضاً لِيُخْتَبَرُوا أَوَّلاً، ثُمَّ يَتَشَمَّسُوا إِنْ كَانُوا بِلاَ لَوْمٍ.
\par 11 كَذَلِكَ يَجِبُ أَنْ تَكُونَ النِّسَاءُ ذَوَاتِ وَقَارٍ، غَيْرَ ثَالِبَاتٍ، صَاحِيَاتٍ، أَمِينَاتٍ فِي كُلِّ شَيْءٍ.
\par 12 لِيَكُنِ الشَّمَامِسَةُ كُلٌّ بَعْلَ امْرَأَةٍ وَاحِدَةٍ، مُدَبِّرِينَ أَوْلاَدَهُمْ وَبُيُوتَهُمْ حَسَناً،
\par 13 لأَنَّ الَّذِينَ تَشَمَّسُوا حَسَناً يَقْتَنُونَ لأَنْفُسِهِمْ دَرَجَةً حَسَنَةً وَثِقَةً كَثِيرَةً فِي الإِيمَانِ الَّذِي بِالْمَسِيحِ يَسُوعَ.
\par 14 هَذَا أَكْتُبُهُ إِلَيْكَ رَاجِياً أَنْ آتِيَ إِلَيْكَ عَنْ قَرِيبٍ.
\par 15 وَلَكِنْ إِنْ كُنْتُ أُبْطِئُ فَلِكَيْ تَعْلَمَ كَيْفَ يَجِبُ أَنْ تَتَصَرَّفَ فِي بَيْتِ اللهِ، الَّذِي هُوَ كَنِيسَةُ اللهِ الْحَيِّ، عَمُودُ الْحَقِّ وَقَاعِدَتُهُ.
\par 16 وَبِالإِجْمَاعِ عَظِيمٌ هُوَ سِرُّ التَّقْوَى: اللهُ ظَهَرَ فِي الْجَسَدِ، تَبَرَّرَ فِي الرُّوحِ، تَرَاءَى لِمَلاَئِكَةٍ، كُرِزَ بِهِ بَيْنَ الأُمَمِ، أُومِنَ بِهِ فِي الْعَالَمِ، رُفِعَ فِي الْمَجْدِ.

\chapter{4}

\par 1 وَلَكِنَّ الرُّوحَ يَقُولُ صَرِيحاً: إِنَّهُ فِي الأَزْمِنَةِ الأَخِيرَةِ يَرْتَدُّ قَوْمٌ عَنِ الإِيمَانِ، تَابِعِينَ أَرْوَاحاً مُضِلَّةً وَتَعَالِيمَ شَيَاطِينَ،
\par 2 فِي رِيَاءِ أَقْوَالٍ كَاذِبَةٍ، مَوْسُومَةً ضَمَائِرُهُمْ،
\par 3 مَانِعِينَ عَنِ الزِّوَاجِ، وَآمِرِينَ أَنْ يُمْتَنَعَ عَنْ أَطْعِمَةٍ قَدْ خَلَقَهَا اللهُ لِتُتَنَاوَلَ بِالشُّكْرِ مِنَ الْمُؤْمِنِينَ وَعَارِفِي الْحَقِّ.
\par 4 لأَنَّ كُلَّ خَلِيقَةِ اللهِ جَيِّدَةٌ، وَلاَ يُرْفَضُ شَيْءٌ إِذَا أُخِذَ مَعَ الشُّكْرِ،
\par 5 لأَنَّهُ يُقَدَّسُ بِكَلِمَةِ اللهِ وَالصَّلاَةِ.
\par 6 إِنْ فَكَّرْتَ الإِخْوَةَ بِهَذَا تَكُونُ خَادِماً صَالِحاً لِيَسُوعَ الْمَسِيحِ، مُتَرَبِّياً بِكَلاَمِ الإِيمَانِ وَالتَّعْلِيمِ الْحَسَنِ الَّذِي تَتَبَّعْتَهُ.
\par 7 وَأَمَّا الْخُرَافَاتُ الدَّنِسَةُ الْعَجَائِزِيَّةُ فَارْفُضْهَا، وَرَوِّضْ نَفْسَكَ لِلتَّقْوَى.
\par 8 لأَنَّ الرِّيَاضَةَ الْجَسَدِيَّةَ نَافِعَةٌ لِقَلِيلٍ، وَلَكِنَّ التَّقْوَى نَافِعَةٌ لِكُلِّ شَيْءٍ، إِذْ لَهَا مَوْعِدُ الْحَيَاةِ الْحَاضِرَةِ وَالْعَتِيدَةِ.
\par 9 صَادِقَةٌ هِيَ الْكَلِمَةُ وَمُسْتَحِقَّةٌ كُلَّ قُبُولٍ.
\par 10 لأَنَّنَا لِهَذَا نَتْعَبُ وَنُعَيَّرُ، لأَنَّنَا قَدْ أَلْقَيْنَا رَجَاءَنَا عَلَى اللهِ الْحَيِّ، الَّذِي هُوَ مُخَلِّصُ جَمِيعِ النَّاسِ وَلاَ سِيَّمَا الْمُؤْمِنِينَ.
\par 11 أَوْصِ بِهَذَا وَعَلِّمْ.
\par 12 لاَ يَسْتَهِنْ أَحَدٌ بِحَدَاثَتِكَ، بَلْ كُنْ قُدْوَةً لِلْمُؤْمِنِينَ فِي الْكَلاَمِ، فِي التَّصَرُّفِ، فِي الْمَحَبَّةِ، فِي الرُّوحِ، فِي الإِيمَانِ، فِي الطَّهَارَةِ.
\par 13 إِلَى أَنْ أَجِيءَ اعْكُفْ عَلَى الْقِرَاءَةِ وَالْوَعْظِ وَالتَّعْلِيمِ.
\par 14 لاَ تُهْمِلِ الْمَوْهِبَةَ الَّتِي فِيكَ الْمُعْطَاةَ لَكَ بِالنُّبُوَّةِ مَعَ وَضْعِ أَيْدِي الْمَشْيَخَةِ.
\par 15 اهْتَمَّ بِهَذَا. كُنْ فِيهِ، لِكَيْ يَكُونَ تَقَدُّمُكَ ظَاهِراً فِي كُلِّ شَيْءٍ.
\par 16 لاَحِظْ نَفْسَكَ وَالتَّعْلِيمَ وَدَاوِمْ عَلَى ذَلِكَ، لأَنَّكَ إِذَا فَعَلْتَ هَذَا تُخَلِّصُ نَفْسَكَ وَالَّذِينَ يَسْمَعُونَكَ أَيْضاً.

\chapter{5}

\par 1 لاَ تَزْجُرْ شَيْخاً بَلْ عِظْهُ كَأَبٍ، وَالأَحْدَاثَ كَإِخْوَةٍ،
\par 2 وَالْعَجَائِزَ كَأُمَّهَاتٍ، وَالْحَدَثَاتِ كَأَخَوَاتٍ، بِكُلِّ طَهَارَةٍ.
\par 3 أَكْرِمِ الأَرَامِلَ اللَّوَاتِي هُنَّ بِالْحَقِيقَةِ أَرَامِلُ.
\par 4 وَلَكِنْ إِنْ كَانَتْ أَرْمَلَةٌ لَهَا أَوْلاَدٌ أَوْ حَفَدَةٌ، فَلْيَتَعَلَّمُوا أَوَّلاً أَنْ يُوَقِّرُوا أَهْلَ بَيْتِهِمْ وَيُوفُوا وَالِدِيهِمِ الْمُكَافَأَةَ، لأَنَّ هَذَا صَالِحٌ وَمَقْبُولٌ أَمَامَ اللهِ.
\par 5 وَلَكِنَّ الَّتِي هِيَ بِالْحَقِيقَةِ أَرْمَلَةٌ وَوَحِيدَةٌ، فَقَدْ أَلْقَتْ رَجَاءَهَا عَلَى اللهِ، وَهِيَ تُواظِبُ عَلَى الطِّلْبَاتِ وَالصَّلَوَاتِ لَيْلاً وَنَهَاراً.
\par 6 وَأَمَّا الْمُتَنَعِّمَةُ فَقَدْ مَاتَتْ وَهِيَ حَيَّةٌ.
\par 7 فَأَوْصِ بِهَذَا لِكَيْ يَكُنَّ بِلاَ لَوْمٍ.
\par 8 وَإِنْ كَانَ أَحَدٌ لاَ يَعْتَنِي بِخَاصَّتِهِ، وَلاَ سِيَّمَا أَهْلُ بَيْتِهِ، فَقَدْ أَنْكَرَ الإِيمَانَ، وَهُوَ شَرٌّ مِنْ غَيْرِ الْمُؤْمِنِ.
\par 9 لِتُكْتَتَبْ أَرْمَلَةٌ إِنْ لَمْ يَكُنْ عُمْرُهَا أَقَلَّ مِنْ سِتِّينَ سَنَةً، امْرَأَةَ رَجُلٍ وَاحِدٍ،
\par 10 مَشْهُوداً لَهَا فِي أَعْمَالٍ صَالِحَةٍ، إِنْ تَكُنْ قَدْ رَبَّتِ الأَوْلاَدَ، أَضَافَتِ الْغُرَبَاءَ، غَسَّلَتْ أَرْجُلَ الْقِدِّيسِينَ، سَاعَدَتِ الْمُتَضَايِقِينَ، اتَّبَعَتْ كُلَّ عَمَلٍ صَالِحٍ.
\par 11 أَمَّا الأَرَامِلُ الْحَدَثَاتُ فَارْفُضْهُنَّ، لأَنَّهُنَّ مَتَى بَطِرْنَ عَلَى الْمَسِيحِ يُرِدْنَ أَنْ يَتَزَوَّجْنَ،
\par 12 وَلَهُنَّ دَيْنُونَةٌ لأَنَّهُنَّ رَفَضْنَ الإِيمَانَ الأَوَّلَ.
\par 13 وَمَعَ ذَلِكَ أَيْضاً يَتَعَلَّمْنَ أَنْ يَكُنَّ بَطَّالاَتٍ، يَطُفْنَ فِي الْبُيُوتِ. وَلَسْنَ بَطَّالاَتٍ فَقَطْ بَلْ مِهْذَارَاتٌ أَيْضاً، وَفُضُولِيَّاتٌ، يَتَكَلَّمْنَ بِمَا لاَ يَجِبُ.
\par 14 فَأُرِيدُ أَنَّ الْحَدَثَاتِ يَتَزَوَّجْنَ وَيَلِدْنَ الأَوْلاَدَ وَيُدَبِّرْنَ الْبُيُوتَ، وَلاَ يُعْطِينَ عِلَّةً لِلْمُقَاوِمِ مِنْ أَجْلِ الشَّتْمِ.
\par 15 فَإِنَّ بَعْضَهُنَّ قَدِ انْحَرَفْنَ وَرَاءَ الشَّيْطَانِ.
\par 16 إِنْ كَانَ لِمُؤْمِنٍ أَوْ مُؤْمِنَةٍ أَرَامِلُ فَلْيُسَاعِدْهُنَّ وَلاَ يُثَقَّلْ عَلَى الْكَنِيسَةِ، لِكَيْ تُسَاعِدَ هِيَ اللَّوَاتِي هُنَّ بِالْحَقِيقَةِ أَرَامِلُ.
\par 17 أَمَّا الشُّيُوخُ الْمُدَبِّرُونَ حَسَناً فَلْيُحْسَبُوا أَهْلاً لِكَرَامَةٍ مُضَاعَفَةٍ، وَلاَ سِيَّمَا الَّذِينَ يَتْعَبُونَ فِي الْكَلِمَةِ وَالتَّعْلِيمِ،
\par 18 لأَنَّ الْكِتَابَ يَقُولُ: «لاَ تَكُمَّ ثَوْراً دَارِساً، وَالْفَاعِلُ مُسْتَحِقٌّ أُجْرَتَهُ».
\par 19 لاَ تَقْبَلْ شِكَايَةً عَلَى شَيْخٍ إِلاَّ عَلَى شَاهِدَيْنِ أَوْ ثَلاَثَةِ شُهُودٍ.
\par 20 اَلَّذِينَ يُخْطِئُونَ وَبِّخْهُمْ أَمَامَ الْجَمِيعِ لِكَيْ يَكُونَ عِنْدَ الْبَاقِينَ خَوْفٌ.
\par 21 أُنَاشِدُكَ أَمَامَ اللهِ وَالرَّبِّ يَسُوعَ الْمَسِيحِ وَالْمَلاَئِكَةِ الْمُخْتَارِينَ أَنْ تَحْفَظَ هَذَا بِدُونِ غَرَضٍ، وَلاَ تَعْمَلَ شَيْئاً بِمُحَابَاةٍ.
\par 22 لاَ تَضَعْ يَداً عَلَى أَحَدٍ بِالْعَجَلَةِ، وَلاَ تَشْتَرِكْ فِي خَطَايَا الآخَرِينَ. احْفَظْ نَفْسَكَ طَاهِراً.
\par 23 لاَ تَكُنْ فِي مَا بَعْدُ شَرَّابَ مَاءٍ، بَلِ اسْتَعْمِلْ خَمْراً قَلِيلاً مِنْ أَجْلِ مَعِدَتِكَ وَأَسْقَامِكَ الْكَثِيرَةِ.
\par 24 خَطَايَا بَعْضِ النَّاسِ وَاضِحَةٌ تَتَقَدَّمُ إِلَى الْقَضَاءِ، وَأَمَّا الْبَعْضُ فَتَتْبَعُهُمْ.
\par 25 كَذَلِكَ أَيْضاً الأَعْمَالُ الصَّالِحَةُ وَاضِحَةٌ، وَالَّتِي هِيَ خِلاَفُ ذَلِكَ لاَ يُمْكِنُ أَنْ تُخْفى.

\chapter{6}

\par 1 جَمِيعُ الَّذِينَ هُمْ عَبِيدٌ تَحْتَ نِيرٍ فَلْيَحْسِبُوا سَادَتَهُمْ مُسْتَحِقِّينَ كُلَّ إِكْرَامٍ، لِئَلاَّ يُفْتَرَى عَلَى اسْمِ اللهِ وَتَعْلِيمِهِ.
\par 2 وَالَّذِينَ لَهُمْ سَادَةٌ مُؤْمِنُونَ لاَ يَسْتَهِينُوا بِهِمْ لأَنَّهُمْ إِخْوَةٌ، بَلْ لِيَخْدِمُوهُمْ أَكْثَرَ، لأَنَّ الَّذِينَ يَتَشَارَكُونَ فِي الْفَائِدَةِ هُمْ مُؤْمِنُونَ وَمَحْبُوبُونَ. عَلِّمْ وَعِظْ بِهَذَا.
\par 3 إِنْ كَانَ أَحَدٌ يُعَلِّمُ تَعْلِيماً آخَرَ، وَلاَ يُوافِقُ كَلِمَاتِ رَبِّنَا يَسُوعَ الْمَسِيحِ الصَّحِيحَةَ، وَالتَّعْلِيمَ الَّذِي هُوَ حَسَبَ التَّقْوَى
\par 4 فَقَدْ تَصَلَّفَ، وَهُوَ لاَ يَفْهَمُ شَيْئاً، بَلْ هُوَ مُتَعَلِّلٌ بِمُبَاحَثَاتٍ وَمُمَاحَكَاتِ الْكَلاَمِ الَّتِي مِنْهَا يَحْصُلُ الْحَسَدُ وَالْخِصَامُ وَالاِفْتِرَاءُ وَالظُّنُونُ الرَّدِيَّةُ،
\par 5 وَمُنَازَعَاتُ أُنَاسٍ فَاسِدِي الذِّهْنِ وَعَادِمِي الْحَقِّ، يَظُنُّونَ أَنَّ التَّقْوَى تِجَارَةٌ. تَجَنَّبْ مِثْلَ هَؤُلاَءِ.
\par 6 وَأَمَّا التَّقْوَى مَعَ الْقَنَاعَةِ فَهِيَ تِجَارَةٌ عَظِيمَةٌ،
\par 7 لأَنَّنَا لَمْ نَدْخُلِ الْعَالَمَ بِشَيْءٍ، وَوَاضِحٌ أَنَّنَا لاَ نَقْدِرُ أَنْ نَخْرُجَ مِنْهُ بِشَيْءٍ.
\par 8 فَإِنْ كَانَ لَنَا قُوتٌ وَكِسْوَةٌ فَلْنَكْتَفِ بِهِمَا.
\par 9 وَأَمَّا الَّذِينَ يُرِيدُونَ أَنْ يَكُونُوا أَغْنِيَاءَ فَيَسْقُطُونَ فِي تَجْرِبَةٍ وَفَخٍّ وَشَهَوَاتٍ كَثِيرَةٍ غَبِيَّةٍ وَمُضِرَّةٍ تُغَرِّقُ النَّاسَ فِي الْعَطَبِ وَالْهَلاَكِ،
\par 10 لأَنَّ مَحَبَّةَ الْمَالِ أَصْلٌ لِكُلِّ الشُّرُورِ، الَّذِي إِذِ ابْتَغَاهُ قَوْمٌ ضَلُّوا عَنِ الإِيمَانِ، وَطَعَنُوا أَنْفُسَهُمْ بِأَوْجَاعٍ كَثِيرَةٍ.
\par 11 وَأَمَّا أَنْتَ يَا إِنْسَانَ اللهِ فَاهْرُبْ مِنْ هَذَا، وَاتْبَعِ الْبِرَّ وَالتَّقْوَى وَالإِيمَانَ وَالْمَحَبَّةَ وَالصَّبْرَ وَالْوَدَاعَةَ.
\par 12 جَاهِدْ جِهَادَ الإِيمَانِ الْحَسَنَ، وَأَمْسِكْ بِالْحَيَاةِ الأَبَدِيَّةِ الَّتِي إِلَيْهَا دُعِيتَ أَيْضاً، وَاعْتَرَفْتَ الاِعْتِرَافَ الْحَسَنَ أَمَامَ شُهُودٍ كَثِيرِينَ.
\par 13 أُوصِيكَ أَمَامَ اللهِ الَّذِي يُحْيِي الْكُلَّ وَالْمَسِيحِ يَسُوعَ الَّذِي شَهِدَ لَدَى بِيلاَطُسَ الْبُنْطِيِّ بِالاِعْتِرَافِ الْحَسَنِ:
\par 14 أَنْ تَحْفَظَ الْوَصِيَّةَ بِلاَ دَنَسٍ وَلاَ لَوْمٍ إِلَى ظُهُورِ رَبِّنَا يَسُوعَ الْمَسِيحِ،
\par 15 الَّذِي سَيُبَيِّنُهُ فِي أَوْقَاتِهِ الْمُبَارَكُ الْعَزِيزُ الْوَحِيدُ، مَلِكُ الْمُلُوكِ وَرَبُّ الأَرْبَابِ،
\par 16 الَّذِي وَحْدَهُ لَهُ عَدَمُ الْمَوْتِ، سَاكِناً فِي نُورٍ لاَ يُدْنَى مِنْهُ، الَّذِي لَمْ يَرَهُ أَحَدٌ مِنَ النَّاسِ وَلاَ يَقْدِرُ أَنْ يَرَاهُ، الَّذِي لَهُ الْكَرَامَةُ وَالْقُدْرَةُ الأَبَدِيَّةُ. آمِينَ.
\par 17 أَوْصِ الأَغْنِيَاءَ فِي الدَّهْرِ الْحَاضِرِ أَنْ لاَ يَسْتَكْبِرُوا، وَلاَ يُلْقُوا رَجَاءَهُمْ عَلَى غَيْرِ يَقِينِيَّةِ الْغِنَى، بَلْ عَلَى اللهِ الْحَيِّ الَّذِي يَمْنَحُنَا كُلَّ شَيْءٍ بِغِنًى لِلتَّمَتُّعِ.
\par 18 وَأَنْ يَصْنَعُوا صَلاَحاً، وَأَنْ يَكُونُوا أَغْنِيَاءَ فِي أَعْمَالٍ صَالِحَةٍ، وَأَنْ يَكُونُوا أَسْخِيَاءَ فِي الْعَطَاءِ كُرَمَاءَ فِي التَّوْزِيعِ،
\par 19 مُدَّخِرِينَ لأَنْفُسِهِمْ أَسَاساً حَسَناً لِلْمُسْتَقْبَِلِ، لِكَيْ يُمْسِكُوا بِالْحَيَاةِ الأَبَدِيَّةِ.
\par 20 يَا تِيمُوثَاوُسُ، احْفَظِ الْوَدِيعَةَ، مُعْرِضاً عَنِ الْكَلاَمِ الْبَاطِلِ الدَّنِسِ، وَمُخَالَفَاتِ الْعِلْمِ الْكَاذِبِ الاِسْمِ،
\par 21 الَّذِي إِذْ تَظَاهَرَ بِهِ قَوْمٌ زَاغُوا مِنْ جِهَةِ الإِيمَانِ.

\end{document}