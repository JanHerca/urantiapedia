\begin{document}

\title{فلسفة زائفة}

\chapter{1}

\par 1 بداية العالم. أنجب آدم ثلاثة أبناء وبنت واحدة، وهم قابيل، ونواب، وهابيل، وشيث.

\par 2 وعاش آدم بعد أن ولد شيثًا 700 سنة، وولد 12 ابنًا و8 بنات

\par 3 وهذه أسماء الذكور: أليشعيل، وسوريس، وعيلميئيل، وبرابل، ونعت، وزرمة، وزسام، ومعثل، وعناة

\par 4 وهؤلاء بناته: فوا، إيكتاس، أريبيكا، سيفا، تيسيا، سابا، آسين

\par 5 وعاش شيث 105 سنين وولد أنوش. وعاش شيث بعد أن ولد أنوش 707 سنين، وولد 3 أبناء وبنتين

\par 6 وهذه أسماء أبنائه: إيليديا، وفونا، وماثا، وابنتيه ماليدا وثيلا

\par 7 وعاش أنوش 180 سنة وولد قينان. وعاش أنوش بعد أن ولد قينان 715 سنة وولد ولدين وبنتًا

\par 8 وهذه أسماء أبنائه: فوي وثال، وابنته كاتيناث

\par 9 وعاش قينان 520 سنة وولد ملالِك. وعاش قينان بعدما ولد ملالِك 730 سنة وولد 3 بنين وبنتين

\par 10 وهذه أسماء الذكور: أثاخ، وسوكر، ولوفا، وأسماء البنات: آنا، وليوا

\par 11 وعاش ملالك 165 سنة وولد ياريث. وعاش ملالك بعدما ولد ياريث 730 سنة وولد 7 بنين و5 بنات

\par 12 وهذه أسماء الذكور: ليتا، ماثا، سيثار، ميلي، سوريال، لودو، أوثيم. وهذه أسماء البنات: آدا، ونوا، عيبال، مادا، سيلا

\par 13 وعاش يرث 172 سنة وولد حنوك. وعاش يرث بعد أن ولد حنوك 800 سنة وولد 4 أبناء وبنتين

\par 14 وهذه أسماء الذكور: ليد، وعناك، وسوبواك، وإيكتار، والبنات: تتزيكو، وليسي

\par 15 وعاش أخنوخ 165 سنة وولد متوشالام. وعاش أخنوخ بعد أن ولد متوشالام 200 سنة وولد 5 أبناء و3 بنات

\par 16 لكن أخنوخ أرضى الله في ذلك الوقت، فلم يُوجد، لأن الله نقله

\par 17 وأسماء أبنائه هي: عناز، زيوم، أخون، فليدي، إيليث؛ ومن بناته: ثعز، ليفيث، ليث

\par 18 وعاش متوشالام 187 سنة وولد لامك. وعاش متوشالام بعد أن ولد لامك 782 سنة، وولد ابنين وبنتين

\par 19 وهذه أسماء الذكور: إيناب ورافو؛ والبنات: ألوما وأموجا

\par 20 وعاش لامك 182 سنة وولد ابنًا، وسماه نوحًا على اسم ميلاده، قائلًا: هذا الطفل سيعطي راحة لنا وللأرض ممن فيها، الذين (أو في اليوم الذي) يُعاقبون فيه بسبب إثم أعمالهم الشريرة

\par 21 وعاش لامك بعد أن ولد نوحًا 585 سنة.

\par 22 وعاش نوح ثلاثمائة سنة وولد ثلاثة بنين ساماً وحاماً ويافث.

\chapter{2}

\par 1 وأما قابيل فسكن في الأرض مرتعدًا، كما أمره الله بعد أن قتل هابيل أخاه، وكان اسم امرأته ثيمك

\par 2 وعرف قابيل تيموش امرأته فحبلت وولدت حنوك.

\par 3 وكان قابيل ابن خمس عشرة سنة حين فعل هذه الأمور، ومن ذلك الوقت بدأ ببناء المدن حتى أسس سبع مدن. وهذه أسماء المدن: اسم المدينة الأولى على اسم ابنه حنوك، واسم المدينة الثانية مولي، واسم المدينة الثالثة ليث، واسم المدينة الرابعة تيزه، واسم المدينة الخامسة ييسكا، واسم المدينة السادسة سيليث، واسم المدينة السابعة ييبّات.

\par 4 وعاش قابيل بعدما ولد حنوك 715 سنة، وولد 3 بنين وبنتين. وهذه أسماء أبنائه: أولاد، وليزاف، وفصال، ومن بناته: قيثا، وماك. وكانت كل أيام قابيل 730 سنة، ومات

\par 5 ثم اتخذ حنوك زوجة من بنات شيث، فولدت له سيرام وكوث وماداب. وأما سيرام فولد متوشائيل، وموشائيل فولد لامك

\par 6 واتخذ لامك لنفسه امرأتين: اسم إحداهما عادا، واسم الأخرى سيلا

\par 7 فولدت له عادة يوباب، وكان أبًا لكل ساكني الخيام ورعاة الغنم. ثم ولدت له أيضًا يوبال، الذي كان أول من علم كل عزف على الآلات الموسيقية (حرفيًا: كل مزمور للأرغن).

\par 8 وفي ذلك الوقت، عندما بدأ سكان الأرض يفعلون الشر، كل واحد مع امرأة قريبه، ينجسهم، غضب الله. وبدأ يعزف على العود (kinnor) والقيثارة وعلى كل آلة موسيقية حلوة (حرفيًا: عزف المزامير)، ويفسد الأرض

\par 9 لكن سيلا أنجبت توبال وميسا وثيفا، وهذا هو توبال الذي أظهر للبشر فنون الرصاص والقصدير والحديد والنحاس والفضة والذهب: ثم بدأ سكان الأرض يصنعون تماثيل منحوتة ويعبدونها

\par 10 فقال لامك لامرأتيه عادا وسلة: اسمعا صوتي يا امرأتي لامك، وأنصتا إلى وصيتي، لأني أفسدت الناس لنفسي، ونزعت الرضعات من الثديين، لأُري أبنائي كيف يصنعون الشر، وسكان الأرض. والآن يُنتقم من قابيل سبعة أضعاف، ومن لامك سبعين مرة سبعة أضعاف

\chapter{3}

\par 1 وحدث لما بدأ الناس يتكاثرون على الأرض أنه وُلدت لهم بنات جميلات. ورأى أبناء الله بنات الناس أنهن جميلات جدًا، فاتخذوا لأنفسهم نساءً من كل ما اختاروه

\par 2 وقال الله: لا يدين روحي هؤلاء الناس إلى الأبد، لأنهم بشر، ولكن ستكون سنوهم 120. الذين وضع عليهم (أو حيث وضعت) أقاصي العالم، وفي أيديهم لم تُطفأ الشرور (أو لن يُطفأ الناموس).

\par 3 ورأى الله أن أعمال الشر قد تمت في جميع سكان الأرض، ولأن تفكيرهم كان في الإثم كل أيامهم، قال الله: سأمحو الإنسان وكل ما أزهر على الأرض، لأنه ندم على أني عملته

\par 4 لكن نوح وجد نعمة ورحمة لدى الرب، وهذه هي أجياله. نوح، الذي كان رجلاً بارًا وبلا دنس في جيله، أرضى الرب. الذي قال له الله: قد جاء وقت جميع البشر الذين يسكنون على الأرض، لأن أعمالهم شريرة جدًا. والآن اصنع لك تابوتًا من خشب الأرز، وهكذا تصنعه. يكون طوله ثلاثمائة ذراع، وعرضه ذراع، وارتفاعه ثلاثون ذراعًا. وتدخل الفلك أنت وامرأتك وبنوك ونساء بنيك معك. وأقطع عهدي معك لأهلك جميع سكان الأرض. والآن تأخذ من البهائم الطاهرة ومن طيور السماء الطاهرة سبعة سبعة ذكرًا وأنثى، لكي ينجو نسلهم على الأرض. وأما البهائم النجسة والطيور فتحصل لك اثنين اثنين ذكرا وأنثى وتصنع زادا لك ولها أيضا.

\par 5 ففعل نوح ما أمره به الله ودخل الفلك هو وجميع أبنائه معه. وكان بعد 7 أيام أن مياه الطوفان ابتدأت تنزل على الأرض. وفي ذلك اليوم انفتحت كل اللجج وانفتح ينبوع الماء العظيم وطاقات السماء، وكان مطر على الأرض 40 يومًا و40 ليلة

\par 6 وكان ذلك في السنة 1652 (1656) من الوقت الذي خلق فيه الله السماء والأرض، يوم فسدت الأرض وسكانها بسبب إثم أعمالهم

\par 7 ولما استمر الطوفان 140 يومًا على الأرض، لم يبقَ على قيد الحياة إلا نوح والذين معه في الفلك. وعندما ذكر الله نوحًا، قلّص الماء

\par 8 وفي اليوم التسعين جفف الله الأرض، وقال لنوح: اخرج من الفلك أنت وكل من معك، وانموا واكثروا على الأرض. فخرج نوح من الفلك هو وبنوه ونساء بنيه، وجميع الوحوش والدبابات والطيور والماشية التي أخرجها معه كما أمره الله. ثم بنى نوح مذبحًا للرب، وأخذ من جميع البهائم والطيور الطاهرة وأصعد محرقات على المذبح، فقبلها الرب رائحة راحة

\par 9 وقال الله: لن ألعن الأرض مرة أخرى من أجل الإنسان، لأن قناع قلب الإنسان قد زال منذ شبابه. ولذلك لن أهلك مرة أخرى كل الأحياء كما فعلت. ولكن عندما يخطئ سكان الأرض، سأدينهم بالمجاعة أو بالسيف أو بالنار أو بالوباء (حرفيًا الموت)، وستكون هناك زلازل، وسيُشتتون إلى أماكن غير مأهولة (أو ستُشتت أماكن سكنهم). لكنني لن أُفسد الأرض مرة أخرى بماء الطوفان، وفي كل أيام الأرض - وقت البذر والحصاد، البرد والحر، الصيف والخريف، النهار والليل - لن يزول، حتى أذكر سكان الأرض، حتى تتم الأزمنة

\par 10 ولكن عندما تكتمل سنو العالم، حينئذٍ يتوقف النور وتنطفئ الظلمة، وسأحيي الموتى وأقيم من الأرض النائمين، وسيدفع الجحيم دينه، وسيرد الهلاك ما أُسلم إليه، لأجازي كل إنسان حسب أعماله وثمرة خيالاته، حتى أحكم بين النفس والجسد. وسيستريح العالم، وسيُخمد الموت، وستغلق الجحيم فمه. ولن تكون الأرض بلا ولادة ولا قاحلة لساكنيها، ولن يتنجس أحد ممن تبرر فيّ. وستكون هناك أرض أخرى وسماء أخرى، مسكن أبدي.

\par 11 ثم كلم الرب نوحًا وبنيه قائلًا: ها أنا أقطع عهدي معكم ومع نسلكم من بعدكم، ولن أعود أُفسد الأرض بمياه الطوفان. وكل ما يدب فيها يكون لكم طعامًا. إلا أن لحمًا مع دم نفس لا تأكلوه. لأن من يسفك دم الإنسان يُسفك دمه، لأنه على صورة الله خُلق الإنسان. وأنتم انموا واكثروا واملأوا الأرض ككثرة السمك الذي يتكاثر في المياه. وقال الله: هذا هو العهد الذي قطعته بيني وبينكم، ويكون حينما أُغطي السماء بالسحاب أن قوسي يظهر في السحاب، فيكون تذكارًا للعهد بيني وبينكم وبين جميع سكان الأرض

\chapter{4}

\par 1 وكان بنو نوح الذين خرجوا من الفلك: سام وحام ويافث

\par 2 بنو يافث: جومر، ماجوج، وماداي، نيدياعزك، توبال، موكتراس، جنيز، ريفاث، وتوغرمة، إليسا، دسين، كيثين، تودانت

وبنو جومر: ثلز، ولود، ودبرت.

وبنو ماجوج: قيس، ثيبا، فروتة، عميئيل، فيمي، جلوزة، سمانخ.

\par وأبناء دودن: سلوس، فيلوكتا فاليتا.

\par وأبناء توبال: فاناتونوفا، أتيفا.

\par وبنو تيراس: معاك، تابل، بلانة، سمبلاميك، إيلاز.

\par وبنو ملك: أمبورادات، أوراش، بصرة.

\par وأبناء [عس] سنيز: جبال، زردانة، عناق.

\par وبنو هيري: فودت، دواد، ديفادزيت، اينوك.

وبنو توغورما: أبيود، وسافاث، وأسابلي، وزفتير.

\par وأبناء إليسا: إتزاك، زينيز، ماستيسا، ريرا.

وبنو زبتي: ماكصيئيل، تمناع، أيلة، فينون.

\par وبنو تسيس: مككول، لون، زلاطابان.

وبنو دودينين: إيثب، وبيث، وفينك.

\par 3 وهؤلاء هم الذين تشتتوا وسكنوا الأرض مع الفرس والماديين وفي الجزائر التي في البحر. فصعد فينك بن دوديني وأمر بصنع سفن البحر، فانقسم ثلث الأرض

\par 4 دوميرث وأبناؤه أخذوا لادك؛ وماجوج وأبناؤه أخذوا ديجال؛ ومدام وأبناؤها أخذوا بستو؛ ويوبان (من جاوة) وأبناؤه أخذوا سيل؛ وتوبال وأبناؤه أخذوا فيد؛ وميشك وأبناؤه أخذوا نفثي؛ وتيراس وأبناؤه أخذوا رو؛ ودودنوت وأبناؤه أخذوا جودا؛ وريفاث وأبناؤه أخذوا بوسارا؛ وتورغوما وأبناؤه أخذوا فود؛ وإليشع وأبناؤه أخذوا تابولا؛ وثيسيس (من ترشيش) وأبناؤه أخذوا ماريشام؛ وسيثيم وأبناؤه أخذوا ثان؛ ودودينين وأبناؤه أخذوا كاروبا.

\par 5 ثم ابتدأوا يفلحون الأرض ويزرعونها. ولما عطشت الأرض، صرخ سكانها إلى الرب، فاستجاب لهم وأعطى مطرًا غزيرًا. وكان لما نزل المطر على الأرض أن القوس ظهرت في السحاب، فرأى سكان الأرض تذكار العهد، وسقطوا على وجوههم، وقدموا ذبائح محرقات للرب

\par 6 وكان بنو حام: كوش، ومسترة، وفوني، وكنعان.

\par وابنا كوش: شبا، و... تودان

\par وأبناء فوني: [إفونتينوس]، زيليوتيلوب، جيلوك، ليفوك.

وكان بنو كنعان صيدونا وعندعين وراسين وسمين وأوروين ونوجين وأماثين ونفيتي وطلع وأيلات وكوزن.

\par 7 وولد خوس نمبروث. وبدأ يتكبر أمام الرب.

\par وأما مسترام فولد لودين وميغيمين ولابين ولاتوين وبيترسونوين وسيسلون. ومن هناك خرج الفلسطينيون والكبادوكيون

\par 8 ثم بدأوا أيضًا في بناء المدن. وهذه هي المدن التي بنوها: صيدون وما حولها، وهي رسن، وبعوسة، ومزة، وجرجرة، وعسقلان، ودبير، وكامو، وتلون، ولقيس، وسدوم، وعمورة، وأدمة، وصبويم

\par 9 وبنو سام: عيلام، وأشور، وأرفقشا، ولوزي، وأرام. وبنو أرام: جدروم، وعيسى. وأرفقشا ولد سال، وشال ولد حابر، وولد لحابر ابنان: اسم أحدهما فالك، لأنه في أيامه قسمت الأرض، واسم أخيه يقطان

\par 10 وولد يقطان هلمدام وسلسترا ومعزام ورع ودورا وعزيا ودجلبال وميموئيل وسبثفين وإيفلاق ويوباب

وبنو فالك: رقاو، ورفوث، وزفرام، وأكولون، وساخار، وسيفاز، ونبي، وصوري، وسكيور، وفلقوس، ورافو، وفلثيا، وصلديفال، وزافيس، وأرطامان، وحليفا. هؤلاء بنو فالك، وهذه أسماؤهم، واتخذوا لأنفسهم نساءً من بنات يقطان، وولدوا بنين وبنات، وملأوا الأرض.

\par 11 واتخذه راغاو زوجةً لمَلْخا ابنة راعوث، فولدت له سَرْوخ. ولما جاء يوم ولادتها قالت: سيولد من هذا الطفل في الجيل الرابع من يجعل مسكنه في الأعالي، ويُدعى كاملاً وبلا دنس، ويكون أبًا للأمم، ولا يُنقض عهده، ويكثر نسله إلى الأبد

\par 12 وعاش رقاو بعد أن ولد سروخ مئة وتسع عشرة سنة، وولد سبعة بنين وخمس بنات. وهذه أسماء أبنائه: أبيئيل، وعوبيد، وسلمى، ودداسال، وزينزة، وأكور، ونفيس. وهذه أسماء بناته: سيديما، ودريسا، وسيفا، وفريطة، وتيلا.

\par 13 وعاش سروخ تسعًا وعشرين سنة وولد ناحور. وعاش سروخ بعد أن ولد ناحور سبعًا وستين سنة وولد أربعة بنين وثلاث بنات. وهذه أسماء الذكور: زيلع، وصوبا، وديكا، وفوده. وهذه بناته: تفيلا، وعوده، وسليفة

\par 14 وعاش ناحور أربعًا وثلاثين سنة وولد ثارا. وعاش ناحور بعد أن ولد ثارا مئتي سنة وولد ثمانية أبناء وخمس بنات. وهذه أسماء الذكور: رَكاب، دَدِياب، بَريخاب، إيوساك، سِثال، نِساب، ناداب، كاموئيل. وهذه بناته: إسكا، ثيفا، برونا، سينيتا

\par 15 وعاش ثارا 70 سنة وولد أبرام وناحور وأرام. وولد أرام لوث

\par 16 حينئذٍ ابتدأ سكان الأرض ينظرون إلى النجوم، ويبدأون بالتنبؤ بها، ويعرِّفون، ويُمرِّرون أبنائهم وبناتهم في النار. أما سروخ وبنوه فلم يسلكوا مثلها

\par 17 وهذه أجيال نوح على الأرض حسب لغاتهم وقبائلهم، الذين تفرقت منهم الأمم على الأرض بعد الطوفان

\chapter{5}

\par 1 ثم جاء بنو حام وجعلوا نمبروث أميرًا عليهم، وأما بنو يافث فجعلوا فينك رئيسًا عليهم، فاجتمع بنو سام وجعلوا عليهم يقطان أميرًا عليهم

\par 2 ولما اجتمع هؤلاء الثلاثة، تشاوروا على أن ينظروا إلى أتباعهم ويأخذوا برأيهم. وقد تم ذلك بينما كان نوح لا يزال حيًا، وذلك حتى يجتمع جميع الناس معًا: وعاشوا معًا، وكانت الأرض في سلام

\par 3 وفي السنة 340 لخروج نوح من الفلك، بعد أن جفف الله الطوفان، أخذ الرؤساء في الحسبان شعبهم

\par 4 فنظر إليهم أولاً فينك بن يافث.

\par وكان جميع بني جومر الذين مرّوا حسب صولجانات رؤساءهم خمسة آلاف وثمانمائة

وأما بنو ماجوج فكل من مر منهم حسب صولجان قيادته كان عددهم ستة آلاف ومائتين.

ومن بني ماداي كل من مر حسب صولجانات رؤساءه كان عددهم خمسة آلاف وسبع مئة.

\par وكان بنو توبال كل من مروا حسب صولجانات رؤساءهم عددهم تسعة آلاف وأربعمائة.

وأما بنو مسكة فكل من مر منهم حسب صولجانات رؤساءه كان عددهم خمسة آلاف وست مئة.

وكان جميع بني تيراس الذين مروا حسب صولجانات رؤساءهم اثنا عشر ألفا وثلاثمائة.

وكان بنو ريفا الذين مروا حسب صولجانات رؤساءهم أربعة عشر ألفا وخمس مئة.

وكان عدد أبناء توجرما الذين مروا حسب صولجانات قيادتهم 14400.

وأما بنو إليشع الذين مروا حسب صولجانات قيادتهم فكان عددهم أربعة عشر ألفا وتسع مئة.

وأما بنو تيرسيس فكل من مر منهم حسب صولجانات قيادته فكان عددهم اثنا عشر ألفا ومئة.

وكان جميع بني كيثين الذين مروا حسب صولجانات قيادتهم سبعة عشر ألفا وثلاثمائة.

وكان بنو دوين الذين مروا حسب صولجانات رؤساءهم سبعة عشر ألفا وسبع مئة.

وكان عدد جيش بني يافث كلهم ​​رجال بأس، جميعهم متقلدون بسلاحهم، الذين وضعوا أمام عيون رؤسائهم مئة وأربعين ألفا ومئتين واثنين، ما عدا النساء والأطفال.

أما قصة يافث كاملة فكانت في العدد 142000.

\par 5 ومر نمبروث، هو وأبناء شام، وجميعهم مروا حسب صولجانات قياداتهم، وُجدوا في العدد 24800

وكان عدد بني فوا، كل من مر من هنا حسب صولجانات قياداته، سبعة وعشرون ألفا وسبعمائة.

\par وجميع بني كنعان الذين مروا حسب صولجانات رؤساءهم وجدوا في العدد اثنين وثلاثين ألفا وثمان مئة.

\par أبناء سوبا جميعهم مروا حسب صولجانات قياداتهم وجدوا في العدد 4300.

\par أبناء لبيلا جميعهم مروا حسب صولجانات قيادتهم وجدوا عددهم 22300.

\par وأما أبناء ساتا فكل من مر منهم حسب صولجانات قيادته وجد عددهم 25300.

\par وأما بنو رماة فكل من مر منهم حسب صولجانات رؤساءه وجد عددهم 30 ألفا وستمائة.

\par وأما بنو سباكا فكل من مر منهم حسب صولجانات رؤساءه وجد عددهم ستة وأربعمائة.

وكان عدد جيش بني حام كلهم ​​رجال جبابرة مسلحون بالسلاح الذين وقفوا أمام عيون رؤسائهم مئتان وأربعة وتسع مئة ما عدا النساء والأطفال.

\par 6 ونظر يقطان بن سام إلى بني عيلام، فإذا جميعهم يمرون حسب عدد صولجانات رؤساءهم، سبعة وأربعون ألفًا

وأما بنو آشور فكل من مر منهم حسب صولجانات رؤساءه وجدوا ثلاثة وسبعين ألفا.

\par وجميع بني أرام الذين مروا حسب صولجانات رؤساءهم وجدوا في العدد سبعة وثمانون ألفا وثلاث مئة.

\par وأما بنو لود فكل من مر منهم حسب صولجانات رؤساءه وجد عددهم 30 ألفا وستمائة.

\par [كان عدد أبناء شام 73000.]

وأما بنو أرفاكس فكل من مر منهم حسب صولجانات قياداته كان عددهم مائة وأربعة عشر ألفا وستمائة.

\par وكان عددهم 347600.

\par 7 وكان عدد جيش بني سام، كلهم ​​خارجين في البأس وفي أمر الحرب أمام رؤسائهم 9، ما عدا النساء والأطفال.

\par 8 وهذه هي أجيال نوح مرتبة على حدة، وكان مجموعها 914,000. وقد أُحصي كل هؤلاء عندما كان نوح حيًا، وأمام نوح بعد الطوفان بثلاثمائة وخمسين سنة. وكانت كل أيام نوح تسعمائة وخمسين سنة، ومات

\chapter{6}

\par 1 ثم اجتمع هناك جميع الذين تفرقوا وسكنوا على الأرض، وسكنوا معًا. ارتحلوا من المشرق ووجدوا سهلًا في أرض بابل. وسكنوا هناك، وقالوا كل واحد لجاره: هوذا سيكون أننا سنتشتت كل واحد عن أخيه، وفي الأيام الأخيرة سنتقاتل بعضنا ضد بعض. فالآن تعالوا نبني لأنفسنا برجًا، رأسه إلى السماء، ونجعل لأنفسنا اسمًا وصيتًا على الأرض

\par 2 وقال كل واحد لجاره: لنأخذ لبنا (حرفيًا حجارة)، ولنكتب كل واحد أسماءنا على اللبن ونحرقه بالنار. فما احترق تمامًا يكون للطين والطوب. (لعل ما لم يحترق تمامًا يكون للطين، وما احترق تمامًا يكون للطوب.)

\par 3 فأخذ كل واحد لبنه، إلا 12 رجلاً لم يأخذوها، وهذه أسماؤهم: إبراهيم، ناحور، لوث، روج، تنوت، زابا، أرمودت، يوباب، إيسار، أبيمايل، شبا، عوفين

\par 4 فوضع أهل الأرض أيديهم عليهم وأحضروهم أمام رؤسائهم وقالوا: هؤلاء هم الرجال الذين تعدوا على مشوراتنا ولن يسلكوا في طرقنا. فقال لهم الرؤساء: لماذا لا تضعون كل واحد لبناتكم مع أهل الأرض؟ فأجابوا وقالوا: لا نضع لبنا معكم ولا ننضم إلى رغبتكم. نحن نعلم ربًا واحدًا وإياه نعبد. وإذا ألقيتمونا في النار مع لبناتكم فلن نرضى بكم

\par 5 فغضب الأمراء وقالوا: كما قالوا، فافعلوا بهم كذلك، وإن لم يوافقوا على وضع الطوب معكم، فأحرقوهم بالنار مع طوبكم

\par 6 فأجاب جَكْتان، وهو أول رئيس للرؤساء: لا، بل يُعطَونَ مهلة سبعة أيام. ويكون أنهم إن تابوا عن مشوراتهم الشريرة، ووضعوا الطوب معنا، فسيحيون. وإلا، فليُحرَقوا حسب قولك. لكنه طلب كيف ينقذهم من أيدي الشعب؛ لأنه كان من سبطهم، وكان يعبد الله

\par 7 ولما قال هذا أخذهم وحبسهم في بيت الملك. ولما كان المساء أمر الأمير باستدعاء خمسين رجلاً من جبابرة الشجاعة، وقال لهم: اخرجوا الليلة وخذوا هؤلاء الرجال المحبوسين في بيتي، وضعوا لهم زاداً من بيتي على عشرة بهائم، وأتوا أنتم الرجال إليّ، وأما زادهم مع البهائم فخذوه إلى الجبال وانتظروهم هناك. واعلموا هذا أنه إن علم أحد بما قلته لكم أحرقكم بالنار.

\par 8 وانطلق الرجال وفعلوا كل ما أمرهم به رئيسهم، وأخذوا الرجال من بيته ليلاً، وأخذوا زادًا ووضعوه على الدواب، وذهبوا بهم إلى الجبل كما أمرهم

\par 9 فدعا الأمير هؤلاء الرجال الاثني عشر وقال لهم: تشجعوا ولا تخافوا، لأنكم لن تموتوا. لأن الله الذي تتوكلون عليه قدير، فاثبتوا فيه، لأنه سينقذكم ويخلصكم. والآن ها أنا قد أمرت رجالاً أن يأخذوا لكم زاداً من بيتي، ويسبقوكم إلى الجبل وينتظروكم في الوادي، وسأعطيكم خمسين رجلاً آخرين يرشدونكم إلى هناك. فاذهبوا واختبئوا هناك في الوادي، واشربوا الماء المتدفق من الصخور. ابقوا هناك 30 يوماً، حتى يهدأ غضب أهل الأرض، ويرسل الله غضبه عليهم ويحطمهم. لأني أعلم أن مشورة الإثم التي اتفقوا على تنفيذها لن تثبت، لأن فكرهم باطل ويكون إذا انقضت سبعة أيام وطلبوا عنكم، أقول لهم: قد خرجوا وكسروا باب السجن الذي كانوا محبوسين فيه وهربوا ليلاً، فأرسلت مئة رجل يطلبونهم، فأردهم عن جنونهم الذي عليهم.

\par 10 فأجابه أحد عشر رجلاً قائلين: قد وجد عبيدك نعمة في عينيك، إذ تحررنا من أيدي هؤلاء الرجال المتكبرين

\par 11 لكن أبرام اكتفى بالصمت، فقال له الأمير: لماذا لا تجيبني يا أبرام يا عبد الله؟ أجاب أبرام وقال: ها أنا أهرب اليوم إلى الجبال، وإن نجوت من النار، ستخرج الوحوش من الجبال وتلتهمنا. وإلا سينفد طعامنا ونموت جوعًا، ونُوجد هاربين من أهل الأرض ونسقط في خطايانا. والآن، بما أنه حيٌّ فيمن أثق فيه، فلن أنتقل من مكاني الذي وضعوني فيه، وإن كانت لي أي خطيئة تُحرقني، فلتكن مشيئة الله. فقال له الأمير: دمك على رأسك إن رفضت الخروج بهذه. ولكن إن وافقت، ستنجو. وإن بقيت، فابق كما أنت. فقال أبرام: لن أخرج، بل سأبقى هنا

\par 12 فأخذ الأمير أولئك الأحد عشر رجلاً وأرسل معهم خمسين آخرين، وأمرهم قائلاً: ابقوا أنتم أيضاً في الجبل خمسة عشر يوماً مع الخمسين الذين أرسلوا قبلكم، وبعد ذلك ترجعون وتقولون لم نجدهم كما قلت للأوائل. واعلموا أن من يتعدى على واحدة من جميع هذه الكلمات التي كلمتكم بها يُحرق بالنار. فخرج الرجال، فأخذ أبرام وحده وحبسه حيث كان محبوساً من قبل.

\par 13 وبعد مرور سبعة أيام، اجتمع الشعب وكلموا رئيسهم قائلين: رد إلينا الرجال الذين لم يوافقوا علينا، لنحرقهم بالنار. فأرسلوا رؤساء ليحضروهم، فلم يجدوهم إلا أبرام وحده. فجمعوهم كلهم ​​إلى رئيسهم قائلين: إن الرجال الذين حبستموهم قد هربوا ونجوا مما نصحناهم به

\par 14 فقال فينك ونمروث لجكتان: أين الرجال الذين حبستهم؟ فقال: لقد هربوا من السجن وهربوا ليلًا، لكنني أرسلت 100 رجل للبحث عنهم، وأمرتهم إذا وجدوهم ألا يحرقوهم بالنار فحسب، بل أن يسلموا أجسادهم لطيور السماء، وهكذا يهلكونهم

\par 15 فقالوا: هذا الذي يوجد وحده فلنحرقه. فأخذوا أبرام وأحضروه أمام أمرائهم وقالوا له: أين الذين كانوا معك؟ فقال: إني نمت في الليل، ولما استيقظت لم أجدهم

\par 16 فأخذوه وبنوا تنورًا وأشعلوه بالنار، ووضعوا فيه طوبًا محترقًا بالنار. فتعجب الأمير يقطان (حرفيًا: ذاب) فأخذ أبرام ووضعه مع الطوب في تنور النار

\par 17 فأثار الله زلزلة عظيمة، فاندفعت نار الأتون وتطايرت إلى لهيب وشرارات من نار، وأكلت كل من وقف حوله أمام الأتون، وكان جميع الذين احترقوا في ذلك اليوم ثلاثة وثمانين ألفًا وخمسمائة. وأما أبرام فلم يصبه أذى من احتراق النار

\par 18 فقام أبرام من الأتون، فسقط أتون النار، ونجا أبرام. وذهب إلى الأحد عشر رجلاً الذين كانوا مختبئين في الجبل، وأخبرهم بكل ما أصابه، فنزلوا معه من الجبل فرحين باسم الرب، ولم يصادفهم أحد ليخيفهم في ذلك اليوم. ودعوا ذلك المكان باسم أبرام، وبلسانة الكلدانيين دلّي، الذي يُترجم إلى الله



\chapter{7}

\par 1 وحدث بعد هذه الأمور أن شعب الأرض لم يرجعوا عن أفكارهم الشريرة، بل اجتمعوا أيضًا على رؤسائهم وقالوا: لن يُهزم الشعب إلى الأبد، والآن هلم نجتمع ونبني لأنفسنا مدينة وبرجًا لا يتزعزعان أبدًا

\par 2 "ولما بدأوا البناء رأى الله المدينة والبرج اللذين كان بنو البشر يبنيانهما فقال هوذا هذا شعب واحد وكلمتهم واحدة وما بدأوا في بنائه لا تتحمله الأرض ولا تسمح به السماء وهي تنظر إليه ويكون إن لم يعيقهم الآن أنهم يجرؤون على فعل كل ما يفكرون فيه."

\par 3 لذلك، ها أنا أُقسّم كلامهم، وأُشتّتهم في جميع البلدان، حتى لا يعرف كل إنسان أخاه، ولا يفهم كل إنسان كلام جاره. وأُسلّمهم إلى الصخور، فيبنون لأنفسهم خيامًا من القش والتبن، ويحفرون لأنفسهم كهوفًا ويعيشون فيها كوحوش الحقل، وهكذا سيبقون أمام وجهي إلى الأبد، فلا يُفكّرون في مثل هذه الأمور. وسأحسبهم كقطرة ماء، وأُشبّههم بالبصاق: فيأتي بعضهم نهايتهم بالماء، ويجفّ بعضهم الآخر عطشًا

\par 4 وأختار من بينهم جميعًا عبدي أبرام، وأخرجه من أرضهم، وأقوده إلى الأرض التي رأتها عيني منذ البدء، حين أخطأ جميع سكان الأرض أمام وجهي، وجلبت عليهم ماء الطوفان. ولم أهلك تلك الأرض، بل حفظتها. لذلك لم تتفجر فيها ينابيع غضبي، ولم ينزل عليها ماء هلاكي. لأني هناك أُسكن عبدي أبرام، وأقطع معه عهدي، وأبارك نسله، ويُدعى إلهه إلى الأبد

\par 5 ولما بدأ سكان الأرض في بناء البرج، فرق الله كلامهم وغير صورهم. فلم يعرف كل واحد أخاه، ولم يفهم كل واحد كلام جاره. فكان إذا أمر البناؤون مساعديهم بإحضار اللبن، كانوا يحضرون ماءً، وإذا طلبوا ماءً، كان الآخرون يحضرون لهم تبنًا. وهكذا نقضوا مشورتهم وكفوا عن بناء المدينة، وبددهم الله من هناك على وجه كل الأرض. لذلك سمي ذلك المكان "بلبلة"، لأن الله هناك بلبل كلامهم وبددهم من هناك على وجه كل الأرض

\chapter{8}

\par 1 فخرج أبرام من هناك وسكن في أرض كنعان، وأخذ معه لوث ابن أخيه، وساراي امرأته. ولأن ساراي كانت عاقرًا وليس لها ولد، أخذ أبرام هاجر جاريتها، فولدت له إسماعيل. وولد إسماعيل اثني عشر ابنًا

\par 2 ثم فارق لوط أبرام وسكن في سدوم [لكن أبرام سكن في أرض كام]. وكان رجال سدوم أشرارًا وخطاة جدًا

\par 3 وظهر الله لإبراهيم قائلاً: لنسلك أعطي هذه الأرض، فيُدعى اسمك إبراهيم، وتُدعى ساراي امرأتك سارة. وأعطيك منها نسلاً أبدياً، وأقطع عهدي معك. وعرف إبراهيم سارة امرأته، فحبلت وولدت إسحاق.

\par 4 واتخذ إسحاق زوجة من أرام النهرين، ابنة بتوئيل، فحبلت وولدت له عيسو ويعقوب

\par 5 واتخذ عيسو لنفسه نساءً: يودِين ابنة بيرو، وبسمة ابنة إيلون، وأليبيمة ابنة عنان، ومنعم ابنة سماحيل

وولدت له باسمة عدليفان، وكان بنو عدليفان تيمار وعمر وسفور وجيثان وتناز وأماليق. وولدت يودِن تنازيس ويرؤبيماس وباسمين وروجيل. وكان بنو روجيل نيزر وسمعة. وأليبمة ولدت عوز ويولام وقورو.

\par مانيم باري تينتدي، ثيناتيلا.

\par 6 واتخذ يعقوب لنفسه زوجاتٍ لبنتَي لابان الأرامي، ليئة وراحيل، وسريتين هما بالا وزلفا. فولدت له ليئة رأوبين وشمعون ولاوي ويهوذا وإيساكر وزبولون ودينا أختهم

وأما راحيل فولدت يوسف وبنيامين.

وبالا ولدت دان ونبتاليم، وزلفا ولدت جاد وأسير.

\par هؤلاء هم أبناء يعقوب الإثني عشر وابنته الواحدة.

\par 7 وسكن يعقوب في أرض كنعان، فقهر شكيم بن حمور الكوري ابنته دينا وأذلها. فدخل شمعون ولاوي ابنا يعقوب وقتلا كل مدينتهما بحد السيف، وأخذا دينا أختهما، وخرجا من هناك

\par 8 وبعد ذلك اتخذها أيوب زوجةً وولد منها 14 ابنًا و6 بنات، حتى 7 أبناء و3 بنات قبل أن يُصاب بالضيق، وبعد ذلك عندما شُفي 7 أبناء و3 بنات. وهذه أسماؤهم: أليفاق، وإرينوي، ودياشت، وفيلياس، ودفار، وزلود، وتيلون، وبناته: مرو، وليتاز، وزلي. وكما كانت أسماء الأولين، كانت أسماء الثانيات أيضًا

\par 9 وكان يعقوب وأبناؤه الاثنا عشر يسكنون في أرض كنعان. وكان أبناؤه يكرهون أخاهم يوسف، فسلموه أيضًا إلى مصر لبطفرس رئيس طهاة فرعون، فأقام عنده أربع عشرة سنة

\par 10 وحدث بعد أن رأى ملك مصر حلمًا أنهم أخبروه عن يوسف، فأخبره بالأحلام. وكان بعد أن أخبره بأحلامه أن فرعون جعله رئيسًا على كل أرض مصر. وفي ذلك الوقت كان هناك جوع في كل الأرض كما تنبأ يوسف. فنزل إخوته إلى مصر ليشتروا طعامًا، لأنه لم يكن في مصر إلا طعام. فعرف يوسف إخوته، فأُخبر بهم، ولم يسيء إليهم. وأرسل ودعا أباه من أرض كنعان، فنزل إليه

\par 11 وهذه أسماء بني إسرائيل الذين نزلوا إلى مصر مع يعقوب، كل واحد وبيته: بنو رأوبين: حنوك وفلود وحصرم وكرمين، وبنو شمعون: نموئيل ويامين ودوت وياخين، وشاول ابن الكنعانية.

وبنو لاوي جرشون وقعات ومراري. وأما بنو يهوذا فعونة وسيلون وفارص وزريمي.

بنو يساكر: تولاع وفوا، أيوب وسومبرام. بنو زبولون: شرلون ويائيل. ودينا أختهم ولدت أربعة عشر ابنًا وست بنات.

وهذه مواليد ليا التي ولدتها ليعقوب. جميع نفوس البنين والبنات اثنتان وسبعون.

\par 12 وكان بنو دان أوسينام. وبنو نبتاليم: بيتعال، ونيمو، وشورم، وأوبتسرئيل. وهذه مواليد بلة التي ولدتها
\par bare ليعقوب. جميع النفوس كانت ٨.

\par 13 وأما بنو جاد: ... شريئيل، وسوع، ويشوي، وموفاط، وسار. وأختهم ابنة شريئيل، ملكيئيل. هذه مواليد زلفة، امرأة يعقوب التي ولدتها له. وجميع نفوس البنين والبنات في العدد 10

\par 14 وابنا يوسف: أفرايم ومنسن، وبنيامين ولد: جيلا، وإسبيل، وأبوكمفيق، وأوتنديوس. وهذه هي النفوس التي ولدتها راحيل ليعقوب، 14

ثم نزلوا إلى مصر وأقاموا هناك مئتين وعشر سنين.

\chapter{9}

\par 1 وحدث بعد ذهاب يوسف أن بني إسرائيل كثروا وتكاثروا جدًا. وقام ملك آخر في مصر لم يكن يعرف يوسف، فقال لشعبه: هوذا هذا الشعب أكثر منا. تعالوا نتشاور عليهم لئلا يكثروا. وأمر ملك مصر جميع شعبه قائلًا: كل ابن يولد للرب يُطرح في النهر، وأما الإناث فيُستبقين. فأجاب المصريون ملكهم قائلين: نقتل ذكورهم ونستبقي إناثهم لنعطيهم لعبيدنا زوجات، والذي يولد منهم يكون عبدًا ويخدمنا. وهذا ما ظهر شرًا أمام الرب

\par 2 ثم جمع شيوخ الشعب الشعب في حزن ونوحوا ونوحوا قائلين: لقد عانت بطون نسائنا من ولادة سقطة. لقد سُلِّمت ثمرة بطوننا إلى أعدائنا، والآن قد قُطِعنا. ولكن فلنضع لأنفسنا فريضة ألا يقترب أحد من امرأته، لئلا يتنجس ثمرة بطنها، وتعبد أحشائنا الأصنام. لأنه من الأفضل أن نموت بلا أولاد حتى نعرف ما سيفعله الله

\par 3 فأجاب عمرام وقال: إن زوال الدهر وسقوط العالم اللامحدود، أو غرق قلب الأعماق في النجوم، أقرب من اندثار نسل بني إسرائيل. ويكون ذلك عندما يتم العهد الذي قطعه الله لإبراهيم قائلًا: "إن بنيك سيسكنون في أرض ليست لهم، وسيُستعبدون ويُذلون أربعمائة عام". وها هوذا منذ أن صدرت الكلمة التي كلم الله بها إبراهيم، ثلاثمائة وخمسون عامًا. ومنذ أن كنا في عبودية مصر، صارت مئة وثلاثين عامًا.

\par 4 والآن لن ألتزم بما تأمرون به، بل سأدخل وآخذ زوجتي وأنجب بنين، لكي نكثر على الأرض. لأن الله لن يستمر في غضبه، ولن ينسى شعبه إلى الأبد، ولن يطرد جنس إسرائيل إلى العدم على الأرض، ولم يقطع عهده مع آبائنا عبثًا: نعم، عندما لم نكن بعد، تكلم الله عن هذه الأمور

\par 5 فالآن سأذهب وآخذ زوجتي، ولن أوافق على أمر هذا الملك. وإن حسن في أعينكم، فلنفعل كلنا كذلك، لأنه عندما تحبل نساؤنا، لا يُعرفن أنهن حوامل حتى تتم ثلاثة أشهر، كما فعلت أيضًا أمنا ثامار، لأنها لم تكن تنوي الزنا، ولكن لأنها لم تشأ أن تنفصل عن بني إسرائيل، فكرت وقالت: من الأفضل لي أن أموت بسبب خطيئتي مع حمي من أن ألتصق بالأمم. وأخفت ثمرة بطنها إلى الشهر الثالث، لأنه حينئذٍ عُرف. ولما ذهبت لتقتل أكدت ذلك قائلة: الرجل الذي له هذه العصا وهذه الخاتمة وجلد الماعز، منه حبلت. وقد أنقذها تدبيرها من كل خطر

\par 6 والآن فلنفعل هكذا أيضًا. ويكون أنه عندما يحين وقت الولادة، إن أمكن، لن نطرح ثمرة بطوننا. ومن يدري لعل الله يستفز بذلك لينقذنا من ذلّنا؟

\par 7 وكان الكلام الذي كان في قلب عمرام مقبولًا لدى الله. فقال الله: من أجل أن كلام عمرام مقبول لدي، ولم ينقض العهد الذي قطع بيني وبين آبائه، فهوذا الآن المولود منه يخدمني إلى الأبد، وبه أصنع عجائب في بيت يعقوب، وأصنع بيده آيات وعجائب لشعبي لم أصنعها لغيري، وأصنع فيهم مجدي وأخبرهم بطرقي

\par 8 أنا الرب، سأشعل له سراجي ليسكن فيه، وأريه عهدي الذي لم يره أحد، وأظهر له عظمة بركتي ​​وعدلتي وأحكامي، وأضيء له نورًا أبديًا. لأني في الأيام القديمة فكرت فيه قائلًا: لن يكون روحي وسيطًا بين هؤلاء البشر إلى الأبد، لأنهم بشر، وأيامهم مئة وعشرين سنة

\par 9 فخرج عمرام من سبط لاوي وأخذ امرأة من سبطه، وكان لما أخذها أن الباقين فعلوا مثله وأخذوا نساءهم، وكان له ابن واحد وبنت واحدة اسمهما هارون ومريم.

\par 10 وحلَّ روح الله على ماريا ليلًا، فرأت حلمًا، وأخبرت والديها في الصباح قائلًا: رأيتُ هذه الليلة، وإذا برجلٍ يرتدي ثوبًا كتانًا واقفًا وقال لي: اذهبي وقولي لوالديكِ: هوذا الذي سيولد منكِ سيُلقى في الماء، لأنه بواسطته سيُجفف الماء، وبواسطته سأصنع الآيات، وسأخلص شعبي، ويكون له السيادة عليه إلى الأبد. وعندما أخبرت ماريا حلمها، لم يُصدقها والداها

\par 11 لكن كلمة ملك مصر سادت على بني إسرائيل، فأذلوا واضطهدوا في عمل الطوب

\par 12 فحبلت يوخابيث بعمرام وأخفت الولد في بطنها ثلاثة أشهر، لأنها لم تستطع إخفاءه أكثر من ذلك، لأن ملك مصر عيّن نظارًا على المنطقة، حتى إذا ولدت العبرانيات، ألقوا الذكور في النهر فورًا. فأخذت ولدها وصنعت له فلكًا من قشر شجرة صنوبر، ووضعت الفلك على حافة النهر

\par 13 وُلد الصبي في عهد الله وفي عهد جسده

\par 14 ولما أخرجوه، اجتمع جميع الشيوخ وجادلوا عمرام قائلين: أليس هذا هو الكلام الذي تكلمنا به قائلين: «خير لنا أن نموت بلا أولاد من أن يُلقى ثمرنا في الماء؟» ولما قالوا ذلك، لم يسمع لهم عمرام

\par 15 فنزلت ابنة فرعون لتغتسل في النهر كما رأت في الحلم، ورأت جارياتها التابوت، فأرسلت إحداهن وأخذته وفتحته. ولما رأت الطفل ونظرت إلى العهد، أي إلى العهد في جسده، قالت: إنه من بني العبرانيين

\par 16 فأخذته وربته، فصار لها ابنًا، ودعت اسمه موسى. أما أمه فقد دعته ملكيئيل. فرُبي الطفل وصار مجيدًا فوق كل الناس، وعلى يده خلص الله بني إسرائيل كما قال

\chapter{10}

\par 1 ولما مات ملك مصر، قام ملك آخر، وأزعج جميع شعب إسرائيل. فصرخوا إلى الرب فاستجاب لهم، فأرسل موسى وأنقذهم من أرض مصر. فأرسل الله عليهم أيضًا عشر ضربات فضربهم. وهذه هي الضربات: الدم، والضفادع، وكل نوع من الذباب، والبرد، وموت الماشية، والجراد والبعوض، والظلام المحسوس، وموت الأبكار

\par 2 ولما خرجوا من هناك وارتحلوا، عادت قلوب المصريين تقسو، فطاردوهم، فوجدوهم عند البحر الأحمر. فصرخ بنو إسرائيل إلى إلههم وكلموا موسى قائلين: هوذا قد حان وقت هلاكنا، فالبحر أمامنا، وجيش الأعداء خلفنا، ونحن في وسطهم. فهل لهذا أخرجنا الله، أم هذه هي العهود التي قطعها مع آبائنا قائلاً: لنسلكم أعطي الأرض التي أنتم ساكنون فيها؟ فليفعل بنا ما يحسن في عينيه.

\par 3 حينئذٍ قسم بنو إسرائيل مشوراتهم إلى ثلاثة أقسام مشورات، بسبب خوف الوقت. لأن سبط رأوبين ويساكر وزبولون وشمعون قالوا: هلموا نلقي بأنفسنا في البحر، لأنه خير لنا أن نموت في الماء من أن نقتل على أيدي أعدائنا. وقال سبط جاد وأشير ودان ونبتاليم: لا، بل نرجع معهم، وإن أعطونا أرواحنا نخدمهم. وأما سبط لاوي ويهوذا ويوسف وسبط بنيامين فقالوا: لا، بل نأخذ أسلحتنا ونقاتلهم، فيكون الله معنا

\par 4 فصرخ موسى أيضًا إلى الرب وقال: أيها الرب إله آبائنا، ألم تقل لي: اذهب وقل لبني ليئة: قد أرسلني الله إليك؟ والآن ها أنت قد أتيت بشعبك إلى حافة البحر، والعدو يتبعهم. أما أنت يا رب فاذكر اسمك

\par 5 وقال الله: بما أنك صرخت إليّ، خذ عصاك واضرب البحر فينشف. ولما فعل موسى كل هذا، وبخ الله البحر، فينشف البحر: توقفت بحار المياه، وظهرت أعماق الأرض، وانكشفت أساسات المسكن من صوت خوف الله ومن نفخة غضب سيدي

\par 6 وعبر بنو إسرائيل على اليابسة في وسط البحر. فلما رأى المصريون ذلك، لحقوا بهم، فشدد الله قلوبهم، فلم يعلموا أنهم داخلون في البحر. وكان أنه بينما كان المصريون في البحر، أمر الله البحر مرة أخرى، وقال لموسى: اضرب البحر مرة أخرى. ففعل كذلك. فأمر الرب البحر فعاد إلى أمواجه، وغطى المصريين ومركباتهم وفرسانهم إلى هذا اليوم

\par 7 أما شعبه، فقد أخرجهم إلى البرية: أربعين سنة أمطر لهم خبزًا من السماء، وأخرج لهم السلوى من البحر، وأخرج لهم بئر ماء تابعًا لهم. وفي عمود سحاب كان يهديهم نهارًا، وفي عمود نار ليلًا كان ينير لهم

\chapter{11}

\par 1 وفي الشهر الثالث من خروج بني إسرائيل من أرض مصر، وصلوا إلى برية سيناء. فتذكر الله كلامه وقال: سأُنير العالم، وأُنير المساكن، وأُبرم عهدي مع بني البشر، وأُمجّد شعبي فوق جميع الأمم، لأني سأُقيم لهم سُرورًا أبديًا يكون لهم نورًا، وللكافرين عقابًا.

\par 2 وقال لموسى: ها أنا أدعوك غدًا. كن مستعدًا وقل لشعبي: "ثلاثة أيام لا يقترب رجل من امرأته"، وفي اليوم الثالث أكلمك وإياهم، وبعد ذلك تصعد إليّ. وأضع كلامي في فمك، فتنير شعبي. لأني قد دفعت إلى يديك شريعة أبدية أحكم بها على كل العالم. لأن هذه ستكون شهادة. لأنه إن قال الناس: "لم نعرفك، ولذلك لم نعبدك"، فإني أنتقم منهم، لأنهم لم يعرفوا شريعتي

\par 3 ففعل موسى كما أمره الله، وقدس الشعب، وقال لهم: كونوا مستعدين في اليوم الثالث، لأنه بعد ثلاثة أيام يقطع الله عهده معكم. فتقدس الشعب

\par 4 وفي اليوم الثالث، إذا بأصوات رعود وبروق لامعة وأصوات آلات تُسمع. وكان خوف على جميع الشعب الذين في المحلة. فأخرج موسى الشعب لملاقاة الله

\par 5 وإذا الجبال تحترق بالنار، والأرض تزلزلت، والتلال تزعزعت، والجبال تزعزعت، والأعماق تغلي، وكل المساكن تزلزلت، والسماوات انطوت، وسحبت السحب ماءً. وأشرق لهيب النار، وكثرت الرعود والبروق، وهبت الرياح والعواصف، وتجمعت النجوم، وركض الملائكة أمامها، حتى أقام الله شريعة عهد أبدي مع بني إسرائيل، وأعطاهم وصية أبدية لا تزول

\par 6 وفي ذلك الوقت كلم الرب شعبه بجميع هذه الكلمات قائلاً: أنا الرب إلهك الذي أخرجك من أرض مصر، من بيت العبودية. لا تصنع لك آلهة منحوتة، ولا تصنع لك تمثالاً قبيحاً للشمس أو القمر أو شيئاً من زينة السماء، ولا صورة لكل ما على الأرض ولا مما يدب في المياه أو على الأرض. أنا الرب إلهك إله غيور، أجازي خطايا الذين ينامون على أبناء الأشرار الأحياء، إذا سلكوا في طرق آبائهم، إلى الجيل الثالث والرابع، وأصنع (أو أظهر) رحمة إلى ألف جيل مع الذين يحبونني ويحفظون وصاياي

\par 7 لا تنطق باسم الرب إلهك باطلا، لئلا تصير طرقي باطلة، لأن الله يبغض من ينطق باسمه باطلا.

\par 8 احفظ يوم السبت لتقديسه. ستة أيام تعمل عملك، وأما اليوم السابع فهو سبت للرب. لا تعمل فيه عملاً أنت وجميع عمالك، إلا أن تسبّحوا الرب في جماعة الشيوخ، وتمجّدوا القدير في مجلس الشيوخ. لأنه في ستة أيام صنع الرب السماء والأرض، والبحر وكل ما فيها، وكل المسكونة، والبرية غير المسكونة، وكل ما يعمل، وكل نظام السماء، واستراح الله في اليوم السابع. لذلك قدس الله اليوم السابع لأنه استراح فيه

\par 9 تُحِبُّ أَبَاكَ وَأُمِّي وَتَهْتَمُّهُما. وَحِينَئِذٍ يُشْرِقُ نُورُكَ، وَأُؤْمِرُ السَّمَاءَ فَتُعْطِيكَ مَطَرَهَا، وَتُسْرِعُ الأَرْضُ فِي إِثْمِهَا، وَتَكُونُ أَيَّامُكَ كَثِيرَةً، وَتَسْكُنُ فِي أَرْضِكَ وَلَا تَكُونُ بِطُفْلًا، لأَنَّ نَسْلَكَ لَنْ يَفْنَى، نَسْلُ السَّاكِنِينَ فِيهَا

\par 10 لا تزنِ، لأن أعداءك لم يزنوا معك، بل خرجت بيد عالية

\par 11 لا تقتل، لأن أعداءك لم يسيطروا عليك ليقتلوك، لكنك شاهدت موتهم

\par 12 لا تشهد على قريبك شهادة زور، لئلا يتكلم عليك رقباءك زورًا

\par 13 لا تشتهِ بيتَ قريبِك ولا ما له، لئلا يشتهي الآخرون أرضَك

\par 14 ولما فرغ الرب من الكلام، خاف الشعب خوفًا عظيمًا، ورأوا الجبل يشتعل بمشاعل نار، فقالوا لموسى: تكلم أنت معنا، ولا تكلمنا الله لئلا نموت. لأنه هوذا نعلم اليوم أن الله يكلم الإنسان وجهًا لوجه، فيحيا الإنسان. والآن قد أدركنا حقيقة أن الأرض ارتجفت عند سماع صوت الله. فقال لهم موسى: لا تخافوا، لأن هذا هو السبب الذي من أجله جاءكم هذا الصوت لكي لا تخطئوا (أو من أجل هذا السبب لكي يختبركم، جاءكم الله لكي تنالوا مخافته عليكم لكي لا تخطئوا).

\par 15 ووقف جميع الشعب من بعيد، أما موسى فتقدم نحو السحابة عالماً أن الله هناك. فكلمه الله بره وأحكامه، وحفظه عنده أربعين نهاراً وأربعين ليلة. وهناك أوصاه بأمور كثيرة، وأراه شجرة الحياة التي قطعها وأخذها ووضعها في مارة، فحلت مياه مارة، وتبعهم في البرية أربعين سنة، وصعد معهم إلى الجبال ونزل إلى السهل. وأوصاه أيضاً بشأن المسكن وتابوت الرب، وذبيحة المحرقة والبخور، وترتيب المائدة والمنارة، والمرحضة وقاعدتها، والكتف والصدرة، والحجارة الكريمة التي يصنعها بنو إسرائيل هكذا، وأراه شبهها ليصنعها على النموذج الذي رآه. وقال له: اصنع لي مقدساً، ويكون مسكن مجدي في وسطكم.

\chapter{12}

\par 1 فنزل موسى، وكان مُغطّى بنور غير مرئي - لأنه نزل إلى المكان الذي فيه نور الشمس والقمر - غلب نور وجهه على لمعان الشمس والقمر، فلم يعلم. وكان كذلك، لما نزل إلى بني إسرائيل، رأوه ولم يعرفوه. ولكن لما تكلم عرفوه. وكان هذا مثل ما حدث في مصر عندما عرف يوسف إخوته ولم يعرفوه. وكان بعد ذلك، لما علم موسى أن وجهه قد صار مجيدًا، جعل له برقعًا ليغطي وجهه

\par 2 ولكن بينما كان في الجبل، فسد قلب الشعب، واجتمعوا على هارون قائلين: اصنع لنا آلهة لنعبدها كما فعل سائر الأمم. لأن هذا موسى الذي صنعت به العجائب أمامنا، قد رُفع عنا. فقال لهم هارون: اصبروا، لأن موسى سيأتي ويقرّب إلينا القضاء، وينير لنا شريعة، ويُظهر من فمه عظمة الله، ويضع أحكامًا لشعبنا

\par 3 ولما قال هذا، لم يسمعوا له، لكي يتم القول الذي قيل يوم أخطأ الشعب في بناء البرج، حين قال الله: والآن إن لم أنهاهم، فسيغامرون بكل ما يفكرون أن يفعلوه، بل أسوأ. لكن هارون خاف، لأن الشعب كان قد قوي جدًا، وقال لهم: هاتوا لنا أقراط نسائكم. فطلب الرجال كل واحد امرأته، فأعطوها في الحال، ووضعوها في النار، فصاروا تمثالًا، فخرج عجل مسبوك

\par 4 وقال الرب لموسى: أسرع من هنا، لأن الشعب قد فسد وغش في طرقي التي أوصيتهم بها. ماذا لو انتهت الوعود التي قطعتها لآبائهم عندما قلت: هل أعطي نسلك هذه الأرض التي تسكنون فيها؟ لأنه هوذا الشعب لم يدخل الأرض بعد، مع أنهم يحملون أحكامي، إلا أنهم تركوني. ولذلك أعلم أنهم إذا دخلوا الأرض سيفعلون إثمًا أعظم. والآن سأتركهم أيضًا: وسأعود وأصالحهم، فيبنى لي بيت في وسطهم؛ وسيُهلك ذلك البيت أيضًا، لأنهم سيخطئون إليّ، وستكون سلالة البشر بالنسبة لي كقطرة جرة، وستُحسب كالبصاق.

\par 5 فأسرع موسى ونزل ورأى العجل، ونظر إلى اللوحين فرأى أنهما غير مكتوبين، فأسرع وكسرهما، فانفتحت يداه وصار كامرأة تلد ابنها البكر، الذي عندما تأخذه مخاضه تكون يداها على حضنه، ولا قوة لها على ولادته

\par 6 وبعد ساعة قال في نفسه: المرارة لا تدوم إلى الأبد، والشر لا يسود إلى الأبد. الآن سأقوم وأقوي حقويّ، لأنه وإن أخطأوا، فلن تضيع هذه الأمور التي أُخبرت بها أعلاه

\par 7 فقام وكسر العجل وألقاه في الماء وسقى الشعب. وكان كذلك، إذا أراد أحد في نفسه أن يصنع العجل، يُقطع لسانه، ولكن إذا اضطر أحد إلى ذلك خوفًا، أضاء وجهه

\par 8 ثم صعد موسى إلى الجبل وصلى إلى الرب قائلاً: هوذا أنت الله الذي غرست هذا الكرم، وغرست جذوره في العمق، ومددت أغصانه إلى عرشه الأعلى. انظر إليه الآن، فإن الكرم قد أثمر ولم يعرف فلاحه. والآن إن كنت غاضبًا على كرمك واقتلعته من العمق، وأذبلت أغصان عرشه الأعلى الأبدي، فلن يأتي الغمر بعد ليغذيه، ولا عرشك لينعش كرمك الذي أحرقته

\par 9 لأنك أنت النور كله، وقد زينت منزلك بالأحجار الكريمة والذهب والعطور والتوابل (أو واليشب)، وخشب البلسم والقرفة، وبجذور المر والأزياء نثرت منزلك، وبأطعمة متنوعة وحلاوة الكثير من المشروبات أشبعته. لذلك إذا لم ترحم كرمك، فكل هذه الأشياء تكون عبثًا يا رب، ولن يكون لديك من يمجدك. لأنه حتى لو زرعت كرمًا آخر، فلن يثق بك ذلك الآخر أيضًا، لأنك دمرت الأول. لأنه إذا كنت حقًا تترك العالم، فمن سيفعل لك ما تكلمت به كإله؟ والآن دع غضبك يكبح جماح كرمك أكثر [بسبب] ما قلته وما تبقى ليقال، ولا تدع عملك يذهب سدى، ولا تدع ميراثك يتمزق في الذل.

\par 10 فقال له الله: ها أنا قد صرت رحيمًا حسب كلامك. فانحت لك لوحين من حجر من المكان الذي نحتت فيه الأول، واكتب عليهما مرة أخرى أحكامي التي كانت على الأولين

\chapter{13}

\par 1 فأسرع موسى وفعل كل ما أمره به الله، ونزل وصنع الموائد [والمسكن] وأوانيها، والتابوت والمصابيح والمائدة ومذبح المحرقة ومذبح البخور والكتف والصدرة والحجارة الكريمة والمرحضة والقواعد وكل ما أُري له. وأمر بجميع ثياب الكهنة، المناطق وباقي الملابس، والتاج، وصفيحة الذهب، والإكليل المقدس. وصنع أيضًا زيت المسحة للكهنة، وقدس الكهنة أنفسهم. ولما كمل كل شيء غطتها السحابة جميعًا

\par 2 ثم صرخ موسى إلى الرب، فكلمه الله من المسكن قائلاً: هذه هي شريعة المذبح التي بها تذبحون لي وتصلون لأجل نفوسكم. وأما ما تقدمونه لي، فمن البقر العجل والغنم والمعز، ومن الطير السلحفاة والحمامة

\par 3 وإن كان في أرضكم برص، وطهر الأبرص، فليأخذوا للرب فرخين حيين، وخشب أرز وزوفا وقرمزًا، ويأتي إلى الكاهن، فيقتل أحدهما ويحفظ الآخر. فيأمر الأبرص بكل ما أمرت به في شريعتي

\par 4 ويكون عندما تأتي الأوقات إليكم أنكم تقدسونني في عيد، وتفرحون أمامي في عيد الفطير، وتضعون أمامي خبزًا، وتصنعون عيدًا للذكرى لأنكم في ذلك اليوم خرجتم من أرض مصر

\par 5 وفي عيد الأسابيع تضعون أمامي خبزًا، وتجعلون لي تقدمة من ثماركم

\par 6 لكن عيد الأبواق سيكون ذبيحة لمراقبيكم، لأني فيه أشرفت على خلقي، لكي تتذكروا العالم أجمع. في بداية العام، عندما تُرونيهم، سأعرف عدد الموتى والمواليد، وصوم الرحمة. لأنكم ستصومون لي من أجل نفوسكم، لكي تتم وعود آبائكم.

\par 7 أحضروا لي أيضًا عيد المظال. تأخذون لي ثمر الشجرة اللذيذ، وأغصان النخيل والصفصاف والأرز، وأغصان المر. وسأذكر الأرض كلها بالمطر، وسيُثبَّت مقياس الفصول، وسأنظم النجوم وأأمر السحب، وستُدوّي الرياح والبروق، وستكون هناك عاصفة رعدية، وسيكون ذلك علامة أبدية. أيضًا ستُنتج الليالي ندى، كما تكلمت بعد طوفان الأرض

\par 8 عندما أعطيته (أو أعطيته) وصية تتعلق بسنة حياة نوح، وقلت له: هذه هي السنوات التي حددتها بعد الأسابيع التي زرت فيها مدينة البشر، وفي أي وقت أريتهم (أو أريته) مكان الميلاد واللون (أو والثعبان)، وقلت (أو قال): هذا هو المكان الذي علمت عنه الإنسان الأول قائلاً: إن لم تتعد ما أمرتك به، فكل شيء سيكون خاضعًا لك. لكنه تعدى طرقي واقتنع بامرأته، وخدعتها الثعبان. ثم قُدِّر الموت على أجيال البشر

\par 9 وعلاوة على ذلك، أراه الرب (أو قال الرب أيضًا: أريته) طرق الجنة وقال له: هذه هي الطرق التي ضل الناس عنها لعدم سلوكهم فيها، لأنهم أخطأوا إليّ

\par 10 وأمره الرب بشأن خلاص نفوس الشعب وقال: إن سلكوا في طرقي فلن أتركهم، بل سأرحمهم إلى الأبد، وسأبارك نسلهم، وستُسرع الأرض لتعطي ثمرها، وسيكون هناك مطر لهم ليزيدوا من مكاسبهم، ولن تكون الأرض قاحلة. ومع ذلك، فإني أعلم حقًا أنهم سيفسدون طرقهم، وسأتركهم، وسينسون العهود التي قطعتها مع آبائهم. ومع ذلك لن أنساهم إلى الأبد، لأنهم في الأيام الأخيرة سيعلمون أنه بسبب خطاياهم قد تُرك نسلهم، لأني أمين في طرقي

\chapter{14}

\par 1 في ذلك الوقت قال له الله: ابدأ في إحصاء شعبي من عشرين سنة فصاعدًا إلى أربعين سنة، لكي أُري أسباطك كل ما أخبرت به آباءهم في أرض غريبة. لأني أخرجتهم من أرض مصر إلى الخمسين جزءًا، ومات منهم أربعون وتسعة أجزاء في أرض مصر

\par 2 عندما تأمرهم وتحصيهم (أو، بينما كنتم هناك. وعندما تحصيهم، إلخ)، اكتب أخبارهم، حتى أُتم كل ما كلمت آباءهم به، وأُرسخهم في أرضهم: لأني لن أُنقص من كلمة واحدة من تلك التي كلمت آباءهم بها، حتى من تلك التي قلت لهم: سيكون نسلكم كنجوم السماء في الكثرة. بالعدد سيدخلون الأرض، وسرعان ما سيُصبحون بلا عدد.

\par 3 ثم نزل موسى وأحصى الشعب، فكان عدد الشعب ستمائة وأربعة آلاف وخمسمائة وخمسين. وأما سبط لاوي فلم يعده بينهم، لأنه هكذا أُمر. إنما أحصى من كان عمره فوق خمسين سنة، فكان عددهم سبعة وأربعين ألفًا وثلاثمائة. وأحصى أيضًا من كان عمره أقل من عشرين سنة، فكان عددهم ثمانمائة وخمسين ألفًا وثمانمائة وخمسين. ثم نظر إلى سبط لاوي، فكان مجموع عددهم: CXX، CCXD، DCXX، CC، DCCC

\par 4 فأخبر موسى الله بعددهم، فقال له الله: هذه هي الكلمات التي كلمت بها آباءهم في أرض مصر، وحددت عددًا، مئتين وعشر سنين، لكل من رأى عجائبي. وكان عددهم جميعًا تسعة آلاف عشرة آلاف، مئتي ألف خمسة وتسعين ألف رجل، ما عدا النساء، فقتلت كل جمهورهم لأنهم لم يؤمنوا بي، وأبقيت الخمسين منهم، وقدستهم لنفسي. لذلك، أنا آمُر جيل شعبي أن يُعطوني عُشر ثمارهم، لتكون أمامي تذكارًا لعظم الظلم الذي رفعته عنهم

\par 5 ولما نزل موسى وأخبر الشعب بهذه الأمور، حزنوا وناحوا ومكثوا في البرية سنتين

\chapter{15}

\par 1 فأرسل موسى جواسيس ليتجسسوا الأرض، اثني عشر رجلاً، لأنه هكذا أُمر. ولما صعدوا ورأوا الأرض، رجعوا إليه حاملين من ثمر الأرض، وأزعجوا قلوب الشعب قائلين: لن تقدروا أن ترثوا الأرض، لأنها مغلقة بقضبان حديدية من قبل جبابرةهم

\par 2 لكن رجلين من الاثني عشر لم يتكلما هكذا، بل قالا: كما أن الحديد الصلب يتغلب على النجوم، أو كما أن الأسلحة تتغلب على الصواعق، أو كما أن طيور السماء تطفئ الرعد، كذلك يستطيع هؤلاء الرجال مقاومة الرب. لأنهم رأوا كيف أشرقت بروق النجوم وهم يصعدون، وتبعتها الرعود، ودوّت معها

\par 3 وهذه أسماء الرجال: كالب بن يفون، بن بري، بن باتوئيل، بن غليفا، بن زنين، بن سليم بن سيلون، بن يهوذا. والآخر: يسوع بن نوح، بن أليفات، بن غال، بن نفليان، بن إيمون، بن شاول، بن دبرة، بن أفرام، بن يوسف.

\par 4 ولكن الشعب لم يسمعوا صوت الاثنين، بل اضطربوا جدا وتكلموا قائلين: أهذه هي الكلمات التي كلمنا بها الله قائلا: سأدخلكم إلى أرض تفيض لبنا وعسلا؟ والآن كيف يصعدنا حتى نسقط بالسيف وتذهب نساؤنا إلى السبي؟

\par 5 ولما قالوا هذا، ظهر مجد الله بغتة، فقال لموسى: هل يستمر هذا الشعب هكذا في الاستماع إليّ؟ هوذا الآن المشورة التي خرجت مني لن تذهب سدى. سأرسل عليهم ملاك غضبي ليحطم أجسادهم بالنار في البرية. وسأوصي ملائكتي الذين يحرسهم ألا يصلوا من أجلهم، لأني سأحبس أرواحهم في مخازن الظلمة، وسأقول لعبيدي آباءهم: هوذا هذا هو النسل الذي كلمته قائلاً: سيأتي نسلك إلى أرض ليست لهم، والأمة التي سيعبدون لها سأدينها. وأتممت كلامي وأذبت أعداءهم، وأخضعت الملائكة تحت أقدامهم، وجعلت سحابة غطاءً لرؤوسهم، وأمرت البحر، فانكسرت اللجج أمام وجوههم، وقامت أسوار المياه

\par 6 ولم يكن مثل هذه الكلمة منذ اليوم الذي قلت فيه: لتجتمع المياه تحت السماء في مكان واحد، إلى هذا اليوم. وأخرجتهم، وقتلت أعداءهم، وهديتهم أمامي إلى جبل سيناء. وسجدت للسماء ونزلت لأشعل سراجًا لشعبي، ولأضع حدودًا لجميع المخلوقات. وعلمتهم أن يصنعوا لي مقدسًا لأسكن بينهم. لكنهم تركوني وأصبحوا غير مخلصين لكلامي، وفتور عقولهم، والآن ها هي الأيام ستأتي عندما أفعل بهم كما أرادوا، وسأطرح جثثهم في البرية

\par 7 فقال موسى: قبل أن تأخذ بذرًا لتصنع به الإنسان على الأرض، هيأت طرقه؟ فالآن فلتكن رحمتك علينا إلى النهاية، ورأفتك طول الأيام

\chapter{16}

\par 1 في ذلك الوقت أعطاه أمرًا بشأن الأطراف، فتمرد حوريب ومئتا رجل معه وتكلموا قائلين: ماذا لو سُنّ علينا ناموس لا نطيق تحمله؟

\par 2 فغضب الله وقال: أنا أمرت الأرض فأعطتني إنسانًا، وولد له ابنان. فقام الأكبر وقتل الأصغر، فأسرعت الأرض وابتلعت دمه. أما أنا فطردت قابيل، ولعنت الأرض، وكلمت صهيون قائلًا: لن تبتلعي دمًا بعد الآن. والآن قد دنست أفكار البشر جدًا

\par 3 ها أنا آمر الأرض فتبتلع الجسد والنفس معًا، ويكون مسكنهم في ظلام وهلاك، ولا يموتون بل يذوون حتى أذكر العالم وأجدد الأرض. وحينئذٍ يموتون ولا يعيشون، وتنزع حياتهم من بين جميع البشر، ولا تقذفهم الجحيم مرة أخرى، ولا يتذكرهم الهلاك، ويكون رحيلهم كرحيل قبيلة الأمم التي قلت: "لن أذكرهم"، أي معسكر المصريين، والشعب الذي أهلكته بماء الطوفان. فتبتلعهم الأرض، ولا أعود أجعلهم يفعلون بي شيئًا.

\par 4 ولما تكلم موسى بجميع هذا الكلام للشعب، كان حوريب ورجاله لا يزالون غير مؤمنين. فأرسل حوريب ليستدعي أبناءه السبعة الذين لم يكونوا على رأيه

\par 5 فأرسلوا إليه جوابًا قائلين: كما أن الرسام لا يعرض صورة مصنوعة بفنه إلا إذا تعلم أولاً، هكذا نحن أيضًا عندما أخذنا شريعة القدير التي تعلمنا طرقه، لم ندخلها إلا لكي نسلك فيها. أبونا لم يولدنا، لكن القدير كوننا، والآن إذا سلكنا في طرقه سنكون أولاده. ولكن إن لم تؤمن، فاذهب في طريقك. ولم يأتوا إليه

\par 6 وحدث بعد ذلك أن الأرض انفتحت أمامهم، فأرسل إليه أبناؤه قائلين: إن كان جنونك لا يزال عليك، فمن يعينك في يوم هلاكك؟ فلم يسمع لهم. ففتحت الأرض فاها وابتلعتهم وبيوتهم، وهزت أساسات الأرض أربع مرات لتبتلع الناس كما أُمرت. وبعد ذلك أنين حوريب وجماعته حتى استُرجع جلد الأرض

\par 7 فقالت جماعات الشعب لموسى: لا نستطيع أن نقيم حول هذا المكان الذي ابتُلِعَ فيه حوريب ورجاله. فقال لهم: انقلوا خيامكم من حولهم، ولا تلتصقوا بخطاياهم. ففعلوا كذلك

\chapter{17}

\par 1 ثم أُعلن عن نسب كهنة الله باختيار سبط، وقيل لموسى: خذ من كل سبط عصا واحدة وضعها في المسكن، فحينئذٍ يزدهر قضيب من يُكلَّم له مجدي، وأزيل التذمُّر عن شعبي

\par 2 ففعل موسى كذلك، ووضع اثنتي عشرة قضيبًا، فخرجت قضيب هارون، وأخرجت زهرًا وأخرجت بذرًا لوزًا

\par 3 وهذا الشبه الذي وُلد هناك كان يشبه العمل الذي عمل إسرائيل وهو في بلاد ما بين النهرين مع لابان الأرامي، حين أخذ قضبان لوز، وجعلها عند مجمع الماء، فجاءت الماشية لتشرب، وقُسِّمت بين القضبان المقشرة، فولدت جديًا أبيض ومنقطًا ومشقوقًا

\par 4 لذلك صار مجمع الشعب كقطيع الغنم، وكما كانت الماشية تخرج حسب أعواد اللوز، كذلك كانت الكهنوت يثبت بواسطة أعواد اللوز.

\chapter{18}

\par 1 في ذلك الوقت، قتل موسى سيعون وعوج ملكي الأموريين، وقسم جميع أرضهم على شعبه، فسكنوا بها

\par 2 وكان بالاق ملك موآب الساكن مقابلهم، فخاف خوفًا شديدًا، فأرسل إلى بلعام بن بعور مفسر الأحلام، الذي كان يسكن في بلاد ما بين النهرين، وأوصاه قائلًا: هوذا أعلم أنه في عهد أبي سافور، حين حاربه الأموريون، لعنتهم، فسلموا أمامه. والآن تعال والعن هذا الشعب، لأنهم أكثر منا، وسنكرمك شرفًا عظيمًا

\par 3 فقال بلعام: هوذا هذا حسن في عيني بالاق، ولكنه لا يعلم أن مشورة الله ليست كمشورة الإنسان. ولا يعلم أن الروح الذي يُعطى لنا يُعطى إلى حين، وأن طرقنا لا تُهدى إلا بإرادة الله. فالآن امكثوا هنا، وسأرى ما سيقوله لي الرب في هذه الليلة

\par 4 وفي الليل قال له الله: من هم الرجال الذين أتوا إليك؟ فقال بلعام: لماذا يا رب تُجرب الجنس البشري؟ لذلك لا يستطيعون تحمله، لأنك كنت تعرف أكثر منهم كل ما في العالم قبل أن تُؤسسه. والآن أنر عبدك إن كان من الصواب أن أذهب معهم

\par 5 فقال له الله: أليس عن هذا الشعب كلمت إبراهيم في رؤيا قائلاً: يكون نسلك كنجوم السماء حين رفعته فوق الجلد وأريته جميع ترتيبات النجوم وطلبت منه ابنه محرقة؟ فجاء به ليوضع على المذبح، فأعدته إلى أبيه. ولأنه لم يقاوم، صار قربانه مقبولاً في عيني، ومن أجل دمه اخترت هذا الشعب. ثم قلت للملائكة العاملين بالمكر: ألم أقل له: هل أخبر إبراهيم بكل ما أفعل؟

\par 6 يعقوب أيضًا، حين تصارع في التراب مع الملاك الذي على التسبيح، لم يتركه حتى باركه. والآن ها أنت ذا تظن أن تذهب مع هؤلاء وتلعن الذين اخترتهم. ولكن إن لعنتهم، فمن هو الذي يباركك؟

\par 7 فقام بلعام في الصباح وقال: اذهب، لأن الله لا يريدني أن آتي معك. فذهبوا وأخبروا بالاق بكل ما قيل عن بلعام. فأرسل بالاق أيضًا رجالًا آخرين إلى بلعام قائلًا: هوذا أنا أعلم أنه عندما تُقرب محرقات لله، فإن الله يصالح الإنسان، والآن اسأل ربك أيضًا، وتضرع إليه بمحرقات، كل من يشاء. لأنه إن رضي عني، فستكون لك مكافأتك، وإن رضي الله فليقبل تقدماتك

\par 8 فقال لهم بلعام: «هوذا ابن صفور أحمق، ولا يعلم أنه يسكن قرب الموتى. والآن امكث هنا الليلة فأرى ما يقوله الله لي». فقال له الله: «اذهب معهم، فيكون طريقك عثرة، ويذهب بالاق نفسه إلى الهلاك». فقام وذهب معهم.

\par 9 فجاءت حمارته في طريق البرية، فأبصرت الملاك، ففتحت عيني بلعام، فرأى الملاك، وسجد له على الأرض. فقال له الملاك: أسرع واذهب، لأن ما تقوله سيحدث له

\par 10 وجاء إلى أرض موآب وبنى مذبحًا وقدم ذبائح. ولما رأى قسمًا من الشعب، لم يسكن فيه روح الله، فأخذ مثله وقال: هوذا بالاق قد أتى بي إلى هنا إلى الجبل قائلًا: تعال اركض إلى نار هؤلاء الرجال. هوذا لا أطيق تلك [النار] التي تطفئها المياه، ولكن تلك النار التي تأكل الماء، فمن يتحمل؟ فقال له: من الأسهل إزالة الأساسات وكل قمتها، وإطفاء نور الشمس، وإظلام ضوء القمر، من أن يقتلع من يريد غرس القدير أو يخرب كرمه. وبالاق نفسه لم يعلم ذلك، لأن قلبه منتفخ، لكي يأتي هلاكه سريعًا

\par 11 لأني ها أنا أرى الميراث الذي أراني إياه القدير في الليل، وها هي الأيام تأتي حين تتعجب موآب مما يصيبها، لأن بالاق أراد أن يقنع القدير بالهدايا ويشتري الحكم بالمال. أما كان ينبغي لك أن تسأل ماذا أرسل على فرعون وعلى أرضه ليُدخلهم في العبودية؟ ها هي كرمة مُظللة، شهية للغاية، ومن يغار منها لأنها لا تذبل؟ ولكن إن قال أحد في مشورته إن القدير قد تعب عبثًا أو اختارهم بلا غرض، فها أنا الآن أرى خلاص النجاة الذي سيأتيهم. أنا مقيد في نطق صوتي ولا أستطيع التعبير عما أراه بعيني، لأنه لم يبق لي إلا القليل من الروح القدس الذي يحل فيّ، لأني أعلم أنني في اقتناعي ببالاق قد ضيعت أيام حياتي

\par 12 هوذا أرى ميراث مسكن هذا الشعب، ونوره يفوق لمعان البرق، وجريه أسرع من السهام. وسيأتي وقتٌ يئن فيه موآب، ويضعف الذين يخدمون حام (كيموش؟)، حتى أولئك الذين اتخذوا هذه النصيحة ضدهم. ولكنني سأصرّ على أسناني لأني خُدعت وتجاوزت ما قيل لي في الليل. ومع ذلك، ستبقى نبوتي واضحة، وستحيا كلماتي، وسيتذكر الحكماء والفهماء كلماتي، لأني عندما لعنت هلكت، ومع أنني باركت لم أُبارك. وعندما قال ذلك سكت. وقال بالاق: لقد حرمك إلهك من عطايا كثيرة مني.

\par 13 فقال له بلعام: تعالَ فنتشاور ماذا تفعل بهن. اختر أجمل النساء اللواتي بينكنّ واللاتي في مديان، وأوقفهنّ أمامهنّ عاريات، مزينات بالذهب والجواهر، ويكون عندما يرونهنّ ويضاجعونهنّ، أنّهنّ يُخطئن إلى ربهنّ ويقعن في يديك، وإلاّ فلا تستطيع إخضاعهنّ

\par 14 وبعد أن قال هذا، انصرف بلعام ورجع إلى مكانه. وبعد ذلك انخدع الشعب وراء بنات موآب، لأن بالاق فعل كل ما أراه إياه بلعام

\chapter{19}

\par 1 في ذلك الوقت، قتل موسى الأمم، وأعطى نصف الغنائم للشعب، وبدأ يُخبرهم بكلام الشريعة الذي كلّمهم الله به في غراب

\par 2 وكلمهم قائلاً: ها أنا أنام مع آبائي وأذهب إلى شعبي. ولكني أعلم أنكم ستقومون وتتركون الكلام الذي أمرتكم به، فيغضب الله عليكم ويترككم ويخرج من أرضكم، ويجلب عليكم مبغضيكم، فيتسلطون عليكم، ولكن ليس إلى النهاية، لأنه سيتذكر العهد الذي قطعه مع آبائكم

\par 3 ولكن حينئذٍ ستقومون أنتم وأبناؤكم وكل أجيالكم من بعدكم وتطلبون يوم وفاتي وتقولون في قلوبهم: من يعطينا راعيًا مثل موسى، أو قاضيًا آخر مثله لبني إسرائيل، ليصلي من أجل خطايانا في كل وقت، ويُسمع لآثامنا؟

\par 4 لكن اليوم أشهد عليكم السماء والأرض، لأن السماء ستسمع هذا، والأرض ستسمعه بأذنيها، أن الله قد كشف نهاية العالم، ليعقد عهدًا معكم على مرتفعاته، وأشعل سراجًا أبديًا في وسطكم. اذكروا أيها الأشرار كيف أجبتم حين كلمتكم قائلين: كل ما قاله الله لنا نسمعه ونفعله. ولكن إن تعدينا أو أفسدنا طرقنا، فإنه يشهد علينا ويقطعنا

\par 5 لكن اعلموا أنكم أكلتم خبز الملائكة 40 سنة. والآن ها أنا أبارك أسباطكم قبل أن تأتي نهايتي. أما أنتم فاعلموا تعبي الذي تعبت فيه معكم منذ يوم صعودكم من أرض مصر

\par 6 ولما قال هذا كلمه الله ثالثة قائلا هوذا أنت تضطجع مع آبائك، فيقوم هذا الشعب ويطلبني وينسى شريعتي التي أنرتهم بها، وأترك ​​نسلهم إلى حين.

\par 7 ولكني سأريك الأرض قبل أن تموت، ولكنك لن تدخلها في هذا العصر، لئلا ترى التماثيل المنحوتة التي سيُخدع بها هذا الشعب ويُضل عن الطريق. سأريك المكان الذي سيخدمونني فيه 740 (1850) سنة. وبعد ذلك تُدفع إلى أيدي أعدائهم، فيدمرونها، ويحيط بها الغرباء، ويكون في ذلك اليوم كما كان يوم كسرت فيه لوحي العهد الذي قطعته معك في غراب: وعندما أخطأوا، زال ما هو مكتوب فيه. وكان ذلك اليوم هو اليوم السابع عشر من الشهر الرابع

\par 8 فصعد موسى إلى جبل غراب، كما أمره الله، وصلى قائلاً: ها أنا قد أكملت مدة حياتي، 120 عامًا. والآن أطلب منك يا رب أن تكون رحمتك مع شعبك، وأن تستمر رحمتك على ميراثك، وأن تستمر أناتك في مكانك على النسل الذي تختاره، لأنك أحببتهم أكثر من الجميع

\par 9 وأنت تعلم أني كنت راعي غنم، وعندما كنت أرعى القطيع في البرية، أتيت بهم إلى جبل غراب، وحينئذ رأيتُ أولًا ملاكك في نار من العليقة. لكنك دعوتني من العليقة، فخفتُ وحولتُ وجهي، فأرسلتني إليهم، وأنقذتهم من مصر، وغرقت أعدائهم في الماء. وأعطيتهم شريعة وأحكامًا ليحيوا بها. لأنه أي إنسان لم يخطئ إليك؟ كيف يثبت ميراثك إن لم ترحمهم؟ أو من سيولد بعد بلا خطيئة؟ ومع ذلك ستؤدبهم إلى حين، ولكن ليس بغضب

\par 10 ثم أراه الرب الأرض وكل ما فيها، وقال: هذه هي الأرض التي أعطيها لشعبي. وأراه المكان الذي تسحب منه السحب الماء لتسقي كل الأرض، والمكان الذي يأخذ منه النهر ماءه، وأرض مصر، وموضع الجلد الذي منه تشرب الأرض المقدسة وحدها. وأراه أيضًا المكان الذي أمطر منه المن للشعب، وحتى سبل الفردوس. وأراه مقاييس المقدس، وعدد القرابين، والعلامة التي بها ينظر الناس إلى السماء، وقال: هذه هي الأشياء التي حُرمت على بني البشر لأنهم أخطأوا

\par 11 والآن، يكون عصاك التي صُنعت بها الآيات شاهدًا بيني وبين شعبي. فإذا أخطأوا، أغضب عليهم وأتذكر عصاي وأعفو عنهم حسب رحمتي، ويكون عصاك في عينيّ تذكارًا كل الأيام، ويكون كالقوس الذي عاهدت به نوحًا حين خرج من الفلك قائلًا: سأضع قوسي في السحاب، فيكون علامة بيني وبين الناس على أن مياه الطوفان لا تكون على الأرض بعد.

\par 12 لكني سآخذك من هنا، وأمنحك نومًا مع آبائك، وأريحك في نومك، وأدفنك بسلام، وستنوح عليك جميع الملائكة، وتحزن أجناد السماء. لكن لن يعرف أحد، من الملائكة أو البشر، قبرك الذي ستُدفن فيه، لكنك سترتاح فيه حتى أزور العالم، وأُقيمك أنت وآباءك من الأرض [مصر] التي ستنامون فيها، وتجتمعون معًا وتسكنون في مسكن خالد لا يخضع للزمن

\par 13 لكن هذه السماء ستكون في نظري كسحابة عابرة، ومثل الأمس بعد أن مضى، وستكون عندما أقترب لزيارة العالم، سأأمر بالسنين وأحاسب الأزمنة، وستُقصَّر، وستُعجَّل النجوم، وسيُسرِّع ضوء الشمس في الغروب، ولن يدوم ضوء القمر، لأني سأُعجِّل لإيقاظكم أيها النائمون، حتى يسكن فيه كل من يستطيع أن يعيش في مكان التقديس الذي أريتكم إياه

\par 14 فقال موسى: يا رب، إن سألتك شيئًا واحدًا أيضًا، فبحسب كثرة رحمتك، فلا تغضب عليّ. وأرني كم من الزمان مضى وما بقى

\par 15 فقال له الرب: لحظة، ورأس كف، وملء لحظة، وقطرة كأس. وقد استكمل الزمان كل شيء. لأن أربع ساعات ونصفًا قد مضت، وبقي اثنان ونصف

\par 16 فلما سمع موسى امتلأ فهمًا، وتغيرت صورته مجيدة، ومات في مجد حسب فم الرب، ودفنه كما وعده، وناح الملائكة على موته، وسارت أمامه بروق ومشاعل وسهام بصوت واحد. وفي ذلك اليوم لم يُنشد ترنيمة الجنود بسبب رحيل موسى. ولم يكن مثله يوم منذ خلق الرب الإنسان على الأرض، ولن يكون إلى الأبد يوم كهذا يوقف ترنيمة الملائكة بسبب إنسان؛ لأنه أحبه كثيرًا، ودفنه بيديه على مكان مرتفع من الأرض، وفي نور العالم كله

\chapter{20}

\par 1 وفي ذلك الوقت، قطع الله عهده مع يسوع بن نوح الذي بقي من الرجال الذين تجسسوا الأرض، لأنه وقعت عليهم القرعة أن لا يروا الأرض لأنهم تكلموا عليها بشر، ولهذا السبب مات ذلك الجيل

\par 2 ثم قال الله ليسوع بن نوح: لماذا تحزن، ولماذا ترجو عبثًا، ظانًّا أن موسى سيحيا؟ والآن أنت لا تنتظر شيئًا، لأن موسى قد مات. خذ ثياب حكمته والبسها، وشد حقويك بمنطقة معرفته، فتتغير وتصبح إنسانًا آخر. ألم أُكلِّم موسى عبدي عنك قائلًا: «هو سيقود شعبي وراءك، وأُسلِّم ملوك الأموريين إلى يده»؟

\par 3 فأخذ يسوع ثياب الحكمة ولبسها، وشد حقويه بمنطقة الفهم. ولما لبسها، ثار قلبه وثارت روحه، وقال للشعب: هوذا الجيل الأول مات في البرية لأنهم تكلموا على إلههم. والآن، اعلموا يا جميع القادة اليوم أنه إذا خرجتم في طرق إلهكم، تُقوّم سبلكم

\par 4 ولكن إن لم تطيعوا صوته، وكنتم مثل آبائكم، فستُنهب أعمالكم، وتتكسرون، ويبيد اسمكم من الأرض، فأين حينئذٍ الكلام الذي كلم به الله آباءكم؟ لأنه حتى لو قال الوثنيون: قد يكون الله قد فشل، لأنه لم ينقذ شعبه، إلا أنهم بينما يدركون أنه اختار لنفسه شعوبًا أخرى، صانعًا لهم عجائب عظيمة، سيفهمون أن القدير لا يقبل الأشخاص. ولكن لأنكم أخطأتم بالباطل، لذلك أخذ سلطانه منكم وأخضعكم. والآن قم واضبط قلبك للسير في طرق ربك، وهو سيهديكم

\par 5 فقال له الشعب: هوذا اليوم نرى ما تنبأ به إلداد ومودات في أيام موسى قائلين: بعد أن يستريح موسى، تُعطى رئاسة موسى ليسوع بن نوح. فلم يحسد موسى، بل فرح عندما سمعهم. ومنذ ذلك الحين آمن جميع الشعب بأنك ستقودهم وتقسم لهم الأرض بسلام. والآن أيضًا، إذا حدث نزاع، فكن قويًا وافعل بشجاعة، لأنك وحدك ستكون قائدًا في إسرائيل

\par 6 ولما سمع يسوع ذلك، فكّر في إرسال جواسيس إلى أريحا. فدعا قنز وسيناميا أخاه، ابني خليفة، وكلمهما قائلًا: أنا وأبوكما أرسلنا من قبل موسى إلى البرية وصعدنا مع عشرة رجال آخرين، فرجعوا وتكلموا بالسوء على الأراضي وأذابوا قلوب الشعب، فتشتّتوا وقلوب الشعب معهم. أما أنا وأبوكما فلم نتمّم إلا كلمة الرب، وها نحن أحياء اليوم. والآن أرسلكما لتتجسسا أرض أريحا. افعلا مثل أبيكما فتحيا أنتم أيضًا

\par 7 فصعدوا وتجسّسوا المدينة. ولما جاءوا بالخبر، صعد الشعب وحاصروا المدينة وأحرقوها بالنار

\par 8 وبعد وفاة موسى، انقطع المن عن بني إسرائيل، فبدأوا يأكلون من ثمار الأرض. وهذه هي الأشياء الثلاثة التي وهبها الله لشعبه من أجل ثلاثة أشخاص: بئر ماء مارة من أجل مريم، وعمود السحاب من أجل هارون، والمن من أجل موسى. ولما فرغت هذه الثلاثة، سُحبت منهم تلك الهدايا الثلاث.

\par 9 وحارب الشعب ويسوع الأموريين، ولما اشتدت المعركة ضد أعدائهم طوال أيام يسوع، انقرض 30 و9 ملوك كانوا يسكنون الأرض. وأعطى يسوع الأرض بالقرعة للشعب، لكل سبط حسب القرعة، كما أخذ الوصية

\par 10 ثم جاء إليه الخليفة وقال: أنت تعلم أننا أرسلنا بالقرعة من قبل موسى للذهاب مع الجواسيس، ولأننا أتممنا كلمة الرب، ها نحن أحياء اليوم. والآن، إن كان ذلك مقبولًا في عينيك، فليُعطَ لابني قنز حصة من أرض الأبراج الثلاثة (أو سبط) . فباركه يسوع وفعل ذلك

\chapter{21}

\par 1 ولما شاخ يسوع وتقدم في السن، قال له الله: ها أنت قد تقدمت في السن وتقدمت في الأيام، والأرض قد عظمت جدًا، وليس من يقسمها (أو يأخذها بالقرعة)، ويكون بعد ذهابك أن يختلط هذا الشعب بسكان الأرض ويضلون وراء آلهة أخرى، وسأتركهم كما شهدت في كلامي لموسى. ولكن اشهد لهم قبل أن تموت

\par 2 فقال يسوع: أنت تعلم يا رب أكثر من الجميع ما يحرك قلب البحر قبل أن يهيج، وقد حددت الأبراج وأحصيت النجوم ورتبت المطر. أنت تعرف عقول جميع الأجيال قبل أن تولد. والآن يا رب، أعط شعبك قلب حكمة وعقل فطنة، ويكون عندما تعطي هذه الفرائض لميراثك، أنهم لن يخطئوا أمامك ولن تغضب عليهم

\par 3 أليست هذه هي الكلمات التي تكلمت بها أمامك يا رب، حين سرق عخار اللعنة، وسُلم الشعب أمامك، وصليتُ أمامك وقلت: أليس خيرًا لنا يا رب لو متنا في البحر الأحمر الذي أغرقت فيه أعداءنا؟ أو متنا في البرية مثل آبائنا، من أن نُسلم إلى أيدي الأموريين فنباد إلى الأبد؟

\par 4 وإن كانت كلمتك عنا، فلن يصيبنا شر. فحتى وإن حلّ الموت بنا، فأنت حيّ قبل العالم وبعده. ولأن الإنسان لا يقدر أن يميّز جيلاً عن جيل، يقول: "أهلك الله شعبه الذي اختاره". وها نحن في جهنم. ومع ذلك، ستحيي كلمتك. والآن، فليكن لك ملء رحمتك صبر على شعبك، واختر لميراثك رجلاً يحكم شعبك، هو وجيله.

\par 5 ألم يكن لهذا تكلم أبونا يعقوب قائلًا: لا يزول رئيس من يهوذا، ولا قائد من صلبه. والآن أثبت الكلمات التي قيلت سابقًا، لكي تتعلم أمم الأرض وقبائل العالم أنك أنت أبدي

\par 6 وقال أيضًا: يا رب، ها أيام تأتي ويكون بيت إسرائيل كحمامة حاضنة تضع فراخها في العش ولا تتركها ولا تنسى مكانها. كذلك هؤلاء أيضًا يتراجعون عن أعمالهم ويحاربون الخلاص الذي يولد لهم

\par 7 فنزل يسوع من الجلجال وبنى مذبحًا من حجارة عظيمة جدًا، ولم يضع عليها حديدًا كما أمر موسى، وأقام حجارة عظيمة على جبل جبيل، وبيضها وكتب عليها كلمات الشريعة نقشًا واضحًا، وجمع كل الشعب وقرأ في مسامعهم جميع كلمات الشريعة

\par 8 ونزل معهم وقدم على المذبح ذبائح سلامة، فغنوا تسابيح كثيرة، ورفعوا تابوت عهد الرب من المسكن مع الدفوف والرقص والعود والقيثارات والرباب وكل آلات العزف العذب

\par 9 وكان الكهنة واللاويون يصعدون أمام التابوت ويفرحون بالمزامير، ووضعوا التابوت أمام المذبح، ورفعوا عليه أيضًا ذبائح سلامة كثيرة جدًا، وترنم كل بيت إسرائيل بصوت عظيم قائلين: هوذا ربنا قد أكمل ما تكلم به مع آبائنا قائلًا: لنسلك أعطي أرضًا للسكنى، أرضًا تفيض لبنا وعسلا. وها هو قد أدخلنا إلى أرض أعدائنا ودفعهم منكسري القلوب أمامنا، وهو الإله الذي أرسل إلى آبائنا في مخابئ النفوس قائلًا: هوذا الرب قد فعل كل ما تكلم به إلينا. والآن نعلم حقًا أن الله قد أقر جميع أقوال الشريعة التي كلمنا بها في غراب، فإذا حفظ قلبنا طرقه يكون لنا ولأبنائنا من بعدنا خير

\par 10 وباركهم يسوع وقال: ليعطِ الرب قلوبكم أن تثبتوا فيه كل الأيام، وإن لم تحيدوا عن اسمه، فإن عهد الرب يدوم معكم. وليعطِ ألا يفسد، بل أن يُبنى مسكن الله بينكم، كما قال حين أرسلكم إلى ميراثه بفرح وابتهاج



\chapter{22}

\par 1 وحدث بعد هذه الأمور أن يسوع وجميع إسرائيل سمعوا أن بني رأوبين وبني جاد ونصف سبط منسى الساكنين حول الأردن قد بنوا لأنفسهم مذبحاً وأصعدوا عليه ذبائح وعينوا كهنة للمقدس، فاضطرب كل الشعب جداً وجاءوا إليهم إلى سيلون.

\par 2 فكلمهم يسوع وجميع الشيوخ قائلين: ما هذه الأعمال التي تُصنع بينكم ونحن لم نستقر بعد في أرضنا؟ أليست هذه هي الكلمات التي كلّمكم بها موسى في البرية قائلاً: انظروا متى دخلتم الأرض لا تفسدون أعمالكم وتفسدون جميع الشعب؟ والآن لماذا كثر أعداؤنا إلا لأنكم تفسدون طرقكم وتسببون كل هذا البلاء، ولذلك سيجتمعون علينا ويغلبوننا

\par 3 فقال بنو رأوبين وبنو جاد ونصف سبط منسى ليسوع وجميع شعب إسرائيل: هوذا الآن قد وسع الله ثمرة بطن البشر، وأقام سراجًا ليبصر من في الظلمة، لأنه يعلم ما في خفايا الغمر، وعنده يستقر النور. والرب إله آبائنا يعلم إن كان أحد منا أو من أنفسنا قد فعلنا هذا الأمر إثمًا، ولكن من أجل ذريتنا فقط، لكي لا ينفصل قلبهم عن الرب إلهنا لئلا يقولوا لنا: هوذا الآن لإخوتنا الذين في عبر الأردن مذبح ليقدموا عليه ذبائح، وأما نحن في هذا المكان الذي ليس لنا مذبح، فلنبتعد عن الرب إلهنا، لأن إلهنا قد أبعدنا عن طرقه حتى لا نعبده

\par 4 ثم قلنا فيما بيننا: لنصنع لأنفسنا مذبحًا، حتى يجتهدوا في طلب الرب. وهناك من يقف بجانبنا ويعلم أننا إخوتكم، ونقف أبرياء أمامكم. فافعلوا ما يرضي الرب

\par 5 فقال يسوع: أليس الرب ملكنا أقوى من ذبائح الووشو؟ ولماذا لم تعلموا أبناءكم كلام الرب الذي سمعتموه منا؟ لأنه لو كان أبناؤكم مشغولين بالتأمل في شريعة الرب، لما انحرفت عقولهم وراء مقدس مصنوع بالأيادي. أم لا تعلمون أنه عندما تُرك الشعب لحظة في البرية عندما صعد موسى ليأخذ الموائد، انحرفت عقولهم وصنعوا لأنفسهم أصنامًا؟ ولولا رحمة إله آبائكم التي حفظتنا، لكانت جميع المجامع قد صارت مثلًا، ولأُحرقت جميع خطايا الشعب بسبب جهالةكم

\par 6 فالآن اذهبوا واهدموا المقادس التي بنيتموها لأنفسكم، وعلموا أبناءكم الشريعة، فيتأملون فيها نهارًا وليلًا، ليكون الرب معهم شاهدًا وقاضيًا لهم كل أيام حياتهم. ويكون الله شاهدًا وقاضيًا بيني وبينكم، وبين قلبي وقلبكم، أنه إن كنتم قد فعلتم هذا الأمر بمكر، فإنه يُنتقم منكم لأنكم تريدون إخوتكم، وإن كنتم قد فعلتموه بجهل كما تقولون، فإن الله يرحمكم من أجل أبنائكم. فأجاب جميع الشعب: آمين، آمين.

\par 7 فقرب يسوع وجميع شعب إسرائيل عنهم ألف كبش ذبيحة خطية (حرفيًا: كلمة تبرير)، وصلوا لأجلهم وأطلقوهم بسلام. فمضوا وهدموا المقدس، وصاموا وبكوا هم وبنوهم، وصلوا وقالوا: يا إله آبائنا، الذي يعلم قلوب كل الناس، أنت تعلم أن طرقنا لم تُصنع بالإثم أمام عينيك، ولم نحد عن طرقك، بل خدمناك كلنا، لأننا عمل يديك. والآن اذكر عهدك مع أبناء عبيدك

\par 8 وبعد ذلك صعد يسوع إلى الجلجال، وأقام مسكن الرب وتابوت العهد وجميع آنيته، ووضعه في صومعة، ووضع هناك البرهان والحق (أي الأوريم والتميم). وفي ذلك الوقت، كان ألعازار الكاهن الذي يخدم المذبح يُعلّم بالبرهان جميع الشعب الذين جاءوا ليسألوا الرب، لأنه بذلك أُظهر لهم. أما في المقدس الجديد الذي في الجلجال، فقد عيّن يسوع إلى هذا اليوم المحرقات التي كان بنو إسرائيل يُقدمونها كل سنة

\par 9 لأنه إلى أن بُني بيت الرب في أورشليم، وطالما كانت القرابين تُقدم في المقدس الجديد، لم يكن الشعب ممنوعًا من تقديم القرابين فيه، لأن الحق والبرهان كشفا كل شيء في شيلوه. وإلى أن وضع سليمان التابوت في شيلوه، استمروا في تقديم الذبائح هناك إلى ذلك اليوم. أما ألعازار بن هارون كاهن الرب، فكان يخدم في شيلوه



\chapter{23}

\par 1 وأمر يسوع بن نوح الشعب وقسم لهم الأرض، وكان جبار بأس. وبينما كان أعداء إسرائيل في الأرض، اقتربت أيام يسوع لموته، فأرسل ودعا جميع إسرائيل في كل أرضهم مع نسائهم وأولادهم، وقال لهم: اجتمعوا أمام تابوت عهد الرب في سيلوه، فأقطع معكم عهدًا قبل أن أموت

\par 2 "ولما اجتمع كل الشعب في اليوم السادس عشر من الشهر الثالث أمام وجه الرب في سيلو مع نسائهم وأولادهم، قال لهم يسوع: اسمع يا إسرائيل، ها أنا أقطع معك عهد هذه الشريعة التي وضعها الرب مع آبائنا في غراب، ولذلك امكثوا هنا الليلة وانظروا ماذا يقول لي الله عنكم."

\par 3 وفيما كان الشعب ينتظر هناك تلك الليلة، ظهر الرب ليسوع في رؤيا وتكلم قائلاً: بحسب جميع هذه الكلمات سأكلم هذا الشعب

\par 4 فجاء يسوع في الصباح وجمع كل الشعب وقال لهم: هكذا قال الرب: كانت هناك صخرة واحدة حفرت منها أباكم، فخرج من قطع تلك الصخرة رجلان اسمهما إبراهيم وناحور، ومن نحت ذلك المكان ولدت امرأتان اسمهما سارة وملكه. وسكنتا معًا في عبر النهر. فتزوج إبراهيم سارة، وتزوج ناحور ملكه

\par 5 ولما ضلّ شعب الأرض، كل واحدٍ حسب تدبيره، آمن بي إبراهيم ولم ينحرف وراءهم. فأنقذته من النار وأخذته وأدخلته إلى كل أرض كنعان. وكلمته في رؤيا قائلاً: لنسلك أعطي هذه الأرض. فقال لي: هوذا الآن قد أعطيتني امرأة وهي عاقر. فكيف يكون لي نسل من ذلك الرحم المحبوس؟

\par 6 فقلت له: خذ لي عجلاً ابن ثلاث سنين، وعنزة ابن ثلاث سنين، وكبشاً ابن ثلاث سنين، ويمامة وحمامة. فأخذها كما أمرته. وألقيت عليه سباتاً، وأحاطت به خوفاً، وجعلت أمامه موضع النار الذي يُنتقم فيه من أعمال الذين يذنبون إليّ، وأريته مشاعل النار التي بها يستنير الصالحون الذين آمنوا بي

\par 7 فقلت له: هذه ستكون شهادة بيني وبينك على أني أعطيك نسلًا من الرحم المغلق. وأشبهك بالحمامة، لأنك أخذت لي المدينة التي سيبدأ بنوكَ في بنائها أمامي. وأشبه اليمامة بالأنبياء الذين سيولدون منك. وأشبه الكبش بالحكماء الذين سيولدون منك والذين ينورون أبنائك. وأشبه العجل بجمهور الشعوب الذين سيكثرون منك. وأشبه العنزة بالنساء اللواتي سأفتح أرحامهن ويلدن. هذه ستكون شهادة بيننا على أني لن أتجاوز كلامي

\par 8 وأعطيته إسحاق، وصوّرته في بطن التي ولدته، وأمرته أن يردّه سريعًا ويردّه لي في الشهر السابع. ولهذا فإن كل امرأة تلد في الشهر السابع يحيا طفلها، لأني عليه دعوت مجدي، وأظهرت الدهر الجديد

\par 9 وأعطيت إسحاق يعقوب وعيسو، وأعطيت عيسو أرض سعير ميراثًا. ونزل يعقوب وبنوه إلى مصر. وأذلّ المصريون آباءكم كما تعلمون، فذكرت آباءكم وأرسلت موسى صديقي، فأنقذتهم من هناك وهزمت أعداءهم.

\par 10 وأخرجتهم بيدٍ عليا، وقادتهم عبر البحر الأحمر، وجعلت السحابة تحت أقدامهم، وأخرجتهم من العمق، وأدخلتهم إلى أسفل جبل سيناء، وأحنيت السماء ونزلت، وجمدت لهيب النار، وسدت ينابيع الغمر، وأعاقت مسار النجوم، وروضت صوت الرعد، وأطفأت ملء الريح، وانتهرت كثرة السحب، وأوقفت حركتها، وقاطعت عاصفة الجيوش، لكي لا أنقض عهدي، لأن كل شيء تحرك عند نزولي، وكل شيء تنشط عند مجيئي، ولم أدع شعبي يتشتت، بل أعطيتهم شريعتي، وأنرتهم، حتى إذا فعلوا هذه الأشياء يعيشون ويكون لهم طول الأيام ولا يموتون

\par 11 وأدخلتكم إلى هذه الأرض وأعطيتكم كرومًا. أنتم تسكنون في مدن لم تبنوها. وقد أكملت العهد الذي كلمت به آباءكم

\par 12 والآن إن أطعتم آباءكم، فسأضع قلبي عليكم إلى الأبد، وسأظللكم، ولن يحاربكم أعداؤكم بعد الآن، وستكون أرضكم مشهورة في كل العالم، وسيُختار نسلكم بين الشعوب، الذين سيقولون: هوذا الشعب الأمين؛ لأنهم آمنوا بالرب، لذلك خلصهم الرب وغرسهم. ولذلك سأغرسكم ككرمة شهية، وسأحكمكم كقطيع محبوب، وسأشرف على المطر والندى، فيشبعانكم كل أيام حياتكم

\par 13 ويكون في النهاية أن نصيب كل واحد منكم هو الحياة الأبدية، لكم ولنسلكم، وسأستقبل أرواحكم وأضعها في سلام، حتى يتم وقت الدهر، وأعيدكم إلى آبائكم وآباؤكم إليكم، فيعلمون على أيديكم أنه ليس عبثًا أني اخترتكم. هذه هي الكلمات التي كلمني بها الرب هذه الليلة

\par 14 فأجاب جميع الشعب وقالوا: الرب إلهنا، وإياه وحده نعبد. وصنع جميع الشعب وليمة عظيمة في ذلك اليوم، وجددوها لمدة 28 يومًا

\chapter{24}

\par 1 وبعد هذه الأيام، جمع يسوع بن نوح كل الشعب أيضًا، وقال لهم: «ها قد شهد الرب لكم اليوم: أشهدت لكم السماء والأرض أنكم إن واصلتم عبادة الرب تكونون له شعبًا خاصًا. وإن لم تعبدوه وأطعتم آلهة الأموريين الذين أنتم ساكنون في أرضهم، فقولوا ذلك اليوم أمام الرب واخرجوا. أما أنا وبيتي فنعبد الرب».

\par 2 فرفع جميع الشعب أصواتهم وبكوا قائلين: لعل الرب يحسبنا مستحقين، وخير لنا أن نموت في خوفه من أن نباد من الأرض

\par 3 وبارك يسوع بن نوح الشعب وقبلهم وقال لهم: لتكن كلماتكم رحمة أمام ربنا، فيرسل ملاكه ويحفظكم. اذكروني بعد موتي، واذكروا موسى خليل الرب. ولا تزول عنكم كلمات العهد الذي قطعه معكم كل أيام حياتكم. ثم أطلقهم، فانطلق كل واحد إلى ميراثه

\par 4 فاضطجع يسوع على فراشه، وأرسل ودعا فينحاس بن ألعازار الكاهن، وقال له: ها أنا أرى بعيني معصية هذا الشعب التي سيبدأون بها بالضلال. أما أنت فشدد يديك في الوقت الذي تكون فيه معهم. وقبله هو وأباه وبنيه وباركه، وقال: الرب إله آبائك يهدي طرقك وطرق هذا الشعب

\par 5 ولما فرغ من الكلام معهم، ضم رجليه إلى الفراش واضطجع مع آبائه. ووضع أبناؤه أيديهم على عينيه

\par 6 ثم اجتمع كل إسرائيل لدفنه، وناحوا عليه ناحةً عظيمة، وقالوا في نحيبهم: ابكوا على جناح هذا النسر السريع، لأنه طار عنا. وابكوا على قوة شبل هذا الأسد، لأنه مختبئ عنا. فمن يذهب الآن ويخبر موسى البار، أنه كان لنا قائد مثله أربعين سنة؟ وأتموا نوحهم ودفنوه بأيديهم في جبل أفرايم، وعاد كل واحد إلى خيمته. وبعد موت يسوع، استراحت أرض إسرائيل

\chapter{25}

\par 1 وطلب الفلسطينيون قتال رجال إسرائيل، فسألوا الرب وقالوا: أنصعد ونحارب الفلسطينيين؟ فقال لهم الله: إن صعدتم بقلب نقي فقاتلوا، وإن كان قلبكم نجسًا فلا تصعدوا. ثم سألوا أيضًا قائلين: كيف نعرف إن كانت قلوب جميع الشعب واحدة؟ فقال لهم الله: ألقوا قرعة بين أسباطكم، فيكون لكل سبط تدخل عليه القرعة أن يفرز قرعة واحدة، فتعرفون أي قلب هو نقي وأي قلب هو نجس

\par 2 فقال الشعب: لنُعيّن علينا أولًا أميرًا، فنُلْقِ قرعةً. فقال لهم ملاك الرب: عيّنوا. فقال الشعب: من نُعيّن جديرًا يا رب؟ فقال لهم ملاك الرب: ألقوا قرعةً على سبط كالب، فمن تظهر له القرعة يكون رئيسًا لكم. فألقوا قرعةً على سبط كالب، فخرجت على قنص، فأقاموه رئيسًا على إسرائيل.

\par 3 وقال كينيز للشعب: أحضروا أسباطكم إليّ واسمعوا كلمة الرب. فاجتمع الشعب، وقال لهم كينيز: أنتم تعلمون ما أوصاكم به موسى خليل الرب، ألا تتعدوا الشريعة يمينًا أو يسارًا. ويسوع أيضًا الذي بعده أوصاكم بنفس الوصية. والآن، ها قد سمعنا من فم الرب أن قلوبكم قد تنجست. وقد أوصانا الرب أن نلقي قرعة بين أسباطكم لنعرف أي قلب انحرف عن الرب إلهنا. ألا يأتي غضب الغضب على الشعب؟ ولكني أعدكم اليوم أنه حتى لو خرج رجل من بيتي في قرعة الخطيئة، فلن ينجو حيًا، بل سيُحرق بالنار. وقال الشعب: لقد تكلمت بنصيحة صالحة لتعمل بها

\par 4 فأُحضرت الأسباط أمامه، فوجدت من سبط يهوذا 345 رجلاً، ومن سبط رأوبين 560، ومن سبط شمعون 775، ومن سبط لاوي 150، ومن سبط زبولون 655 (أو 645)، ومن سبط يساكر 665، ومن سبط جاد 380، ومن سبط أشير 665، ومن سبط منسى 480، ومن سبط أفرايم 468، ومن سبط بنيامين 267. وكان جميع عدد الذين وُجدوا بقرعة الخطيئة 6110. فأخذهم كنِص جميعهم وأغلق عليهم السجن حتى يعلم ماذا يُفعل بهم

\par 5 فقال تشينيز: أليس عن هذا تكلم موسى خليل الرب قائلاً: فيكم أصل قوي يُخرج المرارة والسوء؟ الآن تبارك الرب الذي كشف جميع مكائد هؤلاء الرجال، ولم يدعهم يفسدون شعبه بأعمالهم الشريرة. فأحضروا الدليل والحق، واستدعوا أليعازار الكاهن، فلنسأل الرب بواسطته

\par 6 ثم صلى قنز وألعازار وجميع الشيوخ والمجمع كله بصوت واحد قائلين: أيها الرب إله آبائنا، أظهر لعبيدك الحقيقة، لأننا وجدنا غير مصدقين للعجائب التي صنعت لآبائنا منذ أخرجتهم من أرض مصر إلى هذا اليوم. فأجاب الرب وقال: اسألوا أولًا الذين وُجدوا، وليعترفوا بأعمالهم التي صنعوها بمكر، وبعد ذلك يُحرقون بالنار

\par 7 فأخرجهم قنز وقال لهم: "ها أنتم تعلمون كيف اعترف أخيار حين وقعت عليه القرعة، وأخبر بكل ما صنع. والآن أخبروني بكل شروركم وبدعكم. من يعلم إن صدقتمونا، حتى لو متم الآن، أن الله سيرحمكم عندما يحيي الموتى؟"

\par 8 فقال له واحد منهم اسمه إيلاس: ألا يأتي علينا الموت الآن فنموت بالنار؟ ومع ذلك أقول لك يا سيدي: ليس مثل هذه الاختراعات التي صنعناها شرًا. ولكن إن أردت أن تبحث عن الحقيقة بوضوح، فاسأل رجال كل سبط على حدة، وهكذا سيدرك أحد الواقفين منهم الفرق بين خطاياهم

\par 9 وسأل كنيز رجال سبطه، فقالوا له: أردنا أن نقتدي بالعجل الذي صنعوه في البرية ونصنعه. وبعد ذلك سأل رجال سبط رأوبين، فقالوا: أردنا أن نذبح لآلهة سكان الأرض. وسأل رجال سبط لاوي، فقالوا: نريد أن نختبر المسكن هل هو مقدس. وسأل بقية سبط يساكر، فقالوا: نريد أن نفحص بأرواح الأصنام الشريرة لنرى هل تكشف جليًا. وسأل رجال سبط زبولون، فقالوا: أردنا أن نأكل لحم أولادنا ونعلم هل يهتم الله بهم. وسأل بقية سبط دان، فقالوا: علمنا الأموريون ما فعلوا لنعلم أولادنا. وها هم مختبئون تحت خيمة إيلاس، الذي قال لك أن تسألنا فأرسل فتجدهم، فأرسل تشنيز فوجدهم.

\par 10 ثم سأل الذين بقوا من سبط جاد، فقالوا: زنا بعضنا مع نساء بعض. ثم سأل رجال سبط أشير، فقالوا: وجدنا سبعة تماثيل من ذهب سماها الأموريون الحوريات المقدسات، فأخذناها مع الأحجار الكريمة التي كانت عليها، وأخفيناها. وها هي الآن مدفونة تحت قمة جبل شكيم. فأرسل فتجدها. فأرسل كنيز رجالاً ونقلهم من هناك

\par 11 هؤلاء هنّ الحوريات اللواتي، حين دُعين، أظهرن للأموريين أعمالهنّ في كل ساعة. هؤلاء هم الذين اخترعهم سبعة أشرار بعد الطوفان، أسماؤهم: [? ] كنعان، فوث، سيلات، نمبروث، أيلة، دسواث. ولن يكون في العالم بعدُ مثيلٌ له نحته صانعٌ وزيّنه فنّانٌ متنوع، بل نُصب ورُكّب لتكريس (أي المكان المقدس؟) للأصنام. وكانت الحجارة ثمينة، جلبت من أرض إيويلاث، وكان من بينها بلورة وبراس (أو واحدة بلورية وأخرى خضراء)، وقد أظهرت شكلها، حيث تم نحتها على طريقة الحجر المثقوب بفتحة، وكان آخر منها محفورًا في الأعلى، وآخر كما لو كان مميزًا بالبقع (أو مثل الكريسوبراسي المرقط) فكان يلمع بنقوشه كما لو كان يظهر مياه الأعماق الواقعة تحته.

\par 12 وهذه هي الأحجار الكريمة التي كانت لدى الأموريين في مقدساتهم، وكان ثمنها باهظًا. لأنه عندما دخل أحد ليلًا، لم يكن بحاجة إلى ضوء فانوس، بل كان نور الحجارة الطبيعي يضيء. وكان الحجر الذي يُعطي أعظم نور هو الحجر المقطوع على شكل حجر مثقوب ومُنظف بشعيرات؛ لأنه إذا كان أحد الأموريين أعمى، كان يذهب ويضع عينيه عليه فيُبصر. ولما وجدها كينز، عزلها ووضعها حتى يعرف ماذا سيحدث لها

\par 13 وبعد ذلك سأل الذين بقوا من سبط منسى، فقالوا: إننا لم ندنس سبوت الرب إلا. وسأل المتروكين من سبط أفرايم، فقالوا: أردنا أن نمرر بنينا وبناتنا في النار، لنعلم هل ما قيل واضح؟ وسأل المتروكين من سبط بنيامين، فقالوا: أردنا في هذا الوقت أن نفحص سفر التوراة، هل كتب الله ما فيه بوضوح، أم أن موسى قد علمه من نفسه

\chapter{26}

\par 1 فأخذ تشينز كل هذا الكلام وكتبه في كتاب وقرأه أمام الرب، وقال له الله: خذ الرجال وما وجد معهم وكل أموالهم وضعها في مجرى نهر فيشون، وأحرقها بالنار لكي يكف غضبي عنهم

\par 2 فقال تشينيز: أنحرق هذه الأحجار الكريمة أيضًا بالنار، أم نقدسها لك، لأنه ليس بيننا مثلها؟ فقال له الله: إن قبل الله باسمه شيئًا من المحرمات، فماذا يفعل الإنسان؟ فالآن خذ هذه الأحجار الكريمة وكل ما وُجد، من كتب ورجال، وعندما تفعل ذلك بالرجال، خصص هذه الحجارة مع الكتب، لأن النار لا تحرقها، وبعد ذلك سأريك كيف يجب عليك إتلافها. أما الرجال وكل ما وُجد، فأحرقه بالنار. واجمع كل الشعب وقل لهم: هكذا يُفعل بكل من ينحرف قلبه عن إلهه.

\par 3 وعندما تلتهم النار هؤلاء الرجال، فإن الكتب والأحجار الكريمة التي لا تُحرق بالنار، ولا تُقطع بالحديد، ولا تُمحى بالماء، ضعوها على قمة الجبل بجانب المذبح الجديد؛ وسأأمر سحابة، فتذهب وتلتقط الندى وتسكبه على الكتب، وتمحو ما هو مكتوب فيها، لأنه لا يمكن محوها بماء آخر غير الذي لم يُخدم البشر قط. وبعد ذلك سأرسل برقي، فيحرق الكتب نفسها

\par 4 وأما الأحجار الكريمة، فسآمر ملاكي فيأخذها ويذهب ويطرحها في أعماق البحر، وأدفع العمق فيبتلعها، لأنها لا تستطيع البقاء في العالم لأنها تنجست بأصنام الأموريين. وسأأمر ملاكا آخر فيأخذ لي اثني عشر حجرًا من الموضع الذي أُخذت منه هذه السبعة؛ ومتى وجدتها في رأس الجبل الذي سيضعها فيه، فخذها وضعها على الكتف مقابل الاثني عشر حجرًا التي وضعها موسى فيها في البرية، وقدسها في الصدرة (حرفيًا: وحي) حسب الأسباط الاثني عشر. ولا تقل: كيف أعرف أي حجر أضع لكل سبط؟ ها أنا أخبرك باسم السبط مُقابل اسم الحجر، فتجد كليهما منحوتًا

\par 5 فذهب كينيز وأخذ كل ما وجد والرجال معه، وجمع كل الشعب مرة أخرى، وقال لهم: هوذا قد رأيتم جميع العجائب التي أرانا الله إياها إلى هذا اليوم، وهوذا حين طلبنا كل الذين دبروا الشر ضد الرب وضد إسرائيل، كشفهم الله حسب أعمالهم، والآن ملعون كل من يدبر ليفعل مثله بينكم أيها الإخوة. فأجاب جميع الشعب: آمين، آمين. ولما قال هذا، أحرق جميع الرجال بالنار، وكل ما وجد معهم، إلا الأحجار الكريمة

\par 6 وبعد ذلك أراد تشينيز أن يثبت ما إذا كانت الحجارة تُحرق بالنار، فألقى بها في النار. وكان الأمر كذلك، أنه عندما سقطت في النار، انطفأت النار على الفور. فأخذ تشينيز حديدًا ليكسرها، وعندما لمسها السيف انصهر حديدها؛ وبعد ذلك أراد على الأقل أن يمحو الكتب بالماء؛ ولكن حدث أن الماء عندما سقط عليها تجمد. وعندما رأى ذلك، قال: تبارك الله الذي صنع هذه العجائب العظيمة لبني البشر، وجعل آدم أول المخلوقات وأراه كل شيء؛ حتى أنه عندما أخطأ آدم بذلك، يجب أن ينكره على كل هذه الأشياء، لئلا إذا أظهرها للبشرية تكون لهم السيادة عليها.

\par 7 ولما قال ذلك، أخذ الكتب والحجارة ووضعها على رأس الجبل بجانب المذبح الجديد كما أمره الرب، وأخذ ذبيحة سلامة ومحرقات، وأصعد على المذبح الجديد ألفي ذبيحة، وقدمها كلها محرقة. وفي ذلك اليوم عملوا وليمة عظيمة هو وكل الشعب معًا

\par 8 وفعل الله في تلك الليلة ما قاله لتشينيز، إذ أمر سحابة، فذهبت وأخذت ندى من جليد الجنة وسكبته على الكتب ومحاها. وبعد ذلك جاء ملاك وأحرقها، وأخذ ملاك آخر الأحجار الكريمة وطرحها في قلب البحر، فملأ عمق البحر فابتلعها. وذهب ملاك آخر وأحضر اثني عشر حجرًا ووضعها في مكانها عند الموضع الذي أخذ منه تلك السبعة. ونقش عليها أسماء الأسباط الاثني عشر

\par 9 وقام تشينيز في الغد فوجد تلك الحجارة الاثني عشر على قمة الجبل حيث وضع تلك الحجارة السبعة. وكان نقشها كأنه رسم عليها شكل عيون

\par 10 وكان الحجر الأول، الذي كُتب عليه اسم سبط رأوبين، يشبه حجر السردين. وكان الحجر الثاني منقوشًا بسن (أو عاج)، ومنقوشًا عليه اسم سبط شمعون، وظهر فيه شبه حجر توباز؛ وعلى الحجر الثالث منقوشًا عليه اسم سبط لاوي، وكان يشبه الزمرد. أما الحجر الرابع فكان يُسمى بلورة، ومنقوشًا عليه اسم سبط يهوذا، وكان يشبه الجمشت. وكان الحجر الخامس أخضر، ومنقوشًا عليه اسم سبط يساكر، وكان لون حجر الياقوت فيه. وكان الحجر السادس منقوشًا كما لو كان منقوشًا (أو كحجر عقيق أحمر) مرقطًا بعلامات مختلفة، ومنقوشًا عليه سبط زبولون، وكان حجر اليشب شبيهًا به

\par 11 الحجر السابع أشرق النقش عليه، وظهر كأنه يحيط بمياه الغمر، وكتب عليه اسم سبط دان، وكان الحجر شبيهًا بالرباط. أما الحجر الثامن فكان مقطوعًا بالماس، وكتب عليه اسم سبط نبتاليم، وكان يشبه الجمشت. أما الحجر التاسع فكان نقشه مثقوبًا، وكان من جبل أوفير، وكتب عليه سبط جاد، وشُبّه بحجر عقيق. أما الحجر العاشر فكان نقشه مجوفًا، فأعطى صورة حجر تيمان، وكتب عليه سبط آشير، وشُبّه بحجر زبرجد. أما الحجر الحادي عشر فكان حجرًا مختارًا من لبنان، وكتب عليه اسم سبط يوسف، وشُبّه بحجر زبرجد. والحجر الثاني عشر قطع من مرتفع صهيون، وكان مكتوبا عليه سبط بنيامين، وكان حجر الجزع شبيها به.

\par 12 وقال الله لقنز: خذ هذه الحجارة وضعها في تابوت عهد الرب مع لوحي العهد اللذين أعطيتهما لموسى في غراب، فتكون هناك معها حتى يقوم ياعيل ليبني بيتًا باسمي، ثم يضعها أمامي على الكروبيم، فتكون أمام عيني تذكارًا لبيت إسرائيل

\par 13 ويكون عندما تكتمل خطايا شعبي، ويسيطر أعداؤهم على بيتهم، أني سآخذ هذه الحجارة وتلك مع الألواح، وأضعها في المكان الذي أُخرجت منه في البدء، وستكون هناك حتى أتذكر العالم، وأزور سكان الأرض. وحينئذ سآخذها وكثيرين غيرهم ممن هم أفضل منها، من ذلك المكان الذي لم تره عين ولم تسمع به أذن ولم يصعد إلى قلب بشر، حتى يحدث مثله في العالم، ولا يحتاج الأبرار إلى نور الشمس ولا إلى سطوع القمر، لأن نور الأحجار الكريمة سيكون نورهم

\par 14 فقام تشينيز وقال: انظروا إلى ما صنع الله من خير للبشر، وبسبب خطاياهم حُرموا منه جميعًا. والآن أعلم اليوم أن الجنس البشري ضعيف، وأن حياتهم ستُحسب لا شيء

\par 15 وقال ذلك، وأخذ الحجارة من مكانها، وبينما كان يأخذها، إذا بنور الشمس يسكب عليها، والأرض تشرق بنورها. ووضعها كينيز في تابوت عهد الرب مع اللوحين كما أمره، وهي هناك إلى هذا اليوم

\chapter{27}

\par 1 وبعد ذلك سلح من الشعب 300 ألف رجل وصعد لمحاربة الأموريين، فقتل في اليوم الأول 800 ألف رجل، وفي اليوم الثاني قتل نحو 500 ألف

\par 2 ولما كان اليوم الثالث تكلم رجال الشعب شراً على كينيز قائلين هوذا كينيز يضطجع في بيته وحده مع امرأته وسراريه ويرسلنا إلى الحرب لكي نهلك أمام أعدائنا.

\par 3 فلما سمع عبيد كينيز، أخبروه. فأمر رئيس خمسين، فأحضر منهم سبعة وثلاثين رجلاً ممن تكلموا عليه، وحبسهم في السجن

\par 4 وهذه أسماؤهم: لا وعوص، بتول، إيفال، دلومة، عناف، ديساك، بيساك، جثيل، عنائيل، عنازيم، نواح، كحك، بواق، عوبال، يبال، عينات، بيث، زيلوت، إيفور، عزات، ديساف، أبيدان، إيسار، موآب، دوصل، عزات، فلق، إيغات، صوفال، إليعاصور، عكار، زيبات، سبث، نيساخ، وصريع. ولما حبسهم رئيس الخمسين كما أمر كنز، قال كنز: عندما يصنع الرب خلاصًا لشعبه بيدي، فإني أعاقب هؤلاء الرجال

\par 5 وبعد أن قال ذلك، أمر تشينيز قائد الخمسين قائلاً: اذهب واختر من بين خدمي 300 رجل، ومثلهم من الخيول، ولا تدع أحدًا من الشعب يعرف الساعة التي سأخرج فيها إلى المعركة؛ بل فقط في أي ساعة سأخبرك، جهز الرجال ليكونوا مستعدين هذه الليلة

\par 6 فأرسل كنيز رسلًا جواسيس ليرى أين جمهور محلة الأموريين. فذهب الرسل وتجسسوا، فرأوا جمهور محلة الأموريين يتحرك بين الصخور متآمرًا للقدوم ومحاربة إسرائيل. فرجع الرسل وأخبروه بهذا الأمر. فقام كنيز ليلًا هو وثلاثمائة فارس معه، وأخذ بوقًا بيده وبدأ ينزل مع الثلاثمائة رجل. ولما اقترب من محلة الأموريين، قال لعبيده: امكثوا هنا، وأنا أنزل وحدي وأنظر محلة الأموريين. ويكون: إن نفخت بالبوق تنزلون، وإلا فانتظروني هنا

\par 7 "ونزل كنيز وحده وقبل أن ينزل صلى وقال: أيها الرب إله آبائنا، لقد أريت عبدك العجائب التي أعددتها لتصنعها بعهدك في الأيام الأخيرة، والآن أرسل إلى عبدك إحدى عجائبك فأهزم أعداءك، لكي يعلموا هم وجميع الأمم وشعبك أن الرب لا ينجّي بكثرة الجيش ولا بقوة الفرسان، حين يرون علامة الخلاص التي تصنعها لي اليوم (أو الفرسان، وأنك يا رب تصنع معي علامة الخلاص اليوم). ها أنا أستل سيفي من غمده فيتلألأ في محلة الأموريين، فإذا علم الأموريون أنني أنا كنيز، عرفت أنك أسلمتهم إلى يدي." وإن لم يدركوا أنني أنا، وظنوا أنه غيري، فأعلم أنك لم تسمع لي، بل أسلمتني إلى أعدائي. وإن أسلمت إلى الموت، فأعلم أن الرب لم يسمعني بسبب آثامي، بل أسلمني إلى أعدائي، ولكنه لا يبيد ميراثه بموتي.

\par 8 ثم انطلق بعد أن صلى، فسمع جمهور الأموريين يقولون: لنقم ونقاتل إسرائيل، لأننا نعلم أن حورياتنا المقدسات هناك بينهم، وسيسلمونهم إلى أيدينا

\par 9 فقام كنيز، لأن روح الرب لبسه ثوبًا، واستل سيفه، فلما أشرق نوره على الأموريين كالبرق الحاد، رأوه، فقالوا: أليس هذا سيف كنيز الذي كثر جرحانا؟ الآن قد صدق الكلام الذي تكلمنا به قائلين إن حورياتنا المقدسات قد سلمتهن إلى أيدينا. هوذا اليوم يكون وليمة للأموريين، حين يُسلم إلينا عدونا. فالآن قوموا وليتقلد كل واحد سيفه ويبدأ المعركة

\par 10 ولما سمع كنيز كلامهم، لبس روح القوة وتحول إلى رجل آخر، ونزل إلى معسكر الأموريين وبدأ يضربهم. فأرسل الرب أمام وجهه الملاك إنجثيل (أو جثيل)، الموكل على الأمور الخفية، والذي يعمل في الخفاء، وملاك قوة آخر يساعده. فضرب إنجثيل الأموريين بالعمى، حتى أن كل من رأى قريبه عدّهم أعداءه، فقتلوا بعضهم بعضًا. ورفع الملاك زروئيل، الموكل على القوة، ذراعي كنيز لئلا يلاحظوه. فضرب كنيز من الأموريين خمسة وأربعين ألف رجل، فقتلوا هم بعضهم بعضًا، فسقط خمسة وأربعون ألف رجل

\par 11 وكان تشينيز، بعد أن ضرب جمعًا غفيرًا، أن يخلع يده من سيفه، لأن مقبض السيف كان لا يمكن أن يخلع، وكانت يده اليمنى قد أخذت فيها قوة السيف

\par ثم هرب الذين بقوا من الأموريين إلى الجبال. لكن كينيز طلب كيف يطلق يده: فنظر بعينيه فرأى رجلاً من الأموريين هارباً، فأمسكه وقال له: أنا أعلم أن الأموريين ماكرون، فأرني الآن كيف أطلق يدي من هذا السيف فأطلقك. فقال الأموري: اذهب وخذ رجلاً من العبرانيين واقتله، وما دام دمه دافئاً فامسك يدك من تحت وخذ دمه، فتنطلق يدك. فقال كينيز: حي هو الرب، لو قلت: خذ رجلاً من الأموريين، لأخذت واحداً منهم وأبقيتك حياً. ولكن بما أنك قلت "من العبرانيين" لإظهار بغضك، فسيكون فمك عليك، وكما قلت كذلك أفعل بك. وبعد أن قال هذا قتله تشينيز، وبينما كان دمه لا يزال دافئًا، وضع يده تحته واستقبلها، فانطلقت.

\par 12 فذهب كينيز وخلع ثيابه، وألقى نفسه في النهر واغتسل، ثم صعد وبدل ثيابه، ورجع إلى غلمانه. فألقى الرب عليهم نومًا ثقيلًا في الليل، فناموا ولم يعلموا شيئًا من كل ما فعله كينيز. فجاء كينيز وأيقظهم من النوم، فنظروا بأعينهم ورأوا، وإذا الحقل ممتلئ جثثًا، فتعجبوا، ونظر كل واحد إلى قريبه. فقال لهم كينيز: ما بالكم تتعجبون؟ هل طرق الرب كطرق الناس؟ لأن الناس يغلبون الكثير، أما الله فما يقدره. ولذلك، إن كان الله قد أراد أن يعمل خلاصًا لهذا الشعب بيدي، فلماذا تتعجبون؟ قوموا وتتقلدوا سيوفكم على كل واحد، وسنعود إلى إخوتنا

\par 13 ولما سمع جميع إسرائيل بالخلاص الذي حدث على يدي قنص، خرج جميع الشعب بصوت واحد للقائه، وقالوا: مبارك الرب الذي أقامك رئيسًا على شعبه، وأثبت صحة ما قاله لك. الذي سمعناه بالكلام نراه الآن بأعيننا، لأن عمل كلمة الله واضح

\par 14 فقال لهم كينيز: اسألوا إخوتكم، فيخبروكم كم تعبوا معي في الحرب. فقال الرجال الذين معه: حي هو الرب، إننا لم نقاتل، ولم نعلم شيئًا، إلا عندما استيقظنا، فرأينا الحقل مليئًا بالجثث. فأجاب الشعب: الآن نعلم أنه عندما يعين الرب شعبه ليعمل خلاصًا، لا يحتاج إلى جمع، بل إلى تقديس فقط

\par 15 وقال تشينيز لرئيس الخمسين الذين حبسوا أولئك الرجال في السجن: أخرج هؤلاء الرجال حتى نسمع كلامهم. ولما أخرجهم قال لهم تشينيز: أخبروني ماذا رأيتم فيّ حتى تذمرتم به بين الشعب؟ فقالوا: لماذا تسألنا؟ لماذا تسألنا؟ فالآن فأمر أن نحرق بالنار، لأننا لا نموت من أجل هذه الخطيئة التي تكلمنا عنها الآن، بل من أجل تلك الخطيئة الأولى التي أخذ بها أولئك الرجال الذين احترقوا في خطاياهم؛ لأننا حينئذ رضينا بخطيئتهم قائلين: لعل الشعب لا يدركنا؛ وحينئذ نجونا من الشعب. ولكن الآن قد أصبحنا (بحق) عبرة علنية بخطايانا إذ وقعنا في الافتراء عليك. وقال تشينيز: إذا كنتم تشهدون على أنفسكم فكيف أشفق عليكم؟ فأمر كينيز أن يحرقوهم بالنار ويطرحوا رمادهم في الموضع الذي أحرقوا فيه جمهور الخطاة في وادي فيشون.

\par 16 وملك كنيز على شعبه سبعًا وخمسين سنة، وكان الرعب على جميع أعدائه كل أيامه

\chapter{28}

\par 1 ولما دنت أيام كينيز من وفاته، أرسل ودعا جميع الرجال (أو جميع الشيوخ)، والنبيين يابيس وفينحاس، وفينحاس بن ألعازار الكاهن، وقال لهم: هوذا الرب قد أراني جميع عجائبه التي أعدها ليصنعها لشعبه في الأيام الأخيرة

\par 2 والآن أقطع عهدي معكم اليوم أن لا تتركوا الرب إلهكم بعد رحيلي. لأنكم رأيتم كل العجائب التي أصابت الذين أخطأوا، وكل ما أعلنوه معترفين بخطاياهم من تلقاء أنفسهم، وكيف أهلكهم الرب إلهنا لأنهم تعدوا عهده. فالآن احفظوهم من بيتكم وأولادكم ، واثبتوا في طرق الرب إلهكم، لئلا يهلك الرب ميراثه

\par 3 فقال فينحاس بن ألعازار الكاهن: إذا أمرني قنز الرئيس والأنبياء والشعب والشيوخ، فإني أتكلم بكلمة سمعتها من أبي عند وفاته، ولا أسكت عن الوصية التي أوصاني بها عند قبض روحه. فقال قنز الرئيس والأنبياء: فليقل فينحاس أيضًا. هل يتكلم أحد آخر أمام الكاهن الذي يحفظ وصايا الرب إلهنا، والذي يخرج من فمه الحق، ومن قلبه سراج منير؟

\par 4 "ثم قال فينحاس: أوصاني أبي حين كان يحتضر قائلاً: هكذا تقول لبني إسرائيل حين يجتمعون إلى الجماعة: ظهر لي الرب قبل هذا بثالثة أيام في حلم في الليل وقال لي: انظر أنت وأبوك قبلك كيف تعبت من أجل شعبي ويكون بعد وفاتك أن يقوم هذا الشعب ويفسد طرقه ويبتعد عن وصاياي وأغضب عليهم غضباً شديداً. ومع ذلك سأتذكر الزمان الذي كان قبل الدهور حين لم يكن إنسان ولم يكن فيه إثم حين قلت أن العالم يجب أن يكون وأن الذين يأتون يمدحونني فيه وأغرس كرما عظيما وأختار منه غرسا وأرتبه وأدعوه باسمي فيكون لي إلى الأبد." ولكن بعد أن أفعل كل ما تكلمت به، فإن غرسي الذي دُعي باسمي لن يعرفني أنا زارعه، بل يفسد ثمرته ولا يعطيني ثمرها. هذه هي الأمور التي أوصاني أبي أن أكلم بها هذا الشعب.

\par 5 فرفع تشينيز صوته، والشيوخ، وكل الشعب بقلب واحد، وبكى بكاءً عظيمًا إلى المساء، وقال: هل يُهلك الراعي قطيعه عبثًا إن لم يستمر في إثمه؟ ألا يكون هو الذي يشفق حسب كثرة رحمته، إذ قد بذل علينا جهدًا كبيرًا؟

\par 6 وبينما هم جالسون، قفز عليه الروح القدس الذي كان يسكن في تشينيز، وأخذ منه إدراكه الجسدي، فبدأ يتنبأ قائلاً: ها أنا الآن أرى ما لم أكن أنتظره، وأدركت أني لم أكن أعرفه. اسمعوا الآن يا سكان الأرض، كما تنبأ الذين كانوا فيها قبلي، حين رأوا هذه الساعة، قبل أن تفسد الأرض، لكي تعرفوا النبوات المعينة من قبل، يا جميع سكانها

\par 7 ها أنا الآن أرى ألسنة لهب لا تحترق، وأسمع ينابيع ماء تستيقظ من النوم، وليس لها أساس، ولا أرى قمم الجبال، ولا قبة السماء، بل كل الأشياء غير الظاهرة وغير المرئية، التي ليس لها مكان على الإطلاق، وعلى الرغم من أن عيني لا تعرف ما تراه، فإن قلبي سيكتشف ما قد يتعلمه (أو يقوله).

\par 8 ومن اللهيب الذي رأيته ولم يحترق، نظرتُ، وإذا بشرارة خرجت، وكأنها بنيت لنفسها أرضية تحت السماء، وكان شكل أرضها كغزل العنكبوت، على شكل ترس. ولما وُضع الأساس، نظرتُ، ومن ذلك النبع ثار كزبد يغلي، وإذا به قد تحول كأنه أساس آخر؛ وبين الأساسين، العلوي والسفلي، اقترب من نور المكان غير المنظور كأشكال بشرية، وساروا ذهابًا وإيابًا. وإذا بصوت يقول: "يكون هؤلاء أساسًا للناس، ويسكنون فيه سبعة آلاف سنة".

\par 9 وكان الأساس السفلي رصيفًا، وكان العلوي زبدًا، والذين خرجوا من نور المكان غير المرئي هم الذين سيسكنون فيه، واسم ذلك الإنسان هو [آدم]. ويكون، عندما يخطئ (أو يخطئون) ضدي وينتهي الوقت، أن الشرارة ستنطفئ ويتوقف الينبوع، وهكذا سيتغيرون

\par 10 وبعد أن نطق كينيز بهذه الكلمات، استيقظ فعاد إليه رشده، لكنه لم يعلم ما نطق به ولا ما رآه، بل قال للشعب: إن كان باقي الصالحين كذلك بعد موتهم، فمن الأفضل لهم أن يموتوا للعالم الفاسد، حتى لا يروا خطيئة. ولما قال كينيز هذا، مات ونام مع آبائه، وندبه الشعب 30 يومًا



\chapter{29}

\par 1 وبعد هذه الأمور، جعل الشعب زبول رئيسًا عليهم، وفي ذلك الوقت جمع الشعب وقال لهم: هوذا قد عرفنا كل التعب الذي تعب به كنيز معنا في أيام حياته. لو كان له بنون لكانوا رؤساء على الشعب، ولكن بما أن بناته ما زلن على قيد الحياة، فليأخذن ميراثًا أعظم بين الشعب، لأن أباهن في حياته رفض أن يعطيهن إياه، لئلا يُدعى طماعًا وطامعًا في الربح. فقال الشعب: افعل كل ما هو مستقيم في عينيك

\par 2 وكان لكينيز ثلاث بنات، هذه أسماؤهن: إيثيما البكر، والثانية فيلة، والثالثة زلفة. فأعطى زبول البكر كل ما حول أرض الفينيقيين، والثانية أعطاها كرم زيتون عقرون، والثالثة كل الأرض المزروعة التي حول أشدود. وأعطاهن أزواجًا: البكر أليصافان، والثانية عديئيل، والثالثة دوئيل

\par 3 في تلك الأيام، أقام زبول خزانة للرب، وقال للشعب: "انظروا، إن أراد أحد أن يقدس للرب ذهبًا وفضة، فليأتِ بهما إلى خزانة الرب في سيلوا. ولكن لا يخطر ببال أحد ممن عنده أشياء للأوثان أن يقدسها لخزائن الرب، لأن الرب لا يريد رجاسات المحرمات، لئلا تعكروا صفو مجمع الرب، لأنه يكفي الغضب العابر". فأتى كل الشعب بما حثتهم قلوبهم، رجالًا ونساءً، من ذهب وفضة. ووُزن كل ما أُحضر، فكان عشرون وزنة من الذهب، ومائتين وخمسين وزنة من الفضة.

\par 4 وقضى زبول للشعب خمسًا وعشرين سنة. ولما أكمل مدته، أرسل ودعا جميع الشعب وقال: ها أنا ذا أمضي لأموت. انظروا إلى الشهادات التي شهد بها الذين سبقونا، ولا يكن قلبكم كأمواج البحر، بل كما أن موج البحر لا يحتمل إلا ما في البحر فقط، هكذا لا يفكر قلبكم أيضًا إلا فيما يتعلق بالناموس فقط. واضطجع زبول مع آبائه، ودُفن في قبر أبيه

\chapter{30}

\par 1 حينئذ لم يكن لبني إسرائيل من يقيمون عليه قاضيًا، فسقط قلبهم ونسوا الوعد، وتعدوا في الطريق الذي أوصاهم به موسى ويسوع عبدا الرب، وانساقوا وراء بنات الأموريين وعبدوا آلهتهم

\par 2 فغضب الرب عليهم، وأرسل ملاكه وقال: ها أنا ذا اخترت لنفسي شعبًا واحدًا من جميع قبائل الأرض، وقلت إن مجدي يجب أن يبقى معهم في هذا العالم، وأرسلت إليهم موسى عبدي، لأُخبرهم بعظمة جلالي وأحكامي، وقد تعدوا طرقي. والآن ها أنا أُثير أعداءهم فيتسلطون عليهم، ويقول جميع الشعوب: لأننا تعدينا طرق الله وآبائنا، لذلك أتت علينا هذه الأمور. ولكن ستتسلط عليهم امرأة تُنير لهم أربعين سنة

\par 3 وبعد هذه الأمور، أثار الرب عليهم يابين ملك حاسور، فبدأ بمحاربتهم، وكان معه رئيس جبروته سيسرا، وكان معه 8000 مركبة من حديد. وجاء إلى جبل أفرام وحارب الشعب، فخافه إسرائيل خوفًا شديدًا، ولم يستطع الشعب الصمود كل أيام سيسرا

\par 4 ولما ذلت إسرائيل جدًا، اجتمع جميع بني إسرائيل معًا على جبل يهوذا وقالوا: كنا نسمي أنفسنا مباركين أكثر من جميع الشعوب، والآن، ها نحن قد ذللت أكثر من جميع الأمم، حتى أننا لا نستطيع أن نسكن في أرضنا، وأعداؤنا يتسلطون علينا. والآن من فعل بنا كل هذا؟ أليست آثامنا لأننا تركنا الرب إله آبائنا وسلكنا في أمور لا تنفعنا؟ فالآن فلنصم سبعة أيام، رجالًا ونساءً، من الصغير إلى الرضيع. من يعلم هل يرضى الله عن ميراثه فلا يهلك غرس كرمه؟

\par 5 وبعد أن صام الشعب سبعة أيام، وجلسوا في المسوح، أرسل الرب إليهم في اليوم السابع دبورة، فقالت لهم: هل تستطيع الشاة المعينة على الذبح أن تجيب أمام من يذبحها، عندما يصمت كل من القاتل [...] والمذبوح، عندما يُستفز أحيانًا؟ الآن وُلدتم لتكونوا قطيعًا أمام ربنا. وقادكم إلى علو السحاب، وأخضع ملائكة تحت أقدامكم، ووضع لكم شريعة، وأعطاكم وصايا بالأنبياء، ووبخكم بالحكام، وأظهر لكم عجائب ليست بقليلة، ومن أجلكم أمر الأنوار فوقفوا في الأماكن التي أُمروا بها، وعندما جاء أعداؤكم أمطرهم بحجارة برد وأهلكهم، وأعطاكم موسى وعيسى وزبول وصايا. فلم تطيعواها

\par 6 لأنه بينما كانوا يعيشون، كنتم تظهرون أنفسكم كأنكم مطيعون لإلهكم، ولكن عندما ماتوا، مات قلبكم أيضًا. وأصبحتم مثل الحديد الذي يُلقى في النار، الذي عندما ينصهر باللهب يصير كالماء، ولكن عندما يخرج من النار يعود إلى صلابته. هكذا أنتم أيضًا، بينما يحرقكم أولئك الذين ينصحونكم، تظهرون التأثير، وعندما يموتون تنسى كل شيء

\par 7 والآن هوذا الرب يرحمكم اليوم، ليس لأجلكم، بل لأجل عهده الذي قطعه مع آبائكم، ولأجل قسمه الذي أقسم أنه لا يترككم إلى الأبد. ولكن اعلموا أنكم بعد وفاتي ستبدأون بالخطيئة في آخر أيامكم. لذلك سيصنع الرب بينكم عجائب، ويدفع أعداءكم إلى أيديكم. لأن آباءكم قد ماتوا، ولكن الله الذي قطع معهم العهد هو الحياة

\chapter{31}

\par 1 فأرسلت دبّورة ودعت باراخ وقالت له: قم وشد أزرك كرجل، وانزل وحارب سيسارا، فإني أرى الأبراج تتحرك في صفوفها وتستعد للقتال من أجلك. وأرى أيضًا البروق ثابتة في مساراتها، وهي تنطلق لتوقف عجلات مركبات أولئك الذين يفتخرون بقوة سيسارا، الذي يقول: سأنزل بذراع قوتي لأحارب إسرائيل، وسأقسم غنائمهم على عبيدي، وسأتخذ نسائهم الجميلات سراري. لذلك تكلم الرب عنه أن ذراع امرأة ضعيفة ستتغلب عليه، وستأخذ العذارى غنائمه، وهو أيضًا سيقع في أيدي امرأة.

\par 2 ولما نزلت دبورة والشعب وباراخ لملاقاة أعدائهم، زعزع الرب في الحال مسارات نجومه، وكلمهم قائلاً: أسرعوا واذهبوا، لأن أعداءنا (أو أعدائكم) ينقضون عليكم. فبلبلوا أذرعهم واكسروا قوة قلوبهم، لأني جئت لكي ينتصر شعبي. لأنه وإن أخطأ شعبي، فإني سأرحمهم. ولما قيل هذا، خرجت النجوم كما أُمرت وأحرقت أعداءها. وكان عدد الذين جُمعوا (أو أُحرقوا) وقُتلوا في ساعة واحدة سبعة وتسعين ألف رجل. أما سيسرا فلم يُهلكوها، لأنه هكذا أُمروا

\par 3 ولما هرب سيسرا على حصانه لينقذ روحه، تزينت ياهيل امرأة عابر القينائي بزينتها وخرجت للقائه. وكانت المرأة الآن جميلة جدًا. ولما رأته قالت: ادخل وخذ طعامًا ونم. وفي المساء سأرسل خدمي معك، لأني أعلم أنك ستتذكرني وتكافئني. فدخل سيسرا، ولما رأى الورود متناثرة على السرير، قال: إن نجوت يا ياهيل، فسأذهب إلى أمي وستكونين (أو ستكون ياهيل) زوجتي

\par 4 وبعد ذلك عطش سيسرا، فقال لياهل: أعطني قليلًا من الماء، فأنا منهك وروحي تحترق من شدة اللهيب الذي رأيته في النجوم. فقال له ياهل: استرح قليلًا ثم اشرب

\par 5 ولما نام سيسرا، ذهبت ياهيل إلى الغنم وحلبت منها حليبًا. وبينما هي تحلب قالت: انظر، اذكر يا رب، حين فرقت كل قبيلة وأمة على الأرض، ألم تختر إسرائيل وحده، ولم تشبهه بأي حيوان إلا بالكبش الذي يسير أمام الغنم ويقودها؟ فانظر كيف فكّر سيسرا في قلبه قائلًا: سأذهب وأعاقب غنم القدير. وها أنا آخذ من لبن البهائم التي شبهت بها شعبك، وأذهب وأسقيه، فإذا شرب ضعف، وبعد ذلك أقتله. وهذه هي العلامة التي تعطيني إياها يا رب: إذا استيقظ سيسرا وأنا نائم، فعندما أدخل، وطلب مني فورًا قائلًا: أعطني ماءً لأشرب، فأعلم أن صلاتي قد استُجيبت.

\par 6 فرجع ياهيل ودخل، فاستيقظ سيسارا وقال لها: اسقيني، فإني أحترق بشدة ونفسي ملتهبة. فأخذ ياهيل خمرًا وخلطها باللبن وسقاه، فشرب ونام

\par 7 فأخذت ياهيل وتدًا بيدها اليسرى وتقدمت إليه قائلة: إن أعطاني الرب هذه العلامة، فأعلم أن سيسارا سيسقط في يدي. ها أنا أطرحه على الأرض عن السرير الذي ينام عليه، وإن لم يشعر، فأعلم أنه قد أُسلم. فأخذ ياهيل سيسارا ودفعه عن السرير إلى الأرض، لكنه لم يشعر، لأنه كان منهكًا جدًا. فقال ياهيل: شدد ذراعي يا رب اليوم من أجلك ومن أجل شعبك ومن أجل الذين يتوكلون عليك. فأخذ ياهيل الوتد ووضعه على صدغه وضربه بالمطرقة. ولما مات، قال سيسارا لياهيل: هوذا قد أصابني وجع يا ياهيل، وأموت كامرأة فقال له ياعيل: اذهب وافتخر أمام أبيك في الجحيم، وأخبره أنك وقعت في يد امرأة، فقتلته ووضعت جثته هناك حتى يعود باراخ.

\par 8 وكانت أم سيسارا تُدعى ثيمك، فأرسلت إلى صديقاتها قائلة: هلموا نخرج معًا للقاء ابني، فترين بنات العبرانيين اللواتي سيحضرهن ابني إلى هنا سراريه

\par 9 فرجع باراخ من اتباع سيسرا، واغتاظ بشدة لأنه لم يجده، فخرج ياعيل للقائه وقال: تعال، ادخل يا مبارك الله، فأنقذك عدوك الذي اتبعته ولم تجده. فدخل باراخ ووجد سيسرا ميتًا، فقال: تبارك الرب الذي أرسل روحه وقال: في يدي امرأة يُسلم سيسرا. ولما قال ذلك، قطع رأس سيسرا وأرسله إلى أمه، وأعطاها رسالة تقول: خذي ابنك الذي كنت تنتظرينه ليأتي بالغنيمة



\chapter{32}

\par 1 فغنت دبورة وباراخ بن أبينو وكل الشعب معًا ترنيمة للرب في ذلك اليوم قائلين: هوذا الرب من العلاء أراه مجده كما فعل سابقًا حين أرسل صوته ليبلبل ألسنة البشر. فاختار أمتنا، وأخذ إبراهيم أبانا من النار، واختاره من بين جميع إخوته، وحفظه من النار، وأنقذه من لبنات بناء البرج، ورزقه ابنًا في آخر شيخوخته، وأخرجه من الرحم العاقر، فغار عليه كل الملائكة، وحسده رؤساء الجيوش

\par 2 وحدث لما حسدوه أن الله قال له: اذبح لي ثمرة بطنك وقدم عني ما أعطيتك. فلم يعارضه إبراهيم وانطلق للوقت. وفيما هو خارج قال لابنه: ها أنا ذا يا ابني أقدمك محرقة وأسلمك إلى يديه الذي أعطانيك

\par 3 فقال الابن لأبيه: اسمع لي يا أبي. إن كان خروف من الغنم مقبولاً قربانًا للرب رائحة طيبة، وإن كان الغنم تُذبح لأجل آثام الناس، ولكن الإنسان وُضع ليرث العالم، فكيف تقول لي الآن: تعالَ ورث حياةً آمنة، ووقتًا لا يُقاس؟ ماذا لو لم أولد في العالم لأُقدم ذبيحة للذي خلقني؟ وستكون لي سعادة تفوق كل الناس، لأنه لن يكون شيء آخر مثله، وفيّ تتأدب الأجيال، وبيّ تفهم الشعوب أن الرب قد حسب نفس الإنسان مستحقة أن تكون ذبيحة له

\par 4 ولما قدمه أبوه على المذبح وربط قدميه ليقتله، أسرع القدير وأرسل صوته من العلى قائلاً: لا تقتل ابنك ولا تميت ثمرة بطنك، لأني الآن أظهرت نفسي لأظهر للذين لا يعرفونني، وأسد أفواه الذين يتكلمون عليك بالشر دائمًا. ويكون ذكرك أمامي إلى الأبد، واسمك واسم ابنك هذا من جيل إلى جيل

\par 5 وأعطى إسحاق ابنين، وكانا أيضًا من رحم مغلق، لأن أمهما كانت في ذلك الوقت في السنة الثالثة من زواجها. ولا يكون هكذا مع امرأة أخرى، ولا تفتخر امرأة تقترب من زوجها في السنة الثالثة. وولد له ابنان، يعقوب وعيسو. وأحب الله يعقوب، وأبغض عيسو بسبب أعماله

\par 6 وحدث في شيخوخة أبيهم أن إسحاق بارك يعقوب وأرسله إلى بلاد ما بين النهرين، فولد هناك 12 ابنًا، فنزلوا إلى مصر وأقاموا هناك

\par 7 "ولما عاملهم أعداؤهم بالسوء، صرخ الشعب إلى الرب، فاستجيبت صلاتهم، فأخرجهم من هناك، وقادهم إلى جبل سيناء، وأخرج لهم أساس الفهم الذي أعده منذ نشأة العالم؛ ثم تحرك الأساس، وأطلقت الجيوش البرق في مساراتها، ودوّت الرياح من مخازنها، وتحركت الأرض من أساسها، وارتجفت الجبال والصخور في أوتارها، ورفعت السحب أمواجها ضد لهيب النار حتى لا يلتهم العالم.

\par 8 حينئذٍ استيقظ العمق من ينابيعه، واجتمعت كل أمواج البحر. حينئذٍ أخرجت الجنة نسمة ثمارها، وانتزع أرز لبنان من جذوره. وارتعبت وحوش الحقل في مساكن الغابات، واجتمعت كل أعماله لتشهد الرب حين قطع عهدًا مع بني إسرائيل. وكل ما قاله القدير، هذا ما رآه، وشاهده موسى حبيبه

\par 9 ولما كان يحتضر، عيّن الله له الجلد، وأراه هؤلاء الشهود الذين لدينا الآن، قائلاً: لتكن السماء التي دخلتها والأرض التي سرت فيها حتى الآن شاهدة بيني وبينك وبين شعبي. لأن الشمس والقمر والنجوم ستكون لنا (أو لكم).

\par 10 ولما قام يسوع ليحكم الشعب، حدث في اليوم الذي كان يحارب فيه الأعداء أن المساء اقترب، والقتال لا يزال شديدًا، فقال يسوع للشمس والقمر: أيها الوزراء الذين عُيّنوا بين القدير وأبنائه، ها هي المعركة مستمرة، فهل تتركون مناصبكم؟ فاثبتوا اليوم وأعطوا أبنائه نورًا، واجعلوا الظلمة على أعدائنا. ففعلوا ذلك

\par 11 والآن في تلك الأيام قام سيسرا ليجعلنا عبيدًا له، فصرخنا إلى الرب إلهنا، فأمر النجوم وقال: اخرجوا من صفوفكم، وأحرقوا أعدائي، لكي يعرفوا قوتي. فنزلت النجوم وقلبت معسكرهم، وحفظتنا سالمين دون عناء

\par 12 لذلك لن نكف عن التسبيح، ولن تصمت أفواهنا عن الحديث عن عجائبه. لأنه تذكر وعوده الجديدة والقديمة، وأرانا خلاصه. ولذلك تفتخر ياهيل بين النساء، لأنها وحدها هي التي أوصلت هذا الطريق الصالح إلى النجاح، إذ قتلت سيسرا بيديها

\par 13 يا أرض، انطلقي، انطلقي أيتها السماوات والبروق، انطلقي أيها الملائكة والجنود، وأخبري الآباء في خزائن نفوسهم، وقولوا: إن القدير لم ينسَ أقل الوعود التي قطعها لنا، قائلاً: سأصنع عجائب كثيرة لأبنائكم. ومن الآن فصاعدًا، سيُعلم أن كل ما قاله الله للبشر أنه سيفعله، سيفعله، حتى لو مات الإنسان.

\par 14 سبحي، سبحي يا دبور (أو إن تأخر الإنسان عن سبح الله، فسبحي أنتِ يا دبور)، ولتستيقظ فيكِ نعمة الروح القدس، وابدئي بتسبيح أعمال الرب: لأنه لن يقوم بعد ذلك يومٌ تحمل فيه النجوم أخبارًا وتتغلب على أعداء إسرائيل كما أُمرت. من الآن فصاعدًا، إذا وقع إسرائيل في مأزق، فليدع هؤلاء شهوده مع خدامهم، وليذهبوا في رحلة إلى العلي، وسيتذكر هذا اليوم، وسيرسل خلاصًا لعهده

\par 15 وأنتِ يا دبوره، ابدئي بالحديث عما رأيتِه في الحقل: كيف سار الشعب وخرج سالمًا، وقاتلت النجوم من جانبهم (أو كيف سارت النجوم وقاتلت كما سارت الشعوب). افرحي أيتها الأرض على سكانك، لأن فيكِ معرفة الرب الذي يبني حصنه فيكِ. لأنه كان من العدل أن يأخذ الله منكِ ضلع الذي جُبل أولًا، عالمًا أنه من ضلعه يولد إسرائيل. ويكون تكوينكِ شهادة على ما صنع الرب لشعبه

\par 16 تمهلي يا ساعات النهار، ولا تسرعي، حتى نخبر بما يمكن أن يأتي به فهمنا، لأن الليل سيأتي علينا. ويكون كالليلة التي ضرب فيها الله أبكار المصريين من أجل أبكاره

\par 17 وحينئذٍ سأكف عن ترنيمتي لأن الوقت سيُعجَّل (أو يُهيَّأ) لأبراره. لأني سأُرنِّم له كما في تجديد الخليقة، وسيتذكر الشعب هذا الخلاص، وسيكون شهادة لهم. فليشهد البحر أيضًا مع أعماقه، لأن الله لم يجففه أمام آبائنا فحسب، بل قلب أيضًا المعسكر من مغربه وهزم أعدائنا

\par 18 ولما فرغت دبور من كلامها، صعدت مع الشعب معًا إلى سيلو، فقدّموا ذبائح ومحرقات ونفخوا في الأبواق العريضة. ولما نفخوا وقدموا الذبائح، قالت دبور: «هذا يكون شهادة للأبواق بين النجوم وربها».

\chapter{33}

\par 1 ونزلت دبّورة من هناك، وقضَت لإسرائيل أربعين سنة. ولما دنا يوم وفاتها، أرسلت وجمعت كل الشعب وقالت لهم: اسمعوا يا شعبي. ها أنا أُنذركم كامرأة من الله، وأُنير لكم كواحدة من النساء؛ أطيعوني الآن كأمكم، وأصغوا إلى كلامي كرجالٍ سيموتون.

\par 2 ها أنا ذاهب لأموت في طريق كل بشر، الذي ستسلكونه أنتم أيضًا. فقط وجِّه قلبك إلى الرب إلهك في وقت حياتك، لأنه بعد موتك لن تتمكن من التوبة عما أنت عليه

\par 3 لأن الموت قد خُتم الآن، واكتمل، والمقدار والوقت والسنين قد ردت ما أُسلم إليها. لأنه حتى لو سعيتم لفعل الشر في الجحيم بعد موتكم، فلن تتمكنوا من ذلك، لأن رغبة الخطيئة ستنتهي، وستفقد الخليقة الشريرة قوتها، والجحيم، الذي يستقبل ما أُسلم إليه، لن يعيده إلا إذا طالب به من ارتكبه. والآن، يا أبنائي، أطيعوا صوتي ما دام لديكم وقت الحياة ونور الشريعة، وأصلحوا طرقكم

\par 4 وعندما تكلمت دبوره بهذه الكلمات، رفع جميع الشعب أصواتهم معًا وبكوا قائلين: ها أنت ذا يا أمي تموتين وتتركين أبناءك، ولمن تعهدينهم؟ صلي إذن من أجلنا، وبعد رحيلك ستذكرنا روحك إلى الأبد

\par 5 فأجابت دبوره وقالت للشعب: ما دام الإنسان حيًا، فإنه يستطيع أن يصلي من أجل نفسه ومن أجل أبنائه، ولكن بعد وفاته لن يكون قادرًا على التوسل إلى أحد أو تذكر أحد. لذلك لا تعتمدوا على آبائكم، لأنهم لن ينفعوكم ما لم توجدوا مثلهم. فحينئذٍ يكون شبهكم كنجوم السماء التي تجلت لكم في هذا الوقت

\par 6 وماتت دبوره واضطجعت مع آبائها ودُفنت في مدينة آبائها، وناح عليها الشعب سبعين يومًا. وبينما كانوا ينوحون عليها، كانوا ينوحون قائلين: هوذا أم قد هلكت من إسرائيل، وقديسة تولت الحكم في بيت يعقوب، التي ثبتت سياج جيلها، وسيطلبها جيلها. وبعد وفاتها استراحت الأرض سبع سنين

\chapter{34}

\par 1 وفي ذلك الوقت، صعد رجل يُدعى عود من كهنة مديان، وكان ساحرًا، وكلم إسرائيل قائلًا: لماذا تصغون إلى شريعتكم؟ تعالوا فأريكم شيئًا ليس في شريعتكم. فقال الشعب: ماذا تُرينا مما ليس في شريعتنا؟ فقال للشعب: هل رأيتم الشمس ليلًا قط؟ فقالوا: لا. فقال: متى شئتم، أُريكم إياه، لتعلموا أن لآلهتنا سلطانًا، ولا تخدع عبادها. فقالوا: أرنا

\par 2 ثم انطلق وكان يصنع سحره ويأمر الملائكة الموكلين على السحر لأنه كان يذبح لهم زمانا طويلا.

\par 3 [لأن هذا كان سابقًا في سلطة الملائكة] وكان الملائكة يقومون به قبل أن يُدانوا، وكانوا سيدمرون العالم الذي لا يُقاس؛ ولأنهم تجاوزوا الحدود، فقد لم يعد للملائكة السلطة. لأنه عندما حُكم عليهم، لم يُسلم السلطة إلى الباقين: وبهذه الآيات (أو القوى) يعملون الذين يخدمون البشر بالسحر، إلى أن يأتي الدهر الذي لا يُقاس

\par 4 وفي ذلك الوقت، أراه داود بفن السحر الشمس ليلاً. فدهش الشعب وقالوا: انظروا، ما أعظم ما تستطيع آلهة المديانيين أن تفعله، ونحن لم نعلم!

\par 5 وأراد الله أن يمتحن إسرائيل هل ما زالوا في إثم، فاحتمل الملائكة، ونجح عملهم، وانخدع بنو إسرائيل وبدأوا يخدمون آلهة المديانيين. وقال الله: سأسلمهم إلى أيدي المديانيين، لأنهم قد انخدعوا بهم. فسلمهم إلى أيديهم، وبدأ المديانيون يستعبدون إسرائيل

\chapter{35}

\par 1 وكان جدعون بن يوآت، أعظم رجل بين جميع إخوته. ولما كان وقت الصيف، جاء إلى الجبل ومعه حزم ليدرسها هناك، وينجو من المديانيين الذين كانوا يضغطون عليه. فالتقاه ملاك الرب وقال له: من أين أتيت وأين تدخل؟

\par 2 قال له: لماذا تسألني من أين أتيت؟ لأن الضيق يحيط بي، وقد سقط إسرائيل في ضيق، وقد أُسلموا حقًا إلى أيدي المديانيين. وأين العجائب التي أخبرنا بها آباؤنا قائلين: إن الرب اختار إسرائيل وحده دون جميع شعوب الأرض؟ ها هو الآن قد أسلمنا، ونسي الوعود التي قطعها لآبائنا. لأنه كان ينبغي لنا أن نختار أن نُسلم إلى الموت مرة واحدة وإلى الأبد، على أن يُعاقب شعبه هكذا مرارًا وتكرارًا

\par 3 فقال له ملاك الرب: ليس عبثًا أن تُسلَّم، بل بتدبيرك جلبت عليك هذه الأمور، لأنكم تركتم مواعيد الرب التي أخذتموها، فأصابتكم هذه الشرور، ولم تذكروا وصايا الله التي أوصاكم بها الذين كانوا قبلكم. لذلك صرتم في سخط إلهكم. لكنه سيرحمكم كما لا يرحم أحد حتى شعب إسرائيل، وليس ذلك من أجلكم، بل من أجل الذين سقطوا.

\par 4 والآن تعالَ أرسلك، فتُخلّص إسرائيل من يد المديانيين. لأنه هكذا قال الرب: وإن لم يكن إسرائيل بارًا، فإن المديانيين خطاة، لذلك، إذ أعلم إثم شعبي، أغفر لهم، ثم أوبخهم على شرهم، وأنتقم من المديانيين سريعًا.

\par 5 فقال جدعون: من أنا وما هو بيت أبي حتى أذهب للقتال ضد المديانيين؟ فقال له الملاك: لعلك تظن أنه كطريق الإنسان كذلك طريق الله. لأن الناس ينظرون إلى مجد العالم وإلى الغنى، وأما الله فينظر إلى ما هو مستقيم وصالح وإلى الوداعة. فالآن اذهب، شد حقويك، فيكون الرب معك، لأنه اختار الانتقام من أعدائه كما هوذا قد أمرك

\par 6 فقال له جدعون: لا يغضب سيدي إن تكلمت بكلمة. هوذا موسى، أول الأنبياء، طلب من الرب آية، فأعطيت له. ولكن من أنا إن لم يُعطني الرب الذي اختارني آية لأعلم أني أسير على الطريق الصحيح. فقال له ملاك الرب: اركض وخذ لي ماءً من البئر هناك واسكبه على هذه الصخرة، فأعطيك آية. فذهب وأخذها كما أمره

\par 7 فقال له الملاك: قبل أن تصب الماء على الصخرة، اسأل ماذا تريد أن تصير، إما دمًا، أو نارًا، أو أن لا تظهر أبدًا. فقال جدعون: ليكن نصفه دمًا ونصفه نارًا. فصب جدعون الماء على الصخرة، وكان لما سكبه أن النصف صار لهيبًا، والنصف الآخر دمًا، واختلطا معًا، أي النار والدم، لكن الدم لم يطفئ النار، ولم تستهلك النار الدم. ولما رأى جدعون ذلك، طلب آيات أخرى، فأعطيت له. أليست هذه مكتوبة في سفر القضاة؟

\chapter{36}

\par 1 فأخذ جدعون 300 رجل ومضى ووصل إلى أقصى محلة مديان، وسمع كل رجل يكلم جاره ويقول: سترون فوضى لا توصف، من سيف جدعون قادمًا علينا، لأن الله قد دفع إلى يديه محلة المديانيين، وسيبدأ في إهلاكنا، حتى الأم مع الأولاد، لأن خطايانا قد امتلأت، كما أظهرت لنا آلهتنا أيضًا ولم نصدقها. والآن قم فلننجِ نفوسنا ونهرب

\par 2 فلما سمع جدعون هذا الكلام، لبس روح الرب في الحال، ووهبه سلطان، فقال للثلاث مئة رجل: قوموا وليتقلد كل واحد منكم سيفه، فقد أُسلم المديانيون إلى أيدينا. فنزل الرجال معه، فتقدم وبدأ يقاتل. فنفخوا في البوق وصرخوا معًا قائلين: سيف الرب علينا. فقتلوا من المديانيين نحو مئة وعشرين ألف رجل، وهرب من بقي من المديانيين.

\par 3 وبعد هذه الأمور، جاء جدعون وجمع بني إسرائيل وقال لهم: هوذا الرب أرسلني لأقاتل معركتكم، فذهبت كما أمرني. والآن أسألكم طلبًا واحدًا: لا تردوا وجوهكم، وليعطني كل رجل منكم أساور الذهب التي على أيديكم. ففرش جدعون قميصًا، وألقى كل رجل أساوره عليه، فوزنت جميعها، فوجدت اثنتي عشرة وزنة (أو اثنتي عشرة ألف شاقل). فأخذها جدعون، وصنع منها أصنامًا وسجد لها

\par 4 فقال الله: قد وُضعت طريق واحدة، ألا أؤدب جدعون في حياته، لأنه لما هدم هيكل البعل، قال جميع الناس: لينتقم البعل لنفسه. والآن، إن كنتُ أؤدبه على ما فعل بي من شر، فستقولون: لم يكن الله هو الذي عاقبه، بل البعل، لأنه أخطأ إليه سابقًا. لذلك الآن يموت جدعون بشيخوخة صالحة، حتى لا يكون لهم ما يتكلمون به. ولكن بعد موت جدعون، سأعاقبه مرة واحدة، لأنه أخطأ إليّ. فمات جدعون بشيخوخة صالحة، ودُفن في مدينته



\chapter{37}

\par 1 وكان له ابن من سرية اسمها أبيمالك، وهو قتل جميع إخوته، راغبًا في أن يكون رئيسًا على الشعب

\par [ورقة ذهبت.]

\par 2 ثم اجتمعت جميع أشجار الحقل على شجرة التين وقالت: تعالَ، امتلكنا. فقالت شجرة التين: هل وُلدتُ حقًا في المملكة أو في سلطان الأشجار؟ أم غُرستُ لذلك لكي أملككم؟ ولذلك، كما لا أستطيع أن أملك عليكم، فلن ينال أبيمالك استمرارًا في سلطانه. بعد ذلك، اجتمعت الأشجار على الكرمة وقالت: تعالَ، امتلكنا. فقالت الكرمة: لقد غُرستُ لأُعطي الناس حلاوة الخمر، وأنا محفوظٌ بتقديم ثمري لهم. ولكن كما لا أستطيع أن أملك عليكم، كذلك يُطلب دم أبيمالك من يدك. وبعد ذلك، جاءت الأشجار إلى التفاحة وقالت: تعالَ، امتلكنا. فقال: أُمرتُ أن أُعطي الناس ثمرة ذات رائحة طيبة. لذلك لا أستطيع أن أملك عليكم، ويموت أبيمالك بالحجارة

\par 3 ثم جاءت الأشجار إلى العليق وقالت: تعال، احكم علينا. فقال العليق: عندما وُلدت الشوكة، أشرقت الحقيقة في شكل شوكة. وعندما حُكم على والدنا الأول بالموت، حُكم على الأرض بإنتاج الأشواك والحسك. وعندما أنار الحق موسى، كان ذلك من خلال شجيرة شوك التي أنارته. الآن لذلك سيكون من خلالي أن يُسمع الحق عنكم. الآن إذا كنت قد تكلمت بصدق إلى العليق أنه يجب أن يسود عليك بالحق، فاجلس تحت ظله: ولكن إذا كان ذلك بالخداع، فدع النار تخرج وتلتهم وتستهلك أشجار الحقل. لأن شجرة التفاح جُعلت للمعاقبين، وشجرة التين جُعلت للشعب، والكرم جُعل للذين كانوا قبلنا.

\par 4 والآن يكون لكم العُلَّيق كأبيمالك الذي قتل إخوته ظلما، ويريد أن يسود عليكم. فإن كان أبيمالك أهلا لهم (أو ليكن أبيمالك نارا لهم) الذين يريد أن يسودهم، فليكن كالعُلَّيق الذي خُلِقَ لتوبيخ السُفهاء من الشعب. فخرجت نار من العُلَّيق وأكلت الأشجار التي في الحقل

\par 5 بعد ذلك، حكم أبيمالك الشعب سنة وستة أشهر، ومات موتًا بليغًا عند برج، حيث ألقت عليه امرأة نصف حجر رحى

\par [فجوة ذات طول غير محدد في النص.]

\chapter{38}

\par 1 (ثم قضى يائير على إسرائيل 22 سنة). وهو بنى مقدسًا للبعل، وأضل الشعب قائلاً: كل من لا يذبح للبعل يموت. وعندما ذبح جميع الشعب، كان سبعة رجال فقط هم الذين امتنعوا عن الذبح، وهذه أسماؤهم: ديفال، وأبيزدريل، وجتليبعل، وسلومي، وأشور، ويونادالي، وميميهل

\par 2 فأجاب وقال ليائير: هوذا قد تذكرنا الوصايا التي أوصانا بها الذين كانوا قبلنا، ودبورة أمنا قائلين: احترزوا من أن تميلوا قلوبكم يمينًا أو يسارًا، بل انتبهوا إلى شريعة الرب نهارًا وليلًا. والآن لماذا تفسدون شعب الرب وتخدعونهم قائلًا: بعل هو الله فلنسجد له؟ والآن إن كان هو الله كما تقول، فليتكلم كإله، وحينئذ نذبح له

\par 3 فقال يائير: أحرقوهم بالنار، لأنهم جدفوا على البعل. فأخذهم عبيده ليحرقوهم بالنار. ولما ألقوهم على النار، خرج نثنائيل، الملاك الذي على النار، وأطفأ النار وأحرق عبيد يائير. أما الرجال السبعة فقد أنقذهم، فلم يبصرهم أحد من الشعب، لأنه كان قد ضرب الشعب بالعمى

\par 4 ولما وصل يائير إلى المكان (أو وصل إلى مكان يائير) احترق هو أيضًا. ولكن قبل أن يحرقه، قال له ملاك الرب: اسمع كلام الرب قبل أن تموت. هكذا قال الرب: أنا رفعتك من أرض مصر، وجعلتك رئيسًا على شعبي. ولكنك قمت وأفسدت عهدي، وأضللتهم، وسعيت إلى حرق عبيدي باللهيب، لأنهم وبخوك، الذين كانوا يحرقون بنار فانية، والآن يُحيون بنار حية ويخلصون. لكنك ستموت، يقول الرب، وفي النار التي ستموت فيها، يكون لك مسكنك. وبعد ذلك أحرقه، وجاء حتى إلى عمود البعل وقلبه، وأحرق البعل مع الشعب الواقفين، حتى 1000 رجل.

\chapter{39}

\par 1 وبعد هذه الأمور، جاء بنو عمون وبدأوا يحاربون إسرائيل واستولوا على العديد من مدنهم. ولما ضاق الشعب جدًا، اجتمعوا في المصفاة، قائلين كل رجل لجيرانه: هوذا نحن نرى الضيق الذي يحيط بنا، والرب قد فارقنا ولم يعد معنا، وقد استولى أعداؤنا على مدننا، وليس هناك قائد يدخل ويخرج أمام وجهنا. والآن فلننظر من نقيم علينا لخوض معركتنا

\par 2 وكان يفثان الجلعادي جبار بأس، ولأنه كان يغار من إخوته، طردوه من أرضه، فذهب وأقام في أرض طوبيا. فاجتمع إليه رجال طوافون وأقاموا عنده

\par 3 ولما انهزم إسرائيل في المعركة، جاءوا إلى أرض طوبيا إلى يفثان وقالوا له: تعالَ، تسلط على الشعب. فمن يعلم هل نجوتَ إلى هذا اليوم أم نجوتَ من أيدي إخوتك حتى تتسلط على شعبك في هذا الوقت؟

\par 4 فقال لهم يفثان: هل هكذا تعود المحبة بعد البغضاء، أم يغلب الزمن كل شيء؟ لأنكم طردتموني من أرضي ومن بيت أبي، والآن أتيتم إليّ وأنتم في ضيق؟ فقالوا له: إن كان إله آبائنا لم يذكر خطايانا، بل أنقذنا حين أخطأنا إليه، ودفعنا أمام أعدائنا، وظلمونا منهم، فلماذا تتذكر أنت أيها الإنسان الآثام التي أصابتنا في وقت ضيقنا؟ لذلك لا يكون كذلك أمامك يا سيد

\par 5 فقال يافثان: إن الله قادر على أن يغفل عن خطايانا، إذ له وقت ومكان ليستريح من طول أناته، لأنه هو الله. أما أنا فانٍ، مخلوق من تراب: إلى أين أعود، وأين أطرح غضبي والظلم الذي آذيتموني به؟ فقال له الشعب: لتعلمك الحمامة ما شُبِّه به إسرائيل، فإنه وإن أُخِذَ صغارها منها، فإنها لا تفارق مكانها، بل ترفض شرها وتنسى كما في قعر الغمر.

\par 6 فقام يفثان ومضى معهم وجمع كل الشعب، وقال لهم: أنتم تعلمون أنه عندما كان أمراؤنا أحياء، نصحونا باتباع شريعتنا. فصدّ عمون وبنوه الشعب عن طريقهم الذي سلكوه ليعبدوا آلهة أخرى تُهلكهم. فالآن اجعلوا قلوبكم في شريعة الرب إلهكم، ولنتضرع إليه بصوت واحد. وهكذا سنحارب أعداءنا، ونثق ونرجو الرب أنه لن يُسلمنا إلى الأبد. لأنه وإن كثرت خطايانا، إلا أن رحمته تملأ كل الأرض

\par 7 وصلى جميع الشعب بصوت واحد، رجالاً ونساءً، صبياناً ورضعاً. وصلوا وقالوا: انظر يا رب إلى الشعب الذي اخترته، ولا تفسد الكرمة التي غرستها يمينك، لكي يكون هذا الشعب أمامك ميراثاً، الذي امتلكته من البدء، وفضلته إلى الأبد، ومن أجله هيأت له المساكن، وأدخلته إلى الأرض التي أقسمت له. لا تنقذنا أمام مبغضيك يا رب

\par 8 فندم الله على غضبه وشدد روح يفثان. وأرسل رسالة إلى جيتال ملك بني عمون وقال: لماذا تضايق أرضنا وتأخذ مدننا، أو لماذا تضايقنا؟ لم يؤمرك إله إسرائيل أن تهلك سكان الأرض. فالآن رد إليّ مدنّي، فيكفّ غضبي عنك. وإلا فاعلم أني أصعد إليك وأجازيك على ما سبق، وأرد شرك على رأسك. ألا تتذكر كيف خدعت بني إسرائيل في البرية؟ وكلم رسل يفثان ملك بني عمون بهذه الكلمات

\par 9 فقال جيتال: هل فكر إسرائيل حين استولى على أرض الأموريين؟ فقل: اعلموا الآن أني سآخذ منكم ما تبقى من مدنكم، وأجازيكم على ذنبكم، وأنتقم للأموريين الذين ظلمتموهم. فأرسل يافثان أيضًا إلى ملك بني عمون قائلًا: الحق أني أرى أن الله قد أتى بكم إلى هنا لأهليكم، إن لم ترتاحوا من إثمكم الذي تضايقون به إسرائيل. ولذلك آتي إليكم وأظهر لكم نفسي. لأنكم لستم آلهة كما تقولون، بل لأنكم أضللتموهم وراء الحجارة، فالنار ستتبعكم للانتقام.

\par 10 ولأن ملك بني عمون لم يسمع لصوت يفثان، قام يفثان وسلّح جميع الشعب ليخرجوا ويحاربوا في الحدود قائلاً: عندما يُسلّم بنو عمون إلى يدي وأُرجع، فإن كل من يصادفني أولاً يكون محرقة للرب

\par 11 فغضب الرب جدًا وقال: هوذا يفثان قد نذر أن يقدم لي ما يلتقي به أولًا. والآن، إذا التقى كلب بيفثان أولًا، فهل يُقدم لي كلب؟ والآن فليكن نذر يفثان على بكره، على ثمرة بطنه، وصلاته على وحيدته. ولكني سأخلص شعبي في هذا الوقت، ليس من أجله، بل من أجل الصلاة التي صلاها إسرائيل

\chapter{40}

\par 1 فجاء يفثان وحارب بني عمون، فدفعهم الرب إلى يده، وضرب ستين مدينة منهم. ورجع يفثان بسلام. وخرجت النساء للقائه بالرقص. وكانت له ابنة وحيدة، هي التي خرجت أولًا في الرقص للقائه. فلما رآها يفثان غشي عليه وقال: بحق يُدعى اسمك سيلا، حتى تُذبحي ذبيحة. والآن من يضع قلبي في الميزان ويزن نفسي؟ فأقف وأنظر أيهما يرجح على الآخر، الفرح الآتي أم الضيق الذي يأتي علي؟ لأني إذ فتحت فمي لربي بترنيمة نذوري، لا أستطيع أن أرجعه بعد

\par 2 فقالت له سيلا ابنته: ومن ذا الذي يحزن على موته حين يرى الشعب قد نجا؟ ألا تذكر ما كان في أيام آبائنا حين كان الأب يقدم ابنه محرقة، فلا ينكر عليه، بل يرضى به فرحًا؟ وكان الذي قُدِّم مستعدًا، والذي قُدِّم فرح

\par 3 الآن فلا تلغِ أي شيء مما نذرته، بل امنحني صلاة واحدة. أسألك قبل أن أموت طلبًا صغيرًا: أتوسل إليك أنه قبل أن أسلم روحي، أسمح لي أن أذهب إلى الجبال وأتجول (أو أقيم) بين التلال وأتجول بين الصخور، أنا والعذارى اللواتي هن رفيقاتي، وأسكب دموعي هناك وأخبر بمحنة شبابي؛ وستبكيني أشجار الحقل وسترثي عليّ وحوش الحقل؛ لأني لست حزينًا على موتي، ولا يحزنني أن أسلم روحي: ولكن بما أن والدي قد سبق في نذره، [و] إذا لم أقدم نفسي طواعيةً للتضحية، أخشى أن لا يكون موتي مقبولاً، وأن أفقد حياتي بلا سبب. سأخبر الجبال بهذه الأشياء، وبعد ذلك سأعود. وقال أبوها: اذهبي.

\par 4 فخرجت سيلا ابنة يفثان هي والعذارى رفيقاتها، وأتين وأخبرن حكماء الشعب. فلم يستطع أحد أن يجيب على كلامها. وبعد ذلك دخلت جبل ستيلق، وفي الليل فكر فيها الرب وقال: هوذا الآن قد أغلقت لسان حكماء شعبي أمام هذا الجيل، فلم يستطيعوا أن يجيبوا على كلام ابنة يفثان، لكي يتم كلامي ولا ينقض مشورتي التي فكرت بها. ورأيت أنها أحكم من أبيها، وأعقل من كل الحكماء الذين هنا. والآن لتُعطى حياتها بناءً على طلبها، ويكون موتها عزيزًا في عينيّ دائمًا

\par 5 ولما وصلت ابنة ييفثان إلى جبل ستيلاك، بدأت بالندب. وهذا هو ندبها الذي ندبت به وناحت به قبل أن تغادر، وقالت: اسمعي أيتها الجبال ندبائي، وانظري أيتها التلال إلى دموع عيني، واشهدي أيتها الصخور على ندب نفسي. انظري كيف أُتهم، لكن نفسي لن تُؤخذ عبثًا. لتخرج كلماتي إلى السماء، ولتكتب دموعي أمام وجه الفلك، حتى لا يغلب الأب (أو لا يحارب) ابنته التي نذر أن يقدمها، حتى يسمع حاكمها أن ابنته الوحيدة قد وُعدت بالتضحية

\par 6 ومع ذلك، لم أشبع من فراش زواجي، ولم أمتلئ بأكاليل زفافي. لأني لم أتلألأ بالتألق، جالسةً في بكارتي؛ ولم أستخدم دهنًا ثمينًا، ولم تذق روحي زيت المسحة الذي أُعدّ لي. يا أمي، لم تلد ابنتك الوحيدة، ولم تلدها على الأرض، لأن الجحيم صار حجرة زواجي. فليُسكب كل مزيج الزيت الذي أعددته لي، وليأكله العثّ الثوب الأبيض الذي نسجته أمي لي، وليذبل إكليل الزهور الذي ضفرتْه لي مرضعتي من قبل، وليفسده الدود، وعندما تخبرني العذارى رفيقاتي عني، فليبكنّ عليّ بالتأوّه أيامًا كثيرة.

\par 7 انحنِ أغصانكِ أيتها الأشجار، وانوحي على شبابي. تعالي يا وحوش الغابة، ودوسي عذريتي. لأن سنيني قد انقطعت، وأيام حياتي شاخت في الظلمة

\par 8 ولما قالت ذلك، رجعت سيلا إلى أبيها، ففعل كل ما نذره وأصعد محرقات. فاجتمعت جميع فتيات إسرائيل ودفنت ابنة يفثان وناحت عليها. وعمل بنو إسرائيل مناحة عظيمة، وحددوا في ذلك الشهر، في اليوم الرابع عشر من الشهر، أن يجتمعوا كل سنة وينوحوا على ابنة يفثان أربعة أيام. ودعوا اسم قبرها باسمها سيلا

\par 9 وقضى يبثان لبني إسرائيل عشر سنين، ومات ودُفن مع آبائه

\chapter{41}

\par 1 وبعده قام قاضٍ في إسرائيل، وهو أدو بن ألك البرتوني، وقضى أيضًا لبني إسرائيل ثماني سنوات. في أيامه أرسل إليه ملك موآب رسلًا يقول: هوذا الآن قد علمت أن إسرائيل قد أخذ مدنّي، فأعدها الآن جزاءً. فقال أدو: ألم تتعلموا بعد مما أصاب بني عمون، لعلّ خطايا موآب تُكَمَّل؟ فأرسل أدو وأخذ من الشعب عشرين ألف رجل، وجاء إلى موآب، وحاربهم، وقتل منهم خمسة وأربعين ألف رجل. وهرب الباقون من أمامه. ورجع أدو بسلام، وأصعد محرقات وذبائح لسيده، ومات، ودُفن في أفراتة مدينته

\par 2 وفي ذلك الوقت اختار الشعب إيلون وجعلوه قاضيًا عليهم، فقضى لإسرائيل عشرين سنة. في تلك الأيام حاربوا الفلسطينيين وأخذوا منهم اثنتي عشرة مدينة. ومات إيلون ودُفن في مدينته

\par 3 لكن بنو إسرائيل نسوا الرب إلههم وعبدوا آلهة سكان الأرض. لذلك أُسلموا للفلسطينيين وخدموهم أربعين سنة

\chapter{42}

\par 1 كان رجل من سبط دان اسمه مانويه بن إيدوك بن أودو بن إيريدن بن فدهسور بن ديما بن سوسي بن دان. وكانت له امرأة اسمها إلوما بنت رماك. وكانت عاقرًا ولم تلد له. وكان مانويه يقول لها يومًا فيومًا: هوذا الرب قد أغلق رحمك حتى لا تلد، فأطلقيني إذن لأتخذ زوجة أخرى لئلا أموت بلا ذرية. فقالت: لم يمنعني الرب عن الولادة، بل منعك أنت حتى لا أثمر. فقال لها: لتوضح الشريعة محنتنا.

\par 2 وبينما كانا يتخاصمان يومًا بعد يوم، وكلاهما كان حزينًا جدًا لعدم ثمرهما، في ليلة ما، صعدت المرأة إلى العلية وصلّت قائلة: أيها الرب إله كل ذي جسد، اكشف لي ما إذا كان لا يُعطى لزوجي أم لي الإنجاب، أو لمن يُحظر عليه أو لمن يُسمح له بالإثمار، حتى يحزن من يُحظر عليه على خطاياه، لأنه يبقى بلا ثمر. أو إذا حُرم كلانا، فاكشف لنا هذا أيضًا، حتى نحمل خطايانا ونصمت أمامك

\par 3 فسمع الرب لصوتها وأرسل لها ملاكه في الصباح وقال لها: أنتِ العاقر التي لا تلد، والرحم الممنوع من أن يثمر. والآن قد سمع الرب صوتكِ ونظر إلى دموعكِ وفتح رحمكِ. وها أنتِ تحبلين وتلدين ابنًا وتدعين اسمه شمشون، لأنه يكون مقدسًا لربك. ولكن احذري أن لا يذوق شيئًا من ثمر الكرمة، ولا يأكل شيئًا نجسًا، لأنه كما قال هو ينقذ إسرائيل من يد الفلسطينيين. ولما تكلم ملاك الرب بهذا الكلام ذهب من عندها

\par 4 فدخلت إلى زوجها إلى البيت وقالت له: ها أنا أضع يدي على فمي وأصمت أمامك كل أيامي، لأنه باطلا كنت أفتخر ولم أصدق كلامك. لأن ملاك الرب جاء إلي اليوم وأراني قائلا: يا إلوما، أنت عاقر، لكنك ستحبلين وتلدين ابنا

\par 5 ولم يُصدِّق مانوي زوجته. فخجل وحزن، وصعد هو أيضًا إلى العلية وصلى قائلًا: ها أنا لستُ مستحقًا أن أسمع الآيات والعجائب التي صنعها الله فينا، ولا أن أرى وجه رسوله

\par 6 وبينما هو يتكلم هكذا، جاء ملاك الرب أيضًا إلى امرأته. وكانت في الحقل ومانوي في بيته. فقال لها الملاك: اركضي وادعي زوجك، لأن الله قد حسبه أهلاً لسماع صوتي

\par 7 فركضت المرأة ونادت زوجها، فأسرع وجاء إلى الملاك في الحقل في عمّو، فقال له: ادخل على امرأتك وافعل كل هذه الأمور سريعًا. فقال له: يا رب، انتبه أن يتم كلامك على عبدك. فقال: يكون كذلك.

\par 8 فقال له مانويه: لو كنت قادرًا لأقنعتك بالدخول إلى بيتي وتناول الخبز معي، وعلمت أنه عندما تذهب سأعطيك هدايا لتأخذها معك لتقدم ذبيحة للرب إلهك. فقال له الملاك: لن أدخل معك إلى بيتك، ولن آكل خبزك، ولن أقبل هداياك. لأنه إذا قدمت ذبيحة مما ليس لك، فلا أستطيع أن أُظهر لك معروفًا

\par 9 فبنى مانوي مذبحًا على الصخرة، وقدم ذبائح ومحرقات. ولما قطع اللحم ووضعه على القدس، مدّ الملاك يده ولمسه بطرف صولجانه. فخرجت نار من الصخرة وأكلت المحرقات والذبائح. ثم صعد الملاك عنه بلهيب النار

\par 10 فلما رأى مانويه وزوجته ذلك، سقطا على وجهيهما وقالا: نموت موتًا، لأننا رأينا الرب وجهًا لوجه. ولم يكفني أني رأيته، بل سألت أيضًا عن اسمه، غير عالم أنه خادم الله. وكان الملاك الذي جاء يُدعى فدائيل



\chapter{43}

\par 1 وحدث في وقت تلك الأيام أن إيلوما حبلت وولدت ابنًا ودعت اسمه شمشون. وكان الرب معه. ولما كبر وطلب محاربة الفلسطينيين، اتخذ لنفسه امرأة من الفلسطينيين. فأحرقها الفلسطينيون بالنار، لأن شمشون أذلهم جدًا

\par 2 وبعد ذلك دخل شمشون أشدود (أو غضب عليه). فأغلقوا عليه أبواب المدينة وقالوا: هوذا الآن قد سُلِّم خصمنا إلى أيدينا. فلنجتمع الآن ونساعد بعضنا بعضًا. ولما استيقظ شمشون في الليل ورأى المدينة مغلقة، قال: هوذا الآن قد حبستني هذه البراغيث في مدينتهم. والآن يكون الرب معي، فأخرج من أبوابهم وأقاتلهم

\par 3 فذهب ووضع يده اليسرى تحت مِزلاج الباب ونفضه وهدم باب السور. وأمسك أحد البابين في يده اليمنى كترس، ووضع الآخر على كتفيه وخلعه، ولأنه لم يكن معه سيف، طارد به الفلسطينيين، فقتل به خمسة وعشرين ألف رجل. ورفع جميع أمتعة الباب ونصبها على جبل

\par 4 وأما الأسد الذي قتله، وفك الحمار الذي ضرب به الفلسطينيين، والربطات التي قطعها عن ذراعيه كأنها من ذاتها، والثعالب التي أمسكها، أليست هذه الأمور مكتوبة في سفر القضاة؟

\par 5 ثم نزل شمشون إلى جرارة، مدينة الفلسطينيين، فرأى هناك زانية اسمها دليلة، فانساق وراءها واتخذها له زوجة. وقال الله: هوذا شمشون الآن قد أضلته عيناه ونسي الجبابرة التي صنعت معه، واختلط ببنات الفلسطينيين، ولم يفكر في عبدي يوسف الذي كان في أرض غريبة وصار تاجًا لإخوته لأنه لم يشأ أن يذل نسله. فالآن تكون شهوته عثرة لشمشون، واختلاطه هلاكًا له، وسأسلمه إلى أعدائه فيعمونه. ولكن في ساعة موته سأذكره، وأنتقم له مرة أخرى من الفلسطينيين

\par 6 وبعد هذه الأمور، ألحّت عليه امرأته قائلةً: أرني قوتك، وأين قوتك. فأعلم أنك تحبني. ولما خدعها شمشون ثلاث مرات، وظلت تُلح عليه كل يوم، في المرة الرابعة أظهر لها قلبه. فأسكرته، ولما نام دعت حلاقًا، فحلق خصل رأسه السبع، فزال عنه سلطانه، لأنه هكذا كشف لها. ودعت الفلسطينيين، فضربوا شمشون، وأعموه، ووضعوه في السجن

\par 7 وفي يوم وليمة لهم، دعوا شمشون ليسخروا منه. وكان مقيدًا بين عمودين، يصلي قائلًا: أيها الرب إله آبائي، استجب لي هذه المرة أيضًا، وقوّني لأموت مع هؤلاء الفلسطينيين، لأن هذا المنظر الذي أخذوه مني قد وهبته لي طوعًا. وأضاف شمشون قائلًا: اخرجي يا نفسي ولا تحزني. مت يا جسدي ولا تبكي على نفسك

\par 8 وأمسك بعمودي البيت وهزهما. فسقط البيت وكل ما فيه وقتل جميع من حوله، وكان عددهم أربعين ألف رجل وامرأة. ونزل إخوة شمشون وكل بيت أبيه، وأخذوه ودفنوه في قبر أبيه. وقضى لإسرائيل عشرين سنة

\chapter{44}

\par 1 وفي تلك الأيام لم يكن رئيس في إسرائيل، بل كان كل إنسان يفعل ما يرضي في عينيه

\par 2 في ذلك الوقت قام ميخا بن ديديلا أم حليو، وكان معه ألف درهم من الذهب وأربعة أسافين من الذهب المسبوك وأربعون درهمًا من الفضة. فقالت له أمه ديديلا: يا بني، اسمع صوتي فاصنع لنفسك اسمًا قبل موتك. خذ هذا الذهب واصهره واصنع لنفسك أصنامًا فتكون لك آلهة وتكون لها كاهنًا.

\par 3 ويكون أن كل من يسأل عنهم يأتون إليك فتجيبهم. ويكون في بيتك مذبح وعمود مبنيان، ومن الذهب الذي لديك تشتري لنفسك بخورًا للحرق وغنمًا للذبائح. ويكون أن كل من يقدم ذبيحة، يعطي عن الغنم 7 دراهم، وعن البخور، إذا أحرقه، يعطي درهمًا واحدًا من الفضة بوزن كامل. ويكون اسمك كاهنًا، وتُدعى عابدًا للآلهة

\par 4 فقال لها ميخا: لقد أحسنتِ نصحي يا أمي كيف أعيش، والآن سيكون اسمك أعظم من اسمي، وفي آخر الأيام يُطلب منك هذه الأشياء

\par 5 فذهب ميخا وفعل كل ما أمرته به أمه. ونحت وصنع لنفسه ثلاثة تماثيل أولاد وعجول وأسد ونسر وتنين وحمامة. وكان كل من ضلوا يأتون إليه، وإذا طلب أحد منهم زوجات سألوه بالحمامة، وإذا طلب بنين فصورة الأولاد. وأما من طلب غنى فاستشاره بشبه النسر، ومن طلب قوة فصورة الأسد. وأيضًا، إذا طلبوا رجالًا وفتيات سألوا بتماثيل العجول، ولكن إذا طلبوا طول الأيام سألوا بصورة التنين. وكان إثمه متعدد الأشكال، وكان كفره مليئًا بالمكر

\par 6 لذلك، عندما ارتحل بنو إسرائيل عن الرب، قال الرب: ها أنا ذا أستأصل الأرض وأهلك كل جنس البشر، لأنه عندما عينت أشياء عظيمة على جبل سيناء، أظهرت نفسي لبني إسرائيل في العاصفة وقلت لهم ألا يصنعوا أصنامًا، فوافقوا على ألا ينحتوا صور الآلهة. وأمرتهم ألا ينطقوا باسمي باطلا، فاختاروا هذا، حتى ألا ينطقوا باسمي باطلا. وأمرتهم بحفظ يوم السبت، فوافقوا لي على تقديس أنفسهم. وقلت لهم أن يكرمون أباهم وأمهم، فوعدوا بذلك. وأمرتهم ألا يسرقوا، فوافقوا. وأمرتهم ألا يقتلوا، فقبلوا ذلك، ألا يفعلوا. وأمرتهم ألا يزنوا، فلم يرفضوا وأوصيتهم أن لا يشهدوا بالزور، ولا يشتهوا كل واحد امرأة قريبه أو بيته أو أي شيء مما له. فقبلوا ذلك.

\par 7 والآن، بعد أن نصحتهم ألا يصنعوا أصنامًا، صنعوا أعمال كل تلك الآلهة المولودة من الفساد باسم تمثال منحوت. وأيضًا من أولئك الذين فسد كل شيء بسببهم. لأن البشر صنعوها، والنار خدمت في صهرها. عمل البشر خلقها، والأيدي صنعتها، والعقل دبّرها. وبما أنهم قبلوها، فقد اتخذوا اسمي باطلا، وأعطوا اسمي لتماثيل منحوتة، وفي يوم السبت الذي قبلوه لحفظه، صنعوا منه رجاسات. لأني قلت لهم أن يحبوا أباهم وأمهم، فقد أهانوني أنا خالقهم. ولأني قلت لهم ألا يسرقوا، فقد سرقوا في عقولهم بالتماثيل المنحوتة. وبما أنني قلت لهم ألا يقتلوا، فإنهم يقتلونهم عندما يخدعون. وبعد أن أوصيتهم ألا يزنوا، تظاهروا بالزنى بغيرتهم. وحيث اختاروا ألا يشهدوا بالزور، فقد تلقوا شهادة زور ممن طردوهم، واشتهوا نساءً أجنبيات.

\par 8 لذلك، ها أنا أمقت الجنس البشري، ولكي أستأصل خليقتي، فإن الذين يموتون سيكثرون أكثر من عدد الذين يولدون. لأن بيت يعقوب قد دنس بالآثام، وكثرت ذنوب إسرائيل، ولا أستطيع [بعض الكلمات مفقودة] أن أبيد سبط بنيامين تمامًا، لأنهم سُحِبوا أولًا بعد ميخا. ولن يفلت شعب إسرائيل أيضًا من العقاب، بل سيكون ذلك لهم ذنبًا أبديًا لذاكرة الأجيال كلها

\par 9 لكن ميخا سأُسلِّمه إلى النار. وستذبل أمه أمامه، تعيش على الأرض، ويخرج الدود من بطنها. وعندما يُكلِّمان بعضهما البعض، تقول كأم تُوبِّخ ابنها: انظر أي خطيئة ارتكبتَها. فيُجيب كابن مطيع لأمه ومُحتال: وقد فعلتَ إثمًا أعظم. ويكون شبه الحمامة التي صنعها لفقء عينيه، وشبه النسر لسكب نار من جناحيه، وتماثيل الأولاد التي صنعها لخدش جنبيه، وشبه الأسد الذي صنعه يكون له كالجبابرة الذين يُعذبونه

\par 10 وهكذا سأفعل ليس فقط مع ميخا، بل أيضًا مع كل من يخطئ ضدي. والآن فليعلم جنس البشر أنهم لن يغيظني باختراعاتهم. ولن يأتي هذا العقاب على صانعي الأصنام فقط، بل سيكون على كل إنسان، بما أخطأ به من خطيئة، سيُدان. لذلك، إن تكلموا بالكذب أمامي، فسآمر السماء فتحرمهم من المطر. وإن طمع أحد في خيرات قريبه، فسآمر الموت فيحرمه من ثمرة جسده. وإن أقسموا باسمي زورًا، فلن أستجيب لدعائهم. وعندما تنفصل الروح عن الجسد، سيقولون: لا نحزن على ما عانيناه، بل على كل ما اخترعناه، سننال أيضًا.

\chapter{45}

\par 1 وحدث في ذلك الوقت أن رجلاً من سبط لاوي جاء إلى جبعون، ولما أراد أن يقيم هناك غربت الشمس. ولما أراد أن يدخل هناك لم يدعه الساكنون هناك. فقال لغلامه: انطلق، قد البغل، فنذهب إلى مدينة نوبا، لعلهم يأذنون لنا بالدخول إلى هناك. فجاء إلى هناك وجلس في ساحة المدينة. ولم يقل له أحد: تعال إلى بيتي

\par 2 وكان هناك لاوي اسمه بيثاق. رآه ذلك وقال له: هل أنت بعيل من سبطنا؟ فقال: أنا هو. فقال له: ألا تعلم شرور سكان هذه المدينة؟ من أشار عليك بالدخول إلى هنا؟ أسرع واخرج من هنا، وادخل بيتي الذي أسكن فيه، وأقم هناك اليوم، فيغلق الرب قلوبهم أمامنا كما أغلق رجال سدوم أمام لوط. فدخل المدينة وبات هناك تلك الليلة

\par 3 فاجتمع جميع سكان المدينة وقالوا لبيت قا: أخرجي الذين جاءوا إليكِ اليوم، وإلا أحرقناهم وإياكِ بالنار. فخرج إليهم وقال لهم: أليسوا إخوتنا؟ لا نعاملهم بشر لئلا تكثر خطايانا علينا. فقالوا: لم يكن هكذا قط أن يأمر الغرباء السكان. فدخلوا بعنف وأخرجوه هو وسريته وطردوهما، وقالوا: أطلقوا الرجل. أما سريته فقد أساءوا إليها حتى ماتت؛ لأنها أخطأت على زوجها ذات مرة بمعصيتها مع العماليقيين، ولذلك أسلمها الرب الإله إلى أيدي الخطاة

\par 4 ولما كان النهار، خرج بعيل فوجد سريته ميتة. فحملها على البغل وأسرع وخرج وجاء إلى قادس. وأخذ جثتها وقسمها وأرسلها إلى جميع الأسباط الاثني عشر، قائلاً: "لقد حدث لي هذا في مدينة نوبا، إذ قام عليّ سكانها ليقتلوني، فأخذوا سريتي وسجنوني وقتلوها. وإن كان هذا مقبولاً لديكم، فاصمتوا، وليحكم الرب، وإن أردتم الانتقام، فسينصركم الرب".

\par 5 فخجل جميع الرجال، حتى الأسباط الاثنا عشر. واجتمعوا إلى سيلو وقال كل رجل لصاحبه: هل حدث مثل هذا الإثم في إسرائيل؟

\par 6 فقال الرب للخصم: أرأيت كيف انزعج هذا الشعب الأحمق؟ في الساعة التي كان من المفترض أن يموتوا فيها، حتى عندما تعامل ميخا بمكر ليخدع الشعب بهذه، أي بالحمامة والنسر وبصورة البشر والعجول والأسد والتنين، حينئذ لم يتأثروا. ولذلك لأنهم لم يغضبوا، فلتكن مشورتهم الآن باطلة ولتتحرك قلوبهم، حتى يُهلك الذين يسمحون بالشر كما يُهلك الخطاة

\chapter{46}

\par 1 ولما كان النهار، تأثر بنو إسرائيل جدًا وقالوا: لنصعد ونفحص الخطيئة التي ارتُكبت، فيُرفع الإثم عنا. وتكلموا هكذا، وقالوا: لنسأل الرب أولًا ونعلم هل يُسلم إخوتنا إلى أيدينا؟ وإلا فلنمتنع. فقال لهم فينحاس: لنقدم البرهان والحق. فأجابهم الرب وقال: اصعدوا، لأني سأسلمهم إلى أيديكم. فخدعهم لكي يُتم كلامه

\par 2 فصعدوا إلى المعركة، ووصلوا إلى مدينة بنيامين، وأرسلوا رسلًا قائلين: أرسلوا إلينا الرجال الذين فعلوا هذا الشر، وسنعفو عنكم، ولكن جازوا كل إنسان شره. فقسى بنو بنيامين قلوبهم، وقالوا لبني إسرائيل: لماذا نسلم إخوتنا إليكم؟ إن لم تعفوا عنهم، فسنقاتلكم. فخرج بنو بنيامين للقاء بني إسرائيل، وطاردوهم، فسقط بنو إسرائيل أمامهم، وضربوا منهم خمسة وأربعين ألف رجل

\par 3 فاغتاظ قلب الشعب جدًا، وجاءوا باكين ونائحين إلى سيلو وقالوا: هوذا الرب قد أسلمنا أمام سكان نوبا. والآن لنسأل الرب من أخطأ بيننا. فسألوا الرب، فقال لهم: إن شئتم فاصعدوا وقاتلوا، فيُسلموا إلى أيديكم، وحينئذ يُخبرون لماذا سقطتم أمامهم. فذهبوا في اليوم الثاني لمحاربتهم. فخرج بنو بنيامين وطاردوا إسرائيل وضربوا منهم ستة وأربعين ألف رجل

\par 4 "فذاب قلب الشعب تمامًا وقالوا: هل أراد الله أن يخدع شعبه؟ أم أمر بسبب الشر الذي يحدث أن يسقط الأبرياء أيضًا كفاعلي الشر؟ ولما تكلموا هكذا سقطوا أمام تابوت عهد الرب ومزقوا ثيابهم ووضعوا رمادًا على رؤوسهم هم وفينحاس بن العازار الكاهن، الذي صلى وقال: ما هذا الخداع الذي خدعتنا به يا رب؟ إن كان عادلًا أمام وجهك الذي فعله بنو بنيامين، فلماذا لم تخبرنا به حتى نتأمل فيه؟ وإن لم يكن مرضيًا في عينيك، فلماذا سمحت لنا أن نسقط أمامهم؟"



\chapter{47}

\par 1 وأضاف فينحاس وقال: يا إله آبائنا، اسمع صوتي، وأخبر عبدك اليوم إن كان قد أحسن في عينيك، أو إن كان الشعب قد أخطأ وأردت أن تبيد شرهم، لكي تصلح بيننا أيضًا الذين أخطأوا إليك. لأني أتذكر في صباي أن يَمبري أخطأ في أيام موسى عبدك، فدخلت حقًا، وغارت في نفسي، ورفعتهما كليهما على سيفي، وكان البقية يقومون عليّ ليقتلوني، فأرسلت ملاكك وضربت منهم أربعة وعشرين ألف رجل وأنقذتني من أيديهم

\par 2 والآن أرسلتَ الأسباط الأحد عشر وأحضرتهم إلى هنا قائلًا: اذهبوا واضربوهم. وعندما ذهبوا أُسْلِموا. والآن يقولون إن أقوال حقك مكشوفة أمامك. والآن يا رب إله آبائنا، لا تُخفِ ذلك عن عبدك، بل أخبرنا لماذا فعلتَ هذا الإثم ضدنا

\par 3 فلما رأى الرب أن فينحاس قد صلى أمامه صلاةً حارة، قال له: بذاتي أقسمت، يقول الرب، أنه لو لم أقسم لما تذكرتك فيما تكلمت به، ولا لما أجبتك اليوم. والآن قل للشعب: قوموا واسمعوا كلام الرب،

\par 4 هكذا قال الرب: كان في وسط الغابة أسد جبار، وقد أوكلت إليه جميع الوحوش الغابة ليحرسها بقوته، لئلا تأتي وحوش أخرى وتدمرها. وبينما كان الأسد يحرسها، جاءت وحوش الحقل من غابة أخرى، وأكلت جميع صغار الوحوش، وأتلفت ثمرة أجسادها، فرأى الأسد ذلك فسكت. أما الوحوش فكانت في سلام، لأنها أوكلت الغابة إلى الأسد، ولم تشعر أن صغارها قد هلكت

\par 5 وبعد مدة، نهض وحش صغير جدًا من أولئك الذين أسلموا الغابة للأسد، والتهم أصغر أشبال وحش آخر شرير جدًا. وإذا بالأسد يصرخ ويهيج جميع وحوش الغابة، فتقاتلوا فيما بينهم، وحارب كل واحد جاره

\par 6 ولما هلك كثير من الوحوش، رآه جرو آخر من غابة أخرى مثله، فقال: ألم تقتل مثل هذا العدد من الوحوش؟ ما هذا الإثم، أنه في البدء حين هلك كثير من الوحوش وصغارها ظلماً على يد وحوش شريرة أخرى، وحين كان على جميع الوحوش أن تنتقم لأنفسها، إذ رأت ثمرة أجسادها قد أُهدرت بلا نفع، صمتت ولم تتكلم، ولكن الآن هلك جرو واحد من وحش شرير، وأثارت الغابة كلها حتى تفترس الوحوش بعضها بعضاً بلا سبب، فتتقلص الغابة. فالآن يجب أن تُهلك أولاً، وهكذا يُثبت الباقي. فلما سمع صغار الوحوش ذلك، قتلوا الأسد أولاً، ووضعوا فوقهم الجرو مكانه، وهكذا خضعت بقية الوحوش معاً.

\par 7 قام ميخا وأغناكم بما فعله هو وأمه. وكانت شرورٌ وشرٌّ لم يخترعه أحدٌ قبلهم، لكنه بمكرِه صنع تماثيل منحوتة لم تُصنع إلى ذلك اليوم، ولم يُغضب أحد، بل أنتم جميعًا ضللتم، ورأيتم ثمرة أجسادكم تُفسد، وسكتم كالأسد الشرير

\par 8 والآن لما رأيتم كيف ماتت سرية هذا الرجل التي عانت من الشر، تأثرتم جميعًا وأتيتم إليّ قائلين: أتسلم بني بنيامين إلى أيدينا؟ لذلك خدعتكم وقلت: سأسلمهم إليكم. والآن أهلكتُ الذين سكتوا حينئذ، وهكذا سأنتقم من كل من أساء إليّ. أما أنتم، فاصعدوا الآن، لأني سأسلمهم إليكم

\par 9 فقام جميع الشعب معًا وذهبوا. فخرج بنو بنيامين للقائهم وظنوا أنهم سيغلبونهم كما فعلوا من قبل. ولم يعلموا أن شرهم قد تم عليهم. ولما لحقوا بهم كما في البداية وطاردوهم، هرب الشعب من أمامهم ليعطوهم مكانًا، ثم قاموا من كمينهم، وكان بنو بنيامين في وسطهم

\par 10 ثم رجع الهاربون، وقُتل رجال مدينة نوبا، رجالاً ونساءً، خمسة وثمانون ألف رجل، وأحرق بنو إسرائيل المدينة، ونهبوا الغنائم، ودمروا كل شيء بحد السيف. ولم يبق من بني بنيامين أحد إلا ستمائة رجل هربوا ولم يُعثر عليهم في المعركة. فرجع جميع الشعب إلى سيلو ومعهم فينحاس بن ألعازار الكاهن

\par 11 "وهؤلاء هم الذين بقوا من نسل بنيامين رؤساء السبط من عشر عائلات أسماؤهم هذه: من العائلة الأولى: إزبيل، زيب، بالاق، ريندباك، بلوخ. ومن العائلة الثانية: نيثاك، زينيب، فينوخ، دمك، جريسارز. ومن العائلة الثالثة: يريموث، فيلوث، عميبئيل، جينوث، نفوت، فينا. ومن المدينة الرابعة: جموف، إيليئيل، جموث، سولف، رافاف، ودفو. ومن العائلة الخامسة: أنوئيل، كود، فريتان، ريمون، بيكان، نبث. ومن العائلة السادسة: رفاز، سفت، أرافاز، متاخ، أدوك، بلنوك. ومن العائلة السابعة: بنين، مفيز، أراف، رويمل، بيلون، يعل، أباك. ومن البيت (الثامن والتاسع والعاشر): إينوفلاسا، ومليك، ومتوريا، وميك؛ وبقية أمراء السبط الذين بقوا، عددهم ستون.

\par 12 وفي ذلك الوقت، رد الرب على ميخا وأمه كل ما تكلم به. فسُوِّجَ ميخا بالنار، وأمه ذابت، كما تكلم الرب عنهما

\chapter{48}

\par 1 في ذلك الوقت أيضًا، اضطجع فينحاس ليموت، فقال له الرب: هوذا قد تجاوزت المئة والعشرين سنة المحددة لجميع البشر. والآن قم واذهب من هنا واسكن في جبل دانابِن وأقم هناك سنينًا كثيرة، وسأأمر نسري فيرعاك هناك، ولن تنزل إلى البشر بعد حتى يأتي الوقت وتُختبر في الوقت المناسب. وحينئذٍ تغلق السماء، وعند كلمتك تُفتح. وبعد ذلك تُرفع إلى المكان الذي رُفع إليه الذين كانوا قبلك، وتكون هناك حتى أتذكر العالم. وحينئذٍ آتي بك فتذوق ما هو الموت

\par 2 فصعد فينحاس وفعل كل ما أمره به الرب. وفي الأيام التي أقامه فيها كاهنًا، مسحه في سيلو

\par 3 وفي ذلك الوقت، عند صعوده، حدث أن بني إسرائيل، عند احتفالهم بالفصح، أمروا بني بنيامين قائلين: اصعدوا واغتصبوا لأنفسكم نساءً، لأننا لا نستطيع أن نعطيكم بناتنا، لأننا حلفنا في وقت غضبنا، ولا يمكن أن يهلك سبط من إسرائيل. فصعد بنو بنيامين واغتصبوا لأنفسهم نساءً، وبنوا لأنفسهم جبعون، وسكنوا هناك

\par 4 وبينما كان بنو إسرائيل في هذه الأثناء في راحة، لم يكن لهم أمير في تلك الأيام، وكان كل إنسان يفعل ما هو صواب في عينيه

\par 5 هذه هي الوصايا والأحكام والشهادات والبيانات التي كانت في أيام قضاة إسرائيل، قبل أن يملك عليهم ملك

\chapter{49}

\par 1 وفي ذلك الوقت، بدأ بنو إسرائيل يسألون الرب، وقالوا: لنلقِ جميعًا قرعةً لنرى من هو الذي يحكمنا مثل كينيز، فلعلَّنا نجد رجلاً ينقذنا من ضيقاتنا، لأنه ليس من المناسب أن يكون الشعب بلا أمير

\par 2 فألقوا القرعة فلم يجدوا أحدًا، فحزن الشعب حزنًا شديدًا وقالوا: ليس الشعب أهلًا لأن يسمع الرب صوته، لأنه لم يُجبنا. فلنُلقِ الآن قرعةً على الأسباط، لعل الله يرضى بالكثرة، لأننا نعلم أنه سيُصالح من يستحقونه. فألقوا قرعةً على الأسباط، فلم تخرج القرعة على أي سبط. فقال إسرائيل: لنختر من أنفسنا واحدًا، فنحن في ضيق، إذ نرى أن الله يمقت شعبه، وأن نفسه ساخطة علينا.

\par 3 فأجاب واحد وقال للشعب، واسمه نيثيز: ليس هو الذي يكرهنا، بل نحن من جعلنا أنفسنا مكروهين حتى يتركنا الله. ولذلك، حتى لو متنا، فلا نتركه، بل لنلجأ إليه؛ لأننا سلكنا في طرقنا الشريرة ولم نعرف خالقنا، ولذلك ستكون مكرتنا باطلة. لأني أعلم أن الله لن يرفضنا إلى الأبد، ولن يكره شعبه إلى الأجيال كلها. لذلك تقووا الآن ولنصلي مرة أخرى ولنلقِ قرعة على المدن، لأنه وإن كثرت خطايانا، فإن طول أناته لن يفشل

\par 4 وألقوا قرعة على المدن، فأتت القرعة على أرماتيم. فقال الشعب: هل حُسبت أرماتيم بارة على جميع مدن إسرائيل حتى اختارها هكذا دون جميع المدن؟ فقال كل واحد لصاحبه: في تلك المدينة التي خرجت بالقرعة، لنلقِ قرعة على الرجال، ولننظر من اختار الرب منها

\par 5 وألقوا القرعة على الرجال، فلم يُؤخذ أحد إلا إلخانة، لأن القرعة قفزت عليه، فأخذه الشعب وقالوا: تعالَ وكن رئيسًا علينا. فقال إلخانة للشعب: لا أستطيع أن أكون رئيسًا على هذا الشعب، ولا أستطيع أن أحكم من يكون رئيسًا عليكم. ولكن إن كشفتني خطاياي، فقفزت عليّ القرعة، فسأقتل نفسي حتى لا تنجسوني؛ لأنه من العدل أن أموت من أجل خطاياي فقط، ولا أضطر إلى حمل ثقل الشعب

\par 6 ولما رأى الشعب أنه ليس من إرادة إلخانة أن يتولى القيادة عليهم، صلوا أيضًا إلى الرب قائلين: يا رب إله إسرائيل، لماذا تركت شعبك في نصر العدو وأهملت ميراثك في وقت الضيق؟ هوذا حتى الذي أُخذ بالقرعة لم يُنجز أمرك؛ ولكن ما حدث هو أن القرعة قفزت عليه، فظننا أن لنا أميرًا. وها هو أيضًا يُعارض القرعة. من نطلب بعد، أو إلى من نهرب، وأين مكان راحتنا؟ لأنه إن كانت الأوامر التي وضعتها مع آبائنا صحيحة قائلًا: سأُكثر نسلكم، فيعلمون بذلك، فكان خيرًا أن تقول لنا: سأقطع نسلكم، من أن لا تُبالي بأصلنا

\par 7 فقال لهم الله: إن كنتُ قد كافأتكم حسب أعمالكم الشريرة، فلا ينبغي لي أن أصغي إلى شعبكم، ولكن ماذا أفعل لأن اسمي يُدعى عليكم؟ والآن اعلموا أن إلخانة الذي وقعت عليه القرعة لا يستطيع أن يسود عليكم، بل ابنه هو الذي يولد منه، وهو يكون رئيسًا عليكم ويتنبأ، ومن الآن فصاعدًا لا يُنقص لكم رئيسٌ سنين طويلة.

\par 8 فقال الشعب: هوذا يا رب، لإيلخانة عشرة أبناء، فمن منهم يكون أميرًا أو يتنبأ؟ فقال الله: لا يقدر أحد من بني فينينا أن يكون أميرًا على الشعب، إلا من ولد من المرأة العاقر التي أعطيته إياها زوجة، فهو يكون نبيًا أمامي، وأحبه كما أحببت إسحاق، ويكون اسمه أمامي إلى الأبد. وقال الشعب: هوذا الآن، لعل الله قد ذكرنا لينقذنا من أيدي مبغضينا. وفي ذلك اليوم ذبحوا ذبائح سلامة وأكلوا حسب طقوسهم

\chapter{50}

\par 1 الآن [بينما] كان لإيلخانة زوجتان، اسم إحداهما حنة واسم الأخرى فينينا. ولأن فينينا كان لها أبناء، ولم يكن لدى حنة أحد، وبختها فينينا قائلة: ما المنفعة لكِ أن يحبكِ زوجكِ إلخانة؟ وأنتِ شجرة يابسة. وأنا أعلم أيضًا أنه سيحبني، لأنه يُسر برؤية أبنائي واقفين حوله كغرس كرم زيتون

\par 2 وهكذا، عندما كانت تعاتبها كل يوم، وكانت حنة حزينة القلب للغاية، وكانت تخشى الله منذ صغرها، حدث عندما اقترب يوم الفصح الجيد، وصعد زوجها لتقديم الذبيحة، أن فينينا شتمت حنة قائلة: المرأة ليست محبوبة حقًا حتى لو أحبها زوجها أو أحب جمالها. فلا تفتخر حنة بجمالها، بل من يفتخر فليفتخر عندما يرى نسله أمام وجهه؛ وعندما لا يكون الأمر كذلك بين النساء، حتى ثمرة بطونهن، فحينئذٍ يصير الحب بلا قيمة. لأنه ماذا نفع راحيل أن يحبها يعقوب؟ لو لم تُعطَ ثمرة بطنها، لما كان حبه ذا فائدة؟ وعندما سمعت حنة ذلك، ذابت روحها في أحشائها وذرفت عيناها الدموع

\par 3 فرآها زوجها فقال: لماذا أنتِ حزينة ولا تأكلين، ولماذا قلبكِ حزين في داخلكِ؟ أليس سلوككِ أفضل من أبناء فينينا العشرة؟ فأصغت له حنة وقامت بعد أن أكلت، وجاءت إلى سيلو إلى بيت الرب حيث كان يقيم هالي الكاهن، الذي قدمه فينيص بن ألعازار الكاهن كما أمره

\par 4 فصلّت حنة وقالت: "ألم تفحص يا رب قلوب جميع الأجيال قبل أن تُنشئ العالم؟ ولكن ما الرحم الذي يولد مفتوحًا، وما الذي يموت منغلقًا إن لم تشاء؟" والآن، فلتصعد صلاتي أمامك اليوم، لئلا أنزل من هنا فارغًا، لأنك تعرف قلبي، كيف سلكت أمامك منذ أيام شبابي.

\par 5 ولم تُرِد آنا أن تُصلي بصوت عالٍ كما يفعل جميع الرجال، لأنها فكرت في ذلك الوقت قائلة: لئلا لا أكون مستحقة أن أُسمع، فيحسدني فينينا أكثر ويوبخني كما تقول كل يوم: أين إلهك الذي تتوكلين عليه؟ وأنا أعلم أن من لديها أبناء كثيرون ليست هي التي تُغنى، ولا من تفتقر إليهم هي الفقيرة، بل من تكثر في إرادة الله هي التي تُغنى. لأن أولئك الذين يعرفون ما صليتُ لأجله، إذا أدركوا أنني لم أُسْتَمِع في صلاتي، فسوف يجدفون. ولن يكون لي شاهد في روحي فحسب، لأن دموعي أيضًا خادمة لصلواتي

\par 6 وبينما كانت تصلي، رأى هالي الكاهن أنها كانت حزينة في نفسها وتحمل نفسها كالسكرى، فقال لها: اذهبي، أبعدي عنك خمرك. فقالت: هل سمعت صلاتي حتى أُدعى سكرانة؟ حقًا إني سكرانة من الحزن وشربت كأس بكائي

\par 7 فقال لها هالي الكاهن: أخبريني بتوبيخك. فقالت له: أنا زوجة إلخانة، ولأن الله قد أغلق رحمي، لذلك صليت أمامه ألا أغادر هذا العالم إليه بلا ثمر، ولا أموت دون أن أترك صورتي. فقال لها هالي الكاهن: اذهبي، لأني أعلم لماذا صليتِ، وقد سُمعت صلاتك

\par 8 لكن هيلي الكاهن لم يُخبرها أن نبيًا قد قُدِّر له أن يولد منها، لأنه سمع عندما تكلم الرب عنه. وجاءت حنة إلى بيتها، وتعزَّت من حزنها، لكنها لم تُخبر أحدًا بما صلَّت من أجله

\chapter{51}

\par 1 وفي وقت تلك الأيام، حبلت وولدت ابنًا ودعت اسمه صموئيل، الذي يُفسر "قدير"، كما دعا الله اسمه عندما تنبأ عنه. وجلست حنة وأرضعت الولد حتى بلغ من العمر سنتين، ولما فطمته صعدت معه وهي تحمل هدايا في يديها، وكان الولد جميلًا جدًا وكان الرب معه

\par 2 ووضعت حنة الطفل أمام وجه هالي وقالت له: هذه هي الرغبة التي رغبتُ بها، وهذه هي الطلب الذي سعيتُ إليه. فقال لها هالي: لم تطلبيه أنتِ فقط، بل صلّى الناس أيضًا من أجله. إنه ليس طلبك وحدك، بل وُعِدَ به سابقًا للأسباط؛ وبهذا الطفل تبرر رحمك، أن تُقيمي النبوة أمام الشعب، وتُجعلي لبن ثدييك ينبوعًا للأسباط الاثني عشر

\par 3 فلما سمعت حنة ذلك، صلّت وقالت: "استمعوا لصوتي يا جميع الشعوب، وأصغوا إلى كلامي يا جميع الممالك، لأن فمي مفتوح لأتكلم، وشفتاي مأمورتان لأنشد للرب تسبيحًا. اهتفوا يا ثديي وقدموا شهاداتكم، لأنه قد قُدّر لكم أن ترضعوا. لأن من ترضعونه يُقام، وبكلماته تستنير الشعوب، ويُظهر للأمم حدودها، ويرتفع قرنه إلى أسمى مدى".

\par 4 ولذلك سأنطق بكلامي علانية، لأنه مني سينشأ نظام الرب، وسيجد جميع الناس الحق. لا تسرعوا إلى الكلام بكبرياء، ولا تنطقوا بكلمات عالية من أفواهكم، بل استمتعوا بالتفاخر عندما ينبثق النور الذي ستولد منه الحكمة، حتى لا يُدعى الميسورون أغنياء، ولا تُدعى اللواتي ولدن بكثرة أمهات: لأن العاقر قد شبعت، والتي كثرت في البنين صارت فارغة

\par 5 لأن الرب يُميت بالدينونة، ويُحيي بالرحمة. لأن الأشرار موجودون في هذا العالم. لذلك يُحيي الأبرار متى شاء، أما الأشرار فيغلق عليهم في الظلمة. أما الأبرار فيحفظ نورهم، وعندما يموت الأشرار، فحينئذٍ يهلكون، وعندما ينام الأبرار، فحينئذٍ يُنجَزون. وهكذا تبقى كل دينونة إلى أن يُستعلن الذي يحملها

\par 6 تكلمي، تكلمي يا حنة، ولا تسكتي. رنمي يا ابنة بثوئيل، لأجل عجائبك التي صنعها الله معك. من هي حنة حتى يخرج منها نبي؟ أو من هي ابنة بثوئيل حتى تلد نورًا يزعج الشعوب؟ قومي أنت أيضًا يا إلخانة، وشددي حقويك. رنمي لآيات الرب: لأن آساف تنبأ عن ابنك في البرية قائلًا: موسى وهارون بين كهنته وصموئيل بينهم. هوذا الكلمة قد تمت والنبوة قد تحققت. وتستمر هذه الأمور هكذا، حتى يعطي قرنًا لمسيحه، ويلتصق سلطان بعرش ملكه. ولكن ليقف ابني هنا ويخدم، حتى يشرق نور لهذا الشعب

\par 7 فانطلقوا من هناك وانطلقوا فرحين، فرحين، مبتهجين بقلوبهم لكل المجد الذي صنعه الله معهم. ونزل الشعب بنفس واحدة إلى سيلو بالدفوف والرقص، والعود والقيثارات، وجاءوا إلى حالي الكاهن وقدموا له صموئيل، فوضعوه أمام وجه الرب ومسحوه وقالوا: ليسكن النبي بين الشعب، فيكون نورًا لهذه الأمة

\chapter{52}

\par 1 وكان صموئيل صبيًا صغيرًا جدًا، لا يعرف شيئًا عن كل هذه الأمور. وبينما كان يخدم الرب، بدأ ابنا هالي، اللذان لم يسلكا في طريق آبائهما، يرتكبان الشر بالشعب ويزيدان من آثامهما. وسكنا في بيت حَقّ، فلما اجتمع الشعب لتقديم الذبائح، جاء عفني وفينحاس وأغاظا الشعب، وأخذا القرابين قبل تقديم الأقداس للرب.

\par 2 ولم يُرضِ هذا الأمر الرب، ولا الشعب، ولا أباهم. فكلمهم أبوهم هكذا: ما هذا الخبر الذي أسمعه عنكم؟ ألا تعلمون أني قد أخذتُ المكان الذي عهد إليّ به فينحاس؟ وإن كنا نُضيع ما أخذناه، فماذا نقول إن طالبنا به الذي عهد إلينا وأغضبنا على ما عهد إلينا به؟ فالآن قوموا طرقكم، وامشوا في سبل صالحة، فتثبت أعمالكم. ولكن إن قاومتموني ولم تمتنعوا عن مكائدكم الشريرة، فإنكم تُهلكون أنفسكم، ويصير الكهنوت باطلاً، ويزول ما قُدِّس. وحينئذٍ يقولون: باطلا لم تنبت عصا هارون، وباطلا الزهرة التي ولدت منها

\par 3 لذلك، ما دمتم قادرين يا أبنائي، صححوا ما فعلتم من سوء، وسيصلي من أجلكم الرجال الذين أخطأتم في حقهم. ولكن إن لم تفعلوا، بل أصررتم على آثامكم، فسأكون بريئًا، ولن أحزن فقط خشية أن أسمع بيوم موتكم قبل أن أموت، بل سأتبرأ أيضًا إذا حدث هذا (أو حتى لو لم يحدث هذا): وحتى لو كنتُ مبتلى، فستهلكون أنتم مع ذلك

\par 4 ولم يُطعه بنوه، لأن الرب حكم عليهم بالموت لأنهم أخطأوا. لأنه عندما قال لهم أبوهم: توبوا عن طريقكم الرديء، قالوا: عندما نشيخ، حينئذ نتوب. ولهذا السبب لم يُعطَ لهم أن يتوبوا عندما وبخهم أبيهم، لأنهم كانوا دائمًا متمردين، وعملوا ظلمًا كبيرًا في نهب إسرائيل. لكن الرب غضب على هالي

\chapter{53}

\par 1 وكان صموئيل يخدم أمام الرب ولم يكن يعرف بعد ما هي أقوال الرب، لأنه لم يكن قد سمع أقوال الرب بعد، لأنه كان ابن ثماني سنين

\par 2 ولكن عندما تذكر الله إسرائيل، أراد أن يكشف كلماته لصموئيل، وكان صموئيل نائماً في هيكل الرب. وحدث عندما دعاه الله، أنه فكر أولاً، وقال: هوذا الآن، صموئيل صغير ليكون (أو على الرغم من أنه) محبوباً في عيني؛ ومع ذلك لأنه لم يسمع بعد صوت الرب، ولم يثبت لصوت العلي، إلا أنه يشبه موسى عبدي: لقد كلمت موسى عندما كان ابن 80 سنة، وكان صموئيل ابن 8 سنوات. ورأى موسى النار أولاً فخاف قلبه. وإذا رأى صموئيل النار الآن، فكيف يتحملها؟ لذلك سيأتي إليه الآن صوت كصوت إنسان، لا صوت إله. وعندما يفهم، سأتحدث إليه كإله.

\par 3 وفي منتصف الليل ناداه صوت من السماء. فاستيقظ صموئيل وسمع كصوت هالي، فركض إليه وتكلم قائلاً: لماذا أيقظتني يا أبي؟ لأني كنت خائفًا، لأنك لم تدعني قط في الليل. فقال هالي: ويل لي، أيمكن أن يكون روح نجس قد خدع ابني صموئيل؟ فقال له: اذهب ونم، لأني لم أدعك. ولكن أخبرني إن كنت تتذكر كم مرة صرخ الذي دعاك. فقال: مرتين. فقال له هالي: قل الآن، من عرفت صوته يا ابني؟ فقال: من صوتك، لذلك ركضت إليك

\par 4 فقال هالي: فيك أرى العلامة التي ستكون للناس من هذا اليوم فصاعدًا إلى الأبد، أنه إذا دعا أحدٌ آخر مرتين في الليل أو في الظهيرة، فسيعرفون أنه روح شرير. ولكن إذا دعا مرة ثالثة، فسيعرفون أنه ملاك. ومضى صموئيل ونام

\par 5 فسمع في المرة الثانية صوتًا من السماء، فقام وركض إلى هالي وقال له: لماذا دعاني وأنا سمعت صوت إلخانة أبي؟ ففهم هالي أن الله بدأ يدعوه. فقال هالي: بهذين الصوتين اللذين دعاك بهما الله، شبّه نفسه بأبيك وبسيدك، أما الآن فسيتكلم كإله

\par 6 فقال له: أصغِ بأذنك اليمنى، وباليسرى لَكَ. لأن فينحاس الكاهن أوصانا قائلاً: الأذن اليمنى تسمع الرب ليلاً، واليسرى ملاك. فإن كنت تسمع بأذنك اليمنى، فقل هكذا: تكلم بما تريد، لأني أسمعك، لأنك أنت شكّلتني. وإن كنت تسمع باليسرى، فتعالَ وأخبرني. فمضى صموئيل ونام كما أمره هالي

\par 7 ثم أضاف الرب وتكلم أيضًا ثالثة، فامتلأت أذن صموئيل اليمنى بالصوت. ولما علم أن كلام أبيه قد نزل إليه، التفت صموئيل إلى جنبه الآخر وقال: إن كنت أستطيع فتكلم، لأنك أنت من كوّنتني (أو تعرفني جيدًا).

\par 8 وقال له الله: إني أنرت بيت إسرائيل في مصر، واخترت لنفسي في ذلك الوقت موسى عبدي نبيا، وبواسطته صنعت عجائب لشعبي، وانتقمت لهم من أعدائي كما أريد، وأخذت شعبي إلى البرية، وأنارتهم كما نظروا.

\par 9 ولما قام سبط على سبط آخر قائلًا: لماذا الكهنة وحدهم مقدسون؟ لم أُرِد أن أُهلكهم، بل قلت لهم: أعطوا كل واحد عصاه، فمن أزهر عصاه فقد اخترته للكهنوت. ولما أعطوا كل واحد عصيه كما أمرت، أمرت أرض المسكن أن يُزهر عصا هارون، لكي يظهر نسله أيامًا كثيرة. والآن أولئك الذين أزهروا قد كرهوا أقداسي

\par 10 لذلك، هوذا أيام تأتي وأقطع (حرفيًا: أوقف) الزهرة التي خرجت في ذلك الوقت، وأخرج عليهم لأنهم يتعدون على الكلام الذي كلمت به عبدي موسى قائلًا: إذا صادفت عشًا، فلا تأخذ الأم مع الصغار، فيصيبهم أن الأمهات تموت مع الأولاد، ويهلك الآباء مع الأبناء

\par 11 فلما سمع صموئيل هذه الكلمات ذاب قلبه، وقال: هل أصابني في صباي أن أتنبأ لهلاك من رباني؟ فكيف أُعطيت بناءً على طلب أمي؟ ومن هو الذي رباني؟ كيف أوصاني أن أبشر بأخبار سيئة؟

\par 12 فقام صموئيل في الصباح ولم يشأ أن يخبر هالي. فقال له هالي: اسمع يا ابني. هوذا قبل أن تولد وعد الله بني إسرائيل أنه سيرسلك إليهم لتتنبأ. والآن، لما جاءت أمك إلى هنا وصلّت، لأنها لم تعلم ما حدث، قلت لها: اذهبي، لأن الذي يولد منك يكون لي ابنًا. هكذا كلمت أمك، وهكذا أرشد الرب طريقك. وإن كنت تؤدب أباك المرضع، حي هو الرب، فلا تخف عني ما سمعت

\par 13 فخاف صموئيل وأخبره بكل الكلام الذي سمعه. فقال: هل يستطيع الشيء المصنوع أن يجيب صانعه؟ هكذا أنا أيضًا لا أستطيع أن أجيب متى أخذ ما أعطاه، حتى المعطي الأمين، القدوس الذي تنبأ، لأني خاضع لسلطانه



\chapter{54}

\par 1 وفي تلك الأيام، جمع الفلسطينيون معسكرهم لمحاربة إسرائيل، فخرج بنو إسرائيل لمحاربتهم. ولما انهزم بنو إسرائيل في المعركة الأولى، قالوا: لنصعد تابوت عهد الرب، لعله يحاربنا، لأن فيه شهادات الرب التي أمر بها لآبائنا في غراب

\par 2 ولما صعد التابوت معهم، ودخل المحلة، رعد الرب وقال: «سيكون هذا الوقت كالذي كان في البرية، حين أخذوا التابوت بغير أمري، فحل بهم الهلاك. كذلك في هذا الوقت أيضًا يسقط الشعب، ويُؤخذ التابوت، لأعاقب أعداء شعبي بسبب التابوت، وأوبخ شعبي على خطاياهم».

\par 3 ولما دخل التابوت إلى المعركة، خرج الفلسطينيون للقاء بني إسرائيل وضربوهم. وكان هناك رجل اسمه جوليا، وهو فلسطيني، قد جاء إلى التابوت، وكان عفني وفينحاس ابنا حالي وشاول بن قيس ممسكين بالتابوت. فأخذه جوليا بيده اليسرى وقتل عفني وفينحاس

\par 4 أما شاول، فكان خفيفًا على قدميه، فهرب من أمامه، ومزق ثيابه، ووضع رمادًا على رأسه، وجاء إلى عالي الكاهن. فقال له عالي: أخبرني ماذا حدث في المحلة؟ فقال له شاول: لماذا تسألني عن هذا؟ لأن الشعب قد انهزم، والله قد ترك إسرائيل، والكهنة أيضًا قُتلوا بالسيف، وسُلِّم التابوت إلى الفلسطينيين

\par 5 ولما سمع هالي بأخذ التابوت، قال: هوذا صموئيل تنبأ عني وعن أبنائي أننا سنموت معًا، ولكن التابوت لم يسمِّه لي. والآن قد سُلِّمت الشهادات للعدو، فماذا أقول بعد؟ هوذا إسرائيل قد هلك عن الحق، لأن الأحكام قد رُفعت عنه. ولأن هالي قد يئس تمامًا، سقط عن كرسيه. وماتوا في يوم واحد، هالي وعفني وفينحاس أبناؤه

\par 6 وجلست امرأة ابن هالي وتولدت. ولما سمعت هذا ذابت أحشاؤها كلها. فقالت لها القابلة: ثقي ولا تكل نفسك، لأنه قد ولد لك ابن. فقالت لها: هوذا الآن ولدت نفس واحدة ونحن أربعة نموت، أبي وابناه وكنته. ودعت اسمه: أين المجد؟ قائلة: قد هلك مجد الله في إسرائيل لأن تابوت الرب قد أُخذ. ولما قالت هذا أسلمت روحها

\chapter{55}

\par 1 لكن صموئيل لم يكن يعلم شيئًا عن كل هذه الأمور، لأنه قبل المعركة بثلاثة أيام أرسله الله قائلًا له: اذهب وانظر إلى موضع الرامة، هناك يكون مسكنك. ولما سمع صموئيل بما حل بإسرائيل، جاء وصلى إلى الرب قائلًا: هوذا الآن، عبثًا يُحرمني الفهم أن أرى هلاك شعبي. والآن أخشى أن تشيخ أيامي في الشر وتنتهي سني بالحزن، لأنه طالما أن تابوت الرب ليس معي، فلماذا أعيش بعد؟

\par 2 فقال له الرب: لا تحزن يا صموئيل لأن التابوت قد أُخذ. سأعيده، وأُبيد الذين أخذوه، وأنتقم لشعبي من أعدائهم. فقال صموئيل: هوذا، حتى لو انتقمت لهم في الوقت المناسب، حسب طول أناتك، فماذا نفعل نحن الذين نموت الآن؟ فقال له الله: قبل أن تموت، سترى النهاية التي أجلبها على أعدائي، حيث يهلك الفلسطينيون ويقتلون بالعقارب وكل أنواع الزاحفات المزعجة.

\par 3 ولما وضع الفلسطينيون تابوت الرب الذي أُخذ في هيكل داجون إلههم، وجاءوا ليستفسروا من داجون عن خروجهم، وجدوه ساقطًا على وجهه ويداه ورجلاه موضوعة أمام التابوت. فخرجوا في الصباح الأول وقد صلبوا كهنته. وفي اليوم الثاني جاؤوا فوجدوا كما في اليوم السابق، وقد كثر الخراب بينهم جدًا

\par 4 لذلك اجتمع الفلسطينيون في عكارون، وقال كل رجل لجاره: هوذا نرى أن الهلاك قد كثر بيننا، وثمرة أجسادنا قد هلكت، لأن الدبابير المرسلة علينا تُهلك الحوامل والرضع والمرضعات أيضًا. وقالوا: لننظر لماذا تقوى يد الرب علينا. هل من أجل التابوت؟ لأنه كل يوم يوجد إلهنا ساقطًا على وجهه أمام التابوت، وقد قتلنا كهنتنا عبثًا مرارًا وتكرارًا

\par 5 فقال حكماء الفلسطينيين: هوذا الآن يمكننا أن نعرف بهذا هل أرسل الرب علينا هلاكًا من أجل تابوته، أم هل أصابتنا بلاء عرضي إلى حين؟

\par 6 والآن، بينما تموت كل الحوامل والمرضعات، والمرضعات يبقين بلا أولاد، والمرضعات يهلكن، فإننا سنأخذ أيضًا البقرات المرضعات ونربطها على عجلة جديدة، ونضع الفلك عليها، ونحبس صغار البقر. ويكون أنه إذا خرجت البقرات ولم ترجع إلى صغارها، سنعلم أننا عانينا هذه الأمور من أجل الفلك؛ ولكن إذا رفضت الذهاب، وهي تتوق إلى صغارها، سنعلم أن وقت سقوطنا قد حان علينا

\par 7 فأجاب بعض الحكماء والعرافين: لا تكتفوا بهذا، بل لنضع البقرات على رأس الطرق الثلاثة المحيطة بعكارون. لأن الطريق الأوسط يؤدي إلى عكارون، والطريق الأيمن إلى اليهودية، والطريق الأيسر إلى السامرة. ووجهوا البقرات التي تحمل التابوت إلى الطريق الأوسط. فإن سارت في الطريق الأيمن مستقيمة إلى اليهودية، فسنعلم أن إله اليهود قد أهلكنا حقًا؛ ولكن إن سلكت تلك الطرق الأخرى، فسنعلم أن زمنًا سيئًا قد أصابنا، لأننا الآن أنكرنا آلهتنا

\par 8 فأخذ الفلسطينيون بقراتٍ مرضعةً وربطوها على عجلةٍ جديدةٍ ووضعوا التابوت عليها، وأقاموها في رأس الطرق الثلاثة، وحبسوا صغارها في البيوت. ومع أن البقرات كانت تخور وتشتاق إلى صغارها، إلا أنها سارت في الطريق الأيمن المؤدي إلى اليهودية. فعرفوا حينئذٍ أنهم خربوا من أجل التابوت.

\par 9 فاجتمع جميع الفلسطينيين وأعادوا التابوت إلى سيلو بدفوف ومزامير ورقصات. وبسبب الدبابات الكريهة التي كانت تُخربهم، صنعوا مقاعد من ذهب وقدسوا التابوت

\par 10 وفي ضربة الفلسطينيين تلك، كان عدد الحوامل اللاتي ماتوا 75 ألفًا، والمرضعات 65 ألفًا، والمرضعات 55 ألفًا، والرجال 25 ألفًا. واستراحت الأرض سبع سنين

\chapter{56}

\par 1 وفي ذلك الوقت طلب بنو إسرائيل ملكًا على أبرارهم. فاجتمعوا إلى صموئيل وقالوا: هوذا أنت قد شخت، وابناك لا يسلكان في طرق الرب. فالآن اجعل علينا ملكًا يقضي بيننا، لأنه قد تم القول الذي كلم به موسى آباءنا في البرية قائلًا: تجعل عليك أميرًا من إخوتك

\par 2 ولما سمع صموئيل ذكر المملكة، حزن حزنًا شديدًا في قلبه، وقال: ها أنا أرى الآن أنه ليس لنا وقت (أو ليس بعد) لمملكة أبدية، ولا لبناء بيت الرب إلهنا، لأن هؤلاء يطلبون ملكًا قبل الوقت. والآن، إذا رفض الرب ذلك رفضًا قاطعًا (أو حتى لو أراد الرب ذلك)، فيبدو لي أنه لا يمكن إقامة ملك

\par 3 فقال له الرب في الليل: لا تحزن، لأني سأرسل لهم ملاكًا يُهلكهم، وهو نفسه يُهلك بعد ذلك. والآن، من يأتي إليك غدًا في الساعة السادسة، فهو الذي يملك عليهم

\par 4 وفي الغد، كان شاول بن قيس قادمًا من جبل أفرام يطلب حمير أبيه. ولما وصل إلى رماتيم، دخل إلى صموئيل ليسأله عن حمير أبيه. وكان يمشي مسرعًا بجانب بعْم، فقال له شاول: أين الذي يبصر؟ لأنه كان في ذلك الوقت نبي يُدعى الرائي. فقال له صموئيل: أنا هو الذي يبصر. فقال: أتستطيع أن تخبرني عن حمير أبي؟ لأنها ضائعة

\par 5 فقال له صموئيل: تكلّم معي اليوم، وفي الصباح أخبرك بما جئت لتستفسر عنه. فقال صموئيل للرب: وجّه يا رب شعبك، وأظهر لي ما قررته بشأنهم. فتكلّم شاول مع صموئيل في ذلك اليوم وقام في الصباح. فقال له صموئيل: انظر، اعلم أن الرب قد اختارك لتكون رئيسًا على شعبه في هذا الوقت، وأصلح طرقك، فيُرشد وقتك

\par 6 فقال شاول لصموئيل: من أنا، وما هو بيت أبي حتى يُكلِّمني سيدي هكذا؟ لأني لا أفهم ما تقول، لأني شاب. فقال صموئيل لشاول: من يُعطي كلامك أن يتمَّ حتى تعيش أيامًا كثيرة؟ لكن انظر هذا، أن كلامك يُشبَّه بكلام نبيٍّ اسمه هيرميا.

\par 7 ولما انصرف شاول في ذلك اليوم، جاء الشعب إلى صموئيل قائلين: أعطنا ملكًا كما وعدتنا. فقال لهم: هوذا الملك يأتي إليكم بعد ثلاثة أيام. وإذا شاول قد جاء. وحدثت له جميع الآيات التي أخبره بها صموئيل. أليست هذه الأمور مكتوبة في سفر الملوك؟

\chapter{57}

\par 1 فأرسل صموئيل وجمع كل الشعب، وقال لهم: هوذا أنتم وملككم ههنا، وأنا بينكم كما أمرني الرب

\par 2 ولذلك أقول لكم أمام وجه ملككم، كما قال سيدي موسى، عبد الله، لآبائكم في البرية، حين قام عليه مجمع قورح: أنتم تعلمون أني لم آخذ منكم شيئًا، ولا أخطأت في حق أحد منكم. ولأن قومًا كذبوا في ذلك الوقت وقالوا: أخذت، ابتلعتهم الأرض

\par 3 والآن، أنتم الذين لم يعاقبهم الرب، أجيبوا أمام الرب وأمام مسيحه، إن كنتم قد طلبتم ملكًا لهذا السبب، لأني أسأت إليكم، والرب سيكون شاهدًا لكم. ولكن إن تمت كلمة الرب، فأنا حر وبيت أبي

\par 4 فأجاب الشعب: نحن عبيدك وملكنا معنا، لأننا لسنا مستحقين أن يحكم علينا نبي، لذلك قلنا: اجعل علينا ملكًا يحكم علينا. فبكى جميع الشعب والملك بكاءً عظيمًا، وقالوا: ليحيَ صموئيل النبي. ولما عُيّن الملك ذبحوا ذبائح للرب

\par 5 وبعد ذلك حارب شاول الفلسطينيين سنة واحدة، وازدهرت الحرب كثيرًا

\chapter{58}

\par 1 وفي ذلك الوقت قال الرب لصموئيل: اذهب وقل لشاول: أنت مُرسل لتهلك عماليق، لكي يتم الكلام الذي تكلم به موسى عبدي قائلاً: سأمحو اسم عماليق من الأرض التي تكلمت عنها بغضبي. ولا تنسَ أن تهلك كل نفس منهم كما أمرتك

\par 2 فذهب شاول وحارب عماليق، وأنقذ أجاج ملك عماليق، لأنه قال له: أريك كنوزًا مخفية. فأحياه وأنقذه، وجاء به إلى أرماتيم.

\par 3 فقال الله لصموئيل: أرأيت كيف فسد الملك بالفضة لحظة، وأنقذ أجاج ملك عماليق وزوجته؟ فالآن دع أجاج وزوجته يجتمعان هذه الليلة، وغدا تقتله. أما زوجته فيبقون عليها حتى تلد ذكرًا، ثم تموت هي أيضًا، ويكون المولود منها عثرة لشاول. أما أنت، فقم في الغد واقتل أجاج، لأن خطيئة شاول مكتوبة أمامي دائمًا.

\par 4 ولما قام صموئيل في الغد، خرج شاول للقائه وقال له: قد دفع الرب أعداءنا في أيدينا كما قال. فقال صموئيل لشاول: من ظلم إسرائيل؟ لأنه قبل أن يأتي الوقت الذي يملك فيه ملك عليه، طلبك ملكًا، وأنت لما أُرسلت لتفعل مشيئة الرب، تعديتها. لذلك فإن من أنقذته حيًا سيموت الآن، والكنوز المخبأة التي تكلم عنها لن يُريكها، وكل من يولد منه يكون عثرة لك. فجاء صموئيل إلى أجاج بالسيف وقتله، ورجع إلى بيته

\chapter{59}

\par 1 فقال له الرب: اذهب وامسح الذي أقول لك، لأنه قد كمل الوقت الذي يأتي فيه ملكوته. فقال صموئيل: هوذا الآن تمحو مملكة شاول؟ فقال: سأمحوها

\par 2 فخرج صموئيل إلى بيت إيل، وقدس الشيوخ ويسى وبنيه. فجاء إليآب بكر يسى. فقال صموئيل: هوذا القدوس مسيح الرب. فقال له الرب: أين رؤياك التي أبصرها قلبك؟ ألست أنت الذي قلت لشاول: أنا هو البصير؟ وكيف لا تعرف من تمسح؟ والآن يكفيك هذا التوبيخ، وابحث عن الراعي الأصغر بينهم جميعًا، وامسحه

\par 3 فقال صموئيل ليسى: اسمع يا يسى، أرسل وقدم ابنك من الغنم، لأنه قد اختاره الله. فأرسل يسى وأحضر داود، ومسحه صموئيل في وسط إخوته. وكان الرب معه من ذلك اليوم فصاعدًا

\par 4 ثم بدأ داود يترنم بهذا المزمور، وقال: في أقاصي الأرض أبدأ أمجده، وإلى الأيام الأبدية أُرنم. كان هابيل في البداية، حين رعى الغنم، مقبولاً أكثر من ذبيحة أخيه. فحسده أخوه فقتله. أما أنا، فليس الأمر كذلك، لأن الله حفظني، وأسلمني إلى ملائكته ومراقبيه ليحفظوني، لأن إخوتي حسدوني، وأبي وأمي أهملاني، ولما جاء النبي لم يدعوا لي، ولما بُشّر بمسيح الرب نسياني. لكن الله اقترب إليّ بيمينه ورحمته، لذلك لا أكف عن التسبيح كل أيام حياتي.

\par 5 وبينما كان داود يتكلم، إذا بأسدٍ شرسٍ من الغابة ودبةٍ من الجبل قد أخذا ثيران داود. فقال داود: "ها هي علامةٌ لي على بداية انتصاري العظيم في المعركة. سأخرج وراءهم وأنقذ ما يُسلب وأقتلهم". فخرج داود وراءهم وأخذ حجارةً من الغابة وقتلهم. فقال له الله: "ها هي ذي الحجارة قد خلصتك من هذه الحيوانات أمام عينيك. وهذه علامةٌ لك على أنك ستقتل بالحجارة بعد ذلك عدو شعبي".

\chapter{60}

\par 1 وفي ذلك الوقت، ارتفع روح الرب عن شاول، وخنقه روح شرير. فأرسل شاول وأخذ داود، وعزف على قيثارته في الليل. وهذا هو المزمور الذي غنّاه لشاول ليذهب عنه الروح الشرير

\par 2 كان هناك ظلام وصمت قبل أن يكون العالم، وتكلم الصمت، وأصبح الظلام مرئيًا. ثم خُلق اسمك، حتى عند جمع ما كان ممتدًا، حيث سُميت العليا سماءً والسفلية أرضًا. وأُمر العليا أن تمطر حسب موسمها، والسفلى أن تُنتج طعامًا للإنسان يُصنع. وبعد ذلك خُلقت قبيلة أرواحك

\par 3 الآن، لا تكن مؤذيًا، ما دمت خليقة ثانية، ولكن إن لم تكن كذلك، فتذكر الجحيم (حرفيًا: انتبه لتارتاروس) الذي كنت تسير فيه. أو ألا يكفيك أن تسمع أنني من خلال ما يتردد أمامك أغني للكثيرين؟ أم تنسى أنه من صدى مرتد في الهاوية (أو الفوضى) وُلدت خليقتك؟ لكن ذلك الرحم الجديد سيوبخك، الذي منه أنا ولدت، والذي منه سيولد بعد زمن من صلبي من سيخضعك

\par وعندما غنى داود التسبيح، أنقذ الروح شاول.

\chapter{61}

\par 1 وبعد هذه الأمور، جاء الفلسطينيون لمحاربة إسرائيل. فعاد داود إلى البرية ليرعى غنمه، فجاء المديانيون وأرادوا أن يأخذوا غنمه، فنزل إليهم وحاربهم وقتل منهم خمسة عشر ألف رجل. هذه هي المعركة الأولى التي خاضها داود في البرية

\par 2 فخرج رجل من معسكر الفلسطينيين اسمه جوليا، فرأى شاول وإسرائيل، فقال: ألست أنت شاول الذي هرب من أمامي حين أخذت التابوت منكم وقتلت كهنتكم؟ والآن وقد ملكت، أتنزل إليّ كرجل وملك وتحاربنا؟ وإلا فسآتي إليك وأأسرك وأعبد شعبك آلهتنا. فلما سمع شاول وإسرائيل ذلك خافا خوفًا شديدًا. فقال الفلسطيني: حسب عدد الأيام التي احتفل فيها إسرائيل حين أخذوا الشريعة في البرية، أربعين يومًا، سأوبخهم، وبعد ذلك أحاربهم.

\par 3 "وحدث لما تمت الأربعون يوما وجاء داود لينظر حرب إخوته أنه سمع الكلام الذي تكلم به الفلسطيني فقال لعل هذا هو الوقت الذي قال لي الله عنه أدفع عدو شعبي في يدك بالحجارة؟"

\par 4 فسمع شاول هذا الكلام فأرسل وأخذه وقال: ما هو الكلام الذي كلمت به الشعب؟ فقال داود: لا تخف أيها الملك، لأني سأذهب وأحارب الفلسطيني، فينزع الله البغضاء والعار عن إسرائيل

\par 5 فخرج داود وأخذ سبعة أحجار وكتب عليها أسماء آبائه: إبراهيم وإسحاق ويعقوب وموسى وهارون، واسمه واسم القدير. وأرسل الله كرڤيهيل، الملاك الذي على القوة

\par 6 فخرج داود إلى جوليا وقال له: اسمع كلمة قبل أن تموت. أليست المرأتان اللتان وُلدتُ منهما أنا وأنت أختين؟ وكانت أمك عُرفة وأمي راعوث. فاختارت عُرفة لنفسها آلهة الفلسطينيين وسارت وراءهم، أما راعوث فاختارت لنفسها طرق الأقوياء وسارت فيها. والآن أنت وإخوتك من عُرفة، وكما أنك قمت اليوم وجئت لتخريب إسرائيل، ها أنا أيضًا المولود من عشيرتك قد أتيت لأنتقم لشعبي. لأن إخوتك الثلاثة أيضًا سيقعون في يدي بعد وفاتك. وحينئذ تقول لأمك: الذي وُلِد من أختك لم يشفق علينا

\par 7 فوضع داود حجرًا في مقلاعه وضرب الفلسطيني في جبهته، وركض نحوه واستل سيفه من غمده وأخذ رأسه منه. فقال له جوليا وهو لا يزال حيًا: أسرع واقتلني وافرح

\par 8 فقال له داود: قبل أن تموت، افتح عينيك وانظر إلى قاتلك الذي قتلك. فنظر الفلسطيني ورأى الملاك وقال: لم تقتلني وحدك، بل الذي كان معك الذي صورته ليست كصورة إنسان. ثم أخذ داود رأسه عنه

\par 9 ورفع ملاك الرب وجه داود، فلم يعرفه أحد. ولما رأى شاول داود سأله من هو، فلم يكن أحد يعرفه من هو

\chapter{62}

\par 1 وبعد هذه الأمور، حسد شاول داود وطلب قتله. ولكن داود ويوناثان ابن شاول قطعا عهدًا معًا. ولما رأى داود أن شاول يطلب قتله، هرب إلى أرماتيم، وخرج شاول وراءه

\par 2 وحلَّ الروح على شاول، فتنبأ قائلاً: لماذا تضلُّ يا شاول، أو من تضطهد عبثًا؟ قد كمل وقت ملكك. اذهب إلى مكانك، لأنك ستموت وداود سيملك. ألا تموت أنت وابنك معًا؟ وحينئذٍ تظهر مملكة داود. فذهب الروح من شاول، ولم يكن يعلم ما تنبأ به

\par 3 فجاء داود إلى يوناثان وقال له: تعالَ نعقد عهدًا قبل أن نفترق. لأن شاول أباك يطلب قتلي بلا سبب، ولأنه علم أنك تحبني، لم يُخبرك بما يُفكّر فيّ.

\par 4 لكن لهذا السبب يكرهني، لأنك تحبني، ولئلا أملك عوضًا عنه. وبما أنني أحسنت إليه، فإنه يجازيني بالشر. وبما أنني قتلت جوليا بكلمة القدير، فانظر أي عاقبة ينويها لي. لأنه عزم على بيت أبي أن يهلكه. وليُوضع حكم الحق في الميزان، حتى يسمع جمهور الحكماء الحكم

\par 5 والآن أخشى أن يقتلني ويفقد حياته من أجلي. لأنه لن يسفك دمًا بريئًا دون عقاب. لماذا تُضطهد نفسي؟ لأني كنت الأصغر بين إخوتي، أرعى الغنم، فلماذا أنا في خطر الموت؟ لأني بار وليس لي إثم. ولماذا يبغضني أبوك؟ وبر أبي سيساعدني على ألا أقع في يدي أبيك. وبما أنني صغير السن وطري السن، فلا يحسدني شاول

\par 6 لو كنت قد ظلمته، لكنت طلبت منه أن يغفر لي خطيئتي. لأنه إن كان الله يغفر الإثم، فكم بالحري أبوك الذي هو من لحم ودم؟ لقد سلكت في بيته بقلب كامل، نعم، نشأت أمام وجهه كالنسر السريع، وضعت يدي على القيثارة وباركته بالأغاني، وقد خطط لقتلي، ومثل عصفور يهرب من وجه الصقر، كذلك هربت من وجهه

\par 7 لمن كلمتُ هذا، أو لمن أخبرتُ بالأمور التي عانيتُ منها إلا أنتِ وملكول أختكِ؟ لأننا كلينا لنذهب معًا بالحق

\par 8 وكان خيرًا يا أخي أن أُقتل في المعركة من أن أقع في يد أبيك، لأن عيني كانتا تنظران في المعركة من كل جانب لأدافع عنه من أعدائه. يا أخي يوناثان، اسمع كلامي، وإن كان فيّ إثم، فبخني

\par 9 فأجاب يوناثان وقال: تعال إليّ يا أخي داود، فأخبرك ببرك. نفسي تذبل بشدة من حزنك لأننا الآن قد افترقنا بعضنا عن بعض. وهذا ما أجبرتنا عليه خطايانا، أن نفترق بعضنا عن بعض. ولكن دعونا نتذكر بعضنا البعض نهارًا وليلًا ونحن أحياء. وحتى لو فرقنا الموت، فأنا أعلم أن أرواحنا ستعرف بعضها البعض. لأن لك الملكوت في هذا العالم، ومنك تكون بداية الملكوت، وهو يأتي في وقته

\par 10 والآن، كما يفطم الطفل عن أمه، كذلك يكون فراقنا. لتشهد السماء وتشهد الأرض على ما تكلمنا به معًا. ولنبك بعضنا مع بعض، ولنجمع دموعنا في إناء واحد، ونضع الإناء على الأرض، فيكون لنا شهادة. فناح كل واحد منهما على الآخر بحرارة، وقبل بعضهما بعضًا. فخاف يوناثان وقال لداود: لنتذكر يا أخي العهد الذي بيننا والقسم الذي في قلوبنا. وإن متُّ أمامك وملكتَ كما تكلم الرب، فلا تذكر غضب أبي، بل العهد الذي بيني وبينك. ولا تفكر في البغضاء الذي يبغضك به أبي باطلا، بل في محبتي التي أحببتك بها. لا تُفكّر فيما ظلمك به أبي، بل تذكّر المائدة التي تناولناها معًا. ولا تُفكّر في الحسد الذي حسدك به أبي، بل في الإيمان الذي نحفظه أنا وأنت. ولا تُبالي بالكذب الذي كذب به شاول، بل بالأيمان التي حلفناها. وقبلا بعضنا بعضًا. وبعد ذلك انصرف داود إلى البرية، ودخل يوناثان المدينة.

\chapter{63}

\par 1 في ذلك الوقت، كان الكهنة الساكنون في نوبا ينجسون أقداس الرب ويجعلون الباكورة عارا على الشعب. فغضب الله وقال: هأنذا أبيد الكهنة الساكنين في نوبا، لأنهم يسلكون في طرق بني هالي

\par 2 وفي ذلك الوقت جاء دوك الأرامي الذي كان على بغال شاول إلى شاول وقال له: ألا تعلم أن أبيمالك الكاهن قد تشاور مع داود وأعطاه سيفًا وأطلقه بسلام؟ فأرسل شاول ودعا أبيمالك وقال له: تموت موتًا لأنك تشاورت مع عدوي. فقتل شاول أبيمالك وكل بيت أبيه، ولم ينجو من سبطه أحد إلا أبياثار ابنه فقط. فجاء هو إلى داود وأخبره بكل ما أصابه

\par 3 وقال الله: هوذا في السنة التي ملك فيها شاول، حين أخطأ يوناثان وأراد قتله، قام هذا الشعب ولم يدعه، والآن حين قُتل الكهنة، 385 رجلاً، سكتوا ولم يتكلموا. لذلك، هوذا أيام تأتي سريعًا وأسلمهم في أيدي أعدائهم، فيسقطون جرحى هم وملكهم

\par 4 ولِدَوْخَ الأَرَامِيِّ هكَذَا قَالَ الرَّبُّ: هُوَذَا أَيَّامٌ سَتَأْتِي سَرِيعًا وَتَطْعَمُ الدَّوْدُ عَلَى لَسَانِهِ وَتُصِيبَهُ بِالذَّبْلِ، وَيَكُونُ مَسْكَنُهُ مَعَ يَائِيرَ إِلَى الأَبَدِ فِي النَّارِ الَّتِي لَا تُطْفَأُ

\par 5 وكل ما فعله شاول، وبقية أقواله، وكيف سعى وراء داود، أليست مكتوبة في سفر ملوك إسرائيل؟

\par 6 وبعد هذه الأمور مات صموئيل، فاجتمع كل إسرائيل وندبوه ودفنوه.

\chapter{64}

\par 1 ثم فكر شاول قائلاً: سأزيل السحرة من أرض إسرائيل. فيذكرني الناس بعد ذهابي. فبدد شاول جميع السحرة من الأرض. وقال الله: هوذا شاول قد أزال السحرة من الأرض، ليس من أجل خوفي، بل ليصنع لنفسه اسمًا. هوذا من بدد، فليأتِ إليهم ويأخذ عرافة منهم، لأنه ليس له أنبياء

\par 2 في ذلك الوقت، قال الفلسطينيون، كل رجل لجاره: هوذا صموئيل النبي قد مات، وليس هناك من يصلي من أجل إسرائيل. وداود أيضًا، الذي حارب عنهم، قد صار خصمًا لشاول، وهو ليس معهم. فالآن، فلنقم ونقاتلهم بشدة، وننتقم لدماء آبائنا. فاجتمع الفلسطينيون وصعدوا للقتال

\par 3 ولما رأى شاول أن صموئيل قد مات وداود ليس معه، انفكت يداه. فسأل الرب فلم يسمع له. وطلب أنبياء فلم يظهر له أحد. فقال شاول للشعب: لنبحث عن عراف ونسأله عما يجول في خاطري. فأجابه الشعب: هوذا امرأة اسمها صدقة بنت دبين (أو أدود) المدياني، قد خدعت بني إسرائيل بالسحر، وها هي ساكنة في عين دور

\par 4 فلبس شاول ثيابا رثة وذهب إليها هو ورجلان معه ليلًا وقال لها: ارفعي لي صموئيل. فقالت: إني خائفة من الملك شاول. فقال لها شاول: لا يضرك شاول في هذا الأمر. وقال شاول في نفسه: لما كنت ملكًا في إسرائيل، لم يرني الأمم، لكنهم عرفوا أني شاول. فسأل شاول المرأة قائلًا: هل رأيتِ شاول قط؟ فقالت: مرات عديدة. فخرج شاول وبكى وقال: هوذا الآن علمت أن جمالي قد تغير، وأن مجد مملكتي قد زال عني

\par 5 وكان لما رأت المرأة صموئيل قادمًا، ورأت شاول معه، أنها صرخت وقالت: هوذا أنت شاول، لماذا خدعتني؟ فقال لها: لا تخافي، بل أخبريني بما رأيت. فقالت: هوذا هذه الأربعون سنة أقمت موتى للفلسطينيين، ولم يُرَ هذا المنظر، ولن يُرى بعد

\par 6 فقال لها شاول: ما هي صورته؟ فقالت: أنت تسألني عن الآلهة. هوذا صورته ليست صورة إنسان. فهو لابس ثوبًا أبيض وعليه رداء، ويقوده ملاكان. فتذكر شاول الرداء الذي مزقه صموئيل في حياته، فصفق بيديه وألقى نفسه على الأرض

\par 7 فقال له صموئيل: لماذا أزعجتني بإقامتي؟ ظننتُ أن الوقت قد حان لأُجازي على أعمالي. فلا تفتخر أيها الملك، ولا أنتِ أيتها المرأة. لأنكم لستم أنتم من أقامني، بل الوصية التي كلمني بها الله وأنا حي، أن آتي وأخبرك أنك أخطأت مرة أخرى بإهمالك الله. لهذا السبب، تهتز عظامي بعد أن بذلت نفسي، لأُكلّمك، ولكي أُسمع كلامي كحيّ وأنا ميت.

\par 8 فالآن غدًا تكون أنت وبنوك معي، حين يُسلَّم الشعب إلى أيدي الفلسطينيين. ولأن أحشائك قد تحركت غيرةً، فسيُؤخذ منك ما هو لك. فسمع شاول كلام صموئيل، فذابت نفسه وقال: ها أنا ذاهب لأموت مع أبنائي، لعلَّ هلاكي يكون تكفيرًا عن آثامي. فقام شاول ومضى من هناك

\chapter{65}

\par 1 وحارب الفلسطينيون إسرائيل. فخرج شاول للقتال. فهرب إسرائيل أمام الفلسطينيين. ولما رأى شاول أن القتال قد اشتد جدًا، قال في قلبه: لماذا تتشدد للحياة وقد نادى صموئيل بالموت لك ولأبنائك؟

\par 2 فقال شاول للحامل سلاحه: خذ سيفك واقتلني قبل أن يأتي الفلسطينيون ويسبوني. فلم يشأ الحامل سلاحه أن يمد يده عليه

\par 3 وانحنى هو نفسه على سيفه، ولم يستطع أن يموت. والتفت خلفه فرأى رجلاً يركض فناداه وقال: خذ سيفي واقتلني. لأن حياتي بعد فيّ

\par 4 فجاء ليقتله. فقال له شاول: قبل أن تقتلني، أخبرني من أنت؟ فقال له: أنا إيداب ابن أجاج ملك العماليقيين. فقال شاول: هوذا كلام صموئيل قد جاء عليّ كما قال: كل من يولد لأجاج يكون لك عثرة

\par 5 بل اذهب وقل لداود: لقد قتلت عدوك. وقل له: هكذا قال شاول: لا تذكر بغضتي ولا إثمي...



\end{document}