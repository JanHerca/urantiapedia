\begin{document}

\title{تيطس}


\chapter{1}

\par 1 بُولُسُ، عَبْدُ اللهِ، وَرَسُولُ يَسُوعَ الْمَسِيحِ، لأَجْلِ إِيمَانِ مُخْتَارِي اللهِ وَمَعْرِفَةِ الْحَقِّ، الَّذِي هُوَ حَسَبُ التَّقْوَى،
\par 2 عَلَى رَجَاءِ الْحَيَاةِ الأَبَدِيَّةِ، الَّتِي وَعَدَ بِهَا اللهُ الْمُنَّزَهُ عَنِ الْكَذِبِ، قَبْلَ الأَزْمِنَةِ الأَزَلِيَّةِ،
\par 3 وَإِنَّمَا أَظْهَرَ كَلِمَتَهُ فِي أَوْقَاتِهَا الْخَاصَّةِ، بِالْكِرَازَةِ الَّتِي اؤْتُمِنْتُ أَنَا عَلَيْهَا، بِحَسَبِ أَمْرِ مُخَلِّصِنَا اللهِ،
\par 4 إِلَى تِيطُسَ، الاِبْنِ الصَّرِيحِ حَسَبَ الإِيمَانِ الْمُشْتَرَكِ. نِعْمَةٌ وَرَحْمَةٌ وَسَلاَمٌ مِنَ اللهِ الآبِ وَالرَّبِّ يَسُوعَ الْمَسِيحِ مُخَلِّصِنَا.
\par 5 مِنْ أَجْلِ هَذَا تَرَكْتُكَ فِي كِرِيتَ لِكَيْ تُكَمِّلَ تَرْتِيبَ الأُمُورِ النَّاقِصَةِ، وَتُقِيمَ فِي كُلِّ مَدِينَةٍ شُيُوخاً كَمَا أَوْصَيْتُكَ.
\par 6 إِنْ كَانَ أَحَدٌ بِلاَ لَوْمٍ، بَعْلَ امْرَأَةٍ وَاحِدَةٍ، لَهُ أَوْلاَدٌ مُؤْمِنُونَ لَيْسُوا فِي شِكَايَةِ الْخَلاَعَةِ وَلاَ مُتَمَرِّدِينَ -
\par 7 لأَنَّهُ يَجِبُ أَنْ يَكُونَ الأُسْقُفُ بِلاَ لَوْمٍ كَوَكِيلِ اللهِ، غَيْرَ مُعْجِبٍ بِنَفْسِهِ، وَلاَ غَضُوبٍ، وَلاَ مُدْمِنِ الْخَمْرِ، وَلاَ ضَرَّابٍ، وَلاَ طَامِعٍ فِي الرِّبْحِ الْقَبِيحِ،
\par 8 بَلْ مُضِيفاً لِلْغُرَبَاءِ، مُحِبّاً لِلْخَيْرِ، مُتَعَقِّلاً، بَارّاً، وَرِعاً، ضَابِطاً لِنَفْسِهِ،
\par 9 مُلاَزِماً لِلْكَلِمَةِ الصَّادِقَةِ الَّتِي بِحَسَبِ التَّعْلِيمِ، لِكَيْ يَكُونَ قَادِراً أَنْ يَعِظَ بِالتَّعْلِيمِ الصَّحِيحِ وَيُوَبِّخَ الْمُنَاقِضِينَ.
\par 10 فَإِنَّهُ يُوجَدُ كَثِيرُونَ مُتَمَرِّدِينَ يَتَكَلَّمُونَ بِالْبَاطِلِ، وَيَخْدَعُونَ الْعُقُولَ، وَلاَ سِيَّمَا الَّذِينَ مِنَ الْخِتَانِ -
\par 11 الَّذِينَ يَجِبُ سَدُّ أَفْوَاهِهِمْ، فَإِنَّهُمْ يَقْلِبُونَ بُيُوتاً بِجُمْلَتِهَا، مُعَلِّمِينَ مَا لاَ يَجِبُ، مِنْ أَجْلِ الرِّبْحِ الْقَبِيحِ.
\par 12 قَالَ وَاحِدٌ مِنْهُمْ - وَهُوَ نَبِيٌّ لَهُمْ خَاصٌّ: «الْكِرِيتِيُّونَ دَائِماً كَذَّابُونَ. وُحُوشٌ رَدِيَّةٌ. بُطُونٌ بَطَّالَةٌ».
\par 13 هَذِهِ الشَّهَادَةُ صَادِقَةٌ. فَلِهَذَا السَّبَبِ وَبِّخْهُمْ بِصَرَامَةٍ لِكَيْ يَكُونُوا أَصِحَّاءَ فِي الإِيمَانِ،
\par 14 لاَ يُصْغُونَ إِلَى خُرَافَاتٍ يَهُودِيَّةٍ وَوَصَايَا أُنَاسٍ مُرْتَدِّينَ عَنِ الْحَقِّ.
\par 15 كُلُّ شَيْءٍ طَاهِرٌ لِلطَّاهِرِينَ، وَأَمَّا لِلنَّجِسِينَ وَغَيْرِ الْمُؤْمِنِينَ فَلَيْسَ شَيْءٌ طَاهِراً، بَلْ قَدْ تَنَجَّسَ ذِهْنُهُمْ أَيْضاً وَضَمِيرُهُمْ.
\par 16 يَعْتَرِفُونَ بِأَنَّهُمْ يَعْرِفُونَ اللهَ، وَلَكِنَّهُمْ بِالأَعْمَالِ يُنْكِرُونَهُ، إِذْ هُمْ رَجِسُونَ غَيْرُ طَائِعِينَ، وَمِنْ جِهَةِ كُلِّ عَمَلٍ صَالِحٍ مَرْفُوضُونَ.

\chapter{2}

\par 1 وَأَمَّا أَنْتَ فَتَكَلَّمْ بِمَا يَلِيقُ بِالتَّعْلِيمِ الصَّحِيحِ:
\par 2 أَنْ يَكُونَ الأَشْيَاخُ صَاحِينَ، ذَوِي وَقَارٍ، مُتَعَقِّلِينَ، أَصِحَّاءَ فِي الإِيمَانِ وَالْمَحَبَّةِ وَالصَّبْرِ.
\par 3 كَذَلِكَ الْعَجَائِزُ فِي سِيرَةٍ تَلِيقُ بِالْقَدَاسَةِ، غَيْرَ ثَالِبَاتٍ، غَيْرَ مُسْتَعْبَدَاتٍ لِلْخَمْرِ الْكَثِيرِ، مُعَلِّمَاتٍ الصَّلاَحَ،
\par 4 لِكَيْ يَنْصَحْنَ الْحَدَثَاتِ أَنْ يَكُنَّ مُحِبَّاتٍ لِرِجَالِهِنَّ وَيُحْبِبْنَ أَوْلاَدَهُنَّ،
\par 5 مُتَعَقِّلاَتٍ، عَفِيفَاتٍ، مُلاَزِمَاتٍ بُيُوتَهُنَّ، صَالِحَاتٍ، خَاضِعَاتٍ لِرِجَالِهِنَّ، لِكَيْ لاَ يُجَدَّفَ عَلَى كَلِمَةِ اللهِ.
\par 6 كَذَلِكَ عِظِ الأَحْدَاثَ أَنْ يَكُونُوا مُتَعَقِّلِينَ،
\par 7 مُقَدِّماً نَفْسَكَ فِي كُلِّ شَيْءٍ قُدْوَةً لِلأَعْمَالِ الْحَسَنَةِ، وَمُقَدِّماً فِي التَّعْلِيمِ نَقَاوَةً، وَوَقَاراً، وَإِخْلاَصاً،
\par 8 وَكَلاَماً صَحِيحاً غَيْرَ مَلُومٍ، لِكَيْ يُخْزَى الْمُضَادُّ، إِذْ لَيْسَ لَهُ شَيْءٌ رَدِيءٌ يَقُولُهُ عَنْكُمْ.
\par 9 وَالْعَبِيدَ أَنْ يَخْضَعُوا لِسَادَتِهِمْ، وَيُرْضُوهُمْ فِي كُلِّ شَيْءٍ، غَيْرَ مُنَاقِضِينَ،
\par 10 غَيْرَ مُخْتَلِسِينَ، بَلْ مُقَدِّمِينَ كُلَّ أَمَانَةٍ صَالِحَةٍ، لِكَيْ يُزَيِّنُوا تَعْلِيمَ مُخَلِّصِنَا اللهِ فِي كُلِّ شَيْءٍ.
\par 11 لأَنَّهُ قَدْ ظَهَرَتْ نِعْمَةُ اللهِ الْمُخَلِّصَةُ لِجَمِيعِ النَّاسِ،
\par 12 مُعَلِّمَةً إِيَّانَا أَنْ نُنْكِرَ الْفُجُورَ وَالشَّهَوَاتِ الْعَالَمِيَّةَ، وَنَعِيشَ بِالتَّعَقُّلِ وَالْبِرِّ وَالتَّقْوَى فِي الْعَالَمِ الْحَاضِرِ،
\par 13 مُنْتَظِرِينَ الرَّجَاءَ الْمُبَارَكَ وَظُهُورَ مَجْدِ اللهِ الْعَظِيمِ وَمُخَلِّصِنَا يَسُوعَ الْمَسِيحِ،
\par 14 الَّذِي بَذَلَ نَفْسَهُ لأَجْلِنَا، لِكَيْ يَفْدِيَنَا مِنْ كُلِّ إِثْمٍ، وَيُطَهِّرَ لِنَفْسِهِ شَعْباً خَاصّاً غَيُوراً فِي أَعْمَالٍ حَسَنَةٍ.
\par 15 تَكَلَّمْ بِهَذِهِ وَعِظْ وَوَبِّخْ بِكُلِّ سُلْطَانٍ. لاَ يَسْتَهِنْ بِكَ أَحَدٌ.

\chapter{3}

\par 1 ذَكِّرْهُمْ أَنْ يَخْضَعُوا لِلرِّيَاسَاتِ وَالسَّلاَطِينِ وَيُطِيعُوا، وَيَكُونُوا مُسْتَعِدِّينَ لِكُلِّ عَمَلٍ صَالِحٍ،
\par 2 وَلاَ يَطْعَنُوا فِي أَحَدٍ، وَيَكُونُوا غَيْرَ مُخَاصِمِينَ، حُلَمَاءَ، مُظْهِرِينَ كُلَّ وَدَاعَةٍ لِجَمِيعِ النَّاسِ.
\par 3 لأَنَّنَا كُنَّا نَحْنُ أَيْضاً قَبْلاً أَغْبِيَاءَ، غَيْرَ طَائِعِينَ، ضَالِّينَ، مُسْتَعْبَدِينَ لِشَهَوَاتٍ وَلَذَّاتٍ مُخْتَلِفَةٍ، عَائِشِينَ فِي الْخُبْثِ وَالْحَسَدِ، مَمْقُوتِينَ، مُبْغِضِينَ بَعْضُنَا بَعْضاً.
\par 4 وَلَكِنْ حِينَ ظَهَرَ لُطْفُ مُخَلِّصِنَا اللهِ وَإِحْسَانُهُ -
\par 5 لاَ بِأَعْمَالٍ فِي بِرٍّ عَمِلْنَاهَا نَحْنُ، بَلْ بِمُقْتَضَى رَحْمَتِهِ -خَلَّصَنَا بِغَسْلِ الْمِيلاَدِ الثَّانِي وَتَجْدِيدِ الرُّوحِ الْقُدُسِ،
\par 6 الَّذِي سَكَبَهُ بِغِنًى عَلَيْنَا بِيَسُوعَ الْمَسِيحِ مُخَلِّصِنَا.
\par 7 حَتَّى إِذَا تَبَرَّرْنَا بِنِعْمَتِهِ نَصِيرُ وَرَثَةً حَسَبَ رَجَاءِ الْحَيَاةِ الأَبَدِيَّةِ.
\par 8 صَادِقَةٌ هِيَ الْكَلِمَةُ. وَأُرِيدُ أَنْ تُقَرِّرَ هَذِهِ الأُمُورَ، لِكَيْ يَهْتَمَّ الَّذِينَ آمَنُوا بِاللهِ أَنْ يُمَارِسُوا أَعْمَالاً حَسَنَةً. فَإِنَّ هَذِهِ الأُمُورَ هِيَ الْحَسَنَةُ وَالنَّافِعَةُ لِلنَّاسِ.
\par 9 وَأَمَّا الْمُبَاحَثَاتُ الْغَبِيَّةُ وَالأَنْسَابُ وَالْخُصُومَاتُ وَالْمُنَازَعَاتُ النَّامُوسِيَّةُ فَاجْتَنِبْهَا، لأَنَّهَا غَيْرُ نَافِعَةٍ، وَبَاطِلَةٌ.
\par 10 اَلرَّجُلُ الْمُبْتَدِعُ بَعْدَ الإِنْذَارِ مَرَّةً وَمَرَّتَيْنِ أَعْرِضْ عَنْهُ.
\par 11 عَالِماً أَنَّ مِثْلَ هَذَا قَدِ انْحَرَفَ، وَهُوَ يُخْطِئُ مَحْكُوماً عَلَيْهِ مِنْ نَفْسِهِ.
\par 12 حِينَمَا أُرْسِلُ إِلَيْكَ أَرْتِيمَاسَ أَوْ تِيخِيكُسَ بَادِرْ أَنْ تَأْتِيَ إِلَيَّ إِلَى نِيكُوبُولِيسَ، لأَنِّي عَزَمْتُ أَنْ أُشَتِّيَ هُنَاكَ.
\par 13 جَهِّزْ زِينَاسَ النَّامُوسِيَّ وَأَبُلُّوسَ بِاجْتِهَادٍ لِلسَّفَرِ حَتَّى لاَ يُعْوِزَهُمَا شَيْءٌ.
\par 14 وَلْيَتَعَلَّمْ مَنْ لَنَا أَيْضاً أَنْ يُمَارِسُوا أَعْمَالاً حَسَنَةً لِلْحَاجَاتِ الضَّرُورِيَّةِ، حَتَّى لاَ يَكُونُوا بِلاَ ثَمَرٍ.
\par 15 يُسَلِّمُ عَلَيْكَ الَّذِينَ مَعِي جَمِيعاً. سَلِّمْ عَلَى الَّذِينَ يُحِبُّونَنَا فِي الإِيمَانِ. النِّعْمَةُ مَعَ جَمِيعِكُمْ. آمِينَ.

\end{document}