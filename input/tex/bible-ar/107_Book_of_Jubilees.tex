\begin{document}

\title{كتاب اليوبيلات}

\chapter{1}

\par \textit{يتلقى موسى ألواح الشريعة وتعليمات عن التاريخ الماضي والمستقبل ليكتبها في كتاب، 1-4. ارتداد إسرائيل، 5-9. سبي إسرائيل ويهوذا، 10-13. عودة يهوذا وإعادة بناء الهيكل، 15-18. صلاة موسى من أجل إسرائيل، 19-21. وعد الله بالفداء والسكنى معهم، 22-5، 28. أمر موسى بكتابة تاريخ العالم المستقبلي (كتاب اليوبيلات؟)، 26. وملاك لكتابة الشريعة، 27. يأخذ هذا الملاك الألواح الزمنية السماوية ليمليها على موسى، 29.}

هذا هو تاريخ تقسيم أيام الناموس والشهادة، وأحداث السنين، وأسابيعها (السنة)، ويوبيلاتها في كل سني العالم، كما كلم الرب موسى على جبل سيناء حين صعد ليأخذ لوحي الناموس والوصية، حسب صوت الله حين قال له: اصعد إلى رأس الجبل.

\par 1 وحدث في السنة الأولى لخروج بني إسرائيل من مصر، في الشهر الثالث، في اليوم السادس عشر من الشهر، [2450 Anno Mundi]، أن الله كلم موسى قائلاً: «اصعد إليّ إلى الجبل، فأعطيك لوحي الحجر من الناموس والوصية التي كتبتها، لتعليمهم».
\par 2 فصعد موسى إلى جبل الله، وحل مجد الرب على جبل سيناء، وظللته سحابة ستة أيام
\par 3 ونادى موسى في اليوم السابع من وسط السحاب، وكان منظر مجد الرب كشعلة نار على رأس الجبل
\par 4 وكان موسى على الجبل أربعين نهارًا وأربعين ليلة، وعلمه الله التاريخ السابق واللاحق لتقسيم جميع أيام الناموس والشهادة
\par 5 وقال: «أمل قلبك إلى كل كلمة أكلمك بها على هذا الجبل، واكتبها في كتاب لكي ترى أجيالهم أني لم أتركهم من أجل كل الشر الذي عملوه بتعديهم العهد الذي أقيمه بيني وبينك لأجيالهم هذا اليوم على جبل سيناء».
\par 6 «وهكذا سيحدث عندما تأتي عليهم كل هذه الأمور، أنهم سيدركون أنني أكثر برًا منهم في جميع أحكامهم وفي جميع أفعالهم، وسيدركون أنني كنت معهم حقًا.»
\par 7 «واكتب لنفسك جميع هذه الكلمات التي أُخبرك بها اليوم، لأني أعلم تمردهم وصلابة أعناقهم، قبل أن أُدخلهم الأرض التي أقسمت لآبائهم إبراهيم وإسحاق ويعقوب قائلاً: لنسلك أعطي أرضًا تفيض لبنا وعسلاً.»
\par 8 فيأكلون ويشبعون، ويلجأون إلى آلهة غريبة لا تنقذهم من شيء من ضيقهم، ويُسمع هذا الشاهد عليهم. لأنهم سينسون جميع وصاياي، كل ما أوصيهم به، ويتبعون الأمم، ونجاستهم، وعارهم، ويعبدون آلهتهم، فيكون ذلك لهم معصيةً وشقاءً وشرًا وفخًا.
\par 9 «ويهلك كثيرون ويؤخذون أسرى، ويسقطون في أيدي العدو، لأنهم تركوا أحكامي ووصاياي، وأعياد عهدي، وسبوتي، وقدسي الذي قدسته لنفسي في وسطهم، ومسكني ومقدسي الذي قدسته لنفسي في وسط الأرض، لأضع اسمي عليه، وليسكن هناك».
\par 10 «وسيصنعون لأنفسهم مرتفعات وأودية وتماثيل منحوتة، ويسجدون لكل واحد تمثاله المنحوت، حتى يضلوا، ويذبحون أولادهم للشياطين، ولكل أعمال ضلال قلوبهم.»
\par 11 «وسأرسل إليهم شهودًا لأشهد عليهم، لكنهم لن يسمعوا، وسيقتلون الشهود أيضًا، ويضطهدون الذين يطلبون الناموس، وسينسخون ويغيرون كل شيء ليعملوا الشر أمام عينيّ.»
\par 12 «وأحجب وجهي عنهم، وأسلمهم إلى أيدي الأمم للسبي والنهب والأكل، وأنزعهم من وسط الأرض، وأشتتهم بين الأمم».
\par 13 «وينسون كل شريعتي وكل وصاياي وكل أحكامي، ويضلّون في رؤوس الشهور والسبوت والأعياد واليوبيلات والأحكام.»
\par 14 «وبعد هذا يرجعون إليّ من بين الأمم بكل قلوبهم وكل أنفسهم وكل قوتهم، وأجمعهم من بين كل الأمم، فيطلبونني لأُوجد فيهم، إذا طلبوني بكل قلوبهم وكل أنفسهم.»
\par 15 «وأُظهِر لهم سلامًا كثيرًا مع البر، وأُزيل عنهم غرس الاستقامة بكل قلبي وكل نفسي، فيكونون بركة لا لعنة، ويكونون رأسًا لا ذنبًا.»
\par 16 «وأبني مقدسي في وسطهم، وأسكن معهم، وأكون لهم إلهًا وهم يكونون لي شعبًا بالحق والبر.»
\par 17 «ولا أتركهم ولا أهملهم، لأني أنا الرب إلههم».
\par 18 فسقط موسى على وجهه وصلى وقال: أيها الرب إلهي لا تترك شعبك وميراثك فيضلوا في ضلال قلوبهم، ولا تسلمهم إلى أيدي أعدائهم الأمم، لئلا يتسلطوا عليهم ويجعلوهم يخطئون إليك.
\par 19 «يا رب، لترتفع رحمتك على شعبك، واخلق فيهم روحًا مستقيمًا، ولا يتسلط عليهم روح بليعار ليتهمهم أمامك، ويوقعهم في شرك من جميع سبل البر، فيبيدوا من أمام وجهك.»
\par 20 «بل هم شعبك وميراثك الذي خلصته بقوتك العظيمة من أيدي المصريين. اخلق فيهم قلبًا نقيًا وروحًا قديسًا، ولا يقعوا في شرك خطاياهم من الآن إلى الأبد.»
\par 21 فقال الرب لموسى: «أنا أعلم تمردهم وأفكارهم وتصلب رقابهم، ولن يطيعوا حتى يعترفوا بخطيئتهم وخطيئة آبائهم».
\par 22 «وبعد هذا يرجعون إليّ بكل استقامة ومن كل قلوبهم ومن كل أنفسهم، وأختن غلفة قلوبهم وغلفة قلب نسلهم، وأخلق فيهم روحًا قدسًا، وأطهرهم فلا يرتدّون عني من ذلك اليوم إلى الأبد.»
\par 23 «وتلتصق نفوسهم بي وبجميع وصاياي، ويعملون بوصاياي، وأكون لهم أبًا ويكونون لي أبناءً.»
\par 24 «وسيُدْعَونَ جَمِيعُهُمْ أَوْلاَدَ اللهِ الْحَيِّ، وَسَيَعْلَمُ كُلُّ مَلاَكٍ وَكُلُّ رُوحٍ، بَلْ يَعْلَمُونَ أَنَّ هَؤُلاَءِ أَوْلاَدِي، وَأَنِّي أَبَاهُمْ بِالْبِرِّ وَالْاستِقَامَةِ، وَأَنِّي أُحِبُّهُمْ.»
\par 25 «واكتب لنفسك جميع هذه الكلمات التي أُخبرك بها على هذا الجبل، الأولى والأخيرة، التي ستتم في جميع أقسام الأيام في الناموس وفي الشهادة وفي الأسابيع واليوبيلات إلى الأبد، حتى أنزل وأسكن معهم طوال الأبد.»
\par 26 وقال لملاك الحضور: «اكتب لموسى من بدء الخليقة إلى أن يُبنى هيكلي بينهم إلى الأبد».
\par 27 «وسيظهر الرب لأعين الجميع، وسيعلم الجميع أنني أنا إله إسرائيل وأبو جميع أبناء يعقوب، وملك على جبل صهيون إلى الأبد. وستكون صهيون وأورشليم مقدستين.»
\par 28 "وأخذ ملاك الحضور السائر أمام محلة إسرائيل ألواح أقسام السنين من وقت الخلق والناموس وشهادة أسابيع اليوبيلات حسب السنين حسب كل عدد اليوبيلات [حسب، السنين] من يوم الخلق [الجديد] حين تتجدد السماوات والأرض وكل خليقتها حسب قوات السماء وحسب كل خليقة الأرض إلى أن يصنع مقدس الرب في أورشليم على جبل صهيون وتتجدد كل النجوم للشفاء والسلام والبركة لجميع مختاري إسرائيل وهكذا يكون من ذلك اليوم وإلى كل أيام الأرض."

\chapter{2}

\par \textit{تاريخ أفعال الخلق الاثنين والعشرين المتميزة في الأيام الستة، 1-16. تأسيس السبت: مراعاته من قبل الملائكة الأعلى، الذين سيرتبط بهم إسرائيل فيما بعد، 17-32. (راجع تكوين 1-2: 3.)}

\par 1 وكلم ملاك الحضور موسى حسب كلام الرب قائلاً: اكتب تاريخ الخليقة كاملاً، كيف أنهى الرب الإله في ستة أيام جميع أعماله وكل ما خلقه، وحفظ السبت في اليوم السابع وقدسه إلى الأبد، وجعله علامة لجميع أعماله
\par 2 لأنه في اليوم الأول خلق السماوات التي فوق، والأرض والمياه، وكل الأرواح التي تخدم أمامه - ملائكة الحضور، وملائكة التقديس، وملائكة روح النار وملائكة روح الرياح، وملائكة روح السحب، والظلام، والثلج والبرد والصقيع، وملائكة الأصوات والرعد والبرق، وملائكة أرواح البرد والحر، والشتاء والربيع والخريف والصيف، وكل أرواح مخلوقاته التي في السماوات وعلى الأرض، (خلق) الهاوية والظلام، والمساء (والليل)، والنور، والفجر والنهار، الذي أعده بمعرفة قلبه
\par 3 وعندئذٍ رأينا أعماله، فسبحناه، وحمدناه أمامه على جميع أعماله؛ لأنه خلق سبعة أعمال عظيمة في اليوم الأول
\par 4 وفي اليوم الثاني خلق الجلد في وسط المياه، فانشقت المياه في ذلك اليوم - نصفها صعد إلى فوق ونصفها نزل إلى أسفل الجلد الذي كان في الوسط على وجه كل الأرض. وكان هذا هو العمل الوحيد الذي خلقه (الله) في اليوم الثاني
\par 5 وفي اليوم الثالث أمر الله المياه أن تزول عن وجه كل الأرض إلى مكان واحد، وأن تظهر اليابسة
\par 6 ففعلت المياه كما أمرها، وارتحلت عن وجه الأرض إلى مكان واحد خارج هذا الجلد، وظهرت اليابسة
\par 7 وفي ذلك اليوم خلق لهم كل البحار حسب مجمعاتها، وكل الأنهار، ومجتمعات المياه في الجبال وعلى كل الأرض، وكل البحيرات، وكل ندى الأرض، والبذر الذي يزرع، وكل ما ينبت، وشجر يحمل ثمرًا، وشجر الغاب، وجنة عدن في عدن، وكل عشب كجنسه.
\par 8 خلق الله هذه الأعمال الأربعة العظيمة في اليوم الثالث. وفي اليوم الرابع خلق الشمس والقمر والنجوم، وجعلها في جلد السماء، لتنير على كل الأرض، ولتتحكم في النهار والليل، وتفصل بين النور والظلمة
\par 9 وجعل الله الشمس آية عظيمة على الأرض للأيام والسبوت والأشهر والأعياد والسنين وسبوت السنين واليوبيلات وكل فصول السنين
\par 10 وهو يفصل بين النور والظلمة، ويؤدي إلى الرخاء، لكي تزدهر كل الأشياء التي تنبت وتنمو على الأرض
\par 11 هذه الأنواع الثلاثة خلقها في اليوم الرابع. وفي اليوم الخامس خلق وحوشًا بحرية عظيمة في أعماق المياه، لأن هذه كانت أول الكائنات الحية التي خلقها بيديه: السمك وكل ما يتحرك في المياه، وكل ما يطير، والطيور وكل أجناسها
\par 12 وأشرقت الشمس فوقهم لتزدهر، وفوق كل ما كان على الأرض، كل ما ينبت من الأرض، وكل شجرة مثمرة، وكل جسد
\par 13 خلق هذه الأنواع الثلاثة في اليوم الخامس. وفي اليوم السادس خلق جميع حيوانات الأرض، وجميع البهائم، وكل ما يدب على الأرض
\par 14 وبعد كل هذا خلق الإنسان، رجلاً وامرأة خلقهما، وأعطاه سلطانًا على كل ما على الأرض، وفي البحار، وعلى كل ما يطير، وعلى البهائم، وعلى البهائم، وعلى كل ما يدب على الأرض، وعلى كل الأرض، وعلى كل هذا أعطاه سلطانًا
\par 15 وخلق هذه الأنواع الأربعة في اليوم السادس. فكان مجموعها اثنين وعشرين نوعًا
\par 16 وانتهى من جميع عمله في اليوم السادس - كل ما في السماوات وعلى الأرض، وفي البحار وفي اللجج، وفي النور وفي الظلمة، وفي كل شيء
\par 17 وأعطانا آية عظيمة، يوم السبت، أن نعمل ستة أيام، ونحفظ السبت في اليوم السابع من كل عمل
\par 18 وجميع ملائكة الحضور، وجميع ملائكة التقديس، هاتان الفئتان العظيمتان - أمرنا أن نحفظ السبت معه في السماء وعلى الأرض
\par 19 وقال لنا: ها أنا أفرز لنفسي شعباً من بين جميع الشعوب، فيحفظون يوم السبت، وأقدسهم لنفسي شعبي، وأباركهم. كما قدست يوم السبت وقدسته لنفسي، كذلك أباركهم، فيكونون لي شعباً وأنا أكون لهم إلهاً.
\par 20 «ولقد اخترتُ نسل يعقوب من بين كل ما رأيتُ، وكتبتُه ابني البكر، وقدّسته لنفسي إلى الأبد، وأعلّمهم يوم السبت، لكي يحفظوا فيه السبت من كل عمل.»
\par 21 وهكذا خلق فيها آيةً لكي يُحفظوا السبت معنا في اليوم السابع، ليأكلوا ويشربوا، وليباركوا من خلق كل الأشياء كما بارك وقدس لنفسه شعبًا خاصًا فوق جميع الشعوب، وأن يُحفظوا السبت معنا
\par 22 وأصعد أوامره كرائحة طيبة مقبولة أمامه كل الأيام ...
\par 23 كان هناك اثنان وعشرون رأسًا للبشرية من آدم إلى يعقوب، وصُنع اثنان وعشرون نوعًا من العمل حتى اليوم السابع؛ هذا مبارك ومقدس؛ والأول أيضًا مبارك ومقدس؛ وهذا يخدم مع ذاك للتقديس والبركة
\par 24 ولهذا (يعقوب ونسله) أُعطي أن يكونوا دائمًا المباركين والقديسين في الشهادة الأولى والناموس، كما قدس وبارك يوم السبت في اليوم السابع
\par 25 خلق السماء والأرض وكل ما خلقه في ستة أيام، وقدس الله اليوم السابع لجميع أعماله، لذلك أمر من أجله أن كل من عمل فيه يموت، ومن ينجسه يموت موتًا
\par 26 لذلك تأمر بني إسرائيل أن يحفظوا هذا اليوم ليقدسوه ولا يعملوا فيه عملاً ولا ينجسوه، لأنه أقدس من سائر الأيام
\par 27 وكل من ينجسه يموت موتًا، وكل من عمل فيه عملًا يموت موتًا أبديًا، لكي يحتفل بنو إسرائيل بهذا اليوم مدى أجيالهم، ولا يُقتلعون من الأرض؛ لأنه يوم مقدس ويوم مبارك
\par 28 وكل من يحفظه ويحفظ السبت فيه من جميع أعماله، يكون مقدسًا ومباركًا طوال الأيام مثلنا
\par 29 أعلن وقل لبني إسرائيل شريعة هذا اليوم، أن يحفظوا فيه السبت، وأن لا يتركوه بضلال قلوبهم؛ وأنه لا يجوز عمل أي عمل غير لائق فيه، أو العمل فيه على هواهم، ولا أن يعدوا فيه شيئًا يؤكل أو يشرب، ولا أن يستقوا فيه ماءً، أو أن يدخلوا أو يخرجوا منه عبر أبوابهم أي حمل لم يعدوه لأنفسهم في اليوم السادس في مساكنهم
\par 30 ولا يدخلون ولا يخرجون من بيت إلى بيت في ذلك اليوم، لأن ذلك اليوم أقدس وأبرك من جميع أيام اليوبيل، لأنه في هذا اليوم سبتنا في السماء قبل أن يعلم كل ذي جسد أن يسبته على الأرض.
\par 31 وباركه خالق كل الأشياء، ولكنه لم يقدس جميع الشعوب والأمم لحفظ السبت فيه، بل قدس إسرائيل وحدهم: هم وحدهم سمح لهم أن يأكلوا ويشربوا ويحفظوا السبت فيه على الأرض
\par 32 وبارك خالق كل شيء هذا اليوم الذي خلقه للبركة والقداسة والمجد فوق كل الأيام
\par 33 أُعطيت هذه الشريعة وهذه الشهادة لبني إسرائيل شريعة أبدية لأجيالهم

\chapter{3}

\par \textit{آدم يسمي جميع المخلوقات، 1-3. خلق حواء وسنّ قوانين التطهير اللاوية، 4-14. آدم وحواء في الجنة: خطيئتهما وطردهما، 15-29. سنّ قانون ستر العار، 30-2. آدم وحواء يعيشان في إيليدا، 32-5. (راجع تكوين 2: 18-25، 3)}

\par 1 وفي الأيام الستة من الأسبوع الثاني، أحضرنا إلى آدم، بحسب كلمة الله، جميع الحيوانات، وجميع البهائم، وجميع الطيور، وكل ما يدب على الأرض، وكل ما يدب في الماء، بحسب أجناسها وأنواعها: البهائم في اليوم الأول، والبهائم في اليوم الثاني، والطيور في اليوم الثالث، وكل ما يدب على الأرض في اليوم الرابع، وما يدب في الماء في اليوم الخامس
\par 2 فسمى آدم كل واحد منهم باسمه، وكما دعاهم كانت أسماؤهم
\par 3 وفي تلك الأيام الخمسة، رأى آدم جميع هؤلاء، ذكرًا وأنثى، حسب كل جنس كان على الأرض، ولكنه كان وحيدًا ولم يجد له معينًا
\par 4 فقال لنا الرب: «ليس جيدًا أن يكون الإنسان وحده. فلنصنع له معينًا».
\par 5 فأوقع عليه الرب إلهنا سباتًا عميقًا، فنام، وأخذ للمرأة ضلعًا واحدة من بين أضلاعه، وكانت هذه الضلع أصل المرأة من بين أضلاعه، وبنى اللحم مكانها، وبنى المرأة
\par 6 وأيقظ آدم من نومه، فاستيقظ وقام في اليوم السادس، وأحضرها إليه، فعرفها، وقال لها: «هذه الآن عظم من عظامي ولحم من لحمي، تُدعى امرأتي لأنها أُخذت من رجلها».
\par 7 لذلك يكون الرجل والمرأة جسدًا واحدًا، ولذلك يترك الرجل أباه وأمه ويلتصق بامرأته ويكونان جسدًا واحدًا
\par 8 في الأسبوع الأول خُلِقَ آدم، والضلع امرأته. وفي الأسبوع الثاني أراها له. ولهذا أُعطيت الوصية أن يُحفظ في نجاسة الذكر سبعة أيام، والأنثى مرتين سبعة أيام
\par 9 وبعد أن أكمل آدم أربعين يوما في الأرض التي خلقها أدخلناه جنة عدن ليعملها ويحفظها. وأما امرأته فأدخلتها بعد ثمانين يوما.
\par 10 ولهذا السبب كُتبت الوصية على الألواح السماوية فيما يتعلق بالمولودة: «إذا ولدت ذكرًا، تبقى في نجاستها سبعة أيام حسب الأسبوع الأول من الأيام، وثلاثة وثلاثين يومًا تبقى في دم تطهيرها، ولا تمس شيئًا مقدسًا، ولا تدخل المقدس، حتى تُكمل هذه الأيام المفروضة على المولود الذكر».
\par 11 «وأما الأنثى فتبقى في نجاستها أسبوعين أيامًا كالأسبوعين الأولين، وستة وستين يومًا في دم تطهيرها، وتكون في كل ثمانين يومًا.»
\par 12 ولما أكملت هذه الثمانين يومًا أدخلناها إلى جنة عدن، لأنها أقدس من كل ما في الأرض، وكل شجرة مغروسة فيها مقدسة
\par 13 لذلك، فُرض على من تلد ذكرًا أو أنثى فريضة تلك الأيام أن لا تمس شيئًا مقدسًا، ولا تدخل المقدس حتى تكمل أيام ولادة الذكر أو الأنثى
\par 14 هذه هي الشريعة والشهادة التي كُتبت لإسرائيل لكي يحفظوها كل الأيام
\par 15 وفي الأسبوع الأول من اليوبيل الأول، [1-7 صباحًا] كان آدم وامرأته في جنة عدن سبع سنوات يفلحونها ويحرسونها، وأعطيناه عملاً وأمرناه أن يعمل كل ما يصلح للفلاحة
\par 16 وكان يعمل (في الحديقة)، وكان عريانًا، ولم يعلم، ولم يخجل، وكان يحفظ الحديقة من الطير والوحوش والماشية، وكان يجمع ثمرها، ويأكل، ويضع الباقي لنفسه ولزوجته [ويضع جانبًا ما كان محفوظًا].
\par 17 وبعد إتمام السنوات السبع التي أكملها هناك، سبع سنوات بالضبط، [8 صباحًا] وفي الشهر الثاني، في اليوم السابع عشر (من الشهر)، جاءت الحية واقتربت من المرأة، فقالت الحية للمرأة: «هل أوصاكِ الله قائلًا: لا تأكلا من كل شجر الجنة؟»
\par 18 فقالت له: «من جميع ثمر شجر الجنة قال لنا الله: كلوا، وأما ثمر الشجرة التي في وسط الجنة فقال لنا الله: لا تأكلا منه ولا تمساه لئلا تموتا».
\par 19 فقالت الحية للمرأة: لن تموتا، بل الله عالم أنه يوم تأكلان منه تنفتح أعينكما وتكونان كالله، وتعرفان الخير والشر
\par 20 فرأت المرأة الشجرة أنها بهجة وبهجة للنظر، وأن ثمرها جيد للأكل، فأخذت منه وأكلت
\par 21 فحين غطت عورتها أولا بأوراق التين، أعطت آدم منه فأكل، فانفتحت عيناه فرأى أنه عريان.
\par 22 فأخذ أوراق تين وخاطها، وصنع لنفسه مئزرًا، وستر عورته
\par 23 ولعن الله الحية وغضب عليها إلى الأبد...
\par 24 فغضب على المرأة لأنها سمعت لصوت الحية وأكلت. وقال لها: «أكثر حزنك وأوجاعك تكثيراً. بالحزن تلدين أولاداً، وإلى رجلك يكون رجوعك وهو يسود عليك».
\par 25 وقال لآدم أيضًا: «لأنك سمعت لقول امرأتك، وأكلت من الشجرة التي أوصيتك ألا تأكل منها، ملعونة الأرض بسببك. شوكًا وحسكًا تنبت لك، وتأكل خبزك بعرق وجهك، حتى تعود إلى الأرض التي أُخذت منها؛ لأنك تراب وإلى تراب تعود».
\par 26 فصنع لهم أقمصة من جلد وألبسهم وأخرجهم من جنة عدن
\par 27 وفي ذلك اليوم الذي خرج فيه آدم من الجنة، قدم رائحة طيبة، لبانًا، ولبانًا، وقصبًا، وأطيابًا في الصباح مع طلوع الشمس من اليوم الذي ستر فيه عورته
\par 28 وفي ذلك اليوم أُغلقت أفواه جميع البهائم، والبهائم، والطيور، وكل ما يمشي، وكل ما يتحرك، حتى لم تعد قادرة على التكلم، لأنها كانت جميعها تتكلم بعضها مع بعض بشفة واحدة ولسان واحد
\par 29 وأرسل من جنة عدن كل ذي جسد كان في جنة عدن، وتشتت كل ذي جسد حسب أجناسه وأنواعه إلى الأماكن التي خُلقت له
\par 30 ولآدم وحده أعطى (ما يكفي) لتغطية عورته، من بين جميع الوحوش والماشية
\par 31 لهذا السبب، كُتب على الألواح السماوية فيما يتعلق بكل من يعرف حكم الشريعة، أن يستر عورته، ولا يكشف نفسه كما يكشف الأمم أنفسهم
\par 32 وفي بداية الشهر الرابع، خرج آدم وزوجته من جنة عدن، وسكنا في أرض إيلدا، أرض خلقهما
\par 33 ودعا آدم اسم امرأته حواء.
\par 34 ولم يكن لهما ابن إلا في اليوبيل الأول [8 صباحًا] وبعد ذلك عرفها.
\par 35 فراح يزرع الأرض كما أُمر في جنة عدن

\chapter{4}

\par \textit{قابيل وهابيل وأبناء آدم الآخرون، 1-12. أنوش، قينان، مهللئيل، يارد، 13-15. أخنوخ وتاريخه، 16-25. أربعة أماكن مقدسة، 26. متوشالح، لامك، نوح، 27، 28. وفاة آدم وقايين، 29-32. سام، وحام، ويافث، 32. (راجع تكوين 4-5)}

\par 1 وفي الأسبوع الثالث في اليوبيل الثاني [64-70 ص] ولدت قابيل، وفي الأسبوع الرابع [71-77 ص] ولدت هابيل، وفي الأسبوع الخامس [78-84 ص] ولدت ابنتها عوان.
\par 2 وفي السنة الأولى من اليوبيل الثالث [99-105 صباحًا]، قتل قابيل هابيل لأن الله قبل ذبيحة هابيل، ولم يقبل قربان قابيل
\par 3 فقتله في الحقل، فصرخ دمه من الأرض إلى السماء، متذمرًا لأنه قتله
\par 4 ووبخ الرب قابيل بسبب هابيل لأنه قتله، وجعله هاربًا على الأرض بسبب دم أخيه، ولعنه على الأرض
\par 5 ولهذا مكتوب على الألواح السماوية: «ملعون من يضرب قريبه غدرًا، وليقل كل من رأى وسمع: فليكن كذلك، ومن رأى ولم يخبر فليكن ملعونًا كالآخر».
\par 6 ولهذا السبب نُعلن عندما نأتي أمام الرب إلهنا عن كل الخطايا التي ارتُكبت في السماء وعلى الأرض، وفي النور وفي الظلمة، وفي كل مكان
\par 7 وناح آدم وزوجته على هابيل أربعة أسابيع من السنين، [99-127 صباحًا]، وفي السنة الرابعة من الأسبوع الخامس [130 صباحًا] فرحا، وعرف آدم امرأته مرة أخرى، فولدت له ابنًا، ودعا اسمه شيثًا، لأنه قال: «قد أقام لنا الله نسلًا ثانيًا على الأرض عوضًا عن هابيل، لأن قابيل قتله».
\par 8 وفي الأسبوع السادس [134-40 صباحًا] ولد ابنته أزورا.
\par 9 "واتخذ قابيل عوان أخته امرأة له فولدت له حنوك في ختام اليوبيل الرابع. [190-196 ص] وفي السنة الأولى من الأسبوع الأول من اليوبيل الخامس، [197 ص] بُنيت بيوت على الأرض، وبنى قابيل مدينة ودعا اسمها باسم ابنه حنوك."
\par 10 وعرف آدم حواء امرأته فولدت تسعة أبناء أيضًا.
\par 11 وفي الأسبوع الخامس من اليوبيل الخامس [225-31 صباحًا] اتخذ شيث أزورا أخته زوجة له، وفي السنة الرابعة من الأسبوع السادس [235 صباحًا] ولدت له أنوش.
\par 12 فابتدأ يدعو باسم الرب على الأرض.
\par 13 وفي اليوبيل السابع في الأسبوع الثالث [309-15 ص] اتخذ أنوش نوعم أخته امرأة له، فولدت له ابنا في السنة الثالثة من الأسبوع الخامس، ودعا اسمه قينان.
\par 14 وفي ختام اليوبيل الثامن [325، 386-3992 صباحًا] اتخذ قينان موهلليت أخته زوجة له، فولدت له ابنًا في اليوبيل التاسع، في الأسبوع الأول من السنة الثالثة من هذا الأسبوع [395 صباحًا] ودعا اسمه مهللئيل
\par 15 "وفي الأسبوع الثاني من اليوبيل العاشر اتخذ مهللئيل زوجة له ​​دينا ابنة بركيئيل ابنة أخيه، فولدت له ابناً في الأسبوع الثالث في السنة السادسة، ودعا اسمه يارد، لأنه في أيامه نزل ملائكة الرب على الأرض، أولئك الذين يسمون المراقبين، لكي يعلّموا بني البشر، ولكي يعملوا الحق والاستقامة على الأرض."
\par 16 وفي اليوبيل الحادي عشر [512-18 صباحًا] اتخذ يارد لنفسه زوجة اسمها بركة، ابنة راسوجال، ابنة عم أبيه، في الأسبوع الرابع من هذا اليوبيل [522 صباحًا]، وولدت له ابنًا في الأسبوع الخامس، في السنة الرابعة من اليوبيل، ودعا اسمه حنوك
\par 17 وكان أول من وُلِد على الأرض من البشر الذين تعلموا الكتابة والمعرفة والحكمة، وكتبوا علامات السماء حسب ترتيب أشهرها في كتاب، حتى يعرف الناس فصول السنين حسب ترتيب أشهرها المنفصلة
\par 18 وكان أول من كتب شهادة، وشهد لبني البشر بين أجيال الأرض، وأحصى أسابيع اليوبيلات، وعرفهم أيام السنين، ورتب الأشهر، وأحصى سبوت السنين كما عرفناها له
\par 19 ورأى في رؤيا نومه ما كان وما سيكون، كما سيحدث لأبناء البشر عبر أجيالهم حتى يوم الدينونة. رأى وفهم كل شيء، وكتب شهادته، ووضع الشهادة على الأرض لجميع أبناء البشر ولأجيالهم
\par 20 وفي اليوبيل الثاني عشر، [582-588] في الأسبوع السابع منه، اتخذ لنفسه امرأة اسمها عدنة، ابنة دانيل، ابنة أخي أبيه، وفي السنة السادسة من هذا الأسبوع [587 صباحًا] ولدت له ابنًا فدعا اسمه متوشالح
\par 21 وكان أيضًا مع ملائكة الله هذه اليوبيلات الستة من السنين، فأروه كل ما على الأرض وما في السماء، وحكم الشمس، وكتب كل شيء
\par 22 وشهد للمراقبين، الذين أخطأوا مع بنات البشر؛ لأنهم بدأوا يتحدون، ليتدنسوا، مع بنات البشر، وشهد أخنوخ ضدهم جميعًا
\par 23 وأُخذ من بين بني البشر، وأدخلناه جنة عدن بجلال وإكرام، وها هو ذا يكتب دينونة العالم وحكمه، وكل شرور بني البشر
\par 24 وبسببه جلب (الله) مياه الطوفان على كل أرض عدن، لأنه وُضع هناك كعلامة، وليشهد على جميع بني البشر، ويروي جميع أعمال الأجيال إلى يوم الدينونة
\par 25 وأوقد على الجبل بخورا القدس عطرا مقبولا أمام الرب.
\par 26 لأن للرب أربعة أماكن على الأرض: جنة عدن، وجبل المشرق، وهذا الجبل الذي أنت عليه اليوم، جبل سيناء، وجبل صهيون (الذي) سيُقدَّس في الخليقة الجديدة لتقديس الأرض؛ ومن خلاله ستُقدَّس الأرض من كل إثمها ونجاستها على مدى أجيال العالم
\par 27 وفي اليوبيل الرابع عشر [652 صباحًا] اتخذ متوشالح لنفسه زوجة، عدنة ابنة عزرئيل، ابنة أخيه، في الأسبوع الثالث، في السنة الأولى من هذا الأسبوع [701-7 صباحًا] وولد ابنًا ودعا اسمه لامك
\par 28 وفي اليوبيل الخامس عشر، في الأسبوع الثالث، اتخذ لامك لنفسه امرأة اسمها بيتنوس ابنة بركئيل، ابنة عم أبيه، وفي ذلك الأسبوع ولدت له ابنًا، فسمى اسمه نوحًا، قائلًا: «هذا يُعزيني عن تعبي وعن كل عملي وعن الأرض التي لعنها الرب».
\par 29 وفي ختام اليوبيل التاسع عشر، في الأسبوع السابع في السنة السادسة منه [930 صباحًا]، مات آدم، ودفنه جميع أبنائه في أرض خلقته، وكان أول من دُفن في الأرض
\par 30 وكان ينقصه سبعون سنة من ألف سنة؛ لأن ألف سنة كيوم واحد في شهادة السماوات، ولذلك كُتب عن شجرة المعرفة: «يوم تأكل منها تموت». لهذا السبب لم يُكمل سنين هذا اليوم؛ لأنه مات خلاله
\par 31 في ختام هذا اليوبيل، قُتل قابيل بعده في نفس السنة؛ لأن بيته سقط عليه، فمات في وسط بيته، وقُتل بحجارته؛ لأنه قتل هابيل بحجر، وبحجر قُتل بحكم عادل
\par 32 لهذا السبب كُتب على الألواح السماوية: «بنفس الأداة التي يقتل بها الإنسان جاره يُقتل؛ وبنفس الطريقة التي جرحه بها يُعاملونه».
\par 33 وفي اليوبيل الخامس والعشرين [1205 صباحًا]، اتخذ نوح لنفسه زوجة، وكان اسمها إمزارا، ابنة راكئيل، ابنة أخي أبيه، في السنة الأولى في الأسبوع الخامس [1207 صباحًا]: وفي السنة الثالثة منها ولدت له سامًا، وفي السنة الخامسة منها [1209 صباحًا] ولدت له حامًا، وفي السنة الأولى في الأسبوع السادس [1212 صباحًا] ولدت له يافث

\chapter{5}

\par \textit{ملائكة الله يتزوجون بنات البشر، 1. فساد الخليقة كلها، 2-3. عقاب الملائكة الساقطين وأطفالهم، 4-9أ. إعلان الدينونة النهائية، 9ب-16. يوم الكفارة، 17-18. الطوفان المُتنبأ به، نوح يبني الفلك، الطوفان، 19-32. (راجع تكوين 6-8: 19.)}

\par 1 وحدث لما ابتدأ بنو البشر يكثرون على وجه الأرض وولد لهم بنات أن ملائكة الله رآهم في إحدى سنوات هذا اليوبيل أنهم كانوا حسني المنظر فاتخذوا لأنفسهم نساء من كل من اختاروا وولدوا لهم بنين فكانوا جبابرة.
\par 2 وازداد الإثم على الأرض، وأفسد كل ذي جسد طريقه، سواءً الناس أو البهائم أو الوحوش أو الطيور أو كل ما يمشي على الأرض - كلهم ​​أفسدوا طرقهم وأنظمتهم، وبدأوا يلتهمون بعضهم بعضًا، وازداد الإثم على الأرض، وكان كل تصور لأفكار جميع البشر شريرًا باستمرار
\par 3 ونظر الله إلى الأرض فإذا هي فاسدة، وكل ذي جسد قد أفسد نظامه، وكل من على الأرض عمل كل أنواع الشر أمام عينيه
\par 4 وقال إنه سيهلك الإنسان وكل ذي جسد على وجه الأرض التي خلقها
\par 5 لكن نوحًا وجد نعمة لدى الرب.
\par 6 "وعلى الملائكة الذين أرسلهم على الأرض غضب غضبا شديدا، وأمر أن نقتلعهم من كل سلطانهم، وأمرنا أن نربطهم في أعماق الأرض، وها هم مقيدين في وسطهم، ومنفصلين."
\par 7 وعلى أبنائهم خرج أمر من أمام وجهه أن يضربوا بالسيف وينزعوا من تحت السماء
\par 8 وقال: «لا يمكث روحي على الإنسان إلى الأبد، لأنهم أيضًا بشر، وتكون أيامهم مئة وعشرين سنة».
\par 9 فأرسل سيفه في وسطهم ليقتل كل واحد جاره، فبدأوا يقتلون بعضهم بعضًا حتى سقطوا جميعًا بالسيف وبادوا من على الأرض
\par 10 وكان آباؤهم شهودًا (على هلاكهم)، وبعد ذلك كانوا مقيدين في أعماق الأرض إلى الأبد، إلى يوم الدينونة العظيمة، حين يُنفَّذ الحكم على كل الذين أفسدوا طرقهم وأعمالهم أمام الرب
\par 11 فأباد الجميع من أماكنهم، ولم يبق منهم أحد إلا وحكم عليهم حسب كل شرورهم
\par 12 وخلق لجميع أعماله طبيعة جديدة وبارة، حتى لا يخطئوا في كل طبيعتهم إلى الأبد، بل يكون الجميع أبرارًا كل واحد في نوعه إلى الأبد
\par 13 ودينونة الجميع مكتوبة ومُقدَّرة على الألواح السماوية بالعدل - حتى دينونة كل من يحيد عن الطريق الذي قُدِّر له أن يسلكه؛ وإن لم يسلك فيه، فالدينونة مكتوبة على كل مخلوق وكل نوع
\par 14 وليس في السماء ولا على الأرض، ولا في النور ولا في الظلمة، ولا في الهاوية ولا في العمق، ولا في موضع الظلمة (إلا يُدان)؛ وكل أحكامهم محكومة ومكتوبة ومنقوشة
\par 15 فيحكم على الجميع، العظيم حسب عظمته، والصغير حسب صغره، وكل واحد حسب طريقه.
\par 16 وهو ليس ممن ينظر إلى شخص (أي شخص)، ولا ممن يقبل الهدايا، إذا قال إنه سيُنفِّذ الحكم على كل شخص: إذا أعطى أحد كل ما على الأرض، فلن ينظر إلى الهدايا أو شخص (أي شخص)، ولن يقبل أي شيء من يديه، لأنه قاضٍ عادل
\par 17 [وعلى بني إسرائيل كُتب وقُدِّر: إن تابوا إليه ببرٍّ، فإنه يغفر لهم جميع ذنوبهم ويتجاوز عن جميع خطاياهم
\par 18 مكتوب ومرسوم أنه سيُظهر الرحمة لكل من يتوب عن جميع ذنوبه مرة واحدة كل عام.]
\par 19 وأما كل الذين أفسدوا طرقهم وأفكارهم قبل الطوفان، فلم يُقبل أحدٌ إلا نوح وحده؛ لأنه قُبل من أجل أبنائه الذين أنقذهم (الله) من مياه الطوفان بسببه؛ لأن قلبه كان بارًا في جميع طرقه كما أُمر به، ولم يحد عن شيء مما كُتب له
\par 20 وقال الرب إنه سيهلك كل ما على الأرض، من الناس والبهائم،
\par 21 البهائم، وطيور السماء، وما يدب على الأرض. وأمر نوحًا أن يصنع له فلكًا لينقذ نفسه من مياه الطوفان
\par 22 فصنع نوح الفلك من جميع الوجوه كما أمره، في اليوبيل السابع والعشرين من السنين، في الأسبوع الخامس من السنة الخامسة (في الهلال من الشهر الأول). [1307 صباحًا]
\par 23 ودخل في السنة السادسة منه، في الشهر الثاني، في هلال الشهر الثاني، إلى اليوم السادس عشر. ودخل هو وكل ما أحضرناه إليه إلى الفلك، وأغلقه الرب من الخارج في مساء اليوم السابع عشر
\par 24 ففتح الرب سبعة أبواب طاقية للسماء،  
\par     وأفواه ينابيع الغمر العظيم، سبعة أفواه في العدد
\par 25 وبدأت بوابات الطوفان تصب الماء من السماء أربعين يومًا وأربعين ليلة،  
وأرسلت ينابيع الغمر أيضًا مياهًا حتى امتلأت الأرض كلها بالماء.
\par 26 وتزايدت المياه على الأرض.  
\par     ارتفعت المياه خمسة عشر ذراعًا فوق جميع الجبال العالية،  
\par    وارتفع الفلك عن الأرض،  
\par    وكان يتحرك على وجه المياه.
\par 27 وظلت المياه على وجه الأرض خمسة أشهر - مئة وخمسين يومًا
\par 28 وانطلق الفلك واستقر على قمة لوبار، أحد جبال أراراط
\par 29 وفي الشهر الرابع انغلقت ينابيع الغمر العظيم وتوقفت طاقات السماء وفي الشهر السابع انفتحت كل أفواه أعماق الأرض وابتدأت المياه تنزل إلى الغمر الأسفل.
\par 30 وفي هلال الشهر العاشر ظهرت قمم الجبال، وفي هلال الشهر الأول ظهرت الأرض
\par 31 واختفت المياه من فوق الأرض في الأسبوع الخامس من السنة السابعة منها، وفي اليوم السابع عشر من الشهر الثاني جفت الأرض
\par 32 وفي اليوم السابع والعشرين منه فتح الفلك، وأخرج منه البهائم والبهائم والطيور وكل دابة

\chapter{6}

\par \textit{ذبيحة نوح، 1-3 (راجع تكوين 7: 20-2). عهد الله مع نوح، وتحريم أكل الدم، 4-10 (راجع تكوين 9: 1-17). أمر موسى بتجديد هذه الشريعة ضد أكل الدم، 11-14. وضع القوس في السحاب كعلامة، 15-16. تأسيس عيد الأسابيع، وتاريخ احتفالاته، 17-22. عيد الأقمار، 23-8. تقسيم السنة إلى 364 يومًا، 29-38.}

\par 1 وفي مطلع الشهر الثالث خرج من الفلك وبنى مذبحًا على ذلك الجبل
\par 2 وكفّر عن الأرض، فأخذ جديًا وكفّر بدمه عن كل ذنوب الأرض، إذ هلك كل ما كان عليها، إلا من كان في الفلك مع نوح
\par 3 ووضع شحمها على المذبح، وأخذ ثورًا ومعزًا وغنمًا وجديًا وملحًا ويمامة وفرخ حمامة، ووضع محرقة على المذبح، وسكب عليها تقدمة ممزوجة بزيت، ورشّ خمرًا ونثر لبانًا على كل شيء، فأضاء رائحة طيبة مقبولة أمام الرب
\par 4 فتنسم الرب الرائحة الطيبة، وقطع معه عهدًا ألا يكون هناك طوفان يُخرب الأرض؛ وأن كل أيام الأرض - البذر والحصاد - لا تنتهي أبدًا؛ وأن البرد والحر، والصيف والشتاء، والنهار والليل - لا تتغير ترتيبها ولا تتوقف إلى الأبد
\par 5 «وأنتم، فاكثروا وتكاثروا على الأرض، وكونوا كثيرين عليها، وكونوا بركة عليها. سأُلهم خوفكم ورهبتكم في كل ما على الأرض وفي البحر.»
\par 6 «وها أنا قد أعطيتكم كل البهائم، وكل الأجنحة، وكل ما يدب على الأرض، والأسماك التي في المياه، وكل ما هو للطعام. كما أعطيتكم الأعشاب الخضراء كل شيء لتأكلوه.»
\par 7 «لكن لحمًا بنفسه مع دمه، لا تأكلوه. لأن نفس كل جسد هي في الدم، لئلا يُطلب دم أنفسكم. من يد كل إنسان، من يد كل (حيوان) أطلب دم الإنسان.»
\par 8 «كل من سفك دم الإنسان بالإنسان يُسفك دمه، لأنه على صورة الله عمل الإنسان.»
\par 9 «وأنتم، انمُوا واكثروا في الأرض.»
\par 10 وحلف نوح وبنوه أنهم لا يأكلون دماً كان في جسد ما، وقطع عهداً أمام الرب الإله أبدياً في جميع أجيال الأرض في هذا الشهر.
\par 11 لأجل هذا قال لك أن تقطع عهدًا مع بني إسرائيل في هذا الشهر على الجبل بقسم، وأن ترش عليهم الدم بسبب جميع كلمات العهد الذي قطعه الرب معهم إلى الأبد
\par 12 وكُتبت هذه الشهادة عنكم لكي تعملوا بها دائمًا، حتى لا تأكلوا في يوم ما شيئًا من دم بهائم أو طيور أو بهائم طوال أيام الأرض. وكل إنسان يأكل دم بهائم أو بهائم أو طيور طوال أيام الأرض، يُقتلع هو ونسله من الأرض
\par 13 وأنتَ تُوصِي بني إسرائيلَ ألا يأكلوا دمًا، لكي تكون أسماؤهم ونسلهم أمام الرب إلهنا دائمًا
\par 14 ولا حدود لأيام هذه الشريعة، لأنها إلى الأبد. سيحفظونها مدى أجيالهم، لكي يستمروا في التضرع من أجلكم بالدم أمام المذبح. كل يوم وفي وقت الصباح والمساء، سيطلبون الغفران من أجلكم دائمًا أمام الرب لكي يحفظوها ولا يُقتلعوا
\par 15 وأعطى نوحًا وبنيه علامةً ألا يكون هناك طوفانٌ على الأرض أيضًا
\par 16 وضع قوسه في السحاب علامةً على العهد الأبدي بأنه لن يكون هناك طوفانٌ آخر على الأرض ليهلكها كل أيام الأرض
\par 17 لهذا السبب، قُدِّرَ وكُتِبَ على الألواح السماوية، أن يحتفلوا بعيد الأسابيع في هذا الشهر مرة واحدة في السنة، لتجديد العهد كل عام
\par 18 وكان يُحتفل بهذا العيد بأكمله في السماء من يوم الخلق إلى أيام نوح - ستة وعشرون يوبيلًا وخمسة أسابيع من السنين [1309-1659 صباحًا]: واحتفل به نوح وأبناؤه لمدة سبعة يوبيلات وأسبوع واحد من السنين، حتى يوم وفاة نوح، ومن يوم وفاة نوح ألغى أبناؤه (ذلك) حتى أيام إبراهيم، وهم يأكلون الدم
\par 19 لكن إبراهيم حفظه، وإسحاق ويعقوب وبنوه حفظوه إلى أيامك، وفي أيامك نسيه بنو إسرائيل حتى أعددتموه على هذا الجبل
\par 20 وأنتَ تُوصِي بني إسرائيلَ أن يُحيوا هذا العيدَ في جميعِ أجيالِهم وصيةً لهم: يومًا واحدًا في السنةِ في هذا الشهرِ يحتفلونَ بالعيد
\par 21 لأنه عيد الأسابيع وعيد الباكورة: هذا العيد ذو وجهين وطبيعة مزدوجة: بحسب ما هو مكتوب ومنقوش عنه، احتفلوا به
\par 22 لأني كتبت في سفر التوراة الأول الذي كتبته لك أنك تعيده في وقته يوما واحدا في السنة، وشرحت لك ذبائحه لكي يذكرها بنو إسرائيل ويعيدوها في أجيالهم في هذا الشهر يوما واحدا في كل سنة.
\par 23 وفي هلال الشهر الأول، وفي هلال الشهر الرابع، وفي هلال الشهر السابع، وفي هلال الشهر العاشر، تكون أيام الذكرى، وأيام الفصول في أقسام السنة الأربعة. تُكتب هذه وتُرسم شهادةً إلى الأبد
\par 24 وجعلها نوح لنفسه أعيادًا للأجيال إلى الأبد، حتى صارت له بذلك تذكارًا
\par 25 وفي بداية الشهر الأول أُمر أن يصنع لنفسه فلكًا، وفي ذلك اليوم جفت الأرض ففتح (الفلك) ورأى الأرض
\par 26 وفي هلال الشهر الرابع، أُغلقت أفواه أعماق الهاوية السفلى. وفي هلال الشهر السابع، انفتحت جميع أفواه هاويات الأرض، وبدأت المياه تنزل إليها
\par 27 وفي هلال الشهر العاشر، ظهرت رؤوس الجبال، ففرح نوح
\par 28 ولهذا جعلها لنفسه أعيادًا لذكرى أبدية، وهكذا هي مُقامة
\par 29 ووضعوها على الألواح السماوية، وكان لكل منها ثلاثة عشر أسبوعًا؛ من واحد إلى آخر (مر) تذكاره، من الأول إلى الثاني، ومن الثاني إلى الثالث، ومن الثالث إلى الرابع
\par 30 وستكون جميع أيام الوصية اثنين وخمسين أسبوعًا من الأيام، وستُكمل السنة بأكملها. وهكذا تُنقش وتُرسم على الألواح السماوية
\par 31 ولا إفراط في هذه الوصية سنة ولا سنة فسنة
\par 32 وأمر بني إسرائيل أن يحفظوا السنين على هذا الحساب - ثلاثمائة وأربعة وستين يومًا، وتكون سنة كاملة، ولا يخلون بوقتها من أيامها ومن أعيادها؛ لأن كل شيء يقع فيها حسب شهادتهم، ولا يتركون يومًا ولا يخلون بأي عيد
\par 33 ولكن إن أهملوا ولم يلتزموا بها حسب أمره، فإنهم سيُخِلّون بجميع فصول السنة، وستُخْرَج السنين من هذا (النظام)، [وسيُخْرِبون الفصول، وستُخْرَج السنين]، وسيُهملون أحكامهم
\par 34 وينسى جميع بني إسرائيل ولن يجدوا مسار السنين، وينسون رؤوس الشهور والأوقات والسبت، ويخطئون في كل ترتيب السنين
\par 35 لأني أعلمه ومن الآن أخبرك به، وهو ليس من تأليفي، لأن الكتاب مكتوب أمامي، وعلى الألواح السماوية قسمة الأيام، لئلا ينسوا أعياد العهد ويسلكوا حسب أعياد الأمم حسب ضلالهم وجهلهم.
\par 36 لأنه سيكون هناك من سيقومون بالتأكيد بمراقبة القمر - كيف يزعج الفصول ويأتي من سنة إلى أخرى مبكرًا جدًا بعشرة أيام
\par 37 لهذا السبب ستأتي عليهم سنون يُخِلّون فيها بالنظام، ويجعلون يومًا رجسًا يوم شهادة، ويومًا نجسًا عيدًا، ويخلطون جميع الأيام، المقدس بالنجس، واليوم النجس بالمقدس؛ لأنهم سيخطئون في تحديد الأشهر والسبوت والأعياد واليوبيلات
\par 38 لهذا السبب، أوصيك وأشهد لك حتى تشهد عليهم؛ لأنه بعد وفاتك، سيُزعجهم أبناؤك، حتى لا يجعلوا السنة ثلاثمائة وأربعة وستين يومًا فقط، ولهذا السبب سيخطئون في تحديد رؤوس الشهور والأوقات والسبت والأعياد، وسيأكلون كل أنواع الدم مع كل أنواع اللحم

\chapter{7}

\par \textit{نوح يغرس كرمًا ويقدم ذبيحة، 1-5. يسكر ويكشف عن نفسه، 6-9. لعنة كنعان ومباركة سام ويافث، 10-12 (قارن تكوين 9: 20-8). أبناء نوح وأحفاده ومدنهم، 13-19. يعلم نوح أبناءه عن أسباب الطوفان ويحذرهم من أكل الدم والقتل، وحفظ الشريعة المتعلقة بأشجار الفاكهة وترك الأرض بورًا كل سبع سنوات، كما أوصى أخنوخ، 20-39.}

\par 1 وفي الأسبوع السابع من السنة الأولى [1317 صباحًا] منه، في هذا اليوبيل، غرس نوح كرومًا على الجبل الذي استقرت عليه السفينة، واسمه لوبار، أحد جبال أراراط، فأثمرت في السنة الرابعة [1320 صباحًا]، فحفظ ثمرها وجمعه في تلك السنة في الشهر السابع
\par 2 وصنع منها خمرًا وجعلها في إناء، وحفظها إلى السنة الخامسة، [1321 صباحًا] إلى اليوم الأول، في الهلال من الشهر الأول
\par 3 واحتفل بفرح في يوم هذا العيد، وقدم محرقة للرب: ثورًا واحدًا ابن بقر وكبشًا واحدًا، وسبعة خراف حولية، وجدي من المعز، لكي يكفر بها عن نفسه وعن بنيه
\par 4 ثم هيأ المعزى أولاً، ووضع من دمه على اللحم الذي على المذبح الذي صنعه، وكل الشحم وضعه على المذبح الذي صنع فيه المحرقة، والثور والكبش والغنم، ووضع كل لحمها على المذبح
\par 5 ووضع عليه جميع قرابينهم ممزوجة بزيت، وبعد ذلك رش الخمر على النار التي أشعلها أولا على المذبح، ووضع البخور على المذبح فأصعد رائحة سرور مقبولة أمام الرب إلهه.
\par 6 ففرح وشرب من هذا الخمر هو وبنوه بفرح.
\par 7 وكان المساء فدخل خيمته وسكر فاضطجع ونام، وتكشف في خيمته وهو نائم.
\par 8 فرأى حام نوحًا أباه عريانًا، فخرج وأخبر أخويه خارجًا
\par 9 فأخذ سام ثوبه وقام هو ويافث، ووضعا الثوب على أكتافهما ومشيا إلى الوراء وسترا عورة أبيهما، وكان وجهاهما إلى الوراء
\par 10 فاستيقظ نوح من نومه وعلم كل ما فعل به ابنه الأصغر، فلعن ابنه وقال: ملعون كنعان، عبد يكون لإخوته
\par 11 وبارك سامًا وقال: «مبارك الرب إله سام، ويكون كنعان عبدًا له».
\par 12 «سيُوسِّع الله يافث، ويسكن الله في مسكن سام، ويكون كنعان خادمًا له.»
\par 13 فعلم حام أن أباه قد لعن ابنه الأصغر، فاغتاظ لأنه لعن ابنه وانفصل عن أبيه هو وبنوه معه، كوش ومصرايم وفوط وكنعان
\par 14 وبنى لنفسه مدينةً ودعا اسمها باسم زوجته نئلاتاماوك
\par 15 فرأى يافث ذلك، فحسد أخاه، وبنى هو أيضًا لنفسه مدينة، ودعا اسمها باسم زوجته أداتانيسيس
\par 16 وأقام سام مع أبيه نوح، وبنى مدينة بالقرب من أبيه في الجبل، ودعا اسمها أيضًا باسم زوجته صدقت لباب
\par 17 وها هي هذه المدن الثلاث بالقرب من جبل لوبار؛ صدقتلباب تواجه الجبل من شرقه؛ ونائلتماؤك من الجنوب؛ وأداتانيسيس من الغرب
\par 18 وهؤلاء بنو سام: عيلام، وأشور، وأرفكشاد - هذا (الابن) وُلد بعد الطوفان بسنتين - ولود، وأرام
\par 19 بنو يافث: جومر وماجوج وماداي ويوان وتوبال وماشك وتيراس. هؤلاء هم بنو نوح
\par 20 وفي اليوبيل الثامن والعشرين [1324-1372 صباحًا] بدأ نوح يفرض على أبناء أبنائه الفرائض والوصايا، وجميع الأحكام التي عرفها، وحثّ أبناءه على مراعاة البر، وستر عار أجسادهم، ومباركة خالقهم، وإكرام الأب والأم، وحب قريبهم، وحفظ نفوسهم من الزنا والنجاسة وكل إثم
\par 21 "فإن بسبب هذه الأشياء الثلاثة جاء الطوفان على الأرض، أي بسبب الزنا الذي ارتكبه المراقبون ضد ناموس أحكامهم، فزنوا وراء بنات الناس، واتخذوا لأنفسهم نساء من كل ما اختاروه، فجعلوا ابتداء النجاسة."
\par 22 وأنجبوا أبناءً من النفيديم، وكانوا جميعًا مختلفين، وكانوا يلتهمون بعضهم بعضًا: فقتل العمالقة النفيديم، وقتل النفيديم الإيلو، وبشر الإيلو، ورجلًا آخر
\par 23 وباع كل واحد نفسه ليعمل الإثم ولسفك دماء كثيرة، فامتلأت الأرض إثمًا
\par 24 وبعد ذلك أخطأوا ضد الوحوش والطيور، وكل ما يتحرك ويمشي على الأرض، وسُفكت دماء كثيرة على الأرض، وتخيل كل خيال ورغبة من البشر الباطل والشر باستمرار
\par 25 فأباد الرب كل شيء عن وجه الأرض، بسبب شرور أعمالهم، وبسبب الدماء التي سفكوها في وسط الأرض، أهلك كل شيء
\par 26 وبقينا أنا وأنتم يا أبنائي وكل من دخل معنا إلى الفلك، وها أنا أرى أعمالكم أمامي أنكم لا تسلكون في البر، لأنكم في طريق الهلاك بدأتم تسيرون، وأنتم منفصلون عن بعضكم البعض، وتحسدون بعضكم بعضًا، وهكذا لا تتفقون يا أبنائي، كل واحد مع أخيه
\par 27 لأني أرى، وأرى، أن الشياطين قد بدأت إغواءاتها ضدك وضد أولادك، والآن أخشى عليك، أنه بعد موتي ستسفك دماء البشر على الأرض، وأنك أنت أيضًا ستُباد من على وجه الأرض
\par 28 لأن كل من سفك دم الإنسان، وكل من أكل دم أي جسد، سيباد جميعهم من على الأرض
\par 29 ولن يبقى إنسان يأكل دمًا، أو يسفك دم الإنسان على الأرض، ولا يبقى له نسل ولا ذرية ساكنة تحت السماء. لأنهم إلى الهاوية يذهبون، وإلى موضع الدينونة ينزلون، وإلى ظلمة الغمر يُنقلون جميعًا بموت عنيف
\par 30 لا يُرى عليكم دم من كل دم، كل الأيام التي تذبحون فيها شيئًا من البهائم أو الماشية أو كل ما يطير على الأرض، وتعملون عملاً صالحًا لنفوسكم بتغطية ما سُفك على وجه الأرض
\par 31 ولا تكونوا كمن يأكل الدم، بل احذروا لئلا يأكل أحد الدم أمامكم. غطوا الدم، لأني هكذا أُمرت أن أشهد لكم ولأولادكم مع كل ذي جسد
\par 32 ولا تدع النفس تؤكل مع الجسد، حتى لا يُطلب دمك، الذي هو حياتك، من يد أي جسد يسفكه على الأرض
\par 33 لأن الأرض لن تطهر من الدم الذي سفك عليها. لأنه بدم سافكه فقط تتطهر الأرض في كل أجيالها.
\par 34 والآن يا أبنائي، اسمعوا: اعملوا الحق والبر لكي تغرسوا بالبر على وجه كل الأرض، ويرتفع مجدكم أمام إلهي الذي خلصني من مياه الطوفان
\par 35 وها أنتم تذهبون وتبنون لأنفسكم مدنًا، وتغرسون فيها كل البقل الذي على الأرض، بل وكل شجرة مثمرة أيضًا
\par 36 لمدة ثلاث سنوات، لا يُجمع ثمر كل ما يؤكل. وفي السنة الرابعة، يُقدّس ثمرها، ويُقدّم باكورة، مقبولة أمام الله العلي، خالق السماء والأرض وكل شيء. فليُقدّموا بكثرة باكورة الخمر والزيت على مذبح الرب الذي يأخذه، وما يتبقى يأكله خدام بيت الرب أمام المذبح الذي يأخذه
\par 37 وفي السنة الخامسة، تُطلقون سراحه، فتُطلقونه في برٍّ واستقامة، وتكونون بارين، وكل ما تزرعونه يُفلح
\par 38 لأنه هكذا أوصى أخنوخ، أبو أبيك، متوشالح ابنه، ومتوشالح ابنه لامك، وأوصاني لامك بكل ما أوصاه به آباؤه
\par 39 وأنا أيضًا أوصيكم يا أبنائي كما أوصى أخنوخ ابنه في اليوبيلات الأولى: وهو لا يزال حيًا، السابع في جيله، أوصى وشهد لابنه ولأبناء ابنه إلى يوم وفاته

\chapter{8}

\par \textit{يكتشف كاينم نقشًا يتعلق بالشمس والنجوم، 1-4. أبناؤه، 5-8. أبناء نوح ونوح يقسمون الأرض، 10-11. ميراث سام، 12-21: ميراث حام، 22-4: ميراث يافث، 25-30. (راجع تكوين 10)}

\par 1 في اليوبيل التاسع والعشرين، في الأسبوع الأول، [1373 صباحًا] في بدايته، اتخذ أرفكشاد لنفسه زوجة اسمها رسوعجة، ابنة سوسن، ابنة عيلام، وولدت له ابنًا في السنة الثالثة من هذا الأسبوع، [1375 صباحًا] ودعا اسمه قاينام
\par 2 وكبر الابن، وعلمه أبوه الكتابة، وذهب يبحث لنفسه عن مكان يملك فيه لنفسه مدينة
\par 3 فوجد كتابةً نقشها الأولون على الصخر، فقرأ ما عليها، فنسخها وأثم بسببها؛ لأنها احتوت على تعليم الناظرين الذي كانوا يراقبون به طوائف الشمس والقمر والنجوم في جميع علامات السماء
\par 4 فكتبه ولم يقل فيه شيئًا، لأنه خاف أن يُكلِّم نوحًا عنه لئلا يغضب عليه بسببه
\par 5 "وفي اليوبيل الثلاثين، في الأسبوع الثاني، في السنة الأولى منه، اتخذ لنفسه امرأة، وكان اسمها ملكة بنت ماداي بن يافث، وفي السنة الرابعة، ولد ابنا، ودعا اسمه شيلة، لأنه قال: «حقا قد أرسلت»."
\par 6 [وفي السنة الرابعة وُلِد]، وكبر شيلة واتخذ لنفسه زوجة اسمها موك، ابنة قيس، أخي أبيه، في اليوبيل الحادي والثلاثين، في الأسبوع الخامس، في السنة الأولى منه
\par 7 فولدت له ابنًا في السنة الخامسة منها [1503 صباحًا]، فدعا اسمه عابر. واتخذ لنفسه زوجة اسمها أزوراد، ابنة نبرود، في اليوبيل الثاني والثلاثين، في الأسبوع السابع، في السنة الثالثة منها. [1564 صباحًا]
\par 8 وفي السنة السادسة منها ولدت له ابنًا، فدعا اسمه فالج، لأنه في الأيام التي ولد فيها، بدأ بنو نوح يقسمون الأرض فيما بينهم، لذلك دعا اسمه فالج
\par 9 وقسموها بينهم سرًا، وأخبروا بها نوحًا
\par 10 وحدث في بداية اليوبيل الثالث والثلاثين [1569 صباحًا] أنهم قسموا الأرض إلى ثلاثة أقسام، لسام وحام ويافث، حسب ميراث كل واحد، في السنة الأولى في الأسبوع الأول، عندما كان واحد منا قد أُرسل معهم
\par 11 ثم دعا بنيه، فتقدموا إليه هم وأولادهم، وقسم الأرض قرعة، ليرثها أبناؤه الثلاثة، ومدوا أيديهم، وأخذوا الكتابة من حضن نوح أبيهم
\par 12 وخرج على الكتابة قرعة سام وسط الأرض التي سيأخذها ميراثًا لنفسه ولأبنائه إلى الأبد، من وسط سلسلة جبال رافا، من مصب نهر تينا، ويمتد نصيبه غربًا عبر وسط هذا النهر، ويمتد حتى يصل إلى مياه الهاوية، التي يخرج منها هذا النهر ويصب مياهه في بحر ميت، ويتدفق هذا النهر إلى البحر الكبير. وكل ما هو شمالًا ليافث، وكل ما هو جنوبًا لسام
\par 13 ويمتد حتى يصل إلى كاراسو: هذا في صدر اللسان المتجه نحو الجنوب
\par 14 ويمتد نصيبه على طول البحر الكبير، ويمتد في خط مستقيم حتى يصل إلى غرب اللسان المتجه نحو الجنوب: لأن هذا البحر يُسمى لسان البحر المصري
\par 15 وينعطف من هنا جنوبًا نحو مصب البحر الكبير على شاطئ مياهه، ويمتد غربًا إلى عفرا، ويمتد حتى يصل إلى مياه نهر جيحون، وإلى جنوب مياه جيحون، إلى ضفاف هذا النهر
\par 16 "ويمتد شرقاً حتى يصل إلى جنة عدن جنوبها، ومن شرق كل أرض عدن والشرق كله ينعطف شرقاً ويسير حتى يصل إلى شرق الجبل المسمى رافا، وينزل إلى ضفة مصب نهر تينا."
\par 17 وخرج هذا النصيب بالقرعة لسام وبنيه، لكي يمتلكوه إلى الأبد في أجياله إلى الأبد
\par 18 ففرح نوح لأن هذا النصيب قد خرج لسام ولبنيه، وتذكر كل ما تكلم به بفمه في النبوة، لأنه قال: «مبارك الرب إله سام، وليسكن الرب في مسكن سام».
\par 19 وعلّم أن جنة عدن هي قدس الأقداس، ومسكن الرب، وجبل سيناء قلب البرية، وجبل صهيون قلب سرة الأرض. هذه الثلاثة خُلقت كأماكن مقدسة متقابلة
\par 20 وبارك إله الآلهة الذي جعل كلمة الرب في فمه، والرب إلى الأبد
\par 21 وعلم أن نصيبًا مباركًا وبركة قد أتيا لسام وبنيه إلى الأبد - كل أرض عدن، وكل أرض البحر الأحمر، وكل أرض المشرق والهند، وعلى البحر الأحمر وجباله، وكل أرض باشان، وكل أرض لبنان وجزائر كفتور، وكل جبال شنير وأمانة، وجبال أشور في الشمال، وكل أرض عيلام وأشور وبابل وسوسن ومعداي، وكل جبال أراراط، وكل ما وراء البحر، التي هي وراء جبال أشور نحو الشمال، أرض مباركة وواسعة، وكل ما فيها جيد جدًا
\par 22 وخرج لحام الجزء الثاني، من وراء جيحون نحو الجنوب إلى يمين الجنة، ويمتد نحو الجنوب ويمتد إلى جميع جبال النار، ويمتد نحو الغرب إلى بحر آتيل ويمتد نحو الغرب حتى يصل إلى بحر معوك - ذلك (البحر) الذي ينزل فيه كل ما لم يهلك
\par 23 ويخرج شمالاً إلى حدود غدير، ويخرج إلى ساحل مياه البحر إلى مياه البحر الكبير حتى يقترب من نهر جيحون، ويسير على طول نهر جيحون حتى يصل إلى يمين جنة عدن
\par 24 وهذه هي الأرض التي خرجت لحام نصيبًا له ليملكها إلى الأبد لنفسه ولأبنائه مدى أجيالهم إلى الأبد
\par 25 وخرج ليافث القسم الثالث من عبر نهر تينا إلى الشمال من مخرج مياهه، ويمتد شمالاً شرقياً إلى كل منطقة جوج، وإلى كل الأرض شرقيها
\par 26 ويمتد شمالاً إلى الشمال، ويمتد إلى جبال القلط شمالاً، ونحو بحر معوك، ويخرج إلى شرقي غدير إلى منطقة مياه البحر.
\par 27 ويمتد حتى يقترب من غرب فرعا ويعود نحو عفرَق، ويمتد شرقًا إلى مياه بحر مِعت
\par 28 ويمتد إلى منطقة نهر الطينة باتجاه الشمال الشرقي حتى يقترب من حدود مياهه باتجاه جبل رافا، ثم ينعطف نحو الشمال
\par 29 هذه هي الأرض التي خرجت ليافث وبنيه نصيبًا من ميراثه الذي يملكه لنفسه ولبنيه لأجيالهم إلى الأبد: خمس جزر عظيمة، وأرض عظيمة في الشمال
\par 30 لكنها باردة، وأرض حام حارة، وأرض سام ليست حارة ولا باردة، بل هي مزيج من البرد والحر

\chapter{9}

\par \textit{تقسيم الأنصبة الثلاثة بين أحفاد نوح. بين أبناء حام، 1: سام، 2-6: يافث، 7-13. قسم أبناء نوح، 14-15.}

\par 1 وقسم حام بين أبنائه، فخرج القسم الأول لكوش شرقًا، وغربًا منه لمصرايم، وغربًا منه لفوط، وغربًا منه [وغربها] على البحر لكنعان
\par 2 وقسم سام أيضًا بين أبنائه، فخرج الجزء الأول لحام وبنيه، شرقي نهر دجلة حتى يقترب من الشرق، كل أرض الهند، وعلى البحر الأحمر على سواحله، ومياه ددان، وجميع جبال مبري وأيلا، وكل أرض سوسن وكل ما على جانب فرناق إلى البحر الأحمر ونهر تينا
\par 3 وخرج لأشور القسم الثاني، كل أرض أشور ونينوى وشنعار وحتى حدود الهند، ويرتفع ويحاذي النهر
\par 4 وخرج لأرفكشاد القسم الثالث، كل أرض منطقة الكلدانيين شرقي الفرات، المتاخمة للبحر الأحمر، وكل مياه البرية التي عند لسان البحر المتجه نحو مصر، وكل أرض لبنان وشنير وأمانة إلى تخوم الفرات
\par 5 وخرج لأرام الحصة الرابعة، كل أرض ما بين النهرين من دجلة والفرات إلى شمال الكلدانيين إلى حدود جبال أشور وأرض عرعرة
\par 6 وخرج للود الجزء الخامس، جبال أشور وكل ما يتعلق بها حتى تصل إلى البحر الكبير، وحتى تصل إلى شرقي أشور أخيه
\par 7 وقسم يافث أيضًا أرض ميراثه بين بنيه
\par 8 وخرج القسم الأول لجومر شرقا من جانب الشمال إلى نهر تينة، وخرج إلى الشمال لماجوج كل أقاصي الشمال حتى وصل إلى بحر مئت.
\par 9 وخرج ماداي نصيبه أن يمتلك من غرب أخويه إلى الجزر، وإلى سواحل الجزر
\par 10 وأما ياوان فخرج الربع لكل جزيرة والجزائر التي نحو حدود لود
\par 11 وخرج لتوبال الجزء الخامس في وسط اللسان الذي يقترب من طرف جزء لود إلى اللسان الثاني، إلى المنطقة التي وراء اللسان الثاني إلى اللسان الثالث
\par 12 وخرج لماشك الجزء السادس، كل المنطقة الواقعة وراء اللسان الثالث حتى تقترب من شرقي جادير
\par 13 وخرج لتيراس الحصة السابعة، أربع جزر عظيمة في وسط البحر، وهي تصل إلى حصة حام [وخرجت جزر كاماتوري بالقرعة لأبناء أرفكشاد ميراثًا له].
\par 14 وهكذا قسم بنو نوح على أبنائهم بحضور نوح أبيهم، وربطهم جميعًا بيمين، ولعن كل من سعى إلى الاستيلاء على النصيب الذي لم يسقط له بقرعته
\par 15 فقالوا جميعهم: «ليكن، فليكن» لأنفسهم ولأبنائهم إلى الأبد في أجيالهم إلى يوم الدين، الذي فيه يدينهم الرب الإله بالسيف والنار على كل شرور خطاياهم النجسة التي ملأوا بها الأرض معصيةً ونجاسةً وزنىً وخطيئة

\chapter{10}

\par \textit{الأرواح الشريرة تُضل أبناء نوح، 1-2. صلاة نوح، 3-6. سُمح لماستيما بالاحتفاظ بعُشر أرواحه الخاضعة، 7-11. علّم نوح الملائكة استخدام الأعشاب لمقاومة الشياطين، 12-14. موت نوح، 15-17. بناء بابل وبلبلة الألسنة، 18-27. استيلاء كنعان على فلسطين، 29-34. استلام ماداي لميديا، 33-6.}

\par 1 وفي الأسبوع الثالث من هذا اليوبيل، بدأت الشياطين النجسة تضل أبناء أبناء نوح، وتجعلهم يضللون ويهلكونهم
\par 2 فجاء بنو نوح إلى نوح أبيهم، وأخبروه عن الشياطين التي تُضل أبناء أبنائه وتُعميهم وتقتلهم
\par 3 وصلى أمام الرب إلهه وقال:
\par    
\par    'إله أرواح كل جسد، الذي أظهر لي الرحمة  
\par    وخلصتني وأبنائي من مياه الطوفان،  
ولم تُهلِكْني كما هَلَكْتَ أبناءَ الهلاك.
\par    
\par    لأن نعمتك كانت عظيمة عليّ،  
ولقد كانت رحمتك عظيمة على نفسي.
\par    
\par    لترتفع نعمتك على أبنائي،  
\par    ولا تسلط عليهم الأرواح الشريرة  
لئلا يبيدوهم من الأرض.
\par    
\par 4 لكن باركني وأبنائي، لكي نتكاثر ونكثر ونملأ الأرض
\par 5 وأنت تعلم كيف تصرف مراقبوك، آباء هذه الأرواح، في أيامي: أما هذه الأرواح الحية، فاسجنها واحتفظ بها في مكان الدينونة، ولا تدعها تجلب الهلاك على أبناء عبدك، إلهي؛ لأنها خبيثة، وخُلقت من أجل الهلاك
\par 6 ولا يتسلطوا على أرواح الأحياء، لأنك وحدك القادر على التسلط عليهم. ولا يتسلطوا على أبناء الصالحين من الآن وإلى الأبد
\par 7 وأمرنا الرب إلهنا أن نربط الجميع.
\par 8 "فجاء رئيس الأرواح، ماستيما، وقال: "أيها الرب الخالق، دع بعضهم يبقوا أمامي، ودعهم يستمعون إلى صوتي، ويفعلون كل ما أقول لهم؛ لأنه إذا لم يُترك لي بعضهم، فلن أتمكن من تنفيذ قوة إرادتي على أبناء البشر؛ لأن هؤلاء هم للفساد والضلال أمام دينونتي، لأن شرور أبناء البشر عظيمة".
\par 9 فقال: «ليبقِ العُشرُ منها أمامه، ولينزل تسعةُ أجزاءٍ إلى موضعِ الهلاك».
\par 10 وأمرنا أن نعلم نوحًا جميع أدويتهم، لأنه علم أنهم لا يسلكون في الاستقامة، ولا يجاهدون في البر
\par 11 وفعلنا حسب جميع أقواله: قيدنا جميع الأشرار المفسدين في موضع الدينونة، وتركنا عشرهم ليكونوا خاضعين للشيطان على الأرض
\par 12 وشرحنا لنوح جميع أدوية أمراضهم، مع غواياتهم، كيف يشفيهم بأعشاب الأرض
\par 13 وكتب نوح كل شيء في كتاب، وأوعزنا إليه من كل نوع من الأدوية. وهكذا منعت الأرواح الشريرة عن (إيذاء) أبناء نوح
\par 14 وأعطى سامًا ابنه الأكبر كل ما كتبه، لأنه أحبه أكثر من جميع أبنائه
\par 15 واضطجع نوح مع آبائه، ودُفن في جبل لوبار في أرض أراراط
\par 16 أكمل في حياته تسعمائة وخمسين عامًا، منها تسعة عشر يوبيلًا وأسبوعان وخمس سنوات. [1659 صباحًا]
\par 17 وفي حياته على الأرض، تفوق على بني البشر ما عدا أخنوخ بسبب البر الذي كان كاملاً فيه. لأن وظيفة أخنوخ كانت مُعَيَّنة للشهادة لأجيال العالم، حتى يروي جميع أعمال جيل إلى جيل، إلى يوم الدينونة
\par 18 وفي اليوبيل الثالث والثلاثين في السنة الأولى في الأسبوع الثاني اتخذ فالج امرأة اسمها لمنة ابنة شنعار فولدت له ابنا في السنة الرابعة من هذا الأسبوع ودعا اسمه رعو لأنه قال هوذا بنو البشر قد أفسدوا بنوا لأنفسهم مدينة وبرجا في أرض شنعار.
\par 19 لأنهم ارتحلوا من أرض أراراط شرقًا إلى شنعار، لأنهم في أيامه بنوا المدينة والبرج، قائلين: هلموا نصعد إلى السماء
\par 20 وشرعوا في البناء، وفي الأسبوع الرابع صنعوا لبنا بالنار، فاستخدموه حجرًا، وكان الطين الذي ألصقوه به أسفلتًا يخرج من البحر ومن ينابيع الماء في أرض شنعار
\par 21 وبنوه: ثلاث وأربعون سنة [1645-1688 صباحًا]، وكان عرضه مئتين وثلاث لبنات، وارتفاعه ثلث لبنة، وكان ارتفاعه خمسة آلاف وأربعمائة وثلاثة وثلاثين ذراعًا ونخلتين، وكان طول أحد الجدران ثلاثة عشر غلوة (وطول الآخر ثلاثين غلوة).
\par 22 وقال لنا الرب إلهنا: «هوذا هم شعب واحد، وهذا ما بدأوا يفعلونه، والآن لن يُمنع عنهم شيء. هلموا، لننزل ونبلبل لغتهم، حتى لا يفهم بعضهم كلام بعض، ويتشتتوا في مدن وأمم، ولا يبقى لهم رأي واحد إلى يوم الدين».
\par 23 ونزل الرب، ونزلنا معه لننظر المدينة والبرج اللذين بناهما بنو البشر
\par 24 فبلبل لسانهم، فلم يعودوا يفهمون كلام بعضهم بعضًا، فتوقفوا عن بناء المدينة والبرج
\par 25 لهذا السبب تُدعى كل أرض شنعار بابل، لأن الرب هناك بلبل كل لسان بني البشر، ومن هناك تشتتوا في مدنهم، كل واحد حسب لغته وأمته
\par 26 فأرسل الرب ريحًا شديدة على البرج فقلبه إلى الأرض، وإذا هو بين أشور وبابل في أرض شنعار، فدعوا اسمه قلب
\par 27 في الأسبوع الرابع من السنة الأولى [1688 صباحًا] في بدايتها، في اليوبيل الرابع والثلاثين، تم تفريقهم من أرض شنعار
\par 28 وذهب حام وبنوه إلى الأرض التي كان عليه أن يسكنها، والتي حصل عليها كنصيب له في أرض الجنوب
\par 29 ورأى كنعان أرض لبنان إلى نهر مصر أنها جيدة جدًا، فلم يدخل أرض ميراثه غرب البحر، وسكن في أرض لبنان شرقًا وغربًا من تخوم الأردن ومن تخوم البحر
\par 30 فقال له حام أبوه وكوش ومصرايم إخوته: «لقد سكنت أرضاً ليست لك ولم تصيبنا بالقرعة. لا تفعل هكذا، لأنك إن فعلت هكذا تسقط أنت وبنوك في الأرض وتلعنون بالفتنة. لأنكم بالفتنة سكنتم، وبالفتنة يسقط أبناؤكم، وتستأصلون إلى الأبد».
\par 31 «لا تسكنوا في مسكن سام، لأنه لسام ولأبنائه جاء ذلك بقرعة».
\par 32 «ملعون أنت، وستكون ملعونًا من بين جميع أبناء نوح، باللعنة التي ألزمنا أنفسنا بها بقسم أمام القاضي القدوس، وأمام نوح أبينا.»
\par 33 لكنه لم يسمع لهم، وأقام في أرض لبنان من حماة إلى مدخل مصر، هو وبنوه إلى هذا اليوم
\par 34 ولهذا السبب سُميت تلك الأرض كنعان.
\par 35 وذهب يافث وبنوه نحو البحر وسكنوا في أرض نصيبهم، فنظر ماداي أرض البحر فلم تعجبه، فطلب نصيباً من حام وأشور وأرفكشاد أخي امرأته، وسكن في أرض ميديا ​​عند أخي امرأته إلى هذا اليوم.
\par 36 ودعا مسكنه ومسكن بنيه ميديا، على اسم أبيهم ماداي

\chapter{11}

\par \textit{رعو وسروج، 1 (قارن تكوين 11: 20، 21). نشوء الحرب وسفك الدماء وأكل الدماء وعبادة الأصنام، 2-7. ناحور وتارح، 8-14 (قارن تكوين 11: 22-30). معرفة أبرام بالله وأعماله العجيبة، 15-24.}

\par 1 وفي اليوبيل الخامس والثلاثين، في الأسبوع الثالث، في السنة الأولى منه، اتخذ رعو لنفسه امرأة اسمها عورة، ابنة عور بن قيس، وولدت له ابنًا، فدعا اسمه سروه، في السنة السابعة من هذا الأسبوع في هذا اليوبيل. [1687 صباحًا]
\par 2 وبدأ بنو نوح يتحاربون فيما بينهم، ويأسرون ويقتلون بعضهم بعضًا، ويسفكون دماء الناس على الأرض، ويأكلون الدماء، ويبنون مدنًا قوية وأسوارًا وأبراجًا، وبدأ أفراد يتسامون على الأمة، ويؤسسون بدايات الممالك، ويخوضون الحروب شعبًا ضد شعب، وأمة ضد أمة، ومدينة ضد مدينة، وبدأ الجميع يفعلون الشر، ويقتنون السلاح، ويعلمون أبناءهم الحرب، وبدأوا يستولون على المدن، ويبيعون العبيد والإماء
\par 3 وبنى أور بن قيس مدينة عَرَّة الكلدانية، ودعا اسمها باسمه واسم أبيه. وصنعوا لأنفسهم تماثيل مسبوكة، وسجد كل واحد للصنم، التمثال المسبوك الذي صنعوه لأنفسهم، وبدأوا يصنعون تماثيل منحوتة وأمثالًا نجسة، وساعدتهم أرواح شريرة وأغوتهم لارتكاب المعصية والنجاسة
\par 4 فبذل الأمير مستيما كل جهده للقيام بكل هذا، وأرسل أرواحًا أخرى، تلك التي وضعت تحت يده، لتفعل كل أنواع الخطأ والخطيئة، وكل أنواع المعصية، وتفسد وتدمر، وتسفك الدماء على الأرض.
\par 5 لهذا السبب دعا اسم سروح، سروج، لأن كل إنسان انقلب إلى فعل كل أنواع الخطيئة والمعصية
\par 6 ونشأ وسكن في أور الكلدانيين، بالقرب من أبي أم امرأته، وكان يعبد الأصنام، وتزوج امرأة في اليوبيل السادس والثلاثين، في الأسبوع الخامس، في السنة الأولى منه، [1744 صباحًا]، وكان اسمها ملكة بنت كابر بنت أخيه
\par 7 فولدت له ناحور في السنة الأولى من هذا الأسبوع، فنشأ وسكن في أور الكلدانيين، وعلمه أبوه علوم الكلدانيين في العرافة والتنجيم حسب آيات السماء
\par 8 وفي اليوبيل السابع والثلاثين، في الأسبوع السادس، في السنة الأولى منه، [1800 صباحًا] اتخذ لنفسه زوجة، وكان اسمها إياسكة، ابنة نيستاغ الكلداني
\par 9 فولدت له تارح في السنة السابعة من هذا الأسبوع. [1806 صباحًا]
\par 10 فأرسل الأمير مستيما الغربان والطيور لتلتهم البذر المزروع في الأرض، لتدميرها، وسلب أبناء البشر تعبهم. وقبل أن يحرثوا البذر، التقطته الغربان من على سطح الأرض.
\par 11 ولهذا السبب دعا اسمه تارح لأن الغربان والطيور أفقرتهم وأكلت بذورهم
\par 12 وبدأت السنين قاحلة بسبب الطيور، فأكلت كل ثمر الشجر من الأشجار: ولم يكن بوسعهم إلا بجهد كبير أن يدخروا القليل من كل ثمر الأرض في أيامهم
\par 13 وفي هذا اليوبيل التاسع والثلاثين، في الأسبوع الثاني من السنة الأولى [1870 صباحًا]، اتخذ تارح لنفسه زوجة، وكان اسمها عدنة، ابنة أبرام، ابنة عم أبيه. وفي السنة السابعة من هذا الأسبوع [1876 صباحًا] ولدت له ابنًا، فدعا اسمه أبرام، باسم أبي أمه
\par 14 لأنه مات قبل أن تحمل ابنته بابن.
\par 15 "وبدأ الطفل يفهم أخطاء الأرض التي ضلت كلها وراء الصور المنحوتة والنجاسة، وعلمه أبوه الكتابة، وكان عمره أسبوعين من السنين، [1890 ش] وانفصل عن أبيه، حتى لا يعبد الأصنام معه."
\par 16 وبدأ يصلي إلى خالق كل شيء ليخلصه من أخطاء بني البشر، وألا يقع نصيبه في الخطأ بعد النجاسة والدناءة
\par 17 وجاء وقت زرع البذر على الأرض، فخرجوا جميعا لحماية نسلهم من الغربان، وخرج أبرام مع الذين ذهبوا، وكان الصبي صبيا ابن أربع عشرة سنة.
\par 18 فجاءت سحابة من الغربان لتأكل النسل، فركض أبرام للقائها قبل أن تستقر على الأرض، ونادى عليها قبل أن تستقر على الأرض لتأكل النسل، وقال: «لا تنزلوا. ارجعوا إلى المكان الذي جئتم منه». فعادوا أدراجهم
\par 19 فأرجع سحب الغربان في ذلك اليوم سبعين مرة، ولم يبق من جميع الغربان التي في كل الأرض التي كان أبرام هناك غربان واحد
\par 20 فرأى جميع الذين كانوا معه في كل الأرض صراخه، ورجعت جميع الغربان إلى الوراء، وعظم اسمه في كل أرض الكلدانيين
\par 21 وجاء إليه هذا العام جميع الذين أرادوا أن يزرعوا، فذهب معهم حتى انتهى وقت الزراعة. فزرعوا أرضهم، وفي تلك السنة أحضروا ما يكفي من الحبوب إلى بيوتهم وأكلوا وشبعوا
\par 22 وفي السنة الأولى من الأسبوع الخامس [1891 صباحًا] علّم أبرام صانعي أدوات الثيران، الصناع من الخشب، فصنعوا إناءً فوق الأرض، مواجهًا إطار المحراث، ليضعوا البذرة عليه، فسقطت البذرة منه على نصيب المحراث، واختفت في الأرض، ولم يعودوا يخافون الغربان
\par 23 وعلى هذا المنوال صنعوا أوعية فوق الأرض على جميع هياكل المحاريث، وزرعوا وفلحون كل الأرض، كما أمرهم أبرام، ولم يعودوا يخافون الطيور

\chapter{12}

\par \textit{سعى أبرام إلى إبعاد تارح عن عبادة الأصنام، 1-8. تزوج ساراي، 9. حاران وناحور، 9-11. أحرق أبرام الأصنام: موت حاران، 12-14 (راجع تكوين 11: 28). ذهب تارح وعائلته إلى حاران، 15. راقب أبرام النجوم وصلى، 16-21. أُمر بالذهاب إلى كنعان ومُبارك، 22-4. مُنح القدرة على التحدث بالعبرية، 25-7. غادر حاران إلى كنعان، 28-31. (راجع تكوين 11: 31-12: 3.)}

\par 1 وفي الأسبوع السادس، في السنة السابعة منه، [1904 صباحًا]، قال أبرام لتارح أبيه: «يا أبي!»
\par 2 فقال: ها أنا ذا يا ابني. فقال:
\par    
\par     'ما هي المساعدة والفائدة التي لدينا من تلك الأصنام التي تعبدها،  
\par    وأمام أي شيء تنحني؟
\par    
\par 3 لأنه ليس فيها روح،  
\par     لأنها أشكال بكماء، وتضليل للقلب  
\par    لا تعبدوهم:
\par    
\par 4 اعبدوا إله السماء،  
\par     الذي يُنزِل المطر والندى على الأرض  
\par    ويفعل كل شيء على الأرض،
\par    
\par    وخلق كل شيء بكلمته،  
\par    وكل حياة من أمام وجهه.
\par    
\par 5 لماذا تعبدون أشياءً لا روح فيها؟  
\par    لأنها من عمل أيدي (الرجال)،
\par    
\par    وعلى أكتافك تحملهم،  
\par    وليس لكم منهم عون،
\par    
\par    ولكنها سبب عار كبير لأولئك الذين يصنعونها،  
\par    وإضلال القلوب للذين يعبدوهم:  
\par    لا تعبدوهم.'
\par    
\par 6 فقال له أبوه: وأنا أعلم ذلك أيضًا يا ابني، ولكن ماذا أفعل بقوم أعبدوني أمامهم؟
\par 7 وإن أخبرتهم الحقيقة، سيقتلونني؛ لأن نفوسهم ملتصقة بهم لتعبّدهم وتكريمهم
\par 8 اصمت يا بني لئلا يقتلوك. وقال هذه الكلمات لأخويه، فغضبا عليه فسكت
\par 9 وفي اليوبيل الأربعين، في الأسبوع الثاني، في السنة السابعة منه، [1925 صباحًا] اتخذ أبرام لنفسه امرأة اسمها ساراي، ابنة أبيه، وصارت له زوجة
\par 10 وتزوج هاران أخوه امرأة في السنة الثالثة من الأسبوع الثالث، [1928 صباحًا]، وولدت له ابنًا في السنة السابعة من هذا الأسبوع، [1932 صباحًا]، ودعا اسمه لوطًا
\par 11 واتخذ ناحور أخوه زوجة.
\par 12 وفي السنة الستين من حياة إبرام، أي في الأسبوع الرابع، في السنة الرابعة منه، قام إبرام ليلا وأحرق بيت الأصنام، وأحرق كل ما في البيت ولم يعلم أحد.
\par 13 وقاموا في الليل وطلبوا إنقاذ آلهتهم من وسط النار
\par 14 فأسرع هاران لإنقاذهم، لكن النار التهمت فوقه، فاحترق في النار، ومات في أور الكلدانيين أمام تارح أبيه، فدفنوه في أور الكلدانيين
\par 15 وخرج تارح من أور الكلدانيين هو وبنوه ليذهب إلى أرض لبنان وأرض كنعان، وأقام في أرض حاران. وأقام أبرام مع تارح أبيه في حاران أسبوعين من السنين
\par 16 وفي الأسبوع السادس، في السنة الخامسة منه، [1951 صباحًا] جلس أبرام طوال الليل في هلال الشهر السابع ليراقب النجوم من المساء إلى الصباح، ليرى ما ستكون عليه السنة من حيث الأمطار، وكان جالسًا بمفرده ويراقب
\par 17 وخطر بباله كلمة فقال: جميع علامات النجوم، وعلامات القمر والشمس، كلها في يد الرب. لماذا أبحث عنها؟
\par    
\par 18 إن شاء أمطرها صباحًا ومساءً.  
\par    وإن شاء أمسكها  
وكل شيء في يده.
\par    
\par 19 وصلى تلك الليلة وقال:  
\par     'إلهي، الله العلي، أنت وحدك إلهي،  
وإياك وسلطانك اخترت.  
\par    وأنت خلقت كل شيء،  
وكل ما هو عمل يديك.
\par    
\par 20 نجني من أيدي الأرواح الشريرة التي تتحكم في أفكار قلوب البشر،  
ولا يضليني عنك يا إلهي.
\par    
\par    وأثبتني ونسلي إلى الأبد  
\par    أن لا نضل من الآن وإلى الأبد.
\par    
\par 21 فقال: «أَأَرْجِعُ إِلَى أُورِ الْكَلْدَانِيِّينَ الَّذِينَ يَطْلُبُونَ وَجْهِي لِأَرْجِعَ إِلَيْهِمْ؟ أَأَبْقَى هُنَا فِي هَذَا الْمَكَانِ؟ السَّبِيرُ الْمُسْتَقِيمَ أَمَامَكَ، فَأَنْجِحْهُ فِي يَدِي عَبْدِكَ لِكَيْ يُكَمِّلَهُ، وَلاَ أَسْلُكُ فِي غُرَّةِ قَلْبِي، يَا إِلَهِي».
\par 22 ولما فرغ من الكلام والصلاة، إذا كلمة الرب قد أُرسلت إليه بواسطتي قائلة: «اصعد من أرضك ومن عشيرتك ومن بيت أبيك إلى الأرض التي أريكها، فأجعلك أمة عظيمة وكثيرة.»
\par    
\par 23 وأباركك  
\par     وأعظم اسمك،  
\par    وتُبارَكُ في الأرضِ،  
\par    وفيك تتبارك جميع قبائل الأرض،  
\par    وأبارك مباركيك،  
\par    والعن لاعنيك.
\par    
\par 24 وأكون إلها لك ولابنك ولابن ابنك ولجميع نسلك. لا تخف، من الآن فصاعدا وإلى جميع أجيال الأرض أنا إلهك
\par 25 وقال الرب الإله: «افتحوا فمه وأذنيه، فيسمع ويتكلم بفمه، باللسان الذي أُعلن»، لأنه توقف عن أفواه جميع بني البشر منذ يوم الانقلاب (بابل).
\par 26 وفتحت فمه وأذنيه وشفتيه، وبدأت أتحدث معه بالعبرية بلسان الخليقة
\par 27 وأخذ كتب آبائه، وكانت مكتوبة بالعبرية، ونسخها، وبدأ من الآن فصاعدًا يدرسها، وأعلمته ما لم يستطع فهمه، ودرسها خلال الأشهر الستة الممطرة
\par 28 وفي السنة السابعة من الأسبوع السادس [1953 صباحًا]، كلم أباه وأخبره أنه سيخرج من حاران ويذهب إلى أرض كنعان ليرىها ويرجع إليه
\par 29 فقال له تارح أبوه: اذهب بسلام
\par    
\par    ليجعل الله الأبدي طريقك مستقيمًا.  
\par    و الرب معك و يحفظك من كل شر،  
ويمنحك النعمة والرحمة والفضل أمام من يرونك،  
ولا يجوز لأحد من أبناء البشر أن يكون له عليك سلطان أن يؤذيك؛  
\par    اذهب بسلام.
\par    
\par 30 وإن رأيت أرضًا شهية لعينيك للسكنى فيها، فقم وخذني إليك، وخذ معك لوطا ابن هاران أخيك كابنك. ليكن الرب معك
\par 31 وناحور أخوك اذهب معي حتى ترجع بسلام، ونذهب معك جميعًا

\chapter{13}

\par \textit{يسافر أبرام من حاران إلى شكيم في كنعان، ومن ثم إلى حبرون ومن ثم إلى مصر، 1-14أ. يعود إلى كنعان حيث انفصل عنه لوط، وينال وعد كنعان ويسافر إلى حبرون، 14ب-21. هجوم كدرلعومر على سدوم وعمورة: أسر لوط، 22-4. سن قانون العشور، 25-9. (راجع تكوين 12: 4-10، 15-17، 19-20؛ 13: 11-18؛ 14: 8-14؛ 21-4.)}

\par 1 ثم ارتحل أبرام من حاران، وأخذ ساراي امرأته ولوطًا ابن أخيه هاران إلى أرض كنعان، وجاء إلى أشور، ومضى إلى شكيم، وأقام عند بلوطة عالية
\par 2 فنظر وإذا الأرض من مدخل حماة إلى البلوطة العالية جميلة جدًا
\par 3 وقال له الرب: لك ولنسلك أعطي هذه الأرض
\par 4 فبنى هناك مذبحًا، وأصعد عليه محرقة للرب الذي ظهر له
\par 5 ثم انتقل من هناك إلى الجبل ... بيت إيل من الغرب وعاي من الشرق، ونصب خيمته هناك
\par 6 فرأى وإذا بالأرض واسعة وجيدة جدًا، وكل شيء ينمو فيها: كرم وتين ورمان، وبلوط وبلوط أخضر، وبطم وأشجار زيتون، وأرز وسرو ونخيل، وكل أشجار الحقل، وكان ماء على الجبال
\par 7 وبارك الرب الذي أخرجه من أور الكلدانيين وأتى به إلى هذه الأرض
\par 8 وحدث في السنة الأولى، في الأسبوع السابع، في هلال الشهر الأول، سنة 1954 صباحًا] أنه بنى مذبحًا على هذا الجبل، ودعا باسم الرب: «أنت الإله الأزلي، إلهي».
\par 9 وأصعد على المذبح محرقة للرب ليكون معه ولا يتركه كل أيام حياته
\par 10 ثم انتقل من هناك واتجه نحو الجنوب، ووصل إلى حبرون، وكانت حبرون مبنية في ذلك الوقت، وأقام هناك سنتين، ثم ذهب (من هناك) إلى أرض الجنوب، إلى بعلوت، وكان هناك مجاعة في الأرض
\par 11 فذهب أبرام إلى مصر في السنة الثالثة من الأسبوع، وأقام في مصر خمس سنوات قبل أن تُفصل عنه امرأته
\par 12 وكانت مدينة طنيس في مصر قد بُنيت في ذلك الوقت، بعد الخليل بسبع سنوات
\par 13 وحدث لما أمسك فرعون ساراي امرأة أبرام أن الرب ضرب فرعون وبيته ضربات عظيمة بسبب ساراي امرأة أبرام
\par 14 وكان أبرام مجيدًا جدًا بما يملك من غنم وبقر وحمير وخيول وإبل وعبيد وإماء، ومن فضة وذهب كثيرًا جدًا. وكان لوط أيضًا ابن أخيه غنيًا
\par 15 فرد فرعون ساراي امرأة أبرام، فأخرجه من أرض مصر، فسافر إلى المكان الذي نصب فيه خيمته في البداءة، إلى مكان المذبح، وعاي من الشرق، وبيت إيل من الغرب، وبارك الرب إلهه الذي رده بسلام
\par 16 وكان في اليوبيل الحادي والأربعين في السنة الثالثة من الأسبوع الأول، أنه رجع إلى هذا المكان وأصعد عليه محرقة، ودعا باسم الرب، وقال: أنت الإله العلي إلهي إلى الأبد
\par 17 وفي السنة الرابعة من هذا الأسبوع [1964 صباحًا] انفصل عنه لوط، وسكن لوط في سدوم، وكان رجال سدوم خطاة جدًا
\par 18 فحزن في قلبه لأن ابن أخيه قد افترق عنه، إذ لم يكن له أولاد
\par 19 في تلك السنة التي أُسر فيها لوط، قال الرب لأبرام، بعد أن افترق عنه لوط، في السنة الرابعة من هذا الأسبوع: «ارفع عينيك عن المكان الذي أنت مقيم فيه، شمالاً وجنوباً، وغرباً وشرقاً
\par 20 لأن جميع الأرض التي أنت ترى لك أعطيها ولنسلك إلى الأبد، وأجعل نسلك كرمل البحر. وإن أحصى أحد تراب الأرض، فلن يُحصى نسلك
\par 21 قم امشِ في الأرض طولها وعرضها، وانظرها كلها، لأني لنسلك أعطيها. وذهب أبرام إلى حبرون وأقام هناك
\par 22 وفي تلك السنة جاء كدرلعومر ملك عيلام، وأمرافل ملك شنعار، وأريوك ملك سلاسار، وترجال ملك الأمم، وقتلوا ملك عمورة، فهرب ملك سدوم، وسقط كثيرون متأثرين بالجروح في عمق السديم عند بحر الملح
\par 23 وأسروا سدوم وآدم وصبويم، وأسروا لوطا أيضًا ابن أخي أبرام وكل ممتلكاته، وذهبوا إلى دان
\par 24 فجاء واحد من الذين نجوا وأخبر أبرام أن ابن أخيه قد أُسر، وسلّح (أبرام) عبيد بيته...
\par 25 . . . . . لأبرام ولنسله عُشر البواكير للرب، وجعلها الرب فريضة أبدية ليعطوها للكهنة الذين يخدمون أمامه، فيمتلكونها إلى الأبد
\par 26 "ولهذا الناموس ليس له حد أيام، لأنه جعله للأجيال إلى الأبد أن يعطوا للرب العشر من كل شيء من البذر والخمر والزيت والبقر والغنم."
\par 27 وأعطاها لكهنته ليأكلوا ويشربوا بفرح أمامه
\par 28 فتقدم إليه ملك سدوم وسجد له وقال: يا سيدنا أبرام، أعطنا النفوس التي أنقذتها، ولكن ليكن لك الغنيمة
\par 29 فقال له أبرام: «أرفع يدي إلى الله العلي، أني لا آخذ منكَ شيئًا لا خيطًا ولا شراك نعل، لئلا تقول: أنا غنيتُ أبرام، إلا ما أكله الغلمان، ونصيب الرجال الذين ذهبوا معي: عانر، وأشكول، وممرا. هؤلاء يأخذون نصيبهم».

\chapter{14}

\par \textit{ينال أبرام الوعد بابن ونسل لا يُحصى، 1-7. يقدم ذبيحة ويُخبر عن نسله في مصر، 8-17. عهد الله مع أبرام، 18-20. هاجر تلد إسماعيل، 21-4. (راجع تكوين 15؛ 16: 1-4، 11.)}


\par 1 بعد هذه الأمور، في السنة الرابعة من هذا الأسبوع، في بداية الشهر الثالث، كانت كلمة الرب إلى أبرام في الحلم قائلة: «لا تخف يا أبرام، أنا حاميك، وأجرك عظيم جدًا».
\par 2 فقال: «يا رب، يا رب، ماذا تعطيني وأنا ماضٍ بلا أولاد، وابن ماسق، ابن أمتي، هو أليعازر الدمشقي. هو وارثي، ولم تعطني نسلا».
\par 3 فقال له: «لن يكون هذا وريثك، بل الذي يخرج من أحشائك. هو يكون وريثك».
\par 4 فأخرجه إلى خارج، وقال له: «انظر إلى السماء، وعد النجوم إن استطعت أن تعدها».
\par 5 ونظر نحو السماء ونظر إلى النجوم. فقال له: «هكذا يكون نسلك».
\par 6 فآمن بالرب، فحسب له ذلك برًا
\par 7 فقال له: «أنا الرب الذي أخرجك من أور الكلدانيين ليعطيك أرض الكنعانيين لتمتلكها إلى الأبد، وأكون إلهًا لك ولنسلك من بعدك».
\par 8 فقال: يا رب، يا رب، بماذا أعلم أني أرثها؟
\par 9 فقال له: خذ لي عجلة ثلاث سنين، وعنزة ثلاث سنين، وخروفًا ثلاث سنين، ويمامة وحمامة
\par 10 فأخذ كل هذه في منتصف الشهر، وأقام عند بلوطة ممرا التي بالقرب من حبرون
\par 11 فبنى هناك مذبحًا، وذبح كل هذه، وسكب دمها على المذبح، وشقها في الوسط، وجعلها بعضها مقابل بعض. أما الطيور فلم يشقها
\par 12 فنزلت الطيور على القطع، فطردها أبرام، ولم يدع الطيور تمسها
\par 13 "وحدث لما غربت الشمس أن دهشة وقعت على أبرام، وإذا رعب من ظلام عظيم وقع عليه، وقيل لأبرام: "اعلم يقينا أن نسلك سيكون غريبا في أرض ليست لهم، فيستعبدونهم ويذلونهم أربعمائة سنة".
\par 14 «والأمة التي يستعبدون لها سأدينها أيضًا، وبعد ذلك يخرجون من هناك بمال كثير.»
\par 15 «وتذهب إلى آبائك بسلام، وتُدفن بشيبة صالحة.»
\par 16 «ولكن في الجيل الرابع يرجعون إلى هنا، لأن إثم الأموريين لم يكتمل بعد.»
\par 17 فاستيقظ من نومه وقام، وكانت الشمس قد غربت، وإذا لهيب، وإذا بتنور يدخن، ولهيب نار يمر بين القطع
\par 18 وفي ذلك اليوم قطع الرب مع أبرام عهدًا قائلًا: «لنسلك أعطي هذه الأرض، من نهر مصر إلى النهر الكبير، نهر الفرات، القينيين، والقنزيين، والقدمونيين، والفرزيين، والرفائيين، والفقوريين، والحويين، والأموريين، والكنعانيين، والجرجاشيين، واليبوسيين».
\par 19 ومضى النهار، فقدّم أبرام القطع والطيور وتقدمتها وسكائبها، فأكلتها النار
\par 20 وفي ذلك اليوم، قطعنا عهدًا مع أبرام، كما قطعنا عهدًا مع نوح في هذا الشهر، وجدد أبرام لنفسه العيد والفريضة إلى الأبد
\par 21 ففرح أبرام وأخبر ساراي امرأته بكل هذه الأمور، فآمن أنه سيكون له نسل، ولكنها لم تلد
\par 22 فأشارت ساراي على زوجها أبرام وقالت له: «ادخل على هاجر جاريتي المصرية لعلي أبني لك منها نسلا».
\par 23 فسمع أبرام لصوت ساراي امرأته وقال لها: افعلي هكذا. فأخذت ساراي هاجر المصرية جاريتها وأعطتها لأبرام زوجها زوجة له
\par 24 فدخل عليها، فحبلت وولدت له ابنًا، ودعا اسمه إسماعيل، في السنة الخامسة من هذا الأسبوع [1965 صباحًا]، وكانت هذه السنة السادسة والثمانين من حياة أبرام

\chapter{15}

\par \textit{يحتفل أبرام بعيد البواكير، 1-2: يتغير اسمه ويُشرع الختان، 3-14. يتغير اسم ساراي ويُوعد إسحاق، 15-21. إبراهيم وإسماعيل وجميع أهل بيته مختونون، 22-4. الختان رسامة أبدية، 25، 26. يشترك إسرائيل في هذا الشرف مع أعلى الملائكة الذين خُلقوا مختونين، 27-9. إسرائيل خاضعة لله وحده: الأمم الأخرى للملائكة، 30-2. عدم أمانة إسرائيل في المستقبل، 33-4. (راجع تكوين 17)}

\par 1 وفي السنة الخامسة من الأسبوع الرابع من هذا اليوبيل، [1979 صباحًا] في الشهر الثالث، في منتصف الشهر، احتفل أبرام بعيد باكورة حصاد الحبوب
\par 2 وأصعد على المذبح ذبائح جديدة باكورة الغلة للرب، عجلة ومعزة وخروفا على المذبح محرقة للرب. وقدم تقدمات ثمرها وسكيبها على المذبح مع اللبان.
\par 3 وظهر الرب لأبرام وقال له: أنا الله القدير. اثبت أمامي وكن كاملاً
\par 4 «وأجعل عهدي بيني وبينك، وأكثرك كثيرًا جدًا.»
\par 5 فسقط أبرام على وجهه، وتكلم الله معه وقال:
\par    
\par 6 هوذا حُكمي معك،  
\par     وتكون أبًا لأممٍ كثيرة
\par    
\par 7 ولا يُدعى اسمك بعد أبرام،  
\par     بل يكون اسمك من الآن فصاعدًا إلى الأبد إبراهيم  
لأني جعلتك أبا لأمم كثيرة.
\par    
\par 8 وسأجعلك عظيمًا جدًا،  
\par     وسأجعلك أممًا،  
ويخرج منك ملوك.
\par    
\par 9 وأقيم عهدي بيني وبينك، وبين نسلك من بعدك، في أجيالهم، عهدًا أبديًا، لأكون إلهًا لك ولنسلك من بعدك
\par 10 (وأعطي لك ولنسلك من بعدك) الأرض التي كنتَ فيها غريبًا، أرض كنعان، لتمتلكها إلى الأبد، وأكون لهم إلهًا
\par 11 وقال الرب لإبراهيم: «وأما أنت فاحفظ عهدي أنت ونسلك من بعدك، وختنوا كل ذكر منكم، وختنوا غرلاتكم، فيكون علامة عهد أبدي بيني وبينكم».
\par 12 «وفي اليوم الثامن تختنون كل ذكر في أجيالكم، مواليد البيت، أو الذي اشتريتموه بفضة من كل غريب اقتنيتموه وهو ليس من نسلكم.»
\par 13 «المولود في بيتك يُختتن، والذين اشتريتهم بفضة يُختتنون، ويكون عهدي في أجسادكم فريضة أبدية.»
\par 14 «وأما الذكر الأغلف الذي لا يختن في لحم غرلته ففي اليوم الثامن تُقطع تلك النفس من شعبها لأنها نكثت عهدي.»
\par 15 وقال الله لإبراهيم: «أما ساراي امرأتك فلا يُدعى اسمها بعد ساراي، بل سارة يكون اسمها».
\par 16 «وأباركها، وأعطيك منها ابنًا، وأباركه، فيكون أمة، ويخرج منه ملوك أمم.»
\par 17 فسقط إبراهيم على وجهه وفرح، وقال في قلبه: «هل يولد لابن مئة سنة ابن، وهل تلد سارة وهي ابنة تسعين سنة؟»
\par 18 فقال إبراهيم لله: ليت إسماعيل يعيش أمامك!
\par 19 وقال الله: «وسارة أيضًا ستلد لك ابنًا وتدعو اسمه إسحاق، وأقيم عهدي معه عهدًا أبديًا ولنسله من بعده».
\par 20 «وأما إسماعيل أيضًا فقد سمعت لك، وها أنا أباركه وأعظمه وأكثره كثيرًا جدًا، ويلد اثني عشر رئيسًا، وأجعله أمة عظيمة.»
\par 21 «ولكن عهدي أقيمه مع إسحاق الذي تلده لك سارة في هذه الأيام، في السنة القادمة.»
\par 22 فتوقف عن الكلام معه، وصعد الله عن إبراهيم.
\par 23 ففعل إبراهيم كما قال له الله، وأخذ إسماعيل ابنه، وجميع ولدان بيته الذين اشتراهم بفضته، كل ذكر في بيته، وختن لحم غرلتهم.
\par 24 وفي ذلك اليوم عينه ختن إبراهيم، وجميع رجال بيته (والمولودين في البيت)، وجميع الذين اشتراهم بفضة من أولاد الغريب، ختنوا معه
\par 25 هذه الشريعة لجميع الأجيال إلى الأبد، ولا ختان للأيام، ولا حذف ليوم واحد من الأيام الثمانية؛ لأنها فريضة أبدية، مقررة ومكتوبة على الألواح السماوية
\par 26 وكل من يولد ولم يُختتن لحم غرلته في اليوم الثامن، فهو ليس من أبناء العهد الذي قطعه الرب مع إبراهيم، بل من أبناء الهلاك. وليس عليه علامة أنه للرب، بل (مقدر له) أن يُهلك ويُقتلع من الأرض، ويُقتلع من الأرض، لأنه نكث عهد الرب إلهنا
\par 27 لأن جميع ملائكة الحضور وجميع ملائكة التقديس قد خُلقوا هكذا منذ يوم خلقهم، وقد قدس إسرائيل أمام ملائكة الحضور وملائكة التقديس، ليكونوا معه ومع ملائكته القديسين
\par 28 وأنتَ تُوصِي بَني إسرائيلَ، وليحفظوا علامةَ هذا العهدِ مدى أجيالِهم فريضةً أبديّةً، فلا يُقتلَعونَ من الأرضِ
\par 29 لأن الأمر قد وضع عهدًا، لكي يحفظوه إلى الأبد بين جميع بني إسرائيل
\par 30 أما إسماعيل وأبنائه وإخوته وعيسو، فلم يجعل الرب يقتربون منه، ولم يختارهم لأنهم أبناء إبراهيم، لأنه كان يعرفهم، بل اختار إسرائيل ليكونوا شعبه
\par 31 وقدسه، وجمعه من بين جميع بني البشر؛ لأن هناك أممًا كثيرة وشعوبًا كثيرة، وكلها له، وعلى الكل وضع أرواحًا في سلطان لإضلالهم عنه
\par 32 "ولكنه لم يعين على إسرائيل ملاكًا أو روحًا، لأنه هو وحده حاكمهم، وهو يحفظهم ويطلبهم على يد ملائكته وأرواحه، وعلى يد كل قواته، لكي يحفظهم ويباركهم، ويكونوا له ويكون لهم من الآن فصاعدًا إلى الأبد.
\par 33 والآن أُعلن لك أن بني إسرائيل لن يحفظوا هذه الشريعة، ولن يختنوا أبناءهم حسب كل هذه الشريعة؛ لأنهم في لحم ختانهم سيُغفلون هذا الختان لأبنائهم، وجميعهم، أبناء بليعار، سيتركون أبناءهم غير مختونين كما ولدوا
\par 34 ويكون غضب الرب عظيمًا على بني إسرائيل لأنهم تركوا عهده وزاغوا عن كلمته، وأثاروا وجدفوا، لأنهم لم يحفظوا حكم هذه الشريعة؛ لأنهم عاملوا أفرادهم كالأمم، حتى يُقتلعوا من الأرض. ولن يكون لهم بعد ذلك عفو أو غفران [بحيث يكون هناك غفران وصفح] عن كل خطيئة هذا الخطأ الأبدي

\chapter{16}

\par \textit{ظهور الملائكة لإبراهيم في الخليل ووعد إسحاق مرة أخرى، 1-4. تدمير سدوم وخلاص لوط، 5-9. إبراهيم في بئر سبع: ولادة إسحاق وختانه، الذي كان نسله نصيب الله، 10-19. تأسيس عيد المظال، 20-31. (راجع تكوين 18: 1، 10، 12؛ 19: 24، 29، 33-7؛ 20: 1، 4، 8؛ 21: 1-4.)}

\par 1 وفي هلال الشهر الرابع، ظهرنا لإبراهيم عند بلوطة ممرا، وتحدثنا معه، وبشرناه أن سارة امرأته سترزقه بابن
\par 2 فضحكت سارة، لأنها سمعت أننا تكلمنا مع إبراهيم بهذا الكلام، فوبخناها، فخافت وأنكرت أنها ضحكت بسبب الكلام
\par 3 وأخبرناها باسم ابنها كما هو مكتوب في الألواح السماوية، إسحاق،
\par 4 وإذا رجعنا إليها في أجل مسمى لَحَمِلَتْ بِغَلام
\par 5 وفي هذا الشهر أجرى الرب أحكامه على سدوم وعمورة وصبويم وكل منطقة الأردن، وأحرقهم بالنار والكبريت، وأهلكهم إلى هذا اليوم، كما أخبرتك بجميع أعمالهم، أنهم أشرار وخطاة جدًا، وأنهم ينجسون أنفسهم ويزنون في أجسادهم، ويصنعون نجاسة على الأرض
\par 6 وكذلك يُنفِّذ الله دينونة على الأماكن التي فعلوا فيها حسب نجاسة سدوم، مثل دينونة سدوم
\par 7 ولكننا خلصنا لوطًا، لأن الله ذكر إبراهيم وأخرجه من وسط الانقلاب
\par 8 ففعل هو وبناته خطية على الأرض لم تكن على الأرض منذ أيام آدم إلى زمانه، إذ اضطجع آدم مع بناته.
\par 9 وها هوذا قد أُمر ونُقش على جميع نسله، على الألواح السماوية، لإزالتهم واستئصالهم، وإجراء حكم عليهم مثل حكم سدوم، وعدم ترك أي نسل للإنسان على الأرض يوم الدينونة
\par 10 وفي ذلك الشهر انتقل إبراهيم من حبرون، وارتحل وأقام بين قادش وشور في جبال جرار
\par 11 وفي منتصف الشهر الخامس انتقل من هناك، وأقام عند بئر القسم
\par 12 وفي منتصف الشهر السادس افتقد الرب سارة وفعل لها كما قال، فحبلت
\par 13 فولدت ابنًا في الشهر الثالث، وفي منتصف الشهر، في الوقت الذي كلم الرب إبراهيم عنه، في عيد باكورة الحصاد، وُلد إسحاق
\par 14 وختن إبراهيم ابنه في اليوم الثامن. وكان أول من ختن حسب العهد المرسوم إلى الأبد
\par 15 وفي السنة السادسة من الأسبوع الرابع، أتينا إلى إبراهيم، إلى بئر القسم، وظهرنا له [كما قلنا لسارة أن نعود إليها، وأنها ستحبل بابن].
\par 16 ورجعنا في الشهر السابع، فوجدنا سارة حاملاً أمامنا، فباركناه، وأخبرناه بكل ما قُضي عليه، أنه لا يموت حتى يلد ستة بنين آخرين، ويراه قبل وفاته، وأن يُدعى اسمه ونسله بإسحاق
\par 17 وأن يكون جميع نسل أبنائه أمميين، ويُحسبون مع الأمم. وأما من أبناء إسحاق فيكون واحد منهم نسلًا مقدسًا، ولا يُحسب مع الأمم
\par 18 لأنه كان سيصبح نصيبًا للعلي، وكل نسله قد وقع في ملك الله، ليكون للرب شعبًا مقتنيًا فوق كل الأمم، ويكون له مملكة وكهنة وأمة مقدسة
\par 19 فذهبنا وأخبرنا سارة بكل ما قلنا له، ففرحتا كلتاهما فرحًا عظيمًا جدًا
\par 20 وبنى هناك مذبحًا للرب الذي أنقذه، والذي كان يُفرحه في أرض غربته، واحتفل بعيد فرح في هذا الشهر سبعة أيام، بالقرب من المذبح الذي بناه عند بئر القسم
\par 21 وبنى لنفسه ولخدمه مظلات في هذا العيد، وكان أول من احتفل بعيد المظال على الأرض
\par 22 وكان في تلك الأيام السبعة يأتي كل يوم بمحرقة للرب إلى المذبح، ثورين وكبشين وسبعة خراف وتيس واحد، ذبيحة خطية، ليُكفِّر بها عن نفسه وعن نسله
\par 23 كذبيحة شكر سبعة كباش وسبعة جداء وسبعة غنم وسبعة تيوس مع تقدماتها وسكيبها وأوقد كل شحمها على المذبح ذبيحة مختارة للرب رائحة سرور.
\par 24 وكان يحرق صباحًا ومساءً العطور: اللبان، والجلبانوم، والمسك، والناردين، والمر، والتوابل، والملابس. جميع هذه السبعة قدمها مسحوقة، ممزوجة معًا بأجزاء متساوية ونقيّة
\par 25 واحتفل بهذا العيد سبعة أيام، فرحًا من كل قلبه ومن كل نفسه، هو وجميع الذين في بيته، ولم يكن معه غريب ولا أحد غير مختون
\par 26 وبارك خالقه الذي خلقه في جيله، لأنه خلقه حسب مسرته، إذ علم وأدرك أنه منه يخرج غرس البر إلى الأجيال الأبدية، ومنه زرع مقدس، ليكون مثل الذي خلق كل شيء
\par 27 وبارك وفرح، ودعا اسم هذا العيد عيد الرب، فرحًا مرضيًا لدى الله العلي
\par 28 وباركناه إلى الأبد، وجميع نسله من بعده في جميع أجيال الأرض، لأنه احتفل بهذا العيد في وقته، حسب شهادة الألواح السماوية
\par 29 لهذا السبب، كُتب على الألواح السماوية فيما يتعلق بإسرائيل، أن يحتفلوا بعيد المظال سبعة أيام بفرح، في الشهر السابع، مقبولاً أمام الرب - فريضة أبدية في أجيالهم كل سنة
\par 30 وليس لهذا حدٌّ زمنيّ؛ لأنه مُقدَّرٌ إلى الأبد على إسرائيل أن يحتفلوا به ويسكنوا في مظال، ويضعوا أكاليل الزهور على رؤوسهم، ويأخذوا أغصانًا مورقةً وصفصافًا من الوادي
\par 31 فأخذ إبراهيم سعف النخيل، وثمر أشجار طيبة، وكان كل يوم يطوف حول المذبح بأغصانها سبع مرات في الصباح، ويسبح إلهه ويحمده على كل شيء بفرح

\chapter{17}

\par \textit{طرد هاجر وإسماعيل، 1-14. يقترح مستيما أن يطلب الله من إبراهيم التضحية بإسحاق لاختبار محبته وطاعته: ابتلاءات إبراهيم العشر، 15-18. (راجع تكوين 21: 8-21)}

\par 1 وفي السنة الأولى من الأسبوع الخامس فُطم إسحاق في هذا اليوبيل، [1982 صباحًا]، وأقام إبراهيم وليمة عظيمة في الشهر الثالث، يوم فطام ابنه إسحاق
\par 2 وكان إسماعيل بن هاجر المصرية أمام إبراهيم أبيه في مكانه، ففرح إبراهيم وبارك الله لأنه رأى بنيه ولم يمت بلا أولاد
\par 3 فتذكر الكلام الذي قاله له يوم فارقه لوط، وفرح لأن الرب أعطاه بذرًا على الأرض ليرث الأرض، وبارك بكل فمه خالق كل الأشياء
\par 4 ورأت سارة إسماعيل يلعب ويرقص، وإبراهيم يفرح فرحاً عظيماً، فغارت من إسماعيل وقالت لإبراهيم: اطرد هذه الجارية وابنها، لأن ابن هذه الجارية لا يرث مع ابني إسحق.
\par 5 فَقَدْ عَظُمَ الأَمْرُ فِي عَيْنَيِ إِبْرَاهِيمِ مِنْ أَجْلِ أَمَتِهِ وَمِنْ أَجْلِ ابْنِهِ أَنْ يَطْرَدَهُمَا مِنْهُ
\par 6 وقال الله لإبراهيم: «لا يقبح في عينيك من أجل الطفل ومن أجل الجارية. في كل ما قالت لك سارة، اسمع كلامها واعمل به، لأنه بإسحاق يُدعى اسمك ونسلك».
\par 7 «وأما ابن هذه الجارية فسأجعله أمة عظيمة، لأنه من نسلك.»
\par 8 فبكر إبراهيم في الصباح، وأخذ خبزًا وقربة ماء، ووضعهما على كتفي هاجر والطفل، وصرفها
\par 9 فانطلقت وتاهت في برية بئر سبع، فنفذ الماء من القربة، وعطش الصبي، ولم يستطع المشي فسقط
\par 10 فأخذته أمه وطرحته تحت شجرة زيتون، وذهبت وجلست مقابله على بُعد رمية قوس، لأنها قالت: لا أرى موت ابني، وبينما هي جالسة كانت تبكي
\par 11 فقال لها ملاك الله، أحد القديسين: «لماذا تبكين يا هاجر؟ قومي وخذي الطفل وامسكيه بيدك، لأن الله قد سمع صوتك ورأى الطفل».
\par 12 ففتحت عينيها، فأبصرت بئر ماء، فذهبت وملأت قربتها ماءً وسقت ولدها، ثم قامت وذهبت نحو برية فاران
\par 13 فكبر الصبي وأصبح راميًا، وكان الله معه، وأخذت له أمه زوجة من بين بنات مصر
\par 14 فولدت له ابنًا، فدعا اسمه نبايوت، لأنها قالت: «الرب قريب مني حين دعوته».
\par 15 وحدث في الأسبوع السابع، في السنة الأولى منه، [2003 صباحًا] في الشهر الأول من هذا اليوبيل، في الثاني عشر من هذا الشهر، أن أصواتًا في السماء عن إبراهيم أنه كان أمينًا في كل ما قاله له، وأنه أحب الرب، وأنه في كل ضيقة كان أمينًا
\par 16 فجاء الأمير مستيما وقال أمام الله: هوذا إبراهيم يحب إسحاق ابنه، وهو يُسر به فوق كل شيء آخر. اطلب منه أن يقدمه محرقة على المذبح، وسترى إن كان سيفعل هذا الأمر، وستعرف إن كان أمينًا في كل ما تختبره فيه
\par 17 وعرف الرب أن إبراهيم كان أمينًا في جميع ضيقاته؛ لأنه امتحنه في أرضه وفي المجاعة، وامتحنه بثروات الملوك، ثم امتحنه مرة أخرى في امرأته عندما انتزعت منه وفي الختان؛ وامتحنه في إسماعيل وهاجر أمته عندما صرفهما
\par 18 وفي كل ما امتحنه به وجد أميناً، ولم تكن نفسه متسرعة، ولم يتباطأ في العمل، لأنه كان أميناً ومحباً للرب.

\chapter{18}

\par \textit{ذبيحة إسحاق: إحراج مستيما، 1-13. مباركة إبراهيم مرة أخرى: العودة إلى بئر سبع 14-19. (راجع تكوين 22: 1-19.)}

\par 1 فقال له الله: «إبراهيم، إبراهيم». فقال: «ها أنا ذا».
\par 2 فقال: «خذ ابنك الحبيب الذي تحبه إسحاق، واذهب إلى المرتفعات، وقدمه على أحد الجبال الذي أشير لك إليه».
\par 3 فبكر في الصباح وشد على حماره، وأخذ غلاميه معه، وإسحاق ابنه، وشقّ حطب المحرقة، وذهب إلى المكان في اليوم الثالث، فنظر المكان من بعيد
\par 4 فجاء إلى بئر ماء، فقال لغلاميه: «اجلسوا أنتم هنا مع الحمار، وأنا والغلام نذهب (إلى هناك)، وعندما ننتهي من السجود نعود إليكما».
\par 5 فأخذ حطب المحرقة ووضعه على إسحاق ابنه، وأخذ في يده النار والسكين، وذهبا كلاهما معًا إلى ذلك المكان
\par 6 فقال إسحاق لأبيه: يا أبي، فقال: ها أنا ذا يا ابني. فقال له: هوذا النار والسكين والحطب، ولكن أين الخروف للمحرقة يا أبي؟
\par 7 فقال: «الله يرى له خروفًا للمحرقة يا ابني». وتقدم إلى موضع جبل الله
\par 8 فبنى مذبحًا، ووضع الحطب على المذبح، وربط إسحاق ابنه، ووضعه على الحطب الذي على المذبح، ومد يده ليأخذ السكين ليذبح إسحاق ابنه
\par 9 ووقفتُ أمامه، وأمام الأمير مستيما، فقال الرب: «لا تأمره أن لا يمد يده على الغلام، ولا أن يفعل به شيئًا، لأني قد أظهرتُ أنه يخاف الرب».
\par 10 وناديته من السماء وقلت له: يا إبراهيم، يا إبراهيم. فخاف وقال: ها أنا ذا
\par 11 فقلت له: لا تمد يدك إلى الغلام ولا تفعل به شيئًا، لأني الآن قد أظهرت أنك تخاف الرب ولم تمسك ابنك البكر عني
\par 12 فخجل الأمير مستيما، فرفع إبراهيم عينيه ونظر، وإذا كبش ممسوك ... بقرنيه، فذهب إبراهيم وأخذ الكبش وأصعده محرقة عوضًا عن ابنه
\par 13 ودعا إبراهيم ذلك المكان «الرب قد رأى»، حتى أنه يُقال في الجبل «الرب قد رأى»، أي جبل صهيون
\par 14 ودعا الرب إبراهيم باسمه ثانيةً من السماء، كما أظهرنا لنكلمَهُ باسم الرب
\par 15 فقال: «بذاتي أقسمت، يقول الرب،»
\par    
\par    لأنك فعلت هذا الأمر،  
ولم تحجب ابنك الحبيب عني،  
\par    أني سأباركك بالبركة،
\par    
\par      وبالتكثير سأكثر نسلك  
\par    كنجوم السماء، وكالرمل الذي على شاطئ البحر.
\par    
\par    ويرث نسلك مدن أعدائه،  
\par    
\par 16 ويتبارك في نسلك جميع أمم الأرض.
\par    
\par     لأنك أطعت صوتي،  
\par    ولقد أظهرت للجميع أنك أمين لي في كل ما قلته لك.
\par    
\par    اذهب بسلام.'
\par    
\par 17 وذهب إبراهيم إلى غلاميه، فقاموا وذهبوا معًا إلى بئر سبع، وكان إبراهيم [2010 صباحًا] ساكنًا عند بئر القسم
\par 18 وكان يحتفل بهذا العيد كل سنة سبعة أيام بفرح، ودعاه عيد الرب، نسبة إلى الأيام السبعة التي ذهب فيها ورجع بسلام
\par 19 وبناءً على ذلك، فقد قُدِّرَ وكُتِبَ على الألواح السماوية فيما يتعلق بإسرائيل ونسلها أن يحتفلوا بهذا العيد سبعة أيام بفرح العيد

\chapter{19}

\par \textit{عودة إبراهيم إلى الخليل. موت سارة ودفنها، 1-9. زواج إسحاق والزواج الثاني لإبراهيم. ولادة عيسو ويعقوب، 10-14. يوصي إبراهيم يعقوب بربيكا ويباركه، 15-31. (راجع تكوين 23: 1-4، 11-16؛ 24: 15؛ 25: 1-2، 25-7؛ 13: 16.)}

\par 1 وفي السنة الأولى من الأسبوع الأول، في اليوبيل الثاني والأربعين، رجع إبراهيم وسكن مقابل حبرون، وهي قرية أربع، أسبوعين من السنين
\par 2 وفي السنة الأولى من الأسبوع الثالث من هذا اليوبيل، تمت أيام حياة سارة، وماتت في حبرون
\par 3 فذهب إبراهيم ليندبها ويدفنها، وجربناه هل تصبر روحه ولا يغتاظ من كلامه، فوجده صابرا على ذلك ولم ينزعج
\par 4 لأنه بصبر روحه، تحدث مع أبناء حث، بقصد أن يمنحوه مكانًا يدفن فيه موتاه
\par 5 فأعطاه الرب نعمة أمام كل من رآه، وطلب بلطف إلى بني حث، فأعطوه أرض المغارة المزدوجة التي مقابل ممرا، التي هي حبرون، بأربعمائة قطعة من الفضة
\par 6 فتوسلوا إليه قائلين: نعطيك إياه مجانًا، فلم يشأ أن يأخذه من أيديهم مجانًا، لأنه دفع ثمن المكان، الفضة كاملة، وسجد لهم مرتين، وبعد ذلك دفن ميته في المغارة المزدوجة
\par 7 وكانت كل أيام حياة سارة مائة وسبعا وعشرين سنة يوبيلين وأربعة أسابيع وسنة واحدة. هذه هي أيام سني حياة سارة.
\par 8 هذه هي التجربة العاشرة التي جُرِّب بها إبراهيم، فوجده أمينًا صبورًا في الروح
\par 9 ولم ينطق بكلمة واحدة عن الشائعة في الأرض أن الله قال إنه سيعطيها له ولنسله من بعده، وطلب مكانًا هناك لدفن موتاه؛ لأنه وُجد أمينًا، وسُجِّل على الألواح السماوية كصديق لله
\par 10 وفي السنة الرابعة منها اتخذ زوجة لابنه إسحاق، واسمها رفقة [2020 AM] [ابنة بتوئيل بن ناحور أخي إبراهيم] أخت لابان وابنة بتوئيل. وكان بتوئيل ابن ملكة، امرأة ناحور أخي إبراهيم
\par 11 واتخذ إبراهيم لنفسه زوجة ثالثة اسمها قطورة من بين بنات عبيده، لأن هاجر كانت قد ماتت قبل سارة. فولدت له ستة بنين: زمرام، ويقشان، ومدان، ومديان، ويشباق، وشوح، في أسبوعين السنين
\par 12 وفي الأسبوع السادس، في السنة الثانية منه، ولدت رفقة لإسحاق ولدين، يعقوب وعيسو،
\par 13 و [2046 صباحًا] كان يعقوب رجلاً أملسًا ومستقيمًا، وكان عيسو شرسًا، رجل حقل، وأشعر، وكان يعقوب يسكن في الخيام
\par 14 وكبر الغلامان، وتعلم يعقوب الكتابة، أما عيسو فلم يتعلم، لأنه كان إنسان برية وصيادًا، وتعلم الحرب، وكانت كل أعماله شرسة
\par 15 وأحب إبراهيم يعقوب، وأما إسحاق فقد أحب عيسو.
\par 16 ونظر إبراهيم أعمال عيسو، فعلم أنه في يعقوب يدعى اسمه ونسله. فدعا رفقة وأوصى بشأن يعقوب، لأنه علم أنها أيضاً تحب يعقوب أكثر من عيسو.
\par 17 فقال لها:
\par    
\par    يا ابنتي، احرسي ابني يعقوب،  
\par    لأنه يكون مكاني على الأرض،  
\par    ولبركة في وسط بني البشر،  
\par    ولمجد كل نسل سام.
\par    
\par 18 لأني أعلم أن الرب سيختاره ليكون له شعبًا مقتنيًا فوق جميع الشعوب الذين على وجه الأرض
\par 19 وهوذا إسحاق ابني يحب عيسو أكثر من يعقوب، ولكني أرى أنك تحب يعقوب حبًا حقيقيًا
\par    
\par 20 زد من لطفك به،  
\par     ولتكن عيناك عليه في حب؛  
لأنه يكون لنا بركة على الأرض من الآن إلى أجيال الأرض كلها.
\par    
\par 21 لتتشدد يداك  
\par     وليفرح قلبك بابنك يعقوب؛  
لأني أحببته أكثر من جميع أبنائي.
\par    
\par     سيكون مباركًا إلى الأبد،  
\par    ويملأ زرعه كل الأرض.
\par    
\par 22 إن استطاع الإنسان أن يحصي رمل الأرض،  
\par     فإن نسله أيضًا سيُحصى
\par    
\par 23 وجميع البركات التي باركني بها الرب ونسلي تكون ليعقوب ونسله إلى الأبد
\par 24 وفي نسله يتبارك اسمي واسم آبائي: سام، ونواب، وحنوك، ومهللئيل، وأنوش، وشيث، وآدم
\par 25 وهؤلاء يخدمون
\par    
\par     لوضع أسس السماء،  
\par    ولتقوية الأرض،  
ولتجديد جميع الأنوار التي في الجلد.
\par    
\par 26 ودعا يعقوب أمام رفقة أمه، وقبله وباركه، وقال:
\par 27 «يا يعقوب، ابني الحبيب الذي تحبه نفسي، يباركك الله من فوق الجلد، ويمنحك كل البركات التي بارك بها آدم، وحنوك، ونوح، وسام؛ وكل ما أخبرني به، وكل ما وعدني به، ليجعله يلتصق بك وبنسلك إلى الأبد، كأيام السماء على الأرض.»
\par 28 «ولن تتسلط عليك أرواح ماستيما ولا على نسلك لتبعدك عن الرب الذي هو إلهك من الآن وإلى الأبد.»
\par 29 «وَلْيَكُنِ الرَّبُّ الإِلهُ أَبًا لَكَ وَلِلاَبْنِ الْبِكْرِ وَلِلشَّعْبِ إِلَى الأَبَدِ.»
\par 30 «اذهب بسلام يا ابني». فخرجا كلاهما معًا من عند إبراهيم
\par 31 وأحبت رفقة يعقوب من كل قلبها ومن كل نفسها أكثر من عيسو كثيرًا. أما إسحاق فأحب عيسو أكثر من يعقوب كثيرًا

\chapter{20}

\par \textit{يُوصي إبراهيم أبناءه وأبناء أبنائه بالعمل الصالح، والمحافظة على الختان، والامتناع عن النجاسة وعبادة الأصنام، 1-10. يُصرفهم بالهدايا، 11. مساكن الإسماعيليين وأبناء قطورة، 12-13. (راجع تكوين 25: 5-6.)}

\par 1 وفي اليوبيل الثاني والأربعين، في السنة الأولى من الأسبوع السابع، دعا إبراهيم إسماعيل، [2052 (2045؟) AM] وأبنائه الاثني عشر، وإسحاق وابنيه، وأبناء قطورة الستة وأبنائهم
\par 2 وأمرهم أن يحفظوا طريق الرب، وأن يعملوا البر، ويحبوا كل واحد قريبه، ويعملوا بهذه الطريقة بين جميع الناس، وأن يسلكوا كل واحد منهم هكذا أمامهم ليعملوا الحق والعدل على الأرض
\par 3 أن يختنوا بنيهم حسب العهد الذي قطعه معهم، ولا يحيدوا يمنة ولا يسرة عن جميع الطرق التي أوصانا بها الرب، وأن نحفظ أنفسنا من كل زنا ونجاسة، [ونبذ من بيننا كل زنا ونجاسة].
\par 4 وإذا زنت امرأة أو عذراء بينكم فأحرقوها بالنار. ولا يزنوا معها حسب عيونهم وقلوبهم. ولا يتخذوا لأنفسهم نساءً من بنات كنعان. لأنه سيُقتلع نسل كنعان من الأرض
\par 5 وأخبرهم عن دينونة العمالقة، ودينونة أهل سدوم، كيف حُكم عليهم بسبب شرهم، وماتوا بسبب زناهم، ونجاستهم، وفسادهم المتبادل بسبب الزنا
\par    
\par 6 «واحفظوا أنفسكم من كل زنا ونجاسة،  
\par    ومن كل دنس الخطيئة،
\par    
\par     لئلا تجعلوا اسمنا لعنة،  
\par     وحياتك كلها هسهسة،
\par    
\par    وجميع أبنائك يُبادون بالسيف،  
\par     وتصبحون ملعونين مثل سدوم،  
وكل بقيتك كأبناء عمورة.
\par    
\par 7 أتوسل إليكم يا أبنائي أن تحبوا إله السماء  
\par     والتزموا بجميع وصاياه
\par    
\par    ولا تسيروا وراء أصنامهم ونجاساتهم،  
\par    
\par 8 ولا تصنعوا لكم آلهة مسبوكة أو منحوتة
\par    
\par     فإنها باطل،  
\par    وليس فيهم روح؛
\par    
\par    لأنها من عمل أيدي (الرجال)،  
وكل من يثق بهم، لا يثق بشيء.
\par    
\par 9 لا تعبدوها ولا تسجدوا لها،  
\par     بل اعبدوا الله العلي واعبدوه دائمًا  
\par    ورجاء وجهه دائمًا،  
\par    وأعملوا الاستقامة والعدل أمامه،
\par    
\par    لكي يرضى عنكم ويرحمكم،  
\par    وأرسل المطر عليك صباحًا ومساءً،
\par    
وبارك جميع أعمالك التي عملتها على الأرض،  
\par    وبارك خبزك وماءك،
\par    
\par    وبارك ثمرة بطنك وثمرة أرضك،  
\par    وقطعان ماشيتك وقطعان غنمك.
\par    
\par 10 وتكونون نعمة في الأرض،  
\par    وتشتاق إليك كل أمم الأرض،
\par    
\par وبارك أبناءك باسمي،  
لكي يكونوا مباركين كما أنا.
\par    
\par 11 وأعطى إسماعيل وبنيه وبني قطورة عطايا، وصرفها عن إسحاق ابنه، وأعطى كل شيء لإسحاق ابنه
\par 12 وذهب إسماعيل وبنوه وبنو قطورة وأبناؤهم معًا وسكنوا من فاران إلى مدخل بابل في كل الأرض التي نحو المشرق تجاه البرية
\par 13 فاختلط هؤلاء ببعضهم، فدُعي اسمهم عربًا وإسماعيليين

\chapter{21}

\par \textit{كلمات إبراهيم الأخيرة لإسحاق بشأن عبادة الأصنام، وأكل الدم، وتقديم الذبائح المختلفة، واستخدام الملح، 1-11. وكذلك بشأن الأخشاب المستخدمة في الذبائح، وواجب الغسل قبل الذبائح، وتغطية الدم، وما إلى ذلك، 12-25.}

\par 1 وفي السنة السادسة من الأسبوع السابع من هذا اليوبيل، دعا إبراهيم إسحاق ابنه، وأوصاه قائلاً: «لقد شخت، ولا أعرف يوم وفاتي، وقد شبعت أيامي».
\par 2 «وها أنا ابن مئة وخمسة وسبعين سنة، وقد ذكرت الرب طوال أيام حياتي، وحرصت بكل قلبي على أن أعمل مشيئته، وأن أسلك مستقيمًا في كل طرقه.»
\par 3 «كرهت نفسي الأصنام، واحتقرت من تعبّدها، وأعطيت قلبي وروحي» لأحرص على أن أعمل مشيئة خالقي
\par 4 «لأنه هو الإله الحي، وهو قدوس وأمين، وهو بار فوق الجميع، وليس عنده قبول وجوه (البشر) ولا قبول عطايا. لأن الله بار، وينفذ الدينونة على كل من يتعدى على وصاياه ويحتقر عهده.»
\par 5 «وأنت يا ابني، احفظ وصاياه وأحكامه، ولا تسلك وراء الأرجاس والتماثيل المنحوتة والمسبوكات.»
\par 6 «ولا تأكلوا دمًا قط من بهائم ولا بهائم، ولا من طير يطير في السماء».
\par 7 «وإذا ذبحت ذبيحة سلامة مقبولة، فاذبحها واسكب دمها على المذبح، وكل شحم التقدمة على المذبح مع دقيق وتقدمة اللحم ملتوتة بزيت مع سكيبه. قدمها جميعًا معًا على مذبح المحرقة. إنها رائحة سرور أمام الرب.»
\par 8 «وتقرب شحم ذبيحة الشكر على النار التي على المذبح، والشحم الذي على البطن، وكل شحم الأحشاء والكليتين، وكل الشحم الذي عليهما، وعن الخاصرتين والكبد تزيله مع الكليتين».
\par 9 "وتقدم كل هذه رائحة سرور مقبولة أمام الرب مع تقدمتها وسكيبها رائحة سرور خبز تقدمة للرب."
\par 10 «وكلوا لحمه في ذلك اليوم وفي اليوم الثاني، ولا تغرب عليه الشمس في اليوم الثاني حتى يؤكل، ولا يبق شيء لليوم الثالث؛ لأنه غير مقبول [لأنه غير معتمد] ولا يؤكل بعد، وكل من يأكل منه يجلب على نفسه إثمًا؛ لأني هكذا وجدته مكتوبًا في كتب آبائي، وفي أقوال أخنوخ، وفي أقوال نوح.»
\par 11 «وعلى جميع قرابينك تنضح ملحًا، ولا ينقص ملح العهد في جميع قرابينك أمام الرب.»
\par 12 «وأما خشب الذبائح، فاحذر أن تُحضر خشبًا (آخر) للمذبح بالإضافة إلى هذه: السرو، الغار، اللوز، التنوب، الصنوبر، الأرز، الزان، التين، الزيتون، المر، الغار، الأسبالثوس.»
\par 13 «ومن هذه الأنواع من الأخشاب، ضع على المذبح تحت الذبيحة ما تم اختباره من حيث مظهره، ولا تضع (عليه) خشبًا مشقوقًا أو داكنًا، بل خشبًا صلبًا ونظيفًا بلا عيب، سليمًا وجديدًا؛ ولا تضع (عليه) خشبًا قديمًا، [لأن رائحته قد ذهبت] لأنه لم يعد فيه رائحة كما كان من قبل.»
\par 14 «لا تضعوا غير هذه الأنواع من الخشب على المذبح، لأن رائحته تنتشر، ولا ترتفع رائحة عطره إلى السماء.»
\par 15 «احفظ هذه الوصية واعمل بها يا ابني، لكي تكون مستقيمًا في جميع أعمالك.»
\par 16 «وكن طاهرًا في جميع الأوقات، واغتسل بالماء قبل أن تتقدم للتقدمة على المذبح، واغسل يديك وقدميك قبل أن تقترب من المذبح؛ وعندما تنتهي من الذبح، اغسل يديك وقدميك مرة أخرى.»
\par 17 «ولا يظهر دم عليك ولا على ثيابك. احذر يا بني من الدم، احذر بشدة، غطه بالتراب.»
\par 18 «ولا تأكلوا دمًا، فهو النفس؛ لا تأكلوا دمًا على الإطلاق».
\par 19 «ولا تقبلوا قرابين من أجل دم الإنسان، لئلا يُسفك بلا عقاب، بلا دينونة؛ لأن الدم المسفوك هو الذي يجعل الأرض تُخطئ، والأرض لا يمكن تطهيرها من دم الإنسان إلا بدم سافكه.»
\par 20 «ولا تقبلوا هدية أو هبة من أجل دم إنسان: دم بدم، لكي تُقبلوا أمام الرب الإله العلي؛ لأنه هو حامي الخير، ولكي تُحفظوا من كل شر، ولكي يُخلصكم من كل نوع من الموت.»
\par    
\par 21 أرى يا بني،  
\par    أن جميع أعمال بني البشر هي خطيئة وشر،  
\par    وكل أعمالهم نجاسة ورجس ونجاسة،  
\par    ولا بر عندهم.
\par    
\par 22 احذر أن تسلك في طرقهم  
\par    وامش في مساراتهم،  
\par    وتخطئ خطيئة للموت أمام الله العلي.
\par    
\par     وإلا فإنه سيخفي وجهه عنك  
\par    و] يعيدك إلى أيدي معصيتك،  
\par    ويستأصلك من الأرض ونسلك كذلك من تحت السماء،  
ويبيد اسمك ونسلك من كل الأرض.
\par    
\par 23 ابتعدوا عن جميع أعمالهم وعن جميع نجاساتهم،  
\par     واحفظوا حكم الله العلي،  
وافعل مشيئته وكن مستقيمًا في كل شيء.
\par    
\par 24 ويباركك في جميع أعمالك،  
\par     ويُنبت منك غرسة بر في كل الأرض، طوال أجيال الأرض،  
ولا يُنسى اسمي واسمك تحت السماء إلى الأبد.
\par    
\par 25 اذهب يا ابني بسلام.  
\par     ليقويك الله العلي، إلهي وإلهك، على فعل مشيئته،  
ويبارك كل نسلك وبقية نسلك إلى الأجيال إلى الأبد بكل البركات الصالحة.  
لكي تكون بركة على كل الأرض.
\par    
\par 26 فخرج منه فرحًا.

\chapter{22}

\par \textit{احتفل إسحاق وإسماعيل ويعقوب بعيد البواكير في بئر سبع مع إبراهيم، ١-٥. صلاة إبراهيم، ٦-٩. كلمات إبراهيم الأخيرة ليعقوب وبركاته، ١٠-٣٠.}

\par 1 وحدث في الأسبوع الأول من اليوبيل الرابع والأربعين، في السنة الثانية، أي السنة التي مات فيها إبراهيم، أن إسحاق وإسماعيل جاءا من بئر القسم ليحتفلا بعيد الأسابيع - أي عيد باكورة الحصاد - إلى إبراهيم أبيهما، ففرح إبراهيم لأن ابنيه قد جاءا
\par 2 لأنه كان لإسحاق ممتلكات كثيرة في بئر سبع، وكان إسحاق معتادا على الذهاب ورؤية ممتلكاته والرجوع إلى أبيه
\par 3 وفي تلك الأيام جاء إسماعيل ليرى أبيه، فاجتمعا كلاهما، فقدّم إسحاق ذبيحة محرقة، وقدمها على مذبح أبيه الذي صنعه في حبرون
\par 4 وقدم ذبيحة شكر وصنع وليمة فرح أمام إسماعيل أخيه. وصنعت رفقة أقراصًا جديدة من السنبلة الجديدة، وأعطتها ليعقوب ابنها ليأخذها لإبراهيم أبيه من باكورة الأرض، ليأكلها ويبارك خالق كل شيء قبل وفاته
\par 5 وأرسل إسحق أيضاً بيد يعقوب إلى إبراهيم ذبيحة شكر عظيمة لكي يأكل ويشرب.
\par 6 فأكل وشرب وبارك الله العلي،
\par    
\par    الذي خلق السماء والأرض،  
\par    الذي صنع كل دهون الأرض،  
\par    وأعطوها لأبناء البشر  
لكي يأكلوا ويشربوا ويباركوا خالقهم.
\par    
\par 7 «والآن أحمدك يا ​​إلهي لأنك أريتني هذا اليوم. ها أنا ذا عمري مئة وثلاثة وخمسة وسبعون سنة، شيخ وشبعان أيامًا، وكانت لي كل أيامي سلامًا.»
\par 8 «لم يغلبني سيف العدو في كل ما أعطيتني وأولادي كل أيام حياتي إلى هذا اليوم.»
\par 9 «إلهي، رحمتك وسلامك على عبدك وعلى نسل أبنائه، ليكونوا لك أمة مختارة وميراثًا من بين جميع أمم الأرض من الآن فصاعدًا إلى كل أيام أجيال الأرض، إلى كل الدهور.»
\par 10 ثم دعا يعقوب وقال: «يا ابني يعقوب، يباركك إله الكل ويقويك لتصنع البر وتفعل مشيئته أمامه، ويختارك ونسلك لتكون له شعبًا ميراثًا حسب مشيئته إلى الأبد».
\par 11 «وأنت يا ابني يعقوب، تقدم وقبلني». فتقدم وقبله، وقال:
\par    
\par     'مبارك ابني يعقوب'  
وجميع أبناء الله العلي إلى كل الدهور.
\par    
\par     فليمنحك الله بذرة البر؛  
\par    ويُقَدِّسُ بَعْضًا مِنْ أَبْنِكَ فِي وَسَطِ كُلِّ الأَرْضِ.
\par    
\par    لتخدمك الأمم،  
وكل الأمم يسجدون أمام نسلك.
\par    
\par 12 كن قويًا في حضور الرجال،  
\par     وتسلط على كل نسل شيث
\par    
\par    حينئذٍ تتبرر طرقك وطرق أبنائك،  
\par    لكي يكونوا أمة مقدسة.
\par    
\par 13 الله العلي العظيم يمنحك كل البركات  
\par     التي باركني بها
\par    
\par    والذي بارك به نوحا وآدم؛  
\par    ليستريحوا على رأس نسلك المقدس من جيل إلى جيل إلى الأبد.
\par    
\par 14 ويطهرك من كل إثم ودنس،  
\par     لكي تُغفر لك جميع ذنوبك التي ارتكبتها بجهل
\par    
\par     ويقويك،  
\par    وباركك.  
\par    وترث الأرض كلها،
\par    
\par 15 ويجدد عهده معك  
لكي تكون له أمة ميراثًا إلى الأبد،  
"وأن يكون لك ولنسلك إلها بالحق والبر كل أيام الأرض."
\par    
\par 16 وأنت يا ابني يعقوب، تذكر كلامي،  
\par     واحفظ وصايا إبراهيم أبيك:
\par    
\par    انفصل عن الأمم،  
\par    ولا تأكلوا معهم:
\par    
\par    ولا تعملوا حسب أعمالهم،  
\par    ولا تكونوا لهم شركاء؛
\par    
\par    لأن أعمالهم نجسة،  
\par    وكل طرقهم نجاسة ورجس ونجاسة.
\par    
\par 17 يقدمون ذبائحهم للموتى  
\par     ويعبدون الأرواح الشريرة،
\par    
\par    ويأكلون على القبور،  
وكل أعمالهم باطل ولا شيء.
\par    
\par 18 ليس لديهم قلب ليفهموا  
\par     ولا ترى أعينهم ما هي أعمالهم،
\par    
\par    وكيف يخطئون حين يقولون لقطعة خشب: أنت إلهي،  
\par    وإلى الحجر: أنت سيدي وأنت منقذي.  
\par    [وليس لهم قلب.]
\par    
\par 19 وأما أنت يا ابني يعقوب،  
\par     فليعينك الله العلي  
\par    ويباركك إله السماء  
\par    وأخرجك من نجاساتهم ومن كل ضلالهم.
\par    
\par 20 احذر يا ابني يعقوب من أن تأخذ زوجة من أي نسل من بنات كنعان؛
\par    
لأن كل زرعه سيُقتلع من الأرض.
\par    
\par 21 لأنه بسبب معصية حام، أخطأ كنعان،  
\par     ويُباد كل نسله وكل ما بقي منه عن الأرض،  
ولا ينجو منه يوم القيامة أحد من بعده.
\par    
\par 22 وأما جميع عابدي الأصنام والمدنسين  
\par     (ب) فلا رجاء لهم في أرض الأحياء؛  
(ج) ولا يكون لهم ذكر في الأرض.  
\par    (ج) لأنهم سيهبطون إلى الهاوية،  
\par    (د) وإلى موضع الدينونة يذهبون،
\par    
\par    كما أُخِذَ بَنُو سَدُومَ مِنَ الأَرْضِ  
\par     فيُزال كل الذين يعبدون الأصنام.
\par    
\par 23 لا تخف يا ابني يعقوب،  
\par    ولا ترتعب يا ابن إبراهيم.
\par    
\par    حفظك الله العلي من الهلاك،  
\par    ومن جميع سبل الضلال فلينقذك.
\par    
\par 24 هذا البيت بنيتُ لنفسي لأضع اسمي عليه في الأرض. [يُعطى لك ولنسلك إلى الأبد]، وسيُدعى بيت إبراهيم. يُعطى لك ولنسلك إلى الأبد. لأنك أنت ستبني بيتي وتثبت اسمي أمام الله إلى الأبد. سيثبت نسلك واسمك طوال أجيال الأرض
\par    
\par 25 فتوقف عن أمره وبركته.
\par 26 واضطجع الاثنان على فراش واحد، ونام يعقوب في حضن إبراهيم أبيه، فقبله سبع مرات، ففرحت عليه عاطفته وقلبه.
\par 27 وباركه من كل قلبه وقال: «الله العلي، إله الكل، وخالق الكل، الذي أخرجني من أور الكلدانيين ليعطيني هذه الأرض لأرثها إلى الأبد، ولكي أقيم نسلًا مقدسًا - مبارك هو العلي إلى الأبد».
\par 28 وبارك يعقوب وقال: «يا ابني الذي أفرح به بكل قلبي ومحبتي، لترتفع نعمتك ورحمتك عليه وعلى نسله إلى الأبد».
\par 29 «ولا تتركه، ولا تحتقره من الآن إلى أيام الأبد، ولتفتح عيناك عليه وعلى نسله، لكي تحفظه وتباركه وتقدسه أمة لميراثك».
\par 30 «وباركه بكل بركاتك من الآن فصاعدًا إلى كل أيام الأبد، وجدد عهدك ونعمتك معه ومع نسله حسب كل مسرتك لجميع أجيال الأرض.»

\chapter{23}

\par \textit{موت إبراهيم ودفنه، 1-8 (راجع تكوين 25: 7-10). تناقص السنين وتزايد فساد البشرية: الويلات المسيحانية: صراع عالمي: انتفاضة المؤمنين لاستعادة غير المؤمنين: غزو إسرائيل من قبل خطاة الأمم، 11-25. دراسة متجددة للشريعة وتجديد البشرية: الملكوت المسيحاني: الخلود المبارك للأبرار، 26-31.}

\par 1 ووضع إصبعي يعقوب على عينيه، وبارك إله الآلهة، وغطى وجهه ومد رجليه، ونام نومة الأبد، وانضم إلى آبائه
\par 2 ومع كل هذا كان يعقوب مضطجعًا في حضنه، ولم يكن يعلم أن إبراهيم، أبو أبيه، قد مات
\par 3 فاستيقظ يعقوب من نومه وإذا إبراهيم بارد كالثلج، فقال: يا أبي يا أبي، ولم يكن أحد يتكلم، فعلم أنه قد مات.
\par 4 فقام من حضنه وركض وأخبر رفقة أمه. فذهبت رفقة إلى إسحاق ليلاً وأخبرته. فذهبا معًا ويعقوب معهما، وكان سراج في يده. ولما دخلا وجدا إبراهيم ميتًا
\par 5 فسقط إسحاق على وجه أبيه وبكى وقبله.
\par 6 وسمعت الأصوات في بيت إبراهيم، فقام إسماعيل ابنه وذهب إلى إبراهيم أبيه وبكى على إبراهيم أبيه هو وكل بيت إبراهيم وبكوا بكاءً عظيماً.
\par 7 فدفنه ابناه إسحاق وإسماعيل في المغارة المزدوجة، بالقرب من سارة امرأته، وبكوا عليه أربعين يومًا، جميع رجال بيته، وإسحاق وإسماعيل، وكل أبنائهما، وجميع بني قطورة في أماكنهم. وانتهت أيام بكاء إبراهيم
\par 8 وعاش ثلاثة يوبيلات وأربعة أسابيع سنين، أي مئة وخمسًا وسبعين سنة، وأكمل أيام حياته شيخًا وشبعان أيامًا
\par 9 لأن أيام الآباء، من حياتهم، كانت تسعة عشر يوبيلًا؛ وبعد الطوفان بدأوا ينقصون عن تسعة عشر يوبيلًا، وينقصون في اليوبيلات، ويشيخون سريعًا، ويشبعون من أيامهم بسبب الضيقات الكثيرة وشر طرقهم، باستثناء إبراهيم
\par 10 لأن إبراهيم كان كاملاً في جميع أعماله مع الرب، ومرضياً بالبر كل أيام حياته. وها هو لم يُكمل أربعة يوبيلات في حياته، بعدما شاخ بسبب الشر، وشبع أيامه
\par 11 وجميع الأجيال التي تقوم من هذا الوقت إلى يوم الدينونة العظيمة ستشيخ سريعًا، قبل أن تُكمل يوبيلين، وستتخلى عنهم معرفتهم بسبب شيخوختهم، وستتلاشى كل معرفتهم
\par 12 وفي تلك الأيام، إذا عاش رجل يوبيلًا ونصفًا من السنين، سيقال عنه: «لقد عاش طويلًا، ومعظم أيامه وجع وحزن وبؤس، ولا سلام».
\par 13 «لأن مصيبة تلو مصيبة، وجرحًا تلو جرح، وضيقًا تلو ضيق، وأخبارًا سيئة تلو أخبار سيئة، ومرضًا تلو مرض، وكل أحكام شريرة كهذه، واحدة تلو الأخرى، مرض وانقلاب، وثلج والصقيع والجليد، وحمى وقشعريرة وخمول، ومجاعة، وموت، وسيف، وأسر، وجميع أنواع الكوارث والآلام.»
\par 14 ويأتي كل هؤلاء على جيل شرير، يتعدى على الأرض. أعمالهم نجاسة وزنا ودنس ورجاسات
\par 15 "حينئذ يقولون: أيام الآباء كانت كثيرة إلى ألف سنة وكانت جيدة، ولكن هوذا أيام حياتنا إن عاش الإنسان فهي سبعون سنة، وإن كان قويا فأربعون سنة، وتلك شريرة، وليس سلام في أيام هذا الجيل الشرير."
\par 16 وفي ذلك الجيل يُوبِّخ الأبناء آباءهم وشيوخهم على الخطيئة والإثم، وعلى أقوال أفواههم، وعلى الشرور العظيمة التي يرتكبونها، وعلى تركهم العهد الذي قطعه الرب بينهم وبينه، أن يحفظوا ويعملوا بجميع وصاياه وأحكامه وجميع شرائعه، لا يحيدوا يمينًا ولا شمالًا
\par 17 لأن الجميع عملوا الشر، وكل فم تكلم بالإثم، وكل أعمالهم نجاسة ورجس، وكل طرقهم نجاسة ونجاسة وهلاك
\par 18 هوذا الأرض تُدمر بسبب جميع أعمالهم، ولن يكون هناك بذر كرمة، ولا زيت؛ لأن أعمالهم كلها خيانة، وسيهلكون جميعًا معًا، الوحوش والبهائم والطيور، وجميع أسماك البحر، بسبب بني البشر
\par 19 ويتشاجرون، الشاب مع الشيخ، والشيخ مع الشاب، والفقراء مع الأغنياء، والوضيع مع الكبير، والفقراء مع الأمير، بسبب الشريعة والعهد؛ لأنهم نسوا الوصية والعهد والأعياد والأشهر والسبوت واليوبيلات وجميع الأحكام
\par 20 ويقومون (بالأقواس والسيوف) ويحاربون ليردوهم إلى الطريق، لكنهم لا يرجعون حتى يُسفك دماء كثيرة على الأرض، واحدًا تلو الآخر
\par 21 والذين هربوا لن يرجعوا عن شرهم إلى طريق البر، بل سيرتفعون جميعًا إلى الغش والغنى، ليأخذ كل واحد منهم كل ما لقريبه، وسيسمون الاسم العظيم، ولكن ليس بالحق ولا بالبر، وسينجسون قدس الأقداس بنجاستهم وفساد نجاستهم
\par 22 وسيصيب عقاب عظيم أعمال هذا الجيل من قبل الرب، فيسلمهم إلى السيف والدينونة والسبي والنهب والأكل
\par 23 وسيوقظ عليهم خطاة الأمم، الذين لا رحمة لهم ولا شفقة، ولا يحترمون أحدًا، لا كبيرًا ولا صغيرًا، ولا أحدًا، لأنهم أشرّ وأقدر على فعل الشر من جميع بني البشر
\par    
\par     ويستخدمون العنف على إسرائيل والعدوان على يعقوب،  
\par    ويُسفَكُ دماءٌ كثيرةٌ على الأرض،  
ولا يكون من يجمع ولا من يدفن.
\par    
\par 24 في تلك الأيام سيصرخون بصوت عالٍ،  
\par    وادعوا وصلّوا لكي يخلصوا من أيدي الخطاة الأمم.  
\par    ولكن لن يخلص أحد.
\par    
\par 25 وتكون رؤوس الأطفال بيضاء وشيبًا،  
\par    ويكون الطفل الذي عمره ثلاثة أسابيع شيخًا كرجل ابن مئة سنة،  
\par    وتُهدم قامتهم بالضيق والضيق.
\par    
\par 26 وفي تلك الأيام سيبدأ الأطفال بدراسة القوانين،  
\par     والسعي وراء الوصايا،  
\par    والرجوع إلى طريق الصواب.
\par    
\par 27 وستبدأ الأيام في النمو والازدياد بين أبناء البشر هؤلاء  
حتى تقترب أيامهم إلى ألف سنة.  
وإلى سنين أكثر مما كانت عدد الأيام.
\par    
\par 28 ولن يكون هناك شيخ  
\par     ولا من لا يرضى بأيامه،  
\par    لأن الجميع سيكونون أطفالاً وشباباً.
\par    
\par 29 ويُكملون كل أيامهم ويعيشون في سلام وفرح،  
\par    ولن يكون هناك شيطان ولا مهلك شرير.  
لأن كل أيامهم تكون أيام بركة وشفاء.
\par    
\par 30 وفي ذلك الوقت يشفي الرب عبيده،  
\par     ويقومون ويرون سلامًا عظيمًا،  
\par    وأخرج أعداءهم.
\par    
\par    والصديق يرى فيشكر،  
\par    وافرحوا بفرح إلى الأبد،  
\par    ويرى جميع أحكامهم وجميع لعناتهم على أعدائهم.
\par    
\par 31 وستستقر عظامهم في الأرض،  
\par     وستفرح أرواحهم كثيرًا،  
\par    فيعلمون أن الرب هو منفذ الحكم،  
\par    ويظهر الرحمة للمئات والآلاف ولكل الذين يحبونه
\par    
\par 32 وأنت يا موسى، اكتب هذه الكلمات؛ لأنها هكذا كُتبت، وهي تُسجل على الألواح السماوية شهادةً للأجيال إلى الأبد

\chapter{24}

\par \textit{إسحاق عند بئر الرؤية، 1 (راجع تكوين 25: 11). عيسو يبيع بكوريته، 2-7 (راجع تكوين 25: 29-34).}

\par 1 وكان بعد وفاة إبراهيم أن الرب بارك إسحاق ابنه، فقام من حبرون وذهب وأقام عند بئر الرؤيا في السنة الأولى من الأسبوع الثالث [2073 صباحًا] من هذا اليوبيل، سبع سنوات
\par 2 وفي السنة الأولى من الأسبوع الرابع، بدأ جوع في الأرض، غير الجوع الأول الذي كان في أيام إبراهيم.
\par 3 فأكل يعقوب عِدسًا، فجاء عيسو من الحقل جائعًا. فقال ليعقوب أخاه: أعطني من هذا العدس الأحمر. فقال له يعقوب: بِعْ لي بكوريتك فأعطيك خبزًا، وأيضًا من عدس هذا العدس
\par 4 فقال عيسو في قلبه: «أموت، فما المنفعة لي من هذه البكورية؟»
\par 5 فقال ليعقوب: أعطيك إياه. فقال يعقوب: احلف لي اليوم. فحلف له
\par 6 وأعطى يعقوب أخاه عيسو خبزًا وطبيخًا، فأكل حتى شبع، واحتقر عيسو بكوريته. لذلك دُعي اسم عيسو أدوم، بسبب الطبيخ الأحمر الذي أعطاه إياه يعقوب بكوريته
\par 7 فصار يعقوب شيخًا، وأُنزل عيسو عن كرامته
\par 8 وكان الجوع في الأرض، فذهب إسحاق لينزل إلى مصر في السنة الثانية من هذا الأسبوع، وذهب إلى ملك الفلسطينيين إلى جرار، إلى أبيمالك
\par 9 فظهر له الرب وقال له: «لا تنزل إلى مصر، اسكن في الأرض التي أقول لك عنها، وتغرب في هذه الأرض، وأنا أكون معك وأباركك».
\par 10 «لأني لك ولنسلك أعطي كل هذه الأرض، وأوفي قسمي الذي أقسمت لإبراهيم أبيك، وأكثر نسلك كنجوم السماء، وأعطي نسلك كل هذه الأرض.»
\par 11 «ويتبارك في نسلك جميع أمم الأرض، من أجل أن أباك أطاع صوتي، وحفظ ما يحفظه من أوامري ووصاياي وشرائعي وأحكامي وعهدي. والآن أطع صوتي واسكن في هذه الأرض.»
\par 12 وأقام في جيلار ثلاثة أسابيع من السنين.
\par 13 "وأوصاه أبيمالك عليه وعلى كل ما له قائلا: كل إنسان يمسه أو يمس شيئا مما له يموت موتا."
\par 14 وتقوى إسحاق بين الفلسطينيين، وصار له أملاك كثيرة: بقرًا وغنمًا وجمالًا وحميرًا وبيتًا كبيرًا
\par 15 وزرع في أرض الفلسطينيين فأتى مئة ضعف، وعظم إسحاق جدًا، فحسده الفلسطينيون
\par 16 وجميع الآبار التي حفرها عبيد إبراهيم في حياة إبراهيم، ردمها الفلسطينيون بعد وفاة إبراهيم، وملأوها ترابًا
\par 17 فقال أبيمالك لإسحاق: «اذهب من عندنا، لأنك أقوى منا بكثير». فانطلق إسحاق من هناك في السنة الأولى من الأسبوع السابع، وتغرب في أودية جرار
\par 18 وحفروا أيضًا آبار الماء التي حفرها عبيد إبراهيم أبيه، والتي سدها الفلسطينيون بعد وفاة إبراهيم أبيه، فدعا أسماءها كما سماها إبراهيم أبوه
\par 19 وحفر عبيد إسحق بئراً في الوادي فوجدوا ماءً حياً، وخاصم رعاة جرار رعاة إسحق قائلين: لنا الماء، فدعا إسحق اسم البئر "انحرافاً" لأنهم انحرافوا معنا.
\par 20 وحفروا بئرًا ثانية، وتخاصموا عليها أيضًا، فسمى اسمها عداوة. ثم قام من هناك وحفروا بئرًا أخرى، ولم يتخاصموا عليها، فسمى اسمها رُوْم. وقال إسحاق: الآن قد أرحب لنا الرب، وكثرنا في الأرض
\par 21 وصعد من هناك إلى بئر القسم في السنة الأولى من الأسبوع الأول في اليوبيل الرابع والأربعين [2108 صباحًا].
\par 22 فظهر له الرب في تلك الليلة، في هلال الشهر الأول، وقال له: «أنا إله إبراهيم أبيك. لا تخف، لأني معك، وسأباركك، وسأكثر نسلك تكثيرًا كرمل الأرض، من أجل إبراهيم عبدي».
\par 23 فبنى هناك مذبحًا كان إبراهيم أبوه قد بناه أولًا، ودعا باسم الرب، وقدم ذبيحة لإله إبراهيم أبيه
\par 24 فحفروا بئرًا فوجدوا ماءً حيًا.
\par 25 وحفر عبيد إسحق بئرا أخرى ولم يجدوا ماء، فذهبوا وأخبروا إسحق أنهم لم يجدوا ماء. فقال إسحق: قد أقسمت اليوم للفلسطينيين وقد أخبرنا بهذا الأمر.
\par 26 ودعا اسم ذلك المكان بئر القسم، لأنه هناك حلف لأبيمالك وأحزات صاحبه وفيكول قائد جيشه
\par 27 فعلم إسحاق في ذلك اليوم أنه أقسم لهم مكرهًا أن يصنع معهم صلحًا
\par 28 ولعن إسحاق في ذلك اليوم الفلسطينيين وقال: ملعون الفلسطينيون إلى يوم الغضب والسخط من بين جميع الأمم. ليجعلهم الله سخرية ولعنة وغضبًا وسخطًا في أيدي الخطاة الأمم وفي أيدي كتيم
\par 29 وكل من ينجو من سيف العدو والكتيم، فلتستأصله الأمة الصالحة بالدينونة من تحت السماء، لأنهم سيكونون أعداءً وأعداءً لأولادي مدى أجيالهم على الأرض
\par    
\par 30 ولا يبقى لهم بقية،  
\par    ولا أحد يخلص في يوم غضب الدينونة؛  
\par    لأن الهلاك والقلع والطرد من الأرض هو كل نسل الفلسطينيين (المحفوظ)،  
ولا يبقى بعد لهؤلاء الكفتوريين اسم ولا نسل على الأرض.
\par    
\par 31 لأنه وإن صعد إلى السماء،  
\par     ومن هناك سيُنزل،
\par    
\par    وإن جعل نفسه قويًا على الأرض،  
\par     ومن هناك يُسحب،
\par    
\par    وإن اختبأ بين الأمم،  
\par     ومن هناك يُقتلع.
\par    
\par    ورغم أنه نزل إلى الهاوية،  
\par    هناك أيضًا يكون عقابه عظيمًا،  
\par    وهناك أيضا لن يكون له سلام.  
\par    
\par 32 وإن ذهب إلى الأسر،  
\par     على أيدي طالبي نفسه يقتلونه في الطريق،  
ولا يبقى له اسم ولا نسل في كل الأرض.  
"فإنه إلى اللعنة الأبدية يذهب."
\par    
\par 33 وهكذا كُتب ونُقش عنه على الألواح السماوية، ليُفعل به يوم القيامة، حتى يُقتلع من الأرض

\chapter{25}

\par \textit{نصحت رفقة يعقوب ألا يتزوج امرأة كنعانية، 1-3. وعد يعقوب بالزواج من ابنة لابان على الرغم من طلبات عيسو المُلحة بأن يتزوج امرأة كنعانية، 4-10. باركت رفقة يعقوب، 11-23. (قارن تكوين 28: 1-4.)}

\par 1 وفي السنة الثانية من هذا الأسبوع في هذا اليوبيل، دعت رفقة يعقوب ابنها، وكلمته قائلة: يا ابني، لا تأخذ زوجة من بنات كنعان كما فعل عيسو أخاك، الذي اتخذ لنفسه امرأتين من بنات كنعان، وقد مررن نفسي بكل أعمالهن النجسة، لأن جميع أعمالهن زنا وشهوة، وليس عندهن بر، لأنها شريرة
\par 2 وأنا يا ابني أحبك حبًا جمًا، ويباركك قلبي ومودتي في كل ساعة من النهار وحراسة الليل
\par 3 والآن يا ابني، اسمع لقولي، وافعل مشيئة أمك، ولا تتزوج امرأة من بنات هذه الأرض، بل من بيت أبي ومن عشيرته. تتزوج امرأة من بيت أبي، فيباركك الله العلي، ويكون أولادك جيلاً بارًا وذرية مقدسة
\par 4 ثم كلم يعقوب رفقة أمه وقال لها: «ها أنا يا أمي ابن تسعة أسابيع، ولم أعرف امرأة ولم أمسس امرأة، ولم أخطب واحدة، ولا أفكر حتى في أن أتخذ لنفسي زوجة من بنات كنعان».
\par 5 «لأني تذكرت يا أمي كلام إبراهيم أبينا، أنه أوصاني ألا أتزوج زوجة من بنات كنعان، بل أن أتزوج زوجة من نسل بيت أبي ومن عشيرتي.»
\par 6 «لقد سمعتُ سابقًا أنه قد وُلِدَت بناتٌ للابان أخيك، وقد عزمتُ على أن أتخذَ لهنَّ زوجةً من بينهنَّ.»
\par 7 "ولهذا السبب حفظت نفسي بروحي لئلا أخطئ أو أفسد في كل طرقي طوال أيام حياتي. لأنه في الشهوة والزنى أوصاني إبراهيم أبي كثيرا."
\par 8 «ورغم كل ما أمرني به، فقد ناضل أخي معي طوال هذه السنتين والعشرين، وتحدث إليّ كثيرًا وقال: يا أخي، تزوج أختًا من زوجتيّ الاثنتين؛ لكنني أرفض أن أفعل كما فعل.»
\par 9 «أقسم أمامك يا أمي، أنني لن أتخذ زوجة من بنات نسل كنعان طوال أيام حياتي، ولن أفعل الشر كما فعل أخي.»
\par 10 «لا تخافي يا أمي؛ كوني على يقين أنني سأفعل مشيئتكِ وأسلك في الاستقامة، ولن أفسد طرقي إلى الأبد.»
\par 11 ثم رفعت وجهها نحو السماء ومدت أصابع يديها وفتحت فمها وباركت الله العلي الذي خالق السماء والأرض وحمدته وسبحته
\par 12 وقالت: «مبارك الرب الإله، ومبارك اسمه القدوس إلى الأبد، الذي أعطاني يعقوب ابنًا طاهرًا وذرية مقدسة، لأنه لك، ويكون نسله لك دائمًا وفي كل الأجيال إلى الأبد».
\par 13 باركه يا رب، وضع في فمي بركة البر، فأباركه
\par 14 وفي تلك الساعة، عندما نزل روح البر في فمها، وضعت يديها على رأس يعقوب، وقالت:
\par    
\par 15 مبارك أنت، رب البر وإله الدهور  
ويباركك الرب ويباركك إلى أبد الأجيال.
\par    
\par    ليهديك يا ابني طريق البر،  
وأظهر البر لنسلك.
\par    
\par 16 ويكثر من أبنائك في حياتك،  
\par     وليقموا على عدد أشهر السنة  
\par     وليكثر أبناؤهم ويعظموا فوق نجوم السماء،  
ويكون عددهم أكثر من رمل البحر.
\par    
\par 17 وليمنحهم هذه الأرض الطيبة - كما قال إنه سيعطيها لإبراهيم ولنسله من بعده إلى الأبد -  
\par     وليحتفظوا بها كملك إلى الأبد.
\par    
\par 18 وأسأل الله أن أرى لك يا بني أطفالًا مباركين خلال حياتي،  
ولتكن كل نسلك مباركة مقدسة.
\par    
\par 19 وكما أنعشتَ روحَ والدتكَ في حياتها،  
\par    إن رحم من حملت بك يباركك هكذا،
\par    
\par    [حبي] وصدري يباركانك  
وفمي ولساني يسبحونك كثيرًا.
\par    
\par 20 تتزايد وتنتشر في جميع أنحاء الأرض،  
وليكن نسلك كاملاً في فرح السماء والأرض إلى الأبد؛
\par    
\par      وليفرح نسلك،  
وفي يوم السلام العظيم فليكن له السلام.
\par    
\par 21 وليبقِ اسمك ونسلك إلى كل الدهور،  
\par     وليكن الله العلي إلههم،
\par    
\par    وليحل معهم إله البر،  
وبهم يُبنى مقدسه إلى كل الدهور.
\par    
\par 22 تبارك من يباركك،  
\par     وكل جسد يلعنك زورًا، فليكن ملعونًا
\par    
\par 23 فقبلته وقالت له:  
\par     «ليحبك رب العالمين»  
\par    'كما يفرح بك قلب أمك ومحبتها ويباركك.''
\par    وكفت عن البركة.

\chapter{26}

\textit{إسحاق يطلب من عيسو لحم غزال، 1-4. رفقة تُرشد يعقوب للحصول على البركة، 5-9. يعقوب في هيئة عيسو يحصل عليها، 10-24. يُحضر عيسو صيده، وبإلحاحه يحصل على البركة، 25-34. يُهدد يعقوب، 35. (راجع تكوين 27)}

\par 1 وفي السنة السابعة من هذا الأسبوع، دعا إسحاق عيسو ابنه الأكبر، وقال له: «أنا [2114 AM] شيخ يا ابني، وها أنا ذا عيناي قد كلتا في البصر، ولا أعرف يوم وفاتي».
\par 2 «والآن خذ أدوات صيدك، جعبتك وقوسك، واخرج إلى الحقل، واصطد لي (لحم الغزال) يا ابني، واصنع لي لحمًا لذيذًا كما تحبه نفسي، وأتني به لآكله، وحتى تباركك نفسي قبل أن أموت.»
\par 3 لكن رفقة سمعت إسحاق يتحدث إلى عيسو.
\par 4 فخرج عيسو مبكرا إلى البرية ليصطاد ويصيد ويأتي بأبيه إلى البيت.
\par 5 فدعت رفقة يعقوب ابنها وقالت له: هوذا قد سمعت إسحاق أباك يكلم عيسو أخاك قائلًا: اصطد لي واصنع لي طعامًا مطيبًا وأتني بما
\par 6 لآكل وأباركك أمام الرب قبل أن أموت. والآن يا بني، أطع صوتي في ما آمرك به: اذهب إلى قطيعك وأحضر لي جديين جيدين من المعزى، وسأصنعهما طعامًا لذيذًا لأبيك كما يحب، وأحضره إلى أبيك ليأكل ويباركك أمام الرب قبل أن يموت، ولتكن مباركًا
\par 7 فقال يعقوب لرفقة أمه: «يا أمي، لا أمنع عن أبي شيئًا مما يأكله ويرضيه، إنما أخاف يا أمي أن يعرف صوتي ويريد أن يلمسني».
\par 8 "وأنت تعلم أني أملس، وعيسو أخي أشعر، فأظهر أمام عينيه كفاعل شر، وأفعل أمراً لم يأمرني به، فيغضب عليّ، وأجلب على نفسي لعنة لا بركة."
\par 9 فقالت له رفقة أمه: «لعنتك عليّ يا ابني، اسمع لقولي فقط».
\par 10 فأطاع يعقوب قول رفقة أمه، فذهب وأخذ جديين من المعزى جيدين وسمينين، وأتى بهما إلى أمه، فصنعتهما أمه طعامًا لذيذًا كما كان يحب
\par 11 فأخذت رفقة ثياب عيسو ابنها الأكبر الفاخرة التي كانت عندها في البيت، وألبست يعقوب ابنها الأصغر، ووضعت جلدي المعزى على يديه وعلى كشوفات عنقه
\par 12 وأعطت اللحم والخبز اللذين أعدتهما في يد ابنها يعقوب
\par 13 فدخل يعقوب على أبيه وقال: «أنا ابنك، قد فعلت كما أمرتني. قم واجلس وكل مما اصطدت يا أبي، حتى تباركني نفسك».
\par 14 فقال إسحاق لابنه: «كيف وجدتَ بهذه السرعة يا ابني؟»
\par 15 فقال يعقوب: لأن إلهك هو الذي وجدني.
\par 16 فقال له إسحاق: «تقدم لأجسّك يا ابني، إن كنت ابني عيسو أم لا».
\par 17 فتقدم يعقوب إلى إسحاق أبيه، فجسّه وقال: الصوت صوت يعقوب، ولكن اليدين يدا عيسو
\par 18 ولم يُميزه، لأنه كان تدبيرًا من السماء لنزع إدراكه، ولم يُميز إسحاق، لأن يديه كانتا مشعرتين كيدي أخيه عيسو، فباركه
\par 19 فقال: «أأنت ابني عيسو؟» فقال: «أنا ابنك». فقال: «تقدم إليّ لآكل مما اصطدت يا ابني، حتى تباركك نفسي».
\par 20 فتقدم إليه فأكل، وأحضر له خمرًا فشرب
\par 21 فقال له إسحاق أبوه: تقدم وقبلني يا ابني
\par 22 فتقدم وقبّله. فشم رائحة ثيابه، وباركه وقال: «هوذا رائحة ابني كرائحة حقل ممتلئ باركه الرب».
\par    
\par 23 وليُعطِكَ الربُّ من ندى السماء  
\par    ومن ندى الأرض وكثرة القمح والزيت:
\par    
\par    لتخدمك الأمم،  
\par    ويسجد لك الشعوب.
\par    
\par 24 كُنْ سَيِّدًا عَلَى إِخْوَتِكَ،  
\par     وَلْيَسْجُدْ لَكَ بَنُو أُمِّكَ
\par    
ولتحل جميع البركات التي باركني بها الرب وبارك بها إبراهيم أبي.  
\par    يُمنح لك ولنسلك إلى الأبد:
\par    
\par    ملعون من لاعنك،  
\par    وطوبى لمن يباركك.
\par    
\par 25 وكان لما فرغ إسحاق من مباركة ابنه يعقوب، وخرج يعقوب من عند إسحاق أبيه، أنه اختبأ، فجاء عيسو أخوه من صيده
\par 26 فصنع أيضًا طعامًا شهيًا، وأتى به إلى أبيه، وقال لأبيه: «ليقم أبي ويأكل من صيدي حتى تباركني نفسك».
\par 27 فقال له إسحاق أبوه: من أنت؟ فقال له: أنا بكرك، ابنك عيسو. قد فعلت كما أمرتني
\par 28 فتعجب إسحاق دهشة شديدة، وقال: «من هو الذي اصطاد والتقط وأتى بي، فأكلت من الكل قبل أن تجيء، وباركته. ويكون مباركًا هو وكل نسله إلى الأبد».
\par 29 فلما سمع عيسو كلام أبيه إسحاق صرخ صرخة عظيمة ومرة ​​جدًا، وقال لأبيه: باركني أنا أيضًا يا أبي
\par 30 فقال له: «جاء أخوك بمكر وأخذ بركتك». فقال: «الآن عرفت لماذا سمي يعقوب: هوذا قد تخلف عني هاتين المرتين: أخذ بكوريتي، والآن أخذ بركتي».
\par 31 فقال: «ألم تحفظ لي بركة يا أبي؟» فأجاب إسحاق وقال لعيسو:
\par    
\par     'هوذا قد جعلته سيدك،  
\par    وجميع إخوته أعطيتهم له عبيدًا،  
\par    وبكثرة القمح والخمر والزيت قويته.
\par    
وماذا أفعل الآن من أجلك يا ابني؟
\par    
\par 32 فقال عيسو لإسحاق أبيه:  
\par     «ألديك بركة واحدة فقط يا أبي؟»  
\par   باركني أنا أيضًا يا أبتِ: '
\par    
\par 33 فرفع عيسو صوته وبكى. فأجاب إسحاق وقال له:
\par    
\par     'هوذا، بعيدًا عن ندى الأرض، سيكون مسكنك،  
\par    وبعيدًا عن ندى السماء من فوق.
\par    
\par 34 وبسيفك تحيا،  
\par     وتخدم أخاك
\par    
\par    وسيحدث ذلك عندما تصبح عظيمًا،  
\par    وأنفض نيره عن عنقك،  
\par    إنك تخطئ خطيئة كاملة حتى الموت،  
\par    ويُقتلَع نسلك من تحت السماء.
\par    
\par 35 وكان عيسو يهدد يعقوب من أجل البركة التي باركه بها أبوه، وقال في قلبه: الآن تأتي أيام مناحة أبي حتى أقتل يعقوب أخي.

\chapter{27}

\par \textit{رفقة، التي شعرت بالقلق من تهديدات عيسو، تقنع إسحاق بإرسال يعقوب إلى بلاد ما بين النهرين، 1-12. إسحاق يعزي رفقة بشأن رحيل يعقوب، 13-18. حلم يعقوب ونذره في بيت إيل، 19-27. (راجع تكوين 28)}

\par 1 فأُخبرت رفقة بكلام عيسو ابنها الأكبر في الحلم، فأرسلت رفقة ودعت يعقوب ابنها الأصغر،
\par 2 وقال له: هوذا عيسو أخوك سينتقم منك ليقتلك
\par 3 «فالآن يا ابني، اسمع لقولي، وقم واهرب إلى أخي لابان، إلى حاران، وأقم عنده أيامًا قليلة حتى يرتد غضب أخيك، ويرفع غضبه عنك، وينسى كل ما فعلت، ثم أرسل وآخذك من هناك.»
\par 4 فقال يعقوب: «لست خائفًا. إن أراد قتلي قتلته».
\par 5 فقالت له: لا أحرم ابنيّ في يوم واحد
\par 6 فقال يعقوب لرفقة أمه: «هوذا أنتِ تعلمين أن أبي قد شاخ، ولا يبصر لأن عينيه كليلتان، وإن تركته يكون شرًا في عينيه، لأني أتركه وأذهب من عندك، فيغضب أبي ويلعنني. لا أذهب. متى أرسلني، فحينئذٍ أذهب فقط».
\par 7 فقالت رفقة ليعقوب: «سأدخل وأكلمُه فيُطلقك».
\par 8 فدخلت رفقة وقالت لإسحاق: «إني أكره حياتي بسبب ابنتي حث اللتين اتخذهما عيسو له زوجتيهما. وإن اتخذ يعقوب زوجة من بنات الأرض مثل هاتين، فلماذا أعيش بعد؟ إن بنات كنعان شريرات».
\par 9 فدعا إسحاق يعقوب وباركه، ووعظه وقال له: لا تأخذ زوجة من إحدى بنات كنعان،
\par 10 قم واذهب إلى بلاد ما بين النهرين، إلى بيت بتوئيل أبي أمك، وخذ لك زوجة من هناك من بنات لابان أخي أمك
\par 11 «والله القدير يباركك ويكثرك ويكثرك حتى تصير جمهورًا من الأمم، ويعطيك بركات أبي إبراهيم لك ولنسلك من بعدك، حتى ترث أرض غربتك وكل الأرض التي أعطاها الله لإبراهيم. اذهب يا ابني بسلام.»
\par 12 فأرسل إسحاق يعقوب، فذهب إلى أرام ما بين النهرين، إلى لابان بن بتوئيل السرياني، أخي رفقة أم يعقوب
\par 13 وحدث بعد قيام يعقوب ليذهب إلى بلاد ما بين النهرين أن روح رفقة حزنت على ابنها وبكت
\par 14 فقال إسحاق لرفقة: «يا أختي، لا تبكي على يعقوب ابني، فإنه ذاهب بسلام، وبسلام يرجع».
\par 15 "ويحفظه الله العلي من كل شر، ويكون معه، لأنه لا يتركه كل أيامه".
\par 16 «لأني أعلم أن طرقه تنجح في كل شيء حيثما ذهب، حتى يرجع إلينا بسلام، ونراه بسلام.»
\par 17 لا تخافي عليه يا أختي، فهو على الطريق المستقيم وهو رجل كامل، وهو أمين ولن يهلك. لا تبكي
\par 18 وعزى إسحاق رفقة بسبب ابنها يعقوب وباركه
\par 19 وذهب يعقوب من بئر القسم ليذهب إلى حاران في السنة الأولى من الأسبوع الثاني في اليوبيل الرابع والأربعين، وجاء إلى لوز على الجبال، وهي بيت إيل، في هلال الشهر الأول من هذا الأسبوع، [21:15 صباحًا] وجاء إلى المكان عند المساء وانحرف عن الطريق غربي الطريق في تلك الليلة، ونام هناك لأن الشمس كانت قد غربت
\par 20 فأخذ حجرًا من ذلك المكان ووضعه عند رأسه تحت الشجرة، وكان مسافرًا وحده، فنام
\par 21 وحلم في تلك الليلة، وإذا بسلم منصوب على الأرض، ورأسه يمس السماء، وإذا ملائكة الرب صعدوا ونزلوا عليه، وإذا الرب واقف عليه
\par 22 ثم كلم يعقوب وقال: «أنا الرب إله إبراهيم أبيك وإله إسحاق. الأرض التي أنت نائم عليها لك أعطيها ولنسلك من بعدك».
\par 23 «ويكون نسلك كتراب الأرض، وتنمو غربًا وشرقًا، شمالًا وجنوبًا، وتتبارك فيك وفي نسلك جميع قبائل الأمم.»
\par 24 «وها أنا أكون معك، وأحفظك أينما تذهب، وأردك إلى هذه الأرض بسلام، لأني لا أتركك حتى أفعل كل ما قلته لك».
\par 25 فاستيقظ يعقوب من نومه وقال: «حقًا هذا المكان بيت الرب، وأنا لم أعلم». فخاف وقال: «مخيف هذا المكان الذي ليس إلا بيت الله، وهذا باب السماء».
\par 26 فبكر يعقوب في الصباح، وأخذ الحجر الذي وضعه تحت رأسه، وأقامه عمودًا لعلامة، وصب زيتًا على رأسه. ودعا اسم ذلك المكان بيت إيل، ولكن اسم المكان كان لوز أولًا
\par 27 ونذر يعقوب نذرًا للرب قائلًا: «إن كان الرب معي، وحفظني في هذا الطريق الذي أسلكه، وأعطاني خبزًا لآكل وثيابا لألبس، ورجعت بسلام إلى بيت أبي، يكون الرب لي إلهًا، وهذا الحجر الذي أقمته عمودًا علامة في هذا المكان يكون بيت الرب، وكل ما تعطيني أعشره لك يا إلهي».

\chapter{28}

يعقوب يتزوج ليئة وراحيل، ١-١٠. أولاده من ليئة وراحيل ومن جاريتيهما، ١١-٢٤. سعى يعقوب لترك لابان، ٢٥، لكنه بقي بأجر معين، ٢٦-٨. اغتنى يعقوب، ٢٩-٣٠. (راجع تكوين ٢٩: ١، ١٧، ١٨، ٢١-٣٥؛ ٣٣: ١-١٣، ١٧-٢٢، ٢٤، ٢٥، ٢٨، ٣٢، ٣٩، ٤٣؛ ٣١: ١، ٢.)

\par 1 ثم انطلق في رحلته، وجاء إلى أرض المشرق، إلى لابان أخي رفقة، وكان معه، وخدمه لأجل راحيل ابنته أسبوعًا واحدًا
\par 2 وفي السنة الأولى من الأسبوع الثالث [2122 صباحًا] قال له: أعطني امرأتي التي خدمتك بها سبع سنوات. فقال لابان ليعقوب: أعطيك امرأتك
\par 3 فصنع لابان وليمة، وأخذ ليئة ابنته الكبرى، وأعطاها ليعقوب زوجة، وأعطاها زلفة جاريته جارية، ولم يعلم يعقوب، لأنه ظن أنها راحيل
\par 4 فدخل عليها، فإذا هي ليئة. فغضب يعقوب على لابان وقال له: "لماذا فعلت بي هكذا؟ ألم أخدمك من أجل راحيل وليس من أجل ليئة؟ لماذا ظلمتني؟"
\par 5 «خذ ابنتك وأنا أذهب، لأنك أساءت إليّ». لأن يعقوب أحب راحيل أكثر من ليئة، لأن عيني ليئة كانتا ضعيفتين، لكن شكلها كان جميلاً جداً، أما راحيل فكانت ذات عيون جميلة وشكل جميل وجميل جداً
\par 6 فقال لابان ليعقوب: «لا يُفعل هكذا في بلادنا أن تُعطى الصغيرة قبل الكبيرة». وليس من الصواب أن نفعل هذا؛ لأنه هكذا هو مكتوب ومُرسوم في الألواح السماوية: لا ينبغي لأحد أن يُعطي ابنته الصغيرة قبل الكبيرة؛ بل تُعطى الكبيرة أولاً، ثم الصغيرة بعدها. ومن يفعل ذلك، يُكتب عليه إثم في السماء، وليس أحد بارًا يفعل هذا الأمر، لأن هذا الفعل شرير أمام الرب
\par 7 وأمر بني إسرائيل أن لا يفعلوا هذا الأمر. لا يأخذوا ولا يعطوا الأصغر قبل أن يعطوا الأكبر، لأنه شرير جدًا
\par 8 فقال لابان ليعقوب: «لتمر سبعة أيام عيد هذه، فأعطيك راحيل، فتخدمني سبع سنين أخرى، فترعى غنمي كما فعلت في الأسبوع الأول».
\par 9 وفي اليوم الذي انقضت فيه سبعة أيام وليمة ليئة، أعطى لابان راحيل ليعقوب ليخدمه سبع سنين أخرى، وأعطى راحيل بلهة أخت زلفة جارية
\par 10 وخدم أيضًا سبع سنين أخرى لراحيل، لأن ليئة أُعطيت له مجانًا
\par 11 وفتح الرب رحم ليئة، فحبلت وولدت ليعقوب ابنًا، فدعا اسمه رأوبين، في اليوم الرابع عشر من الشهر التاسع، في السنة الأولى من الأسبوع الثالث. [2122 صباحًا]
\par 12 وأما رحم راحيل فكان مغلقًا، لأن الرب رأى أن ليئة مكروهة وراحيل محبوبة
\par 13 "ودخل يعقوب أيضا على ليئة فحبلت وولدت ليعقوب ابنا ثانيا ودعا اسمه شمعون في اليوم الحادي والعشرين من الشهر العاشر وفي السنة الثالثة من هذا الأسبوع."
\par 14 ثم دخل يعقوب أيضًا على ليئة، فحبلت وولدت له ابنًا ثالثًا، فدعا اسمه لاوي، في هلال الشهر الأول في السنة السادسة من هذا الأسبوع. [2127 صباحًا]
\par 15 ثم دخل عليها يعقوب أيضًا، فحبلت وولدت له ابنًا رابعًا، فدعا اسمه يهوذا، في اليوم الخامس عشر من الشهر الثالث، في السنة الأولى من الأسبوع الرابع. [2129 صباحًا]
\par 16 ومن أجل كل هذا، غارت راحيل من ليئة، لأنها لم تلد، فقالت ليعقوب: «أعطني بنين». فقال يعقوب: «هل منعت عنك ثمرات بطنك؟ هل تركتك؟»
\par 17 ولما رأت راحيل أن ليئة قد ولدت ليعقوب أربعة أبناء: رأوبين وشمعون ولاوي ويهوذا، قالت له: «ادخل على بلهة جاريتي فتحبل وتلد لي ابنًا». (فأعطته بلهة جاريتها زوجة).
\par 18 فدخل عليها، فحبلت وولدت له ابنًا، فدعا اسمه دان، في التاسع من الشهر السادس، في السنة السادسة من الأسبوع الثالث. [2127 صباحًا]
\par 19 ثم دخل يعقوب أيضًا على بلهة ثانية، فحبلت وولدت ليعقوب ابنًا آخر، ودعت راحيل اسمه نفتالي، في اليوم الخامس من الشهر السابع، في السنة الثانية من الأسبوع الرابع. [2130 صباحًا]
\par 20 ولما رأت ليئة أنها عاقرة ولم تلد، غارت من راحيل، وأعطت أيضًا جاريتها زلفة ليعقوب زوجة، فحملت وولدت ابنًا، ودعت ليئة اسمه جاد، في الثاني عشر من الشهر الثامن، في السنة الثالثة من الأسبوع الرابع. [2131 صباحًا]
\par 21 ثم دخل عليها أيضًا، فحبلت وولدت له ابنًا ثانيًا، ودعت ليئة اسمه أشير، في الثاني من الشهر الحادي عشر، في السنة الخامسة من الأسبوع الرابع. [2133 صباحًا]
\par 22 فدخل يعقوب على ليئة، فحبلت وولدت ابنًا، ودعت اسمه يساكر، في الرابع من الشهر الخامس، في السنة الرابعة من الأسبوع الرابع، وأعطته لمرضعة
\par 23 ثم دخل عليها يعقوب أيضًا، فحبلت وولدت اثنين، ابنًا وبنتًا، ودعت اسم الابن زبولون، واسم الابنة دينة، في السابع من الشهر السابع، في السنة السادسة من الأسبوع الرابع. [2134 صباحًا]
\par 24 ونعم الرب على راحيل، وفتح رحمها، فحملت وولدت ابنًا، ودعت اسمه يوسف، في هلال الشهر الرابع، في السنة السادسة من هذا الأسبوع الرابع. [2134 صباحًا]
\par 25 وفي الأيام التي ولد فيها يوسف قال يعقوب للابان: أعطني نسائي وأولادي فأذهب إلى أبي إسحق فأبني لنفسي بيتاً لأني قد أكملت سني خدمتك بابنتيك فأذهب إلى بيت أبي.
\par 26 فقال لابان ليعقوب: «أقم عندي لأجل أجرتك، وارع غنمي لي بعد، وخذ أجرتك».
\par 27 واتفقوا فيما بينهم على أن يُعطيه أجره من الحملان والجداء المولودة سوداء ومنقطّة وبيضاء، (هذه) تكون أجره
\par 28 فولدت جميع الغنم بقعًا ورقطًا وسوداء، مُعَلَّمة بأنواع مختلفة، وولدت أيضًا خرافًا مثلها. كل ما كان بُقَطًا كان ليعقوب، وما لم يكن فهو للابان
\par 29 وكثرت ممتلكات يعقوب جدًا، فصار له ثيران وغنم وحمير وجمال وعبيد وإماء
\par 30 فحسد لابان وبنوه يعقوب، فاسترد لابان غنمه منه، ونظر إليه نظرة شريرة

\chapter{29}

\par \textit{يعقوب يغادر سرًا، 1-4. لابان يطارده، 5-6. عهد يعقوب ولابان، 7-8. دُمِّرت مساكن الأموريين (الرفائيين سابقًا) في زمن الكاتب، 9-11. لابان يغادر، 12. يعقوب يُصالح مع عيسو، 13. يعقوب يُرسل مؤنًا من الطعام إلى والديه أربع مرات في السنة إلى الخليل، 14-17، 19-20. عيسو يتزوج مرة أخرى، 18. (راجع تكوين 31: 3، 4، 10، 13، 19، 21، 23، 24، 46، 47؛ 32: 22؛ 33: 10، 16.)}

\par 1 وحدث لما ولدت راحيل يوسف أن لابان ذهب ليجز غنمه، لأنها كانت بعيدة عنه مسيرة ثلاثة أيام
\par 2 ولما رأى يعقوب أن لابان ذاهب ليجز غنمه، دعا يعقوب ليئة وراحيل وكلمهما بلطف أن تأتيا معه إلى أرض كنعان
\par 3 لأنه أخبرهم كيف رأى كل شيء في الحلم، حتى كل ما قاله له أنه يجب أن يعود إلى بيت أبيه، فقالوا: «إلى كل مكان تذهب إليه نذهب معك».
\par 4 وبارك يعقوب إله إسحاق أبيه وإله إبراهيم جد أبيه، وقام وركب نساءه وأولاده، وأخذ كل ما يملك وعبر النهر، وجاء إلى أرض جلعاد. وأخفى يعقوب نيته عن لابان ولم يخبره
\par 5 وفي السنة السابعة من الأسبوع الرابع، توجه يعقوب نحو جلعاد في الشهر الأول، في الحادي والعشرين منه. [2135 صباحًا] فتبعه لابان فأدرك يعقوب في جبل جلعاد في الشهر الثالث، في الثالث عشر منه
\par 6 ولم يدع الرب يعقوب يؤذيه، لأنه تراءى له في حلم الليل. وكلم لابان يعقوب
\par 7 وفي اليوم الخامس عشر من تلك الأيام صنع يعقوب وليمة للابان ولكل من جاء معه. وحلف يعقوب للابان في ذلك اليوم، وحلف لابان أيضا ليعقوب أن لا يعبر أحدهما جبل جلعاد إلى الآخر بقصد شرير.
\par 8 فصنع هناك كومة شاهد، ولذلك سمي ذلك المكان كومة الشهادة، نسبةً إلى هذه الكومة
\par 9 ولكن قبل ذلك كانوا يُسمون أرض جلعاد أرض الرفائيين؛ لأنها كانت أرض الرفائيين، وهناك وُلد الرفائيون، عمالقة كان طولهم عشرة أذرع، وتسعة أذرع، وثمانية أذرع، إلى سبعة أذرع
\par 10 وكان مسكنهم من أرض بني عمون إلى جبل حرمون، وكانت مراكز مملكتهم قرنائيم وعشتاروث وأذرعي ومصور وبعون
\par 11 فأهلكهم الرب بسبب شر أعمالهم، لأنهم كانوا أشرارًا جدًا، فسكن الأموريون مكانهم، أشرارًا وخطاة، ولا يوجد اليوم شعب قد عمل كل خطاياه، ولم تعد لهم مدة حياة على الأرض
\par 12 فأرسل يعقوب لابان، فانطلق إلى أرام ما بين النهرين، أرض المشرق، ورجع يعقوب إلى أرض جلعاد
\par 13 وعبر يعقوب يبوق في الشهر التاسع، في الحادي عشر منه. وفي ذلك اليوم جاء إليه عيسو أخوه، فاصطلح معه، وذهب من عنده إلى أرض سعير، وأما يعقوب فكان ساكنًا في الخيام
\par 14 وفي السنة الأولى من الأسبوع الخامس في هذا اليوبيل [2136 صباحًا] عبر الأردن، وأقام في عبر الأردن، ورعى غنمه من بحر الرمة إلى بيت شان، وإلى دوثان، وإلى غابة عقربيم
\par 15 وأرسل إلى أبيه إسحاق من كل ما يملك، من ملابس وطعام ولحم وشراب ولبن وزبدة وجبن وبعض تمر الوادي
\par 16 ولأمه رفقة أيضًا أربع مرات في السنة، بين أوقات الشهور، وبين الحرث والحصاد، وبين الخريف والمطر، وبين الشتاء والربيع، إلى برج إبراهيم
\par 17 لأن إسحاق كان قد رجع من بئر القسم وصعد إلى برج أبيه إبراهيم، وأقام هناك منفردًا مع ابنه عيسو
\par 18 لأنه في الأيام التي ذهب فيها يعقوب إلى بلاد ما بين النهرين، اتخذ عيسو لنفسه امرأة محلة ابنة إسماعيل، وجمع كل غنم أبيه ونساءه، وصعد وأقام في جبل سعير، وترك إسحاق أباه عند بئر الحلف وحده
\par 19 وصعد إسحاق من بئر القسم وسكن في برج إبراهيم أبيه في جبال حبرون،
\par 20 وأرسل يعقوب إلى هناك كل ما كان يرسله إلى أبيه وأمه من حين لآخر، كل ما احتاجا إليه، وباركا يعقوب بكل قلوبهما وكل أنفسهما

\chapter{30}

\par \textit{دينة تُغتصب، ١-٣. مذبحة أهل شكيم، ٤-٦. شرائع تحريم الزواج بين بني إسرائيل والوثنيين، ٧-١٧. اختيار لاوي للكهنوت بسبب مذبحته لأهل شكيم، ١٨-٢٣. شفاء دينة، ٢٤. توبيخ يعقوب، ٢٥-٦. (قارن تكوين ٣٣: ١٨، ٣٤: ٢، ٤، ٧، ١٣-١٤، ٢٥-٣٠، ٣٥: ٥.)}

\par 1 وفي السنة الأولى من الأسبوع السادس [2143 صباحًا] صعد إلى ساليم، شرقي شكيم، بسلام، في الشهر الرابع
\par 2 وهناك أخذوا دينة ابنة يعقوب إلى بيت شكيم بن حمور الحوي رئيس الأرض، واضطجع معها ونجّسها، وكانت فتاة صغيرة ابنة اثنتي عشرة سنة
\par 3 وطلب إلى أبيه وإخوتها أن يزوجوها له زوجة. فغضب يعقوب وبنوه على أهل شكيم، لأنهم نجّسوا دينة أختهم، وكلموهم بالسوء وخدعوهم وخدعوهم
\par 4 فجاء شمعون ولاوي إلى شكيم بغتة، ونفذا حكمًا على جميع أهل شكيم، وقتلا جميع الرجال الذين وجدوهم فيها، ولم يُبقوا فيها أحدًا. قتلا الجميع تحت التعذيب لأنهم أهانوا أختهما دينة
\par 5 وهكذا لا يُفعل بعد الآن أن تُدنس ابنة إسرائيل، لأنه قد قُدِّر عليهم في السماء أن يُهلكوهم بالسيف جميع رجال شكيم لأنهم صنعوا العار في إسرائيل
\par 6 فدفعهم الرب إلى أيدي بني يعقوب لكي يبيدوهم بالسيف ويحكموا عليهم، وحتى لا يُفعل مثل هذا أيضًا في إسرائيل حتى تتنجس عذراء إسرائيل
\par 7 وإن أراد أحد في إسرائيل أن يعطي ابنته أو أخته لرجل من نسل الأمم، فإنه يموت موتًا، ويرجمونه بالحجارة لأنه عمل عارًا في إسرائيل، وتحرق المرأة بالنار لأنها أهانت اسم بيت أبيها، فتُقتلع من إسرائيل
\par 8 ولا توجد زانية ولا نجاسة في إسرائيل طوال أيام أجيال الأرض، لأن إسرائيل مقدسة للرب، وكل من نجسها يموت موتًا. يرجمونه بالحجارة
\par 9 لأنه هكذا رُسِم وكُتِبَ في الألواح السماوية بشأن جميع نسل إسرائيل: من دنسها يموت موتًا ويُرجم بالحجارة
\par 10 وليس لهذه الناموسة حد أيام، ولا مغفرة، ولا كفارة. أما الرجل الذي نجس ابنته فيُقتلع من وسط كل إسرائيل، لأنه أعطى من نسله لمولك، وعمل شرًا لينجسها
\par 11 وأنت يا موسى، فأوصِ بني إسرائيل وحثهم أن لا يعطوا بناتهم للأمم، ولا يأخذوا لأبنائهم من بنات الأمم، لأن هذا مكروه لدى الرب.
\par 12 لهذا السبب كتبت لك في أقوال الشريعة جميع أعمال أهل شكيم التي فعلوها على دينة، وكيف تكلم بنو يعقوب قائلين: لا نعطي ابنتنا لرجل أغلف، لأن ذلك كان عارًا علينا
\par 13 وهو عار على إسرائيل وعلى الأحياء وعلى الذين يأخذون بنات الأمم، لأن هذا نجس ومكروه لإسرائيل
\par 14 ولن يكون إسرائيل خاليًا من هذه النجاسة إذا كانت له امرأة من بنات الأمم، أو أعطى إحدى بناته لرجل من أحد الأمم
\par 15 لأنه سيكون وباء على وباء، ولعنة على لعنة، وكل دينونة ووباء ولعنة ستأتي عليه. إن فعل هذا الأمر، أو حجب عينيه عن الذين يرتكبون النجاسة، أو الذين ينجسون قدس الرب، أو الذين ينتهكون اسمه القدوس، (حينئذٍ) ستُدان الأمة كلها معًا على كل نجاسة هذا الإنسان وتدنيسه
\par 16 ولا محاباة للأشخاص [ولا اعتبار للأشخاص]، ولا يُقبل من يديه ثمر ولا قرابين ولا محرقات ولا شحم، ولا رائحة رائحة سرور، حتى يقبلها. وهكذا يكون مصير كل رجل أو امرأة في إسرائيل ينجس المقدس
\par 17 لهذا السبب أمرتك قائلاً: «أشهد بهذه الشهادة لإسرائيل: انظر كيف كان حال أهل شكيم وأبنائهم: كيف أُسلموا إلى أيدي ابني يعقوب، فقتلوهما تحت التعذيب، فحُسب لهما ذلك برًا، وكُتب لهما برًا».
\par 18 واختير نسل لاوي للكهنوت، وليكونوا لاويين، ليخدموا أمام الرب مثلنا دائمًا، وليُبارك لاوي وأبناؤه إلى الأبد، لأنه كان غيورًا على إجراء البر والقضاء والانتقام على كل من قام على إسرائيل
\par 19 وهكذا يكتبون على الألواح السماوية شهادةً لصالحه البركة والبر أمام إله الجميع:
\par 20 ونذكر البر الذي عمله الإنسان في حياته، في جميع أوقات السنة؛ إلى أن يكتبوه لألف جيل، وسيأتي إليه وإلى نسله من بعده، وقد سُجِّل على الألواح السماوية صديقًا وبارًا
\par 21 كل هذا الكلام كتبته لك، وأمرتك أن تقول لبني إسرائيل: ألا يرتكبوا معصية، ولا يتجاوزوا الفرائض، ولا ينقضوا العهد الذي فُرض عليهم، بل أن يوفوا به ويُسجلوا أصدقاء
\par 22 "ولكن إن تعدوا وعملوا نجاسة بكل وجه، فسوف يكتبون على الألواح السماوية كأعداء، وسوف يبادون من سفر الحياة، وسوف يكتبون في سفر الذين يهلكون ومع الذين يستأصلون من الأرض.
\par 23 وفي اليوم الذي قتل فيه بنو يعقوب شكيم، سُجِّلت كتابة لصالحهم في السماء أنهم صنعوا برًا واستقامة وانتقامًا على الخطاة، وكُتبت للبركة
\par 24 وأخرجوا دينة أختهم من بيت شكيم، وسبوا كل ما في شكيم: غنمهم وبقرهم وحميرهم، وكل ثروتهم وكل مواشيهم، وأتوا بها كلها إلى يعقوب أبيهم
\par 25 ووبخهم لأنهم ضربوا المدينة بحد السيف، لأنه خاف من سكان الأرض، الكنعانيين والفرزيين
\par 26 وكان رعب الرب على جميع المدن التي حول شكيم، فلم يقوموا ليتبعوا بني يعقوب، لأن الرعب وقع عليهم

\chapter{31}

\par \textit{يعقوب يذهب إلى بيت إيل لتقديم ذبيحة، 1-3 (راجع تكوين 35: 2-4، 7، 14). إسحاق يبارك لاوي، 4-17، ويهوذا، 18-22. يعقوب يروي لإسحاق كيف أنعم الله عليه، 24. يعقوب يذهب إلى بيت إيل مع رفقة ودبورة، 26-30. يعقوب يبارك إله آبائه، 31-2.}

\par 1 وفي بداية الشهر، كلم يعقوب جميع أهل بيته قائلاً: «تطهروا وغيروا ثيابكم، ولنقم ونصعد إلى بيت إيل، حيث نذرت له نذرًا يوم هربت من وجه عيسو أخي، لأنه كان معي وأتى بي إلى هذه الأرض بسلام، وأزلوا الآلهة الغريبة التي بينكم».
\par 2 فتركوا الآلهة الغريبة وما كان في آذانهم وما كان على أعناقهم، والأصنام التي سرقتها راحيل من لابان أبيها، وأعطتها كاملة ليعقوب. فأحرقها وكسرها ودمرها، وأخفاها تحت البلوطة التي في أرض شكيم
\par 3 وصعد في هلال الشهر السابع إلى بيت إيل. وبنى مذبحًا في المكان الذي نام فيه، وأقام هناك عمودًا، وأرسل إلى أبيه إسحاق أن يأتي إليه ليذبح ذبيحته، وإلى أمه رفقة
\par 4 فقال إسحاق: دع ابني يعقوب يأتي فأراه قبل أن أموت
\par 5 وذهب يعقوب إلى أبيه إسحاق وأمه رفقة، إلى بيت أبيه إبراهيم، وأخذ معه اثنين من أبنائه، لاوي ويهوذا، وجاء إلى أبيه إسحاق وأمه رفقة
\par 6 فخرجت رفقة من البرج إلى أمامه لتقبل يعقوب وتعانقه، لأن روحها انتعشت عندما سمعت: «هوذا يعقوب ابنك قد جاء»، وقبلته
\par 7 فرأت ابنيه فعرفتهما وقالت له: هل هذان ابناك يا ابني؟ واحتضنتهما وقبلتهما وباركتهما وقالت: فيكما تتمجد نسل إبراهيم وتكونان بركة على الأرض.
\par 8 فدخل يعقوب على إسحاق أبيه، إلى الحجرة التي كان مضطجعًا فيها، وكان ابناه معه، فأمسك بيد أبيه وانحنى وقبله، فتمسك إسحاق بعنق يعقوب ابنه وبكى على عنقه
\par 9 فذهب الظلام عن عيني إسحاق، ورأى ابني يعقوب، لاوي ويهوذا، فقال: «هل هذان ابناك يا ابني؟ لأنهما مثلك».
\par 10 فقال له إنهم أبناؤه حقًا: «وقد رأيت حقًا أنهم أبنائي حقًا».
\par 11 فتقدما إليه، فالتفت وقبلهما واحتضنهما معًا
\par 12 فنزل روح النبوة في فمه، وأخذ لاوي بيمينه ويهوذا بيساره
\par 13 والتفت إلى لاوي أولاً، وبدأ يباركه أولاً، وقال له: ليباركك إله الكل، رب كل العصور، أنت وأولادك إلى الأبد
\par 14 وليعطِ الربُّ لك ولنسلك عظمةً ومجدًا عظيمًا، ويجعلك أنت ونسلك من بين كل ذي جسدٍ تقترب إليه لتخدم في قدسه كملائكة الحضور والقديسين. (حتى) مثلهم، يكون نسل أبنائك للمجد والعظمة والقداسة، ويجعلهم عظماء إلى كل العصور
\par 15 ويكونون قضاةً ورؤساءً ورؤساءً لجميع نسل بني يعقوب
\par    
\par     ويتكلمون بكلمة الرب بالبر،  
\par    ويحكمون بجميع أحكامه بالعدل.
\par    
\par    ويخبرون بطرقي ليعقوب  
\par    وسبلاتي إلى إسرائيل.
\par    
\par    تعطى بركة الرب في أفواههم  
\par    لتبارك كل بذور الحبيب.
\par    
\par 16 دعت أمك اسمك لاوي،  
\par     وبحق دعت اسمك؛
\par    
\par    ستكون مرتبطًا بالرب  
\par    وكن رفيقًا لجميع بني يعقوب؛
\par    
\par    لتكن مائدته لك،  
وتأكل أنت وبنوك منه؛
\par    
ولتكن مائدتك ممتلئة إلى كل الأجيال،  
ولا ينفد طعامك إلى كل الدهور.
\par    
\par 17 وَلْيَسْقُطْ أَمَامَكَ كُلُّ مَنْ يُغْضِبُونَكَ،  
\par     وَلِيُقْتَلَعْ وَيَهْلِكْ جَمِيعُ أَعْدَائِكَ
\par    
\par    وتبارك من يباركك،  
وتكون ملعونة كل أمة تلعنك.
\par    
\par 18 وقال ليهوذا:  
\par     'ليمنحك الرب القوة والقدرة
\par    
\par    لسحق كل من يكرهك؛  
\par      وتكون رئيسًا أنت وواحد من بنيك على بني يعقوب.
\par    
\par    ليخرج اسمك واسم أبنائك في كل أرض وكل منطقة.  
\par    حينئذٍ يخاف الأمم من وجهك،
\par    
\par    وترتجف كل الأمم  
\par    [وترتعد جميع الشعوب].
\par    
\par 19 فيك يكون عون يعقوب،  
\par     وفيك يوجد خلاص إسرائيل
\par    
\par 20 وعندما تجلس على عرش شرف برك  
\par     سيكون هناك سلام عظيم لجميع نسل أبناء الحبيب؛
\par    
\par    تبارك الذي يباركك،  
\par     وكل الذين يبغضونك ويضايقونك ويلعنونك  
\par    سيُقتلَع من الأرض ويُباد ويُلعن.
\par    
\par 21 فالتفت وقبله أيضًا واحتضنه، وفرح فرحًا عظيمًا، لأنه رأى أبناء يعقوب ابنه بصدق
\par 22 فخرج من بين رجليه وسقط وسجد له، وباركهم واضطجع هناك مع إسحاق أبيه تلك الليلة، فأكلوا وشربوا بفرح
\par 23 وأنام ابني يعقوب، واحدًا عن يمينه والآخر عن يساره، فحسب له ذلك برا
\par 24 وأخبر يعقوب أباه بكل شيء في الليل، كيف أظهر له الرب رحمة عظيمة، وكيف نجح في كل طرقه، وحفظه من كل شر
\par 25 وبارك إسحاق إله أبيه إبراهيم الذي لم يمنع رحمته وبره عن أبناء عبده إسحاق
\par 26 وفي الصباح أخبر يعقوب أباه إسحاق بالنذر الذي نذره للرب، والرؤيا التي رآها، وأنه بنى مذبحًا، وأن كل شيء جاهز للذبيحة أمام الرب كما نذر، وأنه جاء ليركبه على حمار
\par 27 فقال إسحاق ليعقوب ابنه: «لا أستطيع أن أذهب معك، لأني قد شخت ولا أقدر على الطريق. اذهب يا ابني بسلام، لأني ابن مئة وخمس وستين سنة هذا اليوم، لا أستطيع السفر بعد. اركب أمك (على حمار) ودعها تذهب معك».
\par 28 «وأنا أعلم يا بني أنك أتيت من أجلي، وليكن هذا اليوم مباركًا لأنك رأيتني حيًا، وقد رأيتك أنا أيضًا يا بني.»
\par 29 "فلتفلح وأوف بنذرك الذي نذرته، ولا ترجع عن نذرك لأنك ستسأل عن النذر، والآن أسرع إلى تنفيذه، وليرضى عن الذي صنع كل شيء، الذي نذرت له النذر".
\par 30 وقال لرفقة: اذهبي مع يعقوب ابنك. فذهبت رفقة مع يعقوب ابنها، ودبورة معها، وأتيتا إلى بيت إيل
\par 31 فتذكر يعقوب الصلاة التي باركه بها أبوه وابنيه لاوي ويهوذا، وفرح وبارك إله آبائه إبراهيم وإسحاق
\par 32 فقال: «الآن علمت أن لي رجاءً أبديًا، ولأبنائي أيضًا، أمام إله الكل». وهكذا قُدِّر الأمر بالنسبة لهما؛ وقد سجلا ذلك كشهادة أبدية لهما على الألواح السماوية كيف باركهما إسحاق

\chapter{32}

\par \textit{حلم لاوي في بيت إيل، 1. اختيار لاوي للكهنوت، باعتباره الابن العاشر، 2-3. احتفال يعقوب بعيد المظال وتقديم العشور من خلال لاوي: وكذلك العُشر الثاني، 4-9. قانون العشور المُرسَم، 10-15. رؤى يعقوب التي يقرأ فيها يعقوب على الألواح السماوية مستقبله ومستقبل نسله، 16-26. الاحتفال باليوم الثمانين من عيد المظال، 27-9. وفاة دبوراه، 30. ولادة بنيامين ووفاة راحيل، 33-4. (راجع تكوين 35: 8، 10، 11، 13، 16-20.)}

\par 1 وبات تلك الليلة في بيت إيل، فحلم لاوي أنهم رسّموه وجعلوه كاهنًا لله العلي هو وبنوه إلى الأبد، فاستيقظ من نومه وبارك الرب
\par 2 فبكر يعقوب في الصباح، في الرابع عشر من هذا الشهر، وأعطى عشر كل ما جاء معه، من الناس والبهائم، من الذهب وكل إناء وثوب، نعم، أعطى عشر كل شيء
\par 3 وفي تلك الأيام حملت راحيل بابنها بنيامين. فعدّ يعقوب بنيه من فوقه، فسقط لاوي في نصيب الرب، وألبسه أبوه ثياب الكهنوت وملأ يديه
\par 4 وفي اليوم الخامس عشر من هذا الشهر، قدم إلى المذبح أربعة عشر ثورًا من بين الماشية، وثمانية وعشرين كبشًا، وتسعة وأربعين خروفًا، وسبعة خراف، وواحدًا وعشرين جديًا من المعز، محرقة على مذبح الذبيحة، رائحة سرور أمام الله
\par 5 كان هذا قربانه، بناءً على النذر الذي نذره أن يُعطي العُشر، مع تقدمة ثمرهم وسكائبهم
\par 6 ولما أكلته النار، أوقد بخورا على النار فوق النار، وذبيحة شكر: ثورين وأربعة كباش وأربعة غنم، وأربعة تيوس، وخروفين حوليين، وجديين من المعزى. وكان يفعل هكذا كل يوم لمدة سبعة أيام
\par 7 وكان هو وجميع بنيه ورجاله يأكلون هذا هناك بفرح سبعة أيام، ويباركون ويشكرون الرب الذي أنقذه من كل ضيقه وأعطاه نذره
\par 8 وعشر جميع البهائم الطاهرة وأصعد محرقة، وأما البهائم النجسة فأعطاها لاوي ابنه، وأعطاه جميع نفوس الرجال.
\par 9 فتولى لاوي الكهنوت في بيت إيل أمام يعقوب أبيه مُفضّلاً على إخوته العشرة، وكان هناك كاهنًا، ونذر يعقوب نذره، فعشر أيضًا العشر للرب وقدّسه، فصار مقدسًا له
\par 10 ولهذا السبب، كُتب على الألواح السماوية كقانون لتقديم العشور مرة أخرى، ليؤكل العشور أمام الرب من سنة إلى سنة، في المكان الذي اختير أن يحل فيه اسمه، وليس لهذا القانون حد أيام إلى الأبد
\par 11 كُتبت هذه الفريضة لكي تُستكمل من سنة إلى سنة بأكل العُشر الثاني أمام الرب في المكان الذي اختير فيه، ولا يبقى منها شيء من هذه السنة إلى السنة التي تليها
\par 12 لأنه في سنته يؤكل البذر إلى أيام جمع بذر السنة، والخمر إلى أيام الخمر، والزيت إلى أيام موسمه
\par 13 وكل ما بقي منه وعتق، فليُعتبر نجسًا. ليُحرق بالنار، لأنه نجس
\par 14 وهكذا يأكلونه معًا في المقدس، ولا يدعوه يشيخ
\par 15 وتكون جميع أعشار البقر والغنم مقدسة للرب، وتكون لكهنته، يأكلونها أمامه من سنة إلى سنة. لأنه هكذا نُظِّم ونُقش بشأن العشور على الألواح السماوية
\par 16 وفي الليلة التالية، في اليوم الثاني والعشرين من هذا الشهر، عزم يعقوب على بناء ذلك المكان، وإحاطة الدار بسور، وتقديسها وجعلها مقدسة إلى الأبد، لنفسه ولأولاده من بعده
\par 17 وظهر له الرب ليلاً وباركه وقال له: «لا يُدعى اسمك يعقوب، بل إسرائيل يُسمون اسمك».
\par 18 فقال له أيضًا: «أنا الرب خالق السماء والأرض، وسأزيدك وأكثرك كثيرًا جدًا، ويخرج منك ملوك، ويحكمون في كل مكان وطأت فيه أقدام بني البشر».
\par 19 «وأعطي لنسلك كل الأرض التي تحت السماء، فيحكمون جميع الأمم حسب رغباتهم، وبعد ذلك يرثون الأرض كلها ويرثونها إلى الأبد.»
\par 20 ولما فرغ من الكلام معه، صعد من عنده، ونظر يعقوب حتى صعد إلى السماء
\par 21 ورأى في رؤيا الليل، وإذا ملاك نازِل من السماء وفي يديه سبعة ألواح، وأعطاها ليعقوب، فقرأها وعرف كل ما هو مكتوب فيها مما سيصيبه وأبنائه عبر العصور
\par 22 فأراه كل ما هو مكتوب على اللوحين، وقال له: لا تبنِ هذا المكان، ولا تجعله مقدسًا أبديًا، ولا تسكن هنا، لأنه ليس هذا المكان. اذهب إلى بيت إبراهيم أبيك، وأقم مع إسحاق أبيك إلى يوم وفاة أبيك.
\par 23 «لأنك في مصر تموت بسلام، وفي هذه الأرض تُدفن بإكرام في قبر آبائك مع إبراهيم وإسحاق.»
\par 24 «لا تخف، لأنه كما رأيت وقرأت، هكذا سيكون كل شيء؛ واكتب كل شيء كما رأيت وقرأت.»
\par 25 فقال يعقوب: يا رب، كيف أذكر كل ما قرأته ورأيته؟ فقال له: سأذكرك بكل شيء
\par 26 ثم صعد عنه، واستيقظ من نومه، فتذكر كل ما قرأه ورأه، وكتب كل الكلام الذي قرأه ورأه
\par 27 واحتفل هناك أيضًا بيوم آخر، وذبح فيه مثل كل ما ذبح في الأيام الأولى، ودعا اسمه "إضافة"، لأن هذا اليوم قد أُضيف، ودعا الأيام الأولى "عيدًا".
\par 28 وهكذا ظهر أنه ينبغي أن يكون، وهو مكتوب على الألواح السماوية: لذلك أُوحي إليه أن يحتفل به، ويضيفه إلى أيام العيد السبعة
\par 29 وسُميَ اسمُه جمعًا، لأنه كان يُسجَّل ضمن أيام الأعياد، حسب عدد أيام السنة
\par 30 وفي الليلة الثالثة والعشرين من هذا الشهر، ماتت دبوراه مرضعة رفقة، فدفنوها تحت المدينة تحت سنديانة النهر، فسمى اسم ذلك المكان «نهر دبوراه»، والسنديانة «سنديانة حزن دبوراه».
\par 31 فمضت رفقة ورجعت إلى بيتها إلى أبيه إسحاق، فأرسل يعقوب بيدها كباشًا وغنمًا ومعزى لتُعدَّ لأبيه طعامًا كما يريد
\par 32 وتبع أمه حتى وصل إلى أرض كبريتان، وأقام هناك
\par 33 فولدت راحيل ابنًا في الليل، ودعت اسمه «ابن حزني» لأنها تألمت في ولادته. أما أبوه فدعاه بنيامين، في الحادي عشر من الشهر الثامن، في الأسبوع الأول من الأسبوع السادس من هذا اليوبيل. [2143 صباحًا]
\par 34 وماتت راحيل هناك ودُفنت في أرض أفراتة، التي هي بيت لحم، فبنى يعقوب عمودًا على قبر راحيل، في الطريق فوق قبرها

\chapter{33}

\par \textit{رأوبين يخطئ مع بلهة، 1-9 (قارن تكوين 35: 21، 22). الشرائع المتعلقة بسفاح القربى، 10-20. أبناء يعقوب، 22. (قارن تكوين 35: 23-7.)}

\par 1 وذهب يعقوب وسكن في جنوب مجدلدرائف. وذهب إلى أبيه إسحاق هو وليئة امرأته في بداية الشهر العاشر
\par 2 ورأى رأوبين بلهة، جارية راحيل، سرية أبيه، تستحم بماء في مكان مخفي، فأحبها
\par 3 فاختبأ في الليل ودخل بيت بلهة فوجدها نائمة على سرير في بيتها وحدها.
\par 4 فاضطجع معها، فاستيقظت ونظرت، وإذا رأوبين مضطجع معها على الفراش، فكشفت طرف غطائها وأمسكته وصرخت، فاكتشفت أنه رأوبين
\par 5 فخجلت منه، وأطلقت يدها عنه، فهرب
\par 6 فَنَكَتَتْ عَلَى هَذَا الأَمْرِ نَأْمًا جِدًّا، وَلَمْ تُخْبِرْ بِهِ أَحَدًا
\par 7 فرجع يعقوب وطلبها، فقالت له: «لست طاهرة لك، لأني قد تنجست بك، لأن رأوبين نجسني واضطجع معي في الليل، وكنت نائمة، ولم أكتشف حتى كشف ذيلي واضطجع معي».
\par 8 فاغتاظ يعقوب على رأوبين غضبًا شديدًا لأنه اضطجع مع بلهة، لأنه كشف ذيل أبيه
\par 9 ولم يعد يعقوب يقترب منها لأن رأوبين نجسها. وكل رجل يكشف ثوب أبيه ففعله شرير جدًا، لأنه مكروه لدى الرب
\par 10 لهذا السبب كُتب وحُدد على الألواح السماوية أنه لا ينبغي للرجل أن يضطجع مع امرأة أبيه، ولا يكشف ذيل أبيه، لأن هذا نجس. يموتان معًا، الرجل الذي يضطجع مع امرأة أبيه والمرأة أيضًا، لأنهما صنعا نجاسة على الأرض
\par 11 ولا يكون شيء نجس أمام إلهنا في الأمة التي اختارها لنفسه ملكًا
\par 12 ومكتوب أيضًا مرة ثانية: «ملعون من يضطجع مع امرأة أبيه، لأنه قد كشف عورة أبيه». وقال جميع قديسي الرب: «ليكن، فليكن».
\par 13 وأنت يا موسى، فأمر بني إسرائيل أن يحفظوا هذه الكلمة؛ لأنها تستلزم عقوبة الموت؛ وهي نجس، وليس هناك كفارة إلى الأبد للتكفير عن الرجل الذي ارتكب هذا، بل يجب أن يُقتل ويُقتل ويُرجم بالحجارة ويُقتلع من وسط شعب إلهنا
\par 14 لأنه لا يجوز لأي إنسان يفعل ذلك في إسرائيل أن يبقى حيًا يومًا واحدًا على الأرض، لأنه رجس ونجس
\par 15 ولا يقولون: إن رأوبين أُعطي حياةً ومغفرةً بعد أن اضطجع مع سرية أبيه، ولها أيضًا مع أنها كانت متزوجة، وكان زوجها يعقوب أبوه لا يزال حيًا
\par 16 لأنه إلى ذلك الوقت لم يُكشف الحكم والحكم والناموس بكمالهما للجميع، إلا في أيامك (وقد أُعلن) كناموس للأزمنة والأيام، وناموس أبدي للأجيال الأبدية
\par 17 ولا يوجد لهذا القانون اكتمال أيام، ولا كفارة عنه، بل يجب استئصالهما من وسط الأمة: في اليوم الذي ارتكباه فيه يقتلونهما
\par 18 وأنت يا موسى اكتبها لإسرائيل لكي يحفظوها ويعملوا حسب هذه الكلمات ولا يرتكبوا خطية للموت لأن الرب إلهنا هو الديان الذي لا يحابي الوجوه ولا يقبل الهدايا.
\par 19 وأخبرهم بهذه كلمات العهد، لكي يسمعوها ويحفظوها، فيحذروا منها، فلا يُبادوا ويُقتلعوا من الأرض، لأن نجاسة ورجسًا ودنسًا وإفسادًا هم كل من يفعلونها على الأرض أمام إلهنا
\par 20 وليس هناك خطيئة أعظم من الزنا الذي يرتكبونه على الأرض؛ لأن إسرائيل أمة مقدسة للرب إلهها، وأمة ميراث، وأمة كهنوتية وملكية وملك خاص؛ ولن تظهر مثل هذه النجاسة في وسط الأمة المقدسة
\par 21 وفي السنة الثالثة من هذا الأسبوع السادس [2145 صباحًا] ذهب يعقوب وجميع أبنائه وسكنوا في بيت إبراهيم، عند إسحاق أبيه ورفقة أمه
\par 22 وهذه أسماء بني يعقوب: البكر رأوبين، شمعون، لاوي، يهوذا، يساكر، زبولون، أبناء ليئة؛ وبنو راحيل يوسف وبنيامين؛ وبنو بلهة دان ونفتالي؛ وبنو زلفة جاد وأشير؛ ودينة ابنة ليئة، ابنة يعقوب الوحيدة
\par 23 فجاءوا وسجدوا لإسحاق ورفقة، ولما رأوهم باركوا يعقوب وجميع بنيه، ففرح إسحاق فرحًا عظيمًا لأنه رأى بني يعقوب ابنه الأصغر فباركهم

\chapter{34}

\par \textit{حرب ملوك الأموريين ضد يعقوب وأبنائه، 1-9. يعقوب يرسل يوسف لزيارة إخوته، 10. بيع يوسف وسبي إلى مصر، 11-12 (قارن تكوين 37: 14، 17، 18، 25، 32-36). وفاة بلهة ودينة، 15. حزن يعقوب على يوسف، 13، 14، 17. تأسيس يوم الكفارة في اليوم الذي وصل فيه خبر وفاة يوسف، 18-19. زوجات أبناء يعقوب، 20-1.}

\par 1 وفي السنة السادسة من هذا الأسبوع من هذا اليوبيل الرابع والأربعين [2148 صباحًا] أرسل يعقوب أبناءه لرعي غنمهم، وعبيده معهم إلى مراعي شكيم
\par 2 فاجتمع عليهم ملوك الأموريين السبعة ليقتلوهم، ويختبئوا تحت الأشجار، ويغنموا مواشيهم
\par 3 وكان يعقوب ولاوي ويهوذا ويوسف في البيت مع إسحاق أبيهم، لأن روحه كانت حزينة، ولم يستطيعوا أن يتركوه. وكان بنيامين الأصغر، ولذلك بقي مع أبيه
\par 4 وجاء ملك/ملوك تافو، وملك/ملوك أريسا، وملك/ملوك سراجان، وملك/ملوك سيلو، وملك/ملوك جااس، وملك بيثورون، وملك معني ساكير، وكل من يسكن هذه الجبال، ومن يسكن الغابات في أرض كنعان
\par 5 فأخبروا يعقوب قائلين: هوذا ملوك الأموريين قد أحاطوا ببنيك ونهبوا مواشيهم
\par 6 وقام من بيته هو وأبناؤه الثلاثة وجميع عبيد أبيه وعبيده، وسار لملاقاتهم بستة آلاف رجل حاملين السيوف
\par 7 فقتلهم في مراعي شكيم، وطارد الهاربين، وقتلهم بحد السيف، وقتل أريسة وتفو وسريغان وسلو وأماني شاكر وجاجاس، واستعاد قطعانه
\par 8 وتغلب عليهم، وفرض عليهم الجزية ليدفعوا له الجزية، وهي خمسة من ثمار أرضهم، وبنى روبل وتمناتاريس
\par 9 ثم رجع بسلام، وصالحهم، فصاروا له عبيدًا إلى يوم نزوله هو وبنوه إلى مصر
\par 10 وفي السنة السابعة من هذا الأسبوع [2149 صباحًا] أرسل يوسف ليسأل عن أحوال إخوته من بيته إلى أرض شكيم، فوجدهم في أرض دوثان
\par 11 فغدروا به، وتآمروا عليه ليقتلوه، ولكنهم عدلوا عن رأيهم، وباعوه لتجار إسماعيليين، وأنزلوه إلى مصر، وباعوه لفوطيفار، خصي فرعون، رئيس الطباخين، كاهن مدينة إيليو
\par 12 فذبح بنو يعقوب جديًا، وغمسوا قميص يوسف في الدم، وأرسلوه إلى يعقوب أبيهم في العاشر من الشهر السابع
\par 13 فحزن طوال تلك الليلة، لأنهم أتوا به إليه في المساء، فحمى حزنًا على موته، وقال: وحش رديء أكل يوسف، وكان جميع أهل بيته حزينين معه في ذلك اليوم، وكانوا يحزنون وينوحون معه طوال ذلك اليوم
\par 14 فقام بنوه وابنته ليعزوه، فأبى أن يتعزى عن ابنه
\par 15 وفي ذلك اليوم سمعت بلهة أن يوسف قد هلك، فماتت حزينة عليه، وكانت ساكنة في كفر طف، وماتت أيضًا دينة ابنته بعد هلاك يوسف
\par 16 وجاءت هذه المناحات الثلاثة على إسرائيل في شهر واحد. فدفنوا بلهة مقابل قبر راحيل، ودفنوا دينة أيضًا ابنته هناك
\par 17 وناح على يوسف سنة واحدة ولم يكف، لأنه قال: دعني أنزل إلى القبر نائحًا على ابني
\par 18 لذلك، فُرض على بني إسرائيل أن يذلوا أنفسهم في العاشر من الشهر السابع - يوم وصول الخبر الذي بكى فيه يعقوب أبيه على يوسف - لكي يكفروا عن أنفسهم به بجدي من المعز في العاشر من الشهر السابع، مرة في السنة، عن خطاياهم؛ لأنهم أحزنوا شفقة أبيهم على يوسف ابنه
\par 19 وقد قُدِّم هذا اليوم لكي يحزنوا فيه على خطاياهم، وعلى كل ذنوبهم، وعلى كل زلاتهم، لكي يتطهروا فيه مرة كل سنة.
\par 20 وبعد هلاك يوسف اتخذ بنو يعقوب لأنفسهم نساء. اسم امرأة رأوبين عادا، واسم امرأة شمعون عدلبعة الكنعانية، واسم امرأة لاوي ملكة من بنات آرام من نسل بني تارح، واسم امرأة يهوذا بطاسوئيل الكنعانية، واسم امرأة يساكر حزقة، واسم امرأة زبولون نعمان، واسم امرأة دان عجلة، واسم امرأة نفتالي رسوعو من أرام النهرين، واسم امرأة جاد ماكا، واسم امرأة أشير يونة، واسم امرأة يوسف أسنات المصرية، واسم امرأة بنيامين ياساكا.
\par 21 فتاب شمعون، واتخذ زوجة ثانية من بلاد ما بين النهرين كأخوته

\chapter{35}

\par \textit{نصيحة رفقة ليعقوب ورده، 1-8. رفقة تطلب من إسحاق أن يُقسم عيسو أنه لن يؤذي يعقوب، 9-12. يوافق إسحاق، 13-17. يُقسم عيسو، وكذلك يعقوب، 18-26. موت رفقة، 27.}

\par 1 وفي السنة الأولى من الأسبوع الأول من اليوبيل الخامس والأربعين [2157 صباحًا]، دعت رفقة يعقوب ابنها، وأوصته بشأن أبيه وأخيه أن يكرمهما كل أيام حياته
\par 2 فقال يعقوب: «أفعل كل ما أمرتني به، لأن هذا الأمر يكون لي كرامة وعظمة وبرا أمام الرب لأكرمهم».
\par 3 وأنتِ أيضًا يا أمي، تعلمين منذ ولادتي حتى هذا اليوم، كل أعمالي وكل ما في قلبي، أنني دائمًا أفكر بالخير في كل شيء
\par 4 «وكيف لا أفعل هذا الأمر الذي أمرتني به لأكرم أبي وأخي!»
\par 5 «أخبريني يا أمي، ما هو الانحراف الذي رأيته فيّ، وسأبتعد عنه، وستكون عليّ الرحمة.»
\par 6 فقالت له: يا بني، لم أرَ فيك كل أيامي انحرافًا، بل أعمالًا صالحة. ومع ذلك، سأخبرك بالحق يا بني: سأموت هذا العام، ولن أنجو من هذا العام في حياتي؛ لأني رأيت في المنام يوم وفاتي، أنني لن أعيش أكثر من مئة وخمس وخمسين عامًا: وها أنا قد أكملت كل أيام حياتي التي سأعيشها
\par 7 فضحك يعقوب من كلام أمه، لأن أمه قالت له إنها ستموت، وكانت جالسة قبالته في قوتها، ولم تكن ضعيفة في قوتها، لأنها كانت تدخل وتخرج وتنظر، وأسنانها قوية، ولم يمسسها داء كل أيام حياتها
\par 8 فقال لها يعقوب: طوبى لي يا أمي إن اقتربت أيامي من أيام حياتك، وبقيت قوتي معي كقوتك، ولن تموتي، لأنكِ تمزحين معي بشأن موتك
\par 9 فدخلت على إسحاق وقالت له: «طلب واحد أسألك إياه: استحلف عيسو أنه لا يؤذي يعقوب ولا يطارده بالعداوة، لأنك تعلم أن أفكار عيسو ملتوية منذ صغره، وليس فيه صلاح، لأنه يريد بعد موتك أن يقتله».
\par 10 وأنت تعلم كل ما فعل منذ اليوم الذي ذهب فيه يعقوب أخوه إلى حاران إلى هذا اليوم: كيف تركنا بكل قلبه، وفعل بنا الشر. اقتنى لنفسه غنمك، ونهب كل مقتناك من أمام وجهك
\par 11 «وعندما توسلنا إليه وطلبنا منه ما لنا، فعل كإنسان يشفق علينا.»
\par 12 «وهو مرير عليكِ لأنكِ باركتِ يعقوب ابنكِ الكامل والبار؛ لأنه ليس فيه شر بل خير فقط، ومنذ جاء من حاران إلى هذا اليوم لم يسلبنا شيئًا، لأنه يأتينا بكل شيء في حينه دائمًا، ويفرح من كل قلبه عندما نأخذ من يديه ويباركنا، ولم يفارقنا منذ جاء من حاران إلى هذا اليوم، وهو يبقى معنا دائمًا في المنزل يكرمنا.»
\par 13 فقال لها إسحاق: «أنا أيضًا أعرف وأرى أعمال يعقوب الذي معنا، كيف يكرمنا من كل قلبه. أما أنا فقد أحببت عيسو سابقًا أكثر من يعقوب لأنه كان البكر. والآن أحب يعقوب أكثر من عيسو، لأنه عمل شرورًا كثيرة، وليس فيه بر، لأن كل طرقه إثم وعنف، [ولا بر حوله]».
\par 14 «والآن قلبي مضطرب بسبب جميع أعماله، ولن يخلص هو ولا نسله، لأنهم هم الذين سيُبادون من الأرض ويُقتلعون من تحت السماء، لأنه ترك إله إبراهيم وتبع زوجاته ونجاستهن وضلالهن، هو وأولاده.»
\par 15 «وأنت تطلب مني أن أحلفه أنه لن يقتل يعقوب أخاه؛ حتى لو أقسم فلن يفي بقسمه، ولن يفعل خيرًا بل شرًا فقط.»
\par 16 «ولكن إن أراد أن يقتل يعقوب أخاه في يدي يعقوب، يُعطى، ولا ينجو من يديه، [لأنه سينزل إلى يديه]».
\par 17 «ولا تخف من يعقوب، لأن ولي يعقوب عظيم وقوي ومكرم ومدح أكثر من ولي عيسو.»
\par 18 فأرسلت رفقة ودعت عيسو، فجاء إليها. فقالت له: «يا ابني، لي طلب أريد أن أطلبه منك، وأنت تعدني أن تفعله يا ابني».
\par 19 فقال: «سأفعل كل ما تقوله لي، ولن أرفض طلبك».
\par 20 فقالت له: أسألك أنه في يوم وفاتي تأخذني وتدفنني عند سارة أم أبيك، وأن تحب أنت ويعقوب بعضكما البعض ولا يريد أحدكما الشر للآخر بل المحبة المتبادلة فقط، وهكذا تنجحون يا أبنائي وتكرمون في الأرض ولا يفرح بكم عدو وتكونون نعمة ورحمة في عيون كل الذين يحبونكم.
\par 21 فقال: «سأفعل كل ما أمرتني به، وسأدفنك يوم وفاتك عند سارة أم أبي، كما أردت أن تكون عظامها قريبة من عظامك».
\par 22 «ويعقوب أخي أيضًا أحبه أكثر من كل جسد؛ لأنه ليس لي أخ في كل الأرض سواه وحده؛ وليس هذا فضلًا عظيمًا لي إن أحببته؛ لأنه أخي، وقد زُرعنا معًا في جسدك، وخرجنا معًا من بطنك، فإن لم أحب أخي، فمن أحب؟»
\par 23 وأنا نفسي أتوسل إليك أن تحرض يعقوب بشأني وبشأن أبنائي، لأني أعلم أنه سيكون ملكًا عليّ وعلى أبنائي، لأنه في اليوم الذي باركه أبي جعله هو الأعلى وجعلني الأدنى
\par 24 وأقسم لك أني سأحبه، ولن أرغب فيه شرًا كل أيام حياتي، بل الخير فقط
\par 25 فحلف لها على كل هذا الأمر. ودعت يعقوب أمام عيني عيسو، وأوصته حسب الكلام الذي كلمت به عيسو
\par 26 فقال: «سأفعل ما ترضاه؛ صدقني أنه لن يأتي شر مني أو من أبنائي على عيسو، وسأكون أول من لا شيء إلا في المحبة فقط».
\par 27 فأكلت وشربت هي وبنوها في تلك الليلة، وماتت في تلك الليلة، ابنة ثلاثة يوبيلات وأسبوع وسنة، فدفنها ابناها عيسو ويعقوب في المغارة المزدوجة بالقرب من سارة أم أبيهما

\chapter{36}

\par \textit{إسحاق يُعطي توجيهات لأبنائه بشأن دفنه: يحثهم على محبة بعضهم البعض ويجعلهم يلعنون الهلاك على من يؤذي أخاه، 1-11. يقسم ممتلكاته، ويعطي الجزء الأكبر ليعقوب، ويموت، 12-18. تموت ليا: يأتي أبناء يعقوب لتعزيته، 21-4.}

\par 1 وفي السنة السادسة من هذا الأسبوع [2162 صباحًا] دعا إسحاق ابنيه عيسو ويعقوب، فأتيا إليه، فقال لهما: «يا ابناي، أنا ذاهب في طريق آبائي، إلى البيت الأبدي حيث آبائي».
\par 2 «لذلك ادفنوني بالقرب من إبراهيم أبي، في المغارة المزدوجة في حقل عفرون الحثي، حيث اشترى إبراهيم قبرًا ليدفن فيه. في القبر الذي حفرت لنفسي، هناك ادفنوني.»
\par 3 «وهذا أوصيكم به يا أبنائي أن تعملوا البر والاستقامة على الأرض، لكي يُجري الرب عليكم كل ما تكلم الرب أنه سيفعله لإبراهيم ونسله».
\par 4 "وأحبوا بعضكم بعضاً يا أبنائي إخوتكم كما يحب الرجل نفسه، وليعمل كل واحد ما ينفع أخاه، وليعملوا معاً على الأرض، وليحبوا بعضهم بعضاً كما يحب كل واحد نفسه."
\par 5 «وأما فيما يتعلق بمسألة الأصنام، فإني آمرك وأنصحك برفضها وكرهها، وعدم محبتها، لأنها مليئة بالخداع لمن يعبدها ولمن يسجد لها.»
\par 6 «اذكروا يا أبنائي الرب إله إبراهيم أبيكم، كيف سجدتُ له وعبدته في بر وفرح، لكي يكثركم ويكثر نسلكم كنجوم السماء في الكثرة، ويثبتكم على الأرض كغرس البر الذي لا يُقتلع إلى الأجيال كلها إلى الأبد.»
\par 7 والآن سأُقسم لكم يمينًا عظيمًا - لأنه لا يوجد يمين أعظم منه باسم المجيد والمُشرَّف والعظيم والباهر والعجيب والقدير، الذي خلق السماوات والأرض وكل الأشياء معًا - أن تخافوه وتعبدوه
\par 8 «وأن يحب كل واحد أخاه بالمودة والبر، ولا يرغب أحدٌ بشرٍّ بأخيه من الآن وإلى الأبد كل أيام حياتكم، لكي تنجحوا في جميع أعمالكم ولا تهلكوا.»
\par 9 «وإن تآمر أحدكم على أخيه شرًا، فاعلموا أنه من الآن فصاعدًا كل من تآمر على أخيه شرًا يقع في يده، ويُقتلع من أرض الأحياء، ويُباد نسله من تحت السماء.»
\par 10 «ولكن في يوم الاضطراب واللعن والسخط والغضب، بنار ملتهبة آكلة كما أحرق سدوم، كذلك سيحرق أرضه ومدينته وكل ما له، وسيُمحى من سفر تأديب بني البشر، ولن يُكتب في سفر الحياة، بل في ذلك المُعد للهلاك، وسيُغادر إلى اللعن الأبدي؛ حتى تتجدد إدانتهم دائمًا في الكراهية واللعن والغضب والعذاب والسخط والأوبئة والمرض إلى الأبد.»
\par 11 «أقول وأشهد لكم يا أبنائي، حسب الدينونة التي تأتي على الرجل الذي يريد أن يؤذي أخاه.»
\par 12 «وقسم جميع ممتلكاته بين الاثنين في ذلك اليوم، وأعطى القسم الأكبر للبكر، والبرج وكل ما حوله، وكل ما ملك إبراهيم عند بئر القسم.»
\par 13 فقال: «هذا النصيب الأكبر سأعطيه للبكر».
\par 14 فقال عيسو: قد بعت ليعقوب وأعطيت بكوريتي ليعقوب. لتعط له. وليس لي فيها كلمة أقولها لأنها له.
\par 15 فقال إسحق: «لتكن البركة عليكم يا بني وعلى نسلكم اليوم، لأنكم أرحتموني، ولم يحزن قلبي على البكورية لئلا تعمل شراً بسببها».
\par 16 «ليبارك الله العلي الرجل الذي يعمل البر، هو ونسله إلى الأبد.»
\par 17 ثم انتهى من أمرهم وبركتهم، فأكلوا وشربوا معًا أمامه، وفرح لأنه كان بينهم رأي واحد، فخرجوا من عنده واستراحوا في ذلك اليوم وناموا
\par 18 ونام إسحاق على فراشه في ذلك اليوم فرحًا، ونام النوم الأبدي، ومات ابن مئة وثمانين سنة. وأكمل خمسة وعشرين أسبوعًا وخمس سنوات، ودفنه ابناه عيسو ويعقوب
\par 19 وذهب عيسو إلى أرض أدوم، إلى جبل سعير، وأقام هناك
\par 20 وسكن يعقوب في جبال حبرون، في برج أرض غربة أبيه إبراهيم، وكان يعبد الرب بكل قلبه وحسب الوصايا المنظورة، كما قسم أيام أجياله
\par 21 وماتت ليئة امرأته في السنة الرابعة من الأسبوع الثاني من اليوبيل الخامس والأربعين، [2167 صباحًا]، فدفنها في المغارة المزدوجة بالقرب من رفقة أمه على يسار قبر سارة أم أبيه
\par 22 وجاء جميع أبنائها وأبناؤه ليندبوا ليئة امرأته معه ويعزوه عنها، لأنه كان يندبها لأنه أحبها حبًا عظيمًا بعد وفاة راحيل أختها
\par 23 لأنها كانت كاملة ومستقيمة في جميع طرقها، وأكرمت يعقوب، ولم يسمع من فمها كلمة قاسية كل الأيام التي عاشتها معه، لأنها كانت وديعة ومسالمة ومستقيمة ومحترمة
\par 24 فتذكر جميع أعمالها التي عملت في حياتها، وندبها ندمًا شديدًا، لأنه أحبها من كل قلبه ومن كل نفسه

\chapter{37}

\par \textit{يوبخ أبناء عيسو على خضوعه ليعقوب، ويرغمونه على الحرب بمساعدة 4000 مرتزق ضد يعقوب، 1-15. يعقوب يوبخ عيسو، 16-17. رد عيسو، 18-25.}

\par 1 وفي اليوم الذي مات فيه إسحاق أبو يعقوب وعيسو، [2162 صباحًا] سمع أبناء عيسو أن إسحاق أعطى نصيب الأكبر لابنه الأصغر يعقوب، فغضبوا جدًا
\par 2 فخاصموا أباهم قائلين: لماذا أعطى أبوك يعقوب نصيب الأكبر وتجاوز عنك وأنت الأكبر ويعقوب الأصغر؟
\par 3 فقال لهم: «لأني بعت بكوريتي ليعقوب مقابل قليل من العدس، وفي اليوم الذي أرسلني فيه أبي لأصطاد وأمسك وأحضر له شيئًا ليأكله ويباركني، جاء بمكر وأحضر لأبي طعامًا وشرابًا، فباركه أبي وجعلني تحت يده».
\par 4 "والآن أقسم لنا أبونا أنا وهو أن لا نفكر في الشر ضد أخيه، وأن نستمر في المحبة والسلام كل واحد مع أخيه، ولا نجعل طرقنا فاسدة."
\par 5 فقالوا له: «لا نسمع لك لنُصالحه، لأن قوتنا أعظم من قوته، ونحن أقدر منه. سنذهب إليه ونقتله، ونُهلكه هو وبنيه. وإن لم تذهب معنا، نؤذيك أيضًا».
\par 6 «والآن اسمعوا لنا: لنرسل إلى أرام وفلسطين وموآب وعمون، ولنختر لأنفسنا رجالاً مختارين متحمسين للقتال، ولنذهب ضده ونحاربه، ولنبيده من الأرض قبل أن يقوى.»
\par 7 فقال لهم أبوهم: لا تذهبوا ولا تحاربوه لئلا تسقطوا أمامه
\par 8 فقالوا له: «هكذا أيضًا تفعل منذ شبابك إلى هذا اليوم، وأنت تضع عنقك تحت نيره».
\par 9 لن نسمع لهذه الكلمات. فأرسلوا إلى أرام، وإلى أدورام إلى صديق أبيهم، واستأجروا معهم ألف رجل محارب، رجال حرب مختارين
\par 10 فجاء إليهم من موآب ومن بني عمون، المأجورين، ألف رجل مختار، ومن فلسطين ألف رجل حرب مختار، ومن أدوم والحوريين ألف رجل حرب مختار، ومن كتيم رجال حرب أشداء
\par 11 فقالوا لأبيهم: اخرج معهم وقُدهم، وإلا قتلناك
\par 12 فامتلأ غضبًا وسخطًا حين رأى أن بنيه يُجبرونه على أن يسبقهم ليقودهم ضد يعقوب أخيه
\par 13 ولكن بعد ذلك تذكر كل الشر الذي كان مخفيًا في قلبه على يعقوب أخيه، ولم يذكر القسم الذي أقسم لأبيه وأمه أنه لا يفكر في شر كل أيامه على يعقوب أخيه
\par 14 ومع كل هذا، لم يكن يعقوب يعلم أنهم قادمون عليه للقتال، وكان يبكي على ليئة زوجته حتى اقتربوا جدًا من البرج ومعهم أربعة آلاف محارب ورجال حرب مختارون
\par 15 فأرسل إليه رجال حبرون قائلين: «هوذا أخوك قد جاء إليك ليحاربك، بأربعة آلاف متسلح بالسيف، وهم يحملون أتراسًا وسلاحًا»، لأنهم أحبوا يعقوب أكثر من عيسو. فأخبروه بذلك، لأن يعقوب كان رجلاً أكرم وأرحم من عيسو
\par 16 لكن يعقوب لم يصدق حتى اقتربا جدًا من البرج.
\par 17 ثم أغلق أبواب البرج، ووقف على الأسوار وكلم أخاه عيسو وقال: «ما أكرم التعزية التي جئت بها لتعزيني في زوجتي التي ماتت. أهذا هو القسم الذي أقسمت به لأبيك وأمك قبل موتهما؟ لقد نقضت القسم، وفي تلك اللحظة التي أقسمت فيها لأبيك حُكم عليك».
\par 18 فأجاب عيسو وقال له: «ليس لبني البشر ولا لوحوش الأرض يمين بر أقسموا به إلى الأبد، بل كل يوم يدبرون الشر بعضهم ضد بعض، وكيف يقتل كل واحد خصمه وعدوه.»
\par 19 وأنت تكرهني أنا وأولادي إلى الأبد. ولا مجال لربط الأخوة بك
\par 20 اسمع هذه الكلمات التي أقولها لك،
\par    
\par    إذا كان الخنزير قادرًا على تغيير جلده وجعل شعيراته ناعمة مثل الصوف،  
\par    أو إذا استطاع أن ينبت على رأسه قرونًا كقرون الأيل أو الشاة،  
\par    ثم سأحافظ على رابطة الأخوة معك  
وإن انفصلت الثديان عن أمهما، فأنت لم تكن لي أخا.  
\par    
\par 21 وإذا صالحت الذئاب الحملان لئلا تفترسها أو تعنفها،  
\par    وإن كانت قلوبهم نحوهم خيراً،  
\par    ثم سيكون هناك سلام في قلبي تجاهك
\par    
\par 22 وإذا أصبح الأسد صديقًا للثور وصالحه  
\par    وإن كان مقيدًا معه تحت نير واحد ويحرث معه،  
\par    حينئذٍ أصنع معك السلام.  
\par    
\par 23 وإذا صار الغراب أبيض كالرازة،  
\par     فاعلم أني أحببتك  
\par     ويصنع معك السلام  
\par    سوف يتم اقتلاعك،  
\par    وسيُقتلع أبناؤك،  
\par    ولن يكون لك سلام
\par    
\par 24 ولما رأى يعقوب أنه قد أساء إليه بقلبه وبكل نفسه ليقتله، وأنه جاء وثبًا كالخنزير البري الذي يأتي على الرمح الذي يطعنه ويقتله ولا يتراجع عنه؛
\par 25 ثم أمر خاصته وخدمه بمهاجمته هو وجميع رفاقه

\chapter{38}

\par \textit{الحرب بين يعقوب وعيسو. موت عيسو وهزيمة قواته، 1-10. خضوع أدوم للعبودية "حتى هذا اليوم"، 11-14. ملوك أدوم، 15-24. (قارن تكوين 36: 31-9)}

\par 1 وبعد ذلك كلم يهوذا يعقوب أبيه وقال له: "أحنِ قوسك يا أبي وأرسل سهامك وألقِ الخصم واقتل العدو، وليكن لك سلطان، لأننا لا نقتل أخاك، لأنه مثلك وهو مثلك، فلنعطه هذا الشرف".
\par 2 ثم ثنى يعقوب قوسه وأرسل السهم وضرب عيسو أخاه (في صدره الأيمن) فقتله
\par 3 ثم أرسل سهمًا أيضًا فأصاب أدوران الآرامي في صدره الأيسر، فدفعه إلى الوراء فقتله
\par 4 ثم خرج بنو يعقوب هم وعبيدهم وانقسموا فرقًا على الجوانب الأربعة للبرج
\par 5 فخرج يهوذا من الأمام، ونفتالي وجاد معه وخمسون خادمًا معه إلى الجانب الجنوبي من البرج، وقتلوا كل من وجدوه أمامهم، ولم ينجُ منهم أحد
\par 6 وخرج لاوي ودان وأشير إلى الجانب الشرقي من البرج، ومعهم خمسون رجلاً، وضربوا رجال موآب وعمون المحاربين
\par 7 وخرج رأوبين ويساكر وزبولون إلى الجانب الشمالي من البرج، ومعهم خمسون رجلاً، وضربوا رجال حرب الفلسطينيين
\par 8 وخرج شمعون وبنيامين وحنوك بن رأوبين إلى الجانب الغربي من البرج، وخمسون رجلاً معهم، وقتلوا من أدوم والحوريين أربعمائة رجل من أشداء الحرب، وهرب ستمائة رجل، وهرب معهم أربعة من أبناء عيسو، وتركوا أباهم مقتولاً كما سقط على التل الذي في أدورام
\par 9 وتبعهم بنو يعقوب إلى جبال سعير. ودفن يعقوب أخاه في التل الذي في عدورام، ثم رجع إلى بيته
\par 10 فضغط بنو يعقوب بشدة على بني عيسو في جبال سعير، وأحنوا أعناقهم حتى صاروا عبيدًا لأبناء يعقوب
\par 11 فأرسلوا إلى أبيهم يسألهم: هل يُصَالِحُونَ أم يَقْتُلُونَهم؟
\par 12 فأرسل يعقوب إلى بنيه أن يصنعوا الصلح، فصنعوا معهم الصلح، ووضعوا عليهم نير العبودية، فكانوا يؤدون الجزية ليعقوب ولبنيه كل الأيام
\par 13 واستمروا في دفع الجزية ليعقوب إلى يوم نزوله إلى مصر
\par 14 ولم يتخلص بنو أدوم من نير العبودية الذي فرضه عليهم أبناء يعقوب الاثنا عشر إلى هذا اليوم
\par 15 وهؤلاء هم الملوك الذين ملكوا في أدوم قبل أن يملك ملك على بني إسرائيل [إلى هذا اليوم] في أرض أدوم
\par 16 وملك بالاق بن بعور في أدوم، وكان اسم مدينته دانابا
\par 17 ومات بالاق، فملك مكانه يوباب بن زارح من بوصر
\par 18 ومات يوباب، فملك مكانه عاصم من أرض تيمان
\par 19 ومات آسام فملك مكانه عادات بن باراد الذي قتل مديان في بلاد موآب وكان اسم مدينته عويت.
\par 20 ومات عادث، فملك مكانه سلمان من عمصقا
\par 21 ومات سلمان، فملك مكانه شاول من رعبوت النهر
\par 22 ومات شاول، فملك مكانه بعلونان بن عكبور
\par 23 ومات بعلونان بن عكبور، فملكت عادات مكانه، وكان اسم امرأته مايطابيث بنت مطرة بنت متعبدزاب
\par 24 هؤلاء هم الملوك الذين ملكوا في أرض أدوم.

\chapter{39}

\par \textit{يوسف يُشرف على بيت فوطيفار، ١-٤. طهارته وسجنه، ٥-١٣. سجن كبير سقاة فرعون ورئيس خبازيه اللذين فسّر يوسف أحلامهما، ١٤-١٨. (قارن تكوين ٣٧: ٢؛ ٣٩: ٣-٨، ١٢-١٥، ١٧-٢٣؛ ٤٠: ١-٥، ٢١-٣؛ ٤١: ١.)}

\par 1 وسكن يعقوب في أرض غربة أبيه في أرض كنعان. هذه مواليد يعقوب
\par 2 وكان يوسف ابن سبع عشرة سنة لما أنزلوه إلى أرض مصر، فاشتراه فوطيفار، خصي فرعون، رئيس الطباخين
\par 3 وأقام يوسف على جميع بيته، فجاءت بركة الرب على بيت المصري بسبب يوسف، وأنجحه الرب في كل ما صنع
\par 4 فدفع المصري كل شيء إلى يد يوسف، لأنه رأى أن الرب معه، وأن الرب كان ينجحه في كل ما يصنع
\par 5 وكان منظر يوسف جميلاً [وكان منظره جميلاً جداً]، فرفعت امرأة سيده عينيها ورأت يوسف، فأحبته وطلبت إليه أن يضطجع معها
\par 6 لكنه لم يُسلم نفسه، بل تذكر الرب والكلام الذي كان يعقوب أبوه يقرأه من بين كلام إبراهيم، أنه لا ينبغي لرجل أن يزني مع امرأة لها زوج؛ وأن عقاب الموت قد قُدِّر عليه في السماوات أمام الله العلي، وستُسجَّل خطيئته في الكتب الأبدية دائمًا أمام الرب
\par 7 فتذكر يوسف هذا الكلام، وأبى أن يضطجع معها.
\par 8 فتوسلت إليه مدة سنة، فأبى ولم يسمع.
\par 9 فاحتضنته وأمسكته في البيت لتجبره على مضاجعة ابنها، وأغلقت أبواب البيت وأمسكته، فترك ثوبه في يديها وكسر الباب وهرب من وجهها إلى خارج
\par 10 فلما رأت المرأة أنه لا يريد أن يضطجع معها، وبّخته أمام سيده قائلة: «عبدك العبراني الذي تحبه طلب أن يغويني ليضاجع معي، وكان لما رفعت صوتي أنه هرب وترك ثوبه في يدي حين أمسكته، فكسر الباب».
\par 11 فنظر المصري ثوب يوسف والباب المكسور وسمع كلام امرأته، فألقى يوسف في السجن في المكان الذي كان المسجونون الذين حبسهم الملك.
\par 12 وكان هناك في السجن، فأعطى الرب يوسف نعمة في عيني رئيس سجّان السجن ورحمة أمامه، لأنه رأى أن الرب معه، وأن الرب كان يُنجح كل ما يصنع
\par 13 فدفع كل شيء إلى يديه، ولم يكن رئيس السجن يعلم شيئًا مما عنده، لأن يوسف كان يعمل كل شيء، والرب أكمله
\par 14 ومكث هناك سنتين. وفي تلك الأيام غضب فرعون ملك مصر على خصيّيه، رئيس السقاة، ورئيس الخبازين، فوضعهم في حبس بيت رئيس الطباخين، في السجن الذي كان يوسف محبوسًا فيه
\par 15 فأقام رئيس حراس السجن يوسف لخدمتهم، فخدم أمامهم
\par 16 وحلما كلاهما حلمًا، رئيس السقاة ورئيس الخبازين، وقصاه على يوسف
\par 17 وكما فسر لهم كذلك أصابهم، فرد فرعون رئيس السقاة إلى منصبه، وذبح رئيس الخبازين كما فسر لهم يوسف
\par 18 لكن رئيس السقاة نسي يوسف في السجن، مع أنه كان قد أخبره بما سيحدث له، ولم يتذكر أن يخبر فرعون بما أخبره به يوسف، لأنه نسي

\chapter{40}

\par \textit{أحلام فرعون وتفسيرها، 1-4. ترقية يوسف وزواجه، 5-13. (راجع تكوين 41: 1-5، 7-9، 14 وما يليه، 25، 29-30، 34، 36، 38-43، 45-6، 49.)}

\par 1 وفي تلك الأيام حلم فرعون حلمين في ليلة واحدة عن مجاعة ستكون في كل الأرض، فاستيقظ من نومه ودعا جميع مفسرين الأحلام الذين في مصر والسحرة، وقصّ عليهم حلميه، فلم يستطيعوا أن ينطقوهما
\par 2 فتذكر رئيس السقاة يوسف وكلم الملك عنه، فأخرجه من السجن، ورأى حلميه أمامه
\par 3 وقال أمام فرعون إن حلميه هما حلم واحد، وقال له: «ستأتي سبع سنين يكون فيها شبع على كل أرض مصر، وبعدها سبع سنين جوع، جوع لم يكن في كل الأرض».
\par 4 «والآن فليُوَكِّل فرعون نظارًا على كل أرض مصر، فيخزنوا طعامًا في كل مدينة طوال أيام سني الشبع، فيكون هناك طعام لسبع سنوات الجوع، ولا تهلك الأرض بسبب الجوع، لأنه سيكون شديدًا جدًا.»
\par 5 فأعطى الرب يوسف نعمة ورحمة في عيني فرعون، وقال فرعون لعبيده: «لا نجد رجلاً حكيماً وفطناً مثل هذا الرجل، لأن روح الرب معه».
\par 6 وجعله ثانياً في كل مملكته وأعطاه سلطاناً على كل مصر، وأركبه في مركبة فرعون الثانية.
\par 7 وألبسه ثيابًا صوفية، ووضع سلسلة من ذهب على عنقه، ونادى أمامه: «إيل إيل وعبيرر»، وجعل خاتمًا في يده، وجعله حاكمًا على كل بيته، وعظمه، وقال له: «فقط على العرش أكون أعظم منك».
\par 8 وكان يوسف متسلطًا على كل أرض مصر، وكان جميع رؤساء فرعون وجميع عبيده وجميع العاملين بأعمال الملك يحبونه، لأنه كان يسلك بالاستقامة، إذ لم يكن فيه كبرياء ولا غرور، ولم يكن يحابى الوجوه، ولم يكن يقبل الهدايا، بل كان يقضي بالاستقامة لجميع شعب الأرض
\par 9 وكانت أرض مصر في سلام أمام فرعون بسبب يوسف، لأن الرب كان معه، وأعطاه نعمة ورحمة إلى كل أجياله أمام كل من عرفه والسمع عنه. وكانت مملكة فرعون منظمة، ولم يكن فيها شيطان ولا إنسان شرير
\par 10 ودعا الملك اسم يوسف سفانطيفانس، وأعطى يوسف زوجةً ابنة فوطيفار، ابنة كاهن هليوبوليس، رئيس الطباخين
\par 11 ويوم وقف يوسف أمام فرعون كان ابن ثلاثين سنة
\par 12 وفي تلك السنة مات إسحاق. وكان كما قال يوسف في تفسير حلميه، حسب قوله، كانت سبع سنين شبعًا على كل أرض مصر، وأثمرت أرض مصر بكثرة، فأنتجت المكيال الواحد ألفًا وثمانمائة مكيال
\par 13 وجمع يوسف طعامًا في كل مدينة حتى امتلأت بالقمح حتى لم يعد بإمكانهم عدّه وقياسه لكثرته

\chapter{41}

\par \textit{أبناء يهوذا وثامار، 1-7. زنا المحارم بين يهوذا وثامار، 8-18. ثامار تلد توأمين، 21-2. غُفر ليهوذا، لأنه أخطأ بجهل وتاب عندما أُدين، ولأن زواج ثامار من أبنائه لم يكن قد اكتمل، 23-8. (راجع تكوين 38: 6-18، 20-6، 29-30؛ 41: 13.)}

\par 1 وفي اليوبيل الخامس والأربعين، في الأسبوع الثاني، وفي السنة الثانية، [2165 صباحًا] اتخذ يهوذا لبكره عيرًا، امرأة من بنات آرام اسمها ثامار
\par 2 لكنه كرهها ولم يضاجعها، لأن أمه كانت من بنات كنعان، وأراد أن يتزوج من قريبات أمه، ولكن يهوذا أبوه لم يدعه
\par 3 وكان عير، بكر يهوذا، شريرًا، فأماتَهُ الرب
\par 4 فقال يهوذا لأونان أخاه: «ادخل على امرأة أخيك، وأقم لها واجب أخ الزوج، وأقم نسلا لأخيك».
\par 5 وعلم أونان أن النسل ليس له بل لأخيه وحده، فدخل بيت امرأة أخيه وسكب البذر على الأرض، فساء في عيني الرب فأماتَهُ.
\par 6 فقال يهوذا لثامار كنته: «اجلسي في بيت أبيك أرملة حتى يكبر شيلة ابني فأعطيك له زوجة».
\par 7 فكبر، ولكن بشوئيل، امرأة يهوذا، لم تدع شيلة ابنها يتزوج. وماتت بشوئيل، امرأة يهوذا، في السنة الخامسة من هذا الأسبوع
\par 8 وفي السنة السادسة صعد يهوذا ليجز غنمه في تمنة. [2169 صباحًا] فقالوا لثامار: «هوذا حموك صاعد إلى تمنة ليجز غنمه».
\par 9 فخلعت ثياب ترملها، ووضعت عليها بُرقعًا، وتزينت، وجلست في الباب الذي بجانب طريق تمنة
\par 10 وفيما كان يهوذا ماشيًا وجدها، فظنها زانية، فقال لها: دعيني أدخل عليكِ. فقالت له: ادخل، فدخل
\par 11 فقالت له: أعطني أجرتي. فقال لها: ليس في يدي شيء إلا خاتمي الذي في إصبعي وقلادتي وعصاي التي في يدي
\par 12 فقالت له: أعطني إياها حتى ترسل لي أجرتي. فقال لها: أرسل إليك جديًا من المعزى. فأعطاها إياها، ودخل عليها، فحبلت منه
\par 13 وذهب يهوذا إلى غنمه، وذهبت هي إلى بيت أبيها
\par 14 فأرسل يهوذا جديًا من المعزى بيد راعيه العدلامي، فلم يجدها. فسأل أهل المكان قائلًا: «أين الزانية التي كانت هنا؟» فقالوا له: «ليس عندنا ها هنا زانية».
\par 15 فعاد فأخبره، وقال له إنه لم يجدها: سألت أهل المكان، فقالوا لي: لا زانٍ هنا
\par 16 فقال: «لتحفظهن لئلا نصبح سخرية». ولما أكملت ثلاثة أشهر، تبين أنها حبلى، فأخبروا يهوذا قائلين: «هوذا ثامار كنتك حبلى من الزنا».
\par 17 فذهبت يهوذا إلى بيت أبيها، وقالت لأبيها وإخوتها: «أخرجوها فيحرقوها، لأنها عملت نجاسة في إسرائيل».
\par 18 وكان لما أخرجوها ليحرقوها أنها أرسلت إلى حميها الخاتم والقلادة والعصا قائلة: «انظر لمن هذه، لأني حبلى منه».
\par 19 فاعترف يهوذا وقال: «ثامار أبر مني».
\par 20 «لذلك لا يحرقونها». ولذلك لم تُعطَ لشيلة، ولم يعد يقترب منها
\par 21 وبعد ذلك ولدت ابنين، فارص وزارح، في السنة السابعة من هذا الأسبوع الثاني.
\par 22 ثم تمت سبع سنوات الإثمار التي تحدث عنها يوسف لفرعون
\par 23 واعترف يهوذا بأن الفعل الذي فعله كان شريرًا، لأنه اضطجع مع كنته، فحَسِبَ ذلك كراهةً في عينيه، واعترف بأنه تعدى وضل، لأنه كشف ذيل ابنه، وبدأ يندب ويتضرع إلى الرب بسبب تعديه
\par 24 وأخبرناه في المنام أنه قد غفر له لأنه دعا بصدق وندّى ولم يعد إليه
\par 25 ونال الغفران لأنه رجع عن خطيئته وجهله، لأنه تجاوز حدًا كبيرًا أمام إلهنا. وكل من يفعل هكذا، كل من يضطجع مع حماته، فليحرقوه بالنار ليحترق بها، لأن عليه نجاسة ودنسًا، فليحرقوه بالنار
\par 26 وأنتَ تُوصِي بَني إِسرائيلَ أَنْ لاَ يَكُونَ نَجْسٌ بَيْنَهُمْ، لأَنَّ كُلَّ مَنْ اضْطَجَعَ مَعَ كُنَّتِهِ أَوْ مَعَ حَمَاتِهِ فَقَدْ فَعَلَ نَجْسًا. لِيَحْرِقُوا بِالنَّارِ الرَّجُلَ الَّذِي اضْطَجَعَ مَعَهَا، وَالْمَرْأَةَ أَيْضًا، فَيَرْفَعُ السَّخَطَ وَالْقَذَابَ عَنْ إِسْرَائِيلَ
\par 27 وقلنا ليهوذا إن ابنيه لم يضطجعا معها، ولهذا السبب ثُبِّت نسله لجيل ثانٍ ولن يُقتلع
\par 28 لأنه بنظرة ثاقبة ذهب وطلب العقاب، أي حسب حكم إبراهيم الذي أمر به أبنائه، سعت يهوذا إلى حرقها بالنار

\chapter{42}

\par \textit{بسبب المجاعة، أرسل يعقوب أبناءه إلى مصر للحصول على القمح، 1-4. تعرف عليهم يوسف واحتفظ بشمعون، وطلب منهم إحضار بنيامين عند عودتهم، 5-12. وعلى الرغم من تردد يعقوب، أخذ أبناؤه بنيامين معهم في رحلتهم الثانية واستضافهم يوسف، 13-25. (راجع تكوين 41: 54، 56؛ 42: 7-9، 13، 17، 20، 24-5، 29-30، 34-8؛ 43: 1-2، 4-5، 8-9، 11، 15، 23، 26، 29، 34؛ 44: 1-2.)}

\par 1 وفي السنة الأولى من الأسبوع الثالث من اليوبيل الخامس والأربعين، بدأت المجاعة تأتي إلى الأرض، ورفض المطر أن يُعطى للأرض، لأنه لم يسقط شيء منها
\par 2 وأصبحت الأرض قاحلة، ولكن في أرض مصر كان هناك طعام، لأن يوسف كان قد جمع بذار الأرض في سبع سني الشبع وحفظه
\par 3 فجاء المصريون إلى يوسف ليعطيهم طعامًا، ففتح مخازن القمح السنوي، وباعه لأهل الأرض بذهب
\par 4 (وكان الجوع شديدا جدا في أرض كنعان)، وسمع يعقوب أن في مصر طعاما، فأرسل أبناءه العشرة لكي يجدوا له طعاما في مصر. وأما بنيامين فلم يرسله، فجاء (أبناء يعقوب العشرة) بين الذين ذهبوا (إلى هناك).
\par 5 فعرفهم يوسف، لكنهم لم يعرفوه، فكلمهم وسألهم، وقال لهم: «أما أنتم جواسيس، وأتيتم لتتفحصوا نواحي الأرض؟» ووضعهم في الحبس
\par 6 وبعد ذلك أطلق سراحهم مرة أخرى، وأمسك بشمعون وحده، وأرسل إخوته التسعة
\par 7 فملأ أكياسهم قمحًا، وجعل ذهبهم في أكياسهم، فلم يعلموا
\par 8 وأمرهم بإحضار أخيهم الأصغر، لأنهم أخبروه أن أباهم وأخاهم الأصغر حيّان
\par 9 فصعدوا من أرض مصر وجاءوا إلى أرض كنعان، وأخبروا أباهم بكل ما أصابهم، وكيف كلمهم سيد الأرض بقسوة، وكيف أمسك شمعون حتى يأتي ببنيامين
\par 10 فقال يعقوب: «لقد حرمتموني من أولادي! يوسف ليس موجودًا، وشمعون أيضًا ليس موجودًا، وستأخذون بنيامين. عليّ جاء شركم».
\par 11 فقال: «لن ينزل ابني معكم لئلا يمرض، لأن أمهما ولدت ابنين، فهلك واحد، وهذا أيضًا تأخذونه مني. إن أصيب بحمى في الطريق، تنزلون شيخوختي بالحزن إلى الموت».
\par 12 لأنه رأى أن أموالهم قد أُعيدت إلى كل رجل في كَوْله، ولهذا السبب خشي أن يرسله
\par 13 واشتدت المجاعة واشتدت في أرض كنعان، وفي جميع الأراضي ما عدا أرض مصر، لأن كثيرين من أبناء المصريين كانوا قد خزنوا بذورهم للطعام منذ أن رأوا يوسف يجمع البذور ويضعها في المخازن ويحفظها لسنوات المجاعة
\par 14 وأكل شعب مصر منه في السنة الأولى من مجاعةهم
\par 15 ولما رأى إسرائيل أن الجوع شديد في الأرض، وأنه لا خلاص، قال لبنيه: «ارجعوا واشتروا لنا طعامًا حتى لا نموت».
\par 16 فقالوا: لا نذهب إلا إذا ذهب أخونا الأصغر معنا فلا نذهب
\par 17 ورأى إسرائيل أنه إن لم يرسله معهم، فسيهلكون جميعًا بسبب المجاعة
\par 18 فقال رأوبين: «سلمه إلى يدي، وإن لم أرجعه إليك فاقتل ابنيّ بدلًا من نفسه».
\par 19 فقال له: لا يذهب معك. فتقدم يهوذا وقال: أرسله معي، وإن لم أرجعه إليك، فدعني أحمل الذنب أمامك كل أيام حياتي
\par 20 فأرسله معهم في السنة الثانية من هذا الأسبوع في اليوم الأول من الشهر، فجاءوا إلى أرض مصر مع جميع الذين ذهبوا، وكان في أيديهم هدايا من السمسم واللوز والبطم والعسل النقي.
\par 21 فذهبوا ووقفوا أمام يوسف، فرأى بنيامين أخاه، فعرفه، فقال لهم: أهذا أخاكم الأصغر؟ فقالوا له: هو هو. فقال: لينعم عليك الرب يا ابني!
\par 22 فأرسله إلى بيته، فأخرج إليهم سمعان، فصنع لهم وليمة، فقدّموا له الهدية التي أتوا بها في أيديهم
\par 23 فأكلوا أمامه فأعطاهم جميعًا نصيبًا، فكان نصيب بنيامين أكبر من نصيب أي واحد منهم بسبعة أضعاف
\par 24 فأكلوا وشربوا وقاموا ومكثوا مع حميرهم.
\par 25 "ووضع يوسف خطة ليعرف أفكارهم هل تسود أفكار السلام بينهم، فقال للمدير الذي على بيته: "املأ كل أكياسهم طعامًا، وأرجع فضتهم إليهم في أوعيتهم، وكأسي، الكأس الفضية التي أشرب منها، ضعها في كيس الأصغر، وأطلقهم".

\chapter{43}

\par \textit{خطة يوسف للبقاء مع إخوته، 1-10. توسّل يهوذا، 11-13. يوسف يُعرّف نفسه لإخوته ويرسلهم إلى أبيه، 14-24. (راجع تكوين 44: 3-10، 12-18، 27-8، 30-2؛ 45: 1-2، 5-9، 12، 18، 20-1، 23، 25-8.)}

\par 1 ففعل كما قال له يوسف، وملأ كل عدالهم طعامًا، ووضع فضتهم في عدالهم، ووضع الكأس في عدال بنيامين
\par 2 وفي الصباح الباكر انطلقوا، ولما انصرفوا من هناك قال يوسف لصاحب بيته: «اتبعهم، اركض وأمسكهم، قائلاً: لقد كافأنيتموني بالشر خيرًا، لقد سرقتم مني كأس الفضة التي يشرب منها سيدي. وأرجع إليّ أخاهم الأصغر، وأحضره سريعًا قبل أن أخرج إلى كرسي قضائي».
\par 3 فركض وراءهم وقال لهم مثل هذا الكلام.
\par 4 فقالوا له حاشا لعبيدك أن يفعلوا هذا الأمر ويسرقوا من بيت سيدك شيئاً من الآنية. وأيضاً الفضة التي وجدناها في عدالنا أول مرة أحضرناها نحن عبيدك من أرض كنعان.
\par 5 «فكيف نسرق أي أداة إذن؟ ها نحن هنا وأكياسنا نبحث، وأينما تجد الكأس في كيس أي رجل من بيننا، فليُقتل، ونحن وحميرنا نخدم سيدك.»
\par 6 فقال لهم: «لا، بل الرجل الذي أجد عنده، هو وحده سآخذه خادمًا، وترجعون بسلام إلى بيتكم».
\par 7 وبينما كان يفتش في آنيتهم، مبتدئًا من الأكبر إلى الأصغر، وجد في جراب بنيامين
\par 8 فمزقوا ثيابهم وحملوا حميرهم ورجعوا إلى المدينة وجاءوا إلى بيت يوسف وسجدوا له كلهم ​​على وجوههم إلى الأرض.
\par 9 فقال لهم يوسف: لقد أسأتم. فقالوا: ماذا نقول وكيف ندافع عن أنفسنا؟ لقد اكتشف سيدنا ذنب عبيده. ها نحن عبيد سيدنا وحميرنا أيضًا
\par 10 فقال لهم يوسف: «أنا أيضًا أخاف الرب. أما أنتم فاذهبوا إلى بيوتكم وليكن أخاكم خادمًا لي، لأنكم فعلتم الشر. ألا تعلمون أن أحدًا يسر بكأسه كما يسرني بهذه الكأس؟ وقد سرقتموها مني».
\par 11 فقال يهوذا: يا سيدي، ليتكلم عبدك كلمة في أذن سيدي. ولدت أم عبدك لأبينا أخوين: أحدهما ذهب وكان ضالاً ولم يوجد، وهو وحده بقي لأمه، وعبدك أبونا يحبه، وحياته أيضًا مرتبطة بحياة هذا (الولد).
\par 12 «ويكون متى ذهبنا إلى عبدك أبينا والغلام ليس معنا أنه يموت، فننزل أبانا إلى الموت بحزن.»
\par 13 «الآن دعني أنا خادمك أمكث بدلًا من الغلام عبدًا لسيدي، وأطلق الغلام مع إخوته، لأني كفلته عند عبدك أبينا، وإن لم أعده، فسيسمع عبدك اللوم على أبينا إلى الأبد.»
\par 14 ورأى يوسف أنهم جميعًا مُتَوَافِقُونَ في الخير بعضهم مع بعض، فلم يستطع أن يكبح نفسه، فأخبرهم أنه يوسف
\par 15 وكان يُكلِّمهم باللغة العبرية، وسقط على أعناقهم وبكى
\par 16 لكنهم لم يعرفوه فبكوا. فقال لهم: لا تبكوا عليّ، بل أسرعوا وأحضروا أبي إليّ، وانظروا أن فمي هو الذي يتكلم، وعينا أخي بنيامين تبصران
\par 17 «فها هي السنة الثانية من المجاعة، ولا تزال هناك خمس سنوات بلا حصاد أو ثمار أشجار أو حرث.»
\par 18 انزلوا سريعًا أنتم وأهل بيوتكم، لئلا تهلكوا بالجوع، ولا تحزنوا على أموالكم، لأن الرب أرسلني أمامكم لأرتب الأمور ليحيا شعب كثير
\par 19 «وأخبروا أبي أني ما زلت حيًا، فترون أن الرب جعلني أبًا لفرعون، ومتسلّطًا على بيته وعلى كل أرض مصر.»
\par 20 «وأخبر أبي بكل مجدي، وبكل الغنى والمجد اللذين أعطاني إياهما الرب.»
\par 21 وبأمر فرعون أعطاهم مركبات ومؤنًا للطريق، وأعطاهم جميعًا ثيابًا ملونة وفضة
\par 22 وأرسل إلى أبيهم ثيابًا وفضةً وعشرة حمير حاملة قمحًا، فصرفهم
\par 23 فصعدوا وأخبروا أباهم أن يوسف حي، وأنه يقيس القمح لجميع أمم الأرض، وأنه هو المتسلط على كل أرض مصر.
\par 24 ولم يُصدِّق أبوهم ذلك، لأنه كان خارجًا عن عقله. ولكن عندما رأى العربات التي أرسلها يوسف، انتعشت روحه، وقال: «يكفيني أن يعيش يوسف. سأنزل وأراه قبل أن أموت».

\chapter{44}

\par \textit{يحتفل يعقوب بعيد الباكورة، ويتشجع برؤية ينزل إلى مصر، 1-10. أسماء نسله، 11-34. (راجع تكوين 46: 1-28.)}

\par 1 وارتحل إسرائيل من حاران من بيته في هلال الشهر الثالث، وذهب في طريق بئر القسم، وقدم ذبيحة لإله أبيه إسحاق في السابع من هذا الشهر
\par 2 فتذكر يعقوب الحلم الذي رآه في بيت إيل، فخاف أن ينزل إلى مصر
\par 3 وبينما كان يفكر في إرسال رسالة إلى يوسف ليأتي إليه، وأنه لن ينزل، مكث هناك سبعة أيام، عسى أن يرى رؤيا عما إذا كان يجب عليه البقاء أو النزول
\par 4 واحتفل بعيد حصاد باكورة الحبوب القديمة، لأنه في كل أرض كنعان لم يكن هناك حفنة بذر، لأن الجوع كان على جميع الحيوانات والبهائم والطيور، وكذلك على الناس
\par 5 وفي اليوم السادس عشر ظهر له الرب وقال له: «يعقوب، يعقوب». فقال: «ها أنا ذا». فقال له: «أنا إله آبائك، إله إبراهيم وإسحاق. لا تخف من النزول إلى مصر، لأني سأجعلك هناك أمة عظيمة. سأنزل معك، وسأصعدك (مرة أخرى)، وفي هذه الأرض ستُدفن، ويضع يوسف يديه على عينيك».
\par 6 «لا تخف، انزل إلى مصر.»
\par 7 فقام أبناؤه وأبناء أبنائه، ووضعوا أباهم وأمتعتهم على عربات
\par 8 وقام إسرائيل من بئر القسم في اليوم السادس عشر من هذا الشهر الثالث، وذهب إلى أرض مصر
\par 9 وأرسل إسرائيل يهوذا أمامه إلى ابنه يوسف ليتفقد أرض جاسان، لأن يوسف كان قد أخبر إخوته أن يأتوا ويسكنوا هناك ليكونوا بالقرب منه
\par 10 وكانت هذه أفضل (أرض) في أرض مصر، وأقرب إليه، للجميع، وللبهائم أيضًا
\par 11 وهذه أسماء أبناء يعقوب الذين دخلوا مصر مع يعقوب أبيهم
\par 12 رأوبين، بكر إسرائيل، وهذه أسماء أبنائه: حنوك، وفلو، وحصرون، وكرمي فايف
\par 13 شمعون وأبناؤه، وهذه أسماء أبنائه: يموئيل، ويامين، وأوهد، وياكين، وصوحر، وشاول ابن الصفاثية - سبعة
\par 14 لاوي وأبناؤه، وهذه أسماء أبنائه: جرشون، وقهات، ومراري، أربعة
\par 15 ويهوذا وبنوه وهذه أسماء بنيه شيلة وفارص وزارح الأربعة.
\par 16 يساكر وبنوه، وهذه أسماء بنيه: تولاع، وفوعة، وياشوب، وشمرون، خمسة
\par 17 زبولون وبنوه، وهذه أسماء بنيه: سارد، وإيلون، ويحلئيل
\par 18 وهؤلاء هم بنو يعقوب وبنوهم الذين ولدتهم ليعقوب في بلاد ما بين النهرين، ستة، وأختهم الوحيدة دينة، وجميع نفوس بني ليئة، وبنوهم الذين ذهبوا مع يعقوب أبيهم إلى مصر تسعة وعشرون، ويعقوب أبوهم معهم، كانوا ثلاثين
\par 19 وأبناء زلفة جارية ليئة امرأة يعقوب التي ولدت ليعقوب جاد وأشور
\par 20 وهذه أسماء أبنائهم الذين ذهبوا معه إلى مصر. بنو جاد: صفيون، وحجّي، وشوني، وأصبون (وعيري)، وأرئيلي، وأرودي
\par 21 وبنو أشير: يمنة، ويشوة (ويشوي)، وبريعة، وسارح، أختهم الوحيدة، ستة
\par 22 جميع النفوس كانت أربع عشرة، وجميع نفوس ليئة كانت أربعًا وأربعين.
\par 23 وابنا راحيل امرأة يعقوب يوسف وبنيامين.
\par 24 وولد ليوسف في مصر قبل أن يأتي أبوه إلى مصر، اللذان ولدتهما له أسنات ابنة فوطيفار كاهن هليوبوليس: منسى وأفرايم، ثلاثة
\par 25 وبنو بنيامين: بالع، وباكر، وأشبيل، وجيرا، ونعمان، وإيهي، وروش، ومفيم، وحفيم، وأرد الحادي عشر
\par 26 وكانت جميع نفوس راحيل أربع عشرة.
\par 27 وكان ابنا بلهة جارية راحيل امرأة يعقوب التي ولدتها ليعقوب دان ونفتالي.
\par 28 وهذه أسماء أبنائهم الذين ذهبوا معهم إلى مصر. وكان بنو دان: حوشيم، وسامون، وأسودي، ويعيق، وسليمان، وستة
\par 29 وماتوا في السنة التي دخلوا فيها مصر، وبقي لدان حوشيم وحده
\par 30 وهذه أسماء بني نفتالي: ياحصيئيل، وجوني، ويصر، وشلوم، وعاوي
\par 31 ومات إيف، الذي وُلد بعد سنوات المجاعة، في مصر.
\par 32 وكانت جميع نفوس راحيل ستاً وعشرين.
\par 33 وكانت جميع نفوس يعقوب التي دخلت مصر سبعين نفسًا. هؤلاء هم بنوه وبنات بنيه، جميعهم سبعون، ولكن مات خمسة في مصر قبل يوسف، ولم يكن لهم أولاد
\par 34 وفي أرض كنعان مات ابنان ليهوذا، عير وأونان، ولم يكن لهما أولاد، فدفن بنو إسرائيل من هلك، وأُحصوا من بين السبعين أمة من الأمم

\chapter{45}

\par \textit{يوسف يستقبل يعقوب، ويعطيه جاسان، 1-7. يوسف يستحوذ على كل الأرض وسكانها لفرعون، 8-12. يعقوب يموت ويُدفن في حبرون، 13-15. كتبه تُعطى للاوي، 16. (راجع تكوين 46: 28-30؛ 47: 11-13، 19، 20، 23، 24، 28؛ 13.)}

\par 1 فذهب إسرائيل إلى أرض مصر، إلى أرض جاسان، في رأس هلال الشهر الرابع، في السنة الثانية من الأسبوع الثالث من اليوبيل الخامس والأربعين.
\par 2 وذهب يوسف لملاقاة أبيه يعقوب إلى أرض جاسان، ووقع على عنق أبيه وبكى
\par 3 فقال إسرائيل ليوسف: دعني أموت الآن بعد أن رأيتك، والآن فليتبارك الرب إله إسرائيل، إله إبراهيم وإله إسحاق الذي لم يمنع رحمته ونعمته عن عبده يعقوب
\par 4 «يكفيني أن أرى وجهك وأنا بعد على قيد الحياة. نعم، إن الرؤيا التي رأيتها في بيت إيل حق. مبارك الرب إلهي من الدهر والأبد، ومبارك اسمه.»
\par 5 وأكل يوسف وإخوته خبزًا أمام أبيهم وشربوا خمرًا، وفرح يعقوب فرحًا عظيمًا جدًا لأنه رأى يوسف يأكل مع إخوته ويشرب أمامه، وبارك خالق كل شيء الذي حفظه، وحفظ له أبناءه الاثني عشر
\par 6 وأعطى يوسف لأبيه وإخوته هبةً حق السكنى في أرض جاسان وفي رعمسيس وفي كل الكورة المحيطة التي كان متسلطًا عليها أمام فرعون. وسكن إسرائيل وبنوه في أرض جاسان، أفضل أرض مصر، وكان إسرائيل ابن مئة وثلاثين سنة عندما جاء إلى مصر
\par 7 وأطعم يوسف أباه وإخوته وأملاكهم أيضًا بالخبز بقدر ما كفاهم سبع سنوات المجاعة
\par 8 وعانت أرض مصر من المجاعة، فامتلك يوسف كل أرض مصر لفرعون مقابل الطعام، وامتلك الشعب ومواشيهم وكل شيء لفرعون
\par 9 واستكملت سنو المجاعة، وأعطى يوسف للشعب في الأرض بذارًا وطعامًا ليزرعوا الأرض في السنة الثامنة، لأن النهر فاض على كل أرض مصر
\par 10 لأنه في سنوات المجاعة السبع، لم يفيض النهر، ولم يسقِ إلا أماكن قليلة على ضفاف النهر، أما الآن فقد فاض، وزرع المصريون الأرض، فأثمرت قمحًا كثيرًا في تلك السنة
\par 11 وكان هذا في السنة الأولى من [2178 صباحًا] الأسبوع الرابع من اليوبيل الخامس والأربعين
\par 12 فأخذ يوسف من قمح الحصاد الخمس للملك، وترك لهم أربعة أجزاء للطعام والبذار، وجعلها يوسف فريضة على أرض مصر إلى هذا اليوم
\par 13 وعاش إسرائيل في أرض مصر سبع عشرة سنة، وكانت كل أيام حياته ثلاثة يوبيلات، مئة وسبعًا وأربعين سنة، ومات في السنة الرابعة من الأسبوع الخامس من اليوبيل الخامس والأربعين
\par 14 وبارك إسرائيل بنيه قبل موته وأخبرهم بكل ما يصيبهم في أرض مصر، وأعلمهم بما يأتي عليهم في الأيام الأخيرة، وباركهم، وأعطى يوسف نصيبين في الأرض.
\par 15 واضطجع مع آبائه، ودُفن في المغارة المزدوجة في أرض كنعان، بالقرب من إبراهيم أبيه في القبر الذي حفره لنفسه في المغارة المزدوجة في أرض الخليل
\par 16 وأعطى جميع كتبه وكتب آبائه لاوي ابنه ليحفظها ويجددها لأولاده إلى هذا اليوم

\chapter{46}

\par \textit{ازدهار إسرائيل في مصر، 1-2. موت يوسف، 3-5. الحرب بين مصر وكنعان التي دُفنت خلالها عظام جميع أبناء يعقوب باستثناء يوسف في الخليل، 6-11. مصر تضطهد إسرائيل، 12-16. (راجع تكوين 1: 22، 25-26؛ خروج 1: 6-14.)}

\par 1 وحدث أنه بعد وفاة يعقوب، تكاثر بنو إسرائيل في أرض مصر، وصاروا أمة عظيمة، وكانوا على قلب واحد، حتى أحب الأخ أخاه، وساعد كل واحد أخاه، وتكاثروا كثيرًا وكثروا جدًا، عشرة أسابيع سنين، كل أيام حياة يوسف
\par 2 ولم يكن شيطان ولا شر كل أيام حياة يوسف التي عاشها بعد أبيه يعقوب، لأن جميع المصريين كانوا يكرمون بني إسرائيل كل أيام حياة يوسف
\par 3 ومات يوسف وهو ابن مئة وعشر سنين، فعاش في أرض كنعان سبع عشرة سنة، وعاش عشر سنين عبدًا، وثلاث سنين في السجن، وثمانين سنة تحت حكم الملك، متسلطًا على كل أرض مصر
\par 4 ومات هو وجميع إخوته وكل ذلك الجيل.
\par 5 وأوصى بني إسرائيل قبل وفاته أن يحملوا عظامه معهم عندما يخرجون من أرض مصر.
\par 6 واستحلفهم على عظامه، لأنه كان يعلم أن المصريين لن يعودوا ليُخرجوه ويدفنوه في أرض كنعان، لأن مقمرون ملك كنعان، وهو ساكن في أرض أشور، حارب في الوادي ملك مصر وقتله هناك، وطارد المصريين إلى أبواب حرمون
\par 7 لكنه لم يستطع الدخول، لأن ملكًا جديدًا آخر قد تولى ملك مصر، وكان أقوى منه، فعاد إلى أرض كنعان، وكانت أبواب مصر مغلقة، فلم يخرج أحد ولم يدخل أحد إلى مصر
\par 8 ومات يوسف في اليوبيل السادس والأربعين، في الأسبوع السادس، في السنة الثانية، فدفنوه في أرض مصر، ومات بعده جميع إخوته
\par 9 "وخرج ملك مصر للحرب على ملك كنعان [2263 ص.م] في اليوبيل السابع والأربعين في الأسبوع الثاني في السنة الثانية، وأخرج بنو إسرائيل كل عظام بني يعقوب إلا عظام يوسف، فدفنوها في الحقل في المغارة المزدوجة في الجبل."
\par 10 فرجع أكثرهم إلى مصر، وبقي منهم قليل في جبال حبرون، وبقي عمرام أبوك معهم
\par 11 وانتصر ملك كنعان على ملك مصر، وأغلق أبواب مصر
\par 12 ودبر مكيدة شريرة ضد بني إسرائيل لإيذائهم، وقال لشعب مصر: «هوذا شعب بني إسرائيل قد كثر وكثر أكثر منا».
\par 13 «هلموا فلنتعقل معهم قبل أن يكثروا، ولنستعبدهم قبل أن تأتي علينا الحرب وقبل أن يحاربونا هم أيضًا؛ وإلا فإنهم سينضمون إلى أعدائنا ويخرجونهم من أرضنا، لأن قلوبهم ووجوههم متجهة نحو أرض كنعان.»
\par 14 وجعل عليهم رؤساء تسخير ليستعبدوهم، فبنوا لفرعون مدنًا حصينة: فيثوم، ورعمسيس، وبنوا جميع الأسوار وجميع الحصون التي سقطت في مدن مصر
\par 15 واستعبدوهم بقسوة، وكلما أساءوا إليهم زادوا وتكاثروا
\par 16 وكره شعب مصر بني إسرائيل

\chapter{47}

ميلاد موسى، ١-٤. تبنيه ابنة فرعون، ٥-٩. قتل مصريًا وهرب (إلى مديان)، ١٠-١٢. (قارن خروج ١: ٢٢؛ ٢: ٢-١٥)

\par 1 وفي الأسبوع السابع، في السنة السابعة، في اليوبيل السابع والأربعين، خرج أبوك [2303 صباحًا] من أرض كنعان، وولدت في الأسبوع الرابع، في السنة السادسة منها، في [2330 صباحًا] اليوبيل الثامن والأربعين. كان هذا وقت الضيق على بني إسرائيل
\par 2 وأصدر فرعون ملك مصر أمرًا بشأنهم بأن يطرحوا جميع أولادهم الذكور المولودين في النهر
\par 3 وألقوهم سبعة أشهر حتى يوم ولادتك
\par 4 وأخفتك أمك ثلاثة أشهر، وأخبروا عنها. فصنعت لك فلكًا، وطلته بالزفت والأسفلت، ووضعته في الأعلام على شاطئ النهر، وأسكنتك فيه سبعة أيام، وكانت أمك تأتي ليلًا وترضعك، وفي النهار كانت مريم أختك تحرسك من الطيور
\par 5 وفي تلك الأيام، جاءت ثارموث، ابنة فرعون، لتغتسل في النهر، فسمعت صوتك تبكي، فقالت لعذارىها أن يخرجوكِ، فأتين بكِ إليها
\par 6 وأخرجتك من الفلك ورحمتك.
\par 7 فقالت لها أختك: هل أذهب وأدعو لك إحدى العبرانيات لترضع لك هذا الطفل وترضعه؟
\par 8 فقالت لها: اذهبي. فذهبت ودعت أمك يوكابد، وأعطتها أجرتها، فأرضعتك
\par 9 وبعد ذلك، لما كبرت، جاءوا بك إلى ابنة فرعون، فصرت لها ابنًا، وعلمك عمرام أبوك الكتابة، وبعد أن أكملت ثلاثة أسابيع جاءوا بك إلى البلاط الملكي
\par 10 وقضيتَ ثلاثة أسابيع من السنين في البلاط حتى الوقت [2351-] عندما خرجتَ من البلاط الملكي ورأيتَ مصريًا يضرب صديقك الذي كان [2372 صباحًا] من بني إسرائيل، فقتلته وأخفيته في الرمال
\par 11 وفي اليوم الثاني تخاصمت أنت واثنان من بني إسرائيل، وقلت للمخطئ: لماذا تضرب أخاك؟
\par 12 فغضب وغضب وقال: "من جعلك أميرًا وقاضيًا علينا؟ أتظن أنك ستقتلني كما قتلت المصري أمس؟" فخفت وهربت بسبب هذه الكلمات

\chapter{48}

\par \textit{موسى يعود من مديان إلى مصر. يسعى مستيما لقتله في الطريق، 1-3. الضربات العشر، 4-11. خروج إسرائيل من مصر: هلاك المصريين على البحر الأحمر، 12-19. (قارن خروج 2: 15؛ 4: 19، 24؛ 7: 12 وما يليه)}

\par 1 وفي السنة السادسة من الأسبوع الثالث من اليوبيل التاسع والأربعين، خرجتَ وأقمتَ (في [2372 AM] أرض مديان)، خمسة أسابيع وسنة واحدة. ورجعتَ إلى مصر في الأسبوع الثاني من السنة الثانية في اليوبيل الخمسين
\par 2 وأنتَ أنتَ تَعلَمُ ما كَلَّمَكَ بِهِ على [٢٤١٠ صباحًا] طُورِ سِينَاء، وما أرادَ الأميرُ مُستِمّا أن يفعلَ بِكَ عندما كُنتَ عائدًا إلى مِصْرَ (في الطَّريقِ عندما قابَلْتَهُ في المَنزل).
\par 3 ألم يسعَ بكل قوته إلى قتلك وإنقاذ المصريين من يدك عندما رأى أنك أُرسلت لتنفيذ الحكم والانتقام من المصريين؟
\par 4 وأنقذتك من يده، فصنعت الآيات والعجائب التي أُرسلت لتصنعها في مصر ضد فرعون، وضد كل بيته، وضد عبيده وشعبه
\par 5 وأجرى الرب عليهم نقمة عظيمة من أجل إسرائيل، فضربهم بالدم والضفادع والقمل والذباب والقروح الخبيثة التي تنفجر في البثور، ومواشيهم بالموت، وبحجارة البرد، فأهلك بذلك كل ما نبت لهم، وبالجراد الذي أكل البقايا التي تركها البرد، وبالظلام، وبموت أبكار الناس والحيوانات، وعلى جميع أصنامهم انتقم الرب وأحرقها بالنار
\par 6 "وكان كل شيء قد أرسل بيدك لكي تخبر (بهذه الأمور) قبل أن تتم، وتكلمت مع ملك مصر أمام جميع عبيده وأمام شعبه."
\par 7 وحدث كل شيء حسب أقوالك. وجاءت عشرة أحكام عظيمة ورهيبة على أرض مصر لتنتقم منها لإسرائيل
\par 8 وفعل الرب كل شيء من أجل إسرائيل، وحسب عهده الذي قطعه مع إبراهيم، لينتقم منهم كما استعبدوهم بالقوة
\par 9 ووقف الأمير مستيما ضدك، وسعى إلى إلقائك في يد فرعون، وساعد السحرة المصريين،
\par 10 وقاموا وفعلوا أمامك الشرور التي سمحنا لهم بفعلها، لكن لم نسمح لهم بفعل العلاجات بأيديهم
\par 11 فضربهم الرب بقرحات خبيثة، فلم يستطيعوا الوقوف، لأننا أهلكناهم حتى لم يتمكنوا من عمل آية واحدة
\par 12 وعلى الرغم من كل هذه الآيات والعجائب، لم يخجل الأمير مستيما، لأنه تشجع ونادى المصريين أن يطاردوك بكل قوات المصريين، وبمركباتهم، وبخيلهم، وبكل جيوش شعوب مصر
\par 13 ووقفتُ بين المصريين وإسرائيل، وأنقذنا إسرائيل من يده، ومن يد شعبه، وأخرجهم الرب في وسط البحر كما لو كان يابسة
\par 14 وجميع الشعوب الذين أتى بهم ليتبعوا إسرائيل، طرحهم الرب إلهنا في وسط البحر، في أعماق الهاوية تحت بني إسرائيل، كما طرح بنو مصر أطفالهم في النهر. وانتقم من مليون منهم، وهلك ألف رجل قوي ونشيط بسبب طفل رضيع واحد من أطفال شعبك ألقوه في النهر
\par 15 وفي اليوم الرابع عشر، والخامس عشر، والسادس عشر، والسابع عشر، والثامن عشر، تم تقييد الأمير مستيما وسجنه خلف بني إسرائيل حتى لا يتهمهم
\par 16 وفي اليوم التاسع عشر أطلقناهم ليساعدوا المصريين ويطاردوا بني إسرائيل
\par 17 فشدد قلوبهم وصلبها، ودبّر الرب إلهنا مكيدة ليضرب المصريين ويلقيهم في البحر
\par 18 وفي اليوم الرابع عشر قيدناه حتى لا يتهم بني إسرائيل يوم طلبوا من المصريين آنية وثيابًا، آنية فضة، وآنية ذهب، وآنية نحاس، لينهبوا المصريين مقابل العبودية التي أجبروهم على خدمتها
\par 19 وما أخرجنا بني إسرائيل من مصر فارغين

\chapter{49}

عيد الفصح: أحكام الاحتفال به. (راجع خروج ١٢: ٦، ٩، ١١، ١٣، ٢٢-٣، ٣٠، ٤٦؛ ١٥: ٢٢)

\par 1 اذكر الوصية التي أوصاك بها الرب بشأن الفصح، أن تعيده في وقته في الرابع عشر من الشهر الأول، وأن تذبحه قبل المساء، وأن يأكلوه ليلاً في مساء اليوم الخامس عشر من وقت غروب الشمس
\par 2 ففي تلك الليلة - بداية العيد وبداية الفرح - كنتم تأكلون الفصح في مصر، عندما أُطلقت جميع قوى المستيما لقتل كل بكر في أرض مصر، من بكر فرعون إلى بكر الأمة الأسيرة في الطاحونة، وحتى الماشية
\par 3 وهذه هي العلامة التي أعطاهم الرب: أن في كل بيت رأوا على أعتابه دم خروف حولي، لا يدخلون ذلك البيت ليذبحوا، بل يمرون به، لكي يخلص جميع الذين في البيت، لأن علامة الدم كانت على أعتابه
\par 4 ففعلت قوات الرب كل شيء كما أمرها الرب، ومرت على جميع بني إسرائيل، ولم يأتِ عليهم الوباء ليهلك منهم نفسًا واحدة، لا من بهيمة ولا من إنسان ولا من كلب
\par 5 وكان الطاعون شديدًا جدًا في مصر، ولم يكن هناك بيت في مصر إلا وفيه ميت، وكان البكاء والنحيب
\par 6 وكان جميع إسرائيل يأكلون لحم خروف الفصح ويشربون الخمر ويسبحون ويباركون ويحمدون الرب إله آبائهم، وكانوا على استعداد للخروج من تحت نير مصر ومن العبودية الشريرة
\par 7 واذكر هذا اليوم كل أيام حياتك، واحفظه من سنة إلى سنة كل أيام حياتك، مرة في السنة، في يومه، حسب كل شريعته، ولا تؤجله من يوم إلى يوم، أو من شهر إلى شهر
\par 8 لأنه فريضة أبدية، محفورة على الألواح السماوية لجميع بني إسرائيل، ليحفظوها كل عام في يومها مرة واحدة في السنة، طوال أجيالهم كلها؛ وليس هناك حد للأيام، لأن هذا مرسوم إلى الأبد
\par 9 وأما الرجل الذي يكون طاهرًا من النجاسة، ولا يأتي ليحتفل به في يومه، ليأتي بقربان مقبول أمام الرب، ويأكل ويشرب أمام الرب في يوم عيده، فإنه يُقطع ذلك الرجل الطاهر القريب. لأنه لم يقرب قربان الرب في وقته، فإنه يحمل إثمه على نفسه
\par 10 ليأتِ بنو إسرائيل ويحتفلوا بالفصح في يوم وقته، في اليوم الرابع عشر من الشهر الأول، بين المساءين، من ثلث النهار إلى ثلث الليل، لأن النهار يُعطى حصتين للضوء، وثلثًا للمساء
\par 11 هذا ما أمرك الرب أن تصنعه بين المساءين.
\par 12 ولا يجوز ذبحه في أي وقت من أوقات النور، إلا في الفترة التي تقارب المساء، ويأكلونه في وقت المساء إلى ثلث الليل، وما بقى من جميع لحمه من ثلث الليل فصاعدًا فليحرقوه بالنار
\par 13 ولا يطبخونه بالماء، ولا يأكلونه نيئًا، بل يشويونه على النار. يأكلونه بحرص، رأسه مع أحشائه، وأرجله يشويها بالنار، ولا يكسرون منه عظمًا. لأنه لا يسحق عظم من بني إسرائيل
\par 14 لهذا السبب أمر الرب بني إسرائيل أن يحفظوا الفصح في يوم عيده، ولا يكسروا منه عظمًا؛ لأنه يوم عيد، ويوم مأمور به، ولا يجوز أن يكون هناك تجاوز من يوم إلى يوم، أو من شهر إلى شهر، بل في يوم عيده فليُحفظ
\par 15 وأنتَ تُوصِي بني إسرائيلَ أن يُعيدوا الفصحَ طوالَ أيامِهم، كلَّ سنةٍ، مرةً في السنةِ في يومِ وقتِهِ، فيأتي تذكارًا مُرضيًا أمامَ الربِّ، ولا يُصيبُهم وباءٌ للقتلِ أو الضربِ في تلكَ السنةِ التي يُعيدونَ فيها الفصحَ في وقتِهِ من كلِّ وجهٍ حسبَ أمرِهِ
\par 16 ولا يأكلونه خارجًا عن مقدس الرب، بل أمام مقدس الرب، ويعيده جميع شعب جماعة إسرائيل في وقته
\par 17 وكل إنسان أتى في يومه يأكله في مقدس إلهك أمام الرب من ابن عشرين سنة فصاعدًا، لأنه هكذا هو مكتوب ومرسوم أن يأكلوه في مقدس الرب
\par 18 ومتى جاء بنو إسرائيل إلى الأرض التي سيمتلكونها، إلى أرض كنعان، ونصبوا مسكن الرب في وسط الأرض في أحد أسباطهم حتى يبنى مقدس الرب في الأرض، فليأتوا ويحتفلوا بالفصح في وسط مسكن الرب، ويذبحوه أمام الرب من سنة إلى سنة
\par 19 وفي الأيام التي يُبنى فيها البيت باسم الرب في أرض ميراثهم، يذهبون إلى هناك ويذبحون الفصح في المساء، عند غروب الشمس، في ثلث النهار
\par 20 ويقربون دمه على عتبة المذبح، ويضعون شحمه على النار التي على المذبح، ويأكلون لحمه مشويًا بالنار في دار البيت المقدس باسم الرب
\par 21 ولا يجوز لهم أن يحتفلوا بالفصح في مدنهم، ولا في أي مكان إلا أمام مسكن الرب، أو أمام بيته حيث سكن اسمه، ولا يضلون عن الرب
\par 22 "وأنت يا موسى، أمر بني إسرائيل أن يحفظوا فرائض الفصح كما أمرك. أعلمهم كل سنة ويوم أيامه وعيد الفطير، حتى يأكلوا فطيرًا سبعة أيام، ويحفظوا عيده، ويقدموا تقدمة كل يوم في تلك الأيام السبعة من الفرح أمام الرب على مذبح إلهكم.
\par 23 لأنكم احتفلتم بهذا العيد على عجل حين خرجتم من مصر حتى دخلتم برية شور، لأنكم أكملتموه على شاطئ البحر

\chapter{50}

\par \textit{القوانين المتعلقة باليوبيلات، 1-5، والسبت، 6-13.}

\par 1 وبعد هذه الشريعة عرفتك أيام السبوت في برية سيناء التي بين إيليم وسيناء
\par 2 وأخبرتك عن سبوت الأرض التي في جبل سيناء، وأخبرتك عن سنوات اليوبيل في سبوت السنين. ولكن لم أخبرك عن سنتها حتى تدخلوا الأرض التي ستمتلكونها
\par 3 وتحفظ الأرض أيضًا سبوتها ما داموا ساكنين فيها، ويعرفون سنة اليوبيل
\par 4 لذلك عيّنت لك السنة - الأسابيع والسنين واليوبيلات: هناك تسعة وأربعون يوبيلًا من أيام آدم إلى هذا اليوم، [2410 صباحًا] وأسبوع واحد وسنتان: ولا يزال هناك أربعون سنة قادمة (حرفيًا "بعيدة") لتعلم [2450 صباحًا] وصايا الرب، حتى يعبروا إلى أرض كنعان، ويعبروا الأردن غربًا
\par 5 وستمر اليوبيلات حتى يتطهر إسرائيل من كل ذنب زنا، ونجاسة، ودنس، وخطيئة، وخطأ، ويسكن بثقة في كل الأرض، ولن يكون هناك شيطان أو أي شرير بعد الآن، وتكون الأرض نقية من ذلك الوقت إلى الأبد
\par 6 وانظر إلى الوصية المتعلقة بالسبت - التي كتبتها لك - وجميع أحكام شرائعها
\par 7 ستة أيام تعمل، وأما اليوم السابع ففيه سبت للرب إلهك. لا تعمل فيه عملاً ما أنت وبنوك وعبيدك وإماؤك وكل بهائمك والنزيل الذي معك
\par 8 ويموت كل من عمل فيه عملاً: من دنس في ذلك اليوم، ومن اضطجع مع امرأته، أو من قال إنه سيعمل فيه شيئاً، فإنه يخرج فيه في سفر من أجل أي شراء أو بيع، ومن استقى منه ماءً لم يعده لنفسه في اليوم السادس، ومن حمل حملاً من خيمته أو من بيته ليحمله، يموت
\par 9 "لا تعملوا عملاً ما في يوم السبت إلا ما أعددتموه لأنفسكم في اليوم السادس للأكل والشرب والراحة، وحفظ السبت من كل عمل في ذلك اليوم، ولتباركوا الرب إلهكم الذي أعطاكم يوم عيد ويوم مقدس، ويوم ملك مقدس لكل إسرائيل هو هذا اليوم من أيامهم إلى الأبد."
\par 10 لأنَّ الكرامةَ التي أعطاها الربُّ لإسرائيلَ عظيمةٌ، أن يأكلوا ويشربوا ويشبعوا في يومِ العيدِ هذا، ويستريحوا فيه من كلِّ تعبٍ من تعبِ بني البشرِ ما عدا إحراقِ اللبانِ وتقديمِ القرابينِ والذبائحِ أمامَ الربِّ أيامًا وسبوتًا
\par 11 يُعمل هذا العمل وحده في أيام السبت في مقدس الرب إلهك، لكي يكفروا عن إسرائيل بذبيحة دائمة من يوم إلى يوم، تذكارًا مرضيًا أمام الرب، ولكي يقبلهم دائمًا من يوم إلى يوم كما أمرت
\par 12 وكل من عمل فيه، أو سافر، أو حرث مزرعته، سواء في بيته أو في أي مكان آخر، وكل من أشعل نارًا، أو ركب دابة، أو سافر في سفينة في البحر، وكل من ضرب أو قتل شيئًا، أو ذبح دابة أو طائرًا، أو من اصطاد حيوانًا أو طائرًا أو سمكة، أو من صام أو حارب في السبت:
\par 13 كل من يفعل شيئًا من هذه الأشياء في السبت يموت، فيحفظ بنو إسرائيل السبوت حسب الوصايا المتعلقة بسبوت الأرض، كما هو مكتوب في الألواح التي دفعها إلى يدي لأكتب لك شرائع الفصول، والأوقات حسب تقسيم أيامها
\par    
\par      بهذا يكتمل حساب تقسيم الأيام

\end{document}