\begin{document}

\title{نهاية العالم لآدم}

\chapter{1}

\par 1 كشف أصل آدم كما رواه لابنه شيث

\par 2 الوحي الذي علّمه آدم ابنه شيث في السنة السبعمائة، قائلاً: اسمع كلامي يا ابني شيث. عندما خلقني الله من الأرض مع حواء، أمك، تجولت معها في مجد رأته في الدهر الذي خرجنا منه. علمتني كلمة معرفة من الله الأزلي. وكنا نشبه الملائكة العظماء، لأننا كنا أعلى من الله الذي خلقنا والقوى التي معه، والتي لم نكن نعرفها

\par 3 ثم قسمنا الله، حاكم الدهور والقوى، في غضب. ثم أصبحنا دهرين. وتركنا المجد في قلوبنا، أنا وأمك حواء، مع المعرفة الأولى التي نفخت فينا. وهرب المجد منا؛ ليس من هذا الدهر الذي خرجنا منه، أنا وحواء أمك. لكن المعرفة دخلت في بذرة الدهور العظيمة. لهذا السبب دعوتك أنا نفسي باسم ذلك الرجل الذي هو بذرة الجيل العظيم أو الذي جاء منه. بعد تلك الأيام، انسحبت المعرفة الأبدية لإله الحق مني ومن أمك حواء. منذ ذلك الوقت تعلمنا عن الأشياء الميتة، مثل البشر. ثم تعرفنا على الله الذي خلقنا. لأننا لم نكن غرباء عن قدراته. وخدمناه في خوف وعبودية. وبعد هذه الأحداث أظلمت قلوبنا. الآن نمت في فكر قلبي

\par 4 ورأيت ثلاثة رجال أمامي لم أستطع التعرف على أشكالهم، لأنهم لم يكونوا من قوى الله الذي خلقنا. لقد فاقوا المجد، والبشر، يقولون لي: "قم يا آدم من نوم الموت واسمع عن الدهر ونسل ذلك الرجل الذي أتت إليه الحياة، الذي جاء منك ومن حواء، زوجتك"

\par 5 عندما سمعت هذه الكلمات من الرجال العظماء الواقفين أمامي، تنهدنا أنا وحواء في قلوبنا. ووقف أمامنا الرب الإله الذي خلقنا. وقال لنا: "يا آدم، لماذا كنتما تتنهدان في قلوبكما؟ ألا تعلمان أني أنا الإله الذي خلقكما؟ ونفخت فيكما روح حياة كنفس حية". ثم أظلمت أعيننا

\par 6 ثم خلق الله الذي خلقنا ابنًا منه ومن حواء، أمك. عرفتُ اشتياقًا حارًا لأمك. ثم تحطمت فينا قوة معرفتنا الأبدية، وطاردنا الضعف. لذلك أصبحت أيام حياتنا قليلة. لأني علمتُ أنني قد أصبحتُ تحت سلطان الموت

\par 7 والآن يا ابني شيث، سأكشف لك الأمور التي كشفها لي أولئك الرجال الذين رأيتهم قبلي أولاً بعد أن أكملت أزمنة هذا الجيل واكتمال سنين الجيل

\par 8 "فإن أمطار الله القدير ستُسكب لكي يُهلك كل جسد الله القدير، لكي يُهلك كل جسد من الأرض بواسطة ما حوله، مع أولئك الذين من نسل الرجال الذين نقلوا إليهم حياة المعرفة التي جاءت مني ومن حواء أمك. لأنهم كانوا غرباء عنه. وبعد ذلك سيأتي الملائكة العظماء على سحاب عالٍ، وسيُحضرون هؤلاء الرجال إلى المكان الذي تسكن فيه روح الحياة في المجد هناك. ثم يُترك كل جمع الجسد في المياه.

\par 9 حينئذٍ يستريح الله من غضبه. ويُلقي سلطانه على المياه، ويُعطي سلطانًا لسلطانه لأبنائه ونسائهم بواسطة الفلك، مع الحيوانات التي يشاء، وطيور السماء التي دعاها وأطلقها على الأرض. ويقول الله لنوح - الذي ستدعوه الأجيال ديوكاليون - "ها أنا قد حفظتك في الفلك مع امرأتك وأبنائك ونسائهم وحيواناتهم وطيور السماء التي دعوتها وأطلقتها على الأرض. لذلك سأعطيك الأرض - أنت وأبنائك. ستحكم عليها ملكيًا - أنت وأبنائك. ولن يأتي منك نسل من البشر الذين لن يقفوا أمامي في مجد آخر."

\par 10 ثم سيصبحون كسحابة النور العظيم. سيأتي أولئك الرجال الذين طُردوا من معرفة الدهر العظيم والملائكة. سيقفون أمام نوح والدهور. وسيقول الله لنوح: "لماذا حدت عما قلته لك؟ لقد خلقت جيلًا آخر حتى تحتقر قوتي". ثم سيقول نوح: "سأشهد أمام قوتك أن جيل هؤلاء الرجال لم يأتِ مني ولا من أبنائي".

\par 11 وسيُدخل هؤلاء الرجال إلى أرضهم الخاصة ويبني لهم مسكنًا مقدسًا. وسيُدعون بذلك الاسم ويسكنون هناك ستمائة عام في معرفة الخلود. وستسكن معهم ملائكة النور العظيم. ولن يسكن في قلوبهم أي عمل شرير، بل معرفة الإله الحقيقي فقط

\par 12 سيقسم نوح الأرض كلها بين أبنائه، حام ويافث وسام. سيقول لهم: "يا أبنائي، اسمعوا كلامي. ها أنا قد قسمت الأرض بينكم. لكن اعبدوه بخوف وعبودية كل أيام حياتكم. لا تدع نسلك يغيب عن وجه الله القدير. ستكون نسلي مرضية أمامك وأمام قدرتك. اختمها بيدك القوية بالخوف والأمر، حتى لا ينحرف كل النسل الذي خرج مني عنك وعن الله القدير، بل يخدم بتواضع وخوف من معرفته."

\par 13 ثم سيأتي آخرون من نسل حام ويافث، أربعمائة ألف رجل، ويدخلون أرضًا أخرى ويقيمون مع أولئك الرجال الذين خرجوا من المعرفة الأبدية العظيمة. لأن ظل قوتهم سيحمي من أقاموا معهم من كل شر ومن كل رغبة دنيئة. ثم سيشكل نسل حام ويافث اثنتي عشرة مملكة، وسيدخل نسلهم أيضًا مملكة شعب آخر، ويأخذون بمشورة الدهور العظيمة الخالدة. وسيذهبون إلى ساكلا، إلههم. أولئك الذين يدخلون إلى القوى، يتهمون العظماء الذين هم في مجدهم.

\par 14 سيقولون لساكلا: "ما هي قوة هؤلاء الرجال الذين وقفوا في حضرتك، والذين أُخذوا من نسل حام ويافث، والذين سيبلغ عددهم أربعمائة ألف رجل؟ لقد تم استقبالهم في دهر آخر خرجوا منه، وقلبوا كل مجد قدرتك وسيطرة يدك. لأن نسل نوح من خلال ابنه قد فعل كل إرادتك، وكذلك فعلت جميع القوى في الدهور التي تحكمها قدرتك، في حين أن هؤلاء الرجال والغرباء في مجدهم لم يفعلوا إرادتك. لكنهم ضللوا كل حشدك".

\par 15 ثم سيعطيهم إله الدهور بعضًا ممن يخدمونه. سيأتون إلى تلك الأرض حيث سيكون الرجال العظماء الذين لم يدنسوا، ولن يدنسوا بأي رغبة. لأن أرواحهم لم تأت من يد نجسة، بل جاءت من أمر عظيم للملاك الأبدي. ثم ستُلقى النار والكبريت والأسفلت على هؤلاء الرجال، وستحل النار والضباب المبهر على تلك الدهور، وستُظلم عيون قوى المنورين، ولن تراهم الدهور في تلك الأيام. وستنزل سحب النور العظيمة، وستنزل عليهم سحب أخرى من النور من الدهور العظيمة

\par 16 سينزل أبراساكس وسابلو وغاماليئيل ويخرجون هؤلاء الرجال من النار والغضب، ويأخذونهم فوق الدهور وحكام القوى، ويأخذونهم إلى هناك، مع الملائكة القديسين والدهور. سيكون الرجال مثل هؤلاء الملائكة، لأنهم ليسوا غرباء عنهم. لكنهم يعملون في البذرة الخالدة

\par 17 مرة أخرى، وللمرة الثالثة، سيمر مُنير المعرفة بمجد عظيم، ليترك شيئًا من نسل نوح وأبناء حام ويافث - ليترك لنفسه أشجارًا مثمرة. وسيُخلص أرواحهم من يوم الموت. لأن الخليقة كلها التي جاءت من الأرض الميتة ستكون تحت سلطان الموت. لكن أولئك الذين يتأملون في معرفة الله الأبدي في قلوبهم لن يهلكوا. لأنهم لم يتلقوا الروح من هذه المملكة وحدها، بل تلقوها من أحد الملائكة الأبديين. سيأتي المُنير. وسيصنع آيات وعجائب ليسخر من القوى وحاكمها

\par 18 "حينئذٍ ينزعج إله القوات قائلاً: ما هي قوة هذا الرجل الذي هو أعلى منا؟ حينئذٍ يثير غضبًا عظيمًا على ذلك الرجل. وتنسحب المجد وتسكن في البيوت المقدسة التي اختارتها لنفسها. ولن تراها القوات بأعينها، ولن يروا المنير أيضًا. حينئذٍ سيعاقبون جسد الإنسان الذي حل عليه الروح القدس.

\par 19 ثم سيستخدم الملائكة وجميع أجيال القوى الاسم خطأً، متسائلين: "من أين جاء الخطأ؟" أو "من أين جاءت كلمات الخداع التي فشلت جميع القوى في اكتشافها؟"

\par 20 الآن تقول المملكة الأولى عنه.....[]
\par 21 لقد تغذّى في السماوات.
\par 22 نال مجد ذلك الواحد والقوة.
\par 23 لقد جاء إلى حضن أمه.
\par 24 وهكذا وصل إلى الماء.
\par 25 وتقول المملكة الثانية عنه إنه جاء من عند نبي عظيم. وجاء طائر، وأخذ الطفل المولود، وحمله إلى جبل عالٍ. وتغذّى من طائر السماء. وخرج ملاك من هناك. وقال له: "قم! لقد مجدك الله".

\par 26 نال المجد والقوة.
\par 27 وهكذا وصل إلى الماء.
\par 28 تقول المملكة الثالثة عنه إنه خرج من رحم عذراء. طُرد هو وأمه من مدينته، ​​وأُحضر إلى مكانٍ قفر.

\par 29 لقد تغذّى هناك.
\par 30 وهكذا وصل إلى الماء.
\par 31 تقول المملكة الرابعة عنه إنه جاء من عذراء... فبحث عنها سليمان، هو وفرسالو وشاول وجيوشه الذين أُرسلوا. أرسل سليمان نفسه جيشه من الشياطين للبحث عن العذراء. فلم يجدوا من طلبوها، بل العذراء التي وهبتهم إياها. هي التي أخذوها. فأخذها سليمان. فحملت العذراء وولدت هناك.

\par 32 لقد غذته على حدود الصحراء. وبعد أن تغذى، نال المجد والقوة من البذرة التي وُلد منها. وهكذا جاء إلى الماء

\par 33 وتقول المملكة الخامسة عنه إنه جاء من قطرة من السماء. أُلقي في البحر. استقبلته الهاوية، وأنجبته، وأحضرته إلى السماء
\par 34 أخذ المجد والقدرة.
\par 35 وهكذا وصل إلى الماء.
\par 36 وتقول المملكة السادسة إنه [...] إلى الدهر الذي في الأسفل، لجمع الزهور. حملت من شهوة الزهور. أنجبته في ذلك المكان. غذّاه ملائكة حديقة الزهور. هناك نال المجد والقوة. وهكذا وصل إلى الماء.

\par 37 وتقول عنه المملكة السابعة إنه قطرة. جاء من السماء إلى الأرض. أنزلته التنانين إلى الكهوف. أصبح طفلاً. حلت عليه روح ورفعته إلى الأعلى إلى المكان الذي خرجت منه القطرة. هناك نال المجد والقوة. وهكذا جاء إلى الماء

\par 38 وتقول المملكة الثامنة عنه إن سحابةً حلّت على الأرض وغطّت صخرةً، فخرج منها، وغذّاه الملائكة الذين كانوا فوق السحابة، ونال المجد والقوة هناك، وهكذا وصل إلى الماء.

\par 39 وتقول المملكة التاسعة عنه إنه من بين الإلهامات التسع انفصلت واحدة. جاءت إلى جبل عالٍ وقضت بعض الوقت جالسة هناك، حتى إنها رغبت في أن تكون وحيدة لكي تصبح خنثى. حققت رغبتها وحملت من رغبتها. وُلد. أطعمته الملائكة الذين كانوا فوق الرغبة. ونال المجد والقوة هناك
\par 40 وهكذا وصل إلى الماء.
\par 41 تقول المملكة العاشرة عنه إن إلهه أحب سحابة من الشهوة. وولد بيده وألقى على السحابة التي فوقه بعضًا من القطرة، فولد. ونال المجد والقوة هناك
\par 42 وهكذا وصل إلى الماء.
\par 43 تقول المملكة الحادية عشرة عنه إن الأب اشتهى ​​ابنته. حملت هي نفسها من أبيها. ألقت [....] قبرًا في البرية. غذّاه الملاك هناك
\par 44 وهكذا وصل إلى الماء.

\par 45 وتقول المملكة الثانية عشرة عنه إنه جاء من منورين. وقد تغذى هناك
\par 46 أخذ المجد والقدرة.
\par 47 وهكذا وصل إلى الماء.

\par 48 وتقول المملكة الثالثة عشرة عنه إن كل ميلاد لحاكمها هو كلمة. وقد نالت هذه الكلمة تفويضًا هناك، ونالت المجد والقوة.
\par 49 وهكذا وصل إلى الماء.
\par 50 لكن الجيل الذي ليس عليه ملك يقول إن الله اختاره من بين جميع الدهور. لقد جعل معرفة الحقيقة النقية تأتي فيه. قال: "من هواء غريب، من دهر عظيم، خرج المُنير العظيم. وجعل جيل أولئك الرجال الذين اختارهم لنفسه يتألقون، حتى يتألقوا على الدهر بأكمله"

\par 51 حينئذٍ، فإن النسل، أولئك الذين سينالون اسمه على الماء واسم جميعهم، سيقاتلون القوة. وستحل عليهم سحابة من الظلام

\par 52 حينئذٍ تهتف الشعوب بصوت عظيم قائلين: "طوبى لنفوس أولئك الرجال لأنهم عرفوا الله بمعرفة الحق! سيحيون إلى الأبد، لأنهم لم يفسدوا بشهواتهم مع الملائكة، ولم يتمموا أعمال القوات، بل وقفوا في حضرته بمعرفة الله كالنور الذي خرج من النار والدم. أما نحن فقد عملنا كل عمل من أعمال القوات عبثًا. افتخرنا بمعصيتنا لجميع أعمالنا. صرخنا على إله الحق لأن كل عمله أبدي. هذه الأعمال ضد أرواحنا. لأننا الآن قد علمنا أن نفوسنا ستموت موتًا."

\par 53 ثم جاءهم صوت يقول "ميشو وميخا ومنسينوس، الذين هم على المعمودية المقدسة والماء الحي، لماذا كنتم تصرخون ضد الله الحي بأصوات آثمة وألسنة بلا قانون عليهم، ونفوس مليئة بالدماء وأعمال قذرة؟ أنتم ممتلئون بأعمال ليست من الحق، لكن طرقكم مليئة بالفرح والابتهاج. بعد أن دنستم ماء الحياة، جذبتموه داخل إرادة القوى التي أعطيت لكم لخدمتها. وفكركم ليس مثل تفكير أولئك الرجال الذين تضطهدونهم. ثمارهم لا تذبل. لكنهم سيعرفون حتى الدهور العظيمة، لأن الكلمات التي احتفظوا بها، عن إله الدهور، لم تُسلم إلى الكتاب، ولم تُكتب. لكن الكائنات الملائكية ستجلبهم، الذين لن تعرفهم جميع أجيال البشر. لأنهم سيكونون على الجبل العالي، على صخرة الحقيقة. لذلك سيُطلق عليهم "كلمات الخلود والحقيقة" لأولئك الذين "اعرفوا الله الأزلي بالحكمة والمعرفة وتعليم الملائكة إلى الأبد، لأنه يعلم كل شيء."

\par 54 هذه هي الوحي الذي أطلعه آدم على ابنه شيث، وعلم ابنه نسله عنها. هذه هي معرفة آدم الخفية، التي أعطاها لشيث، وهي المعمودية المقدسة لأولئك الذين يعرفون المعرفة الأبدية من خلال المولودين من الكلمة والمستنيرين الخالدين، الذين جاءوا من النسل المقدس: يشوع. مازاريوس يشوع، الماء الحي


\end{document}