\begin{document}


\title{سلم يعقوب}

\chapter{1}

\par 1 فذهب يعقوب إلى لابان عمه.

\par 2 وجد مكانًا، ووضع رأسه على حجر، ونام هناك، لأن الشمس كانت قد غربت

\par 3 رأى حلمًا. وإذا بسلم منصوب على الأرض، رأسه يمس السماء

\par 4 وكان أعلى السلم وجهًا كوجه إنسان منحوتًا من نار

\par 5 كانت هناك اثنتي عشرة درجة تؤدي إلى قمة السلم، وعلى كل درجة إلى الأعلى كان هناك وجهان بشريان، على اليمين وعلى اليسار، أربعة وعشرون وجهًا (أو تمثالًا نصفيًا) بما في ذلك صدورهم

\par 6 وكان الوجه في المنتصف أعلى من كل ما رأيت، وجه النار، بما في ذلك الكتفين والذراعين، أكثر رعبًا من تلك الوجوه الأربعة والعشرين

\par 7 وبينما كنت لا أزال أنظر إليه، إذا ملائكة الله يصعدون وينزلون عليه

\par 8 وكان الله واقفًا على أعلى وجهها، فناداني من هناك قائلًا: «يعقوب، يعقوب!» فقلت: «هأنذا يا رب!»

\par 9 وقال لي: «الأرض التي أنت نائم عليها لك أعطيها ولنسلك من بعدك

\par 10 وسأكثّر نسلك كنجوم السماء ورمل البحر

\par 11 ومن خلال نسلك، ستُبارك كل الأرض وكل من يعيش عليها في آخر سنوات اكتمالها

\par 12 بركتي ​​التي باركتك بها ستتدفق منك إلى الأبد
"في كل جيل يكون الشرق والغرب مملوءين من سبطك."

\chapter{2}

\par 1 ولما سمعتُ هذا من العُلى، وقعت عليَّ رعدةٌ وخوفٌ

\par 2 ثم نهضت من حلمي، والصوت لا يزال في أذني، فقلت:

\par 3 ما أرعب هذا المكان! ما هذا إلا بيت الله، وهذا باب السماء.

\par 4 وأقمت الحجر الذي كان وسادتي عموداً، وصببت زيتاً على رأسه، وسميت اسم ذلك المكان بيت الله.

\par 5 ووقفت وبدأت بالغناء وقلت:

\par 6 "الرب إله آدم مخلوقك و
\par الرب إله إبراهيم وإسحق آبائي
\par وكل الذين ساروا قبلك في البر!

\par 7 "أنت الذي تجلس بثبات على الكروبيم وعلى عرش المجد الناري... والعيون الكثيرة كما رأيت في حلمي،

\par 8 حاملاً الكروبيم ذو الوجوه الأربعة،
\par تحمل أيضًا السرافيم ذو العيون الكثيرة،

\par 9 حاملاً العالم كله تحت ذراعك،
\par ولكن لم يتحمله أحد؛

\par 10 أنت الذي ثبتت السماء لمجد اسمك.

\par 11 ممتدًا على سحابتين سماويتين السماء التي تتلألأ تحتك،

\par 12 لكي تجعل الشمس تجري تحتها وتخفيها أثناء الليل حتى لا تبدو وكأنها إله؛

\par 13 (أنت) الذي جعلت عليهم سبيلاً للقمر والنجوم؛

\par 14 وتجعل القمر يتزايد ويتناقص، وتقدر النجوم أن تمر
\par حتى لا يبدو أنهم آلهة أيضًا.

\par 15 من أمام وجه مجدك، يخاف السرافيم ذوو الأجنحة الستة، وهم
\par يغطون أقدامهم ووجوههم بأجنحتهم، أثناء الطيران بأجنحتهم.
\par أخرى (أجنحة)، وهم يغنون بلا انقطاع ترنيمة:

\par 16 '... الذي أقدسه الآن (ترنيمة) جديدة

\par 17 ذو اثني عشر رأسًا، ذو اثني عشر وجهًا، ذو أسماء كثيرة، ناري!
\par أيها القديس ذو العيون الساطعة!

\par 18 قدوس، قدوس، قدوس، ياو، ياوفا، ياويل، ياو،
\par كادوس، تشافود، سافاوث،

\par 19 أومليملك إيل أفير أميسيمي فاريش،
\par ملك أبدي، عظيم، قوي، عظيم للغاية،
\par المريض المبارك!'

\par 20 أنت الذي تملأ السماء والأرض والبحر والهاوية
\par وكل العصور بمجدك،

\par 21 اسمع أغنيتي التي غنيتها لك، وامنحني الطلب الذي أطلبه منك،

\par 22 وأخبرني بتفسير حلمي،
\par لأنك إله عظيم، قادر، ومجيد،
"أنا إله قدوس، ربي ورب آبائي."

\chapter{3}


\par 1 وبينما كنتُ أردد هذه الصلاة، إذا بصوتٍ أمام وجهي يقول:

\par 2 «يا سريئيل، قائد المخدوعين، يا من هو المسؤول عن الأحلام، اذهب وأفهم يعقوب معنى الحلم الذي رآه، واشرح له كل ما رآه؛ ولكن باركه أولًا.»

\par 3 وجاء إليّ سريئيل رئيس الملائكة ورأيته، وكان منظره جميلاً ومهيباً جداً

\par 4 لكنني لم أندهش من مظهره، لأن الرؤيا التي رأيتها في حلمي كانت أفظع منه

\par 5 ولم أخشَ رؤيا الملاك.

\chapter{4}

\par 1 فقال لي الملاك: ما اسمك؟

\par 2 فقلت: يعقوب.

\par 3 (وأعلن) «لا يُدعى اسمك بعد يعقوب، بل يكون اسمك مثل اسمي إسرائيل».

\par 4 "وحين كنت ذاهباً من فندنا السورية للقاء عيسو أخي، جاء إلي وباركني ودعاني إسرائيل."

\par 5 ولم يخبرني باسمه حتى أحلف له.

\par 6 ثم قال لي: "بما أنك كنتَ تُبقي زولًا..."

\chapter{5}

\par 1 فقال لي: لقد رأيت سلماً من اثنتي عشرة درجة، في كل درجة وجهان إنسانيان يتغير مظهرهما باستمرار.

\par 2 السلم هو هذا العصر،

\par 3 والخطوات الاثنتي عشرة هي فترات هذا العصر.

\par 4 وأما الوجوه الأربعة والعشرون فهم ملوك الأمم الفاسقين في هذا الدهر.

\par 5 وفي عهد هؤلاء الملوك يتم استجواب أبناء أبنائك وأجيال أبنائك.

\par 6 هؤلاء سوف يقومون على إثم بنيك.

\par 7 وسيُصبح هذا المكان مهجورًا بسبب الصعودات الأربعة ... بسبب خطايا أحفادك

\par 8 ويُبنى حول ممتلكات آبائك قصر، هيكل باسم إلهك وإله آبائك،

\par 9 وفي استفزازات أطفالك، ستصبح مهجورة في الصعودات الأربعة لهذا العصر

\par 10 لأنكم رأيتم التماثيل الأربعة الأولى التي كانت تصطدم بالدرج ...

\par 11 الملائكة تصعد وتنزل، والتماثيل النصفية وسط الدرجات.

\par 12 ويقيم العلي ملوكاً من بني عيسو أخيك، فيأخذون جميع عظماء قبائل الأرض الذين أساؤوا إلى نسلك.

\par 13 ويُسَلَّمون إلى يديه فيُضطَرّ بهم.

\par 14 فيُمسكهم بالقوة ويتسلط عليهم، ولن يتمكنوا من مقاومته حتى يأتي اليوم الذي تخرج فيه أفكاره عليهم ليعبدوا الأصنام ويقدموا ذبائح الأموات

\par 15 . . . (سيُمارس) العنف على جميع من في مملكته ممن سينكشف عن ذنبهم، سواءً أعلى رجل من قبيلتك أو كفالكوناغارغايليويا

\par 16 اعلم يا يعقوب أن نسلك سيكونون منفيين في أرض غريبة، وسيذلونهم بالعبودية ويصيبونهم بالجراح كل يوم

\par 17 لكن الرب سيدين الشعب الذي يستعبدون له.

\chapter{6}

\par 1 «وعندما يقوم الملك، سيأتي الحكم أيضًا على ذلك المكان.»

\par 2 "حينئذ يخرج نسلك إسرائيل من عبودية الأمم التي تغتصبه، ويكون حراً من كل توبيخ من أعدائك."

\par 3 لأن هذا الملك هو رأس كل انتقام وانتقام من الذين أساءوا إليك يا إسرائيل، وإلى نهاية العالم

\par 4 لأن المُرّين سيقومون، ويصرخون، فيسمعهم الرب ويقبل تضرّعهم

\par 5 وسيتوب القدير من آلامهم.

\par 6 فإن الملائكة ورؤساء الملائكة سيقذفون عليهم صواعقهم من أجل خلاص قبيلتك.

\par 7 وستنالون رحمة العلي.

\par 8 ثم تلد زوجاتهم أولاداً كثيرين.

\par 9 وبعد ذلك، سيُحارب الرب عن سبطكم بآيات عظيمة ورهيبة ضد الذين استعبدوهم

\par 10 ملأ مخازنهم، وستُوجد فارغة.

\par 11 كانت أرضهم مليئة بالزواحف وكل أنواع الأشياء المميتة.

\par 12 ستكون هناك زلازل وخسائر فادحة.

\par 13 ويصب الرب غضبه على ليفياثان تنين البحر، ويقتل الصقر الشرير بالسيف، لأنه يثير غضب إله الآلهة بكبريائه.

\par 14 "وحينئذٍ يظهر برّك يا يعقوب وبرّ أولادك الذين يكونون من بعدك والذين يسلكون في برّك. وحينئذٍ يقرع نسلك الباب، وتبيد مملكة أدوم كلها مع جميع شعوب موآب."

\chapter{7}

\par 1 «وأما الملائكة الذين رأيتهم ينزلون ويصعدون على السلم،

\par 2 في السنوات الأخيرة سيكون هناك رجل من الأعلى، وسيرغب في ضم الأشياء العليا إلى الأشياء السفلى.

\par 3 وقبل مجيئه يخبر عنه أبناؤكم وبناتكم، ويرى شبابكم رؤى عنه.

\par 4 وهذه هي العلامات في وقت مجيئه:

\par 5 الشجرة المقطوعة بالفأس سوف تنزف؛

\par 6 الأطفال في عمر الثلاثة أشهر سيتحدثون بلغة مفهومة؛

\par 7 الجنين في بطن أمه يتكلم عن طريقه.

\par 8 سيكون الشاب مثل الشيخ.

\par 9 ثم يأتي المنتظر الذي لن يلاحظ أحد طريقه.

\par 10 "حينئذ تتمجد الأرض وتتلقى مجداً سماوياً.

\par 11 ما كان أعلاه سيكون أدناه أيضًا.

\par 12 ومن نسلك ينبت أصل الملوك.

\par 13 سيظهر ويطيح بقوة الشر.

\par 14 وسيكون هو نفسه المخلص لكل أرض والراحة للمتعبين، وسحابة تظلل العالم أجمع من الحر الشديد.

\par 15 وإلا فلن تتم السيطرة على ما هو غير خاضع للرقابة.

\par 16 إذا لم يأتِ، فلا يمكن جمع الأشياء السفلية مع العليا.

\par 17 عند مجيئه، ستُصدر أصنام النحاس والحجر وكل نوع من المنحوتات صوتًا لمدة ثلاثة أيام

\par 18 سيخبرون الحكماء عنه ويخبرونهم بما سيكون على الأرض

\par 19 بواسطة نجم، أولئك الذين يرغبون في رؤية من لا تراه الملائكة في الأعلى على الأرض سيجدون الطريق إليه

\par 20 ثم يكون القدير على الأرض بجسده، ويحتضنه ذراعاه الجسديتان، فيعيد المادة البشرية

\par 21 وسيُحيي حواء التي ماتت من ثمرة الشجرة.

\par 22 حينئذٍ ينكشف خداع الأشرار، وتسقط كل الأصنام على وجهها.

\par 23 لأنهم سيخجلون من قبل صاحب المقام.

\par 24 لأنهم كانوا يكذبون بالأوهام، فلن يتمكنوا من الحكم أو النبوة بعد الآن.

\par 25 سيُنتزع منهم الشرف وسيبقون بلا مجد.

\par 26 "لأن الذي يأتي يأخذ القوة والقدرة، ويعطي إبراهيم الحق الذي قاله له سابقًا.

\par 27 سيجعل كل شيء حاد باهتًا، والخشن ناعمًا.

\par 28 ويطرح جميع الأشرار في أعماق البحر.

\par 29 سيصنع عجائب في السماء وعلى الأرض.

\par 30 ويُجرح في وسط بيته الحبيب.

\par 31 وعندما يُجرح، يكون الخلاص مُستعدًا، ونهاية كل هلاك

\par 32 لأن أولئك الذين جرحوه سيصابون بجرح لن يُشفى فيهم إلى الأبد

\par 33 وستنحني كل الخليقة للذي جُرح، وسيُحبط فيه كثيرون

\par 34 وسيُعرف في كل مكان في جميع الأراضي، ولن يخجل من يعترف باسمه

\par 35 ستكون سيطرته وسنواته لا نهاية لها إلى الأبد.

\end{document}