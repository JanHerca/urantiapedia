\begin{document}

\title{امثال}


\chapter{1}

\par 1 أَمْثَالُ سُلَيْمَانَ بْنِ دَاوُدَ مَلِكِ إِسْرَائِيلَ:
\par 2 لِمَعْرِفَةِ حِكْمَةٍ وَأَدَبٍ لإِدْرَاكِ أَقْوَالِ الْفَهْمِ.
\par 3 لِقُبُولِ تَأْدِيبِ الْمَعْرِفَةِ وَالْعَدْلِ وَالْحَقِّ وَالاِسْتِقَامَةِ.
\par 4 لِتُعْطِيَ الْجُهَّالَ ذَكَاءً وَالشَّابَّ مَعْرِفَةً وَتَدَبُّراً.
\par 5 يَسْمَعُهَا الْحَكِيمُ فَيَزْدَادُ عِلْماً وَالْفَهِيمُ يَكْتَسِبُ تَدْبِيراً.
\par 6 لِفَهْمِ الْمَثَلِ وَاللُّغْزِ أَقْوَالِ الْحُكَمَاءِ وَغَوَامِضِهِمْ.
\par 7 مَخَافَةُ الرَّبِّ رَأْسُ الْمَعْرِفَةِ. أَمَّا الْجَاهِلُونَ فَيَحْتَقِرُونَ الْحِكْمَةَ وَالأَدَبَ.
\par 8 اِسْمَعْ يَا ابْنِي تَأْدِيبَ أَبِيكَ وَلاَ تَرْفُضْ شَرِيعَةَ أُمِّكَ
\par 9 لأَنَّهُمَا إِكْلِيلُ نِعْمَةٍ لِرَأْسِكَ وَقَلاَئِدُ لِعُنُقِك.
\par 10 يَا ابْنِي إِنْ تَمَلَّقَكَ الْخُطَاةُ فَلاَ تَرْضَ.
\par 11 إِنْ قَالُوا: «هَلُمَّ مَعَنَا لِنَكْمُنْ لِلدَّمِ. لِنَخْتَفِ لِلْبَرِيءِ بَاطِلاً.
\par 12 لِنَبْتَلِعْهُمْ أَحْيَاءً كَالْهَاوِيَةِ وَصِحَاحاً كَالْهَابِطِينَ فِي الْجُبِّ
\par 13 فَنَجِدَ كُلَّ قِنْيَةٍ فَاخِرَةٍ نَمْلَأَ بُيُوتَنَا غَنِيمَةً.
\par 14 تُلْقِي قُرْعَتَكَ وَسَطَنَا. يَكُونُ لَنَا جَمِيعاً كِيسٌ وَاحِدٌ».
\par 15 يَا ابْنِي لاَ تَسْلُكْ فِي الطَّرِيقِ مَعَهُمْ. امْنَعْ رِجْلَكَ عَنْ مَسَالِكِهِمْ.
\par 16 لأَنَّ أَرْجُلَهُمْ تَجْرِي إِلَى الشَّرِّ وَتُسْرِعُ إِلَى سَفْكِ الدَّمِ.
\par 17 لأَنَّهُ بَاطِلاً تُنْصَبُ الشَّبَكَةُ فِي عَيْنَيْ كُلِّ ذِي جَنَاحٍ.
\par 18 أَمَّا هُمْ فَيَكْمُنُونَ لِدَمِ أَنْفُسِهِمْ. يَخْتَفُونَ لأَنْفُسِهِمْ.
\par 19 هَكَذَا طُرُقُ كُلِّ مُولَعٍ بِكَسْبٍ. يَأْخُذُ نَفْسَ مُقْتَنِيهِ!
\par 20 اَلْحِكْمَةُ تُنَادِي فِي الْخَارِجِ. فِي الشَّوَارِعِ تُعْطِي صَوْتَهَا.
\par 21 تَدْعُو فِي رُؤُوسِ الأَسْوَاقِ فِي مَدَاخِلِ الأَبْوَابِ. فِي الْمَدِينَةِ تُبْدِي كَلاَمَهَا
\par 22 قَائِلَةً: «إِلَى مَتَى أَيُّهَا الْجُهَّالُ تُحِبُّونَ الْجَهْلَ وَالْمُسْتَهْزِئُونَ يُسَرُّونَ بِالاِسْتِهْزَاءِ وَالْحَمْقَى يُبْغِضُونَ الْعِلْمَ؟
\par 23 اِرْجِعُوا عِنْدَ تَوْبِيخِي. هَئَنَذَا أُفِيضُ لَكُمْ رُوحِي. أُعَلِّمُكُمْ كَلِمَاتِي.
\par 24 «لأَنِّي دَعَوْتُ فَأَبَيْتُمْ وَمَدَدْتُ يَدِي وَلَيْسَ مَنْ يُبَالِي
\par 25 بَلْ رَفَضْتُمْ كُلَّ مَشُورَتِي وَلَمْ تَرْضُوا تَوْبِيخِي.
\par 26 فَأَنَا أَيْضاً أَضْحَكُ عِنْدَ بَلِيَّتِكُمْ. أَشْمَتُ عِنْدَ مَجِيءِ خَوْفِكُمْ.
\par 27 إِذَا جَاءَ خَوْفُكُمْ كَعَاصِفَةٍ وَأَتَتْ بَلِيَّتُكُمْ كَالزَّوْبَعَةِ إِذَا جَاءَتْ عَلَيْكُمْ شِدَّةٌ وَضِيقٌ
\par 28 حِينَئِذٍ يَدْعُونَنِي فَلاَ أَسْتَجِيبُ. يُبَكِّرُونَ إِلَيَّ فَلاَ يَجِدُونَنِي.
\par 29 لأَنَّهُمْ أَبْغَضُوا الْعِلْمَ وَلَمْ يَخْتَارُوا مَخَافَةَ الرَّبِّ.
\par 30 لَمْ يَرْضُوا مَشُورَتِي. رَذَلُوا كُلَّ تَوْبِيخِي.
\par 31 فَلِذَلِكَ يَأْكُلُونَ مِنْ ثَمَرِ طَرِيقِهِمْ وَيَشْبَعُونَ مِنْ مُؤَامَرَاتِهِمْ.
\par 32 لأَنَّ ارْتِدَادَ الْحَمْقَى يَقْتُلُهُمْ وَرَاحَةَ الْجُهَّالِ تُبِيدُهُمْ.
\par 33 أَمَّا الْمُسْتَمِعُ لِي فَيَسْكُنُ آمِناً وَيَسْتَرِيحُ مِنْ خَوْفِ الشَّرِّ».

\chapter{2}

\par 1 يَا ابْنِي إِنْ قَبِلْتَ كَلاَمِي وَخَبَّأْتَ وَصَايَايَ عِنْدَكَ
\par 2 حَتَّى تُمِيلَ أُذْنَكَ إِلَى الْحِكْمَةِ وَتُعَطِّفَ قَلْبَكَ عَلَى الْفَهْمِ -
\par 3 إِنْ دَعَوْتَ الْمَعْرِفَةَ وَرَفَعْتَ صَوْتَكَ إِلَى الْفَهْمِ
\par 4 إِنْ طَلَبْتَهَا كَالْفِضَّةِ وَبَحَثْتَ عَنْهَا كَالْكُنُوزِ
\par 5 فَحِينَئِذٍ تَفْهَمُ مَخَافَةَ الرَّبِّ وَتَجِدُ مَعْرِفَةَ اللَّهِ.
\par 6 لأَنَّ الرَّبَّ يُعْطِي حِكْمَةً. مِنْ فَمِهِ الْمَعْرِفَةُ وَالْفَهْمُ.
\par 7 يَذْخَرُ مَعُونَةً لِلْمُسْتَقِيمِينَ. هُوَ مِجَنٌّ لِلسَّالِكِينَ بِالْكَمَالِ
\par 8 لِنَصْرِ مَسَالِكِ الْحَقِّ وَحِفْظِ طَرِيقِ أَتْقِيَائِهِ.
\par 9 حِينَئِذٍ تَفْهَمُ الْعَدْلَ وَالْحَقَّ وَالاِسْتِقَامَةَ: كُلَّ سَبِيلٍ صَالِحٍ.
\par 10 إِذَا دَخَلَتِ الْحِكْمَةُ قَلْبَكَ وَلَذَّتِ الْمَعْرِفَةُ لِنَفْسِكَ
\par 11 فَالْعَقْلُ يَحْفَظُكَ وَالْفَهْمُ يَنْصُرُكَ
\par 12 لإِنْقَاذِكَ مِنْ طَرِيقِ الشِّرِّيرِ وَمِنَ الإِنْسَانِ الْمُتَكَلِّمِ بِالأَكَاذِيبِ
\par 13 التَّارِكِينَ سُبُلَ الاِسْتِقَامَةِ لِلسُّلُوكِ فِي مَسَالِكِ الظُّلْمَةِ
\par 14 الْفَرِحِينَ بِفَعْلِ السُّوءِ الْمُبْتَهِجِينَ بِأَكَاذِيبِ الشَّرِّ
\par 15 الَّذِينَ طُرُقُهُمْ مُعَوَّجَةٌ وَهُمْ مُلْتَوُونَ فِي سُبُلِهِمْ.
\par 16 لإِنْقَاذِكَ مِنَ الْمَرْأَةِ الأَجْنَبِيَّةِ مِنَ الْغَرِيبَةِ الْمُتَمَلِّقَةِ بِكَلاَمِهَا
\par 17 التَّارِكَةِ أَلِيفَ صِبَاهَا وَالنَّاسِيَةِ عَهْدَ إِلَهِهَا.
\par 18 لأَنَّ بَيْتَهَا يَسُوخُ إِلَى الْمَوْتِ وَسُبُلُهَا إِلَى الأَخِيلَةِ.
\par 19 كُلُّ مَنْ دَخَلَ إِلَيْهَا لاَ يَرْجِعُ وَلاَ يَبْلُغُونَ سُبُلَ الْحَيَاةِ.
\par 20 حَتَّى تَسْلُكَ فِي طَرِيقِ الصَّالِحِينَ وَتَحْفَظَ سُبُلَ الصِّدِّيقِينَ.
\par 21 لأَنَّ الْمُسْتَقِيمِينَ يَسْكُنُونَ الأَرْضَ وَالْكَامِلِينَ يَبْقُونَ فِيهَا.
\par 22 أَمَّا الأَشْرَارُ فَيَنْقَرِضُونَ مِنَ الأَرْضِ وَالْغَادِرُونَ يُسْتَأْصَلُونَ مِنْهَا.

\chapter{3}

\par 1 يَا ابْنِي لاَ تَنْسَ شَرِيعَتِي بَلْ لِيَحْفَظْ قَلْبُكَ وَصَايَايَ.
\par 2 فَإِنَّهَا تَزِيدُكَ طُولَ أَيَّامٍ وَسِنِي حَيَاةٍ وَسَلاَمَةً.
\par 3 لاَ تَدَعِ الرَّحْمَةَ وَالْحَقَّ يَتْرُكَانِكَ. تَقَلَّدْهُمَا عَلَى عُنُقِكَ. اكْتُبْهُمَا عَلَى لَوْحِ قَلْبِكَ
\par 4 فَتَجِدَ نِعْمَةً وَفِطْنَةً صَالِحَةً فِي أَعْيُنِ اللَّهِ وَالنَّاسِ.
\par 5 تَوَكَّلْ عَلَى الرَّبِّ بِكُلِّ قَلْبِكَ وَعَلَى فَهْمِكَ لاَ تَعْتَمِدْ.
\par 6 فِي كُلِّ طُرُقِكَ اعْرِفْهُ وَهُوَ يُقَوِّمُ سُبُلَكَ.
\par 7 لاَ تَكُنْ حَكِيماً فِي عَيْنَيْ نَفْسِكَ. اتَّقِ الرَّبَّ وَابْعُدْ عَنِ الشَّرِّ
\par 8 فَيَكُونَ شِفَاءً لِسُرَّتِكَ وَسَقَاءً لِعِظَامِكَ.
\par 9 أَكْرِمِ الرَّبَّ مِنْ مَالِكَ وَمِنْ كُلِّ بَاكُورَاتِ غَلَّتِكَ
\par 10 فَتَمْتَلِئَ خَزَائِنُكَ شِبَعاً وَتَفِيضَ مَعَاصِرُكَ مِسْطَاراً.
\par 11 يَا ابْنِي لاَ تَحْتَقِرْ تَأْدِيبَ الرَّبِّ وَلاَ تَكْرَهْ تَوْبِيخَهُ
\par 12 لأَنَّ الَّذِي يُحِبُّهُ الرَّبُّ يُؤَدِّبُهُ وَكَأَبٍ بِابْنٍ يُسَرُّ بِهِ.
\par 13 طُوبَى لِلإِنْسَانِ الَّذِي يَجِدُ الْحِكْمَةَ وَلِلرَّجُلِ الَّذِي يَنَالُ الْفَهْمَ
\par 14 لأَنَّ تِجَارَتَهَا خَيْرٌ مِنْ تِجَارَةِ الْفِضَّةِ وَرِبْحَهَا خَيْرٌ مِنَ الذَّهَبِ الْخَالِصِ.
\par 15 هِيَ أَثْمَنُ مِنَ اللَّآلِئِ وَكُلُّ جَوَاهِرِكَ لاَ تُسَاوِيهَا.
\par 16 فِي يَمِينِهَا طُولُ أَيَّامٍ وَفِي يَسَارِهَا الْغِنَى وَالْمَجْدُ.
\par 17 طُرُقُهَا طُرُقُ نِعَمٍ وَكُلُّ مَسَالِكِهَا سَلاَمٌ.
\par 18 هِيَ شَجَرَةُ حَيَاةٍ لِمُمْسِكِيهَا وَالْمُتَمَسِّكُ بِهَا مَغْبُوطٌ.
\par 19 الرَّبُّ بِالْحِكْمَةِ أَسَّسَ الأَرْضَ. أَثْبَتَ السَّمَاوَاتِ بِالْفَهْمِ.
\par 20 بِعِلْمِهِ انْشَقَّتِ اللُّجَجُ وَتَقْطُرُ السَّحَابُ نَدًى.
\par 21 يَا ابْنِي لاَ تَبْرَحْ هَذِهِ مِنْ عَيْنَيْكَ. احْفَظِ الرَّأْيَ وَالتَّدْبِيرَ
\par 22 فَيَكُونَا حَيَاةً لِنَفْسِكَ وَنِعْمَةً لِعُنُقِكَ.
\par 23 حِينَئِذٍ تَسْلُكُ فِي طَرِيقِكَ آمِناً وَلاَ تَعْثُرُ رِجْلُكَ.
\par 24 إِذَا اضْطَجَعْتَ فَلاَ تَخَافُ بَلْ تَضْطَجِعُ وَيَلُذُّ نَوْمُكَ.
\par 25 لاَ تَخْشَى مِنْ خَوْفٍ بَاغِتٍ وَلاَ مِنْ خَرَابِ الأَشْرَارِ إِذَا جَاءَ.
\par 26 لأَنَّ الرَّبَّ يَكُونُ مُعْتَمَدَكَ وَيَصُونُ رِجْلَكَ مِنْ أَنْ تُؤْخَذَ.
\par 27 لاَ تَمْنَعِ الْخَيْرَ عَنْ أَهْلِهِ حِينَ يَكُونُ فِي طَاقَةِ يَدِكَ أَنْ تَفْعَلَهُ.
\par 28 لاَ تَقُلْ لِصَاحِبِكَ: «اذْهَبْ وَعُدْ فَأُعْطِيَكَ غَداً» وَمَوْجُودٌ عِنْدَكَ.
\par 29 لاَ تَخْتَرِعْ شَرّاً عَلَى صَاحِبِكَ وَهُوَ سَاكِنٌ لَدَيْكَ آمِناً.
\par 30 لاَ تُخَاصِمْ إِنْسَاناً بِدُونِ سَبَبٍ إِنْ لَمْ يَكُنْ قَدْ صَنَعَ مَعَكَ شَرّاً.
\par 31 لاَ تَحْسِدِ الظَّالِمَ وَلاَ تَخْتَرْ شَيْئاً مِنْ طُرُقِهِ
\par 32 لأَنَّ الْمُلْتَوِيَ رِجْسٌ عِنْدَ الرَّبِّ. أَمَّا سِرُّهُ فَعِنْدَ الْمُسْتَقِيمِينَ.
\par 33 لَعْنَةُ الرَّبِّ فِي بَيْتِ الشِّرِّيرِ لَكِنَّهُ يُبَارِكُ مَسْكَنَ الصِّدِّيقِينَ.
\par 34 كَمَا أَنَّهُ يَسْتَهْزِئُ بِالْمُسْتَهْزِئِينَ هَكَذَا يُعْطِي نِعْمَةً لِلْمُتَوَاضِعِينَ.
\par 35 الْحُكَمَاءُ يَرِثُونَ مَجْداً وَالْحَمْقَى يَحْمِلُونَ هَوَاناً.

\chapter{4}

\par 1 اِسْمَعُوا أَيُّهَا الْبَنُونَ تَأْدِيبَ الأَبِ وَاصْغُوا لأَجْلِ مَعْرِفَةِ الْفَهْمِ
\par 2 لأَنِّي أُعْطِيكُمْ تَعْلِيماً صَالِحاً فَلاَ تَتْرُكُوا شَرِيعَتِي.
\par 3 فَإِنِّي كُنْتُ ابْناً لأَبِي غَضّاً وَوَحِيداً عِنْدَ أُمِّي
\par 4 وَكَانَ يُرِينِي وَيَقُولُ لِي: «لِيَضْبِطْ قَلْبُكَ كَلاَمِي. احْفَظْ وَصَايَايَ فَتَحْيَا.
\par 5 اِقْتَنِ الْحِكْمَةَ. اقْتَنِ الْفَهْمَ. لاَ تَنْسَ وَلاَ تُعْرِضْ عَنْ كَلِمَاتِ فَمِي.
\par 6 لاَ تَتْرُكْهَا فَتَحْفَظَكَ. أَحْبِبْهَا فَتَصُونَكَ.
\par 7 الْحِكْمَةُ هِيَ الرَّأْسُ فَاقْتَنِ الْحِكْمَةَ وَبِكُلِّ مُقْتَنَاكَ اقْتَنِ الْفَهْمَ.
\par 8 ارْفَعْهَا فَتُعَلِّيَكَ. تُمَجِّدُكَ إِذَا اعْتَنَقْتَهَا.
\par 9 تُعْطِي رَأْسَكَ إِكْلِيلَ نِعْمَةٍ. تَاجَ جَمَالٍ تَمْنَحُكَ».
\par 10 اِسْمَعْ يَا ابْنِي وَاقْبَلْ أَقْوَالِي فَتَكْثُرَ سِنُو حَيَاتِكَ.
\par 11 أَرَيْتُكَ طَرِيقَ الْحِكْمَةِ. هَدَيْتُكَ سُبُلَ الاِسْتِقَامَةِ.
\par 12 إِذَا سِرْتَ فَلاَ تَضِيقُ خَطَوَاتُكَ وَإِذَا سَعَيْتَ فَلاَ تَعْثُرُ.
\par 13 تَمَسَّكْ بِالأَدَبِ. لاَ تَرْخِهِ. احْفَظْهُ فَإِنَّهُ هُوَ حَيَاتُكَ.
\par 14 لاَ تَدْخُلْ فِي سَبِيلِ الأَشْرَارِ وَلاَ تَسِرْ فِي طَرِيقِ الأَثَمَةِ.
\par 15 تَنَكَّبْ عَنْهُ. لاَ تَمُرَّ بِهِ. حِدْ عَنْهُ وَاعْبُرْ
\par 16 لأَنَّهُمْ لاَ يَنَامُونَ إِنْ لَمْ يَفْعَلُوا سُوءاً وَيُنْزَعُ نَوْمُهُمْ إِنْ لَمْ يُسْقِطُوا أَحَداً.
\par 17 لأَنَّهُمْ يَطْعَمُونَ خُبْزَ الشَّرِّ وَيَشْرَبُونَ خَمْرَ الظُّلْمِ.
\par 18 أَمَّا سَبِيلُ الصِّدِّيقِينَ فَكَنُورٍ مُشْرِقٍ يَتَزَايَدُ وَيُنِيرُ إِلَى النَّهَارِ الْكَامِلِ.
\par 19 أَمَّا طَرِيقُ الأَشْرَارِ فَكَالظَّلاَمِ. لاَ يَعْلَمُونَ مَا يَعْثُرُونَ بِهِ.
\par 20 يَا ابْنِي أَصْغِ إِلَى كَلاَمِي. أَمِلْ أُذْنَكَ إِلَى أَقْوَالِي.
\par 21 لاَ تَبْرَحْ عَنْ عَيْنَيْكَ. احْفَظْهَا فِي وَسَطِ قَلْبِكَ.
\par 22 لأَنَّهَا هِيَ حَيَاةٌ لِلَّذِينَ يَجِدُونَهَا وَدَوَاءٌ لِكُلِّ الْجَسَدِ.
\par 23 فَوْقَ كُلِّ تَحَفُّظٍ احْفَظْ قَلْبَكَ لأَنَّ مِنْهُ مَخَارِجَ الْحَيَاةِ.
\par 24 انْزِعْ عَنْكَ الْتِوَاءَ الْفَمِ وَأَبْعِدْ عَنْكَ انْحِرَافَ الشَّفَتَيْنِ.
\par 25 لِتَنْظُرْ عَيْنَاكَ إِلَى قُدَّامِكَ وَأَجْفَانُكَ إِلَى أَمَامِكَ مُسْتَقِيماً.
\par 26 مَهِّدْ سَبِيلَ رِجْلِكَ فَتَثْبُتَ كُلُّ طُرُقِكَ.
\par 27 لاَ تَمِلْ يَمْنَةً وَلاَ يَسْرَةً. بَاعِدْ رِجْلَكَ عَنِ الشَّرِّ.

\chapter{5}

\par 1 يَا ابْنِي أَصْغِ إِلَى حِكْمَتِي. أَمِلْ أُذْنَكَ إِلَى فَهْمِي
\par 2 لِحِفْظِ التَّدَابِيرِ وَلِتَحْفَظَ شَفَتَاكَ مَعْرِفَةً.
\par 3 لأَنَّ شَفَتَيِ الْمَرْأَةِ الأَجْنَبِيَّةِ تَقْطُرَانِ عَسَلاً وَحَنَكُهَا أَنْعَمُ مِنَ الزَّيْتِ.
\par 4 لَكِنَّ عَاقِبَتَهَا مُرَّةٌ كَالأَفْسَنْتِينِ. حَادَّةٌ كَسَيْفٍ ذِي حَدَّيْنِ.
\par 5 قَدَمَاهَا تَنْحَدِرَانِ إِلَى الْمَوْتِ. خَطَوَاتُهَا تَتَمَسَّكُ بِالْهَاوِيَةِ.
\par 6 لِئَلاَّ تَتَأَمَّلَ طَرِيقَ الْحَيَاةِ. تَمَايَلَتْ خَطَوَاتُهَا وَلاَ تَشْعُرُ.
\par 7 وَالآنَ أَيُّهَا الْبَنُونَ اسْمَعُوا لِي وَلاَ تَرْتَدُّوا عَنْ كَلِمَاتِ فَمِي.
\par 8 أَبْعِدْ طَرِيقَكَ عَنْهَا وَلاَ تَقْرُبْ إِلَى بَابِ بَيْتِهَا
\par 9 لِئَلاَّ تُعْطِيَ زَهْرَكَ لآخَرِينَ وَسِنِينَكَ لِلْقَاسِي.
\par 10 لِئَلاَّ تَشْبَعَ الأَجَانِبُ مِنْ قُوَّتِكَ وَتَكُونَ أَتْعَابُكَ فِي بَيْتِ غَرِيبٍ.
\par 11 فَتَنُوحَ فِي أَوَاخِرِكَ عِنْدَ فَنَاءِ لَحْمِكَ وَجِسْمِكَ
\par 12 فَتَقُولَ: «كَيْفَ أَنِّي أَبْغَضْتُ الأَدَبَ وَرَذَلَ قَلْبِي التَّوْبِيخَ!
\par 13 وَلَمْ أَسْمَعْ لِصَوْتِ مُرْشِدِيَّ وَلَمْ أَمِلْ أُذُنِي إِلَى مُعَلِّمِيَّ.
\par 14 لَوْلاَ قَلِيلٌ لَكُنْتُ فِي كُلِّ شَرٍّ فِي وَسَطِ الزُّمْرَةِ وَالْجَمَاعَةِ».
\par 15 اِشْرَبْ مِيَاهاً مِنْ جُبِّكَ وَمِيَاهاً جَارِيَةً مِنْ بِئْرِكَ.
\par 16 لاَ تَفِضْ يَنَابِيعُكَ إِلَى الْخَارِجِ سَوَاقِيَ مِيَاهٍ فِي الشَّوَارِعِ.
\par 17 لِتَكُنْ لَكَ وَحْدَكَ وَلَيْسَ لأَجَانِبَ مَعَكَ.
\par 18 لِيَكُنْ يَنْبُوعُكَ مُبَارَكاً وَافْرَحْ بِامْرَأَةِ شَبَابِكَ
\par 19 الظَّبْيَةِ الْمَحْبُوبَةِ وَالْوَعْلَةِ الزَّهِيَّةِ. لِيُرْوِكَ ثَدْيَاهَا فِي كُلِّ وَقْتٍ وَبِمَحَبَّتِهَا اسْكَرْ دَائِماً.
\par 20 فَلِمَاذَا تُفْتَنُ يَا ابْنِي بِأَجْنَبِيَّةٍ وَتَحْتَضِنُ غَرِيبَةً
\par 21 لأَنَّ طُرُقَ الإِنْسَانِ أَمَامَ عَيْنَيِ الرَّبِّ وَهُوَ يَزِنُ كُلَّ سُبُلِهِ.
\par 22 الشِّرِّيرُ تَأْخُذُهُ آثَامُهُ وَبِحِبَالِ خَطِيَّتِهِ يُمْسَكُ.
\par 23 إِنَّهُ يَمُوتُ مِنْ عَدَمِ الأَدَبِ وَبِفَرْطِ حُمْقِهِ يَتَهَوَّرُ.

\chapter{6}

\par 1 يَا ابْنِي إِنْ ضَمِنْتَ صَاحِبَكَ إِنْ صَفَّقْتَ كَفَّكَ لِغَرِيبٍ
\par 2 إِنْ عَلِقْتَ فِي كَلاَمِ فَمِكَ إِنْ أُخِذْتَ بِكَلاَمِ فِيكَ.
\par 3 إِذاً فَافْعَلْ هَذَا يَا ابْنِي وَنَجِّ نَفْسَكَ إِذَا صِرْتَ فِي يَدِ صَاحِبِكَ: اذْهَبْ تَرَامَ وَأَلِحَّ عَلَى صَاحِبِكَ.
\par 4 لاَ تُعْطِ عَيْنَيْكَ نَوْماً وَلاَ أَجْفَانَكَ نُعَاساً.
\par 5 نَجِّ نَفْسَكَ كَالظَّبْيِ مِنَ الْيَدِ كَالْعُصْفُورِ مِنْ يَدِ الصَّيَّادِ.
\par 6 اِذْهَبْ إِلَى النَّمْلَةِ أَيُّهَا الْكَسْلاَنُ. تَأَمَّلْ طُرُقَهَا وَكُنْ حَكِيماً.
\par 7 الَّتِي لَيْسَ لَهَا قَائِدٌ أَوْ عَرِيفٌ أَوْ مُتَسَلِّطٌ
\par 8 وَتُعِدُّ فِي الصَّيْفِ طَعَامَهَا وَتَجْمَعُ فِي الْحَصَادِ أَكْلَهَا.
\par 9 إِلَى مَتَى تَنَامُ أَيُّهَا الْكَسْلاَنُ؟ مَتَى تَنْهَضُ مِنْ نَوْمِكَ؟
\par 10 قَلِيلُ نَوْمٍ بَعْدُ قَلِيلُ نُعَاسٍ وَطَيُّ الْيَدَيْنِ قَلِيلاً لِلرُّقُودِ
\par 11 فَيَأْتِي فَقْرُكَ كَسَاعٍ وَعَوَزُكَ كَغَازٍ!
\par 12 اَلرَّجُلُ اللَّئِيمُ الرَّجُلُ الأَثِيمُ يَسْعَى بِاعْوِجَاجِ الْفَمِ.
\par 13 يَغْمِزُ بِعَيْنَيْهِ. يَقُولُ بِرِجْلِهِ. يُشِيرُ بِأَصَابِعِهِ.
\par 14 فِي قَلْبِهِ أَكَاذِيبُ. يَخْتَرِعُ الشَّرَّ فِي كُلِّ حِينٍ. يَزْرَعُ خُصُومَاتٍ.
\par 15 لأَجْلِ ذَلِكَ بَغْتَةً تُفَاجِئُهُ بَلِيَّتُهُ. فِي لَحْظَةٍ يَنْكَسِرُ وَلاَ شَِفَاءَ.
\par 16 هَذِهِ السِّتَّةُ يُبْغِضُهَا الرَّبُّ وَسَبْعَةٌ هِيَ مَكْرُهَةُ نَفْسِهِ:
\par 17 عُيُونٌ مُتَعَالِيَةٌ لِسَانٌ كَاذِبٌ أَيْدٍ سَافِكَةٌ دَماً بَرِيئاً
\par 18 قَلْبٌ يُنْشِئُ أَفْكَاراً رَدِيئَةً أَرْجُلٌ سَرِيعَةُ الْجَرَيَانِ إِلَى السُّوءِ
\par 19 شَاهِدُ زُورٍ يَفُوهُ بِالأَكَاذِيبِ وَزَارِعُ خُصُومَاتٍ بَيْنَ إِخْوَةٍ.
\par 20 يَا ابْنِي احْفَظْ وَصَايَا أَبِيكَ وَلاَ تَتْرُكْ شَرِيعَةَ أُمِّكَ.
\par 21 اُرْبُطْهَا عَلَى قَلْبِكَ دَائِماً. قَلِّدْ بِهَا عُنُقَكَ.
\par 22 إِذَا ذَهَبْتَ تَهْدِيكَ. إِذَا نِمْتَ تَحْرُسُكَ وَإِذَا اسْتَيْقَظْتَ فَهِيَ تُحَدِّثُكَ.
\par 23 لأَنَّ الْوَصِيَّةَ مِصْبَاحٌ وَالشَّرِيعَةَ نُورٌ وَتَوْبِيخَاتِ الأَدَبِ طَرِيقُ الْحَيَاةِ.
\par 24 لِحِفْظِكَ مِنَ الْمَرْأَةِ الشِّرِّيرَةِ مِنْ مَلَقِ لِسَانِ الأَجْنَبِيَّةِ.
\par 25 لاَ تَشْتَهِيَنَّ جَمَالَهَا بِقَلْبِكَ وَلاَ تَأْخُذْكَ بِهُدُبِهَا.
\par 26 لأَنَّهُ بِسَبَبِ امْرَأَةٍ زَانِيَةٍ يَفْتَقِرُ الْمَرْءُ إِلَى رَغِيفِ خُبْزٍ وَامْرَأَةُ رَجُلٍ آخَرَ تَقْتَنِصُ النَّفْسَ الْكَرِيمَةَ.
\par 27 أَيَأْخُذُ إِنْسَانٌ نَاراً فِي حِضْنِهِ وَلاَ تَحْتَرِقُ ثِيَابُهُ؟
\par 28 أَوَ يَمْشِي إِنْسَانٌ عَلَى الْجَمْرِ وَلاَ تَكْتَوِي رِجْلاَهُ؟
\par 29 هَكَذَا مَنْ يَدْخُلُ عَلَى امْرَأَةِ صَاحِبِهِ. كُلُّ مَنْ يَمَسُّهَا لاَ يَكُونُ بَرِيئاً.
\par 30 لاَ يَسْتَخِفُّونَ بِالسَّارِقِ وَلَوْ سَرِقَ لِيُشْبِعَ نَفْسَهُ وَهُوَ جَوْعَانٌ.
\par 31 إِنْ وُجِدَ يَرُدُّ سَبْعَةَ أَضْعَافٍ وَيُعْطِي كُلَّ قِنْيَةِ بَيْتِهِ.
\par 32 أَمَّا الزَّانِي بِامْرَأَةٍ فَعَدِيمُ الْعَقْلِ. الْمُهْلِكُ نَفْسَهُ هُوَ يَفْعَلُهُ.
\par 33 ضَرْباً وَخِزْياً يَجِدُ وَعَارُهُ لاَ يُمْحَى.
\par 34 لأَنَّ الْغَيْرَةَ هِيَ حَمِيَّةُ الرَّجُلِ فَلاَ يُشْفِقُ فِي يَوْمِ الاِنْتِقَامِ.
\par 35 لاَ يَنْظُرُ إِلَى فِدْيَةٍ مَا وَلاَ يَرْضَى وَلَوْ أَكْثَرْتَ الرَّشْوَةَ.

\chapter{7}

\par 1 يَا ابْنِي احْفَظْ كَلاَمِي وَاذْخَرْ وَصَايَايَ عِنْدَكَ.
\par 2 احْفَظْ وَصَايَايَ فَتَحْيَا وَشَرِيعَتِي كَحَدَقَةِ عَيْنِكَ.
\par 3 اُرْبُطْهَا عَلَى أَصَابِعِكَ. اكْتُبْهَا عَلَى لَوْحِ قَلْبِكَ.
\par 4 قُلْ لِلْحِكْمَةِ: «أَنْتِ أُخْتِي» وَادْعُ الْفَهْمَ ذَا قَرَابَةٍ.
\par 5 لِتَحْفَظَكَ مِنَ الْمَرْأَةِ الأَجْنَبِيَّةِ مِنَ الْغَرِيبَةِ الْمَلِقَةِ بِكَلاَمِهَا.
\par 6 لأَنِّي مِنْ كُوَّةِ بَيْتِي مِنْ وَرَاءِ شُبَّاكِي تَطَلَّعْتُ
\par 7 فَرَأَيْتُ بَيْنَ الْجُهَّالِ لاَحَظْتُ بَيْنَ الْبَنِينَ غُلاَماً عَدِيمَ الْفَهْمِ
\par 8 عَابِراً فِي الشَّارِعِ عِنْدَ زَاوِيَتِهَا وَصَاعِداً فِي طَرِيقِ بَيْتِهَا.
\par 9 فِي الْعِشَاءِ فِي مَسَاءِ الْيَوْمِ فِي حَدَقَةِ اللَّيْلِ وَالظَّلاَمِ.
\par 10 وَإِذَا بِامْرَأَةٍ اسْتَقْبَلَتْهُ فِي زِيِّ زَانِيَةٍ وَخَبِيثَةُ الْقَلْبِ.
\par 11 صَخَّابَةٌ هِيَ وَجَامِحَةٌ. فِي بَيْتِهَا لاَ تَسْتَقِرُّ قَدَمَاهَا.
\par 12 تَارَةً فِي الْخَارِجِ وَأُخْرَى فِي الشَّوَارِعِ. وَعِنْدَ كُلِّ زَاوِيَةٍ تَكْمُنُ.
\par 13 فَأَمْسَكَتْهُ وَقَبَّلَتْهُ. أَوْقَحَتْ وَجْهَهَا وَقَالَتْ لَهُ:
\par 14 «عَلَيَّ ذَبَائِحُ السَّلاَمَةِ. الْيَوْمَ أَوْفَيْتُ نُذُورِي.
\par 15 فَلِذَلِكَ خَرَجْتُ لِلِقَائِكَ لأَطْلُبَ وَجْهَكَ حَتَّى أَجِدَكَ.
\par 16 بِالدِّيبَاجِ فَرَشْتُ سَرِيرِي بِمُوَشَّى كَتَّانٍ مِنْ مِصْرَ.
\par 17 عَطَّرْتُ فِرَاشِي بِمُرٍّ وَعُودٍ وَقِرْفَةٍ.
\par 18 هَلُمَّ نَرْتَوِ وُدّاً إِلَى الصَّبَاحِ. نَتَلَذَّذُ بِالْحُبِّ.
\par 19 لأَنَّ الرَّجُلَ لَيْسَ فِي الْبَيْتِ. ذَهَبَ فِي طَرِيقٍ بَعِيدَةٍ.
\par 20 أَخَذَ صُرَّةَ الْفِضَّةِ بِيَدِهِ. يَوْمَ الْهِلاَلِ يَأْتِي إِلَى بَيْتِهِ».
\par 21 أَغْوَتْهُ بِكَثْرَةِ فُنُونِهَا بِمَلْثِ شَفَتَيْهَا طَوَّحَتْهُ.
\par 22 ذَهَبَ وَرَاءَهَا لِوَقْتِهِ كَثَوْرٍ يَذْهَبُ إِلَى الذَّبْحِ أَوْ كَالْغَبِيِّ إِلَى قَيْدِ الْقِصَاصِ
\par 23 حَتَّى يَشُقَّ سَهْمٌ كَبِدَهُ. كَطَيْرٍ يُسْرِعُ إِلَى الْفَخِّ وَلاَ يَدْرِي أَنَّهُ لِنَفْسِهِ.
\par 24 وَالآنَ أَيُّهَا الأَبْنَاءُ اسْمَعُوا لِي وَأَصْغُوا لِكَلِمَاتِ فَمِي.
\par 25 لاَ يَمِلْ قَلْبُكَ إِلَى طُرُقِهَا وَلاَ تَشْرُدْ فِي مَسَالِكِهَا.
\par 26 لأَنَّهَا طَرَحَتْ كَثِيرِينَ جَرْحَى وَكُلُّ قَتْلاَهَا أَقْوِيَاءُ.
\par 27 طُرُقُ الْهَاوِيَةِ بَيْتُهَا هَابِطَةٌ إِلَى خُدُورِ الْمَوْتِ.

\chapter{8}

\par 1 أَلَعَلَّ الْحِكْمَةَ لاَ تُنَادِي وَالْفَهْمَ أَلاَ يُعْطِي صَوْتَهُ؟
\par 2 عِنْدَ رُؤُوسِ الشَّوَاهِقِ عِنْدَ الطَّرِيقِ بَيْنَ الْمَسَالِكِ تَقِفُ.
\par 3 بِجَانِبِ الأَبْوَابِ عِنْدَ ثَغْرِ الْمَدِينَةِ عِنْدَ مَدْخَلِ الأَبْوَابِ تُصَرِّحُ:
\par 4 «لَكُمْ أَيُّهَا النَّاسُ أُنَادِي وَصَوْتِي إِلَى بَنِي آدَمَ.
\par 5 أَيُّهَا الْحَمْقَى تَعَلَّمُوا ذَكَاءً وَيَا جُهَّالُ تَعَلَّمُوا فَهْماً.
\par 6 اِسْمَعُوا فَإِنِّي أَتَكَلَّمُ بِأُمُورٍ شَرِيفَةٍ وَافْتِتَاحُ شَفَتَيَّ اسْتِقَامَةٌ.
\par 7 لأَنَّ حَنَكِي يَلْهَجُ بِالصِّدْقِ وَمَكْرَهَةُ شَفَتَيَّ الْكَذِبُ.
\par 8 كُلُّ كَلِمَاتِ فَمِي بِالْحَقِّ. لَيْسَ فِيهَا عَوَجٌ وَلاَ الْتِوَاءٌ.
\par 9 كُلُّهَا وَاضِحَةٌ لَدَى الْفَهِيمِ وَمُسْتَقِيمَةٌ لَدَى الَّذِينَ يَجِدُونَ الْمَعْرِفَةَ.
\par 10 خُذُوا تَأْدِيبِي لاَ الْفِضَّةَ. وَالْمَعْرِفَةَ أَكْثَرَ مِنَ الذَّهَبِ الْمُخْتَارِ.
\par 11 لأَنَّ الْحِكْمَةَ خَيْرٌ مِنَ اللّآلِئِ وَكُلُّ الْجَوَاهِرِ لاَ تُسَاوِيهَا.
\par 12 «أَنَا الْحِكْمَةُ أَسْكُنُ الذَّكَاءَ وَأَجِدُ مَعْرِفَةَ التَّدَابِيرِ.
\par 13 مَخَافَةُ الرَّبِّ بُغْضُ الشَّرِّ. الْكِبْرِيَاءَ وَالتَّعَظُّمَ وَطَرِيقَ الشَّرِّ وَفَمَ الأَكَاذِيبِ أَبْغَضْتُ.
\par 14 لِي الْمَشُورَةُ وَالرَّأْيُ. أَنَا الْفَهْمُ. لِي الْقُدْرَةُ.
\par 15 بِي تَمْلِكُ الْمُلُوكُ وَتَقْضِي الْعُظَمَاءُ عَدْلاً.
\par 16 بِي تَتَرَأَّسُ الرُّؤَسَاءُ وَالشُّرَفَاءُ كُلُّ قُضَاةِ الأَرْضِ.
\par 17 أَنَا أُحِبُّ الَّذِينَ يُحِبُّونَنِي وَالَّذِينَ يُبَكِّرُونَ إِلَيَّ يَجِدُونَنِي.
\par 18 عِنْدِي الْغِنَى وَالْكَرَامَةُ. قِنْيَةٌ فَاخِرَةٌ وَحَظٌّ.
\par 19 ثَمَرِي خَيْرٌ مِنَ الذَّهَبِ وَمِنَ الإِبْرِيزِ وَغَلَّتِي خَيْرٌ مِنَ الْفِضَّةِ الْمُخْتَارَةِ.
\par 20 فِي طَرِيقِ الْعَدْلِ أَتَمَشَّى فِي وَسَطِ سُبُلِ الْحَقِّ
\par 21 فَأُوَرِّثُ مُحِبِّيَّ رِزْقاً وَأَمْلَأُ خَزَائِنَهُمْ.
\par 22 «اَلرَّبُّ قَنَانِي أَوَّلَ طَرِيقِهِ مِنْ قَبْلِ أَعْمَالِهِ مُنْذُ الْقِدَمِ.
\par 23 مُنْذُ الأَزَلِ مُسِحْتُ مُنْذُ الْبَدْءِ مُنْذُ أَوَائِلِ الأَرْضِ.
\par 24 إِذْ لَمْ يَكُنْ غَمْرٌ أُبْدِئْتُ. إِذْ لَمْ تَكُنْ يَنَابِيعُ كَثِيرَةُ الْمِيَاهِ.
\par 25 مِنْ قَبْلِ أَنْ تَقَرَّرَتِ الْجِبَالُ قَبْلَ التِّلاَلِ أُبْدِئْتُ.
\par 26 إِذْ لَمْ يَكُنْ قَدْ صَنَعَ الأَرْضَ بَعْدُ وَلاَ الْبَرَارِيَّ وَلاَ أَوَّلَ أَعْفَارِ الْمَسْكُونَةِ.
\par 27 لَمَّا ثَبَّتَ السَّمَاوَاتِ كُنْتُ هُنَاكَ أَنَا. لَمَّا رَسَمَ دَائِرَةً عَلَى وَجْهِ الْغَمْرِ.
\par 28 لَمَّا أَثْبَتَ السُّحُبَ مِنْ فَوْقُ. لَمَّا تَشَدَّدَتْ يَنَابِيعُ الْغَمْرِ.
\par 29 لَمَّا وَضَعَ لِلْبَحْرِ حَدَّهُ فَلاَ تَتَعَدَّى الْمِيَاهُ تُخْمَهُ لَمَّا رَسَمَ أُسُسَ الأَرْضِ
\par 30 كُنْتُ عِنْدَهُ صَانِعاً وَكُنْتُ كُلَّ يَوْمٍ لَذَّتَهُ فَرِحَةً دَائِماً قُدَّامَهُ.
\par 31 فَرِحَةً فِي مَسْكُونَةِ أَرْضِهِ وَلَذَّاتِي مَعَ بَنِي آدَمٍَ.
\par 32 «فَالآنَ أَيُّهَا الْبَنُونَ اسْمَعُوا لِي - فَطُوبَى لِلَّذِينَ يَحْفَظُونَ طُرُقِي.
\par 33 اسْمَعُوا التَّعْلِيمَ وَكُونُوا حُكَمَاءَ وَلاَ تَرْفُضُوهُ.
\par 34 طُوبَى لِلإِنْسَانِ الَّذِي يَسْمَعُ لِي سَاهِراً كُلَّ يَوْمٍ عِنْدَ مَصَارِيعِي حَافِظاً قَوَائِمَ أَبْوَابِي.
\par 35 لأَنَّ مَنْ يَجِدُنِي يَجِدُ الْحَيَاةَ وَيَنَالُ رِضًى مِنَ الرَّبِّ
\par 36 وَمَنْ يُخْطِئُ عَنِّي يَضُرُّ نَفْسَهُ. كُلُّ مُبْغِضِيَّ يُحِبُّونَ الْمَوْتَ».

\chapter{9}

\par 1 اَلْحِكْمَةُ بَنَتْ بَيْتَهَا. نَحَتَتْ أَعْمِدَتَهَا السَّبْعَةَ.
\par 2 ذَبَحَتْ ذَبْحَهَا. مَزَجَتْ خَمْرَهَا. أَيْضاً رَتَّبَتْ مَائِدَتَهَا.
\par 3 أَرْسَلَتْ جَوَارِيَهَا تُنَادِي عَلَى ظُهُورِ أَعَالِي الْمَدِينَةِ:
\par 4 «مَنْ هُوَ جَاهِلٌ فَلِْيَمِلْ إِلَى هُنَا». وَالنَّاقِصُ الْفَهْمِ قَالَتْ لَهُ:
\par 5 «هَلُمُّوا كُلُوا مِنْ طَعَامِي وَاشْرَبُوا مِنَ الْخَمْرِ الَّتِي مَزَجْتُهَا.
\par 6 اُتْرُكُوا الْجَهَالاَتِ فَتَحْيُوا وَسِيرُوا فِي طَرِيقِ الْفَهْمِ.
\par 7 «مَنْ يُوَبِّخُ مُسْتَهْزِئاً يَكْسَبُ لِنَفْسِهِ هَوَاناً وَمَنْ يُنْذِرُ شِرِّيراً يَكْسَبُ عَيْباً.
\par 8 لاَ تُوَبِّخْ مُسْتَهْزِئاً لِئَلاَّ يُبْغِضَكَ. وَبِّخْ حَكِيماً فَيُحِبَّكَ.
\par 9 أَعْطِ حَكِيماً فَيَكُونَ أَوْفَرَ حِكْمَةً. عَلِّمْ صِدِّيقاً فَيَزْدَادَ عِلْماً.
\par 10 بَدْءُ الْحِكْمَةِ مَخَافَةُ الرَّبِّ وَمَعْرِفَةُ الْقُدُّوسِ فَهْمٌ.
\par 11 لأَنَّهُ بِي تَكْثُرُ أَيَّامُكَ وَتَزْدَادُ لَكَ سِنُو حَيَاةٍ.
\par 12 إِنْ كُنْتَ حَكِيماً فَأَنْتَ حَكِيمٌ لِنَفْسِكَ وَإِنِ اسْتَهْزَأْتَ فَأَنْتَ وَحْدَكَ تَتَحَمَّلُ».
\par 13 اَلْمَرْأَةُ الْجَاهِلَةُ صَخَّابَةٌ حَمْقَاءُ وَلاَ تَدْرِي شَيْئاً
\par 14 فَتَقْعُدُ عِنْدَ بَابِ بَيْتِهَا عَلَى كُرْسِيٍّ فِي أَعَالِي الْمَدِينَةِ
\par 15 لِتُنَادِيَ عَابِرِي السَّبِيلِ الْمُقَوِّمِينَ طُرُقَهُمْ:
\par 16 «مَنْ هُوَ جَاهِلٌ فَلْيَمِلْ إِلَى هُنَا». وَالنَّاقِصُ الْفَهْمِ تَقُولُ لَهُ:
\par 17 «الْمِيَاهُ الْمَسْرُوقَةُ حُلْوَةٌ وَخُبْزُ الْخُفْيَةِ لَذِيذٌ».
\par 18 وَلاَ يَعْلَمُ أَنَّ الأَخْيِلَةَ هُنَاكَ وَأَنَّ فِي أَعْمَاقِ الْهَاوِيَةِ ضُيُوفَهَا.

\chapter{10}

\par 1 أَمْثَالُ سُلَيْمَانَ - الاِبْنُ الْحَكِيمُ يَسُرُّ أَبَاهُ وَالاِبْنُ الْجَاهِلُ حُزْنُ أُمِّهِ.
\par 2 كُنُوزُ الشَّرِّ لاَ تَنْفَعُ أَمَّا الْبِرُّ فَيُنَجِّي مِنَ الْمَوْتِ.
\par 3 اَلرَّبُّ لاَ يُجِيعُ نَفْسَ الصِّدِّيقِ وَلَكِنَّهُ يَدْفَعُ هَوَى الأَشْرَارِ.
\par 4 اَلْعَامِلُ بِيَدٍ رَخْوَةٍ يَفْتَقِرُ أَمَّا يَدُ الْمُجْتَهِدِينَ فَتُغْنِي.
\par 5 مَنْ يَجْمَعُ فِي الصَّيْفِ فَهُوَ ابْنٌ عَاقِلٌ وَمَنْ يَنَامُ فِي الْحَصَادِ فَهُوَ ابْنٌ مُخْزٍ.
\par 6 بَرَكَاتٌ عَلَى رَأْسِ الصِّدِّيقِ أَمَّا فَمُ الأَشْرَارِ فَيَغْشَاهُ ظُلْمٌ.
\par 7 ذِكْرُ الصِّدِّيقِ لِلْبَرَكَةِ وَاسْمُ الأَشْرَارِ يَنْخَرُ.
\par 8 حَكِيمُ الْقَلْبِ يَقْبَلُ الْوَصَايَا وَغَبِيُّ الشَّفَتَيْنِ يُصْرَعُ.
\par 9 مَنْ يَسْلُكُ بِالاِسْتِقَامَةِ يَسْلُكُ بِالأَمَانِ وَمَنْ يُعَوِّجُ طُرُقَهُ يُعَرَّفُ.
\par 10 مَنْ يَغْمِزُ بِالْعَيْنِ يُسَبِّبُ حُزْناً وَالْغَبِيُّ الشَّفَتَيْنِ يُصْرَعُ.
\par 11 فَمُ الصِّدِّيقِ يَنْبُوعُ حَيَاةٍ وَفَمُ الأَشْرَارِ يَغْشَاهُ ظُلْمٌ.
\par 12 اَلْبُغْضَةُ تُهَيِّجُ خُصُومَاتٍ وَالْمَحَبَّةُ تَسْتُرُ كُلَّ الذُّنُوبِ.
\par 13 فِي شَفَتَيِ الْعَاقِلِ تُوجَدُ حِكْمَةٌ وَالْعَصَا لِظَهْرِ النَّاقِصِ الْفَهْمِ.
\par 14 اَلْحُكَمَاءُ يَذْخَرُونَ مَعْرِفَةً أَمَّا فَمُ الْغَبِيِّ فَهَلاَكٌ قَرِيبٌ.
\par 15 ثَرْوَةُ الْغَنِيِّ مَدِينَتُهُ الْحَصِينَةُ. هَلاَكُ الْمَسَاكِينِ فَقْرُهُمْ.
\par 16 عَمَلُ الصِّدِّيقِ لِلْحَيَاةِ. رِبْحُ الشِّرِّيرِ لِلْخَطِيَّةِ.
\par 17 حَافِظُ التَّعْلِيمِ هُوَ فِي طَرِيقِ الْحَيَاةِ وَرَافِضُ التَّأْدِيبِ ضَالٌّ.
\par 18 مَنْ يُخْفِي الْبُغْضَةَ فَشَفَتَاهُ كَاذِبَتَانِ وَمُشِيعُ الْمَذَمَّةِ هُوَ جَاهِلٌ.
\par 19 كَثْرَةُ الْكَلاَمِ لاَ تَخْلُو مِنْ مَعْصِيَةٍ أَمَّا الضَّابِطُ شَفَتَيْهِ فَعَاقِلٌ.
\par 20 لِسَانُ الصِّدِّيقِ فِضَّةٌ مُخْتَارَةٌ. قَلْبُ الأَشْرَارِ كَشَيْءٍ زَهِيدٍ.
\par 21 شَفَتَا الصِّدِّيقِ تَهْدِيَانِ كَثِيرِينَ أَمَّا الأَغْبِيَاءُ فَيَمُوتُونَ مِنْ نَقْصِ الْفَهْمِ.
\par 22 بَرَكَةُ الرَّبِّ هِيَ تُغْنِي وَلاَ يَزِيدُ الرَّبُّ مَعَهَا تَعَباً.
\par 23 فِعْلُ الرَّذِيلَةِ عِنْدَ الْجَاهِلِ كَالضِّحْكِ أَمَّا الْحِكْمَةُ فَلِذِي فَهْمٍ.
\par 24 خَوْفُ الشِّرِّيرِ هُوَ يَأْتِيهِ وَشَهْوَةُ الصِّدِّيقِينَ تُمْنَحُ.
\par 25 كَعُبُورِ الزَّوْبَعَةِ فَلاَ يَكُونُ الشِّرِّيرُ أَمَّا الصِّدِّيقُ فَأَسَاسٌ مُؤَبَّدٌ.
\par 26 كَالْخَلِّ لِلأَسْنَانِ وَكَالدُّخَانِ لِلْعَيْنَيْنِ كَذَلِكَ الْكَسْلاَنُ لِلَّذِينَ أَرْسَلُوهُ.
\par 27 مَخَافَةُ الرَّبِّ تَزِيدُ الأَيَّامَ أَمَّا سِنُو الأَشْرَارِ فَتُقْصَرُ.
\par 28 مُنْتَظَرُ الصِّدِّيقِينَ مُفَرِّحٌ أَمَّا رَجَاءُ الأَشْرَارِ فَيَبِيدُ.
\par 29 حِصْنٌ لِلاِسْتِقَامَةِ طَرِيقُ الرَّبِّ وَالْهَلاَكُ لِفَاعِلِي الإِثْمِ.
\par 30 اَلصِّدِّيقُ لَنْ يُزَحْزَحَ أَبَداً وَالأَشْرَارُ لَنْ يَسْكُنُوا الأَرْضَ.
\par 31 فَمُ الصِّدِّيقِ يُنْبِتُ الْحِكْمَةَ أَمَّا لِسَانُ الأَكَاذِيبِ فَيُقْطَعُ.
\par 32 شَفَتَا الصِّدِّيقِ تَعْرِفَانِ الْمَرْضِيَّ وَفَمُ الأَشْرَارِ أَكَاذِيبُ.

\chapter{11}

\par 1 مَوَازِينُ غِشٍّ مَكْرَهَةُ الرَّبِّ وَالْوَزْنُ الصَّحِيحُ رِضَاهُ.
\par 2 تَأْتِي الْكِبْرِيَاءُ فَيَأْتِي الْهَوَانُ وَمَعَ الْمُتَوَاضِعِينَ حِكْمَةٌ.
\par 3 اِسْتِقَامَةُ الْمُسْتَقِيمِينَ تَهْدِيهِمْ وَاعْوِجَاجُ الْغَادِرِينَ يُخْرِبُهُمْ.
\par 4 لاَ يَنْفَعُ الْغِنَى فِي يَوْمِ السَّخَطِ أَمَّا الْبِرُّ فَيُنَجِّي مِنَ الْمَوْتِ.
\par 5 بِرُّ الْكَامِلِ يُقَوِّمُ طَرِيقَهُ أَمَّا الشِّرِّيرُ فَيَسْقُطُ بِشَرِّهِ.
\par 6 بِرُّ الْمُسْتَقِيمِينَ يُنَجِّيهِمْ أَمَّا الْغَادِرُونَ فَيُؤْخَذُونَ بِفَسَادِهِمْ.
\par 7 عِنْدَ مَوْتِ إِنْسَانٍ شِرِّيرٍ يَهْلِكُ رَجَاؤُهُ وَمُنْتَظَرُ الأَثَمَةِ يَبِيدُ.
\par 8 اَلصِّدِّيقُ يَنْجُو مِنَ الضِّيقِ وَيَأْتِي الشِّرِّيرُ مَكَانَهُ.
\par 9 بِالْفَمِ يُخْرِبُ الْمُنَافِقُ صَاحِبَهُ وَبِالْمَعْرِفَةِ يَنْجُو الصِّدِّيقُونَ.
\par 10 بِخَيْرِ الصِّدِّيقِينَ تَفْرَحُ الْمَدِينَةُ وَعِنْدَ هَلاَكِ الأَشْرَارِ هُتَافٌ.
\par 11 بِبَرَكَةِ الْمُسْتَقِيمِينَ تَعْلُو الْمَدِينَةُ وَبِفَمِ الأَشْرَارِ تُهْدَمُ.
\par 12 اَلْمُحْتَقِرُ صَاحِبَهُ هُوَ نَاقِصُ الْفَهْمِ أَمَّا ذُو الْفَهْمِ فَيَسْكُتُ.
\par 13 السَّاعِي بِالْوِشَايَةِ يُفْشِي السِّرَّ وَالأَمِينُ الرُّوحِ يَكْتُمُ الأَمْرَ.
\par 14 حَيْثُ لاَ تَدْبِيرٌ يَسْقُطُ الشَّعْبُ أَمَّا الْخَلاَصُ فَبِكَثْرَةِ الْمُشِيرِينَ.
\par 15 ضَرَراً يُضَرُّ مَنْ يَضْمَنُ غَرِيباً وَمَنْ يُبْغِضُ صَفْقَ الأَيْدِي مُطْمَئِنٌّ.
\par 16 اَلْمَرْأَةُ ذَاتُ النِّعْمَةِ تُحَصِّلُ كَرَامَةً وَالأَشِدَّاءُ يُحَصِّلُونَ غِنًى.
\par 17 اَلرَّجُلُ الرَّحِيمُ يُحْسِنُ إِلَى نَفْسِهِ وَالْقَاسِي يُكَدِّرُ لَحْمَهُ.
\par 18 اَلشِّرِّيرُ يَكْسَبُ أُجْرَةَ غِشٍّ وَالزَّارِعُ الْبِرَّ أُجْرَةَ أَمَانَةٍ.
\par 19 كَمَا أَنَّ الْبِرَّ يَؤُولُ إِلَى الْحَيَاةِ كَذَلِكَ مَنْ يَتْبَعُ الشَّرَّ فَإِلَى مَوْتِهِ.
\par 20 كَرَاهَةُ الرَّبِّ مُلْتَوُو الْقَلْبِ وَرِضَاهُ مُسْتَقِيمُو الطَّرِيقِ.
\par 21 يَدٌ لِيَدٍ لاَ يَتَبَرَّرُ الشِّرِّيرُ أَمَّا نَسْلُ الصِّدِّيقِينَ فَيَنْجُو.
\par 22 خِزَامَةُ ذَهَبٍ فِي فِنْطِيسَةِ خِنْزِيرَةٍ الْمَرْأَةُ الْجَمِيلَةُ الْعَدِيمَةُ الْعَقْلِ.
\par 23 شَهْوَةُ الأَبْرَارِ خَيْرٌ فَقَطْ. رَجَاءُ الأَشْرَارِ سَخَطٌ.
\par 24 يُوجَدُ مَنْ يُفَرِّقُ فَيَزْدَادُ أَيْضاً وَمَنْ يُمْسِكُ أَكْثَرَ مِنَ اللاَّئِقِ وَإِنَّمَا إِلَى الْفَقْرِ.
\par 25 النَّفْسُ السَّخِيَّةُ تُسَمَّنُ وَالْمُرْوِي هُوَ أَيْضاً يُرْوَى.
\par 26 مُحْتَكِرُ الْحِنْطَةِ يَلْعَنُهُ الشَّعْبُ وَالْبَرَكَةُ عَلَى رَأْسِ الْبَائِعِ.
\par 27 مَنْ يَطْلُبُ الْخَيْرَ يَلْتَمِسُ الرِّضَا وَمَنْ يَطْلُبُ الشَّرَّ فَالشَّرُّ يَأْتِيهِ.
\par 28 مَنْ يَتَّكِلْ عَلَى غِنَاهُ يَسْقُطْ أَمَّا الصِّدِّيقُونَ فَيَزْهُونَ كَالْوَرَقِ.
\par 29 مَنْ يُكَدِّرْ بَيْتَهُ يَرِثِ الرِّيحَ وَالْغَبِيُّ خَادِمٌ لِحَكِيمِ الْقَلْبِ.
\par 30 ثَمَرُ الصِّدِّيقِ شَجَرَةُ حَيَاةٍ وَرَابِحُ النُّفُوسِ حَكِيمٌ.
\par 31 هُوَذَا الصِّدِّيقُ يُجَازَى فِي الأَرْضِ فَكَمْ بِالْحَرِيِّ الشِّرِّيرُ وَالْخَاطِئُ!

\chapter{12}

\par 1 مَنْ يُحِبُّ التَّأْدِيبَ يُحِبُّ الْمَعْرِفَةَ وَمَنْ يُبْغِضُ التَّوْبِيخَ فَهُوَ بَلِيدٌ.
\par 2 الصَّالِحُ يَنَالُ رِضىً مِنْ الرَّبِّ أَمَّا رَجُلُ الْمَكَايِدِ فَيَحْكُمُ عَلَيْهِ.
\par 3 لاَ يُثَبَّتُ الإِنْسَانُ بِالشَّرِّ أَمَّا أَصْلُ الصِّدِّيقِينَ فَلاَ يَتَقَلْقَلُ.
\par 4 اَلْمَرْأَةُ الْفَاضِلَةُ تَاجٌ لِبَعْلِهَا أَمَّا الْمُخْزِيَةُ فَكَنَخْرٍ فِي عِظَامِهِ.
\par 5 أَفْكَارُ الصِّدِّيقِينَ عَدْلٌ. تَدَابِيرُ الأَشْرَارِ غِشٌّ.
\par 6 كَلاَمُ الأَشْرَارِ كُمُونٌ لِلدَّمِ أَمَّا فَمُ الْمُسْتَقِيمِينَ فَيُنَجِّيهِمْ.
\par 7 تَنْقَلِبُ الأَشْرَارُ وَلاَ يَكُونُونَ أَمَّا بَيْتُ الصِّدِّيقِينَ فَيَثْبُتُ.
\par 8 بِحَسَبِ فِطْنَتِهِ يُحْمَدُ الإِنْسَانُ أَمَّا الْمُلْتَوِي الْقَلْبِ فَيَكُونُ لِلْهَوَانِ.
\par 9 اَلْحَقِيرُ وَلَهُ عَبْدٌ خَيْرٌ مِنَ الْمُتَمَجِّدِ وَيُعْوِزُهُ الْخُبْزُ.
\par 10 الصِّدِّيقُ يُرَاعِي نَفْسَ بَهِيمَتِهِ أَمَّا مَرَاحِمُ الأَشْرَارِ فَقَاسِيَةٌ.
\par 11 مَنْ يَشْتَغِلُ بِحَقْلِهِ يَشْبَعُ خُبْزاً أَمَّا تَابِعُ الْبَطَّالِينَ فَهُوَ عَدِيمُ الْفَهْمِ.
\par 12 اِشْتَهَى الشِّرِّيرُ صَيْدَ الأَشْرَارِ وَأَصْلُ الصِّدِّيقِينَ يُجْدِي.
\par 13 فِي مَعْصِيَةِ الشَّفَتَيْنِ شَرَكُ الشِّرِّيرِ أَمَّا الصِّدِّيقُ فَيَخْرُجُ مِنَ الضِّيقِ.
\par 14 الإِنْسَانُ يَشْبَعُ خَيْراً مِنْ ثَمَرِ فَمِهِ وَمُكَافَأَةُ يَدَيِ الإِنْسَانِ تُرَدُّ لَهُ.
\par 15 طَرِيقُ الْجَاهِلِ مُسْتَقِيمٌ فِي عَيْنَيْهِ أَمَّا سَامِعُ الْمَشُورَةِ فَهُوَ حَكِيمٌ.
\par 16 غَضَبُ الْجَاهِلِ يُعْرَفُ فِي يَوْمِهِ أَمَّا سَاتِرُ الْهَوَانِ فَهُوَ ذَكِيٌّ.
\par 17 مَنْ يَتَفَوَّهُ بِالْحَقِّ يُظْهِرُ الْعَدْلَ وَالشَّاهِدُ الْكَاذِبُ يُظْهِرُ غِشّاً.
\par 18 يُوجَدُ مَنْ يَهْذُرُ مِثْلَ طَعْنِ السَّيْفِ أَمَّا لِسَانُ الْحُكَمَاءِ فَشِفَاءٌ.
\par 19 شَفَةُ الصِّدْقِ تَثْبُتُ إِلَى الأَبَدِ وَلِسَانُ الْكَذِبِ إِنَّمَا هُوَ إِلَى طَرْفَةِ الْعَيْنِ.
\par 20 اَلْغِشُّ فِي قَلْبِ الَّذِينَ يُفَكِّرُونَ فِي الشَّرِّ أَمَّا الْمُشِيرُونَ بِالسَّلاَمِ فَلَهُمْ فَرَحٌ.
\par 21 لاَ يُصِيبُ الصِّدِّيقَ شَرٌّ أَمَّا الأَشْرَارُ فَيَمْتَلِئُونَ سُوءاً.
\par 22 كَرَاهَةُ الرَّبِّ شَفَتَا كَذِبٍ أَمَّا الْعَامِلُونَ بِالصِّدْقِ فَرِضَاهُ.
\par 23 اَلرَّجُلُ الذَّكِيُّ يَسْتُرُ الْمَعْرِفَةَ وَقَلْبُ الْجَاهِلِ يُنَادِي بِالْحَمَقِ.
\par 24 يَدُ الْمُجْتَهِدِينَ تَسُودُ أَمَّا الرَّخْوَةُ فَتَكُونُ تَحْتَ الْجِزْيَةِ.
\par 25 الْغَمُّ فِي قَلْبِ الرَّجُلِ يُحْنِيهِ وَالْكَلِمَةُ الطَّيِّبَةُ تُفَرِّحُهُ.
\par 26 الصِّدِّيقُ يَهْدِي صَاحِبَهُ أَمَّا طَرِيقُ الأَشْرَارِ فَتُضِلُّهُمْ.
\par 27 الرَّخَاوَةُ لاَ تَمْسِكُ صَيْداً أَمَّا ثَرْوَةُ الإِنْسَانِ الْكَرِيمَةُ فَهِيَ الاِجْتِهَادُ.
\par 28 فِي سَبِيلِ الْبِرِّ حَيَاةٌ وَفِي طَرِيقِ مَسْلِكِهِ لاَ مَوْتَ.

\chapter{13}

\par 1 اَلاِبْنُ الْحَكِيمُ يَقْبَلُ تَأْدِيبَ أَبِيهِ وَالْمُسْتَهْزِئُ لاَ يَسْمَعُ انْتِهَاراً.
\par 2 مِنْ ثَمَرَةِ فَمِهِ يَأْكُلُ الإِنْسَانُ خَيْراً وَمَرَامُ الْغَادِرِينَ ظُلْمٌ.
\par 3 مَنْ يَحْفَظُ فَمَهُ يَحْفَظُ نَفْسَهُ. مَنْ يَفْغَرُ شَفَتَيْهِ فَلَهُ هَلاَكٌ.
\par 4 نَفْسُ الْكَسْلاَنِ تَشْتَهِي وَلاَ شَيْءَ لَهَا وَنَفْسُ الْمُجْتَهِدِينَ تَسْمَنُ.
\par 5 اَلصِّدِّيقُ يُبْغِضُ كَلاَمَ كَذِبٍ وَالشِّرِّيرُ يُخْزِي وَيُخْجِلُ.
\par 6 اَلْبِرُّ يَحْفَظُ الْكَامِلَ طَرِيقَهُ وَالشَّرُّ يَقْلِبُ الْخَاطِئَ.
\par 7 يُوجَدُ مَنْ يَتَغَانَى وَلاَ شَيْءَ عِنْدَهُ وَمَنْ يَتَفَاقَرُ وَعِنْدَهُ غِنًى جَزِيلٌ.
\par 8 فِدْيَةُ نَفْسِ رَجُلٍ غِنَاهُ أَمَّا الْفَقِيرُ فَلاَ يَسْمَعُ انْتِهَاراً.
\par 9 نُورُ الصِّدِّيقِينَ يُفَرِّحُ وَسِرَاجُ الأَشْرَارِ يَنْطَفِئُ.
\par 10 اَلْخِصَامُ إِنَّمَا يَصِيرُ بِالْكِبْرِيَاءِ وَمَعَ الْمُتَشَاوِرِينَ حِكْمَةٌ.
\par 11 غِنَى الْبُطْلِ يَقِلُّ وَالْجَامِعُ بِيَدِهِ يَزْدَادُ.
\par 12 الرَّجَاءُ الْمُمَاطَلُ يُمْرِضُ الْقَلْبَ وَالشَّهْوَةُ الْمُتَمَّمَةُ شَجَرَةُ حَيَاةٍ.
\par 13 مَنِ ازْدَرَى بِالْكَلِمَةِ يُخْرِبُ نَفْسَهُ وَمَنْ خَشِيَ الْوَصِيَّةَ يُكَافَأُ.
\par 14 شَرِيعَةُ الْحَكِيمِ يَنْبُوعُ حَيَاةٍ لِلْحَيَدَانِ عَنْ أَشْرَاكِ الْمَوْتِ.
\par 15 اَلْفِطْنَةُ الْجَيِّدَةُ تَمْنَحُ نِعْمَةً أَمَّا طَرِيقُ الْغَادِرِينَ فَأَوْعَرُ.
\par 16 كُلُّ ذَكِيٍّ يَعْمَلُ بِالْمَعْرِفَةِ وَالْجَاهِلُ يَنْشُرُ حُمْقاً.
\par 17 اَلرَّسُولُ الشِّرِّيرُ يَقَعُ فِي الشَّرِّ وَالسَّفِيرُ الأَمِينُ شِفَاءٌ.
\par 18 قَفْرٌ وَهَوَانٌ لِمَنْ يَرْفُضُ التَّأْدِيبَ وَمَنْ يُلاَحِظُ التَّوْبِيخَ يُكْرَمُ.
\par 19 اَلشَّهْوَةُ الْحَاصِلَةُ تَلُذُّ النَّفْسَ أَمَّا كَرَاهَةُ الْجُهَّالِ فَهِيَ الْحَيَدَانُ عَنِ الشَّرِّ.
\par 20 اَلْمُسَايِرُ الْحُكَمَاءَ يَصِيرُ حَكِيماً وَرَفِيقُ الْجُهَّالِ يُضَرُّ.
\par 21 اَلشَّرُّ يَتْبَعُ الْخَاطِئِينَ وَالصِّدِّيقُونَ يُجَازَوْنَ خَيْراً.
\par 22 اَلصَّالِحُ يُورِثُ بَنِي الْبَنِينَ وَثَرْوَةُ الْخَاطِئِ تُذْخَرُ لِلصِّدِّيقِ.
\par 23 فِي حَرْثِ الْفُقَرَاءِ طَعَامٌ كَثِيرٌ وَيُوجَدُ هَالِكٌ مِنْ عَدَمِ الْحَقِّ.
\par 24 مَنْ يَمْنَعُ عَصَاهُ يَمْقُتُ ابْنَهُ وَمَنْ أَحَبَّهُ يَطْلُبُ لَهُ التَّأْدِيبَ.
\par 25 اَلصِّدِّيقُ يَأْكُلُ لِشَبَعِ نَفْسِهِ أَمَّا بَطْنُ الأَشْرَارِ فَيَحْتَاجُ.

\chapter{14}

\par 1 حِكْمَةُ الْمَرْأَةِ تَبْنِي بَيْتَهَا وَالْحَمَاقَةُ تَهْدِمُهُ بِيَدِهَا.
\par 2 اَلسَّالِكُ بِاسْتِقَامَتِهِ يَتَّقِي الرَّبَّ وَالْمُعَوِّجُ طُرُقَهُ يَحْتَقِرُهُ.
\par 3 فِي فَمِ الْجَاهِلِ قَضِيبٌ لِكِبْرِيَائِهِ أَمَّا شِفَاهُ الْحُكَمَاءِ فَتَحْفَظُهُمْ.
\par 4 حَيْثُ لاَ بَقَرٌ فَالْمَعْلَفُ فَارِغٌ وَكَثْرَةُ الْغَلَّةِ بِقُوَّةِ الثَّوْرِ.
\par 5 اَلشَّاهِدُ الأَمِينُ لَنْ يَكْذِبَ وَالشَّاهِدُ الزُّورُ يَتَفَوَّهُ بِالأَكَاذِيبِ.
\par 6 اَلْمُسْتَهْزِئُ يَطْلُبُ الْحِكْمَةَ وَلاَ يَجِدُهَا وَالْمَعْرِفَةُ هَيِّنَةٌ لِلْفَهِيمِ.
\par 7 اِذْهَبْ مِنْ قُدَّامِ رَجُلٍ جَاهِلٍ إِذْ لاَ تَشْعُرُ بِشَفَتَيْ مَعْرِفَةٍ.
\par 8 حِكْمَةُ الذَّكِيِّ فَهْمُ طَرِيقِهِ وَغَبَاوَةُ الْجُهَّالِ غِشٌّ.
\par 9 اَلْجُهَّالُ يَسْتَهْزِئُونَ بِالإِثْمِ وَبَيْنَ الْمُسْتَقِيمِينَ رِضىً.
\par 10 اَلْقَلْبُ يَعْرِفُ مَرَارَةَ نَفْسِهِ وَبِفَرَحِهِ لاَ يُشَارِكُهُ غَرِيبٌ.
\par 11 بَيْتُ الأَشْرَارِ يُخْرَبُ وَخَيْمَةُ الْمُسْتَقِيمِينَ تُزْهِرُ.
\par 12 تُوجَدُ طَرِيقٌ تَظْهَرُ لِلإِنْسَانِ مُسْتَقِيمَةً وَعَاقِبَتُهَا طُرُقُ الْمَوْتِ.
\par 13 أَيْضاً فِي الضِّحْكِ يَكْتَئِبُ الْقَلْبُ وَعَاقِبَةُ الْفَرَحِ حُزْنٌ.
\par 14 اَلْمُرْتَدُّ فِي الْقَلْبِ يَشْبَعُ مِنْ طُرُقِهِ وَالرَّجُلُ الصَّالِحُ مِمَّا عِنْدَهُ.
\par 15 اَلْغَبِيُّ يُصَدِّقُ كُلَّ كَلِمَةٍ وَالذَّكِيُّ يَنْتَبِهُ إِلَى خَطَوَاتِهِ.
\par 16 اَلْحَكِيمُ يَخْشَى وَيَحِيدُ عَنِ الشَّرِّ وَالْجَاهِلُ يَتَصَلَّفُ وَيَثِقُ.
\par 17 اَلسَّرِيعُ الْغَضَبِ يَعْمَلُ بِالْحَمَقِ وَذُو الْمَكَايِدِ يُشْنَأُ.
\par 18 اَلأَغْبِيَاءُ يَرِثُونَ الْحَمَاقَةَ وَالأَذْكِيَاءُ يُتَوَّجُونَ بِالْمَعْرِفَةِ.
\par 19 الأَشْرَارُ يَنْحَنُونَ أَمَامَ الأَخْيَارِ وَالأَثَمَةُ لَدَى أَبْوَابِ الصِّدِّيقِ.
\par 20 أَيْضاً مِنْ قَرِيبِهِ يُبْغَضُ الْفَقِيرُ وَمُحِبُّو الْغَنِيِّ كَثِيرُونَ.
\par 21 مَنْ يَحْتَقِرُ قَرِيبَهُ يُخْطِئُ وَمَنْ يَرْحَمُ الْمَسَاكِينَ فَطُوبَى لَهُ.
\par 22 أَمَا يَضِلُّ مُخْتَرِعُو الشَّرِّ أَمَّا الرَّحْمَةُ وَالْحَقُّ فَيَهْدِيَانِ مُخْتَرِعِي الْخَيْرِ.
\par 23 فِي كُلِّ تَعَبٍ مَنْفَعَةٌ وَكَلاَمُ الشَّفَتَيْنِ إِنَّمَا هُوَ إِلَى الْفَقْرِ.
\par 24 تَاجُ الْحُكَمَاءِ غِنَاهُمْ. تَقَدُّمُ الْجُهَّالِ حَمَاقَةٌ.
\par 25 اَلشَّاهِدُ الأَمِينُ مُنَجِّي النُّفُوسِ وَمَنْ يَتَفَوَّهُ بِالأَكَاذِيبِ فَغِشٌّ.
\par 26 فِي مَخَافَةِ الرَّبِّ ثِقَةٌ شَدِيدَةٌ وَيَكُونُ لِبَنِيهِ مَلْجَأٌ.
\par 27 مَخَافَةُ الرَّبِّ يَنْبُوعُ حَيَاةٍ لِلْحَيَدَانِ عَنْ أَشْرَاكِ الْمَوْتِ.
\par 28 فِي كَثْرَةِ الشَّعْبِ زِينَةُ الْمَلِكِ وَفِي عَدَمِ الْقَوْمِ هَلاَكُ الأَمِيرِ.
\par 29 بَطِيءُ الْغَضَبِ كَثِيرُ الْفَهْمِ وَقَصِيرُ الرُّوحِ مُعَلِّي الْحَمَقِ.
\par 30 حَيَاةُ الْجَسَدِ هُدُوءُ الْقَلْبِ وَنَخْرُ الْعِظَامِ الْحَسَدُ.
\par 31 ظَالِمُ الْفَقِيرِ يُعَيِّرُ خَالِقَهُ وَيُمَجِّدُهُ رَاحِمُ الْمِسْكِينِ.
\par 32 اَلشِّرِّيرُ يُطْرَدُ بِشَرِّهِ أَمَّا الصِّدِّيقُ فَوَاثِقٌ عِنْدَ مَوْتِهِ.
\par 33 فِي قَلْبِ الْفَهِيمِ تَسْتَقِرُّ الْحِكْمَةُ وَمَا فِي دَاخِلِ الْجُهَّالِ يُعْرَفُ.
\par 34 اَلْبِرُّ يَرْفَعُ شَأْنَ الأُمَّةِ وَعَارُ الشُّعُوبِ الْخَطِيَّةُ.
\par 35 رِضْوَانُ الْمَلِكِ عَلَى الْعَبْدِ الْفَطِنِ وَسَخَطُهُ يَكُونُ عَلَى الْمُخْزِي.

\chapter{15}

\par 1 اَلْجَوَابُ اللَّيِّنُ يَصْرِفُ الْغَضَبَ وَالْكَلاَمُ الْمُوجِعُ يُهَيِّجُ السَّخَطَ.
\par 2 لِسَانُ الْحُكَمَاءِ يُحَسِّنُ الْمَعْرِفَةَ وَفَمُ الْجُهَّالِ يُنْبِعُ حَمَاقَةً.
\par 3 فِي كُلِّ مَكَانٍ عَيْنَا الرَّبِّ مُرَاقِبَتَيْنِ الطَّالِحِينَ وَالصَّالِحِينَ.
\par 4 هُدُوءُ اللِّسَانِ شَجَرَةُ حَيَاةٍ وَاعْوِجَاجُهُ سَحْقٌ فِي الرُّوحِ.
\par 5 اَلأَحْمَقُ يَسْتَهِينُ بِتَأْدِيبِ أَبِيهِ أَمَّا مُرَاعِي التَّوْبِيخِ فَيَذْكَى.
\par 6 فِي بَيْتِ الصِّدِّيقِ كَنْزٌ عَظِيمٌ وَفِي دَخْلِ الأَشْرَارِ كَدَرٌ.
\par 7 شِفَاهُ الْحُكَمَاءِ تَذُرُّ مَعْرِفَةً أَمَّا قَلْبُ الْجُهَّالِ فَلَيْسَ كَذَلِكَ.
\par 8 ذَبِيحَةُ الأَشْرَارِ مَكْرَهَةُ الرَّبِّ وَصَلاَةُ الْمُسْتَقِيمِينَ مَرْضَاتُهُ.
\par 9 مَكْرَهَةُ الرَّبِّ طَرِيقُ الشِّرِّيرِ وَتَابِعُ الْبِرِّ يُحِبُّهُ.
\par 10 تَأْدِيبُ شَرٍّ لِتَارِكِ الطَّرِيقِ. مُبْغِضُ التَّوْبِيخِ يَمُوتُ.
\par 11 اَلْهَاوِيَةُ وَالْهَلاَكُ أَمَامَ الرَّبِّ. كَمْ بِالْحَرِيِّ قُلُوبُ بَنِي آدَمَ!
\par 12 اَلْمُسْتَهْزِئُ لاَ يُحِبُّ مُوَبِّخَهُ. إِلَى الْحُكَمَاءِ لاَ يَذْهَبُ.
\par 13 اَلْقَلْبُ الْفَرْحَانُ يَجْعَلُ الْوَجْهَ طَلِقاً وَبِحُزْنِ الْقَلْبِ تَنْسَحِقُ الرُّوحُ.
\par 14 قَلْبُ الْفَهِيمِ يَطْلُبُ مَعْرِفَةً وَفَمُ الْجُهَّالِ يَرْعَى حَمَاقَةً.
\par 15 كُلُّ أَيَّامِ الْحَزِينِ شَقِيَّةٌ أَمَّا طَيِّبُ الْقَلْبِ فَوَلِيمَةٌ دَائِمَةٌ.
\par 16 اَلْقَلِيلُ مَعَ مَخَافَةِ الرَّبِّ خَيْرٌ مِنْ كَنْزٍ عَظِيمٍ مَعَ هَمٍّ.
\par 17 أَكْلَةٌ مِنَ الْبُقُولِ حَيْثُ تَكُونُ الْمَحَبَّةُ خَيْرٌ مِنْ ثَوْرٍ مَعْلُوفٍ وَمَعَهُ بُغْضَةٌ.
\par 18 اَلرَّجُلُ الْغَضُوبُ يُهَيِّجُ الْخُصُومَةَ وَبَطِيءُ الْغَضَبِ يُسَكِّنُ الْخِصَامَ.
\par 19 طَرِيقُ الْكَسْلاَنِ كَسِيَاجٍ مِنْ شَوْكٍ وَطَرِيقُ الْمُسْتَقِيمِينَ مَنْهَجٌ.
\par 20 اَلاِبْنُ الْحَكِيمُ يَسُرُّ أَبَاهُ وَالرَّجُلُ الْجَاهِلُ يَحْتَقِرُ أُمَّهُ.
\par 21 الْحَمَاقَةُ فَرَحٌ لِنَاقِصِ الْفَهْمِ أَمَّا ذُو الْفَهْمِ فَيُقَوِّمُ سُلُوكَهُ.
\par 22 مَقَاصِدُ بِغَيْرِ مَشُورَةٍ تَبْطُلُ وَبِكَثْرَةِ الْمُشِيرِينَ تَقُومُ.
\par 23 لِلإِنْسَانِ فَرَحٌ بِجَوَابِ فَمِهِ وَالْكَلِمَةُ فِي وَقْتِهَا مَا أَحْسَنَهَا.
\par 24 طَرِيقُ الْحَيَاةِ لِلْفَطِنِ إِلَى فَوْقُ لِلْحَيَدَانِ عَنِ الْهَاوِيَةِ مِنْ تَحْتُ.
\par 25 اَلرَّبُّ يَقْلَعُ بَيْتَ الْمُتَكَبِّرِينَ وَيُوَطِّدُ تُخْمَ الأَرْمَلَةِ.
\par 26 مَكْرَهَةُ الرَّبِّ أَفْكَارُ الشِّرِّيرِ وَلِلأَطْهَارِ كَلاَمٌ حَسَنٌ.
\par 27 اَلْمُولَعُ بِالْكَسْبِ يُكَدِّرُ بَيْتَهُ وَالْكَارِهُ الْهَدَايَا يَعِيشُ.
\par 28 قَلْبُ الصِّدِّيقِ يَتَفَكَّرُ بِالْجَوَابِ وَفَمُ الأَشْرَارِ يُنْبِعُ شُرُوراً.
\par 29 اَلرَّبُّ بَعِيدٌ عَنِ الأَشْرَارِ وَيَسْمَعُ صَلاَةَ الصِّدِّيقِينَ.
\par 30 نُورُ الْعَيْنَيْنِ يُفَرِّحُ الْقَلْبَ. الْخَبَرُ الطَّيِّبُ يُسَمِّنُ الْعِظَامَ.
\par 31 اَلأُذُنُ السَّامِعَةُ تَوْبِيخَ الْحَيَاةِ تَسْتَقِرُّ بَيْنَ الْحُكَمَاءِ.
\par 32 مَنْ يَرْفُضُ التَّأْدِيبَ يَرْذُلُ نَفْسَهُ وَمَنْ يَسْمَعُ لِلتَّوْبِيخِ يَقْتَنِي فَهْماً.
\par 33 مَخَافَةُ الرَّبِّ أَدَبُ حِكْمَةٍ وَقَبْلَ الْكَرَامَةِ التَّوَاضُعُ.

\chapter{16}

\par 1 لِلإِنْسَانِ تَدَابِيرُ الْقَلْبِ وَمِنَ الرَّبِّ جَوَابُ اللِّسَانِ.
\par 2 كُلُّ طُرُقِ الإِنْسَانِ نَقِيَّةٌ فِي عَيْنَيْ نَفْسِهِ وَالرَّبُّ وَازِنُ الأَرْوَاحِ.
\par 3 أَلْقِ عَلَى الرَّبِّ أَعْمَالَكَ فَتُثَبَّتَ أَفْكَارُكَ.
\par 4 اَلرَّبُّ صَنَعَ الْكُلَّ لِغَرَضِهِ وَالشِّرِّيرَ أَيْضاً لِيَوْمِ الشَّرِّ.
\par 5 مَكْرَهَةُ الرَّبِّ كُلُّ مُتَشَامِخِ الْقَلْبِ. يَداً لِيَدٍ لاَ يَتَبَرَّأُ.
\par 6 بِالرَّحْمَةِ وَالْحَقِّ يُسْتَرُ الإِثْمُ وَفِي مَخَافَةِ الرَّبِّ الْحَيَدَانُ عَنِ الشَّرِّ.
\par 7 إِذَا أَرْضَتِ الرَّبَّ طُرُقُ إِنْسَانٍ جَعَلَ أَعْدَاءَهُ أَيْضاً يُسَالِمُونَهُ.
\par 8 اَلْقَلِيلُ مَعَ الْعَدْلِ خَيْرٌ مِنْ دَخْلٍ جَزِيلٍ بِغَيْرِ حَقٍّ.
\par 9 قَلْبُ الإِنْسَانِ يُفَكِّرُ فِي طَرِيقِهِ وَالرَّبُّ يَهْدِي خَطْوَتَهُ.
\par 10 فِي شَفَتَيِ الْمَلِكِ وَحْيٌ. فِي الْقَضَاءِ فَمُهُ لاَ يَخُونُ.
\par 11 قَبَّانُ الْحَقِّ وَمَوَازِينُهُ لِلرَّبِّ. كُلُّ مَعَايِيرِ الْكِيسِ عَمَلُهُ.
\par 12 مَكْرَهَةُ الْمُلُوكِ فِعْلُ الشَّرِّ لأَنَّ الْكُرْسِيَّ يُثَبَّتُ بِالْبِرِّ.
\par 13 مَرْضَاةُ الْمُلُوكِ شَفَتَا حَقٍّ وَالْمُتَكَلِّمُ بِالْمُسْتَقِيمَاتِ يُحَبُّ.
\par 14 غَضَبُ الْمَلِكِ رُسُلُ الْمَوْتِ وَالإِنْسَانُ الْحَكِيمُ يَسْتَعْطِفُهُ.
\par 15 فِي نُورِ وَجْهِ الْمَلِكِ حَيَاةٌ وَرِضَاهُ كَسَحَابِ الْمَطَرِ الْمُتَأَخِّرِ.
\par 16 قِنْيَةُ الْحِكْمَةِ كَمْ هِيَ خَيْرٌ مِنَ الذَّهَبِ وَقِنْيَةُ الْفَهْمِ تُخْتَارُ عَلَى الْفِضَّةِ!
\par 17 مَنْهَجُ الْمُسْتَقِيمِينَ الْحَيَدَانُ عَنِ الشَّرِّ. حَافِظٌ نَفْسَهُ حَافِظٌ طَرِيقَهُ.
\par 18 قَبْلَ الْكَسْرِ الْكِبْرِيَاءُ وَقَبْلَ السُّقُوطِ تَشَامُخُ الرُّوحِ.
\par 19 تَوَاضُعُ الرُّوحِ مَعَ الْوُدَعَاءِ خَيْرٌ مِنْ قَسْمِ الْغَنِيمَةِ مَعَ الْمُتَكَبِّرِينَ.
\par 20 الْفَطِنُ مِنْ جِهَةِ أَمْرٍ يَجِدُ خَيْراً وَمَنْ يَتَّكِلُ عَلَى الرَّبِّ فَطُوبَى لَهُ.
\par 21 حَكِيمُ الْقَلْبِ يُدْعَى فَهِيماً وَحَلاَوَةُ الشَّفَتَيْنِ تَزِيدُ عِلْماً.
\par 22 الْفِطْنَةُ يَنْبُوعُ حَيَاةٍ لِصَاحِبِهَا وَتَأْدِيبُ الْحَمْقَى حَمَاقَةٌ.
\par 23 قَلْبُ الْحَكِيمِ يُرْشِدُ فَمَهُ وَيَزِيدُ شَفَتَيْهِ عِلْماً.
\par 24 اَلْكَلاَمُ الْحَسَنُ شَهْدُ عَسَلٍ حُلْوٌ لِلنَّفْسِ وَشِفَاءٌ لِلْعِظَامِ.
\par 25 تُوجَدُ طَرِيقٌ تَظْهَرُ لِلإِنْسَانِ مُسْتَقِيمَةً وَعَاقِبَتُهَا طُرُقُ الْمَوْتِ.
\par 26 نَفْسُ التَّعِبِ تُتْعِبُ لَهُ لأَنَّ فَمَهُ يَحِثُّهُ.
\par 27 الرَّجُلُ اللَّئِيمُ يَنْبُشُ الشَّرَّ وَعَلَى شَفَتَيْهِ كَالنَّارِ الْمُتَّقِدَةِ.
\par 28 رَجُلُ الأَكَاذِيبِ يُطْلِقُ الْخُصُومَةَ وَالنَّمَّامُ يُفَرِّقُ الأَصْدِقَاءَ.
\par 29 اَلرَّجُلُ الظَّالِمُ يُغْوِي صَاحِبَهُ وَيَسُوقُهُ إِلَى طَرِيقٍ غَيْرِ صَالِحَةٍ.
\par 30 مَنْ يُغَمِّضُ عَيْنَيْهِ لِيُفَكِّرَ فِي الأَكَاذِيبِ وَمَنْ يَعَضُّ شَفَتَيْهِ فَقَدْ أَكْمَلَ شَرّاً.
\par 31 تَاجُ جَمَالٍ: شَيْبَةٌ تُوجَدُ فِي طَرِيقِ الْبِرِّ.
\par 32 اَلْبَطِيءُ الْغَضَبِ خَيْرٌ مِنَ الْجَبَّارِ وَمَالِكُ رُوحِهِ خَيْرٌ مِمَّنْ يَأْخُذُ مَدِينَةً.
\par 33 الْقُرْعَةُ تُلْقَى فِي الْحِضْنِ وَمِنَ الرَّبِّ كُلُّ حُكْمِهَا.

\chapter{17}

\par 1 لُقْمَةٌ يَابِسَةٌ وَمَعَهَا سَلاَمَةٌ خَيْرٌ مِنْ بَيْتٍ مَلآنٍ ذَبَائِحَ مَعَ خِصَامٍ.
\par 2 اَلْعَبْدُ الْفَطِنُ يَتَسَلَّطُ عَلَى الاِبْنِ الْمُخْزِي وَيُقَاسِمُ الإِخْوَةَ الْمِيرَاثَ.
\par 3 الْبُوطَةُ لِلْفِضَّةِ وَالْكُورُ لِلذَّهَبِ وَمُمْتَحِنُ الْقُلُوبِ الرَّبُّ.
\par 4 الْفَاعِلُ الشَّرَّ يُصْغِي إِلَى شَفَةِ الإِثْمِ وَالْكَاذِبُ يَأْذَنُ لِلِسَانِ فَسَادٍ.
\par 5 الْمُسْتَهْزِئُ بِالْفَقِيرِ يُعَيِّرُ خَالِقَهُ. الْفَرْحَانُ بِبَلِيَّةٍ لاَ يَتَبَرَّأُ.
\par 6 تَاجُ الشُّيُوخِ بَنُو الْبَنِينَ وَفَخْرُ الْبَنِينَ آبَاؤُهُمْ.
\par 7 لاَ تَلِيقُ بِالأَحْمَقِ شَفَةُ السُّودَدِ. كَمْ بِالأَحْرَى شَفَةُ الْكَذِبِ بِالشَّرِيفِ!
\par 8 الْهَدِيَّةُ حَجَرٌ كَرِيمٌ فِي عَيْنَيْ قَابِلِهَا حَيْثُمَا تَتَوَجَّهُ تُفْلِحُ.
\par 9 مَنْ يَسْتُرْ مَعْصِيَةً يَطْلُبُ الْمَحَبَّةَ وَمَنْ يُكَرِّرُ أَمْراً يُفَرِّقُ بَيْنَ الأَصْدِقَاءِ.
\par 10 اَلاِنْتِهَارُ يُؤَثِّرُ فِي الْحَكِيمِ أَكْثَرَ مِنْ مِئَةِ جَلْدَةٍ فِي الْجَاهِلِ.
\par 11 اَلشِّرِّيرُ إِنَّمَا يَطْلُبُ التَّمَرُّدَ فَيُطْلَقُ عَلَيْهِ رَسُولٌ قَاسٍ.
\par 12 لِيُصَادِفِ الإِنْسَانَ دُبَّةٌ ثَكُولٌ وَلاَ جَاهِلٌ فِي حَمَاقَتِهِ.
\par 13 مَنْ يُجَازِي عَنْ خَيْرٍ بِشَرٍّ لَنْ يَبْرَحَ الشَّرُّ مِنْ بَيْتِهِ.
\par 14 اِبْتِدَاءُ الْخِصَامِ إِطْلاَقُ الْمَاءِ فَقَبْلَ أَنْ تَدْفُقَ الْمُخَاصَمَةُ اتْرُكْهَا.
\par 15 مُبَرِّئُ الْمُذْنِبَ وَمُذَنِّبُ الْبَرِيءَ كِلاَهُمَا مَكْرَهَةُ الرَّبِّ.
\par 16 لِمَاذَا فِي يَدِ الْجَاهِلِ ثَمَنٌ؟ هَلْ لاِقْتِنَاءِ الْحِكْمَةِ وَلَيْسَ لَهُ فَهْمٌ؟
\par 17 اَلصَّدِيقُ يُحِبُّ فِي كُلِّ وَقْتٍ أَمَّا الأَخُ فَلِلشِّدَّةِ يُولَدُ.
\par 18 اَلإِنْسَانُ النَّاقِصُ الْفَهْمِ يَصْفِقُ كَفّاً وَيَضْمَنُ صَاحِبَهُ ضَمَاناً.
\par 19 مُحِبُّ الْمَعْصِيَةِ مُحِبُّ الْخِصَامِ. الْمُعَلِّي بَابَهُ يَطْلُبُ الْكَسْرَ.
\par 20 الْمُلْتَوِي الْقَلْبِ لاَ يَجِدُ خَيْراً وَالْمُتَقَلِّبُ اللِّسَانِ يَقَعُ فِي السُّوءِ.
\par 21 مَنْ يَلِدُ جَاهِلاً فَلِحُزْنِهِ وَلاَ يَفْرَحُ أَبُو الأَحْمَقِ.
\par 22 الْقَلْبُ الْفَرْحَانُ يُطَيِّبُ الْجِسْمَ وَالرُّوحُ الْمُنْسَحِقَةُ تُجَفِّفُ الْعَظْمَ.
\par 23 الشِّرِّيرُ يَأْخُذُ الرَّشْوَةَ مِنَ الْحِضْنِ لِيُعَوِّجَ طُرُقَ الْقَضَاءِ.
\par 24 الْحِكْمَةُ عِنْدَ الْفَهِيمِ وَعَيْنَا الْجَاهِلِ فِي أَقْصَى الأَرْضِ.
\par 25 الاِبْنُ الْجَاهِلُ غَمٌّ لأَبِيهِ وَمَرَارَةٌ لِلَّتِي وَلَدَتْهُ.
\par 26 أَيْضاً تَغْرِيمُ الْبَرِيءِ لَيْسَ بِحَسَنٍ وَكَذَلِكَ ضَرْبُ الشُّرَفَاءِ لأَجْلِ الاِسْتِقَامَةِ.
\par 27 ذُو الْمَعْرِفَةِ يُبْقِي كَلاَمَهُ وَذُو الْفَهْمِ وَقُورُ الرُّوحِ.
\par 28 بَلِ الأَحْمَقُ إِذَا سَكَتَ يُحْسَبُ حَكِيماً وَمَنْ ضَمَّ شَفَتَيْهِ فَهِيماً!

\chapter{18}

\par 1 الْمُعْتَزِلُ يَطْلُبُ شَهْوَتَهُ. بِكُلِّ مَشُورَةٍ يَغْتَاظُ.
\par 2 اَلْجَاهِلُ لاَ يُسَرُّ بِالْفَهْمِ بَلْ بِكَشْفِ قَلْبِهِ.
\par 3 إِذَا جَاءَ الشِّرِّيرُ جَاءَ الاِحْتِقَارُ أَيْضاً وَمَعَ الْهَوَانِ عَارٌ.
\par 4 كَلِمَاتُ فَمِ الإِنْسَانِ مِيَاهٌ عَمِيقَةٌ. نَبْعُ الْحِكْمَةِ نَهْرٌ مُنْدَفِقٌ.
\par 5 رَفْعُ وَجْهِ الشِّرِّيرِ لَيْسَ حَسَناً لإِخْطَاءِ الصِّدِّيقِ فِي الْقَضَاءِ.
\par 6 شَفَتَا الْجَاهِلِ تُدَاخِلاَنِ فِي الْخُصُومَةِ وَفَمُهُ يَدْعُو بِضَرَبَاتٍ.
\par 7 فَمُ الْجَاهِلِ مَهْلَكَةٌ لَهُ وَشَفَتَاهُ شَرَكٌ لِنَفْسِهِ.
\par 8 كَلاَمُ النَّمَّامِ مِثْلُ لُقَمٍ حُلْوَةٍ وَهُوَ يَنْزِلُ إِلَى مَخَادِعِ الْبَطْنِ.
\par 9 أَيْضاً الْمُتَرَاخِي فِي عَمَلِهِ هُوَ أَخُو الْمُسْرِفِ.
\par 10 اِسْمُ الرَّبِّ بُرْجٌ حَصِينٌ يَرْكُضُ إِلَيْهِ الصِّدِّيقُ وَيَتَمَنَّعُ.
\par 11 ثَرْوَةُ الْغَنِيِّ مَدِينَتُهُ الْحَصِينَةُ وَمِثْلُ سُورٍ عَالٍ فِي تَصَوُّرِهِ.
\par 12 قَبْلَ الْكَسْرِ يَتَكَبَّرُ قَلْبُ الإِنْسَانِ وَقَبْلَ الْكَرَامَةِ التَّوَاضُعُ.
\par 13 مَنْ يُجِيبُ عَنْ أَمْرٍ قَبْلَ أَنْ يَسْمَعَهُ فَلَهُ حَمَاقَةٌ وَعَارٌ.
\par 14 رُوحُ الإِنْسَانِ تَحْتَمِلُ مَرَضَهُ أَمَّا الرُّوحُ الْمَكْسُورَةُ فَمَنْ يَحْمِلُهَا؟
\par 15 قَلْبُ الْفَهِيمِ يَقْتَنِي مَعْرِفَةً وَأُذُنُ الْحُكَمَاءِ تَطْلُبُ عِلْماً.
\par 16 هَدِيَّةُ الإِنْسَانِ تُرَحِّبُ لَهُ وَتَهْدِيهِ إِلَى أَمَامِ الْعُظَمَاءِ.
\par 17 اَلأَوَّلُ فِي دَعْوَاهُ مُحِقٌّ فَيَأْتِي رَفِيقُهُ وَيَفْحَصُهُ.
\par 18 اَلْقُرْعَةُ تُبَطِّلُ الْخُصُومَاتِ وَتَفْصِلُ بَيْنَ الأَقْوِيَاءِ.
\par 19 اَلأَخُ أَمْنَعُ مِنْ مَدِينَةٍ حَصِينَةٍ وَالْمُخَاصَمَاتُ كَعَارِضَةِ قَلْعَةٍ.
\par 20 مِنْ ثَمَرِ فَمِ الإِنْسَانِ يَشْبَعُ بَطْنُهُ مِنْ غَلَّةِ شَفَتَيْهِ يَشْبَعُ.
\par 21 اَلْمَوْتُ وَالْحَيَاةُ فِي يَدِ اللِّسَانِ وَأَحِبَّاؤُهُ يَأْكُلُونَ ثَمَرَهُ.
\par 22 مَنْ يَجِدُ زَوْجَةً يَجِدُ خَيْراً وَيَنَالُ رِضًى مِنَ الرَّبِّ.
\par 23 بِتَضَرُّعَاتٍ يَتَكَلَّمُ الْفَقِيرُ وَالْغَنِيُّ يُجَاوِبُ بِخُشُونَةٍ.
\par 24 اَلْمُكْثِرُ الأَصْحَابِ يُخْرِبُ نَفْسَهُ وَلَكِنْ يُوجَدْ مُحِبٌّ أَلْزَقُ مِنَ الأَخِ.

\chapter{19}

\par 1 اَلْفَقِيرُ السَّالِكُ بِكَمَالِهِ خَيْرٌ مِنْ مُلْتَوِي الشَّفَتَيْنِ وَهُوَ جَاهِلٌ.
\par 2 أَيْضاً كَوْنُ النَّفْسِ بِلاَ مَعْرِفَةٍ لَيْسَ حَسَناً وَالْمُسْتَعْجِلُ بِرِجْلَيْهِ يُخْطِئُ.
\par 3 حَمَاقَةُ الرَّجُلِ تُعَوِّجُ طَرِيقَهُ وَعَلَى الرَّبِّ يَحْنَقُ قَلْبُهُ.
\par 4 اَلْغِنَى يُكْثِرُ الأَصْحَابَ وَالْفَقِيرُ مُنْفَصِلٌ عَنْ قَرِيبِهِ.
\par 5 شَاهِدُ الزُّورِ لاَ يَتَبَرَّأُ وَالْمُتَكَلِّمُ بِالأَكَاذِيبِ لاَ يَنْجُو.
\par 6 كَثِيرُونَ يَسْتَعْطِفُونَ وَجْهَ الشَّرِيفِ وَكُلٌّ صَاحِبٌ لِذِي الْعَطَايَا.
\par 7 كُلُّ إِخْوَةِ الْفَقِيرِ يُبْغِضُونَهُ فَكَمْ بِالْحَرِيِّ أَصْدِقَاؤُهُ يَبْتَعِدُونَ عَنْهُ! مَنْ يَتْبَعُ أَقْوَالاً فَهِيَ لَهُ.
\par 8 اَلْمُقْتَنِي الْحِكْمَةَ يُحِبُّ نَفْسَهُ. الْحَافِظُ الْفَهْمِ يَجِدُ خَيْراً.
\par 9 شَاهِدُ الزُّورِ لاَ يَتَبَرَّأُ وَالْمُتَكَلِّمُ بِالأَكَاذِيبِ يَهْلِكُ.
\par 10 اَلتَّنَعُّمُ لاَ يَلِيقُ بِالْجَاهِلِ. كَمْ بِالأَوْلَى لاَ يَلِيقُ بِالْعَبْدِ أَنْ يَتَسَلَّطَ عَلَى الرُّؤَسَاءِ!
\par 11 تَعَقُّلُ الإِنْسَانِ يُبْطِئُ غَضَبَهُ وَفَخْرُهُ الصَّفْحُ عَنْ مَعْصِيَةٍ.
\par 12 كَزَمْجَرَةِ الأَسَدِ حَنَقُ الْمَلِكِ وَكَالطَّلِّ عَلَى الْعُشْبِ رِضْوَانُهُ.
\par 13 اَلاِبْنُ الْجَاهِلُ مُصِيبَةٌ عَلَى أَبِيهِ وَمُخَاصَمَاتُ الزَّوْجَةِ كَالْوَكْفِ الْمُتَتَابِعِ.
\par 14 اَلْبَيْتُ وَالثَّرْوَةُ مِيرَاثٌ مِنَ الآبَاءِ أَمَّا الزَّوْجَةُ الْمُتَعَقِّلَةُ فَمِنْ عِنْدِ الرَّبِّ.
\par 15 اَلْكَسَلُ يُلْقِي فِي السُّبَاتِ وَالنَّفْسُ الْمُتَرَاخِيَةُ تَجُوعُ.
\par 16 حَافِظُ الْوَصِيَّةِ حَافِظٌ نَفْسَهُ وَالْمُتَهَاوِنُ بِطُرُقِهِ يَمُوتُ.
\par 17 مَنْ يَرْحَمُ الْفَقِيرَ يُقْرِضُ الرَّبَّ وَعَنْ مَعْرُوفِهِ يُجَازِيهِ.
\par 18 أَدِّبِ ابْنَكَ لأَنَّ فِيهِ رَجَاءً وَلَكِنْ عَلَى إِمَاتَتِهِ لاَ تَحْمِلْ نَفْسَكَ.
\par 19 اَلشَّدِيدُ الْغَضَبِ يَحْمِلُ عُقُوبَةً لأَنَّكَ إِذَا نَجَّيْتَهُ فَبَعْدُ تُعِيدُ.
\par 20 اِسْمَعِ الْمَشُورَةَ وَاقْبَلِ التَّأْدِيبَ لِكَيْ تَكُونَ حَكِيماً فِي آخِرَتِكَ.
\par 21 فِي قَلْبِ الإِنْسَانِ أَفْكَارٌ كَثِيرَةٌ لَكِنْ مَشُورَةُ الرَّبِّ هِيَ تَثْبُتُ.
\par 22 زِينَةُ الإِنْسَانِ مَعْرُوفُهُ وَالْفَقِيرُ خَيْرٌ مِنَ الْكَذُوبِ.
\par 23 مَخَافَةُ الرَّبِّ لِلْحَيَاةِ. يَبِيتُ شَبْعَانَ لاَ يَتَعَهَّدُهُ شَرٌّ.
\par 24 اَلْكَسْلاَنُ يُخْفِي يَدَهُ فِي الصَّحْفَةِ وَأَيْضاً إِلَى فَمِهِ لاَ يَرُدُّهَا.
\par 25 اِضْرِبِ الْمُسْتَهْزِئَ فَيَتَذَكَّى الأَحْمَقُ وَوَبِّخْ فَهِيماً فَيَفْهَمَ مَعْرِفَةً.
\par 26 الْمُخَرِّبُ أَبَاهُ وَالطَّارِدُ أُمَّهُ هُوَ ابْنٌ مُخْزٍ وَمُخْجِلٌ.
\par 27 كُفَّ يَا ابْنِي عَنِ اسْتِمَاعِ التَّعْلِيمِ لِلضَّلاَلَةِ عَنْ كَلاَمِ الْمَعْرِفَةِ.
\par 28 اَلشَّاهِدُ اللَّئِيمُ يَسْتَهْزِئُ بِالْحَقِّ وَفَمُ الأَشْرَارِ يَبْلَعُ الإِثْمَ.
\par 29 اَلْقِصَاصُ مُعَدٌّ لِلْمُسْتَهْزِئِينَ وَالضَّرْبُ لِظَهْرِ الْجُهَّالِ.

\chapter{20}

\par 1 اَلْخَمْرُ مُسْتَهْزِئَةٌ. الْمُسْكِرُ عَجَّاجٌ وَمَنْ يَتَرَنَّحُ بِهِمَا فَلَيْسَ بِحَكِيمٍ.
\par 2 رُعْبُ الْمَلِكِ كَزَمْجَرَةِ الأَسَدِ. الَّذِي يُغِيظُهُ يُخْطِئُ إِلَى نَفْسِهِ.
\par 3 مَجْدُ الرَّجُلِ أَنْ يَبْتَعِدَ عَنِ الْخِصَامِ وَكُلُّ أَحْمَقٍ يُنَازِعُ.
\par 4 اَلْكَسْلاَنُ لاَ يَحْرُثُ بِسَبَبِ الشِّتَاءِ فَيَسْتَعْطِي فِي الْحَصَادِ وَلاَ يُعْطَى.
\par 5 اَلْمَشُورَةُ فِي قَلْبِ الرَّجُلِ مِيَاهٌ عَمِيقَةٌ وَذُو الْفِطْنَةِ يَسْتَقِيهَا.
\par 6 أَكْثَرُ النَّاسِ يُنَادُونَ كُلُّ وَاحِدٍ بِصَلاَحِهِ أَمَّا الرَّجُلُ الأَمِينُ فَمَنْ يَجِدُهُ؟
\par 7 اَلصِّدِّيقُ يَسْلُكُ بِكَمَالِهِ. طُوبَى لِبَنِيهِ بَعْدَهُ.
\par 8 اَلْمَلِكُ الْجَالِسُ عَلَى كُرْسِيِّ الْقَضَاءِ يُذَرِّي بِعَيْنِهِ كُلَّ شَرٍّ.
\par 9 مَنْ يَقُولُ: «إِنِّي زَكَّيْتُ قَلْبِي تَطَهَّرْتُ مِنْ خَطِيَّتِي؟»
\par 10 مِعْيَارٌ فَمِعْيَارٌ مِكْيَالٌ فَمِكْيَالٌ كِلاَهُمَا مُكْرُهَةٌ عِنْدَ الرَّبِّ.
\par 11 اَلْوَلَدُ أَيْضاً يُعْرَفُ بِأَفْعَالِهِ هَلْ عَمَلُهُ نَقِيٌّ وَمُسْتَقِيمٌ؟
\par 12 اَلأُذُنُ السَّامِعَةُ وَالْعَيْنُ الْبَاصِرَةُ الرَّبُّ صَنَعَهُمَا كِلْتَيْهِمَا.
\par 13 لاَ تُحِبَّ النَّوْمَ لِئَلاَّ تَفْتَقِرَ. افْتَحْ عَيْنَيْكَ تَشْبَعْ خُبْزاً.
\par 14 «رَدِيءٌ رَدِيءٌ» يَقُولُ الْمُشْتَرِي وَإِذَا ذَهَبَ فَحِينَئِذٍ يَفْتَخِرُ!
\par 15 يُوجَدُ ذَهَبٌ وَكَثْرَةُ لَآلِئَ أَمَّا شِفَاهُ الْمَعْرِفَةِ فَمَتَاعٌ ثَمِينٌ.
\par 16 خُذْ ثَوْبَهُ لأَنَّهُ ضَمِنَ غَرِيباً وَلأَجْلِ الأَجَانِبِ ارْتَهِنْ مِنْهُ.
\par 17 خُبْزُ الْكَذِبِ لَذِيذٌ لِلإِنْسَانِ وَمِنْ بَعْدُ يَمْتَلِئُ فَمُهُ حَصًى.
\par 18 اَلْمَقَاصِدُ تُثَبَّتُ بِالْمَشُورَةِ وَبِالتَّدَابِيرِ اعْمَلْ حَرْباً.
\par 19 اَلسَّاعِي بِالْوِشَايَةِ يُفْشِي السِّرَّ فَلاَ تُخَالِطِ الْمُفَتِّحَ شَفَتَيْهِ.
\par 20 مَنْ سَبَّ أَبَاهُ أَوْ أُمَّهُ يَنْطَفِئُ سِرَاجُهُ فِي حَدَقَةِ الظَّلاَمِ.
\par 21 رُبَّ مُلْكٍ مُعَجِّلٍ فِي أَوَّلِهِ أَمَّا آخِرَتُهُ فَلاَ تُبَارَكُ.
\par 22 لاَ تَقُلْ: «إِنِّي أُجَازِي شَرّاً». انْتَظِرِ الرَّبَّ فَيُخَلِّصَكَ.
\par 23 مِعْيَارٌ فَمِعْيَارٌ مَكْرُهَةُ الرَّبِّ وَمَوَازِينُ الْغِشِّ غَيْرُ صَالِحَةٍ.
\par 24 مِنَ الرَّبِّ خَطَوَاتُ الرَّجُلِ. أَمَّا الإِنْسَانُ فَكَيْفَ يَفْهَمُ طَرِيقَهُ؟
\par 25 هُوَ شَرَكٌ لِلإِنْسَانِ أَنْ يَلْغُوَ قَائِلاً: «مُقَدَّسٌ». وَبَعْدَ النَّذْرِ أَنْ يَسْأَلَ!
\par 26 اَلْمَلِكُ الْحَكِيمُ يُشَتِّتُ الأَشْرَارَ وَيَرُدُّ عَلَيْهِمِ النَّوْرَجَ.
\par 27 نَفْسُ الإِنْسَانِ سِرَاجُ الرَّبِّ يُفَتِّشُ كُلَّ مَخَادِعِ الْبَطْنِ.
\par 28 الرَّحْمَةُ وَالْحَقُّ يَحْفَظَانِ الْمَلِكَ وَكُرْسِيُّهُ يُسْنَدُ بِالرَّحْمَةِ.
\par 29 فَخْرُ الشُّبَّانِ قُوَّتُهُمْ وَبَهَاءُ الشُّيُوخِ الشَّيْبُ.
\par 30 حُبُرُ جُرْحٍ مُنَقِّيَةٌ لِلشِّرِّيرِ وَضَرَبَاتٌ بَالِغَةٌ مَخَادِعَ الْبَطْنِ.

\chapter{21}

\par 1 قَلْبُ الْمَلِكِ فِي يَدِ الرَّبِّ كَجَدَاوِلِ مِيَاهٍ حَيْثُمَا شَاءَ يُمِيلُهُ.
\par 2 كُلُّ طُرُقِ الإِنْسَانِ مُسْتَقِيمَةٌ فِي عَيْنَيْهِ وَالرَّبُّ وَازِنُ الْقُلُوبِ.
\par 3 فِعْلُ الْعَدْلِ وَالْحَقِّ أَفْضَلُ عِنْدَ الرَّبِّ مِنَ الذَّبِيحَةِ.
\par 4 طُمُوحُ الْعَيْنَيْنِ وَانْتِفَاخُ الْقَلْبِ نُورُ الأَشْرَارِ خَطِيَّةٌ.
\par 5 أَفْكَارُ الْمُجْتَهِدِ إِنَّمَا هِيَ لِلْخِصْبِ وَكُلُّ عَجُولٍ إِنَّمَا هُوَ لِلْعَوَزِ.
\par 6 جَمْعُ الْكُنُوزِ بِلِسَانٍ كَاذِبٍ هُوَ بُخَارٌ مَطْرُودٌ لِطَالِبِي الْمَوْتِ.
\par 7 اِغْتِصَابُ الأَشْرَارِ يَجْرُفُهُمْ لأَنَّهُمْ أَبُوا إِجْرَاءَ الْعَدْلِ.
\par 8 طَرِيقُ رَجُلٍ مَوْزُورٍ هِيَ مُلْتَوِيَةٌ أَمَّا الزَّكِيُّ فَعَمَلُهُ مُسْتَقِيمٌ.
\par 9 اَلسُّكْنَى فِي زَاوِيَةِ السَّطْحِ خَيْرٌ مِنِ امْرَأَةٍ مُخَاصِمَةٍ وَبَيْتٍ مُشْتَرِكٍ.
\par 10 نَفْسُ الشِّرِّيرِ تَشْتَهِي الشَّرَّ. قَرِيبُهُ لاَ يَجِدُ نِعْمَةً فِي عَيْنَيْهِ.
\par 11 بِمُعَاقَبَةِ الْمُسْتَهْزِئِ يَصِيرُ الأَحْمَقُ حَكِيماً وَالْحَكِيمُ بِالإِرْشَادِ يَقْبَلُ مَعْرِفَةً.
\par 12 اَلْبَارُّ يَتَأَمَّلُ بَيْتَ الشِّرِّيرِ وَيَقْلِبُ الأَشْرَارَ فِي الشَّرِّ.
\par 13 مَنْ يَسُدُّ أُذُنَيْهِ عَنْ صُرَاخِ الْمِسْكِينِ فَهُوَ أَيْضاً يَصْرُخُ وَلاَ يُسْتَجَابُ.
\par 14 اَلْهَدِيَّةُ فِي الْخَفَاءِ تَفْثَأُ الْغَضَبَ وَالرَّشْوَةُ فِي الْحِضْنِ تَفْثَأُ السَّخَطَ الشَّدِيدَ.
\par 15 إِجْرَاءُ الْحَقِّ فَرَحٌ لِلصِّدِّيقِ وَالْهَلاَكُ لِفَاعِلِي الإِثْمِ.
\par 16 اَلرَّجُلُ الضَّالُّ عَنْ طَرِيقِ الْمَعْرِفَةِ يَسْكُنُ بَيْنَ جَمَاعَةِ الأَخِيلَةِ.
\par 17 مُحِبُّ الْفَرَحِ إِنْسَانٌ مُعْوِزٌ. مُحِبُّ الْخَمْرِ وَالدُّهْنِ لاَ يَسْتَغْنِي.
\par 18 اَلشِّرِّيرُ فِدْيَةُ الصِّدِّيقِ وَمَكَانَ الْمُسْتَقِيمِينَ الْغَادِرُ.
\par 19 اَلسُّكْنَى فِي أَرْضٍ بَرِّيَّةٍ خَيْرٌ مِنِ امْرَأَةٍ مُخَاصِمَةٍ حَرِدَةٍ.
\par 20 كَنْزٌ مُشْتَهًى وَزَيْتٌ فِي بَيْتِ الْحَكِيمِ أَمَّا الرَّجُلُ الْجَاهِلُ فَيُتْلِفُهُ.
\par 21 اَلتَّابِعُ الْعَدْلَ وَالرَّحْمَةَ يَجِدُ حَيَاةً حَظّاً وَكَرَامَةً.
\par 22 اَلْحَكِيمُ يَتَسَوَّرُ مَدِينَةَ الْجَبَابِرَةِ وَيُسْقِطُ قُوَّةَ مُعْتَمَدِهَا.
\par 23 مَنْ يَحْفَظُ فَمَهُ وَلِسَانَهُ يَحْفَظُ مِنَ الضِّيقَاتِ نَفْسَهُ.
\par 24 اَلْمُنْتَفِخُ الْمُتَكَبِّرُ اسْمُهُ «مُسْتَهْزِئٌ» عَامِلٌ بِفَيَضَانِ الْكِبْرِيَاءِ.
\par 25 شَهْوَةُ الْكَسْلاَنِ تَقْتُلُهُ لأَنَّ يَدَيْهِ تَأْبَيَانِ الشُّغْلَ.
\par 26 اَلْيَوْمَ كُلَّهُ يَشْتَهِي شَهْوَةً أَمَّا الصِّدِّيقُ فَيُعْطِي وَلاَ يُمْسِكُ.
\par 27 ذَبِيحَةُ الشِّرِّيرِ مَكْرَهَةٌ فَكَمْ بِالْحَرِيِّ حِينَ يُقَدِّمُهَا بِغِشٍّ!
\par 28 شَاهِدُ الزُّورِ يَهْلِكُ وَالرَّجُلُ السَّامِعُ لِلْحَقِّ يَتَكَلَّمُ.
\par 29 اَلشِّرِّيرُ يُوقِحُ وَجْهَهُ أَمَّا الْمُسْتَقِيمُ فَيُثَبِّتُ طُرُقَهُ.
\par 30 لَيْسَ حِكْمَةٌ وَلاَ فِطْنَةٌ وَلاَ مَشُورَةٌ تُجَاهَ الرَّبِّ.
\par 31 اَلْفَرَسُ مُعَدٌّ لِيَوْمِ الْحَرْبِ أَمَّا النُّصْرَةُ فَمِنَ الرَّبِّ.

\chapter{22}

\par 1 اَلصِّيتُ أَفْضَلُ مِنَ الْغِنَى الْعَظِيمِ وَالنِّعْمَةُ الصَّالِحَةُ أَفْضَلُ مِنَ الْفِضَّةِ وَالذَّهَبِ.
\par 2 اَلْغَنِيُّ وَالْفَقِيرُ يَتَلاَقَيَانِ. صَانِعُهُمَا كِلَيْهِمَا الرَّبُّ.
\par 3 اَلذَّكِيُّ يُبْصِرُ الشَّرَّ فَيَتَوَارَى وَالْحَمْقَى يَعْبُرُونَ فَيُعَاقَبُونَ.
\par 4 ثَوَابُ التَّوَاضُعِ وَمَخَافَةِ الرَّبِّ هُوَ غِنًى وَكَرَامَةٌ وَحَيَاةٌ.
\par 5 شَوْكٌ وَفُخُوخٌ فِي طَرِيقِ الْمُلْتَوِي. مَنْ يَحْفَظُ نَفْسَهُ يَبْتَعِدُ عَنْهَا.
\par 6 رَبِّ الْوَلَدَ فِي طَرِيقِهِ فَمَتَى شَاخَ أَيْضاً لاَ يَحِيدُ عَنْهُ.
\par 7 اَلْغَنِيُّ يَتَسَلَّطُ عَلَى الْفَقِيرِ وَالْمُقْتَرِضُ عَبْدٌ لِلْمُقْرِضِ.
\par 8 الزَّارِعُ إِثْماً يَحْصُدُ بَلِيَّةً وَعَصَا سَخَطِهِ تَفْنَى.
\par 9 اَلصَّالِحُ الْعَيْنِ هُوَ يُبَارَكُ لأَنَّهُ يُعْطِي مِنْ خُبْزِهِ لِلْفَقِيرِ.
\par 10 أُطْرُدِ الْمُسْتَهْزِئَ فَيَخْرُجَ الْخِصَامُ وَيَبْطُلَ النِّزَاعُ وَالْخِزْيُ.
\par 11 مَنْ أَحَبَّ طَهَارَةَ الْقَلْبِ فَلِنِعْمَةِ شَفَتَيْهِ يَكُونُ الْمَلِكُ صَدِيقَهُ.
\par 12 عَيْنَا الرَّبِّ تَحْفَظَانِ الْمَعْرِفَةَ وَهُوَ يَقْلِبُ كَلاَمَ الْغَادِرِينَ.
\par 13 قَالَ الْكَسْلاَنُ: «الأَسَدُ فِي الْخَارِجِ فَأُقْتَلُ فِي الشَّوَارِعِ!»
\par 14 فَمُ الأَجْنَبِيَّاتِ هُوَّةٌ عَمِيقَةٌ. مَمْقُوتُ الرَّبِّ يَسْقُطُ فِيهَا.
\par 15 اَلْجَهَالَةُ مُرْتَبِطَةٌ بِقَلْبِ الْوَلَدِ. عَصَا التَّأْدِيبِ تُبْعِدُهَا عَنْهُ.
\par 16 ظَالِمُ الْفَقِيرِ تَكْثِيراً لِمَا لَهُ وَمُعْطِي الْغَنِيِّ إِنَّمَا هُمَا لِلْعَوَزِ.
\par 17 أَمِلْ أُذْنَكَ وَاسْمَعْ كَلاَمَ الْحُكَمَاءِ وَوَجِّهْ قَلْبَكَ إِلَى مَعْرِفَتِي
\par 18 لأَنَّهُ حَسَنٌ إِنْ حَفِظْتَهَا فِي جَوْفِكَ إِنْ تَتَثَبَّتْ جَمِيعاً عَلَى شَفَتَيْكَ.
\par 19 لِيَكُونَ اتِّكَالُكَ عَلَى الرَّبِّ عَرَّفْتُكَ أَنْتَ الْيَوْمَ.
\par 20 أَلَمْ أَكْتُبْ لَكَ أُمُوراً شَرِيفَةً مِنْ جِهَةِ مُؤَامَرَةٍ وَمَعْرِفَةٍ
\par 21 لِأُعَلِّمَكَ قِسْطَ كَلاَمِ الْحَقِّ لِتَرُدَّ جَوَابَ الْحَقِّ لِلَّذِينَ أَرْسَلُوكَ؟
\par 22 لاَ تَسْلِبِ الْفَقِيرَ لِكَوْنِهِ فَقِيراً وَلاَ تَسْحَقِ الْمِسْكِينَ فِي الْبَابِ
\par 23 لأَنَّ الرَّبَّ يُقِيمُ دَعْوَاهُمْ وَيَسْلِبُ سَالِبِي أَنْفُسِهِمْ.
\par 24 لاَ تَسْتَصْحِبْ غَضُوباً وَمَعَ رَجُلٍ سَاخِطٍ لاَ تَجِئْ
\par 25 لِئَلاَّ تَأْلَفَ طُرُقَهُ وَتَأْخُذَ شَرَكاً إِلَى نَفْسِكَ.
\par 26 لاَ تَكُنْ مِنْ صَافِقِي الْكَفِّ وَلاَ مِنْ ضَامِنِي الدُّيُونِ.
\par 27 إِنْ لَمْ يَكُنْ لَكَ مَا تَفِي فَلِمَاذَا يَأْخُذُ فِرَاشَكَ مِنْ تَحْتِكَ؟
\par 28 لاَ تَنْقُلِ التُّخُمَ الْقَدِيمَ الَّذِي وَضَعَهُ آبَاؤُكَ.
\par 29 أَرَأَيْتَ رَجُلاً مُجْتَهِداً فِي عَمَلِهِ؟ أَمَامَ الْمُلُوكِ يَقِفُ. لاَ يَقِفُ أَمَامَ الرَّعَاعِ!

\chapter{23}

\par 1 إِذَا جَلَسْتَ تَأْكُلُ مَعَ مُتَسَلِّطٍ فَتَأَمَّلْ مَا هُوَ أَمَامَكَ تَأَمُّلاً
\par 2 وَضَعْ سِكِّيناً لِحَنْجَرَتِكَ إِنْ كُنْتَ شَرِهاً!
\par 3 لاَ تَشْتَهِ أَطَايِبَهُ لأَنَّهَا خُبْزُ أَكَاذِيبَ.
\par 4 لاَ تَتْعَبْ لِكَيْ تَصِيرَ غَنِيّاً. كُفَّ عَنْ فِطْنَتِكَ.
\par 5 هَلْ تُطَيِّرُ عَيْنَيْكَ نَحْوَهُ وَلَيْسَ هُوَ؟ لأَنَّهُ إِنَّمَا يَصْنَعُ لِنَفْسِهِ أَجْنِحَةً. كَالنَّسْرِ يَطِيرُ نَحْوَ السَّمَاءِ.
\par 6 لاَ تَأْكُلْ خُبْزَ ذِي عَيْنٍ شِرِّيرَةٍ وَلاَ تَشْتَهِ أَطَايِبَهُ.
\par 7 لأَنَّهُ كَمَا شَعَرَ فِي نَفْسِهِ هَكَذَا هُوَ. يَقُولُ لَكَ: «كُلْ وَاشْرَبْ» وَقَلْبُهُ لَيْسَ مَعَكَ.
\par 8 اللُّقْمَةُ الَّتِي أَكَلْتَهَا تَتَقَيَّأُهَا وَتَخْسَرُ كَلِمَاتِكَ الْحُلْوَةَ.
\par 9 فِي أُذُنَيْ جَاهِلٍ لاَ تَتَكَلَّمْ لأَنَّهُ يَحْتَقِرُ حِكْمَةَ كَلاَمِكَ.
\par 10 لاَ تَنْقُلِ التُّخْمَ الْقَدِيمَ وَلاَ تَدْخُلْ حُقُولَ الأَيْتَامِ
\par 11 لأَنَّ وَلِيَّهُمْ قَوِيٌّ. هُوَ يُقِيمُ دَعْوَاهُمْ عَلَيْكَ.
\par 12 وَجِّهْ قَلْبَكَ إِلَى الأَدَبِ وَأُذُنَيْكَ إِلَى كَلِمَاتِ الْمَعْرِفَةِ.
\par 13 لاَ تَمْنَعِ التَّأْدِيبَ عَنِ الْوَلَدِ لأَنَّكَ إِنْ ضَرَبْتَهُ بِعَصاً لاَ يَمُوتُ.
\par 14 تَضْرِبُهُ أَنْتَ بِعَصاً فَتُنْقِذُ نَفْسَهُ مِنَ الْهَاوِيَةِ.
\par 15 يَا ابْنِي إِنْ كَانَ قَلْبُكَ حَكِيماً يَفْرَحُ قَلْبِي أَنَا أَيْضاً
\par 16 وَتَبْتَهِجُ كِلْيَتَايَ إِذَا تَكَلَّمَتْ شَفَتَاكَ بِالْمُسْتَقِيمَاتِ.
\par 17 لاَ يَحْسِدَنَّ قَلْبُكَ الْخَاطِئِينَ بَلْ كُنْ فِي مَخَافَةِ الرَّبِّ الْيَوْمَ كُلَّهُ.
\par 18 لأَنَّهُ لاَ بُدَّ مِنْ ثَوَابٍ وَرَجَاؤُكَ لاَ يَخِيبُ.
\par 19 اِسْمَعْ أَنْتَ يَا ابْنِي وَكُنْ حَكِيماً وَأَرْشِدْ قَلْبَكَ فِي الطَّرِيقِ.
\par 20 لاَ تَكُنْ بَيْنَ شِرِّيبِي الْخَمْرِ بَيْنَ الْمُتْلِفِينَ أَجْسَادَهُمْ
\par 21 لأَنَّ السِّكِّيرَ وَالْمُسْرِفَ يَفْتَقِرَانِ وَالنَّوْمُ يَكْسُو الْخِرَقَ.
\par 22 اِسْمَعْ لأَبِيكَ الَّذِي وَلَدَكَ وَلاَ تَحْتَقِرْ أُمَّكَ إِذَا شَاخَتْ.
\par 23 اِقْتَنِ الْحَقَّ وَلاَ تَبِعْهُ وَالْحِكْمَةَ وَالأَدَبَ وَالْفَهْمَ.
\par 24 أَبُو الصِّدِّيقِ يَبْتَهِجُ ابْتِهَاجاً وَمَنْ وَلَدَ حَكِيماً يُسَرُّ بِهِ.
\par 25 يَفْرَحُ أَبُوكَ وَأُمُّكَ وَتَبْتَهِجُ الَّتِي وَلَدَتْكَ.
\par 26 يَا ابْنِي أَعْطِنِي قَلْبَكَ وَلْتُلاَحِظْ عَيْنَاكَ طُرُقِي.
\par 27 لأَنَّ الزَّانِيَةَ هُوَّةٌ عَمِيقَةٌ وَالأَجْنَبِيَّةَ حُفْرَةٌ ضَيِّقَةٌ.
\par 28 هِيَ أَيْضاً كَلِصٍّ تَكْمُنُ وَتَزِيدُ الْغَادِرِينَ بَيْنَ النَّاسِ.
\par 29 لِمَنِ الْوَيْلُ؟ لِمَنِ الشَّقَاوَةُ؟ لِمَنِ الْمُخَاصَمَاتُ؟ لِمَنِ الْكَرْبُ لِمَنِ الْجُرُوحُ بِلاَ سَبَبٍ؟ لِمَنِ ازْمِهْرَارُ الْعَيْنَيْنِ؟
\par 30 لِلَّذِينَ يُدْمِنُونَ الْخَمْرَ الَّذِينَ يَدْخُلُونَ فِي طَلَبِ الشَّرَابِ الْمَمْزُوجِ.
\par 31 لاَ تَنْظُرْ إِلَى الْخَمْرِ إِذَا احْمَرَّتْ حِينَ تُظْهِرُ حِبَابَهَا فِي الْكَأْسِ وَسَاغَتْ مُرَقْرِقَةً.
\par 32 فِي الآخِرِ تَلْسَعُ كَالْحَيَّةِ وَتَلْدَغُ كَالأُفْعُوانِ.
\par 33 عَيْنَاكَ تَنْظُرَانِ الأَجْنَبِيَّاتِ وَقَلْبُكَ يَنْطِقُ بِأُمُورٍ مُلْتَوِيَةٍ.
\par 34 وَتَكُونُ كَمُضْطَجِعٍ فِي قَلْبِ الْبَحْرِ أَوْ كَمُضْطَجِعٍ عَلَى رَأْسِ سَارِيَةٍ.
\par 35 يَقُولُ: «ضَرَبُونِي وَلَمْ أَتَوَجَّعْ. لَقَدْ لَكَأُونِي وَلَمْ أَعْرِفْ. مَتَى أَسْتَيْقِظُ أَعُودُ أَطْلُبُهَا بَعْدُ!»

\chapter{24}

\par 1 لاَ تَحْسِدْ أَهْلَ الشَّرِّ وَلاَ تَشْتَهِ أَنْ تَكُونَ مَعَهُمْ
\par 2 لأَنَّ قَلْبَهُمْ يَلْهَجُ بِالاِغْتِصَابِ وَشِفَاهَهُمْ تَتَكَلَّمُ بِالْمَشَقَّةِ.
\par 3 بِالْحِكْمَةِ يُبْنَى الْبَيْتُ وَبِالْفَهْمِ يُثَبَّتُ
\par 4 وَبِالْمَعْرِفَةِ تَمْتَلِئُ الْمَخَادِعُ مِنْ كُلِّ ثَرْوَةٍ كَرِيمَةٍ وَنَفِيسَةٍ.
\par 5 اَلرَّجُلُ الْحَكِيمُ فِي عِزٍّ وَذُو الْمَعْرِفَةِ مُتَشَدِّدُ الْقُوَّةِ.
\par 6 لأَنَّكَ بِالتَّدَابِيرِ تَعْمَلُ حَرْبَكَ وَالْخَلاَصُ بِكَثْرَةِ الْمُشِيرِينَ.
\par 7 اَلْحِكَمُ عَالِيَةٌ عَنِ الأَحْمَقِ. لاَ يَفْتَحْ فَمَهُ فِي الْبَابِ.
\par 8 اَلْمُتَفَكِّرُ فِي عَمَلِ الشَّرِّ يُدْعَى مُفْسِداً.
\par 9 فِكْرُ الْحَمَاقَةِ خَطِيَّةٌ وَمَكْرَهَةُ النَّاسِ الْمُسْتَهْزِئُ.
\par 10 إِنِ ارْتَخَيْتَ فِي يَوْمِ الضِّيقِ ضَاقَتْ قُوَّتُكَ.
\par 11 أَنْقِذِ الْمُنْقَادِينَ إِلَى الْمَوْتِ وَالْمَمْدُودِينَ لِلْقَتْلِ. لاَ تَمْتَنِعْ.
\par 12 إِنْ قُلْتَ: «هُوَذَا لَمْ نَعْرِفْ هَذَا» - أَفَلاَ يَفْهَمُ وَازِنُ الْقُلُوبِ وَحَافِظُ نَفْسِكَ أَلاَ يَعْلَمُ؟ فَيَرُدُّ عَلَى الإِنْسَانِ مِثْلَ عَمَلِهِ.
\par 13 يَا ابْنِي كُلْ عَسَلاً لأَنَّهُ طَيِّبٌ وَقَطْرَ الْعَسَلِ حُلْوٌ فِي حَنَكِكَ.
\par 14 كَذَلِكَ مَعْرِفَةُ الْحِكْمَةِ لِنَفْسِكَ. إِذَا وَجَدْتَهَا فَلاَ بُدَّ مِنْ ثَوَابٍ وَرَجَاؤُكَ لاَ يَخِيبُ.
\par 15 لاَ تَكْمُنْ أَيُّهَا الشِّرِّيرُ لِمَسْكَنِ الصِّدِّيقِ. لاَ تُخْرِبْ رَبْعَهُ.
\par 16 لأَنَّ الصِّدِّيقَ يَسْقُطُ سَبْعَ مَرَّاتٍ وَيَقُومُ. أَمَّا الأَشْرَارُ فَيَعْثُرُونَ بِالشَّرِّ.
\par 17 لاَ تَفْرَحْ بِسُقُوطِ عَدُوِّكَ وَلاَ يَبْتَهِجْ قَلْبُكَ إِذَا عَثَرَ
\par 18 لِئَلاَّ يَرَى الرَّبُّ وَيَسُوءَ ذَلِكَ فِي عَيْنَيْهِ فَيَرُدَّ عَنْهُ غَضَبَهُ.
\par 19 لاَ تَغَرْ مِنَ الأَشْرَارِ وَلاَ تَحْسِدِ الأَثَمَةَ.
\par 20 لأَنَّهُ لاَ يَكُونُ ثَوَابٌ لِلأَشْرَارِ. سِرَاجُ الأَثَمَةِ يَنْطَفِئُ.
\par 21 يَا ابْنِي اخْشَ الرَّبَّ وَالْمَلِكَ. لاَ تُخَالِطِ الْمُتَقَلِّبِينَ
\par 22 لأَنَّ بَلِيَّتَهُمْ تَقُومُ بَغْتَةً وَمَنْ يَعْلَمُ بَلاَءَهُمَا كِلَيْهِمَا.
\par 23 هَذِهِ أَيْضاً لِلْحُكَمَاءِ: مُحَابَاةُ الْوُجُوهِ فِي الْحُكْمِ لَيْسَتْ صَالِحَةً.
\par 24 مَنْ يَقُولُ لِلشِّرِّيرِ: «أَنْتَ صِدِّيقٌ» تَسُبُّهُ الْعَامَّةُ. تَلْعَنُهُ الشُّعُوبُ.
\par 25 أَمَّا الَّذِينَ يُؤَدِّبُونَ فَيَنْعَمُونَ وَبَرَكَةُ خَيْرٍ تَأْتِي عَلَيْهِمْ.
\par 26 تُقَبَّلُ شَفَتَا مَنْ يُجَاوِبُ بِكَلاَمٍ مُسْتَقِيمٍ.
\par 27 هَيِّئْ عَمَلَكَ فِي الْخَارِجِ وَأَعِدَّهُ فِي حَقْلِكَ. بَعْدُ تَبْنِي بَيْتَكَ.
\par 28 لاَ تَكُنْ شَاهِداً عَلَى قَرِيبِكَ بِلاَ سَبَبٍ فَهَلْ تُخَادِعُ بِشَفَتَيْكَ؟
\par 29 لاَ تَقُلْ: «كَمَا فَعَلَ بِي هَكَذَا أَفْعَلُ بِهِ. أَرُدُّ عَلَى الإِنْسَانِ مِثْلَ عَمَلِهِ».
\par 30 عَبَرْتُ بِحَقْلِ الْكَسْلاَنِ وَبِكَرْمِ الرَّجُلِ النَّاقِصِ الْفَهْمِ
\par 31 فَإِذَا هُوَ قَدْ عَلاَهُ كُلَّهُ الْقَرِيصُ وَقَدْ غَطَّى الْعَوْسَجُ وَجْهَهُ وَجِدَارُ حِجَارَتِهِ انْهَدَمَ.
\par 32 ثُمَّ نَظَرْتُ وَوَجَّهْتُ قَلْبِي. رَأَيْتُ وَقَبِلْتُ تَعْلِيماً.
\par 33 نَوْمٌ قَلِيلٌ بَعْدُ نُعَاسٌ قَلِيلٌ وَطَيُّ الْيَدَيْنِ قَلِيلاً لِلرُّقُودِ
\par 34 فَيَأْتِي فَقْرُكَ كَعَدَّاءٍ وَعَوَزُكَ كَغَازٍ!

\chapter{25}

\par 1 هَذِهِ أَيْضاً أَمْثَالُ سُلَيْمَانَ الَّتِي نَقَلَهَا رِجَالُ حَزَقِيَّا مَلِكِ يَهُوذَا:
\par 2 مَجْدُ اللَّهِ إِخْفَاءُ الأَمْرِ وَمَجْدُ الْمُلُوكِ فَحْصُ الأَمْرِ.
\par 3 اَلسَّمَاءُ لِلْعُلُوِّ وَالأَرْضُ لِلْعُمْقِ وَقُلُوبُ الْمُلُوكِ لاَ تُفْحَصُ.
\par 4 أَزِلِ الزَّغَلَ مِنَ الْفِضَّةِ فَيَخْرُجَ إِنَاءٌ لِلصَّائِغِ.
\par 5 أَزِلِ الشِّرِّيرَ مِنْ قُدَّامِ الْمَلِكِ فَيُثَبَّتَ كُرْسِيُّهُ بِالْعَدْلِ.
\par 6 لاَ تَتَفَاخَرْ أَمَامَ الْمَلِكِ وَلاَ تَقِفْ فِي مَكَانِ الْعُظَمَاءِ
\par 7 لأَنَّهُ خَيْرٌ أَنْ يُقَالَ لَكَ ارْتَفِعْ إِلَى هُنَا مِنْ أَنْ تُحَطَّ فِي حَضْرَةِ الرَّئِيسِ الَّذِي رَأَتْهُ عَيْنَاكَ.
\par 8 لاَ تَبْرُزْ عَاجِلاً إِلَى الْخِصَامِ لِئَلاَّ تَفْعَلَ شَيْئاً فِي الآخِرِ حِينَ يُخْزِيكَ قَرِيبُكَ.
\par 9 أَقِمْ دَعْوَاكَ مَعَ قَرِيبِكَ وَلاَ تُبِحْ بِسِرِّ غَيْرِكَ
\par 10 لِئَلاَّ يُعَيِّرَكَ السَّامِعُ فَلاَ تَنْصَرِفَ فَضِيحَتُكَ.
\par 11 تُفَّاحٌ مِنْ ذَهَبٍ فِي مَصُوغٍ مِنْ فِضَّةٍ كَلِمَةٌ مَقُولَةٌ فِي مَحَلِّهَا.
\par 12 قُرْطٌ مِنْ ذَهَبٍ وَحُلِيٌّ مِنْ إِبْرِيزٍ الْمُوَبِّخُ الْحَكِيمُ لِأُذُنٍ سَامِعَةٍ.
\par 13 كَبَرْدِ الثَّلْجِ فِي يَوْمِ الْحَصَادِ الرَّسُولُ الأَمِينُ لِمُرْسِلِيهِ لأَنَّهُ يَرُدُّ نَفْسَ سَادَتِهِ.
\par 14 سَحَابٌ وَرِيحٌ بِلاَ مَطَرٍ الرَّجُلُ الْمُفْتَخِرُ بِهَدِيَّةِ كَذِبٍ.
\par 15 بِبُطْءِ الْغَضَبِ يُقْنَعُ الرَّئِيسُ وَاللِّسَانُ اللَّيِّنُ يَكْسِرُ الْعَظْمَ.
\par 16 أَوَجَدْتَ عَسَلاً؟ فَكُلْ كِفَايَتَكَ لِئَلاَّ تَتَّخِمَ فَتَتَقَيَّأَهُ.
\par 17 اِجْعَلْ رِجْلَكَ عَزِيزَةً فِي بَيْتِ قَرِيبِكَ لِئَلاَّ يَمَلَّ مِنْكَ فَيُبْغِضَكَ.
\par 18 مِقْمَعَةٌ وَسَيْفٌ وَسَهْمٌ حَادٌّ الرَّجُلُ الْمُجِيبُ قَرِيبَهُ بِشَهَادَةِ زُورٍ.
\par 19 سِنٌّ مَهْتُومَةٌ وَرِجْلٌ مُخَلَّعَةٌ الثِّقَةُ بِالْخَائِنِ فِي يَوْمِ الضِّيقِ.
\par 20 كَنَزْعِ الثَّوْبِ فِي يَوْمِ الْبَرْدِ كَخَلٍّ عَلَى نَطْرُونٍ مَنْ يُغَنِّي أَغَانِيَّ لِقَلْبٍ كَئِيبٍ.
\par 21 إِنْ جَاعَ عَدُوُّكَ فَأَطْعِمْهُ خُبْزاً وَإِنْ عَطِشَ فَاسْقِهِ مَاءً
\par 22 فَإِنَّكَ تَجْمَعُ جَمْراً عَلَى رَأْسِهِ وَالرَّبُّ يُجَازِيكَ.
\par 23 رِيحُ الشِّمَالِ تَطْرُدُ الْمَطَرَ وَالْوَجْهُ الْمُعْبِسُ يَطْرُدُ لِسَاناً ثَالِباً.
\par 24 اَلسُّكْنَى فِي زَاوِيَةِ السَّطْحِ خَيْرٌ مِنِ امْرَأَةٍ مُخَاصِمَةٍ فِي بَيْتٍ مُشْتَرِكٍ.
\par 25 مِيَاهٌ بَارِدَةٌ لِنَفْسٍ عَطْشَانَةٍ الْخَبَرُ الطَّيِّبُ مِنْ أَرْضٍ بَعِيدَةٍ.
\par 26 عَيْنٌ مُكَدَّرَةٌ وَيَنْبُوعٌ فَاسِدٌ الصِّدِّيقُ الْمُنْحَنِي أَمَامَ الشِّرِّيرِ.
\par 27 أَكْلُ كَثِيرٍ مِنَ الْعَسَلِ لَيْسَ بِحَسَنٍ وَطَلَبُ النَّاسِ مَجْدَ أَنْفُسِهِمْ ثَقِيلٌ.
\par 28 مَدِينَةٌ مُنْهَدِمَةٌ بِلاَ سُورٍ الرَّجُلُ الَّذِي لَيْسَ لَهُ سُلْطَانٌ عَلَى رُوحِهِ.

\chapter{26}

\par 1 كَالثَّلْجِ فِي الصَّيْفِ وَكَالْمَطَرِ فِي الْحَصَادِ هَكَذَا الْكَرَامَةُ غَيْرُ لاَئِقَةٍ بِالْجَاهِلِ.
\par 2 كَالْعُصْفُورِ لِلْفَرَارِ وَكَالسُّنُونَةِ لِلطَّيَرَانِ كَذَلِكَ لَعْنَةٌ بِلاَ سَبَبٍ لاَ تَأْتِي.
\par 3 اَلسَّوْطُ لِلْفَرَسِ وَاللِّجَامُ لِلْحِمَارِ وَالْعَصَا لِظَهْرِ الْجُهَّالِ.
\par 4 لاَ تُجَاوِبِ الْجَاهِلَ حَسَبَ حَمَاقَتِهِ لِئَلاَّ تَعْدِلَهُ أَنْتَ.
\par 5 جَاوِبِ الْجَاهِلَ حَسَبَ حَمَاقَتِهِ لِئَلاَّ يَكُونَ حَكِيماً فِي عَيْنَيْ نَفْسِهِ.
\par 6 يَقْطَعُ الرِّجْلَيْنِ يَشْرَبُ ظُلْماً مَنْ يُرْسِلُ كَلاَماً عَنْ يَدِ جَاهِلٍ.
\par 7 سَاقَا الأَعْرَجِ مُتَدَلْدِلَتَانِ وَكَذَا الْمَثَلُ فِي فَمِ الْجُهَّالِ.
\par 8 كَصُرَّةِ حِجَارَةٍ كَرِيمَةٍ فِي رُجْمَةٍ هَكَذَا الْمُعْطِي كَرَامَةً لِلْجَاهِلِ.
\par 9 شَوْكٌ مُرْتَفِعٌ بِيَدِ سَكْرَانٍ مِثْلُ الْمَثَلِ فِي فَمِ الْجُهَّالِ.
\par 10 رَامٍ يَطْعَنُ الْكُلَّ هَكَذَا مَنْ يَسْتَأْجِرُ الْجَاهِلَ أَوْ يَسْتَأْجِرُ الْمُحْتَالِينَ.
\par 11 كَمَا يَعُودُ الْكَلْبُ إِلَى قَيْئِهِ هَكَذَا الْجَاهِلُ يُعِيدُ حَمَاقَتَهُ.
\par 12 أَرَأَيْتَ رَجُلاً حَكِيماً فِي عَيْنَيْ نَفْسِهِ؟ الرَّجَاءُ بِالْجَاهِلِ أَكْثَرُ مِنَ الرَّجَاءِ بِهِ!
\par 13 قَالَ الْكَسْلاَنُ: «الأَسَدُ فِي الطَّرِيقِ الشِّبْلُ فِي الشَّوَارِعِ».
\par 14 اَلْبَابُ يَدُورُ عَلَى صَائِرِهِ وَالْكَسْلاَنُ عَلَى فِرَاشِهِ.
\par 15 اَلْكَسْلاَنُ يُخْفِي يَدَهُ فِي الصَّحْفَةِ وَيَشُقُّ عَلَيْهِ أَنْ يَرُدَّهَا إِلَى فَمِهِ.
\par 16 اَلْكَسْلاَنُ أَوْفَرُ حِكْمَةً فِي عَيْنَيْ نَفْسِهِ مِنَ السَّبْعَةِ الْمُجِيبِينَ بِعَقْلٍ.
\par 17 كَمُمْسِكٍ أُذُنَيْ كَلْبٍ هَكَذَا مَنْ يَعْبُرُ وَيَتَعَرَّضُ لِمُشَاجَرَةٍ لاَ تَعْنِيهِ.
\par 18 مِثْلُ الْمَجْنُونِ الَّذِي يَرْمِي نَاراً وَسِهَاماً وَمَوْتاً
\par 19 هَكَذَا الرَّجُلُ الْخَادِعُ قَرِيبَهُ وَيَقُولُ: «أَلَمْ أَلْعَبْ أَنَا!»
\par 20 بِعَدَمِ الْحَطَبِ تَنْطَفِئُ النَّارُ وَحَيْثُ لاَ نَمَّامَ يَهْدَأُ الْخِصَامُ.
\par 21 فَحْمٌ لِلْجَمْرِ وَحَطَبٌ لِلنَّارِ هَكَذَا الرَّجُلُ الْمُخَاصِمُ لِتَهْيِيجِ النِّزَاعِ.
\par 22 كَلاَمُ النَّمَّامِ مِثْلُ لُقَمٍ حُلْوَةٍ فَيَنْزِلُ إِلَى مَخَادِعِ الْبَطْنِ.
\par 23 فِضَّةُ زَغَلٍ تُغَشِّي شَقْفَةً هَكَذَا الشَّفَتَانِ الْمُتَوَقِّدَتَانِ وَالْقَلْبُ الشِّرِّيرُ.
\par 24 بِشَفَتَيْهِ يَتَنَكَّرُ الْمُبْغِضُ وَفِي جَوْفِهِ يَضَعُ غِشّاً.
\par 25 إِذَا حَسَّنَ صَوْتَهُ فَلاَ تَأْتَمِنْهُ لأَنَّ فِي قَلْبِهِ سَبْعَ رَجَاسَاتٍ.
\par 26 مَنْ يُغَطِّي بُغْضَةً بِمَكْرٍ يَكْشِفُ خُبْثَهُ بَيْنَ الْجَمَاعَةِ.
\par 27 مَنْ يَحْفُرُ حُفْرَةً يَسْقُطُ فِيهَا وَمَنْ يُدَحْرِجُ حَجَراً يَرْجِعُ عَلَيْهِ.
\par 28 اَللِّسَانُ الْكَاذِبُ يُبْغِضُ مُنْسَحِقِيهِ وَالْفَمُ الْمَلِقُ يُعِدُّ خَرَاباً.

\chapter{27}

\par 1 لاَ تَفْتَخِرْ بِالْغَدِ لأَنَّكَ لاَ تَعْلَمُ مَاذَا يَلِدُهُ يَوْمٌ.
\par 2 لِيَمْدَحْكَ الْغَرِيبُ لاَ فَمُكَ الأَجْنَبِيُّ لاَ شَفَتَاكَ.
\par 3 اَلْحَجَرُ ثَقِيلٌ وَالرَّمْلُ ثَقِيلٌ وَغَضَبُ الْجَاهِلِ أَثْقَلُ مِنْهُمَا كِلَيْهِمَا.
\par 4 اَلْغَضَبُ قَسَاوَةٌ وَالسَّخَطُ جُرَافٌ وَمَنْ يَقِفُ قُدَّامَ الْحَسَدِ؟
\par 5 اَلتَّوْبِيخُ الظَّاهِرُ خَيْرٌ مِنَ الْحُبِّ الْمُسْتَتِرِ.
\par 6 أَمِينَةٌ هِيَ جُرُوحُ الْمُحِبِّ وَغَاشَّةٌ هِيَ قُبْلاَتُ الْعَدُوِّ.
\par 7 اَلنَّفْسُ الشَّبْعَانَةُ تَدُوسُ الْعَسَلَ وَلِلنَّفْسِ الْجَائِعَةِ كُلُّ مُرٍّ حُلْوٌ.
\par 8 مِثْلُ الْعُصْفُورِ التَّائِهِ مِنْ عُشِّهِ هَكَذَا الرَّجُلُ التَّائِهُ مِنْ مَكَانِهِ.
\par 9 اَلدُّهْنُ وَالْبَخُورُ يُفَرِّحَانِ الْقَلْبَ وَحَلاَوَةُ الصَّدِيقِ مِنْ مَشُورَةِ النَّفْسِ.
\par 10 لاَ تَتْرُكْ صَدِيقَكَ وَصَدِيقَ أَبِيكَ وَلاَ تَدْخُلْ بَيْتَ أَخِيكَ فِي يَوْمِ بَلِيَّتِكَ. الْجَارُ الْقَرِيبُ خَيْرٌ مِنَ الأَخِ الْبَعِيدِ.
\par 11 يَا ابْنِي كُنْ حَكِيماً وَفَرِّحْ قَلْبِي فَأُجِيبَ مَنْ يُعَيِّرُنِي كَلِمَةً.
\par 12 الذَّكِيُّ يُبْصِرُ الشَّرَّ فَيَتَوَارَى. الأَغْبِيَاءُ يَعْبُرُونَ فَيُعَاقَبُونَ.
\par 13 خُذْ ثَوْبَهُ لأَنَّهُ ضَمِنَ غَرِيباً وَلأَجْلِ الأَجَانِبِ ارْتَهِنَ مِنْهُ.
\par 14 مَنْ يُبَارِكُ قَرِيبَهُ بِصَوْتٍ عَالٍ فِي الصَّبَاحِ بَاكِراً يُحْسَبُ لَهُ لَعْناً.
\par 15 اَلْوَكْفُ الْمُتَتَابِعُ فِي يَوْمٍ مُمْطِرٍ وَالْمَرْأَةُ الْمُخَاصِمَةُ سِيَّانِ
\par 16 مَنْ يُخَبِّئُهَا يُخَبِّئُ الرِّيحَ وَيَمِينُهُ تَقْبِضُ عَلَى زَيْتٍ!
\par 17 الْحَدِيدُ بِالْحَدِيدِ يُحَدَّدُ وَالإِنْسَانُ يُحَدِّدُ وَجْهَ صَاحِبِهِ.
\par 18 مَنْ يَحْمِي تِينَةً يَأْكُلُ ثَمَرَتَهَا وَحَافِظُ سَيِّدِهِ يُكْرَمُ.
\par 19 كَمَا فِي الْمَاءِ الْوَجْهُ لِلْوَجْهِ كَذَلِكَ قَلْبُ الإِنْسَانِ لِلإِنْسَانِ.
\par 20 اَلْهَاوِيَةُ وَالْهَلاَكُ لاَ يَشْبَعَانِ وَكَذَا عَيْنَا الإِنْسَانِ لاَ تَشْبَعَانِ.
\par 21 اَلْبُوطَةُ لِلْفِضَّةِ وَالْكُورُ لِلذَّهَبِ كَذَا الإِنْسَانُ لِفَمِ مَادِحِهِ.
\par 22 إِنْ دَقَقْتَ الأَحْمَقَ فِي هَاوُنٍ بَيْنَ السَّمِيذِ بِمِدَقٍّ لاَ تَبْرَحُ عَنْهُ حَمَاقَتُهُ.
\par 23 مَعْرِفَةً اعْرِفْ حَالَ غَنَمِكَ وَاجْعَلْ قَلْبَكَ إِلَى قُطْعَانِكَ
\par 24 لأَنَّ الْغِنَى لَيْسَ بِدَائِمٍ وَلاَ التَّاجُ لِدَوْرٍ فَدَوْرٍ.
\par 25 فَنِيَ الْحَشِيشُ وَظَهَرَ الْعُشْبُ وَاجْتَمَعَ نَبَاتُ الْجِبَالِ.
\par 26 الْحُمْلاَنُ لِلِبَاسِكَ وَثَمَنُ حَقْلٍ أَعْتِدَةٌ.
\par 27 وَكِفَايَةٌ مِنْ لَبَنِ الْمَعْزِ لِطَعَامِكَ لِقُوتِ بَيْتِكَ وَمَعِيشَةِ فَتَيَاتِكَ.

\chapter{28}

\par 1 اَلشِّرِّيرُ يَهْرُبُ وَلاَ طَارِدَ أَمَّا الصِّدِّيقُونَ فَكَشِبْلٍ ثَبِيتٍ.
\par 2 لِمَعْصِيَةِ أَرْضٍ تَكْثُرُ رُؤَسَاؤُهَا لَكِنْ بِذِي فَهْمٍ وَمَعْرِفَةٍ تَدُومُ.
\par 3 اَلرَّجُلُ الْفَقِيرُ الَّذِي يَظْلِمُ فُقَرَاءَ هُوَ مَطَرٌ جَارِفٌ لاَ يُبْقِي طَعَاماً.
\par 4 تَارِكُو الشَّرِيعَةِ يَمْدَحُونَ الأَشْرَارَ وَحَافِظُو الشَّرِيعَةِ يُخَاصِمُونَهُمْ.
\par 5 اَلنَّاسُ الأَشْرَارُ لاَ يَفْهَمُونَ الْحَقَّ وَطَالِبُو الرَّبِّ يَفْهَمُونَ كُلَّ شَيْءٍ.
\par 6 اَلْفَقِيرُ السَّالِكُ بِاسْتِقَامَتِهِ خَيْرٌ مِنْ مُعَوَّجِ الطُّرُقِ وَهُوَ غَنِيٌّ.
\par 7 اَلْحَافِظُ الشَّرِيعَةَ هُوَ ابْنٌ فَهِيمٌ وَصَاحِبُ الْمُسْرِفِينَ يُخْجِلُ أَبَاهُ.
\par 8 اَلْمُكْثِرُ مَالَهُ بِالرِّبَا وَالْمُرَابَحَةِ فَلِمَنْ يَرْحَمُ الْفُقَرَاءَ يَجْمَعُهُ!
\par 9 مَنْ يُحَوِّلُ أُذْنَهُ عَنْ سَمَاعِ الشَّرِيعَةِ فَصَلاَتُهُ أَيْضاً مَكْرَهَةٌ.
\par 10 مَنْ يُضِلُّ الْمُسْتَقِيمِينَ فِي طَرِيقٍ رَدِيئَةٍ فَفِي حُفْرَتِهِ يَسْقُطُ هُوَ. أَمَّا الْكَمَلَةُ فَيَمْتَلِكُونَ خَيْراً.
\par 11 اَلرَّجُلُ الْغَنِيُّ حَكِيمٌ فِي عَيْنَيْ نَفْسِهِ وَالْفَقِيرُ الْفَهِيمُ يَفْحَصُهُ.
\par 12 إِذَا فَرِحَ الصِّدِّيقُونَ عَظُمَ الْفَخْرُ وَعِنْدَ قِيَامِ الأَشْرَارِ تَخْتَفِي النَّاسُ.
\par 13 مَنْ يَكْتُمُ خَطَايَاهُ لاَ يَنْجَحُ وَمَنْ يُقِرُّ بِهَا وَيَتْرُكُهَا يُرْحَمُ.
\par 14 طُوبَى لِلإِنْسَانِ الْمُتَّقِي دَائِماً أَمَّا الْمُقَسِّي قَلْبَهُ فَيَسْقُطُ فِي الشَّرِّ.
\par 15 أَسَدٌ زَائِرٌ وَدُبٌّ ثَائِرٌ الْمُتَسَلِّطُ الشِّرِّيرُ عَلَى شَعْبٍ فَقِيرٍ.
\par 16 رَئِيسٌ نَاقِصُ الْفَهْمِ وَكَثِيرُ الْمَظَالِمِ. مُبْغِضُ الرَّشْوَةِ تَطُولُ أَيَّامُهُ.
\par 17 اَلرَّجُلُ الْمُثَقَّلُ بِدَمِ نَفْسٍ يَهْرُبُ إِلَى الْجُبِّ. لاَ يُمْسِكَنَّهُ أَحَدٌ.
\par 18 اَلسَّالِكُ بِالْكَمَالِ يَخْلُصُ وَالْمُلْتَوِي فِي طَرِيقَيْنِ يَسْقُطُ فِي إِحْدَاهُمَا.
\par 19 اَلْمُشْتَغِلُ بِأَرْضِهِ يَشْبَعُ خُبْزاً وَتَابِعُ الْبَطَّالِينَ يَشْبَعُ فَقْراً.
\par 20 اَلرَّجُلُ الأَمِينُ كَثِيرُ الْبَرَكَاتِ وَالْمُسْتَعْجِلُ إِلَى الْغِنَى لاَ يُبْرَأُ.
\par 21 مُحَابَاةُ الْوُجُوهِ لَيْسَتْ صَالِحَةً فَيُذْنِبُ الإِنْسَانُ لأَجْلِ كِسْرَةِ خُبْزٍ.
\par 22 ذُو الْعَيْنِ الشِّرِّيرَةِ يَعْجَلُ إِلَى الْغِنَى وَلاَ يَعْلَمُ أَنَّ الْفَقْرَ يَأْتِيهِ.
\par 23 مَنْ يُوَبِّخُ إِنْسَاناً يَجِدُ أَخِيراً نِعْمَةً أَكْثَرَ مِنَ الْمُطْرِي بِاللِّسَانِ.
\par 24 السَّالِبُ أَبَاهُ أَوْ أُمَّهُ وَهُوَ يَقُولُ: «لاَ بَأْسَ» فَهُوَ رَفِيقٌ لِرَجُلٍ مُخْرِبٍ.
\par 25 اَلْمُنْتَفِخُ النَّفْسُ يُهَيِّجُ الْخِصَامَ وَالْمُتَّكِلُ عَلَى الرَّبِّ يُسَمَّنُ.
\par 26 اَلْمُتَّكِلُ عَلَى قَلْبِهِ هُوَ جَاهِلٌ وَالسَّالِكُ بِحِكْمَةٍ هُوَ يَنْجُو.
\par 27 مَنْ يُعْطِي الْفَقِيرَ لاَ يَحْتَاجُ وَلِمَنْ يَحْجِبُ عَنْهُ عَيْنَيْهِ لَعَنَاتٌ كَثِيرَةٌ.
\par 28 عِنْدَ قِيَامِ الأَشْرَارِ تَخْتَبِئُ النَّاسُ وَبِهَلاَكِهِمْ يَكْثُرُ الصِّدِّيقُونَ.

\chapter{29}

\par 1 اَلْكَثِيرُ التَّوَبُّخِ الْمُقَسِّي عُنُقَهُ بَغْتَةً يُكَسَّرُ وَلاَ شِفَاءَ.
\par 2 إِذَا سَادَ الصِّدِّيقُونَ فَرِحَ الشَّعْبُ وَإِذَا تَسَلَّطَ الشِّرِّيرُ يَئِنُّ الشَّعْبُ.
\par 3 مَنْ يُحِبُّ الْحِكْمَةَ يُفَرِّحُ أَبَاهُ وَرَفِيقُ الزَّوَانِي يُبَدِّدُ مَالاً.
\par 4 اَلْمَلِكُ بِالْعَدْلِ يُثَبِّتُ الأَرْضَ وَالْقَابِلُ الْهَدَايَا يُدَمِّرُهَا.
\par 5 اَلرَّجُلُ الَّذِي يُطْرِي صَاحِبَهُ يَبْسُطُ شَبَكَةً لِرِجْلَيْهِ.
\par 6 فِي مَعْصِيَةِ رَجُلٍ شِرِّيرٍ شَرَكٌ أَمَّا الصِّدِّيقُ فَيَتَرَنَّمُ وَيَفْرَحُ.
\par 7 الصِّدِّيقُ يَعْرِفُ دَعْوَى الْفُقَرَاءِ أَمَّا الشِّرِّيرُ فَلاَ يَفْهَمُ مَعْرِفَةً.
\par 8 اَلنَّاسُ الْمُسْتَهْزِئُونَ يَفْتِنُونَ الْمَدِينَةَ أَمَّا الْحُكَمَاءُ فَيَصْرِفُونَ الْغَضَبَ.
\par 9 رَجُلٌ حَكِيمٌ إِنْ حَاكَمَ رَجُلاً أَحْمَقَ فَإِنْ غَضِبَ وَإِنْ ضَحِكَ فَلاَ رَاحَةَ.
\par 10 أَهْلُ الدِّمَاءِ يُبْغِضُونَ الْكَامِلَ أَمَّا الْمُسْتَقِيمُونَ فَيَسْأَلُونَ عَنْ نَفْسِهِ.
\par 11 اَلْجَاهِلُ يُظْهِرُ كُلَّ غَيْظِهِ وَالْحَكِيمُ يُسَكِّنُهُ أَخِيراً.
\par 12 اَلْحَاكِمُ الْمُصْغِي إِلَى كَلاَمِ كَذِبٍ كُلُّ خُدَّامِهِ أَشْرَارٌ.
\par 13 اَلْفَقِيرُ وَالْظَّالِمُ يَتَلاَقَيَانِ. الرَّبُّ يُنَوِّرُ أَعْيُنَ كِلَيْهِمَا.
\par 14 اَلْمَلِكُ الْحَاكِمُ بِالْحَقِّ لِلْفُقَرَاءِ يُثَبَّتُ كُرْسِيُّهُ إِلَى الأَبَدِ.
\par 15 اَلْعَصَا وَالتَّوْبِيخُ يُعْطِيَانِ حِكْمَةً وَالصَّبِيُّ الْمُطْلَقُ إِلَى هَوَاهُ يُخْجِلُ أُمَّهُ.
\par 16 إِذَا سَادَ الأَشْرَارُ كَثُرَتِ الْمَعَاصِي. أَمَّا الصِّدِّيقُونَ فَيَنْظُرُونَ سُقُوطَهُمْ.
\par 17 أَدِّبِ ابْنَكَ فَيُرِيحَكَ وَيُعْطِيَ نَفْسَكَ لَذَّاتٍ.
\par 18 بِلاَ رُؤْيَا يَجْمَحُ الشَّعْبُ أَمَّا حَافِظُ الشَّرِيعَةِ فَطُوبَاهُ.
\par 19 بِالْكَلاَمِ لاَ يُؤَدَّبُ الْعَبْدُ لأَنَّهُ يَفْهَمُ وَلاَ يُعْنَى.
\par 20 أَرَأَيْتَ إِنْسَاناً عَجُولاً فِي كَلاَمِهِ؟ الرَّجَاءُ بِالْجَاهِلِ أَكْثَرُ مِنَ الرَّجَاءِ بِهِ.
\par 21 مَنْ دَلَّلَ عَبْدَهُ مِنْ حَدَاثَتِهِ فَفِي آخِرَتِهِ يَصِيرُ مَنُوناً.
\par 22 اَلرَّجُلُ الْغَضُوبُ يُهَيِّجُ الْخِصَامَ وَالرَّجُلُ السَّخُوطُ كَثِيرُ الْمَعَاصِي.
\par 23 كِبْرِيَاءُ الإِنْسَانِ تَضَعُهُ وَالْوَضِيعُ الرُّوحِ يَنَالُ مَجْداً.
\par 24 مَنْ يُقَاسِمْ سَارِقاً يُبْغِضْ نَفْسَهُ. يَسْمَعُ اللَّعْنَ وَلاَ يُقِرُّ.
\par 25 خَشْيَةُ الإِنْسَانِ تَضَعُ شَرَكاً وَالْمُتَّكِلُ عَلَى الرَّبِّ يُرْفَعُ.
\par 26 كَثِيرُونَ يَطْلُبُونَ وَجْهَ الْمُتَسَلِّطِ أَمَّا حَقُّ الإِنْسَانِ فَمِنَ الرَّبِّ.
\par 27 اَلرَّجُلُ الظَّالِمُ مَكْرَهَةُ الصِّدِّيقِينَ وَالْمُسْتَقِيمُ الطَّرِيقِ مَكْرَهَةُ الشِّرِّيرِ.

\chapter{30}

\par 1 كَلاَمُ أَجُورَ ابْنِ مُتَّقِيَةِ مَسَّا. وَحْيُ هَذَا الرَّجُلِ إِلَى إِيثِيئِيلَ. إِلَى إِيثِيئِيلَ وَأُكَّالَ:
\par 2 إِنِّي أَبْلَدُ مِن كُلِّ إِنْسَانٍ وَلَيْسَ لِي فَهْمُ إِنْسَانٍ
\par 3 وَلَمْ أَتَعَلَّمِ الْحِكْمَةَ وَلَمْ أَعْرِفْ مَعْرِفَةَ الْقُدُّوسِ.
\par 4 مَن صَعِدَ إِلَى السَّمَاوَاتِ وَنَزَلَ؟ مَن جَمَعَ الرِّيحَ في حُفْنَتَيْهِ؟ مَن صَرَّ الْمِيَاهَ في ثَوْبٍ؟ مَن ثَبَّتَ جَمِيعَ أَطْرَافِ الأَرْضِ؟ مَا اسْمُهُ وَمَا اسْمُ ابْنِهِ إِنْ عَرَفْتَ؟
\par 5 كُلُّ كَلِمَةٍ مِنَ اللَّهِ نَقِيَّةٌ. تُرْسٌ هُوَ لِلْمُحْتَمِينَ بِهِ.
\par 6 لاَ تَزِدْ عَلَى كَلِمَاتِهِ لِئَلاَّ يُوَبِّخَكَ فَتُكَذَّبَ.
\par 7 اِثْنَتَيْنِ سَأَلْتُ مِنْكَ فَلاَ تَمْنَعْهُمَا عَنِّي قَبْلَ أَنْ أَمُوتَ:
\par 8 أَبْعِدْ عَنِّي الْبَاطِلَ وَالْكَذِبَ. لاَ تُعْطِنِي فَقْراً وَلاَ غِنىً. أَطْعِمْنِي خُبْزَ فَرِيضَتِي
\par 9 لِئَلاَّ أَشْبَعَ وَأَكْفُرَ وَأَقُولَ: «مَنْ هُوَ الرَّبُّ؟» أَوْ لِئَلاَّ أَفْتَقِرَ وَأَسْرِقَ وَأَتَّخِذَ اسْمَ إِلَهِي بَاطِلاً.
\par 10 لاَ تَشْكُ عَبْداً إِلَى سَيِّدِهِ لِئَلاَّ يَلْعَنَكَ فَتَأْثَمَ.
\par 11 جِيلٌ يَلْعَنُ أَبَاهُ وَلاَ يُبَارِكُ أُمَّهُ
\par 12 جِيلٌ طَاهِرٌ فِي عَيْنَيْ نَفْسِهِ وَهُوَ لَمْ يَغْتَسِلْ مِنْ قَذَرِهِ
\par 13 جِيلٌ مَا أَرْفَعَ عَيْنَيْهِ وَحَوَاجِبُهُ مُرْتَفِعَةٌ
\par 14 جِيلٌ أَسْنَانُهُ سُيُوفٌ وَأَضْرَاسُهُ سَكَاكِينُ لأَكْلِ الْمَسَاكِينِ عَنِ الأَرْضِ وَالْفُقَرَاءِ مِنْ بَيْنِ النَّاسِ.
\par 15 لِلْعَلُوقَةِ ابْنَتَانِ: «هَاتِ هَاتِ!» ثَلاَثَةٌ لاَ تَشْبَعُ. أَرْبَعَةٌ لاَ تَقُولُ: «كَفَا»:
\par 16 الْهَاوِيَةُ وَالرَّحِمُ الْعَقِيمُ وَأَرْضٌ لاَ تَشْبَعُ مَاءً وَالنَّارُ لاَ تَقُولُ: «كَفَا».
\par 17 اَلْعَيْنُ الْمُسْتَهْزِئَةُ بِأَبِيهَا وَالْمُحْتَقِرَةُ إِطَاعَةَ أُمِّهَا تُقَوِّرُهَا غِرْبَانُ الْوَادِي وَتَأْكُلُهَا فِرَاخُ النَّسْرِ.
\par 18 ثَلاَثَةٌ عَجِيبَةٌ فَوْقِي وَأَرْبَعَةٌ لاَ أَعْرِفُهَا:
\par 19 طَرِيقَ نَسْرٍ فِي السَّمَاوَاتِ وَطَرِيقَ حَيَّةٍ عَلَى صَخْرٍ وَطَرِيقَ سَفِينَةٍ فِي قَلْبِ الْبَحْرِ وَطَرِيقَ رَجُلٍ بِفَتَاةٍ.
\par 20 كَذَلِكَ طَرِيقُ الْمَرْأَةِ الزَّانِيَةِ. أَكَلَتْ وَمَسَحَتْ فَمَهَا وَقَالَتْ: «مَا عَمِلْتُ إِثْماً!».
\par 21 تَحْتَ ثَلاَثَةٍ تَضْطَرِبُ الأَرْضُ وَأَرْبَعَةٌ لاَ تَسْتَطِيعُ احْتِمَالَهَا:
\par 22 تَحْتَ عَبْدٍ إِذَا مَلَكَ وَأَحْمَقَ إِذَا شَبِعَ خُبْزاً.
\par 23 تَحْتَ شَنِيعَةٍ إِذَا تَزَوَّجَتْ وَأَمَةٍ إِذَا وَرَثَتْ سَيِّدَتَهَا.
\par 24 أَرْبَعَةٌ هِيَ الأَصْغَرُ فِي الأَرْضِ وَلَكِنَّهَا حَكِيمَةٌ جِدّاً:
\par 25 النَّمْلُ طَائِفَةٌ غَيْرُ قَوِيَّةٍ وَلَكِنَّهُ يُعِدُّ طَعَامَهُ فِي الصَّيْفِ.
\par 26 الْوِبَارُ طَائِفَةٌ ضَعِيفَةٌ وَلَكِنَّهَا تَضَعُ بُيُوتَهَا فِي الصَّخْرِ.
\par 27 الْجَرَادُ لَيْسَ لَهُ مَلِكٌ وَلَكِنَّهُ يَخْرُجُ كُلُّهُ فِرَقاً فِرَقاً.
\par 28 الْعَنْكَبُوتُ تُمْسِكُ بِيَدَيْهَا وَهِيَ فِي قُصُورِ الْمُلُوكِ.
\par 29 ثَلاَثَةٌ هِيَ حَسَنَةُ التَّخَطِّي وَأَرْبَعَةٌ مَشْيُهَا مُسْتَحْسَنٌ:
\par 30 اَلأَسَدُ جَبَّارُ الْوُحُوشِ وَلاَ يَرْجِعُ مِنْ قُدَّامِ أَحَدٍ
\par 31 ضَامِرُ الشَّاكِلَةِ وَالتَّيْسُ وَالْمَلِكُ الَّذِي لاَ يُقَاوَمُ.
\par 32 إِنْ حَمِقْتَ بِالتَّرَفُّعِ وَإِنْ تَآمَرْتَ فَضَعْ يَدَكَ عَلَى فَمِكَ.
\par 33 لأَنَّ عَصْرَ اللَّبَنِ يُخْرِجُ جُبْناً وَعَصْرَ الأَنْفِ يُخْرِجُ دَماً وَعَصْرَ الْغَضَبِ يُخْرِجُ خِصَاماً.

\chapter{31}

\par 1 كَلاَمُ لَمُوئِيلَ مَلِكِ مَسَّا. عَلَّمَتْهُ إِيَّاهُ أُمُّهُ:
\par 2 مَاذَا يَا ابْنِي ثُمَّ مَاذَا يَا ابْنَ رَحِمِي ثُمَّ مَاذَا يَا ابْنَ نُذُورِي؟ -
\par 3 لاَ تُعْطِ حَيْلَكَ لِلنِّسَاءِ وَلاَ طُرُقَكَ لِمُهْلِكَاتِ الْمُلُوكِ.
\par 4 لَيْسَ لِلْمُلُوكِ يَا لَمُوئِيلُ لَيْسَ لِلْمُلُوكِ أَنْ يَشْرَبُوا خَمْراً وَلاَ لِلْعُظَمَاءِ الْمُسْكِرُ.
\par 5 لِئَلاَّ يَشْرَبُوا وَيَنْسُوُا الْمَفْرُوضَ وَيُغَيِّرُوا حُجَّةَ كُلِّ بَنِي الْمَذَلَّةِ.
\par 6 أَعْطُوا مُسْكِراً لِهَالِكٍ وَخَمْراً لِمُرِّي النَّفْسِ.
\par 7 يَشْرَبُ وَيَنْسَى فَقْرَهُ وَلاَ يَذْكُرُ تَعَبَهُ بَعْدُ.
\par 8 اِفْتَحْ فَمَكَ لأَجْلِ الأَخْرَسِ فِي دَعْوَى كُلِّ يَتِيمٍ.
\par 9 اِفْتَحْ فَمَكَ. اقْضِ بِالْعَدْلِ وَحَامِ عَنِ الْفَقِيرِ وَالْمِسْكِينِ.
\par 10 اِمْرَأَةٌ فَاضِلَةٌ مَنْ يَجِدُهَا؟ لأَنَّ ثَمَنَهَا يَفُوقُ اللَّآلِئَ.
\par 11 بِهَا يَثِقُ قَلْبُ زَوْجِهَا فَلاَ يَحْتَاجُ إِلَى غَنِيمَةٍ.
\par 12 تَصْنَعُ لَهُ خَيْراً لاَ شَرّاً كُلَّ أَيَّامِ حَيَاتِهَا.
\par 13 تَطْلُبُ صُوفاً وَكَتَّاناً وَتَشْتَغِلُ بِيَدَيْنِ رَاضِيَتَيْنِ.
\par 14 هِيَ كَسُفُنِ التَّاجِرِ. تَجْلِبُ طَعَامَهَا مِنْ بَعِيدٍ.
\par 15 وَتَقُومُ إِذِ اللَّيْلُ بَعْدُ وَتُعْطِي أَكْلاً لأَهْلِ بَيْتِهَا وَفَرِيضَةً لِفَتَيَاتِهَا.
\par 16 تَتَأَمَّلُ حَقْلاً فَتَأْخُذُهُ وَبِثَمَرِ يَدَيْهَا تَغْرِسُ كَرْماً.
\par 17 تُنَطِّقُ حَقَوَيْهَا بِالْقُوَّةِ وَتُشَدِّدُ ذِرَاعَيْهَا.
\par 18 تَشْعُرُ أَنَّ تِجَارَتَهَا جَيِّدَةٌ. سِرَاجُهَا لاَ يَنْطَفِئُ فِي اللَّيْلِ.
\par 19 تَمُدُّ يَدَيْهَا إِلَى الْمِغْزَلِ وَتُمْسِكُ كَفَّاهَا بِالْفَلْكَةِ.
\par 20 تَبْسُطُ كَفَّيْهَا لِلْفَقِيرِ وَتَمُدُّ يَدَيْهَا إِلَى الْمِسْكِينِ.
\par 21 لاَ تَخْشَى عَلَى بَيْتِهَا مِنَ الثَّلْجِ لأَنَّ كُلَّ أَهْلِ بَيْتِهَا لاَبِسُونَ حُلَلاً.
\par 22 تَعْمَلُ لِنَفْسِهَا مُوَشَّيَاتٍ. لِبْسُهَا بُوصٌ وَأُرْجُوانٌ.
\par 23 زَوْجُهَا مَعْرُوفٌ فِي الأَبْوَابِ حِينَ يَجْلِسُ بَيْنَ مَشَايِخِ الأَرْضِ.
\par 24 تَصْنَعُ قُمْصَاناً وَتَبِيعُهَا وَتَعْرِضُ مَنَاطِقَ عَلَى الْكَنْعَانِيِّ.
\par 25 اَلْعِزُّ وَالْبَهَاءُ لِبَاسُهَا وَتَضْحَكُ عَلَى الزَّمَنِ الآتِي.
\par 26 تَفْتَحُ فَمَهَا بِالْحِكْمَةِ وَفِي لِسَانِهَا سُنَّةُ الْمَعْرُوفِ.
\par 27 تُرَاقِبُ طُرُقَ أَهْلِ بَيْتِهَا وَلاَ تَأْكُلُ خُبْزَ الْكَسَلِ.
\par 28 يَقُومُ أَوْلاَدُهَا وَيُطَوِّبُونَهَا. زَوْجُهَا أَيْضاً فَيَمْدَحُهَا.
\par 29 بَنَاتٌ كَثِيرَاتٌ عَمِلْنَ فَضْلاً أَمَّا أَنْتِ فَفُقْتِ عَلَيْهِنَّ جَمِيعاً.
\par 30 اَلْحُسْنُ غِشٌّ وَالْجَمَالُ بَاطِلٌ أَمَّا الْمَرْأَةُ الْمُتَّقِيَةُ الرَّبَّ فَهِيَ تُمْدَحُ.
\par 31 أَعْطُوهَا مِنْ ثَمَرِ يَدَيْهَا وَلْتَمْدَحْهَا أَعْمَالُهَا فِي الأَبْوَابِ.


\end{document}