\begin{document}

\title{يوحنا}


\chapter{1}

\par 1 فِي الْبَدْءِ كَانَ الْكَلِمَةُ وَالْكَلِمَةُ كَانَ عِنْدَ اللَّهِ وَكَانَ الْكَلِمَةُ اللَّهَ.
\par 2 هَذَا كَانَ فِي الْبَدْءِ عِنْدَ اللَّهِ.
\par 3 كُلُّ شَيْءٍ بِهِ كَانَ وَبِغَيْرِهِ لَمْ يَكُنْ شَيْءٌ مِمَّا كَانَ.
\par 4 فِيهِ كَانَتِ الْحَيَاةُ وَالْحَيَاةُ كَانَتْ نُورَ النَّاسِ
\par 5 وَالنُّورُ يُضِيءُ فِي الظُّلْمَةِ وَالظُّلْمَةُ لَمْ تُدْرِكْهُ.
\par 6 كَانَ إِنْسَانٌ مُرْسَلٌ مِنَ اللَّهِ اسْمُهُ يُوحَنَّا.
\par 7 هَذَا جَاءَ لِلشَّهَادَةِ لِيَشْهَدَ لِلنُّورِ لِكَيْ يُؤْمِنَ الْكُلُّ بِوَاسِطَتِهِ.
\par 8 لَمْ يَكُنْ هُوَ النُّورَ بَلْ لِيَشْهَدَ لِلنُّورِ.
\par 9 كَانَ النُّورُ الْحَقِيقِيُّ الَّذِي يُنِيرُ كُلَّ إِنْسَانٍ آتِياً إِلَى الْعَالَمِ.
\par 10 كَانَ فِي الْعَالَمِ وَكُوِّنَ الْعَالَمُ بِهِ وَلَمْ يَعْرِفْهُ الْعَالَمُ.
\par 11 إِلَى خَاصَّتِهِ جَاءَ وَخَاصَّتُهُ لَمْ تَقْبَلْهُ.
\par 12 وَأَمَّا كُلُّ الَّذِينَ قَبِلُوهُ فَأَعْطَاهُمْ سُلْطَاناً أَنْ يَصِيرُوا أَوْلاَدَ اللَّهِ أَيِ الْمُؤْمِنُونَ بِاسْمِهِ.
\par 13 اَلَّذِينَ وُلِدُوا لَيْسَ مِنْ دَمٍ وَلاَ مِنْ مَشِيئَةِ جَسَدٍ وَلاَ مِنْ مَشِيئَةِ رَجُلٍ بَلْ مِنَ اللَّهِ.
\par 14 وَالْكَلِمَةُ صَارَ جَسَداً وَحَلَّ بَيْنَنَا وَرَأَيْنَا مَجْدَهُ مَجْداً كَمَا لِوَحِيدٍ مِنَ الآبِ مَمْلُوءاً نِعْمَةً وَحَقّاً.
\par 15 يُوحَنَّا شَهِدَ لَهُ وَنَادَى: «هَذَا هُوَ الَّذِي قُلْتُ عَنْهُ: إِنَّ الَّذِي يَأْتِي بَعْدِي صَارَ قُدَّامِي لأَنَّهُ كَانَ قَبْلِي».
\par 16 وَمِنْ مِلْئِهِ نَحْنُ جَمِيعاً أَخَذْنَا وَنِعْمَةً فَوْقَ نِعْمَةٍ.
\par 17 لأَنَّ النَّامُوسَ بِمُوسَى أُعْطِيَ أَمَّا النِّعْمَةُ وَالْحَقُّ فَبِيَسُوعَ الْمَسِيحِ صَارَا.
\par 18 اَللَّهُ لَمْ يَرَهُ أَحَدٌ قَطُّ. اَلاِبْنُ الْوَحِيدُ الَّذِي هُوَ فِي حِضْنِ الآبِ هُوَ خَبَّرَ.
\par 19 وَهَذِهِ هِيَ شَهَادَةُ يُوحَنَّا حِينَ أَرْسَلَ الْيَهُودُ مِنْ أُورُشَلِيمَ كَهَنَةً وَلاَوِيِّينَ لِيَسْأَلُوهُ: «مَنْ أَنْتَ؟»
\par 20 فَاعْتَرَفَ وَلَمْ يُنْكِرْ وَأَقَرَّ أَنِّي لَسْتُ أَنَا الْمَسِيحَ.
\par 21 فَسَأَلُوهُ: «إِذاً مَاذَا؟ إِيلِيَّا أَنْتَ؟» فَقَالَ: «لَسْتُ أَنَا». «أَلنَّبِيُّ أَنْتَ؟» فَأَجَابَ: «لاَ».
\par 22 فَقَالُوا لَهُ: «مَنْ أَنْتَ لِنُعْطِيَ جَوَاباً لِلَّذِينَ أَرْسَلُونَا؟ مَاذَا تَقُولُ عَنْ نَفْسِكَ؟»
\par 23 قَالَ: «أَنَا صَوْتُ صَارِخٍ فِي الْبَرِّيَّةِ: قَوِّمُوا طَرِيقَ الرَّبِّ كَمَا قَالَ إِشَعْيَاءُ النَّبِيُّ».
\par 24 وَكَانَ الْمُرْسَلُونَ مِنَ الْفَرِّيسِيِّينَ
\par 25 فَسَأَلُوهُ: «فَمَا بَالُكَ تُعَمِّدُ إِنْ كُنْتَ لَسْتَ الْمَسِيحَ وَلاَ إِيلِيَّا وَلاَ النَّبِيَّ؟»
\par 26 أَجَابَهُمْ يُوحَنَّا: «أَنَا أُعَمِّدُ بِمَاءٍ وَلَكِنْ فِي وَسَطِكُمْ قَائِمٌ الَّذِي لَسْتُمْ تَعْرِفُونَهُ.
\par 27 هُوَ الَّذِي يَأْتِي بَعْدِي الَّذِي صَارَ قُدَّامِي الَّذِي لَسْتُ بِمُسْتَحِقٍّ أَنْ أَحُلَّ سُيُورَ حِذَائِهِ».
\par 28 هَذَا كَانَ فِي بَيْتِ عَبْرَةَ فِي عَبْرِ الأُرْدُنِّ حَيْثُ كَانَ يُوحَنَّا يُعَمِّدُ.
\par 29 وَفِي الْغَدِ نَظَرَ يُوحَنَّا يَسُوعَ مُقْبِلاً إِلَيْهِ فَقَالَ: «هُوَذَا حَمَلُ اللَّهِ الَّذِي يَرْفَعُ خَطِيَّةَ الْعَالَمِ.
\par 30 هَذَا هُوَ الَّذِي قُلْتُ عَنْهُ يَأْتِي بَعْدِي رَجُلٌ صَارَ قُدَّامِي لأَنَّهُ كَانَ قَبْلِي.
\par 31 وَأَنَا لَمْ أَكُنْ أَعْرِفُهُ. لَكِنْ لِيُظْهَرَ لِإِسْرَائِيلَ لِذَلِكَ جِئْتُ أُعَمِّدُ بِالْمَاءِ».
\par 32 وَشَهِدَ يُوحَنَّا: «إِنِّي قَدْ رَأَيْتُ الرُّوحَ نَازِلاً مِثْلَ حَمَامَةٍ مِنَ السَّمَاءِ فَاسْتَقَرَّ عَلَيْهِ.
\par 33 وَأَنَا لَمْ أَكُنْ أَعْرِفُهُ لَكِنَّ الَّذِي أَرْسَلَنِي لِأُعَمِّدَ بِالْمَاءِ ذَاكَ قَالَ لِي: الَّذِي تَرَى الرُّوحَ نَازِلاً وَمُسْتَقِرّاً عَلَيْهِ فَهَذَا هُوَ الَّذِي يُعَمِّدُ بِالرُّوحِ الْقُدُسِ.
\par 34 وَأَنَا قَدْ رَأَيْتُ وَشَهِدْتُ أَنَّ هَذَا هُوَ ابْنُ اللَّهِ».
\par 35 وَفِي الْغَدِ أَيْضاً كَانَ يُوحَنَّا وَاقِفاً هُوَ وَاثْنَانِ مِنْ تلاَمِيذِهِ
\par 36 فَنَظَرَ إِلَى يَسُوعَ مَاشِياً فَقَالَ: «هُوَذَا حَمَلُ اللَّهِ».
\par 37 فَسَمِعَهُ التِّلْمِيذَانِ يَتَكَلَّمُ فَتَبِعَا يَسُوعَ.
\par 38 فَالْتَفَتَ يَسُوعُ وَنَظَرَهُمَا يَتْبَعَانِ فَقَالَ لَهُمَا: «مَاذَا تَطْلُبَانِ؟» فَقَالاَ: «رَبِّي (الَّذِي تَفْسِيرُهُ: يَا مُعَلِّمُ) أَيْنَ تَمْكُثُ؟»
\par 39 فَقَالَ لَهُمَا: «تَعَالَيَا وَانْظُرَا». فَأَتَيَا وَنَظَرَا أَيْنَ كَانَ يَمْكُثُ وَمَكَثَا عِنْدَهُ ذَلِكَ الْيَوْمَ. وَكَانَ نَحْوَ السَّاعَةِ الْعَاشِرَةِ.
\par 40 كَانَ أَنْدَرَاوُسُ أَخُو سِمْعَانَ بُطْرُسَ وَاحِداً مِنَ الاِثْنَيْنِ اللَّذَيْنِ سَمِعَا يُوحَنَّا وَتَبِعَاهُ.
\par 41 هَذَا وَجَدَ أَوَّلاً أَخَاهُ سِمْعَانَ فَقَالَ لَهُ: «قَدْ وَجَدْنَا مَسِيَّا» (الَّذِي تَفْسِيرُهُ: الْمَسِيحُ).
\par 42 فَجَاءَ بِهِ إِلَى يَسُوعَ. فَنَظَرَ إِلَيْهِ يَسُوعُ وَقَالَ: «أَنْتَ سِمْعَانُ بْنُ يُونَا. أَنْتَ تُدْعَى صَفَا» (الَّذِي تَفْسِيرُهُ: بُطْرُسُ).
\par 43 فِي الْغَدِ أَرَادَ يَسُوعُ أَنْ يَخْرُجَ إِلَى الْجَلِيلِ فَوَجَدَ فِيلُبُّسَ فَقَالَ لَهُ: «اتْبَعْنِي».
\par 44 وَكَانَ فِيلُبُّسُ مِنْ بَيْتِ صَيْدَا مِنْ مَدِينَةِ أَنْدَرَاوُسَ وَبُطْرُسَ.
\par 45 فِيلُبُّسُ وَجَدَ نَثَنَائِيلَ وَقَالَ لَهُ: «وَجَدْنَا الَّذِي كَتَبَ عَنْهُ مُوسَى فِي النَّامُوسِ وَالأَنْبِيَاءُ: يَسُوعَ ابْنَ يُوسُفَ الَّذِي مِنَ النَّاصِرَةِ».
\par 46 فَقَالَ لَهُ نَثَنَائِيلُ: «أَمِنَ النَّاصِرَةِ يُمْكِنُ أَنْ يَكُونَ شَيْءٌ صَالِحٌ؟» قَالَ لَهُ فِيلُبُّسُ: «تَعَالَ وَانْظُرْ».
\par 47 وَرَأَى يَسُوعُ نَثَنَائِيلَ مُقْبِلاً إِلَيْهِ فَقَالَ عَنْهُ: «هُوَذَا إِسْرَائِيلِيٌّ حَقّاً لاَ غِشَّ فِيهِ».
\par 48 قَالَ لَهُ نَثَنَائِيلُ: «مِنْ أَيْنَ تَعْرِفُنِي؟» أَجَابَ يَسُوعُ: «قَبْلَ أَنْ دَعَاكَ فِيلُبُّسُ وَأَنْتَ تَحْتَ التِّينَةِ رَأَيْتُكَ».
\par 49 فَقَالَ نَثَنَائِيلُ: «يَا مُعَلِّمُ أَنْتَ ابْنُ اللَّهِ! أَنْتَ مَلِكُ إِسْرَائِيلَ!»
\par 50 أَجَابَ يَسُوعُ: «هَلْ آمَنْتَ لأَنِّي قُلْتُ لَكَ إِنِّي رَأَيْتُكَ تَحْتَ التِّينَةِ؟ سَوْفَ تَرَى أَعْظَمَ مِنْ هَذَا!»
\par 51 وَقَالَ لَهُ: «الْحَقَّ الْحَقَّ أَقُولُ لَكُمْ: مِنَ الآنَ تَرَوْنَ السَّمَاءَ مَفْتُوحَةً وَملاَئِكَةَ اللَّهِ يَصْعَدُونَ وَيَنْزِلُونَ عَلَى ابْنِ الإِنْسَانِ».

\chapter{2}

\par 1 وَفِي الْيَوْمِ الثَّالِثِ كَانَ عُرْسٌ فِي قَانَا الْجَلِيلِ وَكَانَتْ أُمُّ يَسُوعَ هُنَاكَ.
\par 2 وَدُعِيَ أَيْضاً يَسُوعُ وَتلاَمِيذُهُ إِلَى الْعُرْسِ.
\par 3 وَلَمَّا فَرَغَتِ الْخَمْرُ قَالَتْ أُمُّ يَسُوعَ لَهُ: «لَيْسَ لَهُمْ خَمْرٌ».
\par 4 قَالَ لَهَا يَسُوعُ: «مَا لِي وَلَكِ يَا امْرَأَةُ! لَمْ تَأْتِ سَاعَتِي بَعْدُ».
\par 5 قَالَتْ أُمُّهُ لِلْخُدَّامِ: «مَهْمَا قَالَ لَكُمْ فَافْعَلُوهُ».
\par 6 وَكَانَتْ سِتَّةُ أَجْرَانٍ مِنْ حِجَارَةٍ مَوْضُوعَةً هُنَاكَ حَسَبَ تَطْهِيرِ الْيَهُودِ يَسَعُ كُلُّ وَاحِدٍ مِطْرَيْنِ أَوْ ثلاَثَةً.
\par 7 قَالَ لَهُمْ يَسُوعُ: «امْلَأُوا الأَجْرَانَ مَاءً». فَمَلَأُوهَا إِلَى فَوْقُ.
\par 8 ثُمَّ قَالَ لَهُمُ: «اسْتَقُوا الآنَ وَقَدِّمُوا إِلَى رَئِيسِ الْمُتَّكَإِ». فَقَدَّمُوا.
\par 9 فَلَمَّا ذَاقَ رَئِيسُ الْمُتَّكَإِ الْمَاءَ الْمُتَحَوِّلَ خَمْراً وَلَمْ يَكُنْ يَعْلَمُ مِنْ أَيْنَ هِيَ - لَكِنَّ الْخُدَّامَ الَّذِينَ كَانُوا قَدِ اسْتَقَوُا الْمَاءَ عَلِمُوا - دَعَا رَئِيسُ الْمُتَّكَإِ الْعَرِيسَ
\par 10 وَقَالَ لَهُ: «كُلُّ إِنْسَانٍ إِنَّمَا يَضَعُ الْخَمْرَ الْجَيِّدَةَ أَوَّلاً وَمَتَى سَكِرُوا فَحِينَئِذٍ الدُّونَ. أَمَّا أَنْتَ فَقَدْ أَبْقَيْتَ الْخَمْرَ الْجَيِّدَةَ إِلَى الآنَ».
\par 11 هَذِهِ بِدَايَةُ الآيَاتِ فَعَلَهَا يَسُوعُ فِي قَانَا الْجَلِيلِ وَأَظْهَرَ مَجْدَهُ فَآمَنَ بِهِ تلاَمِيذُهُ.
\par 12 وَبَعْدَ هَذَا انْحَدَرَ إِلَى كَفْرِنَاحُومَ هُوَ وَأُمُّهُ وَإِخْوَتُهُ وَتلاَمِيذُهُ وَأَقَامُوا هُنَاكَ أَيَّاماً لَيْسَتْ كَثِيرَةً
\par 13 وَكَانَ فِصْحُ الْيَهُودِ قَرِيباً فَصَعِدَ يَسُوعُ إِلَى أُورُشَلِيمَ
\par 14 وَوَجَدَ فِي الْهَيْكَلِ الَّذِينَ كَانُوا يَبِيعُونَ بَقَراً وَغَنَماً وَحَمَاماً وَالصَّيَارِفَ جُلُوساً.
\par 15 فَصَنَعَ سَوْطاً مِنْ حِبَالٍ وَطَرَدَ الْجَمِيعَ مِنَ الْهَيْكَلِ اَلْغَنَمَ وَالْبَقَرَ وَكَبَّ دَرَاهِمَ الصَّيَارِفِ وَقَلَّبَ مَوَائِدَهُمْ.
\par 16 وَقَالَ لِبَاعَةِ الْحَمَامِ: «ارْفَعُوا هَذِهِ مِنْ هَهُنَا. لاَ تَجْعَلُوا بَيْتَ أَبِي بَيْتَ تِجَارَةٍ».
\par 17 فَتَذَكَّرَ تلاَمِيذُهُ أَنَّهُ مَكْتُوبٌ: «غَيْرَةُ بَيْتِكَ أَكَلَتْنِي».
\par 18 فَسَأَلَهُ الْيَهُودُ: «أَيَّةَ آيَةٍ تُرِينَا حَتَّى تَفْعَلَ هَذَا؟»
\par 19 أَجَابَ يَسُوعُ: «انْقُضُوا هَذَا الْهَيْكَلَ وَفِي ثلاَثَةِ أَيَّامٍ أُقِيمُهُ».
\par 20 فَقَالَ الْيَهُودُ: «فِي سِتٍّ وَأَرْبَعِينَ سَنَةً بُنِيَ هَذَا الْهَيْكَلُ أَفَأَنْتَ فِي ثلاَثَةِ أَيَّامٍ تُقِيمُهُ؟»
\par 21 وَأَمَّا هُوَ فَكَانَ يَقُولُ عَنْ هَيْكَلِ جَسَدِهِ.
\par 22 فَلَمَّا قَامَ مِنَ الأَمْوَاتِ تَذَكَّرَ تلاَمِيذُهُ أَنَّهُ قَالَ هَذَا فَآمَنُوا بِالْكِتَابِ وَالْكلاَمِ الَّذِي قَالَهُ يَسُوعُ.
\par 23 وَلَمَّا كَانَ فِي أُورُشَلِيمَ فِي عِيدِ الْفِصْحِ آمَنَ كَثِيرُونَ بِاسْمِهِ إِذْ رَأَوُا الآيَاتِ الَّتِي صَنَعَ.
\par 24 لَكِنَّ يَسُوعَ لَمْ يَأْتَمِنْهُمْ عَلَى نَفْسِهِ لأَنَّهُ كَانَ يَعْرِفُ الْجَمِيعَ.
\par 25 وَلأَنَّهُ لَمْ يَكُنْ مُحْتَاجاً أَنْ يَشْهَدَ أَحَدٌ عَنِ الإِنْسَانِ لأَنَّهُ عَلِمَ مَا كَانَ فِي الإِنْسَانِ.

\chapter{3}

\par 1 كَانَ إِنْسَانٌ مِنَ الْفَرِّيسِيِّينَ اسْمُهُ نِيقُودِيمُوسُ رَئِيسٌ لِلْيَهُودِ.
\par 2 هَذَا جَاءَ إِلَى يَسُوعَ لَيْلاً وَقَالَ لَهُ: «يَا مُعَلِّمُ نَعْلَمُ أَنَّكَ قَدْ أَتَيْتَ مِنَ اللَّهِ مُعَلِّماً لأَنْ لَيْسَ أَحَدٌ يَقْدِرُ أَنْ يَعْمَلَ هَذِهِ الآيَاتِ الَّتِي أَنْتَ تَعْمَلُ إِنْ لَمْ يَكُنِ اللَّهُ مَعَهُ».
\par 3 فَقَالَ يَسُوعُ: «الْحَقَّ الْحَقَّ أَقُولُ لَكَ: إِنْ كَانَ أَحَدٌ لاَ يُولَدُ مِنْ فَوْقُ لاَ يَقْدِرُ أَنْ يَرَى مَلَكُوتَ اللَّهِ».
\par 4 قَالَ لَهُ نِيقُودِيمُوسُ: «كَيْفَ يُمْكِنُ الإِنْسَانَ أَنْ يُولَدَ وَهُوَ شَيْخٌ؟ أَلَعَلَّهُ يَقْدِرُ أَنْ يَدْخُلَ بَطْنَ أُمِّهِ ثَانِيَةً وَيُولَدَ؟»
\par 5 أَجَابَ يَسُوعُ: «الْحَقَّ الْحَقَّ أَقُولُ لَكَ: إِنْ كَانَ أَحَدٌ لاَ يُولَدُ مِنَ الْمَاءِ وَالرُّوحِ لاَ يَقْدِرُ أَنْ يَدْخُلَ مَلَكُوتَ اللَّهِ.
\par 6 اَلْمَوْلُودُ مِنَ الْجَسَدِ جَسَدٌ هُوَ وَالْمَوْلُودُ مِنَ الرُّوحِ هُوَ رُوحٌ.
\par 7 لاَ تَتَعَجَّبْ أَنِّي قُلْتُ لَكَ: يَنْبَغِي أَنْ تُولَدُوا مِنْ فَوْقُ.
\par 8 اَلرِّيحُ تَهُبُّ حَيْثُ تَشَاءُ وَتَسْمَعُ صَوْتَهَا لَكِنَّكَ لاَ تَعْلَمُ مِنْ أَيْنَ تَأْتِي وَلاَ إِلَى أَيْنَ تَذْهَبُ. هَكَذَا كُلُّ مَنْ وُلِدَ مِنَ الرُّوحِ».
\par 9 فَسَأَلَهُ نِيقُودِيمُوسُ: «كَيْفَ يُمْكِنُ أَنْ يَكُونَ هَذَا؟»
\par 10 أَجَابَ يَسُوعُ: «أَنْتَ مُعَلِّمُ إِسْرَائِيلَ وَلَسْتَ تَعْلَمُ هَذَا!
\par 11 اَلْحَقَّ الْحَقَّ أَقُولُ لَكَ: إِنَّنَا إِنَّمَا نَتَكَلَّمُ بِمَا نَعْلَمُ وَنَشْهَدُ بِمَا رَأَيْنَا وَلَسْتُمْ تَقْبَلُونَ شَهَادَتَنَا.
\par 12 إِنْ كُنْتُ قُلْتُ لَكُمُ الأَرْضِيَّاتِ وَلَسْتُمْ تُؤْمِنُونَ فَكَيْفَ تُؤْمِنُونَ إِنْ قُلْتُ لَكُمُ السَّمَاوِيَّاتِ؟
\par 13 وَلَيْسَ أَحَدٌ صَعِدَ إِلَى السَّمَاءِ إِلاَّ الَّذِي نَزَلَ مِنَ السَّمَاءِ ابْنُ الإِنْسَانِ الَّذِي هُوَ فِي السَّمَاءِ.
\par 14 «وَكَمَا رَفَعَ مُوسَى الْحَيَّةَ فِي الْبَرِّيَّةِ هَكَذَا يَنْبَغِي أَنْ يُرْفَعَ ابْنُ الإِنْسَانِ
\par 15 لِكَيْ لاَ يَهْلِكَ كُلُّ مَنْ يُؤْمِنُ بِهِ بَلْ تَكُونُ لَهُ الْحَيَاةُ الأَبَدِيَّةُ.
\par 16 لأَنَّهُ هَكَذَا أَحَبَّ اللَّهُ الْعَالَمَ حَتَّى بَذَلَ ابْنَهُ الْوَحِيدَ لِكَيْ لاَ يَهْلِكَ كُلُّ مَنْ يُؤْمِنُ بِهِ بَلْ تَكُونُ لَهُ الْحَيَاةُ الأَبَدِيَّةُ.
\par 17 لأَنَّهُ لَمْ يُرْسِلِ اللَّهُ ابْنَهُ إِلَى الْعَالَمِ لِيَدِينَ الْعَالَمَ بَلْ لِيَخْلُصَ بِهِ الْعَالَمُ.
\par 18 اَلَّذِي يُؤْمِنُ بِهِ لاَ يُدَانُ وَالَّذِي لاَ يُؤْمِنُ قَدْ دِينَ لأَنَّهُ لَمْ يُؤْمِنْ بِاسْمِ ابْنِ اللَّهِ الْوَحِيدِ.
\par 19 وَهَذِهِ هِيَ الدَّيْنُونَةُ: إِنَّ النُّورَ قَدْ جَاءَ إِلَى الْعَالَمِ وَأَحَبَّ النَّاسُ الظُّلْمَةَ أَكْثَرَ مِنَ النُّورِ لأَنَّ أَعْمَالَهُمْ كَانَتْ شِرِّيرَةً.
\par 20 لأَنَّ كُلَّ مَنْ يَعْمَلُ السَّيِّآتِ يُبْغِضُ النُّورَ وَلاَ يَأْتِي إِلَى النُّورِ لِئَلَّا تُوَبَّخَ أَعْمَالُهُ.
\par 21 وَأَمَّا مَنْ يَفْعَلُ الْحَقَّ فَيُقْبِلُ إِلَى النُّورِ لِكَيْ تَظْهَرَ أَعْمَالُهُ أَنَّهَا بِاللَّهِ مَعْمُولَةٌ».
\par 22 وَبَعْدَ هَذَا جَاءَ يَسُوعُ وَتلاَمِيذُهُ إِلَى أَرْضِ الْيَهُودِيَّةِ وَمَكَثَ مَعَهُمْ هُنَاكَ وَكَانَ يُعَمِّدُ.
\par 23 وَكَانَ يُوحَنَّا أَيْضاً يُعَمِّدُ فِي عَيْنِ نُونٍ بِقُرْبِ سَالِيمَ لأَنَّهُ كَانَ هُنَاكَ مِيَاهٌ كَثِيرَةٌ وَكَانُوا يَأْتُونَ وَيَعْتَمِدُونَ -
\par 24 لأَنَّهُ لَمْ يَكُنْ يُوحَنَّا قَدْ أُلْقِيَ بَعْدُ فِي السِّجْنِ.
\par 25 وَحَدَثَتْ مُبَاحَثَةٌ مِنْ تلاَمِيذِ يُوحَنَّا مَعَ يَهُودٍ مِنْ جِهَةِ التَّطْهِيرِ.
\par 26 فَجَاءُوا إِلَى يُوحَنَّا وَقَالُوا لَهُ: «يَا مُعَلِّمُ هُوَذَا الَّذِي كَانَ مَعَكَ فِي عَبْرِ الأُرْدُنِّ الَّذِي أَنْتَ قَدْ شَهِدْتَ لَهُ هُوَ يُعَمِّدُ وَالْجَمِيعُ يَأْتُونَ إِلَيْهِ»
\par 27 فَقَالَ يُوحَنَّا: «لاَ يَقْدِرُ إِنْسَانٌ أَنْ يَأْخُذَ شَيْئاً إِنْ لَمْ يَكُنْ قَدْ أُعْطِيَ مِنَ السَّمَاءِ.
\par 28 أَنْتُمْ أَنْفُسُكُمْ تَشْهَدُونَ لِي أَنِّي قُلْتُ: لَسْتُ أَنَا الْمَسِيحَ بَلْ إِنِّي مُرْسَلٌ أَمَامَهُ.
\par 29 مَنْ لَهُ الْعَرُوسُ فَهُوَ الْعَرِيسُ وَأَمَّا صَدِيقُ الْعَرِيسِ الَّذِي يَقِفُ وَيَسْمَعُهُ فَيَفْرَحُ فَرَحاً مِنْ أَجْلِ صَوْتِ الْعَرِيسِ. إِذاً فَرَحِي هَذَا قَدْ كَمَلَ.
\par 30 يَنْبَغِي أَنَّ ذَلِكَ يَزِيدُ وَأَنِّي أَنَا أَنْقُصُ.
\par 31 اَلَّذِي يَأْتِي مِنْ فَوْقُ هُوَ فَوْقَ الْجَمِيعِ وَالَّذِي مِنَ الأَرْضِ هُوَ أَرْضِيٌّ وَمِنَ الأَرْضِ يَتَكَلَّمُ. اَلَّذِي يَأْتِي مِنَ السَّمَاءِ هُوَ فَوْقَ الْجَمِيعِ
\par 32 وَمَا رَآهُ وَسَمِعَهُ بِهِ يَشْهَدُ وَشَهَادَتُهُ لَيْسَ أَحَدٌ يَقْبَلُهَا.
\par 33 وَمَنْ قَبِلَ شَهَادَتَهُ فَقَدْ خَتَمَ أَنَّ اللَّهَ صَادِقٌ
\par 34 لأَنَّ الَّذِي أَرْسَلَهُ اللَّهُ يَتَكَلَّمُ بِكلاَمِ اللَّهِ. لأَنَّهُ لَيْسَ بِكَيْلٍ يُعْطِي اللَّهُ الرُّوحَ.
\par 35 اَلآبُ يُحِبُّ الاِبْنَ وَقَدْ دَفَعَ كُلَّ شَيْءٍ فِي يَدِهِ.
\par 36 اَلَّذِي يُؤْمِنُ بِالاِبْنِ لَهُ حَيَاةٌ أَبَدِيَّةٌ وَالَّذِي لاَ يُؤْمِنُ بِالاِبْنِ لَنْ يَرَى حَيَاةً بَلْ يَمْكُثُ عَلَيْهِ غَضَبُ اللَّهِ».

\chapter{4}

\par 1 فَلَمَّا عَلِمَ الرَّبُّ أَنَّ الْفَرِّيسِيِّينَ سَمِعُوا أَنَّ يَسُوعَ يُصَيِّرُ وَيُعَمِّدُ تلاَمِيذَ أَكْثَرَ مِنْ يُوحَنَّا -
\par 2 مَعَ أَنَّ يَسُوعَ نَفْسَهُ لَمْ يَكُنْ يُعَمِّدُ بَلْ تلاَمِيذُهُ -
\par 3 تَرَكَ الْيَهُودِيَّةَ وَمَضَى أَيْضاً إِلَى الْجَلِيلِ.
\par 4 وَكَانَ لاَ بُدَّ لَهُ أَنْ يَجْتَازَ السَّامِرَةَ.
\par 5 فَأَتَى إِلَى مَدِينَةٍ مِنَ السَّامِرَةِ يُقَالُ لَهَا سُوخَارُ بِقُرْبِ الضَّيْعَةِ الَّتِي وَهَبَهَا يَعْقُوبُ لِيُوسُفَ ابْنِهِ.
\par 6 وَكَانَتْ هُنَاكَ بِئْرُ يَعْقُوبَ. فَإِذْ كَانَ يَسُوعُ قَدْ تَعِبَ مِنَ السَّفَرِ جَلَسَ هَكَذَا عَلَى الْبِئْرِ وَكَانَ نَحْوَ السَّاعَةِ السَّادِسَةِ.
\par 7 فَجَاءَتِ امْرَأَةٌ مِنَ السَّامِرَةِ لِتَسْتَقِيَ مَاءً فَقَالَ لَهَا يَسُوعُ: «أَعْطِينِي لأَشْرَبَ» -
\par 8 لأَنَّ تلاَمِيذَهُ كَانُوا قَدْ مَضَوْا إِلَى الْمَدِينَةِ لِيَبْتَاعُوا طَعَاماً.
\par 9 فَقَالَتْ لَهُ الْمَرْأَةُ السَّامِرِيَّةُ: «كَيْفَ تَطْلُبُ مِنِّي لِتَشْرَبَ وَأَنْتَ يَهُودِيٌّ وَأَنَا امْرَأَةٌ سَامِرِيَّةٌ؟» لأَنَّ الْيَهُودَ لاَ يُعَامِلُونَ السَّامِرِيِّينَ.
\par 10 أَجَابَ يَسُوعُ: «لَوْ كُنْتِ تَعْلَمِينَ عَطِيَّةَ اللَّهِ وَمَنْ هُوَ الَّذِي يَقُولُ لَكِ أَعْطِينِي لأَشْرَبَ لَطَلَبْتِ أَنْتِ مِنْهُ فَأَعْطَاكِ مَاءً حَيّاً».
\par 11 قَالَتْ لَهُ الْمَرْأَةُ: «يَا سَيِّدُ لاَ دَلْوَ لَكَ وَالْبِئْرُ عَمِيقَةٌ. فَمِنْ أَيْنَ لَكَ الْمَاءُ الْحَيُّ؟
\par 12 أَلَعَلَّكَ أَعْظَمُ مِنْ أَبِينَا يَعْقُوبَ الَّذِي أَعْطَانَا الْبِئْرَ وَشَرِبَ مِنْهَا هُوَ وَبَنُوهُ وَمَوَاشِيهِ؟»
\par 13 أَجَابَ يَسُوعُ: «كُلُّ مَنْ يَشْرَبُ مِنْ هَذَا الْمَاءِ يَعْطَشُ أَيْضاً.
\par 14 وَلَكِنْ مَنْ يَشْرَبُ مِنَ الْمَاءِ الَّذِي أُعْطِيهِ أَنَا فَلَنْ يَعْطَشَ إِلَى الأَبَدِ بَلِ الْمَاءُ الَّذِي أُعْطِيهِ يَصِيرُ فِيهِ يَنْبُوعَ مَاءٍ يَنْبَعُ إِلَى حَيَاةٍ أَبَدِيَّةٍ».
\par 15 قَالَتْ لَهُ الْمَرْأَةُ: «يَا سَيِّدُ أَعْطِنِي هَذَا الْمَاءَ لِكَيْ لاَ أَعْطَشَ وَلاَ آتِيَ إِلَى هُنَا لأَسْتَقِيَ».
\par 16 قَالَ لَهَا يَسُوعُ: «اذْهَبِي وَادْعِي زَوْجَكِ وَتَعَالَيْ إِلَى هَهُنَا»
\par 17 أَجَابَتِ الْمَرْأَةُ: «لَيْسَ لِي زَوْجٌ». قَالَ لَهَا يَسُوعُ: «حَسَناً قُلْتِ لَيْسَ لِي زَوْجٌ
\par 18 لأَنَّهُ كَانَ لَكِ خَمْسَةُ أَزْوَاجٍ وَالَّذِي لَكِ الآنَ لَيْسَ هُوَ زَوْجَكِ. هَذَا قُلْتِ بِالصِّدْقِ».
\par 19 قَالَتْ لَهُ الْمَرْأَةُ: «يَا سَيِّدُ أَرَى أَنَّكَ نَبِيٌّ!
\par 20 آبَاؤُنَا سَجَدُوا فِي هَذَا الْجَبَلِ وَأَنْتُمْ تَقُولُونَ إِنَّ فِي أُورُشَلِيمَ الْمَوْضِعَ الَّذِي يَنْبَغِي أَنْ يُسْجَدَ فِيهِ».
\par 21 قَالَ لَهَا يَسُوعُ: «يَا امْرَأَةُ صَدِّقِينِي أَنَّهُ تَأْتِي سَاعَةٌ لاَ فِي هَذَا الْجَبَلِ وَلاَ فِي أُورُشَلِيمَ تَسْجُدُونَ لِلآبِ.
\par 22 أَنْتُمْ تَسْجُدُونَ لِمَا لَسْتُمْ تَعْلَمُونَ أَمَّا نَحْنُ فَنَسْجُدُ لِمَا نَعْلَمُ - لأَنَّ الْخلاَصَ هُوَ مِنَ الْيَهُودِ.
\par 23 وَلَكِنْ تَأْتِي سَاعَةٌ وَهِيَ الآنَ حِينَ السَّاجِدُونَ الْحَقِيقِيُّونَ يَسْجُدُونَ لِلآبِ بِالرُّوحِ وَالْحَقِّ لأَنَّ الآبَ طَالِبٌ مِثْلَ هَؤُلاَءِ السَّاجِدِينَ لَهُ.
\par 24 اَللَّهُ رُوحٌ. وَالَّذِينَ يَسْجُدُونَ لَهُ فَبِالرُّوحِ وَالْحَقِّ يَنْبَغِي أَنْ يَسْجُدُوا».
\par 25 قَالَتْ لَهُ الْمَرْأَةُ: «أَنَا أَعْلَمُ أَنَّ مَسِيَّا الَّذِي يُقَالُ لَهُ الْمَسِيحُ يَأْتِي. فَمَتَى جَاءَ ذَاكَ يُخْبِرُنَا بِكُلِّ شَيْءٍ».
\par 26 قَالَ لَهَا يَسُوعُ: «أَنَا الَّذِي أُكَلِّمُكِ هُوَ».
\par 27 وَعِنْدَ ذَلِكَ جَاءَ تلاَمِيذُهُ وَكَانُوا يَتَعَجَّبُونَ أَنَّهُ يَتَكَلَّمُ مَعَ امْرَأَةٍ. وَلَكِنْ لَمْ يَقُلْ أَحَدٌ: مَاذَا تَطْلُبُ أَوْ لِمَاذَا تَتَكَلَّمُ مَعَهَا.
\par 28 فَتَرَكَتِ الْمَرْأَةُ جَرَّتَهَا وَمَضَتْ إِلَى الْمَدِينَةِ وَقَالَتْ لِلنَّاسِ:
\par 29 «هَلُمُّوا انْظُرُوا إِنْسَاناً قَالَ لِي كُلَّ مَا فَعَلْتُ. أَلَعَلَّ هَذَا هُوَ الْمَسِيحُ؟».
\par 30 فَخَرَجُوا مِنَ الْمَدِينَةِ وَأَتَوْا إِلَيْهِ.
\par 31 وَفِي أَثْنَاءِ ذَلِكَ سَأَلَهُ تلاَمِيذُهُ: «يَا مُعَلِّمُ كُلْ»
\par 32 فَقَالَ لَهُمْ: «أَنَا لِي طَعَامٌ لِآكُلَ لَسْتُمْ تَعْرِفُونَهُ أَنْتُمْ».
\par 33 فَقَالَ التّلاَمِيذُ بَعْضُهُمْ لِبَعْضٍ: «أَلَعَلَّ أَحَداً أَتَاهُ بِشَيْءٍ لِيَأْكُلَ؟»
\par 34 قَالَ لَهُمْ يَسُوعُ: «طَعَامِي أَنْ أَعْمَلَ مَشِيئَةَ الَّذِي أَرْسَلَنِي وَأُتَمِّمَ عَمَلَهُ.
\par 35 أَمَا تَقُولُونَ إِنَّهُ يَكُونُ أَرْبَعَةُ أَشْهُرٍ ثُمَّ يَأْتِي الْحَصَادُ؟ هَا أَنَا أَقُولُ لَكُمُ: ارْفَعُوا أَعْيُنَكُمْ وَانْظُرُوا الْحُقُولَ إِنَّهَا قَدِ ابْيَضَّتْ لِلْحَصَادِ.
\par 36 وَالْحَاصِدُ يَأْخُذُ أُجْرَةً وَيَجْمَعُ ثَمَراً لِلْحَيَاةِ الأَبَدِيَّةِ لِكَيْ يَفْرَحَ الزَّارِعُ وَالْحَاصِدُ مَعاً.
\par 37 لأَنَّهُ فِي هَذَا يَصْدُقُ الْقَوْلُ: إِنَّ وَاحِداً يَزْرَعُ وَآخَرَ يَحْصُدُ.
\par 38 أَنَا أَرْسَلْتُكُمْ لِتَحْصُدُوا مَا لَمْ تَتْعَبُوا فِيهِ. آخَرُونَ تَعِبُوا وَأَنْتُمْ قَدْ دَخَلْتُمْ عَلَى تَعَبِهِمْ».
\par 39 فَآمَنَ بِهِ مِنْ تِلْكَ الْمَدِينَةِ كَثِيرُونَ مِنَ السَّامِرِيِّينَ بِسَبَبِ كلاَمِ الْمَرْأَةِ الَّتِي كَانَتْ تَشْهَدُ أَنَّهُ: «قَالَ لِي كُلَّ مَا فَعَلْتُ».
\par 40 فَلَمَّا جَاءَ إِلَيْهِ السَّامِرِيُّونَ سَأَلُوهُ أَنْ يَمْكُثَ عِنْدَهُمْ فَمَكَثَ هُنَاكَ يَوْمَيْنِ.
\par 41 فَآمَنَ بِهِ أَكْثَرُ جِدّاً بِسَبَبِ كلاَمِهِ.
\par 42 وَقَالُوا لِلْمَرْأَةِ: «إِنَّنَا لَسْنَا بَعْدُ بِسَبَبِ كلاَمِكِ نُؤْمِنُ لأَنَّنَا نَحْنُ قَدْ سَمِعْنَا وَنَعْلَمُ أَنَّ هَذَا هُوَ بِالْحَقِيقَةِ الْمَسِيحُ مُخَلِّصُ الْعَالَمِ».
\par 43 وَبَعْدَ الْيَوْمَيْنِ خَرَجَ مِنْ هُنَاكَ وَمَضَى إِلَى الْجَلِيلِ
\par 44 لأَنَّ يَسُوعَ نَفْسَهُ شَهِدَ أَنْ: «لَيْسَ لِنَبِيٍّ كَرَامَةٌ فِي وَطَنِهِ».
\par 45 فَلَمَّا جَاءَ إِلَى الْجَلِيلِ قَبِلَهُ الْجَلِيلِيُّونَ إِذْ كَانُوا قَدْ عَايَنُوا كُلَّ مَا فَعَلَ فِي أُورُشَلِيمَ فِي الْعِيدِ لأَنَّهُمْ هُمْ أَيْضاً جَاءُوا إِلَى الْعِيدِ.
\par 46 فَجَاءَ يَسُوعُ أَيْضاً إِلَى قَانَا الْجَلِيلِ حَيْثُ صَنَعَ الْمَاءَ خَمْراً. وَكَانَ خَادِمٌ لِلْمَلِكِ ابْنُهُ مَرِيضٌ فِي كَفْرِنَاحُومَ.
\par 47 هَذَا إِذْ سَمِعَ أَنَّ يَسُوعَ قَدْ جَاءَ مِنَ الْيَهُودِيَّةِ إِلَى الْجَلِيلِ انْطَلَقَ إِلَيْهِ وَسَأَلَهُ أَنْ يَنْزِلَ وَيَشْفِيَ ابْنَهُ لأَنَّهُ كَانَ مُشْرِفاً عَلَى الْمَوْتِ.
\par 48 فَقَالَ لَهُ يَسُوعُ: «لاَ تُؤْمِنُونَ إِنْ لَمْ تَرَوْا آيَاتٍ وَعَجَائِبَ!»
\par 49 قَالَ لَهُ خَادِمُ الْمَلِكِ: «يَا سَيِّدُ انْزِلْ قَبْلَ أَنْ يَمُوتَ ابْنِي».
\par 50 قَالَ لَهُ يَسُوعُ: «اذْهَبْ. اِبْنُكَ حَيٌّ». فَآمَنَ الرَّجُلُ بِالْكَلِمَةِ الَّتِي قَالَهَا لَهُ يَسُوعُ وَذَهَبَ.
\par 51 وَفِيمَا هُوَ نَازِلٌ اسْتَقْبَلَهُ عَبِيدُهُ وَأَخْبَرُوهُ قَائِلِينَ: «إِنَّ ابْنَكَ حَيٌّ».
\par 52 فَاسْتَخْبَرَهُمْ عَنِ السَّاعَةِ الَّتِي فِيهَا أَخَذَ يَتَعَافَى فَقَالُوا لَهُ: «أَمْسٍ فِي السَّاعَةِ السَّابِعَةِ تَرَكَتْهُ الْحُمَّى».
\par 53 فَفَهِمَ الأَبُ أَنَّهُ فِي تِلْكَ السَّاعَةِ الَّتِي قَالَ لَهُ فِيهَا يَسُوعُ إِنَّ ابْنَكَ حَيٌّ. فَآمَنَ هُوَ وَبَيْتُهُ كُلُّهُ.
\par 54 هَذِهِ أَيْضاً آيَةٌ ثَانِيَةٌ صَنَعَهَا يَسُوعُ لَمَّا جَاءَ مِنَ الْيَهُودِيَّةِ إِلَى الْجَلِيلِ.

\chapter{5}

\par 1 وَبَعْدَ هَذَا كَانَ عِيدٌ لِلْيَهُودِ فَصَعِدَ يَسُوعُ إِلَى أُورُشَلِيمَ.
\par 2 وَفِي أُورُشَلِيمَ عِنْدَ بَابِ الضَّأْنِ بِرْكَةٌ يُقَالُ لَهَا بِالْعِبْرَانِيَّةِ «بَيْتُ حِسْدَا» لَهَا خَمْسَةُ أَرْوِقَةٍ.
\par 3 فِي هَذِهِ كَانَ مُضْطَجِعاً جُمْهُورٌ كَثِيرٌ مِنْ مَرْضَى وَعُمْيٍ وَعُرْجٍ وَعُسْمٍ يَتَوَقَّعُونَ تَحْرِيكَ الْمَاءِ.
\par 4 لأَنَّ ملاَكاً كَانَ يَنْزِلُ أَحْيَاناً فِي الْبِرْكَةِ وَيُحَرِّكُ الْمَاءَ. فَمَنْ نَزَلَ أَوَّلاً بَعْدَ تَحْرِيكِ الْمَاءِ كَانَ يَبْرَأُ مِنْ أَيِّ مَرَضٍ اعْتَرَاهُ.
\par 5 وَكَانَ هُنَاكَ إِنْسَانٌ بِهِ مَرَضٌ مُنْذُ ثَمَانٍ وَثلاَثِينَ سَنَةً.
\par 6 هَذَا رَآهُ يَسُوعُ مُضْطَجِعاً وَعَلِمَ أَنَّ لَهُ زَمَاناً كَثِيراً فَقَالَ لَهُ: «أَتُرِيدُ أَنْ تَبْرَأَ؟»
\par 7 أَجَابَهُ الْمَرِيضُ: «يَا سَيِّدُ لَيْسَ لِي إِنْسَانٌ يُلْقِينِي فِي الْبِرْكَةِ مَتَى تَحَرَّكَ الْمَاءُ. بَلْ بَيْنَمَا أَنَا آتٍ يَنْزِلُ قُدَّامِي آخَرُ».
\par 8 قَالَ لَهُ يَسُوعُ: «قُمِ. احْمِلْ سَرِيرَكَ وَامْشِ».
\par 9 فَحَالاً بَرِئَ الإِنْسَانُ وَحَمَلَ سَرِيرَهُ وَمَشَى. وَكَانَ فِي ذَلِكَ الْيَوْمِ سَبْتٌ.
\par 10 فَقَالَ الْيَهُودُ لِلَّذِي شُفِيَ: «إِنَّهُ سَبْتٌ! لاَ يَحِلُّ لَكَ أَنْ تَحْمِلَ سَرِيرَكَ».
\par 11 أَجَابَهُمْ: «إِنَّ الَّذِي أَبْرَأَنِي هُوَ قَالَ لِي احْمِلْ سَرِيرَكَ وَامْشِ».
\par 12 فَسَأَلُوهُ: «مَنْ هُوَ الإِنْسَانُ الَّذِي قَالَ لَكَ احْمِلْ سَرِيرَكَ وَامْشِ؟».
\par 13 أَمَّا الَّذِي شُفِيَ فَلَمْ يَكُنْ يَعْلَمُ مَنْ هُوَ لأَنَّ يَسُوعَ اعْتَزَلَ إِذْ كَانَ فِي الْمَوْضِعِ جَمْعٌ.
\par 14 بَعْدَ ذَلِكَ وَجَدَهُ يَسُوعُ فِي الْهَيْكَلِ وَقَالَ لَهُ: «هَا أَنْتَ قَدْ بَرِئْتَ فلاَ تُخْطِئْ أَيْضاً لِئَلَّا يَكُونَ لَكَ أَشَرُّ».
\par 15 فَمَضَى الإِنْسَانُ وَأَخْبَرَ الْيَهُودَ أَنَّ يَسُوعَ هُوَ الَّذِي أَبْرَأَهُ.
\par 16 وَلِهَذَا كَانَ الْيَهُودُ يَطْرُدُونَ يَسُوعَ وَيَطْلُبُونَ أَنْ يَقْتُلُوهُ لأَنَّهُ عَمِلَ هَذَا فِي سَبْتٍ.
\par 17 فَأَجَابَهُمْ يَسُوعُ: «أَبِي يَعْمَلُ حَتَّى الآنَ وَأَنَا أَعْمَلُ».
\par 18 فَمِنْ أَجْلِ هَذَا كَانَ الْيَهُودُ يَطْلُبُونَ أَكْثَرَ أَنْ يَقْتُلُوهُ لأَنَّهُ لَمْ يَنْقُضِ السَّبْتَ فَقَطْ بَلْ قَالَ أَيْضاً إِنَّ اللَّهَ أَبُوهُ مُعَادِلاً نَفْسَهُ بِاللَّهِ.
\par 19 فَقَالَ يَسُوعُ لَهُمُ: «الْحَقَّ الْحَقَّ أَقُولُ لَكُمْ: لاَ يَقْدِرُ الاِبْنُ أَنْ يَعْمَلَ مِنْ نَفْسِهِ شَيْئاً إِلاَّ مَا يَنْظُرُ الآبَ يَعْمَلُ. لأَنْ مَهْمَا عَمِلَ ذَاكَ فَهَذَا يَعْمَلُهُ الاِبْنُ كَذَلِكَ.
\par 20 لأَنَّ الآبَ يُحِبُّ الاِبْنَ وَيُرِيهِ جَمِيعَ مَا هُوَ يَعْمَلُهُ وَسَيُرِيهِ أَعْمَالاً أَعْظَمَ مِنْ هَذِهِ لِتَتَعَجَّبُوا أَنْتُمْ.
\par 21 لأَنَّهُ كَمَا أَنَّ الآبَ يُقِيمُ الأَمْوَاتَ وَيُحْيِي كَذَلِكَ الاِبْنُ أَيْضاً يُحْيِي مَنْ يَشَاءُ.
\par 22 لأَنَّ الآبَ لاَ يَدِينُ أَحَداً بَلْ قَدْ أَعْطَى كُلَّ الدَّيْنُونَةِ لِلاِبْنِ
\par 23 لِكَيْ يُكْرِمَ الْجَمِيعُ الاِبْنَ كَمَا يُكْرِمُونَ الآبَ. مَنْ لاَ يُكْرِمُ الاِبْنَ لاَ يُكْرِمُ الآبَ الَّذِي أَرْسَلَهُ.
\par 24 «اَلْحَقَّ الْحَقَّ أَقُولُ لَكُمْ: إِنَّ مَنْ يَسْمَعُ كلاَمِي وَيُؤْمِنُ بِالَّذِي أَرْسَلَنِي فَلَهُ حَيَاةٌ أَبَدِيَّةٌ وَلاَ يَأْتِي إِلَى دَيْنُونَةٍ بَلْ قَدِ انْتَقَلَ مِنَ الْمَوْتِ إِلَى الْحَيَاةِ.
\par 25 اَلْحَقَّ الْحَقَّ أَقُولُ لَكُمْ: إِنَّهُ تَأْتِي سَاعَةٌ وَهِيَ الآنَ حِينَ يَسْمَعُ الأَمْوَاتُ صَوْتَ ابْنِ اللَّهِ وَالسَّامِعُونَ يَحْيَوْنَ.
\par 26 لأَنَّهُ كَمَا أَنَّ الآبَ لَهُ حَيَاةٌ فِي ذَاتِهِ كَذَلِكَ أَعْطَى الاِبْنَ أَيْضاً أَنْ تَكُونَ لَهُ حَيَاةٌ فِي ذَاتِهِ
\par 27 وَأَعْطَاهُ سُلْطَاناً أَنْ يَدِينَ أَيْضاً لأَنَّهُ ابْنُ الإِنْسَانِ.
\par 28 لاَ تَتَعَجَّبُوا مِنْ هَذَا فَإِنَّهُ تَأْتِي سَاعَةٌ فِيهَا يَسْمَعُ جَمِيعُ الَّذِينَ فِي الْقُبُورِ صَوْتَهُ
\par 29 فَيَخْرُجُ الَّذِينَ فَعَلُوا الصَّالِحَاتِ إِلَى قِيَامَةِ الْحَيَاةِ وَالَّذِينَ عَمِلُوا السَّيِّئَاتِ إِلَى قِيَامَةِ الدَّيْنُونَةِ.
\par 30 أَنَا لاَ أَقْدِرُ أَنْ أَفْعَلَ مِنْ نَفْسِي شَيْئاً. كَمَا أَسْمَعُ أَدِينُ وَدَيْنُونَتِي عَادِلَةٌ لأَنِّي لاَ أَطْلُبُ مَشِيئَتِي بَلْ مَشِيئَةَ الآبِ الَّذِي أَرْسَلَنِي.
\par 31 «إِنْ كُنْتُ أَشْهَدُ لِنَفْسِي فَشَهَادَتِي لَيْسَتْ حَقّاً.
\par 32 الَّذِي يَشْهَدُ لِي هُوَ آخَرُ وَأَنَا أَعْلَمُ أَنَّ شَهَادَتَهُ الَّتِي يَشْهَدُهَا لِي هِيَ حَقٌّ.
\par 33 أَنْتُمْ أَرْسَلْتُمْ إِلَى يُوحَنَّا فَشَهِدَ لِلْحَقِّ.
\par 34 وَأَنَا لاَ أَقْبَلُ شَهَادَةً مِنْ إِنْسَانٍ وَلَكِنِّي أَقُولُ هَذَا لِتَخْلُصُوا أَنْتُمْ.
\par 35 كَانَ هُوَ السِّرَاجَ الْمُوقَدَ الْمُنِيرَ وَأَنْتُمْ أَرَدْتُمْ أَنْ تَبْتَهِجُوا بِنُورِهِ سَاعَةً.
\par 36 وَأَمَّا أَنَا فَلِي شَهَادَةٌ أَعْظَمُ مِنْ يُوحَنَّا لأَنَّ الأَعْمَالَ الَّتِي أَعْطَانِي الآبُ لِأُكَمِّلَهَا هَذِهِ الأَعْمَالُ بِعَيْنِهَا الَّتِي أَنَا أَعْمَلُهَا هِيَ تَشْهَدُ لِي أَنَّ الآبَ قَدْ أَرْسَلَنِي.
\par 37 وَالآبُ نَفْسُهُ الَّذِي أَرْسَلَنِي يَشْهَدُ لِي. لَمْ تَسْمَعُوا صَوْتَهُ قَطُّ وَلاَ أَبْصَرْتُمْ هَيْئَتَهُ
\par 38 وَلَيْسَتْ لَكُمْ كَلِمَتُهُ ثَابِتَةً فِيكُمْ لأَنَّ الَّذِي أَرْسَلَهُ هُوَ لَسْتُمْ أَنْتُمْ تُؤْمِنُونَ بِهِ.
\par 39 فَتِّشُوا الْكُتُبَ لأَنَّكُمْ تَظُنُّونَ أَنَّ لَكُمْ فِيهَا حَيَاةً أَبَدِيَّةً. وَهِيَ الَّتِي تَشْهَدُ لِي.
\par 40 ولاَ تُرِيدُونَ أَنْ تَأْتُوا إِلَيَّ لِتَكُونَ لَكُمْ حَيَاةٌ.
\par 41 «مَجْداً مِنَ النَّاسِ لَسْتُ أَقْبَلُ
\par 42 وَلَكِنِّي قَدْ عَرَفْتُكُمْ أَنْ لَيْسَتْ لَكُمْ مَحَبَّةُ اللَّهِ فِي أَنْفُسِكُمْ.
\par 43 أَنَا قَدْ أَتَيْتُ بِاسْمِ أَبِي وَلَسْتُمْ تَقْبَلُونَنِي. إِنْ أَتَى آخَرُ بِاسْمِ نَفْسِهِ فَذَلِكَ تَقْبَلُونَهُ.
\par 44 كَيْفَ تَقْدِرُونَ أَنْ تُؤْمِنُوا وَأَنْتُمْ تَقْبَلُونَ مَجْداً بَعْضُكُمْ مِنْ بَعْضٍ؟ وَالْمَجْدُ الَّذِي مِنَ الإِلَهِ الْوَاحِدِ لَسْتُمْ تَطْلُبُونَهُ؟
\par 45 «لاَ تَظُنُّوا أَنِّي أَشْكُوكُمْ إِلَى الآبِ. يُوجَدُ الَّذِي يَشْكُوكُمْ وَهُوَ مُوسَى الَّذِي عَلَيْهِ رَجَاؤُكُمْ.
\par 46 لأَنَّكُمْ لَوْ كُنْتُمْ تُصَدِّقُونَ مُوسَى لَكُنْتُمْ تُصَدِّقُونَنِي لأَنَّهُ هُوَ كَتَبَ عَنِّي.
\par 47 فَإِنْ كُنْتُمْ لَسْتُمْ تُصَدِّقُونَ كُتُبَ ذَاكَ فَكَيْفَ تُصَدِّقُونَ كلاَمِي؟».

\chapter{6}

\par 1 إشباع الخمسة الآلاف رَجُل بَعْدَ هَذَا مَضَى يَسُوعُ إِلَى عَبْرِ بَحْرِ الْجَلِيلِ وَهُوَ بَحْرُ طَبَرِيَّةَ.
\par 2 وَتَبِعَهُ جَمْعٌ كَثِيرٌ لأَنَّهُمْ أَبْصَرُوا آيَاتِهِ الَّتِي كَانَ يَصْنَعُهَا فِي الْمَرْضَى.
\par 3 فَصَعِدَ يَسُوعُ إِلَى جَبَلٍ وَجَلَسَ هُنَاكَ مَعَ تلاَمِيذِهِ.
\par 4 وَكَانَ الْفِصْحُ عِيدُ الْيَهُودِ قَرِيباً.
\par 5 فَرَفَعَ يَسُوعُ عَيْنَيْهِ وَنَظَرَ أَنَّ جَمْعاً كَثِيراً مُقْبِلٌ إِلَيْهِ فَقَالَ لِفِيلُبُّسَ: «مِنْ أَيْنَ نَبْتَاعُ خُبْزاً لِيَأْكُلَ هَؤُلاَءِ؟»
\par 6 وَإِنَّمَا قَالَ هَذَا لِيَمْتَحِنَهُ لأَنَّهُ هُوَ عَلِمَ مَا هُوَ مُزْمِعٌ أَنْ يَفْعَلَ.
\par 7 أَجَابَهُ فِيلُبُّسُ: «لاَ يَكْفِيهِمْ خُبْزٌ بِمِئَتَيْ دِينَارٍ لِيَأْخُذَ كُلُّ وَاحِدٍ مِنْهُمْ شَيْئاً يَسِيراً».
\par 8 قَالَ لَهُ وَاحِدٌ مِنْ تلاَمِيذِهِ وَهُوَ أَنْدَرَاوُسُ أَخُو سِمْعَانَ بُطْرُسَ:
\par 9 «هُنَا غُلاَمٌ مَعَهُ خَمْسَةُ أَرْغِفَةِ شَعِيرٍ وَسَمَكَتَانِ وَلَكِنْ مَا هَذَا لِمِثْلِ هَؤُلاَءِ؟»
\par 10 فَقَالَ يَسُوعُ: «اجْعَلُوا النَّاسَ يَتَّكِئُونَ». وَكَانَ فِي الْمَكَانِ عُشْبٌ كَثِيرٌ فَاتَّكَأَ الرِّجَالُ وَعَدَدُهُمْ نَحْوُ خَمْسَةِ آلاَفٍ.
\par 11 وَأَخَذَ يَسُوعُ الأَرْغِفَةَ وَشَكَرَ وَوَزَّعَ عَلَى التّلاَمِيذِ وَالتّلاَمِيذُ أَعْطَوُا الْمُتَّكِئِينَ. وَكَذَلِكَ مِنَ السَّمَكَتَيْنِ بِقَدْرِ مَا شَاءُوا.
\par 12 فَلَمَّا شَبِعُوا قَالَ لِتلاَمِيذِهِ: «اجْمَعُوا الْكِسَرَ الْفَاضِلَةَ لِكَيْ لاَ يَضِيعَ شَيْءٌ».
\par 13 فَجَمَعُوا وَمَلَأُوا اثْنَتَيْ عَشْرَةَ قُفَّةً مِنَ الْكِسَرِ مِنْ خَمْسَةِ أَرْغِفَةِ الشَّعِيرِ الَّتِي فَضَلَتْ عَنِ الآكِلِينَ.
\par 14 فَلَمَّا رَأَى النَّاسُ الآيَةَ الَّتِي صَنَعَهَا يَسُوعُ قَالُوا: «إِنَّ هَذَا هُوَ بِالْحَقِيقَةِ النَّبِيُّ الآتِي إِلَى الْعَالَمِ!»
\par 15 وَأَمَّا يَسُوعُ فَإِذْ عَلِمَ أَنَّهُمْ مُزْمِعُونَ أَنْ يَأْتُوا وَيَخْتَطِفُوهُ لِيَجْعَلُوهُ مَلِكاً انْصَرَفَ أَيْضاً إِلَى الْجَبَلِ وَحْدَهُ.
\par 16 وَلَمَّا كَانَ الْمَسَاءُ نَزَلَ تلاَمِيذُهُ إِلَى الْبَحْرِ
\par 17 فَدَخَلُوا السَّفِينَةَ وَكَانُوا يَذْهَبُونَ إِلَى عَبْرِ الْبَحْرِ إِلَى كَفْرِنَاحُومَ. وَكَانَ الظّلاَمُ قَدْ أَقْبَلَ وَلَمْ يَكُنْ يَسُوعُ قَدْ أَتَى إِلَيْهِمْ.
\par 18 وَهَاجَ الْبَحْرُ مِنْ رِيحٍ عَظِيمَةٍ تَهُبُّ.
\par 19 فَلَمَّا كَانُوا قَدْ جَذَّفُوا نَحْوَ خَمْسٍ وَعِشْرِينَ أَوْ ثلاَثِينَ غَلْوَةً نَظَرُوا يَسُوعَ مَاشِياً عَلَى الْبَحْرِ مُقْتَرِباً مِنَ السَّفِينَةِ فَخَافُوا.
\par 20 فَقَالَ لَهُمْ: «أَنَا هُوَ لاَ تَخَافُوا».
\par 21 فَرَضُوا أَنْ يَقْبَلُوهُ فِي السَّفِينَةِ. وَلِلْوَقْتِ صَارَتِ السَّفِينَةُ إِلَى الأَرْضِ الَّتِي كَانُوا ذَاهِبِينَ إِلَيْهَا.
\par 22 وَفِي الْغَدِ لَمَّا رَأَى الْجَمْعُ الَّذِينَ كَانُوا وَاقِفِينَ فِي عَبْرِ الْبَحْرِ أَنَّهُ لَمْ تَكُنْ هُنَاكَ سَفِينَةٌ أُخْرَى سِوَى وَاحِدَةٍ وَهِيَ تِلْكَ الَّتِي دَخَلَهَا تلاَمِيذُهُ وَأَنَّ يَسُوعَ لَمْ يَدْخُلِ السَّفِينَةَ مَعَ تلاَمِيذِهِ بَلْ مَضَى تلاَمِيذُهُ وَحْدَهُمْ -
\par 23 غَيْرَ أَنَّهُ جَاءَتْ سُفُنٌ مِنْ طَبَرِيَّةَ إِلَى قُرْبِ الْمَوْضِعِ الَّذِي أَكَلُوا فِيهِ الْخُبْزَ إِذْ شَكَرَ الرَّبُّ -
\par 24 فَلَمَّا رَأَى الْجَمْعُ أَنَّ يَسُوعَ لَيْسَ هُوَ هُنَاكَ وَلاَ تلاَمِيذُهُ دَخَلُوا هُمْ أَيْضاً السُّفُنَ وَجَاءُوا إِلَى كَفْرِنَاحُومَ يَطْلُبُونَ يَسُوعَ.
\par 25 وَلَمَّا وَجَدُوهُ فِي عَبْرِ الْبَحْرِ قَالُوا لَهُ: «يَا مُعَلِّمُ مَتَى صِرْتَ هُنَا؟»
\par 26 أَجَابَهُمْ يَسُوعُ: «الْحَقَّ الْحَقَّ أَقُولُ لَكُمْ: أَنْتُمْ تَطْلُبُونَنِي لَيْسَ لأَنَّكُمْ رَأَيْتُمْ آيَاتٍ بَلْ لأَنَّكُمْ أَكَلْتُمْ مِنَ الْخُبْزِ فَشَبِعْتُمْ.
\par 27 اِعْمَلُوا لاَ لِلطَّعَامِ الْبَائِدِ بَلْ لِلطَّعَامِ الْبَاقِي لِلْحَيَاةِ الأَبَدِيَّةِ الَّذِي يُعْطِيكُمُ ابْنُ الإِنْسَانِ لأَنَّ هَذَا اللَّهُ الآبُ قَدْ خَتَمَهُ».
\par 28 فَقَالُوا لَهُ: «مَاذَا نَفْعَلُ حَتَّى نَعْمَلَ أَعْمَالَ اللَّهِ؟»
\par 29 أَجَابَ يَسُوعُ: «هَذَا هُوَ عَمَلُ اللَّهِ: أَنْ تُؤْمِنُوا بِالَّذِي هُوَ أَرْسَلَهُ».
\par 30 فَقَالُوا لَهُ: «فَأَيَّةَ آيَةٍ تَصْنَعُ لِنَرَى وَنُؤْمِنَ بِكَ؟ مَاذَا تَعْمَلُ؟
\par 31 آبَاؤُنَا أَكَلُوا الْمَنَّ فِي الْبَرِّيَّةِ كَمَا هُوَ مَكْتُوبٌ: أَنَّهُ أَعْطَاهُمْ خُبْزاً مِنَ السَّمَاءِ لِيَأْكُلُوا».
\par 32 فَقَالَ لَهُمْ يَسُوعُ: «الْحَقَّ الْحَقَّ أَقُولُ لَكُمْ: لَيْسَ مُوسَى أَعْطَاكُمُ الْخُبْزَ مِنَ السَّمَاءِ بَلْ أَبِي يُعْطِيكُمُ الْخُبْزَ الْحَقِيقِيَّ مِنَ السَّمَاءِ
\par 33 لأَنَّ خُبْزَ اللَّهِ هُوَ النَّازِلُ مِنَ السَّمَاءِ الْوَاهِبُ حَيَاةً لِلْعَالَمِ».
\par 34 فَقَالُوا لَهُ: «يَا سَيِّدُ أَعْطِنَا فِي كُلِّ حِينٍ هَذَا الْخُبْزَ».
\par 35 فَقَالَ لَهُمْ يَسُوعُ: «أَنَا هُوَ خُبْزُ الْحَيَاةِ. مَنْ يُقْبِلْ إِلَيَّ فلاَ يَجُوعُ وَمَنْ يُؤْمِنْ بِي فلاَ يَعْطَشُ أَبَداً.
\par 36 وَلَكِنِّي قُلْتُ لَكُمْ إِنَّكُمْ قَدْ رَأَيْتُمُونِي وَلَسْتُمْ تُؤْمِنُونَ.
\par 37 كُلُّ مَا يُعْطِينِي الآبُ فَإِلَيَّ يُقْبِلُ وَمَنْ يُقْبِلْ إِلَيَّ لاَ أُخْرِجْهُ خَارِجاً.
\par 38 لأَنِّي قَدْ نَزَلْتُ مِنَ السَّمَاءِ لَيْسَ لأَعْمَلَ مَشِيئَتِي بَلْ مَشِيئَةَ الَّذِي أَرْسَلَنِي.
\par 39 وَهَذِهِ مَشِيئَةُ الآبِ الَّذِي أَرْسَلَنِي: أَنَّ كُلَّ مَا أَعْطَانِي لاَ أُتْلِفُ مِنْهُ شَيْئاً بَلْ أُقِيمُهُ فِي الْيَوْمِ الأَخِيرِ.
\par 40 لأَنَّ هَذِهِ هِيَ مَشِيئَةُ الَّذِي أَرْسَلَنِي: أَنَّ كُلَّ مَنْ يَرَى الاِبْنَ وَيُؤْمِنُ بِهِ تَكُونُ لَهُ حَيَاةٌ أَبَدِيَّةٌ وَأَنَا أُقِيمُهُ فِي الْيَوْمِ الأَخِيرِ».
\par 41 فَكَانَ الْيَهُودُ يَتَذَمَّرُونَ عَلَيْهِ لأَنَّهُ قَالَ: «أَنَا هُوَ الْخُبْزُ الَّذِي نَزَلَ مِنَ السَّمَاءِ».
\par 42 وَقَالُوا: «أَلَيْسَ هَذَا هُوَ يَسُوعَ بْنَ يُوسُفَ الَّذِي نَحْنُ عَارِفُونَ بِأَبِيهِ وَأُمِّهِ. فَكَيْفَ يَقُولُ هَذَا: إِنِّي نَزَلْتُ مِنَ السَّمَاءِ؟»
\par 43 فَأَجَابَ يَسُوعُ: «لاَ تَتَذَمَّرُوا فِيمَا بَيْنَكُمْ.
\par 44 لاَ يَقْدِرُ أَحَدٌ أَنْ يُقْبِلَ إِلَيَّ إِنْ لَمْ يَجْتَذِبْهُ الآبُ الَّذِي أَرْسَلَنِي وَأَنَا أُقِيمُهُ فِي الْيَوْمِ الأَخِيرِ.
\par 45 إِنَّهُ مَكْتُوبٌ فِي الأَنْبِيَاءِ: وَيَكُونُ الْجَمِيعُ مُتَعَلِّمِينَ مِنَ اللَّهِ. فَكُلُّ مَنْ سَمِعَ مِنَ الآبِ وَتَعَلَّمَ يُقْبِلُ إِلَيَّ.
\par 46 لَيْسَ أَنَّ أَحَداً رَأَى الآبَ إِلاَّ الَّذِي مِنَ اللَّهِ. هَذَا قَدْ رَأَى الآبَ.
\par 47 اَلْحَقَّ الْحَقَّ أَقُولُ لَكُمْ: مَنْ يُؤْمِنُ بِي فَلَهُ حَيَاةٌ أَبَدِيَّةٌ.
\par 48 أَنَا هُوَ خُبْزُ الْحَيَاةِ.
\par 49 آبَاؤُكُمْ أَكَلُوا الْمَنَّ فِي الْبَرِّيَّةِ وَمَاتُوا.
\par 50 هَذَا هُوَ الْخُبْزُ النَّازِلُ مِنَ السَّمَاءِ لِكَيْ يَأْكُلَ مِنْهُ الإِنْسَانُ وَلاَ يَمُوتَ.
\par 51 أَنَا هُوَ الْخُبْزُ الْحَيُّ الَّذِي نَزَلَ مِنَ السَّمَاءِ. إِنْ أَكَلَ أَحَدٌ مِنْ هَذَا الْخُبْزِ يَحْيَا إِلَى الأَبَدِ. وَالْخُبْزُ الَّذِي أَنَا أُعْطِي هُوَ جَسَدِي الَّذِي أَبْذِلُهُ مِنْ أَجْلِ حَيَاةِ الْعَالَمِ».
\par 52 فَخَاصَمَ الْيَهُودُ بَعْضُهُمْ بَعْضاً قَائِلِينَ: «كَيْفَ يَقْدِرُ هَذَا أَنْ يُعْطِيَنَا جَسَدَهُ لِنَأْكُلَ؟»
\par 53 فَقَالَ لَهُمْ يَسُوعُ: «الْحَقَّ الْحَقَّ أَقُولُ لَكُمْ: إِنْ لَمْ تَأْكُلُوا جَسَدَ ابْنِ الإِنْسَانِ وَتَشْرَبُوا دَمَهُ فَلَيْسَ لَكُمْ حَيَاةٌ فِيكُمْ.
\par 54 مَنْ يَأْكُلُ جَسَدِي وَيَشْرَبُ دَمِي فَلَهُ حَيَاةٌ أَبَدِيَّةٌ وَأَنَا أُقِيمُهُ فِي الْيَوْمِ الأَخِيرِ
\par 55 لأَنَّ جَسَدِي مَأْكَلٌ حَقٌّ وَدَمِي مَشْرَبٌ حَقٌّ.
\par 56 مَنْ يَأْكُلْ جَسَدِي وَيَشْرَبْ دَمِي يَثْبُتْ فِيَّ وَأَنَا فِيهِ.
\par 57 كَمَا أَرْسَلَنِي الآبُ الْحَيُّ وَأَنَا حَيٌّ بِالآبِ فَمَنْ يَأْكُلْنِي فَهُوَ يَحْيَا بِي.
\par 58 هَذَا هُوَ الْخُبْزُ الَّذِي نَزَلَ مِنَ السَّمَاءِ. لَيْسَ كَمَا أَكَلَ آبَاؤُكُمُ الْمَنَّ وَمَاتُوا. مَنْ يَأْكُلْ هَذَا الْخُبْزَ فَإِنَّهُ يَحْيَا إِلَى الأَبَدِ».
\par 59 قَالَ هَذَا فِي الْمَجْمَعِ وَهُوَ يُعَلِّمُ فِي كَفْرِنَاحُومَ.
\par 60 فَقَالَ كَثِيرُونَ مِنْ تلاَمِيذِهِ إِذْ سَمِعُوا: «إِنَّ هَذَا الْكلاَمَ صَعْبٌ! مَنْ يَقْدِرُ أَنْ يَسْمَعَهُ؟»
\par 61 فَعَلِمَ يَسُوعُ فِي نَفْسِهِ أَنَّ تلاَمِيذَهُ يَتَذَمَّرُونَ عَلَى هَذَا فَقَالَ لَهُمْ: «أَهَذَا يُعْثِرُكُمْ؟
\par 62 فَإِنْ رَأَيْتُمُ ابْنَ الإِنْسَانِ صَاعِداً إِلَى حَيْثُ كَانَ أَوَّلاً!
\par 63 اَلرُّوحُ هُوَ الَّذِي يُحْيِي. أَمَّا الْجَسَدُ فلاَ يُفِيدُ شَيْئاً. اَلْكلاَمُ الَّذِي أُكَلِّمُكُمْ بِهِ هُوَ رُوحٌ وَحَيَاةٌ
\par 64 وَلَكِنْ مِنْكُمْ قَوْمٌ لاَ يُؤْمِنُونَ». لأَنَّ يَسُوعَ مِنَ الْبَدْءِ عَلِمَ مَنْ هُمُ الَّذِينَ لاَ يُؤْمِنُونَ وَمَنْ هُوَ الَّذِي يُسَلِّمُهُ.
\par 65 فَقَالَ: «لِهَذَا قُلْتُ لَكُمْ إِنَّهُ لاَ يَقْدِرُ أَحَدٌ أَنْ يَأْتِيَ إِلَيَّ إِنْ لَمْ يُعْطَ مِنْ أَبِي».
\par 66 مِنْ هَذَا الْوَقْتِ رَجَعَ كَثِيرُونَ مِنْ تلاَمِيذِهِ إِلَى الْوَرَاءِ وَلَمْ يَعُودُوا يَمْشُونَ مَعَهُ.
\par 67 فَقَالَ يَسُوعُ لِلاِثْنَيْ عَشَرَ: «أَلَعَلَّكُمْ أَنْتُمْ أَيْضاً تُرِيدُونَ أَنْ تَمْضُوا؟»
\par 68 فَأَجَابَهُ سِمْعَانُ بُطْرُسُ: «يَا رَبُّ إِلَى مَنْ نَذْهَبُ؟ كلاَمُ الْحَيَاةِ الأَبَدِيَّةِ عِنْدَكَ
\par 69 وَنَحْنُ قَدْ آمَنَّا وَعَرَفْنَا أَنَّكَ أَنْتَ الْمَسِيحُ ابْنُ اللَّهِ الْحَيِّ».
\par 70 أَجَابَهُمْ يَسُوعُ: «أَلَيْسَ أَنِّي أَنَا اخْتَرْتُكُمْ الاِثْنَيْ عَشَرَ؟ وَوَاحِدٌ مِنْكُمْ شَيْطَانٌ!»
\par 71 قَالَ عَنْ يَهُوذَا سِمْعَانَ الإِسْخَرْيُوطِيِّ لأَنَّ هَذَا كَانَ مُزْمِعاً أَنْ يُسَلِّمَهُ وَهُوَ وَاحِدٌ مِنَ الاِثْنَيْ عَشَرَ.

\chapter{7}

\par 1 وَكَانَ يَسُوعُ يَتَرَدَّدُ بَعْدَ هَذَا فِي الْجَلِيلِ لأَنَّهُ لَمْ يُرِدْ أَنْ يَتَرَدَّدَ فِي الْيَهُودِيَّةِ لأَنَّ الْيَهُودَ كَانُوا يَطْلُبُونَ أَنْ يَقْتُلُوهُ.
\par 2 وَكَانَ عِيدُ الْيَهُودِ عِيدُ الْمَظَالِّ قَرِيباً
\par 3 فَقَالَ لَهُ إِخْوَتُهُ: «انْتَقِلْ مِنْ هُنَا وَاذْهَبْ إِلَى الْيَهُودِيَّةِ لِكَيْ يَرَى تلاَمِيذُكَ أَيْضاً أَعْمَالَكَ الَّتِي تَعْمَلُ
\par 4 لأَنَّهُ لَيْسَ أَحَدٌ يَعْمَلُ شَيْئاً فِي الْخَفَاءِ وَهُوَ يُرِيدُ أَنْ يَكُونَ علاَنِيَةً. إِنْ كُنْتَ تَعْمَلُ هَذِهِ الأَشْيَاءَ فَأَظْهِرْ نَفْسَكَ لِلْعَالَمِ».
\par 5 لأَنَّ إِخْوَتَهُ أَيْضاً لَمْ يَكُونُوا يُؤْمِنُونَ بِهِ.
\par 6 فَقَالَ لَهُمْ يَسُوعُ: «إِنَّ وَقْتِي لَمْ يَحْضُرْ بَعْدُ وَأَمَّا وَقْتُكُمْ فَفِي كُلِّ حِينٍ حَاضِرٌ.
\par 7 لاَ يَقْدِرُ الْعَالَمُ أَنْ يُبْغِضَكُمْ وَلَكِنَّهُ يُبْغِضُنِي أَنَا لأَنِّي أَشْهَدُ عَلَيْهِ أَنَّ أَعْمَالَهُ شِرِّيرَةٌ.
\par 8 اِصْعَدُوا أَنْتُمْ إِلَى هَذَا الْعِيدِ. أَنَا لَسْتُ أَصْعَدُ بَعْدُ إِلَى هَذَا الْعِيدِ لأَنَّ وَقْتِي لَمْ يُكْمَلْ بَعْدُ».
\par 9 قَالَ لَهُمْ هَذَا وَمَكَثَ فِي الْجَلِيلِ.
\par 10 وَلَمَّا كَانَ إِخْوَتُهُ قَدْ صَعِدُوا حِينَئِذٍ صَعِدَ هُوَ أَيْضاً إِلَى الْعِيدِ لاَ ظَاهِراً بَلْ كَأَنَّهُ فِي الْخَفَاءِ.
\par 11 فَكَانَ الْيَهُودُ يَطْلُبُونَهُ فِي الْعِيدِ وَيَقُولُونَ: «أَيْنَ ذَاكَ؟»
\par 12 وَكَانَ فِي الْجُمُوعِ مُنَاجَاةٌ كَثِيرَةٌ مِنْ نَحْوِهِ. بَعْضُهُمْ يَقُولُونَ: «إِنَّهُ صَالِحٌ». وَآخَرُونَ يَقُولُونَ: «لاَ بَلْ يُضِلُّ الشَّعْبَ».
\par 13 وَلَكِنْ لَمْ يَكُنْ أَحَدٌ يَتَكَلَّمُ عَنْهُ جِهَاراً لِسَبَبِ الْخَوْفِ مِنَ الْيَهُودِ.
\par 14 وَلَمَّا كَانَ الْعِيدُ قَدِ انْتَصَفَ صَعِدَ يَسُوعُ إِلَى الْهَيْكَلِ وَكَانَ يُعَلِّمُ.
\par 15 فَتَعَجَّبَ الْيَهُودُ قَائِلِينَ: «كَيْفَ هَذَا يَعْرِفُ الْكُتُبَ وَهُوَ لَمْ يَتَعَلَّمْ؟»
\par 16 أَجَابَهُمْ يَسُوعُ: «تَعْلِيمِي لَيْسَ لِي بَلْ لِلَّذِي أَرْسَلَنِي.
\par 17 إِنْ شَاءَ أَحَدٌ أَنْ يَعْمَلَ مَشِيئَتَهُ يَعْرِفُ التَّعْلِيمَ هَلْ هُوَ مِنَ اللَّهِ أَمْ أَتَكَلَّمُ أَنَا مِنْ نَفْسِي.
\par 18 مَنْ يَتَكَلَّمُ مِنْ نَفْسِهِ يَطْلُبُ مَجْدَ نَفْسِهِ وَأَمَّا مَنْ يَطْلُبُ مَجْدَ الَّذِي أَرْسَلَهُ فَهُوَ صَادِقٌ وَلَيْسَ فِيهِ ظُلْمٌ.
\par 19 أَلَيْسَ مُوسَى قَدْ أَعْطَاكُمُ النَّامُوسَ؟ وَلَيْسَ أَحَدٌ مِنْكُمْ يَعْمَلُ النَّامُوسَ! لِمَاذَا تَطْلُبُونَ أَنْ تَقْتُلُونِي؟»
\par 20 أَجَابَ الْجَمْعُ: «بِكَ شَيْطَانٌ. مَنْ يَطْلُبُ أَنْ يَقْتُلَكَ؟»
\par 21 فَقَالَ يَسُوعُ لَهُمْ: «عَمَلاً وَاحِداً عَمِلْتُ فَتَتَعَجَّبُونَ جَمِيعاً.
\par 22 لِهَذَا أَعْطَاكُمْ مُوسَى الْخِتَانَ لَيْسَ أَنَّهُ مِنْ مُوسَى بَلْ مِنَ الآبَاءِ. فَفِي السَّبْتِ تَخْتِنُونَ الإِنْسَانَ.
\par 23 فَإِنْ كَانَ الإِنْسَانُ يَقْبَلُ الْخِتَانَ فِي السَّبْتِ لِئَلَّا يُنْقَضَ نَامُوسُ مُوسَى أَفَتَسْخَطُونَ عَلَيَّ لأَنِّي شَفَيْتُ إِنْسَاناً كُلَّهُ فِي السَّبْتِ؟
\par 24 لاَ تَحْكُمُوا حَسَبَ الظَّاهِرِ بَلِ احْكُمُوا حُكْماً عَادِلاً».
\par 25 فَقَالَ قَوْمٌ مِنْ أَهْلِ أُورُشَلِيمَ: «أَلَيْسَ هَذَا هُوَ الَّذِي يَطْلُبُونَ أَنْ يَقْتُلُوهُ؟
\par 26 وَهَا هُوَ يَتَكَلَّمُ جِهَاراً وَلاَ يَقُولُونَ لَهُ شَيْئاً! أَلَعَلَّ الرُّؤَسَاءَ عَرَفُوا يَقِيناً أَنَّ هَذَا هُوَ الْمَسِيحُ حَقّاً؟
\par 27 وَلَكِنَّ هَذَا نَعْلَمُ مِنْ أَيْنَ هُوَ وَأَمَّا الْمَسِيحُ فَمَتَى جَاءَ لاَ يَعْرِفُ أَحَدٌ مِنْ أَيْنَ هُوَ».
\par 28 فَنَادَى يَسُوعُ وَهُوَ يُعَلِّمُ فِي الْهَيْكَلِ: «تَعْرِفُونَنِي وَتَعْرِفُونَ مِنْ أَيْنَ أَنَا وَمِنْ نَفْسِي لَمْ آتِ بَلِ الَّذِي أَرْسَلَنِي هُوَ حَقٌّ الَّذِي أَنْتُمْ لَسْتُمْ تَعْرِفُونَهُ.
\par 29 أَنَا أَعْرِفُهُ لأَنِّي مِنْهُ وَهُوَ أَرْسَلَنِي».
\par 30 فَطَلَبُوا أَنْ يُمْسِكُوهُ وَلَمْ يُلْقِ أَحَدٌ يَداً عَلَيْهِ لأَنَّ سَاعَتَهُ لَمْ تَكُنْ قَدْ جَاءَتْ بَعْدُ.
\par 31 فَآمَنَ بِهِ كَثِيرُونَ مِنَ الْجَمْعِ وَقَالُوا: «أَلَعَلَّ الْمَسِيحَ مَتَى جَاءَ يَعْمَلُ آيَاتٍ أَكْثَرَ مِنْ هَذِهِ الَّتِي عَمِلَهَا هَذَا؟».
\par 32 سَمِعَ الْفَرِّيسِيُّونَ الْجَمْعَ يَتَنَاجَوْنَ بِهَذَا مِنْ نَحْوِهِ فَأَرْسَلَ الْفَرِّيسِيُّونَ وَرُؤَسَاءُ الْكَهَنَةِ خُدَّاماً لِيُمْسِكُوهُ.
\par 33 فَقَالَ لَهُمْ يَسُوعُ: «أَنَا مَعَكُمْ زَمَاناً يَسِيراً بَعْدُ ثُمَّ أَمْضِي إِلَى الَّذِي أَرْسَلَنِي.
\par 34 سَتَطْلُبُونَنِي وَلاَ تَجِدُونَنِي وَحَيْثُ أَكُونُ أَنَا لاَ تَقْدِرُونَ أَنْتُمْ أَنْ تَأْتُوا».
\par 35 فَقَالَ الْيَهُودُ فِيمَا بَيْنَهُمْ: «إِلَى أَيْنَ هَذَا مُزْمِعٌ أَنْ يَذْهَبَ حَتَّى لاَ نَجِدَهُ نَحْنُ؟ أَلَعَلَّهُ مُزْمِعٌ أَنْ يَذْهَبَ إِلَى شَتَاتِ الْيُونَانِيِّينَ وَيُعَلِّمَ الْيُونَانِيِّينَ؟
\par 36 مَا هَذَا الْقَوْلُ الَّذِي قَالَ: سَتَطْلُبُونَنِي وَلاَ تَجِدُونَنِي وَحَيْثُ أَكُونُ أَنَا لاَ تَقْدِرُونَ أَنْتُمْ أَنْ تَأْتُوا؟».
\par 37 وَفِي الْيَوْمِ الأَخِيرِ الْعَظِيمِ مِنَ الْعِيدِ وَقَفَ يَسُوعُ وَنَادَى: «إِنْ عَطِشَ أَحَدٌ فَلْيُقْبِلْ إِلَيَّ وَيَشْرَبْ.
\par 38 مَنْ آمَنَ بِي كَمَا قَالَ الْكِتَابُ تَجْرِي مِنْ بَطْنِهِ أَنْهَارُ مَاءٍ حَيٍّ».
\par 39 قَالَ هَذَا عَنِ الرُّوحِ الَّذِي كَانَ الْمُؤْمِنُونَ بِهِ مُزْمِعِينَ أَنْ يَقْبَلُوهُ لأَنَّ الرُّوحَ الْقُدُسَ لَمْ يَكُنْ قَدْ أُعْطِيَ بَعْدُ لأَنَّ يَسُوعَ لَمْ يَكُنْ قَدْ مُجِّدَ بَعْدُ.
\par 40 فَكَثِيرُونَ مِنَ الْجَمْعِ لَمَّا سَمِعُوا هَذَا الْكلاَمَ قَالُوا: «هَذَا بِالْحَقِيقَةِ هُوَ النَّبِيُّ».
\par 41 آخَرُونَ قَالُوا: «هَذَا هُوَ الْمَسِيحُ». وَآخَرُونَ قَالُوا: «أَلَعَلَّ الْمَسِيحَ مِنَ الْجَلِيلِ يَأْتِي؟
\par 42 أَلَمْ يَقُلِ الْكِتَابُ إِنَّهُ مِنْ نَسْلِ دَاوُدَ وَمِنْ بَيْتِ لَحْمٍ الْقَرْيَةِ الَّتِي كَانَ دَاوُدُ فِيهَا يَأْتِي الْمَسِيحُ؟»
\par 43 فَحَدَثَ انْشِقَاقٌ فِي الْجَمْعِ لِسَبَبِهِ.
\par 44 وَكَانَ قَوْمٌ مِنْهُمْ يُرِيدُونَ أَنْ يُمْسِكُوهُ وَلَكِنْ لَمْ يُلْقِ أَحَدٌ عَلَيْهِ الأَيَادِيَ.
\par 45 فَجَاءَ الْخُدَّامُ إِلَى رُؤَسَاءِ الْكَهَنَةِ وَالْفَرِّيسِيِّينَ. فَقَالَ هَؤُلاَءِ لَهُمْ: «لِمَاذَا لَمْ تَأْتُوا بِهِ؟»
\par 46 أَجَابَ الْخُدَّامُ: «لَمْ يَتَكَلَّمْ قَطُّ إِنْسَانٌ هَكَذَا مِثْلَ هَذَا الإِنْسَانِ».
\par 47 فَأَجَابَهُمُ الْفَرِّيسِيُّونَ: «أَلَعَلَّكُمْ أَنْتُمْ أَيْضاً قَدْ ضَلَلْتُمْ؟
\par 48 أَلَعَلَّ أَحَداً مِنَ الرُّؤَسَاءِ أَوْ مِنَ الْفَرِّيسِيِّينَ آمَنَ بِهِ؟
\par 49 وَلَكِنَّ هَذَا الشَّعْبَ الَّذِي لاَ يَفْهَمُ النَّامُوسَ هُوَ مَلْعُونٌ».
\par 50 قَالَ لَهُمْ نِيقُودِيمُوسُ الَّذِي جَاءَ إِلَيْهِ لَيْلاً وَهُوَ وَاحِدٌ مِنْهُمْ:
\par 51 «أَلَعَلَّ نَامُوسَنَا يَدِينُ إِنْسَاناً لَمْ يَسْمَعْ مِنْهُ أَوَّلاً وَيَعْرِفْ مَاذَا فَعَلَ؟»
\par 52 أَجَابُوا: «أَلَعَلَّكَ أَنْتَ أَيْضاً مِنَ الْجَلِيلِ؟ فَتِّشْ وَانْظُرْ! إِنَّهُ لَمْ يَقُمْ نَبِيٌّ مِنَ الْجَلِيلِ».
\par 53 فَمَضَى كُلُّ وَاحِدٍ إِلَى بَيْتِهِ.

\chapter{8}

\par 1 أَمَّا يَسُوعُ فَمَضَى إِلَى جَبَلِ الزَّيْتُونِ.
\par 2 ثُمَّ حَضَرَ أَيْضاً إِلَى الْهَيْكَلِ فِي الصُّبْحِ وَجَاءَ إِلَيْهِ جَمِيعُ الشَّعْبِ فَجَلَسَ يُعَلِّمُهُمْ.
\par 3 وَقَدَّمَ إِلَيْهِ الْكَتَبَةُ وَالْفَرِّيسِيُّونَ امْرَأَةً أُمْسِكَتْ فِي زِناً. وَلَمَّا أَقَامُوهَا فِي الْوَسَطِ
\par 4 قَالُوا لَهُ: «يَا مُعَلِّمُ هَذِهِ الْمَرْأَةُ أُمْسِكَتْ وَهِيَ تَزْنِي فِي ذَاتِ الْفِعْلِ
\par 5 وَمُوسَى فِي النَّامُوسِ أَوْصَانَا أَنَّ مِثْلَ هَذِهِ تُرْجَمُ. فَمَاذَا تَقُولُ أَنْتَ؟»
\par 6 قَالُوا هَذَا لِيُجَرِّبُوهُ لِكَيْ يَكُونَ لَهُمْ مَا يَشْتَكُونَ بِهِ عَلَيْهِ. وَأَمَّا يَسُوعُ فَانْحَنَى إِلَى أَسْفَلُ وَكَانَ يَكْتُبُ بِإِصْبِعِهِ عَلَى الأَرْضِ.
\par 7 وَلَمَّا اسْتَمَرُّوا يَسْأَلُونَهُ انْتَصَبَ وَقَالَ لَهُمْ: «مَنْ كَانَ مِنْكُمْ بِلاَ خَطِيَّةٍ فَلْيَرْمِهَا أَوَّلاً بِحَجَرٍ!»
\par 8 ثُمَّ انْحَنَى أَيْضاً إِلَى أَسْفَلُ وَكَانَ يَكْتُبُ عَلَى الأَرْضِ.
\par 9 وَأَمَّا هُمْ فَلَمَّا سَمِعُوا وَكَانَتْ ضَمَائِرُهُمْ تُبَكِّتُهُمْ خَرَجُوا وَاحِداً فَوَاحِداً مُبْتَدِئِينَ مِنَ الشُّيُوخِ إِلَى الآخِرِينَ. وَبَقِيَ يَسُوعُ وَحْدَهُ وَالْمَرْأَةُ وَاقِفَةٌ فِي الْوَسَطِ.
\par 10 فَلَمَّا انْتَصَبَ يَسُوعُ وَلَمْ يَنْظُرْ أَحَداً سِوَى الْمَرْأَةِ قَالَ لَهَا: «يَا امْرَأَةُ أَيْنَ هُمْ أُولَئِكَ الْمُشْتَكُونَ عَلَيْكِ؟ أَمَا دَانَكِ أَحَدٌ؟»
\par 11 فَقَالَتْ: «لاَ أَحَدَ يَا سَيِّدُ». فَقَالَ لَهَا يَسُوعُ: «ولاَ أَنَا أَدِينُكِ. اذْهَبِي وَلاَ تُخْطِئِي أَيْضاً».
\par 12 ثُمَّ كَلَّمَهُمْ يَسُوعُ أَيْضاً قَائِلاً: «أَنَا هُوَ نُورُ الْعَالَمِ. مَنْ يَتْبَعْنِي فلاَ يَمْشِي فِي الظُّلْمَةِ بَلْ يَكُونُ لَهُ نُورُ الْحَيَاةِ».
\par 13 فَقَالَ لَهُ الْفَرِّيسِيُّونَ: «أَنْتَ تَشْهَدُ لِنَفْسِكَ. شَهَادَتُكَ لَيْسَتْ حَقّاً».
\par 14 أَجَابَ يَسُوعُ: «وَإِنْ كُنْتُ أَشْهَدُ لِنَفْسِي فَشَهَادَتِي حَقٌّ لأَنِّي أَعْلَمُ مِنْ أَيْنَ أَتَيْتُ وَإِلَى أَيْنَ أَذْهَبُ. وَأَمَّا أَنْتُمْ فلاَ تَعْلَمُونَ مِنْ أَيْنَ آتِي وَلاَ إِلَى أَيْنَ أَذْهَبُ.
\par 15 أَنْتُمْ حَسَبَ الْجَسَدِ تَدِينُونَ أَمَّا أَنَا فَلَسْتُ أَدِينُ أَحَداً.
\par 16 وَإِنْ كُنْتُ أَنَا أَدِينُ فَدَيْنُونَتِي حَقٌّ لأَنِّي لَسْتُ وَحْدِي بَلْ أَنَا وَالآبُ الَّذِي أَرْسَلَنِي.
\par 17 وَأَيْضاً فِي نَامُوسِكُمْ مَكْتُوبٌ: أَنَّ شَهَادَةَ رَجُلَيْنِ حَقٌّ.
\par 18 أَنَا هُوَ الشَّاهِدُ لِنَفْسِي وَيَشْهَدُ لِي الآبُ الَّذِي أَرْسَلَنِي».
\par 19 فَقَالُوا لَهُ: «أَيْنَ هُوَ أَبُوكَ؟» أَجَابَ يَسُوعُ: «لَسْتُمْ تَعْرِفُونَنِي أَنَا وَلاَ أَبِي. لَوْ عَرَفْتُمُونِي لَعَرَفْتُمْ أَبِي أَيْضاً».
\par 20 هَذَا الْكلاَمُ قَالَهُ يَسُوعُ فِي الْخِزَانَةِ وَهُوَ يُعَلِّمُ فِي الْهَيْكَلِ. وَلَمْ يُمْسِكْهُ أَحَدٌ لأَنَّ سَاعَتَهُ لَمْ تَكُنْ قَدْ جَاءَتْ بَعْدُ.
\par 21 قَالَ لَهُمْ يَسُوعُ أَيْضاً: «أَنَا أَمْضِي وَسَتَطْلُبُونَنِي وَتَمُوتُونَ فِي خَطِيَّتِكُمْ. حَيْثُ أَمْضِي أَنَا لاَ تَقْدِرُونَ أَنْتُمْ أَنْ تَأْتُوا»
\par 22 فَقَالَ الْيَهُودُ: «أَلَعَلَّهُ يَقْتُلُ نَفْسَهُ حَتَّى يَقُولُ: حَيْثُ أَمْضِي أَنَا لاَ تَقْدِرُونَ أَنْتُمْ أَنْ تَأْتُوا؟»
\par 23 فَقَالَ لَهُمْ: « أَنْتُمْ مِنْ أَسْفَلُ أَمَّا أَنَا فَمِنْ فَوْقُ. أَنْتُمْ مِنْ هَذَا الْعَالَمِ أَمَّا أَنَا فَلَسْتُ مِنْ هَذَا الْعَالَمِ.
\par 24 فَقُلْتُ لَكُمْ إِنَّكُمْ تَمُوتُونَ فِي خَطَايَاكُمْ لأَنَّكُمْ إِنْ لَمْ تُؤْمِنُوا أَنِّي أَنَا هُوَ تَمُوتُونَ فِي خَطَايَاكُمْ».
\par 25 فَقَالُوا لَهُ: «مَنْ أَنْتَ؟» فَقَالَ لَهُمْ يَسُوعُ: «أَنَا مِنَ الْبَدْءِ مَا أُكَلِّمُكُمْ أَيْضاً بِهِ.
\par 26 إِنَّ لِي أَشْيَاءَ كَثِيرَةً أَتَكَلَّمُ وَأَحْكُمُ بِهَا مِنْ نَحْوِكُمْ لَكِنَّ الَّذِي أَرْسَلَنِي هُوَ حَقٌّ. وَأَنَا مَا سَمِعْتُهُ مِنْهُ فَهَذَا أَقُولُهُ لِلْعَالَمِ».
\par 27 وَلَمْ يَفْهَمُوا أَنَّهُ كَانَ يَقُولُ لَهُمْ عَنِ الآبِ.
\par 28 فَقَالَ لَهُمْ يَسُوعُ: «مَتَى رَفَعْتُمُ ابْنَ الإِنْسَانِ فَحِينَئِذٍ تَفْهَمُونَ أَنِّي أَنَا هُوَ وَلَسْتُ أَفْعَلُ شَيْئاً مِنْ نَفْسِي بَلْ أَتَكَلَّمُ بِهَذَا كَمَا عَلَّمَنِي أَبِي.
\par 29 وَالَّذِي أَرْسَلَنِي هُوَ مَعِي وَلَمْ يَتْرُكْنِي الآبُ وَحْدِي لأَنِّي فِي كُلِّ حِينٍ أَفْعَلُ مَا يُرْضِيهِ».
\par 30 وَبَيْنَمَا هُوَ يَتَكَلَّمُ بِهَذَا آمَنَ بِهِ كَثِيرُونَ.
\par 31 فَقَالَ يَسُوعُ لِلْيَهُودِ الَّذِينَ آمَنُوا بِهِ: «إِنَّكُمْ إِنْ ثَبَتُّمْ فِي كلاَمِي فَبِالْحَقِيقَةِ تَكُونُونَ تلاَمِيذِي
\par 32 وَتَعْرِفُونَ الْحَقَّ وَالْحَقُّ يُحَرِّرُكُمْ».
\par 33 أَجَابُوهُ: «إِنَّنَا ذُرِّيَّةُ إِبْرَاهِيمَ وَلَمْ نُسْتَعْبَدْ لأَحَدٍ قَطُّ. كَيْفَ تَقُولُ أَنْتَ: إِنَّكُمْ تَصِيرُونَ أَحْرَاراً؟»
\par 34 أَجَابَهُمْ يَسُوعُ: «الْحَقَّ الْحَقَّ أَقُولُ لَكُمْ: إِنَّ كُلَّ مَنْ يَعْمَلُ الْخَطِيَّةَ هُوَ عَبْدٌ لِلْخَطِيَّةِ.
\par 35 وَالْعَبْدُ لاَ يَبْقَى فِي الْبَيْتِ إِلَى الأَبَدِ أَمَّا الاِبْنُ فَيَبْقَى إِلَى الأَبَدِ.
\par 36 فَإِنْ حَرَّرَكُمْ الاِبْنُ فَبِالْحَقِيقَةِ تَكُونُونَ أَحْرَاراً.
\par 37 أَنَا عَالِمٌ أَنَّكُمْ ذُرِّيَّةُ إِبْرَاهِيمَ. لَكِنَّكُمْ تَطْلُبُونَ أَنْ تَقْتُلُونِي لأَنَّ كلاَمِي لاَ مَوْضِعَ لَهُ فِيكُمْ.
\par 38 أَنَا أَتَكَلَّمُ بِمَا رَأَيْتُ عِنْدَ أَبِي وَأَنْتُمْ تَعْمَلُونَ مَا رَأَيْتُمْ عِنْدَ أَبِيكُمْ».
\par 39 أَجَابُوا: «أَبُونَا هُوَ إِبْرَاهِيمُ». قَالَ لَهُمْ يَسُوعُ: «لَوْ كُنْتُمْ أَوْلاَدَ إِبْرَاهِيمَ لَكُنْتُمْ تَعْمَلُونَ أَعْمَالَ إِبْرَاهِيمَ!
\par 40 وَلَكِنَّكُمُ الآنَ تَطْلُبُونَ أَنْ تَقْتُلُونِي وَأَنَا إِنْسَانٌ قَدْ كَلَّمَكُمْ بِالْحَقِّ الَّذِي سَمِعَهُ مِنَ اللَّهِ. هَذَا لَمْ يَعْمَلْهُ إِبْرَاهِيمُ.
\par 41 أَنْتُمْ تَعْمَلُونَ أَعْمَالَ أَبِيكُمْ». فَقَالُوا لَهُ: «إِنَّنَا لَمْ نُولَدْ مِنْ زِناً. لَنَا أَبٌ وَاحِدٌ وَهُوَ اللَّهُ».
\par 42 فَقَالَ لَهُمْ يَسُوعُ: «لَوْ كَانَ اللَّهُ أَبَاكُمْ لَكُنْتُمْ تُحِبُّونَنِي لأَنِّي خَرَجْتُ مِنْ قِبَلِ اللَّهِ وَأَتَيْتُ. لأَنِّي لَمْ آتِ مِنْ نَفْسِي بَلْ ذَاكَ أَرْسَلَنِي.
\par 43 لِمَاذَا لاَ تَفْهَمُونَ كلاَمِي؟ لأَنَّكُمْ لاَ تَقْدِرُونَ أَنْ تَسْمَعُوا قَوْلِي.
\par 44 أَنْتُمْ مِنْ أَبٍ هُوَ إِبْلِيسُ وَشَهَوَاتِ أَبِيكُمْ تُرِيدُونَ أَنْ تَعْمَلُوا. ذَاكَ كَانَ قَتَّالاً لِلنَّاسِ مِنَ الْبَدْءِ وَلَمْ يَثْبُتْ فِي الْحَقِّ لأَنَّهُ لَيْسَ فِيهِ حَقٌّ. مَتَى تَكَلَّمَ بِالْكَذِبِ فَإِنَّمَا يَتَكَلَّمُ مِمَّا لَهُ لأَنَّهُ كَذَّابٌ وَأَبُو الْكَذَّابِ.
\par 45 وَأَمَّا أَنَا فَلأَنِّي أَقُولُ الْحَقَّ لَسْتُمْ تُؤْمِنُونَ بِي.
\par 46 مَنْ مِنْكُمْ يُبَكِّتُنِي عَلَى خَطِيَّةٍ؟ فَإِنْ كُنْتُ أَقُولُ الْحَقَّ فَلِمَاذَا لَسْتُمْ تُؤْمِنُونَ بِي؟
\par 47 اَلَّذِي مِنَ اللَّهِ يَسْمَعُ كلاَمَ اللَّهِ. لِذَلِكَ أَنْتُمْ لَسْتُمْ تَسْمَعُونَ لأَنَّكُمْ لَسْتُمْ مِنَ اللَّهِ».
\par 48 فَقَالَ الْيَهُودُ: «أَلَسْنَا نَقُولُ حَسَناً إِنَّكَ سَامِرِيٌّ وَبِكَ شَيْطَانٌ؟»
\par 49 أَجَابَ يَسُوعُ: «أَنَا لَيْسَ بِي شَيْطَانٌ لَكِنِّي أُكْرِمُ أَبِي وَأَنْتُمْ تُهِينُونَنِي.
\par 50 أَنَا لَسْتُ أَطْلُبُ مَجْدِي. يُوجَدُ مَنْ يَطْلُبُ وَيَدِينُ.
\par 51 اَلْحَقَّ الْحَقَّ أَقُولُ لَكُمْ: إِنْ كَانَ أَحَدٌ يَحْفَظُ كلاَمِي فَلَنْ يَرَى الْمَوْتَ إِلَى الأَبَدِ».
\par 52 فَقَالَ لَهُ الْيَهُودُ: «الآنَ عَلِمْنَا أَنَّ بِكَ شَيْطَاناً. قَدْ مَاتَ إِبْرَاهِيمُ وَالأَنْبِيَاءُ وَأَنْتَ تَقُولُ: «إِنْ كَانَ أَحَدٌ يَحْفَظُ كلاَمِي فَلَنْ يَذُوقَ الْمَوْتَ إِلَى الأَبَدِ».
\par 53 أَلَعَلَّكَ أَعْظَمُ مِنْ أَبِينَا إِبْرَاهِيمَ الَّذِي مَاتَ. وَالأَنْبِيَاءُ مَاتُوا. مَنْ تَجْعَلُ نَفْسَكَ؟»
\par 54 أَجَابَ يَسُوعُ: «إِنْ كُنْتُ أُمَجِّدُ نَفْسِي فَلَيْسَ مَجْدِي شَيْئاً. أَبِي هُوَ الَّذِي يُمَجِّدُنِي الَّذِي تَقُولُونَ أَنْتُمْ إِنَّهُ إِلَهُكُمْ
\par 55 وَلَسْتُمْ تَعْرِفُونَهُ. وَأَمَّا أَنَا فَأَعْرِفُهُ. وَإِنْ قُلْتُ إِنِّي لَسْتُ أَعْرِفُهُ أَكُونُ مِثْلَكُمْ كَاذِباً لَكِنِّي أَعْرِفُهُ وَأَحْفَظُ قَوْلَهُ.
\par 56 أَبُوكُمْ إِبْرَاهِيمُ تَهَلَّلَ بِأَنْ يَرَى يَوْمِي فَرَأَى وَفَرِحَ».
\par 57 فَقَالَ لَهُ الْيَهُودُ: «لَيْسَ لَكَ خَمْسُونَ سَنَةً بَعْدُ أَفَرَأَيْتَ إِبْرَاهِيمَ؟»
\par 58 قَالَ لَهُمْ يَسُوعُ: «الْحَقَّ الْحَقَّ أَقُولُ لَكُمْ: قَبْلَ أَنْ يَكُونَ إِبْرَاهِيمُ أَنَا كَائِنٌ».
\par 59 فَرَفَعُوا حِجَارَةً لِيَرْجُمُوهُ. أَمَّا يَسُوعُ فَاخْتَفَى وَخَرَجَ مِنَ الْهَيْكَلِ مُجْتَازاً فِي وَسْطِهِمْ وَمَضَى هَكَذَا.

\chapter{9}

\par 1 وَفِيمَا هُوَ مُجْتَازٌ رَأَى إِنْسَاناً أَعْمَى مُنْذُ وِلاَدَتِهِ
\par 2 فَسَأَلَهُ تلاَمِيذُهُ: «يَا مُعَلِّمُ مَنْ أَخْطَأَ: هَذَا أَمْ أَبَوَاهُ حَتَّى وُلِدَ أَعْمَى؟»
\par 3 أَجَابَ يَسُوعُ: «لاَ هَذَا أَخْطَأَ وَلاَ أَبَوَاهُ لَكِنْ لِتَظْهَرَ أَعْمَالُ اللَّهِ فِيهِ.
\par 4 يَنْبَغِي أَنْ أَعْمَلَ أَعْمَالَ الَّذِي أَرْسَلَنِي مَا دَامَ نَهَارٌ. يَأْتِي لَيْلٌ حِينَ لاَ يَسْتَطِيعُ أَحَدٌ أَنْ يَعْمَلَ.
\par 5 مَا دُمْتُ فِي الْعَالَمِ فَأَنَا نُورُ الْعَالَمِ».
\par 6 قَالَ هَذَا وَتَفَلَ عَلَى الأَرْضِ وَصَنَعَ مِنَ التُّفْلِ طِيناً وَطَلَى بِالطِّينِ عَيْنَيِ الأَعْمَى.
\par 7 وَقَالَ لَهُ: «اذْهَبِ اغْتَسِلْ فِي بِرْكَةِ سِلْوَامَ». الَّذِي تَفْسِيرُهُ مُرْسَلٌ. فَمَضَى وَاغْتَسَلَ وَأَتَى بَصِيراً.
\par 8 فَالْجِيرَانُ وَالَّذِينَ كَانُوا يَرَوْنَهُ قَبْلاً أَنَّهُ كَانَ أَعْمَى قَالُوا: «أَلَيْسَ هَذَا هُوَ الَّذِي كَانَ يَجْلِسُ وَيَسْتَعْطِي؟»
\par 9 آخَرُونَ قَالُوا: «هَذَا هُوَ». وَآخَرُونَ: «إِنَّهُ يُشْبِهُهُ». وَأَمَّا هُوَ فَقَالَ: «إِنِّي أَنَا هُوَ».
\par 10 فَقَالُوا لَهُ: «كَيْفَ انْفَتَحَتْ عَيْنَاكَ؟»
\par 11 أَجَابَ: «إِنْسَانٌ يُقَالُ لَهُ يَسُوعُ صَنَعَ طِيناً وَطَلَى عَيْنَيَّ وَقَالَ لِي: اذْهَبْ إِلَى بِرْكَةِ سِلْوَامَ وَاغْتَسِلْ. فَمَضَيْتُ وَاغْتَسَلْتُ فَأَبْصَرْتُ».
\par 12 فَقَالُوا لَهُ: «أَيْنَ ذَاكَ؟» قَالَ: «لاَ أَعْلَمُ».
\par 13 فَأَتَوْا إِلَى الْفَرِّيسِيِّينَ بِالَّذِي كَانَ قَبْلاً أَعْمَى.
\par 14 وَكَانَ سَبْتٌ حِينَ صَنَعَ يَسُوعُ الطِّينَ وَفَتَحَ عَيْنَيْهِ.
\par 15 فَسَأَلَهُ الْفَرِّيسِيُّونَ أَيْضاً كَيْفَ أَبْصَرَ فَقَالَ لَهُمْ: «وَضَعَ طِيناً عَلَى عَيْنَيَّ وَاغْتَسَلْتُ فَأَنَا أُبْصِرُ».
\par 16 فَقَالَ قَوْمٌ مِنَ الْفَرِّيسِيِّينَ: «هَذَا الإِنْسَانُ لَيْسَ مِنَ اللَّهِ لأَنَّهُ لاَ يَحْفَظُ السَّبْتَ». آخَرُونَ قَالُوا: «كَيْفَ يَقْدِرُ إِنْسَانٌ خَاطِئٌ أَنْ يَعْمَلَ مِثْلَ هَذِهِ الآيَاتِ؟» وَكَانَ بَيْنَهُمُ انْشِقَاقٌ.
\par 17 قَالُوا أَيْضاً لِلأَعْمَى: «مَاذَا تَقُولُ أَنْتَ عَنْهُ مِنْ حَيْثُ إِنَّهُ فَتَحَ عَيْنَيْكَ؟» فَقَالَ: «إِنَّهُ نَبِيٌّ».
\par 18 فَلَمْ يُصَدِّقِ الْيَهُودُ عَنْهُ أَنَّهُ كَانَ أَعْمَى فَأَبْصَرَ حَتَّى دَعَوْا أَبَوَيِ الَّذِي أَبْصَرَ.
\par 19 فَسَأَلُوهُمَا: «أَهَذَا ابْنُكُمَا الَّذِي تَقُولاَنِ إِنَّهُ وُلِدَ أَعْمَى؟ فَكَيْفَ يُبْصِرُ الآنَ؟»
\par 20 أَجَابَهُمْ أَبَوَاهُ: «نَعْلَمُ أَنَّ هَذَا ابْنُنَا وَأَنَّهُ وُلِدَ أَعْمَى
\par 21 وَأَمَّا كَيْفَ يُبْصِرُ الآنَ فلاَ نَعْلَمُ. أَوْ مَنْ فَتَحَ عَيْنَيْهِ فلاَ نَعْلَمُ. هُوَ كَامِلُ السِّنِّ. اسْأَلُوهُ فَهُوَ يَتَكَلَّمُ عَنْ نَفْسِهِ».
\par 22 قَالَ أَبَوَاهُ هَذَا لأَنَّهُمَا كَانَا يَخَافَانِ مِنَ الْيَهُودِ لأَنَّ الْيَهُودَ كَانُوا قَدْ تَعَاهَدُوا أَنَّهُ إِنِ اعْتَرَفَ أَحَدٌ بِأَنَّهُ الْمَسِيحُ يُخْرَجُ مِنَ الْمَجْمَعِ.
\par 23 لِذَلِكَ قَالَ أَبَوَاهُ: «إِنَّهُ كَامِلُ السِّنِّ اسْأَلُوهُ».
\par 24 فَدَعَوْا ثَانِيَةً الإِنْسَانَ الَّذِي كَانَ أَعْمَى وَقَالُوا لَهُ: «أَعْطِ مَجْداً لِلَّهِ. نَحْنُ نَعْلَمُ أَنَّ هَذَا الإِنْسَانَ خَاطِئٌ».
\par 25 فَأَجَابَ: «أَخَاطِئٌ هُوَ؟ لَسْتُ أَعْلَمُ. إِنَّمَا أَعْلَمُ شَيْئاً وَاحِداً: أَنِّي كُنْتُ أَعْمَى وَالآنَ أُبْصِرُ».
\par 26 فَقَالُوا لَهُ أَيْضاً: «مَاذَا صَنَعَ بِكَ؟ كَيْفَ فَتَحَ عَيْنَيْكَ؟»
\par 27 أَجَابَهُمْ: «قَدْ قُلْتُ لَكُمْ وَلَمْ تَسْمَعُوا. لِمَاذَا تُرِيدُونَ أَنْ تَسْمَعُوا أَيْضاً؟ أَلَعَلَّكُمْ أَنْتُمْ تُرِيدُونَ أَنْ تَصِيرُوا لَهُ تلاَمِيذَ؟»
\par 28 فَشَتَمُوهُ وَقَالُوا: «أَنْتَ تِلْمِيذُ ذَاكَ وَأَمَّا نَحْنُ فَإِنَّنَا تلاَمِيذُ مُوسَى.
\par 29 نَحْنُ نَعْلَمُ أَنَّ مُوسَى كَلَّمَهُ اللَّهُ وَأَمَّا هَذَا فَمَا نَعْلَمُ مِنْ أَيْنَ هُوَ».
\par 30 أَجَابَ الرَّجُلُ: «إِنَّ فِي هَذَا عَجَباً! إِنَّكُمْ لَسْتُمْ تَعْلَمُونَ مِنْ أَيْنَ هُوَ وَقَدْ فَتَحَ عَيْنَيَّ.
\par 31 وَنَعْلَمُ أَنَّ اللَّهَ لاَ يَسْمَعُ لِلْخُطَاةِ. وَلَكِنْ إِنْ كَانَ أَحَدٌ يَتَّقِي اللَّهَ وَيَفْعَلُ مَشِيئَتَهُ فَلِهَذَا يَسْمَعُ.
\par 32 مُنْذُ الدَّهْرِ لَمْ يُسْمَعْ أَنَّ أَحَداً فَتَحَ عَيْنَيْ مَوْلُودٍ أَعْمَى.
\par 33 لَوْ لَمْ يَكُنْ هَذَا مِنَ اللَّهِ لَمْ يَقْدِرْ أَنْ يَفْعَلَ شَيْئاً».
\par 34 قَالُوا لَهُ: «فِي الْخَطَايَا وُلِدْتَ أَنْتَ بِجُمْلَتِكَ وَأَنْتَ تُعَلِّمُنَا!» فَأَخْرَجُوهُ خَارِجاً.
\par 35 فَسَمِعَ يَسُوعُ أَنَّهُمْ أَخْرَجُوهُ خَارِجاً فَوَجَدَهُ وَقَالَ لَهُ: «أَتُؤْمِنُ بِابْنِ اللَّهِ؟»
\par 36 أَجَابَ: «مَنْ هُوَ يَا سَيِّدُ لِأُومِنَ بِهِ؟»
\par 37 فَقَالَ لَهُ يَسُوعُ: «قَدْ رَأَيْتَهُ وَالَّذِي يَتَكَلَّمُ مَعَكَ هُوَ هُوَ».
\par 38 فَقَالَ: «أُومِنُ يَا سَيِّدُ». وَسَجَدَ لَهُ.
\par 39 فَقَالَ يَسُوعُ: « لِدَيْنُونَةٍ أَتَيْتُ أَنَا إِلَى هَذَا الْعَالَمِ حَتَّى يُبْصِرَ الَّذِينَ لاَ يُبْصِرُونَ وَيَعْمَى الَّذِينَ يُبْصِرُونَ».
\par 40 فَسَمِعَ هَذَا الَّذِينَ كَانُوا مَعَهُ مِنَ الْفَرِّيسِيِّينَ وَقَالُوا لَهُ: «أَلَعَلَّنَا نَحْنُ أَيْضاً عُمْيَانٌ؟»
\par 41 قَالَ لَهُمْ يَسُوعُ: «لَوْ كُنْتُمْ عُمْيَاناً لَمَا كَانَتْ لَكُمْ خَطِيَّةٌ. وَلَكِنِ الآنَ تَقُولُونَ إِنَّنَا نُبْصِرُ فَخَطِيَّتُكُمْ بَاقِيَةٌ».

\chapter{10}

\par 1 «اَلْحَقَّ الْحَقَّ أَقُولُ لَكُمْ: إِنَّ الَّذِي لاَ يَدْخُلُ مِنَ الْبَابِ إِلَى حَظِيرَةِ الْخِرَافِ بَلْ يَطْلَعُ مِنْ مَوْضِعٍ آخَرَ فَذَاكَ سَارِقٌ وَلِصٌّ.
\par 2 وَأَمَّا الَّذِي يَدْخُلُ مِنَ الْبَابِ فَهُوَ رَاعِي الْخِرَافِ.
\par 3 لِهَذَا يَفْتَحُ الْبَوَّابُ وَالْخِرَافُ تَسْمَعُ صَوْتَهُ فَيَدْعُو خِرَافَهُ الْخَاصَّةَ بِأَسْمَاءٍ وَيُخْرِجُهَا.
\par 4 وَمَتَى أَخْرَجَ خِرَافَهُ الْخَاصَّةَ يَذْهَبُ أَمَامَهَا وَالْخِرَافُ تَتْبَعُهُ لأَنَّهَا تَعْرِفُ صَوْتَهُ.
\par 5 وَأَمَّا الْغَرِيبُ فلاَ تَتْبَعُهُ بَلْ تَهْرُبُ مِنْهُ لأَنَّهَا لاَ تَعْرِفُ صَوْتَ الْغُرَبَاءِ».
\par 6 هَذَا الْمَثَلُ قَالَهُ لَهُمْ يَسُوعُ وَأَمَّا هُمْ فَلَمْ يَفْهَمُوا مَا هُوَ الَّذِي كَانَ يُكَلِّمُهُمْ بِهِ.
\par 7 فَقَالَ لَهُمْ يَسُوعُ أَيْضاً: «الْحَقَّ الْحَقَّ أَقُولُ لَكُمْ: إِنِّي أَنَا بَابُ الْخِرَافِ.
\par 8 جَمِيعُ الَّذِينَ أَتَوْا قَبْلِي هُمْ سُرَّاقٌ وَلُصُوصٌ وَلَكِنَّ الْخِرَافَ لَمْ تَسْمَعْ لَهُمْ.
\par 9 أَنَا هُوَ الْبَابُ. إِنْ دَخَلَ بِي أَحَدٌ فَيَخْلُصُ وَيَدْخُلُ وَيَخْرُجُ وَيَجِدُ مَرْعًى.
\par 10 اَلسَّارِقُ لاَ يَأْتِي إِلاَّ لِيَسْرِقَ وَيَذْبَحَ وَيُهْلِكَ وَأَمَّا أَنَا فَقَدْ أَتَيْتُ لِتَكُونَ لَهُمْ حَيَاةٌ وَلِيَكُونَ لَهُمْ أَفْضَلُ.
\par 11 أَنَا هُوَ الرَّاعِي الصَّالِحُ وَالرَّاعِي الصَّالِحُ يَبْذِلُ نَفْسَهُ عَنِ الْخِرَافِ.
\par 12 وَأَمَّا الَّذِي هُوَ أَجِيرٌ وَلَيْسَ رَاعِياً الَّذِي لَيْسَتِ الْخِرَافُ لَهُ فَيَرَى الذِّئْبَ مُقْبِلاً وَيَتْرُكُ الْخِرَافَ وَيَهْرُبُ فَيَخْطَفُ الذِّئْبُ الْخِرَافَ وَيُبَدِّدُهَا.
\par 13 وَالأَجِيرُ يَهْرُبُ لأَنَّهُ أَجِيرٌ وَلاَ يُبَالِي بِالْخِرَافِ.
\par 14 أَمَّا أَنَا فَإِنِّي الرَّاعِي الصَّالِحُ وَأَعْرِفُ خَاصَّتِي وَخَاصَّتِي تَعْرِفُنِي
\par 15 كَمَا أَنَّ الآبَ يَعْرِفُنِي وَأَنَا أَعْرِفُ الآبَ. وَأَنَا أَضَعُ نَفْسِي عَنِ الْخِرَافِ.
\par 16 وَلِي خِرَافٌ أُخَرُ لَيْسَتْ مِنْ هَذِهِ الْحَظِيرَةِ يَنْبَغِي أَنْ آتِيَ بِتِلْكَ أَيْضاً فَتَسْمَعُ صَوْتِي وَتَكُونُ رَعِيَّةٌ وَاحِدَةٌ وَرَاعٍ وَاحِدٌ.
\par 17 لِهَذَا يُحِبُّنِي الآبُ لأَنِّي أَضَعُ نَفْسِي لِآخُذَهَا أَيْضاً.
\par 18 لَيْسَ أَحَدٌ يَأْخُذُهَا مِنِّي بَلْ أَضَعُهَا أَنَا مِنْ ذَاتِي. لِي سُلْطَانٌ أَنْ أَضَعَهَا وَلِي سُلْطَانٌ أَنْ آخُذَهَا أَيْضاً. هَذِهِ الْوَصِيَّةُ قَبِلْتُهَا مِنْ أَبِي».
\par 19 فَحَدَثَ أَيْضاً انْشِقَاقٌ بَيْنَ الْيَهُودِ بِسَبَبِ هَذَا الْكلاَمِ.
\par 20 فَقَالَ كَثِيرُونَ مِنْهُمْ: «بِهِ شَيْطَانٌ وَهُوَ يَهْذِي. لِمَاذَا تَسْتَمِعُونَ لَهُ؟»
\par 21 آخَرُونَ قَالُوا: «لَيْسَ هَذَا كلاَمَ مَنْ بِهِ شَيْطَانٌ. أَلَعَلَّ شَيْطَاناً يَقْدِرُ أَنْ يَفْتَحَ أَعْيُنَ الْعُمْيَانِ؟».
\par 22 وَكَانَ عِيدُ التَّجْدِيدِ فِي أُورُشَلِيمَ وَكَانَ شِتَاءٌ.
\par 23 وَكَانَ يَسُوعُ يَتَمَشَّى فِي الْهَيْكَلِ فِي رِوَاقِ سُلَيْمَانَ
\par 24 فَاحْتَاطَ بِهِ الْيَهُودُ وَقَالُوا لَهُ: «إِلَى مَتَى تُعَلِّقُ أَنْفُسَنَا؟ إِنْ كُنْتَ أَنْتَ الْمَسِيحَ فَقُلْ لَنَا جَهْراً».
\par 25 أَجَابَهُمْ يَسُوعُ: «إِنِّي قُلْتُ لَكُمْ وَلَسْتُمْ تُؤْمِنُونَ. اَلأَعْمَالُ الَّتِي أَنَا أَعْمَلُهَا بِاسْمِ أَبِي هِيَ تَشْهَدُ لِي.
\par 26 وَلَكِنَّكُمْ لَسْتُمْ تُؤْمِنُونَ لأَنَّكُمْ لَسْتُمْ مِنْ خِرَافِي كَمَا قُلْتُ لَكُمْ.
\par 27 خِرَافِي تَسْمَعُ صَوْتِي وَأَنَا أَعْرِفُهَا فَتَتْبَعُنِي.
\par 28 وَأَنَا أُعْطِيهَا حَيَاةً أَبَدِيَّةً وَلَنْ تَهْلِكَ إِلَى الأَبَدِ وَلاَ يَخْطَفُهَا أَحَدٌ مِنْ يَدِي.
\par 29 أَبِي الَّذِي أَعْطَانِي إِيَّاهَا هُوَ أَعْظَمُ مِنَ الْكُلِّ وَلاَ يَقْدِرُ أَحَدٌ أَنْ يَخْطَفَ مِنْ يَدِ أَبِي.
\par 30 أَنَا وَالآبُ وَاحِدٌ».
\par 31 فَتَنَاوَلَ الْيَهُودُ أَيْضاً حِجَارَةً لِيَرْجُمُوهُ.
\par 32 فَقَالَ يَسُوعُ: «أَعْمَالاً كَثِيرَةً حَسَنَةً أَرَيْتُكُمْ مِنْ عِنْدِ أَبِي - بِسَبَبِ أَيِّ عَمَلٍ مِنْهَا تَرْجُمُونَنِي؟»
\par 33 أَجَابَهُ الْيَهُودُ: «لَسْنَا نَرْجُمُكَ لأَجْلِ عَمَلٍ حَسَنٍ بَلْ لأَجْلِ تَجْدِيفٍ فَإِنَّكَ وَأَنْتَ إِنْسَانٌ تَجْعَلُ نَفْسَكَ إِلَهاً»
\par 34 أَجَابَهُمْ يَسُوعُ: «أَلَيْسَ مَكْتُوباً فِي نَامُوسِكُمْ: أَنَا قُلْتُ إِنَّكُمْ آلِهَةٌ؟
\par 35 إِنْ قَالَ آلِهَةٌ لِأُولَئِكَ الَّذِينَ صَارَتْ إِلَيْهِمْ كَلِمَةُ اللَّهِ وَلاَ يُمْكِنُ أَنْ يُنْقَضَ الْمَكْتُوبُ
\par 36 فَالَّذِي قَدَّسَهُ الآبُ وَأَرْسَلَهُ إِلَى الْعَالَمِ أَتَقُولُونَ لَهُ: إِنَّكَ تُجَدِّفُ لأَنِّي قُلْتُ إِنِّي ابْنُ اللَّهِ؟
\par 37 إِنْ كُنْتُ لَسْتُ أَعْمَلُ أَعْمَالَ أَبِي فلاَ تُؤْمِنُوا بِي.
\par 38 وَلَكِنْ إِنْ كُنْتُ أَعْمَلُ فَإِنْ لَمْ تُؤْمِنُوا بِي فَآمِنُوا بِالأَعْمَالِ لِكَيْ تَعْرِفُوا وَتُؤْمِنُوا أَنَّ الآبَ فِيَّ وَأَنَا فِيهِ».
\par 39 فَطَلَبُوا أَيْضاً أَنْ يُمْسِكُوهُ فَخَرَجَ مِنْ أَيْدِيهِمْ
\par 40 وَمَضَى أَيْضاً إِلَى عَبْرِ الأُرْدُنِّ إِلَى الْمَكَانِ الَّذِي كَانَ يُوحَنَّا يُعَمِّدُ فِيهِ أَوَّلاً وَمَكَثَ هُنَاكَ.
\par 41 فَأَتَى إِلَيْهِ كَثِيرُونَ وَقَالُوا: «إِنَّ يُوحَنَّا لَمْ يَفْعَلْ آيَةً وَاحِدَةً وَلَكِنْ كُلُّ مَا قَالَهُ يُوحَنَّا عَنْ هَذَا كَانَ حَقّاً».
\par 42 فَآمَنَ كَثِيرُونَ بِهِ هُنَاكَ.

\chapter{11}

\par 1 وَكَانَ إِنْسَانٌ مَرِيضاً وَهُوَ لِعَازَرُ مِنْ بَيْتِ عَنْيَا مِنْ قَرْيَةِ مَرْيَمَ وَمَرْثَا أُخْتِهَا.
\par 2 وَكَانَتْ مَرْيَمُ الَّتِي كَانَ لِعَازَرُ أَخُوهَا مَرِيضاً هِيَ الَّتِي دَهَنَتِ الرَّبَّ بِطِيبٍ وَمَسَحَتْ رِجْلَيْهِ بِشَعْرِهَا.
\par 3 فَأَرْسَلَتِ الأُخْتَانِ إِلَيْهِ قَائِلَتَيْنِ: «يَا سَيِّدُ هُوَذَا الَّذِي تُحِبُّهُ مَرِيضٌ».
\par 4 فَلَمَّا سَمِعَ يَسُوعُ قَالَ: «هَذَا الْمَرَضُ لَيْسَ لِلْمَوْتِ بَلْ لأَجْلِ مَجْدِ اللَّهِ لِيَتَمَجَّدَ ابْنُ اللَّهِ بِهِ».
\par 5 وَكَانَ يَسُوعُ يُحِبُّ مَرْثَا وَأُخْتَهَا وَلِعَازَرَ.
\par 6 فَلَمَّا سَمِعَ أَنَّهُ مَرِيضٌ مَكَثَ حِينَئِذٍ فِي الْمَوْضِعِ الَّذِي كَانَ فِيهِ يَوْمَيْنِ.
\par 7 ثُمَّ بَعْدَ ذَلِكَ قَالَ لِتلاَمِيذِهِ: «لِنَذْهَبْ إِلَى الْيَهُودِيَّةِ أَيْضاً».
\par 8 قَالَ لَهُ التّلاَمِيذُ: «يَا مُعَلِّمُ الآنَ كَانَ الْيَهُودُ يَطْلُبُونَ أَنْ يَرْجُمُوكَ وَتَذْهَبُ أَيْضاً إِلَى هُنَاكَ».
\par 9 أَجَابَ يَسُوعُ: «أَلَيْسَتْ سَاعَاتُ النَّهَارِ اثْنَتَيْ عَشْرَةَ؟ إِنْ كَانَ أَحَدٌ يَمْشِي فِي النَّهَارِ لاَ يَعْثُرُ لأَنَّهُ يَنْظُرُ نُورَ هَذَا الْعَالَمِ
\par 10 وَلَكِنْ إِنْ كَانَ أَحَدٌ يَمْشِي فِي اللَّيْلِ يَعْثُرُ لأَنَّ النُّورَ لَيْسَ فِيهِ».
\par 11 قَالَ هَذَا وَبَعْدَ ذَلِكَ قَالَ لَهُمْ: «لِعَازَرُ حَبِيبُنَا قَدْ نَامَ. لَكِنِّي أَذْهَبُ لِأُوقِظَهُ».
\par 12 فَقَالَ تلاَمِيذُهُ: «يَا سَيِّدُ إِنْ كَانَ قَدْ نَامَ فَهُوَ يُشْفَى».
\par 13 وَكَانَ يَسُوعُ يَقُولُ عَنْ مَوْتِهِ وَهُمْ ظَنُّوا أَنَّهُ يَقُولُ عَنْ رُقَادِ النَّوْمِ.
\par 14 فَقَالَ لَهُمْ يَسُوعُ حِينَئِذٍ علاَنِيَةً: «لِعَازَرُ مَاتَ.
\par 15 وَأَنَا أَفْرَحُ لأَجْلِكُمْ إِنِّي لَمْ أَكُنْ هُنَاكَ لِتُؤْمِنُوا. وَلَكِنْ لِنَذْهَبْ إِلَيْهِ».
\par 16 فَقَالَ تُومَا الَّذِي يُقَالُ لَهُ التَّوْأَمُ لِلتّلاَمِيذِ رُفَقَائِهِ: «لِنَذْهَبْ نَحْنُ أَيْضاً لِكَيْ نَمُوتَ مَعَهُ».
\par 17 فَلَمَّا أَتَى يَسُوعُ وَجَدَ أَنَّهُ قَدْ صَارَ لَهُ أَرْبَعَةُ أَيَّامٍ فِي الْقَبْرِ.
\par 18 وَكَانَتْ بَيْتُ عَنْيَا قَرِيبَةً مِنْ أُورُشَلِيمَ نَحْوَ خَمْسَ عَشْرَةَ غَلْوَةً.
\par 19 وَكَانَ كَثِيرُونَ مِنَ الْيَهُودِ قَدْ جَاءُوا إِلَى مَرْثَا وَمَرْيَمَ لِيُعَزُّوهُمَا عَنْ أَخِيهِمَا.
\par 20 فَلَمَّا سَمِعَتْ مَرْثَا أَنَّ يَسُوعَ آتٍ لاَقَتْهُ وَأَمَّا مَرْيَمُ فَاسْتَمَرَّتْ جَالِسَةً فِي الْبَيْتِ.
\par 21 فَقَالَتْ مَرْثَا لِيَسُوعَ: «يَا سَيِّدُ لَوْ كُنْتَ هَهُنَا لَمْ يَمُتْ أَخِي.
\par 22 لَكِنِّي الآنَ أَيْضاً أَعْلَمُ أَنَّ كُلَّ مَا تَطْلُبُ مِنَ اللَّهِ يُعْطِيكَ اللَّهُ إِيَّاهُ».
\par 23 قَالَ لَهَا يَسُوعُ: «سَيَقُومُ أَخُوكِ».
\par 24 قَالَتْ لَهُ مَرْثَا: «أَنَا أَعْلَمُ أَنَّهُ سَيَقُومُ فِي الْقِيَامَةِ فِي الْيَوْمِ الأَخِيرِ».
\par 25 قَالَ لَهَا يَسُوعُ: «أَنَا هُوَ الْقِيَامَةُ وَالْحَيَاةُ. مَنْ آمَنَ بِي وَلَوْ مَاتَ فَسَيَحْيَا
\par 26 وَكُلُّ مَنْ كَانَ حَيّاً وَآمَنَ بِي فَلَنْ يَمُوتَ إِلَى الأَبَدِ. أَتُؤْمِنِينَ بِهَذَا؟»
\par 27 قَالَتْ لَهُ: «نَعَمْ يَا سَيِّدُ. أَنَا قَدْ آمَنْتُ أَنَّكَ أَنْتَ الْمَسِيحُ ابْنُ اللَّهِ الآتِي إِلَى الْعَالَمِ».
\par 28 وَلَمَّا قَالَتْ هَذَا مَضَتْ وَدَعَتْ مَرْيَمَ أُخْتَهَا سِرّاً قَائِلَةً: «الْمُعَلِّمُ قَدْ حَضَرَ وَهُوَ يَدْعُوكِ».
\par 29 أَمَّا تِلْكَ فَلَمَّا سَمِعَتْ قَامَتْ سَرِيعاً وَجَاءَتْ إِلَيْهِ.
\par 30 وَلَمْ يَكُنْ يَسُوعُ قَدْ جَاءَ إِلَى الْقَرْيَةِ بَلْ كَانَ فِي الْمَكَانِ الَّذِي لاَقَتْهُ فِيهِ مَرْثَا.
\par 31 ثُمَّ إِنَّ الْيَهُودَ الَّذِينَ كَانُوا مَعَهَا فِي الْبَيْتِ يُعَزُّونَهَا لَمَّا رَأَوْا مَرْيَمَ قَامَتْ عَاجِلاً وَخَرَجَتْ تَبِعُوهَا قَائِلِينَ: «إِنَّهَا تَذْهَبُ إِلَى الْقَبْرِ لِتَبْكِيَ هُنَاكَ».
\par 32 فَمَرْيَمُ لَمَّا أَتَتْ إِلَى حَيْثُ كَانَ يَسُوعُ وَرَأَتْهُ خَرَّتْ عِنْدَ رِجْلَيْهِ قَائِلَةً لَهُ: «يَا سَيِّدُ لَوْ كُنْتَ هَهُنَا لَمْ يَمُتْ أَخِي».
\par 33 فَلَمَّا رَآهَا يَسُوعُ تَبْكِي وَالْيَهُودُ الَّذِينَ جَاءُوا مَعَهَا يَبْكُونَ انْزَعَجَ بِالرُّوحِ وَاضْطَرَبَ
\par 34 وَقَالَ: «أَيْنَ وَضَعْتُمُوهُ؟» قَالُوا لَهُ: «يَا سَيِّدُ تَعَالَ وَانْظُرْ».
\par 35 بَكَى يَسُوعُ.
\par 36 فَقَالَ الْيَهُودُ: «انْظُرُوا كَيْفَ كَانَ يُحِبُّهُ».
\par 37 وَقَالَ بَعْضٌ مِنْهُمْ: «أَلَمْ يَقْدِرْ هَذَا الَّذِي فَتَحَ عَيْنَيِ الأَعْمَى أَنْ يَجْعَلَ هَذَا أَيْضاً لاَ يَمُوتُ؟».
\par 38 فَانْزَعَجَ يَسُوعُ أَيْضاً فِي نَفْسِهِ وَجَاءَ إِلَى الْقَبْرِ وَكَانَ مَغَارَةً وَقَدْ وُضِعَ عَلَيْهِ حَجَرٌ.
\par 39 قَالَ يَسُوعُ: «ارْفَعُوا الْحَجَرَ». قَالَتْ لَهُ مَرْثَا أُخْتُ الْمَيْتِ: «يَا سَيِّدُ قَدْ أَنْتَنَ لأَنَّ لَهُ أَرْبَعَةَ أَيَّامٍ».
\par 40 قَالَ لَهَا يَسُوعُ: «أَلَمْ أَقُلْ لَكِ: إِنْ آمَنْتِ تَرَيْنَ مَجْدَ اللَّهِ؟».
\par 41 فَرَفَعُوا الْحَجَرَ حَيْثُ كَانَ الْمَيْتُ مَوْضُوعاً وَرَفَعَ يَسُوعُ عَيْنَيْهِ إِلَى فَوْقُ وَقَالَ: «أَيُّهَا الآبُ أَشْكُرُكَ لأَنَّكَ سَمِعْتَ لِي
\par 42 وَأَنَا عَلِمْتُ أَنَّكَ فِي كُلِّ حِينٍ تَسْمَعُ لِي. وَلَكِنْ لأَجْلِ هَذَا الْجَمْعِ الْوَاقِفِ قُلْتُ لِيُؤْمِنُوا أَنَّكَ أَرْسَلْتَنِي».
\par 43 وَلَمَّا قَالَ هَذَا صَرَخَ بِصَوْتٍ عَظِيمٍ: «لِعَازَرُ هَلُمَّ خَارِجاً»
\par 44 فَخَرَجَ الْمَيْتُ وَيَدَاهُ وَرِجْلاَهُ مَرْبُوطَاتٌ بِأَقْمِطَةٍ وَوَجْهُهُ مَلْفُوفٌ بِمِنْدِيلٍ. فَقَالَ لَهُمْ يَسُوعُ: «حُلُّوهُ وَدَعُوهُ يَذْهَبْ».
\par 45 فَكَثِيرُونَ مِنَ الْيَهُودِ الَّذِينَ جَاءُوا إِلَى مَرْيَمَ وَنَظَرُوا مَا فَعَلَ يَسُوعُ آمَنُوا بِهِ.
\par 46 وَأَمَّا قَوْمٌ مِنْهُمْ فَمَضَوْا إِلَى الْفَرِّيسِيِّينَ وَقَالُوا لَهُمْ عَمَّا فَعَلَ يَسُوعُ.
\par 47 فَجَمَعَ رُؤَسَاءُ الْكَهَنَةِ وَالْفَرِّيسِيُّونَ مَجْمَعاً وَقَالُوا: «مَاذَا نَصْنَعُ؟ فَإِنَّ هَذَا الإِنْسَانَ يَعْمَلُ آيَاتٍ كَثِيرَةً.
\par 48 إِنْ تَرَكْنَاهُ هَكَذَا يُؤْمِنُ الْجَمِيعُ بِهِ فَيَأْتِي الرُّومَانِيُّونَ وَيَأْخُذُونَ مَوْضِعَنَا وَأُمَّتَنَا».
\par 49 فَقَالَ لَهُمْ وَاحِدٌ مِنْهُمْ وَهُوَ قَيَافَا كَانَ رَئِيساً لِلْكَهَنَةِ فِي تِلْكَ السَّنَةِ: «أَنْتُمْ لَسْتُمْ تَعْرِفُونَ شَيْئاً
\par 50 ولاَ تُفَكِّرُونَ أَنَّهُ خَيْرٌ لَنَا أَنْ يَمُوتَ إِنْسَانٌ وَاحِدٌ عَنِ الشَّعْبِ وَلاَ تَهْلِكَ الأُمَّةُ كُلُّهَا».
\par 51 وَلَمْ يَقُلْ هَذَا مِنْ نَفْسِهِ بَلْ إِذْ كَانَ رَئِيساً لِلْكَهَنَةِ فِي تِلْكَ السَّنَةِ تَنَبَّأَ أَنَّ يَسُوعَ مُزْمِعٌ أَنْ يَمُوتَ عَنِ الأُمَّةِ
\par 52 وَلَيْسَ عَنِ الأُمَّةِ فَقَطْ بَلْ لِيَجْمَعَ أَبْنَاءَ اللَّهِ الْمُتَفَرِّقِينَ إِلَى وَاحِدٍ.
\par 53 فَمِنْ ذَلِكَ الْيَوْمِ تَشَاوَرُوا لِيَقْتُلُوهُ.
\par 54 فَلَمْ يَكُنْ يَسُوعُ أَيْضاً يَمْشِي بَيْنَ الْيَهُودِ علاَنِيَةً بَلْ مَضَى مِنْ هُنَاكَ إِلَى الْكُورَةِ الْقَرِيبَةِ مِنَ الْبَرِّيَّةِ إِلَى مَدِينَةٍ يُقَالُ لَهَا أَفْرَايِمُ وَمَكَثَ هُنَاكَ مَعَ تلاَمِيذِهِ.
\par 55 وَكَانَ فِصْحُ الْيَهُودِ قَرِيباً. فَصَعِدَ كَثِيرُونَ مِنَ الْكُوَرِ إِلَى أُورُشَلِيمَ قَبْلَ الْفِصْحِ لِيُطَهِّرُوا أَنْفُسَهُمْ.
\par 56 فَكَانُوا يَطْلُبُونَ يَسُوعَ وَيَقُولُونَ فِيمَا بَيْنَهُمْ وَهُمْ وَاقِفُونَ فِي الْهَيْكَلِ: «مَاذَا تَظُنُّونَ؟ هَلْ هُوَ لاَ يَأْتِي إِلَى الْعِيدِ؟»
\par 57 وَكَانَ أَيْضاً رُؤَسَاءُ الْكَهَنَةِ وَالْفَرِّيسِيُّونَ قَدْ أَصْدَرُوا أَمْراً أَنَّهُ إِنْ عَرَفَ أَحَدٌ أَيْنَ هُوَ فَلْيَدُلَّ عَلَيْهِ لِكَيْ يُمْسِكُوهُ.

\chapter{12}

\par 1 ثُمَّ قَبْلَ الْفِصْحِ بِسِتَّةِ أَيَّامٍ أَتَى يَسُوعُ إِلَى بَيْتِ عَنْيَا حَيْثُ كَانَ لِعَازَرُ الْمَيْتُ الَّذِي أَقَامَهُ مِنَ الأَمْوَاتِ.
\par 2 فَصَنَعُوا لَهُ هُنَاكَ عَشَاءً. وَكَانَتْ مَرْثَا تَخْدِمُ وَأَمَّا لِعَازَرُ فَكَانَ أَحَدَ الْمُتَّكِئِينَ مَعَهُ.
\par 3 فَأَخَذَتْ مَرْيَمُ مَناً مِنْ طِيبِ نَارِدِينٍ خَالِصٍ كَثِيرِ الثَّمَنِ وَدَهَنَتْ قَدَمَيْ يَسُوعَ وَمَسَحَتْ قَدَمَيْهِ بِشَعْرِهَا فَامْتَلَأَ الْبَيْتُ مِنْ رَائِحَةِ الطِّيبِ.
\par 4 فَقَالَ وَاحِدٌ مِنْ تلاَمِيذِهِ وَهُوَ يَهُوذَا سِمْعَانُ الإِسْخَرْيُوطِيُّ الْمُزْمِعُ أَنْ يُسَلِّمَهُ:
\par 5 «لِمَاذَا لَمْ يُبَعْ هَذَا الطِّيبُ بِثلاَثَمِئَةِ دِينَارٍ وَيُعْطَ لِلْفُقَرَاءِ؟»
\par 6 قَالَ هَذَا لَيْسَ لأَنَّهُ كَانَ يُبَالِي بِالْفُقَرَاءِ بَلْ لأَنَّهُ كَانَ سَارِقاً وَكَانَ الصُّنْدُوقُ عِنْدَهُ وَكَانَ يَحْمِلُ مَا يُلْقَى فِيهِ.
\par 7 فَقَالَ يَسُوعُ: «اتْرُكُوهَا. إِنَّهَا لِيَوْمِ تَكْفِينِي قَدْ حَفِظَتْهُ
\par 8 لأَنَّ الْفُقَرَاءَ مَعَكُمْ فِي كُلِّ حِينٍ وَأَمَّا أَنَا فَلَسْتُ مَعَكُمْ فِي كُلِّ حِينٍ».
\par 9 فَعَلِمَ جَمْعٌ كَثِيرٌ مِنَ الْيَهُودِ أَنَّهُ هُنَاكَ فَجَاءُوا لَيْسَ لأَجْلِ يَسُوعَ فَقَطْ بَلْ لِيَنْظُرُوا أَيْضاً لِعَازَرَ الَّذِي أَقَامَهُ مِنَ الأَمْوَاتِ.
\par 10 فَتَشَاوَرَ رُؤَسَاءُ الْكَهَنَةِ لِيَقْتُلُوا لِعَازَرَ أَيْضاً
\par 11 لأَنَّ كَثِيرِينَ مِنَ الْيَهُودِ كَانُوا بِسَبَبِهِ يَذْهَبُونَ وَيُؤْمِنُونَ بِيَسُوعَ.
\par 12 وَفِي الْغَدِ سَمِعَ الْجَمْعُ الْكَثِيرُ الَّذِي جَاءَ إِلَى الْعِيدِ أَنَّ يَسُوعَ آتٍ إِلَى أُورُشَلِيمَ
\par 13 فَأَخَذُوا سُعُوفَ النَّخْلِ وَخَرَجُوا لِلِقَائِهِ وَكَانُوا يَصْرُخُونَ: «أُوصَنَّا! مُبَارَكٌ الآتِي بِاسْمِ الرَّبِّ مَلِكُ إِسْرَائِيلَ!»
\par 14 وَوَجَدَ يَسُوعُ جَحْشاً فَجَلَسَ عَلَيْهِ كَمَا هُوَ مَكْتُوبٌ:
\par 15 «لاَ تَخَافِي يَا ابْنَةَ صَِهْيَوْنَ. هُوَذَا مَلِكُكِ يَأْتِي جَالِساً عَلَى جَحْشِ أَتَانٍ».
\par 16 وَهَذِهِ الأُمُورُ لَمْ يَفْهَمْهَا تلاَمِيذُهُ أَوَّلاً وَلَكِنْ لَمَّا تَمَجَّدَ يَسُوعُ حِينَئِذٍ تَذَكَّرُوا أَنَّ هَذِهِ كَانَتْ مَكْتُوبَةً عَنْهُ وَأَنَّهُمْ صَنَعُوا هَذِهِ لَهُ.
\par 17 وَكَانَ الْجَمْعُ الَّذِي مَعَهُ يَشْهَدُ أَنَّهُ دَعَا لِعَازَرَ مِنَ الْقَبْرِ وَأَقَامَهُ مِنَ الأَمْوَاتِ.
\par 18 لِهَذَا أَيْضاً لاَقَاهُ الْجَمْعُ لأَنَّهُمْ سَمِعُوا أَنَّهُ كَانَ قَدْ صَنَعَ هَذِهِ الآيَةَ.
\par 19 فَقَالَ الْفَرِّيسِيُّونَ بَعْضُهُمْ لِبَعْضٍ: «انْظُرُوا! إِنَّكُمْ لاَ تَنْفَعُونَ شَيْئاً! هُوَذَا الْعَالَمُ قَدْ ذَهَبَ وَرَاءَهُ!».
\par 20 وَكَانَ أُنَاسٌ يُونَانِيُّونَ مِنَ الَّذِينَ صَعِدُوا لِيَسْجُدُوا فِي الْعِيدِ.
\par 21 فَتَقَدَّمَ هَؤُلاَءِ إِلَى فِيلُبُّسَ الَّذِي مِنْ بَيْتِ صَيْدَا الْجَلِيلِ وَسَأَلُوهُ: «يَا سَيِّدُ نُرِيدُ أَنْ نَرَى يَسُوعَ»
\par 22 فَأَتَى فِيلُبُّسُ وَقَالَ لأَنْدَرَاوُسَ ثُمَّ قَالَ أَنْدَرَاوُسُ وَفِيلُبُّسُ لِيَسُوعَ.
\par 23 وَأَمَّا يَسُوعُ فَأَجَابَهُمَا: «قَدْ أَتَتِ السَّاعَةُ لِيَتَمَجَّدَ ابْنُ الإِنْسَانِ.
\par 24 اَلْحَقَّ الْحَقَّ أَقُولُ لَكُمْ: إِنْ لَمْ تَقَعْ حَبَّةُ الْحِنْطَةِ فِي الأَرْضِ وَتَمُتْ فَهِيَ تَبْقَى وَحْدَهَا. وَلَكِنْ إِنْ مَاتَتْ تَأْتِي بِثَمَرٍ كَثِيرٍ.
\par 25 مَنْ يُحِبُّ نَفْسَهُ يُهْلِكُهَا وَمَنْ يُبْغِضُ نَفْسَهُ فِي هَذَا الْعَالَمِ يَحْفَظُهَا إِلَى حَيَاةٍ أَبَدِيَّةٍ.
\par 26 إِنْ كَانَ أَحَدٌ يَخْدِمُنِي فَلْيَتْبَعْنِي وَحَيْثُ أَكُونُ أَنَا هُنَاكَ أَيْضاً يَكُونُ خَادِمِي. وَإِنْ كَانَ أَحَدٌ يَخْدِمُنِي يُكْرِمُهُ الآبُ.
\par 27 اَلآنَ نَفْسِي قَدِ اضْطَرَبَتْ. وَمَاذَا أَقُولُ؟ أَيُّهَا الآبُ نَجِّنِي مِنْ هَذِهِ السَّاعَةِ. وَلَكِنْ لأَجْلِ هَذَا أَتَيْتُ إِلَى هَذِهِ السَّاعَةِ.
\par 28 أَيُّهَا الآبُ مَجِّدِ اسْمَكَ». فَجَاءَ صَوْتٌ مِنَ السَّمَاءِ: «مَجَّدْتُ وَأُمَجِّدُ أَيْضاً».
\par 29 فَالْجَمْعُ الَّذِي كَانَ وَاقِفاً وَسَمِعَ قَالَ: «قَدْ حَدَثَ رَعْدٌ». وَآخَرُونَ قَالُوا: «قَدْ كَلَّمَهُ ملاَكٌ».
\par 30 أَجَابَ يَسُوعُ: «لَيْسَ مِنْ أَجْلِي صَارَ هَذَا الصَّوْتُ بَلْ مِنْ أَجْلِكُمْ.
\par 31 اَلآنَ دَيْنُونَةُ هَذَا الْعَالَمِ. اَلآنَ يُطْرَحُ رَئِيسُ هَذَا الْعَالَمِ خَارِجاً.
\par 32 وَأَنَا إِنِ ارْتَفَعْتُ عَنِ الأَرْضِ أَجْذِبُ إِلَيَّ الْجَمِيعَ».
\par 33 قَالَ هَذَا مُشِيراً إِلَى أَيَّةِ مِيتَةٍ كَانَ مُزْمِعاً أَنْ يَمُوتَ.
\par 34 فَأَجَابَهُ الْجَمْعُ: «نَحْنُ سَمِعْنَا مِنَ النَّامُوسِ أَنَّ الْمَسِيحَ يَبْقَى إِلَى الأَبَدِ فَكَيْفَ تَقُولُ أَنْتَ إِنَّهُ يَنْبَغِي أَنْ يَرْتَفِعَ ابْنُ الإِنْسَانِ؟ مَنْ هُوَ هَذَا ابْنُ الإِنْسَانِ؟»
\par 35 فَقَالَ لَهُمْ يَسُوعُ: «النُّورُ مَعَكُمْ زَمَاناً قَلِيلاً بَعْدُ فَسِيرُوا مَا دَامَ لَكُمُ النُّورُ لِئَلَّا يُدْرِكَكُمُ الظّلاَمُ. وَالَّذِي يَسِيرُ فِي الظّلاَمِ لاَ يَعْلَمُ إِلَى أَيْنَ يَذْهَبُ.
\par 36 مَا دَامَ لَكُمُ النُّورُ آمِنُوا بِالنُّورِ لِتَصِيرُوا أَبْنَاءَ النُّورِ». تَكَلَّمَ يَسُوعُ بِهَذَا ثُمَّ مَضَى وَاخْتَفَى عَنْهُمْ.
\par 37 وَمَعَ أَنَّهُ كَانَ قَدْ صَنَعَ أَمَامَهُمْ آيَاتٍ هَذَا عَدَدُهَا لَمْ يُؤْمِنُوا بِهِ
\par 38 لِيَتِمَّ قَوْلُ إِشَعْيَاءَ النَّبِيِّ: «يَا رَبُّ مَنْ صَدَّقَ خَبَرَنَا وَلِمَنِ اسْتُعْلِنَتْ ذِرَاعُ الرَّبِّ؟»
\par 39 لِهَذَا لَمْ يَقْدِرُوا أَنْ يُؤْمِنُوا. لأَنَّ إِشَعْيَاءَ قَالَ أَيْضاً:
\par 40 «قَدْ أَعْمَى عُيُونَهُمْ وَأَغْلَظَ قُلُوبَهُمْ لِئَلَّا يُبْصِرُوا بِعُيُونِهِمْ وَيَشْعُرُوا بِقُلُوبِهِمْ وَيَرْجِعُوا فَأَشْفِيَهُمْ».
\par 41 قَالَ إِشَعْيَاءُ هَذَا حِينَ رَأَى مَجْدَهُ وَتَكَلَّمَ عَنْهُ.
\par 42 وَلَكِنْ مَعَ ذَلِكَ آمَنَ بِهِ كَثِيرُونَ مِنَ الرُّؤَسَاءِ أَيْضاً غَيْرَ أَنَّهُمْ لِسَبَبِ الْفَرِّيسِيِّينَ لَمْ يَعْتَرِفُوا بِهِ لِئَلَّا يَصِيرُوا خَارِجَ الْمَجْمَعِ
\par 43 لأَنَّهُمْ أَحَبُّوا مَجْدَ النَّاسِ أَكْثَرَ مِنْ مَجْدِ اللَّهِ.
\par 44 فَنَادَى يَسُوعُ: «الَّذِي يُؤْمِنُ بِي لَيْسَ يُؤْمِنُ بِي بَلْ بِالَّذِي أَرْسَلَنِي.
\par 45 وَالَّذِي يَرَانِي يَرَى الَّذِي أَرْسَلَنِي.
\par 46 أَنَا قَدْ جِئْتُ نُوراً إِلَى الْعَالَمِ حَتَّى كُلُّ مَنْ يُؤْمِنُ بِي لاَ يَمْكُثُ فِي الظُّلْمَةِ.
\par 47 وَإِنْ سَمِعَ أَحَدٌ كلاَمِي وَلَمْ يُؤْمِنْ فَأَنَا لاَ أَدِينُهُ لأَنِّي لَمْ آتِ لأَدِينَ الْعَالَمَ بَلْ لِأُخَلِّصَ الْعَالَمَ.
\par 48 مَنْ رَذَلَنِي وَلَمْ يَقْبَلْ كلاَمِي فَلَهُ مَنْ يَدِينُهُ. اَلْكلاَمُ الَّذِي تَكَلَّمْتُ بِهِ هُوَ يَدِينُهُ فِي الْيَوْمِ الأَخِيرِ
\par 49 لأَنِّي لَمْ أَتَكَلَّمْ مِنْ نَفْسِي لَكِنَّ الآبَ الَّذِي أَرْسَلَنِي هُوَ أَعْطَانِي وَصِيَّةً: مَاذَا أَقُولُ وَبِمَاذَا أَتَكَلَّمُ.
\par 50 وَأَنَا أَعْلَمُ أَنَّ وَصِيَّتَهُ هِيَ حَيَاةٌ أَبَدِيَّةٌ. فَمَا أَتَكَلَّمُ أَنَا بِهِ فَكَمَا قَالَ لِي الآبُ هَكَذَا أَتَكَلَّمُ».

\chapter{13}

\par 1 أَمَّا يَسُوعُ قَبْلَ عِيدِ الْفِصْحِ وَهُوَ عَالِمٌ أَنَّ سَاعَتَهُ قَدْ جَاءَتْ لِيَنْتَقِلَ مِنْ هَذَا الْعَالَمِ إِلَى الآبِ إِذْ كَانَ قَدْ أَحَبَّ خَاصَّتَهُ الَّذِينَ فِي الْعَالَمِ أَحَبَّهُمْ إِلَى الْمُنْتَهَى.
\par 2 فَحِينَ كَانَ الْعَشَاءُ وَقَدْ أَلْقَى الشَّيْطَانُ فِي قَلْبِ يَهُوذَا سِمْعَانَ الإِسْخَرْيُوطِيِّ أَنْ يُسَلِّمَهُ
\par 3 يَسُوعُ وَهُوَ عَالِمٌ أَنَّ الآبَ قَدْ دَفَعَ كُلَّ شَيْءٍ إِلَى يَدَيْهِ وَأَنَّهُ مِنْ عِنْدِ اللَّهِ خَرَجَ وَإِلَى اللَّهِ يَمْضِي
\par 4 قَامَ عَنِ الْعَشَاءِ وَخَلَعَ ثِيَابَهُ وَأَخَذَ مِنْشَفَةً وَاتَّزَرَ بِهَا
\par 5 ثُمَّ صَبَّ مَاءً فِي مِغْسَلٍ وَابْتَدَأَ يَغْسِلُ أَرْجُلَ التّلاَمِيذِ وَيَمْسَحُهَا بِالْمِنْشَفَةِ الَّتِي كَانَ مُتَّزِراً بِهَا.
\par 6 فَجَاءَ إِلَى سِمْعَانَ بُطْرُسَ. فَقَالَ لَهُ ذَاكَ: «يَا سَيِّدُ أَنْتَ تَغْسِلُ رِجْلَيَّ!»
\par 7 أَجَابَ يَسُوعُ: «لَسْتَ تَعْلَمُ أَنْتَ الآنَ مَا أَنَا أَصْنَعُ وَلَكِنَّكَ سَتَفْهَمُ فِيمَا بَعْدُ».
\par 8 قَالَ لَهُ بُطْرُسُ: «لَنْ تَغْسِلَ رِجْلَيَّ أَبَداً!» أَجَابَهُ يَسُوعُ: «إِنْ كُنْتُ لاَ أَغْسِلُكَ فَلَيْسَ لَكَ مَعِي نَصِيبٌ».
\par 9 قَالَ لَهُ سِمْعَانُ بُطْرُسُ: «يَا سَيِّدُ لَيْسَ رِجْلَيَّ فَقَطْ بَلْ أَيْضاً يَدَيَّ وَرَأْسِي».
\par 10 قَالَ لَهُ يَسُوعُ: «الَّذِي قَدِ اغْتَسَلَ لَيْسَ لَهُ حَاجَةٌ إِلاَّ إِلَى غَسْلِ رِجْلَيْهِ بَلْ هُوَ طَاهِرٌ كُلُّهُ. وَأَنْتُمْ طَاهِرُونَ وَلَكِنْ لَيْسَ كُلُّكُمْ».
\par 11 لأَنَّهُ عَرَفَ مُسَلِّمَهُ لِذَلِكَ قَالَ: «لَسْتُمْ كُلُّكُمْ طَاهِرِينَ».
\par 12 فَلَمَّا كَانَ قَدْ غَسَلَ أَرْجُلَهُمْ وَأَخَذَ ثِيَابَهُ وَاتَّكَأَ أَيْضاً قَالَ لَهُمْ: «أَتَفْهَمُونَ مَا قَدْ صَنَعْتُ بِكُمْ؟
\par 13 أَنْتُمْ تَدْعُونَنِي مُعَلِّماً وَسَيِّداً وَحَسَناً تَقُولُونَ لأَنِّي أَنَا كَذَلِكَ.
\par 14 فَإِنْ كُنْتُ وَأَنَا السَّيِّدُ وَالْمُعَلِّمُ قَدْ غَسَلْتُ أَرْجُلَكُمْ فَأَنْتُمْ يَجِبُ عَلَيْكُمْ أَنْ يَغْسِلَ بَعْضُكُمْ أَرْجُلَ بَعْضٍ
\par 15 لأَنِّي أَعْطَيْتُكُمْ مِثَالاً حَتَّى كَمَا صَنَعْتُ أَنَا بِكُمْ تَصْنَعُونَ أَنْتُمْ أَيْضاً.
\par 16 اَلْحَقَّ الْحَقَّ أَقُولُ لَكُمْ: إِنَّهُ لَيْسَ عَبْدٌ أَعْظَمَ مِنْ سَيِّدِهِ وَلاَ رَسُولٌ أَعْظَمَ مِنْ مُرْسِلِهِ.
\par 17 إِنْ عَلِمْتُمْ هَذَا فَطُوبَاكُمْ إِنْ عَمِلْتُمُوهُ.
\par 18 لَسْتُ أَقُولُ عَنْ جَمِيعِكُمْ. أَنَا أَعْلَمُ الَّذِينَ اخْتَرْتُهُمْ. لَكِنْ لِيَتِمَّ الْكِتَابُ: اَلَّذِي يَأْكُلُ مَعِي الْخُبْزَ رَفَعَ عَلَيَّ عَقِبَهُ.
\par 19 أَقُولُ لَكُمُ الآنَ قَبْلَ أَنْ يَكُونَ حَتَّى مَتَى كَانَ تُؤْمِنُونَ أَنِّي أَنَا هُوَ.
\par 20 اَلْحَقَّ الْحَقَّ أَقُولُ لَكُمُ: الَّذِي يَقْبَلُ مَنْ أُرْسِلُهُ يَقْبَلُنِي وَالَّذِي يَقْبَلُنِي يَقْبَلُ الَّذِي أَرْسَلَنِي».
\par 21 لَمَّا قَالَ يَسُوعُ هَذَا اضْطَرَبَ بِالرُّوحِ وَشَهِدَ وَقَالَ: «الْحَقَّ الْحَقَّ أَقُولُ لَكُمْ: إِنَّ وَاحِداً مِنْكُمْ سَيُسَلِّمُنِي».
\par 22 فَكَانَ التّلاَمِيذُ يَنْظُرُونَ بَعْضُهُمْ إِلَى بَعْضٍ وَهُمْ مُحْتَارُونَ فِي مَنْ قَالَ عَنْهُ.
\par 23 وَكَانَ مُتَّكِئاً فِي حِضْنِ يَسُوعَ وَاحِدٌ مِنْ تلاَمِيذِهِ كَانَ يَسُوعُ يُحِبُّهُ.
\par 24 فَأَوْمَأَ إِلَيْهِ سِمْعَانُ بُطْرُسُ أَنْ يَسْأَلَ مَنْ عَسَى أَنْ يَكُونَ الَّذِي قَالَ عَنْهُ.
\par 25 فَاتَّكَأَ ذَاكَ عَلَى صَدْرِ يَسُوعَ وَقَالَ لَهُ: «يَا سَيِّدُ مَنْ هُوَ؟»
\par 26 أَجَابَ يَسُوعُ: «هُوَ ذَاكَ الَّذِي أَغْمِسُ أَنَا اللُّقْمَةَ وَأُعْطِيهِ». فَغَمَسَ اللُّقْمَةَ وَأَعْطَاهَا لِيَهُوذَا سِمْعَانَ الإِسْخَرْيُوطِيِّ.
\par 27 فَبَعْدَ اللُّقْمَةِ دَخَلَهُ الشَّيْطَانُ. فَقَالَ لَهُ يَسُوعُ: «مَا أَنْتَ تَعْمَلُهُ فَاعْمَلْهُ بِأَكْثَرِ سُرْعَةٍ».
\par 28 وَأَمَّا هَذَا فَلَمْ يَفْهَمْ أَحَدٌ مِنَ الْمُتَّكِئِينَ لِمَاذَا كَلَّمَهُ بِه
\par 29 لأَنَّ قَوْماً إِذْ كَانَ الصُّنْدُوقُ مَعَ يَهُوذَا ظَنُّوا أَنَّ يَسُوعَ قَالَ لَهُ: اشْتَرِ مَا نَحْتَاجُ إِلَيْهِ لِلْعِيدِ أَوْ أَنْ يُعْطِيَ شَيْئاً لِلْفُقَرَاءِ.
\par 30 فَذَاكَ لَمَّا أَخَذَ اللُّقْمَةَ خَرَجَ لِلْوَقْتِ. وَكَانَ لَيْلاً.
\par 31 فَلَمَّا خَرَجَ قَالَ يَسُوعُ: «الآنَ تَمَجَّدَ ابْنُ الإِنْسَانِ وَتَمَجَّدَ اللَّهُ فِيهِ.
\par 32 إِنْ كَانَ اللَّهُ قَدْ تَمَجَّدَ فِيهِ فَإِنَّ اللَّهَ سَيُمَجِّدُهُ فِي ذَاتِهِ وَيُمَجِّدُهُ سَرِيعاً.
\par 33 يَا أَوْلاَدِي أَنَا مَعَكُمْ زَمَاناً قَلِيلاً بَعْدُ. سَتَطْلُبُونَنِي وَكَمَا قُلْتُ لِلْيَهُودِ: حَيْثُ أَذْهَبُ أَنَا لاَ تَقْدِرُونَ أَنْتُمْ أَنْ تَأْتُوا أَقُولُ لَكُمْ أَنْتُمُ الآنَ.
\par 34 وَصِيَّةً جَدِيدَةً أَنَا أُعْطِيكُمْ: أَنْ تُحِبُّوا بَعْضُكُمْ بَعْضاً. كَمَا أَحْبَبْتُكُمْ أَنَا تُحِبُّونَ أَنْتُمْ أَيْضاً بَعْضُكُمْ بَعْضاً.
\par 35 بِهَذَا يَعْرِفُ الْجَمِيعُ أَنَّكُمْ تلاَمِيذِي: إِنْ كَانَ لَكُمْ حُبٌّ بَعْضاً لِبَعْضٍ».
\par 36 قَالَ لَهُ سِمْعَانُ بُطْرُسُ: «يَا سَيِّدُ إِلَى أَيْنَ تَذْهَبُ؟» أَجَابَهُ يَسُوعُ: «حَيْثُ أَذْهَبُ لاَ تَقْدِرُ الآنَ أَنْ تَتْبَعَنِي وَلَكِنَّكَ سَتَتْبَعُنِي أَخِيراً».
\par 37 قَالَ لَهُ بُطْرُسُ: «يَا سَيِّدُ لِمَاذَا لاَ أَقْدِرُ أَنْ أَتْبَعَكَ الآنَ؟ إِنِّي أَضَعُ نَفْسِي عَنْكَ».
\par 38 أَجَابَهُ يَسُوعُ: «أَتَضَعُ نَفْسَكَ عَنِّي؟ اَلْحَقَّ الْحَقَّ أَقُولُ لَكَ: لاَ يَصِيحُ الدِّيكُ حَتَّى تُنْكِرَنِي ثلاَثَ مَرَّاتٍ».

\chapter{14}

\par 1 «لاَ تَضْطَرِبْ قُلُوبُكُمْ. أَنْتُمْ تُؤْمِنُونَ بِاللَّهِ فَآمِنُوا بِي.
\par 2 فِي بَيْتِ أَبِي مَنَازِلُ كَثِيرَةٌ وَإِلاَّ فَإِنِّي كُنْتُ قَدْ قُلْتُ لَكُمْ. أَنَا أَمْضِي لِأُعِدَّ لَكُمْ مَكَاناً
\par 3 وَإِنْ مَضَيْتُ وَأَعْدَدْتُ لَكُمْ مَكَاناً آتِي أَيْضاً وَآخُذُكُمْ إِلَيَّ حَتَّى حَيْثُ أَكُونُ أَنَا تَكُونُونَ أَنْتُمْ أَيْضاً
\par 4 وَتَعْلَمُونَ حَيْثُ أَنَا أَذْهَبُ وَتَعْلَمُونَ الطَّرِيقَ».
\par 5 قَالَ لَهُ تُومَا: «يَا سَيِّدُ لَسْنَا نَعْلَمُ أَيْنَ تَذْهَبُ فَكَيْفَ نَقْدِرُ أَنْ نَعْرِفَ الطَّرِيقَ؟»
\par 6 قَالَ لَهُ يَسُوعُ: «أَنَا هُوَ الطَّرِيقُ وَالْحَقُّ وَالْحَيَاةُ. لَيْسَ أَحَدٌ يَأْتِي إِلَى الآبِ إِلاَّ بِي.
\par 7 لَوْ كُنْتُمْ قَدْ عَرَفْتُمُونِي لَعَرَفْتُمْ أَبِي أَيْضاً. وَمِنَ الآنَ تَعْرِفُونَهُ وَقَدْ رَأَيْتُمُوهُ».
\par 8 قَالَ لَهُ فِيلُبُّسُ: «يَا سَيِّدُ أَرِنَا الآبَ وَكَفَانَا».
\par 9 قَالَ لَهُ يَسُوعُ: «أَنَا مَعَكُمْ زَمَاناً هَذِهِ مُدَّتُهُ وَلَمْ تَعْرِفْنِي يَا فِيلُبُّسُ! اَلَّذِي رَآنِي فَقَدْ رَأَى الآبَ فَكَيْفَ تَقُولُ أَنْتَ أَرِنَا الآبَ؟
\par 10 أَلَسْتَ تُؤْمِنُ أَنِّي أَنَا فِي الآبِ وَالآبَ فِيَّ؟ الْكلاَمُ الَّذِي أُكَلِّمُكُمْ بِهِ لَسْتُ أَتَكَلَّمُ بِهِ مِنْ نَفْسِي لَكِنَّ الآبَ الْحَالَّ فِيَّ هُوَ يَعْمَلُ الأَعْمَالَ.
\par 11 صَدِّقُونِي أَنِّي فِي الآبِ وَالآبَ فِيَّ وَإِلاَّ فَصَدِّقُونِي لِسَبَبِ الأَعْمَالِ نَفْسِهَا.
\par 12 اَلْحَقَّ الْحَقَّ أَقُولُ لَكُمْ: مَنْ يُؤْمِنُ بِي فَالأَعْمَالُ الَّتِي أَنَا أَعْمَلُهَا يَعْمَلُهَا هُوَ أَيْضاً وَيَعْمَلُ أَعْظَمَ مِنْهَا لأَنِّي مَاضٍ إِلَى أَبِي.
\par 13 وَمَهْمَا سَأَلْتُمْ بِاسْمِي فَذَلِكَ أَفْعَلُهُ لِيَتَمَجَّدَ الآبُ بِالاِبْنِ.
\par 14 إِنْ سَأَلْتُمْ شَيْئاً بِاسْمِي فَإِنِّي أَفْعَلُهُ.
\par 15 «إِنْ كُنْتُمْ تُحِبُّونَنِي فَاحْفَظُوا وَصَايَايَ
\par 16 وَأَنَا أَطْلُبُ مِنَ الآبِ فَيُعْطِيكُمْ مُعَزِّياً آخَرَ لِيَمْكُثَ مَعَكُمْ إِلَى الأَبَدِ
\par 17 رُوحُ الْحَقِّ الَّذِي لاَ يَسْتَطِيعُ الْعَالَمُ أَنْ يَقْبَلَهُ لأَنَّهُ لاَ يَرَاهُ وَلاَ يَعْرِفُهُ وَأَمَّا أَنْتُمْ فَتَعْرِفُونَهُ لأَنَّهُ مَاكِثٌ مَعَكُمْ وَيَكُونُ فِيكُمْ.
\par 18 لاَ أَتْرُكُكُمْ يَتَامَى. إِنِّي آتِي إِلَيْكُمْ.
\par 19 بَعْدَ قَلِيلٍ لاَ يَرَانِي الْعَالَمُ أَيْضاً وَأَمَّا أَنْتُمْ فَتَرَوْنَنِي. إِنِّي أَنَا حَيٌّ فَأَنْتُمْ سَتَحْيَوْنَ.
\par 20 فِي ذَلِكَ الْيَوْمِ تَعْلَمُونَ أَنِّي أَنَا فِي أَبِي وَأَنْتُمْ فِيَّ وَأَنَا فِيكُمْ.
\par 21 اَلَّذِي عِنْدَهُ وَصَايَايَ وَيَحْفَظُهَا فَهُوَ الَّذِي يُحِبُّنِي وَالَّذِي يُحِبُّنِي يُحِبُّهُ أَبِي وَأَنَا أُحِبُّهُ وَأُظْهِرُ لَهُ ذَاتِي».
\par 22 قَالَ لَهُ يَهُوذَا لَيْسَ الإِسْخَرْيُوطِيَّ: «يَا سَيِّدُ مَاذَا حَدَثَ حَتَّى إِنَّكَ مُزْمِعٌ أَنْ تُظْهِرَ ذَاتَكَ لَنَا وَلَيْسَ لِلْعَالَمِ؟»
\par 23 أَجَابَ يَسُوعُ: «إِنْ أَحَبَّنِي أَحَدٌ يَحْفَظْ كلاَمِي وَيُحِبُّهُ أَبِي وَإِلَيْهِ نَأْتِي وَعِنْدَهُ نَصْنَعُ مَنْزِلاً.
\par 24 اَلَّذِي لاَ يُحِبُّنِي لاَ يَحْفَظُ كلاَمِي. وَالْكلاَمُ الَّذِي تَسْمَعُونَهُ لَيْسَ لِي بَلْ لِلآبِ الَّذِي أَرْسَلَنِي.
\par 25 بِهَذَا كَلَّمْتُكُمْ وَأَنَا عِنْدَكُمْ.
\par 26 وَأَمَّا الْمُعَزِّي الرُّوحُ الْقُدُسُ الَّذِي سَيُرْسِلُهُ الآبُ بِاسْمِي فَهُوَ يُعَلِّمُكُمْ كُلَّ شَيْءٍ وَيُذَكِّرُكُمْ بِكُلِّ مَا قُلْتُهُ لَكُمْ.
\par 27 «سلاَماً أَتْرُكُ لَكُمْ. سلاَمِي أُعْطِيكُمْ. لَيْسَ كَمَا يُعْطِي الْعَالَمُ أُعْطِيكُمْ أَنَا. لاَ تَضْطَرِبْ قُلُوبُكُمْ وَلاَ تَرْهَبْ.
\par 28 سَمِعْتُمْ أَنِّي قُلْتُ لَكُمْ أَنَا أَذْهَبُ ثُمَّ آتِي إِلَيْكُمْ. لَوْ كُنْتُمْ تُحِبُّونَنِي لَكُنْتُمْ تَفْرَحُونَ لأَنِّي قُلْتُ أَمْضِي إِلَى الآبِ لأَنَّ أَبِي أَعْظَمُ مِنِّي.
\par 29 وَقُلْتُ لَكُمُ الآنَ قَبْلَ أَنْ يَكُونَ حَتَّى مَتَى كَانَ تُؤْمِنُونَ.
\par 30 لاَ أَتَكَلَّمُ أَيْضاً مَعَكُمْ كَثِيراً لأَنَّ رَئِيسَ هَذَا الْعَالَمِ يَأْتِي وَلَيْسَ لَهُ فِيَّ شَيْءٌ.
\par 31 وَلَكِنْ لِيَفْهَمَ الْعَالَمُ أَنِّي أُحِبُّ الآبَ وَكَمَا أَوْصَانِي الآبُ هَكَذَا أَفْعَلُ. قُومُوا نَنْطَلِقْ مِنْ هَهُنَا».

\chapter{15}

\par 1 «أَنَا الْكَرْمَةُ الْحَقِيقِيَّةُ وَأَبِي الْكَرَّامُ.
\par 2 كُلُّ غُصْنٍ فِيَّ لاَ يَأْتِي بِثَمَرٍ يَنْزِعُهُ وَكُلُّ مَا يَأْتِي بِثَمَرٍ يُنَقِّيهِ لِيَأْتِيَ بِثَمَرٍ أَكْثَرَ.
\par 3 أَنْتُمُ الآنَ أَنْقِيَاءُ لِسَبَبِ الْكلاَمِ الَّذِي كَلَّمْتُكُمْ بِهِ.
\par 4 اُثْبُتُوا فِيَّ وَأَنَا فِيكُمْ. كَمَا أَنَّ الْغُصْنَ لاَ يَقْدِرُ أَنْ يَأْتِيَ بِثَمَرٍ مِنْ ذَاتِهِ إِنْ لَمْ يَثْبُتْ فِي الْكَرْمَةِ كَذَلِكَ أَنْتُمْ أَيْضاً إِنْ لَمْ تَثْبُتُوا فِيَّ.
\par 5 أَنَا الْكَرْمَةُ وَأَنْتُمُ الأَغْصَانُ. الَّذِي يَثْبُتُ فِيَّ وَأَنَا فِيهِ هَذَا يَأْتِي بِثَمَرٍ كَثِيرٍ لأَنَّكُمْ بِدُونِي لاَ تَقْدِرُونَ أَنْ تَفْعَلُوا شَيْئاً.
\par 6 إِنْ كَانَ أَحَدٌ لاَ يَثْبُتُ فِيَّ يُطْرَحُ خَارِجاً كَالْغُصْنِ فَيَجِفُّ وَيَجْمَعُونَهُ وَيَطْرَحُونَهُ فِي النَّارِ فَيَحْتَرِقُ.
\par 7 إِنْ ثَبَتُّمْ فِيَّ وَثَبَتَ كلاَمِي فِيكُمْ تَطْلُبُونَ مَا تُرِيدُونَ فَيَكُونُ لَكُمْ.
\par 8 بِهَذَا يَتَمَجَّدُ أَبِي أَنْ تَأْتُوا بِثَمَرٍ كَثِيرٍ فَتَكُونُونَ تلاَمِيذِي.
\par 9 كَمَا أَحَبَّنِي الآبُ كَذَلِكَ أَحْبَبْتُكُمْ أَنَا. اُثْبُتُوا فِي مَحَبَّتِي.
\par 10 إِنْ حَفِظْتُمْ وَصَايَايَ تَثْبُتُونَ فِي مَحَبَّتِي كَمَا أَنِّي أَنَا قَدْ حَفِظْتُ وَصَايَا أَبِي وَأَثْبُتُ فِي مَحَبَّتِهِ.
\par 11 كَلَّمْتُكُمْ بِهَذَا لِكَيْ يَثْبُتَ فَرَحِي فِيكُمْ وَيُكْمَلَ فَرَحُكُمْ.
\par 12 «هَذِهِ هِيَ وَصِيَّتِي أَنْ تُحِبُّوا بَعْضُكُمْ بَعْضاً كَمَا أَحْبَبْتُكُمْ.
\par 13 لَيْسَ لأَحَدٍ حُبٌّ أَعْظَمُ مِنْ هَذَا أَنْ يَضَعَ أَحَدٌ نَفْسَهُ لأَجْلِ أَحِبَّائِهِ.
\par 14 أَنْتُمْ أَحِبَّائِي إِنْ فَعَلْتُمْ مَا أُوصِيكُمْ بِهِ.
\par 15 لاَ أَعُودُ أُسَمِّيكُمْ عَبِيداً لأَنَّ الْعَبْدَ لاَ يَعْلَمُ مَا يَعْمَلُ سَيِّدُهُ لَكِنِّي قَدْ سَمَّيْتُكُمْ أَحِبَّاءَ لأَنِّي أَعْلَمْتُكُمْ بِكُلِّ مَا سَمِعْتُهُ مِنْ أَبِي.
\par 16 لَيْسَ أَنْتُمُ اخْتَرْتُمُونِي بَلْ أَنَا اخْتَرْتُكُمْ وَأَقَمْتُكُمْ لِتَذْهَبُوا وَتَأْتُوا بِثَمَرٍ وَيَدُومَ ثَمَرُكُمْ لِكَيْ يُعْطِيَكُمُ الآبُ كُلَّ مَا طَلَبْتُمْ بِاسْمِي.
\par 17 بِهَذَا أُوصِيكُمْ حَتَّى تُحِبُّوا بَعْضُكُمْ بَعْضاً.
\par 18 «إِنْ كَانَ الْعَالَمُ يُبْغِضُكُمْ فَاعْلَمُوا أَنَّهُ قَدْ أَبْغَضَنِي قَبْلَكُمْ.
\par 19 لَوْ كُنْتُمْ مِنَ الْعَالَمِ لَكَانَ الْعَالَمُ يُحِبُّ خَاصَّتَهُ. وَلَكِنْ لأَنَّكُمْ لَسْتُمْ مِنَ الْعَالَمِ بَلْ أَنَا اخْتَرْتُكُمْ مِنَ الْعَالَمِ لِذَلِكَ يُبْغِضُكُمُ الْعَالَمُ.
\par 20 اُذْكُرُوا الْكلاَمَ الَّذِي قُلْتُهُ لَكُمْ: لَيْسَ عَبْدٌ أَعْظَمَ مِنْ سَيِّدِهِ. إِنْ كَانُوا قَدِ اضْطَهَدُونِي فَسَيَضْطَهِدُونَكُمْ وَإِنْ كَانُوا قَدْ حَفِظُوا كلاَمِي فَسَيَحْفَظُونَ كلاَمَكُمْ.
\par 21 لَكِنَّهُمْ إِنَّمَا يَفْعَلُونَ بِكُمْ هَذَا كُلَّهُ مِنْ أَجْلِ اسْمِي لأَنَّهُمْ لاَ يَعْرِفُونَ الَّذِي أَرْسَلَنِي.
\par 22 لَوْ لَمْ أَكُنْ قَدْ جِئْتُ وَكَلَّمْتُهُمْ لَمْ تَكُنْ لَهُمْ خَطِيَّةٌ وَأَمَّا الآنَ فَلَيْسَ لَهُمْ عُذْرٌ فِي خَطِيَّتِهِمْ.
\par 23 اَلَّذِي يُبْغِضُنِي يُبْغِضُ أَبِي أَيْضاً.
\par 24 لَوْ لَمْ أَكُنْ قَدْ عَمِلْتُ بَيْنَهُمْ أَعْمَالاً لَمْ يَعْمَلْهَا أَحَدٌ غَيْرِي لَمْ تَكُنْ لَهُمْ خَطِيَّةٌ وَأَمَّا الآنَ فَقَدْ رَأَوْا وَأَبْغَضُونِي أَنَا وَأَبِي.
\par 25 لَكِنْ لِكَيْ تَتِمَّ الْكَلِمَةُ الْمَكْتُوبَةُ فِي نَامُوسِهِمْ: إِنَّهُمْ أَبْغَضُونِي بِلاَ سَبَبٍ.
\par 26 «وَمَتَى جَاءَ الْمُعَزِّي الَّذِي سَأُرْسِلُهُ أَنَا إِلَيْكُمْ مِنَ الآبِ رُوحُ الْحَقِّ الَّذِي مِنْ عِنْدِ الآبِ يَنْبَثِقُ فَهُوَ يَشْهَدُ لِي.
\par 27 وَتَشْهَدُونَ أَنْتُمْ أَيْضاً لأَنَّكُمْ مَعِي مِنَ الاِبْتِدَاءِ».

\chapter{16}

\par 1 «قَدْ كَلَّمْتُكُمْ بِهَذَا لِكَيْ لاَ تَعْثُرُوا.
\par 2 سَيُخْرِجُونَكُمْ مِنَ الْمَجَامِعِ بَلْ تَأْتِي سَاعَةٌ فِيهَا يَظُنُّ كُلُّ مَنْ يَقْتُلُكُمْ أَنَّهُ يُقَدِّمُ خِدْمَةً لِلَّهِ.
\par 3 وَسَيَفْعَلُونَ هَذَا بِكُمْ لأَنَّهُمْ لَمْ يَعْرِفُوا الآبَ وَلاَ عَرَفُونِي.
\par 4 لَكِنِّي قَدْ كَلَّمْتُكُمْ بِهَذَا حَتَّى إِذَا جَاءَتِ السَّاعَةُ تَذْكُرُونَ أَنِّي أَنَا قُلْتُهُ لَكُمْ. وَلَمْ أَقُلْ لَكُمْ مِنَ الْبِدَايَةِ لأَنِّي كُنْتُ مَعَكُمْ.
\par 5 وَأَمَّا الآنَ فَأَنَا مَاضٍ إِلَى الَّذِي أَرْسَلَنِي وَلَيْسَ أَحَدٌ مِنْكُمْ يَسْأَلُنِي أَيْنَ تَمْضِي.
\par 6 لَكِنْ لأَنِّي قُلْتُ لَكُمْ هَذَا قَدْ مَلَأَ الْحُزْنُ قُلُوبَكُمْ.
\par 7 لَكِنِّي أَقُولُ لَكُمُ الْحَقَّ إِنَّهُ خَيْرٌ لَكُمْ أَنْ أَنْطَلِقَ لأَنَّهُ إِنْ لَمْ أَنْطَلِقْ لاَ يَأْتِيكُمُ الْمُعَزِّي وَلَكِنْ إِنْ ذَهَبْتُ أُرْسِلُهُ إِلَيْكُمْ.
\par 8 وَمَتَى جَاءَ ذَاكَ يُبَكِّتُ الْعَالَمَ عَلَى خَطِيَّةٍ وَعَلَى بِرٍّ وَعَلَى دَيْنُونَةٍ.
\par 9 أَمَّا عَلَى خَطِيَّةٍ فَلأَنَّهُمْ لاَ يُؤْمِنُونَ بِي.
\par 10 وَأَمَّا عَلَى بِرٍّ فَلأَنِّي ذَاهِبٌ إِلَى أَبِي وَلاَ تَرَوْنَنِي أَيْضاً.
\par 11 وَأَمَّا عَلَى دَيْنُونَةٍ فَلأَنَّ رَئِيسَ هَذَا الْعَالَمِ قَدْ دِينَ.
\par 12 «إِنَّ لِي أُمُوراً كَثِيرَةً أَيْضاً لأَقُولَ لَكُمْ وَلَكِنْ لاَ تَسْتَطِيعُونَ أَنْ تَحْتَمِلُوا الآنَ.
\par 13 وَأَمَّا مَتَى جَاءَ ذَاكَ رُوحُ الْحَقِّ فَهُوَ يُرْشِدُكُمْ إِلَى جَمِيعِ الْحَقِّ لأَنَّهُ لاَ يَتَكَلَّمُ مِنْ نَفْسِهِ بَلْ كُلُّ مَا يَسْمَعُ يَتَكَلَّمُ بِهِ وَيُخْبِرُكُمْ بِأُمُورٍ آتِيَةٍ.
\par 14 ذَاكَ يُمَجِّدُنِي لأَنَّهُ يَأْخُذُ مِمَّا لِي وَيُخْبِرُكُمْ.
\par 15 كُلُّ مَا لِلآبِ هُوَ لِي. لِهَذَا قُلْتُ إِنَّهُ يَأْخُذُ مِمَّا لِي وَيُخْبِرُكُمْ.
\par 16 بَعْدَ قَلِيلٍ لاَ تُبْصِرُونَنِي ثُمَّ بَعْدَ قَلِيلٍ أَيْضاً تَرَوْنَنِي لأَنِّي ذَاهِبٌ إِلَى الآبِ».
\par 17 فَقَالَ قَوْمٌ مِنْ تلاَمِيذِهِ بَعْضُهُمْ لِبَعْضٍ: «مَا هُوَ هَذَا الَّذِي يَقُولُهُ لَنَا: بَعْدَ قَلِيلٍ لاَ تُبْصِرُونَنِي ثُمَّ بَعْدَ قَلِيلٍ أَيْضاً تَرَوْنَنِي وَلأَنِّي ذَاهِبٌ إِلَى الآبِ؟».
\par 18 فَتَسَاءَلُوا: «مَا هُوَ هَذَا الْقَلِيلُ الَّذِي يَقُولُ عَنْهُ؟ لَسْنَا نَعْلَمُ بِمَاذَا يَتَكَلَّمُ».
\par 19 فَعَلِمَ يَسُوعُ أَنَّهُمْ كَانُوا يُرِيدُونَ أَنْ يَسْأَلُوهُ فَقَالَ لَهُمْ: «أَعَنْ هَذَا تَتَسَاءَلُونَ فِيمَا بَيْنَكُمْ لأَنِّي قُلْتُ: بَعْدَ قَلِيلٍ لاَ تُبْصِرُونَنِي ثُمَّ بَعْدَ قَلِيلٍ أَيْضاً تَرَوْنَنِي
\par 20 اَلْحَقَّ الْحَقَّ أَقُولُ لَكُمْ: إِنَّكُمْ سَتَبْكُونَ وَتَنُوحُونَ وَالْعَالَمُ يَفْرَحُ. أَنْتُمْ سَتَحْزَنُونَ وَلَكِنَّ حُزْنَكُمْ يَتَحَوَّلُ إِلَى فَرَحٍ.
\par 21 اَلْمَرْأَةُ وَهِيَ تَلِدُ تَحْزَنُ لأَنَّ سَاعَتَهَا قَدْ جَاءَتْ وَلَكِنْ مَتَى وَلَدَتِ الطِّفْلَ لاَ تَعُودُ تَذْكُرُ الشِّدَّةَ لِسَبَبِ الْفَرَحِ لأَنَّهُ قَدْ وُلِدَ إِنْسَانٌ فِي الْعَالَمِ.
\par 22 فَأَنْتُمْ كَذَلِكَ عِنْدَكُمُ الآنَ حُزْنٌ. وَلَكِنِّي سَأَرَاكُمْ أَيْضاً فَتَفْرَحُ قُلُوبُكُمْ وَلاَ يَنْزِعُ أَحَدٌ فَرَحَكُمْ مِنْكُمْ.
\par 23 وَفِي ذَلِكَ الْيَوْمِ لاَ تَسْأَلُونَنِي شَيْئاً. اَلْحَقَّ الْحَقَّ أَقُولُ لَكُمْ: إِنَّ كُلَّ مَا طَلَبْتُمْ مِنَ الآبِ بِاسْمِي يُعْطِيكُمْ.
\par 24 إِلَى الآنَ لَمْ تَطْلُبُوا شَيْئاً بِاسْمِي. اُطْلُبُوا تَأْخُذُوا لِيَكُونَ فَرَحُكُمْ كَامِلاً.
\par 25 «قَدْ كَلَّمْتُكُمْ بِهَذَا بِأَمْثَالٍ وَلَكِنْ تَأْتِي سَاعَةٌ حِينَ لاَ أُكَلِّمُكُمْ أَيْضاً بِأَمْثَالٍ بَلْ أُخْبِرُكُمْ عَنِ الآبِ علاَنِيَةً.
\par 26 فِي ذَلِكَ الْيَوْمِ تَطْلُبُونَ بِاسْمِي. وَلَسْتُ أَقُولُ لَكُمْ إِنِّي أَنَا أَسْأَلُ الآبَ مِنْ أَجْلِكُمْ
\par 27 لأَنَّ الآبَ نَفْسَهُ يُحِبُّكُمْ لأَنَّكُمْ قَدْ أَحْبَبْتُمُونِي وَآمَنْتُمْ أَنِّي مِنْ عِنْدِ اللَّهِ خَرَجْتُ.
\par 28 خَرَجْتُ مِنْ عِنْدِ الآبِ وَقَدْ أَتَيْتُ إِلَى الْعَالَمِ وَأَيْضاً أَتْرُكُ الْعَالَمَ وَأَذْهَبُ إِلَى الآبِ».
\par 29 قَالَ لَهُ تلاَمِيذُهُ: «هُوَذَا الآنَ تَتَكَلَّمُ علاَنِيَةً وَلَسْتَ تَقُولُ مَثَلاً وَاحِداً!
\par 30 اَلآنَ نَعْلَمُ أَنَّكَ عَالِمٌ بِكُلِّ شَيْءٍ وَلَسْتَ تَحْتَاجُ أَنْ يَسْأَلَكَ أَحَدٌ. لِهَذَا نُؤْمِنُ أَنَّكَ مِنَ اللَّهِ خَرَجْتَ».
\par 31 أَجَابَهُمْ يَسُوعُ: «أَلآنَ تُؤْمِنُونَ؟
\par 32 هُوَذَا تَأْتِي سَاعَةٌ وَقَدْ أَتَتِ الآنَ تَتَفَرَّقُونَ فِيهَا كُلُّ وَاحِدٍ إِلَى خَاصَّتِهِ وَتَتْرُكُونَنِي وَحْدِي. وَأَنَا لَسْتُ وَحْدِي لأَنَّ الآبَ مَعِي.
\par 33 قَدْ كَلَّمْتُكُمْ بِهَذَا لِيَكُونَ لَكُمْ فِيَّ سلاَمٌ. فِي الْعَالَمِ سَيَكُونُ لَكُمْ ضِيقٌ وَلَكِنْ ثِقُوا: أَنَا قَدْ غَلَبْتُ الْعَالَمَ».

\chapter{17}

\par 1 تَكَلَّمَ يَسُوعُ بِهَذَا وَرَفَعَ عَيْنَيْهِ نَحْوَ السَّمَاءِ وَقَالَ: «أَيُّهَا الآبُ قَدْ أَتَتِ السَّاعَةُ. مَجِّدِ ابْنَكَ لِيُمَجِّدَكَ ابْنُكَ أَيْضاً
\par 2 إِذْ أَعْطَيْتَهُ سُلْطَاناً عَلَى كُلِّ جَسَدٍ لِيُعْطِيَ حَيَاةً أَبَدِيَّةً لِكُلِّ مَنْ أَعْطَيْتَهُ.
\par 3 وَهَذِهِ هِيَ الْحَيَاةُ الأَبَدِيَّةُ: أَنْ يَعْرِفُوكَ أَنْتَ الإِلَهَ الْحَقِيقِيَّ وَحْدَكَ وَيَسُوعَ الْمَسِيحَ الَّذِي أَرْسَلْتَهُ.
\par 4 أَنَا مَجَّدْتُكَ عَلَى الأَرْضِ. الْعَمَلَ الَّذِي أَعْطَيْتَنِي لأَعْمَلَ قَدْ أَكْمَلْتُهُ.
\par 5 وَالآنَ مَجِّدْنِي أَنْتَ أَيُّهَا الآبُ عِنْدَ ذَاتِكَ بِالْمَجْدِ الَّذِي كَانَ لِي عِنْدَكَ قَبْلَ كَوْنِ الْعَالَمِ.
\par 6 «أَنَا أَظْهَرْتُ اسْمَكَ لِلنَّاسِ الَّذِينَ أَعْطَيْتَنِي مِنَ الْعَالَمِ. كَانُوا لَكَ وَأَعْطَيْتَهُمْ لِي وَقَدْ حَفِظُوا كلاَمَكَ.
\par 7 وَالآنَ عَلِمُوا أَنَّ كُلَّ مَا أَعْطَيْتَنِي هُوَ مِنْ عِنْدِكَ
\par 8 لأَنَّ الْكلاَمَ الَّذِي أَعْطَيْتَنِي قَدْ أَعْطَيْتُهُمْ وَهُمْ قَبِلُوا وَعَلِمُوا يَقِيناً أَنِّي خَرَجْتُ مِنْ عِنْدِكَ وَآمَنُوا أَنَّكَ أَنْتَ أَرْسَلْتَنِي.
\par 9 مِنْ أَجْلِهِمْ أَنَا أَسْأَلُ. لَسْتُ أَسْأَلُ مِنْ أَجْلِ الْعَالَمِ بَلْ مِنْ أَجْلِ الَّذِينَ أَعْطَيْتَنِي لأَنَّهُمْ لَكَ.
\par 10 وَكُلُّ مَا هُوَ لِي فَهُوَ لَكَ وَمَا هُوَ لَكَ فَهُوَ لِي وَأَنَا مُمَجَّدٌ فِيهِمْ.
\par 11 وَلَسْتُ أَنَا بَعْدُ فِي الْعَالَمِ وَأَمَّا هَؤُلاَءِ فَهُمْ فِي الْعَالَمِ وَأَنَا آتِي إِلَيْكَ. أَيُّهَا الآبُ الْقُدُّوسُ احْفَظْهُمْ فِي اسْمِكَ. الَّذِينَ أَعْطَيْتَنِي لِيَكُونُوا وَاحِداً كَمَا نَحْنُ.
\par 12 حِينَ كُنْتُ مَعَهُمْ فِي الْعَالَمِ كُنْتُ أَحْفَظُهُمْ فِي اسْمِكَ. الَّذِينَ أَعْطَيْتَنِي حَفِظْتُهُمْ وَلَمْ يَهْلِكْ مِنْهُمْ أَحَدٌ إِلاَّ ابْنُ الْهلاَكِ لِيَتِمَّ الْكِتَابُ.
\par 13 أَمَّا الآنَ فَإِنِّي آتِي إِلَيْكَ. وَأَتَكَلَّمُ بِهَذَا فِي الْعَالَمِ لِيَكُونَ لَهُمْ فَرَحِي كَامِلاً فِيهِمْ.
\par 14 أَنَا قَدْ أَعْطَيْتُهُمْ كلاَمَكَ وَالْعَالَمُ أَبْغَضَهُمْ لأَنَّهُمْ لَيْسُوا مِنَ الْعَالَمِ كَمَا أَنِّي أَنَا لَسْتُ مِنَ الْعَالَمِ
\par 15 لَسْتُ أَسْأَلُ أَنْ تَأْخُذَهُمْ مِنَ الْعَالَمِ بَلْ أَنْ تَحْفَظَهُمْ مِنَ الشِّرِّيرِ.
\par 16 لَيْسُوا مِنَ الْعَالَمِ كَمَا أَنِّي أَنَا لَسْتُ مِنَ الْعَالَمِ.
\par 17 قَدِّسْهُمْ فِي حَقِّكَ. كلاَمُكَ هُوَ حَقٌّ.
\par 18 كَمَا أَرْسَلْتَنِي إِلَى الْعَالَمِ أَرْسَلْتُهُمْ أَنَا إِلَى الْعَالَمِ
\par 19 وَلأَجْلِهِمْ أُقَدِّسُ أَنَا ذَاتِي لِيَكُونُوا هُمْ أَيْضاً مُقَدَّسِينَ فِي الْحَقِّ.
\par 20 «وَلَسْتُ أَسْأَلُ مِنْ أَجْلِ هَؤُلاَءِ فَقَطْ بَلْ أَيْضاً مِنْ أَجْلِ الَّذِينَ يُؤْمِنُونَ بِي بِكلاَمِهِمْ
\par 21 لِيَكُونَ الْجَمِيعُ وَاحِداً كَمَا أَنَّكَ أَنْتَ أَيُّهَا الآبُ فِيَّ وَأَنَا فِيكَ لِيَكُونُوا هُمْ أَيْضاً وَاحِداً فِينَا لِيُؤْمِنَ الْعَالَمُ أَنَّكَ أَرْسَلْتَنِي.
\par 22 وَأَنَا قَدْ أَعْطَيْتُهُمُ الْمَجْدَ الَّذِي أَعْطَيْتَنِي لِيَكُونُوا وَاحِداً كَمَا أَنَّنَا نَحْنُ وَاحِدٌ.
\par 23 أَنَا فِيهِمْ وَأَنْتَ فِيَّ لِيَكُونُوا مُكَمَّلِينَ إِلَى وَاحِدٍ وَلِيَعْلَمَ الْعَالَمُ أَنَّكَ أَرْسَلْتَنِي وَأَحْبَبْتَهُمْ كَمَا أَحْبَبْتَنِي.
\par 24 أَيُّهَا الآبُ أُرِيدُ أَنَّ هَؤُلاَءِ الَّذِينَ أَعْطَيْتَنِي يَكُونُونَ مَعِي حَيْثُ أَكُونُ أَنَا لِيَنْظُرُوا مَجْدِي الَّذِي أَعْطَيْتَنِي لأَنَّكَ أَحْبَبْتَنِي قَبْلَ إِنْشَاءِ الْعَالَمِ.
\par 25 أَيُّهَا الآبُ الْبَارُّ إِنَّ الْعَالَمَ لَمْ يَعْرِفْكَ أَمَّا أَنَا فَعَرَفْتُكَ وَهَؤُلاَءِ عَرَفُوا أَنَّكَ أَنْتَ أَرْسَلْتَنِي.
\par 26 وَعَرَّفْتُهُمُ اسْمَكَ وَسَأُعَرِّفُهُمْ لِيَكُونَ فِيهِمُ الْحُبُّ الَّذِي أَحْبَبْتَنِي بِهِ وَأَكُونَ أَنَا فِيهِمْ».

\chapter{18}

\par 1 قَالَ يَسُوعُ هَذَا وَخَرَجَ مَعَ تلاَمِيذِهِ إِلَى عَبْرِ وَادِي قَدْرُونَ حَيْثُ كَانَ بُسْتَانٌ دَخَلَهُ هُوَ وَتلاَمِيذُهُ.
\par 2 وَكَانَ يَهُوذَا مُسَلِّمُهُ يَعْرِفُ الْمَوْضِعَ لأَنَّ يَسُوعَ اجْتَمَعَ هُنَاكَ كَثِيراً مَعَ تلاَمِيذِهِ.
\par 3 فَأَخَذَ يَهُوذَا الْجُنْدَ وَخُدَّاماً مِنْ عِنْدِ رُؤَسَاءِ الْكَهَنَةِ وَالْفَرِّيسِيِّينَ وَجَاءَ إِلَى هُنَاكَ بِمَشَاعِلَ وَمَصَابِيحَ وَسِلاَحٍ.
\par 4 فَخَرَجَ يَسُوعُ وَهُوَ عَالِمٌ بِكُلِّ مَا يَأْتِي عَلَيْهِ وَقَالَ لَهُمْ: «مَنْ تَطْلُبُونَ؟»
\par 5 أَجَابُوهُ: «يَسُوعَ النَّاصِرِيَّ». قَالَ لَهُمْ: «أَنَا هُوَ». وَكَانَ يَهُوذَا مُسَلِّمُهُ أَيْضاً وَاقِفاً مَعَهُمْ.
\par 6 فَلَمَّا قَالَ لَهُمْ: «إِنِّي أَنَا هُوَ» رَجَعُوا إِلَى الْوَرَاءِ وَسَقَطُوا عَلَى الأَرْضِ.
\par 7 فَسَأَلَهُمْ أَيْضاً: «مَنْ تَطْلُبُونَ؟» فَقَالُوا: «يَسُوعَ النَّاصِرِيَّ».
\par 8 أَجَابَ: «قَدْ قُلْتُ لَكُمْ إِنِّي أَنَا هُوَ. فَإِنْ كُنْتُمْ تَطْلُبُونَنِي فَدَعُوا هَؤُلاَءِ يَذْهَبُونَ».
\par 9 لِيَتِمَّ الْقَوْلُ الَّذِي قَالَهُ: «إِنَّ الَّذِينَ أَعْطَيْتَنِي لَمْ أُهْلِكْ مِنْهُمْ أَحَداً».
\par 10 ثُمَّ إِنَّ سِمْعَانَ بُطْرُسَ كَانَ مَعَهُ سَيْفٌ فَاسْتَلَّهُ وَضَرَبَ عَبْدَ رَئِيسِ الْكَهَنَةِ فَقَطَعَ أُذْنَهُ الْيُمْنَى. وَكَانَ اسْمُ الْعَبْدِ مَلْخُسَ.
\par 11 فَقَالَ يَسُوعُ لِبُطْرُسَ: «اجْعَلْ سَيْفَكَ فِي الْغِمْدِ. الْكَأْسُ الَّتِي أَعْطَانِي الآبُ ألاَ أَشْرَبُهَا؟».
\par 12 ثُمَّ إِنَّ الْجُنْدَ وَالْقَائِدَ وَخُدَّامَ الْيَهُودِ قَبَضُوا عَلَى يَسُوعَ وَأَوْثَقُوهُ
\par 13 وَمَضَوْا بِهِ إِلَى حَنَّانَ أَوَّلاً لأَنَّهُ كَانَ حَمَا قَيَافَا الَّذِي كَانَ رَئِيساً لِلْكَهَنَةِ فِي تِلْكَ السَّنَةِ.
\par 14 وَكَانَ قَيَافَا هُوَ الَّذِي أَشَارَ عَلَى الْيَهُودِ أَنَّهُ خَيْرٌ أَنْ يَمُوتَ إِنْسَانٌ وَاحِدٌ عَنِ الشَّعْبِ.
\par 15 وَكَانَ سِمْعَانُ بُطْرُسُ وَالتِّلْمِيذُ الآخَرُ يَتْبَعَانِ يَسُوعَ وَكَانَ ذَلِكَ التِّلْمِيذُ مَعْرُوفاً عِنْدَ رَئِيسِ الْكَهَنَةِ فَدَخَلَ مَعَ يَسُوعَ إِلَى دَارِ رَئِيسِ الْكَهَنَةِ.
\par 16 وَأَمَّا بُطْرُسُ فَكَانَ وَاقِفاً عِنْدَ الْبَابِ خَارِجاً. فَخَرَجَ التِّلْمِيذُ الآخَرُ الَّذِي كَانَ مَعْرُوفاً عِنْدَ رَئِيسِ الْكَهَنَةِ وَكَلَّمَ الْبَوَّابَةَ فَأَدْخَلَ بُطْرُسَ.
\par 17 فَقَالَتِ الْجَارِيَةُ الْبَوَّابَةُ لِبُطْرُسَ: «أَلَسْتَ أَنْتَ أَيْضاً مِنْ تلاَمِيذِ هَذَا الإِنْسَانِ؟» قَالَ ذَاكَ: «لَسْتُ أَنَا».
\par 18 وَكَانَ الْعَبِيدُ وَالْخُدَّامُ وَاقِفِينَ وَهُمْ قَدْ أَضْرَمُوا جَمْراً لأَنَّهُ كَانَ بَرْدٌ وَكَانُوا يَصْطَلُونَ وَكَانَ بُطْرُسُ وَاقِفاً مَعَهُمْ يَصْطَلِي.
\par 19 فَسَأَلَ رَئِيسُ الْكَهَنَةِ يَسُوعَ عَنْ تلاَمِيذِهِ وَعَنْ تَعْلِيمِهِ.
\par 20 أَجَابَهُ يَسُوعُ: «أَنَا كَلَّمْتُ الْعَالَمَ علاَنِيَةً. أَنَا عَلَّمْتُ كُلَّ حِينٍ فِي الْمَجْمَعِ وَفِي الْهَيْكَلِ حَيْثُ يَجْتَمِعُ الْيَهُودُ دَائِماً. وَفِي الْخَفَاءِ لَمْ أَتَكَلَّمْ بِشَيْءٍ.
\par 21 لِمَاذَا تَسْأَلُنِي أَنَا؟ اِسْأَلِ الَّذِينَ قَدْ سَمِعُوا مَاذَا كَلَّمْتُهُمْ. هُوَذَا هَؤُلاَءِ يَعْرِفُونَ مَاذَا قُلْتُ أَنَا».
\par 22 وَلَمَّا قَالَ هَذَا لَطَمَ يَسُوعَ وَاحِدٌ مِنَ الْخُدَّامِ كَانَ وَاقِفاً قَائِلاً: «أَهَكَذَا تُجَاوِبُ رَئِيسَ الْكَهَنَةِ؟»
\par 23 أَجَابَهُ يَسُوعُ: «إِنْ كُنْتُ قَدْ تَكَلَّمْتُ رَدِيّاً فَاشْهَدْ عَلَى الرَّدِيِّ وَإِنْ حَسَناً فَلِمَاذَا تَضْرِبُنِي؟»
\par 24 وَكَانَ حَنَّانُ قَدْ أَرْسَلَهُ مُوثَقاً إِلَى قَيَافَا رَئِيسِ الْكَهَنَةِ.
\par 25 وَسِمْعَانُ بُطْرُسُ كَانَ وَاقِفاً يَصْطَلِي. فَقَالُوا لَهُ: «أَلَسْتَ أَنْتَ أَيْضاً مِنْ تلاَمِيذِهِ؟» فَأَنْكَرَ ذَاكَ وَقَالَ: «لَسْتُ أَنَا».
\par 26 قَالَ وَاحِدٌ مِنْ عَبِيدِ رَئِيسِ الْكَهَنَةِ وَهُوَ نَسِيبُ الَّذِي قَطَعَ بُطْرُسُ أُذْنَهُ: «أَمَا رَأَيْتُكَ أَنَا مَعَهُ فِي الْبُسْتَانِ؟»
\par 27 فَأَنْكَرَ بُطْرُسُ أَيْضاً. وَلِلْوَقْتِ صَاحَ الدِّيكُ.
\par 28 ثُمَّ جَاءُوا بِيَسُوعَ مِنْ عِنْدِ قَيَافَا إِلَى دَارِ الْوِلاَيَةِ وَكَانَ صُبْحٌ. وَلَمْ يَدْخُلُوا هُمْ إِلَى دَارِ الْوِلاَيَةِ لِكَيْ لاَ يَتَنَجَّسُوا فَيَأْكُلُونَ الْفِصْحَ.
\par 29 فَخَرَجَ بِيلاَطُسُ إِلَيْهِمْ وَقَالَ: «أَيَّةَ شِكَايَةٍ تُقَدِّمُونَ عَلَى هَذَا الإِنْسَانِ؟»
\par 30 أَجَابُوا: «لَوْ لَمْ يَكُنْ فَاعِلَ شَرٍّ لَمَا كُنَّا قَدْ سَلَّمْنَاهُ إِلَيْكَ!»
\par 31 فَقَالَ لَهُمْ بِيلاَطُسُ: «خُذُوهُ أَنْتُمْ وَاحْكُمُوا عَلَيْهِ حَسَبَ نَامُوسِكُمْ». فَقَالَ لَهُ الْيَهُودُ: «لاَ يَجُوزُ لَنَا أَنْ نَقْتُلَ أَحَداً».
\par 32 لِيَتِمَّ قَوْلُ يَسُوعَ الَّذِي قَالَهُ مُشِيراً إِلَى أَيَّةِ مِيتَةٍ كَانَ مُزْمِعاً أَنْ يَمُوتَ.
\par 33 ثُمَّ دَخَلَ بِيلاَطُسُ أَيْضاً إِلَى دَارِ الْوِلاَيَةِ وَدَعَا يَسُوعَ وَقَالَ لَهُ: «أَأَنْتَ مَلِكُ الْيَهُودِ؟»
\par 34 أَجَابَهُ يَسُوعُ: «أَمِنْ ذَاتِكَ تَقُولُ هَذَا أَمْ آخَرُونَ قَالُوا لَكَ عَنِّي؟»
\par 35 أَجَابَهُ بِيلاَطُسُ: «أَلَعَلِّي أَنَا يَهُودِيٌّ؟ أُمَّتُكَ وَرُؤَسَاءُ الْكَهَنَةِ أَسْلَمُوكَ إِلَيَّ. مَاذَا فَعَلْتَ؟»
\par 36 أَجَابَ يَسُوعُ: «مَمْلَكَتِي لَيْسَتْ مِنْ هَذَا الْعَالَمِ. لَوْ كَانَتْ مَمْلَكَتِي مِنْ هَذَا الْعَالَمِ لَكَانَ خُدَّامِي يُجَاهِدُونَ لِكَيْ لاَ أُسَلَّمَ إِلَى الْيَهُودِ. وَلَكِنِ الآنَ لَيْسَتْ مَمْلَكَتِي مِنْ هُنَا».
\par 37 فَقَالَ لَهُ بِيلاَطُسُ: «أَفَأَنْتَ إِذاً مَلِكٌ؟» أَجَابَ يَسُوعُ: «أَنْتَ تَقُولُ إِنِّي مَلِكٌ. لِهَذَا قَدْ وُلِدْتُ أَنَا وَلِهَذَا قَدْ أَتَيْتُ إِلَى الْعَالَمِ لأَشْهَدَ لِلْحَقِّ. كُلُّ مَنْ هُوَ مِنَ الْحَقِّ يَسْمَعُ صَوْتِي».
\par 38 قَالَ لَهُ بِيلاَطُسُ: «مَا هُوَ الْحَقُّ؟». وَلَمَّا قَالَ هَذَا خَرَجَ أَيْضاً إِلَى الْيَهُودِ وَقَالَ لَهُمْ: «أَنَا لَسْتُ أَجِدُ فِيهِ عِلَّةً وَاحِدَةً.
\par 39 وَلَكُمْ عَادَةٌ أَنْ أُطْلِقَ لَكُمْ وَاحِداً فِي الْفِصْحِ. أَفَتُرِيدُونَ أَنْ أُطْلِقَ لَكُمْ مَلِكَ الْيَهُودِ؟».
\par 40 فَصَرَخُوا أَيْضاً جَمِيعُهُمْ: «لَيْسَ هَذَا بَلْ بَارَابَاسَ». وَكَانَ بَارَابَاسُ لِصّاً.

\chapter{19}

\par 1 فَحِينَئِذٍ أَخَذَ بِيلاَطُسُ يَسُوعَ وَجَلَدَهُ.
\par 2 وَضَفَرَ الْعَسْكَرُ إِكْلِيلاً مِنْ شَوْكٍ وَوَضَعُوهُ عَلَى رَأْسِهِ وَأَلْبَسُوهُ ثَوْبَ أُرْجُوانٍ
\par 3 وَكَانُوا يَقُولُونَ: «السّلاَمُ يَا مَلِكَ الْيَهُودِ». وَكَانُوا يَلْطِمُونَهُ.
\par 4 فَخَرَجَ بِيلاَطُسُ أَيْضاً خَارِجاً وَقَالَ لَهُمْ: «هَا أَنَا أُخْرِجُهُ إِلَيْكُمْ لِتَعْلَمُوا أَنِّي لَسْتُ أَجِدُ فِيهِ عِلَّةً وَاحِدَةً».
\par 5 فَخَرَجَ يَسُوعُ خَارِجاً وَهُوَ حَامِلٌ إِكْلِيلَ الشَّوْكِ وَثَوْبَ الأُرْجُوانِ. فَقَالَ لَهُمْ بِيلاَطُسُ: «هُوَذَا الإِنْسَانُ».
\par 6 فَلَمَّا رَآهُ رُؤَسَاءُ الْكَهَنَةِ وَالْخُدَّامُ صَرَخُوا: «اصْلِبْهُ! اصْلِبْهُ!» قَالَ لَهُمْ بِيلاَطُسُ: «خُذُوهُ أَنْتُمْ وَاصْلِبُوهُ لأَنِّي لَسْتُ أَجِدُ فِيهِ عِلَّةً».
\par 7 أَجَابَهُ الْيَهُودُ: «لَنَا نَامُوسٌ وَحَسَبَ نَامُوسِنَا يَجِبُ أَنْ يَمُوتَ لأَنَّهُ جَعَلَ نَفْسَهُ ابْنَ اللَّهِ».
\par 8 فَلَمَّا سَمِعَ بِيلاَطُسُ هَذَا الْقَوْلَ ازْدَادَ خَوْفاً.
\par 9 فَدَخَلَ أَيْضاً إِلَى دَارِ الْوِلاَيَةِ وَقَالَ لِيَسُوعَ: «مِنْ أَيْنَ أَنْتَ؟» وَأَمَّا يَسُوعُ فَلَمْ يُعْطِهِ جَوَاباً.
\par 10 فَقَالَ لَهُ بِيلاَطُسُ: «أَمَا تُكَلِّمُنِي؟ أَلَسْتَ تَعْلَمُ أَنَّ لِي سُلْطَاناً أَنْ أَصْلِبَكَ وَسُلْطَاناً أَنْ أُطْلِقَكَ؟»
\par 11 أَجَابَ يَسُوعُ: « لَمْ يَكُنْ لَكَ عَلَيَّ سُلْطَانٌ الْبَتَّةَ لَوْ لَمْ تَكُنْ قَدْ أُعْطِيتَ مِنْ فَوْقُ. لِذَلِكَ الَّذِي أَسْلَمَنِي إِلَيْكَ لَهُ خَطِيَّةٌ أَعْظَمُ».
\par 12 مِنْ هَذَا الْوَقْتِ كَانَ بِيلاَطُسُ يَطْلُبُ أَنْ يُطْلِقَهُ وَلَكِنَّ الْيَهُودَ كَانُوا يَصْرُخُونَ: «إِنْ أَطْلَقْتَ هَذَا فَلَسْتَ مُحِبّاً لِقَيْصَرَ. كُلُّ مَنْ يَجْعَلُ نَفْسَهُ مَلِكاً يُقَاوِمُ قَيْصَرَ».
\par 13 فَلَمَّا سَمِعَ بِيلاَطُسُ هَذَا الْقَوْلَ أَخْرَجَ يَسُوعَ وَجَلَسَ عَلَى كُرْسِيِّ الْوِلاَيَةِ فِي مَوْضِعٍ يُقَالُ لَهُ «الْبلاَطُ» وَبِالْعِبْرَانِيَّةِ «جَبَّاثَا».
\par 14 وَكَانَ اسْتِعْدَادُ الْفِصْحِ وَنَحْوُ السَّاعَةِ السَّادِسَةِ. فَقَالَ لِلْيَهُودِ: «هُوَذَا مَلِكُكُمْ».
\par 15 فَصَرَخُوا: «خُذْهُ! خُذْهُ اصْلِبْهُ!» قَالَ لَهُمْ بِيلاَطُسُ: «أَأَصْلِبُ مَلِكَكُمْ؟» أَجَابَ رُؤَسَاءُ الْكَهَنَةِ: «لَيْسَ لَنَا مَلِكٌ إِلاَّ قَيْصَرُ».
\par 16 فَحِينَئِذٍ أَسْلَمَهُ إِلَيْهِمْ لِيُصْلَبَ. فَأَخَذُوا يَسُوعَ وَمَضَوْا بِهِ.
\par 17 فَخَرَجَ وَهُوَ حَامِلٌ صَلِيبَهُ إِلَى الْمَوْضِعِ الَّذِي يُقَالُ لَهُ «مَوْضِعُ الْجُمْجُمَةِ» وَيُقَالُ لَهُ بِالْعِبْرَانِيَّةِ «جُلْجُثَةُ»
\par 18 حَيْثُ صَلَبُوهُ وَصَلَبُوا اثْنَيْنِ آخَرَيْنِ مَعَهُ مِنْ هُنَا وَمِنْ هُنَا وَيَسُوعُ فِي الْوَسْطِ.
\par 19 وَكَتَبَ بِيلاَطُسُ عُنْوَاناً وَوَضَعَهُ عَلَى الصَّلِيبِ. وَكَانَ مَكْتُوباً: «يَسُوعُ النَّاصِرِيُّ مَلِكُ الْيَهُودِ».
\par 20 فَقَرَأَ هَذَا الْعُنْوَانَ كَثِيرُونَ مِنَ الْيَهُودِ لأَنَّ الْمَكَانَ الَّذِي صُلِبَ فِيهِ يَسُوعُ كَانَ قَرِيباً مِنَ الْمَدِينَةِ. وَكَانَ مَكْتُوباً بِالْعِبْرَانِيَّةِ وَالْيُونَانِيَّةِ وَاللَّاتِينِيَّةِ.
\par 21 فَقَالَ رُؤَسَاءُ كَهَنَةِ الْيَهُودِ لِبِيلاَطُسَ: «لاَ تَكْتُبْ: مَلِكُ الْيَهُودِ بَلْ: إِنَّ ذَاكَ قَالَ أَنَا مَلِكُ الْيَهُودِ».
\par 22 أَجَابَ بِيلاَطُسُ: «مَا كَتَبْتُ قَدْ كَتَبْتُ».
\par 23 ثُمَّ إِنَّ الْعَسْكَرَ لَمَّا كَانُوا قَدْ صَلَبُوا يَسُوعَ أَخَذُوا ثِيَابَهُ وَجَعَلُوهَا أَرْبَعَةَ أَقْسَامٍ لِكُلِّ عَسْكَرِيٍّ قِسْماً. وَأَخَذُوا الْقَمِيصَ أَيْضاً. وَكَانَ الْقَمِيصُ بِغَيْرِ خِيَاطَةٍ مَنْسُوجاً كُلُّهُ مِنْ فَوْقُ.
\par 24 فَقَالَ بَعْضُهُمْ لِبَعْضٍ: «لاَ نَشُقُّهُ بَلْ نَقْتَرِعُ عَلَيْهِ لِمَنْ يَكُونُ». لِيَتِمَّ الْكِتَابُ الْقَائِلُ: «اقْتَسَمُوا ثِيَابِي بَيْنَهُمْ وَعَلَى لِبَاسِي أَلْقَوْا قُرْعَةً». هَذَا فَعَلَهُ الْعَسْكَرُ.
\par 25 وَكَانَتْ وَاقِفَاتٍ عِنْدَ صَلِيبِ يَسُوعَ أُمُّهُ وَأُخْتُ أُمِّهِ مَرْيَمُ زَوْجَةُ كِلُوبَا وَمَرْيَمُ الْمَجْدَلِيَّةُ.
\par 26 فَلَمَّا رَأَى يَسُوعُ أُمَّهُ وَالتِّلْمِيذَ الَّذِي كَانَ يُحِبُّهُ وَاقِفاً قَالَ لِأُمِّهِ: «يَا امْرَأَةُ هُوَذَا ابْنُكِ».
\par 27 ثُمَّ قَالَ لِلتِّلْمِيذِ: «هُوَذَا أُمُّكَ». وَمِنْ تِلْكَ السَّاعَةِ أَخَذَهَا التِّلْمِيذُ إِلَى خَاصَّتِهِ.
\par 28 بَعْدَ هَذَا رَأَى يَسُوعُ أَنَّ كُلَّ شَيْءٍ قَدْ كَمَلَ فَلِكَيْ يَتِمَّ الْكِتَابُ قَالَ: «أَنَا عَطْشَانُ».
\par 29 وَكَانَ إِنَاءٌ مَوْضُوعاً مَمْلُوّاً خَلاًّ فَمَلَأُوا إِسْفِنْجَةً مِنَ الْخَلِّ وَوَضَعُوهَا عَلَى زُوفَا وَقَدَّمُوهَا إِلَى فَمِهِ.
\par 30 فَلَمَّا أَخَذَ يَسُوعُ الْخَلَّ قَالَ: «قَدْ أُكْمِلَ». وَنَكَّسَ رَأْسَهُ وَأَسْلَمَ الرُّوحَ.
\par 31 ثُمَّ إِذْ كَانَ اسْتِعْدَادٌ فَلِكَيْ لاَ تَبْقَى الأَجْسَادُ عَلَى الصَّلِيبِ فِي السَّبْتِ لأَنَّ يَوْمَ ذَلِكَ السَّبْتِ كَانَ عَظِيماً سَأَلَ الْيَهُودُ بِيلاَطُسَ أَنْ تُكْسَرَ سِيقَانُهُمْ وَيُرْفَعُوا.
\par 32 فَأَتَى الْعَسْكَرُ وَكَسَرُوا سَاقَيِ الأَوَّلِ وَالآخَرِ الْمَصْلُوبَيْنِ مَعَهُ.
\par 33 وَأَمَّا يَسُوعُ فَلَمَّا جَاءُوا إِلَيْهِ لَمْ يَكْسِرُوا سَاقَيْهِ لأَنَّهُمْ رَأَوْهُ قَدْ مَاتَ.
\par 34 لَكِنَّ وَاحِداً مِنَ الْعَسْكَرِ طَعَنَ جَنْبَهُ بِحَرْبَةٍ وَلِلْوَقْتِ خَرَجَ دَمٌ وَمَاءٌ.
\par 35 وَالَّذِي عَايَنَ شَهِدَ وَشَهَادَتُهُ حَقٌّ وَهُوَ يَعْلَمُ أَنَّهُ يَقُولُ الْحَقَّ لِتُؤْمِنُوا أَنْتُمْ.
\par 36 لأَنَّ هَذَا كَانَ لِيَتِمَّ الْكِتَابُ الْقَائِلُ: «عَظْمٌ لاَ يُكْسَرُ مِنْهُ».
\par 37 وَأَيْضاً يَقُولُ كِتَابٌ آخَرُ: «سَيَنْظُرُونَ إِلَى الَّذِي طَعَنُوهُ».
\par 38 ثُمَّ إِنَّ يُوسُفَ الَّذِي مِنَ الرَّامَةِ وَهُوَ تِلْمِيذُ يَسُوعَ وَلَكِنْ خُفْيَةً لِسَبَبِ الْخَوْفِ مِنَ الْيَهُودِ سَأَلَ بِيلاَطُسَ أَنْ يَأْخُذَ جَسَدَ يَسُوعَ فَأَذِنَ بِيلاَطُسُ. فَجَاءَ وَأَخَذَ جَسَدَ يَسُوعَ.
\par 39 وَجَاءَ أَيْضاً نِيقُودِيمُوسُ الَّذِي أَتَى أَوَّلاً إِلَى يَسُوعَ لَيْلاً وَهُوَ حَامِلٌ مَزِيجَ مُرٍّ وَعُودٍ نَحْوَ مِئَةِ مَناً.
\par 40 فَأَخَذَا جَسَدَ يَسُوعَ وَلَفَّاهُ بِأَكْفَانٍ مَعَ الأَطْيَابِ كَمَا لِلْيَهُودِ عَادَةٌ أَنْ يُكَفِّنُوا.
\par 41 وَكَانَ فِي الْمَوْضِعِ الَّذِي صُلِبَ فِيهِ بُسْتَانٌ وَفِي الْبُسْتَانِ قَبْرٌ جَدِيدٌ لَمْ يُوضَعْ فِيهِ أَحَدٌ قَطُّ.
\par 42 فَهُنَاكَ وَضَعَا يَسُوعَ لِسَبَبِ اسْتِعْدَادِ الْيَهُودِ لأَنَّ الْقَبْرَ كَانَ قَرِيباً.

\chapter{20}

\par 1 ٍوَفِي أَوَّلِ الأُسْبُوعِ جَاءَتْ مَرْيَمُ الْمَجْدَلِيَّةُ إِلَى الْقَبْرِ بَاكِراً وَالظّلاَمُ بَاقٍ. فَنَظَرَتِ الْحَجَرَ مَرْفُوعاً عَنِ الْقَبْرِ.
\par 2 فَرَكَضَتْ وَجَاءَتْ إِلَى سِمْعَانَ بُطْرُسَ وَإِلَى التِّلْمِيذِ الآخَرِ الَّذِي كَانَ يَسُوعُ يُحِبُّهُ وَقَالَتْ لَهُمَا: «أَخَذُوا السَّيِّدَ مِنَ الْقَبْرِ وَلَسْنَا نَعْلَمُ أَيْنَ وَضَعُوهُ».
\par 3 فَخَرَجَ بُطْرُسُ وَالتِّلْمِيذُ الآخَرُ وَأَتَيَا إِلَى الْقَبْرِ.
\par 4 وَكَانَ الاِثْنَانِ يَرْكُضَانِ مَعاً. فَسَبَقَ التِّلْمِيذُ الآخَرُ بُطْرُسَ وَجَاءَ أَوَّلاً إِلَى الْقَبْرِ
\par 5 وَانْحَنَى فَنَظَرَ الأَكْفَانَ مَوْضُوعَةً وَلَكِنَّهُ لَمْ يَدْخُلْ.
\par 6 ثُمَّ جَاءَ سِمْعَانُ بُطْرُسُ يَتْبَعُهُ وَدَخَلَ الْقَبْرَ وَنَظَرَ الأَكْفَانَ مَوْضُوعَةً
\par 7 وَالْمِنْدِيلَ الَّذِي كَانَ عَلَى رَأْسِهِ لَيْسَ مَوْضُوعاً مَعَ الأَكْفَانِ بَلْ مَلْفُوفاً فِي مَوْضِعٍ وَحْدَهُ.
\par 8 فَحِينَئِذٍ دَخَلَ أَيْضاً التِّلْمِيذُ الآخَرُ الَّذِي جَاءَ أَوَّلاً إِلَى الْقَبْرِ وَرَأَى فَآمَنَ
\par 9 لأَنَّهُمْ لَمْ يَكُونُوا بَعْدُ يَعْرِفُونَ الْكِتَابَ: أَنَّهُ يَنْبَغِي أَنْ يَقُومَ مِنَ الأَمْوَاتِ.
\par 10 فَمَضَى التِّلْمِيذَانِ أَيْضاً إِلَى مَوْضِعِهِمَا.
\par 11 أَمَّا مَرْيَمُ فَكَانَتْ وَاقِفَةً عِنْدَ الْقَبْرِ خَارِجاً تَبْكِي. وَفِيمَا هِيَ تَبْكِي انْحَنَتْ إِلَى الْقَبْرِ
\par 12 فَنَظَرَتْ ملاَكَيْنِ بِثِيَابٍ بِيضٍ جَالِسَيْنِ وَاحِداً عِنْدَ الرَّأْسِ وَالآخَرَ عِنْدَ الرِّجْلَيْنِ حَيْثُ كَانَ جَسَدُ يَسُوعَ مَوْضُوعاً.
\par 13 فَقَالاَ لَهَا: «يَا امْرَأَةُ لِمَاذَا تَبْكِينَ؟» قَالَتْ لَهُمَا: «إِنَّهُمْ أَخَذُوا سَيِّدِي وَلَسْتُ أَعْلَمُ أَيْنَ وَضَعُوهُ».
\par 14 وَلَمَّا قَالَتْ هَذَا الْتَفَتَتْ إِلَى الْوَرَاءِ فَنَظَرَتْ يَسُوعَ وَاقِفاً وَلَمْ تَعْلَمْ أَنَّهُ يَسُوعُ.
\par 15 قَالَ لَهَا يَسُوعُ: «يَا امْرَأَةُ لِمَاذَا تَبْكِينَ؟ مَنْ تَطْلُبِينَ؟» فَظَنَّتْ تِلْكَ أَنَّهُ الْبُسْتَانِيُّ فَقَالَتْ لَهُ: «يَا سَيِّدُ إِنْ كُنْتَ أَنْتَ قَدْ حَمَلْتَهُ فَقُلْ لِي أَيْنَ وَضَعْتَهُ وَأَنَا آخُذُهُ».
\par 16 قَالَ لَهَا يَسُوعُ: «يَا مَرْيَمُ!» فَالْتَفَتَتْ تِلْكَ وَقَالَتْ لَهُ: «رَبُّونِي» الَّذِي تَفْسِيرُهُ يَا مُعَلِّمُ.
\par 17 قَالَ لَهَا يَسُوعُ: «لاَ تَلْمِسِينِي لأَنِّي لَمْ أَصْعَدْ بَعْدُ إِلَى أَبِي. وَلَكِنِ اذْهَبِي إِلَى إِخْوَتِي وَقُولِي لَهُمْ: إِنِّي أَصْعَدُ إِلَى أَبِي وَأَبِيكُمْ وَإِلَهِي وَإِلَهِكُمْ».
\par 18 فَجَاءَتْ مَرْيَمُ الْمَجْدَلِيَّةُ وَأَخْبَرَتِ التّلاَمِيذَ أَنَّهَا رَأَتِ الرَّبَّ وَأَنَّهُ قَالَ لَهَا هَذَا.
\par 19 وَلَمَّا كَانَتْ عَشِيَّةُ ذَلِكَ الْيَوْمِ وَهُوَ أَوَّلُ الأُسْبُوعِ وَكَانَتِ الأَبْوَابُ مُغَلَّقَةً حَيْثُ كَانَ التّلاَمِيذُ مُجْتَمِعِينَ لِسَبَبِ الْخَوْفِ مِنَ الْيَهُودِ جَاءَ يَسُوعُ وَوَقَفَ فِي الْوَسَطِ وَقَالَ لَهُمْ: «سلاَمٌ لَكُمْ».
\par 20 وَلَمَّا قَالَ هَذَا أَرَاهُمْ يَدَيْهِ وَجَنْبَهُ فَفَرِحَ التّلاَمِيذُ إِذْ رَأَوُا الرَّبَّ.
\par 21 فَقَالَ لَهُمْ يَسُوعُ أَيْضاً: «سلاَمٌ لَكُمْ. كَمَا أَرْسَلَنِي الآبُ أُرْسِلُكُمْ أَنَا».
\par 22 وَلَمَّا قَالَ هَذَا نَفَخَ وَقَالَ لَهُمُ: «اقْبَلُوا الرُّوحَ الْقُدُسَ.
\par 23 مَنْ غَفَرْتُمْ خَطَايَاهُ تُغْفَرُ لَهُ وَمَنْ أَمْسَكْتُمْ خَطَايَاهُ أُمْسِكَتْ».
\par 24 أَمَّا تُومَا أَحَدُ الاِثْنَيْ عَشَرَ الَّذِي يُقَالُ لَهُ التَّوْأَمُ فَلَمْ يَكُنْ مَعَهُمْ حِينَ جَاءَ يَسُوعُ.
\par 25 فَقَالَ لَهُ التّلاَمِيذُ الآخَرُونَ: «قَدْ رَأَيْنَا الرَّبَّ». فَقَالَ لَهُمْ: «إِنْ لَمْ أُبْصِرْ فِي يَدَيْهِ أَثَرَ الْمَسَامِيرِ وَأَضَعْ إِصْبِعِي فِي أَثَرِ الْمَسَامِيرِ وَأَضَعْ يَدِي فِي جَنْبِهِ لاَ أُومِنْ».
\par 26 وَبَعْدَ ثَمَانِيَةِ أَيَّامٍ كَانَ تلاَمِيذُهُ أَيْضاً دَاخِلاً وَتُومَا مَعَهُمْ. فَجَاءَ يَسُوعُ وَالأَبْوَابُ مُغَلَّقَةٌ وَوَقَفَ فِي الْوَسَطِ وَقَالَ: «سلاَمٌ لَكُمْ».
\par 27 ثُمَّ قَالَ لِتُومَا: «هَاتِ إِصْبِعَكَ إِلَى هُنَا وَأَبْصِرْ يَدَيَّ وَهَاتِ يَدَكَ وَضَعْهَا فِي جَنْبِي وَلاَ تَكُنْ غَيْرَ مُؤْمِنٍ بَلْ مُؤْمِناً».
\par 28 أَجَابَ تُومَا: «رَبِّي وَإِلَهِي».
\par 29 قَالَ لَهُ يَسُوعُ: «لأَنَّكَ رَأَيْتَنِي يَا تُومَا آمَنْتَ! طُوبَى لِلَّذِينَ آمَنُوا وَلَمْ يَرَوْا».
\par 30 وَآيَاتٍ أُخَرَ كَثِيرَةً صَنَعَ يَسُوعُ قُدَّامَ تلاَمِيذِهِ لَمْ تُكْتَبْ فِي هَذَا الْكِتَابِ.
\par 31 وَأَمَّا هَذِهِ فَقَدْ كُتِبَتْ لِتُؤْمِنُوا أَنَّ يَسُوعَ هُوَ الْمَسِيحُ ابْنُ اللَّهِ وَلِكَيْ تَكُونَ لَكُمْ إِذَا آمَنْتُمْ حَيَاةٌ بِاسْمِهِ.

\chapter{21}

\par 1 بَعْدَ هَذَا أَظْهَرَ أَيْضاً يَسُوعُ نَفْسَهُ لِلتّلاَمِيذِ عَلَى بَحْرِ طَبَرِيَّةَ. ظَهَرَ هَكَذَا:
\par 2 كَانَ سِمْعَانُ بُطْرُسُ وَتُومَا الَّذِي يُقَالُ لَهُ التَّوْأَمُ وَنَثَنَائِيلُ الَّذِي مِنْ قَانَا الْجَلِيلِ وَابْنَا زَبْدِي وَاثْنَانِ آخَرَانِ مِنْ تلاَمِيذِهِ مَعَ بَعْضِهِمْ.
\par 3 قَالَ لَهُمْ سِمْعَانُ بُطْرُسُ: «أَنَا أَذْهَبُ لأَتَصَيَّدَ». قَالُوا لَهُ: «نَذْهَبُ نَحْنُ أَيْضاً مَعَكَ». فَخَرَجُوا وَدَخَلُوا السَّفِينَةَ لِلْوَقْتِ. وَفِي تِلْكَ اللَّيْلَةِ لَمْ يُمْسِكُوا شَيْئاً.
\par 4 وَلَمَّا كَانَ الصُّبْحُ وَقَفَ يَسُوعُ عَلَى الشَّاطِئِ. وَلَكِنَّ التّلاَمِيذَ لَمْ يَكُونُوا يَعْلَمُونَ أَنَّهُ يَسُوعُ.
\par 5 فَقَالَ لَهُمْ يَسُوعُ: «يَا غِلْمَانُ أَلَعَلَّ عِنْدَكُمْ إِدَاماً؟». أَجَابُوهُ: «لاَ!»
\par 6 فَقَالَ لَهُمْ: «أَلْقُوا الشَّبَكَةَ إِلَى جَانِبِ السَّفِينَةِ الأَيْمَنِ فَتَجِدُوا». فَأَلْقَوْا وَلَمْ يَعُودُوا يَقْدِرُونَ أَنْ يَجْذِبُوهَا مِنْ كَثْرَةِ السَّمَكِ.
\par 7 فَقَالَ ذَلِكَ التِّلْمِيذُ الَّذِي كَانَ يَسُوعُ يُحِبُّهُ لِبُطْرُسَ: «هُوَ الرَّبُّ». فَلَمَّا سَمِعَ سِمْعَانُ بُطْرُسُ أَنَّهُ الرَّبُّ اتَّزَرَ بِثَوْبِهِ لأَنَّهُ كَانَ عُرْيَاناً وَأَلْقَى نَفْسَهُ فِي الْبَحْرِ.
\par 8 وَأَمَّا التّلاَمِيذُ الآخَرُونَ فَجَاءُوا بِالسَّفِينَةِ لأَنَّهُمْ لَمْ يَكُونُوا بَعِيدِينَ عَنِ الأَرْضِ إِلاَّ نَحْوَ مِئَتَيْ ذِرَاعٍ وَهُمْ يَجُرُّونَ شَبَكَةَ السَّمَكِ.
\par 9 فَلَمَّا خَرَجُوا إِلَى الأَرْضِ نَظَرُوا جَمْراً مَوْضُوعاً وَسَمَكاً مَوْضُوعاً عَلَيْهِ وَخُبْزاً.
\par 10 قَالَ لَهُمْ يَسُوعُ: «قَدِّمُوا مِنَ السَّمَكِ الَّذِي أَمْسَكْتُمُ الآنَ».
\par 11 فَصَعِدَ سِمْعَانُ بُطْرُسُ وَجَذَبَ الشَّبَكَةَ إِلَى الأَرْضِ مُمْتَلِئَةً سَمَكاً كَبِيراً مِئَةً وَثلاَثاً وَخَمْسِينَ. وَمَعْ هَذِهِ الْكَثْرَةِ لَمْ تَتَخَرَّقِ الشَّبَكَةُ.
\par 12 قَالَ لَهُمْ يَسُوعُ: «هَلُمُّوا تَغَدَّوْا». وَلَمْ يَجْسُرْ أَحَدٌ مِنَ التّلاَمِيذِ أَنْ يَسْأَلَهُ: مَنْ أَنْتَ؟ إِذْ كَانُوا يَعْلَمُونَ أَنَّهُ الرَّبُّ.
\par 13 ثُمَّ جَاءَ يَسُوعُ وَأَخَذَ الْخُبْزَ وَأَعْطَاهُمْ وَكَذَلِكَ السَّمَكَ.
\par 14 هَذِهِ مَرَّةٌ ثَالِثَةٌ ظَهَرَ يَسُوعُ لِتلاَمِيذِهِ بَعْدَمَا قَامَ مِنَ الأَمْوَاتِ.
\par 15 فَبَعْدَ مَا تَغَدَّوْا قَالَ يَسُوعُ لِسِمْعَانَ بُطْرُسَ: «يَا سِمْعَانُ بْنَ يُونَا أَتُحِبُّنِي أَكْثَرَ مِنْ هَؤُلاَءِ؟» قَالَ لَهُ: «نَعَمْ يَا رَبُّ أَنْتَ تَعْلَمُ أَنِّي أُحِبُّكَ». قَالَ لَهُ: «ارْعَ خِرَافِي».
\par 16 قَالَ لَهُ أَيْضاً ثَانِيَةً: «يَا سِمْعَانُ بْنَ يُونَا أَتُحِبُّنِي؟» قَالَ لَهُ: «نَعَمْ يَا رَبُّ أَنْتَ تَعْلَمُ أَنِّي أُحِبُّكَ». قَالَ لَهُ: «ارْعَ غَنَمِي».
\par 17 قَالَ لَهُ ثَالِثَةً: «يَا سِمْعَانُ بْنَ يُونَا أَتُحِبُّنِي؟» فَحَزِنَ بُطْرُسُ لأَنَّهُ قَالَ لَهُ ثَالِثَةً: أَتُحِبُّنِي؟ فَقَالَ لَهُ: «يَا رَبُّ أَنْتَ تَعْلَمُ كُلَّ شَيْءٍ. أَنْتَ تَعْرِفُ أَنِّي أُحِبُّكَ». قَالَ لَهُ يَسُوعُ: «ارْعَ غَنَمِي.
\par 18 اَلْحَقَّ الْحَقَّ أَقُولُ لَكَ: لَمَّا كُنْتَ أَكْثَرَ حَدَاثَةً كُنْتَ تُمَنْطِقُ ذَاتَكَ وَتَمْشِي حَيْثُ تَشَاءُ. وَلَكِنْ مَتَى شِخْتَ فَإِنَّكَ تَمُدُّ يَدَيْكَ وَآخَرُ يُمَنْطِقُكَ وَيَحْمِلُكَ حَيْثُ لاَ تَشَاءُ».
\par 19 قَالَ هَذَا مُشِيراً إِلَى أَيَّةِ مِيتَةٍ كَانَ مُزْمِعاً أَنْ يُمَجِّدَ اللَّهَ بِهَا. وَلَمَّا قَالَ هَذَا قَالَ لَهُ: «اتْبَعْنِي».
\par 20 فَالْتَفَتَ بُطْرُسُ وَنَظَرَ التِّلْمِيذَ الَّذِي كَانَ يَسُوعُ يُحِبُّهُ يَتْبَعُهُ وَهُوَ أَيْضاً الَّذِي اتَّكَأَ عَلَى صَدْرِهِ وَقْتَ الْعَشَاءِ وَقَالَ: « يَا سَيِّدُ مَنْ هُوَ الَّذِي يُسَلِّمُكَ؟»
\par 21 فَلَمَّا رَأَى بُطْرُسُ هَذَا قَالَ لِيَسُوعَ: «يَا رَبُّ وَهَذَا مَا لَهُ؟»
\par 22 قَالَ لَهُ يَسُوعُ: «إِنْ كُنْتُ أَشَاءُ أَنَّهُ يَبْقَى حَتَّى أَجِيءَ فَمَاذَا لَكَ؟ اتْبَعْنِي أَنْتَ».
\par 23 فَذَاعَ هَذَا الْقَوْلُ بَيْنَ الإِخْوَةِ: إِنَّ ذَلِكَ التِّلْمِيذَ لاَ يَمُوتُ. وَلَكِنْ لَمْ يَقُلْ لَهُ يَسُوعُ إِنَّهُ لاَ يَمُوتُ بَلْ: «إِنْ كُنْتُ أَشَاءُ أَنَّهُ يَبْقَى حَتَّى أَجِيءَ فَمَاذَا لَكَ؟».
\par 24 هَذَا هُوَ التِّلْمِيذُ الَّذِي يَشْهَدُ بِهَذَا وَكَتَبَ هَذَا. وَنَعْلَمُ أَنَّ شَهَادَتَهُ حَقٌّ.
\par 25 وَأَشْيَاءُ أُخَرُ كَثِيرَةٌ صَنَعَهَا يَسُوعُ إِنْ كُتِبَتْ وَاحِدَةً وَاحِدَةً فَلَسْتُ أَظُنُّ أَنَّ الْعَالَمَ نَفْسَهُ يَسَعُ الْكُتُبَ الْمَكْتُوبَةَ. آمِينَ.

\end{document}