\begin{document}

\title{ناحوم}


\chapter{1}

\par 1 وَحْيٌ عَلَى نِينَوَى. سِفْرُ رُؤْيَا نَاحُومَ الأَلْقُوشِيِّ:
\par 2 اَلرَّبُّ إِلَهٌ غَيُورٌ وَمُنْتَقِمٌ. الرَّبُّ مُنْتَقِمٌ وَذُو سَخَطٍ. الرَّبُّ مُنْتَقِمٌ مِن مُبْغِضِيهِ وَحَافِظٌ غَضَبَهُ علَى أَعْدَائِهِ.
\par 3 الرَّبُّ بَطِيءُ الْغَضَبِ وَعَظِيمُ الْقُدْرَةِ وَلَكِنَّهُ لاَ يُبَرِّئُ الْبَتَّةَ. الرَّبُّ فِي الزَّوْبَعَةِ وَفِي الْعَاصِفِ طَرِيقُهُ وَالسَّحَابُ غُبَارُ رِجْلَيْهِ.
\par 4 يَنْتَهِرُ الْبَحْرَ فَيُنَشِّفُهُ وَيُجَفِّفُ جَمِيعَ الأَنْهَارِ. يَذْبُلُ بَاشَانُ وَالْكَرْمَلُ وَزَهْرُ لُبْنَانَ يَذْبُلُ.
\par 5 اَلْجِبَالُ تَرْجُفُ مِنْهُ وَالتِّلاَلُ تَذُوبُ وَالأَرْضُ تُرْفَعُ مِنْ وَجْهِهِ وَالْعَالَمُ وَكُلُّ السَّاكِنِينَ فِيهِ.
\par 6 مَنْ يَقِفُ أَمَامَ سَخَطِهِ وَمَنْ يَقُومُ فِي حُمُوِّ غَضَبِهِ؟ غَيْظُهُ يَنْسَكِبُ كَالنَّارِ وَالصُّخُورُ تَنْهَدِمُ مِنْهُ.
\par 7 صَالِحٌ هُوَ الرَّبُّ. حِصْنٌ فِي يَوْمِ الضَّيقِ وَهُوَ يَعْرِفُ الْمُتَوَكِّلِينَ عَلَيْهِ.
\par 8 وَلكِنْ بِطُوفَانٍ عَابِرٍ يَصْنَعُ هَلاَكاً تَامّاً لِمَوْضِعِهَا وَأَعْدَاؤُهُ يَتْبَعُهُمْ ظَلاَمٌ.
\par 9 مَاذَا تَفْتَكِرُونَ عَلَى الرَّبِّ؟ هُوَ صَانِعٌ هَلاَكاً تَامّاً. لاَ يَقُومُ الضِّيقُ مَرَّتَيْنِ.
\par 10 فَإِنَّهُمْ وَهُمْ مُشْتَبِكُونَ مِثْلَ الشَّوْكِ وَسَكْرَانُونَ كَمِنْ خَمْرِهِمْ يُؤْكَلُونَ كَالْقَشِّ الْيَابِسِ بِالْكَمَالِ.
\par 11 مِنْكِ خَرَجَ الْمُفْتَكِرُ عَلَى الرَّبِّ شَرّاً الْمُشِيرُ بِالْهَلاَكِ.
\par 12 هَكَذَا قَالَ الرَّبُّ: «إِنْ كَانُوا سَالِمِينَ وَكَثِيرِينَ هَكَذَا فَهَكَذَا يُجَزُّونَ فَيَعْبُرُ. أَذْلَلْتُكِ. لاَ أُذِلُّكِ ثَانِيَةً.
\par 13 وَالآنَ أَكْسِرُ نِيرَهُ عَنْكِ وَأَقْطَعُ رُبُطَكِ».
\par 14 وَلكِنْ قَدْ أَوْصَى عَنْكَ الرَّبُّ: «لاَ يُزْرَعُ مِنِ اسْمِكَ فِي مَا بَعْدُ. إِنِّي أَقْطَعُ مِنْ بَيْتِ إِلَهِكَ التَّمَاثِيلَ الْمَنْحُوتَةَ وَالْمَسْبُوكَةَ. أَجْعَلُهُ قَبْرَكَ لأَنَّكَ صِرْتَ حَقِيراً».
\par 15 هُوَذَا عَلَى الْجِبَالِ قَدَمَا مُبَشِّرٍ مُنَادٍ بِالسَّلاَمِ: عَيِّدِي يَا يَهُوذَا أَعْيَادَكِ. أَوْفِي نُذُورَكِ فَإِنَّهُ لاَ يَعُودُ يَعْبُرُ فِيكِ أَيْضاً الْمُهْلِكُ. قَدِ انْقَرَضَ كُلُّهُ.

\chapter{2}

\par 1 قَدِ ارْتَفَعَتِ الْمِقْمَعَةُ عَلَى وَجْهِكِ. احْرُسِ الْحِصْنَ. رَاقِبِ الطَّرِيقَ. شَدِّدِ الْحَقَوَيْنِ. مَكِّنِ الْقُوَّةَ جِدّاً.
\par 2 فَإِنَّ الرَّبَّ يَرُدُّ عَظَمَةَ يَعْقُوبَ كَعَظَمَةِ إِسْرَائِيلَ لأَنَّ السَّالِبِينَ قَدْ سَلَبُوهُمْ وَأَتْلَفُوا قُضْبَانَ كُرُومِهِمْ.
\par 3 تُرْسُ أَبْطَالِهِ مُحَمَّرٌ. رِجَالُ الْجَيْشِ قِرْمِزِيُّونَ. الْمَرْكَبَاتُ بِنَارِ الْفُولاَذِ فِي يَوْمِ إِعْدَادِهِ. وَالسَّرْوُ يَهْتَزُّ.
\par 4 تَهِيجُ الْمَرْكَبَاتُ فِي الأَزِقَّةِ. تَتَرَاكَضُ فِي السَّاحَاتِ. مَنْظَرُهَا كَمَصَابِيحَ. تَجْرِي كَالْبُرُوقِ.
\par 5 يَذْكُرُ عُظَمَاءَهُ. يَتَعَثَّرُونَ فِي مَشْيِهِمْ. يُسْرِعُونَ إِلَى سُورِهَا وَقَدْ أُقِيمَتِ الْمِتْرَسَةُ.
\par 6 أَبْوَابُ الأَنْهَارِ انْفَتَحَتْ وَالْقَصْرُ قَدْ ذَابَ.
\par 7 وَهُصَّبُ قَدِ انْكَشَفَتْ. أُطْلِعَتْ. وَجَوَارِيهَا تَئِنُّ كَصَوْتِ الْحَمَامِ ضَارِبَاتٍ عَلَى صُدُورِهِنَّ.
\par 8 وَنِينَوَى كَبِرْكَةِ مَاءٍ مُنْذُ كَانَتْ وَلَكِنَّهُمُ الآنَ هَارِبُونَ. «قِفُوا قِفُوا!» وَلاَ مُلْتَفِتٌ.
\par 9 اِنْهَبُوا فِضَّةً. انْهَبُوا ذَهَباً فَلاَ نِهَايَةَ لِلتُّحَفِ لِلْكَثْرَةِ مِنْ كُلِّ مَتَاعٍ شَيْءٍ.
\par 10 فَرَاغٌ وَخَلاَءٌ وَخَرَابٌ وَقَلْبٌ ذَائِبٌ وَارْتِخَاءُ رُكَبٍ وَوَجَعٌ فِي كُلِّ حَقْوٍ. وَأَوْجُهُ جَمِيعِهِمْ تَجْمَعُ حُمْرَةً.
\par 11 أَيْنَ مَأْوَى الأُسُودِ وَمَرْعَى أَشْبَالِ الأُسُودِ؟ حَيْثُ يَمْشِي الأَسَدُ وَاللَّبْوَةُ وَشِبْلُ الأَسَدِ وَلَيْسَ مَنْ يُخَوِّفُ.
\par 12 الأَسَدُ الْمُفْتَرِسُ لِحَاجةِ جِرَائِهِ وَالْخَانِقُ لأَجْلِ لَبْواتِهِ حَتَّى مَلَأَ مَغَارَاتِهِ فَرَائِسَ وَمَآوِيَهُ مُفْتَرَسَاتٍ.
\par 13 «هَا أَنَا عَلَيْكِ يَقُولُ رَبُّ الْجُنُودِ. فَأُحْرِقُ مَرْكَبَاتِكِ دُخَاناً وَأَشْبَالُكِ يَأْكُلُهَا السَّيْفُ وَأَقْطَعُ مِنَ الأَرْضِ فَرَائِسَكِ وَلاَ يُسْمَعُ أَيْضاً صَوْتُ رُسُلُكِ».

\chapter{3}

\par 1 وَيْلٌ لِمَدِينَةِ الدِّمَاءِ. كُلُّهَا مَلآنَةٌ كَذِباً وَخَطْفاً. لاَ يَزُولُ الاِفْتِرَاسُ.
\par 2 صَوْتُ السَّوطِ وَصَوْتُ رَعْشَةِ الْبَكَرِ وَخَيْلٌ تَخُبُّ وَمَرْكَبَاتٌ تَقْفِزُ
\par 3 وَفُرْسَانٌ تَنْهَضُ وَلَهِيبُ السَّيْفِ وَبَرِيقُ الرُّمْحِ وَكَثْرَةُ جَرْحَى وَوَفْرَةُ قَتْلَى وَلاَ نِهَايَةَ لِلْجُثَثِ. يَعْثُرُونَ بِجُثَثِهِمْ.
\par 4 مِنْ أَجْلِ زِنَى الزَّانِيَةِ الْحَسَنَةِ الْجَمَالِ صَاحِبَةِ السِّحْرِ الْبَائِعَةِ أُمَماً بِزِنَاهَا وَقَبَائِلَ بِسِحْرِهَا.
\par 5 «هَئَنَذَا عَلَيْكِ يَقُولُ رَبُّ الْجُنُودِ فَأَكْشِفُ أَذْيَالَكِ إِلَى فَوْقِ وَجْهِكِ وَأُرِي الأُمَمَ عَوْرَتَكِ وَالْمَمَالِكَ خِزْيَكِ.
\par 6 وَأَطْرَحُ عَلَيْكِ أَوْسَاخاً وَأُهِينُكِ وَأَجْعَلُكِ عِبْرَةً.
\par 7 وَيَكُونُ كُلُّ مَنْ يَرَاكِ يَهْرُبُ مِنْكِ وَيَقُولُ: خَرِبَتْ نِينَوَى مَنْ يَرْثِي لَهَا: مِنْ أَيْنَ أَطْلُبُ لَكِ مُعَزِّينَ؟».
\par 8 هَلْ أَنْتِ أَفْضَلُ مِنْ نُوَ أَمُونَ الْجَالِسَةِ بَيْنَ الأَنْهَارِ حَوْلَهَا الْمِيَاهُ الَّتِي هِيَ حِصْنُ الْبَحْرِ وَمِنَ الْبَحْرِ سُورُهَا؟
\par 9 كُوشٌ قُوَّتُهَا مَعَ مِصْرَ وَلَيْسَتْ نِهَايَةٌ. فُوطٌ وَلُوبِيمُ كَانُوا مَعُونَتَكِ.
\par 10 هِيَ أَيْضاً قَدْ مَضَتْ إِلَى الْمَنْفَى بِالسَّبْيِ وَأَطْفَالُهَا حُطِّمَتْ فِي رَأْسِ جَمِيعِ الأَزِقَّةِ وَعَلَى أَشْرَافِهَا أَلْقُوا قُرْعَةً وَجَمِيعُ عُظَمَائِهَا تَقَيَّدُوا بِالْقُيُودِ.
\par 11 أَنْتِ أَيْضاً تَسْكَرِينَ. تَكُونِينَ خَافِيَةً. أَنْتِ أَيْضاً تَطْلُبِينَ حِصْناً بِسَبَبِ الْعَدُوِّ.
\par 12 جَمِيعُ قِلاَعِكِ أَشْجَارُ تِينٍ بِالْبَوَاكِيرِ إِذَا انْهَزَّتْ تَسْقُطُ فِي فَمِ الآكِلِ.
\par 13 هُوَذَا شَعْبُكِ نِسَاءٌ فِي وَسَطِكِ. تَنْفَتِحُ لأَعْدَائِكِ أَبْوَابُ أَرْضِكِ. تَأْكُلُ النَّارُ مَغَالِيقَكِ.
\par 14 اِسْتَقِي لِنَفْسِكِ مَاءً لِلْحِصَارِ. أَصْلِحِي قِلاَعَكِ. ادْخُلِي فِي الطِّينِ وَدُوسِي فِي الْمِلاَطِ. أَصْلِحِي الْمِلْبَنَ.
\par 15 هُنَاكَ تَأْكُلُكِ نَارٌ. يَقْطَعُكِ سَيْفٌ. يَأْكُلُكِ كَالْغَوْغَاءِ. تَكَاثَرِي كَالْغَوْغَاءِ. تَعَاظَمِي كَالْجَرَادِ.
\par 16 أَكْثَرْتِ تُجَّارَكِ أَكْثَرَ مِنْ نُجُومِ السَّمَاءِ. الْغَوْغَاءُ جَنَّحَتْ وَطَارَتْ.
\par 17 رُؤَسَاؤُكِ كَالْجَرَادِ وَوُلاَتُكِ كَحَرْجَلَةِ الْجَرَادِ الْحَالَّةِ عَلَى الْجُدْرَانِ فِي يَوْمِ الْبَرْدِ. تُشْرِقُ الشَّمْسُ فَتَطِيرُ وَلاَ يُعْرَفُ مَكَانُهَا أَيْنَ هُوَ.
\par 18 نَعِسَتْ رُعَاتُكَ يَا مَلِكَ أَشُّورَ. اضْطَجَعَتْ عُظَمَاؤُكَ. تَشَتَّتَ شَعْبُكَ عَلَى الْجِبَالِ وَلاَ مَنْ يَجْمَعُ.
\par 19 لَيْسَ جَبْرٌ لاِنْكِسَارِكَ. جُرْحُكَ عَدِيمُ الشِّفَاءِ. كُلُّ الَّذِينَ يَسْمَعُونَ خَبَرَكَ يُصَفِّقُونَ بِأَيْدِيهِمْ عَلَيْكَ لأَنَّهُ عَلَى مَنْ لَمْ يَمُرَّ شَرُّكَ عَلَى الدَّوَامِ؟


\end{document}