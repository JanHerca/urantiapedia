\begin{document}

\title{دانيال}


\chapter{1}

\par 1 فِي السَّنَةِ الثَّالِثَةِ مِنْ مُلْكِ يَهُويَاقِيمَ مَلِكِ يَهُوذَا ذَهَبَ نَبُوخَذْنَصَّرُ مَلِكُ بَابِلَ إِلَى أُورُشَلِيمَ وَحَاصَرَهَا.
\par 2 وَسَلَّمَ الرَّبُّ بِيَدِهِ يَهُويَاقِيمَ مَلِكَ يَهُوذَا مَعَ بَعْضِ آنِيَةِ بَيْتِ اللَّهِ فَجَاءَ بِهَا إِلَى أَرْضِ شِنْعَارَ إِلَى بَيْتِ إِلَهِهِ وَأَدْخَلَ الآنِيَةَ إِلَى خِزَانَةِ بَيْتِ إِلَهِهِ.
\par 3 وَأَمَرَ الْمَلِكُ أَشْفَنَزَ رَئِيسَ خِصْيَانِهِ بِأَنْ يُحْضِرَ مِنْ بَنِي إِسْرَائِيلَ وَمِنْ نَسْلِ الْمُلْكِ وَمِنَ الشُّرَفَاءِ
\par 4 فِتْيَاناً لاَ عَيْبَ فِيهِمْ حِسَانَ الْمَنْظَرِ حَاذِقِينَ فِي كُلِّ حِكْمَةٍ وَعَارِفِينَ مَعْرِفَةً وَذَوِي فَهْمٍ بِالْعِلْمِ وَالَّذِينَ فِيهِمْ قُوَّةٌ عَلَى الْوُقُوفِ فِي قَصْرِ الْمَلِكِ فَيُعَلِّمُوهُمْ كِتَابَةَ الْكِلْدَانِيِّينَ وَلِسَانَهُمْ.
\par 5 وَعَيَّنَ لَهُمُ الْمَلِكُ وَظِيفَةً كُلَّ يَوْمٍ بِيَوْمِهِ مِنْ أَطَايِبِ الْمَلِكِ وَمِنْ خَمْرِ مَشْرُوبِهِ لِتَرْبِيَتِهِمْ ثَلاَثَ سِنِينَ وَعِنْدَ نِهَايَتِهَا يَقِفُونَ أَمَامَ الْمَلِكِ.
\par 6 وَكَانَ بَيْنَهُمْ مِنْ بَنِي يَهُوذَا: دَانِيآلُ وَحَنَنْيَا وَمِيشَائِيلُ وَعَزَرْيَا.
\par 7 فَجَعَلَ لَهُمْ رَئِيسُ الْخِصْيَانِ أَسْمَاءً فَسَمَّى دَانِيآلَ [بَلْطَشَاصَّرَ] وَحَنَنْيَا [شَدْرَخَ] وَمِيشَائِيلَ [مِيشَخَ] وَعَزَرْيَا [عَبْدَنَغُوَ].
\par 8 أَمَّا دَانِيآلُ فَجَعَلَ فِي قَلْبِهِ أَنَّهُ لاَ يَتَنَجَّسُ بِأَطَايِبِ الْمَلِكِ وَلاَ بِخَمْرِ مَشْرُوبِهِ فَطَلَبَ مِنْ رَئِيسِ الْخِصْيَانِ أَنْ لاَ يَتَنَجَّسَ.
\par 9 وَأَعْطَى اللَّهُ دَانِيآلَ نِعْمَةً وَرَحْمَةً عِنْدَ رَئِيسِ الْخِصْيَانِ.
\par 10 فَقَالَ رَئِيسُ الْخِصْيَانِ لِدَانِيآلَ: [إِنِّي أَخَافُ سَيِّدِي الْمَلِكَ الَّذِي عَيَّنَ طَعَامَكُمْ وَشَرَابَكُمْ. فَلِمَاذَا يَرَى وُجُوهَكُمْ أَهْزَلَ مِنَ الْفِتْيَانِ الَّذِينَ مِنْ جِيلِكُمْ فَتُدَيِّنُونَ رَأْسِي لِلْمَلِكِ؟]
\par 11 فَقَالَ دَانِيآلُ لِرَئِيسِ السُّقَاةِ الَّذِي وَلاَّهُ رَئِيسُ الْخِصْيَانِ عَلَى دَانِيآلَ وَحَنَنْيَا وَمِيشَائِيلَ وَعَزَرْيَا:
\par 12 [جَرِّبْ عَبِيدَكَ عَشَرَةَ أَيَّامٍ. فَلْيُعْطُونَا الْقَطَانِيَّ لِنَأْكُلَ وَمَاءً لِنَشْرَبَ.
\par 13 وَلْيَنْظُرُوا إِلَى مَنَاظِرِنَا أَمَامَكَ وَإِلَى مَنَاظِرِ الْفِتْيَانِ الَّذِينَ يَأْكُلُونَ مِنْ أَطَايِبِ الْمَلِكِ. ثُمَّ اصْنَعْ بِعَبِيدِكَ كَمَا تَرَى].
\par 14 فَسَمِعَ لَهُمْ هَذَا الْكَلاَمَ وَجَرَّبَهُمْ عَشَرَةَ أَيَّامٍ.
\par 15 وَعِنْدَ نِهَايَةِ الْعَشَرَةِ الأَيَّامِ ظَهَرَتْ مَنَاظِرُهُمْ أَحْسَنَ وَأَسْمَنَ لَحْماً مِنْ كُلِّ الْفِتْيَانِ الآكِلِينَ مِنْ أَطَايِبِ الْمَلِكِ.
\par 16 فَكَانَ رَئِيسُ السُّقَاةِ يَرْفَعُ أَطَايِبَهُمْ وَخَمْرَ مَشْرُوبِهِمْ وَيُعْطِيهِمْ قَطَانِيَّ.
\par 17 أَمَّا هَؤُلاَءِ الْفِتْيَانُ الأَرْبَعَةُ فَأَعْطَاهُمُ اللَّهُ مَعْرِفَةً وَعَقْلاً فِي كُلِّ كِتَابَةٍ وَحِكْمَةٍ وَكَانَ دَانِيآلُ فَهِيماً بِكُلِّ الرُّؤَى وَالأَحْلاَمِ.
\par 18 وَعِنْدَ نِهَايَةِ الأَيَّامِ الَّتِي قَالَ الْمَلِكُ أَنْ يُدْخِلُوهُمْ بَعْدَهَا أَتَى بِهِمْ رَئِيسُ الْخِصْيَانِ إِلَى أَمَامِ نَبُوخَذْنَصَّرَ
\par 19 وَكَلَّمَهُمُ الْمَلِكُ فَلَمْ يُوجَدْ بَيْنَهُمْ كُلِّهِمْ مِثْلُ دَانِيآلَ وَحَنَنْيَا وَمِيشَائِيلَ وَعَزَرْيَا. فَوَقَفُوا أَمَامَ الْمَلِكِ.
\par 20 وَفِي كُلِّ أَمْرِ حِكْمَةِ فَهْمٍ الَّذِي سَأَلَهُمْ عَنْهُ الْمَلِكُ وَجَدَهُمْ عَشَرَةَ أَضْعَافٍ فَوْقَ كُلِّ الْمَجُوسِ وَالسَّحَرَةِ الَّذِينَ فِي كُلِّ مَمْلَكَتِهِ.
\par 21 وَكَانَ دَانِيآلُ إِلَى السَّنَةِ الأُولَى لِكُورَشَ الْمَلِكِ.

\chapter{2}

\par 1 وَفِي السَّنَةِ الثَّانِيَةِ مِنْ مُلْكِ نَبُوخَذْنَصَّرَ حَلُمَ نَبُوخَذْنَصَّرُ أَحْلاَماً فَانْزَعَجَتْ رُوحُهُ وَطَارَ عَنْهُ نَوْمُهُ.
\par 2 فَأَمَرَ الْمَلِكُ بِأَنْ يُسْتَدْعَى الْمَجُوسُ وَالسَّحَرَةُ وَالْعَرَّافُونَ وَالْكِلْدَانِيُّونَ لِيُخْبِرُوا الْمَلِكَ بِأَحْلاَمِهِ. فَأَتُوا وَوَقَفُوا أَمَامَ الْمَلِكِ.
\par 3 فَقَالَ لَهُمُ الْمَلِكُ: [قَدْ حَلُمْتُ حُلْماً وَانْزَعَجَتْ رُوحِي لِمَعْرِفَةِ الْحُلْمِ].
\par 4 فَكَلَّمَ الْكِلْدَانِيُّونَ الْمَلِكَ بِالأَرَامِيَّةِ: [عِشْ أَيُّهَا الْمَلِكُ إِلَى الأَبَدِ. أَخْبِرْ عَبِيدَكَ بِالْحُلْمِ فَنُبَيِّنَ تَعْبِيرَهُ].
\par 5 فَقَالَ الْمَلِكُ لِلْكِلْدَانِيِّينَ: [قَدْ خَرَجَ مِنِّي الْقَوْلُ: إِنْ لَمْ تُنْبِئُونِي بِالْحُلْمِ وَبِتَعْبِيرِهِ تُصَيَّرُونَ إِرْباً إِرْباً وَتُجْعَلُ بُيُوتُكُمْ مَزْبَلَةً.
\par 6 وَإِنْ بَيَّنْتُمُ الْحُلْمَ وَتَعْبِيرَهُ تَنَالُونَ مِنْ قِبَلِي هَدَايَا وَحَلاَوِينَ وَإِكْرَاماً عَظِيماً. فَبَيِّنُوا لِي الْحُلْمَ وَتَعْبِيرَهُ].
\par 7 فَأَجَابُوا ثَانِيَةً: [لِيُخْبِرِ الْمَلِكُ عَبِيدَهُ بِالْحُلْمِ فَنُبَيِّنَ تَعْبِيرَهُ].
\par 8 قَالَ الْمَلِكُ: [إِنِّي أَعْلَمُ يَقِيناً أَنَّكُمْ تَكْتَسِبُونَ وَقْتاً إِذْ رَأَيْتُمْ أَنَّ الْقَوْلَ قَدْ خَرَجَ مِنِّي
\par 9 بِأَنَّهُ إِنْ لَمْ تُنْبِئُونِي بِالْحُلْمِ فَقَضَاؤُكُمْ وَاحِدٌ. لأَنَّكُمْ قَدِ اتَّفَقْتُمْ عَلَى كَلاَمٍ كَذِبٍ وَفَاسِدٍ لِتَتَكَلَّمُوا بِهِ قُدَّامِي إِلَى أَنْ يَتَحَوَّلَ الْوَقْتُ. فَأَخْبِرُونِي بِالْحُلْمِ فَأَعْلَمَ أَنَّكُمْ تُبَيِّنُونَ لِي تَعْبِيرَهُ].
\par 10 أَجَابَ الْكِلْدَانِيُّونَ قُدَّامَ الْمَلِكِ: [لَيْسَ عَلَى الأَرْضِ إِنْسَانٌ يَسْتَطِيعُ أَنْ يُبَيِّنَ أَمْرَ الْمَلِكِ. لِذَلِكَ لَيْسَ مَلِكٌ عَظِيمٌ ذُو سُلْطَانٍ سَأَلَ أَمْراً مِثْلَ هَذَا مِنْ مَجُوسِيٍّ أَوْ سَاحِرٍ أَوْ كِلْدَانِيٍّ.
\par 11 وَالأَمْرُ الَّذِي يَطْلُبُهُ الْمَلِكُ عَسِرٌ وَلَيْسَ آخَرُ يُبَيِّنُهُ قُدَّامَ الْمَلِكِ غَيْرَ الآلِهَةِ الَّذِينَ لَيْسَتْ سُكْنَاهُمْ مَعَ الْبَشَرِ].
\par 12 لأَجْلِ ذَلِكَ غَضِبَ الْمَلِكُ وَاغْتَاظَ جِدّاً وَأَمَرَ بِإِبَادَةِ كُلِّ حُكَمَاءِ بَابِلَ.
\par 13 فَخَرَجَ الأَمْرُ وَكَانَ الْحُكَمَاءُ يُقْتَلُونَ. فَطَلَبُوا دَانِيآلَ وَأَصْحَابَهُ لِيَقْتُلُوهُمْ.
\par 14 حِينَئِذٍ أَجَابَ دَانِيآلُ بِحِكْمَةٍ وَعَقْلٍ لأَرْيُوخَ رَئِيسِ شُرَطِ الْمَلِكِ الَّذِي خَرَجَ لِيَقْتُلَ حُكَمَاءَ بَابِلَ:
\par 15 [لِمَاذَا اشْتَدَّ الأَمْرُ مِنْ قِبَلِ الْمَلِكِ؟] حِينَئِذٍ أَخْبَرَ أَرْيُوخُ دَانِيآلَ بِالأَمْرِ.
\par 16 فَدَخَلَ دَانِيآلُ وَطَلَبَ مِنَ الْمَلِكِ أَنْ يُعْطِيَهُ وَقْتاً فَيُبَيِّنُ لِلْمَلِكِ التَّعْبِيرَ.
\par 17 حِينَئِذٍ مَضَى دَانِيآلُ إِلَى بَيْتِهِ وَأَعْلَمَ حَنَنْيَا وَمِيشَائِيلَ وَعَزَرْيَا أَصْحَابَهُ بِالأَمْرِ
\par 18 لِيَطْلُبُوا الْمَرَاحِمَ مِنْ قِبَلِ إِلَهِ السَّمَاوَاتِ مِنْ جِهَةِ هَذَا السِّرِّ لِكَيْ لاَ يَهْلِكَ دَانِيآلُ وَأَصْحَابُهُ مَعَ سَائِرِ حُكَمَاءِ بَابِلَ.
\par 19 حِينَئِذٍ كُشِفَ السِّرُّ لِدَانِيآلَ فِي رُؤْيَا اللَّيْلِ. فَبَارَكَ دَانِيآلُ إِلَهَ السَّمَاوَاتِ.
\par 20 فَقَالَ دَانِيآلُ: [لِيَكُنِ اسْمُ اللَّهِ مُبَارَكاً مِنَ الأَزَلِ وَإِلَى الأَبَدِ لأَنَّ لَهُ الْحِكْمَةَ وَالْجَبَرُوتَ.
\par 21 وَهُوَ يُغَيِّرُ الأَوْقَاتَ وَالأَزْمِنَةَ. يَعْزِلُ مُلُوكاً وَيُنَصِّبُ مُلُوكاً. يُعْطِي الْحُكَمَاءَ حِكْمَةً وَيُعَلِّمُ الْعَارِفِينَ فَهْماً.
\par 22 هُوَ يَكْشِفُ الْعَمَائِقَ وَالأَسْرَارَ. يَعْلَمُ مَا هُوَ فِي الظُّلْمَةِ وَعِنْدَهُ يَسْكُنُ النُّورُ.
\par 23 إِيَّاكَ يَا إِلَهَ آبَائِي أَحْمَدُ وَأُسَبِّحُ الَّذِي أَعْطَانِي الْحِكْمَةَ وَالْقُوَّةَ وَأَعْلَمَنِي الآنَ مَا طَلَبْنَاهُ مِنْكَ لأَنَّكَ أَعْلَمْتَنَا أَمْرَ الْمَلِكِ].
\par 24 فَمِنْ أَجْلِ ذَلِكَ دَخَلَ دَانِيآلُ إِلَى أَرْيُوخَ الَّذِي عَيَّنَهُ الْمَلِكُ لِإِبَادَةِ حُكَمَاءِ بَابِلَ وَقَالَ لَهُ: [لاَ تُبِدْ حُكَمَاءَ بَابِلَ. أَدْخِلْنِي إِلَى قُدَّامِ الْمَلِكِ فَأُبَيِّنَ لِلْمَلِكِ التَّعْبِيرَ].
\par 25 حِينَئِذٍ دَخَلَ أَرْيُوخُ بِدَانِيآلَ إِلَى قُدَّامِ الْمَلِكِ مُسْرِعاً وَقَالَ لَهُ: [قَدْ وَجَدْتُ رَجُلاً مِنْ بَنِي سَبْيِ يَهُوذَا الَّذِي يُعَرِّفُ الْمَلِكَ بِالتَّعْبِيرِ].
\par 26 فَقَالَ الْمَلِكُ لِدَانِيآلَ (الَّذِي اسْمُهُ بَلْطَشَاصَّرُ): [هَلْ تَسْتَطِيعُ أَنْتَ عَلَى أَنْ تُعَرِّفَنِي بِالْحُلْمِ الَّذِي رَأَيْتُ وَبِتَعْبِيرِهِ؟]
\par 27 أَجَابَ دَانِيآلُ قُدَّامَ الْمَلِكِ: [السِّرُّ الَّذِي طَلَبَهُ الْمَلِكُ لاَ تَقْدِرُ الْحُكَمَاءُ وَلاَ السَّحَرَةُ وَلاَ الْمَجُوسُ وَلاَ الْمُنَجِّمُونَ عَلَى أَنْ يُبَيِّنُوهُ لِلْمَلِكِ.
\par 28 لَكِنْ يُوجَدُ إِلَهٌ فِي السَّمَاوَاتِ كَاشِفُ الأَسْرَارِ وَقَدْ عَرَّفَ الْمَلِكَ نَبُوخَذْنَصَّرَ مَا يَكُونُ فِي الأَيَّامِ الأَخِيرَةِ. حُلْمُكَ وَرُؤْيَا رَأْسِكَ عَلَى فِرَاشِكَ هُوَ هَذَا:
\par 29 أَنْتَ يَا أَيُّهَا الْمَلِكُ أَفْكَارُكَ عَلَى فِرَاشِكَ صَعِدَتْ إِلَى مَا يَكُونُ مِنْ بَعْدِ هَذَا وَكَاشِفُ الأَسْرَارِ يُعَرِّفُكَ بِمَا يَكُونُ.
\par 30 أَمَّا أَنَا فَلَمْ يُكْشَفْ لِي هَذَا السِّرُّ لِحِكْمَةٍ فِيَّ أَكْثَرَ مِنْ كُلِّ الأَحْيَاءِ. وَلَكِنْ لِيُعَرَّفَ الْمَلِكُ بِالتَّعْبِيرِ وَلِتَعْلَمَ أَفْكَارَ قَلْبِكَ.
\par 31 [أَنْتَ أَيُّهَا الْمَلِكُ كُنْتَ تَنْظُرُ وَإِذَا بِتِمْثَالٍ عَظِيمٍ. هَذَا التِّمْثَالُ الْعَظِيمُ الْبَهِيُّ جِدّاً وَقَفَ قُبَالَتَكَ وَمَنْظَرُهُ هَائِلٌ.
\par 32 رَأْسُ هَذَا التِّمْثَالِ مِنْ ذَهَبٍ جَيِّدٍ. صَدْرُهُ وَذِرَاعَاهُ مِنْ فِضَّةٍ. بَطْنُهُ وَفَخْذَاهُ مِنْ نُحَاسٍ.
\par 33 سَاقَاهُ مِنْ حَدِيدٍ. قَدَمَاهُ بَعْضُهُمَا مِنْ حَدِيدٍ وَالْبَعْضُ مِنْ خَزَفٍ.
\par 34 كُنْتَ تَنْظُرُ إِلَى أَنْ قُطِعَ حَجَرٌ بِغَيْرِ يَدَيْنِ فَضَرَبَ التِّمْثَالَ عَلَى قَدَمَيْهِ اللَّتَيْنِ مِنْ حَدِيدٍ وَخَزَفٍ فَسَحَقَهُمَا.
\par 35 فَانْسَحَقَ حِينَئِذٍ الْحَدِيدُ وَالْخَزَفُ وَالنُّحَاسُ وَالْفِضَّةُ وَالذَّهَبُ مَعاً وَصَارَتْ كَعُصَافَةِ الْبَيْدَرِ فِي الصَّيْفِ فَحَمَلَتْهَا الرِّيحُ فَلَمْ يُوجَدْ لَهَا مَكَانٌ. أَمَّا الْحَجَرُ الَّذِي ضَرَبَ التِّمْثَالَ فَصَارَ جَبَلاً كَبِيراً وَمَلَأَ الأَرْضَ كُلَّهَا.
\par 36 هَذَا هُوَ الْحُلْمُ. فَنُخْبِرُ بِتَعْبِيرِهِ قُدَّامَ الْمَلِكِ:
\par 37 [أَنْتَ أَيُّهَا الْمَلِكُ مَلِكُ مُلُوكٍ لأَنَّ إِلَهَ السَّمَاوَاتِ أَعْطَاكَ مَمْلَكَةً وَاقْتِدَاراً وَسُلْطَاناً وَفَخْراً.
\par 38 وَحَيْثُمَا يَسْكُنُ بَنُو الْبَشَرِ وَوُحُوشُ الْبَرِّ وَطُيُورُ السَّمَاءِ دَفَعَهَا لِيَدِكَ وَسَلَّطَكَ عَلَيْهَا جَمِيعِهَا. فَأَنْتَ هَذَا الرَّأْسُ مِنْ ذَهَبٍ.
\par 39 وَبَعْدَكَ تَقُومُ مَمْلَكَةٌ أُخْرَى أَصْغَرُ مِنْكَ وَمَمْلَكَةٌ ثَالِثَةٌ أُخْرَى مِنْ نُحَاسٍ فَتَتَسَلَّطُ عَلَى كُلِّ الأَرْضِ.
\par 40 وَتَكُونُ مَمْلَكَةٌ رَابِعَةٌ صَلِبَةٌ كَالْحَدِيدِ لأَنَّ الْحَدِيدَ يَدُقُّ وَيَسْحَقُ كُلَّ شَيْءٍ. وَكَالْحَدِيدِ الَّذِي يُكَسِّرُ تَسْحَقُ وَتُكَسِّرُ كُلَّ هَؤُلاَءِ.
\par 41 وَبِمَا رَأَيْتَ الْقَدَمَيْنِ وَالأَصَابِعَ بَعْضُهَا مِنْ خَزَفٍ وَالْبَعْضُ مِنْ حَدِيدٍ فَالْمَمْلَكَةُ تَكُونُ مُنْقَسِمَةً وَيَكُونُ فِيهَا قُوَّةُ الْحَدِيدِ مِنْ حَيْثُ إِنَّكَ رَأَيْتَ الْحَدِيدَ مُخْتَلِطاً بِخَزَفِ الطِّينِ.
\par 42 وَأَصَابِعُ الْقَدَمَيْنِ بَعْضُهَا مِنْ حَدِيدٍ وَالْبَعْضُ مِنْ خَزَفٍ فَبَعْضُ الْمَمْلَكَةِ يَكُونُ قَوِيّاً وَالْبَعْضُ قَصِماً.
\par 43 وَبِمَا رَأَيْتَ الْحَدِيدَ مُخْتَلِطاً بِخَزَفِ الطِّينِ فَإِنَّهُمْ يَخْتَلِطُونَ بِنَسْلِ النَّاسِ وَلَكِنْ لاَ يَتَلاَصَقُ هَذَا بِذَاكَ كَمَا أَنَّ الْحَدِيدَ لاَ يَخْتَلِطُ بِالْخَزَفِ.
\par 44 وَفِي أَيَّامِ هَؤُلاَءِ الْمُلُوكِ يُقِيمُ إِلَهُ السَّمَاوَاتِ مَمْلَكَةً لَنْ تَنْقَرِضَ أَبَداً وَمَلِكُهَا لاَ يُتْرَكُ لِشَعْبٍ آخَرَ وَتَسْحَقُ وَتُفْنِي كُلَّ هَذِهِ الْمَمَالِكِ وَهِيَ تَثْبُتُ إِلَى الأَبَدِ.
\par 45 لأَنَّكَ رَأَيْتَ أَنَّهُ قَدْ قُطِعَ حَجَرٌ مِنْ جَبَلٍ لاَ بِيَدَيْنِ فَسَحَقَ الْحَدِيدَ وَالنُّحَاسَ وَالْخَزَفَ وَالْفِضَّةَ وَالذَّهَبَ. اللَّهُ الْعَظِيمُ قَدْ عَرَّفَ الْمَلِكَ مَا سَيَأْتِي بَعْدَ هَذَا. الْحُلْمُ حَقٌّ وَتَعْبِيرُهُ يَقِينٌ].
\par 46 حِينَئِذٍ خَرَّ نَبُوخَذْنَصَّرُ عَلَى وَجْهِهِ وَسَجَدَ لِدَانِيآلَ وَأَمَرَ بِأَنْ يُقَدِّمُوا لَهُ تَقْدِمَةً وَرَوَائِحَ سُرُورٍ.
\par 47 وَقَالَ الْمَلِكُ لِدَانِيآلَ: [حَقّاً إِنَّ إِلَهَكُمْ إِلَهُ الآلِهَةِ وَرَبُّ الْمُلُوكِ وَكَاشِفُ الأَسْرَارِ إِذِ اسْتَطَعْتَ عَلَى كَشْفِ هَذَا السِّرِّ].
\par 48 حِينَئِذٍ عَظَّمَ الْمَلِكُ دَانِيآلَ وَأَعْطَاهُ عَطَايَا كَثِيرَةً وَسَلَّطَهُ عَلَى كُلِّ وِلاَيَةِ بَابِلَ وَجَعَلَهُ رَئِيسَ الشِّحَنِ عَلَى جَمِيعِ حُكَمَاءِ بَابِلَ.
\par 49 فَطَلَبَ دَانِيآلُ مِنَ الْمَلِكِ فَوَلَّى شَدْرَخَ وَمِيشَخَ وَعَبْدَنَغُوَ عَلَى أَعْمَالِ وِلاَيَةِ بَابِلَ. أَمَّا دَانِيآلُ فَكَانَ فِي بَابِ الْمَلِكِ.

\chapter{3}

\par 1 نَبُوخَذْنَصَّرُ الْمَلِكُ صَنَعَ تِمْثَالاً مِنْ ذَهَبٍ طُولُهُ سِتُّونَ ذِرَاعاً وَعَرْضُهُ سِتُّ أَذْرُعٍ وَنَصَبَهُ فِي بُقْعَةِ دُورَا فِي وِلاَيَةِ بَابِلَ.
\par 2 ثُمَّ أَرْسَلَ نَبُوخَذْنَصَّرُ الْمَلِكُ لِيَجْمَعَ الْمَرَازِبَةَ وَالشِّحَنَ وَالْوُلاَةَ وَالْقُضَاةَ وَالْخَزَنَةَ وَالْفُقَهَاءَ وَالْمُفْتِينَ وَكُلَّ حُكَّامِ الْوِلاَيَاتِ لِيَأْتُوا لِتَدْشِينِ التِّمْثَالِ الَّذِي نَصَبَهُ نَبُوخَذْنَصَّرُ الْمَلِكُ.
\par 3 حِينَئِذٍ اجْتَمَعَ الْمَرَازِبَةُ وَالشِّحَنُ وَالْوُلاَةُ وَالْقُضَاةُ وَالْخَزَنَةُ وَالْفُقَهَاءُ وَالْمُفْتُونَ وَكُلُّ حُكَّامِ الْوِلاَيَاتِ لِتَدْشِينِ التِّمْثَالِ الَّذِي نَصَبَهُ نَبُوخَذْنَصَّرُ الْمَلِكُ وَوَقَفُوا أَمَامَ التِّمْثَالِ الَّذِي نَصَبَهُ نَبُوخَذْنَصَّرُ.
\par 4 وَنَادَى مُنَادٍ بِشِدَّةٍ: [قَدْ أُمِرْتُمْ أَيُّهَا الشُّعُوبُ وَالأُمَمُ وَالأَلْسِنَةُ
\par 5 عِنْدَمَا تَسْمَعُونَ صَوْتَ الْقَرْنِ وَالنَّايِ وَالْعُودِ وَالرَّبَابِ وَالسِّنْطِيرِ وَالْمِزْمَارِ وَكُلِّ أَنْوَاعِ الْعَزْفِ أَنْ تَخِرُّوا وَتَسْجُدُوا لِتِمْثَالِ الذَّهَبِ الَّذِي نَصَبَهُ نَبُوخَذْنَصَّرُ الْمَلِكُ.
\par 6 وَمَنْ لاَ يَخِرُّ وَيَسْجُدُ فَفِي تِلْكَ السَّاعَةِ يُلْقَى فِي وَسَطِ أَتُونِ نَارٍ مُتَّقِدَةٍ].
\par 7 لأَجْلِ ذَلِكَ وَقْتَمَا سَمِعَ كُلُّ الشُّعُوبِ صَوْتَ الْقَرْنِ وَالنَّايِ وَالْعُودِ وَالرَّبَابِ وَالسِّنْطِيرِ وَكُلِّ أَنْوَاعِ الْعَزْفِ خَرَّ كُلُّ الشُّعُوبِ وَالأُمَمِ وَالأَلْسِنَةِ وَسَجَدُوا لِتِمْثَالِ الذَّهَبِ الَّذِي نَصَبَهُ نَبُوخَذْنَصَّرُ الْمَلِكُ.
\par 8 لأَجْلِ ذَلِكَ تَقَدَّمَ حِينَئِذٍ رِجَالٌ كِلْدَانِيُّونَ وَاشْتَكُوا عَلَى الْيَهُودِ
\par 9 وَقَالُوا لِلْمَلِكِ نَبُوخَذْنَصَّرَ: [أَيُّهَا الْمَلِكُ عِشْ إِلَى الأَبَدِ!
\par 10 أَنْتَ أَيُّهَا الْمَلِكُ قَدْ أَصْدَرْتَ أَمْراً بِأَنَّ كُلَّ إِنْسَانٍ يَسْمَعُ صَوْتَ الْقَرْنِ وَالنَّايِ وَالْعُودِ وَالرَّبَابِ وَالسِّنْطِيرِ وَالْمِزْمَارِ وَكُلِّ أَنْوَاعِ الْعَزْفِ يَخِرُّ وَيَسْجُدُ لِتِمْثَالِ الذَّهَبِ.
\par 11 وَمَنْ لاَ يَخِرُّ وَيَسْجُدُ فَإِنَّهُ يُلْقَى فِي وَسَطِ أَتُونِ نَارٍ مُتَّقِدَةٍ.
\par 12 يُوجَدُ رِجَالٌ يَهُودٌ الَّذِينَ وَكَّلْتَهُمْ عَلَى أَعْمَالِ وِلاَيَةِ بَابِلَ: شَدْرَخُ وَمِيشَخُ وَعَبْدَنَغُو. هَؤُلاَءِ الرِّجَالُ لَمْ يَجْعَلُوا لَكَ أَيُّهَا الْمَلِكُ اعْتِبَاراً. آلِهَتُكَ لاَ يَعْبُدُونَ وَلِتِمْثَالِ الذَّهَبِ الَّذِي نَصَبْتَ لاَ يَسْجُدُونَ].
\par 13 حِينَئِذٍ أَمَرَ نَبُوخَذْنَصَّرُ بِغَضَبٍ وَغَيْظٍ بِإِحْضَارِ شَدْرَخَ وَمِيشَخَ وَعَبْدَنَغُوَ. فَأَتُوا بِهَؤُلاَءِ الرِّجَالِ قُدَّامَ الْمَلِكِ.
\par 14 فَسأَلَهُمْ نَبُوخَذْنَصَّرُ: [تَعَمُّداً يَا شَدْرَخُ وَمِيشَخُ وَعَبْدَنَغُو لاَ تَعْبُدُونَ آلِهَتِي وَلاَ تَسْجُدُونَ لِتِمْثَالِ الذَّهَبِ الَّذِي نَصَبْتُ؟
\par 15 فَإِنْ كُنْتُمُ الآنَ مُسْتَعِدِّينَ عِنْدَمَا تَسْمَعُونَ صَوْتَ الْقَرْنِ وَالنَّايِ وَالْعُودِ وَالرَّبَابِ وَالسِّنْطِيرِ وَالْمِزْمَارِ وَكُلِّ أَنْوَاعِ الْعَزْفِ إِلَى أَنْ تَخِرُّوا وَتَسْجُدُوا لِلتِّمْثَالِ الَّذِي عَمِلْتُهُ. وَإِنْ لَمْ تَسْجُدُوا فَفِي تِلْكَ السَّاعَةِ تُلْقَوْنَ فِي وَسَطِ أَتُونِ النَّارِ الْمُتَّقِدَةِ. وَمَنْ هُوَ الإِلَهُ الَّذِي يُنْقِذُكُمْ مِنْ يَدَيَّ؟]
\par 16 فَأَجَابَ شَدْرَخُ وَمِيشَخُ وَعَبْدَنَغُو: [يَا نَبُوخَذْنَصَّرُ لاَ يَلْزَمُنَا أَنْ نُجِيبَكَ عَنْ هَذَا الأَمْرِ.
\par 17 هُوَذَا يُوجَدُ إِلَهُنَا الَّذِي نَعْبُدُهُ يَسْتَطِيعُ أَنْ يُنَجِّيَنَا مِنْ أَتُونِ النَّارِ الْمُتَّقِدَةِ وَأَنْ يُنْقِذَنَا مِنْ يَدِكَ أَيُّهَا الْمَلِكُ.
\par 18 وَإِلاَّ فَلِْيَكُنْ مَعْلُوماً لَكَ أَيُّهَا الْمَلِكُ أَنَّنَا لاَ نَعْبُدُ آلِهَتَكَ وَلاَ نَسْجُدُ لِتِمْثَالِ الذَّهَبِ الَّذِي نَصَبْتَهُ].
\par 19 حِينَئِذٍ امْتَلَأَ نَبُوخَذْنَصَّرُ غَيْظاً وَتَغَيَّرَ مَنْظَرُ وَجْهِهِ عَلَى شَدْرَخَ وَمِيشَخَ وَعَبْدَنَغُو وَأَمَرَ بِأَنْ يَحْمُوا الأَتُونَ سَبْعَةَ أَضْعَافٍ أَكْثَرَ مِمَّا كَانَ مُعْتَاداً أَنْ يُحْمَى.
\par 20 وَأَمَرَ جَبَابِرَةَ الْقُوَّةِ فِي جَيْشِهِ بِأَنْ يُوثِقُوا شَدْرَخَ وَمِيشَخَ وَعَبْدَنَغُوَ وَيُلْقُوهُمْ فِي أَتُونِ النَّارِ الْمُتَّقِدَةِ.
\par 21 ثُمَّ أُوثِقَ هَؤُلاَءِ الرِّجَالُ فِي سَرَاوِيلِهِمْ وَأَقْمِصَتِهِمْ وَأَرْدِيَتِهِمْ وَلِبَاسِهِمْ وَأُلْقُوا فِي وَسَطِ أَتُونِ النَّارِ الْمُتَّقِدَةِ.
\par 22 وَمِنْ حَيْثُ إِنَّ كَلِمَةَ الْمَلِكِ شَدِيدَةٌ وَالأَتُونَ قَدْ حَمِيَ جِدّاً قَتَلَ لَهِيبُ النَّارِ الرِّجَالَ الَّذِينَ رَفَعُوا شَدْرَخَ وَمِيشَخَ وَعَبْدَنَغُوَ.
\par 23 وَهَؤُلاَءِ الثَّلاَثَةُ الرِّجَالِ شَدْرَخُ وَمِيشَخُ وَعَبْدَنَغُو سَقَطُوا مُوثَقِينَ فِي وَسَطِ أَتُونِ النَّارِ الْمُتَّقِدَةِ.
\par 24 حِينَئِذٍ تَحَيَّرَ نَبُوخَذْنَصَّرُ الْمَلِكُ وَقَامَ مُسْرِعاً وَسَأَلَ مُشِيرِيهِ: [أَلَمْ نُلْقِ ثَلاَثَةَ رِجَالٍ مُوثَقِينَ فِي وَسَطِ النَّارِ؟] فَأَجَابُوا: [صَحِيحٌ أَيُّهَا الْمَلِكُ].
\par 25 فَقَالَ: [هَا أَنَا نَاظِرٌ أَرْبَعَةَ رِجَالٍ مَحْلُولِينَ يَتَمَشُّونَ فِي وَسَطِ النَّارِ وَمَا بِهِمْ ضَرَرٌ وَمَنْظَرُ الرَّابِعِ شَبِيهٌ بِابْنِ الآلِهَةِ].
\par 26 ثُمَّ اقْتَرَبَ نَبُوخَذْنَصَّرُ إِلَى بَابِ أَتُونِ النَّارِ الْمُتَّقِدَةِ وَنَادَى: [يَا شَدْرَخُ وَمِيشَخُ وَعَبْدَنَغُو يَا عَبِيدَ اللَّهِ الْعَلِيِّ اخْرُجُوا وَتَعَالُوا]. فَخَرَجَ شَدْرَخُ وَمِيشَخُ وَعَبْدَنَغُو مِنْ وَسَطِ النَّارِ.
\par 27 فَاجْتَمَعَتِ الْمَرَازِبَةُ وَالشِّحَنُ وَالْوُلاَةُ وَمُشِيرُو الْمَلِكِ وَرَأُوا هَؤُلاَءِ الرِّجَالَ الَّذِينَ لَمْ تَكُنْ لِلنَّارِ قُوَّةٌ عَلَى أَجْسَامِهِمْ وَشَعْرَةٌ مِنْ رُؤُوسِهِمْ لَمْ تَحْتَرِقْ وَسَرَاوِيلُهُمْ لَمْ تَتَغَيَّرْ وَرَائِحَةُ النَّارِ لَمْ تَأْتِ عَلَيْهِمْ.
\par 28 فَقَالَ نَبُوخَذْنَصَّرُ: [تَبَارَكَ إِلَهُ شَدْرَخَ وَمِيشَخَ وَعَبْدَنَغُو الَّذِي أَرْسَلَ مَلاَكَهُ وَأَنْقَذَ عَبِيدَهُ الَّذِينَ اتَّكَلُوا عَلَيْهِ وَغَيَّرُوا كَلِمَةَ الْمَلِكِ وَأَسْلَمُوا أَجْسَادَهُمْ لِكَيْ لاَ يَعْبُدُوا أَوْ يَسْجُدُوا لِإِلَهٍ غَيْرِ إِلَهِهِمْ.
\par 29 فَمِنِّي قَدْ صَدَرَ أَمْرٌ بِأَنَّ كُلَّ شَعْبٍ وَأُمَّةٍ وَلِسَانٍ يَتَكَلَّمُونَ بِالسُّوءِ عَلَى إِلَهِ شَدْرَخَ وَمِيشَخَ وَعَبْدَنَغُو فَإِنَّهُمْ يُصَيَّرُونَ إِرْباً إِرْباً وَتُجْعَلُ بُيُوتُهُمْ مَزْبَلَةً إِذْ لَيْسَ إِلَهٌ آخَرُ يَسْتَطِيعُ أَنْ يُنَجِّيَ هَكَذَا].
\par 30 حِينَئِذٍ قَدَّمَ الْمَلِكُ شَدْرَخَ وَمِيشَخَ وَعَبْدَنَغُوَ فِي وِلاَيَةِ بَابِلَ.

\chapter{4}

\par 1 مِنْ نَبُوخَذْنَصَّرَ الْمَلِكِ إِلَى كُلِّ الشُّعُوبِ وَالأُمَمِ وَالأَلْسِنَةِ السَّاكِنِينَ فِي الأَرْضِ كُلِّهَا. لِيَكْثُرْ سَلاَمُكُمْ.
\par 2 اَلآيَاتُ وَالْعَجَائِبُ الَّتِي صَنَعَهَا مَعِي اللَّهُ الْعَلِيُّ حَسُنَ عِنْدِي أَنْ أُخْبِرَ بِهَا.
\par 3 آيَاتُهُ مَا أَعْظَمَهَا وَعَجَائِبُهُ مَا أَقْوَاهَا! مَلَكُوتُهُ مَلَكُوتٌ أَبَدِيٌّ وَسُلْطَانُهُ إِلَى دَوْرٍ فَدَوْرٍ.
\par 4 أَنَا نَبُوخَذْنَصَّرُ قَدْ كُنْتُ مُطْمَئِنّاً فِي بَيْتِي وَنَاضِراً فِي قَصْرِي.
\par 5 رَأَيْتُ حُلْماً فَرَوَّعَنِي وَالأَفْكَارُ عَلَى فِرَاشِي وَرُؤَى رَأْسِي أَفْزَعَتْنِي.
\par 6 فَصَدَرَ مِنِّي أَمْرٌ بِإِحْضَارِ جَمِيعِ حُكَمَاءِ بَابِلَ قُدَّامِي لِيُعَرِّفُونِي بِتَعْبِيرِ الْحُلْمِ.
\par 7 حِينَئِذٍ حَضَرَ الْمَجُوسُ وَالسَّحَرَةُ وَالْكِلْدَانِيُّونَ وَالْمُنَجِّمُونَ وَقَصَصْتُ الْحُلْمَ عَلَيْهِمْ فَلَمْ يُعَرِّفُونِي بِتَعْبِيرِهِ.
\par 8 أَخِيراً دَخَلَ قُدَّامِي دَانِيآلُ الَّذِي اسْمُهُ بَلْطَشَاصَّرُ كَاسْمِ إِلَهِي وَالَّذِي فِيهِ رُوحُ الآلِهَةِ الْقُدُّوسِينَ فَقَصَصْتُ الْحُلْمَ قُدَّامَهُ.
\par 9 [يَا بَلْطَشَاصَّرُ كَبِيرُ الْمَجُوسِ مِنْ حَيْثُ إِنِّي أَعْلَمُ أَنَّ فِيكَ رُوحَ الآلِهَةِ الْقُدُّوسِينَ وَلاَ يَعْسُرُ عَلَيْكَ سِرٌّ فَأَخْبِرْنِي بِرُؤَى حُلْمِي الَّذِي رَأَيْتُهُ وَبِتَعْبِيرِهِ.
\par 10 فَرُؤَى رَأْسِي عَلَى فِرَاشِي هِيَ أَنِّي كُنْتُ أَرَى فَإِذَا بِشَجَرَةٍ فِي وَسَطِ الأَرْضِ وَطُولُهَا عَظِيمٌ.
\par 11 فَكَبُرَتِ الشَّجَرَةُ وَقَوِيَتْ فَبَلَغَ عُلُوُّهَا إِلَى السَّمَاءِ وَمَنْظَرُهَا إِلَى أَقْصَى كُلِّ الأَرْضِ.
\par 12 أَوْرَاقُهَا جَمِيلَةٌ وَثَمَرُهَا كَثِيرٌ وَفِيهَا طَعَامٌ لِلْجَمِيعِ وَتَحْتَهَا اسْتَظَلَّ حَيَوَانُ الْبَرِّ وَفِي أَغْصَانِهَا سَكَنَتْ طُيُورُ السَّمَاءِ وَطَعِمَ مِنْهَا كُلُّ الْبَشَرِ.
\par 13 كُنْتُ أَرَى فِي رُؤَى رَأْسِي عَلَى فِرَاشِي وَإِذَا بِسَاهِرٍ وَقُدُّوسٍ نَزَلَ مِنَ السَّمَاءِ
\par 14 فَصَرَخَ بِشِدَّةٍ: [اقْطَعُوا الشَّجَرَةَ وَاقْضِبُوا أَغْصَانَهَا وَانْثُرُوا أَوْرَاقَهَا وَابْذُرُوا ثَمَرَهَا لِيَهْرُبَ الْحَيَوَانُ مِنْ تَحْتِهَا وَالطُّيُورُ مِنْ أَغْصَانِهَا.
\par 15 وَلَكِنِ اتْرُكُوا سَاقَ أَصْلِهَا فِي الأَرْضِ وَبِقَيْدٍ مِنْ حَدِيدٍ وَنُحَاسٍ فِي عُشْبِ الْحَقْلِ وَلْيَبْتَلَّ بِنَدَى السَّمَاءِ وَلْيَكُنْ نَصِيبُهُ مَعَ الْحَيَوَانِ فِي عُشْبِ الْحَقْلِ.
\par 16 لِيَتَغَيَّرْ قَلْبُهُ عَنِ الإِنْسَانِيَّةِ وَلِْيُعْطَ قَلْبَ حَيَوَانٍ وَلْتَمْضِ عَلَيْهِ سَبْعَةُ أَزْمِنَةٍ.
\par 17 هَذَا الأَمْرُ بِقَضَاءِ السَّاهِرِينَ وَالْحُكْمُ بِكَلِمَةِ الْقُدُّوسِينَ لِتَعْلَمَ الأَحْيَاءُ أَنَّ الْعَلِيَّ مُتَسَلِّطٌ فِي مَمْلَكَةِ النَّاسِ فَيُعْطِيهَا مَنْ يَشَاءُ وَيُنَصِّبَ عَلَيْهَا أَدْنَى النَّاسِ.
\par 18 هَذَا الْحُلْمُ رَأَيْتُهُ أَنَا نَبُوخَذْنَصَّرَ الْمَلِكَ. أَمَّا أَنْتَ يَا بَلْطَشَاصَّرُ فَبَيِّنْ تَعْبِيرَهُ لأَنَّ كُلَّ حُكَمَاءِ مَمْلَكَتِي لاَ يَسْتَطِيعُونَ أَنْ يُعَرِّفُونِي بِالتَّعْبِيرِ. أَمَّا أَنْتَ فَتَسْتَطِيعُ لأَنَّ فِيكَ رُوحَ الآلِهَةِ الْقُدُّوسِينَ].
\par 19 حِينَئِذٍ تَحَيَّرَ دَانِيآلُ (الَّذِي اسْمُهُ بَلْطَشَاصَّرُ) سَاعَةً وَاحِدَةً وَأَفْزَعَتْهُ أَفْكَارُهُ. فَقَالَ الْمَلِكُ: [يَا بَلْطَشَاصَّرُ لاَ يُفْزِعُكَ الْحُلْمُ وَلاَ تَعْبِيرُهُ]. فَأَجَابَ بَلْطَشَاصَّرُ: [يَا سَيِّدِي الْحُلْمُ لِمُبْغِضِيكَ وَتَعْبِيرُهُ لأَعَادِيكَ.
\par 20 اَلشَّجَرَةُ الَّتِي رَأَيْتَهَا الَّتِي كَبِرَتْ وَقَوِيَتْ وَبَلَغَ عُلُوُّهَا إِلَى السَّمَاءِ وَمَنْظَرُهَا إِلَى كُلِّ الأَرْضِ
\par 21 وَأَوْرَاقُهَا جَمِيلَةٌ وَثَمَرُهَا كَثِيرٌ وَفِيهَا طَعَامٌ لِلْجَمِيعِ وَتَحْتَهَا سَكَنَ حَيَوَانُ الْبَرِّ وَفِي أَغْصَانِهَا سَكَنَتْ طُيُورُ السَّمَاءِ
\par 22 إِنَّمَا هِيَ أَنْتَ يَا أَيُّهَا الْمَلِكُ الَّذِي كَبِرْتَ وَتَقَوَّيْتَ وَعَظَمَتُكَ قَدْ زَادَتْ وَبَلَغَتْ إِلَى السَّمَاءِ وَسُلْطَانُكَ إِلَى أَقْصَى الأَرْضِ.
\par 23 وَحَيْثُ رَأَى الْمَلِكُ سَاهِراً وَقُدُّوساً نَزَلَ مِنَ السَّمَاءِ وَقَالَ: اقْطَعُوا الشَّجَرَةَ وَأَهْلِكُوهَا وَلَكِنِ اتْرُكُوا سَاقَ أَصْلِهَا فِي الأَرْضِ وَبِقَيْدٍ مِنْ حَدِيدٍ وَنُِحَاسٍ فِي عُشْبِ الْحَقْلِ وَلْيَبْتَلَّ بِنَدَى السَّمَاءِ وَلْيَكُنْ نَصِيبُهُ مَعَ حَيَوَانِ الْبَرِّ حَتَّى تَمْضِيَ عَلَيْهِ سَبْعَةُ أَزْمِنَةٍ.
\par 24 فَهَذَا هُوَ التَّعْبِيرُ أَيُّهَا الْمَلِكُ وَهَذَا هُوَ قَضَاءُ الْعَلِيِّ الَّذِي يَأْتِي عَلَى سَيِّدِي الْمَلِكِ:
\par 25 يَطْرُدُونَكَ مِنْ بَيْنِ النَّاسِ وَتَكُونُ سُكْنَاكَ مَعَ حَيَوَانِ الْبَرِّ وَيُطْعِمُونَكَ الْعُشْبَ كَالثِّيرَانِ وَيَبُلُّونَكَ بِنَدَى السَّمَاءِ فَتَمْضِي عَلَيْكَ سَبْعَةُ أَزْمِنَةٍ حَتَّى تَعْلَمَ أَنَّ الْعَلِيَّ مُتَسَلِّطٌ فِي مَمْلَكَةِ النَّاسِ وَيُعْطِيهَا مَنْ يَشَاءُ.
\par 26 وَحَيْثُ أَمَرُوا بِتَرْكِ سَاقِ أُصُولِ الشَّجَرَةِ فَإِنَّ مَمْلَكَتَكَ تَثْبُتُ لَكَ عِنْدَمَا تَعْلَمُ أَنَّ السَّمَاءَ سُلْطَانٌ.
\par 27 لِذَلِكَ أَيُّهَا الْمَلِكُ فَلْتَكُنْ مَشُورَتِي مَقْبُولَةً لَدَيْكَ وَفَارِقْ خَطَايَاكَ بِالْبِرِّ وَآثَامَكَ بِالرَّحْمَةِ لِلْمَسَاكِينِ لَعَلَّهُ يُطَالُ اطْمِئْنَانُكَ].
\par 28 كُلُّ هَذَا جَاءَ عَلَى نَبُوخَذْنَصَّرَ الْمَلِكِ.
\par 29 عِنْدَ نِهَايَةِ اثْنَيْ عَشَرَ شَهْراً كَانَ يَتَمَشَّى عَلَى قَصْرِ مَمْلَكَةِ بَابِلَ.
\par 30 فَقَالَ: [أَلَيْسَتْ هَذِهِ بَابِلَ الْعَظِيمَةَ الَّتِي بَنَيْتُهَا لِبَيْتِ الْمُلْكِ بِقُوَّةِ اقْتِدَارِي وَلِجَلاَلِ مَجْدِي!]
\par 31 وَالْكَلِمَةُ بَعْدُ بِفَمِ الْمَلِكِ وَقَعَ صَوْتٌ مِنَ السَّمَاءِ: [لَكَ يَقُولُونَ يَا نَبُوخَذْنَصَّرُ الْمَلِكُ إِنَّ الْمُلْكَ قَدْ زَالَ عَنْكَ
\par 32 وَيَطْرُدُونَكَ مِنْ بَيْنِ النَّاسِ وَتَكُونُ سُكْنَاكَ مَعَ حَيَوَانِ الْبَرِّ وَيُطْعِمُونَكَ الْعُشْبَ كَالثِّيرَانِ فَتَمْضِي عَلَيْكَ سَبْعَةُ أَزْمِنَةٍ حَتَّى تَعْلَمَ أَنَّ الْعَلِيَّ مُتَسَلِّطٌ فِي مَمْلَكَةِ النَّاسِ وَأَنَّهُ يُعْطِيهَا مَنْ يَشَاءُ].
\par 33 فِي تِلْكَ السَّاعَةِ تَمَّ الأَمْرُ عَلَى نَبُوخَذْنَصَّرَ فَطُرِدَ مِنْ بَيْنِ النَّاسِ وَأَكَلَ الْعُشْبَ كَالثِّيرَانِ وَابْتَلَّ جِسْمُهُ بِنَدَى السَّمَاءِ حَتَّى طَالَ شَعْرُهُ مِثْلَ النُّسُورِ وَأَظْفَارُهُ مِثْلَ الطُّيُورِ.
\par 34 وَعِنْدَ انْتِهَاءِ الأَيَّامِ: [أَنَا نَبُوخَذْنَصَّرُ رَفَعْتُ عَيْنَيَّ إِلَى السَّمَاءِ فَرَجَعَ إِلَيَّ عَقْلِي وَبَارَكْتُ الْعَلِيَّ وَسَبَّحْتُ وَحَمَدْتُ الْحَيَّ إِلَى الأَبَدِ الَّذِي سُلْطَانُهُ سُلْطَانٌ أَبَدِيٌّ وَمَلَكُوتُهُ إِلَى دَوْرٍ فَدَوْرٍ.
\par 35 وَحُسِبَتْ جَمِيعُ سُكَّانِ الأَرْضِ كَلاَ شَيْءَ وَهُوَ يَفْعَلُ كَمَا يَشَاءُ فِي جُنْدِ السَّمَاءِ وَسُكَّانِ الأَرْضِ وَلاَ يُوجَدُ مَنْ يَمْنَعُ يَدَهُ أَوْ يَقُولُ لَهُ: مَاذَا تَفْعَلُ؟
\par 36 فِي ذَلِكَ الْوَقْتِ رَجَعَ إِلَيَّ عَقْلِي وَعَادَ إِلَيَّ جَلاَلُ مَمْلَكَتِي وَمَجْدِي وَبَهَائِي وَطَلَبَنِي مُشِيرِيَّ وَعُظَمَائِي وَتَثَبَّتُّ عَلَى مَمْلَكَتِي وَازْدَادَتْ لِي عَظَمَةٌ كَثِيرَةٌ.
\par 37 فَالآنَ أَنَا نَبُوخَذْنَصَّرُ أُسَبِّحُ وَأُعَظِّمُ وَأَحْمَدُ مَلِكَ السَّمَاءِ الَّذِي كُلُّ أَعْمَالِهِ حَقٌّ وَطُرُقِهِ عَدْلٌ وَمَنْ يَسْلُكُ بِالْكِبْرِيَاءِ فَهُوَ قَادِرٌ عَلَى أَنْ يُذِلَّهُ].

\chapter{5}

\par 1 بَيْلْشَاصَّرُ الْمَلِكُ صَنَعَ وَلِيمَةً عَظِيمَةً لِعُظَمَائِهِ الأَلْفِ وَشَرِبَ خَمْراً قُدَّامَ الأَلْفِ.
\par 2 وَإِذْ كَانَ بَيْلْشَاصَّرُ يَذُوقُ الْخَمْرَ أَمَرَ بِإِحْضَارِ آنِيَةِ الذَّهَبِ وَالْفِضَّةِ الَّتِي أَخْرَجَهَا نَبُوخَذْنَصَّرُ أَبُوهُ مِنَ الْهَيْكَلِ الَّذِي فِي أُورُشَلِيمَ لِيَشْرَبَ بِهَا الْمَلِكُ وَعُظَمَاؤُهُ وَزَوْجَاتُهُ وَسَرَارِيهِ.
\par 3 حِينَئِذٍ أَحْضَرُوا آنِيَةَ الذَّهَبِ الَّتِي أُخْرِجَتْ مِنْ هَيْكَلِ بَيْتِ اللَّهِ الَّذِي فِي أُورُشَلِيمَ وَشَرِبَ بِهَا الْمَلِكُ وَعُظَمَاؤُهُ وَزَوْجَاتُهُ وَسَرَارِيهِ.
\par 4 كَانُوا يَشْرَبُونَ الْخَمْرَ وَيُسَبِّحُونَ آلِهَةَ الذَّهَبِ وَالْفِضَّةِ وَالنُّحَاسِ وَالْحَدِيدِ وَالْخَشَبِ وَالْحَجَرِ.
\par 5 فِي تِلْكَ السَّاعَةِ ظَهَرَتْ أَصَابِعُ يَدِ إِنْسَانٍ وَكَتَبَتْ بِإِزَاءِ النِّبْرَاسِ عَلَى مُكَلَّسِ حَائِطِ قَصْرِ الْمَلِكِ وَالْمَلِكُ يَنْظُرُ طَرَفَ الْيَدِ الْكَاتِبَةِ.
\par 6 حِينَئِذٍ تَغَيَّرَتْ هَيْئَةُ الْمَلِكِ وَأَفْزَعَتْهُ أَفْكَارُهُ وَانْحَلَّتْ خَرَزُ حَقَوَيْهِ وَاصْطَكَّتْ رُكْبَتَاهُ.
\par 7 فَصَرَخَ الْمَلِكُ بِشِدَّةٍ لِإِدْخَالِ السَّحَرَةِ وَالْكِلْدَانِيِّينَ وَالْمُنَجِّمِينَ وَقَالَ الْمَلِكُ لِحُكَمَاءِ بَابِلَ: [أَيُّ رَجُلٍ يَقْرَأُ هَذِهِ الْكِتَابَةَ وَيُبَيِّنُ لِي تَفْسِيرَهَا فَإِنَّهُ يُلَبَّسُ الأُرْجُوَانَ وَقِلاَدَةً مِنْ ذَهَبٍ فِي عُنُقِهِ وَيَتَسَلَّطُ ثَالِثاً فِي الْمَمْلَكَةِ].
\par 8 ثُمَّ دَخَلَ كُلُّ حُكَمَاءِ الْمَلِكِ فَلَمْ يَسْتَطِيعُوا أَنْ يَقْرَأُوا الْكِتَابَةَ وَلاَ أَنْ يُعَرِّفُوا الْمَلِكَ بِتَفْسِيرِهَا.
\par 9 فَفَزَعَ الْمَلِكُ بَيْلْشَاصَّرُ جِدّاً وَتَغَيَّرَتْ فِيهِ هَيْئَتُهُ وَاضْطَرَبَ عُظَمَاؤُهُ.
\par 10 أَمَّا الْمَلِكَةُ فَلِسَبَبِ كَلاَمِ الْمَلِكِ وَعُظَمَائِهِ دَخَلَتْ بَيْتَ الْوَلِيمَةِ وَقَالَتْ: [أَيُّهَا الْمَلِكُ عِشْ إِلَى الأَبَدِ! لاَ تُفَزِّعْكَ أَفْكَارُكَ وَلاَ تَتَغَيَّرْ هَيْئَتُكَ.
\par 11 يُوجَدُ فِي مَمْلَكَتِكَ رَجُلٌ فِيهِ رُوحُ الآلِهَةِ الْقُدُّوسِينَ وَفِي أَيَّامِ أَبِيكَ وُجِدَتْ فِيهِ نَيِّرَةٌ وَفِطْنَةٌ وَحِكْمَةٌ كَحِكْمَةِ الآلِهَةِ وَالْمَلِكُ نَبُوخَذْنَصَّرُ أَبُوكَ جَعَلَهُ كَبِيرَ الْمَجُوسِ وَالسَّحَرَةِ وَالْكِلْدَانِيِّينَ وَالْمُنَجِّمِينَ.
\par 12 مِنْ حَيْثُ إِنَّ رُوحاً فَاضِلَةً وَمَعْرِفَةً وَفِطْنَةً وَتَعْبِيرَ الأَحْلاَمِ وَتَبْيِينَ أَلْغَازٍ وَحَلَّ عُقَدٍ وُجِدَتْ فِي دَانِيآلَ هَذَا الَّذِي سَمَّاهُ الْمَلِكُ بَلْطَشَاصَّرَ. فَلْيُدْعَ الآنَ دَانِيآلُ فَيُبَيِّنَ التَّفْسِيرَ].
\par 13 حِينَئِذٍ أُدْخِلَ دَانِيآلُ إِلَى قُدَّامِ الْمَلِكِ. فَسَأَلَ الْمَلِكُ دَانِيآلَ: [أَأَنْتَ هُوَ دَانِيآلُ مِنْ بَنِي سَبْيِ يَهُوذَا الَّذِي جَلَبَهُ أَبِي الْمَلِكُ مِنْ يَهُوذَا؟
\par 14 قَدْ سَمِعْتُ عَنْكَ أَنَّ فِيكَ رُوحَ الآلِهَةِ وَأَنَّ فِيكَ نَيِّرَةً وَفِطْنَةً وَحِكْمَةً فَاضِلَةً.
\par 15 وَالآنَ أُدْخِلَ قُدَّامِي الْحُكَمَاءُ وَالسَّحَرَةُ لِيَقْرَأُوا هَذِهِ الْكِتَابَةَ وَيُعَرِّفُونِي بِتَفْسِيرِهَا فَلَمْ يَسْتَطِيعُوا أَنْ يُبَيِّنُوا تَفْسِيرَ الْكَلاَمِ.
\par 16 وَأَنَا قَدْ سَمِعْتُ عَنْكَ أَنَّكَ تَسْتَطِيعُ أَنْ تُفَسِّرَ تَفْسِيراً وَتَحِلَّ عُقَداً. فَإِنِ اسْتَطَعْتَ الآنَ أَنْ تَقْرَأَ الْكِتَابَةَ وَتُعَرِّفَنِي بِتَفْسِيرِهَا فَتُلَبَّسُ الأُرْجُوانَ وَقِلاَدَةً مِنْ ذَهَبٍ فِي عُنُقِكَ وَتَتَسَلَّطُ ثَالِثاً فِي الْمَمْلَكَةِ].
\par 17 فَأَجَابَ دَانِيآلُ الْمَلِكَ: [لِتَكُنْ عَطَايَاكَ لِنَفْسِكَ وَهَبْ هِبَاتِكَ لِغَيْرِي. لَكِنِّي أَقْرَأُ الْكِتَابَةَ لِلْمَلِكِ وَأُعَرِّفُهُ بِالتَّفْسِيرِ.
\par 18 أَنْتَ أَيُّهَا الْمَلِكُ فَاللَّهُ الْعَلِيُّ أَعْطَى أَبَاكَ نَبُوخَذْنَصَّرَ مَلَكُوتاً وَعَظَمَةً وَجَلاَلاً وَبَهَاءً.
\par 19 وَلِلْعَظَمَةِ الَّتِي أَعْطَاهُ إِيَّاهَا كَانَتْ تَرْتَعِدُ وَتَفْزَعُ قُدَّامَهُ جَمِيعُ الشُّعُوبِ وَالأُمَمِ وَالأَلْسِنَةِ. فَأَيّاً شَاءَ قَتَلَ وَأَيّاً شَاءَ اسْتَحْيَا وَأَيّاً شَاءَ رَفَعَ وَأَيّاً شَاءَ وَضَعَ.
\par 20 فَلَمَّا ارْتَفَعَ قَلْبُهُ وَقَسَتْ رُوحُهُ تَجَبُّراً انْحَطَّ عَنْ كُرْسِيِّ مُلْكِهِ وَنَزَعُوا عَنْهُ جَلاَلَهُ
\par 21 وَطُرِدَ مِنْ بَيْنِ النَّاسِ وَتَسَاوَى قَلْبُهُ بِالْحَيَوَانِ وَكَانَتْ سُكْنَاهُ مَعَ الْحَمِيرِ الْوَحْشِيَّةِ فَأَطْعَمُوهُ الْعُشْبَ كَالثِّيرَانِ وَابْتَلَّ جِسْمُهُ بِنَدَى السَّمَاءِ حَتَّى عَلِمَ أَنَّ اللَّهَ الْعَلِيَّ سُلْطَانٌ فِي مَمْلَكَةِ النَّاسِ وَأَنَّهُ يُقِيمُ عَلَيْهَا مَنْ يَشَاءُ.
\par 22 وَأَنْتَ يَا بَيْلْشَاصَّرُ ابْنَهُ لَمْ تَضَعْ قَلَبَكَ مَعَ أَنَّكَ عَرَفْتَ كُلَّ هَذَا.
\par 23 بَلْ تَعَظَّمْتَ عَلَى رَبِّ السَّمَاءِ فَأَحْضَرُوا قُدَّامَكَ آنِيَةَ بَيْتِهِ وَأَنْتَ وَعُظَمَاؤُكَ وَزَوْجَاتُكَ وَسَرَارِيكَ شَرِبْتُمْ بِهَا الْخَمْرَ وَسَبَّحْتَ آلِهَةَ الْفِضَّةِ وَالذَّهَبِ وَالنِّحَاسِ وَالْحَدِيدِ وَالْخَشَبِ وَالْحَجَرِ الَّتِي لاَ تُبْصِرُ وَلاَ تَسْمَعُ وَلاَ تَعْرِفُ. أَمَّا اللَّهُ الَّذِي بِيَدِهِ نَسَمَتُكَ وَلَهُ كُلُّ طُرُقِكَ فَلَمْ تُمَجِّدْهُ.
\par 24 حِينَئِذٍ أُرْسِلَ مِنْ قِبَلِهِ طَرَفُ الْيَدِ فَكُتِبَتْ هَذِهِ الْكِتَابَةُ.
\par 25 وَهَذِهِ هِيَ الْكِتَابَةُ الَّتِي سُطِّرَتْ: مَنَا مَنَا تَقَيْلُ وَفَرْسِينُ.
\par 26 وَهَذَا تَفْسِيرُ الْكَلاَمِ. [مَنَا] أَحْصَى اللَّهُ مَلَكُوتَكَ وَأَنْهَاهُ.
\par 27 [تَقَيْلُ] وُزِنْتَ بِالْمَوَازِينِ فَوُجِدْتَ نَاقِصاً.
\par 28 [فَرْسِ] قُسِمَتْ مَمْلَكَتُكَ وَأُعْطِيَتْ لِمَادِي وَفَارِسَ].
\par 29 حِينَئِذٍ أَمَرَ بَيْلْشَاصَّرُ أَنْ يُلْبِسُوا دَانِيآلَ الأَرْجُوانَ وَقِلاَدَةً مِنْ ذَهَبٍ فِي عُنُقِهِ وَيُنَادُوا عَلَيْهِ أَنَّهُ يَكُونُ مُتَسَلِّطاً ثَالِثاً فِي الْمَمْلَكَةِ.
\par 30 فِي تِلْكَ اللَّيْلَةِ قُتِلَ بَيْلْشَاصَّرُ مَلِكُ الْكِلْدَانِيِّينَ
\par 31 فَأَخَذَ الْمَمْلَكَةَ دَارِيُوسُ الْمَادِيُّ وَهُوَ ابْنُ اثْنَتَيْنِ وَسِتِّينَ سَنَةً.

\chapter{6}

\par 1 حَسُنَ عِنْدَ دَارِيُوسَ أَنْ يُوَلِّيَ عَلَى الْمَمْلَكَةِ مِئَةً وَعِشْرِينَ مَرْزُبَاناً يَكُونُونَ عَلَى الْمَمْلَكَةِ كُلِّهَا.
\par 2 وَعَلَى هَؤُلاَءِ ثَلاَثَةَ وُزَرَاءَ أَحَدُهُمْ دَانِيآلُ لِتُؤَدِّيَ الْمَرَازِبَةُ إِلَيْهِمِ الْحِسَابَ فَلاَ تُصِيبَ الْمَلِكَ خَسَارَةٌ.
\par 3 فَفَاقَ دَانِيآلُ هَذَا عَلَى الْوُزَرَاءِ وَالْمَرَازِبَةِ لأَنَّ فِيهِ رُوحاً فَاضِلَةً. وَفَكَّرَ الْمَلِكُ فِي أَنْ يُوَلِّيَهُ عَلَى الْمَمْلَكَةِ كُلِّهَا.
\par 4 ثُمَّ إِنَّ الْوُزَرَاءَ وَالْمَرَازِبَةَ كَانُوا يَطْلُبُونَ عِلَّةً يَجِدُونَهَا عَلَى دَانِيآلَ مِنْ جِهَةِ الْمَمْلَكَةِ فَلَمْ يَقْدِرُوا أَنْ يَجِدُوا عِلَّةً وَلاَ ذَنْباً لأَنَّهُ كَانَ أَمِيناً وَلَمْ يُوجَدْ فِيهِ خَطَأٌ وَلاَ ذَنْبٌ.
\par 5 فَقَالَ هَؤُلاَءِ الرِّجَالُ: [لاَ نَجِدُ عَلَى دَانِيآلَ هَذَا عِلَّةً إِلاَّ أَنْ نَجِدَهَا مِنْ جِهَةِ شَرِيعَةِ إِلَهِهِ].
\par 6 حِينَئِذٍ اجْتَمَعَ هَؤُلاَءِ الْوُزَرَاءُ وَالْمَرَازِبَةُ عِنْدَ الْمَلِكِ وَقَالُوا لَهُ: [أَيُّهَا الْمَلِكُ دَارِيُوسُ عِشْ إِلَى الأَبَدِ!
\par 7 إِنَّ جَمِيعَ وُزَرَاءِ الْمَمْلَكَةِ وَالشِّحَنِ وَالْمَرَازِبَةِ وَالْمُشِيرِينَ وَالْوُلاَةِ قَدْ تَشَاوَرُوا عَلَى أَنْ يَضَعُوا أَمْراً مَلَكِيّاً وَيُشَدِّدُوا نَهْياً بِأَنَّ كُلَّ مَنْ يَطْلُبُ طِلْبَةً حَتَّى ثَلاَثِينَ يَوْماً مِنْ إِلَهٍ أَوْ إِنْسَانٍ إِلاَّ مِنْكَ أَيُّهَا الْمَلِكُ يُطْرَحُ فِي جُبِّ الأُسُودِ.
\par 8 فَثَبِّتِ الآنَ النَّهْيَ أَيُّهَا الْمَلِكُ وَأَمْضِ الْكِتَابَةَ لِكَيْ لاَ تَتَغَيَّرَ كَشَرِيعَةِ مَادِي وَفَارِسَ الَّتِي لاَ تُنْسَخُ].
\par 9 لأَجْلِ ذَلِكَ أَمْضَى الْمَلِكُ دَارِيُوسُ الْكِتَابَةَ وَالنَّهْيَ.
\par 10 فَلَمَّا عَلِمَ دَانِيآلُ بِإِمْضَاءِ الْكِتَابَةِ ذَهَبَ إِلَى بَيْتِهِ وَكُواهُ مَفْتُوحَةٌ فِي عُلِّيَّتِهِ نَحْوَ أُورُشَلِيمَ فَجَثَا عَلَى رُكْبَتَيْهِ ثَلاَثَ مَرَّاتٍ فِي الْيَوْمِ وَصَلَّى وَحَمَدَ قُدَّامَ إِلَهِهِ كَمَا كَانَ يَفْعَلُ قَبْلَ ذَلِكَ.
\par 11 فَاجْتَمَعَ حِينَئِذٍ هَؤُلاَءِ الرِّجَالُ فَوَجَدُوا دَانِيآلَ يَطْلُبُ وَيَتَضَرَّعُ قُدَّامَ إِلَهِهِ.
\par 12 فَتَقَدَّمُوا وَتَكَلَّمُوا قُدَّامَ الْمَلِكِ فِي نَهْيِ الْمَلِكِ: [أَلَمْ تُمْضِ أَيُّهَا الْمَلِكُ نَهْياً بِأَنَّ كُلَّ إِنْسَانٍ يَطْلُبُ مِنْ إِلَهٍ أَوْ إِنْسَانٍ حَتَّى ثَلاَثِينَ يَوْماً إِلاَّ مِنْكَ أَيُّهَا الْمَلِكُ يُطْرَحُ فِي جُبِّ الأُسُودِ؟] فَأَجَابَ الْمَلِكُ: [الأَمْرُ صَحِيحٌ كَشَرِيعَةِ مَادِي وَفَارِسَ الَّتِي لاَ تُنْسَخُ].
\par 13 حِينَئِذٍ قَالُوا لِلْمَلِكِ: [إِنَّ دَانِيآلَ الَّذِي مِنْ بَنِي سَبْيِ يَهُوذَا لَمْ يَجْعَلْ لَكَ أَيُّهَا الْمَلِكُ اعْتِبَاراً وَلاَ لِلنَّهْيِ الَّذِي أَمْضَيْتَهُ بَلْ ثَلاَثَ مَرَّاتٍ فِي الْيَوْمِ يَطْلُبُ طِلْبَتَهُ].
\par 14 فَلَمَّا سَمِعَ الْمَلِكُ هَذَا الْكَلاَمَ اغْتَاظَ عَلَى نَفْسِهِ جِدّاً وَجَعَلَ قَلْبَهُ عَلَى دَانِيآلَ لِيُنَجِّيَهُ وَاجْتَهَدَ إِلَى غُرُوبِ الشَّمْسِ لِيُنْقِذَهُ.
\par 15 فَاجْتَمَعَ أُولَئِكَ الرِّجَالُ إِلَى الْمَلِكِ وَقَالُوا: [اعْلَمْ أَيُّهَا الْمَلِكُ أَنَّ شَرِيعَةَ مَادِي وَفَارِسَ هِيَ أَنَّ كُلَّ نَهْيٍ أَوْ أَمْرٍ يَضَعُهُ الْمَلِكُ لاَ يَتَغَيَّرُ].
\par 16 حِينَئِذٍ أَمَرَ الْمَلِكُ فَأَحْضَرُوا دَانِيآلَ وَطَرَحُوهُ فِي جُبِّ الأُسُودِ. وَقَالَ الْمَلِكُ لِدَانِيآلَ: [إِنَّ إِلَهَكَ الَّذِي تَعْبُدُهُ دَائِماً هُوَ يُنَجِّيكَ].
\par 17 وَأُتِيَ بِحَجَرٍ وَوُضِعَ عَلَى فَمِ الْجُبِّ وَخَتَمَهُ الْمَلِكُ بِخَاتِمِهِ وَخَاتِمِ عُظَمَائِهِ لِئَلاَّ يَتَغَيَّرَ الْقَصْدُ فِي دَانِيآلَ.
\par 18 حِينَئِذٍ مَضَى الْمَلِكُ إِلَى قَصْرِهِ وَبَاتَ صَائِماً وَلَمْ يُؤْتَ قُدَّامَهُ بِسَرَارِيهِ وَطَارَ عَنْهُ نَوْمُهُ.
\par 19 ثُمَّ قَامَ الْمَلِكُ بَاكِراً عِنْدَ الْفَجْرِ وَذَهَبَ مُسْرِعاً إِلَى جُبِّ الأُسُودِ.
\par 20 فَلَمَّا اقْتَرَبَ إِلَى الْجُبِّ نَادَى دَانِيآلَ بِصَوْتٍ أَسِيفٍ: [يَا دَانِيآلُ عَبْدَ اللَّهِ الْحَيِّ هَلْ إِلَهُكَ الَّذِي تَعْبُدُهُ دَائِماً قَدِرَ عَلَى أَنْ يُنَجِّيَكَ مِنَ الأُسُودِ؟]
\par 21 فَتَكَلَّمَ دَانِيآلُ مَعَ الْمَلِكِ: [يَا أَيُّهَا الْمَلِكُ عِشْ إِلَى الأَبَدِ!
\par 22 إِلَهِي أَرْسَلَ مَلاَكَهُ وَسَدَّ أَفْوَاهَ الأُسُودِ فَلَمْ تَضُرَّنِي لأَنِّي وُجِدْتُ بَرِيئاً قُدَّامَهُ وَقُدَّامَكَ أَيْضاً أَيُّهَا الْمَلِكُ. لَمْ أَفْعَلْ ذَنْباً].
\par 23 حِينَئِذٍ فَرِحَ الْمَلِكُ بِهِ وَأَمَرَ بِأَنْ يُصْعَدَ دَانِيآلُ مِنَ الْجُبِّ. فَأُصْعِدَ دَانِيآلُ مِنَ الْجُبِّ وَلَمْ يُوجَدْ فِيهِ ضَرَرٌ لأَنَّهُ آمَنَ بِإِلَهِهِ.
\par 24 فَأَمَرَ الْمَلِكُ فَأَحْضَرُوا أُولَئِكَ الرِّجَالَ الَّذِينَ اشْتَكُوا عَلَى دَانِيآلَ وَطَرَحُوهُمْ فِي جُبِّ الأُسُودِ هُمْ وَأَوْلاَدَهُمْ وَنِسَاءَهُمْ. وَلَمْ يَصِلُوا إِلَى أَسْفَلِ الْجُبِّ حَتَّى بَطَشَتْ بِهِمِ الأُسُودُ وَسَحَقَتْ كُلَّ عِظَامِهِمْ.
\par 25 ثُمَّ كَتَبَ الْمَلِكُ دَارِيُوسُ إِلَى كُلِّ الشُّعُوبِ وَالأُمَمِ وَالأَلْسِنَةِ السَّاكِنِينَ فِي الأَرْضِ كُلِّهَا: [لِيَكْثُرْ سَلاَمُكُمْ.
\par 26 مِنْ قِبَلِي صَدَرَ أَمْرٌ بِأَنَّهُ فِي كُلِّ سُلْطَانِ مَمْلَكَتِي يَرْتَعِدُونَ وَيَخَافُونَ قُدَّامَ إِلَهِ دَانِيآلَ لأَنَّهُ هُوَ الإِلَهُ الْحَيُّ الْقَيُّومُ إِلَى الأَبَدِ وَمَلَكُوتُهُ لَنْ يَزُولَ وَسُلْطَانُهُ إِلَى الْمُنْتَهَى.
\par 27 هُوَ يُنَجِّي وَيُنْقِذُ وَيَعْمَلُ الآيَاتِ وَالْعَجَائِبَ فِي السَّمَاوَاتِ وَفِي الأَرْضِ. هُوَ الَّذِي نَجَّى دَانِيآلَ مِنْ يَدِ الأُسُودِ].
\par 28 فَنَجَحَ دَانِيآلُ هَذَا فِي مُلْكِ دَارِيُوسَ وَفِي مُلْكِ كُورَشَ الْفَارِسِيِّ.

\chapter{7}

\par 1 فِي السَّنَةِ الأُولَى لِبَيْلْشَاصَّرَ مَلِكِ بَابِلَ رَأَى دَانِيآلُ حُلْماً وَرُؤَى رَأْسِهِ عَلَى فِرَاشِهِ. حِينَئِذٍ كَتَبَ الْحُلْمَ وَأَخْبَرَ بِرَأْسِ الْكَلاَمِ.
\par 2 قَالَ دَانِيآلُ: [كُنْتُ أَرَى فِي رُؤْيَايَ لَيْلاً وَإِذَا بِأَرْبَعِ رِيَاحِ السَّمَاءِ هَجَمَتْ عَلَى الْبَحْرِ الْكَبِيرِ.
\par 3 وَصَعِدَ مِنَ الْبَحْرِ أَرْبَعَةُ حَيَوَانَاتٍ عَظِيمَةٍ هَذَا مُخَالِفٌ ذَاكَ.
\par 4 الأَوَّلُ كَالأَسَدِ وَلَهُ جَنَاحَا نَسْرٍ. وَكُنْتُ أَنْظُرُ حَتَّى انْتَتَفَ جَنَاحَاهُ وَانْتَصَبَ عَنِ الأَرْضِ وَأُوقِفَ عَلَى رِجْلَيْنِ كَإِنْسَانٍ وَأُعْطِيَ قَلْبَ إِنْسَانٍ.
\par 5 وَإِذَا بِحَيَوَانٍ آخَرَ ثَانٍ شَبِيهٍ بِالدُّبِّ فَارْتَفَعَ عَلَى جَنْبٍ وَاحِدٍ وَفِي فَمِهِ ثَلاَثُ أَضْلُعٍ بَيْنَ أَسْنَانِهِ فَقَالُوا لَهُ: [قُمْ كُلْ لَحْماً كَثِيراً.
\par 6 وَبَعْدَ هَذَا كُنْتُ أَرَى وَإِذَا بِآخَرَ مِثْلِ النَّمِرِ وَلَهُ عَلَى ظَهْرِهِ أَرْبَعَةُ أَجْنِحَةِ طَائِرٍ. وَكَانَ لِلْحَيَوَانِ أَرْبَعَةُ رُؤُوسٍ وَأُعْطِيَ سُلْطَاناً.
\par 7 بَعْدَ هَذَا كُنْتُ أَرَى فِي رُؤَى اللَّيْلِ وَإِذَا بِحَيَوَانٍ رَابِعٍ هَائِلٍ وَقَوِيٍّ وَشَدِيدٍ جِدّاً وَلَهُ أَسْنَانٌ مِنْ حَدِيدٍ كَبِيرَةٌ. أَكَلَ وَسَحَقَ وَدَاسَ الْبَاقِيَ بِرِجْلَيْهِ. وَكَانَ مُخَالِفاً لِكُلِّ الْحَيَوَانَاتِ الَّذِينَ قَبْلَهُ. وَلَهُ عَشَرَةُ قُرُونٍ.
\par 8 كُنْتُ مُتَأَمِّلاً بِالْقُرُونِ وَإِذَا بِقَرْنٍ آخَرَ صَغِيرٍ طَلَعَ بَيْنَهَا وَقُلِعَتْ ثَلاَثَةٌ مِنَ الْقُرُونِ الأُولَى مِنْ قُدَّامِهِ وَإِذَا بِعُيُونٍ كَعُيُونِ الإِنْسَانِ فِي هَذَا الْقَرْنِ وَفَمٍ مُتَكَلِّمٍ بِعَظَائِمَ.
\par 9 كُنْتُ أَرَى أَنَّهُ وُضِعَتْ عُرُوشٌ وَجَلَسَ الْقَدِيمُ الأَيَّامِ. لِبَاسُهُ أَبْيَضُ كَالثَّلْجِ وَشَعْرُ رَأْسِهِ كَالصُّوفِ النَّقِيِّ وَعَرْشُهُ لَهِيبُ نَارٍ وَبَكَرَاتُهُ نَارٌ مُتَّقِدَةٌ.
\par 10 نَهْرُ نَارٍ جَرَى وَخَرَجَ مِنْ قُدَّامِهِ. أُلُوفُ أُلُوفٍ تَخْدِمُهُ وَرَبَوَاتُ رَبَوَاتٍ وُقُوفٌ قُدَّامَهُ. فَجَلَسَ الدِّينُ وَفُتِحَتِ الأَسْفَارُ.
\par 11 كُنْتُ أَنْظُرُ حِينَئِذٍ مِنْ أَجْلِ صَوْتِ الْكَلِمَاتِ الْعَظِيمَةِ الَّتِي تَكَلَّمَ بِهَا الْقَرْنُ. كُنْتُ أَرَى إِلَى أَنْ قُتِلَ الْحَيَوَانُ وَهَلَكَ جِسْمُهُ وَدُفِعَ لِوَقِيدِ النَّارِ.
\par 12 أَمَّا بَاقِي الْحَيَوَانَاتِ فَنُزِعَ عَنْهُمْ سُلْطَانُهُمْ وَلَكِنْ أُعْطُوا طُولَ حَيَاةٍ إِلَى زَمَانٍ وَوَقْتٍ.
\par 13 [كُنْتُ أَرَى فِي رُؤَى اللَّيْلِ وَإِذَا مَعَ سُحُبِ السَّمَاءِ مِثْلُ ابْنِ إِنْسَانٍ أَتَى وَجَاءَ إِلَى الْقَدِيمِ الأَيَّامِ فَقَرَّبُوهُ قُدَّامَهُ.
\par 14 فَأُعْطِيَ سُلْطَاناً وَمَجْداً وَمَلَكُوتاً لِتَتَعَبَّدَ لَهُ كُلُّ الشُّعُوبِ وَالأُمَمِ وَالأَلْسِنَةِ. سُلْطَانُهُ سُلْطَانٌ أَبَدِيٌّ مَا لَنْ يَزُولَ وَمَلَكُوتُهُ مَا لاَ يَنْقَرِضُ.
\par 15 [أَمَّا أَنَا دَانِيآلَ فَحَزِنَتْ رُوحِي فِي وَسَطِ جِسْمِي وَأَفْزَعَتْنِي رُؤَى رَأْسِي.
\par 16 فَاقْتَرَبْتُ إِلَى وَاحِدٍ مِنَ الْوُقُوفِ وَطَلَبْتُ مِنْهُ الْحَقِيقَةَ فِي كُلِّ هَذَا. فَأَخْبَرَنِي وَعَرَّفَنِي تَفْسِيرَ الأُمُورِ:
\par 17 هَؤُلاَءِ الْحَيَوَانَاتُ الْعَظِيمَةُ الَّتِي هِيَ أَرْبَعَةٌ هِيَ أَرْبَعَةُ مُلُوكٍ يَقُومُونَ عَلَى الأَرْضِ.
\par 18 أَمَّا قِدِّيسُو الْعَلِيِّ فَيَأْخُذُونَ الْمَمْلَكَةَ وَيَمْتَلِكُونَ الْمَمْلَكَةَ إِلَى الأَبَدِ وَإِلَى أَبَدِ الآبِدِينَ.
\par 19 حِينَئِذٍ رُمْتُ الْحَقِيقَةَ مِنْ جِهَةِ الْحَيَوَانِ الرَّابِعِ الَّذِي كَانَ مُخَالِفاً لِكُلِّهَا وَهَائِلاً جِدّاً وَأَسْنَانُهُ مِنْ حَدِيدٍ وَأَظْفَارُهُ مِنْ نُحَاسٍ وَقَدْ أَكَلَ وَسَحَقَ وَدَاسَ الْبَاقِيَ بِرِجْلَيْهِ
\par 20 وَعَنِ الْقُرُونِ الْعَشَرَةِ الَّتِي بِرَأْسِهِ وَعَنِ الآخَرِ الَّذِي طَلَعَ فَسَقَطَتْ قُدَّامَهُ ثَلاَثَةٌ. وَهَذَا الْقَرْنُ لَهُ عُيُونٌ وَفَمٌ مُتَكَلِّمٌ بِعَظَائِمَ وَمَنْظَرُهُ أَشَدُّ مِنْ رُفَقَائِهِ.
\par 21 وَكُنْتُ أَنْظُرُ وَإِذَا هَذَا الْقَرْنُ يُحَارِبُ الْقِدِّيسِينَ فَغَلَبَهُمْ
\par 22 حَتَّى جَاءَ الْقَدِيمُ الأَيَّامِ وَأُعْطِيَ الدِّينُ لِقِدِّيسِيِ الْعَلِيِّ وَبَلَغَ الْوَقْتُ فَامْتَلَكَ الْقِدِّيسُونَ الْمَمْلَكَةَ].
\par 23 فَقَالَ: [أَمَّا الْحَيَوَانُ الرَّابِعُ فَتَكُونُ مَمْلَكَةٌ رَابِعَةٌ عَلَى الأَرْضِ مُخَالِفَةٌ لِسَائِرِ الْمَمَالِكِ فَتَأْكُلُ الأَرْضَ كُلَّهَا وَتَدُوسُهَا وَتَسْحَقُهَا.
\par 24 وَالْقُرُونُ الْعَشَرَةُ مِنْ هَذِهِ الْمَمْلَكَةِ هِيَ عَشَرَةُ مُلُوكٍ يَقُومُونَ وَيَقُومُ بَعْدَهُمْ آخَرُ وَهُوَ مُخَالِفٌ الأَوَّلِينَ وَيُذِلُّ ثَلاَثَةَ مُلُوكٍ.
\par 25 وَيَتَكَلَّمُ بِكَلاَمٍ ضِدَّ الْعَلِيِّ وَيُبْلِي قِدِّيسِي الْعَلِيِّ وَيَظُنُّ أَنَّهُ يُغَيِّرُ الأَوْقَاتَ وَالسُّنَّةَ وَيُسَلَّمُونَ لِيَدِهِ إِلَى زَمَانٍ وَأَزْمِنَةٍ وَنِصْفِ زَمَانٍ.
\par 26 فَيَجْلِسُ الدِّينُ وَيَنْزِعُونَ عَنْهُ سُلْطَانَهُ لِيَفْنُوا وَيَبِيدُوا إِلَى الْمُنْتَهَى.
\par 27 وَالْمَمْلَكَةُ وَالسُّلْطَانُ وَعَظَمَةُ الْمَمْلَكَةِ تَحْتَ كُلِّ السَّمَاءِ تُعْطَى لِشَعْبِ قِدِّيسِي الْعَلِيِّ. مَلَكُوتُهُ مَلَكُوتٌ أَبَدِيٌّ وَجَمِيعُ السَّلاَطِينِ إِيَّاهُ يَعْبُدُونَ وَيُطِيعُونَ.
\par 28 إِلَى هُنَا نِهَايَةُ الأَمْرِ. أَمَّا أَنَا دَانِيآلَ فَأَفْكَارِي أَفْزَعَتْنِي كَثِيراً وَتَغَيَّرَتْ عَلَيَّ هَيْئَتِي وَحَفِظْتُ الأَمْرَ فِي قَلْبِي].

\chapter{8}

\par 1 فِي السَّنَةِ الثَّالِثَةِ مِنْ مُلْكِ بَيْلْشَاصَّرَ الْمَلِكِ ظَهَرَتْ لِي أَنَا دَانِيآلَ رُؤْيَا بَعْدَ الَّتِي ظَهَرَتْ لِي فِي الاِبْتِدَاءِ.
\par 2 فَرَأَيْتُ فِي الرُّؤْيَا وَكَانَ فِي رُؤْيَايَ وَأَنَا فِي شُوشَنَ الْقَصْرِ الَّذِي فِي وِلاَيَةِ عِيلاَمَ وَرَأَيْتُ فِي الرُّؤْيَا وَأَنَا عِنْدَ نَهْرِ أُولاَيَ.
\par 3 فَرَفَعْتُ عَيْنَيَّ وَرَأَيْتُ وَإِذَا بِكَبْشٍ وَاقِفٍ عَُِنْدَ النَّهْرِ وَلَهُ قَرْنَانِ وَالْقَرْنَانِ عَالِيَانِ وَالْوَاحِدُ أَعْلَى مِنَ الآخَرِ وَالأَعْلَى طَالِعٌ أَخِيراً.
\par 4 رَأَيْتُ الْكَبْشَ يَنْطَحُ غَرْباً وَشِمَالاً وَجَنُوباً فَلَمْ يَقِفْ حَيَوَانٌ قُدَّامَهُ وَلاَ مُنْقِذٌ مِنْ يَدِهِ وَفَعَلَ كَمَرْضَاتِهِ وَعَظُمَ.
\par 5 وَبَيْنَمَا كُنْتُ مُتَأَمِّلاً إِذَا بِتَيْسٍ مِنَ الْمَعْزِ جَاءَ مِنَ الْمَغْرِبِ عَلَى وَجْهِ كُلِّ الأَرْضِ وَلَمْ يَمَسَّ الأَرْضَ وَلِلتَّيْسِ قَرْنٌ مُعْتَبَرٌ بَيْنَ عَيْنَيْهِ.
\par 6 وَجَاءَ إِلَى الْكَبْشِ صَاحِبِ الْقَرْنَيْنِ الَّذِي رَأَيْتُهُ وَاقِفاً عِنْدَ النَّهْرِ وَرَكَضَ إِلَيْهِ بِشِدَّةِ قُوَّتِهِ.
\par 7 وَرَأَيْتُهُ قَدْ وَصَلَ إِلَى جَانِبِ الْكَبْشِ فَاسْتَشَاطَ عَلَيْهِ وَضَرَبَ الْكَبْشَ وَكَسَرَ قَرْنَيْهِ فَلَمْ تَكُنْ لِلْكَبْشِ قُوَّةٌ عَلَى الْوُقُوفِ أَمَامَهُ وَطَرَحَهُ عَلَى الأَرْضِ وَدَاسَهُ وَلَمْ يَكُنْ لِلْكَبْشِ مُنْقِذٌ مِنْ يَدِهِ.
\par 8 فتَعَظَّمَ تَيْسُ المَعْزِ جِدّاً. وَلَمَّا اعْتَزَّ انْكَسَرَ الْقَرْنُ الْعَظِيمُ وَطَلَعَ عِوَضاً عَنْهُ أَرْبَعَةُ قُرُونٍ مُعْتَبَرَةٍ نَحْوَ رِيَاحِ السَّمَاءِ الأَرْبَعِ.
\par 9 وَمِنْ وَاحِدٍ مِنْهَا خَرَجَ قَرْنٌ صَغِيرٌ وَعَظُمَ جِدّاً نَحْوَ الْجَنُوبِ وَنَحْوَ الشَّرْقِ وَنَحْوَ فَخْرِ الأَرَاضِي.
\par 10 وَتَعَظَّمَ حَتَّى إِلَى جُنْدِ السَّمَاوَاتِ وَطَرَحَ بَعْضاً مِنَ الْجُنْدِ وَالنُّجُومِ إِلَى الأَرْضِ وَدَاسَهُمْ.
\par 11 وَحَتَّى إِلَى رَئِيسِ الْجُنْدِ تَعَظَّمَ وَبِهِ أُبْطِلَتِ الْمُحْرَقَةُ الدَّائِمَةُ وَهُدِمَ مَسْكَنُ مَقْدِسِهِ.
\par 12 وَجُعِلَ جُنْدٌ عَلَى الْمُحْرَقَةِ الدَّائِمَةِ بِالْمَعْصِيَةِ فَطَرَحَ الْحَقَّ عَلَى الأَرْضِ وَفَعَلَ وَنَجَحَ.
\par 13 فَسَمِعْتُ قُدُّوساً وَاحِداً يَتَكَلَّمُ. فَقَالَ قُدُّوسٌ وَاحِدٌ لِفُلاَنٍ الْمُتَكَلِّمِ: [إِلَى مَتَى الرُّؤْيَا مِنْ جِهَةِ الْمُحْرَقَةِ الدَّائِمَةِ وَمَعْصِيَةِ الْخَرَابِ لِبَذْلِ الْقُدْسِ وَالْجُنْدِ مَدُوسَيْنِ؟]
\par 14 فَقَالَ لِي: [إِلَى أَلْفَيْنِ وَثَلاَثِ مِئَةِ صَبَاحٍ وَمَسَاءٍ فَيَتَبَرَّأُ الْقُدْسُ].
\par 15 وَكَانَ لَمَّا رَأَيْتُ أَنَا دَانِيآلَ الرُّؤْيَا وَطَلَبْتُ الْمَعْنَى إِذَا بِشِبْهِ إِنْسَانٍ وَاقِفٍ قُبَالَتِي.
\par 16 وَسَمِعْتُ صَوْتَ إِنْسَانٍ بَيْنَ أُولاَيَ فَنَادَى وَقَالَ: [يَا جِبْرَائِيلُ فَهِّمْ هَذَا الرَّجُلَ الرُّؤْيَا].
\par 17 فَجَاءَ إِلَى حَيْثُ وَقَفْتُ. وَلَمَّا جَاءَ خِفْتُ وَخَرَرْتُ عَلَى وَجْهِي. فَقَالَ لِي: [افْهَمْ يَا ابْنَ آدَمَ. إِنَّ الرُّؤْيَا لِوَقْتِ الْمُنْتَهَى].
\par 18 وَإِذْ كَانَ يَتَكَلَّمُ مَعِي كُنْتُ مُسَبَّخاً عَلَى وَجْهِي إِلَى الأَرْضِ فَلَمَسَنِي وَأَوْقَفَنِي عَلَى مَقَامِي.
\par 19 وَقَالَ: [هَئَنَذَا أُعَرِّفُكَ مَا يَكُونُ فِي آخِرِ السَّخَطِ. لأَنَّ لِمِيعَادِ الاِنْتِهَاءَ.
\par 20 أَمَّا الْكَبْشُ الَّذِي رَأَيْتَهُ ذَا الْقَرْنَيْنِ فَهُوَ مُلُوكُ مَادِي وَفَارِسَ.
\par 21 وَالتَّيْسُ الْعَافِي مَلِكُ الْيُونَانِ وَالْقَرْنُ الْعَظِيمُ الَّذِي بَيْنَ عَيْنَيْهِ هُوَ الْمَلِكُ الأَوَّلُ.
\par 22 وَإِذِ انْكَسَرَ وَقَامَ أَرْبَعَةٌ عِوَضاً عَنْهُ فَسَتَقُومُ أَرْبَعُ مَمَالِكَ مِنَ الأُمَّةِ وَلَكِنْ لَيْسَ فِي قُوَّتِهِ.
\par 23 وَفِي آخِرِ مَمْلَكَتِهِمْ عِنْدَ تَمَامِ الْمَعَاصِي يَقُومُ مَلِكٌ جَافِي الْوَجْهِ وَفَاهِمُ الْحِيَلِ.
\par 24 وَتَعْظُمُ قُوَّتُهُ وَلَكِنْ لَيْسَ بِقُوَّتِهِ. يُهْلِكُ عَجَباً وَيَنْجَحُ وَيَفْعَلُ وَيُبِيدُ الْعُظَمَاءَ وَشَعْبَ الْقِدِّيسِينَ.
\par 25 وَبِحَذَاقَتِهِ يَنْجَحُ أَيْضاً الْمَكْرُ فِي يَدِهِ وَيَتَعَظَّمُ بِقَلْبِهِ. وَفِي الاِطْمِئْنَانِ يُهْلِكُ كَثِيرِينَ وَيَقُومُ عَلَى رَئِيسِ الرُّؤَسَاءِ وَبِلاَ يَدٍ يَنْكَسِرُ.
\par 26 فَرُؤْيَا الْمَسَاءِ وَالصَّبَاحِ الَّتِي قِيلَتْ هِيَ حَقٌّ. أَمَّا أَنْتَ فَاكْتُمِ الرُّؤْيَا لأَنَّهَا إِلَى أَيَّامٍ كَثِيرَةٍ].
\par 27 وَأَنَا دَانِيآلَ ضَعُفْتُ وَنَحَلْتُ أَيَّاماً ثُمَّ قُمْتُ وَبَاشَرْتُ أَعْمَالَ الْمَلِكِ. وَكُنْتُ مُتَحَيِّراً مِنَ الرُّؤْيَا وَلاَ فَاهِمَ.

\chapter{9}

\par 1 فِي السَّنَةِ الأُولَى لِدَارِيُوسَ بْنِ أَحْشَوِيرُوشَ مِنْ نَسْلِ الْمَادِيِّينَ الَّذِي مُلِّكَ عَلَى مَمْلَكَةِ الْكِلْدَانِيِّينَ
\par 2 فِي السَّنَةِ الأُولَى مِنْ مُلْكِهِ أَنَا دَانِيآلَ فَهِمْتُ مِنَ الْكُتُبِ عَدَدَ السِّنِينَ الَّتِي كَانَتْ عَنْهَا كَلِمَةُ الرَّبِّ إِلَى إِرْمِيَا النَّبِيِّ لِكَمَالَةِ سَبْعِينَ سَنَةً عَلَى خَرَابِ أُورُشَلِيمَ.
\par 3 فَوَجَّهْتُ وَجْهِي إِلَى اللَّهِ السَّيِّدِ طَالِباً بِالصَّلاَةِ وَالتَّضَرُّعَاتِ بِالصَّوْمِ وَالْمَسْحِ وَالرَّمَادِ.
\par 4 وَصَلَّيْتُ إِلَى الرَّبِّ إِلَهِي وَاعْتَرَفْتُ وَقُلْتُ: [أَيُّهَا الرَّبُّ الإِلَهُ الْعَظِيمُ الْمَهُوبُ حَافِظَ الْعَهْدِ وَالرَّحْمَةِ لِمُحِبِّيهِ وَحَافِظِي وَصَايَاهُ.
\par 5 أَخْطَأْنَا وَأَثِمْنَا وَعَمِلْنَا الشَّرَّ وَتَمَرَّدْنَا وَحِدْنَا عَنْ وَصَايَاكَ وَعَنْ أَحْكَامِكَ.
\par 6 وَمَا سَمِعْنَا مِنْ عَبِيدِكَ الأَنْبِيَاءِ الَّذِينَ بِاسْمِكَ كَلَّمُوا مُلُوكَنَا وَرُؤَسَاءَنَا وَآبَاءَنَا وَكُلَّ شَعْبِ الأَرْضِ.
\par 7 لَكَ يَا سَيِّدُ الْبِرُّ أَمَّا لَنَا فَخِزْيُ الْوُجُوهِ كَمَا هُوَ الْيَوْمَ لِرِجَالِ يَهُوذَا وَلِسُكَّانِ أُورُشَلِيمَ وَلِكُلِّ إِسْرَائِيلَ الْقَرِيبِينَ وَالْبَعِيدِينَ فِي كُلِّ الأَرَاضِي الَّتِي طَرَدْتَهُمْ إِلَيْهَا مِنْ أَجْلِ خِيَانَتِهِمِ الَّتِي خَانُوكَ إِيَّاهَا.
\par 8 يَا سَيِّدُ لَنَا خِزْيُ الْوُجُوهِ لِمُلُوكِنَا لِرُؤَسَائِنَا وَلِآبَائِنَا لأَنَّنَا أَخْطَأْنَا إِلَيْكَ.
\par 9 لِلرَّبِّ إِلَهِنَا الْمَرَاحِمُ وَالْمَغْفِرَةُ لأَنَّنَا تَمَرَّدْنَا عَلَيْهِ.
\par 10 وَمَا سَمِعْنَا صَوْتَ الرَّبِّ إِلَهِنَا لِنَسْلُكَ فِي شَرَائِعِهِ الَّتِي جَعَلَهَا أَمَامَنَا عَنْ يَدِ عَبِيدِهِ الأَنْبِيَاءِ.
\par 11 وَكُلُّ إِسْرَائِيلَ قَدْ تَعَدَّى عَلَى شَرِيعَتِكَ وَحَادُوا لِئَلاَّ يَسْمَعُوا صَوْتَكَ فَسَكَبْتَ عَلَيْنَا اللَّعْنَةَ وَالْحَلْفَ الْمَكْتُوبَ فِي شَرِيعَةِ مُوسَى عَبْدِ اللَّهِ لأَنَّنَا أَخْطَأْنَا إِلَيْهِ.
\par 12 وَقَدْ أَقَامَ كَلِمَاتِهِ الَّتِي تَكَلَّمَ بِهَا عَلَيْنَا وَعَلَى قُضَاتِنَا الَّذِينَ قَضُوا لَنَا لِيَجْلِبَ عَلَيْنَا شَرّاً عَظِيماً مَا لَمْ يُجْرَ تَحْتَ السَّمَاوَاتِ كُلِّهَا كَمَا أُجْرِيَ عَلَى أُورُشَلِيمَ.
\par 13 كَمَا كُتِبَ فِي شَرِيعَةِ مُوسَى قَدْ جَاءَ عَلَيْنَا كُلُّ هَذَا الشَّرِّ وَلَمْ نَتَضَرَّعْ إِلَى وَجْهِ الرَّبِّ إِلَهِنَا لِنَرْجِعَ مِنْ آثَامِنَا وَنَفْطِنَ بِحَقِّكَ.
\par 14 فَسَهِرَ الرَّبُّ عَلَى الشَّرِّ وَجَلَبَهُ عَلَيْنَا لأَنَّ الرَّبَّ إِلَهَنَا بَارٌّ فِي كُلِّ أَعْمَالِهِ الَّتِي عَمِلَهَا إِذْ لَمْ نَسْمَعْ صَوْتَهُ.
\par 15 وَالآنَ أَيُّهَا السَّيِّدُ إِلَهُنَا الَّذِي أَخْرَجْتَ شَعْبَكَ مِنْ أَرْضِ مِصْرَ بِيَدٍ قَوِيَّةٍ وَجَعَلْتَ لِنَفْسِكَ اسْماً كَمَا هُوَ هَذَا الْيَوْمَ قَدْ أَخْطَأْنَا. عَمِلْنَا شَرّاً.
\par 16 يَا سَيِّدُ حَسَبَ كُلِّ رَحْمَتِكَ اصْرِفْ سَخَطَكَ وَغَضَبَكَ عَنْ مَدِينَتِكَ أُورُشَلِيمَ جَبَلِ قُدْسِكَ إِذْ لِخَطَايَانَا وَلِآثَامِ آبَائِنَا صَارَتْ أُورُشَلِيمُ وَشَعْبُكَ عَاراً عِنْدَ جَمِيعِ الَّذِينَ حَوْلَنَا.
\par 17 فَاسْمَعِ الآنَ يَا إِلَهَنَا صَلاَةَ عَبْدِكَ وَتَضَرُّعَاتِهِ وَأَضِئْ بِوَجْهِكَ عَلَى مَقْدِسِكَ الْخَرِبِ مِنْ أَجْلِ السَّيِّدِ.
\par 18 أَمِلْ أُذُنَكَ يَا إِلَهِي وَاسْمَعْ. افْتَحْ عَيْنَيْكَ وَانْظُرْ خِرَبَنَا وَالْمَدِينَةَ الَّتِي دُعِيَ اسْمُكَ عَلَيْهَا لأَنَّهُ لاَ لأَجْلِ بِرِّنَا نَطْرَحُ تَضَرُّعَاتِنَا أَمَامَ وَجْهِكَ بَلْ لأَجْلِ مَرَاحِمِكَ الْعَظِيمَةِ.
\par 19 يَا سَيِّدُ اسْمَعْ. يَا سَيِّدُ اغْفِرْ. يَا سَيِّدُ أَصْغِ وَاصْنَعْ. لاَ تُؤَخِّرْ مِنْ أَجْلِ نَفْسِكَ يَا إِلَهِي لأَنَّ اسْمَكَ دُعِيَ عَلَى مَدِينَتِكَ وَعَلَى شَعْبِكَ].
\par 20 وَبَيْنَمَا أَنَا أَتَكَلَّمُ وَأُصَلِّي وَأَعْتَرِفُ بِخَطِيَّتِي وَخَطِيَّةِ شَعْبِي إِسْرَائِيلَ وَأَطْرَحُ تَضَرُّعِي أَمَامَ الرَّبِّ إِلَهِي عَنْ جَبَلِ قُدْسِ إِلَهِي
\par 21 وَأَنَا مُتَكَلِّمٌ بَعْدُ بِالصَّلاَةِ إِذَا بِالرَّجُلِ جِبْرَائِيلَ الَّذِي رَأَيْتُهُ فِي الرُّؤْيَا فِي الاِبْتِدَاءِ مُطَاراً وَاغِفاً لَمَسَنِي عِنْدَ وَقْتِ تَقْدِمَةِ الْمَسَاءِ.
\par 22 وَفَهَّمَنِي وَتَكَلَّمَ مَعِي وَقَالَ: [يَا دَانِيآلُ إِنِّي خَرَجْتُ الآنَ لِأُعَلِّمَكَ الْفَهْمَ.
\par 23 فِي ابْتِدَاءِ تَضَرُّعَاتِكَ خَرَجَ الأَمْرُ وَأَنَا جِئْتُ لِأُخْبِرَكَ لأَنَّكَ أَنْتَ مَحْبُوبٌ. فَتَأَمَّلِ الْكَلاَمَ وَافْهَمِ الرُّؤْيَا.
\par 24 سَبْعُونَ أُسْبُوعاً قُضِيَتْ عَلَى شَعْبِكَ وَعَلَى مَدِينَتِكَ الْمُقَدَّسَةِ لِتَكْمِيلِ الْمَعْصِيَةِ وَتَتْمِيمِ الْخَطَايَا وَلِكَفَّارَةِ الإِثْمِ وَلِيُؤْتَى بِالْبِرِّ الأَبَدِيِّ وَلِخَتْمِ الرُّؤْيَا وَالنُّبُوَّةِ وَلِمَسْحِ قُدُّوسِ الْقُدُّوسِينَ.
\par 25 فَاعْلَمْ وَافْهَمْ أَنَّهُ مِنْ خُرُوجِ الأَمْرِ لِتَجْدِيدِ أُورُشَلِيمَ وَبَنَائِهَا إِلَى الْمَسِيحِ الرَّئِيسِ سَبْعَةُ أَسَابِيعَ وَاثْنَانِ وَسِتُّونَ أُسْبُوعاً يَعُودُ وَيُبْنَى سُوقٌ وَخَلِيجٌ فِي ضِيقِ الأَزْمِنَةِ.
\par 26 وَبَعْدَ اثْنَيْنِ وَسِتِّينَ أُسْبُوعاً يُقْطَعُ الْمَسِيحُ وَلَيْسَ لَهُ وَشَعْبُ رَئِيسٍ آتٍ يُخْرِبُ الْمَدِينَةَ وَالْقُدْسَ وَانْتِهَاؤُهُ بِغَمَارَةٍ وَإِلَى النِّهَايَةِ حَرْبٌ وَخِرَبٌ قُضِيَ بِهَا.
\par 27 وَيُثَبِّتُ عَهْداً مَعَ كَثِيرِينَ فِي أُسْبُوعٍ وَاحِدٍ وَفِي وَسَطِ الأُسْبُوعِ يُبَطِّلُ الذَّبِيحَةَ وَالتَّقْدِمَةَ وَعَلَى جَنَاحِ الأَرْجَاسِ مُخَرَّبٌ حَتَّى يَتِمَّ وَيُصَبَّ الْمَقْضِيُّ عَلَى الْمُخَرَِّبِ].

\chapter{10}

\par 1 فِي السَّنَةِ الثَّالِثَةِ لِكُورَشَ مَلِكِ فَارِسَ كُشِفَ أَمْرٌ لِدَانِيآلَ الَّذِي سُمِّيَ بِاسْمِ بَلْطْشَاصَّرَ. وَالأَمْرُ حَقٌّ وَالْجِهَادُ عَظِيمٌ وَفَهِمَ الأَمْرَ وَلَهُ مَعْرِفَةُ الرُّؤْيَا.
\par 2 فِي تِلْكَ الأَيَّامِ أَنَا دَانِيآلَ كُنْتُ نَائِحاً ثَلاَثَةَ أَسَابِيعِ أَيَّامٍ
\par 3 لَمْ آكُلْ طَعَاماً شَهِيّاً وَلَمْ يَدْخُلْ فِي فَمِي لَحْمٌ وَلاَ خَمْرٌ وَلَمْ أَدَّهِنْ حَتَّى تَمَّتْ ثَلاَثَةُ أَسَابِيعِ أَيَّامٍ.
\par 4 وَفِي الْيَوْمِ الرَّابِعِ وَالْعِشْرِينَ مِنَ الشَّهْرِ الأَوَّلِ إِذْ كُنْتُ عَلَى جَانِبِ النَّهْرِ الْعَظِيمِ (هُوَ دِجْلَةُ)
\par 5 رَفَعْتُ وَنَظَرْتُ فَإِذَا بِرَجُلٍ لاَبِسٍ كَتَّاناً وَحَقَوَاهُ مُتَنَطِّقَانِ بِذَهَبِ أُوفَازَ
\par 6 وَجِسْمُهُ كَالزَّبَرْجَدِ وَوَجْهُهُ كَمَنْظَرِ الْبَرْقِ وَعَيْنَاهُ كَمِصْبَاحَيْ نَارٍ وَذِرَاعَاهُ وَرِجْلاَهُ كَعَيْنِ النُّحَاسِ الْمَصْقُولِ وَصَوْتُ كَلاَمِهِ كَصَوْتِ جُمْهُورٍ.
\par 7 فَرَأَيْتُ أَنَا دَانِيآلُ الرُّؤْيَا وَحْدِي وَالرِّجَالُ الَّذِينَ كَانُوا مَعِي لَمْ يَرُوا الرُّؤْيَا لَكِنْ وَقَعَ عَلَيْهِمِ ارْتِعَادٌ عَظِيمٌ فَهَرَبُوا لِيَخْتَبِئُوا.
\par 8 فَبَقِيتُ أَنَا وَحْدِي وَرَأَيْتُ هَذِهِ الرُّؤْيَا الْعَظِيمَةَ. وَلَمْ تَبْقَ فِيَّ قُوَّةٌ وَنَضَارَتِي تَحَوَّلَتْ فِيَّ إِلَى فَسَادٍ وَلَمْ أَضْبِطْ قُوَّةً.
\par 9 وَسَمِعْتُ صَوْتَ كَلاَمِهِ. وَلَمَّا سَمِعْتُ صَوْتَ كَلاَمِهِ كُنْتُ مُسَبَّخاً عَلَى وَجْهِي وَوَجْهِي إِلَى الأَرْضِ.
\par 10 وَإِذَا بِيَدٍ لَمَسَتْنِي وَأَقَامَتْنِي مُرْتَجِفاً عَلَى رُكْبَتَيَّ وَعَلَى كَفَّيْ يَدَيَّ.
\par 11 وَقَالَ لِي: [يَا دَانِيآلُ أَيُّهَا الرَّجُلُ الْمَحْبُوبُ افْهَمِ الْكَلاَمَ الَّذِي أُكَلِّمُكَ بِهِ وَقُمْ عَلَى مَقَامِكَ لأَنِّي الآنَ أُرْسِلْتُ إِلَيْكَ]. وَلَمَّا تَكَلَّمَ مَعِي بِهَذَا الْكَلاَمِ قُمْتُ مُرْتَعِداً.
\par 12 فَقَالَ لِي: [لاَ تَخَفْ يَا دَانِيآلُ لأَنَّهُ مِنَ الْيَوْمِ الأَوَّلِ الَّذِي فِيهِ جَعَلْتَ قَلْبَكَ لِلْفَهْمِ وَلِإِذْلاَلِ نَفْسِكَ قُدَّامَ إِلَهِكَ سُمِعَ كَلاَمُكَ وَأَنَا أَتَيْتُ لأَجْلِ كَلاَمِكَ.
\par 13 وَرَئِيسُ مَمْلَكَةِ فَارِسَ وَقَفَ مُقَابِلِي وَاحِداً وَعِشْرِينَ يَوْماً وَهُوَذَا مِيخَائِيلُ وَاحِدٌ مِنَ الرُّؤَسَاءِ الأَوَّلِينَ جَاءَ لِإِعَانَتِي وَأَنَا أُبْقِيتُ هُنَاكَ عِنْدَ مُلُوكِ فَارِسَ.
\par 14 وَجِئْتُ لِأُفْهِمَكَ مَا يُصِيبُ شَعْبَكَ فِي الأَيَّامِ الأَخِيرَةِ لأَنَّ الرُّؤْيَا إِلَى أَيَّامٍ بَعْدُ].
\par 15 فَلَمَّا تَكَلَّمَ مَعِي بِمِثْلِ هَذَا الْكَلاَمِ جَعَلْتُ وَجْهِي إِلَى الأَرْضِ وَصَمَتُّ.
\par 16 وَهُوَذَا كَشِبْهِ بَنِي آدَمَ لَمَسَ شَفَتَيَّ فَفَتَحْتُ فَمِي وَتَكَلَّمْتُ وَقُلْتُ لِلْوَاقِفِ أَمَامِي: [يَا سَيِّدِي بِالرُّؤْيَا انْقَلَبَتْ عَلَيَّ أَوْجَاعِي فَمَا ضَبَطْتُ قُوَّةً.
\par 17 فَكَيْفَ يَسْتَطِيعُ عَبْدُ سَيِّدِي هَذَا أَنْ يَتَكَلَّمَ مَعَ سَيِّدِي هَذَا وَأَنَا فَحَالاً لَمْ تَثْبُتْ فِيَّ قُوَّةٌ وَلَمْ تَبْقَ فِيَّ نَسَمَةٌ].
\par 18 فَعَادَ وَلَمَسَنِي كَمَنْظَرِ إِنْسَانٍ وَقَوَّانِي.
\par 19 وَقَالَ: [لاَ تَخَفْ أَيُّهَا الرَّجُلُ الْمَحْبُوبُ. سَلاَمٌ لَكَ. تَشَدَّدْ. تَقَوَّ]. وَلَمَّا كَلَّمَنِي تَقَوَّيْتُ وَقُلْتُ: [لِيَتَكَلَّمْ سَيِّدِي لأَنَّكَ قَوَّيْتَنِي].
\par 20 فَقَالَ: [هَلْ عَرَفْتَ لِمَاذَا جِئْتُ إِلَيْكَ؟ فَالآنَ أَرْجِعُ وَأُحَارِبُ رَئِيسَ فَارِسَ. فَإِذَا خَرَجْتُ هُوَذَا رَئِيسُ الْيُونَانِ يَأْتِي.
\par 21 وَلَكِنِّي أُخْبِرُكَ بِالْمَرْسُومِ فِي كِتَابِ الْحَقِّ. وَلاَ أَحَدٌ يَتَمَسَّكُ مَعِي عَلَى هَؤُلاَءِ إِلاَّ مِيخَائِيلُ رَئِيسُكُمْ].

\chapter{11}

\par 1 [وَأَنَا فِي السَّنَةِ الأُولَى لِدَارِيُوسَ الْمَادِيِّ وَقَفْتُ لِأُشَدِّدَهُ وَأُقَوِّيَهُ.
\par 2 وَالآنَ أُخْبِرُكَ بِالْحَقِّ. هُوَذَا ثَلاَثَةُ مُلُوكٍ أَيْضاً يَقُومُونَ فِي فَارِسَ وَالرَّابِعُ يَسْتَغْنِي بِغِنًى أَوْفَرَ مِنْ جَمِيعِهِمْ وَحَسَبَ قُوَّتِهِ بِغِنَاهُ يُهَيِّجُ الْجَمِيعَ عَلَى مَمْلَكَةِ الْيُونَانِ.
\par 3 وَيَقُومُ مَلِكٌ جَبَّارٌ وَيَتَسَلَّطُ تَسَلُّطاً عَظِيماً وَيَفْعَلُ حَسَبَ إِرَادَتِهِ.
\par 4 وَكَقِيَامِهِ تَنْكَسِرُ مَمْلَكَتُهُ وَتَنْقَسِمُ إِلَى رِيَاحِ السَّمَاءِ الأَرْبَعِ وَلاَ لِعَقِبِهِ وَلاَ حَسَبَ سُلْطَانِهِ الَّذِي تَسَلَّطَ بِهِ لأَنَّ مَمْلَكَتَهُ تَنْقَرِضُ وَتَكُونُ لِآخَرِينَ غَيْرِ أُولَئِكَ.
\par 5 وَيَتَقَوَّى مَلِكُ الْجَنُوبِ. وَمِنْ رُؤَسَائِهِ مَنْ يَقْوَى عَلَيْهِ وَيَتَسَلَّطُ. تَسَلُّطٌ عَظِيمٌ تَسَلُّطُهُ.
\par 6 وَبَعْدَ سِنِينَ يَتَعَاهَدَانِ وَبِنْتُ مَلِكِ الْجَنُوبِ تَأْتِي إِلَى مَلِكِ الشِّمَالِ لِإِجْرَاءِ الاِتِّفَاقِ وَلَكِنْ لاَ تَضْبِطُ الذِّرَاعُ قُوَّةً وَلاَ يَقُومُ هُوَ وَلاَ ذِرَاعُهُ. وَتُسَلَّمُ هِيَ وَالَّذِينَ أَتُوا بِهَا وَالَّذِي وَلَدَهَا وَمَنْ قَوَّاهَا فِي تِلْكَ الأَوْقَاتِ.
\par 7 وَيَقُومُ مِنْ فَرْعِ أُصُولِهَا قَائِمٌ مَكَانَهُ وَيَأْتِي إِلَى الْجَيْشِ وَيَدْخُلُ حِصْنَ مَلِكِ الشِّمَالِ وَيَعْمَلُ بِهِمْ وَيَقْوَى.
\par 8 وَيَسْبِي إِلَى مِصْرَ آلِهَتَهُمْ أَيْضاً مَعَ مَسْبُوكَاتِهِمْ وَآنِيَتِهِمِ الثَّمِينَةِ مِنْ فِضَّةٍ وَذَهَبٍ وَيَقْتَصِرُ سِنِينَ عَنْ مَلِكِ الشِّمَالِ.
\par 9 فَيَدْخُلُ مَلِكُ الْجَنُوبِ إِلَى مَمْلَكَتِهِ وَيَرْجِعُ إِلَى أَرْضِهِ.
\par 10 [وَبَنُوهُ يَتَهَيَّجُونَ فَيَجْمَعُونَ جُمْهُورَ جُيُوشٍ عَظِيمَةٍ وَيَأْتِي آتٍ وَيَغْمُرُ وَيَطْمُو وَيَرْجِعُ وَيُحَارِبُ حَتَّى إِلَى حِصْنِهِ.
\par 11 وَيَغْتَاظُ مَلِكُ الْجَنُوبِ وَيَخْرُجُ وَيُحَارِبُ مَلِكَ الشِّمَالِ وَيُقِيمُ جُمْهُوراً عَظِيماً فَيُسَلَّمُ الْجُمْهُورُ فِي يَدِهِ.
\par 12 فَإِذَا رُفِعَ الْجُمْهُورُ يَرْتَفِعُ قَلْبُهُ وَيَطْرَحُ رَبَوَاتٍ وَلاَ يَعْتَزُّ.
\par 13 فَيَرْجِعُ مَلِكُ الشِّمَالِ وَيُقِيمُ جُمْهُوراً أَكْثَرَ مِنَ الأَوَّلِ وَيَأْتِي بَعْدَ حِينٍ بَعْدَ سِنِينَ بِجَيْشٍ عَظِيمٍ وَثَرْوَةٍ جَزِيلَةٍ.
\par 14 وَفِي تِلْكَ الأَوْقَاتِ يَقُومُ كَثِيرُونَ عَلَى مَلِكِ الْجَنُوبِ وَبَنُو الْعُتَاةِ مِنْ شَعْبِكَ يَقُومُونَ لِإِثْبَاتِ الرُّؤْيَا وَيَعْثُرُونَ.
\par 15 فَيَأْتِي مَلِكُ الشِّمَالِ وَيُقِيمُ مِتْرَسَةً وَيَأْخُذُ الْمَدِينَةَ الْحَصِينَةَ فَلاَ تَقُومُ أَمَامَهُ ذِرَاعَا الْجَنُوبِ وَلاَ قَوْمُهُ الْمُنْتَخَبُ وَلاَ تَكُونُ لَهُ قُوَّةٌ لِلْمُقَاوَمَةِ.
\par 16 وَالآتِي عَلَيْهِ يَفْعَلُ كَإِرَادَتِهِ وَلَيْسَ مَنْ يَقِفُ أَمَامَهُ وَيَقُومُ فِي الأَرْضِ الْبَهِيَّةِ وَهِيَ بِالتَّمَامِ بِيَدِهِ.
\par 17 وَيَجْعَلُ وَجْهَهُ لِيَدْخُلَ بِسُلْطَانِ كُلِّ مَمْلَكَتِهِ وَيَجْعَلُ مَعَهُ صُلْحاً وَيُعْطِيهِ بِنْتَ النِّسَاءِ لِيُفْسِدَهَا فَلاَ تَثْبُتَ وَلاَ تَكُونَ لَهُ.
\par 18 وَيُحَوِّلُ وَجْهَهُ إِلَى الْجَزَائِرِ وَيَأْخُذُ كَثِيراً مِنْهَا وَيُزِيلُ رَئِيسٌ تَعْيِيرَهُ فَضْلاً عَنْ رَدِّ تَعْيِيرِهِ عَلَيْهِ.
\par 19 وَيُحَوِّلُ وَجْهَهُ إِلَى حُصُونِ أَرْضِهِ وَيَعْثُرُ وَيَسْقُطُ وَلاَ يُوجَدُ.
\par 20 [فَيَقُومُ مَكَانَهُ مَنْ يُعَبِّرُ جَابِيَ الْجِزْيَةِ فِي فَخْرِ الْمَمْلَكَةِ وَفِي أَيَّامٍ قَلِيلَةٍ يَنْكَسِرُ لاَ بِغَضَبٍ وَلاَ بِحَرْبٍ.
\par 21 فَيَقُومُ مَكَانَهُ مُحْتَقَرٌ لَمْ يَجْعَلُوا عَلَيْهِ فَخْرَ الْمَمْلَكَةِ وَيَأْتِي بَغْتَةً وَيُمْسِكُ الْمَمْلَكَةَ بِالتَّمَلُّقَاتِ.
\par 22 وَأَذْرُعُ الْجَارِفِ تُجْرَفُ مِنْ قُدَّامِهِ وَتَنْكَسِرُ وَكَذَلِكَ رَئِيسُ الْعَهْدِ.
\par 23 وَمِنَ الْمُعَاهَدَةِ مَعَهُ يَعْمَلُ بِالْمَكْرِ وَيَصْعَدُ وَيَعْظُمُ بِقَوْمٍ قَلِيلٍ.
\par 24 يَدْخُلُ بَغْتَةً عَلَى أَسْمَنِ الْبِلاَدِ وَيَفْعَلُ مَا لَمْ يَفْعَلْهُ آبَاؤُهُ وَلاَ آبَاءُ آبَائِهِ. يَبْذُرُ بَيْنَهُمْ نَهْباً وَغَنِيمَةً وَغِنًى وَيُفَكِّرُ أَفْكَارَهُ عَلَى الْحُصُونِ وَذَلِكَ إِلَى حِينٍ.
\par 25 وَيُنْهِضُ قُوَّتَهُ وَقَلْبَهُ عَلَى مَلِكِ الْجَنُوبِ بِجَيْشٍ عَظِيمٍ وَمَلِكُ الْجَنُوبِ يَتَهَيَّجُ إِلَى الْحَرْبِ بِجَيْشٍ عَظِيمٍ وَقَوِيٍّ جِدّاً وَلَكِنَّهُ لاَ يَثْبُتُ لأَنَّهُمْ يُدَبِّرُونَ عَلَيْهِ تَدَابِيرَ.
\par 26 وَالآكِلُونَ أَطَايِبَهُ يَكْسِرُونَهُ وَجَيْشُهُ يَطْمُو وَيَسْقُطُ كَثِيرُونَ قَتْلَى.
\par 27 وَهَذَانِ الْمَلِكَانِ قَلْبُهُمَا لِفِعْلِ الشَّرِّ وَيَتَكَلَّمَانِ بِالْكَذِبِ عَلَى مَائِدَةٍ وَاحِدَةٍ وَلاَ يَنْجَحُ لأَنَّ الاِنْتِهَاءَ بَعْدُ إِلَى مِيعَادٍ.
\par 28 فَيَرْجِعُ إِلَى أَرْضِهِ بِغِنًى جَزِيلٍ وَقَلْبُهُ عَلَى الْعَهْدِ الْمُقَدَّسِ فَيَعْمَلُ وَيَرْجِعُ إِلَى أَرْضِهِ.
\par 29 [وَفِي الْمِيعَادِ يَعُودُ وَيَدْخُلُ الْجَنُوبَ وَلَكِنْ لاَ يَكُونُ الآخِرُ كَالأَوَّلِ.
\par 30 فَتَأْتِي عَلَيْهِ سُفُنٌ مِنْ كِتِّيمَ فَيَيْئَسُ وَيَرْجِعُ وَيَغْتَاظُ عَلَى الْعَهْدِ الْمُقَدَّسِ وَيَعْمَلُ وَيَرْجِعُ وَيَصْغَى إِلَى الَّذِينَ تَرَكُوا الْعَهْدَ الْمُقَدَّسَ.
\par 31 وَتَقُومُ مِنْهُ أَذْرُعٌ وَتُنَجِّسُ الْمَقْدِسَ الْحَصِينَ وَتَنْزِعُ الْمُحْرَقَةَ الدَّائِمَةَ وَتَجْعَلُ الرِّجْسَ الْمُخَرِّبَ.
\par 32 وَالْمُتَعَدُّونَ عَلَى الْعَهْدِ يُغْوِيهِمْ بِالتَّمَلُّقَاتِ. أَمَّا الشَّعْبُ الَّذِينَ يَعْرِفُونَ إِلَهَهُمْ فَيَقْوُونَ وَيَعْمَلُونَ.
\par 33 وَالْفَاهِمُونَ مِنَ الشَّعْبِ يُعَلِّمُونَ كَثِيرِينَ. وَيَعْثُرُونَ بِالسَّيْفِ وَبِاللَّهِيبِ وَبِالسَّبْيِ وَبِالنَّهْبِ أَيَّاماً.
\par 34 فَإِذَا عَثَرُوا يُعَانُونَ عَوْناً قَلِيلاً وَيَتَّصِلُ بِهِمْ كَثِيرُونَ بِالتَّمَلُّقَاتِ.
\par 35 وَبَعْضُ الْفَاهِمِينَ يَعْثُرُونَ امْتِحَاناً لَهُمْ لِلتَّطْهِيرِ وَلِلتَّبْيِيضِ إِلَى وَقْتِ النِّهَايَةِ. لأَنَّهُ بَعْدُ إِلَى الْمِيعَادِ.
\par 36 وَيَفْعَلُ الْمَلِكُ كَإِرَادَتِهِ وَيَرْتَفِعُ وَيَتَعَظَّمُ عَلَى كُلِّ إِلَهٍ وَيَتَكَلَّمُ بِأُمُورٍ عَجِيبَةٍ عَلَى إِلَهِ الآلِهَةِ وَيَنْجَحُ إِلَى إِتْمَامِ الْغَضَبِ لأَنَّ الْمَقْضِيَّ بِهِ يُجْرَى.
\par 37 وَلاَ يُبَالِي بِآلِهَةِ آبَائِهِ وَلاَ بِشَهْوَةِ النِّسَاءِ وَبِكُلِّ إِلَهٍ لاَ يُبَالِي لأَنَّهُ يَتَعَظَّمُ عَلَى الْكُلِّ.
\par 38 وَيُكْرِمُ إِلَهَ الْحُصُونِ فِي مَكَانِهِ وَإِلَهاً لَمْ تَعْرِفْهُ آبَاؤُهُ يُكْرِمُهُ بِالذَّهَبِ وَالْفِضَّةِ وَبِالْحِجَارَةِ الْكَرِيمَةِ وَالنَّفَائِسِ.
\par 39 وَيَفْعَلُ فِي الْحُصُونِ الْحَصِينَةِ بِإِلَهٍ غَرِيبٍ. مَنْ يَعْرِفُهُ يَزِيدُهُ مَجْداً وَيُسَلِّطُهُمْ عَلَى كَثِيرِينَ وَيَقْسِمُ الأَرْضَ أُجْرَةً.
\par 40 [فَفِي وَقْتِ النِّهَايَةِ يُحَارِبُهُ مَلِكُ الْجَنُوبِ فَيَثُورُ عَلَيْهِ مَلِكُ الشِّمَالِ بِمَرْكَبَاتٍ وَفُرْسَانٍ وَسُفُنٍ كَثِيرَةٍ وَيَدْخُلُ الأَرَاضِيَ وَيَجْرُفُ وَيَطْمُو.
\par 41 وَيَدْخُلُ إِلَى الأَرْضِ الْبَهِيَّةِ فَيُعْثَرُ كَثِيرُونَ وَهَؤُلاَءِ يُفْلِتُونَ مِنْ يَدِهِ: أَدُومُ وَمُوآبُ وَرُؤَسَاءُ بَنِي عَمُّونَ.
\par 42 وَيَمُدُّ يَدَهُ عَلَى الأَرَاضِي وَأَرْضُ مِصْرَ لاَ تَنْجُو.
\par 43 وَيَتَسَلَّطُ عَلَى كُنُوزِ الذَّهَبِ وَالْفِضَّةِ وَعَلَى كُلِّ نَفَائِسِ مِصْرَ. وَاللُّوبِيُّونَ وَالْكُوشِيُّونَ عِنْدَ خَطَوَاتِهِ.
\par 44 وَتُفْزِعُهُ أَخْبَارٌ مِنَ الشَّرْقِ وَمِنَ الشِّمَالِ فَيَخْرُجُ بِغَضَبٍ عَظِيمٍ لِيُخْرِبَ وَلِيُحَرِّمَ كَثِيرِينَ.
\par 45 وَيَنْصُبُ فُسْطَاطَهُ بَيْنَ الْبُحُورِ وَجَبَلِ بَهَاءِ الْقُدْسِ وَيَبْلُغُ نِهَايَتَهُ وَلاَ مُعِينَ لَهُ.

\chapter{12}

\par 1 [وَفِي ذَلِكَ الْوَقْتِ يَقُومُ مِيخَائِيلُ الرَّئِيسُ الْعَظِيمُ الْقَائِمُ لِبَنِي شَعْبِكَ وَيَكُونُ زَمَانُ ضِيقٍ لَمْ يَكُنْ مُنْذُ كَانَتْ أُمَّةٌ إِلَى ذَلِكَ الْوَقْتِ. وَفِي ذَلِكَ الْوَقْتِ يُنَجَّى شَعْبُكَ كُلُّ مَنْ يُوجَدُ مَكْتُوباً فِي السِّفْرِ.
\par 2 وَكَثِيرُونَ مِنَ الرَّاقِدِينَ فِي تُرَابِ الأَرْضِ يَسْتَيْقِظُونَ هَؤُلاَءِ إِلَى الْحَيَاةِ الأَبَدِيَّةِ وَهَؤُلاَءِ إِلَى الْعَارِ لِلاِزْدِرَاءِ الأَبَدِيِّ.
\par 3 وَالْفَاهِمُونَ يَضِيئُونَ كَضِيَاءِ الْجَلَدِ وَالَّذِينَ رَدُّوا كَثِيرِينَ إِلَى الْبِرِّ كَالْكَوَاكِبِ إِلَى أَبَدِ الدُّهُورِ.
\par 4 [أَمَّا أَنْتَ يَا دَانِيآلُ فَأَخْفِ الْكَلاَمَ وَاخْتِمِ السِّفْرَ إِلَى وَقْتِ النِّهَايَةِ. كَثِيرُونَ يَتَصَفَّحُونَهُ وَالْمَعْرِفَةُ تَزْدَادُ].
\par 5 فَنَظَرْتُ أَنَا دَانِيآلَ وَإِذَا بِاثْنَيْنِ آخَرَيْنِ قَدْ وَقَفَا وَاحِدٌ مِنْ هُنَا عَلَى شَاطِئِ النَّهْرِ وَآخَرُ مِنْ هُنَاكَ عَلَى شَاطِئِ النَّهْرِ.
\par 6 وَقَالَ لِلرَّجُلِ اللاَّبِسِ الْكَتَّانِ الَّذِي مِنْ فَوْقِ مِيَاهِ النَّهْرِ: [إِلَى مَتَى انْتِهَاءُ الْعَجَائِبِ؟]
\par 7 فَسَمِعْتُ الرَّجُلَ اللاَّبِسَ الْكَتَّانِ الَّذِي مِنْ فَوْقِ مِيَاهِ النَّهْرِ إِذْ رَفَعَ يُمْنَاهُ وَيُسْرَاهُ نَحْوَ السَّمَاوَاتِ وَحَلَفَ بِالْحَيِّ إِلَى الأَبَدِ: [ إِنَّهُ إِلَى زَمَانٍ وَزَمَانَيْنِ وَنِصْفٍ. فَإِذَا تَمَّ تَفْرِيقُ أَيْدِي الشَّعْبِ الْمُقَدَّسِ تَتِمُّ كُلُّ هَذِهِ].
\par 8 وَأَنَا سَمِعْتُ وَمَا فَهِمْتُ. فَقُلْتُ: [يَا سَيِّدِي مَا هِيَ آخِرُ هَذِهِ؟]
\par 9 فَقَالَ: [اذْهَبْ يَا دَانِيآلُ لأَنَّ الْكَلِمَاتِ مَخْفِيَّةٌ وَمَخْتُومَةٌ إِلَى وَقْتِ النِّهَايَةِ.
\par 10 كَثِيرُونَ يَتَطَهَّرُونَ وَيُبَيَّضُونَ وَيُمَحَّصُونَ أَمَّا الأَشْرَارُ فَيَفْعَلُونَ شَرّاً. وَلاَ يَفْهَمُ أَحَدُ الأَشْرَارِ لَكِنِ الْفَاهِمُونَ يَفْهَمُونَ.
\par 11 وَمِنْ وَقْتِ إِزَالَةِ الْمُحْرَقَةِ الدَّائِمَةِ وَإِقَامَةِ رِجْسِ الْمُخَرَّبِ أَلْفٌ وَمِئَتَانِ وَتِسْعُونَ يَوْماً.
\par 12 طُوبَى لِمَنْ يَنْتَظِرُ وَيَبْلُغُ إِلَى الأَلْفِ وَالثَّلاَثِ مِئَةٍ وَالْخَمْسَةِ وَالثَّلاَثِينَ يَوْماً.
\par 13 أَمَّا أَنْتَ فَاذْهَبْ إِلَى النِّهَايةِ فتَسْتَريحَ وتَقُومَ لِقُرعَتِكَ فِي نِهَايَةِ الأَيَّامِ].

\end{document}