\begin{document}

\title{مزامير داود}

\chapter{152}

\par \textit{قالها داود عندما كان يصارع الأسد والذئب اللذين أخذا خروفًا من قطيعه.}

\par 1 يا الله، يا الله، تعال لمساعدتي؛ ساعدني وخلصني؛ نجِّ روحي من القاتل

\par 2 هل أنزل إلى الهاوية بفم الأسد؟ أم يخزيني الذئب؟

\par 3 ألم يكفيهم أنهم كمنوا لقطيع أبي، ومزقوا شاة من قطيع أبي، بل أرادوا أن يهلكوا نفسي أيضًا؟

\par 4 ارحم يا رب، وخلص قدوسك من الهلاك، لكي يُتمجد بمجدك في كل أوقاته، ويُسبّح اسمك العظيم

\par 5 عندما أنقذته من يدي الأسد المهلك والذئب المفترس، وعندما أنقذت أسري من أيدي الوحوش البرية

\par 6 يا سيدي (أدوناي)، أرسل من أمامك منقذًا سريعًا، وأخرجني من الحفرة السحيقة التي تسجنني في أعماقها

\chapter{153}

\par \textit{قالها داود عند شكره لله الذي نجّاه من الأسد والذئب وقتلهما معًا.}

\par 1 سبحوا الرب يا جميع الأمم، مجدوه وباركوا اسمه.

\par 2 الذي أنقذ نفس مختاريه من أيدي الموت، ونجا قديسيه من الهلاك.

\par 3 وأنقذني من فخاخ الهاوية، ونفسي من الحفرة التي لا عمق لها.

\par 4 لأنه قبل أن يخرج خلاصي من أمامه، كنت على وشك أن أتمزق إلى نصفين بواسطة وحشين بريين.

\par 5 "ولكنه أرسل ملاكه وأغلق عني الأفواه الفاغرة وأنقذ نفسي من الهلاك."

\par 6 تُمَجِّدُهُ نَفْسِي وَتُرَفِّعُهُ مِنْ أَجْلِ كُلِّ إِحْسَانِهِ الَّذِي صَنَعَ وَيَصْنَعُهُ إِلَيَّ.

\chapter{154}

\par \textit{صلاة حزقيا عندما حاصره الأعداء.}

\par 1 بأصوات عظيمة مجّدوا الله، وفي جماعة كثيرة أخبروا بمجده.

\par 2 في وسط جمهور المستقيمين مجدوه، ومع الصديقين تكلموا بمجده.

\par 3 انضموا (حرفيًا، أرواحكم) إلى الخير والكامل، لتمجيد العلي.

\par 4 اجتمعوا معًا لإظهار قوته، ولا تتباطأوا في إظهار خلاصه [وقوته] ومجده لجميع الأطفال.

\par 5 لكي يُعرَف مجد الرب أُعطيت الحكمة، ولكي يُخبَر الناس بأعماله عُرِفَت.

\par 6 ليُعَرِّف الأطفال بقوته، وليُعَلِّم الذين يفتقرون إلى الفهم (حرفيًا، القلب) مجده.

\par 7 البعيدون عن مداخله والبعيدون عن أبوابه:

\par 8 لأن رب يعقوب قد ارتفع ومجده على كل أعماله.

\par 9 والإنسان الذي يُمجِّد العليَّ، فبه يُسرُّ، كما بمن يُقدِّم طعامًا فاخرًا، وكما بمن يُقدِّم تيوسًا وعجولًا

\par 10 وكما في من يُسمِّن المذبح بكثرة من المحرقات، وكرائحة البخور من أيدي الصديقين

\par 11 من أبوابك المستقيمة يسمع صوته، ومن صوت التنديد المستقيم.

\par 12 ويكون في أكلهم شبعان بالحق، وفي شربهم إذا اشتركوا

\par 13 مسكنهم في شريعة العلي، وكلامهم للتعريف بقوته

\par 14 ما أبعد كلام الأشرار عنه، وما أبعد كل المخالفين عن معرفته!

\par 15 هوذا عين الرب ترحم الصالحين، ويكثر الرحمة للذين يمجدونه، ومن وقت الشر ينجّي نفوسهم

\par 16 تبارك الرب الذي أنقذ البائسين من يد الأشرار، الذي أقام قرنًا من يعقوب، وقاضيًا للأمم من إسرائيل

\par 17 لكي يُطيل إقامته في صهيون، ويُزين عصرنا في أورشليم

\chapter{155}

\par \textit{عندما حصل الشعب على إذن من كورش بالعودة إلى دياره.}

\par 1 يا رب، إليك صرختُ، فاستجب لي.

\par 2 رفعت يدي إلى مسكنك المقدس، أمل أذنك إلي.

\par 3 وأعطني سؤلي، ولا تحجبني عني صلاتي.

\par 4 ابنِ نفسي ولا تهلكها، ولا تكشفها أمام الأشرار.

\par 5 أولئك الذين يجازون الشرور، أبعدهم عني، يا قاضي الحق

\par 6 يا رب، لا تحاكمني حسب خطاياي، لأنه ليس هناك جسد بريء أمامك

\par 7 أوضح لي يا رب شريعتك، وعلمني أحكامك.

\par 8 فيسمع كثيرون بأعمالك، وتحمد الأمم مجدك.

\par 9 اذكرني ولا تنساني، ولا تدخلني في ما يشق عليّ

\par 10 خطايا شبابي تجعلك تعبر عني، وتأديبي لا يذكرني

\par 11 طهّرني يا رب من البرص الشرير، ولا يعود يأتي إليّ

\par 12 جفف جذورها في داخلي، ولا تنبت أوراقها في داخلي.

\par 13 عظيم أنت يا رب، ولذلك تتحقق طلباتي من أمامك.

\par 14 إلى من أشكو فيعطيني؟ وماذا تزيدني قوة الناس؟

\par 15 من أمامك يا رب ثقتي. صرخت إلى الرب فاستجاب لي وشفى كسر قلبي.

\par 16 لقد نمت ونمت، حلمت فأغيثت، وكان الرب يعينني.

\par 17 لقد جرحوا قلبي بشدة، وسأشكرهم لأن الرب خلصني.

\par 18 الآن سأفرح بخزيهم. عليك توكلت ولن أخزى

\par 19 أعطِك الشرف إلى الأبد، وإلى أبد الآبدين.

\par 20 أنقذ إسرائيل مختاريك، وبيت يعقوب مجربك.

\end{document}