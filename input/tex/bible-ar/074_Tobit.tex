\begin{document}

\title{طوبيا}


\chapter{1}

\par 1 سفر أقوال طوبيا بن طوبيئيل بن عنانئيل بن عدويل بن جبائيل من نسل عسائيل من سبط نفتالي.
\par 2 الذي في زمن إنيمسار ملك الآشوريين، أُخذ أسيرًا من ثيسبي، التي على يمين تلك المدينة، التي تُدعى نفتالي تحديدًا في الجليل فوق آشير
\par 3 أنا طوبيا، سلكتُ كل أيام حياتي في طرق الحق والعدل، وصنعتُ صدقات كثيرة لإخوتي ولأمتي الذين جاءوا معي إلى نينوى، إلى أرض الآشوريين
\par 4 ولما كنت في وطني، في أرض إسرائيل، وكنت فتىً صغيرًا، سقط كل سبط نفتالي أبي من بيت أورشليم، التي اختيرت من بين جميع أسباط إسرائيل، لكي تذبح جميع الأسباط هناك، حيث كُرِّس هيكل مسكن العلي وبُني إلى الأبد
\par 5 والآن جميع الأسباط الذين ثاروا معًا، وبيت أبي نفتالي، ذبحوا للبعل البقرة
\par 6 ولكنني كنت أذهب وحدي إلى أورشليم في الأعياد، كما هو مرسوم لجميع شعب إسرائيل بأمر أبدي، مع باكورة الثمار وأعشار الغلة مع أول ما جزّ، وكنت أقدمها على المذبح للكهنة بني هارون
\par 7 العشر الأول من كل الزاد أعطيته لبني هارون الذين خدموا في أورشليم. وعشر آخر بعته وذهبت وأنفقته كل سنة في أورشليم
\par 8 وأعطيت الثلث لمن يستحقه، كما أمرتني دبوره والدة أبي، لأني تركت يتيمة من أبي
\par 9 علاوة على ذلك، عندما بلغت سن الرشد، تزوجت حنة من عشيرتي، وأنجبت منها طوبيا
\par 10 ولما سُبحنا إلى نينوى، أكل جميع إخوتي وأقاربي من خبز الأمم
\par 11 ولكني منعت نفسي من الأكل.
\par 12 لأني ذكرت الله بكل قلبي
\par 13 ومنحني العلي نعمة ورضا أمام عدو، حتى أصبحتُ مُموِّنه
\par 14 وذهبت إلى ميديا، وأودعت عند جابائيل، شقيق جابريا، في راجيس، مدينة ميديا، عشرة وزنات من الفضة
\par 15 وبعد أن مات عدو مِصر، ملك سنحاريب ابنه مكانه، الذي اضطربت حالته، لدرجة أنني لم أستطع دخول ميديا
\par 16 وفي زمن عدوي، قدمت صدقات كثيرة لإخوتي، وأعطيت خبزي للجائعين،
\par 17 وملابسي للعريان، وإذا رأيت أحدًا من أمتي ميتًا أو مطروحًا على أسوار نينوى، دفنته
\par 18 وإن كان الملك سنحاريب قد قتل أحدًا عند مجيئه وهروبه من اليهودية، فقد دفنتهم سرًا؛ لأنه في غضبه قتل كثيرين؛ ولكن لم يتم العثور على الجثث عندما طلبها الملك
\par 19 ولما جاء واحد من أهل نينوى وشكا إليّ إلى الملك أني دفنتهم واختبأت، وعلمت أني مطلوب للقتل، انسحبت خوفاً.
\par 20 ثم أُخذت جميع ممتلكاتي بالقوة، ولم يبق لي شيء سوى زوجتي حنة وابني طوبيا
\par 21 ولم تمضِ خمسة وخمسون يومًا حتى قتله اثنان من أبنائه، فهربا إلى جبال أراراط، وملك ابنه سركيدونوس مكانه، ووُكِّل على حسابات أبيه وعلى جميع شؤونه أخياكاروس ابن عنائيل أخي
\par 22 فرجعتُ إلى نينوى بعد أن طلب مني أخياكاروس. وكان أخياكاروس ساقيًا وحافظًا للخاتم ومدبِّرًا ومشرفًا على الحسابات، فجعله سركيدونوس بجانبه، وكان ابن أخي

\chapter{2}

\par 1 ولما رجعت إلى البيت، وعادت إليّ زوجتي حنة مع ابني طوبيا، في عيد الخمسين، وهو عيد الأسابيع السبعة المقدس، أُعدّ لي عشاء جيد، فجلست لأتناوله
\par 2 ولما رأيتُ وفرةً من الطعام، قلتُ لابني: اذهب وأحضر أيَّ فقيرٍ تجده من إخوتنا ممن يذكر الرب، وها أنا أنتظرك
\par 3 فجاء أيضًا وقال: يا أبتاه، إن واحدًا من أمتنا مخنوق ومُلقى في السوق
\par 4 ثم قبل أن أتذوق أي لحم، نهضت، وأخذته إلى غرفة حتى غروب الشمس
\par 5 ثم رجعت، واغتسلت، وأكلت لحمي بثقل،
\par 6 متذكرين نبوءة عاموس حين قال: "تتحول أعيادكم إلى نوح، وكل أفراحكم إلى رثاء".
\par 7 لذلك بكيت، وبعد غروب الشمس ذهبت وحفرت قبرًا ودفنته
\par 8 لكن جيراني سخروا مني وقالوا: هذا الرجل لا يخاف بعد أن يُقتل من أجل هذا الأمر، فقد هرب، ومع ذلك فهو يدفن الموتى أيضًا
\par 9 وفي تلك الليلة أيضًا رجعتُ من الدفن، ونمت بجانب حائط داري، وأنا متنجسة ووجهي مكشوف
\par 10 ولم أكن أعلم بوجود عصافير في الجدار، ولأن عينيّ كانتا مفتوحتين، رمت العصافير روثًا دافئًا في عينيّ، وظهر بياض في عينيّ. فذهبت إلى الأطباء، لكنهم لم يساعدوني. علاوة على ذلك، كان أخياكاروس يطعمني حتى دخلت إليمايس
\par 11 وكانت زوجتي آنا تأخذ أعمال النساء للقيام بها.
\par 12 وأرسلتهم إلى أصحابها، فدفعوا لها أجرتها، وأعطوها أيضاً جدياً أيضاً.
\par 13 ولما كان في بيتي وبدأ يبكي، قلت لها: من أين هذا الجدي؟ أليس مسروقًا؟ رديه إلى أصحابه، لأنه لا يحل أكل المسروق
\par 14 فأجابتني: أُعطيت هبةً أكثر من الأجرة. فلم أصدقها، بل أمرتها أن تردها إلى أصحابها، فخجلتُ منها. فقالت لي: أين صدقتك وبرك؟ ها أنت وجميع أعمالك معروفة.

\chapter{3}

\par 1 ثم حزنتُ، فبكيت، وصليتُ في حزني قائلًا:
\par 2 يا رب، أنت عادل، وجميع أعمالك وجميع طرقك رحمة وحق، وأنت تحكم بالحق والعدل إلى الأبد
\par 3 اذكرني وانظر إليّ، ولا تعاقبني على خطاياي وجهلي وخطايا آبائي الذين أخطأوا قبلك
\par 4 لأنهم لم يطيعوا وصاياك، فأسلمتنا غنيمة وسبيًا وموتًا، ومثلًا للعار لجميع الأمم الذين تشتتنا بينهم
\par 5 والآن أحكامك كثيرة وحق. عاملني حسب خطاياي وخطايا آبائي، لأننا لم نحفظ وصاياك، ولم نسلك بالحق أمامك
\par 6 فالآن، عاملني بما تراه مناسبًا، وأمر أن تُؤخذ روحي مني، لأذوب وأصير ترابًا. لأنه خير لي أن أموت من أن أعيش، لأني سمعتُ تعييراتٍ كاذبة، وحزنتُ كثيرًا. فأمرني الآن أن أُخلَّص من هذا الضيق، وأذهب إلى المكان الأبدي. لا تصرف وجهك عني
\par 7 وحدث في ذلك اليوم نفسه، أنه في إكباتان، إحدى مدن ميديا، تعرضت سارة ابنة رعوئيل أيضًا للتوبيخ من قبل جواري أبيها؛
\par 8 لأنها كانت متزوجة من سبعة أزواج، قتلهم أسموديوس الروح الشرير، قبل أن يضطجعوا معها. قالوا: ألا تعلمين أنكِ خنقتِ أزواجكِ؟ لقد كان لكِ سبعة أزواج بالفعل، ولم تُسمَّي على اسم أيٍّ منهم
\par 9 لماذا تضربنا من أجلهم؟ إن ماتوا، فاذهب وراءهم، ولا نراك أبدًا يا ابني ولا يا ابنتي
\par 10 فلما سمعت هذه الأمور، حزنت حزنًا شديدًا حتى إنها ظنت أنها تخنق نفسها، وقالت: أنا ابنة وحيدة لأبي، وإن فعلت هذا، فسيكون ذلك عارا عليه، وسأجلب شيخوخته بحزن إلى القبر
\par 11 ثم صلت نحو النافذة وقالت: مبارك أنت أيها الرب إلهي، واسمك القدوس المجيد مبارك ومكرم إلى الأبد. فلتحمدك جميع أعمالك إلى الأبد
\par 12 والآن يا رب، وجهتُ عينيّ ووجهي نحوك،
\par 13 وقل أخرجني من الأرض فلا أسمع بعد عارا.
\par 14 أنت تعلم يا رب أنني طاهر من كل خطيئة مع الإنسان،
\par 15 "وأني لم أدنس اسمي ولا اسم أبي في أرض سبيّ، فأنا ابنة أبي الوحيدة، وليس له ولد يرثه، ولا قريب، ولا ابن حيّ له أتخذه زوجة. أزواجي السبعة ماتوا، فلماذا أعيش؟ ولكن إن كنت لا ترضين أن أموت، فأمري أن يُحترم أمري، وأن يُشفق عليّ، حتى لا أسمع عتابًا بعد."
\par 16 فاستجابت صلاتهما أمام جلال الله العظيم
\par 17 فأُرسل رافائيل ليشفيهما كليهما، أي ليزيل بياض عيني طوبيا، وليعطي سارة ابنة رعوئيل زوجة لطوبيا بن طوبيا، وليقيد أسموداوس الروح الشرير؛ لأنها كانت ملكًا لطوبيا بحق الميراث. وفي ذلك الوقت، عاد طوبيا إلى بيته، ودخل بيته، فنزلت سارة ابنة رعوئيل من عليتها

\chapter{4}

\par 1 في ذلك اليوم تذكر طوبيا المال الذي أودعه لدى جابائيل في غضب ميديا،
\par 2 وقال في نفسه: لقد تمنيتُ الموت، فلماذا لا أستدعي ابني طوبيا لأعطيه المال قبل أن أموت؟
\par 3 ثم دعاه وقال: يا ابني، عندما أموت فادفنني، ولا تحتقر أمك، بل أكرمها كل أيام حياتك، وافعل ما يرضيها ولا تحزنها
\par 4 اذكر يا بني أنها رأت لك مخاطر كثيرة عندما كنت في رحمها، وعندما تموت، ادفنها بجانبي في قبر واحد
\par 5 يا ابني، اذكر الرب إلهنا كل أيامك، ولا تدع إرادتك تُعرّضك للخطيئة أو لتعدي وصاياه. افعل الاستقامة كل حياتك، ولا تتبع طرق الإثم
\par 6 لأنه إذا تعاملت بالحق، فسوف تنجح أعمالك لك ولجميع الذين يعيشون بالعدل
\par 7 تصدّق من مالك، ومتى تصدّق فلا تحسد، ولا تصرف وجهك عن فقير، ووجه الله لا يصرف عنك
\par 8 إذا كان لديك وفرة، فتصدق بما يناسبها: إذا كان لديك القليل، فلا تخف من العطاء بما يتناسب مع ذلك القليل:
\par 9 فإنك تكنز لنفسك كنزًا جيدًا ليوم الضرورة
\par 10 لأن الصدقة تُنجي من الموت، ولا تدع أحدًا يدخل الظلمة
\par 11 لأن الصدقة عطية صالحة لكل من يعطيها أمام العلي
\par 12 احذر من كل زنا يا ابني، واختار زوجة من نسل آبائك، ولا تتزوج امرأة أجنبية ليست من سبط أبيك، لأننا أبناء الأنبياء نوح وإبراهيم وإسحاق ويعقوب. اذكر يا ابني أن آباءنا من البدء تزوجوا جميعًا نساء من أقربائهم، وباركوا في أولادهم، وسيرث نسلهم الأرض
\par 13 فالآن يا ابني أحب إخوتك ولا تحتقر في قلبك إخوتك بني وبنات شعبك حتى لا تأخذ منهم امرأة. لأن في الكبرياء هلاكاً وضيقاً كثيراً، وفي الفجور انحلالاً وفقراً عظيماً. لأن الفجور أم الجوع.
\par 14 لا تَبْقَ لديك أجرة أي إنسان عمل لك، بل أعطه إياها من يدك. لأنه إن كنت تخدم الله، فسوف يكافئك أيضًا. كن حذرًا يا ابني في كل ما تفعله، وكن حكيمًا في كل تصرفاتك
\par 15 لا تفعل ذلك بأحد تكرهه: لا تشرب خمرًا تُسكرك، ولا تدع السكر يرافقك في رحلتك
\par 16 أعطِ من خبزك للجائع، ومن ثيابك للعراة. وحسب فضلتك، أعطِ صدقة. ولا تحسد عينك إذا صنعت صدقة
\par 17 اسكب خبزك على دفن البار، ولا تعطي للأشرار شيئًا
\par 18 اطلب المشورة من كل حكماء، ولا تحتقر أي مشورة نافعة
\par 19 بارك الرب إلهك كل حين، واطلب منه أن يُستقام طريقك، وأن تنجح جميع سبلك ومشوراتك. لأنه ليس لكل أمة مشورة. لكن الرب نفسه يُعطي كل الخير، ويُذل من يشاء كما يشاء. والآن يا ابني، تذكر وصاياي، ولا تنزعها من ذهنك
\par 20 والآن أُعلن لهم أنني قد عهدت بعشر وزنات لجابائيل بن جبرياس في راجيس في ميديا
\par 21 ولا تخف يا بني من أن نصبح فقراء، فلديك ثروة طائلة إذا تقيتَ الله، وابتعدتَ عن كل معصية، وفعلتَ ما يُرضيه

\chapter{5}

\par 1 فأجاب طوبيا وقال: يا أبتِ، سأفعل كل ما أمرتني به
\par 2 ولكن كيف يمكنني استلام المال، وأنا لا أعرفه؟
\par 3 ثم أعطاه الصك وقال له: اطلب لك رجلاً يذهب معك ما دمت حياً، فأعطيه أجرته، واذهب وخذ الفضة.
\par 4 لذلك عندما ذهب ليبحث عن رجل، وجد رافائيل، وكان ملاكًا
\par 5 لكنه لم يكن يعلم، فقال له: هل تستطيع أن تذهب معي إلى راجيس؟ وهل تعرف تلك الأماكن جيدًا؟
\par 6 فقال له الملاك: أنا ذاهب معك، وأنا أعرف الطريق جيدًا، لأني نزلت عند أخينا جبائيل
\par 7 فقال له طوبيا: انتظرني حتى أخبر أبي.
\par 8 فقال له: اذهب ولا تبقَ. فدخل وقال لأبيه: هوذا قد وجدتُ من يذهب معي. فقال: ادعه إليّ لأعرف من أي سبط هو، وهل هو رجلٌ أمينٌ للذهاب معك.
\par 9 فناداه، فدخل، وسلم بعضهم على بعض.
\par 10 فقال له طوبيا: يا أخي، أرني من أي سبط أنت ومن أي عشيرة أنت.
\par 11 فقال له: أتبحث عن سبط أو بيت أو أجير يذهب مع ابنك؟ فقال له طوبيا: أريد أن أعرف عشيرتك واسم ابنك أيها الأخ.
\par 12 فقال: أنا عزريا ابن حننيا الكبير ومن إخوتك
\par 13 فقال طوبيا: أهلاً وسهلاً بك يا أخي. لا تغضب عليّ الآن، لأني سألتُ عن سبطك وعائلتك، لأنك أخي من نسلٍ شريفٍ وجيد. لأني أعرف حننيا ويوناتاس، ابني ذلك السمايا العظيم، عندما ذهبنا معًا إلى أورشليم للسجود، وقدمنا ​​الأبكار وأعشار الثمار، ولم ينخدعا بضلال إخوتنا. يا أخي، أنت من نسلٍ شريف
\par 14 لكن أخبرني، ما الأجر الذي سأعطيك إياه؟ هل تريد درهمًا واحدًا في اليوم، وأشياء ضرورية، كما هو الحال بالنسبة لابني؟
\par 15 نعم، علاوة على ذلك، إذا عدت سالمًا، فسأضيف شيئًا إلى أجرك.
\par 16 ففرحوا. ثم قال لطوبيا: هيئ نفسك للسفر، فيُسهّل الله عليك السفر. ولما هيأ ابنه كل شيء للسفر، قال أبوه: اذهب مع هذا الرجل، وليُسهّل الله سفركما، وليُرافقكما ملاك الله. فخرجا كلاهما، وكلب الشاب معهما.
\par 17 فبكت أمه حنة وقالت لطوبيا: لماذا أرسلت ابننا؟ أليس هو عصا يدنا في الدخول والخروج أمامنا؟
\par 18 لا تكن جشعًا في إضافة المال إلى المال: بل دعه يكون قمامةً تجاه طفلنا
\par 19 لأن ما أعطانا الرب لنعيش به يكفينا.
\par 20 فقال لها طوبيا لا تقلقي يا أختي فإنه يرجع بسلامة وعيناك تنظرانه.
\par 21 لأن الملاك الصالح سيرافقه، وستكون رحلته مزدهرة، وسيعود سالمًا
\par 22 ثم توقفت عن البكاء.

\chapter{6}

\par 1 وفيما هم سائرون في رحلتهم وصلوا في المساء إلى نهر دجلة وباتوا هناك.
\par 2 وعندما نزل الشاب ليغتسل، قفزت سمكة من النهر، وكادت أن تلتهمه
\par 3 ثم قال له الملاك: خذ السمكة. فأمسك الشاب السمكة وجذبها إلى الأرض
\par 4 الذي قال له الملاك: افتح السمكة، وخذ القلب والكبد والمرارة، وضعها في مكان آمن
\par 5 ففعل الشاب كما أمره الملاك، وبعد أن شويا السمك، أكلاه، ثم مضى كلاهما في طريقهما حتى اقتربا من إكباتان
\par 6 ثم قال الشاب للملاك: يا أخي عزريا، ما فائدة قلب وكبد ولحم الحوت؟
\par 7 فقال له: «أما القلب والكبد، فإن أزعج أحدًا شيطان أو روح شريرة، فعلينا أن ندخن منه أمام الرجل أو المرأة، فلا يضطرب الجمع بعد».
\par 8 وأما المرارة، فمن الجيد أن تدهن بها رجلاً في عينيه بياض، فيشفى
\par 9 ولما اقتربوا من راجيس،
\par 10 قال الملاك للشاب: يا أخي، سننزل اليوم عند رعوئيل، وهو ابن عمك. وله ابنة واحدة فقط اسمها سارة. سأتكلم عنها حتى تتزوجك
\par 11 لأن حقها لك، لأنك وحدك من أقاربها
\par 12 والفتاة جميلة وحكيمة: فاسمعني الآن، وسأكلم أباها؛ وعندما نعود من راجيس سنحتفل بالزواج: لأني أعلم أن رعوئيل لا يستطيع أن يزوجها لآخر وفقًا لشريعة موسى، بل سيكون مذنبًا بالموت، لأن حق الميراث لك أكثر من أي شخص آخر
\par 13 فأجاب الشاب الملاك: سمعت يا أخي عزريا أن هذه الفتاة قد أعطيت لسبعة رجال، ماتوا جميعًا في حجرة الزواج
\par 14 وأنا الآن الابن الوحيد لأبي، وأخشى أن أموت إذا دخلت عليها، كما حدث مع الأخرى من قبل: لأن روحًا شريرة تحبها، لا تؤذي أحدًا إلا من يأتي إليها؛ لذلك أخشى أيضًا أن أموت، فأُلقي بحياة أبي وأمي بسببي في القبر بحزن: لأنه ليس لهما ابن آخر ليدفنهما
\par 15 فقال له الملاك: ألا تذكر الوصايا التي أوصاك بها أبوك أن تتزوج امرأة من عشيرتك؟ فاسمع لي يا أخي، فإنها ستُزوج لك، ولا تحاسب على الروح الشرير، فإنها في هذه الليلة ستُزوج لك
\par 16 وعندما تدخل إلى حجرة الزواج، خذ رماد العطر، وضع عليه بعضًا من قلب وكبد السمكة، وأشعله بالدخان
\par 17 وسيشمها الشيطان ويهرب، ولن يعود أبدًا. ولكن عندما تأتيان إليها، انهضا كلاكما وصليا إلى الله الرحيم، الذي سيرحمكما ويخلصكما. لا تخافي، لأنها قد عُينت لكِ منذ البداية؛ وستحفظينها، وستذهب معكِ. علاوة على ذلك، أفترض أنها ستلد لكِ أولادًا. فلما سمع طوبيا هذه الأمور، أحبها، وارتبط قلبه بها تمامًا

\chapter{7}

\par 1 ولما وصلوا إلى إكباتان، أتوا إلى بيت رعوئيل، فاستقبلتهم سارة، وبعد أن سلموا على بعضهم البعض، أدخلتهم إلى البيت
\par 2 ثم قال راجويل لإدنا زوجته: ما أشبه هذا الشاب بابن عمي طوبيا!
\par 3 فسألهم رعوئيل: من أين أنتم أيها الإخوة؟ فقالوا لهم: نحن من بني نفتاليم المسبيين في نينوى
\par 4 ثم قال لهم: هل تعرفون طوبيا قريبنا؟ فقالوا: نعرفه. ثم قال: هل هو بصحة جيدة؟
\par 5 فقالوا: إنه حيّ وبصحة جيدة. فقال طوبيا: هو أبي
\par 6 ثم قفز راجويل وقبّله وبكى،
\par 7 وباركه وقال له: أنت ابن رجل أمين وصالح. فلما سمع أن طوبيا أعمى حزن وبكى.
\par 8 وكذلك بكت إدنا زوجته وسارة ابنته. بل استقبلوهما بفرح، وبعد أن ذبحوا كبشًا من الغنم، وضعوا مؤونة من اللحم على المائدة. ثم قال طوبيا لرافائيل، يا أخي عزريا، تحدث عن الأمور التي تحدثت عنها في الطريق، وليُنجز هذا العمل
\par 9 فأخبر رعوئيل بالأمر، فقال رعوئيل لطوبيا: كل واشرب وفرح
\par 10 لأنه يليق بك أن تتزوج ابنتي. ومع ذلك سأخبرك بالحق
\par 11 لقد زوجت ابنتي الرجال السبعة الذين ماتوا في تلك الليلة التي دخلوا فيها عليها. ومع ذلك، افرحوا الآن. لكن طوبيا قال: لن آكل شيئًا هنا حتى نتفق ونقسم بعضنا لبعض
\par 12 قال راجويل: إذن خذها من الآن فصاعدًا حسب العادة، فأنت ابن عمها، وهي لك، والله الرحيم يمنحك النجاح في كل شيء
\par 13 ثم دعا سارة ابنته، فجاءت إلى أبيها، فأمسك بيدها وأعطاها زوجة لطوبيا، قائلاً: انظر، خذها حسب شريعة موسى، واذهب بها إلى أبيك. وباركهما
\par 14 ودعا إدنا زوجته، وأخذ ورقة وكتب صك عهد وختمه
\par 15 ثم بدأوا في تناول الطعام.
\par 16 بعد ذلك، دعا راجويل زوجته إدنا، وقال لها: يا أختي، جهزي مخدعًا آخر، وأدخليها إلى هناك
\par 17 ففعلت كما أمرها، وأتت بها إلى هناك، وبكت، فقبلت دموع ابنتها، وقالت لها:
\par 18 تعزّي يا ابنتي، رب السماء والأرض يُفرّحكِ في حزنكِ هذا: تعزّي يا ابنتي

\chapter{8}

\par 1 وبعد أن تعشوا، أدخلوا طوبيا إليها
\par 2 وبينما هو ذاهب، تذكر كلام رافائيل، فأخذ رماد العطور، ووضع عليه قلب وكبد السمكة، وأشعل به دخانًا
\par 3 عندما شم الروح الشرير رائحته، هرب إلى أقصى أنحاء مصر، وقيده الملاك
\par 4 وبعد أن حُبستا معًا، نهض طوبيا من الفراش وقال: يا أختي، قومي ولنصلِّ لكي يرحمنا الله
\par 5 فابتدأ طوبيا يقول: مبارك أنت يا إله آبائنا، ومبارك اسمك القدوس المجيد إلى الأبد. لتباركك السماوات وكل خلائقك
\par 6 أنت خلقت آدم، وأعطيته حواء امرأته معينًا ومسندًا. ومنهم جاء البشر. أنت قلت: ليس جيدًا أن يكون الإنسان وحيدًا. فلنصنع له معينًا مثله
\par 7 والآن يا رب، لا أعتبر أختي هذه كرمًا، بل استقامة: لذلك قدر لنا برحمتك أن نشيخ معًا
\par 8 فقالت معه آمين.
\par 9 فنام كلاهما تلك الليلة. وقام رعوئيل وذهب وحفر قبرًا،
\par 10 قائلاً: أخشى أن يكون هو أيضًا قد مات.
\par 11 ولما دخل رعوئيل بيته،
\par 12 قال لزوجته إدنا: أرسلي إحدى الجاريات، ودعيها ترى إن كان حيًا. وإن لم يكن، فلندفنه ولا يعلم أحد
\par 13 ففتحت الجارية الباب ودخلت، فوجدتهما نائمين،
\par 14 فخرج وأخبرهم أنه حي.
\par 15 ثم سبح رعوئيل الله وقال: يا الله، أنت مستحق أن تُحمد بكل تسبيح طاهر ومقدس. لذلك فليسبحك قديسوك مع كل خلائقك، وليسبحك جميع ملائكتك ومختاروك إلى الأبد.
\par 16 أنت جدير بالثناء، لأنك أسعدتني؛ ولم يأتِ إليّ ما كنت أظنه؛ بل تعاملت معنا وفقًا لرحمتك العظيمة
\par 17 أنت جدير بالثناء لأنك رحمت اثنين كانا الابنين الوحيدين لوالديهما: ارحمهما يا رب، وأتم حياتهما بصحة وفرح ورحمة
\par 18 ثم أمر راجويل خدمه بملء القبر.
\par 19 وأقام العرس أربعة عشر يوما.
\par 20 لأنه قبل انتهاء أيام الزواج، أقسم له راجويل أنه لن يغادر حتى تنتهي أربعة عشر يومًا من الزواج؛
\par 21 ثم يأخذ نصف ممتلكاته، ويذهب بأمان إلى والده؛ ويأخذ الباقي بعد وفاتي أنا وزوجتي

\chapter{9}

\par 1 ثم دعا طوبيا رافائيل وقال له:
\par 2 أيها الأخ عزريا، خذ معك خادمًا وجملين، واذهب إلى راجيس ميديا ​​إلى غابائيل، وأحضر لي المال، وأحضره إلى العرس
\par 3 لأن راجويل أقسم أنني لن أرحل.
\par 4 ولكن أبي يحسب الأيام، وإذا تأخرت يحزن جداً.
\par 5 فخرج رافائيل وأقام عند جابائيل، وأعطاه الصك، فأخرج أكياسًا مختومة، وأعطاها له
\par 6 وفي الصباح الباكر خرجا كلاهما معًا، وجاءا إلى العرس. وبارك طوبيا امرأته

\chapter{10}

\par 1 وكان طوبيا أبوه يحسب كل يوم، ولما انقضت أيام السفر ولم يأتِ،
\par 2 فقال طوبيا: هل هم محاصرون أم مات جبائيل وليس من يعطيه الفضة؟
\par 3 ولذلك كان حزينًا جدًا.
\par 4 فقالت له امرأته: ابني مات لأنه طال إقامته، وبدأت تبكي عليه وتقول:
\par 5 والآن لم أعد أهتم بشيء يا ابني، منذ أن تركتك، يا نور عيني.
\par 6 فقال له طوبيا: اسكت ولا تهتم فإنه في أمان.
\par 7 فقالت: "اصمت ولا تخدعني، ابني قد مات". وكانت تخرج كل يوم في الطريق الذي سلكوه، ولا تأكل لحمًا في النهار، ولا تطيل الليالي تبكي على ابنها طوبيا، حتى انقضت أيام العرس الأربعة عشر التي أقسم رعوئيل أن يقضيها هناك. فقال طوبيا لرعوئيل: "أطلقني، لأن أبي وأمي لم يعودا يريانني".
\par 8 فقال له حماه: امكث عندي فأرسل إلى أبيك فيخبرانه كيف حالك
\par 9 فقال طوبيا: لا، بل دعني أذهب إلى أبي.
\par 10 فقام رعوئيل وأعطاه سارة امرأته ونصف أمواله وخدمه ومواشيه وفضة.
\par 11 وباركهم وأطلقهم قائلًا: «إله السماء يُهيئ لكم طريقًا موفقًا يا أبنائي».
\par 12 وقال لابنته: أكرم أباك وحماتك اللذين هما الآن والداكِ، حتى أسمع عنكِ خيرًا. وقبلها. وقالت إدنا أيضًا لطوبيا: ليُعِدْكَ رب السماء يا أخي الحبيب، وليُمْنحني أن أرى أولادك من ابنتي سارة قبل أن أموت، حتى أفرح أمام الرب. ها أنا أستودعك ابنتي أمانة خاصة؛ فأين هما فلا تُؤذِها

\chapter{11}

\par 1 بعد هذه الأمور، مضى طوبيا في طريقه، وهو يحمد الله لأنه وهبه رحلة موفقة، وبارك رعوئيل وإدنة زوجته، ومضى في طريقه حتى اقترب من نينوى
\par 2 فقال رافائيل لطوبيا: أنت تعلم يا أخي كيف تركت أباك
\par 3 لنسرع أمام زوجتك ونُعِدّ البيت.
\par 4 وخذ بيدك مرارة الحوت، فانطلقا، ​​فتبعهما الكلب.
\par 5 جلست آنا الآن تنظر نحو الطريق بحثًا عن ابنها.
\par 6 فلما رأته مقبلاً قالت لأبيه هوذا ابنك قادم والرجل الذي ذهب معه.
\par 7 فقال رافائيل: أعلم يا طوبيا أن أباك سيفتح عينيه
\par 8 لذلك ادهن عينيه بالمرارة، ثم وخزها وفركها، فيذهب البياض عن عينيه، فيراك.
\par 9 فركضت حنة وسقطت على عنق ابنها وقالت له: بما أنني رأيتك يا بني، فأنا راضية بالموت من الآن فصاعدًا. وبكى كلاهما.
\par 10 فخرج طوبيا أيضاً نحو الباب فتعثر، فركض إليه ابنه.
\par 11 وأمسك بأبيه وضرب بالمرارة في عيني أبيه قائلاً: كن مطمئناً يا أبي.
\par 12 ولما بدأت عيناه تحمران، فركهما؛
\par 13 وتقشر البياض من جانبي عينيه، وعندما رأى ابنه، سقط على رقبته
\par 14 فبكى وقال: مبارك أنت يا الله، ومبارك اسمك إلى الأبد، ومبارك جميع ملائكتك القديسين
\par 15 لأنكَ جلدتَني ورحمتَني، وها أنا أرى ابني طوبيا. فدخل ابنه فرحًا، وأخبر أباه بما حدث له في ميديا.
\par 16 فخرج طوبيا لاستقبال كنته إلى باب نينوى فرحًا ومسبحًا الله. والذين رأوه منطلقًا تعجبوا لأنه أبصر
\par 17 فشكر طوبيا أمامهم، لأن الله قد رحمه. ولما تقدم إلى سارة كنته، باركها قائلاً: مرحباً بكِ يا ابنة، مبارك الله الذي أتى بكِ إلينا، ومبارك أباك وأمك. وكان فرح في جميع إخوته الذين في نينوى
\par 18 وجاء أخياكاروس ونسباس ابن أخيه
\par 19 وأقيم عرس طوبيا سبعة أيام بفرح عظيم.

\chapter{12}

\par 1 ثم دعا طوبيا ابنه وقال له: يا ابني، انظر أن للرجل أجرته التي ذهبت معك، ويجب أن تعطيه أكثر
\par 2 فقال له طوبيا: يا أبتِ، ليس عليَّ بأس أن أعطيه نصف ما أحضرته
\par 3 لأنه أعادني إليك سالمًا، وشفى زوجتي، وأتى لي بالفضة، وكذلك شفاك
\par 4 ثم قال الرجل العجوز: إنه من حقه.
\par 5 فدعا الملاك وقال له: خذ نصف كل ما جئت به واذهب آمناً.
\par 6 ثم أخذهما كليهما على انفراد، وقال لهما: باركا الله، وسبحاه، وعظّماه، واحمداه على ما صنع معكما أمام جميع الأحياء. من الجيد أن نحمد الله، ونمجّد اسمه، وأن نُظهر أعمال الله بشرف؛ لذلك لا تتكاسلا عن تسبيحه
\par 7 من الجيد كتمان سر الملك، ولكن من الشرف الكشف عن أعمال الله. افعل الخير، ولن يمسك الشر
\par 8 الصلاة صالحة مع الصوم والصدقة والبر. القليل مع البر خير من الكثير مع الإثم. الصدقة خير من ادخار الذهب
\par 9 لأن الصدقة تُنجي من الموت، وتُزيل كل خطيئة. من يمارس الصدقة والبر يمتلئ بالحياة
\par 10 أما الذين يخطئون فهم أعداء لأنفسهم.
\par 11 لا أخفي عنك شيئًا، لأني قلت: إنه حسنٌ أن يُكتم سر الملك، ولكن من الكرامة أن تُكشف أعمال الله.
\par 12 والآن، عندما كنتِ تصلين، وكنتِ سارة، كنتُ أحضر تذكار صلواتكم أمام القدوس، وعندما كنتِ تدفنين الموتى، كنتُ معكِ أيضًا
\par 13 وعندما لم تتأخر في القيام وترك عشاءك والذهاب لتغطية الموتى، لم يكن عملك الصالح مخفيًا عني، بل كنت معك
\par 14 والآن أرسلني الله لأشفيك وسارة كنتك.
\par 15 أنا رافائيل، أحد الملائكة القديسين السبعة، الذين يقدمون صلوات القديسين، والذين يدخلون ويخرجون أمام مجد القدوس.
\par 16 فَاضْطَرَبَا كلاهما وسَقَطَا على وَجْهَيْهِمَيْهِمَا، لأَنَّهُمَا كَانَا خَائِفَينَ
\par 17 فقال لهم: لا تخافوا، لأنه سيكون لكم خير. فاحمدوا الله إذن
\par 18 لأني لم آتِ من فضلك، بل بمشيئة إلهنا. لذلك أحمده إلى الأبد
\par 19 كنتُ أظهِرُ لكم كلَّ هذه الأيام، ولكنني لم آكل ولم أشرب، ولكنكم رأيتم رؤيا
\par 20 والآن اشكروا الله، لأني ماضٍ إلى الذي أرسلني، وأكتب كل ما يحدث في كتاب
\par 21 فلما قاموا لم يروه أيضًا.
\par 22 ثم اعترفوا بأعمال الله العظيمة والعجيبة، وكيف ظهر لهم ملاك الرب.

\chapter{13}

\par 1 ثم كتب طوبيا صلاة فرح وقال: مبارك الله الحي إلى الأبد، ومبارك ملكوته
\par 2 لأنه يجلد ويرحم. يهبط إلى الجحيم ويصعد. وليس من ينجو من يده
\par 3 اعترفوا به أمام الأمم يا بني إسرائيل، لأنه شتتنا بينهم
\par 4 هناك أعلنوا عظمته، ومجدوه أمام كل الأحياء: لأنه ربنا، وهو الله أبونا إلى الأبد
\par 5 ويجلدنا على آثامنا، ويرحمنا أيضًا، ويجمعنا من كل الأمم الذين بددنا بينهم
\par 6 إن رجعتم إليه بكل قلوبكم وبكل عقولكم، وتصرفتم بالاستقامة أمامه، فإنه يرجع إليكم ولا يحجب وجهه عنكم. لذلك انظروا ماذا يفعل بكم، واعترفوا به بكل أفواهكم، وسبحوا رب القوة، وسبحوا الملك الأبدي. في أرض سبيي أسبحه، وأخبر بقوته وجلاله لأمة خاطئة. أيها الخطاة، رجعوا وأجروا العدل أمامه: من يعلم إن كان سيقبلكم ويرحمكم؟
\par 7 أُسبِّح إلهي، وتُسبِّح نفسي ملك السماء، وتبتهج بعظمته
\par 8 فليتكلم جميع الناس، وليحمده الجميع على بره.
\par 9 يا أورشليم المدينة المقدسة، فإنه سيجلدك على أعمال أبنائك، ويرحم أيضًا أبناء الصديقين.
\par 10 سبحوا الرب لأنه صالح، وسبحوا الملك الأبدي، لكي يُبنى فيكم مسكنه بفرح، وليفرح فيكم الأسرى، وليحب فيكم البائسين إلى الأبد
\par 11 ستأتي أمم كثيرة من بعيد إلى اسم الرب الإله بهدايا في أيديهم، حتى هدايا لملك السماء؛ ستحمدك جميع الأجيال بفرح عظيم
\par 12 ملعون كل من يبغضك، ومبارك كل من يحبك إلى الأبد
\par 13 افرحوا وتهللوا لأبناء الصديقين، لأنهم يجتمعون ويباركون رب الصديقين.
\par 14 طوبى لمن يحبونك، لأنهم سيبتهجون بسلامك. طوبى لمن حزنوا على كل بلاياك، لأنهم سيبتهجون بك عندما يرون كل مجدك، ويفرحون إلى الأبد
\par 15 لتبارك روحي الله الملك العظيم.
\par 16 لأن أورشليم ستُبنى بالياقوت الأزرق والزمرد والأحجار الكريمة، وأسوارها وأبراجها وأسوارها من الذهب الخالص
\par 17 وتُرصَف شوارع أورشليم بالزبرجد والزمرد وحجارة أوفير
\par 18 وتقول جميع شوارعها: هللويا، ويسبحونه قائلين: مبارك الله الذي رفعه إلى الأبد

\chapter{14}

\par 1 فتوقف طوبيا عن تسبيح الله
\par 2 وكان عمره ثماني وخمسين سنة حين فقد بصره، فعاد إليه بعد ثماني سنين، وكان يتصدق، ويزداد في مخافة الرب الإله، ويحمده
\par 3 ولما كبر كثيرًا، دعا ابنه وأبناء ابنه، وقال له: يا ابني، خذ أولادك، لأني ها أنا قد كبرتُ، وأنا على وشك الرحيل من هذه الحياة
\par 4 اذهب إلى ميديا ​​يا بني، لأني أؤمن إيمانًا راسخًا بما قاله يونان النبي عن نينوى، أنها ستُقلب، وأن السلام سيسود ميديا ​​إلى حين، وأن إخوتنا سيتشتتون في الأرض من تلك الأرض الصالحة، وأن أورشليم ستكون خربة، وبيت الله فيها سيُحرق، وسيكون خربة إلى حين
\par 5 وأن الله سيرحمهم مرة أخرى، ويعيدهم إلى الأرض، حيث سيبنون هيكلًا، ولكن ليس مثل الهيكل الأول، حتى يتم وقت ذلك الدهر؛ وبعد ذلك سيعودون من جميع أماكن سبيهم، ويبنون أورشليم مجيدة، وسيُبنى بيت الله فيها إلى الأبد بناءً مجيدًا، كما تكلم الأنبياء عن ذلك
\par 6 وترجع جميع الأمم، وتتقي الرب الإله حقًا، وتدفن أصنامها
\par 7 هكذا تسبح جميع الأمم الرب، ويعترف شعبه بالله، ويرفع الرب شعبه، ويفرح جميع الذين يحبون الرب الإله بالحق والعدل، معطيين الرحمة لإخوتنا
\par 8 والآن يا بني، اذهب من نينوى، لأن ما تكلم به النبي يونان سيتحقق لا محالة
\par 9 بل احفظ الناموس والوصايا، وأظهر نفسك رحيمًا وعادلاً، لكي يحالفك الخير
\par 10 وادفنني دفنًا لائقًا وأمك معي، ولا تبقَ في نينوى بعد الآن. تذكر يا بني كيف عامل هامان أخياكاروس الذي رفعه، وكيف أخرجه من النور إلى الظلمة، وكيف كافأه مرة أخرى: ومع ذلك نجا أخياكاروس، أما الآخر فقد نال مكافأته لأنه نزل إلى الظلمة. تصدّق منسى، ونجا من فخاخ الموت التي نصبوها له، أما هامان فسقط في الفخ وهلك.
\par 11 والآن يا بني، انظر ماذا تفعل الصدقة، وكيف ينجّي البر. ولما قال هذا، أسلم الروح على الفراش، وهو ابن مئة وثمانية وخمسين سنة، ودفنه بإكرام
\par 12 ولما ماتت حنة أمه، دفنها مع أبيه. أما طوبيا، فذهب مع زوجته وأولاده إلى إكباتان إلى راجويل حميه،
\par 13 حيث شاخ بشرف، ودفن أباه وحماته بإكرام، ورث أموالهما، وأموال أبيه طوبيا
\par 14 وتوفي في إكباتان في ميديا، وكان عمره مائة وسبعة وعشرين سنة
\par 15 ولكن قبل وفاته، سمع عن دمار نينوى، التي استولى عليها نبوخذنصر وأشوروش: وقبل وفاته، فرح بنينوى

\end{document}