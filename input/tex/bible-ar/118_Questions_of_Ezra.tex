\begin{document}

\title{أسئلة عزرا}

\chapter{1}

\par \textit{المراجعة أ}

\par \textit{ما هو مصير الصالحين والخاطئين؟}

\par 1 رأى عزرا النبي ملاك الله وسأله سؤالاً تلو الآخر

\par 2 فاقترب منه الملاك وقال له: ماذا سيكون عند الانتهاء؟ فسأل النبي الملاك وقال: "ماذا أعد الله للأبرار والخطاة؟ وعند حلول يوم النهاية، ماذا سيحل بهم؟ إلى أين يذهبون، إلى التكريم أم إلى العذاب؟"

\par 3 أجاب الملاك وقال للنبي: «أُعِدَّ فرحٌ عظيمٌ ونورٌ أبديٌّ للأبرار، وأُعِدَّ للخطاة الظلمة الخارجية والنار الأبدية».

\par 4 قال النبي للملاك: «يا رب، من من الأحياء لم يخطئ إلى الله؟»

\par 5 وإن كان الأمر كذلك، فطوبى للوحوش والطيور التي لا تنتظر القيامة ولا تتوقع النهاية

\par 6 إذا كنت ستتوج الصالحين، الذين تحملوا كل العذابات، والأنبياء والشهداء عندما كانوا يأخذون الحجارة ويضربون وجوههم بالمطرقة حتى ظهرت أحشاؤهم،

\par 7 لقد عُذِّبوا من أجلك. ارحمنا نحن الخطاة الذين استحوذ عليهم الشيطان واستولى عليهم

\par \textit{وبَّخ النبي}

\par 8 أجاب الملاك وقال: "إذا كان هناك من هو فوقك، فلا تتحدث معه بعد الآن، وإلا فسيصيبك شر عظيم."

\par 9 قال النبي للملاك: "يا رب، أريد أن أتحدث معك أكثر قليلاً، فأجبني!"

\par 10 عندما يأتي يوم النهاية ويأخذ الروح، هل سيُرسلها إلى مكان العقاب أم إلى مكان الشرف حتى المجيء الثاني؟ [...]"

\par \textit{يوم النهاية}

\par 11 أجاب الملاك وقال: "لا تنتظروا يوم النهاية، بل سارعوا كالنسر الطائر إلى فعل الخير والرحمة

\par 12 لأن ذلك اليوم مخيف، عاجل، ودقيق.

\par 13 لا يسمح برعاية الأطفال أو الممتلكات. يأتي فجأةً كشخصٍ عديم الرحمة والنزاهة، يأخذ الأسير فجأةً، لا محالة. سواءٌ بكى أم صمت، لن يرحم.

\par \textit{الملائكة الطيبون والشريرون}

\par 14 «ولكن عندما يأتي يوم النهاية، يأتي ملاك صالح إلى الروح الصالحة، وملاك شرير إلى الروح الشريرة. تمامًا كما يُرسل الملوك شخصًا إلى فاعلي الشرور

\par 15 والأعمال الصالحة تكافئ الخير بالخير والشر بالشر، بنفس الطريقة التي يأتي بها الملاك الصالح إلى الروح الصالحة والشر بالشر. ليس أن الملاك شرير، بل أعمال كل إنسان شريرة

\par 16 يأخذ الروح، ويحملها إلى الشرق؛ تمر عبر الصقيع، عبر الثلج، عبر الظلام، عبر البَرَد، عبر الجليد، عبر العاصفة، عبر جيوش الشيطان، عبر الجداول، عبر رياح الأمطار الرهيبة، عبر مسارات رهيبة ومذهلة، عبر منحدرات ضيقة، وعبر جبال عالية

\par 17 "يا لها من طريقة عجيبة، لأن قدمًا واحدة خلف الأخرى، وأمامها أنهار نارية!"

\par 18 فذهل النبي وقال: "يا لهذا الطريق العجيب والرهيب!"

\par \textit{الخطوات السبع نحو الألوهية}

\par 19 قال الملاك: "إلى هذا الطريق سبعة معسكرات وسبع درجات إلى الألوهية، إذا استطعت أن أجعل (شخصًا) يمر عبره

\par 20 لأن المساكن الأولى سيئة وعجيبة؛ والثانية مخيفة ولا توصف؛ والثالثة جحيم وبرد قارس؛ والرابع مشاجرات وحروب؛ وفي الخامسة، إذًا، التحقيق - إذا كان بارًا، فإنه يضيء، وإذا كان خاطئًا، فإنه يُظلم؛ وفي السادسة، إذًا، تتألق روح الرجل الصالح كالشمس؛

\par 21 في اليوم السابع، بعد أن أحضرته، أجعله يقترب من عرش الألوهية العظيم، مقابل الحديقة، مواجهًا مجد الله حيث النور السامي

\par \textit{لا يمكن رؤية الله}

\par 22 قال النبي للملاك: "يا سيدي، عندما تجعله يمر بمثل هذه الأهوال، من خلال المشاجرات، من خلال الحروب، من خلال الحرارة الشديدة، لماذا لا تجعله يلتقي بالألوهية، بدلاً من أن تجعله يقترب من العرش فقط؟"

\par 23 قال الملاك للنبي: "أنت من الحمقى وتفكر وفقًا للطبيعة البشرية

\par 24 أنا ملاك، وأخدم الله دائمًا، ولم أرَ وجه الله. كيف تقول إن الإنسان الخاطئ يجب أن يُجبر على مقابلة الألوهية؟

\par 25 لأن اللاهوت مخيف وعجيب، ومن يجرؤ على النظر نحو اللاهوت غير المخلوق؟

\par 26 إذا نظر الإنسان، فسوف يذوب كالشمع أمام وجه الله: لأن اللاهوت ناري وعجيب. لأن مثل هؤلاء الحراس يقفون حول عرش اللاهوت

\par \textit{أولئك حول العرش الإلهي}

\par 27 «هناك محطات، [...] تجاويف، وأخرى نارية، وحاملو أحزمة، وفوانيس.»

\par 28 في ذلك المكان، هناك رعود، وزلازل، ومشاجرات، وحروب، وحرارة شديدة، وأهل النار، وأهل اللهب، و(و) جيوش نارية

\par 29 حوله سيرافيم غير متجسدين، كروبيم بستة أجنحة: بجناحين يغطون وجوههم، وبجناحين أرجلهم، ويطيرون باثنين، ويصرخون: «قدوس، قدوس، (قدوس) رب الجنود، السماء والأرض مملوءتان من مجدك».

\par 30 يقف هؤلاء الأوصياء حول عرش الألوهية.

\par \textit{تحرير الروح من الشيطان}

\par 31 سأل النبي الملاك وقال: "يا رب، ماذا سيحدث لنا، فنحن جميعًا خطاة وأُسرنا في يد الشيطان؟ والآن، بأي وسيلة ننجو أو من سيُخرجنا من يديه؟"

\par 32 أجاب الملاك وقال: "إذا بقي أحد بعد الموت، أبًا أو أمًا أو أخًا أو أختًا أو ابنًا أو ابنة أو أي مسيحي آخر، وقدم صلوات مع صوم لمدة أربعين يومًا، فسيكون هناك راحة عظيمة ورحمة من خلال ذبيحة المسيح

\par 33 لأن المسيح قُدِّم ذبيحةً من أجلنا على الصليب، ولستة عصور خلّص أرواحنا من يدي الشيطان

\par 34 كيف يتم خلاص الروح من خلال ما يقدمه الكاهن بتوقير، إذا أكمل الأربعين يومًا بطريقة ترضي الله!

\par 35 سيبقى في الكنيسة لمدة أربعين يومًا، ولن يذهب إلى الأماكن العامة، ولكنه سيتلو من وقت لآخر مزامير داود مع الصلوات

\par 36 هذا ما يُخرجنا من أيدي الشيطان. وإن لم يكن كذلك، فأعطِ الفقراء

\par \textit{طبيعة الصلاة}

\par 37 «لأن صلواتكم هي هكذا: كما يخرج المزارع ليزرع، فيخرج الغصن فرحًا ورشيقًا ويرغب في إنتاج ثمار كثيرة، وتأتي الأشواك والأعشاب أيضًا فتخنقه ولا تسمح له بجمع ثمار كثيرة

\par 38 وبالمثل، أنت أيضًا، عندما تدخل الكنيسة وترغب في تقديم الصلوات أمام الإله، فإن هموم هذا العالم وخداع العظمة (الثروة) تخرج وتخنقك ولا تدع ثمارًا كثيرة تُزرع

\par 39 لأنه لو كانت صلاتكم كصلاة موسى التي بكى عليها أربعين يومًا وتكلم مع الله فمًا إلى فم،


\par 40 وكذلك أيضًا رُفع إيليا إلى السماء في مركبة نارية، وكذلك دانيال أيضًا صلى في جب الأسد ..."

\par \textit{(انظر المراجعة ب، 10-14.)}

\chapter{2}

\par \textit{Recention B}

\par 1 رأى ملاك الله وسأل عن الأبرار والخطاة عندما يخرجون من هذا العالم

\par 2 قال الملاك: "للأبرار نور وراحة، حياة أبدية، أما الخطاة، فللخطاة عذاب لا ينتهي".

\par 3 قال عزرا: "إن كان الأمر كذلك، فطوبى للوحوش ووحوش البرية والزحافات وطيور السماء التي لا تنتظر القيامة والدينونة."

\par 4 قال الملاك: "أنت تُخطئ بقولك هذا، لأن الله خلق كل شيء من أجل الإنسان، والإنسان من أجل الله. وتلك الأشياء التي يجد الله فيها الإنسان، فبها يُحاسب."

\par 5 قال عزرا: "عندما تأخذ أرواح البشر، إلى أين ستأتي بها؟"

\par 6 قال الملاك: "أحضر أرواح الصالحين لعبادة الله وأُقيمهم في الغلاف الجوي العلوي، وأما أرواح الخطاة فتأسرها الشياطين المسجونة في الغلاف الجوي."

\par 7 فقال عزرا: «ومتى تتحرر النفس التي استولى عليها الشيطان؟»

\par 8 قال الملاك: «إذا كان للنفس ذكرٌ حسنٌ في الدنيا، فإنها تُخرجه من الشيطان بالدعاء والرحمة».

\par 9 قال عزرا: "بأي وسيلة؟" قال الملاك: "بالصلاة، وبالرحمة، وبالذبائح." (قال عزرا)

\par 10 «إذا لم يكن لروح الخاطئ ذكرى طيبة تساعده، فماذا سيحدث له؟»

\par 11 قال له الملاك: «مثل هذا في يد الشيطان إلى أن يأتي المسيح، حين ينفخ بوق جبرائيل

\par 12 ثم تتحرر الأرواح من أيدي الشيطان وترتفع من الغلاف الجوي

\par 13 "ويأتون ويتحد كل واحد منهم بجسده الذي كان قد عاد إلى التراب والذي بناه صوت البوق وأيقظه وجدده."

\par 14 ويرفعه أمام المسيح إلهنا الذي يأتي ليدين من على الأرض، أي الصالحين والأشرار، ويجازي كل واحد على أعماله. من خلال التماس أنبيائكم الذين روى عنهم الله، ارحموا قراء هذه الكتابة

\end{document}