\begin{document}

\title{بيل والتنين}

\chapter{1}

\par 1 وانضم الملك أستياجيس إلى آبائه، وحصل كورش الفارسي على مملكته.
\par 2 وتحدث دانيال مع الملك، وكان مُكرَّمًا فوق جميع أصدقائه
\par 3 وكان للبابليين صنم يُدعى بال، وكانوا يُنفقون عليه كل يوم اثنتي عشرة مكيالًا من الدقيق الفاخر، وأربعين شاة، وستة أوانٍ من الخمر
\par 4 وكان الملك يعبده ويذهب كل يوم ليسجد له. أما دانيال فكان يعبد إلهه. فقال له الملك: لماذا لا تعبد بيل؟
\par 5 فأجاب وقال: لأني لا أقدر أن أعبد أصنامًا مصنوعة بالأيادي، بل الله الحي الذي خالق السماء والأرض، والذي له السلطان على كل ذي جسد
\par 6 فقال له الملك: ألا تظن أن بيل إله حي؟ ألا ترى كم يأكل ويشرب كل يوم؟
\par 7 فابتسم دانيال وقال: أيها الملك، لا تغرّ، فهذا من داخل طين، ومن خارج نحاس، ولم يأكل ولم يشرب شيئًا قط
\par 8 فغضب الملك، ودعا كهنته، وقال لهم: إن لم تخبروني من هذا الذي يأكل هذه النفقات، فستموتون
\par 9 لكن إن استطعتم أن تثبتوا لي أن بال يلتهمهم، فإن دانيال يموت، لأنه جدف على بال. فقال دانيال للملك: فليكن حسب قولك
\par 10 وكان كهنة بل سبعين كهنة، ما عدا نسائهم وأولادهم. فدخل الملك مع دانيال إلى هيكل بل
\par 11 فقال كهنة بل: ها نحن نخرج. وأنت أيها الملك، ضع الطعام وجهّز الخمر، وأغلق الباب بإحكام واختمه بخاتمك
\par 12 وغدًا عندما تعود، إن لم تجد أن بيل قد أكل كل شيء، فسنموت، وإلا فدانيال الذي يتكلم علينا كذبًا
\par 13 ولم يكترثوا للأمر كثيرًا: فقد صنعوا تحت المائدة مدخلًا خاصًا، يدخلون منه باستمرار، ويستهلكون تلك الأشياء
\par 14 فلما خرجوا، وضع الملك مائدة أمام بال. وكان دانيال قد أمر عبيده بإحضار رماد، فنثروه في جميع أنحاء الهيكل أمام الملك وحده. ثم خرجوا وأغلقوا الباب وختموه بخاتم الملك، وانصرفوا هكذا
\par 15 وفي الليل، جاء الكهنة مع زوجاتهم وأطفالهم، كما اعتادوا أن يفعلوا، وأكلوا وشربوا كل شيء
\par 16 في الصباح، استيقظ الملك ودانيال معه.
\par 17 فقال الملك: يا دانيال، هل الخواتم سليمة؟ فقال: نعم أيها الملك، إنها سليمة.
\par 18 وبمجرد أن فتح الباب، نظر الملك إلى المائدة، وصرخ بصوت عالٍ: عظيم أنت يا بيل، وليس معك غش على الإطلاق
\par 19 فضحك دانيال، وأمسك الملك لئلا يدخل، وقال: انظر إلى الرصيف، ولاحظ جيدًا لمن هذه آثار الأقدام
\par 20 فقال الملك: أرى آثار أقدام رجال ونساء وأطفال. فغضب الملك.
\par 21 وأخذ الكهنة مع نسائهم وأولادهم، فأروه الأبواب السرية التي يدخلون منها، ويأكلون مما كان على المائدة
\par 22 لذلك قتلهم الملك، وأسلم بيل إلى يد دانيال، الذي هدمه هو وهيكله
\par 23 وفي ذلك المكان نفسه كان هناك تنين عظيم، وكان أهل بابل يعبدونه
\par 24 فقال الملك لدانيال: أتقول أيضًا إن هذا من نحاس؟ ها هو حي يأكل ويشرب. لا تستطيع أن تقول إنه ليس إلهًا حيًا، فاعبده
\par 25 ثم قال دانيال للملك: «سأسجد للرب إلهي لأنه هو الإله الحي».
\par 26 لكن أعطني الإذن، أيها الملك، وسأقتل هذا التنين بلا سيف ولا عصا. قال الملك: أعطيك الإذن
\par 27 فأخذ دانيال زفتا وشحما وشعرا، وطبخها معا، وصنع منها كتلا، ووضعها في فم التنين، فانشق التنين. فقال دانيال: هوذا هذه هي الآلهة التي تعبدونها
\par 28 فلما سمع أهل بابل ذلك، غضبوا غضبًا شديدًا، وتآمروا على الملك قائلين: إن الملك قد صار يهوديًا، وهلك بيل، وقتل التنين، وحكم على الكهنة بالموت
\par 29 فجاءوا إلى الملك وقالوا: سلم لنا دانيال وإلا هدمناك أنت وبيتك
\par 30 فلما رأى الملك أنهم يضغطون عليه بشدة، اضطر إلى دفع دانيال إليهم
\par 31 الذي ألقاه في جب الأسود: حيث بقي ستة أيام.
\par 32 وكان في الجب سبعة أسود، وكانوا يعطون لها كل يوم جثتين ونعجتين، فلم تعط لها حينئذ، لكي تفترس دانيال.
\par 33 وكان في اليهودية نبي يُدعى حبقوق، كان قد صنع عِصيدة، وكسر خبزًا في قصعة، وكان ينطلق إلى الحقل ليأتي به إلى الحصادين
\par 34 فقال ملاك الرب لحبقوق: اذهب واحمل الغداء الذي معك إلى بابل إلى دانيال الذي هو في جب الأسود
\par 35 فقال حبقوق: يا رب، لم أرَ بابل قط، ولا أعرف أين الجب
\par 36 ثم أخذ ملاك الرب من رأسه، وحمله من شعر رأسه، ووضعه في بابل على الجب بقوة روحه
\par 37 فنادى حبقوق قائلًا: يا دانيال، يا دانيال، خذ الغداء الذي أرسله لك الله
\par 38 فقال دانيال: يا الله، لقد ذكرتني، ولم تترك الذين يبحثون عنك ويحبونك
\par 39 فقام دانيال وأكل، وأعاد ملاك الرب حبقوق إلى مكانه في الحال
\par 40 وفي اليوم السابع ذهب الملك ليبكي على دانيال، فلما جاء إلى الجب نظر إلى الداخل، فإذا دانيال جالس
\par 41 ثم صرخ الملك بصوت عظيم وقال: عظيم أنت الرب إله دانيال وليس آخر غيرك.
\par 42 ثم سحبه وألقى في الجب أولئك الذين كانوا سبب هلاكه، فاُتِلِكوا في لحظة أمام وجهه

\end{document}