\begin{document}

\title{عزرا الثاني}

\chapter{1}

\par 1 الكتاب الثاني للنبي عزرا بن سرايا بن عزريا بن حلقيا بن صادومياس بن صادوق بن أخيطوب،
\par 2 ابن أخيا، بن فينحاس، بن هالي، بن أمريا، بن عزي، بن ماريموت، ابن بوريث، بن أبيساي، بن فينحاس، بن ألعازار،
\par 3 ابن هارون، من سبط لاوي، الذي كان أسيرًا في أرض مادي، في عهد أرتحشستا ملك الفرس
\par 4 وكان إليّ كلام الرب قائلاً:
\par 5 اذهب وأخبر شعبي بأفعالهم الخاطئة، وأولادهم بالشر الذي فعلوه ضدي، لكي يخبروا أبناء أبنائهم.
\par 6 لأن خطايا آبائهم قد كثرت فيهم، فقد نسوا أمري وذبحوا لآلهة غريبة
\par 7 ألست أنا الذي أخرجهم من أرض مصر، من بيت العبودية؟ لكنهم أغضبوني واستهانوا بمشوراتي
\par 8 فانزع شعر رأسك وألق عليهم كل شر، لأنهم لم يطيعوا شريعتي، بل هم شعب متمرد
\par 9 إلى متى أتحمل أولئك الذين أحسنت إليهم كثيرًا؟
\par 10 لقد أهلكت من أجلهم ملوكاً كثيرين، وضربت فرعون وعبيده وكل قوته.
\par 11 أهلكت جميع الأمم من أمامهم، وفي الشرق شتتت شعوب إقليمين، صور وصيدا، وقتلت جميع أعدائهم
\par 12 فكلمهم قائلاً: هكذا قال الرب،
\par 13 "أنا قدتكم عبر البحر، وفي البدء أعطيتكم طريقاً واسعاً وآمناً، وأعطيتكم موسى قائداً، وهارون كاهناً."
\par 14 أعطيتكم نورًا في عمود نار، وصنعت بينكم عجائب عظيمة، ولكنكم نسيتموني، يقول الرب
\par 15 هكذا قال الرب القدير: كانت السلوى علامة لكم، وأعطيتكم خيامًا لحماية أنفسكم، ومع ذلك تذمرتم هناك،
\par 16 ولم تنتصروا باسمي لهلاك أعدائكم، بل تتذمرون إلى هذا اليوم
\par 17 أين هي النعم التي صنعتها لكم؟ ألم تصرخوا إليّ حين كنتم جائعين وعطشانين في البرية،
\par 18 قائلين: لماذا أتيت بنا إلى هذه البرية لتقتلنا؟ كان خيرًا لنا أن نخدم المصريين من أن نموت في هذه البرية
\par 19 ثم شفقت على أحزانكم، وأعطيتكم المن لتأكلوه، فأكلتم خبز الملائكة
\par 20 ألم أشق الصخرة حين عطشتم، ففاضت المياه إلى شبعكم؟ من أجل الحر غطيتكم بأوراق الأشجار
\par 21 قسمت بينكم أرضًا مثمرة، وطردت من أمامكم الكنعانيين والفرزيين والفلسطينيين. فماذا أفعل لكم بعد؟ يقول الرب
\par 22 هكذا قال الرب القدير حين كنتم في البرية عند نهر الأموريين عطشانا ومجدفين على اسمي.
\par 23 لم أعطيكم نارًا لتجديفاتكم، بل ألقيت شجرة في الماء، فجعلت النهر عذبًا
\par 24 ماذا أفعل بك يا يعقوب؟ وأنت يا يهوذا لم تطيعني. ألجأ إلى أمم أخرى، فأعطي اسمي لهم، فيحفظوا فرائضي
\par 25 بما أنكم تركتموني، فسأترككم أيضًا؛ عندما تريدونني أن أكون كريمًا معكم، فلن أرحمكم
\par 26 في كل مرة تدعونني لا أسمعكم، لأنكم نجست أيديكم بالدماء، وأرجلكم سريعة إلى القتل
\par 27 لم تتركوني كأني، بل أنفسكم، يقول الرب
\par 28 هكذا قال الرب القدير: ألم أطلب إليك كأب لأبنائه، وكأم لبناتها، وكمرضعة لأطفالها،
\par 29 أن تكونوا لي شعبًا، وأنا أكون لكم إلهًا، وأن تكونوا لي أبناءً، وأنا أكون لكم أباءً؟
\par 30 جمعتكم كما تجمع الدجاجة فراخها تحت جناحيها، ولكن الآن ماذا أفعل بكم؟ سأطردكم من وجهي
\par 31 عندما تقدمون لي، سأحول وجهي عنكم، لأني تركت أعيادكم المقدسة، ورؤوس شهوركم، وختانكم
\par 32 أرسلت إليكم عبيدي الأنبياء، الذين أخذتموهم وقتلتموهم ومزقتم أجسادهم، وأطلب دمهم من أيديكم، يقول الرب
\par 33 هكذا قال الرب القدير: بيتكم خراب، سأطردكم كما تطرد الريح القش
\par 34 ولن يُثمر أولادكم، لأنهم احتقروا وصيتي، وعملوا الشر أمامي
\par 35 سأعطي بيوتكم لشعب قادم، لم يسمع بي بعد، فيؤمن بي، ولم أره آيات، ومع ذلك سيفعلون ما أوصيتهم به
\par 36 لم يروا أنبياء، ومع ذلك سيذكرون خطاياهم ويعترفون بها
\par 37 أشهد نعمة الناس القادمين، الذين يبتهج صغارهم فرحًا: ومع أنهم لم يروني بأعينهم الجسدية، إلا أنهم يؤمنون بالروح بما أقول
\par 38 والآن يا أخي، انظر إلى أي مجد هذا؟ وانظر إلى الناس القادمين من الشرق:
\par 39 الذين سأعطيهم قادةً: إبراهيم وإسحاق ويعقوب، وأوشع وعاموس وميخا، ويوئيل وعبديا ويونس،
\par 40 ناحوم، وأباكوق، وصفونيا، وحجّاس، وزكريا، وملاخي، الذي يُدعى أيضًا ملاك الرب

\chapter{2}

\par 1 هكذا قال الرب: إني أخرجت هذا الشعب من العبودية، وأعطيتهم وصاياي عن يد عبيد الأنبياء، فلم يسمعوا، بل احتقروا مشوراتي
\par 2 قالت لهم أمهم التي ولدتهم: اذهبوا أيها الأولاد، لأني أرملة ومتروكة
\par 3 لقد ربيتكم بالفرح، ولكن بالحزن والغم أهلكتكم، لأنكم أخطأتم أمام الرب إلهكم، وعملتم الشر أمامه.
\par 4 ولكن ماذا أفعل بكم الآن؟ أنا أرملة ومتروكة. اذهبوا يا أبنائي واطلبوا رحمة الرب
\par 5 أما أنا يا أبت، فأستحلفك أن تشهد على أم هؤلاء الأطفال التي لم تحفظ عهدي،
\par 6 لكي تُخزيهم، وتُسلب أمهم، فلا يكون لهم ذرية
\par 7 ليتبددوا بين الأمم، ولتُمحَ أسماؤهم من الأرض، لأنهم احتقروا عهدي
\par 8 ويل لك يا أشور، يا من تخفي الأشرار فيك! أيها الأشرار، تذكروا ما فعلته بسدوم وعمورة؛
\par 9 الذين أرضهم أكوام من الزفت والرماد، هكذا أفعل أيضًا بالذين لا يسمعون لي، يقول الرب القدير
\par 10 هكذا قال الرب لعزرا: قل لشعبي أني سأعطيهم مملكة أورشليم التي كنت سأعطيها لإسرائيل
\par 11 وسآخذ مجدهم أيضًا، وأعطيهم المظال الأبدية التي أعددتها لهم
\par 12 ستكون لهم شجرة الحياة دهنًا طيب الرائحة، لا يتعبون ولا يكلون
\par 13 اذهبوا، فتنالوا. صلوا من أجل أيام قليلة، لكي تختصر. الملكوت مُعدّ لكم بالفعل. اسهروا
\par 14 اشهدوا السماء والأرض، لأني حطمت الشر وخلقت الخير، لأني أنا حي، يقول الرب
\par 15 يا أم، احتضني أولادك، وربيهم بفرح، واجعلي أقدامهم ثابتة كالعمود، لأني اخترتك، يقول الرب
\par 16 وسأقيم الأموات من أماكنهم، وأخرجهم من القبور، لأني عرفت اسمي في إسرائيل
\par 17 لا تخافي يا أم البنين، لأني اخترتك، يقول الرب
\par 18 لمساعدتك، سأرسل خادميّ عيسو وإرميا، اللذين بناءً على مشورتهما قدّستُ وأعددتُ لك اثنتي عشرة شجرة محملة بثمار متنوعة،
\par 19 وكينابيع كثيرة تتدفق لبنًا وعسلًا، وسبعة جبال عظيمة، تنمو عليها الورود والزنابق، والتي سأملأ بها أطفالك بالفرح
\par 20 أنصف الأرملة، اقضِ لليتيم، أعطِ الفقير، دافع عن اليتيم، كسِ العريان،
\par 21 اشفِ المنكسر والضعيف، ولا تستهزئ بالأعرج، ودافع عن المشوه، ودع الأعمى يدخل إلى بصر صفائي
\par 22 أبقِ الكبار والصغار داخل جدرانك.
\par 23 أينما وجدت الموتى، خذهم وادفنهم، وسأعطيك المكان الأول في قيامتي.
\par 24 اسكنوا يا شعبي، واستريحوا، لأن هدوئكم لا يزال قادمًا
\par 25 أطعمي أطفالك، أيتها المرضعة الصالحة؛ ثبتي أقدامهم.
\par 26 وأما العبيد الذين أعطيتك إياهم فلا يهلك منهم أحد لأني من بينك أطلبهم.
\par 27 لا تكل، لأنه عندما يأتي يوم الضيق والثقل، سيبكي الآخرون ويحزنون، لكنك ستفرح وتنعم بالوفرة
\par 28 سيحسدك الأمم، لكنهم لن يستطيعوا أن يفعلوا ضدك شيئًا، يقول الرب
\par 29 ستغطيك يداي، حتى لا يرى أطفالك الجحيم.
\par 30 افرحي أيتها الأم بأولادك، لأني سأنقذك، يقول الرب.
\par 31 اذكر أولادك الراقدين، لأني سأخرجهم من أقاصي الأرض، وأرحمهم، لأني أنا رحيم، يقول الرب القدير
\par 32 احتضني أطفالك حتى آتي وأظهر لهم الرحمة، لأن آباري تفيض، ونعمتي لا تزول
\par 33 أنا عزرا تلقيت وصية من الرب على جبل غراب، بأن أذهب إلى إسرائيل؛ ولكن عندما أتيت إليهم، استهزأوا بي، واحتقروا وصية الرب
\par 34 ولذلك أقول لكم، أيها الوثنيون، الذين يسمعون ويفهمون، ابحثوا عن راعيكم، فهو يعطيكم راحة أبدية، لأنه قريب، الذي سيأتي في نهاية العالم
\par 35 استعدوا لمكافأة الملكوت، لأن النور الأبدي سيضيء عليكم إلى الأبد
\par 36 اهرب من ظل هذا العالم، واستقبل بهجة مجدك: أشهد لمخلصي علانية
\par 37 اقبلوا العطية التي أُعطيت لكم، وافرحوا، شاكرين من قادكم إلى الملكوت السماوي
\par 38 قم وقف، وانظر عدد الذين خُتموا في وليمة الرب؛
\par 39 الذين رحلوا عن ظل العالم، وتلقوا ثياب الرب المجيدة
\par 40 خذي عددك يا ​​صهيون، واغلقي على من يرتدون البياض من بنيك، الذين أتمّوا ناموس الرب
\par 41 لقد اكتمل عدد أبنائك الذين تمنيتهم: اطلب قوة الرب، لكي يتقدس شعبك الذي دُعي منذ البداية
\par 42 أنا عزرا رأيت على جبل صهيون شعبًا عظيمًا لم أستطع أن أحصيه، وكانوا جميعًا يسبحون الرب بالأغاني
\par 43 وكان في وسطهم شاب طويل القامة، أطول من كل الباقين، وعلى كل رأس منهم وضع تيجانًا، وكان أعلى، الأمر الذي تعجبت منه جدًا
\par 44 فسألت الملاك وقلت: يا سيدي، ما هذه؟
\par 45 فأجابني وقال لي: هؤلاء هم الذين خلعوا الثوب المميت ولبسوا الخالد واعترفوا باسم الله. الآن قد تكللوا وأخذوا سعف النخل.
\par 46 ثم قلت للملاك: من هو الشاب الذي يتوجهم ويعطيهم سعف النخيل في أيديهم؟
\par 47 فأجابني وقال لي: هو ابن الله الذي اعترفوا به في العالم. فبدأت أُشيد بشدة بالذين ثبتوا في سبيل اسم الرب.
\par 48 ثم قال لي الملاك: اذهب وأخبر شعبي بما رأيت من عجائب الرب إلهك وما أعظمها

\chapter{3}

\par 1 في السنة الثلاثين بعد خراب المدينة، كنت في بابل، وكنتُ مُضطجعًا على فراشي مضطربًا، وخطرتُ أفكاري على قلبي
\par 2 لأني رأيت خراب صهيون، وغنى سكان بابل
\par 3 وتأثرت روحي بشدة، حتى أنني بدأت أتحدث بكلمات مليئة بالخوف إلى العلي، وقلت،
\par 4 أيها الرب، الذي يحكم، لقد تكلمت في البداية، عندما غرست الأرض، وذلك بنفسك وحدك، وأمرت الشعب،
\par 5 وأعطيت آدم جسدًا بلا روح، وهو عمل يديك، ونفخت فيه نسمة حياة، فعاش أمامك
\par 6 وأدخلته إلى الفردوس الذي غرسته يمينك قبل أن تظهر الأرض
\par 7 وأوصيته أن يحب طريقك، فتعداه، وعينت للوقت الموت فيه وفي أجياله، التي خرجت منها شعوب وقبائل وقبائل بلا عدد
\par 8 وسار كل شعب وراء إرادته، وصنعوا عجائب أمامك، واحتقروا وصاياك
\par 9 ومرة ​​أخرى، مع مرور الوقت، جلبت الطوفان على سكان العالم، وأهلكتهم
\par 10 وحدث في كل واحد منهم أنه كما كان الموت لآدم، كذلك كان الطوفان لهؤلاء
\par 11 مع ذلك تركت واحدًا منهم، وهو نوح وأهل بيته، الذين خرج منهم جميع الصالحين
\par 12 وحدث أنه عندما بدأ سكان الأرض يتكاثرون، ورزقوا بأولاد كثيرين، وأصبحوا شعبًا عظيمًا، بدأوا مرة أخرى يكونون أكثر شرًا من الأولين
\par 13 ولما كانوا شريرين أمامك، اخترت لنفسك من بينهم رجلاً اسمه إبراهيم
\par 14 الذي أحببته، وله وحده أظهرت مشيئتك.
\par 15 وأبرمت معه عهدًا أبديًا، ووعدته بأنك لن تترك نسله أبدًا.
\par 16 وأعطيته إسحاق، وأعطيت إسحاق أيضًا يعقوب وعيسو. أما يعقوب فاخترته لنفسك ووضعته بجانب عيسو، فصار يعقوب جمعًا كثيرًا
\par 17 وكان لما أخرجت نسله من مصر، أصعدتهم إلى جبل سيناء
\par 18 وبانحناء السماوات، ثبّتتَ الأرض، وحرّكتَ العالم أجمع، وجعلتَ الأعماق ترتجف، وأزعجتَ رجال ذلك العصر
\par 19 ومجَّدكَ عبر أربعة أبواب: النار، والزلزال، والريح، والبرد، لكي تُعطي الشريعة لنسل يعقوب، والاجتهاد لجيل إسرائيل
\par 20 ومع ذلك، فلم تنزع منهم قلبًا شريرًا، لكي تُثمر شريعتك فيهم
\par 21 لأن آدم الأول الذي حمل قلبًا شريرًا تعدى فانهزم، وهكذا يكون كل من ولدوا منه
\par 22 وهكذا أصبح الضعف دائمًا؛ والشريعة (أيضًا) في قلوب الناس بخبث الجذر؛ حتى رحل الخير، وبقي الشر
\par 23 ثم مضت الأزمنة، وانتهت السنون، فأقمت لنفسك عبدًا اسمه داود
\par 24 الذي أمرته أن يبني مدينة لاسمك، وأن يصعد لك فيها بخورًا وقرابين
\par 25 ولما مضى على ذلك سنين عديدة، تركك سكان المدينة،
\par 26 وفي كل شيء فعلوا كما فعل آدم وكل أجياله، لأنهم هم أيضًا كان لهم قلب شرير
\par 27 وهكذا سلمت مدينتك إلى أيدي أعدائك.
\par 28 فهل أعمالهم أفضل من أعمال ساكني بابل حتى يكون لهم السيادة على صهيون؟
\par 29 لأنه عندما وصلت إلى هناك، ورأيت فجورًا لا يُحصى، رأت روحي العديد من الأشرار في هذه السنة الثلاثين، حتى خذلني قلبي
\par 30 لأني رأيت كيف أنك تسمح لهم بالخطيئة، وتحفظ الأشرار، وتهلك شعبك، وتحفظ أعداءك، ولم تلمح بذلك
\par 31 لا أتذكر كيف يمكن ترك هذا الطريق: فهل أهل بابل أفضل من أهل صهيون؟
\par 32 أم يوجد شعب آخر يعرفك غير إسرائيل؟ أو أي جيل آمن بعهودك مثل يعقوب؟
\par 33 ومع ذلك، لم يظهر أجرهم، ولم يُثمر عملهم: لأني تجولت هنا وهناك بين الوثنيين، ورأيتهم يفيضون ثراءً، ولا يفكرون في وصاياك
\par 34 زن الآن شرورنا في الميزان، وشرور الذين يسكنون العالم أيضًا، وهكذا لن يوجد اسمك في أي مكان إلا في إسرائيل
\par 35 أو متى لم يخطئ أمامك سكان الأرض؟ أو أي شعبٍ حفظ وصاياك؟
\par 36 ستجد أن إسرائيل بالاسم قد حفظت وصاياك، أما الوثنيون فليسوا كذلك

\chapter{4}

\par 1 والملاك الذي أُرسل إليّ، واسمه أورييل، أعطاني جوابًا،
\par 2 وقال: لقد ذهب قلبك بعيدًا في هذا العالم، وتظن أنك تفهم طريق العلي؟
\par 3 فقلت: نعم يا سيدي. فأجابني وقال: أُرسلت لأريك ثلاث طرق، وأعرض أمامك ثلاث مثل:
\par 4 إن استطعت أن تُعلن لي واحدًا منها، فسأريك أيضًا الطريق الذي ترغب في رؤيته، وسأريك من أين يأتي القلب الشرير
\par 5 فقلت: أخبر يا سيدي. فقال لي: اذهب، زن لي وزن النار، أو قِس لي نفخة الريح، أو عد إليّ في اليوم الماضي
\par 6 فأجبت وقلت: أي إنسان يستطيع أن يفعل ذلك حتى تطلب مني مثل هذه الأمور؟
\par 7 فقال لي: إذا سألتك عن عدد المساكن في وسط البحر، أو كم ينبوعًا في أول الغمر، أو كم ينبوعًا فوق السماء، أو ما هي مخارج الجنة:
\par 8 ربما تقول لي: لم أنزل إلى العمق قط، ولا إلى الجحيم بعد، ولم أصعد إلى السماء قط
\par 9 ومع ذلك، فقد سألتك الآن فقط عن النار والريح، وعن اليوم الذي مررت به، وعن الأشياء التي لا يمكنك الانفصال عنها، ومع ذلك لا يمكنك أن تعطيني إجابة عنها
\par 10 ثم قال لي: أما أنت فلا تعرف أشياءك الخاصة، وما نشأ معك
\par 11 كيف يُمكن لوعائك إذن أن يفهم طريق العلي، وكيف يُمكن للعالم الآن، وقد أصبح فاسدًا ظاهريًا، أن يفهم الفساد الواضح في نظري؟
\par 12 ثم قلت له: من الأفضل ألا نكون موجودين على الإطلاق، من أن نعيش بعد في الشر، ونعاني، ولا نعرف لماذا
\par 13 أجابني وقال: دخلتُ غابةً إلى سهل، والأشجارُ تشاورت،
\par 14 وقال: هلموا نذهب ونحارب البحر حتى يبتعد عنا ونصنع لأنفسنا المزيد من الغابات
\par 15 وتشاورت أنهار البحر أيضًا، وقالت: هلموا نصعد ونخضع غابات السهل، لكي نصنع لأنفسنا هناك أيضًا أرضًا أخرى
\par 16 كانت فكرة الحطب عبثًا، لأن النار جاءت والتهمته
\par 17 وكذلك تبددت فكرة فيضانات البحر، إذ وقفت الرمال وأوقفتها
\par 18 لو كنت تحكم الآن بين هذين الاثنين، فمن تريد أن تبدأ بتبريره؟ أو من تريد أن تدينه؟
\par 19 أجبتُ وقلتُ: حقًا إنها فكرة حمقاء أنهما فكرا بها، لأن الأرض قد أُعطيت للغابة، وللبحر أيضًا مكانه ليحمل فيضانه
\par 20 فأجابني وقال: لقد حكمت حكمًا صحيحًا، ولكن لماذا لا تحكم على نفسك أيضًا؟
\par 21 لأنه كما أن الأرض مُنحت للغابة، والبحر لغزارته، كذلك فإن سكان الأرض لا يفهمون إلا ما على الأرض، ومن يسكن فوق السماوات لا يفهم إلا ما فوق علو السماوات
\par 22 فأجبتُ وقلتُ: يا رب، أطلب إليكَ أن تُفهمني
\par 23 لأنه لم يكن من قصدي أن أكون فضوليًا بشأن الأمور السامية، بل بشأن أولئك الذين يمرون بنا يوميًا، أي لماذا تُسلم إسرائيل كعار للأمم، ولماذا يُسلم الشعب الذي أحببته إلى أمم غير مؤمنة، ولماذا تُبطل شريعة أجدادنا، وتبطل العهود المكتوبة،
\par 24 ونخرج من العالم كالجراد، وحياتنا دهشة وخوف، ولسنا مستحقين للرحمة.
\par 25 فماذا يفعل باسمه الذي دُعينا به؟ عن هذه الأمور سألت
\par 26 فأجابني وقال: كلما بحثت أكثر، كلما تعجبت أكثر، لأن العالم سريع الزوال،
\par 27 ولا يستطيعون إدراك الأمور التي وُعد بها الصالحون في المستقبل، لأن هذا العالم مليء بالإثم والضعف
\par 28 وأما ما تسألني عنه فسأخبرك به، لأن الشر قد زُرع، ولكن هلاكه لم يأتِ بعد
\par 29 فإن لم ينقلب المزروع، ولم يزل المكان الذي زُرع فيه الشر، فلا يمكن أن يأتي المزروع بالخير
\par 30 لأن بذرة الشر قد زُرعت في قلب آدم منذ البدء، فكم من الإثم جلبت إلى هذا الوقت؟ وكم ستُنتج حتى يأتي وقت الدرس؟
\par 31 تأمل الآن في نفسك، كم أنبتت حبة البذرة الشريرة من ثمار الشر
\par 32 وعندما تُقطع السنابل التي لا عدد لها، فكم من أرض تملأ؟
\par 33 ثم أجبت وقلت: كيف ومتى تحدث هذه الأمور؟ لماذا تكون سنواتنا قليلة وشريرة؟
\par 34 فأجابني قائلاً: لا تتعجل فوق العلي، لأن تعجلك باطلٌ أن تكون فوقه، لأنك قد تجاوزت كثيرًا
\par 35 ألم تسأل نفوس الأبرار أيضًا عن هذه الأمور في غرفها قائلة: إلى متى أنتظر على هذا المنوال؟ متى يأتي ثمر أرض مكافأتنا؟
\par 36 وأجابهم أورييل رئيس الملائكة على هذه الأمور، وقال: حتى عندما يمتلئ فيكم عدد البذور، لأنه وزن العالم بالميزان
\par 37 بالمكيال قدّر الأزمنة، وبالعدد أحصى الأزمنة، ولا يحرّكها ولا يحرّكها حتى يتمّ المكيال المذكور
\par 38 ثم أجبت وقلت: يا رب، أيها الحاكم الحكيم، حتى نحن جميعًا مليئون بالفجور
\par 39 ولعله من أجلنا لا تُملأ بيوت الصديقين بسبب خطايا الساكنين على الأرض
\par 40 فأجابني وقال: اذهب إلى امرأة حبلى، واسألها متى أتمت تسعة أشهر، هل يستطيع رحمها أن يحفظ الولادة في بطنها بعد
\par 41 فقلتُ: لا يا رب، إنها لا تستطيع. فقال لي: في القبر غرف الأرواح كرحم امرأة
\par 42 فكما أن المرأة التي تلد تسرع في الهروب من ضرورة المخاض، كذلك تُسرع هذه الأماكن في تسليم الأشياء الموكلة إليها
\par 43 من البداية، انظر، ما ترغب في رؤيته، سيُظهر لك
\par 44 فأجبت وقلت: إن كنت قد وجدت نعمة في عينيك، وإن أمكن، وإن كنت مستحقا لذلك،
\par 45 أرني إذًا ما إذا كان هناك المزيد مما مضى، أو ما مضى أكثر مما سيأتي
\par 46 أعرف ما مضى، لكني لا أعرف ما هو آتٍ.
\par 47 فقال لي قم عن الجانب الأيمن فأفسر لك المثل.
\par 48 فوقفتُ ونظرتُ، وإذا بتنورٍ مُتقدٍ يمرُّ أمامي. ولما ذهب اللهيب، نظرتُ، وإذا الدخانُ قد توقف
\par 49 بعد ذلك، مرت أمامي سحابة مائيّة، وأرسلت مطرًا غزيرًا مع عاصفة؛ وعندما انقضى المطر العاصف، بقيت القطرات ساكنة
\par 50 ثم قال لي: فكّر في نفسك؛ فكما أن المطر أكثر من القطرات، وكما أن النار أكبر من الدخان، لكن القطرات والدخان تبقى، فكذلك الكمية التي مضت زادت
\par 51 ثم صليت وقلت: أتظن أني قد أعيش إلى ذلك الوقت؟ أو ماذا سيحدث في تلك الأيام؟
\par 52 أجابني وقال: أما العلامات التي تسألني عنها، فسأخبرك بجزء منها، وأما ما يتعلق بحياتك، فلم أُرسل لأريك إياها، لأني لا أعرفها

\chapter{5}

\par 1 ولكن مع مجيء العلامات، هوذا أيام ستأتي، فيها يُؤخذ سكان الأرض بأعداد كبيرة، ويُخفى طريق الحق، وتُصبح الأرض قاحلة من الإيمان
\par 2 بل سيزداد الإثم فوق ما تراه الآن، أو ما سمعته منذ زمن طويل
\par 3 والأرض التي تراها الآن متأصلة، ستراها خربة فجأة
\par 4 ولكن إن منحك العلي الحياة، فسترى بعد البوق الثالث أن الشمس ستشرق فجأة مرة أخرى في الليل، والقمر ثلاث مرات في النهار:
\par 5 ويسيل الدم من الخشب، ويعطي الحجر صوته، ويضطرب الشعب
\par 6 وهو الذي سيحكم، الذي لا ينتظرونه من الساكنين على الأرض، والطيور ستطير معًا
\par 7 ويطرح بحر سدوم سمكًا، ويُحدث ضجيجًا في الليل لم يعرفه كثيرون، لكنهم سيسمعون جميعًا صوته
\par 8 ستكون هناك أيضًا حالة من الفوضى في أماكن كثيرة، وستُطلق النار مرارًا وتكرارًا، وستغير الوحوش البرية أماكنها، وستلد النساء الحائضات وحوشًا:
\par 9 وستُوجد مياه مالحة في المياه العذبة، وسيدمر كل الأصدقاء بعضهم بعضًا؛ حينئذٍ سيختبئ العقل، وينسحب الفهم إلى حجرته السرية،
\par 10 ويُطلَب من كثيرين فلا يُوجَدون، فحينئذٍ يكثر الإثم وعدم الانضباط على الأرض
\par 11 تسأل أرضٌ أرضًا أخرى وتقول: هل مرّ بكِ البرُّ الذي يجعل الإنسان بارًا؟ فيقول: لا
\par 12 وفي الوقت نفسه يأمل الرجال ولكن لا ينالون شيئا. وسوف يتعبون ولكن طرقهم لن تنجح.
\par 13 لقد أذنتُ لكَ أن أُريكَ مثل هذه العلامات؛ وإذا صليت مرة أخرى، وبكيت كما الآن، وصمتَ ولو لأيام، فستسمع أمورًا أعظم
\par 14 ثم استيقظت، وسرى خوف شديد في جميع أنحاء جسدي، واضطرب عقلي حتى أغمي عليه
\par 15 فأمسكني الملاك الذي جاء ليتحدث معي، وعزاني، وأقامني على قدمي
\par 16 وفي الليلة الثانية، جاء إليّ شألتيئيل رئيس الشعب قائلًا: أين كنت؟ ولماذا وجهك ثقيل؟
\par 17 أما تعلم أن إسرائيل مُستَعْتَمَدٌ إليك في أرض سبيهم؟
\par 18 قم إذًا، وكل خبزًا، ولا تتركنا، كما يترك الراعي قطيعه في أيدي ذئاب ضارية
\par 19 فقلت له: اذهب عني ولا تقترب مني. فسمع ما قلته وذهب من عندي
\par 20 فصمتُ سبعة أيامٍ، حزينًا وباكيًا، كما أمرني الملاك أورييل
\par 21 وبعد سبعة أيام، كانت أفكار قلبي حزينة جدًا عليّ مرة أخرى،
\par 22 واستعادت روحي روح الفهم، وبدأت أتحدث مع العلي مرة أخرى،
\par 23 وقال: أيها السيد المتسلط، من كل غابة الأرض، ومن جميع أشجارها، اخترت لنفسك كرمة واحدة فقط
\par 24 ومن بين جميع أراضي العالم كله، اخترت لنفسك حفرة واحدة، ومن بين جميع أزهارها زنبقة واحدة
\par 25 ومن كل أعماق البحر ملأتَ لك نهرًا واحدًا، ومن كل المدن المبنية قدست صهيون لنفسك
\par 26 ومن جميع الطيور التي خلقت سميت لنفسك حمامة واحدة، ومن جميع البهائم التي خلقت جعلت لك خروفًا واحدًا
\par 27 ومن بين جميع جموع الشعوب، حصلت على شعب واحد. ولهذا الشعب الذي أحببته، أعطيت شريعة مقبولة لدى الجميع
\par 28 والآن يا رب، لماذا أسلمت هذا الشعب الواحد إلى كثيرين؟ وعلى أصل واحد هيأت آخرين، ولماذا شتتت شعبك الوحيد بين كثيرين؟
\par 29 والذين خالفوا مواعيدك ولم يؤمنوا بعهودك داسوها
\par 30 إذا كنت تكره شعبك إلى هذا الحد، فهل يجب عليك أن تعاقبهم بيديك
\par 31 بعد أن نطقت بهذه الكلمات، أُرسل إليّ الملاك الذي جاء إليّ في الليلة السابقة،
\par 32 وقال لي: اسمع لي فأُعلِّمك. أصغِ إلى ما أقوله فأُخبرك أكثر
\par 33 فقلت: تكلم يا سيدي. فقال لي: أنت مضطرب النفس من أجل إسرائيل. هل تحب هذا الشعب أكثر ممن خلقه؟
\par 34 فقلت لا يا رب، ولكني تكلمت بحزن شديد، لأن كليتي تؤلمانني كل ساعة، بينما أتعب في فهم طريق العلي، والبحث عن جزء من دينونته.
\par 35 فقال لي: لا تستطيع. فقلت: لماذا يا رب؟ وأين ولدت إذًا؟ ولماذا لم يكن بطن أمي قبري حتى لا أرى تعب يعقوب، وتعب بني إسرائيل المرهق؟
\par 36 وقال لي: أحصِ لي الأشياء التي لم تأتِ بعد، واجمع لي الزغل المبعثر، واجعل لي الزهور الذابلة خضراء من جديد،
\par 37 افتح لي الأماكن المغلقة، وأخرج لي الرياح المغلقة فيها، أرني صورة صوت، فحينئذٍ أخبرك بالأمر الذي تتعب في معرفته
\par 38 فقلت: أيها السيد الحاكم، من يعلم هذه الأمور إلا من ليس له مسكن مع الناس؟
\par 39 أما أنا، فأنا جاهل، فكيف لي أن أتحدث عن هذه الأمور التي تسألني عنها؟
\par 40 ثم قال لي: كما أنك لا تستطيع أن تفعل شيئًا من هذه الأشياء التي تحدثت عنها، كذلك لا تستطيع أن تعرف حكمي، أو في النهاية المحبة التي وعدت بها شعبي
\par 41 فقلت: هوذا يا رب، أنت قريب من الذين سيبقون إلى النهاية. فماذا سيفعل الذين كانوا قبلي، أو نحن الموجودون الآن، أو الذين يأتون بعدنا؟
\par 42 فقال لي: سأشبّه حكمي بخاتم، كما أنه لا تباطؤ في الأخير، كذلك لا سرعة في الأول
\par 43 فأجبت وقلت: ألا تستطيع أن تصنع ما خُلِقَ، وهو كائن الآن، وما هو آتٍ، في الحال، حتى تُظهر دينونتك في أسرع وقت؟
\par 44 فأجابني وقال: لا يجوز للمخلوق أن يسرع فوق الخالق، ولا يجوز للعالم أن يستوعب من سيُخلق فيه دفعة واحدة
\par 45 فقلتُ: كما قلتَ لعبدك، إنك أنت الذي تُحيي الجميع، قد وهبتَ الحياةَ دفعةً واحدةً للمخلوق الذي خلقته، فحملته، هكذا قد تلد الآن أيضًا الحاضرين الآن دفعةً واحدة
\par 46 فقال لي: اسأل رحم امرأة وقل لها: إن كنتِ تلدين أولادًا فلماذا لا تلدين معًا، بل واحدًا تلو الآخر؟ فاطلب منها أن تلد عشرة أطفال دفعة واحدة
\par 47 فقلتُ: لا تستطيع، لكن يجب أن تفعل ذلك بمرور الوقت.
\par 48 ثم قال لي: هكذا أعطيت رحم الأرض للمزروعين فيها في أوقاتهم.
\par 49 فكما أن الطفل الصغير لا يستطيع أن يُنتج الأشياء التي تخص كبار السن، هكذا رتبتُ العالم الذي خلقته
\par 50 فسألتُ وقلتُ: بما أنك قد أرشدتني الآن، فسأتحدث أمامك، لأن أمنا التي أخبرتني أنها صغيرة، قد اقتربت من السن
\par 51 فأجابني وقال: اسأل امرأة تلد فتخبرك.
\par 52 قل لها: لماذا أصبح الذين ولدتهم الآن مثل الذين كانوا من قبل، ولكن أقل قامة؟
\par 53 فتجيبك: إن الذين يولدون في عز الشباب هم على نوع واحد، والذين يولدون في وقت الشيخوخة، عندما يفشل الرحم، هم على نوع آخر
\par 54 فاعتبروا أنتم أيضًا أنكم أقصر قامة من الذين كانوا قبلكم
\par 55 وكذلك الذين يأتون بعدكم أقل منكم، كالمخلوقات التي بدأت الآن تشيخ، وتخطت قوة الشباب
\par 56 ثم قلت يا رب، أتوسل إليك، إن كنت قد وجدت نعمة في عينيك، فأرِ عبدك الذي به تفتقد خليقتك

\chapter{6}

\par 1 وقال لي: في البدء، حين خُلقت الأرض، قبل أن تقف حدود العالم، أو هبت الرياح،
\par 2 قبل أن يرعد ويبرق، أو تُوضع أسس الجنة،
\par 3 قبل أن تُرى الزهور الجميلة، أو تُثبت القوى المتحركة، قبل أن تتجمع الحشود التي لا تُحصى من الملائكة،
\par 4 أو حتى ارتفعت مرتفعات الهواء، قبل أن تُسمى مقاييس السماء، أو حتى اشتعلت مداخن صهيون،
\par 5 وقبل أن يتم البحث عن السنوات الحالية، أو حتى اختراعات أولئك الذين تحولت خطيتهم الآن، قبل أن يُختم أولئك الذين جمعوا الإيمان كنزًا:
\par 6 ثم فكرت في هذه الأشياء، وكلها كانت من خلالي وحدي، وليس من خلال أي شخص آخر: من خلالي أيضًا ستنتهي، وليس من خلال أي شخص آخر
\par 7 فأجبتُ وقلتُ: ما هو انقسام الأزمنة؟ أو متى تكون نهاية الأول وبداية الذي يليه؟
\par 8 وقال لي: من إبراهيم إلى إسحاق، حين ولد منه يعقوب وعيسو، أمسكت يد يعقوب أولاً بعقب عيسو
\par 9 لأن عيسو هو نهاية العالم، ويعقوب هو بدايته التي تليها
\par 10 يد الإنسان بين الكعب واليد: سؤال آخر يا عزرا، لا تسأل
\par 11 فأجبتُ حينئذٍ وقلتُ: أيها السيدُ الذي يحكمُ، إن كنتُ قد وجدتُ نعمةً في عينيك،
\par 12 أتوسل إليك، أن تُري خادمك نهاية علاماتك التي أريتني جزءًا منها الليلة الماضية
\par 13 فأجابني وقال لي: قم على رجليك واسمع صوتًا عظيمًا
\par 14 ويكون ذلك كحركة عظيمة، ولكن المكان الذي تقف فيه لن يتزعزع
\par 15 لذلك متى تكلم فلا تخف، لأن الكلمة هي للنهاية، وأساس الأرض مفهوم
\par 16 ولماذا؟ لأن كلام هذه الأشياء يرتعد ويضطرب، لأنه يعلم أن نهاية هذه الأشياء لا بد أن تتغير
\par 17 فحدث حين سمعت ذلك فنهضت على رجليّ وأصغيت وإذا صوت يتكلم وصوته كصوت مياه كثيرة.
\par 18 وقال: هوذا أيام تأتي، وأبدأ بالاقتراب، وأفتقد سكان الأرض،
\par 19 وسيبدأون في استجوابهم، ماذا عساهم أناسًا آذوا ظلمًا بظلمهم، وعندما يتم ضيق صهيون؛
\par 20 وعندما ينتهي العالم الذي سيبدأ بالزوال، سأُظهر هذه العلامات: ستُفتح الكتب أمام السماء، وسيرون كل شيء معًا:
\par 21 ويتكلم أطفال السنة بأصواتهم، وتلد الحوامل أطفالاً سقطاً من ابن ثلاثة أو أربعة أشهر، فيعيشون ويترعرعون
\par 22 وفجأة ستظهر الأماكن المزروعة غير مزروعة، وستُوجد المخازن الممتلئة فارغة فجأة:
\par 23 ويصدر البوق صوتًا، فإذا سمعه كل إنسان خافوا فجأة
\par 24 في ذلك الوقت، سيتقاتل الأصدقاء بعضهم بعضًا كأعداء، وستقف الأرض في خوف مع سكانها، وستتوقف ينابيع الينابيع، وفي غضون ثلاث ساعات لن تتدفق
\par 25 كل من يبقى من كل هذه التي أخبرتك بها ينجو ويرى خلاصي ونهاية عالمك
\par 26 وسيرى ذلك الرجال الذين تم قبولهم، والذين لم يذوقوا الموت منذ ولادتهم: وسيتغير قلب السكان، ويتحول إلى معنى آخر
\par 27 لأن الشر يُخمد، والخداع يُطفأ.
\par 28 وأما الإيمان فسوف يزدهر، والفساد سوف يتغلب عليه، والحقيقة التي كانت لفترة طويلة بلا ثمر سوف تُعلن.
\par 29 ولما تكلم معي، إذا بي أنظر قليلاً فقليلاً إلى من كنت أقف أمامه
\par 30 وقال لي هذه الكلمات: جئت لأريك وقت الليل الآتي
\par 31 إذا صليت أكثر، وصمت سبعة أيام أخرى، فسأخبرك كل يوم بأشياء أعظم مما سمعت
\par 32 لأن صوتك مسموع لدى العلي، لأن القدير قد رأى عدلك، ورأى أيضًا عفتك التي كانت لديك منذ شبابك
\par 33 ولذلك أرسلني لأريك كل هذه الأمور، وأقول لك: تعزَّ ولا تخف
\par 34 ولا تتعجل بالأزمنة الماضية لتفكر في أمور باطل، حتى لا تتعجل بالأزمنة الأخيرة
\par 35 وحدث بعد ذلك أنني بكيت أيضًا، وصمت سبعة أيام كذلك، لكي أُكمل الأسابيع الثلاثة التي قال لي
\par 36 وفي الليلة الثامنة، انزعج قلبي مرة أخرى، وبدأت أتحدث أمام العلي
\par 37 لأن روحي اشتعلت بشدة، ونفسي كانت في ضيق.
\par 38 "وقلت يا رب أنت تكلمت من بدء الخليقة من اليوم الأول وقلت هكذا: لتكن السماء والأرض. وكانت كلمتك عملاً تاماً."
\par 39 ثم كان الروح، وكان الظلام والصمت في كل مكان؛ ولم يكن صوت الإنسان قد تشكل بعد
\par 40 ثم أمرت بإخراج نور جميل من كنوزك، حتى يظهر عملك
\par 41 في اليوم الثاني صنعتَ روحَ الجلد، وأمرتَه أن ينشقَّ، وأن يصنعَ شقًّا بين المياه، بحيث يصعد جزءٌ إلى الأعلى، ويبقى الآخرُ في الأسفل
\par 42 في اليوم الثالث، أمرتَ بجمع المياه في سابع جزء من الأرض: جففتَ ست قطع، واحتفظتَ بها، لكي يخدمك منها ما زرعه الله وزرعه
\par 43 لأنه حالما خرجت كلمتك، تم العمل.
\par 44 فصار في الحال فاكهة كثيرة لا تحصى، ومتع كثيرة ومتنوعة للتذوق، وأزهار ذات لون لا يتغير، وروائح عطرة. وحدث هذا في اليوم الثالث.
\par 45 في اليوم الرابع أمرتَ أن تُشرق الشمس، وأن يُعطي القمر ضوءه، وأن تُرتّب النجوم:
\par 46 وأعطاهم وصية لخدمة الإنسان، وكان من المقرر أن يتم ذلك
\par 47 وفي اليوم الخامس قلت للجزء السابع، حيث تجتمع المياه، أن يُخرج كائنات حية، طيورًا وأسماكًا. فحدث ذلك
\par 48 لأن الماء الصامت عديم الحياة أنتج كائنات حية بأمر الله، لكي يسبح جميع الناس أعمالك العجيبة
\par 49 ثم جعلت مخلوقين حيين، أحدهما سميته حنوك والآخر لوياثان
\par 50 وفصل الواحد عن الآخر، لأن الجزء السابع، أي حيث تجمع الماء، لم يستطع أن يستوعبهما معًا
\par 51 أعطيت أخنوخ جزءًا واحدًا، جف في اليوم الثالث، لكي يسكن في ذلك الجزء الذي فيه ألف جبل
\par 52 وأما ليفياثان فأعطيته السبع، أي الرطب، وحفظته ليؤكل من تشاء ومتى تشاء
\par 53 في اليوم السادس، أمرتَ الأرضَ أن تُخرِج أمامكَ وحوشًا وبهائمَ ودبابات
\par 54 وبعد هؤلاء، آدم أيضًا، الذي جعلته سيدًا على جميع مخلوقاتك. منه أتينا نحن جميعًا، وكذلك الشعب الذي اخترته
\par 55 كل هذا تكلمت به أمامك يا رب، لأنك خلقت العالم من أجلنا
\par 56 أما بالنسبة للشعوب الأخرى، التي تأتي أيضًا من آدم، فقد قلت إنهم ليسوا شيئًا، بل هم كالبصاق، وشبهت كثرتهم بقطرة تسقط من إناء
\par 57 والآن يا رب، انظر، هؤلاء الوثنيون، الذين لطالما اعتبروا لا شيء، قد بدأوا يتسلطون علينا ويفترسوننا
\par 58 وأما نحن شعبك الذي دعوته بكرك ووحيدك ومحبك الشديد فقد أسلمنا إلى أيديهم.
\par 59 إذا كان العالم الآن قد خُلِقَ من أجلنا، فلماذا لا نملك ميراثًا مع العالم؟ إلى متى سيستمر هذا؟

\chapter{7}

\par 1 ولما انتهيتُ من نطق هذه الكلمات، أُرسل إليّ الملاك الذي أُرسل إليّ في الليالي السابقة:
\par 2 وقال لي: قم يا عزرا، واسمع الكلام الذي جئت لأخبرك به
\par 3 فقلت: تكلم يا إلهي. فقال لي: البحر واسعٌ ليكون عميقًا وعظيمًا
\par 4 ولكن ضع في اعتبارك أن المدخل كان ضيقًا، ومثل النهر؛
\par 5 فمن يستطيع إذن أن يذهب إلى البحر لينظر إليه ويحكمه؟ إن لم يكن قد عبر الضيق فكيف يستطيع أن يدخل إلى الواسع؟
\par 6 هناك أيضًا شيء آخر؛ تُبنى مدينة، وتُقام على حقل واسع، وهي مليئة بكل الأشياء الجيدة
\par 7 مدخله ضيق، ومُقام في مكان خطر السقوط، كما لو كانت هناك نار على اليمين، وعلى اليسار مياه عميقة:
\par 8 ومسار واحد فقط بينهما، حتى بين النار والماء، صغير جدًا لدرجة أنه لا يمكن أن يذهب إليه إلا رجل واحد في آن واحد
\par 9 إذا أُعطيت هذه المدينة الآن لرجل ميراثًا، وإذا لم يتجاوز الخطر الموضوع أمامها، فكيف سيحصل على هذا الميراث؟
\par 10 فقلت: هو كذلك يا رب. فقال لي: هكذا أيضًا نصيب إسرائيل
\par 11 لأني من أجلهم خلقت العالم، وعندما تعدى آدم فرائضي، قُدِّر أن الآن قد تم
\par 12 ثم ضاقت مداخل هذا العالم، مليئة بالحزن والمحن: فهي قليلة وشريرة، مليئة بالمخاطر، ومؤلمة للغاية
\par 13 لأن مداخل العالم القديم كانت واسعة ومؤكدة، وجلبت ثمارًا خالدة
\par 14 إذا كان الأحياء لا يتعبون في الدخول في هذه الأمور الضيقة والباطلة، فلن يتمكنوا أبدًا من الحصول على تلك المدخرة لهم
\par 15 فلماذا تقلق الآن، وأنت لست سوى إنسان قابل للفساد؟ ولماذا تتأثر، وأنت لست سوى فانٍ؟
\par 16 لماذا لم تفكر في هذا الأمر الآتي بدلًا من الأمر الحاضر؟
\par 17 فأجبت وقلت: أيها الرب المتسلط، لقد جعلت في ناموسك أن يرث الأبرار هذه الأشياء، ويهلك الأشرار
\par 18 ومع ذلك، فإن الصديق سيعاني من الضيقات، ويرجو خيرًا واسعًا. لأن الذين فعلوا الشر قد عانوا من الضيقات، ومع ذلك لن يروا الخير الواسع
\par 19 فقال لي: ليس قاضٍ فوق الله، ولا من له فهم فوق العلي
\par 20 لأن كثيرين يهلكون في هذه الحياة، لأنهم يحتقرون شريعة الله الموضوعة أمامهم
\par 21 لأن الله أعطى وصية صارمة للذين أتوا، ماذا ينبغي لهم أن يفعلوا لكي يعيشوا كما أتوا، وماذا ينبغي لهم أن يلتزموا لكي يتجنبوا العقاب.
\par 22 لكنهم لم يطيعوا أمره، بل تكلموا عليه، وتفكروا في أمور باطلة
\par 23 وخدعوا أنفسهم بأعمالهم الشريرة، وقالوا عن العلي إنه ليس موجودًا، ولم يعرفوا طرقه
\par 24 لكنهم استهينوا بشريعةِهِ وأنكروا عهودَهُ، ولم يكونوا أمناءَ في فرائضِهِ، ولم يُنجزوا أعمالَهُ
\par 25 ولذلك يا عزرا، فإن الفراغ هو الأشياء الفارغة، والملء هو الأشياء الممتلئة
\par 26 هوذا الوقت سيأتي، وستتحقق فيه هذه العلامات التي أخبرتك بها، وستظهر العروس، وستُرى وهي تخرج، وقد سُحبت الآن من الأرض
\par 27 وكل من نجا من الشرور المذكورة سيرى عجائبي
\par 28 لأن ابني يسوع سيُظهر مع الذين معه، والذين بقوا سيفرحون بعد أربعمائة عام
\par 29 بعد هذه السنين سيموت ابني المسيح، وكل من له حياة
\par 30 ويعود العالم إلى الصمت القديم سبعة أيام، كما في الأحكام السابقة: حتى لا يبقى إنسان
\par 31 وبعد سبعة أيام، سيقوم العالم الذي لم يستيقظ بعد، ويموت الفاسد
\par 32 وستعيد الأرض من ناموا فيها، وكذلك التراب من يسكنون في صمت، وستُخلّص الأماكن السرية تلك النفوس التي أُودعت فيها
\par 33 وسيظهر العلي على كرسي الدينونة، وسيزول الشقاء، وسيكون لطول الأناة نهاية
\par 34 لكن الدينونة وحدها ستبقى، والحق سيصمد، والإيمان سيزداد قوة
\par 35 وسيأتي العمل، وسيظهر الجزاء، وستكون الأعمال الصالحة مؤثرة، ولن يكون للأعمال السيئة حكم
\par 36 فقلتُ: إن إبراهيم صلّى أولًا لأجل سدوميين، وموسى صلّى أولًا لأجل الآباء الذين أخطأوا في البرية:
\par 37 ويسوع بعده لإسرائيل في زمن عخان:
\par 38 وصموئيل وداود للهلاك وسليمان للقادمين إلى المقدس.
\par 39 وهيلياس للذين نالوا المطر، وللأموات لكي يحيا
\par 40 وحزقيا للشعب في زمن سنحاريب، وكثيرون لكثيرين
\par 41 هكذا الآن، إذ قد كبر الفساد، وكثر الشر، وصلى الأبرار من أجل الأشرار، فلماذا لا يكون الأمر كذلك الآن أيضًا؟
\par 42 أجابني وقال: هذه الحياة الدنيا ليست هي النهاية حيث يبقى مجد كثير؛ لذلك صلوا من أجل الضعفاء
\par 43 لكن يوم القيامة سيكون نهاية هذا الزمان، وبداية الخلود الآتي، حيث يكون الفساد قد مضى،
\par 44 انتهى الإسراف، وانقطع الخيانة، ونما البر، وانبثقت الحقيقة
\par 45 حينئذ لا يستطيع أحد أن ينقذ الهالك، ولا أن يظلم المنتصر.
\par 46 فأجبتُ حينئذٍ وقلتُ: هذا هو قولي الأول والأخير، أنه كان من الأفضل ألا أُعطي الأرض لآدم، وإلا، عندما أُعطيت له، كنتُ أمنعه عن الخطيئة
\par 47 ما الفائدة التي تعود على البشر الآن في هذا الزمان أن يعيشوا في حزن، وبعد الموت ينتظرون العقاب؟
\par 48 يا آدم، ماذا فعلت؟ فمع أنك أنت الذي أخطأت، إلا أنك لست وحدك الساقط، بل نحن جميعًا الذين نأتي منك
\par 49 ما الفائدة لنا إذا وُعدنا بوقت خالد، بينما نحن قد عملنا الأعمال التي تجلب الموت؟
\par 50 وأن هناك وعدًا لنا برجاء أبدي، بينما نحن الأكثر شرًا نُصبح باطلين؟
\par 51 وأن هناك مساكن معدة لنا للصحة والسلامة، بينما عشنا في الشر؟
\par 52 وأن مجد العلي محفوظ للدفاع عن أولئك الذين عاشوا حياة حذرة، بينما سلكنا نحن في أكثر الطرق شرًا على الإطلاق؟
\par 53 وأن يُظهر لنا جنة، ثمرها يدوم إلى الأبد، فيها الأمن والدواء، بما أننا لن ندخلها؟
\par 54 (لأننا سلكنا في أماكن غير سارة.)
\par 55 وأن وجوه الذين امتنعوا عن الطعام تشرق فوق النجوم، بينما وجوهنا تكون أشد سواداً من الظلام؟
\par 56 لأنه بينما كنا نعيش ونرتكب الإثم، لم نكن نفكر في أننا سنبدأ في المعاناة بسببه بعد الموت
\par 57 فأجابني وقال: هذه هي حالة المعركة التي سيخوضها الإنسان المولود على الأرض؛
\par 58 أنه إذا هُزم، فسوف يعاني كما قلت: ولكن إذا حصل على النصر، فسوف يتلقى ما أقوله
\par 59 لأن هذه هي الحياة التي كلم عنها موسى الشعب وهو حي، قائلاً: اختر لنفسك حياة تحيا بها
\par 60 ومع ذلك لم يؤمنوا به، ولا بالأنبياء من بعده، ولا بي أنا الذي كلمتهم،
\par 61 لكي لا يكون هناك ثقل في هلاكهم، كما يكون فرح على الذين اقتنعوا بالخلاص
\par 62 فأجبتُ وقلتُ: أعلمُ يا ربُّ أن العليَّ يُدعى رحيمًا، لأنه يرحمُ الذين لم يأتوا بعدُ إلى العالم،
\par 63 وعلى الذين يلجأون إلى شريعته أيضًا؛
\par 64 وأنه صبور ويتحمل المخطئين كخليقته.
\par 65 وأنه كريم، لأنه مستعد للعطاء حيثما يحتاج الأمر؛
\par 66 وإنه ذو رحمة عظيمة، فإنه يضاعف رحمته أكثر فأكثر للحاضرين، وللماضيين، وللقادمين أيضاً.
\par 67 لأنه إن لم يُكثِّر رحمته، فلن يستمر العالم مع الذين يرثونه
\par 68 "وهو يغفر، لأنه لو لم يفعل ذلك من أجل صلاحه لكي يرتاح من ارتكبوا الآثام، لما بقي عشرة آلاف من البشر أحياء."
\par 69 وهو قاضٍ، إن لم يغفر للذين شُفوا بكلامه، ويطفئ كثرة الخصومة،
\par 70 ربما يتبقى عدد قليل جدًا في عدد لا يحصى

\chapter{8}

\par 1 فأجابني قائلاً: إن العلي خلق هذا العالم لكثيرين، أما العالم الآتي فللقليلين
\par 2 سأخبرك بمثل يا عزرا؛ كما لو أنك تسأل الأرض، فتقول لك إنها تعطي الكثير من القالب الذي تُصنع منه الأواني الفخارية، لكنها لا تعطي سوى القليل من الغبار الذي يُستخرج منه الذهب: هكذا هو مجرى هذا العالم الحاضر
\par 3 سيُخلَق الكثيرون، لكن قليلين سيخلصون.
\par 4 فأجبت وقلت: يا نفسي، ابتلعي الفهم والتهمي الحكمة.
\par 5 لأنك وافقت على الإصغاء، وأنت على استعداد للتنبؤ: لأنه لم يعد لديك متسع إلا للعيش فقط
\par 6 يا رب، إن لم تدع عبدك يصلي أمامك، وتعطينا بذرة لقلوبنا، وثقافة لفهمنا، حتى يأتي منها ثمر؛ فكيف يعيش كل إنسان فاسد، وهو يشغل مكان الإنسان؟
\par 7 لأنك أنت وحدك، ونحن جميعًا صنعة يديك، كما قلت
\par 8 لأنه عندما يُصنع الجسم الآن في رحم الأم، وتُعطيه أعضاءً، يُحفظ مخلوقك في النار والماء، وتسعة أشهر يتحمل صنعك مخلوقك الذي خُلِق فيها
\par 9 ولكن ما يحفظ ويُحفظ سيُحفظ كلاهما. وعندما يحين الوقت، تُخرج الرحم المحفوظة الأشياء التي نمت فيها
\par 10 لأنك أمرت من أعضاء الجسد، أي من الثديين، أن يُعطى اللبن، الذي هو ثمرة الثديين،
\par 11 حتى يُغذى الشيء المصنوع لفترة من الوقت، حتى تُخضعه لرحمتك
\par 12 ربيته ببرك، وربيته في شريعتك، وأصلحته بحكمك
\par 13 وتُميته كخليقتك، وتُحييه كعملك
\par 14 إذا كنت تريد أن تدمر ما تم صنعه بجهد كبير، فمن السهل أن تُأمر بأمرك، حتى يتم الحفاظ على الشيء الذي تم صنعه
\par 15 الآن، يا رب، سأتحدث؛ فيما يتعلق بالإنسان بشكل عام، أنت أعلم بذلك؛ ولكن فيما يتعلق بشعبك، الذي أنا آسف من أجله؛
\par 16 ومن أجل ميراثك الذي أنوح عليه، ومن أجل إسرائيل الذي أثقل عليه، ومن أجل يعقوب الذي أحزن من أجله،
\par 17 لذلك سأبدأ بالصلاة أمامك من أجلي ومن أجلهم، لأني أرى سقوطنا نحن سكان الأرض
\par 18 ولكني سمعت سرعة القاضي الذي سيأتي.
\par 19 فاسمع صوتي وافهم كلامي، فأتكلم أمامك. هذه بداية كلام عزرا قبل صعوده، وقلت:
\par 20 يا رب، أنت الساكن في الأبدية، الذي ينظر من فوق إلى ما في السماء وفي الهواء؛
\par 21 الذي عرشه لا يُقدر بثمن، ومجده لا يُدرك، وأمامه تقف جحافل الملائكة مرتعدة،
\par 22 الذي خدمته خبيرة بالريح والنار، وكلامه حق، وأقواله ثابتة، ووصيته قوية، وأمره مخيف،
\par 23 الذي نظرته تُجفف الأعماق، وسخطه يُذيب الجبال؛ وهو ما تشهد به الحقيقة:
\par 24 اسمع دعاء عبدك، وأنصت إلى دعاء خليقتك
\par 25 سأتحدث ما دمت حيًا، وما دام لدي فهم فسأجيب
\par 26 لا تنظر إلى خطايا شعبك، بل إلى الذين يخدمونك بالحق
\par 27 لا تلتفت إلى اختراعات الأمم الشريرة، بل إلى رغبة الذين يحفظون شهاداتك في الضيقات
\par 28 لا تفكر في أولئك الذين ساروا أمامك متصنعين، بل تذكر الذين عرفوا خوفك حسب إرادتك
\par 29 لا تكن مشيئتك أن تهلك من عاشوا كالوحوش، بل أن تنظر إلى من علموا شريعتك بوضوح
\par 30 لا تغضب على من يُعتبرون أسوأ من الوحوش؛ بل أحبّ من يضعون ثقتهم دائمًا في برك ومجدك
\par 31 لأننا نحن وآباؤنا نعاني من مثل هذه الأمراض، ولكن بسببنا نحن الخطاة ستُدعى رحماء
\par 32 لأنه إن أردت أن ترحمنا، فسوف تُدعى رحيمًا بنا، أي نحن الذين ليس لنا أعمال بر
\par 33 لأن الأبرار الذين لديهم أعمال صالحة كثيرة مدخرة لديك، سينالون مكافأة على أعمالهم الخاصة
\par 34 فما هو الإنسان حتى تغضب عليه؟ أو ما هو الجيل الفاسد حتى تكون مريرًا تجاهه؟
\par 35 لأنه في الحقيقة ليس بين المولودين إنسان إلا وقد فعل الشر، وليس بين المؤمنين أحد لم يفعل الخطأ
\par 36 لأنه في هذا يا رب، سيُعلن برك وصلاحك، إذا كنت رحيمًا بمن ليس لديهم ثقة الأعمال الصالحة
\par 37 فأجابني وقال: لقد تكلمت بشكل صحيح، وسيحدث حسب أقوالك
\par 38 لأني لا أفكر في مصير الذين أخطأوا قبل الموت، قبل الدينونة، قبل الهلاك:
\par 39 لكنني سأفرح بتصرفات الصديقين، وسأتذكر أيضًا رحلتهم، والخلاص، والمكافأة التي سيحصلون عليها
\par 40 كما تكلمت الآن، فسيحدث ذلك.
\par 41 فكما أن الفلاح يزرع بذارا كثيرة في الأرض ويغرس أشجارا كثيرة، ولكن المزروع جيدا في حينه لا ينمو، وكل المزروع لا يتأصل، هكذا أيضا الذين يزرعون في العالم، ليس جميعهم يخلصون.
\par 42 أجبتُ حينها وقلتُ: إن كنتُ قد وجدتُ نعمةً، فدعني أتكلم.
\par 43 كما أن زرع الفلاح يفسد إذا لم ينبت ولم ينل المطر في حينه، أو إذا كثر المطر فأفسده.
\par 44 هكذا يهلك أيضًا الإنسان الذي خُلِقَ بيديك، والذي يُدعى صورتك، لأنك تشبهه، الذي من أجله خلقت كل الأشياء، وشبهته بزرع الفلاح
\par 45 لا تغضب علينا، بل ارحم شعبك، وارحم ميراثك، لأنك رحيم بخليقتك
\par 46 ثم أجابني وقال: الحاضر للحاضر، والآتي للقادم
\par 47 لأنكَ أبعد ما يكون عن أن تُحبَّ خليقتي أكثر مني، لكنني كثيرًا ما اقتربتُ إليكَ وإليها، ولكن لم أقترب أبدًا من الأشرار
\par 48 في هذا أيضًا أنت عجيب أمام العلي:
\par 49 لأنك تواضعت كما يليق بك، ولم تعتبر نفسك أهلاً لأن تُمَجَّد كثيراً بين الأبرار.
\par 50 لأنه سيُصاب أولئك الذين سيسكنون العالم في الزمان الأخير بمصائب عظيمة، لأنهم سلكوا بكبرياء عظيم
\par 51 لكن افهم بنفسك، واطلب المجد لمن هم مثلك
\par 52 لأنه قد فُتح لكم الفردوس، وغُرست شجرة الحياة، وأُعدَّ الزمان الآتي، وأُعِدَّ الوفرة، وبُنيت مدينة، وسُمح بالراحة، نعم، صلاح وحكمة كاملين
\par 53 لقد سُدّ عنك جذر الشر، وأُخفي عنك الضعف والعثة، وهرب الفساد إلى الجحيم ليُنسى:
\par 54 تمر الأحزان، وفي النهاية يُكشف كنز الخلود
\par 55 ولذلك لا تسأل بعد الآن عن كثرة الذين يهلكون
\par 56 لأنهم عندما أخذوا حريتهم، احتقروا العلي، واستهزأوا بشريعته، وتركوا طرقه
\par 57 علاوة على ذلك، فقد داسوا صالحه،
\par 58 وقالوا في قلوبهم أنه ليس إله، وأنه إذا علموا يجب أن يموتوا.
\par 59 لأنه كما أن الأشياء المذكورة آنفًا ستقبلكم، فكذلك العطش والألم مُهيأ لها: لأنه لم تكن مشيئته أن يُفنى البشر
\par 60 لكن المخلوقين دنسوا اسم خالقهم، ولم يشكروا الذي أعد لهم الحياة
\par 61 ولذلك فإن حكمي الآن بين يدي.
\par 62 لم أُرِ هذه الأمورَ لجميعِ الناس، بل لكَ ولقليلٍ مثلكَ. فأجبتُ وقلتُ:
\par 63 هوذا يا رب الآن قد أريتني كثرة العجائب التي ستبتدئ تصنعها في الأزمنة الأخيرة وأما متى فلم تريني.

\chapter{9}

\par 1 فأجابني وقال: قس الوقت بدقة في حد ذاته، وعندما ترى جزءًا من العلامات التي أخبرتك بها سابقًا،
\par 2 حينئذٍ ستفهم أن هذا هو الوقت نفسه الذي سيبدأ فيه العلي بزيارة العالم الذي خلقه
\par 3 لذلك، عندما تُرى زلازل واضطرابات الناس في العالم:
\par 4 حينئذٍ ستفهم جيدًا أن العلي تكلم عن تلك الأمور منذ الأيام التي كانت قبلك، منذ البدء
\par 5 فكما أن لكل ما هو مصنوع في العالم بداية ونهاية، والنهاية ظاهرة:
\par 6 وهكذا فإن أزمنة العلي لها بدايات واضحة في العجائب والأعمال الجبارة، ونهايات في الآثار والعلامات
\par 7 وكل من يخلص ويستطيع أن ينجو بأعماله وبالإيمان الذي آمنتم به،
\par 8 سأُحفظ من المخاطر المذكورة، وسأرى خلاصي في أرضي وداخل حدودي: لأني قدّستهم لي منذ البداية
\par 9 حينئذٍ يكون في حالة يرثى لها أولئك الذين أساءوا استخدام طرقي الآن، والذين رفضوهم بازدراء سيسكنون في العذاب
\par 10 لأن الذين نالوا نعمة في حياتهم ولم يعرفوني
\par 11 والذين كرهوا شريعتي، بينما كانت لهم الحرية، وعندما كان لا يزال أمامهم مجال للتوبة، لم يفهموها، بل احتقروها؛
\par 12 يجب أن يعرفه الشخص نفسه بعد الموت بسبب الألم.
\par 13 ولذلك لا تكن فضوليًا بشأن كيفية معاقبة الأشرار، ومتى: ولكن اسأل عن كيفية خلاص الصالحين، ولمن هذا العالم، ولمن خُلق العالم.
\par 14 ثم أجبت وقلت،
\par 15 لقد قلتُ سابقًا، وأتحدث الآن، وسأتحدث أيضًا فيما بعد، أن من يهلكون سيكونون أكثر بكثير من الذين يخلصون
\par 16 كما أن الموجة أعظم من القطرة.
\par 17 فأجابني قائلاً: كما هو الحقل، كذلك البذرة أيضًا؛ وكما هي الأزهار، كذلك الألوان أيضًا؛ وكما هو العامل، كذلك العمل أيضًا؛ وكما هو الفلاح نفسه، كذلك عمله أيضًا؛ لأنه كان زمن العالم
\par 18 والآن، عندما أعددتُ العالم الذي لم يُخلق بعد، حتى يسكن فيه الذين يعيشون الآن، لم يتكلم أحد عليّ
\par 19 لأن الجميع كانوا يطيعون آنذاك، أما الآن فإن عادات أولئك الذين خُلقوا في هذا العالم المصنوع قد فسدت ببذرة أبدية، وبناموس لا يُستقصى، يتخلصون من أنفسهم
\par 20 فتأملت العالم، وإذا به معرض للخطر بسبب المكائد التي دُبِّرت عليه
\par 21 ورأيتُ، ونجيتُ منه كثيرًا، وأبقيتُ لنفسي عنقودًا من العنب، وغرسًا لشعبٍ عظيم
\par 22 فليهلك الجمع الذي ولد باطلا، وليحفظ عنبي وغرستي، لأني بتعب عظيم أكملتها.
\par 23 ومع ذلك، إذا كنت ستفطر سبعة أيام أخرى (ولكن لا تصوم فيها،
\par 24 بل اذهب إلى حقل زهور، حيث لا يُبنى بيت، وكل فقط زهور الحقل؛ لا تذق لحمًا، ولا تشرب خمرًا، بل كل الزهور فقط؛)
\par 25 وصلِّ إلى العلي باستمرار، فحينئذٍ سآتي وأتحدث إليك
\par 26 فذهبتُ إلى الحقل الذي يُدعى أرداث، كما أمرني، وجلستُ هناك بين الزهور، وأكلتُ من عشب الحقل، فأشبعني لحمه
\par 27 بعد سبعة أيام، جلست على العشب، وكان قلبي مضطربًا في داخلي، كما كان من قبل:
\par 28 ففتحت فمي، وبدأت أتحدث أمام العلي، وقلت:
\par 29 يا رب، أنت الذي أظهرت نفسك لنا، لقد أظهرت لآبائنا في البرية، في مكان لا يدوسه أحد، في مكان قاحل، عندما خرجوا من مصر
\par 30 وتكلمت قائلاً: اسمع لي يا إسرائيل، واسمع كلامي يا نسل يعقوب
\par 31 لأني ها أنا أزرع شريعتي فيكم، فتُثمر فيكم، وتُكرَّمون فيها إلى الأبد
\par 32 لكن آباءنا الذين أخذوا الناموس لم يحفظوه ولم يحفظوا أحكامك. ومع أن ثمرة ناموسك لم تهلك، فإنه لم يكن ليهلك هو أيضًا، لأنه كان لك
\par 33 ولكن الذين أخذوه هلكوا، لأنهم لم يحفظوا ما زُرع فيهم
\par 34 وها هي العادة، عندما تتلقى الأرض بذرًا، أو البحر سفينة، أو أي إناء طعام أو شراب، أن يفسد ما زُرع فيه أو أُلقي فيه،
\par 35 وأيضًا ما زُرع أو أُلقي فيه أو أُخذ، يهلك ولا يبقى معنا. أما نحن فلم يحدث هكذا
\par 36 لأننا نحن الذين أخذنا الناموس نهلك بالخطية، وقلبنا أيضًا الذي أخذه
\par 37 مع أن الناموس لا يهلك، بل يبقى قائمًا.
\par 38 "وعندما تكلمت بهذا في قلبي، نظرت إلى الوراء بعيني، وإذا بامرأة على الجانب الأيمن، وإذا هي تبكي وتئن بصوت عظيم، وكانت حزينة جدًا في قلبها، وكانت ثيابها ممزقة، وعلى رأسها رماد.
\par 39 ثم تركتُ أفكاري التي كنتُ فيها، والتفتُ إليها،
\par 40 فقال لها لماذا تبكين ولماذا حزنت في قلبك؟
\par 41 فقالت لي: يا سيدي، دعني وشأني، لأبكي على نفسي وأزيد من حزني، لأني أشعر بضيق شديد في قلبي، وشعور بالانحطاط الشديد
\par 42 فقلت لها: ما بكِ؟ أخبريني.
\par 43 فقالت لي: كنت أمتك عاقراً ولم يكن لي ولد، وكان لي زوج منذ ثلاثين سنة،
\par 44 وفي تلك السنوات الثلاثين، لم أفعل شيئًا آخر، ليلًا ونهارًا، وفي كل ساعة، سوى صلاتي إلى العلي
\par 45 وبعد ثلاثين سنة سمع الله أمتك، ونظر إلى بؤسي، وتأمل في ضيقي، وأعطاني ابناً. ففرحت به جداً، وكذلك زوجي، وجميع جيراني. وأعطينا القدير كرامة عظيمة.
\par 46 وربيته بتعب عظيم.
\par 47 فلما كبر وجاء الوقت الذي يجب أن تكون له زوجة، قمت بإعداد وليمة.

\chapter{10}

\par 1 وحدث أنه لما دخل ابني حجرة عرسهم سقط ومات
\par 2 ثم أطفئنا الأنوار جميعًا، وقام جميع جيراني لتعزيتي، فاسترحت إلى اليوم الثاني ليلًا
\par 3 وحدث، عندما انتهى الجميع من تعزيتي، حتى أتمكن من الهدوء، ثم قمت ليلًا وهربت، ووصلت إلى هنا إلى هذا الحقل، كما ترى
\par 4 وأنا الآن أنوي ألا أعود إلى المدينة، بل أن أبقى هنا، لا أن آكل ولا أشرب، بل أن أظل حزينًا وأصوم حتى أموت
\par 5 ثم تركتُ التأملات التي كنتُ فيها، وتحدثتُ إليها في غضبٍ قائلًا:
\par 6 أيتها المرأة الحمقاء فوق كل شيء، ألا ترين حزننا، وماذا يحدث لنا؟
\par 7 كيف أن أمنا صهيون مملوءة بكل ثقل، ومتواضعة للغاية، وتحزن بشدة؟
\par 8 والآن، بما أننا جميعًا نحزن ونتحسر، لأننا جميعًا في حزن شديد، فهل أنتِ حزينة على ابن واحد؟
\par 9 فاسأل الأرض، وستخبرك أنها هي التي ينبغي أن تحزن على سقوط الكثيرين الذين ينمون عليها
\par 10 لأنه منها خرج الجميع أولًا، ومنها سيأتي الآخرون جميعًا، وها هم يسيرون جميعًا تقريبًا إلى الهلاك، وقد استُأصلَ كثيرٌ منهم تمامًا
\par 11 فمن إذن يستطيع أن يُحزن أكثر منها، على فقدان هذا الجمع العظيم؛ أليس أنت، الذي لا يحزن إلا على واحد؟
\par 12 ولكن إن قلت لي: إن رثائي ليس كرثاء الأرض، لأني فقدت ثمرة بطني التي ولدتها بآلام، وحملتها بأحزان؛
\par 13 لكن الأرض ليست كذلك، لأن الجمع الموجود فيها حسب مسار الأرض قد مضى كما جاء
\par 14 ثم أقول لك: كما ولدت بتعب، هكذا أعطت الأرض أيضًا ثمرها، أي الإنسان، منذ البدء لمن خلقها
\par 15 الآن، احتفظ بحزنك لنفسك، واتحمل بشجاعة ما أصابك
\par 16 لأنه إذا اعترفتِ بقضاء الله العادل، فسوف تتلقين ابنكِ في الوقت المناسب، وستُمدحين بين النساء
\par 17 ثم اذهبي إلى المدينة إلى زوجك.
\par 18 فقالت لي: لا أفعل ذلك. لا أدخل المدينة، بل هنا أموت.
\par 19 لذلك واصلتُ الحديث معها، وقلتُ،
\par 20 لا تفعلوا هذا، بل استفيدوا مني: لأنه كم هي كثيرة شدائد صهيون؟ تعزوا من أجل حزن أورشليم.
\par 21 لأنك ترى أن قدسنا قد خرب، ومذبحنا هدم، وهيكلنا دمر.
\par 22 لقد وُضعت مزاميرنا على الأرض، وأُسكتت ترنيمتنا، وانتهى فرحنا، وأُطفئ نور منارتنا، ودُنس تابوت عهدنا، وتدنست أقداسنا، وكاد الاسم الذي دُعي علينا أن يُدنّس: أُخزي أطفالنا، وأُحرق كهنتنا، وسُبي لاويونا، وتنجست عذارانا، وفُتحت نسائنا؛ وسُبي أبرارنا، وهلك أطفالنا، واستُعبد شبابنا، وضعف رجالنا الأقوياء؛
\par 23 والأعظم من ذلك كله، أن خاتم صهيون قد فقد شرفه الآن؛ لأنه سُلِّم إلى أيدي مبغضينا
\par 24 ولذلك، انفض عنك ثقلك العظيم، واطرح عنك كثرة الأحزان، حتى يرحمك القدير مرة أخرى، ويمنحك العلي الراحة والسكينة من تعبك
\par 25 وبينما كنت أتحدث معها، إذا بوجهها فجأة أشرق بشدة، وأشرق وجهها، حتى خفت منها، وفكرت فيما قد يكون
\par 26 وإذا بها تصرخ صرخة عظيمة مخيفة جدًا، حتى اهتزت الأرض من صوت المرأة
\par 27 فنظرت وإذا المرأة لم تعد تظهر لي، وإذا مدينة مبنية، ومكان عظيم ظاهر من الأساسات. فخفت وصرخت بصوت عظيم وقلت:
\par 28 أين أورييل الملاك الذي جاء إليّ في البداية؟ لأنه جعلني أقع في غيبوبة كثيرة، وتحولت نهايتي إلى فساد، وصلاتي إلى توبيخ
\par 29 وبينما كنت أتكلم بهذه الكلمات، إذا هو قد جاء إليّ ونظر إليّ
\par 30 وها أنا أضطجع كميت، وقد انتُزع مني عقلي، فأخذني بيدي اليمنى وعزاني وأقامني على قدميَّ، وقال لي:
\par 31 ما بك؟ ولماذا أنت مضطرب هكذا؟ ولماذا اضطرب فهمك وأفكار قلبك؟
\par 32 فقلتُ: لأنكَ تركتني، وفعلتُ حسب كلامك، وخرجتُ إلى الحقل، وها أنا قد رأيتُ، وأرى ما لا أستطيع التعبير عنه
\par 33 فقال لي: قم بشجاعة، وسأنصحك.
\par 34 ثم قلت: تكلم يا سيدي في نفسي، ولكن لا تتركني لئلا أموت محبطًا من رجائي.
\par 35 لأني رأيتُ أنني لم أعرف، وسمعتُ أنني لم أعرف.
\par 36 أم أن حواسي مخدوعة أو روحي في حلم؟
\par 37 والآن أتوسل إليك أن تُري خادمك هذه الرؤيا
\par 38 فأجابني حينئذٍ وقال: اسمع لي، وسأخبرك وأخبرك لماذا أنت خائف، لأن العلي سيكشف لك أسرارًا كثيرة
\par 39 قد رأى أن طريقك مستقيم، لأنك تحزن على شعبك دائمًا، وتُكثر النحيب على صهيون
\par 40 هذا إذن هو معنى الرؤيا التي رأيتها مؤخرًا:
\par 41 رأيت امرأةً حزينةً، فبدأت تُعزيها:
\par 42 ولكنك الآن لا ترى شبه المرأة أيضاً، بل ظهرت لك مدينة مبنية.
\par 43 وبما أنها أخبرتك بوفاة ابنها، فهذا هو الحل:
\par 44 هذه المرأة التي رأيتها هي صهيون. وبما أنها قالت لك، حتى تلك التي تراها كمدينة مبنية،
\par 45 وأما أنا فأقول إنها قالت لك إنها كانت عاقرا ثلاثين سنة. تلك هي الثلاثين سنة التي لم يُقدم فيها قربان لها
\par 46 وبعد ثلاثين سنة، بنى سليمان المدينة وقدم قرابين، ثم ولد للعاقر ابنًا
\par 47 وبما أنها أخبرتك أنها أطعمته بتعب، كان ذلك مسكنها في أورشليم
\par 48 ولكن بينما قالت لك: إن ابني لما دخل حجرة زواجه حدث له خلل ومات، كان هذا هو الهلاك الذي جاء إلى أورشليم
\par 49 وها أنت ذا قد رأيتَ شكلها، ولأنها حزنت على ابنها، بدأتَ تُعزيها. وهذه الأمور التي حدثت، ستُكشف لك
\par 50 لأن العلي يرى الآن أنك تحزن بلا تظاهر، وتتألم من كل قلبك لأجلها، لذلك أراك سطوع مجدها، وجمال جمالها
\par 51 ولذلك أمرتك بالبقاء في الحقل حيث لم يُبنَ بيت:
\par 52 لأني كنت أعلم أن العلي سيُريك هذا.
\par 53 لذلك أمرتك أن تذهب إلى الحقل حيث لم يكن هناك أساس لأي بناء.
\par 54 لأنه في المكان الذي يبدأ فيه العلي إظهار مدينته، ​​لا يمكن لأي بناء إنسان أن يقف
\par 55 لذلك لا تخف، ولا يرتعب قلبك، بل ادخل، وانظر إلى جمال وعظمة المبنى، بقدر ما تستطيع عيناك أن تراه
\par 56 وحينئذٍ ستسمع بقدر ما تستطيع أذناك استيعابه.
\par 57 لأنك مبارك فوق كثيرين غيرك، ومدعو من العلي، وهكذا قليلون.
\par 58 لكن غدًا في الليل ستبقى هنا؛
\par 59 وهكذا يُريكَ العليُّ رؤى العُلى، ما سيصنعُهُ العليُّ لساكني الأرضِ في الأيامِ الأخيرة. فنمتُ تلكَ الليلةَ وليلةً أخرى كما أمرني.

\chapter{11}

\par 1 ثم رأيت حلمًا، وإذا نسر صاعد من البحر، وله اثنا عشر جناحًا من ريش، وثلاثة رؤوس
\par 2 ونظرت فإذا هي تبسط جناحيها على كل الأرض، وهبت عليها كل رياح الهواء، فاجتمعت
\par 3 ونظرت، فإذا من ريشها قد نبت ريش آخر معاكس، فأصبح ريشًا صغيرًا وصغيرًا
\par 4 أما رؤوسها فكانت في حالة راحة: كان الرأس في المنتصف أكبر من الآخر، ومع ذلك كان يرتاح مع الباقي
\par 5 ثم نظرت وإذا نسر يطير بريشه ويملك على الأرض وعلى الساكنين فيها.
\par 6 ورأيت أن كل شيء تحت السماء خاضع لها، ولم يتكلم عليها أحد، ولا خليقة واحدة على الأرض
\par 7 ونظرتُ، فإذا النسر قد نهض على مخالبه، وكلم ريشه قائلًا:
\par 8 لا تراقبوا الجميع في آن واحد: لينام كل واحد في مكانه، وراقبوا بالترتيب:
\par 9 ولكن فلتحفظ الرؤوس للأخير.
\par 10 فنظرت وإذا الصوت لم يخرج من رأسها بل من وسط جسدها.
\par 11 وأحصيتُ ريشها المتقابل، فإذا هي ثمانية
\par 12 ونظرت وإذا ريشة واحدة على الجانب الأيمن ترتفع وتملك على كل الأرض
\par 13 وهكذا، عندما حكم، جاءت نهايته، ولم يعد مكانه يظهر: فقام من بعده وحكم، وقضى وقتًا رائعًا؛
\par 14 وحدث أنه لما ملك، جاءت نهايته أيضًا كالأولى، حتى إنه لم يظهر بعد
\par 15 ثم جاءه صوت وقال:
\par 16 اسمع أيها الذي حكمت الأرض طويلاً: هذا أقول لك، قبل أن لا تبدأ في الظهور بعد الآن،
\par 17 لن يبلغ أحد بعدك وقتك ولا نصفه
\par 18 ثم قام الثالث، وملك كالآخر من قبل، ولم يظهر أيضًا بعد ذلك
\par 19 وهكذا سارت الأمور مع كل البقية واحدًا تلو الآخر، حيث حكم كل واحد منهم، ثم اختفى
\par 20 ثم نظرتُ، وإذا بالريش الذي تبعني قد وقف على الجانب الأيمن مع مرور الوقت، ليحكم أيضًا؛ وحكم بعضهم، لكن بعد قليل اختفوا
\par 21 لأن بعضها أُقيم، لكن لم يُحكم.
\par 22 وبعد هذا نظرت وإذا الريشات الاثنتي عشرة لم تظهر بعد ولا الريشتان الصغيرتان.
\par 23 ولم يبق على جسد النسر إلا ثلاثة رؤوس مستقرة، وستة أجنحة صغيرة
\par 24 ثم رأيت أيضًا أن ريشتين صغيرتين انفصلتا عن الريش الست، وبقيتا تحت الرأس الذي على الجانب الأيمن، لأن الريش الأربع بقيت في مكانها
\par 25 ونظرتُ، وإذا بالريش الذي كان تحت الجناح يظن أنه سينصب نفسه وسيحكم
\par 26 ونظرت، وإذا بواحدة منها منصوبة، ولكنها سرعان ما اختفت
\par 27 وكان الثاني أسرع من الأول.
\par 28 ونظرت وإذا الاثنان الباقيان يفكران في أنفسهما أن يملكا.
\par 29 ولما ظنوا ذلك، إذا بأحد الرؤوس الساكنة قد استيقظ، وهو الذي في الوسط، لأنه كان أعظم من الرأسين الآخرين
\par 30 ثم رأيت أن الرأسين الآخرين متصلان به.
\par 31 وإذا الرأس قد تحول مع الذين معه، وأكل الريشتين اللتين تحت الجناح اللذين كانا يملكان.
\par 32 "ولكن هذا الرأس جعل الأرض كلها في خوف، وسيطر على كل من سكنوا الأرض بظلم عظيم، وكان له حكم العالم أكثر من كل الأجنحة التي كانت."
\par 33 وبعد هذا نظرتُ، وإذا الرأس الذي كان في الوسط قد اختفى فجأةً، كالأجنحة
\par 34 ولكن بقي الرأسان اللذان حكما أيضًا بنفس الطريقة على الأرض وعلى من سكنوها
\par 35 ونظرت فإذا الرأس الذي على اليمين يلتهم الذي على اليسار
\par 36 ثم سمعت صوتًا قال لي: انظر أمامك، وتأمل ما تراه
\par 37 ونظرت، وإذا بأسد كأنه زائر طُرد من الغابة، ورأيت أنه أرسل صوت إنسان إلى النسر وقال:
\par 38 اسمع، سأتحدث معك، وسيقول لك العلي،
\par 39 ألست أنت الباقي من الوحوش الأربعة، الذين جعلتهم يحكمون عالمي، حتى تأتي نهاية أزمنتهم من خلالهم؟
\par 40 وجاء الرابع، وغلب جميع الوحوش التي كانت ماضية، وتسلط على العالم بخوف عظيم، وعلى كل دائرة الأرض بظلم شرير كثير؛ وسكن على الأرض زمانًا طويلًا بمكر
\par 41 لأن الأرض لم تحكم بالحق.
\par 42 لأنك أحزنت الودعاء، وآذيت المسالمين، وأحببت الكاذبين، ودمرت مساكن الذين صنعوا ثمرًا، وهدمت أسوار الذين لم يضروك.
\par 43 لذلك صعد ظلمك إلى العلي، وكبريائك إلى القدير
\par 44 ونظر العلي أيضًا إلى الأزمنة الكبرياء، وها هي قد انقضت، ورجاساته قد تمت
\par 45 ولذلك لا تظهر بعد الآن أيها النسر، ولا أجنحتك المروعة، ولا ريشك الشرير، ولا رؤوسك الخبيثة، ولا مخالبك المؤذية، ولا كل جسدك الباطل:
\par 46 لكي تتجدد الأرض كلها، وتعود، وقد تحررت من بطشكَ، ولكي ترجو دينونة ورحمة من خلقها

\chapter{12}

\par 1 وحدث، بينما كان الأسد يكلم النسر بهذه الكلمات، رأيت،
\par 2 وإذا بالرأس الذي بقي والأجنحة الأربعة لم تظهر بعد، فذهب الاثنان إليه ونصبا نفسيهما للملك، وكانت مملكتهما صغيرة ومليئة بالضجيج
\par 3 ونظرت، وإذا هم قد اختفوا، واحترق جسد النسر كله حتى ارتجفت الأرض خوفًا عظيمًا. ثم استيقظت من قلقي وغيبوبة عقلي، ومن خوفي العظيم، وقلت لروحي:
\par 4 ها قد فعلتَ بي هذا، إذ بحثتَ عن سبل العلي
\par 5 ها أنا ذا متعبٌ عقليًا، وضعيفٌ جدًا روحيًا، وقوتي قليلةٌ في داخلي، بسبب الخوف الشديد الذي أصابني هذه الليلة
\par 6 لذلك سأطلب الآن من العلي أن يعزيني حتى النهاية.
\par 7 فقلتُ: يا ربُّ، إن كنتُ قد وجدتُ نعمةً أمامكَ، وإن تبرَّرتُ لديكَ أمامَ كثيرينَ، وإن صعدتْ صلاتي أمامَكَ،
\par 8 عزّني إذن، وأرني عبدك تفسير هذه الرؤيا المرعبة والفرق الواضح بينها، حتى تُعزي نفسي تمامًا
\par 9 لأنك حكمت عليّ أن أُرِيَني الأزمنة الأخيرة.
\par 10 فقال لي هذا تفسير الرؤيا.
\par 11 النسر الذي رأيته صاعدًا من البحر هو المملكة التي ظهرت في رؤيا أخيك دانيال
\par 12 ولكن لم يُشرح له، لذلك أخبرك به الآن
\par 13 هوذا أيام تأتي وتقوم مملكة على الأرض، وتكون مخيفة فوق جميع الممالك التي قبلها
\par 14 في ذلك العام يملك اثنا عشر ملكًا، واحدًا تلو الآخر:
\par 15 والذي يبدأ الثاني في الحكم، ويكون له وقت أطول من أي من الإثني عشر.
\par 16 وهذا ما ترمز إليه الأجنحة الاثني عشر التي رأيتها.
\par 17 وأما الصوت الذي سمعته يتكلم والذي رأيته لم يخرج من رؤوسه بل من وسط جسده فهذا هو التفسير.
\par 18 أنه بعد زمن تلك المملكة، ستنشأ صراعات عظيمة، وستكون معرضة لخطر الفشل: ومع ذلك، فلن تسقط حينئذٍ، بل ستعود مرة أخرى إلى بدايتها
\par 19 وبما أنك رأيت الريشات الثمانية الصغيرة السفلية ملتصقة بجناحيها، فهذا هو التفسير:
\par 20 أنه فيه سيقوم ثمانية ملوك، تكون أزمنتهم قصيرة، وسنيهم سريعة
\par 21 ويهلك اثنان منهم، مع اقتراب منتصف الوقت. يُحفظ أربعة حتى تقترب نهايتهم. ويُحفظ اثنان حتى النهاية
\par 22 وحيث رأيت ثلاثة رؤوس راسخة، فهذا هو التأويل:
\par 23 في أيامه الأخيرة، سيُقيم العلي ثلاث ممالك، ويُجدد فيها أشياء كثيرة، ويكون لهم سلطان على الأرض،
\par 24 والذين يسكنونها، بظلم كثير، فوق كل الذين كانوا قبلهم: لذلك يُدعون رؤوس النسر
\par 25 لأن هؤلاء هم الذين سيتممون شره، ويتممون نهايته
\par 26 وبما أنك رأيت أن الرأس الكبير لم يعد يظهر، فهذا يدل على أن أحدهم سيموت على فراشه، ومع ذلك يتألم
\par 27 أما الاثنان الباقيان فيُقتلان بالسيف.
\par 28 لأن سيف الواحد يأكل الآخر، ولكن في النهاية يسقط هو نفسه بالسيف.
\par 29 وحيث رأيت ريشتين تحت الجناحين سائرتين فوق الرأس الذي على الجانب الأيمن؛
\par 30 وهذا يدل على أن هؤلاء هم الذين حفظهم الله إلى نهايتهم: هذه هي المملكة الصغيرة والمليئة بالمتاعب، كما رأيت.
\par 31 والأسد الذي رأيته صاعدًا من الغابة وزائرًا، ويكلم النسر ويوبخه على إثمه بكل الكلام الذي سمعته؛
\par 32 هذا هو الممسوح الذي حفظه العلي لهم ولشرورهم إلى النهاية. هو يوبخهم ويوبخهم على قسوتهم
\par 33 لأنه سيُقيمهم أمامه أحياءً للحكم، ويوبخهم ويؤدبهم
\par 34 أما بقية شعبي فينقذهم بالرحمة، أولئك الذين ضُغطوا على تخومي، ويفرحهم حتى مجيء يوم الدينونة الذي تكلمت بك عنه منذ البدء
\par 35 هذا هو الحلم الذي رأيته، وهذه هي التفسيرات
\par 36 لقد وُجِّهتَ فقط لمعرفة هذا السر الأسمى.
\par 37 لذلك اكتب كل هذه الأشياء التي رأيتها في كتاب وأخفها.
\par 38 وعلمها لحكماء الناس، الذين تعرف قلوبهم، فيستطيعون فهم هذه الأسرار وحفظها
\par 39 لكن انتظر هنا سبعة أيام أخرى، حتى يُرى لك ما يحلو للعلي أن يُعلنه لك. ومع ذلك مضى في طريقه
\par 40 ولما رأى جميع الشعب أن الأيام السبعة قد مضت، ولم أرجع إلى المدينة، جمعوهم كلهم ​​من الصغير إلى الكبير، وجاءوا إليّ وقالوا:
\par 41 ما الذي أسأنا إليكِ؟ وأي شر فعلناه لكِ حتى تركتِنا وجلستِ هنا في هذا المكان؟
\par 42 فمن بين جميع الأنبياء، لم يبقَ لنا إلا أنت، كعنقود من العنب، وكشمعة في مكان مظلم، وكملاذ أو سفينة محفوظة من العاصفة
\par 43 أليست الشرور التي حلت بنا كافية؟
\par 44 إن تركتنا فكم كان خيراً لنا لو احترقنا أيضاً في وسط صهيون؟
\par 45 لأننا لسنا أفضل من الذين ماتوا هناك. فبكوا بصوت عظيم. فأجبتهم وقلت:
\par 46 تعزَّ يا إسرائيل، ولا تثقل يا بيت يعقوب.
\par 47 لأن العلي يذكرك، والقادر لم ينساك في التجربة.
\par 48 أما أنا، فلم أتركك، ولم أبتعد عنك، بل أتيت إلى هذا المكان لأصلي من أجل خراب صهيون، ولكي أطلب الرحمة من أجل وضع قدسك المتدني
\par 49 والآن اذهبوا إلى بيوتكم كل واحد، وبعد هذه الأيام سآتي إليكم
\par 50 فذهب الشعب إلى المدينة كما أمرتهم:
\par 51 أما أنا فبقيت في الحقل سبعة أيام كما أمرني الملاك، ولم آكل في تلك الأيام إلا من أزهار الحقل، وكان طعامي من الأعشاب

\chapter{13}

\par 1 وبعد سبعة أيام حلمت حلما في الليل:
\par 2 وإذا ريح قد هبت من البحر فحركت أمواجه كلها
\par 3 ونظرت، وإذا ذلك الرجل يتقوى مع ألوف السماء، وعندما التفت لينظر، ارتجف كل ما يُرى تحته
\par 4 وكان كلما خرج الصوت من فمه، يحترق كل من سمع صوته، كما تتلاشى الأرض عندما تشعر بالنار
\par 5 وبعد هذا نظرتُ، وإذا بحشدٍ من الرجال قد اجتمعوا، لا عدد لهم، من رياح السماء الأربع، لإخضاع الرجل الذي خرج من البحر
\par 6 ولكني نظرتُ، فإذا هو قد نقش لنفسه جبلًا عظيمًا، وطار عليه
\par 7 لكنني كنت سأرى المنطقة أو المكان الذي نُقش فيه التل، ولم أستطع
\par 8 وبعد هذا نظرت، وإذا كل الذين اجتمعوا لإخضاعه كانوا خائفين جدًا، ومع ذلك تجرأوا على القتال
\par 9 وإذا به إذ رأى عنف الجمع الذي جاء، لم يرفع يده، ولم يحمل سيفًا، ولا شيئًا من أدوات الحرب
\par 10 ولكنني رأيت فقط أنه أرسل من فمه كما لو كان نفخة من نار، ومن شفتيه نفسًا ملتهبًا، ومن لسانه أطلق شرارات وعواصف
\par 11 واختلطت كلها معًا؛ نفخة النار، ونفخة اللهب، والعاصفة العظيمة؛ وسقطت بعنف على الحشد الذي كان مستعدًا للقتال، وأحرقتهم جميعًا، حتى أنه فجأة لم يعد يُرى شيء من هذا الحشد الذي لا يُحصى، إلا الغبار ورائحة الدخان: عندما رأيت هذا خفت
\par 12 بعد ذلك رأيتُ نفس الرجل نازِلاً من الجبل، ودعا إليه جمعًا مسالمًا آخر
\par 13 فجاء إليه جمع كثير، ففرح بعضهم، وحزن بعضهم، وأوثق بعضهم، وأحضر آخرون من المذبوحين. ثم مرضت خوفًا شديدًا، فاستيقظت وقلت:
\par 14 لقد أريت عبدك هذه العجائب منذ البداية، وحسبتني مستحقًا أن تقبل صلاتي
\par 15 أرني الآن تفسير هذا الحلم.
\par 16 لأني كما أتصور في ذهني ويل للذين يبقون في تلك الأيام، وويل بالأكثر للذين لم يبقوا!
\par 17 لأن الذين لم يبقوا كانوا في حزن.
\par 18 الآن أفهم الأمور المدخرة في الأيام الأخيرة، التي ستصيبهم، والذين تركوا وراءهم.
\par 19 لذلك يتعرضون لمخاطر عظيمة وضرورات كثيرة، كما تعلن هذه الأحلام
\par 20 ولكن هل من الأسهل على من هو في خطر أن يأتي إلى هذه الأمور، من أن يختفي كسحابة من العالم، ولا يرى الأمور التي تحدث في الأيام الأخيرة. فأجابني وقال:
\par 21 سأريك تفسير الرؤيا، وأكشف لك الأمر الذي طلبته
\par 22 وبما أنك قلت عن المتخلفين، فهذا هو التأويل:
\par 23 من يتحمل الخطر في ذلك الوقت فقد حفظ نفسه. أما الذين يقعون في الخطر فهم من لديهم أعمال وإيمان بالله القدير
\par 24 فاعلم هذا إذًا: أن الباقين هم أسعد من الأموات
\par 25 هذا هو معنى الرؤيا: إذ رأيتَ رجلاً صاعدًا من وسط البحر،
\par 26 هو نفسه الذي جعل الله العلي له موعدًا عظيمًا، وهو الذي سيخلص خليقته بنفسه، ويأمر من تخلفوا عنه
\par 27 وبينما رأيت أنه خرج من فمه نفخة ريح ونار وعاصفة؛
\par 28 وأنه لم يكن يحمل سيفًا ولا آلة حرب، بل إن اندفاعه أهلك كل الجمع الذي جاء لإخضاعه. هذا هو التفسير:
\par 29 هوذا أيام تأتي حين يبدأ العلي بإنقاذ من على الأرض
\par 30 ويأتي إلى دهشة سكان الأرض
\par 31 ويتعهد المرء بمحاربة الآخر، ومدينة ضد أخرى، ومكان ضد آخر، وشعب ضد آخر، ومملكة ضد أخرى
\par 32 وسيكون الوقت الذي تحدث فيه هذه الأمور، وستحدث العلامات التي أريتك إياها سابقًا، وحينئذٍ يُعلن ابني الذي رأيته كإنسان صاعد
\par 33 وعندما يسمع جميع الشعب صوته، يغادر كل إنسان في أرضه المعركة التي خاضها ضد الآخر
\par 34 وسيُجمع جمع لا يُحصى، كما رأيتهم، مستعدين للمجيء والتغلب عليه بالقتال
\par 35 لكنه سيقف على رأس جبل صهيون.
\par 36 وتأتي صهيون وتظهر لجميع الناس مهيأة ومبنية كما رأيت الجبل المنحوت بغير أياد.
\par 37 وهذا ابني سيوبخ اختراعات تلك الأمم الشريرة، التي سقطت في العاصفة بسبب حياتها الشريرة؛
\par 38 ويضع أمامهم أفكارهم الشريرة، والعذابات التي سيبدأون بتعذيبهم بها، والتي هي مثل لهيب، ويهلكهم بلا عناء بالشريعة التي تشبه شريعتي
\par 39 ولما رأيت أنه جمع إليه جمعًا آخر مسالمًا؛
\par 40 هؤلاء هم الأسباط العشرة الذين سُبوا أسرى من أرضهم في زمن الملك هوشع، الذين أسرهم شلمنصر ملك آشور، وعبر بهم المياه، وهكذا وصلوا إلى أرض أخرى
\par 41 لكنهم اتفقوا على هذه النصيحة فيما بينهم، أنهم سيتركون جموع الوثنيين، ويخرجون إلى بلد أبعد، حيث لم يسكن بشر قط،
\par 42 لكي يحفظوا هناك فرائضهم التي لم يحفظوها قط في أرضهم
\par 43 ودخلوا إلى نهر الفرات من مضايق النهر.
\par 44 فأظهر لهم العلي آيات وأوقف الطوفان حتى عبروا.
\par 45 لأنه كان هناك طريق طويل لقطعه عبر تلك البلاد، أي سنة ونصف: وتُدعى تلك المنطقة أساريث
\par 46 ثم أقاموا هناك إلى الوقت الأخير، والآن عندما يبدأون في القدوم،
\par 47 سيمنع العلي ينابيع النهر مرة أخرى، حتى تتمكن من الجريان. لذلك رأيت الجموع بسلام
\par 48 وأما الباقون من شعبك فهم الذين وُجدوا داخل حدودي
\par 49 والآن عندما يُهلك جمهور الأمم المُجتمعة، فإنه سيدافع عن شعبه الباقي
\par 50 وحينئذٍ يُريهم عجائب عظيمة.
\par 51 فقلت يا سيد الدببة أرني هذا. لماذا رأيت الإنسان صاعدا من وسط البحر؟
\par 52 فقال لي: كما أنك لا تستطيع أن تبحث ولا أن تعرف الأشياء التي في عمق البحر، هكذا لا يستطيع أحد على الأرض أن يرى ابني، أو الذين معه، إلا في النهار
\par 53 هذا هو تفسير الحلم الذي رأيته، والذي به تُنير لك هذه الدنيا فقط
\par 54 لأنك تركت طريقك، واجتهدت في شريعتي وطلبتها
\par 55 لقد رتبت حياتك بالحكمة، وسميت الفهم أمك
\par 56 ولذلك أريتُك كنوز العلي. بعد ثلاثة أيام أخرى، سأُكلِّمك بأشياء أخرى، وأخبرك بأشياء عظيمة وعجيبة
\par 57 ثم خرجتُ إلى الحقل، أُسبِّح وأشكر العليَّ كثيرًا على عجائبه التي صنعها في حينه؛
\par 58 ولأنه يحكم الشيء نفسه، والأشياء التي تقع في أوقاتها: وجلست هناك ثلاثة أيام

\chapter{14}

\par 1 وفي اليوم الثالث، جلستُ تحت شجرة بلوط، وإذا بصوتٍ خرج من شجيرةٍ قبالتي، وقال: عزرا، عزرا
\par 2 فقلت: هأنذا يا رب، ووقفت على قدمي.
\par 3 ثم قال لي: في العليقة ظهرت لموسى وتكلمت معه حين خدم شعبي في مصر.
\par 4 وأرسلته وأخرجت شعبي من مصر، وأصعدته إلى الجبل الذي احتجزته فيه زمانًا طويلاً،
\par 5 وأخبروه بأمور عجيبة كثيرة، وأروه أسرار الأزمنة والنهاية، وأوصوه قائلين:
\par 6 هذه الكلمات ستُعلنها، وهذه ستُخفيها.
\par 7 والآن أقول لك،
\par 8 أن تحفظ في قلبك الآيات التي أريتها، والأحلام التي رأيتها، والتأويلات التي سمعتها
\par 9 لأنك ستُؤخذ بعيدًا عن الجميع، ومن الآن فصاعدًا ستبقى مع ابني، ومع من هم مثلك، حتى تنتهي الأزمنة
\par 10 لأن العالم قد فقد شبابه، والأزمنة بدأت تشيخ.
\par 11 فإن العالم مقسم إلى اثني عشر جزءًا، والأجزاء العشرة منه قد زالت بالفعل، ونصف الجزء العاشر.
\par 12 ويبقى ما بعد نصف العشر.
\par 13 الآن رتب بيتك ووبخ شعبك وعزِّ من هم في ضيق، والآن انبذ الفساد.
\par 14 دع عنك الأفكار الفانية، وتخلص من أعباء الإنسان، واخلع الآن الطبيعة الضعيفة،
\par 15 وتجاهل الأفكار التي تثقل كاهلك، وسارع إلى الفرار من هذه الأوقات
\par 16 لأن شرورًا أعظم من تلك التي رأيتها تحدث ستحدث فيما بعد
\par 17 فانظروا كم سيضعف العالم مع مرور الزمن، وكلما ازدادت الشرور على من يسكنونه
\par 18 لأن الوقت قد مضى بعيدًا، والاختيار صعب المنال، لأن الرؤيا التي رأيتها الآن تسرع إلى المجيء
\par 19 ثم أجبتُ أمامك وقلتُ:
\par 20 ها أنا يا رب، ماضٍ كما أمرتني، وأوبخ الشعب الحاضر. أما الذين سيولدون بعد ذلك، فمن ينذرهم؟ هكذا غُلق العالم في ظلام، والذين يسكنون فيه هم بلا نور
\par 21 لأن شريعتك قد احترقت، لذلك لا أحد يعرف ما الذي يحدث منك، أو العمل الذي سيبدأ
\par 22 ولكن إن كنت قد وجدت نعمة لديك، فأرسل الروح القدس إليّ، وسأكتب كل ما حدث في العالم منذ البدء، والذي كُتب في ناموسك، حتى يجد الناس سبيلك، ويحيا الذين يعيشون في الأيام الأخيرة
\par 23 فأجابني قائلاً: اذهب، اجمع الشعب، وقل لهم: لا يطلبوك أربعين يومًا
\par 24 لكن انظر، جهّز لنفسك الكثير من أشجار البقس، وخذ معك ساريا، ودبريه، وسليمية، وإيكانوس، وأسيل، هؤلاء الخمسة المستعدين للكتابة بسرعة؛
\par 25 وتعالَ إلى هنا، وسأشعلُ شمعةَ فهمٍ في قلبك، لن تنطفئ حتى تُنجزَ الأمورُ التي ستبدأُ في كتابتها
\par 26 وعندما تنتهي، ستنشر بعض الأمور، وستُظهر بعض الأمور سرًا للحكماء: غدًا في هذه الساعة ستبدأ في الكتابة
\par 27 ثم خرجت كما أمر، وجمعت كل الشعب، وقلت:
\par 28 اسمع هذه الكلمات يا إسرائيل.
\par 29 كان آباؤنا في البدء غرباء في مصر، ومن هناك أُنقذوا
\par 30 وأخذوا ناموس الحياة الذي لم يحفظوه، والذي تعديتموه أنتم أيضًا بعدهم
\par 31 حينئذٍ قُسِّمت الأرض، أرض صهيون، بينكم بالقرعة. ولكن آباءكم وأنتم فعلتم الإثم، ولم تحفظوا الطرق التي أوصاكم بها العلي
\par 32 وبما أنه قاضٍ عادل، فقد أخذ منكم في الوقت المناسب الشيء الذي كان قد أعطاكم إياه
\par 33 والآن أنتم ههنا، وإخوتكم بينكم.
\par 34 لذلك، إذا قمتم بإخضاع فهمكم وإصلاح قلوبكم، فسوف تبقون على قيد الحياة، وبعد الموت سوف تحصلون على الرحمة.
\par 35 لأنه بعد الموت سيأتي الدينونة، حين نحيا مرة أخرى، وحينئذٍ ستظهر أسماء الأبرار، وتُعلن أعمال الأشرار
\par 36 فلا يأتي إليّ أحد الآن، ولا يبحث عني هذه الأيام الأربعين
\par 37 فأخذت الرجال الخمسة كما أمرني، ودخلنا الحقل وأقمنا هناك
\par 38 وفي الغد، إذا بصوت يناديني قائلًا: يا عزرا، افتح فمك واشرب مما أسقيك
\par 39 ثم فتحت فمي، وإذا به يمد لي كأسًا ممتلئة، ممتلئة كما لو كانت ماءً، لكن لونها كان كالنار
\par 40 فأخذته وشربته، فلما شربته، نطق قلبي بالفهم، ونمت الحكمة في صدري، لأن روحي شددت ذاكرتي
\par 41 وانفتح فمي ولم يعد مغلقًا.
\par 42 فأعطى العلي فهماً للرجال الخمسة، فكتبوا عجائب الليل التي حدثت والتي لم يعرفوها. وجلسوا أربعين يوماً وكتبوا في النهار، وفي الليل كانوا يأكلون خبزاً.
\par 43 أما أنا، فقد تكلمت في النهار، ولم أمسك لساني في الليل.
\par 44 في أربعين يومًا كتبوا مائتين وأربعة كتابًا.
\par 45 ولما انقضت الأربعون يومًا، تكلم العلي قائلًا: «أول ما كتبته أنشره علانية، حتى يقرأه المستحقون وغير المستحقين.»
\par 46 لكن أبقِ السبعين في النهاية، لكي لا تُسلمها إلا للحكماء من الشعب
\par 47 ففيهم ينبوع الفهم، وينبوع الحكمة، وسيل المعرفة
\par 48 وفعلت ذلك.

\chapter{15}

\par 1 هوذا تكلم في مسامع شعبي بكلام النبوة الذي أضعه في فمك يقول الرب.
\par 2 واكتبوها على ورق، فإنها أمينة وصادقة
\par 3 لا تخف من التخيلات ضدك، ولا تدع عدم تصديق من يتحدثون ضدك يزعجك
\par 4 لأن كل الخائن سيموت في خيانته.
\par 5 ها أنا ذا، يقول الرب، أجلب على العالم أوبئة، سيفاً وجوعاً وموتاً وهلاكاً.
\par 6 لأن الشر قد لوث الأرض كلها فسادًا، وقد تمت أعمالهم المؤذية
\par 7 لذلك قال الرب،
\par 8 لن أمسك لساني بعد الآن عن شرورهم التي يرتكبونها بجُنوح، ولن أسمح لهم بالأشياء التي يمارسون بها الشر. هوذا دم البريء والبار يصرخ إليّ، ونفوس الأبرار تشكو باستمرار
\par 9 ولذلك، يقول الرب، سأنتقم لهم انتقامًا، وأستعيد لنفسي كل دم بريء من بينهم
\par 10 هوذا شعبي يُساق كغنم إلى الذبح. لا أدعهم الآن يسكنون في أرض مصر
\par 11 ولكني سأجلبهم بيد شديدة وذراع ممدودة، وأضرب مصر بالوبايات كما في السابق، وأهلك كل أرضها
\par 12 ستحزن مصر، وسيُصاب أساسها بالوباء والعقاب الذي سينزله الله عليها
\par 13 الذين يفلحون الأرض سينوحون، لأن بذورهم ستذبل بسبب القصف والبرد، ومع وجود كوكبة مخيفة
\par 14 ويل للعالم وللساكنين فيه!
\par 15 لأنه قد اقترب السيف وهلاكهم، ويقوم شعب ويحارب شعبا آخر، والسيوف في أيديهم.
\par 16 لأنه ستكون هناك فتنة بين الناس، وغزو بعضهم بعضًا؛ ولن يراعوا ملوكهم ولا أمراءهم، وسيظل مسار أعمالهم في قبضتهم
\par 17 يرغب الرجل في الدخول إلى مدينة، ولا يستطيع.
\par 18 لأنه بسبب كبريائهم تضطرب المدن، وتدمر البيوت، ويخاف الناس.
\par 19 لا يرحم الإنسان جاره، بل يهدم بيوته بالسيف، وينهب أملاكه، بسبب قلة الخبز، وبسبب ضيق عظيم
\par 20 هأنذا، يقول الله، أدعو إلى إجلالي كل ملوك الأرض، الذين من مشرق الشمس، ومن الجنوب، ومن المشرق، ولبنان، ليثوروا بعضهم على بعض، ويجازوا ما فعلوا بهم
\par 21 كما يفعلون اليوم بمختاري، كذلك أفعل أنا أيضًا، وأجازي في حضنهم. هكذا قال السيد الرب:
\par 22 يميني لن تشفق على الخطاة، وسيفي لن يسكت عن سفاكي دماء الأبرياء على الأرض
\par 23 خرجت نار غضبه، وأكلت أسس الأرض والخطاة، كالقش المشتعل
\par 24 ويلٌ للذين يخطئون ولا يحفظون وصاياي! يقول الرب.
\par 25 لا أشفق عليهم. اذهبوا أيها الأولاد من القوة، لا تنجسوا مقدسي.
\par 26 لأن الرب يعلم جميع الذين يخطئون إليه، ولذلك يسلمهم إلى الموت والهلاك
\par 27 الآنَ قَدْ جَاءَتِ الْبَلاَئِسُ عَلَى الأَرْضِ كُلِّهَا، وَسَتَمْكُثُونَ فِيهَا. لأَنَّ اللهَ لَنْ يُخَلِّصَكُمْ لأَنَّكُمْ أَخْطَأْتُمْ إِلَيْهِ
\par 28 انظر إلى رؤيا مروعة، ومنظرها من الشرق:
\par 29 حيث تخرج أمم التنانين العربية بمركبات كثيرة، وتحمل جموعهم كالريح على الأرض، حتى أن كل من يسمعهم يخاف ويرتعد.
\par 30 وسيخرج الكرمانيون أيضًا غاضبين كخنازير الغابة البرية، وسيأتون بقوة عظيمة، وينضمون إليهم في المعركة، ويدمرون جزءًا من أرض الآشوريين
\par 31 وحينها ستكون للتنانين اليد العليا، متذكرةً طبيعتها؛ وإذا انقلبت، متآمرةً معًا بقوة عظيمة لاضطهادها،
\par 32 حينئذٍ سيضطرب هؤلاء، ويصمتون بقوتهم، ويهربون
\par 33 ومن أرض آشور يحاصرهم العدو، ويفنون بعضهم، ويكون في جيشهم خوف ورعب، وخصام بين ملوكهم
\par 34 انظر إلى السحب من الشرق ومن الشمال إلى الجنوب، وهي مريعة المنظر للغاية، مليئة بالغضب والعاصفة
\par 35 سيصطدم بعضهم ببعض، وسيسقطون عددًا كبيرًا من النجوم على الأرض، حتى نجمهم، ويكون الدم من السيف إلى البطن،
\par 36 وروث الناس إلى جنب البعير.
\par 37 ويكون خوف عظيم ورعدة على الأرض والذين يرون الغضب يرتعدون ويأتي عليهم الرعدة.
\par 38 ثم ستأتي عواصف شديدة من الجنوب، ومن الشمال، وجزء آخر من الغرب
\par 39 وتهب رياح شديدة من المشرق فتفتحه، فتُهلك السحابة التي أثارها غضبًا، والنجم الذي حركه ليُثير الرعب نحو الريح الشرقية والغربية
\par 40 وتنتفخ السحب العظيمة والقوية غضبًا، والنجم، فتُرهب كل الأرض وساكنيها، وتسكب على كل مكان مرتفع ومشهور كوكبًا مخيفًا،
\par 41 نار، وبَرَد، وسيوف طائرة، ومياه كثيرة، حتى تمتلئ جميع الحقول، وجميع الأنهار، بوفرة المياه الكثيرة
\par 42 ويهدمون المدن والأسوار والجبال والتلال وأشجار الغاب وعشب المروج وقمحها
\par 43 ويذهبون بثبات إلى بابل ويرعبونها.
\par 44 فيأتون إليها ويحاصرونها، ويصبون عليها النجم وكل الغضب، ثم يصعد الغبار والدخان إلى السماء، وينوح عليها كل من حولها.
\par 45 والذين يبقون تحتها يخدمون الذين أخافوها
\par 46 وأنتِ يا آسيا، شريكة رجاء بابل، ومجد جوهرها:
\par 47 ويل لك أيها البائس، لأنك جعلت نفسك مثلها، وزيّنت بناتك بالزنى، لكي يرضين ويفتخرن بعشاقك الذين لطالما رغبوا في ممارسة الزنا معك
\par 48 لقد تبعتَ المكروهة في جميع أعمالها واختراعاتها. لذلك قال الله،
\par 49 سأرسل عليكِ الضربات؛ الترمل، والفقر، والمجاعة، والسيف، والوباء، لتخريب بيوتكِ بالدمار والموت
\par 50 وسيجف مجد قوتك كزهرة، وسترتفع الحرارة المرسلة عليك
\par 51 ستضعفين كامرأة فقيرة مجروحة، وكشخص مُؤدب بالجروح، حتى لا يتمكن الأقوياء والعشاق من استقبالك
\par 52 ليت غيرتي كانت ستفعل بك هكذا، يقول الرب،
\par 53 لو لم تقتل مختاري دائمًا، وتعظم ضربة يديك، وتقول على موتاهم، وأنت سكران،
\par 54 أأظهر جمال وجهك؟
\par 55 "إن جزاء زناك سيكون في حضنك، لذلك سوف تتلقى الجزاء."
\par 56 كما فعلت بمختاري، يقول الرب، هكذا يفعل الله بك، ويسلمك إلى الشر
\par 57 سيموت أطفالك جوعًا، وستسقطين بالسيف. ستُهدم مدنك، ويهلك جميع سكانك بالسيف في الحقل
\par 58 سيموت سكان الجبال من الجوع، ويأكلون لحومهم، ويشربون دمائهم، من شدة الجوع إلى الخبز، والعطش إلى الماء
\par 59 ستأتي أنت التعيس عبر البحر، وستتلقى الأوبئة مرة أخرى
\par 60 وفي الطريق، ينقضون على المدينة المهجورة، ويدمرون جزءًا من أرضك، ويستهلكون جزءًا من مجدك، ويعودون إلى بابل المدمرة
\par 61 وتطرحهم كالقش، ويكونون لك كالنار
\par 62 ويأكلك أنت ومدنك وأرضك وجبالك، ويحرقون بالنار جميع غاباتك وأشجارك المثمرة
\par 63 سيأسرون أولادك، وانظروا، سينهبون ما لديك، ويشوهون جمال وجهك

\chapter{16}

\par 1 ويلٌ لكِ يا بابل وآسيا! ويلٌ لكِ يا مصر وسوريا!
\par 2 تحزموا بأثواب من قماش وشعر، وابكوا على أولادكم، واندموا؛ لأن هلاككم قريب
\par 3 أُرسِلَ عليك سيف، فمن يرده؟
\par 4 أُرسِلَتْ عَلَيْكُمْ نَارٌ فَمَنْ يُطْفِئُهَا؟
\par 5 أُرسِلَتْ إِلَيْكُمُ الْأَوْبَاءُ، فَمَنْ هُوَ الَّذِي يَدْفِعُهَا؟
\par 6 هل يجوز لأحد أن يطرد أسدًا جائعًا في الغابة؟ أو هل يجوز لأحد أن يطفئ النار في القش عندما تبدأ في الاشتعال؟
\par 7 هل يجوز رد السهم الذي أطلقه رامٍ قوي؟
\par 8 الرب القدير يرسل الضربات فمن هو الذي يستطيع أن يطردها؟
\par 9 تخرج نار من غضبه، فمن ذا الذي يطفئها؟
\par 10 يُلقي بروقًا، فمن لا يخاف؟ يُرعد، فمن لا يخاف؟
\par 11 الرب يهدد، فمن لا يُسحق أمامه؟
\par 12 الأرض تزلزلت وأساساتها، والبحر يرتفع بأمواج من العمق، وأمواجه تضطرب، وأسماكه أيضًا، أمام الرب وأمام مجد قدرته
\par 13 لأن يمينه قوية التي تحني القوس، وسهامه التي يرميها حادة، ولن تخطئ، عندما تبدأ في الرمي إلى أقاصي العالم
\par 14 هوذا الضربات مُرسَلة، ولن تعود حتى تأتي على الأرض
\par 15 النار مشتعلة، ولن تنطفئ حتى تلتهم أساس الأرض
\par 16 كما أن السهم الذي يطلقه رامي عظيم لا يرجع إلى الوراء، كذلك الضربات التي ترسل على الأرض لا تعود أيضا.
\par 17 ويل لي! ويل لي! من يُنقذني في تلك الأيام؟
\par 18 بداية الأحزان والحزن العظيم، بداية المجاعة والموت العظيم، بداية الحروب، وستقف القوى خائفة، بداية الشرور! ماذا أفعل عندما تأتي هذه الشرور؟
\par 19 هوذا المجاعة والطاعون، والضيق والضيق، تُرسل كسياط للإصلاح
\par 20 ولكن مع كل هذه الأمور، لن يتراجعوا عن شرورهم، ولن يتذكروا دائمًا الضربات
\par 21 هوذا الطعام سيكون رخيصًا جدًا على الأرض، لدرجة أنهم سيظنون أنهم في حالة جيدة، وحتى حينها ستنمو الشرور على الأرض، سيف، ومجاعة، وفوضى عارمة
\par 22 لأن كثيرين من سكان الأرض سيهلكون من الجوع، والذين ينجو من الجوع سيهلكهم السيف
\par 23 ويُطرح الأموات كالزبل، ولا يكون من يُعزيهم، لأن الأرض تُخْرَب، والمدن تُهدم
\par 24 لن يبقى إنسان ليعمل الأرض ويزرعها
\par 25 والأشجار تعطي ثمرها فمن يجمعها؟
\par 26 ينضج العنب، فمن يدوسه؟ لأن كل الأماكن تكون خربة من الناس
\par 27 حتى يرغب الإنسان في رؤية الآخر وسماع صوته.
\par 28 لأنه يبقى من المدينة عشرة ومن الحقل اثنان يختبئون في الغابات الكثيفة وفي شقوق الصخور.
\par 29 كما في بستان زيتون، على كل شجرة يبقى ثلاث أو أربع حبات زيتون؛
\par 30 أو كما يُجمع كرم، تبقى مجموعات ممن يبحثون باجتهاد في الكرم:
\par 31 هكذا في تلك الأيام يبقى ثلاثة أو أربعة ممن يفتشون بيوتهم بالسيف
\par 32 وتخرب الأرض، وتشيخ حقولها، وتمتلئ طرقها وكل مسالكها أشواكًا، لأنه لن يعبرها أحد
\par 33 ستحزن العذارى بلا عرسان، وستحزن النساء بلا أزواج، وستحزن بناتهن بلا معينين
\par 34 في الحروب يُهلك عرسانهن، ويهلك أزواجهن من الجوع
\par 35 اسمعوا الآن هذه الأمور وافهموها يا عبيد الرب.
\par 36 هوذا كلام الرب اقبلوه. لا تصدقوا الآلهة التي تكلم عنها الرب.
\par 37 هوذا الضربات تقترب ولا تهدأ.
\par 38 كما أن المرأة عندما تلد ابنها في الشهر التاسع، بعد ساعتين أو ثلاث من ولادتها، يحيط ببطنها آلام عظيمة، ولا تتأخر هذه الآلام لحظة واحدة عندما يخرج الطفل.
\par 39 هكذا لن تتأخر الضربات عن المجيء إلى الأرض، وسيحزن العالم، وستأتي عليه الأحزان من كل جانب
\par 40 يا شعبي، اسمعوا كلمتي. استعدوا لمعركتكم، وفي تلك الشرور كونوا كالغرباء على الأرض.
\par 41 من يبيع فليكن كالهارب، ومن يشتري فليكن كالخاسر
\par 42 من يتاجر بالتجارة كمن لا ربح له منها، ومن يبني كمن لا يسكن فيه
\par 43 من يزرع كأنه لا يحصد، وكذلك من يغرس الكرم كأنه لا يجمع
\par 44 الذين يتزوجون كمن لن ينجبوا، والذين لا يتزوجون كمن هم الأرامل
\par 45 ولذلك فإن الذين يعملون يعملون عبثًا:
\par 46 لأن الغرباء يحصدون أثمارهم وينهبون أملاكهم ويهدمون بيوتهم ويسبيون أولادهم، لأنهم في السبي والجوع يولدون أولاداً.
\par 47 والذين يتاجرون ببضائعهم بالنهب، يزيّنون مدنهم وبيوتهم وممتلكاتهم وأشخاصهم:
\par 48 كلما زاد غضبي عليهم بسبب خطيئتهم، يقول الرب.
\par 49 كما أن الزانية تحسد المرأة الصالحة الفاضلة:
\par 50 هكذا تبغض البر الإثم، حين تزين نفسها، وتتهمها في وجهها، حين يأتي من يدافع عن من يفحص كل خطيئة على الأرض باجتهاد
\par 51 ولذلك لا تكونوا مثله ولا مثل أعماله.
\par 52 لأنه بعد قليل يُنزع الإثم من الأرض ويملك البر بينكم.
\par 53 لا يقل الخاطئ إنه لم يخطئ، لأن الله يحرق جمر نار على رأسه، وهو القائل أمام الرب الإله ومجده: لم أخطئ
\par 54 هوذا الرب يعلم جميع أعمال البشر، وتصوراتهم، وأفكارهم، وقلوبهم
\par 55 الذي لم ينطق إلا بكلمة: لتكن الأرض فكانت. لتكن السماء فكانت
\par 56 في كلمته خُلقت النجوم، وهو يعلم عددها.
\par 57 هو يفتش اللجج وكنوزه، ويقيس البحر وما فيه.
\par 58 أغلق البحر في وسط المياه، وبكلمته علق الأرض على المياه
\par 59 يبسط السماوات كقبة، على المياه أسسها
\par 60 جعل في البرية ينابيع ماء، وبركًا على رؤوس الجبال، لكي تتدفق السيول من الصخور العالية لتسقي الأرض
\par 61 خلق الإنسان، ووضع قلبه في وسط الجسد، وأعطاه نفسًا وحياة وفهمًا
\par 62 نعم، وروح الله القدير، الذي خلق كل الأشياء، ويبحث عن كل الأشياء المخفية في أسرار الأرض،
\par 63 إنه يعلم أفكاركم وما تفكرون به في قلوبكم، حتى أولئك الذين يخطئون ويريدون أن يكتموا خطيئتهم
\par 64 لذلك فحص الرب جميع أعمالكم بدقة، وسيخزيكم ​​جميعًا
\par 65 وعندما تُذكر خطاياكم، تُخزون أمام الناس، وتكون خطاياكم هي المشتكين لكم في ذلك اليوم.
\par 66 ماذا ستفعلون؟ أو كيف ستخفون خطاياكم أمام الله وملائكته؟
\par 67 هوذا الله نفسه هو القاضي، فاخشوه. اتركوا خطاياكم، وانسي آثامكم، ولا تتدخلوا فيها إلى الأبد. هكذا يهديكم الله، وينقذكم من كل ضيق
\par 68 لأنه هوذا غضب جمع كثير متقد عليكم، فيأخذون بعضًا منكم ويطعمونكم بطالين مما ذبح للأصنام
\par 69 والذين يوافقون عليها يُستهزأ بهم ويُهانون ويُداسون بالأقدام
\par 70 لأنه سيكون في كل مكان، وفي المدن المجاورة، تمرد عظيم على الذين يخافون الرب
\par 71 سيكونون كالمجانين، لا يبقوا على أحد، بل يستمرون في إفساد وتدمير من يخاف الرب
\par 72 لأنهم سيُبددون ويسلبون أملاكهم ويطردونهم من بيوتهم
\par 73 حينئذٍ سيُعرف مختاريّ، وسيُمتحنون كالذهب في النار
\par 74 اسمعوا يا أحبائي، يقول الرب: هوذا أيام الضيق قريبة، ولكني سأنقذكم منها
\par 75 لا تخافوا ولا تشكوا، فالله هو هاديكم،
\par 76 "وقائد الذين يحفظون وصاياي وشرائعي يقول السيد الرب لا تثقل عليكم خطاياكم ولا ترتفع آثامكم."
\par 77 ويلٌ للذين رُبطوا بخطاياهم، وتغطوا بآثامهم، كما يُغطى الحقل بالشجيرات، ويُغطى طريقه بالأشواك، فلا يستطيع أحدٌ أن يسلكه!
\par 78 يُترك عاريًا، ويُلقى في النار ليُحرق بها

\end{document}