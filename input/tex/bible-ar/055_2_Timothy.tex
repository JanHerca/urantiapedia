\begin{document}

\title{تيموثاوس2}


\chapter{1}

\par 1 بُولُسُ، رَسُولُ يَسُوعَ الْمَسِيحِ بِمَشِيئَةِ اللهِ، لأَجْلِ وَعْدِ الْحَيَاةِ الَّتِي فِي يَسُوعَ الْمَسِيحِ.
\par 2 إِلَى تِيمُوثَاوُسَ الاِبْنِ الْحَبِيبِ. نِعْمَةٌ وَرَحْمَةٌ وَسَلاَمٌ مِنَ اللهِ الآبِ وَالْمَسِيحِ يَسُوعَ رَبِّنَا.
\par 3 إِنِّي أَشْكُرُ اللهَ الَّذِي أَعْبُدُهُ مِنْ أَجْدَادِي بِضَمِيرٍ طَاهِرٍ، كَمَا أَذْكُرُكَ بِلاَ انْقِطَاعٍ فِي طِلْبَاتِي لَيْلاً وَنَهَاراً،
\par 4 مُشْتَاقاً أَنْ أَرَاكَ، ذَاكِراً دُمُوعَكَ لِكَيْ أَمْتَلِئَ فَرَحاً،
\par 5 إِذْ أَتَذَكَّرُ الإِيمَانَ الْعَدِيمَ الرِّيَاءِ الَّذِي فِيكَ، الَّذِي سَكَنَ أَوَّلاً فِي جَدَّتِكَ لَوْئِيسَ وَأُمِّكَ أَفْنِيكِي، وَلَكِنِّي مُوقِنٌ أَنَّهُ فِيكَ أَيْضاً.
\par 6 فَلِهَذَا السَّبَبِ أُذَكِّرُكَ أَنْ تُضْرِمَ أَيْضاً مَوْهِبَةَ اللهِ الَّتِي فِيكَ بِوَضْعِ يَدَيَّ،
\par 7 لأَنَّ اللهَ لَمْ يُعْطِنَا رُوحَ الْفَشَلِ، بَلْ رُوحَ الْقُوَّةِ وَالْمَحَبَّةِ وَالنُّصْحِ.
\par 8 فَلاَ تَخْجَلْ بِشَهَادَةِ رَبِّنَا، وَلاَ بِي أَنَا أَسِيرَهُ، بَلِ اشْتَرِكْ فِي احْتِمَالِ الْمَشَقَّاتِ لأَجْلِ الإِنْجِيلِ بِحَسَبِ قُوَّةِ اللهِ،
\par 9 الَّذِي خَلَّصَنَا وَدَعَانَا دَعْوَةً مُقَدَّسَةً، لاَ بِمُقْتَضَى أَعْمَالِنَا، بَلْ بِمُقْتَضَى الْقَصْدِ وَالنِّعْمَةِ الَّتِي أُعْطِيَتْ لَنَا فِي الْمَسِيحِ يَسُوعَ قَبْلَ الأَزْمِنَةِ الأَزَلِيَّةِ،
\par 10 وَإِنَّمَا أُظْهِرَتِ الآنَ بِظُهُورِ مُخَلِّصِنَا يَسُوعَ الْمَسِيحِ، الَّذِي أَبْطَلَ الْمَوْتَ وَأَنَارَ الْحَيَاةَ وَالْخُلُودَ بِوَاسِطَةِ الإِنْجِيلِ.
\par 11 الَّذِي جُعِلْتُ أَنَا لَهُ كَارِزاً وَرَسُولاً وَمُعَلِّماً لِلأُمَمِ.
\par 12 لِهَذَا السَّبَبِ أَحْتَمِلُ هَذِهِ الأُمُورَ أَيْضاً. لَكِنَّنِي لَسْتُ أَخْجَلُ، لأَنَّنِي عَالِمٌ بِمَنْ آمَنْتُ، وَمُوقِنٌ أَنَّهُ قَادِرٌ أَنْ يَحْفَظَ وَدِيعَتِي إِلَى ذَلِكَ الْيَوْمِ.
\par 13 تَمَسَّكْ بِصُورَةِ الْكَلاَمِ الصَّحِيحِ الَّذِي سَمِعْتَهُ مِنِّي، فِي الإِيمَانِ وَالْمَحَبَّةِ الَّتِي فِي الْمَسِيحِ يَسُوعَ.
\par 14 اِحْفَظِ الْوَدِيعَةَ الصَّالِحَةَ بِالرُّوحِ الْقُدُسِ السَّاكِنِ فِينَا.
\par 15 أَنْتَ تَعْلَمُ هَذَا أَنَّ جَمِيعَ الَّذِينَ فِي أَسِيَّا ارْتَدُّوا عَنِّي، الَّذِينَ مِنْهُمْ فِيجَلُّسُ وَهَرْمُوجَانِسُ.
\par 16 لِيُعْطِ الرَّبُّ رَحْمَةً لِبَيْتِ أُنِيسِيفُورُسَ، لأَنَّهُ مِرَاراً كَثِيرَةً أَرَاحَنِي وَلَمْ يَخْجَلْ بِسِلْسِلَتِي،
\par 17 بَلْ لَمَّا كَانَ فِي رُومِيَةَ طَلَبَنِي بِأَوْفَرِ اجْتِهَادٍ فَوَجَدَنِي.
\par 18 لِيُعْطِهِ الرَّبُّ أَنْ يَجِدَ رَحْمَةً مِنَ الرَّبِّ فِي ذَلِكَ الْيَوْمِ. وَكُلُّ مَا كَانَ يَخْدِمُ فِي أَفَسُسَ أَنْتَ تَعْرِفُهُ جَيِّداً.

\chapter{2}

\par 1 فَتَقَوَّ أَنْتَ يَا ابْنِي بِالنِّعْمَةِ الَّتِي فِي الْمَسِيحِ يَسُوعَ.
\par 2 وَمَا سَمِعْتَهُ مِنِّي بِشُهُودٍ كَثِيرِينَ، أَوْدِعْهُ أُنَاساً أُمَنَاءَ، يَكُونُونَ أَكْفَاءً أَنْ يُعَلِّمُوا آخَرِينَ أَيْضاً.
\par 3 فَاشْتَرِكْ أَنْتَ فِي احْتِمَالِ الْمَشَقَّاتِ كَجُنْدِيٍّ صَالِحٍ لِيَسُوعَ الْمَسِيحِ.
\par 4 لَيْسَ أَحَدٌ وَهُوَ يَتَجَنَّدُ يَرْتَبِكُ بِأَعْمَالِ الْحَيَاةِ لِكَيْ يُرْضِيَ مَنْ جَنَّدَهُ،
\par 5 وَأَيْضاً إِنْ كَانَ أَحَدٌ يُجَاهِدُ لاَ يُكَلَّلُ إِنْ لَمْ يُجَاهِدْ قَانُونِيّاً.
\par 6 يَجِبُ أَنَّ الْحَرَّاثَ الَّذِي يَتْعَبُ يَشْتَرِكُ هُوَ أَوَّلاً فِي الأَثْمَارِ.
\par 7 افْهَمْ مَا أَقُولُ. فَلْيُعْطِكَ الرَّبُّ فَهْماً فِي كُلِّ شَيْءٍ.
\par 8 اُذْكُرْ يَسُوعَ الْمَسِيحَ الْمُقَامَ مِنَ الأَمْوَاتِ مِنْ نَسْلِ دَاوُدَ بِحَسَبِ إِنْجِيلِي،
\par 9 الَّذِي فِيهِ أَحْتَمِلُ الْمَشَقَّاتِ حَتَّى الْقُيُودَ كَمُذْنِبٍ. لَكِنَّ كَلِمَةَ اللهِ لاَ تُقَيَّدُ.
\par 10 لأَجْلِ ذَلِكَ أَنَا أَصْبِرُ عَلَى كُلِّ شَيْءٍ لأَجْلِ الْمُخْتَارِينَ، لِكَيْ يَحْصُلُوا هُمْ أَيْضاً عَلَى الْخَلاَصِ الَّذِي فِي الْمَسِيحِ يَسُوعَ مَعَ مَجْدٍ أَبَدِيٍّ.
\par 11 صَادِقَةٌ هِيَ الْكَلِمَةُ: أَنَّهُ إِنْ كُنَّا قَدْ مُتْنَا مَعَهُ فَسَنَحْيَا أَيْضاً مَعَهُ.
\par 12 إِنْ كُنَّا نَصْبِرُ فَسَنَمْلِكُ أَيْضاً مَعَهُ. إِنْ كُنَّا نُنْكِرُهُ فَهُوَ أَيْضاً سَيُنْكِرُنَا.
\par 13 إِنْ كُنَّا غَيْرَ أُمَنَاءَ فَهُوَ يَبْقَى أَمِيناً، لَنْ يَقْدِرَ أَنْ يُنْكِرَ نَفْسَهُ.
\par 14 فَكِّرْ بِهَذِهِ الأُمُورِ مُنَاشِداً قُدَّامَ الرَّبِّ أَنْ لاَ يَتَمَاحَكُوا بِالْكَلاَمِ، الأَمْرُ غَيْرُ النَّافِعِ لِشَيْءٍ، لِهَدْمِ السَّامِعِينَ.
\par 15 اجْتَهِدْ أَنْ تُقِيمَ نَفْسَكَ ِللهِ مُزَكّىً، عَامِلاً لاَ يُخْزَى، مُفَصِّلاً كَلِمَةَ الْحَقِّ بِالاِسْتِقَامَةِ.
\par 16 وَأَمَّا الأَقْوَالُ الْبَاطِلَةُ الدَّنِسَةُ فَاجْتَنِبْهَا، لأَنَّهُمْ يَتَقَدَّمُونَ إِلَى أَكْثَرِ فُجُورٍ،
\par 17 وَكَلِمَتُهُمْ تَرْعَى كَآكِلَةٍ، الَّذِينَ مِنْهُمْ هِيمِينَايُسُ وَفِيلِيتُسُ،
\par 18 اللَّذَانِ زَاغَا عَنِ الْحَقِّ، قَائِلَيْنِ: «إِنَّ الْقِيَامَةَ قَدْ صَارَتْ» فَيَقْلِبَانِ إِيمَانَ قَوْمٍ.
\par 19 وَلَكِنَّ أَسَاسَ اللهِ الرَّاسِخَ قَدْ ثَبَتَ، إِذْ لَهُ هَذَا الْخَتْمُ. يَعْلَمُ الرَّبُّ الَّذِينَ هُمْ لَهُ. وَلْيَتَجَنَّبِ الإِثْمَ كُلُّ مَنْ يُسَمِّي اسْمَ الْمَسِيحِ.
\par 20 وَلَكِنْ فِي بَيْتٍ كَبِيرٍ لَيْسَ آنِيَةٌ مِنْ ذَهَبٍ وَفِضَّةٍ فَقَطْ، بَلْ مِنْ خَشَبٍ وَخَزَفٍ أَيْضاً، وَتِلْكَ لِلْكَرَامَةِ وَهَذِهِ لِلْهَوَانِ.
\par 21 فَإِنْ طَهَّرَ أَحَدٌ نَفْسَهُ مِنْ هَذِهِ يَكُونُ إِنَاءً لِلْكَرَامَةِ، مُقَدَّساً، نَافِعاً لِلسَّيِّدِ، مُسْتَعَدّاً لِكُلِّ عَمَلٍ صَالِحٍ.
\par 22 أَمَّا الشَّهَوَاتُ الشَّبَابِيَّةُ فَاهْرُبْ مِنْهَا، وَاتْبَعِ الْبِرَّ وَالإِيمَانَ وَالْمَحَبَّةَ وَالسَّلاَمَ مَعَ الَّذِينَ يَدْعُونَ الرَّبَّ مِنْ قَلْبٍ نَقِيٍّ.
\par 23 وَالْمُبَاحَثَاتُ الْغَبِيَّةُ وَالسَّخِيفَةُ اجْتَنِبْهَا، عَالِماً أَنَّهَا تُوَلِّدُ خُصُومَاتٍ،
\par 24 وَعَبْدُ الرَّبِّ لاَ يَجِبُ أَنْ يُخَاصِمَ، بَلْ يَكُونُ مُتَرَفِّقاً بِالْجَمِيعِ، صَالِحاً لِلتَّعْلِيمِ، صَبُوراً عَلَى الْمَشَقَّاتِ،
\par 25 مُؤَدِّباً بِالْوَدَاعَةِ الْمُقَاوِمِينَ، عَسَى أَنْ يُعْطِيَهُمُ اللهُ تَوْبَةً لِمَعْرِفَةِ الْحَقِّ،
\par 26 فَيَسْتَفِيقُوا مِنْ فَخِّ إِبْلِيسَ إِذْ قَدِ اقْتَنَصَهُمْ لإِرَادَتِهِ.

\chapter{3}

\par 1 وَلَكِنِ اعْلَمْ هَذَا أَنَّهُ فِي الأَيَّامِ الأَخِيرَةِ سَتَأْتِي أَزْمِنَةٌ صَعْبَةٌ،
\par 2 لأَنَّ النَّاسَ يَكُونُونَ مُحِبِّينَ لأَنْفُسِهِمْ، مُحِبِّينَ لِلْمَالِ، مُتَعَظِّمِينَ، مُسْتَكْبِرِينَ، مُجَدِّفِينَ، غَيْرَ طَائِعِينَ لِوَالِدِيهِمْ، غَيْرَ شَاكِرِينَ، دَنِسِينَ،
\par 3 بِلاَ حُنُوٍّ، بِلاَ رِضىً، ثَالِبِينَ، عَدِيمِي النَّزَاهَةِ، شَرِسِينَ، غَيْرَ مُحِبِّينَ لِلصَّلاَحِ،
\par 4 خَائِنِينَ، مُقْتَحِمِينَ، مُتَصَلِّفِينَ، مُحِبِّينَ لِلَّذَّاتِ دُونَ مَحَبَّةٍ لِلَّهِ،
\par 5 لَهُمْ صُورَةُ التَّقْوَى وَلَكِنَّهُمْ مُنْكِرُونَ قُوَّتَهَا. فَأَعْرِضْ عَنْ هَؤُلاَءِ.
\par 6 فَإِنَّهُ مِنْ هَؤُلاَءِ هُمُ الَّذِينَ يَدْخُلُونَ الْبُيُوتَ، وَيَسْبُونَ نُسَيَّاتٍ مُحَمَّلاَتٍ خَطَايَا، مُنْسَاقَاتٍ بِشَهَوَاتٍ مُخْتَلِفَةٍ.
\par 7 يَتَعَلَّمْنَ فِي كُلِّ حِينٍ، وَلاَ يَسْتَطِعْنَ أَنْ يُقْبِلْنَ إِلَى مَعْرِفَةِ الْحَقِّ أَبَداً.
\par 8 وَكَمَا قَاوَمَ يَنِّيسُ وَيَمْبِرِيسُ مُوسَى، كَذَلِكَ هَؤُلاَءِ أَيْضاً يُقَاوِمُونَ الْحَقَّ. أُنَاسٌ فَاسِدَةٌ أَذْهَانُهُمْ، وَمِنْ جِهَةِ الإِيمَانِ مَرْفُوضُونَ.
\par 9 لَكِنَّهُمْ لاَ يَتَقَدَّمُونَ أَكْثَرَ، لأَنَّ حُمْقَهُمْ سَيَكُونُ وَاضِحاً لِلْجَمِيعِ، كَمَا كَانَ حُمْقُ ذَيْنِكَ أَيْضاً.
\par 10 وَأَمَّا أَنْتَ فَقَدْ تَبِعْتَ تَعْلِيمِي، وَسِيرَتِي، وَقَصْدِي، وَإِيمَانِي، وَأَنَاتِي، وَمَحَبَّتِي، وَصَبْرِي،
\par 11 وَاضْطِهَادَاتِي، وَآلاَمِي، مِثْلَ مَا أَصَابَنِي فِي أَنْطَاكِيَةَ وَإِيقُونِيَّةَ وَلِسْتِرَةَ. أَيَّةَ اضْطِهَادَاتٍ احْتَمَلْتُ! وَمِنَ الْجَمِيعِ أَنْقَذَنِي الرَّبُّ.
\par 12 وَجَمِيعُ الَّذِينَ يُرِيدُونَ أَنْ يَعِيشُوا بِالتَّقْوَى فِي الْمَسِيحِ يَسُوعَ يُضْطَهَدُونَ.
\par 13 وَلَكِنَّ النَّاسَ الأَشْرَارَ الْمُزَوِّرِينَ سَيَتَقَدَّمُونَ إِلَى أَرْدَأَ، مُضِلِّينَ وَمُضَلِّينَ.
\par 14 وَأَمَّا أَنْتَ فَاثْبُتْ عَلَى مَا تَعَلَّمْتَ وَأَيْقَنْتَ، عَارِفاً مِمَّنْ تَعَلَّمْتَ.
\par 15 وَأَنَّكَ مُنْذُ الطُّفُولِيَّةِ تَعْرِفُ الْكُتُبَ الْمُقَدَّسَةَ، الْقَادِرَةَ أَنْ تُحَكِّمَكَ لِلْخَلاَصِ، بِالإِيمَانِ الَّذِي فِي الْمَسِيحِ يَسُوعَ.
\par 16 كُلُّ الْكِتَابِ هُوَ مُوحىً بِهِ مِنَ اللهِ، وَنَافِعٌ لِلتَّعْلِيمِ وَالتَّوْبِيخِ، لِلتَّقْوِيمِ وَالتَّأْدِيبِ الَّذِي فِي الْبِرِّ،
\par 17 لِكَيْ يَكُونَ إِنْسَانُ اللهِ كَامِلاً، مُتَأَهِّباً لِكُلِّ عَمَلٍ صَالِحٍ.

\chapter{4}

\par 1 أَنَا أُنَاشِدُكَ إِذاً أَمَامَ اللهِ وَالرَّبِّ يَسُوعَ الْمَسِيحِ، الْعَتِيدِ أَنْ يَدِينَ الأَحْيَاءَ وَالأَمْوَاتَ، عِنْدَ ظُهُورِهِ وَمَلَكُوتِهِ:
\par 2 اكْرِزْ بِالْكَلِمَةِ. اعْكُفْ عَلَى ذَلِكَ فِي وَقْتٍ مُنَاسِبٍ وَغَيْرِ مُنَاسِبٍ. وَبِّخِ، انْتَهِرْ، عِظْ بِكُلِّ أَنَاةٍ وَتَعْلِيمٍ.
\par 3 لأَنَّهُ سَيَكُونُ وَقْتٌ لاَ يَحْتَمِلُونَ فِيهِ التَّعْلِيمَ الصَّحِيحَ، بَلْ حَسَبَ شَهَوَاتِهِمُ الْخَاصَّةِ يَجْمَعُونَ لَهُمْ مُعَلِّمِينَ مُسْتَحِكَّةً مَسَامِعُهُمْ،
\par 4 فَيَصْرِفُونَ مَسَامِعَهُمْ عَنِ الْحَقِّ، وَيَنْحَرِفُونَ إِلَى الْخُرَافَاتِ.
\par 5 وَأَمَّا أَنْتَ فَاصْحُ فِي كُلِّ شَيْءٍ. احْتَمِلِ الْمَشَقَّاتِ. اعْمَلْ عَمَلَ الْمُبَشِّرِ. تَمِّمْ خِدْمَتَكَ.
\par 6 فَإِنِّي أَنَا الآنَ أُسْكَبُ سَكِيباً، وَوَقْتُ انْحِلاَلِي قَدْ حَضَرَ.
\par 7 قَدْ جَاهَدْتُ الْجِهَادَ الْحَسَنَ، أَكْمَلْتُ السَّعْيَ، حَفِظْتُ الإِيمَانَ،
\par 8 وَأَخِيراً قَدْ وُضِعَ لِي إِكْلِيلُ الْبِرِّ، الَّذِي يَهَبُهُ لِي فِي ذَلِكَ الْيَوْمِ الرَّبُّ الدَّيَّانُ الْعَادِلُ، وَلَيْسَ لِي فَقَطْ، بَلْ لِجَمِيعِ الَّذِينَ يُحِبُّونَ ظُهُورَهُ أَيْضاً.
\par 9 بَادِرْ أَنْ تَجِيءَ إِلَيَّ سَرِيعاً،
\par 10 لأَنَّ دِيمَاسَ قَدْ تَرَكَنِي إِذْ أَحَبَّ الْعَالَمَ الْحَاضِرَ وَذَهَبَ إِلَى تَسَالُونِيكِي، وَكِرِيسْكِيسَ إِلَى غَلاَطِيَّةَ، وَتِيطُسَ إِلَى دَلْمَاطِيَّةَ.
\par 11 لُوقَا وَحْدَهُ مَعِي. خُذْ مَرْقُسَ وَأَحْضِرْهُ مَعَكَ لأَنَّهُ نَافِعٌ لِي لِلْخِدْمَةِ.
\par 12 أَمَّا تِيخِيكُسُ فَقَدْ أَرْسَلْتُهُ إِلَى أَفَسُسَ.
\par 13 اَلرِّدَاءَ الَّذِي تَرَكْتُهُ فِي تَرُواسَ عِنْدَ كَارْبُسَ أَحْضِرْهُ مَتَى جِئْتَ، وَالْكُتُبَ أَيْضاً وَلاَ سِيَّمَا الرُّقُوقَ.
\par 14 إِسْكَنْدَرُ النَّحَّاسُ أَظْهَرَ لِي شُرُوراً كَثِيرَةً. لِيُجَازِهِ الرَّبُّ حَسَبَ أَعْمَالِهِ.
\par 15 فَاحْتَفِظْ مِنْهُ أَنْتَ أَيْضاً لأَنَّهُ قَاوَمَ أَقْوَالَنَا جِدّاً.
\par 16 فِي احْتِجَاجِي الأَوَّلِ لَمْ يَحْضُرْ أَحَدٌ مَعِي، بَلِ الْجَمِيعُ تَرَكُونِي. لاَ يُحْسَبْ عَلَيْهِمْ.
\par 17 وَلَكِنَّ الرَّبَّ وَقَفَ مَعِي وَقَوَّانِي، لِكَيْ تُتَمَّ بِي الْكِرَازَةُ، وَيَسْمَعَ جَمِيعُ الأُمَمِ، فَأُنْقِذْتُ مِنْ فَمِ الأَسَدِ.
\par 18 وَسَيُنْقِذُنِي الرَّبُّ مِنْ كُلِّ عَمَلٍ رَدِيءٍ وَيُخَلِّصُنِي لِمَلَكُوتِهِ السَّمَاوِيِّ. الَّذِي لَهُ الْمَجْدُ إِلَى دَهْرِ الدُّهُورِ. آمِينَ.
\par 19 سَلِّمْ عَلَى فِرِسْكَا وَأَكِيلاَ وَبَيْتِ أُنِيسِيفُورُسَ.
\par 20 أَرَاسْتُسُ بَقِيَ فِي كُورِنْثُوسَ. وَأَمَّا تُرُوفِيمُسُ فَتَرَكْتُهُ فِي مِيلِيتُسَ مَرِيضاً.
\par 21 بَادِرْ أَنْ تَجِيءَ قَبْلَ الشِّتَاءِ. يُسَلِّمُ عَلَيْكَ أَفْبُولُسُ وَبُودِيسُ و

\end{document}