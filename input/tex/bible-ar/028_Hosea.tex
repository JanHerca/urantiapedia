\begin{document}

\title{هوشع}


\chapter{1}

\par 1 قَوْلُ الرَّبِّ الَّذِي صَارَ إِلَى هُوشَعَ بْنِ بِئِيرِي فِي أَيَّامِ عُزِّيَّا وَيُوثَامَ وَآحَازَ وَحَزَقِيَّا مُلُوكِ يَهُوذَا وَفِي أَيَّامِ يَرُبْعَامَ بْنِ يُوآشَ مَلِكِ إِسْرَائِيلَ:
\par 2 أَوَّّلَ مَا كَلَّمَ الرَّبُّ هُوشَعَ قَالَ الرَّبُّ لِهُوشَعَ: «اذْهَبْ خُذْ لِنَفْسِكَ امْرَأَةَ زِنًى وَأَوْلاَدَ زِنًى لأَنَّ الأَرْضَ قَدْ زَنَتْ زِنًى تَارِكَةً الرَّبَّ!».
\par 3 فَذَهَبَ وَأَخَذَ جُومَرَ بِنْتَ دِبْلاَيِمَ فَحَبِلَتْ وَوَلَدَتْ لَهُ ابْناً.
\par 4 فَقَالَ لَهُ الرَّبُّ: «ادْعُ اسْمَهُ يَزْرَعِيلَ لأَنَّنِي بَعْدَ قَلِيلٍ أُعَاقِبُ بَيْتَ يَاهُو عَلَى دَمِ يَزْرَعِيلَ وَأُبِيدُ مَمْلَكَةَ بَيْتِ إِسْرَائِيلَ.
\par 5 وَيَكُونُ فِي ذَلِكَ الْيَوْمِ أَنِّي أَكْسِرُ قَوْسَ إِسْرَائِيلَ فِي وَادِي يَزْرَعِيلَ».
\par 6 ثُمَّ حَبِلَتْ أَيْضاً وَوَلَدَتْ بِنْتاً فَقَالَ لَهُ: «ادْعُ اسْمَهَا لُورُحَامَةَ لأَنِّي لاَ أَعُودُ أَرْحَمُ بَيْتَ إِسْرَائِيلَ أَيْضاً بَلْ أَنْزِعُهُمْ نَزْعاً.
\par 7 وَأَمَّا بَيْتُ يَهُوذَا فَأَرْحَمُهُمْ وَأُخَلِّصُهُمْ بِالرَّبِّ إِلَهِهِمْ وَلاَ أُخَلِّصُهُمْ بِقَوْسٍ وَبِسَيْفٍ وَبِحَرْبٍ وَبِخَيْلٍ وَبِفُرْسَانٍ».
\par 8 ثُمَّ فَطَمَتْ لُورُحَامَةَ وَحَبِلَتْ فَوَلَدَتِ ابْناً.
\par 9 فَقَالَ: «ادْعُ اسْمَهُ لُوعَمِّي لأَنَّكُمْ لَسْتُمْ شَعْبِي وَأَنَا لاَ أَكُونُ لَكُمْ.
\par 10 لَكِنْ يَكُونُ عَدَدُ بَنِي إِسْرَائِيلَ كَرَمْلِ الْبَحْرِ الَّذِي لاَ يُكَالُ وَلاَ يُعَدُّ وَيَكُونُ عِوَضاً عَنْ أَنْ يُقَالَ لَهُمْ: لَسْتُمْ شَعْبِي يُقَالُ لَهُمْ: أَبْنَاءُ اللَّهِ الْحَيِّ.
\par 11 وَيُجْمَعُ بَنُو يَهُوذَا وَبَنُو إِسْرَائِيلَ مَعاً وَيَجْعَلُونَ لأَنْفُسِهِمْ رَأْساً وَاحِداً وَيَصْعَدُونَ مِنَ الأَرْضِ لأَنَّ يَوْمَ يَزْرَعِيلَ عَظِيمٌ».

\chapter{2}

\par 1 «قُولُوا لإِخْوَتِكُمْ «عَمِّي» وَلأَخَوَاتِكُمْ «رُحَامَةَ».
\par 2 حَاكِمُوا أُمَّكُمْ حَاكِمُوا لأَنَّهَا لَيْسَتِ امْرَأَتِي وَأَنَا لَسْتُ رَجُلَهَا لِتَعْزِلَ زِنَاهَا عَنْ وَجْهِهَا وَفِسْقَهَا مِنْ بَيْنِ ثَدْيَيْهَا
\par 3 لِئَلاَّ أُجَرِّدَهَا عُرْيَانَةً وَأَوْقِفَهَا كَيَوْمِ وِلاَدَتِهَا وَأَجْعَلَهَا كَقَفْرٍ وَأُصَيِّرَهَا كَأَرْضٍ يَابِسَةٍ وَأُمِيتَهَا بِالْعَطَشِ.
\par 4 وَلاَ أَرْحَمُ أَوْلاَدَهَا لأَنَّهُمْ أَوْلاَدُ زِنًى.
\par 5 «لأَنَّ أُمَّهُمْ قَدْ زَنَتِ. الَّتِي حَبِلَتْ بِهِمْ صَنَعَتْ خِزْياً. لأَنَّهَا قَالَتْ: أَذْهَبُ وَرَاءَ مُحِبِّيَّ الَّذِينَ يُعْطُونَ خُبْزِي وَمَائِي صُوفِي وَكَتَّانِي زَيْتِي وَأَشْرِبَتِي.
\par 6 لِذَلِكَ هَئَنَذَا أُسَيِّجُ طَرِيقَكِ بِالشَّوْكِ وَأَبْنِي حَائِطَهَا حَتَّى لاَ تَجِدَ مَسَالِكَهَا.
\par 7 فَتَتْبَعُ مُحِبِّيهَا وَلاَ تُدْرِكُهُمْ وَتُفَتِّشُ عَلَيْهِمْ وَلاَ تَجِدُهُمْ. فَتَقُولُ: أَذْهَبُ وَأَرْجِعُ إِلَى رَجُلِي الأَوَّّلِ لأَنَّهُ حِينَئِذٍ كَانَ خَيْرٌ لِي مِنَ الآنَ.
\par 8 «وَهِيَ لَمْ تَعْرِفْ أَنِّي أَنَا أَعْطَيْتُهَا الْقَمْحَ وَالْمِسْطَارَ وَالزَّيْتَ وَكَثَّرْتُ لَهَا فِضَّةً وَذَهَباً جَعَلُوهُ لِبَعْلٍ.
\par 9 لِذَلِكَ أَرْجِعُ وَآخُذُ قَمْحِي فِي حِينِهِ وَمِسْطَارِي فِي وَقْتِهِ وَأَنْزِعُ صُوفِي وَكَتَّانِي اللَّذَيْنِ لِسَتْرِ عَوْرَتِهَا.
\par 10 وَالآنَ أَكْشِفُ عَوْرَتَهَا أَمَامَ عُيُونِ مُحِبِّيهَا وَلاَ يُنْقِذُهَا أَحَدٌ مِنْ يَدِي.
\par 11 وَأُبَطِّلُ كُلَّ أَفْرَاحِهَا: أَعْيَادَهَا وَرُؤُوسَ شُهُورِهَا وَسُبُوتَهَا وَجَمِيعَ مَوَاسِمِهَا.
\par 12 وَأُخَرِّبُ كَرْمَهَا وَتِينَهَا اللَّذَيْنِ قَالَتْ: هُمَا أُجْرَتِي الَّتِي أَعْطَانِيهَا مُحِبِّيَّ وَأَجْعَلُهُمَا وَعْراً فَيَأْكُلُهُمَا حَيَوَانُ الْبَرِّيَّةِ.
\par 13 وَأُعَاقِبُهَا عَلَى أَيَّامِ بَعْلِيمَ الَّتِي فِيهَا كَانَتْ تُبَخِّرُ لَهُمْ وَتَتَزَيَّنُ بِخَزَائِمِهَا وَحُلِيِّهَا وَتَذْهَبُ وَرَاءَ مُحِبِّيهَا وَتَنْسَانِي أَنَا يَقُولُ الرَّبُّ.
\par 14 «لَكِنْ هَئَنَذَا أَتَمَلَّقُهَا وَأَذْهَبُ بِهَا إِلَى الْبَرِّيَّةِ وَأُلاَطِفُهَا
\par 15 وَأُعْطِيهَا كُرُومَهَا مِنْ هُنَاكَ وَوَادِي عَخُورَ بَاباً لِلرَّجَاءِ. وَهِيَ تُغَنِّي هُنَاكَ كَأَيَّامِ صِبَاهَا وَكَيَوْمِ صُعُودِهَا مِنْ أَرْضِ مِصْرَ.
\par 16 وَيَكُونُ فِي ذَلِكَ الْيَوْمِ يَقُولُ الرَّبُّ أَنَّكِ تَدْعِينَنِي «رَجُلِي» وَلاَ تَدْعِينَنِي بَعْدُ «بَعْلِي».
\par 17 وَأَنْزِعُ أَسْمَاءَ الْبَعْلِيمِ مِنْ فَمِهَا فَلاَ تُذْكَرُ أَيْضاً بِأَسْمَائِهَا.
\par 18 وَأَقْطَعُ لَهُمْ عَهْداً فِي ذَلِكَ الْيَوْمِ مَعَ حَيَوَانِ الْبَرِّيَّةِ وَطُيُورِ السَّمَاءِ وَدَبَّابَاتِ الأَرْضِ وَأَكْسِرُ الْقَوْسَ وَالسَّيْفَ وَالْحَرْبَ مِنَ الأَرْضِ وَأَجْعَلُهُمْ يَضْطَجِعُونَ آمِنِينَ.
\par 19 وَأَخْطُبُكِ لِنَفْسِي إِلَى الأَبَدِ. وَأَخْطُبُكِ لِنَفْسِي بِالْعَدْلِ وَالْحَقِّ وَالإِحْسَانِ وَالْمَرَاحِمِ.
\par 20 أَخْطُبُكِ لِنَفْسِي بِالأَمَانَةِ فَتَعْرِفِينَ الرَّبَّ.
\par 21 وَيَكُونُ فِي ذَلِكَ الْيَوْمِ أَنِّي أَسْتَجِيبُ يَقُولُ الرَّبُّ أَسْتَجِيبُ السَّمَاوَاتِ وَهِيَ تَسْتَجِيبُ الأَرْضَ
\par 22 وَالأَرْضُ تَسْتَجِيبُ الْقَمْحَ وَالْمِسْطَارَ وَالزَّيْتَ وَهِيَ تَسْتَجِيبُ يَزْرَعِيلَ.
\par 23 وَأَزْرَعُهَا لِنَفْسِي فِي الأَرْضِ وَأَرْحَمُ لُورُحَامَةَ وَأَقُولُ لِلُوعَمِّي: أَنْتَ شَعْبِي وَهُوَ يَقُولُ: أَنْتَ إِلَهِي».

\chapter{3}

\par 1 وَقَالَ الرَّبُّ لِي: «اذْهَبْ أَيْضاً أَحْبِبِ امْرَأَةً حَبِيبَةَ صَاحِبٍ وَزَانِيَةً كَمَحَبَّةِ الرَّبِّ لِبَنِي إِسْرَائِيلَ وَهُمْ مُلْتَفِتُونَ إِلَى آلِهَةٍ أُخْرَى وَمُحِبُّونَ لأَقْرَاصِ الزَّبِيبِ».
\par 2 فَاشْتَرَيْتُهَا لِنَفْسِي بِخَمْسَةَ عَشَرَ شَاقِلَ فِضَّةٍ وَبِحُومَرَ وَلَثَكِ شَعِيرٍ.
\par 3 وَقُلْتُ لَهَا: «تَقْعُدِينَ أَيَّاماً كَثِيرَةً لاَ تَزْنِي وَلاَ تَكُونِي لِرَجُلٍ وَأَنَا كَذَلِكَ لَكِ».
\par 4 لأَنَّ بَنِي إِسْرَائِيلَ سَيَقْعُدُونَ أَيَّاماً كَثِيرَةً بِلاَ مَلِكٍ وَبِلاَ رَئِيسٍ وَبِلاَ ذَبِيحَةٍ وَبِلاَ تِمْثَالٍ وَبِلاَ أَفُودٍ وَتَرَافِيمَ.
\par 5 بَعْدَ ذَلِكَ يَعُودُ بَنُو إِسْرَائِيلَ وَيَطْلُبُونَ الرَّبَّ إِلَهَهُمْ وَدَاوُدَ مَلِكَهُمْ وَيَفْزَعُونَ إِلَى الرَّبِّ وَإِلَى جُودِهِ فِي آخِرِ الأَيَّامِ».

\chapter{4}

\par 1 اِسْمَعُوا قَوْلَ الرَّبِّ يَا بَنِي إِسْرَائِيلَ: «إِنَّ لِلرَّبِّ مُحَاكَمَةً مَعَ سُكَّانِ الأَرْضِ لأَنَّهُ لاَ أَمَانَةَ وَلاَ إِحْسَانَ وَلاَ مَعْرِفَةَ اللَّهِ فِي الأَرْضِ.
\par 2 لَعْنٌ وَكَذِبٌ وَقَتْلٌ وَسِرْقَةٌ وَفِسْقٌ. يَعْتَنِفُونَ وَدِمَاءٌ تَلْحَقُ دِمَاءً.
\par 3 لِذَلِكَ تَنُوحُ الأَرْضُ وَيَذْبُلُ كُلُّ مَنْ يَسْكُنُ فِيهَا مَعَ حَيَوَانِ الْبَرِّيَّةِ وَطُيُورِ السَّمَاءِ وَأَسْمَاكِ الْبَحْرِ أَيْضاً تَنْتَزِعُ.
\par 4 «وَلَكِنْ لاَ يُحَاكِمْ أَحَدٌ وَلاَ يُعَاتِبْ أَحَدٌ. وَشَعْبُكَ كَمَنْ يُخَاصِمُ كَاهِناً.
\par 5 فَتَتَعَثَّرُ فِي النَّهَارِ وَيَتَعَثَّرُ أَيْضاً النَّبِيُّ مَعَكَ فِي اللَّيْلِ وَأَنَا أَخْرِبُ أُمَّكَ.
\par 6 قَدْ هَلَكَ شَعْبِي مِنْ عَدَمِ الْمَعْرِفَةِ. لأَنَّكَ أَنْتَ رَفَضْتَ الْمَعْرِفَةَ أَرْفُضُكَ أَنَا حَتَّى لاَ تَكْهَنَ لِي. وَلأَنَّكَ نَسِيتَ شَرِيعَةَ إِلَهِكَ أَنْسَى أَنَا أَيْضاً بَنِيكَ.
\par 7 عَلَى حَسْبَمَا كَثُرُوا هَكَذَا أَخْطَأُوا إِلَيَّ فَأُبْدِلُ كَرَامَتَهُمْ بِهَوَانٍ.
\par 8 يَأْكُلُونَ خَطِيَّةَ شَعْبِي وَإِلَى إِثْمِهِمْ يَحْمِلُونَ نُفُوسَهُمْ.
\par 9 فَيَكُونُ كَمَا الشَّعْبُ هَكَذَا الْكَاهِنُ. وَأُعَاقِبُهُمْ عَلَى طُرُقِهِمْ وَأَرُدُّ أَعْمَالَهُمْ عَلَيْهِمْ.
\par 10 فَيَأْكُلُونَ وَلاَ يَشْبَعُونَ وَيَزْنُونَ وَلاَ يَكْثُرُونَ لأَنَّهُمْ قَدْ تَرَكُوا عِبَادَةَ الرَّبِّ.
\par 11 «اَلزِّنَى وَالْخَمْرُ وَالسُّلاَفَةُ تَخْلِبُ الْقَلْبَ.
\par 12 شَعْبِي يَسْأَلُ خَشَبَهُ وَعَصَاهُ تُخْبِرُهُ لأَنَّ رُوحَ الزِّنَى قَدْ أَضَلَّهُمْ فَزَنُوا مِنْ تَحْتِ إِلَهِهِمْ.
\par 13 يَذْبَحُونَ عَلَى رُؤُوسِ الْجِبَالِ وَيُبَخِّرُونَ عَلَى التِّلاَلِ تَحْتَ الْبَلُّوطِ وَاللُّبْنَى وَالْبُطْمِ لأَنَّ ظِلَّهَا حَسَنٌ! لِذَلِكَ تَزْنِي بَنَاتُكُمْ وَتَفْسِقُ كَنَّاتُكُمْ.
\par 14 لاَ أُعَاقِبُ بَنَاتِكُمْ لأَنَّهُنَّ يَزْنِينَ وَلاَ كَنَّاتِكُمْ لأَنَّهُنَّ يَفْسِقْنَ. لأَنَّهُمْ يَعْتَزِلُونَ مَعَ الزَّانِيَاتِ وَيَذْبَحُونَ مَعَ النَّاذِرَاتِ الزِّنَى. وَشَعْبٌ لاَ يَعْقِلُ يُصْرَعُ.
\par 15 «إِنْ كُنْتَ أَنْتَ زَانِياً يَا إِسْرَائِيلُ فَلاَ يَأْثَمُ يَهُوذَا. وَلاَ تَأْتُوا إِلَى الْجِلْجَالِ وَلاَ تَصْعَدُوا إِلَى بَيْتِ آوَنَ وَلاَ تَحْلِفُوا: حَيٌّ هُوَ الرَّبُّ.
\par 16 إِنَّهُ قَدْ جَمَحَ إِسْرَائِيلُ كَبَقَرَةٍ جَامِحَةٍ. الآنَ يَرْعَاهُمُ الرَّبُّ كَخَرُوفٍ فِي مَكَانٍ وَاسِعٍ.
\par 17 أَفْرَايِمُ مُوثَقٌ بِالأَصْنَامِ. اتْرُكُوهُ.
\par 18 مَتَى انْتَهَتْ مُنَادَمَتُهُمْ زَنِوا زِنًى. أَحَبَّ مَجَانُّهَا أَحَبُّوا الْهَوَانَ.
\par 19 قَدْ صَرَّتْهَا الرِّيحُ فِي أَجْنِحَتِهَا وَخَجِلُوا مِنْ ذَبَائِحِهِمْ.

\chapter{5}

\par 1 «اِسْمَعُوا هَذَا أَيُّهَا الْكَهَنَةُ وَانْصِتُوا يَا بَيْتَ إِسْرَائِيلَ وَأَصْغُوا يَا بَيْتَ الْمَلِكِ لأَنَّ عَلَيْكُمُ الْقَضَاءَ إِذْ صِرْتُمْ فَخّاً فِي مِصْفَاةَ وَشَبَكَةً مَبْسُوطَةً عَلَى تَابُورَ.
\par 2 وَقَدْ تَوَغَّلُوا فِي ذَبَائِحِ الزَّيَغَانِ فَأَنَا تَأْدِيبٌ لِجَمِيعِهِمْ.
\par 3 أَنَا أَعْرِفُ أَفْرَايِمَ. وَإِسْرَائِيلُ لَيْسَ مَخْفِيّاً عَنِّي. إِنَّكَ الآنَ زَنَيْتَ يَا أَفْرَايِمُ. قَدْ تَنَجَّسَ إِسْرَائِيلُ.
\par 4 أَفْعَالُهُمْ لاَ تَدَعُهُمْ يَرْجِعُونَ إِلَى إِلَهِهِمْ لأَنَّ رُوحَ الزِّنَى فِي بَاطِنِهِمْ وَهُمْ لاَ يَعْرِفُونَ الرَّبَّ.
\par 5 وَقَدْ أُذِلَّتْ عَظَمَةُ إِسْرَائِيلَ فِي وَجْهِهِ فَيَتَعَثَّرُ إِسْرَائِيلُ وَأَفْرَايِمُ فِي إِثْمِهِمَا وَيَتَعَثَّرُ يَهُوذَا أَيْضاً مَعَهُمَا.
\par 6 يَذْهَبُونَ بِغَنَمِهِمْ وَبَقَرِهِمْ لِيَطْلُبُوا الرَّبَّ وَلاَ يَجِدُونَهُ. قَدْ تَنَحَّى عَنْهُمْ.
\par 7 قَدْ غَدَرُوا بِالرَّبِّ. لأَنَّهُمْ وَلَدُوا أَوْلاَداً أَجْنَبِيِّينَ الآنَ يَأْكُلُهُمْ شَهْرٌ مَعَ أَنْصِبَتِهِمْ.
\par 8 «اِضْرِبُوا بِالْبُوقِ فِي جِبْعَةَ بِالْقَرْنِ فِي الرَّامَةِ. اصْرُخُوا فِي بَيْتِ آوَنَ. وَرَاءَكَ يَا بِنْيَامِينُ.
\par 9 يَصِيرُ أَفْرَايِمُ خَرَاباً فِي يَوْمِ التَّأْدِيبِ. فِي أَسْبَاطِ إِسْرَائِيلَ أَعْلَمْتُ الْيَقِينَ.
\par 10 صَارَتْ رُؤَسَاءُ يَهُوذَا كَنَاقِلِي التُّخُومِ. فَأَسْكُبُ عَلَيْهِمْ سَخَطِي كَالْمَاءِ.
\par 11 أَفْرَايِمُ مَظْلُومٌ مَسْحُوقُ الْقَضَاءِ لأَنَّهُ ارْتَضَى أَنْ يَمْضِيَ وَرَاءَ الْوَصِيَّةِ.
\par 12 فَأَنَا لأَفْرَايِمَ كَالْعُثِّ وَلِبَيْتِ يَهُوذَا كَالسُّوسِ.
\par 13 «وَرَأَى أَفْرَايِمُ مَرَضَهُ وَيَهُوذَا جُرْحَهُ فَمَضَى أَفْرَايِمُ إِلَى أَشُّورَ وَأَرْسَلَ إِلَى مَلِكٍ عَدُوٍّّ. وَلَكِنَّهُ لاَ يَسْتَطِيعُ أَنْ يَشْفِيَكُمْ وَلاَ أَنْ يُزِيلَ مِنْكُمُ الْجُرْحَ.
\par 14 لأَنِّي لأَفْرَايِمَ كَالأَسَدِ وَلِبَيْتِ يَهُوذَا كَشِبْلِ الأَسَدِ. فَإِنِّي أَنَا أَفْتَرِسُ وَأَمْضِي وَآخُذُ وَلاَ مُنْقِذٌ.
\par 15 أَذْهَبُ وَأَرْجِعُ إِلَى مَكَانِي حَتَّى يُجَازَوْا وَيَطْلُبُوا وَجْهِي. فِي ضِيقِهِمْ يُبَكِّرُونَ إِلَيَّ».

\chapter{6}

\par 1 هَلُمَّ نَرْجِعُ إِلَى الرَّبِّ لأَنَّهُ هُوَ افْتَرَسَ فَيَشْفِينَا ضَرَبَ فَيَجْبِرُنَا.
\par 2 يُحْيِينَا بَعْدَ يَوْمَيْنِ. فِي الْيَوْمِ الثَّالِثِ يُقِيمُنَا فَنَحْيَا أَمَامَهُ.
\par 3 لِنَعْرِفْ فَلْنَتَتَبَّعْ لِنَعْرِفَ الرَّبَّ. خُرُوجُهُ يَقِينٌ كَالْفَجْرِ. يَأْتِي إِلَيْنَا كَالْمَطَرِ. كَمَطَرٍ مُتَأَخِّرٍ يَسْقِي الأَرْضَ.
\par 4 «مَاذَا أَصْنَعُ بِكَ يَا أَفْرَايِمُ؟ مَاذَا أَصْنَعُ بِكَ يَا يَهُوذَا؟ فَإِنَّ إِحْسَانَكُمْ كَسَحَابِ الصُّبْحِ وَكَالنَّدَى الْمَاضِي بَاكِراً.
\par 5 لِذَلِكَ أَقْرِضُهُمْ بِالأَنْبِيَاءِ أَقْتُلُهُمْ بِأَقْوَالِ فَمِي. وَالْقَضَاءُ عَلَيْكَ كَنُورٍ قَدْ خَرَجَ.
\par 6 «إِنِّي أُرِيدُ رَحْمَةً لاَ ذَبِيحَةً وَمَعْرِفَةَ اللَّهِ أَكْثَرَ مِنْ مُحْرَقَاتٍ.
\par 7 وَلَكِنَّهُمْ كَآدَمَ تَعَدَّوُا الْعَهْدَ. هُنَاكَ غَدَرُوا بِي.
\par 8 جِلْعَادُ قَرْيَةُ فَاعِلِي الإِثْمِ مَدُوسَةٌ بِالدَّمِ.
\par 9 وَكَمَا يُكْمُنُ لُصُوصٌ لإِنْسَانٍ كَذَلِكَ زُمْرَةُ الْكَهَنَةِ فِي الطَّرِيقِ يَقْتُلُونَ نَحْوَ شَكِيمَ. إِنَّهُمْ قَدْ صَنَعُوا فَاحِشَةً.
\par 10 في بَيْتِ إِسْرَائِيلَ رَأَيْتُ أَمْراً فَظِيعاً. هُنَاكَ زَنَى أَفْرَايِمُ. تَنَجَّسَ إِسْرَائِيلُ.
\par 11 وَأَنْتَ أَيْضاً يَا يَهُوذَا قَدْ أُعِدَّ لَكَ حَصَادٌ عِنْدَمَا أَرُدُّ سَبْيَ شَعْبِي.

\chapter{7}

\par 1 «حِينَمَا كُنْتُ أَشْفِي إِسْرَائِيلَ أُعْلِنَ إِثْمُ أَفْرَايِمَ وَشُرُورُ السَّامِرَةِ فَإِنَّهُمْ قَدْ صَنَعُوا غِشّاً. السَّارِقُ دَخَلَ وَالْغُزَاةُ نَهَبُوا فِي الْخَارِجِ.
\par 2 وَلاَ يَفْتَكِرُونَ فِي قُلُوبِهِمْ أَنِّي قَدْ تَذَكَّرْتُ كُلَّ شَرِّهِمْ. الآنَ قَدْ أَحَاطَتْ بِهِمْ أَفْعَالُهُمْ. صَارَتْ أَمَامَ وَجْهِي.
\par 3 «بِشَرِّهِمْ يُفَرِّحُونَ الْمَلِكَ وَبِكَذِبِهِمِ الرُّؤَسَاءَ.
\par 4 كُلُّهُمْ فَاسِقُونَ كَتَنُّورٍ مُحْمًى مِنَ الْخَبَّازِ. يُبَطِّلُ الإِيقَادَ مِنْ وَقْتِمَا يَعْجِنُ الْعَجِينَ إِلَى أَنْ يَخْتَمِرَ.
\par 5 يَوْمُ مَلِكِنَا يَمْرَضُ الرُّؤَسَاءُ مِنْ سَوْرَةِ الْخَمْرِ. يَبْسُطُ يَدَهُ مَعَ الْمُسْتَهْزِئِينَ.
\par 6 لأَنَّهُمْ يُقَرِّبُونَ قُلُوبَهُمْ فِي مَكِيدَتِهِمْ كَالتَّنُّورِ. كُلَّ اللَّيْلِ يَنَامُ خَبَّازُهُمْ وَفِي الصَّبَاحِ يَكُونُ مُحْمًى كَنَارٍ مُلْتَهِبَةٍ.
\par 7 كُلُّهُمْ حَامُونَ كَالتَّنُّورِ وَأَكَلُوا قُضَاتَهُمْ. جَمِيعُ مُلُوكِهِمْ سَقَطُوا. لَيْسَ بَيْنَهُمْ مَنْ يَدْعُو إِلَيَّ.
\par 8 «أَفْرَايِمُ يَخْتَلِطُ بِالشُّعُوبِ. أَفْرَايِمُ صَارَ خُبْزَ مَلَّةٍ لَمْ يُقْلَبْ.
\par 9 أَكَلَ الْغُرَبَاءُ ثَرْوَتَهُ وَهُوَ لاَ يَعْرِفُ وَقَدْ رُشَّ عَلَيْهِ الشَّيْبُ وَهُوَ لاَ يَعْرِفُ.
\par 10 وَقَدْ أُذِلَّتْ عَظَمَةُ إِسْرَائِيلَ فِي وَجْهِهِ وَهُمْ لاَ يَرْجِعُونَ إِلَى الرَّبِّ إِلَهِهِمْ وَلاَ يَطْلُبُونَهُ مَعَ كُلِّ هَذَا.
\par 11 وَصَارَ أَفْرَايِمُ كَحَمَامَةٍ رَعْنَاءَ بِلاَ قَلْبٍ. يَدْعُونَ مِصْرَ. يَمْضُونَ إِلَى أَشُّورَ.
\par 12 عِنْدَمَا يَمْضُونَ أَبْسُطُ عَلَيْهِمْ شَبَكَتِي. أُلْقِيهِمْ كَطُيُورِ السَّمَاءِ. أُؤَدِّبُهُمْ بِحَسَبِ خَبَرِ جَمَاعَتِهِمْ.
\par 13 «وَيْلٌ لَهُمْ لأَنَّهُمْ هَرَبُوا عَنِّي. تَبّاً لَهُمْ لأَنَّهُمْ أَذْنَبُوا إِلَيَّ. أَنَا أَفْدِيهِمْ وَهُمْ تَكَلَّمُوا عَلَيَّ بِكَذِبٍ.
\par 14 وَلاَ يَصْرُخُونَ إِلَيَّ بِقُلُوبِهِمْ حِينَمَا يُوَلْوِلُونَ عَلَى مَضَاجِعِهِمْ. يَتَجَمَّعُونَ لأَجْلِ الْقَمْحِ وَالْخَمْرِ وَيَرْتَدُّونَ عَنِّي.
\par 15 وَأَنَا أَنْذَرْتُهُمْ وَشَدَّدْتُ أَذْرُعَهُمْ وَهُمْ يُفَكِّرُونَ عَلَيَّ بِالشَّرِّ.
\par 16 يَرْجِعُونَ لَيْسَ إِلَى الْعَلِيِّ. قَدْ صَارُوا كَقَوْسٍ مُخْطِئَةٍ. يَسْقُطُ رُؤَسَاؤُهُمْ بِالسَّيْفِ مِنْ أَجْلِ سَخَطِ أَلْسِنَتِهِمْ. هَذَا هُزْؤُهُمْ فِي أَرْضِ مِصْرَ».

\chapter{8}

\par 1 «إِلَى فَمِكَ بِالْبُوقِ! كَالنَّسْرِ عَلَى بَيْتِ الرَّبِّ. لأَنَّهُمْ قَدْ تَجَاوَزُوا عَهْدِي وَتَعَدُّوا عَلَى شَرِيعَتِي.
\par 2 إِلَيَّ يَصْرُخُونَ: يَا إِلَهِي نَعْرِفُكَ نَحْنُ إِسْرَائِيلَ.
\par 3 «قَدْ كَرِهَ إِسْرَائِيلُ الصَّلاَحَ فَيَتْبَعُهُ الْعَدُوُّّ.
\par 4 هُمْ أَقَامُوا مُلُوكاً وَلَيْسَ مِنِّي. أَقَامُوا رُؤَسَاءَ وَأَنَا لَمْ أَعْرِفْ. صَنَعُوا لأَنْفُسِهِمْ مِنْ فِضَّتِهِمْ وَذَهَبِهِمْ أَصْنَاماً لِيَنْقَرِضُوا.
\par 5 قَدْ زَنِخَ عِجْلُكِ يَا سَامِرَةُ. حَمِيَ غَضَبِي عَلَيْهِمْ. إِلَى مَتَى لاَ يَسْتَطِيعُونَ النَّقَاوَةَ!
\par 6 إِنَّهُ هُوَ أَيْضاً مِنْ إِسْرَائِيلَ. صَنَعَهُ الصَّانِعُ وَلَيْسَ هُوَ إِلَهاً. إِنَّ عِجْلَ السَّامِرَةِ يَصِيرُ كِسَراً.
\par 7 «إِنَّهُمْ يَزْرَعُونَ الرِّيحَ وَيَحْصُدُونَ الزَّوْبَعَةَ. زَرْعٌ لَيْسَ لَهُ غَلَّةٌ لاَ يَصْنَعُ دَقِيقاً. وَإِنْ صَنَعَ فَالْغُرَبَاءُ تَبْتَلِعُهُ.
\par 8 قَدِ ابْتُلِعَ إِسْرَائِيلُ. الآنَ صَارُوا بَيْنَ الأُمَمِ كَإِنَاءٍ لاَ مَسَرَّةَ فِيهِ.
\par 9 لأَنَّهُمْ صَعِدُوا إِلَى أَشُّورَ مِثْلَ حِمَارٍ وَحْشِيٍّ مُعْتَزِلٍ بِنَفْسِهِ. اسْتَأْجَرَ أَفْرَايِمُ مُحِبِّينَ.
\par 10 إِنِّي وَإِنْ كَانُوا يَسْتَأْجِرُونَ بَيْنَ الأُمَمِ الآنَ أَجْمَعُهُمْ فَيَنْفَكُّونَ قَلِيلاً مِنْ ثِقْلِ مَلِكِ الرُّؤَسَاءِ.
\par 11 «لأَنَّ أَفْرَايِمَ كَثَّرَ مَذَابِحَ لِلْخَطِيَّةِ صَارَتْ لَهُ الْمَذَابِحُ لِلْخَطِيَّةِ.
\par 12 أَكْتُبُ لَهُ كَثْرَةَ شَرَائِعِي فَهِيَ تُحْسَبُ أَجْنَبِيَّةً.
\par 13 أَمَّا ذَبَائِحُ تَقْدِمَاتِي فَيَذْبَحُونَ لَحْماً وَيَأْكُلُونَ. الرَّبُّ لاَ يَرْتَضِيهَا. الآنَ يَذْكُرُ إِثْمَهُمْ وَيُعَاقِبُ خَطِيَّتَهُمْ. إِنَّهُمْ إِلَى مِصْرَ يَرْجِعُونَ.
\par 14 وَقَدْ نَسِيَ إِسْرَائِيلُ صَانِعَهُ وَبَنَى قُصُوراً وَكَثَّرَ يَهُوذَا مُدُناً حَصِينَةً. لَكِنِّي أُرْسِلُ عَلَى مُدُنِهِ نَاراً فَتَأْكُلُ قُصُورَهُ».

\chapter{9}

\par 1 لاَ تَفْرَحْ يَا إِسْرَائِيلُ طَرَباً كَالشُّعُوبِ لأَنَّكَ قَدْ زَنَيْتَ عَنْ إِلَهِكَ. أَحْبَبْتَ الأُجْرَةَ عَلَى جَمِيعِ بَيَادِرِ الْحِنْطَةِ.
\par 2 لاَ يُطْعِمُهُمُ الْبَيْدَرُ وَالْمِعْصَرَةُ وَيَكْذِبُ عَلَيْهِمِ الْمِسْطَارُ.
\par 3 لاَ يَسْكُنُونَ فِي أَرْضِ الرَّبِّ بَلْ يَرْجِعُ أَفْرَايِمُ إِلَى مِصْرَ وَيَأْكُلُونَ النَّجِسَ فِي أَشُّورَ.
\par 4 لاَ يَسْكُبُونَ لِلرَّبِّ خَمْراً وَلاَ تَسُرُّهُ ذَبَائِحُهُمْ. إِنَّهَا لَهُمْ كَخُبْزِ الْحُزْنِ. كُلُّ مَنْ أَكَلَهُ يَتَنَجَّسُ. إِنَّ خُبْزَهُمْ لِنَفْسِهِمْ. لاَ يَدْخُلُ بَيْتَ الرَّبِّ.
\par 5 مَاذَا تَصْنَعُونَ فِي يَوْمِ الْمَوْسِمِ وَفِي يَوْمِ عِيدِ الرَّبِّ؟
\par 6 إِنَّهُمْ قَدْ ذَهَبُوا مِنَ الْخَرَابِ. تَجْمَعُهُمْ مِصْرُ. تَدْفِنُهُمْ مُوفُ. يَرِثُ الْقَرِيصُ نَفَائِسَ فِضَّتِهِمْ. يَكُونُ الْعَوْسَجُ فِي مَنَازِلِهِمْ.
\par 7 جَاءَتْ أَيَّامُ الْعِقَابِ. جَاءَتْ أَيَّامُ الْجَزَاءِ. سَيَعْرِفُ إِسْرَائِيلُ. النَّبِيُّ أَحْمَقُ. إِنْسَانُ الرُّوحِ مَجْنُونٌ مِنْ كَثْرَةِ إِثْمِكَ وَكَثْرَةِ الْحِقْدِ.
\par 8 أَفْرَايِمُ مُنْتَظَرٌ عِنْدَ إِلَهِي. النَّبِيُّ فَخُّ صَيَّادٍ عَلَى جَمِيعِ طُرُقِهِ. حَِقْدٌ فِي بَيْتِ إِلَهِهِ.
\par 9 قَدْ تَوَغَّلُوا فَسَدُوا كَأَيَّامِ جِبْعَةَ. سَيَذْكُرُ إِثْمَهُمْ. سَيُعَاقِبُ خَطَايَاهُمْ.
\par 10 «وَجَدْتُ إِسْرَائِيلَ كَعِنَبٍ فِي الْبَرِّيَّةِ. رَأَيْتُ آبَاءَكُمْ كَبَاكُورَةٍ عَلَى تِينَةٍ فِي أَوَّّلِهَا. أَمَّا هُمْ فَجَاءُوا إِلَى بَعْلِ فَغُورَ وَنَذَرُوا أَنْفُسَهُمْ لِلْخِزْيِ وَصَارُوا رِجْساً كَمَا أَحَبُّوا.
\par 11 أَفْرَايِمُ تَطِيرُ كَرَامَتُهُمْ كَطَائِرٍ مِنَ الْوِلاَدَةِ وَمِنَ الْبَطْنِ وَمِنَ الْحَبَلِ.
\par 12 وَإِنْ رَبُّوا أَوْلاَدَهُمْ أُثْكِلُهُمْ إِيَّاهُمْ حَتَّى لاَ يَكُونَ إِنْسَانٌ. وَيْلٌ لَهُمْ أَيْضاً مَتَى انْصَرَفْتُ عَنْهُمْ.
\par 13 أَفْرَايِمُ كَمَا أَرَى مِثْلُ صُورٍ مَغْرُوسٌ فِي مَرْعًى وَلَكِنَّ أَفْرَايِمَ سَيُخْرِجُ بَنِيهِ إِلَى الْقَاتِلِ».
\par 14 أَعْطِهِمْ يَا رَبُّ. مَاذَا تُعْطِي؟ أَعْطِهِمْ رَحِماً مُسْقِطاً وَثَدْيَيْنِ يَبِسَيْنِ.
\par 15 «كُلُّ شَرِّهِمْ فِي الْجِلْجَالِ. إِنِّي هُنَاكَ أَبْغَضْتُهُمْ. مِنْ أَجْلِ سُوءِ أَفْعَالِهِمْ أَطْرُدُهُمْ مِنْ بَيْتِي. لاَ أَعُودُ أُحِبُّهُمْ. جَمِيعُ رُؤَسَائِهِمْ مُتَمَرِّدُونَ.
\par 16 أَفْرَايِمُ مَضْرُوبٌ. أَصْلُهُمْ قَدْ جَفَّ. لاَ يَصْنَعُونَ ثَمَراً. وَإِنْ وَلَدُوا أُمِيتُ مُشْتَهَيَاتِ بُطُونِهِمْ».
\par 17 يَرْفُضُهُمْ إِلَهِي لأَنَّهُمْ لَمْ يَسْمَعُوا لَهُ فَيَكُونُونَ تَائِهِينَ بَيْنَ الأُمَمِ.

\chapter{10}

\par 1 إِسْرَائِيلُ جَفْنَةٌ مُمْتَدَّةٌ. يُخْرِجُ ثَمَراً لِنَفْسِهِ. عَلَى حَسَبِ كَثْرَةِ ثَمَرِهِ قَدْ كَثَّرَ الْمَذَابِحَ. عَلَى حَسَبِ جُودَةِ أَرْضِهِ أَجَادَ الأَنْصَابَ.
\par 2 قَدْ قَسَمُوا قُلُوبَهُمْ. الآنَ يُعَاقَبُونَ. هُوَ يُحَطِّمُ مَذَابِحَهُمْ يُخْرِبُ أَنْصَابَهُمْ.
\par 3 إِنَّهُمُ الآنَ يَقُولُونَ: «لاَ مَلِكَ لَنَا لأَنَّنَا لاَ نَخَافُ الرَّبَّ فَالْمَلِكُ مَاذَا يَصْنَعُ بِنَا؟»
\par 4 يَتَكَلَّمُونَ كَلاَماً بِأَقْسَامٍ بَاطِلَةٍ. يَقْطَعُونَ عَهْداً فَيَنْبُتُ الْقَضَاءُ عَلَيْهِمْ كَالْعَلْقَمِ فِي أَتْلاَمِ الْحَقْلِ.
\par 5 عَلَى عُجُولِ بَيْتِ آوَنَ يَخَافُ سُكَّانُ السَّامِرَةِ. إِنَّ شَعْبَهُ يَنُوحُ عَلَيْهِ وَكَهَنَتَهُ عَلَيْهِ يَرْتَعِدُونَ عَلَى مَجْدِهِ لأَنَّهُ انْتَفَى عَنْهُ.
\par 6 وَهُوَ أَيْضاً يُجْلَبُ إِلَى أَشُّورَ هَدِيَّةً لِمَلِكٍ عَدُوٍّّ. يَأْخُذُ أَفْرَايِمُ خِزْياً وَيَخْجَلُ إِسْرَائِيلُ عَلَى رَأْيِهِ.
\par 7 اَلسَّامِرَةُ مَلِكُهَا يَبِيدُ كَغُثَاءٍ عَلَى وَجْهِ الْمَاءِ
\par 8 وَتُخْرَبُ شَوَامِخُ آوَنَ خَطِيَّةُ إِسْرَائِيلَ. يَطْلُعُ الشَّوْكُ وَالْحَسَكُ عَلَى مَذَابِحِهِمْ وَيَقُولُونَ لِلْجِبَالِ: غَطِّينَا وَلِلتِّلاَلِ: اسْقُطِي عَلَيْنَا.
\par 9 «مِنْ أَيَّامِ جِبْعَةَ أَخْطَأْتَ يَا إِسْرَائِيلُ. هُنَاكَ وَقَفُوا. لَمْ تُدْرِكْهُمْ فِي جِبْعَةَ الْحَرْبُ عَلَى بَنِي الإِثْمِ.
\par 10 حِينَمَا أُرِيدُ أُؤَدِّبُهُمْ وَيَجْتَمِعُ عَلَيْهِمْ شُعُوبٌ فِي ارْتِبَاطِهِمْ بِإِثْمَيْهِمْ.
\par 11 وَأَفْرَايِمُ عِجْلَةٌ مُتَمَرِّنَةٌ تُحِبُّ الدِّرَاسَ وَلَكِنِّي أَجْتَازُ عَلَى عُنُقِهَا الْحَسَنِ. أُرْكِبُ عَلَى أَفْرَايِمَ. يَفْلَحُ يَهُوذَا يُمَهِّدُ يَعْقُوبُ!
\par 12 «اِزْرَعُوا لأَنْفُسِكُمْ بِالْبِرِّ. احْصُدُوا بِحَسَبِ الصَّلاَحِ. احْرُثُوا لأَنْفُسِكُمْ حَرْثاً فَإِنَّهُ وَقْتٌ لِطَلَبِ الرَّبِّ حَتَّى يَأْتِيَ وَيُعَلِّمَكُمُ الْبِرَّ.
\par 13 قَدْ حَرَثْتُمُ النِّفَاقَ حَصَدْتُمُ الإِثْمَ أَكَلْتُمْ ثَمَرَ الْكَذِبِ. لأَنَّكَ وَثَقْتَ بِطَرِيقِكَ بِكَثْرَةِ أَبْطَالِكَ.
\par 14 يَقُومُ ضَجِيجٌ فِي شُعُوبِكَ وَتُخْرَبُ جَمِيعُ حُصُونِكَ كَإِخْرَابِ شَلْمَانَ بَيْتَ أَرَبْئِيلَ فِي يَوْمِ الْحَرْبِ. الأُمُّ مَعَ الأَوْلاَدِ حُطِّمَتْ.
\par 15 هَكَذَا تَصْنَعُ بِكُمْ بَيْتُ إِيلَ مِنْ أَجْلِ رَدَاءَةِ شَرِّكُمْ. فِي الصُّبْحِ يَهْلِكُ مَلِكُ إِسْرَائِيلَ هَلاَكاً».

\chapter{11}

\par 1 «لَمَّا كَانَ إِسْرَائِيلُ غُلاَماً أَحْبَبْتُهُ وَمِنْ مِصْرَ دَعَوْتُ ابْنِي.
\par 2 كُلَّ مَا دَعُوهُمْ ذَهَبُوا مِنْ أَمَامِهِمْ يَذْبَحُونَ لِلْبَعْلِيمِ وَيُبَخِّرُونَ لِلتَّمَاثِيلِ الْمَنْحُوتَةِ.
\par 3 وَأَنَا دَرَّجْتُ أَفْرَايِمَ مُمْسِكاً إِيَّاهُمْ بِأَذْرُعِهِمْ فَلَمْ يَعْرِفُوا أَنِّي شَفَيْتُهُمْ.
\par 4 كُنْتُ أَجْذِبُهُمْ بِحِبَالِ الْبَشَرِ بِرُبُطِ الْمَحَبَّةِ وَكُنْتُ لَهُمْ كَمَنْ يَرْفَعُ النِّيرَ عَنْ أَعْنَاقِهِمْ وَمَدَدْتُ إِلَيْهِ مُطْعِماً إِيَّاهُ.
\par 5 «لاَ يَرْجِعُ إِلَى أَرْضِ مِصْرَ بَلْ أَشُّورُ هُوَ مَلِكُهُ. لأَنَّهُمْ أَبُوا أَنْ يَرْجِعُوا
\par 6 يَثُورُ السَّيْفُ فِي مُدُنِهِمْ وَيُتْلِفُ عِصِيَّهَا وَيَأْكُلُهُمْ مِنْ أَجْلِ آرَائِهِمْ.
\par 7 وَشَعْبِي جَانِحُونَ إِلَى الاِرْتِدَادِ عَنِّي فَيَدْعُونَهُمْ إِلَى الْعَلِيِّ وَلاَ أَحَدٌ يَرْفَعُهُ.
\par 8 كَيْفَ أَجْعَلُكَ يَا أَفْرَايِمُ أُصَيِّرُكَ يَا إِسْرَائِيلُ؟! كَيْفَ أَجْعَلُكَ كَأَدَمَةَ أَصْنَعُكَ كَصَبُويِيمَ؟! قَدِ انْقَلَبَ عَلَيَّ قَلْبِي. اضْطَرَمَتْ مَرَاحِمِي جَمِيعاً!
\par 9 «لاَ أُجْرِي حُمُوَّّ غَضَبِي. لاَ أَعُودُ أَخْرِبُ أَفْرَايِمَ لأَنِّي اللَّهُ لاَ إِنْسَانٌ الْقُدُّوسُ فِي وَسَطِكَ فَلاَ آتِي بِسَخَطٍ.
\par 10 «وَرَاءَ الرَّبِّ يَمْشُونَ. كَأَسَدٍ يُزَمْجِرُ. فَإِنَّهُ يُزَمْجِرُ فَيُسْرِعُ الْبَنُونَ مِنَ الْبَحْرِ.
\par 11 يُسْرِعُونَ كَعُصْفُورٍ مِنْ مِصْرَ وَكَحَمَامَةٍ مِنْ أَرْضِ أَشُّورَ فَأُسْكِنُهُمْ فِي بُيُوتِهِمْ يَقُولُ الرَّبُّ.
\par 12 قَدْ أَحَاطَ بِي أَفْرَايِمُ بِالْكَذِبِ وَبَيْتُ إِسْرَائِيلَ بِالْمَكْرِ وَلَمْ يَزَلْ يَهُوذَا شَارِداً عَنِ اللَّهِ وَعَنِ الْقُدُّوسِ الأَمِينِ».

\chapter{12}

\par 1 «أَفْرَايِمُ رَاعِي الرِّيحِ وَتَابِعُ الرِّيحِ الشَّرْقِيَّةِ. كُلَّ يَوْمٍ يُكَثِّرُ الْكَذِبَ وَالاِغْتِصَابَ وَيَقْطَعُونَ مَعَ أَشُّورَ عَهْداً وَالزَّيْتُ إِلَى مِصْرَ يُجْلَبُ.
\par 2 فَلِلرَّبِّ خِصَامٌ مَعَ يَهُوذَا وَهُوَ مُزْمِعٌ أَنْ يُعَاقِبَ يَعْقُوبَ بِحَسَبِ طُرُقِهِ. بِحَسَبِ أَفْعَالِهِ يَرُدُّ عَلَيْهِ.
\par 3 «فِي الْبَطْنِ قَبَضَ بِعَقِبِ أَخِيهِ وَبِقُوَّّتِهِ جَاهَدَ مَعَ اللَّهِ.
\par 4 جَاهَدَ مَعَ الْمَلاَكِ وَغَلَبَ. بَكَى وَاسْتَرْحَمَهُ. وَجَدَهُ فِي بَيْتِ إِيلَ وَهُنَاكَ تَكَلَّمَ مَعَنَا.
\par 5 وَالرَّبُّ إِلَهُ الْجُنُودِ يَهْوَهُ اسْمُهُ.
\par 6 وَأَنْتَ فَارْجِعْ إِلَى إِلَهِكَ. احْفَظِ الرَّحْمَةَ وَالْحَقَّ وَانْتَظِرْ إِلَهَكَ دَائِماً.
\par 7 مِثْلُ الْكَنْعَانِيِّ فِي يَدِهِ مَوَازِينُ الْغِشِّ. يُحِبُّ أَنْ يَظْلِمَ.
\par 8 فَقَالَ أَفْرَايِمُ: إِنِّي صِرْتُ غَنِيّاً. وَجَدْتُ لِنَفْسِي ثَرْوَةً. جَمِيعُ أَتْعَابِي لاَ يَجِدُونَ لِي فِيهَا ذَنْباً هُوَ خَطِيَّةٌ.
\par 9 وَأَنَا الرَّبُّ إِلَهُكَ مِنْ أَرْضِ مِصْرَ حَتَّى أُسْكِنَكَ الْخِيَامَ كَأَيَّامِ الْمَوْسِمِ.
\par 10 وَكَلَّمْتُ الأَنْبِيَاءَ وَكَثَّرْتُ الرُّؤَى وَبِيَدِ الأَنْبِيَاءِ مَثَّلْتُ أَمْثَالاً».
\par 11 إِنَّهُمْ فِي جِلْعَادَ قَدْ صَارُوا إِثْماً بُطْلاً لاَ غَيْرُ. فِي الْجِلْجَالِ ذَبَحُوا ثِيرَاناً وَمَذَابِحُهُمْ كَرُجَمٍ فِي أَتْلاَمِ الْحَقْلِ.
\par 12 وَهَرَبَ يَعْقُوبُ إِلَى صَحْرَاءِ أَرَامَ وَخَدَمَ إِسْرَائِيلُ لأَجْلِ امْرَأَةٍ وَلأَجْلِ امْرَأَةٍ رَعَى.
\par 13 وَبِنَبِيٍّ أَصْعَدَ الرَّبُّ إِسْرَائِيلَ مِنْ مِصْرَ وَبِنَبِيٍّ حُفِظَ.
\par 14 أَغَاظَهُ إِسْرَائِيلُ بِمَرَارَةٍ فَيَتْرُكُ دِمَاءَهُ عَلَيْهِ وَيَرُدُّ سَيِّدُهُ عَارَهُ عَلَيْهِ.

\chapter{13}

\par 1 لَمَّا تَكَلَّمَ أَفْرَايِمُ بِرَعْدَةٍ تَرَفَّعَ فِي إِسْرَائِيلَ. وَلَمَّا أَثِمَ بِبَعْلٍ مَاتَ.
\par 2 وَالآنَ يَزْدَادُونَ خَطِيَّةً وَيَصْنَعُونَ لأَنْفُسِهِمْ تَمَاثِيلَ مَسْبُوكَةً مِنْ فِضَّتِهِمْ أَصْنَاماً بِحَذَاقَتِهِمْ كُلُّهَا عَمَلُ الصُّنَّاعِ. عَنْهَا هُمْ يَقُولُونَ: «ذَابِحُو النَّاسِ يُقَبِّلُونَ الْعُجُولَ».
\par 3 لِذَلِكَ يَكُونُونَ كَسَحَابِ الصُّبْحِ وَكَالنَّدَى الْمَاضِي بَاكِراً. كَعُصَافَةٍ تُخْطَفُ مِنَ الْبَيْدَرِ وَكَدُخَانٍ مِنَ الْكُوَّّةِ.
\par 4 «وَأَنَا الرَّبُّ إِلَهُكَ مِنْ أَرْضِ مِصْرَ وَإِلَهاً سُِوَايَ لَسْتَ تَعْرِفُ وَلاَ مُخَلِّصَ غَيْرِي.
\par 5 أَنَا عَرَفْتُكَ فِي الْبَرِّيَّةِ فِي أَرْضِ الْعَطَشِ.
\par 6 لَمَّا رَعُوا شَبِعُوا. شَبِعُوا وَارْتَفَعَتْ قُلُوبُهُمْ لِذَلِكَ نَسُونِي.
\par 7 «فَأَكُونُ لَهُمْ كَأَسَدٍ. أَرْصُدُ عَلَى الطَّرِيقِ كَنَمِرٍ.
\par 8 أَصْدِمُهُمْ كَدُبَّةٍ مُثْكِلٍ وَأَشُقُّ شَغَافَ قَلْبِهِمْ وَآكُلُهُمْ هُنَاكَ كَلَبْوَةٍ. يُمَزِّقُهُمْ وَحْشُ الْبَرِّيَّةِ.
\par 9 «هَلاَكُكَ يَا إِسْرَائِيلُ أَنَّكَ عَلَيَّ عَلَى عَوْنِكَ.
\par 10 فَأَيْنَ هُوَ مَلِكُكَ حَتَّى يُخَلِّصَكَ فِي جَمِيعِ مُدُنِكَ؟ وَقُضَاتُكَ حَيْثُ قُلْتَ: أَعْطِنِي مَلِكاً وَرُؤَسَاءَ؟
\par 11 أَنَا أَعْطَيْتُكَ مَلِكاً بِغَضَبِي وَأَخَذْتُهُ بِسَخَطِي.
\par 12 «إِثْمُ أَفْرَايِمَ مَصْرُورٌ. خَطِيَّتُهُ مَكْنُوزَةٌ.
\par 13 مَخَاضُ الْوَالِدَةِ يَأْتِي عَلَيْهِ. هُوَ ابْنٌ غَيْرُ حَكِيمٍ إِذْ لَمْ يَقِفْ فِي الْوَقْتِ فِي مَوْلِدِ الْبَنِينَ.
\par 14 «مِنْ يَدِ الْهَاوِيَةِ أَفْدِيهِمْ. مِنَ الْمَوْتِ أُخَلِّصُهُمْ. أَيْنَ أَوْبَاؤُكَ يَا مَوْتُ؟ أَيْنَ شَوْكَتُكِ يَا هَاوِيَةُ؟ تَخْتَفِي النَّدَامَةُ عَنْ عَيْنَيَّ».
\par 15 وَإِنْ كَانَ مُثْمِراً بَيْنَ إِخْوَةٍ تَأْتِي رِيحٌ شَرْقِيَّةٌ. رِيحُ الرَّبِّ طَالِعَةً مِنَ الْقَفْرِ فَتَجِفُّ عَيْنُهُ وَيَيْبَسُ يَنْبُوعُهُ. هِيَ تَنْهَبُ كَنْزَ كُلِّ مَتَاعٍ شَهِيٍّ.
\par 16 تُجَازَى السَّامِرَةُ لأَنَّهَا قَدْ تَمَرَّدَتْ عَلَى إِلَهِهَا. بِالسَّيْفِ يَسْقُطُونَ. تُحَطَّمُ أَطْفَالُهُمْ وَالْحَوَامِلُ تُشَقُّ.

\chapter{14}

\par 1 ارْجِعْ يَا إِسْرَائِيلُ إِلَى الرَّبِّ إِلَهِكَ لأَنَّكَ قَدْ تَعَثَّرْتَ بِإِثْمِكَ.
\par 2 خُذُوا مَعَكُمْ كَلاَماً وَارْجِعُوا إِلَى الرَّبِّ. قُولُوا لَهُ: «ارْفَعْ كُلَّ إِثْمٍ وَاقْبَلْ حَسَناً فَنُقَدِّمَ عُجُولَ شِفَاهِنَا.
\par 3 لاَ يُخَلِّصُنَا أَشُّورُ. لاَ نَرْكَبُ عَلَى الْخَيْلِ وَلاَ نَقُولُ أَيْضاً لِعَمَلِ أَيْدِينَا: آلِهَتَنَا. إِنَّهُ بِكَ يُرْحَمُ الْيَتِيمُ».
\par 4 «أَنَا أَشْفِي ارْتِدَادَهُمْ. أُحِبُّهُمْ فَضْلاً لأَنَّ غَضَبِي قَدِ ارْتَدَّ عَنْهُ.
\par 5 أَكُونُ لإِسْرَائِيلَ كَالنَّدَى. يُزْهِرُ كَالسَّوْسَنِ وَيَضْرِبُ أُصُولَهُ كَلُبْنَانَ.
\par 6 تَمْتَدُّ خَرَاعِيبُهُ وَيَكُونُ بَهَاؤُهُ كَالزَّيْتُونَةِ وَلَهُ رَائِحَةٌ كَلُبْنَانَ.
\par 7 يَعُودُ السَّاكِنُونَ فِي ظِلِّهِ يُحْيُونَ حِنْطَةً وَيُزْهِرُونَ كَجَفْنَةٍ. يَكُونُ ذِكْرُهُمْ كَخَمْرِ لُبْنَانَ.
\par 8 يَقُولُ أَفْرَايِمُ: مَا لِي أَيْضاً وَلِلأَصْنَامِ؟ أَنَا قَدْ أَجَبْتُ فَأُلاَحِظُهُ. أَنَا كَسَرْوَةٍ خَضْرَاءَ. مِنْ قِبَلِي يُوجَدُ ثَمَرُكِ».
\par 9 مَنْ هُوَ حَكِيمٌ حَتَّى يَفْهَمَ هَذِهِ الأُمُورَ وَفَهِيمٌ حَتَّى يَعْرِفَهَا؟ فَإِنَّ طُرُقَ الرَّبِّ مُسْتَقِيمَةٌ وَالأَبْرَارَ يَسْلُكُونَ فِيهَا. وَأَمَّا الْمُنَافِقُونَ فَيَعْثُرُونَ فِيهَا.

\end{document}